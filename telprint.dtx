% \iffalse meta-comment
%
% File: telprint.dtx
% Version: 2016/05/16 v1.11
% Info: Format German phone numbers
%
% Copyright (C) 1996, 1997, 2004-2008 by
%    Heiko Oberdiek <heiko.oberdiek at googlemail.com>
%    2016
%    https://github.com/ho-tex/oberdiek/issues
%
% This work may be distributed and/or modified under the
% conditions of the LaTeX Project Public License, either
% version 1.3c of this license or (at your option) any later
% version. This version of this license is in
%    https://www.latex-project.org/lppl/lppl-1-3c.txt
% and the latest version of this license is in
%    https://www.latex-project.org/lppl.txt
% and version 1.3 or later is part of all distributions of
% LaTeX version 2005/12/01 or later.
%
% This work has the LPPL maintenance status "maintained".
%
% The Current Maintainers of this work are
% Heiko Oberdiek and the Oberdiek Package Support Group
% https://github.com/ho-tex/oberdiek/issues
%
% The Base Interpreter refers to any `TeX-Format',
% because some files are installed in TDS:tex/generic//.
%
% This work consists of the main source file telprint.dtx
% and the derived files
%    telprint.sty, telprint.pdf, telprint.ins, telprint.drv,
%    telprint-test1.tex.
%
% Distribution:
%    CTAN:macros/latex/contrib/oberdiek/telprint.dtx
%    CTAN:macros/latex/contrib/oberdiek/telprint.pdf
%
% Unpacking:
%    (a) If telprint.ins is present:
%           tex telprint.ins
%    (b) Without telprint.ins:
%           tex telprint.dtx
%    (c) If you insist on using LaTeX
%           latex \let\install=y% \iffalse meta-comment
%
% File: telprint.dtx
% Version: 2016/05/16 v1.11
% Info: Format German phone numbers
%
% Copyright (C) 1996, 1997, 2004-2008 by
%    Heiko Oberdiek <heiko.oberdiek at googlemail.com>
%    2016
%    https://github.com/ho-tex/oberdiek/issues
%
% This work may be distributed and/or modified under the
% conditions of the LaTeX Project Public License, either
% version 1.3c of this license or (at your option) any later
% version. This version of this license is in
%    http://www.latex-project.org/lppl/lppl-1-3c.txt
% and the latest version of this license is in
%    http://www.latex-project.org/lppl.txt
% and version 1.3 or later is part of all distributions of
% LaTeX version 2005/12/01 or later.
%
% This work has the LPPL maintenance status "maintained".
%
% This Current Maintainer of this work is Heiko Oberdiek.
%
% The Base Interpreter refers to any `TeX-Format',
% because some files are installed in TDS:tex/generic//.
%
% This work consists of the main source file telprint.dtx
% and the derived files
%    telprint.sty, telprint.pdf, telprint.ins, telprint.drv,
%    telprint-test1.tex.
%
% Distribution:
%    CTAN:macros/latex/contrib/oberdiek/telprint.dtx
%    CTAN:macros/latex/contrib/oberdiek/telprint.pdf
%
% Unpacking:
%    (a) If telprint.ins is present:
%           tex telprint.ins
%    (b) Without telprint.ins:
%           tex telprint.dtx
%    (c) If you insist on using LaTeX
%           latex \let\install=y% \iffalse meta-comment
%
% File: telprint.dtx
% Version: 2016/05/16 v1.11
% Info: Format German phone numbers
%
% Copyright (C) 1996, 1997, 2004-2008 by
%    Heiko Oberdiek <heiko.oberdiek at googlemail.com>
%    2016
%    https://github.com/ho-tex/oberdiek/issues
%
% This work may be distributed and/or modified under the
% conditions of the LaTeX Project Public License, either
% version 1.3c of this license or (at your option) any later
% version. This version of this license is in
%    http://www.latex-project.org/lppl/lppl-1-3c.txt
% and the latest version of this license is in
%    http://www.latex-project.org/lppl.txt
% and version 1.3 or later is part of all distributions of
% LaTeX version 2005/12/01 or later.
%
% This work has the LPPL maintenance status "maintained".
%
% This Current Maintainer of this work is Heiko Oberdiek.
%
% The Base Interpreter refers to any `TeX-Format',
% because some files are installed in TDS:tex/generic//.
%
% This work consists of the main source file telprint.dtx
% and the derived files
%    telprint.sty, telprint.pdf, telprint.ins, telprint.drv,
%    telprint-test1.tex.
%
% Distribution:
%    CTAN:macros/latex/contrib/oberdiek/telprint.dtx
%    CTAN:macros/latex/contrib/oberdiek/telprint.pdf
%
% Unpacking:
%    (a) If telprint.ins is present:
%           tex telprint.ins
%    (b) Without telprint.ins:
%           tex telprint.dtx
%    (c) If you insist on using LaTeX
%           latex \let\install=y% \iffalse meta-comment
%
% File: telprint.dtx
% Version: 2016/05/16 v1.11
% Info: Format German phone numbers
%
% Copyright (C) 1996, 1997, 2004-2008 by
%    Heiko Oberdiek <heiko.oberdiek at googlemail.com>
%    2016
%    https://github.com/ho-tex/oberdiek/issues
%
% This work may be distributed and/or modified under the
% conditions of the LaTeX Project Public License, either
% version 1.3c of this license or (at your option) any later
% version. This version of this license is in
%    http://www.latex-project.org/lppl/lppl-1-3c.txt
% and the latest version of this license is in
%    http://www.latex-project.org/lppl.txt
% and version 1.3 or later is part of all distributions of
% LaTeX version 2005/12/01 or later.
%
% This work has the LPPL maintenance status "maintained".
%
% This Current Maintainer of this work is Heiko Oberdiek.
%
% The Base Interpreter refers to any `TeX-Format',
% because some files are installed in TDS:tex/generic//.
%
% This work consists of the main source file telprint.dtx
% and the derived files
%    telprint.sty, telprint.pdf, telprint.ins, telprint.drv,
%    telprint-test1.tex.
%
% Distribution:
%    CTAN:macros/latex/contrib/oberdiek/telprint.dtx
%    CTAN:macros/latex/contrib/oberdiek/telprint.pdf
%
% Unpacking:
%    (a) If telprint.ins is present:
%           tex telprint.ins
%    (b) Without telprint.ins:
%           tex telprint.dtx
%    (c) If you insist on using LaTeX
%           latex \let\install=y\input{telprint.dtx}
%        (quote the arguments according to the demands of your shell)
%
% Documentation:
%    (a) If telprint.drv is present:
%           latex telprint.drv
%    (b) Without telprint.drv:
%           latex telprint.dtx; ...
%    The class ltxdoc loads the configuration file ltxdoc.cfg
%    if available. Here you can specify further options, e.g.
%    use A4 as paper format:
%       \PassOptionsToClass{a4paper}{article}
%
%    Programm calls to get the documentation (example):
%       pdflatex telprint.dtx
%       makeindex -s gind.ist telprint.idx
%       pdflatex telprint.dtx
%       makeindex -s gind.ist telprint.idx
%       pdflatex telprint.dtx
%
% Installation:
%    TDS:tex/generic/oberdiek/telprint.sty
%    TDS:doc/latex/oberdiek/telprint.pdf
%    TDS:doc/latex/oberdiek/test/telprint-test1.tex
%    TDS:source/latex/oberdiek/telprint.dtx
%
%<*ignore>
\begingroup
  \catcode123=1 %
  \catcode125=2 %
  \def\x{LaTeX2e}%
\expandafter\endgroup
\ifcase 0\ifx\install y1\fi\expandafter
         \ifx\csname processbatchFile\endcsname\relax\else1\fi
         \ifx\fmtname\x\else 1\fi\relax
\else\csname fi\endcsname
%</ignore>
%<*install>
\input docstrip.tex
\Msg{************************************************************************}
\Msg{* Installation}
\Msg{* Package: telprint 2016/05/16 v1.11 Format German phone numbers (HO)}
\Msg{************************************************************************}

\keepsilent
\askforoverwritefalse

\let\MetaPrefix\relax
\preamble

This is a generated file.

Project: telprint
Version: 2016/05/16 v1.11

Copyright (C) 1996, 1997, 2004-2008 by
   Heiko Oberdiek <heiko.oberdiek at googlemail.com>

This work may be distributed and/or modified under the
conditions of the LaTeX Project Public License, either
version 1.3c of this license or (at your option) any later
version. This version of this license is in
   http://www.latex-project.org/lppl/lppl-1-3c.txt
and the latest version of this license is in
   http://www.latex-project.org/lppl.txt
and version 1.3 or later is part of all distributions of
LaTeX version 2005/12/01 or later.

This work has the LPPL maintenance status "maintained".

This Current Maintainer of this work is Heiko Oberdiek.

The Base Interpreter refers to any `TeX-Format',
because some files are installed in TDS:tex/generic//.

This work consists of the main source file telprint.dtx
and the derived files
   telprint.sty, telprint.pdf, telprint.ins, telprint.drv,
   telprint-test1.tex.

\endpreamble
\let\MetaPrefix\DoubleperCent

\generate{%
  \file{telprint.ins}{\from{telprint.dtx}{install}}%
  \file{telprint.drv}{\from{telprint.dtx}{driver}}%
  \usedir{tex/generic/oberdiek}%
  \file{telprint.sty}{\from{telprint.dtx}{package}}%
%  \usedir{doc/latex/oberdiek/test}%
%  \file{telprint-test1.tex}{\from{telprint.dtx}{test1}}%
  \nopreamble
  \nopostamble
%  \usedir{source/latex/oberdiek/catalogue}%
%  \file{telprint.xml}{\from{telprint.dtx}{catalogue}}%
}

\catcode32=13\relax% active space
\let =\space%
\Msg{************************************************************************}
\Msg{*}
\Msg{* To finish the installation you have to move the following}
\Msg{* file into a directory searched by TeX:}
\Msg{*}
\Msg{*     telprint.sty}
\Msg{*}
\Msg{* To produce the documentation run the file `telprint.drv'}
\Msg{* through LaTeX.}
\Msg{*}
\Msg{* Happy TeXing!}
\Msg{*}
\Msg{************************************************************************}

\endbatchfile
%</install>
%<*ignore>
\fi
%</ignore>
%<*driver>
\NeedsTeXFormat{LaTeX2e}
\ProvidesFile{telprint.drv}%
  [2016/05/16 v1.11 Format German phone numbers (HO)]%
\documentclass{ltxdoc}
\usepackage{holtxdoc}[2011/11/22]
\usepackage[ngerman,english]{babel}
\begin{document}
  \DocInput{telprint.dtx}%
\end{document}
%</driver>
% \fi
%
%
% \CharacterTable
%  {Upper-case    \A\B\C\D\E\F\G\H\I\J\K\L\M\N\O\P\Q\R\S\T\U\V\W\X\Y\Z
%   Lower-case    \a\b\c\d\e\f\g\h\i\j\k\l\m\n\o\p\q\r\s\t\u\v\w\x\y\z
%   Digits        \0\1\2\3\4\5\6\7\8\9
%   Exclamation   \!     Double quote  \"     Hash (number) \#
%   Dollar        \$     Percent       \%     Ampersand     \&
%   Acute accent  \'     Left paren    \(     Right paren   \)
%   Asterisk      \*     Plus          \+     Comma         \,
%   Minus         \-     Point         \.     Solidus       \/
%   Colon         \:     Semicolon     \;     Less than     \<
%   Equals        \=     Greater than  \>     Question mark \?
%   Commercial at \@     Left bracket  \[     Backslash     \\
%   Right bracket \]     Circumflex    \^     Underscore    \_
%   Grave accent  \`     Left brace    \{     Vertical bar  \|
%   Right brace   \}     Tilde         \~}
%
% \GetFileInfo{telprint.drv}
%
% \title{The \xpackage{telprint} package}
% \date{2016/05/16 v1.11}
% \author{Heiko Oberdiek\thanks
% {Please report any issues at https://github.com/ho-tex/oberdiek/issues}\\
% \xemail{heiko.oberdiek at googlemail.com}}
%
% \maketitle
%
% \begin{abstract}
% Package \xpackage{telprint} provides \cs{telprint} for formatting
% German phone numbers.
% \end{abstract}
%
% \tableofcontents
%
% \section{Documentation}
%
% \subsection{Introduction}
%
%            This is a very old package that I have written
%            to format phone numbers. It follows German
%            conventions and the documentation is mainly in German.
%
% \subsection{Short overview in English}
%
% \LaTeX:
% \begin{quote}
% |\usepackage{telprint}|\\
% |\telprint{123/456-789}|\\
% \end{quote}
% \plainTeX:
% \begin{quote}
%   |\input telprint.sty|\\
%   |\telprint{123/456-789}|
% \end{quote}
%
% \DescribeMacro\telprint
% |\telprint{...}| formats the explicitly given number.
%     Digits, spaces and some special characters
%     ('+', '/', '-', '(', ')', '\textasciitilde', ' ') are supported.
%     Numbers are divided into groups of two digits from the right.
% Examples:
% \begin{quote}
%     |\telprint{0761/12345}     ==> 07\,61/1\,23\,45|\\
%     |\telprint{01234/567-89}   ==> 0\,12\,34/5\,67\leavevmode\hbox{-}89|\\
%     |\telprint{+49 (6221) 297} ==> +49~(62\,21)~2\,97|
% \end{quote}
%
% \subsubsection{Configuration}
%
% The output of the symbols can be configured by
% \cs{telhyphen}, \cs{telslash}, \cs{telleftparen}, \cs{telrightparen},
% \cs{telplus}, \cs{teltilde}.
% Example:
% \begin{quote}
%   |\telslash{\,/\,}\\|
%   |\telprint{12/34} ==> 12\,/\,34|
% \end{quote}
%
% \DescribeMacro\telspace
% \cs{telspace} configures the space between digit groups.
%
% \DescribeMacro\telnumber
% \cs{telnumber} only formats a number in digit groups; special
%    characters are not recognized.
%
% \subsection{Documentation in German}
%
% \begin{otherlanguage*}{ngerman}
% \hyphenation{To-ken-ma-kros}
% \begin{itemize}
% \item \DescribeMacro\telprint |telprint#1|\\
%   Der eigentliche Anwenderbefehl zur formatierten Ausgabe von
%   Telefonnummern. Diese d\"urfen dabei nur als Zahlen angegeben
%   werden(, da sie tokenweise analysiert werden).
%   Als Trenn- oder Sonderzeichen werden unterst\"utzt:
%   '+', '/', '-', '(', ')', '\textasciitilde', ' '
%   Einfache Leerzeichen werden erkannt und durch Tilden ersetzt, um
%   Trennungen in der Telefonnummer zu verhindern. (Man beachte aus
%   gleichem Grunde die \cs{hbox} bei '-'.)
%   Beispiele:
%   \begin{quote}
%     |\telprint{0761/12345}     ==> 07\,61/1\,23\,45|\\
%     |\telprint{01234/567-89}   ==> 0\,12\,34/5\,67\leavevmode\hbox{-}89|\\
%     |\telprint{+49 (6221) 297} ==> +49~(62\,21)~2\,97|
%   \end{quote}
% \end{itemize}
% Der Rest enth\"alt eher Technisches:
% \begin{itemize}
% \item \DescribeMacro\telspace |\telspace#1|\\
%   Mit diesem Befehl wird der Abstand zwischen den Zifferngruppen
%   angegeben (Default: |\,|).
%   (Durch |\telspace{}| kann dieser zusaetzliche Abstand abgestellt
%   werden.)
% \item \DescribeMacro\telhyphen |\telhyphen#1|\\
%   Dieser Befehl gibt die Art des Bindestriches, wie er ausgegeben
%   werden soll. In der Eingabe darf jedoch nur der einfache
%   Bindestrich stehen:
%   |\telprint{123-45}|, jedoch NIE |\telprint{123--45}|!
%   Kopka-Bindestrich-Fans geben an:
%   |\telhyphen{\leavevmode\hbox{--}}|
% \item
%   \DescribeMacro{\telslash}
%   \DescribeMacro{\telleftparen}
%   \DescribeMacro{\telrightparen}
%   \DescribeMacro{\telplus}
%   \DescribeMacro{\teltilde}
%   |\telslash#1|, |\telleftparen#1|, |\telrightparen#1|, |\telplus#1|,
%   |\teltilde|\\
%   Diese Befehle konfigurieren die Zeichen '/', '(', ')', '+'
%   und '\textasciitilde'. Sie funktionieren analog zu \cs{telhyphen}.
% \item \DescribeMacro\telnumber |\telnumber#1|\\
%   Richtung interner Befehl: Er dient dazu, eine Zifferngruppe
%   in Zweiergruppen auszugeben.
%   Die einzelnen Zahlen werden im Tokenregister \cs{TELtoks}
%   gespeichert. Abwechselnd werden dabei zwischen zwei Token
%   (Zahlen) \cs{TELx} bzw. \cs{TELy} eingefuegt, abh\"angig von dem
%   wechselnden Wert von \cs{TELswitch}. Zum Schluss kann dann einfach
%   festgestellt werden ob die Nummer nun eine geradzahlige oder
%   ungeradzahlige Zahl von Ziffern aufwies. Dem entsprechend wird
%   \cs{TELx} mit dem Zusatzabstand belegt und \cs{TELy} leer definiert
%   oder umgekehrt. )
% \item |\TEL...| interne Befehle, Technisches:\\
%   \cs{TELsplit} dient zur Aufteilung einer zusammengesetzten
%   Telefonnummer (Vorwahl, Hauptnummer, Nebenstelle). In dieser
%   Implementation werden als Trennzeichen nur '/' und '-' erkannt.
%   Die einzelnen Bestandteile wie Vorwahl werden dann dem Befehl
%   \cs{telnumber} zur Formatierung uebergeben.
% \item Die Erkennung von einfachen Leerzeichen ist um einiges
%   schwieriger: Die Tokentrennung ueber Parameter |#1#2| funktioniert
%   nicht f\"ur einfache Leerzeichen, da TeX sie \emph{niemals} als
%   eigenst\"andige Argumente behandelt! (The TeXbook, Chapter 20,
%   p. 201)
%
%   (Anmerkung am Rande: Deshalb funktionieren die entsprechenden
%   Tokenmakros auf S. 149 des Buches "`Einf\"uhrung in TeX"' von
%   N. Schwarz (3. Aufl.) nicht, wenn im Tokenregister als erstes
%   ein einfaches Leerzeichen steht!)
% \end{itemize}
% \end{otherlanguage*}
%
% \StopEventually{
% }
%
% \section{Implementation}
%
%    \begin{macrocode}
%<*package>
%    \end{macrocode}
%
% \subsection{Reload check and package identification}
%    Reload check, especially if the package is not used with \LaTeX.
%    \begin{macrocode}
\begingroup\catcode61\catcode48\catcode32=10\relax%
  \catcode13=5 % ^^M
  \endlinechar=13 %
  \catcode35=6 % #
  \catcode39=12 % '
  \catcode44=12 % ,
  \catcode45=12 % -
  \catcode46=12 % .
  \catcode58=12 % :
  \catcode64=11 % @
  \catcode123=1 % {
  \catcode125=2 % }
  \expandafter\let\expandafter\x\csname ver@telprint.sty\endcsname
  \ifx\x\relax % plain-TeX, first loading
  \else
    \def\empty{}%
    \ifx\x\empty % LaTeX, first loading,
      % variable is initialized, but \ProvidesPackage not yet seen
    \else
      \expandafter\ifx\csname PackageInfo\endcsname\relax
        \def\x#1#2{%
          \immediate\write-1{Package #1 Info: #2.}%
        }%
      \else
        \def\x#1#2{\PackageInfo{#1}{#2, stopped}}%
      \fi
      \x{telprint}{The package is already loaded}%
      \aftergroup\endinput
    \fi
  \fi
\endgroup%
%    \end{macrocode}
%    Package identification:
%    \begin{macrocode}
\begingroup\catcode61\catcode48\catcode32=10\relax%
  \catcode13=5 % ^^M
  \endlinechar=13 %
  \catcode35=6 % #
  \catcode39=12 % '
  \catcode40=12 % (
  \catcode41=12 % )
  \catcode44=12 % ,
  \catcode45=12 % -
  \catcode46=12 % .
  \catcode47=12 % /
  \catcode58=12 % :
  \catcode64=11 % @
  \catcode91=12 % [
  \catcode93=12 % ]
  \catcode123=1 % {
  \catcode125=2 % }
  \expandafter\ifx\csname ProvidesPackage\endcsname\relax
    \def\x#1#2#3[#4]{\endgroup
      \immediate\write-1{Package: #3 #4}%
      \xdef#1{#4}%
    }%
  \else
    \def\x#1#2[#3]{\endgroup
      #2[{#3}]%
      \ifx#1\@undefined
        \xdef#1{#3}%
      \fi
      \ifx#1\relax
        \xdef#1{#3}%
      \fi
    }%
  \fi
\expandafter\x\csname ver@telprint.sty\endcsname
\ProvidesPackage{telprint}%
  [2016/05/16 v1.11 Format German phone numbers (HO)]%
%    \end{macrocode}
%
% \subsection{Catcodes}
%
%    \begin{macrocode}
\begingroup\catcode61\catcode48\catcode32=10\relax%
  \catcode13=5 % ^^M
  \endlinechar=13 %
  \catcode123=1 % {
  \catcode125=2 % }
  \catcode64=11 % @
  \def\x{\endgroup
    \expandafter\edef\csname TELAtEnd\endcsname{%
      \endlinechar=\the\endlinechar\relax
      \catcode13=\the\catcode13\relax
      \catcode32=\the\catcode32\relax
      \catcode35=\the\catcode35\relax
      \catcode61=\the\catcode61\relax
      \catcode64=\the\catcode64\relax
      \catcode123=\the\catcode123\relax
      \catcode125=\the\catcode125\relax
    }%
  }%
\x\catcode61\catcode48\catcode32=10\relax%
\catcode13=5 % ^^M
\endlinechar=13 %
\catcode35=6 % #
\catcode64=11 % @
\catcode123=1 % {
\catcode125=2 % }
\def\TMP@EnsureCode#1#2{%
  \edef\TELAtEnd{%
    \TELAtEnd
    \catcode#1=\the\catcode#1\relax
  }%
  \catcode#1=#2\relax
}
\TMP@EnsureCode{33}{12}% !
\TMP@EnsureCode{36}{3}% $
\TMP@EnsureCode{40}{12}% (
\TMP@EnsureCode{41}{12}% )
\TMP@EnsureCode{42}{12}% *
\TMP@EnsureCode{43}{12}% +
\TMP@EnsureCode{44}{12}% ,
\TMP@EnsureCode{45}{12}% -
\TMP@EnsureCode{46}{12}% .
\TMP@EnsureCode{47}{12}% /
\TMP@EnsureCode{91}{12}% [
\TMP@EnsureCode{93}{12}% ]
\TMP@EnsureCode{126}{13}% ~ (active)
\edef\TELAtEnd{\TELAtEnd\noexpand\endinput}
%    \end{macrocode}
%
% \subsection{Package macros}
%    \begin{macrocode}
\ifx\DeclareRobustCommand\UnDeFiNeD
  \def\DeclareRobustCommand*#1[1]{\def#1##1}%
  \def\TELreset{\let\DeclareRobustCommand=\UnDeFiNeD}%
  \input infwarerr.sty\relax
  \@PackageInfo{telprint}{%
    Macros are not robust!%
  }%
\else
  \let\TELreset=\relax
\fi
%    \end{macrocode}
%    \begin{macro}{\telspace}
%    \begin{macrocode}
\DeclareRobustCommand*{\telspace}[1]{\def\TELspace{#1}}
\telspace{{}$\,${}}
%    \end{macrocode}
%    \end{macro}
%    \begin{macro}{\telhyphen}
%    \begin{macrocode}
\DeclareRobustCommand*{\telhyphen}[1]{\def\TELhyphen{#1}}
\telhyphen{\leavevmode\hbox{-}}% \hbox zur Verhinderung der Trennung
%    \end{macrocode}
%    \end{macro}
%    \begin{macro}{\telslash}
%    \begin{macrocode}
\DeclareRobustCommand*{\telslash}[1]{\def\TELslash{#1}}
\telslash{/}%
%    \end{macrocode}
%    \end{macro}
%    \begin{macro}{\telleftparen}
%    \begin{macrocode}
\DeclareRobustCommand*{\telleftparen}[1]{\def\TELleftparen{#1}}
\telleftparen{(}%
%    \end{macrocode}
%    \end{macro}
%    \begin{macro}{\telrightparen}
%    \begin{macrocode}
\DeclareRobustCommand*{\telrightparen}[1]{\def\TELrightparen{#1}}
\telrightparen{)}%
%    \end{macrocode}
%    \end{macro}
%    \begin{macro}{\telplus}
%    \begin{macrocode}
\DeclareRobustCommand*{\telplus}[1]{\def\TELplus{#1}}
\telplus{+}%
%    \end{macrocode}
%    \end{macro}
%    \begin{macro}{\teltilde}
%    \begin{macrocode}
\DeclareRobustCommand*{\teltilde}[1]{\def\TELtilde{#1}}
\teltilde{~}%
%    \end{macrocode}
%    \end{macro}
%    \begin{macro}{\TELtoks}
%    \begin{macrocode}
\newtoks\TELtoks
%    \end{macrocode}
%    \end{macro}
%    \begin{macro}{\TELnumber}
%    \begin{macrocode}
\def\TELnumber#1#2\TELnumberEND{%
  \begingroup
  \def\0{#2}%
  \expandafter\endgroup
  \ifx\0\empty
    \TELtoks=\expandafter{\the\TELtoks#1}%
    \ifnum\TELswitch=0 %
      \def\TELx{\TELspace}\def\TELy{}%
    \else
      \def\TELx{}\def\TELy{\TELspace}%
    \fi
    \the\TELtoks
  \else
    \ifnum\TELswitch=0 %
      \TELtoks=\expandafter{\the\TELtoks#1\TELx}%
      \def\TELswitch{1}%
    \else
      \TELtoks=\expandafter{\the\TELtoks#1\TELy}%
      \def\TELswitch{0}%
    \fi
    \TELnumber#2\TELnumberEND
  \fi
}
%    \end{macrocode}
%    \end{macro}
%    \begin{macro}{\telnumber}
%    \begin{macrocode}
\DeclareRobustCommand*{\telnumber}[1]{%
  \TELtoks={}%
  \def\TELswitch{0}%
  \TELnumber#1{}\TELnumberEND
}
%    \end{macrocode}
%    \end{macro}
%    \begin{macro}{\TELsplit}
%    \begin{macrocode}
\def\TELsplit{\futurelet\TELfuture\TELdosplit}
%    \end{macrocode}
%    \end{macro}
%    \begin{macro}{\TELdosplit}
%    \begin{macrocode}
\def\TELdosplit#1#2\TELsplitEND
{%
  \def\TELsp{ }%
  \expandafter\ifx\TELsp\TELfuture
    \let\TELfuture=\relax
    \expandafter\telnumber\expandafter{\the\TELtoks}~%
    \telprint{#1#2}% Das Leerzeichen kann nicht #1 sein!
  \else
    \def\TELfirst{#1}%
    \ifx\TELfirst\empty
      \expandafter\telnumber\expandafter{\the\TELtoks}%
      \TELtoks={}%
    \else\if-\TELfirst
      \expandafter\telnumber\expandafter{\the\TELtoks}\TELhyphen
      \telprint{#2}%
    \else\if/\TELfirst
      \expandafter\telnumber\expandafter{\the\TELtoks}\TELslash
      \telprint{#2}%
    \else\if(\TELfirst
      \expandafter\telnumber\expandafter{\the\TELtoks}\TELleftparen
      \telprint{#2}%
    \else\if)\TELfirst
      \expandafter\telnumber\expandafter{\the\TELtoks}\TELrightparen
      \telprint{#2}%
    \else\if+\TELfirst
      \expandafter\telnumber\expandafter{\the\TELtoks}\TELplus
      \telprint{#2}%
    \else\def\TELtemp{~}\ifx\TELtemp\TELfirst
      \expandafter\telnumber\expandafter{\the\TELtoks}\TELtilde
      \telprint{#2}%
    \else
      \TELtoks=\expandafter{\the\TELtoks#1}%
      \TELsplit#2{}\TELsplitEND
    \fi\fi\fi\fi\fi\fi\fi
  \fi
}
%    \end{macrocode}
%    \end{macro}
%    \begin{macro}{\telprint}
%    \begin{macrocode}
\DeclareRobustCommand*{\telprint}[1]{%
  \TELtoks={}%
  \TELsplit#1{}\TELsplitEND
}
%    \end{macrocode}
%    \end{macro}
%    \begin{macrocode}
\TELreset\let\TELreset=\UnDeFiNeD
%    \end{macrocode}
%
%    \begin{macrocode}
\TELAtEnd%
%</package>
%    \end{macrocode}
%
% \section{Test}
%
% \subsection{Catcode checks for loading}
%
%    \begin{macrocode}
%<*test1>
%    \end{macrocode}
%    \begin{macrocode}
\catcode`\{=1 %
\catcode`\}=2 %
\catcode`\#=6 %
\catcode`\@=11 %
\expandafter\ifx\csname count@\endcsname\relax
  \countdef\count@=255 %
\fi
\expandafter\ifx\csname @gobble\endcsname\relax
  \long\def\@gobble#1{}%
\fi
\expandafter\ifx\csname @firstofone\endcsname\relax
  \long\def\@firstofone#1{#1}%
\fi
\expandafter\ifx\csname loop\endcsname\relax
  \expandafter\@firstofone
\else
  \expandafter\@gobble
\fi
{%
  \def\loop#1\repeat{%
    \def\body{#1}%
    \iterate
  }%
  \def\iterate{%
    \body
      \let\next\iterate
    \else
      \let\next\relax
    \fi
    \next
  }%
  \let\repeat=\fi
}%
\def\RestoreCatcodes{}
\count@=0 %
\loop
  \edef\RestoreCatcodes{%
    \RestoreCatcodes
    \catcode\the\count@=\the\catcode\count@\relax
  }%
\ifnum\count@<255 %
  \advance\count@ 1 %
\repeat

\def\RangeCatcodeInvalid#1#2{%
  \count@=#1\relax
  \loop
    \catcode\count@=15 %
  \ifnum\count@<#2\relax
    \advance\count@ 1 %
  \repeat
}
\def\RangeCatcodeCheck#1#2#3{%
  \count@=#1\relax
  \loop
    \ifnum#3=\catcode\count@
    \else
      \errmessage{%
        Character \the\count@\space
        with wrong catcode \the\catcode\count@\space
        instead of \number#3%
      }%
    \fi
  \ifnum\count@<#2\relax
    \advance\count@ 1 %
  \repeat
}
\def\space{ }
\expandafter\ifx\csname LoadCommand\endcsname\relax
  \def\LoadCommand{\input telprint.sty\relax}%
\fi
\def\Test{%
  \RangeCatcodeInvalid{0}{47}%
  \RangeCatcodeInvalid{58}{64}%
  \RangeCatcodeInvalid{91}{96}%
  \RangeCatcodeInvalid{123}{255}%
  \catcode`\@=12 %
  \catcode`\\=0 %
  \catcode`\%=14 %
  \LoadCommand
  \RangeCatcodeCheck{0}{36}{15}%
  \RangeCatcodeCheck{37}{37}{14}%
  \RangeCatcodeCheck{38}{47}{15}%
  \RangeCatcodeCheck{48}{57}{12}%
  \RangeCatcodeCheck{58}{63}{15}%
  \RangeCatcodeCheck{64}{64}{12}%
  \RangeCatcodeCheck{65}{90}{11}%
  \RangeCatcodeCheck{91}{91}{15}%
  \RangeCatcodeCheck{92}{92}{0}%
  \RangeCatcodeCheck{93}{96}{15}%
  \RangeCatcodeCheck{97}{122}{11}%
  \RangeCatcodeCheck{123}{255}{15}%
  \RestoreCatcodes
}
\Test
\csname @@end\endcsname
\end
%    \end{macrocode}
%    \begin{macrocode}
%</test1>
%    \end{macrocode}
%
% \section{Installation}
%
% \subsection{Download}
%
% \paragraph{Package.} This package is available on
% CTAN\footnote{\url{https://ctan.org/pkg/telprint}}:
% \begin{description}
% \item[\CTAN{macros/latex/contrib/oberdiek/telprint.dtx}] The source file.
% \item[\CTAN{macros/latex/contrib/oberdiek/telprint.pdf}] Documentation.
% \end{description}
%
%
% \paragraph{Bundle.} All the packages of the bundle `oberdiek'
% are also available in a TDS compliant ZIP archive. There
% the packages are already unpacked and the documentation files
% are generated. The files and directories obey the TDS standard.
% \begin{description}
% \item[\CTANinstall{install/macros/latex/contrib/oberdiek.tds.zip}]
% \end{description}
% \emph{TDS} refers to the standard ``A Directory Structure
% for \TeX\ Files'' (\CTAN{tds/tds.pdf}). Directories
% with \xfile{texmf} in their name are usually organized this way.
%
% \subsection{Bundle installation}
%
% \paragraph{Unpacking.} Unpack the \xfile{oberdiek.tds.zip} in the
% TDS tree (also known as \xfile{texmf} tree) of your choice.
% Example (linux):
% \begin{quote}
%   |unzip oberdiek.tds.zip -d ~/texmf|
% \end{quote}
%
% \paragraph{Script installation.}
% Check the directory \xfile{TDS:scripts/oberdiek/} for
% scripts that need further installation steps.
% Package \xpackage{attachfile2} comes with the Perl script
% \xfile{pdfatfi.pl} that should be installed in such a way
% that it can be called as \texttt{pdfatfi}.
% Example (linux):
% \begin{quote}
%   |chmod +x scripts/oberdiek/pdfatfi.pl|\\
%   |cp scripts/oberdiek/pdfatfi.pl /usr/local/bin/|
% \end{quote}
%
% \subsection{Package installation}
%
% \paragraph{Unpacking.} The \xfile{.dtx} file is a self-extracting
% \docstrip\ archive. The files are extracted by running the
% \xfile{.dtx} through \plainTeX:
% \begin{quote}
%   \verb|tex telprint.dtx|
% \end{quote}
%
% \paragraph{TDS.} Now the different files must be moved into
% the different directories in your installation TDS tree
% (also known as \xfile{texmf} tree):
% \begin{quote}
% \def\t{^^A
% \begin{tabular}{@{}>{\ttfamily}l@{ $\rightarrow$ }>{\ttfamily}l@{}}
%   telprint.sty & tex/generic/oberdiek/telprint.sty\\
%   telprint.pdf & doc/latex/oberdiek/telprint.pdf\\
%   test/telprint-test1.tex & doc/latex/oberdiek/test/telprint-test1.tex\\
%   telprint.dtx & source/latex/oberdiek/telprint.dtx\\
% \end{tabular}^^A
% }^^A
% \sbox0{\t}^^A
% \ifdim\wd0>\linewidth
%   \begingroup
%     \advance\linewidth by\leftmargin
%     \advance\linewidth by\rightmargin
%   \edef\x{\endgroup
%     \def\noexpand\lw{\the\linewidth}^^A
%   }\x
%   \def\lwbox{^^A
%     \leavevmode
%     \hbox to \linewidth{^^A
%       \kern-\leftmargin\relax
%       \hss
%       \usebox0
%       \hss
%       \kern-\rightmargin\relax
%     }^^A
%   }^^A
%   \ifdim\wd0>\lw
%     \sbox0{\small\t}^^A
%     \ifdim\wd0>\linewidth
%       \ifdim\wd0>\lw
%         \sbox0{\footnotesize\t}^^A
%         \ifdim\wd0>\linewidth
%           \ifdim\wd0>\lw
%             \sbox0{\scriptsize\t}^^A
%             \ifdim\wd0>\linewidth
%               \ifdim\wd0>\lw
%                 \sbox0{\tiny\t}^^A
%                 \ifdim\wd0>\linewidth
%                   \lwbox
%                 \else
%                   \usebox0
%                 \fi
%               \else
%                 \lwbox
%               \fi
%             \else
%               \usebox0
%             \fi
%           \else
%             \lwbox
%           \fi
%         \else
%           \usebox0
%         \fi
%       \else
%         \lwbox
%       \fi
%     \else
%       \usebox0
%     \fi
%   \else
%     \lwbox
%   \fi
% \else
%   \usebox0
% \fi
% \end{quote}
% If you have a \xfile{docstrip.cfg} that configures and enables \docstrip's
% TDS installing feature, then some files can already be in the right
% place, see the documentation of \docstrip.
%
% \subsection{Refresh file name databases}
%
% If your \TeX~distribution
% (\teTeX, \mikTeX, \dots) relies on file name databases, you must refresh
% these. For example, \teTeX\ users run \verb|texhash| or
% \verb|mktexlsr|.
%
% \subsection{Some details for the interested}
%
% \paragraph{Attached source.}
%
% The PDF documentation on CTAN also includes the
% \xfile{.dtx} source file. It can be extracted by
% AcrobatReader 6 or higher. Another option is \textsf{pdftk},
% e.g. unpack the file into the current directory:
% \begin{quote}
%   \verb|pdftk telprint.pdf unpack_files output .|
% \end{quote}
%
% \paragraph{Unpacking with \LaTeX.}
% The \xfile{.dtx} chooses its action depending on the format:
% \begin{description}
% \item[\plainTeX:] Run \docstrip\ and extract the files.
% \item[\LaTeX:] Generate the documentation.
% \end{description}
% If you insist on using \LaTeX\ for \docstrip\ (really,
% \docstrip\ does not need \LaTeX), then inform the autodetect routine
% about your intention:
% \begin{quote}
%   \verb|latex \let\install=y\input{telprint.dtx}|
% \end{quote}
% Do not forget to quote the argument according to the demands
% of your shell.
%
% \paragraph{Generating the documentation.}
% You can use both the \xfile{.dtx} or the \xfile{.drv} to generate
% the documentation. The process can be configured by the
% configuration file \xfile{ltxdoc.cfg}. For instance, put this
% line into this file, if you want to have A4 as paper format:
% \begin{quote}
%   \verb|\PassOptionsToClass{a4paper}{article}|
% \end{quote}
% An example follows how to generate the
% documentation with pdf\LaTeX:
% \begin{quote}
%\begin{verbatim}
%pdflatex telprint.dtx
%makeindex -s gind.ist telprint.idx
%pdflatex telprint.dtx
%makeindex -s gind.ist telprint.idx
%pdflatex telprint.dtx
%\end{verbatim}
% \end{quote}
%
% \begin{History}
%   \begin{Version}{1996/11/28 v1.0}
%   \item
%     Erste lauff\"ahige Version.
%   \item
%     Nur '-' und '/' als zul\"assige Sonderzeichen.
%   \end{Version}
%   \begin{Version}{1997/09/16 v1.1}
%   \item
%     Dokumentation und Kommentare (Posting in de.comp.text.tex).
%   \item
%     Erweiterung um Sonderzeichen '(', ')', '+', '\textasciitilde' und ' '.
%   \item
%     Trennungsverhinderung am 'hyphen'.
%   \end{Version}
%   \begin{Version}{1997/10/16 v1.2}
%   \item
%     Schutz vor wiederholtem Einlesen.
%   \item
%     Unter \LaTeXe\ Nutzung des \cs{DeclareRobustCommand}-Features.
%   \end{Version}
%   \begin{Version}{1997/12/09 v1.3}
%   \item
%     Tempor\"are Variable eingespart.
%   \item
%     Posted in newsgroup \xnewsgroup{de.comp.text.tex}:\\
%     \URL{``\link{Re: Generisches Markup f\"ur Telefonnummern?}''}^^A
%     {http://groups.google.com/group/de.comp.text.tex/msg/86b3a86140007309}
%   \end{Version}
%   \begin{Version}{2004/11/02 v1.4}
%   \item
%     Fehler in der Dokumentation korrigiert.
%   \end{Version}
%   \begin{Version}{2005/09/30 v1.5}
%   \item
%     Konfigurierbare Symbole: '/', '(', ')', '+' und '\textasciitilde'.
%   \end{Version}
%   \begin{Version}{2006/02/12 v1.6}
%   \item
%     LPPL 1.3.
%   \item
%     Kurze \"Ubersicht in Englisch.
%   \item
%     CTAN.
%   \end{Version}
%   \begin{Version}{2006/08/26 v1.7}
%   \item
%     New DTX framework.
%   \end{Version}
%   \begin{Version}{2007/04/11 v1.8}
%   \item
%     Line ends sanitized.
%   \end{Version}
%   \begin{Version}{2007/09/09 v1.9}
%   \item
%     Catcode section added.
%   \item
%     Missing docstrip tag added.
%   \end{Version}
%   \begin{Version}{2008/08/11 v1.10}
%   \item
%     Code is not changed.
%   \item
%     URLs updated.
%   \end{Version}
%   \begin{Version}{2016/05/16 v1.11}
%   \item
%     Documentation updates.
%   \end{Version}
% \end{History}
%
% \PrintIndex
%
% \Finale
\endinput

%        (quote the arguments according to the demands of your shell)
%
% Documentation:
%    (a) If telprint.drv is present:
%           latex telprint.drv
%    (b) Without telprint.drv:
%           latex telprint.dtx; ...
%    The class ltxdoc loads the configuration file ltxdoc.cfg
%    if available. Here you can specify further options, e.g.
%    use A4 as paper format:
%       \PassOptionsToClass{a4paper}{article}
%
%    Programm calls to get the documentation (example):
%       pdflatex telprint.dtx
%       makeindex -s gind.ist telprint.idx
%       pdflatex telprint.dtx
%       makeindex -s gind.ist telprint.idx
%       pdflatex telprint.dtx
%
% Installation:
%    TDS:tex/generic/oberdiek/telprint.sty
%    TDS:doc/latex/oberdiek/telprint.pdf
%    TDS:doc/latex/oberdiek/test/telprint-test1.tex
%    TDS:source/latex/oberdiek/telprint.dtx
%
%<*ignore>
\begingroup
  \catcode123=1 %
  \catcode125=2 %
  \def\x{LaTeX2e}%
\expandafter\endgroup
\ifcase 0\ifx\install y1\fi\expandafter
         \ifx\csname processbatchFile\endcsname\relax\else1\fi
         \ifx\fmtname\x\else 1\fi\relax
\else\csname fi\endcsname
%</ignore>
%<*install>
\input docstrip.tex
\Msg{************************************************************************}
\Msg{* Installation}
\Msg{* Package: telprint 2016/05/16 v1.11 Format German phone numbers (HO)}
\Msg{************************************************************************}

\keepsilent
\askforoverwritefalse

\let\MetaPrefix\relax
\preamble

This is a generated file.

Project: telprint
Version: 2016/05/16 v1.11

Copyright (C) 1996, 1997, 2004-2008 by
   Heiko Oberdiek <heiko.oberdiek at googlemail.com>

This work may be distributed and/or modified under the
conditions of the LaTeX Project Public License, either
version 1.3c of this license or (at your option) any later
version. This version of this license is in
   http://www.latex-project.org/lppl/lppl-1-3c.txt
and the latest version of this license is in
   http://www.latex-project.org/lppl.txt
and version 1.3 or later is part of all distributions of
LaTeX version 2005/12/01 or later.

This work has the LPPL maintenance status "maintained".

This Current Maintainer of this work is Heiko Oberdiek.

The Base Interpreter refers to any `TeX-Format',
because some files are installed in TDS:tex/generic//.

This work consists of the main source file telprint.dtx
and the derived files
   telprint.sty, telprint.pdf, telprint.ins, telprint.drv,
   telprint-test1.tex.

\endpreamble
\let\MetaPrefix\DoubleperCent

\generate{%
  \file{telprint.ins}{\from{telprint.dtx}{install}}%
  \file{telprint.drv}{\from{telprint.dtx}{driver}}%
  \usedir{tex/generic/oberdiek}%
  \file{telprint.sty}{\from{telprint.dtx}{package}}%
%  \usedir{doc/latex/oberdiek/test}%
%  \file{telprint-test1.tex}{\from{telprint.dtx}{test1}}%
  \nopreamble
  \nopostamble
%  \usedir{source/latex/oberdiek/catalogue}%
%  \file{telprint.xml}{\from{telprint.dtx}{catalogue}}%
}

\catcode32=13\relax% active space
\let =\space%
\Msg{************************************************************************}
\Msg{*}
\Msg{* To finish the installation you have to move the following}
\Msg{* file into a directory searched by TeX:}
\Msg{*}
\Msg{*     telprint.sty}
\Msg{*}
\Msg{* To produce the documentation run the file `telprint.drv'}
\Msg{* through LaTeX.}
\Msg{*}
\Msg{* Happy TeXing!}
\Msg{*}
\Msg{************************************************************************}

\endbatchfile
%</install>
%<*ignore>
\fi
%</ignore>
%<*driver>
\NeedsTeXFormat{LaTeX2e}
\ProvidesFile{telprint.drv}%
  [2016/05/16 v1.11 Format German phone numbers (HO)]%
\documentclass{ltxdoc}
\usepackage{holtxdoc}[2011/11/22]
\usepackage[ngerman,english]{babel}
\begin{document}
  \DocInput{telprint.dtx}%
\end{document}
%</driver>
% \fi
%
%
% \CharacterTable
%  {Upper-case    \A\B\C\D\E\F\G\H\I\J\K\L\M\N\O\P\Q\R\S\T\U\V\W\X\Y\Z
%   Lower-case    \a\b\c\d\e\f\g\h\i\j\k\l\m\n\o\p\q\r\s\t\u\v\w\x\y\z
%   Digits        \0\1\2\3\4\5\6\7\8\9
%   Exclamation   \!     Double quote  \"     Hash (number) \#
%   Dollar        \$     Percent       \%     Ampersand     \&
%   Acute accent  \'     Left paren    \(     Right paren   \)
%   Asterisk      \*     Plus          \+     Comma         \,
%   Minus         \-     Point         \.     Solidus       \/
%   Colon         \:     Semicolon     \;     Less than     \<
%   Equals        \=     Greater than  \>     Question mark \?
%   Commercial at \@     Left bracket  \[     Backslash     \\
%   Right bracket \]     Circumflex    \^     Underscore    \_
%   Grave accent  \`     Left brace    \{     Vertical bar  \|
%   Right brace   \}     Tilde         \~}
%
% \GetFileInfo{telprint.drv}
%
% \title{The \xpackage{telprint} package}
% \date{2016/05/16 v1.11}
% \author{Heiko Oberdiek\thanks
% {Please report any issues at https://github.com/ho-tex/oberdiek/issues}\\
% \xemail{heiko.oberdiek at googlemail.com}}
%
% \maketitle
%
% \begin{abstract}
% Package \xpackage{telprint} provides \cs{telprint} for formatting
% German phone numbers.
% \end{abstract}
%
% \tableofcontents
%
% \section{Documentation}
%
% \subsection{Introduction}
%
%            This is a very old package that I have written
%            to format phone numbers. It follows German
%            conventions and the documentation is mainly in German.
%
% \subsection{Short overview in English}
%
% \LaTeX:
% \begin{quote}
% |\usepackage{telprint}|\\
% |\telprint{123/456-789}|\\
% \end{quote}
% \plainTeX:
% \begin{quote}
%   |\input telprint.sty|\\
%   |\telprint{123/456-789}|
% \end{quote}
%
% \DescribeMacro\telprint
% |\telprint{...}| formats the explicitly given number.
%     Digits, spaces and some special characters
%     ('+', '/', '-', '(', ')', '\textasciitilde', ' ') are supported.
%     Numbers are divided into groups of two digits from the right.
% Examples:
% \begin{quote}
%     |\telprint{0761/12345}     ==> 07\,61/1\,23\,45|\\
%     |\telprint{01234/567-89}   ==> 0\,12\,34/5\,67\leavevmode\hbox{-}89|\\
%     |\telprint{+49 (6221) 297} ==> +49~(62\,21)~2\,97|
% \end{quote}
%
% \subsubsection{Configuration}
%
% The output of the symbols can be configured by
% \cs{telhyphen}, \cs{telslash}, \cs{telleftparen}, \cs{telrightparen},
% \cs{telplus}, \cs{teltilde}.
% Example:
% \begin{quote}
%   |\telslash{\,/\,}\\|
%   |\telprint{12/34} ==> 12\,/\,34|
% \end{quote}
%
% \DescribeMacro\telspace
% \cs{telspace} configures the space between digit groups.
%
% \DescribeMacro\telnumber
% \cs{telnumber} only formats a number in digit groups; special
%    characters are not recognized.
%
% \subsection{Documentation in German}
%
% \begin{otherlanguage*}{ngerman}
% \hyphenation{To-ken-ma-kros}
% \begin{itemize}
% \item \DescribeMacro\telprint |telprint#1|\\
%   Der eigentliche Anwenderbefehl zur formatierten Ausgabe von
%   Telefonnummern. Diese d\"urfen dabei nur als Zahlen angegeben
%   werden(, da sie tokenweise analysiert werden).
%   Als Trenn- oder Sonderzeichen werden unterst\"utzt:
%   '+', '/', '-', '(', ')', '\textasciitilde', ' '
%   Einfache Leerzeichen werden erkannt und durch Tilden ersetzt, um
%   Trennungen in der Telefonnummer zu verhindern. (Man beachte aus
%   gleichem Grunde die \cs{hbox} bei '-'.)
%   Beispiele:
%   \begin{quote}
%     |\telprint{0761/12345}     ==> 07\,61/1\,23\,45|\\
%     |\telprint{01234/567-89}   ==> 0\,12\,34/5\,67\leavevmode\hbox{-}89|\\
%     |\telprint{+49 (6221) 297} ==> +49~(62\,21)~2\,97|
%   \end{quote}
% \end{itemize}
% Der Rest enth\"alt eher Technisches:
% \begin{itemize}
% \item \DescribeMacro\telspace |\telspace#1|\\
%   Mit diesem Befehl wird der Abstand zwischen den Zifferngruppen
%   angegeben (Default: |\,|).
%   (Durch |\telspace{}| kann dieser zusaetzliche Abstand abgestellt
%   werden.)
% \item \DescribeMacro\telhyphen |\telhyphen#1|\\
%   Dieser Befehl gibt die Art des Bindestriches, wie er ausgegeben
%   werden soll. In der Eingabe darf jedoch nur der einfache
%   Bindestrich stehen:
%   |\telprint{123-45}|, jedoch NIE |\telprint{123--45}|!
%   Kopka-Bindestrich-Fans geben an:
%   |\telhyphen{\leavevmode\hbox{--}}|
% \item
%   \DescribeMacro{\telslash}
%   \DescribeMacro{\telleftparen}
%   \DescribeMacro{\telrightparen}
%   \DescribeMacro{\telplus}
%   \DescribeMacro{\teltilde}
%   |\telslash#1|, |\telleftparen#1|, |\telrightparen#1|, |\telplus#1|,
%   |\teltilde|\\
%   Diese Befehle konfigurieren die Zeichen '/', '(', ')', '+'
%   und '\textasciitilde'. Sie funktionieren analog zu \cs{telhyphen}.
% \item \DescribeMacro\telnumber |\telnumber#1|\\
%   Richtung interner Befehl: Er dient dazu, eine Zifferngruppe
%   in Zweiergruppen auszugeben.
%   Die einzelnen Zahlen werden im Tokenregister \cs{TELtoks}
%   gespeichert. Abwechselnd werden dabei zwischen zwei Token
%   (Zahlen) \cs{TELx} bzw. \cs{TELy} eingefuegt, abh\"angig von dem
%   wechselnden Wert von \cs{TELswitch}. Zum Schluss kann dann einfach
%   festgestellt werden ob die Nummer nun eine geradzahlige oder
%   ungeradzahlige Zahl von Ziffern aufwies. Dem entsprechend wird
%   \cs{TELx} mit dem Zusatzabstand belegt und \cs{TELy} leer definiert
%   oder umgekehrt. )
% \item |\TEL...| interne Befehle, Technisches:\\
%   \cs{TELsplit} dient zur Aufteilung einer zusammengesetzten
%   Telefonnummer (Vorwahl, Hauptnummer, Nebenstelle). In dieser
%   Implementation werden als Trennzeichen nur '/' und '-' erkannt.
%   Die einzelnen Bestandteile wie Vorwahl werden dann dem Befehl
%   \cs{telnumber} zur Formatierung uebergeben.
% \item Die Erkennung von einfachen Leerzeichen ist um einiges
%   schwieriger: Die Tokentrennung ueber Parameter |#1#2| funktioniert
%   nicht f\"ur einfache Leerzeichen, da TeX sie \emph{niemals} als
%   eigenst\"andige Argumente behandelt! (The TeXbook, Chapter 20,
%   p. 201)
%
%   (Anmerkung am Rande: Deshalb funktionieren die entsprechenden
%   Tokenmakros auf S. 149 des Buches "`Einf\"uhrung in TeX"' von
%   N. Schwarz (3. Aufl.) nicht, wenn im Tokenregister als erstes
%   ein einfaches Leerzeichen steht!)
% \end{itemize}
% \end{otherlanguage*}
%
% \StopEventually{
% }
%
% \section{Implementation}
%
%    \begin{macrocode}
%<*package>
%    \end{macrocode}
%
% \subsection{Reload check and package identification}
%    Reload check, especially if the package is not used with \LaTeX.
%    \begin{macrocode}
\begingroup\catcode61\catcode48\catcode32=10\relax%
  \catcode13=5 % ^^M
  \endlinechar=13 %
  \catcode35=6 % #
  \catcode39=12 % '
  \catcode44=12 % ,
  \catcode45=12 % -
  \catcode46=12 % .
  \catcode58=12 % :
  \catcode64=11 % @
  \catcode123=1 % {
  \catcode125=2 % }
  \expandafter\let\expandafter\x\csname ver@telprint.sty\endcsname
  \ifx\x\relax % plain-TeX, first loading
  \else
    \def\empty{}%
    \ifx\x\empty % LaTeX, first loading,
      % variable is initialized, but \ProvidesPackage not yet seen
    \else
      \expandafter\ifx\csname PackageInfo\endcsname\relax
        \def\x#1#2{%
          \immediate\write-1{Package #1 Info: #2.}%
        }%
      \else
        \def\x#1#2{\PackageInfo{#1}{#2, stopped}}%
      \fi
      \x{telprint}{The package is already loaded}%
      \aftergroup\endinput
    \fi
  \fi
\endgroup%
%    \end{macrocode}
%    Package identification:
%    \begin{macrocode}
\begingroup\catcode61\catcode48\catcode32=10\relax%
  \catcode13=5 % ^^M
  \endlinechar=13 %
  \catcode35=6 % #
  \catcode39=12 % '
  \catcode40=12 % (
  \catcode41=12 % )
  \catcode44=12 % ,
  \catcode45=12 % -
  \catcode46=12 % .
  \catcode47=12 % /
  \catcode58=12 % :
  \catcode64=11 % @
  \catcode91=12 % [
  \catcode93=12 % ]
  \catcode123=1 % {
  \catcode125=2 % }
  \expandafter\ifx\csname ProvidesPackage\endcsname\relax
    \def\x#1#2#3[#4]{\endgroup
      \immediate\write-1{Package: #3 #4}%
      \xdef#1{#4}%
    }%
  \else
    \def\x#1#2[#3]{\endgroup
      #2[{#3}]%
      \ifx#1\@undefined
        \xdef#1{#3}%
      \fi
      \ifx#1\relax
        \xdef#1{#3}%
      \fi
    }%
  \fi
\expandafter\x\csname ver@telprint.sty\endcsname
\ProvidesPackage{telprint}%
  [2016/05/16 v1.11 Format German phone numbers (HO)]%
%    \end{macrocode}
%
% \subsection{Catcodes}
%
%    \begin{macrocode}
\begingroup\catcode61\catcode48\catcode32=10\relax%
  \catcode13=5 % ^^M
  \endlinechar=13 %
  \catcode123=1 % {
  \catcode125=2 % }
  \catcode64=11 % @
  \def\x{\endgroup
    \expandafter\edef\csname TELAtEnd\endcsname{%
      \endlinechar=\the\endlinechar\relax
      \catcode13=\the\catcode13\relax
      \catcode32=\the\catcode32\relax
      \catcode35=\the\catcode35\relax
      \catcode61=\the\catcode61\relax
      \catcode64=\the\catcode64\relax
      \catcode123=\the\catcode123\relax
      \catcode125=\the\catcode125\relax
    }%
  }%
\x\catcode61\catcode48\catcode32=10\relax%
\catcode13=5 % ^^M
\endlinechar=13 %
\catcode35=6 % #
\catcode64=11 % @
\catcode123=1 % {
\catcode125=2 % }
\def\TMP@EnsureCode#1#2{%
  \edef\TELAtEnd{%
    \TELAtEnd
    \catcode#1=\the\catcode#1\relax
  }%
  \catcode#1=#2\relax
}
\TMP@EnsureCode{33}{12}% !
\TMP@EnsureCode{36}{3}% $
\TMP@EnsureCode{40}{12}% (
\TMP@EnsureCode{41}{12}% )
\TMP@EnsureCode{42}{12}% *
\TMP@EnsureCode{43}{12}% +
\TMP@EnsureCode{44}{12}% ,
\TMP@EnsureCode{45}{12}% -
\TMP@EnsureCode{46}{12}% .
\TMP@EnsureCode{47}{12}% /
\TMP@EnsureCode{91}{12}% [
\TMP@EnsureCode{93}{12}% ]
\TMP@EnsureCode{126}{13}% ~ (active)
\edef\TELAtEnd{\TELAtEnd\noexpand\endinput}
%    \end{macrocode}
%
% \subsection{Package macros}
%    \begin{macrocode}
\ifx\DeclareRobustCommand\UnDeFiNeD
  \def\DeclareRobustCommand*#1[1]{\def#1##1}%
  \def\TELreset{\let\DeclareRobustCommand=\UnDeFiNeD}%
  \input infwarerr.sty\relax
  \@PackageInfo{telprint}{%
    Macros are not robust!%
  }%
\else
  \let\TELreset=\relax
\fi
%    \end{macrocode}
%    \begin{macro}{\telspace}
%    \begin{macrocode}
\DeclareRobustCommand*{\telspace}[1]{\def\TELspace{#1}}
\telspace{{}$\,${}}
%    \end{macrocode}
%    \end{macro}
%    \begin{macro}{\telhyphen}
%    \begin{macrocode}
\DeclareRobustCommand*{\telhyphen}[1]{\def\TELhyphen{#1}}
\telhyphen{\leavevmode\hbox{-}}% \hbox zur Verhinderung der Trennung
%    \end{macrocode}
%    \end{macro}
%    \begin{macro}{\telslash}
%    \begin{macrocode}
\DeclareRobustCommand*{\telslash}[1]{\def\TELslash{#1}}
\telslash{/}%
%    \end{macrocode}
%    \end{macro}
%    \begin{macro}{\telleftparen}
%    \begin{macrocode}
\DeclareRobustCommand*{\telleftparen}[1]{\def\TELleftparen{#1}}
\telleftparen{(}%
%    \end{macrocode}
%    \end{macro}
%    \begin{macro}{\telrightparen}
%    \begin{macrocode}
\DeclareRobustCommand*{\telrightparen}[1]{\def\TELrightparen{#1}}
\telrightparen{)}%
%    \end{macrocode}
%    \end{macro}
%    \begin{macro}{\telplus}
%    \begin{macrocode}
\DeclareRobustCommand*{\telplus}[1]{\def\TELplus{#1}}
\telplus{+}%
%    \end{macrocode}
%    \end{macro}
%    \begin{macro}{\teltilde}
%    \begin{macrocode}
\DeclareRobustCommand*{\teltilde}[1]{\def\TELtilde{#1}}
\teltilde{~}%
%    \end{macrocode}
%    \end{macro}
%    \begin{macro}{\TELtoks}
%    \begin{macrocode}
\newtoks\TELtoks
%    \end{macrocode}
%    \end{macro}
%    \begin{macro}{\TELnumber}
%    \begin{macrocode}
\def\TELnumber#1#2\TELnumberEND{%
  \begingroup
  \def\0{#2}%
  \expandafter\endgroup
  \ifx\0\empty
    \TELtoks=\expandafter{\the\TELtoks#1}%
    \ifnum\TELswitch=0 %
      \def\TELx{\TELspace}\def\TELy{}%
    \else
      \def\TELx{}\def\TELy{\TELspace}%
    \fi
    \the\TELtoks
  \else
    \ifnum\TELswitch=0 %
      \TELtoks=\expandafter{\the\TELtoks#1\TELx}%
      \def\TELswitch{1}%
    \else
      \TELtoks=\expandafter{\the\TELtoks#1\TELy}%
      \def\TELswitch{0}%
    \fi
    \TELnumber#2\TELnumberEND
  \fi
}
%    \end{macrocode}
%    \end{macro}
%    \begin{macro}{\telnumber}
%    \begin{macrocode}
\DeclareRobustCommand*{\telnumber}[1]{%
  \TELtoks={}%
  \def\TELswitch{0}%
  \TELnumber#1{}\TELnumberEND
}
%    \end{macrocode}
%    \end{macro}
%    \begin{macro}{\TELsplit}
%    \begin{macrocode}
\def\TELsplit{\futurelet\TELfuture\TELdosplit}
%    \end{macrocode}
%    \end{macro}
%    \begin{macro}{\TELdosplit}
%    \begin{macrocode}
\def\TELdosplit#1#2\TELsplitEND
{%
  \def\TELsp{ }%
  \expandafter\ifx\TELsp\TELfuture
    \let\TELfuture=\relax
    \expandafter\telnumber\expandafter{\the\TELtoks}~%
    \telprint{#1#2}% Das Leerzeichen kann nicht #1 sein!
  \else
    \def\TELfirst{#1}%
    \ifx\TELfirst\empty
      \expandafter\telnumber\expandafter{\the\TELtoks}%
      \TELtoks={}%
    \else\if-\TELfirst
      \expandafter\telnumber\expandafter{\the\TELtoks}\TELhyphen
      \telprint{#2}%
    \else\if/\TELfirst
      \expandafter\telnumber\expandafter{\the\TELtoks}\TELslash
      \telprint{#2}%
    \else\if(\TELfirst
      \expandafter\telnumber\expandafter{\the\TELtoks}\TELleftparen
      \telprint{#2}%
    \else\if)\TELfirst
      \expandafter\telnumber\expandafter{\the\TELtoks}\TELrightparen
      \telprint{#2}%
    \else\if+\TELfirst
      \expandafter\telnumber\expandafter{\the\TELtoks}\TELplus
      \telprint{#2}%
    \else\def\TELtemp{~}\ifx\TELtemp\TELfirst
      \expandafter\telnumber\expandafter{\the\TELtoks}\TELtilde
      \telprint{#2}%
    \else
      \TELtoks=\expandafter{\the\TELtoks#1}%
      \TELsplit#2{}\TELsplitEND
    \fi\fi\fi\fi\fi\fi\fi
  \fi
}
%    \end{macrocode}
%    \end{macro}
%    \begin{macro}{\telprint}
%    \begin{macrocode}
\DeclareRobustCommand*{\telprint}[1]{%
  \TELtoks={}%
  \TELsplit#1{}\TELsplitEND
}
%    \end{macrocode}
%    \end{macro}
%    \begin{macrocode}
\TELreset\let\TELreset=\UnDeFiNeD
%    \end{macrocode}
%
%    \begin{macrocode}
\TELAtEnd%
%</package>
%    \end{macrocode}
%
% \section{Test}
%
% \subsection{Catcode checks for loading}
%
%    \begin{macrocode}
%<*test1>
%    \end{macrocode}
%    \begin{macrocode}
\catcode`\{=1 %
\catcode`\}=2 %
\catcode`\#=6 %
\catcode`\@=11 %
\expandafter\ifx\csname count@\endcsname\relax
  \countdef\count@=255 %
\fi
\expandafter\ifx\csname @gobble\endcsname\relax
  \long\def\@gobble#1{}%
\fi
\expandafter\ifx\csname @firstofone\endcsname\relax
  \long\def\@firstofone#1{#1}%
\fi
\expandafter\ifx\csname loop\endcsname\relax
  \expandafter\@firstofone
\else
  \expandafter\@gobble
\fi
{%
  \def\loop#1\repeat{%
    \def\body{#1}%
    \iterate
  }%
  \def\iterate{%
    \body
      \let\next\iterate
    \else
      \let\next\relax
    \fi
    \next
  }%
  \let\repeat=\fi
}%
\def\RestoreCatcodes{}
\count@=0 %
\loop
  \edef\RestoreCatcodes{%
    \RestoreCatcodes
    \catcode\the\count@=\the\catcode\count@\relax
  }%
\ifnum\count@<255 %
  \advance\count@ 1 %
\repeat

\def\RangeCatcodeInvalid#1#2{%
  \count@=#1\relax
  \loop
    \catcode\count@=15 %
  \ifnum\count@<#2\relax
    \advance\count@ 1 %
  \repeat
}
\def\RangeCatcodeCheck#1#2#3{%
  \count@=#1\relax
  \loop
    \ifnum#3=\catcode\count@
    \else
      \errmessage{%
        Character \the\count@\space
        with wrong catcode \the\catcode\count@\space
        instead of \number#3%
      }%
    \fi
  \ifnum\count@<#2\relax
    \advance\count@ 1 %
  \repeat
}
\def\space{ }
\expandafter\ifx\csname LoadCommand\endcsname\relax
  \def\LoadCommand{\input telprint.sty\relax}%
\fi
\def\Test{%
  \RangeCatcodeInvalid{0}{47}%
  \RangeCatcodeInvalid{58}{64}%
  \RangeCatcodeInvalid{91}{96}%
  \RangeCatcodeInvalid{123}{255}%
  \catcode`\@=12 %
  \catcode`\\=0 %
  \catcode`\%=14 %
  \LoadCommand
  \RangeCatcodeCheck{0}{36}{15}%
  \RangeCatcodeCheck{37}{37}{14}%
  \RangeCatcodeCheck{38}{47}{15}%
  \RangeCatcodeCheck{48}{57}{12}%
  \RangeCatcodeCheck{58}{63}{15}%
  \RangeCatcodeCheck{64}{64}{12}%
  \RangeCatcodeCheck{65}{90}{11}%
  \RangeCatcodeCheck{91}{91}{15}%
  \RangeCatcodeCheck{92}{92}{0}%
  \RangeCatcodeCheck{93}{96}{15}%
  \RangeCatcodeCheck{97}{122}{11}%
  \RangeCatcodeCheck{123}{255}{15}%
  \RestoreCatcodes
}
\Test
\csname @@end\endcsname
\end
%    \end{macrocode}
%    \begin{macrocode}
%</test1>
%    \end{macrocode}
%
% \section{Installation}
%
% \subsection{Download}
%
% \paragraph{Package.} This package is available on
% CTAN\footnote{\url{https://ctan.org/pkg/telprint}}:
% \begin{description}
% \item[\CTAN{macros/latex/contrib/oberdiek/telprint.dtx}] The source file.
% \item[\CTAN{macros/latex/contrib/oberdiek/telprint.pdf}] Documentation.
% \end{description}
%
%
% \paragraph{Bundle.} All the packages of the bundle `oberdiek'
% are also available in a TDS compliant ZIP archive. There
% the packages are already unpacked and the documentation files
% are generated. The files and directories obey the TDS standard.
% \begin{description}
% \item[\CTANinstall{install/macros/latex/contrib/oberdiek.tds.zip}]
% \end{description}
% \emph{TDS} refers to the standard ``A Directory Structure
% for \TeX\ Files'' (\CTAN{tds/tds.pdf}). Directories
% with \xfile{texmf} in their name are usually organized this way.
%
% \subsection{Bundle installation}
%
% \paragraph{Unpacking.} Unpack the \xfile{oberdiek.tds.zip} in the
% TDS tree (also known as \xfile{texmf} tree) of your choice.
% Example (linux):
% \begin{quote}
%   |unzip oberdiek.tds.zip -d ~/texmf|
% \end{quote}
%
% \paragraph{Script installation.}
% Check the directory \xfile{TDS:scripts/oberdiek/} for
% scripts that need further installation steps.
% Package \xpackage{attachfile2} comes with the Perl script
% \xfile{pdfatfi.pl} that should be installed in such a way
% that it can be called as \texttt{pdfatfi}.
% Example (linux):
% \begin{quote}
%   |chmod +x scripts/oberdiek/pdfatfi.pl|\\
%   |cp scripts/oberdiek/pdfatfi.pl /usr/local/bin/|
% \end{quote}
%
% \subsection{Package installation}
%
% \paragraph{Unpacking.} The \xfile{.dtx} file is a self-extracting
% \docstrip\ archive. The files are extracted by running the
% \xfile{.dtx} through \plainTeX:
% \begin{quote}
%   \verb|tex telprint.dtx|
% \end{quote}
%
% \paragraph{TDS.} Now the different files must be moved into
% the different directories in your installation TDS tree
% (also known as \xfile{texmf} tree):
% \begin{quote}
% \def\t{^^A
% \begin{tabular}{@{}>{\ttfamily}l@{ $\rightarrow$ }>{\ttfamily}l@{}}
%   telprint.sty & tex/generic/oberdiek/telprint.sty\\
%   telprint.pdf & doc/latex/oberdiek/telprint.pdf\\
%   test/telprint-test1.tex & doc/latex/oberdiek/test/telprint-test1.tex\\
%   telprint.dtx & source/latex/oberdiek/telprint.dtx\\
% \end{tabular}^^A
% }^^A
% \sbox0{\t}^^A
% \ifdim\wd0>\linewidth
%   \begingroup
%     \advance\linewidth by\leftmargin
%     \advance\linewidth by\rightmargin
%   \edef\x{\endgroup
%     \def\noexpand\lw{\the\linewidth}^^A
%   }\x
%   \def\lwbox{^^A
%     \leavevmode
%     \hbox to \linewidth{^^A
%       \kern-\leftmargin\relax
%       \hss
%       \usebox0
%       \hss
%       \kern-\rightmargin\relax
%     }^^A
%   }^^A
%   \ifdim\wd0>\lw
%     \sbox0{\small\t}^^A
%     \ifdim\wd0>\linewidth
%       \ifdim\wd0>\lw
%         \sbox0{\footnotesize\t}^^A
%         \ifdim\wd0>\linewidth
%           \ifdim\wd0>\lw
%             \sbox0{\scriptsize\t}^^A
%             \ifdim\wd0>\linewidth
%               \ifdim\wd0>\lw
%                 \sbox0{\tiny\t}^^A
%                 \ifdim\wd0>\linewidth
%                   \lwbox
%                 \else
%                   \usebox0
%                 \fi
%               \else
%                 \lwbox
%               \fi
%             \else
%               \usebox0
%             \fi
%           \else
%             \lwbox
%           \fi
%         \else
%           \usebox0
%         \fi
%       \else
%         \lwbox
%       \fi
%     \else
%       \usebox0
%     \fi
%   \else
%     \lwbox
%   \fi
% \else
%   \usebox0
% \fi
% \end{quote}
% If you have a \xfile{docstrip.cfg} that configures and enables \docstrip's
% TDS installing feature, then some files can already be in the right
% place, see the documentation of \docstrip.
%
% \subsection{Refresh file name databases}
%
% If your \TeX~distribution
% (\teTeX, \mikTeX, \dots) relies on file name databases, you must refresh
% these. For example, \teTeX\ users run \verb|texhash| or
% \verb|mktexlsr|.
%
% \subsection{Some details for the interested}
%
% \paragraph{Attached source.}
%
% The PDF documentation on CTAN also includes the
% \xfile{.dtx} source file. It can be extracted by
% AcrobatReader 6 or higher. Another option is \textsf{pdftk},
% e.g. unpack the file into the current directory:
% \begin{quote}
%   \verb|pdftk telprint.pdf unpack_files output .|
% \end{quote}
%
% \paragraph{Unpacking with \LaTeX.}
% The \xfile{.dtx} chooses its action depending on the format:
% \begin{description}
% \item[\plainTeX:] Run \docstrip\ and extract the files.
% \item[\LaTeX:] Generate the documentation.
% \end{description}
% If you insist on using \LaTeX\ for \docstrip\ (really,
% \docstrip\ does not need \LaTeX), then inform the autodetect routine
% about your intention:
% \begin{quote}
%   \verb|latex \let\install=y% \iffalse meta-comment
%
% File: telprint.dtx
% Version: 2016/05/16 v1.11
% Info: Format German phone numbers
%
% Copyright (C) 1996, 1997, 2004-2008 by
%    Heiko Oberdiek <heiko.oberdiek at googlemail.com>
%    2016
%    https://github.com/ho-tex/oberdiek/issues
%
% This work may be distributed and/or modified under the
% conditions of the LaTeX Project Public License, either
% version 1.3c of this license or (at your option) any later
% version. This version of this license is in
%    http://www.latex-project.org/lppl/lppl-1-3c.txt
% and the latest version of this license is in
%    http://www.latex-project.org/lppl.txt
% and version 1.3 or later is part of all distributions of
% LaTeX version 2005/12/01 or later.
%
% This work has the LPPL maintenance status "maintained".
%
% This Current Maintainer of this work is Heiko Oberdiek.
%
% The Base Interpreter refers to any `TeX-Format',
% because some files are installed in TDS:tex/generic//.
%
% This work consists of the main source file telprint.dtx
% and the derived files
%    telprint.sty, telprint.pdf, telprint.ins, telprint.drv,
%    telprint-test1.tex.
%
% Distribution:
%    CTAN:macros/latex/contrib/oberdiek/telprint.dtx
%    CTAN:macros/latex/contrib/oberdiek/telprint.pdf
%
% Unpacking:
%    (a) If telprint.ins is present:
%           tex telprint.ins
%    (b) Without telprint.ins:
%           tex telprint.dtx
%    (c) If you insist on using LaTeX
%           latex \let\install=y\input{telprint.dtx}
%        (quote the arguments according to the demands of your shell)
%
% Documentation:
%    (a) If telprint.drv is present:
%           latex telprint.drv
%    (b) Without telprint.drv:
%           latex telprint.dtx; ...
%    The class ltxdoc loads the configuration file ltxdoc.cfg
%    if available. Here you can specify further options, e.g.
%    use A4 as paper format:
%       \PassOptionsToClass{a4paper}{article}
%
%    Programm calls to get the documentation (example):
%       pdflatex telprint.dtx
%       makeindex -s gind.ist telprint.idx
%       pdflatex telprint.dtx
%       makeindex -s gind.ist telprint.idx
%       pdflatex telprint.dtx
%
% Installation:
%    TDS:tex/generic/oberdiek/telprint.sty
%    TDS:doc/latex/oberdiek/telprint.pdf
%    TDS:doc/latex/oberdiek/test/telprint-test1.tex
%    TDS:source/latex/oberdiek/telprint.dtx
%
%<*ignore>
\begingroup
  \catcode123=1 %
  \catcode125=2 %
  \def\x{LaTeX2e}%
\expandafter\endgroup
\ifcase 0\ifx\install y1\fi\expandafter
         \ifx\csname processbatchFile\endcsname\relax\else1\fi
         \ifx\fmtname\x\else 1\fi\relax
\else\csname fi\endcsname
%</ignore>
%<*install>
\input docstrip.tex
\Msg{************************************************************************}
\Msg{* Installation}
\Msg{* Package: telprint 2016/05/16 v1.11 Format German phone numbers (HO)}
\Msg{************************************************************************}

\keepsilent
\askforoverwritefalse

\let\MetaPrefix\relax
\preamble

This is a generated file.

Project: telprint
Version: 2016/05/16 v1.11

Copyright (C) 1996, 1997, 2004-2008 by
   Heiko Oberdiek <heiko.oberdiek at googlemail.com>

This work may be distributed and/or modified under the
conditions of the LaTeX Project Public License, either
version 1.3c of this license or (at your option) any later
version. This version of this license is in
   http://www.latex-project.org/lppl/lppl-1-3c.txt
and the latest version of this license is in
   http://www.latex-project.org/lppl.txt
and version 1.3 or later is part of all distributions of
LaTeX version 2005/12/01 or later.

This work has the LPPL maintenance status "maintained".

This Current Maintainer of this work is Heiko Oberdiek.

The Base Interpreter refers to any `TeX-Format',
because some files are installed in TDS:tex/generic//.

This work consists of the main source file telprint.dtx
and the derived files
   telprint.sty, telprint.pdf, telprint.ins, telprint.drv,
   telprint-test1.tex.

\endpreamble
\let\MetaPrefix\DoubleperCent

\generate{%
  \file{telprint.ins}{\from{telprint.dtx}{install}}%
  \file{telprint.drv}{\from{telprint.dtx}{driver}}%
  \usedir{tex/generic/oberdiek}%
  \file{telprint.sty}{\from{telprint.dtx}{package}}%
%  \usedir{doc/latex/oberdiek/test}%
%  \file{telprint-test1.tex}{\from{telprint.dtx}{test1}}%
  \nopreamble
  \nopostamble
%  \usedir{source/latex/oberdiek/catalogue}%
%  \file{telprint.xml}{\from{telprint.dtx}{catalogue}}%
}

\catcode32=13\relax% active space
\let =\space%
\Msg{************************************************************************}
\Msg{*}
\Msg{* To finish the installation you have to move the following}
\Msg{* file into a directory searched by TeX:}
\Msg{*}
\Msg{*     telprint.sty}
\Msg{*}
\Msg{* To produce the documentation run the file `telprint.drv'}
\Msg{* through LaTeX.}
\Msg{*}
\Msg{* Happy TeXing!}
\Msg{*}
\Msg{************************************************************************}

\endbatchfile
%</install>
%<*ignore>
\fi
%</ignore>
%<*driver>
\NeedsTeXFormat{LaTeX2e}
\ProvidesFile{telprint.drv}%
  [2016/05/16 v1.11 Format German phone numbers (HO)]%
\documentclass{ltxdoc}
\usepackage{holtxdoc}[2011/11/22]
\usepackage[ngerman,english]{babel}
\begin{document}
  \DocInput{telprint.dtx}%
\end{document}
%</driver>
% \fi
%
%
% \CharacterTable
%  {Upper-case    \A\B\C\D\E\F\G\H\I\J\K\L\M\N\O\P\Q\R\S\T\U\V\W\X\Y\Z
%   Lower-case    \a\b\c\d\e\f\g\h\i\j\k\l\m\n\o\p\q\r\s\t\u\v\w\x\y\z
%   Digits        \0\1\2\3\4\5\6\7\8\9
%   Exclamation   \!     Double quote  \"     Hash (number) \#
%   Dollar        \$     Percent       \%     Ampersand     \&
%   Acute accent  \'     Left paren    \(     Right paren   \)
%   Asterisk      \*     Plus          \+     Comma         \,
%   Minus         \-     Point         \.     Solidus       \/
%   Colon         \:     Semicolon     \;     Less than     \<
%   Equals        \=     Greater than  \>     Question mark \?
%   Commercial at \@     Left bracket  \[     Backslash     \\
%   Right bracket \]     Circumflex    \^     Underscore    \_
%   Grave accent  \`     Left brace    \{     Vertical bar  \|
%   Right brace   \}     Tilde         \~}
%
% \GetFileInfo{telprint.drv}
%
% \title{The \xpackage{telprint} package}
% \date{2016/05/16 v1.11}
% \author{Heiko Oberdiek\thanks
% {Please report any issues at https://github.com/ho-tex/oberdiek/issues}\\
% \xemail{heiko.oberdiek at googlemail.com}}
%
% \maketitle
%
% \begin{abstract}
% Package \xpackage{telprint} provides \cs{telprint} for formatting
% German phone numbers.
% \end{abstract}
%
% \tableofcontents
%
% \section{Documentation}
%
% \subsection{Introduction}
%
%            This is a very old package that I have written
%            to format phone numbers. It follows German
%            conventions and the documentation is mainly in German.
%
% \subsection{Short overview in English}
%
% \LaTeX:
% \begin{quote}
% |\usepackage{telprint}|\\
% |\telprint{123/456-789}|\\
% \end{quote}
% \plainTeX:
% \begin{quote}
%   |\input telprint.sty|\\
%   |\telprint{123/456-789}|
% \end{quote}
%
% \DescribeMacro\telprint
% |\telprint{...}| formats the explicitly given number.
%     Digits, spaces and some special characters
%     ('+', '/', '-', '(', ')', '\textasciitilde', ' ') are supported.
%     Numbers are divided into groups of two digits from the right.
% Examples:
% \begin{quote}
%     |\telprint{0761/12345}     ==> 07\,61/1\,23\,45|\\
%     |\telprint{01234/567-89}   ==> 0\,12\,34/5\,67\leavevmode\hbox{-}89|\\
%     |\telprint{+49 (6221) 297} ==> +49~(62\,21)~2\,97|
% \end{quote}
%
% \subsubsection{Configuration}
%
% The output of the symbols can be configured by
% \cs{telhyphen}, \cs{telslash}, \cs{telleftparen}, \cs{telrightparen},
% \cs{telplus}, \cs{teltilde}.
% Example:
% \begin{quote}
%   |\telslash{\,/\,}\\|
%   |\telprint{12/34} ==> 12\,/\,34|
% \end{quote}
%
% \DescribeMacro\telspace
% \cs{telspace} configures the space between digit groups.
%
% \DescribeMacro\telnumber
% \cs{telnumber} only formats a number in digit groups; special
%    characters are not recognized.
%
% \subsection{Documentation in German}
%
% \begin{otherlanguage*}{ngerman}
% \hyphenation{To-ken-ma-kros}
% \begin{itemize}
% \item \DescribeMacro\telprint |telprint#1|\\
%   Der eigentliche Anwenderbefehl zur formatierten Ausgabe von
%   Telefonnummern. Diese d\"urfen dabei nur als Zahlen angegeben
%   werden(, da sie tokenweise analysiert werden).
%   Als Trenn- oder Sonderzeichen werden unterst\"utzt:
%   '+', '/', '-', '(', ')', '\textasciitilde', ' '
%   Einfache Leerzeichen werden erkannt und durch Tilden ersetzt, um
%   Trennungen in der Telefonnummer zu verhindern. (Man beachte aus
%   gleichem Grunde die \cs{hbox} bei '-'.)
%   Beispiele:
%   \begin{quote}
%     |\telprint{0761/12345}     ==> 07\,61/1\,23\,45|\\
%     |\telprint{01234/567-89}   ==> 0\,12\,34/5\,67\leavevmode\hbox{-}89|\\
%     |\telprint{+49 (6221) 297} ==> +49~(62\,21)~2\,97|
%   \end{quote}
% \end{itemize}
% Der Rest enth\"alt eher Technisches:
% \begin{itemize}
% \item \DescribeMacro\telspace |\telspace#1|\\
%   Mit diesem Befehl wird der Abstand zwischen den Zifferngruppen
%   angegeben (Default: |\,|).
%   (Durch |\telspace{}| kann dieser zusaetzliche Abstand abgestellt
%   werden.)
% \item \DescribeMacro\telhyphen |\telhyphen#1|\\
%   Dieser Befehl gibt die Art des Bindestriches, wie er ausgegeben
%   werden soll. In der Eingabe darf jedoch nur der einfache
%   Bindestrich stehen:
%   |\telprint{123-45}|, jedoch NIE |\telprint{123--45}|!
%   Kopka-Bindestrich-Fans geben an:
%   |\telhyphen{\leavevmode\hbox{--}}|
% \item
%   \DescribeMacro{\telslash}
%   \DescribeMacro{\telleftparen}
%   \DescribeMacro{\telrightparen}
%   \DescribeMacro{\telplus}
%   \DescribeMacro{\teltilde}
%   |\telslash#1|, |\telleftparen#1|, |\telrightparen#1|, |\telplus#1|,
%   |\teltilde|\\
%   Diese Befehle konfigurieren die Zeichen '/', '(', ')', '+'
%   und '\textasciitilde'. Sie funktionieren analog zu \cs{telhyphen}.
% \item \DescribeMacro\telnumber |\telnumber#1|\\
%   Richtung interner Befehl: Er dient dazu, eine Zifferngruppe
%   in Zweiergruppen auszugeben.
%   Die einzelnen Zahlen werden im Tokenregister \cs{TELtoks}
%   gespeichert. Abwechselnd werden dabei zwischen zwei Token
%   (Zahlen) \cs{TELx} bzw. \cs{TELy} eingefuegt, abh\"angig von dem
%   wechselnden Wert von \cs{TELswitch}. Zum Schluss kann dann einfach
%   festgestellt werden ob die Nummer nun eine geradzahlige oder
%   ungeradzahlige Zahl von Ziffern aufwies. Dem entsprechend wird
%   \cs{TELx} mit dem Zusatzabstand belegt und \cs{TELy} leer definiert
%   oder umgekehrt. )
% \item |\TEL...| interne Befehle, Technisches:\\
%   \cs{TELsplit} dient zur Aufteilung einer zusammengesetzten
%   Telefonnummer (Vorwahl, Hauptnummer, Nebenstelle). In dieser
%   Implementation werden als Trennzeichen nur '/' und '-' erkannt.
%   Die einzelnen Bestandteile wie Vorwahl werden dann dem Befehl
%   \cs{telnumber} zur Formatierung uebergeben.
% \item Die Erkennung von einfachen Leerzeichen ist um einiges
%   schwieriger: Die Tokentrennung ueber Parameter |#1#2| funktioniert
%   nicht f\"ur einfache Leerzeichen, da TeX sie \emph{niemals} als
%   eigenst\"andige Argumente behandelt! (The TeXbook, Chapter 20,
%   p. 201)
%
%   (Anmerkung am Rande: Deshalb funktionieren die entsprechenden
%   Tokenmakros auf S. 149 des Buches "`Einf\"uhrung in TeX"' von
%   N. Schwarz (3. Aufl.) nicht, wenn im Tokenregister als erstes
%   ein einfaches Leerzeichen steht!)
% \end{itemize}
% \end{otherlanguage*}
%
% \StopEventually{
% }
%
% \section{Implementation}
%
%    \begin{macrocode}
%<*package>
%    \end{macrocode}
%
% \subsection{Reload check and package identification}
%    Reload check, especially if the package is not used with \LaTeX.
%    \begin{macrocode}
\begingroup\catcode61\catcode48\catcode32=10\relax%
  \catcode13=5 % ^^M
  \endlinechar=13 %
  \catcode35=6 % #
  \catcode39=12 % '
  \catcode44=12 % ,
  \catcode45=12 % -
  \catcode46=12 % .
  \catcode58=12 % :
  \catcode64=11 % @
  \catcode123=1 % {
  \catcode125=2 % }
  \expandafter\let\expandafter\x\csname ver@telprint.sty\endcsname
  \ifx\x\relax % plain-TeX, first loading
  \else
    \def\empty{}%
    \ifx\x\empty % LaTeX, first loading,
      % variable is initialized, but \ProvidesPackage not yet seen
    \else
      \expandafter\ifx\csname PackageInfo\endcsname\relax
        \def\x#1#2{%
          \immediate\write-1{Package #1 Info: #2.}%
        }%
      \else
        \def\x#1#2{\PackageInfo{#1}{#2, stopped}}%
      \fi
      \x{telprint}{The package is already loaded}%
      \aftergroup\endinput
    \fi
  \fi
\endgroup%
%    \end{macrocode}
%    Package identification:
%    \begin{macrocode}
\begingroup\catcode61\catcode48\catcode32=10\relax%
  \catcode13=5 % ^^M
  \endlinechar=13 %
  \catcode35=6 % #
  \catcode39=12 % '
  \catcode40=12 % (
  \catcode41=12 % )
  \catcode44=12 % ,
  \catcode45=12 % -
  \catcode46=12 % .
  \catcode47=12 % /
  \catcode58=12 % :
  \catcode64=11 % @
  \catcode91=12 % [
  \catcode93=12 % ]
  \catcode123=1 % {
  \catcode125=2 % }
  \expandafter\ifx\csname ProvidesPackage\endcsname\relax
    \def\x#1#2#3[#4]{\endgroup
      \immediate\write-1{Package: #3 #4}%
      \xdef#1{#4}%
    }%
  \else
    \def\x#1#2[#3]{\endgroup
      #2[{#3}]%
      \ifx#1\@undefined
        \xdef#1{#3}%
      \fi
      \ifx#1\relax
        \xdef#1{#3}%
      \fi
    }%
  \fi
\expandafter\x\csname ver@telprint.sty\endcsname
\ProvidesPackage{telprint}%
  [2016/05/16 v1.11 Format German phone numbers (HO)]%
%    \end{macrocode}
%
% \subsection{Catcodes}
%
%    \begin{macrocode}
\begingroup\catcode61\catcode48\catcode32=10\relax%
  \catcode13=5 % ^^M
  \endlinechar=13 %
  \catcode123=1 % {
  \catcode125=2 % }
  \catcode64=11 % @
  \def\x{\endgroup
    \expandafter\edef\csname TELAtEnd\endcsname{%
      \endlinechar=\the\endlinechar\relax
      \catcode13=\the\catcode13\relax
      \catcode32=\the\catcode32\relax
      \catcode35=\the\catcode35\relax
      \catcode61=\the\catcode61\relax
      \catcode64=\the\catcode64\relax
      \catcode123=\the\catcode123\relax
      \catcode125=\the\catcode125\relax
    }%
  }%
\x\catcode61\catcode48\catcode32=10\relax%
\catcode13=5 % ^^M
\endlinechar=13 %
\catcode35=6 % #
\catcode64=11 % @
\catcode123=1 % {
\catcode125=2 % }
\def\TMP@EnsureCode#1#2{%
  \edef\TELAtEnd{%
    \TELAtEnd
    \catcode#1=\the\catcode#1\relax
  }%
  \catcode#1=#2\relax
}
\TMP@EnsureCode{33}{12}% !
\TMP@EnsureCode{36}{3}% $
\TMP@EnsureCode{40}{12}% (
\TMP@EnsureCode{41}{12}% )
\TMP@EnsureCode{42}{12}% *
\TMP@EnsureCode{43}{12}% +
\TMP@EnsureCode{44}{12}% ,
\TMP@EnsureCode{45}{12}% -
\TMP@EnsureCode{46}{12}% .
\TMP@EnsureCode{47}{12}% /
\TMP@EnsureCode{91}{12}% [
\TMP@EnsureCode{93}{12}% ]
\TMP@EnsureCode{126}{13}% ~ (active)
\edef\TELAtEnd{\TELAtEnd\noexpand\endinput}
%    \end{macrocode}
%
% \subsection{Package macros}
%    \begin{macrocode}
\ifx\DeclareRobustCommand\UnDeFiNeD
  \def\DeclareRobustCommand*#1[1]{\def#1##1}%
  \def\TELreset{\let\DeclareRobustCommand=\UnDeFiNeD}%
  \input infwarerr.sty\relax
  \@PackageInfo{telprint}{%
    Macros are not robust!%
  }%
\else
  \let\TELreset=\relax
\fi
%    \end{macrocode}
%    \begin{macro}{\telspace}
%    \begin{macrocode}
\DeclareRobustCommand*{\telspace}[1]{\def\TELspace{#1}}
\telspace{{}$\,${}}
%    \end{macrocode}
%    \end{macro}
%    \begin{macro}{\telhyphen}
%    \begin{macrocode}
\DeclareRobustCommand*{\telhyphen}[1]{\def\TELhyphen{#1}}
\telhyphen{\leavevmode\hbox{-}}% \hbox zur Verhinderung der Trennung
%    \end{macrocode}
%    \end{macro}
%    \begin{macro}{\telslash}
%    \begin{macrocode}
\DeclareRobustCommand*{\telslash}[1]{\def\TELslash{#1}}
\telslash{/}%
%    \end{macrocode}
%    \end{macro}
%    \begin{macro}{\telleftparen}
%    \begin{macrocode}
\DeclareRobustCommand*{\telleftparen}[1]{\def\TELleftparen{#1}}
\telleftparen{(}%
%    \end{macrocode}
%    \end{macro}
%    \begin{macro}{\telrightparen}
%    \begin{macrocode}
\DeclareRobustCommand*{\telrightparen}[1]{\def\TELrightparen{#1}}
\telrightparen{)}%
%    \end{macrocode}
%    \end{macro}
%    \begin{macro}{\telplus}
%    \begin{macrocode}
\DeclareRobustCommand*{\telplus}[1]{\def\TELplus{#1}}
\telplus{+}%
%    \end{macrocode}
%    \end{macro}
%    \begin{macro}{\teltilde}
%    \begin{macrocode}
\DeclareRobustCommand*{\teltilde}[1]{\def\TELtilde{#1}}
\teltilde{~}%
%    \end{macrocode}
%    \end{macro}
%    \begin{macro}{\TELtoks}
%    \begin{macrocode}
\newtoks\TELtoks
%    \end{macrocode}
%    \end{macro}
%    \begin{macro}{\TELnumber}
%    \begin{macrocode}
\def\TELnumber#1#2\TELnumberEND{%
  \begingroup
  \def\0{#2}%
  \expandafter\endgroup
  \ifx\0\empty
    \TELtoks=\expandafter{\the\TELtoks#1}%
    \ifnum\TELswitch=0 %
      \def\TELx{\TELspace}\def\TELy{}%
    \else
      \def\TELx{}\def\TELy{\TELspace}%
    \fi
    \the\TELtoks
  \else
    \ifnum\TELswitch=0 %
      \TELtoks=\expandafter{\the\TELtoks#1\TELx}%
      \def\TELswitch{1}%
    \else
      \TELtoks=\expandafter{\the\TELtoks#1\TELy}%
      \def\TELswitch{0}%
    \fi
    \TELnumber#2\TELnumberEND
  \fi
}
%    \end{macrocode}
%    \end{macro}
%    \begin{macro}{\telnumber}
%    \begin{macrocode}
\DeclareRobustCommand*{\telnumber}[1]{%
  \TELtoks={}%
  \def\TELswitch{0}%
  \TELnumber#1{}\TELnumberEND
}
%    \end{macrocode}
%    \end{macro}
%    \begin{macro}{\TELsplit}
%    \begin{macrocode}
\def\TELsplit{\futurelet\TELfuture\TELdosplit}
%    \end{macrocode}
%    \end{macro}
%    \begin{macro}{\TELdosplit}
%    \begin{macrocode}
\def\TELdosplit#1#2\TELsplitEND
{%
  \def\TELsp{ }%
  \expandafter\ifx\TELsp\TELfuture
    \let\TELfuture=\relax
    \expandafter\telnumber\expandafter{\the\TELtoks}~%
    \telprint{#1#2}% Das Leerzeichen kann nicht #1 sein!
  \else
    \def\TELfirst{#1}%
    \ifx\TELfirst\empty
      \expandafter\telnumber\expandafter{\the\TELtoks}%
      \TELtoks={}%
    \else\if-\TELfirst
      \expandafter\telnumber\expandafter{\the\TELtoks}\TELhyphen
      \telprint{#2}%
    \else\if/\TELfirst
      \expandafter\telnumber\expandafter{\the\TELtoks}\TELslash
      \telprint{#2}%
    \else\if(\TELfirst
      \expandafter\telnumber\expandafter{\the\TELtoks}\TELleftparen
      \telprint{#2}%
    \else\if)\TELfirst
      \expandafter\telnumber\expandafter{\the\TELtoks}\TELrightparen
      \telprint{#2}%
    \else\if+\TELfirst
      \expandafter\telnumber\expandafter{\the\TELtoks}\TELplus
      \telprint{#2}%
    \else\def\TELtemp{~}\ifx\TELtemp\TELfirst
      \expandafter\telnumber\expandafter{\the\TELtoks}\TELtilde
      \telprint{#2}%
    \else
      \TELtoks=\expandafter{\the\TELtoks#1}%
      \TELsplit#2{}\TELsplitEND
    \fi\fi\fi\fi\fi\fi\fi
  \fi
}
%    \end{macrocode}
%    \end{macro}
%    \begin{macro}{\telprint}
%    \begin{macrocode}
\DeclareRobustCommand*{\telprint}[1]{%
  \TELtoks={}%
  \TELsplit#1{}\TELsplitEND
}
%    \end{macrocode}
%    \end{macro}
%    \begin{macrocode}
\TELreset\let\TELreset=\UnDeFiNeD
%    \end{macrocode}
%
%    \begin{macrocode}
\TELAtEnd%
%</package>
%    \end{macrocode}
%
% \section{Test}
%
% \subsection{Catcode checks for loading}
%
%    \begin{macrocode}
%<*test1>
%    \end{macrocode}
%    \begin{macrocode}
\catcode`\{=1 %
\catcode`\}=2 %
\catcode`\#=6 %
\catcode`\@=11 %
\expandafter\ifx\csname count@\endcsname\relax
  \countdef\count@=255 %
\fi
\expandafter\ifx\csname @gobble\endcsname\relax
  \long\def\@gobble#1{}%
\fi
\expandafter\ifx\csname @firstofone\endcsname\relax
  \long\def\@firstofone#1{#1}%
\fi
\expandafter\ifx\csname loop\endcsname\relax
  \expandafter\@firstofone
\else
  \expandafter\@gobble
\fi
{%
  \def\loop#1\repeat{%
    \def\body{#1}%
    \iterate
  }%
  \def\iterate{%
    \body
      \let\next\iterate
    \else
      \let\next\relax
    \fi
    \next
  }%
  \let\repeat=\fi
}%
\def\RestoreCatcodes{}
\count@=0 %
\loop
  \edef\RestoreCatcodes{%
    \RestoreCatcodes
    \catcode\the\count@=\the\catcode\count@\relax
  }%
\ifnum\count@<255 %
  \advance\count@ 1 %
\repeat

\def\RangeCatcodeInvalid#1#2{%
  \count@=#1\relax
  \loop
    \catcode\count@=15 %
  \ifnum\count@<#2\relax
    \advance\count@ 1 %
  \repeat
}
\def\RangeCatcodeCheck#1#2#3{%
  \count@=#1\relax
  \loop
    \ifnum#3=\catcode\count@
    \else
      \errmessage{%
        Character \the\count@\space
        with wrong catcode \the\catcode\count@\space
        instead of \number#3%
      }%
    \fi
  \ifnum\count@<#2\relax
    \advance\count@ 1 %
  \repeat
}
\def\space{ }
\expandafter\ifx\csname LoadCommand\endcsname\relax
  \def\LoadCommand{\input telprint.sty\relax}%
\fi
\def\Test{%
  \RangeCatcodeInvalid{0}{47}%
  \RangeCatcodeInvalid{58}{64}%
  \RangeCatcodeInvalid{91}{96}%
  \RangeCatcodeInvalid{123}{255}%
  \catcode`\@=12 %
  \catcode`\\=0 %
  \catcode`\%=14 %
  \LoadCommand
  \RangeCatcodeCheck{0}{36}{15}%
  \RangeCatcodeCheck{37}{37}{14}%
  \RangeCatcodeCheck{38}{47}{15}%
  \RangeCatcodeCheck{48}{57}{12}%
  \RangeCatcodeCheck{58}{63}{15}%
  \RangeCatcodeCheck{64}{64}{12}%
  \RangeCatcodeCheck{65}{90}{11}%
  \RangeCatcodeCheck{91}{91}{15}%
  \RangeCatcodeCheck{92}{92}{0}%
  \RangeCatcodeCheck{93}{96}{15}%
  \RangeCatcodeCheck{97}{122}{11}%
  \RangeCatcodeCheck{123}{255}{15}%
  \RestoreCatcodes
}
\Test
\csname @@end\endcsname
\end
%    \end{macrocode}
%    \begin{macrocode}
%</test1>
%    \end{macrocode}
%
% \section{Installation}
%
% \subsection{Download}
%
% \paragraph{Package.} This package is available on
% CTAN\footnote{\url{https://ctan.org/pkg/telprint}}:
% \begin{description}
% \item[\CTAN{macros/latex/contrib/oberdiek/telprint.dtx}] The source file.
% \item[\CTAN{macros/latex/contrib/oberdiek/telprint.pdf}] Documentation.
% \end{description}
%
%
% \paragraph{Bundle.} All the packages of the bundle `oberdiek'
% are also available in a TDS compliant ZIP archive. There
% the packages are already unpacked and the documentation files
% are generated. The files and directories obey the TDS standard.
% \begin{description}
% \item[\CTANinstall{install/macros/latex/contrib/oberdiek.tds.zip}]
% \end{description}
% \emph{TDS} refers to the standard ``A Directory Structure
% for \TeX\ Files'' (\CTAN{tds/tds.pdf}). Directories
% with \xfile{texmf} in their name are usually organized this way.
%
% \subsection{Bundle installation}
%
% \paragraph{Unpacking.} Unpack the \xfile{oberdiek.tds.zip} in the
% TDS tree (also known as \xfile{texmf} tree) of your choice.
% Example (linux):
% \begin{quote}
%   |unzip oberdiek.tds.zip -d ~/texmf|
% \end{quote}
%
% \paragraph{Script installation.}
% Check the directory \xfile{TDS:scripts/oberdiek/} for
% scripts that need further installation steps.
% Package \xpackage{attachfile2} comes with the Perl script
% \xfile{pdfatfi.pl} that should be installed in such a way
% that it can be called as \texttt{pdfatfi}.
% Example (linux):
% \begin{quote}
%   |chmod +x scripts/oberdiek/pdfatfi.pl|\\
%   |cp scripts/oberdiek/pdfatfi.pl /usr/local/bin/|
% \end{quote}
%
% \subsection{Package installation}
%
% \paragraph{Unpacking.} The \xfile{.dtx} file is a self-extracting
% \docstrip\ archive. The files are extracted by running the
% \xfile{.dtx} through \plainTeX:
% \begin{quote}
%   \verb|tex telprint.dtx|
% \end{quote}
%
% \paragraph{TDS.} Now the different files must be moved into
% the different directories in your installation TDS tree
% (also known as \xfile{texmf} tree):
% \begin{quote}
% \def\t{^^A
% \begin{tabular}{@{}>{\ttfamily}l@{ $\rightarrow$ }>{\ttfamily}l@{}}
%   telprint.sty & tex/generic/oberdiek/telprint.sty\\
%   telprint.pdf & doc/latex/oberdiek/telprint.pdf\\
%   test/telprint-test1.tex & doc/latex/oberdiek/test/telprint-test1.tex\\
%   telprint.dtx & source/latex/oberdiek/telprint.dtx\\
% \end{tabular}^^A
% }^^A
% \sbox0{\t}^^A
% \ifdim\wd0>\linewidth
%   \begingroup
%     \advance\linewidth by\leftmargin
%     \advance\linewidth by\rightmargin
%   \edef\x{\endgroup
%     \def\noexpand\lw{\the\linewidth}^^A
%   }\x
%   \def\lwbox{^^A
%     \leavevmode
%     \hbox to \linewidth{^^A
%       \kern-\leftmargin\relax
%       \hss
%       \usebox0
%       \hss
%       \kern-\rightmargin\relax
%     }^^A
%   }^^A
%   \ifdim\wd0>\lw
%     \sbox0{\small\t}^^A
%     \ifdim\wd0>\linewidth
%       \ifdim\wd0>\lw
%         \sbox0{\footnotesize\t}^^A
%         \ifdim\wd0>\linewidth
%           \ifdim\wd0>\lw
%             \sbox0{\scriptsize\t}^^A
%             \ifdim\wd0>\linewidth
%               \ifdim\wd0>\lw
%                 \sbox0{\tiny\t}^^A
%                 \ifdim\wd0>\linewidth
%                   \lwbox
%                 \else
%                   \usebox0
%                 \fi
%               \else
%                 \lwbox
%               \fi
%             \else
%               \usebox0
%             \fi
%           \else
%             \lwbox
%           \fi
%         \else
%           \usebox0
%         \fi
%       \else
%         \lwbox
%       \fi
%     \else
%       \usebox0
%     \fi
%   \else
%     \lwbox
%   \fi
% \else
%   \usebox0
% \fi
% \end{quote}
% If you have a \xfile{docstrip.cfg} that configures and enables \docstrip's
% TDS installing feature, then some files can already be in the right
% place, see the documentation of \docstrip.
%
% \subsection{Refresh file name databases}
%
% If your \TeX~distribution
% (\teTeX, \mikTeX, \dots) relies on file name databases, you must refresh
% these. For example, \teTeX\ users run \verb|texhash| or
% \verb|mktexlsr|.
%
% \subsection{Some details for the interested}
%
% \paragraph{Attached source.}
%
% The PDF documentation on CTAN also includes the
% \xfile{.dtx} source file. It can be extracted by
% AcrobatReader 6 or higher. Another option is \textsf{pdftk},
% e.g. unpack the file into the current directory:
% \begin{quote}
%   \verb|pdftk telprint.pdf unpack_files output .|
% \end{quote}
%
% \paragraph{Unpacking with \LaTeX.}
% The \xfile{.dtx} chooses its action depending on the format:
% \begin{description}
% \item[\plainTeX:] Run \docstrip\ and extract the files.
% \item[\LaTeX:] Generate the documentation.
% \end{description}
% If you insist on using \LaTeX\ for \docstrip\ (really,
% \docstrip\ does not need \LaTeX), then inform the autodetect routine
% about your intention:
% \begin{quote}
%   \verb|latex \let\install=y\input{telprint.dtx}|
% \end{quote}
% Do not forget to quote the argument according to the demands
% of your shell.
%
% \paragraph{Generating the documentation.}
% You can use both the \xfile{.dtx} or the \xfile{.drv} to generate
% the documentation. The process can be configured by the
% configuration file \xfile{ltxdoc.cfg}. For instance, put this
% line into this file, if you want to have A4 as paper format:
% \begin{quote}
%   \verb|\PassOptionsToClass{a4paper}{article}|
% \end{quote}
% An example follows how to generate the
% documentation with pdf\LaTeX:
% \begin{quote}
%\begin{verbatim}
%pdflatex telprint.dtx
%makeindex -s gind.ist telprint.idx
%pdflatex telprint.dtx
%makeindex -s gind.ist telprint.idx
%pdflatex telprint.dtx
%\end{verbatim}
% \end{quote}
%
% \begin{History}
%   \begin{Version}{1996/11/28 v1.0}
%   \item
%     Erste lauff\"ahige Version.
%   \item
%     Nur '-' und '/' als zul\"assige Sonderzeichen.
%   \end{Version}
%   \begin{Version}{1997/09/16 v1.1}
%   \item
%     Dokumentation und Kommentare (Posting in de.comp.text.tex).
%   \item
%     Erweiterung um Sonderzeichen '(', ')', '+', '\textasciitilde' und ' '.
%   \item
%     Trennungsverhinderung am 'hyphen'.
%   \end{Version}
%   \begin{Version}{1997/10/16 v1.2}
%   \item
%     Schutz vor wiederholtem Einlesen.
%   \item
%     Unter \LaTeXe\ Nutzung des \cs{DeclareRobustCommand}-Features.
%   \end{Version}
%   \begin{Version}{1997/12/09 v1.3}
%   \item
%     Tempor\"are Variable eingespart.
%   \item
%     Posted in newsgroup \xnewsgroup{de.comp.text.tex}:\\
%     \URL{``\link{Re: Generisches Markup f\"ur Telefonnummern?}''}^^A
%     {http://groups.google.com/group/de.comp.text.tex/msg/86b3a86140007309}
%   \end{Version}
%   \begin{Version}{2004/11/02 v1.4}
%   \item
%     Fehler in der Dokumentation korrigiert.
%   \end{Version}
%   \begin{Version}{2005/09/30 v1.5}
%   \item
%     Konfigurierbare Symbole: '/', '(', ')', '+' und '\textasciitilde'.
%   \end{Version}
%   \begin{Version}{2006/02/12 v1.6}
%   \item
%     LPPL 1.3.
%   \item
%     Kurze \"Ubersicht in Englisch.
%   \item
%     CTAN.
%   \end{Version}
%   \begin{Version}{2006/08/26 v1.7}
%   \item
%     New DTX framework.
%   \end{Version}
%   \begin{Version}{2007/04/11 v1.8}
%   \item
%     Line ends sanitized.
%   \end{Version}
%   \begin{Version}{2007/09/09 v1.9}
%   \item
%     Catcode section added.
%   \item
%     Missing docstrip tag added.
%   \end{Version}
%   \begin{Version}{2008/08/11 v1.10}
%   \item
%     Code is not changed.
%   \item
%     URLs updated.
%   \end{Version}
%   \begin{Version}{2016/05/16 v1.11}
%   \item
%     Documentation updates.
%   \end{Version}
% \end{History}
%
% \PrintIndex
%
% \Finale
\endinput
|
% \end{quote}
% Do not forget to quote the argument according to the demands
% of your shell.
%
% \paragraph{Generating the documentation.}
% You can use both the \xfile{.dtx} or the \xfile{.drv} to generate
% the documentation. The process can be configured by the
% configuration file \xfile{ltxdoc.cfg}. For instance, put this
% line into this file, if you want to have A4 as paper format:
% \begin{quote}
%   \verb|\PassOptionsToClass{a4paper}{article}|
% \end{quote}
% An example follows how to generate the
% documentation with pdf\LaTeX:
% \begin{quote}
%\begin{verbatim}
%pdflatex telprint.dtx
%makeindex -s gind.ist telprint.idx
%pdflatex telprint.dtx
%makeindex -s gind.ist telprint.idx
%pdflatex telprint.dtx
%\end{verbatim}
% \end{quote}
%
% \begin{History}
%   \begin{Version}{1996/11/28 v1.0}
%   \item
%     Erste lauff\"ahige Version.
%   \item
%     Nur '-' und '/' als zul\"assige Sonderzeichen.
%   \end{Version}
%   \begin{Version}{1997/09/16 v1.1}
%   \item
%     Dokumentation und Kommentare (Posting in de.comp.text.tex).
%   \item
%     Erweiterung um Sonderzeichen '(', ')', '+', '\textasciitilde' und ' '.
%   \item
%     Trennungsverhinderung am 'hyphen'.
%   \end{Version}
%   \begin{Version}{1997/10/16 v1.2}
%   \item
%     Schutz vor wiederholtem Einlesen.
%   \item
%     Unter \LaTeXe\ Nutzung des \cs{DeclareRobustCommand}-Features.
%   \end{Version}
%   \begin{Version}{1997/12/09 v1.3}
%   \item
%     Tempor\"are Variable eingespart.
%   \item
%     Posted in newsgroup \xnewsgroup{de.comp.text.tex}:\\
%     \URL{``\link{Re: Generisches Markup f\"ur Telefonnummern?}''}^^A
%     {http://groups.google.com/group/de.comp.text.tex/msg/86b3a86140007309}
%   \end{Version}
%   \begin{Version}{2004/11/02 v1.4}
%   \item
%     Fehler in der Dokumentation korrigiert.
%   \end{Version}
%   \begin{Version}{2005/09/30 v1.5}
%   \item
%     Konfigurierbare Symbole: '/', '(', ')', '+' und '\textasciitilde'.
%   \end{Version}
%   \begin{Version}{2006/02/12 v1.6}
%   \item
%     LPPL 1.3.
%   \item
%     Kurze \"Ubersicht in Englisch.
%   \item
%     CTAN.
%   \end{Version}
%   \begin{Version}{2006/08/26 v1.7}
%   \item
%     New DTX framework.
%   \end{Version}
%   \begin{Version}{2007/04/11 v1.8}
%   \item
%     Line ends sanitized.
%   \end{Version}
%   \begin{Version}{2007/09/09 v1.9}
%   \item
%     Catcode section added.
%   \item
%     Missing docstrip tag added.
%   \end{Version}
%   \begin{Version}{2008/08/11 v1.10}
%   \item
%     Code is not changed.
%   \item
%     URLs updated.
%   \end{Version}
%   \begin{Version}{2016/05/16 v1.11}
%   \item
%     Documentation updates.
%   \end{Version}
% \end{History}
%
% \PrintIndex
%
% \Finale
\endinput

%        (quote the arguments according to the demands of your shell)
%
% Documentation:
%    (a) If telprint.drv is present:
%           latex telprint.drv
%    (b) Without telprint.drv:
%           latex telprint.dtx; ...
%    The class ltxdoc loads the configuration file ltxdoc.cfg
%    if available. Here you can specify further options, e.g.
%    use A4 as paper format:
%       \PassOptionsToClass{a4paper}{article}
%
%    Programm calls to get the documentation (example):
%       pdflatex telprint.dtx
%       makeindex -s gind.ist telprint.idx
%       pdflatex telprint.dtx
%       makeindex -s gind.ist telprint.idx
%       pdflatex telprint.dtx
%
% Installation:
%    TDS:tex/generic/oberdiek/telprint.sty
%    TDS:doc/latex/oberdiek/telprint.pdf
%    TDS:doc/latex/oberdiek/test/telprint-test1.tex
%    TDS:source/latex/oberdiek/telprint.dtx
%
%<*ignore>
\begingroup
  \catcode123=1 %
  \catcode125=2 %
  \def\x{LaTeX2e}%
\expandafter\endgroup
\ifcase 0\ifx\install y1\fi\expandafter
         \ifx\csname processbatchFile\endcsname\relax\else1\fi
         \ifx\fmtname\x\else 1\fi\relax
\else\csname fi\endcsname
%</ignore>
%<*install>
\input docstrip.tex
\Msg{************************************************************************}
\Msg{* Installation}
\Msg{* Package: telprint 2016/05/16 v1.11 Format German phone numbers (HO)}
\Msg{************************************************************************}

\keepsilent
\askforoverwritefalse

\let\MetaPrefix\relax
\preamble

This is a generated file.

Project: telprint
Version: 2016/05/16 v1.11

Copyright (C) 1996, 1997, 2004-2008 by
   Heiko Oberdiek <heiko.oberdiek at googlemail.com>

This work may be distributed and/or modified under the
conditions of the LaTeX Project Public License, either
version 1.3c of this license or (at your option) any later
version. This version of this license is in
   http://www.latex-project.org/lppl/lppl-1-3c.txt
and the latest version of this license is in
   http://www.latex-project.org/lppl.txt
and version 1.3 or later is part of all distributions of
LaTeX version 2005/12/01 or later.

This work has the LPPL maintenance status "maintained".

This Current Maintainer of this work is Heiko Oberdiek.

The Base Interpreter refers to any `TeX-Format',
because some files are installed in TDS:tex/generic//.

This work consists of the main source file telprint.dtx
and the derived files
   telprint.sty, telprint.pdf, telprint.ins, telprint.drv,
   telprint-test1.tex.

\endpreamble
\let\MetaPrefix\DoubleperCent

\generate{%
  \file{telprint.ins}{\from{telprint.dtx}{install}}%
  \file{telprint.drv}{\from{telprint.dtx}{driver}}%
  \usedir{tex/generic/oberdiek}%
  \file{telprint.sty}{\from{telprint.dtx}{package}}%
%  \usedir{doc/latex/oberdiek/test}%
%  \file{telprint-test1.tex}{\from{telprint.dtx}{test1}}%
  \nopreamble
  \nopostamble
%  \usedir{source/latex/oberdiek/catalogue}%
%  \file{telprint.xml}{\from{telprint.dtx}{catalogue}}%
}

\catcode32=13\relax% active space
\let =\space%
\Msg{************************************************************************}
\Msg{*}
\Msg{* To finish the installation you have to move the following}
\Msg{* file into a directory searched by TeX:}
\Msg{*}
\Msg{*     telprint.sty}
\Msg{*}
\Msg{* To produce the documentation run the file `telprint.drv'}
\Msg{* through LaTeX.}
\Msg{*}
\Msg{* Happy TeXing!}
\Msg{*}
\Msg{************************************************************************}

\endbatchfile
%</install>
%<*ignore>
\fi
%</ignore>
%<*driver>
\NeedsTeXFormat{LaTeX2e}
\ProvidesFile{telprint.drv}%
  [2016/05/16 v1.11 Format German phone numbers (HO)]%
\documentclass{ltxdoc}
\usepackage{holtxdoc}[2011/11/22]
\usepackage[ngerman,english]{babel}
\begin{document}
  \DocInput{telprint.dtx}%
\end{document}
%</driver>
% \fi
%
%
% \CharacterTable
%  {Upper-case    \A\B\C\D\E\F\G\H\I\J\K\L\M\N\O\P\Q\R\S\T\U\V\W\X\Y\Z
%   Lower-case    \a\b\c\d\e\f\g\h\i\j\k\l\m\n\o\p\q\r\s\t\u\v\w\x\y\z
%   Digits        \0\1\2\3\4\5\6\7\8\9
%   Exclamation   \!     Double quote  \"     Hash (number) \#
%   Dollar        \$     Percent       \%     Ampersand     \&
%   Acute accent  \'     Left paren    \(     Right paren   \)
%   Asterisk      \*     Plus          \+     Comma         \,
%   Minus         \-     Point         \.     Solidus       \/
%   Colon         \:     Semicolon     \;     Less than     \<
%   Equals        \=     Greater than  \>     Question mark \?
%   Commercial at \@     Left bracket  \[     Backslash     \\
%   Right bracket \]     Circumflex    \^     Underscore    \_
%   Grave accent  \`     Left brace    \{     Vertical bar  \|
%   Right brace   \}     Tilde         \~}
%
% \GetFileInfo{telprint.drv}
%
% \title{The \xpackage{telprint} package}
% \date{2016/05/16 v1.11}
% \author{Heiko Oberdiek\thanks
% {Please report any issues at https://github.com/ho-tex/oberdiek/issues}\\
% \xemail{heiko.oberdiek at googlemail.com}}
%
% \maketitle
%
% \begin{abstract}
% Package \xpackage{telprint} provides \cs{telprint} for formatting
% German phone numbers.
% \end{abstract}
%
% \tableofcontents
%
% \section{Documentation}
%
% \subsection{Introduction}
%
%            This is a very old package that I have written
%            to format phone numbers. It follows German
%            conventions and the documentation is mainly in German.
%
% \subsection{Short overview in English}
%
% \LaTeX:
% \begin{quote}
% |\usepackage{telprint}|\\
% |\telprint{123/456-789}|\\
% \end{quote}
% \plainTeX:
% \begin{quote}
%   |\input telprint.sty|\\
%   |\telprint{123/456-789}|
% \end{quote}
%
% \DescribeMacro\telprint
% |\telprint{...}| formats the explicitly given number.
%     Digits, spaces and some special characters
%     ('+', '/', '-', '(', ')', '\textasciitilde', ' ') are supported.
%     Numbers are divided into groups of two digits from the right.
% Examples:
% \begin{quote}
%     |\telprint{0761/12345}     ==> 07\,61/1\,23\,45|\\
%     |\telprint{01234/567-89}   ==> 0\,12\,34/5\,67\leavevmode\hbox{-}89|\\
%     |\telprint{+49 (6221) 297} ==> +49~(62\,21)~2\,97|
% \end{quote}
%
% \subsubsection{Configuration}
%
% The output of the symbols can be configured by
% \cs{telhyphen}, \cs{telslash}, \cs{telleftparen}, \cs{telrightparen},
% \cs{telplus}, \cs{teltilde}.
% Example:
% \begin{quote}
%   |\telslash{\,/\,}\\|
%   |\telprint{12/34} ==> 12\,/\,34|
% \end{quote}
%
% \DescribeMacro\telspace
% \cs{telspace} configures the space between digit groups.
%
% \DescribeMacro\telnumber
% \cs{telnumber} only formats a number in digit groups; special
%    characters are not recognized.
%
% \subsection{Documentation in German}
%
% \begin{otherlanguage*}{ngerman}
% \hyphenation{To-ken-ma-kros}
% \begin{itemize}
% \item \DescribeMacro\telprint |telprint#1|\\
%   Der eigentliche Anwenderbefehl zur formatierten Ausgabe von
%   Telefonnummern. Diese d\"urfen dabei nur als Zahlen angegeben
%   werden(, da sie tokenweise analysiert werden).
%   Als Trenn- oder Sonderzeichen werden unterst\"utzt:
%   '+', '/', '-', '(', ')', '\textasciitilde', ' '
%   Einfache Leerzeichen werden erkannt und durch Tilden ersetzt, um
%   Trennungen in der Telefonnummer zu verhindern. (Man beachte aus
%   gleichem Grunde die \cs{hbox} bei '-'.)
%   Beispiele:
%   \begin{quote}
%     |\telprint{0761/12345}     ==> 07\,61/1\,23\,45|\\
%     |\telprint{01234/567-89}   ==> 0\,12\,34/5\,67\leavevmode\hbox{-}89|\\
%     |\telprint{+49 (6221) 297} ==> +49~(62\,21)~2\,97|
%   \end{quote}
% \end{itemize}
% Der Rest enth\"alt eher Technisches:
% \begin{itemize}
% \item \DescribeMacro\telspace |\telspace#1|\\
%   Mit diesem Befehl wird der Abstand zwischen den Zifferngruppen
%   angegeben (Default: |\,|).
%   (Durch |\telspace{}| kann dieser zusaetzliche Abstand abgestellt
%   werden.)
% \item \DescribeMacro\telhyphen |\telhyphen#1|\\
%   Dieser Befehl gibt die Art des Bindestriches, wie er ausgegeben
%   werden soll. In der Eingabe darf jedoch nur der einfache
%   Bindestrich stehen:
%   |\telprint{123-45}|, jedoch NIE |\telprint{123--45}|!
%   Kopka-Bindestrich-Fans geben an:
%   |\telhyphen{\leavevmode\hbox{--}}|
% \item
%   \DescribeMacro{\telslash}
%   \DescribeMacro{\telleftparen}
%   \DescribeMacro{\telrightparen}
%   \DescribeMacro{\telplus}
%   \DescribeMacro{\teltilde}
%   |\telslash#1|, |\telleftparen#1|, |\telrightparen#1|, |\telplus#1|,
%   |\teltilde|\\
%   Diese Befehle konfigurieren die Zeichen '/', '(', ')', '+'
%   und '\textasciitilde'. Sie funktionieren analog zu \cs{telhyphen}.
% \item \DescribeMacro\telnumber |\telnumber#1|\\
%   Richtung interner Befehl: Er dient dazu, eine Zifferngruppe
%   in Zweiergruppen auszugeben.
%   Die einzelnen Zahlen werden im Tokenregister \cs{TELtoks}
%   gespeichert. Abwechselnd werden dabei zwischen zwei Token
%   (Zahlen) \cs{TELx} bzw. \cs{TELy} eingefuegt, abh\"angig von dem
%   wechselnden Wert von \cs{TELswitch}. Zum Schluss kann dann einfach
%   festgestellt werden ob die Nummer nun eine geradzahlige oder
%   ungeradzahlige Zahl von Ziffern aufwies. Dem entsprechend wird
%   \cs{TELx} mit dem Zusatzabstand belegt und \cs{TELy} leer definiert
%   oder umgekehrt. )
% \item |\TEL...| interne Befehle, Technisches:\\
%   \cs{TELsplit} dient zur Aufteilung einer zusammengesetzten
%   Telefonnummer (Vorwahl, Hauptnummer, Nebenstelle). In dieser
%   Implementation werden als Trennzeichen nur '/' und '-' erkannt.
%   Die einzelnen Bestandteile wie Vorwahl werden dann dem Befehl
%   \cs{telnumber} zur Formatierung uebergeben.
% \item Die Erkennung von einfachen Leerzeichen ist um einiges
%   schwieriger: Die Tokentrennung ueber Parameter |#1#2| funktioniert
%   nicht f\"ur einfache Leerzeichen, da TeX sie \emph{niemals} als
%   eigenst\"andige Argumente behandelt! (The TeXbook, Chapter 20,
%   p. 201)
%
%   (Anmerkung am Rande: Deshalb funktionieren die entsprechenden
%   Tokenmakros auf S. 149 des Buches "`Einf\"uhrung in TeX"' von
%   N. Schwarz (3. Aufl.) nicht, wenn im Tokenregister als erstes
%   ein einfaches Leerzeichen steht!)
% \end{itemize}
% \end{otherlanguage*}
%
% \StopEventually{
% }
%
% \section{Implementation}
%
%    \begin{macrocode}
%<*package>
%    \end{macrocode}
%
% \subsection{Reload check and package identification}
%    Reload check, especially if the package is not used with \LaTeX.
%    \begin{macrocode}
\begingroup\catcode61\catcode48\catcode32=10\relax%
  \catcode13=5 % ^^M
  \endlinechar=13 %
  \catcode35=6 % #
  \catcode39=12 % '
  \catcode44=12 % ,
  \catcode45=12 % -
  \catcode46=12 % .
  \catcode58=12 % :
  \catcode64=11 % @
  \catcode123=1 % {
  \catcode125=2 % }
  \expandafter\let\expandafter\x\csname ver@telprint.sty\endcsname
  \ifx\x\relax % plain-TeX, first loading
  \else
    \def\empty{}%
    \ifx\x\empty % LaTeX, first loading,
      % variable is initialized, but \ProvidesPackage not yet seen
    \else
      \expandafter\ifx\csname PackageInfo\endcsname\relax
        \def\x#1#2{%
          \immediate\write-1{Package #1 Info: #2.}%
        }%
      \else
        \def\x#1#2{\PackageInfo{#1}{#2, stopped}}%
      \fi
      \x{telprint}{The package is already loaded}%
      \aftergroup\endinput
    \fi
  \fi
\endgroup%
%    \end{macrocode}
%    Package identification:
%    \begin{macrocode}
\begingroup\catcode61\catcode48\catcode32=10\relax%
  \catcode13=5 % ^^M
  \endlinechar=13 %
  \catcode35=6 % #
  \catcode39=12 % '
  \catcode40=12 % (
  \catcode41=12 % )
  \catcode44=12 % ,
  \catcode45=12 % -
  \catcode46=12 % .
  \catcode47=12 % /
  \catcode58=12 % :
  \catcode64=11 % @
  \catcode91=12 % [
  \catcode93=12 % ]
  \catcode123=1 % {
  \catcode125=2 % }
  \expandafter\ifx\csname ProvidesPackage\endcsname\relax
    \def\x#1#2#3[#4]{\endgroup
      \immediate\write-1{Package: #3 #4}%
      \xdef#1{#4}%
    }%
  \else
    \def\x#1#2[#3]{\endgroup
      #2[{#3}]%
      \ifx#1\@undefined
        \xdef#1{#3}%
      \fi
      \ifx#1\relax
        \xdef#1{#3}%
      \fi
    }%
  \fi
\expandafter\x\csname ver@telprint.sty\endcsname
\ProvidesPackage{telprint}%
  [2016/05/16 v1.11 Format German phone numbers (HO)]%
%    \end{macrocode}
%
% \subsection{Catcodes}
%
%    \begin{macrocode}
\begingroup\catcode61\catcode48\catcode32=10\relax%
  \catcode13=5 % ^^M
  \endlinechar=13 %
  \catcode123=1 % {
  \catcode125=2 % }
  \catcode64=11 % @
  \def\x{\endgroup
    \expandafter\edef\csname TELAtEnd\endcsname{%
      \endlinechar=\the\endlinechar\relax
      \catcode13=\the\catcode13\relax
      \catcode32=\the\catcode32\relax
      \catcode35=\the\catcode35\relax
      \catcode61=\the\catcode61\relax
      \catcode64=\the\catcode64\relax
      \catcode123=\the\catcode123\relax
      \catcode125=\the\catcode125\relax
    }%
  }%
\x\catcode61\catcode48\catcode32=10\relax%
\catcode13=5 % ^^M
\endlinechar=13 %
\catcode35=6 % #
\catcode64=11 % @
\catcode123=1 % {
\catcode125=2 % }
\def\TMP@EnsureCode#1#2{%
  \edef\TELAtEnd{%
    \TELAtEnd
    \catcode#1=\the\catcode#1\relax
  }%
  \catcode#1=#2\relax
}
\TMP@EnsureCode{33}{12}% !
\TMP@EnsureCode{36}{3}% $
\TMP@EnsureCode{40}{12}% (
\TMP@EnsureCode{41}{12}% )
\TMP@EnsureCode{42}{12}% *
\TMP@EnsureCode{43}{12}% +
\TMP@EnsureCode{44}{12}% ,
\TMP@EnsureCode{45}{12}% -
\TMP@EnsureCode{46}{12}% .
\TMP@EnsureCode{47}{12}% /
\TMP@EnsureCode{91}{12}% [
\TMP@EnsureCode{93}{12}% ]
\TMP@EnsureCode{126}{13}% ~ (active)
\edef\TELAtEnd{\TELAtEnd\noexpand\endinput}
%    \end{macrocode}
%
% \subsection{Package macros}
%    \begin{macrocode}
\ifx\DeclareRobustCommand\UnDeFiNeD
  \def\DeclareRobustCommand*#1[1]{\def#1##1}%
  \def\TELreset{\let\DeclareRobustCommand=\UnDeFiNeD}%
  \input infwarerr.sty\relax
  \@PackageInfo{telprint}{%
    Macros are not robust!%
  }%
\else
  \let\TELreset=\relax
\fi
%    \end{macrocode}
%    \begin{macro}{\telspace}
%    \begin{macrocode}
\DeclareRobustCommand*{\telspace}[1]{\def\TELspace{#1}}
\telspace{{}$\,${}}
%    \end{macrocode}
%    \end{macro}
%    \begin{macro}{\telhyphen}
%    \begin{macrocode}
\DeclareRobustCommand*{\telhyphen}[1]{\def\TELhyphen{#1}}
\telhyphen{\leavevmode\hbox{-}}% \hbox zur Verhinderung der Trennung
%    \end{macrocode}
%    \end{macro}
%    \begin{macro}{\telslash}
%    \begin{macrocode}
\DeclareRobustCommand*{\telslash}[1]{\def\TELslash{#1}}
\telslash{/}%
%    \end{macrocode}
%    \end{macro}
%    \begin{macro}{\telleftparen}
%    \begin{macrocode}
\DeclareRobustCommand*{\telleftparen}[1]{\def\TELleftparen{#1}}
\telleftparen{(}%
%    \end{macrocode}
%    \end{macro}
%    \begin{macro}{\telrightparen}
%    \begin{macrocode}
\DeclareRobustCommand*{\telrightparen}[1]{\def\TELrightparen{#1}}
\telrightparen{)}%
%    \end{macrocode}
%    \end{macro}
%    \begin{macro}{\telplus}
%    \begin{macrocode}
\DeclareRobustCommand*{\telplus}[1]{\def\TELplus{#1}}
\telplus{+}%
%    \end{macrocode}
%    \end{macro}
%    \begin{macro}{\teltilde}
%    \begin{macrocode}
\DeclareRobustCommand*{\teltilde}[1]{\def\TELtilde{#1}}
\teltilde{~}%
%    \end{macrocode}
%    \end{macro}
%    \begin{macro}{\TELtoks}
%    \begin{macrocode}
\newtoks\TELtoks
%    \end{macrocode}
%    \end{macro}
%    \begin{macro}{\TELnumber}
%    \begin{macrocode}
\def\TELnumber#1#2\TELnumberEND{%
  \begingroup
  \def\0{#2}%
  \expandafter\endgroup
  \ifx\0\empty
    \TELtoks=\expandafter{\the\TELtoks#1}%
    \ifnum\TELswitch=0 %
      \def\TELx{\TELspace}\def\TELy{}%
    \else
      \def\TELx{}\def\TELy{\TELspace}%
    \fi
    \the\TELtoks
  \else
    \ifnum\TELswitch=0 %
      \TELtoks=\expandafter{\the\TELtoks#1\TELx}%
      \def\TELswitch{1}%
    \else
      \TELtoks=\expandafter{\the\TELtoks#1\TELy}%
      \def\TELswitch{0}%
    \fi
    \TELnumber#2\TELnumberEND
  \fi
}
%    \end{macrocode}
%    \end{macro}
%    \begin{macro}{\telnumber}
%    \begin{macrocode}
\DeclareRobustCommand*{\telnumber}[1]{%
  \TELtoks={}%
  \def\TELswitch{0}%
  \TELnumber#1{}\TELnumberEND
}
%    \end{macrocode}
%    \end{macro}
%    \begin{macro}{\TELsplit}
%    \begin{macrocode}
\def\TELsplit{\futurelet\TELfuture\TELdosplit}
%    \end{macrocode}
%    \end{macro}
%    \begin{macro}{\TELdosplit}
%    \begin{macrocode}
\def\TELdosplit#1#2\TELsplitEND
{%
  \def\TELsp{ }%
  \expandafter\ifx\TELsp\TELfuture
    \let\TELfuture=\relax
    \expandafter\telnumber\expandafter{\the\TELtoks}~%
    \telprint{#1#2}% Das Leerzeichen kann nicht #1 sein!
  \else
    \def\TELfirst{#1}%
    \ifx\TELfirst\empty
      \expandafter\telnumber\expandafter{\the\TELtoks}%
      \TELtoks={}%
    \else\if-\TELfirst
      \expandafter\telnumber\expandafter{\the\TELtoks}\TELhyphen
      \telprint{#2}%
    \else\if/\TELfirst
      \expandafter\telnumber\expandafter{\the\TELtoks}\TELslash
      \telprint{#2}%
    \else\if(\TELfirst
      \expandafter\telnumber\expandafter{\the\TELtoks}\TELleftparen
      \telprint{#2}%
    \else\if)\TELfirst
      \expandafter\telnumber\expandafter{\the\TELtoks}\TELrightparen
      \telprint{#2}%
    \else\if+\TELfirst
      \expandafter\telnumber\expandafter{\the\TELtoks}\TELplus
      \telprint{#2}%
    \else\def\TELtemp{~}\ifx\TELtemp\TELfirst
      \expandafter\telnumber\expandafter{\the\TELtoks}\TELtilde
      \telprint{#2}%
    \else
      \TELtoks=\expandafter{\the\TELtoks#1}%
      \TELsplit#2{}\TELsplitEND
    \fi\fi\fi\fi\fi\fi\fi
  \fi
}
%    \end{macrocode}
%    \end{macro}
%    \begin{macro}{\telprint}
%    \begin{macrocode}
\DeclareRobustCommand*{\telprint}[1]{%
  \TELtoks={}%
  \TELsplit#1{}\TELsplitEND
}
%    \end{macrocode}
%    \end{macro}
%    \begin{macrocode}
\TELreset\let\TELreset=\UnDeFiNeD
%    \end{macrocode}
%
%    \begin{macrocode}
\TELAtEnd%
%</package>
%    \end{macrocode}
%
% \section{Test}
%
% \subsection{Catcode checks for loading}
%
%    \begin{macrocode}
%<*test1>
%    \end{macrocode}
%    \begin{macrocode}
\catcode`\{=1 %
\catcode`\}=2 %
\catcode`\#=6 %
\catcode`\@=11 %
\expandafter\ifx\csname count@\endcsname\relax
  \countdef\count@=255 %
\fi
\expandafter\ifx\csname @gobble\endcsname\relax
  \long\def\@gobble#1{}%
\fi
\expandafter\ifx\csname @firstofone\endcsname\relax
  \long\def\@firstofone#1{#1}%
\fi
\expandafter\ifx\csname loop\endcsname\relax
  \expandafter\@firstofone
\else
  \expandafter\@gobble
\fi
{%
  \def\loop#1\repeat{%
    \def\body{#1}%
    \iterate
  }%
  \def\iterate{%
    \body
      \let\next\iterate
    \else
      \let\next\relax
    \fi
    \next
  }%
  \let\repeat=\fi
}%
\def\RestoreCatcodes{}
\count@=0 %
\loop
  \edef\RestoreCatcodes{%
    \RestoreCatcodes
    \catcode\the\count@=\the\catcode\count@\relax
  }%
\ifnum\count@<255 %
  \advance\count@ 1 %
\repeat

\def\RangeCatcodeInvalid#1#2{%
  \count@=#1\relax
  \loop
    \catcode\count@=15 %
  \ifnum\count@<#2\relax
    \advance\count@ 1 %
  \repeat
}
\def\RangeCatcodeCheck#1#2#3{%
  \count@=#1\relax
  \loop
    \ifnum#3=\catcode\count@
    \else
      \errmessage{%
        Character \the\count@\space
        with wrong catcode \the\catcode\count@\space
        instead of \number#3%
      }%
    \fi
  \ifnum\count@<#2\relax
    \advance\count@ 1 %
  \repeat
}
\def\space{ }
\expandafter\ifx\csname LoadCommand\endcsname\relax
  \def\LoadCommand{\input telprint.sty\relax}%
\fi
\def\Test{%
  \RangeCatcodeInvalid{0}{47}%
  \RangeCatcodeInvalid{58}{64}%
  \RangeCatcodeInvalid{91}{96}%
  \RangeCatcodeInvalid{123}{255}%
  \catcode`\@=12 %
  \catcode`\\=0 %
  \catcode`\%=14 %
  \LoadCommand
  \RangeCatcodeCheck{0}{36}{15}%
  \RangeCatcodeCheck{37}{37}{14}%
  \RangeCatcodeCheck{38}{47}{15}%
  \RangeCatcodeCheck{48}{57}{12}%
  \RangeCatcodeCheck{58}{63}{15}%
  \RangeCatcodeCheck{64}{64}{12}%
  \RangeCatcodeCheck{65}{90}{11}%
  \RangeCatcodeCheck{91}{91}{15}%
  \RangeCatcodeCheck{92}{92}{0}%
  \RangeCatcodeCheck{93}{96}{15}%
  \RangeCatcodeCheck{97}{122}{11}%
  \RangeCatcodeCheck{123}{255}{15}%
  \RestoreCatcodes
}
\Test
\csname @@end\endcsname
\end
%    \end{macrocode}
%    \begin{macrocode}
%</test1>
%    \end{macrocode}
%
% \section{Installation}
%
% \subsection{Download}
%
% \paragraph{Package.} This package is available on
% CTAN\footnote{\url{https://ctan.org/pkg/telprint}}:
% \begin{description}
% \item[\CTAN{macros/latex/contrib/oberdiek/telprint.dtx}] The source file.
% \item[\CTAN{macros/latex/contrib/oberdiek/telprint.pdf}] Documentation.
% \end{description}
%
%
% \paragraph{Bundle.} All the packages of the bundle `oberdiek'
% are also available in a TDS compliant ZIP archive. There
% the packages are already unpacked and the documentation files
% are generated. The files and directories obey the TDS standard.
% \begin{description}
% \item[\CTANinstall{install/macros/latex/contrib/oberdiek.tds.zip}]
% \end{description}
% \emph{TDS} refers to the standard ``A Directory Structure
% for \TeX\ Files'' (\CTAN{tds/tds.pdf}). Directories
% with \xfile{texmf} in their name are usually organized this way.
%
% \subsection{Bundle installation}
%
% \paragraph{Unpacking.} Unpack the \xfile{oberdiek.tds.zip} in the
% TDS tree (also known as \xfile{texmf} tree) of your choice.
% Example (linux):
% \begin{quote}
%   |unzip oberdiek.tds.zip -d ~/texmf|
% \end{quote}
%
% \paragraph{Script installation.}
% Check the directory \xfile{TDS:scripts/oberdiek/} for
% scripts that need further installation steps.
% Package \xpackage{attachfile2} comes with the Perl script
% \xfile{pdfatfi.pl} that should be installed in such a way
% that it can be called as \texttt{pdfatfi}.
% Example (linux):
% \begin{quote}
%   |chmod +x scripts/oberdiek/pdfatfi.pl|\\
%   |cp scripts/oberdiek/pdfatfi.pl /usr/local/bin/|
% \end{quote}
%
% \subsection{Package installation}
%
% \paragraph{Unpacking.} The \xfile{.dtx} file is a self-extracting
% \docstrip\ archive. The files are extracted by running the
% \xfile{.dtx} through \plainTeX:
% \begin{quote}
%   \verb|tex telprint.dtx|
% \end{quote}
%
% \paragraph{TDS.} Now the different files must be moved into
% the different directories in your installation TDS tree
% (also known as \xfile{texmf} tree):
% \begin{quote}
% \def\t{^^A
% \begin{tabular}{@{}>{\ttfamily}l@{ $\rightarrow$ }>{\ttfamily}l@{}}
%   telprint.sty & tex/generic/oberdiek/telprint.sty\\
%   telprint.pdf & doc/latex/oberdiek/telprint.pdf\\
%   test/telprint-test1.tex & doc/latex/oberdiek/test/telprint-test1.tex\\
%   telprint.dtx & source/latex/oberdiek/telprint.dtx\\
% \end{tabular}^^A
% }^^A
% \sbox0{\t}^^A
% \ifdim\wd0>\linewidth
%   \begingroup
%     \advance\linewidth by\leftmargin
%     \advance\linewidth by\rightmargin
%   \edef\x{\endgroup
%     \def\noexpand\lw{\the\linewidth}^^A
%   }\x
%   \def\lwbox{^^A
%     \leavevmode
%     \hbox to \linewidth{^^A
%       \kern-\leftmargin\relax
%       \hss
%       \usebox0
%       \hss
%       \kern-\rightmargin\relax
%     }^^A
%   }^^A
%   \ifdim\wd0>\lw
%     \sbox0{\small\t}^^A
%     \ifdim\wd0>\linewidth
%       \ifdim\wd0>\lw
%         \sbox0{\footnotesize\t}^^A
%         \ifdim\wd0>\linewidth
%           \ifdim\wd0>\lw
%             \sbox0{\scriptsize\t}^^A
%             \ifdim\wd0>\linewidth
%               \ifdim\wd0>\lw
%                 \sbox0{\tiny\t}^^A
%                 \ifdim\wd0>\linewidth
%                   \lwbox
%                 \else
%                   \usebox0
%                 \fi
%               \else
%                 \lwbox
%               \fi
%             \else
%               \usebox0
%             \fi
%           \else
%             \lwbox
%           \fi
%         \else
%           \usebox0
%         \fi
%       \else
%         \lwbox
%       \fi
%     \else
%       \usebox0
%     \fi
%   \else
%     \lwbox
%   \fi
% \else
%   \usebox0
% \fi
% \end{quote}
% If you have a \xfile{docstrip.cfg} that configures and enables \docstrip's
% TDS installing feature, then some files can already be in the right
% place, see the documentation of \docstrip.
%
% \subsection{Refresh file name databases}
%
% If your \TeX~distribution
% (\teTeX, \mikTeX, \dots) relies on file name databases, you must refresh
% these. For example, \teTeX\ users run \verb|texhash| or
% \verb|mktexlsr|.
%
% \subsection{Some details for the interested}
%
% \paragraph{Attached source.}
%
% The PDF documentation on CTAN also includes the
% \xfile{.dtx} source file. It can be extracted by
% AcrobatReader 6 or higher. Another option is \textsf{pdftk},
% e.g. unpack the file into the current directory:
% \begin{quote}
%   \verb|pdftk telprint.pdf unpack_files output .|
% \end{quote}
%
% \paragraph{Unpacking with \LaTeX.}
% The \xfile{.dtx} chooses its action depending on the format:
% \begin{description}
% \item[\plainTeX:] Run \docstrip\ and extract the files.
% \item[\LaTeX:] Generate the documentation.
% \end{description}
% If you insist on using \LaTeX\ for \docstrip\ (really,
% \docstrip\ does not need \LaTeX), then inform the autodetect routine
% about your intention:
% \begin{quote}
%   \verb|latex \let\install=y% \iffalse meta-comment
%
% File: telprint.dtx
% Version: 2016/05/16 v1.11
% Info: Format German phone numbers
%
% Copyright (C) 1996, 1997, 2004-2008 by
%    Heiko Oberdiek <heiko.oberdiek at googlemail.com>
%    2016
%    https://github.com/ho-tex/oberdiek/issues
%
% This work may be distributed and/or modified under the
% conditions of the LaTeX Project Public License, either
% version 1.3c of this license or (at your option) any later
% version. This version of this license is in
%    http://www.latex-project.org/lppl/lppl-1-3c.txt
% and the latest version of this license is in
%    http://www.latex-project.org/lppl.txt
% and version 1.3 or later is part of all distributions of
% LaTeX version 2005/12/01 or later.
%
% This work has the LPPL maintenance status "maintained".
%
% This Current Maintainer of this work is Heiko Oberdiek.
%
% The Base Interpreter refers to any `TeX-Format',
% because some files are installed in TDS:tex/generic//.
%
% This work consists of the main source file telprint.dtx
% and the derived files
%    telprint.sty, telprint.pdf, telprint.ins, telprint.drv,
%    telprint-test1.tex.
%
% Distribution:
%    CTAN:macros/latex/contrib/oberdiek/telprint.dtx
%    CTAN:macros/latex/contrib/oberdiek/telprint.pdf
%
% Unpacking:
%    (a) If telprint.ins is present:
%           tex telprint.ins
%    (b) Without telprint.ins:
%           tex telprint.dtx
%    (c) If you insist on using LaTeX
%           latex \let\install=y% \iffalse meta-comment
%
% File: telprint.dtx
% Version: 2016/05/16 v1.11
% Info: Format German phone numbers
%
% Copyright (C) 1996, 1997, 2004-2008 by
%    Heiko Oberdiek <heiko.oberdiek at googlemail.com>
%    2016
%    https://github.com/ho-tex/oberdiek/issues
%
% This work may be distributed and/or modified under the
% conditions of the LaTeX Project Public License, either
% version 1.3c of this license or (at your option) any later
% version. This version of this license is in
%    http://www.latex-project.org/lppl/lppl-1-3c.txt
% and the latest version of this license is in
%    http://www.latex-project.org/lppl.txt
% and version 1.3 or later is part of all distributions of
% LaTeX version 2005/12/01 or later.
%
% This work has the LPPL maintenance status "maintained".
%
% This Current Maintainer of this work is Heiko Oberdiek.
%
% The Base Interpreter refers to any `TeX-Format',
% because some files are installed in TDS:tex/generic//.
%
% This work consists of the main source file telprint.dtx
% and the derived files
%    telprint.sty, telprint.pdf, telprint.ins, telprint.drv,
%    telprint-test1.tex.
%
% Distribution:
%    CTAN:macros/latex/contrib/oberdiek/telprint.dtx
%    CTAN:macros/latex/contrib/oberdiek/telprint.pdf
%
% Unpacking:
%    (a) If telprint.ins is present:
%           tex telprint.ins
%    (b) Without telprint.ins:
%           tex telprint.dtx
%    (c) If you insist on using LaTeX
%           latex \let\install=y\input{telprint.dtx}
%        (quote the arguments according to the demands of your shell)
%
% Documentation:
%    (a) If telprint.drv is present:
%           latex telprint.drv
%    (b) Without telprint.drv:
%           latex telprint.dtx; ...
%    The class ltxdoc loads the configuration file ltxdoc.cfg
%    if available. Here you can specify further options, e.g.
%    use A4 as paper format:
%       \PassOptionsToClass{a4paper}{article}
%
%    Programm calls to get the documentation (example):
%       pdflatex telprint.dtx
%       makeindex -s gind.ist telprint.idx
%       pdflatex telprint.dtx
%       makeindex -s gind.ist telprint.idx
%       pdflatex telprint.dtx
%
% Installation:
%    TDS:tex/generic/oberdiek/telprint.sty
%    TDS:doc/latex/oberdiek/telprint.pdf
%    TDS:doc/latex/oberdiek/test/telprint-test1.tex
%    TDS:source/latex/oberdiek/telprint.dtx
%
%<*ignore>
\begingroup
  \catcode123=1 %
  \catcode125=2 %
  \def\x{LaTeX2e}%
\expandafter\endgroup
\ifcase 0\ifx\install y1\fi\expandafter
         \ifx\csname processbatchFile\endcsname\relax\else1\fi
         \ifx\fmtname\x\else 1\fi\relax
\else\csname fi\endcsname
%</ignore>
%<*install>
\input docstrip.tex
\Msg{************************************************************************}
\Msg{* Installation}
\Msg{* Package: telprint 2016/05/16 v1.11 Format German phone numbers (HO)}
\Msg{************************************************************************}

\keepsilent
\askforoverwritefalse

\let\MetaPrefix\relax
\preamble

This is a generated file.

Project: telprint
Version: 2016/05/16 v1.11

Copyright (C) 1996, 1997, 2004-2008 by
   Heiko Oberdiek <heiko.oberdiek at googlemail.com>

This work may be distributed and/or modified under the
conditions of the LaTeX Project Public License, either
version 1.3c of this license or (at your option) any later
version. This version of this license is in
   http://www.latex-project.org/lppl/lppl-1-3c.txt
and the latest version of this license is in
   http://www.latex-project.org/lppl.txt
and version 1.3 or later is part of all distributions of
LaTeX version 2005/12/01 or later.

This work has the LPPL maintenance status "maintained".

This Current Maintainer of this work is Heiko Oberdiek.

The Base Interpreter refers to any `TeX-Format',
because some files are installed in TDS:tex/generic//.

This work consists of the main source file telprint.dtx
and the derived files
   telprint.sty, telprint.pdf, telprint.ins, telprint.drv,
   telprint-test1.tex.

\endpreamble
\let\MetaPrefix\DoubleperCent

\generate{%
  \file{telprint.ins}{\from{telprint.dtx}{install}}%
  \file{telprint.drv}{\from{telprint.dtx}{driver}}%
  \usedir{tex/generic/oberdiek}%
  \file{telprint.sty}{\from{telprint.dtx}{package}}%
%  \usedir{doc/latex/oberdiek/test}%
%  \file{telprint-test1.tex}{\from{telprint.dtx}{test1}}%
  \nopreamble
  \nopostamble
%  \usedir{source/latex/oberdiek/catalogue}%
%  \file{telprint.xml}{\from{telprint.dtx}{catalogue}}%
}

\catcode32=13\relax% active space
\let =\space%
\Msg{************************************************************************}
\Msg{*}
\Msg{* To finish the installation you have to move the following}
\Msg{* file into a directory searched by TeX:}
\Msg{*}
\Msg{*     telprint.sty}
\Msg{*}
\Msg{* To produce the documentation run the file `telprint.drv'}
\Msg{* through LaTeX.}
\Msg{*}
\Msg{* Happy TeXing!}
\Msg{*}
\Msg{************************************************************************}

\endbatchfile
%</install>
%<*ignore>
\fi
%</ignore>
%<*driver>
\NeedsTeXFormat{LaTeX2e}
\ProvidesFile{telprint.drv}%
  [2016/05/16 v1.11 Format German phone numbers (HO)]%
\documentclass{ltxdoc}
\usepackage{holtxdoc}[2011/11/22]
\usepackage[ngerman,english]{babel}
\begin{document}
  \DocInput{telprint.dtx}%
\end{document}
%</driver>
% \fi
%
%
% \CharacterTable
%  {Upper-case    \A\B\C\D\E\F\G\H\I\J\K\L\M\N\O\P\Q\R\S\T\U\V\W\X\Y\Z
%   Lower-case    \a\b\c\d\e\f\g\h\i\j\k\l\m\n\o\p\q\r\s\t\u\v\w\x\y\z
%   Digits        \0\1\2\3\4\5\6\7\8\9
%   Exclamation   \!     Double quote  \"     Hash (number) \#
%   Dollar        \$     Percent       \%     Ampersand     \&
%   Acute accent  \'     Left paren    \(     Right paren   \)
%   Asterisk      \*     Plus          \+     Comma         \,
%   Minus         \-     Point         \.     Solidus       \/
%   Colon         \:     Semicolon     \;     Less than     \<
%   Equals        \=     Greater than  \>     Question mark \?
%   Commercial at \@     Left bracket  \[     Backslash     \\
%   Right bracket \]     Circumflex    \^     Underscore    \_
%   Grave accent  \`     Left brace    \{     Vertical bar  \|
%   Right brace   \}     Tilde         \~}
%
% \GetFileInfo{telprint.drv}
%
% \title{The \xpackage{telprint} package}
% \date{2016/05/16 v1.11}
% \author{Heiko Oberdiek\thanks
% {Please report any issues at https://github.com/ho-tex/oberdiek/issues}\\
% \xemail{heiko.oberdiek at googlemail.com}}
%
% \maketitle
%
% \begin{abstract}
% Package \xpackage{telprint} provides \cs{telprint} for formatting
% German phone numbers.
% \end{abstract}
%
% \tableofcontents
%
% \section{Documentation}
%
% \subsection{Introduction}
%
%            This is a very old package that I have written
%            to format phone numbers. It follows German
%            conventions and the documentation is mainly in German.
%
% \subsection{Short overview in English}
%
% \LaTeX:
% \begin{quote}
% |\usepackage{telprint}|\\
% |\telprint{123/456-789}|\\
% \end{quote}
% \plainTeX:
% \begin{quote}
%   |\input telprint.sty|\\
%   |\telprint{123/456-789}|
% \end{quote}
%
% \DescribeMacro\telprint
% |\telprint{...}| formats the explicitly given number.
%     Digits, spaces and some special characters
%     ('+', '/', '-', '(', ')', '\textasciitilde', ' ') are supported.
%     Numbers are divided into groups of two digits from the right.
% Examples:
% \begin{quote}
%     |\telprint{0761/12345}     ==> 07\,61/1\,23\,45|\\
%     |\telprint{01234/567-89}   ==> 0\,12\,34/5\,67\leavevmode\hbox{-}89|\\
%     |\telprint{+49 (6221) 297} ==> +49~(62\,21)~2\,97|
% \end{quote}
%
% \subsubsection{Configuration}
%
% The output of the symbols can be configured by
% \cs{telhyphen}, \cs{telslash}, \cs{telleftparen}, \cs{telrightparen},
% \cs{telplus}, \cs{teltilde}.
% Example:
% \begin{quote}
%   |\telslash{\,/\,}\\|
%   |\telprint{12/34} ==> 12\,/\,34|
% \end{quote}
%
% \DescribeMacro\telspace
% \cs{telspace} configures the space between digit groups.
%
% \DescribeMacro\telnumber
% \cs{telnumber} only formats a number in digit groups; special
%    characters are not recognized.
%
% \subsection{Documentation in German}
%
% \begin{otherlanguage*}{ngerman}
% \hyphenation{To-ken-ma-kros}
% \begin{itemize}
% \item \DescribeMacro\telprint |telprint#1|\\
%   Der eigentliche Anwenderbefehl zur formatierten Ausgabe von
%   Telefonnummern. Diese d\"urfen dabei nur als Zahlen angegeben
%   werden(, da sie tokenweise analysiert werden).
%   Als Trenn- oder Sonderzeichen werden unterst\"utzt:
%   '+', '/', '-', '(', ')', '\textasciitilde', ' '
%   Einfache Leerzeichen werden erkannt und durch Tilden ersetzt, um
%   Trennungen in der Telefonnummer zu verhindern. (Man beachte aus
%   gleichem Grunde die \cs{hbox} bei '-'.)
%   Beispiele:
%   \begin{quote}
%     |\telprint{0761/12345}     ==> 07\,61/1\,23\,45|\\
%     |\telprint{01234/567-89}   ==> 0\,12\,34/5\,67\leavevmode\hbox{-}89|\\
%     |\telprint{+49 (6221) 297} ==> +49~(62\,21)~2\,97|
%   \end{quote}
% \end{itemize}
% Der Rest enth\"alt eher Technisches:
% \begin{itemize}
% \item \DescribeMacro\telspace |\telspace#1|\\
%   Mit diesem Befehl wird der Abstand zwischen den Zifferngruppen
%   angegeben (Default: |\,|).
%   (Durch |\telspace{}| kann dieser zusaetzliche Abstand abgestellt
%   werden.)
% \item \DescribeMacro\telhyphen |\telhyphen#1|\\
%   Dieser Befehl gibt die Art des Bindestriches, wie er ausgegeben
%   werden soll. In der Eingabe darf jedoch nur der einfache
%   Bindestrich stehen:
%   |\telprint{123-45}|, jedoch NIE |\telprint{123--45}|!
%   Kopka-Bindestrich-Fans geben an:
%   |\telhyphen{\leavevmode\hbox{--}}|
% \item
%   \DescribeMacro{\telslash}
%   \DescribeMacro{\telleftparen}
%   \DescribeMacro{\telrightparen}
%   \DescribeMacro{\telplus}
%   \DescribeMacro{\teltilde}
%   |\telslash#1|, |\telleftparen#1|, |\telrightparen#1|, |\telplus#1|,
%   |\teltilde|\\
%   Diese Befehle konfigurieren die Zeichen '/', '(', ')', '+'
%   und '\textasciitilde'. Sie funktionieren analog zu \cs{telhyphen}.
% \item \DescribeMacro\telnumber |\telnumber#1|\\
%   Richtung interner Befehl: Er dient dazu, eine Zifferngruppe
%   in Zweiergruppen auszugeben.
%   Die einzelnen Zahlen werden im Tokenregister \cs{TELtoks}
%   gespeichert. Abwechselnd werden dabei zwischen zwei Token
%   (Zahlen) \cs{TELx} bzw. \cs{TELy} eingefuegt, abh\"angig von dem
%   wechselnden Wert von \cs{TELswitch}. Zum Schluss kann dann einfach
%   festgestellt werden ob die Nummer nun eine geradzahlige oder
%   ungeradzahlige Zahl von Ziffern aufwies. Dem entsprechend wird
%   \cs{TELx} mit dem Zusatzabstand belegt und \cs{TELy} leer definiert
%   oder umgekehrt. )
% \item |\TEL...| interne Befehle, Technisches:\\
%   \cs{TELsplit} dient zur Aufteilung einer zusammengesetzten
%   Telefonnummer (Vorwahl, Hauptnummer, Nebenstelle). In dieser
%   Implementation werden als Trennzeichen nur '/' und '-' erkannt.
%   Die einzelnen Bestandteile wie Vorwahl werden dann dem Befehl
%   \cs{telnumber} zur Formatierung uebergeben.
% \item Die Erkennung von einfachen Leerzeichen ist um einiges
%   schwieriger: Die Tokentrennung ueber Parameter |#1#2| funktioniert
%   nicht f\"ur einfache Leerzeichen, da TeX sie \emph{niemals} als
%   eigenst\"andige Argumente behandelt! (The TeXbook, Chapter 20,
%   p. 201)
%
%   (Anmerkung am Rande: Deshalb funktionieren die entsprechenden
%   Tokenmakros auf S. 149 des Buches "`Einf\"uhrung in TeX"' von
%   N. Schwarz (3. Aufl.) nicht, wenn im Tokenregister als erstes
%   ein einfaches Leerzeichen steht!)
% \end{itemize}
% \end{otherlanguage*}
%
% \StopEventually{
% }
%
% \section{Implementation}
%
%    \begin{macrocode}
%<*package>
%    \end{macrocode}
%
% \subsection{Reload check and package identification}
%    Reload check, especially if the package is not used with \LaTeX.
%    \begin{macrocode}
\begingroup\catcode61\catcode48\catcode32=10\relax%
  \catcode13=5 % ^^M
  \endlinechar=13 %
  \catcode35=6 % #
  \catcode39=12 % '
  \catcode44=12 % ,
  \catcode45=12 % -
  \catcode46=12 % .
  \catcode58=12 % :
  \catcode64=11 % @
  \catcode123=1 % {
  \catcode125=2 % }
  \expandafter\let\expandafter\x\csname ver@telprint.sty\endcsname
  \ifx\x\relax % plain-TeX, first loading
  \else
    \def\empty{}%
    \ifx\x\empty % LaTeX, first loading,
      % variable is initialized, but \ProvidesPackage not yet seen
    \else
      \expandafter\ifx\csname PackageInfo\endcsname\relax
        \def\x#1#2{%
          \immediate\write-1{Package #1 Info: #2.}%
        }%
      \else
        \def\x#1#2{\PackageInfo{#1}{#2, stopped}}%
      \fi
      \x{telprint}{The package is already loaded}%
      \aftergroup\endinput
    \fi
  \fi
\endgroup%
%    \end{macrocode}
%    Package identification:
%    \begin{macrocode}
\begingroup\catcode61\catcode48\catcode32=10\relax%
  \catcode13=5 % ^^M
  \endlinechar=13 %
  \catcode35=6 % #
  \catcode39=12 % '
  \catcode40=12 % (
  \catcode41=12 % )
  \catcode44=12 % ,
  \catcode45=12 % -
  \catcode46=12 % .
  \catcode47=12 % /
  \catcode58=12 % :
  \catcode64=11 % @
  \catcode91=12 % [
  \catcode93=12 % ]
  \catcode123=1 % {
  \catcode125=2 % }
  \expandafter\ifx\csname ProvidesPackage\endcsname\relax
    \def\x#1#2#3[#4]{\endgroup
      \immediate\write-1{Package: #3 #4}%
      \xdef#1{#4}%
    }%
  \else
    \def\x#1#2[#3]{\endgroup
      #2[{#3}]%
      \ifx#1\@undefined
        \xdef#1{#3}%
      \fi
      \ifx#1\relax
        \xdef#1{#3}%
      \fi
    }%
  \fi
\expandafter\x\csname ver@telprint.sty\endcsname
\ProvidesPackage{telprint}%
  [2016/05/16 v1.11 Format German phone numbers (HO)]%
%    \end{macrocode}
%
% \subsection{Catcodes}
%
%    \begin{macrocode}
\begingroup\catcode61\catcode48\catcode32=10\relax%
  \catcode13=5 % ^^M
  \endlinechar=13 %
  \catcode123=1 % {
  \catcode125=2 % }
  \catcode64=11 % @
  \def\x{\endgroup
    \expandafter\edef\csname TELAtEnd\endcsname{%
      \endlinechar=\the\endlinechar\relax
      \catcode13=\the\catcode13\relax
      \catcode32=\the\catcode32\relax
      \catcode35=\the\catcode35\relax
      \catcode61=\the\catcode61\relax
      \catcode64=\the\catcode64\relax
      \catcode123=\the\catcode123\relax
      \catcode125=\the\catcode125\relax
    }%
  }%
\x\catcode61\catcode48\catcode32=10\relax%
\catcode13=5 % ^^M
\endlinechar=13 %
\catcode35=6 % #
\catcode64=11 % @
\catcode123=1 % {
\catcode125=2 % }
\def\TMP@EnsureCode#1#2{%
  \edef\TELAtEnd{%
    \TELAtEnd
    \catcode#1=\the\catcode#1\relax
  }%
  \catcode#1=#2\relax
}
\TMP@EnsureCode{33}{12}% !
\TMP@EnsureCode{36}{3}% $
\TMP@EnsureCode{40}{12}% (
\TMP@EnsureCode{41}{12}% )
\TMP@EnsureCode{42}{12}% *
\TMP@EnsureCode{43}{12}% +
\TMP@EnsureCode{44}{12}% ,
\TMP@EnsureCode{45}{12}% -
\TMP@EnsureCode{46}{12}% .
\TMP@EnsureCode{47}{12}% /
\TMP@EnsureCode{91}{12}% [
\TMP@EnsureCode{93}{12}% ]
\TMP@EnsureCode{126}{13}% ~ (active)
\edef\TELAtEnd{\TELAtEnd\noexpand\endinput}
%    \end{macrocode}
%
% \subsection{Package macros}
%    \begin{macrocode}
\ifx\DeclareRobustCommand\UnDeFiNeD
  \def\DeclareRobustCommand*#1[1]{\def#1##1}%
  \def\TELreset{\let\DeclareRobustCommand=\UnDeFiNeD}%
  \input infwarerr.sty\relax
  \@PackageInfo{telprint}{%
    Macros are not robust!%
  }%
\else
  \let\TELreset=\relax
\fi
%    \end{macrocode}
%    \begin{macro}{\telspace}
%    \begin{macrocode}
\DeclareRobustCommand*{\telspace}[1]{\def\TELspace{#1}}
\telspace{{}$\,${}}
%    \end{macrocode}
%    \end{macro}
%    \begin{macro}{\telhyphen}
%    \begin{macrocode}
\DeclareRobustCommand*{\telhyphen}[1]{\def\TELhyphen{#1}}
\telhyphen{\leavevmode\hbox{-}}% \hbox zur Verhinderung der Trennung
%    \end{macrocode}
%    \end{macro}
%    \begin{macro}{\telslash}
%    \begin{macrocode}
\DeclareRobustCommand*{\telslash}[1]{\def\TELslash{#1}}
\telslash{/}%
%    \end{macrocode}
%    \end{macro}
%    \begin{macro}{\telleftparen}
%    \begin{macrocode}
\DeclareRobustCommand*{\telleftparen}[1]{\def\TELleftparen{#1}}
\telleftparen{(}%
%    \end{macrocode}
%    \end{macro}
%    \begin{macro}{\telrightparen}
%    \begin{macrocode}
\DeclareRobustCommand*{\telrightparen}[1]{\def\TELrightparen{#1}}
\telrightparen{)}%
%    \end{macrocode}
%    \end{macro}
%    \begin{macro}{\telplus}
%    \begin{macrocode}
\DeclareRobustCommand*{\telplus}[1]{\def\TELplus{#1}}
\telplus{+}%
%    \end{macrocode}
%    \end{macro}
%    \begin{macro}{\teltilde}
%    \begin{macrocode}
\DeclareRobustCommand*{\teltilde}[1]{\def\TELtilde{#1}}
\teltilde{~}%
%    \end{macrocode}
%    \end{macro}
%    \begin{macro}{\TELtoks}
%    \begin{macrocode}
\newtoks\TELtoks
%    \end{macrocode}
%    \end{macro}
%    \begin{macro}{\TELnumber}
%    \begin{macrocode}
\def\TELnumber#1#2\TELnumberEND{%
  \begingroup
  \def\0{#2}%
  \expandafter\endgroup
  \ifx\0\empty
    \TELtoks=\expandafter{\the\TELtoks#1}%
    \ifnum\TELswitch=0 %
      \def\TELx{\TELspace}\def\TELy{}%
    \else
      \def\TELx{}\def\TELy{\TELspace}%
    \fi
    \the\TELtoks
  \else
    \ifnum\TELswitch=0 %
      \TELtoks=\expandafter{\the\TELtoks#1\TELx}%
      \def\TELswitch{1}%
    \else
      \TELtoks=\expandafter{\the\TELtoks#1\TELy}%
      \def\TELswitch{0}%
    \fi
    \TELnumber#2\TELnumberEND
  \fi
}
%    \end{macrocode}
%    \end{macro}
%    \begin{macro}{\telnumber}
%    \begin{macrocode}
\DeclareRobustCommand*{\telnumber}[1]{%
  \TELtoks={}%
  \def\TELswitch{0}%
  \TELnumber#1{}\TELnumberEND
}
%    \end{macrocode}
%    \end{macro}
%    \begin{macro}{\TELsplit}
%    \begin{macrocode}
\def\TELsplit{\futurelet\TELfuture\TELdosplit}
%    \end{macrocode}
%    \end{macro}
%    \begin{macro}{\TELdosplit}
%    \begin{macrocode}
\def\TELdosplit#1#2\TELsplitEND
{%
  \def\TELsp{ }%
  \expandafter\ifx\TELsp\TELfuture
    \let\TELfuture=\relax
    \expandafter\telnumber\expandafter{\the\TELtoks}~%
    \telprint{#1#2}% Das Leerzeichen kann nicht #1 sein!
  \else
    \def\TELfirst{#1}%
    \ifx\TELfirst\empty
      \expandafter\telnumber\expandafter{\the\TELtoks}%
      \TELtoks={}%
    \else\if-\TELfirst
      \expandafter\telnumber\expandafter{\the\TELtoks}\TELhyphen
      \telprint{#2}%
    \else\if/\TELfirst
      \expandafter\telnumber\expandafter{\the\TELtoks}\TELslash
      \telprint{#2}%
    \else\if(\TELfirst
      \expandafter\telnumber\expandafter{\the\TELtoks}\TELleftparen
      \telprint{#2}%
    \else\if)\TELfirst
      \expandafter\telnumber\expandafter{\the\TELtoks}\TELrightparen
      \telprint{#2}%
    \else\if+\TELfirst
      \expandafter\telnumber\expandafter{\the\TELtoks}\TELplus
      \telprint{#2}%
    \else\def\TELtemp{~}\ifx\TELtemp\TELfirst
      \expandafter\telnumber\expandafter{\the\TELtoks}\TELtilde
      \telprint{#2}%
    \else
      \TELtoks=\expandafter{\the\TELtoks#1}%
      \TELsplit#2{}\TELsplitEND
    \fi\fi\fi\fi\fi\fi\fi
  \fi
}
%    \end{macrocode}
%    \end{macro}
%    \begin{macro}{\telprint}
%    \begin{macrocode}
\DeclareRobustCommand*{\telprint}[1]{%
  \TELtoks={}%
  \TELsplit#1{}\TELsplitEND
}
%    \end{macrocode}
%    \end{macro}
%    \begin{macrocode}
\TELreset\let\TELreset=\UnDeFiNeD
%    \end{macrocode}
%
%    \begin{macrocode}
\TELAtEnd%
%</package>
%    \end{macrocode}
%
% \section{Test}
%
% \subsection{Catcode checks for loading}
%
%    \begin{macrocode}
%<*test1>
%    \end{macrocode}
%    \begin{macrocode}
\catcode`\{=1 %
\catcode`\}=2 %
\catcode`\#=6 %
\catcode`\@=11 %
\expandafter\ifx\csname count@\endcsname\relax
  \countdef\count@=255 %
\fi
\expandafter\ifx\csname @gobble\endcsname\relax
  \long\def\@gobble#1{}%
\fi
\expandafter\ifx\csname @firstofone\endcsname\relax
  \long\def\@firstofone#1{#1}%
\fi
\expandafter\ifx\csname loop\endcsname\relax
  \expandafter\@firstofone
\else
  \expandafter\@gobble
\fi
{%
  \def\loop#1\repeat{%
    \def\body{#1}%
    \iterate
  }%
  \def\iterate{%
    \body
      \let\next\iterate
    \else
      \let\next\relax
    \fi
    \next
  }%
  \let\repeat=\fi
}%
\def\RestoreCatcodes{}
\count@=0 %
\loop
  \edef\RestoreCatcodes{%
    \RestoreCatcodes
    \catcode\the\count@=\the\catcode\count@\relax
  }%
\ifnum\count@<255 %
  \advance\count@ 1 %
\repeat

\def\RangeCatcodeInvalid#1#2{%
  \count@=#1\relax
  \loop
    \catcode\count@=15 %
  \ifnum\count@<#2\relax
    \advance\count@ 1 %
  \repeat
}
\def\RangeCatcodeCheck#1#2#3{%
  \count@=#1\relax
  \loop
    \ifnum#3=\catcode\count@
    \else
      \errmessage{%
        Character \the\count@\space
        with wrong catcode \the\catcode\count@\space
        instead of \number#3%
      }%
    \fi
  \ifnum\count@<#2\relax
    \advance\count@ 1 %
  \repeat
}
\def\space{ }
\expandafter\ifx\csname LoadCommand\endcsname\relax
  \def\LoadCommand{\input telprint.sty\relax}%
\fi
\def\Test{%
  \RangeCatcodeInvalid{0}{47}%
  \RangeCatcodeInvalid{58}{64}%
  \RangeCatcodeInvalid{91}{96}%
  \RangeCatcodeInvalid{123}{255}%
  \catcode`\@=12 %
  \catcode`\\=0 %
  \catcode`\%=14 %
  \LoadCommand
  \RangeCatcodeCheck{0}{36}{15}%
  \RangeCatcodeCheck{37}{37}{14}%
  \RangeCatcodeCheck{38}{47}{15}%
  \RangeCatcodeCheck{48}{57}{12}%
  \RangeCatcodeCheck{58}{63}{15}%
  \RangeCatcodeCheck{64}{64}{12}%
  \RangeCatcodeCheck{65}{90}{11}%
  \RangeCatcodeCheck{91}{91}{15}%
  \RangeCatcodeCheck{92}{92}{0}%
  \RangeCatcodeCheck{93}{96}{15}%
  \RangeCatcodeCheck{97}{122}{11}%
  \RangeCatcodeCheck{123}{255}{15}%
  \RestoreCatcodes
}
\Test
\csname @@end\endcsname
\end
%    \end{macrocode}
%    \begin{macrocode}
%</test1>
%    \end{macrocode}
%
% \section{Installation}
%
% \subsection{Download}
%
% \paragraph{Package.} This package is available on
% CTAN\footnote{\url{https://ctan.org/pkg/telprint}}:
% \begin{description}
% \item[\CTAN{macros/latex/contrib/oberdiek/telprint.dtx}] The source file.
% \item[\CTAN{macros/latex/contrib/oberdiek/telprint.pdf}] Documentation.
% \end{description}
%
%
% \paragraph{Bundle.} All the packages of the bundle `oberdiek'
% are also available in a TDS compliant ZIP archive. There
% the packages are already unpacked and the documentation files
% are generated. The files and directories obey the TDS standard.
% \begin{description}
% \item[\CTANinstall{install/macros/latex/contrib/oberdiek.tds.zip}]
% \end{description}
% \emph{TDS} refers to the standard ``A Directory Structure
% for \TeX\ Files'' (\CTAN{tds/tds.pdf}). Directories
% with \xfile{texmf} in their name are usually organized this way.
%
% \subsection{Bundle installation}
%
% \paragraph{Unpacking.} Unpack the \xfile{oberdiek.tds.zip} in the
% TDS tree (also known as \xfile{texmf} tree) of your choice.
% Example (linux):
% \begin{quote}
%   |unzip oberdiek.tds.zip -d ~/texmf|
% \end{quote}
%
% \paragraph{Script installation.}
% Check the directory \xfile{TDS:scripts/oberdiek/} for
% scripts that need further installation steps.
% Package \xpackage{attachfile2} comes with the Perl script
% \xfile{pdfatfi.pl} that should be installed in such a way
% that it can be called as \texttt{pdfatfi}.
% Example (linux):
% \begin{quote}
%   |chmod +x scripts/oberdiek/pdfatfi.pl|\\
%   |cp scripts/oberdiek/pdfatfi.pl /usr/local/bin/|
% \end{quote}
%
% \subsection{Package installation}
%
% \paragraph{Unpacking.} The \xfile{.dtx} file is a self-extracting
% \docstrip\ archive. The files are extracted by running the
% \xfile{.dtx} through \plainTeX:
% \begin{quote}
%   \verb|tex telprint.dtx|
% \end{quote}
%
% \paragraph{TDS.} Now the different files must be moved into
% the different directories in your installation TDS tree
% (also known as \xfile{texmf} tree):
% \begin{quote}
% \def\t{^^A
% \begin{tabular}{@{}>{\ttfamily}l@{ $\rightarrow$ }>{\ttfamily}l@{}}
%   telprint.sty & tex/generic/oberdiek/telprint.sty\\
%   telprint.pdf & doc/latex/oberdiek/telprint.pdf\\
%   test/telprint-test1.tex & doc/latex/oberdiek/test/telprint-test1.tex\\
%   telprint.dtx & source/latex/oberdiek/telprint.dtx\\
% \end{tabular}^^A
% }^^A
% \sbox0{\t}^^A
% \ifdim\wd0>\linewidth
%   \begingroup
%     \advance\linewidth by\leftmargin
%     \advance\linewidth by\rightmargin
%   \edef\x{\endgroup
%     \def\noexpand\lw{\the\linewidth}^^A
%   }\x
%   \def\lwbox{^^A
%     \leavevmode
%     \hbox to \linewidth{^^A
%       \kern-\leftmargin\relax
%       \hss
%       \usebox0
%       \hss
%       \kern-\rightmargin\relax
%     }^^A
%   }^^A
%   \ifdim\wd0>\lw
%     \sbox0{\small\t}^^A
%     \ifdim\wd0>\linewidth
%       \ifdim\wd0>\lw
%         \sbox0{\footnotesize\t}^^A
%         \ifdim\wd0>\linewidth
%           \ifdim\wd0>\lw
%             \sbox0{\scriptsize\t}^^A
%             \ifdim\wd0>\linewidth
%               \ifdim\wd0>\lw
%                 \sbox0{\tiny\t}^^A
%                 \ifdim\wd0>\linewidth
%                   \lwbox
%                 \else
%                   \usebox0
%                 \fi
%               \else
%                 \lwbox
%               \fi
%             \else
%               \usebox0
%             \fi
%           \else
%             \lwbox
%           \fi
%         \else
%           \usebox0
%         \fi
%       \else
%         \lwbox
%       \fi
%     \else
%       \usebox0
%     \fi
%   \else
%     \lwbox
%   \fi
% \else
%   \usebox0
% \fi
% \end{quote}
% If you have a \xfile{docstrip.cfg} that configures and enables \docstrip's
% TDS installing feature, then some files can already be in the right
% place, see the documentation of \docstrip.
%
% \subsection{Refresh file name databases}
%
% If your \TeX~distribution
% (\teTeX, \mikTeX, \dots) relies on file name databases, you must refresh
% these. For example, \teTeX\ users run \verb|texhash| or
% \verb|mktexlsr|.
%
% \subsection{Some details for the interested}
%
% \paragraph{Attached source.}
%
% The PDF documentation on CTAN also includes the
% \xfile{.dtx} source file. It can be extracted by
% AcrobatReader 6 or higher. Another option is \textsf{pdftk},
% e.g. unpack the file into the current directory:
% \begin{quote}
%   \verb|pdftk telprint.pdf unpack_files output .|
% \end{quote}
%
% \paragraph{Unpacking with \LaTeX.}
% The \xfile{.dtx} chooses its action depending on the format:
% \begin{description}
% \item[\plainTeX:] Run \docstrip\ and extract the files.
% \item[\LaTeX:] Generate the documentation.
% \end{description}
% If you insist on using \LaTeX\ for \docstrip\ (really,
% \docstrip\ does not need \LaTeX), then inform the autodetect routine
% about your intention:
% \begin{quote}
%   \verb|latex \let\install=y\input{telprint.dtx}|
% \end{quote}
% Do not forget to quote the argument according to the demands
% of your shell.
%
% \paragraph{Generating the documentation.}
% You can use both the \xfile{.dtx} or the \xfile{.drv} to generate
% the documentation. The process can be configured by the
% configuration file \xfile{ltxdoc.cfg}. For instance, put this
% line into this file, if you want to have A4 as paper format:
% \begin{quote}
%   \verb|\PassOptionsToClass{a4paper}{article}|
% \end{quote}
% An example follows how to generate the
% documentation with pdf\LaTeX:
% \begin{quote}
%\begin{verbatim}
%pdflatex telprint.dtx
%makeindex -s gind.ist telprint.idx
%pdflatex telprint.dtx
%makeindex -s gind.ist telprint.idx
%pdflatex telprint.dtx
%\end{verbatim}
% \end{quote}
%
% \begin{History}
%   \begin{Version}{1996/11/28 v1.0}
%   \item
%     Erste lauff\"ahige Version.
%   \item
%     Nur '-' und '/' als zul\"assige Sonderzeichen.
%   \end{Version}
%   \begin{Version}{1997/09/16 v1.1}
%   \item
%     Dokumentation und Kommentare (Posting in de.comp.text.tex).
%   \item
%     Erweiterung um Sonderzeichen '(', ')', '+', '\textasciitilde' und ' '.
%   \item
%     Trennungsverhinderung am 'hyphen'.
%   \end{Version}
%   \begin{Version}{1997/10/16 v1.2}
%   \item
%     Schutz vor wiederholtem Einlesen.
%   \item
%     Unter \LaTeXe\ Nutzung des \cs{DeclareRobustCommand}-Features.
%   \end{Version}
%   \begin{Version}{1997/12/09 v1.3}
%   \item
%     Tempor\"are Variable eingespart.
%   \item
%     Posted in newsgroup \xnewsgroup{de.comp.text.tex}:\\
%     \URL{``\link{Re: Generisches Markup f\"ur Telefonnummern?}''}^^A
%     {http://groups.google.com/group/de.comp.text.tex/msg/86b3a86140007309}
%   \end{Version}
%   \begin{Version}{2004/11/02 v1.4}
%   \item
%     Fehler in der Dokumentation korrigiert.
%   \end{Version}
%   \begin{Version}{2005/09/30 v1.5}
%   \item
%     Konfigurierbare Symbole: '/', '(', ')', '+' und '\textasciitilde'.
%   \end{Version}
%   \begin{Version}{2006/02/12 v1.6}
%   \item
%     LPPL 1.3.
%   \item
%     Kurze \"Ubersicht in Englisch.
%   \item
%     CTAN.
%   \end{Version}
%   \begin{Version}{2006/08/26 v1.7}
%   \item
%     New DTX framework.
%   \end{Version}
%   \begin{Version}{2007/04/11 v1.8}
%   \item
%     Line ends sanitized.
%   \end{Version}
%   \begin{Version}{2007/09/09 v1.9}
%   \item
%     Catcode section added.
%   \item
%     Missing docstrip tag added.
%   \end{Version}
%   \begin{Version}{2008/08/11 v1.10}
%   \item
%     Code is not changed.
%   \item
%     URLs updated.
%   \end{Version}
%   \begin{Version}{2016/05/16 v1.11}
%   \item
%     Documentation updates.
%   \end{Version}
% \end{History}
%
% \PrintIndex
%
% \Finale
\endinput

%        (quote the arguments according to the demands of your shell)
%
% Documentation:
%    (a) If telprint.drv is present:
%           latex telprint.drv
%    (b) Without telprint.drv:
%           latex telprint.dtx; ...
%    The class ltxdoc loads the configuration file ltxdoc.cfg
%    if available. Here you can specify further options, e.g.
%    use A4 as paper format:
%       \PassOptionsToClass{a4paper}{article}
%
%    Programm calls to get the documentation (example):
%       pdflatex telprint.dtx
%       makeindex -s gind.ist telprint.idx
%       pdflatex telprint.dtx
%       makeindex -s gind.ist telprint.idx
%       pdflatex telprint.dtx
%
% Installation:
%    TDS:tex/generic/oberdiek/telprint.sty
%    TDS:doc/latex/oberdiek/telprint.pdf
%    TDS:doc/latex/oberdiek/test/telprint-test1.tex
%    TDS:source/latex/oberdiek/telprint.dtx
%
%<*ignore>
\begingroup
  \catcode123=1 %
  \catcode125=2 %
  \def\x{LaTeX2e}%
\expandafter\endgroup
\ifcase 0\ifx\install y1\fi\expandafter
         \ifx\csname processbatchFile\endcsname\relax\else1\fi
         \ifx\fmtname\x\else 1\fi\relax
\else\csname fi\endcsname
%</ignore>
%<*install>
\input docstrip.tex
\Msg{************************************************************************}
\Msg{* Installation}
\Msg{* Package: telprint 2016/05/16 v1.11 Format German phone numbers (HO)}
\Msg{************************************************************************}

\keepsilent
\askforoverwritefalse

\let\MetaPrefix\relax
\preamble

This is a generated file.

Project: telprint
Version: 2016/05/16 v1.11

Copyright (C) 1996, 1997, 2004-2008 by
   Heiko Oberdiek <heiko.oberdiek at googlemail.com>

This work may be distributed and/or modified under the
conditions of the LaTeX Project Public License, either
version 1.3c of this license or (at your option) any later
version. This version of this license is in
   http://www.latex-project.org/lppl/lppl-1-3c.txt
and the latest version of this license is in
   http://www.latex-project.org/lppl.txt
and version 1.3 or later is part of all distributions of
LaTeX version 2005/12/01 or later.

This work has the LPPL maintenance status "maintained".

This Current Maintainer of this work is Heiko Oberdiek.

The Base Interpreter refers to any `TeX-Format',
because some files are installed in TDS:tex/generic//.

This work consists of the main source file telprint.dtx
and the derived files
   telprint.sty, telprint.pdf, telprint.ins, telprint.drv,
   telprint-test1.tex.

\endpreamble
\let\MetaPrefix\DoubleperCent

\generate{%
  \file{telprint.ins}{\from{telprint.dtx}{install}}%
  \file{telprint.drv}{\from{telprint.dtx}{driver}}%
  \usedir{tex/generic/oberdiek}%
  \file{telprint.sty}{\from{telprint.dtx}{package}}%
%  \usedir{doc/latex/oberdiek/test}%
%  \file{telprint-test1.tex}{\from{telprint.dtx}{test1}}%
  \nopreamble
  \nopostamble
%  \usedir{source/latex/oberdiek/catalogue}%
%  \file{telprint.xml}{\from{telprint.dtx}{catalogue}}%
}

\catcode32=13\relax% active space
\let =\space%
\Msg{************************************************************************}
\Msg{*}
\Msg{* To finish the installation you have to move the following}
\Msg{* file into a directory searched by TeX:}
\Msg{*}
\Msg{*     telprint.sty}
\Msg{*}
\Msg{* To produce the documentation run the file `telprint.drv'}
\Msg{* through LaTeX.}
\Msg{*}
\Msg{* Happy TeXing!}
\Msg{*}
\Msg{************************************************************************}

\endbatchfile
%</install>
%<*ignore>
\fi
%</ignore>
%<*driver>
\NeedsTeXFormat{LaTeX2e}
\ProvidesFile{telprint.drv}%
  [2016/05/16 v1.11 Format German phone numbers (HO)]%
\documentclass{ltxdoc}
\usepackage{holtxdoc}[2011/11/22]
\usepackage[ngerman,english]{babel}
\begin{document}
  \DocInput{telprint.dtx}%
\end{document}
%</driver>
% \fi
%
%
% \CharacterTable
%  {Upper-case    \A\B\C\D\E\F\G\H\I\J\K\L\M\N\O\P\Q\R\S\T\U\V\W\X\Y\Z
%   Lower-case    \a\b\c\d\e\f\g\h\i\j\k\l\m\n\o\p\q\r\s\t\u\v\w\x\y\z
%   Digits        \0\1\2\3\4\5\6\7\8\9
%   Exclamation   \!     Double quote  \"     Hash (number) \#
%   Dollar        \$     Percent       \%     Ampersand     \&
%   Acute accent  \'     Left paren    \(     Right paren   \)
%   Asterisk      \*     Plus          \+     Comma         \,
%   Minus         \-     Point         \.     Solidus       \/
%   Colon         \:     Semicolon     \;     Less than     \<
%   Equals        \=     Greater than  \>     Question mark \?
%   Commercial at \@     Left bracket  \[     Backslash     \\
%   Right bracket \]     Circumflex    \^     Underscore    \_
%   Grave accent  \`     Left brace    \{     Vertical bar  \|
%   Right brace   \}     Tilde         \~}
%
% \GetFileInfo{telprint.drv}
%
% \title{The \xpackage{telprint} package}
% \date{2016/05/16 v1.11}
% \author{Heiko Oberdiek\thanks
% {Please report any issues at https://github.com/ho-tex/oberdiek/issues}\\
% \xemail{heiko.oberdiek at googlemail.com}}
%
% \maketitle
%
% \begin{abstract}
% Package \xpackage{telprint} provides \cs{telprint} for formatting
% German phone numbers.
% \end{abstract}
%
% \tableofcontents
%
% \section{Documentation}
%
% \subsection{Introduction}
%
%            This is a very old package that I have written
%            to format phone numbers. It follows German
%            conventions and the documentation is mainly in German.
%
% \subsection{Short overview in English}
%
% \LaTeX:
% \begin{quote}
% |\usepackage{telprint}|\\
% |\telprint{123/456-789}|\\
% \end{quote}
% \plainTeX:
% \begin{quote}
%   |\input telprint.sty|\\
%   |\telprint{123/456-789}|
% \end{quote}
%
% \DescribeMacro\telprint
% |\telprint{...}| formats the explicitly given number.
%     Digits, spaces and some special characters
%     ('+', '/', '-', '(', ')', '\textasciitilde', ' ') are supported.
%     Numbers are divided into groups of two digits from the right.
% Examples:
% \begin{quote}
%     |\telprint{0761/12345}     ==> 07\,61/1\,23\,45|\\
%     |\telprint{01234/567-89}   ==> 0\,12\,34/5\,67\leavevmode\hbox{-}89|\\
%     |\telprint{+49 (6221) 297} ==> +49~(62\,21)~2\,97|
% \end{quote}
%
% \subsubsection{Configuration}
%
% The output of the symbols can be configured by
% \cs{telhyphen}, \cs{telslash}, \cs{telleftparen}, \cs{telrightparen},
% \cs{telplus}, \cs{teltilde}.
% Example:
% \begin{quote}
%   |\telslash{\,/\,}\\|
%   |\telprint{12/34} ==> 12\,/\,34|
% \end{quote}
%
% \DescribeMacro\telspace
% \cs{telspace} configures the space between digit groups.
%
% \DescribeMacro\telnumber
% \cs{telnumber} only formats a number in digit groups; special
%    characters are not recognized.
%
% \subsection{Documentation in German}
%
% \begin{otherlanguage*}{ngerman}
% \hyphenation{To-ken-ma-kros}
% \begin{itemize}
% \item \DescribeMacro\telprint |telprint#1|\\
%   Der eigentliche Anwenderbefehl zur formatierten Ausgabe von
%   Telefonnummern. Diese d\"urfen dabei nur als Zahlen angegeben
%   werden(, da sie tokenweise analysiert werden).
%   Als Trenn- oder Sonderzeichen werden unterst\"utzt:
%   '+', '/', '-', '(', ')', '\textasciitilde', ' '
%   Einfache Leerzeichen werden erkannt und durch Tilden ersetzt, um
%   Trennungen in der Telefonnummer zu verhindern. (Man beachte aus
%   gleichem Grunde die \cs{hbox} bei '-'.)
%   Beispiele:
%   \begin{quote}
%     |\telprint{0761/12345}     ==> 07\,61/1\,23\,45|\\
%     |\telprint{01234/567-89}   ==> 0\,12\,34/5\,67\leavevmode\hbox{-}89|\\
%     |\telprint{+49 (6221) 297} ==> +49~(62\,21)~2\,97|
%   \end{quote}
% \end{itemize}
% Der Rest enth\"alt eher Technisches:
% \begin{itemize}
% \item \DescribeMacro\telspace |\telspace#1|\\
%   Mit diesem Befehl wird der Abstand zwischen den Zifferngruppen
%   angegeben (Default: |\,|).
%   (Durch |\telspace{}| kann dieser zusaetzliche Abstand abgestellt
%   werden.)
% \item \DescribeMacro\telhyphen |\telhyphen#1|\\
%   Dieser Befehl gibt die Art des Bindestriches, wie er ausgegeben
%   werden soll. In der Eingabe darf jedoch nur der einfache
%   Bindestrich stehen:
%   |\telprint{123-45}|, jedoch NIE |\telprint{123--45}|!
%   Kopka-Bindestrich-Fans geben an:
%   |\telhyphen{\leavevmode\hbox{--}}|
% \item
%   \DescribeMacro{\telslash}
%   \DescribeMacro{\telleftparen}
%   \DescribeMacro{\telrightparen}
%   \DescribeMacro{\telplus}
%   \DescribeMacro{\teltilde}
%   |\telslash#1|, |\telleftparen#1|, |\telrightparen#1|, |\telplus#1|,
%   |\teltilde|\\
%   Diese Befehle konfigurieren die Zeichen '/', '(', ')', '+'
%   und '\textasciitilde'. Sie funktionieren analog zu \cs{telhyphen}.
% \item \DescribeMacro\telnumber |\telnumber#1|\\
%   Richtung interner Befehl: Er dient dazu, eine Zifferngruppe
%   in Zweiergruppen auszugeben.
%   Die einzelnen Zahlen werden im Tokenregister \cs{TELtoks}
%   gespeichert. Abwechselnd werden dabei zwischen zwei Token
%   (Zahlen) \cs{TELx} bzw. \cs{TELy} eingefuegt, abh\"angig von dem
%   wechselnden Wert von \cs{TELswitch}. Zum Schluss kann dann einfach
%   festgestellt werden ob die Nummer nun eine geradzahlige oder
%   ungeradzahlige Zahl von Ziffern aufwies. Dem entsprechend wird
%   \cs{TELx} mit dem Zusatzabstand belegt und \cs{TELy} leer definiert
%   oder umgekehrt. )
% \item |\TEL...| interne Befehle, Technisches:\\
%   \cs{TELsplit} dient zur Aufteilung einer zusammengesetzten
%   Telefonnummer (Vorwahl, Hauptnummer, Nebenstelle). In dieser
%   Implementation werden als Trennzeichen nur '/' und '-' erkannt.
%   Die einzelnen Bestandteile wie Vorwahl werden dann dem Befehl
%   \cs{telnumber} zur Formatierung uebergeben.
% \item Die Erkennung von einfachen Leerzeichen ist um einiges
%   schwieriger: Die Tokentrennung ueber Parameter |#1#2| funktioniert
%   nicht f\"ur einfache Leerzeichen, da TeX sie \emph{niemals} als
%   eigenst\"andige Argumente behandelt! (The TeXbook, Chapter 20,
%   p. 201)
%
%   (Anmerkung am Rande: Deshalb funktionieren die entsprechenden
%   Tokenmakros auf S. 149 des Buches "`Einf\"uhrung in TeX"' von
%   N. Schwarz (3. Aufl.) nicht, wenn im Tokenregister als erstes
%   ein einfaches Leerzeichen steht!)
% \end{itemize}
% \end{otherlanguage*}
%
% \StopEventually{
% }
%
% \section{Implementation}
%
%    \begin{macrocode}
%<*package>
%    \end{macrocode}
%
% \subsection{Reload check and package identification}
%    Reload check, especially if the package is not used with \LaTeX.
%    \begin{macrocode}
\begingroup\catcode61\catcode48\catcode32=10\relax%
  \catcode13=5 % ^^M
  \endlinechar=13 %
  \catcode35=6 % #
  \catcode39=12 % '
  \catcode44=12 % ,
  \catcode45=12 % -
  \catcode46=12 % .
  \catcode58=12 % :
  \catcode64=11 % @
  \catcode123=1 % {
  \catcode125=2 % }
  \expandafter\let\expandafter\x\csname ver@telprint.sty\endcsname
  \ifx\x\relax % plain-TeX, first loading
  \else
    \def\empty{}%
    \ifx\x\empty % LaTeX, first loading,
      % variable is initialized, but \ProvidesPackage not yet seen
    \else
      \expandafter\ifx\csname PackageInfo\endcsname\relax
        \def\x#1#2{%
          \immediate\write-1{Package #1 Info: #2.}%
        }%
      \else
        \def\x#1#2{\PackageInfo{#1}{#2, stopped}}%
      \fi
      \x{telprint}{The package is already loaded}%
      \aftergroup\endinput
    \fi
  \fi
\endgroup%
%    \end{macrocode}
%    Package identification:
%    \begin{macrocode}
\begingroup\catcode61\catcode48\catcode32=10\relax%
  \catcode13=5 % ^^M
  \endlinechar=13 %
  \catcode35=6 % #
  \catcode39=12 % '
  \catcode40=12 % (
  \catcode41=12 % )
  \catcode44=12 % ,
  \catcode45=12 % -
  \catcode46=12 % .
  \catcode47=12 % /
  \catcode58=12 % :
  \catcode64=11 % @
  \catcode91=12 % [
  \catcode93=12 % ]
  \catcode123=1 % {
  \catcode125=2 % }
  \expandafter\ifx\csname ProvidesPackage\endcsname\relax
    \def\x#1#2#3[#4]{\endgroup
      \immediate\write-1{Package: #3 #4}%
      \xdef#1{#4}%
    }%
  \else
    \def\x#1#2[#3]{\endgroup
      #2[{#3}]%
      \ifx#1\@undefined
        \xdef#1{#3}%
      \fi
      \ifx#1\relax
        \xdef#1{#3}%
      \fi
    }%
  \fi
\expandafter\x\csname ver@telprint.sty\endcsname
\ProvidesPackage{telprint}%
  [2016/05/16 v1.11 Format German phone numbers (HO)]%
%    \end{macrocode}
%
% \subsection{Catcodes}
%
%    \begin{macrocode}
\begingroup\catcode61\catcode48\catcode32=10\relax%
  \catcode13=5 % ^^M
  \endlinechar=13 %
  \catcode123=1 % {
  \catcode125=2 % }
  \catcode64=11 % @
  \def\x{\endgroup
    \expandafter\edef\csname TELAtEnd\endcsname{%
      \endlinechar=\the\endlinechar\relax
      \catcode13=\the\catcode13\relax
      \catcode32=\the\catcode32\relax
      \catcode35=\the\catcode35\relax
      \catcode61=\the\catcode61\relax
      \catcode64=\the\catcode64\relax
      \catcode123=\the\catcode123\relax
      \catcode125=\the\catcode125\relax
    }%
  }%
\x\catcode61\catcode48\catcode32=10\relax%
\catcode13=5 % ^^M
\endlinechar=13 %
\catcode35=6 % #
\catcode64=11 % @
\catcode123=1 % {
\catcode125=2 % }
\def\TMP@EnsureCode#1#2{%
  \edef\TELAtEnd{%
    \TELAtEnd
    \catcode#1=\the\catcode#1\relax
  }%
  \catcode#1=#2\relax
}
\TMP@EnsureCode{33}{12}% !
\TMP@EnsureCode{36}{3}% $
\TMP@EnsureCode{40}{12}% (
\TMP@EnsureCode{41}{12}% )
\TMP@EnsureCode{42}{12}% *
\TMP@EnsureCode{43}{12}% +
\TMP@EnsureCode{44}{12}% ,
\TMP@EnsureCode{45}{12}% -
\TMP@EnsureCode{46}{12}% .
\TMP@EnsureCode{47}{12}% /
\TMP@EnsureCode{91}{12}% [
\TMP@EnsureCode{93}{12}% ]
\TMP@EnsureCode{126}{13}% ~ (active)
\edef\TELAtEnd{\TELAtEnd\noexpand\endinput}
%    \end{macrocode}
%
% \subsection{Package macros}
%    \begin{macrocode}
\ifx\DeclareRobustCommand\UnDeFiNeD
  \def\DeclareRobustCommand*#1[1]{\def#1##1}%
  \def\TELreset{\let\DeclareRobustCommand=\UnDeFiNeD}%
  \input infwarerr.sty\relax
  \@PackageInfo{telprint}{%
    Macros are not robust!%
  }%
\else
  \let\TELreset=\relax
\fi
%    \end{macrocode}
%    \begin{macro}{\telspace}
%    \begin{macrocode}
\DeclareRobustCommand*{\telspace}[1]{\def\TELspace{#1}}
\telspace{{}$\,${}}
%    \end{macrocode}
%    \end{macro}
%    \begin{macro}{\telhyphen}
%    \begin{macrocode}
\DeclareRobustCommand*{\telhyphen}[1]{\def\TELhyphen{#1}}
\telhyphen{\leavevmode\hbox{-}}% \hbox zur Verhinderung der Trennung
%    \end{macrocode}
%    \end{macro}
%    \begin{macro}{\telslash}
%    \begin{macrocode}
\DeclareRobustCommand*{\telslash}[1]{\def\TELslash{#1}}
\telslash{/}%
%    \end{macrocode}
%    \end{macro}
%    \begin{macro}{\telleftparen}
%    \begin{macrocode}
\DeclareRobustCommand*{\telleftparen}[1]{\def\TELleftparen{#1}}
\telleftparen{(}%
%    \end{macrocode}
%    \end{macro}
%    \begin{macro}{\telrightparen}
%    \begin{macrocode}
\DeclareRobustCommand*{\telrightparen}[1]{\def\TELrightparen{#1}}
\telrightparen{)}%
%    \end{macrocode}
%    \end{macro}
%    \begin{macro}{\telplus}
%    \begin{macrocode}
\DeclareRobustCommand*{\telplus}[1]{\def\TELplus{#1}}
\telplus{+}%
%    \end{macrocode}
%    \end{macro}
%    \begin{macro}{\teltilde}
%    \begin{macrocode}
\DeclareRobustCommand*{\teltilde}[1]{\def\TELtilde{#1}}
\teltilde{~}%
%    \end{macrocode}
%    \end{macro}
%    \begin{macro}{\TELtoks}
%    \begin{macrocode}
\newtoks\TELtoks
%    \end{macrocode}
%    \end{macro}
%    \begin{macro}{\TELnumber}
%    \begin{macrocode}
\def\TELnumber#1#2\TELnumberEND{%
  \begingroup
  \def\0{#2}%
  \expandafter\endgroup
  \ifx\0\empty
    \TELtoks=\expandafter{\the\TELtoks#1}%
    \ifnum\TELswitch=0 %
      \def\TELx{\TELspace}\def\TELy{}%
    \else
      \def\TELx{}\def\TELy{\TELspace}%
    \fi
    \the\TELtoks
  \else
    \ifnum\TELswitch=0 %
      \TELtoks=\expandafter{\the\TELtoks#1\TELx}%
      \def\TELswitch{1}%
    \else
      \TELtoks=\expandafter{\the\TELtoks#1\TELy}%
      \def\TELswitch{0}%
    \fi
    \TELnumber#2\TELnumberEND
  \fi
}
%    \end{macrocode}
%    \end{macro}
%    \begin{macro}{\telnumber}
%    \begin{macrocode}
\DeclareRobustCommand*{\telnumber}[1]{%
  \TELtoks={}%
  \def\TELswitch{0}%
  \TELnumber#1{}\TELnumberEND
}
%    \end{macrocode}
%    \end{macro}
%    \begin{macro}{\TELsplit}
%    \begin{macrocode}
\def\TELsplit{\futurelet\TELfuture\TELdosplit}
%    \end{macrocode}
%    \end{macro}
%    \begin{macro}{\TELdosplit}
%    \begin{macrocode}
\def\TELdosplit#1#2\TELsplitEND
{%
  \def\TELsp{ }%
  \expandafter\ifx\TELsp\TELfuture
    \let\TELfuture=\relax
    \expandafter\telnumber\expandafter{\the\TELtoks}~%
    \telprint{#1#2}% Das Leerzeichen kann nicht #1 sein!
  \else
    \def\TELfirst{#1}%
    \ifx\TELfirst\empty
      \expandafter\telnumber\expandafter{\the\TELtoks}%
      \TELtoks={}%
    \else\if-\TELfirst
      \expandafter\telnumber\expandafter{\the\TELtoks}\TELhyphen
      \telprint{#2}%
    \else\if/\TELfirst
      \expandafter\telnumber\expandafter{\the\TELtoks}\TELslash
      \telprint{#2}%
    \else\if(\TELfirst
      \expandafter\telnumber\expandafter{\the\TELtoks}\TELleftparen
      \telprint{#2}%
    \else\if)\TELfirst
      \expandafter\telnumber\expandafter{\the\TELtoks}\TELrightparen
      \telprint{#2}%
    \else\if+\TELfirst
      \expandafter\telnumber\expandafter{\the\TELtoks}\TELplus
      \telprint{#2}%
    \else\def\TELtemp{~}\ifx\TELtemp\TELfirst
      \expandafter\telnumber\expandafter{\the\TELtoks}\TELtilde
      \telprint{#2}%
    \else
      \TELtoks=\expandafter{\the\TELtoks#1}%
      \TELsplit#2{}\TELsplitEND
    \fi\fi\fi\fi\fi\fi\fi
  \fi
}
%    \end{macrocode}
%    \end{macro}
%    \begin{macro}{\telprint}
%    \begin{macrocode}
\DeclareRobustCommand*{\telprint}[1]{%
  \TELtoks={}%
  \TELsplit#1{}\TELsplitEND
}
%    \end{macrocode}
%    \end{macro}
%    \begin{macrocode}
\TELreset\let\TELreset=\UnDeFiNeD
%    \end{macrocode}
%
%    \begin{macrocode}
\TELAtEnd%
%</package>
%    \end{macrocode}
%
% \section{Test}
%
% \subsection{Catcode checks for loading}
%
%    \begin{macrocode}
%<*test1>
%    \end{macrocode}
%    \begin{macrocode}
\catcode`\{=1 %
\catcode`\}=2 %
\catcode`\#=6 %
\catcode`\@=11 %
\expandafter\ifx\csname count@\endcsname\relax
  \countdef\count@=255 %
\fi
\expandafter\ifx\csname @gobble\endcsname\relax
  \long\def\@gobble#1{}%
\fi
\expandafter\ifx\csname @firstofone\endcsname\relax
  \long\def\@firstofone#1{#1}%
\fi
\expandafter\ifx\csname loop\endcsname\relax
  \expandafter\@firstofone
\else
  \expandafter\@gobble
\fi
{%
  \def\loop#1\repeat{%
    \def\body{#1}%
    \iterate
  }%
  \def\iterate{%
    \body
      \let\next\iterate
    \else
      \let\next\relax
    \fi
    \next
  }%
  \let\repeat=\fi
}%
\def\RestoreCatcodes{}
\count@=0 %
\loop
  \edef\RestoreCatcodes{%
    \RestoreCatcodes
    \catcode\the\count@=\the\catcode\count@\relax
  }%
\ifnum\count@<255 %
  \advance\count@ 1 %
\repeat

\def\RangeCatcodeInvalid#1#2{%
  \count@=#1\relax
  \loop
    \catcode\count@=15 %
  \ifnum\count@<#2\relax
    \advance\count@ 1 %
  \repeat
}
\def\RangeCatcodeCheck#1#2#3{%
  \count@=#1\relax
  \loop
    \ifnum#3=\catcode\count@
    \else
      \errmessage{%
        Character \the\count@\space
        with wrong catcode \the\catcode\count@\space
        instead of \number#3%
      }%
    \fi
  \ifnum\count@<#2\relax
    \advance\count@ 1 %
  \repeat
}
\def\space{ }
\expandafter\ifx\csname LoadCommand\endcsname\relax
  \def\LoadCommand{\input telprint.sty\relax}%
\fi
\def\Test{%
  \RangeCatcodeInvalid{0}{47}%
  \RangeCatcodeInvalid{58}{64}%
  \RangeCatcodeInvalid{91}{96}%
  \RangeCatcodeInvalid{123}{255}%
  \catcode`\@=12 %
  \catcode`\\=0 %
  \catcode`\%=14 %
  \LoadCommand
  \RangeCatcodeCheck{0}{36}{15}%
  \RangeCatcodeCheck{37}{37}{14}%
  \RangeCatcodeCheck{38}{47}{15}%
  \RangeCatcodeCheck{48}{57}{12}%
  \RangeCatcodeCheck{58}{63}{15}%
  \RangeCatcodeCheck{64}{64}{12}%
  \RangeCatcodeCheck{65}{90}{11}%
  \RangeCatcodeCheck{91}{91}{15}%
  \RangeCatcodeCheck{92}{92}{0}%
  \RangeCatcodeCheck{93}{96}{15}%
  \RangeCatcodeCheck{97}{122}{11}%
  \RangeCatcodeCheck{123}{255}{15}%
  \RestoreCatcodes
}
\Test
\csname @@end\endcsname
\end
%    \end{macrocode}
%    \begin{macrocode}
%</test1>
%    \end{macrocode}
%
% \section{Installation}
%
% \subsection{Download}
%
% \paragraph{Package.} This package is available on
% CTAN\footnote{\url{https://ctan.org/pkg/telprint}}:
% \begin{description}
% \item[\CTAN{macros/latex/contrib/oberdiek/telprint.dtx}] The source file.
% \item[\CTAN{macros/latex/contrib/oberdiek/telprint.pdf}] Documentation.
% \end{description}
%
%
% \paragraph{Bundle.} All the packages of the bundle `oberdiek'
% are also available in a TDS compliant ZIP archive. There
% the packages are already unpacked and the documentation files
% are generated. The files and directories obey the TDS standard.
% \begin{description}
% \item[\CTANinstall{install/macros/latex/contrib/oberdiek.tds.zip}]
% \end{description}
% \emph{TDS} refers to the standard ``A Directory Structure
% for \TeX\ Files'' (\CTAN{tds/tds.pdf}). Directories
% with \xfile{texmf} in their name are usually organized this way.
%
% \subsection{Bundle installation}
%
% \paragraph{Unpacking.} Unpack the \xfile{oberdiek.tds.zip} in the
% TDS tree (also known as \xfile{texmf} tree) of your choice.
% Example (linux):
% \begin{quote}
%   |unzip oberdiek.tds.zip -d ~/texmf|
% \end{quote}
%
% \paragraph{Script installation.}
% Check the directory \xfile{TDS:scripts/oberdiek/} for
% scripts that need further installation steps.
% Package \xpackage{attachfile2} comes with the Perl script
% \xfile{pdfatfi.pl} that should be installed in such a way
% that it can be called as \texttt{pdfatfi}.
% Example (linux):
% \begin{quote}
%   |chmod +x scripts/oberdiek/pdfatfi.pl|\\
%   |cp scripts/oberdiek/pdfatfi.pl /usr/local/bin/|
% \end{quote}
%
% \subsection{Package installation}
%
% \paragraph{Unpacking.} The \xfile{.dtx} file is a self-extracting
% \docstrip\ archive. The files are extracted by running the
% \xfile{.dtx} through \plainTeX:
% \begin{quote}
%   \verb|tex telprint.dtx|
% \end{quote}
%
% \paragraph{TDS.} Now the different files must be moved into
% the different directories in your installation TDS tree
% (also known as \xfile{texmf} tree):
% \begin{quote}
% \def\t{^^A
% \begin{tabular}{@{}>{\ttfamily}l@{ $\rightarrow$ }>{\ttfamily}l@{}}
%   telprint.sty & tex/generic/oberdiek/telprint.sty\\
%   telprint.pdf & doc/latex/oberdiek/telprint.pdf\\
%   test/telprint-test1.tex & doc/latex/oberdiek/test/telprint-test1.tex\\
%   telprint.dtx & source/latex/oberdiek/telprint.dtx\\
% \end{tabular}^^A
% }^^A
% \sbox0{\t}^^A
% \ifdim\wd0>\linewidth
%   \begingroup
%     \advance\linewidth by\leftmargin
%     \advance\linewidth by\rightmargin
%   \edef\x{\endgroup
%     \def\noexpand\lw{\the\linewidth}^^A
%   }\x
%   \def\lwbox{^^A
%     \leavevmode
%     \hbox to \linewidth{^^A
%       \kern-\leftmargin\relax
%       \hss
%       \usebox0
%       \hss
%       \kern-\rightmargin\relax
%     }^^A
%   }^^A
%   \ifdim\wd0>\lw
%     \sbox0{\small\t}^^A
%     \ifdim\wd0>\linewidth
%       \ifdim\wd0>\lw
%         \sbox0{\footnotesize\t}^^A
%         \ifdim\wd0>\linewidth
%           \ifdim\wd0>\lw
%             \sbox0{\scriptsize\t}^^A
%             \ifdim\wd0>\linewidth
%               \ifdim\wd0>\lw
%                 \sbox0{\tiny\t}^^A
%                 \ifdim\wd0>\linewidth
%                   \lwbox
%                 \else
%                   \usebox0
%                 \fi
%               \else
%                 \lwbox
%               \fi
%             \else
%               \usebox0
%             \fi
%           \else
%             \lwbox
%           \fi
%         \else
%           \usebox0
%         \fi
%       \else
%         \lwbox
%       \fi
%     \else
%       \usebox0
%     \fi
%   \else
%     \lwbox
%   \fi
% \else
%   \usebox0
% \fi
% \end{quote}
% If you have a \xfile{docstrip.cfg} that configures and enables \docstrip's
% TDS installing feature, then some files can already be in the right
% place, see the documentation of \docstrip.
%
% \subsection{Refresh file name databases}
%
% If your \TeX~distribution
% (\teTeX, \mikTeX, \dots) relies on file name databases, you must refresh
% these. For example, \teTeX\ users run \verb|texhash| or
% \verb|mktexlsr|.
%
% \subsection{Some details for the interested}
%
% \paragraph{Attached source.}
%
% The PDF documentation on CTAN also includes the
% \xfile{.dtx} source file. It can be extracted by
% AcrobatReader 6 or higher. Another option is \textsf{pdftk},
% e.g. unpack the file into the current directory:
% \begin{quote}
%   \verb|pdftk telprint.pdf unpack_files output .|
% \end{quote}
%
% \paragraph{Unpacking with \LaTeX.}
% The \xfile{.dtx} chooses its action depending on the format:
% \begin{description}
% \item[\plainTeX:] Run \docstrip\ and extract the files.
% \item[\LaTeX:] Generate the documentation.
% \end{description}
% If you insist on using \LaTeX\ for \docstrip\ (really,
% \docstrip\ does not need \LaTeX), then inform the autodetect routine
% about your intention:
% \begin{quote}
%   \verb|latex \let\install=y% \iffalse meta-comment
%
% File: telprint.dtx
% Version: 2016/05/16 v1.11
% Info: Format German phone numbers
%
% Copyright (C) 1996, 1997, 2004-2008 by
%    Heiko Oberdiek <heiko.oberdiek at googlemail.com>
%    2016
%    https://github.com/ho-tex/oberdiek/issues
%
% This work may be distributed and/or modified under the
% conditions of the LaTeX Project Public License, either
% version 1.3c of this license or (at your option) any later
% version. This version of this license is in
%    http://www.latex-project.org/lppl/lppl-1-3c.txt
% and the latest version of this license is in
%    http://www.latex-project.org/lppl.txt
% and version 1.3 or later is part of all distributions of
% LaTeX version 2005/12/01 or later.
%
% This work has the LPPL maintenance status "maintained".
%
% This Current Maintainer of this work is Heiko Oberdiek.
%
% The Base Interpreter refers to any `TeX-Format',
% because some files are installed in TDS:tex/generic//.
%
% This work consists of the main source file telprint.dtx
% and the derived files
%    telprint.sty, telprint.pdf, telprint.ins, telprint.drv,
%    telprint-test1.tex.
%
% Distribution:
%    CTAN:macros/latex/contrib/oberdiek/telprint.dtx
%    CTAN:macros/latex/contrib/oberdiek/telprint.pdf
%
% Unpacking:
%    (a) If telprint.ins is present:
%           tex telprint.ins
%    (b) Without telprint.ins:
%           tex telprint.dtx
%    (c) If you insist on using LaTeX
%           latex \let\install=y\input{telprint.dtx}
%        (quote the arguments according to the demands of your shell)
%
% Documentation:
%    (a) If telprint.drv is present:
%           latex telprint.drv
%    (b) Without telprint.drv:
%           latex telprint.dtx; ...
%    The class ltxdoc loads the configuration file ltxdoc.cfg
%    if available. Here you can specify further options, e.g.
%    use A4 as paper format:
%       \PassOptionsToClass{a4paper}{article}
%
%    Programm calls to get the documentation (example):
%       pdflatex telprint.dtx
%       makeindex -s gind.ist telprint.idx
%       pdflatex telprint.dtx
%       makeindex -s gind.ist telprint.idx
%       pdflatex telprint.dtx
%
% Installation:
%    TDS:tex/generic/oberdiek/telprint.sty
%    TDS:doc/latex/oberdiek/telprint.pdf
%    TDS:doc/latex/oberdiek/test/telprint-test1.tex
%    TDS:source/latex/oberdiek/telprint.dtx
%
%<*ignore>
\begingroup
  \catcode123=1 %
  \catcode125=2 %
  \def\x{LaTeX2e}%
\expandafter\endgroup
\ifcase 0\ifx\install y1\fi\expandafter
         \ifx\csname processbatchFile\endcsname\relax\else1\fi
         \ifx\fmtname\x\else 1\fi\relax
\else\csname fi\endcsname
%</ignore>
%<*install>
\input docstrip.tex
\Msg{************************************************************************}
\Msg{* Installation}
\Msg{* Package: telprint 2016/05/16 v1.11 Format German phone numbers (HO)}
\Msg{************************************************************************}

\keepsilent
\askforoverwritefalse

\let\MetaPrefix\relax
\preamble

This is a generated file.

Project: telprint
Version: 2016/05/16 v1.11

Copyright (C) 1996, 1997, 2004-2008 by
   Heiko Oberdiek <heiko.oberdiek at googlemail.com>

This work may be distributed and/or modified under the
conditions of the LaTeX Project Public License, either
version 1.3c of this license or (at your option) any later
version. This version of this license is in
   http://www.latex-project.org/lppl/lppl-1-3c.txt
and the latest version of this license is in
   http://www.latex-project.org/lppl.txt
and version 1.3 or later is part of all distributions of
LaTeX version 2005/12/01 or later.

This work has the LPPL maintenance status "maintained".

This Current Maintainer of this work is Heiko Oberdiek.

The Base Interpreter refers to any `TeX-Format',
because some files are installed in TDS:tex/generic//.

This work consists of the main source file telprint.dtx
and the derived files
   telprint.sty, telprint.pdf, telprint.ins, telprint.drv,
   telprint-test1.tex.

\endpreamble
\let\MetaPrefix\DoubleperCent

\generate{%
  \file{telprint.ins}{\from{telprint.dtx}{install}}%
  \file{telprint.drv}{\from{telprint.dtx}{driver}}%
  \usedir{tex/generic/oberdiek}%
  \file{telprint.sty}{\from{telprint.dtx}{package}}%
%  \usedir{doc/latex/oberdiek/test}%
%  \file{telprint-test1.tex}{\from{telprint.dtx}{test1}}%
  \nopreamble
  \nopostamble
%  \usedir{source/latex/oberdiek/catalogue}%
%  \file{telprint.xml}{\from{telprint.dtx}{catalogue}}%
}

\catcode32=13\relax% active space
\let =\space%
\Msg{************************************************************************}
\Msg{*}
\Msg{* To finish the installation you have to move the following}
\Msg{* file into a directory searched by TeX:}
\Msg{*}
\Msg{*     telprint.sty}
\Msg{*}
\Msg{* To produce the documentation run the file `telprint.drv'}
\Msg{* through LaTeX.}
\Msg{*}
\Msg{* Happy TeXing!}
\Msg{*}
\Msg{************************************************************************}

\endbatchfile
%</install>
%<*ignore>
\fi
%</ignore>
%<*driver>
\NeedsTeXFormat{LaTeX2e}
\ProvidesFile{telprint.drv}%
  [2016/05/16 v1.11 Format German phone numbers (HO)]%
\documentclass{ltxdoc}
\usepackage{holtxdoc}[2011/11/22]
\usepackage[ngerman,english]{babel}
\begin{document}
  \DocInput{telprint.dtx}%
\end{document}
%</driver>
% \fi
%
%
% \CharacterTable
%  {Upper-case    \A\B\C\D\E\F\G\H\I\J\K\L\M\N\O\P\Q\R\S\T\U\V\W\X\Y\Z
%   Lower-case    \a\b\c\d\e\f\g\h\i\j\k\l\m\n\o\p\q\r\s\t\u\v\w\x\y\z
%   Digits        \0\1\2\3\4\5\6\7\8\9
%   Exclamation   \!     Double quote  \"     Hash (number) \#
%   Dollar        \$     Percent       \%     Ampersand     \&
%   Acute accent  \'     Left paren    \(     Right paren   \)
%   Asterisk      \*     Plus          \+     Comma         \,
%   Minus         \-     Point         \.     Solidus       \/
%   Colon         \:     Semicolon     \;     Less than     \<
%   Equals        \=     Greater than  \>     Question mark \?
%   Commercial at \@     Left bracket  \[     Backslash     \\
%   Right bracket \]     Circumflex    \^     Underscore    \_
%   Grave accent  \`     Left brace    \{     Vertical bar  \|
%   Right brace   \}     Tilde         \~}
%
% \GetFileInfo{telprint.drv}
%
% \title{The \xpackage{telprint} package}
% \date{2016/05/16 v1.11}
% \author{Heiko Oberdiek\thanks
% {Please report any issues at https://github.com/ho-tex/oberdiek/issues}\\
% \xemail{heiko.oberdiek at googlemail.com}}
%
% \maketitle
%
% \begin{abstract}
% Package \xpackage{telprint} provides \cs{telprint} for formatting
% German phone numbers.
% \end{abstract}
%
% \tableofcontents
%
% \section{Documentation}
%
% \subsection{Introduction}
%
%            This is a very old package that I have written
%            to format phone numbers. It follows German
%            conventions and the documentation is mainly in German.
%
% \subsection{Short overview in English}
%
% \LaTeX:
% \begin{quote}
% |\usepackage{telprint}|\\
% |\telprint{123/456-789}|\\
% \end{quote}
% \plainTeX:
% \begin{quote}
%   |\input telprint.sty|\\
%   |\telprint{123/456-789}|
% \end{quote}
%
% \DescribeMacro\telprint
% |\telprint{...}| formats the explicitly given number.
%     Digits, spaces and some special characters
%     ('+', '/', '-', '(', ')', '\textasciitilde', ' ') are supported.
%     Numbers are divided into groups of two digits from the right.
% Examples:
% \begin{quote}
%     |\telprint{0761/12345}     ==> 07\,61/1\,23\,45|\\
%     |\telprint{01234/567-89}   ==> 0\,12\,34/5\,67\leavevmode\hbox{-}89|\\
%     |\telprint{+49 (6221) 297} ==> +49~(62\,21)~2\,97|
% \end{quote}
%
% \subsubsection{Configuration}
%
% The output of the symbols can be configured by
% \cs{telhyphen}, \cs{telslash}, \cs{telleftparen}, \cs{telrightparen},
% \cs{telplus}, \cs{teltilde}.
% Example:
% \begin{quote}
%   |\telslash{\,/\,}\\|
%   |\telprint{12/34} ==> 12\,/\,34|
% \end{quote}
%
% \DescribeMacro\telspace
% \cs{telspace} configures the space between digit groups.
%
% \DescribeMacro\telnumber
% \cs{telnumber} only formats a number in digit groups; special
%    characters are not recognized.
%
% \subsection{Documentation in German}
%
% \begin{otherlanguage*}{ngerman}
% \hyphenation{To-ken-ma-kros}
% \begin{itemize}
% \item \DescribeMacro\telprint |telprint#1|\\
%   Der eigentliche Anwenderbefehl zur formatierten Ausgabe von
%   Telefonnummern. Diese d\"urfen dabei nur als Zahlen angegeben
%   werden(, da sie tokenweise analysiert werden).
%   Als Trenn- oder Sonderzeichen werden unterst\"utzt:
%   '+', '/', '-', '(', ')', '\textasciitilde', ' '
%   Einfache Leerzeichen werden erkannt und durch Tilden ersetzt, um
%   Trennungen in der Telefonnummer zu verhindern. (Man beachte aus
%   gleichem Grunde die \cs{hbox} bei '-'.)
%   Beispiele:
%   \begin{quote}
%     |\telprint{0761/12345}     ==> 07\,61/1\,23\,45|\\
%     |\telprint{01234/567-89}   ==> 0\,12\,34/5\,67\leavevmode\hbox{-}89|\\
%     |\telprint{+49 (6221) 297} ==> +49~(62\,21)~2\,97|
%   \end{quote}
% \end{itemize}
% Der Rest enth\"alt eher Technisches:
% \begin{itemize}
% \item \DescribeMacro\telspace |\telspace#1|\\
%   Mit diesem Befehl wird der Abstand zwischen den Zifferngruppen
%   angegeben (Default: |\,|).
%   (Durch |\telspace{}| kann dieser zusaetzliche Abstand abgestellt
%   werden.)
% \item \DescribeMacro\telhyphen |\telhyphen#1|\\
%   Dieser Befehl gibt die Art des Bindestriches, wie er ausgegeben
%   werden soll. In der Eingabe darf jedoch nur der einfache
%   Bindestrich stehen:
%   |\telprint{123-45}|, jedoch NIE |\telprint{123--45}|!
%   Kopka-Bindestrich-Fans geben an:
%   |\telhyphen{\leavevmode\hbox{--}}|
% \item
%   \DescribeMacro{\telslash}
%   \DescribeMacro{\telleftparen}
%   \DescribeMacro{\telrightparen}
%   \DescribeMacro{\telplus}
%   \DescribeMacro{\teltilde}
%   |\telslash#1|, |\telleftparen#1|, |\telrightparen#1|, |\telplus#1|,
%   |\teltilde|\\
%   Diese Befehle konfigurieren die Zeichen '/', '(', ')', '+'
%   und '\textasciitilde'. Sie funktionieren analog zu \cs{telhyphen}.
% \item \DescribeMacro\telnumber |\telnumber#1|\\
%   Richtung interner Befehl: Er dient dazu, eine Zifferngruppe
%   in Zweiergruppen auszugeben.
%   Die einzelnen Zahlen werden im Tokenregister \cs{TELtoks}
%   gespeichert. Abwechselnd werden dabei zwischen zwei Token
%   (Zahlen) \cs{TELx} bzw. \cs{TELy} eingefuegt, abh\"angig von dem
%   wechselnden Wert von \cs{TELswitch}. Zum Schluss kann dann einfach
%   festgestellt werden ob die Nummer nun eine geradzahlige oder
%   ungeradzahlige Zahl von Ziffern aufwies. Dem entsprechend wird
%   \cs{TELx} mit dem Zusatzabstand belegt und \cs{TELy} leer definiert
%   oder umgekehrt. )
% \item |\TEL...| interne Befehle, Technisches:\\
%   \cs{TELsplit} dient zur Aufteilung einer zusammengesetzten
%   Telefonnummer (Vorwahl, Hauptnummer, Nebenstelle). In dieser
%   Implementation werden als Trennzeichen nur '/' und '-' erkannt.
%   Die einzelnen Bestandteile wie Vorwahl werden dann dem Befehl
%   \cs{telnumber} zur Formatierung uebergeben.
% \item Die Erkennung von einfachen Leerzeichen ist um einiges
%   schwieriger: Die Tokentrennung ueber Parameter |#1#2| funktioniert
%   nicht f\"ur einfache Leerzeichen, da TeX sie \emph{niemals} als
%   eigenst\"andige Argumente behandelt! (The TeXbook, Chapter 20,
%   p. 201)
%
%   (Anmerkung am Rande: Deshalb funktionieren die entsprechenden
%   Tokenmakros auf S. 149 des Buches "`Einf\"uhrung in TeX"' von
%   N. Schwarz (3. Aufl.) nicht, wenn im Tokenregister als erstes
%   ein einfaches Leerzeichen steht!)
% \end{itemize}
% \end{otherlanguage*}
%
% \StopEventually{
% }
%
% \section{Implementation}
%
%    \begin{macrocode}
%<*package>
%    \end{macrocode}
%
% \subsection{Reload check and package identification}
%    Reload check, especially if the package is not used with \LaTeX.
%    \begin{macrocode}
\begingroup\catcode61\catcode48\catcode32=10\relax%
  \catcode13=5 % ^^M
  \endlinechar=13 %
  \catcode35=6 % #
  \catcode39=12 % '
  \catcode44=12 % ,
  \catcode45=12 % -
  \catcode46=12 % .
  \catcode58=12 % :
  \catcode64=11 % @
  \catcode123=1 % {
  \catcode125=2 % }
  \expandafter\let\expandafter\x\csname ver@telprint.sty\endcsname
  \ifx\x\relax % plain-TeX, first loading
  \else
    \def\empty{}%
    \ifx\x\empty % LaTeX, first loading,
      % variable is initialized, but \ProvidesPackage not yet seen
    \else
      \expandafter\ifx\csname PackageInfo\endcsname\relax
        \def\x#1#2{%
          \immediate\write-1{Package #1 Info: #2.}%
        }%
      \else
        \def\x#1#2{\PackageInfo{#1}{#2, stopped}}%
      \fi
      \x{telprint}{The package is already loaded}%
      \aftergroup\endinput
    \fi
  \fi
\endgroup%
%    \end{macrocode}
%    Package identification:
%    \begin{macrocode}
\begingroup\catcode61\catcode48\catcode32=10\relax%
  \catcode13=5 % ^^M
  \endlinechar=13 %
  \catcode35=6 % #
  \catcode39=12 % '
  \catcode40=12 % (
  \catcode41=12 % )
  \catcode44=12 % ,
  \catcode45=12 % -
  \catcode46=12 % .
  \catcode47=12 % /
  \catcode58=12 % :
  \catcode64=11 % @
  \catcode91=12 % [
  \catcode93=12 % ]
  \catcode123=1 % {
  \catcode125=2 % }
  \expandafter\ifx\csname ProvidesPackage\endcsname\relax
    \def\x#1#2#3[#4]{\endgroup
      \immediate\write-1{Package: #3 #4}%
      \xdef#1{#4}%
    }%
  \else
    \def\x#1#2[#3]{\endgroup
      #2[{#3}]%
      \ifx#1\@undefined
        \xdef#1{#3}%
      \fi
      \ifx#1\relax
        \xdef#1{#3}%
      \fi
    }%
  \fi
\expandafter\x\csname ver@telprint.sty\endcsname
\ProvidesPackage{telprint}%
  [2016/05/16 v1.11 Format German phone numbers (HO)]%
%    \end{macrocode}
%
% \subsection{Catcodes}
%
%    \begin{macrocode}
\begingroup\catcode61\catcode48\catcode32=10\relax%
  \catcode13=5 % ^^M
  \endlinechar=13 %
  \catcode123=1 % {
  \catcode125=2 % }
  \catcode64=11 % @
  \def\x{\endgroup
    \expandafter\edef\csname TELAtEnd\endcsname{%
      \endlinechar=\the\endlinechar\relax
      \catcode13=\the\catcode13\relax
      \catcode32=\the\catcode32\relax
      \catcode35=\the\catcode35\relax
      \catcode61=\the\catcode61\relax
      \catcode64=\the\catcode64\relax
      \catcode123=\the\catcode123\relax
      \catcode125=\the\catcode125\relax
    }%
  }%
\x\catcode61\catcode48\catcode32=10\relax%
\catcode13=5 % ^^M
\endlinechar=13 %
\catcode35=6 % #
\catcode64=11 % @
\catcode123=1 % {
\catcode125=2 % }
\def\TMP@EnsureCode#1#2{%
  \edef\TELAtEnd{%
    \TELAtEnd
    \catcode#1=\the\catcode#1\relax
  }%
  \catcode#1=#2\relax
}
\TMP@EnsureCode{33}{12}% !
\TMP@EnsureCode{36}{3}% $
\TMP@EnsureCode{40}{12}% (
\TMP@EnsureCode{41}{12}% )
\TMP@EnsureCode{42}{12}% *
\TMP@EnsureCode{43}{12}% +
\TMP@EnsureCode{44}{12}% ,
\TMP@EnsureCode{45}{12}% -
\TMP@EnsureCode{46}{12}% .
\TMP@EnsureCode{47}{12}% /
\TMP@EnsureCode{91}{12}% [
\TMP@EnsureCode{93}{12}% ]
\TMP@EnsureCode{126}{13}% ~ (active)
\edef\TELAtEnd{\TELAtEnd\noexpand\endinput}
%    \end{macrocode}
%
% \subsection{Package macros}
%    \begin{macrocode}
\ifx\DeclareRobustCommand\UnDeFiNeD
  \def\DeclareRobustCommand*#1[1]{\def#1##1}%
  \def\TELreset{\let\DeclareRobustCommand=\UnDeFiNeD}%
  \input infwarerr.sty\relax
  \@PackageInfo{telprint}{%
    Macros are not robust!%
  }%
\else
  \let\TELreset=\relax
\fi
%    \end{macrocode}
%    \begin{macro}{\telspace}
%    \begin{macrocode}
\DeclareRobustCommand*{\telspace}[1]{\def\TELspace{#1}}
\telspace{{}$\,${}}
%    \end{macrocode}
%    \end{macro}
%    \begin{macro}{\telhyphen}
%    \begin{macrocode}
\DeclareRobustCommand*{\telhyphen}[1]{\def\TELhyphen{#1}}
\telhyphen{\leavevmode\hbox{-}}% \hbox zur Verhinderung der Trennung
%    \end{macrocode}
%    \end{macro}
%    \begin{macro}{\telslash}
%    \begin{macrocode}
\DeclareRobustCommand*{\telslash}[1]{\def\TELslash{#1}}
\telslash{/}%
%    \end{macrocode}
%    \end{macro}
%    \begin{macro}{\telleftparen}
%    \begin{macrocode}
\DeclareRobustCommand*{\telleftparen}[1]{\def\TELleftparen{#1}}
\telleftparen{(}%
%    \end{macrocode}
%    \end{macro}
%    \begin{macro}{\telrightparen}
%    \begin{macrocode}
\DeclareRobustCommand*{\telrightparen}[1]{\def\TELrightparen{#1}}
\telrightparen{)}%
%    \end{macrocode}
%    \end{macro}
%    \begin{macro}{\telplus}
%    \begin{macrocode}
\DeclareRobustCommand*{\telplus}[1]{\def\TELplus{#1}}
\telplus{+}%
%    \end{macrocode}
%    \end{macro}
%    \begin{macro}{\teltilde}
%    \begin{macrocode}
\DeclareRobustCommand*{\teltilde}[1]{\def\TELtilde{#1}}
\teltilde{~}%
%    \end{macrocode}
%    \end{macro}
%    \begin{macro}{\TELtoks}
%    \begin{macrocode}
\newtoks\TELtoks
%    \end{macrocode}
%    \end{macro}
%    \begin{macro}{\TELnumber}
%    \begin{macrocode}
\def\TELnumber#1#2\TELnumberEND{%
  \begingroup
  \def\0{#2}%
  \expandafter\endgroup
  \ifx\0\empty
    \TELtoks=\expandafter{\the\TELtoks#1}%
    \ifnum\TELswitch=0 %
      \def\TELx{\TELspace}\def\TELy{}%
    \else
      \def\TELx{}\def\TELy{\TELspace}%
    \fi
    \the\TELtoks
  \else
    \ifnum\TELswitch=0 %
      \TELtoks=\expandafter{\the\TELtoks#1\TELx}%
      \def\TELswitch{1}%
    \else
      \TELtoks=\expandafter{\the\TELtoks#1\TELy}%
      \def\TELswitch{0}%
    \fi
    \TELnumber#2\TELnumberEND
  \fi
}
%    \end{macrocode}
%    \end{macro}
%    \begin{macro}{\telnumber}
%    \begin{macrocode}
\DeclareRobustCommand*{\telnumber}[1]{%
  \TELtoks={}%
  \def\TELswitch{0}%
  \TELnumber#1{}\TELnumberEND
}
%    \end{macrocode}
%    \end{macro}
%    \begin{macro}{\TELsplit}
%    \begin{macrocode}
\def\TELsplit{\futurelet\TELfuture\TELdosplit}
%    \end{macrocode}
%    \end{macro}
%    \begin{macro}{\TELdosplit}
%    \begin{macrocode}
\def\TELdosplit#1#2\TELsplitEND
{%
  \def\TELsp{ }%
  \expandafter\ifx\TELsp\TELfuture
    \let\TELfuture=\relax
    \expandafter\telnumber\expandafter{\the\TELtoks}~%
    \telprint{#1#2}% Das Leerzeichen kann nicht #1 sein!
  \else
    \def\TELfirst{#1}%
    \ifx\TELfirst\empty
      \expandafter\telnumber\expandafter{\the\TELtoks}%
      \TELtoks={}%
    \else\if-\TELfirst
      \expandafter\telnumber\expandafter{\the\TELtoks}\TELhyphen
      \telprint{#2}%
    \else\if/\TELfirst
      \expandafter\telnumber\expandafter{\the\TELtoks}\TELslash
      \telprint{#2}%
    \else\if(\TELfirst
      \expandafter\telnumber\expandafter{\the\TELtoks}\TELleftparen
      \telprint{#2}%
    \else\if)\TELfirst
      \expandafter\telnumber\expandafter{\the\TELtoks}\TELrightparen
      \telprint{#2}%
    \else\if+\TELfirst
      \expandafter\telnumber\expandafter{\the\TELtoks}\TELplus
      \telprint{#2}%
    \else\def\TELtemp{~}\ifx\TELtemp\TELfirst
      \expandafter\telnumber\expandafter{\the\TELtoks}\TELtilde
      \telprint{#2}%
    \else
      \TELtoks=\expandafter{\the\TELtoks#1}%
      \TELsplit#2{}\TELsplitEND
    \fi\fi\fi\fi\fi\fi\fi
  \fi
}
%    \end{macrocode}
%    \end{macro}
%    \begin{macro}{\telprint}
%    \begin{macrocode}
\DeclareRobustCommand*{\telprint}[1]{%
  \TELtoks={}%
  \TELsplit#1{}\TELsplitEND
}
%    \end{macrocode}
%    \end{macro}
%    \begin{macrocode}
\TELreset\let\TELreset=\UnDeFiNeD
%    \end{macrocode}
%
%    \begin{macrocode}
\TELAtEnd%
%</package>
%    \end{macrocode}
%
% \section{Test}
%
% \subsection{Catcode checks for loading}
%
%    \begin{macrocode}
%<*test1>
%    \end{macrocode}
%    \begin{macrocode}
\catcode`\{=1 %
\catcode`\}=2 %
\catcode`\#=6 %
\catcode`\@=11 %
\expandafter\ifx\csname count@\endcsname\relax
  \countdef\count@=255 %
\fi
\expandafter\ifx\csname @gobble\endcsname\relax
  \long\def\@gobble#1{}%
\fi
\expandafter\ifx\csname @firstofone\endcsname\relax
  \long\def\@firstofone#1{#1}%
\fi
\expandafter\ifx\csname loop\endcsname\relax
  \expandafter\@firstofone
\else
  \expandafter\@gobble
\fi
{%
  \def\loop#1\repeat{%
    \def\body{#1}%
    \iterate
  }%
  \def\iterate{%
    \body
      \let\next\iterate
    \else
      \let\next\relax
    \fi
    \next
  }%
  \let\repeat=\fi
}%
\def\RestoreCatcodes{}
\count@=0 %
\loop
  \edef\RestoreCatcodes{%
    \RestoreCatcodes
    \catcode\the\count@=\the\catcode\count@\relax
  }%
\ifnum\count@<255 %
  \advance\count@ 1 %
\repeat

\def\RangeCatcodeInvalid#1#2{%
  \count@=#1\relax
  \loop
    \catcode\count@=15 %
  \ifnum\count@<#2\relax
    \advance\count@ 1 %
  \repeat
}
\def\RangeCatcodeCheck#1#2#3{%
  \count@=#1\relax
  \loop
    \ifnum#3=\catcode\count@
    \else
      \errmessage{%
        Character \the\count@\space
        with wrong catcode \the\catcode\count@\space
        instead of \number#3%
      }%
    \fi
  \ifnum\count@<#2\relax
    \advance\count@ 1 %
  \repeat
}
\def\space{ }
\expandafter\ifx\csname LoadCommand\endcsname\relax
  \def\LoadCommand{\input telprint.sty\relax}%
\fi
\def\Test{%
  \RangeCatcodeInvalid{0}{47}%
  \RangeCatcodeInvalid{58}{64}%
  \RangeCatcodeInvalid{91}{96}%
  \RangeCatcodeInvalid{123}{255}%
  \catcode`\@=12 %
  \catcode`\\=0 %
  \catcode`\%=14 %
  \LoadCommand
  \RangeCatcodeCheck{0}{36}{15}%
  \RangeCatcodeCheck{37}{37}{14}%
  \RangeCatcodeCheck{38}{47}{15}%
  \RangeCatcodeCheck{48}{57}{12}%
  \RangeCatcodeCheck{58}{63}{15}%
  \RangeCatcodeCheck{64}{64}{12}%
  \RangeCatcodeCheck{65}{90}{11}%
  \RangeCatcodeCheck{91}{91}{15}%
  \RangeCatcodeCheck{92}{92}{0}%
  \RangeCatcodeCheck{93}{96}{15}%
  \RangeCatcodeCheck{97}{122}{11}%
  \RangeCatcodeCheck{123}{255}{15}%
  \RestoreCatcodes
}
\Test
\csname @@end\endcsname
\end
%    \end{macrocode}
%    \begin{macrocode}
%</test1>
%    \end{macrocode}
%
% \section{Installation}
%
% \subsection{Download}
%
% \paragraph{Package.} This package is available on
% CTAN\footnote{\url{https://ctan.org/pkg/telprint}}:
% \begin{description}
% \item[\CTAN{macros/latex/contrib/oberdiek/telprint.dtx}] The source file.
% \item[\CTAN{macros/latex/contrib/oberdiek/telprint.pdf}] Documentation.
% \end{description}
%
%
% \paragraph{Bundle.} All the packages of the bundle `oberdiek'
% are also available in a TDS compliant ZIP archive. There
% the packages are already unpacked and the documentation files
% are generated. The files and directories obey the TDS standard.
% \begin{description}
% \item[\CTANinstall{install/macros/latex/contrib/oberdiek.tds.zip}]
% \end{description}
% \emph{TDS} refers to the standard ``A Directory Structure
% for \TeX\ Files'' (\CTAN{tds/tds.pdf}). Directories
% with \xfile{texmf} in their name are usually organized this way.
%
% \subsection{Bundle installation}
%
% \paragraph{Unpacking.} Unpack the \xfile{oberdiek.tds.zip} in the
% TDS tree (also known as \xfile{texmf} tree) of your choice.
% Example (linux):
% \begin{quote}
%   |unzip oberdiek.tds.zip -d ~/texmf|
% \end{quote}
%
% \paragraph{Script installation.}
% Check the directory \xfile{TDS:scripts/oberdiek/} for
% scripts that need further installation steps.
% Package \xpackage{attachfile2} comes with the Perl script
% \xfile{pdfatfi.pl} that should be installed in such a way
% that it can be called as \texttt{pdfatfi}.
% Example (linux):
% \begin{quote}
%   |chmod +x scripts/oberdiek/pdfatfi.pl|\\
%   |cp scripts/oberdiek/pdfatfi.pl /usr/local/bin/|
% \end{quote}
%
% \subsection{Package installation}
%
% \paragraph{Unpacking.} The \xfile{.dtx} file is a self-extracting
% \docstrip\ archive. The files are extracted by running the
% \xfile{.dtx} through \plainTeX:
% \begin{quote}
%   \verb|tex telprint.dtx|
% \end{quote}
%
% \paragraph{TDS.} Now the different files must be moved into
% the different directories in your installation TDS tree
% (also known as \xfile{texmf} tree):
% \begin{quote}
% \def\t{^^A
% \begin{tabular}{@{}>{\ttfamily}l@{ $\rightarrow$ }>{\ttfamily}l@{}}
%   telprint.sty & tex/generic/oberdiek/telprint.sty\\
%   telprint.pdf & doc/latex/oberdiek/telprint.pdf\\
%   test/telprint-test1.tex & doc/latex/oberdiek/test/telprint-test1.tex\\
%   telprint.dtx & source/latex/oberdiek/telprint.dtx\\
% \end{tabular}^^A
% }^^A
% \sbox0{\t}^^A
% \ifdim\wd0>\linewidth
%   \begingroup
%     \advance\linewidth by\leftmargin
%     \advance\linewidth by\rightmargin
%   \edef\x{\endgroup
%     \def\noexpand\lw{\the\linewidth}^^A
%   }\x
%   \def\lwbox{^^A
%     \leavevmode
%     \hbox to \linewidth{^^A
%       \kern-\leftmargin\relax
%       \hss
%       \usebox0
%       \hss
%       \kern-\rightmargin\relax
%     }^^A
%   }^^A
%   \ifdim\wd0>\lw
%     \sbox0{\small\t}^^A
%     \ifdim\wd0>\linewidth
%       \ifdim\wd0>\lw
%         \sbox0{\footnotesize\t}^^A
%         \ifdim\wd0>\linewidth
%           \ifdim\wd0>\lw
%             \sbox0{\scriptsize\t}^^A
%             \ifdim\wd0>\linewidth
%               \ifdim\wd0>\lw
%                 \sbox0{\tiny\t}^^A
%                 \ifdim\wd0>\linewidth
%                   \lwbox
%                 \else
%                   \usebox0
%                 \fi
%               \else
%                 \lwbox
%               \fi
%             \else
%               \usebox0
%             \fi
%           \else
%             \lwbox
%           \fi
%         \else
%           \usebox0
%         \fi
%       \else
%         \lwbox
%       \fi
%     \else
%       \usebox0
%     \fi
%   \else
%     \lwbox
%   \fi
% \else
%   \usebox0
% \fi
% \end{quote}
% If you have a \xfile{docstrip.cfg} that configures and enables \docstrip's
% TDS installing feature, then some files can already be in the right
% place, see the documentation of \docstrip.
%
% \subsection{Refresh file name databases}
%
% If your \TeX~distribution
% (\teTeX, \mikTeX, \dots) relies on file name databases, you must refresh
% these. For example, \teTeX\ users run \verb|texhash| or
% \verb|mktexlsr|.
%
% \subsection{Some details for the interested}
%
% \paragraph{Attached source.}
%
% The PDF documentation on CTAN also includes the
% \xfile{.dtx} source file. It can be extracted by
% AcrobatReader 6 or higher. Another option is \textsf{pdftk},
% e.g. unpack the file into the current directory:
% \begin{quote}
%   \verb|pdftk telprint.pdf unpack_files output .|
% \end{quote}
%
% \paragraph{Unpacking with \LaTeX.}
% The \xfile{.dtx} chooses its action depending on the format:
% \begin{description}
% \item[\plainTeX:] Run \docstrip\ and extract the files.
% \item[\LaTeX:] Generate the documentation.
% \end{description}
% If you insist on using \LaTeX\ for \docstrip\ (really,
% \docstrip\ does not need \LaTeX), then inform the autodetect routine
% about your intention:
% \begin{quote}
%   \verb|latex \let\install=y\input{telprint.dtx}|
% \end{quote}
% Do not forget to quote the argument according to the demands
% of your shell.
%
% \paragraph{Generating the documentation.}
% You can use both the \xfile{.dtx} or the \xfile{.drv} to generate
% the documentation. The process can be configured by the
% configuration file \xfile{ltxdoc.cfg}. For instance, put this
% line into this file, if you want to have A4 as paper format:
% \begin{quote}
%   \verb|\PassOptionsToClass{a4paper}{article}|
% \end{quote}
% An example follows how to generate the
% documentation with pdf\LaTeX:
% \begin{quote}
%\begin{verbatim}
%pdflatex telprint.dtx
%makeindex -s gind.ist telprint.idx
%pdflatex telprint.dtx
%makeindex -s gind.ist telprint.idx
%pdflatex telprint.dtx
%\end{verbatim}
% \end{quote}
%
% \begin{History}
%   \begin{Version}{1996/11/28 v1.0}
%   \item
%     Erste lauff\"ahige Version.
%   \item
%     Nur '-' und '/' als zul\"assige Sonderzeichen.
%   \end{Version}
%   \begin{Version}{1997/09/16 v1.1}
%   \item
%     Dokumentation und Kommentare (Posting in de.comp.text.tex).
%   \item
%     Erweiterung um Sonderzeichen '(', ')', '+', '\textasciitilde' und ' '.
%   \item
%     Trennungsverhinderung am 'hyphen'.
%   \end{Version}
%   \begin{Version}{1997/10/16 v1.2}
%   \item
%     Schutz vor wiederholtem Einlesen.
%   \item
%     Unter \LaTeXe\ Nutzung des \cs{DeclareRobustCommand}-Features.
%   \end{Version}
%   \begin{Version}{1997/12/09 v1.3}
%   \item
%     Tempor\"are Variable eingespart.
%   \item
%     Posted in newsgroup \xnewsgroup{de.comp.text.tex}:\\
%     \URL{``\link{Re: Generisches Markup f\"ur Telefonnummern?}''}^^A
%     {http://groups.google.com/group/de.comp.text.tex/msg/86b3a86140007309}
%   \end{Version}
%   \begin{Version}{2004/11/02 v1.4}
%   \item
%     Fehler in der Dokumentation korrigiert.
%   \end{Version}
%   \begin{Version}{2005/09/30 v1.5}
%   \item
%     Konfigurierbare Symbole: '/', '(', ')', '+' und '\textasciitilde'.
%   \end{Version}
%   \begin{Version}{2006/02/12 v1.6}
%   \item
%     LPPL 1.3.
%   \item
%     Kurze \"Ubersicht in Englisch.
%   \item
%     CTAN.
%   \end{Version}
%   \begin{Version}{2006/08/26 v1.7}
%   \item
%     New DTX framework.
%   \end{Version}
%   \begin{Version}{2007/04/11 v1.8}
%   \item
%     Line ends sanitized.
%   \end{Version}
%   \begin{Version}{2007/09/09 v1.9}
%   \item
%     Catcode section added.
%   \item
%     Missing docstrip tag added.
%   \end{Version}
%   \begin{Version}{2008/08/11 v1.10}
%   \item
%     Code is not changed.
%   \item
%     URLs updated.
%   \end{Version}
%   \begin{Version}{2016/05/16 v1.11}
%   \item
%     Documentation updates.
%   \end{Version}
% \end{History}
%
% \PrintIndex
%
% \Finale
\endinput
|
% \end{quote}
% Do not forget to quote the argument according to the demands
% of your shell.
%
% \paragraph{Generating the documentation.}
% You can use both the \xfile{.dtx} or the \xfile{.drv} to generate
% the documentation. The process can be configured by the
% configuration file \xfile{ltxdoc.cfg}. For instance, put this
% line into this file, if you want to have A4 as paper format:
% \begin{quote}
%   \verb|\PassOptionsToClass{a4paper}{article}|
% \end{quote}
% An example follows how to generate the
% documentation with pdf\LaTeX:
% \begin{quote}
%\begin{verbatim}
%pdflatex telprint.dtx
%makeindex -s gind.ist telprint.idx
%pdflatex telprint.dtx
%makeindex -s gind.ist telprint.idx
%pdflatex telprint.dtx
%\end{verbatim}
% \end{quote}
%
% \begin{History}
%   \begin{Version}{1996/11/28 v1.0}
%   \item
%     Erste lauff\"ahige Version.
%   \item
%     Nur '-' und '/' als zul\"assige Sonderzeichen.
%   \end{Version}
%   \begin{Version}{1997/09/16 v1.1}
%   \item
%     Dokumentation und Kommentare (Posting in de.comp.text.tex).
%   \item
%     Erweiterung um Sonderzeichen '(', ')', '+', '\textasciitilde' und ' '.
%   \item
%     Trennungsverhinderung am 'hyphen'.
%   \end{Version}
%   \begin{Version}{1997/10/16 v1.2}
%   \item
%     Schutz vor wiederholtem Einlesen.
%   \item
%     Unter \LaTeXe\ Nutzung des \cs{DeclareRobustCommand}-Features.
%   \end{Version}
%   \begin{Version}{1997/12/09 v1.3}
%   \item
%     Tempor\"are Variable eingespart.
%   \item
%     Posted in newsgroup \xnewsgroup{de.comp.text.tex}:\\
%     \URL{``\link{Re: Generisches Markup f\"ur Telefonnummern?}''}^^A
%     {http://groups.google.com/group/de.comp.text.tex/msg/86b3a86140007309}
%   \end{Version}
%   \begin{Version}{2004/11/02 v1.4}
%   \item
%     Fehler in der Dokumentation korrigiert.
%   \end{Version}
%   \begin{Version}{2005/09/30 v1.5}
%   \item
%     Konfigurierbare Symbole: '/', '(', ')', '+' und '\textasciitilde'.
%   \end{Version}
%   \begin{Version}{2006/02/12 v1.6}
%   \item
%     LPPL 1.3.
%   \item
%     Kurze \"Ubersicht in Englisch.
%   \item
%     CTAN.
%   \end{Version}
%   \begin{Version}{2006/08/26 v1.7}
%   \item
%     New DTX framework.
%   \end{Version}
%   \begin{Version}{2007/04/11 v1.8}
%   \item
%     Line ends sanitized.
%   \end{Version}
%   \begin{Version}{2007/09/09 v1.9}
%   \item
%     Catcode section added.
%   \item
%     Missing docstrip tag added.
%   \end{Version}
%   \begin{Version}{2008/08/11 v1.10}
%   \item
%     Code is not changed.
%   \item
%     URLs updated.
%   \end{Version}
%   \begin{Version}{2016/05/16 v1.11}
%   \item
%     Documentation updates.
%   \end{Version}
% \end{History}
%
% \PrintIndex
%
% \Finale
\endinput
|
% \end{quote}
% Do not forget to quote the argument according to the demands
% of your shell.
%
% \paragraph{Generating the documentation.}
% You can use both the \xfile{.dtx} or the \xfile{.drv} to generate
% the documentation. The process can be configured by the
% configuration file \xfile{ltxdoc.cfg}. For instance, put this
% line into this file, if you want to have A4 as paper format:
% \begin{quote}
%   \verb|\PassOptionsToClass{a4paper}{article}|
% \end{quote}
% An example follows how to generate the
% documentation with pdf\LaTeX:
% \begin{quote}
%\begin{verbatim}
%pdflatex telprint.dtx
%makeindex -s gind.ist telprint.idx
%pdflatex telprint.dtx
%makeindex -s gind.ist telprint.idx
%pdflatex telprint.dtx
%\end{verbatim}
% \end{quote}
%
% \begin{History}
%   \begin{Version}{1996/11/28 v1.0}
%   \item
%     Erste lauff\"ahige Version.
%   \item
%     Nur '-' und '/' als zul\"assige Sonderzeichen.
%   \end{Version}
%   \begin{Version}{1997/09/16 v1.1}
%   \item
%     Dokumentation und Kommentare (Posting in de.comp.text.tex).
%   \item
%     Erweiterung um Sonderzeichen '(', ')', '+', '\textasciitilde' und ' '.
%   \item
%     Trennungsverhinderung am 'hyphen'.
%   \end{Version}
%   \begin{Version}{1997/10/16 v1.2}
%   \item
%     Schutz vor wiederholtem Einlesen.
%   \item
%     Unter \LaTeXe\ Nutzung des \cs{DeclareRobustCommand}-Features.
%   \end{Version}
%   \begin{Version}{1997/12/09 v1.3}
%   \item
%     Tempor\"are Variable eingespart.
%   \item
%     Posted in newsgroup \xnewsgroup{de.comp.text.tex}:\\
%     \URL{``\link{Re: Generisches Markup f\"ur Telefonnummern?}''}^^A
%     {http://groups.google.com/group/de.comp.text.tex/msg/86b3a86140007309}
%   \end{Version}
%   \begin{Version}{2004/11/02 v1.4}
%   \item
%     Fehler in der Dokumentation korrigiert.
%   \end{Version}
%   \begin{Version}{2005/09/30 v1.5}
%   \item
%     Konfigurierbare Symbole: '/', '(', ')', '+' und '\textasciitilde'.
%   \end{Version}
%   \begin{Version}{2006/02/12 v1.6}
%   \item
%     LPPL 1.3.
%   \item
%     Kurze \"Ubersicht in Englisch.
%   \item
%     CTAN.
%   \end{Version}
%   \begin{Version}{2006/08/26 v1.7}
%   \item
%     New DTX framework.
%   \end{Version}
%   \begin{Version}{2007/04/11 v1.8}
%   \item
%     Line ends sanitized.
%   \end{Version}
%   \begin{Version}{2007/09/09 v1.9}
%   \item
%     Catcode section added.
%   \item
%     Missing docstrip tag added.
%   \end{Version}
%   \begin{Version}{2008/08/11 v1.10}
%   \item
%     Code is not changed.
%   \item
%     URLs updated.
%   \end{Version}
%   \begin{Version}{2016/05/16 v1.11}
%   \item
%     Documentation updates.
%   \end{Version}
% \end{History}
%
% \PrintIndex
%
% \Finale
\endinput

%        (quote the arguments according to the demands of your shell)
%
% Documentation:
%    (a) If telprint.drv is present:
%           latex telprint.drv
%    (b) Without telprint.drv:
%           latex telprint.dtx; ...
%    The class ltxdoc loads the configuration file ltxdoc.cfg
%    if available. Here you can specify further options, e.g.
%    use A4 as paper format:
%       \PassOptionsToClass{a4paper}{article}
%
%    Programm calls to get the documentation (example):
%       pdflatex telprint.dtx
%       makeindex -s gind.ist telprint.idx
%       pdflatex telprint.dtx
%       makeindex -s gind.ist telprint.idx
%       pdflatex telprint.dtx
%
% Installation:
%    TDS:tex/generic/oberdiek/telprint.sty
%    TDS:doc/latex/oberdiek/telprint.pdf
%    TDS:doc/latex/oberdiek/test/telprint-test1.tex
%    TDS:source/latex/oberdiek/telprint.dtx
%
%<*ignore>
\begingroup
  \catcode123=1 %
  \catcode125=2 %
  \def\x{LaTeX2e}%
\expandafter\endgroup
\ifcase 0\ifx\install y1\fi\expandafter
         \ifx\csname processbatchFile\endcsname\relax\else1\fi
         \ifx\fmtname\x\else 1\fi\relax
\else\csname fi\endcsname
%</ignore>
%<*install>
\input docstrip.tex
\Msg{************************************************************************}
\Msg{* Installation}
\Msg{* Package: telprint 2016/05/16 v1.11 Format German phone numbers (HO)}
\Msg{************************************************************************}

\keepsilent
\askforoverwritefalse

\let\MetaPrefix\relax
\preamble

This is a generated file.

Project: telprint
Version: 2016/05/16 v1.11

Copyright (C) 1996, 1997, 2004-2008 by
   Heiko Oberdiek <heiko.oberdiek at googlemail.com>

This work may be distributed and/or modified under the
conditions of the LaTeX Project Public License, either
version 1.3c of this license or (at your option) any later
version. This version of this license is in
   https://www.latex-project.org/lppl/lppl-1-3c.txt
and the latest version of this license is in
   https://www.latex-project.org/lppl.txt
and version 1.3 or later is part of all distributions of
LaTeX version 2005/12/01 or later.

This work has the LPPL maintenance status "maintained".

The Current Maintainers of this work are
Heiko Oberdiek and the Oberdiek Package Support Group
https://github.com/ho-tex/oberdiek/issues


The Base Interpreter refers to any `TeX-Format',
because some files are installed in TDS:tex/generic//.

This work consists of the main source file telprint.dtx
and the derived files
   telprint.sty, telprint.pdf, telprint.ins, telprint.drv,
   telprint-test1.tex.

\endpreamble
\let\MetaPrefix\DoubleperCent

\generate{%
  \file{telprint.ins}{\from{telprint.dtx}{install}}%
  \file{telprint.drv}{\from{telprint.dtx}{driver}}%
  \usedir{tex/generic/oberdiek}%
  \file{telprint.sty}{\from{telprint.dtx}{package}}%
%  \usedir{doc/latex/oberdiek/test}%
%  \file{telprint-test1.tex}{\from{telprint.dtx}{test1}}%
  \nopreamble
  \nopostamble
%  \usedir{source/latex/oberdiek/catalogue}%
%  \file{telprint.xml}{\from{telprint.dtx}{catalogue}}%
}

\catcode32=13\relax% active space
\let =\space%
\Msg{************************************************************************}
\Msg{*}
\Msg{* To finish the installation you have to move the following}
\Msg{* file into a directory searched by TeX:}
\Msg{*}
\Msg{*     telprint.sty}
\Msg{*}
\Msg{* To produce the documentation run the file `telprint.drv'}
\Msg{* through LaTeX.}
\Msg{*}
\Msg{* Happy TeXing!}
\Msg{*}
\Msg{************************************************************************}

\endbatchfile
%</install>
%<*ignore>
\fi
%</ignore>
%<*driver>
\NeedsTeXFormat{LaTeX2e}
\ProvidesFile{telprint.drv}%
  [2016/05/16 v1.11 Format German phone numbers (HO)]%
\documentclass{ltxdoc}
\usepackage{holtxdoc}[2011/11/22]
\usepackage[ngerman,english]{babel}
\begin{document}
  \DocInput{telprint.dtx}%
\end{document}
%</driver>
% \fi
%
%
% \CharacterTable
%  {Upper-case    \A\B\C\D\E\F\G\H\I\J\K\L\M\N\O\P\Q\R\S\T\U\V\W\X\Y\Z
%   Lower-case    \a\b\c\d\e\f\g\h\i\j\k\l\m\n\o\p\q\r\s\t\u\v\w\x\y\z
%   Digits        \0\1\2\3\4\5\6\7\8\9
%   Exclamation   \!     Double quote  \"     Hash (number) \#
%   Dollar        \$     Percent       \%     Ampersand     \&
%   Acute accent  \'     Left paren    \(     Right paren   \)
%   Asterisk      \*     Plus          \+     Comma         \,
%   Minus         \-     Point         \.     Solidus       \/
%   Colon         \:     Semicolon     \;     Less than     \<
%   Equals        \=     Greater than  \>     Question mark \?
%   Commercial at \@     Left bracket  \[     Backslash     \\
%   Right bracket \]     Circumflex    \^     Underscore    \_
%   Grave accent  \`     Left brace    \{     Vertical bar  \|
%   Right brace   \}     Tilde         \~}
%
% \GetFileInfo{telprint.drv}
%
% \title{The \xpackage{telprint} package}
% \date{2016/05/16 v1.11}
% \author{Heiko Oberdiek\thanks
% {Please report any issues at \url{https://github.com/ho-tex/oberdiek/issues}}}
%
% \maketitle
%
% \begin{abstract}
% Package \xpackage{telprint} provides \cs{telprint} for formatting
% German phone numbers.
% \end{abstract}
%
% \tableofcontents
%
% \section{Documentation}
%
% \subsection{Introduction}
%
%            This is a very old package that I have written
%            to format phone numbers. It follows German
%            conventions and the documentation is mainly in German.
%
% \subsection{Short overview in English}
%
% \LaTeX:
% \begin{quote}
% |\usepackage{telprint}|\\
% |\telprint{123/456-789}|\\
% \end{quote}
% \plainTeX:
% \begin{quote}
%   |\input telprint.sty|\\
%   |\telprint{123/456-789}|
% \end{quote}
%
% \DescribeMacro\telprint
% |\telprint{...}| formats the explicitly given number.
%     Digits, spaces and some special characters
%     ('+', '/', '-', '(', ')', '\textasciitilde', ' ') are supported.
%     Numbers are divided into groups of two digits from the right.
% Examples:
% \begin{quote}
%     |\telprint{0761/12345}     ==> 07\,61/1\,23\,45|\\
%     |\telprint{01234/567-89}   ==> 0\,12\,34/5\,67\leavevmode\hbox{-}89|\\
%     |\telprint{+49 (6221) 297} ==> +49~(62\,21)~2\,97|
% \end{quote}
%
% \subsubsection{Configuration}
%
% The output of the symbols can be configured by
% \cs{telhyphen}, \cs{telslash}, \cs{telleftparen}, \cs{telrightparen},
% \cs{telplus}, \cs{teltilde}.
% Example:
% \begin{quote}
%   |\telslash{\,/\,}\\|
%   |\telprint{12/34} ==> 12\,/\,34|
% \end{quote}
%
% \DescribeMacro\telspace
% \cs{telspace} configures the space between digit groups.
%
% \DescribeMacro\telnumber
% \cs{telnumber} only formats a number in digit groups; special
%    characters are not recognized.
%
% \subsection{Documentation in German}
%
% \begin{otherlanguage*}{ngerman}
% \hyphenation{To-ken-ma-kros}
% \begin{itemize}
% \item \DescribeMacro\telprint |telprint#1|\\
%   Der eigentliche Anwenderbefehl zur formatierten Ausgabe von
%   Telefonnummern. Diese d\"urfen dabei nur als Zahlen angegeben
%   werden(, da sie tokenweise analysiert werden).
%   Als Trenn- oder Sonderzeichen werden unterst\"utzt:
%   '+', '/', '-', '(', ')', '\textasciitilde', ' '
%   Einfache Leerzeichen werden erkannt und durch Tilden ersetzt, um
%   Trennungen in der Telefonnummer zu verhindern. (Man beachte aus
%   gleichem Grunde die \cs{hbox} bei '-'.)
%   Beispiele:
%   \begin{quote}
%     |\telprint{0761/12345}     ==> 07\,61/1\,23\,45|\\
%     |\telprint{01234/567-89}   ==> 0\,12\,34/5\,67\leavevmode\hbox{-}89|\\
%     |\telprint{+49 (6221) 297} ==> +49~(62\,21)~2\,97|
%   \end{quote}
% \end{itemize}
% Der Rest enth\"alt eher Technisches:
% \begin{itemize}
% \item \DescribeMacro\telspace |\telspace#1|\\
%   Mit diesem Befehl wird der Abstand zwischen den Zifferngruppen
%   angegeben (Default: |\,|).
%   (Durch |\telspace{}| kann dieser zusaetzliche Abstand abgestellt
%   werden.)
% \item \DescribeMacro\telhyphen |\telhyphen#1|\\
%   Dieser Befehl gibt die Art des Bindestriches, wie er ausgegeben
%   werden soll. In der Eingabe darf jedoch nur der einfache
%   Bindestrich stehen:
%   |\telprint{123-45}|, jedoch NIE |\telprint{123--45}|!
%   Kopka-Bindestrich-Fans geben an:
%   |\telhyphen{\leavevmode\hbox{--}}|
% \item
%   \DescribeMacro{\telslash}
%   \DescribeMacro{\telleftparen}
%   \DescribeMacro{\telrightparen}
%   \DescribeMacro{\telplus}
%   \DescribeMacro{\teltilde}
%   |\telslash#1|, |\telleftparen#1|, |\telrightparen#1|, |\telplus#1|,
%   |\teltilde|\\
%   Diese Befehle konfigurieren die Zeichen '/', '(', ')', '+'
%   und '\textasciitilde'. Sie funktionieren analog zu \cs{telhyphen}.
% \item \DescribeMacro\telnumber |\telnumber#1|\\
%   Richtung interner Befehl: Er dient dazu, eine Zifferngruppe
%   in Zweiergruppen auszugeben.
%   Die einzelnen Zahlen werden im Tokenregister \cs{TELtoks}
%   gespeichert. Abwechselnd werden dabei zwischen zwei Token
%   (Zahlen) \cs{TELx} bzw. \cs{TELy} eingefuegt, abh\"angig von dem
%   wechselnden Wert von \cs{TELswitch}. Zum Schluss kann dann einfach
%   festgestellt werden ob die Nummer nun eine geradzahlige oder
%   ungeradzahlige Zahl von Ziffern aufwies. Dem entsprechend wird
%   \cs{TELx} mit dem Zusatzabstand belegt und \cs{TELy} leer definiert
%   oder umgekehrt. )
% \item |\TEL...| interne Befehle, Technisches:\\
%   \cs{TELsplit} dient zur Aufteilung einer zusammengesetzten
%   Telefonnummer (Vorwahl, Hauptnummer, Nebenstelle). In dieser
%   Implementation werden als Trennzeichen nur '/' und '-' erkannt.
%   Die einzelnen Bestandteile wie Vorwahl werden dann dem Befehl
%   \cs{telnumber} zur Formatierung uebergeben.
% \item Die Erkennung von einfachen Leerzeichen ist um einiges
%   schwieriger: Die Tokentrennung ueber Parameter |#1#2| funktioniert
%   nicht f\"ur einfache Leerzeichen, da TeX sie \emph{niemals} als
%   eigenst\"andige Argumente behandelt! (The TeXbook, Chapter 20,
%   p. 201)
%
%   (Anmerkung am Rande: Deshalb funktionieren die entsprechenden
%   Tokenmakros auf S. 149 des Buches "`Einf\"uhrung in TeX"' von
%   N. Schwarz (3. Aufl.) nicht, wenn im Tokenregister als erstes
%   ein einfaches Leerzeichen steht!)
% \end{itemize}
% \end{otherlanguage*}
%
% \StopEventually{
% }
%
% \section{Implementation}
%
%    \begin{macrocode}
%<*package>
%    \end{macrocode}
%
% \subsection{Reload check and package identification}
%    Reload check, especially if the package is not used with \LaTeX.
%    \begin{macrocode}
\begingroup\catcode61\catcode48\catcode32=10\relax%
  \catcode13=5 % ^^M
  \endlinechar=13 %
  \catcode35=6 % #
  \catcode39=12 % '
  \catcode44=12 % ,
  \catcode45=12 % -
  \catcode46=12 % .
  \catcode58=12 % :
  \catcode64=11 % @
  \catcode123=1 % {
  \catcode125=2 % }
  \expandafter\let\expandafter\x\csname ver@telprint.sty\endcsname
  \ifx\x\relax % plain-TeX, first loading
  \else
    \def\empty{}%
    \ifx\x\empty % LaTeX, first loading,
      % variable is initialized, but \ProvidesPackage not yet seen
    \else
      \expandafter\ifx\csname PackageInfo\endcsname\relax
        \def\x#1#2{%
          \immediate\write-1{Package #1 Info: #2.}%
        }%
      \else
        \def\x#1#2{\PackageInfo{#1}{#2, stopped}}%
      \fi
      \x{telprint}{The package is already loaded}%
      \aftergroup\endinput
    \fi
  \fi
\endgroup%
%    \end{macrocode}
%    Package identification:
%    \begin{macrocode}
\begingroup\catcode61\catcode48\catcode32=10\relax%
  \catcode13=5 % ^^M
  \endlinechar=13 %
  \catcode35=6 % #
  \catcode39=12 % '
  \catcode40=12 % (
  \catcode41=12 % )
  \catcode44=12 % ,
  \catcode45=12 % -
  \catcode46=12 % .
  \catcode47=12 % /
  \catcode58=12 % :
  \catcode64=11 % @
  \catcode91=12 % [
  \catcode93=12 % ]
  \catcode123=1 % {
  \catcode125=2 % }
  \expandafter\ifx\csname ProvidesPackage\endcsname\relax
    \def\x#1#2#3[#4]{\endgroup
      \immediate\write-1{Package: #3 #4}%
      \xdef#1{#4}%
    }%
  \else
    \def\x#1#2[#3]{\endgroup
      #2[{#3}]%
      \ifx#1\@undefined
        \xdef#1{#3}%
      \fi
      \ifx#1\relax
        \xdef#1{#3}%
      \fi
    }%
  \fi
\expandafter\x\csname ver@telprint.sty\endcsname
\ProvidesPackage{telprint}%
  [2016/05/16 v1.11 Format German phone numbers (HO)]%
%    \end{macrocode}
%
% \subsection{Catcodes}
%
%    \begin{macrocode}
\begingroup\catcode61\catcode48\catcode32=10\relax%
  \catcode13=5 % ^^M
  \endlinechar=13 %
  \catcode123=1 % {
  \catcode125=2 % }
  \catcode64=11 % @
  \def\x{\endgroup
    \expandafter\edef\csname TELAtEnd\endcsname{%
      \endlinechar=\the\endlinechar\relax
      \catcode13=\the\catcode13\relax
      \catcode32=\the\catcode32\relax
      \catcode35=\the\catcode35\relax
      \catcode61=\the\catcode61\relax
      \catcode64=\the\catcode64\relax
      \catcode123=\the\catcode123\relax
      \catcode125=\the\catcode125\relax
    }%
  }%
\x\catcode61\catcode48\catcode32=10\relax%
\catcode13=5 % ^^M
\endlinechar=13 %
\catcode35=6 % #
\catcode64=11 % @
\catcode123=1 % {
\catcode125=2 % }
\def\TMP@EnsureCode#1#2{%
  \edef\TELAtEnd{%
    \TELAtEnd
    \catcode#1=\the\catcode#1\relax
  }%
  \catcode#1=#2\relax
}
\TMP@EnsureCode{33}{12}% !
\TMP@EnsureCode{36}{3}% $
\TMP@EnsureCode{40}{12}% (
\TMP@EnsureCode{41}{12}% )
\TMP@EnsureCode{42}{12}% *
\TMP@EnsureCode{43}{12}% +
\TMP@EnsureCode{44}{12}% ,
\TMP@EnsureCode{45}{12}% -
\TMP@EnsureCode{46}{12}% .
\TMP@EnsureCode{47}{12}% /
\TMP@EnsureCode{91}{12}% [
\TMP@EnsureCode{93}{12}% ]
\TMP@EnsureCode{126}{13}% ~ (active)
\edef\TELAtEnd{\TELAtEnd\noexpand\endinput}
%    \end{macrocode}
%
% \subsection{Package macros}
%    \begin{macrocode}
\ifx\DeclareRobustCommand\UnDeFiNeD
  \def\DeclareRobustCommand*#1[1]{\def#1##1}%
  \def\TELreset{\let\DeclareRobustCommand=\UnDeFiNeD}%
  \input infwarerr.sty\relax
  \@PackageInfo{telprint}{%
    Macros are not robust!%
  }%
\else
  \let\TELreset=\relax
\fi
%    \end{macrocode}
%    \begin{macro}{\telspace}
%    \begin{macrocode}
\DeclareRobustCommand*{\telspace}[1]{\def\TELspace{#1}}
\telspace{{}$\,${}}
%    \end{macrocode}
%    \end{macro}
%    \begin{macro}{\telhyphen}
%    \begin{macrocode}
\DeclareRobustCommand*{\telhyphen}[1]{\def\TELhyphen{#1}}
\telhyphen{\leavevmode\hbox{-}}% \hbox zur Verhinderung der Trennung
%    \end{macrocode}
%    \end{macro}
%    \begin{macro}{\telslash}
%    \begin{macrocode}
\DeclareRobustCommand*{\telslash}[1]{\def\TELslash{#1}}
\telslash{/}%
%    \end{macrocode}
%    \end{macro}
%    \begin{macro}{\telleftparen}
%    \begin{macrocode}
\DeclareRobustCommand*{\telleftparen}[1]{\def\TELleftparen{#1}}
\telleftparen{(}%
%    \end{macrocode}
%    \end{macro}
%    \begin{macro}{\telrightparen}
%    \begin{macrocode}
\DeclareRobustCommand*{\telrightparen}[1]{\def\TELrightparen{#1}}
\telrightparen{)}%
%    \end{macrocode}
%    \end{macro}
%    \begin{macro}{\telplus}
%    \begin{macrocode}
\DeclareRobustCommand*{\telplus}[1]{\def\TELplus{#1}}
\telplus{+}%
%    \end{macrocode}
%    \end{macro}
%    \begin{macro}{\teltilde}
%    \begin{macrocode}
\DeclareRobustCommand*{\teltilde}[1]{\def\TELtilde{#1}}
\teltilde{~}%
%    \end{macrocode}
%    \end{macro}
%    \begin{macro}{\TELtoks}
%    \begin{macrocode}
\newtoks\TELtoks
%    \end{macrocode}
%    \end{macro}
%    \begin{macro}{\TELnumber}
%    \begin{macrocode}
\def\TELnumber#1#2\TELnumberEND{%
  \begingroup
  \def\0{#2}%
  \expandafter\endgroup
  \ifx\0\empty
    \TELtoks=\expandafter{\the\TELtoks#1}%
    \ifnum\TELswitch=0 %
      \def\TELx{\TELspace}\def\TELy{}%
    \else
      \def\TELx{}\def\TELy{\TELspace}%
    \fi
    \the\TELtoks
  \else
    \ifnum\TELswitch=0 %
      \TELtoks=\expandafter{\the\TELtoks#1\TELx}%
      \def\TELswitch{1}%
    \else
      \TELtoks=\expandafter{\the\TELtoks#1\TELy}%
      \def\TELswitch{0}%
    \fi
    \TELnumber#2\TELnumberEND
  \fi
}
%    \end{macrocode}
%    \end{macro}
%    \begin{macro}{\telnumber}
%    \begin{macrocode}
\DeclareRobustCommand*{\telnumber}[1]{%
  \TELtoks={}%
  \def\TELswitch{0}%
  \TELnumber#1{}\TELnumberEND
}
%    \end{macrocode}
%    \end{macro}
%    \begin{macro}{\TELsplit}
%    \begin{macrocode}
\def\TELsplit{\futurelet\TELfuture\TELdosplit}
%    \end{macrocode}
%    \end{macro}
%    \begin{macro}{\TELdosplit}
%    \begin{macrocode}
\def\TELdosplit#1#2\TELsplitEND
{%
  \def\TELsp{ }%
  \expandafter\ifx\TELsp\TELfuture
    \let\TELfuture=\relax
    \expandafter\telnumber\expandafter{\the\TELtoks}~%
    \telprint{#1#2}% Das Leerzeichen kann nicht #1 sein!
  \else
    \def\TELfirst{#1}%
    \ifx\TELfirst\empty
      \expandafter\telnumber\expandafter{\the\TELtoks}%
      \TELtoks={}%
    \else\if-\TELfirst
      \expandafter\telnumber\expandafter{\the\TELtoks}\TELhyphen
      \telprint{#2}%
    \else\if/\TELfirst
      \expandafter\telnumber\expandafter{\the\TELtoks}\TELslash
      \telprint{#2}%
    \else\if(\TELfirst
      \expandafter\telnumber\expandafter{\the\TELtoks}\TELleftparen
      \telprint{#2}%
    \else\if)\TELfirst
      \expandafter\telnumber\expandafter{\the\TELtoks}\TELrightparen
      \telprint{#2}%
    \else\if+\TELfirst
      \expandafter\telnumber\expandafter{\the\TELtoks}\TELplus
      \telprint{#2}%
    \else\def\TELtemp{~}\ifx\TELtemp\TELfirst
      \expandafter\telnumber\expandafter{\the\TELtoks}\TELtilde
      \telprint{#2}%
    \else
      \TELtoks=\expandafter{\the\TELtoks#1}%
      \TELsplit#2{}\TELsplitEND
    \fi\fi\fi\fi\fi\fi\fi
  \fi
}
%    \end{macrocode}
%    \end{macro}
%    \begin{macro}{\telprint}
%    \begin{macrocode}
\DeclareRobustCommand*{\telprint}[1]{%
  \TELtoks={}%
  \TELsplit#1{}\TELsplitEND
}
%    \end{macrocode}
%    \end{macro}
%    \begin{macrocode}
\TELreset\let\TELreset=\UnDeFiNeD
%    \end{macrocode}
%
%    \begin{macrocode}
\TELAtEnd%
%</package>
%    \end{macrocode}
%
% \section{Test}
%
% \subsection{Catcode checks for loading}
%
%    \begin{macrocode}
%<*test1>
%    \end{macrocode}
%    \begin{macrocode}
\catcode`\{=1 %
\catcode`\}=2 %
\catcode`\#=6 %
\catcode`\@=11 %
\expandafter\ifx\csname count@\endcsname\relax
  \countdef\count@=255 %
\fi
\expandafter\ifx\csname @gobble\endcsname\relax
  \long\def\@gobble#1{}%
\fi
\expandafter\ifx\csname @firstofone\endcsname\relax
  \long\def\@firstofone#1{#1}%
\fi
\expandafter\ifx\csname loop\endcsname\relax
  \expandafter\@firstofone
\else
  \expandafter\@gobble
\fi
{%
  \def\loop#1\repeat{%
    \def\body{#1}%
    \iterate
  }%
  \def\iterate{%
    \body
      \let\next\iterate
    \else
      \let\next\relax
    \fi
    \next
  }%
  \let\repeat=\fi
}%
\def\RestoreCatcodes{}
\count@=0 %
\loop
  \edef\RestoreCatcodes{%
    \RestoreCatcodes
    \catcode\the\count@=\the\catcode\count@\relax
  }%
\ifnum\count@<255 %
  \advance\count@ 1 %
\repeat

\def\RangeCatcodeInvalid#1#2{%
  \count@=#1\relax
  \loop
    \catcode\count@=15 %
  \ifnum\count@<#2\relax
    \advance\count@ 1 %
  \repeat
}
\def\RangeCatcodeCheck#1#2#3{%
  \count@=#1\relax
  \loop
    \ifnum#3=\catcode\count@
    \else
      \errmessage{%
        Character \the\count@\space
        with wrong catcode \the\catcode\count@\space
        instead of \number#3%
      }%
    \fi
  \ifnum\count@<#2\relax
    \advance\count@ 1 %
  \repeat
}
\def\space{ }
\expandafter\ifx\csname LoadCommand\endcsname\relax
  \def\LoadCommand{\input telprint.sty\relax}%
\fi
\def\Test{%
  \RangeCatcodeInvalid{0}{47}%
  \RangeCatcodeInvalid{58}{64}%
  \RangeCatcodeInvalid{91}{96}%
  \RangeCatcodeInvalid{123}{255}%
  \catcode`\@=12 %
  \catcode`\\=0 %
  \catcode`\%=14 %
  \LoadCommand
  \RangeCatcodeCheck{0}{36}{15}%
  \RangeCatcodeCheck{37}{37}{14}%
  \RangeCatcodeCheck{38}{47}{15}%
  \RangeCatcodeCheck{48}{57}{12}%
  \RangeCatcodeCheck{58}{63}{15}%
  \RangeCatcodeCheck{64}{64}{12}%
  \RangeCatcodeCheck{65}{90}{11}%
  \RangeCatcodeCheck{91}{91}{15}%
  \RangeCatcodeCheck{92}{92}{0}%
  \RangeCatcodeCheck{93}{96}{15}%
  \RangeCatcodeCheck{97}{122}{11}%
  \RangeCatcodeCheck{123}{255}{15}%
  \RestoreCatcodes
}
\Test
\csname @@end\endcsname
\end
%    \end{macrocode}
%    \begin{macrocode}
%</test1>
%    \end{macrocode}
%
% \section{Installation}
%
% \subsection{Download}
%
% \paragraph{Package.} This package is available on
% CTAN\footnote{\CTANpkg{telprint}}:
% \begin{description}
% \item[\CTAN{macros/latex/contrib/oberdiek/telprint.dtx}] The source file.
% \item[\CTAN{macros/latex/contrib/oberdiek/telprint.pdf}] Documentation.
% \end{description}
%
%
% \paragraph{Bundle.} All the packages of the bundle `oberdiek'
% are also available in a TDS compliant ZIP archive. There
% the packages are already unpacked and the documentation files
% are generated. The files and directories obey the TDS standard.
% \begin{description}
% \item[\CTANinstall{install/macros/latex/contrib/oberdiek.tds.zip}]
% \end{description}
% \emph{TDS} refers to the standard ``A Directory Structure
% for \TeX\ Files'' (\CTAN{tds/tds.pdf}). Directories
% with \xfile{texmf} in their name are usually organized this way.
%
% \subsection{Bundle installation}
%
% \paragraph{Unpacking.} Unpack the \xfile{oberdiek.tds.zip} in the
% TDS tree (also known as \xfile{texmf} tree) of your choice.
% Example (linux):
% \begin{quote}
%   |unzip oberdiek.tds.zip -d ~/texmf|
% \end{quote}
%
% \paragraph{Script installation.}
% Check the directory \xfile{TDS:scripts/oberdiek/} for
% scripts that need further installation steps.
% Package \xpackage{attachfile2} comes with the Perl script
% \xfile{pdfatfi.pl} that should be installed in such a way
% that it can be called as \texttt{pdfatfi}.
% Example (linux):
% \begin{quote}
%   |chmod +x scripts/oberdiek/pdfatfi.pl|\\
%   |cp scripts/oberdiek/pdfatfi.pl /usr/local/bin/|
% \end{quote}
%
% \subsection{Package installation}
%
% \paragraph{Unpacking.} The \xfile{.dtx} file is a self-extracting
% \docstrip\ archive. The files are extracted by running the
% \xfile{.dtx} through \plainTeX:
% \begin{quote}
%   \verb|tex telprint.dtx|
% \end{quote}
%
% \paragraph{TDS.} Now the different files must be moved into
% the different directories in your installation TDS tree
% (also known as \xfile{texmf} tree):
% \begin{quote}
% \def\t{^^A
% \begin{tabular}{@{}>{\ttfamily}l@{ $\rightarrow$ }>{\ttfamily}l@{}}
%   telprint.sty & tex/generic/oberdiek/telprint.sty\\
%   telprint.pdf & doc/latex/oberdiek/telprint.pdf\\
%   test/telprint-test1.tex & doc/latex/oberdiek/test/telprint-test1.tex\\
%   telprint.dtx & source/latex/oberdiek/telprint.dtx\\
% \end{tabular}^^A
% }^^A
% \sbox0{\t}^^A
% \ifdim\wd0>\linewidth
%   \begingroup
%     \advance\linewidth by\leftmargin
%     \advance\linewidth by\rightmargin
%   \edef\x{\endgroup
%     \def\noexpand\lw{\the\linewidth}^^A
%   }\x
%   \def\lwbox{^^A
%     \leavevmode
%     \hbox to \linewidth{^^A
%       \kern-\leftmargin\relax
%       \hss
%       \usebox0
%       \hss
%       \kern-\rightmargin\relax
%     }^^A
%   }^^A
%   \ifdim\wd0>\lw
%     \sbox0{\small\t}^^A
%     \ifdim\wd0>\linewidth
%       \ifdim\wd0>\lw
%         \sbox0{\footnotesize\t}^^A
%         \ifdim\wd0>\linewidth
%           \ifdim\wd0>\lw
%             \sbox0{\scriptsize\t}^^A
%             \ifdim\wd0>\linewidth
%               \ifdim\wd0>\lw
%                 \sbox0{\tiny\t}^^A
%                 \ifdim\wd0>\linewidth
%                   \lwbox
%                 \else
%                   \usebox0
%                 \fi
%               \else
%                 \lwbox
%               \fi
%             \else
%               \usebox0
%             \fi
%           \else
%             \lwbox
%           \fi
%         \else
%           \usebox0
%         \fi
%       \else
%         \lwbox
%       \fi
%     \else
%       \usebox0
%     \fi
%   \else
%     \lwbox
%   \fi
% \else
%   \usebox0
% \fi
% \end{quote}
% If you have a \xfile{docstrip.cfg} that configures and enables \docstrip's
% TDS installing feature, then some files can already be in the right
% place, see the documentation of \docstrip.
%
% \subsection{Refresh file name databases}
%
% If your \TeX~distribution
% (\teTeX, \mikTeX, \dots) relies on file name databases, you must refresh
% these. For example, \teTeX\ users run \verb|texhash| or
% \verb|mktexlsr|.
%
% \subsection{Some details for the interested}
%
% \paragraph{Unpacking with \LaTeX.}
% The \xfile{.dtx} chooses its action depending on the format:
% \begin{description}
% \item[\plainTeX:] Run \docstrip\ and extract the files.
% \item[\LaTeX:] Generate the documentation.
% \end{description}
% If you insist on using \LaTeX\ for \docstrip\ (really,
% \docstrip\ does not need \LaTeX), then inform the autodetect routine
% about your intention:
% \begin{quote}
%   \verb|latex \let\install=y% \iffalse meta-comment
%
% File: telprint.dtx
% Version: 2016/05/16 v1.11
% Info: Format German phone numbers
%
% Copyright (C) 1996, 1997, 2004-2008 by
%    Heiko Oberdiek <heiko.oberdiek at googlemail.com>
%    2016
%    https://github.com/ho-tex/oberdiek/issues
%
% This work may be distributed and/or modified under the
% conditions of the LaTeX Project Public License, either
% version 1.3c of this license or (at your option) any later
% version. This version of this license is in
%    http://www.latex-project.org/lppl/lppl-1-3c.txt
% and the latest version of this license is in
%    http://www.latex-project.org/lppl.txt
% and version 1.3 or later is part of all distributions of
% LaTeX version 2005/12/01 or later.
%
% This work has the LPPL maintenance status "maintained".
%
% This Current Maintainer of this work is Heiko Oberdiek.
%
% The Base Interpreter refers to any `TeX-Format',
% because some files are installed in TDS:tex/generic//.
%
% This work consists of the main source file telprint.dtx
% and the derived files
%    telprint.sty, telprint.pdf, telprint.ins, telprint.drv,
%    telprint-test1.tex.
%
% Distribution:
%    CTAN:macros/latex/contrib/oberdiek/telprint.dtx
%    CTAN:macros/latex/contrib/oberdiek/telprint.pdf
%
% Unpacking:
%    (a) If telprint.ins is present:
%           tex telprint.ins
%    (b) Without telprint.ins:
%           tex telprint.dtx
%    (c) If you insist on using LaTeX
%           latex \let\install=y% \iffalse meta-comment
%
% File: telprint.dtx
% Version: 2016/05/16 v1.11
% Info: Format German phone numbers
%
% Copyright (C) 1996, 1997, 2004-2008 by
%    Heiko Oberdiek <heiko.oberdiek at googlemail.com>
%    2016
%    https://github.com/ho-tex/oberdiek/issues
%
% This work may be distributed and/or modified under the
% conditions of the LaTeX Project Public License, either
% version 1.3c of this license or (at your option) any later
% version. This version of this license is in
%    http://www.latex-project.org/lppl/lppl-1-3c.txt
% and the latest version of this license is in
%    http://www.latex-project.org/lppl.txt
% and version 1.3 or later is part of all distributions of
% LaTeX version 2005/12/01 or later.
%
% This work has the LPPL maintenance status "maintained".
%
% This Current Maintainer of this work is Heiko Oberdiek.
%
% The Base Interpreter refers to any `TeX-Format',
% because some files are installed in TDS:tex/generic//.
%
% This work consists of the main source file telprint.dtx
% and the derived files
%    telprint.sty, telprint.pdf, telprint.ins, telprint.drv,
%    telprint-test1.tex.
%
% Distribution:
%    CTAN:macros/latex/contrib/oberdiek/telprint.dtx
%    CTAN:macros/latex/contrib/oberdiek/telprint.pdf
%
% Unpacking:
%    (a) If telprint.ins is present:
%           tex telprint.ins
%    (b) Without telprint.ins:
%           tex telprint.dtx
%    (c) If you insist on using LaTeX
%           latex \let\install=y% \iffalse meta-comment
%
% File: telprint.dtx
% Version: 2016/05/16 v1.11
% Info: Format German phone numbers
%
% Copyright (C) 1996, 1997, 2004-2008 by
%    Heiko Oberdiek <heiko.oberdiek at googlemail.com>
%    2016
%    https://github.com/ho-tex/oberdiek/issues
%
% This work may be distributed and/or modified under the
% conditions of the LaTeX Project Public License, either
% version 1.3c of this license or (at your option) any later
% version. This version of this license is in
%    http://www.latex-project.org/lppl/lppl-1-3c.txt
% and the latest version of this license is in
%    http://www.latex-project.org/lppl.txt
% and version 1.3 or later is part of all distributions of
% LaTeX version 2005/12/01 or later.
%
% This work has the LPPL maintenance status "maintained".
%
% This Current Maintainer of this work is Heiko Oberdiek.
%
% The Base Interpreter refers to any `TeX-Format',
% because some files are installed in TDS:tex/generic//.
%
% This work consists of the main source file telprint.dtx
% and the derived files
%    telprint.sty, telprint.pdf, telprint.ins, telprint.drv,
%    telprint-test1.tex.
%
% Distribution:
%    CTAN:macros/latex/contrib/oberdiek/telprint.dtx
%    CTAN:macros/latex/contrib/oberdiek/telprint.pdf
%
% Unpacking:
%    (a) If telprint.ins is present:
%           tex telprint.ins
%    (b) Without telprint.ins:
%           tex telprint.dtx
%    (c) If you insist on using LaTeX
%           latex \let\install=y\input{telprint.dtx}
%        (quote the arguments according to the demands of your shell)
%
% Documentation:
%    (a) If telprint.drv is present:
%           latex telprint.drv
%    (b) Without telprint.drv:
%           latex telprint.dtx; ...
%    The class ltxdoc loads the configuration file ltxdoc.cfg
%    if available. Here you can specify further options, e.g.
%    use A4 as paper format:
%       \PassOptionsToClass{a4paper}{article}
%
%    Programm calls to get the documentation (example):
%       pdflatex telprint.dtx
%       makeindex -s gind.ist telprint.idx
%       pdflatex telprint.dtx
%       makeindex -s gind.ist telprint.idx
%       pdflatex telprint.dtx
%
% Installation:
%    TDS:tex/generic/oberdiek/telprint.sty
%    TDS:doc/latex/oberdiek/telprint.pdf
%    TDS:doc/latex/oberdiek/test/telprint-test1.tex
%    TDS:source/latex/oberdiek/telprint.dtx
%
%<*ignore>
\begingroup
  \catcode123=1 %
  \catcode125=2 %
  \def\x{LaTeX2e}%
\expandafter\endgroup
\ifcase 0\ifx\install y1\fi\expandafter
         \ifx\csname processbatchFile\endcsname\relax\else1\fi
         \ifx\fmtname\x\else 1\fi\relax
\else\csname fi\endcsname
%</ignore>
%<*install>
\input docstrip.tex
\Msg{************************************************************************}
\Msg{* Installation}
\Msg{* Package: telprint 2016/05/16 v1.11 Format German phone numbers (HO)}
\Msg{************************************************************************}

\keepsilent
\askforoverwritefalse

\let\MetaPrefix\relax
\preamble

This is a generated file.

Project: telprint
Version: 2016/05/16 v1.11

Copyright (C) 1996, 1997, 2004-2008 by
   Heiko Oberdiek <heiko.oberdiek at googlemail.com>

This work may be distributed and/or modified under the
conditions of the LaTeX Project Public License, either
version 1.3c of this license or (at your option) any later
version. This version of this license is in
   http://www.latex-project.org/lppl/lppl-1-3c.txt
and the latest version of this license is in
   http://www.latex-project.org/lppl.txt
and version 1.3 or later is part of all distributions of
LaTeX version 2005/12/01 or later.

This work has the LPPL maintenance status "maintained".

This Current Maintainer of this work is Heiko Oberdiek.

The Base Interpreter refers to any `TeX-Format',
because some files are installed in TDS:tex/generic//.

This work consists of the main source file telprint.dtx
and the derived files
   telprint.sty, telprint.pdf, telprint.ins, telprint.drv,
   telprint-test1.tex.

\endpreamble
\let\MetaPrefix\DoubleperCent

\generate{%
  \file{telprint.ins}{\from{telprint.dtx}{install}}%
  \file{telprint.drv}{\from{telprint.dtx}{driver}}%
  \usedir{tex/generic/oberdiek}%
  \file{telprint.sty}{\from{telprint.dtx}{package}}%
%  \usedir{doc/latex/oberdiek/test}%
%  \file{telprint-test1.tex}{\from{telprint.dtx}{test1}}%
  \nopreamble
  \nopostamble
%  \usedir{source/latex/oberdiek/catalogue}%
%  \file{telprint.xml}{\from{telprint.dtx}{catalogue}}%
}

\catcode32=13\relax% active space
\let =\space%
\Msg{************************************************************************}
\Msg{*}
\Msg{* To finish the installation you have to move the following}
\Msg{* file into a directory searched by TeX:}
\Msg{*}
\Msg{*     telprint.sty}
\Msg{*}
\Msg{* To produce the documentation run the file `telprint.drv'}
\Msg{* through LaTeX.}
\Msg{*}
\Msg{* Happy TeXing!}
\Msg{*}
\Msg{************************************************************************}

\endbatchfile
%</install>
%<*ignore>
\fi
%</ignore>
%<*driver>
\NeedsTeXFormat{LaTeX2e}
\ProvidesFile{telprint.drv}%
  [2016/05/16 v1.11 Format German phone numbers (HO)]%
\documentclass{ltxdoc}
\usepackage{holtxdoc}[2011/11/22]
\usepackage[ngerman,english]{babel}
\begin{document}
  \DocInput{telprint.dtx}%
\end{document}
%</driver>
% \fi
%
%
% \CharacterTable
%  {Upper-case    \A\B\C\D\E\F\G\H\I\J\K\L\M\N\O\P\Q\R\S\T\U\V\W\X\Y\Z
%   Lower-case    \a\b\c\d\e\f\g\h\i\j\k\l\m\n\o\p\q\r\s\t\u\v\w\x\y\z
%   Digits        \0\1\2\3\4\5\6\7\8\9
%   Exclamation   \!     Double quote  \"     Hash (number) \#
%   Dollar        \$     Percent       \%     Ampersand     \&
%   Acute accent  \'     Left paren    \(     Right paren   \)
%   Asterisk      \*     Plus          \+     Comma         \,
%   Minus         \-     Point         \.     Solidus       \/
%   Colon         \:     Semicolon     \;     Less than     \<
%   Equals        \=     Greater than  \>     Question mark \?
%   Commercial at \@     Left bracket  \[     Backslash     \\
%   Right bracket \]     Circumflex    \^     Underscore    \_
%   Grave accent  \`     Left brace    \{     Vertical bar  \|
%   Right brace   \}     Tilde         \~}
%
% \GetFileInfo{telprint.drv}
%
% \title{The \xpackage{telprint} package}
% \date{2016/05/16 v1.11}
% \author{Heiko Oberdiek\thanks
% {Please report any issues at https://github.com/ho-tex/oberdiek/issues}\\
% \xemail{heiko.oberdiek at googlemail.com}}
%
% \maketitle
%
% \begin{abstract}
% Package \xpackage{telprint} provides \cs{telprint} for formatting
% German phone numbers.
% \end{abstract}
%
% \tableofcontents
%
% \section{Documentation}
%
% \subsection{Introduction}
%
%            This is a very old package that I have written
%            to format phone numbers. It follows German
%            conventions and the documentation is mainly in German.
%
% \subsection{Short overview in English}
%
% \LaTeX:
% \begin{quote}
% |\usepackage{telprint}|\\
% |\telprint{123/456-789}|\\
% \end{quote}
% \plainTeX:
% \begin{quote}
%   |\input telprint.sty|\\
%   |\telprint{123/456-789}|
% \end{quote}
%
% \DescribeMacro\telprint
% |\telprint{...}| formats the explicitly given number.
%     Digits, spaces and some special characters
%     ('+', '/', '-', '(', ')', '\textasciitilde', ' ') are supported.
%     Numbers are divided into groups of two digits from the right.
% Examples:
% \begin{quote}
%     |\telprint{0761/12345}     ==> 07\,61/1\,23\,45|\\
%     |\telprint{01234/567-89}   ==> 0\,12\,34/5\,67\leavevmode\hbox{-}89|\\
%     |\telprint{+49 (6221) 297} ==> +49~(62\,21)~2\,97|
% \end{quote}
%
% \subsubsection{Configuration}
%
% The output of the symbols can be configured by
% \cs{telhyphen}, \cs{telslash}, \cs{telleftparen}, \cs{telrightparen},
% \cs{telplus}, \cs{teltilde}.
% Example:
% \begin{quote}
%   |\telslash{\,/\,}\\|
%   |\telprint{12/34} ==> 12\,/\,34|
% \end{quote}
%
% \DescribeMacro\telspace
% \cs{telspace} configures the space between digit groups.
%
% \DescribeMacro\telnumber
% \cs{telnumber} only formats a number in digit groups; special
%    characters are not recognized.
%
% \subsection{Documentation in German}
%
% \begin{otherlanguage*}{ngerman}
% \hyphenation{To-ken-ma-kros}
% \begin{itemize}
% \item \DescribeMacro\telprint |telprint#1|\\
%   Der eigentliche Anwenderbefehl zur formatierten Ausgabe von
%   Telefonnummern. Diese d\"urfen dabei nur als Zahlen angegeben
%   werden(, da sie tokenweise analysiert werden).
%   Als Trenn- oder Sonderzeichen werden unterst\"utzt:
%   '+', '/', '-', '(', ')', '\textasciitilde', ' '
%   Einfache Leerzeichen werden erkannt und durch Tilden ersetzt, um
%   Trennungen in der Telefonnummer zu verhindern. (Man beachte aus
%   gleichem Grunde die \cs{hbox} bei '-'.)
%   Beispiele:
%   \begin{quote}
%     |\telprint{0761/12345}     ==> 07\,61/1\,23\,45|\\
%     |\telprint{01234/567-89}   ==> 0\,12\,34/5\,67\leavevmode\hbox{-}89|\\
%     |\telprint{+49 (6221) 297} ==> +49~(62\,21)~2\,97|
%   \end{quote}
% \end{itemize}
% Der Rest enth\"alt eher Technisches:
% \begin{itemize}
% \item \DescribeMacro\telspace |\telspace#1|\\
%   Mit diesem Befehl wird der Abstand zwischen den Zifferngruppen
%   angegeben (Default: |\,|).
%   (Durch |\telspace{}| kann dieser zusaetzliche Abstand abgestellt
%   werden.)
% \item \DescribeMacro\telhyphen |\telhyphen#1|\\
%   Dieser Befehl gibt die Art des Bindestriches, wie er ausgegeben
%   werden soll. In der Eingabe darf jedoch nur der einfache
%   Bindestrich stehen:
%   |\telprint{123-45}|, jedoch NIE |\telprint{123--45}|!
%   Kopka-Bindestrich-Fans geben an:
%   |\telhyphen{\leavevmode\hbox{--}}|
% \item
%   \DescribeMacro{\telslash}
%   \DescribeMacro{\telleftparen}
%   \DescribeMacro{\telrightparen}
%   \DescribeMacro{\telplus}
%   \DescribeMacro{\teltilde}
%   |\telslash#1|, |\telleftparen#1|, |\telrightparen#1|, |\telplus#1|,
%   |\teltilde|\\
%   Diese Befehle konfigurieren die Zeichen '/', '(', ')', '+'
%   und '\textasciitilde'. Sie funktionieren analog zu \cs{telhyphen}.
% \item \DescribeMacro\telnumber |\telnumber#1|\\
%   Richtung interner Befehl: Er dient dazu, eine Zifferngruppe
%   in Zweiergruppen auszugeben.
%   Die einzelnen Zahlen werden im Tokenregister \cs{TELtoks}
%   gespeichert. Abwechselnd werden dabei zwischen zwei Token
%   (Zahlen) \cs{TELx} bzw. \cs{TELy} eingefuegt, abh\"angig von dem
%   wechselnden Wert von \cs{TELswitch}. Zum Schluss kann dann einfach
%   festgestellt werden ob die Nummer nun eine geradzahlige oder
%   ungeradzahlige Zahl von Ziffern aufwies. Dem entsprechend wird
%   \cs{TELx} mit dem Zusatzabstand belegt und \cs{TELy} leer definiert
%   oder umgekehrt. )
% \item |\TEL...| interne Befehle, Technisches:\\
%   \cs{TELsplit} dient zur Aufteilung einer zusammengesetzten
%   Telefonnummer (Vorwahl, Hauptnummer, Nebenstelle). In dieser
%   Implementation werden als Trennzeichen nur '/' und '-' erkannt.
%   Die einzelnen Bestandteile wie Vorwahl werden dann dem Befehl
%   \cs{telnumber} zur Formatierung uebergeben.
% \item Die Erkennung von einfachen Leerzeichen ist um einiges
%   schwieriger: Die Tokentrennung ueber Parameter |#1#2| funktioniert
%   nicht f\"ur einfache Leerzeichen, da TeX sie \emph{niemals} als
%   eigenst\"andige Argumente behandelt! (The TeXbook, Chapter 20,
%   p. 201)
%
%   (Anmerkung am Rande: Deshalb funktionieren die entsprechenden
%   Tokenmakros auf S. 149 des Buches "`Einf\"uhrung in TeX"' von
%   N. Schwarz (3. Aufl.) nicht, wenn im Tokenregister als erstes
%   ein einfaches Leerzeichen steht!)
% \end{itemize}
% \end{otherlanguage*}
%
% \StopEventually{
% }
%
% \section{Implementation}
%
%    \begin{macrocode}
%<*package>
%    \end{macrocode}
%
% \subsection{Reload check and package identification}
%    Reload check, especially if the package is not used with \LaTeX.
%    \begin{macrocode}
\begingroup\catcode61\catcode48\catcode32=10\relax%
  \catcode13=5 % ^^M
  \endlinechar=13 %
  \catcode35=6 % #
  \catcode39=12 % '
  \catcode44=12 % ,
  \catcode45=12 % -
  \catcode46=12 % .
  \catcode58=12 % :
  \catcode64=11 % @
  \catcode123=1 % {
  \catcode125=2 % }
  \expandafter\let\expandafter\x\csname ver@telprint.sty\endcsname
  \ifx\x\relax % plain-TeX, first loading
  \else
    \def\empty{}%
    \ifx\x\empty % LaTeX, first loading,
      % variable is initialized, but \ProvidesPackage not yet seen
    \else
      \expandafter\ifx\csname PackageInfo\endcsname\relax
        \def\x#1#2{%
          \immediate\write-1{Package #1 Info: #2.}%
        }%
      \else
        \def\x#1#2{\PackageInfo{#1}{#2, stopped}}%
      \fi
      \x{telprint}{The package is already loaded}%
      \aftergroup\endinput
    \fi
  \fi
\endgroup%
%    \end{macrocode}
%    Package identification:
%    \begin{macrocode}
\begingroup\catcode61\catcode48\catcode32=10\relax%
  \catcode13=5 % ^^M
  \endlinechar=13 %
  \catcode35=6 % #
  \catcode39=12 % '
  \catcode40=12 % (
  \catcode41=12 % )
  \catcode44=12 % ,
  \catcode45=12 % -
  \catcode46=12 % .
  \catcode47=12 % /
  \catcode58=12 % :
  \catcode64=11 % @
  \catcode91=12 % [
  \catcode93=12 % ]
  \catcode123=1 % {
  \catcode125=2 % }
  \expandafter\ifx\csname ProvidesPackage\endcsname\relax
    \def\x#1#2#3[#4]{\endgroup
      \immediate\write-1{Package: #3 #4}%
      \xdef#1{#4}%
    }%
  \else
    \def\x#1#2[#3]{\endgroup
      #2[{#3}]%
      \ifx#1\@undefined
        \xdef#1{#3}%
      \fi
      \ifx#1\relax
        \xdef#1{#3}%
      \fi
    }%
  \fi
\expandafter\x\csname ver@telprint.sty\endcsname
\ProvidesPackage{telprint}%
  [2016/05/16 v1.11 Format German phone numbers (HO)]%
%    \end{macrocode}
%
% \subsection{Catcodes}
%
%    \begin{macrocode}
\begingroup\catcode61\catcode48\catcode32=10\relax%
  \catcode13=5 % ^^M
  \endlinechar=13 %
  \catcode123=1 % {
  \catcode125=2 % }
  \catcode64=11 % @
  \def\x{\endgroup
    \expandafter\edef\csname TELAtEnd\endcsname{%
      \endlinechar=\the\endlinechar\relax
      \catcode13=\the\catcode13\relax
      \catcode32=\the\catcode32\relax
      \catcode35=\the\catcode35\relax
      \catcode61=\the\catcode61\relax
      \catcode64=\the\catcode64\relax
      \catcode123=\the\catcode123\relax
      \catcode125=\the\catcode125\relax
    }%
  }%
\x\catcode61\catcode48\catcode32=10\relax%
\catcode13=5 % ^^M
\endlinechar=13 %
\catcode35=6 % #
\catcode64=11 % @
\catcode123=1 % {
\catcode125=2 % }
\def\TMP@EnsureCode#1#2{%
  \edef\TELAtEnd{%
    \TELAtEnd
    \catcode#1=\the\catcode#1\relax
  }%
  \catcode#1=#2\relax
}
\TMP@EnsureCode{33}{12}% !
\TMP@EnsureCode{36}{3}% $
\TMP@EnsureCode{40}{12}% (
\TMP@EnsureCode{41}{12}% )
\TMP@EnsureCode{42}{12}% *
\TMP@EnsureCode{43}{12}% +
\TMP@EnsureCode{44}{12}% ,
\TMP@EnsureCode{45}{12}% -
\TMP@EnsureCode{46}{12}% .
\TMP@EnsureCode{47}{12}% /
\TMP@EnsureCode{91}{12}% [
\TMP@EnsureCode{93}{12}% ]
\TMP@EnsureCode{126}{13}% ~ (active)
\edef\TELAtEnd{\TELAtEnd\noexpand\endinput}
%    \end{macrocode}
%
% \subsection{Package macros}
%    \begin{macrocode}
\ifx\DeclareRobustCommand\UnDeFiNeD
  \def\DeclareRobustCommand*#1[1]{\def#1##1}%
  \def\TELreset{\let\DeclareRobustCommand=\UnDeFiNeD}%
  \input infwarerr.sty\relax
  \@PackageInfo{telprint}{%
    Macros are not robust!%
  }%
\else
  \let\TELreset=\relax
\fi
%    \end{macrocode}
%    \begin{macro}{\telspace}
%    \begin{macrocode}
\DeclareRobustCommand*{\telspace}[1]{\def\TELspace{#1}}
\telspace{{}$\,${}}
%    \end{macrocode}
%    \end{macro}
%    \begin{macro}{\telhyphen}
%    \begin{macrocode}
\DeclareRobustCommand*{\telhyphen}[1]{\def\TELhyphen{#1}}
\telhyphen{\leavevmode\hbox{-}}% \hbox zur Verhinderung der Trennung
%    \end{macrocode}
%    \end{macro}
%    \begin{macro}{\telslash}
%    \begin{macrocode}
\DeclareRobustCommand*{\telslash}[1]{\def\TELslash{#1}}
\telslash{/}%
%    \end{macrocode}
%    \end{macro}
%    \begin{macro}{\telleftparen}
%    \begin{macrocode}
\DeclareRobustCommand*{\telleftparen}[1]{\def\TELleftparen{#1}}
\telleftparen{(}%
%    \end{macrocode}
%    \end{macro}
%    \begin{macro}{\telrightparen}
%    \begin{macrocode}
\DeclareRobustCommand*{\telrightparen}[1]{\def\TELrightparen{#1}}
\telrightparen{)}%
%    \end{macrocode}
%    \end{macro}
%    \begin{macro}{\telplus}
%    \begin{macrocode}
\DeclareRobustCommand*{\telplus}[1]{\def\TELplus{#1}}
\telplus{+}%
%    \end{macrocode}
%    \end{macro}
%    \begin{macro}{\teltilde}
%    \begin{macrocode}
\DeclareRobustCommand*{\teltilde}[1]{\def\TELtilde{#1}}
\teltilde{~}%
%    \end{macrocode}
%    \end{macro}
%    \begin{macro}{\TELtoks}
%    \begin{macrocode}
\newtoks\TELtoks
%    \end{macrocode}
%    \end{macro}
%    \begin{macro}{\TELnumber}
%    \begin{macrocode}
\def\TELnumber#1#2\TELnumberEND{%
  \begingroup
  \def\0{#2}%
  \expandafter\endgroup
  \ifx\0\empty
    \TELtoks=\expandafter{\the\TELtoks#1}%
    \ifnum\TELswitch=0 %
      \def\TELx{\TELspace}\def\TELy{}%
    \else
      \def\TELx{}\def\TELy{\TELspace}%
    \fi
    \the\TELtoks
  \else
    \ifnum\TELswitch=0 %
      \TELtoks=\expandafter{\the\TELtoks#1\TELx}%
      \def\TELswitch{1}%
    \else
      \TELtoks=\expandafter{\the\TELtoks#1\TELy}%
      \def\TELswitch{0}%
    \fi
    \TELnumber#2\TELnumberEND
  \fi
}
%    \end{macrocode}
%    \end{macro}
%    \begin{macro}{\telnumber}
%    \begin{macrocode}
\DeclareRobustCommand*{\telnumber}[1]{%
  \TELtoks={}%
  \def\TELswitch{0}%
  \TELnumber#1{}\TELnumberEND
}
%    \end{macrocode}
%    \end{macro}
%    \begin{macro}{\TELsplit}
%    \begin{macrocode}
\def\TELsplit{\futurelet\TELfuture\TELdosplit}
%    \end{macrocode}
%    \end{macro}
%    \begin{macro}{\TELdosplit}
%    \begin{macrocode}
\def\TELdosplit#1#2\TELsplitEND
{%
  \def\TELsp{ }%
  \expandafter\ifx\TELsp\TELfuture
    \let\TELfuture=\relax
    \expandafter\telnumber\expandafter{\the\TELtoks}~%
    \telprint{#1#2}% Das Leerzeichen kann nicht #1 sein!
  \else
    \def\TELfirst{#1}%
    \ifx\TELfirst\empty
      \expandafter\telnumber\expandafter{\the\TELtoks}%
      \TELtoks={}%
    \else\if-\TELfirst
      \expandafter\telnumber\expandafter{\the\TELtoks}\TELhyphen
      \telprint{#2}%
    \else\if/\TELfirst
      \expandafter\telnumber\expandafter{\the\TELtoks}\TELslash
      \telprint{#2}%
    \else\if(\TELfirst
      \expandafter\telnumber\expandafter{\the\TELtoks}\TELleftparen
      \telprint{#2}%
    \else\if)\TELfirst
      \expandafter\telnumber\expandafter{\the\TELtoks}\TELrightparen
      \telprint{#2}%
    \else\if+\TELfirst
      \expandafter\telnumber\expandafter{\the\TELtoks}\TELplus
      \telprint{#2}%
    \else\def\TELtemp{~}\ifx\TELtemp\TELfirst
      \expandafter\telnumber\expandafter{\the\TELtoks}\TELtilde
      \telprint{#2}%
    \else
      \TELtoks=\expandafter{\the\TELtoks#1}%
      \TELsplit#2{}\TELsplitEND
    \fi\fi\fi\fi\fi\fi\fi
  \fi
}
%    \end{macrocode}
%    \end{macro}
%    \begin{macro}{\telprint}
%    \begin{macrocode}
\DeclareRobustCommand*{\telprint}[1]{%
  \TELtoks={}%
  \TELsplit#1{}\TELsplitEND
}
%    \end{macrocode}
%    \end{macro}
%    \begin{macrocode}
\TELreset\let\TELreset=\UnDeFiNeD
%    \end{macrocode}
%
%    \begin{macrocode}
\TELAtEnd%
%</package>
%    \end{macrocode}
%
% \section{Test}
%
% \subsection{Catcode checks for loading}
%
%    \begin{macrocode}
%<*test1>
%    \end{macrocode}
%    \begin{macrocode}
\catcode`\{=1 %
\catcode`\}=2 %
\catcode`\#=6 %
\catcode`\@=11 %
\expandafter\ifx\csname count@\endcsname\relax
  \countdef\count@=255 %
\fi
\expandafter\ifx\csname @gobble\endcsname\relax
  \long\def\@gobble#1{}%
\fi
\expandafter\ifx\csname @firstofone\endcsname\relax
  \long\def\@firstofone#1{#1}%
\fi
\expandafter\ifx\csname loop\endcsname\relax
  \expandafter\@firstofone
\else
  \expandafter\@gobble
\fi
{%
  \def\loop#1\repeat{%
    \def\body{#1}%
    \iterate
  }%
  \def\iterate{%
    \body
      \let\next\iterate
    \else
      \let\next\relax
    \fi
    \next
  }%
  \let\repeat=\fi
}%
\def\RestoreCatcodes{}
\count@=0 %
\loop
  \edef\RestoreCatcodes{%
    \RestoreCatcodes
    \catcode\the\count@=\the\catcode\count@\relax
  }%
\ifnum\count@<255 %
  \advance\count@ 1 %
\repeat

\def\RangeCatcodeInvalid#1#2{%
  \count@=#1\relax
  \loop
    \catcode\count@=15 %
  \ifnum\count@<#2\relax
    \advance\count@ 1 %
  \repeat
}
\def\RangeCatcodeCheck#1#2#3{%
  \count@=#1\relax
  \loop
    \ifnum#3=\catcode\count@
    \else
      \errmessage{%
        Character \the\count@\space
        with wrong catcode \the\catcode\count@\space
        instead of \number#3%
      }%
    \fi
  \ifnum\count@<#2\relax
    \advance\count@ 1 %
  \repeat
}
\def\space{ }
\expandafter\ifx\csname LoadCommand\endcsname\relax
  \def\LoadCommand{\input telprint.sty\relax}%
\fi
\def\Test{%
  \RangeCatcodeInvalid{0}{47}%
  \RangeCatcodeInvalid{58}{64}%
  \RangeCatcodeInvalid{91}{96}%
  \RangeCatcodeInvalid{123}{255}%
  \catcode`\@=12 %
  \catcode`\\=0 %
  \catcode`\%=14 %
  \LoadCommand
  \RangeCatcodeCheck{0}{36}{15}%
  \RangeCatcodeCheck{37}{37}{14}%
  \RangeCatcodeCheck{38}{47}{15}%
  \RangeCatcodeCheck{48}{57}{12}%
  \RangeCatcodeCheck{58}{63}{15}%
  \RangeCatcodeCheck{64}{64}{12}%
  \RangeCatcodeCheck{65}{90}{11}%
  \RangeCatcodeCheck{91}{91}{15}%
  \RangeCatcodeCheck{92}{92}{0}%
  \RangeCatcodeCheck{93}{96}{15}%
  \RangeCatcodeCheck{97}{122}{11}%
  \RangeCatcodeCheck{123}{255}{15}%
  \RestoreCatcodes
}
\Test
\csname @@end\endcsname
\end
%    \end{macrocode}
%    \begin{macrocode}
%</test1>
%    \end{macrocode}
%
% \section{Installation}
%
% \subsection{Download}
%
% \paragraph{Package.} This package is available on
% CTAN\footnote{\url{https://ctan.org/pkg/telprint}}:
% \begin{description}
% \item[\CTAN{macros/latex/contrib/oberdiek/telprint.dtx}] The source file.
% \item[\CTAN{macros/latex/contrib/oberdiek/telprint.pdf}] Documentation.
% \end{description}
%
%
% \paragraph{Bundle.} All the packages of the bundle `oberdiek'
% are also available in a TDS compliant ZIP archive. There
% the packages are already unpacked and the documentation files
% are generated. The files and directories obey the TDS standard.
% \begin{description}
% \item[\CTANinstall{install/macros/latex/contrib/oberdiek.tds.zip}]
% \end{description}
% \emph{TDS} refers to the standard ``A Directory Structure
% for \TeX\ Files'' (\CTAN{tds/tds.pdf}). Directories
% with \xfile{texmf} in their name are usually organized this way.
%
% \subsection{Bundle installation}
%
% \paragraph{Unpacking.} Unpack the \xfile{oberdiek.tds.zip} in the
% TDS tree (also known as \xfile{texmf} tree) of your choice.
% Example (linux):
% \begin{quote}
%   |unzip oberdiek.tds.zip -d ~/texmf|
% \end{quote}
%
% \paragraph{Script installation.}
% Check the directory \xfile{TDS:scripts/oberdiek/} for
% scripts that need further installation steps.
% Package \xpackage{attachfile2} comes with the Perl script
% \xfile{pdfatfi.pl} that should be installed in such a way
% that it can be called as \texttt{pdfatfi}.
% Example (linux):
% \begin{quote}
%   |chmod +x scripts/oberdiek/pdfatfi.pl|\\
%   |cp scripts/oberdiek/pdfatfi.pl /usr/local/bin/|
% \end{quote}
%
% \subsection{Package installation}
%
% \paragraph{Unpacking.} The \xfile{.dtx} file is a self-extracting
% \docstrip\ archive. The files are extracted by running the
% \xfile{.dtx} through \plainTeX:
% \begin{quote}
%   \verb|tex telprint.dtx|
% \end{quote}
%
% \paragraph{TDS.} Now the different files must be moved into
% the different directories in your installation TDS tree
% (also known as \xfile{texmf} tree):
% \begin{quote}
% \def\t{^^A
% \begin{tabular}{@{}>{\ttfamily}l@{ $\rightarrow$ }>{\ttfamily}l@{}}
%   telprint.sty & tex/generic/oberdiek/telprint.sty\\
%   telprint.pdf & doc/latex/oberdiek/telprint.pdf\\
%   test/telprint-test1.tex & doc/latex/oberdiek/test/telprint-test1.tex\\
%   telprint.dtx & source/latex/oberdiek/telprint.dtx\\
% \end{tabular}^^A
% }^^A
% \sbox0{\t}^^A
% \ifdim\wd0>\linewidth
%   \begingroup
%     \advance\linewidth by\leftmargin
%     \advance\linewidth by\rightmargin
%   \edef\x{\endgroup
%     \def\noexpand\lw{\the\linewidth}^^A
%   }\x
%   \def\lwbox{^^A
%     \leavevmode
%     \hbox to \linewidth{^^A
%       \kern-\leftmargin\relax
%       \hss
%       \usebox0
%       \hss
%       \kern-\rightmargin\relax
%     }^^A
%   }^^A
%   \ifdim\wd0>\lw
%     \sbox0{\small\t}^^A
%     \ifdim\wd0>\linewidth
%       \ifdim\wd0>\lw
%         \sbox0{\footnotesize\t}^^A
%         \ifdim\wd0>\linewidth
%           \ifdim\wd0>\lw
%             \sbox0{\scriptsize\t}^^A
%             \ifdim\wd0>\linewidth
%               \ifdim\wd0>\lw
%                 \sbox0{\tiny\t}^^A
%                 \ifdim\wd0>\linewidth
%                   \lwbox
%                 \else
%                   \usebox0
%                 \fi
%               \else
%                 \lwbox
%               \fi
%             \else
%               \usebox0
%             \fi
%           \else
%             \lwbox
%           \fi
%         \else
%           \usebox0
%         \fi
%       \else
%         \lwbox
%       \fi
%     \else
%       \usebox0
%     \fi
%   \else
%     \lwbox
%   \fi
% \else
%   \usebox0
% \fi
% \end{quote}
% If you have a \xfile{docstrip.cfg} that configures and enables \docstrip's
% TDS installing feature, then some files can already be in the right
% place, see the documentation of \docstrip.
%
% \subsection{Refresh file name databases}
%
% If your \TeX~distribution
% (\teTeX, \mikTeX, \dots) relies on file name databases, you must refresh
% these. For example, \teTeX\ users run \verb|texhash| or
% \verb|mktexlsr|.
%
% \subsection{Some details for the interested}
%
% \paragraph{Attached source.}
%
% The PDF documentation on CTAN also includes the
% \xfile{.dtx} source file. It can be extracted by
% AcrobatReader 6 or higher. Another option is \textsf{pdftk},
% e.g. unpack the file into the current directory:
% \begin{quote}
%   \verb|pdftk telprint.pdf unpack_files output .|
% \end{quote}
%
% \paragraph{Unpacking with \LaTeX.}
% The \xfile{.dtx} chooses its action depending on the format:
% \begin{description}
% \item[\plainTeX:] Run \docstrip\ and extract the files.
% \item[\LaTeX:] Generate the documentation.
% \end{description}
% If you insist on using \LaTeX\ for \docstrip\ (really,
% \docstrip\ does not need \LaTeX), then inform the autodetect routine
% about your intention:
% \begin{quote}
%   \verb|latex \let\install=y\input{telprint.dtx}|
% \end{quote}
% Do not forget to quote the argument according to the demands
% of your shell.
%
% \paragraph{Generating the documentation.}
% You can use both the \xfile{.dtx} or the \xfile{.drv} to generate
% the documentation. The process can be configured by the
% configuration file \xfile{ltxdoc.cfg}. For instance, put this
% line into this file, if you want to have A4 as paper format:
% \begin{quote}
%   \verb|\PassOptionsToClass{a4paper}{article}|
% \end{quote}
% An example follows how to generate the
% documentation with pdf\LaTeX:
% \begin{quote}
%\begin{verbatim}
%pdflatex telprint.dtx
%makeindex -s gind.ist telprint.idx
%pdflatex telprint.dtx
%makeindex -s gind.ist telprint.idx
%pdflatex telprint.dtx
%\end{verbatim}
% \end{quote}
%
% \begin{History}
%   \begin{Version}{1996/11/28 v1.0}
%   \item
%     Erste lauff\"ahige Version.
%   \item
%     Nur '-' und '/' als zul\"assige Sonderzeichen.
%   \end{Version}
%   \begin{Version}{1997/09/16 v1.1}
%   \item
%     Dokumentation und Kommentare (Posting in de.comp.text.tex).
%   \item
%     Erweiterung um Sonderzeichen '(', ')', '+', '\textasciitilde' und ' '.
%   \item
%     Trennungsverhinderung am 'hyphen'.
%   \end{Version}
%   \begin{Version}{1997/10/16 v1.2}
%   \item
%     Schutz vor wiederholtem Einlesen.
%   \item
%     Unter \LaTeXe\ Nutzung des \cs{DeclareRobustCommand}-Features.
%   \end{Version}
%   \begin{Version}{1997/12/09 v1.3}
%   \item
%     Tempor\"are Variable eingespart.
%   \item
%     Posted in newsgroup \xnewsgroup{de.comp.text.tex}:\\
%     \URL{``\link{Re: Generisches Markup f\"ur Telefonnummern?}''}^^A
%     {http://groups.google.com/group/de.comp.text.tex/msg/86b3a86140007309}
%   \end{Version}
%   \begin{Version}{2004/11/02 v1.4}
%   \item
%     Fehler in der Dokumentation korrigiert.
%   \end{Version}
%   \begin{Version}{2005/09/30 v1.5}
%   \item
%     Konfigurierbare Symbole: '/', '(', ')', '+' und '\textasciitilde'.
%   \end{Version}
%   \begin{Version}{2006/02/12 v1.6}
%   \item
%     LPPL 1.3.
%   \item
%     Kurze \"Ubersicht in Englisch.
%   \item
%     CTAN.
%   \end{Version}
%   \begin{Version}{2006/08/26 v1.7}
%   \item
%     New DTX framework.
%   \end{Version}
%   \begin{Version}{2007/04/11 v1.8}
%   \item
%     Line ends sanitized.
%   \end{Version}
%   \begin{Version}{2007/09/09 v1.9}
%   \item
%     Catcode section added.
%   \item
%     Missing docstrip tag added.
%   \end{Version}
%   \begin{Version}{2008/08/11 v1.10}
%   \item
%     Code is not changed.
%   \item
%     URLs updated.
%   \end{Version}
%   \begin{Version}{2016/05/16 v1.11}
%   \item
%     Documentation updates.
%   \end{Version}
% \end{History}
%
% \PrintIndex
%
% \Finale
\endinput

%        (quote the arguments according to the demands of your shell)
%
% Documentation:
%    (a) If telprint.drv is present:
%           latex telprint.drv
%    (b) Without telprint.drv:
%           latex telprint.dtx; ...
%    The class ltxdoc loads the configuration file ltxdoc.cfg
%    if available. Here you can specify further options, e.g.
%    use A4 as paper format:
%       \PassOptionsToClass{a4paper}{article}
%
%    Programm calls to get the documentation (example):
%       pdflatex telprint.dtx
%       makeindex -s gind.ist telprint.idx
%       pdflatex telprint.dtx
%       makeindex -s gind.ist telprint.idx
%       pdflatex telprint.dtx
%
% Installation:
%    TDS:tex/generic/oberdiek/telprint.sty
%    TDS:doc/latex/oberdiek/telprint.pdf
%    TDS:doc/latex/oberdiek/test/telprint-test1.tex
%    TDS:source/latex/oberdiek/telprint.dtx
%
%<*ignore>
\begingroup
  \catcode123=1 %
  \catcode125=2 %
  \def\x{LaTeX2e}%
\expandafter\endgroup
\ifcase 0\ifx\install y1\fi\expandafter
         \ifx\csname processbatchFile\endcsname\relax\else1\fi
         \ifx\fmtname\x\else 1\fi\relax
\else\csname fi\endcsname
%</ignore>
%<*install>
\input docstrip.tex
\Msg{************************************************************************}
\Msg{* Installation}
\Msg{* Package: telprint 2016/05/16 v1.11 Format German phone numbers (HO)}
\Msg{************************************************************************}

\keepsilent
\askforoverwritefalse

\let\MetaPrefix\relax
\preamble

This is a generated file.

Project: telprint
Version: 2016/05/16 v1.11

Copyright (C) 1996, 1997, 2004-2008 by
   Heiko Oberdiek <heiko.oberdiek at googlemail.com>

This work may be distributed and/or modified under the
conditions of the LaTeX Project Public License, either
version 1.3c of this license or (at your option) any later
version. This version of this license is in
   http://www.latex-project.org/lppl/lppl-1-3c.txt
and the latest version of this license is in
   http://www.latex-project.org/lppl.txt
and version 1.3 or later is part of all distributions of
LaTeX version 2005/12/01 or later.

This work has the LPPL maintenance status "maintained".

This Current Maintainer of this work is Heiko Oberdiek.

The Base Interpreter refers to any `TeX-Format',
because some files are installed in TDS:tex/generic//.

This work consists of the main source file telprint.dtx
and the derived files
   telprint.sty, telprint.pdf, telprint.ins, telprint.drv,
   telprint-test1.tex.

\endpreamble
\let\MetaPrefix\DoubleperCent

\generate{%
  \file{telprint.ins}{\from{telprint.dtx}{install}}%
  \file{telprint.drv}{\from{telprint.dtx}{driver}}%
  \usedir{tex/generic/oberdiek}%
  \file{telprint.sty}{\from{telprint.dtx}{package}}%
%  \usedir{doc/latex/oberdiek/test}%
%  \file{telprint-test1.tex}{\from{telprint.dtx}{test1}}%
  \nopreamble
  \nopostamble
%  \usedir{source/latex/oberdiek/catalogue}%
%  \file{telprint.xml}{\from{telprint.dtx}{catalogue}}%
}

\catcode32=13\relax% active space
\let =\space%
\Msg{************************************************************************}
\Msg{*}
\Msg{* To finish the installation you have to move the following}
\Msg{* file into a directory searched by TeX:}
\Msg{*}
\Msg{*     telprint.sty}
\Msg{*}
\Msg{* To produce the documentation run the file `telprint.drv'}
\Msg{* through LaTeX.}
\Msg{*}
\Msg{* Happy TeXing!}
\Msg{*}
\Msg{************************************************************************}

\endbatchfile
%</install>
%<*ignore>
\fi
%</ignore>
%<*driver>
\NeedsTeXFormat{LaTeX2e}
\ProvidesFile{telprint.drv}%
  [2016/05/16 v1.11 Format German phone numbers (HO)]%
\documentclass{ltxdoc}
\usepackage{holtxdoc}[2011/11/22]
\usepackage[ngerman,english]{babel}
\begin{document}
  \DocInput{telprint.dtx}%
\end{document}
%</driver>
% \fi
%
%
% \CharacterTable
%  {Upper-case    \A\B\C\D\E\F\G\H\I\J\K\L\M\N\O\P\Q\R\S\T\U\V\W\X\Y\Z
%   Lower-case    \a\b\c\d\e\f\g\h\i\j\k\l\m\n\o\p\q\r\s\t\u\v\w\x\y\z
%   Digits        \0\1\2\3\4\5\6\7\8\9
%   Exclamation   \!     Double quote  \"     Hash (number) \#
%   Dollar        \$     Percent       \%     Ampersand     \&
%   Acute accent  \'     Left paren    \(     Right paren   \)
%   Asterisk      \*     Plus          \+     Comma         \,
%   Minus         \-     Point         \.     Solidus       \/
%   Colon         \:     Semicolon     \;     Less than     \<
%   Equals        \=     Greater than  \>     Question mark \?
%   Commercial at \@     Left bracket  \[     Backslash     \\
%   Right bracket \]     Circumflex    \^     Underscore    \_
%   Grave accent  \`     Left brace    \{     Vertical bar  \|
%   Right brace   \}     Tilde         \~}
%
% \GetFileInfo{telprint.drv}
%
% \title{The \xpackage{telprint} package}
% \date{2016/05/16 v1.11}
% \author{Heiko Oberdiek\thanks
% {Please report any issues at https://github.com/ho-tex/oberdiek/issues}\\
% \xemail{heiko.oberdiek at googlemail.com}}
%
% \maketitle
%
% \begin{abstract}
% Package \xpackage{telprint} provides \cs{telprint} for formatting
% German phone numbers.
% \end{abstract}
%
% \tableofcontents
%
% \section{Documentation}
%
% \subsection{Introduction}
%
%            This is a very old package that I have written
%            to format phone numbers. It follows German
%            conventions and the documentation is mainly in German.
%
% \subsection{Short overview in English}
%
% \LaTeX:
% \begin{quote}
% |\usepackage{telprint}|\\
% |\telprint{123/456-789}|\\
% \end{quote}
% \plainTeX:
% \begin{quote}
%   |\input telprint.sty|\\
%   |\telprint{123/456-789}|
% \end{quote}
%
% \DescribeMacro\telprint
% |\telprint{...}| formats the explicitly given number.
%     Digits, spaces and some special characters
%     ('+', '/', '-', '(', ')', '\textasciitilde', ' ') are supported.
%     Numbers are divided into groups of two digits from the right.
% Examples:
% \begin{quote}
%     |\telprint{0761/12345}     ==> 07\,61/1\,23\,45|\\
%     |\telprint{01234/567-89}   ==> 0\,12\,34/5\,67\leavevmode\hbox{-}89|\\
%     |\telprint{+49 (6221) 297} ==> +49~(62\,21)~2\,97|
% \end{quote}
%
% \subsubsection{Configuration}
%
% The output of the symbols can be configured by
% \cs{telhyphen}, \cs{telslash}, \cs{telleftparen}, \cs{telrightparen},
% \cs{telplus}, \cs{teltilde}.
% Example:
% \begin{quote}
%   |\telslash{\,/\,}\\|
%   |\telprint{12/34} ==> 12\,/\,34|
% \end{quote}
%
% \DescribeMacro\telspace
% \cs{telspace} configures the space between digit groups.
%
% \DescribeMacro\telnumber
% \cs{telnumber} only formats a number in digit groups; special
%    characters are not recognized.
%
% \subsection{Documentation in German}
%
% \begin{otherlanguage*}{ngerman}
% \hyphenation{To-ken-ma-kros}
% \begin{itemize}
% \item \DescribeMacro\telprint |telprint#1|\\
%   Der eigentliche Anwenderbefehl zur formatierten Ausgabe von
%   Telefonnummern. Diese d\"urfen dabei nur als Zahlen angegeben
%   werden(, da sie tokenweise analysiert werden).
%   Als Trenn- oder Sonderzeichen werden unterst\"utzt:
%   '+', '/', '-', '(', ')', '\textasciitilde', ' '
%   Einfache Leerzeichen werden erkannt und durch Tilden ersetzt, um
%   Trennungen in der Telefonnummer zu verhindern. (Man beachte aus
%   gleichem Grunde die \cs{hbox} bei '-'.)
%   Beispiele:
%   \begin{quote}
%     |\telprint{0761/12345}     ==> 07\,61/1\,23\,45|\\
%     |\telprint{01234/567-89}   ==> 0\,12\,34/5\,67\leavevmode\hbox{-}89|\\
%     |\telprint{+49 (6221) 297} ==> +49~(62\,21)~2\,97|
%   \end{quote}
% \end{itemize}
% Der Rest enth\"alt eher Technisches:
% \begin{itemize}
% \item \DescribeMacro\telspace |\telspace#1|\\
%   Mit diesem Befehl wird der Abstand zwischen den Zifferngruppen
%   angegeben (Default: |\,|).
%   (Durch |\telspace{}| kann dieser zusaetzliche Abstand abgestellt
%   werden.)
% \item \DescribeMacro\telhyphen |\telhyphen#1|\\
%   Dieser Befehl gibt die Art des Bindestriches, wie er ausgegeben
%   werden soll. In der Eingabe darf jedoch nur der einfache
%   Bindestrich stehen:
%   |\telprint{123-45}|, jedoch NIE |\telprint{123--45}|!
%   Kopka-Bindestrich-Fans geben an:
%   |\telhyphen{\leavevmode\hbox{--}}|
% \item
%   \DescribeMacro{\telslash}
%   \DescribeMacro{\telleftparen}
%   \DescribeMacro{\telrightparen}
%   \DescribeMacro{\telplus}
%   \DescribeMacro{\teltilde}
%   |\telslash#1|, |\telleftparen#1|, |\telrightparen#1|, |\telplus#1|,
%   |\teltilde|\\
%   Diese Befehle konfigurieren die Zeichen '/', '(', ')', '+'
%   und '\textasciitilde'. Sie funktionieren analog zu \cs{telhyphen}.
% \item \DescribeMacro\telnumber |\telnumber#1|\\
%   Richtung interner Befehl: Er dient dazu, eine Zifferngruppe
%   in Zweiergruppen auszugeben.
%   Die einzelnen Zahlen werden im Tokenregister \cs{TELtoks}
%   gespeichert. Abwechselnd werden dabei zwischen zwei Token
%   (Zahlen) \cs{TELx} bzw. \cs{TELy} eingefuegt, abh\"angig von dem
%   wechselnden Wert von \cs{TELswitch}. Zum Schluss kann dann einfach
%   festgestellt werden ob die Nummer nun eine geradzahlige oder
%   ungeradzahlige Zahl von Ziffern aufwies. Dem entsprechend wird
%   \cs{TELx} mit dem Zusatzabstand belegt und \cs{TELy} leer definiert
%   oder umgekehrt. )
% \item |\TEL...| interne Befehle, Technisches:\\
%   \cs{TELsplit} dient zur Aufteilung einer zusammengesetzten
%   Telefonnummer (Vorwahl, Hauptnummer, Nebenstelle). In dieser
%   Implementation werden als Trennzeichen nur '/' und '-' erkannt.
%   Die einzelnen Bestandteile wie Vorwahl werden dann dem Befehl
%   \cs{telnumber} zur Formatierung uebergeben.
% \item Die Erkennung von einfachen Leerzeichen ist um einiges
%   schwieriger: Die Tokentrennung ueber Parameter |#1#2| funktioniert
%   nicht f\"ur einfache Leerzeichen, da TeX sie \emph{niemals} als
%   eigenst\"andige Argumente behandelt! (The TeXbook, Chapter 20,
%   p. 201)
%
%   (Anmerkung am Rande: Deshalb funktionieren die entsprechenden
%   Tokenmakros auf S. 149 des Buches "`Einf\"uhrung in TeX"' von
%   N. Schwarz (3. Aufl.) nicht, wenn im Tokenregister als erstes
%   ein einfaches Leerzeichen steht!)
% \end{itemize}
% \end{otherlanguage*}
%
% \StopEventually{
% }
%
% \section{Implementation}
%
%    \begin{macrocode}
%<*package>
%    \end{macrocode}
%
% \subsection{Reload check and package identification}
%    Reload check, especially if the package is not used with \LaTeX.
%    \begin{macrocode}
\begingroup\catcode61\catcode48\catcode32=10\relax%
  \catcode13=5 % ^^M
  \endlinechar=13 %
  \catcode35=6 % #
  \catcode39=12 % '
  \catcode44=12 % ,
  \catcode45=12 % -
  \catcode46=12 % .
  \catcode58=12 % :
  \catcode64=11 % @
  \catcode123=1 % {
  \catcode125=2 % }
  \expandafter\let\expandafter\x\csname ver@telprint.sty\endcsname
  \ifx\x\relax % plain-TeX, first loading
  \else
    \def\empty{}%
    \ifx\x\empty % LaTeX, first loading,
      % variable is initialized, but \ProvidesPackage not yet seen
    \else
      \expandafter\ifx\csname PackageInfo\endcsname\relax
        \def\x#1#2{%
          \immediate\write-1{Package #1 Info: #2.}%
        }%
      \else
        \def\x#1#2{\PackageInfo{#1}{#2, stopped}}%
      \fi
      \x{telprint}{The package is already loaded}%
      \aftergroup\endinput
    \fi
  \fi
\endgroup%
%    \end{macrocode}
%    Package identification:
%    \begin{macrocode}
\begingroup\catcode61\catcode48\catcode32=10\relax%
  \catcode13=5 % ^^M
  \endlinechar=13 %
  \catcode35=6 % #
  \catcode39=12 % '
  \catcode40=12 % (
  \catcode41=12 % )
  \catcode44=12 % ,
  \catcode45=12 % -
  \catcode46=12 % .
  \catcode47=12 % /
  \catcode58=12 % :
  \catcode64=11 % @
  \catcode91=12 % [
  \catcode93=12 % ]
  \catcode123=1 % {
  \catcode125=2 % }
  \expandafter\ifx\csname ProvidesPackage\endcsname\relax
    \def\x#1#2#3[#4]{\endgroup
      \immediate\write-1{Package: #3 #4}%
      \xdef#1{#4}%
    }%
  \else
    \def\x#1#2[#3]{\endgroup
      #2[{#3}]%
      \ifx#1\@undefined
        \xdef#1{#3}%
      \fi
      \ifx#1\relax
        \xdef#1{#3}%
      \fi
    }%
  \fi
\expandafter\x\csname ver@telprint.sty\endcsname
\ProvidesPackage{telprint}%
  [2016/05/16 v1.11 Format German phone numbers (HO)]%
%    \end{macrocode}
%
% \subsection{Catcodes}
%
%    \begin{macrocode}
\begingroup\catcode61\catcode48\catcode32=10\relax%
  \catcode13=5 % ^^M
  \endlinechar=13 %
  \catcode123=1 % {
  \catcode125=2 % }
  \catcode64=11 % @
  \def\x{\endgroup
    \expandafter\edef\csname TELAtEnd\endcsname{%
      \endlinechar=\the\endlinechar\relax
      \catcode13=\the\catcode13\relax
      \catcode32=\the\catcode32\relax
      \catcode35=\the\catcode35\relax
      \catcode61=\the\catcode61\relax
      \catcode64=\the\catcode64\relax
      \catcode123=\the\catcode123\relax
      \catcode125=\the\catcode125\relax
    }%
  }%
\x\catcode61\catcode48\catcode32=10\relax%
\catcode13=5 % ^^M
\endlinechar=13 %
\catcode35=6 % #
\catcode64=11 % @
\catcode123=1 % {
\catcode125=2 % }
\def\TMP@EnsureCode#1#2{%
  \edef\TELAtEnd{%
    \TELAtEnd
    \catcode#1=\the\catcode#1\relax
  }%
  \catcode#1=#2\relax
}
\TMP@EnsureCode{33}{12}% !
\TMP@EnsureCode{36}{3}% $
\TMP@EnsureCode{40}{12}% (
\TMP@EnsureCode{41}{12}% )
\TMP@EnsureCode{42}{12}% *
\TMP@EnsureCode{43}{12}% +
\TMP@EnsureCode{44}{12}% ,
\TMP@EnsureCode{45}{12}% -
\TMP@EnsureCode{46}{12}% .
\TMP@EnsureCode{47}{12}% /
\TMP@EnsureCode{91}{12}% [
\TMP@EnsureCode{93}{12}% ]
\TMP@EnsureCode{126}{13}% ~ (active)
\edef\TELAtEnd{\TELAtEnd\noexpand\endinput}
%    \end{macrocode}
%
% \subsection{Package macros}
%    \begin{macrocode}
\ifx\DeclareRobustCommand\UnDeFiNeD
  \def\DeclareRobustCommand*#1[1]{\def#1##1}%
  \def\TELreset{\let\DeclareRobustCommand=\UnDeFiNeD}%
  \input infwarerr.sty\relax
  \@PackageInfo{telprint}{%
    Macros are not robust!%
  }%
\else
  \let\TELreset=\relax
\fi
%    \end{macrocode}
%    \begin{macro}{\telspace}
%    \begin{macrocode}
\DeclareRobustCommand*{\telspace}[1]{\def\TELspace{#1}}
\telspace{{}$\,${}}
%    \end{macrocode}
%    \end{macro}
%    \begin{macro}{\telhyphen}
%    \begin{macrocode}
\DeclareRobustCommand*{\telhyphen}[1]{\def\TELhyphen{#1}}
\telhyphen{\leavevmode\hbox{-}}% \hbox zur Verhinderung der Trennung
%    \end{macrocode}
%    \end{macro}
%    \begin{macro}{\telslash}
%    \begin{macrocode}
\DeclareRobustCommand*{\telslash}[1]{\def\TELslash{#1}}
\telslash{/}%
%    \end{macrocode}
%    \end{macro}
%    \begin{macro}{\telleftparen}
%    \begin{macrocode}
\DeclareRobustCommand*{\telleftparen}[1]{\def\TELleftparen{#1}}
\telleftparen{(}%
%    \end{macrocode}
%    \end{macro}
%    \begin{macro}{\telrightparen}
%    \begin{macrocode}
\DeclareRobustCommand*{\telrightparen}[1]{\def\TELrightparen{#1}}
\telrightparen{)}%
%    \end{macrocode}
%    \end{macro}
%    \begin{macro}{\telplus}
%    \begin{macrocode}
\DeclareRobustCommand*{\telplus}[1]{\def\TELplus{#1}}
\telplus{+}%
%    \end{macrocode}
%    \end{macro}
%    \begin{macro}{\teltilde}
%    \begin{macrocode}
\DeclareRobustCommand*{\teltilde}[1]{\def\TELtilde{#1}}
\teltilde{~}%
%    \end{macrocode}
%    \end{macro}
%    \begin{macro}{\TELtoks}
%    \begin{macrocode}
\newtoks\TELtoks
%    \end{macrocode}
%    \end{macro}
%    \begin{macro}{\TELnumber}
%    \begin{macrocode}
\def\TELnumber#1#2\TELnumberEND{%
  \begingroup
  \def\0{#2}%
  \expandafter\endgroup
  \ifx\0\empty
    \TELtoks=\expandafter{\the\TELtoks#1}%
    \ifnum\TELswitch=0 %
      \def\TELx{\TELspace}\def\TELy{}%
    \else
      \def\TELx{}\def\TELy{\TELspace}%
    \fi
    \the\TELtoks
  \else
    \ifnum\TELswitch=0 %
      \TELtoks=\expandafter{\the\TELtoks#1\TELx}%
      \def\TELswitch{1}%
    \else
      \TELtoks=\expandafter{\the\TELtoks#1\TELy}%
      \def\TELswitch{0}%
    \fi
    \TELnumber#2\TELnumberEND
  \fi
}
%    \end{macrocode}
%    \end{macro}
%    \begin{macro}{\telnumber}
%    \begin{macrocode}
\DeclareRobustCommand*{\telnumber}[1]{%
  \TELtoks={}%
  \def\TELswitch{0}%
  \TELnumber#1{}\TELnumberEND
}
%    \end{macrocode}
%    \end{macro}
%    \begin{macro}{\TELsplit}
%    \begin{macrocode}
\def\TELsplit{\futurelet\TELfuture\TELdosplit}
%    \end{macrocode}
%    \end{macro}
%    \begin{macro}{\TELdosplit}
%    \begin{macrocode}
\def\TELdosplit#1#2\TELsplitEND
{%
  \def\TELsp{ }%
  \expandafter\ifx\TELsp\TELfuture
    \let\TELfuture=\relax
    \expandafter\telnumber\expandafter{\the\TELtoks}~%
    \telprint{#1#2}% Das Leerzeichen kann nicht #1 sein!
  \else
    \def\TELfirst{#1}%
    \ifx\TELfirst\empty
      \expandafter\telnumber\expandafter{\the\TELtoks}%
      \TELtoks={}%
    \else\if-\TELfirst
      \expandafter\telnumber\expandafter{\the\TELtoks}\TELhyphen
      \telprint{#2}%
    \else\if/\TELfirst
      \expandafter\telnumber\expandafter{\the\TELtoks}\TELslash
      \telprint{#2}%
    \else\if(\TELfirst
      \expandafter\telnumber\expandafter{\the\TELtoks}\TELleftparen
      \telprint{#2}%
    \else\if)\TELfirst
      \expandafter\telnumber\expandafter{\the\TELtoks}\TELrightparen
      \telprint{#2}%
    \else\if+\TELfirst
      \expandafter\telnumber\expandafter{\the\TELtoks}\TELplus
      \telprint{#2}%
    \else\def\TELtemp{~}\ifx\TELtemp\TELfirst
      \expandafter\telnumber\expandafter{\the\TELtoks}\TELtilde
      \telprint{#2}%
    \else
      \TELtoks=\expandafter{\the\TELtoks#1}%
      \TELsplit#2{}\TELsplitEND
    \fi\fi\fi\fi\fi\fi\fi
  \fi
}
%    \end{macrocode}
%    \end{macro}
%    \begin{macro}{\telprint}
%    \begin{macrocode}
\DeclareRobustCommand*{\telprint}[1]{%
  \TELtoks={}%
  \TELsplit#1{}\TELsplitEND
}
%    \end{macrocode}
%    \end{macro}
%    \begin{macrocode}
\TELreset\let\TELreset=\UnDeFiNeD
%    \end{macrocode}
%
%    \begin{macrocode}
\TELAtEnd%
%</package>
%    \end{macrocode}
%
% \section{Test}
%
% \subsection{Catcode checks for loading}
%
%    \begin{macrocode}
%<*test1>
%    \end{macrocode}
%    \begin{macrocode}
\catcode`\{=1 %
\catcode`\}=2 %
\catcode`\#=6 %
\catcode`\@=11 %
\expandafter\ifx\csname count@\endcsname\relax
  \countdef\count@=255 %
\fi
\expandafter\ifx\csname @gobble\endcsname\relax
  \long\def\@gobble#1{}%
\fi
\expandafter\ifx\csname @firstofone\endcsname\relax
  \long\def\@firstofone#1{#1}%
\fi
\expandafter\ifx\csname loop\endcsname\relax
  \expandafter\@firstofone
\else
  \expandafter\@gobble
\fi
{%
  \def\loop#1\repeat{%
    \def\body{#1}%
    \iterate
  }%
  \def\iterate{%
    \body
      \let\next\iterate
    \else
      \let\next\relax
    \fi
    \next
  }%
  \let\repeat=\fi
}%
\def\RestoreCatcodes{}
\count@=0 %
\loop
  \edef\RestoreCatcodes{%
    \RestoreCatcodes
    \catcode\the\count@=\the\catcode\count@\relax
  }%
\ifnum\count@<255 %
  \advance\count@ 1 %
\repeat

\def\RangeCatcodeInvalid#1#2{%
  \count@=#1\relax
  \loop
    \catcode\count@=15 %
  \ifnum\count@<#2\relax
    \advance\count@ 1 %
  \repeat
}
\def\RangeCatcodeCheck#1#2#3{%
  \count@=#1\relax
  \loop
    \ifnum#3=\catcode\count@
    \else
      \errmessage{%
        Character \the\count@\space
        with wrong catcode \the\catcode\count@\space
        instead of \number#3%
      }%
    \fi
  \ifnum\count@<#2\relax
    \advance\count@ 1 %
  \repeat
}
\def\space{ }
\expandafter\ifx\csname LoadCommand\endcsname\relax
  \def\LoadCommand{\input telprint.sty\relax}%
\fi
\def\Test{%
  \RangeCatcodeInvalid{0}{47}%
  \RangeCatcodeInvalid{58}{64}%
  \RangeCatcodeInvalid{91}{96}%
  \RangeCatcodeInvalid{123}{255}%
  \catcode`\@=12 %
  \catcode`\\=0 %
  \catcode`\%=14 %
  \LoadCommand
  \RangeCatcodeCheck{0}{36}{15}%
  \RangeCatcodeCheck{37}{37}{14}%
  \RangeCatcodeCheck{38}{47}{15}%
  \RangeCatcodeCheck{48}{57}{12}%
  \RangeCatcodeCheck{58}{63}{15}%
  \RangeCatcodeCheck{64}{64}{12}%
  \RangeCatcodeCheck{65}{90}{11}%
  \RangeCatcodeCheck{91}{91}{15}%
  \RangeCatcodeCheck{92}{92}{0}%
  \RangeCatcodeCheck{93}{96}{15}%
  \RangeCatcodeCheck{97}{122}{11}%
  \RangeCatcodeCheck{123}{255}{15}%
  \RestoreCatcodes
}
\Test
\csname @@end\endcsname
\end
%    \end{macrocode}
%    \begin{macrocode}
%</test1>
%    \end{macrocode}
%
% \section{Installation}
%
% \subsection{Download}
%
% \paragraph{Package.} This package is available on
% CTAN\footnote{\url{https://ctan.org/pkg/telprint}}:
% \begin{description}
% \item[\CTAN{macros/latex/contrib/oberdiek/telprint.dtx}] The source file.
% \item[\CTAN{macros/latex/contrib/oberdiek/telprint.pdf}] Documentation.
% \end{description}
%
%
% \paragraph{Bundle.} All the packages of the bundle `oberdiek'
% are also available in a TDS compliant ZIP archive. There
% the packages are already unpacked and the documentation files
% are generated. The files and directories obey the TDS standard.
% \begin{description}
% \item[\CTANinstall{install/macros/latex/contrib/oberdiek.tds.zip}]
% \end{description}
% \emph{TDS} refers to the standard ``A Directory Structure
% for \TeX\ Files'' (\CTAN{tds/tds.pdf}). Directories
% with \xfile{texmf} in their name are usually organized this way.
%
% \subsection{Bundle installation}
%
% \paragraph{Unpacking.} Unpack the \xfile{oberdiek.tds.zip} in the
% TDS tree (also known as \xfile{texmf} tree) of your choice.
% Example (linux):
% \begin{quote}
%   |unzip oberdiek.tds.zip -d ~/texmf|
% \end{quote}
%
% \paragraph{Script installation.}
% Check the directory \xfile{TDS:scripts/oberdiek/} for
% scripts that need further installation steps.
% Package \xpackage{attachfile2} comes with the Perl script
% \xfile{pdfatfi.pl} that should be installed in such a way
% that it can be called as \texttt{pdfatfi}.
% Example (linux):
% \begin{quote}
%   |chmod +x scripts/oberdiek/pdfatfi.pl|\\
%   |cp scripts/oberdiek/pdfatfi.pl /usr/local/bin/|
% \end{quote}
%
% \subsection{Package installation}
%
% \paragraph{Unpacking.} The \xfile{.dtx} file is a self-extracting
% \docstrip\ archive. The files are extracted by running the
% \xfile{.dtx} through \plainTeX:
% \begin{quote}
%   \verb|tex telprint.dtx|
% \end{quote}
%
% \paragraph{TDS.} Now the different files must be moved into
% the different directories in your installation TDS tree
% (also known as \xfile{texmf} tree):
% \begin{quote}
% \def\t{^^A
% \begin{tabular}{@{}>{\ttfamily}l@{ $\rightarrow$ }>{\ttfamily}l@{}}
%   telprint.sty & tex/generic/oberdiek/telprint.sty\\
%   telprint.pdf & doc/latex/oberdiek/telprint.pdf\\
%   test/telprint-test1.tex & doc/latex/oberdiek/test/telprint-test1.tex\\
%   telprint.dtx & source/latex/oberdiek/telprint.dtx\\
% \end{tabular}^^A
% }^^A
% \sbox0{\t}^^A
% \ifdim\wd0>\linewidth
%   \begingroup
%     \advance\linewidth by\leftmargin
%     \advance\linewidth by\rightmargin
%   \edef\x{\endgroup
%     \def\noexpand\lw{\the\linewidth}^^A
%   }\x
%   \def\lwbox{^^A
%     \leavevmode
%     \hbox to \linewidth{^^A
%       \kern-\leftmargin\relax
%       \hss
%       \usebox0
%       \hss
%       \kern-\rightmargin\relax
%     }^^A
%   }^^A
%   \ifdim\wd0>\lw
%     \sbox0{\small\t}^^A
%     \ifdim\wd0>\linewidth
%       \ifdim\wd0>\lw
%         \sbox0{\footnotesize\t}^^A
%         \ifdim\wd0>\linewidth
%           \ifdim\wd0>\lw
%             \sbox0{\scriptsize\t}^^A
%             \ifdim\wd0>\linewidth
%               \ifdim\wd0>\lw
%                 \sbox0{\tiny\t}^^A
%                 \ifdim\wd0>\linewidth
%                   \lwbox
%                 \else
%                   \usebox0
%                 \fi
%               \else
%                 \lwbox
%               \fi
%             \else
%               \usebox0
%             \fi
%           \else
%             \lwbox
%           \fi
%         \else
%           \usebox0
%         \fi
%       \else
%         \lwbox
%       \fi
%     \else
%       \usebox0
%     \fi
%   \else
%     \lwbox
%   \fi
% \else
%   \usebox0
% \fi
% \end{quote}
% If you have a \xfile{docstrip.cfg} that configures and enables \docstrip's
% TDS installing feature, then some files can already be in the right
% place, see the documentation of \docstrip.
%
% \subsection{Refresh file name databases}
%
% If your \TeX~distribution
% (\teTeX, \mikTeX, \dots) relies on file name databases, you must refresh
% these. For example, \teTeX\ users run \verb|texhash| or
% \verb|mktexlsr|.
%
% \subsection{Some details for the interested}
%
% \paragraph{Attached source.}
%
% The PDF documentation on CTAN also includes the
% \xfile{.dtx} source file. It can be extracted by
% AcrobatReader 6 or higher. Another option is \textsf{pdftk},
% e.g. unpack the file into the current directory:
% \begin{quote}
%   \verb|pdftk telprint.pdf unpack_files output .|
% \end{quote}
%
% \paragraph{Unpacking with \LaTeX.}
% The \xfile{.dtx} chooses its action depending on the format:
% \begin{description}
% \item[\plainTeX:] Run \docstrip\ and extract the files.
% \item[\LaTeX:] Generate the documentation.
% \end{description}
% If you insist on using \LaTeX\ for \docstrip\ (really,
% \docstrip\ does not need \LaTeX), then inform the autodetect routine
% about your intention:
% \begin{quote}
%   \verb|latex \let\install=y% \iffalse meta-comment
%
% File: telprint.dtx
% Version: 2016/05/16 v1.11
% Info: Format German phone numbers
%
% Copyright (C) 1996, 1997, 2004-2008 by
%    Heiko Oberdiek <heiko.oberdiek at googlemail.com>
%    2016
%    https://github.com/ho-tex/oberdiek/issues
%
% This work may be distributed and/or modified under the
% conditions of the LaTeX Project Public License, either
% version 1.3c of this license or (at your option) any later
% version. This version of this license is in
%    http://www.latex-project.org/lppl/lppl-1-3c.txt
% and the latest version of this license is in
%    http://www.latex-project.org/lppl.txt
% and version 1.3 or later is part of all distributions of
% LaTeX version 2005/12/01 or later.
%
% This work has the LPPL maintenance status "maintained".
%
% This Current Maintainer of this work is Heiko Oberdiek.
%
% The Base Interpreter refers to any `TeX-Format',
% because some files are installed in TDS:tex/generic//.
%
% This work consists of the main source file telprint.dtx
% and the derived files
%    telprint.sty, telprint.pdf, telprint.ins, telprint.drv,
%    telprint-test1.tex.
%
% Distribution:
%    CTAN:macros/latex/contrib/oberdiek/telprint.dtx
%    CTAN:macros/latex/contrib/oberdiek/telprint.pdf
%
% Unpacking:
%    (a) If telprint.ins is present:
%           tex telprint.ins
%    (b) Without telprint.ins:
%           tex telprint.dtx
%    (c) If you insist on using LaTeX
%           latex \let\install=y\input{telprint.dtx}
%        (quote the arguments according to the demands of your shell)
%
% Documentation:
%    (a) If telprint.drv is present:
%           latex telprint.drv
%    (b) Without telprint.drv:
%           latex telprint.dtx; ...
%    The class ltxdoc loads the configuration file ltxdoc.cfg
%    if available. Here you can specify further options, e.g.
%    use A4 as paper format:
%       \PassOptionsToClass{a4paper}{article}
%
%    Programm calls to get the documentation (example):
%       pdflatex telprint.dtx
%       makeindex -s gind.ist telprint.idx
%       pdflatex telprint.dtx
%       makeindex -s gind.ist telprint.idx
%       pdflatex telprint.dtx
%
% Installation:
%    TDS:tex/generic/oberdiek/telprint.sty
%    TDS:doc/latex/oberdiek/telprint.pdf
%    TDS:doc/latex/oberdiek/test/telprint-test1.tex
%    TDS:source/latex/oberdiek/telprint.dtx
%
%<*ignore>
\begingroup
  \catcode123=1 %
  \catcode125=2 %
  \def\x{LaTeX2e}%
\expandafter\endgroup
\ifcase 0\ifx\install y1\fi\expandafter
         \ifx\csname processbatchFile\endcsname\relax\else1\fi
         \ifx\fmtname\x\else 1\fi\relax
\else\csname fi\endcsname
%</ignore>
%<*install>
\input docstrip.tex
\Msg{************************************************************************}
\Msg{* Installation}
\Msg{* Package: telprint 2016/05/16 v1.11 Format German phone numbers (HO)}
\Msg{************************************************************************}

\keepsilent
\askforoverwritefalse

\let\MetaPrefix\relax
\preamble

This is a generated file.

Project: telprint
Version: 2016/05/16 v1.11

Copyright (C) 1996, 1997, 2004-2008 by
   Heiko Oberdiek <heiko.oberdiek at googlemail.com>

This work may be distributed and/or modified under the
conditions of the LaTeX Project Public License, either
version 1.3c of this license or (at your option) any later
version. This version of this license is in
   http://www.latex-project.org/lppl/lppl-1-3c.txt
and the latest version of this license is in
   http://www.latex-project.org/lppl.txt
and version 1.3 or later is part of all distributions of
LaTeX version 2005/12/01 or later.

This work has the LPPL maintenance status "maintained".

This Current Maintainer of this work is Heiko Oberdiek.

The Base Interpreter refers to any `TeX-Format',
because some files are installed in TDS:tex/generic//.

This work consists of the main source file telprint.dtx
and the derived files
   telprint.sty, telprint.pdf, telprint.ins, telprint.drv,
   telprint-test1.tex.

\endpreamble
\let\MetaPrefix\DoubleperCent

\generate{%
  \file{telprint.ins}{\from{telprint.dtx}{install}}%
  \file{telprint.drv}{\from{telprint.dtx}{driver}}%
  \usedir{tex/generic/oberdiek}%
  \file{telprint.sty}{\from{telprint.dtx}{package}}%
%  \usedir{doc/latex/oberdiek/test}%
%  \file{telprint-test1.tex}{\from{telprint.dtx}{test1}}%
  \nopreamble
  \nopostamble
%  \usedir{source/latex/oberdiek/catalogue}%
%  \file{telprint.xml}{\from{telprint.dtx}{catalogue}}%
}

\catcode32=13\relax% active space
\let =\space%
\Msg{************************************************************************}
\Msg{*}
\Msg{* To finish the installation you have to move the following}
\Msg{* file into a directory searched by TeX:}
\Msg{*}
\Msg{*     telprint.sty}
\Msg{*}
\Msg{* To produce the documentation run the file `telprint.drv'}
\Msg{* through LaTeX.}
\Msg{*}
\Msg{* Happy TeXing!}
\Msg{*}
\Msg{************************************************************************}

\endbatchfile
%</install>
%<*ignore>
\fi
%</ignore>
%<*driver>
\NeedsTeXFormat{LaTeX2e}
\ProvidesFile{telprint.drv}%
  [2016/05/16 v1.11 Format German phone numbers (HO)]%
\documentclass{ltxdoc}
\usepackage{holtxdoc}[2011/11/22]
\usepackage[ngerman,english]{babel}
\begin{document}
  \DocInput{telprint.dtx}%
\end{document}
%</driver>
% \fi
%
%
% \CharacterTable
%  {Upper-case    \A\B\C\D\E\F\G\H\I\J\K\L\M\N\O\P\Q\R\S\T\U\V\W\X\Y\Z
%   Lower-case    \a\b\c\d\e\f\g\h\i\j\k\l\m\n\o\p\q\r\s\t\u\v\w\x\y\z
%   Digits        \0\1\2\3\4\5\6\7\8\9
%   Exclamation   \!     Double quote  \"     Hash (number) \#
%   Dollar        \$     Percent       \%     Ampersand     \&
%   Acute accent  \'     Left paren    \(     Right paren   \)
%   Asterisk      \*     Plus          \+     Comma         \,
%   Minus         \-     Point         \.     Solidus       \/
%   Colon         \:     Semicolon     \;     Less than     \<
%   Equals        \=     Greater than  \>     Question mark \?
%   Commercial at \@     Left bracket  \[     Backslash     \\
%   Right bracket \]     Circumflex    \^     Underscore    \_
%   Grave accent  \`     Left brace    \{     Vertical bar  \|
%   Right brace   \}     Tilde         \~}
%
% \GetFileInfo{telprint.drv}
%
% \title{The \xpackage{telprint} package}
% \date{2016/05/16 v1.11}
% \author{Heiko Oberdiek\thanks
% {Please report any issues at https://github.com/ho-tex/oberdiek/issues}\\
% \xemail{heiko.oberdiek at googlemail.com}}
%
% \maketitle
%
% \begin{abstract}
% Package \xpackage{telprint} provides \cs{telprint} for formatting
% German phone numbers.
% \end{abstract}
%
% \tableofcontents
%
% \section{Documentation}
%
% \subsection{Introduction}
%
%            This is a very old package that I have written
%            to format phone numbers. It follows German
%            conventions and the documentation is mainly in German.
%
% \subsection{Short overview in English}
%
% \LaTeX:
% \begin{quote}
% |\usepackage{telprint}|\\
% |\telprint{123/456-789}|\\
% \end{quote}
% \plainTeX:
% \begin{quote}
%   |\input telprint.sty|\\
%   |\telprint{123/456-789}|
% \end{quote}
%
% \DescribeMacro\telprint
% |\telprint{...}| formats the explicitly given number.
%     Digits, spaces and some special characters
%     ('+', '/', '-', '(', ')', '\textasciitilde', ' ') are supported.
%     Numbers are divided into groups of two digits from the right.
% Examples:
% \begin{quote}
%     |\telprint{0761/12345}     ==> 07\,61/1\,23\,45|\\
%     |\telprint{01234/567-89}   ==> 0\,12\,34/5\,67\leavevmode\hbox{-}89|\\
%     |\telprint{+49 (6221) 297} ==> +49~(62\,21)~2\,97|
% \end{quote}
%
% \subsubsection{Configuration}
%
% The output of the symbols can be configured by
% \cs{telhyphen}, \cs{telslash}, \cs{telleftparen}, \cs{telrightparen},
% \cs{telplus}, \cs{teltilde}.
% Example:
% \begin{quote}
%   |\telslash{\,/\,}\\|
%   |\telprint{12/34} ==> 12\,/\,34|
% \end{quote}
%
% \DescribeMacro\telspace
% \cs{telspace} configures the space between digit groups.
%
% \DescribeMacro\telnumber
% \cs{telnumber} only formats a number in digit groups; special
%    characters are not recognized.
%
% \subsection{Documentation in German}
%
% \begin{otherlanguage*}{ngerman}
% \hyphenation{To-ken-ma-kros}
% \begin{itemize}
% \item \DescribeMacro\telprint |telprint#1|\\
%   Der eigentliche Anwenderbefehl zur formatierten Ausgabe von
%   Telefonnummern. Diese d\"urfen dabei nur als Zahlen angegeben
%   werden(, da sie tokenweise analysiert werden).
%   Als Trenn- oder Sonderzeichen werden unterst\"utzt:
%   '+', '/', '-', '(', ')', '\textasciitilde', ' '
%   Einfache Leerzeichen werden erkannt und durch Tilden ersetzt, um
%   Trennungen in der Telefonnummer zu verhindern. (Man beachte aus
%   gleichem Grunde die \cs{hbox} bei '-'.)
%   Beispiele:
%   \begin{quote}
%     |\telprint{0761/12345}     ==> 07\,61/1\,23\,45|\\
%     |\telprint{01234/567-89}   ==> 0\,12\,34/5\,67\leavevmode\hbox{-}89|\\
%     |\telprint{+49 (6221) 297} ==> +49~(62\,21)~2\,97|
%   \end{quote}
% \end{itemize}
% Der Rest enth\"alt eher Technisches:
% \begin{itemize}
% \item \DescribeMacro\telspace |\telspace#1|\\
%   Mit diesem Befehl wird der Abstand zwischen den Zifferngruppen
%   angegeben (Default: |\,|).
%   (Durch |\telspace{}| kann dieser zusaetzliche Abstand abgestellt
%   werden.)
% \item \DescribeMacro\telhyphen |\telhyphen#1|\\
%   Dieser Befehl gibt die Art des Bindestriches, wie er ausgegeben
%   werden soll. In der Eingabe darf jedoch nur der einfache
%   Bindestrich stehen:
%   |\telprint{123-45}|, jedoch NIE |\telprint{123--45}|!
%   Kopka-Bindestrich-Fans geben an:
%   |\telhyphen{\leavevmode\hbox{--}}|
% \item
%   \DescribeMacro{\telslash}
%   \DescribeMacro{\telleftparen}
%   \DescribeMacro{\telrightparen}
%   \DescribeMacro{\telplus}
%   \DescribeMacro{\teltilde}
%   |\telslash#1|, |\telleftparen#1|, |\telrightparen#1|, |\telplus#1|,
%   |\teltilde|\\
%   Diese Befehle konfigurieren die Zeichen '/', '(', ')', '+'
%   und '\textasciitilde'. Sie funktionieren analog zu \cs{telhyphen}.
% \item \DescribeMacro\telnumber |\telnumber#1|\\
%   Richtung interner Befehl: Er dient dazu, eine Zifferngruppe
%   in Zweiergruppen auszugeben.
%   Die einzelnen Zahlen werden im Tokenregister \cs{TELtoks}
%   gespeichert. Abwechselnd werden dabei zwischen zwei Token
%   (Zahlen) \cs{TELx} bzw. \cs{TELy} eingefuegt, abh\"angig von dem
%   wechselnden Wert von \cs{TELswitch}. Zum Schluss kann dann einfach
%   festgestellt werden ob die Nummer nun eine geradzahlige oder
%   ungeradzahlige Zahl von Ziffern aufwies. Dem entsprechend wird
%   \cs{TELx} mit dem Zusatzabstand belegt und \cs{TELy} leer definiert
%   oder umgekehrt. )
% \item |\TEL...| interne Befehle, Technisches:\\
%   \cs{TELsplit} dient zur Aufteilung einer zusammengesetzten
%   Telefonnummer (Vorwahl, Hauptnummer, Nebenstelle). In dieser
%   Implementation werden als Trennzeichen nur '/' und '-' erkannt.
%   Die einzelnen Bestandteile wie Vorwahl werden dann dem Befehl
%   \cs{telnumber} zur Formatierung uebergeben.
% \item Die Erkennung von einfachen Leerzeichen ist um einiges
%   schwieriger: Die Tokentrennung ueber Parameter |#1#2| funktioniert
%   nicht f\"ur einfache Leerzeichen, da TeX sie \emph{niemals} als
%   eigenst\"andige Argumente behandelt! (The TeXbook, Chapter 20,
%   p. 201)
%
%   (Anmerkung am Rande: Deshalb funktionieren die entsprechenden
%   Tokenmakros auf S. 149 des Buches "`Einf\"uhrung in TeX"' von
%   N. Schwarz (3. Aufl.) nicht, wenn im Tokenregister als erstes
%   ein einfaches Leerzeichen steht!)
% \end{itemize}
% \end{otherlanguage*}
%
% \StopEventually{
% }
%
% \section{Implementation}
%
%    \begin{macrocode}
%<*package>
%    \end{macrocode}
%
% \subsection{Reload check and package identification}
%    Reload check, especially if the package is not used with \LaTeX.
%    \begin{macrocode}
\begingroup\catcode61\catcode48\catcode32=10\relax%
  \catcode13=5 % ^^M
  \endlinechar=13 %
  \catcode35=6 % #
  \catcode39=12 % '
  \catcode44=12 % ,
  \catcode45=12 % -
  \catcode46=12 % .
  \catcode58=12 % :
  \catcode64=11 % @
  \catcode123=1 % {
  \catcode125=2 % }
  \expandafter\let\expandafter\x\csname ver@telprint.sty\endcsname
  \ifx\x\relax % plain-TeX, first loading
  \else
    \def\empty{}%
    \ifx\x\empty % LaTeX, first loading,
      % variable is initialized, but \ProvidesPackage not yet seen
    \else
      \expandafter\ifx\csname PackageInfo\endcsname\relax
        \def\x#1#2{%
          \immediate\write-1{Package #1 Info: #2.}%
        }%
      \else
        \def\x#1#2{\PackageInfo{#1}{#2, stopped}}%
      \fi
      \x{telprint}{The package is already loaded}%
      \aftergroup\endinput
    \fi
  \fi
\endgroup%
%    \end{macrocode}
%    Package identification:
%    \begin{macrocode}
\begingroup\catcode61\catcode48\catcode32=10\relax%
  \catcode13=5 % ^^M
  \endlinechar=13 %
  \catcode35=6 % #
  \catcode39=12 % '
  \catcode40=12 % (
  \catcode41=12 % )
  \catcode44=12 % ,
  \catcode45=12 % -
  \catcode46=12 % .
  \catcode47=12 % /
  \catcode58=12 % :
  \catcode64=11 % @
  \catcode91=12 % [
  \catcode93=12 % ]
  \catcode123=1 % {
  \catcode125=2 % }
  \expandafter\ifx\csname ProvidesPackage\endcsname\relax
    \def\x#1#2#3[#4]{\endgroup
      \immediate\write-1{Package: #3 #4}%
      \xdef#1{#4}%
    }%
  \else
    \def\x#1#2[#3]{\endgroup
      #2[{#3}]%
      \ifx#1\@undefined
        \xdef#1{#3}%
      \fi
      \ifx#1\relax
        \xdef#1{#3}%
      \fi
    }%
  \fi
\expandafter\x\csname ver@telprint.sty\endcsname
\ProvidesPackage{telprint}%
  [2016/05/16 v1.11 Format German phone numbers (HO)]%
%    \end{macrocode}
%
% \subsection{Catcodes}
%
%    \begin{macrocode}
\begingroup\catcode61\catcode48\catcode32=10\relax%
  \catcode13=5 % ^^M
  \endlinechar=13 %
  \catcode123=1 % {
  \catcode125=2 % }
  \catcode64=11 % @
  \def\x{\endgroup
    \expandafter\edef\csname TELAtEnd\endcsname{%
      \endlinechar=\the\endlinechar\relax
      \catcode13=\the\catcode13\relax
      \catcode32=\the\catcode32\relax
      \catcode35=\the\catcode35\relax
      \catcode61=\the\catcode61\relax
      \catcode64=\the\catcode64\relax
      \catcode123=\the\catcode123\relax
      \catcode125=\the\catcode125\relax
    }%
  }%
\x\catcode61\catcode48\catcode32=10\relax%
\catcode13=5 % ^^M
\endlinechar=13 %
\catcode35=6 % #
\catcode64=11 % @
\catcode123=1 % {
\catcode125=2 % }
\def\TMP@EnsureCode#1#2{%
  \edef\TELAtEnd{%
    \TELAtEnd
    \catcode#1=\the\catcode#1\relax
  }%
  \catcode#1=#2\relax
}
\TMP@EnsureCode{33}{12}% !
\TMP@EnsureCode{36}{3}% $
\TMP@EnsureCode{40}{12}% (
\TMP@EnsureCode{41}{12}% )
\TMP@EnsureCode{42}{12}% *
\TMP@EnsureCode{43}{12}% +
\TMP@EnsureCode{44}{12}% ,
\TMP@EnsureCode{45}{12}% -
\TMP@EnsureCode{46}{12}% .
\TMP@EnsureCode{47}{12}% /
\TMP@EnsureCode{91}{12}% [
\TMP@EnsureCode{93}{12}% ]
\TMP@EnsureCode{126}{13}% ~ (active)
\edef\TELAtEnd{\TELAtEnd\noexpand\endinput}
%    \end{macrocode}
%
% \subsection{Package macros}
%    \begin{macrocode}
\ifx\DeclareRobustCommand\UnDeFiNeD
  \def\DeclareRobustCommand*#1[1]{\def#1##1}%
  \def\TELreset{\let\DeclareRobustCommand=\UnDeFiNeD}%
  \input infwarerr.sty\relax
  \@PackageInfo{telprint}{%
    Macros are not robust!%
  }%
\else
  \let\TELreset=\relax
\fi
%    \end{macrocode}
%    \begin{macro}{\telspace}
%    \begin{macrocode}
\DeclareRobustCommand*{\telspace}[1]{\def\TELspace{#1}}
\telspace{{}$\,${}}
%    \end{macrocode}
%    \end{macro}
%    \begin{macro}{\telhyphen}
%    \begin{macrocode}
\DeclareRobustCommand*{\telhyphen}[1]{\def\TELhyphen{#1}}
\telhyphen{\leavevmode\hbox{-}}% \hbox zur Verhinderung der Trennung
%    \end{macrocode}
%    \end{macro}
%    \begin{macro}{\telslash}
%    \begin{macrocode}
\DeclareRobustCommand*{\telslash}[1]{\def\TELslash{#1}}
\telslash{/}%
%    \end{macrocode}
%    \end{macro}
%    \begin{macro}{\telleftparen}
%    \begin{macrocode}
\DeclareRobustCommand*{\telleftparen}[1]{\def\TELleftparen{#1}}
\telleftparen{(}%
%    \end{macrocode}
%    \end{macro}
%    \begin{macro}{\telrightparen}
%    \begin{macrocode}
\DeclareRobustCommand*{\telrightparen}[1]{\def\TELrightparen{#1}}
\telrightparen{)}%
%    \end{macrocode}
%    \end{macro}
%    \begin{macro}{\telplus}
%    \begin{macrocode}
\DeclareRobustCommand*{\telplus}[1]{\def\TELplus{#1}}
\telplus{+}%
%    \end{macrocode}
%    \end{macro}
%    \begin{macro}{\teltilde}
%    \begin{macrocode}
\DeclareRobustCommand*{\teltilde}[1]{\def\TELtilde{#1}}
\teltilde{~}%
%    \end{macrocode}
%    \end{macro}
%    \begin{macro}{\TELtoks}
%    \begin{macrocode}
\newtoks\TELtoks
%    \end{macrocode}
%    \end{macro}
%    \begin{macro}{\TELnumber}
%    \begin{macrocode}
\def\TELnumber#1#2\TELnumberEND{%
  \begingroup
  \def\0{#2}%
  \expandafter\endgroup
  \ifx\0\empty
    \TELtoks=\expandafter{\the\TELtoks#1}%
    \ifnum\TELswitch=0 %
      \def\TELx{\TELspace}\def\TELy{}%
    \else
      \def\TELx{}\def\TELy{\TELspace}%
    \fi
    \the\TELtoks
  \else
    \ifnum\TELswitch=0 %
      \TELtoks=\expandafter{\the\TELtoks#1\TELx}%
      \def\TELswitch{1}%
    \else
      \TELtoks=\expandafter{\the\TELtoks#1\TELy}%
      \def\TELswitch{0}%
    \fi
    \TELnumber#2\TELnumberEND
  \fi
}
%    \end{macrocode}
%    \end{macro}
%    \begin{macro}{\telnumber}
%    \begin{macrocode}
\DeclareRobustCommand*{\telnumber}[1]{%
  \TELtoks={}%
  \def\TELswitch{0}%
  \TELnumber#1{}\TELnumberEND
}
%    \end{macrocode}
%    \end{macro}
%    \begin{macro}{\TELsplit}
%    \begin{macrocode}
\def\TELsplit{\futurelet\TELfuture\TELdosplit}
%    \end{macrocode}
%    \end{macro}
%    \begin{macro}{\TELdosplit}
%    \begin{macrocode}
\def\TELdosplit#1#2\TELsplitEND
{%
  \def\TELsp{ }%
  \expandafter\ifx\TELsp\TELfuture
    \let\TELfuture=\relax
    \expandafter\telnumber\expandafter{\the\TELtoks}~%
    \telprint{#1#2}% Das Leerzeichen kann nicht #1 sein!
  \else
    \def\TELfirst{#1}%
    \ifx\TELfirst\empty
      \expandafter\telnumber\expandafter{\the\TELtoks}%
      \TELtoks={}%
    \else\if-\TELfirst
      \expandafter\telnumber\expandafter{\the\TELtoks}\TELhyphen
      \telprint{#2}%
    \else\if/\TELfirst
      \expandafter\telnumber\expandafter{\the\TELtoks}\TELslash
      \telprint{#2}%
    \else\if(\TELfirst
      \expandafter\telnumber\expandafter{\the\TELtoks}\TELleftparen
      \telprint{#2}%
    \else\if)\TELfirst
      \expandafter\telnumber\expandafter{\the\TELtoks}\TELrightparen
      \telprint{#2}%
    \else\if+\TELfirst
      \expandafter\telnumber\expandafter{\the\TELtoks}\TELplus
      \telprint{#2}%
    \else\def\TELtemp{~}\ifx\TELtemp\TELfirst
      \expandafter\telnumber\expandafter{\the\TELtoks}\TELtilde
      \telprint{#2}%
    \else
      \TELtoks=\expandafter{\the\TELtoks#1}%
      \TELsplit#2{}\TELsplitEND
    \fi\fi\fi\fi\fi\fi\fi
  \fi
}
%    \end{macrocode}
%    \end{macro}
%    \begin{macro}{\telprint}
%    \begin{macrocode}
\DeclareRobustCommand*{\telprint}[1]{%
  \TELtoks={}%
  \TELsplit#1{}\TELsplitEND
}
%    \end{macrocode}
%    \end{macro}
%    \begin{macrocode}
\TELreset\let\TELreset=\UnDeFiNeD
%    \end{macrocode}
%
%    \begin{macrocode}
\TELAtEnd%
%</package>
%    \end{macrocode}
%
% \section{Test}
%
% \subsection{Catcode checks for loading}
%
%    \begin{macrocode}
%<*test1>
%    \end{macrocode}
%    \begin{macrocode}
\catcode`\{=1 %
\catcode`\}=2 %
\catcode`\#=6 %
\catcode`\@=11 %
\expandafter\ifx\csname count@\endcsname\relax
  \countdef\count@=255 %
\fi
\expandafter\ifx\csname @gobble\endcsname\relax
  \long\def\@gobble#1{}%
\fi
\expandafter\ifx\csname @firstofone\endcsname\relax
  \long\def\@firstofone#1{#1}%
\fi
\expandafter\ifx\csname loop\endcsname\relax
  \expandafter\@firstofone
\else
  \expandafter\@gobble
\fi
{%
  \def\loop#1\repeat{%
    \def\body{#1}%
    \iterate
  }%
  \def\iterate{%
    \body
      \let\next\iterate
    \else
      \let\next\relax
    \fi
    \next
  }%
  \let\repeat=\fi
}%
\def\RestoreCatcodes{}
\count@=0 %
\loop
  \edef\RestoreCatcodes{%
    \RestoreCatcodes
    \catcode\the\count@=\the\catcode\count@\relax
  }%
\ifnum\count@<255 %
  \advance\count@ 1 %
\repeat

\def\RangeCatcodeInvalid#1#2{%
  \count@=#1\relax
  \loop
    \catcode\count@=15 %
  \ifnum\count@<#2\relax
    \advance\count@ 1 %
  \repeat
}
\def\RangeCatcodeCheck#1#2#3{%
  \count@=#1\relax
  \loop
    \ifnum#3=\catcode\count@
    \else
      \errmessage{%
        Character \the\count@\space
        with wrong catcode \the\catcode\count@\space
        instead of \number#3%
      }%
    \fi
  \ifnum\count@<#2\relax
    \advance\count@ 1 %
  \repeat
}
\def\space{ }
\expandafter\ifx\csname LoadCommand\endcsname\relax
  \def\LoadCommand{\input telprint.sty\relax}%
\fi
\def\Test{%
  \RangeCatcodeInvalid{0}{47}%
  \RangeCatcodeInvalid{58}{64}%
  \RangeCatcodeInvalid{91}{96}%
  \RangeCatcodeInvalid{123}{255}%
  \catcode`\@=12 %
  \catcode`\\=0 %
  \catcode`\%=14 %
  \LoadCommand
  \RangeCatcodeCheck{0}{36}{15}%
  \RangeCatcodeCheck{37}{37}{14}%
  \RangeCatcodeCheck{38}{47}{15}%
  \RangeCatcodeCheck{48}{57}{12}%
  \RangeCatcodeCheck{58}{63}{15}%
  \RangeCatcodeCheck{64}{64}{12}%
  \RangeCatcodeCheck{65}{90}{11}%
  \RangeCatcodeCheck{91}{91}{15}%
  \RangeCatcodeCheck{92}{92}{0}%
  \RangeCatcodeCheck{93}{96}{15}%
  \RangeCatcodeCheck{97}{122}{11}%
  \RangeCatcodeCheck{123}{255}{15}%
  \RestoreCatcodes
}
\Test
\csname @@end\endcsname
\end
%    \end{macrocode}
%    \begin{macrocode}
%</test1>
%    \end{macrocode}
%
% \section{Installation}
%
% \subsection{Download}
%
% \paragraph{Package.} This package is available on
% CTAN\footnote{\url{https://ctan.org/pkg/telprint}}:
% \begin{description}
% \item[\CTAN{macros/latex/contrib/oberdiek/telprint.dtx}] The source file.
% \item[\CTAN{macros/latex/contrib/oberdiek/telprint.pdf}] Documentation.
% \end{description}
%
%
% \paragraph{Bundle.} All the packages of the bundle `oberdiek'
% are also available in a TDS compliant ZIP archive. There
% the packages are already unpacked and the documentation files
% are generated. The files and directories obey the TDS standard.
% \begin{description}
% \item[\CTANinstall{install/macros/latex/contrib/oberdiek.tds.zip}]
% \end{description}
% \emph{TDS} refers to the standard ``A Directory Structure
% for \TeX\ Files'' (\CTAN{tds/tds.pdf}). Directories
% with \xfile{texmf} in their name are usually organized this way.
%
% \subsection{Bundle installation}
%
% \paragraph{Unpacking.} Unpack the \xfile{oberdiek.tds.zip} in the
% TDS tree (also known as \xfile{texmf} tree) of your choice.
% Example (linux):
% \begin{quote}
%   |unzip oberdiek.tds.zip -d ~/texmf|
% \end{quote}
%
% \paragraph{Script installation.}
% Check the directory \xfile{TDS:scripts/oberdiek/} for
% scripts that need further installation steps.
% Package \xpackage{attachfile2} comes with the Perl script
% \xfile{pdfatfi.pl} that should be installed in such a way
% that it can be called as \texttt{pdfatfi}.
% Example (linux):
% \begin{quote}
%   |chmod +x scripts/oberdiek/pdfatfi.pl|\\
%   |cp scripts/oberdiek/pdfatfi.pl /usr/local/bin/|
% \end{quote}
%
% \subsection{Package installation}
%
% \paragraph{Unpacking.} The \xfile{.dtx} file is a self-extracting
% \docstrip\ archive. The files are extracted by running the
% \xfile{.dtx} through \plainTeX:
% \begin{quote}
%   \verb|tex telprint.dtx|
% \end{quote}
%
% \paragraph{TDS.} Now the different files must be moved into
% the different directories in your installation TDS tree
% (also known as \xfile{texmf} tree):
% \begin{quote}
% \def\t{^^A
% \begin{tabular}{@{}>{\ttfamily}l@{ $\rightarrow$ }>{\ttfamily}l@{}}
%   telprint.sty & tex/generic/oberdiek/telprint.sty\\
%   telprint.pdf & doc/latex/oberdiek/telprint.pdf\\
%   test/telprint-test1.tex & doc/latex/oberdiek/test/telprint-test1.tex\\
%   telprint.dtx & source/latex/oberdiek/telprint.dtx\\
% \end{tabular}^^A
% }^^A
% \sbox0{\t}^^A
% \ifdim\wd0>\linewidth
%   \begingroup
%     \advance\linewidth by\leftmargin
%     \advance\linewidth by\rightmargin
%   \edef\x{\endgroup
%     \def\noexpand\lw{\the\linewidth}^^A
%   }\x
%   \def\lwbox{^^A
%     \leavevmode
%     \hbox to \linewidth{^^A
%       \kern-\leftmargin\relax
%       \hss
%       \usebox0
%       \hss
%       \kern-\rightmargin\relax
%     }^^A
%   }^^A
%   \ifdim\wd0>\lw
%     \sbox0{\small\t}^^A
%     \ifdim\wd0>\linewidth
%       \ifdim\wd0>\lw
%         \sbox0{\footnotesize\t}^^A
%         \ifdim\wd0>\linewidth
%           \ifdim\wd0>\lw
%             \sbox0{\scriptsize\t}^^A
%             \ifdim\wd0>\linewidth
%               \ifdim\wd0>\lw
%                 \sbox0{\tiny\t}^^A
%                 \ifdim\wd0>\linewidth
%                   \lwbox
%                 \else
%                   \usebox0
%                 \fi
%               \else
%                 \lwbox
%               \fi
%             \else
%               \usebox0
%             \fi
%           \else
%             \lwbox
%           \fi
%         \else
%           \usebox0
%         \fi
%       \else
%         \lwbox
%       \fi
%     \else
%       \usebox0
%     \fi
%   \else
%     \lwbox
%   \fi
% \else
%   \usebox0
% \fi
% \end{quote}
% If you have a \xfile{docstrip.cfg} that configures and enables \docstrip's
% TDS installing feature, then some files can already be in the right
% place, see the documentation of \docstrip.
%
% \subsection{Refresh file name databases}
%
% If your \TeX~distribution
% (\teTeX, \mikTeX, \dots) relies on file name databases, you must refresh
% these. For example, \teTeX\ users run \verb|texhash| or
% \verb|mktexlsr|.
%
% \subsection{Some details for the interested}
%
% \paragraph{Attached source.}
%
% The PDF documentation on CTAN also includes the
% \xfile{.dtx} source file. It can be extracted by
% AcrobatReader 6 or higher. Another option is \textsf{pdftk},
% e.g. unpack the file into the current directory:
% \begin{quote}
%   \verb|pdftk telprint.pdf unpack_files output .|
% \end{quote}
%
% \paragraph{Unpacking with \LaTeX.}
% The \xfile{.dtx} chooses its action depending on the format:
% \begin{description}
% \item[\plainTeX:] Run \docstrip\ and extract the files.
% \item[\LaTeX:] Generate the documentation.
% \end{description}
% If you insist on using \LaTeX\ for \docstrip\ (really,
% \docstrip\ does not need \LaTeX), then inform the autodetect routine
% about your intention:
% \begin{quote}
%   \verb|latex \let\install=y\input{telprint.dtx}|
% \end{quote}
% Do not forget to quote the argument according to the demands
% of your shell.
%
% \paragraph{Generating the documentation.}
% You can use both the \xfile{.dtx} or the \xfile{.drv} to generate
% the documentation. The process can be configured by the
% configuration file \xfile{ltxdoc.cfg}. For instance, put this
% line into this file, if you want to have A4 as paper format:
% \begin{quote}
%   \verb|\PassOptionsToClass{a4paper}{article}|
% \end{quote}
% An example follows how to generate the
% documentation with pdf\LaTeX:
% \begin{quote}
%\begin{verbatim}
%pdflatex telprint.dtx
%makeindex -s gind.ist telprint.idx
%pdflatex telprint.dtx
%makeindex -s gind.ist telprint.idx
%pdflatex telprint.dtx
%\end{verbatim}
% \end{quote}
%
% \begin{History}
%   \begin{Version}{1996/11/28 v1.0}
%   \item
%     Erste lauff\"ahige Version.
%   \item
%     Nur '-' und '/' als zul\"assige Sonderzeichen.
%   \end{Version}
%   \begin{Version}{1997/09/16 v1.1}
%   \item
%     Dokumentation und Kommentare (Posting in de.comp.text.tex).
%   \item
%     Erweiterung um Sonderzeichen '(', ')', '+', '\textasciitilde' und ' '.
%   \item
%     Trennungsverhinderung am 'hyphen'.
%   \end{Version}
%   \begin{Version}{1997/10/16 v1.2}
%   \item
%     Schutz vor wiederholtem Einlesen.
%   \item
%     Unter \LaTeXe\ Nutzung des \cs{DeclareRobustCommand}-Features.
%   \end{Version}
%   \begin{Version}{1997/12/09 v1.3}
%   \item
%     Tempor\"are Variable eingespart.
%   \item
%     Posted in newsgroup \xnewsgroup{de.comp.text.tex}:\\
%     \URL{``\link{Re: Generisches Markup f\"ur Telefonnummern?}''}^^A
%     {http://groups.google.com/group/de.comp.text.tex/msg/86b3a86140007309}
%   \end{Version}
%   \begin{Version}{2004/11/02 v1.4}
%   \item
%     Fehler in der Dokumentation korrigiert.
%   \end{Version}
%   \begin{Version}{2005/09/30 v1.5}
%   \item
%     Konfigurierbare Symbole: '/', '(', ')', '+' und '\textasciitilde'.
%   \end{Version}
%   \begin{Version}{2006/02/12 v1.6}
%   \item
%     LPPL 1.3.
%   \item
%     Kurze \"Ubersicht in Englisch.
%   \item
%     CTAN.
%   \end{Version}
%   \begin{Version}{2006/08/26 v1.7}
%   \item
%     New DTX framework.
%   \end{Version}
%   \begin{Version}{2007/04/11 v1.8}
%   \item
%     Line ends sanitized.
%   \end{Version}
%   \begin{Version}{2007/09/09 v1.9}
%   \item
%     Catcode section added.
%   \item
%     Missing docstrip tag added.
%   \end{Version}
%   \begin{Version}{2008/08/11 v1.10}
%   \item
%     Code is not changed.
%   \item
%     URLs updated.
%   \end{Version}
%   \begin{Version}{2016/05/16 v1.11}
%   \item
%     Documentation updates.
%   \end{Version}
% \end{History}
%
% \PrintIndex
%
% \Finale
\endinput
|
% \end{quote}
% Do not forget to quote the argument according to the demands
% of your shell.
%
% \paragraph{Generating the documentation.}
% You can use both the \xfile{.dtx} or the \xfile{.drv} to generate
% the documentation. The process can be configured by the
% configuration file \xfile{ltxdoc.cfg}. For instance, put this
% line into this file, if you want to have A4 as paper format:
% \begin{quote}
%   \verb|\PassOptionsToClass{a4paper}{article}|
% \end{quote}
% An example follows how to generate the
% documentation with pdf\LaTeX:
% \begin{quote}
%\begin{verbatim}
%pdflatex telprint.dtx
%makeindex -s gind.ist telprint.idx
%pdflatex telprint.dtx
%makeindex -s gind.ist telprint.idx
%pdflatex telprint.dtx
%\end{verbatim}
% \end{quote}
%
% \begin{History}
%   \begin{Version}{1996/11/28 v1.0}
%   \item
%     Erste lauff\"ahige Version.
%   \item
%     Nur '-' und '/' als zul\"assige Sonderzeichen.
%   \end{Version}
%   \begin{Version}{1997/09/16 v1.1}
%   \item
%     Dokumentation und Kommentare (Posting in de.comp.text.tex).
%   \item
%     Erweiterung um Sonderzeichen '(', ')', '+', '\textasciitilde' und ' '.
%   \item
%     Trennungsverhinderung am 'hyphen'.
%   \end{Version}
%   \begin{Version}{1997/10/16 v1.2}
%   \item
%     Schutz vor wiederholtem Einlesen.
%   \item
%     Unter \LaTeXe\ Nutzung des \cs{DeclareRobustCommand}-Features.
%   \end{Version}
%   \begin{Version}{1997/12/09 v1.3}
%   \item
%     Tempor\"are Variable eingespart.
%   \item
%     Posted in newsgroup \xnewsgroup{de.comp.text.tex}:\\
%     \URL{``\link{Re: Generisches Markup f\"ur Telefonnummern?}''}^^A
%     {http://groups.google.com/group/de.comp.text.tex/msg/86b3a86140007309}
%   \end{Version}
%   \begin{Version}{2004/11/02 v1.4}
%   \item
%     Fehler in der Dokumentation korrigiert.
%   \end{Version}
%   \begin{Version}{2005/09/30 v1.5}
%   \item
%     Konfigurierbare Symbole: '/', '(', ')', '+' und '\textasciitilde'.
%   \end{Version}
%   \begin{Version}{2006/02/12 v1.6}
%   \item
%     LPPL 1.3.
%   \item
%     Kurze \"Ubersicht in Englisch.
%   \item
%     CTAN.
%   \end{Version}
%   \begin{Version}{2006/08/26 v1.7}
%   \item
%     New DTX framework.
%   \end{Version}
%   \begin{Version}{2007/04/11 v1.8}
%   \item
%     Line ends sanitized.
%   \end{Version}
%   \begin{Version}{2007/09/09 v1.9}
%   \item
%     Catcode section added.
%   \item
%     Missing docstrip tag added.
%   \end{Version}
%   \begin{Version}{2008/08/11 v1.10}
%   \item
%     Code is not changed.
%   \item
%     URLs updated.
%   \end{Version}
%   \begin{Version}{2016/05/16 v1.11}
%   \item
%     Documentation updates.
%   \end{Version}
% \end{History}
%
% \PrintIndex
%
% \Finale
\endinput

%        (quote the arguments according to the demands of your shell)
%
% Documentation:
%    (a) If telprint.drv is present:
%           latex telprint.drv
%    (b) Without telprint.drv:
%           latex telprint.dtx; ...
%    The class ltxdoc loads the configuration file ltxdoc.cfg
%    if available. Here you can specify further options, e.g.
%    use A4 as paper format:
%       \PassOptionsToClass{a4paper}{article}
%
%    Programm calls to get the documentation (example):
%       pdflatex telprint.dtx
%       makeindex -s gind.ist telprint.idx
%       pdflatex telprint.dtx
%       makeindex -s gind.ist telprint.idx
%       pdflatex telprint.dtx
%
% Installation:
%    TDS:tex/generic/oberdiek/telprint.sty
%    TDS:doc/latex/oberdiek/telprint.pdf
%    TDS:doc/latex/oberdiek/test/telprint-test1.tex
%    TDS:source/latex/oberdiek/telprint.dtx
%
%<*ignore>
\begingroup
  \catcode123=1 %
  \catcode125=2 %
  \def\x{LaTeX2e}%
\expandafter\endgroup
\ifcase 0\ifx\install y1\fi\expandafter
         \ifx\csname processbatchFile\endcsname\relax\else1\fi
         \ifx\fmtname\x\else 1\fi\relax
\else\csname fi\endcsname
%</ignore>
%<*install>
\input docstrip.tex
\Msg{************************************************************************}
\Msg{* Installation}
\Msg{* Package: telprint 2016/05/16 v1.11 Format German phone numbers (HO)}
\Msg{************************************************************************}

\keepsilent
\askforoverwritefalse

\let\MetaPrefix\relax
\preamble

This is a generated file.

Project: telprint
Version: 2016/05/16 v1.11

Copyright (C) 1996, 1997, 2004-2008 by
   Heiko Oberdiek <heiko.oberdiek at googlemail.com>

This work may be distributed and/or modified under the
conditions of the LaTeX Project Public License, either
version 1.3c of this license or (at your option) any later
version. This version of this license is in
   http://www.latex-project.org/lppl/lppl-1-3c.txt
and the latest version of this license is in
   http://www.latex-project.org/lppl.txt
and version 1.3 or later is part of all distributions of
LaTeX version 2005/12/01 or later.

This work has the LPPL maintenance status "maintained".

This Current Maintainer of this work is Heiko Oberdiek.

The Base Interpreter refers to any `TeX-Format',
because some files are installed in TDS:tex/generic//.

This work consists of the main source file telprint.dtx
and the derived files
   telprint.sty, telprint.pdf, telprint.ins, telprint.drv,
   telprint-test1.tex.

\endpreamble
\let\MetaPrefix\DoubleperCent

\generate{%
  \file{telprint.ins}{\from{telprint.dtx}{install}}%
  \file{telprint.drv}{\from{telprint.dtx}{driver}}%
  \usedir{tex/generic/oberdiek}%
  \file{telprint.sty}{\from{telprint.dtx}{package}}%
%  \usedir{doc/latex/oberdiek/test}%
%  \file{telprint-test1.tex}{\from{telprint.dtx}{test1}}%
  \nopreamble
  \nopostamble
%  \usedir{source/latex/oberdiek/catalogue}%
%  \file{telprint.xml}{\from{telprint.dtx}{catalogue}}%
}

\catcode32=13\relax% active space
\let =\space%
\Msg{************************************************************************}
\Msg{*}
\Msg{* To finish the installation you have to move the following}
\Msg{* file into a directory searched by TeX:}
\Msg{*}
\Msg{*     telprint.sty}
\Msg{*}
\Msg{* To produce the documentation run the file `telprint.drv'}
\Msg{* through LaTeX.}
\Msg{*}
\Msg{* Happy TeXing!}
\Msg{*}
\Msg{************************************************************************}

\endbatchfile
%</install>
%<*ignore>
\fi
%</ignore>
%<*driver>
\NeedsTeXFormat{LaTeX2e}
\ProvidesFile{telprint.drv}%
  [2016/05/16 v1.11 Format German phone numbers (HO)]%
\documentclass{ltxdoc}
\usepackage{holtxdoc}[2011/11/22]
\usepackage[ngerman,english]{babel}
\begin{document}
  \DocInput{telprint.dtx}%
\end{document}
%</driver>
% \fi
%
%
% \CharacterTable
%  {Upper-case    \A\B\C\D\E\F\G\H\I\J\K\L\M\N\O\P\Q\R\S\T\U\V\W\X\Y\Z
%   Lower-case    \a\b\c\d\e\f\g\h\i\j\k\l\m\n\o\p\q\r\s\t\u\v\w\x\y\z
%   Digits        \0\1\2\3\4\5\6\7\8\9
%   Exclamation   \!     Double quote  \"     Hash (number) \#
%   Dollar        \$     Percent       \%     Ampersand     \&
%   Acute accent  \'     Left paren    \(     Right paren   \)
%   Asterisk      \*     Plus          \+     Comma         \,
%   Minus         \-     Point         \.     Solidus       \/
%   Colon         \:     Semicolon     \;     Less than     \<
%   Equals        \=     Greater than  \>     Question mark \?
%   Commercial at \@     Left bracket  \[     Backslash     \\
%   Right bracket \]     Circumflex    \^     Underscore    \_
%   Grave accent  \`     Left brace    \{     Vertical bar  \|
%   Right brace   \}     Tilde         \~}
%
% \GetFileInfo{telprint.drv}
%
% \title{The \xpackage{telprint} package}
% \date{2016/05/16 v1.11}
% \author{Heiko Oberdiek\thanks
% {Please report any issues at https://github.com/ho-tex/oberdiek/issues}\\
% \xemail{heiko.oberdiek at googlemail.com}}
%
% \maketitle
%
% \begin{abstract}
% Package \xpackage{telprint} provides \cs{telprint} for formatting
% German phone numbers.
% \end{abstract}
%
% \tableofcontents
%
% \section{Documentation}
%
% \subsection{Introduction}
%
%            This is a very old package that I have written
%            to format phone numbers. It follows German
%            conventions and the documentation is mainly in German.
%
% \subsection{Short overview in English}
%
% \LaTeX:
% \begin{quote}
% |\usepackage{telprint}|\\
% |\telprint{123/456-789}|\\
% \end{quote}
% \plainTeX:
% \begin{quote}
%   |\input telprint.sty|\\
%   |\telprint{123/456-789}|
% \end{quote}
%
% \DescribeMacro\telprint
% |\telprint{...}| formats the explicitly given number.
%     Digits, spaces and some special characters
%     ('+', '/', '-', '(', ')', '\textasciitilde', ' ') are supported.
%     Numbers are divided into groups of two digits from the right.
% Examples:
% \begin{quote}
%     |\telprint{0761/12345}     ==> 07\,61/1\,23\,45|\\
%     |\telprint{01234/567-89}   ==> 0\,12\,34/5\,67\leavevmode\hbox{-}89|\\
%     |\telprint{+49 (6221) 297} ==> +49~(62\,21)~2\,97|
% \end{quote}
%
% \subsubsection{Configuration}
%
% The output of the symbols can be configured by
% \cs{telhyphen}, \cs{telslash}, \cs{telleftparen}, \cs{telrightparen},
% \cs{telplus}, \cs{teltilde}.
% Example:
% \begin{quote}
%   |\telslash{\,/\,}\\|
%   |\telprint{12/34} ==> 12\,/\,34|
% \end{quote}
%
% \DescribeMacro\telspace
% \cs{telspace} configures the space between digit groups.
%
% \DescribeMacro\telnumber
% \cs{telnumber} only formats a number in digit groups; special
%    characters are not recognized.
%
% \subsection{Documentation in German}
%
% \begin{otherlanguage*}{ngerman}
% \hyphenation{To-ken-ma-kros}
% \begin{itemize}
% \item \DescribeMacro\telprint |telprint#1|\\
%   Der eigentliche Anwenderbefehl zur formatierten Ausgabe von
%   Telefonnummern. Diese d\"urfen dabei nur als Zahlen angegeben
%   werden(, da sie tokenweise analysiert werden).
%   Als Trenn- oder Sonderzeichen werden unterst\"utzt:
%   '+', '/', '-', '(', ')', '\textasciitilde', ' '
%   Einfache Leerzeichen werden erkannt und durch Tilden ersetzt, um
%   Trennungen in der Telefonnummer zu verhindern. (Man beachte aus
%   gleichem Grunde die \cs{hbox} bei '-'.)
%   Beispiele:
%   \begin{quote}
%     |\telprint{0761/12345}     ==> 07\,61/1\,23\,45|\\
%     |\telprint{01234/567-89}   ==> 0\,12\,34/5\,67\leavevmode\hbox{-}89|\\
%     |\telprint{+49 (6221) 297} ==> +49~(62\,21)~2\,97|
%   \end{quote}
% \end{itemize}
% Der Rest enth\"alt eher Technisches:
% \begin{itemize}
% \item \DescribeMacro\telspace |\telspace#1|\\
%   Mit diesem Befehl wird der Abstand zwischen den Zifferngruppen
%   angegeben (Default: |\,|).
%   (Durch |\telspace{}| kann dieser zusaetzliche Abstand abgestellt
%   werden.)
% \item \DescribeMacro\telhyphen |\telhyphen#1|\\
%   Dieser Befehl gibt die Art des Bindestriches, wie er ausgegeben
%   werden soll. In der Eingabe darf jedoch nur der einfache
%   Bindestrich stehen:
%   |\telprint{123-45}|, jedoch NIE |\telprint{123--45}|!
%   Kopka-Bindestrich-Fans geben an:
%   |\telhyphen{\leavevmode\hbox{--}}|
% \item
%   \DescribeMacro{\telslash}
%   \DescribeMacro{\telleftparen}
%   \DescribeMacro{\telrightparen}
%   \DescribeMacro{\telplus}
%   \DescribeMacro{\teltilde}
%   |\telslash#1|, |\telleftparen#1|, |\telrightparen#1|, |\telplus#1|,
%   |\teltilde|\\
%   Diese Befehle konfigurieren die Zeichen '/', '(', ')', '+'
%   und '\textasciitilde'. Sie funktionieren analog zu \cs{telhyphen}.
% \item \DescribeMacro\telnumber |\telnumber#1|\\
%   Richtung interner Befehl: Er dient dazu, eine Zifferngruppe
%   in Zweiergruppen auszugeben.
%   Die einzelnen Zahlen werden im Tokenregister \cs{TELtoks}
%   gespeichert. Abwechselnd werden dabei zwischen zwei Token
%   (Zahlen) \cs{TELx} bzw. \cs{TELy} eingefuegt, abh\"angig von dem
%   wechselnden Wert von \cs{TELswitch}. Zum Schluss kann dann einfach
%   festgestellt werden ob die Nummer nun eine geradzahlige oder
%   ungeradzahlige Zahl von Ziffern aufwies. Dem entsprechend wird
%   \cs{TELx} mit dem Zusatzabstand belegt und \cs{TELy} leer definiert
%   oder umgekehrt. )
% \item |\TEL...| interne Befehle, Technisches:\\
%   \cs{TELsplit} dient zur Aufteilung einer zusammengesetzten
%   Telefonnummer (Vorwahl, Hauptnummer, Nebenstelle). In dieser
%   Implementation werden als Trennzeichen nur '/' und '-' erkannt.
%   Die einzelnen Bestandteile wie Vorwahl werden dann dem Befehl
%   \cs{telnumber} zur Formatierung uebergeben.
% \item Die Erkennung von einfachen Leerzeichen ist um einiges
%   schwieriger: Die Tokentrennung ueber Parameter |#1#2| funktioniert
%   nicht f\"ur einfache Leerzeichen, da TeX sie \emph{niemals} als
%   eigenst\"andige Argumente behandelt! (The TeXbook, Chapter 20,
%   p. 201)
%
%   (Anmerkung am Rande: Deshalb funktionieren die entsprechenden
%   Tokenmakros auf S. 149 des Buches "`Einf\"uhrung in TeX"' von
%   N. Schwarz (3. Aufl.) nicht, wenn im Tokenregister als erstes
%   ein einfaches Leerzeichen steht!)
% \end{itemize}
% \end{otherlanguage*}
%
% \StopEventually{
% }
%
% \section{Implementation}
%
%    \begin{macrocode}
%<*package>
%    \end{macrocode}
%
% \subsection{Reload check and package identification}
%    Reload check, especially if the package is not used with \LaTeX.
%    \begin{macrocode}
\begingroup\catcode61\catcode48\catcode32=10\relax%
  \catcode13=5 % ^^M
  \endlinechar=13 %
  \catcode35=6 % #
  \catcode39=12 % '
  \catcode44=12 % ,
  \catcode45=12 % -
  \catcode46=12 % .
  \catcode58=12 % :
  \catcode64=11 % @
  \catcode123=1 % {
  \catcode125=2 % }
  \expandafter\let\expandafter\x\csname ver@telprint.sty\endcsname
  \ifx\x\relax % plain-TeX, first loading
  \else
    \def\empty{}%
    \ifx\x\empty % LaTeX, first loading,
      % variable is initialized, but \ProvidesPackage not yet seen
    \else
      \expandafter\ifx\csname PackageInfo\endcsname\relax
        \def\x#1#2{%
          \immediate\write-1{Package #1 Info: #2.}%
        }%
      \else
        \def\x#1#2{\PackageInfo{#1}{#2, stopped}}%
      \fi
      \x{telprint}{The package is already loaded}%
      \aftergroup\endinput
    \fi
  \fi
\endgroup%
%    \end{macrocode}
%    Package identification:
%    \begin{macrocode}
\begingroup\catcode61\catcode48\catcode32=10\relax%
  \catcode13=5 % ^^M
  \endlinechar=13 %
  \catcode35=6 % #
  \catcode39=12 % '
  \catcode40=12 % (
  \catcode41=12 % )
  \catcode44=12 % ,
  \catcode45=12 % -
  \catcode46=12 % .
  \catcode47=12 % /
  \catcode58=12 % :
  \catcode64=11 % @
  \catcode91=12 % [
  \catcode93=12 % ]
  \catcode123=1 % {
  \catcode125=2 % }
  \expandafter\ifx\csname ProvidesPackage\endcsname\relax
    \def\x#1#2#3[#4]{\endgroup
      \immediate\write-1{Package: #3 #4}%
      \xdef#1{#4}%
    }%
  \else
    \def\x#1#2[#3]{\endgroup
      #2[{#3}]%
      \ifx#1\@undefined
        \xdef#1{#3}%
      \fi
      \ifx#1\relax
        \xdef#1{#3}%
      \fi
    }%
  \fi
\expandafter\x\csname ver@telprint.sty\endcsname
\ProvidesPackage{telprint}%
  [2016/05/16 v1.11 Format German phone numbers (HO)]%
%    \end{macrocode}
%
% \subsection{Catcodes}
%
%    \begin{macrocode}
\begingroup\catcode61\catcode48\catcode32=10\relax%
  \catcode13=5 % ^^M
  \endlinechar=13 %
  \catcode123=1 % {
  \catcode125=2 % }
  \catcode64=11 % @
  \def\x{\endgroup
    \expandafter\edef\csname TELAtEnd\endcsname{%
      \endlinechar=\the\endlinechar\relax
      \catcode13=\the\catcode13\relax
      \catcode32=\the\catcode32\relax
      \catcode35=\the\catcode35\relax
      \catcode61=\the\catcode61\relax
      \catcode64=\the\catcode64\relax
      \catcode123=\the\catcode123\relax
      \catcode125=\the\catcode125\relax
    }%
  }%
\x\catcode61\catcode48\catcode32=10\relax%
\catcode13=5 % ^^M
\endlinechar=13 %
\catcode35=6 % #
\catcode64=11 % @
\catcode123=1 % {
\catcode125=2 % }
\def\TMP@EnsureCode#1#2{%
  \edef\TELAtEnd{%
    \TELAtEnd
    \catcode#1=\the\catcode#1\relax
  }%
  \catcode#1=#2\relax
}
\TMP@EnsureCode{33}{12}% !
\TMP@EnsureCode{36}{3}% $
\TMP@EnsureCode{40}{12}% (
\TMP@EnsureCode{41}{12}% )
\TMP@EnsureCode{42}{12}% *
\TMP@EnsureCode{43}{12}% +
\TMP@EnsureCode{44}{12}% ,
\TMP@EnsureCode{45}{12}% -
\TMP@EnsureCode{46}{12}% .
\TMP@EnsureCode{47}{12}% /
\TMP@EnsureCode{91}{12}% [
\TMP@EnsureCode{93}{12}% ]
\TMP@EnsureCode{126}{13}% ~ (active)
\edef\TELAtEnd{\TELAtEnd\noexpand\endinput}
%    \end{macrocode}
%
% \subsection{Package macros}
%    \begin{macrocode}
\ifx\DeclareRobustCommand\UnDeFiNeD
  \def\DeclareRobustCommand*#1[1]{\def#1##1}%
  \def\TELreset{\let\DeclareRobustCommand=\UnDeFiNeD}%
  \input infwarerr.sty\relax
  \@PackageInfo{telprint}{%
    Macros are not robust!%
  }%
\else
  \let\TELreset=\relax
\fi
%    \end{macrocode}
%    \begin{macro}{\telspace}
%    \begin{macrocode}
\DeclareRobustCommand*{\telspace}[1]{\def\TELspace{#1}}
\telspace{{}$\,${}}
%    \end{macrocode}
%    \end{macro}
%    \begin{macro}{\telhyphen}
%    \begin{macrocode}
\DeclareRobustCommand*{\telhyphen}[1]{\def\TELhyphen{#1}}
\telhyphen{\leavevmode\hbox{-}}% \hbox zur Verhinderung der Trennung
%    \end{macrocode}
%    \end{macro}
%    \begin{macro}{\telslash}
%    \begin{macrocode}
\DeclareRobustCommand*{\telslash}[1]{\def\TELslash{#1}}
\telslash{/}%
%    \end{macrocode}
%    \end{macro}
%    \begin{macro}{\telleftparen}
%    \begin{macrocode}
\DeclareRobustCommand*{\telleftparen}[1]{\def\TELleftparen{#1}}
\telleftparen{(}%
%    \end{macrocode}
%    \end{macro}
%    \begin{macro}{\telrightparen}
%    \begin{macrocode}
\DeclareRobustCommand*{\telrightparen}[1]{\def\TELrightparen{#1}}
\telrightparen{)}%
%    \end{macrocode}
%    \end{macro}
%    \begin{macro}{\telplus}
%    \begin{macrocode}
\DeclareRobustCommand*{\telplus}[1]{\def\TELplus{#1}}
\telplus{+}%
%    \end{macrocode}
%    \end{macro}
%    \begin{macro}{\teltilde}
%    \begin{macrocode}
\DeclareRobustCommand*{\teltilde}[1]{\def\TELtilde{#1}}
\teltilde{~}%
%    \end{macrocode}
%    \end{macro}
%    \begin{macro}{\TELtoks}
%    \begin{macrocode}
\newtoks\TELtoks
%    \end{macrocode}
%    \end{macro}
%    \begin{macro}{\TELnumber}
%    \begin{macrocode}
\def\TELnumber#1#2\TELnumberEND{%
  \begingroup
  \def\0{#2}%
  \expandafter\endgroup
  \ifx\0\empty
    \TELtoks=\expandafter{\the\TELtoks#1}%
    \ifnum\TELswitch=0 %
      \def\TELx{\TELspace}\def\TELy{}%
    \else
      \def\TELx{}\def\TELy{\TELspace}%
    \fi
    \the\TELtoks
  \else
    \ifnum\TELswitch=0 %
      \TELtoks=\expandafter{\the\TELtoks#1\TELx}%
      \def\TELswitch{1}%
    \else
      \TELtoks=\expandafter{\the\TELtoks#1\TELy}%
      \def\TELswitch{0}%
    \fi
    \TELnumber#2\TELnumberEND
  \fi
}
%    \end{macrocode}
%    \end{macro}
%    \begin{macro}{\telnumber}
%    \begin{macrocode}
\DeclareRobustCommand*{\telnumber}[1]{%
  \TELtoks={}%
  \def\TELswitch{0}%
  \TELnumber#1{}\TELnumberEND
}
%    \end{macrocode}
%    \end{macro}
%    \begin{macro}{\TELsplit}
%    \begin{macrocode}
\def\TELsplit{\futurelet\TELfuture\TELdosplit}
%    \end{macrocode}
%    \end{macro}
%    \begin{macro}{\TELdosplit}
%    \begin{macrocode}
\def\TELdosplit#1#2\TELsplitEND
{%
  \def\TELsp{ }%
  \expandafter\ifx\TELsp\TELfuture
    \let\TELfuture=\relax
    \expandafter\telnumber\expandafter{\the\TELtoks}~%
    \telprint{#1#2}% Das Leerzeichen kann nicht #1 sein!
  \else
    \def\TELfirst{#1}%
    \ifx\TELfirst\empty
      \expandafter\telnumber\expandafter{\the\TELtoks}%
      \TELtoks={}%
    \else\if-\TELfirst
      \expandafter\telnumber\expandafter{\the\TELtoks}\TELhyphen
      \telprint{#2}%
    \else\if/\TELfirst
      \expandafter\telnumber\expandafter{\the\TELtoks}\TELslash
      \telprint{#2}%
    \else\if(\TELfirst
      \expandafter\telnumber\expandafter{\the\TELtoks}\TELleftparen
      \telprint{#2}%
    \else\if)\TELfirst
      \expandafter\telnumber\expandafter{\the\TELtoks}\TELrightparen
      \telprint{#2}%
    \else\if+\TELfirst
      \expandafter\telnumber\expandafter{\the\TELtoks}\TELplus
      \telprint{#2}%
    \else\def\TELtemp{~}\ifx\TELtemp\TELfirst
      \expandafter\telnumber\expandafter{\the\TELtoks}\TELtilde
      \telprint{#2}%
    \else
      \TELtoks=\expandafter{\the\TELtoks#1}%
      \TELsplit#2{}\TELsplitEND
    \fi\fi\fi\fi\fi\fi\fi
  \fi
}
%    \end{macrocode}
%    \end{macro}
%    \begin{macro}{\telprint}
%    \begin{macrocode}
\DeclareRobustCommand*{\telprint}[1]{%
  \TELtoks={}%
  \TELsplit#1{}\TELsplitEND
}
%    \end{macrocode}
%    \end{macro}
%    \begin{macrocode}
\TELreset\let\TELreset=\UnDeFiNeD
%    \end{macrocode}
%
%    \begin{macrocode}
\TELAtEnd%
%</package>
%    \end{macrocode}
%
% \section{Test}
%
% \subsection{Catcode checks for loading}
%
%    \begin{macrocode}
%<*test1>
%    \end{macrocode}
%    \begin{macrocode}
\catcode`\{=1 %
\catcode`\}=2 %
\catcode`\#=6 %
\catcode`\@=11 %
\expandafter\ifx\csname count@\endcsname\relax
  \countdef\count@=255 %
\fi
\expandafter\ifx\csname @gobble\endcsname\relax
  \long\def\@gobble#1{}%
\fi
\expandafter\ifx\csname @firstofone\endcsname\relax
  \long\def\@firstofone#1{#1}%
\fi
\expandafter\ifx\csname loop\endcsname\relax
  \expandafter\@firstofone
\else
  \expandafter\@gobble
\fi
{%
  \def\loop#1\repeat{%
    \def\body{#1}%
    \iterate
  }%
  \def\iterate{%
    \body
      \let\next\iterate
    \else
      \let\next\relax
    \fi
    \next
  }%
  \let\repeat=\fi
}%
\def\RestoreCatcodes{}
\count@=0 %
\loop
  \edef\RestoreCatcodes{%
    \RestoreCatcodes
    \catcode\the\count@=\the\catcode\count@\relax
  }%
\ifnum\count@<255 %
  \advance\count@ 1 %
\repeat

\def\RangeCatcodeInvalid#1#2{%
  \count@=#1\relax
  \loop
    \catcode\count@=15 %
  \ifnum\count@<#2\relax
    \advance\count@ 1 %
  \repeat
}
\def\RangeCatcodeCheck#1#2#3{%
  \count@=#1\relax
  \loop
    \ifnum#3=\catcode\count@
    \else
      \errmessage{%
        Character \the\count@\space
        with wrong catcode \the\catcode\count@\space
        instead of \number#3%
      }%
    \fi
  \ifnum\count@<#2\relax
    \advance\count@ 1 %
  \repeat
}
\def\space{ }
\expandafter\ifx\csname LoadCommand\endcsname\relax
  \def\LoadCommand{\input telprint.sty\relax}%
\fi
\def\Test{%
  \RangeCatcodeInvalid{0}{47}%
  \RangeCatcodeInvalid{58}{64}%
  \RangeCatcodeInvalid{91}{96}%
  \RangeCatcodeInvalid{123}{255}%
  \catcode`\@=12 %
  \catcode`\\=0 %
  \catcode`\%=14 %
  \LoadCommand
  \RangeCatcodeCheck{0}{36}{15}%
  \RangeCatcodeCheck{37}{37}{14}%
  \RangeCatcodeCheck{38}{47}{15}%
  \RangeCatcodeCheck{48}{57}{12}%
  \RangeCatcodeCheck{58}{63}{15}%
  \RangeCatcodeCheck{64}{64}{12}%
  \RangeCatcodeCheck{65}{90}{11}%
  \RangeCatcodeCheck{91}{91}{15}%
  \RangeCatcodeCheck{92}{92}{0}%
  \RangeCatcodeCheck{93}{96}{15}%
  \RangeCatcodeCheck{97}{122}{11}%
  \RangeCatcodeCheck{123}{255}{15}%
  \RestoreCatcodes
}
\Test
\csname @@end\endcsname
\end
%    \end{macrocode}
%    \begin{macrocode}
%</test1>
%    \end{macrocode}
%
% \section{Installation}
%
% \subsection{Download}
%
% \paragraph{Package.} This package is available on
% CTAN\footnote{\url{https://ctan.org/pkg/telprint}}:
% \begin{description}
% \item[\CTAN{macros/latex/contrib/oberdiek/telprint.dtx}] The source file.
% \item[\CTAN{macros/latex/contrib/oberdiek/telprint.pdf}] Documentation.
% \end{description}
%
%
% \paragraph{Bundle.} All the packages of the bundle `oberdiek'
% are also available in a TDS compliant ZIP archive. There
% the packages are already unpacked and the documentation files
% are generated. The files and directories obey the TDS standard.
% \begin{description}
% \item[\CTANinstall{install/macros/latex/contrib/oberdiek.tds.zip}]
% \end{description}
% \emph{TDS} refers to the standard ``A Directory Structure
% for \TeX\ Files'' (\CTAN{tds/tds.pdf}). Directories
% with \xfile{texmf} in their name are usually organized this way.
%
% \subsection{Bundle installation}
%
% \paragraph{Unpacking.} Unpack the \xfile{oberdiek.tds.zip} in the
% TDS tree (also known as \xfile{texmf} tree) of your choice.
% Example (linux):
% \begin{quote}
%   |unzip oberdiek.tds.zip -d ~/texmf|
% \end{quote}
%
% \paragraph{Script installation.}
% Check the directory \xfile{TDS:scripts/oberdiek/} for
% scripts that need further installation steps.
% Package \xpackage{attachfile2} comes with the Perl script
% \xfile{pdfatfi.pl} that should be installed in such a way
% that it can be called as \texttt{pdfatfi}.
% Example (linux):
% \begin{quote}
%   |chmod +x scripts/oberdiek/pdfatfi.pl|\\
%   |cp scripts/oberdiek/pdfatfi.pl /usr/local/bin/|
% \end{quote}
%
% \subsection{Package installation}
%
% \paragraph{Unpacking.} The \xfile{.dtx} file is a self-extracting
% \docstrip\ archive. The files are extracted by running the
% \xfile{.dtx} through \plainTeX:
% \begin{quote}
%   \verb|tex telprint.dtx|
% \end{quote}
%
% \paragraph{TDS.} Now the different files must be moved into
% the different directories in your installation TDS tree
% (also known as \xfile{texmf} tree):
% \begin{quote}
% \def\t{^^A
% \begin{tabular}{@{}>{\ttfamily}l@{ $\rightarrow$ }>{\ttfamily}l@{}}
%   telprint.sty & tex/generic/oberdiek/telprint.sty\\
%   telprint.pdf & doc/latex/oberdiek/telprint.pdf\\
%   test/telprint-test1.tex & doc/latex/oberdiek/test/telprint-test1.tex\\
%   telprint.dtx & source/latex/oberdiek/telprint.dtx\\
% \end{tabular}^^A
% }^^A
% \sbox0{\t}^^A
% \ifdim\wd0>\linewidth
%   \begingroup
%     \advance\linewidth by\leftmargin
%     \advance\linewidth by\rightmargin
%   \edef\x{\endgroup
%     \def\noexpand\lw{\the\linewidth}^^A
%   }\x
%   \def\lwbox{^^A
%     \leavevmode
%     \hbox to \linewidth{^^A
%       \kern-\leftmargin\relax
%       \hss
%       \usebox0
%       \hss
%       \kern-\rightmargin\relax
%     }^^A
%   }^^A
%   \ifdim\wd0>\lw
%     \sbox0{\small\t}^^A
%     \ifdim\wd0>\linewidth
%       \ifdim\wd0>\lw
%         \sbox0{\footnotesize\t}^^A
%         \ifdim\wd0>\linewidth
%           \ifdim\wd0>\lw
%             \sbox0{\scriptsize\t}^^A
%             \ifdim\wd0>\linewidth
%               \ifdim\wd0>\lw
%                 \sbox0{\tiny\t}^^A
%                 \ifdim\wd0>\linewidth
%                   \lwbox
%                 \else
%                   \usebox0
%                 \fi
%               \else
%                 \lwbox
%               \fi
%             \else
%               \usebox0
%             \fi
%           \else
%             \lwbox
%           \fi
%         \else
%           \usebox0
%         \fi
%       \else
%         \lwbox
%       \fi
%     \else
%       \usebox0
%     \fi
%   \else
%     \lwbox
%   \fi
% \else
%   \usebox0
% \fi
% \end{quote}
% If you have a \xfile{docstrip.cfg} that configures and enables \docstrip's
% TDS installing feature, then some files can already be in the right
% place, see the documentation of \docstrip.
%
% \subsection{Refresh file name databases}
%
% If your \TeX~distribution
% (\teTeX, \mikTeX, \dots) relies on file name databases, you must refresh
% these. For example, \teTeX\ users run \verb|texhash| or
% \verb|mktexlsr|.
%
% \subsection{Some details for the interested}
%
% \paragraph{Attached source.}
%
% The PDF documentation on CTAN also includes the
% \xfile{.dtx} source file. It can be extracted by
% AcrobatReader 6 or higher. Another option is \textsf{pdftk},
% e.g. unpack the file into the current directory:
% \begin{quote}
%   \verb|pdftk telprint.pdf unpack_files output .|
% \end{quote}
%
% \paragraph{Unpacking with \LaTeX.}
% The \xfile{.dtx} chooses its action depending on the format:
% \begin{description}
% \item[\plainTeX:] Run \docstrip\ and extract the files.
% \item[\LaTeX:] Generate the documentation.
% \end{description}
% If you insist on using \LaTeX\ for \docstrip\ (really,
% \docstrip\ does not need \LaTeX), then inform the autodetect routine
% about your intention:
% \begin{quote}
%   \verb|latex \let\install=y% \iffalse meta-comment
%
% File: telprint.dtx
% Version: 2016/05/16 v1.11
% Info: Format German phone numbers
%
% Copyright (C) 1996, 1997, 2004-2008 by
%    Heiko Oberdiek <heiko.oberdiek at googlemail.com>
%    2016
%    https://github.com/ho-tex/oberdiek/issues
%
% This work may be distributed and/or modified under the
% conditions of the LaTeX Project Public License, either
% version 1.3c of this license or (at your option) any later
% version. This version of this license is in
%    http://www.latex-project.org/lppl/lppl-1-3c.txt
% and the latest version of this license is in
%    http://www.latex-project.org/lppl.txt
% and version 1.3 or later is part of all distributions of
% LaTeX version 2005/12/01 or later.
%
% This work has the LPPL maintenance status "maintained".
%
% This Current Maintainer of this work is Heiko Oberdiek.
%
% The Base Interpreter refers to any `TeX-Format',
% because some files are installed in TDS:tex/generic//.
%
% This work consists of the main source file telprint.dtx
% and the derived files
%    telprint.sty, telprint.pdf, telprint.ins, telprint.drv,
%    telprint-test1.tex.
%
% Distribution:
%    CTAN:macros/latex/contrib/oberdiek/telprint.dtx
%    CTAN:macros/latex/contrib/oberdiek/telprint.pdf
%
% Unpacking:
%    (a) If telprint.ins is present:
%           tex telprint.ins
%    (b) Without telprint.ins:
%           tex telprint.dtx
%    (c) If you insist on using LaTeX
%           latex \let\install=y% \iffalse meta-comment
%
% File: telprint.dtx
% Version: 2016/05/16 v1.11
% Info: Format German phone numbers
%
% Copyright (C) 1996, 1997, 2004-2008 by
%    Heiko Oberdiek <heiko.oberdiek at googlemail.com>
%    2016
%    https://github.com/ho-tex/oberdiek/issues
%
% This work may be distributed and/or modified under the
% conditions of the LaTeX Project Public License, either
% version 1.3c of this license or (at your option) any later
% version. This version of this license is in
%    http://www.latex-project.org/lppl/lppl-1-3c.txt
% and the latest version of this license is in
%    http://www.latex-project.org/lppl.txt
% and version 1.3 or later is part of all distributions of
% LaTeX version 2005/12/01 or later.
%
% This work has the LPPL maintenance status "maintained".
%
% This Current Maintainer of this work is Heiko Oberdiek.
%
% The Base Interpreter refers to any `TeX-Format',
% because some files are installed in TDS:tex/generic//.
%
% This work consists of the main source file telprint.dtx
% and the derived files
%    telprint.sty, telprint.pdf, telprint.ins, telprint.drv,
%    telprint-test1.tex.
%
% Distribution:
%    CTAN:macros/latex/contrib/oberdiek/telprint.dtx
%    CTAN:macros/latex/contrib/oberdiek/telprint.pdf
%
% Unpacking:
%    (a) If telprint.ins is present:
%           tex telprint.ins
%    (b) Without telprint.ins:
%           tex telprint.dtx
%    (c) If you insist on using LaTeX
%           latex \let\install=y\input{telprint.dtx}
%        (quote the arguments according to the demands of your shell)
%
% Documentation:
%    (a) If telprint.drv is present:
%           latex telprint.drv
%    (b) Without telprint.drv:
%           latex telprint.dtx; ...
%    The class ltxdoc loads the configuration file ltxdoc.cfg
%    if available. Here you can specify further options, e.g.
%    use A4 as paper format:
%       \PassOptionsToClass{a4paper}{article}
%
%    Programm calls to get the documentation (example):
%       pdflatex telprint.dtx
%       makeindex -s gind.ist telprint.idx
%       pdflatex telprint.dtx
%       makeindex -s gind.ist telprint.idx
%       pdflatex telprint.dtx
%
% Installation:
%    TDS:tex/generic/oberdiek/telprint.sty
%    TDS:doc/latex/oberdiek/telprint.pdf
%    TDS:doc/latex/oberdiek/test/telprint-test1.tex
%    TDS:source/latex/oberdiek/telprint.dtx
%
%<*ignore>
\begingroup
  \catcode123=1 %
  \catcode125=2 %
  \def\x{LaTeX2e}%
\expandafter\endgroup
\ifcase 0\ifx\install y1\fi\expandafter
         \ifx\csname processbatchFile\endcsname\relax\else1\fi
         \ifx\fmtname\x\else 1\fi\relax
\else\csname fi\endcsname
%</ignore>
%<*install>
\input docstrip.tex
\Msg{************************************************************************}
\Msg{* Installation}
\Msg{* Package: telprint 2016/05/16 v1.11 Format German phone numbers (HO)}
\Msg{************************************************************************}

\keepsilent
\askforoverwritefalse

\let\MetaPrefix\relax
\preamble

This is a generated file.

Project: telprint
Version: 2016/05/16 v1.11

Copyright (C) 1996, 1997, 2004-2008 by
   Heiko Oberdiek <heiko.oberdiek at googlemail.com>

This work may be distributed and/or modified under the
conditions of the LaTeX Project Public License, either
version 1.3c of this license or (at your option) any later
version. This version of this license is in
   http://www.latex-project.org/lppl/lppl-1-3c.txt
and the latest version of this license is in
   http://www.latex-project.org/lppl.txt
and version 1.3 or later is part of all distributions of
LaTeX version 2005/12/01 or later.

This work has the LPPL maintenance status "maintained".

This Current Maintainer of this work is Heiko Oberdiek.

The Base Interpreter refers to any `TeX-Format',
because some files are installed in TDS:tex/generic//.

This work consists of the main source file telprint.dtx
and the derived files
   telprint.sty, telprint.pdf, telprint.ins, telprint.drv,
   telprint-test1.tex.

\endpreamble
\let\MetaPrefix\DoubleperCent

\generate{%
  \file{telprint.ins}{\from{telprint.dtx}{install}}%
  \file{telprint.drv}{\from{telprint.dtx}{driver}}%
  \usedir{tex/generic/oberdiek}%
  \file{telprint.sty}{\from{telprint.dtx}{package}}%
%  \usedir{doc/latex/oberdiek/test}%
%  \file{telprint-test1.tex}{\from{telprint.dtx}{test1}}%
  \nopreamble
  \nopostamble
%  \usedir{source/latex/oberdiek/catalogue}%
%  \file{telprint.xml}{\from{telprint.dtx}{catalogue}}%
}

\catcode32=13\relax% active space
\let =\space%
\Msg{************************************************************************}
\Msg{*}
\Msg{* To finish the installation you have to move the following}
\Msg{* file into a directory searched by TeX:}
\Msg{*}
\Msg{*     telprint.sty}
\Msg{*}
\Msg{* To produce the documentation run the file `telprint.drv'}
\Msg{* through LaTeX.}
\Msg{*}
\Msg{* Happy TeXing!}
\Msg{*}
\Msg{************************************************************************}

\endbatchfile
%</install>
%<*ignore>
\fi
%</ignore>
%<*driver>
\NeedsTeXFormat{LaTeX2e}
\ProvidesFile{telprint.drv}%
  [2016/05/16 v1.11 Format German phone numbers (HO)]%
\documentclass{ltxdoc}
\usepackage{holtxdoc}[2011/11/22]
\usepackage[ngerman,english]{babel}
\begin{document}
  \DocInput{telprint.dtx}%
\end{document}
%</driver>
% \fi
%
%
% \CharacterTable
%  {Upper-case    \A\B\C\D\E\F\G\H\I\J\K\L\M\N\O\P\Q\R\S\T\U\V\W\X\Y\Z
%   Lower-case    \a\b\c\d\e\f\g\h\i\j\k\l\m\n\o\p\q\r\s\t\u\v\w\x\y\z
%   Digits        \0\1\2\3\4\5\6\7\8\9
%   Exclamation   \!     Double quote  \"     Hash (number) \#
%   Dollar        \$     Percent       \%     Ampersand     \&
%   Acute accent  \'     Left paren    \(     Right paren   \)
%   Asterisk      \*     Plus          \+     Comma         \,
%   Minus         \-     Point         \.     Solidus       \/
%   Colon         \:     Semicolon     \;     Less than     \<
%   Equals        \=     Greater than  \>     Question mark \?
%   Commercial at \@     Left bracket  \[     Backslash     \\
%   Right bracket \]     Circumflex    \^     Underscore    \_
%   Grave accent  \`     Left brace    \{     Vertical bar  \|
%   Right brace   \}     Tilde         \~}
%
% \GetFileInfo{telprint.drv}
%
% \title{The \xpackage{telprint} package}
% \date{2016/05/16 v1.11}
% \author{Heiko Oberdiek\thanks
% {Please report any issues at https://github.com/ho-tex/oberdiek/issues}\\
% \xemail{heiko.oberdiek at googlemail.com}}
%
% \maketitle
%
% \begin{abstract}
% Package \xpackage{telprint} provides \cs{telprint} for formatting
% German phone numbers.
% \end{abstract}
%
% \tableofcontents
%
% \section{Documentation}
%
% \subsection{Introduction}
%
%            This is a very old package that I have written
%            to format phone numbers. It follows German
%            conventions and the documentation is mainly in German.
%
% \subsection{Short overview in English}
%
% \LaTeX:
% \begin{quote}
% |\usepackage{telprint}|\\
% |\telprint{123/456-789}|\\
% \end{quote}
% \plainTeX:
% \begin{quote}
%   |\input telprint.sty|\\
%   |\telprint{123/456-789}|
% \end{quote}
%
% \DescribeMacro\telprint
% |\telprint{...}| formats the explicitly given number.
%     Digits, spaces and some special characters
%     ('+', '/', '-', '(', ')', '\textasciitilde', ' ') are supported.
%     Numbers are divided into groups of two digits from the right.
% Examples:
% \begin{quote}
%     |\telprint{0761/12345}     ==> 07\,61/1\,23\,45|\\
%     |\telprint{01234/567-89}   ==> 0\,12\,34/5\,67\leavevmode\hbox{-}89|\\
%     |\telprint{+49 (6221) 297} ==> +49~(62\,21)~2\,97|
% \end{quote}
%
% \subsubsection{Configuration}
%
% The output of the symbols can be configured by
% \cs{telhyphen}, \cs{telslash}, \cs{telleftparen}, \cs{telrightparen},
% \cs{telplus}, \cs{teltilde}.
% Example:
% \begin{quote}
%   |\telslash{\,/\,}\\|
%   |\telprint{12/34} ==> 12\,/\,34|
% \end{quote}
%
% \DescribeMacro\telspace
% \cs{telspace} configures the space between digit groups.
%
% \DescribeMacro\telnumber
% \cs{telnumber} only formats a number in digit groups; special
%    characters are not recognized.
%
% \subsection{Documentation in German}
%
% \begin{otherlanguage*}{ngerman}
% \hyphenation{To-ken-ma-kros}
% \begin{itemize}
% \item \DescribeMacro\telprint |telprint#1|\\
%   Der eigentliche Anwenderbefehl zur formatierten Ausgabe von
%   Telefonnummern. Diese d\"urfen dabei nur als Zahlen angegeben
%   werden(, da sie tokenweise analysiert werden).
%   Als Trenn- oder Sonderzeichen werden unterst\"utzt:
%   '+', '/', '-', '(', ')', '\textasciitilde', ' '
%   Einfache Leerzeichen werden erkannt und durch Tilden ersetzt, um
%   Trennungen in der Telefonnummer zu verhindern. (Man beachte aus
%   gleichem Grunde die \cs{hbox} bei '-'.)
%   Beispiele:
%   \begin{quote}
%     |\telprint{0761/12345}     ==> 07\,61/1\,23\,45|\\
%     |\telprint{01234/567-89}   ==> 0\,12\,34/5\,67\leavevmode\hbox{-}89|\\
%     |\telprint{+49 (6221) 297} ==> +49~(62\,21)~2\,97|
%   \end{quote}
% \end{itemize}
% Der Rest enth\"alt eher Technisches:
% \begin{itemize}
% \item \DescribeMacro\telspace |\telspace#1|\\
%   Mit diesem Befehl wird der Abstand zwischen den Zifferngruppen
%   angegeben (Default: |\,|).
%   (Durch |\telspace{}| kann dieser zusaetzliche Abstand abgestellt
%   werden.)
% \item \DescribeMacro\telhyphen |\telhyphen#1|\\
%   Dieser Befehl gibt die Art des Bindestriches, wie er ausgegeben
%   werden soll. In der Eingabe darf jedoch nur der einfache
%   Bindestrich stehen:
%   |\telprint{123-45}|, jedoch NIE |\telprint{123--45}|!
%   Kopka-Bindestrich-Fans geben an:
%   |\telhyphen{\leavevmode\hbox{--}}|
% \item
%   \DescribeMacro{\telslash}
%   \DescribeMacro{\telleftparen}
%   \DescribeMacro{\telrightparen}
%   \DescribeMacro{\telplus}
%   \DescribeMacro{\teltilde}
%   |\telslash#1|, |\telleftparen#1|, |\telrightparen#1|, |\telplus#1|,
%   |\teltilde|\\
%   Diese Befehle konfigurieren die Zeichen '/', '(', ')', '+'
%   und '\textasciitilde'. Sie funktionieren analog zu \cs{telhyphen}.
% \item \DescribeMacro\telnumber |\telnumber#1|\\
%   Richtung interner Befehl: Er dient dazu, eine Zifferngruppe
%   in Zweiergruppen auszugeben.
%   Die einzelnen Zahlen werden im Tokenregister \cs{TELtoks}
%   gespeichert. Abwechselnd werden dabei zwischen zwei Token
%   (Zahlen) \cs{TELx} bzw. \cs{TELy} eingefuegt, abh\"angig von dem
%   wechselnden Wert von \cs{TELswitch}. Zum Schluss kann dann einfach
%   festgestellt werden ob die Nummer nun eine geradzahlige oder
%   ungeradzahlige Zahl von Ziffern aufwies. Dem entsprechend wird
%   \cs{TELx} mit dem Zusatzabstand belegt und \cs{TELy} leer definiert
%   oder umgekehrt. )
% \item |\TEL...| interne Befehle, Technisches:\\
%   \cs{TELsplit} dient zur Aufteilung einer zusammengesetzten
%   Telefonnummer (Vorwahl, Hauptnummer, Nebenstelle). In dieser
%   Implementation werden als Trennzeichen nur '/' und '-' erkannt.
%   Die einzelnen Bestandteile wie Vorwahl werden dann dem Befehl
%   \cs{telnumber} zur Formatierung uebergeben.
% \item Die Erkennung von einfachen Leerzeichen ist um einiges
%   schwieriger: Die Tokentrennung ueber Parameter |#1#2| funktioniert
%   nicht f\"ur einfache Leerzeichen, da TeX sie \emph{niemals} als
%   eigenst\"andige Argumente behandelt! (The TeXbook, Chapter 20,
%   p. 201)
%
%   (Anmerkung am Rande: Deshalb funktionieren die entsprechenden
%   Tokenmakros auf S. 149 des Buches "`Einf\"uhrung in TeX"' von
%   N. Schwarz (3. Aufl.) nicht, wenn im Tokenregister als erstes
%   ein einfaches Leerzeichen steht!)
% \end{itemize}
% \end{otherlanguage*}
%
% \StopEventually{
% }
%
% \section{Implementation}
%
%    \begin{macrocode}
%<*package>
%    \end{macrocode}
%
% \subsection{Reload check and package identification}
%    Reload check, especially if the package is not used with \LaTeX.
%    \begin{macrocode}
\begingroup\catcode61\catcode48\catcode32=10\relax%
  \catcode13=5 % ^^M
  \endlinechar=13 %
  \catcode35=6 % #
  \catcode39=12 % '
  \catcode44=12 % ,
  \catcode45=12 % -
  \catcode46=12 % .
  \catcode58=12 % :
  \catcode64=11 % @
  \catcode123=1 % {
  \catcode125=2 % }
  \expandafter\let\expandafter\x\csname ver@telprint.sty\endcsname
  \ifx\x\relax % plain-TeX, first loading
  \else
    \def\empty{}%
    \ifx\x\empty % LaTeX, first loading,
      % variable is initialized, but \ProvidesPackage not yet seen
    \else
      \expandafter\ifx\csname PackageInfo\endcsname\relax
        \def\x#1#2{%
          \immediate\write-1{Package #1 Info: #2.}%
        }%
      \else
        \def\x#1#2{\PackageInfo{#1}{#2, stopped}}%
      \fi
      \x{telprint}{The package is already loaded}%
      \aftergroup\endinput
    \fi
  \fi
\endgroup%
%    \end{macrocode}
%    Package identification:
%    \begin{macrocode}
\begingroup\catcode61\catcode48\catcode32=10\relax%
  \catcode13=5 % ^^M
  \endlinechar=13 %
  \catcode35=6 % #
  \catcode39=12 % '
  \catcode40=12 % (
  \catcode41=12 % )
  \catcode44=12 % ,
  \catcode45=12 % -
  \catcode46=12 % .
  \catcode47=12 % /
  \catcode58=12 % :
  \catcode64=11 % @
  \catcode91=12 % [
  \catcode93=12 % ]
  \catcode123=1 % {
  \catcode125=2 % }
  \expandafter\ifx\csname ProvidesPackage\endcsname\relax
    \def\x#1#2#3[#4]{\endgroup
      \immediate\write-1{Package: #3 #4}%
      \xdef#1{#4}%
    }%
  \else
    \def\x#1#2[#3]{\endgroup
      #2[{#3}]%
      \ifx#1\@undefined
        \xdef#1{#3}%
      \fi
      \ifx#1\relax
        \xdef#1{#3}%
      \fi
    }%
  \fi
\expandafter\x\csname ver@telprint.sty\endcsname
\ProvidesPackage{telprint}%
  [2016/05/16 v1.11 Format German phone numbers (HO)]%
%    \end{macrocode}
%
% \subsection{Catcodes}
%
%    \begin{macrocode}
\begingroup\catcode61\catcode48\catcode32=10\relax%
  \catcode13=5 % ^^M
  \endlinechar=13 %
  \catcode123=1 % {
  \catcode125=2 % }
  \catcode64=11 % @
  \def\x{\endgroup
    \expandafter\edef\csname TELAtEnd\endcsname{%
      \endlinechar=\the\endlinechar\relax
      \catcode13=\the\catcode13\relax
      \catcode32=\the\catcode32\relax
      \catcode35=\the\catcode35\relax
      \catcode61=\the\catcode61\relax
      \catcode64=\the\catcode64\relax
      \catcode123=\the\catcode123\relax
      \catcode125=\the\catcode125\relax
    }%
  }%
\x\catcode61\catcode48\catcode32=10\relax%
\catcode13=5 % ^^M
\endlinechar=13 %
\catcode35=6 % #
\catcode64=11 % @
\catcode123=1 % {
\catcode125=2 % }
\def\TMP@EnsureCode#1#2{%
  \edef\TELAtEnd{%
    \TELAtEnd
    \catcode#1=\the\catcode#1\relax
  }%
  \catcode#1=#2\relax
}
\TMP@EnsureCode{33}{12}% !
\TMP@EnsureCode{36}{3}% $
\TMP@EnsureCode{40}{12}% (
\TMP@EnsureCode{41}{12}% )
\TMP@EnsureCode{42}{12}% *
\TMP@EnsureCode{43}{12}% +
\TMP@EnsureCode{44}{12}% ,
\TMP@EnsureCode{45}{12}% -
\TMP@EnsureCode{46}{12}% .
\TMP@EnsureCode{47}{12}% /
\TMP@EnsureCode{91}{12}% [
\TMP@EnsureCode{93}{12}% ]
\TMP@EnsureCode{126}{13}% ~ (active)
\edef\TELAtEnd{\TELAtEnd\noexpand\endinput}
%    \end{macrocode}
%
% \subsection{Package macros}
%    \begin{macrocode}
\ifx\DeclareRobustCommand\UnDeFiNeD
  \def\DeclareRobustCommand*#1[1]{\def#1##1}%
  \def\TELreset{\let\DeclareRobustCommand=\UnDeFiNeD}%
  \input infwarerr.sty\relax
  \@PackageInfo{telprint}{%
    Macros are not robust!%
  }%
\else
  \let\TELreset=\relax
\fi
%    \end{macrocode}
%    \begin{macro}{\telspace}
%    \begin{macrocode}
\DeclareRobustCommand*{\telspace}[1]{\def\TELspace{#1}}
\telspace{{}$\,${}}
%    \end{macrocode}
%    \end{macro}
%    \begin{macro}{\telhyphen}
%    \begin{macrocode}
\DeclareRobustCommand*{\telhyphen}[1]{\def\TELhyphen{#1}}
\telhyphen{\leavevmode\hbox{-}}% \hbox zur Verhinderung der Trennung
%    \end{macrocode}
%    \end{macro}
%    \begin{macro}{\telslash}
%    \begin{macrocode}
\DeclareRobustCommand*{\telslash}[1]{\def\TELslash{#1}}
\telslash{/}%
%    \end{macrocode}
%    \end{macro}
%    \begin{macro}{\telleftparen}
%    \begin{macrocode}
\DeclareRobustCommand*{\telleftparen}[1]{\def\TELleftparen{#1}}
\telleftparen{(}%
%    \end{macrocode}
%    \end{macro}
%    \begin{macro}{\telrightparen}
%    \begin{macrocode}
\DeclareRobustCommand*{\telrightparen}[1]{\def\TELrightparen{#1}}
\telrightparen{)}%
%    \end{macrocode}
%    \end{macro}
%    \begin{macro}{\telplus}
%    \begin{macrocode}
\DeclareRobustCommand*{\telplus}[1]{\def\TELplus{#1}}
\telplus{+}%
%    \end{macrocode}
%    \end{macro}
%    \begin{macro}{\teltilde}
%    \begin{macrocode}
\DeclareRobustCommand*{\teltilde}[1]{\def\TELtilde{#1}}
\teltilde{~}%
%    \end{macrocode}
%    \end{macro}
%    \begin{macro}{\TELtoks}
%    \begin{macrocode}
\newtoks\TELtoks
%    \end{macrocode}
%    \end{macro}
%    \begin{macro}{\TELnumber}
%    \begin{macrocode}
\def\TELnumber#1#2\TELnumberEND{%
  \begingroup
  \def\0{#2}%
  \expandafter\endgroup
  \ifx\0\empty
    \TELtoks=\expandafter{\the\TELtoks#1}%
    \ifnum\TELswitch=0 %
      \def\TELx{\TELspace}\def\TELy{}%
    \else
      \def\TELx{}\def\TELy{\TELspace}%
    \fi
    \the\TELtoks
  \else
    \ifnum\TELswitch=0 %
      \TELtoks=\expandafter{\the\TELtoks#1\TELx}%
      \def\TELswitch{1}%
    \else
      \TELtoks=\expandafter{\the\TELtoks#1\TELy}%
      \def\TELswitch{0}%
    \fi
    \TELnumber#2\TELnumberEND
  \fi
}
%    \end{macrocode}
%    \end{macro}
%    \begin{macro}{\telnumber}
%    \begin{macrocode}
\DeclareRobustCommand*{\telnumber}[1]{%
  \TELtoks={}%
  \def\TELswitch{0}%
  \TELnumber#1{}\TELnumberEND
}
%    \end{macrocode}
%    \end{macro}
%    \begin{macro}{\TELsplit}
%    \begin{macrocode}
\def\TELsplit{\futurelet\TELfuture\TELdosplit}
%    \end{macrocode}
%    \end{macro}
%    \begin{macro}{\TELdosplit}
%    \begin{macrocode}
\def\TELdosplit#1#2\TELsplitEND
{%
  \def\TELsp{ }%
  \expandafter\ifx\TELsp\TELfuture
    \let\TELfuture=\relax
    \expandafter\telnumber\expandafter{\the\TELtoks}~%
    \telprint{#1#2}% Das Leerzeichen kann nicht #1 sein!
  \else
    \def\TELfirst{#1}%
    \ifx\TELfirst\empty
      \expandafter\telnumber\expandafter{\the\TELtoks}%
      \TELtoks={}%
    \else\if-\TELfirst
      \expandafter\telnumber\expandafter{\the\TELtoks}\TELhyphen
      \telprint{#2}%
    \else\if/\TELfirst
      \expandafter\telnumber\expandafter{\the\TELtoks}\TELslash
      \telprint{#2}%
    \else\if(\TELfirst
      \expandafter\telnumber\expandafter{\the\TELtoks}\TELleftparen
      \telprint{#2}%
    \else\if)\TELfirst
      \expandafter\telnumber\expandafter{\the\TELtoks}\TELrightparen
      \telprint{#2}%
    \else\if+\TELfirst
      \expandafter\telnumber\expandafter{\the\TELtoks}\TELplus
      \telprint{#2}%
    \else\def\TELtemp{~}\ifx\TELtemp\TELfirst
      \expandafter\telnumber\expandafter{\the\TELtoks}\TELtilde
      \telprint{#2}%
    \else
      \TELtoks=\expandafter{\the\TELtoks#1}%
      \TELsplit#2{}\TELsplitEND
    \fi\fi\fi\fi\fi\fi\fi
  \fi
}
%    \end{macrocode}
%    \end{macro}
%    \begin{macro}{\telprint}
%    \begin{macrocode}
\DeclareRobustCommand*{\telprint}[1]{%
  \TELtoks={}%
  \TELsplit#1{}\TELsplitEND
}
%    \end{macrocode}
%    \end{macro}
%    \begin{macrocode}
\TELreset\let\TELreset=\UnDeFiNeD
%    \end{macrocode}
%
%    \begin{macrocode}
\TELAtEnd%
%</package>
%    \end{macrocode}
%
% \section{Test}
%
% \subsection{Catcode checks for loading}
%
%    \begin{macrocode}
%<*test1>
%    \end{macrocode}
%    \begin{macrocode}
\catcode`\{=1 %
\catcode`\}=2 %
\catcode`\#=6 %
\catcode`\@=11 %
\expandafter\ifx\csname count@\endcsname\relax
  \countdef\count@=255 %
\fi
\expandafter\ifx\csname @gobble\endcsname\relax
  \long\def\@gobble#1{}%
\fi
\expandafter\ifx\csname @firstofone\endcsname\relax
  \long\def\@firstofone#1{#1}%
\fi
\expandafter\ifx\csname loop\endcsname\relax
  \expandafter\@firstofone
\else
  \expandafter\@gobble
\fi
{%
  \def\loop#1\repeat{%
    \def\body{#1}%
    \iterate
  }%
  \def\iterate{%
    \body
      \let\next\iterate
    \else
      \let\next\relax
    \fi
    \next
  }%
  \let\repeat=\fi
}%
\def\RestoreCatcodes{}
\count@=0 %
\loop
  \edef\RestoreCatcodes{%
    \RestoreCatcodes
    \catcode\the\count@=\the\catcode\count@\relax
  }%
\ifnum\count@<255 %
  \advance\count@ 1 %
\repeat

\def\RangeCatcodeInvalid#1#2{%
  \count@=#1\relax
  \loop
    \catcode\count@=15 %
  \ifnum\count@<#2\relax
    \advance\count@ 1 %
  \repeat
}
\def\RangeCatcodeCheck#1#2#3{%
  \count@=#1\relax
  \loop
    \ifnum#3=\catcode\count@
    \else
      \errmessage{%
        Character \the\count@\space
        with wrong catcode \the\catcode\count@\space
        instead of \number#3%
      }%
    \fi
  \ifnum\count@<#2\relax
    \advance\count@ 1 %
  \repeat
}
\def\space{ }
\expandafter\ifx\csname LoadCommand\endcsname\relax
  \def\LoadCommand{\input telprint.sty\relax}%
\fi
\def\Test{%
  \RangeCatcodeInvalid{0}{47}%
  \RangeCatcodeInvalid{58}{64}%
  \RangeCatcodeInvalid{91}{96}%
  \RangeCatcodeInvalid{123}{255}%
  \catcode`\@=12 %
  \catcode`\\=0 %
  \catcode`\%=14 %
  \LoadCommand
  \RangeCatcodeCheck{0}{36}{15}%
  \RangeCatcodeCheck{37}{37}{14}%
  \RangeCatcodeCheck{38}{47}{15}%
  \RangeCatcodeCheck{48}{57}{12}%
  \RangeCatcodeCheck{58}{63}{15}%
  \RangeCatcodeCheck{64}{64}{12}%
  \RangeCatcodeCheck{65}{90}{11}%
  \RangeCatcodeCheck{91}{91}{15}%
  \RangeCatcodeCheck{92}{92}{0}%
  \RangeCatcodeCheck{93}{96}{15}%
  \RangeCatcodeCheck{97}{122}{11}%
  \RangeCatcodeCheck{123}{255}{15}%
  \RestoreCatcodes
}
\Test
\csname @@end\endcsname
\end
%    \end{macrocode}
%    \begin{macrocode}
%</test1>
%    \end{macrocode}
%
% \section{Installation}
%
% \subsection{Download}
%
% \paragraph{Package.} This package is available on
% CTAN\footnote{\url{https://ctan.org/pkg/telprint}}:
% \begin{description}
% \item[\CTAN{macros/latex/contrib/oberdiek/telprint.dtx}] The source file.
% \item[\CTAN{macros/latex/contrib/oberdiek/telprint.pdf}] Documentation.
% \end{description}
%
%
% \paragraph{Bundle.} All the packages of the bundle `oberdiek'
% are also available in a TDS compliant ZIP archive. There
% the packages are already unpacked and the documentation files
% are generated. The files and directories obey the TDS standard.
% \begin{description}
% \item[\CTANinstall{install/macros/latex/contrib/oberdiek.tds.zip}]
% \end{description}
% \emph{TDS} refers to the standard ``A Directory Structure
% for \TeX\ Files'' (\CTAN{tds/tds.pdf}). Directories
% with \xfile{texmf} in their name are usually organized this way.
%
% \subsection{Bundle installation}
%
% \paragraph{Unpacking.} Unpack the \xfile{oberdiek.tds.zip} in the
% TDS tree (also known as \xfile{texmf} tree) of your choice.
% Example (linux):
% \begin{quote}
%   |unzip oberdiek.tds.zip -d ~/texmf|
% \end{quote}
%
% \paragraph{Script installation.}
% Check the directory \xfile{TDS:scripts/oberdiek/} for
% scripts that need further installation steps.
% Package \xpackage{attachfile2} comes with the Perl script
% \xfile{pdfatfi.pl} that should be installed in such a way
% that it can be called as \texttt{pdfatfi}.
% Example (linux):
% \begin{quote}
%   |chmod +x scripts/oberdiek/pdfatfi.pl|\\
%   |cp scripts/oberdiek/pdfatfi.pl /usr/local/bin/|
% \end{quote}
%
% \subsection{Package installation}
%
% \paragraph{Unpacking.} The \xfile{.dtx} file is a self-extracting
% \docstrip\ archive. The files are extracted by running the
% \xfile{.dtx} through \plainTeX:
% \begin{quote}
%   \verb|tex telprint.dtx|
% \end{quote}
%
% \paragraph{TDS.} Now the different files must be moved into
% the different directories in your installation TDS tree
% (also known as \xfile{texmf} tree):
% \begin{quote}
% \def\t{^^A
% \begin{tabular}{@{}>{\ttfamily}l@{ $\rightarrow$ }>{\ttfamily}l@{}}
%   telprint.sty & tex/generic/oberdiek/telprint.sty\\
%   telprint.pdf & doc/latex/oberdiek/telprint.pdf\\
%   test/telprint-test1.tex & doc/latex/oberdiek/test/telprint-test1.tex\\
%   telprint.dtx & source/latex/oberdiek/telprint.dtx\\
% \end{tabular}^^A
% }^^A
% \sbox0{\t}^^A
% \ifdim\wd0>\linewidth
%   \begingroup
%     \advance\linewidth by\leftmargin
%     \advance\linewidth by\rightmargin
%   \edef\x{\endgroup
%     \def\noexpand\lw{\the\linewidth}^^A
%   }\x
%   \def\lwbox{^^A
%     \leavevmode
%     \hbox to \linewidth{^^A
%       \kern-\leftmargin\relax
%       \hss
%       \usebox0
%       \hss
%       \kern-\rightmargin\relax
%     }^^A
%   }^^A
%   \ifdim\wd0>\lw
%     \sbox0{\small\t}^^A
%     \ifdim\wd0>\linewidth
%       \ifdim\wd0>\lw
%         \sbox0{\footnotesize\t}^^A
%         \ifdim\wd0>\linewidth
%           \ifdim\wd0>\lw
%             \sbox0{\scriptsize\t}^^A
%             \ifdim\wd0>\linewidth
%               \ifdim\wd0>\lw
%                 \sbox0{\tiny\t}^^A
%                 \ifdim\wd0>\linewidth
%                   \lwbox
%                 \else
%                   \usebox0
%                 \fi
%               \else
%                 \lwbox
%               \fi
%             \else
%               \usebox0
%             \fi
%           \else
%             \lwbox
%           \fi
%         \else
%           \usebox0
%         \fi
%       \else
%         \lwbox
%       \fi
%     \else
%       \usebox0
%     \fi
%   \else
%     \lwbox
%   \fi
% \else
%   \usebox0
% \fi
% \end{quote}
% If you have a \xfile{docstrip.cfg} that configures and enables \docstrip's
% TDS installing feature, then some files can already be in the right
% place, see the documentation of \docstrip.
%
% \subsection{Refresh file name databases}
%
% If your \TeX~distribution
% (\teTeX, \mikTeX, \dots) relies on file name databases, you must refresh
% these. For example, \teTeX\ users run \verb|texhash| or
% \verb|mktexlsr|.
%
% \subsection{Some details for the interested}
%
% \paragraph{Attached source.}
%
% The PDF documentation on CTAN also includes the
% \xfile{.dtx} source file. It can be extracted by
% AcrobatReader 6 or higher. Another option is \textsf{pdftk},
% e.g. unpack the file into the current directory:
% \begin{quote}
%   \verb|pdftk telprint.pdf unpack_files output .|
% \end{quote}
%
% \paragraph{Unpacking with \LaTeX.}
% The \xfile{.dtx} chooses its action depending on the format:
% \begin{description}
% \item[\plainTeX:] Run \docstrip\ and extract the files.
% \item[\LaTeX:] Generate the documentation.
% \end{description}
% If you insist on using \LaTeX\ for \docstrip\ (really,
% \docstrip\ does not need \LaTeX), then inform the autodetect routine
% about your intention:
% \begin{quote}
%   \verb|latex \let\install=y\input{telprint.dtx}|
% \end{quote}
% Do not forget to quote the argument according to the demands
% of your shell.
%
% \paragraph{Generating the documentation.}
% You can use both the \xfile{.dtx} or the \xfile{.drv} to generate
% the documentation. The process can be configured by the
% configuration file \xfile{ltxdoc.cfg}. For instance, put this
% line into this file, if you want to have A4 as paper format:
% \begin{quote}
%   \verb|\PassOptionsToClass{a4paper}{article}|
% \end{quote}
% An example follows how to generate the
% documentation with pdf\LaTeX:
% \begin{quote}
%\begin{verbatim}
%pdflatex telprint.dtx
%makeindex -s gind.ist telprint.idx
%pdflatex telprint.dtx
%makeindex -s gind.ist telprint.idx
%pdflatex telprint.dtx
%\end{verbatim}
% \end{quote}
%
% \begin{History}
%   \begin{Version}{1996/11/28 v1.0}
%   \item
%     Erste lauff\"ahige Version.
%   \item
%     Nur '-' und '/' als zul\"assige Sonderzeichen.
%   \end{Version}
%   \begin{Version}{1997/09/16 v1.1}
%   \item
%     Dokumentation und Kommentare (Posting in de.comp.text.tex).
%   \item
%     Erweiterung um Sonderzeichen '(', ')', '+', '\textasciitilde' und ' '.
%   \item
%     Trennungsverhinderung am 'hyphen'.
%   \end{Version}
%   \begin{Version}{1997/10/16 v1.2}
%   \item
%     Schutz vor wiederholtem Einlesen.
%   \item
%     Unter \LaTeXe\ Nutzung des \cs{DeclareRobustCommand}-Features.
%   \end{Version}
%   \begin{Version}{1997/12/09 v1.3}
%   \item
%     Tempor\"are Variable eingespart.
%   \item
%     Posted in newsgroup \xnewsgroup{de.comp.text.tex}:\\
%     \URL{``\link{Re: Generisches Markup f\"ur Telefonnummern?}''}^^A
%     {http://groups.google.com/group/de.comp.text.tex/msg/86b3a86140007309}
%   \end{Version}
%   \begin{Version}{2004/11/02 v1.4}
%   \item
%     Fehler in der Dokumentation korrigiert.
%   \end{Version}
%   \begin{Version}{2005/09/30 v1.5}
%   \item
%     Konfigurierbare Symbole: '/', '(', ')', '+' und '\textasciitilde'.
%   \end{Version}
%   \begin{Version}{2006/02/12 v1.6}
%   \item
%     LPPL 1.3.
%   \item
%     Kurze \"Ubersicht in Englisch.
%   \item
%     CTAN.
%   \end{Version}
%   \begin{Version}{2006/08/26 v1.7}
%   \item
%     New DTX framework.
%   \end{Version}
%   \begin{Version}{2007/04/11 v1.8}
%   \item
%     Line ends sanitized.
%   \end{Version}
%   \begin{Version}{2007/09/09 v1.9}
%   \item
%     Catcode section added.
%   \item
%     Missing docstrip tag added.
%   \end{Version}
%   \begin{Version}{2008/08/11 v1.10}
%   \item
%     Code is not changed.
%   \item
%     URLs updated.
%   \end{Version}
%   \begin{Version}{2016/05/16 v1.11}
%   \item
%     Documentation updates.
%   \end{Version}
% \end{History}
%
% \PrintIndex
%
% \Finale
\endinput

%        (quote the arguments according to the demands of your shell)
%
% Documentation:
%    (a) If telprint.drv is present:
%           latex telprint.drv
%    (b) Without telprint.drv:
%           latex telprint.dtx; ...
%    The class ltxdoc loads the configuration file ltxdoc.cfg
%    if available. Here you can specify further options, e.g.
%    use A4 as paper format:
%       \PassOptionsToClass{a4paper}{article}
%
%    Programm calls to get the documentation (example):
%       pdflatex telprint.dtx
%       makeindex -s gind.ist telprint.idx
%       pdflatex telprint.dtx
%       makeindex -s gind.ist telprint.idx
%       pdflatex telprint.dtx
%
% Installation:
%    TDS:tex/generic/oberdiek/telprint.sty
%    TDS:doc/latex/oberdiek/telprint.pdf
%    TDS:doc/latex/oberdiek/test/telprint-test1.tex
%    TDS:source/latex/oberdiek/telprint.dtx
%
%<*ignore>
\begingroup
  \catcode123=1 %
  \catcode125=2 %
  \def\x{LaTeX2e}%
\expandafter\endgroup
\ifcase 0\ifx\install y1\fi\expandafter
         \ifx\csname processbatchFile\endcsname\relax\else1\fi
         \ifx\fmtname\x\else 1\fi\relax
\else\csname fi\endcsname
%</ignore>
%<*install>
\input docstrip.tex
\Msg{************************************************************************}
\Msg{* Installation}
\Msg{* Package: telprint 2016/05/16 v1.11 Format German phone numbers (HO)}
\Msg{************************************************************************}

\keepsilent
\askforoverwritefalse

\let\MetaPrefix\relax
\preamble

This is a generated file.

Project: telprint
Version: 2016/05/16 v1.11

Copyright (C) 1996, 1997, 2004-2008 by
   Heiko Oberdiek <heiko.oberdiek at googlemail.com>

This work may be distributed and/or modified under the
conditions of the LaTeX Project Public License, either
version 1.3c of this license or (at your option) any later
version. This version of this license is in
   http://www.latex-project.org/lppl/lppl-1-3c.txt
and the latest version of this license is in
   http://www.latex-project.org/lppl.txt
and version 1.3 or later is part of all distributions of
LaTeX version 2005/12/01 or later.

This work has the LPPL maintenance status "maintained".

This Current Maintainer of this work is Heiko Oberdiek.

The Base Interpreter refers to any `TeX-Format',
because some files are installed in TDS:tex/generic//.

This work consists of the main source file telprint.dtx
and the derived files
   telprint.sty, telprint.pdf, telprint.ins, telprint.drv,
   telprint-test1.tex.

\endpreamble
\let\MetaPrefix\DoubleperCent

\generate{%
  \file{telprint.ins}{\from{telprint.dtx}{install}}%
  \file{telprint.drv}{\from{telprint.dtx}{driver}}%
  \usedir{tex/generic/oberdiek}%
  \file{telprint.sty}{\from{telprint.dtx}{package}}%
%  \usedir{doc/latex/oberdiek/test}%
%  \file{telprint-test1.tex}{\from{telprint.dtx}{test1}}%
  \nopreamble
  \nopostamble
%  \usedir{source/latex/oberdiek/catalogue}%
%  \file{telprint.xml}{\from{telprint.dtx}{catalogue}}%
}

\catcode32=13\relax% active space
\let =\space%
\Msg{************************************************************************}
\Msg{*}
\Msg{* To finish the installation you have to move the following}
\Msg{* file into a directory searched by TeX:}
\Msg{*}
\Msg{*     telprint.sty}
\Msg{*}
\Msg{* To produce the documentation run the file `telprint.drv'}
\Msg{* through LaTeX.}
\Msg{*}
\Msg{* Happy TeXing!}
\Msg{*}
\Msg{************************************************************************}

\endbatchfile
%</install>
%<*ignore>
\fi
%</ignore>
%<*driver>
\NeedsTeXFormat{LaTeX2e}
\ProvidesFile{telprint.drv}%
  [2016/05/16 v1.11 Format German phone numbers (HO)]%
\documentclass{ltxdoc}
\usepackage{holtxdoc}[2011/11/22]
\usepackage[ngerman,english]{babel}
\begin{document}
  \DocInput{telprint.dtx}%
\end{document}
%</driver>
% \fi
%
%
% \CharacterTable
%  {Upper-case    \A\B\C\D\E\F\G\H\I\J\K\L\M\N\O\P\Q\R\S\T\U\V\W\X\Y\Z
%   Lower-case    \a\b\c\d\e\f\g\h\i\j\k\l\m\n\o\p\q\r\s\t\u\v\w\x\y\z
%   Digits        \0\1\2\3\4\5\6\7\8\9
%   Exclamation   \!     Double quote  \"     Hash (number) \#
%   Dollar        \$     Percent       \%     Ampersand     \&
%   Acute accent  \'     Left paren    \(     Right paren   \)
%   Asterisk      \*     Plus          \+     Comma         \,
%   Minus         \-     Point         \.     Solidus       \/
%   Colon         \:     Semicolon     \;     Less than     \<
%   Equals        \=     Greater than  \>     Question mark \?
%   Commercial at \@     Left bracket  \[     Backslash     \\
%   Right bracket \]     Circumflex    \^     Underscore    \_
%   Grave accent  \`     Left brace    \{     Vertical bar  \|
%   Right brace   \}     Tilde         \~}
%
% \GetFileInfo{telprint.drv}
%
% \title{The \xpackage{telprint} package}
% \date{2016/05/16 v1.11}
% \author{Heiko Oberdiek\thanks
% {Please report any issues at https://github.com/ho-tex/oberdiek/issues}\\
% \xemail{heiko.oberdiek at googlemail.com}}
%
% \maketitle
%
% \begin{abstract}
% Package \xpackage{telprint} provides \cs{telprint} for formatting
% German phone numbers.
% \end{abstract}
%
% \tableofcontents
%
% \section{Documentation}
%
% \subsection{Introduction}
%
%            This is a very old package that I have written
%            to format phone numbers. It follows German
%            conventions and the documentation is mainly in German.
%
% \subsection{Short overview in English}
%
% \LaTeX:
% \begin{quote}
% |\usepackage{telprint}|\\
% |\telprint{123/456-789}|\\
% \end{quote}
% \plainTeX:
% \begin{quote}
%   |\input telprint.sty|\\
%   |\telprint{123/456-789}|
% \end{quote}
%
% \DescribeMacro\telprint
% |\telprint{...}| formats the explicitly given number.
%     Digits, spaces and some special characters
%     ('+', '/', '-', '(', ')', '\textasciitilde', ' ') are supported.
%     Numbers are divided into groups of two digits from the right.
% Examples:
% \begin{quote}
%     |\telprint{0761/12345}     ==> 07\,61/1\,23\,45|\\
%     |\telprint{01234/567-89}   ==> 0\,12\,34/5\,67\leavevmode\hbox{-}89|\\
%     |\telprint{+49 (6221) 297} ==> +49~(62\,21)~2\,97|
% \end{quote}
%
% \subsubsection{Configuration}
%
% The output of the symbols can be configured by
% \cs{telhyphen}, \cs{telslash}, \cs{telleftparen}, \cs{telrightparen},
% \cs{telplus}, \cs{teltilde}.
% Example:
% \begin{quote}
%   |\telslash{\,/\,}\\|
%   |\telprint{12/34} ==> 12\,/\,34|
% \end{quote}
%
% \DescribeMacro\telspace
% \cs{telspace} configures the space between digit groups.
%
% \DescribeMacro\telnumber
% \cs{telnumber} only formats a number in digit groups; special
%    characters are not recognized.
%
% \subsection{Documentation in German}
%
% \begin{otherlanguage*}{ngerman}
% \hyphenation{To-ken-ma-kros}
% \begin{itemize}
% \item \DescribeMacro\telprint |telprint#1|\\
%   Der eigentliche Anwenderbefehl zur formatierten Ausgabe von
%   Telefonnummern. Diese d\"urfen dabei nur als Zahlen angegeben
%   werden(, da sie tokenweise analysiert werden).
%   Als Trenn- oder Sonderzeichen werden unterst\"utzt:
%   '+', '/', '-', '(', ')', '\textasciitilde', ' '
%   Einfache Leerzeichen werden erkannt und durch Tilden ersetzt, um
%   Trennungen in der Telefonnummer zu verhindern. (Man beachte aus
%   gleichem Grunde die \cs{hbox} bei '-'.)
%   Beispiele:
%   \begin{quote}
%     |\telprint{0761/12345}     ==> 07\,61/1\,23\,45|\\
%     |\telprint{01234/567-89}   ==> 0\,12\,34/5\,67\leavevmode\hbox{-}89|\\
%     |\telprint{+49 (6221) 297} ==> +49~(62\,21)~2\,97|
%   \end{quote}
% \end{itemize}
% Der Rest enth\"alt eher Technisches:
% \begin{itemize}
% \item \DescribeMacro\telspace |\telspace#1|\\
%   Mit diesem Befehl wird der Abstand zwischen den Zifferngruppen
%   angegeben (Default: |\,|).
%   (Durch |\telspace{}| kann dieser zusaetzliche Abstand abgestellt
%   werden.)
% \item \DescribeMacro\telhyphen |\telhyphen#1|\\
%   Dieser Befehl gibt die Art des Bindestriches, wie er ausgegeben
%   werden soll. In der Eingabe darf jedoch nur der einfache
%   Bindestrich stehen:
%   |\telprint{123-45}|, jedoch NIE |\telprint{123--45}|!
%   Kopka-Bindestrich-Fans geben an:
%   |\telhyphen{\leavevmode\hbox{--}}|
% \item
%   \DescribeMacro{\telslash}
%   \DescribeMacro{\telleftparen}
%   \DescribeMacro{\telrightparen}
%   \DescribeMacro{\telplus}
%   \DescribeMacro{\teltilde}
%   |\telslash#1|, |\telleftparen#1|, |\telrightparen#1|, |\telplus#1|,
%   |\teltilde|\\
%   Diese Befehle konfigurieren die Zeichen '/', '(', ')', '+'
%   und '\textasciitilde'. Sie funktionieren analog zu \cs{telhyphen}.
% \item \DescribeMacro\telnumber |\telnumber#1|\\
%   Richtung interner Befehl: Er dient dazu, eine Zifferngruppe
%   in Zweiergruppen auszugeben.
%   Die einzelnen Zahlen werden im Tokenregister \cs{TELtoks}
%   gespeichert. Abwechselnd werden dabei zwischen zwei Token
%   (Zahlen) \cs{TELx} bzw. \cs{TELy} eingefuegt, abh\"angig von dem
%   wechselnden Wert von \cs{TELswitch}. Zum Schluss kann dann einfach
%   festgestellt werden ob die Nummer nun eine geradzahlige oder
%   ungeradzahlige Zahl von Ziffern aufwies. Dem entsprechend wird
%   \cs{TELx} mit dem Zusatzabstand belegt und \cs{TELy} leer definiert
%   oder umgekehrt. )
% \item |\TEL...| interne Befehle, Technisches:\\
%   \cs{TELsplit} dient zur Aufteilung einer zusammengesetzten
%   Telefonnummer (Vorwahl, Hauptnummer, Nebenstelle). In dieser
%   Implementation werden als Trennzeichen nur '/' und '-' erkannt.
%   Die einzelnen Bestandteile wie Vorwahl werden dann dem Befehl
%   \cs{telnumber} zur Formatierung uebergeben.
% \item Die Erkennung von einfachen Leerzeichen ist um einiges
%   schwieriger: Die Tokentrennung ueber Parameter |#1#2| funktioniert
%   nicht f\"ur einfache Leerzeichen, da TeX sie \emph{niemals} als
%   eigenst\"andige Argumente behandelt! (The TeXbook, Chapter 20,
%   p. 201)
%
%   (Anmerkung am Rande: Deshalb funktionieren die entsprechenden
%   Tokenmakros auf S. 149 des Buches "`Einf\"uhrung in TeX"' von
%   N. Schwarz (3. Aufl.) nicht, wenn im Tokenregister als erstes
%   ein einfaches Leerzeichen steht!)
% \end{itemize}
% \end{otherlanguage*}
%
% \StopEventually{
% }
%
% \section{Implementation}
%
%    \begin{macrocode}
%<*package>
%    \end{macrocode}
%
% \subsection{Reload check and package identification}
%    Reload check, especially if the package is not used with \LaTeX.
%    \begin{macrocode}
\begingroup\catcode61\catcode48\catcode32=10\relax%
  \catcode13=5 % ^^M
  \endlinechar=13 %
  \catcode35=6 % #
  \catcode39=12 % '
  \catcode44=12 % ,
  \catcode45=12 % -
  \catcode46=12 % .
  \catcode58=12 % :
  \catcode64=11 % @
  \catcode123=1 % {
  \catcode125=2 % }
  \expandafter\let\expandafter\x\csname ver@telprint.sty\endcsname
  \ifx\x\relax % plain-TeX, first loading
  \else
    \def\empty{}%
    \ifx\x\empty % LaTeX, first loading,
      % variable is initialized, but \ProvidesPackage not yet seen
    \else
      \expandafter\ifx\csname PackageInfo\endcsname\relax
        \def\x#1#2{%
          \immediate\write-1{Package #1 Info: #2.}%
        }%
      \else
        \def\x#1#2{\PackageInfo{#1}{#2, stopped}}%
      \fi
      \x{telprint}{The package is already loaded}%
      \aftergroup\endinput
    \fi
  \fi
\endgroup%
%    \end{macrocode}
%    Package identification:
%    \begin{macrocode}
\begingroup\catcode61\catcode48\catcode32=10\relax%
  \catcode13=5 % ^^M
  \endlinechar=13 %
  \catcode35=6 % #
  \catcode39=12 % '
  \catcode40=12 % (
  \catcode41=12 % )
  \catcode44=12 % ,
  \catcode45=12 % -
  \catcode46=12 % .
  \catcode47=12 % /
  \catcode58=12 % :
  \catcode64=11 % @
  \catcode91=12 % [
  \catcode93=12 % ]
  \catcode123=1 % {
  \catcode125=2 % }
  \expandafter\ifx\csname ProvidesPackage\endcsname\relax
    \def\x#1#2#3[#4]{\endgroup
      \immediate\write-1{Package: #3 #4}%
      \xdef#1{#4}%
    }%
  \else
    \def\x#1#2[#3]{\endgroup
      #2[{#3}]%
      \ifx#1\@undefined
        \xdef#1{#3}%
      \fi
      \ifx#1\relax
        \xdef#1{#3}%
      \fi
    }%
  \fi
\expandafter\x\csname ver@telprint.sty\endcsname
\ProvidesPackage{telprint}%
  [2016/05/16 v1.11 Format German phone numbers (HO)]%
%    \end{macrocode}
%
% \subsection{Catcodes}
%
%    \begin{macrocode}
\begingroup\catcode61\catcode48\catcode32=10\relax%
  \catcode13=5 % ^^M
  \endlinechar=13 %
  \catcode123=1 % {
  \catcode125=2 % }
  \catcode64=11 % @
  \def\x{\endgroup
    \expandafter\edef\csname TELAtEnd\endcsname{%
      \endlinechar=\the\endlinechar\relax
      \catcode13=\the\catcode13\relax
      \catcode32=\the\catcode32\relax
      \catcode35=\the\catcode35\relax
      \catcode61=\the\catcode61\relax
      \catcode64=\the\catcode64\relax
      \catcode123=\the\catcode123\relax
      \catcode125=\the\catcode125\relax
    }%
  }%
\x\catcode61\catcode48\catcode32=10\relax%
\catcode13=5 % ^^M
\endlinechar=13 %
\catcode35=6 % #
\catcode64=11 % @
\catcode123=1 % {
\catcode125=2 % }
\def\TMP@EnsureCode#1#2{%
  \edef\TELAtEnd{%
    \TELAtEnd
    \catcode#1=\the\catcode#1\relax
  }%
  \catcode#1=#2\relax
}
\TMP@EnsureCode{33}{12}% !
\TMP@EnsureCode{36}{3}% $
\TMP@EnsureCode{40}{12}% (
\TMP@EnsureCode{41}{12}% )
\TMP@EnsureCode{42}{12}% *
\TMP@EnsureCode{43}{12}% +
\TMP@EnsureCode{44}{12}% ,
\TMP@EnsureCode{45}{12}% -
\TMP@EnsureCode{46}{12}% .
\TMP@EnsureCode{47}{12}% /
\TMP@EnsureCode{91}{12}% [
\TMP@EnsureCode{93}{12}% ]
\TMP@EnsureCode{126}{13}% ~ (active)
\edef\TELAtEnd{\TELAtEnd\noexpand\endinput}
%    \end{macrocode}
%
% \subsection{Package macros}
%    \begin{macrocode}
\ifx\DeclareRobustCommand\UnDeFiNeD
  \def\DeclareRobustCommand*#1[1]{\def#1##1}%
  \def\TELreset{\let\DeclareRobustCommand=\UnDeFiNeD}%
  \input infwarerr.sty\relax
  \@PackageInfo{telprint}{%
    Macros are not robust!%
  }%
\else
  \let\TELreset=\relax
\fi
%    \end{macrocode}
%    \begin{macro}{\telspace}
%    \begin{macrocode}
\DeclareRobustCommand*{\telspace}[1]{\def\TELspace{#1}}
\telspace{{}$\,${}}
%    \end{macrocode}
%    \end{macro}
%    \begin{macro}{\telhyphen}
%    \begin{macrocode}
\DeclareRobustCommand*{\telhyphen}[1]{\def\TELhyphen{#1}}
\telhyphen{\leavevmode\hbox{-}}% \hbox zur Verhinderung der Trennung
%    \end{macrocode}
%    \end{macro}
%    \begin{macro}{\telslash}
%    \begin{macrocode}
\DeclareRobustCommand*{\telslash}[1]{\def\TELslash{#1}}
\telslash{/}%
%    \end{macrocode}
%    \end{macro}
%    \begin{macro}{\telleftparen}
%    \begin{macrocode}
\DeclareRobustCommand*{\telleftparen}[1]{\def\TELleftparen{#1}}
\telleftparen{(}%
%    \end{macrocode}
%    \end{macro}
%    \begin{macro}{\telrightparen}
%    \begin{macrocode}
\DeclareRobustCommand*{\telrightparen}[1]{\def\TELrightparen{#1}}
\telrightparen{)}%
%    \end{macrocode}
%    \end{macro}
%    \begin{macro}{\telplus}
%    \begin{macrocode}
\DeclareRobustCommand*{\telplus}[1]{\def\TELplus{#1}}
\telplus{+}%
%    \end{macrocode}
%    \end{macro}
%    \begin{macro}{\teltilde}
%    \begin{macrocode}
\DeclareRobustCommand*{\teltilde}[1]{\def\TELtilde{#1}}
\teltilde{~}%
%    \end{macrocode}
%    \end{macro}
%    \begin{macro}{\TELtoks}
%    \begin{macrocode}
\newtoks\TELtoks
%    \end{macrocode}
%    \end{macro}
%    \begin{macro}{\TELnumber}
%    \begin{macrocode}
\def\TELnumber#1#2\TELnumberEND{%
  \begingroup
  \def\0{#2}%
  \expandafter\endgroup
  \ifx\0\empty
    \TELtoks=\expandafter{\the\TELtoks#1}%
    \ifnum\TELswitch=0 %
      \def\TELx{\TELspace}\def\TELy{}%
    \else
      \def\TELx{}\def\TELy{\TELspace}%
    \fi
    \the\TELtoks
  \else
    \ifnum\TELswitch=0 %
      \TELtoks=\expandafter{\the\TELtoks#1\TELx}%
      \def\TELswitch{1}%
    \else
      \TELtoks=\expandafter{\the\TELtoks#1\TELy}%
      \def\TELswitch{0}%
    \fi
    \TELnumber#2\TELnumberEND
  \fi
}
%    \end{macrocode}
%    \end{macro}
%    \begin{macro}{\telnumber}
%    \begin{macrocode}
\DeclareRobustCommand*{\telnumber}[1]{%
  \TELtoks={}%
  \def\TELswitch{0}%
  \TELnumber#1{}\TELnumberEND
}
%    \end{macrocode}
%    \end{macro}
%    \begin{macro}{\TELsplit}
%    \begin{macrocode}
\def\TELsplit{\futurelet\TELfuture\TELdosplit}
%    \end{macrocode}
%    \end{macro}
%    \begin{macro}{\TELdosplit}
%    \begin{macrocode}
\def\TELdosplit#1#2\TELsplitEND
{%
  \def\TELsp{ }%
  \expandafter\ifx\TELsp\TELfuture
    \let\TELfuture=\relax
    \expandafter\telnumber\expandafter{\the\TELtoks}~%
    \telprint{#1#2}% Das Leerzeichen kann nicht #1 sein!
  \else
    \def\TELfirst{#1}%
    \ifx\TELfirst\empty
      \expandafter\telnumber\expandafter{\the\TELtoks}%
      \TELtoks={}%
    \else\if-\TELfirst
      \expandafter\telnumber\expandafter{\the\TELtoks}\TELhyphen
      \telprint{#2}%
    \else\if/\TELfirst
      \expandafter\telnumber\expandafter{\the\TELtoks}\TELslash
      \telprint{#2}%
    \else\if(\TELfirst
      \expandafter\telnumber\expandafter{\the\TELtoks}\TELleftparen
      \telprint{#2}%
    \else\if)\TELfirst
      \expandafter\telnumber\expandafter{\the\TELtoks}\TELrightparen
      \telprint{#2}%
    \else\if+\TELfirst
      \expandafter\telnumber\expandafter{\the\TELtoks}\TELplus
      \telprint{#2}%
    \else\def\TELtemp{~}\ifx\TELtemp\TELfirst
      \expandafter\telnumber\expandafter{\the\TELtoks}\TELtilde
      \telprint{#2}%
    \else
      \TELtoks=\expandafter{\the\TELtoks#1}%
      \TELsplit#2{}\TELsplitEND
    \fi\fi\fi\fi\fi\fi\fi
  \fi
}
%    \end{macrocode}
%    \end{macro}
%    \begin{macro}{\telprint}
%    \begin{macrocode}
\DeclareRobustCommand*{\telprint}[1]{%
  \TELtoks={}%
  \TELsplit#1{}\TELsplitEND
}
%    \end{macrocode}
%    \end{macro}
%    \begin{macrocode}
\TELreset\let\TELreset=\UnDeFiNeD
%    \end{macrocode}
%
%    \begin{macrocode}
\TELAtEnd%
%</package>
%    \end{macrocode}
%
% \section{Test}
%
% \subsection{Catcode checks for loading}
%
%    \begin{macrocode}
%<*test1>
%    \end{macrocode}
%    \begin{macrocode}
\catcode`\{=1 %
\catcode`\}=2 %
\catcode`\#=6 %
\catcode`\@=11 %
\expandafter\ifx\csname count@\endcsname\relax
  \countdef\count@=255 %
\fi
\expandafter\ifx\csname @gobble\endcsname\relax
  \long\def\@gobble#1{}%
\fi
\expandafter\ifx\csname @firstofone\endcsname\relax
  \long\def\@firstofone#1{#1}%
\fi
\expandafter\ifx\csname loop\endcsname\relax
  \expandafter\@firstofone
\else
  \expandafter\@gobble
\fi
{%
  \def\loop#1\repeat{%
    \def\body{#1}%
    \iterate
  }%
  \def\iterate{%
    \body
      \let\next\iterate
    \else
      \let\next\relax
    \fi
    \next
  }%
  \let\repeat=\fi
}%
\def\RestoreCatcodes{}
\count@=0 %
\loop
  \edef\RestoreCatcodes{%
    \RestoreCatcodes
    \catcode\the\count@=\the\catcode\count@\relax
  }%
\ifnum\count@<255 %
  \advance\count@ 1 %
\repeat

\def\RangeCatcodeInvalid#1#2{%
  \count@=#1\relax
  \loop
    \catcode\count@=15 %
  \ifnum\count@<#2\relax
    \advance\count@ 1 %
  \repeat
}
\def\RangeCatcodeCheck#1#2#3{%
  \count@=#1\relax
  \loop
    \ifnum#3=\catcode\count@
    \else
      \errmessage{%
        Character \the\count@\space
        with wrong catcode \the\catcode\count@\space
        instead of \number#3%
      }%
    \fi
  \ifnum\count@<#2\relax
    \advance\count@ 1 %
  \repeat
}
\def\space{ }
\expandafter\ifx\csname LoadCommand\endcsname\relax
  \def\LoadCommand{\input telprint.sty\relax}%
\fi
\def\Test{%
  \RangeCatcodeInvalid{0}{47}%
  \RangeCatcodeInvalid{58}{64}%
  \RangeCatcodeInvalid{91}{96}%
  \RangeCatcodeInvalid{123}{255}%
  \catcode`\@=12 %
  \catcode`\\=0 %
  \catcode`\%=14 %
  \LoadCommand
  \RangeCatcodeCheck{0}{36}{15}%
  \RangeCatcodeCheck{37}{37}{14}%
  \RangeCatcodeCheck{38}{47}{15}%
  \RangeCatcodeCheck{48}{57}{12}%
  \RangeCatcodeCheck{58}{63}{15}%
  \RangeCatcodeCheck{64}{64}{12}%
  \RangeCatcodeCheck{65}{90}{11}%
  \RangeCatcodeCheck{91}{91}{15}%
  \RangeCatcodeCheck{92}{92}{0}%
  \RangeCatcodeCheck{93}{96}{15}%
  \RangeCatcodeCheck{97}{122}{11}%
  \RangeCatcodeCheck{123}{255}{15}%
  \RestoreCatcodes
}
\Test
\csname @@end\endcsname
\end
%    \end{macrocode}
%    \begin{macrocode}
%</test1>
%    \end{macrocode}
%
% \section{Installation}
%
% \subsection{Download}
%
% \paragraph{Package.} This package is available on
% CTAN\footnote{\url{https://ctan.org/pkg/telprint}}:
% \begin{description}
% \item[\CTAN{macros/latex/contrib/oberdiek/telprint.dtx}] The source file.
% \item[\CTAN{macros/latex/contrib/oberdiek/telprint.pdf}] Documentation.
% \end{description}
%
%
% \paragraph{Bundle.} All the packages of the bundle `oberdiek'
% are also available in a TDS compliant ZIP archive. There
% the packages are already unpacked and the documentation files
% are generated. The files and directories obey the TDS standard.
% \begin{description}
% \item[\CTANinstall{install/macros/latex/contrib/oberdiek.tds.zip}]
% \end{description}
% \emph{TDS} refers to the standard ``A Directory Structure
% for \TeX\ Files'' (\CTAN{tds/tds.pdf}). Directories
% with \xfile{texmf} in their name are usually organized this way.
%
% \subsection{Bundle installation}
%
% \paragraph{Unpacking.} Unpack the \xfile{oberdiek.tds.zip} in the
% TDS tree (also known as \xfile{texmf} tree) of your choice.
% Example (linux):
% \begin{quote}
%   |unzip oberdiek.tds.zip -d ~/texmf|
% \end{quote}
%
% \paragraph{Script installation.}
% Check the directory \xfile{TDS:scripts/oberdiek/} for
% scripts that need further installation steps.
% Package \xpackage{attachfile2} comes with the Perl script
% \xfile{pdfatfi.pl} that should be installed in such a way
% that it can be called as \texttt{pdfatfi}.
% Example (linux):
% \begin{quote}
%   |chmod +x scripts/oberdiek/pdfatfi.pl|\\
%   |cp scripts/oberdiek/pdfatfi.pl /usr/local/bin/|
% \end{quote}
%
% \subsection{Package installation}
%
% \paragraph{Unpacking.} The \xfile{.dtx} file is a self-extracting
% \docstrip\ archive. The files are extracted by running the
% \xfile{.dtx} through \plainTeX:
% \begin{quote}
%   \verb|tex telprint.dtx|
% \end{quote}
%
% \paragraph{TDS.} Now the different files must be moved into
% the different directories in your installation TDS tree
% (also known as \xfile{texmf} tree):
% \begin{quote}
% \def\t{^^A
% \begin{tabular}{@{}>{\ttfamily}l@{ $\rightarrow$ }>{\ttfamily}l@{}}
%   telprint.sty & tex/generic/oberdiek/telprint.sty\\
%   telprint.pdf & doc/latex/oberdiek/telprint.pdf\\
%   test/telprint-test1.tex & doc/latex/oberdiek/test/telprint-test1.tex\\
%   telprint.dtx & source/latex/oberdiek/telprint.dtx\\
% \end{tabular}^^A
% }^^A
% \sbox0{\t}^^A
% \ifdim\wd0>\linewidth
%   \begingroup
%     \advance\linewidth by\leftmargin
%     \advance\linewidth by\rightmargin
%   \edef\x{\endgroup
%     \def\noexpand\lw{\the\linewidth}^^A
%   }\x
%   \def\lwbox{^^A
%     \leavevmode
%     \hbox to \linewidth{^^A
%       \kern-\leftmargin\relax
%       \hss
%       \usebox0
%       \hss
%       \kern-\rightmargin\relax
%     }^^A
%   }^^A
%   \ifdim\wd0>\lw
%     \sbox0{\small\t}^^A
%     \ifdim\wd0>\linewidth
%       \ifdim\wd0>\lw
%         \sbox0{\footnotesize\t}^^A
%         \ifdim\wd0>\linewidth
%           \ifdim\wd0>\lw
%             \sbox0{\scriptsize\t}^^A
%             \ifdim\wd0>\linewidth
%               \ifdim\wd0>\lw
%                 \sbox0{\tiny\t}^^A
%                 \ifdim\wd0>\linewidth
%                   \lwbox
%                 \else
%                   \usebox0
%                 \fi
%               \else
%                 \lwbox
%               \fi
%             \else
%               \usebox0
%             \fi
%           \else
%             \lwbox
%           \fi
%         \else
%           \usebox0
%         \fi
%       \else
%         \lwbox
%       \fi
%     \else
%       \usebox0
%     \fi
%   \else
%     \lwbox
%   \fi
% \else
%   \usebox0
% \fi
% \end{quote}
% If you have a \xfile{docstrip.cfg} that configures and enables \docstrip's
% TDS installing feature, then some files can already be in the right
% place, see the documentation of \docstrip.
%
% \subsection{Refresh file name databases}
%
% If your \TeX~distribution
% (\teTeX, \mikTeX, \dots) relies on file name databases, you must refresh
% these. For example, \teTeX\ users run \verb|texhash| or
% \verb|mktexlsr|.
%
% \subsection{Some details for the interested}
%
% \paragraph{Attached source.}
%
% The PDF documentation on CTAN also includes the
% \xfile{.dtx} source file. It can be extracted by
% AcrobatReader 6 or higher. Another option is \textsf{pdftk},
% e.g. unpack the file into the current directory:
% \begin{quote}
%   \verb|pdftk telprint.pdf unpack_files output .|
% \end{quote}
%
% \paragraph{Unpacking with \LaTeX.}
% The \xfile{.dtx} chooses its action depending on the format:
% \begin{description}
% \item[\plainTeX:] Run \docstrip\ and extract the files.
% \item[\LaTeX:] Generate the documentation.
% \end{description}
% If you insist on using \LaTeX\ for \docstrip\ (really,
% \docstrip\ does not need \LaTeX), then inform the autodetect routine
% about your intention:
% \begin{quote}
%   \verb|latex \let\install=y% \iffalse meta-comment
%
% File: telprint.dtx
% Version: 2016/05/16 v1.11
% Info: Format German phone numbers
%
% Copyright (C) 1996, 1997, 2004-2008 by
%    Heiko Oberdiek <heiko.oberdiek at googlemail.com>
%    2016
%    https://github.com/ho-tex/oberdiek/issues
%
% This work may be distributed and/or modified under the
% conditions of the LaTeX Project Public License, either
% version 1.3c of this license or (at your option) any later
% version. This version of this license is in
%    http://www.latex-project.org/lppl/lppl-1-3c.txt
% and the latest version of this license is in
%    http://www.latex-project.org/lppl.txt
% and version 1.3 or later is part of all distributions of
% LaTeX version 2005/12/01 or later.
%
% This work has the LPPL maintenance status "maintained".
%
% This Current Maintainer of this work is Heiko Oberdiek.
%
% The Base Interpreter refers to any `TeX-Format',
% because some files are installed in TDS:tex/generic//.
%
% This work consists of the main source file telprint.dtx
% and the derived files
%    telprint.sty, telprint.pdf, telprint.ins, telprint.drv,
%    telprint-test1.tex.
%
% Distribution:
%    CTAN:macros/latex/contrib/oberdiek/telprint.dtx
%    CTAN:macros/latex/contrib/oberdiek/telprint.pdf
%
% Unpacking:
%    (a) If telprint.ins is present:
%           tex telprint.ins
%    (b) Without telprint.ins:
%           tex telprint.dtx
%    (c) If you insist on using LaTeX
%           latex \let\install=y\input{telprint.dtx}
%        (quote the arguments according to the demands of your shell)
%
% Documentation:
%    (a) If telprint.drv is present:
%           latex telprint.drv
%    (b) Without telprint.drv:
%           latex telprint.dtx; ...
%    The class ltxdoc loads the configuration file ltxdoc.cfg
%    if available. Here you can specify further options, e.g.
%    use A4 as paper format:
%       \PassOptionsToClass{a4paper}{article}
%
%    Programm calls to get the documentation (example):
%       pdflatex telprint.dtx
%       makeindex -s gind.ist telprint.idx
%       pdflatex telprint.dtx
%       makeindex -s gind.ist telprint.idx
%       pdflatex telprint.dtx
%
% Installation:
%    TDS:tex/generic/oberdiek/telprint.sty
%    TDS:doc/latex/oberdiek/telprint.pdf
%    TDS:doc/latex/oberdiek/test/telprint-test1.tex
%    TDS:source/latex/oberdiek/telprint.dtx
%
%<*ignore>
\begingroup
  \catcode123=1 %
  \catcode125=2 %
  \def\x{LaTeX2e}%
\expandafter\endgroup
\ifcase 0\ifx\install y1\fi\expandafter
         \ifx\csname processbatchFile\endcsname\relax\else1\fi
         \ifx\fmtname\x\else 1\fi\relax
\else\csname fi\endcsname
%</ignore>
%<*install>
\input docstrip.tex
\Msg{************************************************************************}
\Msg{* Installation}
\Msg{* Package: telprint 2016/05/16 v1.11 Format German phone numbers (HO)}
\Msg{************************************************************************}

\keepsilent
\askforoverwritefalse

\let\MetaPrefix\relax
\preamble

This is a generated file.

Project: telprint
Version: 2016/05/16 v1.11

Copyright (C) 1996, 1997, 2004-2008 by
   Heiko Oberdiek <heiko.oberdiek at googlemail.com>

This work may be distributed and/or modified under the
conditions of the LaTeX Project Public License, either
version 1.3c of this license or (at your option) any later
version. This version of this license is in
   http://www.latex-project.org/lppl/lppl-1-3c.txt
and the latest version of this license is in
   http://www.latex-project.org/lppl.txt
and version 1.3 or later is part of all distributions of
LaTeX version 2005/12/01 or later.

This work has the LPPL maintenance status "maintained".

This Current Maintainer of this work is Heiko Oberdiek.

The Base Interpreter refers to any `TeX-Format',
because some files are installed in TDS:tex/generic//.

This work consists of the main source file telprint.dtx
and the derived files
   telprint.sty, telprint.pdf, telprint.ins, telprint.drv,
   telprint-test1.tex.

\endpreamble
\let\MetaPrefix\DoubleperCent

\generate{%
  \file{telprint.ins}{\from{telprint.dtx}{install}}%
  \file{telprint.drv}{\from{telprint.dtx}{driver}}%
  \usedir{tex/generic/oberdiek}%
  \file{telprint.sty}{\from{telprint.dtx}{package}}%
%  \usedir{doc/latex/oberdiek/test}%
%  \file{telprint-test1.tex}{\from{telprint.dtx}{test1}}%
  \nopreamble
  \nopostamble
%  \usedir{source/latex/oberdiek/catalogue}%
%  \file{telprint.xml}{\from{telprint.dtx}{catalogue}}%
}

\catcode32=13\relax% active space
\let =\space%
\Msg{************************************************************************}
\Msg{*}
\Msg{* To finish the installation you have to move the following}
\Msg{* file into a directory searched by TeX:}
\Msg{*}
\Msg{*     telprint.sty}
\Msg{*}
\Msg{* To produce the documentation run the file `telprint.drv'}
\Msg{* through LaTeX.}
\Msg{*}
\Msg{* Happy TeXing!}
\Msg{*}
\Msg{************************************************************************}

\endbatchfile
%</install>
%<*ignore>
\fi
%</ignore>
%<*driver>
\NeedsTeXFormat{LaTeX2e}
\ProvidesFile{telprint.drv}%
  [2016/05/16 v1.11 Format German phone numbers (HO)]%
\documentclass{ltxdoc}
\usepackage{holtxdoc}[2011/11/22]
\usepackage[ngerman,english]{babel}
\begin{document}
  \DocInput{telprint.dtx}%
\end{document}
%</driver>
% \fi
%
%
% \CharacterTable
%  {Upper-case    \A\B\C\D\E\F\G\H\I\J\K\L\M\N\O\P\Q\R\S\T\U\V\W\X\Y\Z
%   Lower-case    \a\b\c\d\e\f\g\h\i\j\k\l\m\n\o\p\q\r\s\t\u\v\w\x\y\z
%   Digits        \0\1\2\3\4\5\6\7\8\9
%   Exclamation   \!     Double quote  \"     Hash (number) \#
%   Dollar        \$     Percent       \%     Ampersand     \&
%   Acute accent  \'     Left paren    \(     Right paren   \)
%   Asterisk      \*     Plus          \+     Comma         \,
%   Minus         \-     Point         \.     Solidus       \/
%   Colon         \:     Semicolon     \;     Less than     \<
%   Equals        \=     Greater than  \>     Question mark \?
%   Commercial at \@     Left bracket  \[     Backslash     \\
%   Right bracket \]     Circumflex    \^     Underscore    \_
%   Grave accent  \`     Left brace    \{     Vertical bar  \|
%   Right brace   \}     Tilde         \~}
%
% \GetFileInfo{telprint.drv}
%
% \title{The \xpackage{telprint} package}
% \date{2016/05/16 v1.11}
% \author{Heiko Oberdiek\thanks
% {Please report any issues at https://github.com/ho-tex/oberdiek/issues}\\
% \xemail{heiko.oberdiek at googlemail.com}}
%
% \maketitle
%
% \begin{abstract}
% Package \xpackage{telprint} provides \cs{telprint} for formatting
% German phone numbers.
% \end{abstract}
%
% \tableofcontents
%
% \section{Documentation}
%
% \subsection{Introduction}
%
%            This is a very old package that I have written
%            to format phone numbers. It follows German
%            conventions and the documentation is mainly in German.
%
% \subsection{Short overview in English}
%
% \LaTeX:
% \begin{quote}
% |\usepackage{telprint}|\\
% |\telprint{123/456-789}|\\
% \end{quote}
% \plainTeX:
% \begin{quote}
%   |\input telprint.sty|\\
%   |\telprint{123/456-789}|
% \end{quote}
%
% \DescribeMacro\telprint
% |\telprint{...}| formats the explicitly given number.
%     Digits, spaces and some special characters
%     ('+', '/', '-', '(', ')', '\textasciitilde', ' ') are supported.
%     Numbers are divided into groups of two digits from the right.
% Examples:
% \begin{quote}
%     |\telprint{0761/12345}     ==> 07\,61/1\,23\,45|\\
%     |\telprint{01234/567-89}   ==> 0\,12\,34/5\,67\leavevmode\hbox{-}89|\\
%     |\telprint{+49 (6221) 297} ==> +49~(62\,21)~2\,97|
% \end{quote}
%
% \subsubsection{Configuration}
%
% The output of the symbols can be configured by
% \cs{telhyphen}, \cs{telslash}, \cs{telleftparen}, \cs{telrightparen},
% \cs{telplus}, \cs{teltilde}.
% Example:
% \begin{quote}
%   |\telslash{\,/\,}\\|
%   |\telprint{12/34} ==> 12\,/\,34|
% \end{quote}
%
% \DescribeMacro\telspace
% \cs{telspace} configures the space between digit groups.
%
% \DescribeMacro\telnumber
% \cs{telnumber} only formats a number in digit groups; special
%    characters are not recognized.
%
% \subsection{Documentation in German}
%
% \begin{otherlanguage*}{ngerman}
% \hyphenation{To-ken-ma-kros}
% \begin{itemize}
% \item \DescribeMacro\telprint |telprint#1|\\
%   Der eigentliche Anwenderbefehl zur formatierten Ausgabe von
%   Telefonnummern. Diese d\"urfen dabei nur als Zahlen angegeben
%   werden(, da sie tokenweise analysiert werden).
%   Als Trenn- oder Sonderzeichen werden unterst\"utzt:
%   '+', '/', '-', '(', ')', '\textasciitilde', ' '
%   Einfache Leerzeichen werden erkannt und durch Tilden ersetzt, um
%   Trennungen in der Telefonnummer zu verhindern. (Man beachte aus
%   gleichem Grunde die \cs{hbox} bei '-'.)
%   Beispiele:
%   \begin{quote}
%     |\telprint{0761/12345}     ==> 07\,61/1\,23\,45|\\
%     |\telprint{01234/567-89}   ==> 0\,12\,34/5\,67\leavevmode\hbox{-}89|\\
%     |\telprint{+49 (6221) 297} ==> +49~(62\,21)~2\,97|
%   \end{quote}
% \end{itemize}
% Der Rest enth\"alt eher Technisches:
% \begin{itemize}
% \item \DescribeMacro\telspace |\telspace#1|\\
%   Mit diesem Befehl wird der Abstand zwischen den Zifferngruppen
%   angegeben (Default: |\,|).
%   (Durch |\telspace{}| kann dieser zusaetzliche Abstand abgestellt
%   werden.)
% \item \DescribeMacro\telhyphen |\telhyphen#1|\\
%   Dieser Befehl gibt die Art des Bindestriches, wie er ausgegeben
%   werden soll. In der Eingabe darf jedoch nur der einfache
%   Bindestrich stehen:
%   |\telprint{123-45}|, jedoch NIE |\telprint{123--45}|!
%   Kopka-Bindestrich-Fans geben an:
%   |\telhyphen{\leavevmode\hbox{--}}|
% \item
%   \DescribeMacro{\telslash}
%   \DescribeMacro{\telleftparen}
%   \DescribeMacro{\telrightparen}
%   \DescribeMacro{\telplus}
%   \DescribeMacro{\teltilde}
%   |\telslash#1|, |\telleftparen#1|, |\telrightparen#1|, |\telplus#1|,
%   |\teltilde|\\
%   Diese Befehle konfigurieren die Zeichen '/', '(', ')', '+'
%   und '\textasciitilde'. Sie funktionieren analog zu \cs{telhyphen}.
% \item \DescribeMacro\telnumber |\telnumber#1|\\
%   Richtung interner Befehl: Er dient dazu, eine Zifferngruppe
%   in Zweiergruppen auszugeben.
%   Die einzelnen Zahlen werden im Tokenregister \cs{TELtoks}
%   gespeichert. Abwechselnd werden dabei zwischen zwei Token
%   (Zahlen) \cs{TELx} bzw. \cs{TELy} eingefuegt, abh\"angig von dem
%   wechselnden Wert von \cs{TELswitch}. Zum Schluss kann dann einfach
%   festgestellt werden ob die Nummer nun eine geradzahlige oder
%   ungeradzahlige Zahl von Ziffern aufwies. Dem entsprechend wird
%   \cs{TELx} mit dem Zusatzabstand belegt und \cs{TELy} leer definiert
%   oder umgekehrt. )
% \item |\TEL...| interne Befehle, Technisches:\\
%   \cs{TELsplit} dient zur Aufteilung einer zusammengesetzten
%   Telefonnummer (Vorwahl, Hauptnummer, Nebenstelle). In dieser
%   Implementation werden als Trennzeichen nur '/' und '-' erkannt.
%   Die einzelnen Bestandteile wie Vorwahl werden dann dem Befehl
%   \cs{telnumber} zur Formatierung uebergeben.
% \item Die Erkennung von einfachen Leerzeichen ist um einiges
%   schwieriger: Die Tokentrennung ueber Parameter |#1#2| funktioniert
%   nicht f\"ur einfache Leerzeichen, da TeX sie \emph{niemals} als
%   eigenst\"andige Argumente behandelt! (The TeXbook, Chapter 20,
%   p. 201)
%
%   (Anmerkung am Rande: Deshalb funktionieren die entsprechenden
%   Tokenmakros auf S. 149 des Buches "`Einf\"uhrung in TeX"' von
%   N. Schwarz (3. Aufl.) nicht, wenn im Tokenregister als erstes
%   ein einfaches Leerzeichen steht!)
% \end{itemize}
% \end{otherlanguage*}
%
% \StopEventually{
% }
%
% \section{Implementation}
%
%    \begin{macrocode}
%<*package>
%    \end{macrocode}
%
% \subsection{Reload check and package identification}
%    Reload check, especially if the package is not used with \LaTeX.
%    \begin{macrocode}
\begingroup\catcode61\catcode48\catcode32=10\relax%
  \catcode13=5 % ^^M
  \endlinechar=13 %
  \catcode35=6 % #
  \catcode39=12 % '
  \catcode44=12 % ,
  \catcode45=12 % -
  \catcode46=12 % .
  \catcode58=12 % :
  \catcode64=11 % @
  \catcode123=1 % {
  \catcode125=2 % }
  \expandafter\let\expandafter\x\csname ver@telprint.sty\endcsname
  \ifx\x\relax % plain-TeX, first loading
  \else
    \def\empty{}%
    \ifx\x\empty % LaTeX, first loading,
      % variable is initialized, but \ProvidesPackage not yet seen
    \else
      \expandafter\ifx\csname PackageInfo\endcsname\relax
        \def\x#1#2{%
          \immediate\write-1{Package #1 Info: #2.}%
        }%
      \else
        \def\x#1#2{\PackageInfo{#1}{#2, stopped}}%
      \fi
      \x{telprint}{The package is already loaded}%
      \aftergroup\endinput
    \fi
  \fi
\endgroup%
%    \end{macrocode}
%    Package identification:
%    \begin{macrocode}
\begingroup\catcode61\catcode48\catcode32=10\relax%
  \catcode13=5 % ^^M
  \endlinechar=13 %
  \catcode35=6 % #
  \catcode39=12 % '
  \catcode40=12 % (
  \catcode41=12 % )
  \catcode44=12 % ,
  \catcode45=12 % -
  \catcode46=12 % .
  \catcode47=12 % /
  \catcode58=12 % :
  \catcode64=11 % @
  \catcode91=12 % [
  \catcode93=12 % ]
  \catcode123=1 % {
  \catcode125=2 % }
  \expandafter\ifx\csname ProvidesPackage\endcsname\relax
    \def\x#1#2#3[#4]{\endgroup
      \immediate\write-1{Package: #3 #4}%
      \xdef#1{#4}%
    }%
  \else
    \def\x#1#2[#3]{\endgroup
      #2[{#3}]%
      \ifx#1\@undefined
        \xdef#1{#3}%
      \fi
      \ifx#1\relax
        \xdef#1{#3}%
      \fi
    }%
  \fi
\expandafter\x\csname ver@telprint.sty\endcsname
\ProvidesPackage{telprint}%
  [2016/05/16 v1.11 Format German phone numbers (HO)]%
%    \end{macrocode}
%
% \subsection{Catcodes}
%
%    \begin{macrocode}
\begingroup\catcode61\catcode48\catcode32=10\relax%
  \catcode13=5 % ^^M
  \endlinechar=13 %
  \catcode123=1 % {
  \catcode125=2 % }
  \catcode64=11 % @
  \def\x{\endgroup
    \expandafter\edef\csname TELAtEnd\endcsname{%
      \endlinechar=\the\endlinechar\relax
      \catcode13=\the\catcode13\relax
      \catcode32=\the\catcode32\relax
      \catcode35=\the\catcode35\relax
      \catcode61=\the\catcode61\relax
      \catcode64=\the\catcode64\relax
      \catcode123=\the\catcode123\relax
      \catcode125=\the\catcode125\relax
    }%
  }%
\x\catcode61\catcode48\catcode32=10\relax%
\catcode13=5 % ^^M
\endlinechar=13 %
\catcode35=6 % #
\catcode64=11 % @
\catcode123=1 % {
\catcode125=2 % }
\def\TMP@EnsureCode#1#2{%
  \edef\TELAtEnd{%
    \TELAtEnd
    \catcode#1=\the\catcode#1\relax
  }%
  \catcode#1=#2\relax
}
\TMP@EnsureCode{33}{12}% !
\TMP@EnsureCode{36}{3}% $
\TMP@EnsureCode{40}{12}% (
\TMP@EnsureCode{41}{12}% )
\TMP@EnsureCode{42}{12}% *
\TMP@EnsureCode{43}{12}% +
\TMP@EnsureCode{44}{12}% ,
\TMP@EnsureCode{45}{12}% -
\TMP@EnsureCode{46}{12}% .
\TMP@EnsureCode{47}{12}% /
\TMP@EnsureCode{91}{12}% [
\TMP@EnsureCode{93}{12}% ]
\TMP@EnsureCode{126}{13}% ~ (active)
\edef\TELAtEnd{\TELAtEnd\noexpand\endinput}
%    \end{macrocode}
%
% \subsection{Package macros}
%    \begin{macrocode}
\ifx\DeclareRobustCommand\UnDeFiNeD
  \def\DeclareRobustCommand*#1[1]{\def#1##1}%
  \def\TELreset{\let\DeclareRobustCommand=\UnDeFiNeD}%
  \input infwarerr.sty\relax
  \@PackageInfo{telprint}{%
    Macros are not robust!%
  }%
\else
  \let\TELreset=\relax
\fi
%    \end{macrocode}
%    \begin{macro}{\telspace}
%    \begin{macrocode}
\DeclareRobustCommand*{\telspace}[1]{\def\TELspace{#1}}
\telspace{{}$\,${}}
%    \end{macrocode}
%    \end{macro}
%    \begin{macro}{\telhyphen}
%    \begin{macrocode}
\DeclareRobustCommand*{\telhyphen}[1]{\def\TELhyphen{#1}}
\telhyphen{\leavevmode\hbox{-}}% \hbox zur Verhinderung der Trennung
%    \end{macrocode}
%    \end{macro}
%    \begin{macro}{\telslash}
%    \begin{macrocode}
\DeclareRobustCommand*{\telslash}[1]{\def\TELslash{#1}}
\telslash{/}%
%    \end{macrocode}
%    \end{macro}
%    \begin{macro}{\telleftparen}
%    \begin{macrocode}
\DeclareRobustCommand*{\telleftparen}[1]{\def\TELleftparen{#1}}
\telleftparen{(}%
%    \end{macrocode}
%    \end{macro}
%    \begin{macro}{\telrightparen}
%    \begin{macrocode}
\DeclareRobustCommand*{\telrightparen}[1]{\def\TELrightparen{#1}}
\telrightparen{)}%
%    \end{macrocode}
%    \end{macro}
%    \begin{macro}{\telplus}
%    \begin{macrocode}
\DeclareRobustCommand*{\telplus}[1]{\def\TELplus{#1}}
\telplus{+}%
%    \end{macrocode}
%    \end{macro}
%    \begin{macro}{\teltilde}
%    \begin{macrocode}
\DeclareRobustCommand*{\teltilde}[1]{\def\TELtilde{#1}}
\teltilde{~}%
%    \end{macrocode}
%    \end{macro}
%    \begin{macro}{\TELtoks}
%    \begin{macrocode}
\newtoks\TELtoks
%    \end{macrocode}
%    \end{macro}
%    \begin{macro}{\TELnumber}
%    \begin{macrocode}
\def\TELnumber#1#2\TELnumberEND{%
  \begingroup
  \def\0{#2}%
  \expandafter\endgroup
  \ifx\0\empty
    \TELtoks=\expandafter{\the\TELtoks#1}%
    \ifnum\TELswitch=0 %
      \def\TELx{\TELspace}\def\TELy{}%
    \else
      \def\TELx{}\def\TELy{\TELspace}%
    \fi
    \the\TELtoks
  \else
    \ifnum\TELswitch=0 %
      \TELtoks=\expandafter{\the\TELtoks#1\TELx}%
      \def\TELswitch{1}%
    \else
      \TELtoks=\expandafter{\the\TELtoks#1\TELy}%
      \def\TELswitch{0}%
    \fi
    \TELnumber#2\TELnumberEND
  \fi
}
%    \end{macrocode}
%    \end{macro}
%    \begin{macro}{\telnumber}
%    \begin{macrocode}
\DeclareRobustCommand*{\telnumber}[1]{%
  \TELtoks={}%
  \def\TELswitch{0}%
  \TELnumber#1{}\TELnumberEND
}
%    \end{macrocode}
%    \end{macro}
%    \begin{macro}{\TELsplit}
%    \begin{macrocode}
\def\TELsplit{\futurelet\TELfuture\TELdosplit}
%    \end{macrocode}
%    \end{macro}
%    \begin{macro}{\TELdosplit}
%    \begin{macrocode}
\def\TELdosplit#1#2\TELsplitEND
{%
  \def\TELsp{ }%
  \expandafter\ifx\TELsp\TELfuture
    \let\TELfuture=\relax
    \expandafter\telnumber\expandafter{\the\TELtoks}~%
    \telprint{#1#2}% Das Leerzeichen kann nicht #1 sein!
  \else
    \def\TELfirst{#1}%
    \ifx\TELfirst\empty
      \expandafter\telnumber\expandafter{\the\TELtoks}%
      \TELtoks={}%
    \else\if-\TELfirst
      \expandafter\telnumber\expandafter{\the\TELtoks}\TELhyphen
      \telprint{#2}%
    \else\if/\TELfirst
      \expandafter\telnumber\expandafter{\the\TELtoks}\TELslash
      \telprint{#2}%
    \else\if(\TELfirst
      \expandafter\telnumber\expandafter{\the\TELtoks}\TELleftparen
      \telprint{#2}%
    \else\if)\TELfirst
      \expandafter\telnumber\expandafter{\the\TELtoks}\TELrightparen
      \telprint{#2}%
    \else\if+\TELfirst
      \expandafter\telnumber\expandafter{\the\TELtoks}\TELplus
      \telprint{#2}%
    \else\def\TELtemp{~}\ifx\TELtemp\TELfirst
      \expandafter\telnumber\expandafter{\the\TELtoks}\TELtilde
      \telprint{#2}%
    \else
      \TELtoks=\expandafter{\the\TELtoks#1}%
      \TELsplit#2{}\TELsplitEND
    \fi\fi\fi\fi\fi\fi\fi
  \fi
}
%    \end{macrocode}
%    \end{macro}
%    \begin{macro}{\telprint}
%    \begin{macrocode}
\DeclareRobustCommand*{\telprint}[1]{%
  \TELtoks={}%
  \TELsplit#1{}\TELsplitEND
}
%    \end{macrocode}
%    \end{macro}
%    \begin{macrocode}
\TELreset\let\TELreset=\UnDeFiNeD
%    \end{macrocode}
%
%    \begin{macrocode}
\TELAtEnd%
%</package>
%    \end{macrocode}
%
% \section{Test}
%
% \subsection{Catcode checks for loading}
%
%    \begin{macrocode}
%<*test1>
%    \end{macrocode}
%    \begin{macrocode}
\catcode`\{=1 %
\catcode`\}=2 %
\catcode`\#=6 %
\catcode`\@=11 %
\expandafter\ifx\csname count@\endcsname\relax
  \countdef\count@=255 %
\fi
\expandafter\ifx\csname @gobble\endcsname\relax
  \long\def\@gobble#1{}%
\fi
\expandafter\ifx\csname @firstofone\endcsname\relax
  \long\def\@firstofone#1{#1}%
\fi
\expandafter\ifx\csname loop\endcsname\relax
  \expandafter\@firstofone
\else
  \expandafter\@gobble
\fi
{%
  \def\loop#1\repeat{%
    \def\body{#1}%
    \iterate
  }%
  \def\iterate{%
    \body
      \let\next\iterate
    \else
      \let\next\relax
    \fi
    \next
  }%
  \let\repeat=\fi
}%
\def\RestoreCatcodes{}
\count@=0 %
\loop
  \edef\RestoreCatcodes{%
    \RestoreCatcodes
    \catcode\the\count@=\the\catcode\count@\relax
  }%
\ifnum\count@<255 %
  \advance\count@ 1 %
\repeat

\def\RangeCatcodeInvalid#1#2{%
  \count@=#1\relax
  \loop
    \catcode\count@=15 %
  \ifnum\count@<#2\relax
    \advance\count@ 1 %
  \repeat
}
\def\RangeCatcodeCheck#1#2#3{%
  \count@=#1\relax
  \loop
    \ifnum#3=\catcode\count@
    \else
      \errmessage{%
        Character \the\count@\space
        with wrong catcode \the\catcode\count@\space
        instead of \number#3%
      }%
    \fi
  \ifnum\count@<#2\relax
    \advance\count@ 1 %
  \repeat
}
\def\space{ }
\expandafter\ifx\csname LoadCommand\endcsname\relax
  \def\LoadCommand{\input telprint.sty\relax}%
\fi
\def\Test{%
  \RangeCatcodeInvalid{0}{47}%
  \RangeCatcodeInvalid{58}{64}%
  \RangeCatcodeInvalid{91}{96}%
  \RangeCatcodeInvalid{123}{255}%
  \catcode`\@=12 %
  \catcode`\\=0 %
  \catcode`\%=14 %
  \LoadCommand
  \RangeCatcodeCheck{0}{36}{15}%
  \RangeCatcodeCheck{37}{37}{14}%
  \RangeCatcodeCheck{38}{47}{15}%
  \RangeCatcodeCheck{48}{57}{12}%
  \RangeCatcodeCheck{58}{63}{15}%
  \RangeCatcodeCheck{64}{64}{12}%
  \RangeCatcodeCheck{65}{90}{11}%
  \RangeCatcodeCheck{91}{91}{15}%
  \RangeCatcodeCheck{92}{92}{0}%
  \RangeCatcodeCheck{93}{96}{15}%
  \RangeCatcodeCheck{97}{122}{11}%
  \RangeCatcodeCheck{123}{255}{15}%
  \RestoreCatcodes
}
\Test
\csname @@end\endcsname
\end
%    \end{macrocode}
%    \begin{macrocode}
%</test1>
%    \end{macrocode}
%
% \section{Installation}
%
% \subsection{Download}
%
% \paragraph{Package.} This package is available on
% CTAN\footnote{\url{https://ctan.org/pkg/telprint}}:
% \begin{description}
% \item[\CTAN{macros/latex/contrib/oberdiek/telprint.dtx}] The source file.
% \item[\CTAN{macros/latex/contrib/oberdiek/telprint.pdf}] Documentation.
% \end{description}
%
%
% \paragraph{Bundle.} All the packages of the bundle `oberdiek'
% are also available in a TDS compliant ZIP archive. There
% the packages are already unpacked and the documentation files
% are generated. The files and directories obey the TDS standard.
% \begin{description}
% \item[\CTANinstall{install/macros/latex/contrib/oberdiek.tds.zip}]
% \end{description}
% \emph{TDS} refers to the standard ``A Directory Structure
% for \TeX\ Files'' (\CTAN{tds/tds.pdf}). Directories
% with \xfile{texmf} in their name are usually organized this way.
%
% \subsection{Bundle installation}
%
% \paragraph{Unpacking.} Unpack the \xfile{oberdiek.tds.zip} in the
% TDS tree (also known as \xfile{texmf} tree) of your choice.
% Example (linux):
% \begin{quote}
%   |unzip oberdiek.tds.zip -d ~/texmf|
% \end{quote}
%
% \paragraph{Script installation.}
% Check the directory \xfile{TDS:scripts/oberdiek/} for
% scripts that need further installation steps.
% Package \xpackage{attachfile2} comes with the Perl script
% \xfile{pdfatfi.pl} that should be installed in such a way
% that it can be called as \texttt{pdfatfi}.
% Example (linux):
% \begin{quote}
%   |chmod +x scripts/oberdiek/pdfatfi.pl|\\
%   |cp scripts/oberdiek/pdfatfi.pl /usr/local/bin/|
% \end{quote}
%
% \subsection{Package installation}
%
% \paragraph{Unpacking.} The \xfile{.dtx} file is a self-extracting
% \docstrip\ archive. The files are extracted by running the
% \xfile{.dtx} through \plainTeX:
% \begin{quote}
%   \verb|tex telprint.dtx|
% \end{quote}
%
% \paragraph{TDS.} Now the different files must be moved into
% the different directories in your installation TDS tree
% (also known as \xfile{texmf} tree):
% \begin{quote}
% \def\t{^^A
% \begin{tabular}{@{}>{\ttfamily}l@{ $\rightarrow$ }>{\ttfamily}l@{}}
%   telprint.sty & tex/generic/oberdiek/telprint.sty\\
%   telprint.pdf & doc/latex/oberdiek/telprint.pdf\\
%   test/telprint-test1.tex & doc/latex/oberdiek/test/telprint-test1.tex\\
%   telprint.dtx & source/latex/oberdiek/telprint.dtx\\
% \end{tabular}^^A
% }^^A
% \sbox0{\t}^^A
% \ifdim\wd0>\linewidth
%   \begingroup
%     \advance\linewidth by\leftmargin
%     \advance\linewidth by\rightmargin
%   \edef\x{\endgroup
%     \def\noexpand\lw{\the\linewidth}^^A
%   }\x
%   \def\lwbox{^^A
%     \leavevmode
%     \hbox to \linewidth{^^A
%       \kern-\leftmargin\relax
%       \hss
%       \usebox0
%       \hss
%       \kern-\rightmargin\relax
%     }^^A
%   }^^A
%   \ifdim\wd0>\lw
%     \sbox0{\small\t}^^A
%     \ifdim\wd0>\linewidth
%       \ifdim\wd0>\lw
%         \sbox0{\footnotesize\t}^^A
%         \ifdim\wd0>\linewidth
%           \ifdim\wd0>\lw
%             \sbox0{\scriptsize\t}^^A
%             \ifdim\wd0>\linewidth
%               \ifdim\wd0>\lw
%                 \sbox0{\tiny\t}^^A
%                 \ifdim\wd0>\linewidth
%                   \lwbox
%                 \else
%                   \usebox0
%                 \fi
%               \else
%                 \lwbox
%               \fi
%             \else
%               \usebox0
%             \fi
%           \else
%             \lwbox
%           \fi
%         \else
%           \usebox0
%         \fi
%       \else
%         \lwbox
%       \fi
%     \else
%       \usebox0
%     \fi
%   \else
%     \lwbox
%   \fi
% \else
%   \usebox0
% \fi
% \end{quote}
% If you have a \xfile{docstrip.cfg} that configures and enables \docstrip's
% TDS installing feature, then some files can already be in the right
% place, see the documentation of \docstrip.
%
% \subsection{Refresh file name databases}
%
% If your \TeX~distribution
% (\teTeX, \mikTeX, \dots) relies on file name databases, you must refresh
% these. For example, \teTeX\ users run \verb|texhash| or
% \verb|mktexlsr|.
%
% \subsection{Some details for the interested}
%
% \paragraph{Attached source.}
%
% The PDF documentation on CTAN also includes the
% \xfile{.dtx} source file. It can be extracted by
% AcrobatReader 6 or higher. Another option is \textsf{pdftk},
% e.g. unpack the file into the current directory:
% \begin{quote}
%   \verb|pdftk telprint.pdf unpack_files output .|
% \end{quote}
%
% \paragraph{Unpacking with \LaTeX.}
% The \xfile{.dtx} chooses its action depending on the format:
% \begin{description}
% \item[\plainTeX:] Run \docstrip\ and extract the files.
% \item[\LaTeX:] Generate the documentation.
% \end{description}
% If you insist on using \LaTeX\ for \docstrip\ (really,
% \docstrip\ does not need \LaTeX), then inform the autodetect routine
% about your intention:
% \begin{quote}
%   \verb|latex \let\install=y\input{telprint.dtx}|
% \end{quote}
% Do not forget to quote the argument according to the demands
% of your shell.
%
% \paragraph{Generating the documentation.}
% You can use both the \xfile{.dtx} or the \xfile{.drv} to generate
% the documentation. The process can be configured by the
% configuration file \xfile{ltxdoc.cfg}. For instance, put this
% line into this file, if you want to have A4 as paper format:
% \begin{quote}
%   \verb|\PassOptionsToClass{a4paper}{article}|
% \end{quote}
% An example follows how to generate the
% documentation with pdf\LaTeX:
% \begin{quote}
%\begin{verbatim}
%pdflatex telprint.dtx
%makeindex -s gind.ist telprint.idx
%pdflatex telprint.dtx
%makeindex -s gind.ist telprint.idx
%pdflatex telprint.dtx
%\end{verbatim}
% \end{quote}
%
% \begin{History}
%   \begin{Version}{1996/11/28 v1.0}
%   \item
%     Erste lauff\"ahige Version.
%   \item
%     Nur '-' und '/' als zul\"assige Sonderzeichen.
%   \end{Version}
%   \begin{Version}{1997/09/16 v1.1}
%   \item
%     Dokumentation und Kommentare (Posting in de.comp.text.tex).
%   \item
%     Erweiterung um Sonderzeichen '(', ')', '+', '\textasciitilde' und ' '.
%   \item
%     Trennungsverhinderung am 'hyphen'.
%   \end{Version}
%   \begin{Version}{1997/10/16 v1.2}
%   \item
%     Schutz vor wiederholtem Einlesen.
%   \item
%     Unter \LaTeXe\ Nutzung des \cs{DeclareRobustCommand}-Features.
%   \end{Version}
%   \begin{Version}{1997/12/09 v1.3}
%   \item
%     Tempor\"are Variable eingespart.
%   \item
%     Posted in newsgroup \xnewsgroup{de.comp.text.tex}:\\
%     \URL{``\link{Re: Generisches Markup f\"ur Telefonnummern?}''}^^A
%     {http://groups.google.com/group/de.comp.text.tex/msg/86b3a86140007309}
%   \end{Version}
%   \begin{Version}{2004/11/02 v1.4}
%   \item
%     Fehler in der Dokumentation korrigiert.
%   \end{Version}
%   \begin{Version}{2005/09/30 v1.5}
%   \item
%     Konfigurierbare Symbole: '/', '(', ')', '+' und '\textasciitilde'.
%   \end{Version}
%   \begin{Version}{2006/02/12 v1.6}
%   \item
%     LPPL 1.3.
%   \item
%     Kurze \"Ubersicht in Englisch.
%   \item
%     CTAN.
%   \end{Version}
%   \begin{Version}{2006/08/26 v1.7}
%   \item
%     New DTX framework.
%   \end{Version}
%   \begin{Version}{2007/04/11 v1.8}
%   \item
%     Line ends sanitized.
%   \end{Version}
%   \begin{Version}{2007/09/09 v1.9}
%   \item
%     Catcode section added.
%   \item
%     Missing docstrip tag added.
%   \end{Version}
%   \begin{Version}{2008/08/11 v1.10}
%   \item
%     Code is not changed.
%   \item
%     URLs updated.
%   \end{Version}
%   \begin{Version}{2016/05/16 v1.11}
%   \item
%     Documentation updates.
%   \end{Version}
% \end{History}
%
% \PrintIndex
%
% \Finale
\endinput
|
% \end{quote}
% Do not forget to quote the argument according to the demands
% of your shell.
%
% \paragraph{Generating the documentation.}
% You can use both the \xfile{.dtx} or the \xfile{.drv} to generate
% the documentation. The process can be configured by the
% configuration file \xfile{ltxdoc.cfg}. For instance, put this
% line into this file, if you want to have A4 as paper format:
% \begin{quote}
%   \verb|\PassOptionsToClass{a4paper}{article}|
% \end{quote}
% An example follows how to generate the
% documentation with pdf\LaTeX:
% \begin{quote}
%\begin{verbatim}
%pdflatex telprint.dtx
%makeindex -s gind.ist telprint.idx
%pdflatex telprint.dtx
%makeindex -s gind.ist telprint.idx
%pdflatex telprint.dtx
%\end{verbatim}
% \end{quote}
%
% \begin{History}
%   \begin{Version}{1996/11/28 v1.0}
%   \item
%     Erste lauff\"ahige Version.
%   \item
%     Nur '-' und '/' als zul\"assige Sonderzeichen.
%   \end{Version}
%   \begin{Version}{1997/09/16 v1.1}
%   \item
%     Dokumentation und Kommentare (Posting in de.comp.text.tex).
%   \item
%     Erweiterung um Sonderzeichen '(', ')', '+', '\textasciitilde' und ' '.
%   \item
%     Trennungsverhinderung am 'hyphen'.
%   \end{Version}
%   \begin{Version}{1997/10/16 v1.2}
%   \item
%     Schutz vor wiederholtem Einlesen.
%   \item
%     Unter \LaTeXe\ Nutzung des \cs{DeclareRobustCommand}-Features.
%   \end{Version}
%   \begin{Version}{1997/12/09 v1.3}
%   \item
%     Tempor\"are Variable eingespart.
%   \item
%     Posted in newsgroup \xnewsgroup{de.comp.text.tex}:\\
%     \URL{``\link{Re: Generisches Markup f\"ur Telefonnummern?}''}^^A
%     {http://groups.google.com/group/de.comp.text.tex/msg/86b3a86140007309}
%   \end{Version}
%   \begin{Version}{2004/11/02 v1.4}
%   \item
%     Fehler in der Dokumentation korrigiert.
%   \end{Version}
%   \begin{Version}{2005/09/30 v1.5}
%   \item
%     Konfigurierbare Symbole: '/', '(', ')', '+' und '\textasciitilde'.
%   \end{Version}
%   \begin{Version}{2006/02/12 v1.6}
%   \item
%     LPPL 1.3.
%   \item
%     Kurze \"Ubersicht in Englisch.
%   \item
%     CTAN.
%   \end{Version}
%   \begin{Version}{2006/08/26 v1.7}
%   \item
%     New DTX framework.
%   \end{Version}
%   \begin{Version}{2007/04/11 v1.8}
%   \item
%     Line ends sanitized.
%   \end{Version}
%   \begin{Version}{2007/09/09 v1.9}
%   \item
%     Catcode section added.
%   \item
%     Missing docstrip tag added.
%   \end{Version}
%   \begin{Version}{2008/08/11 v1.10}
%   \item
%     Code is not changed.
%   \item
%     URLs updated.
%   \end{Version}
%   \begin{Version}{2016/05/16 v1.11}
%   \item
%     Documentation updates.
%   \end{Version}
% \end{History}
%
% \PrintIndex
%
% \Finale
\endinput
|
% \end{quote}
% Do not forget to quote the argument according to the demands
% of your shell.
%
% \paragraph{Generating the documentation.}
% You can use both the \xfile{.dtx} or the \xfile{.drv} to generate
% the documentation. The process can be configured by the
% configuration file \xfile{ltxdoc.cfg}. For instance, put this
% line into this file, if you want to have A4 as paper format:
% \begin{quote}
%   \verb|\PassOptionsToClass{a4paper}{article}|
% \end{quote}
% An example follows how to generate the
% documentation with pdf\LaTeX:
% \begin{quote}
%\begin{verbatim}
%pdflatex telprint.dtx
%makeindex -s gind.ist telprint.idx
%pdflatex telprint.dtx
%makeindex -s gind.ist telprint.idx
%pdflatex telprint.dtx
%\end{verbatim}
% \end{quote}
%
% \begin{History}
%   \begin{Version}{1996/11/28 v1.0}
%   \item
%     Erste lauff\"ahige Version.
%   \item
%     Nur '-' und '/' als zul\"assige Sonderzeichen.
%   \end{Version}
%   \begin{Version}{1997/09/16 v1.1}
%   \item
%     Dokumentation und Kommentare (Posting in de.comp.text.tex).
%   \item
%     Erweiterung um Sonderzeichen '(', ')', '+', '\textasciitilde' und ' '.
%   \item
%     Trennungsverhinderung am 'hyphen'.
%   \end{Version}
%   \begin{Version}{1997/10/16 v1.2}
%   \item
%     Schutz vor wiederholtem Einlesen.
%   \item
%     Unter \LaTeXe\ Nutzung des \cs{DeclareRobustCommand}-Features.
%   \end{Version}
%   \begin{Version}{1997/12/09 v1.3}
%   \item
%     Tempor\"are Variable eingespart.
%   \item
%     Posted in newsgroup \xnewsgroup{de.comp.text.tex}:\\
%     \URL{``\link{Re: Generisches Markup f\"ur Telefonnummern?}''}^^A
%     {http://groups.google.com/group/de.comp.text.tex/msg/86b3a86140007309}
%   \end{Version}
%   \begin{Version}{2004/11/02 v1.4}
%   \item
%     Fehler in der Dokumentation korrigiert.
%   \end{Version}
%   \begin{Version}{2005/09/30 v1.5}
%   \item
%     Konfigurierbare Symbole: '/', '(', ')', '+' und '\textasciitilde'.
%   \end{Version}
%   \begin{Version}{2006/02/12 v1.6}
%   \item
%     LPPL 1.3.
%   \item
%     Kurze \"Ubersicht in Englisch.
%   \item
%     CTAN.
%   \end{Version}
%   \begin{Version}{2006/08/26 v1.7}
%   \item
%     New DTX framework.
%   \end{Version}
%   \begin{Version}{2007/04/11 v1.8}
%   \item
%     Line ends sanitized.
%   \end{Version}
%   \begin{Version}{2007/09/09 v1.9}
%   \item
%     Catcode section added.
%   \item
%     Missing docstrip tag added.
%   \end{Version}
%   \begin{Version}{2008/08/11 v1.10}
%   \item
%     Code is not changed.
%   \item
%     URLs updated.
%   \end{Version}
%   \begin{Version}{2016/05/16 v1.11}
%   \item
%     Documentation updates.
%   \end{Version}
% \end{History}
%
% \PrintIndex
%
% \Finale
\endinput
|
% \end{quote}
% Do not forget to quote the argument according to the demands
% of your shell.
%
% \paragraph{Generating the documentation.}
% You can use both the \xfile{.dtx} or the \xfile{.drv} to generate
% the documentation. The process can be configured by the
% configuration file \xfile{ltxdoc.cfg}. For instance, put this
% line into this file, if you want to have A4 as paper format:
% \begin{quote}
%   \verb|\PassOptionsToClass{a4paper}{article}|
% \end{quote}
% An example follows how to generate the
% documentation with pdf\LaTeX:
% \begin{quote}
%\begin{verbatim}
%pdflatex telprint.dtx
%makeindex -s gind.ist telprint.idx
%pdflatex telprint.dtx
%makeindex -s gind.ist telprint.idx
%pdflatex telprint.dtx
%\end{verbatim}
% \end{quote}
%
% \begin{History}
%   \begin{Version}{1996/11/28 v1.0}
%   \item
%     Erste lauff\"ahige Version.
%   \item
%     Nur '-' und '/' als zul\"assige Sonderzeichen.
%   \end{Version}
%   \begin{Version}{1997/09/16 v1.1}
%   \item
%     Dokumentation und Kommentare (Posting in de.comp.text.tex).
%   \item
%     Erweiterung um Sonderzeichen '(', ')', '+', '\textasciitilde' und ' '.
%   \item
%     Trennungsverhinderung am 'hyphen'.
%   \end{Version}
%   \begin{Version}{1997/10/16 v1.2}
%   \item
%     Schutz vor wiederholtem Einlesen.
%   \item
%     Unter \LaTeXe\ Nutzung des \cs{DeclareRobustCommand}-Features.
%   \end{Version}
%   \begin{Version}{1997/12/09 v1.3}
%   \item
%     Tempor\"are Variable eingespart.
%   \item
%     Posted in newsgroup \xnewsgroup{de.comp.text.tex}:\\
%     \URL{``\link{Re: Generisches Markup f\"ur Telefonnummern?}''}^^A
%     {https://groups.google.com/group/de.comp.text.tex/msg/86b3a86140007309}
%   \end{Version}
%   \begin{Version}{2004/11/02 v1.4}
%   \item
%     Fehler in der Dokumentation korrigiert.
%   \end{Version}
%   \begin{Version}{2005/09/30 v1.5}
%   \item
%     Konfigurierbare Symbole: '/', '(', ')', '+' und '\textasciitilde'.
%   \end{Version}
%   \begin{Version}{2006/02/12 v1.6}
%   \item
%     LPPL 1.3.
%   \item
%     Kurze \"Ubersicht in Englisch.
%   \item
%     CTAN.
%   \end{Version}
%   \begin{Version}{2006/08/26 v1.7}
%   \item
%     New DTX framework.
%   \end{Version}
%   \begin{Version}{2007/04/11 v1.8}
%   \item
%     Line ends sanitized.
%   \end{Version}
%   \begin{Version}{2007/09/09 v1.9}
%   \item
%     Catcode section added.
%   \item
%     Missing docstrip tag added.
%   \end{Version}
%   \begin{Version}{2008/08/11 v1.10}
%   \item
%     Code is not changed.
%   \item
%     URLs updated.
%   \end{Version}
%   \begin{Version}{2016/05/16 v1.11}
%   \item
%     Documentation updates.
%   \end{Version}
% \end{History}
%
% \PrintIndex
%
% \Finale
\endinput
