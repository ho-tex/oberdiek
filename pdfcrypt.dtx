% \iffalse meta-comment
%
% File: pdfcrypt.dtx
% Version: 2016/05/16 v1.1
% Info: Allows the setting of PDF encryption
%
% Copyright (C) 2001, 2006, 2007 by
%    Heiko Oberdiek <heiko.oberdiek at googlemail.com>
%    2016
%    https://github.com/ho-tex/oberdiek/issues
%
% This work may be distributed and/or modified under the
% conditions of the LaTeX Project Public License, either
% version 1.3c of this license or (at your option) any later
% version. This version of this license is in
%    https://www.latex-project.org/lppl/lppl-1-3c.txt
% and the latest version of this license is in
%    https://www.latex-project.org/lppl.txt
% and version 1.3 or later is part of all distributions of
% LaTeX version 2005/12/01 or later.
%
% This work has the LPPL maintenance status "maintained".
%
% The Current Maintainers of this work are
% Heiko Oberdiek and the Oberdiek Package Support Group
% https://github.com/ho-tex/oberdiek/issues
%
% The Base Interpreter refers to any `TeX-Format',
% because some files are installed in TDS:tex/generic//.
%
% This work consists of the main source file pdfcrypt.dtx
% and the derived files
%    pdfcrypt.sty, pdfcrypt.pdf, pdfcrypt.ins, pdfcrypt.drv.
%
% Distribution:
%    CTAN:macros/latex/contrib/oberdiek/pdfcrypt.dtx
%    CTAN:macros/latex/contrib/oberdiek/pdfcrypt.pdf
%
% Unpacking:
%    (a) If pdfcrypt.ins is present:
%           tex pdfcrypt.ins
%    (b) Without pdfcrypt.ins:
%           tex pdfcrypt.dtx
%    (c) If you insist on using LaTeX
%           latex \let\install=y% \iffalse meta-comment
%
% File: pdfcrypt.dtx
% Version: 2016/05/16 v1.1
% Info: Allows the setting of PDF encryption
%
% Copyright (C)
%    2001, 2006, 2007 Heiko Oberdiek
%    2016-2019 Oberdiek Package Support Group
%    https://github.com/ho-tex/oberdiek/issues
%
% This work may be distributed and/or modified under the
% conditions of the LaTeX Project Public License, either
% version 1.3c of this license or (at your option) any later
% version. This version of this license is in
%    https://www.latex-project.org/lppl/lppl-1-3c.txt
% and the latest version of this license is in
%    https://www.latex-project.org/lppl.txt
% and version 1.3 or later is part of all distributions of
% LaTeX version 2005/12/01 or later.
%
% This work has the LPPL maintenance status "maintained".
%
% The Current Maintainers of this work are
% Heiko Oberdiek and the Oberdiek Package Support Group
% https://github.com/ho-tex/oberdiek/issues
%
% The Base Interpreter refers to any `TeX-Format',
% because some files are installed in TDS:tex/generic//.
%
% This work consists of the main source file pdfcrypt.dtx
% and the derived files
%    pdfcrypt.sty, pdfcrypt.pdf, pdfcrypt.ins, pdfcrypt.drv.
%
% Distribution:
%    CTAN:macros/latex/contrib/oberdiek/pdfcrypt.dtx
%    CTAN:macros/latex/contrib/oberdiek/pdfcrypt.pdf
%
% Unpacking:
%    (a) If pdfcrypt.ins is present:
%           tex pdfcrypt.ins
%    (b) Without pdfcrypt.ins:
%           tex pdfcrypt.dtx
%    (c) If you insist on using LaTeX
%           latex \let\install=y% \iffalse meta-comment
%
% File: pdfcrypt.dtx
% Version: 2016/05/16 v1.1
% Info: Allows the setting of PDF encryption
%
% Copyright (C)
%    2001, 2006, 2007 Heiko Oberdiek
%    2016-2019 Oberdiek Package Support Group
%    https://github.com/ho-tex/oberdiek/issues
%
% This work may be distributed and/or modified under the
% conditions of the LaTeX Project Public License, either
% version 1.3c of this license or (at your option) any later
% version. This version of this license is in
%    https://www.latex-project.org/lppl/lppl-1-3c.txt
% and the latest version of this license is in
%    https://www.latex-project.org/lppl.txt
% and version 1.3 or later is part of all distributions of
% LaTeX version 2005/12/01 or later.
%
% This work has the LPPL maintenance status "maintained".
%
% The Current Maintainers of this work are
% Heiko Oberdiek and the Oberdiek Package Support Group
% https://github.com/ho-tex/oberdiek/issues
%
% The Base Interpreter refers to any `TeX-Format',
% because some files are installed in TDS:tex/generic//.
%
% This work consists of the main source file pdfcrypt.dtx
% and the derived files
%    pdfcrypt.sty, pdfcrypt.pdf, pdfcrypt.ins, pdfcrypt.drv.
%
% Distribution:
%    CTAN:macros/latex/contrib/oberdiek/pdfcrypt.dtx
%    CTAN:macros/latex/contrib/oberdiek/pdfcrypt.pdf
%
% Unpacking:
%    (a) If pdfcrypt.ins is present:
%           tex pdfcrypt.ins
%    (b) Without pdfcrypt.ins:
%           tex pdfcrypt.dtx
%    (c) If you insist on using LaTeX
%           latex \let\install=y% \iffalse meta-comment
%
% File: pdfcrypt.dtx
% Version: 2016/05/16 v1.1
% Info: Allows the setting of PDF encryption
%
% Copyright (C)
%    2001, 2006, 2007 Heiko Oberdiek
%    2016-2019 Oberdiek Package Support Group
%    https://github.com/ho-tex/oberdiek/issues
%
% This work may be distributed and/or modified under the
% conditions of the LaTeX Project Public License, either
% version 1.3c of this license or (at your option) any later
% version. This version of this license is in
%    https://www.latex-project.org/lppl/lppl-1-3c.txt
% and the latest version of this license is in
%    https://www.latex-project.org/lppl.txt
% and version 1.3 or later is part of all distributions of
% LaTeX version 2005/12/01 or later.
%
% This work has the LPPL maintenance status "maintained".
%
% The Current Maintainers of this work are
% Heiko Oberdiek and the Oberdiek Package Support Group
% https://github.com/ho-tex/oberdiek/issues
%
% The Base Interpreter refers to any `TeX-Format',
% because some files are installed in TDS:tex/generic//.
%
% This work consists of the main source file pdfcrypt.dtx
% and the derived files
%    pdfcrypt.sty, pdfcrypt.pdf, pdfcrypt.ins, pdfcrypt.drv.
%
% Distribution:
%    CTAN:macros/latex/contrib/oberdiek/pdfcrypt.dtx
%    CTAN:macros/latex/contrib/oberdiek/pdfcrypt.pdf
%
% Unpacking:
%    (a) If pdfcrypt.ins is present:
%           tex pdfcrypt.ins
%    (b) Without pdfcrypt.ins:
%           tex pdfcrypt.dtx
%    (c) If you insist on using LaTeX
%           latex \let\install=y\input{pdfcrypt.dtx}
%        (quote the arguments according to the demands of your shell)
%
% Documentation:
%    (a) If pdfcrypt.drv is present:
%           latex pdfcrypt.drv
%    (b) Without pdfcrypt.drv:
%           latex pdfcrypt.dtx; ...
%    The class ltxdoc loads the configuration file ltxdoc.cfg
%    if available. Here you can specify further options, e.g.
%    use A4 as paper format:
%       \PassOptionsToClass{a4paper}{article}
%
%    Programm calls to get the documentation (example):
%       pdflatex pdfcrypt.dtx
%       makeindex -s gind.ist pdfcrypt.idx
%       pdflatex pdfcrypt.dtx
%       makeindex -s gind.ist pdfcrypt.idx
%       pdflatex pdfcrypt.dtx
%
% Installation:
%    TDS:tex/generic/oberdiek/pdfcrypt.sty
%    TDS:doc/latex/oberdiek/pdfcrypt.pdf
%    TDS:source/latex/oberdiek/pdfcrypt.dtx
%
%<*ignore>
\begingroup
  \catcode123=1 %
  \catcode125=2 %
  \def\x{LaTeX2e}%
\expandafter\endgroup
\ifcase 0\ifx\install y1\fi\expandafter
         \ifx\csname processbatchFile\endcsname\relax\else1\fi
         \ifx\fmtname\x\else 1\fi\relax
\else\csname fi\endcsname
%</ignore>
%<*install>
\input docstrip.tex
\Msg{************************************************************************}
\Msg{* Installation}
\Msg{* Package: pdfcrypt 2016/05/16 v1.1 Allows the setting of PDF encryption (HO)}
\Msg{************************************************************************}

\keepsilent
\askforoverwritefalse

\let\MetaPrefix\relax
\preamble

This is a generated file.

Project: pdfcrypt
Version: 2016/05/16 v1.1

Copyright (C)
   2001, 2006, 2007 Heiko Oberdiek
   2016-2019 Oberdiek Package Support Group

This work may be distributed and/or modified under the
conditions of the LaTeX Project Public License, either
version 1.3c of this license or (at your option) any later
version. This version of this license is in
   https://www.latex-project.org/lppl/lppl-1-3c.txt
and the latest version of this license is in
   https://www.latex-project.org/lppl.txt
and version 1.3 or later is part of all distributions of
LaTeX version 2005/12/01 or later.

This work has the LPPL maintenance status "maintained".

The Current Maintainers of this work are
Heiko Oberdiek and the Oberdiek Package Support Group
https://github.com/ho-tex/oberdiek/issues


The Base Interpreter refers to any `TeX-Format',
because some files are installed in TDS:tex/generic//.

This work consists of the main source file pdfcrypt.dtx
and the derived files
   pdfcrypt.sty, pdfcrypt.pdf, pdfcrypt.ins, pdfcrypt.drv.

\endpreamble
\let\MetaPrefix\DoubleperCent

\generate{%
  \file{pdfcrypt.ins}{\from{pdfcrypt.dtx}{install}}%
  \file{pdfcrypt.drv}{\from{pdfcrypt.dtx}{driver}}%
  \usedir{tex/generic/oberdiek}%
  \file{pdfcrypt.sty}{\from{pdfcrypt.dtx}{package}}%
  \nopreamble
  \nopostamble
%  \usedir{source/latex/oberdiek/catalogue}%
%  \file{pdfcrypt.xml}{\from{pdfcrypt.dtx}{catalogue}}%
}

\catcode32=13\relax% active space
\let =\space%
\Msg{************************************************************************}
\Msg{*}
\Msg{* To finish the installation you have to move the following}
\Msg{* file into a directory searched by TeX:}
\Msg{*}
\Msg{*     pdfcrypt.sty}
\Msg{*}
\Msg{* To produce the documentation run the file `pdfcrypt.drv'}
\Msg{* through LaTeX.}
\Msg{*}
\Msg{* Happy TeXing!}
\Msg{*}
\Msg{************************************************************************}

\endbatchfile
%</install>
%<*ignore>
\fi
%</ignore>
%<*driver>
\NeedsTeXFormat{LaTeX2e}
\ProvidesFile{pdfcrypt.drv}%
  [2016/05/16 v1.1 Allows the setting of PDF encryption (HO)]%
\documentclass{ltxdoc}
\usepackage{holtxdoc}[2011/11/22]
\begin{document}
  \DocInput{pdfcrypt.dtx}%
\end{document}
%</driver>
% \fi
%
%
% \CharacterTable
%  {Upper-case    \A\B\C\D\E\F\G\H\I\J\K\L\M\N\O\P\Q\R\S\T\U\V\W\X\Y\Z
%   Lower-case    \a\b\c\d\e\f\g\h\i\j\k\l\m\n\o\p\q\r\s\t\u\v\w\x\y\z
%   Digits        \0\1\2\3\4\5\6\7\8\9
%   Exclamation   \!     Double quote  \"     Hash (number) \#
%   Dollar        \$     Percent       \%     Ampersand     \&
%   Acute accent  \'     Left paren    \(     Right paren   \)
%   Asterisk      \*     Plus          \+     Comma         \,
%   Minus         \-     Point         \.     Solidus       \/
%   Colon         \:     Semicolon     \;     Less than     \<
%   Equals        \=     Greater than  \>     Question mark \?
%   Commercial at \@     Left bracket  \[     Backslash     \\
%   Right bracket \]     Circumflex    \^     Underscore    \_
%   Grave accent  \`     Left brace    \{     Vertical bar  \|
%   Right brace   \}     Tilde         \~}
%
% \GetFileInfo{pdfcrypt.drv}
%
% \title{The \xpackage{pdfcrypt} package}
% \date{2016/05/16 v1.1}
% \author{Heiko Oberdiek\thanks
% {Please report any issues at \url{https://github.com/ho-tex/oberdiek/issues}}}
%
% \maketitle
%
% \begin{abstract}
% This package supports the setting of pdf encryption options
% for \VTeX\ and some older versions of \pdfTeX.
% \end{abstract}
%
% \tableofcontents
%
% \section{Documentation}
%
% This package allows the setting of pdf encryption
% options for
% \begin{itemize}
% \item
%   \VTeX, since version 7.35, \url{http://www.micropress-inc.com/}.
% \item
%   \pdfTeX, patched by
%   \NameEmail{Ricardo Sanchez Carmenes}{carmenes@bioquimica.uniovi.es}.\\
%   Some supported versions are:
%   \begin{quote}
%     \pdfTeX-1.00a,\\
%     \pdfTeX-0.14h-pdfcrypt-20010310,\\
%     pdfcrypt-20010331
%   \end{quote}
% \end{itemize}
%
% \noindent
% \textcolor{red}{Note:} Since \pdfTeX-1.10a (2003-01-16) encryption
% support was dropped! Thus the package is now obsolete for recent
% \pdfTeX\ versions.
%
% \subsection{Alternatives}
%
% There are free alternatives, programs that can
% be used for postprocessing the pdf file:
% \begin{itemize}
% \item \textsf{pdftk}\\
%   \url{http://www.accesspdf.com/pdftk/}
% \item \textsf{Multivalent}\\
%   \url{http://multivalent.sourceforge.net/}
% \item \textsf{PDFBox}\\
%   \url{http://www.pdfbox.org/}
% \item \textsf{PDFTrans}\\
%   \url{http://maddingue.free.fr/softwares/pdftrans.html}
% \item \dots
% \end{itemize}
%
% \subsection{Usage}
%
% The options can be set as package options or
% with the command \cs{pdfcryptsetup}:
% \begin{quote}
%   |owner=|\meta{owner password}\\
%   |user=|\meta{user password}\\
%   |print|, |copy|, |edit|, |annotate=true|, |false|\\
%   |all|, |none|\\
%   |debug|
% \end{quote}
% The encryption is set at |\begin{document}| by default,
% but this can be forced for an earlier time
% using the option |set|.
% Example:
% \begin{quote}
%   |\usepackage[owner=Donald,user=Knuth,print=false]{pdfcrypt}|\\
%   \dots\\
%   |\begin{document}|
% \end{quote}
% or
% \begin{quote}
%   |\usepackage{pdfcrypt}|\\
%   |\pdfcryptsetup{owner=Donald,user=Knut}|\\
%   |\pdfcryptsetup{print=false,set}|\\
%   \dots\\
%   |\begin{document}|
% \end{quote}
%
% \subsubsection{Special characters}
%
% The characters `|{|', `|}|', and `|\|' have to be
% escaped with `|\|'. Then the string should not
% be specified as package option but with
% the macro \cs{pdfcryptsetup}, eg:
% \begin{quote}
%   |\pdfcryptsetup{user={\{\\Hello\}},print=false}|
% \end{quote}
% The password here is ``|{\Hello}|''.
% Active characters can be used and are not expanded.
% Macros except for |\{|, |\}|, and |\\| should not be used
% and are not expanded.
%
% \subsubsection{\cs{nopdfcrypt}}
%
% Often the whole process of pdf generation includes
% several \TeX\ compilations and perhaps other
% postprocessing steps. Therefore a feature would
% be useful to disable the encryption stuff in order
% to speed up the first compilations.
% Therefore package \xpackage{pdfcrypt} will look for
% the macro \cs{nopdfcrypt}. If it has the meaning of the
% letter `|Y|', the package will be disabled.
% Example:
% \begin{quote}
%   |pdflatex '\let\nopdfcrypt=Y\input myfile'|\\
%   |thumbpdf myfile|\\
%   |pdfelatex myfile|
% \end{quote}
%
% \subsubsection{Configuration file}
%
% If the file \xfile{pdfcrypt.cfg} exists it is loaded
% and can be used for common settings.
%
% \subsubsection{Support for \plainTeX}
%
% The package can also be used with \plainTeX. It defines the
% missing \LaTeX\ macros and loads package \xpackage{keyval}.
% The configuration file is not supported and automatically
% loaded.
%
% \subsection{Hints for \pdfTeX}
%
% This section is out of date, because the encryption support
% was removed from \pdfTeX. Therefore this hints are of
% historical interest only.
%
%\begin{verbatim}
%* There are packages such as thumbpdf that ship out
%  some stuff (\immediate\pdfobj{...}). In these cases
%  \pdfcrypt will be ignored without warning or error
%  message. Therefore the package pdfcrypt should be
%  loaded before and option "set" should be used in
%  order to force the call of \pdfcrypt, for example:
%     \usepackage{pdfcrypt}
%     \pdfcryptsetup{..., set}
%     ...
%     \usepackage{thumbpdf}
%
%* Compiling pdfTeX (eg. version 1.00a-pretest-20010804):
%  pdfTeX versions are available at
%      ftp://ftp.cstug.cz/pub/tex/local/cstug/thanh/pdftex/snapshots/
%  Unpack the latest .tgz file:
%      > tar xzf pdftex-20010804.tgz
%  Enable encryption support:
%      src> cd src/texk/web2c/pdftexdir
%      src/texk/web2c/pdftexdir> ln -fs pdfcrypt-full.c pdfcrypt.c
%  Compile:
%      src/texk/web2c/pdftexdir> cd ../../..
%      src> ./configure
%      src> cd texk/web2c
%      src/texk/web2c> make pdftexbin
%  At last install the binary and pool files.
%\end{verbatim}
%
% \StopEventually{
% }
%
% \section{Implementation}
%
%    \begin{macrocode}
%<*package>
%    \end{macrocode}
%
% \subsection{Help macros for \plainTeX}
%
%    \begin{macrocode}
\expandafter\edef\csname pc@endinput\endcsname{%
  \catcode`\noexpand\@=\the\catcode`\@\relax
  \noexpand\endinput
}
\catcode`\@=11 %

\expandafter\ifx\csname @firstoftwo\endcsname\relax
  \long\def\@firstoftwo#1#2{#1}%
\fi

\expandafter\ifx\csname @secondoftwo\endcsname\relax
  \long\def\@secondoftwo#1#2{#2}%
\fi

\expandafter\ifx\csname @ifundefined\endcsname\relax
  \def\@ifundefined#1{%
    \expandafter\ifx\csname #1\endcsname\relax
      \expandafter\@firstoftwo
    \else
      \expandafter\@secondoftwo
    \fi
  }%
\fi

\@ifundefined{@ifnextchar}{%
  \long\def\@ifnextchar#1#2#3{%
    \let\reserved@d=#1%
    \def\reserved@a{#2}%
    \def\reserved@b{#3}%
    \futurelet\@let@token\@ifnch
  }%
  \def\@ifnch{%
    \ifx\@let@token\@sptoken
      \let\reserved@c\@xifnch
    \else
      \ifx\@let@token\reserved@d
        \let\reserved@c\reserved@a
      \else
        \let\reserved@c\reserved@b
      \fi
    \fi
    \reserved@c
  }%
  \begingroup
    \def\:{\global\let\@sptoken= }%
    \: %
    \def\:{\@xifnch}%
    \expandafter\gdef\: {\futurelet\@let@token\@ifnch}%
  \endgroup
}{}

\@ifundefined{ProvidesPackage}{%
  \def\ProvidesPackage#1{%
    \@ifnextchar[{\pc@ProvidesPackage{#1}}%
                 {\pc@ProvidesPackage{#1}[]}%
  }%
  \def\pc@ProvidesPackage#1[#2]{%
    \immediate\write-1{Package: #1 #2}%
  }%
}{}

\begingroup\expandafter\expandafter\expandafter\endgroup
\expandafter\ifx\csname RequirePackage\endcsname\relax
  \input infwarerr.sty\relax
\else
  \RequirePackage{infwarerr}%
\fi

\@ifundefined{@gobble}{%
  \long\def\@gobble#1{}%
}{}

\@ifundefined{@empty}{%
  \def\@empty{}%
}{}
%    \end{macrocode}
%
% \subsection{Package Identification and checks}
%
%    \begin{macrocode}
\ProvidesPackage{pdfcrypt}%
  [2016/05/16 v1.1 Allows the setting of PDF encryption (HO)]%
%    \end{macrocode}
%
%    \begin{macrocode}
\@ifundefined{pdfcryptsetup}{%
  \let\pdfcryptsetup\@gobble
}{%
  \@PackageWarningNoLine{pdfcrypt}{Package pdfcrypt is already loaded}%
  \pc@endinput
}
%    \end{macrocode}
%    Support for \cs{nopdfcrypt}.
%    \begin{macrocode}
\newif\ifpc@nopdfcrypt
\ifx Y\nopdfcrypt
  \@PackageWarningNoLine{pdfcrypt}{%
    Encryption disabled by \string \nopdfcrypt\space request%
  }%
  \global\pc@nopdfcrypttrue
\fi
%    \end{macrocode}
%
% \subsection{Driver detection}
%
%    \begin{macrocode}
\let\pc@driver\@empty
\begingroup
  % pdfTeX detection
  \@ifundefined{pdfoutput}{%
  }{%
    \ifcase\pdfoutput
    \else
      \@ifundefined{pdfcrypt}{%
        \@PackageError{pdfcrypt}{%
          PDF encryption is not supported with this pdfTeX%
        }{%
          Encryption support was added in 0.14h (2001/03/10)\MessageBreak
          and removed in 1.10a (2003/01/16).%
        }%
        \endgroup
        \pc@endinput
      }{%
        \gdef\pc@driver{pdftex}%
      }%
    \fi
  }%
  % VTeX detection
  \@ifundefined{OpMode}{%
  }{%
    \ifnum\OpMode=1 %
      \ifnum\@ifundefined{VTeXversion}0\VTeXversion<735 %
        \@PackageError{pdfcrypt}{%
          PDF encryption is not supported with this VTeX%
        }{%
          You need VTeX 7.35 or higher.%
        }%
        \endgroup
        \pc@endinput
      \else
        \gdef\pc@driver{vtex}%
      \fi
    \fi
  }%
\endgroup
%    \end{macrocode}
%
% \subsection{Load package \xpackage{keyval}}
%
%    \begin{macrocode}
\@ifundefined{@makeother}{%
  \def\@makeother#1{\catcode`#1=12\relax}%
}{}

\@ifundefined{g@addto@macro}{%
  \long\def\g@addto@macro#1#2{%
    \begingroup
      \toks@\expandafter{#1#2}%
      \xdef#1{\the\toks@}%
    \endgroup
  }%
}{}

\@ifundefined{@namedef}{%
  \def\@namedef#1{\expandafter\def\csname#1\endcsname}%
}{}

\@ifundefined{@nameuse}{%
  \def\@nameuse#1{\csname #1\endcsname}%
}{}

\def\pc@KeyvalRestore{%
  \let\pc@KeyvalRestore\@undefined
}

\let\pcOrg@NeedsTeXFormat\NeedsTeXFormat
\@ifundefined{NeedsTeXFormat}{%
  \def\NeedsTeXFormat#1{}%
  \g@addto@macro\pc@KeyvalRestore{%
    \let\NeedsTeXFormat\pcOrg@NeedsTeXFormat
  }%
}{}

\let\pcOrg@DeclareOption\DeclareOption
\@ifundefined{DeclareOption}{%
  \def\DeclareOption#1#2{#2}%
  \g@addto@macro\pc@KeyvalRestore{%
    \let\DeclareOption\pcOrg@DeclareOption
  }%
}{}

\let\pcOrg@ExecuteOptions\ExecuteOptions
\@ifundefined{ExecuteOptions}{%
  \def\ExecuteOptions#1{}%
  \g@addto@macro\pc@KeyvalRestore{%
    \let\ExecuteOptions\pcOrg@ExecuteOptions
  }%
}{}

\let\pcOrg@ProcessOptions\ProcessOptions
\@ifundefined{ProcessOptions}{%
  \def\ProcessOptions{}%
  \g@addto@macro\pc@KeyvalRestore{%
    \let\ProcessOptions\pcOrg@ProcessOptions
  }%
}{}%

\begingroup\expandafter\expandafter\expandafter\endgroup
\expandafter\ifx\csname RequirePackage\endcsname\relax
  \input keyval.sty\relax
\else
  \RequirePackage{keyval}%
\fi
\pc@KeyvalRestore
%    \end{macrocode}
%
% \subsection{Define options}
%
%    \begin{macrocode}
\@ifundefined{@dblarg}{%
  \long\def\@dblarg#1{\@ifnextchar[{#1}{\@xdblarg{#1}}}%
  \long\def\@xdblarg#1#2{#1[{#2}]{#2}}%
}{}

\newif\ifpc@set
\newif\ifpc@print
\newif\ifpc@copy
\newif\ifpc@edit
\newif\ifpc@annotate
\newif\ifpc@debug
\let\pc@owner\@empty
\let\pc@user\@empty

% default: allow all
\pc@printtrue
\pc@copytrue
\pc@edittrue
\pc@annotatetrue

\edef\pc@temp{\catcode`\noexpand\"=\the\catcode`\"\relax}
\@makeother\"
\def\pc@set{%
  \@PackageInfo{pdfcrypt}{%
    \ifpc@debug
      \ifx\pc@owner\@empty
        No owner password%
      \else
        Owner password: `\pc@owner'%
      \fi
      \MessageBreak
      \ifx\pc@user\@empty
        No user password%
      \else
        User password: `\pc@user'%
      \fi
      \MessageBreak
      Flags: %
      \ifpc@print   \else no\fi print, %
      \ifpc@copy    \else no\fi copy, %
      \ifpc@edit    \else no\fi edit, %
      \ifpc@annotate\else no\fi annotate%
      \MessageBreak
    \fi
    \ifpc@nopdfcrypt
      Encryption is disabled by `\string\nopdfcrypt'%
    \else
      Encryption is set for `\pc@driver'%
    \fi
  }%
  \ifpc@nopdfcrypt
  \else
    \@ifundefined{pc@set@\pc@driver}{%
      \ifx\pc@driver\@empty
        \@PackageError{pdfcrypt}{No driver for encryption %
          support found}\@ehc
      \else
        \@PackageError{pdfcrypt}{Cannot set encryption for %
          unknown driver `\pc@driver'}\@ehc
      \fi
    }{%
      \@nameuse{pc@set@\pc@driver}%
    }%
  \fi
}
\def\pc@set@pdftex{%
  \ifnum\pdftexversion<100 %
    \pc@set@pdftexold
  \else
    \pc@set@pdftexnew
  \fi
}
\def\pc@set@pdftexold{%
  \pdfcrypt{%
    owner "\pc@owner" %
    user "\pc@user" %
    \ifpc@print   \else no\fi print %
    \ifpc@copy    \else no\fi copy %
    \ifpc@edit    \else no\fi edit %
    \ifpc@annotate\else no\fi annotate%
  }%
}
\def\pc@set@pdftexnew{%
  \pdfcrypt
    owner{\pc@owner}%
    user{\pc@user}%
    \ifpc@print   \else no\fi print %
    \ifpc@copy    \else no\fi copy %
    \ifpc@edit    \else no\fi edit %
    \ifpc@annotate\else no\fi annotate%
  \relax
}
\def\pc@set@vtex{%
  \immediate\special{!security %
    O=\pc@MakeVTeXString\pc@owner,%
    U=\pc@MakeVTeXString\pc@user,%
    P\ifpc@print +\else -\fi,%
    C\ifpc@copy +\else -\fi,%
    M\ifpc@edit +\else -\fi,%
    A\ifpc@annotate +\else -\fi
  }%
}
\def\pc@MakeVTeXString#1{%
  "\expandafter\pc@@MakeVTeXString#1"\@nil"%
}
\def\pc@@MakeVTeXString#1"#2\@nil{%
  #1%
  \ifx\\#2\\%
  \else
    ""%
    \@ReturnAfterFi{%
      \pc@@MakeVTeXString#2\@nil
    }%
  \fi
}
\long\def\@ReturnAfterFi#1\fi{\fi#1}
\pc@temp

\begingroup
  \catcode`\ =12 \gdef\pc@spaceother{ }\catcode`\ =10\relax
  \catcode`\|=0 %
  \catcode`\\=12 %
  |gdef|pc@DefString#1#2{%
    |def#1{#2}%
    |edef#1{|expandafter|strip@prefix|meaning#1}%
    |edef#1{|expandafter|pc@SpaceToOther#1 |@nil}%
    |edef#1{|expandafter|pc@EscapeRemove#1|@empty\|@empty|@nil}%
  }%
  |gdef|pc@EscapeRemove#1\#2#3|@nil{%
    #1#2%
    |ifx|\#3|\%%
    |else
      |@ReturnAfterFi{%
        |pc@EscapeRemove#3|@nil
      }%
    |fi
  }%
|endgroup
\def\pc@SpaceToOther#1 #2\@nil{%
  #1%
  \ifx\\#2\\%
  \else
    \pc@spaceother
    \@ReturnAfterFi{%
      \pc@SpaceToOther#2\@nil
    }%
  \fi
}

\def\pc@boolkey{\@dblarg\pc@@boolkey}
\def\pc@@boolkey[#1]#2#3{%
  \lowercase{\def\pc@temp{#3}}%
  \ifx\pc@temp\@empty
    \let\pc@temp\pc@true
  \fi
  \ifx\pc@temp\pc@true
  \else
    \ifx\pc@temp\pc@false
    \else
      \let\pc@temp\relax
    \fi
  \fi
  \ifx\pc@temp\relax
    \@PackageWarning{pdfcrypt}{%
      Unexpected value \string`#3\string' of %
      option \string`#2\string'\MessageBreak
      instead of %
      \string`true\string' or \string`false\string'%
    }%
  \else
    \csname pc@#2\pc@temp\endcsname
  \fi
}
\def\pc@true{true}
\def\pc@false{false}

\define@key{pc}{set}[true]{%
  \pc@boolkey{set}{#1}%
}
\define@key{pc}{pdftex}[]{%
  \def\pc@driver{pdftex}%
}
\define@key{pc}{vtex}[]{%
  \def\pc@driver{vtex}%
}
\define@key{pc}{print}[true]{%
  \pc@boolkey{print}{#1}%
}
\define@key{pc}{copy}[true]{%
  \pc@boolkey{copy}{#1}%
}
\define@key{pc}{edit}[true]{%
  \pc@boolkey{edit}{#1}%
}
\define@key{pc}{annotate}[true]{%
  \pc@boolkey{annotate}{#1}%
}
\define@key{pc}{all}[]{%
  \pc@boolkey{print}{true}%
  \pc@boolkey{copy}{true}%
  \pc@boolkey{edit}{true}%
  \pc@boolkey{annotate}{true}%
}
\define@key{pc}{none}[]{%
  \pc@boolkey{print}{false}%
  \pc@boolkey{copy}{false}%
  \pc@boolkey{edit}{false}%
  \pc@boolkey{annotate}{false}%
}

\define@key{pc}{owner}{%
  \pc@DefString\pc@owner{#1}%
}
\define@key{pc}{user}{%
  \pc@DefString\pc@user{#1}%
}
\define@key{pc}{debug}[true]{%
  \pc@boolkey{debug}{#1}%
}

\def\pdfcryptsetup#1{%
  \setkeys{pc}{#1}%
  \ifpc@set
    \pc@set
    \global\let\pc@set\relax
    \gdef\pdfcryptsetup##1{%
      \@PackageWarning{pdfcrypt}{%
        Encryption options are already set\MessageBreak
        new values are ignored%
      }%
    }%
  \fi
}
\begingroup\expandafter\expandafter\expandafter\endgroup
\expandafter\ifx\csname @onlypreamble\endcsname\relax
\else
  \@onlypreamble\pdfcryptsetup
\fi
%    \end{macrocode}
%
% \subsection{support of configuration file}
%    \begin{macrocode}
\begingroup\expandafter\expandafter\expandafter\endgroup
\expandafter\ifx\csname InputIfFileExists\endcsname\relax
  \@PackageInfo{pdfcrypt}{%
    Configuration file pdfcrypt.cfg not supported.%
  }%
\else
  \let\pc@ExecuteOptions\ExecuteOptions
  \InputIfFileExists{pdfcrypt.cfg}{}{}%
  \let\ExecuteOptions\pc@ExecuteOptions
\fi
%    \end{macrocode}
%
% \subsection{Package options}
%
%    Plain format does not know package options.
%    \begin{macrocode}
\begingroup\expandafter\expandafter\expandafter\endgroup
\expandafter\ifx\csname @classoptionslist\endcsname\relax
  \expandafter\pc@endinput
\fi
%    \end{macrocode}
%    Process global and local options.
%    \begin{macrocode}
\def\pc@ProcessOptionsWithKV{%
  \let\pc@temp\@empty
  \@for\CurrentOption:=\@classoptionslist\do{%
    \@ifundefined{KV@pc@\CurrentOption}{}{%
      \edef\pc@temp{\pc@temp,\CurrentOption,}%
      \@expandtwoargs\@removeelement\CurrentOption
        \@unusedoptionlist\@unusedoptionlist
    }%
  }%
  \edef\pc@temp{%
    \noexpand\pdfcryptsetup{%
      \pc@temp\@ptionlist{\@currname.\@currext}%
    }%
  }%
  \pc@temp
}
\pc@ProcessOptionsWithKV
\AtEndOfPackage{\let\@unprocessedoptions\relax}
\AtBeginDocument{\pc@set}

\pc@endinput
%    \end{macrocode}
%
%    \begin{macrocode}
%</package>
%    \end{macrocode}
%
% \section{Installation}
%
% \subsection{Download}
%
% \paragraph{Package.} This package is available on
% CTAN\footnote{\CTANpkg{pdfcrypt}}:
% \begin{description}
% \item[\CTAN{macros/latex/contrib/oberdiek/pdfcrypt.dtx}] The source file.
% \item[\CTAN{macros/latex/contrib/oberdiek/pdfcrypt.pdf}] Documentation.
% \end{description}
%
%
% \paragraph{Bundle.} All the packages of the bundle `oberdiek'
% are also available in a TDS compliant ZIP archive. There
% the packages are already unpacked and the documentation files
% are generated. The files and directories obey the TDS standard.
% \begin{description}
% \item[\CTANinstall{install/macros/latex/contrib/oberdiek.tds.zip}]
% \end{description}
% \emph{TDS} refers to the standard ``A Directory Structure
% for \TeX\ Files'' (\CTANpkg{tds}). Directories
% with \xfile{texmf} in their name are usually organized this way.
%
% \subsection{Bundle installation}
%
% \paragraph{Unpacking.} Unpack the \xfile{oberdiek.tds.zip} in the
% TDS tree (also known as \xfile{texmf} tree) of your choice.
% Example (linux):
% \begin{quote}
%   |unzip oberdiek.tds.zip -d ~/texmf|
% \end{quote}
%
% \subsection{Package installation}
%
% \paragraph{Unpacking.} The \xfile{.dtx} file is a self-extracting
% \docstrip\ archive. The files are extracted by running the
% \xfile{.dtx} through \plainTeX:
% \begin{quote}
%   \verb|tex pdfcrypt.dtx|
% \end{quote}
%
% \paragraph{TDS.} Now the different files must be moved into
% the different directories in your installation TDS tree
% (also known as \xfile{texmf} tree):
% \begin{quote}
% \def\t{^^A
% \begin{tabular}{@{}>{\ttfamily}l@{ $\rightarrow$ }>{\ttfamily}l@{}}
%   pdfcrypt.sty & tex/generic/oberdiek/pdfcrypt.sty\\
%   pdfcrypt.pdf & doc/latex/oberdiek/pdfcrypt.pdf\\
%   pdfcrypt.dtx & source/latex/oberdiek/pdfcrypt.dtx\\
% \end{tabular}^^A
% }^^A
% \sbox0{\t}^^A
% \ifdim\wd0>\linewidth
%   \begingroup
%     \advance\linewidth by\leftmargin
%     \advance\linewidth by\rightmargin
%   \edef\x{\endgroup
%     \def\noexpand\lw{\the\linewidth}^^A
%   }\x
%   \def\lwbox{^^A
%     \leavevmode
%     \hbox to \linewidth{^^A
%       \kern-\leftmargin\relax
%       \hss
%       \usebox0
%       \hss
%       \kern-\rightmargin\relax
%     }^^A
%   }^^A
%   \ifdim\wd0>\lw
%     \sbox0{\small\t}^^A
%     \ifdim\wd0>\linewidth
%       \ifdim\wd0>\lw
%         \sbox0{\footnotesize\t}^^A
%         \ifdim\wd0>\linewidth
%           \ifdim\wd0>\lw
%             \sbox0{\scriptsize\t}^^A
%             \ifdim\wd0>\linewidth
%               \ifdim\wd0>\lw
%                 \sbox0{\tiny\t}^^A
%                 \ifdim\wd0>\linewidth
%                   \lwbox
%                 \else
%                   \usebox0
%                 \fi
%               \else
%                 \lwbox
%               \fi
%             \else
%               \usebox0
%             \fi
%           \else
%             \lwbox
%           \fi
%         \else
%           \usebox0
%         \fi
%       \else
%         \lwbox
%       \fi
%     \else
%       \usebox0
%     \fi
%   \else
%     \lwbox
%   \fi
% \else
%   \usebox0
% \fi
% \end{quote}
% If you have a \xfile{docstrip.cfg} that configures and enables \docstrip's
% TDS installing feature, then some files can already be in the right
% place, see the documentation of \docstrip.
%
% \subsection{Refresh file name databases}
%
% If your \TeX~distribution
% (\TeX\,Live, \mikTeX, \dots) relies on file name databases, you must refresh
% these. For example, \TeX\,Live\ users run \verb|texhash| or
% \verb|mktexlsr|.
%
% \subsection{Some details for the interested}
%
% \paragraph{Unpacking with \LaTeX.}
% The \xfile{.dtx} chooses its action depending on the format:
% \begin{description}
% \item[\plainTeX:] Run \docstrip\ and extract the files.
% \item[\LaTeX:] Generate the documentation.
% \end{description}
% If you insist on using \LaTeX\ for \docstrip\ (really,
% \docstrip\ does not need \LaTeX), then inform the autodetect routine
% about your intention:
% \begin{quote}
%   \verb|latex \let\install=y\input{pdfcrypt.dtx}|
% \end{quote}
% Do not forget to quote the argument according to the demands
% of your shell.
%
% \paragraph{Generating the documentation.}
% You can use both the \xfile{.dtx} or the \xfile{.drv} to generate
% the documentation. The process can be configured by the
% configuration file \xfile{ltxdoc.cfg}. For instance, put this
% line into this file, if you want to have A4 as paper format:
% \begin{quote}
%   \verb|\PassOptionsToClass{a4paper}{article}|
% \end{quote}
% An example follows how to generate the
% documentation with pdf\LaTeX:
% \begin{quote}
%\begin{verbatim}
%pdflatex pdfcrypt.dtx
%makeindex -s gind.ist pdfcrypt.idx
%pdflatex pdfcrypt.dtx
%makeindex -s gind.ist pdfcrypt.idx
%pdflatex pdfcrypt.dtx
%\end{verbatim}
% \end{quote}
%
% \begin{History}
%   \begin{Version}{2001/04/02 v0.1}
%   \item
%     First public version, published in the pdftex
%     mailing list for testing with pdfcrypt-20010331
%   \end{Version}
%   \begin{Version}{2001/07/19 v0.2}
%   \item
%     Default: all allowed.
%   \item
%     Support for \VTeX\ added.
%   \end{Version}
%   \begin{Version}{2001/07/19 v0.3}
%   \item
%     Bug fix: \VTeX\ letter for edit is M (modify).
%   \end{Version}
%   \begin{Version}{2001/07/19 v0.4}
%   \item
%     Bug fix: \cs{VTeXversion} is correct after
%     regenerating the format file.
%   \end{Version}
%   \begin{Version}{2001/08/05 v0.5}
%   \item
%     Syntax change in \pdfTeX\ 1.00a.
%   \end{Version}
%   \begin{Version}{2001/08/09 v0.6}
%   \item
%     Support of special characters:\\
%     input: |\{|, |\}|, |\\| for |{|, |}|, |\|\\
%     output: |"| in \VTeX
%   \item
%     Option debug added.
%   \end{Version}
%   \begin{Version}{2001/10/28 v0.7}
%   \item
%     Plain compatibility.
%   \item
%     \cs{nopdfcrypt} added.
%   \item
%     Typos corrected.
%   \end{Version}
%   \begin{Version}{2006/02/20 v0.8}
%   \item
%     Obsolete remarks for \pdfTeX.
%   \item
%     DTX framework.
%   \item
%     LPPL 1.3
%   \end{Version}
%   \begin{Version}{2007/04/11 v0.9}
%   \item
%     Line ends sanitized.
%   \end{Version}
%   \begin{Version}{2007/04/26 v1.0}
%   \item
%     Use of package \xpackage{infwarerr}.
%   \end{Version}
%   \begin{Version}{2016/05/16 v1.1}
%   \item
%     Documentation updates.
%   \end{Version}
% \end{History}
%
% \PrintIndex
%
% \Finale
\endinput

%        (quote the arguments according to the demands of your shell)
%
% Documentation:
%    (a) If pdfcrypt.drv is present:
%           latex pdfcrypt.drv
%    (b) Without pdfcrypt.drv:
%           latex pdfcrypt.dtx; ...
%    The class ltxdoc loads the configuration file ltxdoc.cfg
%    if available. Here you can specify further options, e.g.
%    use A4 as paper format:
%       \PassOptionsToClass{a4paper}{article}
%
%    Programm calls to get the documentation (example):
%       pdflatex pdfcrypt.dtx
%       makeindex -s gind.ist pdfcrypt.idx
%       pdflatex pdfcrypt.dtx
%       makeindex -s gind.ist pdfcrypt.idx
%       pdflatex pdfcrypt.dtx
%
% Installation:
%    TDS:tex/generic/oberdiek/pdfcrypt.sty
%    TDS:doc/latex/oberdiek/pdfcrypt.pdf
%    TDS:source/latex/oberdiek/pdfcrypt.dtx
%
%<*ignore>
\begingroup
  \catcode123=1 %
  \catcode125=2 %
  \def\x{LaTeX2e}%
\expandafter\endgroup
\ifcase 0\ifx\install y1\fi\expandafter
         \ifx\csname processbatchFile\endcsname\relax\else1\fi
         \ifx\fmtname\x\else 1\fi\relax
\else\csname fi\endcsname
%</ignore>
%<*install>
\input docstrip.tex
\Msg{************************************************************************}
\Msg{* Installation}
\Msg{* Package: pdfcrypt 2016/05/16 v1.1 Allows the setting of PDF encryption (HO)}
\Msg{************************************************************************}

\keepsilent
\askforoverwritefalse

\let\MetaPrefix\relax
\preamble

This is a generated file.

Project: pdfcrypt
Version: 2016/05/16 v1.1

Copyright (C)
   2001, 2006, 2007 Heiko Oberdiek
   2016-2019 Oberdiek Package Support Group

This work may be distributed and/or modified under the
conditions of the LaTeX Project Public License, either
version 1.3c of this license or (at your option) any later
version. This version of this license is in
   https://www.latex-project.org/lppl/lppl-1-3c.txt
and the latest version of this license is in
   https://www.latex-project.org/lppl.txt
and version 1.3 or later is part of all distributions of
LaTeX version 2005/12/01 or later.

This work has the LPPL maintenance status "maintained".

The Current Maintainers of this work are
Heiko Oberdiek and the Oberdiek Package Support Group
https://github.com/ho-tex/oberdiek/issues


The Base Interpreter refers to any `TeX-Format',
because some files are installed in TDS:tex/generic//.

This work consists of the main source file pdfcrypt.dtx
and the derived files
   pdfcrypt.sty, pdfcrypt.pdf, pdfcrypt.ins, pdfcrypt.drv.

\endpreamble
\let\MetaPrefix\DoubleperCent

\generate{%
  \file{pdfcrypt.ins}{\from{pdfcrypt.dtx}{install}}%
  \file{pdfcrypt.drv}{\from{pdfcrypt.dtx}{driver}}%
  \usedir{tex/generic/oberdiek}%
  \file{pdfcrypt.sty}{\from{pdfcrypt.dtx}{package}}%
  \nopreamble
  \nopostamble
%  \usedir{source/latex/oberdiek/catalogue}%
%  \file{pdfcrypt.xml}{\from{pdfcrypt.dtx}{catalogue}}%
}

\catcode32=13\relax% active space
\let =\space%
\Msg{************************************************************************}
\Msg{*}
\Msg{* To finish the installation you have to move the following}
\Msg{* file into a directory searched by TeX:}
\Msg{*}
\Msg{*     pdfcrypt.sty}
\Msg{*}
\Msg{* To produce the documentation run the file `pdfcrypt.drv'}
\Msg{* through LaTeX.}
\Msg{*}
\Msg{* Happy TeXing!}
\Msg{*}
\Msg{************************************************************************}

\endbatchfile
%</install>
%<*ignore>
\fi
%</ignore>
%<*driver>
\NeedsTeXFormat{LaTeX2e}
\ProvidesFile{pdfcrypt.drv}%
  [2016/05/16 v1.1 Allows the setting of PDF encryption (HO)]%
\documentclass{ltxdoc}
\usepackage{holtxdoc}[2011/11/22]
\begin{document}
  \DocInput{pdfcrypt.dtx}%
\end{document}
%</driver>
% \fi
%
%
% \CharacterTable
%  {Upper-case    \A\B\C\D\E\F\G\H\I\J\K\L\M\N\O\P\Q\R\S\T\U\V\W\X\Y\Z
%   Lower-case    \a\b\c\d\e\f\g\h\i\j\k\l\m\n\o\p\q\r\s\t\u\v\w\x\y\z
%   Digits        \0\1\2\3\4\5\6\7\8\9
%   Exclamation   \!     Double quote  \"     Hash (number) \#
%   Dollar        \$     Percent       \%     Ampersand     \&
%   Acute accent  \'     Left paren    \(     Right paren   \)
%   Asterisk      \*     Plus          \+     Comma         \,
%   Minus         \-     Point         \.     Solidus       \/
%   Colon         \:     Semicolon     \;     Less than     \<
%   Equals        \=     Greater than  \>     Question mark \?
%   Commercial at \@     Left bracket  \[     Backslash     \\
%   Right bracket \]     Circumflex    \^     Underscore    \_
%   Grave accent  \`     Left brace    \{     Vertical bar  \|
%   Right brace   \}     Tilde         \~}
%
% \GetFileInfo{pdfcrypt.drv}
%
% \title{The \xpackage{pdfcrypt} package}
% \date{2016/05/16 v1.1}
% \author{Heiko Oberdiek\thanks
% {Please report any issues at \url{https://github.com/ho-tex/oberdiek/issues}}}
%
% \maketitle
%
% \begin{abstract}
% This package supports the setting of pdf encryption options
% for \VTeX\ and some older versions of \pdfTeX.
% \end{abstract}
%
% \tableofcontents
%
% \section{Documentation}
%
% This package allows the setting of pdf encryption
% options for
% \begin{itemize}
% \item
%   \VTeX, since version 7.35, \url{http://www.micropress-inc.com/}.
% \item
%   \pdfTeX, patched by
%   \NameEmail{Ricardo Sanchez Carmenes}{carmenes@bioquimica.uniovi.es}.\\
%   Some supported versions are:
%   \begin{quote}
%     \pdfTeX-1.00a,\\
%     \pdfTeX-0.14h-pdfcrypt-20010310,\\
%     pdfcrypt-20010331
%   \end{quote}
% \end{itemize}
%
% \noindent
% \textcolor{red}{Note:} Since \pdfTeX-1.10a (2003-01-16) encryption
% support was dropped! Thus the package is now obsolete for recent
% \pdfTeX\ versions.
%
% \subsection{Alternatives}
%
% There are free alternatives, programs that can
% be used for postprocessing the pdf file:
% \begin{itemize}
% \item \textsf{pdftk}\\
%   \url{http://www.accesspdf.com/pdftk/}
% \item \textsf{Multivalent}\\
%   \url{http://multivalent.sourceforge.net/}
% \item \textsf{PDFBox}\\
%   \url{http://www.pdfbox.org/}
% \item \textsf{PDFTrans}\\
%   \url{http://maddingue.free.fr/softwares/pdftrans.html}
% \item \dots
% \end{itemize}
%
% \subsection{Usage}
%
% The options can be set as package options or
% with the command \cs{pdfcryptsetup}:
% \begin{quote}
%   |owner=|\meta{owner password}\\
%   |user=|\meta{user password}\\
%   |print|, |copy|, |edit|, |annotate=true|, |false|\\
%   |all|, |none|\\
%   |debug|
% \end{quote}
% The encryption is set at |\begin{document}| by default,
% but this can be forced for an earlier time
% using the option |set|.
% Example:
% \begin{quote}
%   |\usepackage[owner=Donald,user=Knuth,print=false]{pdfcrypt}|\\
%   \dots\\
%   |\begin{document}|
% \end{quote}
% or
% \begin{quote}
%   |\usepackage{pdfcrypt}|\\
%   |\pdfcryptsetup{owner=Donald,user=Knut}|\\
%   |\pdfcryptsetup{print=false,set}|\\
%   \dots\\
%   |\begin{document}|
% \end{quote}
%
% \subsubsection{Special characters}
%
% The characters `|{|', `|}|', and `|\|' have to be
% escaped with `|\|'. Then the string should not
% be specified as package option but with
% the macro \cs{pdfcryptsetup}, eg:
% \begin{quote}
%   |\pdfcryptsetup{user={\{\\Hello\}},print=false}|
% \end{quote}
% The password here is ``|{\Hello}|''.
% Active characters can be used and are not expanded.
% Macros except for |\{|, |\}|, and |\\| should not be used
% and are not expanded.
%
% \subsubsection{\cs{nopdfcrypt}}
%
% Often the whole process of pdf generation includes
% several \TeX\ compilations and perhaps other
% postprocessing steps. Therefore a feature would
% be useful to disable the encryption stuff in order
% to speed up the first compilations.
% Therefore package \xpackage{pdfcrypt} will look for
% the macro \cs{nopdfcrypt}. If it has the meaning of the
% letter `|Y|', the package will be disabled.
% Example:
% \begin{quote}
%   |pdflatex '\let\nopdfcrypt=Y\input myfile'|\\
%   |thumbpdf myfile|\\
%   |pdfelatex myfile|
% \end{quote}
%
% \subsubsection{Configuration file}
%
% If the file \xfile{pdfcrypt.cfg} exists it is loaded
% and can be used for common settings.
%
% \subsubsection{Support for \plainTeX}
%
% The package can also be used with \plainTeX. It defines the
% missing \LaTeX\ macros and loads package \xpackage{keyval}.
% The configuration file is not supported and automatically
% loaded.
%
% \subsection{Hints for \pdfTeX}
%
% This section is out of date, because the encryption support
% was removed from \pdfTeX. Therefore this hints are of
% historical interest only.
%
%\begin{verbatim}
%* There are packages such as thumbpdf that ship out
%  some stuff (\immediate\pdfobj{...}). In these cases
%  \pdfcrypt will be ignored without warning or error
%  message. Therefore the package pdfcrypt should be
%  loaded before and option "set" should be used in
%  order to force the call of \pdfcrypt, for example:
%     \usepackage{pdfcrypt}
%     \pdfcryptsetup{..., set}
%     ...
%     \usepackage{thumbpdf}
%
%* Compiling pdfTeX (eg. version 1.00a-pretest-20010804):
%  pdfTeX versions are available at
%      ftp://ftp.cstug.cz/pub/tex/local/cstug/thanh/pdftex/snapshots/
%  Unpack the latest .tgz file:
%      > tar xzf pdftex-20010804.tgz
%  Enable encryption support:
%      src> cd src/texk/web2c/pdftexdir
%      src/texk/web2c/pdftexdir> ln -fs pdfcrypt-full.c pdfcrypt.c
%  Compile:
%      src/texk/web2c/pdftexdir> cd ../../..
%      src> ./configure
%      src> cd texk/web2c
%      src/texk/web2c> make pdftexbin
%  At last install the binary and pool files.
%\end{verbatim}
%
% \StopEventually{
% }
%
% \section{Implementation}
%
%    \begin{macrocode}
%<*package>
%    \end{macrocode}
%
% \subsection{Help macros for \plainTeX}
%
%    \begin{macrocode}
\expandafter\edef\csname pc@endinput\endcsname{%
  \catcode`\noexpand\@=\the\catcode`\@\relax
  \noexpand\endinput
}
\catcode`\@=11 %

\expandafter\ifx\csname @firstoftwo\endcsname\relax
  \long\def\@firstoftwo#1#2{#1}%
\fi

\expandafter\ifx\csname @secondoftwo\endcsname\relax
  \long\def\@secondoftwo#1#2{#2}%
\fi

\expandafter\ifx\csname @ifundefined\endcsname\relax
  \def\@ifundefined#1{%
    \expandafter\ifx\csname #1\endcsname\relax
      \expandafter\@firstoftwo
    \else
      \expandafter\@secondoftwo
    \fi
  }%
\fi

\@ifundefined{@ifnextchar}{%
  \long\def\@ifnextchar#1#2#3{%
    \let\reserved@d=#1%
    \def\reserved@a{#2}%
    \def\reserved@b{#3}%
    \futurelet\@let@token\@ifnch
  }%
  \def\@ifnch{%
    \ifx\@let@token\@sptoken
      \let\reserved@c\@xifnch
    \else
      \ifx\@let@token\reserved@d
        \let\reserved@c\reserved@a
      \else
        \let\reserved@c\reserved@b
      \fi
    \fi
    \reserved@c
  }%
  \begingroup
    \def\:{\global\let\@sptoken= }%
    \: %
    \def\:{\@xifnch}%
    \expandafter\gdef\: {\futurelet\@let@token\@ifnch}%
  \endgroup
}{}

\@ifundefined{ProvidesPackage}{%
  \def\ProvidesPackage#1{%
    \@ifnextchar[{\pc@ProvidesPackage{#1}}%
                 {\pc@ProvidesPackage{#1}[]}%
  }%
  \def\pc@ProvidesPackage#1[#2]{%
    \immediate\write-1{Package: #1 #2}%
  }%
}{}

\begingroup\expandafter\expandafter\expandafter\endgroup
\expandafter\ifx\csname RequirePackage\endcsname\relax
  \input infwarerr.sty\relax
\else
  \RequirePackage{infwarerr}%
\fi

\@ifundefined{@gobble}{%
  \long\def\@gobble#1{}%
}{}

\@ifundefined{@empty}{%
  \def\@empty{}%
}{}
%    \end{macrocode}
%
% \subsection{Package Identification and checks}
%
%    \begin{macrocode}
\ProvidesPackage{pdfcrypt}%
  [2016/05/16 v1.1 Allows the setting of PDF encryption (HO)]%
%    \end{macrocode}
%
%    \begin{macrocode}
\@ifundefined{pdfcryptsetup}{%
  \let\pdfcryptsetup\@gobble
}{%
  \@PackageWarningNoLine{pdfcrypt}{Package pdfcrypt is already loaded}%
  \pc@endinput
}
%    \end{macrocode}
%    Support for \cs{nopdfcrypt}.
%    \begin{macrocode}
\newif\ifpc@nopdfcrypt
\ifx Y\nopdfcrypt
  \@PackageWarningNoLine{pdfcrypt}{%
    Encryption disabled by \string \nopdfcrypt\space request%
  }%
  \global\pc@nopdfcrypttrue
\fi
%    \end{macrocode}
%
% \subsection{Driver detection}
%
%    \begin{macrocode}
\let\pc@driver\@empty
\begingroup
  % pdfTeX detection
  \@ifundefined{pdfoutput}{%
  }{%
    \ifcase\pdfoutput
    \else
      \@ifundefined{pdfcrypt}{%
        \@PackageError{pdfcrypt}{%
          PDF encryption is not supported with this pdfTeX%
        }{%
          Encryption support was added in 0.14h (2001/03/10)\MessageBreak
          and removed in 1.10a (2003/01/16).%
        }%
        \endgroup
        \pc@endinput
      }{%
        \gdef\pc@driver{pdftex}%
      }%
    \fi
  }%
  % VTeX detection
  \@ifundefined{OpMode}{%
  }{%
    \ifnum\OpMode=1 %
      \ifnum\@ifundefined{VTeXversion}0\VTeXversion<735 %
        \@PackageError{pdfcrypt}{%
          PDF encryption is not supported with this VTeX%
        }{%
          You need VTeX 7.35 or higher.%
        }%
        \endgroup
        \pc@endinput
      \else
        \gdef\pc@driver{vtex}%
      \fi
    \fi
  }%
\endgroup
%    \end{macrocode}
%
% \subsection{Load package \xpackage{keyval}}
%
%    \begin{macrocode}
\@ifundefined{@makeother}{%
  \def\@makeother#1{\catcode`#1=12\relax}%
}{}

\@ifundefined{g@addto@macro}{%
  \long\def\g@addto@macro#1#2{%
    \begingroup
      \toks@\expandafter{#1#2}%
      \xdef#1{\the\toks@}%
    \endgroup
  }%
}{}

\@ifundefined{@namedef}{%
  \def\@namedef#1{\expandafter\def\csname#1\endcsname}%
}{}

\@ifundefined{@nameuse}{%
  \def\@nameuse#1{\csname #1\endcsname}%
}{}

\def\pc@KeyvalRestore{%
  \let\pc@KeyvalRestore\@undefined
}

\let\pcOrg@NeedsTeXFormat\NeedsTeXFormat
\@ifundefined{NeedsTeXFormat}{%
  \def\NeedsTeXFormat#1{}%
  \g@addto@macro\pc@KeyvalRestore{%
    \let\NeedsTeXFormat\pcOrg@NeedsTeXFormat
  }%
}{}

\let\pcOrg@DeclareOption\DeclareOption
\@ifundefined{DeclareOption}{%
  \def\DeclareOption#1#2{#2}%
  \g@addto@macro\pc@KeyvalRestore{%
    \let\DeclareOption\pcOrg@DeclareOption
  }%
}{}

\let\pcOrg@ExecuteOptions\ExecuteOptions
\@ifundefined{ExecuteOptions}{%
  \def\ExecuteOptions#1{}%
  \g@addto@macro\pc@KeyvalRestore{%
    \let\ExecuteOptions\pcOrg@ExecuteOptions
  }%
}{}

\let\pcOrg@ProcessOptions\ProcessOptions
\@ifundefined{ProcessOptions}{%
  \def\ProcessOptions{}%
  \g@addto@macro\pc@KeyvalRestore{%
    \let\ProcessOptions\pcOrg@ProcessOptions
  }%
}{}%

\begingroup\expandafter\expandafter\expandafter\endgroup
\expandafter\ifx\csname RequirePackage\endcsname\relax
  \input keyval.sty\relax
\else
  \RequirePackage{keyval}%
\fi
\pc@KeyvalRestore
%    \end{macrocode}
%
% \subsection{Define options}
%
%    \begin{macrocode}
\@ifundefined{@dblarg}{%
  \long\def\@dblarg#1{\@ifnextchar[{#1}{\@xdblarg{#1}}}%
  \long\def\@xdblarg#1#2{#1[{#2}]{#2}}%
}{}

\newif\ifpc@set
\newif\ifpc@print
\newif\ifpc@copy
\newif\ifpc@edit
\newif\ifpc@annotate
\newif\ifpc@debug
\let\pc@owner\@empty
\let\pc@user\@empty

% default: allow all
\pc@printtrue
\pc@copytrue
\pc@edittrue
\pc@annotatetrue

\edef\pc@temp{\catcode`\noexpand\"=\the\catcode`\"\relax}
\@makeother\"
\def\pc@set{%
  \@PackageInfo{pdfcrypt}{%
    \ifpc@debug
      \ifx\pc@owner\@empty
        No owner password%
      \else
        Owner password: `\pc@owner'%
      \fi
      \MessageBreak
      \ifx\pc@user\@empty
        No user password%
      \else
        User password: `\pc@user'%
      \fi
      \MessageBreak
      Flags: %
      \ifpc@print   \else no\fi print, %
      \ifpc@copy    \else no\fi copy, %
      \ifpc@edit    \else no\fi edit, %
      \ifpc@annotate\else no\fi annotate%
      \MessageBreak
    \fi
    \ifpc@nopdfcrypt
      Encryption is disabled by `\string\nopdfcrypt'%
    \else
      Encryption is set for `\pc@driver'%
    \fi
  }%
  \ifpc@nopdfcrypt
  \else
    \@ifundefined{pc@set@\pc@driver}{%
      \ifx\pc@driver\@empty
        \@PackageError{pdfcrypt}{No driver for encryption %
          support found}\@ehc
      \else
        \@PackageError{pdfcrypt}{Cannot set encryption for %
          unknown driver `\pc@driver'}\@ehc
      \fi
    }{%
      \@nameuse{pc@set@\pc@driver}%
    }%
  \fi
}
\def\pc@set@pdftex{%
  \ifnum\pdftexversion<100 %
    \pc@set@pdftexold
  \else
    \pc@set@pdftexnew
  \fi
}
\def\pc@set@pdftexold{%
  \pdfcrypt{%
    owner "\pc@owner" %
    user "\pc@user" %
    \ifpc@print   \else no\fi print %
    \ifpc@copy    \else no\fi copy %
    \ifpc@edit    \else no\fi edit %
    \ifpc@annotate\else no\fi annotate%
  }%
}
\def\pc@set@pdftexnew{%
  \pdfcrypt
    owner{\pc@owner}%
    user{\pc@user}%
    \ifpc@print   \else no\fi print %
    \ifpc@copy    \else no\fi copy %
    \ifpc@edit    \else no\fi edit %
    \ifpc@annotate\else no\fi annotate%
  \relax
}
\def\pc@set@vtex{%
  \immediate\special{!security %
    O=\pc@MakeVTeXString\pc@owner,%
    U=\pc@MakeVTeXString\pc@user,%
    P\ifpc@print +\else -\fi,%
    C\ifpc@copy +\else -\fi,%
    M\ifpc@edit +\else -\fi,%
    A\ifpc@annotate +\else -\fi
  }%
}
\def\pc@MakeVTeXString#1{%
  "\expandafter\pc@@MakeVTeXString#1"\@nil"%
}
\def\pc@@MakeVTeXString#1"#2\@nil{%
  #1%
  \ifx\\#2\\%
  \else
    ""%
    \@ReturnAfterFi{%
      \pc@@MakeVTeXString#2\@nil
    }%
  \fi
}
\long\def\@ReturnAfterFi#1\fi{\fi#1}
\pc@temp

\begingroup
  \catcode`\ =12 \gdef\pc@spaceother{ }\catcode`\ =10\relax
  \catcode`\|=0 %
  \catcode`\\=12 %
  |gdef|pc@DefString#1#2{%
    |def#1{#2}%
    |edef#1{|expandafter|strip@prefix|meaning#1}%
    |edef#1{|expandafter|pc@SpaceToOther#1 |@nil}%
    |edef#1{|expandafter|pc@EscapeRemove#1|@empty\|@empty|@nil}%
  }%
  |gdef|pc@EscapeRemove#1\#2#3|@nil{%
    #1#2%
    |ifx|\#3|\%%
    |else
      |@ReturnAfterFi{%
        |pc@EscapeRemove#3|@nil
      }%
    |fi
  }%
|endgroup
\def\pc@SpaceToOther#1 #2\@nil{%
  #1%
  \ifx\\#2\\%
  \else
    \pc@spaceother
    \@ReturnAfterFi{%
      \pc@SpaceToOther#2\@nil
    }%
  \fi
}

\def\pc@boolkey{\@dblarg\pc@@boolkey}
\def\pc@@boolkey[#1]#2#3{%
  \lowercase{\def\pc@temp{#3}}%
  \ifx\pc@temp\@empty
    \let\pc@temp\pc@true
  \fi
  \ifx\pc@temp\pc@true
  \else
    \ifx\pc@temp\pc@false
    \else
      \let\pc@temp\relax
    \fi
  \fi
  \ifx\pc@temp\relax
    \@PackageWarning{pdfcrypt}{%
      Unexpected value \string`#3\string' of %
      option \string`#2\string'\MessageBreak
      instead of %
      \string`true\string' or \string`false\string'%
    }%
  \else
    \csname pc@#2\pc@temp\endcsname
  \fi
}
\def\pc@true{true}
\def\pc@false{false}

\define@key{pc}{set}[true]{%
  \pc@boolkey{set}{#1}%
}
\define@key{pc}{pdftex}[]{%
  \def\pc@driver{pdftex}%
}
\define@key{pc}{vtex}[]{%
  \def\pc@driver{vtex}%
}
\define@key{pc}{print}[true]{%
  \pc@boolkey{print}{#1}%
}
\define@key{pc}{copy}[true]{%
  \pc@boolkey{copy}{#1}%
}
\define@key{pc}{edit}[true]{%
  \pc@boolkey{edit}{#1}%
}
\define@key{pc}{annotate}[true]{%
  \pc@boolkey{annotate}{#1}%
}
\define@key{pc}{all}[]{%
  \pc@boolkey{print}{true}%
  \pc@boolkey{copy}{true}%
  \pc@boolkey{edit}{true}%
  \pc@boolkey{annotate}{true}%
}
\define@key{pc}{none}[]{%
  \pc@boolkey{print}{false}%
  \pc@boolkey{copy}{false}%
  \pc@boolkey{edit}{false}%
  \pc@boolkey{annotate}{false}%
}

\define@key{pc}{owner}{%
  \pc@DefString\pc@owner{#1}%
}
\define@key{pc}{user}{%
  \pc@DefString\pc@user{#1}%
}
\define@key{pc}{debug}[true]{%
  \pc@boolkey{debug}{#1}%
}

\def\pdfcryptsetup#1{%
  \setkeys{pc}{#1}%
  \ifpc@set
    \pc@set
    \global\let\pc@set\relax
    \gdef\pdfcryptsetup##1{%
      \@PackageWarning{pdfcrypt}{%
        Encryption options are already set\MessageBreak
        new values are ignored%
      }%
    }%
  \fi
}
\begingroup\expandafter\expandafter\expandafter\endgroup
\expandafter\ifx\csname @onlypreamble\endcsname\relax
\else
  \@onlypreamble\pdfcryptsetup
\fi
%    \end{macrocode}
%
% \subsection{support of configuration file}
%    \begin{macrocode}
\begingroup\expandafter\expandafter\expandafter\endgroup
\expandafter\ifx\csname InputIfFileExists\endcsname\relax
  \@PackageInfo{pdfcrypt}{%
    Configuration file pdfcrypt.cfg not supported.%
  }%
\else
  \let\pc@ExecuteOptions\ExecuteOptions
  \InputIfFileExists{pdfcrypt.cfg}{}{}%
  \let\ExecuteOptions\pc@ExecuteOptions
\fi
%    \end{macrocode}
%
% \subsection{Package options}
%
%    Plain format does not know package options.
%    \begin{macrocode}
\begingroup\expandafter\expandafter\expandafter\endgroup
\expandafter\ifx\csname @classoptionslist\endcsname\relax
  \expandafter\pc@endinput
\fi
%    \end{macrocode}
%    Process global and local options.
%    \begin{macrocode}
\def\pc@ProcessOptionsWithKV{%
  \let\pc@temp\@empty
  \@for\CurrentOption:=\@classoptionslist\do{%
    \@ifundefined{KV@pc@\CurrentOption}{}{%
      \edef\pc@temp{\pc@temp,\CurrentOption,}%
      \@expandtwoargs\@removeelement\CurrentOption
        \@unusedoptionlist\@unusedoptionlist
    }%
  }%
  \edef\pc@temp{%
    \noexpand\pdfcryptsetup{%
      \pc@temp\@ptionlist{\@currname.\@currext}%
    }%
  }%
  \pc@temp
}
\pc@ProcessOptionsWithKV
\AtEndOfPackage{\let\@unprocessedoptions\relax}
\AtBeginDocument{\pc@set}

\pc@endinput
%    \end{macrocode}
%
%    \begin{macrocode}
%</package>
%    \end{macrocode}
%
% \section{Installation}
%
% \subsection{Download}
%
% \paragraph{Package.} This package is available on
% CTAN\footnote{\CTANpkg{pdfcrypt}}:
% \begin{description}
% \item[\CTAN{macros/latex/contrib/oberdiek/pdfcrypt.dtx}] The source file.
% \item[\CTAN{macros/latex/contrib/oberdiek/pdfcrypt.pdf}] Documentation.
% \end{description}
%
%
% \paragraph{Bundle.} All the packages of the bundle `oberdiek'
% are also available in a TDS compliant ZIP archive. There
% the packages are already unpacked and the documentation files
% are generated. The files and directories obey the TDS standard.
% \begin{description}
% \item[\CTANinstall{install/macros/latex/contrib/oberdiek.tds.zip}]
% \end{description}
% \emph{TDS} refers to the standard ``A Directory Structure
% for \TeX\ Files'' (\CTANpkg{tds}). Directories
% with \xfile{texmf} in their name are usually organized this way.
%
% \subsection{Bundle installation}
%
% \paragraph{Unpacking.} Unpack the \xfile{oberdiek.tds.zip} in the
% TDS tree (also known as \xfile{texmf} tree) of your choice.
% Example (linux):
% \begin{quote}
%   |unzip oberdiek.tds.zip -d ~/texmf|
% \end{quote}
%
% \subsection{Package installation}
%
% \paragraph{Unpacking.} The \xfile{.dtx} file is a self-extracting
% \docstrip\ archive. The files are extracted by running the
% \xfile{.dtx} through \plainTeX:
% \begin{quote}
%   \verb|tex pdfcrypt.dtx|
% \end{quote}
%
% \paragraph{TDS.} Now the different files must be moved into
% the different directories in your installation TDS tree
% (also known as \xfile{texmf} tree):
% \begin{quote}
% \def\t{^^A
% \begin{tabular}{@{}>{\ttfamily}l@{ $\rightarrow$ }>{\ttfamily}l@{}}
%   pdfcrypt.sty & tex/generic/oberdiek/pdfcrypt.sty\\
%   pdfcrypt.pdf & doc/latex/oberdiek/pdfcrypt.pdf\\
%   pdfcrypt.dtx & source/latex/oberdiek/pdfcrypt.dtx\\
% \end{tabular}^^A
% }^^A
% \sbox0{\t}^^A
% \ifdim\wd0>\linewidth
%   \begingroup
%     \advance\linewidth by\leftmargin
%     \advance\linewidth by\rightmargin
%   \edef\x{\endgroup
%     \def\noexpand\lw{\the\linewidth}^^A
%   }\x
%   \def\lwbox{^^A
%     \leavevmode
%     \hbox to \linewidth{^^A
%       \kern-\leftmargin\relax
%       \hss
%       \usebox0
%       \hss
%       \kern-\rightmargin\relax
%     }^^A
%   }^^A
%   \ifdim\wd0>\lw
%     \sbox0{\small\t}^^A
%     \ifdim\wd0>\linewidth
%       \ifdim\wd0>\lw
%         \sbox0{\footnotesize\t}^^A
%         \ifdim\wd0>\linewidth
%           \ifdim\wd0>\lw
%             \sbox0{\scriptsize\t}^^A
%             \ifdim\wd0>\linewidth
%               \ifdim\wd0>\lw
%                 \sbox0{\tiny\t}^^A
%                 \ifdim\wd0>\linewidth
%                   \lwbox
%                 \else
%                   \usebox0
%                 \fi
%               \else
%                 \lwbox
%               \fi
%             \else
%               \usebox0
%             \fi
%           \else
%             \lwbox
%           \fi
%         \else
%           \usebox0
%         \fi
%       \else
%         \lwbox
%       \fi
%     \else
%       \usebox0
%     \fi
%   \else
%     \lwbox
%   \fi
% \else
%   \usebox0
% \fi
% \end{quote}
% If you have a \xfile{docstrip.cfg} that configures and enables \docstrip's
% TDS installing feature, then some files can already be in the right
% place, see the documentation of \docstrip.
%
% \subsection{Refresh file name databases}
%
% If your \TeX~distribution
% (\TeX\,Live, \mikTeX, \dots) relies on file name databases, you must refresh
% these. For example, \TeX\,Live\ users run \verb|texhash| or
% \verb|mktexlsr|.
%
% \subsection{Some details for the interested}
%
% \paragraph{Unpacking with \LaTeX.}
% The \xfile{.dtx} chooses its action depending on the format:
% \begin{description}
% \item[\plainTeX:] Run \docstrip\ and extract the files.
% \item[\LaTeX:] Generate the documentation.
% \end{description}
% If you insist on using \LaTeX\ for \docstrip\ (really,
% \docstrip\ does not need \LaTeX), then inform the autodetect routine
% about your intention:
% \begin{quote}
%   \verb|latex \let\install=y% \iffalse meta-comment
%
% File: pdfcrypt.dtx
% Version: 2016/05/16 v1.1
% Info: Allows the setting of PDF encryption
%
% Copyright (C)
%    2001, 2006, 2007 Heiko Oberdiek
%    2016-2019 Oberdiek Package Support Group
%    https://github.com/ho-tex/oberdiek/issues
%
% This work may be distributed and/or modified under the
% conditions of the LaTeX Project Public License, either
% version 1.3c of this license or (at your option) any later
% version. This version of this license is in
%    https://www.latex-project.org/lppl/lppl-1-3c.txt
% and the latest version of this license is in
%    https://www.latex-project.org/lppl.txt
% and version 1.3 or later is part of all distributions of
% LaTeX version 2005/12/01 or later.
%
% This work has the LPPL maintenance status "maintained".
%
% The Current Maintainers of this work are
% Heiko Oberdiek and the Oberdiek Package Support Group
% https://github.com/ho-tex/oberdiek/issues
%
% The Base Interpreter refers to any `TeX-Format',
% because some files are installed in TDS:tex/generic//.
%
% This work consists of the main source file pdfcrypt.dtx
% and the derived files
%    pdfcrypt.sty, pdfcrypt.pdf, pdfcrypt.ins, pdfcrypt.drv.
%
% Distribution:
%    CTAN:macros/latex/contrib/oberdiek/pdfcrypt.dtx
%    CTAN:macros/latex/contrib/oberdiek/pdfcrypt.pdf
%
% Unpacking:
%    (a) If pdfcrypt.ins is present:
%           tex pdfcrypt.ins
%    (b) Without pdfcrypt.ins:
%           tex pdfcrypt.dtx
%    (c) If you insist on using LaTeX
%           latex \let\install=y\input{pdfcrypt.dtx}
%        (quote the arguments according to the demands of your shell)
%
% Documentation:
%    (a) If pdfcrypt.drv is present:
%           latex pdfcrypt.drv
%    (b) Without pdfcrypt.drv:
%           latex pdfcrypt.dtx; ...
%    The class ltxdoc loads the configuration file ltxdoc.cfg
%    if available. Here you can specify further options, e.g.
%    use A4 as paper format:
%       \PassOptionsToClass{a4paper}{article}
%
%    Programm calls to get the documentation (example):
%       pdflatex pdfcrypt.dtx
%       makeindex -s gind.ist pdfcrypt.idx
%       pdflatex pdfcrypt.dtx
%       makeindex -s gind.ist pdfcrypt.idx
%       pdflatex pdfcrypt.dtx
%
% Installation:
%    TDS:tex/generic/oberdiek/pdfcrypt.sty
%    TDS:doc/latex/oberdiek/pdfcrypt.pdf
%    TDS:source/latex/oberdiek/pdfcrypt.dtx
%
%<*ignore>
\begingroup
  \catcode123=1 %
  \catcode125=2 %
  \def\x{LaTeX2e}%
\expandafter\endgroup
\ifcase 0\ifx\install y1\fi\expandafter
         \ifx\csname processbatchFile\endcsname\relax\else1\fi
         \ifx\fmtname\x\else 1\fi\relax
\else\csname fi\endcsname
%</ignore>
%<*install>
\input docstrip.tex
\Msg{************************************************************************}
\Msg{* Installation}
\Msg{* Package: pdfcrypt 2016/05/16 v1.1 Allows the setting of PDF encryption (HO)}
\Msg{************************************************************************}

\keepsilent
\askforoverwritefalse

\let\MetaPrefix\relax
\preamble

This is a generated file.

Project: pdfcrypt
Version: 2016/05/16 v1.1

Copyright (C)
   2001, 2006, 2007 Heiko Oberdiek
   2016-2019 Oberdiek Package Support Group

This work may be distributed and/or modified under the
conditions of the LaTeX Project Public License, either
version 1.3c of this license or (at your option) any later
version. This version of this license is in
   https://www.latex-project.org/lppl/lppl-1-3c.txt
and the latest version of this license is in
   https://www.latex-project.org/lppl.txt
and version 1.3 or later is part of all distributions of
LaTeX version 2005/12/01 or later.

This work has the LPPL maintenance status "maintained".

The Current Maintainers of this work are
Heiko Oberdiek and the Oberdiek Package Support Group
https://github.com/ho-tex/oberdiek/issues


The Base Interpreter refers to any `TeX-Format',
because some files are installed in TDS:tex/generic//.

This work consists of the main source file pdfcrypt.dtx
and the derived files
   pdfcrypt.sty, pdfcrypt.pdf, pdfcrypt.ins, pdfcrypt.drv.

\endpreamble
\let\MetaPrefix\DoubleperCent

\generate{%
  \file{pdfcrypt.ins}{\from{pdfcrypt.dtx}{install}}%
  \file{pdfcrypt.drv}{\from{pdfcrypt.dtx}{driver}}%
  \usedir{tex/generic/oberdiek}%
  \file{pdfcrypt.sty}{\from{pdfcrypt.dtx}{package}}%
  \nopreamble
  \nopostamble
%  \usedir{source/latex/oberdiek/catalogue}%
%  \file{pdfcrypt.xml}{\from{pdfcrypt.dtx}{catalogue}}%
}

\catcode32=13\relax% active space
\let =\space%
\Msg{************************************************************************}
\Msg{*}
\Msg{* To finish the installation you have to move the following}
\Msg{* file into a directory searched by TeX:}
\Msg{*}
\Msg{*     pdfcrypt.sty}
\Msg{*}
\Msg{* To produce the documentation run the file `pdfcrypt.drv'}
\Msg{* through LaTeX.}
\Msg{*}
\Msg{* Happy TeXing!}
\Msg{*}
\Msg{************************************************************************}

\endbatchfile
%</install>
%<*ignore>
\fi
%</ignore>
%<*driver>
\NeedsTeXFormat{LaTeX2e}
\ProvidesFile{pdfcrypt.drv}%
  [2016/05/16 v1.1 Allows the setting of PDF encryption (HO)]%
\documentclass{ltxdoc}
\usepackage{holtxdoc}[2011/11/22]
\begin{document}
  \DocInput{pdfcrypt.dtx}%
\end{document}
%</driver>
% \fi
%
%
% \CharacterTable
%  {Upper-case    \A\B\C\D\E\F\G\H\I\J\K\L\M\N\O\P\Q\R\S\T\U\V\W\X\Y\Z
%   Lower-case    \a\b\c\d\e\f\g\h\i\j\k\l\m\n\o\p\q\r\s\t\u\v\w\x\y\z
%   Digits        \0\1\2\3\4\5\6\7\8\9
%   Exclamation   \!     Double quote  \"     Hash (number) \#
%   Dollar        \$     Percent       \%     Ampersand     \&
%   Acute accent  \'     Left paren    \(     Right paren   \)
%   Asterisk      \*     Plus          \+     Comma         \,
%   Minus         \-     Point         \.     Solidus       \/
%   Colon         \:     Semicolon     \;     Less than     \<
%   Equals        \=     Greater than  \>     Question mark \?
%   Commercial at \@     Left bracket  \[     Backslash     \\
%   Right bracket \]     Circumflex    \^     Underscore    \_
%   Grave accent  \`     Left brace    \{     Vertical bar  \|
%   Right brace   \}     Tilde         \~}
%
% \GetFileInfo{pdfcrypt.drv}
%
% \title{The \xpackage{pdfcrypt} package}
% \date{2016/05/16 v1.1}
% \author{Heiko Oberdiek\thanks
% {Please report any issues at \url{https://github.com/ho-tex/oberdiek/issues}}}
%
% \maketitle
%
% \begin{abstract}
% This package supports the setting of pdf encryption options
% for \VTeX\ and some older versions of \pdfTeX.
% \end{abstract}
%
% \tableofcontents
%
% \section{Documentation}
%
% This package allows the setting of pdf encryption
% options for
% \begin{itemize}
% \item
%   \VTeX, since version 7.35, \url{http://www.micropress-inc.com/}.
% \item
%   \pdfTeX, patched by
%   \NameEmail{Ricardo Sanchez Carmenes}{carmenes@bioquimica.uniovi.es}.\\
%   Some supported versions are:
%   \begin{quote}
%     \pdfTeX-1.00a,\\
%     \pdfTeX-0.14h-pdfcrypt-20010310,\\
%     pdfcrypt-20010331
%   \end{quote}
% \end{itemize}
%
% \noindent
% \textcolor{red}{Note:} Since \pdfTeX-1.10a (2003-01-16) encryption
% support was dropped! Thus the package is now obsolete for recent
% \pdfTeX\ versions.
%
% \subsection{Alternatives}
%
% There are free alternatives, programs that can
% be used for postprocessing the pdf file:
% \begin{itemize}
% \item \textsf{pdftk}\\
%   \url{http://www.accesspdf.com/pdftk/}
% \item \textsf{Multivalent}\\
%   \url{http://multivalent.sourceforge.net/}
% \item \textsf{PDFBox}\\
%   \url{http://www.pdfbox.org/}
% \item \textsf{PDFTrans}\\
%   \url{http://maddingue.free.fr/softwares/pdftrans.html}
% \item \dots
% \end{itemize}
%
% \subsection{Usage}
%
% The options can be set as package options or
% with the command \cs{pdfcryptsetup}:
% \begin{quote}
%   |owner=|\meta{owner password}\\
%   |user=|\meta{user password}\\
%   |print|, |copy|, |edit|, |annotate=true|, |false|\\
%   |all|, |none|\\
%   |debug|
% \end{quote}
% The encryption is set at |\begin{document}| by default,
% but this can be forced for an earlier time
% using the option |set|.
% Example:
% \begin{quote}
%   |\usepackage[owner=Donald,user=Knuth,print=false]{pdfcrypt}|\\
%   \dots\\
%   |\begin{document}|
% \end{quote}
% or
% \begin{quote}
%   |\usepackage{pdfcrypt}|\\
%   |\pdfcryptsetup{owner=Donald,user=Knut}|\\
%   |\pdfcryptsetup{print=false,set}|\\
%   \dots\\
%   |\begin{document}|
% \end{quote}
%
% \subsubsection{Special characters}
%
% The characters `|{|', `|}|', and `|\|' have to be
% escaped with `|\|'. Then the string should not
% be specified as package option but with
% the macro \cs{pdfcryptsetup}, eg:
% \begin{quote}
%   |\pdfcryptsetup{user={\{\\Hello\}},print=false}|
% \end{quote}
% The password here is ``|{\Hello}|''.
% Active characters can be used and are not expanded.
% Macros except for |\{|, |\}|, and |\\| should not be used
% and are not expanded.
%
% \subsubsection{\cs{nopdfcrypt}}
%
% Often the whole process of pdf generation includes
% several \TeX\ compilations and perhaps other
% postprocessing steps. Therefore a feature would
% be useful to disable the encryption stuff in order
% to speed up the first compilations.
% Therefore package \xpackage{pdfcrypt} will look for
% the macro \cs{nopdfcrypt}. If it has the meaning of the
% letter `|Y|', the package will be disabled.
% Example:
% \begin{quote}
%   |pdflatex '\let\nopdfcrypt=Y\input myfile'|\\
%   |thumbpdf myfile|\\
%   |pdfelatex myfile|
% \end{quote}
%
% \subsubsection{Configuration file}
%
% If the file \xfile{pdfcrypt.cfg} exists it is loaded
% and can be used for common settings.
%
% \subsubsection{Support for \plainTeX}
%
% The package can also be used with \plainTeX. It defines the
% missing \LaTeX\ macros and loads package \xpackage{keyval}.
% The configuration file is not supported and automatically
% loaded.
%
% \subsection{Hints for \pdfTeX}
%
% This section is out of date, because the encryption support
% was removed from \pdfTeX. Therefore this hints are of
% historical interest only.
%
%\begin{verbatim}
%* There are packages such as thumbpdf that ship out
%  some stuff (\immediate\pdfobj{...}). In these cases
%  \pdfcrypt will be ignored without warning or error
%  message. Therefore the package pdfcrypt should be
%  loaded before and option "set" should be used in
%  order to force the call of \pdfcrypt, for example:
%     \usepackage{pdfcrypt}
%     \pdfcryptsetup{..., set}
%     ...
%     \usepackage{thumbpdf}
%
%* Compiling pdfTeX (eg. version 1.00a-pretest-20010804):
%  pdfTeX versions are available at
%      ftp://ftp.cstug.cz/pub/tex/local/cstug/thanh/pdftex/snapshots/
%  Unpack the latest .tgz file:
%      > tar xzf pdftex-20010804.tgz
%  Enable encryption support:
%      src> cd src/texk/web2c/pdftexdir
%      src/texk/web2c/pdftexdir> ln -fs pdfcrypt-full.c pdfcrypt.c
%  Compile:
%      src/texk/web2c/pdftexdir> cd ../../..
%      src> ./configure
%      src> cd texk/web2c
%      src/texk/web2c> make pdftexbin
%  At last install the binary and pool files.
%\end{verbatim}
%
% \StopEventually{
% }
%
% \section{Implementation}
%
%    \begin{macrocode}
%<*package>
%    \end{macrocode}
%
% \subsection{Help macros for \plainTeX}
%
%    \begin{macrocode}
\expandafter\edef\csname pc@endinput\endcsname{%
  \catcode`\noexpand\@=\the\catcode`\@\relax
  \noexpand\endinput
}
\catcode`\@=11 %

\expandafter\ifx\csname @firstoftwo\endcsname\relax
  \long\def\@firstoftwo#1#2{#1}%
\fi

\expandafter\ifx\csname @secondoftwo\endcsname\relax
  \long\def\@secondoftwo#1#2{#2}%
\fi

\expandafter\ifx\csname @ifundefined\endcsname\relax
  \def\@ifundefined#1{%
    \expandafter\ifx\csname #1\endcsname\relax
      \expandafter\@firstoftwo
    \else
      \expandafter\@secondoftwo
    \fi
  }%
\fi

\@ifundefined{@ifnextchar}{%
  \long\def\@ifnextchar#1#2#3{%
    \let\reserved@d=#1%
    \def\reserved@a{#2}%
    \def\reserved@b{#3}%
    \futurelet\@let@token\@ifnch
  }%
  \def\@ifnch{%
    \ifx\@let@token\@sptoken
      \let\reserved@c\@xifnch
    \else
      \ifx\@let@token\reserved@d
        \let\reserved@c\reserved@a
      \else
        \let\reserved@c\reserved@b
      \fi
    \fi
    \reserved@c
  }%
  \begingroup
    \def\:{\global\let\@sptoken= }%
    \: %
    \def\:{\@xifnch}%
    \expandafter\gdef\: {\futurelet\@let@token\@ifnch}%
  \endgroup
}{}

\@ifundefined{ProvidesPackage}{%
  \def\ProvidesPackage#1{%
    \@ifnextchar[{\pc@ProvidesPackage{#1}}%
                 {\pc@ProvidesPackage{#1}[]}%
  }%
  \def\pc@ProvidesPackage#1[#2]{%
    \immediate\write-1{Package: #1 #2}%
  }%
}{}

\begingroup\expandafter\expandafter\expandafter\endgroup
\expandafter\ifx\csname RequirePackage\endcsname\relax
  \input infwarerr.sty\relax
\else
  \RequirePackage{infwarerr}%
\fi

\@ifundefined{@gobble}{%
  \long\def\@gobble#1{}%
}{}

\@ifundefined{@empty}{%
  \def\@empty{}%
}{}
%    \end{macrocode}
%
% \subsection{Package Identification and checks}
%
%    \begin{macrocode}
\ProvidesPackage{pdfcrypt}%
  [2016/05/16 v1.1 Allows the setting of PDF encryption (HO)]%
%    \end{macrocode}
%
%    \begin{macrocode}
\@ifundefined{pdfcryptsetup}{%
  \let\pdfcryptsetup\@gobble
}{%
  \@PackageWarningNoLine{pdfcrypt}{Package pdfcrypt is already loaded}%
  \pc@endinput
}
%    \end{macrocode}
%    Support for \cs{nopdfcrypt}.
%    \begin{macrocode}
\newif\ifpc@nopdfcrypt
\ifx Y\nopdfcrypt
  \@PackageWarningNoLine{pdfcrypt}{%
    Encryption disabled by \string \nopdfcrypt\space request%
  }%
  \global\pc@nopdfcrypttrue
\fi
%    \end{macrocode}
%
% \subsection{Driver detection}
%
%    \begin{macrocode}
\let\pc@driver\@empty
\begingroup
  % pdfTeX detection
  \@ifundefined{pdfoutput}{%
  }{%
    \ifcase\pdfoutput
    \else
      \@ifundefined{pdfcrypt}{%
        \@PackageError{pdfcrypt}{%
          PDF encryption is not supported with this pdfTeX%
        }{%
          Encryption support was added in 0.14h (2001/03/10)\MessageBreak
          and removed in 1.10a (2003/01/16).%
        }%
        \endgroup
        \pc@endinput
      }{%
        \gdef\pc@driver{pdftex}%
      }%
    \fi
  }%
  % VTeX detection
  \@ifundefined{OpMode}{%
  }{%
    \ifnum\OpMode=1 %
      \ifnum\@ifundefined{VTeXversion}0\VTeXversion<735 %
        \@PackageError{pdfcrypt}{%
          PDF encryption is not supported with this VTeX%
        }{%
          You need VTeX 7.35 or higher.%
        }%
        \endgroup
        \pc@endinput
      \else
        \gdef\pc@driver{vtex}%
      \fi
    \fi
  }%
\endgroup
%    \end{macrocode}
%
% \subsection{Load package \xpackage{keyval}}
%
%    \begin{macrocode}
\@ifundefined{@makeother}{%
  \def\@makeother#1{\catcode`#1=12\relax}%
}{}

\@ifundefined{g@addto@macro}{%
  \long\def\g@addto@macro#1#2{%
    \begingroup
      \toks@\expandafter{#1#2}%
      \xdef#1{\the\toks@}%
    \endgroup
  }%
}{}

\@ifundefined{@namedef}{%
  \def\@namedef#1{\expandafter\def\csname#1\endcsname}%
}{}

\@ifundefined{@nameuse}{%
  \def\@nameuse#1{\csname #1\endcsname}%
}{}

\def\pc@KeyvalRestore{%
  \let\pc@KeyvalRestore\@undefined
}

\let\pcOrg@NeedsTeXFormat\NeedsTeXFormat
\@ifundefined{NeedsTeXFormat}{%
  \def\NeedsTeXFormat#1{}%
  \g@addto@macro\pc@KeyvalRestore{%
    \let\NeedsTeXFormat\pcOrg@NeedsTeXFormat
  }%
}{}

\let\pcOrg@DeclareOption\DeclareOption
\@ifundefined{DeclareOption}{%
  \def\DeclareOption#1#2{#2}%
  \g@addto@macro\pc@KeyvalRestore{%
    \let\DeclareOption\pcOrg@DeclareOption
  }%
}{}

\let\pcOrg@ExecuteOptions\ExecuteOptions
\@ifundefined{ExecuteOptions}{%
  \def\ExecuteOptions#1{}%
  \g@addto@macro\pc@KeyvalRestore{%
    \let\ExecuteOptions\pcOrg@ExecuteOptions
  }%
}{}

\let\pcOrg@ProcessOptions\ProcessOptions
\@ifundefined{ProcessOptions}{%
  \def\ProcessOptions{}%
  \g@addto@macro\pc@KeyvalRestore{%
    \let\ProcessOptions\pcOrg@ProcessOptions
  }%
}{}%

\begingroup\expandafter\expandafter\expandafter\endgroup
\expandafter\ifx\csname RequirePackage\endcsname\relax
  \input keyval.sty\relax
\else
  \RequirePackage{keyval}%
\fi
\pc@KeyvalRestore
%    \end{macrocode}
%
% \subsection{Define options}
%
%    \begin{macrocode}
\@ifundefined{@dblarg}{%
  \long\def\@dblarg#1{\@ifnextchar[{#1}{\@xdblarg{#1}}}%
  \long\def\@xdblarg#1#2{#1[{#2}]{#2}}%
}{}

\newif\ifpc@set
\newif\ifpc@print
\newif\ifpc@copy
\newif\ifpc@edit
\newif\ifpc@annotate
\newif\ifpc@debug
\let\pc@owner\@empty
\let\pc@user\@empty

% default: allow all
\pc@printtrue
\pc@copytrue
\pc@edittrue
\pc@annotatetrue

\edef\pc@temp{\catcode`\noexpand\"=\the\catcode`\"\relax}
\@makeother\"
\def\pc@set{%
  \@PackageInfo{pdfcrypt}{%
    \ifpc@debug
      \ifx\pc@owner\@empty
        No owner password%
      \else
        Owner password: `\pc@owner'%
      \fi
      \MessageBreak
      \ifx\pc@user\@empty
        No user password%
      \else
        User password: `\pc@user'%
      \fi
      \MessageBreak
      Flags: %
      \ifpc@print   \else no\fi print, %
      \ifpc@copy    \else no\fi copy, %
      \ifpc@edit    \else no\fi edit, %
      \ifpc@annotate\else no\fi annotate%
      \MessageBreak
    \fi
    \ifpc@nopdfcrypt
      Encryption is disabled by `\string\nopdfcrypt'%
    \else
      Encryption is set for `\pc@driver'%
    \fi
  }%
  \ifpc@nopdfcrypt
  \else
    \@ifundefined{pc@set@\pc@driver}{%
      \ifx\pc@driver\@empty
        \@PackageError{pdfcrypt}{No driver for encryption %
          support found}\@ehc
      \else
        \@PackageError{pdfcrypt}{Cannot set encryption for %
          unknown driver `\pc@driver'}\@ehc
      \fi
    }{%
      \@nameuse{pc@set@\pc@driver}%
    }%
  \fi
}
\def\pc@set@pdftex{%
  \ifnum\pdftexversion<100 %
    \pc@set@pdftexold
  \else
    \pc@set@pdftexnew
  \fi
}
\def\pc@set@pdftexold{%
  \pdfcrypt{%
    owner "\pc@owner" %
    user "\pc@user" %
    \ifpc@print   \else no\fi print %
    \ifpc@copy    \else no\fi copy %
    \ifpc@edit    \else no\fi edit %
    \ifpc@annotate\else no\fi annotate%
  }%
}
\def\pc@set@pdftexnew{%
  \pdfcrypt
    owner{\pc@owner}%
    user{\pc@user}%
    \ifpc@print   \else no\fi print %
    \ifpc@copy    \else no\fi copy %
    \ifpc@edit    \else no\fi edit %
    \ifpc@annotate\else no\fi annotate%
  \relax
}
\def\pc@set@vtex{%
  \immediate\special{!security %
    O=\pc@MakeVTeXString\pc@owner,%
    U=\pc@MakeVTeXString\pc@user,%
    P\ifpc@print +\else -\fi,%
    C\ifpc@copy +\else -\fi,%
    M\ifpc@edit +\else -\fi,%
    A\ifpc@annotate +\else -\fi
  }%
}
\def\pc@MakeVTeXString#1{%
  "\expandafter\pc@@MakeVTeXString#1"\@nil"%
}
\def\pc@@MakeVTeXString#1"#2\@nil{%
  #1%
  \ifx\\#2\\%
  \else
    ""%
    \@ReturnAfterFi{%
      \pc@@MakeVTeXString#2\@nil
    }%
  \fi
}
\long\def\@ReturnAfterFi#1\fi{\fi#1}
\pc@temp

\begingroup
  \catcode`\ =12 \gdef\pc@spaceother{ }\catcode`\ =10\relax
  \catcode`\|=0 %
  \catcode`\\=12 %
  |gdef|pc@DefString#1#2{%
    |def#1{#2}%
    |edef#1{|expandafter|strip@prefix|meaning#1}%
    |edef#1{|expandafter|pc@SpaceToOther#1 |@nil}%
    |edef#1{|expandafter|pc@EscapeRemove#1|@empty\|@empty|@nil}%
  }%
  |gdef|pc@EscapeRemove#1\#2#3|@nil{%
    #1#2%
    |ifx|\#3|\%%
    |else
      |@ReturnAfterFi{%
        |pc@EscapeRemove#3|@nil
      }%
    |fi
  }%
|endgroup
\def\pc@SpaceToOther#1 #2\@nil{%
  #1%
  \ifx\\#2\\%
  \else
    \pc@spaceother
    \@ReturnAfterFi{%
      \pc@SpaceToOther#2\@nil
    }%
  \fi
}

\def\pc@boolkey{\@dblarg\pc@@boolkey}
\def\pc@@boolkey[#1]#2#3{%
  \lowercase{\def\pc@temp{#3}}%
  \ifx\pc@temp\@empty
    \let\pc@temp\pc@true
  \fi
  \ifx\pc@temp\pc@true
  \else
    \ifx\pc@temp\pc@false
    \else
      \let\pc@temp\relax
    \fi
  \fi
  \ifx\pc@temp\relax
    \@PackageWarning{pdfcrypt}{%
      Unexpected value \string`#3\string' of %
      option \string`#2\string'\MessageBreak
      instead of %
      \string`true\string' or \string`false\string'%
    }%
  \else
    \csname pc@#2\pc@temp\endcsname
  \fi
}
\def\pc@true{true}
\def\pc@false{false}

\define@key{pc}{set}[true]{%
  \pc@boolkey{set}{#1}%
}
\define@key{pc}{pdftex}[]{%
  \def\pc@driver{pdftex}%
}
\define@key{pc}{vtex}[]{%
  \def\pc@driver{vtex}%
}
\define@key{pc}{print}[true]{%
  \pc@boolkey{print}{#1}%
}
\define@key{pc}{copy}[true]{%
  \pc@boolkey{copy}{#1}%
}
\define@key{pc}{edit}[true]{%
  \pc@boolkey{edit}{#1}%
}
\define@key{pc}{annotate}[true]{%
  \pc@boolkey{annotate}{#1}%
}
\define@key{pc}{all}[]{%
  \pc@boolkey{print}{true}%
  \pc@boolkey{copy}{true}%
  \pc@boolkey{edit}{true}%
  \pc@boolkey{annotate}{true}%
}
\define@key{pc}{none}[]{%
  \pc@boolkey{print}{false}%
  \pc@boolkey{copy}{false}%
  \pc@boolkey{edit}{false}%
  \pc@boolkey{annotate}{false}%
}

\define@key{pc}{owner}{%
  \pc@DefString\pc@owner{#1}%
}
\define@key{pc}{user}{%
  \pc@DefString\pc@user{#1}%
}
\define@key{pc}{debug}[true]{%
  \pc@boolkey{debug}{#1}%
}

\def\pdfcryptsetup#1{%
  \setkeys{pc}{#1}%
  \ifpc@set
    \pc@set
    \global\let\pc@set\relax
    \gdef\pdfcryptsetup##1{%
      \@PackageWarning{pdfcrypt}{%
        Encryption options are already set\MessageBreak
        new values are ignored%
      }%
    }%
  \fi
}
\begingroup\expandafter\expandafter\expandafter\endgroup
\expandafter\ifx\csname @onlypreamble\endcsname\relax
\else
  \@onlypreamble\pdfcryptsetup
\fi
%    \end{macrocode}
%
% \subsection{support of configuration file}
%    \begin{macrocode}
\begingroup\expandafter\expandafter\expandafter\endgroup
\expandafter\ifx\csname InputIfFileExists\endcsname\relax
  \@PackageInfo{pdfcrypt}{%
    Configuration file pdfcrypt.cfg not supported.%
  }%
\else
  \let\pc@ExecuteOptions\ExecuteOptions
  \InputIfFileExists{pdfcrypt.cfg}{}{}%
  \let\ExecuteOptions\pc@ExecuteOptions
\fi
%    \end{macrocode}
%
% \subsection{Package options}
%
%    Plain format does not know package options.
%    \begin{macrocode}
\begingroup\expandafter\expandafter\expandafter\endgroup
\expandafter\ifx\csname @classoptionslist\endcsname\relax
  \expandafter\pc@endinput
\fi
%    \end{macrocode}
%    Process global and local options.
%    \begin{macrocode}
\def\pc@ProcessOptionsWithKV{%
  \let\pc@temp\@empty
  \@for\CurrentOption:=\@classoptionslist\do{%
    \@ifundefined{KV@pc@\CurrentOption}{}{%
      \edef\pc@temp{\pc@temp,\CurrentOption,}%
      \@expandtwoargs\@removeelement\CurrentOption
        \@unusedoptionlist\@unusedoptionlist
    }%
  }%
  \edef\pc@temp{%
    \noexpand\pdfcryptsetup{%
      \pc@temp\@ptionlist{\@currname.\@currext}%
    }%
  }%
  \pc@temp
}
\pc@ProcessOptionsWithKV
\AtEndOfPackage{\let\@unprocessedoptions\relax}
\AtBeginDocument{\pc@set}

\pc@endinput
%    \end{macrocode}
%
%    \begin{macrocode}
%</package>
%    \end{macrocode}
%
% \section{Installation}
%
% \subsection{Download}
%
% \paragraph{Package.} This package is available on
% CTAN\footnote{\CTANpkg{pdfcrypt}}:
% \begin{description}
% \item[\CTAN{macros/latex/contrib/oberdiek/pdfcrypt.dtx}] The source file.
% \item[\CTAN{macros/latex/contrib/oberdiek/pdfcrypt.pdf}] Documentation.
% \end{description}
%
%
% \paragraph{Bundle.} All the packages of the bundle `oberdiek'
% are also available in a TDS compliant ZIP archive. There
% the packages are already unpacked and the documentation files
% are generated. The files and directories obey the TDS standard.
% \begin{description}
% \item[\CTANinstall{install/macros/latex/contrib/oberdiek.tds.zip}]
% \end{description}
% \emph{TDS} refers to the standard ``A Directory Structure
% for \TeX\ Files'' (\CTANpkg{tds}). Directories
% with \xfile{texmf} in their name are usually organized this way.
%
% \subsection{Bundle installation}
%
% \paragraph{Unpacking.} Unpack the \xfile{oberdiek.tds.zip} in the
% TDS tree (also known as \xfile{texmf} tree) of your choice.
% Example (linux):
% \begin{quote}
%   |unzip oberdiek.tds.zip -d ~/texmf|
% \end{quote}
%
% \subsection{Package installation}
%
% \paragraph{Unpacking.} The \xfile{.dtx} file is a self-extracting
% \docstrip\ archive. The files are extracted by running the
% \xfile{.dtx} through \plainTeX:
% \begin{quote}
%   \verb|tex pdfcrypt.dtx|
% \end{quote}
%
% \paragraph{TDS.} Now the different files must be moved into
% the different directories in your installation TDS tree
% (also known as \xfile{texmf} tree):
% \begin{quote}
% \def\t{^^A
% \begin{tabular}{@{}>{\ttfamily}l@{ $\rightarrow$ }>{\ttfamily}l@{}}
%   pdfcrypt.sty & tex/generic/oberdiek/pdfcrypt.sty\\
%   pdfcrypt.pdf & doc/latex/oberdiek/pdfcrypt.pdf\\
%   pdfcrypt.dtx & source/latex/oberdiek/pdfcrypt.dtx\\
% \end{tabular}^^A
% }^^A
% \sbox0{\t}^^A
% \ifdim\wd0>\linewidth
%   \begingroup
%     \advance\linewidth by\leftmargin
%     \advance\linewidth by\rightmargin
%   \edef\x{\endgroup
%     \def\noexpand\lw{\the\linewidth}^^A
%   }\x
%   \def\lwbox{^^A
%     \leavevmode
%     \hbox to \linewidth{^^A
%       \kern-\leftmargin\relax
%       \hss
%       \usebox0
%       \hss
%       \kern-\rightmargin\relax
%     }^^A
%   }^^A
%   \ifdim\wd0>\lw
%     \sbox0{\small\t}^^A
%     \ifdim\wd0>\linewidth
%       \ifdim\wd0>\lw
%         \sbox0{\footnotesize\t}^^A
%         \ifdim\wd0>\linewidth
%           \ifdim\wd0>\lw
%             \sbox0{\scriptsize\t}^^A
%             \ifdim\wd0>\linewidth
%               \ifdim\wd0>\lw
%                 \sbox0{\tiny\t}^^A
%                 \ifdim\wd0>\linewidth
%                   \lwbox
%                 \else
%                   \usebox0
%                 \fi
%               \else
%                 \lwbox
%               \fi
%             \else
%               \usebox0
%             \fi
%           \else
%             \lwbox
%           \fi
%         \else
%           \usebox0
%         \fi
%       \else
%         \lwbox
%       \fi
%     \else
%       \usebox0
%     \fi
%   \else
%     \lwbox
%   \fi
% \else
%   \usebox0
% \fi
% \end{quote}
% If you have a \xfile{docstrip.cfg} that configures and enables \docstrip's
% TDS installing feature, then some files can already be in the right
% place, see the documentation of \docstrip.
%
% \subsection{Refresh file name databases}
%
% If your \TeX~distribution
% (\TeX\,Live, \mikTeX, \dots) relies on file name databases, you must refresh
% these. For example, \TeX\,Live\ users run \verb|texhash| or
% \verb|mktexlsr|.
%
% \subsection{Some details for the interested}
%
% \paragraph{Unpacking with \LaTeX.}
% The \xfile{.dtx} chooses its action depending on the format:
% \begin{description}
% \item[\plainTeX:] Run \docstrip\ and extract the files.
% \item[\LaTeX:] Generate the documentation.
% \end{description}
% If you insist on using \LaTeX\ for \docstrip\ (really,
% \docstrip\ does not need \LaTeX), then inform the autodetect routine
% about your intention:
% \begin{quote}
%   \verb|latex \let\install=y\input{pdfcrypt.dtx}|
% \end{quote}
% Do not forget to quote the argument according to the demands
% of your shell.
%
% \paragraph{Generating the documentation.}
% You can use both the \xfile{.dtx} or the \xfile{.drv} to generate
% the documentation. The process can be configured by the
% configuration file \xfile{ltxdoc.cfg}. For instance, put this
% line into this file, if you want to have A4 as paper format:
% \begin{quote}
%   \verb|\PassOptionsToClass{a4paper}{article}|
% \end{quote}
% An example follows how to generate the
% documentation with pdf\LaTeX:
% \begin{quote}
%\begin{verbatim}
%pdflatex pdfcrypt.dtx
%makeindex -s gind.ist pdfcrypt.idx
%pdflatex pdfcrypt.dtx
%makeindex -s gind.ist pdfcrypt.idx
%pdflatex pdfcrypt.dtx
%\end{verbatim}
% \end{quote}
%
% \begin{History}
%   \begin{Version}{2001/04/02 v0.1}
%   \item
%     First public version, published in the pdftex
%     mailing list for testing with pdfcrypt-20010331
%   \end{Version}
%   \begin{Version}{2001/07/19 v0.2}
%   \item
%     Default: all allowed.
%   \item
%     Support for \VTeX\ added.
%   \end{Version}
%   \begin{Version}{2001/07/19 v0.3}
%   \item
%     Bug fix: \VTeX\ letter for edit is M (modify).
%   \end{Version}
%   \begin{Version}{2001/07/19 v0.4}
%   \item
%     Bug fix: \cs{VTeXversion} is correct after
%     regenerating the format file.
%   \end{Version}
%   \begin{Version}{2001/08/05 v0.5}
%   \item
%     Syntax change in \pdfTeX\ 1.00a.
%   \end{Version}
%   \begin{Version}{2001/08/09 v0.6}
%   \item
%     Support of special characters:\\
%     input: |\{|, |\}|, |\\| for |{|, |}|, |\|\\
%     output: |"| in \VTeX
%   \item
%     Option debug added.
%   \end{Version}
%   \begin{Version}{2001/10/28 v0.7}
%   \item
%     Plain compatibility.
%   \item
%     \cs{nopdfcrypt} added.
%   \item
%     Typos corrected.
%   \end{Version}
%   \begin{Version}{2006/02/20 v0.8}
%   \item
%     Obsolete remarks for \pdfTeX.
%   \item
%     DTX framework.
%   \item
%     LPPL 1.3
%   \end{Version}
%   \begin{Version}{2007/04/11 v0.9}
%   \item
%     Line ends sanitized.
%   \end{Version}
%   \begin{Version}{2007/04/26 v1.0}
%   \item
%     Use of package \xpackage{infwarerr}.
%   \end{Version}
%   \begin{Version}{2016/05/16 v1.1}
%   \item
%     Documentation updates.
%   \end{Version}
% \end{History}
%
% \PrintIndex
%
% \Finale
\endinput
|
% \end{quote}
% Do not forget to quote the argument according to the demands
% of your shell.
%
% \paragraph{Generating the documentation.}
% You can use both the \xfile{.dtx} or the \xfile{.drv} to generate
% the documentation. The process can be configured by the
% configuration file \xfile{ltxdoc.cfg}. For instance, put this
% line into this file, if you want to have A4 as paper format:
% \begin{quote}
%   \verb|\PassOptionsToClass{a4paper}{article}|
% \end{quote}
% An example follows how to generate the
% documentation with pdf\LaTeX:
% \begin{quote}
%\begin{verbatim}
%pdflatex pdfcrypt.dtx
%makeindex -s gind.ist pdfcrypt.idx
%pdflatex pdfcrypt.dtx
%makeindex -s gind.ist pdfcrypt.idx
%pdflatex pdfcrypt.dtx
%\end{verbatim}
% \end{quote}
%
% \begin{History}
%   \begin{Version}{2001/04/02 v0.1}
%   \item
%     First public version, published in the pdftex
%     mailing list for testing with pdfcrypt-20010331
%   \end{Version}
%   \begin{Version}{2001/07/19 v0.2}
%   \item
%     Default: all allowed.
%   \item
%     Support for \VTeX\ added.
%   \end{Version}
%   \begin{Version}{2001/07/19 v0.3}
%   \item
%     Bug fix: \VTeX\ letter for edit is M (modify).
%   \end{Version}
%   \begin{Version}{2001/07/19 v0.4}
%   \item
%     Bug fix: \cs{VTeXversion} is correct after
%     regenerating the format file.
%   \end{Version}
%   \begin{Version}{2001/08/05 v0.5}
%   \item
%     Syntax change in \pdfTeX\ 1.00a.
%   \end{Version}
%   \begin{Version}{2001/08/09 v0.6}
%   \item
%     Support of special characters:\\
%     input: |\{|, |\}|, |\\| for |{|, |}|, |\|\\
%     output: |"| in \VTeX
%   \item
%     Option debug added.
%   \end{Version}
%   \begin{Version}{2001/10/28 v0.7}
%   \item
%     Plain compatibility.
%   \item
%     \cs{nopdfcrypt} added.
%   \item
%     Typos corrected.
%   \end{Version}
%   \begin{Version}{2006/02/20 v0.8}
%   \item
%     Obsolete remarks for \pdfTeX.
%   \item
%     DTX framework.
%   \item
%     LPPL 1.3
%   \end{Version}
%   \begin{Version}{2007/04/11 v0.9}
%   \item
%     Line ends sanitized.
%   \end{Version}
%   \begin{Version}{2007/04/26 v1.0}
%   \item
%     Use of package \xpackage{infwarerr}.
%   \end{Version}
%   \begin{Version}{2016/05/16 v1.1}
%   \item
%     Documentation updates.
%   \end{Version}
% \end{History}
%
% \PrintIndex
%
% \Finale
\endinput

%        (quote the arguments according to the demands of your shell)
%
% Documentation:
%    (a) If pdfcrypt.drv is present:
%           latex pdfcrypt.drv
%    (b) Without pdfcrypt.drv:
%           latex pdfcrypt.dtx; ...
%    The class ltxdoc loads the configuration file ltxdoc.cfg
%    if available. Here you can specify further options, e.g.
%    use A4 as paper format:
%       \PassOptionsToClass{a4paper}{article}
%
%    Programm calls to get the documentation (example):
%       pdflatex pdfcrypt.dtx
%       makeindex -s gind.ist pdfcrypt.idx
%       pdflatex pdfcrypt.dtx
%       makeindex -s gind.ist pdfcrypt.idx
%       pdflatex pdfcrypt.dtx
%
% Installation:
%    TDS:tex/generic/oberdiek/pdfcrypt.sty
%    TDS:doc/latex/oberdiek/pdfcrypt.pdf
%    TDS:source/latex/oberdiek/pdfcrypt.dtx
%
%<*ignore>
\begingroup
  \catcode123=1 %
  \catcode125=2 %
  \def\x{LaTeX2e}%
\expandafter\endgroup
\ifcase 0\ifx\install y1\fi\expandafter
         \ifx\csname processbatchFile\endcsname\relax\else1\fi
         \ifx\fmtname\x\else 1\fi\relax
\else\csname fi\endcsname
%</ignore>
%<*install>
\input docstrip.tex
\Msg{************************************************************************}
\Msg{* Installation}
\Msg{* Package: pdfcrypt 2016/05/16 v1.1 Allows the setting of PDF encryption (HO)}
\Msg{************************************************************************}

\keepsilent
\askforoverwritefalse

\let\MetaPrefix\relax
\preamble

This is a generated file.

Project: pdfcrypt
Version: 2016/05/16 v1.1

Copyright (C)
   2001, 2006, 2007 Heiko Oberdiek
   2016-2019 Oberdiek Package Support Group

This work may be distributed and/or modified under the
conditions of the LaTeX Project Public License, either
version 1.3c of this license or (at your option) any later
version. This version of this license is in
   https://www.latex-project.org/lppl/lppl-1-3c.txt
and the latest version of this license is in
   https://www.latex-project.org/lppl.txt
and version 1.3 or later is part of all distributions of
LaTeX version 2005/12/01 or later.

This work has the LPPL maintenance status "maintained".

The Current Maintainers of this work are
Heiko Oberdiek and the Oberdiek Package Support Group
https://github.com/ho-tex/oberdiek/issues


The Base Interpreter refers to any `TeX-Format',
because some files are installed in TDS:tex/generic//.

This work consists of the main source file pdfcrypt.dtx
and the derived files
   pdfcrypt.sty, pdfcrypt.pdf, pdfcrypt.ins, pdfcrypt.drv.

\endpreamble
\let\MetaPrefix\DoubleperCent

\generate{%
  \file{pdfcrypt.ins}{\from{pdfcrypt.dtx}{install}}%
  \file{pdfcrypt.drv}{\from{pdfcrypt.dtx}{driver}}%
  \usedir{tex/generic/oberdiek}%
  \file{pdfcrypt.sty}{\from{pdfcrypt.dtx}{package}}%
  \nopreamble
  \nopostamble
%  \usedir{source/latex/oberdiek/catalogue}%
%  \file{pdfcrypt.xml}{\from{pdfcrypt.dtx}{catalogue}}%
}

\catcode32=13\relax% active space
\let =\space%
\Msg{************************************************************************}
\Msg{*}
\Msg{* To finish the installation you have to move the following}
\Msg{* file into a directory searched by TeX:}
\Msg{*}
\Msg{*     pdfcrypt.sty}
\Msg{*}
\Msg{* To produce the documentation run the file `pdfcrypt.drv'}
\Msg{* through LaTeX.}
\Msg{*}
\Msg{* Happy TeXing!}
\Msg{*}
\Msg{************************************************************************}

\endbatchfile
%</install>
%<*ignore>
\fi
%</ignore>
%<*driver>
\NeedsTeXFormat{LaTeX2e}
\ProvidesFile{pdfcrypt.drv}%
  [2016/05/16 v1.1 Allows the setting of PDF encryption (HO)]%
\documentclass{ltxdoc}
\usepackage{holtxdoc}[2011/11/22]
\begin{document}
  \DocInput{pdfcrypt.dtx}%
\end{document}
%</driver>
% \fi
%
%
% \CharacterTable
%  {Upper-case    \A\B\C\D\E\F\G\H\I\J\K\L\M\N\O\P\Q\R\S\T\U\V\W\X\Y\Z
%   Lower-case    \a\b\c\d\e\f\g\h\i\j\k\l\m\n\o\p\q\r\s\t\u\v\w\x\y\z
%   Digits        \0\1\2\3\4\5\6\7\8\9
%   Exclamation   \!     Double quote  \"     Hash (number) \#
%   Dollar        \$     Percent       \%     Ampersand     \&
%   Acute accent  \'     Left paren    \(     Right paren   \)
%   Asterisk      \*     Plus          \+     Comma         \,
%   Minus         \-     Point         \.     Solidus       \/
%   Colon         \:     Semicolon     \;     Less than     \<
%   Equals        \=     Greater than  \>     Question mark \?
%   Commercial at \@     Left bracket  \[     Backslash     \\
%   Right bracket \]     Circumflex    \^     Underscore    \_
%   Grave accent  \`     Left brace    \{     Vertical bar  \|
%   Right brace   \}     Tilde         \~}
%
% \GetFileInfo{pdfcrypt.drv}
%
% \title{The \xpackage{pdfcrypt} package}
% \date{2016/05/16 v1.1}
% \author{Heiko Oberdiek\thanks
% {Please report any issues at \url{https://github.com/ho-tex/oberdiek/issues}}}
%
% \maketitle
%
% \begin{abstract}
% This package supports the setting of pdf encryption options
% for \VTeX\ and some older versions of \pdfTeX.
% \end{abstract}
%
% \tableofcontents
%
% \section{Documentation}
%
% This package allows the setting of pdf encryption
% options for
% \begin{itemize}
% \item
%   \VTeX, since version 7.35, \url{http://www.micropress-inc.com/}.
% \item
%   \pdfTeX, patched by
%   \NameEmail{Ricardo Sanchez Carmenes}{carmenes@bioquimica.uniovi.es}.\\
%   Some supported versions are:
%   \begin{quote}
%     \pdfTeX-1.00a,\\
%     \pdfTeX-0.14h-pdfcrypt-20010310,\\
%     pdfcrypt-20010331
%   \end{quote}
% \end{itemize}
%
% \noindent
% \textcolor{red}{Note:} Since \pdfTeX-1.10a (2003-01-16) encryption
% support was dropped! Thus the package is now obsolete for recent
% \pdfTeX\ versions.
%
% \subsection{Alternatives}
%
% There are free alternatives, programs that can
% be used for postprocessing the pdf file:
% \begin{itemize}
% \item \textsf{pdftk}\\
%   \url{http://www.accesspdf.com/pdftk/}
% \item \textsf{Multivalent}\\
%   \url{http://multivalent.sourceforge.net/}
% \item \textsf{PDFBox}\\
%   \url{http://www.pdfbox.org/}
% \item \textsf{PDFTrans}\\
%   \url{http://maddingue.free.fr/softwares/pdftrans.html}
% \item \dots
% \end{itemize}
%
% \subsection{Usage}
%
% The options can be set as package options or
% with the command \cs{pdfcryptsetup}:
% \begin{quote}
%   |owner=|\meta{owner password}\\
%   |user=|\meta{user password}\\
%   |print|, |copy|, |edit|, |annotate=true|, |false|\\
%   |all|, |none|\\
%   |debug|
% \end{quote}
% The encryption is set at |\begin{document}| by default,
% but this can be forced for an earlier time
% using the option |set|.
% Example:
% \begin{quote}
%   |\usepackage[owner=Donald,user=Knuth,print=false]{pdfcrypt}|\\
%   \dots\\
%   |\begin{document}|
% \end{quote}
% or
% \begin{quote}
%   |\usepackage{pdfcrypt}|\\
%   |\pdfcryptsetup{owner=Donald,user=Knut}|\\
%   |\pdfcryptsetup{print=false,set}|\\
%   \dots\\
%   |\begin{document}|
% \end{quote}
%
% \subsubsection{Special characters}
%
% The characters `|{|', `|}|', and `|\|' have to be
% escaped with `|\|'. Then the string should not
% be specified as package option but with
% the macro \cs{pdfcryptsetup}, eg:
% \begin{quote}
%   |\pdfcryptsetup{user={\{\\Hello\}},print=false}|
% \end{quote}
% The password here is ``|{\Hello}|''.
% Active characters can be used and are not expanded.
% Macros except for |\{|, |\}|, and |\\| should not be used
% and are not expanded.
%
% \subsubsection{\cs{nopdfcrypt}}
%
% Often the whole process of pdf generation includes
% several \TeX\ compilations and perhaps other
% postprocessing steps. Therefore a feature would
% be useful to disable the encryption stuff in order
% to speed up the first compilations.
% Therefore package \xpackage{pdfcrypt} will look for
% the macro \cs{nopdfcrypt}. If it has the meaning of the
% letter `|Y|', the package will be disabled.
% Example:
% \begin{quote}
%   |pdflatex '\let\nopdfcrypt=Y\input myfile'|\\
%   |thumbpdf myfile|\\
%   |pdfelatex myfile|
% \end{quote}
%
% \subsubsection{Configuration file}
%
% If the file \xfile{pdfcrypt.cfg} exists it is loaded
% and can be used for common settings.
%
% \subsubsection{Support for \plainTeX}
%
% The package can also be used with \plainTeX. It defines the
% missing \LaTeX\ macros and loads package \xpackage{keyval}.
% The configuration file is not supported and automatically
% loaded.
%
% \subsection{Hints for \pdfTeX}
%
% This section is out of date, because the encryption support
% was removed from \pdfTeX. Therefore this hints are of
% historical interest only.
%
%\begin{verbatim}
%* There are packages such as thumbpdf that ship out
%  some stuff (\immediate\pdfobj{...}). In these cases
%  \pdfcrypt will be ignored without warning or error
%  message. Therefore the package pdfcrypt should be
%  loaded before and option "set" should be used in
%  order to force the call of \pdfcrypt, for example:
%     \usepackage{pdfcrypt}
%     \pdfcryptsetup{..., set}
%     ...
%     \usepackage{thumbpdf}
%
%* Compiling pdfTeX (eg. version 1.00a-pretest-20010804):
%  pdfTeX versions are available at
%      ftp://ftp.cstug.cz/pub/tex/local/cstug/thanh/pdftex/snapshots/
%  Unpack the latest .tgz file:
%      > tar xzf pdftex-20010804.tgz
%  Enable encryption support:
%      src> cd src/texk/web2c/pdftexdir
%      src/texk/web2c/pdftexdir> ln -fs pdfcrypt-full.c pdfcrypt.c
%  Compile:
%      src/texk/web2c/pdftexdir> cd ../../..
%      src> ./configure
%      src> cd texk/web2c
%      src/texk/web2c> make pdftexbin
%  At last install the binary and pool files.
%\end{verbatim}
%
% \StopEventually{
% }
%
% \section{Implementation}
%
%    \begin{macrocode}
%<*package>
%    \end{macrocode}
%
% \subsection{Help macros for \plainTeX}
%
%    \begin{macrocode}
\expandafter\edef\csname pc@endinput\endcsname{%
  \catcode`\noexpand\@=\the\catcode`\@\relax
  \noexpand\endinput
}
\catcode`\@=11 %

\expandafter\ifx\csname @firstoftwo\endcsname\relax
  \long\def\@firstoftwo#1#2{#1}%
\fi

\expandafter\ifx\csname @secondoftwo\endcsname\relax
  \long\def\@secondoftwo#1#2{#2}%
\fi

\expandafter\ifx\csname @ifundefined\endcsname\relax
  \def\@ifundefined#1{%
    \expandafter\ifx\csname #1\endcsname\relax
      \expandafter\@firstoftwo
    \else
      \expandafter\@secondoftwo
    \fi
  }%
\fi

\@ifundefined{@ifnextchar}{%
  \long\def\@ifnextchar#1#2#3{%
    \let\reserved@d=#1%
    \def\reserved@a{#2}%
    \def\reserved@b{#3}%
    \futurelet\@let@token\@ifnch
  }%
  \def\@ifnch{%
    \ifx\@let@token\@sptoken
      \let\reserved@c\@xifnch
    \else
      \ifx\@let@token\reserved@d
        \let\reserved@c\reserved@a
      \else
        \let\reserved@c\reserved@b
      \fi
    \fi
    \reserved@c
  }%
  \begingroup
    \def\:{\global\let\@sptoken= }%
    \: %
    \def\:{\@xifnch}%
    \expandafter\gdef\: {\futurelet\@let@token\@ifnch}%
  \endgroup
}{}

\@ifundefined{ProvidesPackage}{%
  \def\ProvidesPackage#1{%
    \@ifnextchar[{\pc@ProvidesPackage{#1}}%
                 {\pc@ProvidesPackage{#1}[]}%
  }%
  \def\pc@ProvidesPackage#1[#2]{%
    \immediate\write-1{Package: #1 #2}%
  }%
}{}

\begingroup\expandafter\expandafter\expandafter\endgroup
\expandafter\ifx\csname RequirePackage\endcsname\relax
  \input infwarerr.sty\relax
\else
  \RequirePackage{infwarerr}%
\fi

\@ifundefined{@gobble}{%
  \long\def\@gobble#1{}%
}{}

\@ifundefined{@empty}{%
  \def\@empty{}%
}{}
%    \end{macrocode}
%
% \subsection{Package Identification and checks}
%
%    \begin{macrocode}
\ProvidesPackage{pdfcrypt}%
  [2016/05/16 v1.1 Allows the setting of PDF encryption (HO)]%
%    \end{macrocode}
%
%    \begin{macrocode}
\@ifundefined{pdfcryptsetup}{%
  \let\pdfcryptsetup\@gobble
}{%
  \@PackageWarningNoLine{pdfcrypt}{Package pdfcrypt is already loaded}%
  \pc@endinput
}
%    \end{macrocode}
%    Support for \cs{nopdfcrypt}.
%    \begin{macrocode}
\newif\ifpc@nopdfcrypt
\ifx Y\nopdfcrypt
  \@PackageWarningNoLine{pdfcrypt}{%
    Encryption disabled by \string \nopdfcrypt\space request%
  }%
  \global\pc@nopdfcrypttrue
\fi
%    \end{macrocode}
%
% \subsection{Driver detection}
%
%    \begin{macrocode}
\let\pc@driver\@empty
\begingroup
  % pdfTeX detection
  \@ifundefined{pdfoutput}{%
  }{%
    \ifcase\pdfoutput
    \else
      \@ifundefined{pdfcrypt}{%
        \@PackageError{pdfcrypt}{%
          PDF encryption is not supported with this pdfTeX%
        }{%
          Encryption support was added in 0.14h (2001/03/10)\MessageBreak
          and removed in 1.10a (2003/01/16).%
        }%
        \endgroup
        \pc@endinput
      }{%
        \gdef\pc@driver{pdftex}%
      }%
    \fi
  }%
  % VTeX detection
  \@ifundefined{OpMode}{%
  }{%
    \ifnum\OpMode=1 %
      \ifnum\@ifundefined{VTeXversion}0\VTeXversion<735 %
        \@PackageError{pdfcrypt}{%
          PDF encryption is not supported with this VTeX%
        }{%
          You need VTeX 7.35 or higher.%
        }%
        \endgroup
        \pc@endinput
      \else
        \gdef\pc@driver{vtex}%
      \fi
    \fi
  }%
\endgroup
%    \end{macrocode}
%
% \subsection{Load package \xpackage{keyval}}
%
%    \begin{macrocode}
\@ifundefined{@makeother}{%
  \def\@makeother#1{\catcode`#1=12\relax}%
}{}

\@ifundefined{g@addto@macro}{%
  \long\def\g@addto@macro#1#2{%
    \begingroup
      \toks@\expandafter{#1#2}%
      \xdef#1{\the\toks@}%
    \endgroup
  }%
}{}

\@ifundefined{@namedef}{%
  \def\@namedef#1{\expandafter\def\csname#1\endcsname}%
}{}

\@ifundefined{@nameuse}{%
  \def\@nameuse#1{\csname #1\endcsname}%
}{}

\def\pc@KeyvalRestore{%
  \let\pc@KeyvalRestore\@undefined
}

\let\pcOrg@NeedsTeXFormat\NeedsTeXFormat
\@ifundefined{NeedsTeXFormat}{%
  \def\NeedsTeXFormat#1{}%
  \g@addto@macro\pc@KeyvalRestore{%
    \let\NeedsTeXFormat\pcOrg@NeedsTeXFormat
  }%
}{}

\let\pcOrg@DeclareOption\DeclareOption
\@ifundefined{DeclareOption}{%
  \def\DeclareOption#1#2{#2}%
  \g@addto@macro\pc@KeyvalRestore{%
    \let\DeclareOption\pcOrg@DeclareOption
  }%
}{}

\let\pcOrg@ExecuteOptions\ExecuteOptions
\@ifundefined{ExecuteOptions}{%
  \def\ExecuteOptions#1{}%
  \g@addto@macro\pc@KeyvalRestore{%
    \let\ExecuteOptions\pcOrg@ExecuteOptions
  }%
}{}

\let\pcOrg@ProcessOptions\ProcessOptions
\@ifundefined{ProcessOptions}{%
  \def\ProcessOptions{}%
  \g@addto@macro\pc@KeyvalRestore{%
    \let\ProcessOptions\pcOrg@ProcessOptions
  }%
}{}%

\begingroup\expandafter\expandafter\expandafter\endgroup
\expandafter\ifx\csname RequirePackage\endcsname\relax
  \input keyval.sty\relax
\else
  \RequirePackage{keyval}%
\fi
\pc@KeyvalRestore
%    \end{macrocode}
%
% \subsection{Define options}
%
%    \begin{macrocode}
\@ifundefined{@dblarg}{%
  \long\def\@dblarg#1{\@ifnextchar[{#1}{\@xdblarg{#1}}}%
  \long\def\@xdblarg#1#2{#1[{#2}]{#2}}%
}{}

\newif\ifpc@set
\newif\ifpc@print
\newif\ifpc@copy
\newif\ifpc@edit
\newif\ifpc@annotate
\newif\ifpc@debug
\let\pc@owner\@empty
\let\pc@user\@empty

% default: allow all
\pc@printtrue
\pc@copytrue
\pc@edittrue
\pc@annotatetrue

\edef\pc@temp{\catcode`\noexpand\"=\the\catcode`\"\relax}
\@makeother\"
\def\pc@set{%
  \@PackageInfo{pdfcrypt}{%
    \ifpc@debug
      \ifx\pc@owner\@empty
        No owner password%
      \else
        Owner password: `\pc@owner'%
      \fi
      \MessageBreak
      \ifx\pc@user\@empty
        No user password%
      \else
        User password: `\pc@user'%
      \fi
      \MessageBreak
      Flags: %
      \ifpc@print   \else no\fi print, %
      \ifpc@copy    \else no\fi copy, %
      \ifpc@edit    \else no\fi edit, %
      \ifpc@annotate\else no\fi annotate%
      \MessageBreak
    \fi
    \ifpc@nopdfcrypt
      Encryption is disabled by `\string\nopdfcrypt'%
    \else
      Encryption is set for `\pc@driver'%
    \fi
  }%
  \ifpc@nopdfcrypt
  \else
    \@ifundefined{pc@set@\pc@driver}{%
      \ifx\pc@driver\@empty
        \@PackageError{pdfcrypt}{No driver for encryption %
          support found}\@ehc
      \else
        \@PackageError{pdfcrypt}{Cannot set encryption for %
          unknown driver `\pc@driver'}\@ehc
      \fi
    }{%
      \@nameuse{pc@set@\pc@driver}%
    }%
  \fi
}
\def\pc@set@pdftex{%
  \ifnum\pdftexversion<100 %
    \pc@set@pdftexold
  \else
    \pc@set@pdftexnew
  \fi
}
\def\pc@set@pdftexold{%
  \pdfcrypt{%
    owner "\pc@owner" %
    user "\pc@user" %
    \ifpc@print   \else no\fi print %
    \ifpc@copy    \else no\fi copy %
    \ifpc@edit    \else no\fi edit %
    \ifpc@annotate\else no\fi annotate%
  }%
}
\def\pc@set@pdftexnew{%
  \pdfcrypt
    owner{\pc@owner}%
    user{\pc@user}%
    \ifpc@print   \else no\fi print %
    \ifpc@copy    \else no\fi copy %
    \ifpc@edit    \else no\fi edit %
    \ifpc@annotate\else no\fi annotate%
  \relax
}
\def\pc@set@vtex{%
  \immediate\special{!security %
    O=\pc@MakeVTeXString\pc@owner,%
    U=\pc@MakeVTeXString\pc@user,%
    P\ifpc@print +\else -\fi,%
    C\ifpc@copy +\else -\fi,%
    M\ifpc@edit +\else -\fi,%
    A\ifpc@annotate +\else -\fi
  }%
}
\def\pc@MakeVTeXString#1{%
  "\expandafter\pc@@MakeVTeXString#1"\@nil"%
}
\def\pc@@MakeVTeXString#1"#2\@nil{%
  #1%
  \ifx\\#2\\%
  \else
    ""%
    \@ReturnAfterFi{%
      \pc@@MakeVTeXString#2\@nil
    }%
  \fi
}
\long\def\@ReturnAfterFi#1\fi{\fi#1}
\pc@temp

\begingroup
  \catcode`\ =12 \gdef\pc@spaceother{ }\catcode`\ =10\relax
  \catcode`\|=0 %
  \catcode`\\=12 %
  |gdef|pc@DefString#1#2{%
    |def#1{#2}%
    |edef#1{|expandafter|strip@prefix|meaning#1}%
    |edef#1{|expandafter|pc@SpaceToOther#1 |@nil}%
    |edef#1{|expandafter|pc@EscapeRemove#1|@empty\|@empty|@nil}%
  }%
  |gdef|pc@EscapeRemove#1\#2#3|@nil{%
    #1#2%
    |ifx|\#3|\%%
    |else
      |@ReturnAfterFi{%
        |pc@EscapeRemove#3|@nil
      }%
    |fi
  }%
|endgroup
\def\pc@SpaceToOther#1 #2\@nil{%
  #1%
  \ifx\\#2\\%
  \else
    \pc@spaceother
    \@ReturnAfterFi{%
      \pc@SpaceToOther#2\@nil
    }%
  \fi
}

\def\pc@boolkey{\@dblarg\pc@@boolkey}
\def\pc@@boolkey[#1]#2#3{%
  \lowercase{\def\pc@temp{#3}}%
  \ifx\pc@temp\@empty
    \let\pc@temp\pc@true
  \fi
  \ifx\pc@temp\pc@true
  \else
    \ifx\pc@temp\pc@false
    \else
      \let\pc@temp\relax
    \fi
  \fi
  \ifx\pc@temp\relax
    \@PackageWarning{pdfcrypt}{%
      Unexpected value \string`#3\string' of %
      option \string`#2\string'\MessageBreak
      instead of %
      \string`true\string' or \string`false\string'%
    }%
  \else
    \csname pc@#2\pc@temp\endcsname
  \fi
}
\def\pc@true{true}
\def\pc@false{false}

\define@key{pc}{set}[true]{%
  \pc@boolkey{set}{#1}%
}
\define@key{pc}{pdftex}[]{%
  \def\pc@driver{pdftex}%
}
\define@key{pc}{vtex}[]{%
  \def\pc@driver{vtex}%
}
\define@key{pc}{print}[true]{%
  \pc@boolkey{print}{#1}%
}
\define@key{pc}{copy}[true]{%
  \pc@boolkey{copy}{#1}%
}
\define@key{pc}{edit}[true]{%
  \pc@boolkey{edit}{#1}%
}
\define@key{pc}{annotate}[true]{%
  \pc@boolkey{annotate}{#1}%
}
\define@key{pc}{all}[]{%
  \pc@boolkey{print}{true}%
  \pc@boolkey{copy}{true}%
  \pc@boolkey{edit}{true}%
  \pc@boolkey{annotate}{true}%
}
\define@key{pc}{none}[]{%
  \pc@boolkey{print}{false}%
  \pc@boolkey{copy}{false}%
  \pc@boolkey{edit}{false}%
  \pc@boolkey{annotate}{false}%
}

\define@key{pc}{owner}{%
  \pc@DefString\pc@owner{#1}%
}
\define@key{pc}{user}{%
  \pc@DefString\pc@user{#1}%
}
\define@key{pc}{debug}[true]{%
  \pc@boolkey{debug}{#1}%
}

\def\pdfcryptsetup#1{%
  \setkeys{pc}{#1}%
  \ifpc@set
    \pc@set
    \global\let\pc@set\relax
    \gdef\pdfcryptsetup##1{%
      \@PackageWarning{pdfcrypt}{%
        Encryption options are already set\MessageBreak
        new values are ignored%
      }%
    }%
  \fi
}
\begingroup\expandafter\expandafter\expandafter\endgroup
\expandafter\ifx\csname @onlypreamble\endcsname\relax
\else
  \@onlypreamble\pdfcryptsetup
\fi
%    \end{macrocode}
%
% \subsection{support of configuration file}
%    \begin{macrocode}
\begingroup\expandafter\expandafter\expandafter\endgroup
\expandafter\ifx\csname InputIfFileExists\endcsname\relax
  \@PackageInfo{pdfcrypt}{%
    Configuration file pdfcrypt.cfg not supported.%
  }%
\else
  \let\pc@ExecuteOptions\ExecuteOptions
  \InputIfFileExists{pdfcrypt.cfg}{}{}%
  \let\ExecuteOptions\pc@ExecuteOptions
\fi
%    \end{macrocode}
%
% \subsection{Package options}
%
%    Plain format does not know package options.
%    \begin{macrocode}
\begingroup\expandafter\expandafter\expandafter\endgroup
\expandafter\ifx\csname @classoptionslist\endcsname\relax
  \expandafter\pc@endinput
\fi
%    \end{macrocode}
%    Process global and local options.
%    \begin{macrocode}
\def\pc@ProcessOptionsWithKV{%
  \let\pc@temp\@empty
  \@for\CurrentOption:=\@classoptionslist\do{%
    \@ifundefined{KV@pc@\CurrentOption}{}{%
      \edef\pc@temp{\pc@temp,\CurrentOption,}%
      \@expandtwoargs\@removeelement\CurrentOption
        \@unusedoptionlist\@unusedoptionlist
    }%
  }%
  \edef\pc@temp{%
    \noexpand\pdfcryptsetup{%
      \pc@temp\@ptionlist{\@currname.\@currext}%
    }%
  }%
  \pc@temp
}
\pc@ProcessOptionsWithKV
\AtEndOfPackage{\let\@unprocessedoptions\relax}
\AtBeginDocument{\pc@set}

\pc@endinput
%    \end{macrocode}
%
%    \begin{macrocode}
%</package>
%    \end{macrocode}
%
% \section{Installation}
%
% \subsection{Download}
%
% \paragraph{Package.} This package is available on
% CTAN\footnote{\CTANpkg{pdfcrypt}}:
% \begin{description}
% \item[\CTAN{macros/latex/contrib/oberdiek/pdfcrypt.dtx}] The source file.
% \item[\CTAN{macros/latex/contrib/oberdiek/pdfcrypt.pdf}] Documentation.
% \end{description}
%
%
% \paragraph{Bundle.} All the packages of the bundle `oberdiek'
% are also available in a TDS compliant ZIP archive. There
% the packages are already unpacked and the documentation files
% are generated. The files and directories obey the TDS standard.
% \begin{description}
% \item[\CTANinstall{install/macros/latex/contrib/oberdiek.tds.zip}]
% \end{description}
% \emph{TDS} refers to the standard ``A Directory Structure
% for \TeX\ Files'' (\CTANpkg{tds}). Directories
% with \xfile{texmf} in their name are usually organized this way.
%
% \subsection{Bundle installation}
%
% \paragraph{Unpacking.} Unpack the \xfile{oberdiek.tds.zip} in the
% TDS tree (also known as \xfile{texmf} tree) of your choice.
% Example (linux):
% \begin{quote}
%   |unzip oberdiek.tds.zip -d ~/texmf|
% \end{quote}
%
% \subsection{Package installation}
%
% \paragraph{Unpacking.} The \xfile{.dtx} file is a self-extracting
% \docstrip\ archive. The files are extracted by running the
% \xfile{.dtx} through \plainTeX:
% \begin{quote}
%   \verb|tex pdfcrypt.dtx|
% \end{quote}
%
% \paragraph{TDS.} Now the different files must be moved into
% the different directories in your installation TDS tree
% (also known as \xfile{texmf} tree):
% \begin{quote}
% \def\t{^^A
% \begin{tabular}{@{}>{\ttfamily}l@{ $\rightarrow$ }>{\ttfamily}l@{}}
%   pdfcrypt.sty & tex/generic/oberdiek/pdfcrypt.sty\\
%   pdfcrypt.pdf & doc/latex/oberdiek/pdfcrypt.pdf\\
%   pdfcrypt.dtx & source/latex/oberdiek/pdfcrypt.dtx\\
% \end{tabular}^^A
% }^^A
% \sbox0{\t}^^A
% \ifdim\wd0>\linewidth
%   \begingroup
%     \advance\linewidth by\leftmargin
%     \advance\linewidth by\rightmargin
%   \edef\x{\endgroup
%     \def\noexpand\lw{\the\linewidth}^^A
%   }\x
%   \def\lwbox{^^A
%     \leavevmode
%     \hbox to \linewidth{^^A
%       \kern-\leftmargin\relax
%       \hss
%       \usebox0
%       \hss
%       \kern-\rightmargin\relax
%     }^^A
%   }^^A
%   \ifdim\wd0>\lw
%     \sbox0{\small\t}^^A
%     \ifdim\wd0>\linewidth
%       \ifdim\wd0>\lw
%         \sbox0{\footnotesize\t}^^A
%         \ifdim\wd0>\linewidth
%           \ifdim\wd0>\lw
%             \sbox0{\scriptsize\t}^^A
%             \ifdim\wd0>\linewidth
%               \ifdim\wd0>\lw
%                 \sbox0{\tiny\t}^^A
%                 \ifdim\wd0>\linewidth
%                   \lwbox
%                 \else
%                   \usebox0
%                 \fi
%               \else
%                 \lwbox
%               \fi
%             \else
%               \usebox0
%             \fi
%           \else
%             \lwbox
%           \fi
%         \else
%           \usebox0
%         \fi
%       \else
%         \lwbox
%       \fi
%     \else
%       \usebox0
%     \fi
%   \else
%     \lwbox
%   \fi
% \else
%   \usebox0
% \fi
% \end{quote}
% If you have a \xfile{docstrip.cfg} that configures and enables \docstrip's
% TDS installing feature, then some files can already be in the right
% place, see the documentation of \docstrip.
%
% \subsection{Refresh file name databases}
%
% If your \TeX~distribution
% (\TeX\,Live, \mikTeX, \dots) relies on file name databases, you must refresh
% these. For example, \TeX\,Live\ users run \verb|texhash| or
% \verb|mktexlsr|.
%
% \subsection{Some details for the interested}
%
% \paragraph{Unpacking with \LaTeX.}
% The \xfile{.dtx} chooses its action depending on the format:
% \begin{description}
% \item[\plainTeX:] Run \docstrip\ and extract the files.
% \item[\LaTeX:] Generate the documentation.
% \end{description}
% If you insist on using \LaTeX\ for \docstrip\ (really,
% \docstrip\ does not need \LaTeX), then inform the autodetect routine
% about your intention:
% \begin{quote}
%   \verb|latex \let\install=y% \iffalse meta-comment
%
% File: pdfcrypt.dtx
% Version: 2016/05/16 v1.1
% Info: Allows the setting of PDF encryption
%
% Copyright (C)
%    2001, 2006, 2007 Heiko Oberdiek
%    2016-2019 Oberdiek Package Support Group
%    https://github.com/ho-tex/oberdiek/issues
%
% This work may be distributed and/or modified under the
% conditions of the LaTeX Project Public License, either
% version 1.3c of this license or (at your option) any later
% version. This version of this license is in
%    https://www.latex-project.org/lppl/lppl-1-3c.txt
% and the latest version of this license is in
%    https://www.latex-project.org/lppl.txt
% and version 1.3 or later is part of all distributions of
% LaTeX version 2005/12/01 or later.
%
% This work has the LPPL maintenance status "maintained".
%
% The Current Maintainers of this work are
% Heiko Oberdiek and the Oberdiek Package Support Group
% https://github.com/ho-tex/oberdiek/issues
%
% The Base Interpreter refers to any `TeX-Format',
% because some files are installed in TDS:tex/generic//.
%
% This work consists of the main source file pdfcrypt.dtx
% and the derived files
%    pdfcrypt.sty, pdfcrypt.pdf, pdfcrypt.ins, pdfcrypt.drv.
%
% Distribution:
%    CTAN:macros/latex/contrib/oberdiek/pdfcrypt.dtx
%    CTAN:macros/latex/contrib/oberdiek/pdfcrypt.pdf
%
% Unpacking:
%    (a) If pdfcrypt.ins is present:
%           tex pdfcrypt.ins
%    (b) Without pdfcrypt.ins:
%           tex pdfcrypt.dtx
%    (c) If you insist on using LaTeX
%           latex \let\install=y% \iffalse meta-comment
%
% File: pdfcrypt.dtx
% Version: 2016/05/16 v1.1
% Info: Allows the setting of PDF encryption
%
% Copyright (C)
%    2001, 2006, 2007 Heiko Oberdiek
%    2016-2019 Oberdiek Package Support Group
%    https://github.com/ho-tex/oberdiek/issues
%
% This work may be distributed and/or modified under the
% conditions of the LaTeX Project Public License, either
% version 1.3c of this license or (at your option) any later
% version. This version of this license is in
%    https://www.latex-project.org/lppl/lppl-1-3c.txt
% and the latest version of this license is in
%    https://www.latex-project.org/lppl.txt
% and version 1.3 or later is part of all distributions of
% LaTeX version 2005/12/01 or later.
%
% This work has the LPPL maintenance status "maintained".
%
% The Current Maintainers of this work are
% Heiko Oberdiek and the Oberdiek Package Support Group
% https://github.com/ho-tex/oberdiek/issues
%
% The Base Interpreter refers to any `TeX-Format',
% because some files are installed in TDS:tex/generic//.
%
% This work consists of the main source file pdfcrypt.dtx
% and the derived files
%    pdfcrypt.sty, pdfcrypt.pdf, pdfcrypt.ins, pdfcrypt.drv.
%
% Distribution:
%    CTAN:macros/latex/contrib/oberdiek/pdfcrypt.dtx
%    CTAN:macros/latex/contrib/oberdiek/pdfcrypt.pdf
%
% Unpacking:
%    (a) If pdfcrypt.ins is present:
%           tex pdfcrypt.ins
%    (b) Without pdfcrypt.ins:
%           tex pdfcrypt.dtx
%    (c) If you insist on using LaTeX
%           latex \let\install=y\input{pdfcrypt.dtx}
%        (quote the arguments according to the demands of your shell)
%
% Documentation:
%    (a) If pdfcrypt.drv is present:
%           latex pdfcrypt.drv
%    (b) Without pdfcrypt.drv:
%           latex pdfcrypt.dtx; ...
%    The class ltxdoc loads the configuration file ltxdoc.cfg
%    if available. Here you can specify further options, e.g.
%    use A4 as paper format:
%       \PassOptionsToClass{a4paper}{article}
%
%    Programm calls to get the documentation (example):
%       pdflatex pdfcrypt.dtx
%       makeindex -s gind.ist pdfcrypt.idx
%       pdflatex pdfcrypt.dtx
%       makeindex -s gind.ist pdfcrypt.idx
%       pdflatex pdfcrypt.dtx
%
% Installation:
%    TDS:tex/generic/oberdiek/pdfcrypt.sty
%    TDS:doc/latex/oberdiek/pdfcrypt.pdf
%    TDS:source/latex/oberdiek/pdfcrypt.dtx
%
%<*ignore>
\begingroup
  \catcode123=1 %
  \catcode125=2 %
  \def\x{LaTeX2e}%
\expandafter\endgroup
\ifcase 0\ifx\install y1\fi\expandafter
         \ifx\csname processbatchFile\endcsname\relax\else1\fi
         \ifx\fmtname\x\else 1\fi\relax
\else\csname fi\endcsname
%</ignore>
%<*install>
\input docstrip.tex
\Msg{************************************************************************}
\Msg{* Installation}
\Msg{* Package: pdfcrypt 2016/05/16 v1.1 Allows the setting of PDF encryption (HO)}
\Msg{************************************************************************}

\keepsilent
\askforoverwritefalse

\let\MetaPrefix\relax
\preamble

This is a generated file.

Project: pdfcrypt
Version: 2016/05/16 v1.1

Copyright (C)
   2001, 2006, 2007 Heiko Oberdiek
   2016-2019 Oberdiek Package Support Group

This work may be distributed and/or modified under the
conditions of the LaTeX Project Public License, either
version 1.3c of this license or (at your option) any later
version. This version of this license is in
   https://www.latex-project.org/lppl/lppl-1-3c.txt
and the latest version of this license is in
   https://www.latex-project.org/lppl.txt
and version 1.3 or later is part of all distributions of
LaTeX version 2005/12/01 or later.

This work has the LPPL maintenance status "maintained".

The Current Maintainers of this work are
Heiko Oberdiek and the Oberdiek Package Support Group
https://github.com/ho-tex/oberdiek/issues


The Base Interpreter refers to any `TeX-Format',
because some files are installed in TDS:tex/generic//.

This work consists of the main source file pdfcrypt.dtx
and the derived files
   pdfcrypt.sty, pdfcrypt.pdf, pdfcrypt.ins, pdfcrypt.drv.

\endpreamble
\let\MetaPrefix\DoubleperCent

\generate{%
  \file{pdfcrypt.ins}{\from{pdfcrypt.dtx}{install}}%
  \file{pdfcrypt.drv}{\from{pdfcrypt.dtx}{driver}}%
  \usedir{tex/generic/oberdiek}%
  \file{pdfcrypt.sty}{\from{pdfcrypt.dtx}{package}}%
  \nopreamble
  \nopostamble
%  \usedir{source/latex/oberdiek/catalogue}%
%  \file{pdfcrypt.xml}{\from{pdfcrypt.dtx}{catalogue}}%
}

\catcode32=13\relax% active space
\let =\space%
\Msg{************************************************************************}
\Msg{*}
\Msg{* To finish the installation you have to move the following}
\Msg{* file into a directory searched by TeX:}
\Msg{*}
\Msg{*     pdfcrypt.sty}
\Msg{*}
\Msg{* To produce the documentation run the file `pdfcrypt.drv'}
\Msg{* through LaTeX.}
\Msg{*}
\Msg{* Happy TeXing!}
\Msg{*}
\Msg{************************************************************************}

\endbatchfile
%</install>
%<*ignore>
\fi
%</ignore>
%<*driver>
\NeedsTeXFormat{LaTeX2e}
\ProvidesFile{pdfcrypt.drv}%
  [2016/05/16 v1.1 Allows the setting of PDF encryption (HO)]%
\documentclass{ltxdoc}
\usepackage{holtxdoc}[2011/11/22]
\begin{document}
  \DocInput{pdfcrypt.dtx}%
\end{document}
%</driver>
% \fi
%
%
% \CharacterTable
%  {Upper-case    \A\B\C\D\E\F\G\H\I\J\K\L\M\N\O\P\Q\R\S\T\U\V\W\X\Y\Z
%   Lower-case    \a\b\c\d\e\f\g\h\i\j\k\l\m\n\o\p\q\r\s\t\u\v\w\x\y\z
%   Digits        \0\1\2\3\4\5\6\7\8\9
%   Exclamation   \!     Double quote  \"     Hash (number) \#
%   Dollar        \$     Percent       \%     Ampersand     \&
%   Acute accent  \'     Left paren    \(     Right paren   \)
%   Asterisk      \*     Plus          \+     Comma         \,
%   Minus         \-     Point         \.     Solidus       \/
%   Colon         \:     Semicolon     \;     Less than     \<
%   Equals        \=     Greater than  \>     Question mark \?
%   Commercial at \@     Left bracket  \[     Backslash     \\
%   Right bracket \]     Circumflex    \^     Underscore    \_
%   Grave accent  \`     Left brace    \{     Vertical bar  \|
%   Right brace   \}     Tilde         \~}
%
% \GetFileInfo{pdfcrypt.drv}
%
% \title{The \xpackage{pdfcrypt} package}
% \date{2016/05/16 v1.1}
% \author{Heiko Oberdiek\thanks
% {Please report any issues at \url{https://github.com/ho-tex/oberdiek/issues}}}
%
% \maketitle
%
% \begin{abstract}
% This package supports the setting of pdf encryption options
% for \VTeX\ and some older versions of \pdfTeX.
% \end{abstract}
%
% \tableofcontents
%
% \section{Documentation}
%
% This package allows the setting of pdf encryption
% options for
% \begin{itemize}
% \item
%   \VTeX, since version 7.35, \url{http://www.micropress-inc.com/}.
% \item
%   \pdfTeX, patched by
%   \NameEmail{Ricardo Sanchez Carmenes}{carmenes@bioquimica.uniovi.es}.\\
%   Some supported versions are:
%   \begin{quote}
%     \pdfTeX-1.00a,\\
%     \pdfTeX-0.14h-pdfcrypt-20010310,\\
%     pdfcrypt-20010331
%   \end{quote}
% \end{itemize}
%
% \noindent
% \textcolor{red}{Note:} Since \pdfTeX-1.10a (2003-01-16) encryption
% support was dropped! Thus the package is now obsolete for recent
% \pdfTeX\ versions.
%
% \subsection{Alternatives}
%
% There are free alternatives, programs that can
% be used for postprocessing the pdf file:
% \begin{itemize}
% \item \textsf{pdftk}\\
%   \url{http://www.accesspdf.com/pdftk/}
% \item \textsf{Multivalent}\\
%   \url{http://multivalent.sourceforge.net/}
% \item \textsf{PDFBox}\\
%   \url{http://www.pdfbox.org/}
% \item \textsf{PDFTrans}\\
%   \url{http://maddingue.free.fr/softwares/pdftrans.html}
% \item \dots
% \end{itemize}
%
% \subsection{Usage}
%
% The options can be set as package options or
% with the command \cs{pdfcryptsetup}:
% \begin{quote}
%   |owner=|\meta{owner password}\\
%   |user=|\meta{user password}\\
%   |print|, |copy|, |edit|, |annotate=true|, |false|\\
%   |all|, |none|\\
%   |debug|
% \end{quote}
% The encryption is set at |\begin{document}| by default,
% but this can be forced for an earlier time
% using the option |set|.
% Example:
% \begin{quote}
%   |\usepackage[owner=Donald,user=Knuth,print=false]{pdfcrypt}|\\
%   \dots\\
%   |\begin{document}|
% \end{quote}
% or
% \begin{quote}
%   |\usepackage{pdfcrypt}|\\
%   |\pdfcryptsetup{owner=Donald,user=Knut}|\\
%   |\pdfcryptsetup{print=false,set}|\\
%   \dots\\
%   |\begin{document}|
% \end{quote}
%
% \subsubsection{Special characters}
%
% The characters `|{|', `|}|', and `|\|' have to be
% escaped with `|\|'. Then the string should not
% be specified as package option but with
% the macro \cs{pdfcryptsetup}, eg:
% \begin{quote}
%   |\pdfcryptsetup{user={\{\\Hello\}},print=false}|
% \end{quote}
% The password here is ``|{\Hello}|''.
% Active characters can be used and are not expanded.
% Macros except for |\{|, |\}|, and |\\| should not be used
% and are not expanded.
%
% \subsubsection{\cs{nopdfcrypt}}
%
% Often the whole process of pdf generation includes
% several \TeX\ compilations and perhaps other
% postprocessing steps. Therefore a feature would
% be useful to disable the encryption stuff in order
% to speed up the first compilations.
% Therefore package \xpackage{pdfcrypt} will look for
% the macro \cs{nopdfcrypt}. If it has the meaning of the
% letter `|Y|', the package will be disabled.
% Example:
% \begin{quote}
%   |pdflatex '\let\nopdfcrypt=Y\input myfile'|\\
%   |thumbpdf myfile|\\
%   |pdfelatex myfile|
% \end{quote}
%
% \subsubsection{Configuration file}
%
% If the file \xfile{pdfcrypt.cfg} exists it is loaded
% and can be used for common settings.
%
% \subsubsection{Support for \plainTeX}
%
% The package can also be used with \plainTeX. It defines the
% missing \LaTeX\ macros and loads package \xpackage{keyval}.
% The configuration file is not supported and automatically
% loaded.
%
% \subsection{Hints for \pdfTeX}
%
% This section is out of date, because the encryption support
% was removed from \pdfTeX. Therefore this hints are of
% historical interest only.
%
%\begin{verbatim}
%* There are packages such as thumbpdf that ship out
%  some stuff (\immediate\pdfobj{...}). In these cases
%  \pdfcrypt will be ignored without warning or error
%  message. Therefore the package pdfcrypt should be
%  loaded before and option "set" should be used in
%  order to force the call of \pdfcrypt, for example:
%     \usepackage{pdfcrypt}
%     \pdfcryptsetup{..., set}
%     ...
%     \usepackage{thumbpdf}
%
%* Compiling pdfTeX (eg. version 1.00a-pretest-20010804):
%  pdfTeX versions are available at
%      ftp://ftp.cstug.cz/pub/tex/local/cstug/thanh/pdftex/snapshots/
%  Unpack the latest .tgz file:
%      > tar xzf pdftex-20010804.tgz
%  Enable encryption support:
%      src> cd src/texk/web2c/pdftexdir
%      src/texk/web2c/pdftexdir> ln -fs pdfcrypt-full.c pdfcrypt.c
%  Compile:
%      src/texk/web2c/pdftexdir> cd ../../..
%      src> ./configure
%      src> cd texk/web2c
%      src/texk/web2c> make pdftexbin
%  At last install the binary and pool files.
%\end{verbatim}
%
% \StopEventually{
% }
%
% \section{Implementation}
%
%    \begin{macrocode}
%<*package>
%    \end{macrocode}
%
% \subsection{Help macros for \plainTeX}
%
%    \begin{macrocode}
\expandafter\edef\csname pc@endinput\endcsname{%
  \catcode`\noexpand\@=\the\catcode`\@\relax
  \noexpand\endinput
}
\catcode`\@=11 %

\expandafter\ifx\csname @firstoftwo\endcsname\relax
  \long\def\@firstoftwo#1#2{#1}%
\fi

\expandafter\ifx\csname @secondoftwo\endcsname\relax
  \long\def\@secondoftwo#1#2{#2}%
\fi

\expandafter\ifx\csname @ifundefined\endcsname\relax
  \def\@ifundefined#1{%
    \expandafter\ifx\csname #1\endcsname\relax
      \expandafter\@firstoftwo
    \else
      \expandafter\@secondoftwo
    \fi
  }%
\fi

\@ifundefined{@ifnextchar}{%
  \long\def\@ifnextchar#1#2#3{%
    \let\reserved@d=#1%
    \def\reserved@a{#2}%
    \def\reserved@b{#3}%
    \futurelet\@let@token\@ifnch
  }%
  \def\@ifnch{%
    \ifx\@let@token\@sptoken
      \let\reserved@c\@xifnch
    \else
      \ifx\@let@token\reserved@d
        \let\reserved@c\reserved@a
      \else
        \let\reserved@c\reserved@b
      \fi
    \fi
    \reserved@c
  }%
  \begingroup
    \def\:{\global\let\@sptoken= }%
    \: %
    \def\:{\@xifnch}%
    \expandafter\gdef\: {\futurelet\@let@token\@ifnch}%
  \endgroup
}{}

\@ifundefined{ProvidesPackage}{%
  \def\ProvidesPackage#1{%
    \@ifnextchar[{\pc@ProvidesPackage{#1}}%
                 {\pc@ProvidesPackage{#1}[]}%
  }%
  \def\pc@ProvidesPackage#1[#2]{%
    \immediate\write-1{Package: #1 #2}%
  }%
}{}

\begingroup\expandafter\expandafter\expandafter\endgroup
\expandafter\ifx\csname RequirePackage\endcsname\relax
  \input infwarerr.sty\relax
\else
  \RequirePackage{infwarerr}%
\fi

\@ifundefined{@gobble}{%
  \long\def\@gobble#1{}%
}{}

\@ifundefined{@empty}{%
  \def\@empty{}%
}{}
%    \end{macrocode}
%
% \subsection{Package Identification and checks}
%
%    \begin{macrocode}
\ProvidesPackage{pdfcrypt}%
  [2016/05/16 v1.1 Allows the setting of PDF encryption (HO)]%
%    \end{macrocode}
%
%    \begin{macrocode}
\@ifundefined{pdfcryptsetup}{%
  \let\pdfcryptsetup\@gobble
}{%
  \@PackageWarningNoLine{pdfcrypt}{Package pdfcrypt is already loaded}%
  \pc@endinput
}
%    \end{macrocode}
%    Support for \cs{nopdfcrypt}.
%    \begin{macrocode}
\newif\ifpc@nopdfcrypt
\ifx Y\nopdfcrypt
  \@PackageWarningNoLine{pdfcrypt}{%
    Encryption disabled by \string \nopdfcrypt\space request%
  }%
  \global\pc@nopdfcrypttrue
\fi
%    \end{macrocode}
%
% \subsection{Driver detection}
%
%    \begin{macrocode}
\let\pc@driver\@empty
\begingroup
  % pdfTeX detection
  \@ifundefined{pdfoutput}{%
  }{%
    \ifcase\pdfoutput
    \else
      \@ifundefined{pdfcrypt}{%
        \@PackageError{pdfcrypt}{%
          PDF encryption is not supported with this pdfTeX%
        }{%
          Encryption support was added in 0.14h (2001/03/10)\MessageBreak
          and removed in 1.10a (2003/01/16).%
        }%
        \endgroup
        \pc@endinput
      }{%
        \gdef\pc@driver{pdftex}%
      }%
    \fi
  }%
  % VTeX detection
  \@ifundefined{OpMode}{%
  }{%
    \ifnum\OpMode=1 %
      \ifnum\@ifundefined{VTeXversion}0\VTeXversion<735 %
        \@PackageError{pdfcrypt}{%
          PDF encryption is not supported with this VTeX%
        }{%
          You need VTeX 7.35 or higher.%
        }%
        \endgroup
        \pc@endinput
      \else
        \gdef\pc@driver{vtex}%
      \fi
    \fi
  }%
\endgroup
%    \end{macrocode}
%
% \subsection{Load package \xpackage{keyval}}
%
%    \begin{macrocode}
\@ifundefined{@makeother}{%
  \def\@makeother#1{\catcode`#1=12\relax}%
}{}

\@ifundefined{g@addto@macro}{%
  \long\def\g@addto@macro#1#2{%
    \begingroup
      \toks@\expandafter{#1#2}%
      \xdef#1{\the\toks@}%
    \endgroup
  }%
}{}

\@ifundefined{@namedef}{%
  \def\@namedef#1{\expandafter\def\csname#1\endcsname}%
}{}

\@ifundefined{@nameuse}{%
  \def\@nameuse#1{\csname #1\endcsname}%
}{}

\def\pc@KeyvalRestore{%
  \let\pc@KeyvalRestore\@undefined
}

\let\pcOrg@NeedsTeXFormat\NeedsTeXFormat
\@ifundefined{NeedsTeXFormat}{%
  \def\NeedsTeXFormat#1{}%
  \g@addto@macro\pc@KeyvalRestore{%
    \let\NeedsTeXFormat\pcOrg@NeedsTeXFormat
  }%
}{}

\let\pcOrg@DeclareOption\DeclareOption
\@ifundefined{DeclareOption}{%
  \def\DeclareOption#1#2{#2}%
  \g@addto@macro\pc@KeyvalRestore{%
    \let\DeclareOption\pcOrg@DeclareOption
  }%
}{}

\let\pcOrg@ExecuteOptions\ExecuteOptions
\@ifundefined{ExecuteOptions}{%
  \def\ExecuteOptions#1{}%
  \g@addto@macro\pc@KeyvalRestore{%
    \let\ExecuteOptions\pcOrg@ExecuteOptions
  }%
}{}

\let\pcOrg@ProcessOptions\ProcessOptions
\@ifundefined{ProcessOptions}{%
  \def\ProcessOptions{}%
  \g@addto@macro\pc@KeyvalRestore{%
    \let\ProcessOptions\pcOrg@ProcessOptions
  }%
}{}%

\begingroup\expandafter\expandafter\expandafter\endgroup
\expandafter\ifx\csname RequirePackage\endcsname\relax
  \input keyval.sty\relax
\else
  \RequirePackage{keyval}%
\fi
\pc@KeyvalRestore
%    \end{macrocode}
%
% \subsection{Define options}
%
%    \begin{macrocode}
\@ifundefined{@dblarg}{%
  \long\def\@dblarg#1{\@ifnextchar[{#1}{\@xdblarg{#1}}}%
  \long\def\@xdblarg#1#2{#1[{#2}]{#2}}%
}{}

\newif\ifpc@set
\newif\ifpc@print
\newif\ifpc@copy
\newif\ifpc@edit
\newif\ifpc@annotate
\newif\ifpc@debug
\let\pc@owner\@empty
\let\pc@user\@empty

% default: allow all
\pc@printtrue
\pc@copytrue
\pc@edittrue
\pc@annotatetrue

\edef\pc@temp{\catcode`\noexpand\"=\the\catcode`\"\relax}
\@makeother\"
\def\pc@set{%
  \@PackageInfo{pdfcrypt}{%
    \ifpc@debug
      \ifx\pc@owner\@empty
        No owner password%
      \else
        Owner password: `\pc@owner'%
      \fi
      \MessageBreak
      \ifx\pc@user\@empty
        No user password%
      \else
        User password: `\pc@user'%
      \fi
      \MessageBreak
      Flags: %
      \ifpc@print   \else no\fi print, %
      \ifpc@copy    \else no\fi copy, %
      \ifpc@edit    \else no\fi edit, %
      \ifpc@annotate\else no\fi annotate%
      \MessageBreak
    \fi
    \ifpc@nopdfcrypt
      Encryption is disabled by `\string\nopdfcrypt'%
    \else
      Encryption is set for `\pc@driver'%
    \fi
  }%
  \ifpc@nopdfcrypt
  \else
    \@ifundefined{pc@set@\pc@driver}{%
      \ifx\pc@driver\@empty
        \@PackageError{pdfcrypt}{No driver for encryption %
          support found}\@ehc
      \else
        \@PackageError{pdfcrypt}{Cannot set encryption for %
          unknown driver `\pc@driver'}\@ehc
      \fi
    }{%
      \@nameuse{pc@set@\pc@driver}%
    }%
  \fi
}
\def\pc@set@pdftex{%
  \ifnum\pdftexversion<100 %
    \pc@set@pdftexold
  \else
    \pc@set@pdftexnew
  \fi
}
\def\pc@set@pdftexold{%
  \pdfcrypt{%
    owner "\pc@owner" %
    user "\pc@user" %
    \ifpc@print   \else no\fi print %
    \ifpc@copy    \else no\fi copy %
    \ifpc@edit    \else no\fi edit %
    \ifpc@annotate\else no\fi annotate%
  }%
}
\def\pc@set@pdftexnew{%
  \pdfcrypt
    owner{\pc@owner}%
    user{\pc@user}%
    \ifpc@print   \else no\fi print %
    \ifpc@copy    \else no\fi copy %
    \ifpc@edit    \else no\fi edit %
    \ifpc@annotate\else no\fi annotate%
  \relax
}
\def\pc@set@vtex{%
  \immediate\special{!security %
    O=\pc@MakeVTeXString\pc@owner,%
    U=\pc@MakeVTeXString\pc@user,%
    P\ifpc@print +\else -\fi,%
    C\ifpc@copy +\else -\fi,%
    M\ifpc@edit +\else -\fi,%
    A\ifpc@annotate +\else -\fi
  }%
}
\def\pc@MakeVTeXString#1{%
  "\expandafter\pc@@MakeVTeXString#1"\@nil"%
}
\def\pc@@MakeVTeXString#1"#2\@nil{%
  #1%
  \ifx\\#2\\%
  \else
    ""%
    \@ReturnAfterFi{%
      \pc@@MakeVTeXString#2\@nil
    }%
  \fi
}
\long\def\@ReturnAfterFi#1\fi{\fi#1}
\pc@temp

\begingroup
  \catcode`\ =12 \gdef\pc@spaceother{ }\catcode`\ =10\relax
  \catcode`\|=0 %
  \catcode`\\=12 %
  |gdef|pc@DefString#1#2{%
    |def#1{#2}%
    |edef#1{|expandafter|strip@prefix|meaning#1}%
    |edef#1{|expandafter|pc@SpaceToOther#1 |@nil}%
    |edef#1{|expandafter|pc@EscapeRemove#1|@empty\|@empty|@nil}%
  }%
  |gdef|pc@EscapeRemove#1\#2#3|@nil{%
    #1#2%
    |ifx|\#3|\%%
    |else
      |@ReturnAfterFi{%
        |pc@EscapeRemove#3|@nil
      }%
    |fi
  }%
|endgroup
\def\pc@SpaceToOther#1 #2\@nil{%
  #1%
  \ifx\\#2\\%
  \else
    \pc@spaceother
    \@ReturnAfterFi{%
      \pc@SpaceToOther#2\@nil
    }%
  \fi
}

\def\pc@boolkey{\@dblarg\pc@@boolkey}
\def\pc@@boolkey[#1]#2#3{%
  \lowercase{\def\pc@temp{#3}}%
  \ifx\pc@temp\@empty
    \let\pc@temp\pc@true
  \fi
  \ifx\pc@temp\pc@true
  \else
    \ifx\pc@temp\pc@false
    \else
      \let\pc@temp\relax
    \fi
  \fi
  \ifx\pc@temp\relax
    \@PackageWarning{pdfcrypt}{%
      Unexpected value \string`#3\string' of %
      option \string`#2\string'\MessageBreak
      instead of %
      \string`true\string' or \string`false\string'%
    }%
  \else
    \csname pc@#2\pc@temp\endcsname
  \fi
}
\def\pc@true{true}
\def\pc@false{false}

\define@key{pc}{set}[true]{%
  \pc@boolkey{set}{#1}%
}
\define@key{pc}{pdftex}[]{%
  \def\pc@driver{pdftex}%
}
\define@key{pc}{vtex}[]{%
  \def\pc@driver{vtex}%
}
\define@key{pc}{print}[true]{%
  \pc@boolkey{print}{#1}%
}
\define@key{pc}{copy}[true]{%
  \pc@boolkey{copy}{#1}%
}
\define@key{pc}{edit}[true]{%
  \pc@boolkey{edit}{#1}%
}
\define@key{pc}{annotate}[true]{%
  \pc@boolkey{annotate}{#1}%
}
\define@key{pc}{all}[]{%
  \pc@boolkey{print}{true}%
  \pc@boolkey{copy}{true}%
  \pc@boolkey{edit}{true}%
  \pc@boolkey{annotate}{true}%
}
\define@key{pc}{none}[]{%
  \pc@boolkey{print}{false}%
  \pc@boolkey{copy}{false}%
  \pc@boolkey{edit}{false}%
  \pc@boolkey{annotate}{false}%
}

\define@key{pc}{owner}{%
  \pc@DefString\pc@owner{#1}%
}
\define@key{pc}{user}{%
  \pc@DefString\pc@user{#1}%
}
\define@key{pc}{debug}[true]{%
  \pc@boolkey{debug}{#1}%
}

\def\pdfcryptsetup#1{%
  \setkeys{pc}{#1}%
  \ifpc@set
    \pc@set
    \global\let\pc@set\relax
    \gdef\pdfcryptsetup##1{%
      \@PackageWarning{pdfcrypt}{%
        Encryption options are already set\MessageBreak
        new values are ignored%
      }%
    }%
  \fi
}
\begingroup\expandafter\expandafter\expandafter\endgroup
\expandafter\ifx\csname @onlypreamble\endcsname\relax
\else
  \@onlypreamble\pdfcryptsetup
\fi
%    \end{macrocode}
%
% \subsection{support of configuration file}
%    \begin{macrocode}
\begingroup\expandafter\expandafter\expandafter\endgroup
\expandafter\ifx\csname InputIfFileExists\endcsname\relax
  \@PackageInfo{pdfcrypt}{%
    Configuration file pdfcrypt.cfg not supported.%
  }%
\else
  \let\pc@ExecuteOptions\ExecuteOptions
  \InputIfFileExists{pdfcrypt.cfg}{}{}%
  \let\ExecuteOptions\pc@ExecuteOptions
\fi
%    \end{macrocode}
%
% \subsection{Package options}
%
%    Plain format does not know package options.
%    \begin{macrocode}
\begingroup\expandafter\expandafter\expandafter\endgroup
\expandafter\ifx\csname @classoptionslist\endcsname\relax
  \expandafter\pc@endinput
\fi
%    \end{macrocode}
%    Process global and local options.
%    \begin{macrocode}
\def\pc@ProcessOptionsWithKV{%
  \let\pc@temp\@empty
  \@for\CurrentOption:=\@classoptionslist\do{%
    \@ifundefined{KV@pc@\CurrentOption}{}{%
      \edef\pc@temp{\pc@temp,\CurrentOption,}%
      \@expandtwoargs\@removeelement\CurrentOption
        \@unusedoptionlist\@unusedoptionlist
    }%
  }%
  \edef\pc@temp{%
    \noexpand\pdfcryptsetup{%
      \pc@temp\@ptionlist{\@currname.\@currext}%
    }%
  }%
  \pc@temp
}
\pc@ProcessOptionsWithKV
\AtEndOfPackage{\let\@unprocessedoptions\relax}
\AtBeginDocument{\pc@set}

\pc@endinput
%    \end{macrocode}
%
%    \begin{macrocode}
%</package>
%    \end{macrocode}
%
% \section{Installation}
%
% \subsection{Download}
%
% \paragraph{Package.} This package is available on
% CTAN\footnote{\CTANpkg{pdfcrypt}}:
% \begin{description}
% \item[\CTAN{macros/latex/contrib/oberdiek/pdfcrypt.dtx}] The source file.
% \item[\CTAN{macros/latex/contrib/oberdiek/pdfcrypt.pdf}] Documentation.
% \end{description}
%
%
% \paragraph{Bundle.} All the packages of the bundle `oberdiek'
% are also available in a TDS compliant ZIP archive. There
% the packages are already unpacked and the documentation files
% are generated. The files and directories obey the TDS standard.
% \begin{description}
% \item[\CTANinstall{install/macros/latex/contrib/oberdiek.tds.zip}]
% \end{description}
% \emph{TDS} refers to the standard ``A Directory Structure
% for \TeX\ Files'' (\CTANpkg{tds}). Directories
% with \xfile{texmf} in their name are usually organized this way.
%
% \subsection{Bundle installation}
%
% \paragraph{Unpacking.} Unpack the \xfile{oberdiek.tds.zip} in the
% TDS tree (also known as \xfile{texmf} tree) of your choice.
% Example (linux):
% \begin{quote}
%   |unzip oberdiek.tds.zip -d ~/texmf|
% \end{quote}
%
% \subsection{Package installation}
%
% \paragraph{Unpacking.} The \xfile{.dtx} file is a self-extracting
% \docstrip\ archive. The files are extracted by running the
% \xfile{.dtx} through \plainTeX:
% \begin{quote}
%   \verb|tex pdfcrypt.dtx|
% \end{quote}
%
% \paragraph{TDS.} Now the different files must be moved into
% the different directories in your installation TDS tree
% (also known as \xfile{texmf} tree):
% \begin{quote}
% \def\t{^^A
% \begin{tabular}{@{}>{\ttfamily}l@{ $\rightarrow$ }>{\ttfamily}l@{}}
%   pdfcrypt.sty & tex/generic/oberdiek/pdfcrypt.sty\\
%   pdfcrypt.pdf & doc/latex/oberdiek/pdfcrypt.pdf\\
%   pdfcrypt.dtx & source/latex/oberdiek/pdfcrypt.dtx\\
% \end{tabular}^^A
% }^^A
% \sbox0{\t}^^A
% \ifdim\wd0>\linewidth
%   \begingroup
%     \advance\linewidth by\leftmargin
%     \advance\linewidth by\rightmargin
%   \edef\x{\endgroup
%     \def\noexpand\lw{\the\linewidth}^^A
%   }\x
%   \def\lwbox{^^A
%     \leavevmode
%     \hbox to \linewidth{^^A
%       \kern-\leftmargin\relax
%       \hss
%       \usebox0
%       \hss
%       \kern-\rightmargin\relax
%     }^^A
%   }^^A
%   \ifdim\wd0>\lw
%     \sbox0{\small\t}^^A
%     \ifdim\wd0>\linewidth
%       \ifdim\wd0>\lw
%         \sbox0{\footnotesize\t}^^A
%         \ifdim\wd0>\linewidth
%           \ifdim\wd0>\lw
%             \sbox0{\scriptsize\t}^^A
%             \ifdim\wd0>\linewidth
%               \ifdim\wd0>\lw
%                 \sbox0{\tiny\t}^^A
%                 \ifdim\wd0>\linewidth
%                   \lwbox
%                 \else
%                   \usebox0
%                 \fi
%               \else
%                 \lwbox
%               \fi
%             \else
%               \usebox0
%             \fi
%           \else
%             \lwbox
%           \fi
%         \else
%           \usebox0
%         \fi
%       \else
%         \lwbox
%       \fi
%     \else
%       \usebox0
%     \fi
%   \else
%     \lwbox
%   \fi
% \else
%   \usebox0
% \fi
% \end{quote}
% If you have a \xfile{docstrip.cfg} that configures and enables \docstrip's
% TDS installing feature, then some files can already be in the right
% place, see the documentation of \docstrip.
%
% \subsection{Refresh file name databases}
%
% If your \TeX~distribution
% (\TeX\,Live, \mikTeX, \dots) relies on file name databases, you must refresh
% these. For example, \TeX\,Live\ users run \verb|texhash| or
% \verb|mktexlsr|.
%
% \subsection{Some details for the interested}
%
% \paragraph{Unpacking with \LaTeX.}
% The \xfile{.dtx} chooses its action depending on the format:
% \begin{description}
% \item[\plainTeX:] Run \docstrip\ and extract the files.
% \item[\LaTeX:] Generate the documentation.
% \end{description}
% If you insist on using \LaTeX\ for \docstrip\ (really,
% \docstrip\ does not need \LaTeX), then inform the autodetect routine
% about your intention:
% \begin{quote}
%   \verb|latex \let\install=y\input{pdfcrypt.dtx}|
% \end{quote}
% Do not forget to quote the argument according to the demands
% of your shell.
%
% \paragraph{Generating the documentation.}
% You can use both the \xfile{.dtx} or the \xfile{.drv} to generate
% the documentation. The process can be configured by the
% configuration file \xfile{ltxdoc.cfg}. For instance, put this
% line into this file, if you want to have A4 as paper format:
% \begin{quote}
%   \verb|\PassOptionsToClass{a4paper}{article}|
% \end{quote}
% An example follows how to generate the
% documentation with pdf\LaTeX:
% \begin{quote}
%\begin{verbatim}
%pdflatex pdfcrypt.dtx
%makeindex -s gind.ist pdfcrypt.idx
%pdflatex pdfcrypt.dtx
%makeindex -s gind.ist pdfcrypt.idx
%pdflatex pdfcrypt.dtx
%\end{verbatim}
% \end{quote}
%
% \begin{History}
%   \begin{Version}{2001/04/02 v0.1}
%   \item
%     First public version, published in the pdftex
%     mailing list for testing with pdfcrypt-20010331
%   \end{Version}
%   \begin{Version}{2001/07/19 v0.2}
%   \item
%     Default: all allowed.
%   \item
%     Support for \VTeX\ added.
%   \end{Version}
%   \begin{Version}{2001/07/19 v0.3}
%   \item
%     Bug fix: \VTeX\ letter for edit is M (modify).
%   \end{Version}
%   \begin{Version}{2001/07/19 v0.4}
%   \item
%     Bug fix: \cs{VTeXversion} is correct after
%     regenerating the format file.
%   \end{Version}
%   \begin{Version}{2001/08/05 v0.5}
%   \item
%     Syntax change in \pdfTeX\ 1.00a.
%   \end{Version}
%   \begin{Version}{2001/08/09 v0.6}
%   \item
%     Support of special characters:\\
%     input: |\{|, |\}|, |\\| for |{|, |}|, |\|\\
%     output: |"| in \VTeX
%   \item
%     Option debug added.
%   \end{Version}
%   \begin{Version}{2001/10/28 v0.7}
%   \item
%     Plain compatibility.
%   \item
%     \cs{nopdfcrypt} added.
%   \item
%     Typos corrected.
%   \end{Version}
%   \begin{Version}{2006/02/20 v0.8}
%   \item
%     Obsolete remarks for \pdfTeX.
%   \item
%     DTX framework.
%   \item
%     LPPL 1.3
%   \end{Version}
%   \begin{Version}{2007/04/11 v0.9}
%   \item
%     Line ends sanitized.
%   \end{Version}
%   \begin{Version}{2007/04/26 v1.0}
%   \item
%     Use of package \xpackage{infwarerr}.
%   \end{Version}
%   \begin{Version}{2016/05/16 v1.1}
%   \item
%     Documentation updates.
%   \end{Version}
% \end{History}
%
% \PrintIndex
%
% \Finale
\endinput

%        (quote the arguments according to the demands of your shell)
%
% Documentation:
%    (a) If pdfcrypt.drv is present:
%           latex pdfcrypt.drv
%    (b) Without pdfcrypt.drv:
%           latex pdfcrypt.dtx; ...
%    The class ltxdoc loads the configuration file ltxdoc.cfg
%    if available. Here you can specify further options, e.g.
%    use A4 as paper format:
%       \PassOptionsToClass{a4paper}{article}
%
%    Programm calls to get the documentation (example):
%       pdflatex pdfcrypt.dtx
%       makeindex -s gind.ist pdfcrypt.idx
%       pdflatex pdfcrypt.dtx
%       makeindex -s gind.ist pdfcrypt.idx
%       pdflatex pdfcrypt.dtx
%
% Installation:
%    TDS:tex/generic/oberdiek/pdfcrypt.sty
%    TDS:doc/latex/oberdiek/pdfcrypt.pdf
%    TDS:source/latex/oberdiek/pdfcrypt.dtx
%
%<*ignore>
\begingroup
  \catcode123=1 %
  \catcode125=2 %
  \def\x{LaTeX2e}%
\expandafter\endgroup
\ifcase 0\ifx\install y1\fi\expandafter
         \ifx\csname processbatchFile\endcsname\relax\else1\fi
         \ifx\fmtname\x\else 1\fi\relax
\else\csname fi\endcsname
%</ignore>
%<*install>
\input docstrip.tex
\Msg{************************************************************************}
\Msg{* Installation}
\Msg{* Package: pdfcrypt 2016/05/16 v1.1 Allows the setting of PDF encryption (HO)}
\Msg{************************************************************************}

\keepsilent
\askforoverwritefalse

\let\MetaPrefix\relax
\preamble

This is a generated file.

Project: pdfcrypt
Version: 2016/05/16 v1.1

Copyright (C)
   2001, 2006, 2007 Heiko Oberdiek
   2016-2019 Oberdiek Package Support Group

This work may be distributed and/or modified under the
conditions of the LaTeX Project Public License, either
version 1.3c of this license or (at your option) any later
version. This version of this license is in
   https://www.latex-project.org/lppl/lppl-1-3c.txt
and the latest version of this license is in
   https://www.latex-project.org/lppl.txt
and version 1.3 or later is part of all distributions of
LaTeX version 2005/12/01 or later.

This work has the LPPL maintenance status "maintained".

The Current Maintainers of this work are
Heiko Oberdiek and the Oberdiek Package Support Group
https://github.com/ho-tex/oberdiek/issues


The Base Interpreter refers to any `TeX-Format',
because some files are installed in TDS:tex/generic//.

This work consists of the main source file pdfcrypt.dtx
and the derived files
   pdfcrypt.sty, pdfcrypt.pdf, pdfcrypt.ins, pdfcrypt.drv.

\endpreamble
\let\MetaPrefix\DoubleperCent

\generate{%
  \file{pdfcrypt.ins}{\from{pdfcrypt.dtx}{install}}%
  \file{pdfcrypt.drv}{\from{pdfcrypt.dtx}{driver}}%
  \usedir{tex/generic/oberdiek}%
  \file{pdfcrypt.sty}{\from{pdfcrypt.dtx}{package}}%
  \nopreamble
  \nopostamble
%  \usedir{source/latex/oberdiek/catalogue}%
%  \file{pdfcrypt.xml}{\from{pdfcrypt.dtx}{catalogue}}%
}

\catcode32=13\relax% active space
\let =\space%
\Msg{************************************************************************}
\Msg{*}
\Msg{* To finish the installation you have to move the following}
\Msg{* file into a directory searched by TeX:}
\Msg{*}
\Msg{*     pdfcrypt.sty}
\Msg{*}
\Msg{* To produce the documentation run the file `pdfcrypt.drv'}
\Msg{* through LaTeX.}
\Msg{*}
\Msg{* Happy TeXing!}
\Msg{*}
\Msg{************************************************************************}

\endbatchfile
%</install>
%<*ignore>
\fi
%</ignore>
%<*driver>
\NeedsTeXFormat{LaTeX2e}
\ProvidesFile{pdfcrypt.drv}%
  [2016/05/16 v1.1 Allows the setting of PDF encryption (HO)]%
\documentclass{ltxdoc}
\usepackage{holtxdoc}[2011/11/22]
\begin{document}
  \DocInput{pdfcrypt.dtx}%
\end{document}
%</driver>
% \fi
%
%
% \CharacterTable
%  {Upper-case    \A\B\C\D\E\F\G\H\I\J\K\L\M\N\O\P\Q\R\S\T\U\V\W\X\Y\Z
%   Lower-case    \a\b\c\d\e\f\g\h\i\j\k\l\m\n\o\p\q\r\s\t\u\v\w\x\y\z
%   Digits        \0\1\2\3\4\5\6\7\8\9
%   Exclamation   \!     Double quote  \"     Hash (number) \#
%   Dollar        \$     Percent       \%     Ampersand     \&
%   Acute accent  \'     Left paren    \(     Right paren   \)
%   Asterisk      \*     Plus          \+     Comma         \,
%   Minus         \-     Point         \.     Solidus       \/
%   Colon         \:     Semicolon     \;     Less than     \<
%   Equals        \=     Greater than  \>     Question mark \?
%   Commercial at \@     Left bracket  \[     Backslash     \\
%   Right bracket \]     Circumflex    \^     Underscore    \_
%   Grave accent  \`     Left brace    \{     Vertical bar  \|
%   Right brace   \}     Tilde         \~}
%
% \GetFileInfo{pdfcrypt.drv}
%
% \title{The \xpackage{pdfcrypt} package}
% \date{2016/05/16 v1.1}
% \author{Heiko Oberdiek\thanks
% {Please report any issues at \url{https://github.com/ho-tex/oberdiek/issues}}}
%
% \maketitle
%
% \begin{abstract}
% This package supports the setting of pdf encryption options
% for \VTeX\ and some older versions of \pdfTeX.
% \end{abstract}
%
% \tableofcontents
%
% \section{Documentation}
%
% This package allows the setting of pdf encryption
% options for
% \begin{itemize}
% \item
%   \VTeX, since version 7.35, \url{http://www.micropress-inc.com/}.
% \item
%   \pdfTeX, patched by
%   \NameEmail{Ricardo Sanchez Carmenes}{carmenes@bioquimica.uniovi.es}.\\
%   Some supported versions are:
%   \begin{quote}
%     \pdfTeX-1.00a,\\
%     \pdfTeX-0.14h-pdfcrypt-20010310,\\
%     pdfcrypt-20010331
%   \end{quote}
% \end{itemize}
%
% \noindent
% \textcolor{red}{Note:} Since \pdfTeX-1.10a (2003-01-16) encryption
% support was dropped! Thus the package is now obsolete for recent
% \pdfTeX\ versions.
%
% \subsection{Alternatives}
%
% There are free alternatives, programs that can
% be used for postprocessing the pdf file:
% \begin{itemize}
% \item \textsf{pdftk}\\
%   \url{http://www.accesspdf.com/pdftk/}
% \item \textsf{Multivalent}\\
%   \url{http://multivalent.sourceforge.net/}
% \item \textsf{PDFBox}\\
%   \url{http://www.pdfbox.org/}
% \item \textsf{PDFTrans}\\
%   \url{http://maddingue.free.fr/softwares/pdftrans.html}
% \item \dots
% \end{itemize}
%
% \subsection{Usage}
%
% The options can be set as package options or
% with the command \cs{pdfcryptsetup}:
% \begin{quote}
%   |owner=|\meta{owner password}\\
%   |user=|\meta{user password}\\
%   |print|, |copy|, |edit|, |annotate=true|, |false|\\
%   |all|, |none|\\
%   |debug|
% \end{quote}
% The encryption is set at |\begin{document}| by default,
% but this can be forced for an earlier time
% using the option |set|.
% Example:
% \begin{quote}
%   |\usepackage[owner=Donald,user=Knuth,print=false]{pdfcrypt}|\\
%   \dots\\
%   |\begin{document}|
% \end{quote}
% or
% \begin{quote}
%   |\usepackage{pdfcrypt}|\\
%   |\pdfcryptsetup{owner=Donald,user=Knut}|\\
%   |\pdfcryptsetup{print=false,set}|\\
%   \dots\\
%   |\begin{document}|
% \end{quote}
%
% \subsubsection{Special characters}
%
% The characters `|{|', `|}|', and `|\|' have to be
% escaped with `|\|'. Then the string should not
% be specified as package option but with
% the macro \cs{pdfcryptsetup}, eg:
% \begin{quote}
%   |\pdfcryptsetup{user={\{\\Hello\}},print=false}|
% \end{quote}
% The password here is ``|{\Hello}|''.
% Active characters can be used and are not expanded.
% Macros except for |\{|, |\}|, and |\\| should not be used
% and are not expanded.
%
% \subsubsection{\cs{nopdfcrypt}}
%
% Often the whole process of pdf generation includes
% several \TeX\ compilations and perhaps other
% postprocessing steps. Therefore a feature would
% be useful to disable the encryption stuff in order
% to speed up the first compilations.
% Therefore package \xpackage{pdfcrypt} will look for
% the macro \cs{nopdfcrypt}. If it has the meaning of the
% letter `|Y|', the package will be disabled.
% Example:
% \begin{quote}
%   |pdflatex '\let\nopdfcrypt=Y\input myfile'|\\
%   |thumbpdf myfile|\\
%   |pdfelatex myfile|
% \end{quote}
%
% \subsubsection{Configuration file}
%
% If the file \xfile{pdfcrypt.cfg} exists it is loaded
% and can be used for common settings.
%
% \subsubsection{Support for \plainTeX}
%
% The package can also be used with \plainTeX. It defines the
% missing \LaTeX\ macros and loads package \xpackage{keyval}.
% The configuration file is not supported and automatically
% loaded.
%
% \subsection{Hints for \pdfTeX}
%
% This section is out of date, because the encryption support
% was removed from \pdfTeX. Therefore this hints are of
% historical interest only.
%
%\begin{verbatim}
%* There are packages such as thumbpdf that ship out
%  some stuff (\immediate\pdfobj{...}). In these cases
%  \pdfcrypt will be ignored without warning or error
%  message. Therefore the package pdfcrypt should be
%  loaded before and option "set" should be used in
%  order to force the call of \pdfcrypt, for example:
%     \usepackage{pdfcrypt}
%     \pdfcryptsetup{..., set}
%     ...
%     \usepackage{thumbpdf}
%
%* Compiling pdfTeX (eg. version 1.00a-pretest-20010804):
%  pdfTeX versions are available at
%      ftp://ftp.cstug.cz/pub/tex/local/cstug/thanh/pdftex/snapshots/
%  Unpack the latest .tgz file:
%      > tar xzf pdftex-20010804.tgz
%  Enable encryption support:
%      src> cd src/texk/web2c/pdftexdir
%      src/texk/web2c/pdftexdir> ln -fs pdfcrypt-full.c pdfcrypt.c
%  Compile:
%      src/texk/web2c/pdftexdir> cd ../../..
%      src> ./configure
%      src> cd texk/web2c
%      src/texk/web2c> make pdftexbin
%  At last install the binary and pool files.
%\end{verbatim}
%
% \StopEventually{
% }
%
% \section{Implementation}
%
%    \begin{macrocode}
%<*package>
%    \end{macrocode}
%
% \subsection{Help macros for \plainTeX}
%
%    \begin{macrocode}
\expandafter\edef\csname pc@endinput\endcsname{%
  \catcode`\noexpand\@=\the\catcode`\@\relax
  \noexpand\endinput
}
\catcode`\@=11 %

\expandafter\ifx\csname @firstoftwo\endcsname\relax
  \long\def\@firstoftwo#1#2{#1}%
\fi

\expandafter\ifx\csname @secondoftwo\endcsname\relax
  \long\def\@secondoftwo#1#2{#2}%
\fi

\expandafter\ifx\csname @ifundefined\endcsname\relax
  \def\@ifundefined#1{%
    \expandafter\ifx\csname #1\endcsname\relax
      \expandafter\@firstoftwo
    \else
      \expandafter\@secondoftwo
    \fi
  }%
\fi

\@ifundefined{@ifnextchar}{%
  \long\def\@ifnextchar#1#2#3{%
    \let\reserved@d=#1%
    \def\reserved@a{#2}%
    \def\reserved@b{#3}%
    \futurelet\@let@token\@ifnch
  }%
  \def\@ifnch{%
    \ifx\@let@token\@sptoken
      \let\reserved@c\@xifnch
    \else
      \ifx\@let@token\reserved@d
        \let\reserved@c\reserved@a
      \else
        \let\reserved@c\reserved@b
      \fi
    \fi
    \reserved@c
  }%
  \begingroup
    \def\:{\global\let\@sptoken= }%
    \: %
    \def\:{\@xifnch}%
    \expandafter\gdef\: {\futurelet\@let@token\@ifnch}%
  \endgroup
}{}

\@ifundefined{ProvidesPackage}{%
  \def\ProvidesPackage#1{%
    \@ifnextchar[{\pc@ProvidesPackage{#1}}%
                 {\pc@ProvidesPackage{#1}[]}%
  }%
  \def\pc@ProvidesPackage#1[#2]{%
    \immediate\write-1{Package: #1 #2}%
  }%
}{}

\begingroup\expandafter\expandafter\expandafter\endgroup
\expandafter\ifx\csname RequirePackage\endcsname\relax
  \input infwarerr.sty\relax
\else
  \RequirePackage{infwarerr}%
\fi

\@ifundefined{@gobble}{%
  \long\def\@gobble#1{}%
}{}

\@ifundefined{@empty}{%
  \def\@empty{}%
}{}
%    \end{macrocode}
%
% \subsection{Package Identification and checks}
%
%    \begin{macrocode}
\ProvidesPackage{pdfcrypt}%
  [2016/05/16 v1.1 Allows the setting of PDF encryption (HO)]%
%    \end{macrocode}
%
%    \begin{macrocode}
\@ifundefined{pdfcryptsetup}{%
  \let\pdfcryptsetup\@gobble
}{%
  \@PackageWarningNoLine{pdfcrypt}{Package pdfcrypt is already loaded}%
  \pc@endinput
}
%    \end{macrocode}
%    Support for \cs{nopdfcrypt}.
%    \begin{macrocode}
\newif\ifpc@nopdfcrypt
\ifx Y\nopdfcrypt
  \@PackageWarningNoLine{pdfcrypt}{%
    Encryption disabled by \string \nopdfcrypt\space request%
  }%
  \global\pc@nopdfcrypttrue
\fi
%    \end{macrocode}
%
% \subsection{Driver detection}
%
%    \begin{macrocode}
\let\pc@driver\@empty
\begingroup
  % pdfTeX detection
  \@ifundefined{pdfoutput}{%
  }{%
    \ifcase\pdfoutput
    \else
      \@ifundefined{pdfcrypt}{%
        \@PackageError{pdfcrypt}{%
          PDF encryption is not supported with this pdfTeX%
        }{%
          Encryption support was added in 0.14h (2001/03/10)\MessageBreak
          and removed in 1.10a (2003/01/16).%
        }%
        \endgroup
        \pc@endinput
      }{%
        \gdef\pc@driver{pdftex}%
      }%
    \fi
  }%
  % VTeX detection
  \@ifundefined{OpMode}{%
  }{%
    \ifnum\OpMode=1 %
      \ifnum\@ifundefined{VTeXversion}0\VTeXversion<735 %
        \@PackageError{pdfcrypt}{%
          PDF encryption is not supported with this VTeX%
        }{%
          You need VTeX 7.35 or higher.%
        }%
        \endgroup
        \pc@endinput
      \else
        \gdef\pc@driver{vtex}%
      \fi
    \fi
  }%
\endgroup
%    \end{macrocode}
%
% \subsection{Load package \xpackage{keyval}}
%
%    \begin{macrocode}
\@ifundefined{@makeother}{%
  \def\@makeother#1{\catcode`#1=12\relax}%
}{}

\@ifundefined{g@addto@macro}{%
  \long\def\g@addto@macro#1#2{%
    \begingroup
      \toks@\expandafter{#1#2}%
      \xdef#1{\the\toks@}%
    \endgroup
  }%
}{}

\@ifundefined{@namedef}{%
  \def\@namedef#1{\expandafter\def\csname#1\endcsname}%
}{}

\@ifundefined{@nameuse}{%
  \def\@nameuse#1{\csname #1\endcsname}%
}{}

\def\pc@KeyvalRestore{%
  \let\pc@KeyvalRestore\@undefined
}

\let\pcOrg@NeedsTeXFormat\NeedsTeXFormat
\@ifundefined{NeedsTeXFormat}{%
  \def\NeedsTeXFormat#1{}%
  \g@addto@macro\pc@KeyvalRestore{%
    \let\NeedsTeXFormat\pcOrg@NeedsTeXFormat
  }%
}{}

\let\pcOrg@DeclareOption\DeclareOption
\@ifundefined{DeclareOption}{%
  \def\DeclareOption#1#2{#2}%
  \g@addto@macro\pc@KeyvalRestore{%
    \let\DeclareOption\pcOrg@DeclareOption
  }%
}{}

\let\pcOrg@ExecuteOptions\ExecuteOptions
\@ifundefined{ExecuteOptions}{%
  \def\ExecuteOptions#1{}%
  \g@addto@macro\pc@KeyvalRestore{%
    \let\ExecuteOptions\pcOrg@ExecuteOptions
  }%
}{}

\let\pcOrg@ProcessOptions\ProcessOptions
\@ifundefined{ProcessOptions}{%
  \def\ProcessOptions{}%
  \g@addto@macro\pc@KeyvalRestore{%
    \let\ProcessOptions\pcOrg@ProcessOptions
  }%
}{}%

\begingroup\expandafter\expandafter\expandafter\endgroup
\expandafter\ifx\csname RequirePackage\endcsname\relax
  \input keyval.sty\relax
\else
  \RequirePackage{keyval}%
\fi
\pc@KeyvalRestore
%    \end{macrocode}
%
% \subsection{Define options}
%
%    \begin{macrocode}
\@ifundefined{@dblarg}{%
  \long\def\@dblarg#1{\@ifnextchar[{#1}{\@xdblarg{#1}}}%
  \long\def\@xdblarg#1#2{#1[{#2}]{#2}}%
}{}

\newif\ifpc@set
\newif\ifpc@print
\newif\ifpc@copy
\newif\ifpc@edit
\newif\ifpc@annotate
\newif\ifpc@debug
\let\pc@owner\@empty
\let\pc@user\@empty

% default: allow all
\pc@printtrue
\pc@copytrue
\pc@edittrue
\pc@annotatetrue

\edef\pc@temp{\catcode`\noexpand\"=\the\catcode`\"\relax}
\@makeother\"
\def\pc@set{%
  \@PackageInfo{pdfcrypt}{%
    \ifpc@debug
      \ifx\pc@owner\@empty
        No owner password%
      \else
        Owner password: `\pc@owner'%
      \fi
      \MessageBreak
      \ifx\pc@user\@empty
        No user password%
      \else
        User password: `\pc@user'%
      \fi
      \MessageBreak
      Flags: %
      \ifpc@print   \else no\fi print, %
      \ifpc@copy    \else no\fi copy, %
      \ifpc@edit    \else no\fi edit, %
      \ifpc@annotate\else no\fi annotate%
      \MessageBreak
    \fi
    \ifpc@nopdfcrypt
      Encryption is disabled by `\string\nopdfcrypt'%
    \else
      Encryption is set for `\pc@driver'%
    \fi
  }%
  \ifpc@nopdfcrypt
  \else
    \@ifundefined{pc@set@\pc@driver}{%
      \ifx\pc@driver\@empty
        \@PackageError{pdfcrypt}{No driver for encryption %
          support found}\@ehc
      \else
        \@PackageError{pdfcrypt}{Cannot set encryption for %
          unknown driver `\pc@driver'}\@ehc
      \fi
    }{%
      \@nameuse{pc@set@\pc@driver}%
    }%
  \fi
}
\def\pc@set@pdftex{%
  \ifnum\pdftexversion<100 %
    \pc@set@pdftexold
  \else
    \pc@set@pdftexnew
  \fi
}
\def\pc@set@pdftexold{%
  \pdfcrypt{%
    owner "\pc@owner" %
    user "\pc@user" %
    \ifpc@print   \else no\fi print %
    \ifpc@copy    \else no\fi copy %
    \ifpc@edit    \else no\fi edit %
    \ifpc@annotate\else no\fi annotate%
  }%
}
\def\pc@set@pdftexnew{%
  \pdfcrypt
    owner{\pc@owner}%
    user{\pc@user}%
    \ifpc@print   \else no\fi print %
    \ifpc@copy    \else no\fi copy %
    \ifpc@edit    \else no\fi edit %
    \ifpc@annotate\else no\fi annotate%
  \relax
}
\def\pc@set@vtex{%
  \immediate\special{!security %
    O=\pc@MakeVTeXString\pc@owner,%
    U=\pc@MakeVTeXString\pc@user,%
    P\ifpc@print +\else -\fi,%
    C\ifpc@copy +\else -\fi,%
    M\ifpc@edit +\else -\fi,%
    A\ifpc@annotate +\else -\fi
  }%
}
\def\pc@MakeVTeXString#1{%
  "\expandafter\pc@@MakeVTeXString#1"\@nil"%
}
\def\pc@@MakeVTeXString#1"#2\@nil{%
  #1%
  \ifx\\#2\\%
  \else
    ""%
    \@ReturnAfterFi{%
      \pc@@MakeVTeXString#2\@nil
    }%
  \fi
}
\long\def\@ReturnAfterFi#1\fi{\fi#1}
\pc@temp

\begingroup
  \catcode`\ =12 \gdef\pc@spaceother{ }\catcode`\ =10\relax
  \catcode`\|=0 %
  \catcode`\\=12 %
  |gdef|pc@DefString#1#2{%
    |def#1{#2}%
    |edef#1{|expandafter|strip@prefix|meaning#1}%
    |edef#1{|expandafter|pc@SpaceToOther#1 |@nil}%
    |edef#1{|expandafter|pc@EscapeRemove#1|@empty\|@empty|@nil}%
  }%
  |gdef|pc@EscapeRemove#1\#2#3|@nil{%
    #1#2%
    |ifx|\#3|\%%
    |else
      |@ReturnAfterFi{%
        |pc@EscapeRemove#3|@nil
      }%
    |fi
  }%
|endgroup
\def\pc@SpaceToOther#1 #2\@nil{%
  #1%
  \ifx\\#2\\%
  \else
    \pc@spaceother
    \@ReturnAfterFi{%
      \pc@SpaceToOther#2\@nil
    }%
  \fi
}

\def\pc@boolkey{\@dblarg\pc@@boolkey}
\def\pc@@boolkey[#1]#2#3{%
  \lowercase{\def\pc@temp{#3}}%
  \ifx\pc@temp\@empty
    \let\pc@temp\pc@true
  \fi
  \ifx\pc@temp\pc@true
  \else
    \ifx\pc@temp\pc@false
    \else
      \let\pc@temp\relax
    \fi
  \fi
  \ifx\pc@temp\relax
    \@PackageWarning{pdfcrypt}{%
      Unexpected value \string`#3\string' of %
      option \string`#2\string'\MessageBreak
      instead of %
      \string`true\string' or \string`false\string'%
    }%
  \else
    \csname pc@#2\pc@temp\endcsname
  \fi
}
\def\pc@true{true}
\def\pc@false{false}

\define@key{pc}{set}[true]{%
  \pc@boolkey{set}{#1}%
}
\define@key{pc}{pdftex}[]{%
  \def\pc@driver{pdftex}%
}
\define@key{pc}{vtex}[]{%
  \def\pc@driver{vtex}%
}
\define@key{pc}{print}[true]{%
  \pc@boolkey{print}{#1}%
}
\define@key{pc}{copy}[true]{%
  \pc@boolkey{copy}{#1}%
}
\define@key{pc}{edit}[true]{%
  \pc@boolkey{edit}{#1}%
}
\define@key{pc}{annotate}[true]{%
  \pc@boolkey{annotate}{#1}%
}
\define@key{pc}{all}[]{%
  \pc@boolkey{print}{true}%
  \pc@boolkey{copy}{true}%
  \pc@boolkey{edit}{true}%
  \pc@boolkey{annotate}{true}%
}
\define@key{pc}{none}[]{%
  \pc@boolkey{print}{false}%
  \pc@boolkey{copy}{false}%
  \pc@boolkey{edit}{false}%
  \pc@boolkey{annotate}{false}%
}

\define@key{pc}{owner}{%
  \pc@DefString\pc@owner{#1}%
}
\define@key{pc}{user}{%
  \pc@DefString\pc@user{#1}%
}
\define@key{pc}{debug}[true]{%
  \pc@boolkey{debug}{#1}%
}

\def\pdfcryptsetup#1{%
  \setkeys{pc}{#1}%
  \ifpc@set
    \pc@set
    \global\let\pc@set\relax
    \gdef\pdfcryptsetup##1{%
      \@PackageWarning{pdfcrypt}{%
        Encryption options are already set\MessageBreak
        new values are ignored%
      }%
    }%
  \fi
}
\begingroup\expandafter\expandafter\expandafter\endgroup
\expandafter\ifx\csname @onlypreamble\endcsname\relax
\else
  \@onlypreamble\pdfcryptsetup
\fi
%    \end{macrocode}
%
% \subsection{support of configuration file}
%    \begin{macrocode}
\begingroup\expandafter\expandafter\expandafter\endgroup
\expandafter\ifx\csname InputIfFileExists\endcsname\relax
  \@PackageInfo{pdfcrypt}{%
    Configuration file pdfcrypt.cfg not supported.%
  }%
\else
  \let\pc@ExecuteOptions\ExecuteOptions
  \InputIfFileExists{pdfcrypt.cfg}{}{}%
  \let\ExecuteOptions\pc@ExecuteOptions
\fi
%    \end{macrocode}
%
% \subsection{Package options}
%
%    Plain format does not know package options.
%    \begin{macrocode}
\begingroup\expandafter\expandafter\expandafter\endgroup
\expandafter\ifx\csname @classoptionslist\endcsname\relax
  \expandafter\pc@endinput
\fi
%    \end{macrocode}
%    Process global and local options.
%    \begin{macrocode}
\def\pc@ProcessOptionsWithKV{%
  \let\pc@temp\@empty
  \@for\CurrentOption:=\@classoptionslist\do{%
    \@ifundefined{KV@pc@\CurrentOption}{}{%
      \edef\pc@temp{\pc@temp,\CurrentOption,}%
      \@expandtwoargs\@removeelement\CurrentOption
        \@unusedoptionlist\@unusedoptionlist
    }%
  }%
  \edef\pc@temp{%
    \noexpand\pdfcryptsetup{%
      \pc@temp\@ptionlist{\@currname.\@currext}%
    }%
  }%
  \pc@temp
}
\pc@ProcessOptionsWithKV
\AtEndOfPackage{\let\@unprocessedoptions\relax}
\AtBeginDocument{\pc@set}

\pc@endinput
%    \end{macrocode}
%
%    \begin{macrocode}
%</package>
%    \end{macrocode}
%
% \section{Installation}
%
% \subsection{Download}
%
% \paragraph{Package.} This package is available on
% CTAN\footnote{\CTANpkg{pdfcrypt}}:
% \begin{description}
% \item[\CTAN{macros/latex/contrib/oberdiek/pdfcrypt.dtx}] The source file.
% \item[\CTAN{macros/latex/contrib/oberdiek/pdfcrypt.pdf}] Documentation.
% \end{description}
%
%
% \paragraph{Bundle.} All the packages of the bundle `oberdiek'
% are also available in a TDS compliant ZIP archive. There
% the packages are already unpacked and the documentation files
% are generated. The files and directories obey the TDS standard.
% \begin{description}
% \item[\CTANinstall{install/macros/latex/contrib/oberdiek.tds.zip}]
% \end{description}
% \emph{TDS} refers to the standard ``A Directory Structure
% for \TeX\ Files'' (\CTANpkg{tds}). Directories
% with \xfile{texmf} in their name are usually organized this way.
%
% \subsection{Bundle installation}
%
% \paragraph{Unpacking.} Unpack the \xfile{oberdiek.tds.zip} in the
% TDS tree (also known as \xfile{texmf} tree) of your choice.
% Example (linux):
% \begin{quote}
%   |unzip oberdiek.tds.zip -d ~/texmf|
% \end{quote}
%
% \subsection{Package installation}
%
% \paragraph{Unpacking.} The \xfile{.dtx} file is a self-extracting
% \docstrip\ archive. The files are extracted by running the
% \xfile{.dtx} through \plainTeX:
% \begin{quote}
%   \verb|tex pdfcrypt.dtx|
% \end{quote}
%
% \paragraph{TDS.} Now the different files must be moved into
% the different directories in your installation TDS tree
% (also known as \xfile{texmf} tree):
% \begin{quote}
% \def\t{^^A
% \begin{tabular}{@{}>{\ttfamily}l@{ $\rightarrow$ }>{\ttfamily}l@{}}
%   pdfcrypt.sty & tex/generic/oberdiek/pdfcrypt.sty\\
%   pdfcrypt.pdf & doc/latex/oberdiek/pdfcrypt.pdf\\
%   pdfcrypt.dtx & source/latex/oberdiek/pdfcrypt.dtx\\
% \end{tabular}^^A
% }^^A
% \sbox0{\t}^^A
% \ifdim\wd0>\linewidth
%   \begingroup
%     \advance\linewidth by\leftmargin
%     \advance\linewidth by\rightmargin
%   \edef\x{\endgroup
%     \def\noexpand\lw{\the\linewidth}^^A
%   }\x
%   \def\lwbox{^^A
%     \leavevmode
%     \hbox to \linewidth{^^A
%       \kern-\leftmargin\relax
%       \hss
%       \usebox0
%       \hss
%       \kern-\rightmargin\relax
%     }^^A
%   }^^A
%   \ifdim\wd0>\lw
%     \sbox0{\small\t}^^A
%     \ifdim\wd0>\linewidth
%       \ifdim\wd0>\lw
%         \sbox0{\footnotesize\t}^^A
%         \ifdim\wd0>\linewidth
%           \ifdim\wd0>\lw
%             \sbox0{\scriptsize\t}^^A
%             \ifdim\wd0>\linewidth
%               \ifdim\wd0>\lw
%                 \sbox0{\tiny\t}^^A
%                 \ifdim\wd0>\linewidth
%                   \lwbox
%                 \else
%                   \usebox0
%                 \fi
%               \else
%                 \lwbox
%               \fi
%             \else
%               \usebox0
%             \fi
%           \else
%             \lwbox
%           \fi
%         \else
%           \usebox0
%         \fi
%       \else
%         \lwbox
%       \fi
%     \else
%       \usebox0
%     \fi
%   \else
%     \lwbox
%   \fi
% \else
%   \usebox0
% \fi
% \end{quote}
% If you have a \xfile{docstrip.cfg} that configures and enables \docstrip's
% TDS installing feature, then some files can already be in the right
% place, see the documentation of \docstrip.
%
% \subsection{Refresh file name databases}
%
% If your \TeX~distribution
% (\TeX\,Live, \mikTeX, \dots) relies on file name databases, you must refresh
% these. For example, \TeX\,Live\ users run \verb|texhash| or
% \verb|mktexlsr|.
%
% \subsection{Some details for the interested}
%
% \paragraph{Unpacking with \LaTeX.}
% The \xfile{.dtx} chooses its action depending on the format:
% \begin{description}
% \item[\plainTeX:] Run \docstrip\ and extract the files.
% \item[\LaTeX:] Generate the documentation.
% \end{description}
% If you insist on using \LaTeX\ for \docstrip\ (really,
% \docstrip\ does not need \LaTeX), then inform the autodetect routine
% about your intention:
% \begin{quote}
%   \verb|latex \let\install=y% \iffalse meta-comment
%
% File: pdfcrypt.dtx
% Version: 2016/05/16 v1.1
% Info: Allows the setting of PDF encryption
%
% Copyright (C)
%    2001, 2006, 2007 Heiko Oberdiek
%    2016-2019 Oberdiek Package Support Group
%    https://github.com/ho-tex/oberdiek/issues
%
% This work may be distributed and/or modified under the
% conditions of the LaTeX Project Public License, either
% version 1.3c of this license or (at your option) any later
% version. This version of this license is in
%    https://www.latex-project.org/lppl/lppl-1-3c.txt
% and the latest version of this license is in
%    https://www.latex-project.org/lppl.txt
% and version 1.3 or later is part of all distributions of
% LaTeX version 2005/12/01 or later.
%
% This work has the LPPL maintenance status "maintained".
%
% The Current Maintainers of this work are
% Heiko Oberdiek and the Oberdiek Package Support Group
% https://github.com/ho-tex/oberdiek/issues
%
% The Base Interpreter refers to any `TeX-Format',
% because some files are installed in TDS:tex/generic//.
%
% This work consists of the main source file pdfcrypt.dtx
% and the derived files
%    pdfcrypt.sty, pdfcrypt.pdf, pdfcrypt.ins, pdfcrypt.drv.
%
% Distribution:
%    CTAN:macros/latex/contrib/oberdiek/pdfcrypt.dtx
%    CTAN:macros/latex/contrib/oberdiek/pdfcrypt.pdf
%
% Unpacking:
%    (a) If pdfcrypt.ins is present:
%           tex pdfcrypt.ins
%    (b) Without pdfcrypt.ins:
%           tex pdfcrypt.dtx
%    (c) If you insist on using LaTeX
%           latex \let\install=y\input{pdfcrypt.dtx}
%        (quote the arguments according to the demands of your shell)
%
% Documentation:
%    (a) If pdfcrypt.drv is present:
%           latex pdfcrypt.drv
%    (b) Without pdfcrypt.drv:
%           latex pdfcrypt.dtx; ...
%    The class ltxdoc loads the configuration file ltxdoc.cfg
%    if available. Here you can specify further options, e.g.
%    use A4 as paper format:
%       \PassOptionsToClass{a4paper}{article}
%
%    Programm calls to get the documentation (example):
%       pdflatex pdfcrypt.dtx
%       makeindex -s gind.ist pdfcrypt.idx
%       pdflatex pdfcrypt.dtx
%       makeindex -s gind.ist pdfcrypt.idx
%       pdflatex pdfcrypt.dtx
%
% Installation:
%    TDS:tex/generic/oberdiek/pdfcrypt.sty
%    TDS:doc/latex/oberdiek/pdfcrypt.pdf
%    TDS:source/latex/oberdiek/pdfcrypt.dtx
%
%<*ignore>
\begingroup
  \catcode123=1 %
  \catcode125=2 %
  \def\x{LaTeX2e}%
\expandafter\endgroup
\ifcase 0\ifx\install y1\fi\expandafter
         \ifx\csname processbatchFile\endcsname\relax\else1\fi
         \ifx\fmtname\x\else 1\fi\relax
\else\csname fi\endcsname
%</ignore>
%<*install>
\input docstrip.tex
\Msg{************************************************************************}
\Msg{* Installation}
\Msg{* Package: pdfcrypt 2016/05/16 v1.1 Allows the setting of PDF encryption (HO)}
\Msg{************************************************************************}

\keepsilent
\askforoverwritefalse

\let\MetaPrefix\relax
\preamble

This is a generated file.

Project: pdfcrypt
Version: 2016/05/16 v1.1

Copyright (C)
   2001, 2006, 2007 Heiko Oberdiek
   2016-2019 Oberdiek Package Support Group

This work may be distributed and/or modified under the
conditions of the LaTeX Project Public License, either
version 1.3c of this license or (at your option) any later
version. This version of this license is in
   https://www.latex-project.org/lppl/lppl-1-3c.txt
and the latest version of this license is in
   https://www.latex-project.org/lppl.txt
and version 1.3 or later is part of all distributions of
LaTeX version 2005/12/01 or later.

This work has the LPPL maintenance status "maintained".

The Current Maintainers of this work are
Heiko Oberdiek and the Oberdiek Package Support Group
https://github.com/ho-tex/oberdiek/issues


The Base Interpreter refers to any `TeX-Format',
because some files are installed in TDS:tex/generic//.

This work consists of the main source file pdfcrypt.dtx
and the derived files
   pdfcrypt.sty, pdfcrypt.pdf, pdfcrypt.ins, pdfcrypt.drv.

\endpreamble
\let\MetaPrefix\DoubleperCent

\generate{%
  \file{pdfcrypt.ins}{\from{pdfcrypt.dtx}{install}}%
  \file{pdfcrypt.drv}{\from{pdfcrypt.dtx}{driver}}%
  \usedir{tex/generic/oberdiek}%
  \file{pdfcrypt.sty}{\from{pdfcrypt.dtx}{package}}%
  \nopreamble
  \nopostamble
%  \usedir{source/latex/oberdiek/catalogue}%
%  \file{pdfcrypt.xml}{\from{pdfcrypt.dtx}{catalogue}}%
}

\catcode32=13\relax% active space
\let =\space%
\Msg{************************************************************************}
\Msg{*}
\Msg{* To finish the installation you have to move the following}
\Msg{* file into a directory searched by TeX:}
\Msg{*}
\Msg{*     pdfcrypt.sty}
\Msg{*}
\Msg{* To produce the documentation run the file `pdfcrypt.drv'}
\Msg{* through LaTeX.}
\Msg{*}
\Msg{* Happy TeXing!}
\Msg{*}
\Msg{************************************************************************}

\endbatchfile
%</install>
%<*ignore>
\fi
%</ignore>
%<*driver>
\NeedsTeXFormat{LaTeX2e}
\ProvidesFile{pdfcrypt.drv}%
  [2016/05/16 v1.1 Allows the setting of PDF encryption (HO)]%
\documentclass{ltxdoc}
\usepackage{holtxdoc}[2011/11/22]
\begin{document}
  \DocInput{pdfcrypt.dtx}%
\end{document}
%</driver>
% \fi
%
%
% \CharacterTable
%  {Upper-case    \A\B\C\D\E\F\G\H\I\J\K\L\M\N\O\P\Q\R\S\T\U\V\W\X\Y\Z
%   Lower-case    \a\b\c\d\e\f\g\h\i\j\k\l\m\n\o\p\q\r\s\t\u\v\w\x\y\z
%   Digits        \0\1\2\3\4\5\6\7\8\9
%   Exclamation   \!     Double quote  \"     Hash (number) \#
%   Dollar        \$     Percent       \%     Ampersand     \&
%   Acute accent  \'     Left paren    \(     Right paren   \)
%   Asterisk      \*     Plus          \+     Comma         \,
%   Minus         \-     Point         \.     Solidus       \/
%   Colon         \:     Semicolon     \;     Less than     \<
%   Equals        \=     Greater than  \>     Question mark \?
%   Commercial at \@     Left bracket  \[     Backslash     \\
%   Right bracket \]     Circumflex    \^     Underscore    \_
%   Grave accent  \`     Left brace    \{     Vertical bar  \|
%   Right brace   \}     Tilde         \~}
%
% \GetFileInfo{pdfcrypt.drv}
%
% \title{The \xpackage{pdfcrypt} package}
% \date{2016/05/16 v1.1}
% \author{Heiko Oberdiek\thanks
% {Please report any issues at \url{https://github.com/ho-tex/oberdiek/issues}}}
%
% \maketitle
%
% \begin{abstract}
% This package supports the setting of pdf encryption options
% for \VTeX\ and some older versions of \pdfTeX.
% \end{abstract}
%
% \tableofcontents
%
% \section{Documentation}
%
% This package allows the setting of pdf encryption
% options for
% \begin{itemize}
% \item
%   \VTeX, since version 7.35, \url{http://www.micropress-inc.com/}.
% \item
%   \pdfTeX, patched by
%   \NameEmail{Ricardo Sanchez Carmenes}{carmenes@bioquimica.uniovi.es}.\\
%   Some supported versions are:
%   \begin{quote}
%     \pdfTeX-1.00a,\\
%     \pdfTeX-0.14h-pdfcrypt-20010310,\\
%     pdfcrypt-20010331
%   \end{quote}
% \end{itemize}
%
% \noindent
% \textcolor{red}{Note:} Since \pdfTeX-1.10a (2003-01-16) encryption
% support was dropped! Thus the package is now obsolete for recent
% \pdfTeX\ versions.
%
% \subsection{Alternatives}
%
% There are free alternatives, programs that can
% be used for postprocessing the pdf file:
% \begin{itemize}
% \item \textsf{pdftk}\\
%   \url{http://www.accesspdf.com/pdftk/}
% \item \textsf{Multivalent}\\
%   \url{http://multivalent.sourceforge.net/}
% \item \textsf{PDFBox}\\
%   \url{http://www.pdfbox.org/}
% \item \textsf{PDFTrans}\\
%   \url{http://maddingue.free.fr/softwares/pdftrans.html}
% \item \dots
% \end{itemize}
%
% \subsection{Usage}
%
% The options can be set as package options or
% with the command \cs{pdfcryptsetup}:
% \begin{quote}
%   |owner=|\meta{owner password}\\
%   |user=|\meta{user password}\\
%   |print|, |copy|, |edit|, |annotate=true|, |false|\\
%   |all|, |none|\\
%   |debug|
% \end{quote}
% The encryption is set at |\begin{document}| by default,
% but this can be forced for an earlier time
% using the option |set|.
% Example:
% \begin{quote}
%   |\usepackage[owner=Donald,user=Knuth,print=false]{pdfcrypt}|\\
%   \dots\\
%   |\begin{document}|
% \end{quote}
% or
% \begin{quote}
%   |\usepackage{pdfcrypt}|\\
%   |\pdfcryptsetup{owner=Donald,user=Knut}|\\
%   |\pdfcryptsetup{print=false,set}|\\
%   \dots\\
%   |\begin{document}|
% \end{quote}
%
% \subsubsection{Special characters}
%
% The characters `|{|', `|}|', and `|\|' have to be
% escaped with `|\|'. Then the string should not
% be specified as package option but with
% the macro \cs{pdfcryptsetup}, eg:
% \begin{quote}
%   |\pdfcryptsetup{user={\{\\Hello\}},print=false}|
% \end{quote}
% The password here is ``|{\Hello}|''.
% Active characters can be used and are not expanded.
% Macros except for |\{|, |\}|, and |\\| should not be used
% and are not expanded.
%
% \subsubsection{\cs{nopdfcrypt}}
%
% Often the whole process of pdf generation includes
% several \TeX\ compilations and perhaps other
% postprocessing steps. Therefore a feature would
% be useful to disable the encryption stuff in order
% to speed up the first compilations.
% Therefore package \xpackage{pdfcrypt} will look for
% the macro \cs{nopdfcrypt}. If it has the meaning of the
% letter `|Y|', the package will be disabled.
% Example:
% \begin{quote}
%   |pdflatex '\let\nopdfcrypt=Y\input myfile'|\\
%   |thumbpdf myfile|\\
%   |pdfelatex myfile|
% \end{quote}
%
% \subsubsection{Configuration file}
%
% If the file \xfile{pdfcrypt.cfg} exists it is loaded
% and can be used for common settings.
%
% \subsubsection{Support for \plainTeX}
%
% The package can also be used with \plainTeX. It defines the
% missing \LaTeX\ macros and loads package \xpackage{keyval}.
% The configuration file is not supported and automatically
% loaded.
%
% \subsection{Hints for \pdfTeX}
%
% This section is out of date, because the encryption support
% was removed from \pdfTeX. Therefore this hints are of
% historical interest only.
%
%\begin{verbatim}
%* There are packages such as thumbpdf that ship out
%  some stuff (\immediate\pdfobj{...}). In these cases
%  \pdfcrypt will be ignored without warning or error
%  message. Therefore the package pdfcrypt should be
%  loaded before and option "set" should be used in
%  order to force the call of \pdfcrypt, for example:
%     \usepackage{pdfcrypt}
%     \pdfcryptsetup{..., set}
%     ...
%     \usepackage{thumbpdf}
%
%* Compiling pdfTeX (eg. version 1.00a-pretest-20010804):
%  pdfTeX versions are available at
%      ftp://ftp.cstug.cz/pub/tex/local/cstug/thanh/pdftex/snapshots/
%  Unpack the latest .tgz file:
%      > tar xzf pdftex-20010804.tgz
%  Enable encryption support:
%      src> cd src/texk/web2c/pdftexdir
%      src/texk/web2c/pdftexdir> ln -fs pdfcrypt-full.c pdfcrypt.c
%  Compile:
%      src/texk/web2c/pdftexdir> cd ../../..
%      src> ./configure
%      src> cd texk/web2c
%      src/texk/web2c> make pdftexbin
%  At last install the binary and pool files.
%\end{verbatim}
%
% \StopEventually{
% }
%
% \section{Implementation}
%
%    \begin{macrocode}
%<*package>
%    \end{macrocode}
%
% \subsection{Help macros for \plainTeX}
%
%    \begin{macrocode}
\expandafter\edef\csname pc@endinput\endcsname{%
  \catcode`\noexpand\@=\the\catcode`\@\relax
  \noexpand\endinput
}
\catcode`\@=11 %

\expandafter\ifx\csname @firstoftwo\endcsname\relax
  \long\def\@firstoftwo#1#2{#1}%
\fi

\expandafter\ifx\csname @secondoftwo\endcsname\relax
  \long\def\@secondoftwo#1#2{#2}%
\fi

\expandafter\ifx\csname @ifundefined\endcsname\relax
  \def\@ifundefined#1{%
    \expandafter\ifx\csname #1\endcsname\relax
      \expandafter\@firstoftwo
    \else
      \expandafter\@secondoftwo
    \fi
  }%
\fi

\@ifundefined{@ifnextchar}{%
  \long\def\@ifnextchar#1#2#3{%
    \let\reserved@d=#1%
    \def\reserved@a{#2}%
    \def\reserved@b{#3}%
    \futurelet\@let@token\@ifnch
  }%
  \def\@ifnch{%
    \ifx\@let@token\@sptoken
      \let\reserved@c\@xifnch
    \else
      \ifx\@let@token\reserved@d
        \let\reserved@c\reserved@a
      \else
        \let\reserved@c\reserved@b
      \fi
    \fi
    \reserved@c
  }%
  \begingroup
    \def\:{\global\let\@sptoken= }%
    \: %
    \def\:{\@xifnch}%
    \expandafter\gdef\: {\futurelet\@let@token\@ifnch}%
  \endgroup
}{}

\@ifundefined{ProvidesPackage}{%
  \def\ProvidesPackage#1{%
    \@ifnextchar[{\pc@ProvidesPackage{#1}}%
                 {\pc@ProvidesPackage{#1}[]}%
  }%
  \def\pc@ProvidesPackage#1[#2]{%
    \immediate\write-1{Package: #1 #2}%
  }%
}{}

\begingroup\expandafter\expandafter\expandafter\endgroup
\expandafter\ifx\csname RequirePackage\endcsname\relax
  \input infwarerr.sty\relax
\else
  \RequirePackage{infwarerr}%
\fi

\@ifundefined{@gobble}{%
  \long\def\@gobble#1{}%
}{}

\@ifundefined{@empty}{%
  \def\@empty{}%
}{}
%    \end{macrocode}
%
% \subsection{Package Identification and checks}
%
%    \begin{macrocode}
\ProvidesPackage{pdfcrypt}%
  [2016/05/16 v1.1 Allows the setting of PDF encryption (HO)]%
%    \end{macrocode}
%
%    \begin{macrocode}
\@ifundefined{pdfcryptsetup}{%
  \let\pdfcryptsetup\@gobble
}{%
  \@PackageWarningNoLine{pdfcrypt}{Package pdfcrypt is already loaded}%
  \pc@endinput
}
%    \end{macrocode}
%    Support for \cs{nopdfcrypt}.
%    \begin{macrocode}
\newif\ifpc@nopdfcrypt
\ifx Y\nopdfcrypt
  \@PackageWarningNoLine{pdfcrypt}{%
    Encryption disabled by \string \nopdfcrypt\space request%
  }%
  \global\pc@nopdfcrypttrue
\fi
%    \end{macrocode}
%
% \subsection{Driver detection}
%
%    \begin{macrocode}
\let\pc@driver\@empty
\begingroup
  % pdfTeX detection
  \@ifundefined{pdfoutput}{%
  }{%
    \ifcase\pdfoutput
    \else
      \@ifundefined{pdfcrypt}{%
        \@PackageError{pdfcrypt}{%
          PDF encryption is not supported with this pdfTeX%
        }{%
          Encryption support was added in 0.14h (2001/03/10)\MessageBreak
          and removed in 1.10a (2003/01/16).%
        }%
        \endgroup
        \pc@endinput
      }{%
        \gdef\pc@driver{pdftex}%
      }%
    \fi
  }%
  % VTeX detection
  \@ifundefined{OpMode}{%
  }{%
    \ifnum\OpMode=1 %
      \ifnum\@ifundefined{VTeXversion}0\VTeXversion<735 %
        \@PackageError{pdfcrypt}{%
          PDF encryption is not supported with this VTeX%
        }{%
          You need VTeX 7.35 or higher.%
        }%
        \endgroup
        \pc@endinput
      \else
        \gdef\pc@driver{vtex}%
      \fi
    \fi
  }%
\endgroup
%    \end{macrocode}
%
% \subsection{Load package \xpackage{keyval}}
%
%    \begin{macrocode}
\@ifundefined{@makeother}{%
  \def\@makeother#1{\catcode`#1=12\relax}%
}{}

\@ifundefined{g@addto@macro}{%
  \long\def\g@addto@macro#1#2{%
    \begingroup
      \toks@\expandafter{#1#2}%
      \xdef#1{\the\toks@}%
    \endgroup
  }%
}{}

\@ifundefined{@namedef}{%
  \def\@namedef#1{\expandafter\def\csname#1\endcsname}%
}{}

\@ifundefined{@nameuse}{%
  \def\@nameuse#1{\csname #1\endcsname}%
}{}

\def\pc@KeyvalRestore{%
  \let\pc@KeyvalRestore\@undefined
}

\let\pcOrg@NeedsTeXFormat\NeedsTeXFormat
\@ifundefined{NeedsTeXFormat}{%
  \def\NeedsTeXFormat#1{}%
  \g@addto@macro\pc@KeyvalRestore{%
    \let\NeedsTeXFormat\pcOrg@NeedsTeXFormat
  }%
}{}

\let\pcOrg@DeclareOption\DeclareOption
\@ifundefined{DeclareOption}{%
  \def\DeclareOption#1#2{#2}%
  \g@addto@macro\pc@KeyvalRestore{%
    \let\DeclareOption\pcOrg@DeclareOption
  }%
}{}

\let\pcOrg@ExecuteOptions\ExecuteOptions
\@ifundefined{ExecuteOptions}{%
  \def\ExecuteOptions#1{}%
  \g@addto@macro\pc@KeyvalRestore{%
    \let\ExecuteOptions\pcOrg@ExecuteOptions
  }%
}{}

\let\pcOrg@ProcessOptions\ProcessOptions
\@ifundefined{ProcessOptions}{%
  \def\ProcessOptions{}%
  \g@addto@macro\pc@KeyvalRestore{%
    \let\ProcessOptions\pcOrg@ProcessOptions
  }%
}{}%

\begingroup\expandafter\expandafter\expandafter\endgroup
\expandafter\ifx\csname RequirePackage\endcsname\relax
  \input keyval.sty\relax
\else
  \RequirePackage{keyval}%
\fi
\pc@KeyvalRestore
%    \end{macrocode}
%
% \subsection{Define options}
%
%    \begin{macrocode}
\@ifundefined{@dblarg}{%
  \long\def\@dblarg#1{\@ifnextchar[{#1}{\@xdblarg{#1}}}%
  \long\def\@xdblarg#1#2{#1[{#2}]{#2}}%
}{}

\newif\ifpc@set
\newif\ifpc@print
\newif\ifpc@copy
\newif\ifpc@edit
\newif\ifpc@annotate
\newif\ifpc@debug
\let\pc@owner\@empty
\let\pc@user\@empty

% default: allow all
\pc@printtrue
\pc@copytrue
\pc@edittrue
\pc@annotatetrue

\edef\pc@temp{\catcode`\noexpand\"=\the\catcode`\"\relax}
\@makeother\"
\def\pc@set{%
  \@PackageInfo{pdfcrypt}{%
    \ifpc@debug
      \ifx\pc@owner\@empty
        No owner password%
      \else
        Owner password: `\pc@owner'%
      \fi
      \MessageBreak
      \ifx\pc@user\@empty
        No user password%
      \else
        User password: `\pc@user'%
      \fi
      \MessageBreak
      Flags: %
      \ifpc@print   \else no\fi print, %
      \ifpc@copy    \else no\fi copy, %
      \ifpc@edit    \else no\fi edit, %
      \ifpc@annotate\else no\fi annotate%
      \MessageBreak
    \fi
    \ifpc@nopdfcrypt
      Encryption is disabled by `\string\nopdfcrypt'%
    \else
      Encryption is set for `\pc@driver'%
    \fi
  }%
  \ifpc@nopdfcrypt
  \else
    \@ifundefined{pc@set@\pc@driver}{%
      \ifx\pc@driver\@empty
        \@PackageError{pdfcrypt}{No driver for encryption %
          support found}\@ehc
      \else
        \@PackageError{pdfcrypt}{Cannot set encryption for %
          unknown driver `\pc@driver'}\@ehc
      \fi
    }{%
      \@nameuse{pc@set@\pc@driver}%
    }%
  \fi
}
\def\pc@set@pdftex{%
  \ifnum\pdftexversion<100 %
    \pc@set@pdftexold
  \else
    \pc@set@pdftexnew
  \fi
}
\def\pc@set@pdftexold{%
  \pdfcrypt{%
    owner "\pc@owner" %
    user "\pc@user" %
    \ifpc@print   \else no\fi print %
    \ifpc@copy    \else no\fi copy %
    \ifpc@edit    \else no\fi edit %
    \ifpc@annotate\else no\fi annotate%
  }%
}
\def\pc@set@pdftexnew{%
  \pdfcrypt
    owner{\pc@owner}%
    user{\pc@user}%
    \ifpc@print   \else no\fi print %
    \ifpc@copy    \else no\fi copy %
    \ifpc@edit    \else no\fi edit %
    \ifpc@annotate\else no\fi annotate%
  \relax
}
\def\pc@set@vtex{%
  \immediate\special{!security %
    O=\pc@MakeVTeXString\pc@owner,%
    U=\pc@MakeVTeXString\pc@user,%
    P\ifpc@print +\else -\fi,%
    C\ifpc@copy +\else -\fi,%
    M\ifpc@edit +\else -\fi,%
    A\ifpc@annotate +\else -\fi
  }%
}
\def\pc@MakeVTeXString#1{%
  "\expandafter\pc@@MakeVTeXString#1"\@nil"%
}
\def\pc@@MakeVTeXString#1"#2\@nil{%
  #1%
  \ifx\\#2\\%
  \else
    ""%
    \@ReturnAfterFi{%
      \pc@@MakeVTeXString#2\@nil
    }%
  \fi
}
\long\def\@ReturnAfterFi#1\fi{\fi#1}
\pc@temp

\begingroup
  \catcode`\ =12 \gdef\pc@spaceother{ }\catcode`\ =10\relax
  \catcode`\|=0 %
  \catcode`\\=12 %
  |gdef|pc@DefString#1#2{%
    |def#1{#2}%
    |edef#1{|expandafter|strip@prefix|meaning#1}%
    |edef#1{|expandafter|pc@SpaceToOther#1 |@nil}%
    |edef#1{|expandafter|pc@EscapeRemove#1|@empty\|@empty|@nil}%
  }%
  |gdef|pc@EscapeRemove#1\#2#3|@nil{%
    #1#2%
    |ifx|\#3|\%%
    |else
      |@ReturnAfterFi{%
        |pc@EscapeRemove#3|@nil
      }%
    |fi
  }%
|endgroup
\def\pc@SpaceToOther#1 #2\@nil{%
  #1%
  \ifx\\#2\\%
  \else
    \pc@spaceother
    \@ReturnAfterFi{%
      \pc@SpaceToOther#2\@nil
    }%
  \fi
}

\def\pc@boolkey{\@dblarg\pc@@boolkey}
\def\pc@@boolkey[#1]#2#3{%
  \lowercase{\def\pc@temp{#3}}%
  \ifx\pc@temp\@empty
    \let\pc@temp\pc@true
  \fi
  \ifx\pc@temp\pc@true
  \else
    \ifx\pc@temp\pc@false
    \else
      \let\pc@temp\relax
    \fi
  \fi
  \ifx\pc@temp\relax
    \@PackageWarning{pdfcrypt}{%
      Unexpected value \string`#3\string' of %
      option \string`#2\string'\MessageBreak
      instead of %
      \string`true\string' or \string`false\string'%
    }%
  \else
    \csname pc@#2\pc@temp\endcsname
  \fi
}
\def\pc@true{true}
\def\pc@false{false}

\define@key{pc}{set}[true]{%
  \pc@boolkey{set}{#1}%
}
\define@key{pc}{pdftex}[]{%
  \def\pc@driver{pdftex}%
}
\define@key{pc}{vtex}[]{%
  \def\pc@driver{vtex}%
}
\define@key{pc}{print}[true]{%
  \pc@boolkey{print}{#1}%
}
\define@key{pc}{copy}[true]{%
  \pc@boolkey{copy}{#1}%
}
\define@key{pc}{edit}[true]{%
  \pc@boolkey{edit}{#1}%
}
\define@key{pc}{annotate}[true]{%
  \pc@boolkey{annotate}{#1}%
}
\define@key{pc}{all}[]{%
  \pc@boolkey{print}{true}%
  \pc@boolkey{copy}{true}%
  \pc@boolkey{edit}{true}%
  \pc@boolkey{annotate}{true}%
}
\define@key{pc}{none}[]{%
  \pc@boolkey{print}{false}%
  \pc@boolkey{copy}{false}%
  \pc@boolkey{edit}{false}%
  \pc@boolkey{annotate}{false}%
}

\define@key{pc}{owner}{%
  \pc@DefString\pc@owner{#1}%
}
\define@key{pc}{user}{%
  \pc@DefString\pc@user{#1}%
}
\define@key{pc}{debug}[true]{%
  \pc@boolkey{debug}{#1}%
}

\def\pdfcryptsetup#1{%
  \setkeys{pc}{#1}%
  \ifpc@set
    \pc@set
    \global\let\pc@set\relax
    \gdef\pdfcryptsetup##1{%
      \@PackageWarning{pdfcrypt}{%
        Encryption options are already set\MessageBreak
        new values are ignored%
      }%
    }%
  \fi
}
\begingroup\expandafter\expandafter\expandafter\endgroup
\expandafter\ifx\csname @onlypreamble\endcsname\relax
\else
  \@onlypreamble\pdfcryptsetup
\fi
%    \end{macrocode}
%
% \subsection{support of configuration file}
%    \begin{macrocode}
\begingroup\expandafter\expandafter\expandafter\endgroup
\expandafter\ifx\csname InputIfFileExists\endcsname\relax
  \@PackageInfo{pdfcrypt}{%
    Configuration file pdfcrypt.cfg not supported.%
  }%
\else
  \let\pc@ExecuteOptions\ExecuteOptions
  \InputIfFileExists{pdfcrypt.cfg}{}{}%
  \let\ExecuteOptions\pc@ExecuteOptions
\fi
%    \end{macrocode}
%
% \subsection{Package options}
%
%    Plain format does not know package options.
%    \begin{macrocode}
\begingroup\expandafter\expandafter\expandafter\endgroup
\expandafter\ifx\csname @classoptionslist\endcsname\relax
  \expandafter\pc@endinput
\fi
%    \end{macrocode}
%    Process global and local options.
%    \begin{macrocode}
\def\pc@ProcessOptionsWithKV{%
  \let\pc@temp\@empty
  \@for\CurrentOption:=\@classoptionslist\do{%
    \@ifundefined{KV@pc@\CurrentOption}{}{%
      \edef\pc@temp{\pc@temp,\CurrentOption,}%
      \@expandtwoargs\@removeelement\CurrentOption
        \@unusedoptionlist\@unusedoptionlist
    }%
  }%
  \edef\pc@temp{%
    \noexpand\pdfcryptsetup{%
      \pc@temp\@ptionlist{\@currname.\@currext}%
    }%
  }%
  \pc@temp
}
\pc@ProcessOptionsWithKV
\AtEndOfPackage{\let\@unprocessedoptions\relax}
\AtBeginDocument{\pc@set}

\pc@endinput
%    \end{macrocode}
%
%    \begin{macrocode}
%</package>
%    \end{macrocode}
%
% \section{Installation}
%
% \subsection{Download}
%
% \paragraph{Package.} This package is available on
% CTAN\footnote{\CTANpkg{pdfcrypt}}:
% \begin{description}
% \item[\CTAN{macros/latex/contrib/oberdiek/pdfcrypt.dtx}] The source file.
% \item[\CTAN{macros/latex/contrib/oberdiek/pdfcrypt.pdf}] Documentation.
% \end{description}
%
%
% \paragraph{Bundle.} All the packages of the bundle `oberdiek'
% are also available in a TDS compliant ZIP archive. There
% the packages are already unpacked and the documentation files
% are generated. The files and directories obey the TDS standard.
% \begin{description}
% \item[\CTANinstall{install/macros/latex/contrib/oberdiek.tds.zip}]
% \end{description}
% \emph{TDS} refers to the standard ``A Directory Structure
% for \TeX\ Files'' (\CTANpkg{tds}). Directories
% with \xfile{texmf} in their name are usually organized this way.
%
% \subsection{Bundle installation}
%
% \paragraph{Unpacking.} Unpack the \xfile{oberdiek.tds.zip} in the
% TDS tree (also known as \xfile{texmf} tree) of your choice.
% Example (linux):
% \begin{quote}
%   |unzip oberdiek.tds.zip -d ~/texmf|
% \end{quote}
%
% \subsection{Package installation}
%
% \paragraph{Unpacking.} The \xfile{.dtx} file is a self-extracting
% \docstrip\ archive. The files are extracted by running the
% \xfile{.dtx} through \plainTeX:
% \begin{quote}
%   \verb|tex pdfcrypt.dtx|
% \end{quote}
%
% \paragraph{TDS.} Now the different files must be moved into
% the different directories in your installation TDS tree
% (also known as \xfile{texmf} tree):
% \begin{quote}
% \def\t{^^A
% \begin{tabular}{@{}>{\ttfamily}l@{ $\rightarrow$ }>{\ttfamily}l@{}}
%   pdfcrypt.sty & tex/generic/oberdiek/pdfcrypt.sty\\
%   pdfcrypt.pdf & doc/latex/oberdiek/pdfcrypt.pdf\\
%   pdfcrypt.dtx & source/latex/oberdiek/pdfcrypt.dtx\\
% \end{tabular}^^A
% }^^A
% \sbox0{\t}^^A
% \ifdim\wd0>\linewidth
%   \begingroup
%     \advance\linewidth by\leftmargin
%     \advance\linewidth by\rightmargin
%   \edef\x{\endgroup
%     \def\noexpand\lw{\the\linewidth}^^A
%   }\x
%   \def\lwbox{^^A
%     \leavevmode
%     \hbox to \linewidth{^^A
%       \kern-\leftmargin\relax
%       \hss
%       \usebox0
%       \hss
%       \kern-\rightmargin\relax
%     }^^A
%   }^^A
%   \ifdim\wd0>\lw
%     \sbox0{\small\t}^^A
%     \ifdim\wd0>\linewidth
%       \ifdim\wd0>\lw
%         \sbox0{\footnotesize\t}^^A
%         \ifdim\wd0>\linewidth
%           \ifdim\wd0>\lw
%             \sbox0{\scriptsize\t}^^A
%             \ifdim\wd0>\linewidth
%               \ifdim\wd0>\lw
%                 \sbox0{\tiny\t}^^A
%                 \ifdim\wd0>\linewidth
%                   \lwbox
%                 \else
%                   \usebox0
%                 \fi
%               \else
%                 \lwbox
%               \fi
%             \else
%               \usebox0
%             \fi
%           \else
%             \lwbox
%           \fi
%         \else
%           \usebox0
%         \fi
%       \else
%         \lwbox
%       \fi
%     \else
%       \usebox0
%     \fi
%   \else
%     \lwbox
%   \fi
% \else
%   \usebox0
% \fi
% \end{quote}
% If you have a \xfile{docstrip.cfg} that configures and enables \docstrip's
% TDS installing feature, then some files can already be in the right
% place, see the documentation of \docstrip.
%
% \subsection{Refresh file name databases}
%
% If your \TeX~distribution
% (\TeX\,Live, \mikTeX, \dots) relies on file name databases, you must refresh
% these. For example, \TeX\,Live\ users run \verb|texhash| or
% \verb|mktexlsr|.
%
% \subsection{Some details for the interested}
%
% \paragraph{Unpacking with \LaTeX.}
% The \xfile{.dtx} chooses its action depending on the format:
% \begin{description}
% \item[\plainTeX:] Run \docstrip\ and extract the files.
% \item[\LaTeX:] Generate the documentation.
% \end{description}
% If you insist on using \LaTeX\ for \docstrip\ (really,
% \docstrip\ does not need \LaTeX), then inform the autodetect routine
% about your intention:
% \begin{quote}
%   \verb|latex \let\install=y\input{pdfcrypt.dtx}|
% \end{quote}
% Do not forget to quote the argument according to the demands
% of your shell.
%
% \paragraph{Generating the documentation.}
% You can use both the \xfile{.dtx} or the \xfile{.drv} to generate
% the documentation. The process can be configured by the
% configuration file \xfile{ltxdoc.cfg}. For instance, put this
% line into this file, if you want to have A4 as paper format:
% \begin{quote}
%   \verb|\PassOptionsToClass{a4paper}{article}|
% \end{quote}
% An example follows how to generate the
% documentation with pdf\LaTeX:
% \begin{quote}
%\begin{verbatim}
%pdflatex pdfcrypt.dtx
%makeindex -s gind.ist pdfcrypt.idx
%pdflatex pdfcrypt.dtx
%makeindex -s gind.ist pdfcrypt.idx
%pdflatex pdfcrypt.dtx
%\end{verbatim}
% \end{quote}
%
% \begin{History}
%   \begin{Version}{2001/04/02 v0.1}
%   \item
%     First public version, published in the pdftex
%     mailing list for testing with pdfcrypt-20010331
%   \end{Version}
%   \begin{Version}{2001/07/19 v0.2}
%   \item
%     Default: all allowed.
%   \item
%     Support for \VTeX\ added.
%   \end{Version}
%   \begin{Version}{2001/07/19 v0.3}
%   \item
%     Bug fix: \VTeX\ letter for edit is M (modify).
%   \end{Version}
%   \begin{Version}{2001/07/19 v0.4}
%   \item
%     Bug fix: \cs{VTeXversion} is correct after
%     regenerating the format file.
%   \end{Version}
%   \begin{Version}{2001/08/05 v0.5}
%   \item
%     Syntax change in \pdfTeX\ 1.00a.
%   \end{Version}
%   \begin{Version}{2001/08/09 v0.6}
%   \item
%     Support of special characters:\\
%     input: |\{|, |\}|, |\\| for |{|, |}|, |\|\\
%     output: |"| in \VTeX
%   \item
%     Option debug added.
%   \end{Version}
%   \begin{Version}{2001/10/28 v0.7}
%   \item
%     Plain compatibility.
%   \item
%     \cs{nopdfcrypt} added.
%   \item
%     Typos corrected.
%   \end{Version}
%   \begin{Version}{2006/02/20 v0.8}
%   \item
%     Obsolete remarks for \pdfTeX.
%   \item
%     DTX framework.
%   \item
%     LPPL 1.3
%   \end{Version}
%   \begin{Version}{2007/04/11 v0.9}
%   \item
%     Line ends sanitized.
%   \end{Version}
%   \begin{Version}{2007/04/26 v1.0}
%   \item
%     Use of package \xpackage{infwarerr}.
%   \end{Version}
%   \begin{Version}{2016/05/16 v1.1}
%   \item
%     Documentation updates.
%   \end{Version}
% \end{History}
%
% \PrintIndex
%
% \Finale
\endinput
|
% \end{quote}
% Do not forget to quote the argument according to the demands
% of your shell.
%
% \paragraph{Generating the documentation.}
% You can use both the \xfile{.dtx} or the \xfile{.drv} to generate
% the documentation. The process can be configured by the
% configuration file \xfile{ltxdoc.cfg}. For instance, put this
% line into this file, if you want to have A4 as paper format:
% \begin{quote}
%   \verb|\PassOptionsToClass{a4paper}{article}|
% \end{quote}
% An example follows how to generate the
% documentation with pdf\LaTeX:
% \begin{quote}
%\begin{verbatim}
%pdflatex pdfcrypt.dtx
%makeindex -s gind.ist pdfcrypt.idx
%pdflatex pdfcrypt.dtx
%makeindex -s gind.ist pdfcrypt.idx
%pdflatex pdfcrypt.dtx
%\end{verbatim}
% \end{quote}
%
% \begin{History}
%   \begin{Version}{2001/04/02 v0.1}
%   \item
%     First public version, published in the pdftex
%     mailing list for testing with pdfcrypt-20010331
%   \end{Version}
%   \begin{Version}{2001/07/19 v0.2}
%   \item
%     Default: all allowed.
%   \item
%     Support for \VTeX\ added.
%   \end{Version}
%   \begin{Version}{2001/07/19 v0.3}
%   \item
%     Bug fix: \VTeX\ letter for edit is M (modify).
%   \end{Version}
%   \begin{Version}{2001/07/19 v0.4}
%   \item
%     Bug fix: \cs{VTeXversion} is correct after
%     regenerating the format file.
%   \end{Version}
%   \begin{Version}{2001/08/05 v0.5}
%   \item
%     Syntax change in \pdfTeX\ 1.00a.
%   \end{Version}
%   \begin{Version}{2001/08/09 v0.6}
%   \item
%     Support of special characters:\\
%     input: |\{|, |\}|, |\\| for |{|, |}|, |\|\\
%     output: |"| in \VTeX
%   \item
%     Option debug added.
%   \end{Version}
%   \begin{Version}{2001/10/28 v0.7}
%   \item
%     Plain compatibility.
%   \item
%     \cs{nopdfcrypt} added.
%   \item
%     Typos corrected.
%   \end{Version}
%   \begin{Version}{2006/02/20 v0.8}
%   \item
%     Obsolete remarks for \pdfTeX.
%   \item
%     DTX framework.
%   \item
%     LPPL 1.3
%   \end{Version}
%   \begin{Version}{2007/04/11 v0.9}
%   \item
%     Line ends sanitized.
%   \end{Version}
%   \begin{Version}{2007/04/26 v1.0}
%   \item
%     Use of package \xpackage{infwarerr}.
%   \end{Version}
%   \begin{Version}{2016/05/16 v1.1}
%   \item
%     Documentation updates.
%   \end{Version}
% \end{History}
%
% \PrintIndex
%
% \Finale
\endinput
|
% \end{quote}
% Do not forget to quote the argument according to the demands
% of your shell.
%
% \paragraph{Generating the documentation.}
% You can use both the \xfile{.dtx} or the \xfile{.drv} to generate
% the documentation. The process can be configured by the
% configuration file \xfile{ltxdoc.cfg}. For instance, put this
% line into this file, if you want to have A4 as paper format:
% \begin{quote}
%   \verb|\PassOptionsToClass{a4paper}{article}|
% \end{quote}
% An example follows how to generate the
% documentation with pdf\LaTeX:
% \begin{quote}
%\begin{verbatim}
%pdflatex pdfcrypt.dtx
%makeindex -s gind.ist pdfcrypt.idx
%pdflatex pdfcrypt.dtx
%makeindex -s gind.ist pdfcrypt.idx
%pdflatex pdfcrypt.dtx
%\end{verbatim}
% \end{quote}
%
% \begin{History}
%   \begin{Version}{2001/04/02 v0.1}
%   \item
%     First public version, published in the pdftex
%     mailing list for testing with pdfcrypt-20010331
%   \end{Version}
%   \begin{Version}{2001/07/19 v0.2}
%   \item
%     Default: all allowed.
%   \item
%     Support for \VTeX\ added.
%   \end{Version}
%   \begin{Version}{2001/07/19 v0.3}
%   \item
%     Bug fix: \VTeX\ letter for edit is M (modify).
%   \end{Version}
%   \begin{Version}{2001/07/19 v0.4}
%   \item
%     Bug fix: \cs{VTeXversion} is correct after
%     regenerating the format file.
%   \end{Version}
%   \begin{Version}{2001/08/05 v0.5}
%   \item
%     Syntax change in \pdfTeX\ 1.00a.
%   \end{Version}
%   \begin{Version}{2001/08/09 v0.6}
%   \item
%     Support of special characters:\\
%     input: |\{|, |\}|, |\\| for |{|, |}|, |\|\\
%     output: |"| in \VTeX
%   \item
%     Option debug added.
%   \end{Version}
%   \begin{Version}{2001/10/28 v0.7}
%   \item
%     Plain compatibility.
%   \item
%     \cs{nopdfcrypt} added.
%   \item
%     Typos corrected.
%   \end{Version}
%   \begin{Version}{2006/02/20 v0.8}
%   \item
%     Obsolete remarks for \pdfTeX.
%   \item
%     DTX framework.
%   \item
%     LPPL 1.3
%   \end{Version}
%   \begin{Version}{2007/04/11 v0.9}
%   \item
%     Line ends sanitized.
%   \end{Version}
%   \begin{Version}{2007/04/26 v1.0}
%   \item
%     Use of package \xpackage{infwarerr}.
%   \end{Version}
%   \begin{Version}{2016/05/16 v1.1}
%   \item
%     Documentation updates.
%   \end{Version}
% \end{History}
%
% \PrintIndex
%
% \Finale
\endinput

%        (quote the arguments according to the demands of your shell)
%
% Documentation:
%    (a) If pdfcrypt.drv is present:
%           latex pdfcrypt.drv
%    (b) Without pdfcrypt.drv:
%           latex pdfcrypt.dtx; ...
%    The class ltxdoc loads the configuration file ltxdoc.cfg
%    if available. Here you can specify further options, e.g.
%    use A4 as paper format:
%       \PassOptionsToClass{a4paper}{article}
%
%    Programm calls to get the documentation (example):
%       pdflatex pdfcrypt.dtx
%       makeindex -s gind.ist pdfcrypt.idx
%       pdflatex pdfcrypt.dtx
%       makeindex -s gind.ist pdfcrypt.idx
%       pdflatex pdfcrypt.dtx
%
% Installation:
%    TDS:tex/generic/oberdiek/pdfcrypt.sty
%    TDS:doc/latex/oberdiek/pdfcrypt.pdf
%    TDS:source/latex/oberdiek/pdfcrypt.dtx
%
%<*ignore>
\begingroup
  \catcode123=1 %
  \catcode125=2 %
  \def\x{LaTeX2e}%
\expandafter\endgroup
\ifcase 0\ifx\install y1\fi\expandafter
         \ifx\csname processbatchFile\endcsname\relax\else1\fi
         \ifx\fmtname\x\else 1\fi\relax
\else\csname fi\endcsname
%</ignore>
%<*install>
\input docstrip.tex
\Msg{************************************************************************}
\Msg{* Installation}
\Msg{* Package: pdfcrypt 2016/05/16 v1.1 Allows the setting of PDF encryption (HO)}
\Msg{************************************************************************}

\keepsilent
\askforoverwritefalse

\let\MetaPrefix\relax
\preamble

This is a generated file.

Project: pdfcrypt
Version: 2016/05/16 v1.1

Copyright (C) 2001, 2006, 2007 by
   Heiko Oberdiek <heiko.oberdiek at googlemail.com>

This work may be distributed and/or modified under the
conditions of the LaTeX Project Public License, either
version 1.3c of this license or (at your option) any later
version. This version of this license is in
   https://www.latex-project.org/lppl/lppl-1-3c.txt
and the latest version of this license is in
   https://www.latex-project.org/lppl.txt
and version 1.3 or later is part of all distributions of
LaTeX version 2005/12/01 or later.

This work has the LPPL maintenance status "maintained".

The Current Maintainers of this work are
Heiko Oberdiek and the Oberdiek Package Support Group
https://github.com/ho-tex/oberdiek/issues


The Base Interpreter refers to any `TeX-Format',
because some files are installed in TDS:tex/generic//.

This work consists of the main source file pdfcrypt.dtx
and the derived files
   pdfcrypt.sty, pdfcrypt.pdf, pdfcrypt.ins, pdfcrypt.drv.

\endpreamble
\let\MetaPrefix\DoubleperCent

\generate{%
  \file{pdfcrypt.ins}{\from{pdfcrypt.dtx}{install}}%
  \file{pdfcrypt.drv}{\from{pdfcrypt.dtx}{driver}}%
  \usedir{tex/generic/oberdiek}%
  \file{pdfcrypt.sty}{\from{pdfcrypt.dtx}{package}}%
  \nopreamble
  \nopostamble
%  \usedir{source/latex/oberdiek/catalogue}%
%  \file{pdfcrypt.xml}{\from{pdfcrypt.dtx}{catalogue}}%
}

\catcode32=13\relax% active space
\let =\space%
\Msg{************************************************************************}
\Msg{*}
\Msg{* To finish the installation you have to move the following}
\Msg{* file into a directory searched by TeX:}
\Msg{*}
\Msg{*     pdfcrypt.sty}
\Msg{*}
\Msg{* To produce the documentation run the file `pdfcrypt.drv'}
\Msg{* through LaTeX.}
\Msg{*}
\Msg{* Happy TeXing!}
\Msg{*}
\Msg{************************************************************************}

\endbatchfile
%</install>
%<*ignore>
\fi
%</ignore>
%<*driver>
\NeedsTeXFormat{LaTeX2e}
\ProvidesFile{pdfcrypt.drv}%
  [2016/05/16 v1.1 Allows the setting of PDF encryption (HO)]%
\documentclass{ltxdoc}
\usepackage{holtxdoc}[2011/11/22]
\begin{document}
  \DocInput{pdfcrypt.dtx}%
\end{document}
%</driver>
% \fi
%
%
% \CharacterTable
%  {Upper-case    \A\B\C\D\E\F\G\H\I\J\K\L\M\N\O\P\Q\R\S\T\U\V\W\X\Y\Z
%   Lower-case    \a\b\c\d\e\f\g\h\i\j\k\l\m\n\o\p\q\r\s\t\u\v\w\x\y\z
%   Digits        \0\1\2\3\4\5\6\7\8\9
%   Exclamation   \!     Double quote  \"     Hash (number) \#
%   Dollar        \$     Percent       \%     Ampersand     \&
%   Acute accent  \'     Left paren    \(     Right paren   \)
%   Asterisk      \*     Plus          \+     Comma         \,
%   Minus         \-     Point         \.     Solidus       \/
%   Colon         \:     Semicolon     \;     Less than     \<
%   Equals        \=     Greater than  \>     Question mark \?
%   Commercial at \@     Left bracket  \[     Backslash     \\
%   Right bracket \]     Circumflex    \^     Underscore    \_
%   Grave accent  \`     Left brace    \{     Vertical bar  \|
%   Right brace   \}     Tilde         \~}
%
% \GetFileInfo{pdfcrypt.drv}
%
% \title{The \xpackage{pdfcrypt} package}
% \date{2016/05/16 v1.1}
% \author{Heiko Oberdiek\thanks
% {Please report any issues at \url{https://github.com/ho-tex/oberdiek/issues}}}
%
% \maketitle
%
% \begin{abstract}
% This package supports the setting of pdf encryption options
% for \VTeX\ and some older versions of \pdfTeX.
% \end{abstract}
%
% \tableofcontents
%
% \section{Documentation}
%
% This package allows the setting of pdf encryption
% options for
% \begin{itemize}
% \item
%   \VTeX, since version 7.35, \url{http://www.micropress-inc.com/}.
% \item
%   \pdfTeX, patched by
%   \NameEmail{Ricardo Sanchez Carmenes}{carmenes@bioquimica.uniovi.es}.\\
%   Some supported versions are:
%   \begin{quote}
%     \pdfTeX-1.00a,\\
%     \pdfTeX-0.14h-pdfcrypt-20010310,\\
%     pdfcrypt-20010331
%   \end{quote}
% \end{itemize}
%
% \noindent
% \textcolor{red}{Note:} Since \pdfTeX-1.10a (2003-01-16) encryption
% support was dropped! Thus the package is now obsolete for recent
% \pdfTeX\ versions.
%
% \subsection{Alternatives}
%
% There are free alternatives, programs that can
% be used for postprocessing the pdf file:
% \begin{itemize}
% \item \textsf{pdftk}\\
%   \url{http://www.accesspdf.com/pdftk/}
% \item \textsf{Multivalent}\\
%   \url{http://multivalent.sourceforge.net/}
% \item \textsf{PDFBox}\\
%   \url{http://www.pdfbox.org/}
% \item \textsf{PDFTrans}\\
%   \url{http://maddingue.free.fr/softwares/pdftrans.html}
% \item \dots
% \end{itemize}
%
% \subsection{Usage}
%
% The options can be set as package options or
% with the command \cs{pdfcryptsetup}:
% \begin{quote}
%   |owner=|\meta{owner password}\\
%   |user=|\meta{user password}\\
%   |print|, |copy|, |edit|, |annotate=true|, |false|\\
%   |all|, |none|\\
%   |debug|
% \end{quote}
% The encryption is set at |\begin{document}| by default,
% but this can be forced for an earlier time
% using the option |set|.
% Example:
% \begin{quote}
%   |\usepackage[owner=Donald,user=Knuth,print=false]{pdfcrypt}|\\
%   \dots\\
%   |\begin{document}|
% \end{quote}
% or
% \begin{quote}
%   |\usepackage{pdfcrypt}|\\
%   |\pdfcryptsetup{owner=Donald,user=Knut}|\\
%   |\pdfcryptsetup{print=false,set}|\\
%   \dots\\
%   |\begin{document}|
% \end{quote}
%
% \subsubsection{Special characters}
%
% The characters `|{|', `|}|', and `|\|' have to be
% escaped with `|\|'. Then the string should not
% be specified as package option but with
% the macro \cs{pdfcryptsetup}, eg:
% \begin{quote}
%   |\pdfcryptsetup{user={\{\\Hello\}},print=false}|
% \end{quote}
% The password here is ``|{\Hello}|''.
% Active characters can be used and are not expanded.
% Macros except for |\{|, |\}|, and |\\| should not be used
% and are not expanded.
%
% \subsubsection{\cs{nopdfcrypt}}
%
% Often the whole process of pdf generation includes
% several \TeX\ compilations and perhaps other
% postprocessing steps. Therefore a feature would
% be useful to disable the encryption stuff in order
% to speed up the first compilations.
% Therefore package \xpackage{pdfcrypt} will look for
% the macro \cs{nopdfcrypt}. If it has the meaning of the
% letter `|Y|', the package will be disabled.
% Example:
% \begin{quote}
%   |pdflatex '\let\nopdfcrypt=Y\input myfile'|\\
%   |thumbpdf myfile|\\
%   |pdfelatex myfile|
% \end{quote}
%
% \subsubsection{Configuration file}
%
% If the file \xfile{pdfcrypt.cfg} exists it is loaded
% and can be used for common settings.
%
% \subsubsection{Support for \plainTeX}
%
% The package can also be used with \plainTeX. It defines the
% missing \LaTeX\ macros and loads package \xpackage{keyval}.
% The configuration file is not supported and automatically
% loaded.
%
% \subsection{Hints for \pdfTeX}
%
% This section is out of date, because the encryption support
% was removed from \pdfTeX. Therefore this hints are of
% historical interest only.
%
%\begin{verbatim}
%* There are packages such as thumbpdf that ship out
%  some stuff (\immediate\pdfobj{...}). In these cases
%  \pdfcrypt will be ignored without warning or error
%  message. Therefore the package pdfcrypt should be
%  loaded before and option "set" should be used in
%  order to force the call of \pdfcrypt, for example:
%     \usepackage{pdfcrypt}
%     \pdfcryptsetup{..., set}
%     ...
%     \usepackage{thumbpdf}
%
%* Compiling pdfTeX (eg. version 1.00a-pretest-20010804):
%  pdfTeX versions are available at
%      ftp://ftp.cstug.cz/pub/tex/local/cstug/thanh/pdftex/snapshots/
%  Unpack the latest .tgz file:
%      > tar xzf pdftex-20010804.tgz
%  Enable encryption support:
%      src> cd src/texk/web2c/pdftexdir
%      src/texk/web2c/pdftexdir> ln -fs pdfcrypt-full.c pdfcrypt.c
%  Compile:
%      src/texk/web2c/pdftexdir> cd ../../..
%      src> ./configure
%      src> cd texk/web2c
%      src/texk/web2c> make pdftexbin
%  At last install the binary and pool files.
%\end{verbatim}
%
% \StopEventually{
% }
%
% \section{Implementation}
%
%    \begin{macrocode}
%<*package>
%    \end{macrocode}
%
% \subsection{Help macros for \plainTeX}
%
%    \begin{macrocode}
\expandafter\edef\csname pc@endinput\endcsname{%
  \catcode`\noexpand\@=\the\catcode`\@\relax
  \noexpand\endinput
}
\catcode`\@=11 %

\expandafter\ifx\csname @firstoftwo\endcsname\relax
  \long\def\@firstoftwo#1#2{#1}%
\fi

\expandafter\ifx\csname @secondoftwo\endcsname\relax
  \long\def\@secondoftwo#1#2{#2}%
\fi

\expandafter\ifx\csname @ifundefined\endcsname\relax
  \def\@ifundefined#1{%
    \expandafter\ifx\csname #1\endcsname\relax
      \expandafter\@firstoftwo
    \else
      \expandafter\@secondoftwo
    \fi
  }%
\fi

\@ifundefined{@ifnextchar}{%
  \long\def\@ifnextchar#1#2#3{%
    \let\reserved@d=#1%
    \def\reserved@a{#2}%
    \def\reserved@b{#3}%
    \futurelet\@let@token\@ifnch
  }%
  \def\@ifnch{%
    \ifx\@let@token\@sptoken
      \let\reserved@c\@xifnch
    \else
      \ifx\@let@token\reserved@d
        \let\reserved@c\reserved@a
      \else
        \let\reserved@c\reserved@b
      \fi
    \fi
    \reserved@c
  }%
  \begingroup
    \def\:{\global\let\@sptoken= }%
    \: %
    \def\:{\@xifnch}%
    \expandafter\gdef\: {\futurelet\@let@token\@ifnch}%
  \endgroup
}{}

\@ifundefined{ProvidesPackage}{%
  \def\ProvidesPackage#1{%
    \@ifnextchar[{\pc@ProvidesPackage{#1}}%
                 {\pc@ProvidesPackage{#1}[]}%
  }%
  \def\pc@ProvidesPackage#1[#2]{%
    \immediate\write-1{Package: #1 #2}%
  }%
}{}

\begingroup\expandafter\expandafter\expandafter\endgroup
\expandafter\ifx\csname RequirePackage\endcsname\relax
  \input infwarerr.sty\relax
\else
  \RequirePackage{infwarerr}%
\fi

\@ifundefined{@gobble}{%
  \long\def\@gobble#1{}%
}{}

\@ifundefined{@empty}{%
  \def\@empty{}%
}{}
%    \end{macrocode}
%
% \subsection{Package Identification and checks}
%
%    \begin{macrocode}
\ProvidesPackage{pdfcrypt}%
  [2016/05/16 v1.1 Allows the setting of PDF encryption (HO)]%
%    \end{macrocode}
%
%    \begin{macrocode}
\@ifundefined{pdfcryptsetup}{%
  \let\pdfcryptsetup\@gobble
}{%
  \@PackageWarningNoLine{pdfcrypt}{Package pdfcrypt is already loaded}%
  \pc@endinput
}
%    \end{macrocode}
%    Support for \cs{nopdfcrypt}.
%    \begin{macrocode}
\newif\ifpc@nopdfcrypt
\ifx Y\nopdfcrypt
  \@PackageWarningNoLine{pdfcrypt}{%
    Encryption disabled by \string \nopdfcrypt\space request%
  }%
  \global\pc@nopdfcrypttrue
\fi
%    \end{macrocode}
%
% \subsection{Driver detection}
%
%    \begin{macrocode}
\let\pc@driver\@empty
\begingroup
  % pdfTeX detection
  \@ifundefined{pdfoutput}{%
  }{%
    \ifcase\pdfoutput
    \else
      \@ifundefined{pdfcrypt}{%
        \@PackageError{pdfcrypt}{%
          PDF encryption is not supported with this pdfTeX%
        }{%
          Encryption support was added in 0.14h (2001/03/10)\MessageBreak
          and removed in 1.10a (2003/01/16).%
        }%
        \endgroup
        \pc@endinput
      }{%
        \gdef\pc@driver{pdftex}%
      }%
    \fi
  }%
  % VTeX detection
  \@ifundefined{OpMode}{%
  }{%
    \ifnum\OpMode=1 %
      \ifnum\@ifundefined{VTeXversion}0\VTeXversion<735 %
        \@PackageError{pdfcrypt}{%
          PDF encryption is not supported with this VTeX%
        }{%
          You need VTeX 7.35 or higher.%
        }%
        \endgroup
        \pc@endinput
      \else
        \gdef\pc@driver{vtex}%
      \fi
    \fi
  }%
\endgroup
%    \end{macrocode}
%
% \subsection{Load package \xpackage{keyval}}
%
%    \begin{macrocode}
\@ifundefined{@makeother}{%
  \def\@makeother#1{\catcode`#1=12\relax}%
}{}

\@ifundefined{g@addto@macro}{%
  \long\def\g@addto@macro#1#2{%
    \begingroup
      \toks@\expandafter{#1#2}%
      \xdef#1{\the\toks@}%
    \endgroup
  }%
}{}

\@ifundefined{@namedef}{%
  \def\@namedef#1{\expandafter\def\csname#1\endcsname}%
}{}

\@ifundefined{@nameuse}{%
  \def\@nameuse#1{\csname #1\endcsname}%
}{}

\def\pc@KeyvalRestore{%
  \let\pc@KeyvalRestore\@undefined
}

\let\pcOrg@NeedsTeXFormat\NeedsTeXFormat
\@ifundefined{NeedsTeXFormat}{%
  \def\NeedsTeXFormat#1{}%
  \g@addto@macro\pc@KeyvalRestore{%
    \let\NeedsTeXFormat\pcOrg@NeedsTeXFormat
  }%
}{}

\let\pcOrg@DeclareOption\DeclareOption
\@ifundefined{DeclareOption}{%
  \def\DeclareOption#1#2{#2}%
  \g@addto@macro\pc@KeyvalRestore{%
    \let\DeclareOption\pcOrg@DeclareOption
  }%
}{}

\let\pcOrg@ExecuteOptions\ExecuteOptions
\@ifundefined{ExecuteOptions}{%
  \def\ExecuteOptions#1{}%
  \g@addto@macro\pc@KeyvalRestore{%
    \let\ExecuteOptions\pcOrg@ExecuteOptions
  }%
}{}

\let\pcOrg@ProcessOptions\ProcessOptions
\@ifundefined{ProcessOptions}{%
  \def\ProcessOptions{}%
  \g@addto@macro\pc@KeyvalRestore{%
    \let\ProcessOptions\pcOrg@ProcessOptions
  }%
}{}%

\begingroup\expandafter\expandafter\expandafter\endgroup
\expandafter\ifx\csname RequirePackage\endcsname\relax
  \input keyval.sty\relax
\else
  \RequirePackage{keyval}%
\fi
\pc@KeyvalRestore
%    \end{macrocode}
%
% \subsection{Define options}
%
%    \begin{macrocode}
\@ifundefined{@dblarg}{%
  \long\def\@dblarg#1{\@ifnextchar[{#1}{\@xdblarg{#1}}}%
  \long\def\@xdblarg#1#2{#1[{#2}]{#2}}%
}{}

\newif\ifpc@set
\newif\ifpc@print
\newif\ifpc@copy
\newif\ifpc@edit
\newif\ifpc@annotate
\newif\ifpc@debug
\let\pc@owner\@empty
\let\pc@user\@empty

% default: allow all
\pc@printtrue
\pc@copytrue
\pc@edittrue
\pc@annotatetrue

\edef\pc@temp{\catcode`\noexpand\"=\the\catcode`\"\relax}
\@makeother\"
\def\pc@set{%
  \@PackageInfo{pdfcrypt}{%
    \ifpc@debug
      \ifx\pc@owner\@empty
        No owner password%
      \else
        Owner password: `\pc@owner'%
      \fi
      \MessageBreak
      \ifx\pc@user\@empty
        No user password%
      \else
        User password: `\pc@user'%
      \fi
      \MessageBreak
      Flags: %
      \ifpc@print   \else no\fi print, %
      \ifpc@copy    \else no\fi copy, %
      \ifpc@edit    \else no\fi edit, %
      \ifpc@annotate\else no\fi annotate%
      \MessageBreak
    \fi
    \ifpc@nopdfcrypt
      Encryption is disabled by `\string\nopdfcrypt'%
    \else
      Encryption is set for `\pc@driver'%
    \fi
  }%
  \ifpc@nopdfcrypt
  \else
    \@ifundefined{pc@set@\pc@driver}{%
      \ifx\pc@driver\@empty
        \@PackageError{pdfcrypt}{No driver for encryption %
          support found}\@ehc
      \else
        \@PackageError{pdfcrypt}{Cannot set encryption for %
          unknown driver `\pc@driver'}\@ehc
      \fi
    }{%
      \@nameuse{pc@set@\pc@driver}%
    }%
  \fi
}
\def\pc@set@pdftex{%
  \ifnum\pdftexversion<100 %
    \pc@set@pdftexold
  \else
    \pc@set@pdftexnew
  \fi
}
\def\pc@set@pdftexold{%
  \pdfcrypt{%
    owner "\pc@owner" %
    user "\pc@user" %
    \ifpc@print   \else no\fi print %
    \ifpc@copy    \else no\fi copy %
    \ifpc@edit    \else no\fi edit %
    \ifpc@annotate\else no\fi annotate%
  }%
}
\def\pc@set@pdftexnew{%
  \pdfcrypt
    owner{\pc@owner}%
    user{\pc@user}%
    \ifpc@print   \else no\fi print %
    \ifpc@copy    \else no\fi copy %
    \ifpc@edit    \else no\fi edit %
    \ifpc@annotate\else no\fi annotate%
  \relax
}
\def\pc@set@vtex{%
  \immediate\special{!security %
    O=\pc@MakeVTeXString\pc@owner,%
    U=\pc@MakeVTeXString\pc@user,%
    P\ifpc@print +\else -\fi,%
    C\ifpc@copy +\else -\fi,%
    M\ifpc@edit +\else -\fi,%
    A\ifpc@annotate +\else -\fi
  }%
}
\def\pc@MakeVTeXString#1{%
  "\expandafter\pc@@MakeVTeXString#1"\@nil"%
}
\def\pc@@MakeVTeXString#1"#2\@nil{%
  #1%
  \ifx\\#2\\%
  \else
    ""%
    \@ReturnAfterFi{%
      \pc@@MakeVTeXString#2\@nil
    }%
  \fi
}
\long\def\@ReturnAfterFi#1\fi{\fi#1}
\pc@temp

\begingroup
  \catcode`\ =12 \gdef\pc@spaceother{ }\catcode`\ =10\relax
  \catcode`\|=0 %
  \catcode`\\=12 %
  |gdef|pc@DefString#1#2{%
    |def#1{#2}%
    |edef#1{|expandafter|strip@prefix|meaning#1}%
    |edef#1{|expandafter|pc@SpaceToOther#1 |@nil}%
    |edef#1{|expandafter|pc@EscapeRemove#1|@empty\|@empty|@nil}%
  }%
  |gdef|pc@EscapeRemove#1\#2#3|@nil{%
    #1#2%
    |ifx|\#3|\%%
    |else
      |@ReturnAfterFi{%
        |pc@EscapeRemove#3|@nil
      }%
    |fi
  }%
|endgroup
\def\pc@SpaceToOther#1 #2\@nil{%
  #1%
  \ifx\\#2\\%
  \else
    \pc@spaceother
    \@ReturnAfterFi{%
      \pc@SpaceToOther#2\@nil
    }%
  \fi
}

\def\pc@boolkey{\@dblarg\pc@@boolkey}
\def\pc@@boolkey[#1]#2#3{%
  \lowercase{\def\pc@temp{#3}}%
  \ifx\pc@temp\@empty
    \let\pc@temp\pc@true
  \fi
  \ifx\pc@temp\pc@true
  \else
    \ifx\pc@temp\pc@false
    \else
      \let\pc@temp\relax
    \fi
  \fi
  \ifx\pc@temp\relax
    \@PackageWarning{pdfcrypt}{%
      Unexpected value \string`#3\string' of %
      option \string`#2\string'\MessageBreak
      instead of %
      \string`true\string' or \string`false\string'%
    }%
  \else
    \csname pc@#2\pc@temp\endcsname
  \fi
}
\def\pc@true{true}
\def\pc@false{false}

\define@key{pc}{set}[true]{%
  \pc@boolkey{set}{#1}%
}
\define@key{pc}{pdftex}[]{%
  \def\pc@driver{pdftex}%
}
\define@key{pc}{vtex}[]{%
  \def\pc@driver{vtex}%
}
\define@key{pc}{print}[true]{%
  \pc@boolkey{print}{#1}%
}
\define@key{pc}{copy}[true]{%
  \pc@boolkey{copy}{#1}%
}
\define@key{pc}{edit}[true]{%
  \pc@boolkey{edit}{#1}%
}
\define@key{pc}{annotate}[true]{%
  \pc@boolkey{annotate}{#1}%
}
\define@key{pc}{all}[]{%
  \pc@boolkey{print}{true}%
  \pc@boolkey{copy}{true}%
  \pc@boolkey{edit}{true}%
  \pc@boolkey{annotate}{true}%
}
\define@key{pc}{none}[]{%
  \pc@boolkey{print}{false}%
  \pc@boolkey{copy}{false}%
  \pc@boolkey{edit}{false}%
  \pc@boolkey{annotate}{false}%
}

\define@key{pc}{owner}{%
  \pc@DefString\pc@owner{#1}%
}
\define@key{pc}{user}{%
  \pc@DefString\pc@user{#1}%
}
\define@key{pc}{debug}[true]{%
  \pc@boolkey{debug}{#1}%
}

\def\pdfcryptsetup#1{%
  \setkeys{pc}{#1}%
  \ifpc@set
    \pc@set
    \global\let\pc@set\relax
    \gdef\pdfcryptsetup##1{%
      \@PackageWarning{pdfcrypt}{%
        Encryption options are already set\MessageBreak
        new values are ignored%
      }%
    }%
  \fi
}
\begingroup\expandafter\expandafter\expandafter\endgroup
\expandafter\ifx\csname @onlypreamble\endcsname\relax
\else
  \@onlypreamble\pdfcryptsetup
\fi
%    \end{macrocode}
%
% \subsection{support of configuration file}
%    \begin{macrocode}
\begingroup\expandafter\expandafter\expandafter\endgroup
\expandafter\ifx\csname InputIfFileExists\endcsname\relax
  \@PackageInfo{pdfcrypt}{%
    Configuration file pdfcrypt.cfg not supported.%
  }%
\else
  \let\pc@ExecuteOptions\ExecuteOptions
  \InputIfFileExists{pdfcrypt.cfg}{}{}%
  \let\ExecuteOptions\pc@ExecuteOptions
\fi
%    \end{macrocode}
%
% \subsection{Package options}
%
%    Plain format does not know package options.
%    \begin{macrocode}
\begingroup\expandafter\expandafter\expandafter\endgroup
\expandafter\ifx\csname @classoptionslist\endcsname\relax
  \expandafter\pc@endinput
\fi
%    \end{macrocode}
%    Process global and local options.
%    \begin{macrocode}
\def\pc@ProcessOptionsWithKV{%
  \let\pc@temp\@empty
  \@for\CurrentOption:=\@classoptionslist\do{%
    \@ifundefined{KV@pc@\CurrentOption}{}{%
      \edef\pc@temp{\pc@temp,\CurrentOption,}%
      \@expandtwoargs\@removeelement\CurrentOption
        \@unusedoptionlist\@unusedoptionlist
    }%
  }%
  \edef\pc@temp{%
    \noexpand\pdfcryptsetup{%
      \pc@temp\@ptionlist{\@currname.\@currext}%
    }%
  }%
  \pc@temp
}
\pc@ProcessOptionsWithKV
\AtEndOfPackage{\let\@unprocessedoptions\relax}
\AtBeginDocument{\pc@set}

\pc@endinput
%    \end{macrocode}
%
%    \begin{macrocode}
%</package>
%    \end{macrocode}
%
% \section{Installation}
%
% \subsection{Download}
%
% \paragraph{Package.} This package is available on
% CTAN\footnote{\CTANpkg{pdfcrypt}}:
% \begin{description}
% \item[\CTAN{macros/latex/contrib/oberdiek/pdfcrypt.dtx}] The source file.
% \item[\CTAN{macros/latex/contrib/oberdiek/pdfcrypt.pdf}] Documentation.
% \end{description}
%
%
% \paragraph{Bundle.} All the packages of the bundle `oberdiek'
% are also available in a TDS compliant ZIP archive. There
% the packages are already unpacked and the documentation files
% are generated. The files and directories obey the TDS standard.
% \begin{description}
% \item[\CTANinstall{install/macros/latex/contrib/oberdiek.tds.zip}]
% \end{description}
% \emph{TDS} refers to the standard ``A Directory Structure
% for \TeX\ Files'' (\CTAN{tds/tds.pdf}). Directories
% with \xfile{texmf} in their name are usually organized this way.
%
% \subsection{Bundle installation}
%
% \paragraph{Unpacking.} Unpack the \xfile{oberdiek.tds.zip} in the
% TDS tree (also known as \xfile{texmf} tree) of your choice.
% Example (linux):
% \begin{quote}
%   |unzip oberdiek.tds.zip -d ~/texmf|
% \end{quote}
%
% \paragraph{Script installation.}
% Check the directory \xfile{TDS:scripts/oberdiek/} for
% scripts that need further installation steps.
% Package \xpackage{attachfile2} comes with the Perl script
% \xfile{pdfatfi.pl} that should be installed in such a way
% that it can be called as \texttt{pdfatfi}.
% Example (linux):
% \begin{quote}
%   |chmod +x scripts/oberdiek/pdfatfi.pl|\\
%   |cp scripts/oberdiek/pdfatfi.pl /usr/local/bin/|
% \end{quote}
%
% \subsection{Package installation}
%
% \paragraph{Unpacking.} The \xfile{.dtx} file is a self-extracting
% \docstrip\ archive. The files are extracted by running the
% \xfile{.dtx} through \plainTeX:
% \begin{quote}
%   \verb|tex pdfcrypt.dtx|
% \end{quote}
%
% \paragraph{TDS.} Now the different files must be moved into
% the different directories in your installation TDS tree
% (also known as \xfile{texmf} tree):
% \begin{quote}
% \def\t{^^A
% \begin{tabular}{@{}>{\ttfamily}l@{ $\rightarrow$ }>{\ttfamily}l@{}}
%   pdfcrypt.sty & tex/generic/oberdiek/pdfcrypt.sty\\
%   pdfcrypt.pdf & doc/latex/oberdiek/pdfcrypt.pdf\\
%   pdfcrypt.dtx & source/latex/oberdiek/pdfcrypt.dtx\\
% \end{tabular}^^A
% }^^A
% \sbox0{\t}^^A
% \ifdim\wd0>\linewidth
%   \begingroup
%     \advance\linewidth by\leftmargin
%     \advance\linewidth by\rightmargin
%   \edef\x{\endgroup
%     \def\noexpand\lw{\the\linewidth}^^A
%   }\x
%   \def\lwbox{^^A
%     \leavevmode
%     \hbox to \linewidth{^^A
%       \kern-\leftmargin\relax
%       \hss
%       \usebox0
%       \hss
%       \kern-\rightmargin\relax
%     }^^A
%   }^^A
%   \ifdim\wd0>\lw
%     \sbox0{\small\t}^^A
%     \ifdim\wd0>\linewidth
%       \ifdim\wd0>\lw
%         \sbox0{\footnotesize\t}^^A
%         \ifdim\wd0>\linewidth
%           \ifdim\wd0>\lw
%             \sbox0{\scriptsize\t}^^A
%             \ifdim\wd0>\linewidth
%               \ifdim\wd0>\lw
%                 \sbox0{\tiny\t}^^A
%                 \ifdim\wd0>\linewidth
%                   \lwbox
%                 \else
%                   \usebox0
%                 \fi
%               \else
%                 \lwbox
%               \fi
%             \else
%               \usebox0
%             \fi
%           \else
%             \lwbox
%           \fi
%         \else
%           \usebox0
%         \fi
%       \else
%         \lwbox
%       \fi
%     \else
%       \usebox0
%     \fi
%   \else
%     \lwbox
%   \fi
% \else
%   \usebox0
% \fi
% \end{quote}
% If you have a \xfile{docstrip.cfg} that configures and enables \docstrip's
% TDS installing feature, then some files can already be in the right
% place, see the documentation of \docstrip.
%
% \subsection{Refresh file name databases}
%
% If your \TeX~distribution
% (\teTeX, \mikTeX, \dots) relies on file name databases, you must refresh
% these. For example, \teTeX\ users run \verb|texhash| or
% \verb|mktexlsr|.
%
% \subsection{Some details for the interested}
%
% \paragraph{Attached source.}
%
% The PDF documentation on CTAN also includes the
% \xfile{.dtx} source file. It can be extracted by
% AcrobatReader 6 or higher. Another option is \textsf{pdftk},
% e.g. unpack the file into the current directory:
% \begin{quote}
%   \verb|pdftk pdfcrypt.pdf unpack_files output .|
% \end{quote}
%
% \paragraph{Unpacking with \LaTeX.}
% The \xfile{.dtx} chooses its action depending on the format:
% \begin{description}
% \item[\plainTeX:] Run \docstrip\ and extract the files.
% \item[\LaTeX:] Generate the documentation.
% \end{description}
% If you insist on using \LaTeX\ for \docstrip\ (really,
% \docstrip\ does not need \LaTeX), then inform the autodetect routine
% about your intention:
% \begin{quote}
%   \verb|latex \let\install=y% \iffalse meta-comment
%
% File: pdfcrypt.dtx
% Version: 2016/05/16 v1.1
% Info: Allows the setting of PDF encryption
%
% Copyright (C)
%    2001, 2006, 2007 Heiko Oberdiek
%    2016-2019 Oberdiek Package Support Group
%    https://github.com/ho-tex/oberdiek/issues
%
% This work may be distributed and/or modified under the
% conditions of the LaTeX Project Public License, either
% version 1.3c of this license or (at your option) any later
% version. This version of this license is in
%    https://www.latex-project.org/lppl/lppl-1-3c.txt
% and the latest version of this license is in
%    https://www.latex-project.org/lppl.txt
% and version 1.3 or later is part of all distributions of
% LaTeX version 2005/12/01 or later.
%
% This work has the LPPL maintenance status "maintained".
%
% The Current Maintainers of this work are
% Heiko Oberdiek and the Oberdiek Package Support Group
% https://github.com/ho-tex/oberdiek/issues
%
% The Base Interpreter refers to any `TeX-Format',
% because some files are installed in TDS:tex/generic//.
%
% This work consists of the main source file pdfcrypt.dtx
% and the derived files
%    pdfcrypt.sty, pdfcrypt.pdf, pdfcrypt.ins, pdfcrypt.drv.
%
% Distribution:
%    CTAN:macros/latex/contrib/oberdiek/pdfcrypt.dtx
%    CTAN:macros/latex/contrib/oberdiek/pdfcrypt.pdf
%
% Unpacking:
%    (a) If pdfcrypt.ins is present:
%           tex pdfcrypt.ins
%    (b) Without pdfcrypt.ins:
%           tex pdfcrypt.dtx
%    (c) If you insist on using LaTeX
%           latex \let\install=y% \iffalse meta-comment
%
% File: pdfcrypt.dtx
% Version: 2016/05/16 v1.1
% Info: Allows the setting of PDF encryption
%
% Copyright (C)
%    2001, 2006, 2007 Heiko Oberdiek
%    2016-2019 Oberdiek Package Support Group
%    https://github.com/ho-tex/oberdiek/issues
%
% This work may be distributed and/or modified under the
% conditions of the LaTeX Project Public License, either
% version 1.3c of this license or (at your option) any later
% version. This version of this license is in
%    https://www.latex-project.org/lppl/lppl-1-3c.txt
% and the latest version of this license is in
%    https://www.latex-project.org/lppl.txt
% and version 1.3 or later is part of all distributions of
% LaTeX version 2005/12/01 or later.
%
% This work has the LPPL maintenance status "maintained".
%
% The Current Maintainers of this work are
% Heiko Oberdiek and the Oberdiek Package Support Group
% https://github.com/ho-tex/oberdiek/issues
%
% The Base Interpreter refers to any `TeX-Format',
% because some files are installed in TDS:tex/generic//.
%
% This work consists of the main source file pdfcrypt.dtx
% and the derived files
%    pdfcrypt.sty, pdfcrypt.pdf, pdfcrypt.ins, pdfcrypt.drv.
%
% Distribution:
%    CTAN:macros/latex/contrib/oberdiek/pdfcrypt.dtx
%    CTAN:macros/latex/contrib/oberdiek/pdfcrypt.pdf
%
% Unpacking:
%    (a) If pdfcrypt.ins is present:
%           tex pdfcrypt.ins
%    (b) Without pdfcrypt.ins:
%           tex pdfcrypt.dtx
%    (c) If you insist on using LaTeX
%           latex \let\install=y% \iffalse meta-comment
%
% File: pdfcrypt.dtx
% Version: 2016/05/16 v1.1
% Info: Allows the setting of PDF encryption
%
% Copyright (C)
%    2001, 2006, 2007 Heiko Oberdiek
%    2016-2019 Oberdiek Package Support Group
%    https://github.com/ho-tex/oberdiek/issues
%
% This work may be distributed and/or modified under the
% conditions of the LaTeX Project Public License, either
% version 1.3c of this license or (at your option) any later
% version. This version of this license is in
%    https://www.latex-project.org/lppl/lppl-1-3c.txt
% and the latest version of this license is in
%    https://www.latex-project.org/lppl.txt
% and version 1.3 or later is part of all distributions of
% LaTeX version 2005/12/01 or later.
%
% This work has the LPPL maintenance status "maintained".
%
% The Current Maintainers of this work are
% Heiko Oberdiek and the Oberdiek Package Support Group
% https://github.com/ho-tex/oberdiek/issues
%
% The Base Interpreter refers to any `TeX-Format',
% because some files are installed in TDS:tex/generic//.
%
% This work consists of the main source file pdfcrypt.dtx
% and the derived files
%    pdfcrypt.sty, pdfcrypt.pdf, pdfcrypt.ins, pdfcrypt.drv.
%
% Distribution:
%    CTAN:macros/latex/contrib/oberdiek/pdfcrypt.dtx
%    CTAN:macros/latex/contrib/oberdiek/pdfcrypt.pdf
%
% Unpacking:
%    (a) If pdfcrypt.ins is present:
%           tex pdfcrypt.ins
%    (b) Without pdfcrypt.ins:
%           tex pdfcrypt.dtx
%    (c) If you insist on using LaTeX
%           latex \let\install=y\input{pdfcrypt.dtx}
%        (quote the arguments according to the demands of your shell)
%
% Documentation:
%    (a) If pdfcrypt.drv is present:
%           latex pdfcrypt.drv
%    (b) Without pdfcrypt.drv:
%           latex pdfcrypt.dtx; ...
%    The class ltxdoc loads the configuration file ltxdoc.cfg
%    if available. Here you can specify further options, e.g.
%    use A4 as paper format:
%       \PassOptionsToClass{a4paper}{article}
%
%    Programm calls to get the documentation (example):
%       pdflatex pdfcrypt.dtx
%       makeindex -s gind.ist pdfcrypt.idx
%       pdflatex pdfcrypt.dtx
%       makeindex -s gind.ist pdfcrypt.idx
%       pdflatex pdfcrypt.dtx
%
% Installation:
%    TDS:tex/generic/oberdiek/pdfcrypt.sty
%    TDS:doc/latex/oberdiek/pdfcrypt.pdf
%    TDS:source/latex/oberdiek/pdfcrypt.dtx
%
%<*ignore>
\begingroup
  \catcode123=1 %
  \catcode125=2 %
  \def\x{LaTeX2e}%
\expandafter\endgroup
\ifcase 0\ifx\install y1\fi\expandafter
         \ifx\csname processbatchFile\endcsname\relax\else1\fi
         \ifx\fmtname\x\else 1\fi\relax
\else\csname fi\endcsname
%</ignore>
%<*install>
\input docstrip.tex
\Msg{************************************************************************}
\Msg{* Installation}
\Msg{* Package: pdfcrypt 2016/05/16 v1.1 Allows the setting of PDF encryption (HO)}
\Msg{************************************************************************}

\keepsilent
\askforoverwritefalse

\let\MetaPrefix\relax
\preamble

This is a generated file.

Project: pdfcrypt
Version: 2016/05/16 v1.1

Copyright (C)
   2001, 2006, 2007 Heiko Oberdiek
   2016-2019 Oberdiek Package Support Group

This work may be distributed and/or modified under the
conditions of the LaTeX Project Public License, either
version 1.3c of this license or (at your option) any later
version. This version of this license is in
   https://www.latex-project.org/lppl/lppl-1-3c.txt
and the latest version of this license is in
   https://www.latex-project.org/lppl.txt
and version 1.3 or later is part of all distributions of
LaTeX version 2005/12/01 or later.

This work has the LPPL maintenance status "maintained".

The Current Maintainers of this work are
Heiko Oberdiek and the Oberdiek Package Support Group
https://github.com/ho-tex/oberdiek/issues


The Base Interpreter refers to any `TeX-Format',
because some files are installed in TDS:tex/generic//.

This work consists of the main source file pdfcrypt.dtx
and the derived files
   pdfcrypt.sty, pdfcrypt.pdf, pdfcrypt.ins, pdfcrypt.drv.

\endpreamble
\let\MetaPrefix\DoubleperCent

\generate{%
  \file{pdfcrypt.ins}{\from{pdfcrypt.dtx}{install}}%
  \file{pdfcrypt.drv}{\from{pdfcrypt.dtx}{driver}}%
  \usedir{tex/generic/oberdiek}%
  \file{pdfcrypt.sty}{\from{pdfcrypt.dtx}{package}}%
  \nopreamble
  \nopostamble
%  \usedir{source/latex/oberdiek/catalogue}%
%  \file{pdfcrypt.xml}{\from{pdfcrypt.dtx}{catalogue}}%
}

\catcode32=13\relax% active space
\let =\space%
\Msg{************************************************************************}
\Msg{*}
\Msg{* To finish the installation you have to move the following}
\Msg{* file into a directory searched by TeX:}
\Msg{*}
\Msg{*     pdfcrypt.sty}
\Msg{*}
\Msg{* To produce the documentation run the file `pdfcrypt.drv'}
\Msg{* through LaTeX.}
\Msg{*}
\Msg{* Happy TeXing!}
\Msg{*}
\Msg{************************************************************************}

\endbatchfile
%</install>
%<*ignore>
\fi
%</ignore>
%<*driver>
\NeedsTeXFormat{LaTeX2e}
\ProvidesFile{pdfcrypt.drv}%
  [2016/05/16 v1.1 Allows the setting of PDF encryption (HO)]%
\documentclass{ltxdoc}
\usepackage{holtxdoc}[2011/11/22]
\begin{document}
  \DocInput{pdfcrypt.dtx}%
\end{document}
%</driver>
% \fi
%
%
% \CharacterTable
%  {Upper-case    \A\B\C\D\E\F\G\H\I\J\K\L\M\N\O\P\Q\R\S\T\U\V\W\X\Y\Z
%   Lower-case    \a\b\c\d\e\f\g\h\i\j\k\l\m\n\o\p\q\r\s\t\u\v\w\x\y\z
%   Digits        \0\1\2\3\4\5\6\7\8\9
%   Exclamation   \!     Double quote  \"     Hash (number) \#
%   Dollar        \$     Percent       \%     Ampersand     \&
%   Acute accent  \'     Left paren    \(     Right paren   \)
%   Asterisk      \*     Plus          \+     Comma         \,
%   Minus         \-     Point         \.     Solidus       \/
%   Colon         \:     Semicolon     \;     Less than     \<
%   Equals        \=     Greater than  \>     Question mark \?
%   Commercial at \@     Left bracket  \[     Backslash     \\
%   Right bracket \]     Circumflex    \^     Underscore    \_
%   Grave accent  \`     Left brace    \{     Vertical bar  \|
%   Right brace   \}     Tilde         \~}
%
% \GetFileInfo{pdfcrypt.drv}
%
% \title{The \xpackage{pdfcrypt} package}
% \date{2016/05/16 v1.1}
% \author{Heiko Oberdiek\thanks
% {Please report any issues at \url{https://github.com/ho-tex/oberdiek/issues}}}
%
% \maketitle
%
% \begin{abstract}
% This package supports the setting of pdf encryption options
% for \VTeX\ and some older versions of \pdfTeX.
% \end{abstract}
%
% \tableofcontents
%
% \section{Documentation}
%
% This package allows the setting of pdf encryption
% options for
% \begin{itemize}
% \item
%   \VTeX, since version 7.35, \url{http://www.micropress-inc.com/}.
% \item
%   \pdfTeX, patched by
%   \NameEmail{Ricardo Sanchez Carmenes}{carmenes@bioquimica.uniovi.es}.\\
%   Some supported versions are:
%   \begin{quote}
%     \pdfTeX-1.00a,\\
%     \pdfTeX-0.14h-pdfcrypt-20010310,\\
%     pdfcrypt-20010331
%   \end{quote}
% \end{itemize}
%
% \noindent
% \textcolor{red}{Note:} Since \pdfTeX-1.10a (2003-01-16) encryption
% support was dropped! Thus the package is now obsolete for recent
% \pdfTeX\ versions.
%
% \subsection{Alternatives}
%
% There are free alternatives, programs that can
% be used for postprocessing the pdf file:
% \begin{itemize}
% \item \textsf{pdftk}\\
%   \url{http://www.accesspdf.com/pdftk/}
% \item \textsf{Multivalent}\\
%   \url{http://multivalent.sourceforge.net/}
% \item \textsf{PDFBox}\\
%   \url{http://www.pdfbox.org/}
% \item \textsf{PDFTrans}\\
%   \url{http://maddingue.free.fr/softwares/pdftrans.html}
% \item \dots
% \end{itemize}
%
% \subsection{Usage}
%
% The options can be set as package options or
% with the command \cs{pdfcryptsetup}:
% \begin{quote}
%   |owner=|\meta{owner password}\\
%   |user=|\meta{user password}\\
%   |print|, |copy|, |edit|, |annotate=true|, |false|\\
%   |all|, |none|\\
%   |debug|
% \end{quote}
% The encryption is set at |\begin{document}| by default,
% but this can be forced for an earlier time
% using the option |set|.
% Example:
% \begin{quote}
%   |\usepackage[owner=Donald,user=Knuth,print=false]{pdfcrypt}|\\
%   \dots\\
%   |\begin{document}|
% \end{quote}
% or
% \begin{quote}
%   |\usepackage{pdfcrypt}|\\
%   |\pdfcryptsetup{owner=Donald,user=Knut}|\\
%   |\pdfcryptsetup{print=false,set}|\\
%   \dots\\
%   |\begin{document}|
% \end{quote}
%
% \subsubsection{Special characters}
%
% The characters `|{|', `|}|', and `|\|' have to be
% escaped with `|\|'. Then the string should not
% be specified as package option but with
% the macro \cs{pdfcryptsetup}, eg:
% \begin{quote}
%   |\pdfcryptsetup{user={\{\\Hello\}},print=false}|
% \end{quote}
% The password here is ``|{\Hello}|''.
% Active characters can be used and are not expanded.
% Macros except for |\{|, |\}|, and |\\| should not be used
% and are not expanded.
%
% \subsubsection{\cs{nopdfcrypt}}
%
% Often the whole process of pdf generation includes
% several \TeX\ compilations and perhaps other
% postprocessing steps. Therefore a feature would
% be useful to disable the encryption stuff in order
% to speed up the first compilations.
% Therefore package \xpackage{pdfcrypt} will look for
% the macro \cs{nopdfcrypt}. If it has the meaning of the
% letter `|Y|', the package will be disabled.
% Example:
% \begin{quote}
%   |pdflatex '\let\nopdfcrypt=Y\input myfile'|\\
%   |thumbpdf myfile|\\
%   |pdfelatex myfile|
% \end{quote}
%
% \subsubsection{Configuration file}
%
% If the file \xfile{pdfcrypt.cfg} exists it is loaded
% and can be used for common settings.
%
% \subsubsection{Support for \plainTeX}
%
% The package can also be used with \plainTeX. It defines the
% missing \LaTeX\ macros and loads package \xpackage{keyval}.
% The configuration file is not supported and automatically
% loaded.
%
% \subsection{Hints for \pdfTeX}
%
% This section is out of date, because the encryption support
% was removed from \pdfTeX. Therefore this hints are of
% historical interest only.
%
%\begin{verbatim}
%* There are packages such as thumbpdf that ship out
%  some stuff (\immediate\pdfobj{...}). In these cases
%  \pdfcrypt will be ignored without warning or error
%  message. Therefore the package pdfcrypt should be
%  loaded before and option "set" should be used in
%  order to force the call of \pdfcrypt, for example:
%     \usepackage{pdfcrypt}
%     \pdfcryptsetup{..., set}
%     ...
%     \usepackage{thumbpdf}
%
%* Compiling pdfTeX (eg. version 1.00a-pretest-20010804):
%  pdfTeX versions are available at
%      ftp://ftp.cstug.cz/pub/tex/local/cstug/thanh/pdftex/snapshots/
%  Unpack the latest .tgz file:
%      > tar xzf pdftex-20010804.tgz
%  Enable encryption support:
%      src> cd src/texk/web2c/pdftexdir
%      src/texk/web2c/pdftexdir> ln -fs pdfcrypt-full.c pdfcrypt.c
%  Compile:
%      src/texk/web2c/pdftexdir> cd ../../..
%      src> ./configure
%      src> cd texk/web2c
%      src/texk/web2c> make pdftexbin
%  At last install the binary and pool files.
%\end{verbatim}
%
% \StopEventually{
% }
%
% \section{Implementation}
%
%    \begin{macrocode}
%<*package>
%    \end{macrocode}
%
% \subsection{Help macros for \plainTeX}
%
%    \begin{macrocode}
\expandafter\edef\csname pc@endinput\endcsname{%
  \catcode`\noexpand\@=\the\catcode`\@\relax
  \noexpand\endinput
}
\catcode`\@=11 %

\expandafter\ifx\csname @firstoftwo\endcsname\relax
  \long\def\@firstoftwo#1#2{#1}%
\fi

\expandafter\ifx\csname @secondoftwo\endcsname\relax
  \long\def\@secondoftwo#1#2{#2}%
\fi

\expandafter\ifx\csname @ifundefined\endcsname\relax
  \def\@ifundefined#1{%
    \expandafter\ifx\csname #1\endcsname\relax
      \expandafter\@firstoftwo
    \else
      \expandafter\@secondoftwo
    \fi
  }%
\fi

\@ifundefined{@ifnextchar}{%
  \long\def\@ifnextchar#1#2#3{%
    \let\reserved@d=#1%
    \def\reserved@a{#2}%
    \def\reserved@b{#3}%
    \futurelet\@let@token\@ifnch
  }%
  \def\@ifnch{%
    \ifx\@let@token\@sptoken
      \let\reserved@c\@xifnch
    \else
      \ifx\@let@token\reserved@d
        \let\reserved@c\reserved@a
      \else
        \let\reserved@c\reserved@b
      \fi
    \fi
    \reserved@c
  }%
  \begingroup
    \def\:{\global\let\@sptoken= }%
    \: %
    \def\:{\@xifnch}%
    \expandafter\gdef\: {\futurelet\@let@token\@ifnch}%
  \endgroup
}{}

\@ifundefined{ProvidesPackage}{%
  \def\ProvidesPackage#1{%
    \@ifnextchar[{\pc@ProvidesPackage{#1}}%
                 {\pc@ProvidesPackage{#1}[]}%
  }%
  \def\pc@ProvidesPackage#1[#2]{%
    \immediate\write-1{Package: #1 #2}%
  }%
}{}

\begingroup\expandafter\expandafter\expandafter\endgroup
\expandafter\ifx\csname RequirePackage\endcsname\relax
  \input infwarerr.sty\relax
\else
  \RequirePackage{infwarerr}%
\fi

\@ifundefined{@gobble}{%
  \long\def\@gobble#1{}%
}{}

\@ifundefined{@empty}{%
  \def\@empty{}%
}{}
%    \end{macrocode}
%
% \subsection{Package Identification and checks}
%
%    \begin{macrocode}
\ProvidesPackage{pdfcrypt}%
  [2016/05/16 v1.1 Allows the setting of PDF encryption (HO)]%
%    \end{macrocode}
%
%    \begin{macrocode}
\@ifundefined{pdfcryptsetup}{%
  \let\pdfcryptsetup\@gobble
}{%
  \@PackageWarningNoLine{pdfcrypt}{Package pdfcrypt is already loaded}%
  \pc@endinput
}
%    \end{macrocode}
%    Support for \cs{nopdfcrypt}.
%    \begin{macrocode}
\newif\ifpc@nopdfcrypt
\ifx Y\nopdfcrypt
  \@PackageWarningNoLine{pdfcrypt}{%
    Encryption disabled by \string \nopdfcrypt\space request%
  }%
  \global\pc@nopdfcrypttrue
\fi
%    \end{macrocode}
%
% \subsection{Driver detection}
%
%    \begin{macrocode}
\let\pc@driver\@empty
\begingroup
  % pdfTeX detection
  \@ifundefined{pdfoutput}{%
  }{%
    \ifcase\pdfoutput
    \else
      \@ifundefined{pdfcrypt}{%
        \@PackageError{pdfcrypt}{%
          PDF encryption is not supported with this pdfTeX%
        }{%
          Encryption support was added in 0.14h (2001/03/10)\MessageBreak
          and removed in 1.10a (2003/01/16).%
        }%
        \endgroup
        \pc@endinput
      }{%
        \gdef\pc@driver{pdftex}%
      }%
    \fi
  }%
  % VTeX detection
  \@ifundefined{OpMode}{%
  }{%
    \ifnum\OpMode=1 %
      \ifnum\@ifundefined{VTeXversion}0\VTeXversion<735 %
        \@PackageError{pdfcrypt}{%
          PDF encryption is not supported with this VTeX%
        }{%
          You need VTeX 7.35 or higher.%
        }%
        \endgroup
        \pc@endinput
      \else
        \gdef\pc@driver{vtex}%
      \fi
    \fi
  }%
\endgroup
%    \end{macrocode}
%
% \subsection{Load package \xpackage{keyval}}
%
%    \begin{macrocode}
\@ifundefined{@makeother}{%
  \def\@makeother#1{\catcode`#1=12\relax}%
}{}

\@ifundefined{g@addto@macro}{%
  \long\def\g@addto@macro#1#2{%
    \begingroup
      \toks@\expandafter{#1#2}%
      \xdef#1{\the\toks@}%
    \endgroup
  }%
}{}

\@ifundefined{@namedef}{%
  \def\@namedef#1{\expandafter\def\csname#1\endcsname}%
}{}

\@ifundefined{@nameuse}{%
  \def\@nameuse#1{\csname #1\endcsname}%
}{}

\def\pc@KeyvalRestore{%
  \let\pc@KeyvalRestore\@undefined
}

\let\pcOrg@NeedsTeXFormat\NeedsTeXFormat
\@ifundefined{NeedsTeXFormat}{%
  \def\NeedsTeXFormat#1{}%
  \g@addto@macro\pc@KeyvalRestore{%
    \let\NeedsTeXFormat\pcOrg@NeedsTeXFormat
  }%
}{}

\let\pcOrg@DeclareOption\DeclareOption
\@ifundefined{DeclareOption}{%
  \def\DeclareOption#1#2{#2}%
  \g@addto@macro\pc@KeyvalRestore{%
    \let\DeclareOption\pcOrg@DeclareOption
  }%
}{}

\let\pcOrg@ExecuteOptions\ExecuteOptions
\@ifundefined{ExecuteOptions}{%
  \def\ExecuteOptions#1{}%
  \g@addto@macro\pc@KeyvalRestore{%
    \let\ExecuteOptions\pcOrg@ExecuteOptions
  }%
}{}

\let\pcOrg@ProcessOptions\ProcessOptions
\@ifundefined{ProcessOptions}{%
  \def\ProcessOptions{}%
  \g@addto@macro\pc@KeyvalRestore{%
    \let\ProcessOptions\pcOrg@ProcessOptions
  }%
}{}%

\begingroup\expandafter\expandafter\expandafter\endgroup
\expandafter\ifx\csname RequirePackage\endcsname\relax
  \input keyval.sty\relax
\else
  \RequirePackage{keyval}%
\fi
\pc@KeyvalRestore
%    \end{macrocode}
%
% \subsection{Define options}
%
%    \begin{macrocode}
\@ifundefined{@dblarg}{%
  \long\def\@dblarg#1{\@ifnextchar[{#1}{\@xdblarg{#1}}}%
  \long\def\@xdblarg#1#2{#1[{#2}]{#2}}%
}{}

\newif\ifpc@set
\newif\ifpc@print
\newif\ifpc@copy
\newif\ifpc@edit
\newif\ifpc@annotate
\newif\ifpc@debug
\let\pc@owner\@empty
\let\pc@user\@empty

% default: allow all
\pc@printtrue
\pc@copytrue
\pc@edittrue
\pc@annotatetrue

\edef\pc@temp{\catcode`\noexpand\"=\the\catcode`\"\relax}
\@makeother\"
\def\pc@set{%
  \@PackageInfo{pdfcrypt}{%
    \ifpc@debug
      \ifx\pc@owner\@empty
        No owner password%
      \else
        Owner password: `\pc@owner'%
      \fi
      \MessageBreak
      \ifx\pc@user\@empty
        No user password%
      \else
        User password: `\pc@user'%
      \fi
      \MessageBreak
      Flags: %
      \ifpc@print   \else no\fi print, %
      \ifpc@copy    \else no\fi copy, %
      \ifpc@edit    \else no\fi edit, %
      \ifpc@annotate\else no\fi annotate%
      \MessageBreak
    \fi
    \ifpc@nopdfcrypt
      Encryption is disabled by `\string\nopdfcrypt'%
    \else
      Encryption is set for `\pc@driver'%
    \fi
  }%
  \ifpc@nopdfcrypt
  \else
    \@ifundefined{pc@set@\pc@driver}{%
      \ifx\pc@driver\@empty
        \@PackageError{pdfcrypt}{No driver for encryption %
          support found}\@ehc
      \else
        \@PackageError{pdfcrypt}{Cannot set encryption for %
          unknown driver `\pc@driver'}\@ehc
      \fi
    }{%
      \@nameuse{pc@set@\pc@driver}%
    }%
  \fi
}
\def\pc@set@pdftex{%
  \ifnum\pdftexversion<100 %
    \pc@set@pdftexold
  \else
    \pc@set@pdftexnew
  \fi
}
\def\pc@set@pdftexold{%
  \pdfcrypt{%
    owner "\pc@owner" %
    user "\pc@user" %
    \ifpc@print   \else no\fi print %
    \ifpc@copy    \else no\fi copy %
    \ifpc@edit    \else no\fi edit %
    \ifpc@annotate\else no\fi annotate%
  }%
}
\def\pc@set@pdftexnew{%
  \pdfcrypt
    owner{\pc@owner}%
    user{\pc@user}%
    \ifpc@print   \else no\fi print %
    \ifpc@copy    \else no\fi copy %
    \ifpc@edit    \else no\fi edit %
    \ifpc@annotate\else no\fi annotate%
  \relax
}
\def\pc@set@vtex{%
  \immediate\special{!security %
    O=\pc@MakeVTeXString\pc@owner,%
    U=\pc@MakeVTeXString\pc@user,%
    P\ifpc@print +\else -\fi,%
    C\ifpc@copy +\else -\fi,%
    M\ifpc@edit +\else -\fi,%
    A\ifpc@annotate +\else -\fi
  }%
}
\def\pc@MakeVTeXString#1{%
  "\expandafter\pc@@MakeVTeXString#1"\@nil"%
}
\def\pc@@MakeVTeXString#1"#2\@nil{%
  #1%
  \ifx\\#2\\%
  \else
    ""%
    \@ReturnAfterFi{%
      \pc@@MakeVTeXString#2\@nil
    }%
  \fi
}
\long\def\@ReturnAfterFi#1\fi{\fi#1}
\pc@temp

\begingroup
  \catcode`\ =12 \gdef\pc@spaceother{ }\catcode`\ =10\relax
  \catcode`\|=0 %
  \catcode`\\=12 %
  |gdef|pc@DefString#1#2{%
    |def#1{#2}%
    |edef#1{|expandafter|strip@prefix|meaning#1}%
    |edef#1{|expandafter|pc@SpaceToOther#1 |@nil}%
    |edef#1{|expandafter|pc@EscapeRemove#1|@empty\|@empty|@nil}%
  }%
  |gdef|pc@EscapeRemove#1\#2#3|@nil{%
    #1#2%
    |ifx|\#3|\%%
    |else
      |@ReturnAfterFi{%
        |pc@EscapeRemove#3|@nil
      }%
    |fi
  }%
|endgroup
\def\pc@SpaceToOther#1 #2\@nil{%
  #1%
  \ifx\\#2\\%
  \else
    \pc@spaceother
    \@ReturnAfterFi{%
      \pc@SpaceToOther#2\@nil
    }%
  \fi
}

\def\pc@boolkey{\@dblarg\pc@@boolkey}
\def\pc@@boolkey[#1]#2#3{%
  \lowercase{\def\pc@temp{#3}}%
  \ifx\pc@temp\@empty
    \let\pc@temp\pc@true
  \fi
  \ifx\pc@temp\pc@true
  \else
    \ifx\pc@temp\pc@false
    \else
      \let\pc@temp\relax
    \fi
  \fi
  \ifx\pc@temp\relax
    \@PackageWarning{pdfcrypt}{%
      Unexpected value \string`#3\string' of %
      option \string`#2\string'\MessageBreak
      instead of %
      \string`true\string' or \string`false\string'%
    }%
  \else
    \csname pc@#2\pc@temp\endcsname
  \fi
}
\def\pc@true{true}
\def\pc@false{false}

\define@key{pc}{set}[true]{%
  \pc@boolkey{set}{#1}%
}
\define@key{pc}{pdftex}[]{%
  \def\pc@driver{pdftex}%
}
\define@key{pc}{vtex}[]{%
  \def\pc@driver{vtex}%
}
\define@key{pc}{print}[true]{%
  \pc@boolkey{print}{#1}%
}
\define@key{pc}{copy}[true]{%
  \pc@boolkey{copy}{#1}%
}
\define@key{pc}{edit}[true]{%
  \pc@boolkey{edit}{#1}%
}
\define@key{pc}{annotate}[true]{%
  \pc@boolkey{annotate}{#1}%
}
\define@key{pc}{all}[]{%
  \pc@boolkey{print}{true}%
  \pc@boolkey{copy}{true}%
  \pc@boolkey{edit}{true}%
  \pc@boolkey{annotate}{true}%
}
\define@key{pc}{none}[]{%
  \pc@boolkey{print}{false}%
  \pc@boolkey{copy}{false}%
  \pc@boolkey{edit}{false}%
  \pc@boolkey{annotate}{false}%
}

\define@key{pc}{owner}{%
  \pc@DefString\pc@owner{#1}%
}
\define@key{pc}{user}{%
  \pc@DefString\pc@user{#1}%
}
\define@key{pc}{debug}[true]{%
  \pc@boolkey{debug}{#1}%
}

\def\pdfcryptsetup#1{%
  \setkeys{pc}{#1}%
  \ifpc@set
    \pc@set
    \global\let\pc@set\relax
    \gdef\pdfcryptsetup##1{%
      \@PackageWarning{pdfcrypt}{%
        Encryption options are already set\MessageBreak
        new values are ignored%
      }%
    }%
  \fi
}
\begingroup\expandafter\expandafter\expandafter\endgroup
\expandafter\ifx\csname @onlypreamble\endcsname\relax
\else
  \@onlypreamble\pdfcryptsetup
\fi
%    \end{macrocode}
%
% \subsection{support of configuration file}
%    \begin{macrocode}
\begingroup\expandafter\expandafter\expandafter\endgroup
\expandafter\ifx\csname InputIfFileExists\endcsname\relax
  \@PackageInfo{pdfcrypt}{%
    Configuration file pdfcrypt.cfg not supported.%
  }%
\else
  \let\pc@ExecuteOptions\ExecuteOptions
  \InputIfFileExists{pdfcrypt.cfg}{}{}%
  \let\ExecuteOptions\pc@ExecuteOptions
\fi
%    \end{macrocode}
%
% \subsection{Package options}
%
%    Plain format does not know package options.
%    \begin{macrocode}
\begingroup\expandafter\expandafter\expandafter\endgroup
\expandafter\ifx\csname @classoptionslist\endcsname\relax
  \expandafter\pc@endinput
\fi
%    \end{macrocode}
%    Process global and local options.
%    \begin{macrocode}
\def\pc@ProcessOptionsWithKV{%
  \let\pc@temp\@empty
  \@for\CurrentOption:=\@classoptionslist\do{%
    \@ifundefined{KV@pc@\CurrentOption}{}{%
      \edef\pc@temp{\pc@temp,\CurrentOption,}%
      \@expandtwoargs\@removeelement\CurrentOption
        \@unusedoptionlist\@unusedoptionlist
    }%
  }%
  \edef\pc@temp{%
    \noexpand\pdfcryptsetup{%
      \pc@temp\@ptionlist{\@currname.\@currext}%
    }%
  }%
  \pc@temp
}
\pc@ProcessOptionsWithKV
\AtEndOfPackage{\let\@unprocessedoptions\relax}
\AtBeginDocument{\pc@set}

\pc@endinput
%    \end{macrocode}
%
%    \begin{macrocode}
%</package>
%    \end{macrocode}
%
% \section{Installation}
%
% \subsection{Download}
%
% \paragraph{Package.} This package is available on
% CTAN\footnote{\CTANpkg{pdfcrypt}}:
% \begin{description}
% \item[\CTAN{macros/latex/contrib/oberdiek/pdfcrypt.dtx}] The source file.
% \item[\CTAN{macros/latex/contrib/oberdiek/pdfcrypt.pdf}] Documentation.
% \end{description}
%
%
% \paragraph{Bundle.} All the packages of the bundle `oberdiek'
% are also available in a TDS compliant ZIP archive. There
% the packages are already unpacked and the documentation files
% are generated. The files and directories obey the TDS standard.
% \begin{description}
% \item[\CTANinstall{install/macros/latex/contrib/oberdiek.tds.zip}]
% \end{description}
% \emph{TDS} refers to the standard ``A Directory Structure
% for \TeX\ Files'' (\CTANpkg{tds}). Directories
% with \xfile{texmf} in their name are usually organized this way.
%
% \subsection{Bundle installation}
%
% \paragraph{Unpacking.} Unpack the \xfile{oberdiek.tds.zip} in the
% TDS tree (also known as \xfile{texmf} tree) of your choice.
% Example (linux):
% \begin{quote}
%   |unzip oberdiek.tds.zip -d ~/texmf|
% \end{quote}
%
% \subsection{Package installation}
%
% \paragraph{Unpacking.} The \xfile{.dtx} file is a self-extracting
% \docstrip\ archive. The files are extracted by running the
% \xfile{.dtx} through \plainTeX:
% \begin{quote}
%   \verb|tex pdfcrypt.dtx|
% \end{quote}
%
% \paragraph{TDS.} Now the different files must be moved into
% the different directories in your installation TDS tree
% (also known as \xfile{texmf} tree):
% \begin{quote}
% \def\t{^^A
% \begin{tabular}{@{}>{\ttfamily}l@{ $\rightarrow$ }>{\ttfamily}l@{}}
%   pdfcrypt.sty & tex/generic/oberdiek/pdfcrypt.sty\\
%   pdfcrypt.pdf & doc/latex/oberdiek/pdfcrypt.pdf\\
%   pdfcrypt.dtx & source/latex/oberdiek/pdfcrypt.dtx\\
% \end{tabular}^^A
% }^^A
% \sbox0{\t}^^A
% \ifdim\wd0>\linewidth
%   \begingroup
%     \advance\linewidth by\leftmargin
%     \advance\linewidth by\rightmargin
%   \edef\x{\endgroup
%     \def\noexpand\lw{\the\linewidth}^^A
%   }\x
%   \def\lwbox{^^A
%     \leavevmode
%     \hbox to \linewidth{^^A
%       \kern-\leftmargin\relax
%       \hss
%       \usebox0
%       \hss
%       \kern-\rightmargin\relax
%     }^^A
%   }^^A
%   \ifdim\wd0>\lw
%     \sbox0{\small\t}^^A
%     \ifdim\wd0>\linewidth
%       \ifdim\wd0>\lw
%         \sbox0{\footnotesize\t}^^A
%         \ifdim\wd0>\linewidth
%           \ifdim\wd0>\lw
%             \sbox0{\scriptsize\t}^^A
%             \ifdim\wd0>\linewidth
%               \ifdim\wd0>\lw
%                 \sbox0{\tiny\t}^^A
%                 \ifdim\wd0>\linewidth
%                   \lwbox
%                 \else
%                   \usebox0
%                 \fi
%               \else
%                 \lwbox
%               \fi
%             \else
%               \usebox0
%             \fi
%           \else
%             \lwbox
%           \fi
%         \else
%           \usebox0
%         \fi
%       \else
%         \lwbox
%       \fi
%     \else
%       \usebox0
%     \fi
%   \else
%     \lwbox
%   \fi
% \else
%   \usebox0
% \fi
% \end{quote}
% If you have a \xfile{docstrip.cfg} that configures and enables \docstrip's
% TDS installing feature, then some files can already be in the right
% place, see the documentation of \docstrip.
%
% \subsection{Refresh file name databases}
%
% If your \TeX~distribution
% (\TeX\,Live, \mikTeX, \dots) relies on file name databases, you must refresh
% these. For example, \TeX\,Live\ users run \verb|texhash| or
% \verb|mktexlsr|.
%
% \subsection{Some details for the interested}
%
% \paragraph{Unpacking with \LaTeX.}
% The \xfile{.dtx} chooses its action depending on the format:
% \begin{description}
% \item[\plainTeX:] Run \docstrip\ and extract the files.
% \item[\LaTeX:] Generate the documentation.
% \end{description}
% If you insist on using \LaTeX\ for \docstrip\ (really,
% \docstrip\ does not need \LaTeX), then inform the autodetect routine
% about your intention:
% \begin{quote}
%   \verb|latex \let\install=y\input{pdfcrypt.dtx}|
% \end{quote}
% Do not forget to quote the argument according to the demands
% of your shell.
%
% \paragraph{Generating the documentation.}
% You can use both the \xfile{.dtx} or the \xfile{.drv} to generate
% the documentation. The process can be configured by the
% configuration file \xfile{ltxdoc.cfg}. For instance, put this
% line into this file, if you want to have A4 as paper format:
% \begin{quote}
%   \verb|\PassOptionsToClass{a4paper}{article}|
% \end{quote}
% An example follows how to generate the
% documentation with pdf\LaTeX:
% \begin{quote}
%\begin{verbatim}
%pdflatex pdfcrypt.dtx
%makeindex -s gind.ist pdfcrypt.idx
%pdflatex pdfcrypt.dtx
%makeindex -s gind.ist pdfcrypt.idx
%pdflatex pdfcrypt.dtx
%\end{verbatim}
% \end{quote}
%
% \begin{History}
%   \begin{Version}{2001/04/02 v0.1}
%   \item
%     First public version, published in the pdftex
%     mailing list for testing with pdfcrypt-20010331
%   \end{Version}
%   \begin{Version}{2001/07/19 v0.2}
%   \item
%     Default: all allowed.
%   \item
%     Support for \VTeX\ added.
%   \end{Version}
%   \begin{Version}{2001/07/19 v0.3}
%   \item
%     Bug fix: \VTeX\ letter for edit is M (modify).
%   \end{Version}
%   \begin{Version}{2001/07/19 v0.4}
%   \item
%     Bug fix: \cs{VTeXversion} is correct after
%     regenerating the format file.
%   \end{Version}
%   \begin{Version}{2001/08/05 v0.5}
%   \item
%     Syntax change in \pdfTeX\ 1.00a.
%   \end{Version}
%   \begin{Version}{2001/08/09 v0.6}
%   \item
%     Support of special characters:\\
%     input: |\{|, |\}|, |\\| for |{|, |}|, |\|\\
%     output: |"| in \VTeX
%   \item
%     Option debug added.
%   \end{Version}
%   \begin{Version}{2001/10/28 v0.7}
%   \item
%     Plain compatibility.
%   \item
%     \cs{nopdfcrypt} added.
%   \item
%     Typos corrected.
%   \end{Version}
%   \begin{Version}{2006/02/20 v0.8}
%   \item
%     Obsolete remarks for \pdfTeX.
%   \item
%     DTX framework.
%   \item
%     LPPL 1.3
%   \end{Version}
%   \begin{Version}{2007/04/11 v0.9}
%   \item
%     Line ends sanitized.
%   \end{Version}
%   \begin{Version}{2007/04/26 v1.0}
%   \item
%     Use of package \xpackage{infwarerr}.
%   \end{Version}
%   \begin{Version}{2016/05/16 v1.1}
%   \item
%     Documentation updates.
%   \end{Version}
% \end{History}
%
% \PrintIndex
%
% \Finale
\endinput

%        (quote the arguments according to the demands of your shell)
%
% Documentation:
%    (a) If pdfcrypt.drv is present:
%           latex pdfcrypt.drv
%    (b) Without pdfcrypt.drv:
%           latex pdfcrypt.dtx; ...
%    The class ltxdoc loads the configuration file ltxdoc.cfg
%    if available. Here you can specify further options, e.g.
%    use A4 as paper format:
%       \PassOptionsToClass{a4paper}{article}
%
%    Programm calls to get the documentation (example):
%       pdflatex pdfcrypt.dtx
%       makeindex -s gind.ist pdfcrypt.idx
%       pdflatex pdfcrypt.dtx
%       makeindex -s gind.ist pdfcrypt.idx
%       pdflatex pdfcrypt.dtx
%
% Installation:
%    TDS:tex/generic/oberdiek/pdfcrypt.sty
%    TDS:doc/latex/oberdiek/pdfcrypt.pdf
%    TDS:source/latex/oberdiek/pdfcrypt.dtx
%
%<*ignore>
\begingroup
  \catcode123=1 %
  \catcode125=2 %
  \def\x{LaTeX2e}%
\expandafter\endgroup
\ifcase 0\ifx\install y1\fi\expandafter
         \ifx\csname processbatchFile\endcsname\relax\else1\fi
         \ifx\fmtname\x\else 1\fi\relax
\else\csname fi\endcsname
%</ignore>
%<*install>
\input docstrip.tex
\Msg{************************************************************************}
\Msg{* Installation}
\Msg{* Package: pdfcrypt 2016/05/16 v1.1 Allows the setting of PDF encryption (HO)}
\Msg{************************************************************************}

\keepsilent
\askforoverwritefalse

\let\MetaPrefix\relax
\preamble

This is a generated file.

Project: pdfcrypt
Version: 2016/05/16 v1.1

Copyright (C)
   2001, 2006, 2007 Heiko Oberdiek
   2016-2019 Oberdiek Package Support Group

This work may be distributed and/or modified under the
conditions of the LaTeX Project Public License, either
version 1.3c of this license or (at your option) any later
version. This version of this license is in
   https://www.latex-project.org/lppl/lppl-1-3c.txt
and the latest version of this license is in
   https://www.latex-project.org/lppl.txt
and version 1.3 or later is part of all distributions of
LaTeX version 2005/12/01 or later.

This work has the LPPL maintenance status "maintained".

The Current Maintainers of this work are
Heiko Oberdiek and the Oberdiek Package Support Group
https://github.com/ho-tex/oberdiek/issues


The Base Interpreter refers to any `TeX-Format',
because some files are installed in TDS:tex/generic//.

This work consists of the main source file pdfcrypt.dtx
and the derived files
   pdfcrypt.sty, pdfcrypt.pdf, pdfcrypt.ins, pdfcrypt.drv.

\endpreamble
\let\MetaPrefix\DoubleperCent

\generate{%
  \file{pdfcrypt.ins}{\from{pdfcrypt.dtx}{install}}%
  \file{pdfcrypt.drv}{\from{pdfcrypt.dtx}{driver}}%
  \usedir{tex/generic/oberdiek}%
  \file{pdfcrypt.sty}{\from{pdfcrypt.dtx}{package}}%
  \nopreamble
  \nopostamble
%  \usedir{source/latex/oberdiek/catalogue}%
%  \file{pdfcrypt.xml}{\from{pdfcrypt.dtx}{catalogue}}%
}

\catcode32=13\relax% active space
\let =\space%
\Msg{************************************************************************}
\Msg{*}
\Msg{* To finish the installation you have to move the following}
\Msg{* file into a directory searched by TeX:}
\Msg{*}
\Msg{*     pdfcrypt.sty}
\Msg{*}
\Msg{* To produce the documentation run the file `pdfcrypt.drv'}
\Msg{* through LaTeX.}
\Msg{*}
\Msg{* Happy TeXing!}
\Msg{*}
\Msg{************************************************************************}

\endbatchfile
%</install>
%<*ignore>
\fi
%</ignore>
%<*driver>
\NeedsTeXFormat{LaTeX2e}
\ProvidesFile{pdfcrypt.drv}%
  [2016/05/16 v1.1 Allows the setting of PDF encryption (HO)]%
\documentclass{ltxdoc}
\usepackage{holtxdoc}[2011/11/22]
\begin{document}
  \DocInput{pdfcrypt.dtx}%
\end{document}
%</driver>
% \fi
%
%
% \CharacterTable
%  {Upper-case    \A\B\C\D\E\F\G\H\I\J\K\L\M\N\O\P\Q\R\S\T\U\V\W\X\Y\Z
%   Lower-case    \a\b\c\d\e\f\g\h\i\j\k\l\m\n\o\p\q\r\s\t\u\v\w\x\y\z
%   Digits        \0\1\2\3\4\5\6\7\8\9
%   Exclamation   \!     Double quote  \"     Hash (number) \#
%   Dollar        \$     Percent       \%     Ampersand     \&
%   Acute accent  \'     Left paren    \(     Right paren   \)
%   Asterisk      \*     Plus          \+     Comma         \,
%   Minus         \-     Point         \.     Solidus       \/
%   Colon         \:     Semicolon     \;     Less than     \<
%   Equals        \=     Greater than  \>     Question mark \?
%   Commercial at \@     Left bracket  \[     Backslash     \\
%   Right bracket \]     Circumflex    \^     Underscore    \_
%   Grave accent  \`     Left brace    \{     Vertical bar  \|
%   Right brace   \}     Tilde         \~}
%
% \GetFileInfo{pdfcrypt.drv}
%
% \title{The \xpackage{pdfcrypt} package}
% \date{2016/05/16 v1.1}
% \author{Heiko Oberdiek\thanks
% {Please report any issues at \url{https://github.com/ho-tex/oberdiek/issues}}}
%
% \maketitle
%
% \begin{abstract}
% This package supports the setting of pdf encryption options
% for \VTeX\ and some older versions of \pdfTeX.
% \end{abstract}
%
% \tableofcontents
%
% \section{Documentation}
%
% This package allows the setting of pdf encryption
% options for
% \begin{itemize}
% \item
%   \VTeX, since version 7.35, \url{http://www.micropress-inc.com/}.
% \item
%   \pdfTeX, patched by
%   \NameEmail{Ricardo Sanchez Carmenes}{carmenes@bioquimica.uniovi.es}.\\
%   Some supported versions are:
%   \begin{quote}
%     \pdfTeX-1.00a,\\
%     \pdfTeX-0.14h-pdfcrypt-20010310,\\
%     pdfcrypt-20010331
%   \end{quote}
% \end{itemize}
%
% \noindent
% \textcolor{red}{Note:} Since \pdfTeX-1.10a (2003-01-16) encryption
% support was dropped! Thus the package is now obsolete for recent
% \pdfTeX\ versions.
%
% \subsection{Alternatives}
%
% There are free alternatives, programs that can
% be used for postprocessing the pdf file:
% \begin{itemize}
% \item \textsf{pdftk}\\
%   \url{http://www.accesspdf.com/pdftk/}
% \item \textsf{Multivalent}\\
%   \url{http://multivalent.sourceforge.net/}
% \item \textsf{PDFBox}\\
%   \url{http://www.pdfbox.org/}
% \item \textsf{PDFTrans}\\
%   \url{http://maddingue.free.fr/softwares/pdftrans.html}
% \item \dots
% \end{itemize}
%
% \subsection{Usage}
%
% The options can be set as package options or
% with the command \cs{pdfcryptsetup}:
% \begin{quote}
%   |owner=|\meta{owner password}\\
%   |user=|\meta{user password}\\
%   |print|, |copy|, |edit|, |annotate=true|, |false|\\
%   |all|, |none|\\
%   |debug|
% \end{quote}
% The encryption is set at |\begin{document}| by default,
% but this can be forced for an earlier time
% using the option |set|.
% Example:
% \begin{quote}
%   |\usepackage[owner=Donald,user=Knuth,print=false]{pdfcrypt}|\\
%   \dots\\
%   |\begin{document}|
% \end{quote}
% or
% \begin{quote}
%   |\usepackage{pdfcrypt}|\\
%   |\pdfcryptsetup{owner=Donald,user=Knut}|\\
%   |\pdfcryptsetup{print=false,set}|\\
%   \dots\\
%   |\begin{document}|
% \end{quote}
%
% \subsubsection{Special characters}
%
% The characters `|{|', `|}|', and `|\|' have to be
% escaped with `|\|'. Then the string should not
% be specified as package option but with
% the macro \cs{pdfcryptsetup}, eg:
% \begin{quote}
%   |\pdfcryptsetup{user={\{\\Hello\}},print=false}|
% \end{quote}
% The password here is ``|{\Hello}|''.
% Active characters can be used and are not expanded.
% Macros except for |\{|, |\}|, and |\\| should not be used
% and are not expanded.
%
% \subsubsection{\cs{nopdfcrypt}}
%
% Often the whole process of pdf generation includes
% several \TeX\ compilations and perhaps other
% postprocessing steps. Therefore a feature would
% be useful to disable the encryption stuff in order
% to speed up the first compilations.
% Therefore package \xpackage{pdfcrypt} will look for
% the macro \cs{nopdfcrypt}. If it has the meaning of the
% letter `|Y|', the package will be disabled.
% Example:
% \begin{quote}
%   |pdflatex '\let\nopdfcrypt=Y\input myfile'|\\
%   |thumbpdf myfile|\\
%   |pdfelatex myfile|
% \end{quote}
%
% \subsubsection{Configuration file}
%
% If the file \xfile{pdfcrypt.cfg} exists it is loaded
% and can be used for common settings.
%
% \subsubsection{Support for \plainTeX}
%
% The package can also be used with \plainTeX. It defines the
% missing \LaTeX\ macros and loads package \xpackage{keyval}.
% The configuration file is not supported and automatically
% loaded.
%
% \subsection{Hints for \pdfTeX}
%
% This section is out of date, because the encryption support
% was removed from \pdfTeX. Therefore this hints are of
% historical interest only.
%
%\begin{verbatim}
%* There are packages such as thumbpdf that ship out
%  some stuff (\immediate\pdfobj{...}). In these cases
%  \pdfcrypt will be ignored without warning or error
%  message. Therefore the package pdfcrypt should be
%  loaded before and option "set" should be used in
%  order to force the call of \pdfcrypt, for example:
%     \usepackage{pdfcrypt}
%     \pdfcryptsetup{..., set}
%     ...
%     \usepackage{thumbpdf}
%
%* Compiling pdfTeX (eg. version 1.00a-pretest-20010804):
%  pdfTeX versions are available at
%      ftp://ftp.cstug.cz/pub/tex/local/cstug/thanh/pdftex/snapshots/
%  Unpack the latest .tgz file:
%      > tar xzf pdftex-20010804.tgz
%  Enable encryption support:
%      src> cd src/texk/web2c/pdftexdir
%      src/texk/web2c/pdftexdir> ln -fs pdfcrypt-full.c pdfcrypt.c
%  Compile:
%      src/texk/web2c/pdftexdir> cd ../../..
%      src> ./configure
%      src> cd texk/web2c
%      src/texk/web2c> make pdftexbin
%  At last install the binary and pool files.
%\end{verbatim}
%
% \StopEventually{
% }
%
% \section{Implementation}
%
%    \begin{macrocode}
%<*package>
%    \end{macrocode}
%
% \subsection{Help macros for \plainTeX}
%
%    \begin{macrocode}
\expandafter\edef\csname pc@endinput\endcsname{%
  \catcode`\noexpand\@=\the\catcode`\@\relax
  \noexpand\endinput
}
\catcode`\@=11 %

\expandafter\ifx\csname @firstoftwo\endcsname\relax
  \long\def\@firstoftwo#1#2{#1}%
\fi

\expandafter\ifx\csname @secondoftwo\endcsname\relax
  \long\def\@secondoftwo#1#2{#2}%
\fi

\expandafter\ifx\csname @ifundefined\endcsname\relax
  \def\@ifundefined#1{%
    \expandafter\ifx\csname #1\endcsname\relax
      \expandafter\@firstoftwo
    \else
      \expandafter\@secondoftwo
    \fi
  }%
\fi

\@ifundefined{@ifnextchar}{%
  \long\def\@ifnextchar#1#2#3{%
    \let\reserved@d=#1%
    \def\reserved@a{#2}%
    \def\reserved@b{#3}%
    \futurelet\@let@token\@ifnch
  }%
  \def\@ifnch{%
    \ifx\@let@token\@sptoken
      \let\reserved@c\@xifnch
    \else
      \ifx\@let@token\reserved@d
        \let\reserved@c\reserved@a
      \else
        \let\reserved@c\reserved@b
      \fi
    \fi
    \reserved@c
  }%
  \begingroup
    \def\:{\global\let\@sptoken= }%
    \: %
    \def\:{\@xifnch}%
    \expandafter\gdef\: {\futurelet\@let@token\@ifnch}%
  \endgroup
}{}

\@ifundefined{ProvidesPackage}{%
  \def\ProvidesPackage#1{%
    \@ifnextchar[{\pc@ProvidesPackage{#1}}%
                 {\pc@ProvidesPackage{#1}[]}%
  }%
  \def\pc@ProvidesPackage#1[#2]{%
    \immediate\write-1{Package: #1 #2}%
  }%
}{}

\begingroup\expandafter\expandafter\expandafter\endgroup
\expandafter\ifx\csname RequirePackage\endcsname\relax
  \input infwarerr.sty\relax
\else
  \RequirePackage{infwarerr}%
\fi

\@ifundefined{@gobble}{%
  \long\def\@gobble#1{}%
}{}

\@ifundefined{@empty}{%
  \def\@empty{}%
}{}
%    \end{macrocode}
%
% \subsection{Package Identification and checks}
%
%    \begin{macrocode}
\ProvidesPackage{pdfcrypt}%
  [2016/05/16 v1.1 Allows the setting of PDF encryption (HO)]%
%    \end{macrocode}
%
%    \begin{macrocode}
\@ifundefined{pdfcryptsetup}{%
  \let\pdfcryptsetup\@gobble
}{%
  \@PackageWarningNoLine{pdfcrypt}{Package pdfcrypt is already loaded}%
  \pc@endinput
}
%    \end{macrocode}
%    Support for \cs{nopdfcrypt}.
%    \begin{macrocode}
\newif\ifpc@nopdfcrypt
\ifx Y\nopdfcrypt
  \@PackageWarningNoLine{pdfcrypt}{%
    Encryption disabled by \string \nopdfcrypt\space request%
  }%
  \global\pc@nopdfcrypttrue
\fi
%    \end{macrocode}
%
% \subsection{Driver detection}
%
%    \begin{macrocode}
\let\pc@driver\@empty
\begingroup
  % pdfTeX detection
  \@ifundefined{pdfoutput}{%
  }{%
    \ifcase\pdfoutput
    \else
      \@ifundefined{pdfcrypt}{%
        \@PackageError{pdfcrypt}{%
          PDF encryption is not supported with this pdfTeX%
        }{%
          Encryption support was added in 0.14h (2001/03/10)\MessageBreak
          and removed in 1.10a (2003/01/16).%
        }%
        \endgroup
        \pc@endinput
      }{%
        \gdef\pc@driver{pdftex}%
      }%
    \fi
  }%
  % VTeX detection
  \@ifundefined{OpMode}{%
  }{%
    \ifnum\OpMode=1 %
      \ifnum\@ifundefined{VTeXversion}0\VTeXversion<735 %
        \@PackageError{pdfcrypt}{%
          PDF encryption is not supported with this VTeX%
        }{%
          You need VTeX 7.35 or higher.%
        }%
        \endgroup
        \pc@endinput
      \else
        \gdef\pc@driver{vtex}%
      \fi
    \fi
  }%
\endgroup
%    \end{macrocode}
%
% \subsection{Load package \xpackage{keyval}}
%
%    \begin{macrocode}
\@ifundefined{@makeother}{%
  \def\@makeother#1{\catcode`#1=12\relax}%
}{}

\@ifundefined{g@addto@macro}{%
  \long\def\g@addto@macro#1#2{%
    \begingroup
      \toks@\expandafter{#1#2}%
      \xdef#1{\the\toks@}%
    \endgroup
  }%
}{}

\@ifundefined{@namedef}{%
  \def\@namedef#1{\expandafter\def\csname#1\endcsname}%
}{}

\@ifundefined{@nameuse}{%
  \def\@nameuse#1{\csname #1\endcsname}%
}{}

\def\pc@KeyvalRestore{%
  \let\pc@KeyvalRestore\@undefined
}

\let\pcOrg@NeedsTeXFormat\NeedsTeXFormat
\@ifundefined{NeedsTeXFormat}{%
  \def\NeedsTeXFormat#1{}%
  \g@addto@macro\pc@KeyvalRestore{%
    \let\NeedsTeXFormat\pcOrg@NeedsTeXFormat
  }%
}{}

\let\pcOrg@DeclareOption\DeclareOption
\@ifundefined{DeclareOption}{%
  \def\DeclareOption#1#2{#2}%
  \g@addto@macro\pc@KeyvalRestore{%
    \let\DeclareOption\pcOrg@DeclareOption
  }%
}{}

\let\pcOrg@ExecuteOptions\ExecuteOptions
\@ifundefined{ExecuteOptions}{%
  \def\ExecuteOptions#1{}%
  \g@addto@macro\pc@KeyvalRestore{%
    \let\ExecuteOptions\pcOrg@ExecuteOptions
  }%
}{}

\let\pcOrg@ProcessOptions\ProcessOptions
\@ifundefined{ProcessOptions}{%
  \def\ProcessOptions{}%
  \g@addto@macro\pc@KeyvalRestore{%
    \let\ProcessOptions\pcOrg@ProcessOptions
  }%
}{}%

\begingroup\expandafter\expandafter\expandafter\endgroup
\expandafter\ifx\csname RequirePackage\endcsname\relax
  \input keyval.sty\relax
\else
  \RequirePackage{keyval}%
\fi
\pc@KeyvalRestore
%    \end{macrocode}
%
% \subsection{Define options}
%
%    \begin{macrocode}
\@ifundefined{@dblarg}{%
  \long\def\@dblarg#1{\@ifnextchar[{#1}{\@xdblarg{#1}}}%
  \long\def\@xdblarg#1#2{#1[{#2}]{#2}}%
}{}

\newif\ifpc@set
\newif\ifpc@print
\newif\ifpc@copy
\newif\ifpc@edit
\newif\ifpc@annotate
\newif\ifpc@debug
\let\pc@owner\@empty
\let\pc@user\@empty

% default: allow all
\pc@printtrue
\pc@copytrue
\pc@edittrue
\pc@annotatetrue

\edef\pc@temp{\catcode`\noexpand\"=\the\catcode`\"\relax}
\@makeother\"
\def\pc@set{%
  \@PackageInfo{pdfcrypt}{%
    \ifpc@debug
      \ifx\pc@owner\@empty
        No owner password%
      \else
        Owner password: `\pc@owner'%
      \fi
      \MessageBreak
      \ifx\pc@user\@empty
        No user password%
      \else
        User password: `\pc@user'%
      \fi
      \MessageBreak
      Flags: %
      \ifpc@print   \else no\fi print, %
      \ifpc@copy    \else no\fi copy, %
      \ifpc@edit    \else no\fi edit, %
      \ifpc@annotate\else no\fi annotate%
      \MessageBreak
    \fi
    \ifpc@nopdfcrypt
      Encryption is disabled by `\string\nopdfcrypt'%
    \else
      Encryption is set for `\pc@driver'%
    \fi
  }%
  \ifpc@nopdfcrypt
  \else
    \@ifundefined{pc@set@\pc@driver}{%
      \ifx\pc@driver\@empty
        \@PackageError{pdfcrypt}{No driver for encryption %
          support found}\@ehc
      \else
        \@PackageError{pdfcrypt}{Cannot set encryption for %
          unknown driver `\pc@driver'}\@ehc
      \fi
    }{%
      \@nameuse{pc@set@\pc@driver}%
    }%
  \fi
}
\def\pc@set@pdftex{%
  \ifnum\pdftexversion<100 %
    \pc@set@pdftexold
  \else
    \pc@set@pdftexnew
  \fi
}
\def\pc@set@pdftexold{%
  \pdfcrypt{%
    owner "\pc@owner" %
    user "\pc@user" %
    \ifpc@print   \else no\fi print %
    \ifpc@copy    \else no\fi copy %
    \ifpc@edit    \else no\fi edit %
    \ifpc@annotate\else no\fi annotate%
  }%
}
\def\pc@set@pdftexnew{%
  \pdfcrypt
    owner{\pc@owner}%
    user{\pc@user}%
    \ifpc@print   \else no\fi print %
    \ifpc@copy    \else no\fi copy %
    \ifpc@edit    \else no\fi edit %
    \ifpc@annotate\else no\fi annotate%
  \relax
}
\def\pc@set@vtex{%
  \immediate\special{!security %
    O=\pc@MakeVTeXString\pc@owner,%
    U=\pc@MakeVTeXString\pc@user,%
    P\ifpc@print +\else -\fi,%
    C\ifpc@copy +\else -\fi,%
    M\ifpc@edit +\else -\fi,%
    A\ifpc@annotate +\else -\fi
  }%
}
\def\pc@MakeVTeXString#1{%
  "\expandafter\pc@@MakeVTeXString#1"\@nil"%
}
\def\pc@@MakeVTeXString#1"#2\@nil{%
  #1%
  \ifx\\#2\\%
  \else
    ""%
    \@ReturnAfterFi{%
      \pc@@MakeVTeXString#2\@nil
    }%
  \fi
}
\long\def\@ReturnAfterFi#1\fi{\fi#1}
\pc@temp

\begingroup
  \catcode`\ =12 \gdef\pc@spaceother{ }\catcode`\ =10\relax
  \catcode`\|=0 %
  \catcode`\\=12 %
  |gdef|pc@DefString#1#2{%
    |def#1{#2}%
    |edef#1{|expandafter|strip@prefix|meaning#1}%
    |edef#1{|expandafter|pc@SpaceToOther#1 |@nil}%
    |edef#1{|expandafter|pc@EscapeRemove#1|@empty\|@empty|@nil}%
  }%
  |gdef|pc@EscapeRemove#1\#2#3|@nil{%
    #1#2%
    |ifx|\#3|\%%
    |else
      |@ReturnAfterFi{%
        |pc@EscapeRemove#3|@nil
      }%
    |fi
  }%
|endgroup
\def\pc@SpaceToOther#1 #2\@nil{%
  #1%
  \ifx\\#2\\%
  \else
    \pc@spaceother
    \@ReturnAfterFi{%
      \pc@SpaceToOther#2\@nil
    }%
  \fi
}

\def\pc@boolkey{\@dblarg\pc@@boolkey}
\def\pc@@boolkey[#1]#2#3{%
  \lowercase{\def\pc@temp{#3}}%
  \ifx\pc@temp\@empty
    \let\pc@temp\pc@true
  \fi
  \ifx\pc@temp\pc@true
  \else
    \ifx\pc@temp\pc@false
    \else
      \let\pc@temp\relax
    \fi
  \fi
  \ifx\pc@temp\relax
    \@PackageWarning{pdfcrypt}{%
      Unexpected value \string`#3\string' of %
      option \string`#2\string'\MessageBreak
      instead of %
      \string`true\string' or \string`false\string'%
    }%
  \else
    \csname pc@#2\pc@temp\endcsname
  \fi
}
\def\pc@true{true}
\def\pc@false{false}

\define@key{pc}{set}[true]{%
  \pc@boolkey{set}{#1}%
}
\define@key{pc}{pdftex}[]{%
  \def\pc@driver{pdftex}%
}
\define@key{pc}{vtex}[]{%
  \def\pc@driver{vtex}%
}
\define@key{pc}{print}[true]{%
  \pc@boolkey{print}{#1}%
}
\define@key{pc}{copy}[true]{%
  \pc@boolkey{copy}{#1}%
}
\define@key{pc}{edit}[true]{%
  \pc@boolkey{edit}{#1}%
}
\define@key{pc}{annotate}[true]{%
  \pc@boolkey{annotate}{#1}%
}
\define@key{pc}{all}[]{%
  \pc@boolkey{print}{true}%
  \pc@boolkey{copy}{true}%
  \pc@boolkey{edit}{true}%
  \pc@boolkey{annotate}{true}%
}
\define@key{pc}{none}[]{%
  \pc@boolkey{print}{false}%
  \pc@boolkey{copy}{false}%
  \pc@boolkey{edit}{false}%
  \pc@boolkey{annotate}{false}%
}

\define@key{pc}{owner}{%
  \pc@DefString\pc@owner{#1}%
}
\define@key{pc}{user}{%
  \pc@DefString\pc@user{#1}%
}
\define@key{pc}{debug}[true]{%
  \pc@boolkey{debug}{#1}%
}

\def\pdfcryptsetup#1{%
  \setkeys{pc}{#1}%
  \ifpc@set
    \pc@set
    \global\let\pc@set\relax
    \gdef\pdfcryptsetup##1{%
      \@PackageWarning{pdfcrypt}{%
        Encryption options are already set\MessageBreak
        new values are ignored%
      }%
    }%
  \fi
}
\begingroup\expandafter\expandafter\expandafter\endgroup
\expandafter\ifx\csname @onlypreamble\endcsname\relax
\else
  \@onlypreamble\pdfcryptsetup
\fi
%    \end{macrocode}
%
% \subsection{support of configuration file}
%    \begin{macrocode}
\begingroup\expandafter\expandafter\expandafter\endgroup
\expandafter\ifx\csname InputIfFileExists\endcsname\relax
  \@PackageInfo{pdfcrypt}{%
    Configuration file pdfcrypt.cfg not supported.%
  }%
\else
  \let\pc@ExecuteOptions\ExecuteOptions
  \InputIfFileExists{pdfcrypt.cfg}{}{}%
  \let\ExecuteOptions\pc@ExecuteOptions
\fi
%    \end{macrocode}
%
% \subsection{Package options}
%
%    Plain format does not know package options.
%    \begin{macrocode}
\begingroup\expandafter\expandafter\expandafter\endgroup
\expandafter\ifx\csname @classoptionslist\endcsname\relax
  \expandafter\pc@endinput
\fi
%    \end{macrocode}
%    Process global and local options.
%    \begin{macrocode}
\def\pc@ProcessOptionsWithKV{%
  \let\pc@temp\@empty
  \@for\CurrentOption:=\@classoptionslist\do{%
    \@ifundefined{KV@pc@\CurrentOption}{}{%
      \edef\pc@temp{\pc@temp,\CurrentOption,}%
      \@expandtwoargs\@removeelement\CurrentOption
        \@unusedoptionlist\@unusedoptionlist
    }%
  }%
  \edef\pc@temp{%
    \noexpand\pdfcryptsetup{%
      \pc@temp\@ptionlist{\@currname.\@currext}%
    }%
  }%
  \pc@temp
}
\pc@ProcessOptionsWithKV
\AtEndOfPackage{\let\@unprocessedoptions\relax}
\AtBeginDocument{\pc@set}

\pc@endinput
%    \end{macrocode}
%
%    \begin{macrocode}
%</package>
%    \end{macrocode}
%
% \section{Installation}
%
% \subsection{Download}
%
% \paragraph{Package.} This package is available on
% CTAN\footnote{\CTANpkg{pdfcrypt}}:
% \begin{description}
% \item[\CTAN{macros/latex/contrib/oberdiek/pdfcrypt.dtx}] The source file.
% \item[\CTAN{macros/latex/contrib/oberdiek/pdfcrypt.pdf}] Documentation.
% \end{description}
%
%
% \paragraph{Bundle.} All the packages of the bundle `oberdiek'
% are also available in a TDS compliant ZIP archive. There
% the packages are already unpacked and the documentation files
% are generated. The files and directories obey the TDS standard.
% \begin{description}
% \item[\CTANinstall{install/macros/latex/contrib/oberdiek.tds.zip}]
% \end{description}
% \emph{TDS} refers to the standard ``A Directory Structure
% for \TeX\ Files'' (\CTANpkg{tds}). Directories
% with \xfile{texmf} in their name are usually organized this way.
%
% \subsection{Bundle installation}
%
% \paragraph{Unpacking.} Unpack the \xfile{oberdiek.tds.zip} in the
% TDS tree (also known as \xfile{texmf} tree) of your choice.
% Example (linux):
% \begin{quote}
%   |unzip oberdiek.tds.zip -d ~/texmf|
% \end{quote}
%
% \subsection{Package installation}
%
% \paragraph{Unpacking.} The \xfile{.dtx} file is a self-extracting
% \docstrip\ archive. The files are extracted by running the
% \xfile{.dtx} through \plainTeX:
% \begin{quote}
%   \verb|tex pdfcrypt.dtx|
% \end{quote}
%
% \paragraph{TDS.} Now the different files must be moved into
% the different directories in your installation TDS tree
% (also known as \xfile{texmf} tree):
% \begin{quote}
% \def\t{^^A
% \begin{tabular}{@{}>{\ttfamily}l@{ $\rightarrow$ }>{\ttfamily}l@{}}
%   pdfcrypt.sty & tex/generic/oberdiek/pdfcrypt.sty\\
%   pdfcrypt.pdf & doc/latex/oberdiek/pdfcrypt.pdf\\
%   pdfcrypt.dtx & source/latex/oberdiek/pdfcrypt.dtx\\
% \end{tabular}^^A
% }^^A
% \sbox0{\t}^^A
% \ifdim\wd0>\linewidth
%   \begingroup
%     \advance\linewidth by\leftmargin
%     \advance\linewidth by\rightmargin
%   \edef\x{\endgroup
%     \def\noexpand\lw{\the\linewidth}^^A
%   }\x
%   \def\lwbox{^^A
%     \leavevmode
%     \hbox to \linewidth{^^A
%       \kern-\leftmargin\relax
%       \hss
%       \usebox0
%       \hss
%       \kern-\rightmargin\relax
%     }^^A
%   }^^A
%   \ifdim\wd0>\lw
%     \sbox0{\small\t}^^A
%     \ifdim\wd0>\linewidth
%       \ifdim\wd0>\lw
%         \sbox0{\footnotesize\t}^^A
%         \ifdim\wd0>\linewidth
%           \ifdim\wd0>\lw
%             \sbox0{\scriptsize\t}^^A
%             \ifdim\wd0>\linewidth
%               \ifdim\wd0>\lw
%                 \sbox0{\tiny\t}^^A
%                 \ifdim\wd0>\linewidth
%                   \lwbox
%                 \else
%                   \usebox0
%                 \fi
%               \else
%                 \lwbox
%               \fi
%             \else
%               \usebox0
%             \fi
%           \else
%             \lwbox
%           \fi
%         \else
%           \usebox0
%         \fi
%       \else
%         \lwbox
%       \fi
%     \else
%       \usebox0
%     \fi
%   \else
%     \lwbox
%   \fi
% \else
%   \usebox0
% \fi
% \end{quote}
% If you have a \xfile{docstrip.cfg} that configures and enables \docstrip's
% TDS installing feature, then some files can already be in the right
% place, see the documentation of \docstrip.
%
% \subsection{Refresh file name databases}
%
% If your \TeX~distribution
% (\TeX\,Live, \mikTeX, \dots) relies on file name databases, you must refresh
% these. For example, \TeX\,Live\ users run \verb|texhash| or
% \verb|mktexlsr|.
%
% \subsection{Some details for the interested}
%
% \paragraph{Unpacking with \LaTeX.}
% The \xfile{.dtx} chooses its action depending on the format:
% \begin{description}
% \item[\plainTeX:] Run \docstrip\ and extract the files.
% \item[\LaTeX:] Generate the documentation.
% \end{description}
% If you insist on using \LaTeX\ for \docstrip\ (really,
% \docstrip\ does not need \LaTeX), then inform the autodetect routine
% about your intention:
% \begin{quote}
%   \verb|latex \let\install=y% \iffalse meta-comment
%
% File: pdfcrypt.dtx
% Version: 2016/05/16 v1.1
% Info: Allows the setting of PDF encryption
%
% Copyright (C)
%    2001, 2006, 2007 Heiko Oberdiek
%    2016-2019 Oberdiek Package Support Group
%    https://github.com/ho-tex/oberdiek/issues
%
% This work may be distributed and/or modified under the
% conditions of the LaTeX Project Public License, either
% version 1.3c of this license or (at your option) any later
% version. This version of this license is in
%    https://www.latex-project.org/lppl/lppl-1-3c.txt
% and the latest version of this license is in
%    https://www.latex-project.org/lppl.txt
% and version 1.3 or later is part of all distributions of
% LaTeX version 2005/12/01 or later.
%
% This work has the LPPL maintenance status "maintained".
%
% The Current Maintainers of this work are
% Heiko Oberdiek and the Oberdiek Package Support Group
% https://github.com/ho-tex/oberdiek/issues
%
% The Base Interpreter refers to any `TeX-Format',
% because some files are installed in TDS:tex/generic//.
%
% This work consists of the main source file pdfcrypt.dtx
% and the derived files
%    pdfcrypt.sty, pdfcrypt.pdf, pdfcrypt.ins, pdfcrypt.drv.
%
% Distribution:
%    CTAN:macros/latex/contrib/oberdiek/pdfcrypt.dtx
%    CTAN:macros/latex/contrib/oberdiek/pdfcrypt.pdf
%
% Unpacking:
%    (a) If pdfcrypt.ins is present:
%           tex pdfcrypt.ins
%    (b) Without pdfcrypt.ins:
%           tex pdfcrypt.dtx
%    (c) If you insist on using LaTeX
%           latex \let\install=y\input{pdfcrypt.dtx}
%        (quote the arguments according to the demands of your shell)
%
% Documentation:
%    (a) If pdfcrypt.drv is present:
%           latex pdfcrypt.drv
%    (b) Without pdfcrypt.drv:
%           latex pdfcrypt.dtx; ...
%    The class ltxdoc loads the configuration file ltxdoc.cfg
%    if available. Here you can specify further options, e.g.
%    use A4 as paper format:
%       \PassOptionsToClass{a4paper}{article}
%
%    Programm calls to get the documentation (example):
%       pdflatex pdfcrypt.dtx
%       makeindex -s gind.ist pdfcrypt.idx
%       pdflatex pdfcrypt.dtx
%       makeindex -s gind.ist pdfcrypt.idx
%       pdflatex pdfcrypt.dtx
%
% Installation:
%    TDS:tex/generic/oberdiek/pdfcrypt.sty
%    TDS:doc/latex/oberdiek/pdfcrypt.pdf
%    TDS:source/latex/oberdiek/pdfcrypt.dtx
%
%<*ignore>
\begingroup
  \catcode123=1 %
  \catcode125=2 %
  \def\x{LaTeX2e}%
\expandafter\endgroup
\ifcase 0\ifx\install y1\fi\expandafter
         \ifx\csname processbatchFile\endcsname\relax\else1\fi
         \ifx\fmtname\x\else 1\fi\relax
\else\csname fi\endcsname
%</ignore>
%<*install>
\input docstrip.tex
\Msg{************************************************************************}
\Msg{* Installation}
\Msg{* Package: pdfcrypt 2016/05/16 v1.1 Allows the setting of PDF encryption (HO)}
\Msg{************************************************************************}

\keepsilent
\askforoverwritefalse

\let\MetaPrefix\relax
\preamble

This is a generated file.

Project: pdfcrypt
Version: 2016/05/16 v1.1

Copyright (C)
   2001, 2006, 2007 Heiko Oberdiek
   2016-2019 Oberdiek Package Support Group

This work may be distributed and/or modified under the
conditions of the LaTeX Project Public License, either
version 1.3c of this license or (at your option) any later
version. This version of this license is in
   https://www.latex-project.org/lppl/lppl-1-3c.txt
and the latest version of this license is in
   https://www.latex-project.org/lppl.txt
and version 1.3 or later is part of all distributions of
LaTeX version 2005/12/01 or later.

This work has the LPPL maintenance status "maintained".

The Current Maintainers of this work are
Heiko Oberdiek and the Oberdiek Package Support Group
https://github.com/ho-tex/oberdiek/issues


The Base Interpreter refers to any `TeX-Format',
because some files are installed in TDS:tex/generic//.

This work consists of the main source file pdfcrypt.dtx
and the derived files
   pdfcrypt.sty, pdfcrypt.pdf, pdfcrypt.ins, pdfcrypt.drv.

\endpreamble
\let\MetaPrefix\DoubleperCent

\generate{%
  \file{pdfcrypt.ins}{\from{pdfcrypt.dtx}{install}}%
  \file{pdfcrypt.drv}{\from{pdfcrypt.dtx}{driver}}%
  \usedir{tex/generic/oberdiek}%
  \file{pdfcrypt.sty}{\from{pdfcrypt.dtx}{package}}%
  \nopreamble
  \nopostamble
%  \usedir{source/latex/oberdiek/catalogue}%
%  \file{pdfcrypt.xml}{\from{pdfcrypt.dtx}{catalogue}}%
}

\catcode32=13\relax% active space
\let =\space%
\Msg{************************************************************************}
\Msg{*}
\Msg{* To finish the installation you have to move the following}
\Msg{* file into a directory searched by TeX:}
\Msg{*}
\Msg{*     pdfcrypt.sty}
\Msg{*}
\Msg{* To produce the documentation run the file `pdfcrypt.drv'}
\Msg{* through LaTeX.}
\Msg{*}
\Msg{* Happy TeXing!}
\Msg{*}
\Msg{************************************************************************}

\endbatchfile
%</install>
%<*ignore>
\fi
%</ignore>
%<*driver>
\NeedsTeXFormat{LaTeX2e}
\ProvidesFile{pdfcrypt.drv}%
  [2016/05/16 v1.1 Allows the setting of PDF encryption (HO)]%
\documentclass{ltxdoc}
\usepackage{holtxdoc}[2011/11/22]
\begin{document}
  \DocInput{pdfcrypt.dtx}%
\end{document}
%</driver>
% \fi
%
%
% \CharacterTable
%  {Upper-case    \A\B\C\D\E\F\G\H\I\J\K\L\M\N\O\P\Q\R\S\T\U\V\W\X\Y\Z
%   Lower-case    \a\b\c\d\e\f\g\h\i\j\k\l\m\n\o\p\q\r\s\t\u\v\w\x\y\z
%   Digits        \0\1\2\3\4\5\6\7\8\9
%   Exclamation   \!     Double quote  \"     Hash (number) \#
%   Dollar        \$     Percent       \%     Ampersand     \&
%   Acute accent  \'     Left paren    \(     Right paren   \)
%   Asterisk      \*     Plus          \+     Comma         \,
%   Minus         \-     Point         \.     Solidus       \/
%   Colon         \:     Semicolon     \;     Less than     \<
%   Equals        \=     Greater than  \>     Question mark \?
%   Commercial at \@     Left bracket  \[     Backslash     \\
%   Right bracket \]     Circumflex    \^     Underscore    \_
%   Grave accent  \`     Left brace    \{     Vertical bar  \|
%   Right brace   \}     Tilde         \~}
%
% \GetFileInfo{pdfcrypt.drv}
%
% \title{The \xpackage{pdfcrypt} package}
% \date{2016/05/16 v1.1}
% \author{Heiko Oberdiek\thanks
% {Please report any issues at \url{https://github.com/ho-tex/oberdiek/issues}}}
%
% \maketitle
%
% \begin{abstract}
% This package supports the setting of pdf encryption options
% for \VTeX\ and some older versions of \pdfTeX.
% \end{abstract}
%
% \tableofcontents
%
% \section{Documentation}
%
% This package allows the setting of pdf encryption
% options for
% \begin{itemize}
% \item
%   \VTeX, since version 7.35, \url{http://www.micropress-inc.com/}.
% \item
%   \pdfTeX, patched by
%   \NameEmail{Ricardo Sanchez Carmenes}{carmenes@bioquimica.uniovi.es}.\\
%   Some supported versions are:
%   \begin{quote}
%     \pdfTeX-1.00a,\\
%     \pdfTeX-0.14h-pdfcrypt-20010310,\\
%     pdfcrypt-20010331
%   \end{quote}
% \end{itemize}
%
% \noindent
% \textcolor{red}{Note:} Since \pdfTeX-1.10a (2003-01-16) encryption
% support was dropped! Thus the package is now obsolete for recent
% \pdfTeX\ versions.
%
% \subsection{Alternatives}
%
% There are free alternatives, programs that can
% be used for postprocessing the pdf file:
% \begin{itemize}
% \item \textsf{pdftk}\\
%   \url{http://www.accesspdf.com/pdftk/}
% \item \textsf{Multivalent}\\
%   \url{http://multivalent.sourceforge.net/}
% \item \textsf{PDFBox}\\
%   \url{http://www.pdfbox.org/}
% \item \textsf{PDFTrans}\\
%   \url{http://maddingue.free.fr/softwares/pdftrans.html}
% \item \dots
% \end{itemize}
%
% \subsection{Usage}
%
% The options can be set as package options or
% with the command \cs{pdfcryptsetup}:
% \begin{quote}
%   |owner=|\meta{owner password}\\
%   |user=|\meta{user password}\\
%   |print|, |copy|, |edit|, |annotate=true|, |false|\\
%   |all|, |none|\\
%   |debug|
% \end{quote}
% The encryption is set at |\begin{document}| by default,
% but this can be forced for an earlier time
% using the option |set|.
% Example:
% \begin{quote}
%   |\usepackage[owner=Donald,user=Knuth,print=false]{pdfcrypt}|\\
%   \dots\\
%   |\begin{document}|
% \end{quote}
% or
% \begin{quote}
%   |\usepackage{pdfcrypt}|\\
%   |\pdfcryptsetup{owner=Donald,user=Knut}|\\
%   |\pdfcryptsetup{print=false,set}|\\
%   \dots\\
%   |\begin{document}|
% \end{quote}
%
% \subsubsection{Special characters}
%
% The characters `|{|', `|}|', and `|\|' have to be
% escaped with `|\|'. Then the string should not
% be specified as package option but with
% the macro \cs{pdfcryptsetup}, eg:
% \begin{quote}
%   |\pdfcryptsetup{user={\{\\Hello\}},print=false}|
% \end{quote}
% The password here is ``|{\Hello}|''.
% Active characters can be used and are not expanded.
% Macros except for |\{|, |\}|, and |\\| should not be used
% and are not expanded.
%
% \subsubsection{\cs{nopdfcrypt}}
%
% Often the whole process of pdf generation includes
% several \TeX\ compilations and perhaps other
% postprocessing steps. Therefore a feature would
% be useful to disable the encryption stuff in order
% to speed up the first compilations.
% Therefore package \xpackage{pdfcrypt} will look for
% the macro \cs{nopdfcrypt}. If it has the meaning of the
% letter `|Y|', the package will be disabled.
% Example:
% \begin{quote}
%   |pdflatex '\let\nopdfcrypt=Y\input myfile'|\\
%   |thumbpdf myfile|\\
%   |pdfelatex myfile|
% \end{quote}
%
% \subsubsection{Configuration file}
%
% If the file \xfile{pdfcrypt.cfg} exists it is loaded
% and can be used for common settings.
%
% \subsubsection{Support for \plainTeX}
%
% The package can also be used with \plainTeX. It defines the
% missing \LaTeX\ macros and loads package \xpackage{keyval}.
% The configuration file is not supported and automatically
% loaded.
%
% \subsection{Hints for \pdfTeX}
%
% This section is out of date, because the encryption support
% was removed from \pdfTeX. Therefore this hints are of
% historical interest only.
%
%\begin{verbatim}
%* There are packages such as thumbpdf that ship out
%  some stuff (\immediate\pdfobj{...}). In these cases
%  \pdfcrypt will be ignored without warning or error
%  message. Therefore the package pdfcrypt should be
%  loaded before and option "set" should be used in
%  order to force the call of \pdfcrypt, for example:
%     \usepackage{pdfcrypt}
%     \pdfcryptsetup{..., set}
%     ...
%     \usepackage{thumbpdf}
%
%* Compiling pdfTeX (eg. version 1.00a-pretest-20010804):
%  pdfTeX versions are available at
%      ftp://ftp.cstug.cz/pub/tex/local/cstug/thanh/pdftex/snapshots/
%  Unpack the latest .tgz file:
%      > tar xzf pdftex-20010804.tgz
%  Enable encryption support:
%      src> cd src/texk/web2c/pdftexdir
%      src/texk/web2c/pdftexdir> ln -fs pdfcrypt-full.c pdfcrypt.c
%  Compile:
%      src/texk/web2c/pdftexdir> cd ../../..
%      src> ./configure
%      src> cd texk/web2c
%      src/texk/web2c> make pdftexbin
%  At last install the binary and pool files.
%\end{verbatim}
%
% \StopEventually{
% }
%
% \section{Implementation}
%
%    \begin{macrocode}
%<*package>
%    \end{macrocode}
%
% \subsection{Help macros for \plainTeX}
%
%    \begin{macrocode}
\expandafter\edef\csname pc@endinput\endcsname{%
  \catcode`\noexpand\@=\the\catcode`\@\relax
  \noexpand\endinput
}
\catcode`\@=11 %

\expandafter\ifx\csname @firstoftwo\endcsname\relax
  \long\def\@firstoftwo#1#2{#1}%
\fi

\expandafter\ifx\csname @secondoftwo\endcsname\relax
  \long\def\@secondoftwo#1#2{#2}%
\fi

\expandafter\ifx\csname @ifundefined\endcsname\relax
  \def\@ifundefined#1{%
    \expandafter\ifx\csname #1\endcsname\relax
      \expandafter\@firstoftwo
    \else
      \expandafter\@secondoftwo
    \fi
  }%
\fi

\@ifundefined{@ifnextchar}{%
  \long\def\@ifnextchar#1#2#3{%
    \let\reserved@d=#1%
    \def\reserved@a{#2}%
    \def\reserved@b{#3}%
    \futurelet\@let@token\@ifnch
  }%
  \def\@ifnch{%
    \ifx\@let@token\@sptoken
      \let\reserved@c\@xifnch
    \else
      \ifx\@let@token\reserved@d
        \let\reserved@c\reserved@a
      \else
        \let\reserved@c\reserved@b
      \fi
    \fi
    \reserved@c
  }%
  \begingroup
    \def\:{\global\let\@sptoken= }%
    \: %
    \def\:{\@xifnch}%
    \expandafter\gdef\: {\futurelet\@let@token\@ifnch}%
  \endgroup
}{}

\@ifundefined{ProvidesPackage}{%
  \def\ProvidesPackage#1{%
    \@ifnextchar[{\pc@ProvidesPackage{#1}}%
                 {\pc@ProvidesPackage{#1}[]}%
  }%
  \def\pc@ProvidesPackage#1[#2]{%
    \immediate\write-1{Package: #1 #2}%
  }%
}{}

\begingroup\expandafter\expandafter\expandafter\endgroup
\expandafter\ifx\csname RequirePackage\endcsname\relax
  \input infwarerr.sty\relax
\else
  \RequirePackage{infwarerr}%
\fi

\@ifundefined{@gobble}{%
  \long\def\@gobble#1{}%
}{}

\@ifundefined{@empty}{%
  \def\@empty{}%
}{}
%    \end{macrocode}
%
% \subsection{Package Identification and checks}
%
%    \begin{macrocode}
\ProvidesPackage{pdfcrypt}%
  [2016/05/16 v1.1 Allows the setting of PDF encryption (HO)]%
%    \end{macrocode}
%
%    \begin{macrocode}
\@ifundefined{pdfcryptsetup}{%
  \let\pdfcryptsetup\@gobble
}{%
  \@PackageWarningNoLine{pdfcrypt}{Package pdfcrypt is already loaded}%
  \pc@endinput
}
%    \end{macrocode}
%    Support for \cs{nopdfcrypt}.
%    \begin{macrocode}
\newif\ifpc@nopdfcrypt
\ifx Y\nopdfcrypt
  \@PackageWarningNoLine{pdfcrypt}{%
    Encryption disabled by \string \nopdfcrypt\space request%
  }%
  \global\pc@nopdfcrypttrue
\fi
%    \end{macrocode}
%
% \subsection{Driver detection}
%
%    \begin{macrocode}
\let\pc@driver\@empty
\begingroup
  % pdfTeX detection
  \@ifundefined{pdfoutput}{%
  }{%
    \ifcase\pdfoutput
    \else
      \@ifundefined{pdfcrypt}{%
        \@PackageError{pdfcrypt}{%
          PDF encryption is not supported with this pdfTeX%
        }{%
          Encryption support was added in 0.14h (2001/03/10)\MessageBreak
          and removed in 1.10a (2003/01/16).%
        }%
        \endgroup
        \pc@endinput
      }{%
        \gdef\pc@driver{pdftex}%
      }%
    \fi
  }%
  % VTeX detection
  \@ifundefined{OpMode}{%
  }{%
    \ifnum\OpMode=1 %
      \ifnum\@ifundefined{VTeXversion}0\VTeXversion<735 %
        \@PackageError{pdfcrypt}{%
          PDF encryption is not supported with this VTeX%
        }{%
          You need VTeX 7.35 or higher.%
        }%
        \endgroup
        \pc@endinput
      \else
        \gdef\pc@driver{vtex}%
      \fi
    \fi
  }%
\endgroup
%    \end{macrocode}
%
% \subsection{Load package \xpackage{keyval}}
%
%    \begin{macrocode}
\@ifundefined{@makeother}{%
  \def\@makeother#1{\catcode`#1=12\relax}%
}{}

\@ifundefined{g@addto@macro}{%
  \long\def\g@addto@macro#1#2{%
    \begingroup
      \toks@\expandafter{#1#2}%
      \xdef#1{\the\toks@}%
    \endgroup
  }%
}{}

\@ifundefined{@namedef}{%
  \def\@namedef#1{\expandafter\def\csname#1\endcsname}%
}{}

\@ifundefined{@nameuse}{%
  \def\@nameuse#1{\csname #1\endcsname}%
}{}

\def\pc@KeyvalRestore{%
  \let\pc@KeyvalRestore\@undefined
}

\let\pcOrg@NeedsTeXFormat\NeedsTeXFormat
\@ifundefined{NeedsTeXFormat}{%
  \def\NeedsTeXFormat#1{}%
  \g@addto@macro\pc@KeyvalRestore{%
    \let\NeedsTeXFormat\pcOrg@NeedsTeXFormat
  }%
}{}

\let\pcOrg@DeclareOption\DeclareOption
\@ifundefined{DeclareOption}{%
  \def\DeclareOption#1#2{#2}%
  \g@addto@macro\pc@KeyvalRestore{%
    \let\DeclareOption\pcOrg@DeclareOption
  }%
}{}

\let\pcOrg@ExecuteOptions\ExecuteOptions
\@ifundefined{ExecuteOptions}{%
  \def\ExecuteOptions#1{}%
  \g@addto@macro\pc@KeyvalRestore{%
    \let\ExecuteOptions\pcOrg@ExecuteOptions
  }%
}{}

\let\pcOrg@ProcessOptions\ProcessOptions
\@ifundefined{ProcessOptions}{%
  \def\ProcessOptions{}%
  \g@addto@macro\pc@KeyvalRestore{%
    \let\ProcessOptions\pcOrg@ProcessOptions
  }%
}{}%

\begingroup\expandafter\expandafter\expandafter\endgroup
\expandafter\ifx\csname RequirePackage\endcsname\relax
  \input keyval.sty\relax
\else
  \RequirePackage{keyval}%
\fi
\pc@KeyvalRestore
%    \end{macrocode}
%
% \subsection{Define options}
%
%    \begin{macrocode}
\@ifundefined{@dblarg}{%
  \long\def\@dblarg#1{\@ifnextchar[{#1}{\@xdblarg{#1}}}%
  \long\def\@xdblarg#1#2{#1[{#2}]{#2}}%
}{}

\newif\ifpc@set
\newif\ifpc@print
\newif\ifpc@copy
\newif\ifpc@edit
\newif\ifpc@annotate
\newif\ifpc@debug
\let\pc@owner\@empty
\let\pc@user\@empty

% default: allow all
\pc@printtrue
\pc@copytrue
\pc@edittrue
\pc@annotatetrue

\edef\pc@temp{\catcode`\noexpand\"=\the\catcode`\"\relax}
\@makeother\"
\def\pc@set{%
  \@PackageInfo{pdfcrypt}{%
    \ifpc@debug
      \ifx\pc@owner\@empty
        No owner password%
      \else
        Owner password: `\pc@owner'%
      \fi
      \MessageBreak
      \ifx\pc@user\@empty
        No user password%
      \else
        User password: `\pc@user'%
      \fi
      \MessageBreak
      Flags: %
      \ifpc@print   \else no\fi print, %
      \ifpc@copy    \else no\fi copy, %
      \ifpc@edit    \else no\fi edit, %
      \ifpc@annotate\else no\fi annotate%
      \MessageBreak
    \fi
    \ifpc@nopdfcrypt
      Encryption is disabled by `\string\nopdfcrypt'%
    \else
      Encryption is set for `\pc@driver'%
    \fi
  }%
  \ifpc@nopdfcrypt
  \else
    \@ifundefined{pc@set@\pc@driver}{%
      \ifx\pc@driver\@empty
        \@PackageError{pdfcrypt}{No driver for encryption %
          support found}\@ehc
      \else
        \@PackageError{pdfcrypt}{Cannot set encryption for %
          unknown driver `\pc@driver'}\@ehc
      \fi
    }{%
      \@nameuse{pc@set@\pc@driver}%
    }%
  \fi
}
\def\pc@set@pdftex{%
  \ifnum\pdftexversion<100 %
    \pc@set@pdftexold
  \else
    \pc@set@pdftexnew
  \fi
}
\def\pc@set@pdftexold{%
  \pdfcrypt{%
    owner "\pc@owner" %
    user "\pc@user" %
    \ifpc@print   \else no\fi print %
    \ifpc@copy    \else no\fi copy %
    \ifpc@edit    \else no\fi edit %
    \ifpc@annotate\else no\fi annotate%
  }%
}
\def\pc@set@pdftexnew{%
  \pdfcrypt
    owner{\pc@owner}%
    user{\pc@user}%
    \ifpc@print   \else no\fi print %
    \ifpc@copy    \else no\fi copy %
    \ifpc@edit    \else no\fi edit %
    \ifpc@annotate\else no\fi annotate%
  \relax
}
\def\pc@set@vtex{%
  \immediate\special{!security %
    O=\pc@MakeVTeXString\pc@owner,%
    U=\pc@MakeVTeXString\pc@user,%
    P\ifpc@print +\else -\fi,%
    C\ifpc@copy +\else -\fi,%
    M\ifpc@edit +\else -\fi,%
    A\ifpc@annotate +\else -\fi
  }%
}
\def\pc@MakeVTeXString#1{%
  "\expandafter\pc@@MakeVTeXString#1"\@nil"%
}
\def\pc@@MakeVTeXString#1"#2\@nil{%
  #1%
  \ifx\\#2\\%
  \else
    ""%
    \@ReturnAfterFi{%
      \pc@@MakeVTeXString#2\@nil
    }%
  \fi
}
\long\def\@ReturnAfterFi#1\fi{\fi#1}
\pc@temp

\begingroup
  \catcode`\ =12 \gdef\pc@spaceother{ }\catcode`\ =10\relax
  \catcode`\|=0 %
  \catcode`\\=12 %
  |gdef|pc@DefString#1#2{%
    |def#1{#2}%
    |edef#1{|expandafter|strip@prefix|meaning#1}%
    |edef#1{|expandafter|pc@SpaceToOther#1 |@nil}%
    |edef#1{|expandafter|pc@EscapeRemove#1|@empty\|@empty|@nil}%
  }%
  |gdef|pc@EscapeRemove#1\#2#3|@nil{%
    #1#2%
    |ifx|\#3|\%%
    |else
      |@ReturnAfterFi{%
        |pc@EscapeRemove#3|@nil
      }%
    |fi
  }%
|endgroup
\def\pc@SpaceToOther#1 #2\@nil{%
  #1%
  \ifx\\#2\\%
  \else
    \pc@spaceother
    \@ReturnAfterFi{%
      \pc@SpaceToOther#2\@nil
    }%
  \fi
}

\def\pc@boolkey{\@dblarg\pc@@boolkey}
\def\pc@@boolkey[#1]#2#3{%
  \lowercase{\def\pc@temp{#3}}%
  \ifx\pc@temp\@empty
    \let\pc@temp\pc@true
  \fi
  \ifx\pc@temp\pc@true
  \else
    \ifx\pc@temp\pc@false
    \else
      \let\pc@temp\relax
    \fi
  \fi
  \ifx\pc@temp\relax
    \@PackageWarning{pdfcrypt}{%
      Unexpected value \string`#3\string' of %
      option \string`#2\string'\MessageBreak
      instead of %
      \string`true\string' or \string`false\string'%
    }%
  \else
    \csname pc@#2\pc@temp\endcsname
  \fi
}
\def\pc@true{true}
\def\pc@false{false}

\define@key{pc}{set}[true]{%
  \pc@boolkey{set}{#1}%
}
\define@key{pc}{pdftex}[]{%
  \def\pc@driver{pdftex}%
}
\define@key{pc}{vtex}[]{%
  \def\pc@driver{vtex}%
}
\define@key{pc}{print}[true]{%
  \pc@boolkey{print}{#1}%
}
\define@key{pc}{copy}[true]{%
  \pc@boolkey{copy}{#1}%
}
\define@key{pc}{edit}[true]{%
  \pc@boolkey{edit}{#1}%
}
\define@key{pc}{annotate}[true]{%
  \pc@boolkey{annotate}{#1}%
}
\define@key{pc}{all}[]{%
  \pc@boolkey{print}{true}%
  \pc@boolkey{copy}{true}%
  \pc@boolkey{edit}{true}%
  \pc@boolkey{annotate}{true}%
}
\define@key{pc}{none}[]{%
  \pc@boolkey{print}{false}%
  \pc@boolkey{copy}{false}%
  \pc@boolkey{edit}{false}%
  \pc@boolkey{annotate}{false}%
}

\define@key{pc}{owner}{%
  \pc@DefString\pc@owner{#1}%
}
\define@key{pc}{user}{%
  \pc@DefString\pc@user{#1}%
}
\define@key{pc}{debug}[true]{%
  \pc@boolkey{debug}{#1}%
}

\def\pdfcryptsetup#1{%
  \setkeys{pc}{#1}%
  \ifpc@set
    \pc@set
    \global\let\pc@set\relax
    \gdef\pdfcryptsetup##1{%
      \@PackageWarning{pdfcrypt}{%
        Encryption options are already set\MessageBreak
        new values are ignored%
      }%
    }%
  \fi
}
\begingroup\expandafter\expandafter\expandafter\endgroup
\expandafter\ifx\csname @onlypreamble\endcsname\relax
\else
  \@onlypreamble\pdfcryptsetup
\fi
%    \end{macrocode}
%
% \subsection{support of configuration file}
%    \begin{macrocode}
\begingroup\expandafter\expandafter\expandafter\endgroup
\expandafter\ifx\csname InputIfFileExists\endcsname\relax
  \@PackageInfo{pdfcrypt}{%
    Configuration file pdfcrypt.cfg not supported.%
  }%
\else
  \let\pc@ExecuteOptions\ExecuteOptions
  \InputIfFileExists{pdfcrypt.cfg}{}{}%
  \let\ExecuteOptions\pc@ExecuteOptions
\fi
%    \end{macrocode}
%
% \subsection{Package options}
%
%    Plain format does not know package options.
%    \begin{macrocode}
\begingroup\expandafter\expandafter\expandafter\endgroup
\expandafter\ifx\csname @classoptionslist\endcsname\relax
  \expandafter\pc@endinput
\fi
%    \end{macrocode}
%    Process global and local options.
%    \begin{macrocode}
\def\pc@ProcessOptionsWithKV{%
  \let\pc@temp\@empty
  \@for\CurrentOption:=\@classoptionslist\do{%
    \@ifundefined{KV@pc@\CurrentOption}{}{%
      \edef\pc@temp{\pc@temp,\CurrentOption,}%
      \@expandtwoargs\@removeelement\CurrentOption
        \@unusedoptionlist\@unusedoptionlist
    }%
  }%
  \edef\pc@temp{%
    \noexpand\pdfcryptsetup{%
      \pc@temp\@ptionlist{\@currname.\@currext}%
    }%
  }%
  \pc@temp
}
\pc@ProcessOptionsWithKV
\AtEndOfPackage{\let\@unprocessedoptions\relax}
\AtBeginDocument{\pc@set}

\pc@endinput
%    \end{macrocode}
%
%    \begin{macrocode}
%</package>
%    \end{macrocode}
%
% \section{Installation}
%
% \subsection{Download}
%
% \paragraph{Package.} This package is available on
% CTAN\footnote{\CTANpkg{pdfcrypt}}:
% \begin{description}
% \item[\CTAN{macros/latex/contrib/oberdiek/pdfcrypt.dtx}] The source file.
% \item[\CTAN{macros/latex/contrib/oberdiek/pdfcrypt.pdf}] Documentation.
% \end{description}
%
%
% \paragraph{Bundle.} All the packages of the bundle `oberdiek'
% are also available in a TDS compliant ZIP archive. There
% the packages are already unpacked and the documentation files
% are generated. The files and directories obey the TDS standard.
% \begin{description}
% \item[\CTANinstall{install/macros/latex/contrib/oberdiek.tds.zip}]
% \end{description}
% \emph{TDS} refers to the standard ``A Directory Structure
% for \TeX\ Files'' (\CTANpkg{tds}). Directories
% with \xfile{texmf} in their name are usually organized this way.
%
% \subsection{Bundle installation}
%
% \paragraph{Unpacking.} Unpack the \xfile{oberdiek.tds.zip} in the
% TDS tree (also known as \xfile{texmf} tree) of your choice.
% Example (linux):
% \begin{quote}
%   |unzip oberdiek.tds.zip -d ~/texmf|
% \end{quote}
%
% \subsection{Package installation}
%
% \paragraph{Unpacking.} The \xfile{.dtx} file is a self-extracting
% \docstrip\ archive. The files are extracted by running the
% \xfile{.dtx} through \plainTeX:
% \begin{quote}
%   \verb|tex pdfcrypt.dtx|
% \end{quote}
%
% \paragraph{TDS.} Now the different files must be moved into
% the different directories in your installation TDS tree
% (also known as \xfile{texmf} tree):
% \begin{quote}
% \def\t{^^A
% \begin{tabular}{@{}>{\ttfamily}l@{ $\rightarrow$ }>{\ttfamily}l@{}}
%   pdfcrypt.sty & tex/generic/oberdiek/pdfcrypt.sty\\
%   pdfcrypt.pdf & doc/latex/oberdiek/pdfcrypt.pdf\\
%   pdfcrypt.dtx & source/latex/oberdiek/pdfcrypt.dtx\\
% \end{tabular}^^A
% }^^A
% \sbox0{\t}^^A
% \ifdim\wd0>\linewidth
%   \begingroup
%     \advance\linewidth by\leftmargin
%     \advance\linewidth by\rightmargin
%   \edef\x{\endgroup
%     \def\noexpand\lw{\the\linewidth}^^A
%   }\x
%   \def\lwbox{^^A
%     \leavevmode
%     \hbox to \linewidth{^^A
%       \kern-\leftmargin\relax
%       \hss
%       \usebox0
%       \hss
%       \kern-\rightmargin\relax
%     }^^A
%   }^^A
%   \ifdim\wd0>\lw
%     \sbox0{\small\t}^^A
%     \ifdim\wd0>\linewidth
%       \ifdim\wd0>\lw
%         \sbox0{\footnotesize\t}^^A
%         \ifdim\wd0>\linewidth
%           \ifdim\wd0>\lw
%             \sbox0{\scriptsize\t}^^A
%             \ifdim\wd0>\linewidth
%               \ifdim\wd0>\lw
%                 \sbox0{\tiny\t}^^A
%                 \ifdim\wd0>\linewidth
%                   \lwbox
%                 \else
%                   \usebox0
%                 \fi
%               \else
%                 \lwbox
%               \fi
%             \else
%               \usebox0
%             \fi
%           \else
%             \lwbox
%           \fi
%         \else
%           \usebox0
%         \fi
%       \else
%         \lwbox
%       \fi
%     \else
%       \usebox0
%     \fi
%   \else
%     \lwbox
%   \fi
% \else
%   \usebox0
% \fi
% \end{quote}
% If you have a \xfile{docstrip.cfg} that configures and enables \docstrip's
% TDS installing feature, then some files can already be in the right
% place, see the documentation of \docstrip.
%
% \subsection{Refresh file name databases}
%
% If your \TeX~distribution
% (\TeX\,Live, \mikTeX, \dots) relies on file name databases, you must refresh
% these. For example, \TeX\,Live\ users run \verb|texhash| or
% \verb|mktexlsr|.
%
% \subsection{Some details for the interested}
%
% \paragraph{Unpacking with \LaTeX.}
% The \xfile{.dtx} chooses its action depending on the format:
% \begin{description}
% \item[\plainTeX:] Run \docstrip\ and extract the files.
% \item[\LaTeX:] Generate the documentation.
% \end{description}
% If you insist on using \LaTeX\ for \docstrip\ (really,
% \docstrip\ does not need \LaTeX), then inform the autodetect routine
% about your intention:
% \begin{quote}
%   \verb|latex \let\install=y\input{pdfcrypt.dtx}|
% \end{quote}
% Do not forget to quote the argument according to the demands
% of your shell.
%
% \paragraph{Generating the documentation.}
% You can use both the \xfile{.dtx} or the \xfile{.drv} to generate
% the documentation. The process can be configured by the
% configuration file \xfile{ltxdoc.cfg}. For instance, put this
% line into this file, if you want to have A4 as paper format:
% \begin{quote}
%   \verb|\PassOptionsToClass{a4paper}{article}|
% \end{quote}
% An example follows how to generate the
% documentation with pdf\LaTeX:
% \begin{quote}
%\begin{verbatim}
%pdflatex pdfcrypt.dtx
%makeindex -s gind.ist pdfcrypt.idx
%pdflatex pdfcrypt.dtx
%makeindex -s gind.ist pdfcrypt.idx
%pdflatex pdfcrypt.dtx
%\end{verbatim}
% \end{quote}
%
% \begin{History}
%   \begin{Version}{2001/04/02 v0.1}
%   \item
%     First public version, published in the pdftex
%     mailing list for testing with pdfcrypt-20010331
%   \end{Version}
%   \begin{Version}{2001/07/19 v0.2}
%   \item
%     Default: all allowed.
%   \item
%     Support for \VTeX\ added.
%   \end{Version}
%   \begin{Version}{2001/07/19 v0.3}
%   \item
%     Bug fix: \VTeX\ letter for edit is M (modify).
%   \end{Version}
%   \begin{Version}{2001/07/19 v0.4}
%   \item
%     Bug fix: \cs{VTeXversion} is correct after
%     regenerating the format file.
%   \end{Version}
%   \begin{Version}{2001/08/05 v0.5}
%   \item
%     Syntax change in \pdfTeX\ 1.00a.
%   \end{Version}
%   \begin{Version}{2001/08/09 v0.6}
%   \item
%     Support of special characters:\\
%     input: |\{|, |\}|, |\\| for |{|, |}|, |\|\\
%     output: |"| in \VTeX
%   \item
%     Option debug added.
%   \end{Version}
%   \begin{Version}{2001/10/28 v0.7}
%   \item
%     Plain compatibility.
%   \item
%     \cs{nopdfcrypt} added.
%   \item
%     Typos corrected.
%   \end{Version}
%   \begin{Version}{2006/02/20 v0.8}
%   \item
%     Obsolete remarks for \pdfTeX.
%   \item
%     DTX framework.
%   \item
%     LPPL 1.3
%   \end{Version}
%   \begin{Version}{2007/04/11 v0.9}
%   \item
%     Line ends sanitized.
%   \end{Version}
%   \begin{Version}{2007/04/26 v1.0}
%   \item
%     Use of package \xpackage{infwarerr}.
%   \end{Version}
%   \begin{Version}{2016/05/16 v1.1}
%   \item
%     Documentation updates.
%   \end{Version}
% \end{History}
%
% \PrintIndex
%
% \Finale
\endinput
|
% \end{quote}
% Do not forget to quote the argument according to the demands
% of your shell.
%
% \paragraph{Generating the documentation.}
% You can use both the \xfile{.dtx} or the \xfile{.drv} to generate
% the documentation. The process can be configured by the
% configuration file \xfile{ltxdoc.cfg}. For instance, put this
% line into this file, if you want to have A4 as paper format:
% \begin{quote}
%   \verb|\PassOptionsToClass{a4paper}{article}|
% \end{quote}
% An example follows how to generate the
% documentation with pdf\LaTeX:
% \begin{quote}
%\begin{verbatim}
%pdflatex pdfcrypt.dtx
%makeindex -s gind.ist pdfcrypt.idx
%pdflatex pdfcrypt.dtx
%makeindex -s gind.ist pdfcrypt.idx
%pdflatex pdfcrypt.dtx
%\end{verbatim}
% \end{quote}
%
% \begin{History}
%   \begin{Version}{2001/04/02 v0.1}
%   \item
%     First public version, published in the pdftex
%     mailing list for testing with pdfcrypt-20010331
%   \end{Version}
%   \begin{Version}{2001/07/19 v0.2}
%   \item
%     Default: all allowed.
%   \item
%     Support for \VTeX\ added.
%   \end{Version}
%   \begin{Version}{2001/07/19 v0.3}
%   \item
%     Bug fix: \VTeX\ letter for edit is M (modify).
%   \end{Version}
%   \begin{Version}{2001/07/19 v0.4}
%   \item
%     Bug fix: \cs{VTeXversion} is correct after
%     regenerating the format file.
%   \end{Version}
%   \begin{Version}{2001/08/05 v0.5}
%   \item
%     Syntax change in \pdfTeX\ 1.00a.
%   \end{Version}
%   \begin{Version}{2001/08/09 v0.6}
%   \item
%     Support of special characters:\\
%     input: |\{|, |\}|, |\\| for |{|, |}|, |\|\\
%     output: |"| in \VTeX
%   \item
%     Option debug added.
%   \end{Version}
%   \begin{Version}{2001/10/28 v0.7}
%   \item
%     Plain compatibility.
%   \item
%     \cs{nopdfcrypt} added.
%   \item
%     Typos corrected.
%   \end{Version}
%   \begin{Version}{2006/02/20 v0.8}
%   \item
%     Obsolete remarks for \pdfTeX.
%   \item
%     DTX framework.
%   \item
%     LPPL 1.3
%   \end{Version}
%   \begin{Version}{2007/04/11 v0.9}
%   \item
%     Line ends sanitized.
%   \end{Version}
%   \begin{Version}{2007/04/26 v1.0}
%   \item
%     Use of package \xpackage{infwarerr}.
%   \end{Version}
%   \begin{Version}{2016/05/16 v1.1}
%   \item
%     Documentation updates.
%   \end{Version}
% \end{History}
%
% \PrintIndex
%
% \Finale
\endinput

%        (quote the arguments according to the demands of your shell)
%
% Documentation:
%    (a) If pdfcrypt.drv is present:
%           latex pdfcrypt.drv
%    (b) Without pdfcrypt.drv:
%           latex pdfcrypt.dtx; ...
%    The class ltxdoc loads the configuration file ltxdoc.cfg
%    if available. Here you can specify further options, e.g.
%    use A4 as paper format:
%       \PassOptionsToClass{a4paper}{article}
%
%    Programm calls to get the documentation (example):
%       pdflatex pdfcrypt.dtx
%       makeindex -s gind.ist pdfcrypt.idx
%       pdflatex pdfcrypt.dtx
%       makeindex -s gind.ist pdfcrypt.idx
%       pdflatex pdfcrypt.dtx
%
% Installation:
%    TDS:tex/generic/oberdiek/pdfcrypt.sty
%    TDS:doc/latex/oberdiek/pdfcrypt.pdf
%    TDS:source/latex/oberdiek/pdfcrypt.dtx
%
%<*ignore>
\begingroup
  \catcode123=1 %
  \catcode125=2 %
  \def\x{LaTeX2e}%
\expandafter\endgroup
\ifcase 0\ifx\install y1\fi\expandafter
         \ifx\csname processbatchFile\endcsname\relax\else1\fi
         \ifx\fmtname\x\else 1\fi\relax
\else\csname fi\endcsname
%</ignore>
%<*install>
\input docstrip.tex
\Msg{************************************************************************}
\Msg{* Installation}
\Msg{* Package: pdfcrypt 2016/05/16 v1.1 Allows the setting of PDF encryption (HO)}
\Msg{************************************************************************}

\keepsilent
\askforoverwritefalse

\let\MetaPrefix\relax
\preamble

This is a generated file.

Project: pdfcrypt
Version: 2016/05/16 v1.1

Copyright (C)
   2001, 2006, 2007 Heiko Oberdiek
   2016-2019 Oberdiek Package Support Group

This work may be distributed and/or modified under the
conditions of the LaTeX Project Public License, either
version 1.3c of this license or (at your option) any later
version. This version of this license is in
   https://www.latex-project.org/lppl/lppl-1-3c.txt
and the latest version of this license is in
   https://www.latex-project.org/lppl.txt
and version 1.3 or later is part of all distributions of
LaTeX version 2005/12/01 or later.

This work has the LPPL maintenance status "maintained".

The Current Maintainers of this work are
Heiko Oberdiek and the Oberdiek Package Support Group
https://github.com/ho-tex/oberdiek/issues


The Base Interpreter refers to any `TeX-Format',
because some files are installed in TDS:tex/generic//.

This work consists of the main source file pdfcrypt.dtx
and the derived files
   pdfcrypt.sty, pdfcrypt.pdf, pdfcrypt.ins, pdfcrypt.drv.

\endpreamble
\let\MetaPrefix\DoubleperCent

\generate{%
  \file{pdfcrypt.ins}{\from{pdfcrypt.dtx}{install}}%
  \file{pdfcrypt.drv}{\from{pdfcrypt.dtx}{driver}}%
  \usedir{tex/generic/oberdiek}%
  \file{pdfcrypt.sty}{\from{pdfcrypt.dtx}{package}}%
  \nopreamble
  \nopostamble
%  \usedir{source/latex/oberdiek/catalogue}%
%  \file{pdfcrypt.xml}{\from{pdfcrypt.dtx}{catalogue}}%
}

\catcode32=13\relax% active space
\let =\space%
\Msg{************************************************************************}
\Msg{*}
\Msg{* To finish the installation you have to move the following}
\Msg{* file into a directory searched by TeX:}
\Msg{*}
\Msg{*     pdfcrypt.sty}
\Msg{*}
\Msg{* To produce the documentation run the file `pdfcrypt.drv'}
\Msg{* through LaTeX.}
\Msg{*}
\Msg{* Happy TeXing!}
\Msg{*}
\Msg{************************************************************************}

\endbatchfile
%</install>
%<*ignore>
\fi
%</ignore>
%<*driver>
\NeedsTeXFormat{LaTeX2e}
\ProvidesFile{pdfcrypt.drv}%
  [2016/05/16 v1.1 Allows the setting of PDF encryption (HO)]%
\documentclass{ltxdoc}
\usepackage{holtxdoc}[2011/11/22]
\begin{document}
  \DocInput{pdfcrypt.dtx}%
\end{document}
%</driver>
% \fi
%
%
% \CharacterTable
%  {Upper-case    \A\B\C\D\E\F\G\H\I\J\K\L\M\N\O\P\Q\R\S\T\U\V\W\X\Y\Z
%   Lower-case    \a\b\c\d\e\f\g\h\i\j\k\l\m\n\o\p\q\r\s\t\u\v\w\x\y\z
%   Digits        \0\1\2\3\4\5\6\7\8\9
%   Exclamation   \!     Double quote  \"     Hash (number) \#
%   Dollar        \$     Percent       \%     Ampersand     \&
%   Acute accent  \'     Left paren    \(     Right paren   \)
%   Asterisk      \*     Plus          \+     Comma         \,
%   Minus         \-     Point         \.     Solidus       \/
%   Colon         \:     Semicolon     \;     Less than     \<
%   Equals        \=     Greater than  \>     Question mark \?
%   Commercial at \@     Left bracket  \[     Backslash     \\
%   Right bracket \]     Circumflex    \^     Underscore    \_
%   Grave accent  \`     Left brace    \{     Vertical bar  \|
%   Right brace   \}     Tilde         \~}
%
% \GetFileInfo{pdfcrypt.drv}
%
% \title{The \xpackage{pdfcrypt} package}
% \date{2016/05/16 v1.1}
% \author{Heiko Oberdiek\thanks
% {Please report any issues at \url{https://github.com/ho-tex/oberdiek/issues}}}
%
% \maketitle
%
% \begin{abstract}
% This package supports the setting of pdf encryption options
% for \VTeX\ and some older versions of \pdfTeX.
% \end{abstract}
%
% \tableofcontents
%
% \section{Documentation}
%
% This package allows the setting of pdf encryption
% options for
% \begin{itemize}
% \item
%   \VTeX, since version 7.35, \url{http://www.micropress-inc.com/}.
% \item
%   \pdfTeX, patched by
%   \NameEmail{Ricardo Sanchez Carmenes}{carmenes@bioquimica.uniovi.es}.\\
%   Some supported versions are:
%   \begin{quote}
%     \pdfTeX-1.00a,\\
%     \pdfTeX-0.14h-pdfcrypt-20010310,\\
%     pdfcrypt-20010331
%   \end{quote}
% \end{itemize}
%
% \noindent
% \textcolor{red}{Note:} Since \pdfTeX-1.10a (2003-01-16) encryption
% support was dropped! Thus the package is now obsolete for recent
% \pdfTeX\ versions.
%
% \subsection{Alternatives}
%
% There are free alternatives, programs that can
% be used for postprocessing the pdf file:
% \begin{itemize}
% \item \textsf{pdftk}\\
%   \url{http://www.accesspdf.com/pdftk/}
% \item \textsf{Multivalent}\\
%   \url{http://multivalent.sourceforge.net/}
% \item \textsf{PDFBox}\\
%   \url{http://www.pdfbox.org/}
% \item \textsf{PDFTrans}\\
%   \url{http://maddingue.free.fr/softwares/pdftrans.html}
% \item \dots
% \end{itemize}
%
% \subsection{Usage}
%
% The options can be set as package options or
% with the command \cs{pdfcryptsetup}:
% \begin{quote}
%   |owner=|\meta{owner password}\\
%   |user=|\meta{user password}\\
%   |print|, |copy|, |edit|, |annotate=true|, |false|\\
%   |all|, |none|\\
%   |debug|
% \end{quote}
% The encryption is set at |\begin{document}| by default,
% but this can be forced for an earlier time
% using the option |set|.
% Example:
% \begin{quote}
%   |\usepackage[owner=Donald,user=Knuth,print=false]{pdfcrypt}|\\
%   \dots\\
%   |\begin{document}|
% \end{quote}
% or
% \begin{quote}
%   |\usepackage{pdfcrypt}|\\
%   |\pdfcryptsetup{owner=Donald,user=Knut}|\\
%   |\pdfcryptsetup{print=false,set}|\\
%   \dots\\
%   |\begin{document}|
% \end{quote}
%
% \subsubsection{Special characters}
%
% The characters `|{|', `|}|', and `|\|' have to be
% escaped with `|\|'. Then the string should not
% be specified as package option but with
% the macro \cs{pdfcryptsetup}, eg:
% \begin{quote}
%   |\pdfcryptsetup{user={\{\\Hello\}},print=false}|
% \end{quote}
% The password here is ``|{\Hello}|''.
% Active characters can be used and are not expanded.
% Macros except for |\{|, |\}|, and |\\| should not be used
% and are not expanded.
%
% \subsubsection{\cs{nopdfcrypt}}
%
% Often the whole process of pdf generation includes
% several \TeX\ compilations and perhaps other
% postprocessing steps. Therefore a feature would
% be useful to disable the encryption stuff in order
% to speed up the first compilations.
% Therefore package \xpackage{pdfcrypt} will look for
% the macro \cs{nopdfcrypt}. If it has the meaning of the
% letter `|Y|', the package will be disabled.
% Example:
% \begin{quote}
%   |pdflatex '\let\nopdfcrypt=Y\input myfile'|\\
%   |thumbpdf myfile|\\
%   |pdfelatex myfile|
% \end{quote}
%
% \subsubsection{Configuration file}
%
% If the file \xfile{pdfcrypt.cfg} exists it is loaded
% and can be used for common settings.
%
% \subsubsection{Support for \plainTeX}
%
% The package can also be used with \plainTeX. It defines the
% missing \LaTeX\ macros and loads package \xpackage{keyval}.
% The configuration file is not supported and automatically
% loaded.
%
% \subsection{Hints for \pdfTeX}
%
% This section is out of date, because the encryption support
% was removed from \pdfTeX. Therefore this hints are of
% historical interest only.
%
%\begin{verbatim}
%* There are packages such as thumbpdf that ship out
%  some stuff (\immediate\pdfobj{...}). In these cases
%  \pdfcrypt will be ignored without warning or error
%  message. Therefore the package pdfcrypt should be
%  loaded before and option "set" should be used in
%  order to force the call of \pdfcrypt, for example:
%     \usepackage{pdfcrypt}
%     \pdfcryptsetup{..., set}
%     ...
%     \usepackage{thumbpdf}
%
%* Compiling pdfTeX (eg. version 1.00a-pretest-20010804):
%  pdfTeX versions are available at
%      ftp://ftp.cstug.cz/pub/tex/local/cstug/thanh/pdftex/snapshots/
%  Unpack the latest .tgz file:
%      > tar xzf pdftex-20010804.tgz
%  Enable encryption support:
%      src> cd src/texk/web2c/pdftexdir
%      src/texk/web2c/pdftexdir> ln -fs pdfcrypt-full.c pdfcrypt.c
%  Compile:
%      src/texk/web2c/pdftexdir> cd ../../..
%      src> ./configure
%      src> cd texk/web2c
%      src/texk/web2c> make pdftexbin
%  At last install the binary and pool files.
%\end{verbatim}
%
% \StopEventually{
% }
%
% \section{Implementation}
%
%    \begin{macrocode}
%<*package>
%    \end{macrocode}
%
% \subsection{Help macros for \plainTeX}
%
%    \begin{macrocode}
\expandafter\edef\csname pc@endinput\endcsname{%
  \catcode`\noexpand\@=\the\catcode`\@\relax
  \noexpand\endinput
}
\catcode`\@=11 %

\expandafter\ifx\csname @firstoftwo\endcsname\relax
  \long\def\@firstoftwo#1#2{#1}%
\fi

\expandafter\ifx\csname @secondoftwo\endcsname\relax
  \long\def\@secondoftwo#1#2{#2}%
\fi

\expandafter\ifx\csname @ifundefined\endcsname\relax
  \def\@ifundefined#1{%
    \expandafter\ifx\csname #1\endcsname\relax
      \expandafter\@firstoftwo
    \else
      \expandafter\@secondoftwo
    \fi
  }%
\fi

\@ifundefined{@ifnextchar}{%
  \long\def\@ifnextchar#1#2#3{%
    \let\reserved@d=#1%
    \def\reserved@a{#2}%
    \def\reserved@b{#3}%
    \futurelet\@let@token\@ifnch
  }%
  \def\@ifnch{%
    \ifx\@let@token\@sptoken
      \let\reserved@c\@xifnch
    \else
      \ifx\@let@token\reserved@d
        \let\reserved@c\reserved@a
      \else
        \let\reserved@c\reserved@b
      \fi
    \fi
    \reserved@c
  }%
  \begingroup
    \def\:{\global\let\@sptoken= }%
    \: %
    \def\:{\@xifnch}%
    \expandafter\gdef\: {\futurelet\@let@token\@ifnch}%
  \endgroup
}{}

\@ifundefined{ProvidesPackage}{%
  \def\ProvidesPackage#1{%
    \@ifnextchar[{\pc@ProvidesPackage{#1}}%
                 {\pc@ProvidesPackage{#1}[]}%
  }%
  \def\pc@ProvidesPackage#1[#2]{%
    \immediate\write-1{Package: #1 #2}%
  }%
}{}

\begingroup\expandafter\expandafter\expandafter\endgroup
\expandafter\ifx\csname RequirePackage\endcsname\relax
  \input infwarerr.sty\relax
\else
  \RequirePackage{infwarerr}%
\fi

\@ifundefined{@gobble}{%
  \long\def\@gobble#1{}%
}{}

\@ifundefined{@empty}{%
  \def\@empty{}%
}{}
%    \end{macrocode}
%
% \subsection{Package Identification and checks}
%
%    \begin{macrocode}
\ProvidesPackage{pdfcrypt}%
  [2016/05/16 v1.1 Allows the setting of PDF encryption (HO)]%
%    \end{macrocode}
%
%    \begin{macrocode}
\@ifundefined{pdfcryptsetup}{%
  \let\pdfcryptsetup\@gobble
}{%
  \@PackageWarningNoLine{pdfcrypt}{Package pdfcrypt is already loaded}%
  \pc@endinput
}
%    \end{macrocode}
%    Support for \cs{nopdfcrypt}.
%    \begin{macrocode}
\newif\ifpc@nopdfcrypt
\ifx Y\nopdfcrypt
  \@PackageWarningNoLine{pdfcrypt}{%
    Encryption disabled by \string \nopdfcrypt\space request%
  }%
  \global\pc@nopdfcrypttrue
\fi
%    \end{macrocode}
%
% \subsection{Driver detection}
%
%    \begin{macrocode}
\let\pc@driver\@empty
\begingroup
  % pdfTeX detection
  \@ifundefined{pdfoutput}{%
  }{%
    \ifcase\pdfoutput
    \else
      \@ifundefined{pdfcrypt}{%
        \@PackageError{pdfcrypt}{%
          PDF encryption is not supported with this pdfTeX%
        }{%
          Encryption support was added in 0.14h (2001/03/10)\MessageBreak
          and removed in 1.10a (2003/01/16).%
        }%
        \endgroup
        \pc@endinput
      }{%
        \gdef\pc@driver{pdftex}%
      }%
    \fi
  }%
  % VTeX detection
  \@ifundefined{OpMode}{%
  }{%
    \ifnum\OpMode=1 %
      \ifnum\@ifundefined{VTeXversion}0\VTeXversion<735 %
        \@PackageError{pdfcrypt}{%
          PDF encryption is not supported with this VTeX%
        }{%
          You need VTeX 7.35 or higher.%
        }%
        \endgroup
        \pc@endinput
      \else
        \gdef\pc@driver{vtex}%
      \fi
    \fi
  }%
\endgroup
%    \end{macrocode}
%
% \subsection{Load package \xpackage{keyval}}
%
%    \begin{macrocode}
\@ifundefined{@makeother}{%
  \def\@makeother#1{\catcode`#1=12\relax}%
}{}

\@ifundefined{g@addto@macro}{%
  \long\def\g@addto@macro#1#2{%
    \begingroup
      \toks@\expandafter{#1#2}%
      \xdef#1{\the\toks@}%
    \endgroup
  }%
}{}

\@ifundefined{@namedef}{%
  \def\@namedef#1{\expandafter\def\csname#1\endcsname}%
}{}

\@ifundefined{@nameuse}{%
  \def\@nameuse#1{\csname #1\endcsname}%
}{}

\def\pc@KeyvalRestore{%
  \let\pc@KeyvalRestore\@undefined
}

\let\pcOrg@NeedsTeXFormat\NeedsTeXFormat
\@ifundefined{NeedsTeXFormat}{%
  \def\NeedsTeXFormat#1{}%
  \g@addto@macro\pc@KeyvalRestore{%
    \let\NeedsTeXFormat\pcOrg@NeedsTeXFormat
  }%
}{}

\let\pcOrg@DeclareOption\DeclareOption
\@ifundefined{DeclareOption}{%
  \def\DeclareOption#1#2{#2}%
  \g@addto@macro\pc@KeyvalRestore{%
    \let\DeclareOption\pcOrg@DeclareOption
  }%
}{}

\let\pcOrg@ExecuteOptions\ExecuteOptions
\@ifundefined{ExecuteOptions}{%
  \def\ExecuteOptions#1{}%
  \g@addto@macro\pc@KeyvalRestore{%
    \let\ExecuteOptions\pcOrg@ExecuteOptions
  }%
}{}

\let\pcOrg@ProcessOptions\ProcessOptions
\@ifundefined{ProcessOptions}{%
  \def\ProcessOptions{}%
  \g@addto@macro\pc@KeyvalRestore{%
    \let\ProcessOptions\pcOrg@ProcessOptions
  }%
}{}%

\begingroup\expandafter\expandafter\expandafter\endgroup
\expandafter\ifx\csname RequirePackage\endcsname\relax
  \input keyval.sty\relax
\else
  \RequirePackage{keyval}%
\fi
\pc@KeyvalRestore
%    \end{macrocode}
%
% \subsection{Define options}
%
%    \begin{macrocode}
\@ifundefined{@dblarg}{%
  \long\def\@dblarg#1{\@ifnextchar[{#1}{\@xdblarg{#1}}}%
  \long\def\@xdblarg#1#2{#1[{#2}]{#2}}%
}{}

\newif\ifpc@set
\newif\ifpc@print
\newif\ifpc@copy
\newif\ifpc@edit
\newif\ifpc@annotate
\newif\ifpc@debug
\let\pc@owner\@empty
\let\pc@user\@empty

% default: allow all
\pc@printtrue
\pc@copytrue
\pc@edittrue
\pc@annotatetrue

\edef\pc@temp{\catcode`\noexpand\"=\the\catcode`\"\relax}
\@makeother\"
\def\pc@set{%
  \@PackageInfo{pdfcrypt}{%
    \ifpc@debug
      \ifx\pc@owner\@empty
        No owner password%
      \else
        Owner password: `\pc@owner'%
      \fi
      \MessageBreak
      \ifx\pc@user\@empty
        No user password%
      \else
        User password: `\pc@user'%
      \fi
      \MessageBreak
      Flags: %
      \ifpc@print   \else no\fi print, %
      \ifpc@copy    \else no\fi copy, %
      \ifpc@edit    \else no\fi edit, %
      \ifpc@annotate\else no\fi annotate%
      \MessageBreak
    \fi
    \ifpc@nopdfcrypt
      Encryption is disabled by `\string\nopdfcrypt'%
    \else
      Encryption is set for `\pc@driver'%
    \fi
  }%
  \ifpc@nopdfcrypt
  \else
    \@ifundefined{pc@set@\pc@driver}{%
      \ifx\pc@driver\@empty
        \@PackageError{pdfcrypt}{No driver for encryption %
          support found}\@ehc
      \else
        \@PackageError{pdfcrypt}{Cannot set encryption for %
          unknown driver `\pc@driver'}\@ehc
      \fi
    }{%
      \@nameuse{pc@set@\pc@driver}%
    }%
  \fi
}
\def\pc@set@pdftex{%
  \ifnum\pdftexversion<100 %
    \pc@set@pdftexold
  \else
    \pc@set@pdftexnew
  \fi
}
\def\pc@set@pdftexold{%
  \pdfcrypt{%
    owner "\pc@owner" %
    user "\pc@user" %
    \ifpc@print   \else no\fi print %
    \ifpc@copy    \else no\fi copy %
    \ifpc@edit    \else no\fi edit %
    \ifpc@annotate\else no\fi annotate%
  }%
}
\def\pc@set@pdftexnew{%
  \pdfcrypt
    owner{\pc@owner}%
    user{\pc@user}%
    \ifpc@print   \else no\fi print %
    \ifpc@copy    \else no\fi copy %
    \ifpc@edit    \else no\fi edit %
    \ifpc@annotate\else no\fi annotate%
  \relax
}
\def\pc@set@vtex{%
  \immediate\special{!security %
    O=\pc@MakeVTeXString\pc@owner,%
    U=\pc@MakeVTeXString\pc@user,%
    P\ifpc@print +\else -\fi,%
    C\ifpc@copy +\else -\fi,%
    M\ifpc@edit +\else -\fi,%
    A\ifpc@annotate +\else -\fi
  }%
}
\def\pc@MakeVTeXString#1{%
  "\expandafter\pc@@MakeVTeXString#1"\@nil"%
}
\def\pc@@MakeVTeXString#1"#2\@nil{%
  #1%
  \ifx\\#2\\%
  \else
    ""%
    \@ReturnAfterFi{%
      \pc@@MakeVTeXString#2\@nil
    }%
  \fi
}
\long\def\@ReturnAfterFi#1\fi{\fi#1}
\pc@temp

\begingroup
  \catcode`\ =12 \gdef\pc@spaceother{ }\catcode`\ =10\relax
  \catcode`\|=0 %
  \catcode`\\=12 %
  |gdef|pc@DefString#1#2{%
    |def#1{#2}%
    |edef#1{|expandafter|strip@prefix|meaning#1}%
    |edef#1{|expandafter|pc@SpaceToOther#1 |@nil}%
    |edef#1{|expandafter|pc@EscapeRemove#1|@empty\|@empty|@nil}%
  }%
  |gdef|pc@EscapeRemove#1\#2#3|@nil{%
    #1#2%
    |ifx|\#3|\%%
    |else
      |@ReturnAfterFi{%
        |pc@EscapeRemove#3|@nil
      }%
    |fi
  }%
|endgroup
\def\pc@SpaceToOther#1 #2\@nil{%
  #1%
  \ifx\\#2\\%
  \else
    \pc@spaceother
    \@ReturnAfterFi{%
      \pc@SpaceToOther#2\@nil
    }%
  \fi
}

\def\pc@boolkey{\@dblarg\pc@@boolkey}
\def\pc@@boolkey[#1]#2#3{%
  \lowercase{\def\pc@temp{#3}}%
  \ifx\pc@temp\@empty
    \let\pc@temp\pc@true
  \fi
  \ifx\pc@temp\pc@true
  \else
    \ifx\pc@temp\pc@false
    \else
      \let\pc@temp\relax
    \fi
  \fi
  \ifx\pc@temp\relax
    \@PackageWarning{pdfcrypt}{%
      Unexpected value \string`#3\string' of %
      option \string`#2\string'\MessageBreak
      instead of %
      \string`true\string' or \string`false\string'%
    }%
  \else
    \csname pc@#2\pc@temp\endcsname
  \fi
}
\def\pc@true{true}
\def\pc@false{false}

\define@key{pc}{set}[true]{%
  \pc@boolkey{set}{#1}%
}
\define@key{pc}{pdftex}[]{%
  \def\pc@driver{pdftex}%
}
\define@key{pc}{vtex}[]{%
  \def\pc@driver{vtex}%
}
\define@key{pc}{print}[true]{%
  \pc@boolkey{print}{#1}%
}
\define@key{pc}{copy}[true]{%
  \pc@boolkey{copy}{#1}%
}
\define@key{pc}{edit}[true]{%
  \pc@boolkey{edit}{#1}%
}
\define@key{pc}{annotate}[true]{%
  \pc@boolkey{annotate}{#1}%
}
\define@key{pc}{all}[]{%
  \pc@boolkey{print}{true}%
  \pc@boolkey{copy}{true}%
  \pc@boolkey{edit}{true}%
  \pc@boolkey{annotate}{true}%
}
\define@key{pc}{none}[]{%
  \pc@boolkey{print}{false}%
  \pc@boolkey{copy}{false}%
  \pc@boolkey{edit}{false}%
  \pc@boolkey{annotate}{false}%
}

\define@key{pc}{owner}{%
  \pc@DefString\pc@owner{#1}%
}
\define@key{pc}{user}{%
  \pc@DefString\pc@user{#1}%
}
\define@key{pc}{debug}[true]{%
  \pc@boolkey{debug}{#1}%
}

\def\pdfcryptsetup#1{%
  \setkeys{pc}{#1}%
  \ifpc@set
    \pc@set
    \global\let\pc@set\relax
    \gdef\pdfcryptsetup##1{%
      \@PackageWarning{pdfcrypt}{%
        Encryption options are already set\MessageBreak
        new values are ignored%
      }%
    }%
  \fi
}
\begingroup\expandafter\expandafter\expandafter\endgroup
\expandafter\ifx\csname @onlypreamble\endcsname\relax
\else
  \@onlypreamble\pdfcryptsetup
\fi
%    \end{macrocode}
%
% \subsection{support of configuration file}
%    \begin{macrocode}
\begingroup\expandafter\expandafter\expandafter\endgroup
\expandafter\ifx\csname InputIfFileExists\endcsname\relax
  \@PackageInfo{pdfcrypt}{%
    Configuration file pdfcrypt.cfg not supported.%
  }%
\else
  \let\pc@ExecuteOptions\ExecuteOptions
  \InputIfFileExists{pdfcrypt.cfg}{}{}%
  \let\ExecuteOptions\pc@ExecuteOptions
\fi
%    \end{macrocode}
%
% \subsection{Package options}
%
%    Plain format does not know package options.
%    \begin{macrocode}
\begingroup\expandafter\expandafter\expandafter\endgroup
\expandafter\ifx\csname @classoptionslist\endcsname\relax
  \expandafter\pc@endinput
\fi
%    \end{macrocode}
%    Process global and local options.
%    \begin{macrocode}
\def\pc@ProcessOptionsWithKV{%
  \let\pc@temp\@empty
  \@for\CurrentOption:=\@classoptionslist\do{%
    \@ifundefined{KV@pc@\CurrentOption}{}{%
      \edef\pc@temp{\pc@temp,\CurrentOption,}%
      \@expandtwoargs\@removeelement\CurrentOption
        \@unusedoptionlist\@unusedoptionlist
    }%
  }%
  \edef\pc@temp{%
    \noexpand\pdfcryptsetup{%
      \pc@temp\@ptionlist{\@currname.\@currext}%
    }%
  }%
  \pc@temp
}
\pc@ProcessOptionsWithKV
\AtEndOfPackage{\let\@unprocessedoptions\relax}
\AtBeginDocument{\pc@set}

\pc@endinput
%    \end{macrocode}
%
%    \begin{macrocode}
%</package>
%    \end{macrocode}
%
% \section{Installation}
%
% \subsection{Download}
%
% \paragraph{Package.} This package is available on
% CTAN\footnote{\CTANpkg{pdfcrypt}}:
% \begin{description}
% \item[\CTAN{macros/latex/contrib/oberdiek/pdfcrypt.dtx}] The source file.
% \item[\CTAN{macros/latex/contrib/oberdiek/pdfcrypt.pdf}] Documentation.
% \end{description}
%
%
% \paragraph{Bundle.} All the packages of the bundle `oberdiek'
% are also available in a TDS compliant ZIP archive. There
% the packages are already unpacked and the documentation files
% are generated. The files and directories obey the TDS standard.
% \begin{description}
% \item[\CTANinstall{install/macros/latex/contrib/oberdiek.tds.zip}]
% \end{description}
% \emph{TDS} refers to the standard ``A Directory Structure
% for \TeX\ Files'' (\CTANpkg{tds}). Directories
% with \xfile{texmf} in their name are usually organized this way.
%
% \subsection{Bundle installation}
%
% \paragraph{Unpacking.} Unpack the \xfile{oberdiek.tds.zip} in the
% TDS tree (also known as \xfile{texmf} tree) of your choice.
% Example (linux):
% \begin{quote}
%   |unzip oberdiek.tds.zip -d ~/texmf|
% \end{quote}
%
% \subsection{Package installation}
%
% \paragraph{Unpacking.} The \xfile{.dtx} file is a self-extracting
% \docstrip\ archive. The files are extracted by running the
% \xfile{.dtx} through \plainTeX:
% \begin{quote}
%   \verb|tex pdfcrypt.dtx|
% \end{quote}
%
% \paragraph{TDS.} Now the different files must be moved into
% the different directories in your installation TDS tree
% (also known as \xfile{texmf} tree):
% \begin{quote}
% \def\t{^^A
% \begin{tabular}{@{}>{\ttfamily}l@{ $\rightarrow$ }>{\ttfamily}l@{}}
%   pdfcrypt.sty & tex/generic/oberdiek/pdfcrypt.sty\\
%   pdfcrypt.pdf & doc/latex/oberdiek/pdfcrypt.pdf\\
%   pdfcrypt.dtx & source/latex/oberdiek/pdfcrypt.dtx\\
% \end{tabular}^^A
% }^^A
% \sbox0{\t}^^A
% \ifdim\wd0>\linewidth
%   \begingroup
%     \advance\linewidth by\leftmargin
%     \advance\linewidth by\rightmargin
%   \edef\x{\endgroup
%     \def\noexpand\lw{\the\linewidth}^^A
%   }\x
%   \def\lwbox{^^A
%     \leavevmode
%     \hbox to \linewidth{^^A
%       \kern-\leftmargin\relax
%       \hss
%       \usebox0
%       \hss
%       \kern-\rightmargin\relax
%     }^^A
%   }^^A
%   \ifdim\wd0>\lw
%     \sbox0{\small\t}^^A
%     \ifdim\wd0>\linewidth
%       \ifdim\wd0>\lw
%         \sbox0{\footnotesize\t}^^A
%         \ifdim\wd0>\linewidth
%           \ifdim\wd0>\lw
%             \sbox0{\scriptsize\t}^^A
%             \ifdim\wd0>\linewidth
%               \ifdim\wd0>\lw
%                 \sbox0{\tiny\t}^^A
%                 \ifdim\wd0>\linewidth
%                   \lwbox
%                 \else
%                   \usebox0
%                 \fi
%               \else
%                 \lwbox
%               \fi
%             \else
%               \usebox0
%             \fi
%           \else
%             \lwbox
%           \fi
%         \else
%           \usebox0
%         \fi
%       \else
%         \lwbox
%       \fi
%     \else
%       \usebox0
%     \fi
%   \else
%     \lwbox
%   \fi
% \else
%   \usebox0
% \fi
% \end{quote}
% If you have a \xfile{docstrip.cfg} that configures and enables \docstrip's
% TDS installing feature, then some files can already be in the right
% place, see the documentation of \docstrip.
%
% \subsection{Refresh file name databases}
%
% If your \TeX~distribution
% (\TeX\,Live, \mikTeX, \dots) relies on file name databases, you must refresh
% these. For example, \TeX\,Live\ users run \verb|texhash| or
% \verb|mktexlsr|.
%
% \subsection{Some details for the interested}
%
% \paragraph{Unpacking with \LaTeX.}
% The \xfile{.dtx} chooses its action depending on the format:
% \begin{description}
% \item[\plainTeX:] Run \docstrip\ and extract the files.
% \item[\LaTeX:] Generate the documentation.
% \end{description}
% If you insist on using \LaTeX\ for \docstrip\ (really,
% \docstrip\ does not need \LaTeX), then inform the autodetect routine
% about your intention:
% \begin{quote}
%   \verb|latex \let\install=y% \iffalse meta-comment
%
% File: pdfcrypt.dtx
% Version: 2016/05/16 v1.1
% Info: Allows the setting of PDF encryption
%
% Copyright (C)
%    2001, 2006, 2007 Heiko Oberdiek
%    2016-2019 Oberdiek Package Support Group
%    https://github.com/ho-tex/oberdiek/issues
%
% This work may be distributed and/or modified under the
% conditions of the LaTeX Project Public License, either
% version 1.3c of this license or (at your option) any later
% version. This version of this license is in
%    https://www.latex-project.org/lppl/lppl-1-3c.txt
% and the latest version of this license is in
%    https://www.latex-project.org/lppl.txt
% and version 1.3 or later is part of all distributions of
% LaTeX version 2005/12/01 or later.
%
% This work has the LPPL maintenance status "maintained".
%
% The Current Maintainers of this work are
% Heiko Oberdiek and the Oberdiek Package Support Group
% https://github.com/ho-tex/oberdiek/issues
%
% The Base Interpreter refers to any `TeX-Format',
% because some files are installed in TDS:tex/generic//.
%
% This work consists of the main source file pdfcrypt.dtx
% and the derived files
%    pdfcrypt.sty, pdfcrypt.pdf, pdfcrypt.ins, pdfcrypt.drv.
%
% Distribution:
%    CTAN:macros/latex/contrib/oberdiek/pdfcrypt.dtx
%    CTAN:macros/latex/contrib/oberdiek/pdfcrypt.pdf
%
% Unpacking:
%    (a) If pdfcrypt.ins is present:
%           tex pdfcrypt.ins
%    (b) Without pdfcrypt.ins:
%           tex pdfcrypt.dtx
%    (c) If you insist on using LaTeX
%           latex \let\install=y% \iffalse meta-comment
%
% File: pdfcrypt.dtx
% Version: 2016/05/16 v1.1
% Info: Allows the setting of PDF encryption
%
% Copyright (C)
%    2001, 2006, 2007 Heiko Oberdiek
%    2016-2019 Oberdiek Package Support Group
%    https://github.com/ho-tex/oberdiek/issues
%
% This work may be distributed and/or modified under the
% conditions of the LaTeX Project Public License, either
% version 1.3c of this license or (at your option) any later
% version. This version of this license is in
%    https://www.latex-project.org/lppl/lppl-1-3c.txt
% and the latest version of this license is in
%    https://www.latex-project.org/lppl.txt
% and version 1.3 or later is part of all distributions of
% LaTeX version 2005/12/01 or later.
%
% This work has the LPPL maintenance status "maintained".
%
% The Current Maintainers of this work are
% Heiko Oberdiek and the Oberdiek Package Support Group
% https://github.com/ho-tex/oberdiek/issues
%
% The Base Interpreter refers to any `TeX-Format',
% because some files are installed in TDS:tex/generic//.
%
% This work consists of the main source file pdfcrypt.dtx
% and the derived files
%    pdfcrypt.sty, pdfcrypt.pdf, pdfcrypt.ins, pdfcrypt.drv.
%
% Distribution:
%    CTAN:macros/latex/contrib/oberdiek/pdfcrypt.dtx
%    CTAN:macros/latex/contrib/oberdiek/pdfcrypt.pdf
%
% Unpacking:
%    (a) If pdfcrypt.ins is present:
%           tex pdfcrypt.ins
%    (b) Without pdfcrypt.ins:
%           tex pdfcrypt.dtx
%    (c) If you insist on using LaTeX
%           latex \let\install=y\input{pdfcrypt.dtx}
%        (quote the arguments according to the demands of your shell)
%
% Documentation:
%    (a) If pdfcrypt.drv is present:
%           latex pdfcrypt.drv
%    (b) Without pdfcrypt.drv:
%           latex pdfcrypt.dtx; ...
%    The class ltxdoc loads the configuration file ltxdoc.cfg
%    if available. Here you can specify further options, e.g.
%    use A4 as paper format:
%       \PassOptionsToClass{a4paper}{article}
%
%    Programm calls to get the documentation (example):
%       pdflatex pdfcrypt.dtx
%       makeindex -s gind.ist pdfcrypt.idx
%       pdflatex pdfcrypt.dtx
%       makeindex -s gind.ist pdfcrypt.idx
%       pdflatex pdfcrypt.dtx
%
% Installation:
%    TDS:tex/generic/oberdiek/pdfcrypt.sty
%    TDS:doc/latex/oberdiek/pdfcrypt.pdf
%    TDS:source/latex/oberdiek/pdfcrypt.dtx
%
%<*ignore>
\begingroup
  \catcode123=1 %
  \catcode125=2 %
  \def\x{LaTeX2e}%
\expandafter\endgroup
\ifcase 0\ifx\install y1\fi\expandafter
         \ifx\csname processbatchFile\endcsname\relax\else1\fi
         \ifx\fmtname\x\else 1\fi\relax
\else\csname fi\endcsname
%</ignore>
%<*install>
\input docstrip.tex
\Msg{************************************************************************}
\Msg{* Installation}
\Msg{* Package: pdfcrypt 2016/05/16 v1.1 Allows the setting of PDF encryption (HO)}
\Msg{************************************************************************}

\keepsilent
\askforoverwritefalse

\let\MetaPrefix\relax
\preamble

This is a generated file.

Project: pdfcrypt
Version: 2016/05/16 v1.1

Copyright (C)
   2001, 2006, 2007 Heiko Oberdiek
   2016-2019 Oberdiek Package Support Group

This work may be distributed and/or modified under the
conditions of the LaTeX Project Public License, either
version 1.3c of this license or (at your option) any later
version. This version of this license is in
   https://www.latex-project.org/lppl/lppl-1-3c.txt
and the latest version of this license is in
   https://www.latex-project.org/lppl.txt
and version 1.3 or later is part of all distributions of
LaTeX version 2005/12/01 or later.

This work has the LPPL maintenance status "maintained".

The Current Maintainers of this work are
Heiko Oberdiek and the Oberdiek Package Support Group
https://github.com/ho-tex/oberdiek/issues


The Base Interpreter refers to any `TeX-Format',
because some files are installed in TDS:tex/generic//.

This work consists of the main source file pdfcrypt.dtx
and the derived files
   pdfcrypt.sty, pdfcrypt.pdf, pdfcrypt.ins, pdfcrypt.drv.

\endpreamble
\let\MetaPrefix\DoubleperCent

\generate{%
  \file{pdfcrypt.ins}{\from{pdfcrypt.dtx}{install}}%
  \file{pdfcrypt.drv}{\from{pdfcrypt.dtx}{driver}}%
  \usedir{tex/generic/oberdiek}%
  \file{pdfcrypt.sty}{\from{pdfcrypt.dtx}{package}}%
  \nopreamble
  \nopostamble
%  \usedir{source/latex/oberdiek/catalogue}%
%  \file{pdfcrypt.xml}{\from{pdfcrypt.dtx}{catalogue}}%
}

\catcode32=13\relax% active space
\let =\space%
\Msg{************************************************************************}
\Msg{*}
\Msg{* To finish the installation you have to move the following}
\Msg{* file into a directory searched by TeX:}
\Msg{*}
\Msg{*     pdfcrypt.sty}
\Msg{*}
\Msg{* To produce the documentation run the file `pdfcrypt.drv'}
\Msg{* through LaTeX.}
\Msg{*}
\Msg{* Happy TeXing!}
\Msg{*}
\Msg{************************************************************************}

\endbatchfile
%</install>
%<*ignore>
\fi
%</ignore>
%<*driver>
\NeedsTeXFormat{LaTeX2e}
\ProvidesFile{pdfcrypt.drv}%
  [2016/05/16 v1.1 Allows the setting of PDF encryption (HO)]%
\documentclass{ltxdoc}
\usepackage{holtxdoc}[2011/11/22]
\begin{document}
  \DocInput{pdfcrypt.dtx}%
\end{document}
%</driver>
% \fi
%
%
% \CharacterTable
%  {Upper-case    \A\B\C\D\E\F\G\H\I\J\K\L\M\N\O\P\Q\R\S\T\U\V\W\X\Y\Z
%   Lower-case    \a\b\c\d\e\f\g\h\i\j\k\l\m\n\o\p\q\r\s\t\u\v\w\x\y\z
%   Digits        \0\1\2\3\4\5\6\7\8\9
%   Exclamation   \!     Double quote  \"     Hash (number) \#
%   Dollar        \$     Percent       \%     Ampersand     \&
%   Acute accent  \'     Left paren    \(     Right paren   \)
%   Asterisk      \*     Plus          \+     Comma         \,
%   Minus         \-     Point         \.     Solidus       \/
%   Colon         \:     Semicolon     \;     Less than     \<
%   Equals        \=     Greater than  \>     Question mark \?
%   Commercial at \@     Left bracket  \[     Backslash     \\
%   Right bracket \]     Circumflex    \^     Underscore    \_
%   Grave accent  \`     Left brace    \{     Vertical bar  \|
%   Right brace   \}     Tilde         \~}
%
% \GetFileInfo{pdfcrypt.drv}
%
% \title{The \xpackage{pdfcrypt} package}
% \date{2016/05/16 v1.1}
% \author{Heiko Oberdiek\thanks
% {Please report any issues at \url{https://github.com/ho-tex/oberdiek/issues}}}
%
% \maketitle
%
% \begin{abstract}
% This package supports the setting of pdf encryption options
% for \VTeX\ and some older versions of \pdfTeX.
% \end{abstract}
%
% \tableofcontents
%
% \section{Documentation}
%
% This package allows the setting of pdf encryption
% options for
% \begin{itemize}
% \item
%   \VTeX, since version 7.35, \url{http://www.micropress-inc.com/}.
% \item
%   \pdfTeX, patched by
%   \NameEmail{Ricardo Sanchez Carmenes}{carmenes@bioquimica.uniovi.es}.\\
%   Some supported versions are:
%   \begin{quote}
%     \pdfTeX-1.00a,\\
%     \pdfTeX-0.14h-pdfcrypt-20010310,\\
%     pdfcrypt-20010331
%   \end{quote}
% \end{itemize}
%
% \noindent
% \textcolor{red}{Note:} Since \pdfTeX-1.10a (2003-01-16) encryption
% support was dropped! Thus the package is now obsolete for recent
% \pdfTeX\ versions.
%
% \subsection{Alternatives}
%
% There are free alternatives, programs that can
% be used for postprocessing the pdf file:
% \begin{itemize}
% \item \textsf{pdftk}\\
%   \url{http://www.accesspdf.com/pdftk/}
% \item \textsf{Multivalent}\\
%   \url{http://multivalent.sourceforge.net/}
% \item \textsf{PDFBox}\\
%   \url{http://www.pdfbox.org/}
% \item \textsf{PDFTrans}\\
%   \url{http://maddingue.free.fr/softwares/pdftrans.html}
% \item \dots
% \end{itemize}
%
% \subsection{Usage}
%
% The options can be set as package options or
% with the command \cs{pdfcryptsetup}:
% \begin{quote}
%   |owner=|\meta{owner password}\\
%   |user=|\meta{user password}\\
%   |print|, |copy|, |edit|, |annotate=true|, |false|\\
%   |all|, |none|\\
%   |debug|
% \end{quote}
% The encryption is set at |\begin{document}| by default,
% but this can be forced for an earlier time
% using the option |set|.
% Example:
% \begin{quote}
%   |\usepackage[owner=Donald,user=Knuth,print=false]{pdfcrypt}|\\
%   \dots\\
%   |\begin{document}|
% \end{quote}
% or
% \begin{quote}
%   |\usepackage{pdfcrypt}|\\
%   |\pdfcryptsetup{owner=Donald,user=Knut}|\\
%   |\pdfcryptsetup{print=false,set}|\\
%   \dots\\
%   |\begin{document}|
% \end{quote}
%
% \subsubsection{Special characters}
%
% The characters `|{|', `|}|', and `|\|' have to be
% escaped with `|\|'. Then the string should not
% be specified as package option but with
% the macro \cs{pdfcryptsetup}, eg:
% \begin{quote}
%   |\pdfcryptsetup{user={\{\\Hello\}},print=false}|
% \end{quote}
% The password here is ``|{\Hello}|''.
% Active characters can be used and are not expanded.
% Macros except for |\{|, |\}|, and |\\| should not be used
% and are not expanded.
%
% \subsubsection{\cs{nopdfcrypt}}
%
% Often the whole process of pdf generation includes
% several \TeX\ compilations and perhaps other
% postprocessing steps. Therefore a feature would
% be useful to disable the encryption stuff in order
% to speed up the first compilations.
% Therefore package \xpackage{pdfcrypt} will look for
% the macro \cs{nopdfcrypt}. If it has the meaning of the
% letter `|Y|', the package will be disabled.
% Example:
% \begin{quote}
%   |pdflatex '\let\nopdfcrypt=Y\input myfile'|\\
%   |thumbpdf myfile|\\
%   |pdfelatex myfile|
% \end{quote}
%
% \subsubsection{Configuration file}
%
% If the file \xfile{pdfcrypt.cfg} exists it is loaded
% and can be used for common settings.
%
% \subsubsection{Support for \plainTeX}
%
% The package can also be used with \plainTeX. It defines the
% missing \LaTeX\ macros and loads package \xpackage{keyval}.
% The configuration file is not supported and automatically
% loaded.
%
% \subsection{Hints for \pdfTeX}
%
% This section is out of date, because the encryption support
% was removed from \pdfTeX. Therefore this hints are of
% historical interest only.
%
%\begin{verbatim}
%* There are packages such as thumbpdf that ship out
%  some stuff (\immediate\pdfobj{...}). In these cases
%  \pdfcrypt will be ignored without warning or error
%  message. Therefore the package pdfcrypt should be
%  loaded before and option "set" should be used in
%  order to force the call of \pdfcrypt, for example:
%     \usepackage{pdfcrypt}
%     \pdfcryptsetup{..., set}
%     ...
%     \usepackage{thumbpdf}
%
%* Compiling pdfTeX (eg. version 1.00a-pretest-20010804):
%  pdfTeX versions are available at
%      ftp://ftp.cstug.cz/pub/tex/local/cstug/thanh/pdftex/snapshots/
%  Unpack the latest .tgz file:
%      > tar xzf pdftex-20010804.tgz
%  Enable encryption support:
%      src> cd src/texk/web2c/pdftexdir
%      src/texk/web2c/pdftexdir> ln -fs pdfcrypt-full.c pdfcrypt.c
%  Compile:
%      src/texk/web2c/pdftexdir> cd ../../..
%      src> ./configure
%      src> cd texk/web2c
%      src/texk/web2c> make pdftexbin
%  At last install the binary and pool files.
%\end{verbatim}
%
% \StopEventually{
% }
%
% \section{Implementation}
%
%    \begin{macrocode}
%<*package>
%    \end{macrocode}
%
% \subsection{Help macros for \plainTeX}
%
%    \begin{macrocode}
\expandafter\edef\csname pc@endinput\endcsname{%
  \catcode`\noexpand\@=\the\catcode`\@\relax
  \noexpand\endinput
}
\catcode`\@=11 %

\expandafter\ifx\csname @firstoftwo\endcsname\relax
  \long\def\@firstoftwo#1#2{#1}%
\fi

\expandafter\ifx\csname @secondoftwo\endcsname\relax
  \long\def\@secondoftwo#1#2{#2}%
\fi

\expandafter\ifx\csname @ifundefined\endcsname\relax
  \def\@ifundefined#1{%
    \expandafter\ifx\csname #1\endcsname\relax
      \expandafter\@firstoftwo
    \else
      \expandafter\@secondoftwo
    \fi
  }%
\fi

\@ifundefined{@ifnextchar}{%
  \long\def\@ifnextchar#1#2#3{%
    \let\reserved@d=#1%
    \def\reserved@a{#2}%
    \def\reserved@b{#3}%
    \futurelet\@let@token\@ifnch
  }%
  \def\@ifnch{%
    \ifx\@let@token\@sptoken
      \let\reserved@c\@xifnch
    \else
      \ifx\@let@token\reserved@d
        \let\reserved@c\reserved@a
      \else
        \let\reserved@c\reserved@b
      \fi
    \fi
    \reserved@c
  }%
  \begingroup
    \def\:{\global\let\@sptoken= }%
    \: %
    \def\:{\@xifnch}%
    \expandafter\gdef\: {\futurelet\@let@token\@ifnch}%
  \endgroup
}{}

\@ifundefined{ProvidesPackage}{%
  \def\ProvidesPackage#1{%
    \@ifnextchar[{\pc@ProvidesPackage{#1}}%
                 {\pc@ProvidesPackage{#1}[]}%
  }%
  \def\pc@ProvidesPackage#1[#2]{%
    \immediate\write-1{Package: #1 #2}%
  }%
}{}

\begingroup\expandafter\expandafter\expandafter\endgroup
\expandafter\ifx\csname RequirePackage\endcsname\relax
  \input infwarerr.sty\relax
\else
  \RequirePackage{infwarerr}%
\fi

\@ifundefined{@gobble}{%
  \long\def\@gobble#1{}%
}{}

\@ifundefined{@empty}{%
  \def\@empty{}%
}{}
%    \end{macrocode}
%
% \subsection{Package Identification and checks}
%
%    \begin{macrocode}
\ProvidesPackage{pdfcrypt}%
  [2016/05/16 v1.1 Allows the setting of PDF encryption (HO)]%
%    \end{macrocode}
%
%    \begin{macrocode}
\@ifundefined{pdfcryptsetup}{%
  \let\pdfcryptsetup\@gobble
}{%
  \@PackageWarningNoLine{pdfcrypt}{Package pdfcrypt is already loaded}%
  \pc@endinput
}
%    \end{macrocode}
%    Support for \cs{nopdfcrypt}.
%    \begin{macrocode}
\newif\ifpc@nopdfcrypt
\ifx Y\nopdfcrypt
  \@PackageWarningNoLine{pdfcrypt}{%
    Encryption disabled by \string \nopdfcrypt\space request%
  }%
  \global\pc@nopdfcrypttrue
\fi
%    \end{macrocode}
%
% \subsection{Driver detection}
%
%    \begin{macrocode}
\let\pc@driver\@empty
\begingroup
  % pdfTeX detection
  \@ifundefined{pdfoutput}{%
  }{%
    \ifcase\pdfoutput
    \else
      \@ifundefined{pdfcrypt}{%
        \@PackageError{pdfcrypt}{%
          PDF encryption is not supported with this pdfTeX%
        }{%
          Encryption support was added in 0.14h (2001/03/10)\MessageBreak
          and removed in 1.10a (2003/01/16).%
        }%
        \endgroup
        \pc@endinput
      }{%
        \gdef\pc@driver{pdftex}%
      }%
    \fi
  }%
  % VTeX detection
  \@ifundefined{OpMode}{%
  }{%
    \ifnum\OpMode=1 %
      \ifnum\@ifundefined{VTeXversion}0\VTeXversion<735 %
        \@PackageError{pdfcrypt}{%
          PDF encryption is not supported with this VTeX%
        }{%
          You need VTeX 7.35 or higher.%
        }%
        \endgroup
        \pc@endinput
      \else
        \gdef\pc@driver{vtex}%
      \fi
    \fi
  }%
\endgroup
%    \end{macrocode}
%
% \subsection{Load package \xpackage{keyval}}
%
%    \begin{macrocode}
\@ifundefined{@makeother}{%
  \def\@makeother#1{\catcode`#1=12\relax}%
}{}

\@ifundefined{g@addto@macro}{%
  \long\def\g@addto@macro#1#2{%
    \begingroup
      \toks@\expandafter{#1#2}%
      \xdef#1{\the\toks@}%
    \endgroup
  }%
}{}

\@ifundefined{@namedef}{%
  \def\@namedef#1{\expandafter\def\csname#1\endcsname}%
}{}

\@ifundefined{@nameuse}{%
  \def\@nameuse#1{\csname #1\endcsname}%
}{}

\def\pc@KeyvalRestore{%
  \let\pc@KeyvalRestore\@undefined
}

\let\pcOrg@NeedsTeXFormat\NeedsTeXFormat
\@ifundefined{NeedsTeXFormat}{%
  \def\NeedsTeXFormat#1{}%
  \g@addto@macro\pc@KeyvalRestore{%
    \let\NeedsTeXFormat\pcOrg@NeedsTeXFormat
  }%
}{}

\let\pcOrg@DeclareOption\DeclareOption
\@ifundefined{DeclareOption}{%
  \def\DeclareOption#1#2{#2}%
  \g@addto@macro\pc@KeyvalRestore{%
    \let\DeclareOption\pcOrg@DeclareOption
  }%
}{}

\let\pcOrg@ExecuteOptions\ExecuteOptions
\@ifundefined{ExecuteOptions}{%
  \def\ExecuteOptions#1{}%
  \g@addto@macro\pc@KeyvalRestore{%
    \let\ExecuteOptions\pcOrg@ExecuteOptions
  }%
}{}

\let\pcOrg@ProcessOptions\ProcessOptions
\@ifundefined{ProcessOptions}{%
  \def\ProcessOptions{}%
  \g@addto@macro\pc@KeyvalRestore{%
    \let\ProcessOptions\pcOrg@ProcessOptions
  }%
}{}%

\begingroup\expandafter\expandafter\expandafter\endgroup
\expandafter\ifx\csname RequirePackage\endcsname\relax
  \input keyval.sty\relax
\else
  \RequirePackage{keyval}%
\fi
\pc@KeyvalRestore
%    \end{macrocode}
%
% \subsection{Define options}
%
%    \begin{macrocode}
\@ifundefined{@dblarg}{%
  \long\def\@dblarg#1{\@ifnextchar[{#1}{\@xdblarg{#1}}}%
  \long\def\@xdblarg#1#2{#1[{#2}]{#2}}%
}{}

\newif\ifpc@set
\newif\ifpc@print
\newif\ifpc@copy
\newif\ifpc@edit
\newif\ifpc@annotate
\newif\ifpc@debug
\let\pc@owner\@empty
\let\pc@user\@empty

% default: allow all
\pc@printtrue
\pc@copytrue
\pc@edittrue
\pc@annotatetrue

\edef\pc@temp{\catcode`\noexpand\"=\the\catcode`\"\relax}
\@makeother\"
\def\pc@set{%
  \@PackageInfo{pdfcrypt}{%
    \ifpc@debug
      \ifx\pc@owner\@empty
        No owner password%
      \else
        Owner password: `\pc@owner'%
      \fi
      \MessageBreak
      \ifx\pc@user\@empty
        No user password%
      \else
        User password: `\pc@user'%
      \fi
      \MessageBreak
      Flags: %
      \ifpc@print   \else no\fi print, %
      \ifpc@copy    \else no\fi copy, %
      \ifpc@edit    \else no\fi edit, %
      \ifpc@annotate\else no\fi annotate%
      \MessageBreak
    \fi
    \ifpc@nopdfcrypt
      Encryption is disabled by `\string\nopdfcrypt'%
    \else
      Encryption is set for `\pc@driver'%
    \fi
  }%
  \ifpc@nopdfcrypt
  \else
    \@ifundefined{pc@set@\pc@driver}{%
      \ifx\pc@driver\@empty
        \@PackageError{pdfcrypt}{No driver for encryption %
          support found}\@ehc
      \else
        \@PackageError{pdfcrypt}{Cannot set encryption for %
          unknown driver `\pc@driver'}\@ehc
      \fi
    }{%
      \@nameuse{pc@set@\pc@driver}%
    }%
  \fi
}
\def\pc@set@pdftex{%
  \ifnum\pdftexversion<100 %
    \pc@set@pdftexold
  \else
    \pc@set@pdftexnew
  \fi
}
\def\pc@set@pdftexold{%
  \pdfcrypt{%
    owner "\pc@owner" %
    user "\pc@user" %
    \ifpc@print   \else no\fi print %
    \ifpc@copy    \else no\fi copy %
    \ifpc@edit    \else no\fi edit %
    \ifpc@annotate\else no\fi annotate%
  }%
}
\def\pc@set@pdftexnew{%
  \pdfcrypt
    owner{\pc@owner}%
    user{\pc@user}%
    \ifpc@print   \else no\fi print %
    \ifpc@copy    \else no\fi copy %
    \ifpc@edit    \else no\fi edit %
    \ifpc@annotate\else no\fi annotate%
  \relax
}
\def\pc@set@vtex{%
  \immediate\special{!security %
    O=\pc@MakeVTeXString\pc@owner,%
    U=\pc@MakeVTeXString\pc@user,%
    P\ifpc@print +\else -\fi,%
    C\ifpc@copy +\else -\fi,%
    M\ifpc@edit +\else -\fi,%
    A\ifpc@annotate +\else -\fi
  }%
}
\def\pc@MakeVTeXString#1{%
  "\expandafter\pc@@MakeVTeXString#1"\@nil"%
}
\def\pc@@MakeVTeXString#1"#2\@nil{%
  #1%
  \ifx\\#2\\%
  \else
    ""%
    \@ReturnAfterFi{%
      \pc@@MakeVTeXString#2\@nil
    }%
  \fi
}
\long\def\@ReturnAfterFi#1\fi{\fi#1}
\pc@temp

\begingroup
  \catcode`\ =12 \gdef\pc@spaceother{ }\catcode`\ =10\relax
  \catcode`\|=0 %
  \catcode`\\=12 %
  |gdef|pc@DefString#1#2{%
    |def#1{#2}%
    |edef#1{|expandafter|strip@prefix|meaning#1}%
    |edef#1{|expandafter|pc@SpaceToOther#1 |@nil}%
    |edef#1{|expandafter|pc@EscapeRemove#1|@empty\|@empty|@nil}%
  }%
  |gdef|pc@EscapeRemove#1\#2#3|@nil{%
    #1#2%
    |ifx|\#3|\%%
    |else
      |@ReturnAfterFi{%
        |pc@EscapeRemove#3|@nil
      }%
    |fi
  }%
|endgroup
\def\pc@SpaceToOther#1 #2\@nil{%
  #1%
  \ifx\\#2\\%
  \else
    \pc@spaceother
    \@ReturnAfterFi{%
      \pc@SpaceToOther#2\@nil
    }%
  \fi
}

\def\pc@boolkey{\@dblarg\pc@@boolkey}
\def\pc@@boolkey[#1]#2#3{%
  \lowercase{\def\pc@temp{#3}}%
  \ifx\pc@temp\@empty
    \let\pc@temp\pc@true
  \fi
  \ifx\pc@temp\pc@true
  \else
    \ifx\pc@temp\pc@false
    \else
      \let\pc@temp\relax
    \fi
  \fi
  \ifx\pc@temp\relax
    \@PackageWarning{pdfcrypt}{%
      Unexpected value \string`#3\string' of %
      option \string`#2\string'\MessageBreak
      instead of %
      \string`true\string' or \string`false\string'%
    }%
  \else
    \csname pc@#2\pc@temp\endcsname
  \fi
}
\def\pc@true{true}
\def\pc@false{false}

\define@key{pc}{set}[true]{%
  \pc@boolkey{set}{#1}%
}
\define@key{pc}{pdftex}[]{%
  \def\pc@driver{pdftex}%
}
\define@key{pc}{vtex}[]{%
  \def\pc@driver{vtex}%
}
\define@key{pc}{print}[true]{%
  \pc@boolkey{print}{#1}%
}
\define@key{pc}{copy}[true]{%
  \pc@boolkey{copy}{#1}%
}
\define@key{pc}{edit}[true]{%
  \pc@boolkey{edit}{#1}%
}
\define@key{pc}{annotate}[true]{%
  \pc@boolkey{annotate}{#1}%
}
\define@key{pc}{all}[]{%
  \pc@boolkey{print}{true}%
  \pc@boolkey{copy}{true}%
  \pc@boolkey{edit}{true}%
  \pc@boolkey{annotate}{true}%
}
\define@key{pc}{none}[]{%
  \pc@boolkey{print}{false}%
  \pc@boolkey{copy}{false}%
  \pc@boolkey{edit}{false}%
  \pc@boolkey{annotate}{false}%
}

\define@key{pc}{owner}{%
  \pc@DefString\pc@owner{#1}%
}
\define@key{pc}{user}{%
  \pc@DefString\pc@user{#1}%
}
\define@key{pc}{debug}[true]{%
  \pc@boolkey{debug}{#1}%
}

\def\pdfcryptsetup#1{%
  \setkeys{pc}{#1}%
  \ifpc@set
    \pc@set
    \global\let\pc@set\relax
    \gdef\pdfcryptsetup##1{%
      \@PackageWarning{pdfcrypt}{%
        Encryption options are already set\MessageBreak
        new values are ignored%
      }%
    }%
  \fi
}
\begingroup\expandafter\expandafter\expandafter\endgroup
\expandafter\ifx\csname @onlypreamble\endcsname\relax
\else
  \@onlypreamble\pdfcryptsetup
\fi
%    \end{macrocode}
%
% \subsection{support of configuration file}
%    \begin{macrocode}
\begingroup\expandafter\expandafter\expandafter\endgroup
\expandafter\ifx\csname InputIfFileExists\endcsname\relax
  \@PackageInfo{pdfcrypt}{%
    Configuration file pdfcrypt.cfg not supported.%
  }%
\else
  \let\pc@ExecuteOptions\ExecuteOptions
  \InputIfFileExists{pdfcrypt.cfg}{}{}%
  \let\ExecuteOptions\pc@ExecuteOptions
\fi
%    \end{macrocode}
%
% \subsection{Package options}
%
%    Plain format does not know package options.
%    \begin{macrocode}
\begingroup\expandafter\expandafter\expandafter\endgroup
\expandafter\ifx\csname @classoptionslist\endcsname\relax
  \expandafter\pc@endinput
\fi
%    \end{macrocode}
%    Process global and local options.
%    \begin{macrocode}
\def\pc@ProcessOptionsWithKV{%
  \let\pc@temp\@empty
  \@for\CurrentOption:=\@classoptionslist\do{%
    \@ifundefined{KV@pc@\CurrentOption}{}{%
      \edef\pc@temp{\pc@temp,\CurrentOption,}%
      \@expandtwoargs\@removeelement\CurrentOption
        \@unusedoptionlist\@unusedoptionlist
    }%
  }%
  \edef\pc@temp{%
    \noexpand\pdfcryptsetup{%
      \pc@temp\@ptionlist{\@currname.\@currext}%
    }%
  }%
  \pc@temp
}
\pc@ProcessOptionsWithKV
\AtEndOfPackage{\let\@unprocessedoptions\relax}
\AtBeginDocument{\pc@set}

\pc@endinput
%    \end{macrocode}
%
%    \begin{macrocode}
%</package>
%    \end{macrocode}
%
% \section{Installation}
%
% \subsection{Download}
%
% \paragraph{Package.} This package is available on
% CTAN\footnote{\CTANpkg{pdfcrypt}}:
% \begin{description}
% \item[\CTAN{macros/latex/contrib/oberdiek/pdfcrypt.dtx}] The source file.
% \item[\CTAN{macros/latex/contrib/oberdiek/pdfcrypt.pdf}] Documentation.
% \end{description}
%
%
% \paragraph{Bundle.} All the packages of the bundle `oberdiek'
% are also available in a TDS compliant ZIP archive. There
% the packages are already unpacked and the documentation files
% are generated. The files and directories obey the TDS standard.
% \begin{description}
% \item[\CTANinstall{install/macros/latex/contrib/oberdiek.tds.zip}]
% \end{description}
% \emph{TDS} refers to the standard ``A Directory Structure
% for \TeX\ Files'' (\CTANpkg{tds}). Directories
% with \xfile{texmf} in their name are usually organized this way.
%
% \subsection{Bundle installation}
%
% \paragraph{Unpacking.} Unpack the \xfile{oberdiek.tds.zip} in the
% TDS tree (also known as \xfile{texmf} tree) of your choice.
% Example (linux):
% \begin{quote}
%   |unzip oberdiek.tds.zip -d ~/texmf|
% \end{quote}
%
% \subsection{Package installation}
%
% \paragraph{Unpacking.} The \xfile{.dtx} file is a self-extracting
% \docstrip\ archive. The files are extracted by running the
% \xfile{.dtx} through \plainTeX:
% \begin{quote}
%   \verb|tex pdfcrypt.dtx|
% \end{quote}
%
% \paragraph{TDS.} Now the different files must be moved into
% the different directories in your installation TDS tree
% (also known as \xfile{texmf} tree):
% \begin{quote}
% \def\t{^^A
% \begin{tabular}{@{}>{\ttfamily}l@{ $\rightarrow$ }>{\ttfamily}l@{}}
%   pdfcrypt.sty & tex/generic/oberdiek/pdfcrypt.sty\\
%   pdfcrypt.pdf & doc/latex/oberdiek/pdfcrypt.pdf\\
%   pdfcrypt.dtx & source/latex/oberdiek/pdfcrypt.dtx\\
% \end{tabular}^^A
% }^^A
% \sbox0{\t}^^A
% \ifdim\wd0>\linewidth
%   \begingroup
%     \advance\linewidth by\leftmargin
%     \advance\linewidth by\rightmargin
%   \edef\x{\endgroup
%     \def\noexpand\lw{\the\linewidth}^^A
%   }\x
%   \def\lwbox{^^A
%     \leavevmode
%     \hbox to \linewidth{^^A
%       \kern-\leftmargin\relax
%       \hss
%       \usebox0
%       \hss
%       \kern-\rightmargin\relax
%     }^^A
%   }^^A
%   \ifdim\wd0>\lw
%     \sbox0{\small\t}^^A
%     \ifdim\wd0>\linewidth
%       \ifdim\wd0>\lw
%         \sbox0{\footnotesize\t}^^A
%         \ifdim\wd0>\linewidth
%           \ifdim\wd0>\lw
%             \sbox0{\scriptsize\t}^^A
%             \ifdim\wd0>\linewidth
%               \ifdim\wd0>\lw
%                 \sbox0{\tiny\t}^^A
%                 \ifdim\wd0>\linewidth
%                   \lwbox
%                 \else
%                   \usebox0
%                 \fi
%               \else
%                 \lwbox
%               \fi
%             \else
%               \usebox0
%             \fi
%           \else
%             \lwbox
%           \fi
%         \else
%           \usebox0
%         \fi
%       \else
%         \lwbox
%       \fi
%     \else
%       \usebox0
%     \fi
%   \else
%     \lwbox
%   \fi
% \else
%   \usebox0
% \fi
% \end{quote}
% If you have a \xfile{docstrip.cfg} that configures and enables \docstrip's
% TDS installing feature, then some files can already be in the right
% place, see the documentation of \docstrip.
%
% \subsection{Refresh file name databases}
%
% If your \TeX~distribution
% (\TeX\,Live, \mikTeX, \dots) relies on file name databases, you must refresh
% these. For example, \TeX\,Live\ users run \verb|texhash| or
% \verb|mktexlsr|.
%
% \subsection{Some details for the interested}
%
% \paragraph{Unpacking with \LaTeX.}
% The \xfile{.dtx} chooses its action depending on the format:
% \begin{description}
% \item[\plainTeX:] Run \docstrip\ and extract the files.
% \item[\LaTeX:] Generate the documentation.
% \end{description}
% If you insist on using \LaTeX\ for \docstrip\ (really,
% \docstrip\ does not need \LaTeX), then inform the autodetect routine
% about your intention:
% \begin{quote}
%   \verb|latex \let\install=y\input{pdfcrypt.dtx}|
% \end{quote}
% Do not forget to quote the argument according to the demands
% of your shell.
%
% \paragraph{Generating the documentation.}
% You can use both the \xfile{.dtx} or the \xfile{.drv} to generate
% the documentation. The process can be configured by the
% configuration file \xfile{ltxdoc.cfg}. For instance, put this
% line into this file, if you want to have A4 as paper format:
% \begin{quote}
%   \verb|\PassOptionsToClass{a4paper}{article}|
% \end{quote}
% An example follows how to generate the
% documentation with pdf\LaTeX:
% \begin{quote}
%\begin{verbatim}
%pdflatex pdfcrypt.dtx
%makeindex -s gind.ist pdfcrypt.idx
%pdflatex pdfcrypt.dtx
%makeindex -s gind.ist pdfcrypt.idx
%pdflatex pdfcrypt.dtx
%\end{verbatim}
% \end{quote}
%
% \begin{History}
%   \begin{Version}{2001/04/02 v0.1}
%   \item
%     First public version, published in the pdftex
%     mailing list for testing with pdfcrypt-20010331
%   \end{Version}
%   \begin{Version}{2001/07/19 v0.2}
%   \item
%     Default: all allowed.
%   \item
%     Support for \VTeX\ added.
%   \end{Version}
%   \begin{Version}{2001/07/19 v0.3}
%   \item
%     Bug fix: \VTeX\ letter for edit is M (modify).
%   \end{Version}
%   \begin{Version}{2001/07/19 v0.4}
%   \item
%     Bug fix: \cs{VTeXversion} is correct after
%     regenerating the format file.
%   \end{Version}
%   \begin{Version}{2001/08/05 v0.5}
%   \item
%     Syntax change in \pdfTeX\ 1.00a.
%   \end{Version}
%   \begin{Version}{2001/08/09 v0.6}
%   \item
%     Support of special characters:\\
%     input: |\{|, |\}|, |\\| for |{|, |}|, |\|\\
%     output: |"| in \VTeX
%   \item
%     Option debug added.
%   \end{Version}
%   \begin{Version}{2001/10/28 v0.7}
%   \item
%     Plain compatibility.
%   \item
%     \cs{nopdfcrypt} added.
%   \item
%     Typos corrected.
%   \end{Version}
%   \begin{Version}{2006/02/20 v0.8}
%   \item
%     Obsolete remarks for \pdfTeX.
%   \item
%     DTX framework.
%   \item
%     LPPL 1.3
%   \end{Version}
%   \begin{Version}{2007/04/11 v0.9}
%   \item
%     Line ends sanitized.
%   \end{Version}
%   \begin{Version}{2007/04/26 v1.0}
%   \item
%     Use of package \xpackage{infwarerr}.
%   \end{Version}
%   \begin{Version}{2016/05/16 v1.1}
%   \item
%     Documentation updates.
%   \end{Version}
% \end{History}
%
% \PrintIndex
%
% \Finale
\endinput

%        (quote the arguments according to the demands of your shell)
%
% Documentation:
%    (a) If pdfcrypt.drv is present:
%           latex pdfcrypt.drv
%    (b) Without pdfcrypt.drv:
%           latex pdfcrypt.dtx; ...
%    The class ltxdoc loads the configuration file ltxdoc.cfg
%    if available. Here you can specify further options, e.g.
%    use A4 as paper format:
%       \PassOptionsToClass{a4paper}{article}
%
%    Programm calls to get the documentation (example):
%       pdflatex pdfcrypt.dtx
%       makeindex -s gind.ist pdfcrypt.idx
%       pdflatex pdfcrypt.dtx
%       makeindex -s gind.ist pdfcrypt.idx
%       pdflatex pdfcrypt.dtx
%
% Installation:
%    TDS:tex/generic/oberdiek/pdfcrypt.sty
%    TDS:doc/latex/oberdiek/pdfcrypt.pdf
%    TDS:source/latex/oberdiek/pdfcrypt.dtx
%
%<*ignore>
\begingroup
  \catcode123=1 %
  \catcode125=2 %
  \def\x{LaTeX2e}%
\expandafter\endgroup
\ifcase 0\ifx\install y1\fi\expandafter
         \ifx\csname processbatchFile\endcsname\relax\else1\fi
         \ifx\fmtname\x\else 1\fi\relax
\else\csname fi\endcsname
%</ignore>
%<*install>
\input docstrip.tex
\Msg{************************************************************************}
\Msg{* Installation}
\Msg{* Package: pdfcrypt 2016/05/16 v1.1 Allows the setting of PDF encryption (HO)}
\Msg{************************************************************************}

\keepsilent
\askforoverwritefalse

\let\MetaPrefix\relax
\preamble

This is a generated file.

Project: pdfcrypt
Version: 2016/05/16 v1.1

Copyright (C)
   2001, 2006, 2007 Heiko Oberdiek
   2016-2019 Oberdiek Package Support Group

This work may be distributed and/or modified under the
conditions of the LaTeX Project Public License, either
version 1.3c of this license or (at your option) any later
version. This version of this license is in
   https://www.latex-project.org/lppl/lppl-1-3c.txt
and the latest version of this license is in
   https://www.latex-project.org/lppl.txt
and version 1.3 or later is part of all distributions of
LaTeX version 2005/12/01 or later.

This work has the LPPL maintenance status "maintained".

The Current Maintainers of this work are
Heiko Oberdiek and the Oberdiek Package Support Group
https://github.com/ho-tex/oberdiek/issues


The Base Interpreter refers to any `TeX-Format',
because some files are installed in TDS:tex/generic//.

This work consists of the main source file pdfcrypt.dtx
and the derived files
   pdfcrypt.sty, pdfcrypt.pdf, pdfcrypt.ins, pdfcrypt.drv.

\endpreamble
\let\MetaPrefix\DoubleperCent

\generate{%
  \file{pdfcrypt.ins}{\from{pdfcrypt.dtx}{install}}%
  \file{pdfcrypt.drv}{\from{pdfcrypt.dtx}{driver}}%
  \usedir{tex/generic/oberdiek}%
  \file{pdfcrypt.sty}{\from{pdfcrypt.dtx}{package}}%
  \nopreamble
  \nopostamble
%  \usedir{source/latex/oberdiek/catalogue}%
%  \file{pdfcrypt.xml}{\from{pdfcrypt.dtx}{catalogue}}%
}

\catcode32=13\relax% active space
\let =\space%
\Msg{************************************************************************}
\Msg{*}
\Msg{* To finish the installation you have to move the following}
\Msg{* file into a directory searched by TeX:}
\Msg{*}
\Msg{*     pdfcrypt.sty}
\Msg{*}
\Msg{* To produce the documentation run the file `pdfcrypt.drv'}
\Msg{* through LaTeX.}
\Msg{*}
\Msg{* Happy TeXing!}
\Msg{*}
\Msg{************************************************************************}

\endbatchfile
%</install>
%<*ignore>
\fi
%</ignore>
%<*driver>
\NeedsTeXFormat{LaTeX2e}
\ProvidesFile{pdfcrypt.drv}%
  [2016/05/16 v1.1 Allows the setting of PDF encryption (HO)]%
\documentclass{ltxdoc}
\usepackage{holtxdoc}[2011/11/22]
\begin{document}
  \DocInput{pdfcrypt.dtx}%
\end{document}
%</driver>
% \fi
%
%
% \CharacterTable
%  {Upper-case    \A\B\C\D\E\F\G\H\I\J\K\L\M\N\O\P\Q\R\S\T\U\V\W\X\Y\Z
%   Lower-case    \a\b\c\d\e\f\g\h\i\j\k\l\m\n\o\p\q\r\s\t\u\v\w\x\y\z
%   Digits        \0\1\2\3\4\5\6\7\8\9
%   Exclamation   \!     Double quote  \"     Hash (number) \#
%   Dollar        \$     Percent       \%     Ampersand     \&
%   Acute accent  \'     Left paren    \(     Right paren   \)
%   Asterisk      \*     Plus          \+     Comma         \,
%   Minus         \-     Point         \.     Solidus       \/
%   Colon         \:     Semicolon     \;     Less than     \<
%   Equals        \=     Greater than  \>     Question mark \?
%   Commercial at \@     Left bracket  \[     Backslash     \\
%   Right bracket \]     Circumflex    \^     Underscore    \_
%   Grave accent  \`     Left brace    \{     Vertical bar  \|
%   Right brace   \}     Tilde         \~}
%
% \GetFileInfo{pdfcrypt.drv}
%
% \title{The \xpackage{pdfcrypt} package}
% \date{2016/05/16 v1.1}
% \author{Heiko Oberdiek\thanks
% {Please report any issues at \url{https://github.com/ho-tex/oberdiek/issues}}}
%
% \maketitle
%
% \begin{abstract}
% This package supports the setting of pdf encryption options
% for \VTeX\ and some older versions of \pdfTeX.
% \end{abstract}
%
% \tableofcontents
%
% \section{Documentation}
%
% This package allows the setting of pdf encryption
% options for
% \begin{itemize}
% \item
%   \VTeX, since version 7.35, \url{http://www.micropress-inc.com/}.
% \item
%   \pdfTeX, patched by
%   \NameEmail{Ricardo Sanchez Carmenes}{carmenes@bioquimica.uniovi.es}.\\
%   Some supported versions are:
%   \begin{quote}
%     \pdfTeX-1.00a,\\
%     \pdfTeX-0.14h-pdfcrypt-20010310,\\
%     pdfcrypt-20010331
%   \end{quote}
% \end{itemize}
%
% \noindent
% \textcolor{red}{Note:} Since \pdfTeX-1.10a (2003-01-16) encryption
% support was dropped! Thus the package is now obsolete for recent
% \pdfTeX\ versions.
%
% \subsection{Alternatives}
%
% There are free alternatives, programs that can
% be used for postprocessing the pdf file:
% \begin{itemize}
% \item \textsf{pdftk}\\
%   \url{http://www.accesspdf.com/pdftk/}
% \item \textsf{Multivalent}\\
%   \url{http://multivalent.sourceforge.net/}
% \item \textsf{PDFBox}\\
%   \url{http://www.pdfbox.org/}
% \item \textsf{PDFTrans}\\
%   \url{http://maddingue.free.fr/softwares/pdftrans.html}
% \item \dots
% \end{itemize}
%
% \subsection{Usage}
%
% The options can be set as package options or
% with the command \cs{pdfcryptsetup}:
% \begin{quote}
%   |owner=|\meta{owner password}\\
%   |user=|\meta{user password}\\
%   |print|, |copy|, |edit|, |annotate=true|, |false|\\
%   |all|, |none|\\
%   |debug|
% \end{quote}
% The encryption is set at |\begin{document}| by default,
% but this can be forced for an earlier time
% using the option |set|.
% Example:
% \begin{quote}
%   |\usepackage[owner=Donald,user=Knuth,print=false]{pdfcrypt}|\\
%   \dots\\
%   |\begin{document}|
% \end{quote}
% or
% \begin{quote}
%   |\usepackage{pdfcrypt}|\\
%   |\pdfcryptsetup{owner=Donald,user=Knut}|\\
%   |\pdfcryptsetup{print=false,set}|\\
%   \dots\\
%   |\begin{document}|
% \end{quote}
%
% \subsubsection{Special characters}
%
% The characters `|{|', `|}|', and `|\|' have to be
% escaped with `|\|'. Then the string should not
% be specified as package option but with
% the macro \cs{pdfcryptsetup}, eg:
% \begin{quote}
%   |\pdfcryptsetup{user={\{\\Hello\}},print=false}|
% \end{quote}
% The password here is ``|{\Hello}|''.
% Active characters can be used and are not expanded.
% Macros except for |\{|, |\}|, and |\\| should not be used
% and are not expanded.
%
% \subsubsection{\cs{nopdfcrypt}}
%
% Often the whole process of pdf generation includes
% several \TeX\ compilations and perhaps other
% postprocessing steps. Therefore a feature would
% be useful to disable the encryption stuff in order
% to speed up the first compilations.
% Therefore package \xpackage{pdfcrypt} will look for
% the macro \cs{nopdfcrypt}. If it has the meaning of the
% letter `|Y|', the package will be disabled.
% Example:
% \begin{quote}
%   |pdflatex '\let\nopdfcrypt=Y\input myfile'|\\
%   |thumbpdf myfile|\\
%   |pdfelatex myfile|
% \end{quote}
%
% \subsubsection{Configuration file}
%
% If the file \xfile{pdfcrypt.cfg} exists it is loaded
% and can be used for common settings.
%
% \subsubsection{Support for \plainTeX}
%
% The package can also be used with \plainTeX. It defines the
% missing \LaTeX\ macros and loads package \xpackage{keyval}.
% The configuration file is not supported and automatically
% loaded.
%
% \subsection{Hints for \pdfTeX}
%
% This section is out of date, because the encryption support
% was removed from \pdfTeX. Therefore this hints are of
% historical interest only.
%
%\begin{verbatim}
%* There are packages such as thumbpdf that ship out
%  some stuff (\immediate\pdfobj{...}). In these cases
%  \pdfcrypt will be ignored without warning or error
%  message. Therefore the package pdfcrypt should be
%  loaded before and option "set" should be used in
%  order to force the call of \pdfcrypt, for example:
%     \usepackage{pdfcrypt}
%     \pdfcryptsetup{..., set}
%     ...
%     \usepackage{thumbpdf}
%
%* Compiling pdfTeX (eg. version 1.00a-pretest-20010804):
%  pdfTeX versions are available at
%      ftp://ftp.cstug.cz/pub/tex/local/cstug/thanh/pdftex/snapshots/
%  Unpack the latest .tgz file:
%      > tar xzf pdftex-20010804.tgz
%  Enable encryption support:
%      src> cd src/texk/web2c/pdftexdir
%      src/texk/web2c/pdftexdir> ln -fs pdfcrypt-full.c pdfcrypt.c
%  Compile:
%      src/texk/web2c/pdftexdir> cd ../../..
%      src> ./configure
%      src> cd texk/web2c
%      src/texk/web2c> make pdftexbin
%  At last install the binary and pool files.
%\end{verbatim}
%
% \StopEventually{
% }
%
% \section{Implementation}
%
%    \begin{macrocode}
%<*package>
%    \end{macrocode}
%
% \subsection{Help macros for \plainTeX}
%
%    \begin{macrocode}
\expandafter\edef\csname pc@endinput\endcsname{%
  \catcode`\noexpand\@=\the\catcode`\@\relax
  \noexpand\endinput
}
\catcode`\@=11 %

\expandafter\ifx\csname @firstoftwo\endcsname\relax
  \long\def\@firstoftwo#1#2{#1}%
\fi

\expandafter\ifx\csname @secondoftwo\endcsname\relax
  \long\def\@secondoftwo#1#2{#2}%
\fi

\expandafter\ifx\csname @ifundefined\endcsname\relax
  \def\@ifundefined#1{%
    \expandafter\ifx\csname #1\endcsname\relax
      \expandafter\@firstoftwo
    \else
      \expandafter\@secondoftwo
    \fi
  }%
\fi

\@ifundefined{@ifnextchar}{%
  \long\def\@ifnextchar#1#2#3{%
    \let\reserved@d=#1%
    \def\reserved@a{#2}%
    \def\reserved@b{#3}%
    \futurelet\@let@token\@ifnch
  }%
  \def\@ifnch{%
    \ifx\@let@token\@sptoken
      \let\reserved@c\@xifnch
    \else
      \ifx\@let@token\reserved@d
        \let\reserved@c\reserved@a
      \else
        \let\reserved@c\reserved@b
      \fi
    \fi
    \reserved@c
  }%
  \begingroup
    \def\:{\global\let\@sptoken= }%
    \: %
    \def\:{\@xifnch}%
    \expandafter\gdef\: {\futurelet\@let@token\@ifnch}%
  \endgroup
}{}

\@ifundefined{ProvidesPackage}{%
  \def\ProvidesPackage#1{%
    \@ifnextchar[{\pc@ProvidesPackage{#1}}%
                 {\pc@ProvidesPackage{#1}[]}%
  }%
  \def\pc@ProvidesPackage#1[#2]{%
    \immediate\write-1{Package: #1 #2}%
  }%
}{}

\begingroup\expandafter\expandafter\expandafter\endgroup
\expandafter\ifx\csname RequirePackage\endcsname\relax
  \input infwarerr.sty\relax
\else
  \RequirePackage{infwarerr}%
\fi

\@ifundefined{@gobble}{%
  \long\def\@gobble#1{}%
}{}

\@ifundefined{@empty}{%
  \def\@empty{}%
}{}
%    \end{macrocode}
%
% \subsection{Package Identification and checks}
%
%    \begin{macrocode}
\ProvidesPackage{pdfcrypt}%
  [2016/05/16 v1.1 Allows the setting of PDF encryption (HO)]%
%    \end{macrocode}
%
%    \begin{macrocode}
\@ifundefined{pdfcryptsetup}{%
  \let\pdfcryptsetup\@gobble
}{%
  \@PackageWarningNoLine{pdfcrypt}{Package pdfcrypt is already loaded}%
  \pc@endinput
}
%    \end{macrocode}
%    Support for \cs{nopdfcrypt}.
%    \begin{macrocode}
\newif\ifpc@nopdfcrypt
\ifx Y\nopdfcrypt
  \@PackageWarningNoLine{pdfcrypt}{%
    Encryption disabled by \string \nopdfcrypt\space request%
  }%
  \global\pc@nopdfcrypttrue
\fi
%    \end{macrocode}
%
% \subsection{Driver detection}
%
%    \begin{macrocode}
\let\pc@driver\@empty
\begingroup
  % pdfTeX detection
  \@ifundefined{pdfoutput}{%
  }{%
    \ifcase\pdfoutput
    \else
      \@ifundefined{pdfcrypt}{%
        \@PackageError{pdfcrypt}{%
          PDF encryption is not supported with this pdfTeX%
        }{%
          Encryption support was added in 0.14h (2001/03/10)\MessageBreak
          and removed in 1.10a (2003/01/16).%
        }%
        \endgroup
        \pc@endinput
      }{%
        \gdef\pc@driver{pdftex}%
      }%
    \fi
  }%
  % VTeX detection
  \@ifundefined{OpMode}{%
  }{%
    \ifnum\OpMode=1 %
      \ifnum\@ifundefined{VTeXversion}0\VTeXversion<735 %
        \@PackageError{pdfcrypt}{%
          PDF encryption is not supported with this VTeX%
        }{%
          You need VTeX 7.35 or higher.%
        }%
        \endgroup
        \pc@endinput
      \else
        \gdef\pc@driver{vtex}%
      \fi
    \fi
  }%
\endgroup
%    \end{macrocode}
%
% \subsection{Load package \xpackage{keyval}}
%
%    \begin{macrocode}
\@ifundefined{@makeother}{%
  \def\@makeother#1{\catcode`#1=12\relax}%
}{}

\@ifundefined{g@addto@macro}{%
  \long\def\g@addto@macro#1#2{%
    \begingroup
      \toks@\expandafter{#1#2}%
      \xdef#1{\the\toks@}%
    \endgroup
  }%
}{}

\@ifundefined{@namedef}{%
  \def\@namedef#1{\expandafter\def\csname#1\endcsname}%
}{}

\@ifundefined{@nameuse}{%
  \def\@nameuse#1{\csname #1\endcsname}%
}{}

\def\pc@KeyvalRestore{%
  \let\pc@KeyvalRestore\@undefined
}

\let\pcOrg@NeedsTeXFormat\NeedsTeXFormat
\@ifundefined{NeedsTeXFormat}{%
  \def\NeedsTeXFormat#1{}%
  \g@addto@macro\pc@KeyvalRestore{%
    \let\NeedsTeXFormat\pcOrg@NeedsTeXFormat
  }%
}{}

\let\pcOrg@DeclareOption\DeclareOption
\@ifundefined{DeclareOption}{%
  \def\DeclareOption#1#2{#2}%
  \g@addto@macro\pc@KeyvalRestore{%
    \let\DeclareOption\pcOrg@DeclareOption
  }%
}{}

\let\pcOrg@ExecuteOptions\ExecuteOptions
\@ifundefined{ExecuteOptions}{%
  \def\ExecuteOptions#1{}%
  \g@addto@macro\pc@KeyvalRestore{%
    \let\ExecuteOptions\pcOrg@ExecuteOptions
  }%
}{}

\let\pcOrg@ProcessOptions\ProcessOptions
\@ifundefined{ProcessOptions}{%
  \def\ProcessOptions{}%
  \g@addto@macro\pc@KeyvalRestore{%
    \let\ProcessOptions\pcOrg@ProcessOptions
  }%
}{}%

\begingroup\expandafter\expandafter\expandafter\endgroup
\expandafter\ifx\csname RequirePackage\endcsname\relax
  \input keyval.sty\relax
\else
  \RequirePackage{keyval}%
\fi
\pc@KeyvalRestore
%    \end{macrocode}
%
% \subsection{Define options}
%
%    \begin{macrocode}
\@ifundefined{@dblarg}{%
  \long\def\@dblarg#1{\@ifnextchar[{#1}{\@xdblarg{#1}}}%
  \long\def\@xdblarg#1#2{#1[{#2}]{#2}}%
}{}

\newif\ifpc@set
\newif\ifpc@print
\newif\ifpc@copy
\newif\ifpc@edit
\newif\ifpc@annotate
\newif\ifpc@debug
\let\pc@owner\@empty
\let\pc@user\@empty

% default: allow all
\pc@printtrue
\pc@copytrue
\pc@edittrue
\pc@annotatetrue

\edef\pc@temp{\catcode`\noexpand\"=\the\catcode`\"\relax}
\@makeother\"
\def\pc@set{%
  \@PackageInfo{pdfcrypt}{%
    \ifpc@debug
      \ifx\pc@owner\@empty
        No owner password%
      \else
        Owner password: `\pc@owner'%
      \fi
      \MessageBreak
      \ifx\pc@user\@empty
        No user password%
      \else
        User password: `\pc@user'%
      \fi
      \MessageBreak
      Flags: %
      \ifpc@print   \else no\fi print, %
      \ifpc@copy    \else no\fi copy, %
      \ifpc@edit    \else no\fi edit, %
      \ifpc@annotate\else no\fi annotate%
      \MessageBreak
    \fi
    \ifpc@nopdfcrypt
      Encryption is disabled by `\string\nopdfcrypt'%
    \else
      Encryption is set for `\pc@driver'%
    \fi
  }%
  \ifpc@nopdfcrypt
  \else
    \@ifundefined{pc@set@\pc@driver}{%
      \ifx\pc@driver\@empty
        \@PackageError{pdfcrypt}{No driver for encryption %
          support found}\@ehc
      \else
        \@PackageError{pdfcrypt}{Cannot set encryption for %
          unknown driver `\pc@driver'}\@ehc
      \fi
    }{%
      \@nameuse{pc@set@\pc@driver}%
    }%
  \fi
}
\def\pc@set@pdftex{%
  \ifnum\pdftexversion<100 %
    \pc@set@pdftexold
  \else
    \pc@set@pdftexnew
  \fi
}
\def\pc@set@pdftexold{%
  \pdfcrypt{%
    owner "\pc@owner" %
    user "\pc@user" %
    \ifpc@print   \else no\fi print %
    \ifpc@copy    \else no\fi copy %
    \ifpc@edit    \else no\fi edit %
    \ifpc@annotate\else no\fi annotate%
  }%
}
\def\pc@set@pdftexnew{%
  \pdfcrypt
    owner{\pc@owner}%
    user{\pc@user}%
    \ifpc@print   \else no\fi print %
    \ifpc@copy    \else no\fi copy %
    \ifpc@edit    \else no\fi edit %
    \ifpc@annotate\else no\fi annotate%
  \relax
}
\def\pc@set@vtex{%
  \immediate\special{!security %
    O=\pc@MakeVTeXString\pc@owner,%
    U=\pc@MakeVTeXString\pc@user,%
    P\ifpc@print +\else -\fi,%
    C\ifpc@copy +\else -\fi,%
    M\ifpc@edit +\else -\fi,%
    A\ifpc@annotate +\else -\fi
  }%
}
\def\pc@MakeVTeXString#1{%
  "\expandafter\pc@@MakeVTeXString#1"\@nil"%
}
\def\pc@@MakeVTeXString#1"#2\@nil{%
  #1%
  \ifx\\#2\\%
  \else
    ""%
    \@ReturnAfterFi{%
      \pc@@MakeVTeXString#2\@nil
    }%
  \fi
}
\long\def\@ReturnAfterFi#1\fi{\fi#1}
\pc@temp

\begingroup
  \catcode`\ =12 \gdef\pc@spaceother{ }\catcode`\ =10\relax
  \catcode`\|=0 %
  \catcode`\\=12 %
  |gdef|pc@DefString#1#2{%
    |def#1{#2}%
    |edef#1{|expandafter|strip@prefix|meaning#1}%
    |edef#1{|expandafter|pc@SpaceToOther#1 |@nil}%
    |edef#1{|expandafter|pc@EscapeRemove#1|@empty\|@empty|@nil}%
  }%
  |gdef|pc@EscapeRemove#1\#2#3|@nil{%
    #1#2%
    |ifx|\#3|\%%
    |else
      |@ReturnAfterFi{%
        |pc@EscapeRemove#3|@nil
      }%
    |fi
  }%
|endgroup
\def\pc@SpaceToOther#1 #2\@nil{%
  #1%
  \ifx\\#2\\%
  \else
    \pc@spaceother
    \@ReturnAfterFi{%
      \pc@SpaceToOther#2\@nil
    }%
  \fi
}

\def\pc@boolkey{\@dblarg\pc@@boolkey}
\def\pc@@boolkey[#1]#2#3{%
  \lowercase{\def\pc@temp{#3}}%
  \ifx\pc@temp\@empty
    \let\pc@temp\pc@true
  \fi
  \ifx\pc@temp\pc@true
  \else
    \ifx\pc@temp\pc@false
    \else
      \let\pc@temp\relax
    \fi
  \fi
  \ifx\pc@temp\relax
    \@PackageWarning{pdfcrypt}{%
      Unexpected value \string`#3\string' of %
      option \string`#2\string'\MessageBreak
      instead of %
      \string`true\string' or \string`false\string'%
    }%
  \else
    \csname pc@#2\pc@temp\endcsname
  \fi
}
\def\pc@true{true}
\def\pc@false{false}

\define@key{pc}{set}[true]{%
  \pc@boolkey{set}{#1}%
}
\define@key{pc}{pdftex}[]{%
  \def\pc@driver{pdftex}%
}
\define@key{pc}{vtex}[]{%
  \def\pc@driver{vtex}%
}
\define@key{pc}{print}[true]{%
  \pc@boolkey{print}{#1}%
}
\define@key{pc}{copy}[true]{%
  \pc@boolkey{copy}{#1}%
}
\define@key{pc}{edit}[true]{%
  \pc@boolkey{edit}{#1}%
}
\define@key{pc}{annotate}[true]{%
  \pc@boolkey{annotate}{#1}%
}
\define@key{pc}{all}[]{%
  \pc@boolkey{print}{true}%
  \pc@boolkey{copy}{true}%
  \pc@boolkey{edit}{true}%
  \pc@boolkey{annotate}{true}%
}
\define@key{pc}{none}[]{%
  \pc@boolkey{print}{false}%
  \pc@boolkey{copy}{false}%
  \pc@boolkey{edit}{false}%
  \pc@boolkey{annotate}{false}%
}

\define@key{pc}{owner}{%
  \pc@DefString\pc@owner{#1}%
}
\define@key{pc}{user}{%
  \pc@DefString\pc@user{#1}%
}
\define@key{pc}{debug}[true]{%
  \pc@boolkey{debug}{#1}%
}

\def\pdfcryptsetup#1{%
  \setkeys{pc}{#1}%
  \ifpc@set
    \pc@set
    \global\let\pc@set\relax
    \gdef\pdfcryptsetup##1{%
      \@PackageWarning{pdfcrypt}{%
        Encryption options are already set\MessageBreak
        new values are ignored%
      }%
    }%
  \fi
}
\begingroup\expandafter\expandafter\expandafter\endgroup
\expandafter\ifx\csname @onlypreamble\endcsname\relax
\else
  \@onlypreamble\pdfcryptsetup
\fi
%    \end{macrocode}
%
% \subsection{support of configuration file}
%    \begin{macrocode}
\begingroup\expandafter\expandafter\expandafter\endgroup
\expandafter\ifx\csname InputIfFileExists\endcsname\relax
  \@PackageInfo{pdfcrypt}{%
    Configuration file pdfcrypt.cfg not supported.%
  }%
\else
  \let\pc@ExecuteOptions\ExecuteOptions
  \InputIfFileExists{pdfcrypt.cfg}{}{}%
  \let\ExecuteOptions\pc@ExecuteOptions
\fi
%    \end{macrocode}
%
% \subsection{Package options}
%
%    Plain format does not know package options.
%    \begin{macrocode}
\begingroup\expandafter\expandafter\expandafter\endgroup
\expandafter\ifx\csname @classoptionslist\endcsname\relax
  \expandafter\pc@endinput
\fi
%    \end{macrocode}
%    Process global and local options.
%    \begin{macrocode}
\def\pc@ProcessOptionsWithKV{%
  \let\pc@temp\@empty
  \@for\CurrentOption:=\@classoptionslist\do{%
    \@ifundefined{KV@pc@\CurrentOption}{}{%
      \edef\pc@temp{\pc@temp,\CurrentOption,}%
      \@expandtwoargs\@removeelement\CurrentOption
        \@unusedoptionlist\@unusedoptionlist
    }%
  }%
  \edef\pc@temp{%
    \noexpand\pdfcryptsetup{%
      \pc@temp\@ptionlist{\@currname.\@currext}%
    }%
  }%
  \pc@temp
}
\pc@ProcessOptionsWithKV
\AtEndOfPackage{\let\@unprocessedoptions\relax}
\AtBeginDocument{\pc@set}

\pc@endinput
%    \end{macrocode}
%
%    \begin{macrocode}
%</package>
%    \end{macrocode}
%
% \section{Installation}
%
% \subsection{Download}
%
% \paragraph{Package.} This package is available on
% CTAN\footnote{\CTANpkg{pdfcrypt}}:
% \begin{description}
% \item[\CTAN{macros/latex/contrib/oberdiek/pdfcrypt.dtx}] The source file.
% \item[\CTAN{macros/latex/contrib/oberdiek/pdfcrypt.pdf}] Documentation.
% \end{description}
%
%
% \paragraph{Bundle.} All the packages of the bundle `oberdiek'
% are also available in a TDS compliant ZIP archive. There
% the packages are already unpacked and the documentation files
% are generated. The files and directories obey the TDS standard.
% \begin{description}
% \item[\CTANinstall{install/macros/latex/contrib/oberdiek.tds.zip}]
% \end{description}
% \emph{TDS} refers to the standard ``A Directory Structure
% for \TeX\ Files'' (\CTANpkg{tds}). Directories
% with \xfile{texmf} in their name are usually organized this way.
%
% \subsection{Bundle installation}
%
% \paragraph{Unpacking.} Unpack the \xfile{oberdiek.tds.zip} in the
% TDS tree (also known as \xfile{texmf} tree) of your choice.
% Example (linux):
% \begin{quote}
%   |unzip oberdiek.tds.zip -d ~/texmf|
% \end{quote}
%
% \subsection{Package installation}
%
% \paragraph{Unpacking.} The \xfile{.dtx} file is a self-extracting
% \docstrip\ archive. The files are extracted by running the
% \xfile{.dtx} through \plainTeX:
% \begin{quote}
%   \verb|tex pdfcrypt.dtx|
% \end{quote}
%
% \paragraph{TDS.} Now the different files must be moved into
% the different directories in your installation TDS tree
% (also known as \xfile{texmf} tree):
% \begin{quote}
% \def\t{^^A
% \begin{tabular}{@{}>{\ttfamily}l@{ $\rightarrow$ }>{\ttfamily}l@{}}
%   pdfcrypt.sty & tex/generic/oberdiek/pdfcrypt.sty\\
%   pdfcrypt.pdf & doc/latex/oberdiek/pdfcrypt.pdf\\
%   pdfcrypt.dtx & source/latex/oberdiek/pdfcrypt.dtx\\
% \end{tabular}^^A
% }^^A
% \sbox0{\t}^^A
% \ifdim\wd0>\linewidth
%   \begingroup
%     \advance\linewidth by\leftmargin
%     \advance\linewidth by\rightmargin
%   \edef\x{\endgroup
%     \def\noexpand\lw{\the\linewidth}^^A
%   }\x
%   \def\lwbox{^^A
%     \leavevmode
%     \hbox to \linewidth{^^A
%       \kern-\leftmargin\relax
%       \hss
%       \usebox0
%       \hss
%       \kern-\rightmargin\relax
%     }^^A
%   }^^A
%   \ifdim\wd0>\lw
%     \sbox0{\small\t}^^A
%     \ifdim\wd0>\linewidth
%       \ifdim\wd0>\lw
%         \sbox0{\footnotesize\t}^^A
%         \ifdim\wd0>\linewidth
%           \ifdim\wd0>\lw
%             \sbox0{\scriptsize\t}^^A
%             \ifdim\wd0>\linewidth
%               \ifdim\wd0>\lw
%                 \sbox0{\tiny\t}^^A
%                 \ifdim\wd0>\linewidth
%                   \lwbox
%                 \else
%                   \usebox0
%                 \fi
%               \else
%                 \lwbox
%               \fi
%             \else
%               \usebox0
%             \fi
%           \else
%             \lwbox
%           \fi
%         \else
%           \usebox0
%         \fi
%       \else
%         \lwbox
%       \fi
%     \else
%       \usebox0
%     \fi
%   \else
%     \lwbox
%   \fi
% \else
%   \usebox0
% \fi
% \end{quote}
% If you have a \xfile{docstrip.cfg} that configures and enables \docstrip's
% TDS installing feature, then some files can already be in the right
% place, see the documentation of \docstrip.
%
% \subsection{Refresh file name databases}
%
% If your \TeX~distribution
% (\TeX\,Live, \mikTeX, \dots) relies on file name databases, you must refresh
% these. For example, \TeX\,Live\ users run \verb|texhash| or
% \verb|mktexlsr|.
%
% \subsection{Some details for the interested}
%
% \paragraph{Unpacking with \LaTeX.}
% The \xfile{.dtx} chooses its action depending on the format:
% \begin{description}
% \item[\plainTeX:] Run \docstrip\ and extract the files.
% \item[\LaTeX:] Generate the documentation.
% \end{description}
% If you insist on using \LaTeX\ for \docstrip\ (really,
% \docstrip\ does not need \LaTeX), then inform the autodetect routine
% about your intention:
% \begin{quote}
%   \verb|latex \let\install=y% \iffalse meta-comment
%
% File: pdfcrypt.dtx
% Version: 2016/05/16 v1.1
% Info: Allows the setting of PDF encryption
%
% Copyright (C)
%    2001, 2006, 2007 Heiko Oberdiek
%    2016-2019 Oberdiek Package Support Group
%    https://github.com/ho-tex/oberdiek/issues
%
% This work may be distributed and/or modified under the
% conditions of the LaTeX Project Public License, either
% version 1.3c of this license or (at your option) any later
% version. This version of this license is in
%    https://www.latex-project.org/lppl/lppl-1-3c.txt
% and the latest version of this license is in
%    https://www.latex-project.org/lppl.txt
% and version 1.3 or later is part of all distributions of
% LaTeX version 2005/12/01 or later.
%
% This work has the LPPL maintenance status "maintained".
%
% The Current Maintainers of this work are
% Heiko Oberdiek and the Oberdiek Package Support Group
% https://github.com/ho-tex/oberdiek/issues
%
% The Base Interpreter refers to any `TeX-Format',
% because some files are installed in TDS:tex/generic//.
%
% This work consists of the main source file pdfcrypt.dtx
% and the derived files
%    pdfcrypt.sty, pdfcrypt.pdf, pdfcrypt.ins, pdfcrypt.drv.
%
% Distribution:
%    CTAN:macros/latex/contrib/oberdiek/pdfcrypt.dtx
%    CTAN:macros/latex/contrib/oberdiek/pdfcrypt.pdf
%
% Unpacking:
%    (a) If pdfcrypt.ins is present:
%           tex pdfcrypt.ins
%    (b) Without pdfcrypt.ins:
%           tex pdfcrypt.dtx
%    (c) If you insist on using LaTeX
%           latex \let\install=y\input{pdfcrypt.dtx}
%        (quote the arguments according to the demands of your shell)
%
% Documentation:
%    (a) If pdfcrypt.drv is present:
%           latex pdfcrypt.drv
%    (b) Without pdfcrypt.drv:
%           latex pdfcrypt.dtx; ...
%    The class ltxdoc loads the configuration file ltxdoc.cfg
%    if available. Here you can specify further options, e.g.
%    use A4 as paper format:
%       \PassOptionsToClass{a4paper}{article}
%
%    Programm calls to get the documentation (example):
%       pdflatex pdfcrypt.dtx
%       makeindex -s gind.ist pdfcrypt.idx
%       pdflatex pdfcrypt.dtx
%       makeindex -s gind.ist pdfcrypt.idx
%       pdflatex pdfcrypt.dtx
%
% Installation:
%    TDS:tex/generic/oberdiek/pdfcrypt.sty
%    TDS:doc/latex/oberdiek/pdfcrypt.pdf
%    TDS:source/latex/oberdiek/pdfcrypt.dtx
%
%<*ignore>
\begingroup
  \catcode123=1 %
  \catcode125=2 %
  \def\x{LaTeX2e}%
\expandafter\endgroup
\ifcase 0\ifx\install y1\fi\expandafter
         \ifx\csname processbatchFile\endcsname\relax\else1\fi
         \ifx\fmtname\x\else 1\fi\relax
\else\csname fi\endcsname
%</ignore>
%<*install>
\input docstrip.tex
\Msg{************************************************************************}
\Msg{* Installation}
\Msg{* Package: pdfcrypt 2016/05/16 v1.1 Allows the setting of PDF encryption (HO)}
\Msg{************************************************************************}

\keepsilent
\askforoverwritefalse

\let\MetaPrefix\relax
\preamble

This is a generated file.

Project: pdfcrypt
Version: 2016/05/16 v1.1

Copyright (C)
   2001, 2006, 2007 Heiko Oberdiek
   2016-2019 Oberdiek Package Support Group

This work may be distributed and/or modified under the
conditions of the LaTeX Project Public License, either
version 1.3c of this license or (at your option) any later
version. This version of this license is in
   https://www.latex-project.org/lppl/lppl-1-3c.txt
and the latest version of this license is in
   https://www.latex-project.org/lppl.txt
and version 1.3 or later is part of all distributions of
LaTeX version 2005/12/01 or later.

This work has the LPPL maintenance status "maintained".

The Current Maintainers of this work are
Heiko Oberdiek and the Oberdiek Package Support Group
https://github.com/ho-tex/oberdiek/issues


The Base Interpreter refers to any `TeX-Format',
because some files are installed in TDS:tex/generic//.

This work consists of the main source file pdfcrypt.dtx
and the derived files
   pdfcrypt.sty, pdfcrypt.pdf, pdfcrypt.ins, pdfcrypt.drv.

\endpreamble
\let\MetaPrefix\DoubleperCent

\generate{%
  \file{pdfcrypt.ins}{\from{pdfcrypt.dtx}{install}}%
  \file{pdfcrypt.drv}{\from{pdfcrypt.dtx}{driver}}%
  \usedir{tex/generic/oberdiek}%
  \file{pdfcrypt.sty}{\from{pdfcrypt.dtx}{package}}%
  \nopreamble
  \nopostamble
%  \usedir{source/latex/oberdiek/catalogue}%
%  \file{pdfcrypt.xml}{\from{pdfcrypt.dtx}{catalogue}}%
}

\catcode32=13\relax% active space
\let =\space%
\Msg{************************************************************************}
\Msg{*}
\Msg{* To finish the installation you have to move the following}
\Msg{* file into a directory searched by TeX:}
\Msg{*}
\Msg{*     pdfcrypt.sty}
\Msg{*}
\Msg{* To produce the documentation run the file `pdfcrypt.drv'}
\Msg{* through LaTeX.}
\Msg{*}
\Msg{* Happy TeXing!}
\Msg{*}
\Msg{************************************************************************}

\endbatchfile
%</install>
%<*ignore>
\fi
%</ignore>
%<*driver>
\NeedsTeXFormat{LaTeX2e}
\ProvidesFile{pdfcrypt.drv}%
  [2016/05/16 v1.1 Allows the setting of PDF encryption (HO)]%
\documentclass{ltxdoc}
\usepackage{holtxdoc}[2011/11/22]
\begin{document}
  \DocInput{pdfcrypt.dtx}%
\end{document}
%</driver>
% \fi
%
%
% \CharacterTable
%  {Upper-case    \A\B\C\D\E\F\G\H\I\J\K\L\M\N\O\P\Q\R\S\T\U\V\W\X\Y\Z
%   Lower-case    \a\b\c\d\e\f\g\h\i\j\k\l\m\n\o\p\q\r\s\t\u\v\w\x\y\z
%   Digits        \0\1\2\3\4\5\6\7\8\9
%   Exclamation   \!     Double quote  \"     Hash (number) \#
%   Dollar        \$     Percent       \%     Ampersand     \&
%   Acute accent  \'     Left paren    \(     Right paren   \)
%   Asterisk      \*     Plus          \+     Comma         \,
%   Minus         \-     Point         \.     Solidus       \/
%   Colon         \:     Semicolon     \;     Less than     \<
%   Equals        \=     Greater than  \>     Question mark \?
%   Commercial at \@     Left bracket  \[     Backslash     \\
%   Right bracket \]     Circumflex    \^     Underscore    \_
%   Grave accent  \`     Left brace    \{     Vertical bar  \|
%   Right brace   \}     Tilde         \~}
%
% \GetFileInfo{pdfcrypt.drv}
%
% \title{The \xpackage{pdfcrypt} package}
% \date{2016/05/16 v1.1}
% \author{Heiko Oberdiek\thanks
% {Please report any issues at \url{https://github.com/ho-tex/oberdiek/issues}}}
%
% \maketitle
%
% \begin{abstract}
% This package supports the setting of pdf encryption options
% for \VTeX\ and some older versions of \pdfTeX.
% \end{abstract}
%
% \tableofcontents
%
% \section{Documentation}
%
% This package allows the setting of pdf encryption
% options for
% \begin{itemize}
% \item
%   \VTeX, since version 7.35, \url{http://www.micropress-inc.com/}.
% \item
%   \pdfTeX, patched by
%   \NameEmail{Ricardo Sanchez Carmenes}{carmenes@bioquimica.uniovi.es}.\\
%   Some supported versions are:
%   \begin{quote}
%     \pdfTeX-1.00a,\\
%     \pdfTeX-0.14h-pdfcrypt-20010310,\\
%     pdfcrypt-20010331
%   \end{quote}
% \end{itemize}
%
% \noindent
% \textcolor{red}{Note:} Since \pdfTeX-1.10a (2003-01-16) encryption
% support was dropped! Thus the package is now obsolete for recent
% \pdfTeX\ versions.
%
% \subsection{Alternatives}
%
% There are free alternatives, programs that can
% be used for postprocessing the pdf file:
% \begin{itemize}
% \item \textsf{pdftk}\\
%   \url{http://www.accesspdf.com/pdftk/}
% \item \textsf{Multivalent}\\
%   \url{http://multivalent.sourceforge.net/}
% \item \textsf{PDFBox}\\
%   \url{http://www.pdfbox.org/}
% \item \textsf{PDFTrans}\\
%   \url{http://maddingue.free.fr/softwares/pdftrans.html}
% \item \dots
% \end{itemize}
%
% \subsection{Usage}
%
% The options can be set as package options or
% with the command \cs{pdfcryptsetup}:
% \begin{quote}
%   |owner=|\meta{owner password}\\
%   |user=|\meta{user password}\\
%   |print|, |copy|, |edit|, |annotate=true|, |false|\\
%   |all|, |none|\\
%   |debug|
% \end{quote}
% The encryption is set at |\begin{document}| by default,
% but this can be forced for an earlier time
% using the option |set|.
% Example:
% \begin{quote}
%   |\usepackage[owner=Donald,user=Knuth,print=false]{pdfcrypt}|\\
%   \dots\\
%   |\begin{document}|
% \end{quote}
% or
% \begin{quote}
%   |\usepackage{pdfcrypt}|\\
%   |\pdfcryptsetup{owner=Donald,user=Knut}|\\
%   |\pdfcryptsetup{print=false,set}|\\
%   \dots\\
%   |\begin{document}|
% \end{quote}
%
% \subsubsection{Special characters}
%
% The characters `|{|', `|}|', and `|\|' have to be
% escaped with `|\|'. Then the string should not
% be specified as package option but with
% the macro \cs{pdfcryptsetup}, eg:
% \begin{quote}
%   |\pdfcryptsetup{user={\{\\Hello\}},print=false}|
% \end{quote}
% The password here is ``|{\Hello}|''.
% Active characters can be used and are not expanded.
% Macros except for |\{|, |\}|, and |\\| should not be used
% and are not expanded.
%
% \subsubsection{\cs{nopdfcrypt}}
%
% Often the whole process of pdf generation includes
% several \TeX\ compilations and perhaps other
% postprocessing steps. Therefore a feature would
% be useful to disable the encryption stuff in order
% to speed up the first compilations.
% Therefore package \xpackage{pdfcrypt} will look for
% the macro \cs{nopdfcrypt}. If it has the meaning of the
% letter `|Y|', the package will be disabled.
% Example:
% \begin{quote}
%   |pdflatex '\let\nopdfcrypt=Y\input myfile'|\\
%   |thumbpdf myfile|\\
%   |pdfelatex myfile|
% \end{quote}
%
% \subsubsection{Configuration file}
%
% If the file \xfile{pdfcrypt.cfg} exists it is loaded
% and can be used for common settings.
%
% \subsubsection{Support for \plainTeX}
%
% The package can also be used with \plainTeX. It defines the
% missing \LaTeX\ macros and loads package \xpackage{keyval}.
% The configuration file is not supported and automatically
% loaded.
%
% \subsection{Hints for \pdfTeX}
%
% This section is out of date, because the encryption support
% was removed from \pdfTeX. Therefore this hints are of
% historical interest only.
%
%\begin{verbatim}
%* There are packages such as thumbpdf that ship out
%  some stuff (\immediate\pdfobj{...}). In these cases
%  \pdfcrypt will be ignored without warning or error
%  message. Therefore the package pdfcrypt should be
%  loaded before and option "set" should be used in
%  order to force the call of \pdfcrypt, for example:
%     \usepackage{pdfcrypt}
%     \pdfcryptsetup{..., set}
%     ...
%     \usepackage{thumbpdf}
%
%* Compiling pdfTeX (eg. version 1.00a-pretest-20010804):
%  pdfTeX versions are available at
%      ftp://ftp.cstug.cz/pub/tex/local/cstug/thanh/pdftex/snapshots/
%  Unpack the latest .tgz file:
%      > tar xzf pdftex-20010804.tgz
%  Enable encryption support:
%      src> cd src/texk/web2c/pdftexdir
%      src/texk/web2c/pdftexdir> ln -fs pdfcrypt-full.c pdfcrypt.c
%  Compile:
%      src/texk/web2c/pdftexdir> cd ../../..
%      src> ./configure
%      src> cd texk/web2c
%      src/texk/web2c> make pdftexbin
%  At last install the binary and pool files.
%\end{verbatim}
%
% \StopEventually{
% }
%
% \section{Implementation}
%
%    \begin{macrocode}
%<*package>
%    \end{macrocode}
%
% \subsection{Help macros for \plainTeX}
%
%    \begin{macrocode}
\expandafter\edef\csname pc@endinput\endcsname{%
  \catcode`\noexpand\@=\the\catcode`\@\relax
  \noexpand\endinput
}
\catcode`\@=11 %

\expandafter\ifx\csname @firstoftwo\endcsname\relax
  \long\def\@firstoftwo#1#2{#1}%
\fi

\expandafter\ifx\csname @secondoftwo\endcsname\relax
  \long\def\@secondoftwo#1#2{#2}%
\fi

\expandafter\ifx\csname @ifundefined\endcsname\relax
  \def\@ifundefined#1{%
    \expandafter\ifx\csname #1\endcsname\relax
      \expandafter\@firstoftwo
    \else
      \expandafter\@secondoftwo
    \fi
  }%
\fi

\@ifundefined{@ifnextchar}{%
  \long\def\@ifnextchar#1#2#3{%
    \let\reserved@d=#1%
    \def\reserved@a{#2}%
    \def\reserved@b{#3}%
    \futurelet\@let@token\@ifnch
  }%
  \def\@ifnch{%
    \ifx\@let@token\@sptoken
      \let\reserved@c\@xifnch
    \else
      \ifx\@let@token\reserved@d
        \let\reserved@c\reserved@a
      \else
        \let\reserved@c\reserved@b
      \fi
    \fi
    \reserved@c
  }%
  \begingroup
    \def\:{\global\let\@sptoken= }%
    \: %
    \def\:{\@xifnch}%
    \expandafter\gdef\: {\futurelet\@let@token\@ifnch}%
  \endgroup
}{}

\@ifundefined{ProvidesPackage}{%
  \def\ProvidesPackage#1{%
    \@ifnextchar[{\pc@ProvidesPackage{#1}}%
                 {\pc@ProvidesPackage{#1}[]}%
  }%
  \def\pc@ProvidesPackage#1[#2]{%
    \immediate\write-1{Package: #1 #2}%
  }%
}{}

\begingroup\expandafter\expandafter\expandafter\endgroup
\expandafter\ifx\csname RequirePackage\endcsname\relax
  \input infwarerr.sty\relax
\else
  \RequirePackage{infwarerr}%
\fi

\@ifundefined{@gobble}{%
  \long\def\@gobble#1{}%
}{}

\@ifundefined{@empty}{%
  \def\@empty{}%
}{}
%    \end{macrocode}
%
% \subsection{Package Identification and checks}
%
%    \begin{macrocode}
\ProvidesPackage{pdfcrypt}%
  [2016/05/16 v1.1 Allows the setting of PDF encryption (HO)]%
%    \end{macrocode}
%
%    \begin{macrocode}
\@ifundefined{pdfcryptsetup}{%
  \let\pdfcryptsetup\@gobble
}{%
  \@PackageWarningNoLine{pdfcrypt}{Package pdfcrypt is already loaded}%
  \pc@endinput
}
%    \end{macrocode}
%    Support for \cs{nopdfcrypt}.
%    \begin{macrocode}
\newif\ifpc@nopdfcrypt
\ifx Y\nopdfcrypt
  \@PackageWarningNoLine{pdfcrypt}{%
    Encryption disabled by \string \nopdfcrypt\space request%
  }%
  \global\pc@nopdfcrypttrue
\fi
%    \end{macrocode}
%
% \subsection{Driver detection}
%
%    \begin{macrocode}
\let\pc@driver\@empty
\begingroup
  % pdfTeX detection
  \@ifundefined{pdfoutput}{%
  }{%
    \ifcase\pdfoutput
    \else
      \@ifundefined{pdfcrypt}{%
        \@PackageError{pdfcrypt}{%
          PDF encryption is not supported with this pdfTeX%
        }{%
          Encryption support was added in 0.14h (2001/03/10)\MessageBreak
          and removed in 1.10a (2003/01/16).%
        }%
        \endgroup
        \pc@endinput
      }{%
        \gdef\pc@driver{pdftex}%
      }%
    \fi
  }%
  % VTeX detection
  \@ifundefined{OpMode}{%
  }{%
    \ifnum\OpMode=1 %
      \ifnum\@ifundefined{VTeXversion}0\VTeXversion<735 %
        \@PackageError{pdfcrypt}{%
          PDF encryption is not supported with this VTeX%
        }{%
          You need VTeX 7.35 or higher.%
        }%
        \endgroup
        \pc@endinput
      \else
        \gdef\pc@driver{vtex}%
      \fi
    \fi
  }%
\endgroup
%    \end{macrocode}
%
% \subsection{Load package \xpackage{keyval}}
%
%    \begin{macrocode}
\@ifundefined{@makeother}{%
  \def\@makeother#1{\catcode`#1=12\relax}%
}{}

\@ifundefined{g@addto@macro}{%
  \long\def\g@addto@macro#1#2{%
    \begingroup
      \toks@\expandafter{#1#2}%
      \xdef#1{\the\toks@}%
    \endgroup
  }%
}{}

\@ifundefined{@namedef}{%
  \def\@namedef#1{\expandafter\def\csname#1\endcsname}%
}{}

\@ifundefined{@nameuse}{%
  \def\@nameuse#1{\csname #1\endcsname}%
}{}

\def\pc@KeyvalRestore{%
  \let\pc@KeyvalRestore\@undefined
}

\let\pcOrg@NeedsTeXFormat\NeedsTeXFormat
\@ifundefined{NeedsTeXFormat}{%
  \def\NeedsTeXFormat#1{}%
  \g@addto@macro\pc@KeyvalRestore{%
    \let\NeedsTeXFormat\pcOrg@NeedsTeXFormat
  }%
}{}

\let\pcOrg@DeclareOption\DeclareOption
\@ifundefined{DeclareOption}{%
  \def\DeclareOption#1#2{#2}%
  \g@addto@macro\pc@KeyvalRestore{%
    \let\DeclareOption\pcOrg@DeclareOption
  }%
}{}

\let\pcOrg@ExecuteOptions\ExecuteOptions
\@ifundefined{ExecuteOptions}{%
  \def\ExecuteOptions#1{}%
  \g@addto@macro\pc@KeyvalRestore{%
    \let\ExecuteOptions\pcOrg@ExecuteOptions
  }%
}{}

\let\pcOrg@ProcessOptions\ProcessOptions
\@ifundefined{ProcessOptions}{%
  \def\ProcessOptions{}%
  \g@addto@macro\pc@KeyvalRestore{%
    \let\ProcessOptions\pcOrg@ProcessOptions
  }%
}{}%

\begingroup\expandafter\expandafter\expandafter\endgroup
\expandafter\ifx\csname RequirePackage\endcsname\relax
  \input keyval.sty\relax
\else
  \RequirePackage{keyval}%
\fi
\pc@KeyvalRestore
%    \end{macrocode}
%
% \subsection{Define options}
%
%    \begin{macrocode}
\@ifundefined{@dblarg}{%
  \long\def\@dblarg#1{\@ifnextchar[{#1}{\@xdblarg{#1}}}%
  \long\def\@xdblarg#1#2{#1[{#2}]{#2}}%
}{}

\newif\ifpc@set
\newif\ifpc@print
\newif\ifpc@copy
\newif\ifpc@edit
\newif\ifpc@annotate
\newif\ifpc@debug
\let\pc@owner\@empty
\let\pc@user\@empty

% default: allow all
\pc@printtrue
\pc@copytrue
\pc@edittrue
\pc@annotatetrue

\edef\pc@temp{\catcode`\noexpand\"=\the\catcode`\"\relax}
\@makeother\"
\def\pc@set{%
  \@PackageInfo{pdfcrypt}{%
    \ifpc@debug
      \ifx\pc@owner\@empty
        No owner password%
      \else
        Owner password: `\pc@owner'%
      \fi
      \MessageBreak
      \ifx\pc@user\@empty
        No user password%
      \else
        User password: `\pc@user'%
      \fi
      \MessageBreak
      Flags: %
      \ifpc@print   \else no\fi print, %
      \ifpc@copy    \else no\fi copy, %
      \ifpc@edit    \else no\fi edit, %
      \ifpc@annotate\else no\fi annotate%
      \MessageBreak
    \fi
    \ifpc@nopdfcrypt
      Encryption is disabled by `\string\nopdfcrypt'%
    \else
      Encryption is set for `\pc@driver'%
    \fi
  }%
  \ifpc@nopdfcrypt
  \else
    \@ifundefined{pc@set@\pc@driver}{%
      \ifx\pc@driver\@empty
        \@PackageError{pdfcrypt}{No driver for encryption %
          support found}\@ehc
      \else
        \@PackageError{pdfcrypt}{Cannot set encryption for %
          unknown driver `\pc@driver'}\@ehc
      \fi
    }{%
      \@nameuse{pc@set@\pc@driver}%
    }%
  \fi
}
\def\pc@set@pdftex{%
  \ifnum\pdftexversion<100 %
    \pc@set@pdftexold
  \else
    \pc@set@pdftexnew
  \fi
}
\def\pc@set@pdftexold{%
  \pdfcrypt{%
    owner "\pc@owner" %
    user "\pc@user" %
    \ifpc@print   \else no\fi print %
    \ifpc@copy    \else no\fi copy %
    \ifpc@edit    \else no\fi edit %
    \ifpc@annotate\else no\fi annotate%
  }%
}
\def\pc@set@pdftexnew{%
  \pdfcrypt
    owner{\pc@owner}%
    user{\pc@user}%
    \ifpc@print   \else no\fi print %
    \ifpc@copy    \else no\fi copy %
    \ifpc@edit    \else no\fi edit %
    \ifpc@annotate\else no\fi annotate%
  \relax
}
\def\pc@set@vtex{%
  \immediate\special{!security %
    O=\pc@MakeVTeXString\pc@owner,%
    U=\pc@MakeVTeXString\pc@user,%
    P\ifpc@print +\else -\fi,%
    C\ifpc@copy +\else -\fi,%
    M\ifpc@edit +\else -\fi,%
    A\ifpc@annotate +\else -\fi
  }%
}
\def\pc@MakeVTeXString#1{%
  "\expandafter\pc@@MakeVTeXString#1"\@nil"%
}
\def\pc@@MakeVTeXString#1"#2\@nil{%
  #1%
  \ifx\\#2\\%
  \else
    ""%
    \@ReturnAfterFi{%
      \pc@@MakeVTeXString#2\@nil
    }%
  \fi
}
\long\def\@ReturnAfterFi#1\fi{\fi#1}
\pc@temp

\begingroup
  \catcode`\ =12 \gdef\pc@spaceother{ }\catcode`\ =10\relax
  \catcode`\|=0 %
  \catcode`\\=12 %
  |gdef|pc@DefString#1#2{%
    |def#1{#2}%
    |edef#1{|expandafter|strip@prefix|meaning#1}%
    |edef#1{|expandafter|pc@SpaceToOther#1 |@nil}%
    |edef#1{|expandafter|pc@EscapeRemove#1|@empty\|@empty|@nil}%
  }%
  |gdef|pc@EscapeRemove#1\#2#3|@nil{%
    #1#2%
    |ifx|\#3|\%%
    |else
      |@ReturnAfterFi{%
        |pc@EscapeRemove#3|@nil
      }%
    |fi
  }%
|endgroup
\def\pc@SpaceToOther#1 #2\@nil{%
  #1%
  \ifx\\#2\\%
  \else
    \pc@spaceother
    \@ReturnAfterFi{%
      \pc@SpaceToOther#2\@nil
    }%
  \fi
}

\def\pc@boolkey{\@dblarg\pc@@boolkey}
\def\pc@@boolkey[#1]#2#3{%
  \lowercase{\def\pc@temp{#3}}%
  \ifx\pc@temp\@empty
    \let\pc@temp\pc@true
  \fi
  \ifx\pc@temp\pc@true
  \else
    \ifx\pc@temp\pc@false
    \else
      \let\pc@temp\relax
    \fi
  \fi
  \ifx\pc@temp\relax
    \@PackageWarning{pdfcrypt}{%
      Unexpected value \string`#3\string' of %
      option \string`#2\string'\MessageBreak
      instead of %
      \string`true\string' or \string`false\string'%
    }%
  \else
    \csname pc@#2\pc@temp\endcsname
  \fi
}
\def\pc@true{true}
\def\pc@false{false}

\define@key{pc}{set}[true]{%
  \pc@boolkey{set}{#1}%
}
\define@key{pc}{pdftex}[]{%
  \def\pc@driver{pdftex}%
}
\define@key{pc}{vtex}[]{%
  \def\pc@driver{vtex}%
}
\define@key{pc}{print}[true]{%
  \pc@boolkey{print}{#1}%
}
\define@key{pc}{copy}[true]{%
  \pc@boolkey{copy}{#1}%
}
\define@key{pc}{edit}[true]{%
  \pc@boolkey{edit}{#1}%
}
\define@key{pc}{annotate}[true]{%
  \pc@boolkey{annotate}{#1}%
}
\define@key{pc}{all}[]{%
  \pc@boolkey{print}{true}%
  \pc@boolkey{copy}{true}%
  \pc@boolkey{edit}{true}%
  \pc@boolkey{annotate}{true}%
}
\define@key{pc}{none}[]{%
  \pc@boolkey{print}{false}%
  \pc@boolkey{copy}{false}%
  \pc@boolkey{edit}{false}%
  \pc@boolkey{annotate}{false}%
}

\define@key{pc}{owner}{%
  \pc@DefString\pc@owner{#1}%
}
\define@key{pc}{user}{%
  \pc@DefString\pc@user{#1}%
}
\define@key{pc}{debug}[true]{%
  \pc@boolkey{debug}{#1}%
}

\def\pdfcryptsetup#1{%
  \setkeys{pc}{#1}%
  \ifpc@set
    \pc@set
    \global\let\pc@set\relax
    \gdef\pdfcryptsetup##1{%
      \@PackageWarning{pdfcrypt}{%
        Encryption options are already set\MessageBreak
        new values are ignored%
      }%
    }%
  \fi
}
\begingroup\expandafter\expandafter\expandafter\endgroup
\expandafter\ifx\csname @onlypreamble\endcsname\relax
\else
  \@onlypreamble\pdfcryptsetup
\fi
%    \end{macrocode}
%
% \subsection{support of configuration file}
%    \begin{macrocode}
\begingroup\expandafter\expandafter\expandafter\endgroup
\expandafter\ifx\csname InputIfFileExists\endcsname\relax
  \@PackageInfo{pdfcrypt}{%
    Configuration file pdfcrypt.cfg not supported.%
  }%
\else
  \let\pc@ExecuteOptions\ExecuteOptions
  \InputIfFileExists{pdfcrypt.cfg}{}{}%
  \let\ExecuteOptions\pc@ExecuteOptions
\fi
%    \end{macrocode}
%
% \subsection{Package options}
%
%    Plain format does not know package options.
%    \begin{macrocode}
\begingroup\expandafter\expandafter\expandafter\endgroup
\expandafter\ifx\csname @classoptionslist\endcsname\relax
  \expandafter\pc@endinput
\fi
%    \end{macrocode}
%    Process global and local options.
%    \begin{macrocode}
\def\pc@ProcessOptionsWithKV{%
  \let\pc@temp\@empty
  \@for\CurrentOption:=\@classoptionslist\do{%
    \@ifundefined{KV@pc@\CurrentOption}{}{%
      \edef\pc@temp{\pc@temp,\CurrentOption,}%
      \@expandtwoargs\@removeelement\CurrentOption
        \@unusedoptionlist\@unusedoptionlist
    }%
  }%
  \edef\pc@temp{%
    \noexpand\pdfcryptsetup{%
      \pc@temp\@ptionlist{\@currname.\@currext}%
    }%
  }%
  \pc@temp
}
\pc@ProcessOptionsWithKV
\AtEndOfPackage{\let\@unprocessedoptions\relax}
\AtBeginDocument{\pc@set}

\pc@endinput
%    \end{macrocode}
%
%    \begin{macrocode}
%</package>
%    \end{macrocode}
%
% \section{Installation}
%
% \subsection{Download}
%
% \paragraph{Package.} This package is available on
% CTAN\footnote{\CTANpkg{pdfcrypt}}:
% \begin{description}
% \item[\CTAN{macros/latex/contrib/oberdiek/pdfcrypt.dtx}] The source file.
% \item[\CTAN{macros/latex/contrib/oberdiek/pdfcrypt.pdf}] Documentation.
% \end{description}
%
%
% \paragraph{Bundle.} All the packages of the bundle `oberdiek'
% are also available in a TDS compliant ZIP archive. There
% the packages are already unpacked and the documentation files
% are generated. The files and directories obey the TDS standard.
% \begin{description}
% \item[\CTANinstall{install/macros/latex/contrib/oberdiek.tds.zip}]
% \end{description}
% \emph{TDS} refers to the standard ``A Directory Structure
% for \TeX\ Files'' (\CTANpkg{tds}). Directories
% with \xfile{texmf} in their name are usually organized this way.
%
% \subsection{Bundle installation}
%
% \paragraph{Unpacking.} Unpack the \xfile{oberdiek.tds.zip} in the
% TDS tree (also known as \xfile{texmf} tree) of your choice.
% Example (linux):
% \begin{quote}
%   |unzip oberdiek.tds.zip -d ~/texmf|
% \end{quote}
%
% \subsection{Package installation}
%
% \paragraph{Unpacking.} The \xfile{.dtx} file is a self-extracting
% \docstrip\ archive. The files are extracted by running the
% \xfile{.dtx} through \plainTeX:
% \begin{quote}
%   \verb|tex pdfcrypt.dtx|
% \end{quote}
%
% \paragraph{TDS.} Now the different files must be moved into
% the different directories in your installation TDS tree
% (also known as \xfile{texmf} tree):
% \begin{quote}
% \def\t{^^A
% \begin{tabular}{@{}>{\ttfamily}l@{ $\rightarrow$ }>{\ttfamily}l@{}}
%   pdfcrypt.sty & tex/generic/oberdiek/pdfcrypt.sty\\
%   pdfcrypt.pdf & doc/latex/oberdiek/pdfcrypt.pdf\\
%   pdfcrypt.dtx & source/latex/oberdiek/pdfcrypt.dtx\\
% \end{tabular}^^A
% }^^A
% \sbox0{\t}^^A
% \ifdim\wd0>\linewidth
%   \begingroup
%     \advance\linewidth by\leftmargin
%     \advance\linewidth by\rightmargin
%   \edef\x{\endgroup
%     \def\noexpand\lw{\the\linewidth}^^A
%   }\x
%   \def\lwbox{^^A
%     \leavevmode
%     \hbox to \linewidth{^^A
%       \kern-\leftmargin\relax
%       \hss
%       \usebox0
%       \hss
%       \kern-\rightmargin\relax
%     }^^A
%   }^^A
%   \ifdim\wd0>\lw
%     \sbox0{\small\t}^^A
%     \ifdim\wd0>\linewidth
%       \ifdim\wd0>\lw
%         \sbox0{\footnotesize\t}^^A
%         \ifdim\wd0>\linewidth
%           \ifdim\wd0>\lw
%             \sbox0{\scriptsize\t}^^A
%             \ifdim\wd0>\linewidth
%               \ifdim\wd0>\lw
%                 \sbox0{\tiny\t}^^A
%                 \ifdim\wd0>\linewidth
%                   \lwbox
%                 \else
%                   \usebox0
%                 \fi
%               \else
%                 \lwbox
%               \fi
%             \else
%               \usebox0
%             \fi
%           \else
%             \lwbox
%           \fi
%         \else
%           \usebox0
%         \fi
%       \else
%         \lwbox
%       \fi
%     \else
%       \usebox0
%     \fi
%   \else
%     \lwbox
%   \fi
% \else
%   \usebox0
% \fi
% \end{quote}
% If you have a \xfile{docstrip.cfg} that configures and enables \docstrip's
% TDS installing feature, then some files can already be in the right
% place, see the documentation of \docstrip.
%
% \subsection{Refresh file name databases}
%
% If your \TeX~distribution
% (\TeX\,Live, \mikTeX, \dots) relies on file name databases, you must refresh
% these. For example, \TeX\,Live\ users run \verb|texhash| or
% \verb|mktexlsr|.
%
% \subsection{Some details for the interested}
%
% \paragraph{Unpacking with \LaTeX.}
% The \xfile{.dtx} chooses its action depending on the format:
% \begin{description}
% \item[\plainTeX:] Run \docstrip\ and extract the files.
% \item[\LaTeX:] Generate the documentation.
% \end{description}
% If you insist on using \LaTeX\ for \docstrip\ (really,
% \docstrip\ does not need \LaTeX), then inform the autodetect routine
% about your intention:
% \begin{quote}
%   \verb|latex \let\install=y\input{pdfcrypt.dtx}|
% \end{quote}
% Do not forget to quote the argument according to the demands
% of your shell.
%
% \paragraph{Generating the documentation.}
% You can use both the \xfile{.dtx} or the \xfile{.drv} to generate
% the documentation. The process can be configured by the
% configuration file \xfile{ltxdoc.cfg}. For instance, put this
% line into this file, if you want to have A4 as paper format:
% \begin{quote}
%   \verb|\PassOptionsToClass{a4paper}{article}|
% \end{quote}
% An example follows how to generate the
% documentation with pdf\LaTeX:
% \begin{quote}
%\begin{verbatim}
%pdflatex pdfcrypt.dtx
%makeindex -s gind.ist pdfcrypt.idx
%pdflatex pdfcrypt.dtx
%makeindex -s gind.ist pdfcrypt.idx
%pdflatex pdfcrypt.dtx
%\end{verbatim}
% \end{quote}
%
% \begin{History}
%   \begin{Version}{2001/04/02 v0.1}
%   \item
%     First public version, published in the pdftex
%     mailing list for testing with pdfcrypt-20010331
%   \end{Version}
%   \begin{Version}{2001/07/19 v0.2}
%   \item
%     Default: all allowed.
%   \item
%     Support for \VTeX\ added.
%   \end{Version}
%   \begin{Version}{2001/07/19 v0.3}
%   \item
%     Bug fix: \VTeX\ letter for edit is M (modify).
%   \end{Version}
%   \begin{Version}{2001/07/19 v0.4}
%   \item
%     Bug fix: \cs{VTeXversion} is correct after
%     regenerating the format file.
%   \end{Version}
%   \begin{Version}{2001/08/05 v0.5}
%   \item
%     Syntax change in \pdfTeX\ 1.00a.
%   \end{Version}
%   \begin{Version}{2001/08/09 v0.6}
%   \item
%     Support of special characters:\\
%     input: |\{|, |\}|, |\\| for |{|, |}|, |\|\\
%     output: |"| in \VTeX
%   \item
%     Option debug added.
%   \end{Version}
%   \begin{Version}{2001/10/28 v0.7}
%   \item
%     Plain compatibility.
%   \item
%     \cs{nopdfcrypt} added.
%   \item
%     Typos corrected.
%   \end{Version}
%   \begin{Version}{2006/02/20 v0.8}
%   \item
%     Obsolete remarks for \pdfTeX.
%   \item
%     DTX framework.
%   \item
%     LPPL 1.3
%   \end{Version}
%   \begin{Version}{2007/04/11 v0.9}
%   \item
%     Line ends sanitized.
%   \end{Version}
%   \begin{Version}{2007/04/26 v1.0}
%   \item
%     Use of package \xpackage{infwarerr}.
%   \end{Version}
%   \begin{Version}{2016/05/16 v1.1}
%   \item
%     Documentation updates.
%   \end{Version}
% \end{History}
%
% \PrintIndex
%
% \Finale
\endinput
|
% \end{quote}
% Do not forget to quote the argument according to the demands
% of your shell.
%
% \paragraph{Generating the documentation.}
% You can use both the \xfile{.dtx} or the \xfile{.drv} to generate
% the documentation. The process can be configured by the
% configuration file \xfile{ltxdoc.cfg}. For instance, put this
% line into this file, if you want to have A4 as paper format:
% \begin{quote}
%   \verb|\PassOptionsToClass{a4paper}{article}|
% \end{quote}
% An example follows how to generate the
% documentation with pdf\LaTeX:
% \begin{quote}
%\begin{verbatim}
%pdflatex pdfcrypt.dtx
%makeindex -s gind.ist pdfcrypt.idx
%pdflatex pdfcrypt.dtx
%makeindex -s gind.ist pdfcrypt.idx
%pdflatex pdfcrypt.dtx
%\end{verbatim}
% \end{quote}
%
% \begin{History}
%   \begin{Version}{2001/04/02 v0.1}
%   \item
%     First public version, published in the pdftex
%     mailing list for testing with pdfcrypt-20010331
%   \end{Version}
%   \begin{Version}{2001/07/19 v0.2}
%   \item
%     Default: all allowed.
%   \item
%     Support for \VTeX\ added.
%   \end{Version}
%   \begin{Version}{2001/07/19 v0.3}
%   \item
%     Bug fix: \VTeX\ letter for edit is M (modify).
%   \end{Version}
%   \begin{Version}{2001/07/19 v0.4}
%   \item
%     Bug fix: \cs{VTeXversion} is correct after
%     regenerating the format file.
%   \end{Version}
%   \begin{Version}{2001/08/05 v0.5}
%   \item
%     Syntax change in \pdfTeX\ 1.00a.
%   \end{Version}
%   \begin{Version}{2001/08/09 v0.6}
%   \item
%     Support of special characters:\\
%     input: |\{|, |\}|, |\\| for |{|, |}|, |\|\\
%     output: |"| in \VTeX
%   \item
%     Option debug added.
%   \end{Version}
%   \begin{Version}{2001/10/28 v0.7}
%   \item
%     Plain compatibility.
%   \item
%     \cs{nopdfcrypt} added.
%   \item
%     Typos corrected.
%   \end{Version}
%   \begin{Version}{2006/02/20 v0.8}
%   \item
%     Obsolete remarks for \pdfTeX.
%   \item
%     DTX framework.
%   \item
%     LPPL 1.3
%   \end{Version}
%   \begin{Version}{2007/04/11 v0.9}
%   \item
%     Line ends sanitized.
%   \end{Version}
%   \begin{Version}{2007/04/26 v1.0}
%   \item
%     Use of package \xpackage{infwarerr}.
%   \end{Version}
%   \begin{Version}{2016/05/16 v1.1}
%   \item
%     Documentation updates.
%   \end{Version}
% \end{History}
%
% \PrintIndex
%
% \Finale
\endinput
|
% \end{quote}
% Do not forget to quote the argument according to the demands
% of your shell.
%
% \paragraph{Generating the documentation.}
% You can use both the \xfile{.dtx} or the \xfile{.drv} to generate
% the documentation. The process can be configured by the
% configuration file \xfile{ltxdoc.cfg}. For instance, put this
% line into this file, if you want to have A4 as paper format:
% \begin{quote}
%   \verb|\PassOptionsToClass{a4paper}{article}|
% \end{quote}
% An example follows how to generate the
% documentation with pdf\LaTeX:
% \begin{quote}
%\begin{verbatim}
%pdflatex pdfcrypt.dtx
%makeindex -s gind.ist pdfcrypt.idx
%pdflatex pdfcrypt.dtx
%makeindex -s gind.ist pdfcrypt.idx
%pdflatex pdfcrypt.dtx
%\end{verbatim}
% \end{quote}
%
% \begin{History}
%   \begin{Version}{2001/04/02 v0.1}
%   \item
%     First public version, published in the pdftex
%     mailing list for testing with pdfcrypt-20010331
%   \end{Version}
%   \begin{Version}{2001/07/19 v0.2}
%   \item
%     Default: all allowed.
%   \item
%     Support for \VTeX\ added.
%   \end{Version}
%   \begin{Version}{2001/07/19 v0.3}
%   \item
%     Bug fix: \VTeX\ letter for edit is M (modify).
%   \end{Version}
%   \begin{Version}{2001/07/19 v0.4}
%   \item
%     Bug fix: \cs{VTeXversion} is correct after
%     regenerating the format file.
%   \end{Version}
%   \begin{Version}{2001/08/05 v0.5}
%   \item
%     Syntax change in \pdfTeX\ 1.00a.
%   \end{Version}
%   \begin{Version}{2001/08/09 v0.6}
%   \item
%     Support of special characters:\\
%     input: |\{|, |\}|, |\\| for |{|, |}|, |\|\\
%     output: |"| in \VTeX
%   \item
%     Option debug added.
%   \end{Version}
%   \begin{Version}{2001/10/28 v0.7}
%   \item
%     Plain compatibility.
%   \item
%     \cs{nopdfcrypt} added.
%   \item
%     Typos corrected.
%   \end{Version}
%   \begin{Version}{2006/02/20 v0.8}
%   \item
%     Obsolete remarks for \pdfTeX.
%   \item
%     DTX framework.
%   \item
%     LPPL 1.3
%   \end{Version}
%   \begin{Version}{2007/04/11 v0.9}
%   \item
%     Line ends sanitized.
%   \end{Version}
%   \begin{Version}{2007/04/26 v1.0}
%   \item
%     Use of package \xpackage{infwarerr}.
%   \end{Version}
%   \begin{Version}{2016/05/16 v1.1}
%   \item
%     Documentation updates.
%   \end{Version}
% \end{History}
%
% \PrintIndex
%
% \Finale
\endinput
|
% \end{quote}
% Do not forget to quote the argument according to the demands
% of your shell.
%
% \paragraph{Generating the documentation.}
% You can use both the \xfile{.dtx} or the \xfile{.drv} to generate
% the documentation. The process can be configured by the
% configuration file \xfile{ltxdoc.cfg}. For instance, put this
% line into this file, if you want to have A4 as paper format:
% \begin{quote}
%   \verb|\PassOptionsToClass{a4paper}{article}|
% \end{quote}
% An example follows how to generate the
% documentation with pdf\LaTeX:
% \begin{quote}
%\begin{verbatim}
%pdflatex pdfcrypt.dtx
%makeindex -s gind.ist pdfcrypt.idx
%pdflatex pdfcrypt.dtx
%makeindex -s gind.ist pdfcrypt.idx
%pdflatex pdfcrypt.dtx
%\end{verbatim}
% \end{quote}
%
% \begin{History}
%   \begin{Version}{2001/04/02 v0.1}
%   \item
%     First public version, published in the pdftex
%     mailing list for testing with pdfcrypt-20010331
%   \end{Version}
%   \begin{Version}{2001/07/19 v0.2}
%   \item
%     Default: all allowed.
%   \item
%     Support for \VTeX\ added.
%   \end{Version}
%   \begin{Version}{2001/07/19 v0.3}
%   \item
%     Bug fix: \VTeX\ letter for edit is M (modify).
%   \end{Version}
%   \begin{Version}{2001/07/19 v0.4}
%   \item
%     Bug fix: \cs{VTeXversion} is correct after
%     regenerating the format file.
%   \end{Version}
%   \begin{Version}{2001/08/05 v0.5}
%   \item
%     Syntax change in \pdfTeX\ 1.00a.
%   \end{Version}
%   \begin{Version}{2001/08/09 v0.6}
%   \item
%     Support of special characters:\\
%     input: |\{|, |\}|, |\\| for |{|, |}|, |\|\\
%     output: |"| in \VTeX
%   \item
%     Option debug added.
%   \end{Version}
%   \begin{Version}{2001/10/28 v0.7}
%   \item
%     Plain compatibility.
%   \item
%     \cs{nopdfcrypt} added.
%   \item
%     Typos corrected.
%   \end{Version}
%   \begin{Version}{2006/02/20 v0.8}
%   \item
%     Obsolete remarks for \pdfTeX.
%   \item
%     DTX framework.
%   \item
%     LPPL 1.3
%   \end{Version}
%   \begin{Version}{2007/04/11 v0.9}
%   \item
%     Line ends sanitized.
%   \end{Version}
%   \begin{Version}{2007/04/26 v1.0}
%   \item
%     Use of package \xpackage{infwarerr}.
%   \end{Version}
%   \begin{Version}{2016/05/16 v1.1}
%   \item
%     Documentation updates.
%   \end{Version}
% \end{History}
%
% \PrintIndex
%
% \Finale
\endinput
