% \iffalse meta-comment
%
% File: pdflscape.dtx
% Version: 2016/05/14 v0.11
% Info: Display of landscape pages in PDF
%
% Copyright (C) 2001, 2004, 2006-2008 by
%    Heiko Oberdiek <heiko.oberdiek at googlemail.com>
%    2016
%    https://github.com/ho-tex/oberdiek/issues
%
% This work may be distributed and/or modified under the
% conditions of the LaTeX Project Public License, either
% version 1.3c of this license or (at your option) any later
% version. This version of this license is in
%    https://www.latex-project.org/lppl/lppl-1-3c.txt
% and the latest version of this license is in
%    https://www.latex-project.org/lppl.txt
% and version 1.3 or later is part of all distributions of
% LaTeX version 2005/12/01 or later.
%
% This work has the LPPL maintenance status "maintained".
%
% The Current Maintainers of this work are
% Heiko Oberdiek and the Oberdiek Package Support Group
% https://github.com/ho-tex/oberdiek/issues
%
% This work consists of the main source file pdflscape.dtx
% and the derived files
%    pdflscape.sty, pdflscape.pdf, pdflscape.ins, pdflscape.drv,
%    pdflscape-test1.tex, pdflscape-test2.tex,
%    pdflscape-test3.tex, pdflscape-test4.tex,
%    pdflscape-test5.tex, pdflscape-test6.tex,
%    pdflscape-test6.txt, ExtractRotate.java.
%
% Distribution:
%    CTAN:macros/latex/contrib/oberdiek/pdflscape.dtx
%    CTAN:macros/latex/contrib/oberdiek/pdflscape.pdf
%
% Unpacking:
%    (a) If pdflscape.ins is present:
%           tex pdflscape.ins
%    (b) Without pdflscape.ins:
%           tex pdflscape.dtx
%    (c) If you insist on using LaTeX
%           latex \let\install=y% \iffalse meta-comment
%
% File: pdflscape.dtx
% Version: 2016/05/14 v0.11
% Info: Display of landscape pages in PDF
%
% Copyright (C) 2001, 2004, 2006-2008 by
%    Heiko Oberdiek <heiko.oberdiek at googlemail.com>
%    2016
%    https://github.com/ho-tex/oberdiek/issues
%
% This work may be distributed and/or modified under the
% conditions of the LaTeX Project Public License, either
% version 1.3c of this license or (at your option) any later
% version. This version of this license is in
%    http://www.latex-project.org/lppl/lppl-1-3c.txt
% and the latest version of this license is in
%    http://www.latex-project.org/lppl.txt
% and version 1.3 or later is part of all distributions of
% LaTeX version 2005/12/01 or later.
%
% This work has the LPPL maintenance status "maintained".
%
% This Current Maintainer of this work is Heiko Oberdiek.
%
% This work consists of the main source file pdflscape.dtx
% and the derived files
%    pdflscape.sty, pdflscape.pdf, pdflscape.ins, pdflscape.drv,
%    pdflscape-test1.tex, pdflscape-test2.tex,
%    pdflscape-test3.tex, pdflscape-test4.tex,
%    pdflscape-test5.tex, pdflscape-test6.tex,
%    pdflscape-test6.txt, ExtractRotate.java.
%
% Distribution:
%    CTAN:macros/latex/contrib/oberdiek/pdflscape.dtx
%    CTAN:macros/latex/contrib/oberdiek/pdflscape.pdf
%
% Unpacking:
%    (a) If pdflscape.ins is present:
%           tex pdflscape.ins
%    (b) Without pdflscape.ins:
%           tex pdflscape.dtx
%    (c) If you insist on using LaTeX
%           latex \let\install=y% \iffalse meta-comment
%
% File: pdflscape.dtx
% Version: 2016/05/14 v0.11
% Info: Display of landscape pages in PDF
%
% Copyright (C) 2001, 2004, 2006-2008 by
%    Heiko Oberdiek <heiko.oberdiek at googlemail.com>
%    2016
%    https://github.com/ho-tex/oberdiek/issues
%
% This work may be distributed and/or modified under the
% conditions of the LaTeX Project Public License, either
% version 1.3c of this license or (at your option) any later
% version. This version of this license is in
%    http://www.latex-project.org/lppl/lppl-1-3c.txt
% and the latest version of this license is in
%    http://www.latex-project.org/lppl.txt
% and version 1.3 or later is part of all distributions of
% LaTeX version 2005/12/01 or later.
%
% This work has the LPPL maintenance status "maintained".
%
% This Current Maintainer of this work is Heiko Oberdiek.
%
% This work consists of the main source file pdflscape.dtx
% and the derived files
%    pdflscape.sty, pdflscape.pdf, pdflscape.ins, pdflscape.drv,
%    pdflscape-test1.tex, pdflscape-test2.tex,
%    pdflscape-test3.tex, pdflscape-test4.tex,
%    pdflscape-test5.tex, pdflscape-test6.tex,
%    pdflscape-test6.txt, ExtractRotate.java.
%
% Distribution:
%    CTAN:macros/latex/contrib/oberdiek/pdflscape.dtx
%    CTAN:macros/latex/contrib/oberdiek/pdflscape.pdf
%
% Unpacking:
%    (a) If pdflscape.ins is present:
%           tex pdflscape.ins
%    (b) Without pdflscape.ins:
%           tex pdflscape.dtx
%    (c) If you insist on using LaTeX
%           latex \let\install=y% \iffalse meta-comment
%
% File: pdflscape.dtx
% Version: 2016/05/14 v0.11
% Info: Display of landscape pages in PDF
%
% Copyright (C) 2001, 2004, 2006-2008 by
%    Heiko Oberdiek <heiko.oberdiek at googlemail.com>
%    2016
%    https://github.com/ho-tex/oberdiek/issues
%
% This work may be distributed and/or modified under the
% conditions of the LaTeX Project Public License, either
% version 1.3c of this license or (at your option) any later
% version. This version of this license is in
%    http://www.latex-project.org/lppl/lppl-1-3c.txt
% and the latest version of this license is in
%    http://www.latex-project.org/lppl.txt
% and version 1.3 or later is part of all distributions of
% LaTeX version 2005/12/01 or later.
%
% This work has the LPPL maintenance status "maintained".
%
% This Current Maintainer of this work is Heiko Oberdiek.
%
% This work consists of the main source file pdflscape.dtx
% and the derived files
%    pdflscape.sty, pdflscape.pdf, pdflscape.ins, pdflscape.drv,
%    pdflscape-test1.tex, pdflscape-test2.tex,
%    pdflscape-test3.tex, pdflscape-test4.tex,
%    pdflscape-test5.tex, pdflscape-test6.tex,
%    pdflscape-test6.txt, ExtractRotate.java.
%
% Distribution:
%    CTAN:macros/latex/contrib/oberdiek/pdflscape.dtx
%    CTAN:macros/latex/contrib/oberdiek/pdflscape.pdf
%
% Unpacking:
%    (a) If pdflscape.ins is present:
%           tex pdflscape.ins
%    (b) Without pdflscape.ins:
%           tex pdflscape.dtx
%    (c) If you insist on using LaTeX
%           latex \let\install=y\input{pdflscape.dtx}
%        (quote the arguments according to the demands of your shell)
%
% Documentation:
%    (a) If pdflscape.drv is present:
%           latex pdflscape.drv
%    (b) Without pdflscape.drv:
%           latex pdflscape.dtx; ...
%    The class ltxdoc loads the configuration file ltxdoc.cfg
%    if available. Here you can specify further options, e.g.
%    use A4 as paper format:
%       \PassOptionsToClass{a4paper}{article}
%
%    Programm calls to get the documentation (example):
%       pdflatex pdflscape.dtx
%       makeindex -s gind.ist pdflscape.idx
%       pdflatex pdflscape.dtx
%       makeindex -s gind.ist pdflscape.idx
%       pdflatex pdflscape.dtx
%
% Installation:
%    TDS:tex/latex/oberdiek/pdflscape.sty
%    TDS:doc/latex/oberdiek/pdflscape.pdf
%    TDS:doc/latex/oberdiek/test/pdflscape-test1.tex
%    TDS:doc/latex/oberdiek/test/pdflscape-test2.tex
%    TDS:doc/latex/oberdiek/test/pdflscape-test3.tex
%    TDS:doc/latex/oberdiek/test/pdflscape-test4.tex
%    TDS:doc/latex/oberdiek/test/pdflscape-test5.tex
%    TDS:doc/latex/oberdiek/test/pdflscape-test6.tex
%    TDS:doc/latex/oberdiek/test/pdflscape-test6.txt
%    TDS:doc/latex/oberdiek/test/ExtractRotate.java
%    TDS:source/latex/oberdiek/pdflscape.dtx
%
%<*ignore>
\begingroup
  \catcode123=1 %
  \catcode125=2 %
  \def\x{LaTeX2e}%
\expandafter\endgroup
\ifcase 0\ifx\install y1\fi\expandafter
         \ifx\csname processbatchFile\endcsname\relax\else1\fi
         \ifx\fmtname\x\else 1\fi\relax
\else\csname fi\endcsname
%</ignore>
%<*install>
\input docstrip.tex
\Msg{************************************************************************}
\Msg{* Installation}
\Msg{* Package: pdflscape 2016/05/14 v0.11 Display of landscape pages in PDF (HO)}
\Msg{************************************************************************}

\keepsilent
\askforoverwritefalse

\let\MetaPrefix\relax
\preamble

This is a generated file.

Project: pdflscape
Version: 2016/05/14 v0.11

Copyright (C) 2001, 2004, 2006-2008 by
   Heiko Oberdiek <heiko.oberdiek at googlemail.com>

This work may be distributed and/or modified under the
conditions of the LaTeX Project Public License, either
version 1.3c of this license or (at your option) any later
version. This version of this license is in
   http://www.latex-project.org/lppl/lppl-1-3c.txt
and the latest version of this license is in
   http://www.latex-project.org/lppl.txt
and version 1.3 or later is part of all distributions of
LaTeX version 2005/12/01 or later.

This work has the LPPL maintenance status "maintained".

This Current Maintainer of this work is Heiko Oberdiek.

This work consists of the main source file pdflscape.dtx
and the derived files
   pdflscape.sty, pdflscape.pdf, pdflscape.ins, pdflscape.drv,
   pdflscape-test1.tex, pdflscape-test2.tex,
   pdflscape-test3.tex, pdflscape-test4.tex,
   pdflscape-test5.tex, pdflscape-test6.tex,
   pdflscape-test6.txt, ExtractRotate.java.

\endpreamble
\let\MetaPrefix\DoubleperCent

\generate{%
  \file{pdflscape.ins}{\from{pdflscape.dtx}{install}}%
  \file{pdflscape.drv}{\from{pdflscape.dtx}{driver}}%
  \usedir{tex/latex/oberdiek}%
  \file{pdflscape.sty}{\from{pdflscape.dtx}{package}}%
%  \usedir{doc/latex/oberdiek/test}%
%  \file{pdflscape-test1.tex}{\from{pdflscape.dtx}{test0,test1}}%
%  \file{pdflscape-test2.tex}{\from{pdflscape.dtx}{test0,test2}}%
%  \file{pdflscape-test3.tex}{\from{pdflscape.dtx}{test0,test3}}%
%  \file{pdflscape-test4.tex}{\from{pdflscape.dtx}{test0,test4}}%
%  \file{pdflscape-test5.tex}{\from{pdflscape.dtx}{test5}}%
%  \file{pdflscape-test6.tex}{\from{pdflscape.dtx}{test6}}%
  \nopreamble
  \nopostamble
%  \file{pdflscape-test6.txt}{\from{pdflscape.dtx}{result6}}%
  \file{ExtractRotate.java}{\from{pdflscape.dtx}{ExtractRotate}}%
%  \usedir{source/latex/oberdiek/catalogue}%
%  \file{pdflscape.xml}{\from{pdflscape.dtx}{catalogue}}%
}

\catcode32=13\relax% active space
\let =\space%
\Msg{************************************************************************}
\Msg{*}
\Msg{* To finish the installation you have to move the following}
\Msg{* file into a directory searched by TeX:}
\Msg{*}
\Msg{*     pdflscape.sty}
\Msg{*}
\Msg{* To produce the documentation run the file `pdflscape.drv'}
\Msg{* through LaTeX.}
\Msg{*}
\Msg{* Happy TeXing!}
\Msg{*}
\Msg{************************************************************************}

\endbatchfile
%</install>
%<*ignore>
\fi
%</ignore>
%<*driver>
\NeedsTeXFormat{LaTeX2e}
\ProvidesFile{pdflscape.drv}%
  [2016/05/14 v0.11 Display of landscape pages in PDF (HO)]%
\documentclass{ltxdoc}
\usepackage{holtxdoc}[2011/11/22]
\usepackage{graphics}
\begin{document}
  \DocInput{pdflscape.dtx}%
\end{document}
%</driver>
% \fi
%
%
% \CharacterTable
%  {Upper-case    \A\B\C\D\E\F\G\H\I\J\K\L\M\N\O\P\Q\R\S\T\U\V\W\X\Y\Z
%   Lower-case    \a\b\c\d\e\f\g\h\i\j\k\l\m\n\o\p\q\r\s\t\u\v\w\x\y\z
%   Digits        \0\1\2\3\4\5\6\7\8\9
%   Exclamation   \!     Double quote  \"     Hash (number) \#
%   Dollar        \$     Percent       \%     Ampersand     \&
%   Acute accent  \'     Left paren    \(     Right paren   \)
%   Asterisk      \*     Plus          \+     Comma         \,
%   Minus         \-     Point         \.     Solidus       \/
%   Colon         \:     Semicolon     \;     Less than     \<
%   Equals        \=     Greater than  \>     Question mark \?
%   Commercial at \@     Left bracket  \[     Backslash     \\
%   Right bracket \]     Circumflex    \^     Underscore    \_
%   Grave accent  \`     Left brace    \{     Vertical bar  \|
%   Right brace   \}     Tilde         \~}
%
% \GetFileInfo{pdflscape.drv}
%
% \title{The \xpackage{pdflscape} package}
% \date{2016/05/14 v0.11}
% \author{Heiko Oberdiek\thanks
% {Please report any issues at \url{https://github.com/ho-tex/oberdiek/issues}}\\
% \xemail{heiko.oberdiek at googlemail.com}}
%
% \maketitle
%
% \begin{abstract}
% Package \xpackage{pdflscape} adds PDF support to the
% environment \texttt{landscape} of package \xpackage{lscape}
% by setting the PDF page attribute \texttt{/Rotate}.
% \end{abstract}
%
% \tableofcontents
%
% \section{Documentation}
%
% \subsection{Supported drivers}
%
% \begin{itemize}
% \item \xoption{pdftex},\xoption{luatex}
% \item \xoption{dvips}, \xoption{dvipsone},
%       \xoption{pctex32}, \xoption{pctexps}
%       (and other drivers that provide a non-empty \cs{Gin@PS@raw})
% \item \xoption{dvipdfm}
% \end{itemize}
%
% \subsection{Caveat}
%
% Depending on the configuration Ghostscript
% adds a guessed rotation entry by its own.
% This can lead to two |/Rotate| entries per page.
% To prevent this behaviour set the parameter
% |AutoRotatePages| to |/None|, eg:
% \begin{quote}
%   |ps2pdf -dAutoRotatePages=/None|
% \end{quote}
%
% \subsection{Requirements}
%
% \begin{itemize}
% \item The package \xpackage{lscape}.
% \item The package \xpackage{atbegshi} for all drivers
%        except \xoption{pdftex},\xoption{luatex}.
% \end{itemize}
%
% \subsection{Usage}
%
% Load this package instead of or after package \xpackage{lscape}:
% \begin{quote}
%   |\usepackage{pdflscape}|\\
%   |\begin{landscape}|\dots|\end{landscape}|
% \end{quote}
%
% \StopEventually{
% }
%
% \section{Implementation}
%
% \subsection{Package identification}
%
%    \begin{macrocode}
%<*package>
%    \end{macrocode}
%    Package identification.
%    \begin{macrocode}
\NeedsTeXFormat{LaTeX2e}
\ProvidesPackage{pdflscape}%
  [2016/05/14 v0.11 Display of landscape pages in PDF (HO)]
%    \end{macrocode}
%
% \subsection{Driver options}
%
%    \begin{macrocode}
\let\PLS@option\@empty
\let\PLS@driver\@empty
\def\PLS@temp#1#2{%
  \DeclareOption{#1}{%
    \let\PLS@option\CurrentOption
    \def\PLS@driver{#2}%
  }%
}
\PLS@temp{pdftex}{pdftex}
\PLS@temp{luatex}{pdftex}
\PLS@temp{dvips}{dvips}
\PLS@temp{dvipsone}{dvips}
\PLS@temp{pctexps}{dvips}
\PLS@temp{dvipdfm}{dvipdfm}
\PLS@temp{dvipdfmx}{dvipdfm}
\PLS@temp{xetex}{dvipdfm}
\let\PLS@temp\@empty
\DeclareOption*{%
  \ifx\PLS@temp\@empty
    \let\PLS@temp\CurrentOption
  \else
    \edef\PLS@temp{\PLS@temp,\CurrentOption}%
  \fi
}
\ProcessOptions*
\ifx\PLS@option\@empty
\else
  \expandafter\PassOptionsToPackage\expandafter{\PLS@option}{lscape}%
\fi
\ifx\PLS@temp\@empty
\else
  \expandafter\PassOptionsToPackage\expandafter{\PLS@temp}{lscape}%
\fi
\RequirePackage{lscape}
%    \end{macrocode}
%
% \subsection{Autodetection of driver}
%
% \subsubsection{Driver \xoption{pdftex}}
%
%    \begin{macrocode}
\RequirePackage{ifpdf}[2006/02/20]
\def\PLS@temp{pdftex}
\ifpdf
  \ifx\PLS@driver\@empty
    \let\PLS@driver\PLS@temp
    \PackageInfo{pdflscape}{Auto-detected driver: \PLS@temp}%
  \fi
  \ifx\PLS@driver\PLS@temp
  \else
    \PackageError{pdflscape}{%
      The driver option `\PLS@option' is not compatible\MessageBreak
      with pdfTeX in PDF mode, using driver `pdftex' instead%
    }\@ehc
    \let\PLS@driver\PLS@temp
  \fi
\fi
%    \end{macrocode}
%
% \subsubsection{Driver \xoption{xetex}}
%
%    \begin{macrocode}
\RequirePackage{ifxetex}%
\def\PLS@temp{dvipdfm}
\ifxetex
  \ifx\PLS@driver\@empty
    \let\PLS@driver\PLS@temp
    \PackageInfo{pdflscape}{%
      Auto-detected driver: \PLS@temp\space(xetex)%
    }%
  \fi
  \ifx\PLS@driver\PLS@temp
  \else
    \PackageError{pdflscape}{%
      The driver option `\PLS@option' is not compatible\MessageBreak
      with XeTeX, using driver for XeTeX (dvipdfm) instead%
    }\@ehc
    \let\PLS@driver\PLS@temp
  \fi
\fi
%    \end{macrocode}
%
% \subsubsection{Detect driver based on \cs{Gin@driver}}
%
%    \begin{macrocode}
\def\PLS@temp#1#2{%
  \ifx\PLS@driver\@empty
    \begingroup
      \def\PLS@temp{#1.def}%
    \expandafter\endgroup
    \ifx\PLS@temp\Gin@driver
      \def\PLS@driver{#2}%
      \PackageInfo{pdflscape}{%
        Auto-detected driver: \PLS@driver\space(#1)%
      }%
    \fi
  \fi
}
\PLS@temp{dvipdfm}{dvipdfm}
\PLS@temp{dvipdfmx}{dvipdfm}
\PLS@temp{dvips}{dvips}
\PLS@temp{dvipsone}{dvips}
\PLS@temp{pctexps}{dvips}
%    \end{macrocode}
%
% \subsubsection{Driver \xoption{dvips}}
%
%    \begin{macrocode}
\ifx\PLS@driver\@empty
  \def\PLS@temp{dvips}%
  \ifnum 1=0\ifx\Gin@PS@raw\@undefined
            \else
              \ifx\Gin@PS@raw\relax
              \else
                \ifx\Gin@PS@raw\@empty
                \else
                  1%
                \fi
              \fi
            \fi
            \space
    \let\PLS@driver\PLS@temp
    \PackageInfo{pdflscape}{%
      Auto-detected driver: \PLS@temp\space(PostScript)%
    }%
  \fi
\fi
%    \end{macrocode}
%
%    \begin{macrocode}
\ifx\PLS@driver\@empty
  \PackageError{pdflscape}{%
    Neither pdfTeX, PostScript, nor dvipdfm driver found%
  }\@ehc
\fi
%    \end{macrocode}
%
% \subsection{Driver implementation}
%
% \begin{itemize}
% \item |\PLS@AddRotate #1|\\
%   it expects the correct rotation number in |#1|
%   and implements the adding of the /Rotation entry
%   in the |/Page| object of the current page.
% \item |\PLS@RemoveRotate|\\
%   it removes a previous |/Rotate| entry, if necessary.
% \end{itemize}
%
% \subsubsection{\pdfTeX}
%
% Not too nice is the global setting of \cs{pdfpageattr}.
% Perhaps this can be changed in future versions.
%
% luatex compat.
%    \begin{macrocode}
\ifx\pdfpageattr\@undefined
  \ifx\pdfvariable\@undefined\else 
    \protected\edef\pdfpageattr{\pdfvariable pageattr}
  \fi
\fi
%    \end{macrocode}
%    \begin{macrocode}
\def\PLS@temp{pdftex}
\ifx\PLS@temp\PLS@driver
  \def\PLS@AddRotate#1{%
    \ifnum#1=0 %
%    \end{macrocode}
%    Already the default.
%    \begin{macrocode}
    \else
      \global\pdfpageattr\expandafter{%
        \the\pdfpageattr
        /Rotate #1%
      }%
    \fi
  }%
%    \end{macrocode}
%    Removes a |/Rotate| entry.
%    \begin{macrocode}
  \def\PLS@RemoveRotate{%
    \begingroup
      \global\pdfpageattr\expandafter{\expandafter}%
      \expandafter\PLS@@RemoveRotate
        \the\pdfpageattr /Rotate\@nil
    \endgroup
  }%
  \def\PLS@@RemoveRotate#1/Rotate#2\@nil{%
%    \end{macrocode}
%    Append |/Rotate| free stuff to \cs{pdfpageattr}.
%    \begin{macrocode}
    \global\pdfpageattr\expandafter{\the\pdfpageattr#1}%
    \ifx\\#2\\%
%    \end{macrocode}
%    Ready, because the detected |/Rotate| is part of
%    the end marker: |/Rotate\@nil|
%    \begin{macrocode}
    \else
%    \end{macrocode}
%    First read in the argument of |/Rotate|,
%    then continue parsing.
%    \begin{macrocode}
      \afterassignment\PLS@@RemoveRotate
      \count0=#2\@nil
    \fi
  }%
\fi
%    \end{macrocode}
%
% \subsubsection{PostScript driver}
%
%    \begin{macrocode}
\def\PLS@temp{dvips}
\ifx\PLS@temp\PLS@driver
  \RequirePackage{atbegshi}%
  \AtBeginShipout{\PLS@AtBeginShipout}%
  \let\PLS@AtBeginShipout\@empty
  \def\PLS@AddRotate#1{%
    \ifnum#1=0 %
    \else
      \def\PLS@AtBeginShipout{%
        \global\setbox\AtBeginShipoutBox\vbox{%
          \Gin@PS@raw{%
            [{ThisPage}\string<\string</Rotate #1\string>\string>%
            /PUT pdfmark%
          }%
          \box\AtBeginShipoutBox
        }%
      }%
    \fi
  }%
  \def\PLS@RemoveRotate{\let\PLS@AtBeginShipout\@empty}%
\fi
%    \end{macrocode}
%
% \subsubsection{Driver dvipdfm}
%
%    \begin{macrocode}
\def\PLS@temp{dvipdfm}
\ifx\PLS@temp\PLS@driver
  \RequirePackage{atbegshi}%
  \AtBeginShipout{\PLS@AtBeginShipout}%
  \let\PLS@AtBeginShipout\@empty
  \def\PLS@AddRotate#1{%
    \ifnum#1=0 %
    \else
      \def\PLS@AtBeginShipout{%
        \global\setbox\AtBeginShipoutBox\vbox{%
          \special{pdf: put @thispage <</Rotate #1>>}%
          \box\AtBeginShipoutBox
        }%
      }%
    \fi
  }%
  \def\PLS@RemoveRotate{\let\PLS@AtBeginShipout\@empty}%
\fi
%    \end{macrocode}
%
% \subsection{Driver independent stuff}
%
%    The landscape environment is extended by
%    adding the correct |/Rotate| entries.
%    \begin{macrocode}
\g@addto@macro{\landscape}{\PLS@Rotate{90}}
\g@addto@macro{\endlandscape}{\PLS@Rotate{0}}
%    \end{macrocode}
%
%    \begin{macro}{\PLS@Rotate}
%    Main macro, that sets the |/Rotate| entry.\\
%    Argument: any TeX number or nothing, that means zero.\\
%    Driver independent.
%    \begin{macrocode}
\def\PLS@Rotate#1{%
  \begingroup
%    \end{macrocode}
%    1. Check and validate the argument.
%    \begin{macrocode}
    \PLS@CheckAngle{#1}%
  \expandafter\endgroup
%    \end{macrocode}
%    2. Remove previous |\Rotate| entry.
%    \begin{macrocode}
  \expandafter\PLS@RemoveRotate
%    \end{macrocode}
%    3. Add |/Rotate| entry.
%    \begin{macrocode}
  \expandafter\PLS@AddRotate\expandafter{\the\count@}%
}
%    \end{macrocode}
%    \end{macro}
%
%    \begin{macro}{\PLS@CheckAngle}
%    Validates the rotation angle.\\
%    The result is stored in the count register |\count@|.\\
%    Driver independent.
%    \begin{macrocode}
\def\PLS@CheckAngle#1{%
%    \end{macrocode}
%    Nothing means zero:
%    \begin{macrocode}
  \ifx\\#1\\%
    \count@=0 %
  \else
    \count@=#1\relax
  \fi
%    \end{macrocode}
%    Normalize to interval -360 $<$ |\count@| $<$ 360.
%    \begin{macrocode}
  \@whilenum\count@>359\do{\advance\count@ -360 }%
  \@whilenum\count@<-359\do{\advance\count@ 360 }%
%    \end{macrocode}
%    Check values: 0, +/-90, +/-180, +/-270
%    \begin{macrocode}
  \ifnum 1=0\ifnum\count@=0 \else
            \ifnum\count@=90 \else
            \ifnum\count@=-90 \else
            \ifnum\count@=180 \else
            \ifnum\count@=-180 \else
            \ifnum\count@=270 \else
            \ifnum\count@=-270 \else
           1\fi\fi\fi\fi\fi\fi\fi\relax
    \PackageError{pdflscape}{%
      Invalid value \the\count@\space for /Rotate%
    }\@ehc
    \count@=0 %
  \fi
}
%    \end{macrocode}
%    \end{macro}
%
%    \begin{macrocode}
%</package>
%    \end{macrocode}
%
% \section{Test}
%
% \subsection{Driver detection tests}
%    \begin{macrocode}
%<*test0>
\NeedsTeXFormat{LaTeX2e}
\documentclass{minimal}
\usepackage{qstest}
\IncludeTests{*}
\LogTests{log}{*}{*}
%</test0>
%    \end{macrocode}
%
%    \begin{macrocode}
%<*test1>
\usepackage{ifpdf}
\ifpdf
  \def\ExpectDriver{pdftex}%
\fi
\usepackage{ifxetex}%
\ifxetex
  \def\ExpectDriver{dvipdfm}%
\fi
\usepackage{pdflscape}[2016/05/14]
%</test1>
%    \end{macrocode}
%
%    \begin{macrocode}
%<*test2>
\def\ExpectDriver{pdftex}
\usepackage[pdftex]{pdflscape}[2016/05/14]
%</test2>
%    \end{macrocode}
%
%    \begin{macrocode}
%<*test3>
\def\ExpectDriver{dvipdfm}
\usepackage[dvipdfmx]{pdflscape}[2016/05/14]
%</test3>
%    \end{macrocode}
%
%    \begin{macrocode}
%<*test4>
\def\ExpectDriver{dvips}
\usepackage[dvipsone]{graphics}
\usepackage{pdflscape}[2016/05/14]
%</test4>
%    \end{macrocode}
%
%    \begin{macrocode}
%<*test0>
\begin{document}
\begin{qstest}{driver}{driver}
  \makeatletter
  \Expect*{\PLS@driver}*{\ExpectDriver}%
\end{qstest}
\end{document}
%</test0>
%    \end{macrocode}
%
% \subsection{Test for \cs{PLS@CheckAngle}}
%
%    \begin{macrocode}
%<*test5>
\NeedsTeXFormat{LaTeX2e}
\documentclass{minimal}
\usepackage{qstest}
\IncludeTests{*}
\LogTests{log}{*}{*}
\usepackage{pdflscape}[2016/05/14]
\begin{document}
\begin{qstest}{checkangle}{checkangle}
  \makeatletter
  \def\Result{%
    \def\ResultString{\the\count@}%
  }%
  \def\PackageError#1#2#3{%
    \def\Result{%
      \def\ResultString{error}%
    }%
  }%
  \def\Test#1#2{%
    \begingroup
      \PLS@CheckAngle{#1}%
      \Result
      \Expect*{\ResultString}{#2}%
    \endgroup
  }%
  \Test{0}{0}%
  \Test{90}{90}%
  \Test{180}{180}%
  \Test{270}{270}%
  \Test{360}{0}%
  \Test{450}{90}%
  \Test{540}{180}%
  \Test{630}{270}%
  \Test{720}{0}%
  \Test{3600}{0}%
  \Test{3690}{90}%
  \Test{-90}{270}%
  \Test{-180}{180}%
  \Test{-270}{-90}%
  \Test{-360}{0}%
  \Test{-450}{270}%
  \Test{540}{180}%
  \Test{-630}{90}%
  \Test{-720}{0}%
  \Test{-3600}{0}%
  \Test{-3690}{270}%
  \Test{1}{error}%
  \Test{-1}{error}%
  \Test{123}{error}%
\end{qstest}
\end{document}
%</test5>
%    \end{macrocode}
%
% \subsection{Test for rotate}
%
%    \begin{macrocode}
%<*test6>
\NeedsTeXFormat{LaTeX2e}
\documentclass{article}
\usepackage{pdflscape}[2016/05/14]
\begin{document}
\section{Portrait page}
\begin{landscape}
\section{Landscape page}
Foo bar foo bar foo bar foo bar foo bar
foo bar foo bar foo bar foo bar foo bar.
\end{landscape}
\section{Portrait Section}
Foo bar
\begin{landscape}
\section{Again Landscape Section}
\newpage
\section{Second Page of Landscape Section}
\end{landscape}
\section{Second Last Portrait Page}
\newpage
\section{Last Portrait Page}
\end{document}
%</test6>
%    \end{macrocode}
%
%    \begin{macrocode}
%<*result6>
/Page 1 /Rotate 0
/Page 2 /Rotate 90
/Page 3 /Rotate 0
/Page 4 /Rotate 90
/Page 5 /Rotate 90
/Page 6 /Rotate 0
/Page 7 /Rotate 0
%</result6>
%    \end{macrocode}
%
%    \begin{macrocode}
%<*ExtractRotate>
/**
 * ExtractRotate.java
 *
 * Copyright (C) 2007 by Heiko Oberdiek <heiko.oberdiek at googlemail.com>
 *
 * Requires: PDFBox (http://www.pdfbox.org/)
 *
 * Syntax: java ExtractRotate <pdffile> <textfile>
 *
 * The <pdffile> is analyzed and for each page its rotation
 * setting is printed in the <textfile>. Example:
 *   /Page 1 /Rotate 0
 *   /Page 2 /Rotate 90
 */
import java.io.FileInputStream;
import java.io.FileWriter;
import org.pdfbox.pdfparser.PDFParser;
import org.pdfbox.pdmodel.PDDocument;
import org.pdfbox.pdmodel.PDDocumentCatalog;
import org.pdfbox.pdmodel.PDPage;

public class ExtractRotate {

    public static void main(String[] args) {
        try {
            String infile = args[0];
            String outfile = args[1];
            FileWriter out = new FileWriter(outfile);
            PDFParser parser =
                    new PDFParser(new FileInputStream(infile));
            parser.parse();
            PDDocument document = parser.getPDDocument();
            PDDocumentCatalog catalog = document.getDocumentCatalog();
            int i = 0;
            for (Object page: catalog.getAllPages()) {
                i++;
                out.write("/Page " + i + " " + "/Rotate "
                        + ((PDPage)page).findRotation() + "\n");
            }
            document.close();
            out.close();
        }
        catch (Exception e) {
            e.printStackTrace();
            System.exit(1);
        }
    }
}
%</ExtractRotate>
%    \end{macrocode}
%
% \section{Installation}
%
% \subsection{Download}
%
% \paragraph{Package.} This package is available on
% CTAN\footnote{\CTANpkg{pdflscape}}:
% \begin{description}
% \item[\CTAN{macros/latex/contrib/oberdiek/pdflscape.dtx}] The source file.
% \item[\CTAN{macros/latex/contrib/oberdiek/pdflscape.pdf}] Documentation.
% \end{description}
%
%
% \paragraph{Bundle.} All the packages of the bundle `oberdiek'
% are also available in a TDS compliant ZIP archive. There
% the packages are already unpacked and the documentation files
% are generated. The files and directories obey the TDS standard.
% \begin{description}
% \item[\CTANinstall{install/macros/latex/contrib/oberdiek.tds.zip}]
% \end{description}
% \emph{TDS} refers to the standard ``A Directory Structure
% for \TeX\ Files'' (\CTAN{tds/tds.pdf}). Directories
% with \xfile{texmf} in their name are usually organized this way.
%
% \subsection{Bundle installation}
%
% \paragraph{Unpacking.} Unpack the \xfile{oberdiek.tds.zip} in the
% TDS tree (also known as \xfile{texmf} tree) of your choice.
% Example (linux):
% \begin{quote}
%   |unzip oberdiek.tds.zip -d ~/texmf|
% \end{quote}
%
% \paragraph{Script installation.}
% Check the directory \xfile{TDS:scripts/oberdiek/} for
% scripts that need further installation steps.
% Package \xpackage{attachfile2} comes with the Perl script
% \xfile{pdfatfi.pl} that should be installed in such a way
% that it can be called as \texttt{pdfatfi}.
% Example (linux):
% \begin{quote}
%   |chmod +x scripts/oberdiek/pdfatfi.pl|\\
%   |cp scripts/oberdiek/pdfatfi.pl /usr/local/bin/|
% \end{quote}
%
% \subsection{Package installation}
%
% \paragraph{Unpacking.} The \xfile{.dtx} file is a self-extracting
% \docstrip\ archive. The files are extracted by running the
% \xfile{.dtx} through \plainTeX:
% \begin{quote}
%   \verb|tex pdflscape.dtx|
% \end{quote}
%
% \paragraph{TDS.} Now the different files must be moved into
% the different directories in your installation TDS tree
% (also known as \xfile{texmf} tree):
% \begin{quote}
% \def\t{^^A
% \begin{tabular}{@{}>{\ttfamily}l@{ $\rightarrow$ }>{\ttfamily}l@{}}
%   pdflscape.sty & tex/latex/oberdiek/pdflscape.sty\\
%   pdflscape.pdf & doc/latex/oberdiek/pdflscape.pdf\\
%   test/pdflscape-test1.tex & doc/latex/oberdiek/test/pdflscape-test1.tex\\
%   test/pdflscape-test2.tex & doc/latex/oberdiek/test/pdflscape-test2.tex\\
%   test/pdflscape-test3.tex & doc/latex/oberdiek/test/pdflscape-test3.tex\\
%   test/pdflscape-test4.tex & doc/latex/oberdiek/test/pdflscape-test4.tex\\
%   test/pdflscape-test5.tex & doc/latex/oberdiek/test/pdflscape-test5.tex\\
%   test/pdflscape-test6.tex & doc/latex/oberdiek/test/pdflscape-test6.tex\\
%   test/pdflscape-test6.txt & doc/latex/oberdiek/test/pdflscape-test6.txt\\
%   test/ExtractRotate.java & doc/latex/oberdiek/test/ExtractRotate.java\\
%   pdflscape.dtx & source/latex/oberdiek/pdflscape.dtx\\
% \end{tabular}^^A
% }^^A
% \sbox0{\t}^^A
% \ifdim\wd0>\linewidth
%   \begingroup
%     \advance\linewidth by\leftmargin
%     \advance\linewidth by\rightmargin
%   \edef\x{\endgroup
%     \def\noexpand\lw{\the\linewidth}^^A
%   }\x
%   \def\lwbox{^^A
%     \leavevmode
%     \hbox to \linewidth{^^A
%       \kern-\leftmargin\relax
%       \hss
%       \usebox0
%       \hss
%       \kern-\rightmargin\relax
%     }^^A
%   }^^A
%   \ifdim\wd0>\lw
%     \sbox0{\small\t}^^A
%     \ifdim\wd0>\linewidth
%       \ifdim\wd0>\lw
%         \sbox0{\footnotesize\t}^^A
%         \ifdim\wd0>\linewidth
%           \ifdim\wd0>\lw
%             \sbox0{\scriptsize\t}^^A
%             \ifdim\wd0>\linewidth
%               \ifdim\wd0>\lw
%                 \sbox0{\tiny\t}^^A
%                 \ifdim\wd0>\linewidth
%                   \lwbox
%                 \else
%                   \usebox0
%                 \fi
%               \else
%                 \lwbox
%               \fi
%             \else
%               \usebox0
%             \fi
%           \else
%             \lwbox
%           \fi
%         \else
%           \usebox0
%         \fi
%       \else
%         \lwbox
%       \fi
%     \else
%       \usebox0
%     \fi
%   \else
%     \lwbox
%   \fi
% \else
%   \usebox0
% \fi
% \end{quote}
% If you have a \xfile{docstrip.cfg} that configures and enables \docstrip's
% TDS installing feature, then some files can already be in the right
% place, see the documentation of \docstrip.
%
% \subsection{Refresh file name databases}
%
% If your \TeX~distribution
% (\teTeX, \mikTeX, \dots) relies on file name databases, you must refresh
% these. For example, \teTeX\ users run \verb|texhash| or
% \verb|mktexlsr|.
%
% \subsection{Some details for the interested}
%
% \paragraph{Attached source.}
%
% The PDF documentation on CTAN also includes the
% \xfile{.dtx} source file. It can be extracted by
% AcrobatReader 6 or higher. Another option is \textsf{pdftk},
% e.g. unpack the file into the current directory:
% \begin{quote}
%   \verb|pdftk pdflscape.pdf unpack_files output .|
% \end{quote}
%
% \paragraph{Unpacking with \LaTeX.}
% The \xfile{.dtx} chooses its action depending on the format:
% \begin{description}
% \item[\plainTeX:] Run \docstrip\ and extract the files.
% \item[\LaTeX:] Generate the documentation.
% \end{description}
% If you insist on using \LaTeX\ for \docstrip\ (really,
% \docstrip\ does not need \LaTeX), then inform the autodetect routine
% about your intention:
% \begin{quote}
%   \verb|latex \let\install=y\input{pdflscape.dtx}|
% \end{quote}
% Do not forget to quote the argument according to the demands
% of your shell.
%
% \paragraph{Generating the documentation.}
% You can use both the \xfile{.dtx} or the \xfile{.drv} to generate
% the documentation. The process can be configured by the
% configuration file \xfile{ltxdoc.cfg}. For instance, put this
% line into this file, if you want to have A4 as paper format:
% \begin{quote}
%   \verb|\PassOptionsToClass{a4paper}{article}|
% \end{quote}
% An example follows how to generate the
% documentation with pdf\LaTeX:
% \begin{quote}
%\begin{verbatim}
%pdflatex pdflscape.dtx
%makeindex -s gind.ist pdflscape.idx
%pdflatex pdflscape.dtx
%makeindex -s gind.ist pdflscape.idx
%pdflatex pdflscape.dtx
%\end{verbatim}
% \end{quote}
%
% \begin{History}
%   \begin{Version}{2001/01/15 v0.1}
%   \item
%     First public version,
%     published in \xnewsgroup{de.comp.text.tex}:\\
%     \URL{``\link{Re: Querformat, pdfLaTex und thumbpdf}''}^^A
%     {https://groups.google.com/group/de.comp.text.tex/msg/e054c5795e52d2b8}
%   \end{Version}
%   \begin{Version}{2001/02/04 v0.2}
%   \item
%     Minor documentation update.
%   \item
%     CTAN.
%   \end{Version}
%   \begin{Version}{2004/05/11 v0.3}
%   \item
%     Support for dvipdfm added.
%   \end{Version}
%   \begin{Version}{2004/05/12 v0.4}
%   \item
%     Bug fix: support for multipage landscape environment
%     for all drivers except for \xoption{pdftex}.
%   \end{Version}
%   \begin{Version}{2006/02/20 v0.5}
%   \item
%     DTX framework.
%   \item
%     LPPL 1.3
%   \item
%     Code is not changed.
%   \end{Version}
%   \begin{Version}{2006/04/24 v0.6}
%   \item
%     Bug fix: \cs{PLS@RemoveRotate} reinserted for \cs{PLS@Rotate}.
%   \item
%     Fix for \cs{PLS@RemoveRotate} (dvips/dvipdfm): \cs{PLS@EverhShi}
%     hook is cleared.
%   \end{Version}
%   \begin{Version}{2007/04/11 v0.7}
%   \item
%     Line ends sanitized.
%   \end{Version}
%   \begin{Version}{2007/04/17 v0.8}
%   \item
%     Package \xpackage{atbegshi} replaces \xpackage{everyshi}.
%   \end{Version}
%   \begin{Version}{2007/10/21 v0.9}
%   \item
%     Driver detection for \XeTeX\ added.
%   \item
%     Fix for rotation angles $\le$~-360 or $\ge$~360.
%   \end{Version}
%   \begin{Version}{2008/08/11 v0.10}
%   \item
%     Code is not changed.
%   \item
%     URLs updated.
%   \end{Version}
%   \begin{Version}{2016/05/14 v0.11}
%   \item
%     LuaTeX Compatibility
%   \end{Version}
% \end{History}
%
% \PrintIndex
%
% \Finale
\endinput

%        (quote the arguments according to the demands of your shell)
%
% Documentation:
%    (a) If pdflscape.drv is present:
%           latex pdflscape.drv
%    (b) Without pdflscape.drv:
%           latex pdflscape.dtx; ...
%    The class ltxdoc loads the configuration file ltxdoc.cfg
%    if available. Here you can specify further options, e.g.
%    use A4 as paper format:
%       \PassOptionsToClass{a4paper}{article}
%
%    Programm calls to get the documentation (example):
%       pdflatex pdflscape.dtx
%       makeindex -s gind.ist pdflscape.idx
%       pdflatex pdflscape.dtx
%       makeindex -s gind.ist pdflscape.idx
%       pdflatex pdflscape.dtx
%
% Installation:
%    TDS:tex/latex/oberdiek/pdflscape.sty
%    TDS:doc/latex/oberdiek/pdflscape.pdf
%    TDS:doc/latex/oberdiek/test/pdflscape-test1.tex
%    TDS:doc/latex/oberdiek/test/pdflscape-test2.tex
%    TDS:doc/latex/oberdiek/test/pdflscape-test3.tex
%    TDS:doc/latex/oberdiek/test/pdflscape-test4.tex
%    TDS:doc/latex/oberdiek/test/pdflscape-test5.tex
%    TDS:doc/latex/oberdiek/test/pdflscape-test6.tex
%    TDS:doc/latex/oberdiek/test/pdflscape-test6.txt
%    TDS:doc/latex/oberdiek/test/ExtractRotate.java
%    TDS:source/latex/oberdiek/pdflscape.dtx
%
%<*ignore>
\begingroup
  \catcode123=1 %
  \catcode125=2 %
  \def\x{LaTeX2e}%
\expandafter\endgroup
\ifcase 0\ifx\install y1\fi\expandafter
         \ifx\csname processbatchFile\endcsname\relax\else1\fi
         \ifx\fmtname\x\else 1\fi\relax
\else\csname fi\endcsname
%</ignore>
%<*install>
\input docstrip.tex
\Msg{************************************************************************}
\Msg{* Installation}
\Msg{* Package: pdflscape 2016/05/14 v0.11 Display of landscape pages in PDF (HO)}
\Msg{************************************************************************}

\keepsilent
\askforoverwritefalse

\let\MetaPrefix\relax
\preamble

This is a generated file.

Project: pdflscape
Version: 2016/05/14 v0.11

Copyright (C) 2001, 2004, 2006-2008 by
   Heiko Oberdiek <heiko.oberdiek at googlemail.com>

This work may be distributed and/or modified under the
conditions of the LaTeX Project Public License, either
version 1.3c of this license or (at your option) any later
version. This version of this license is in
   http://www.latex-project.org/lppl/lppl-1-3c.txt
and the latest version of this license is in
   http://www.latex-project.org/lppl.txt
and version 1.3 or later is part of all distributions of
LaTeX version 2005/12/01 or later.

This work has the LPPL maintenance status "maintained".

This Current Maintainer of this work is Heiko Oberdiek.

This work consists of the main source file pdflscape.dtx
and the derived files
   pdflscape.sty, pdflscape.pdf, pdflscape.ins, pdflscape.drv,
   pdflscape-test1.tex, pdflscape-test2.tex,
   pdflscape-test3.tex, pdflscape-test4.tex,
   pdflscape-test5.tex, pdflscape-test6.tex,
   pdflscape-test6.txt, ExtractRotate.java.

\endpreamble
\let\MetaPrefix\DoubleperCent

\generate{%
  \file{pdflscape.ins}{\from{pdflscape.dtx}{install}}%
  \file{pdflscape.drv}{\from{pdflscape.dtx}{driver}}%
  \usedir{tex/latex/oberdiek}%
  \file{pdflscape.sty}{\from{pdflscape.dtx}{package}}%
%  \usedir{doc/latex/oberdiek/test}%
%  \file{pdflscape-test1.tex}{\from{pdflscape.dtx}{test0,test1}}%
%  \file{pdflscape-test2.tex}{\from{pdflscape.dtx}{test0,test2}}%
%  \file{pdflscape-test3.tex}{\from{pdflscape.dtx}{test0,test3}}%
%  \file{pdflscape-test4.tex}{\from{pdflscape.dtx}{test0,test4}}%
%  \file{pdflscape-test5.tex}{\from{pdflscape.dtx}{test5}}%
%  \file{pdflscape-test6.tex}{\from{pdflscape.dtx}{test6}}%
  \nopreamble
  \nopostamble
%  \file{pdflscape-test6.txt}{\from{pdflscape.dtx}{result6}}%
  \file{ExtractRotate.java}{\from{pdflscape.dtx}{ExtractRotate}}%
%  \usedir{source/latex/oberdiek/catalogue}%
%  \file{pdflscape.xml}{\from{pdflscape.dtx}{catalogue}}%
}

\catcode32=13\relax% active space
\let =\space%
\Msg{************************************************************************}
\Msg{*}
\Msg{* To finish the installation you have to move the following}
\Msg{* file into a directory searched by TeX:}
\Msg{*}
\Msg{*     pdflscape.sty}
\Msg{*}
\Msg{* To produce the documentation run the file `pdflscape.drv'}
\Msg{* through LaTeX.}
\Msg{*}
\Msg{* Happy TeXing!}
\Msg{*}
\Msg{************************************************************************}

\endbatchfile
%</install>
%<*ignore>
\fi
%</ignore>
%<*driver>
\NeedsTeXFormat{LaTeX2e}
\ProvidesFile{pdflscape.drv}%
  [2016/05/14 v0.11 Display of landscape pages in PDF (HO)]%
\documentclass{ltxdoc}
\usepackage{holtxdoc}[2011/11/22]
\usepackage{graphics}
\begin{document}
  \DocInput{pdflscape.dtx}%
\end{document}
%</driver>
% \fi
%
%
% \CharacterTable
%  {Upper-case    \A\B\C\D\E\F\G\H\I\J\K\L\M\N\O\P\Q\R\S\T\U\V\W\X\Y\Z
%   Lower-case    \a\b\c\d\e\f\g\h\i\j\k\l\m\n\o\p\q\r\s\t\u\v\w\x\y\z
%   Digits        \0\1\2\3\4\5\6\7\8\9
%   Exclamation   \!     Double quote  \"     Hash (number) \#
%   Dollar        \$     Percent       \%     Ampersand     \&
%   Acute accent  \'     Left paren    \(     Right paren   \)
%   Asterisk      \*     Plus          \+     Comma         \,
%   Minus         \-     Point         \.     Solidus       \/
%   Colon         \:     Semicolon     \;     Less than     \<
%   Equals        \=     Greater than  \>     Question mark \?
%   Commercial at \@     Left bracket  \[     Backslash     \\
%   Right bracket \]     Circumflex    \^     Underscore    \_
%   Grave accent  \`     Left brace    \{     Vertical bar  \|
%   Right brace   \}     Tilde         \~}
%
% \GetFileInfo{pdflscape.drv}
%
% \title{The \xpackage{pdflscape} package}
% \date{2016/05/14 v0.11}
% \author{Heiko Oberdiek\thanks
% {Please report any issues at \url{https://github.com/ho-tex/oberdiek/issues}}\\
% \xemail{heiko.oberdiek at googlemail.com}}
%
% \maketitle
%
% \begin{abstract}
% Package \xpackage{pdflscape} adds PDF support to the
% environment \texttt{landscape} of package \xpackage{lscape}
% by setting the PDF page attribute \texttt{/Rotate}.
% \end{abstract}
%
% \tableofcontents
%
% \section{Documentation}
%
% \subsection{Supported drivers}
%
% \begin{itemize}
% \item \xoption{pdftex},\xoption{luatex}
% \item \xoption{dvips}, \xoption{dvipsone},
%       \xoption{pctex32}, \xoption{pctexps}
%       (and other drivers that provide a non-empty \cs{Gin@PS@raw})
% \item \xoption{dvipdfm}
% \end{itemize}
%
% \subsection{Caveat}
%
% Depending on the configuration Ghostscript
% adds a guessed rotation entry by its own.
% This can lead to two |/Rotate| entries per page.
% To prevent this behaviour set the parameter
% |AutoRotatePages| to |/None|, eg:
% \begin{quote}
%   |ps2pdf -dAutoRotatePages=/None|
% \end{quote}
%
% \subsection{Requirements}
%
% \begin{itemize}
% \item The package \xpackage{lscape}.
% \item The package \xpackage{atbegshi} for all drivers
%        except \xoption{pdftex},\xoption{luatex}.
% \end{itemize}
%
% \subsection{Usage}
%
% Load this package instead of or after package \xpackage{lscape}:
% \begin{quote}
%   |\usepackage{pdflscape}|\\
%   |\begin{landscape}|\dots|\end{landscape}|
% \end{quote}
%
% \StopEventually{
% }
%
% \section{Implementation}
%
% \subsection{Package identification}
%
%    \begin{macrocode}
%<*package>
%    \end{macrocode}
%    Package identification.
%    \begin{macrocode}
\NeedsTeXFormat{LaTeX2e}
\ProvidesPackage{pdflscape}%
  [2016/05/14 v0.11 Display of landscape pages in PDF (HO)]
%    \end{macrocode}
%
% \subsection{Driver options}
%
%    \begin{macrocode}
\let\PLS@option\@empty
\let\PLS@driver\@empty
\def\PLS@temp#1#2{%
  \DeclareOption{#1}{%
    \let\PLS@option\CurrentOption
    \def\PLS@driver{#2}%
  }%
}
\PLS@temp{pdftex}{pdftex}
\PLS@temp{luatex}{pdftex}
\PLS@temp{dvips}{dvips}
\PLS@temp{dvipsone}{dvips}
\PLS@temp{pctexps}{dvips}
\PLS@temp{dvipdfm}{dvipdfm}
\PLS@temp{dvipdfmx}{dvipdfm}
\PLS@temp{xetex}{dvipdfm}
\let\PLS@temp\@empty
\DeclareOption*{%
  \ifx\PLS@temp\@empty
    \let\PLS@temp\CurrentOption
  \else
    \edef\PLS@temp{\PLS@temp,\CurrentOption}%
  \fi
}
\ProcessOptions*
\ifx\PLS@option\@empty
\else
  \expandafter\PassOptionsToPackage\expandafter{\PLS@option}{lscape}%
\fi
\ifx\PLS@temp\@empty
\else
  \expandafter\PassOptionsToPackage\expandafter{\PLS@temp}{lscape}%
\fi
\RequirePackage{lscape}
%    \end{macrocode}
%
% \subsection{Autodetection of driver}
%
% \subsubsection{Driver \xoption{pdftex}}
%
%    \begin{macrocode}
\RequirePackage{ifpdf}[2006/02/20]
\def\PLS@temp{pdftex}
\ifpdf
  \ifx\PLS@driver\@empty
    \let\PLS@driver\PLS@temp
    \PackageInfo{pdflscape}{Auto-detected driver: \PLS@temp}%
  \fi
  \ifx\PLS@driver\PLS@temp
  \else
    \PackageError{pdflscape}{%
      The driver option `\PLS@option' is not compatible\MessageBreak
      with pdfTeX in PDF mode, using driver `pdftex' instead%
    }\@ehc
    \let\PLS@driver\PLS@temp
  \fi
\fi
%    \end{macrocode}
%
% \subsubsection{Driver \xoption{xetex}}
%
%    \begin{macrocode}
\RequirePackage{ifxetex}%
\def\PLS@temp{dvipdfm}
\ifxetex
  \ifx\PLS@driver\@empty
    \let\PLS@driver\PLS@temp
    \PackageInfo{pdflscape}{%
      Auto-detected driver: \PLS@temp\space(xetex)%
    }%
  \fi
  \ifx\PLS@driver\PLS@temp
  \else
    \PackageError{pdflscape}{%
      The driver option `\PLS@option' is not compatible\MessageBreak
      with XeTeX, using driver for XeTeX (dvipdfm) instead%
    }\@ehc
    \let\PLS@driver\PLS@temp
  \fi
\fi
%    \end{macrocode}
%
% \subsubsection{Detect driver based on \cs{Gin@driver}}
%
%    \begin{macrocode}
\def\PLS@temp#1#2{%
  \ifx\PLS@driver\@empty
    \begingroup
      \def\PLS@temp{#1.def}%
    \expandafter\endgroup
    \ifx\PLS@temp\Gin@driver
      \def\PLS@driver{#2}%
      \PackageInfo{pdflscape}{%
        Auto-detected driver: \PLS@driver\space(#1)%
      }%
    \fi
  \fi
}
\PLS@temp{dvipdfm}{dvipdfm}
\PLS@temp{dvipdfmx}{dvipdfm}
\PLS@temp{dvips}{dvips}
\PLS@temp{dvipsone}{dvips}
\PLS@temp{pctexps}{dvips}
%    \end{macrocode}
%
% \subsubsection{Driver \xoption{dvips}}
%
%    \begin{macrocode}
\ifx\PLS@driver\@empty
  \def\PLS@temp{dvips}%
  \ifnum 1=0\ifx\Gin@PS@raw\@undefined
            \else
              \ifx\Gin@PS@raw\relax
              \else
                \ifx\Gin@PS@raw\@empty
                \else
                  1%
                \fi
              \fi
            \fi
            \space
    \let\PLS@driver\PLS@temp
    \PackageInfo{pdflscape}{%
      Auto-detected driver: \PLS@temp\space(PostScript)%
    }%
  \fi
\fi
%    \end{macrocode}
%
%    \begin{macrocode}
\ifx\PLS@driver\@empty
  \PackageError{pdflscape}{%
    Neither pdfTeX, PostScript, nor dvipdfm driver found%
  }\@ehc
\fi
%    \end{macrocode}
%
% \subsection{Driver implementation}
%
% \begin{itemize}
% \item |\PLS@AddRotate #1|\\
%   it expects the correct rotation number in |#1|
%   and implements the adding of the /Rotation entry
%   in the |/Page| object of the current page.
% \item |\PLS@RemoveRotate|\\
%   it removes a previous |/Rotate| entry, if necessary.
% \end{itemize}
%
% \subsubsection{\pdfTeX}
%
% Not too nice is the global setting of \cs{pdfpageattr}.
% Perhaps this can be changed in future versions.
%
% luatex compat.
%    \begin{macrocode}
\ifx\pdfpageattr\@undefined
  \ifx\pdfvariable\@undefined\else 
    \protected\edef\pdfpageattr{\pdfvariable pageattr}
  \fi
\fi
%    \end{macrocode}
%    \begin{macrocode}
\def\PLS@temp{pdftex}
\ifx\PLS@temp\PLS@driver
  \def\PLS@AddRotate#1{%
    \ifnum#1=0 %
%    \end{macrocode}
%    Already the default.
%    \begin{macrocode}
    \else
      \global\pdfpageattr\expandafter{%
        \the\pdfpageattr
        /Rotate #1%
      }%
    \fi
  }%
%    \end{macrocode}
%    Removes a |/Rotate| entry.
%    \begin{macrocode}
  \def\PLS@RemoveRotate{%
    \begingroup
      \global\pdfpageattr\expandafter{\expandafter}%
      \expandafter\PLS@@RemoveRotate
        \the\pdfpageattr /Rotate\@nil
    \endgroup
  }%
  \def\PLS@@RemoveRotate#1/Rotate#2\@nil{%
%    \end{macrocode}
%    Append |/Rotate| free stuff to \cs{pdfpageattr}.
%    \begin{macrocode}
    \global\pdfpageattr\expandafter{\the\pdfpageattr#1}%
    \ifx\\#2\\%
%    \end{macrocode}
%    Ready, because the detected |/Rotate| is part of
%    the end marker: |/Rotate\@nil|
%    \begin{macrocode}
    \else
%    \end{macrocode}
%    First read in the argument of |/Rotate|,
%    then continue parsing.
%    \begin{macrocode}
      \afterassignment\PLS@@RemoveRotate
      \count0=#2\@nil
    \fi
  }%
\fi
%    \end{macrocode}
%
% \subsubsection{PostScript driver}
%
%    \begin{macrocode}
\def\PLS@temp{dvips}
\ifx\PLS@temp\PLS@driver
  \RequirePackage{atbegshi}%
  \AtBeginShipout{\PLS@AtBeginShipout}%
  \let\PLS@AtBeginShipout\@empty
  \def\PLS@AddRotate#1{%
    \ifnum#1=0 %
    \else
      \def\PLS@AtBeginShipout{%
        \global\setbox\AtBeginShipoutBox\vbox{%
          \Gin@PS@raw{%
            [{ThisPage}\string<\string</Rotate #1\string>\string>%
            /PUT pdfmark%
          }%
          \box\AtBeginShipoutBox
        }%
      }%
    \fi
  }%
  \def\PLS@RemoveRotate{\let\PLS@AtBeginShipout\@empty}%
\fi
%    \end{macrocode}
%
% \subsubsection{Driver dvipdfm}
%
%    \begin{macrocode}
\def\PLS@temp{dvipdfm}
\ifx\PLS@temp\PLS@driver
  \RequirePackage{atbegshi}%
  \AtBeginShipout{\PLS@AtBeginShipout}%
  \let\PLS@AtBeginShipout\@empty
  \def\PLS@AddRotate#1{%
    \ifnum#1=0 %
    \else
      \def\PLS@AtBeginShipout{%
        \global\setbox\AtBeginShipoutBox\vbox{%
          \special{pdf: put @thispage <</Rotate #1>>}%
          \box\AtBeginShipoutBox
        }%
      }%
    \fi
  }%
  \def\PLS@RemoveRotate{\let\PLS@AtBeginShipout\@empty}%
\fi
%    \end{macrocode}
%
% \subsection{Driver independent stuff}
%
%    The landscape environment is extended by
%    adding the correct |/Rotate| entries.
%    \begin{macrocode}
\g@addto@macro{\landscape}{\PLS@Rotate{90}}
\g@addto@macro{\endlandscape}{\PLS@Rotate{0}}
%    \end{macrocode}
%
%    \begin{macro}{\PLS@Rotate}
%    Main macro, that sets the |/Rotate| entry.\\
%    Argument: any TeX number or nothing, that means zero.\\
%    Driver independent.
%    \begin{macrocode}
\def\PLS@Rotate#1{%
  \begingroup
%    \end{macrocode}
%    1. Check and validate the argument.
%    \begin{macrocode}
    \PLS@CheckAngle{#1}%
  \expandafter\endgroup
%    \end{macrocode}
%    2. Remove previous |\Rotate| entry.
%    \begin{macrocode}
  \expandafter\PLS@RemoveRotate
%    \end{macrocode}
%    3. Add |/Rotate| entry.
%    \begin{macrocode}
  \expandafter\PLS@AddRotate\expandafter{\the\count@}%
}
%    \end{macrocode}
%    \end{macro}
%
%    \begin{macro}{\PLS@CheckAngle}
%    Validates the rotation angle.\\
%    The result is stored in the count register |\count@|.\\
%    Driver independent.
%    \begin{macrocode}
\def\PLS@CheckAngle#1{%
%    \end{macrocode}
%    Nothing means zero:
%    \begin{macrocode}
  \ifx\\#1\\%
    \count@=0 %
  \else
    \count@=#1\relax
  \fi
%    \end{macrocode}
%    Normalize to interval -360 $<$ |\count@| $<$ 360.
%    \begin{macrocode}
  \@whilenum\count@>359\do{\advance\count@ -360 }%
  \@whilenum\count@<-359\do{\advance\count@ 360 }%
%    \end{macrocode}
%    Check values: 0, +/-90, +/-180, +/-270
%    \begin{macrocode}
  \ifnum 1=0\ifnum\count@=0 \else
            \ifnum\count@=90 \else
            \ifnum\count@=-90 \else
            \ifnum\count@=180 \else
            \ifnum\count@=-180 \else
            \ifnum\count@=270 \else
            \ifnum\count@=-270 \else
           1\fi\fi\fi\fi\fi\fi\fi\relax
    \PackageError{pdflscape}{%
      Invalid value \the\count@\space for /Rotate%
    }\@ehc
    \count@=0 %
  \fi
}
%    \end{macrocode}
%    \end{macro}
%
%    \begin{macrocode}
%</package>
%    \end{macrocode}
%
% \section{Test}
%
% \subsection{Driver detection tests}
%    \begin{macrocode}
%<*test0>
\NeedsTeXFormat{LaTeX2e}
\documentclass{minimal}
\usepackage{qstest}
\IncludeTests{*}
\LogTests{log}{*}{*}
%</test0>
%    \end{macrocode}
%
%    \begin{macrocode}
%<*test1>
\usepackage{ifpdf}
\ifpdf
  \def\ExpectDriver{pdftex}%
\fi
\usepackage{ifxetex}%
\ifxetex
  \def\ExpectDriver{dvipdfm}%
\fi
\usepackage{pdflscape}[2016/05/14]
%</test1>
%    \end{macrocode}
%
%    \begin{macrocode}
%<*test2>
\def\ExpectDriver{pdftex}
\usepackage[pdftex]{pdflscape}[2016/05/14]
%</test2>
%    \end{macrocode}
%
%    \begin{macrocode}
%<*test3>
\def\ExpectDriver{dvipdfm}
\usepackage[dvipdfmx]{pdflscape}[2016/05/14]
%</test3>
%    \end{macrocode}
%
%    \begin{macrocode}
%<*test4>
\def\ExpectDriver{dvips}
\usepackage[dvipsone]{graphics}
\usepackage{pdflscape}[2016/05/14]
%</test4>
%    \end{macrocode}
%
%    \begin{macrocode}
%<*test0>
\begin{document}
\begin{qstest}{driver}{driver}
  \makeatletter
  \Expect*{\PLS@driver}*{\ExpectDriver}%
\end{qstest}
\end{document}
%</test0>
%    \end{macrocode}
%
% \subsection{Test for \cs{PLS@CheckAngle}}
%
%    \begin{macrocode}
%<*test5>
\NeedsTeXFormat{LaTeX2e}
\documentclass{minimal}
\usepackage{qstest}
\IncludeTests{*}
\LogTests{log}{*}{*}
\usepackage{pdflscape}[2016/05/14]
\begin{document}
\begin{qstest}{checkangle}{checkangle}
  \makeatletter
  \def\Result{%
    \def\ResultString{\the\count@}%
  }%
  \def\PackageError#1#2#3{%
    \def\Result{%
      \def\ResultString{error}%
    }%
  }%
  \def\Test#1#2{%
    \begingroup
      \PLS@CheckAngle{#1}%
      \Result
      \Expect*{\ResultString}{#2}%
    \endgroup
  }%
  \Test{0}{0}%
  \Test{90}{90}%
  \Test{180}{180}%
  \Test{270}{270}%
  \Test{360}{0}%
  \Test{450}{90}%
  \Test{540}{180}%
  \Test{630}{270}%
  \Test{720}{0}%
  \Test{3600}{0}%
  \Test{3690}{90}%
  \Test{-90}{270}%
  \Test{-180}{180}%
  \Test{-270}{-90}%
  \Test{-360}{0}%
  \Test{-450}{270}%
  \Test{540}{180}%
  \Test{-630}{90}%
  \Test{-720}{0}%
  \Test{-3600}{0}%
  \Test{-3690}{270}%
  \Test{1}{error}%
  \Test{-1}{error}%
  \Test{123}{error}%
\end{qstest}
\end{document}
%</test5>
%    \end{macrocode}
%
% \subsection{Test for rotate}
%
%    \begin{macrocode}
%<*test6>
\NeedsTeXFormat{LaTeX2e}
\documentclass{article}
\usepackage{pdflscape}[2016/05/14]
\begin{document}
\section{Portrait page}
\begin{landscape}
\section{Landscape page}
Foo bar foo bar foo bar foo bar foo bar
foo bar foo bar foo bar foo bar foo bar.
\end{landscape}
\section{Portrait Section}
Foo bar
\begin{landscape}
\section{Again Landscape Section}
\newpage
\section{Second Page of Landscape Section}
\end{landscape}
\section{Second Last Portrait Page}
\newpage
\section{Last Portrait Page}
\end{document}
%</test6>
%    \end{macrocode}
%
%    \begin{macrocode}
%<*result6>
/Page 1 /Rotate 0
/Page 2 /Rotate 90
/Page 3 /Rotate 0
/Page 4 /Rotate 90
/Page 5 /Rotate 90
/Page 6 /Rotate 0
/Page 7 /Rotate 0
%</result6>
%    \end{macrocode}
%
%    \begin{macrocode}
%<*ExtractRotate>
/**
 * ExtractRotate.java
 *
 * Copyright (C) 2007 by Heiko Oberdiek <heiko.oberdiek at googlemail.com>
 *
 * Requires: PDFBox (http://www.pdfbox.org/)
 *
 * Syntax: java ExtractRotate <pdffile> <textfile>
 *
 * The <pdffile> is analyzed and for each page its rotation
 * setting is printed in the <textfile>. Example:
 *   /Page 1 /Rotate 0
 *   /Page 2 /Rotate 90
 */
import java.io.FileInputStream;
import java.io.FileWriter;
import org.pdfbox.pdfparser.PDFParser;
import org.pdfbox.pdmodel.PDDocument;
import org.pdfbox.pdmodel.PDDocumentCatalog;
import org.pdfbox.pdmodel.PDPage;

public class ExtractRotate {

    public static void main(String[] args) {
        try {
            String infile = args[0];
            String outfile = args[1];
            FileWriter out = new FileWriter(outfile);
            PDFParser parser =
                    new PDFParser(new FileInputStream(infile));
            parser.parse();
            PDDocument document = parser.getPDDocument();
            PDDocumentCatalog catalog = document.getDocumentCatalog();
            int i = 0;
            for (Object page: catalog.getAllPages()) {
                i++;
                out.write("/Page " + i + " " + "/Rotate "
                        + ((PDPage)page).findRotation() + "\n");
            }
            document.close();
            out.close();
        }
        catch (Exception e) {
            e.printStackTrace();
            System.exit(1);
        }
    }
}
%</ExtractRotate>
%    \end{macrocode}
%
% \section{Installation}
%
% \subsection{Download}
%
% \paragraph{Package.} This package is available on
% CTAN\footnote{\CTANpkg{pdflscape}}:
% \begin{description}
% \item[\CTAN{macros/latex/contrib/oberdiek/pdflscape.dtx}] The source file.
% \item[\CTAN{macros/latex/contrib/oberdiek/pdflscape.pdf}] Documentation.
% \end{description}
%
%
% \paragraph{Bundle.} All the packages of the bundle `oberdiek'
% are also available in a TDS compliant ZIP archive. There
% the packages are already unpacked and the documentation files
% are generated. The files and directories obey the TDS standard.
% \begin{description}
% \item[\CTANinstall{install/macros/latex/contrib/oberdiek.tds.zip}]
% \end{description}
% \emph{TDS} refers to the standard ``A Directory Structure
% for \TeX\ Files'' (\CTAN{tds/tds.pdf}). Directories
% with \xfile{texmf} in their name are usually organized this way.
%
% \subsection{Bundle installation}
%
% \paragraph{Unpacking.} Unpack the \xfile{oberdiek.tds.zip} in the
% TDS tree (also known as \xfile{texmf} tree) of your choice.
% Example (linux):
% \begin{quote}
%   |unzip oberdiek.tds.zip -d ~/texmf|
% \end{quote}
%
% \paragraph{Script installation.}
% Check the directory \xfile{TDS:scripts/oberdiek/} for
% scripts that need further installation steps.
% Package \xpackage{attachfile2} comes with the Perl script
% \xfile{pdfatfi.pl} that should be installed in such a way
% that it can be called as \texttt{pdfatfi}.
% Example (linux):
% \begin{quote}
%   |chmod +x scripts/oberdiek/pdfatfi.pl|\\
%   |cp scripts/oberdiek/pdfatfi.pl /usr/local/bin/|
% \end{quote}
%
% \subsection{Package installation}
%
% \paragraph{Unpacking.} The \xfile{.dtx} file is a self-extracting
% \docstrip\ archive. The files are extracted by running the
% \xfile{.dtx} through \plainTeX:
% \begin{quote}
%   \verb|tex pdflscape.dtx|
% \end{quote}
%
% \paragraph{TDS.} Now the different files must be moved into
% the different directories in your installation TDS tree
% (also known as \xfile{texmf} tree):
% \begin{quote}
% \def\t{^^A
% \begin{tabular}{@{}>{\ttfamily}l@{ $\rightarrow$ }>{\ttfamily}l@{}}
%   pdflscape.sty & tex/latex/oberdiek/pdflscape.sty\\
%   pdflscape.pdf & doc/latex/oberdiek/pdflscape.pdf\\
%   test/pdflscape-test1.tex & doc/latex/oberdiek/test/pdflscape-test1.tex\\
%   test/pdflscape-test2.tex & doc/latex/oberdiek/test/pdflscape-test2.tex\\
%   test/pdflscape-test3.tex & doc/latex/oberdiek/test/pdflscape-test3.tex\\
%   test/pdflscape-test4.tex & doc/latex/oberdiek/test/pdflscape-test4.tex\\
%   test/pdflscape-test5.tex & doc/latex/oberdiek/test/pdflscape-test5.tex\\
%   test/pdflscape-test6.tex & doc/latex/oberdiek/test/pdflscape-test6.tex\\
%   test/pdflscape-test6.txt & doc/latex/oberdiek/test/pdflscape-test6.txt\\
%   test/ExtractRotate.java & doc/latex/oberdiek/test/ExtractRotate.java\\
%   pdflscape.dtx & source/latex/oberdiek/pdflscape.dtx\\
% \end{tabular}^^A
% }^^A
% \sbox0{\t}^^A
% \ifdim\wd0>\linewidth
%   \begingroup
%     \advance\linewidth by\leftmargin
%     \advance\linewidth by\rightmargin
%   \edef\x{\endgroup
%     \def\noexpand\lw{\the\linewidth}^^A
%   }\x
%   \def\lwbox{^^A
%     \leavevmode
%     \hbox to \linewidth{^^A
%       \kern-\leftmargin\relax
%       \hss
%       \usebox0
%       \hss
%       \kern-\rightmargin\relax
%     }^^A
%   }^^A
%   \ifdim\wd0>\lw
%     \sbox0{\small\t}^^A
%     \ifdim\wd0>\linewidth
%       \ifdim\wd0>\lw
%         \sbox0{\footnotesize\t}^^A
%         \ifdim\wd0>\linewidth
%           \ifdim\wd0>\lw
%             \sbox0{\scriptsize\t}^^A
%             \ifdim\wd0>\linewidth
%               \ifdim\wd0>\lw
%                 \sbox0{\tiny\t}^^A
%                 \ifdim\wd0>\linewidth
%                   \lwbox
%                 \else
%                   \usebox0
%                 \fi
%               \else
%                 \lwbox
%               \fi
%             \else
%               \usebox0
%             \fi
%           \else
%             \lwbox
%           \fi
%         \else
%           \usebox0
%         \fi
%       \else
%         \lwbox
%       \fi
%     \else
%       \usebox0
%     \fi
%   \else
%     \lwbox
%   \fi
% \else
%   \usebox0
% \fi
% \end{quote}
% If you have a \xfile{docstrip.cfg} that configures and enables \docstrip's
% TDS installing feature, then some files can already be in the right
% place, see the documentation of \docstrip.
%
% \subsection{Refresh file name databases}
%
% If your \TeX~distribution
% (\teTeX, \mikTeX, \dots) relies on file name databases, you must refresh
% these. For example, \teTeX\ users run \verb|texhash| or
% \verb|mktexlsr|.
%
% \subsection{Some details for the interested}
%
% \paragraph{Attached source.}
%
% The PDF documentation on CTAN also includes the
% \xfile{.dtx} source file. It can be extracted by
% AcrobatReader 6 or higher. Another option is \textsf{pdftk},
% e.g. unpack the file into the current directory:
% \begin{quote}
%   \verb|pdftk pdflscape.pdf unpack_files output .|
% \end{quote}
%
% \paragraph{Unpacking with \LaTeX.}
% The \xfile{.dtx} chooses its action depending on the format:
% \begin{description}
% \item[\plainTeX:] Run \docstrip\ and extract the files.
% \item[\LaTeX:] Generate the documentation.
% \end{description}
% If you insist on using \LaTeX\ for \docstrip\ (really,
% \docstrip\ does not need \LaTeX), then inform the autodetect routine
% about your intention:
% \begin{quote}
%   \verb|latex \let\install=y% \iffalse meta-comment
%
% File: pdflscape.dtx
% Version: 2016/05/14 v0.11
% Info: Display of landscape pages in PDF
%
% Copyright (C) 2001, 2004, 2006-2008 by
%    Heiko Oberdiek <heiko.oberdiek at googlemail.com>
%    2016
%    https://github.com/ho-tex/oberdiek/issues
%
% This work may be distributed and/or modified under the
% conditions of the LaTeX Project Public License, either
% version 1.3c of this license or (at your option) any later
% version. This version of this license is in
%    http://www.latex-project.org/lppl/lppl-1-3c.txt
% and the latest version of this license is in
%    http://www.latex-project.org/lppl.txt
% and version 1.3 or later is part of all distributions of
% LaTeX version 2005/12/01 or later.
%
% This work has the LPPL maintenance status "maintained".
%
% This Current Maintainer of this work is Heiko Oberdiek.
%
% This work consists of the main source file pdflscape.dtx
% and the derived files
%    pdflscape.sty, pdflscape.pdf, pdflscape.ins, pdflscape.drv,
%    pdflscape-test1.tex, pdflscape-test2.tex,
%    pdflscape-test3.tex, pdflscape-test4.tex,
%    pdflscape-test5.tex, pdflscape-test6.tex,
%    pdflscape-test6.txt, ExtractRotate.java.
%
% Distribution:
%    CTAN:macros/latex/contrib/oberdiek/pdflscape.dtx
%    CTAN:macros/latex/contrib/oberdiek/pdflscape.pdf
%
% Unpacking:
%    (a) If pdflscape.ins is present:
%           tex pdflscape.ins
%    (b) Without pdflscape.ins:
%           tex pdflscape.dtx
%    (c) If you insist on using LaTeX
%           latex \let\install=y\input{pdflscape.dtx}
%        (quote the arguments according to the demands of your shell)
%
% Documentation:
%    (a) If pdflscape.drv is present:
%           latex pdflscape.drv
%    (b) Without pdflscape.drv:
%           latex pdflscape.dtx; ...
%    The class ltxdoc loads the configuration file ltxdoc.cfg
%    if available. Here you can specify further options, e.g.
%    use A4 as paper format:
%       \PassOptionsToClass{a4paper}{article}
%
%    Programm calls to get the documentation (example):
%       pdflatex pdflscape.dtx
%       makeindex -s gind.ist pdflscape.idx
%       pdflatex pdflscape.dtx
%       makeindex -s gind.ist pdflscape.idx
%       pdflatex pdflscape.dtx
%
% Installation:
%    TDS:tex/latex/oberdiek/pdflscape.sty
%    TDS:doc/latex/oberdiek/pdflscape.pdf
%    TDS:doc/latex/oberdiek/test/pdflscape-test1.tex
%    TDS:doc/latex/oberdiek/test/pdflscape-test2.tex
%    TDS:doc/latex/oberdiek/test/pdflscape-test3.tex
%    TDS:doc/latex/oberdiek/test/pdflscape-test4.tex
%    TDS:doc/latex/oberdiek/test/pdflscape-test5.tex
%    TDS:doc/latex/oberdiek/test/pdflscape-test6.tex
%    TDS:doc/latex/oberdiek/test/pdflscape-test6.txt
%    TDS:doc/latex/oberdiek/test/ExtractRotate.java
%    TDS:source/latex/oberdiek/pdflscape.dtx
%
%<*ignore>
\begingroup
  \catcode123=1 %
  \catcode125=2 %
  \def\x{LaTeX2e}%
\expandafter\endgroup
\ifcase 0\ifx\install y1\fi\expandafter
         \ifx\csname processbatchFile\endcsname\relax\else1\fi
         \ifx\fmtname\x\else 1\fi\relax
\else\csname fi\endcsname
%</ignore>
%<*install>
\input docstrip.tex
\Msg{************************************************************************}
\Msg{* Installation}
\Msg{* Package: pdflscape 2016/05/14 v0.11 Display of landscape pages in PDF (HO)}
\Msg{************************************************************************}

\keepsilent
\askforoverwritefalse

\let\MetaPrefix\relax
\preamble

This is a generated file.

Project: pdflscape
Version: 2016/05/14 v0.11

Copyright (C) 2001, 2004, 2006-2008 by
   Heiko Oberdiek <heiko.oberdiek at googlemail.com>

This work may be distributed and/or modified under the
conditions of the LaTeX Project Public License, either
version 1.3c of this license or (at your option) any later
version. This version of this license is in
   http://www.latex-project.org/lppl/lppl-1-3c.txt
and the latest version of this license is in
   http://www.latex-project.org/lppl.txt
and version 1.3 or later is part of all distributions of
LaTeX version 2005/12/01 or later.

This work has the LPPL maintenance status "maintained".

This Current Maintainer of this work is Heiko Oberdiek.

This work consists of the main source file pdflscape.dtx
and the derived files
   pdflscape.sty, pdflscape.pdf, pdflscape.ins, pdflscape.drv,
   pdflscape-test1.tex, pdflscape-test2.tex,
   pdflscape-test3.tex, pdflscape-test4.tex,
   pdflscape-test5.tex, pdflscape-test6.tex,
   pdflscape-test6.txt, ExtractRotate.java.

\endpreamble
\let\MetaPrefix\DoubleperCent

\generate{%
  \file{pdflscape.ins}{\from{pdflscape.dtx}{install}}%
  \file{pdflscape.drv}{\from{pdflscape.dtx}{driver}}%
  \usedir{tex/latex/oberdiek}%
  \file{pdflscape.sty}{\from{pdflscape.dtx}{package}}%
%  \usedir{doc/latex/oberdiek/test}%
%  \file{pdflscape-test1.tex}{\from{pdflscape.dtx}{test0,test1}}%
%  \file{pdflscape-test2.tex}{\from{pdflscape.dtx}{test0,test2}}%
%  \file{pdflscape-test3.tex}{\from{pdflscape.dtx}{test0,test3}}%
%  \file{pdflscape-test4.tex}{\from{pdflscape.dtx}{test0,test4}}%
%  \file{pdflscape-test5.tex}{\from{pdflscape.dtx}{test5}}%
%  \file{pdflscape-test6.tex}{\from{pdflscape.dtx}{test6}}%
  \nopreamble
  \nopostamble
%  \file{pdflscape-test6.txt}{\from{pdflscape.dtx}{result6}}%
  \file{ExtractRotate.java}{\from{pdflscape.dtx}{ExtractRotate}}%
%  \usedir{source/latex/oberdiek/catalogue}%
%  \file{pdflscape.xml}{\from{pdflscape.dtx}{catalogue}}%
}

\catcode32=13\relax% active space
\let =\space%
\Msg{************************************************************************}
\Msg{*}
\Msg{* To finish the installation you have to move the following}
\Msg{* file into a directory searched by TeX:}
\Msg{*}
\Msg{*     pdflscape.sty}
\Msg{*}
\Msg{* To produce the documentation run the file `pdflscape.drv'}
\Msg{* through LaTeX.}
\Msg{*}
\Msg{* Happy TeXing!}
\Msg{*}
\Msg{************************************************************************}

\endbatchfile
%</install>
%<*ignore>
\fi
%</ignore>
%<*driver>
\NeedsTeXFormat{LaTeX2e}
\ProvidesFile{pdflscape.drv}%
  [2016/05/14 v0.11 Display of landscape pages in PDF (HO)]%
\documentclass{ltxdoc}
\usepackage{holtxdoc}[2011/11/22]
\usepackage{graphics}
\begin{document}
  \DocInput{pdflscape.dtx}%
\end{document}
%</driver>
% \fi
%
%
% \CharacterTable
%  {Upper-case    \A\B\C\D\E\F\G\H\I\J\K\L\M\N\O\P\Q\R\S\T\U\V\W\X\Y\Z
%   Lower-case    \a\b\c\d\e\f\g\h\i\j\k\l\m\n\o\p\q\r\s\t\u\v\w\x\y\z
%   Digits        \0\1\2\3\4\5\6\7\8\9
%   Exclamation   \!     Double quote  \"     Hash (number) \#
%   Dollar        \$     Percent       \%     Ampersand     \&
%   Acute accent  \'     Left paren    \(     Right paren   \)
%   Asterisk      \*     Plus          \+     Comma         \,
%   Minus         \-     Point         \.     Solidus       \/
%   Colon         \:     Semicolon     \;     Less than     \<
%   Equals        \=     Greater than  \>     Question mark \?
%   Commercial at \@     Left bracket  \[     Backslash     \\
%   Right bracket \]     Circumflex    \^     Underscore    \_
%   Grave accent  \`     Left brace    \{     Vertical bar  \|
%   Right brace   \}     Tilde         \~}
%
% \GetFileInfo{pdflscape.drv}
%
% \title{The \xpackage{pdflscape} package}
% \date{2016/05/14 v0.11}
% \author{Heiko Oberdiek\thanks
% {Please report any issues at \url{https://github.com/ho-tex/oberdiek/issues}}\\
% \xemail{heiko.oberdiek at googlemail.com}}
%
% \maketitle
%
% \begin{abstract}
% Package \xpackage{pdflscape} adds PDF support to the
% environment \texttt{landscape} of package \xpackage{lscape}
% by setting the PDF page attribute \texttt{/Rotate}.
% \end{abstract}
%
% \tableofcontents
%
% \section{Documentation}
%
% \subsection{Supported drivers}
%
% \begin{itemize}
% \item \xoption{pdftex},\xoption{luatex}
% \item \xoption{dvips}, \xoption{dvipsone},
%       \xoption{pctex32}, \xoption{pctexps}
%       (and other drivers that provide a non-empty \cs{Gin@PS@raw})
% \item \xoption{dvipdfm}
% \end{itemize}
%
% \subsection{Caveat}
%
% Depending on the configuration Ghostscript
% adds a guessed rotation entry by its own.
% This can lead to two |/Rotate| entries per page.
% To prevent this behaviour set the parameter
% |AutoRotatePages| to |/None|, eg:
% \begin{quote}
%   |ps2pdf -dAutoRotatePages=/None|
% \end{quote}
%
% \subsection{Requirements}
%
% \begin{itemize}
% \item The package \xpackage{lscape}.
% \item The package \xpackage{atbegshi} for all drivers
%        except \xoption{pdftex},\xoption{luatex}.
% \end{itemize}
%
% \subsection{Usage}
%
% Load this package instead of or after package \xpackage{lscape}:
% \begin{quote}
%   |\usepackage{pdflscape}|\\
%   |\begin{landscape}|\dots|\end{landscape}|
% \end{quote}
%
% \StopEventually{
% }
%
% \section{Implementation}
%
% \subsection{Package identification}
%
%    \begin{macrocode}
%<*package>
%    \end{macrocode}
%    Package identification.
%    \begin{macrocode}
\NeedsTeXFormat{LaTeX2e}
\ProvidesPackage{pdflscape}%
  [2016/05/14 v0.11 Display of landscape pages in PDF (HO)]
%    \end{macrocode}
%
% \subsection{Driver options}
%
%    \begin{macrocode}
\let\PLS@option\@empty
\let\PLS@driver\@empty
\def\PLS@temp#1#2{%
  \DeclareOption{#1}{%
    \let\PLS@option\CurrentOption
    \def\PLS@driver{#2}%
  }%
}
\PLS@temp{pdftex}{pdftex}
\PLS@temp{luatex}{pdftex}
\PLS@temp{dvips}{dvips}
\PLS@temp{dvipsone}{dvips}
\PLS@temp{pctexps}{dvips}
\PLS@temp{dvipdfm}{dvipdfm}
\PLS@temp{dvipdfmx}{dvipdfm}
\PLS@temp{xetex}{dvipdfm}
\let\PLS@temp\@empty
\DeclareOption*{%
  \ifx\PLS@temp\@empty
    \let\PLS@temp\CurrentOption
  \else
    \edef\PLS@temp{\PLS@temp,\CurrentOption}%
  \fi
}
\ProcessOptions*
\ifx\PLS@option\@empty
\else
  \expandafter\PassOptionsToPackage\expandafter{\PLS@option}{lscape}%
\fi
\ifx\PLS@temp\@empty
\else
  \expandafter\PassOptionsToPackage\expandafter{\PLS@temp}{lscape}%
\fi
\RequirePackage{lscape}
%    \end{macrocode}
%
% \subsection{Autodetection of driver}
%
% \subsubsection{Driver \xoption{pdftex}}
%
%    \begin{macrocode}
\RequirePackage{ifpdf}[2006/02/20]
\def\PLS@temp{pdftex}
\ifpdf
  \ifx\PLS@driver\@empty
    \let\PLS@driver\PLS@temp
    \PackageInfo{pdflscape}{Auto-detected driver: \PLS@temp}%
  \fi
  \ifx\PLS@driver\PLS@temp
  \else
    \PackageError{pdflscape}{%
      The driver option `\PLS@option' is not compatible\MessageBreak
      with pdfTeX in PDF mode, using driver `pdftex' instead%
    }\@ehc
    \let\PLS@driver\PLS@temp
  \fi
\fi
%    \end{macrocode}
%
% \subsubsection{Driver \xoption{xetex}}
%
%    \begin{macrocode}
\RequirePackage{ifxetex}%
\def\PLS@temp{dvipdfm}
\ifxetex
  \ifx\PLS@driver\@empty
    \let\PLS@driver\PLS@temp
    \PackageInfo{pdflscape}{%
      Auto-detected driver: \PLS@temp\space(xetex)%
    }%
  \fi
  \ifx\PLS@driver\PLS@temp
  \else
    \PackageError{pdflscape}{%
      The driver option `\PLS@option' is not compatible\MessageBreak
      with XeTeX, using driver for XeTeX (dvipdfm) instead%
    }\@ehc
    \let\PLS@driver\PLS@temp
  \fi
\fi
%    \end{macrocode}
%
% \subsubsection{Detect driver based on \cs{Gin@driver}}
%
%    \begin{macrocode}
\def\PLS@temp#1#2{%
  \ifx\PLS@driver\@empty
    \begingroup
      \def\PLS@temp{#1.def}%
    \expandafter\endgroup
    \ifx\PLS@temp\Gin@driver
      \def\PLS@driver{#2}%
      \PackageInfo{pdflscape}{%
        Auto-detected driver: \PLS@driver\space(#1)%
      }%
    \fi
  \fi
}
\PLS@temp{dvipdfm}{dvipdfm}
\PLS@temp{dvipdfmx}{dvipdfm}
\PLS@temp{dvips}{dvips}
\PLS@temp{dvipsone}{dvips}
\PLS@temp{pctexps}{dvips}
%    \end{macrocode}
%
% \subsubsection{Driver \xoption{dvips}}
%
%    \begin{macrocode}
\ifx\PLS@driver\@empty
  \def\PLS@temp{dvips}%
  \ifnum 1=0\ifx\Gin@PS@raw\@undefined
            \else
              \ifx\Gin@PS@raw\relax
              \else
                \ifx\Gin@PS@raw\@empty
                \else
                  1%
                \fi
              \fi
            \fi
            \space
    \let\PLS@driver\PLS@temp
    \PackageInfo{pdflscape}{%
      Auto-detected driver: \PLS@temp\space(PostScript)%
    }%
  \fi
\fi
%    \end{macrocode}
%
%    \begin{macrocode}
\ifx\PLS@driver\@empty
  \PackageError{pdflscape}{%
    Neither pdfTeX, PostScript, nor dvipdfm driver found%
  }\@ehc
\fi
%    \end{macrocode}
%
% \subsection{Driver implementation}
%
% \begin{itemize}
% \item |\PLS@AddRotate #1|\\
%   it expects the correct rotation number in |#1|
%   and implements the adding of the /Rotation entry
%   in the |/Page| object of the current page.
% \item |\PLS@RemoveRotate|\\
%   it removes a previous |/Rotate| entry, if necessary.
% \end{itemize}
%
% \subsubsection{\pdfTeX}
%
% Not too nice is the global setting of \cs{pdfpageattr}.
% Perhaps this can be changed in future versions.
%
% luatex compat.
%    \begin{macrocode}
\ifx\pdfpageattr\@undefined
  \ifx\pdfvariable\@undefined\else 
    \protected\edef\pdfpageattr{\pdfvariable pageattr}
  \fi
\fi
%    \end{macrocode}
%    \begin{macrocode}
\def\PLS@temp{pdftex}
\ifx\PLS@temp\PLS@driver
  \def\PLS@AddRotate#1{%
    \ifnum#1=0 %
%    \end{macrocode}
%    Already the default.
%    \begin{macrocode}
    \else
      \global\pdfpageattr\expandafter{%
        \the\pdfpageattr
        /Rotate #1%
      }%
    \fi
  }%
%    \end{macrocode}
%    Removes a |/Rotate| entry.
%    \begin{macrocode}
  \def\PLS@RemoveRotate{%
    \begingroup
      \global\pdfpageattr\expandafter{\expandafter}%
      \expandafter\PLS@@RemoveRotate
        \the\pdfpageattr /Rotate\@nil
    \endgroup
  }%
  \def\PLS@@RemoveRotate#1/Rotate#2\@nil{%
%    \end{macrocode}
%    Append |/Rotate| free stuff to \cs{pdfpageattr}.
%    \begin{macrocode}
    \global\pdfpageattr\expandafter{\the\pdfpageattr#1}%
    \ifx\\#2\\%
%    \end{macrocode}
%    Ready, because the detected |/Rotate| is part of
%    the end marker: |/Rotate\@nil|
%    \begin{macrocode}
    \else
%    \end{macrocode}
%    First read in the argument of |/Rotate|,
%    then continue parsing.
%    \begin{macrocode}
      \afterassignment\PLS@@RemoveRotate
      \count0=#2\@nil
    \fi
  }%
\fi
%    \end{macrocode}
%
% \subsubsection{PostScript driver}
%
%    \begin{macrocode}
\def\PLS@temp{dvips}
\ifx\PLS@temp\PLS@driver
  \RequirePackage{atbegshi}%
  \AtBeginShipout{\PLS@AtBeginShipout}%
  \let\PLS@AtBeginShipout\@empty
  \def\PLS@AddRotate#1{%
    \ifnum#1=0 %
    \else
      \def\PLS@AtBeginShipout{%
        \global\setbox\AtBeginShipoutBox\vbox{%
          \Gin@PS@raw{%
            [{ThisPage}\string<\string</Rotate #1\string>\string>%
            /PUT pdfmark%
          }%
          \box\AtBeginShipoutBox
        }%
      }%
    \fi
  }%
  \def\PLS@RemoveRotate{\let\PLS@AtBeginShipout\@empty}%
\fi
%    \end{macrocode}
%
% \subsubsection{Driver dvipdfm}
%
%    \begin{macrocode}
\def\PLS@temp{dvipdfm}
\ifx\PLS@temp\PLS@driver
  \RequirePackage{atbegshi}%
  \AtBeginShipout{\PLS@AtBeginShipout}%
  \let\PLS@AtBeginShipout\@empty
  \def\PLS@AddRotate#1{%
    \ifnum#1=0 %
    \else
      \def\PLS@AtBeginShipout{%
        \global\setbox\AtBeginShipoutBox\vbox{%
          \special{pdf: put @thispage <</Rotate #1>>}%
          \box\AtBeginShipoutBox
        }%
      }%
    \fi
  }%
  \def\PLS@RemoveRotate{\let\PLS@AtBeginShipout\@empty}%
\fi
%    \end{macrocode}
%
% \subsection{Driver independent stuff}
%
%    The landscape environment is extended by
%    adding the correct |/Rotate| entries.
%    \begin{macrocode}
\g@addto@macro{\landscape}{\PLS@Rotate{90}}
\g@addto@macro{\endlandscape}{\PLS@Rotate{0}}
%    \end{macrocode}
%
%    \begin{macro}{\PLS@Rotate}
%    Main macro, that sets the |/Rotate| entry.\\
%    Argument: any TeX number or nothing, that means zero.\\
%    Driver independent.
%    \begin{macrocode}
\def\PLS@Rotate#1{%
  \begingroup
%    \end{macrocode}
%    1. Check and validate the argument.
%    \begin{macrocode}
    \PLS@CheckAngle{#1}%
  \expandafter\endgroup
%    \end{macrocode}
%    2. Remove previous |\Rotate| entry.
%    \begin{macrocode}
  \expandafter\PLS@RemoveRotate
%    \end{macrocode}
%    3. Add |/Rotate| entry.
%    \begin{macrocode}
  \expandafter\PLS@AddRotate\expandafter{\the\count@}%
}
%    \end{macrocode}
%    \end{macro}
%
%    \begin{macro}{\PLS@CheckAngle}
%    Validates the rotation angle.\\
%    The result is stored in the count register |\count@|.\\
%    Driver independent.
%    \begin{macrocode}
\def\PLS@CheckAngle#1{%
%    \end{macrocode}
%    Nothing means zero:
%    \begin{macrocode}
  \ifx\\#1\\%
    \count@=0 %
  \else
    \count@=#1\relax
  \fi
%    \end{macrocode}
%    Normalize to interval -360 $<$ |\count@| $<$ 360.
%    \begin{macrocode}
  \@whilenum\count@>359\do{\advance\count@ -360 }%
  \@whilenum\count@<-359\do{\advance\count@ 360 }%
%    \end{macrocode}
%    Check values: 0, +/-90, +/-180, +/-270
%    \begin{macrocode}
  \ifnum 1=0\ifnum\count@=0 \else
            \ifnum\count@=90 \else
            \ifnum\count@=-90 \else
            \ifnum\count@=180 \else
            \ifnum\count@=-180 \else
            \ifnum\count@=270 \else
            \ifnum\count@=-270 \else
           1\fi\fi\fi\fi\fi\fi\fi\relax
    \PackageError{pdflscape}{%
      Invalid value \the\count@\space for /Rotate%
    }\@ehc
    \count@=0 %
  \fi
}
%    \end{macrocode}
%    \end{macro}
%
%    \begin{macrocode}
%</package>
%    \end{macrocode}
%
% \section{Test}
%
% \subsection{Driver detection tests}
%    \begin{macrocode}
%<*test0>
\NeedsTeXFormat{LaTeX2e}
\documentclass{minimal}
\usepackage{qstest}
\IncludeTests{*}
\LogTests{log}{*}{*}
%</test0>
%    \end{macrocode}
%
%    \begin{macrocode}
%<*test1>
\usepackage{ifpdf}
\ifpdf
  \def\ExpectDriver{pdftex}%
\fi
\usepackage{ifxetex}%
\ifxetex
  \def\ExpectDriver{dvipdfm}%
\fi
\usepackage{pdflscape}[2016/05/14]
%</test1>
%    \end{macrocode}
%
%    \begin{macrocode}
%<*test2>
\def\ExpectDriver{pdftex}
\usepackage[pdftex]{pdflscape}[2016/05/14]
%</test2>
%    \end{macrocode}
%
%    \begin{macrocode}
%<*test3>
\def\ExpectDriver{dvipdfm}
\usepackage[dvipdfmx]{pdflscape}[2016/05/14]
%</test3>
%    \end{macrocode}
%
%    \begin{macrocode}
%<*test4>
\def\ExpectDriver{dvips}
\usepackage[dvipsone]{graphics}
\usepackage{pdflscape}[2016/05/14]
%</test4>
%    \end{macrocode}
%
%    \begin{macrocode}
%<*test0>
\begin{document}
\begin{qstest}{driver}{driver}
  \makeatletter
  \Expect*{\PLS@driver}*{\ExpectDriver}%
\end{qstest}
\end{document}
%</test0>
%    \end{macrocode}
%
% \subsection{Test for \cs{PLS@CheckAngle}}
%
%    \begin{macrocode}
%<*test5>
\NeedsTeXFormat{LaTeX2e}
\documentclass{minimal}
\usepackage{qstest}
\IncludeTests{*}
\LogTests{log}{*}{*}
\usepackage{pdflscape}[2016/05/14]
\begin{document}
\begin{qstest}{checkangle}{checkangle}
  \makeatletter
  \def\Result{%
    \def\ResultString{\the\count@}%
  }%
  \def\PackageError#1#2#3{%
    \def\Result{%
      \def\ResultString{error}%
    }%
  }%
  \def\Test#1#2{%
    \begingroup
      \PLS@CheckAngle{#1}%
      \Result
      \Expect*{\ResultString}{#2}%
    \endgroup
  }%
  \Test{0}{0}%
  \Test{90}{90}%
  \Test{180}{180}%
  \Test{270}{270}%
  \Test{360}{0}%
  \Test{450}{90}%
  \Test{540}{180}%
  \Test{630}{270}%
  \Test{720}{0}%
  \Test{3600}{0}%
  \Test{3690}{90}%
  \Test{-90}{270}%
  \Test{-180}{180}%
  \Test{-270}{-90}%
  \Test{-360}{0}%
  \Test{-450}{270}%
  \Test{540}{180}%
  \Test{-630}{90}%
  \Test{-720}{0}%
  \Test{-3600}{0}%
  \Test{-3690}{270}%
  \Test{1}{error}%
  \Test{-1}{error}%
  \Test{123}{error}%
\end{qstest}
\end{document}
%</test5>
%    \end{macrocode}
%
% \subsection{Test for rotate}
%
%    \begin{macrocode}
%<*test6>
\NeedsTeXFormat{LaTeX2e}
\documentclass{article}
\usepackage{pdflscape}[2016/05/14]
\begin{document}
\section{Portrait page}
\begin{landscape}
\section{Landscape page}
Foo bar foo bar foo bar foo bar foo bar
foo bar foo bar foo bar foo bar foo bar.
\end{landscape}
\section{Portrait Section}
Foo bar
\begin{landscape}
\section{Again Landscape Section}
\newpage
\section{Second Page of Landscape Section}
\end{landscape}
\section{Second Last Portrait Page}
\newpage
\section{Last Portrait Page}
\end{document}
%</test6>
%    \end{macrocode}
%
%    \begin{macrocode}
%<*result6>
/Page 1 /Rotate 0
/Page 2 /Rotate 90
/Page 3 /Rotate 0
/Page 4 /Rotate 90
/Page 5 /Rotate 90
/Page 6 /Rotate 0
/Page 7 /Rotate 0
%</result6>
%    \end{macrocode}
%
%    \begin{macrocode}
%<*ExtractRotate>
/**
 * ExtractRotate.java
 *
 * Copyright (C) 2007 by Heiko Oberdiek <heiko.oberdiek at googlemail.com>
 *
 * Requires: PDFBox (http://www.pdfbox.org/)
 *
 * Syntax: java ExtractRotate <pdffile> <textfile>
 *
 * The <pdffile> is analyzed and for each page its rotation
 * setting is printed in the <textfile>. Example:
 *   /Page 1 /Rotate 0
 *   /Page 2 /Rotate 90
 */
import java.io.FileInputStream;
import java.io.FileWriter;
import org.pdfbox.pdfparser.PDFParser;
import org.pdfbox.pdmodel.PDDocument;
import org.pdfbox.pdmodel.PDDocumentCatalog;
import org.pdfbox.pdmodel.PDPage;

public class ExtractRotate {

    public static void main(String[] args) {
        try {
            String infile = args[0];
            String outfile = args[1];
            FileWriter out = new FileWriter(outfile);
            PDFParser parser =
                    new PDFParser(new FileInputStream(infile));
            parser.parse();
            PDDocument document = parser.getPDDocument();
            PDDocumentCatalog catalog = document.getDocumentCatalog();
            int i = 0;
            for (Object page: catalog.getAllPages()) {
                i++;
                out.write("/Page " + i + " " + "/Rotate "
                        + ((PDPage)page).findRotation() + "\n");
            }
            document.close();
            out.close();
        }
        catch (Exception e) {
            e.printStackTrace();
            System.exit(1);
        }
    }
}
%</ExtractRotate>
%    \end{macrocode}
%
% \section{Installation}
%
% \subsection{Download}
%
% \paragraph{Package.} This package is available on
% CTAN\footnote{\CTANpkg{pdflscape}}:
% \begin{description}
% \item[\CTAN{macros/latex/contrib/oberdiek/pdflscape.dtx}] The source file.
% \item[\CTAN{macros/latex/contrib/oberdiek/pdflscape.pdf}] Documentation.
% \end{description}
%
%
% \paragraph{Bundle.} All the packages of the bundle `oberdiek'
% are also available in a TDS compliant ZIP archive. There
% the packages are already unpacked and the documentation files
% are generated. The files and directories obey the TDS standard.
% \begin{description}
% \item[\CTANinstall{install/macros/latex/contrib/oberdiek.tds.zip}]
% \end{description}
% \emph{TDS} refers to the standard ``A Directory Structure
% for \TeX\ Files'' (\CTAN{tds/tds.pdf}). Directories
% with \xfile{texmf} in their name are usually organized this way.
%
% \subsection{Bundle installation}
%
% \paragraph{Unpacking.} Unpack the \xfile{oberdiek.tds.zip} in the
% TDS tree (also known as \xfile{texmf} tree) of your choice.
% Example (linux):
% \begin{quote}
%   |unzip oberdiek.tds.zip -d ~/texmf|
% \end{quote}
%
% \paragraph{Script installation.}
% Check the directory \xfile{TDS:scripts/oberdiek/} for
% scripts that need further installation steps.
% Package \xpackage{attachfile2} comes with the Perl script
% \xfile{pdfatfi.pl} that should be installed in such a way
% that it can be called as \texttt{pdfatfi}.
% Example (linux):
% \begin{quote}
%   |chmod +x scripts/oberdiek/pdfatfi.pl|\\
%   |cp scripts/oberdiek/pdfatfi.pl /usr/local/bin/|
% \end{quote}
%
% \subsection{Package installation}
%
% \paragraph{Unpacking.} The \xfile{.dtx} file is a self-extracting
% \docstrip\ archive. The files are extracted by running the
% \xfile{.dtx} through \plainTeX:
% \begin{quote}
%   \verb|tex pdflscape.dtx|
% \end{quote}
%
% \paragraph{TDS.} Now the different files must be moved into
% the different directories in your installation TDS tree
% (also known as \xfile{texmf} tree):
% \begin{quote}
% \def\t{^^A
% \begin{tabular}{@{}>{\ttfamily}l@{ $\rightarrow$ }>{\ttfamily}l@{}}
%   pdflscape.sty & tex/latex/oberdiek/pdflscape.sty\\
%   pdflscape.pdf & doc/latex/oberdiek/pdflscape.pdf\\
%   test/pdflscape-test1.tex & doc/latex/oberdiek/test/pdflscape-test1.tex\\
%   test/pdflscape-test2.tex & doc/latex/oberdiek/test/pdflscape-test2.tex\\
%   test/pdflscape-test3.tex & doc/latex/oberdiek/test/pdflscape-test3.tex\\
%   test/pdflscape-test4.tex & doc/latex/oberdiek/test/pdflscape-test4.tex\\
%   test/pdflscape-test5.tex & doc/latex/oberdiek/test/pdflscape-test5.tex\\
%   test/pdflscape-test6.tex & doc/latex/oberdiek/test/pdflscape-test6.tex\\
%   test/pdflscape-test6.txt & doc/latex/oberdiek/test/pdflscape-test6.txt\\
%   test/ExtractRotate.java & doc/latex/oberdiek/test/ExtractRotate.java\\
%   pdflscape.dtx & source/latex/oberdiek/pdflscape.dtx\\
% \end{tabular}^^A
% }^^A
% \sbox0{\t}^^A
% \ifdim\wd0>\linewidth
%   \begingroup
%     \advance\linewidth by\leftmargin
%     \advance\linewidth by\rightmargin
%   \edef\x{\endgroup
%     \def\noexpand\lw{\the\linewidth}^^A
%   }\x
%   \def\lwbox{^^A
%     \leavevmode
%     \hbox to \linewidth{^^A
%       \kern-\leftmargin\relax
%       \hss
%       \usebox0
%       \hss
%       \kern-\rightmargin\relax
%     }^^A
%   }^^A
%   \ifdim\wd0>\lw
%     \sbox0{\small\t}^^A
%     \ifdim\wd0>\linewidth
%       \ifdim\wd0>\lw
%         \sbox0{\footnotesize\t}^^A
%         \ifdim\wd0>\linewidth
%           \ifdim\wd0>\lw
%             \sbox0{\scriptsize\t}^^A
%             \ifdim\wd0>\linewidth
%               \ifdim\wd0>\lw
%                 \sbox0{\tiny\t}^^A
%                 \ifdim\wd0>\linewidth
%                   \lwbox
%                 \else
%                   \usebox0
%                 \fi
%               \else
%                 \lwbox
%               \fi
%             \else
%               \usebox0
%             \fi
%           \else
%             \lwbox
%           \fi
%         \else
%           \usebox0
%         \fi
%       \else
%         \lwbox
%       \fi
%     \else
%       \usebox0
%     \fi
%   \else
%     \lwbox
%   \fi
% \else
%   \usebox0
% \fi
% \end{quote}
% If you have a \xfile{docstrip.cfg} that configures and enables \docstrip's
% TDS installing feature, then some files can already be in the right
% place, see the documentation of \docstrip.
%
% \subsection{Refresh file name databases}
%
% If your \TeX~distribution
% (\teTeX, \mikTeX, \dots) relies on file name databases, you must refresh
% these. For example, \teTeX\ users run \verb|texhash| or
% \verb|mktexlsr|.
%
% \subsection{Some details for the interested}
%
% \paragraph{Attached source.}
%
% The PDF documentation on CTAN also includes the
% \xfile{.dtx} source file. It can be extracted by
% AcrobatReader 6 or higher. Another option is \textsf{pdftk},
% e.g. unpack the file into the current directory:
% \begin{quote}
%   \verb|pdftk pdflscape.pdf unpack_files output .|
% \end{quote}
%
% \paragraph{Unpacking with \LaTeX.}
% The \xfile{.dtx} chooses its action depending on the format:
% \begin{description}
% \item[\plainTeX:] Run \docstrip\ and extract the files.
% \item[\LaTeX:] Generate the documentation.
% \end{description}
% If you insist on using \LaTeX\ for \docstrip\ (really,
% \docstrip\ does not need \LaTeX), then inform the autodetect routine
% about your intention:
% \begin{quote}
%   \verb|latex \let\install=y\input{pdflscape.dtx}|
% \end{quote}
% Do not forget to quote the argument according to the demands
% of your shell.
%
% \paragraph{Generating the documentation.}
% You can use both the \xfile{.dtx} or the \xfile{.drv} to generate
% the documentation. The process can be configured by the
% configuration file \xfile{ltxdoc.cfg}. For instance, put this
% line into this file, if you want to have A4 as paper format:
% \begin{quote}
%   \verb|\PassOptionsToClass{a4paper}{article}|
% \end{quote}
% An example follows how to generate the
% documentation with pdf\LaTeX:
% \begin{quote}
%\begin{verbatim}
%pdflatex pdflscape.dtx
%makeindex -s gind.ist pdflscape.idx
%pdflatex pdflscape.dtx
%makeindex -s gind.ist pdflscape.idx
%pdflatex pdflscape.dtx
%\end{verbatim}
% \end{quote}
%
% \begin{History}
%   \begin{Version}{2001/01/15 v0.1}
%   \item
%     First public version,
%     published in \xnewsgroup{de.comp.text.tex}:\\
%     \URL{``\link{Re: Querformat, pdfLaTex und thumbpdf}''}^^A
%     {https://groups.google.com/group/de.comp.text.tex/msg/e054c5795e52d2b8}
%   \end{Version}
%   \begin{Version}{2001/02/04 v0.2}
%   \item
%     Minor documentation update.
%   \item
%     CTAN.
%   \end{Version}
%   \begin{Version}{2004/05/11 v0.3}
%   \item
%     Support for dvipdfm added.
%   \end{Version}
%   \begin{Version}{2004/05/12 v0.4}
%   \item
%     Bug fix: support for multipage landscape environment
%     for all drivers except for \xoption{pdftex}.
%   \end{Version}
%   \begin{Version}{2006/02/20 v0.5}
%   \item
%     DTX framework.
%   \item
%     LPPL 1.3
%   \item
%     Code is not changed.
%   \end{Version}
%   \begin{Version}{2006/04/24 v0.6}
%   \item
%     Bug fix: \cs{PLS@RemoveRotate} reinserted for \cs{PLS@Rotate}.
%   \item
%     Fix for \cs{PLS@RemoveRotate} (dvips/dvipdfm): \cs{PLS@EverhShi}
%     hook is cleared.
%   \end{Version}
%   \begin{Version}{2007/04/11 v0.7}
%   \item
%     Line ends sanitized.
%   \end{Version}
%   \begin{Version}{2007/04/17 v0.8}
%   \item
%     Package \xpackage{atbegshi} replaces \xpackage{everyshi}.
%   \end{Version}
%   \begin{Version}{2007/10/21 v0.9}
%   \item
%     Driver detection for \XeTeX\ added.
%   \item
%     Fix for rotation angles $\le$~-360 or $\ge$~360.
%   \end{Version}
%   \begin{Version}{2008/08/11 v0.10}
%   \item
%     Code is not changed.
%   \item
%     URLs updated.
%   \end{Version}
%   \begin{Version}{2016/05/14 v0.11}
%   \item
%     LuaTeX Compatibility
%   \end{Version}
% \end{History}
%
% \PrintIndex
%
% \Finale
\endinput
|
% \end{quote}
% Do not forget to quote the argument according to the demands
% of your shell.
%
% \paragraph{Generating the documentation.}
% You can use both the \xfile{.dtx} or the \xfile{.drv} to generate
% the documentation. The process can be configured by the
% configuration file \xfile{ltxdoc.cfg}. For instance, put this
% line into this file, if you want to have A4 as paper format:
% \begin{quote}
%   \verb|\PassOptionsToClass{a4paper}{article}|
% \end{quote}
% An example follows how to generate the
% documentation with pdf\LaTeX:
% \begin{quote}
%\begin{verbatim}
%pdflatex pdflscape.dtx
%makeindex -s gind.ist pdflscape.idx
%pdflatex pdflscape.dtx
%makeindex -s gind.ist pdflscape.idx
%pdflatex pdflscape.dtx
%\end{verbatim}
% \end{quote}
%
% \begin{History}
%   \begin{Version}{2001/01/15 v0.1}
%   \item
%     First public version,
%     published in \xnewsgroup{de.comp.text.tex}:\\
%     \URL{``\link{Re: Querformat, pdfLaTex und thumbpdf}''}^^A
%     {https://groups.google.com/group/de.comp.text.tex/msg/e054c5795e52d2b8}
%   \end{Version}
%   \begin{Version}{2001/02/04 v0.2}
%   \item
%     Minor documentation update.
%   \item
%     CTAN.
%   \end{Version}
%   \begin{Version}{2004/05/11 v0.3}
%   \item
%     Support for dvipdfm added.
%   \end{Version}
%   \begin{Version}{2004/05/12 v0.4}
%   \item
%     Bug fix: support for multipage landscape environment
%     for all drivers except for \xoption{pdftex}.
%   \end{Version}
%   \begin{Version}{2006/02/20 v0.5}
%   \item
%     DTX framework.
%   \item
%     LPPL 1.3
%   \item
%     Code is not changed.
%   \end{Version}
%   \begin{Version}{2006/04/24 v0.6}
%   \item
%     Bug fix: \cs{PLS@RemoveRotate} reinserted for \cs{PLS@Rotate}.
%   \item
%     Fix for \cs{PLS@RemoveRotate} (dvips/dvipdfm): \cs{PLS@EverhShi}
%     hook is cleared.
%   \end{Version}
%   \begin{Version}{2007/04/11 v0.7}
%   \item
%     Line ends sanitized.
%   \end{Version}
%   \begin{Version}{2007/04/17 v0.8}
%   \item
%     Package \xpackage{atbegshi} replaces \xpackage{everyshi}.
%   \end{Version}
%   \begin{Version}{2007/10/21 v0.9}
%   \item
%     Driver detection for \XeTeX\ added.
%   \item
%     Fix for rotation angles $\le$~-360 or $\ge$~360.
%   \end{Version}
%   \begin{Version}{2008/08/11 v0.10}
%   \item
%     Code is not changed.
%   \item
%     URLs updated.
%   \end{Version}
%   \begin{Version}{2016/05/14 v0.11}
%   \item
%     LuaTeX Compatibility
%   \end{Version}
% \end{History}
%
% \PrintIndex
%
% \Finale
\endinput

%        (quote the arguments according to the demands of your shell)
%
% Documentation:
%    (a) If pdflscape.drv is present:
%           latex pdflscape.drv
%    (b) Without pdflscape.drv:
%           latex pdflscape.dtx; ...
%    The class ltxdoc loads the configuration file ltxdoc.cfg
%    if available. Here you can specify further options, e.g.
%    use A4 as paper format:
%       \PassOptionsToClass{a4paper}{article}
%
%    Programm calls to get the documentation (example):
%       pdflatex pdflscape.dtx
%       makeindex -s gind.ist pdflscape.idx
%       pdflatex pdflscape.dtx
%       makeindex -s gind.ist pdflscape.idx
%       pdflatex pdflscape.dtx
%
% Installation:
%    TDS:tex/latex/oberdiek/pdflscape.sty
%    TDS:doc/latex/oberdiek/pdflscape.pdf
%    TDS:doc/latex/oberdiek/test/pdflscape-test1.tex
%    TDS:doc/latex/oberdiek/test/pdflscape-test2.tex
%    TDS:doc/latex/oberdiek/test/pdflscape-test3.tex
%    TDS:doc/latex/oberdiek/test/pdflscape-test4.tex
%    TDS:doc/latex/oberdiek/test/pdflscape-test5.tex
%    TDS:doc/latex/oberdiek/test/pdflscape-test6.tex
%    TDS:doc/latex/oberdiek/test/pdflscape-test6.txt
%    TDS:doc/latex/oberdiek/test/ExtractRotate.java
%    TDS:source/latex/oberdiek/pdflscape.dtx
%
%<*ignore>
\begingroup
  \catcode123=1 %
  \catcode125=2 %
  \def\x{LaTeX2e}%
\expandafter\endgroup
\ifcase 0\ifx\install y1\fi\expandafter
         \ifx\csname processbatchFile\endcsname\relax\else1\fi
         \ifx\fmtname\x\else 1\fi\relax
\else\csname fi\endcsname
%</ignore>
%<*install>
\input docstrip.tex
\Msg{************************************************************************}
\Msg{* Installation}
\Msg{* Package: pdflscape 2016/05/14 v0.11 Display of landscape pages in PDF (HO)}
\Msg{************************************************************************}

\keepsilent
\askforoverwritefalse

\let\MetaPrefix\relax
\preamble

This is a generated file.

Project: pdflscape
Version: 2016/05/14 v0.11

Copyright (C) 2001, 2004, 2006-2008 by
   Heiko Oberdiek <heiko.oberdiek at googlemail.com>

This work may be distributed and/or modified under the
conditions of the LaTeX Project Public License, either
version 1.3c of this license or (at your option) any later
version. This version of this license is in
   http://www.latex-project.org/lppl/lppl-1-3c.txt
and the latest version of this license is in
   http://www.latex-project.org/lppl.txt
and version 1.3 or later is part of all distributions of
LaTeX version 2005/12/01 or later.

This work has the LPPL maintenance status "maintained".

This Current Maintainer of this work is Heiko Oberdiek.

This work consists of the main source file pdflscape.dtx
and the derived files
   pdflscape.sty, pdflscape.pdf, pdflscape.ins, pdflscape.drv,
   pdflscape-test1.tex, pdflscape-test2.tex,
   pdflscape-test3.tex, pdflscape-test4.tex,
   pdflscape-test5.tex, pdflscape-test6.tex,
   pdflscape-test6.txt, ExtractRotate.java.

\endpreamble
\let\MetaPrefix\DoubleperCent

\generate{%
  \file{pdflscape.ins}{\from{pdflscape.dtx}{install}}%
  \file{pdflscape.drv}{\from{pdflscape.dtx}{driver}}%
  \usedir{tex/latex/oberdiek}%
  \file{pdflscape.sty}{\from{pdflscape.dtx}{package}}%
%  \usedir{doc/latex/oberdiek/test}%
%  \file{pdflscape-test1.tex}{\from{pdflscape.dtx}{test0,test1}}%
%  \file{pdflscape-test2.tex}{\from{pdflscape.dtx}{test0,test2}}%
%  \file{pdflscape-test3.tex}{\from{pdflscape.dtx}{test0,test3}}%
%  \file{pdflscape-test4.tex}{\from{pdflscape.dtx}{test0,test4}}%
%  \file{pdflscape-test5.tex}{\from{pdflscape.dtx}{test5}}%
%  \file{pdflscape-test6.tex}{\from{pdflscape.dtx}{test6}}%
  \nopreamble
  \nopostamble
%  \file{pdflscape-test6.txt}{\from{pdflscape.dtx}{result6}}%
  \file{ExtractRotate.java}{\from{pdflscape.dtx}{ExtractRotate}}%
%  \usedir{source/latex/oberdiek/catalogue}%
%  \file{pdflscape.xml}{\from{pdflscape.dtx}{catalogue}}%
}

\catcode32=13\relax% active space
\let =\space%
\Msg{************************************************************************}
\Msg{*}
\Msg{* To finish the installation you have to move the following}
\Msg{* file into a directory searched by TeX:}
\Msg{*}
\Msg{*     pdflscape.sty}
\Msg{*}
\Msg{* To produce the documentation run the file `pdflscape.drv'}
\Msg{* through LaTeX.}
\Msg{*}
\Msg{* Happy TeXing!}
\Msg{*}
\Msg{************************************************************************}

\endbatchfile
%</install>
%<*ignore>
\fi
%</ignore>
%<*driver>
\NeedsTeXFormat{LaTeX2e}
\ProvidesFile{pdflscape.drv}%
  [2016/05/14 v0.11 Display of landscape pages in PDF (HO)]%
\documentclass{ltxdoc}
\usepackage{holtxdoc}[2011/11/22]
\usepackage{graphics}
\begin{document}
  \DocInput{pdflscape.dtx}%
\end{document}
%</driver>
% \fi
%
%
% \CharacterTable
%  {Upper-case    \A\B\C\D\E\F\G\H\I\J\K\L\M\N\O\P\Q\R\S\T\U\V\W\X\Y\Z
%   Lower-case    \a\b\c\d\e\f\g\h\i\j\k\l\m\n\o\p\q\r\s\t\u\v\w\x\y\z
%   Digits        \0\1\2\3\4\5\6\7\8\9
%   Exclamation   \!     Double quote  \"     Hash (number) \#
%   Dollar        \$     Percent       \%     Ampersand     \&
%   Acute accent  \'     Left paren    \(     Right paren   \)
%   Asterisk      \*     Plus          \+     Comma         \,
%   Minus         \-     Point         \.     Solidus       \/
%   Colon         \:     Semicolon     \;     Less than     \<
%   Equals        \=     Greater than  \>     Question mark \?
%   Commercial at \@     Left bracket  \[     Backslash     \\
%   Right bracket \]     Circumflex    \^     Underscore    \_
%   Grave accent  \`     Left brace    \{     Vertical bar  \|
%   Right brace   \}     Tilde         \~}
%
% \GetFileInfo{pdflscape.drv}
%
% \title{The \xpackage{pdflscape} package}
% \date{2016/05/14 v0.11}
% \author{Heiko Oberdiek\thanks
% {Please report any issues at \url{https://github.com/ho-tex/oberdiek/issues}}\\
% \xemail{heiko.oberdiek at googlemail.com}}
%
% \maketitle
%
% \begin{abstract}
% Package \xpackage{pdflscape} adds PDF support to the
% environment \texttt{landscape} of package \xpackage{lscape}
% by setting the PDF page attribute \texttt{/Rotate}.
% \end{abstract}
%
% \tableofcontents
%
% \section{Documentation}
%
% \subsection{Supported drivers}
%
% \begin{itemize}
% \item \xoption{pdftex},\xoption{luatex}
% \item \xoption{dvips}, \xoption{dvipsone},
%       \xoption{pctex32}, \xoption{pctexps}
%       (and other drivers that provide a non-empty \cs{Gin@PS@raw})
% \item \xoption{dvipdfm}
% \end{itemize}
%
% \subsection{Caveat}
%
% Depending on the configuration Ghostscript
% adds a guessed rotation entry by its own.
% This can lead to two |/Rotate| entries per page.
% To prevent this behaviour set the parameter
% |AutoRotatePages| to |/None|, eg:
% \begin{quote}
%   |ps2pdf -dAutoRotatePages=/None|
% \end{quote}
%
% \subsection{Requirements}
%
% \begin{itemize}
% \item The package \xpackage{lscape}.
% \item The package \xpackage{atbegshi} for all drivers
%        except \xoption{pdftex},\xoption{luatex}.
% \end{itemize}
%
% \subsection{Usage}
%
% Load this package instead of or after package \xpackage{lscape}:
% \begin{quote}
%   |\usepackage{pdflscape}|\\
%   |\begin{landscape}|\dots|\end{landscape}|
% \end{quote}
%
% \StopEventually{
% }
%
% \section{Implementation}
%
% \subsection{Package identification}
%
%    \begin{macrocode}
%<*package>
%    \end{macrocode}
%    Package identification.
%    \begin{macrocode}
\NeedsTeXFormat{LaTeX2e}
\ProvidesPackage{pdflscape}%
  [2016/05/14 v0.11 Display of landscape pages in PDF (HO)]
%    \end{macrocode}
%
% \subsection{Driver options}
%
%    \begin{macrocode}
\let\PLS@option\@empty
\let\PLS@driver\@empty
\def\PLS@temp#1#2{%
  \DeclareOption{#1}{%
    \let\PLS@option\CurrentOption
    \def\PLS@driver{#2}%
  }%
}
\PLS@temp{pdftex}{pdftex}
\PLS@temp{luatex}{pdftex}
\PLS@temp{dvips}{dvips}
\PLS@temp{dvipsone}{dvips}
\PLS@temp{pctexps}{dvips}
\PLS@temp{dvipdfm}{dvipdfm}
\PLS@temp{dvipdfmx}{dvipdfm}
\PLS@temp{xetex}{dvipdfm}
\let\PLS@temp\@empty
\DeclareOption*{%
  \ifx\PLS@temp\@empty
    \let\PLS@temp\CurrentOption
  \else
    \edef\PLS@temp{\PLS@temp,\CurrentOption}%
  \fi
}
\ProcessOptions*
\ifx\PLS@option\@empty
\else
  \expandafter\PassOptionsToPackage\expandafter{\PLS@option}{lscape}%
\fi
\ifx\PLS@temp\@empty
\else
  \expandafter\PassOptionsToPackage\expandafter{\PLS@temp}{lscape}%
\fi
\RequirePackage{lscape}
%    \end{macrocode}
%
% \subsection{Autodetection of driver}
%
% \subsubsection{Driver \xoption{pdftex}}
%
%    \begin{macrocode}
\RequirePackage{ifpdf}[2006/02/20]
\def\PLS@temp{pdftex}
\ifpdf
  \ifx\PLS@driver\@empty
    \let\PLS@driver\PLS@temp
    \PackageInfo{pdflscape}{Auto-detected driver: \PLS@temp}%
  \fi
  \ifx\PLS@driver\PLS@temp
  \else
    \PackageError{pdflscape}{%
      The driver option `\PLS@option' is not compatible\MessageBreak
      with pdfTeX in PDF mode, using driver `pdftex' instead%
    }\@ehc
    \let\PLS@driver\PLS@temp
  \fi
\fi
%    \end{macrocode}
%
% \subsubsection{Driver \xoption{xetex}}
%
%    \begin{macrocode}
\RequirePackage{ifxetex}%
\def\PLS@temp{dvipdfm}
\ifxetex
  \ifx\PLS@driver\@empty
    \let\PLS@driver\PLS@temp
    \PackageInfo{pdflscape}{%
      Auto-detected driver: \PLS@temp\space(xetex)%
    }%
  \fi
  \ifx\PLS@driver\PLS@temp
  \else
    \PackageError{pdflscape}{%
      The driver option `\PLS@option' is not compatible\MessageBreak
      with XeTeX, using driver for XeTeX (dvipdfm) instead%
    }\@ehc
    \let\PLS@driver\PLS@temp
  \fi
\fi
%    \end{macrocode}
%
% \subsubsection{Detect driver based on \cs{Gin@driver}}
%
%    \begin{macrocode}
\def\PLS@temp#1#2{%
  \ifx\PLS@driver\@empty
    \begingroup
      \def\PLS@temp{#1.def}%
    \expandafter\endgroup
    \ifx\PLS@temp\Gin@driver
      \def\PLS@driver{#2}%
      \PackageInfo{pdflscape}{%
        Auto-detected driver: \PLS@driver\space(#1)%
      }%
    \fi
  \fi
}
\PLS@temp{dvipdfm}{dvipdfm}
\PLS@temp{dvipdfmx}{dvipdfm}
\PLS@temp{dvips}{dvips}
\PLS@temp{dvipsone}{dvips}
\PLS@temp{pctexps}{dvips}
%    \end{macrocode}
%
% \subsubsection{Driver \xoption{dvips}}
%
%    \begin{macrocode}
\ifx\PLS@driver\@empty
  \def\PLS@temp{dvips}%
  \ifnum 1=0\ifx\Gin@PS@raw\@undefined
            \else
              \ifx\Gin@PS@raw\relax
              \else
                \ifx\Gin@PS@raw\@empty
                \else
                  1%
                \fi
              \fi
            \fi
            \space
    \let\PLS@driver\PLS@temp
    \PackageInfo{pdflscape}{%
      Auto-detected driver: \PLS@temp\space(PostScript)%
    }%
  \fi
\fi
%    \end{macrocode}
%
%    \begin{macrocode}
\ifx\PLS@driver\@empty
  \PackageError{pdflscape}{%
    Neither pdfTeX, PostScript, nor dvipdfm driver found%
  }\@ehc
\fi
%    \end{macrocode}
%
% \subsection{Driver implementation}
%
% \begin{itemize}
% \item |\PLS@AddRotate #1|\\
%   it expects the correct rotation number in |#1|
%   and implements the adding of the /Rotation entry
%   in the |/Page| object of the current page.
% \item |\PLS@RemoveRotate|\\
%   it removes a previous |/Rotate| entry, if necessary.
% \end{itemize}
%
% \subsubsection{\pdfTeX}
%
% Not too nice is the global setting of \cs{pdfpageattr}.
% Perhaps this can be changed in future versions.
%
% luatex compat.
%    \begin{macrocode}
\ifx\pdfpageattr\@undefined
  \ifx\pdfvariable\@undefined\else 
    \protected\edef\pdfpageattr{\pdfvariable pageattr}
  \fi
\fi
%    \end{macrocode}
%    \begin{macrocode}
\def\PLS@temp{pdftex}
\ifx\PLS@temp\PLS@driver
  \def\PLS@AddRotate#1{%
    \ifnum#1=0 %
%    \end{macrocode}
%    Already the default.
%    \begin{macrocode}
    \else
      \global\pdfpageattr\expandafter{%
        \the\pdfpageattr
        /Rotate #1%
      }%
    \fi
  }%
%    \end{macrocode}
%    Removes a |/Rotate| entry.
%    \begin{macrocode}
  \def\PLS@RemoveRotate{%
    \begingroup
      \global\pdfpageattr\expandafter{\expandafter}%
      \expandafter\PLS@@RemoveRotate
        \the\pdfpageattr /Rotate\@nil
    \endgroup
  }%
  \def\PLS@@RemoveRotate#1/Rotate#2\@nil{%
%    \end{macrocode}
%    Append |/Rotate| free stuff to \cs{pdfpageattr}.
%    \begin{macrocode}
    \global\pdfpageattr\expandafter{\the\pdfpageattr#1}%
    \ifx\\#2\\%
%    \end{macrocode}
%    Ready, because the detected |/Rotate| is part of
%    the end marker: |/Rotate\@nil|
%    \begin{macrocode}
    \else
%    \end{macrocode}
%    First read in the argument of |/Rotate|,
%    then continue parsing.
%    \begin{macrocode}
      \afterassignment\PLS@@RemoveRotate
      \count0=#2\@nil
    \fi
  }%
\fi
%    \end{macrocode}
%
% \subsubsection{PostScript driver}
%
%    \begin{macrocode}
\def\PLS@temp{dvips}
\ifx\PLS@temp\PLS@driver
  \RequirePackage{atbegshi}%
  \AtBeginShipout{\PLS@AtBeginShipout}%
  \let\PLS@AtBeginShipout\@empty
  \def\PLS@AddRotate#1{%
    \ifnum#1=0 %
    \else
      \def\PLS@AtBeginShipout{%
        \global\setbox\AtBeginShipoutBox\vbox{%
          \Gin@PS@raw{%
            [{ThisPage}\string<\string</Rotate #1\string>\string>%
            /PUT pdfmark%
          }%
          \box\AtBeginShipoutBox
        }%
      }%
    \fi
  }%
  \def\PLS@RemoveRotate{\let\PLS@AtBeginShipout\@empty}%
\fi
%    \end{macrocode}
%
% \subsubsection{Driver dvipdfm}
%
%    \begin{macrocode}
\def\PLS@temp{dvipdfm}
\ifx\PLS@temp\PLS@driver
  \RequirePackage{atbegshi}%
  \AtBeginShipout{\PLS@AtBeginShipout}%
  \let\PLS@AtBeginShipout\@empty
  \def\PLS@AddRotate#1{%
    \ifnum#1=0 %
    \else
      \def\PLS@AtBeginShipout{%
        \global\setbox\AtBeginShipoutBox\vbox{%
          \special{pdf: put @thispage <</Rotate #1>>}%
          \box\AtBeginShipoutBox
        }%
      }%
    \fi
  }%
  \def\PLS@RemoveRotate{\let\PLS@AtBeginShipout\@empty}%
\fi
%    \end{macrocode}
%
% \subsection{Driver independent stuff}
%
%    The landscape environment is extended by
%    adding the correct |/Rotate| entries.
%    \begin{macrocode}
\g@addto@macro{\landscape}{\PLS@Rotate{90}}
\g@addto@macro{\endlandscape}{\PLS@Rotate{0}}
%    \end{macrocode}
%
%    \begin{macro}{\PLS@Rotate}
%    Main macro, that sets the |/Rotate| entry.\\
%    Argument: any TeX number or nothing, that means zero.\\
%    Driver independent.
%    \begin{macrocode}
\def\PLS@Rotate#1{%
  \begingroup
%    \end{macrocode}
%    1. Check and validate the argument.
%    \begin{macrocode}
    \PLS@CheckAngle{#1}%
  \expandafter\endgroup
%    \end{macrocode}
%    2. Remove previous |\Rotate| entry.
%    \begin{macrocode}
  \expandafter\PLS@RemoveRotate
%    \end{macrocode}
%    3. Add |/Rotate| entry.
%    \begin{macrocode}
  \expandafter\PLS@AddRotate\expandafter{\the\count@}%
}
%    \end{macrocode}
%    \end{macro}
%
%    \begin{macro}{\PLS@CheckAngle}
%    Validates the rotation angle.\\
%    The result is stored in the count register |\count@|.\\
%    Driver independent.
%    \begin{macrocode}
\def\PLS@CheckAngle#1{%
%    \end{macrocode}
%    Nothing means zero:
%    \begin{macrocode}
  \ifx\\#1\\%
    \count@=0 %
  \else
    \count@=#1\relax
  \fi
%    \end{macrocode}
%    Normalize to interval -360 $<$ |\count@| $<$ 360.
%    \begin{macrocode}
  \@whilenum\count@>359\do{\advance\count@ -360 }%
  \@whilenum\count@<-359\do{\advance\count@ 360 }%
%    \end{macrocode}
%    Check values: 0, +/-90, +/-180, +/-270
%    \begin{macrocode}
  \ifnum 1=0\ifnum\count@=0 \else
            \ifnum\count@=90 \else
            \ifnum\count@=-90 \else
            \ifnum\count@=180 \else
            \ifnum\count@=-180 \else
            \ifnum\count@=270 \else
            \ifnum\count@=-270 \else
           1\fi\fi\fi\fi\fi\fi\fi\relax
    \PackageError{pdflscape}{%
      Invalid value \the\count@\space for /Rotate%
    }\@ehc
    \count@=0 %
  \fi
}
%    \end{macrocode}
%    \end{macro}
%
%    \begin{macrocode}
%</package>
%    \end{macrocode}
%
% \section{Test}
%
% \subsection{Driver detection tests}
%    \begin{macrocode}
%<*test0>
\NeedsTeXFormat{LaTeX2e}
\documentclass{minimal}
\usepackage{qstest}
\IncludeTests{*}
\LogTests{log}{*}{*}
%</test0>
%    \end{macrocode}
%
%    \begin{macrocode}
%<*test1>
\usepackage{ifpdf}
\ifpdf
  \def\ExpectDriver{pdftex}%
\fi
\usepackage{ifxetex}%
\ifxetex
  \def\ExpectDriver{dvipdfm}%
\fi
\usepackage{pdflscape}[2016/05/14]
%</test1>
%    \end{macrocode}
%
%    \begin{macrocode}
%<*test2>
\def\ExpectDriver{pdftex}
\usepackage[pdftex]{pdflscape}[2016/05/14]
%</test2>
%    \end{macrocode}
%
%    \begin{macrocode}
%<*test3>
\def\ExpectDriver{dvipdfm}
\usepackage[dvipdfmx]{pdflscape}[2016/05/14]
%</test3>
%    \end{macrocode}
%
%    \begin{macrocode}
%<*test4>
\def\ExpectDriver{dvips}
\usepackage[dvipsone]{graphics}
\usepackage{pdflscape}[2016/05/14]
%</test4>
%    \end{macrocode}
%
%    \begin{macrocode}
%<*test0>
\begin{document}
\begin{qstest}{driver}{driver}
  \makeatletter
  \Expect*{\PLS@driver}*{\ExpectDriver}%
\end{qstest}
\end{document}
%</test0>
%    \end{macrocode}
%
% \subsection{Test for \cs{PLS@CheckAngle}}
%
%    \begin{macrocode}
%<*test5>
\NeedsTeXFormat{LaTeX2e}
\documentclass{minimal}
\usepackage{qstest}
\IncludeTests{*}
\LogTests{log}{*}{*}
\usepackage{pdflscape}[2016/05/14]
\begin{document}
\begin{qstest}{checkangle}{checkangle}
  \makeatletter
  \def\Result{%
    \def\ResultString{\the\count@}%
  }%
  \def\PackageError#1#2#3{%
    \def\Result{%
      \def\ResultString{error}%
    }%
  }%
  \def\Test#1#2{%
    \begingroup
      \PLS@CheckAngle{#1}%
      \Result
      \Expect*{\ResultString}{#2}%
    \endgroup
  }%
  \Test{0}{0}%
  \Test{90}{90}%
  \Test{180}{180}%
  \Test{270}{270}%
  \Test{360}{0}%
  \Test{450}{90}%
  \Test{540}{180}%
  \Test{630}{270}%
  \Test{720}{0}%
  \Test{3600}{0}%
  \Test{3690}{90}%
  \Test{-90}{270}%
  \Test{-180}{180}%
  \Test{-270}{-90}%
  \Test{-360}{0}%
  \Test{-450}{270}%
  \Test{540}{180}%
  \Test{-630}{90}%
  \Test{-720}{0}%
  \Test{-3600}{0}%
  \Test{-3690}{270}%
  \Test{1}{error}%
  \Test{-1}{error}%
  \Test{123}{error}%
\end{qstest}
\end{document}
%</test5>
%    \end{macrocode}
%
% \subsection{Test for rotate}
%
%    \begin{macrocode}
%<*test6>
\NeedsTeXFormat{LaTeX2e}
\documentclass{article}
\usepackage{pdflscape}[2016/05/14]
\begin{document}
\section{Portrait page}
\begin{landscape}
\section{Landscape page}
Foo bar foo bar foo bar foo bar foo bar
foo bar foo bar foo bar foo bar foo bar.
\end{landscape}
\section{Portrait Section}
Foo bar
\begin{landscape}
\section{Again Landscape Section}
\newpage
\section{Second Page of Landscape Section}
\end{landscape}
\section{Second Last Portrait Page}
\newpage
\section{Last Portrait Page}
\end{document}
%</test6>
%    \end{macrocode}
%
%    \begin{macrocode}
%<*result6>
/Page 1 /Rotate 0
/Page 2 /Rotate 90
/Page 3 /Rotate 0
/Page 4 /Rotate 90
/Page 5 /Rotate 90
/Page 6 /Rotate 0
/Page 7 /Rotate 0
%</result6>
%    \end{macrocode}
%
%    \begin{macrocode}
%<*ExtractRotate>
/**
 * ExtractRotate.java
 *
 * Copyright (C) 2007 by Heiko Oberdiek <heiko.oberdiek at googlemail.com>
 *
 * Requires: PDFBox (http://www.pdfbox.org/)
 *
 * Syntax: java ExtractRotate <pdffile> <textfile>
 *
 * The <pdffile> is analyzed and for each page its rotation
 * setting is printed in the <textfile>. Example:
 *   /Page 1 /Rotate 0
 *   /Page 2 /Rotate 90
 */
import java.io.FileInputStream;
import java.io.FileWriter;
import org.pdfbox.pdfparser.PDFParser;
import org.pdfbox.pdmodel.PDDocument;
import org.pdfbox.pdmodel.PDDocumentCatalog;
import org.pdfbox.pdmodel.PDPage;

public class ExtractRotate {

    public static void main(String[] args) {
        try {
            String infile = args[0];
            String outfile = args[1];
            FileWriter out = new FileWriter(outfile);
            PDFParser parser =
                    new PDFParser(new FileInputStream(infile));
            parser.parse();
            PDDocument document = parser.getPDDocument();
            PDDocumentCatalog catalog = document.getDocumentCatalog();
            int i = 0;
            for (Object page: catalog.getAllPages()) {
                i++;
                out.write("/Page " + i + " " + "/Rotate "
                        + ((PDPage)page).findRotation() + "\n");
            }
            document.close();
            out.close();
        }
        catch (Exception e) {
            e.printStackTrace();
            System.exit(1);
        }
    }
}
%</ExtractRotate>
%    \end{macrocode}
%
% \section{Installation}
%
% \subsection{Download}
%
% \paragraph{Package.} This package is available on
% CTAN\footnote{\CTANpkg{pdflscape}}:
% \begin{description}
% \item[\CTAN{macros/latex/contrib/oberdiek/pdflscape.dtx}] The source file.
% \item[\CTAN{macros/latex/contrib/oberdiek/pdflscape.pdf}] Documentation.
% \end{description}
%
%
% \paragraph{Bundle.} All the packages of the bundle `oberdiek'
% are also available in a TDS compliant ZIP archive. There
% the packages are already unpacked and the documentation files
% are generated. The files and directories obey the TDS standard.
% \begin{description}
% \item[\CTANinstall{install/macros/latex/contrib/oberdiek.tds.zip}]
% \end{description}
% \emph{TDS} refers to the standard ``A Directory Structure
% for \TeX\ Files'' (\CTAN{tds/tds.pdf}). Directories
% with \xfile{texmf} in their name are usually organized this way.
%
% \subsection{Bundle installation}
%
% \paragraph{Unpacking.} Unpack the \xfile{oberdiek.tds.zip} in the
% TDS tree (also known as \xfile{texmf} tree) of your choice.
% Example (linux):
% \begin{quote}
%   |unzip oberdiek.tds.zip -d ~/texmf|
% \end{quote}
%
% \paragraph{Script installation.}
% Check the directory \xfile{TDS:scripts/oberdiek/} for
% scripts that need further installation steps.
% Package \xpackage{attachfile2} comes with the Perl script
% \xfile{pdfatfi.pl} that should be installed in such a way
% that it can be called as \texttt{pdfatfi}.
% Example (linux):
% \begin{quote}
%   |chmod +x scripts/oberdiek/pdfatfi.pl|\\
%   |cp scripts/oberdiek/pdfatfi.pl /usr/local/bin/|
% \end{quote}
%
% \subsection{Package installation}
%
% \paragraph{Unpacking.} The \xfile{.dtx} file is a self-extracting
% \docstrip\ archive. The files are extracted by running the
% \xfile{.dtx} through \plainTeX:
% \begin{quote}
%   \verb|tex pdflscape.dtx|
% \end{quote}
%
% \paragraph{TDS.} Now the different files must be moved into
% the different directories in your installation TDS tree
% (also known as \xfile{texmf} tree):
% \begin{quote}
% \def\t{^^A
% \begin{tabular}{@{}>{\ttfamily}l@{ $\rightarrow$ }>{\ttfamily}l@{}}
%   pdflscape.sty & tex/latex/oberdiek/pdflscape.sty\\
%   pdflscape.pdf & doc/latex/oberdiek/pdflscape.pdf\\
%   test/pdflscape-test1.tex & doc/latex/oberdiek/test/pdflscape-test1.tex\\
%   test/pdflscape-test2.tex & doc/latex/oberdiek/test/pdflscape-test2.tex\\
%   test/pdflscape-test3.tex & doc/latex/oberdiek/test/pdflscape-test3.tex\\
%   test/pdflscape-test4.tex & doc/latex/oberdiek/test/pdflscape-test4.tex\\
%   test/pdflscape-test5.tex & doc/latex/oberdiek/test/pdflscape-test5.tex\\
%   test/pdflscape-test6.tex & doc/latex/oberdiek/test/pdflscape-test6.tex\\
%   test/pdflscape-test6.txt & doc/latex/oberdiek/test/pdflscape-test6.txt\\
%   test/ExtractRotate.java & doc/latex/oberdiek/test/ExtractRotate.java\\
%   pdflscape.dtx & source/latex/oberdiek/pdflscape.dtx\\
% \end{tabular}^^A
% }^^A
% \sbox0{\t}^^A
% \ifdim\wd0>\linewidth
%   \begingroup
%     \advance\linewidth by\leftmargin
%     \advance\linewidth by\rightmargin
%   \edef\x{\endgroup
%     \def\noexpand\lw{\the\linewidth}^^A
%   }\x
%   \def\lwbox{^^A
%     \leavevmode
%     \hbox to \linewidth{^^A
%       \kern-\leftmargin\relax
%       \hss
%       \usebox0
%       \hss
%       \kern-\rightmargin\relax
%     }^^A
%   }^^A
%   \ifdim\wd0>\lw
%     \sbox0{\small\t}^^A
%     \ifdim\wd0>\linewidth
%       \ifdim\wd0>\lw
%         \sbox0{\footnotesize\t}^^A
%         \ifdim\wd0>\linewidth
%           \ifdim\wd0>\lw
%             \sbox0{\scriptsize\t}^^A
%             \ifdim\wd0>\linewidth
%               \ifdim\wd0>\lw
%                 \sbox0{\tiny\t}^^A
%                 \ifdim\wd0>\linewidth
%                   \lwbox
%                 \else
%                   \usebox0
%                 \fi
%               \else
%                 \lwbox
%               \fi
%             \else
%               \usebox0
%             \fi
%           \else
%             \lwbox
%           \fi
%         \else
%           \usebox0
%         \fi
%       \else
%         \lwbox
%       \fi
%     \else
%       \usebox0
%     \fi
%   \else
%     \lwbox
%   \fi
% \else
%   \usebox0
% \fi
% \end{quote}
% If you have a \xfile{docstrip.cfg} that configures and enables \docstrip's
% TDS installing feature, then some files can already be in the right
% place, see the documentation of \docstrip.
%
% \subsection{Refresh file name databases}
%
% If your \TeX~distribution
% (\teTeX, \mikTeX, \dots) relies on file name databases, you must refresh
% these. For example, \teTeX\ users run \verb|texhash| or
% \verb|mktexlsr|.
%
% \subsection{Some details for the interested}
%
% \paragraph{Attached source.}
%
% The PDF documentation on CTAN also includes the
% \xfile{.dtx} source file. It can be extracted by
% AcrobatReader 6 or higher. Another option is \textsf{pdftk},
% e.g. unpack the file into the current directory:
% \begin{quote}
%   \verb|pdftk pdflscape.pdf unpack_files output .|
% \end{quote}
%
% \paragraph{Unpacking with \LaTeX.}
% The \xfile{.dtx} chooses its action depending on the format:
% \begin{description}
% \item[\plainTeX:] Run \docstrip\ and extract the files.
% \item[\LaTeX:] Generate the documentation.
% \end{description}
% If you insist on using \LaTeX\ for \docstrip\ (really,
% \docstrip\ does not need \LaTeX), then inform the autodetect routine
% about your intention:
% \begin{quote}
%   \verb|latex \let\install=y% \iffalse meta-comment
%
% File: pdflscape.dtx
% Version: 2016/05/14 v0.11
% Info: Display of landscape pages in PDF
%
% Copyright (C) 2001, 2004, 2006-2008 by
%    Heiko Oberdiek <heiko.oberdiek at googlemail.com>
%    2016
%    https://github.com/ho-tex/oberdiek/issues
%
% This work may be distributed and/or modified under the
% conditions of the LaTeX Project Public License, either
% version 1.3c of this license or (at your option) any later
% version. This version of this license is in
%    http://www.latex-project.org/lppl/lppl-1-3c.txt
% and the latest version of this license is in
%    http://www.latex-project.org/lppl.txt
% and version 1.3 or later is part of all distributions of
% LaTeX version 2005/12/01 or later.
%
% This work has the LPPL maintenance status "maintained".
%
% This Current Maintainer of this work is Heiko Oberdiek.
%
% This work consists of the main source file pdflscape.dtx
% and the derived files
%    pdflscape.sty, pdflscape.pdf, pdflscape.ins, pdflscape.drv,
%    pdflscape-test1.tex, pdflscape-test2.tex,
%    pdflscape-test3.tex, pdflscape-test4.tex,
%    pdflscape-test5.tex, pdflscape-test6.tex,
%    pdflscape-test6.txt, ExtractRotate.java.
%
% Distribution:
%    CTAN:macros/latex/contrib/oberdiek/pdflscape.dtx
%    CTAN:macros/latex/contrib/oberdiek/pdflscape.pdf
%
% Unpacking:
%    (a) If pdflscape.ins is present:
%           tex pdflscape.ins
%    (b) Without pdflscape.ins:
%           tex pdflscape.dtx
%    (c) If you insist on using LaTeX
%           latex \let\install=y% \iffalse meta-comment
%
% File: pdflscape.dtx
% Version: 2016/05/14 v0.11
% Info: Display of landscape pages in PDF
%
% Copyright (C) 2001, 2004, 2006-2008 by
%    Heiko Oberdiek <heiko.oberdiek at googlemail.com>
%    2016
%    https://github.com/ho-tex/oberdiek/issues
%
% This work may be distributed and/or modified under the
% conditions of the LaTeX Project Public License, either
% version 1.3c of this license or (at your option) any later
% version. This version of this license is in
%    http://www.latex-project.org/lppl/lppl-1-3c.txt
% and the latest version of this license is in
%    http://www.latex-project.org/lppl.txt
% and version 1.3 or later is part of all distributions of
% LaTeX version 2005/12/01 or later.
%
% This work has the LPPL maintenance status "maintained".
%
% This Current Maintainer of this work is Heiko Oberdiek.
%
% This work consists of the main source file pdflscape.dtx
% and the derived files
%    pdflscape.sty, pdflscape.pdf, pdflscape.ins, pdflscape.drv,
%    pdflscape-test1.tex, pdflscape-test2.tex,
%    pdflscape-test3.tex, pdflscape-test4.tex,
%    pdflscape-test5.tex, pdflscape-test6.tex,
%    pdflscape-test6.txt, ExtractRotate.java.
%
% Distribution:
%    CTAN:macros/latex/contrib/oberdiek/pdflscape.dtx
%    CTAN:macros/latex/contrib/oberdiek/pdflscape.pdf
%
% Unpacking:
%    (a) If pdflscape.ins is present:
%           tex pdflscape.ins
%    (b) Without pdflscape.ins:
%           tex pdflscape.dtx
%    (c) If you insist on using LaTeX
%           latex \let\install=y\input{pdflscape.dtx}
%        (quote the arguments according to the demands of your shell)
%
% Documentation:
%    (a) If pdflscape.drv is present:
%           latex pdflscape.drv
%    (b) Without pdflscape.drv:
%           latex pdflscape.dtx; ...
%    The class ltxdoc loads the configuration file ltxdoc.cfg
%    if available. Here you can specify further options, e.g.
%    use A4 as paper format:
%       \PassOptionsToClass{a4paper}{article}
%
%    Programm calls to get the documentation (example):
%       pdflatex pdflscape.dtx
%       makeindex -s gind.ist pdflscape.idx
%       pdflatex pdflscape.dtx
%       makeindex -s gind.ist pdflscape.idx
%       pdflatex pdflscape.dtx
%
% Installation:
%    TDS:tex/latex/oberdiek/pdflscape.sty
%    TDS:doc/latex/oberdiek/pdflscape.pdf
%    TDS:doc/latex/oberdiek/test/pdflscape-test1.tex
%    TDS:doc/latex/oberdiek/test/pdflscape-test2.tex
%    TDS:doc/latex/oberdiek/test/pdflscape-test3.tex
%    TDS:doc/latex/oberdiek/test/pdflscape-test4.tex
%    TDS:doc/latex/oberdiek/test/pdflscape-test5.tex
%    TDS:doc/latex/oberdiek/test/pdflscape-test6.tex
%    TDS:doc/latex/oberdiek/test/pdflscape-test6.txt
%    TDS:doc/latex/oberdiek/test/ExtractRotate.java
%    TDS:source/latex/oberdiek/pdflscape.dtx
%
%<*ignore>
\begingroup
  \catcode123=1 %
  \catcode125=2 %
  \def\x{LaTeX2e}%
\expandafter\endgroup
\ifcase 0\ifx\install y1\fi\expandafter
         \ifx\csname processbatchFile\endcsname\relax\else1\fi
         \ifx\fmtname\x\else 1\fi\relax
\else\csname fi\endcsname
%</ignore>
%<*install>
\input docstrip.tex
\Msg{************************************************************************}
\Msg{* Installation}
\Msg{* Package: pdflscape 2016/05/14 v0.11 Display of landscape pages in PDF (HO)}
\Msg{************************************************************************}

\keepsilent
\askforoverwritefalse

\let\MetaPrefix\relax
\preamble

This is a generated file.

Project: pdflscape
Version: 2016/05/14 v0.11

Copyright (C) 2001, 2004, 2006-2008 by
   Heiko Oberdiek <heiko.oberdiek at googlemail.com>

This work may be distributed and/or modified under the
conditions of the LaTeX Project Public License, either
version 1.3c of this license or (at your option) any later
version. This version of this license is in
   http://www.latex-project.org/lppl/lppl-1-3c.txt
and the latest version of this license is in
   http://www.latex-project.org/lppl.txt
and version 1.3 or later is part of all distributions of
LaTeX version 2005/12/01 or later.

This work has the LPPL maintenance status "maintained".

This Current Maintainer of this work is Heiko Oberdiek.

This work consists of the main source file pdflscape.dtx
and the derived files
   pdflscape.sty, pdflscape.pdf, pdflscape.ins, pdflscape.drv,
   pdflscape-test1.tex, pdflscape-test2.tex,
   pdflscape-test3.tex, pdflscape-test4.tex,
   pdflscape-test5.tex, pdflscape-test6.tex,
   pdflscape-test6.txt, ExtractRotate.java.

\endpreamble
\let\MetaPrefix\DoubleperCent

\generate{%
  \file{pdflscape.ins}{\from{pdflscape.dtx}{install}}%
  \file{pdflscape.drv}{\from{pdflscape.dtx}{driver}}%
  \usedir{tex/latex/oberdiek}%
  \file{pdflscape.sty}{\from{pdflscape.dtx}{package}}%
%  \usedir{doc/latex/oberdiek/test}%
%  \file{pdflscape-test1.tex}{\from{pdflscape.dtx}{test0,test1}}%
%  \file{pdflscape-test2.tex}{\from{pdflscape.dtx}{test0,test2}}%
%  \file{pdflscape-test3.tex}{\from{pdflscape.dtx}{test0,test3}}%
%  \file{pdflscape-test4.tex}{\from{pdflscape.dtx}{test0,test4}}%
%  \file{pdflscape-test5.tex}{\from{pdflscape.dtx}{test5}}%
%  \file{pdflscape-test6.tex}{\from{pdflscape.dtx}{test6}}%
  \nopreamble
  \nopostamble
%  \file{pdflscape-test6.txt}{\from{pdflscape.dtx}{result6}}%
  \file{ExtractRotate.java}{\from{pdflscape.dtx}{ExtractRotate}}%
%  \usedir{source/latex/oberdiek/catalogue}%
%  \file{pdflscape.xml}{\from{pdflscape.dtx}{catalogue}}%
}

\catcode32=13\relax% active space
\let =\space%
\Msg{************************************************************************}
\Msg{*}
\Msg{* To finish the installation you have to move the following}
\Msg{* file into a directory searched by TeX:}
\Msg{*}
\Msg{*     pdflscape.sty}
\Msg{*}
\Msg{* To produce the documentation run the file `pdflscape.drv'}
\Msg{* through LaTeX.}
\Msg{*}
\Msg{* Happy TeXing!}
\Msg{*}
\Msg{************************************************************************}

\endbatchfile
%</install>
%<*ignore>
\fi
%</ignore>
%<*driver>
\NeedsTeXFormat{LaTeX2e}
\ProvidesFile{pdflscape.drv}%
  [2016/05/14 v0.11 Display of landscape pages in PDF (HO)]%
\documentclass{ltxdoc}
\usepackage{holtxdoc}[2011/11/22]
\usepackage{graphics}
\begin{document}
  \DocInput{pdflscape.dtx}%
\end{document}
%</driver>
% \fi
%
%
% \CharacterTable
%  {Upper-case    \A\B\C\D\E\F\G\H\I\J\K\L\M\N\O\P\Q\R\S\T\U\V\W\X\Y\Z
%   Lower-case    \a\b\c\d\e\f\g\h\i\j\k\l\m\n\o\p\q\r\s\t\u\v\w\x\y\z
%   Digits        \0\1\2\3\4\5\6\7\8\9
%   Exclamation   \!     Double quote  \"     Hash (number) \#
%   Dollar        \$     Percent       \%     Ampersand     \&
%   Acute accent  \'     Left paren    \(     Right paren   \)
%   Asterisk      \*     Plus          \+     Comma         \,
%   Minus         \-     Point         \.     Solidus       \/
%   Colon         \:     Semicolon     \;     Less than     \<
%   Equals        \=     Greater than  \>     Question mark \?
%   Commercial at \@     Left bracket  \[     Backslash     \\
%   Right bracket \]     Circumflex    \^     Underscore    \_
%   Grave accent  \`     Left brace    \{     Vertical bar  \|
%   Right brace   \}     Tilde         \~}
%
% \GetFileInfo{pdflscape.drv}
%
% \title{The \xpackage{pdflscape} package}
% \date{2016/05/14 v0.11}
% \author{Heiko Oberdiek\thanks
% {Please report any issues at \url{https://github.com/ho-tex/oberdiek/issues}}\\
% \xemail{heiko.oberdiek at googlemail.com}}
%
% \maketitle
%
% \begin{abstract}
% Package \xpackage{pdflscape} adds PDF support to the
% environment \texttt{landscape} of package \xpackage{lscape}
% by setting the PDF page attribute \texttt{/Rotate}.
% \end{abstract}
%
% \tableofcontents
%
% \section{Documentation}
%
% \subsection{Supported drivers}
%
% \begin{itemize}
% \item \xoption{pdftex},\xoption{luatex}
% \item \xoption{dvips}, \xoption{dvipsone},
%       \xoption{pctex32}, \xoption{pctexps}
%       (and other drivers that provide a non-empty \cs{Gin@PS@raw})
% \item \xoption{dvipdfm}
% \end{itemize}
%
% \subsection{Caveat}
%
% Depending on the configuration Ghostscript
% adds a guessed rotation entry by its own.
% This can lead to two |/Rotate| entries per page.
% To prevent this behaviour set the parameter
% |AutoRotatePages| to |/None|, eg:
% \begin{quote}
%   |ps2pdf -dAutoRotatePages=/None|
% \end{quote}
%
% \subsection{Requirements}
%
% \begin{itemize}
% \item The package \xpackage{lscape}.
% \item The package \xpackage{atbegshi} for all drivers
%        except \xoption{pdftex},\xoption{luatex}.
% \end{itemize}
%
% \subsection{Usage}
%
% Load this package instead of or after package \xpackage{lscape}:
% \begin{quote}
%   |\usepackage{pdflscape}|\\
%   |\begin{landscape}|\dots|\end{landscape}|
% \end{quote}
%
% \StopEventually{
% }
%
% \section{Implementation}
%
% \subsection{Package identification}
%
%    \begin{macrocode}
%<*package>
%    \end{macrocode}
%    Package identification.
%    \begin{macrocode}
\NeedsTeXFormat{LaTeX2e}
\ProvidesPackage{pdflscape}%
  [2016/05/14 v0.11 Display of landscape pages in PDF (HO)]
%    \end{macrocode}
%
% \subsection{Driver options}
%
%    \begin{macrocode}
\let\PLS@option\@empty
\let\PLS@driver\@empty
\def\PLS@temp#1#2{%
  \DeclareOption{#1}{%
    \let\PLS@option\CurrentOption
    \def\PLS@driver{#2}%
  }%
}
\PLS@temp{pdftex}{pdftex}
\PLS@temp{luatex}{pdftex}
\PLS@temp{dvips}{dvips}
\PLS@temp{dvipsone}{dvips}
\PLS@temp{pctexps}{dvips}
\PLS@temp{dvipdfm}{dvipdfm}
\PLS@temp{dvipdfmx}{dvipdfm}
\PLS@temp{xetex}{dvipdfm}
\let\PLS@temp\@empty
\DeclareOption*{%
  \ifx\PLS@temp\@empty
    \let\PLS@temp\CurrentOption
  \else
    \edef\PLS@temp{\PLS@temp,\CurrentOption}%
  \fi
}
\ProcessOptions*
\ifx\PLS@option\@empty
\else
  \expandafter\PassOptionsToPackage\expandafter{\PLS@option}{lscape}%
\fi
\ifx\PLS@temp\@empty
\else
  \expandafter\PassOptionsToPackage\expandafter{\PLS@temp}{lscape}%
\fi
\RequirePackage{lscape}
%    \end{macrocode}
%
% \subsection{Autodetection of driver}
%
% \subsubsection{Driver \xoption{pdftex}}
%
%    \begin{macrocode}
\RequirePackage{ifpdf}[2006/02/20]
\def\PLS@temp{pdftex}
\ifpdf
  \ifx\PLS@driver\@empty
    \let\PLS@driver\PLS@temp
    \PackageInfo{pdflscape}{Auto-detected driver: \PLS@temp}%
  \fi
  \ifx\PLS@driver\PLS@temp
  \else
    \PackageError{pdflscape}{%
      The driver option `\PLS@option' is not compatible\MessageBreak
      with pdfTeX in PDF mode, using driver `pdftex' instead%
    }\@ehc
    \let\PLS@driver\PLS@temp
  \fi
\fi
%    \end{macrocode}
%
% \subsubsection{Driver \xoption{xetex}}
%
%    \begin{macrocode}
\RequirePackage{ifxetex}%
\def\PLS@temp{dvipdfm}
\ifxetex
  \ifx\PLS@driver\@empty
    \let\PLS@driver\PLS@temp
    \PackageInfo{pdflscape}{%
      Auto-detected driver: \PLS@temp\space(xetex)%
    }%
  \fi
  \ifx\PLS@driver\PLS@temp
  \else
    \PackageError{pdflscape}{%
      The driver option `\PLS@option' is not compatible\MessageBreak
      with XeTeX, using driver for XeTeX (dvipdfm) instead%
    }\@ehc
    \let\PLS@driver\PLS@temp
  \fi
\fi
%    \end{macrocode}
%
% \subsubsection{Detect driver based on \cs{Gin@driver}}
%
%    \begin{macrocode}
\def\PLS@temp#1#2{%
  \ifx\PLS@driver\@empty
    \begingroup
      \def\PLS@temp{#1.def}%
    \expandafter\endgroup
    \ifx\PLS@temp\Gin@driver
      \def\PLS@driver{#2}%
      \PackageInfo{pdflscape}{%
        Auto-detected driver: \PLS@driver\space(#1)%
      }%
    \fi
  \fi
}
\PLS@temp{dvipdfm}{dvipdfm}
\PLS@temp{dvipdfmx}{dvipdfm}
\PLS@temp{dvips}{dvips}
\PLS@temp{dvipsone}{dvips}
\PLS@temp{pctexps}{dvips}
%    \end{macrocode}
%
% \subsubsection{Driver \xoption{dvips}}
%
%    \begin{macrocode}
\ifx\PLS@driver\@empty
  \def\PLS@temp{dvips}%
  \ifnum 1=0\ifx\Gin@PS@raw\@undefined
            \else
              \ifx\Gin@PS@raw\relax
              \else
                \ifx\Gin@PS@raw\@empty
                \else
                  1%
                \fi
              \fi
            \fi
            \space
    \let\PLS@driver\PLS@temp
    \PackageInfo{pdflscape}{%
      Auto-detected driver: \PLS@temp\space(PostScript)%
    }%
  \fi
\fi
%    \end{macrocode}
%
%    \begin{macrocode}
\ifx\PLS@driver\@empty
  \PackageError{pdflscape}{%
    Neither pdfTeX, PostScript, nor dvipdfm driver found%
  }\@ehc
\fi
%    \end{macrocode}
%
% \subsection{Driver implementation}
%
% \begin{itemize}
% \item |\PLS@AddRotate #1|\\
%   it expects the correct rotation number in |#1|
%   and implements the adding of the /Rotation entry
%   in the |/Page| object of the current page.
% \item |\PLS@RemoveRotate|\\
%   it removes a previous |/Rotate| entry, if necessary.
% \end{itemize}
%
% \subsubsection{\pdfTeX}
%
% Not too nice is the global setting of \cs{pdfpageattr}.
% Perhaps this can be changed in future versions.
%
% luatex compat.
%    \begin{macrocode}
\ifx\pdfpageattr\@undefined
  \ifx\pdfvariable\@undefined\else 
    \protected\edef\pdfpageattr{\pdfvariable pageattr}
  \fi
\fi
%    \end{macrocode}
%    \begin{macrocode}
\def\PLS@temp{pdftex}
\ifx\PLS@temp\PLS@driver
  \def\PLS@AddRotate#1{%
    \ifnum#1=0 %
%    \end{macrocode}
%    Already the default.
%    \begin{macrocode}
    \else
      \global\pdfpageattr\expandafter{%
        \the\pdfpageattr
        /Rotate #1%
      }%
    \fi
  }%
%    \end{macrocode}
%    Removes a |/Rotate| entry.
%    \begin{macrocode}
  \def\PLS@RemoveRotate{%
    \begingroup
      \global\pdfpageattr\expandafter{\expandafter}%
      \expandafter\PLS@@RemoveRotate
        \the\pdfpageattr /Rotate\@nil
    \endgroup
  }%
  \def\PLS@@RemoveRotate#1/Rotate#2\@nil{%
%    \end{macrocode}
%    Append |/Rotate| free stuff to \cs{pdfpageattr}.
%    \begin{macrocode}
    \global\pdfpageattr\expandafter{\the\pdfpageattr#1}%
    \ifx\\#2\\%
%    \end{macrocode}
%    Ready, because the detected |/Rotate| is part of
%    the end marker: |/Rotate\@nil|
%    \begin{macrocode}
    \else
%    \end{macrocode}
%    First read in the argument of |/Rotate|,
%    then continue parsing.
%    \begin{macrocode}
      \afterassignment\PLS@@RemoveRotate
      \count0=#2\@nil
    \fi
  }%
\fi
%    \end{macrocode}
%
% \subsubsection{PostScript driver}
%
%    \begin{macrocode}
\def\PLS@temp{dvips}
\ifx\PLS@temp\PLS@driver
  \RequirePackage{atbegshi}%
  \AtBeginShipout{\PLS@AtBeginShipout}%
  \let\PLS@AtBeginShipout\@empty
  \def\PLS@AddRotate#1{%
    \ifnum#1=0 %
    \else
      \def\PLS@AtBeginShipout{%
        \global\setbox\AtBeginShipoutBox\vbox{%
          \Gin@PS@raw{%
            [{ThisPage}\string<\string</Rotate #1\string>\string>%
            /PUT pdfmark%
          }%
          \box\AtBeginShipoutBox
        }%
      }%
    \fi
  }%
  \def\PLS@RemoveRotate{\let\PLS@AtBeginShipout\@empty}%
\fi
%    \end{macrocode}
%
% \subsubsection{Driver dvipdfm}
%
%    \begin{macrocode}
\def\PLS@temp{dvipdfm}
\ifx\PLS@temp\PLS@driver
  \RequirePackage{atbegshi}%
  \AtBeginShipout{\PLS@AtBeginShipout}%
  \let\PLS@AtBeginShipout\@empty
  \def\PLS@AddRotate#1{%
    \ifnum#1=0 %
    \else
      \def\PLS@AtBeginShipout{%
        \global\setbox\AtBeginShipoutBox\vbox{%
          \special{pdf: put @thispage <</Rotate #1>>}%
          \box\AtBeginShipoutBox
        }%
      }%
    \fi
  }%
  \def\PLS@RemoveRotate{\let\PLS@AtBeginShipout\@empty}%
\fi
%    \end{macrocode}
%
% \subsection{Driver independent stuff}
%
%    The landscape environment is extended by
%    adding the correct |/Rotate| entries.
%    \begin{macrocode}
\g@addto@macro{\landscape}{\PLS@Rotate{90}}
\g@addto@macro{\endlandscape}{\PLS@Rotate{0}}
%    \end{macrocode}
%
%    \begin{macro}{\PLS@Rotate}
%    Main macro, that sets the |/Rotate| entry.\\
%    Argument: any TeX number or nothing, that means zero.\\
%    Driver independent.
%    \begin{macrocode}
\def\PLS@Rotate#1{%
  \begingroup
%    \end{macrocode}
%    1. Check and validate the argument.
%    \begin{macrocode}
    \PLS@CheckAngle{#1}%
  \expandafter\endgroup
%    \end{macrocode}
%    2. Remove previous |\Rotate| entry.
%    \begin{macrocode}
  \expandafter\PLS@RemoveRotate
%    \end{macrocode}
%    3. Add |/Rotate| entry.
%    \begin{macrocode}
  \expandafter\PLS@AddRotate\expandafter{\the\count@}%
}
%    \end{macrocode}
%    \end{macro}
%
%    \begin{macro}{\PLS@CheckAngle}
%    Validates the rotation angle.\\
%    The result is stored in the count register |\count@|.\\
%    Driver independent.
%    \begin{macrocode}
\def\PLS@CheckAngle#1{%
%    \end{macrocode}
%    Nothing means zero:
%    \begin{macrocode}
  \ifx\\#1\\%
    \count@=0 %
  \else
    \count@=#1\relax
  \fi
%    \end{macrocode}
%    Normalize to interval -360 $<$ |\count@| $<$ 360.
%    \begin{macrocode}
  \@whilenum\count@>359\do{\advance\count@ -360 }%
  \@whilenum\count@<-359\do{\advance\count@ 360 }%
%    \end{macrocode}
%    Check values: 0, +/-90, +/-180, +/-270
%    \begin{macrocode}
  \ifnum 1=0\ifnum\count@=0 \else
            \ifnum\count@=90 \else
            \ifnum\count@=-90 \else
            \ifnum\count@=180 \else
            \ifnum\count@=-180 \else
            \ifnum\count@=270 \else
            \ifnum\count@=-270 \else
           1\fi\fi\fi\fi\fi\fi\fi\relax
    \PackageError{pdflscape}{%
      Invalid value \the\count@\space for /Rotate%
    }\@ehc
    \count@=0 %
  \fi
}
%    \end{macrocode}
%    \end{macro}
%
%    \begin{macrocode}
%</package>
%    \end{macrocode}
%
% \section{Test}
%
% \subsection{Driver detection tests}
%    \begin{macrocode}
%<*test0>
\NeedsTeXFormat{LaTeX2e}
\documentclass{minimal}
\usepackage{qstest}
\IncludeTests{*}
\LogTests{log}{*}{*}
%</test0>
%    \end{macrocode}
%
%    \begin{macrocode}
%<*test1>
\usepackage{ifpdf}
\ifpdf
  \def\ExpectDriver{pdftex}%
\fi
\usepackage{ifxetex}%
\ifxetex
  \def\ExpectDriver{dvipdfm}%
\fi
\usepackage{pdflscape}[2016/05/14]
%</test1>
%    \end{macrocode}
%
%    \begin{macrocode}
%<*test2>
\def\ExpectDriver{pdftex}
\usepackage[pdftex]{pdflscape}[2016/05/14]
%</test2>
%    \end{macrocode}
%
%    \begin{macrocode}
%<*test3>
\def\ExpectDriver{dvipdfm}
\usepackage[dvipdfmx]{pdflscape}[2016/05/14]
%</test3>
%    \end{macrocode}
%
%    \begin{macrocode}
%<*test4>
\def\ExpectDriver{dvips}
\usepackage[dvipsone]{graphics}
\usepackage{pdflscape}[2016/05/14]
%</test4>
%    \end{macrocode}
%
%    \begin{macrocode}
%<*test0>
\begin{document}
\begin{qstest}{driver}{driver}
  \makeatletter
  \Expect*{\PLS@driver}*{\ExpectDriver}%
\end{qstest}
\end{document}
%</test0>
%    \end{macrocode}
%
% \subsection{Test for \cs{PLS@CheckAngle}}
%
%    \begin{macrocode}
%<*test5>
\NeedsTeXFormat{LaTeX2e}
\documentclass{minimal}
\usepackage{qstest}
\IncludeTests{*}
\LogTests{log}{*}{*}
\usepackage{pdflscape}[2016/05/14]
\begin{document}
\begin{qstest}{checkangle}{checkangle}
  \makeatletter
  \def\Result{%
    \def\ResultString{\the\count@}%
  }%
  \def\PackageError#1#2#3{%
    \def\Result{%
      \def\ResultString{error}%
    }%
  }%
  \def\Test#1#2{%
    \begingroup
      \PLS@CheckAngle{#1}%
      \Result
      \Expect*{\ResultString}{#2}%
    \endgroup
  }%
  \Test{0}{0}%
  \Test{90}{90}%
  \Test{180}{180}%
  \Test{270}{270}%
  \Test{360}{0}%
  \Test{450}{90}%
  \Test{540}{180}%
  \Test{630}{270}%
  \Test{720}{0}%
  \Test{3600}{0}%
  \Test{3690}{90}%
  \Test{-90}{270}%
  \Test{-180}{180}%
  \Test{-270}{-90}%
  \Test{-360}{0}%
  \Test{-450}{270}%
  \Test{540}{180}%
  \Test{-630}{90}%
  \Test{-720}{0}%
  \Test{-3600}{0}%
  \Test{-3690}{270}%
  \Test{1}{error}%
  \Test{-1}{error}%
  \Test{123}{error}%
\end{qstest}
\end{document}
%</test5>
%    \end{macrocode}
%
% \subsection{Test for rotate}
%
%    \begin{macrocode}
%<*test6>
\NeedsTeXFormat{LaTeX2e}
\documentclass{article}
\usepackage{pdflscape}[2016/05/14]
\begin{document}
\section{Portrait page}
\begin{landscape}
\section{Landscape page}
Foo bar foo bar foo bar foo bar foo bar
foo bar foo bar foo bar foo bar foo bar.
\end{landscape}
\section{Portrait Section}
Foo bar
\begin{landscape}
\section{Again Landscape Section}
\newpage
\section{Second Page of Landscape Section}
\end{landscape}
\section{Second Last Portrait Page}
\newpage
\section{Last Portrait Page}
\end{document}
%</test6>
%    \end{macrocode}
%
%    \begin{macrocode}
%<*result6>
/Page 1 /Rotate 0
/Page 2 /Rotate 90
/Page 3 /Rotate 0
/Page 4 /Rotate 90
/Page 5 /Rotate 90
/Page 6 /Rotate 0
/Page 7 /Rotate 0
%</result6>
%    \end{macrocode}
%
%    \begin{macrocode}
%<*ExtractRotate>
/**
 * ExtractRotate.java
 *
 * Copyright (C) 2007 by Heiko Oberdiek <heiko.oberdiek at googlemail.com>
 *
 * Requires: PDFBox (http://www.pdfbox.org/)
 *
 * Syntax: java ExtractRotate <pdffile> <textfile>
 *
 * The <pdffile> is analyzed and for each page its rotation
 * setting is printed in the <textfile>. Example:
 *   /Page 1 /Rotate 0
 *   /Page 2 /Rotate 90
 */
import java.io.FileInputStream;
import java.io.FileWriter;
import org.pdfbox.pdfparser.PDFParser;
import org.pdfbox.pdmodel.PDDocument;
import org.pdfbox.pdmodel.PDDocumentCatalog;
import org.pdfbox.pdmodel.PDPage;

public class ExtractRotate {

    public static void main(String[] args) {
        try {
            String infile = args[0];
            String outfile = args[1];
            FileWriter out = new FileWriter(outfile);
            PDFParser parser =
                    new PDFParser(new FileInputStream(infile));
            parser.parse();
            PDDocument document = parser.getPDDocument();
            PDDocumentCatalog catalog = document.getDocumentCatalog();
            int i = 0;
            for (Object page: catalog.getAllPages()) {
                i++;
                out.write("/Page " + i + " " + "/Rotate "
                        + ((PDPage)page).findRotation() + "\n");
            }
            document.close();
            out.close();
        }
        catch (Exception e) {
            e.printStackTrace();
            System.exit(1);
        }
    }
}
%</ExtractRotate>
%    \end{macrocode}
%
% \section{Installation}
%
% \subsection{Download}
%
% \paragraph{Package.} This package is available on
% CTAN\footnote{\CTANpkg{pdflscape}}:
% \begin{description}
% \item[\CTAN{macros/latex/contrib/oberdiek/pdflscape.dtx}] The source file.
% \item[\CTAN{macros/latex/contrib/oberdiek/pdflscape.pdf}] Documentation.
% \end{description}
%
%
% \paragraph{Bundle.} All the packages of the bundle `oberdiek'
% are also available in a TDS compliant ZIP archive. There
% the packages are already unpacked and the documentation files
% are generated. The files and directories obey the TDS standard.
% \begin{description}
% \item[\CTANinstall{install/macros/latex/contrib/oberdiek.tds.zip}]
% \end{description}
% \emph{TDS} refers to the standard ``A Directory Structure
% for \TeX\ Files'' (\CTAN{tds/tds.pdf}). Directories
% with \xfile{texmf} in their name are usually organized this way.
%
% \subsection{Bundle installation}
%
% \paragraph{Unpacking.} Unpack the \xfile{oberdiek.tds.zip} in the
% TDS tree (also known as \xfile{texmf} tree) of your choice.
% Example (linux):
% \begin{quote}
%   |unzip oberdiek.tds.zip -d ~/texmf|
% \end{quote}
%
% \paragraph{Script installation.}
% Check the directory \xfile{TDS:scripts/oberdiek/} for
% scripts that need further installation steps.
% Package \xpackage{attachfile2} comes with the Perl script
% \xfile{pdfatfi.pl} that should be installed in such a way
% that it can be called as \texttt{pdfatfi}.
% Example (linux):
% \begin{quote}
%   |chmod +x scripts/oberdiek/pdfatfi.pl|\\
%   |cp scripts/oberdiek/pdfatfi.pl /usr/local/bin/|
% \end{quote}
%
% \subsection{Package installation}
%
% \paragraph{Unpacking.} The \xfile{.dtx} file is a self-extracting
% \docstrip\ archive. The files are extracted by running the
% \xfile{.dtx} through \plainTeX:
% \begin{quote}
%   \verb|tex pdflscape.dtx|
% \end{quote}
%
% \paragraph{TDS.} Now the different files must be moved into
% the different directories in your installation TDS tree
% (also known as \xfile{texmf} tree):
% \begin{quote}
% \def\t{^^A
% \begin{tabular}{@{}>{\ttfamily}l@{ $\rightarrow$ }>{\ttfamily}l@{}}
%   pdflscape.sty & tex/latex/oberdiek/pdflscape.sty\\
%   pdflscape.pdf & doc/latex/oberdiek/pdflscape.pdf\\
%   test/pdflscape-test1.tex & doc/latex/oberdiek/test/pdflscape-test1.tex\\
%   test/pdflscape-test2.tex & doc/latex/oberdiek/test/pdflscape-test2.tex\\
%   test/pdflscape-test3.tex & doc/latex/oberdiek/test/pdflscape-test3.tex\\
%   test/pdflscape-test4.tex & doc/latex/oberdiek/test/pdflscape-test4.tex\\
%   test/pdflscape-test5.tex & doc/latex/oberdiek/test/pdflscape-test5.tex\\
%   test/pdflscape-test6.tex & doc/latex/oberdiek/test/pdflscape-test6.tex\\
%   test/pdflscape-test6.txt & doc/latex/oberdiek/test/pdflscape-test6.txt\\
%   test/ExtractRotate.java & doc/latex/oberdiek/test/ExtractRotate.java\\
%   pdflscape.dtx & source/latex/oberdiek/pdflscape.dtx\\
% \end{tabular}^^A
% }^^A
% \sbox0{\t}^^A
% \ifdim\wd0>\linewidth
%   \begingroup
%     \advance\linewidth by\leftmargin
%     \advance\linewidth by\rightmargin
%   \edef\x{\endgroup
%     \def\noexpand\lw{\the\linewidth}^^A
%   }\x
%   \def\lwbox{^^A
%     \leavevmode
%     \hbox to \linewidth{^^A
%       \kern-\leftmargin\relax
%       \hss
%       \usebox0
%       \hss
%       \kern-\rightmargin\relax
%     }^^A
%   }^^A
%   \ifdim\wd0>\lw
%     \sbox0{\small\t}^^A
%     \ifdim\wd0>\linewidth
%       \ifdim\wd0>\lw
%         \sbox0{\footnotesize\t}^^A
%         \ifdim\wd0>\linewidth
%           \ifdim\wd0>\lw
%             \sbox0{\scriptsize\t}^^A
%             \ifdim\wd0>\linewidth
%               \ifdim\wd0>\lw
%                 \sbox0{\tiny\t}^^A
%                 \ifdim\wd0>\linewidth
%                   \lwbox
%                 \else
%                   \usebox0
%                 \fi
%               \else
%                 \lwbox
%               \fi
%             \else
%               \usebox0
%             \fi
%           \else
%             \lwbox
%           \fi
%         \else
%           \usebox0
%         \fi
%       \else
%         \lwbox
%       \fi
%     \else
%       \usebox0
%     \fi
%   \else
%     \lwbox
%   \fi
% \else
%   \usebox0
% \fi
% \end{quote}
% If you have a \xfile{docstrip.cfg} that configures and enables \docstrip's
% TDS installing feature, then some files can already be in the right
% place, see the documentation of \docstrip.
%
% \subsection{Refresh file name databases}
%
% If your \TeX~distribution
% (\teTeX, \mikTeX, \dots) relies on file name databases, you must refresh
% these. For example, \teTeX\ users run \verb|texhash| or
% \verb|mktexlsr|.
%
% \subsection{Some details for the interested}
%
% \paragraph{Attached source.}
%
% The PDF documentation on CTAN also includes the
% \xfile{.dtx} source file. It can be extracted by
% AcrobatReader 6 or higher. Another option is \textsf{pdftk},
% e.g. unpack the file into the current directory:
% \begin{quote}
%   \verb|pdftk pdflscape.pdf unpack_files output .|
% \end{quote}
%
% \paragraph{Unpacking with \LaTeX.}
% The \xfile{.dtx} chooses its action depending on the format:
% \begin{description}
% \item[\plainTeX:] Run \docstrip\ and extract the files.
% \item[\LaTeX:] Generate the documentation.
% \end{description}
% If you insist on using \LaTeX\ for \docstrip\ (really,
% \docstrip\ does not need \LaTeX), then inform the autodetect routine
% about your intention:
% \begin{quote}
%   \verb|latex \let\install=y\input{pdflscape.dtx}|
% \end{quote}
% Do not forget to quote the argument according to the demands
% of your shell.
%
% \paragraph{Generating the documentation.}
% You can use both the \xfile{.dtx} or the \xfile{.drv} to generate
% the documentation. The process can be configured by the
% configuration file \xfile{ltxdoc.cfg}. For instance, put this
% line into this file, if you want to have A4 as paper format:
% \begin{quote}
%   \verb|\PassOptionsToClass{a4paper}{article}|
% \end{quote}
% An example follows how to generate the
% documentation with pdf\LaTeX:
% \begin{quote}
%\begin{verbatim}
%pdflatex pdflscape.dtx
%makeindex -s gind.ist pdflscape.idx
%pdflatex pdflscape.dtx
%makeindex -s gind.ist pdflscape.idx
%pdflatex pdflscape.dtx
%\end{verbatim}
% \end{quote}
%
% \begin{History}
%   \begin{Version}{2001/01/15 v0.1}
%   \item
%     First public version,
%     published in \xnewsgroup{de.comp.text.tex}:\\
%     \URL{``\link{Re: Querformat, pdfLaTex und thumbpdf}''}^^A
%     {https://groups.google.com/group/de.comp.text.tex/msg/e054c5795e52d2b8}
%   \end{Version}
%   \begin{Version}{2001/02/04 v0.2}
%   \item
%     Minor documentation update.
%   \item
%     CTAN.
%   \end{Version}
%   \begin{Version}{2004/05/11 v0.3}
%   \item
%     Support for dvipdfm added.
%   \end{Version}
%   \begin{Version}{2004/05/12 v0.4}
%   \item
%     Bug fix: support for multipage landscape environment
%     for all drivers except for \xoption{pdftex}.
%   \end{Version}
%   \begin{Version}{2006/02/20 v0.5}
%   \item
%     DTX framework.
%   \item
%     LPPL 1.3
%   \item
%     Code is not changed.
%   \end{Version}
%   \begin{Version}{2006/04/24 v0.6}
%   \item
%     Bug fix: \cs{PLS@RemoveRotate} reinserted for \cs{PLS@Rotate}.
%   \item
%     Fix for \cs{PLS@RemoveRotate} (dvips/dvipdfm): \cs{PLS@EverhShi}
%     hook is cleared.
%   \end{Version}
%   \begin{Version}{2007/04/11 v0.7}
%   \item
%     Line ends sanitized.
%   \end{Version}
%   \begin{Version}{2007/04/17 v0.8}
%   \item
%     Package \xpackage{atbegshi} replaces \xpackage{everyshi}.
%   \end{Version}
%   \begin{Version}{2007/10/21 v0.9}
%   \item
%     Driver detection for \XeTeX\ added.
%   \item
%     Fix for rotation angles $\le$~-360 or $\ge$~360.
%   \end{Version}
%   \begin{Version}{2008/08/11 v0.10}
%   \item
%     Code is not changed.
%   \item
%     URLs updated.
%   \end{Version}
%   \begin{Version}{2016/05/14 v0.11}
%   \item
%     LuaTeX Compatibility
%   \end{Version}
% \end{History}
%
% \PrintIndex
%
% \Finale
\endinput

%        (quote the arguments according to the demands of your shell)
%
% Documentation:
%    (a) If pdflscape.drv is present:
%           latex pdflscape.drv
%    (b) Without pdflscape.drv:
%           latex pdflscape.dtx; ...
%    The class ltxdoc loads the configuration file ltxdoc.cfg
%    if available. Here you can specify further options, e.g.
%    use A4 as paper format:
%       \PassOptionsToClass{a4paper}{article}
%
%    Programm calls to get the documentation (example):
%       pdflatex pdflscape.dtx
%       makeindex -s gind.ist pdflscape.idx
%       pdflatex pdflscape.dtx
%       makeindex -s gind.ist pdflscape.idx
%       pdflatex pdflscape.dtx
%
% Installation:
%    TDS:tex/latex/oberdiek/pdflscape.sty
%    TDS:doc/latex/oberdiek/pdflscape.pdf
%    TDS:doc/latex/oberdiek/test/pdflscape-test1.tex
%    TDS:doc/latex/oberdiek/test/pdflscape-test2.tex
%    TDS:doc/latex/oberdiek/test/pdflscape-test3.tex
%    TDS:doc/latex/oberdiek/test/pdflscape-test4.tex
%    TDS:doc/latex/oberdiek/test/pdflscape-test5.tex
%    TDS:doc/latex/oberdiek/test/pdflscape-test6.tex
%    TDS:doc/latex/oberdiek/test/pdflscape-test6.txt
%    TDS:doc/latex/oberdiek/test/ExtractRotate.java
%    TDS:source/latex/oberdiek/pdflscape.dtx
%
%<*ignore>
\begingroup
  \catcode123=1 %
  \catcode125=2 %
  \def\x{LaTeX2e}%
\expandafter\endgroup
\ifcase 0\ifx\install y1\fi\expandafter
         \ifx\csname processbatchFile\endcsname\relax\else1\fi
         \ifx\fmtname\x\else 1\fi\relax
\else\csname fi\endcsname
%</ignore>
%<*install>
\input docstrip.tex
\Msg{************************************************************************}
\Msg{* Installation}
\Msg{* Package: pdflscape 2016/05/14 v0.11 Display of landscape pages in PDF (HO)}
\Msg{************************************************************************}

\keepsilent
\askforoverwritefalse

\let\MetaPrefix\relax
\preamble

This is a generated file.

Project: pdflscape
Version: 2016/05/14 v0.11

Copyright (C) 2001, 2004, 2006-2008 by
   Heiko Oberdiek <heiko.oberdiek at googlemail.com>

This work may be distributed and/or modified under the
conditions of the LaTeX Project Public License, either
version 1.3c of this license or (at your option) any later
version. This version of this license is in
   http://www.latex-project.org/lppl/lppl-1-3c.txt
and the latest version of this license is in
   http://www.latex-project.org/lppl.txt
and version 1.3 or later is part of all distributions of
LaTeX version 2005/12/01 or later.

This work has the LPPL maintenance status "maintained".

This Current Maintainer of this work is Heiko Oberdiek.

This work consists of the main source file pdflscape.dtx
and the derived files
   pdflscape.sty, pdflscape.pdf, pdflscape.ins, pdflscape.drv,
   pdflscape-test1.tex, pdflscape-test2.tex,
   pdflscape-test3.tex, pdflscape-test4.tex,
   pdflscape-test5.tex, pdflscape-test6.tex,
   pdflscape-test6.txt, ExtractRotate.java.

\endpreamble
\let\MetaPrefix\DoubleperCent

\generate{%
  \file{pdflscape.ins}{\from{pdflscape.dtx}{install}}%
  \file{pdflscape.drv}{\from{pdflscape.dtx}{driver}}%
  \usedir{tex/latex/oberdiek}%
  \file{pdflscape.sty}{\from{pdflscape.dtx}{package}}%
%  \usedir{doc/latex/oberdiek/test}%
%  \file{pdflscape-test1.tex}{\from{pdflscape.dtx}{test0,test1}}%
%  \file{pdflscape-test2.tex}{\from{pdflscape.dtx}{test0,test2}}%
%  \file{pdflscape-test3.tex}{\from{pdflscape.dtx}{test0,test3}}%
%  \file{pdflscape-test4.tex}{\from{pdflscape.dtx}{test0,test4}}%
%  \file{pdflscape-test5.tex}{\from{pdflscape.dtx}{test5}}%
%  \file{pdflscape-test6.tex}{\from{pdflscape.dtx}{test6}}%
  \nopreamble
  \nopostamble
%  \file{pdflscape-test6.txt}{\from{pdflscape.dtx}{result6}}%
  \file{ExtractRotate.java}{\from{pdflscape.dtx}{ExtractRotate}}%
%  \usedir{source/latex/oberdiek/catalogue}%
%  \file{pdflscape.xml}{\from{pdflscape.dtx}{catalogue}}%
}

\catcode32=13\relax% active space
\let =\space%
\Msg{************************************************************************}
\Msg{*}
\Msg{* To finish the installation you have to move the following}
\Msg{* file into a directory searched by TeX:}
\Msg{*}
\Msg{*     pdflscape.sty}
\Msg{*}
\Msg{* To produce the documentation run the file `pdflscape.drv'}
\Msg{* through LaTeX.}
\Msg{*}
\Msg{* Happy TeXing!}
\Msg{*}
\Msg{************************************************************************}

\endbatchfile
%</install>
%<*ignore>
\fi
%</ignore>
%<*driver>
\NeedsTeXFormat{LaTeX2e}
\ProvidesFile{pdflscape.drv}%
  [2016/05/14 v0.11 Display of landscape pages in PDF (HO)]%
\documentclass{ltxdoc}
\usepackage{holtxdoc}[2011/11/22]
\usepackage{graphics}
\begin{document}
  \DocInput{pdflscape.dtx}%
\end{document}
%</driver>
% \fi
%
%
% \CharacterTable
%  {Upper-case    \A\B\C\D\E\F\G\H\I\J\K\L\M\N\O\P\Q\R\S\T\U\V\W\X\Y\Z
%   Lower-case    \a\b\c\d\e\f\g\h\i\j\k\l\m\n\o\p\q\r\s\t\u\v\w\x\y\z
%   Digits        \0\1\2\3\4\5\6\7\8\9
%   Exclamation   \!     Double quote  \"     Hash (number) \#
%   Dollar        \$     Percent       \%     Ampersand     \&
%   Acute accent  \'     Left paren    \(     Right paren   \)
%   Asterisk      \*     Plus          \+     Comma         \,
%   Minus         \-     Point         \.     Solidus       \/
%   Colon         \:     Semicolon     \;     Less than     \<
%   Equals        \=     Greater than  \>     Question mark \?
%   Commercial at \@     Left bracket  \[     Backslash     \\
%   Right bracket \]     Circumflex    \^     Underscore    \_
%   Grave accent  \`     Left brace    \{     Vertical bar  \|
%   Right brace   \}     Tilde         \~}
%
% \GetFileInfo{pdflscape.drv}
%
% \title{The \xpackage{pdflscape} package}
% \date{2016/05/14 v0.11}
% \author{Heiko Oberdiek\thanks
% {Please report any issues at \url{https://github.com/ho-tex/oberdiek/issues}}\\
% \xemail{heiko.oberdiek at googlemail.com}}
%
% \maketitle
%
% \begin{abstract}
% Package \xpackage{pdflscape} adds PDF support to the
% environment \texttt{landscape} of package \xpackage{lscape}
% by setting the PDF page attribute \texttt{/Rotate}.
% \end{abstract}
%
% \tableofcontents
%
% \section{Documentation}
%
% \subsection{Supported drivers}
%
% \begin{itemize}
% \item \xoption{pdftex},\xoption{luatex}
% \item \xoption{dvips}, \xoption{dvipsone},
%       \xoption{pctex32}, \xoption{pctexps}
%       (and other drivers that provide a non-empty \cs{Gin@PS@raw})
% \item \xoption{dvipdfm}
% \end{itemize}
%
% \subsection{Caveat}
%
% Depending on the configuration Ghostscript
% adds a guessed rotation entry by its own.
% This can lead to two |/Rotate| entries per page.
% To prevent this behaviour set the parameter
% |AutoRotatePages| to |/None|, eg:
% \begin{quote}
%   |ps2pdf -dAutoRotatePages=/None|
% \end{quote}
%
% \subsection{Requirements}
%
% \begin{itemize}
% \item The package \xpackage{lscape}.
% \item The package \xpackage{atbegshi} for all drivers
%        except \xoption{pdftex},\xoption{luatex}.
% \end{itemize}
%
% \subsection{Usage}
%
% Load this package instead of or after package \xpackage{lscape}:
% \begin{quote}
%   |\usepackage{pdflscape}|\\
%   |\begin{landscape}|\dots|\end{landscape}|
% \end{quote}
%
% \StopEventually{
% }
%
% \section{Implementation}
%
% \subsection{Package identification}
%
%    \begin{macrocode}
%<*package>
%    \end{macrocode}
%    Package identification.
%    \begin{macrocode}
\NeedsTeXFormat{LaTeX2e}
\ProvidesPackage{pdflscape}%
  [2016/05/14 v0.11 Display of landscape pages in PDF (HO)]
%    \end{macrocode}
%
% \subsection{Driver options}
%
%    \begin{macrocode}
\let\PLS@option\@empty
\let\PLS@driver\@empty
\def\PLS@temp#1#2{%
  \DeclareOption{#1}{%
    \let\PLS@option\CurrentOption
    \def\PLS@driver{#2}%
  }%
}
\PLS@temp{pdftex}{pdftex}
\PLS@temp{luatex}{pdftex}
\PLS@temp{dvips}{dvips}
\PLS@temp{dvipsone}{dvips}
\PLS@temp{pctexps}{dvips}
\PLS@temp{dvipdfm}{dvipdfm}
\PLS@temp{dvipdfmx}{dvipdfm}
\PLS@temp{xetex}{dvipdfm}
\let\PLS@temp\@empty
\DeclareOption*{%
  \ifx\PLS@temp\@empty
    \let\PLS@temp\CurrentOption
  \else
    \edef\PLS@temp{\PLS@temp,\CurrentOption}%
  \fi
}
\ProcessOptions*
\ifx\PLS@option\@empty
\else
  \expandafter\PassOptionsToPackage\expandafter{\PLS@option}{lscape}%
\fi
\ifx\PLS@temp\@empty
\else
  \expandafter\PassOptionsToPackage\expandafter{\PLS@temp}{lscape}%
\fi
\RequirePackage{lscape}
%    \end{macrocode}
%
% \subsection{Autodetection of driver}
%
% \subsubsection{Driver \xoption{pdftex}}
%
%    \begin{macrocode}
\RequirePackage{ifpdf}[2006/02/20]
\def\PLS@temp{pdftex}
\ifpdf
  \ifx\PLS@driver\@empty
    \let\PLS@driver\PLS@temp
    \PackageInfo{pdflscape}{Auto-detected driver: \PLS@temp}%
  \fi
  \ifx\PLS@driver\PLS@temp
  \else
    \PackageError{pdflscape}{%
      The driver option `\PLS@option' is not compatible\MessageBreak
      with pdfTeX in PDF mode, using driver `pdftex' instead%
    }\@ehc
    \let\PLS@driver\PLS@temp
  \fi
\fi
%    \end{macrocode}
%
% \subsubsection{Driver \xoption{xetex}}
%
%    \begin{macrocode}
\RequirePackage{ifxetex}%
\def\PLS@temp{dvipdfm}
\ifxetex
  \ifx\PLS@driver\@empty
    \let\PLS@driver\PLS@temp
    \PackageInfo{pdflscape}{%
      Auto-detected driver: \PLS@temp\space(xetex)%
    }%
  \fi
  \ifx\PLS@driver\PLS@temp
  \else
    \PackageError{pdflscape}{%
      The driver option `\PLS@option' is not compatible\MessageBreak
      with XeTeX, using driver for XeTeX (dvipdfm) instead%
    }\@ehc
    \let\PLS@driver\PLS@temp
  \fi
\fi
%    \end{macrocode}
%
% \subsubsection{Detect driver based on \cs{Gin@driver}}
%
%    \begin{macrocode}
\def\PLS@temp#1#2{%
  \ifx\PLS@driver\@empty
    \begingroup
      \def\PLS@temp{#1.def}%
    \expandafter\endgroup
    \ifx\PLS@temp\Gin@driver
      \def\PLS@driver{#2}%
      \PackageInfo{pdflscape}{%
        Auto-detected driver: \PLS@driver\space(#1)%
      }%
    \fi
  \fi
}
\PLS@temp{dvipdfm}{dvipdfm}
\PLS@temp{dvipdfmx}{dvipdfm}
\PLS@temp{dvips}{dvips}
\PLS@temp{dvipsone}{dvips}
\PLS@temp{pctexps}{dvips}
%    \end{macrocode}
%
% \subsubsection{Driver \xoption{dvips}}
%
%    \begin{macrocode}
\ifx\PLS@driver\@empty
  \def\PLS@temp{dvips}%
  \ifnum 1=0\ifx\Gin@PS@raw\@undefined
            \else
              \ifx\Gin@PS@raw\relax
              \else
                \ifx\Gin@PS@raw\@empty
                \else
                  1%
                \fi
              \fi
            \fi
            \space
    \let\PLS@driver\PLS@temp
    \PackageInfo{pdflscape}{%
      Auto-detected driver: \PLS@temp\space(PostScript)%
    }%
  \fi
\fi
%    \end{macrocode}
%
%    \begin{macrocode}
\ifx\PLS@driver\@empty
  \PackageError{pdflscape}{%
    Neither pdfTeX, PostScript, nor dvipdfm driver found%
  }\@ehc
\fi
%    \end{macrocode}
%
% \subsection{Driver implementation}
%
% \begin{itemize}
% \item |\PLS@AddRotate #1|\\
%   it expects the correct rotation number in |#1|
%   and implements the adding of the /Rotation entry
%   in the |/Page| object of the current page.
% \item |\PLS@RemoveRotate|\\
%   it removes a previous |/Rotate| entry, if necessary.
% \end{itemize}
%
% \subsubsection{\pdfTeX}
%
% Not too nice is the global setting of \cs{pdfpageattr}.
% Perhaps this can be changed in future versions.
%
% luatex compat.
%    \begin{macrocode}
\ifx\pdfpageattr\@undefined
  \ifx\pdfvariable\@undefined\else 
    \protected\edef\pdfpageattr{\pdfvariable pageattr}
  \fi
\fi
%    \end{macrocode}
%    \begin{macrocode}
\def\PLS@temp{pdftex}
\ifx\PLS@temp\PLS@driver
  \def\PLS@AddRotate#1{%
    \ifnum#1=0 %
%    \end{macrocode}
%    Already the default.
%    \begin{macrocode}
    \else
      \global\pdfpageattr\expandafter{%
        \the\pdfpageattr
        /Rotate #1%
      }%
    \fi
  }%
%    \end{macrocode}
%    Removes a |/Rotate| entry.
%    \begin{macrocode}
  \def\PLS@RemoveRotate{%
    \begingroup
      \global\pdfpageattr\expandafter{\expandafter}%
      \expandafter\PLS@@RemoveRotate
        \the\pdfpageattr /Rotate\@nil
    \endgroup
  }%
  \def\PLS@@RemoveRotate#1/Rotate#2\@nil{%
%    \end{macrocode}
%    Append |/Rotate| free stuff to \cs{pdfpageattr}.
%    \begin{macrocode}
    \global\pdfpageattr\expandafter{\the\pdfpageattr#1}%
    \ifx\\#2\\%
%    \end{macrocode}
%    Ready, because the detected |/Rotate| is part of
%    the end marker: |/Rotate\@nil|
%    \begin{macrocode}
    \else
%    \end{macrocode}
%    First read in the argument of |/Rotate|,
%    then continue parsing.
%    \begin{macrocode}
      \afterassignment\PLS@@RemoveRotate
      \count0=#2\@nil
    \fi
  }%
\fi
%    \end{macrocode}
%
% \subsubsection{PostScript driver}
%
%    \begin{macrocode}
\def\PLS@temp{dvips}
\ifx\PLS@temp\PLS@driver
  \RequirePackage{atbegshi}%
  \AtBeginShipout{\PLS@AtBeginShipout}%
  \let\PLS@AtBeginShipout\@empty
  \def\PLS@AddRotate#1{%
    \ifnum#1=0 %
    \else
      \def\PLS@AtBeginShipout{%
        \global\setbox\AtBeginShipoutBox\vbox{%
          \Gin@PS@raw{%
            [{ThisPage}\string<\string</Rotate #1\string>\string>%
            /PUT pdfmark%
          }%
          \box\AtBeginShipoutBox
        }%
      }%
    \fi
  }%
  \def\PLS@RemoveRotate{\let\PLS@AtBeginShipout\@empty}%
\fi
%    \end{macrocode}
%
% \subsubsection{Driver dvipdfm}
%
%    \begin{macrocode}
\def\PLS@temp{dvipdfm}
\ifx\PLS@temp\PLS@driver
  \RequirePackage{atbegshi}%
  \AtBeginShipout{\PLS@AtBeginShipout}%
  \let\PLS@AtBeginShipout\@empty
  \def\PLS@AddRotate#1{%
    \ifnum#1=0 %
    \else
      \def\PLS@AtBeginShipout{%
        \global\setbox\AtBeginShipoutBox\vbox{%
          \special{pdf: put @thispage <</Rotate #1>>}%
          \box\AtBeginShipoutBox
        }%
      }%
    \fi
  }%
  \def\PLS@RemoveRotate{\let\PLS@AtBeginShipout\@empty}%
\fi
%    \end{macrocode}
%
% \subsection{Driver independent stuff}
%
%    The landscape environment is extended by
%    adding the correct |/Rotate| entries.
%    \begin{macrocode}
\g@addto@macro{\landscape}{\PLS@Rotate{90}}
\g@addto@macro{\endlandscape}{\PLS@Rotate{0}}
%    \end{macrocode}
%
%    \begin{macro}{\PLS@Rotate}
%    Main macro, that sets the |/Rotate| entry.\\
%    Argument: any TeX number or nothing, that means zero.\\
%    Driver independent.
%    \begin{macrocode}
\def\PLS@Rotate#1{%
  \begingroup
%    \end{macrocode}
%    1. Check and validate the argument.
%    \begin{macrocode}
    \PLS@CheckAngle{#1}%
  \expandafter\endgroup
%    \end{macrocode}
%    2. Remove previous |\Rotate| entry.
%    \begin{macrocode}
  \expandafter\PLS@RemoveRotate
%    \end{macrocode}
%    3. Add |/Rotate| entry.
%    \begin{macrocode}
  \expandafter\PLS@AddRotate\expandafter{\the\count@}%
}
%    \end{macrocode}
%    \end{macro}
%
%    \begin{macro}{\PLS@CheckAngle}
%    Validates the rotation angle.\\
%    The result is stored in the count register |\count@|.\\
%    Driver independent.
%    \begin{macrocode}
\def\PLS@CheckAngle#1{%
%    \end{macrocode}
%    Nothing means zero:
%    \begin{macrocode}
  \ifx\\#1\\%
    \count@=0 %
  \else
    \count@=#1\relax
  \fi
%    \end{macrocode}
%    Normalize to interval -360 $<$ |\count@| $<$ 360.
%    \begin{macrocode}
  \@whilenum\count@>359\do{\advance\count@ -360 }%
  \@whilenum\count@<-359\do{\advance\count@ 360 }%
%    \end{macrocode}
%    Check values: 0, +/-90, +/-180, +/-270
%    \begin{macrocode}
  \ifnum 1=0\ifnum\count@=0 \else
            \ifnum\count@=90 \else
            \ifnum\count@=-90 \else
            \ifnum\count@=180 \else
            \ifnum\count@=-180 \else
            \ifnum\count@=270 \else
            \ifnum\count@=-270 \else
           1\fi\fi\fi\fi\fi\fi\fi\relax
    \PackageError{pdflscape}{%
      Invalid value \the\count@\space for /Rotate%
    }\@ehc
    \count@=0 %
  \fi
}
%    \end{macrocode}
%    \end{macro}
%
%    \begin{macrocode}
%</package>
%    \end{macrocode}
%
% \section{Test}
%
% \subsection{Driver detection tests}
%    \begin{macrocode}
%<*test0>
\NeedsTeXFormat{LaTeX2e}
\documentclass{minimal}
\usepackage{qstest}
\IncludeTests{*}
\LogTests{log}{*}{*}
%</test0>
%    \end{macrocode}
%
%    \begin{macrocode}
%<*test1>
\usepackage{ifpdf}
\ifpdf
  \def\ExpectDriver{pdftex}%
\fi
\usepackage{ifxetex}%
\ifxetex
  \def\ExpectDriver{dvipdfm}%
\fi
\usepackage{pdflscape}[2016/05/14]
%</test1>
%    \end{macrocode}
%
%    \begin{macrocode}
%<*test2>
\def\ExpectDriver{pdftex}
\usepackage[pdftex]{pdflscape}[2016/05/14]
%</test2>
%    \end{macrocode}
%
%    \begin{macrocode}
%<*test3>
\def\ExpectDriver{dvipdfm}
\usepackage[dvipdfmx]{pdflscape}[2016/05/14]
%</test3>
%    \end{macrocode}
%
%    \begin{macrocode}
%<*test4>
\def\ExpectDriver{dvips}
\usepackage[dvipsone]{graphics}
\usepackage{pdflscape}[2016/05/14]
%</test4>
%    \end{macrocode}
%
%    \begin{macrocode}
%<*test0>
\begin{document}
\begin{qstest}{driver}{driver}
  \makeatletter
  \Expect*{\PLS@driver}*{\ExpectDriver}%
\end{qstest}
\end{document}
%</test0>
%    \end{macrocode}
%
% \subsection{Test for \cs{PLS@CheckAngle}}
%
%    \begin{macrocode}
%<*test5>
\NeedsTeXFormat{LaTeX2e}
\documentclass{minimal}
\usepackage{qstest}
\IncludeTests{*}
\LogTests{log}{*}{*}
\usepackage{pdflscape}[2016/05/14]
\begin{document}
\begin{qstest}{checkangle}{checkangle}
  \makeatletter
  \def\Result{%
    \def\ResultString{\the\count@}%
  }%
  \def\PackageError#1#2#3{%
    \def\Result{%
      \def\ResultString{error}%
    }%
  }%
  \def\Test#1#2{%
    \begingroup
      \PLS@CheckAngle{#1}%
      \Result
      \Expect*{\ResultString}{#2}%
    \endgroup
  }%
  \Test{0}{0}%
  \Test{90}{90}%
  \Test{180}{180}%
  \Test{270}{270}%
  \Test{360}{0}%
  \Test{450}{90}%
  \Test{540}{180}%
  \Test{630}{270}%
  \Test{720}{0}%
  \Test{3600}{0}%
  \Test{3690}{90}%
  \Test{-90}{270}%
  \Test{-180}{180}%
  \Test{-270}{-90}%
  \Test{-360}{0}%
  \Test{-450}{270}%
  \Test{540}{180}%
  \Test{-630}{90}%
  \Test{-720}{0}%
  \Test{-3600}{0}%
  \Test{-3690}{270}%
  \Test{1}{error}%
  \Test{-1}{error}%
  \Test{123}{error}%
\end{qstest}
\end{document}
%</test5>
%    \end{macrocode}
%
% \subsection{Test for rotate}
%
%    \begin{macrocode}
%<*test6>
\NeedsTeXFormat{LaTeX2e}
\documentclass{article}
\usepackage{pdflscape}[2016/05/14]
\begin{document}
\section{Portrait page}
\begin{landscape}
\section{Landscape page}
Foo bar foo bar foo bar foo bar foo bar
foo bar foo bar foo bar foo bar foo bar.
\end{landscape}
\section{Portrait Section}
Foo bar
\begin{landscape}
\section{Again Landscape Section}
\newpage
\section{Second Page of Landscape Section}
\end{landscape}
\section{Second Last Portrait Page}
\newpage
\section{Last Portrait Page}
\end{document}
%</test6>
%    \end{macrocode}
%
%    \begin{macrocode}
%<*result6>
/Page 1 /Rotate 0
/Page 2 /Rotate 90
/Page 3 /Rotate 0
/Page 4 /Rotate 90
/Page 5 /Rotate 90
/Page 6 /Rotate 0
/Page 7 /Rotate 0
%</result6>
%    \end{macrocode}
%
%    \begin{macrocode}
%<*ExtractRotate>
/**
 * ExtractRotate.java
 *
 * Copyright (C) 2007 by Heiko Oberdiek <heiko.oberdiek at googlemail.com>
 *
 * Requires: PDFBox (http://www.pdfbox.org/)
 *
 * Syntax: java ExtractRotate <pdffile> <textfile>
 *
 * The <pdffile> is analyzed and for each page its rotation
 * setting is printed in the <textfile>. Example:
 *   /Page 1 /Rotate 0
 *   /Page 2 /Rotate 90
 */
import java.io.FileInputStream;
import java.io.FileWriter;
import org.pdfbox.pdfparser.PDFParser;
import org.pdfbox.pdmodel.PDDocument;
import org.pdfbox.pdmodel.PDDocumentCatalog;
import org.pdfbox.pdmodel.PDPage;

public class ExtractRotate {

    public static void main(String[] args) {
        try {
            String infile = args[0];
            String outfile = args[1];
            FileWriter out = new FileWriter(outfile);
            PDFParser parser =
                    new PDFParser(new FileInputStream(infile));
            parser.parse();
            PDDocument document = parser.getPDDocument();
            PDDocumentCatalog catalog = document.getDocumentCatalog();
            int i = 0;
            for (Object page: catalog.getAllPages()) {
                i++;
                out.write("/Page " + i + " " + "/Rotate "
                        + ((PDPage)page).findRotation() + "\n");
            }
            document.close();
            out.close();
        }
        catch (Exception e) {
            e.printStackTrace();
            System.exit(1);
        }
    }
}
%</ExtractRotate>
%    \end{macrocode}
%
% \section{Installation}
%
% \subsection{Download}
%
% \paragraph{Package.} This package is available on
% CTAN\footnote{\CTANpkg{pdflscape}}:
% \begin{description}
% \item[\CTAN{macros/latex/contrib/oberdiek/pdflscape.dtx}] The source file.
% \item[\CTAN{macros/latex/contrib/oberdiek/pdflscape.pdf}] Documentation.
% \end{description}
%
%
% \paragraph{Bundle.} All the packages of the bundle `oberdiek'
% are also available in a TDS compliant ZIP archive. There
% the packages are already unpacked and the documentation files
% are generated. The files and directories obey the TDS standard.
% \begin{description}
% \item[\CTANinstall{install/macros/latex/contrib/oberdiek.tds.zip}]
% \end{description}
% \emph{TDS} refers to the standard ``A Directory Structure
% for \TeX\ Files'' (\CTAN{tds/tds.pdf}). Directories
% with \xfile{texmf} in their name are usually organized this way.
%
% \subsection{Bundle installation}
%
% \paragraph{Unpacking.} Unpack the \xfile{oberdiek.tds.zip} in the
% TDS tree (also known as \xfile{texmf} tree) of your choice.
% Example (linux):
% \begin{quote}
%   |unzip oberdiek.tds.zip -d ~/texmf|
% \end{quote}
%
% \paragraph{Script installation.}
% Check the directory \xfile{TDS:scripts/oberdiek/} for
% scripts that need further installation steps.
% Package \xpackage{attachfile2} comes with the Perl script
% \xfile{pdfatfi.pl} that should be installed in such a way
% that it can be called as \texttt{pdfatfi}.
% Example (linux):
% \begin{quote}
%   |chmod +x scripts/oberdiek/pdfatfi.pl|\\
%   |cp scripts/oberdiek/pdfatfi.pl /usr/local/bin/|
% \end{quote}
%
% \subsection{Package installation}
%
% \paragraph{Unpacking.} The \xfile{.dtx} file is a self-extracting
% \docstrip\ archive. The files are extracted by running the
% \xfile{.dtx} through \plainTeX:
% \begin{quote}
%   \verb|tex pdflscape.dtx|
% \end{quote}
%
% \paragraph{TDS.} Now the different files must be moved into
% the different directories in your installation TDS tree
% (also known as \xfile{texmf} tree):
% \begin{quote}
% \def\t{^^A
% \begin{tabular}{@{}>{\ttfamily}l@{ $\rightarrow$ }>{\ttfamily}l@{}}
%   pdflscape.sty & tex/latex/oberdiek/pdflscape.sty\\
%   pdflscape.pdf & doc/latex/oberdiek/pdflscape.pdf\\
%   test/pdflscape-test1.tex & doc/latex/oberdiek/test/pdflscape-test1.tex\\
%   test/pdflscape-test2.tex & doc/latex/oberdiek/test/pdflscape-test2.tex\\
%   test/pdflscape-test3.tex & doc/latex/oberdiek/test/pdflscape-test3.tex\\
%   test/pdflscape-test4.tex & doc/latex/oberdiek/test/pdflscape-test4.tex\\
%   test/pdflscape-test5.tex & doc/latex/oberdiek/test/pdflscape-test5.tex\\
%   test/pdflscape-test6.tex & doc/latex/oberdiek/test/pdflscape-test6.tex\\
%   test/pdflscape-test6.txt & doc/latex/oberdiek/test/pdflscape-test6.txt\\
%   test/ExtractRotate.java & doc/latex/oberdiek/test/ExtractRotate.java\\
%   pdflscape.dtx & source/latex/oberdiek/pdflscape.dtx\\
% \end{tabular}^^A
% }^^A
% \sbox0{\t}^^A
% \ifdim\wd0>\linewidth
%   \begingroup
%     \advance\linewidth by\leftmargin
%     \advance\linewidth by\rightmargin
%   \edef\x{\endgroup
%     \def\noexpand\lw{\the\linewidth}^^A
%   }\x
%   \def\lwbox{^^A
%     \leavevmode
%     \hbox to \linewidth{^^A
%       \kern-\leftmargin\relax
%       \hss
%       \usebox0
%       \hss
%       \kern-\rightmargin\relax
%     }^^A
%   }^^A
%   \ifdim\wd0>\lw
%     \sbox0{\small\t}^^A
%     \ifdim\wd0>\linewidth
%       \ifdim\wd0>\lw
%         \sbox0{\footnotesize\t}^^A
%         \ifdim\wd0>\linewidth
%           \ifdim\wd0>\lw
%             \sbox0{\scriptsize\t}^^A
%             \ifdim\wd0>\linewidth
%               \ifdim\wd0>\lw
%                 \sbox0{\tiny\t}^^A
%                 \ifdim\wd0>\linewidth
%                   \lwbox
%                 \else
%                   \usebox0
%                 \fi
%               \else
%                 \lwbox
%               \fi
%             \else
%               \usebox0
%             \fi
%           \else
%             \lwbox
%           \fi
%         \else
%           \usebox0
%         \fi
%       \else
%         \lwbox
%       \fi
%     \else
%       \usebox0
%     \fi
%   \else
%     \lwbox
%   \fi
% \else
%   \usebox0
% \fi
% \end{quote}
% If you have a \xfile{docstrip.cfg} that configures and enables \docstrip's
% TDS installing feature, then some files can already be in the right
% place, see the documentation of \docstrip.
%
% \subsection{Refresh file name databases}
%
% If your \TeX~distribution
% (\teTeX, \mikTeX, \dots) relies on file name databases, you must refresh
% these. For example, \teTeX\ users run \verb|texhash| or
% \verb|mktexlsr|.
%
% \subsection{Some details for the interested}
%
% \paragraph{Attached source.}
%
% The PDF documentation on CTAN also includes the
% \xfile{.dtx} source file. It can be extracted by
% AcrobatReader 6 or higher. Another option is \textsf{pdftk},
% e.g. unpack the file into the current directory:
% \begin{quote}
%   \verb|pdftk pdflscape.pdf unpack_files output .|
% \end{quote}
%
% \paragraph{Unpacking with \LaTeX.}
% The \xfile{.dtx} chooses its action depending on the format:
% \begin{description}
% \item[\plainTeX:] Run \docstrip\ and extract the files.
% \item[\LaTeX:] Generate the documentation.
% \end{description}
% If you insist on using \LaTeX\ for \docstrip\ (really,
% \docstrip\ does not need \LaTeX), then inform the autodetect routine
% about your intention:
% \begin{quote}
%   \verb|latex \let\install=y% \iffalse meta-comment
%
% File: pdflscape.dtx
% Version: 2016/05/14 v0.11
% Info: Display of landscape pages in PDF
%
% Copyright (C) 2001, 2004, 2006-2008 by
%    Heiko Oberdiek <heiko.oberdiek at googlemail.com>
%    2016
%    https://github.com/ho-tex/oberdiek/issues
%
% This work may be distributed and/or modified under the
% conditions of the LaTeX Project Public License, either
% version 1.3c of this license or (at your option) any later
% version. This version of this license is in
%    http://www.latex-project.org/lppl/lppl-1-3c.txt
% and the latest version of this license is in
%    http://www.latex-project.org/lppl.txt
% and version 1.3 or later is part of all distributions of
% LaTeX version 2005/12/01 or later.
%
% This work has the LPPL maintenance status "maintained".
%
% This Current Maintainer of this work is Heiko Oberdiek.
%
% This work consists of the main source file pdflscape.dtx
% and the derived files
%    pdflscape.sty, pdflscape.pdf, pdflscape.ins, pdflscape.drv,
%    pdflscape-test1.tex, pdflscape-test2.tex,
%    pdflscape-test3.tex, pdflscape-test4.tex,
%    pdflscape-test5.tex, pdflscape-test6.tex,
%    pdflscape-test6.txt, ExtractRotate.java.
%
% Distribution:
%    CTAN:macros/latex/contrib/oberdiek/pdflscape.dtx
%    CTAN:macros/latex/contrib/oberdiek/pdflscape.pdf
%
% Unpacking:
%    (a) If pdflscape.ins is present:
%           tex pdflscape.ins
%    (b) Without pdflscape.ins:
%           tex pdflscape.dtx
%    (c) If you insist on using LaTeX
%           latex \let\install=y\input{pdflscape.dtx}
%        (quote the arguments according to the demands of your shell)
%
% Documentation:
%    (a) If pdflscape.drv is present:
%           latex pdflscape.drv
%    (b) Without pdflscape.drv:
%           latex pdflscape.dtx; ...
%    The class ltxdoc loads the configuration file ltxdoc.cfg
%    if available. Here you can specify further options, e.g.
%    use A4 as paper format:
%       \PassOptionsToClass{a4paper}{article}
%
%    Programm calls to get the documentation (example):
%       pdflatex pdflscape.dtx
%       makeindex -s gind.ist pdflscape.idx
%       pdflatex pdflscape.dtx
%       makeindex -s gind.ist pdflscape.idx
%       pdflatex pdflscape.dtx
%
% Installation:
%    TDS:tex/latex/oberdiek/pdflscape.sty
%    TDS:doc/latex/oberdiek/pdflscape.pdf
%    TDS:doc/latex/oberdiek/test/pdflscape-test1.tex
%    TDS:doc/latex/oberdiek/test/pdflscape-test2.tex
%    TDS:doc/latex/oberdiek/test/pdflscape-test3.tex
%    TDS:doc/latex/oberdiek/test/pdflscape-test4.tex
%    TDS:doc/latex/oberdiek/test/pdflscape-test5.tex
%    TDS:doc/latex/oberdiek/test/pdflscape-test6.tex
%    TDS:doc/latex/oberdiek/test/pdflscape-test6.txt
%    TDS:doc/latex/oberdiek/test/ExtractRotate.java
%    TDS:source/latex/oberdiek/pdflscape.dtx
%
%<*ignore>
\begingroup
  \catcode123=1 %
  \catcode125=2 %
  \def\x{LaTeX2e}%
\expandafter\endgroup
\ifcase 0\ifx\install y1\fi\expandafter
         \ifx\csname processbatchFile\endcsname\relax\else1\fi
         \ifx\fmtname\x\else 1\fi\relax
\else\csname fi\endcsname
%</ignore>
%<*install>
\input docstrip.tex
\Msg{************************************************************************}
\Msg{* Installation}
\Msg{* Package: pdflscape 2016/05/14 v0.11 Display of landscape pages in PDF (HO)}
\Msg{************************************************************************}

\keepsilent
\askforoverwritefalse

\let\MetaPrefix\relax
\preamble

This is a generated file.

Project: pdflscape
Version: 2016/05/14 v0.11

Copyright (C) 2001, 2004, 2006-2008 by
   Heiko Oberdiek <heiko.oberdiek at googlemail.com>

This work may be distributed and/or modified under the
conditions of the LaTeX Project Public License, either
version 1.3c of this license or (at your option) any later
version. This version of this license is in
   http://www.latex-project.org/lppl/lppl-1-3c.txt
and the latest version of this license is in
   http://www.latex-project.org/lppl.txt
and version 1.3 or later is part of all distributions of
LaTeX version 2005/12/01 or later.

This work has the LPPL maintenance status "maintained".

This Current Maintainer of this work is Heiko Oberdiek.

This work consists of the main source file pdflscape.dtx
and the derived files
   pdflscape.sty, pdflscape.pdf, pdflscape.ins, pdflscape.drv,
   pdflscape-test1.tex, pdflscape-test2.tex,
   pdflscape-test3.tex, pdflscape-test4.tex,
   pdflscape-test5.tex, pdflscape-test6.tex,
   pdflscape-test6.txt, ExtractRotate.java.

\endpreamble
\let\MetaPrefix\DoubleperCent

\generate{%
  \file{pdflscape.ins}{\from{pdflscape.dtx}{install}}%
  \file{pdflscape.drv}{\from{pdflscape.dtx}{driver}}%
  \usedir{tex/latex/oberdiek}%
  \file{pdflscape.sty}{\from{pdflscape.dtx}{package}}%
%  \usedir{doc/latex/oberdiek/test}%
%  \file{pdflscape-test1.tex}{\from{pdflscape.dtx}{test0,test1}}%
%  \file{pdflscape-test2.tex}{\from{pdflscape.dtx}{test0,test2}}%
%  \file{pdflscape-test3.tex}{\from{pdflscape.dtx}{test0,test3}}%
%  \file{pdflscape-test4.tex}{\from{pdflscape.dtx}{test0,test4}}%
%  \file{pdflscape-test5.tex}{\from{pdflscape.dtx}{test5}}%
%  \file{pdflscape-test6.tex}{\from{pdflscape.dtx}{test6}}%
  \nopreamble
  \nopostamble
%  \file{pdflscape-test6.txt}{\from{pdflscape.dtx}{result6}}%
  \file{ExtractRotate.java}{\from{pdflscape.dtx}{ExtractRotate}}%
%  \usedir{source/latex/oberdiek/catalogue}%
%  \file{pdflscape.xml}{\from{pdflscape.dtx}{catalogue}}%
}

\catcode32=13\relax% active space
\let =\space%
\Msg{************************************************************************}
\Msg{*}
\Msg{* To finish the installation you have to move the following}
\Msg{* file into a directory searched by TeX:}
\Msg{*}
\Msg{*     pdflscape.sty}
\Msg{*}
\Msg{* To produce the documentation run the file `pdflscape.drv'}
\Msg{* through LaTeX.}
\Msg{*}
\Msg{* Happy TeXing!}
\Msg{*}
\Msg{************************************************************************}

\endbatchfile
%</install>
%<*ignore>
\fi
%</ignore>
%<*driver>
\NeedsTeXFormat{LaTeX2e}
\ProvidesFile{pdflscape.drv}%
  [2016/05/14 v0.11 Display of landscape pages in PDF (HO)]%
\documentclass{ltxdoc}
\usepackage{holtxdoc}[2011/11/22]
\usepackage{graphics}
\begin{document}
  \DocInput{pdflscape.dtx}%
\end{document}
%</driver>
% \fi
%
%
% \CharacterTable
%  {Upper-case    \A\B\C\D\E\F\G\H\I\J\K\L\M\N\O\P\Q\R\S\T\U\V\W\X\Y\Z
%   Lower-case    \a\b\c\d\e\f\g\h\i\j\k\l\m\n\o\p\q\r\s\t\u\v\w\x\y\z
%   Digits        \0\1\2\3\4\5\6\7\8\9
%   Exclamation   \!     Double quote  \"     Hash (number) \#
%   Dollar        \$     Percent       \%     Ampersand     \&
%   Acute accent  \'     Left paren    \(     Right paren   \)
%   Asterisk      \*     Plus          \+     Comma         \,
%   Minus         \-     Point         \.     Solidus       \/
%   Colon         \:     Semicolon     \;     Less than     \<
%   Equals        \=     Greater than  \>     Question mark \?
%   Commercial at \@     Left bracket  \[     Backslash     \\
%   Right bracket \]     Circumflex    \^     Underscore    \_
%   Grave accent  \`     Left brace    \{     Vertical bar  \|
%   Right brace   \}     Tilde         \~}
%
% \GetFileInfo{pdflscape.drv}
%
% \title{The \xpackage{pdflscape} package}
% \date{2016/05/14 v0.11}
% \author{Heiko Oberdiek\thanks
% {Please report any issues at \url{https://github.com/ho-tex/oberdiek/issues}}\\
% \xemail{heiko.oberdiek at googlemail.com}}
%
% \maketitle
%
% \begin{abstract}
% Package \xpackage{pdflscape} adds PDF support to the
% environment \texttt{landscape} of package \xpackage{lscape}
% by setting the PDF page attribute \texttt{/Rotate}.
% \end{abstract}
%
% \tableofcontents
%
% \section{Documentation}
%
% \subsection{Supported drivers}
%
% \begin{itemize}
% \item \xoption{pdftex},\xoption{luatex}
% \item \xoption{dvips}, \xoption{dvipsone},
%       \xoption{pctex32}, \xoption{pctexps}
%       (and other drivers that provide a non-empty \cs{Gin@PS@raw})
% \item \xoption{dvipdfm}
% \end{itemize}
%
% \subsection{Caveat}
%
% Depending on the configuration Ghostscript
% adds a guessed rotation entry by its own.
% This can lead to two |/Rotate| entries per page.
% To prevent this behaviour set the parameter
% |AutoRotatePages| to |/None|, eg:
% \begin{quote}
%   |ps2pdf -dAutoRotatePages=/None|
% \end{quote}
%
% \subsection{Requirements}
%
% \begin{itemize}
% \item The package \xpackage{lscape}.
% \item The package \xpackage{atbegshi} for all drivers
%        except \xoption{pdftex},\xoption{luatex}.
% \end{itemize}
%
% \subsection{Usage}
%
% Load this package instead of or after package \xpackage{lscape}:
% \begin{quote}
%   |\usepackage{pdflscape}|\\
%   |\begin{landscape}|\dots|\end{landscape}|
% \end{quote}
%
% \StopEventually{
% }
%
% \section{Implementation}
%
% \subsection{Package identification}
%
%    \begin{macrocode}
%<*package>
%    \end{macrocode}
%    Package identification.
%    \begin{macrocode}
\NeedsTeXFormat{LaTeX2e}
\ProvidesPackage{pdflscape}%
  [2016/05/14 v0.11 Display of landscape pages in PDF (HO)]
%    \end{macrocode}
%
% \subsection{Driver options}
%
%    \begin{macrocode}
\let\PLS@option\@empty
\let\PLS@driver\@empty
\def\PLS@temp#1#2{%
  \DeclareOption{#1}{%
    \let\PLS@option\CurrentOption
    \def\PLS@driver{#2}%
  }%
}
\PLS@temp{pdftex}{pdftex}
\PLS@temp{luatex}{pdftex}
\PLS@temp{dvips}{dvips}
\PLS@temp{dvipsone}{dvips}
\PLS@temp{pctexps}{dvips}
\PLS@temp{dvipdfm}{dvipdfm}
\PLS@temp{dvipdfmx}{dvipdfm}
\PLS@temp{xetex}{dvipdfm}
\let\PLS@temp\@empty
\DeclareOption*{%
  \ifx\PLS@temp\@empty
    \let\PLS@temp\CurrentOption
  \else
    \edef\PLS@temp{\PLS@temp,\CurrentOption}%
  \fi
}
\ProcessOptions*
\ifx\PLS@option\@empty
\else
  \expandafter\PassOptionsToPackage\expandafter{\PLS@option}{lscape}%
\fi
\ifx\PLS@temp\@empty
\else
  \expandafter\PassOptionsToPackage\expandafter{\PLS@temp}{lscape}%
\fi
\RequirePackage{lscape}
%    \end{macrocode}
%
% \subsection{Autodetection of driver}
%
% \subsubsection{Driver \xoption{pdftex}}
%
%    \begin{macrocode}
\RequirePackage{ifpdf}[2006/02/20]
\def\PLS@temp{pdftex}
\ifpdf
  \ifx\PLS@driver\@empty
    \let\PLS@driver\PLS@temp
    \PackageInfo{pdflscape}{Auto-detected driver: \PLS@temp}%
  \fi
  \ifx\PLS@driver\PLS@temp
  \else
    \PackageError{pdflscape}{%
      The driver option `\PLS@option' is not compatible\MessageBreak
      with pdfTeX in PDF mode, using driver `pdftex' instead%
    }\@ehc
    \let\PLS@driver\PLS@temp
  \fi
\fi
%    \end{macrocode}
%
% \subsubsection{Driver \xoption{xetex}}
%
%    \begin{macrocode}
\RequirePackage{ifxetex}%
\def\PLS@temp{dvipdfm}
\ifxetex
  \ifx\PLS@driver\@empty
    \let\PLS@driver\PLS@temp
    \PackageInfo{pdflscape}{%
      Auto-detected driver: \PLS@temp\space(xetex)%
    }%
  \fi
  \ifx\PLS@driver\PLS@temp
  \else
    \PackageError{pdflscape}{%
      The driver option `\PLS@option' is not compatible\MessageBreak
      with XeTeX, using driver for XeTeX (dvipdfm) instead%
    }\@ehc
    \let\PLS@driver\PLS@temp
  \fi
\fi
%    \end{macrocode}
%
% \subsubsection{Detect driver based on \cs{Gin@driver}}
%
%    \begin{macrocode}
\def\PLS@temp#1#2{%
  \ifx\PLS@driver\@empty
    \begingroup
      \def\PLS@temp{#1.def}%
    \expandafter\endgroup
    \ifx\PLS@temp\Gin@driver
      \def\PLS@driver{#2}%
      \PackageInfo{pdflscape}{%
        Auto-detected driver: \PLS@driver\space(#1)%
      }%
    \fi
  \fi
}
\PLS@temp{dvipdfm}{dvipdfm}
\PLS@temp{dvipdfmx}{dvipdfm}
\PLS@temp{dvips}{dvips}
\PLS@temp{dvipsone}{dvips}
\PLS@temp{pctexps}{dvips}
%    \end{macrocode}
%
% \subsubsection{Driver \xoption{dvips}}
%
%    \begin{macrocode}
\ifx\PLS@driver\@empty
  \def\PLS@temp{dvips}%
  \ifnum 1=0\ifx\Gin@PS@raw\@undefined
            \else
              \ifx\Gin@PS@raw\relax
              \else
                \ifx\Gin@PS@raw\@empty
                \else
                  1%
                \fi
              \fi
            \fi
            \space
    \let\PLS@driver\PLS@temp
    \PackageInfo{pdflscape}{%
      Auto-detected driver: \PLS@temp\space(PostScript)%
    }%
  \fi
\fi
%    \end{macrocode}
%
%    \begin{macrocode}
\ifx\PLS@driver\@empty
  \PackageError{pdflscape}{%
    Neither pdfTeX, PostScript, nor dvipdfm driver found%
  }\@ehc
\fi
%    \end{macrocode}
%
% \subsection{Driver implementation}
%
% \begin{itemize}
% \item |\PLS@AddRotate #1|\\
%   it expects the correct rotation number in |#1|
%   and implements the adding of the /Rotation entry
%   in the |/Page| object of the current page.
% \item |\PLS@RemoveRotate|\\
%   it removes a previous |/Rotate| entry, if necessary.
% \end{itemize}
%
% \subsubsection{\pdfTeX}
%
% Not too nice is the global setting of \cs{pdfpageattr}.
% Perhaps this can be changed in future versions.
%
% luatex compat.
%    \begin{macrocode}
\ifx\pdfpageattr\@undefined
  \ifx\pdfvariable\@undefined\else 
    \protected\edef\pdfpageattr{\pdfvariable pageattr}
  \fi
\fi
%    \end{macrocode}
%    \begin{macrocode}
\def\PLS@temp{pdftex}
\ifx\PLS@temp\PLS@driver
  \def\PLS@AddRotate#1{%
    \ifnum#1=0 %
%    \end{macrocode}
%    Already the default.
%    \begin{macrocode}
    \else
      \global\pdfpageattr\expandafter{%
        \the\pdfpageattr
        /Rotate #1%
      }%
    \fi
  }%
%    \end{macrocode}
%    Removes a |/Rotate| entry.
%    \begin{macrocode}
  \def\PLS@RemoveRotate{%
    \begingroup
      \global\pdfpageattr\expandafter{\expandafter}%
      \expandafter\PLS@@RemoveRotate
        \the\pdfpageattr /Rotate\@nil
    \endgroup
  }%
  \def\PLS@@RemoveRotate#1/Rotate#2\@nil{%
%    \end{macrocode}
%    Append |/Rotate| free stuff to \cs{pdfpageattr}.
%    \begin{macrocode}
    \global\pdfpageattr\expandafter{\the\pdfpageattr#1}%
    \ifx\\#2\\%
%    \end{macrocode}
%    Ready, because the detected |/Rotate| is part of
%    the end marker: |/Rotate\@nil|
%    \begin{macrocode}
    \else
%    \end{macrocode}
%    First read in the argument of |/Rotate|,
%    then continue parsing.
%    \begin{macrocode}
      \afterassignment\PLS@@RemoveRotate
      \count0=#2\@nil
    \fi
  }%
\fi
%    \end{macrocode}
%
% \subsubsection{PostScript driver}
%
%    \begin{macrocode}
\def\PLS@temp{dvips}
\ifx\PLS@temp\PLS@driver
  \RequirePackage{atbegshi}%
  \AtBeginShipout{\PLS@AtBeginShipout}%
  \let\PLS@AtBeginShipout\@empty
  \def\PLS@AddRotate#1{%
    \ifnum#1=0 %
    \else
      \def\PLS@AtBeginShipout{%
        \global\setbox\AtBeginShipoutBox\vbox{%
          \Gin@PS@raw{%
            [{ThisPage}\string<\string</Rotate #1\string>\string>%
            /PUT pdfmark%
          }%
          \box\AtBeginShipoutBox
        }%
      }%
    \fi
  }%
  \def\PLS@RemoveRotate{\let\PLS@AtBeginShipout\@empty}%
\fi
%    \end{macrocode}
%
% \subsubsection{Driver dvipdfm}
%
%    \begin{macrocode}
\def\PLS@temp{dvipdfm}
\ifx\PLS@temp\PLS@driver
  \RequirePackage{atbegshi}%
  \AtBeginShipout{\PLS@AtBeginShipout}%
  \let\PLS@AtBeginShipout\@empty
  \def\PLS@AddRotate#1{%
    \ifnum#1=0 %
    \else
      \def\PLS@AtBeginShipout{%
        \global\setbox\AtBeginShipoutBox\vbox{%
          \special{pdf: put @thispage <</Rotate #1>>}%
          \box\AtBeginShipoutBox
        }%
      }%
    \fi
  }%
  \def\PLS@RemoveRotate{\let\PLS@AtBeginShipout\@empty}%
\fi
%    \end{macrocode}
%
% \subsection{Driver independent stuff}
%
%    The landscape environment is extended by
%    adding the correct |/Rotate| entries.
%    \begin{macrocode}
\g@addto@macro{\landscape}{\PLS@Rotate{90}}
\g@addto@macro{\endlandscape}{\PLS@Rotate{0}}
%    \end{macrocode}
%
%    \begin{macro}{\PLS@Rotate}
%    Main macro, that sets the |/Rotate| entry.\\
%    Argument: any TeX number or nothing, that means zero.\\
%    Driver independent.
%    \begin{macrocode}
\def\PLS@Rotate#1{%
  \begingroup
%    \end{macrocode}
%    1. Check and validate the argument.
%    \begin{macrocode}
    \PLS@CheckAngle{#1}%
  \expandafter\endgroup
%    \end{macrocode}
%    2. Remove previous |\Rotate| entry.
%    \begin{macrocode}
  \expandafter\PLS@RemoveRotate
%    \end{macrocode}
%    3. Add |/Rotate| entry.
%    \begin{macrocode}
  \expandafter\PLS@AddRotate\expandafter{\the\count@}%
}
%    \end{macrocode}
%    \end{macro}
%
%    \begin{macro}{\PLS@CheckAngle}
%    Validates the rotation angle.\\
%    The result is stored in the count register |\count@|.\\
%    Driver independent.
%    \begin{macrocode}
\def\PLS@CheckAngle#1{%
%    \end{macrocode}
%    Nothing means zero:
%    \begin{macrocode}
  \ifx\\#1\\%
    \count@=0 %
  \else
    \count@=#1\relax
  \fi
%    \end{macrocode}
%    Normalize to interval -360 $<$ |\count@| $<$ 360.
%    \begin{macrocode}
  \@whilenum\count@>359\do{\advance\count@ -360 }%
  \@whilenum\count@<-359\do{\advance\count@ 360 }%
%    \end{macrocode}
%    Check values: 0, +/-90, +/-180, +/-270
%    \begin{macrocode}
  \ifnum 1=0\ifnum\count@=0 \else
            \ifnum\count@=90 \else
            \ifnum\count@=-90 \else
            \ifnum\count@=180 \else
            \ifnum\count@=-180 \else
            \ifnum\count@=270 \else
            \ifnum\count@=-270 \else
           1\fi\fi\fi\fi\fi\fi\fi\relax
    \PackageError{pdflscape}{%
      Invalid value \the\count@\space for /Rotate%
    }\@ehc
    \count@=0 %
  \fi
}
%    \end{macrocode}
%    \end{macro}
%
%    \begin{macrocode}
%</package>
%    \end{macrocode}
%
% \section{Test}
%
% \subsection{Driver detection tests}
%    \begin{macrocode}
%<*test0>
\NeedsTeXFormat{LaTeX2e}
\documentclass{minimal}
\usepackage{qstest}
\IncludeTests{*}
\LogTests{log}{*}{*}
%</test0>
%    \end{macrocode}
%
%    \begin{macrocode}
%<*test1>
\usepackage{ifpdf}
\ifpdf
  \def\ExpectDriver{pdftex}%
\fi
\usepackage{ifxetex}%
\ifxetex
  \def\ExpectDriver{dvipdfm}%
\fi
\usepackage{pdflscape}[2016/05/14]
%</test1>
%    \end{macrocode}
%
%    \begin{macrocode}
%<*test2>
\def\ExpectDriver{pdftex}
\usepackage[pdftex]{pdflscape}[2016/05/14]
%</test2>
%    \end{macrocode}
%
%    \begin{macrocode}
%<*test3>
\def\ExpectDriver{dvipdfm}
\usepackage[dvipdfmx]{pdflscape}[2016/05/14]
%</test3>
%    \end{macrocode}
%
%    \begin{macrocode}
%<*test4>
\def\ExpectDriver{dvips}
\usepackage[dvipsone]{graphics}
\usepackage{pdflscape}[2016/05/14]
%</test4>
%    \end{macrocode}
%
%    \begin{macrocode}
%<*test0>
\begin{document}
\begin{qstest}{driver}{driver}
  \makeatletter
  \Expect*{\PLS@driver}*{\ExpectDriver}%
\end{qstest}
\end{document}
%</test0>
%    \end{macrocode}
%
% \subsection{Test for \cs{PLS@CheckAngle}}
%
%    \begin{macrocode}
%<*test5>
\NeedsTeXFormat{LaTeX2e}
\documentclass{minimal}
\usepackage{qstest}
\IncludeTests{*}
\LogTests{log}{*}{*}
\usepackage{pdflscape}[2016/05/14]
\begin{document}
\begin{qstest}{checkangle}{checkangle}
  \makeatletter
  \def\Result{%
    \def\ResultString{\the\count@}%
  }%
  \def\PackageError#1#2#3{%
    \def\Result{%
      \def\ResultString{error}%
    }%
  }%
  \def\Test#1#2{%
    \begingroup
      \PLS@CheckAngle{#1}%
      \Result
      \Expect*{\ResultString}{#2}%
    \endgroup
  }%
  \Test{0}{0}%
  \Test{90}{90}%
  \Test{180}{180}%
  \Test{270}{270}%
  \Test{360}{0}%
  \Test{450}{90}%
  \Test{540}{180}%
  \Test{630}{270}%
  \Test{720}{0}%
  \Test{3600}{0}%
  \Test{3690}{90}%
  \Test{-90}{270}%
  \Test{-180}{180}%
  \Test{-270}{-90}%
  \Test{-360}{0}%
  \Test{-450}{270}%
  \Test{540}{180}%
  \Test{-630}{90}%
  \Test{-720}{0}%
  \Test{-3600}{0}%
  \Test{-3690}{270}%
  \Test{1}{error}%
  \Test{-1}{error}%
  \Test{123}{error}%
\end{qstest}
\end{document}
%</test5>
%    \end{macrocode}
%
% \subsection{Test for rotate}
%
%    \begin{macrocode}
%<*test6>
\NeedsTeXFormat{LaTeX2e}
\documentclass{article}
\usepackage{pdflscape}[2016/05/14]
\begin{document}
\section{Portrait page}
\begin{landscape}
\section{Landscape page}
Foo bar foo bar foo bar foo bar foo bar
foo bar foo bar foo bar foo bar foo bar.
\end{landscape}
\section{Portrait Section}
Foo bar
\begin{landscape}
\section{Again Landscape Section}
\newpage
\section{Second Page of Landscape Section}
\end{landscape}
\section{Second Last Portrait Page}
\newpage
\section{Last Portrait Page}
\end{document}
%</test6>
%    \end{macrocode}
%
%    \begin{macrocode}
%<*result6>
/Page 1 /Rotate 0
/Page 2 /Rotate 90
/Page 3 /Rotate 0
/Page 4 /Rotate 90
/Page 5 /Rotate 90
/Page 6 /Rotate 0
/Page 7 /Rotate 0
%</result6>
%    \end{macrocode}
%
%    \begin{macrocode}
%<*ExtractRotate>
/**
 * ExtractRotate.java
 *
 * Copyright (C) 2007 by Heiko Oberdiek <heiko.oberdiek at googlemail.com>
 *
 * Requires: PDFBox (http://www.pdfbox.org/)
 *
 * Syntax: java ExtractRotate <pdffile> <textfile>
 *
 * The <pdffile> is analyzed and for each page its rotation
 * setting is printed in the <textfile>. Example:
 *   /Page 1 /Rotate 0
 *   /Page 2 /Rotate 90
 */
import java.io.FileInputStream;
import java.io.FileWriter;
import org.pdfbox.pdfparser.PDFParser;
import org.pdfbox.pdmodel.PDDocument;
import org.pdfbox.pdmodel.PDDocumentCatalog;
import org.pdfbox.pdmodel.PDPage;

public class ExtractRotate {

    public static void main(String[] args) {
        try {
            String infile = args[0];
            String outfile = args[1];
            FileWriter out = new FileWriter(outfile);
            PDFParser parser =
                    new PDFParser(new FileInputStream(infile));
            parser.parse();
            PDDocument document = parser.getPDDocument();
            PDDocumentCatalog catalog = document.getDocumentCatalog();
            int i = 0;
            for (Object page: catalog.getAllPages()) {
                i++;
                out.write("/Page " + i + " " + "/Rotate "
                        + ((PDPage)page).findRotation() + "\n");
            }
            document.close();
            out.close();
        }
        catch (Exception e) {
            e.printStackTrace();
            System.exit(1);
        }
    }
}
%</ExtractRotate>
%    \end{macrocode}
%
% \section{Installation}
%
% \subsection{Download}
%
% \paragraph{Package.} This package is available on
% CTAN\footnote{\CTANpkg{pdflscape}}:
% \begin{description}
% \item[\CTAN{macros/latex/contrib/oberdiek/pdflscape.dtx}] The source file.
% \item[\CTAN{macros/latex/contrib/oberdiek/pdflscape.pdf}] Documentation.
% \end{description}
%
%
% \paragraph{Bundle.} All the packages of the bundle `oberdiek'
% are also available in a TDS compliant ZIP archive. There
% the packages are already unpacked and the documentation files
% are generated. The files and directories obey the TDS standard.
% \begin{description}
% \item[\CTANinstall{install/macros/latex/contrib/oberdiek.tds.zip}]
% \end{description}
% \emph{TDS} refers to the standard ``A Directory Structure
% for \TeX\ Files'' (\CTAN{tds/tds.pdf}). Directories
% with \xfile{texmf} in their name are usually organized this way.
%
% \subsection{Bundle installation}
%
% \paragraph{Unpacking.} Unpack the \xfile{oberdiek.tds.zip} in the
% TDS tree (also known as \xfile{texmf} tree) of your choice.
% Example (linux):
% \begin{quote}
%   |unzip oberdiek.tds.zip -d ~/texmf|
% \end{quote}
%
% \paragraph{Script installation.}
% Check the directory \xfile{TDS:scripts/oberdiek/} for
% scripts that need further installation steps.
% Package \xpackage{attachfile2} comes with the Perl script
% \xfile{pdfatfi.pl} that should be installed in such a way
% that it can be called as \texttt{pdfatfi}.
% Example (linux):
% \begin{quote}
%   |chmod +x scripts/oberdiek/pdfatfi.pl|\\
%   |cp scripts/oberdiek/pdfatfi.pl /usr/local/bin/|
% \end{quote}
%
% \subsection{Package installation}
%
% \paragraph{Unpacking.} The \xfile{.dtx} file is a self-extracting
% \docstrip\ archive. The files are extracted by running the
% \xfile{.dtx} through \plainTeX:
% \begin{quote}
%   \verb|tex pdflscape.dtx|
% \end{quote}
%
% \paragraph{TDS.} Now the different files must be moved into
% the different directories in your installation TDS tree
% (also known as \xfile{texmf} tree):
% \begin{quote}
% \def\t{^^A
% \begin{tabular}{@{}>{\ttfamily}l@{ $\rightarrow$ }>{\ttfamily}l@{}}
%   pdflscape.sty & tex/latex/oberdiek/pdflscape.sty\\
%   pdflscape.pdf & doc/latex/oberdiek/pdflscape.pdf\\
%   test/pdflscape-test1.tex & doc/latex/oberdiek/test/pdflscape-test1.tex\\
%   test/pdflscape-test2.tex & doc/latex/oberdiek/test/pdflscape-test2.tex\\
%   test/pdflscape-test3.tex & doc/latex/oberdiek/test/pdflscape-test3.tex\\
%   test/pdflscape-test4.tex & doc/latex/oberdiek/test/pdflscape-test4.tex\\
%   test/pdflscape-test5.tex & doc/latex/oberdiek/test/pdflscape-test5.tex\\
%   test/pdflscape-test6.tex & doc/latex/oberdiek/test/pdflscape-test6.tex\\
%   test/pdflscape-test6.txt & doc/latex/oberdiek/test/pdflscape-test6.txt\\
%   test/ExtractRotate.java & doc/latex/oberdiek/test/ExtractRotate.java\\
%   pdflscape.dtx & source/latex/oberdiek/pdflscape.dtx\\
% \end{tabular}^^A
% }^^A
% \sbox0{\t}^^A
% \ifdim\wd0>\linewidth
%   \begingroup
%     \advance\linewidth by\leftmargin
%     \advance\linewidth by\rightmargin
%   \edef\x{\endgroup
%     \def\noexpand\lw{\the\linewidth}^^A
%   }\x
%   \def\lwbox{^^A
%     \leavevmode
%     \hbox to \linewidth{^^A
%       \kern-\leftmargin\relax
%       \hss
%       \usebox0
%       \hss
%       \kern-\rightmargin\relax
%     }^^A
%   }^^A
%   \ifdim\wd0>\lw
%     \sbox0{\small\t}^^A
%     \ifdim\wd0>\linewidth
%       \ifdim\wd0>\lw
%         \sbox0{\footnotesize\t}^^A
%         \ifdim\wd0>\linewidth
%           \ifdim\wd0>\lw
%             \sbox0{\scriptsize\t}^^A
%             \ifdim\wd0>\linewidth
%               \ifdim\wd0>\lw
%                 \sbox0{\tiny\t}^^A
%                 \ifdim\wd0>\linewidth
%                   \lwbox
%                 \else
%                   \usebox0
%                 \fi
%               \else
%                 \lwbox
%               \fi
%             \else
%               \usebox0
%             \fi
%           \else
%             \lwbox
%           \fi
%         \else
%           \usebox0
%         \fi
%       \else
%         \lwbox
%       \fi
%     \else
%       \usebox0
%     \fi
%   \else
%     \lwbox
%   \fi
% \else
%   \usebox0
% \fi
% \end{quote}
% If you have a \xfile{docstrip.cfg} that configures and enables \docstrip's
% TDS installing feature, then some files can already be in the right
% place, see the documentation of \docstrip.
%
% \subsection{Refresh file name databases}
%
% If your \TeX~distribution
% (\teTeX, \mikTeX, \dots) relies on file name databases, you must refresh
% these. For example, \teTeX\ users run \verb|texhash| or
% \verb|mktexlsr|.
%
% \subsection{Some details for the interested}
%
% \paragraph{Attached source.}
%
% The PDF documentation on CTAN also includes the
% \xfile{.dtx} source file. It can be extracted by
% AcrobatReader 6 or higher. Another option is \textsf{pdftk},
% e.g. unpack the file into the current directory:
% \begin{quote}
%   \verb|pdftk pdflscape.pdf unpack_files output .|
% \end{quote}
%
% \paragraph{Unpacking with \LaTeX.}
% The \xfile{.dtx} chooses its action depending on the format:
% \begin{description}
% \item[\plainTeX:] Run \docstrip\ and extract the files.
% \item[\LaTeX:] Generate the documentation.
% \end{description}
% If you insist on using \LaTeX\ for \docstrip\ (really,
% \docstrip\ does not need \LaTeX), then inform the autodetect routine
% about your intention:
% \begin{quote}
%   \verb|latex \let\install=y\input{pdflscape.dtx}|
% \end{quote}
% Do not forget to quote the argument according to the demands
% of your shell.
%
% \paragraph{Generating the documentation.}
% You can use both the \xfile{.dtx} or the \xfile{.drv} to generate
% the documentation. The process can be configured by the
% configuration file \xfile{ltxdoc.cfg}. For instance, put this
% line into this file, if you want to have A4 as paper format:
% \begin{quote}
%   \verb|\PassOptionsToClass{a4paper}{article}|
% \end{quote}
% An example follows how to generate the
% documentation with pdf\LaTeX:
% \begin{quote}
%\begin{verbatim}
%pdflatex pdflscape.dtx
%makeindex -s gind.ist pdflscape.idx
%pdflatex pdflscape.dtx
%makeindex -s gind.ist pdflscape.idx
%pdflatex pdflscape.dtx
%\end{verbatim}
% \end{quote}
%
% \begin{History}
%   \begin{Version}{2001/01/15 v0.1}
%   \item
%     First public version,
%     published in \xnewsgroup{de.comp.text.tex}:\\
%     \URL{``\link{Re: Querformat, pdfLaTex und thumbpdf}''}^^A
%     {https://groups.google.com/group/de.comp.text.tex/msg/e054c5795e52d2b8}
%   \end{Version}
%   \begin{Version}{2001/02/04 v0.2}
%   \item
%     Minor documentation update.
%   \item
%     CTAN.
%   \end{Version}
%   \begin{Version}{2004/05/11 v0.3}
%   \item
%     Support for dvipdfm added.
%   \end{Version}
%   \begin{Version}{2004/05/12 v0.4}
%   \item
%     Bug fix: support for multipage landscape environment
%     for all drivers except for \xoption{pdftex}.
%   \end{Version}
%   \begin{Version}{2006/02/20 v0.5}
%   \item
%     DTX framework.
%   \item
%     LPPL 1.3
%   \item
%     Code is not changed.
%   \end{Version}
%   \begin{Version}{2006/04/24 v0.6}
%   \item
%     Bug fix: \cs{PLS@RemoveRotate} reinserted for \cs{PLS@Rotate}.
%   \item
%     Fix for \cs{PLS@RemoveRotate} (dvips/dvipdfm): \cs{PLS@EverhShi}
%     hook is cleared.
%   \end{Version}
%   \begin{Version}{2007/04/11 v0.7}
%   \item
%     Line ends sanitized.
%   \end{Version}
%   \begin{Version}{2007/04/17 v0.8}
%   \item
%     Package \xpackage{atbegshi} replaces \xpackage{everyshi}.
%   \end{Version}
%   \begin{Version}{2007/10/21 v0.9}
%   \item
%     Driver detection for \XeTeX\ added.
%   \item
%     Fix for rotation angles $\le$~-360 or $\ge$~360.
%   \end{Version}
%   \begin{Version}{2008/08/11 v0.10}
%   \item
%     Code is not changed.
%   \item
%     URLs updated.
%   \end{Version}
%   \begin{Version}{2016/05/14 v0.11}
%   \item
%     LuaTeX Compatibility
%   \end{Version}
% \end{History}
%
% \PrintIndex
%
% \Finale
\endinput
|
% \end{quote}
% Do not forget to quote the argument according to the demands
% of your shell.
%
% \paragraph{Generating the documentation.}
% You can use both the \xfile{.dtx} or the \xfile{.drv} to generate
% the documentation. The process can be configured by the
% configuration file \xfile{ltxdoc.cfg}. For instance, put this
% line into this file, if you want to have A4 as paper format:
% \begin{quote}
%   \verb|\PassOptionsToClass{a4paper}{article}|
% \end{quote}
% An example follows how to generate the
% documentation with pdf\LaTeX:
% \begin{quote}
%\begin{verbatim}
%pdflatex pdflscape.dtx
%makeindex -s gind.ist pdflscape.idx
%pdflatex pdflscape.dtx
%makeindex -s gind.ist pdflscape.idx
%pdflatex pdflscape.dtx
%\end{verbatim}
% \end{quote}
%
% \begin{History}
%   \begin{Version}{2001/01/15 v0.1}
%   \item
%     First public version,
%     published in \xnewsgroup{de.comp.text.tex}:\\
%     \URL{``\link{Re: Querformat, pdfLaTex und thumbpdf}''}^^A
%     {https://groups.google.com/group/de.comp.text.tex/msg/e054c5795e52d2b8}
%   \end{Version}
%   \begin{Version}{2001/02/04 v0.2}
%   \item
%     Minor documentation update.
%   \item
%     CTAN.
%   \end{Version}
%   \begin{Version}{2004/05/11 v0.3}
%   \item
%     Support for dvipdfm added.
%   \end{Version}
%   \begin{Version}{2004/05/12 v0.4}
%   \item
%     Bug fix: support for multipage landscape environment
%     for all drivers except for \xoption{pdftex}.
%   \end{Version}
%   \begin{Version}{2006/02/20 v0.5}
%   \item
%     DTX framework.
%   \item
%     LPPL 1.3
%   \item
%     Code is not changed.
%   \end{Version}
%   \begin{Version}{2006/04/24 v0.6}
%   \item
%     Bug fix: \cs{PLS@RemoveRotate} reinserted for \cs{PLS@Rotate}.
%   \item
%     Fix for \cs{PLS@RemoveRotate} (dvips/dvipdfm): \cs{PLS@EverhShi}
%     hook is cleared.
%   \end{Version}
%   \begin{Version}{2007/04/11 v0.7}
%   \item
%     Line ends sanitized.
%   \end{Version}
%   \begin{Version}{2007/04/17 v0.8}
%   \item
%     Package \xpackage{atbegshi} replaces \xpackage{everyshi}.
%   \end{Version}
%   \begin{Version}{2007/10/21 v0.9}
%   \item
%     Driver detection for \XeTeX\ added.
%   \item
%     Fix for rotation angles $\le$~-360 or $\ge$~360.
%   \end{Version}
%   \begin{Version}{2008/08/11 v0.10}
%   \item
%     Code is not changed.
%   \item
%     URLs updated.
%   \end{Version}
%   \begin{Version}{2016/05/14 v0.11}
%   \item
%     LuaTeX Compatibility
%   \end{Version}
% \end{History}
%
% \PrintIndex
%
% \Finale
\endinput
|
% \end{quote}
% Do not forget to quote the argument according to the demands
% of your shell.
%
% \paragraph{Generating the documentation.}
% You can use both the \xfile{.dtx} or the \xfile{.drv} to generate
% the documentation. The process can be configured by the
% configuration file \xfile{ltxdoc.cfg}. For instance, put this
% line into this file, if you want to have A4 as paper format:
% \begin{quote}
%   \verb|\PassOptionsToClass{a4paper}{article}|
% \end{quote}
% An example follows how to generate the
% documentation with pdf\LaTeX:
% \begin{quote}
%\begin{verbatim}
%pdflatex pdflscape.dtx
%makeindex -s gind.ist pdflscape.idx
%pdflatex pdflscape.dtx
%makeindex -s gind.ist pdflscape.idx
%pdflatex pdflscape.dtx
%\end{verbatim}
% \end{quote}
%
% \begin{History}
%   \begin{Version}{2001/01/15 v0.1}
%   \item
%     First public version,
%     published in \xnewsgroup{de.comp.text.tex}:\\
%     \URL{``\link{Re: Querformat, pdfLaTex und thumbpdf}''}^^A
%     {https://groups.google.com/group/de.comp.text.tex/msg/e054c5795e52d2b8}
%   \end{Version}
%   \begin{Version}{2001/02/04 v0.2}
%   \item
%     Minor documentation update.
%   \item
%     CTAN.
%   \end{Version}
%   \begin{Version}{2004/05/11 v0.3}
%   \item
%     Support for dvipdfm added.
%   \end{Version}
%   \begin{Version}{2004/05/12 v0.4}
%   \item
%     Bug fix: support for multipage landscape environment
%     for all drivers except for \xoption{pdftex}.
%   \end{Version}
%   \begin{Version}{2006/02/20 v0.5}
%   \item
%     DTX framework.
%   \item
%     LPPL 1.3
%   \item
%     Code is not changed.
%   \end{Version}
%   \begin{Version}{2006/04/24 v0.6}
%   \item
%     Bug fix: \cs{PLS@RemoveRotate} reinserted for \cs{PLS@Rotate}.
%   \item
%     Fix for \cs{PLS@RemoveRotate} (dvips/dvipdfm): \cs{PLS@EverhShi}
%     hook is cleared.
%   \end{Version}
%   \begin{Version}{2007/04/11 v0.7}
%   \item
%     Line ends sanitized.
%   \end{Version}
%   \begin{Version}{2007/04/17 v0.8}
%   \item
%     Package \xpackage{atbegshi} replaces \xpackage{everyshi}.
%   \end{Version}
%   \begin{Version}{2007/10/21 v0.9}
%   \item
%     Driver detection for \XeTeX\ added.
%   \item
%     Fix for rotation angles $\le$~-360 or $\ge$~360.
%   \end{Version}
%   \begin{Version}{2008/08/11 v0.10}
%   \item
%     Code is not changed.
%   \item
%     URLs updated.
%   \end{Version}
%   \begin{Version}{2016/05/14 v0.11}
%   \item
%     LuaTeX Compatibility
%   \end{Version}
% \end{History}
%
% \PrintIndex
%
% \Finale
\endinput

%        (quote the arguments according to the demands of your shell)
%
% Documentation:
%    (a) If pdflscape.drv is present:
%           latex pdflscape.drv
%    (b) Without pdflscape.drv:
%           latex pdflscape.dtx; ...
%    The class ltxdoc loads the configuration file ltxdoc.cfg
%    if available. Here you can specify further options, e.g.
%    use A4 as paper format:
%       \PassOptionsToClass{a4paper}{article}
%
%    Programm calls to get the documentation (example):
%       pdflatex pdflscape.dtx
%       makeindex -s gind.ist pdflscape.idx
%       pdflatex pdflscape.dtx
%       makeindex -s gind.ist pdflscape.idx
%       pdflatex pdflscape.dtx
%
% Installation:
%    TDS:tex/latex/oberdiek/pdflscape.sty
%    TDS:doc/latex/oberdiek/pdflscape.pdf
%    TDS:doc/latex/oberdiek/test/pdflscape-test1.tex
%    TDS:doc/latex/oberdiek/test/pdflscape-test2.tex
%    TDS:doc/latex/oberdiek/test/pdflscape-test3.tex
%    TDS:doc/latex/oberdiek/test/pdflscape-test4.tex
%    TDS:doc/latex/oberdiek/test/pdflscape-test5.tex
%    TDS:doc/latex/oberdiek/test/pdflscape-test6.tex
%    TDS:doc/latex/oberdiek/test/pdflscape-test6.txt
%    TDS:doc/latex/oberdiek/test/ExtractRotate.java
%    TDS:source/latex/oberdiek/pdflscape.dtx
%
%<*ignore>
\begingroup
  \catcode123=1 %
  \catcode125=2 %
  \def\x{LaTeX2e}%
\expandafter\endgroup
\ifcase 0\ifx\install y1\fi\expandafter
         \ifx\csname processbatchFile\endcsname\relax\else1\fi
         \ifx\fmtname\x\else 1\fi\relax
\else\csname fi\endcsname
%</ignore>
%<*install>
\input docstrip.tex
\Msg{************************************************************************}
\Msg{* Installation}
\Msg{* Package: pdflscape 2016/05/14 v0.11 Display of landscape pages in PDF (HO)}
\Msg{************************************************************************}

\keepsilent
\askforoverwritefalse

\let\MetaPrefix\relax
\preamble

This is a generated file.

Project: pdflscape
Version: 2016/05/14 v0.11

Copyright (C) 2001, 2004, 2006-2008 by
   Heiko Oberdiek <heiko.oberdiek at googlemail.com>

This work may be distributed and/or modified under the
conditions of the LaTeX Project Public License, either
version 1.3c of this license or (at your option) any later
version. This version of this license is in
   https://www.latex-project.org/lppl/lppl-1-3c.txt
and the latest version of this license is in
   https://www.latex-project.org/lppl.txt
and version 1.3 or later is part of all distributions of
LaTeX version 2005/12/01 or later.

This work has the LPPL maintenance status "maintained".

The Current Maintainers of this work are
Heiko Oberdiek and the Oberdiek Package Support Group
https://github.com/ho-tex/oberdiek/issues


This work consists of the main source file pdflscape.dtx
and the derived files
   pdflscape.sty, pdflscape.pdf, pdflscape.ins, pdflscape.drv,
   pdflscape-test1.tex, pdflscape-test2.tex,
   pdflscape-test3.tex, pdflscape-test4.tex,
   pdflscape-test5.tex, pdflscape-test6.tex,
   pdflscape-test6.txt, ExtractRotate.java.

\endpreamble
\let\MetaPrefix\DoubleperCent

\generate{%
  \file{pdflscape.ins}{\from{pdflscape.dtx}{install}}%
  \file{pdflscape.drv}{\from{pdflscape.dtx}{driver}}%
  \usedir{tex/latex/oberdiek}%
  \file{pdflscape.sty}{\from{pdflscape.dtx}{package}}%
%  \usedir{doc/latex/oberdiek/test}%
%  \file{pdflscape-test1.tex}{\from{pdflscape.dtx}{test0,test1}}%
%  \file{pdflscape-test2.tex}{\from{pdflscape.dtx}{test0,test2}}%
%  \file{pdflscape-test3.tex}{\from{pdflscape.dtx}{test0,test3}}%
%  \file{pdflscape-test4.tex}{\from{pdflscape.dtx}{test0,test4}}%
%  \file{pdflscape-test5.tex}{\from{pdflscape.dtx}{test5}}%
%  \file{pdflscape-test6.tex}{\from{pdflscape.dtx}{test6}}%
  \nopreamble
  \nopostamble
%  \file{pdflscape-test6.txt}{\from{pdflscape.dtx}{result6}}%
  \file{ExtractRotate.java}{\from{pdflscape.dtx}{ExtractRotate}}%
%  \usedir{source/latex/oberdiek/catalogue}%
%  \file{pdflscape.xml}{\from{pdflscape.dtx}{catalogue}}%
}

\catcode32=13\relax% active space
\let =\space%
\Msg{************************************************************************}
\Msg{*}
\Msg{* To finish the installation you have to move the following}
\Msg{* file into a directory searched by TeX:}
\Msg{*}
\Msg{*     pdflscape.sty}
\Msg{*}
\Msg{* To produce the documentation run the file `pdflscape.drv'}
\Msg{* through LaTeX.}
\Msg{*}
\Msg{* Happy TeXing!}
\Msg{*}
\Msg{************************************************************************}

\endbatchfile
%</install>
%<*ignore>
\fi
%</ignore>
%<*driver>
\NeedsTeXFormat{LaTeX2e}
\ProvidesFile{pdflscape.drv}%
  [2016/05/14 v0.11 Display of landscape pages in PDF (HO)]%
\documentclass{ltxdoc}
\usepackage{holtxdoc}[2011/11/22]
\usepackage{graphics}
\begin{document}
  \DocInput{pdflscape.dtx}%
\end{document}
%</driver>
% \fi
%
%
% \CharacterTable
%  {Upper-case    \A\B\C\D\E\F\G\H\I\J\K\L\M\N\O\P\Q\R\S\T\U\V\W\X\Y\Z
%   Lower-case    \a\b\c\d\e\f\g\h\i\j\k\l\m\n\o\p\q\r\s\t\u\v\w\x\y\z
%   Digits        \0\1\2\3\4\5\6\7\8\9
%   Exclamation   \!     Double quote  \"     Hash (number) \#
%   Dollar        \$     Percent       \%     Ampersand     \&
%   Acute accent  \'     Left paren    \(     Right paren   \)
%   Asterisk      \*     Plus          \+     Comma         \,
%   Minus         \-     Point         \.     Solidus       \/
%   Colon         \:     Semicolon     \;     Less than     \<
%   Equals        \=     Greater than  \>     Question mark \?
%   Commercial at \@     Left bracket  \[     Backslash     \\
%   Right bracket \]     Circumflex    \^     Underscore    \_
%   Grave accent  \`     Left brace    \{     Vertical bar  \|
%   Right brace   \}     Tilde         \~}
%
% \GetFileInfo{pdflscape.drv}
%
% \title{The \xpackage{pdflscape} package}
% \date{2016/05/14 v0.11}
% \author{Heiko Oberdiek\thanks
% {Please report any issues at \url{https://github.com/ho-tex/oberdiek/issues}}}
%
% \maketitle
%
% \begin{abstract}
% Package \xpackage{pdflscape} adds PDF support to the
% environment \texttt{landscape} of package \xpackage{lscape}
% by setting the PDF page attribute \texttt{/Rotate}.
% \end{abstract}
%
% \tableofcontents
%
% \section{Documentation}
%
% \subsection{Supported drivers}
%
% \begin{itemize}
% \item \xoption{pdftex},\xoption{luatex}
% \item \xoption{dvips}, \xoption{dvipsone},
%       \xoption{pctex32}, \xoption{pctexps}
%       (and other drivers that provide a non-empty \cs{Gin@PS@raw})
% \item \xoption{dvipdfm}
% \end{itemize}
%
% \subsection{Caveat}
%
% Depending on the configuration Ghostscript
% adds a guessed rotation entry by its own.
% This can lead to two |/Rotate| entries per page.
% To prevent this behaviour set the parameter
% |AutoRotatePages| to |/None|, eg:
% \begin{quote}
%   |ps2pdf -dAutoRotatePages=/None|
% \end{quote}
%
% \subsection{Requirements}
%
% \begin{itemize}
% \item The package \xpackage{lscape}.
% \item The package \xpackage{atbegshi} for all drivers
%        except \xoption{pdftex},\xoption{luatex}.
% \end{itemize}
%
% \subsection{Usage}
%
% Load this package instead of or after package \xpackage{lscape}:
% \begin{quote}
%   |\usepackage{pdflscape}|\\
%   |\begin{landscape}|\dots|\end{landscape}|
% \end{quote}
%
% \StopEventually{
% }
%
% \section{Implementation}
%
% \subsection{Package identification}
%
%    \begin{macrocode}
%<*package>
%    \end{macrocode}
%    Package identification.
%    \begin{macrocode}
\NeedsTeXFormat{LaTeX2e}
\ProvidesPackage{pdflscape}%
  [2016/05/14 v0.11 Display of landscape pages in PDF (HO)]
%    \end{macrocode}
%
% \subsection{Driver options}
%
%    \begin{macrocode}
\let\PLS@option\@empty
\let\PLS@driver\@empty
\def\PLS@temp#1#2{%
  \DeclareOption{#1}{%
    \let\PLS@option\CurrentOption
    \def\PLS@driver{#2}%
  }%
}
\PLS@temp{pdftex}{pdftex}
\PLS@temp{luatex}{pdftex}
\PLS@temp{dvips}{dvips}
\PLS@temp{dvipsone}{dvips}
\PLS@temp{pctexps}{dvips}
\PLS@temp{dvipdfm}{dvipdfm}
\PLS@temp{dvipdfmx}{dvipdfm}
\PLS@temp{xetex}{dvipdfm}
\let\PLS@temp\@empty
\DeclareOption*{%
  \ifx\PLS@temp\@empty
    \let\PLS@temp\CurrentOption
  \else
    \edef\PLS@temp{\PLS@temp,\CurrentOption}%
  \fi
}
\ProcessOptions*
\ifx\PLS@option\@empty
\else
  \expandafter\PassOptionsToPackage\expandafter{\PLS@option}{lscape}%
\fi
\ifx\PLS@temp\@empty
\else
  \expandafter\PassOptionsToPackage\expandafter{\PLS@temp}{lscape}%
\fi
\RequirePackage{lscape}
%    \end{macrocode}
%
% \subsection{Autodetection of driver}
%
% \subsubsection{Driver \xoption{pdftex}}
%
%    \begin{macrocode}
\RequirePackage{ifpdf}[2006/02/20]
\def\PLS@temp{pdftex}
\ifpdf
  \ifx\PLS@driver\@empty
    \let\PLS@driver\PLS@temp
    \PackageInfo{pdflscape}{Auto-detected driver: \PLS@temp}%
  \fi
  \ifx\PLS@driver\PLS@temp
  \else
    \PackageError{pdflscape}{%
      The driver option `\PLS@option' is not compatible\MessageBreak
      with pdfTeX in PDF mode, using driver `pdftex' instead%
    }\@ehc
    \let\PLS@driver\PLS@temp
  \fi
\fi
%    \end{macrocode}
%
% \subsubsection{Driver \xoption{xetex}}
%
%    \begin{macrocode}
\RequirePackage{ifxetex}%
\def\PLS@temp{dvipdfm}
\ifxetex
  \ifx\PLS@driver\@empty
    \let\PLS@driver\PLS@temp
    \PackageInfo{pdflscape}{%
      Auto-detected driver: \PLS@temp\space(xetex)%
    }%
  \fi
  \ifx\PLS@driver\PLS@temp
  \else
    \PackageError{pdflscape}{%
      The driver option `\PLS@option' is not compatible\MessageBreak
      with XeTeX, using driver for XeTeX (dvipdfm) instead%
    }\@ehc
    \let\PLS@driver\PLS@temp
  \fi
\fi
%    \end{macrocode}
%
% \subsubsection{Detect driver based on \cs{Gin@driver}}
%
%    \begin{macrocode}
\def\PLS@temp#1#2{%
  \ifx\PLS@driver\@empty
    \begingroup
      \def\PLS@temp{#1.def}%
    \expandafter\endgroup
    \ifx\PLS@temp\Gin@driver
      \def\PLS@driver{#2}%
      \PackageInfo{pdflscape}{%
        Auto-detected driver: \PLS@driver\space(#1)%
      }%
    \fi
  \fi
}
\PLS@temp{dvipdfm}{dvipdfm}
\PLS@temp{dvipdfmx}{dvipdfm}
\PLS@temp{dvips}{dvips}
\PLS@temp{dvipsone}{dvips}
\PLS@temp{pctexps}{dvips}
%    \end{macrocode}
%
% \subsubsection{Driver \xoption{dvips}}
%
%    \begin{macrocode}
\ifx\PLS@driver\@empty
  \def\PLS@temp{dvips}%
  \ifnum 1=0\ifx\Gin@PS@raw\@undefined
            \else
              \ifx\Gin@PS@raw\relax
              \else
                \ifx\Gin@PS@raw\@empty
                \else
                  1%
                \fi
              \fi
            \fi
            \space
    \let\PLS@driver\PLS@temp
    \PackageInfo{pdflscape}{%
      Auto-detected driver: \PLS@temp\space(PostScript)%
    }%
  \fi
\fi
%    \end{macrocode}
%
%    \begin{macrocode}
\ifx\PLS@driver\@empty
  \PackageError{pdflscape}{%
    Neither pdfTeX, PostScript, nor dvipdfm driver found%
  }\@ehc
\fi
%    \end{macrocode}
%
% \subsection{Driver implementation}
%
% \begin{itemize}
% \item |\PLS@AddRotate #1|\\
%   it expects the correct rotation number in |#1|
%   and implements the adding of the /Rotation entry
%   in the |/Page| object of the current page.
% \item |\PLS@RemoveRotate|\\
%   it removes a previous |/Rotate| entry, if necessary.
% \end{itemize}
%
% \subsubsection{\pdfTeX}
%
% Not too nice is the global setting of \cs{pdfpageattr}.
% Perhaps this can be changed in future versions.
%
% luatex compat.
%    \begin{macrocode}
\ifx\pdfpageattr\@undefined
  \ifx\pdfvariable\@undefined\else 
    \protected\edef\pdfpageattr{\pdfvariable pageattr}
  \fi
\fi
%    \end{macrocode}
%    \begin{macrocode}
\def\PLS@temp{pdftex}
\ifx\PLS@temp\PLS@driver
  \def\PLS@AddRotate#1{%
    \ifnum#1=0 %
%    \end{macrocode}
%    Already the default.
%    \begin{macrocode}
    \else
      \global\pdfpageattr\expandafter{%
        \the\pdfpageattr
        /Rotate #1%
      }%
    \fi
  }%
%    \end{macrocode}
%    Removes a |/Rotate| entry.
%    \begin{macrocode}
  \def\PLS@RemoveRotate{%
    \begingroup
      \global\pdfpageattr\expandafter{\expandafter}%
      \expandafter\PLS@@RemoveRotate
        \the\pdfpageattr /Rotate\@nil
    \endgroup
  }%
  \def\PLS@@RemoveRotate#1/Rotate#2\@nil{%
%    \end{macrocode}
%    Append |/Rotate| free stuff to \cs{pdfpageattr}.
%    \begin{macrocode}
    \global\pdfpageattr\expandafter{\the\pdfpageattr#1}%
    \ifx\\#2\\%
%    \end{macrocode}
%    Ready, because the detected |/Rotate| is part of
%    the end marker: |/Rotate\@nil|
%    \begin{macrocode}
    \else
%    \end{macrocode}
%    First read in the argument of |/Rotate|,
%    then continue parsing.
%    \begin{macrocode}
      \afterassignment\PLS@@RemoveRotate
      \count0=#2\@nil
    \fi
  }%
\fi
%    \end{macrocode}
%
% \subsubsection{PostScript driver}
%
%    \begin{macrocode}
\def\PLS@temp{dvips}
\ifx\PLS@temp\PLS@driver
  \RequirePackage{atbegshi}%
  \AtBeginShipout{\PLS@AtBeginShipout}%
  \let\PLS@AtBeginShipout\@empty
  \def\PLS@AddRotate#1{%
    \ifnum#1=0 %
    \else
      \def\PLS@AtBeginShipout{%
        \global\setbox\AtBeginShipoutBox\vbox{%
          \Gin@PS@raw{%
            [{ThisPage}\string<\string</Rotate #1\string>\string>%
            /PUT pdfmark%
          }%
          \box\AtBeginShipoutBox
        }%
      }%
    \fi
  }%
  \def\PLS@RemoveRotate{\let\PLS@AtBeginShipout\@empty}%
\fi
%    \end{macrocode}
%
% \subsubsection{Driver dvipdfm}
%
%    \begin{macrocode}
\def\PLS@temp{dvipdfm}
\ifx\PLS@temp\PLS@driver
  \RequirePackage{atbegshi}%
  \AtBeginShipout{\PLS@AtBeginShipout}%
  \let\PLS@AtBeginShipout\@empty
  \def\PLS@AddRotate#1{%
    \ifnum#1=0 %
    \else
      \def\PLS@AtBeginShipout{%
        \global\setbox\AtBeginShipoutBox\vbox{%
          \special{pdf: put @thispage <</Rotate #1>>}%
          \box\AtBeginShipoutBox
        }%
      }%
    \fi
  }%
  \def\PLS@RemoveRotate{\let\PLS@AtBeginShipout\@empty}%
\fi
%    \end{macrocode}
%
% \subsection{Driver independent stuff}
%
%    The landscape environment is extended by
%    adding the correct |/Rotate| entries.
%    \begin{macrocode}
\g@addto@macro{\landscape}{\PLS@Rotate{90}}
\g@addto@macro{\endlandscape}{\PLS@Rotate{0}}
%    \end{macrocode}
%
%    \begin{macro}{\PLS@Rotate}
%    Main macro, that sets the |/Rotate| entry.\\
%    Argument: any TeX number or nothing, that means zero.\\
%    Driver independent.
%    \begin{macrocode}
\def\PLS@Rotate#1{%
  \begingroup
%    \end{macrocode}
%    1. Check and validate the argument.
%    \begin{macrocode}
    \PLS@CheckAngle{#1}%
  \expandafter\endgroup
%    \end{macrocode}
%    2. Remove previous |\Rotate| entry.
%    \begin{macrocode}
  \expandafter\PLS@RemoveRotate
%    \end{macrocode}
%    3. Add |/Rotate| entry.
%    \begin{macrocode}
  \expandafter\PLS@AddRotate\expandafter{\the\count@}%
}
%    \end{macrocode}
%    \end{macro}
%
%    \begin{macro}{\PLS@CheckAngle}
%    Validates the rotation angle.\\
%    The result is stored in the count register |\count@|.\\
%    Driver independent.
%    \begin{macrocode}
\def\PLS@CheckAngle#1{%
%    \end{macrocode}
%    Nothing means zero:
%    \begin{macrocode}
  \ifx\\#1\\%
    \count@=0 %
  \else
    \count@=#1\relax
  \fi
%    \end{macrocode}
%    Normalize to interval -360 $<$ |\count@| $<$ 360.
%    \begin{macrocode}
  \@whilenum\count@>359\do{\advance\count@ -360 }%
  \@whilenum\count@<-359\do{\advance\count@ 360 }%
%    \end{macrocode}
%    Check values: 0, +/-90, +/-180, +/-270
%    \begin{macrocode}
  \ifnum 1=0\ifnum\count@=0 \else
            \ifnum\count@=90 \else
            \ifnum\count@=-90 \else
            \ifnum\count@=180 \else
            \ifnum\count@=-180 \else
            \ifnum\count@=270 \else
            \ifnum\count@=-270 \else
           1\fi\fi\fi\fi\fi\fi\fi\relax
    \PackageError{pdflscape}{%
      Invalid value \the\count@\space for /Rotate%
    }\@ehc
    \count@=0 %
  \fi
}
%    \end{macrocode}
%    \end{macro}
%
%    \begin{macrocode}
%</package>
%    \end{macrocode}
%
% \section{Test}
%
% \subsection{Driver detection tests}
%    \begin{macrocode}
%<*test0>
\NeedsTeXFormat{LaTeX2e}
\documentclass{minimal}
\usepackage{qstest}
\IncludeTests{*}
\LogTests{log}{*}{*}
%</test0>
%    \end{macrocode}
%
%    \begin{macrocode}
%<*test1>
\usepackage{ifpdf}
\ifpdf
  \def\ExpectDriver{pdftex}%
\fi
\usepackage{ifxetex}%
\ifxetex
  \def\ExpectDriver{dvipdfm}%
\fi
\usepackage{pdflscape}[2016/05/14]
%</test1>
%    \end{macrocode}
%
%    \begin{macrocode}
%<*test2>
\def\ExpectDriver{pdftex}
\usepackage[pdftex]{pdflscape}[2016/05/14]
%</test2>
%    \end{macrocode}
%
%    \begin{macrocode}
%<*test3>
\def\ExpectDriver{dvipdfm}
\usepackage[dvipdfmx]{pdflscape}[2016/05/14]
%</test3>
%    \end{macrocode}
%
%    \begin{macrocode}
%<*test4>
\def\ExpectDriver{dvips}
\usepackage[dvipsone]{graphics}
\usepackage{pdflscape}[2016/05/14]
%</test4>
%    \end{macrocode}
%
%    \begin{macrocode}
%<*test0>
\begin{document}
\begin{qstest}{driver}{driver}
  \makeatletter
  \Expect*{\PLS@driver}*{\ExpectDriver}%
\end{qstest}
\end{document}
%</test0>
%    \end{macrocode}
%
% \subsection{Test for \cs{PLS@CheckAngle}}
%
%    \begin{macrocode}
%<*test5>
\NeedsTeXFormat{LaTeX2e}
\documentclass{minimal}
\usepackage{qstest}
\IncludeTests{*}
\LogTests{log}{*}{*}
\usepackage{pdflscape}[2016/05/14]
\begin{document}
\begin{qstest}{checkangle}{checkangle}
  \makeatletter
  \def\Result{%
    \def\ResultString{\the\count@}%
  }%
  \def\PackageError#1#2#3{%
    \def\Result{%
      \def\ResultString{error}%
    }%
  }%
  \def\Test#1#2{%
    \begingroup
      \PLS@CheckAngle{#1}%
      \Result
      \Expect*{\ResultString}{#2}%
    \endgroup
  }%
  \Test{0}{0}%
  \Test{90}{90}%
  \Test{180}{180}%
  \Test{270}{270}%
  \Test{360}{0}%
  \Test{450}{90}%
  \Test{540}{180}%
  \Test{630}{270}%
  \Test{720}{0}%
  \Test{3600}{0}%
  \Test{3690}{90}%
  \Test{-90}{270}%
  \Test{-180}{180}%
  \Test{-270}{-90}%
  \Test{-360}{0}%
  \Test{-450}{270}%
  \Test{540}{180}%
  \Test{-630}{90}%
  \Test{-720}{0}%
  \Test{-3600}{0}%
  \Test{-3690}{270}%
  \Test{1}{error}%
  \Test{-1}{error}%
  \Test{123}{error}%
\end{qstest}
\end{document}
%</test5>
%    \end{macrocode}
%
% \subsection{Test for rotate}
%
%    \begin{macrocode}
%<*test6>
\NeedsTeXFormat{LaTeX2e}
\documentclass{article}
\usepackage{pdflscape}[2016/05/14]
\begin{document}
\section{Portrait page}
\begin{landscape}
\section{Landscape page}
Foo bar foo bar foo bar foo bar foo bar
foo bar foo bar foo bar foo bar foo bar.
\end{landscape}
\section{Portrait Section}
Foo bar
\begin{landscape}
\section{Again Landscape Section}
\newpage
\section{Second Page of Landscape Section}
\end{landscape}
\section{Second Last Portrait Page}
\newpage
\section{Last Portrait Page}
\end{document}
%</test6>
%    \end{macrocode}
%
%    \begin{macrocode}
%<*result6>
/Page 1 /Rotate 0
/Page 2 /Rotate 90
/Page 3 /Rotate 0
/Page 4 /Rotate 90
/Page 5 /Rotate 90
/Page 6 /Rotate 0
/Page 7 /Rotate 0
%</result6>
%    \end{macrocode}
%
%    \begin{macrocode}
%<*ExtractRotate>
/**
 * ExtractRotate.java
 *
 * Copyright (C) 2007 by Heiko Oberdiek <heiko.oberdiek at googlemail.com>
 *
 * Requires: PDFBox (http://www.pdfbox.org/)
 *
 * Syntax: java ExtractRotate <pdffile> <textfile>
 *
 * The <pdffile> is analyzed and for each page its rotation
 * setting is printed in the <textfile>. Example:
 *   /Page 1 /Rotate 0
 *   /Page 2 /Rotate 90
 */
import java.io.FileInputStream;
import java.io.FileWriter;
import org.pdfbox.pdfparser.PDFParser;
import org.pdfbox.pdmodel.PDDocument;
import org.pdfbox.pdmodel.PDDocumentCatalog;
import org.pdfbox.pdmodel.PDPage;

public class ExtractRotate {

    public static void main(String[] args) {
        try {
            String infile = args[0];
            String outfile = args[1];
            FileWriter out = new FileWriter(outfile);
            PDFParser parser =
                    new PDFParser(new FileInputStream(infile));
            parser.parse();
            PDDocument document = parser.getPDDocument();
            PDDocumentCatalog catalog = document.getDocumentCatalog();
            int i = 0;
            for (Object page: catalog.getAllPages()) {
                i++;
                out.write("/Page " + i + " " + "/Rotate "
                        + ((PDPage)page).findRotation() + "\n");
            }
            document.close();
            out.close();
        }
        catch (Exception e) {
            e.printStackTrace();
            System.exit(1);
        }
    }
}
%</ExtractRotate>
%    \end{macrocode}
%
% \section{Installation}
%
% \subsection{Download}
%
% \paragraph{Package.} This package is available on
% CTAN\footnote{\CTANpkg{pdflscape}}:
% \begin{description}
% \item[\CTAN{macros/latex/contrib/oberdiek/pdflscape.dtx}] The source file.
% \item[\CTAN{macros/latex/contrib/oberdiek/pdflscape.pdf}] Documentation.
% \end{description}
%
%
% \paragraph{Bundle.} All the packages of the bundle `oberdiek'
% are also available in a TDS compliant ZIP archive. There
% the packages are already unpacked and the documentation files
% are generated. The files and directories obey the TDS standard.
% \begin{description}
% \item[\CTANinstall{install/macros/latex/contrib/oberdiek.tds.zip}]
% \end{description}
% \emph{TDS} refers to the standard ``A Directory Structure
% for \TeX\ Files'' (\CTAN{tds/tds.pdf}). Directories
% with \xfile{texmf} in their name are usually organized this way.
%
% \subsection{Bundle installation}
%
% \paragraph{Unpacking.} Unpack the \xfile{oberdiek.tds.zip} in the
% TDS tree (also known as \xfile{texmf} tree) of your choice.
% Example (linux):
% \begin{quote}
%   |unzip oberdiek.tds.zip -d ~/texmf|
% \end{quote}
%
% \paragraph{Script installation.}
% Check the directory \xfile{TDS:scripts/oberdiek/} for
% scripts that need further installation steps.
%
% \subsection{Package installation}
%
% \paragraph{Unpacking.} The \xfile{.dtx} file is a self-extracting
% \docstrip\ archive. The files are extracted by running the
% \xfile{.dtx} through \plainTeX:
% \begin{quote}
%   \verb|tex pdflscape.dtx|
% \end{quote}
%
% \paragraph{TDS.} Now the different files must be moved into
% the different directories in your installation TDS tree
% (also known as \xfile{texmf} tree):
% \begin{quote}
% \def\t{^^A
% \begin{tabular}{@{}>{\ttfamily}l@{ $\rightarrow$ }>{\ttfamily}l@{}}
%   pdflscape.sty & tex/latex/oberdiek/pdflscape.sty\\
%   pdflscape.pdf & doc/latex/oberdiek/pdflscape.pdf\\
%   test/pdflscape-test1.tex & doc/latex/oberdiek/test/pdflscape-test1.tex\\
%   test/pdflscape-test2.tex & doc/latex/oberdiek/test/pdflscape-test2.tex\\
%   test/pdflscape-test3.tex & doc/latex/oberdiek/test/pdflscape-test3.tex\\
%   test/pdflscape-test4.tex & doc/latex/oberdiek/test/pdflscape-test4.tex\\
%   test/pdflscape-test5.tex & doc/latex/oberdiek/test/pdflscape-test5.tex\\
%   test/pdflscape-test6.tex & doc/latex/oberdiek/test/pdflscape-test6.tex\\
%   test/pdflscape-test6.txt & doc/latex/oberdiek/test/pdflscape-test6.txt\\
%   test/ExtractRotate.java & doc/latex/oberdiek/test/ExtractRotate.java\\
%   pdflscape.dtx & source/latex/oberdiek/pdflscape.dtx\\
% \end{tabular}^^A
% }^^A
% \sbox0{\t}^^A
% \ifdim\wd0>\linewidth
%   \begingroup
%     \advance\linewidth by\leftmargin
%     \advance\linewidth by\rightmargin
%   \edef\x{\endgroup
%     \def\noexpand\lw{\the\linewidth}^^A
%   }\x
%   \def\lwbox{^^A
%     \leavevmode
%     \hbox to \linewidth{^^A
%       \kern-\leftmargin\relax
%       \hss
%       \usebox0
%       \hss
%       \kern-\rightmargin\relax
%     }^^A
%   }^^A
%   \ifdim\wd0>\lw
%     \sbox0{\small\t}^^A
%     \ifdim\wd0>\linewidth
%       \ifdim\wd0>\lw
%         \sbox0{\footnotesize\t}^^A
%         \ifdim\wd0>\linewidth
%           \ifdim\wd0>\lw
%             \sbox0{\scriptsize\t}^^A
%             \ifdim\wd0>\linewidth
%               \ifdim\wd0>\lw
%                 \sbox0{\tiny\t}^^A
%                 \ifdim\wd0>\linewidth
%                   \lwbox
%                 \else
%                   \usebox0
%                 \fi
%               \else
%                 \lwbox
%               \fi
%             \else
%               \usebox0
%             \fi
%           \else
%             \lwbox
%           \fi
%         \else
%           \usebox0
%         \fi
%       \else
%         \lwbox
%       \fi
%     \else
%       \usebox0
%     \fi
%   \else
%     \lwbox
%   \fi
% \else
%   \usebox0
% \fi
% \end{quote}
% If you have a \xfile{docstrip.cfg} that configures and enables \docstrip's
% TDS installing feature, then some files can already be in the right
% place, see the documentation of \docstrip.
%
% \subsection{Refresh file name databases}
%
% If your \TeX~distribution
% (\TeX\,Live, \mikTeX, \dots) relies on file name databases, you must refresh
% these. For example, \TeX\,Live\ users run \verb|texhash| or
% \verb|mktexlsr|.
%
% \subsection{Some details for the interested}
%
% \paragraph{Unpacking with \LaTeX.}
% The \xfile{.dtx} chooses its action depending on the format:
% \begin{description}
% \item[\plainTeX:] Run \docstrip\ and extract the files.
% \item[\LaTeX:] Generate the documentation.
% \end{description}
% If you insist on using \LaTeX\ for \docstrip\ (really,
% \docstrip\ does not need \LaTeX), then inform the autodetect routine
% about your intention:
% \begin{quote}
%   \verb|latex \let\install=y% \iffalse meta-comment
%
% File: pdflscape.dtx
% Version: 2016/05/14 v0.11
% Info: Display of landscape pages in PDF
%
% Copyright (C) 2001, 2004, 2006-2008 by
%    Heiko Oberdiek <heiko.oberdiek at googlemail.com>
%    2016
%    https://github.com/ho-tex/oberdiek/issues
%
% This work may be distributed and/or modified under the
% conditions of the LaTeX Project Public License, either
% version 1.3c of this license or (at your option) any later
% version. This version of this license is in
%    http://www.latex-project.org/lppl/lppl-1-3c.txt
% and the latest version of this license is in
%    http://www.latex-project.org/lppl.txt
% and version 1.3 or later is part of all distributions of
% LaTeX version 2005/12/01 or later.
%
% This work has the LPPL maintenance status "maintained".
%
% This Current Maintainer of this work is Heiko Oberdiek.
%
% This work consists of the main source file pdflscape.dtx
% and the derived files
%    pdflscape.sty, pdflscape.pdf, pdflscape.ins, pdflscape.drv,
%    pdflscape-test1.tex, pdflscape-test2.tex,
%    pdflscape-test3.tex, pdflscape-test4.tex,
%    pdflscape-test5.tex, pdflscape-test6.tex,
%    pdflscape-test6.txt, ExtractRotate.java.
%
% Distribution:
%    CTAN:macros/latex/contrib/oberdiek/pdflscape.dtx
%    CTAN:macros/latex/contrib/oberdiek/pdflscape.pdf
%
% Unpacking:
%    (a) If pdflscape.ins is present:
%           tex pdflscape.ins
%    (b) Without pdflscape.ins:
%           tex pdflscape.dtx
%    (c) If you insist on using LaTeX
%           latex \let\install=y% \iffalse meta-comment
%
% File: pdflscape.dtx
% Version: 2016/05/14 v0.11
% Info: Display of landscape pages in PDF
%
% Copyright (C) 2001, 2004, 2006-2008 by
%    Heiko Oberdiek <heiko.oberdiek at googlemail.com>
%    2016
%    https://github.com/ho-tex/oberdiek/issues
%
% This work may be distributed and/or modified under the
% conditions of the LaTeX Project Public License, either
% version 1.3c of this license or (at your option) any later
% version. This version of this license is in
%    http://www.latex-project.org/lppl/lppl-1-3c.txt
% and the latest version of this license is in
%    http://www.latex-project.org/lppl.txt
% and version 1.3 or later is part of all distributions of
% LaTeX version 2005/12/01 or later.
%
% This work has the LPPL maintenance status "maintained".
%
% This Current Maintainer of this work is Heiko Oberdiek.
%
% This work consists of the main source file pdflscape.dtx
% and the derived files
%    pdflscape.sty, pdflscape.pdf, pdflscape.ins, pdflscape.drv,
%    pdflscape-test1.tex, pdflscape-test2.tex,
%    pdflscape-test3.tex, pdflscape-test4.tex,
%    pdflscape-test5.tex, pdflscape-test6.tex,
%    pdflscape-test6.txt, ExtractRotate.java.
%
% Distribution:
%    CTAN:macros/latex/contrib/oberdiek/pdflscape.dtx
%    CTAN:macros/latex/contrib/oberdiek/pdflscape.pdf
%
% Unpacking:
%    (a) If pdflscape.ins is present:
%           tex pdflscape.ins
%    (b) Without pdflscape.ins:
%           tex pdflscape.dtx
%    (c) If you insist on using LaTeX
%           latex \let\install=y% \iffalse meta-comment
%
% File: pdflscape.dtx
% Version: 2016/05/14 v0.11
% Info: Display of landscape pages in PDF
%
% Copyright (C) 2001, 2004, 2006-2008 by
%    Heiko Oberdiek <heiko.oberdiek at googlemail.com>
%    2016
%    https://github.com/ho-tex/oberdiek/issues
%
% This work may be distributed and/or modified under the
% conditions of the LaTeX Project Public License, either
% version 1.3c of this license or (at your option) any later
% version. This version of this license is in
%    http://www.latex-project.org/lppl/lppl-1-3c.txt
% and the latest version of this license is in
%    http://www.latex-project.org/lppl.txt
% and version 1.3 or later is part of all distributions of
% LaTeX version 2005/12/01 or later.
%
% This work has the LPPL maintenance status "maintained".
%
% This Current Maintainer of this work is Heiko Oberdiek.
%
% This work consists of the main source file pdflscape.dtx
% and the derived files
%    pdflscape.sty, pdflscape.pdf, pdflscape.ins, pdflscape.drv,
%    pdflscape-test1.tex, pdflscape-test2.tex,
%    pdflscape-test3.tex, pdflscape-test4.tex,
%    pdflscape-test5.tex, pdflscape-test6.tex,
%    pdflscape-test6.txt, ExtractRotate.java.
%
% Distribution:
%    CTAN:macros/latex/contrib/oberdiek/pdflscape.dtx
%    CTAN:macros/latex/contrib/oberdiek/pdflscape.pdf
%
% Unpacking:
%    (a) If pdflscape.ins is present:
%           tex pdflscape.ins
%    (b) Without pdflscape.ins:
%           tex pdflscape.dtx
%    (c) If you insist on using LaTeX
%           latex \let\install=y\input{pdflscape.dtx}
%        (quote the arguments according to the demands of your shell)
%
% Documentation:
%    (a) If pdflscape.drv is present:
%           latex pdflscape.drv
%    (b) Without pdflscape.drv:
%           latex pdflscape.dtx; ...
%    The class ltxdoc loads the configuration file ltxdoc.cfg
%    if available. Here you can specify further options, e.g.
%    use A4 as paper format:
%       \PassOptionsToClass{a4paper}{article}
%
%    Programm calls to get the documentation (example):
%       pdflatex pdflscape.dtx
%       makeindex -s gind.ist pdflscape.idx
%       pdflatex pdflscape.dtx
%       makeindex -s gind.ist pdflscape.idx
%       pdflatex pdflscape.dtx
%
% Installation:
%    TDS:tex/latex/oberdiek/pdflscape.sty
%    TDS:doc/latex/oberdiek/pdflscape.pdf
%    TDS:doc/latex/oberdiek/test/pdflscape-test1.tex
%    TDS:doc/latex/oberdiek/test/pdflscape-test2.tex
%    TDS:doc/latex/oberdiek/test/pdflscape-test3.tex
%    TDS:doc/latex/oberdiek/test/pdflscape-test4.tex
%    TDS:doc/latex/oberdiek/test/pdflscape-test5.tex
%    TDS:doc/latex/oberdiek/test/pdflscape-test6.tex
%    TDS:doc/latex/oberdiek/test/pdflscape-test6.txt
%    TDS:doc/latex/oberdiek/test/ExtractRotate.java
%    TDS:source/latex/oberdiek/pdflscape.dtx
%
%<*ignore>
\begingroup
  \catcode123=1 %
  \catcode125=2 %
  \def\x{LaTeX2e}%
\expandafter\endgroup
\ifcase 0\ifx\install y1\fi\expandafter
         \ifx\csname processbatchFile\endcsname\relax\else1\fi
         \ifx\fmtname\x\else 1\fi\relax
\else\csname fi\endcsname
%</ignore>
%<*install>
\input docstrip.tex
\Msg{************************************************************************}
\Msg{* Installation}
\Msg{* Package: pdflscape 2016/05/14 v0.11 Display of landscape pages in PDF (HO)}
\Msg{************************************************************************}

\keepsilent
\askforoverwritefalse

\let\MetaPrefix\relax
\preamble

This is a generated file.

Project: pdflscape
Version: 2016/05/14 v0.11

Copyright (C) 2001, 2004, 2006-2008 by
   Heiko Oberdiek <heiko.oberdiek at googlemail.com>

This work may be distributed and/or modified under the
conditions of the LaTeX Project Public License, either
version 1.3c of this license or (at your option) any later
version. This version of this license is in
   http://www.latex-project.org/lppl/lppl-1-3c.txt
and the latest version of this license is in
   http://www.latex-project.org/lppl.txt
and version 1.3 or later is part of all distributions of
LaTeX version 2005/12/01 or later.

This work has the LPPL maintenance status "maintained".

This Current Maintainer of this work is Heiko Oberdiek.

This work consists of the main source file pdflscape.dtx
and the derived files
   pdflscape.sty, pdflscape.pdf, pdflscape.ins, pdflscape.drv,
   pdflscape-test1.tex, pdflscape-test2.tex,
   pdflscape-test3.tex, pdflscape-test4.tex,
   pdflscape-test5.tex, pdflscape-test6.tex,
   pdflscape-test6.txt, ExtractRotate.java.

\endpreamble
\let\MetaPrefix\DoubleperCent

\generate{%
  \file{pdflscape.ins}{\from{pdflscape.dtx}{install}}%
  \file{pdflscape.drv}{\from{pdflscape.dtx}{driver}}%
  \usedir{tex/latex/oberdiek}%
  \file{pdflscape.sty}{\from{pdflscape.dtx}{package}}%
%  \usedir{doc/latex/oberdiek/test}%
%  \file{pdflscape-test1.tex}{\from{pdflscape.dtx}{test0,test1}}%
%  \file{pdflscape-test2.tex}{\from{pdflscape.dtx}{test0,test2}}%
%  \file{pdflscape-test3.tex}{\from{pdflscape.dtx}{test0,test3}}%
%  \file{pdflscape-test4.tex}{\from{pdflscape.dtx}{test0,test4}}%
%  \file{pdflscape-test5.tex}{\from{pdflscape.dtx}{test5}}%
%  \file{pdflscape-test6.tex}{\from{pdflscape.dtx}{test6}}%
  \nopreamble
  \nopostamble
%  \file{pdflscape-test6.txt}{\from{pdflscape.dtx}{result6}}%
  \file{ExtractRotate.java}{\from{pdflscape.dtx}{ExtractRotate}}%
%  \usedir{source/latex/oberdiek/catalogue}%
%  \file{pdflscape.xml}{\from{pdflscape.dtx}{catalogue}}%
}

\catcode32=13\relax% active space
\let =\space%
\Msg{************************************************************************}
\Msg{*}
\Msg{* To finish the installation you have to move the following}
\Msg{* file into a directory searched by TeX:}
\Msg{*}
\Msg{*     pdflscape.sty}
\Msg{*}
\Msg{* To produce the documentation run the file `pdflscape.drv'}
\Msg{* through LaTeX.}
\Msg{*}
\Msg{* Happy TeXing!}
\Msg{*}
\Msg{************************************************************************}

\endbatchfile
%</install>
%<*ignore>
\fi
%</ignore>
%<*driver>
\NeedsTeXFormat{LaTeX2e}
\ProvidesFile{pdflscape.drv}%
  [2016/05/14 v0.11 Display of landscape pages in PDF (HO)]%
\documentclass{ltxdoc}
\usepackage{holtxdoc}[2011/11/22]
\usepackage{graphics}
\begin{document}
  \DocInput{pdflscape.dtx}%
\end{document}
%</driver>
% \fi
%
%
% \CharacterTable
%  {Upper-case    \A\B\C\D\E\F\G\H\I\J\K\L\M\N\O\P\Q\R\S\T\U\V\W\X\Y\Z
%   Lower-case    \a\b\c\d\e\f\g\h\i\j\k\l\m\n\o\p\q\r\s\t\u\v\w\x\y\z
%   Digits        \0\1\2\3\4\5\6\7\8\9
%   Exclamation   \!     Double quote  \"     Hash (number) \#
%   Dollar        \$     Percent       \%     Ampersand     \&
%   Acute accent  \'     Left paren    \(     Right paren   \)
%   Asterisk      \*     Plus          \+     Comma         \,
%   Minus         \-     Point         \.     Solidus       \/
%   Colon         \:     Semicolon     \;     Less than     \<
%   Equals        \=     Greater than  \>     Question mark \?
%   Commercial at \@     Left bracket  \[     Backslash     \\
%   Right bracket \]     Circumflex    \^     Underscore    \_
%   Grave accent  \`     Left brace    \{     Vertical bar  \|
%   Right brace   \}     Tilde         \~}
%
% \GetFileInfo{pdflscape.drv}
%
% \title{The \xpackage{pdflscape} package}
% \date{2016/05/14 v0.11}
% \author{Heiko Oberdiek\thanks
% {Please report any issues at \url{https://github.com/ho-tex/oberdiek/issues}}\\
% \xemail{heiko.oberdiek at googlemail.com}}
%
% \maketitle
%
% \begin{abstract}
% Package \xpackage{pdflscape} adds PDF support to the
% environment \texttt{landscape} of package \xpackage{lscape}
% by setting the PDF page attribute \texttt{/Rotate}.
% \end{abstract}
%
% \tableofcontents
%
% \section{Documentation}
%
% \subsection{Supported drivers}
%
% \begin{itemize}
% \item \xoption{pdftex},\xoption{luatex}
% \item \xoption{dvips}, \xoption{dvipsone},
%       \xoption{pctex32}, \xoption{pctexps}
%       (and other drivers that provide a non-empty \cs{Gin@PS@raw})
% \item \xoption{dvipdfm}
% \end{itemize}
%
% \subsection{Caveat}
%
% Depending on the configuration Ghostscript
% adds a guessed rotation entry by its own.
% This can lead to two |/Rotate| entries per page.
% To prevent this behaviour set the parameter
% |AutoRotatePages| to |/None|, eg:
% \begin{quote}
%   |ps2pdf -dAutoRotatePages=/None|
% \end{quote}
%
% \subsection{Requirements}
%
% \begin{itemize}
% \item The package \xpackage{lscape}.
% \item The package \xpackage{atbegshi} for all drivers
%        except \xoption{pdftex},\xoption{luatex}.
% \end{itemize}
%
% \subsection{Usage}
%
% Load this package instead of or after package \xpackage{lscape}:
% \begin{quote}
%   |\usepackage{pdflscape}|\\
%   |\begin{landscape}|\dots|\end{landscape}|
% \end{quote}
%
% \StopEventually{
% }
%
% \section{Implementation}
%
% \subsection{Package identification}
%
%    \begin{macrocode}
%<*package>
%    \end{macrocode}
%    Package identification.
%    \begin{macrocode}
\NeedsTeXFormat{LaTeX2e}
\ProvidesPackage{pdflscape}%
  [2016/05/14 v0.11 Display of landscape pages in PDF (HO)]
%    \end{macrocode}
%
% \subsection{Driver options}
%
%    \begin{macrocode}
\let\PLS@option\@empty
\let\PLS@driver\@empty
\def\PLS@temp#1#2{%
  \DeclareOption{#1}{%
    \let\PLS@option\CurrentOption
    \def\PLS@driver{#2}%
  }%
}
\PLS@temp{pdftex}{pdftex}
\PLS@temp{luatex}{pdftex}
\PLS@temp{dvips}{dvips}
\PLS@temp{dvipsone}{dvips}
\PLS@temp{pctexps}{dvips}
\PLS@temp{dvipdfm}{dvipdfm}
\PLS@temp{dvipdfmx}{dvipdfm}
\PLS@temp{xetex}{dvipdfm}
\let\PLS@temp\@empty
\DeclareOption*{%
  \ifx\PLS@temp\@empty
    \let\PLS@temp\CurrentOption
  \else
    \edef\PLS@temp{\PLS@temp,\CurrentOption}%
  \fi
}
\ProcessOptions*
\ifx\PLS@option\@empty
\else
  \expandafter\PassOptionsToPackage\expandafter{\PLS@option}{lscape}%
\fi
\ifx\PLS@temp\@empty
\else
  \expandafter\PassOptionsToPackage\expandafter{\PLS@temp}{lscape}%
\fi
\RequirePackage{lscape}
%    \end{macrocode}
%
% \subsection{Autodetection of driver}
%
% \subsubsection{Driver \xoption{pdftex}}
%
%    \begin{macrocode}
\RequirePackage{ifpdf}[2006/02/20]
\def\PLS@temp{pdftex}
\ifpdf
  \ifx\PLS@driver\@empty
    \let\PLS@driver\PLS@temp
    \PackageInfo{pdflscape}{Auto-detected driver: \PLS@temp}%
  \fi
  \ifx\PLS@driver\PLS@temp
  \else
    \PackageError{pdflscape}{%
      The driver option `\PLS@option' is not compatible\MessageBreak
      with pdfTeX in PDF mode, using driver `pdftex' instead%
    }\@ehc
    \let\PLS@driver\PLS@temp
  \fi
\fi
%    \end{macrocode}
%
% \subsubsection{Driver \xoption{xetex}}
%
%    \begin{macrocode}
\RequirePackage{ifxetex}%
\def\PLS@temp{dvipdfm}
\ifxetex
  \ifx\PLS@driver\@empty
    \let\PLS@driver\PLS@temp
    \PackageInfo{pdflscape}{%
      Auto-detected driver: \PLS@temp\space(xetex)%
    }%
  \fi
  \ifx\PLS@driver\PLS@temp
  \else
    \PackageError{pdflscape}{%
      The driver option `\PLS@option' is not compatible\MessageBreak
      with XeTeX, using driver for XeTeX (dvipdfm) instead%
    }\@ehc
    \let\PLS@driver\PLS@temp
  \fi
\fi
%    \end{macrocode}
%
% \subsubsection{Detect driver based on \cs{Gin@driver}}
%
%    \begin{macrocode}
\def\PLS@temp#1#2{%
  \ifx\PLS@driver\@empty
    \begingroup
      \def\PLS@temp{#1.def}%
    \expandafter\endgroup
    \ifx\PLS@temp\Gin@driver
      \def\PLS@driver{#2}%
      \PackageInfo{pdflscape}{%
        Auto-detected driver: \PLS@driver\space(#1)%
      }%
    \fi
  \fi
}
\PLS@temp{dvipdfm}{dvipdfm}
\PLS@temp{dvipdfmx}{dvipdfm}
\PLS@temp{dvips}{dvips}
\PLS@temp{dvipsone}{dvips}
\PLS@temp{pctexps}{dvips}
%    \end{macrocode}
%
% \subsubsection{Driver \xoption{dvips}}
%
%    \begin{macrocode}
\ifx\PLS@driver\@empty
  \def\PLS@temp{dvips}%
  \ifnum 1=0\ifx\Gin@PS@raw\@undefined
            \else
              \ifx\Gin@PS@raw\relax
              \else
                \ifx\Gin@PS@raw\@empty
                \else
                  1%
                \fi
              \fi
            \fi
            \space
    \let\PLS@driver\PLS@temp
    \PackageInfo{pdflscape}{%
      Auto-detected driver: \PLS@temp\space(PostScript)%
    }%
  \fi
\fi
%    \end{macrocode}
%
%    \begin{macrocode}
\ifx\PLS@driver\@empty
  \PackageError{pdflscape}{%
    Neither pdfTeX, PostScript, nor dvipdfm driver found%
  }\@ehc
\fi
%    \end{macrocode}
%
% \subsection{Driver implementation}
%
% \begin{itemize}
% \item |\PLS@AddRotate #1|\\
%   it expects the correct rotation number in |#1|
%   and implements the adding of the /Rotation entry
%   in the |/Page| object of the current page.
% \item |\PLS@RemoveRotate|\\
%   it removes a previous |/Rotate| entry, if necessary.
% \end{itemize}
%
% \subsubsection{\pdfTeX}
%
% Not too nice is the global setting of \cs{pdfpageattr}.
% Perhaps this can be changed in future versions.
%
% luatex compat.
%    \begin{macrocode}
\ifx\pdfpageattr\@undefined
  \ifx\pdfvariable\@undefined\else 
    \protected\edef\pdfpageattr{\pdfvariable pageattr}
  \fi
\fi
%    \end{macrocode}
%    \begin{macrocode}
\def\PLS@temp{pdftex}
\ifx\PLS@temp\PLS@driver
  \def\PLS@AddRotate#1{%
    \ifnum#1=0 %
%    \end{macrocode}
%    Already the default.
%    \begin{macrocode}
    \else
      \global\pdfpageattr\expandafter{%
        \the\pdfpageattr
        /Rotate #1%
      }%
    \fi
  }%
%    \end{macrocode}
%    Removes a |/Rotate| entry.
%    \begin{macrocode}
  \def\PLS@RemoveRotate{%
    \begingroup
      \global\pdfpageattr\expandafter{\expandafter}%
      \expandafter\PLS@@RemoveRotate
        \the\pdfpageattr /Rotate\@nil
    \endgroup
  }%
  \def\PLS@@RemoveRotate#1/Rotate#2\@nil{%
%    \end{macrocode}
%    Append |/Rotate| free stuff to \cs{pdfpageattr}.
%    \begin{macrocode}
    \global\pdfpageattr\expandafter{\the\pdfpageattr#1}%
    \ifx\\#2\\%
%    \end{macrocode}
%    Ready, because the detected |/Rotate| is part of
%    the end marker: |/Rotate\@nil|
%    \begin{macrocode}
    \else
%    \end{macrocode}
%    First read in the argument of |/Rotate|,
%    then continue parsing.
%    \begin{macrocode}
      \afterassignment\PLS@@RemoveRotate
      \count0=#2\@nil
    \fi
  }%
\fi
%    \end{macrocode}
%
% \subsubsection{PostScript driver}
%
%    \begin{macrocode}
\def\PLS@temp{dvips}
\ifx\PLS@temp\PLS@driver
  \RequirePackage{atbegshi}%
  \AtBeginShipout{\PLS@AtBeginShipout}%
  \let\PLS@AtBeginShipout\@empty
  \def\PLS@AddRotate#1{%
    \ifnum#1=0 %
    \else
      \def\PLS@AtBeginShipout{%
        \global\setbox\AtBeginShipoutBox\vbox{%
          \Gin@PS@raw{%
            [{ThisPage}\string<\string</Rotate #1\string>\string>%
            /PUT pdfmark%
          }%
          \box\AtBeginShipoutBox
        }%
      }%
    \fi
  }%
  \def\PLS@RemoveRotate{\let\PLS@AtBeginShipout\@empty}%
\fi
%    \end{macrocode}
%
% \subsubsection{Driver dvipdfm}
%
%    \begin{macrocode}
\def\PLS@temp{dvipdfm}
\ifx\PLS@temp\PLS@driver
  \RequirePackage{atbegshi}%
  \AtBeginShipout{\PLS@AtBeginShipout}%
  \let\PLS@AtBeginShipout\@empty
  \def\PLS@AddRotate#1{%
    \ifnum#1=0 %
    \else
      \def\PLS@AtBeginShipout{%
        \global\setbox\AtBeginShipoutBox\vbox{%
          \special{pdf: put @thispage <</Rotate #1>>}%
          \box\AtBeginShipoutBox
        }%
      }%
    \fi
  }%
  \def\PLS@RemoveRotate{\let\PLS@AtBeginShipout\@empty}%
\fi
%    \end{macrocode}
%
% \subsection{Driver independent stuff}
%
%    The landscape environment is extended by
%    adding the correct |/Rotate| entries.
%    \begin{macrocode}
\g@addto@macro{\landscape}{\PLS@Rotate{90}}
\g@addto@macro{\endlandscape}{\PLS@Rotate{0}}
%    \end{macrocode}
%
%    \begin{macro}{\PLS@Rotate}
%    Main macro, that sets the |/Rotate| entry.\\
%    Argument: any TeX number or nothing, that means zero.\\
%    Driver independent.
%    \begin{macrocode}
\def\PLS@Rotate#1{%
  \begingroup
%    \end{macrocode}
%    1. Check and validate the argument.
%    \begin{macrocode}
    \PLS@CheckAngle{#1}%
  \expandafter\endgroup
%    \end{macrocode}
%    2. Remove previous |\Rotate| entry.
%    \begin{macrocode}
  \expandafter\PLS@RemoveRotate
%    \end{macrocode}
%    3. Add |/Rotate| entry.
%    \begin{macrocode}
  \expandafter\PLS@AddRotate\expandafter{\the\count@}%
}
%    \end{macrocode}
%    \end{macro}
%
%    \begin{macro}{\PLS@CheckAngle}
%    Validates the rotation angle.\\
%    The result is stored in the count register |\count@|.\\
%    Driver independent.
%    \begin{macrocode}
\def\PLS@CheckAngle#1{%
%    \end{macrocode}
%    Nothing means zero:
%    \begin{macrocode}
  \ifx\\#1\\%
    \count@=0 %
  \else
    \count@=#1\relax
  \fi
%    \end{macrocode}
%    Normalize to interval -360 $<$ |\count@| $<$ 360.
%    \begin{macrocode}
  \@whilenum\count@>359\do{\advance\count@ -360 }%
  \@whilenum\count@<-359\do{\advance\count@ 360 }%
%    \end{macrocode}
%    Check values: 0, +/-90, +/-180, +/-270
%    \begin{macrocode}
  \ifnum 1=0\ifnum\count@=0 \else
            \ifnum\count@=90 \else
            \ifnum\count@=-90 \else
            \ifnum\count@=180 \else
            \ifnum\count@=-180 \else
            \ifnum\count@=270 \else
            \ifnum\count@=-270 \else
           1\fi\fi\fi\fi\fi\fi\fi\relax
    \PackageError{pdflscape}{%
      Invalid value \the\count@\space for /Rotate%
    }\@ehc
    \count@=0 %
  \fi
}
%    \end{macrocode}
%    \end{macro}
%
%    \begin{macrocode}
%</package>
%    \end{macrocode}
%
% \section{Test}
%
% \subsection{Driver detection tests}
%    \begin{macrocode}
%<*test0>
\NeedsTeXFormat{LaTeX2e}
\documentclass{minimal}
\usepackage{qstest}
\IncludeTests{*}
\LogTests{log}{*}{*}
%</test0>
%    \end{macrocode}
%
%    \begin{macrocode}
%<*test1>
\usepackage{ifpdf}
\ifpdf
  \def\ExpectDriver{pdftex}%
\fi
\usepackage{ifxetex}%
\ifxetex
  \def\ExpectDriver{dvipdfm}%
\fi
\usepackage{pdflscape}[2016/05/14]
%</test1>
%    \end{macrocode}
%
%    \begin{macrocode}
%<*test2>
\def\ExpectDriver{pdftex}
\usepackage[pdftex]{pdflscape}[2016/05/14]
%</test2>
%    \end{macrocode}
%
%    \begin{macrocode}
%<*test3>
\def\ExpectDriver{dvipdfm}
\usepackage[dvipdfmx]{pdflscape}[2016/05/14]
%</test3>
%    \end{macrocode}
%
%    \begin{macrocode}
%<*test4>
\def\ExpectDriver{dvips}
\usepackage[dvipsone]{graphics}
\usepackage{pdflscape}[2016/05/14]
%</test4>
%    \end{macrocode}
%
%    \begin{macrocode}
%<*test0>
\begin{document}
\begin{qstest}{driver}{driver}
  \makeatletter
  \Expect*{\PLS@driver}*{\ExpectDriver}%
\end{qstest}
\end{document}
%</test0>
%    \end{macrocode}
%
% \subsection{Test for \cs{PLS@CheckAngle}}
%
%    \begin{macrocode}
%<*test5>
\NeedsTeXFormat{LaTeX2e}
\documentclass{minimal}
\usepackage{qstest}
\IncludeTests{*}
\LogTests{log}{*}{*}
\usepackage{pdflscape}[2016/05/14]
\begin{document}
\begin{qstest}{checkangle}{checkangle}
  \makeatletter
  \def\Result{%
    \def\ResultString{\the\count@}%
  }%
  \def\PackageError#1#2#3{%
    \def\Result{%
      \def\ResultString{error}%
    }%
  }%
  \def\Test#1#2{%
    \begingroup
      \PLS@CheckAngle{#1}%
      \Result
      \Expect*{\ResultString}{#2}%
    \endgroup
  }%
  \Test{0}{0}%
  \Test{90}{90}%
  \Test{180}{180}%
  \Test{270}{270}%
  \Test{360}{0}%
  \Test{450}{90}%
  \Test{540}{180}%
  \Test{630}{270}%
  \Test{720}{0}%
  \Test{3600}{0}%
  \Test{3690}{90}%
  \Test{-90}{270}%
  \Test{-180}{180}%
  \Test{-270}{-90}%
  \Test{-360}{0}%
  \Test{-450}{270}%
  \Test{540}{180}%
  \Test{-630}{90}%
  \Test{-720}{0}%
  \Test{-3600}{0}%
  \Test{-3690}{270}%
  \Test{1}{error}%
  \Test{-1}{error}%
  \Test{123}{error}%
\end{qstest}
\end{document}
%</test5>
%    \end{macrocode}
%
% \subsection{Test for rotate}
%
%    \begin{macrocode}
%<*test6>
\NeedsTeXFormat{LaTeX2e}
\documentclass{article}
\usepackage{pdflscape}[2016/05/14]
\begin{document}
\section{Portrait page}
\begin{landscape}
\section{Landscape page}
Foo bar foo bar foo bar foo bar foo bar
foo bar foo bar foo bar foo bar foo bar.
\end{landscape}
\section{Portrait Section}
Foo bar
\begin{landscape}
\section{Again Landscape Section}
\newpage
\section{Second Page of Landscape Section}
\end{landscape}
\section{Second Last Portrait Page}
\newpage
\section{Last Portrait Page}
\end{document}
%</test6>
%    \end{macrocode}
%
%    \begin{macrocode}
%<*result6>
/Page 1 /Rotate 0
/Page 2 /Rotate 90
/Page 3 /Rotate 0
/Page 4 /Rotate 90
/Page 5 /Rotate 90
/Page 6 /Rotate 0
/Page 7 /Rotate 0
%</result6>
%    \end{macrocode}
%
%    \begin{macrocode}
%<*ExtractRotate>
/**
 * ExtractRotate.java
 *
 * Copyright (C) 2007 by Heiko Oberdiek <heiko.oberdiek at googlemail.com>
 *
 * Requires: PDFBox (http://www.pdfbox.org/)
 *
 * Syntax: java ExtractRotate <pdffile> <textfile>
 *
 * The <pdffile> is analyzed and for each page its rotation
 * setting is printed in the <textfile>. Example:
 *   /Page 1 /Rotate 0
 *   /Page 2 /Rotate 90
 */
import java.io.FileInputStream;
import java.io.FileWriter;
import org.pdfbox.pdfparser.PDFParser;
import org.pdfbox.pdmodel.PDDocument;
import org.pdfbox.pdmodel.PDDocumentCatalog;
import org.pdfbox.pdmodel.PDPage;

public class ExtractRotate {

    public static void main(String[] args) {
        try {
            String infile = args[0];
            String outfile = args[1];
            FileWriter out = new FileWriter(outfile);
            PDFParser parser =
                    new PDFParser(new FileInputStream(infile));
            parser.parse();
            PDDocument document = parser.getPDDocument();
            PDDocumentCatalog catalog = document.getDocumentCatalog();
            int i = 0;
            for (Object page: catalog.getAllPages()) {
                i++;
                out.write("/Page " + i + " " + "/Rotate "
                        + ((PDPage)page).findRotation() + "\n");
            }
            document.close();
            out.close();
        }
        catch (Exception e) {
            e.printStackTrace();
            System.exit(1);
        }
    }
}
%</ExtractRotate>
%    \end{macrocode}
%
% \section{Installation}
%
% \subsection{Download}
%
% \paragraph{Package.} This package is available on
% CTAN\footnote{\CTANpkg{pdflscape}}:
% \begin{description}
% \item[\CTAN{macros/latex/contrib/oberdiek/pdflscape.dtx}] The source file.
% \item[\CTAN{macros/latex/contrib/oberdiek/pdflscape.pdf}] Documentation.
% \end{description}
%
%
% \paragraph{Bundle.} All the packages of the bundle `oberdiek'
% are also available in a TDS compliant ZIP archive. There
% the packages are already unpacked and the documentation files
% are generated. The files and directories obey the TDS standard.
% \begin{description}
% \item[\CTANinstall{install/macros/latex/contrib/oberdiek.tds.zip}]
% \end{description}
% \emph{TDS} refers to the standard ``A Directory Structure
% for \TeX\ Files'' (\CTAN{tds/tds.pdf}). Directories
% with \xfile{texmf} in their name are usually organized this way.
%
% \subsection{Bundle installation}
%
% \paragraph{Unpacking.} Unpack the \xfile{oberdiek.tds.zip} in the
% TDS tree (also known as \xfile{texmf} tree) of your choice.
% Example (linux):
% \begin{quote}
%   |unzip oberdiek.tds.zip -d ~/texmf|
% \end{quote}
%
% \paragraph{Script installation.}
% Check the directory \xfile{TDS:scripts/oberdiek/} for
% scripts that need further installation steps.
% Package \xpackage{attachfile2} comes with the Perl script
% \xfile{pdfatfi.pl} that should be installed in such a way
% that it can be called as \texttt{pdfatfi}.
% Example (linux):
% \begin{quote}
%   |chmod +x scripts/oberdiek/pdfatfi.pl|\\
%   |cp scripts/oberdiek/pdfatfi.pl /usr/local/bin/|
% \end{quote}
%
% \subsection{Package installation}
%
% \paragraph{Unpacking.} The \xfile{.dtx} file is a self-extracting
% \docstrip\ archive. The files are extracted by running the
% \xfile{.dtx} through \plainTeX:
% \begin{quote}
%   \verb|tex pdflscape.dtx|
% \end{quote}
%
% \paragraph{TDS.} Now the different files must be moved into
% the different directories in your installation TDS tree
% (also known as \xfile{texmf} tree):
% \begin{quote}
% \def\t{^^A
% \begin{tabular}{@{}>{\ttfamily}l@{ $\rightarrow$ }>{\ttfamily}l@{}}
%   pdflscape.sty & tex/latex/oberdiek/pdflscape.sty\\
%   pdflscape.pdf & doc/latex/oberdiek/pdflscape.pdf\\
%   test/pdflscape-test1.tex & doc/latex/oberdiek/test/pdflscape-test1.tex\\
%   test/pdflscape-test2.tex & doc/latex/oberdiek/test/pdflscape-test2.tex\\
%   test/pdflscape-test3.tex & doc/latex/oberdiek/test/pdflscape-test3.tex\\
%   test/pdflscape-test4.tex & doc/latex/oberdiek/test/pdflscape-test4.tex\\
%   test/pdflscape-test5.tex & doc/latex/oberdiek/test/pdflscape-test5.tex\\
%   test/pdflscape-test6.tex & doc/latex/oberdiek/test/pdflscape-test6.tex\\
%   test/pdflscape-test6.txt & doc/latex/oberdiek/test/pdflscape-test6.txt\\
%   test/ExtractRotate.java & doc/latex/oberdiek/test/ExtractRotate.java\\
%   pdflscape.dtx & source/latex/oberdiek/pdflscape.dtx\\
% \end{tabular}^^A
% }^^A
% \sbox0{\t}^^A
% \ifdim\wd0>\linewidth
%   \begingroup
%     \advance\linewidth by\leftmargin
%     \advance\linewidth by\rightmargin
%   \edef\x{\endgroup
%     \def\noexpand\lw{\the\linewidth}^^A
%   }\x
%   \def\lwbox{^^A
%     \leavevmode
%     \hbox to \linewidth{^^A
%       \kern-\leftmargin\relax
%       \hss
%       \usebox0
%       \hss
%       \kern-\rightmargin\relax
%     }^^A
%   }^^A
%   \ifdim\wd0>\lw
%     \sbox0{\small\t}^^A
%     \ifdim\wd0>\linewidth
%       \ifdim\wd0>\lw
%         \sbox0{\footnotesize\t}^^A
%         \ifdim\wd0>\linewidth
%           \ifdim\wd0>\lw
%             \sbox0{\scriptsize\t}^^A
%             \ifdim\wd0>\linewidth
%               \ifdim\wd0>\lw
%                 \sbox0{\tiny\t}^^A
%                 \ifdim\wd0>\linewidth
%                   \lwbox
%                 \else
%                   \usebox0
%                 \fi
%               \else
%                 \lwbox
%               \fi
%             \else
%               \usebox0
%             \fi
%           \else
%             \lwbox
%           \fi
%         \else
%           \usebox0
%         \fi
%       \else
%         \lwbox
%       \fi
%     \else
%       \usebox0
%     \fi
%   \else
%     \lwbox
%   \fi
% \else
%   \usebox0
% \fi
% \end{quote}
% If you have a \xfile{docstrip.cfg} that configures and enables \docstrip's
% TDS installing feature, then some files can already be in the right
% place, see the documentation of \docstrip.
%
% \subsection{Refresh file name databases}
%
% If your \TeX~distribution
% (\teTeX, \mikTeX, \dots) relies on file name databases, you must refresh
% these. For example, \teTeX\ users run \verb|texhash| or
% \verb|mktexlsr|.
%
% \subsection{Some details for the interested}
%
% \paragraph{Attached source.}
%
% The PDF documentation on CTAN also includes the
% \xfile{.dtx} source file. It can be extracted by
% AcrobatReader 6 or higher. Another option is \textsf{pdftk},
% e.g. unpack the file into the current directory:
% \begin{quote}
%   \verb|pdftk pdflscape.pdf unpack_files output .|
% \end{quote}
%
% \paragraph{Unpacking with \LaTeX.}
% The \xfile{.dtx} chooses its action depending on the format:
% \begin{description}
% \item[\plainTeX:] Run \docstrip\ and extract the files.
% \item[\LaTeX:] Generate the documentation.
% \end{description}
% If you insist on using \LaTeX\ for \docstrip\ (really,
% \docstrip\ does not need \LaTeX), then inform the autodetect routine
% about your intention:
% \begin{quote}
%   \verb|latex \let\install=y\input{pdflscape.dtx}|
% \end{quote}
% Do not forget to quote the argument according to the demands
% of your shell.
%
% \paragraph{Generating the documentation.}
% You can use both the \xfile{.dtx} or the \xfile{.drv} to generate
% the documentation. The process can be configured by the
% configuration file \xfile{ltxdoc.cfg}. For instance, put this
% line into this file, if you want to have A4 as paper format:
% \begin{quote}
%   \verb|\PassOptionsToClass{a4paper}{article}|
% \end{quote}
% An example follows how to generate the
% documentation with pdf\LaTeX:
% \begin{quote}
%\begin{verbatim}
%pdflatex pdflscape.dtx
%makeindex -s gind.ist pdflscape.idx
%pdflatex pdflscape.dtx
%makeindex -s gind.ist pdflscape.idx
%pdflatex pdflscape.dtx
%\end{verbatim}
% \end{quote}
%
% \begin{History}
%   \begin{Version}{2001/01/15 v0.1}
%   \item
%     First public version,
%     published in \xnewsgroup{de.comp.text.tex}:\\
%     \URL{``\link{Re: Querformat, pdfLaTex und thumbpdf}''}^^A
%     {https://groups.google.com/group/de.comp.text.tex/msg/e054c5795e52d2b8}
%   \end{Version}
%   \begin{Version}{2001/02/04 v0.2}
%   \item
%     Minor documentation update.
%   \item
%     CTAN.
%   \end{Version}
%   \begin{Version}{2004/05/11 v0.3}
%   \item
%     Support for dvipdfm added.
%   \end{Version}
%   \begin{Version}{2004/05/12 v0.4}
%   \item
%     Bug fix: support for multipage landscape environment
%     for all drivers except for \xoption{pdftex}.
%   \end{Version}
%   \begin{Version}{2006/02/20 v0.5}
%   \item
%     DTX framework.
%   \item
%     LPPL 1.3
%   \item
%     Code is not changed.
%   \end{Version}
%   \begin{Version}{2006/04/24 v0.6}
%   \item
%     Bug fix: \cs{PLS@RemoveRotate} reinserted for \cs{PLS@Rotate}.
%   \item
%     Fix for \cs{PLS@RemoveRotate} (dvips/dvipdfm): \cs{PLS@EverhShi}
%     hook is cleared.
%   \end{Version}
%   \begin{Version}{2007/04/11 v0.7}
%   \item
%     Line ends sanitized.
%   \end{Version}
%   \begin{Version}{2007/04/17 v0.8}
%   \item
%     Package \xpackage{atbegshi} replaces \xpackage{everyshi}.
%   \end{Version}
%   \begin{Version}{2007/10/21 v0.9}
%   \item
%     Driver detection for \XeTeX\ added.
%   \item
%     Fix for rotation angles $\le$~-360 or $\ge$~360.
%   \end{Version}
%   \begin{Version}{2008/08/11 v0.10}
%   \item
%     Code is not changed.
%   \item
%     URLs updated.
%   \end{Version}
%   \begin{Version}{2016/05/14 v0.11}
%   \item
%     LuaTeX Compatibility
%   \end{Version}
% \end{History}
%
% \PrintIndex
%
% \Finale
\endinput

%        (quote the arguments according to the demands of your shell)
%
% Documentation:
%    (a) If pdflscape.drv is present:
%           latex pdflscape.drv
%    (b) Without pdflscape.drv:
%           latex pdflscape.dtx; ...
%    The class ltxdoc loads the configuration file ltxdoc.cfg
%    if available. Here you can specify further options, e.g.
%    use A4 as paper format:
%       \PassOptionsToClass{a4paper}{article}
%
%    Programm calls to get the documentation (example):
%       pdflatex pdflscape.dtx
%       makeindex -s gind.ist pdflscape.idx
%       pdflatex pdflscape.dtx
%       makeindex -s gind.ist pdflscape.idx
%       pdflatex pdflscape.dtx
%
% Installation:
%    TDS:tex/latex/oberdiek/pdflscape.sty
%    TDS:doc/latex/oberdiek/pdflscape.pdf
%    TDS:doc/latex/oberdiek/test/pdflscape-test1.tex
%    TDS:doc/latex/oberdiek/test/pdflscape-test2.tex
%    TDS:doc/latex/oberdiek/test/pdflscape-test3.tex
%    TDS:doc/latex/oberdiek/test/pdflscape-test4.tex
%    TDS:doc/latex/oberdiek/test/pdflscape-test5.tex
%    TDS:doc/latex/oberdiek/test/pdflscape-test6.tex
%    TDS:doc/latex/oberdiek/test/pdflscape-test6.txt
%    TDS:doc/latex/oberdiek/test/ExtractRotate.java
%    TDS:source/latex/oberdiek/pdflscape.dtx
%
%<*ignore>
\begingroup
  \catcode123=1 %
  \catcode125=2 %
  \def\x{LaTeX2e}%
\expandafter\endgroup
\ifcase 0\ifx\install y1\fi\expandafter
         \ifx\csname processbatchFile\endcsname\relax\else1\fi
         \ifx\fmtname\x\else 1\fi\relax
\else\csname fi\endcsname
%</ignore>
%<*install>
\input docstrip.tex
\Msg{************************************************************************}
\Msg{* Installation}
\Msg{* Package: pdflscape 2016/05/14 v0.11 Display of landscape pages in PDF (HO)}
\Msg{************************************************************************}

\keepsilent
\askforoverwritefalse

\let\MetaPrefix\relax
\preamble

This is a generated file.

Project: pdflscape
Version: 2016/05/14 v0.11

Copyright (C) 2001, 2004, 2006-2008 by
   Heiko Oberdiek <heiko.oberdiek at googlemail.com>

This work may be distributed and/or modified under the
conditions of the LaTeX Project Public License, either
version 1.3c of this license or (at your option) any later
version. This version of this license is in
   http://www.latex-project.org/lppl/lppl-1-3c.txt
and the latest version of this license is in
   http://www.latex-project.org/lppl.txt
and version 1.3 or later is part of all distributions of
LaTeX version 2005/12/01 or later.

This work has the LPPL maintenance status "maintained".

This Current Maintainer of this work is Heiko Oberdiek.

This work consists of the main source file pdflscape.dtx
and the derived files
   pdflscape.sty, pdflscape.pdf, pdflscape.ins, pdflscape.drv,
   pdflscape-test1.tex, pdflscape-test2.tex,
   pdflscape-test3.tex, pdflscape-test4.tex,
   pdflscape-test5.tex, pdflscape-test6.tex,
   pdflscape-test6.txt, ExtractRotate.java.

\endpreamble
\let\MetaPrefix\DoubleperCent

\generate{%
  \file{pdflscape.ins}{\from{pdflscape.dtx}{install}}%
  \file{pdflscape.drv}{\from{pdflscape.dtx}{driver}}%
  \usedir{tex/latex/oberdiek}%
  \file{pdflscape.sty}{\from{pdflscape.dtx}{package}}%
%  \usedir{doc/latex/oberdiek/test}%
%  \file{pdflscape-test1.tex}{\from{pdflscape.dtx}{test0,test1}}%
%  \file{pdflscape-test2.tex}{\from{pdflscape.dtx}{test0,test2}}%
%  \file{pdflscape-test3.tex}{\from{pdflscape.dtx}{test0,test3}}%
%  \file{pdflscape-test4.tex}{\from{pdflscape.dtx}{test0,test4}}%
%  \file{pdflscape-test5.tex}{\from{pdflscape.dtx}{test5}}%
%  \file{pdflscape-test6.tex}{\from{pdflscape.dtx}{test6}}%
  \nopreamble
  \nopostamble
%  \file{pdflscape-test6.txt}{\from{pdflscape.dtx}{result6}}%
  \file{ExtractRotate.java}{\from{pdflscape.dtx}{ExtractRotate}}%
%  \usedir{source/latex/oberdiek/catalogue}%
%  \file{pdflscape.xml}{\from{pdflscape.dtx}{catalogue}}%
}

\catcode32=13\relax% active space
\let =\space%
\Msg{************************************************************************}
\Msg{*}
\Msg{* To finish the installation you have to move the following}
\Msg{* file into a directory searched by TeX:}
\Msg{*}
\Msg{*     pdflscape.sty}
\Msg{*}
\Msg{* To produce the documentation run the file `pdflscape.drv'}
\Msg{* through LaTeX.}
\Msg{*}
\Msg{* Happy TeXing!}
\Msg{*}
\Msg{************************************************************************}

\endbatchfile
%</install>
%<*ignore>
\fi
%</ignore>
%<*driver>
\NeedsTeXFormat{LaTeX2e}
\ProvidesFile{pdflscape.drv}%
  [2016/05/14 v0.11 Display of landscape pages in PDF (HO)]%
\documentclass{ltxdoc}
\usepackage{holtxdoc}[2011/11/22]
\usepackage{graphics}
\begin{document}
  \DocInput{pdflscape.dtx}%
\end{document}
%</driver>
% \fi
%
%
% \CharacterTable
%  {Upper-case    \A\B\C\D\E\F\G\H\I\J\K\L\M\N\O\P\Q\R\S\T\U\V\W\X\Y\Z
%   Lower-case    \a\b\c\d\e\f\g\h\i\j\k\l\m\n\o\p\q\r\s\t\u\v\w\x\y\z
%   Digits        \0\1\2\3\4\5\6\7\8\9
%   Exclamation   \!     Double quote  \"     Hash (number) \#
%   Dollar        \$     Percent       \%     Ampersand     \&
%   Acute accent  \'     Left paren    \(     Right paren   \)
%   Asterisk      \*     Plus          \+     Comma         \,
%   Minus         \-     Point         \.     Solidus       \/
%   Colon         \:     Semicolon     \;     Less than     \<
%   Equals        \=     Greater than  \>     Question mark \?
%   Commercial at \@     Left bracket  \[     Backslash     \\
%   Right bracket \]     Circumflex    \^     Underscore    \_
%   Grave accent  \`     Left brace    \{     Vertical bar  \|
%   Right brace   \}     Tilde         \~}
%
% \GetFileInfo{pdflscape.drv}
%
% \title{The \xpackage{pdflscape} package}
% \date{2016/05/14 v0.11}
% \author{Heiko Oberdiek\thanks
% {Please report any issues at \url{https://github.com/ho-tex/oberdiek/issues}}\\
% \xemail{heiko.oberdiek at googlemail.com}}
%
% \maketitle
%
% \begin{abstract}
% Package \xpackage{pdflscape} adds PDF support to the
% environment \texttt{landscape} of package \xpackage{lscape}
% by setting the PDF page attribute \texttt{/Rotate}.
% \end{abstract}
%
% \tableofcontents
%
% \section{Documentation}
%
% \subsection{Supported drivers}
%
% \begin{itemize}
% \item \xoption{pdftex},\xoption{luatex}
% \item \xoption{dvips}, \xoption{dvipsone},
%       \xoption{pctex32}, \xoption{pctexps}
%       (and other drivers that provide a non-empty \cs{Gin@PS@raw})
% \item \xoption{dvipdfm}
% \end{itemize}
%
% \subsection{Caveat}
%
% Depending on the configuration Ghostscript
% adds a guessed rotation entry by its own.
% This can lead to two |/Rotate| entries per page.
% To prevent this behaviour set the parameter
% |AutoRotatePages| to |/None|, eg:
% \begin{quote}
%   |ps2pdf -dAutoRotatePages=/None|
% \end{quote}
%
% \subsection{Requirements}
%
% \begin{itemize}
% \item The package \xpackage{lscape}.
% \item The package \xpackage{atbegshi} for all drivers
%        except \xoption{pdftex},\xoption{luatex}.
% \end{itemize}
%
% \subsection{Usage}
%
% Load this package instead of or after package \xpackage{lscape}:
% \begin{quote}
%   |\usepackage{pdflscape}|\\
%   |\begin{landscape}|\dots|\end{landscape}|
% \end{quote}
%
% \StopEventually{
% }
%
% \section{Implementation}
%
% \subsection{Package identification}
%
%    \begin{macrocode}
%<*package>
%    \end{macrocode}
%    Package identification.
%    \begin{macrocode}
\NeedsTeXFormat{LaTeX2e}
\ProvidesPackage{pdflscape}%
  [2016/05/14 v0.11 Display of landscape pages in PDF (HO)]
%    \end{macrocode}
%
% \subsection{Driver options}
%
%    \begin{macrocode}
\let\PLS@option\@empty
\let\PLS@driver\@empty
\def\PLS@temp#1#2{%
  \DeclareOption{#1}{%
    \let\PLS@option\CurrentOption
    \def\PLS@driver{#2}%
  }%
}
\PLS@temp{pdftex}{pdftex}
\PLS@temp{luatex}{pdftex}
\PLS@temp{dvips}{dvips}
\PLS@temp{dvipsone}{dvips}
\PLS@temp{pctexps}{dvips}
\PLS@temp{dvipdfm}{dvipdfm}
\PLS@temp{dvipdfmx}{dvipdfm}
\PLS@temp{xetex}{dvipdfm}
\let\PLS@temp\@empty
\DeclareOption*{%
  \ifx\PLS@temp\@empty
    \let\PLS@temp\CurrentOption
  \else
    \edef\PLS@temp{\PLS@temp,\CurrentOption}%
  \fi
}
\ProcessOptions*
\ifx\PLS@option\@empty
\else
  \expandafter\PassOptionsToPackage\expandafter{\PLS@option}{lscape}%
\fi
\ifx\PLS@temp\@empty
\else
  \expandafter\PassOptionsToPackage\expandafter{\PLS@temp}{lscape}%
\fi
\RequirePackage{lscape}
%    \end{macrocode}
%
% \subsection{Autodetection of driver}
%
% \subsubsection{Driver \xoption{pdftex}}
%
%    \begin{macrocode}
\RequirePackage{ifpdf}[2006/02/20]
\def\PLS@temp{pdftex}
\ifpdf
  \ifx\PLS@driver\@empty
    \let\PLS@driver\PLS@temp
    \PackageInfo{pdflscape}{Auto-detected driver: \PLS@temp}%
  \fi
  \ifx\PLS@driver\PLS@temp
  \else
    \PackageError{pdflscape}{%
      The driver option `\PLS@option' is not compatible\MessageBreak
      with pdfTeX in PDF mode, using driver `pdftex' instead%
    }\@ehc
    \let\PLS@driver\PLS@temp
  \fi
\fi
%    \end{macrocode}
%
% \subsubsection{Driver \xoption{xetex}}
%
%    \begin{macrocode}
\RequirePackage{ifxetex}%
\def\PLS@temp{dvipdfm}
\ifxetex
  \ifx\PLS@driver\@empty
    \let\PLS@driver\PLS@temp
    \PackageInfo{pdflscape}{%
      Auto-detected driver: \PLS@temp\space(xetex)%
    }%
  \fi
  \ifx\PLS@driver\PLS@temp
  \else
    \PackageError{pdflscape}{%
      The driver option `\PLS@option' is not compatible\MessageBreak
      with XeTeX, using driver for XeTeX (dvipdfm) instead%
    }\@ehc
    \let\PLS@driver\PLS@temp
  \fi
\fi
%    \end{macrocode}
%
% \subsubsection{Detect driver based on \cs{Gin@driver}}
%
%    \begin{macrocode}
\def\PLS@temp#1#2{%
  \ifx\PLS@driver\@empty
    \begingroup
      \def\PLS@temp{#1.def}%
    \expandafter\endgroup
    \ifx\PLS@temp\Gin@driver
      \def\PLS@driver{#2}%
      \PackageInfo{pdflscape}{%
        Auto-detected driver: \PLS@driver\space(#1)%
      }%
    \fi
  \fi
}
\PLS@temp{dvipdfm}{dvipdfm}
\PLS@temp{dvipdfmx}{dvipdfm}
\PLS@temp{dvips}{dvips}
\PLS@temp{dvipsone}{dvips}
\PLS@temp{pctexps}{dvips}
%    \end{macrocode}
%
% \subsubsection{Driver \xoption{dvips}}
%
%    \begin{macrocode}
\ifx\PLS@driver\@empty
  \def\PLS@temp{dvips}%
  \ifnum 1=0\ifx\Gin@PS@raw\@undefined
            \else
              \ifx\Gin@PS@raw\relax
              \else
                \ifx\Gin@PS@raw\@empty
                \else
                  1%
                \fi
              \fi
            \fi
            \space
    \let\PLS@driver\PLS@temp
    \PackageInfo{pdflscape}{%
      Auto-detected driver: \PLS@temp\space(PostScript)%
    }%
  \fi
\fi
%    \end{macrocode}
%
%    \begin{macrocode}
\ifx\PLS@driver\@empty
  \PackageError{pdflscape}{%
    Neither pdfTeX, PostScript, nor dvipdfm driver found%
  }\@ehc
\fi
%    \end{macrocode}
%
% \subsection{Driver implementation}
%
% \begin{itemize}
% \item |\PLS@AddRotate #1|\\
%   it expects the correct rotation number in |#1|
%   and implements the adding of the /Rotation entry
%   in the |/Page| object of the current page.
% \item |\PLS@RemoveRotate|\\
%   it removes a previous |/Rotate| entry, if necessary.
% \end{itemize}
%
% \subsubsection{\pdfTeX}
%
% Not too nice is the global setting of \cs{pdfpageattr}.
% Perhaps this can be changed in future versions.
%
% luatex compat.
%    \begin{macrocode}
\ifx\pdfpageattr\@undefined
  \ifx\pdfvariable\@undefined\else 
    \protected\edef\pdfpageattr{\pdfvariable pageattr}
  \fi
\fi
%    \end{macrocode}
%    \begin{macrocode}
\def\PLS@temp{pdftex}
\ifx\PLS@temp\PLS@driver
  \def\PLS@AddRotate#1{%
    \ifnum#1=0 %
%    \end{macrocode}
%    Already the default.
%    \begin{macrocode}
    \else
      \global\pdfpageattr\expandafter{%
        \the\pdfpageattr
        /Rotate #1%
      }%
    \fi
  }%
%    \end{macrocode}
%    Removes a |/Rotate| entry.
%    \begin{macrocode}
  \def\PLS@RemoveRotate{%
    \begingroup
      \global\pdfpageattr\expandafter{\expandafter}%
      \expandafter\PLS@@RemoveRotate
        \the\pdfpageattr /Rotate\@nil
    \endgroup
  }%
  \def\PLS@@RemoveRotate#1/Rotate#2\@nil{%
%    \end{macrocode}
%    Append |/Rotate| free stuff to \cs{pdfpageattr}.
%    \begin{macrocode}
    \global\pdfpageattr\expandafter{\the\pdfpageattr#1}%
    \ifx\\#2\\%
%    \end{macrocode}
%    Ready, because the detected |/Rotate| is part of
%    the end marker: |/Rotate\@nil|
%    \begin{macrocode}
    \else
%    \end{macrocode}
%    First read in the argument of |/Rotate|,
%    then continue parsing.
%    \begin{macrocode}
      \afterassignment\PLS@@RemoveRotate
      \count0=#2\@nil
    \fi
  }%
\fi
%    \end{macrocode}
%
% \subsubsection{PostScript driver}
%
%    \begin{macrocode}
\def\PLS@temp{dvips}
\ifx\PLS@temp\PLS@driver
  \RequirePackage{atbegshi}%
  \AtBeginShipout{\PLS@AtBeginShipout}%
  \let\PLS@AtBeginShipout\@empty
  \def\PLS@AddRotate#1{%
    \ifnum#1=0 %
    \else
      \def\PLS@AtBeginShipout{%
        \global\setbox\AtBeginShipoutBox\vbox{%
          \Gin@PS@raw{%
            [{ThisPage}\string<\string</Rotate #1\string>\string>%
            /PUT pdfmark%
          }%
          \box\AtBeginShipoutBox
        }%
      }%
    \fi
  }%
  \def\PLS@RemoveRotate{\let\PLS@AtBeginShipout\@empty}%
\fi
%    \end{macrocode}
%
% \subsubsection{Driver dvipdfm}
%
%    \begin{macrocode}
\def\PLS@temp{dvipdfm}
\ifx\PLS@temp\PLS@driver
  \RequirePackage{atbegshi}%
  \AtBeginShipout{\PLS@AtBeginShipout}%
  \let\PLS@AtBeginShipout\@empty
  \def\PLS@AddRotate#1{%
    \ifnum#1=0 %
    \else
      \def\PLS@AtBeginShipout{%
        \global\setbox\AtBeginShipoutBox\vbox{%
          \special{pdf: put @thispage <</Rotate #1>>}%
          \box\AtBeginShipoutBox
        }%
      }%
    \fi
  }%
  \def\PLS@RemoveRotate{\let\PLS@AtBeginShipout\@empty}%
\fi
%    \end{macrocode}
%
% \subsection{Driver independent stuff}
%
%    The landscape environment is extended by
%    adding the correct |/Rotate| entries.
%    \begin{macrocode}
\g@addto@macro{\landscape}{\PLS@Rotate{90}}
\g@addto@macro{\endlandscape}{\PLS@Rotate{0}}
%    \end{macrocode}
%
%    \begin{macro}{\PLS@Rotate}
%    Main macro, that sets the |/Rotate| entry.\\
%    Argument: any TeX number or nothing, that means zero.\\
%    Driver independent.
%    \begin{macrocode}
\def\PLS@Rotate#1{%
  \begingroup
%    \end{macrocode}
%    1. Check and validate the argument.
%    \begin{macrocode}
    \PLS@CheckAngle{#1}%
  \expandafter\endgroup
%    \end{macrocode}
%    2. Remove previous |\Rotate| entry.
%    \begin{macrocode}
  \expandafter\PLS@RemoveRotate
%    \end{macrocode}
%    3. Add |/Rotate| entry.
%    \begin{macrocode}
  \expandafter\PLS@AddRotate\expandafter{\the\count@}%
}
%    \end{macrocode}
%    \end{macro}
%
%    \begin{macro}{\PLS@CheckAngle}
%    Validates the rotation angle.\\
%    The result is stored in the count register |\count@|.\\
%    Driver independent.
%    \begin{macrocode}
\def\PLS@CheckAngle#1{%
%    \end{macrocode}
%    Nothing means zero:
%    \begin{macrocode}
  \ifx\\#1\\%
    \count@=0 %
  \else
    \count@=#1\relax
  \fi
%    \end{macrocode}
%    Normalize to interval -360 $<$ |\count@| $<$ 360.
%    \begin{macrocode}
  \@whilenum\count@>359\do{\advance\count@ -360 }%
  \@whilenum\count@<-359\do{\advance\count@ 360 }%
%    \end{macrocode}
%    Check values: 0, +/-90, +/-180, +/-270
%    \begin{macrocode}
  \ifnum 1=0\ifnum\count@=0 \else
            \ifnum\count@=90 \else
            \ifnum\count@=-90 \else
            \ifnum\count@=180 \else
            \ifnum\count@=-180 \else
            \ifnum\count@=270 \else
            \ifnum\count@=-270 \else
           1\fi\fi\fi\fi\fi\fi\fi\relax
    \PackageError{pdflscape}{%
      Invalid value \the\count@\space for /Rotate%
    }\@ehc
    \count@=0 %
  \fi
}
%    \end{macrocode}
%    \end{macro}
%
%    \begin{macrocode}
%</package>
%    \end{macrocode}
%
% \section{Test}
%
% \subsection{Driver detection tests}
%    \begin{macrocode}
%<*test0>
\NeedsTeXFormat{LaTeX2e}
\documentclass{minimal}
\usepackage{qstest}
\IncludeTests{*}
\LogTests{log}{*}{*}
%</test0>
%    \end{macrocode}
%
%    \begin{macrocode}
%<*test1>
\usepackage{ifpdf}
\ifpdf
  \def\ExpectDriver{pdftex}%
\fi
\usepackage{ifxetex}%
\ifxetex
  \def\ExpectDriver{dvipdfm}%
\fi
\usepackage{pdflscape}[2016/05/14]
%</test1>
%    \end{macrocode}
%
%    \begin{macrocode}
%<*test2>
\def\ExpectDriver{pdftex}
\usepackage[pdftex]{pdflscape}[2016/05/14]
%</test2>
%    \end{macrocode}
%
%    \begin{macrocode}
%<*test3>
\def\ExpectDriver{dvipdfm}
\usepackage[dvipdfmx]{pdflscape}[2016/05/14]
%</test3>
%    \end{macrocode}
%
%    \begin{macrocode}
%<*test4>
\def\ExpectDriver{dvips}
\usepackage[dvipsone]{graphics}
\usepackage{pdflscape}[2016/05/14]
%</test4>
%    \end{macrocode}
%
%    \begin{macrocode}
%<*test0>
\begin{document}
\begin{qstest}{driver}{driver}
  \makeatletter
  \Expect*{\PLS@driver}*{\ExpectDriver}%
\end{qstest}
\end{document}
%</test0>
%    \end{macrocode}
%
% \subsection{Test for \cs{PLS@CheckAngle}}
%
%    \begin{macrocode}
%<*test5>
\NeedsTeXFormat{LaTeX2e}
\documentclass{minimal}
\usepackage{qstest}
\IncludeTests{*}
\LogTests{log}{*}{*}
\usepackage{pdflscape}[2016/05/14]
\begin{document}
\begin{qstest}{checkangle}{checkangle}
  \makeatletter
  \def\Result{%
    \def\ResultString{\the\count@}%
  }%
  \def\PackageError#1#2#3{%
    \def\Result{%
      \def\ResultString{error}%
    }%
  }%
  \def\Test#1#2{%
    \begingroup
      \PLS@CheckAngle{#1}%
      \Result
      \Expect*{\ResultString}{#2}%
    \endgroup
  }%
  \Test{0}{0}%
  \Test{90}{90}%
  \Test{180}{180}%
  \Test{270}{270}%
  \Test{360}{0}%
  \Test{450}{90}%
  \Test{540}{180}%
  \Test{630}{270}%
  \Test{720}{0}%
  \Test{3600}{0}%
  \Test{3690}{90}%
  \Test{-90}{270}%
  \Test{-180}{180}%
  \Test{-270}{-90}%
  \Test{-360}{0}%
  \Test{-450}{270}%
  \Test{540}{180}%
  \Test{-630}{90}%
  \Test{-720}{0}%
  \Test{-3600}{0}%
  \Test{-3690}{270}%
  \Test{1}{error}%
  \Test{-1}{error}%
  \Test{123}{error}%
\end{qstest}
\end{document}
%</test5>
%    \end{macrocode}
%
% \subsection{Test for rotate}
%
%    \begin{macrocode}
%<*test6>
\NeedsTeXFormat{LaTeX2e}
\documentclass{article}
\usepackage{pdflscape}[2016/05/14]
\begin{document}
\section{Portrait page}
\begin{landscape}
\section{Landscape page}
Foo bar foo bar foo bar foo bar foo bar
foo bar foo bar foo bar foo bar foo bar.
\end{landscape}
\section{Portrait Section}
Foo bar
\begin{landscape}
\section{Again Landscape Section}
\newpage
\section{Second Page of Landscape Section}
\end{landscape}
\section{Second Last Portrait Page}
\newpage
\section{Last Portrait Page}
\end{document}
%</test6>
%    \end{macrocode}
%
%    \begin{macrocode}
%<*result6>
/Page 1 /Rotate 0
/Page 2 /Rotate 90
/Page 3 /Rotate 0
/Page 4 /Rotate 90
/Page 5 /Rotate 90
/Page 6 /Rotate 0
/Page 7 /Rotate 0
%</result6>
%    \end{macrocode}
%
%    \begin{macrocode}
%<*ExtractRotate>
/**
 * ExtractRotate.java
 *
 * Copyright (C) 2007 by Heiko Oberdiek <heiko.oberdiek at googlemail.com>
 *
 * Requires: PDFBox (http://www.pdfbox.org/)
 *
 * Syntax: java ExtractRotate <pdffile> <textfile>
 *
 * The <pdffile> is analyzed and for each page its rotation
 * setting is printed in the <textfile>. Example:
 *   /Page 1 /Rotate 0
 *   /Page 2 /Rotate 90
 */
import java.io.FileInputStream;
import java.io.FileWriter;
import org.pdfbox.pdfparser.PDFParser;
import org.pdfbox.pdmodel.PDDocument;
import org.pdfbox.pdmodel.PDDocumentCatalog;
import org.pdfbox.pdmodel.PDPage;

public class ExtractRotate {

    public static void main(String[] args) {
        try {
            String infile = args[0];
            String outfile = args[1];
            FileWriter out = new FileWriter(outfile);
            PDFParser parser =
                    new PDFParser(new FileInputStream(infile));
            parser.parse();
            PDDocument document = parser.getPDDocument();
            PDDocumentCatalog catalog = document.getDocumentCatalog();
            int i = 0;
            for (Object page: catalog.getAllPages()) {
                i++;
                out.write("/Page " + i + " " + "/Rotate "
                        + ((PDPage)page).findRotation() + "\n");
            }
            document.close();
            out.close();
        }
        catch (Exception e) {
            e.printStackTrace();
            System.exit(1);
        }
    }
}
%</ExtractRotate>
%    \end{macrocode}
%
% \section{Installation}
%
% \subsection{Download}
%
% \paragraph{Package.} This package is available on
% CTAN\footnote{\CTANpkg{pdflscape}}:
% \begin{description}
% \item[\CTAN{macros/latex/contrib/oberdiek/pdflscape.dtx}] The source file.
% \item[\CTAN{macros/latex/contrib/oberdiek/pdflscape.pdf}] Documentation.
% \end{description}
%
%
% \paragraph{Bundle.} All the packages of the bundle `oberdiek'
% are also available in a TDS compliant ZIP archive. There
% the packages are already unpacked and the documentation files
% are generated. The files and directories obey the TDS standard.
% \begin{description}
% \item[\CTANinstall{install/macros/latex/contrib/oberdiek.tds.zip}]
% \end{description}
% \emph{TDS} refers to the standard ``A Directory Structure
% for \TeX\ Files'' (\CTAN{tds/tds.pdf}). Directories
% with \xfile{texmf} in their name are usually organized this way.
%
% \subsection{Bundle installation}
%
% \paragraph{Unpacking.} Unpack the \xfile{oberdiek.tds.zip} in the
% TDS tree (also known as \xfile{texmf} tree) of your choice.
% Example (linux):
% \begin{quote}
%   |unzip oberdiek.tds.zip -d ~/texmf|
% \end{quote}
%
% \paragraph{Script installation.}
% Check the directory \xfile{TDS:scripts/oberdiek/} for
% scripts that need further installation steps.
% Package \xpackage{attachfile2} comes with the Perl script
% \xfile{pdfatfi.pl} that should be installed in such a way
% that it can be called as \texttt{pdfatfi}.
% Example (linux):
% \begin{quote}
%   |chmod +x scripts/oberdiek/pdfatfi.pl|\\
%   |cp scripts/oberdiek/pdfatfi.pl /usr/local/bin/|
% \end{quote}
%
% \subsection{Package installation}
%
% \paragraph{Unpacking.} The \xfile{.dtx} file is a self-extracting
% \docstrip\ archive. The files are extracted by running the
% \xfile{.dtx} through \plainTeX:
% \begin{quote}
%   \verb|tex pdflscape.dtx|
% \end{quote}
%
% \paragraph{TDS.} Now the different files must be moved into
% the different directories in your installation TDS tree
% (also known as \xfile{texmf} tree):
% \begin{quote}
% \def\t{^^A
% \begin{tabular}{@{}>{\ttfamily}l@{ $\rightarrow$ }>{\ttfamily}l@{}}
%   pdflscape.sty & tex/latex/oberdiek/pdflscape.sty\\
%   pdflscape.pdf & doc/latex/oberdiek/pdflscape.pdf\\
%   test/pdflscape-test1.tex & doc/latex/oberdiek/test/pdflscape-test1.tex\\
%   test/pdflscape-test2.tex & doc/latex/oberdiek/test/pdflscape-test2.tex\\
%   test/pdflscape-test3.tex & doc/latex/oberdiek/test/pdflscape-test3.tex\\
%   test/pdflscape-test4.tex & doc/latex/oberdiek/test/pdflscape-test4.tex\\
%   test/pdflscape-test5.tex & doc/latex/oberdiek/test/pdflscape-test5.tex\\
%   test/pdflscape-test6.tex & doc/latex/oberdiek/test/pdflscape-test6.tex\\
%   test/pdflscape-test6.txt & doc/latex/oberdiek/test/pdflscape-test6.txt\\
%   test/ExtractRotate.java & doc/latex/oberdiek/test/ExtractRotate.java\\
%   pdflscape.dtx & source/latex/oberdiek/pdflscape.dtx\\
% \end{tabular}^^A
% }^^A
% \sbox0{\t}^^A
% \ifdim\wd0>\linewidth
%   \begingroup
%     \advance\linewidth by\leftmargin
%     \advance\linewidth by\rightmargin
%   \edef\x{\endgroup
%     \def\noexpand\lw{\the\linewidth}^^A
%   }\x
%   \def\lwbox{^^A
%     \leavevmode
%     \hbox to \linewidth{^^A
%       \kern-\leftmargin\relax
%       \hss
%       \usebox0
%       \hss
%       \kern-\rightmargin\relax
%     }^^A
%   }^^A
%   \ifdim\wd0>\lw
%     \sbox0{\small\t}^^A
%     \ifdim\wd0>\linewidth
%       \ifdim\wd0>\lw
%         \sbox0{\footnotesize\t}^^A
%         \ifdim\wd0>\linewidth
%           \ifdim\wd0>\lw
%             \sbox0{\scriptsize\t}^^A
%             \ifdim\wd0>\linewidth
%               \ifdim\wd0>\lw
%                 \sbox0{\tiny\t}^^A
%                 \ifdim\wd0>\linewidth
%                   \lwbox
%                 \else
%                   \usebox0
%                 \fi
%               \else
%                 \lwbox
%               \fi
%             \else
%               \usebox0
%             \fi
%           \else
%             \lwbox
%           \fi
%         \else
%           \usebox0
%         \fi
%       \else
%         \lwbox
%       \fi
%     \else
%       \usebox0
%     \fi
%   \else
%     \lwbox
%   \fi
% \else
%   \usebox0
% \fi
% \end{quote}
% If you have a \xfile{docstrip.cfg} that configures and enables \docstrip's
% TDS installing feature, then some files can already be in the right
% place, see the documentation of \docstrip.
%
% \subsection{Refresh file name databases}
%
% If your \TeX~distribution
% (\teTeX, \mikTeX, \dots) relies on file name databases, you must refresh
% these. For example, \teTeX\ users run \verb|texhash| or
% \verb|mktexlsr|.
%
% \subsection{Some details for the interested}
%
% \paragraph{Attached source.}
%
% The PDF documentation on CTAN also includes the
% \xfile{.dtx} source file. It can be extracted by
% AcrobatReader 6 or higher. Another option is \textsf{pdftk},
% e.g. unpack the file into the current directory:
% \begin{quote}
%   \verb|pdftk pdflscape.pdf unpack_files output .|
% \end{quote}
%
% \paragraph{Unpacking with \LaTeX.}
% The \xfile{.dtx} chooses its action depending on the format:
% \begin{description}
% \item[\plainTeX:] Run \docstrip\ and extract the files.
% \item[\LaTeX:] Generate the documentation.
% \end{description}
% If you insist on using \LaTeX\ for \docstrip\ (really,
% \docstrip\ does not need \LaTeX), then inform the autodetect routine
% about your intention:
% \begin{quote}
%   \verb|latex \let\install=y% \iffalse meta-comment
%
% File: pdflscape.dtx
% Version: 2016/05/14 v0.11
% Info: Display of landscape pages in PDF
%
% Copyright (C) 2001, 2004, 2006-2008 by
%    Heiko Oberdiek <heiko.oberdiek at googlemail.com>
%    2016
%    https://github.com/ho-tex/oberdiek/issues
%
% This work may be distributed and/or modified under the
% conditions of the LaTeX Project Public License, either
% version 1.3c of this license or (at your option) any later
% version. This version of this license is in
%    http://www.latex-project.org/lppl/lppl-1-3c.txt
% and the latest version of this license is in
%    http://www.latex-project.org/lppl.txt
% and version 1.3 or later is part of all distributions of
% LaTeX version 2005/12/01 or later.
%
% This work has the LPPL maintenance status "maintained".
%
% This Current Maintainer of this work is Heiko Oberdiek.
%
% This work consists of the main source file pdflscape.dtx
% and the derived files
%    pdflscape.sty, pdflscape.pdf, pdflscape.ins, pdflscape.drv,
%    pdflscape-test1.tex, pdflscape-test2.tex,
%    pdflscape-test3.tex, pdflscape-test4.tex,
%    pdflscape-test5.tex, pdflscape-test6.tex,
%    pdflscape-test6.txt, ExtractRotate.java.
%
% Distribution:
%    CTAN:macros/latex/contrib/oberdiek/pdflscape.dtx
%    CTAN:macros/latex/contrib/oberdiek/pdflscape.pdf
%
% Unpacking:
%    (a) If pdflscape.ins is present:
%           tex pdflscape.ins
%    (b) Without pdflscape.ins:
%           tex pdflscape.dtx
%    (c) If you insist on using LaTeX
%           latex \let\install=y\input{pdflscape.dtx}
%        (quote the arguments according to the demands of your shell)
%
% Documentation:
%    (a) If pdflscape.drv is present:
%           latex pdflscape.drv
%    (b) Without pdflscape.drv:
%           latex pdflscape.dtx; ...
%    The class ltxdoc loads the configuration file ltxdoc.cfg
%    if available. Here you can specify further options, e.g.
%    use A4 as paper format:
%       \PassOptionsToClass{a4paper}{article}
%
%    Programm calls to get the documentation (example):
%       pdflatex pdflscape.dtx
%       makeindex -s gind.ist pdflscape.idx
%       pdflatex pdflscape.dtx
%       makeindex -s gind.ist pdflscape.idx
%       pdflatex pdflscape.dtx
%
% Installation:
%    TDS:tex/latex/oberdiek/pdflscape.sty
%    TDS:doc/latex/oberdiek/pdflscape.pdf
%    TDS:doc/latex/oberdiek/test/pdflscape-test1.tex
%    TDS:doc/latex/oberdiek/test/pdflscape-test2.tex
%    TDS:doc/latex/oberdiek/test/pdflscape-test3.tex
%    TDS:doc/latex/oberdiek/test/pdflscape-test4.tex
%    TDS:doc/latex/oberdiek/test/pdflscape-test5.tex
%    TDS:doc/latex/oberdiek/test/pdflscape-test6.tex
%    TDS:doc/latex/oberdiek/test/pdflscape-test6.txt
%    TDS:doc/latex/oberdiek/test/ExtractRotate.java
%    TDS:source/latex/oberdiek/pdflscape.dtx
%
%<*ignore>
\begingroup
  \catcode123=1 %
  \catcode125=2 %
  \def\x{LaTeX2e}%
\expandafter\endgroup
\ifcase 0\ifx\install y1\fi\expandafter
         \ifx\csname processbatchFile\endcsname\relax\else1\fi
         \ifx\fmtname\x\else 1\fi\relax
\else\csname fi\endcsname
%</ignore>
%<*install>
\input docstrip.tex
\Msg{************************************************************************}
\Msg{* Installation}
\Msg{* Package: pdflscape 2016/05/14 v0.11 Display of landscape pages in PDF (HO)}
\Msg{************************************************************************}

\keepsilent
\askforoverwritefalse

\let\MetaPrefix\relax
\preamble

This is a generated file.

Project: pdflscape
Version: 2016/05/14 v0.11

Copyright (C) 2001, 2004, 2006-2008 by
   Heiko Oberdiek <heiko.oberdiek at googlemail.com>

This work may be distributed and/or modified under the
conditions of the LaTeX Project Public License, either
version 1.3c of this license or (at your option) any later
version. This version of this license is in
   http://www.latex-project.org/lppl/lppl-1-3c.txt
and the latest version of this license is in
   http://www.latex-project.org/lppl.txt
and version 1.3 or later is part of all distributions of
LaTeX version 2005/12/01 or later.

This work has the LPPL maintenance status "maintained".

This Current Maintainer of this work is Heiko Oberdiek.

This work consists of the main source file pdflscape.dtx
and the derived files
   pdflscape.sty, pdflscape.pdf, pdflscape.ins, pdflscape.drv,
   pdflscape-test1.tex, pdflscape-test2.tex,
   pdflscape-test3.tex, pdflscape-test4.tex,
   pdflscape-test5.tex, pdflscape-test6.tex,
   pdflscape-test6.txt, ExtractRotate.java.

\endpreamble
\let\MetaPrefix\DoubleperCent

\generate{%
  \file{pdflscape.ins}{\from{pdflscape.dtx}{install}}%
  \file{pdflscape.drv}{\from{pdflscape.dtx}{driver}}%
  \usedir{tex/latex/oberdiek}%
  \file{pdflscape.sty}{\from{pdflscape.dtx}{package}}%
%  \usedir{doc/latex/oberdiek/test}%
%  \file{pdflscape-test1.tex}{\from{pdflscape.dtx}{test0,test1}}%
%  \file{pdflscape-test2.tex}{\from{pdflscape.dtx}{test0,test2}}%
%  \file{pdflscape-test3.tex}{\from{pdflscape.dtx}{test0,test3}}%
%  \file{pdflscape-test4.tex}{\from{pdflscape.dtx}{test0,test4}}%
%  \file{pdflscape-test5.tex}{\from{pdflscape.dtx}{test5}}%
%  \file{pdflscape-test6.tex}{\from{pdflscape.dtx}{test6}}%
  \nopreamble
  \nopostamble
%  \file{pdflscape-test6.txt}{\from{pdflscape.dtx}{result6}}%
  \file{ExtractRotate.java}{\from{pdflscape.dtx}{ExtractRotate}}%
%  \usedir{source/latex/oberdiek/catalogue}%
%  \file{pdflscape.xml}{\from{pdflscape.dtx}{catalogue}}%
}

\catcode32=13\relax% active space
\let =\space%
\Msg{************************************************************************}
\Msg{*}
\Msg{* To finish the installation you have to move the following}
\Msg{* file into a directory searched by TeX:}
\Msg{*}
\Msg{*     pdflscape.sty}
\Msg{*}
\Msg{* To produce the documentation run the file `pdflscape.drv'}
\Msg{* through LaTeX.}
\Msg{*}
\Msg{* Happy TeXing!}
\Msg{*}
\Msg{************************************************************************}

\endbatchfile
%</install>
%<*ignore>
\fi
%</ignore>
%<*driver>
\NeedsTeXFormat{LaTeX2e}
\ProvidesFile{pdflscape.drv}%
  [2016/05/14 v0.11 Display of landscape pages in PDF (HO)]%
\documentclass{ltxdoc}
\usepackage{holtxdoc}[2011/11/22]
\usepackage{graphics}
\begin{document}
  \DocInput{pdflscape.dtx}%
\end{document}
%</driver>
% \fi
%
%
% \CharacterTable
%  {Upper-case    \A\B\C\D\E\F\G\H\I\J\K\L\M\N\O\P\Q\R\S\T\U\V\W\X\Y\Z
%   Lower-case    \a\b\c\d\e\f\g\h\i\j\k\l\m\n\o\p\q\r\s\t\u\v\w\x\y\z
%   Digits        \0\1\2\3\4\5\6\7\8\9
%   Exclamation   \!     Double quote  \"     Hash (number) \#
%   Dollar        \$     Percent       \%     Ampersand     \&
%   Acute accent  \'     Left paren    \(     Right paren   \)
%   Asterisk      \*     Plus          \+     Comma         \,
%   Minus         \-     Point         \.     Solidus       \/
%   Colon         \:     Semicolon     \;     Less than     \<
%   Equals        \=     Greater than  \>     Question mark \?
%   Commercial at \@     Left bracket  \[     Backslash     \\
%   Right bracket \]     Circumflex    \^     Underscore    \_
%   Grave accent  \`     Left brace    \{     Vertical bar  \|
%   Right brace   \}     Tilde         \~}
%
% \GetFileInfo{pdflscape.drv}
%
% \title{The \xpackage{pdflscape} package}
% \date{2016/05/14 v0.11}
% \author{Heiko Oberdiek\thanks
% {Please report any issues at \url{https://github.com/ho-tex/oberdiek/issues}}\\
% \xemail{heiko.oberdiek at googlemail.com}}
%
% \maketitle
%
% \begin{abstract}
% Package \xpackage{pdflscape} adds PDF support to the
% environment \texttt{landscape} of package \xpackage{lscape}
% by setting the PDF page attribute \texttt{/Rotate}.
% \end{abstract}
%
% \tableofcontents
%
% \section{Documentation}
%
% \subsection{Supported drivers}
%
% \begin{itemize}
% \item \xoption{pdftex},\xoption{luatex}
% \item \xoption{dvips}, \xoption{dvipsone},
%       \xoption{pctex32}, \xoption{pctexps}
%       (and other drivers that provide a non-empty \cs{Gin@PS@raw})
% \item \xoption{dvipdfm}
% \end{itemize}
%
% \subsection{Caveat}
%
% Depending on the configuration Ghostscript
% adds a guessed rotation entry by its own.
% This can lead to two |/Rotate| entries per page.
% To prevent this behaviour set the parameter
% |AutoRotatePages| to |/None|, eg:
% \begin{quote}
%   |ps2pdf -dAutoRotatePages=/None|
% \end{quote}
%
% \subsection{Requirements}
%
% \begin{itemize}
% \item The package \xpackage{lscape}.
% \item The package \xpackage{atbegshi} for all drivers
%        except \xoption{pdftex},\xoption{luatex}.
% \end{itemize}
%
% \subsection{Usage}
%
% Load this package instead of or after package \xpackage{lscape}:
% \begin{quote}
%   |\usepackage{pdflscape}|\\
%   |\begin{landscape}|\dots|\end{landscape}|
% \end{quote}
%
% \StopEventually{
% }
%
% \section{Implementation}
%
% \subsection{Package identification}
%
%    \begin{macrocode}
%<*package>
%    \end{macrocode}
%    Package identification.
%    \begin{macrocode}
\NeedsTeXFormat{LaTeX2e}
\ProvidesPackage{pdflscape}%
  [2016/05/14 v0.11 Display of landscape pages in PDF (HO)]
%    \end{macrocode}
%
% \subsection{Driver options}
%
%    \begin{macrocode}
\let\PLS@option\@empty
\let\PLS@driver\@empty
\def\PLS@temp#1#2{%
  \DeclareOption{#1}{%
    \let\PLS@option\CurrentOption
    \def\PLS@driver{#2}%
  }%
}
\PLS@temp{pdftex}{pdftex}
\PLS@temp{luatex}{pdftex}
\PLS@temp{dvips}{dvips}
\PLS@temp{dvipsone}{dvips}
\PLS@temp{pctexps}{dvips}
\PLS@temp{dvipdfm}{dvipdfm}
\PLS@temp{dvipdfmx}{dvipdfm}
\PLS@temp{xetex}{dvipdfm}
\let\PLS@temp\@empty
\DeclareOption*{%
  \ifx\PLS@temp\@empty
    \let\PLS@temp\CurrentOption
  \else
    \edef\PLS@temp{\PLS@temp,\CurrentOption}%
  \fi
}
\ProcessOptions*
\ifx\PLS@option\@empty
\else
  \expandafter\PassOptionsToPackage\expandafter{\PLS@option}{lscape}%
\fi
\ifx\PLS@temp\@empty
\else
  \expandafter\PassOptionsToPackage\expandafter{\PLS@temp}{lscape}%
\fi
\RequirePackage{lscape}
%    \end{macrocode}
%
% \subsection{Autodetection of driver}
%
% \subsubsection{Driver \xoption{pdftex}}
%
%    \begin{macrocode}
\RequirePackage{ifpdf}[2006/02/20]
\def\PLS@temp{pdftex}
\ifpdf
  \ifx\PLS@driver\@empty
    \let\PLS@driver\PLS@temp
    \PackageInfo{pdflscape}{Auto-detected driver: \PLS@temp}%
  \fi
  \ifx\PLS@driver\PLS@temp
  \else
    \PackageError{pdflscape}{%
      The driver option `\PLS@option' is not compatible\MessageBreak
      with pdfTeX in PDF mode, using driver `pdftex' instead%
    }\@ehc
    \let\PLS@driver\PLS@temp
  \fi
\fi
%    \end{macrocode}
%
% \subsubsection{Driver \xoption{xetex}}
%
%    \begin{macrocode}
\RequirePackage{ifxetex}%
\def\PLS@temp{dvipdfm}
\ifxetex
  \ifx\PLS@driver\@empty
    \let\PLS@driver\PLS@temp
    \PackageInfo{pdflscape}{%
      Auto-detected driver: \PLS@temp\space(xetex)%
    }%
  \fi
  \ifx\PLS@driver\PLS@temp
  \else
    \PackageError{pdflscape}{%
      The driver option `\PLS@option' is not compatible\MessageBreak
      with XeTeX, using driver for XeTeX (dvipdfm) instead%
    }\@ehc
    \let\PLS@driver\PLS@temp
  \fi
\fi
%    \end{macrocode}
%
% \subsubsection{Detect driver based on \cs{Gin@driver}}
%
%    \begin{macrocode}
\def\PLS@temp#1#2{%
  \ifx\PLS@driver\@empty
    \begingroup
      \def\PLS@temp{#1.def}%
    \expandafter\endgroup
    \ifx\PLS@temp\Gin@driver
      \def\PLS@driver{#2}%
      \PackageInfo{pdflscape}{%
        Auto-detected driver: \PLS@driver\space(#1)%
      }%
    \fi
  \fi
}
\PLS@temp{dvipdfm}{dvipdfm}
\PLS@temp{dvipdfmx}{dvipdfm}
\PLS@temp{dvips}{dvips}
\PLS@temp{dvipsone}{dvips}
\PLS@temp{pctexps}{dvips}
%    \end{macrocode}
%
% \subsubsection{Driver \xoption{dvips}}
%
%    \begin{macrocode}
\ifx\PLS@driver\@empty
  \def\PLS@temp{dvips}%
  \ifnum 1=0\ifx\Gin@PS@raw\@undefined
            \else
              \ifx\Gin@PS@raw\relax
              \else
                \ifx\Gin@PS@raw\@empty
                \else
                  1%
                \fi
              \fi
            \fi
            \space
    \let\PLS@driver\PLS@temp
    \PackageInfo{pdflscape}{%
      Auto-detected driver: \PLS@temp\space(PostScript)%
    }%
  \fi
\fi
%    \end{macrocode}
%
%    \begin{macrocode}
\ifx\PLS@driver\@empty
  \PackageError{pdflscape}{%
    Neither pdfTeX, PostScript, nor dvipdfm driver found%
  }\@ehc
\fi
%    \end{macrocode}
%
% \subsection{Driver implementation}
%
% \begin{itemize}
% \item |\PLS@AddRotate #1|\\
%   it expects the correct rotation number in |#1|
%   and implements the adding of the /Rotation entry
%   in the |/Page| object of the current page.
% \item |\PLS@RemoveRotate|\\
%   it removes a previous |/Rotate| entry, if necessary.
% \end{itemize}
%
% \subsubsection{\pdfTeX}
%
% Not too nice is the global setting of \cs{pdfpageattr}.
% Perhaps this can be changed in future versions.
%
% luatex compat.
%    \begin{macrocode}
\ifx\pdfpageattr\@undefined
  \ifx\pdfvariable\@undefined\else 
    \protected\edef\pdfpageattr{\pdfvariable pageattr}
  \fi
\fi
%    \end{macrocode}
%    \begin{macrocode}
\def\PLS@temp{pdftex}
\ifx\PLS@temp\PLS@driver
  \def\PLS@AddRotate#1{%
    \ifnum#1=0 %
%    \end{macrocode}
%    Already the default.
%    \begin{macrocode}
    \else
      \global\pdfpageattr\expandafter{%
        \the\pdfpageattr
        /Rotate #1%
      }%
    \fi
  }%
%    \end{macrocode}
%    Removes a |/Rotate| entry.
%    \begin{macrocode}
  \def\PLS@RemoveRotate{%
    \begingroup
      \global\pdfpageattr\expandafter{\expandafter}%
      \expandafter\PLS@@RemoveRotate
        \the\pdfpageattr /Rotate\@nil
    \endgroup
  }%
  \def\PLS@@RemoveRotate#1/Rotate#2\@nil{%
%    \end{macrocode}
%    Append |/Rotate| free stuff to \cs{pdfpageattr}.
%    \begin{macrocode}
    \global\pdfpageattr\expandafter{\the\pdfpageattr#1}%
    \ifx\\#2\\%
%    \end{macrocode}
%    Ready, because the detected |/Rotate| is part of
%    the end marker: |/Rotate\@nil|
%    \begin{macrocode}
    \else
%    \end{macrocode}
%    First read in the argument of |/Rotate|,
%    then continue parsing.
%    \begin{macrocode}
      \afterassignment\PLS@@RemoveRotate
      \count0=#2\@nil
    \fi
  }%
\fi
%    \end{macrocode}
%
% \subsubsection{PostScript driver}
%
%    \begin{macrocode}
\def\PLS@temp{dvips}
\ifx\PLS@temp\PLS@driver
  \RequirePackage{atbegshi}%
  \AtBeginShipout{\PLS@AtBeginShipout}%
  \let\PLS@AtBeginShipout\@empty
  \def\PLS@AddRotate#1{%
    \ifnum#1=0 %
    \else
      \def\PLS@AtBeginShipout{%
        \global\setbox\AtBeginShipoutBox\vbox{%
          \Gin@PS@raw{%
            [{ThisPage}\string<\string</Rotate #1\string>\string>%
            /PUT pdfmark%
          }%
          \box\AtBeginShipoutBox
        }%
      }%
    \fi
  }%
  \def\PLS@RemoveRotate{\let\PLS@AtBeginShipout\@empty}%
\fi
%    \end{macrocode}
%
% \subsubsection{Driver dvipdfm}
%
%    \begin{macrocode}
\def\PLS@temp{dvipdfm}
\ifx\PLS@temp\PLS@driver
  \RequirePackage{atbegshi}%
  \AtBeginShipout{\PLS@AtBeginShipout}%
  \let\PLS@AtBeginShipout\@empty
  \def\PLS@AddRotate#1{%
    \ifnum#1=0 %
    \else
      \def\PLS@AtBeginShipout{%
        \global\setbox\AtBeginShipoutBox\vbox{%
          \special{pdf: put @thispage <</Rotate #1>>}%
          \box\AtBeginShipoutBox
        }%
      }%
    \fi
  }%
  \def\PLS@RemoveRotate{\let\PLS@AtBeginShipout\@empty}%
\fi
%    \end{macrocode}
%
% \subsection{Driver independent stuff}
%
%    The landscape environment is extended by
%    adding the correct |/Rotate| entries.
%    \begin{macrocode}
\g@addto@macro{\landscape}{\PLS@Rotate{90}}
\g@addto@macro{\endlandscape}{\PLS@Rotate{0}}
%    \end{macrocode}
%
%    \begin{macro}{\PLS@Rotate}
%    Main macro, that sets the |/Rotate| entry.\\
%    Argument: any TeX number or nothing, that means zero.\\
%    Driver independent.
%    \begin{macrocode}
\def\PLS@Rotate#1{%
  \begingroup
%    \end{macrocode}
%    1. Check and validate the argument.
%    \begin{macrocode}
    \PLS@CheckAngle{#1}%
  \expandafter\endgroup
%    \end{macrocode}
%    2. Remove previous |\Rotate| entry.
%    \begin{macrocode}
  \expandafter\PLS@RemoveRotate
%    \end{macrocode}
%    3. Add |/Rotate| entry.
%    \begin{macrocode}
  \expandafter\PLS@AddRotate\expandafter{\the\count@}%
}
%    \end{macrocode}
%    \end{macro}
%
%    \begin{macro}{\PLS@CheckAngle}
%    Validates the rotation angle.\\
%    The result is stored in the count register |\count@|.\\
%    Driver independent.
%    \begin{macrocode}
\def\PLS@CheckAngle#1{%
%    \end{macrocode}
%    Nothing means zero:
%    \begin{macrocode}
  \ifx\\#1\\%
    \count@=0 %
  \else
    \count@=#1\relax
  \fi
%    \end{macrocode}
%    Normalize to interval -360 $<$ |\count@| $<$ 360.
%    \begin{macrocode}
  \@whilenum\count@>359\do{\advance\count@ -360 }%
  \@whilenum\count@<-359\do{\advance\count@ 360 }%
%    \end{macrocode}
%    Check values: 0, +/-90, +/-180, +/-270
%    \begin{macrocode}
  \ifnum 1=0\ifnum\count@=0 \else
            \ifnum\count@=90 \else
            \ifnum\count@=-90 \else
            \ifnum\count@=180 \else
            \ifnum\count@=-180 \else
            \ifnum\count@=270 \else
            \ifnum\count@=-270 \else
           1\fi\fi\fi\fi\fi\fi\fi\relax
    \PackageError{pdflscape}{%
      Invalid value \the\count@\space for /Rotate%
    }\@ehc
    \count@=0 %
  \fi
}
%    \end{macrocode}
%    \end{macro}
%
%    \begin{macrocode}
%</package>
%    \end{macrocode}
%
% \section{Test}
%
% \subsection{Driver detection tests}
%    \begin{macrocode}
%<*test0>
\NeedsTeXFormat{LaTeX2e}
\documentclass{minimal}
\usepackage{qstest}
\IncludeTests{*}
\LogTests{log}{*}{*}
%</test0>
%    \end{macrocode}
%
%    \begin{macrocode}
%<*test1>
\usepackage{ifpdf}
\ifpdf
  \def\ExpectDriver{pdftex}%
\fi
\usepackage{ifxetex}%
\ifxetex
  \def\ExpectDriver{dvipdfm}%
\fi
\usepackage{pdflscape}[2016/05/14]
%</test1>
%    \end{macrocode}
%
%    \begin{macrocode}
%<*test2>
\def\ExpectDriver{pdftex}
\usepackage[pdftex]{pdflscape}[2016/05/14]
%</test2>
%    \end{macrocode}
%
%    \begin{macrocode}
%<*test3>
\def\ExpectDriver{dvipdfm}
\usepackage[dvipdfmx]{pdflscape}[2016/05/14]
%</test3>
%    \end{macrocode}
%
%    \begin{macrocode}
%<*test4>
\def\ExpectDriver{dvips}
\usepackage[dvipsone]{graphics}
\usepackage{pdflscape}[2016/05/14]
%</test4>
%    \end{macrocode}
%
%    \begin{macrocode}
%<*test0>
\begin{document}
\begin{qstest}{driver}{driver}
  \makeatletter
  \Expect*{\PLS@driver}*{\ExpectDriver}%
\end{qstest}
\end{document}
%</test0>
%    \end{macrocode}
%
% \subsection{Test for \cs{PLS@CheckAngle}}
%
%    \begin{macrocode}
%<*test5>
\NeedsTeXFormat{LaTeX2e}
\documentclass{minimal}
\usepackage{qstest}
\IncludeTests{*}
\LogTests{log}{*}{*}
\usepackage{pdflscape}[2016/05/14]
\begin{document}
\begin{qstest}{checkangle}{checkangle}
  \makeatletter
  \def\Result{%
    \def\ResultString{\the\count@}%
  }%
  \def\PackageError#1#2#3{%
    \def\Result{%
      \def\ResultString{error}%
    }%
  }%
  \def\Test#1#2{%
    \begingroup
      \PLS@CheckAngle{#1}%
      \Result
      \Expect*{\ResultString}{#2}%
    \endgroup
  }%
  \Test{0}{0}%
  \Test{90}{90}%
  \Test{180}{180}%
  \Test{270}{270}%
  \Test{360}{0}%
  \Test{450}{90}%
  \Test{540}{180}%
  \Test{630}{270}%
  \Test{720}{0}%
  \Test{3600}{0}%
  \Test{3690}{90}%
  \Test{-90}{270}%
  \Test{-180}{180}%
  \Test{-270}{-90}%
  \Test{-360}{0}%
  \Test{-450}{270}%
  \Test{540}{180}%
  \Test{-630}{90}%
  \Test{-720}{0}%
  \Test{-3600}{0}%
  \Test{-3690}{270}%
  \Test{1}{error}%
  \Test{-1}{error}%
  \Test{123}{error}%
\end{qstest}
\end{document}
%</test5>
%    \end{macrocode}
%
% \subsection{Test for rotate}
%
%    \begin{macrocode}
%<*test6>
\NeedsTeXFormat{LaTeX2e}
\documentclass{article}
\usepackage{pdflscape}[2016/05/14]
\begin{document}
\section{Portrait page}
\begin{landscape}
\section{Landscape page}
Foo bar foo bar foo bar foo bar foo bar
foo bar foo bar foo bar foo bar foo bar.
\end{landscape}
\section{Portrait Section}
Foo bar
\begin{landscape}
\section{Again Landscape Section}
\newpage
\section{Second Page of Landscape Section}
\end{landscape}
\section{Second Last Portrait Page}
\newpage
\section{Last Portrait Page}
\end{document}
%</test6>
%    \end{macrocode}
%
%    \begin{macrocode}
%<*result6>
/Page 1 /Rotate 0
/Page 2 /Rotate 90
/Page 3 /Rotate 0
/Page 4 /Rotate 90
/Page 5 /Rotate 90
/Page 6 /Rotate 0
/Page 7 /Rotate 0
%</result6>
%    \end{macrocode}
%
%    \begin{macrocode}
%<*ExtractRotate>
/**
 * ExtractRotate.java
 *
 * Copyright (C) 2007 by Heiko Oberdiek <heiko.oberdiek at googlemail.com>
 *
 * Requires: PDFBox (http://www.pdfbox.org/)
 *
 * Syntax: java ExtractRotate <pdffile> <textfile>
 *
 * The <pdffile> is analyzed and for each page its rotation
 * setting is printed in the <textfile>. Example:
 *   /Page 1 /Rotate 0
 *   /Page 2 /Rotate 90
 */
import java.io.FileInputStream;
import java.io.FileWriter;
import org.pdfbox.pdfparser.PDFParser;
import org.pdfbox.pdmodel.PDDocument;
import org.pdfbox.pdmodel.PDDocumentCatalog;
import org.pdfbox.pdmodel.PDPage;

public class ExtractRotate {

    public static void main(String[] args) {
        try {
            String infile = args[0];
            String outfile = args[1];
            FileWriter out = new FileWriter(outfile);
            PDFParser parser =
                    new PDFParser(new FileInputStream(infile));
            parser.parse();
            PDDocument document = parser.getPDDocument();
            PDDocumentCatalog catalog = document.getDocumentCatalog();
            int i = 0;
            for (Object page: catalog.getAllPages()) {
                i++;
                out.write("/Page " + i + " " + "/Rotate "
                        + ((PDPage)page).findRotation() + "\n");
            }
            document.close();
            out.close();
        }
        catch (Exception e) {
            e.printStackTrace();
            System.exit(1);
        }
    }
}
%</ExtractRotate>
%    \end{macrocode}
%
% \section{Installation}
%
% \subsection{Download}
%
% \paragraph{Package.} This package is available on
% CTAN\footnote{\CTANpkg{pdflscape}}:
% \begin{description}
% \item[\CTAN{macros/latex/contrib/oberdiek/pdflscape.dtx}] The source file.
% \item[\CTAN{macros/latex/contrib/oberdiek/pdflscape.pdf}] Documentation.
% \end{description}
%
%
% \paragraph{Bundle.} All the packages of the bundle `oberdiek'
% are also available in a TDS compliant ZIP archive. There
% the packages are already unpacked and the documentation files
% are generated. The files and directories obey the TDS standard.
% \begin{description}
% \item[\CTANinstall{install/macros/latex/contrib/oberdiek.tds.zip}]
% \end{description}
% \emph{TDS} refers to the standard ``A Directory Structure
% for \TeX\ Files'' (\CTAN{tds/tds.pdf}). Directories
% with \xfile{texmf} in their name are usually organized this way.
%
% \subsection{Bundle installation}
%
% \paragraph{Unpacking.} Unpack the \xfile{oberdiek.tds.zip} in the
% TDS tree (also known as \xfile{texmf} tree) of your choice.
% Example (linux):
% \begin{quote}
%   |unzip oberdiek.tds.zip -d ~/texmf|
% \end{quote}
%
% \paragraph{Script installation.}
% Check the directory \xfile{TDS:scripts/oberdiek/} for
% scripts that need further installation steps.
% Package \xpackage{attachfile2} comes with the Perl script
% \xfile{pdfatfi.pl} that should be installed in such a way
% that it can be called as \texttt{pdfatfi}.
% Example (linux):
% \begin{quote}
%   |chmod +x scripts/oberdiek/pdfatfi.pl|\\
%   |cp scripts/oberdiek/pdfatfi.pl /usr/local/bin/|
% \end{quote}
%
% \subsection{Package installation}
%
% \paragraph{Unpacking.} The \xfile{.dtx} file is a self-extracting
% \docstrip\ archive. The files are extracted by running the
% \xfile{.dtx} through \plainTeX:
% \begin{quote}
%   \verb|tex pdflscape.dtx|
% \end{quote}
%
% \paragraph{TDS.} Now the different files must be moved into
% the different directories in your installation TDS tree
% (also known as \xfile{texmf} tree):
% \begin{quote}
% \def\t{^^A
% \begin{tabular}{@{}>{\ttfamily}l@{ $\rightarrow$ }>{\ttfamily}l@{}}
%   pdflscape.sty & tex/latex/oberdiek/pdflscape.sty\\
%   pdflscape.pdf & doc/latex/oberdiek/pdflscape.pdf\\
%   test/pdflscape-test1.tex & doc/latex/oberdiek/test/pdflscape-test1.tex\\
%   test/pdflscape-test2.tex & doc/latex/oberdiek/test/pdflscape-test2.tex\\
%   test/pdflscape-test3.tex & doc/latex/oberdiek/test/pdflscape-test3.tex\\
%   test/pdflscape-test4.tex & doc/latex/oberdiek/test/pdflscape-test4.tex\\
%   test/pdflscape-test5.tex & doc/latex/oberdiek/test/pdflscape-test5.tex\\
%   test/pdflscape-test6.tex & doc/latex/oberdiek/test/pdflscape-test6.tex\\
%   test/pdflscape-test6.txt & doc/latex/oberdiek/test/pdflscape-test6.txt\\
%   test/ExtractRotate.java & doc/latex/oberdiek/test/ExtractRotate.java\\
%   pdflscape.dtx & source/latex/oberdiek/pdflscape.dtx\\
% \end{tabular}^^A
% }^^A
% \sbox0{\t}^^A
% \ifdim\wd0>\linewidth
%   \begingroup
%     \advance\linewidth by\leftmargin
%     \advance\linewidth by\rightmargin
%   \edef\x{\endgroup
%     \def\noexpand\lw{\the\linewidth}^^A
%   }\x
%   \def\lwbox{^^A
%     \leavevmode
%     \hbox to \linewidth{^^A
%       \kern-\leftmargin\relax
%       \hss
%       \usebox0
%       \hss
%       \kern-\rightmargin\relax
%     }^^A
%   }^^A
%   \ifdim\wd0>\lw
%     \sbox0{\small\t}^^A
%     \ifdim\wd0>\linewidth
%       \ifdim\wd0>\lw
%         \sbox0{\footnotesize\t}^^A
%         \ifdim\wd0>\linewidth
%           \ifdim\wd0>\lw
%             \sbox0{\scriptsize\t}^^A
%             \ifdim\wd0>\linewidth
%               \ifdim\wd0>\lw
%                 \sbox0{\tiny\t}^^A
%                 \ifdim\wd0>\linewidth
%                   \lwbox
%                 \else
%                   \usebox0
%                 \fi
%               \else
%                 \lwbox
%               \fi
%             \else
%               \usebox0
%             \fi
%           \else
%             \lwbox
%           \fi
%         \else
%           \usebox0
%         \fi
%       \else
%         \lwbox
%       \fi
%     \else
%       \usebox0
%     \fi
%   \else
%     \lwbox
%   \fi
% \else
%   \usebox0
% \fi
% \end{quote}
% If you have a \xfile{docstrip.cfg} that configures and enables \docstrip's
% TDS installing feature, then some files can already be in the right
% place, see the documentation of \docstrip.
%
% \subsection{Refresh file name databases}
%
% If your \TeX~distribution
% (\teTeX, \mikTeX, \dots) relies on file name databases, you must refresh
% these. For example, \teTeX\ users run \verb|texhash| or
% \verb|mktexlsr|.
%
% \subsection{Some details for the interested}
%
% \paragraph{Attached source.}
%
% The PDF documentation on CTAN also includes the
% \xfile{.dtx} source file. It can be extracted by
% AcrobatReader 6 or higher. Another option is \textsf{pdftk},
% e.g. unpack the file into the current directory:
% \begin{quote}
%   \verb|pdftk pdflscape.pdf unpack_files output .|
% \end{quote}
%
% \paragraph{Unpacking with \LaTeX.}
% The \xfile{.dtx} chooses its action depending on the format:
% \begin{description}
% \item[\plainTeX:] Run \docstrip\ and extract the files.
% \item[\LaTeX:] Generate the documentation.
% \end{description}
% If you insist on using \LaTeX\ for \docstrip\ (really,
% \docstrip\ does not need \LaTeX), then inform the autodetect routine
% about your intention:
% \begin{quote}
%   \verb|latex \let\install=y\input{pdflscape.dtx}|
% \end{quote}
% Do not forget to quote the argument according to the demands
% of your shell.
%
% \paragraph{Generating the documentation.}
% You can use both the \xfile{.dtx} or the \xfile{.drv} to generate
% the documentation. The process can be configured by the
% configuration file \xfile{ltxdoc.cfg}. For instance, put this
% line into this file, if you want to have A4 as paper format:
% \begin{quote}
%   \verb|\PassOptionsToClass{a4paper}{article}|
% \end{quote}
% An example follows how to generate the
% documentation with pdf\LaTeX:
% \begin{quote}
%\begin{verbatim}
%pdflatex pdflscape.dtx
%makeindex -s gind.ist pdflscape.idx
%pdflatex pdflscape.dtx
%makeindex -s gind.ist pdflscape.idx
%pdflatex pdflscape.dtx
%\end{verbatim}
% \end{quote}
%
% \begin{History}
%   \begin{Version}{2001/01/15 v0.1}
%   \item
%     First public version,
%     published in \xnewsgroup{de.comp.text.tex}:\\
%     \URL{``\link{Re: Querformat, pdfLaTex und thumbpdf}''}^^A
%     {https://groups.google.com/group/de.comp.text.tex/msg/e054c5795e52d2b8}
%   \end{Version}
%   \begin{Version}{2001/02/04 v0.2}
%   \item
%     Minor documentation update.
%   \item
%     CTAN.
%   \end{Version}
%   \begin{Version}{2004/05/11 v0.3}
%   \item
%     Support for dvipdfm added.
%   \end{Version}
%   \begin{Version}{2004/05/12 v0.4}
%   \item
%     Bug fix: support for multipage landscape environment
%     for all drivers except for \xoption{pdftex}.
%   \end{Version}
%   \begin{Version}{2006/02/20 v0.5}
%   \item
%     DTX framework.
%   \item
%     LPPL 1.3
%   \item
%     Code is not changed.
%   \end{Version}
%   \begin{Version}{2006/04/24 v0.6}
%   \item
%     Bug fix: \cs{PLS@RemoveRotate} reinserted for \cs{PLS@Rotate}.
%   \item
%     Fix for \cs{PLS@RemoveRotate} (dvips/dvipdfm): \cs{PLS@EverhShi}
%     hook is cleared.
%   \end{Version}
%   \begin{Version}{2007/04/11 v0.7}
%   \item
%     Line ends sanitized.
%   \end{Version}
%   \begin{Version}{2007/04/17 v0.8}
%   \item
%     Package \xpackage{atbegshi} replaces \xpackage{everyshi}.
%   \end{Version}
%   \begin{Version}{2007/10/21 v0.9}
%   \item
%     Driver detection for \XeTeX\ added.
%   \item
%     Fix for rotation angles $\le$~-360 or $\ge$~360.
%   \end{Version}
%   \begin{Version}{2008/08/11 v0.10}
%   \item
%     Code is not changed.
%   \item
%     URLs updated.
%   \end{Version}
%   \begin{Version}{2016/05/14 v0.11}
%   \item
%     LuaTeX Compatibility
%   \end{Version}
% \end{History}
%
% \PrintIndex
%
% \Finale
\endinput
|
% \end{quote}
% Do not forget to quote the argument according to the demands
% of your shell.
%
% \paragraph{Generating the documentation.}
% You can use both the \xfile{.dtx} or the \xfile{.drv} to generate
% the documentation. The process can be configured by the
% configuration file \xfile{ltxdoc.cfg}. For instance, put this
% line into this file, if you want to have A4 as paper format:
% \begin{quote}
%   \verb|\PassOptionsToClass{a4paper}{article}|
% \end{quote}
% An example follows how to generate the
% documentation with pdf\LaTeX:
% \begin{quote}
%\begin{verbatim}
%pdflatex pdflscape.dtx
%makeindex -s gind.ist pdflscape.idx
%pdflatex pdflscape.dtx
%makeindex -s gind.ist pdflscape.idx
%pdflatex pdflscape.dtx
%\end{verbatim}
% \end{quote}
%
% \begin{History}
%   \begin{Version}{2001/01/15 v0.1}
%   \item
%     First public version,
%     published in \xnewsgroup{de.comp.text.tex}:\\
%     \URL{``\link{Re: Querformat, pdfLaTex und thumbpdf}''}^^A
%     {https://groups.google.com/group/de.comp.text.tex/msg/e054c5795e52d2b8}
%   \end{Version}
%   \begin{Version}{2001/02/04 v0.2}
%   \item
%     Minor documentation update.
%   \item
%     CTAN.
%   \end{Version}
%   \begin{Version}{2004/05/11 v0.3}
%   \item
%     Support for dvipdfm added.
%   \end{Version}
%   \begin{Version}{2004/05/12 v0.4}
%   \item
%     Bug fix: support for multipage landscape environment
%     for all drivers except for \xoption{pdftex}.
%   \end{Version}
%   \begin{Version}{2006/02/20 v0.5}
%   \item
%     DTX framework.
%   \item
%     LPPL 1.3
%   \item
%     Code is not changed.
%   \end{Version}
%   \begin{Version}{2006/04/24 v0.6}
%   \item
%     Bug fix: \cs{PLS@RemoveRotate} reinserted for \cs{PLS@Rotate}.
%   \item
%     Fix for \cs{PLS@RemoveRotate} (dvips/dvipdfm): \cs{PLS@EverhShi}
%     hook is cleared.
%   \end{Version}
%   \begin{Version}{2007/04/11 v0.7}
%   \item
%     Line ends sanitized.
%   \end{Version}
%   \begin{Version}{2007/04/17 v0.8}
%   \item
%     Package \xpackage{atbegshi} replaces \xpackage{everyshi}.
%   \end{Version}
%   \begin{Version}{2007/10/21 v0.9}
%   \item
%     Driver detection for \XeTeX\ added.
%   \item
%     Fix for rotation angles $\le$~-360 or $\ge$~360.
%   \end{Version}
%   \begin{Version}{2008/08/11 v0.10}
%   \item
%     Code is not changed.
%   \item
%     URLs updated.
%   \end{Version}
%   \begin{Version}{2016/05/14 v0.11}
%   \item
%     LuaTeX Compatibility
%   \end{Version}
% \end{History}
%
% \PrintIndex
%
% \Finale
\endinput

%        (quote the arguments according to the demands of your shell)
%
% Documentation:
%    (a) If pdflscape.drv is present:
%           latex pdflscape.drv
%    (b) Without pdflscape.drv:
%           latex pdflscape.dtx; ...
%    The class ltxdoc loads the configuration file ltxdoc.cfg
%    if available. Here you can specify further options, e.g.
%    use A4 as paper format:
%       \PassOptionsToClass{a4paper}{article}
%
%    Programm calls to get the documentation (example):
%       pdflatex pdflscape.dtx
%       makeindex -s gind.ist pdflscape.idx
%       pdflatex pdflscape.dtx
%       makeindex -s gind.ist pdflscape.idx
%       pdflatex pdflscape.dtx
%
% Installation:
%    TDS:tex/latex/oberdiek/pdflscape.sty
%    TDS:doc/latex/oberdiek/pdflscape.pdf
%    TDS:doc/latex/oberdiek/test/pdflscape-test1.tex
%    TDS:doc/latex/oberdiek/test/pdflscape-test2.tex
%    TDS:doc/latex/oberdiek/test/pdflscape-test3.tex
%    TDS:doc/latex/oberdiek/test/pdflscape-test4.tex
%    TDS:doc/latex/oberdiek/test/pdflscape-test5.tex
%    TDS:doc/latex/oberdiek/test/pdflscape-test6.tex
%    TDS:doc/latex/oberdiek/test/pdflscape-test6.txt
%    TDS:doc/latex/oberdiek/test/ExtractRotate.java
%    TDS:source/latex/oberdiek/pdflscape.dtx
%
%<*ignore>
\begingroup
  \catcode123=1 %
  \catcode125=2 %
  \def\x{LaTeX2e}%
\expandafter\endgroup
\ifcase 0\ifx\install y1\fi\expandafter
         \ifx\csname processbatchFile\endcsname\relax\else1\fi
         \ifx\fmtname\x\else 1\fi\relax
\else\csname fi\endcsname
%</ignore>
%<*install>
\input docstrip.tex
\Msg{************************************************************************}
\Msg{* Installation}
\Msg{* Package: pdflscape 2016/05/14 v0.11 Display of landscape pages in PDF (HO)}
\Msg{************************************************************************}

\keepsilent
\askforoverwritefalse

\let\MetaPrefix\relax
\preamble

This is a generated file.

Project: pdflscape
Version: 2016/05/14 v0.11

Copyright (C) 2001, 2004, 2006-2008 by
   Heiko Oberdiek <heiko.oberdiek at googlemail.com>

This work may be distributed and/or modified under the
conditions of the LaTeX Project Public License, either
version 1.3c of this license or (at your option) any later
version. This version of this license is in
   http://www.latex-project.org/lppl/lppl-1-3c.txt
and the latest version of this license is in
   http://www.latex-project.org/lppl.txt
and version 1.3 or later is part of all distributions of
LaTeX version 2005/12/01 or later.

This work has the LPPL maintenance status "maintained".

This Current Maintainer of this work is Heiko Oberdiek.

This work consists of the main source file pdflscape.dtx
and the derived files
   pdflscape.sty, pdflscape.pdf, pdflscape.ins, pdflscape.drv,
   pdflscape-test1.tex, pdflscape-test2.tex,
   pdflscape-test3.tex, pdflscape-test4.tex,
   pdflscape-test5.tex, pdflscape-test6.tex,
   pdflscape-test6.txt, ExtractRotate.java.

\endpreamble
\let\MetaPrefix\DoubleperCent

\generate{%
  \file{pdflscape.ins}{\from{pdflscape.dtx}{install}}%
  \file{pdflscape.drv}{\from{pdflscape.dtx}{driver}}%
  \usedir{tex/latex/oberdiek}%
  \file{pdflscape.sty}{\from{pdflscape.dtx}{package}}%
%  \usedir{doc/latex/oberdiek/test}%
%  \file{pdflscape-test1.tex}{\from{pdflscape.dtx}{test0,test1}}%
%  \file{pdflscape-test2.tex}{\from{pdflscape.dtx}{test0,test2}}%
%  \file{pdflscape-test3.tex}{\from{pdflscape.dtx}{test0,test3}}%
%  \file{pdflscape-test4.tex}{\from{pdflscape.dtx}{test0,test4}}%
%  \file{pdflscape-test5.tex}{\from{pdflscape.dtx}{test5}}%
%  \file{pdflscape-test6.tex}{\from{pdflscape.dtx}{test6}}%
  \nopreamble
  \nopostamble
%  \file{pdflscape-test6.txt}{\from{pdflscape.dtx}{result6}}%
  \file{ExtractRotate.java}{\from{pdflscape.dtx}{ExtractRotate}}%
%  \usedir{source/latex/oberdiek/catalogue}%
%  \file{pdflscape.xml}{\from{pdflscape.dtx}{catalogue}}%
}

\catcode32=13\relax% active space
\let =\space%
\Msg{************************************************************************}
\Msg{*}
\Msg{* To finish the installation you have to move the following}
\Msg{* file into a directory searched by TeX:}
\Msg{*}
\Msg{*     pdflscape.sty}
\Msg{*}
\Msg{* To produce the documentation run the file `pdflscape.drv'}
\Msg{* through LaTeX.}
\Msg{*}
\Msg{* Happy TeXing!}
\Msg{*}
\Msg{************************************************************************}

\endbatchfile
%</install>
%<*ignore>
\fi
%</ignore>
%<*driver>
\NeedsTeXFormat{LaTeX2e}
\ProvidesFile{pdflscape.drv}%
  [2016/05/14 v0.11 Display of landscape pages in PDF (HO)]%
\documentclass{ltxdoc}
\usepackage{holtxdoc}[2011/11/22]
\usepackage{graphics}
\begin{document}
  \DocInput{pdflscape.dtx}%
\end{document}
%</driver>
% \fi
%
%
% \CharacterTable
%  {Upper-case    \A\B\C\D\E\F\G\H\I\J\K\L\M\N\O\P\Q\R\S\T\U\V\W\X\Y\Z
%   Lower-case    \a\b\c\d\e\f\g\h\i\j\k\l\m\n\o\p\q\r\s\t\u\v\w\x\y\z
%   Digits        \0\1\2\3\4\5\6\7\8\9
%   Exclamation   \!     Double quote  \"     Hash (number) \#
%   Dollar        \$     Percent       \%     Ampersand     \&
%   Acute accent  \'     Left paren    \(     Right paren   \)
%   Asterisk      \*     Plus          \+     Comma         \,
%   Minus         \-     Point         \.     Solidus       \/
%   Colon         \:     Semicolon     \;     Less than     \<
%   Equals        \=     Greater than  \>     Question mark \?
%   Commercial at \@     Left bracket  \[     Backslash     \\
%   Right bracket \]     Circumflex    \^     Underscore    \_
%   Grave accent  \`     Left brace    \{     Vertical bar  \|
%   Right brace   \}     Tilde         \~}
%
% \GetFileInfo{pdflscape.drv}
%
% \title{The \xpackage{pdflscape} package}
% \date{2016/05/14 v0.11}
% \author{Heiko Oberdiek\thanks
% {Please report any issues at \url{https://github.com/ho-tex/oberdiek/issues}}\\
% \xemail{heiko.oberdiek at googlemail.com}}
%
% \maketitle
%
% \begin{abstract}
% Package \xpackage{pdflscape} adds PDF support to the
% environment \texttt{landscape} of package \xpackage{lscape}
% by setting the PDF page attribute \texttt{/Rotate}.
% \end{abstract}
%
% \tableofcontents
%
% \section{Documentation}
%
% \subsection{Supported drivers}
%
% \begin{itemize}
% \item \xoption{pdftex},\xoption{luatex}
% \item \xoption{dvips}, \xoption{dvipsone},
%       \xoption{pctex32}, \xoption{pctexps}
%       (and other drivers that provide a non-empty \cs{Gin@PS@raw})
% \item \xoption{dvipdfm}
% \end{itemize}
%
% \subsection{Caveat}
%
% Depending on the configuration Ghostscript
% adds a guessed rotation entry by its own.
% This can lead to two |/Rotate| entries per page.
% To prevent this behaviour set the parameter
% |AutoRotatePages| to |/None|, eg:
% \begin{quote}
%   |ps2pdf -dAutoRotatePages=/None|
% \end{quote}
%
% \subsection{Requirements}
%
% \begin{itemize}
% \item The package \xpackage{lscape}.
% \item The package \xpackage{atbegshi} for all drivers
%        except \xoption{pdftex},\xoption{luatex}.
% \end{itemize}
%
% \subsection{Usage}
%
% Load this package instead of or after package \xpackage{lscape}:
% \begin{quote}
%   |\usepackage{pdflscape}|\\
%   |\begin{landscape}|\dots|\end{landscape}|
% \end{quote}
%
% \StopEventually{
% }
%
% \section{Implementation}
%
% \subsection{Package identification}
%
%    \begin{macrocode}
%<*package>
%    \end{macrocode}
%    Package identification.
%    \begin{macrocode}
\NeedsTeXFormat{LaTeX2e}
\ProvidesPackage{pdflscape}%
  [2016/05/14 v0.11 Display of landscape pages in PDF (HO)]
%    \end{macrocode}
%
% \subsection{Driver options}
%
%    \begin{macrocode}
\let\PLS@option\@empty
\let\PLS@driver\@empty
\def\PLS@temp#1#2{%
  \DeclareOption{#1}{%
    \let\PLS@option\CurrentOption
    \def\PLS@driver{#2}%
  }%
}
\PLS@temp{pdftex}{pdftex}
\PLS@temp{luatex}{pdftex}
\PLS@temp{dvips}{dvips}
\PLS@temp{dvipsone}{dvips}
\PLS@temp{pctexps}{dvips}
\PLS@temp{dvipdfm}{dvipdfm}
\PLS@temp{dvipdfmx}{dvipdfm}
\PLS@temp{xetex}{dvipdfm}
\let\PLS@temp\@empty
\DeclareOption*{%
  \ifx\PLS@temp\@empty
    \let\PLS@temp\CurrentOption
  \else
    \edef\PLS@temp{\PLS@temp,\CurrentOption}%
  \fi
}
\ProcessOptions*
\ifx\PLS@option\@empty
\else
  \expandafter\PassOptionsToPackage\expandafter{\PLS@option}{lscape}%
\fi
\ifx\PLS@temp\@empty
\else
  \expandafter\PassOptionsToPackage\expandafter{\PLS@temp}{lscape}%
\fi
\RequirePackage{lscape}
%    \end{macrocode}
%
% \subsection{Autodetection of driver}
%
% \subsubsection{Driver \xoption{pdftex}}
%
%    \begin{macrocode}
\RequirePackage{ifpdf}[2006/02/20]
\def\PLS@temp{pdftex}
\ifpdf
  \ifx\PLS@driver\@empty
    \let\PLS@driver\PLS@temp
    \PackageInfo{pdflscape}{Auto-detected driver: \PLS@temp}%
  \fi
  \ifx\PLS@driver\PLS@temp
  \else
    \PackageError{pdflscape}{%
      The driver option `\PLS@option' is not compatible\MessageBreak
      with pdfTeX in PDF mode, using driver `pdftex' instead%
    }\@ehc
    \let\PLS@driver\PLS@temp
  \fi
\fi
%    \end{macrocode}
%
% \subsubsection{Driver \xoption{xetex}}
%
%    \begin{macrocode}
\RequirePackage{ifxetex}%
\def\PLS@temp{dvipdfm}
\ifxetex
  \ifx\PLS@driver\@empty
    \let\PLS@driver\PLS@temp
    \PackageInfo{pdflscape}{%
      Auto-detected driver: \PLS@temp\space(xetex)%
    }%
  \fi
  \ifx\PLS@driver\PLS@temp
  \else
    \PackageError{pdflscape}{%
      The driver option `\PLS@option' is not compatible\MessageBreak
      with XeTeX, using driver for XeTeX (dvipdfm) instead%
    }\@ehc
    \let\PLS@driver\PLS@temp
  \fi
\fi
%    \end{macrocode}
%
% \subsubsection{Detect driver based on \cs{Gin@driver}}
%
%    \begin{macrocode}
\def\PLS@temp#1#2{%
  \ifx\PLS@driver\@empty
    \begingroup
      \def\PLS@temp{#1.def}%
    \expandafter\endgroup
    \ifx\PLS@temp\Gin@driver
      \def\PLS@driver{#2}%
      \PackageInfo{pdflscape}{%
        Auto-detected driver: \PLS@driver\space(#1)%
      }%
    \fi
  \fi
}
\PLS@temp{dvipdfm}{dvipdfm}
\PLS@temp{dvipdfmx}{dvipdfm}
\PLS@temp{dvips}{dvips}
\PLS@temp{dvipsone}{dvips}
\PLS@temp{pctexps}{dvips}
%    \end{macrocode}
%
% \subsubsection{Driver \xoption{dvips}}
%
%    \begin{macrocode}
\ifx\PLS@driver\@empty
  \def\PLS@temp{dvips}%
  \ifnum 1=0\ifx\Gin@PS@raw\@undefined
            \else
              \ifx\Gin@PS@raw\relax
              \else
                \ifx\Gin@PS@raw\@empty
                \else
                  1%
                \fi
              \fi
            \fi
            \space
    \let\PLS@driver\PLS@temp
    \PackageInfo{pdflscape}{%
      Auto-detected driver: \PLS@temp\space(PostScript)%
    }%
  \fi
\fi
%    \end{macrocode}
%
%    \begin{macrocode}
\ifx\PLS@driver\@empty
  \PackageError{pdflscape}{%
    Neither pdfTeX, PostScript, nor dvipdfm driver found%
  }\@ehc
\fi
%    \end{macrocode}
%
% \subsection{Driver implementation}
%
% \begin{itemize}
% \item |\PLS@AddRotate #1|\\
%   it expects the correct rotation number in |#1|
%   and implements the adding of the /Rotation entry
%   in the |/Page| object of the current page.
% \item |\PLS@RemoveRotate|\\
%   it removes a previous |/Rotate| entry, if necessary.
% \end{itemize}
%
% \subsubsection{\pdfTeX}
%
% Not too nice is the global setting of \cs{pdfpageattr}.
% Perhaps this can be changed in future versions.
%
% luatex compat.
%    \begin{macrocode}
\ifx\pdfpageattr\@undefined
  \ifx\pdfvariable\@undefined\else 
    \protected\edef\pdfpageattr{\pdfvariable pageattr}
  \fi
\fi
%    \end{macrocode}
%    \begin{macrocode}
\def\PLS@temp{pdftex}
\ifx\PLS@temp\PLS@driver
  \def\PLS@AddRotate#1{%
    \ifnum#1=0 %
%    \end{macrocode}
%    Already the default.
%    \begin{macrocode}
    \else
      \global\pdfpageattr\expandafter{%
        \the\pdfpageattr
        /Rotate #1%
      }%
    \fi
  }%
%    \end{macrocode}
%    Removes a |/Rotate| entry.
%    \begin{macrocode}
  \def\PLS@RemoveRotate{%
    \begingroup
      \global\pdfpageattr\expandafter{\expandafter}%
      \expandafter\PLS@@RemoveRotate
        \the\pdfpageattr /Rotate\@nil
    \endgroup
  }%
  \def\PLS@@RemoveRotate#1/Rotate#2\@nil{%
%    \end{macrocode}
%    Append |/Rotate| free stuff to \cs{pdfpageattr}.
%    \begin{macrocode}
    \global\pdfpageattr\expandafter{\the\pdfpageattr#1}%
    \ifx\\#2\\%
%    \end{macrocode}
%    Ready, because the detected |/Rotate| is part of
%    the end marker: |/Rotate\@nil|
%    \begin{macrocode}
    \else
%    \end{macrocode}
%    First read in the argument of |/Rotate|,
%    then continue parsing.
%    \begin{macrocode}
      \afterassignment\PLS@@RemoveRotate
      \count0=#2\@nil
    \fi
  }%
\fi
%    \end{macrocode}
%
% \subsubsection{PostScript driver}
%
%    \begin{macrocode}
\def\PLS@temp{dvips}
\ifx\PLS@temp\PLS@driver
  \RequirePackage{atbegshi}%
  \AtBeginShipout{\PLS@AtBeginShipout}%
  \let\PLS@AtBeginShipout\@empty
  \def\PLS@AddRotate#1{%
    \ifnum#1=0 %
    \else
      \def\PLS@AtBeginShipout{%
        \global\setbox\AtBeginShipoutBox\vbox{%
          \Gin@PS@raw{%
            [{ThisPage}\string<\string</Rotate #1\string>\string>%
            /PUT pdfmark%
          }%
          \box\AtBeginShipoutBox
        }%
      }%
    \fi
  }%
  \def\PLS@RemoveRotate{\let\PLS@AtBeginShipout\@empty}%
\fi
%    \end{macrocode}
%
% \subsubsection{Driver dvipdfm}
%
%    \begin{macrocode}
\def\PLS@temp{dvipdfm}
\ifx\PLS@temp\PLS@driver
  \RequirePackage{atbegshi}%
  \AtBeginShipout{\PLS@AtBeginShipout}%
  \let\PLS@AtBeginShipout\@empty
  \def\PLS@AddRotate#1{%
    \ifnum#1=0 %
    \else
      \def\PLS@AtBeginShipout{%
        \global\setbox\AtBeginShipoutBox\vbox{%
          \special{pdf: put @thispage <</Rotate #1>>}%
          \box\AtBeginShipoutBox
        }%
      }%
    \fi
  }%
  \def\PLS@RemoveRotate{\let\PLS@AtBeginShipout\@empty}%
\fi
%    \end{macrocode}
%
% \subsection{Driver independent stuff}
%
%    The landscape environment is extended by
%    adding the correct |/Rotate| entries.
%    \begin{macrocode}
\g@addto@macro{\landscape}{\PLS@Rotate{90}}
\g@addto@macro{\endlandscape}{\PLS@Rotate{0}}
%    \end{macrocode}
%
%    \begin{macro}{\PLS@Rotate}
%    Main macro, that sets the |/Rotate| entry.\\
%    Argument: any TeX number or nothing, that means zero.\\
%    Driver independent.
%    \begin{macrocode}
\def\PLS@Rotate#1{%
  \begingroup
%    \end{macrocode}
%    1. Check and validate the argument.
%    \begin{macrocode}
    \PLS@CheckAngle{#1}%
  \expandafter\endgroup
%    \end{macrocode}
%    2. Remove previous |\Rotate| entry.
%    \begin{macrocode}
  \expandafter\PLS@RemoveRotate
%    \end{macrocode}
%    3. Add |/Rotate| entry.
%    \begin{macrocode}
  \expandafter\PLS@AddRotate\expandafter{\the\count@}%
}
%    \end{macrocode}
%    \end{macro}
%
%    \begin{macro}{\PLS@CheckAngle}
%    Validates the rotation angle.\\
%    The result is stored in the count register |\count@|.\\
%    Driver independent.
%    \begin{macrocode}
\def\PLS@CheckAngle#1{%
%    \end{macrocode}
%    Nothing means zero:
%    \begin{macrocode}
  \ifx\\#1\\%
    \count@=0 %
  \else
    \count@=#1\relax
  \fi
%    \end{macrocode}
%    Normalize to interval -360 $<$ |\count@| $<$ 360.
%    \begin{macrocode}
  \@whilenum\count@>359\do{\advance\count@ -360 }%
  \@whilenum\count@<-359\do{\advance\count@ 360 }%
%    \end{macrocode}
%    Check values: 0, +/-90, +/-180, +/-270
%    \begin{macrocode}
  \ifnum 1=0\ifnum\count@=0 \else
            \ifnum\count@=90 \else
            \ifnum\count@=-90 \else
            \ifnum\count@=180 \else
            \ifnum\count@=-180 \else
            \ifnum\count@=270 \else
            \ifnum\count@=-270 \else
           1\fi\fi\fi\fi\fi\fi\fi\relax
    \PackageError{pdflscape}{%
      Invalid value \the\count@\space for /Rotate%
    }\@ehc
    \count@=0 %
  \fi
}
%    \end{macrocode}
%    \end{macro}
%
%    \begin{macrocode}
%</package>
%    \end{macrocode}
%
% \section{Test}
%
% \subsection{Driver detection tests}
%    \begin{macrocode}
%<*test0>
\NeedsTeXFormat{LaTeX2e}
\documentclass{minimal}
\usepackage{qstest}
\IncludeTests{*}
\LogTests{log}{*}{*}
%</test0>
%    \end{macrocode}
%
%    \begin{macrocode}
%<*test1>
\usepackage{ifpdf}
\ifpdf
  \def\ExpectDriver{pdftex}%
\fi
\usepackage{ifxetex}%
\ifxetex
  \def\ExpectDriver{dvipdfm}%
\fi
\usepackage{pdflscape}[2016/05/14]
%</test1>
%    \end{macrocode}
%
%    \begin{macrocode}
%<*test2>
\def\ExpectDriver{pdftex}
\usepackage[pdftex]{pdflscape}[2016/05/14]
%</test2>
%    \end{macrocode}
%
%    \begin{macrocode}
%<*test3>
\def\ExpectDriver{dvipdfm}
\usepackage[dvipdfmx]{pdflscape}[2016/05/14]
%</test3>
%    \end{macrocode}
%
%    \begin{macrocode}
%<*test4>
\def\ExpectDriver{dvips}
\usepackage[dvipsone]{graphics}
\usepackage{pdflscape}[2016/05/14]
%</test4>
%    \end{macrocode}
%
%    \begin{macrocode}
%<*test0>
\begin{document}
\begin{qstest}{driver}{driver}
  \makeatletter
  \Expect*{\PLS@driver}*{\ExpectDriver}%
\end{qstest}
\end{document}
%</test0>
%    \end{macrocode}
%
% \subsection{Test for \cs{PLS@CheckAngle}}
%
%    \begin{macrocode}
%<*test5>
\NeedsTeXFormat{LaTeX2e}
\documentclass{minimal}
\usepackage{qstest}
\IncludeTests{*}
\LogTests{log}{*}{*}
\usepackage{pdflscape}[2016/05/14]
\begin{document}
\begin{qstest}{checkangle}{checkangle}
  \makeatletter
  \def\Result{%
    \def\ResultString{\the\count@}%
  }%
  \def\PackageError#1#2#3{%
    \def\Result{%
      \def\ResultString{error}%
    }%
  }%
  \def\Test#1#2{%
    \begingroup
      \PLS@CheckAngle{#1}%
      \Result
      \Expect*{\ResultString}{#2}%
    \endgroup
  }%
  \Test{0}{0}%
  \Test{90}{90}%
  \Test{180}{180}%
  \Test{270}{270}%
  \Test{360}{0}%
  \Test{450}{90}%
  \Test{540}{180}%
  \Test{630}{270}%
  \Test{720}{0}%
  \Test{3600}{0}%
  \Test{3690}{90}%
  \Test{-90}{270}%
  \Test{-180}{180}%
  \Test{-270}{-90}%
  \Test{-360}{0}%
  \Test{-450}{270}%
  \Test{540}{180}%
  \Test{-630}{90}%
  \Test{-720}{0}%
  \Test{-3600}{0}%
  \Test{-3690}{270}%
  \Test{1}{error}%
  \Test{-1}{error}%
  \Test{123}{error}%
\end{qstest}
\end{document}
%</test5>
%    \end{macrocode}
%
% \subsection{Test for rotate}
%
%    \begin{macrocode}
%<*test6>
\NeedsTeXFormat{LaTeX2e}
\documentclass{article}
\usepackage{pdflscape}[2016/05/14]
\begin{document}
\section{Portrait page}
\begin{landscape}
\section{Landscape page}
Foo bar foo bar foo bar foo bar foo bar
foo bar foo bar foo bar foo bar foo bar.
\end{landscape}
\section{Portrait Section}
Foo bar
\begin{landscape}
\section{Again Landscape Section}
\newpage
\section{Second Page of Landscape Section}
\end{landscape}
\section{Second Last Portrait Page}
\newpage
\section{Last Portrait Page}
\end{document}
%</test6>
%    \end{macrocode}
%
%    \begin{macrocode}
%<*result6>
/Page 1 /Rotate 0
/Page 2 /Rotate 90
/Page 3 /Rotate 0
/Page 4 /Rotate 90
/Page 5 /Rotate 90
/Page 6 /Rotate 0
/Page 7 /Rotate 0
%</result6>
%    \end{macrocode}
%
%    \begin{macrocode}
%<*ExtractRotate>
/**
 * ExtractRotate.java
 *
 * Copyright (C) 2007 by Heiko Oberdiek <heiko.oberdiek at googlemail.com>
 *
 * Requires: PDFBox (http://www.pdfbox.org/)
 *
 * Syntax: java ExtractRotate <pdffile> <textfile>
 *
 * The <pdffile> is analyzed and for each page its rotation
 * setting is printed in the <textfile>. Example:
 *   /Page 1 /Rotate 0
 *   /Page 2 /Rotate 90
 */
import java.io.FileInputStream;
import java.io.FileWriter;
import org.pdfbox.pdfparser.PDFParser;
import org.pdfbox.pdmodel.PDDocument;
import org.pdfbox.pdmodel.PDDocumentCatalog;
import org.pdfbox.pdmodel.PDPage;

public class ExtractRotate {

    public static void main(String[] args) {
        try {
            String infile = args[0];
            String outfile = args[1];
            FileWriter out = new FileWriter(outfile);
            PDFParser parser =
                    new PDFParser(new FileInputStream(infile));
            parser.parse();
            PDDocument document = parser.getPDDocument();
            PDDocumentCatalog catalog = document.getDocumentCatalog();
            int i = 0;
            for (Object page: catalog.getAllPages()) {
                i++;
                out.write("/Page " + i + " " + "/Rotate "
                        + ((PDPage)page).findRotation() + "\n");
            }
            document.close();
            out.close();
        }
        catch (Exception e) {
            e.printStackTrace();
            System.exit(1);
        }
    }
}
%</ExtractRotate>
%    \end{macrocode}
%
% \section{Installation}
%
% \subsection{Download}
%
% \paragraph{Package.} This package is available on
% CTAN\footnote{\CTANpkg{pdflscape}}:
% \begin{description}
% \item[\CTAN{macros/latex/contrib/oberdiek/pdflscape.dtx}] The source file.
% \item[\CTAN{macros/latex/contrib/oberdiek/pdflscape.pdf}] Documentation.
% \end{description}
%
%
% \paragraph{Bundle.} All the packages of the bundle `oberdiek'
% are also available in a TDS compliant ZIP archive. There
% the packages are already unpacked and the documentation files
% are generated. The files and directories obey the TDS standard.
% \begin{description}
% \item[\CTANinstall{install/macros/latex/contrib/oberdiek.tds.zip}]
% \end{description}
% \emph{TDS} refers to the standard ``A Directory Structure
% for \TeX\ Files'' (\CTAN{tds/tds.pdf}). Directories
% with \xfile{texmf} in their name are usually organized this way.
%
% \subsection{Bundle installation}
%
% \paragraph{Unpacking.} Unpack the \xfile{oberdiek.tds.zip} in the
% TDS tree (also known as \xfile{texmf} tree) of your choice.
% Example (linux):
% \begin{quote}
%   |unzip oberdiek.tds.zip -d ~/texmf|
% \end{quote}
%
% \paragraph{Script installation.}
% Check the directory \xfile{TDS:scripts/oberdiek/} for
% scripts that need further installation steps.
% Package \xpackage{attachfile2} comes with the Perl script
% \xfile{pdfatfi.pl} that should be installed in such a way
% that it can be called as \texttt{pdfatfi}.
% Example (linux):
% \begin{quote}
%   |chmod +x scripts/oberdiek/pdfatfi.pl|\\
%   |cp scripts/oberdiek/pdfatfi.pl /usr/local/bin/|
% \end{quote}
%
% \subsection{Package installation}
%
% \paragraph{Unpacking.} The \xfile{.dtx} file is a self-extracting
% \docstrip\ archive. The files are extracted by running the
% \xfile{.dtx} through \plainTeX:
% \begin{quote}
%   \verb|tex pdflscape.dtx|
% \end{quote}
%
% \paragraph{TDS.} Now the different files must be moved into
% the different directories in your installation TDS tree
% (also known as \xfile{texmf} tree):
% \begin{quote}
% \def\t{^^A
% \begin{tabular}{@{}>{\ttfamily}l@{ $\rightarrow$ }>{\ttfamily}l@{}}
%   pdflscape.sty & tex/latex/oberdiek/pdflscape.sty\\
%   pdflscape.pdf & doc/latex/oberdiek/pdflscape.pdf\\
%   test/pdflscape-test1.tex & doc/latex/oberdiek/test/pdflscape-test1.tex\\
%   test/pdflscape-test2.tex & doc/latex/oberdiek/test/pdflscape-test2.tex\\
%   test/pdflscape-test3.tex & doc/latex/oberdiek/test/pdflscape-test3.tex\\
%   test/pdflscape-test4.tex & doc/latex/oberdiek/test/pdflscape-test4.tex\\
%   test/pdflscape-test5.tex & doc/latex/oberdiek/test/pdflscape-test5.tex\\
%   test/pdflscape-test6.tex & doc/latex/oberdiek/test/pdflscape-test6.tex\\
%   test/pdflscape-test6.txt & doc/latex/oberdiek/test/pdflscape-test6.txt\\
%   test/ExtractRotate.java & doc/latex/oberdiek/test/ExtractRotate.java\\
%   pdflscape.dtx & source/latex/oberdiek/pdflscape.dtx\\
% \end{tabular}^^A
% }^^A
% \sbox0{\t}^^A
% \ifdim\wd0>\linewidth
%   \begingroup
%     \advance\linewidth by\leftmargin
%     \advance\linewidth by\rightmargin
%   \edef\x{\endgroup
%     \def\noexpand\lw{\the\linewidth}^^A
%   }\x
%   \def\lwbox{^^A
%     \leavevmode
%     \hbox to \linewidth{^^A
%       \kern-\leftmargin\relax
%       \hss
%       \usebox0
%       \hss
%       \kern-\rightmargin\relax
%     }^^A
%   }^^A
%   \ifdim\wd0>\lw
%     \sbox0{\small\t}^^A
%     \ifdim\wd0>\linewidth
%       \ifdim\wd0>\lw
%         \sbox0{\footnotesize\t}^^A
%         \ifdim\wd0>\linewidth
%           \ifdim\wd0>\lw
%             \sbox0{\scriptsize\t}^^A
%             \ifdim\wd0>\linewidth
%               \ifdim\wd0>\lw
%                 \sbox0{\tiny\t}^^A
%                 \ifdim\wd0>\linewidth
%                   \lwbox
%                 \else
%                   \usebox0
%                 \fi
%               \else
%                 \lwbox
%               \fi
%             \else
%               \usebox0
%             \fi
%           \else
%             \lwbox
%           \fi
%         \else
%           \usebox0
%         \fi
%       \else
%         \lwbox
%       \fi
%     \else
%       \usebox0
%     \fi
%   \else
%     \lwbox
%   \fi
% \else
%   \usebox0
% \fi
% \end{quote}
% If you have a \xfile{docstrip.cfg} that configures and enables \docstrip's
% TDS installing feature, then some files can already be in the right
% place, see the documentation of \docstrip.
%
% \subsection{Refresh file name databases}
%
% If your \TeX~distribution
% (\teTeX, \mikTeX, \dots) relies on file name databases, you must refresh
% these. For example, \teTeX\ users run \verb|texhash| or
% \verb|mktexlsr|.
%
% \subsection{Some details for the interested}
%
% \paragraph{Attached source.}
%
% The PDF documentation on CTAN also includes the
% \xfile{.dtx} source file. It can be extracted by
% AcrobatReader 6 or higher. Another option is \textsf{pdftk},
% e.g. unpack the file into the current directory:
% \begin{quote}
%   \verb|pdftk pdflscape.pdf unpack_files output .|
% \end{quote}
%
% \paragraph{Unpacking with \LaTeX.}
% The \xfile{.dtx} chooses its action depending on the format:
% \begin{description}
% \item[\plainTeX:] Run \docstrip\ and extract the files.
% \item[\LaTeX:] Generate the documentation.
% \end{description}
% If you insist on using \LaTeX\ for \docstrip\ (really,
% \docstrip\ does not need \LaTeX), then inform the autodetect routine
% about your intention:
% \begin{quote}
%   \verb|latex \let\install=y% \iffalse meta-comment
%
% File: pdflscape.dtx
% Version: 2016/05/14 v0.11
% Info: Display of landscape pages in PDF
%
% Copyright (C) 2001, 2004, 2006-2008 by
%    Heiko Oberdiek <heiko.oberdiek at googlemail.com>
%    2016
%    https://github.com/ho-tex/oberdiek/issues
%
% This work may be distributed and/or modified under the
% conditions of the LaTeX Project Public License, either
% version 1.3c of this license or (at your option) any later
% version. This version of this license is in
%    http://www.latex-project.org/lppl/lppl-1-3c.txt
% and the latest version of this license is in
%    http://www.latex-project.org/lppl.txt
% and version 1.3 or later is part of all distributions of
% LaTeX version 2005/12/01 or later.
%
% This work has the LPPL maintenance status "maintained".
%
% This Current Maintainer of this work is Heiko Oberdiek.
%
% This work consists of the main source file pdflscape.dtx
% and the derived files
%    pdflscape.sty, pdflscape.pdf, pdflscape.ins, pdflscape.drv,
%    pdflscape-test1.tex, pdflscape-test2.tex,
%    pdflscape-test3.tex, pdflscape-test4.tex,
%    pdflscape-test5.tex, pdflscape-test6.tex,
%    pdflscape-test6.txt, ExtractRotate.java.
%
% Distribution:
%    CTAN:macros/latex/contrib/oberdiek/pdflscape.dtx
%    CTAN:macros/latex/contrib/oberdiek/pdflscape.pdf
%
% Unpacking:
%    (a) If pdflscape.ins is present:
%           tex pdflscape.ins
%    (b) Without pdflscape.ins:
%           tex pdflscape.dtx
%    (c) If you insist on using LaTeX
%           latex \let\install=y% \iffalse meta-comment
%
% File: pdflscape.dtx
% Version: 2016/05/14 v0.11
% Info: Display of landscape pages in PDF
%
% Copyright (C) 2001, 2004, 2006-2008 by
%    Heiko Oberdiek <heiko.oberdiek at googlemail.com>
%    2016
%    https://github.com/ho-tex/oberdiek/issues
%
% This work may be distributed and/or modified under the
% conditions of the LaTeX Project Public License, either
% version 1.3c of this license or (at your option) any later
% version. This version of this license is in
%    http://www.latex-project.org/lppl/lppl-1-3c.txt
% and the latest version of this license is in
%    http://www.latex-project.org/lppl.txt
% and version 1.3 or later is part of all distributions of
% LaTeX version 2005/12/01 or later.
%
% This work has the LPPL maintenance status "maintained".
%
% This Current Maintainer of this work is Heiko Oberdiek.
%
% This work consists of the main source file pdflscape.dtx
% and the derived files
%    pdflscape.sty, pdflscape.pdf, pdflscape.ins, pdflscape.drv,
%    pdflscape-test1.tex, pdflscape-test2.tex,
%    pdflscape-test3.tex, pdflscape-test4.tex,
%    pdflscape-test5.tex, pdflscape-test6.tex,
%    pdflscape-test6.txt, ExtractRotate.java.
%
% Distribution:
%    CTAN:macros/latex/contrib/oberdiek/pdflscape.dtx
%    CTAN:macros/latex/contrib/oberdiek/pdflscape.pdf
%
% Unpacking:
%    (a) If pdflscape.ins is present:
%           tex pdflscape.ins
%    (b) Without pdflscape.ins:
%           tex pdflscape.dtx
%    (c) If you insist on using LaTeX
%           latex \let\install=y\input{pdflscape.dtx}
%        (quote the arguments according to the demands of your shell)
%
% Documentation:
%    (a) If pdflscape.drv is present:
%           latex pdflscape.drv
%    (b) Without pdflscape.drv:
%           latex pdflscape.dtx; ...
%    The class ltxdoc loads the configuration file ltxdoc.cfg
%    if available. Here you can specify further options, e.g.
%    use A4 as paper format:
%       \PassOptionsToClass{a4paper}{article}
%
%    Programm calls to get the documentation (example):
%       pdflatex pdflscape.dtx
%       makeindex -s gind.ist pdflscape.idx
%       pdflatex pdflscape.dtx
%       makeindex -s gind.ist pdflscape.idx
%       pdflatex pdflscape.dtx
%
% Installation:
%    TDS:tex/latex/oberdiek/pdflscape.sty
%    TDS:doc/latex/oberdiek/pdflscape.pdf
%    TDS:doc/latex/oberdiek/test/pdflscape-test1.tex
%    TDS:doc/latex/oberdiek/test/pdflscape-test2.tex
%    TDS:doc/latex/oberdiek/test/pdflscape-test3.tex
%    TDS:doc/latex/oberdiek/test/pdflscape-test4.tex
%    TDS:doc/latex/oberdiek/test/pdflscape-test5.tex
%    TDS:doc/latex/oberdiek/test/pdflscape-test6.tex
%    TDS:doc/latex/oberdiek/test/pdflscape-test6.txt
%    TDS:doc/latex/oberdiek/test/ExtractRotate.java
%    TDS:source/latex/oberdiek/pdflscape.dtx
%
%<*ignore>
\begingroup
  \catcode123=1 %
  \catcode125=2 %
  \def\x{LaTeX2e}%
\expandafter\endgroup
\ifcase 0\ifx\install y1\fi\expandafter
         \ifx\csname processbatchFile\endcsname\relax\else1\fi
         \ifx\fmtname\x\else 1\fi\relax
\else\csname fi\endcsname
%</ignore>
%<*install>
\input docstrip.tex
\Msg{************************************************************************}
\Msg{* Installation}
\Msg{* Package: pdflscape 2016/05/14 v0.11 Display of landscape pages in PDF (HO)}
\Msg{************************************************************************}

\keepsilent
\askforoverwritefalse

\let\MetaPrefix\relax
\preamble

This is a generated file.

Project: pdflscape
Version: 2016/05/14 v0.11

Copyright (C) 2001, 2004, 2006-2008 by
   Heiko Oberdiek <heiko.oberdiek at googlemail.com>

This work may be distributed and/or modified under the
conditions of the LaTeX Project Public License, either
version 1.3c of this license or (at your option) any later
version. This version of this license is in
   http://www.latex-project.org/lppl/lppl-1-3c.txt
and the latest version of this license is in
   http://www.latex-project.org/lppl.txt
and version 1.3 or later is part of all distributions of
LaTeX version 2005/12/01 or later.

This work has the LPPL maintenance status "maintained".

This Current Maintainer of this work is Heiko Oberdiek.

This work consists of the main source file pdflscape.dtx
and the derived files
   pdflscape.sty, pdflscape.pdf, pdflscape.ins, pdflscape.drv,
   pdflscape-test1.tex, pdflscape-test2.tex,
   pdflscape-test3.tex, pdflscape-test4.tex,
   pdflscape-test5.tex, pdflscape-test6.tex,
   pdflscape-test6.txt, ExtractRotate.java.

\endpreamble
\let\MetaPrefix\DoubleperCent

\generate{%
  \file{pdflscape.ins}{\from{pdflscape.dtx}{install}}%
  \file{pdflscape.drv}{\from{pdflscape.dtx}{driver}}%
  \usedir{tex/latex/oberdiek}%
  \file{pdflscape.sty}{\from{pdflscape.dtx}{package}}%
%  \usedir{doc/latex/oberdiek/test}%
%  \file{pdflscape-test1.tex}{\from{pdflscape.dtx}{test0,test1}}%
%  \file{pdflscape-test2.tex}{\from{pdflscape.dtx}{test0,test2}}%
%  \file{pdflscape-test3.tex}{\from{pdflscape.dtx}{test0,test3}}%
%  \file{pdflscape-test4.tex}{\from{pdflscape.dtx}{test0,test4}}%
%  \file{pdflscape-test5.tex}{\from{pdflscape.dtx}{test5}}%
%  \file{pdflscape-test6.tex}{\from{pdflscape.dtx}{test6}}%
  \nopreamble
  \nopostamble
%  \file{pdflscape-test6.txt}{\from{pdflscape.dtx}{result6}}%
  \file{ExtractRotate.java}{\from{pdflscape.dtx}{ExtractRotate}}%
%  \usedir{source/latex/oberdiek/catalogue}%
%  \file{pdflscape.xml}{\from{pdflscape.dtx}{catalogue}}%
}

\catcode32=13\relax% active space
\let =\space%
\Msg{************************************************************************}
\Msg{*}
\Msg{* To finish the installation you have to move the following}
\Msg{* file into a directory searched by TeX:}
\Msg{*}
\Msg{*     pdflscape.sty}
\Msg{*}
\Msg{* To produce the documentation run the file `pdflscape.drv'}
\Msg{* through LaTeX.}
\Msg{*}
\Msg{* Happy TeXing!}
\Msg{*}
\Msg{************************************************************************}

\endbatchfile
%</install>
%<*ignore>
\fi
%</ignore>
%<*driver>
\NeedsTeXFormat{LaTeX2e}
\ProvidesFile{pdflscape.drv}%
  [2016/05/14 v0.11 Display of landscape pages in PDF (HO)]%
\documentclass{ltxdoc}
\usepackage{holtxdoc}[2011/11/22]
\usepackage{graphics}
\begin{document}
  \DocInput{pdflscape.dtx}%
\end{document}
%</driver>
% \fi
%
%
% \CharacterTable
%  {Upper-case    \A\B\C\D\E\F\G\H\I\J\K\L\M\N\O\P\Q\R\S\T\U\V\W\X\Y\Z
%   Lower-case    \a\b\c\d\e\f\g\h\i\j\k\l\m\n\o\p\q\r\s\t\u\v\w\x\y\z
%   Digits        \0\1\2\3\4\5\6\7\8\9
%   Exclamation   \!     Double quote  \"     Hash (number) \#
%   Dollar        \$     Percent       \%     Ampersand     \&
%   Acute accent  \'     Left paren    \(     Right paren   \)
%   Asterisk      \*     Plus          \+     Comma         \,
%   Minus         \-     Point         \.     Solidus       \/
%   Colon         \:     Semicolon     \;     Less than     \<
%   Equals        \=     Greater than  \>     Question mark \?
%   Commercial at \@     Left bracket  \[     Backslash     \\
%   Right bracket \]     Circumflex    \^     Underscore    \_
%   Grave accent  \`     Left brace    \{     Vertical bar  \|
%   Right brace   \}     Tilde         \~}
%
% \GetFileInfo{pdflscape.drv}
%
% \title{The \xpackage{pdflscape} package}
% \date{2016/05/14 v0.11}
% \author{Heiko Oberdiek\thanks
% {Please report any issues at \url{https://github.com/ho-tex/oberdiek/issues}}\\
% \xemail{heiko.oberdiek at googlemail.com}}
%
% \maketitle
%
% \begin{abstract}
% Package \xpackage{pdflscape} adds PDF support to the
% environment \texttt{landscape} of package \xpackage{lscape}
% by setting the PDF page attribute \texttt{/Rotate}.
% \end{abstract}
%
% \tableofcontents
%
% \section{Documentation}
%
% \subsection{Supported drivers}
%
% \begin{itemize}
% \item \xoption{pdftex},\xoption{luatex}
% \item \xoption{dvips}, \xoption{dvipsone},
%       \xoption{pctex32}, \xoption{pctexps}
%       (and other drivers that provide a non-empty \cs{Gin@PS@raw})
% \item \xoption{dvipdfm}
% \end{itemize}
%
% \subsection{Caveat}
%
% Depending on the configuration Ghostscript
% adds a guessed rotation entry by its own.
% This can lead to two |/Rotate| entries per page.
% To prevent this behaviour set the parameter
% |AutoRotatePages| to |/None|, eg:
% \begin{quote}
%   |ps2pdf -dAutoRotatePages=/None|
% \end{quote}
%
% \subsection{Requirements}
%
% \begin{itemize}
% \item The package \xpackage{lscape}.
% \item The package \xpackage{atbegshi} for all drivers
%        except \xoption{pdftex},\xoption{luatex}.
% \end{itemize}
%
% \subsection{Usage}
%
% Load this package instead of or after package \xpackage{lscape}:
% \begin{quote}
%   |\usepackage{pdflscape}|\\
%   |\begin{landscape}|\dots|\end{landscape}|
% \end{quote}
%
% \StopEventually{
% }
%
% \section{Implementation}
%
% \subsection{Package identification}
%
%    \begin{macrocode}
%<*package>
%    \end{macrocode}
%    Package identification.
%    \begin{macrocode}
\NeedsTeXFormat{LaTeX2e}
\ProvidesPackage{pdflscape}%
  [2016/05/14 v0.11 Display of landscape pages in PDF (HO)]
%    \end{macrocode}
%
% \subsection{Driver options}
%
%    \begin{macrocode}
\let\PLS@option\@empty
\let\PLS@driver\@empty
\def\PLS@temp#1#2{%
  \DeclareOption{#1}{%
    \let\PLS@option\CurrentOption
    \def\PLS@driver{#2}%
  }%
}
\PLS@temp{pdftex}{pdftex}
\PLS@temp{luatex}{pdftex}
\PLS@temp{dvips}{dvips}
\PLS@temp{dvipsone}{dvips}
\PLS@temp{pctexps}{dvips}
\PLS@temp{dvipdfm}{dvipdfm}
\PLS@temp{dvipdfmx}{dvipdfm}
\PLS@temp{xetex}{dvipdfm}
\let\PLS@temp\@empty
\DeclareOption*{%
  \ifx\PLS@temp\@empty
    \let\PLS@temp\CurrentOption
  \else
    \edef\PLS@temp{\PLS@temp,\CurrentOption}%
  \fi
}
\ProcessOptions*
\ifx\PLS@option\@empty
\else
  \expandafter\PassOptionsToPackage\expandafter{\PLS@option}{lscape}%
\fi
\ifx\PLS@temp\@empty
\else
  \expandafter\PassOptionsToPackage\expandafter{\PLS@temp}{lscape}%
\fi
\RequirePackage{lscape}
%    \end{macrocode}
%
% \subsection{Autodetection of driver}
%
% \subsubsection{Driver \xoption{pdftex}}
%
%    \begin{macrocode}
\RequirePackage{ifpdf}[2006/02/20]
\def\PLS@temp{pdftex}
\ifpdf
  \ifx\PLS@driver\@empty
    \let\PLS@driver\PLS@temp
    \PackageInfo{pdflscape}{Auto-detected driver: \PLS@temp}%
  \fi
  \ifx\PLS@driver\PLS@temp
  \else
    \PackageError{pdflscape}{%
      The driver option `\PLS@option' is not compatible\MessageBreak
      with pdfTeX in PDF mode, using driver `pdftex' instead%
    }\@ehc
    \let\PLS@driver\PLS@temp
  \fi
\fi
%    \end{macrocode}
%
% \subsubsection{Driver \xoption{xetex}}
%
%    \begin{macrocode}
\RequirePackage{ifxetex}%
\def\PLS@temp{dvipdfm}
\ifxetex
  \ifx\PLS@driver\@empty
    \let\PLS@driver\PLS@temp
    \PackageInfo{pdflscape}{%
      Auto-detected driver: \PLS@temp\space(xetex)%
    }%
  \fi
  \ifx\PLS@driver\PLS@temp
  \else
    \PackageError{pdflscape}{%
      The driver option `\PLS@option' is not compatible\MessageBreak
      with XeTeX, using driver for XeTeX (dvipdfm) instead%
    }\@ehc
    \let\PLS@driver\PLS@temp
  \fi
\fi
%    \end{macrocode}
%
% \subsubsection{Detect driver based on \cs{Gin@driver}}
%
%    \begin{macrocode}
\def\PLS@temp#1#2{%
  \ifx\PLS@driver\@empty
    \begingroup
      \def\PLS@temp{#1.def}%
    \expandafter\endgroup
    \ifx\PLS@temp\Gin@driver
      \def\PLS@driver{#2}%
      \PackageInfo{pdflscape}{%
        Auto-detected driver: \PLS@driver\space(#1)%
      }%
    \fi
  \fi
}
\PLS@temp{dvipdfm}{dvipdfm}
\PLS@temp{dvipdfmx}{dvipdfm}
\PLS@temp{dvips}{dvips}
\PLS@temp{dvipsone}{dvips}
\PLS@temp{pctexps}{dvips}
%    \end{macrocode}
%
% \subsubsection{Driver \xoption{dvips}}
%
%    \begin{macrocode}
\ifx\PLS@driver\@empty
  \def\PLS@temp{dvips}%
  \ifnum 1=0\ifx\Gin@PS@raw\@undefined
            \else
              \ifx\Gin@PS@raw\relax
              \else
                \ifx\Gin@PS@raw\@empty
                \else
                  1%
                \fi
              \fi
            \fi
            \space
    \let\PLS@driver\PLS@temp
    \PackageInfo{pdflscape}{%
      Auto-detected driver: \PLS@temp\space(PostScript)%
    }%
  \fi
\fi
%    \end{macrocode}
%
%    \begin{macrocode}
\ifx\PLS@driver\@empty
  \PackageError{pdflscape}{%
    Neither pdfTeX, PostScript, nor dvipdfm driver found%
  }\@ehc
\fi
%    \end{macrocode}
%
% \subsection{Driver implementation}
%
% \begin{itemize}
% \item |\PLS@AddRotate #1|\\
%   it expects the correct rotation number in |#1|
%   and implements the adding of the /Rotation entry
%   in the |/Page| object of the current page.
% \item |\PLS@RemoveRotate|\\
%   it removes a previous |/Rotate| entry, if necessary.
% \end{itemize}
%
% \subsubsection{\pdfTeX}
%
% Not too nice is the global setting of \cs{pdfpageattr}.
% Perhaps this can be changed in future versions.
%
% luatex compat.
%    \begin{macrocode}
\ifx\pdfpageattr\@undefined
  \ifx\pdfvariable\@undefined\else 
    \protected\edef\pdfpageattr{\pdfvariable pageattr}
  \fi
\fi
%    \end{macrocode}
%    \begin{macrocode}
\def\PLS@temp{pdftex}
\ifx\PLS@temp\PLS@driver
  \def\PLS@AddRotate#1{%
    \ifnum#1=0 %
%    \end{macrocode}
%    Already the default.
%    \begin{macrocode}
    \else
      \global\pdfpageattr\expandafter{%
        \the\pdfpageattr
        /Rotate #1%
      }%
    \fi
  }%
%    \end{macrocode}
%    Removes a |/Rotate| entry.
%    \begin{macrocode}
  \def\PLS@RemoveRotate{%
    \begingroup
      \global\pdfpageattr\expandafter{\expandafter}%
      \expandafter\PLS@@RemoveRotate
        \the\pdfpageattr /Rotate\@nil
    \endgroup
  }%
  \def\PLS@@RemoveRotate#1/Rotate#2\@nil{%
%    \end{macrocode}
%    Append |/Rotate| free stuff to \cs{pdfpageattr}.
%    \begin{macrocode}
    \global\pdfpageattr\expandafter{\the\pdfpageattr#1}%
    \ifx\\#2\\%
%    \end{macrocode}
%    Ready, because the detected |/Rotate| is part of
%    the end marker: |/Rotate\@nil|
%    \begin{macrocode}
    \else
%    \end{macrocode}
%    First read in the argument of |/Rotate|,
%    then continue parsing.
%    \begin{macrocode}
      \afterassignment\PLS@@RemoveRotate
      \count0=#2\@nil
    \fi
  }%
\fi
%    \end{macrocode}
%
% \subsubsection{PostScript driver}
%
%    \begin{macrocode}
\def\PLS@temp{dvips}
\ifx\PLS@temp\PLS@driver
  \RequirePackage{atbegshi}%
  \AtBeginShipout{\PLS@AtBeginShipout}%
  \let\PLS@AtBeginShipout\@empty
  \def\PLS@AddRotate#1{%
    \ifnum#1=0 %
    \else
      \def\PLS@AtBeginShipout{%
        \global\setbox\AtBeginShipoutBox\vbox{%
          \Gin@PS@raw{%
            [{ThisPage}\string<\string</Rotate #1\string>\string>%
            /PUT pdfmark%
          }%
          \box\AtBeginShipoutBox
        }%
      }%
    \fi
  }%
  \def\PLS@RemoveRotate{\let\PLS@AtBeginShipout\@empty}%
\fi
%    \end{macrocode}
%
% \subsubsection{Driver dvipdfm}
%
%    \begin{macrocode}
\def\PLS@temp{dvipdfm}
\ifx\PLS@temp\PLS@driver
  \RequirePackage{atbegshi}%
  \AtBeginShipout{\PLS@AtBeginShipout}%
  \let\PLS@AtBeginShipout\@empty
  \def\PLS@AddRotate#1{%
    \ifnum#1=0 %
    \else
      \def\PLS@AtBeginShipout{%
        \global\setbox\AtBeginShipoutBox\vbox{%
          \special{pdf: put @thispage <</Rotate #1>>}%
          \box\AtBeginShipoutBox
        }%
      }%
    \fi
  }%
  \def\PLS@RemoveRotate{\let\PLS@AtBeginShipout\@empty}%
\fi
%    \end{macrocode}
%
% \subsection{Driver independent stuff}
%
%    The landscape environment is extended by
%    adding the correct |/Rotate| entries.
%    \begin{macrocode}
\g@addto@macro{\landscape}{\PLS@Rotate{90}}
\g@addto@macro{\endlandscape}{\PLS@Rotate{0}}
%    \end{macrocode}
%
%    \begin{macro}{\PLS@Rotate}
%    Main macro, that sets the |/Rotate| entry.\\
%    Argument: any TeX number or nothing, that means zero.\\
%    Driver independent.
%    \begin{macrocode}
\def\PLS@Rotate#1{%
  \begingroup
%    \end{macrocode}
%    1. Check and validate the argument.
%    \begin{macrocode}
    \PLS@CheckAngle{#1}%
  \expandafter\endgroup
%    \end{macrocode}
%    2. Remove previous |\Rotate| entry.
%    \begin{macrocode}
  \expandafter\PLS@RemoveRotate
%    \end{macrocode}
%    3. Add |/Rotate| entry.
%    \begin{macrocode}
  \expandafter\PLS@AddRotate\expandafter{\the\count@}%
}
%    \end{macrocode}
%    \end{macro}
%
%    \begin{macro}{\PLS@CheckAngle}
%    Validates the rotation angle.\\
%    The result is stored in the count register |\count@|.\\
%    Driver independent.
%    \begin{macrocode}
\def\PLS@CheckAngle#1{%
%    \end{macrocode}
%    Nothing means zero:
%    \begin{macrocode}
  \ifx\\#1\\%
    \count@=0 %
  \else
    \count@=#1\relax
  \fi
%    \end{macrocode}
%    Normalize to interval -360 $<$ |\count@| $<$ 360.
%    \begin{macrocode}
  \@whilenum\count@>359\do{\advance\count@ -360 }%
  \@whilenum\count@<-359\do{\advance\count@ 360 }%
%    \end{macrocode}
%    Check values: 0, +/-90, +/-180, +/-270
%    \begin{macrocode}
  \ifnum 1=0\ifnum\count@=0 \else
            \ifnum\count@=90 \else
            \ifnum\count@=-90 \else
            \ifnum\count@=180 \else
            \ifnum\count@=-180 \else
            \ifnum\count@=270 \else
            \ifnum\count@=-270 \else
           1\fi\fi\fi\fi\fi\fi\fi\relax
    \PackageError{pdflscape}{%
      Invalid value \the\count@\space for /Rotate%
    }\@ehc
    \count@=0 %
  \fi
}
%    \end{macrocode}
%    \end{macro}
%
%    \begin{macrocode}
%</package>
%    \end{macrocode}
%
% \section{Test}
%
% \subsection{Driver detection tests}
%    \begin{macrocode}
%<*test0>
\NeedsTeXFormat{LaTeX2e}
\documentclass{minimal}
\usepackage{qstest}
\IncludeTests{*}
\LogTests{log}{*}{*}
%</test0>
%    \end{macrocode}
%
%    \begin{macrocode}
%<*test1>
\usepackage{ifpdf}
\ifpdf
  \def\ExpectDriver{pdftex}%
\fi
\usepackage{ifxetex}%
\ifxetex
  \def\ExpectDriver{dvipdfm}%
\fi
\usepackage{pdflscape}[2016/05/14]
%</test1>
%    \end{macrocode}
%
%    \begin{macrocode}
%<*test2>
\def\ExpectDriver{pdftex}
\usepackage[pdftex]{pdflscape}[2016/05/14]
%</test2>
%    \end{macrocode}
%
%    \begin{macrocode}
%<*test3>
\def\ExpectDriver{dvipdfm}
\usepackage[dvipdfmx]{pdflscape}[2016/05/14]
%</test3>
%    \end{macrocode}
%
%    \begin{macrocode}
%<*test4>
\def\ExpectDriver{dvips}
\usepackage[dvipsone]{graphics}
\usepackage{pdflscape}[2016/05/14]
%</test4>
%    \end{macrocode}
%
%    \begin{macrocode}
%<*test0>
\begin{document}
\begin{qstest}{driver}{driver}
  \makeatletter
  \Expect*{\PLS@driver}*{\ExpectDriver}%
\end{qstest}
\end{document}
%</test0>
%    \end{macrocode}
%
% \subsection{Test for \cs{PLS@CheckAngle}}
%
%    \begin{macrocode}
%<*test5>
\NeedsTeXFormat{LaTeX2e}
\documentclass{minimal}
\usepackage{qstest}
\IncludeTests{*}
\LogTests{log}{*}{*}
\usepackage{pdflscape}[2016/05/14]
\begin{document}
\begin{qstest}{checkangle}{checkangle}
  \makeatletter
  \def\Result{%
    \def\ResultString{\the\count@}%
  }%
  \def\PackageError#1#2#3{%
    \def\Result{%
      \def\ResultString{error}%
    }%
  }%
  \def\Test#1#2{%
    \begingroup
      \PLS@CheckAngle{#1}%
      \Result
      \Expect*{\ResultString}{#2}%
    \endgroup
  }%
  \Test{0}{0}%
  \Test{90}{90}%
  \Test{180}{180}%
  \Test{270}{270}%
  \Test{360}{0}%
  \Test{450}{90}%
  \Test{540}{180}%
  \Test{630}{270}%
  \Test{720}{0}%
  \Test{3600}{0}%
  \Test{3690}{90}%
  \Test{-90}{270}%
  \Test{-180}{180}%
  \Test{-270}{-90}%
  \Test{-360}{0}%
  \Test{-450}{270}%
  \Test{540}{180}%
  \Test{-630}{90}%
  \Test{-720}{0}%
  \Test{-3600}{0}%
  \Test{-3690}{270}%
  \Test{1}{error}%
  \Test{-1}{error}%
  \Test{123}{error}%
\end{qstest}
\end{document}
%</test5>
%    \end{macrocode}
%
% \subsection{Test for rotate}
%
%    \begin{macrocode}
%<*test6>
\NeedsTeXFormat{LaTeX2e}
\documentclass{article}
\usepackage{pdflscape}[2016/05/14]
\begin{document}
\section{Portrait page}
\begin{landscape}
\section{Landscape page}
Foo bar foo bar foo bar foo bar foo bar
foo bar foo bar foo bar foo bar foo bar.
\end{landscape}
\section{Portrait Section}
Foo bar
\begin{landscape}
\section{Again Landscape Section}
\newpage
\section{Second Page of Landscape Section}
\end{landscape}
\section{Second Last Portrait Page}
\newpage
\section{Last Portrait Page}
\end{document}
%</test6>
%    \end{macrocode}
%
%    \begin{macrocode}
%<*result6>
/Page 1 /Rotate 0
/Page 2 /Rotate 90
/Page 3 /Rotate 0
/Page 4 /Rotate 90
/Page 5 /Rotate 90
/Page 6 /Rotate 0
/Page 7 /Rotate 0
%</result6>
%    \end{macrocode}
%
%    \begin{macrocode}
%<*ExtractRotate>
/**
 * ExtractRotate.java
 *
 * Copyright (C) 2007 by Heiko Oberdiek <heiko.oberdiek at googlemail.com>
 *
 * Requires: PDFBox (http://www.pdfbox.org/)
 *
 * Syntax: java ExtractRotate <pdffile> <textfile>
 *
 * The <pdffile> is analyzed and for each page its rotation
 * setting is printed in the <textfile>. Example:
 *   /Page 1 /Rotate 0
 *   /Page 2 /Rotate 90
 */
import java.io.FileInputStream;
import java.io.FileWriter;
import org.pdfbox.pdfparser.PDFParser;
import org.pdfbox.pdmodel.PDDocument;
import org.pdfbox.pdmodel.PDDocumentCatalog;
import org.pdfbox.pdmodel.PDPage;

public class ExtractRotate {

    public static void main(String[] args) {
        try {
            String infile = args[0];
            String outfile = args[1];
            FileWriter out = new FileWriter(outfile);
            PDFParser parser =
                    new PDFParser(new FileInputStream(infile));
            parser.parse();
            PDDocument document = parser.getPDDocument();
            PDDocumentCatalog catalog = document.getDocumentCatalog();
            int i = 0;
            for (Object page: catalog.getAllPages()) {
                i++;
                out.write("/Page " + i + " " + "/Rotate "
                        + ((PDPage)page).findRotation() + "\n");
            }
            document.close();
            out.close();
        }
        catch (Exception e) {
            e.printStackTrace();
            System.exit(1);
        }
    }
}
%</ExtractRotate>
%    \end{macrocode}
%
% \section{Installation}
%
% \subsection{Download}
%
% \paragraph{Package.} This package is available on
% CTAN\footnote{\CTANpkg{pdflscape}}:
% \begin{description}
% \item[\CTAN{macros/latex/contrib/oberdiek/pdflscape.dtx}] The source file.
% \item[\CTAN{macros/latex/contrib/oberdiek/pdflscape.pdf}] Documentation.
% \end{description}
%
%
% \paragraph{Bundle.} All the packages of the bundle `oberdiek'
% are also available in a TDS compliant ZIP archive. There
% the packages are already unpacked and the documentation files
% are generated. The files and directories obey the TDS standard.
% \begin{description}
% \item[\CTANinstall{install/macros/latex/contrib/oberdiek.tds.zip}]
% \end{description}
% \emph{TDS} refers to the standard ``A Directory Structure
% for \TeX\ Files'' (\CTAN{tds/tds.pdf}). Directories
% with \xfile{texmf} in their name are usually organized this way.
%
% \subsection{Bundle installation}
%
% \paragraph{Unpacking.} Unpack the \xfile{oberdiek.tds.zip} in the
% TDS tree (also known as \xfile{texmf} tree) of your choice.
% Example (linux):
% \begin{quote}
%   |unzip oberdiek.tds.zip -d ~/texmf|
% \end{quote}
%
% \paragraph{Script installation.}
% Check the directory \xfile{TDS:scripts/oberdiek/} for
% scripts that need further installation steps.
% Package \xpackage{attachfile2} comes with the Perl script
% \xfile{pdfatfi.pl} that should be installed in such a way
% that it can be called as \texttt{pdfatfi}.
% Example (linux):
% \begin{quote}
%   |chmod +x scripts/oberdiek/pdfatfi.pl|\\
%   |cp scripts/oberdiek/pdfatfi.pl /usr/local/bin/|
% \end{quote}
%
% \subsection{Package installation}
%
% \paragraph{Unpacking.} The \xfile{.dtx} file is a self-extracting
% \docstrip\ archive. The files are extracted by running the
% \xfile{.dtx} through \plainTeX:
% \begin{quote}
%   \verb|tex pdflscape.dtx|
% \end{quote}
%
% \paragraph{TDS.} Now the different files must be moved into
% the different directories in your installation TDS tree
% (also known as \xfile{texmf} tree):
% \begin{quote}
% \def\t{^^A
% \begin{tabular}{@{}>{\ttfamily}l@{ $\rightarrow$ }>{\ttfamily}l@{}}
%   pdflscape.sty & tex/latex/oberdiek/pdflscape.sty\\
%   pdflscape.pdf & doc/latex/oberdiek/pdflscape.pdf\\
%   test/pdflscape-test1.tex & doc/latex/oberdiek/test/pdflscape-test1.tex\\
%   test/pdflscape-test2.tex & doc/latex/oberdiek/test/pdflscape-test2.tex\\
%   test/pdflscape-test3.tex & doc/latex/oberdiek/test/pdflscape-test3.tex\\
%   test/pdflscape-test4.tex & doc/latex/oberdiek/test/pdflscape-test4.tex\\
%   test/pdflscape-test5.tex & doc/latex/oberdiek/test/pdflscape-test5.tex\\
%   test/pdflscape-test6.tex & doc/latex/oberdiek/test/pdflscape-test6.tex\\
%   test/pdflscape-test6.txt & doc/latex/oberdiek/test/pdflscape-test6.txt\\
%   test/ExtractRotate.java & doc/latex/oberdiek/test/ExtractRotate.java\\
%   pdflscape.dtx & source/latex/oberdiek/pdflscape.dtx\\
% \end{tabular}^^A
% }^^A
% \sbox0{\t}^^A
% \ifdim\wd0>\linewidth
%   \begingroup
%     \advance\linewidth by\leftmargin
%     \advance\linewidth by\rightmargin
%   \edef\x{\endgroup
%     \def\noexpand\lw{\the\linewidth}^^A
%   }\x
%   \def\lwbox{^^A
%     \leavevmode
%     \hbox to \linewidth{^^A
%       \kern-\leftmargin\relax
%       \hss
%       \usebox0
%       \hss
%       \kern-\rightmargin\relax
%     }^^A
%   }^^A
%   \ifdim\wd0>\lw
%     \sbox0{\small\t}^^A
%     \ifdim\wd0>\linewidth
%       \ifdim\wd0>\lw
%         \sbox0{\footnotesize\t}^^A
%         \ifdim\wd0>\linewidth
%           \ifdim\wd0>\lw
%             \sbox0{\scriptsize\t}^^A
%             \ifdim\wd0>\linewidth
%               \ifdim\wd0>\lw
%                 \sbox0{\tiny\t}^^A
%                 \ifdim\wd0>\linewidth
%                   \lwbox
%                 \else
%                   \usebox0
%                 \fi
%               \else
%                 \lwbox
%               \fi
%             \else
%               \usebox0
%             \fi
%           \else
%             \lwbox
%           \fi
%         \else
%           \usebox0
%         \fi
%       \else
%         \lwbox
%       \fi
%     \else
%       \usebox0
%     \fi
%   \else
%     \lwbox
%   \fi
% \else
%   \usebox0
% \fi
% \end{quote}
% If you have a \xfile{docstrip.cfg} that configures and enables \docstrip's
% TDS installing feature, then some files can already be in the right
% place, see the documentation of \docstrip.
%
% \subsection{Refresh file name databases}
%
% If your \TeX~distribution
% (\teTeX, \mikTeX, \dots) relies on file name databases, you must refresh
% these. For example, \teTeX\ users run \verb|texhash| or
% \verb|mktexlsr|.
%
% \subsection{Some details for the interested}
%
% \paragraph{Attached source.}
%
% The PDF documentation on CTAN also includes the
% \xfile{.dtx} source file. It can be extracted by
% AcrobatReader 6 or higher. Another option is \textsf{pdftk},
% e.g. unpack the file into the current directory:
% \begin{quote}
%   \verb|pdftk pdflscape.pdf unpack_files output .|
% \end{quote}
%
% \paragraph{Unpacking with \LaTeX.}
% The \xfile{.dtx} chooses its action depending on the format:
% \begin{description}
% \item[\plainTeX:] Run \docstrip\ and extract the files.
% \item[\LaTeX:] Generate the documentation.
% \end{description}
% If you insist on using \LaTeX\ for \docstrip\ (really,
% \docstrip\ does not need \LaTeX), then inform the autodetect routine
% about your intention:
% \begin{quote}
%   \verb|latex \let\install=y\input{pdflscape.dtx}|
% \end{quote}
% Do not forget to quote the argument according to the demands
% of your shell.
%
% \paragraph{Generating the documentation.}
% You can use both the \xfile{.dtx} or the \xfile{.drv} to generate
% the documentation. The process can be configured by the
% configuration file \xfile{ltxdoc.cfg}. For instance, put this
% line into this file, if you want to have A4 as paper format:
% \begin{quote}
%   \verb|\PassOptionsToClass{a4paper}{article}|
% \end{quote}
% An example follows how to generate the
% documentation with pdf\LaTeX:
% \begin{quote}
%\begin{verbatim}
%pdflatex pdflscape.dtx
%makeindex -s gind.ist pdflscape.idx
%pdflatex pdflscape.dtx
%makeindex -s gind.ist pdflscape.idx
%pdflatex pdflscape.dtx
%\end{verbatim}
% \end{quote}
%
% \begin{History}
%   \begin{Version}{2001/01/15 v0.1}
%   \item
%     First public version,
%     published in \xnewsgroup{de.comp.text.tex}:\\
%     \URL{``\link{Re: Querformat, pdfLaTex und thumbpdf}''}^^A
%     {https://groups.google.com/group/de.comp.text.tex/msg/e054c5795e52d2b8}
%   \end{Version}
%   \begin{Version}{2001/02/04 v0.2}
%   \item
%     Minor documentation update.
%   \item
%     CTAN.
%   \end{Version}
%   \begin{Version}{2004/05/11 v0.3}
%   \item
%     Support for dvipdfm added.
%   \end{Version}
%   \begin{Version}{2004/05/12 v0.4}
%   \item
%     Bug fix: support for multipage landscape environment
%     for all drivers except for \xoption{pdftex}.
%   \end{Version}
%   \begin{Version}{2006/02/20 v0.5}
%   \item
%     DTX framework.
%   \item
%     LPPL 1.3
%   \item
%     Code is not changed.
%   \end{Version}
%   \begin{Version}{2006/04/24 v0.6}
%   \item
%     Bug fix: \cs{PLS@RemoveRotate} reinserted for \cs{PLS@Rotate}.
%   \item
%     Fix for \cs{PLS@RemoveRotate} (dvips/dvipdfm): \cs{PLS@EverhShi}
%     hook is cleared.
%   \end{Version}
%   \begin{Version}{2007/04/11 v0.7}
%   \item
%     Line ends sanitized.
%   \end{Version}
%   \begin{Version}{2007/04/17 v0.8}
%   \item
%     Package \xpackage{atbegshi} replaces \xpackage{everyshi}.
%   \end{Version}
%   \begin{Version}{2007/10/21 v0.9}
%   \item
%     Driver detection for \XeTeX\ added.
%   \item
%     Fix for rotation angles $\le$~-360 or $\ge$~360.
%   \end{Version}
%   \begin{Version}{2008/08/11 v0.10}
%   \item
%     Code is not changed.
%   \item
%     URLs updated.
%   \end{Version}
%   \begin{Version}{2016/05/14 v0.11}
%   \item
%     LuaTeX Compatibility
%   \end{Version}
% \end{History}
%
% \PrintIndex
%
% \Finale
\endinput

%        (quote the arguments according to the demands of your shell)
%
% Documentation:
%    (a) If pdflscape.drv is present:
%           latex pdflscape.drv
%    (b) Without pdflscape.drv:
%           latex pdflscape.dtx; ...
%    The class ltxdoc loads the configuration file ltxdoc.cfg
%    if available. Here you can specify further options, e.g.
%    use A4 as paper format:
%       \PassOptionsToClass{a4paper}{article}
%
%    Programm calls to get the documentation (example):
%       pdflatex pdflscape.dtx
%       makeindex -s gind.ist pdflscape.idx
%       pdflatex pdflscape.dtx
%       makeindex -s gind.ist pdflscape.idx
%       pdflatex pdflscape.dtx
%
% Installation:
%    TDS:tex/latex/oberdiek/pdflscape.sty
%    TDS:doc/latex/oberdiek/pdflscape.pdf
%    TDS:doc/latex/oberdiek/test/pdflscape-test1.tex
%    TDS:doc/latex/oberdiek/test/pdflscape-test2.tex
%    TDS:doc/latex/oberdiek/test/pdflscape-test3.tex
%    TDS:doc/latex/oberdiek/test/pdflscape-test4.tex
%    TDS:doc/latex/oberdiek/test/pdflscape-test5.tex
%    TDS:doc/latex/oberdiek/test/pdflscape-test6.tex
%    TDS:doc/latex/oberdiek/test/pdflscape-test6.txt
%    TDS:doc/latex/oberdiek/test/ExtractRotate.java
%    TDS:source/latex/oberdiek/pdflscape.dtx
%
%<*ignore>
\begingroup
  \catcode123=1 %
  \catcode125=2 %
  \def\x{LaTeX2e}%
\expandafter\endgroup
\ifcase 0\ifx\install y1\fi\expandafter
         \ifx\csname processbatchFile\endcsname\relax\else1\fi
         \ifx\fmtname\x\else 1\fi\relax
\else\csname fi\endcsname
%</ignore>
%<*install>
\input docstrip.tex
\Msg{************************************************************************}
\Msg{* Installation}
\Msg{* Package: pdflscape 2016/05/14 v0.11 Display of landscape pages in PDF (HO)}
\Msg{************************************************************************}

\keepsilent
\askforoverwritefalse

\let\MetaPrefix\relax
\preamble

This is a generated file.

Project: pdflscape
Version: 2016/05/14 v0.11

Copyright (C) 2001, 2004, 2006-2008 by
   Heiko Oberdiek <heiko.oberdiek at googlemail.com>

This work may be distributed and/or modified under the
conditions of the LaTeX Project Public License, either
version 1.3c of this license or (at your option) any later
version. This version of this license is in
   http://www.latex-project.org/lppl/lppl-1-3c.txt
and the latest version of this license is in
   http://www.latex-project.org/lppl.txt
and version 1.3 or later is part of all distributions of
LaTeX version 2005/12/01 or later.

This work has the LPPL maintenance status "maintained".

This Current Maintainer of this work is Heiko Oberdiek.

This work consists of the main source file pdflscape.dtx
and the derived files
   pdflscape.sty, pdflscape.pdf, pdflscape.ins, pdflscape.drv,
   pdflscape-test1.tex, pdflscape-test2.tex,
   pdflscape-test3.tex, pdflscape-test4.tex,
   pdflscape-test5.tex, pdflscape-test6.tex,
   pdflscape-test6.txt, ExtractRotate.java.

\endpreamble
\let\MetaPrefix\DoubleperCent

\generate{%
  \file{pdflscape.ins}{\from{pdflscape.dtx}{install}}%
  \file{pdflscape.drv}{\from{pdflscape.dtx}{driver}}%
  \usedir{tex/latex/oberdiek}%
  \file{pdflscape.sty}{\from{pdflscape.dtx}{package}}%
%  \usedir{doc/latex/oberdiek/test}%
%  \file{pdflscape-test1.tex}{\from{pdflscape.dtx}{test0,test1}}%
%  \file{pdflscape-test2.tex}{\from{pdflscape.dtx}{test0,test2}}%
%  \file{pdflscape-test3.tex}{\from{pdflscape.dtx}{test0,test3}}%
%  \file{pdflscape-test4.tex}{\from{pdflscape.dtx}{test0,test4}}%
%  \file{pdflscape-test5.tex}{\from{pdflscape.dtx}{test5}}%
%  \file{pdflscape-test6.tex}{\from{pdflscape.dtx}{test6}}%
  \nopreamble
  \nopostamble
%  \file{pdflscape-test6.txt}{\from{pdflscape.dtx}{result6}}%
  \file{ExtractRotate.java}{\from{pdflscape.dtx}{ExtractRotate}}%
%  \usedir{source/latex/oberdiek/catalogue}%
%  \file{pdflscape.xml}{\from{pdflscape.dtx}{catalogue}}%
}

\catcode32=13\relax% active space
\let =\space%
\Msg{************************************************************************}
\Msg{*}
\Msg{* To finish the installation you have to move the following}
\Msg{* file into a directory searched by TeX:}
\Msg{*}
\Msg{*     pdflscape.sty}
\Msg{*}
\Msg{* To produce the documentation run the file `pdflscape.drv'}
\Msg{* through LaTeX.}
\Msg{*}
\Msg{* Happy TeXing!}
\Msg{*}
\Msg{************************************************************************}

\endbatchfile
%</install>
%<*ignore>
\fi
%</ignore>
%<*driver>
\NeedsTeXFormat{LaTeX2e}
\ProvidesFile{pdflscape.drv}%
  [2016/05/14 v0.11 Display of landscape pages in PDF (HO)]%
\documentclass{ltxdoc}
\usepackage{holtxdoc}[2011/11/22]
\usepackage{graphics}
\begin{document}
  \DocInput{pdflscape.dtx}%
\end{document}
%</driver>
% \fi
%
%
% \CharacterTable
%  {Upper-case    \A\B\C\D\E\F\G\H\I\J\K\L\M\N\O\P\Q\R\S\T\U\V\W\X\Y\Z
%   Lower-case    \a\b\c\d\e\f\g\h\i\j\k\l\m\n\o\p\q\r\s\t\u\v\w\x\y\z
%   Digits        \0\1\2\3\4\5\6\7\8\9
%   Exclamation   \!     Double quote  \"     Hash (number) \#
%   Dollar        \$     Percent       \%     Ampersand     \&
%   Acute accent  \'     Left paren    \(     Right paren   \)
%   Asterisk      \*     Plus          \+     Comma         \,
%   Minus         \-     Point         \.     Solidus       \/
%   Colon         \:     Semicolon     \;     Less than     \<
%   Equals        \=     Greater than  \>     Question mark \?
%   Commercial at \@     Left bracket  \[     Backslash     \\
%   Right bracket \]     Circumflex    \^     Underscore    \_
%   Grave accent  \`     Left brace    \{     Vertical bar  \|
%   Right brace   \}     Tilde         \~}
%
% \GetFileInfo{pdflscape.drv}
%
% \title{The \xpackage{pdflscape} package}
% \date{2016/05/14 v0.11}
% \author{Heiko Oberdiek\thanks
% {Please report any issues at \url{https://github.com/ho-tex/oberdiek/issues}}\\
% \xemail{heiko.oberdiek at googlemail.com}}
%
% \maketitle
%
% \begin{abstract}
% Package \xpackage{pdflscape} adds PDF support to the
% environment \texttt{landscape} of package \xpackage{lscape}
% by setting the PDF page attribute \texttt{/Rotate}.
% \end{abstract}
%
% \tableofcontents
%
% \section{Documentation}
%
% \subsection{Supported drivers}
%
% \begin{itemize}
% \item \xoption{pdftex},\xoption{luatex}
% \item \xoption{dvips}, \xoption{dvipsone},
%       \xoption{pctex32}, \xoption{pctexps}
%       (and other drivers that provide a non-empty \cs{Gin@PS@raw})
% \item \xoption{dvipdfm}
% \end{itemize}
%
% \subsection{Caveat}
%
% Depending on the configuration Ghostscript
% adds a guessed rotation entry by its own.
% This can lead to two |/Rotate| entries per page.
% To prevent this behaviour set the parameter
% |AutoRotatePages| to |/None|, eg:
% \begin{quote}
%   |ps2pdf -dAutoRotatePages=/None|
% \end{quote}
%
% \subsection{Requirements}
%
% \begin{itemize}
% \item The package \xpackage{lscape}.
% \item The package \xpackage{atbegshi} for all drivers
%        except \xoption{pdftex},\xoption{luatex}.
% \end{itemize}
%
% \subsection{Usage}
%
% Load this package instead of or after package \xpackage{lscape}:
% \begin{quote}
%   |\usepackage{pdflscape}|\\
%   |\begin{landscape}|\dots|\end{landscape}|
% \end{quote}
%
% \StopEventually{
% }
%
% \section{Implementation}
%
% \subsection{Package identification}
%
%    \begin{macrocode}
%<*package>
%    \end{macrocode}
%    Package identification.
%    \begin{macrocode}
\NeedsTeXFormat{LaTeX2e}
\ProvidesPackage{pdflscape}%
  [2016/05/14 v0.11 Display of landscape pages in PDF (HO)]
%    \end{macrocode}
%
% \subsection{Driver options}
%
%    \begin{macrocode}
\let\PLS@option\@empty
\let\PLS@driver\@empty
\def\PLS@temp#1#2{%
  \DeclareOption{#1}{%
    \let\PLS@option\CurrentOption
    \def\PLS@driver{#2}%
  }%
}
\PLS@temp{pdftex}{pdftex}
\PLS@temp{luatex}{pdftex}
\PLS@temp{dvips}{dvips}
\PLS@temp{dvipsone}{dvips}
\PLS@temp{pctexps}{dvips}
\PLS@temp{dvipdfm}{dvipdfm}
\PLS@temp{dvipdfmx}{dvipdfm}
\PLS@temp{xetex}{dvipdfm}
\let\PLS@temp\@empty
\DeclareOption*{%
  \ifx\PLS@temp\@empty
    \let\PLS@temp\CurrentOption
  \else
    \edef\PLS@temp{\PLS@temp,\CurrentOption}%
  \fi
}
\ProcessOptions*
\ifx\PLS@option\@empty
\else
  \expandafter\PassOptionsToPackage\expandafter{\PLS@option}{lscape}%
\fi
\ifx\PLS@temp\@empty
\else
  \expandafter\PassOptionsToPackage\expandafter{\PLS@temp}{lscape}%
\fi
\RequirePackage{lscape}
%    \end{macrocode}
%
% \subsection{Autodetection of driver}
%
% \subsubsection{Driver \xoption{pdftex}}
%
%    \begin{macrocode}
\RequirePackage{ifpdf}[2006/02/20]
\def\PLS@temp{pdftex}
\ifpdf
  \ifx\PLS@driver\@empty
    \let\PLS@driver\PLS@temp
    \PackageInfo{pdflscape}{Auto-detected driver: \PLS@temp}%
  \fi
  \ifx\PLS@driver\PLS@temp
  \else
    \PackageError{pdflscape}{%
      The driver option `\PLS@option' is not compatible\MessageBreak
      with pdfTeX in PDF mode, using driver `pdftex' instead%
    }\@ehc
    \let\PLS@driver\PLS@temp
  \fi
\fi
%    \end{macrocode}
%
% \subsubsection{Driver \xoption{xetex}}
%
%    \begin{macrocode}
\RequirePackage{ifxetex}%
\def\PLS@temp{dvipdfm}
\ifxetex
  \ifx\PLS@driver\@empty
    \let\PLS@driver\PLS@temp
    \PackageInfo{pdflscape}{%
      Auto-detected driver: \PLS@temp\space(xetex)%
    }%
  \fi
  \ifx\PLS@driver\PLS@temp
  \else
    \PackageError{pdflscape}{%
      The driver option `\PLS@option' is not compatible\MessageBreak
      with XeTeX, using driver for XeTeX (dvipdfm) instead%
    }\@ehc
    \let\PLS@driver\PLS@temp
  \fi
\fi
%    \end{macrocode}
%
% \subsubsection{Detect driver based on \cs{Gin@driver}}
%
%    \begin{macrocode}
\def\PLS@temp#1#2{%
  \ifx\PLS@driver\@empty
    \begingroup
      \def\PLS@temp{#1.def}%
    \expandafter\endgroup
    \ifx\PLS@temp\Gin@driver
      \def\PLS@driver{#2}%
      \PackageInfo{pdflscape}{%
        Auto-detected driver: \PLS@driver\space(#1)%
      }%
    \fi
  \fi
}
\PLS@temp{dvipdfm}{dvipdfm}
\PLS@temp{dvipdfmx}{dvipdfm}
\PLS@temp{dvips}{dvips}
\PLS@temp{dvipsone}{dvips}
\PLS@temp{pctexps}{dvips}
%    \end{macrocode}
%
% \subsubsection{Driver \xoption{dvips}}
%
%    \begin{macrocode}
\ifx\PLS@driver\@empty
  \def\PLS@temp{dvips}%
  \ifnum 1=0\ifx\Gin@PS@raw\@undefined
            \else
              \ifx\Gin@PS@raw\relax
              \else
                \ifx\Gin@PS@raw\@empty
                \else
                  1%
                \fi
              \fi
            \fi
            \space
    \let\PLS@driver\PLS@temp
    \PackageInfo{pdflscape}{%
      Auto-detected driver: \PLS@temp\space(PostScript)%
    }%
  \fi
\fi
%    \end{macrocode}
%
%    \begin{macrocode}
\ifx\PLS@driver\@empty
  \PackageError{pdflscape}{%
    Neither pdfTeX, PostScript, nor dvipdfm driver found%
  }\@ehc
\fi
%    \end{macrocode}
%
% \subsection{Driver implementation}
%
% \begin{itemize}
% \item |\PLS@AddRotate #1|\\
%   it expects the correct rotation number in |#1|
%   and implements the adding of the /Rotation entry
%   in the |/Page| object of the current page.
% \item |\PLS@RemoveRotate|\\
%   it removes a previous |/Rotate| entry, if necessary.
% \end{itemize}
%
% \subsubsection{\pdfTeX}
%
% Not too nice is the global setting of \cs{pdfpageattr}.
% Perhaps this can be changed in future versions.
%
% luatex compat.
%    \begin{macrocode}
\ifx\pdfpageattr\@undefined
  \ifx\pdfvariable\@undefined\else 
    \protected\edef\pdfpageattr{\pdfvariable pageattr}
  \fi
\fi
%    \end{macrocode}
%    \begin{macrocode}
\def\PLS@temp{pdftex}
\ifx\PLS@temp\PLS@driver
  \def\PLS@AddRotate#1{%
    \ifnum#1=0 %
%    \end{macrocode}
%    Already the default.
%    \begin{macrocode}
    \else
      \global\pdfpageattr\expandafter{%
        \the\pdfpageattr
        /Rotate #1%
      }%
    \fi
  }%
%    \end{macrocode}
%    Removes a |/Rotate| entry.
%    \begin{macrocode}
  \def\PLS@RemoveRotate{%
    \begingroup
      \global\pdfpageattr\expandafter{\expandafter}%
      \expandafter\PLS@@RemoveRotate
        \the\pdfpageattr /Rotate\@nil
    \endgroup
  }%
  \def\PLS@@RemoveRotate#1/Rotate#2\@nil{%
%    \end{macrocode}
%    Append |/Rotate| free stuff to \cs{pdfpageattr}.
%    \begin{macrocode}
    \global\pdfpageattr\expandafter{\the\pdfpageattr#1}%
    \ifx\\#2\\%
%    \end{macrocode}
%    Ready, because the detected |/Rotate| is part of
%    the end marker: |/Rotate\@nil|
%    \begin{macrocode}
    \else
%    \end{macrocode}
%    First read in the argument of |/Rotate|,
%    then continue parsing.
%    \begin{macrocode}
      \afterassignment\PLS@@RemoveRotate
      \count0=#2\@nil
    \fi
  }%
\fi
%    \end{macrocode}
%
% \subsubsection{PostScript driver}
%
%    \begin{macrocode}
\def\PLS@temp{dvips}
\ifx\PLS@temp\PLS@driver
  \RequirePackage{atbegshi}%
  \AtBeginShipout{\PLS@AtBeginShipout}%
  \let\PLS@AtBeginShipout\@empty
  \def\PLS@AddRotate#1{%
    \ifnum#1=0 %
    \else
      \def\PLS@AtBeginShipout{%
        \global\setbox\AtBeginShipoutBox\vbox{%
          \Gin@PS@raw{%
            [{ThisPage}\string<\string</Rotate #1\string>\string>%
            /PUT pdfmark%
          }%
          \box\AtBeginShipoutBox
        }%
      }%
    \fi
  }%
  \def\PLS@RemoveRotate{\let\PLS@AtBeginShipout\@empty}%
\fi
%    \end{macrocode}
%
% \subsubsection{Driver dvipdfm}
%
%    \begin{macrocode}
\def\PLS@temp{dvipdfm}
\ifx\PLS@temp\PLS@driver
  \RequirePackage{atbegshi}%
  \AtBeginShipout{\PLS@AtBeginShipout}%
  \let\PLS@AtBeginShipout\@empty
  \def\PLS@AddRotate#1{%
    \ifnum#1=0 %
    \else
      \def\PLS@AtBeginShipout{%
        \global\setbox\AtBeginShipoutBox\vbox{%
          \special{pdf: put @thispage <</Rotate #1>>}%
          \box\AtBeginShipoutBox
        }%
      }%
    \fi
  }%
  \def\PLS@RemoveRotate{\let\PLS@AtBeginShipout\@empty}%
\fi
%    \end{macrocode}
%
% \subsection{Driver independent stuff}
%
%    The landscape environment is extended by
%    adding the correct |/Rotate| entries.
%    \begin{macrocode}
\g@addto@macro{\landscape}{\PLS@Rotate{90}}
\g@addto@macro{\endlandscape}{\PLS@Rotate{0}}
%    \end{macrocode}
%
%    \begin{macro}{\PLS@Rotate}
%    Main macro, that sets the |/Rotate| entry.\\
%    Argument: any TeX number or nothing, that means zero.\\
%    Driver independent.
%    \begin{macrocode}
\def\PLS@Rotate#1{%
  \begingroup
%    \end{macrocode}
%    1. Check and validate the argument.
%    \begin{macrocode}
    \PLS@CheckAngle{#1}%
  \expandafter\endgroup
%    \end{macrocode}
%    2. Remove previous |\Rotate| entry.
%    \begin{macrocode}
  \expandafter\PLS@RemoveRotate
%    \end{macrocode}
%    3. Add |/Rotate| entry.
%    \begin{macrocode}
  \expandafter\PLS@AddRotate\expandafter{\the\count@}%
}
%    \end{macrocode}
%    \end{macro}
%
%    \begin{macro}{\PLS@CheckAngle}
%    Validates the rotation angle.\\
%    The result is stored in the count register |\count@|.\\
%    Driver independent.
%    \begin{macrocode}
\def\PLS@CheckAngle#1{%
%    \end{macrocode}
%    Nothing means zero:
%    \begin{macrocode}
  \ifx\\#1\\%
    \count@=0 %
  \else
    \count@=#1\relax
  \fi
%    \end{macrocode}
%    Normalize to interval -360 $<$ |\count@| $<$ 360.
%    \begin{macrocode}
  \@whilenum\count@>359\do{\advance\count@ -360 }%
  \@whilenum\count@<-359\do{\advance\count@ 360 }%
%    \end{macrocode}
%    Check values: 0, +/-90, +/-180, +/-270
%    \begin{macrocode}
  \ifnum 1=0\ifnum\count@=0 \else
            \ifnum\count@=90 \else
            \ifnum\count@=-90 \else
            \ifnum\count@=180 \else
            \ifnum\count@=-180 \else
            \ifnum\count@=270 \else
            \ifnum\count@=-270 \else
           1\fi\fi\fi\fi\fi\fi\fi\relax
    \PackageError{pdflscape}{%
      Invalid value \the\count@\space for /Rotate%
    }\@ehc
    \count@=0 %
  \fi
}
%    \end{macrocode}
%    \end{macro}
%
%    \begin{macrocode}
%</package>
%    \end{macrocode}
%
% \section{Test}
%
% \subsection{Driver detection tests}
%    \begin{macrocode}
%<*test0>
\NeedsTeXFormat{LaTeX2e}
\documentclass{minimal}
\usepackage{qstest}
\IncludeTests{*}
\LogTests{log}{*}{*}
%</test0>
%    \end{macrocode}
%
%    \begin{macrocode}
%<*test1>
\usepackage{ifpdf}
\ifpdf
  \def\ExpectDriver{pdftex}%
\fi
\usepackage{ifxetex}%
\ifxetex
  \def\ExpectDriver{dvipdfm}%
\fi
\usepackage{pdflscape}[2016/05/14]
%</test1>
%    \end{macrocode}
%
%    \begin{macrocode}
%<*test2>
\def\ExpectDriver{pdftex}
\usepackage[pdftex]{pdflscape}[2016/05/14]
%</test2>
%    \end{macrocode}
%
%    \begin{macrocode}
%<*test3>
\def\ExpectDriver{dvipdfm}
\usepackage[dvipdfmx]{pdflscape}[2016/05/14]
%</test3>
%    \end{macrocode}
%
%    \begin{macrocode}
%<*test4>
\def\ExpectDriver{dvips}
\usepackage[dvipsone]{graphics}
\usepackage{pdflscape}[2016/05/14]
%</test4>
%    \end{macrocode}
%
%    \begin{macrocode}
%<*test0>
\begin{document}
\begin{qstest}{driver}{driver}
  \makeatletter
  \Expect*{\PLS@driver}*{\ExpectDriver}%
\end{qstest}
\end{document}
%</test0>
%    \end{macrocode}
%
% \subsection{Test for \cs{PLS@CheckAngle}}
%
%    \begin{macrocode}
%<*test5>
\NeedsTeXFormat{LaTeX2e}
\documentclass{minimal}
\usepackage{qstest}
\IncludeTests{*}
\LogTests{log}{*}{*}
\usepackage{pdflscape}[2016/05/14]
\begin{document}
\begin{qstest}{checkangle}{checkangle}
  \makeatletter
  \def\Result{%
    \def\ResultString{\the\count@}%
  }%
  \def\PackageError#1#2#3{%
    \def\Result{%
      \def\ResultString{error}%
    }%
  }%
  \def\Test#1#2{%
    \begingroup
      \PLS@CheckAngle{#1}%
      \Result
      \Expect*{\ResultString}{#2}%
    \endgroup
  }%
  \Test{0}{0}%
  \Test{90}{90}%
  \Test{180}{180}%
  \Test{270}{270}%
  \Test{360}{0}%
  \Test{450}{90}%
  \Test{540}{180}%
  \Test{630}{270}%
  \Test{720}{0}%
  \Test{3600}{0}%
  \Test{3690}{90}%
  \Test{-90}{270}%
  \Test{-180}{180}%
  \Test{-270}{-90}%
  \Test{-360}{0}%
  \Test{-450}{270}%
  \Test{540}{180}%
  \Test{-630}{90}%
  \Test{-720}{0}%
  \Test{-3600}{0}%
  \Test{-3690}{270}%
  \Test{1}{error}%
  \Test{-1}{error}%
  \Test{123}{error}%
\end{qstest}
\end{document}
%</test5>
%    \end{macrocode}
%
% \subsection{Test for rotate}
%
%    \begin{macrocode}
%<*test6>
\NeedsTeXFormat{LaTeX2e}
\documentclass{article}
\usepackage{pdflscape}[2016/05/14]
\begin{document}
\section{Portrait page}
\begin{landscape}
\section{Landscape page}
Foo bar foo bar foo bar foo bar foo bar
foo bar foo bar foo bar foo bar foo bar.
\end{landscape}
\section{Portrait Section}
Foo bar
\begin{landscape}
\section{Again Landscape Section}
\newpage
\section{Second Page of Landscape Section}
\end{landscape}
\section{Second Last Portrait Page}
\newpage
\section{Last Portrait Page}
\end{document}
%</test6>
%    \end{macrocode}
%
%    \begin{macrocode}
%<*result6>
/Page 1 /Rotate 0
/Page 2 /Rotate 90
/Page 3 /Rotate 0
/Page 4 /Rotate 90
/Page 5 /Rotate 90
/Page 6 /Rotate 0
/Page 7 /Rotate 0
%</result6>
%    \end{macrocode}
%
%    \begin{macrocode}
%<*ExtractRotate>
/**
 * ExtractRotate.java
 *
 * Copyright (C) 2007 by Heiko Oberdiek <heiko.oberdiek at googlemail.com>
 *
 * Requires: PDFBox (http://www.pdfbox.org/)
 *
 * Syntax: java ExtractRotate <pdffile> <textfile>
 *
 * The <pdffile> is analyzed and for each page its rotation
 * setting is printed in the <textfile>. Example:
 *   /Page 1 /Rotate 0
 *   /Page 2 /Rotate 90
 */
import java.io.FileInputStream;
import java.io.FileWriter;
import org.pdfbox.pdfparser.PDFParser;
import org.pdfbox.pdmodel.PDDocument;
import org.pdfbox.pdmodel.PDDocumentCatalog;
import org.pdfbox.pdmodel.PDPage;

public class ExtractRotate {

    public static void main(String[] args) {
        try {
            String infile = args[0];
            String outfile = args[1];
            FileWriter out = new FileWriter(outfile);
            PDFParser parser =
                    new PDFParser(new FileInputStream(infile));
            parser.parse();
            PDDocument document = parser.getPDDocument();
            PDDocumentCatalog catalog = document.getDocumentCatalog();
            int i = 0;
            for (Object page: catalog.getAllPages()) {
                i++;
                out.write("/Page " + i + " " + "/Rotate "
                        + ((PDPage)page).findRotation() + "\n");
            }
            document.close();
            out.close();
        }
        catch (Exception e) {
            e.printStackTrace();
            System.exit(1);
        }
    }
}
%</ExtractRotate>
%    \end{macrocode}
%
% \section{Installation}
%
% \subsection{Download}
%
% \paragraph{Package.} This package is available on
% CTAN\footnote{\CTANpkg{pdflscape}}:
% \begin{description}
% \item[\CTAN{macros/latex/contrib/oberdiek/pdflscape.dtx}] The source file.
% \item[\CTAN{macros/latex/contrib/oberdiek/pdflscape.pdf}] Documentation.
% \end{description}
%
%
% \paragraph{Bundle.} All the packages of the bundle `oberdiek'
% are also available in a TDS compliant ZIP archive. There
% the packages are already unpacked and the documentation files
% are generated. The files and directories obey the TDS standard.
% \begin{description}
% \item[\CTANinstall{install/macros/latex/contrib/oberdiek.tds.zip}]
% \end{description}
% \emph{TDS} refers to the standard ``A Directory Structure
% for \TeX\ Files'' (\CTAN{tds/tds.pdf}). Directories
% with \xfile{texmf} in their name are usually organized this way.
%
% \subsection{Bundle installation}
%
% \paragraph{Unpacking.} Unpack the \xfile{oberdiek.tds.zip} in the
% TDS tree (also known as \xfile{texmf} tree) of your choice.
% Example (linux):
% \begin{quote}
%   |unzip oberdiek.tds.zip -d ~/texmf|
% \end{quote}
%
% \paragraph{Script installation.}
% Check the directory \xfile{TDS:scripts/oberdiek/} for
% scripts that need further installation steps.
% Package \xpackage{attachfile2} comes with the Perl script
% \xfile{pdfatfi.pl} that should be installed in such a way
% that it can be called as \texttt{pdfatfi}.
% Example (linux):
% \begin{quote}
%   |chmod +x scripts/oberdiek/pdfatfi.pl|\\
%   |cp scripts/oberdiek/pdfatfi.pl /usr/local/bin/|
% \end{quote}
%
% \subsection{Package installation}
%
% \paragraph{Unpacking.} The \xfile{.dtx} file is a self-extracting
% \docstrip\ archive. The files are extracted by running the
% \xfile{.dtx} through \plainTeX:
% \begin{quote}
%   \verb|tex pdflscape.dtx|
% \end{quote}
%
% \paragraph{TDS.} Now the different files must be moved into
% the different directories in your installation TDS tree
% (also known as \xfile{texmf} tree):
% \begin{quote}
% \def\t{^^A
% \begin{tabular}{@{}>{\ttfamily}l@{ $\rightarrow$ }>{\ttfamily}l@{}}
%   pdflscape.sty & tex/latex/oberdiek/pdflscape.sty\\
%   pdflscape.pdf & doc/latex/oberdiek/pdflscape.pdf\\
%   test/pdflscape-test1.tex & doc/latex/oberdiek/test/pdflscape-test1.tex\\
%   test/pdflscape-test2.tex & doc/latex/oberdiek/test/pdflscape-test2.tex\\
%   test/pdflscape-test3.tex & doc/latex/oberdiek/test/pdflscape-test3.tex\\
%   test/pdflscape-test4.tex & doc/latex/oberdiek/test/pdflscape-test4.tex\\
%   test/pdflscape-test5.tex & doc/latex/oberdiek/test/pdflscape-test5.tex\\
%   test/pdflscape-test6.tex & doc/latex/oberdiek/test/pdflscape-test6.tex\\
%   test/pdflscape-test6.txt & doc/latex/oberdiek/test/pdflscape-test6.txt\\
%   test/ExtractRotate.java & doc/latex/oberdiek/test/ExtractRotate.java\\
%   pdflscape.dtx & source/latex/oberdiek/pdflscape.dtx\\
% \end{tabular}^^A
% }^^A
% \sbox0{\t}^^A
% \ifdim\wd0>\linewidth
%   \begingroup
%     \advance\linewidth by\leftmargin
%     \advance\linewidth by\rightmargin
%   \edef\x{\endgroup
%     \def\noexpand\lw{\the\linewidth}^^A
%   }\x
%   \def\lwbox{^^A
%     \leavevmode
%     \hbox to \linewidth{^^A
%       \kern-\leftmargin\relax
%       \hss
%       \usebox0
%       \hss
%       \kern-\rightmargin\relax
%     }^^A
%   }^^A
%   \ifdim\wd0>\lw
%     \sbox0{\small\t}^^A
%     \ifdim\wd0>\linewidth
%       \ifdim\wd0>\lw
%         \sbox0{\footnotesize\t}^^A
%         \ifdim\wd0>\linewidth
%           \ifdim\wd0>\lw
%             \sbox0{\scriptsize\t}^^A
%             \ifdim\wd0>\linewidth
%               \ifdim\wd0>\lw
%                 \sbox0{\tiny\t}^^A
%                 \ifdim\wd0>\linewidth
%                   \lwbox
%                 \else
%                   \usebox0
%                 \fi
%               \else
%                 \lwbox
%               \fi
%             \else
%               \usebox0
%             \fi
%           \else
%             \lwbox
%           \fi
%         \else
%           \usebox0
%         \fi
%       \else
%         \lwbox
%       \fi
%     \else
%       \usebox0
%     \fi
%   \else
%     \lwbox
%   \fi
% \else
%   \usebox0
% \fi
% \end{quote}
% If you have a \xfile{docstrip.cfg} that configures and enables \docstrip's
% TDS installing feature, then some files can already be in the right
% place, see the documentation of \docstrip.
%
% \subsection{Refresh file name databases}
%
% If your \TeX~distribution
% (\teTeX, \mikTeX, \dots) relies on file name databases, you must refresh
% these. For example, \teTeX\ users run \verb|texhash| or
% \verb|mktexlsr|.
%
% \subsection{Some details for the interested}
%
% \paragraph{Attached source.}
%
% The PDF documentation on CTAN also includes the
% \xfile{.dtx} source file. It can be extracted by
% AcrobatReader 6 or higher. Another option is \textsf{pdftk},
% e.g. unpack the file into the current directory:
% \begin{quote}
%   \verb|pdftk pdflscape.pdf unpack_files output .|
% \end{quote}
%
% \paragraph{Unpacking with \LaTeX.}
% The \xfile{.dtx} chooses its action depending on the format:
% \begin{description}
% \item[\plainTeX:] Run \docstrip\ and extract the files.
% \item[\LaTeX:] Generate the documentation.
% \end{description}
% If you insist on using \LaTeX\ for \docstrip\ (really,
% \docstrip\ does not need \LaTeX), then inform the autodetect routine
% about your intention:
% \begin{quote}
%   \verb|latex \let\install=y% \iffalse meta-comment
%
% File: pdflscape.dtx
% Version: 2016/05/14 v0.11
% Info: Display of landscape pages in PDF
%
% Copyright (C) 2001, 2004, 2006-2008 by
%    Heiko Oberdiek <heiko.oberdiek at googlemail.com>
%    2016
%    https://github.com/ho-tex/oberdiek/issues
%
% This work may be distributed and/or modified under the
% conditions of the LaTeX Project Public License, either
% version 1.3c of this license or (at your option) any later
% version. This version of this license is in
%    http://www.latex-project.org/lppl/lppl-1-3c.txt
% and the latest version of this license is in
%    http://www.latex-project.org/lppl.txt
% and version 1.3 or later is part of all distributions of
% LaTeX version 2005/12/01 or later.
%
% This work has the LPPL maintenance status "maintained".
%
% This Current Maintainer of this work is Heiko Oberdiek.
%
% This work consists of the main source file pdflscape.dtx
% and the derived files
%    pdflscape.sty, pdflscape.pdf, pdflscape.ins, pdflscape.drv,
%    pdflscape-test1.tex, pdflscape-test2.tex,
%    pdflscape-test3.tex, pdflscape-test4.tex,
%    pdflscape-test5.tex, pdflscape-test6.tex,
%    pdflscape-test6.txt, ExtractRotate.java.
%
% Distribution:
%    CTAN:macros/latex/contrib/oberdiek/pdflscape.dtx
%    CTAN:macros/latex/contrib/oberdiek/pdflscape.pdf
%
% Unpacking:
%    (a) If pdflscape.ins is present:
%           tex pdflscape.ins
%    (b) Without pdflscape.ins:
%           tex pdflscape.dtx
%    (c) If you insist on using LaTeX
%           latex \let\install=y\input{pdflscape.dtx}
%        (quote the arguments according to the demands of your shell)
%
% Documentation:
%    (a) If pdflscape.drv is present:
%           latex pdflscape.drv
%    (b) Without pdflscape.drv:
%           latex pdflscape.dtx; ...
%    The class ltxdoc loads the configuration file ltxdoc.cfg
%    if available. Here you can specify further options, e.g.
%    use A4 as paper format:
%       \PassOptionsToClass{a4paper}{article}
%
%    Programm calls to get the documentation (example):
%       pdflatex pdflscape.dtx
%       makeindex -s gind.ist pdflscape.idx
%       pdflatex pdflscape.dtx
%       makeindex -s gind.ist pdflscape.idx
%       pdflatex pdflscape.dtx
%
% Installation:
%    TDS:tex/latex/oberdiek/pdflscape.sty
%    TDS:doc/latex/oberdiek/pdflscape.pdf
%    TDS:doc/latex/oberdiek/test/pdflscape-test1.tex
%    TDS:doc/latex/oberdiek/test/pdflscape-test2.tex
%    TDS:doc/latex/oberdiek/test/pdflscape-test3.tex
%    TDS:doc/latex/oberdiek/test/pdflscape-test4.tex
%    TDS:doc/latex/oberdiek/test/pdflscape-test5.tex
%    TDS:doc/latex/oberdiek/test/pdflscape-test6.tex
%    TDS:doc/latex/oberdiek/test/pdflscape-test6.txt
%    TDS:doc/latex/oberdiek/test/ExtractRotate.java
%    TDS:source/latex/oberdiek/pdflscape.dtx
%
%<*ignore>
\begingroup
  \catcode123=1 %
  \catcode125=2 %
  \def\x{LaTeX2e}%
\expandafter\endgroup
\ifcase 0\ifx\install y1\fi\expandafter
         \ifx\csname processbatchFile\endcsname\relax\else1\fi
         \ifx\fmtname\x\else 1\fi\relax
\else\csname fi\endcsname
%</ignore>
%<*install>
\input docstrip.tex
\Msg{************************************************************************}
\Msg{* Installation}
\Msg{* Package: pdflscape 2016/05/14 v0.11 Display of landscape pages in PDF (HO)}
\Msg{************************************************************************}

\keepsilent
\askforoverwritefalse

\let\MetaPrefix\relax
\preamble

This is a generated file.

Project: pdflscape
Version: 2016/05/14 v0.11

Copyright (C) 2001, 2004, 2006-2008 by
   Heiko Oberdiek <heiko.oberdiek at googlemail.com>

This work may be distributed and/or modified under the
conditions of the LaTeX Project Public License, either
version 1.3c of this license or (at your option) any later
version. This version of this license is in
   http://www.latex-project.org/lppl/lppl-1-3c.txt
and the latest version of this license is in
   http://www.latex-project.org/lppl.txt
and version 1.3 or later is part of all distributions of
LaTeX version 2005/12/01 or later.

This work has the LPPL maintenance status "maintained".

This Current Maintainer of this work is Heiko Oberdiek.

This work consists of the main source file pdflscape.dtx
and the derived files
   pdflscape.sty, pdflscape.pdf, pdflscape.ins, pdflscape.drv,
   pdflscape-test1.tex, pdflscape-test2.tex,
   pdflscape-test3.tex, pdflscape-test4.tex,
   pdflscape-test5.tex, pdflscape-test6.tex,
   pdflscape-test6.txt, ExtractRotate.java.

\endpreamble
\let\MetaPrefix\DoubleperCent

\generate{%
  \file{pdflscape.ins}{\from{pdflscape.dtx}{install}}%
  \file{pdflscape.drv}{\from{pdflscape.dtx}{driver}}%
  \usedir{tex/latex/oberdiek}%
  \file{pdflscape.sty}{\from{pdflscape.dtx}{package}}%
%  \usedir{doc/latex/oberdiek/test}%
%  \file{pdflscape-test1.tex}{\from{pdflscape.dtx}{test0,test1}}%
%  \file{pdflscape-test2.tex}{\from{pdflscape.dtx}{test0,test2}}%
%  \file{pdflscape-test3.tex}{\from{pdflscape.dtx}{test0,test3}}%
%  \file{pdflscape-test4.tex}{\from{pdflscape.dtx}{test0,test4}}%
%  \file{pdflscape-test5.tex}{\from{pdflscape.dtx}{test5}}%
%  \file{pdflscape-test6.tex}{\from{pdflscape.dtx}{test6}}%
  \nopreamble
  \nopostamble
%  \file{pdflscape-test6.txt}{\from{pdflscape.dtx}{result6}}%
  \file{ExtractRotate.java}{\from{pdflscape.dtx}{ExtractRotate}}%
%  \usedir{source/latex/oberdiek/catalogue}%
%  \file{pdflscape.xml}{\from{pdflscape.dtx}{catalogue}}%
}

\catcode32=13\relax% active space
\let =\space%
\Msg{************************************************************************}
\Msg{*}
\Msg{* To finish the installation you have to move the following}
\Msg{* file into a directory searched by TeX:}
\Msg{*}
\Msg{*     pdflscape.sty}
\Msg{*}
\Msg{* To produce the documentation run the file `pdflscape.drv'}
\Msg{* through LaTeX.}
\Msg{*}
\Msg{* Happy TeXing!}
\Msg{*}
\Msg{************************************************************************}

\endbatchfile
%</install>
%<*ignore>
\fi
%</ignore>
%<*driver>
\NeedsTeXFormat{LaTeX2e}
\ProvidesFile{pdflscape.drv}%
  [2016/05/14 v0.11 Display of landscape pages in PDF (HO)]%
\documentclass{ltxdoc}
\usepackage{holtxdoc}[2011/11/22]
\usepackage{graphics}
\begin{document}
  \DocInput{pdflscape.dtx}%
\end{document}
%</driver>
% \fi
%
%
% \CharacterTable
%  {Upper-case    \A\B\C\D\E\F\G\H\I\J\K\L\M\N\O\P\Q\R\S\T\U\V\W\X\Y\Z
%   Lower-case    \a\b\c\d\e\f\g\h\i\j\k\l\m\n\o\p\q\r\s\t\u\v\w\x\y\z
%   Digits        \0\1\2\3\4\5\6\7\8\9
%   Exclamation   \!     Double quote  \"     Hash (number) \#
%   Dollar        \$     Percent       \%     Ampersand     \&
%   Acute accent  \'     Left paren    \(     Right paren   \)
%   Asterisk      \*     Plus          \+     Comma         \,
%   Minus         \-     Point         \.     Solidus       \/
%   Colon         \:     Semicolon     \;     Less than     \<
%   Equals        \=     Greater than  \>     Question mark \?
%   Commercial at \@     Left bracket  \[     Backslash     \\
%   Right bracket \]     Circumflex    \^     Underscore    \_
%   Grave accent  \`     Left brace    \{     Vertical bar  \|
%   Right brace   \}     Tilde         \~}
%
% \GetFileInfo{pdflscape.drv}
%
% \title{The \xpackage{pdflscape} package}
% \date{2016/05/14 v0.11}
% \author{Heiko Oberdiek\thanks
% {Please report any issues at \url{https://github.com/ho-tex/oberdiek/issues}}\\
% \xemail{heiko.oberdiek at googlemail.com}}
%
% \maketitle
%
% \begin{abstract}
% Package \xpackage{pdflscape} adds PDF support to the
% environment \texttt{landscape} of package \xpackage{lscape}
% by setting the PDF page attribute \texttt{/Rotate}.
% \end{abstract}
%
% \tableofcontents
%
% \section{Documentation}
%
% \subsection{Supported drivers}
%
% \begin{itemize}
% \item \xoption{pdftex},\xoption{luatex}
% \item \xoption{dvips}, \xoption{dvipsone},
%       \xoption{pctex32}, \xoption{pctexps}
%       (and other drivers that provide a non-empty \cs{Gin@PS@raw})
% \item \xoption{dvipdfm}
% \end{itemize}
%
% \subsection{Caveat}
%
% Depending on the configuration Ghostscript
% adds a guessed rotation entry by its own.
% This can lead to two |/Rotate| entries per page.
% To prevent this behaviour set the parameter
% |AutoRotatePages| to |/None|, eg:
% \begin{quote}
%   |ps2pdf -dAutoRotatePages=/None|
% \end{quote}
%
% \subsection{Requirements}
%
% \begin{itemize}
% \item The package \xpackage{lscape}.
% \item The package \xpackage{atbegshi} for all drivers
%        except \xoption{pdftex},\xoption{luatex}.
% \end{itemize}
%
% \subsection{Usage}
%
% Load this package instead of or after package \xpackage{lscape}:
% \begin{quote}
%   |\usepackage{pdflscape}|\\
%   |\begin{landscape}|\dots|\end{landscape}|
% \end{quote}
%
% \StopEventually{
% }
%
% \section{Implementation}
%
% \subsection{Package identification}
%
%    \begin{macrocode}
%<*package>
%    \end{macrocode}
%    Package identification.
%    \begin{macrocode}
\NeedsTeXFormat{LaTeX2e}
\ProvidesPackage{pdflscape}%
  [2016/05/14 v0.11 Display of landscape pages in PDF (HO)]
%    \end{macrocode}
%
% \subsection{Driver options}
%
%    \begin{macrocode}
\let\PLS@option\@empty
\let\PLS@driver\@empty
\def\PLS@temp#1#2{%
  \DeclareOption{#1}{%
    \let\PLS@option\CurrentOption
    \def\PLS@driver{#2}%
  }%
}
\PLS@temp{pdftex}{pdftex}
\PLS@temp{luatex}{pdftex}
\PLS@temp{dvips}{dvips}
\PLS@temp{dvipsone}{dvips}
\PLS@temp{pctexps}{dvips}
\PLS@temp{dvipdfm}{dvipdfm}
\PLS@temp{dvipdfmx}{dvipdfm}
\PLS@temp{xetex}{dvipdfm}
\let\PLS@temp\@empty
\DeclareOption*{%
  \ifx\PLS@temp\@empty
    \let\PLS@temp\CurrentOption
  \else
    \edef\PLS@temp{\PLS@temp,\CurrentOption}%
  \fi
}
\ProcessOptions*
\ifx\PLS@option\@empty
\else
  \expandafter\PassOptionsToPackage\expandafter{\PLS@option}{lscape}%
\fi
\ifx\PLS@temp\@empty
\else
  \expandafter\PassOptionsToPackage\expandafter{\PLS@temp}{lscape}%
\fi
\RequirePackage{lscape}
%    \end{macrocode}
%
% \subsection{Autodetection of driver}
%
% \subsubsection{Driver \xoption{pdftex}}
%
%    \begin{macrocode}
\RequirePackage{ifpdf}[2006/02/20]
\def\PLS@temp{pdftex}
\ifpdf
  \ifx\PLS@driver\@empty
    \let\PLS@driver\PLS@temp
    \PackageInfo{pdflscape}{Auto-detected driver: \PLS@temp}%
  \fi
  \ifx\PLS@driver\PLS@temp
  \else
    \PackageError{pdflscape}{%
      The driver option `\PLS@option' is not compatible\MessageBreak
      with pdfTeX in PDF mode, using driver `pdftex' instead%
    }\@ehc
    \let\PLS@driver\PLS@temp
  \fi
\fi
%    \end{macrocode}
%
% \subsubsection{Driver \xoption{xetex}}
%
%    \begin{macrocode}
\RequirePackage{ifxetex}%
\def\PLS@temp{dvipdfm}
\ifxetex
  \ifx\PLS@driver\@empty
    \let\PLS@driver\PLS@temp
    \PackageInfo{pdflscape}{%
      Auto-detected driver: \PLS@temp\space(xetex)%
    }%
  \fi
  \ifx\PLS@driver\PLS@temp
  \else
    \PackageError{pdflscape}{%
      The driver option `\PLS@option' is not compatible\MessageBreak
      with XeTeX, using driver for XeTeX (dvipdfm) instead%
    }\@ehc
    \let\PLS@driver\PLS@temp
  \fi
\fi
%    \end{macrocode}
%
% \subsubsection{Detect driver based on \cs{Gin@driver}}
%
%    \begin{macrocode}
\def\PLS@temp#1#2{%
  \ifx\PLS@driver\@empty
    \begingroup
      \def\PLS@temp{#1.def}%
    \expandafter\endgroup
    \ifx\PLS@temp\Gin@driver
      \def\PLS@driver{#2}%
      \PackageInfo{pdflscape}{%
        Auto-detected driver: \PLS@driver\space(#1)%
      }%
    \fi
  \fi
}
\PLS@temp{dvipdfm}{dvipdfm}
\PLS@temp{dvipdfmx}{dvipdfm}
\PLS@temp{dvips}{dvips}
\PLS@temp{dvipsone}{dvips}
\PLS@temp{pctexps}{dvips}
%    \end{macrocode}
%
% \subsubsection{Driver \xoption{dvips}}
%
%    \begin{macrocode}
\ifx\PLS@driver\@empty
  \def\PLS@temp{dvips}%
  \ifnum 1=0\ifx\Gin@PS@raw\@undefined
            \else
              \ifx\Gin@PS@raw\relax
              \else
                \ifx\Gin@PS@raw\@empty
                \else
                  1%
                \fi
              \fi
            \fi
            \space
    \let\PLS@driver\PLS@temp
    \PackageInfo{pdflscape}{%
      Auto-detected driver: \PLS@temp\space(PostScript)%
    }%
  \fi
\fi
%    \end{macrocode}
%
%    \begin{macrocode}
\ifx\PLS@driver\@empty
  \PackageError{pdflscape}{%
    Neither pdfTeX, PostScript, nor dvipdfm driver found%
  }\@ehc
\fi
%    \end{macrocode}
%
% \subsection{Driver implementation}
%
% \begin{itemize}
% \item |\PLS@AddRotate #1|\\
%   it expects the correct rotation number in |#1|
%   and implements the adding of the /Rotation entry
%   in the |/Page| object of the current page.
% \item |\PLS@RemoveRotate|\\
%   it removes a previous |/Rotate| entry, if necessary.
% \end{itemize}
%
% \subsubsection{\pdfTeX}
%
% Not too nice is the global setting of \cs{pdfpageattr}.
% Perhaps this can be changed in future versions.
%
% luatex compat.
%    \begin{macrocode}
\ifx\pdfpageattr\@undefined
  \ifx\pdfvariable\@undefined\else 
    \protected\edef\pdfpageattr{\pdfvariable pageattr}
  \fi
\fi
%    \end{macrocode}
%    \begin{macrocode}
\def\PLS@temp{pdftex}
\ifx\PLS@temp\PLS@driver
  \def\PLS@AddRotate#1{%
    \ifnum#1=0 %
%    \end{macrocode}
%    Already the default.
%    \begin{macrocode}
    \else
      \global\pdfpageattr\expandafter{%
        \the\pdfpageattr
        /Rotate #1%
      }%
    \fi
  }%
%    \end{macrocode}
%    Removes a |/Rotate| entry.
%    \begin{macrocode}
  \def\PLS@RemoveRotate{%
    \begingroup
      \global\pdfpageattr\expandafter{\expandafter}%
      \expandafter\PLS@@RemoveRotate
        \the\pdfpageattr /Rotate\@nil
    \endgroup
  }%
  \def\PLS@@RemoveRotate#1/Rotate#2\@nil{%
%    \end{macrocode}
%    Append |/Rotate| free stuff to \cs{pdfpageattr}.
%    \begin{macrocode}
    \global\pdfpageattr\expandafter{\the\pdfpageattr#1}%
    \ifx\\#2\\%
%    \end{macrocode}
%    Ready, because the detected |/Rotate| is part of
%    the end marker: |/Rotate\@nil|
%    \begin{macrocode}
    \else
%    \end{macrocode}
%    First read in the argument of |/Rotate|,
%    then continue parsing.
%    \begin{macrocode}
      \afterassignment\PLS@@RemoveRotate
      \count0=#2\@nil
    \fi
  }%
\fi
%    \end{macrocode}
%
% \subsubsection{PostScript driver}
%
%    \begin{macrocode}
\def\PLS@temp{dvips}
\ifx\PLS@temp\PLS@driver
  \RequirePackage{atbegshi}%
  \AtBeginShipout{\PLS@AtBeginShipout}%
  \let\PLS@AtBeginShipout\@empty
  \def\PLS@AddRotate#1{%
    \ifnum#1=0 %
    \else
      \def\PLS@AtBeginShipout{%
        \global\setbox\AtBeginShipoutBox\vbox{%
          \Gin@PS@raw{%
            [{ThisPage}\string<\string</Rotate #1\string>\string>%
            /PUT pdfmark%
          }%
          \box\AtBeginShipoutBox
        }%
      }%
    \fi
  }%
  \def\PLS@RemoveRotate{\let\PLS@AtBeginShipout\@empty}%
\fi
%    \end{macrocode}
%
% \subsubsection{Driver dvipdfm}
%
%    \begin{macrocode}
\def\PLS@temp{dvipdfm}
\ifx\PLS@temp\PLS@driver
  \RequirePackage{atbegshi}%
  \AtBeginShipout{\PLS@AtBeginShipout}%
  \let\PLS@AtBeginShipout\@empty
  \def\PLS@AddRotate#1{%
    \ifnum#1=0 %
    \else
      \def\PLS@AtBeginShipout{%
        \global\setbox\AtBeginShipoutBox\vbox{%
          \special{pdf: put @thispage <</Rotate #1>>}%
          \box\AtBeginShipoutBox
        }%
      }%
    \fi
  }%
  \def\PLS@RemoveRotate{\let\PLS@AtBeginShipout\@empty}%
\fi
%    \end{macrocode}
%
% \subsection{Driver independent stuff}
%
%    The landscape environment is extended by
%    adding the correct |/Rotate| entries.
%    \begin{macrocode}
\g@addto@macro{\landscape}{\PLS@Rotate{90}}
\g@addto@macro{\endlandscape}{\PLS@Rotate{0}}
%    \end{macrocode}
%
%    \begin{macro}{\PLS@Rotate}
%    Main macro, that sets the |/Rotate| entry.\\
%    Argument: any TeX number or nothing, that means zero.\\
%    Driver independent.
%    \begin{macrocode}
\def\PLS@Rotate#1{%
  \begingroup
%    \end{macrocode}
%    1. Check and validate the argument.
%    \begin{macrocode}
    \PLS@CheckAngle{#1}%
  \expandafter\endgroup
%    \end{macrocode}
%    2. Remove previous |\Rotate| entry.
%    \begin{macrocode}
  \expandafter\PLS@RemoveRotate
%    \end{macrocode}
%    3. Add |/Rotate| entry.
%    \begin{macrocode}
  \expandafter\PLS@AddRotate\expandafter{\the\count@}%
}
%    \end{macrocode}
%    \end{macro}
%
%    \begin{macro}{\PLS@CheckAngle}
%    Validates the rotation angle.\\
%    The result is stored in the count register |\count@|.\\
%    Driver independent.
%    \begin{macrocode}
\def\PLS@CheckAngle#1{%
%    \end{macrocode}
%    Nothing means zero:
%    \begin{macrocode}
  \ifx\\#1\\%
    \count@=0 %
  \else
    \count@=#1\relax
  \fi
%    \end{macrocode}
%    Normalize to interval -360 $<$ |\count@| $<$ 360.
%    \begin{macrocode}
  \@whilenum\count@>359\do{\advance\count@ -360 }%
  \@whilenum\count@<-359\do{\advance\count@ 360 }%
%    \end{macrocode}
%    Check values: 0, +/-90, +/-180, +/-270
%    \begin{macrocode}
  \ifnum 1=0\ifnum\count@=0 \else
            \ifnum\count@=90 \else
            \ifnum\count@=-90 \else
            \ifnum\count@=180 \else
            \ifnum\count@=-180 \else
            \ifnum\count@=270 \else
            \ifnum\count@=-270 \else
           1\fi\fi\fi\fi\fi\fi\fi\relax
    \PackageError{pdflscape}{%
      Invalid value \the\count@\space for /Rotate%
    }\@ehc
    \count@=0 %
  \fi
}
%    \end{macrocode}
%    \end{macro}
%
%    \begin{macrocode}
%</package>
%    \end{macrocode}
%
% \section{Test}
%
% \subsection{Driver detection tests}
%    \begin{macrocode}
%<*test0>
\NeedsTeXFormat{LaTeX2e}
\documentclass{minimal}
\usepackage{qstest}
\IncludeTests{*}
\LogTests{log}{*}{*}
%</test0>
%    \end{macrocode}
%
%    \begin{macrocode}
%<*test1>
\usepackage{ifpdf}
\ifpdf
  \def\ExpectDriver{pdftex}%
\fi
\usepackage{ifxetex}%
\ifxetex
  \def\ExpectDriver{dvipdfm}%
\fi
\usepackage{pdflscape}[2016/05/14]
%</test1>
%    \end{macrocode}
%
%    \begin{macrocode}
%<*test2>
\def\ExpectDriver{pdftex}
\usepackage[pdftex]{pdflscape}[2016/05/14]
%</test2>
%    \end{macrocode}
%
%    \begin{macrocode}
%<*test3>
\def\ExpectDriver{dvipdfm}
\usepackage[dvipdfmx]{pdflscape}[2016/05/14]
%</test3>
%    \end{macrocode}
%
%    \begin{macrocode}
%<*test4>
\def\ExpectDriver{dvips}
\usepackage[dvipsone]{graphics}
\usepackage{pdflscape}[2016/05/14]
%</test4>
%    \end{macrocode}
%
%    \begin{macrocode}
%<*test0>
\begin{document}
\begin{qstest}{driver}{driver}
  \makeatletter
  \Expect*{\PLS@driver}*{\ExpectDriver}%
\end{qstest}
\end{document}
%</test0>
%    \end{macrocode}
%
% \subsection{Test for \cs{PLS@CheckAngle}}
%
%    \begin{macrocode}
%<*test5>
\NeedsTeXFormat{LaTeX2e}
\documentclass{minimal}
\usepackage{qstest}
\IncludeTests{*}
\LogTests{log}{*}{*}
\usepackage{pdflscape}[2016/05/14]
\begin{document}
\begin{qstest}{checkangle}{checkangle}
  \makeatletter
  \def\Result{%
    \def\ResultString{\the\count@}%
  }%
  \def\PackageError#1#2#3{%
    \def\Result{%
      \def\ResultString{error}%
    }%
  }%
  \def\Test#1#2{%
    \begingroup
      \PLS@CheckAngle{#1}%
      \Result
      \Expect*{\ResultString}{#2}%
    \endgroup
  }%
  \Test{0}{0}%
  \Test{90}{90}%
  \Test{180}{180}%
  \Test{270}{270}%
  \Test{360}{0}%
  \Test{450}{90}%
  \Test{540}{180}%
  \Test{630}{270}%
  \Test{720}{0}%
  \Test{3600}{0}%
  \Test{3690}{90}%
  \Test{-90}{270}%
  \Test{-180}{180}%
  \Test{-270}{-90}%
  \Test{-360}{0}%
  \Test{-450}{270}%
  \Test{540}{180}%
  \Test{-630}{90}%
  \Test{-720}{0}%
  \Test{-3600}{0}%
  \Test{-3690}{270}%
  \Test{1}{error}%
  \Test{-1}{error}%
  \Test{123}{error}%
\end{qstest}
\end{document}
%</test5>
%    \end{macrocode}
%
% \subsection{Test for rotate}
%
%    \begin{macrocode}
%<*test6>
\NeedsTeXFormat{LaTeX2e}
\documentclass{article}
\usepackage{pdflscape}[2016/05/14]
\begin{document}
\section{Portrait page}
\begin{landscape}
\section{Landscape page}
Foo bar foo bar foo bar foo bar foo bar
foo bar foo bar foo bar foo bar foo bar.
\end{landscape}
\section{Portrait Section}
Foo bar
\begin{landscape}
\section{Again Landscape Section}
\newpage
\section{Second Page of Landscape Section}
\end{landscape}
\section{Second Last Portrait Page}
\newpage
\section{Last Portrait Page}
\end{document}
%</test6>
%    \end{macrocode}
%
%    \begin{macrocode}
%<*result6>
/Page 1 /Rotate 0
/Page 2 /Rotate 90
/Page 3 /Rotate 0
/Page 4 /Rotate 90
/Page 5 /Rotate 90
/Page 6 /Rotate 0
/Page 7 /Rotate 0
%</result6>
%    \end{macrocode}
%
%    \begin{macrocode}
%<*ExtractRotate>
/**
 * ExtractRotate.java
 *
 * Copyright (C) 2007 by Heiko Oberdiek <heiko.oberdiek at googlemail.com>
 *
 * Requires: PDFBox (http://www.pdfbox.org/)
 *
 * Syntax: java ExtractRotate <pdffile> <textfile>
 *
 * The <pdffile> is analyzed and for each page its rotation
 * setting is printed in the <textfile>. Example:
 *   /Page 1 /Rotate 0
 *   /Page 2 /Rotate 90
 */
import java.io.FileInputStream;
import java.io.FileWriter;
import org.pdfbox.pdfparser.PDFParser;
import org.pdfbox.pdmodel.PDDocument;
import org.pdfbox.pdmodel.PDDocumentCatalog;
import org.pdfbox.pdmodel.PDPage;

public class ExtractRotate {

    public static void main(String[] args) {
        try {
            String infile = args[0];
            String outfile = args[1];
            FileWriter out = new FileWriter(outfile);
            PDFParser parser =
                    new PDFParser(new FileInputStream(infile));
            parser.parse();
            PDDocument document = parser.getPDDocument();
            PDDocumentCatalog catalog = document.getDocumentCatalog();
            int i = 0;
            for (Object page: catalog.getAllPages()) {
                i++;
                out.write("/Page " + i + " " + "/Rotate "
                        + ((PDPage)page).findRotation() + "\n");
            }
            document.close();
            out.close();
        }
        catch (Exception e) {
            e.printStackTrace();
            System.exit(1);
        }
    }
}
%</ExtractRotate>
%    \end{macrocode}
%
% \section{Installation}
%
% \subsection{Download}
%
% \paragraph{Package.} This package is available on
% CTAN\footnote{\CTANpkg{pdflscape}}:
% \begin{description}
% \item[\CTAN{macros/latex/contrib/oberdiek/pdflscape.dtx}] The source file.
% \item[\CTAN{macros/latex/contrib/oberdiek/pdflscape.pdf}] Documentation.
% \end{description}
%
%
% \paragraph{Bundle.} All the packages of the bundle `oberdiek'
% are also available in a TDS compliant ZIP archive. There
% the packages are already unpacked and the documentation files
% are generated. The files and directories obey the TDS standard.
% \begin{description}
% \item[\CTANinstall{install/macros/latex/contrib/oberdiek.tds.zip}]
% \end{description}
% \emph{TDS} refers to the standard ``A Directory Structure
% for \TeX\ Files'' (\CTAN{tds/tds.pdf}). Directories
% with \xfile{texmf} in their name are usually organized this way.
%
% \subsection{Bundle installation}
%
% \paragraph{Unpacking.} Unpack the \xfile{oberdiek.tds.zip} in the
% TDS tree (also known as \xfile{texmf} tree) of your choice.
% Example (linux):
% \begin{quote}
%   |unzip oberdiek.tds.zip -d ~/texmf|
% \end{quote}
%
% \paragraph{Script installation.}
% Check the directory \xfile{TDS:scripts/oberdiek/} for
% scripts that need further installation steps.
% Package \xpackage{attachfile2} comes with the Perl script
% \xfile{pdfatfi.pl} that should be installed in such a way
% that it can be called as \texttt{pdfatfi}.
% Example (linux):
% \begin{quote}
%   |chmod +x scripts/oberdiek/pdfatfi.pl|\\
%   |cp scripts/oberdiek/pdfatfi.pl /usr/local/bin/|
% \end{quote}
%
% \subsection{Package installation}
%
% \paragraph{Unpacking.} The \xfile{.dtx} file is a self-extracting
% \docstrip\ archive. The files are extracted by running the
% \xfile{.dtx} through \plainTeX:
% \begin{quote}
%   \verb|tex pdflscape.dtx|
% \end{quote}
%
% \paragraph{TDS.} Now the different files must be moved into
% the different directories in your installation TDS tree
% (also known as \xfile{texmf} tree):
% \begin{quote}
% \def\t{^^A
% \begin{tabular}{@{}>{\ttfamily}l@{ $\rightarrow$ }>{\ttfamily}l@{}}
%   pdflscape.sty & tex/latex/oberdiek/pdflscape.sty\\
%   pdflscape.pdf & doc/latex/oberdiek/pdflscape.pdf\\
%   test/pdflscape-test1.tex & doc/latex/oberdiek/test/pdflscape-test1.tex\\
%   test/pdflscape-test2.tex & doc/latex/oberdiek/test/pdflscape-test2.tex\\
%   test/pdflscape-test3.tex & doc/latex/oberdiek/test/pdflscape-test3.tex\\
%   test/pdflscape-test4.tex & doc/latex/oberdiek/test/pdflscape-test4.tex\\
%   test/pdflscape-test5.tex & doc/latex/oberdiek/test/pdflscape-test5.tex\\
%   test/pdflscape-test6.tex & doc/latex/oberdiek/test/pdflscape-test6.tex\\
%   test/pdflscape-test6.txt & doc/latex/oberdiek/test/pdflscape-test6.txt\\
%   test/ExtractRotate.java & doc/latex/oberdiek/test/ExtractRotate.java\\
%   pdflscape.dtx & source/latex/oberdiek/pdflscape.dtx\\
% \end{tabular}^^A
% }^^A
% \sbox0{\t}^^A
% \ifdim\wd0>\linewidth
%   \begingroup
%     \advance\linewidth by\leftmargin
%     \advance\linewidth by\rightmargin
%   \edef\x{\endgroup
%     \def\noexpand\lw{\the\linewidth}^^A
%   }\x
%   \def\lwbox{^^A
%     \leavevmode
%     \hbox to \linewidth{^^A
%       \kern-\leftmargin\relax
%       \hss
%       \usebox0
%       \hss
%       \kern-\rightmargin\relax
%     }^^A
%   }^^A
%   \ifdim\wd0>\lw
%     \sbox0{\small\t}^^A
%     \ifdim\wd0>\linewidth
%       \ifdim\wd0>\lw
%         \sbox0{\footnotesize\t}^^A
%         \ifdim\wd0>\linewidth
%           \ifdim\wd0>\lw
%             \sbox0{\scriptsize\t}^^A
%             \ifdim\wd0>\linewidth
%               \ifdim\wd0>\lw
%                 \sbox0{\tiny\t}^^A
%                 \ifdim\wd0>\linewidth
%                   \lwbox
%                 \else
%                   \usebox0
%                 \fi
%               \else
%                 \lwbox
%               \fi
%             \else
%               \usebox0
%             \fi
%           \else
%             \lwbox
%           \fi
%         \else
%           \usebox0
%         \fi
%       \else
%         \lwbox
%       \fi
%     \else
%       \usebox0
%     \fi
%   \else
%     \lwbox
%   \fi
% \else
%   \usebox0
% \fi
% \end{quote}
% If you have a \xfile{docstrip.cfg} that configures and enables \docstrip's
% TDS installing feature, then some files can already be in the right
% place, see the documentation of \docstrip.
%
% \subsection{Refresh file name databases}
%
% If your \TeX~distribution
% (\teTeX, \mikTeX, \dots) relies on file name databases, you must refresh
% these. For example, \teTeX\ users run \verb|texhash| or
% \verb|mktexlsr|.
%
% \subsection{Some details for the interested}
%
% \paragraph{Attached source.}
%
% The PDF documentation on CTAN also includes the
% \xfile{.dtx} source file. It can be extracted by
% AcrobatReader 6 or higher. Another option is \textsf{pdftk},
% e.g. unpack the file into the current directory:
% \begin{quote}
%   \verb|pdftk pdflscape.pdf unpack_files output .|
% \end{quote}
%
% \paragraph{Unpacking with \LaTeX.}
% The \xfile{.dtx} chooses its action depending on the format:
% \begin{description}
% \item[\plainTeX:] Run \docstrip\ and extract the files.
% \item[\LaTeX:] Generate the documentation.
% \end{description}
% If you insist on using \LaTeX\ for \docstrip\ (really,
% \docstrip\ does not need \LaTeX), then inform the autodetect routine
% about your intention:
% \begin{quote}
%   \verb|latex \let\install=y\input{pdflscape.dtx}|
% \end{quote}
% Do not forget to quote the argument according to the demands
% of your shell.
%
% \paragraph{Generating the documentation.}
% You can use both the \xfile{.dtx} or the \xfile{.drv} to generate
% the documentation. The process can be configured by the
% configuration file \xfile{ltxdoc.cfg}. For instance, put this
% line into this file, if you want to have A4 as paper format:
% \begin{quote}
%   \verb|\PassOptionsToClass{a4paper}{article}|
% \end{quote}
% An example follows how to generate the
% documentation with pdf\LaTeX:
% \begin{quote}
%\begin{verbatim}
%pdflatex pdflscape.dtx
%makeindex -s gind.ist pdflscape.idx
%pdflatex pdflscape.dtx
%makeindex -s gind.ist pdflscape.idx
%pdflatex pdflscape.dtx
%\end{verbatim}
% \end{quote}
%
% \begin{History}
%   \begin{Version}{2001/01/15 v0.1}
%   \item
%     First public version,
%     published in \xnewsgroup{de.comp.text.tex}:\\
%     \URL{``\link{Re: Querformat, pdfLaTex und thumbpdf}''}^^A
%     {https://groups.google.com/group/de.comp.text.tex/msg/e054c5795e52d2b8}
%   \end{Version}
%   \begin{Version}{2001/02/04 v0.2}
%   \item
%     Minor documentation update.
%   \item
%     CTAN.
%   \end{Version}
%   \begin{Version}{2004/05/11 v0.3}
%   \item
%     Support for dvipdfm added.
%   \end{Version}
%   \begin{Version}{2004/05/12 v0.4}
%   \item
%     Bug fix: support for multipage landscape environment
%     for all drivers except for \xoption{pdftex}.
%   \end{Version}
%   \begin{Version}{2006/02/20 v0.5}
%   \item
%     DTX framework.
%   \item
%     LPPL 1.3
%   \item
%     Code is not changed.
%   \end{Version}
%   \begin{Version}{2006/04/24 v0.6}
%   \item
%     Bug fix: \cs{PLS@RemoveRotate} reinserted for \cs{PLS@Rotate}.
%   \item
%     Fix for \cs{PLS@RemoveRotate} (dvips/dvipdfm): \cs{PLS@EverhShi}
%     hook is cleared.
%   \end{Version}
%   \begin{Version}{2007/04/11 v0.7}
%   \item
%     Line ends sanitized.
%   \end{Version}
%   \begin{Version}{2007/04/17 v0.8}
%   \item
%     Package \xpackage{atbegshi} replaces \xpackage{everyshi}.
%   \end{Version}
%   \begin{Version}{2007/10/21 v0.9}
%   \item
%     Driver detection for \XeTeX\ added.
%   \item
%     Fix for rotation angles $\le$~-360 or $\ge$~360.
%   \end{Version}
%   \begin{Version}{2008/08/11 v0.10}
%   \item
%     Code is not changed.
%   \item
%     URLs updated.
%   \end{Version}
%   \begin{Version}{2016/05/14 v0.11}
%   \item
%     LuaTeX Compatibility
%   \end{Version}
% \end{History}
%
% \PrintIndex
%
% \Finale
\endinput
|
% \end{quote}
% Do not forget to quote the argument according to the demands
% of your shell.
%
% \paragraph{Generating the documentation.}
% You can use both the \xfile{.dtx} or the \xfile{.drv} to generate
% the documentation. The process can be configured by the
% configuration file \xfile{ltxdoc.cfg}. For instance, put this
% line into this file, if you want to have A4 as paper format:
% \begin{quote}
%   \verb|\PassOptionsToClass{a4paper}{article}|
% \end{quote}
% An example follows how to generate the
% documentation with pdf\LaTeX:
% \begin{quote}
%\begin{verbatim}
%pdflatex pdflscape.dtx
%makeindex -s gind.ist pdflscape.idx
%pdflatex pdflscape.dtx
%makeindex -s gind.ist pdflscape.idx
%pdflatex pdflscape.dtx
%\end{verbatim}
% \end{quote}
%
% \begin{History}
%   \begin{Version}{2001/01/15 v0.1}
%   \item
%     First public version,
%     published in \xnewsgroup{de.comp.text.tex}:\\
%     \URL{``\link{Re: Querformat, pdfLaTex und thumbpdf}''}^^A
%     {https://groups.google.com/group/de.comp.text.tex/msg/e054c5795e52d2b8}
%   \end{Version}
%   \begin{Version}{2001/02/04 v0.2}
%   \item
%     Minor documentation update.
%   \item
%     CTAN.
%   \end{Version}
%   \begin{Version}{2004/05/11 v0.3}
%   \item
%     Support for dvipdfm added.
%   \end{Version}
%   \begin{Version}{2004/05/12 v0.4}
%   \item
%     Bug fix: support for multipage landscape environment
%     for all drivers except for \xoption{pdftex}.
%   \end{Version}
%   \begin{Version}{2006/02/20 v0.5}
%   \item
%     DTX framework.
%   \item
%     LPPL 1.3
%   \item
%     Code is not changed.
%   \end{Version}
%   \begin{Version}{2006/04/24 v0.6}
%   \item
%     Bug fix: \cs{PLS@RemoveRotate} reinserted for \cs{PLS@Rotate}.
%   \item
%     Fix for \cs{PLS@RemoveRotate} (dvips/dvipdfm): \cs{PLS@EverhShi}
%     hook is cleared.
%   \end{Version}
%   \begin{Version}{2007/04/11 v0.7}
%   \item
%     Line ends sanitized.
%   \end{Version}
%   \begin{Version}{2007/04/17 v0.8}
%   \item
%     Package \xpackage{atbegshi} replaces \xpackage{everyshi}.
%   \end{Version}
%   \begin{Version}{2007/10/21 v0.9}
%   \item
%     Driver detection for \XeTeX\ added.
%   \item
%     Fix for rotation angles $\le$~-360 or $\ge$~360.
%   \end{Version}
%   \begin{Version}{2008/08/11 v0.10}
%   \item
%     Code is not changed.
%   \item
%     URLs updated.
%   \end{Version}
%   \begin{Version}{2016/05/14 v0.11}
%   \item
%     LuaTeX Compatibility
%   \end{Version}
% \end{History}
%
% \PrintIndex
%
% \Finale
\endinput
|
% \end{quote}
% Do not forget to quote the argument according to the demands
% of your shell.
%
% \paragraph{Generating the documentation.}
% You can use both the \xfile{.dtx} or the \xfile{.drv} to generate
% the documentation. The process can be configured by the
% configuration file \xfile{ltxdoc.cfg}. For instance, put this
% line into this file, if you want to have A4 as paper format:
% \begin{quote}
%   \verb|\PassOptionsToClass{a4paper}{article}|
% \end{quote}
% An example follows how to generate the
% documentation with pdf\LaTeX:
% \begin{quote}
%\begin{verbatim}
%pdflatex pdflscape.dtx
%makeindex -s gind.ist pdflscape.idx
%pdflatex pdflscape.dtx
%makeindex -s gind.ist pdflscape.idx
%pdflatex pdflscape.dtx
%\end{verbatim}
% \end{quote}
%
% \begin{History}
%   \begin{Version}{2001/01/15 v0.1}
%   \item
%     First public version,
%     published in \xnewsgroup{de.comp.text.tex}:\\
%     \URL{``\link{Re: Querformat, pdfLaTex und thumbpdf}''}^^A
%     {https://groups.google.com/group/de.comp.text.tex/msg/e054c5795e52d2b8}
%   \end{Version}
%   \begin{Version}{2001/02/04 v0.2}
%   \item
%     Minor documentation update.
%   \item
%     CTAN.
%   \end{Version}
%   \begin{Version}{2004/05/11 v0.3}
%   \item
%     Support for dvipdfm added.
%   \end{Version}
%   \begin{Version}{2004/05/12 v0.4}
%   \item
%     Bug fix: support for multipage landscape environment
%     for all drivers except for \xoption{pdftex}.
%   \end{Version}
%   \begin{Version}{2006/02/20 v0.5}
%   \item
%     DTX framework.
%   \item
%     LPPL 1.3
%   \item
%     Code is not changed.
%   \end{Version}
%   \begin{Version}{2006/04/24 v0.6}
%   \item
%     Bug fix: \cs{PLS@RemoveRotate} reinserted for \cs{PLS@Rotate}.
%   \item
%     Fix for \cs{PLS@RemoveRotate} (dvips/dvipdfm): \cs{PLS@EverhShi}
%     hook is cleared.
%   \end{Version}
%   \begin{Version}{2007/04/11 v0.7}
%   \item
%     Line ends sanitized.
%   \end{Version}
%   \begin{Version}{2007/04/17 v0.8}
%   \item
%     Package \xpackage{atbegshi} replaces \xpackage{everyshi}.
%   \end{Version}
%   \begin{Version}{2007/10/21 v0.9}
%   \item
%     Driver detection for \XeTeX\ added.
%   \item
%     Fix for rotation angles $\le$~-360 or $\ge$~360.
%   \end{Version}
%   \begin{Version}{2008/08/11 v0.10}
%   \item
%     Code is not changed.
%   \item
%     URLs updated.
%   \end{Version}
%   \begin{Version}{2016/05/14 v0.11}
%   \item
%     LuaTeX Compatibility
%   \end{Version}
% \end{History}
%
% \PrintIndex
%
% \Finale
\endinput
|
% \end{quote}
% Do not forget to quote the argument according to the demands
% of your shell.
%
% \paragraph{Generating the documentation.}
% You can use both the \xfile{.dtx} or the \xfile{.drv} to generate
% the documentation. The process can be configured by the
% configuration file \xfile{ltxdoc.cfg}. For instance, put this
% line into this file, if you want to have A4 as paper format:
% \begin{quote}
%   \verb|\PassOptionsToClass{a4paper}{article}|
% \end{quote}
% An example follows how to generate the
% documentation with pdf\LaTeX:
% \begin{quote}
%\begin{verbatim}
%pdflatex pdflscape.dtx
%makeindex -s gind.ist pdflscape.idx
%pdflatex pdflscape.dtx
%makeindex -s gind.ist pdflscape.idx
%pdflatex pdflscape.dtx
%\end{verbatim}
% \end{quote}
%
% \begin{History}
%   \begin{Version}{2001/01/15 v0.1}
%   \item
%     First public version,
%     published in \xnewsgroup{de.comp.text.tex}:\\
%     \URL{``\link{Re: Querformat, pdfLaTex und thumbpdf}''}^^A
%     {https://groups.google.com/group/de.comp.text.tex/msg/e054c5795e52d2b8}
%   \end{Version}
%   \begin{Version}{2001/02/04 v0.2}
%   \item
%     Minor documentation update.
%   \item
%     CTAN.
%   \end{Version}
%   \begin{Version}{2004/05/11 v0.3}
%   \item
%     Support for dvipdfm added.
%   \end{Version}
%   \begin{Version}{2004/05/12 v0.4}
%   \item
%     Bug fix: support for multipage landscape environment
%     for all drivers except for \xoption{pdftex}.
%   \end{Version}
%   \begin{Version}{2006/02/20 v0.5}
%   \item
%     DTX framework.
%   \item
%     LPPL 1.3
%   \item
%     Code is not changed.
%   \end{Version}
%   \begin{Version}{2006/04/24 v0.6}
%   \item
%     Bug fix: \cs{PLS@RemoveRotate} reinserted for \cs{PLS@Rotate}.
%   \item
%     Fix for \cs{PLS@RemoveRotate} (dvips/dvipdfm): \cs{PLS@EverhShi}
%     hook is cleared.
%   \end{Version}
%   \begin{Version}{2007/04/11 v0.7}
%   \item
%     Line ends sanitized.
%   \end{Version}
%   \begin{Version}{2007/04/17 v0.8}
%   \item
%     Package \xpackage{atbegshi} replaces \xpackage{everyshi}.
%   \end{Version}
%   \begin{Version}{2007/10/21 v0.9}
%   \item
%     Driver detection for \XeTeX\ added.
%   \item
%     Fix for rotation angles $\le$~-360 or $\ge$~360.
%   \end{Version}
%   \begin{Version}{2008/08/11 v0.10}
%   \item
%     Code is not changed.
%   \item
%     URLs updated.
%   \end{Version}
%   \begin{Version}{2016/05/14 v0.11}
%   \item
%     LuaTeX Compatibility
%   \end{Version}
% \end{History}
%
% \PrintIndex
%
% \Finale
\endinput
