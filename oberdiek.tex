\NeedsTeXFormat{LaTeX2e}

\newcommand*{\Title}{Overview}
\newcommand*{\CTANdir}{macros/latex/contrib/oberdiek/}
\newcommand*{\CTANroot}{http://mirror.ctan.org/}
\newcommand*{\Subject}{CTAN:\CTANdir}
\newcommand*{\Author}{Heiko Oberdiek}

\newcommand*{\Email}{ho-tex at tug.org}
\newcommand*{\Date}{2019/11/24}

% Copyright (C) 2006-2016 by
%    Heiko Oberdiek
% Copyright (C) 2016-2019 by
%    Oberdiek Package Support Group
%
% This work may be distributed and/or modified under the
% conditions of the LaTeX Project Public License, either
% version 1.3 of this license or (at your option) any later
% version. The latest version of this license is in
%    http://www.latex-project.org/lppl.txt
% and version 1.3 or later is part of all distributions of
% LaTeX version 2003/12/01 or later.
%
% This work has the LPPL maintenance status "maintained".
%
% This Current Maintainer of this work is the Oberdiek Package Support Group.
%
% This work consists of the overview "oberdiek.pdf", its source
% "oberdiek.tex", and the installation script "oberdiek.ins"
% for the projects in CTAN:macros/latex/contrib/oberdiek/.
%
\documentclass[a4paper,12pt]{article}

\usepackage{iftex}
\ifluatex
  \usepackage{fontspec}[2011/09/18]%
  \usepackage{unicode-math}[2011/09/19]%
  \setmathfont{latinmodern-math.otf}%
\fi

\usepackage[
  ignorehead,
  top=1in,
]{geometry}
\usepackage{longtable}
\usepackage{color}
\usepackage[ngerman,english]{babel}
\usepackage{hologo}
\usepackage{biblatex}% for internals in .toc files

\definecolor{link}{rgb}{1,0,0}% red
\definecolor{file}{rgb}{0,0,1}% blue
\definecolor{url}{cmyk}{0.1,1,0,0.1}

\definecolor{file}{rgb}{1,0,0}% red
\definecolor{url}{rgb}{0,0,1}% blue
\definecolor{link}{rgb}{0,0.75,0}%

\usepackage[
  colorlinks,
]{hyperref}[2006/02/12]
\hypersetup{
  pdftitle={CTAN:\CTANdir},
  pdfsubject={Package Overview},
  pdfauthor={\Author\ <\Email>},
  bookmarksnumbered,
  bookmarksopen,
  bookmarksopenlevel=2,
  bookmarksdepth=2,
  filecolor=file,
  urlcolor=url,
  linkcolor=link,
}
\usepackage{bookmark}
\usepackage{hypdestopt}
\setcounter{tocdepth}{1}
\setcounter{secnumdepth}{1}

\title{%
  \href{\CTANroot\CTANdir}{CTAN:\CTANdir}%
}
\author{%
  \Author\thanks
 {Please report any issues at \url{https://github.com/ho-tex/oberdiek/issues}}\\
  \textless
  \href{mailto:\Email}{\texttt{\Email}}%
  \textgreater
}
\date{\Date}

\providecommand*{\pdfTeX}{pdf\TeX}
\providecommand*{\plainTeX}{\mbox{plain-\TeX}}
\providecommand*{\iniTeX}{\mbox{ini-\TeX}}
\providecommand*{\VTeX}{V\TeX}
\providecommand*{\eTeX}{$\csname m@th\endcsname\varepsilon$-\TeX}
\providecommand*{\LuaTeX}{%
  L\textsc{ua}\TeX
}
\newcommand*{\xpackage}[1]{\textsf{#1}}
\newcommand*{\xmodule}[1]{\textsf{#1}}
\newcommand*{\xfile}[1]{\texttt{#1}}
\newcommand*{\xext}[1]{\texttt{.#1}}
\newcommand*{\xoption}[1]{\textsf{#1}}
\newcommand*{\cs}[1]{\texttt{\textbackslash#1}}
\newcommand*{\meta}[1]{%
  \ensuremath\langle
  \textit{#1}%
  \ensuremath\rangle
}

\makeatletter
\g@addto@macro\abstract{\noindent\ignorespaces}

\newcommand*{\tocinclude}[1]{%
  \setcounter{tocdepth}{3}%
  \begingroup
    \makeatletter
    \def\@prj{#1}%
    \let\contentsline\foreign@contentsline
    \input{\@prj.toc}%
  \endgroup
}
\def\foreign@contentsline#1#2#3#4{%
  \ifx\\#4\\%
    \csname l@#1\endcsname{#2}{#3}%
  \else
    \ifHy@linktocpage
      \csname l@#1\endcsname{{#2}}{%
        \hyper@linkfile{#3}{\@prj.pdf}{#4}%
      }%
    \else
      \csname l@#1\endcsname{%
        \hyper@linkfile{#2}{\@prj.pdf}{#4}%
      }{#3}%
    \fi
  \fi
}%

\newenvironment{overview}{%
  \setlength{\tabcolsep}{0.8\tabcolsep}%
  \setlength{\LTleft}{0pt}%
  \longtable{@{}llll@{}}
}{%
  \endlongtable
}
\newcommand*{\entry}[4]{%
  \href{file:#1.pdf}{%
    \bfseries\xpackage{#1}%
  }%
  & #2%
  & v#3%
  & \href{\CTANroot\CTANdir #1.pdf}{[pdf]} %
    \href{\CTANroot\CTANdir #1.dtx}{[dtx]}
  \\*%
  \hyperref[{#1}]{\small (contents)}%
  &
  \multicolumn{2}{l}{%
    #4%
  }%
  \\%
}
\newcommand*{\entrysep}{1.5ex}

\newcommand*{\pkgsectformat}[1]{%
  \texorpdfstring{%
    \textcolor{link}{The} %
    \xpackage{#1} %
    \textcolor{link}{package}%
  }{#1}%
}

\makeatother

\begin{document}
\maketitle

\section{Overview}
\begin{overview}
\entry{accsupp}{2010/01/16}{0.3}{Accessibility support by marked content}%
[\entrysep]
\entry{aliascnt}{2009/09/08}{1.3}{Alias counters}%
[\entrysep]
\entry{alphalph}{2011/05/13}{2.4}{Convert numbers to letters}%
[\entrysep]
\entry{askinclude}{2011/12/02}{2.2}{Interactive asking of included files}%
[\entrysep]
\entry{atbegshi}{2011/10/05}{1.16}{At begin shipout hook}%
[\entrysep]
\entry{atenddvi}{2007/04/17}{1.1}{At end DVI hook}%
[\entrysep]
\entry{attachfile2}{2019/07/24}{2.9}{Attach files into PDF}%
[\entrysep]
\entry{atveryend}{2011/06/30}{1.8}{Hooks at the very end of document}%
[\entrysep]
\entry{auxhook}{2011/03/04}{1.3}{Hooks for auxiliary files}%
[\entrysep]
\entry{bigintcalc}{2012/04/08}{1.3}{Expandable calculations on big integers}%
[\entrysep]
\entry{bitset}{2011/01/30}{1.1}{Handle bit-vector datatype}%
[\entrysep]
\entry{bmpsize}{2009/09/04}{1.6}{Extract size/resolution from bitmap files}%
[\entrysep]
\entry{bookmark}{2011/12/02}{1.24}{PDF bookmarks}%
[\entrysep]
\entry{catchfile}{2011/03/01}{1.6}{Catch the contents of a file}%
[\entrysep]
\entry{centernot}{2011/07/11}{1.3}{Centers the not symbol horizontally}%
[\entrysep]
\entry{chemarr}{2006/02/20}{1.2}{Arrows for chemical reactions}%
[\entrysep]
\entry{classlist}{2011/10/17}{1.4}{Record classes used in a document}%
[\entrysep]
\entry{colonequals}{2006/08/01}{1.0}{Colon equals symbols}%
[\entrysep]
\entry{dvipscol}{2008/08/11}{1.2}{Alter the usage of the dvips color stack}%
[\entrysep]
\entry{embedfile}{2011/04/13}{2.6}{Embed files into PDF}%
[\entrysep]
\entry{engord}{2010/03/01}{1.8}{Provides English ordinal numbers}%
[\entrysep]
\entry{enparen}{2012/01/07}{1.0}{Parentheses nesting}%
[\entrysep]
\entry{eolgrab}{2011/01/12}{1.0}{Catch arguments delimited by end of line}%
[\entrysep]
\entry{etexcmds}{2011/02/16}{1.5}{Avoid name clashes with \hologo{eTeX} commands}%
[\entrysep]
\entry{fibnum}{2012/04/08}{1.0}{Fibonacci numbers}%
[\entrysep]
\entry{flags}{2007/09/30}{0.4}{Setting/clearing of flags in bit fields}%
[\entrysep]
\entry{gettitlestring}{2010/12/03}{1.4}{Cleanup title references}%
[\entrysep]
\entry{grfext}{2010/08/19}{1.1}{Manage graphics extensions}%
[\entrysep]
\entry{hologo}{2012/04/26}{1.10}{A logo collection with bookmark support}%
[\entrysep]
\entry{holtxdoc}{2012/03/21}{0.24}{Private additional ltxdoc support}%
[\entrysep]
\entry{hopatch}{2011/06/24}{1.1}{Wrapper for package hooks}%
[\entrysep]
\entry{hycolor}{2011/01/30}{1.7}{Color options for hyperref/bookmark}%
[\entrysep]
\entry{hypbmsec}{2007/04/11}{2.4}{Bookmarks in sectioning commands}%
[\entrysep]
\entry{hypcap}{2011/02/16}{1.11}{Adjusting the anchors of captions}%
[\entrysep]
\entry{hypdestopt}{2011/05/13}{2.3}{Hyperref destination optimizer}%
[\entrysep]
\entry{hypdoc}{2011/08/19}{1.11}{Hyper extensions for doc.sty}%
[\entrysep]
\entry{hypgotoe}{2007/10/30}{0.1}{Links to embedded files}%
[\entrysep]
\entry{hyphsubst}{2008/06/09}{0.2}{Substitute hyphenation patterns}%
[\entrysep]
\entry{ifdraft}{2008/08/11}{1.3}{Detect class options draft and final}%
[\entrysep]
\entry{iflang}{2007/11/11}{1.5}{Checks for the current language}%
[\entrysep]
% iftex distribution \entry{ifluatex}{2010/03/01}{1.3}{Provides the ifluatex switch}%
% iftex distribution [\entrysep]
% iftex distribution \entry{ifpdf}{2011/01/30}{2.3}{Provides the ifpdf switch}%
% iftex distribution [\entrysep]
% iftex distribution \entry{ifvtex}{2010/03/01}{1.5}{Detect \hologo{VTeX} and its facilities}%
% iftex distribution [\entrysep]
\entry{infwarerr}{2010/04/08}{1.3}{Providing info/warning/error messages}%
[\entrysep]
\entry{inputenx}{2011/05/27}{1.10}{Enhanced input encoding handling}%
[\entrysep]
\entry{intcalc}{2007/09/27}{1.1}{Expandable calculations with integers}%
[\entrysep]
\entry{kvdefinekeys}{2011/04/07}{1.3}{Define keys}%
[\entrysep]
\entry{kvoptions}{2011/06/30}{3.11}{Key value format for package options}%
[\entrysep]
\entry{kvsetkeys}{2012/04/25}{1.16}{Key value parser}%
[\entrysep]
\entry{letltxmacro}{2010/09/02}{1.4}{Let assignment for \hologo{LaTeX} macros}%
[\entrysep]
\entry{listingsutf8}{2011/11/10}{1.2}{Allow UTF-8 in listings input}%
[\entrysep]
\entry{ltxcmds}{2011/11/09}{1.22}{\hologo{LaTeX} kernel commands for general use}%
[\entrysep]
\entry{luacolor}{2019/07/25}{1.12}{Color support via \hologo{LuaTeX}'s attributes}%
[\entrysep]
\entry{magicnum}{2019/07/25}{1.6}{Magic numbers}%
[\entrysep]
\entry{mleftright}{2010/09/25}{1.0}{Math left/right delim.\@ as open/close}%
[\entrysep]
\entry{pagegrid}{2009/12/04}{1.4}{Print page grid in background}%
[\entrysep]
\entry{pagesel}{2008/08/11}{1.8}{Select pages of a document for output}%
[\entrysep]
\entry{pdfcol}{2007/12/12}{1.2}{Handle new color stacks for \hologo{pdfTeX}}%
[\entrysep]
\entry{pdfcolfoot}{2012/01/02}{1.2}{Color stack for footnotes with \hologo{pdfTeX}}%
[\entrysep]
\entry{pdfcolparallel}{2010/01/11}{1.3}{Color stacks support for parallel}%
[\entrysep]
\entry{pdfcolparcolumns}{2010/01/11}{1.3}{Color stacks for parcolumns}%
[\entrysep]
\entry{pdfcrypt}{2007/04/26}{1.0}{Allows the setting of PDF encryption}%
[\entrysep]
\entry{pdfescape}{2011/11/25}{1.13}{Implements \hologo{pdfTeX}'s escape features}%
[\entrysep]
\entry{pdflscape}{2008/08/11}{0.10}{Display of landscape pages in PDF}%
[\entrysep]
\entry{pdfrender}{2010/01/28}{1.2}{Access to some PDF graphics parameters}%
[\entrysep]
\entry{pdftexcmds}{2019/07/25}{0.30}{Utility functions of \hologo{pdfTeX} for \hologo{LuaTeX}}%
[\entrysep]
\entry{picture}{2009/10/11}{1.3}{Dimens for picture macros}%
[\entrysep]
\entry{pmboxdraw}{2011/03/24}{1.1}{Poor man's box drawing characters}%
[\entrysep]
\entry{protecteddef}{2011/01/31}{1.0}{Define protected commands}%
[\entrysep]
\entry{refcount}{2011/10/16}{3.4}{Data extraction from label references}%
[\entrysep]
\entry{rerunfilecheck}{2011/04/15}{1.7}{Rerun checks for auxiliary files}%
[\entrysep]
\entry{resizegather}{2010/03/01}{1.2}{Resize overly large equations}%
[\entrysep]
\entry{rotchiffre}{2010/11/12}{1.0}{Perform simple rotation ciphers}%
[\entrysep]
\entry{scrindex}{2008/08/11}{1.1}{Package index with \hologo{KOMAScript} classes}%
[\entrysep]
\entry{selinput}{2007/09/09}{1.2}{Semi-automatic input encoding detection}%
[\entrysep]
\entry{setouterhbox}{2007/09/09}{1.7}{Set hbox in outer horizontal mode}%
[\entrysep]
\entry{settobox}{2008/08/11}{1.4}{Assign box dimensions to length registers}%
[\entrysep]
\entry{soulutf8}{2007/09/09}{1.0}{Permit use of UTF-8 characters in soul}%
[\entrysep]
\entry{stackrel}{2007/11/11}{1.2}{Adding subscript option to stackrel}%
[\entrysep]
\entry{stampinclude}{2008/07/14}{1.0}{Include files based on time stamps}%
[\entrysep]
\entry{stringenc}{2011/12/02}{1.10}{Convert strings between diff.\@ encodings}%
[\entrysep]
\entry{tabularht}{2007/04/11}{2.5}{Tabular with height specified}%
[\entrysep]
\entry{tabularkv}{2006/02/20}{1.1}{Tabular with key value interface}%
[\entrysep]
\entry{telprint}{2008/08/11}{1.10}{Format German phone numbers}%
[\entrysep]
\entry{thepdfnumber}{2011/11/24}{1.0}{Print PDF numbers with minimal digits}%
[\entrysep]
\entry{transparent}{2007/01/08}{1.0}{Transparency via \hologo{pdfTeX}'s color stack}%
[\entrysep]
\entry{twoopt}{2008/08/11}{1.5}{Definitions with two optional arguments}%
[\entrysep]
\entry{uniquecounter}{2011/01/30}{1.2}{Provide unlimited unique counter}%
[\entrysep]
\entry{zref}{2012/04/04}{2.24}{A new reference scheme for \hologo{LaTeX}}%
\end{overview}

\section{Packages}
\hypersetup{bookmarksnumbered=false}

\subsection{\pkgsectformat{accsupp}}
\label{accsupp}
\begin{abstract}
Since PDF 1.5 portions of a page can be marked for better
accessibility support.
For example, replacement texts or expansions of abbreviations can be
provided. Package \xpackage{accsupp} starts with providing a minimal
low-level interface for programmers. Status is experimental.
\end{abstract}
\tocinclude{accsupp}

\newpage
\subsection{\pkgsectformat{aliascnt}}
\label{aliascnt}
\begin{abstract}
Package \xpackage{aliascnt} introduces \emph{alias counters} that
share the same counter register and clear list.
\end{abstract}
\tocinclude{aliascnt}

\newpage
\subsection{\pkgsectformat{alphalph}}
\label{alphalph}
\begin{abstract}
The package provides methods to represent numbers with a limited
set of symbols. Both \hologo{LaTeX} and \hologo{plainTeX} are supported.
\end{abstract}
\tocinclude{alphalph}

\newpage
\subsection{\pkgsectformat{askinclude}}
\label{askinclude}
\begin{abstract}
This package replaces \cs{includeonly} by an interactive user
interface.
\end{abstract}
\tocinclude{askinclude}

\newpage
\subsection{\pkgsectformat{atbegshi}}
\label{atbegshi}
\begin{abstract}
This package is a modern reimplementation of package \xpackage{everyshi}
without the burden of compatibility. It makes use of \eTeX's if available.
Both \LaTeX\ and \plainTeX\ are supported.
\end{abstract}
\tocinclude{atbegshi}

\newpage
\subsection{\pkgsectformat{atenddvi}}
\label{atenddvi}
\begin{abstract}
\LaTeX\ offers \cs{AtBeginDvi}. This package \xpackage{atenddvi}
provides the counterpart \cs{AtEndDvi}. The execution of its
argument is delayed to the end of the document at the end of the
last page. Thus \cs{special} and \cs{write} remain effective, because
they are put into the last page. This is the main difference
to \cs{AtEndDocument}.
\end{abstract}
\tocinclude{atenddvi}

\newpage
\subsection{\pkgsectformat{attachfile2}}
\label{attachfile2}
\begin{abstract}
This package can be used to attach files to a PDF document.
It is a further development of Scott Pakin's package
\xpackage{attachfile} for \pdfTeX. Apart from bug fixes,
package \xpackage{attachfile2} adds support for \xoption{dvips},
some new options, gets and writes meta information data about
the attached files.
\end{abstract}
\tocinclude{attachfile2}

\newpage
\subsection{\pkgsectformat{atveryend}}
\label{atveryend}
\begin{abstract}
This \LaTeX\ package provides two hooks for \verb|\end{document}|
that are executed after the hook of \cs{AtEndDocument}.
\cs{AfterLastShipout} can be used for code that is to be executed
right after the last \cs{clearpage} before the \xext{aux} file
is closed. \cs{AtVeryEndDocument} is used for code after closing
and final reading of the \xext{aux} file.
\end{abstract}
\tocinclude{atveryend}

\newpage
\subsection{\pkgsectformat{auxhook}}
\label{auxhook}
\begin{abstract}
Package \xpackage{auxhook} provides hooks for adding stuff at
the begin of \xfile{.aux} files.
\end{abstract}
\tocinclude{auxhook}

\newpage
\subsection{\pkgsectformat{bigintcalc}}
\label{bigintcalc}
\begin{abstract}
This package provides expandable arithmetic operations
with big integers that can exceed \TeX's number limits.
\end{abstract}
\tocinclude{bigintcalc}

\newpage
\subsection{\pkgsectformat{bitset}}
\label{bitset}
\begin{abstract}
This package defines and implements the data type bit set,
a vector of bits. The size of the vector may grow dynamically.
Individual bits can be manipulated.
\end{abstract}
\tocinclude{bitset}

\newpage
\subsection{\pkgsectformat{bmpsize}}
\label{bmpsize}
\begin{abstract}
Package \xpackage{bmpsize} analyzes bitmap images to extract
size and resolution data. It adds this feature to the graphics package
that now do not need separate bounding box files for bitmap images.
Additionally the implementation for the inclusion of bitmap images
in some drivers of package \xpackage{graphicx} are rewritten to support
options \xoption{viewport}, \xoption{trim} and \xoption{clip}.
\end{abstract}
\tocinclude{bmpsize}

\newpage
\subsection{\pkgsectformat{bookmark}}
\label{bookmark}
\begin{abstract}
This package implements a new bookmark (outline) organization for
package \xpackage{hyperref}. Bookmark properties such
as style and color can now be set. Other action types
are available (URI, GoToR, Named). The bookmarks are
generated in the first compile run. Package \xpackage{hyperref}
uses two runs.
\end{abstract}
\tocinclude{bookmark}

\newpage
\subsection{\pkgsectformat{catchfile}}
\label{catchfile}
\begin{abstract}
This package catches the contents of a file and puts it in a macro.
It requires \eTeX. Both \LaTeX\ and \plainTeX\ are supported.
\end{abstract}
\tocinclude{catchfile}

\newpage
\subsection{\pkgsectformat{centernot}}
\label{centernot}
\begin{abstract}
This package provides \cs{centernot} that prints the symbol
\cs{not} on the following argument. Unlike \cs{not} the symbol
is horizontally centered.
\end{abstract}
\tocinclude{centernot}

\newpage
\subsection{\pkgsectformat{chemarr}}
\label{chemarr}
\begin{abstract}
Very often chemists need a longer version
of reaction arrows (\cs{rightleftharpoons}) with
the possibility to put text above and below.
Analogous to \xpackage{amsmath}'s \cs{xrightarrow} and
\cs{xleftarrow} this package provides the macro
\cs{xrightleftharpoons}.
\end{abstract}
\tocinclude{chemarr}

\newpage
\subsection{\pkgsectformat{classlist}}
\label{classlist}
\begin{abstract}
This package records the loaded classes and stores
them in a list.
\end{abstract}
\tocinclude{classlist}

\newpage
\subsection{\pkgsectformat{colonequals}}
\label{colonequals}
\begin{abstract}
Package \xpackage{colonequals} defines poor man's symbols
for math relation symbols such as ``colon equals''.
The colon is centered around the horizontal math axis.
\end{abstract}
\tocinclude{colonequals}

\newpage
\subsection{\pkgsectformat{dvipscol}}
\label{dvipscol}
\begin{abstract}
Color support for dvips in \xfile{dvips.def} involves the
color stack of dvips. The package tries to remove unnecessary
uses of the stack to avoid the error ``out of coor stack space''.
\end{abstract}
\tocinclude{dvipscol}

\newpage
\subsection{\pkgsectformat{embedfile}}
\label{embedfile}
\begin{abstract}
This package embeds files to a PDF document.
Currently the only supported driver is \pdfTeX\ $>=$ 1.30 in PDF mode.
\end{abstract}
\tocinclude{embedfile}

\newpage
\subsection{\pkgsectformat{engord}}
\label{engord}
\begin{abstract}
The package generates the suffix of English ordinal numbers.
It can be used with plain and \LaTeX\ formats.
\end{abstract}
\tocinclude{engord}

\newpage
\subsection{\pkgsectformat{enparen}}
\label{enparen}
\begin{abstract}
The package defines macros to set parentheses that automatically
change the symbols from inner to outer fences.
\end{abstract}
\tocinclude{enparen}

\newpage
\subsection{\pkgsectformat{eolgrab}}
\label{eolgrab}
\begin{abstract}
This package implements a generic argument grabber
to catch an argument that is delimited by the line end.
\end{abstract}
\tocinclude{eolgrab}

\newpage
\subsection{\pkgsectformat{etexcmds}}
\label{etexcmds}
\begin{abstract}
New primitive commands are introduced in \eTeX. Sometimes the
names collide with existing macros. This package solves this
name clashes by adding a prefix to \eTeX's commands. For example,
\eTeX's \cs{unexpanded} is provided as \cs{etex@unexpanded}.
\end{abstract}
\tocinclude{etexcmds}

\newpage
\subsection{\pkgsectformat{fibnum}}
\label{fibnum}
\begin{abstract}
The package \xpackage{fibnum} provides expandable fibonacci
numbers for both \hologo{LaTeX} and \hologo{plainTeX}.
\end{abstract}
\tocinclude{fibnum}

\newpage
\subsection{\pkgsectformat{flags}}
\label{flags}
\begin{abstract}
Package \xpackage{flags} allows the setting and clearing
of flags in bit fields and converts the bit field into a
decimal number. Currently the bit field is limited to 31 bits.
\end{abstract}
\tocinclude{flags}

\newpage
\subsection{\pkgsectformat{gettitlestring}}
\label{gettitlestring}
\begin{abstract}
The \LaTeX\ package addresses packages that are dealing with
references to titles (\cs{section}, \cs{caption}, \dots).
The package tries to remove \cs{label} and other
commands from title strings.
\end{abstract}
\tocinclude{gettitlestring}

\newpage
\subsection{\pkgsectformat{grfext}}
\label{grfext}
\begin{abstract}
This package provides macros for adding and reordering
graphics extensions of package \xpackage{graphics}.
\end{abstract}
\tocinclude{grfext}


\newpage
\subsection{\pkgsectformat{hologo}}
\label{hologo}
\begin{abstract}
This package starts a collection of logos with support for bookmarks
strings.
\end{abstract}
\tocinclude{hologo}

\newpage
\subsection{\pkgsectformat{holtxdoc}}
\label{holtxdoc}
\begin{abstract}
The package is used for the documentation of my packages in
DTX format. It contains some private macros and setup for
my needs. Thus do not use it. I have separated the part
that may be useful for others in package \xpackage{hypdoc}.
\end{abstract}
\tocinclude{holtxdoc}

\newpage
\subsection{\pkgsectformat{hopatch}}
\label{hopatch}
\begin{abstract}
This packages provides a wrapper to various package hooks
provided by other packages or classes, but does not define
own hooks.
\end{abstract}
\tocinclude{hopatch}

\newpage
\subsection{\pkgsectformat{hycolor}}
\label{hycolor}
\begin{abstract}
Package \xpackage{hycolor} implements the color option stuff that
is used by packages \xpackage{hyperref} and \xpackage{bookmark}.
It is not intended as package for the user.
\end{abstract}
\tocinclude{hycolor}

\newpage
\subsection{\pkgsectformat{hypbmsec}}
\label{hypbmsec}
\begin{abstract}
This package expands the syntax of the sectioning commands. If the
argument of the sectioning commands isn't usable as outline entry,
a replacement for the bookmarks can be given.
\end{abstract}
\tocinclude{hypbmsec}

\newpage
\subsection{\pkgsectformat{hypcap}}
\label{hypcap}
\begin{abstract}
This package tries a solution of the problem with
hyperref, that links to floats points below the
caption and not at the beginning of the float.
Therefore this package divides the task into two
part, the link setting with \cs{capstart} or
automatically at the beginning of a float and
the rest in the \cs{caption} command.
\end{abstract}
\tocinclude{hypcap}

\newpage
\subsection{\pkgsectformat{hypdestopt}}
\label{hypdestopt}
\begin{abstract}
Package \xpackage{hypdestopt} supports \xpackage{hyperref}'s
\xoption{pdftex} driver. It removes unnecessary destinations
and shortens the destination names or uses numbered destinations
to get smaller PDF files.
\end{abstract}
\tocinclude{hypdestopt}

\newpage
\subsection{\pkgsectformat{hypdoc}}
\label{hypdoc}
\begin{abstract}
This package adds hyper features to the package
\xpackage{doc} that is used in the documentation
system of \LaTeXe. Bookmarks are added and references
are linked as far as possible.
\end{abstract}
\tocinclude{hypdoc}

\newpage
\subsection{\pkgsectformat{hypgotoe}}
\label{hypgotoe}
\begin{abstract}
Experimental package for links to embedded files.
\end{abstract}
\tocinclude{hypgotoe}

\newpage
\subsection{\pkgsectformat{hyphsubst}}
\label{hyphsubst}
\begin{abstract}
A \TeX\ format file may include alternative hyphenation patterns
for a language with a different name. If the naming convention
follows \xpackage{babel's} rules, then the hyphenation patterns
for a language can be replaced by the alternative hyphenation patterns,
provided in the format file.
\end{abstract}
\tocinclude{hyphsubst}

\newpage
\subsection{\pkgsectformat{ifdraft}}
\label{ifdraft}
\begin{abstract}
The package provides an interface for selecting code depending
on the options \xoption{draft} and \xoption{final}.
\end{abstract}
\tocinclude{ifdraft}

\newpage
\subsection{\pkgsectformat{iflang}}
\label{iflang}
\begin{abstract}
This package provides expandible checks for the current language
based on macro \cs{languagename} or hyphenation patterns.
\end{abstract}
\tocinclude{iflang}

\newpage
\subsection{\pkgsectformat{infwarerr}}
\label{infwarerr}
\begin{abstract}
This package provides a complete set of macros for informations,
warnings and error messages with support for \plainTeX.
\end{abstract}
\tocinclude{infwarerr}

\newpage
\subsection{\pkgsectformat{inputenx}}
\label{inputenx}
\begin{abstract}
This package provides input encodings using
standard mappings and covers nearly all slots. Thus it
serves as more uptodate replacement for package \xpackage{inputenc}.
\end{abstract}
\tocinclude{inputenx}

\newpage
\subsection{\pkgsectformat{intcalc}}
\label{intcalc}
\begin{abstract}
This package provides expandable arithmetic operations
with integers.
\end{abstract}
\tocinclude{intcalc}

\newpage
\subsection{\pkgsectformat{kvdefinekeys}}
\label{kvdefinekeys}
\begin{abstract}
Package \xpackage{kvdefinekeys} provides \cs{kv@define@key} to define
keys the same way as \xpackage{keyval}'s \cs{define@key}. However, it
works also using \iniTeX.
\end{abstract}
\tocinclude{kvdefinekeys}

\newpage
\subsection{\pkgsectformat{kvoptions}}
\label{kvoptions}
\begin{abstract}
This package is intended for package authors who want to
use options in key value format for their package options.
\end{abstract}
\tocinclude{kvoptions}

\newpage
\subsection{\pkgsectformat{kvsetkeys}}
\label{kvsetkeys}
\begin{abstract}
Package \xpackage{kvsetkeys} provides \cs{kvsetkeys}, a variant
of package \xpackage{keyval}'s \cs{setkeys}. It allows to specify
a handler that deals with unknown options. Active commas and equal
signs may be used (e.g. see \xpackage{babel}'s shorthands) and
only one level of curly braces is removed from the values.
\end{abstract}
\tocinclude{kvsetkeys}

\newpage
\subsection{\pkgsectformat{letltxmacro}}
\label{letltxmacro}
\begin{abstract}
\TeX's \cs{let} assignment does not work for \LaTeX\ macros
with optional arguments or for macros that are defined
as robust macros by \cs{DeclareRobustCommand}. This package
defines \cs{LetLtxMacro} that also takes care of the involved
internal macros.
\end{abstract}
\tocinclude{letltxmacro}

\newpage
\subsection{\pkgsectformat{listingsutf8}}
\label{listingsutf8}
\begin{abstract}
Package \xpackage{listings} does not support files with multi-byte
encodings such as UTF-8. In case of \cs{lstinputlisting} a simple
workaround is possible if an one-byte encoding exists that the file
can be converted to. Also \eTeX\ and \pdfTeX\ regardless of its mode
are required.
\end{abstract}
\tocinclude{listingsutf8}

\newpage
\subsection{\pkgsectformat{ltxcmds}}
\label{ltxcmds}
\begin{abstract}
The package \xpackage{ltxcmds} exports some utility macros
from the \LaTeX\ kernel into a separate namespace and
also provides them for other formats such as plain-\TeX.
\end{abstract}
\tocinclude{ltxcmds}

\newpage
\subsection{\pkgsectformat{luacolor}}
\label{luacolor}
\begin{abstract}
Package \xpackage{luacolor} implements color support based
on \LuaTeX's node attributes.
\end{abstract}
\tocinclude{luacolor}

\newpage
\subsection{\pkgsectformat{magicnum}}
\label{magicnum}
\begin{abstract}
This packages allows to access magic numbers by a hierarchical
name system.
\end{abstract}
\tocinclude{magicnum}

\newpage
\subsection{\pkgsectformat{mleftright}}
\label{mleftright}
\begin{abstract}
\TeX\ sets subformulas by \cs{left} and \cs{right} as inner formulas
with additional surrounding spaces in some situations. This package
provides \cs{mleft} and \cs{mright} that call \cs{left} and \cs{right},
but the delimiters will act as normal \cs{mathopen} and \cs{mathclose}
delimiters without the additional space of an inner formula.
\end{abstract}
\tocinclude{mleftright}

\newpage
\subsection{\pkgsectformat{pagegrid}}
\label{pagegrid}
\begin{abstract}
The \LaTeX\ package prints a page grid in the background.
\end{abstract}
\tocinclude{pagegrid}

\newpage
\subsection{\pkgsectformat{pagesel}}
\label{pagesel}
\begin{abstract}
Single pages or page areas can be selected for output.
\end{abstract}
\tocinclude{pagesel}

\newpage
\subsection{\pkgsectformat{pdfcol}}
\label{pdfcol}
\begin{abstract}
Since version 1.40 \pdfTeX\ supports color stacks.
The driver file \xfile{pdftex.def} for package \xpackage{color}
defines and uses a main color stack since version v0.04b.
Package \xpackage{pdfcol} is intended for package writers.
It defines macros for setting and maintaining new color stacks.
\end{abstract}
\tocinclude{pdfcol}

\newpage
\subsection{\pkgsectformat{pdfcolfoot}}
\label{pdfcolfoot}
\begin{abstract}
Since version 1.40 \pdfTeX\ supports several color stacks. This
package uses a separate color stack for footnotes that can break
across pages.
\end{abstract}
\tocinclude{pdfcolfoot}

\newpage
\subsection{\pkgsectformat{pdfcolparallel}}
\label{pdfcolparallel}
\begin{abstract}
This packages fixes bugs in \xpackage{parallel} and
improves color support by using several color stacks
that are provided by \pdfTeX\ since version 1.40.
\end{abstract}
\tocinclude{pdfcolparallel}

\newpage
\subsection{\pkgsectformat{pdfcolparcolumns}}
\label{pdfcolparcolumns}
\begin{abstract}
Since version 1.40 \pdfTeX\ supports several color stacks.
This package uses them to fix color problems in
package \xpackage{parcolumns}.
\end{abstract}
\tocinclude{pdfcolparcolumns}

\newpage
\subsection{\pkgsectformat{pdfcrypt}}
\label{pdfcrypt}
\begin{abstract}
This package supports the setting of pdf encryption options
for \VTeX\ and some older versions of \pdfTeX.
\end{abstract}
\tocinclude{pdfcrypt}

\newpage
\subsection{\pkgsectformat{pdfescape}}
\label{pdfescape}
\begin{abstract}
This package implements \pdfTeX's escape features
(\cs{pdfescapehex}, \cs{pdfunescapehex}, \cs{pdfescapename},
\cs{pdfescapestring}) using \TeX\ or \eTeX.
\end{abstract}
\tocinclude{pdfescape}

\newpage
\subsection{\pkgsectformat{pdflscape}}
\label{pdflscape}
\begin{abstract}
Package \xpackage{pdflscape} adds PDF support to the
environment \texttt{landscape} of package \xpackage{lscape}
by setting the PDF page attribute \texttt{/Rotate}.
\end{abstract}
\tocinclude{pdflscape}

\newpage
\subsection{\pkgsectformat{pdfrender}}
\label{pdfrender}
\begin{abstract}
The PDF format has some graphics parameter like
line width or text rendering mode. This package
provides an interface for setting these parameters.
\end{abstract}
\tocinclude{pdfrender}

\newpage
\subsection{\pkgsectformat{pdftexcmds}}
\label{pdftexcmds}
\begin{abstract}
\hologo{LuaTeX} provides most of the commands of \hologo{pdfTeX} 1.40. However
a number of utility functions are removed. This package tries to fill
the gap and implements some of the missing primitive using Lua.
\end{abstract}
\tocinclude{pdftexcmds}

\newpage
\subsection{\pkgsectformat{picture}}
\label{picture}
\begin{abstract}
There are macro and environment arguments that expect numbers
that will internally be multiplicated with \cs{unitlength}.
This package extends the syntax of these arguments that
dimens with calculation support can be added for these arguments.
\end{abstract}
\tocinclude{picture}

\newpage
\subsection{\pkgsectformat{pmboxdraw}}
\label{pmboxdraw}
\begin{abstract}
Package \xpackage{pmboxdraw} declares box drawings characters of
old code pages, e.g. cp437. It uses rules instead of using a font.
\end{abstract}
\tocinclude{pmboxdraw}

\newpage
\subsection{\pkgsectformat{protecteddef}}
\label{protecteddef}
\begin{abstract}
This packages provides \cs{ProtectedDef} for defining
robust macros for both \hologo{plainTeX} and \hologo{LaTeX}.
First \hologo{eTeX}'s \cs{protected} is tried, then
\hologo{LaTeX}'s \cs{DeclareRobustCommand} is used.
Otherwise the macro is not made robust.
\end{abstract}
\tocinclude{protecteddef}

\newpage
\subsection{\pkgsectformat{refcount}}
\label{refcount}
\begin{abstract}
References are not numbers, however they often store numerical
data such as section or page numbers. \cs{ref} or \cs{pageref}
cannot be used for counter assignments or calculations because
they are not expandable, generate warnings, or can even be links.
The package provides expandable macros to extract the data
from references. Packages \xpackage{hyperref}, \xpackage{nameref},
\xpackage{titleref}, and \xpackage{babel} are supported.
\end{abstract}
\tocinclude{refcount}

\newpage
\subsection{\pkgsectformat{rerunfilecheck}}
\label{rerunfilecheck}
\begin{abstract}
The package provides additional rerun warnings if some
auxiliary files have changed. It is based on MD5 checksum,
provided by \pdfTeX.
\end{abstract}
\tocinclude{rerunfilecheck}

\newpage
\subsection{\pkgsectformat{resizegather}}
\label{resizegather}
\begin{abstract}
Equations that are too large are resized to fit the available
space. The environment \textsf{gather} of package \xpackage{amsmath}
is supported. Also the environments \textsf{equation} and
\textsf{displaymath} are redefined using \textsf{gather}
and its starred version.
\end{abstract}
\tocinclude{resizegather}

\newpage
\subsection{\pkgsectformat{rotchiffre}}
\label{rotchiffre}
\begin{abstract}
This package implements chiffres ROT13 with its variants
ROT5, ROT18, and ROT47.
\end{abstract}
\tocinclude{rotchiffre}

\newpage
\subsection{\pkgsectformat{scrindex}}
\label{scrindex}
\begin{abstract}
This package redefines environment `theindex' of package \xpackage{index},
if a class from KOMA-Script is loaded. Also option \xoption{idxtotoc}
is supported. Index preambles can be given either by means of package
\xpackage{index} or via the interface provided by KOMA-Script.
\end{abstract}
\tocinclude{scrindex}

\newpage
\subsection{\pkgsectformat{selinput}}
\label{selinput}
\begin{abstract}
This package selects the input encoding by specifying between
input characters and their glyph names.
\end{abstract}
\tocinclude{selinput}

\newpage
\subsection{\pkgsectformat{setouterhbox}}
\label{setouterhbox}
\begin{abstract}
If math stuff is set in an \cs{hbox}, then TeX
performs some optimization and omits the implicite
penalties \cs{binoppenalty} and \cs{relpenalty}.
This packages tries to put stuff into an \cs{hbox}
without getting lost of those penalties.
\end{abstract}
\tocinclude{setouterhbox}

\newpage
\subsection{\pkgsectformat{settobox}}
\label{settobox}
\begin{abstract}
Commands are defined for getting box sizes similar
to \LaTeX's \cs{settowidth} commands.
\end{abstract}
\tocinclude{settobox}

\newpage
\subsection{\pkgsectformat{soulutf8}}
\label{soulutf8}
\begin{abstract}
This package extends package \xpackage{soul} and adds some support
for UTF-8. Namely the input encodings \xfile{utf8.def}
from package \xpackage{inputenc} and
package \xpackage{ucs}'s \xfile{utf8x.def} are supported.
\end{abstract}
\tocinclude{soulutf8}

\newpage
\subsection{\pkgsectformat{stackrel}}
\label{stackrel}
\begin{abstract}
This package adds an optional argument to \cs{stackrel} for
putting something below the relational symbol and defines
\cs{stackbin} for binary symbols.
\end{abstract}
\tocinclude{stackrel}

\newpage
\subsection{\pkgsectformat{stampinclude}}
\label{stampinclude}
\begin{abstract}
The package replaces \cs{includeonly} and selects the files for
\cs{include} by inspecting the time stamp of the \xext{aux} file.
The file is selected for inclusion if the \xext{aux} file does
not yet exist or is older than the corresponding \xext{tex} file.
\end{abstract}
\tocinclude{stampinclude}

\newpage
\subsection{\pkgsectformat{stringenc}}
\label{stringenc}
\begin{abstract}
This package provides \cs{StringEncodingConvert} for converting
a string between different encodings.
Both \LaTeX\ and \plainTeX\ are supported.
\end{abstract}
\tocinclude{stringenc}

\newpage
\subsection{\pkgsectformat{tabularht}}
\label{tabularht}
\begin{abstract}
This package defines some environments that adds
a height specification to tabular and array.
\end{abstract}
\tocinclude{tabularht}

\newpage
\subsection{\pkgsectformat{tabularkv}}
\label{tabularkv}
\begin{abstract}
This package adds a key value interface for tabular
by the new environment \texttt{tabularkv}. Thus the
\TeX\ source code looks better by named parameters,
especially if package \xpackage{tabularht} is used.
\end{abstract}
\tocinclude{tabularkv}

\newpage
\subsection{\pkgsectformat{telprint}}
\label{telprint}
\begin{abstract}
Package \xpackage{telprint} provides \cs{telprint} for formatting
German phone numbers.
\end{abstract}
\tocinclude{telprint}

\newpage
\subsection{\pkgsectformat{thepdfnumber}}
\label{thepdfnumber}
\begin{abstract}
The package converts real numbers to a minimal representation
that is stripped from leading or trailing zeros,
plus signs and decimal point if not necessary.
\end{abstract}
\tocinclude{thepdfnumber}

\newpage
\subsection{\pkgsectformat{transparent}}
\label{transparent}
\begin{abstract}
Since version 1.40 \pdfTeX\ supports several color stacks. This
package shows, how a separate color stack can be used for transparency,
a property besides color.
\end{abstract}
\tocinclude{transparent}

\newpage
\subsection{\pkgsectformat{twoopt}}
\label{twoopt}
\begin{abstract}
This package provides commands to define macros with two
optional arguments.
\end{abstract}
\tocinclude{twoopt}

\newpage
\subsection{\pkgsectformat{uniquecounter}}
\label{uniquecounter}
\begin{abstract}
This package provides a kind of counter that provides unique
number values. Several counters can be created by different names.
The numeric values are not limited.
\end{abstract}
\tocinclude{uniquecounter}

\newpage
\subsection{\pkgsectformat{zref}}
\label{zref}
\begin{abstract}
Package \xpackage{zref} tries to get rid of the restriction
in \hologo{LaTeX}'s reference system that only two properties are
supported. The package implements an extensible referencing
system, where properties are handled in a more flexible way.
It offers an interface for macro programmers for the access
to the system and some applications that uses the new
reference scheme.
\end{abstract}
\tocinclude{zref}
\end{document}
