% \iffalse meta-comment
%
% File: soulutf8.dtx
% Version: 2016/05/16 v1.1
% Info: Permit use of UTF-8 characters in soul
%
% Copyright (C)
%    2007 Heiko Oberdiek
%    2016-2019 Oberdiek Package Support Group
%    https://github.com/ho-tex/oberdiek/issues
%
% This work may be distributed and/or modified under the
% conditions of the LaTeX Project Public License, either
% version 1.3c of this license or (at your option) any later
% version. This version of this license is in
%    https://www.latex-project.org/lppl/lppl-1-3c.txt
% and the latest version of this license is in
%    https://www.latex-project.org/lppl.txt
% and version 1.3 or later is part of all distributions of
% LaTeX version 2005/12/01 or later.
%
% This work has the LPPL maintenance status "maintained".
%
% The Current Maintainers of this work are
% Heiko Oberdiek and the Oberdiek Package Support Group
% https://github.com/ho-tex/oberdiek/issues
%
% The Base Interpreter refers to any `TeX-Format',
% because some files are installed in TDS:tex/generic//.
%
% This work consists of the main source file soulutf8.dtx
% and the derived files
%    soulutf8.sty, soulutf8.pdf, soulutf8.ins, soulutf8.drv,
%    soulutf8-test1.tex, soulutf8-test2.tex, soulutf8-test3.tex,
%    soulutf8-test4.tex, soulutf8-test5.tex.
%
% Distribution:
%    CTAN:macros/latex/contrib/oberdiek/soulutf8.dtx
%    CTAN:macros/latex/contrib/oberdiek/soulutf8.pdf
%
% Unpacking:
%    (a) If soulutf8.ins is present:
%           tex soulutf8.ins
%    (b) Without soulutf8.ins:
%           tex soulutf8.dtx
%    (c) If you insist on using LaTeX
%           latex \let\install=y% \iffalse meta-comment
%
% File: soulutf8.dtx
% Version: 2016/05/16 v1.1
% Info: Permit use of UTF-8 characters in soul
%
% Copyright (C) 2007 by
%    Heiko Oberdiek <heiko.oberdiek at googlemail.com>
%    2016
%    https://github.com/ho-tex/oberdiek/issues
%
% This work may be distributed and/or modified under the
% conditions of the LaTeX Project Public License, either
% version 1.3c of this license or (at your option) any later
% version. This version of this license is in
%    http://www.latex-project.org/lppl/lppl-1-3c.txt
% and the latest version of this license is in
%    http://www.latex-project.org/lppl.txt
% and version 1.3 or later is part of all distributions of
% LaTeX version 2005/12/01 or later.
%
% This work has the LPPL maintenance status "maintained".
%
% This Current Maintainer of this work is Heiko Oberdiek.
%
% The Base Interpreter refers to any `TeX-Format',
% because some files are installed in TDS:tex/generic//.
%
% This work consists of the main source file soulutf8.dtx
% and the derived files
%    soulutf8.sty, soulutf8.pdf, soulutf8.ins, soulutf8.drv,
%    soulutf8-test1.tex, soulutf8-test2.tex, soulutf8-test3.tex,
%    soulutf8-test4.tex, soulutf8-test5.tex.
%
% Distribution:
%    CTAN:macros/latex/contrib/oberdiek/soulutf8.dtx
%    CTAN:macros/latex/contrib/oberdiek/soulutf8.pdf
%
% Unpacking:
%    (a) If soulutf8.ins is present:
%           tex soulutf8.ins
%    (b) Without soulutf8.ins:
%           tex soulutf8.dtx
%    (c) If you insist on using LaTeX
%           latex \let\install=y% \iffalse meta-comment
%
% File: soulutf8.dtx
% Version: 2016/05/16 v1.1
% Info: Permit use of UTF-8 characters in soul
%
% Copyright (C) 2007 by
%    Heiko Oberdiek <heiko.oberdiek at googlemail.com>
%    2016
%    https://github.com/ho-tex/oberdiek/issues
%
% This work may be distributed and/or modified under the
% conditions of the LaTeX Project Public License, either
% version 1.3c of this license or (at your option) any later
% version. This version of this license is in
%    http://www.latex-project.org/lppl/lppl-1-3c.txt
% and the latest version of this license is in
%    http://www.latex-project.org/lppl.txt
% and version 1.3 or later is part of all distributions of
% LaTeX version 2005/12/01 or later.
%
% This work has the LPPL maintenance status "maintained".
%
% This Current Maintainer of this work is Heiko Oberdiek.
%
% The Base Interpreter refers to any `TeX-Format',
% because some files are installed in TDS:tex/generic//.
%
% This work consists of the main source file soulutf8.dtx
% and the derived files
%    soulutf8.sty, soulutf8.pdf, soulutf8.ins, soulutf8.drv,
%    soulutf8-test1.tex, soulutf8-test2.tex, soulutf8-test3.tex,
%    soulutf8-test4.tex, soulutf8-test5.tex.
%
% Distribution:
%    CTAN:macros/latex/contrib/oberdiek/soulutf8.dtx
%    CTAN:macros/latex/contrib/oberdiek/soulutf8.pdf
%
% Unpacking:
%    (a) If soulutf8.ins is present:
%           tex soulutf8.ins
%    (b) Without soulutf8.ins:
%           tex soulutf8.dtx
%    (c) If you insist on using LaTeX
%           latex \let\install=y% \iffalse meta-comment
%
% File: soulutf8.dtx
% Version: 2016/05/16 v1.1
% Info: Permit use of UTF-8 characters in soul
%
% Copyright (C) 2007 by
%    Heiko Oberdiek <heiko.oberdiek at googlemail.com>
%    2016
%    https://github.com/ho-tex/oberdiek/issues
%
% This work may be distributed and/or modified under the
% conditions of the LaTeX Project Public License, either
% version 1.3c of this license or (at your option) any later
% version. This version of this license is in
%    http://www.latex-project.org/lppl/lppl-1-3c.txt
% and the latest version of this license is in
%    http://www.latex-project.org/lppl.txt
% and version 1.3 or later is part of all distributions of
% LaTeX version 2005/12/01 or later.
%
% This work has the LPPL maintenance status "maintained".
%
% This Current Maintainer of this work is Heiko Oberdiek.
%
% The Base Interpreter refers to any `TeX-Format',
% because some files are installed in TDS:tex/generic//.
%
% This work consists of the main source file soulutf8.dtx
% and the derived files
%    soulutf8.sty, soulutf8.pdf, soulutf8.ins, soulutf8.drv,
%    soulutf8-test1.tex, soulutf8-test2.tex, soulutf8-test3.tex,
%    soulutf8-test4.tex, soulutf8-test5.tex.
%
% Distribution:
%    CTAN:macros/latex/contrib/oberdiek/soulutf8.dtx
%    CTAN:macros/latex/contrib/oberdiek/soulutf8.pdf
%
% Unpacking:
%    (a) If soulutf8.ins is present:
%           tex soulutf8.ins
%    (b) Without soulutf8.ins:
%           tex soulutf8.dtx
%    (c) If you insist on using LaTeX
%           latex \let\install=y\input{soulutf8.dtx}
%        (quote the arguments according to the demands of your shell)
%
% Documentation:
%    (a) If soulutf8.drv is present:
%           latex soulutf8.drv
%    (b) Without soulutf8.drv:
%           latex soulutf8.dtx; ...
%    The class ltxdoc loads the configuration file ltxdoc.cfg
%    if available. Here you can specify further options, e.g.
%    use A4 as paper format:
%       \PassOptionsToClass{a4paper}{article}
%
%    Programm calls to get the documentation (example):
%       pdflatex soulutf8.dtx
%       makeindex -s gind.ist soulutf8.idx
%       pdflatex soulutf8.dtx
%       makeindex -s gind.ist soulutf8.idx
%       pdflatex soulutf8.dtx
%
% Installation:
%    TDS:tex/generic/oberdiek/soulutf8.sty
%    TDS:doc/latex/oberdiek/soulutf8.pdf
%    TDS:doc/latex/oberdiek/test/soulutf8-test1.tex
%    TDS:doc/latex/oberdiek/test/soulutf8-test2.tex
%    TDS:doc/latex/oberdiek/test/soulutf8-test3.tex
%    TDS:doc/latex/oberdiek/test/soulutf8-test4.tex
%    TDS:doc/latex/oberdiek/test/soulutf8-test5.tex
%    TDS:source/latex/oberdiek/soulutf8.dtx
%
%<*ignore>
\begingroup
  \catcode123=1 %
  \catcode125=2 %
  \def\x{LaTeX2e}%
\expandafter\endgroup
\ifcase 0\ifx\install y1\fi\expandafter
         \ifx\csname processbatchFile\endcsname\relax\else1\fi
         \ifx\fmtname\x\else 1\fi\relax
\else\csname fi\endcsname
%</ignore>
%<*install>
\input docstrip.tex
\Msg{************************************************************************}
\Msg{* Installation}
\Msg{* Package: soulutf8 2016/05/16 v1.1 Permit use of UTF-8 characters in soul (HO)}
\Msg{************************************************************************}

\keepsilent
\askforoverwritefalse

\let\MetaPrefix\relax
\preamble

This is a generated file.

Project: soulutf8
Version: 2016/05/16 v1.1

Copyright (C) 2007 by
   Heiko Oberdiek <heiko.oberdiek at googlemail.com>

This work may be distributed and/or modified under the
conditions of the LaTeX Project Public License, either
version 1.3c of this license or (at your option) any later
version. This version of this license is in
   http://www.latex-project.org/lppl/lppl-1-3c.txt
and the latest version of this license is in
   http://www.latex-project.org/lppl.txt
and version 1.3 or later is part of all distributions of
LaTeX version 2005/12/01 or later.

This work has the LPPL maintenance status "maintained".

This Current Maintainer of this work is Heiko Oberdiek.

The Base Interpreter refers to any `TeX-Format',
because some files are installed in TDS:tex/generic//.

This work consists of the main source file soulutf8.dtx
and the derived files
   soulutf8.sty, soulutf8.pdf, soulutf8.ins, soulutf8.drv,
   soulutf8-test1.tex, soulutf8-test2.tex, soulutf8-test3.tex,
   soulutf8-test4.tex, soulutf8-test5.tex.

\endpreamble
\let\MetaPrefix\DoubleperCent

\generate{%
  \file{soulutf8.ins}{\from{soulutf8.dtx}{install}}%
  \file{soulutf8.drv}{\from{soulutf8.dtx}{driver}}%
  \usedir{tex/generic/oberdiek}%
  \file{soulutf8.sty}{\from{soulutf8.dtx}{package}}%
%  \usedir{doc/latex/oberdiek/test}%
%  \file{soulutf8-test1.tex}{\from{soulutf8.dtx}{test1}}%
%  \file{soulutf8-test2.tex}{\from{soulutf8.dtx}{test2,utf8}}%
%  \file{soulutf8-test3.tex}{\from{soulutf8.dtx}{test3,utf8x}}%
%  \file{soulutf8-test4.tex}{\from{soulutf8.dtx}{test4,utf8,noetex}}%
%  \file{soulutf8-test5.tex}{\from{soulutf8.dtx}{test5,utf8x,noetex}}%
  \nopreamble
  \nopostamble
%  \usedir{source/latex/oberdiek/catalogue}%
%  \file{soulutf8.xml}{\from{soulutf8.dtx}{catalogue}}%
}

\catcode32=13\relax% active space
\let =\space%
\Msg{************************************************************************}
\Msg{*}
\Msg{* To finish the installation you have to move the following}
\Msg{* file into a directory searched by TeX:}
\Msg{*}
\Msg{*     soulutf8.sty}
\Msg{*}
\Msg{* To produce the documentation run the file `soulutf8.drv'}
\Msg{* through LaTeX.}
\Msg{*}
\Msg{* Happy TeXing!}
\Msg{*}
\Msg{************************************************************************}

\endbatchfile
%</install>
%<*ignore>
\fi
%</ignore>
%<*driver>
\NeedsTeXFormat{LaTeX2e}
\ProvidesFile{soulutf8.drv}%
  [2016/05/16 v1.1 Permit use of UTF-8 characters in soul (HO)]%
\documentclass{ltxdoc}
\usepackage{holtxdoc}[2011/11/22]
\begin{document}
  \DocInput{soulutf8.dtx}%
\end{document}
%</driver>
% \fi
%
%
% \CharacterTable
%  {Upper-case    \A\B\C\D\E\F\G\H\I\J\K\L\M\N\O\P\Q\R\S\T\U\V\W\X\Y\Z
%   Lower-case    \a\b\c\d\e\f\g\h\i\j\k\l\m\n\o\p\q\r\s\t\u\v\w\x\y\z
%   Digits        \0\1\2\3\4\5\6\7\8\9
%   Exclamation   \!     Double quote  \"     Hash (number) \#
%   Dollar        \$     Percent       \%     Ampersand     \&
%   Acute accent  \'     Left paren    \(     Right paren   \)
%   Asterisk      \*     Plus          \+     Comma         \,
%   Minus         \-     Point         \.     Solidus       \/
%   Colon         \:     Semicolon     \;     Less than     \<
%   Equals        \=     Greater than  \>     Question mark \?
%   Commercial at \@     Left bracket  \[     Backslash     \\
%   Right bracket \]     Circumflex    \^     Underscore    \_
%   Grave accent  \`     Left brace    \{     Vertical bar  \|
%   Right brace   \}     Tilde         \~}
%
% \GetFileInfo{soulutf8.drv}
%
% \title{The \xpackage{soulutf8} package}
% \date{2016/05/16 v1.1}
% \author{Heiko Oberdiek\thanks
% {Please report any issues at https://github.com/ho-tex/oberdiek/issues}\\
% \xemail{heiko.oberdiek at googlemail.com}}
%
% \maketitle
%
% \begin{abstract}
% This package extends package \xpackage{soul} and adds some support
% for UTF-8. Namely the input encodings \xfile{utf8.def}
% from package \xpackage{inputenc} and
% package \xpackage{ucs}'s \xfile{utf8x.def} are supported.
% \end{abstract}
%
% \tableofcontents
%
% \section{Documentation}
%
% This package \xpackage{soulutf8} does not have own options and does not
% define new user commands.
% Any option is passed to package \xpackage{soul} \cite{soul}
% that is loaded first.
% Then some internal macros of \xpackage{soul} are redefined to add
% support for UTF-8.
% The following input encodings are supported:
% \begin{quote}
%   \begin{tabular}{@{}lll@{}}
%     \texttt{utf8} & \LaTeX\ base &
%        \xfile{TDS:tex/latex/base/utf8.def} \cite{utf8}\\
%     \texttt{utf8x} & Package \xpackage{ucs} &
%       \xfile{TDS:tex/latex/ucs/utf8x.def} \cite{ucs}\\
%   \end{tabular}
% \end{quote}
% UTF-8 byte sequences are added as token group to a word, even
% if these UTF-8 characters are some kind of hyphen or space.
% As exception the following three Unicode characters are handled specially:
% \begin{quote}
%   \begin{tabular}{lll}
%     Slot & Name & Action\\
%     \hline
%     \texttt{U+00A0} & NO-BREAK SPACE & like |~|\\
%     \texttt{U+2013} & EN DASH & |--|\\
%     \texttt{U+2014} & EM DASH & |---|
%   \end{tabular}
%
% \subsection{Patch}
%
% Also package \xpackage{soulutf8} tries to patch package \xpackage{soul}
% to improve its behaviour:
% \begin{itemize}
% \item
%   A problem with additional levels of curly braces is fixed.
%   As advantage more implicite kernings are detected.
%   However, the result may be incompatible with the
%   original behaviour of package \xpackage{soul} because
%   of these respected implicite kernings.
% \item
%   \eTeX\ , especially \cs{unexpanded} is supported.
%   This allows a better protection of token groups
%   (|\mbox{|\dots|}|, math, \dots).
% \end{itemize}
% \end{quote}
%
% \subsection{Future}
%
% Currently package \xpackage{soul} does not seem to be
% maintained. Nevertheless if there will be a new version that
% adds support for UTF-8, then this package may become
% obsolete.
%
% \StopEventually{
% }
%
% \section{Implementation}
%
%    \begin{macrocode}
%<*package>
%    \end{macrocode}
%
% \subsection{Reload check and package identification}
%    Reload check, especially if the package is not used with \LaTeX.
%    \begin{macrocode}
\begingroup\catcode61\catcode48\catcode32=10\relax%
  \catcode13=5 % ^^M
  \endlinechar=13 %
  \catcode35=6 % #
  \catcode39=12 % '
  \catcode44=12 % ,
  \catcode45=12 % -
  \catcode46=12 % .
  \catcode58=12 % :
  \catcode64=11 % @
  \catcode123=1 % {
  \catcode125=2 % }
  \expandafter\let\expandafter\x\csname ver@soulutf8.sty\endcsname
  \ifx\x\relax % plain-TeX, first loading
  \else
    \def\empty{}%
    \ifx\x\empty % LaTeX, first loading,
      % variable is initialized, but \ProvidesPackage not yet seen
    \else
      \expandafter\ifx\csname PackageInfo\endcsname\relax
        \def\x#1#2{%
          \immediate\write-1{Package #1 Info: #2.}%
        }%
      \else
        \def\x#1#2{\PackageInfo{#1}{#2, stopped}}%
      \fi
      \x{soulutf8}{The package is already loaded}%
      \aftergroup\endinput
    \fi
  \fi
\endgroup%
%    \end{macrocode}
%    Package identification:
%    \begin{macrocode}
\begingroup\catcode61\catcode48\catcode32=10\relax%
  \catcode13=5 % ^^M
  \endlinechar=13 %
  \catcode35=6 % #
  \catcode39=12 % '
  \catcode40=12 % (
  \catcode41=12 % )
  \catcode44=12 % ,
  \catcode45=12 % -
  \catcode46=12 % .
  \catcode47=12 % /
  \catcode58=12 % :
  \catcode64=11 % @
  \catcode91=12 % [
  \catcode93=12 % ]
  \catcode123=1 % {
  \catcode125=2 % }
  \expandafter\ifx\csname ProvidesPackage\endcsname\relax
    \def\x#1#2#3[#4]{\endgroup
      \immediate\write-1{Package: #3 #4}%
      \xdef#1{#4}%
    }%
  \else
    \def\x#1#2[#3]{\endgroup
      #2[{#3}]%
      \ifx#1\@undefined
        \xdef#1{#3}%
      \fi
      \ifx#1\relax
        \xdef#1{#3}%
      \fi
    }%
  \fi
\expandafter\x\csname ver@soulutf8.sty\endcsname
\ProvidesPackage{soulutf8}%
  [2016/05/16 v1.1 Permit use of UTF-8 characters in soul (HO)]%
%    \end{macrocode}
%
% \subsection{Catcodes}
%
%    \begin{macrocode}
\begingroup\catcode61\catcode48\catcode32=10\relax%
  \catcode13=5 % ^^M
  \endlinechar=13 %
  \catcode123=1 % {
  \catcode125=2 % }
  \catcode64=11 % @
  \def\x{\endgroup
    \expandafter\edef\csname SOuL@AtEnd\endcsname{%
      \endlinechar=\the\endlinechar\relax
      \catcode13=\the\catcode13\relax
      \catcode32=\the\catcode32\relax
      \catcode35=\the\catcode35\relax
      \catcode61=\the\catcode61\relax
      \catcode64=\the\catcode64\relax
      \catcode123=\the\catcode123\relax
      \catcode125=\the\catcode125\relax
    }%
  }%
\x\catcode61\catcode48\catcode32=10\relax%
\catcode13=5 % ^^M
\endlinechar=13 %
\catcode35=6 % #
\catcode64=11 % @
\catcode123=1 % {
\catcode125=2 % }
\def\TMP@EnsureCode#1#2{%
  \edef\SOuL@AtEnd{%
    \SOuL@AtEnd
    \catcode#1=\the\catcode#1\relax
  }%
  \catcode#1=#2\relax
}
\TMP@EnsureCode{10}{12}% ^^J
\TMP@EnsureCode{33}{12}% !
\TMP@EnsureCode{34}{12}% "
\TMP@EnsureCode{36}{3}% $
\TMP@EnsureCode{39}{12}% '
\TMP@EnsureCode{40}{12}% (
\TMP@EnsureCode{41}{12}% )
\TMP@EnsureCode{42}{12}% *
\TMP@EnsureCode{43}{12}% +
\TMP@EnsureCode{44}{12}% ,
\TMP@EnsureCode{45}{12}% -
\TMP@EnsureCode{46}{12}% .
\TMP@EnsureCode{47}{12}% /
\TMP@EnsureCode{58}{12}% :
\TMP@EnsureCode{60}{12}% <
\TMP@EnsureCode{62}{12}% >
\TMP@EnsureCode{91}{12}% [
\TMP@EnsureCode{93}{12}% ]
\TMP@EnsureCode{94}{7}% ^
\TMP@EnsureCode{96}{12}% `
\TMP@EnsureCode{126}\active % ~
\TMP@EnsureCode{128}{12}% ^^80
\TMP@EnsureCode{147}{12}% ^^93
\TMP@EnsureCode{148}{12}% ^^94
\TMP@EnsureCode{160}{12}% ^^a0
\TMP@EnsureCode{194}{12}% ^^c2
\TMP@EnsureCode{226}{12}% ^^e2
\edef\SOuL@AtEnd{\SOuL@AtEnd\noexpand\endinput}
%    \end{macrocode}
%
% \subsection{Loading packages}
%
%    Package \xpackage{soul} uses \cs{documentclass} to detect \LaTeX.
%    \begin{macrocode}
\ifx\documentclass\@undefined
%    \end{macrocode}
%
% \subsubsection{\plainTeX}
%
%    First we check, whether package \xpackage{soul} is already loaded.
%    \begin{macrocode}
  \expandafter\ifx\csname SOUL@\endcsname\relax
%    \end{macrocode}
%    In case of \plainTeX\ package \xpackage{soul} defines some
%    macros in a simple manner that will break the definitions of
%    \xfile{miniltx.tex}, for example. Therefore these macros
%    are first saved and restored afterwards.
%    \begin{macrocode}
    \let\SOuL@orgDeclareRobustCommand\DeclareRobustCommand
    \let\SOuL@orgnewcommand          \newcommand
    \let\SOuL@orgDeclareOption       \DeclareOption
    \let\SOuL@orgPackageError        \PackageError
    \def\SOuL@restorelatexcmds{%
      \let\DeclareRobustCommand\SOuL@orgDeclareRobustCommand
      \let\newcommand          \SOuL@orgnewcommand
      \let\DeclareOption       \SOuL@orgDeclareOption
      \let\PackageError        \SOuL@orgPackageError
    }%
    \input soul.sty\relax
    \SOuL@restorelatexcmds
  \fi
%    \end{macrocode}
%    \begin{macro}{\SOUL@error}
%     Package \xpackage{soul}'s use of \cs{PackageError} is replaced
%     by \cs{@PackageError} of package \xpackage{infwarerr}.
%    \begin{macrocode}
  \input infwarerr.sty\relax
  \let\SOuL@orgSOUL@error\SOUL@error
  \def\SOUL@error{%
    \begingroup
      \let\PackageError\@PackageError
      \SOuL@orgSOUL@error
    \endgroup
  }%
%    \end{macrocode}
%    \end{macro}
%    \begin{macrocode}
  \input etexcmds.sty\relax
%    \end{macrocode}
%    \begin{macro}{\@onelevel@sanitize}
%    Define \LaTeX's \cs{@onelevel@sanitize} if not already available.
%    \begin{macrocode}
  \expandafter\ifx\csname @onelevel@sanitize\endcsname\relax
    \def\@onelevel@sanitize#1{%
      \edef#1{%
        \expandafter\strip@prefix\meaning#1%
      }%
    }%
%    \end{macrocode}
%    \end{macro}
%    \begin{macro}{\strip@prefix}
%    \begin{macrocode}
    \def\strip@prefix#1>{}%
%    \end{macrocode}
%    \end{macro}
%    \begin{macrocode}
  \fi
\else
%    \end{macrocode}
%
% \subsubsection{\LaTeX}
%    \begin{macrocode}
  \DeclareOption*{\PassOptionsToPackage{\CurrentOption}{soul}}%
  \ProcessOptions\relax
  \RequirePackage{soul}[2003/11/17]%
  \RequirePackage{infwarerr}[2016/05/16]%
  \RequirePackage{etexcmds}[2016/05/16]%
\fi
%    \end{macrocode}
%
% \subsubsection{\eTeX}
%
%    In \plainTeX\ command \cs{+} is an \emph{outer} macro.
%    Therefore numbers are used to avoid problems.
%    \begin{macrocode}
\ifetex@unexpanded
  \catcode33=14 % '!': comment
  \catcode43=9  % '+': ignore
\else
  \catcode33=9  % '!': ignore
  \catcode43=14 % '+': comment
\fi
%    \end{macrocode}
%
% \subsection{Macro for redefinitions}
%
%    \begin{macro}{\SOuL@redefine}
%    \begin{macrocode}
\def\SOuL@redefine#1{%
  \begingroup
    \def\SOuL@cmd{#1}%
    \afterassignment\SOuL@cmdcheck
    \def\SOuL@temp
}
%    \end{macrocode}
%    \end{macro}
%    \begin{macro}{\SOuL@cmdcheck}
%    \begin{macrocode}
\def\SOuL@cmdcheck{%
    \expandafter\ifx\SOuL@cmd\SOuL@temp
    \else
      \edef\SOuL@temp*{\expandafter\string\SOuL@cmd}%
      \@PackageWarningNoLine{soulutf8}{%
        Command \SOuL@temp* has changed.\MessageBreak
        Supported versions of package `soul': 2003/11/17.\MessageBreak
        Depending on the unknown changes the redefinition\MessageBreak
        of \SOuL@temp* may not behave correctly%
      }%
    \fi
  \expandafter\endgroup
  \expandafter\def\SOuL@cmd
}
%    \end{macrocode}
%    \end{macro}
%
% \subsection{Redefinition of \cs{SOUL@eval}}
%
%    \begin{macro}{\SOUL@eval}
%    Macro \cs{SOUL@eval} is redefined to add detection of
%    the first byte of a UTF-8 sequence. Because \cs{SOUL@eval}
%    is overwritten, a warning is issued, if the contents of
%    \cs{SOUL@eval} is not as expected.
%    \begin{macrocode}
\SOuL@redefine\SOUL@eval{%
%    \end{macrocode}
%    First the expected definition.
%    \begin{macrocode}
  \def\SOUL@n*##1{\SOUL@scan}%
  \if\noexpand\SOUL@@\SOUL@spc
  \else
    \SOUL@ignorespacesfalse
  \fi
  \ifnum\SOUL@minus=\thr@@
    \SOUL@flushminus
  \else\ifnum\SOUL@comma=\tw@
    \SOUL@flushcomma
  \else\ifnum\SOUL@apo=\tw@
    \SOUL@flushapo
  \else\ifnum\SOUL@grave=\tw@
    \SOUL@flushgrave
  \fi\fi\fi\fi
  \ifx\SOUL@@-\else\SOUL@flushminus\fi
  \ifx\SOUL@@,\else\SOUL@flushcomma\fi
  \ifx\SOUL@@'\else\SOUL@flushapo\fi
  \ifx\SOUL@@`\else\SOUL@flushgrave\fi
  \ifx\SOUL@@-%
    \advance\SOUL@minus\@ne
  \else\ifx\SOUL@@,%
    \advance\SOUL@comma\@ne
  \else\ifx\SOUL@@'%
    \advance\SOUL@apo\@ne
  \else\ifx\SOUL@@`%
    \advance\SOUL@grave\@ne
  \else
    \SOUL@flushminus
    \SOUL@flushcomma
    \SOUL@flushapo
    \SOUL@flushgrave
    \ifx\SOUL@@\SOUL@stop
      \def\SOUL@n*{%
        \SOUL@doword
        \SOUL@eventuallyexhyphen\null
      }%
    \else\ifx\SOUL@@\par
      \def\SOUL@n*\par{\par\leavevmode\SOUL@scan}%
    \else\if\noexpand\SOUL@@\SOUL@spc
      \SOUL@doword
      \SOUL@eventuallyexhyphen\null
      \ifSOUL@ignorespaces
      \else
        \SOUL@everyspace{}%
      \fi
      \def\SOUL@n* {\SOUL@scan}%
    \else\ifx\SOUL@@\\%
      \SOUL@doword
      \SOUL@eventuallyexhyphen\null
      \SOUL@everyspace{\unskip\nobreak\hfil\break}%
      \SOUL@ignorespacestrue
    \else\ifx\SOUL@@~%
      \SOUL@doword
      \SOUL@eventuallyexhyphen\null
      \SOUL@everyspace{\nobreak}%
    \else\ifx\SOUL@@\slash
      \SOUL@doword
      \SOUL@eventuallyexhyphen{/}%
      \SOUL@exhyphen{/}%
    \else\ifx\SOUL@@\mbox
      \def\SOUL@n*{\SOUL@addprotect}%
    \else\ifx\SOUL@@\hbox
      \def\SOUL@n*{\SOUL@addprotect}%
    \else\ifx\SOUL@@\soulomit
      \def\SOUL@n*\soulomit##1{%
        \SOUL@doword
        {\spaceskip\SOUL@spaceskip##1}%
        \SOUL@scan
      }%
    \else\ifx\SOUL@@\break
      \SOUL@doword
      \break
    \else\ifx\SOUL@@\linebreak
      \SOUL@doword
      \SOUL@everyspace{\linebreak}%
    \else\ifcat\bgroup\noexpand\SOUL@@
      \def\SOUL@n*{\SOUL@addgroup{}}%
    \else\ifcat$\noexpand\SOUL@@
      \def\SOUL@n*{\SOUL@addmath}%
    \else
      \def\SOUL@n*{\SOUL@dotoken}%
    \fi\fi\fi\fi\fi\fi\fi\fi\fi\fi\fi\fi\fi
  \fi\fi\fi\fi
  \SOUL@n*%
}{%
%    \end{macrocode}
%    Now the redefined version follows.
%    \begin{macrocode}
  \def\SOUL@n*##1{\SOUL@scan}%
  \if\noexpand\SOUL@@\SOUL@spc
  \else
    \SOUL@ignorespacesfalse
  \fi
  \ifnum\SOUL@minus=\thr@@
    \SOUL@flushminus
  \else\ifnum\SOUL@comma=\tw@
    \SOUL@flushcomma
  \else\ifnum\SOUL@apo=\tw@
    \SOUL@flushapo
  \else\ifnum\SOUL@grave=\tw@
    \SOUL@flushgrave
  \fi\fi\fi\fi
  \ifx\SOUL@@-\else\SOUL@flushminus\fi
  \ifx\SOUL@@,\else\SOUL@flushcomma\fi
  \ifx\SOUL@@'\else\SOUL@flushapo\fi
  \ifx\SOUL@@`\else\SOUL@flushgrave\fi
  \ifx\SOUL@@-%
    \advance\SOUL@minus\@ne
  \else\ifx\SOUL@@,%
    \advance\SOUL@comma\@ne
  \else\ifx\SOUL@@'%
    \advance\SOUL@apo\@ne
  \else\ifx\SOUL@@`%
    \advance\SOUL@grave\@ne
  \else
    \SOUL@flushminus
    \SOUL@flushcomma
    \SOUL@flushapo
    \SOUL@flushgrave
    \ifx\SOUL@@\SOUL@stop
      \def\SOUL@n*{%
          \SOUL@doword
          \SOUL@eventuallyexhyphen\null
      }%
    \else\ifx\SOUL@@\par
      \def\SOUL@n*\par{\par\leavevmode\SOUL@scan}%
    \else\if\noexpand\SOUL@@\SOUL@spc
      \SOUL@doword
      \SOUL@eventuallyexhyphen\null
      \ifSOUL@ignorespaces
      \else
        \SOUL@everyspace{}%
      \fi
      \def\SOUL@n* {\SOUL@scan}%
    \else\ifx\SOUL@@\\%
      \SOUL@doword
      \SOUL@eventuallyexhyphen\null
      \SOUL@everyspace{\unskip\nobreak\hfil\break}%
      \SOUL@ignorespacestrue
    \else\ifx\SOUL@@~%
      \SOUL@doword
      \SOUL@eventuallyexhyphen\null
      \SOUL@everyspace{\nobreak}%
    \else\ifx\SOUL@@\slash
      \SOUL@doword
      \SOUL@eventuallyexhyphen{/}%
      \SOUL@exhyphen{/}%
    \else\ifx\SOUL@@\mbox
      \def\SOUL@n*{\SOUL@addprotect}%
    \else\ifx\SOUL@@\hbox
      \def\SOUL@n*{\SOUL@addprotect}%
    \else\ifx\SOUL@@\soulomit
      \def\SOUL@n*\soulomit##1{%
        \SOUL@doword
        {\spaceskip\SOUL@spaceskip##1}%
        \SOUL@scan
      }%
    \else\ifx\SOUL@@\break
      \SOUL@doword
      \break
    \else\ifx\SOUL@@\linebreak
      \SOUL@doword
      \SOUL@everyspace{\linebreak}%
    \else\ifcat\bgroup\noexpand\SOUL@@
      \def\SOUL@n*{\SOUL@addgroup{}}%
    \else\ifcat$\noexpand\SOUL@@
      \def\SOUL@n*{\SOUL@addmath}%
    \else
%    \end{macrocode}
%    The current token is examined to detect the start of
%    a UTF-8 sequence.
%    \begin{macrocode}
      \SOuL@analyzeutfviii
      \ifcase\SOuL@octets
        \SOuL@analyzeutfviiix
      \fi
      \ifcase\SOuL@octets
        \def\SOUL@n*{\SOUL@dotoken}%
      \or % 1
      \or % 2
        \def\SOUL@n*{\SOuL@addtwooctets}%
      \or % 3
        \def\SOUL@n*{\SOuL@addthreeoctets}%
      \or % 4
        \def\SOUL@n*{\SOuL@addfouroctets}%
      \fi
    \fi\fi\fi\fi\fi\fi\fi\fi\fi\fi\fi\fi\fi
  \fi\fi\fi\fi
  \SOUL@n*%
}
%    \end{macrocode}
%    \end{macro}
%
% \subsection{UTF-8 analysis}
%
% \subsubsection{Help strings}
%
%    \begin{macrocode}
\def\SOuL@defsanitizedstring#1#2{%
  \expandafter\def\csname SOuL@string#1\endcsname{#2}%
  \expandafter\@onelevel@sanitize\csname SOuL@string#1\endcsname
}
\SOuL@defsanitizedstring{UTFviii}{UTFviii@}
\SOuL@defsanitizedstring{octets}{@octets}
\SOuL@defsanitizedstring{two}{two}
\SOuL@defsanitizedstring{three}{three}
\SOuL@defsanitizedstring{four}{four}
\SOuL@defsanitizedstring{macrocolon}{macro:}
\SOuL@defsanitizedstring{csnameu}{csname u8-}
\SOuL@defsanitizedstring{undeferr}{utf@viii@undeferr}
\def\SOuL@stringendash{^^e2^^80^^93}
\def\SOuL@stringemdash{^^e2^^80^^94}
\def\SOuL@stringnobreakspace{^^c2^^a0}
\edef\SOuL@charhash{\string #}
\edef\SOuL@chartwo{\string 2}
\edef\SOuL@charthree{\string 3}
\def\SOuL@empty{}
%    \end{macrocode}
%
% \subsubsection{Support for \xfile{utf8.def}}
%
%    \begin{macro}{\SOuL@analyzeutfviii}
%    \begin{macrocode}
\begingroup
  \edef\x{\endgroup
    \def\noexpand\SOuL@analyzeutfviii{%
      \noexpand\expandafter\noexpand\SOuL@checkutfviii
      \noexpand\meaning\noexpand\SOUL@@
      \SOuL@stringUTFviii\SOuL@stringoctets
      \noexpand\@nil
    }%
    \def\noexpand\SOuL@checkutfviii
      ##1\SOuL@stringUTFviii##2\SOuL@stringoctets##3\noexpand\@nil
  }%
\x{%
  \def\SOuL@temp{#2}%
  \chardef\SOuL@octets=%
      \ifx\SOuL@temp\SOuL@stringtwo
        \tw@
      \else\ifx\SOuL@temp\SOuL@stringthree
        \thr@@
      \else\ifx\SOuL@temp\SOuL@stringfour
        4 %
      \else
        \z@
      \fi\fi\fi
}
%    \end{macrocode}
%    \end{macro}
%
% \subsubsection{Support for \xfile{utf8x.def}}
%
%    \begin{macro}{\SOuL@analyzeutfviiix}
%    \begin{macrocode}
\begingroup
  \edef\x{\endgroup
    \def\noexpand\SOuL@analyzeutfviiix{%
      \noexpand\expandafter\noexpand\SOuL@checkutfviiix
      \noexpand\meaning\noexpand\SOUL@@
      \SOuL@stringmacrocolon\SOuL@charhash1{}{}{}{}%
      \SOuL@stringcsnameu\SOuL@stringundeferr
      \noexpand\@nil
    }%
%    \end{macrocode}
%    \end{macro}
%    \begin{macro}{\SOuL@checkutfviiix}
%    \begin{macrocode}
    \def\noexpand\SOuL@checkutfviiix
      ##1\SOuL@stringmacrocolon\SOuL@charhash1##2##3##4##5##6%
      \SOuL@stringcsnameu##7\SOuL@stringundeferr##8\noexpand\@nil
  }%
\x{%
  \def\SOuL@temp{#7}%
  \ifx\SOuL@temp\SOuL@empty
    \chardef\SOuL@octets=\z@
  \else
    \def\SOuL@temp{#5}%
    \ifx\SOuL@temp\SOuL@charthree
      \chardef\SOuL@octets=4 %
    \else
      \def\SOuL@temp{#3}%
      \ifx\SOuL@temp\SOuL@chartwo
        \chardef\SOuL@octets=\thr@@
      \else
        \chardef\SOuL@octets=\tw@
      \fi
    \fi
  \fi
}
%    \end{macrocode}
%    \end{macro}
%
% \subsection{Actions for UTF-8 sequences}
%
%    \begin{macro}{\SOuL@addtwooctets}
%    \begin{macrocode}
\def\SOuL@addtwooctets#1#2{%
  \def\SOuL@temp{#1#2}%
  \@onelevel@sanitize\SOuL@temp
  \ifx\SOuL@temp\SOuL@stringnobreakspace
    \SOUL@doword
    \SOUL@eventuallyexhyphen\null
    \SOUL@everyspace{\nobreak}%
    \let\SOuL@next\SOUL@scan
  \else
    \def\SOuL@next{%
!     \SOUL@addtoken{{\noexpand#1\noexpand#2}}%
+     \SOUL@addtoken{{\etex@unexpanded{#1#2}}}%
    }%
  \fi
  \SOuL@next
}
%    \end{macrocode}
%    \end{macro}
%    \begin{macro}{\SOuL@addthreeoctets}
%    \begin{macrocode}
\def\SOuL@addthreeoctets#1#2#3{%
  \def\SOuL@temp{#1#2#3}%
  \@onelevel@sanitize\SOuL@temp
  \ifx\SOuL@temp\SOuL@stringendash
    \SOUL@doword
    \SOUL@eventuallyexhyphen{-}%
    \SOUL@exhyphen{--}%
    \let\SOuL@next\SOUL@scan
  \else
    \ifx\SOuL@temp\SOuL@stringemdash
      \SOUL@doword
      \SOUL@eventuallyexhyphen{-}%
      \SOUL@exhyphen{---}%
      \let\SOuL@next\SOUL@scan
    \else
      \def\SOuL@next{%
!       \SOUL@addtoken{{\noexpand#1\noexpand#2\noexpand#3}}%
+       \SOUL@addtoken{{\etex@unexpanded{#1#2#3}}}%
      }%
    \fi
  \fi
  \SOuL@next
}
%    \end{macrocode}
%    \end{macro}
%    \begin{macro}{\SOuL@addfouroctets}
%    \begin{macrocode}
\def\SOuL@addfouroctets#1#2#3#4{%
! \SOUL@addtoken{{\noexpand#1\noexpand#2\noexpand#3\noexpand#4}}%
+ \SOUL@addtoken{{\etex@unexpanded{#1#2#3#4}}}%
}
%    \end{macrocode}
%    \end{macro}
%
% \subsubsection{Redefinition of \cs{SOUL@splittoken}}
%
%    \begin{macro}{\SOUL@splittoken}
%    Macro \cs{SOUL@splittoken} separates the first token
%    or token group from a word and redefines the word to contain
%    the remaining tokens. However if the remaining tokens are
%    a token group, then the curly braces will be removed and
%    the token group is splitted by the next call of \cs{SOUL@splittoken}.
%    The redefinition avoids the removal of curly braces around the
%    remaining tokens.
%    \begin{macrocode}
\SOuL@redefine\SOUL@splittoken#1#2\SOUL@stop{%
  \global\SOUL@token={#1}%
  \global\SOUL@word={#2}%
}#1{%
  \global\SOUL@token={#1}%
  \SOuL@remainingtoken\relax
}
%    \end{macrocode}
%    \end{macro}
%    \begin{macro}{\SOuL@remainingtoken}
%    \begin{macrocode}
\def\SOuL@remainingtoken#1\SOUL@stop{%
  \global\SOUL@word=\expandafter{\@gobble#1}%
}
%    \end{macrocode}
%    \end{macro}
%
% \subsection{Patches}
%
%    The fixed \cs{SOUL@splittoken} allows to remove the double
%    sets of curly braces in other macros of package \xpackage{soul}.
%    The benefit is that implicite kernings are more often detected
%    and fixes a bug in package \xpackage{soul}. The disadvantage is
%    incompatibility. The width of the resulting strings may change.
%    \begin{macro}{\SOUL@flushcomma}
%    \begin{macrocode}
\SOuL@redefine\SOUL@flushcomma{%
  \ifcase\SOUL@comma
  \or
    \edef\x{\SOUL@word={\the\SOUL@word,}}\x
  \or
    \edef\x{\SOUL@word={\the\SOUL@word{{,,}}}}\x
  \fi
  \SOUL@comma\z@
}{%
  \ifcase\SOUL@comma
  \or
    \edef\x{\SOUL@word={\the\SOUL@word,}}\x
  \or
    \edef\x{\SOUL@word={\the\SOUL@word{,,}}}\x
  \fi
  \SOUL@comma\z@
}
%    \end{macrocode}
%    \end{macro}
%    \begin{macro}{\SOUL@flushapo}
%    \begin{macrocode}
\SOuL@redefine\SOUL@flushapo{%
  \ifcase\SOUL@apo
  \or
    \edef\x{\SOUL@word={\the\SOUL@word'}}\x
  \or
    \edef\x{\SOUL@word={\the\SOUL@word{{''}}}}\x
  \fi
  \SOUL@apo\z@
}{%
  \ifcase\SOUL@apo
  \or
    \edef\x{\SOUL@word={\the\SOUL@word'}}\x
  \or
    \edef\x{\SOUL@word={\the\SOUL@word{''}}}\x
  \fi
  \SOUL@apo\z@
}
%    \end{macrocode}
%    \end{macro}
%    \begin{macro}{\SOUL@flushgrave}
%    \begin{macrocode}
\SOuL@redefine\SOUL@flushgrave{%
  \ifcase\SOUL@grave
  \or
    \edef\x{\SOUL@word={\the\SOUL@word`}}\x
  \or
    \edef\x{\SOUL@word={\the\SOUL@word{{``}}}}\x
  \fi
  \SOUL@grave\z@
}{%
  \ifcase\SOUL@grave
  \or
    \edef\x{\SOUL@word={\the\SOUL@word`}}\x
  \or
    \edef\x{\SOUL@word={\the\SOUL@word{``}}}\x
  \fi
  \SOUL@grave\z@
}
%    \end{macrocode}
%    \end{macro}
%    \begin{macro}{\SOUL@addgroup}
%    \begin{macrocode}
\SOuL@redefine\SOUL@addgroup#1#2{%
  {%
    \let\protect\noexpand
    \edef\x{%
      \global\SOUL@word={%
        \the\SOUL@word
        {{\noexpand#1#2}}%
      }%
    }%
    \x
  }%
  \SOUL@scan
}#1#2{%
  \begingroup
    \let\protect\noexpand
    \edef\x{\endgroup
      \SOUL@word={%
        \the\SOUL@word
!       {\noexpand#1{#2}}%
+       {\etex@unexpanded{#1{#2}}}%
      }%
    }%
  \x
  \SOUL@scan
}
%    \end{macrocode}
%    \end{macro}
%    \begin{macro}{\SOUL@addmath}
%    \begin{macrocode}
\SOuL@redefine\SOUL@addmath$#1${%
  {%
    \let\protect\noexpand
    \edef\x{%
      \global\SOUL@word={%
        \the\SOUL@word
        {{\hbox{$#1$}}}%
      }%
    }%
    \x
  }%
  \SOUL@scan
}$#1${%
  \begingroup
    \let\protect\noexpand
    \edef\x{\endgroup
      \SOUL@word={%
        \the\SOUL@word
!       {\hbox{$#1$}}%
+       {\etex@unexpanded{\hbox{$#1$}}}%
      }%
    }%
  \x
  \SOUL@scan
}
%    \end{macrocode}
%    \end{macro}
%    \begin{macro}{\SOUL@addprotect}
%    \begin{macrocode}
\SOuL@redefine\SOUL@addprotect#1#2{%
  {%
    \let\protect\noexpand
    \edef\x{%
      \global\SOUL@word={%
        \the\SOUL@word
        {{\hbox{#2}}}%
      }%
    }%
    \x
  }%
  \SOUL@scan
}#1#2{%
  \begingroup
    \let\protect\noexpand
    \edef\x{\endgroup
      \SOUL@word={%
        \the\SOUL@word
!       {\hbox{#2}}%
+       {\etex@unexpanded{\hbox{#2}}}%
      }%
    }%
  \x
  \SOUL@scan
}
%    \end{macrocode}
%    \end{macro}
%
%    \begin{macro}{\SOUL@addtoken}
%    \begin{macrocode}
+ \SOuL@redefine\SOUL@addtoken#1{%
+   \edef\x{%
+     \SOUL@word={%
+       \the\SOUL@word
+       \noexpand#1%
+     }%
+   }%
+   \x
+   \SOUL@scan
+ }#1{%
+   \edef\x{%
+     \SOUL@word={%
+       \the\SOUL@word
+       \etex@unexpanded{#1}%
+     }%
+   }%
+   \x
+   \SOUL@scan
+ }%
%    \end{macrocode}
%    \end{macro}
%
%    \begin{macrocode}
\SOuL@AtEnd%
%    \end{macrocode}
%
%    \begin{macrocode}
%</package>
%    \end{macrocode}
%
% \section{Test}
%
% \subsection{Catcode checks for loading}
%
%    \begin{macrocode}
%<*test1>
%    \end{macrocode}
%    \begin{macrocode}
\catcode`\{=1 %
\catcode`\}=2 %
\catcode`\#=6 %
\catcode`\@=11 %
\expandafter\ifx\csname count@\endcsname\relax
  \countdef\count@=255 %
\fi
\expandafter\ifx\csname @gobble\endcsname\relax
  \long\def\@gobble#1{}%
\fi
\expandafter\ifx\csname @firstofone\endcsname\relax
  \long\def\@firstofone#1{#1}%
\fi
\expandafter\ifx\csname loop\endcsname\relax
  \expandafter\@firstofone
\else
  \expandafter\@gobble
\fi
{%
  \def\loop#1\repeat{%
    \def\body{#1}%
    \iterate
  }%
  \def\iterate{%
    \body
      \let\next\iterate
    \else
      \let\next\relax
    \fi
    \next
  }%
  \let\repeat=\fi
}%
\def\RestoreCatcodes{}
\count@=0 %
\loop
  \edef\RestoreCatcodes{%
    \RestoreCatcodes
    \catcode\the\count@=\the\catcode\count@\relax
  }%
\ifnum\count@<255 %
  \advance\count@ 1 %
\repeat

\def\RangeCatcodeInvalid#1#2{%
  \count@=#1\relax
  \loop
    \catcode\count@=15 %
  \ifnum\count@<#2\relax
    \advance\count@ 1 %
  \repeat
}
\def\RangeCatcodeCheck#1#2#3{%
  \count@=#1\relax
  \loop
    \ifnum#3=\catcode\count@
    \else
      \errmessage{%
        Character \the\count@\space
        with wrong catcode \the\catcode\count@\space
        instead of \number#3%
      }%
    \fi
  \ifnum\count@<#2\relax
    \advance\count@ 1 %
  \repeat
}
\def\space{ }
\expandafter\ifx\csname LoadCommand\endcsname\relax
  \def\LoadCommand{\input soulutf8.sty\relax}%
\fi
\def\Test{%
  \RangeCatcodeInvalid{0}{47}%
  \RangeCatcodeInvalid{58}{64}%
  \RangeCatcodeInvalid{91}{96}%
  \RangeCatcodeInvalid{123}{255}%
  \catcode`\@=12 %
  \catcode`\\=0 %
  \catcode`\%=14 %
  \LoadCommand
  \RangeCatcodeCheck{0}{36}{15}%
  \RangeCatcodeCheck{37}{37}{14}%
  \RangeCatcodeCheck{38}{47}{15}%
  \RangeCatcodeCheck{48}{57}{12}%
  \RangeCatcodeCheck{58}{63}{15}%
  \RangeCatcodeCheck{64}{64}{12}%
  \RangeCatcodeCheck{65}{90}{11}%
  \RangeCatcodeCheck{91}{91}{15}%
  \RangeCatcodeCheck{92}{92}{0}%
  \RangeCatcodeCheck{93}{96}{15}%
  \RangeCatcodeCheck{97}{122}{11}%
  \RangeCatcodeCheck{123}{255}{15}%
  \RestoreCatcodes
}
\Test
\csname @@end\endcsname
\end
%    \end{macrocode}
%    \begin{macrocode}
%</test1>
%    \end{macrocode}
%
% \subsection{\LaTeX\space tests}
%
%    \begin{macrocode}
%<*test2|test3|test4|test5>
%    \end{macrocode}
%    \begin{macrocode}
%<noetex>\let\SavedUnexpanded\unexpanded
%<noetex>\let\unexpanded\UNDEFINED
\NeedsTeXFormat{LaTeX2e}
\documentclass{article}
\usepackage[T1]{fontenc}
\usepackage{lmodern}
\usepackage{textcomp}
\usepackage[ngerman]{babel}
\expandafter\def\csname ver@hypdestopt.sty\endcsname{}
\usepackage[unicode]{hyperref}
\usepackage{bookmark}
\usepackage{color}
\usepackage{soul}[2003/11/17]
\makeatletter
\newcommand*{\@ifDefined}[1]{%
  \@ifundefined{#1}{%
    \@latex@error{%
      \expandafter\noexpand\csname#1\endcsname undefined%
    }\@ehc
  }%
}
\newcommand*{\SaveOrg}[1]{%
  \@ifdefinable{ORG@SOUL@#1}{%
    \@ifDefined{SOUL@#1}{%
      \expandafter\let\csname ORG@SOUL@#1\expandafter\endcsname
                      \csname SOUL@#1\endcsname
    }%
  }%
}
\newcommand*{\RestoreOrg}[1]{%
  \@ifDefined{ORG@SOUL@#1}{%
    \@ifDefined{SOUL@#1}{%
      \expandafter\let\csname SOUL@#1\expandafter\endcsname
                      \csname ORG@SOUL@#1\endcsname
    }%
  }%
}
\makeatother
\SaveOrg{eval}
\SaveOrg{splittoken}
\SaveOrg{flushcomma}
\SaveOrg{flushapo}
\SaveOrg{grave}
\SaveOrg{addgroup}
\SaveOrg{addmath}
\SaveOrg{addprotect}
\SaveOrg{addtoken}
\usepackage{soulutf8}[2016/05/16]
%    \end{macrocode}
%    \begin{macrocode}
%<noetex>\let\unexpanded\SavedUnexpanded
%    \end{macrocode}
%    Input encoding \xfile{utf8}.
%    \begin{macrocode}
%<*utf8>
\usepackage[utf8]{inputenc}
\DeclareUnicodeCharacter{00A0}{\nobreakspace}
%</utf8>
%    \end{macrocode}
%    Input encoding \xfile{utf8x}. Package \xpackage{ucs}
%    has a nasty bug, that it redefines \cs{show}.
%    \begin{macrocode}
%<*utf8x>
\let\origshow\show
\usepackage[utf8x]{inputenc}
\AtBeginDocument{%
  \let\show\origshow
}
%</utf8x>
%    \end{macrocode}
%
% \subsubsection{Driver definitions}
%
%    \begin{macrocode}
\makeatletter
%    \end{macrocode}
%    \begin{macro}{\an}
%    The definition for \cs{an} is taken from the documentation of
%    package \xpackage{soul}.
%    \begin{macrocode}
\DeclareRobustCommand*{\an}{%
  \def\SOUL@preamble{$^{^P}$}%
  \def\SOUL@everyspace##1{##1\texttt{\char`\ }\linebreak[2]}%
  \def\SOUL@postamble{$^{^E}$}%
  \def\SOUL@everyhyphen{$^{^-}$\linebreak[1]}%
  \def\SOUL@everyexhyphen##1{##1$^{^=}$}%
  \def\SOUL@everysyllable{$^{^S}$}%
  \def\SOUL@everytoken{\the\SOUL@token$^{^T}$}%
  \def\SOUL@everylowerthan{$^{^L}$}%
  \SOUL@
}
%    \end{macrocode}
%    \end{macro}
%
%    \begin{macro}{\sy}
%    The definition for \cs{sy} is taken from the documentation of
%    package \xpackage{soul}.
%    \begin{macrocode}
\DeclareRobustCommand*{\sy}{%
  \SOUL@setup
  \def\SOUL@preamble{%
    \lefthyphenmin=\z@
    \righthyphenmin=\z@
  }%
  \def\SOUL@everyspace##1{##1\space}%
  \def\SOUL@everysyllable{\the\SOUL@syllable}%
  \def\SOUL@everyhyphen{%
    \discretionary{%
      \SOUL@setkern\SOUL@hyphkern
      \SOUL@sethyphenchar
    }{}{%
      \hbox{\kern1pt$\cdot$}%
    }%
  }%
  \def\SOUL@everyexhyphen##1{%
    \SOUL@setkern\SOUL@hyphkern
    \hbox{##1}%
    \discretionary{}{}{%
      \SOUL@setkern\SOUL@charkern
    }%
  }%
  \SOUL@
}
%    \end{macrocode}
%    \end{macro}
%
%    \begin{macro}{\soulH}
%    \begin{macrocode}
\DeclareRobustCommand*{\soulH}{%
  \def\SOUL@preamble{%
    \gdef\SOULH{}%
    \soulHadd{[}%
  }%
  \def\SOUL@everyspace##1{%
    \soulHadd{Sp}%
  }%
  \def\SOUL@postamble{%
    \soulHadd{]}%
  }%
  \def\SOUL@everyhyphen{%
    \soulHadd{Hy}%
  }%
  \def\SOUL@everyexhyphen##1{%
    \soulHadd{Exhy}%
  }%
  \def\SOUL@everysyllable{%
    \soulHadd{Sy}%
  }%
  \def\SOUL@everytoken{%
    \soulHadd{T}%
  }%
  \def\SOUL@everylowerthan{%
    \soulHadd{<}%
  }%
  \SOUL@
}
\newcommand*{\soulHadd}[1]{%
  #1%
  \expandafter\gdef\expandafter\soulHstring\expandafter{%
    \soulHstring
    #1%
  }%
}
%    \end{macrocode}
%    \end{macro}
%
% \subsubsection{Test macros}
%
%    \begin{macrocode}
\usepackage{qstest}
\IncludeTests{*}
\LogTests{log}{*}{*}
%    \end{macrocode}
%    \begin{macrocode}
\newsavebox\boxA
\newsavebox\boxB
\newsavebox\boxC
%    \end{macrocode}
%    \begin{macro}{\TestDriver}
%    \begin{macrocode}
\newcommand*{\TestDriver}[4]{%
  \subsection{Driver #1}%
  \@PackageInfo{soulutf8-test}{* Driver #1}%
  \expandafter\let\expandafter\testdriver\csname #1\endcsname
  \def\teststringA{#3}%
  \def\teststringB{#4}%
  \begin{qstest}{string-#2-driver-#1}{string,#2,driver,#1}%
    \@PackageInfo{soulutf8-test}{* Version: soulutf8}%
    \global\let\soulHstring\@empty
    \setbox0=\hbox{\expandafter\testdriver\expandafter{\teststringA}}%
    \let\HstringA\soulHstring
    \SetTestVBox{\boxA}{\teststringA}%
    %
    \@PackageInfo{soulutf8-test}{* Version: soul + patch}%
    \global\let\soulHstring\@empty
    \RestoreOrg{eval}%
    \setbox2=\hbox{\expandafter\testdriver\expandafter{\teststringB}}%
    \let\HstringB\soulHstring
    \SetTestVBox{\boxB}{\teststringB}%
    %
    \@PackageInfo{soulutf8-test}{* Version: original soul}%
    \global\let\soulHstring\@empty
    \RestoreOrg{splittoken}%
    \RestoreOrg{flushcomma}%
    \RestoreOrg{flushapo}%
    \RestoreOrg{grave}%
    \RestoreOrg{addgroup}%
    \RestoreOrg{addmath}%
    \RestoreOrg{addprotect}%
    \RestoreOrg{addtoken}%
    \setbox4=\hbox{\expandafter\testdriver\expandafter{\teststringB}}%
    \let\HstringC\soulHstring
    \SetTestVBox{\boxC}{\teststringB}%
    %
    \TestInfo{0}{soulutf8}%
    \TestInfo{2}{soul + patch}%
    \TestInfo{4}{original soul}%
    \TestInfoVBox
    \showboxdepth=\maxdimen
    \showboxbreadth=\maxdimen
    \tracingonline=\@ne
    \Expect*{\HstringA}*{\HstringB}%
    \Expect*{\HstringA}*{\HstringC}%
    \Expect*{\the\wd0}*{\the\wd2}%
    \ifdim\wd0=\wd2 %
    \else
      \showbox0 %
      \showbox2 %
    \fi
    \Expect*{\the\ht0}*{\the\ht2}%
    \Expect*{\the\dp0}*{\the\dp2}%
    \Expect*{\the\wd\boxB}*{\the\wd\boxA}%
    \Expect*{\the\wd\boxC}*{\the\wd\boxA}%
    \Expect*{\the\ht\boxB}*{\the\ht\boxA}%
    \ifdim\ht\boxB=\ht\boxA
    \else
      \showbox\boxA
      \showbox\boxB
    \fi
    \Expect*{\the\ht\boxC}*{\the\ht\boxA}%
    \Expect*{\the\dp\boxB}*{\the\dp\boxA}%
    \Expect*{\the\dp\boxC}*{\the\dp\boxA}%
  \end{qstest}%
}
%    \end{macrocode}
%    \end{macro}
%    \begin{macro}{\SetTestVBox}
%    \begin{macrocode}
\newcommand*{\SetTestVBox}[2]{%
  \setbox#1=\vbox{%
    \hsize=0pt\relax
    \hfuzz=100mm\relax
    \expandafter\testdriver\expandafter{#2}%
  }%
}
%    \end{macrocode}
%    \end{macro}
%    \begin{macro}{\TestInfo}
%    \begin{macrocode}
\newcommand*{\TestInfo}[2]{%
  \par
  \noindent
  \copy#1\relax
  \hfill
  \begingroup
    \small
    \ifdim\wd0=\wd2\relax
      \ifdim\wd0=\wd4\relax
      \else
        \ifnum#1=2 %
          \raisebox{.5\baselineskip}[0pt][0pt]{%
            \llap{$\}$ }%
            \rlap{width = \the\wd#1\relax\space$\{$}%
            \phantom{width = 999.99999pt}%
            \qquad
          }%
        \else
          \ifnum#1=4 %
            \rlap{width = \the\wd#1}%
            \phantom{width = 999.99999pt}%
            \qquad
          \fi
        \fi
      \fi
    \else
      \begingroup
        \bfseries
        \rlap{width = \the\wd#1}%
        \phantom{width = 999.99999pt}%
      \endgroup
      \qquad
    \fi
    \rlap{(#2)}%
    \phantom{(original soul)}%
  \endgroup
  \par
}
%    \end{macrocode}
%    \end{macro}
%    \begin{macro}{\TestInfoVBox}
%    \begin{macrocode}
\newcommand*{\TestInfoVBox}{%
  \par
  \noindent
  \mbox{\vrule\relax\space\copy\boxA}%
  \ifdim\ht\boxA=\ht\boxB
  \else
    \hfill
    \begingroup
      \small
      \rlap{(soulutf8)}%
      \phantom{(soulutf8)}%
    \endgroup
    \par
    \noindent
    \mbox{\vrule\relax\space\copy\boxB}%
    \hfill
    \begingroup
      \small
      \rlap{(soul)}%
      \phantom{(soulutf8)}%
    \endgroup
  \fi
  \par
}
%    \end{macrocode}
%    \end{macro}
%    \begin{macro}{\Test}
%    \begin{macrocode}
\makeatletter
\newcommand*{\Test}[3]{%
  \clearpage
  \begingroup
    \def\linebreak{\textbackslash linebreak }%
    \def\\{\textbackslash\textbackslash}%
    \def\soulomit##1{%
      \textbackslash soulomit\textbraceleft##1\textbraceright
    }%
    \def\mbox##1{%
      \textbackslash mbox\textbraceleft##1\textbraceright
    }%
    \def\abc{\textbackslash abc }%
    \def\texttt##1{%
      \textbackslash texttt\textbraceleft##1\textbraceright
    }%
    \def\c##1{%
      \textbackslash c\textbraceleft##1\textbraceright
    }%
%<*utf8x>
    \count@=128 %
    \loop
    \ifnum\count@<256 %
      \lccode`\~=\count@
      \lowercase{%
        \def~{\noexpand~}%
      }%
      \advance\count@\@ne
    \repeat
%</utf8x>
    \csname protected@edef\endcsname\x{\endgroup
      \noexpand\section{String #1: #2}%
    }%
  \x
  \TestDriver{so}{#1}{#2}{#3}%
  \TestDriver{caps}{#1}{#2}{#3}%
  \TestDriver{ul}{#1}{#2}{#3}%
  \TestDriver{st}{#1}{#2}{#3}%
  \TestDriver{hl}{#1}{#2}{#3}%
  \TestDriver{an}{#1}{#2}{#3}%
  \TestDriver{sy}{#1}{#2}{#3}%
  \TestDriver{soulH}{#1}{#2}{#3}%
}
\makeatother
%    \end{macrocode}
%    \end{macro}
%
% \subsubsection{Testing}
%
%    \begin{macrocode}
\begin{document}
  \Test A{}{}%
  \Test B{Hello World}{Hello World}%
  \Test C{%
    Umlauts: ^^c3^^84^^c3^^96^^c3^^9c^^c3^^a4^^c3^^b6^^c3^^bc^^c3^^9f%
  }{%
    Umlauts: \"A\"O\"U\"a\"o\"u\ss
  }%
  \Test D{Euro: ^^e2^^82^^ac}{Euro: \texteuro}%
  \Test E{Stra^^c3^^9fe}{Stra\ss e}%
  \Test F{R^^c3^^b6mer}{R\"omer}%
  \Test G{A^^e2^^80^^93B^^e2^^80^^94C}{A--B---C}%
  \Test H{X^^c2^^a0Y}{X~Y}%
  \Test I{A^^c5^^a4}{A\v T}%
  \Test J{A,B,,C,,,D,,,,E}{A,B,,C,,,D,,,,E}%
  \Test K{A'B''C'''D''''E}{A'B''C'''D''''E}%
  \Test L{A`B``C```D````E}{A`B``C```D````E}%
  \Test M{A$x$B}{A$x$B}%
  \Test N{A\mbox{abc}B}{A\mbox{abc}B}%
  \Test O{,,TA''}{,,TA''}%
  \Test P{{!`}Hola{?`}}{{!`}Hola{?`}}%
  \Test Q{re-sent\slash x}{re-sent\slash x}%
  \Test R{new\\line}{new\\line}%
  \Test S{foo\linebreak bar}{foo\linebreak bar}%
  \Test T{Virchow{sche}}{Virchow{sche}}%
  \Test U{A\soulomit{foo}B}{A\soulomit{foo}B}%
  \Test V{A\texttt{xyz}B}{A\texttt{xyz}B}%
  \Test W{ffi}{ffi}%
  \Test X{f{}f{}i}{f{}f{}i}%
  \Test Y{f\mbox{}f\mbox{}i}{f\mbox{}f\mbox{}i}%
  \Test Z{\c{\"a}}{{\c{\"a}}}%
%<*test2|test3>
  \let\abcx\UNDEFINED
  \def\abc{%
    \begingroup
      \def\abcx{x}%
      \abcx
    \endgroup
  }%
  \Test{AA}{[{x\abc x}]}{[{xxx}]}%
%</test2|test3>
%    \end{macrocode}
%
%    \begin{macrocode}
\end{document}
%    \end{macrocode}
%    \begin{macrocode}
%</test2|test3|test4|test5>
%    \end{macrocode}
%
% \section{Installation}
%
% \subsection{Download}
%
% \paragraph{Package.} This package is available on
% CTAN\footnote{\url{http://ctan.org/pkg/soulutf8}}:
% \begin{description}
% \item[\CTAN{macros/latex/contrib/oberdiek/soulutf8.dtx}] The source file.
% \item[\CTAN{macros/latex/contrib/oberdiek/soulutf8.pdf}] Documentation.
% \end{description}
%
%
% \paragraph{Bundle.} All the packages of the bundle `oberdiek'
% are also available in a TDS compliant ZIP archive. There
% the packages are already unpacked and the documentation files
% are generated. The files and directories obey the TDS standard.
% \begin{description}
% \item[\CTAN{install/macros/latex/contrib/oberdiek.tds.zip}]
% \end{description}
% \emph{TDS} refers to the standard ``A Directory Structure
% for \TeX\ Files'' (\CTAN{tds/tds.pdf}). Directories
% with \xfile{texmf} in their name are usually organized this way.
%
% \subsection{Bundle installation}
%
% \paragraph{Unpacking.} Unpack the \xfile{oberdiek.tds.zip} in the
% TDS tree (also known as \xfile{texmf} tree) of your choice.
% Example (linux):
% \begin{quote}
%   |unzip oberdiek.tds.zip -d ~/texmf|
% \end{quote}
%
% \paragraph{Script installation.}
% Check the directory \xfile{TDS:scripts/oberdiek/} for
% scripts that need further installation steps.
% Package \xpackage{attachfile2} comes with the Perl script
% \xfile{pdfatfi.pl} that should be installed in such a way
% that it can be called as \texttt{pdfatfi}.
% Example (linux):
% \begin{quote}
%   |chmod +x scripts/oberdiek/pdfatfi.pl|\\
%   |cp scripts/oberdiek/pdfatfi.pl /usr/local/bin/|
% \end{quote}
%
% \subsection{Package installation}
%
% \paragraph{Unpacking.} The \xfile{.dtx} file is a self-extracting
% \docstrip\ archive. The files are extracted by running the
% \xfile{.dtx} through \plainTeX:
% \begin{quote}
%   \verb|tex soulutf8.dtx|
% \end{quote}
%
% \paragraph{TDS.} Now the different files must be moved into
% the different directories in your installation TDS tree
% (also known as \xfile{texmf} tree):
% \begin{quote}
% \def\t{^^A
% \begin{tabular}{@{}>{\ttfamily}l@{ $\rightarrow$ }>{\ttfamily}l@{}}
%   soulutf8.sty & tex/generic/oberdiek/soulutf8.sty\\
%   soulutf8.pdf & doc/latex/oberdiek/soulutf8.pdf\\
%   test/soulutf8-test1.tex & doc/latex/oberdiek/test/soulutf8-test1.tex\\
%   test/soulutf8-test2.tex & doc/latex/oberdiek/test/soulutf8-test2.tex\\
%   test/soulutf8-test3.tex & doc/latex/oberdiek/test/soulutf8-test3.tex\\
%   test/soulutf8-test4.tex & doc/latex/oberdiek/test/soulutf8-test4.tex\\
%   test/soulutf8-test5.tex & doc/latex/oberdiek/test/soulutf8-test5.tex\\
%   soulutf8.dtx & source/latex/oberdiek/soulutf8.dtx\\
% \end{tabular}^^A
% }^^A
% \sbox0{\t}^^A
% \ifdim\wd0>\linewidth
%   \begingroup
%     \advance\linewidth by\leftmargin
%     \advance\linewidth by\rightmargin
%   \edef\x{\endgroup
%     \def\noexpand\lw{\the\linewidth}^^A
%   }\x
%   \def\lwbox{^^A
%     \leavevmode
%     \hbox to \linewidth{^^A
%       \kern-\leftmargin\relax
%       \hss
%       \usebox0
%       \hss
%       \kern-\rightmargin\relax
%     }^^A
%   }^^A
%   \ifdim\wd0>\lw
%     \sbox0{\small\t}^^A
%     \ifdim\wd0>\linewidth
%       \ifdim\wd0>\lw
%         \sbox0{\footnotesize\t}^^A
%         \ifdim\wd0>\linewidth
%           \ifdim\wd0>\lw
%             \sbox0{\scriptsize\t}^^A
%             \ifdim\wd0>\linewidth
%               \ifdim\wd0>\lw
%                 \sbox0{\tiny\t}^^A
%                 \ifdim\wd0>\linewidth
%                   \lwbox
%                 \else
%                   \usebox0
%                 \fi
%               \else
%                 \lwbox
%               \fi
%             \else
%               \usebox0
%             \fi
%           \else
%             \lwbox
%           \fi
%         \else
%           \usebox0
%         \fi
%       \else
%         \lwbox
%       \fi
%     \else
%       \usebox0
%     \fi
%   \else
%     \lwbox
%   \fi
% \else
%   \usebox0
% \fi
% \end{quote}
% If you have a \xfile{docstrip.cfg} that configures and enables \docstrip's
% TDS installing feature, then some files can already be in the right
% place, see the documentation of \docstrip.
%
% \subsection{Refresh file name databases}
%
% If your \TeX~distribution
% (\teTeX, \mikTeX, \dots) relies on file name databases, you must refresh
% these. For example, \teTeX\ users run \verb|texhash| or
% \verb|mktexlsr|.
%
% \subsection{Some details for the interested}
%
% \paragraph{Attached source.}
%
% The PDF documentation on CTAN also includes the
% \xfile{.dtx} source file. It can be extracted by
% AcrobatReader 6 or higher. Another option is \textsf{pdftk},
% e.g. unpack the file into the current directory:
% \begin{quote}
%   \verb|pdftk soulutf8.pdf unpack_files output .|
% \end{quote}
%
% \paragraph{Unpacking with \LaTeX.}
% The \xfile{.dtx} chooses its action depending on the format:
% \begin{description}
% \item[\plainTeX:] Run \docstrip\ and extract the files.
% \item[\LaTeX:] Generate the documentation.
% \end{description}
% If you insist on using \LaTeX\ for \docstrip\ (really,
% \docstrip\ does not need \LaTeX), then inform the autodetect routine
% about your intention:
% \begin{quote}
%   \verb|latex \let\install=y\input{soulutf8.dtx}|
% \end{quote}
% Do not forget to quote the argument according to the demands
% of your shell.
%
% \paragraph{Generating the documentation.}
% You can use both the \xfile{.dtx} or the \xfile{.drv} to generate
% the documentation. The process can be configured by the
% configuration file \xfile{ltxdoc.cfg}. For instance, put this
% line into this file, if you want to have A4 as paper format:
% \begin{quote}
%   \verb|\PassOptionsToClass{a4paper}{article}|
% \end{quote}
% An example follows how to generate the
% documentation with pdf\LaTeX:
% \begin{quote}
%\begin{verbatim}
%pdflatex soulutf8.dtx
%makeindex -s gind.ist soulutf8.idx
%pdflatex soulutf8.dtx
%makeindex -s gind.ist soulutf8.idx
%pdflatex soulutf8.dtx
%\end{verbatim}
% \end{quote}
%
% \begin{thebibliography}{9}
%
% \bibitem{soul}
%   Melchior Franz: \textit{The \xpackage{soul} package};
%   2003/11/17;\\
%   \CTAN{macros/latex/contrib/soul/}.
%
% \bibitem{ucs}
%   Dominique P. G. Unruh: \textit{\xpackage{ucs.sty} -- Unicode Support};
%   2004/10/17;\\
%   \CTAN{macros/latex/contrib/unicode/}.
%
% \bibitem{utf8}
%   Frank Mittelbach, Chris Rowley:
%   \textit{Providing some UTF-8 support via \xpackage{inputenc}};
%   2006/03/30;\\
%   \CTAN{macros/latex/base/utf8ienc.dtx}.
%
% \end{thebibliography}
%
% \begin{History}
%   \begin{Version}{2007/09/09 v1.0}
%   \item
%     First version.
%   \end{Version}
%   \begin{Version}{2016/05/16 v1.1}
%   \item
%     Documentation updates.
%   \end{Version}
% \end{History}
%
% \PrintIndex
%
% \Finale
\endinput

%        (quote the arguments according to the demands of your shell)
%
% Documentation:
%    (a) If soulutf8.drv is present:
%           latex soulutf8.drv
%    (b) Without soulutf8.drv:
%           latex soulutf8.dtx; ...
%    The class ltxdoc loads the configuration file ltxdoc.cfg
%    if available. Here you can specify further options, e.g.
%    use A4 as paper format:
%       \PassOptionsToClass{a4paper}{article}
%
%    Programm calls to get the documentation (example):
%       pdflatex soulutf8.dtx
%       makeindex -s gind.ist soulutf8.idx
%       pdflatex soulutf8.dtx
%       makeindex -s gind.ist soulutf8.idx
%       pdflatex soulutf8.dtx
%
% Installation:
%    TDS:tex/generic/oberdiek/soulutf8.sty
%    TDS:doc/latex/oberdiek/soulutf8.pdf
%    TDS:doc/latex/oberdiek/test/soulutf8-test1.tex
%    TDS:doc/latex/oberdiek/test/soulutf8-test2.tex
%    TDS:doc/latex/oberdiek/test/soulutf8-test3.tex
%    TDS:doc/latex/oberdiek/test/soulutf8-test4.tex
%    TDS:doc/latex/oberdiek/test/soulutf8-test5.tex
%    TDS:source/latex/oberdiek/soulutf8.dtx
%
%<*ignore>
\begingroup
  \catcode123=1 %
  \catcode125=2 %
  \def\x{LaTeX2e}%
\expandafter\endgroup
\ifcase 0\ifx\install y1\fi\expandafter
         \ifx\csname processbatchFile\endcsname\relax\else1\fi
         \ifx\fmtname\x\else 1\fi\relax
\else\csname fi\endcsname
%</ignore>
%<*install>
\input docstrip.tex
\Msg{************************************************************************}
\Msg{* Installation}
\Msg{* Package: soulutf8 2016/05/16 v1.1 Permit use of UTF-8 characters in soul (HO)}
\Msg{************************************************************************}

\keepsilent
\askforoverwritefalse

\let\MetaPrefix\relax
\preamble

This is a generated file.

Project: soulutf8
Version: 2016/05/16 v1.1

Copyright (C) 2007 by
   Heiko Oberdiek <heiko.oberdiek at googlemail.com>

This work may be distributed and/or modified under the
conditions of the LaTeX Project Public License, either
version 1.3c of this license or (at your option) any later
version. This version of this license is in
   http://www.latex-project.org/lppl/lppl-1-3c.txt
and the latest version of this license is in
   http://www.latex-project.org/lppl.txt
and version 1.3 or later is part of all distributions of
LaTeX version 2005/12/01 or later.

This work has the LPPL maintenance status "maintained".

This Current Maintainer of this work is Heiko Oberdiek.

The Base Interpreter refers to any `TeX-Format',
because some files are installed in TDS:tex/generic//.

This work consists of the main source file soulutf8.dtx
and the derived files
   soulutf8.sty, soulutf8.pdf, soulutf8.ins, soulutf8.drv,
   soulutf8-test1.tex, soulutf8-test2.tex, soulutf8-test3.tex,
   soulutf8-test4.tex, soulutf8-test5.tex.

\endpreamble
\let\MetaPrefix\DoubleperCent

\generate{%
  \file{soulutf8.ins}{\from{soulutf8.dtx}{install}}%
  \file{soulutf8.drv}{\from{soulutf8.dtx}{driver}}%
  \usedir{tex/generic/oberdiek}%
  \file{soulutf8.sty}{\from{soulutf8.dtx}{package}}%
%  \usedir{doc/latex/oberdiek/test}%
%  \file{soulutf8-test1.tex}{\from{soulutf8.dtx}{test1}}%
%  \file{soulutf8-test2.tex}{\from{soulutf8.dtx}{test2,utf8}}%
%  \file{soulutf8-test3.tex}{\from{soulutf8.dtx}{test3,utf8x}}%
%  \file{soulutf8-test4.tex}{\from{soulutf8.dtx}{test4,utf8,noetex}}%
%  \file{soulutf8-test5.tex}{\from{soulutf8.dtx}{test5,utf8x,noetex}}%
  \nopreamble
  \nopostamble
%  \usedir{source/latex/oberdiek/catalogue}%
%  \file{soulutf8.xml}{\from{soulutf8.dtx}{catalogue}}%
}

\catcode32=13\relax% active space
\let =\space%
\Msg{************************************************************************}
\Msg{*}
\Msg{* To finish the installation you have to move the following}
\Msg{* file into a directory searched by TeX:}
\Msg{*}
\Msg{*     soulutf8.sty}
\Msg{*}
\Msg{* To produce the documentation run the file `soulutf8.drv'}
\Msg{* through LaTeX.}
\Msg{*}
\Msg{* Happy TeXing!}
\Msg{*}
\Msg{************************************************************************}

\endbatchfile
%</install>
%<*ignore>
\fi
%</ignore>
%<*driver>
\NeedsTeXFormat{LaTeX2e}
\ProvidesFile{soulutf8.drv}%
  [2016/05/16 v1.1 Permit use of UTF-8 characters in soul (HO)]%
\documentclass{ltxdoc}
\usepackage{holtxdoc}[2011/11/22]
\begin{document}
  \DocInput{soulutf8.dtx}%
\end{document}
%</driver>
% \fi
%
%
% \CharacterTable
%  {Upper-case    \A\B\C\D\E\F\G\H\I\J\K\L\M\N\O\P\Q\R\S\T\U\V\W\X\Y\Z
%   Lower-case    \a\b\c\d\e\f\g\h\i\j\k\l\m\n\o\p\q\r\s\t\u\v\w\x\y\z
%   Digits        \0\1\2\3\4\5\6\7\8\9
%   Exclamation   \!     Double quote  \"     Hash (number) \#
%   Dollar        \$     Percent       \%     Ampersand     \&
%   Acute accent  \'     Left paren    \(     Right paren   \)
%   Asterisk      \*     Plus          \+     Comma         \,
%   Minus         \-     Point         \.     Solidus       \/
%   Colon         \:     Semicolon     \;     Less than     \<
%   Equals        \=     Greater than  \>     Question mark \?
%   Commercial at \@     Left bracket  \[     Backslash     \\
%   Right bracket \]     Circumflex    \^     Underscore    \_
%   Grave accent  \`     Left brace    \{     Vertical bar  \|
%   Right brace   \}     Tilde         \~}
%
% \GetFileInfo{soulutf8.drv}
%
% \title{The \xpackage{soulutf8} package}
% \date{2016/05/16 v1.1}
% \author{Heiko Oberdiek\thanks
% {Please report any issues at https://github.com/ho-tex/oberdiek/issues}\\
% \xemail{heiko.oberdiek at googlemail.com}}
%
% \maketitle
%
% \begin{abstract}
% This package extends package \xpackage{soul} and adds some support
% for UTF-8. Namely the input encodings \xfile{utf8.def}
% from package \xpackage{inputenc} and
% package \xpackage{ucs}'s \xfile{utf8x.def} are supported.
% \end{abstract}
%
% \tableofcontents
%
% \section{Documentation}
%
% This package \xpackage{soulutf8} does not have own options and does not
% define new user commands.
% Any option is passed to package \xpackage{soul} \cite{soul}
% that is loaded first.
% Then some internal macros of \xpackage{soul} are redefined to add
% support for UTF-8.
% The following input encodings are supported:
% \begin{quote}
%   \begin{tabular}{@{}lll@{}}
%     \texttt{utf8} & \LaTeX\ base &
%        \xfile{TDS:tex/latex/base/utf8.def} \cite{utf8}\\
%     \texttt{utf8x} & Package \xpackage{ucs} &
%       \xfile{TDS:tex/latex/ucs/utf8x.def} \cite{ucs}\\
%   \end{tabular}
% \end{quote}
% UTF-8 byte sequences are added as token group to a word, even
% if these UTF-8 characters are some kind of hyphen or space.
% As exception the following three Unicode characters are handled specially:
% \begin{quote}
%   \begin{tabular}{lll}
%     Slot & Name & Action\\
%     \hline
%     \texttt{U+00A0} & NO-BREAK SPACE & like |~|\\
%     \texttt{U+2013} & EN DASH & |--|\\
%     \texttt{U+2014} & EM DASH & |---|
%   \end{tabular}
%
% \subsection{Patch}
%
% Also package \xpackage{soulutf8} tries to patch package \xpackage{soul}
% to improve its behaviour:
% \begin{itemize}
% \item
%   A problem with additional levels of curly braces is fixed.
%   As advantage more implicite kernings are detected.
%   However, the result may be incompatible with the
%   original behaviour of package \xpackage{soul} because
%   of these respected implicite kernings.
% \item
%   \eTeX\ , especially \cs{unexpanded} is supported.
%   This allows a better protection of token groups
%   (|\mbox{|\dots|}|, math, \dots).
% \end{itemize}
% \end{quote}
%
% \subsection{Future}
%
% Currently package \xpackage{soul} does not seem to be
% maintained. Nevertheless if there will be a new version that
% adds support for UTF-8, then this package may become
% obsolete.
%
% \StopEventually{
% }
%
% \section{Implementation}
%
%    \begin{macrocode}
%<*package>
%    \end{macrocode}
%
% \subsection{Reload check and package identification}
%    Reload check, especially if the package is not used with \LaTeX.
%    \begin{macrocode}
\begingroup\catcode61\catcode48\catcode32=10\relax%
  \catcode13=5 % ^^M
  \endlinechar=13 %
  \catcode35=6 % #
  \catcode39=12 % '
  \catcode44=12 % ,
  \catcode45=12 % -
  \catcode46=12 % .
  \catcode58=12 % :
  \catcode64=11 % @
  \catcode123=1 % {
  \catcode125=2 % }
  \expandafter\let\expandafter\x\csname ver@soulutf8.sty\endcsname
  \ifx\x\relax % plain-TeX, first loading
  \else
    \def\empty{}%
    \ifx\x\empty % LaTeX, first loading,
      % variable is initialized, but \ProvidesPackage not yet seen
    \else
      \expandafter\ifx\csname PackageInfo\endcsname\relax
        \def\x#1#2{%
          \immediate\write-1{Package #1 Info: #2.}%
        }%
      \else
        \def\x#1#2{\PackageInfo{#1}{#2, stopped}}%
      \fi
      \x{soulutf8}{The package is already loaded}%
      \aftergroup\endinput
    \fi
  \fi
\endgroup%
%    \end{macrocode}
%    Package identification:
%    \begin{macrocode}
\begingroup\catcode61\catcode48\catcode32=10\relax%
  \catcode13=5 % ^^M
  \endlinechar=13 %
  \catcode35=6 % #
  \catcode39=12 % '
  \catcode40=12 % (
  \catcode41=12 % )
  \catcode44=12 % ,
  \catcode45=12 % -
  \catcode46=12 % .
  \catcode47=12 % /
  \catcode58=12 % :
  \catcode64=11 % @
  \catcode91=12 % [
  \catcode93=12 % ]
  \catcode123=1 % {
  \catcode125=2 % }
  \expandafter\ifx\csname ProvidesPackage\endcsname\relax
    \def\x#1#2#3[#4]{\endgroup
      \immediate\write-1{Package: #3 #4}%
      \xdef#1{#4}%
    }%
  \else
    \def\x#1#2[#3]{\endgroup
      #2[{#3}]%
      \ifx#1\@undefined
        \xdef#1{#3}%
      \fi
      \ifx#1\relax
        \xdef#1{#3}%
      \fi
    }%
  \fi
\expandafter\x\csname ver@soulutf8.sty\endcsname
\ProvidesPackage{soulutf8}%
  [2016/05/16 v1.1 Permit use of UTF-8 characters in soul (HO)]%
%    \end{macrocode}
%
% \subsection{Catcodes}
%
%    \begin{macrocode}
\begingroup\catcode61\catcode48\catcode32=10\relax%
  \catcode13=5 % ^^M
  \endlinechar=13 %
  \catcode123=1 % {
  \catcode125=2 % }
  \catcode64=11 % @
  \def\x{\endgroup
    \expandafter\edef\csname SOuL@AtEnd\endcsname{%
      \endlinechar=\the\endlinechar\relax
      \catcode13=\the\catcode13\relax
      \catcode32=\the\catcode32\relax
      \catcode35=\the\catcode35\relax
      \catcode61=\the\catcode61\relax
      \catcode64=\the\catcode64\relax
      \catcode123=\the\catcode123\relax
      \catcode125=\the\catcode125\relax
    }%
  }%
\x\catcode61\catcode48\catcode32=10\relax%
\catcode13=5 % ^^M
\endlinechar=13 %
\catcode35=6 % #
\catcode64=11 % @
\catcode123=1 % {
\catcode125=2 % }
\def\TMP@EnsureCode#1#2{%
  \edef\SOuL@AtEnd{%
    \SOuL@AtEnd
    \catcode#1=\the\catcode#1\relax
  }%
  \catcode#1=#2\relax
}
\TMP@EnsureCode{10}{12}% ^^J
\TMP@EnsureCode{33}{12}% !
\TMP@EnsureCode{34}{12}% "
\TMP@EnsureCode{36}{3}% $
\TMP@EnsureCode{39}{12}% '
\TMP@EnsureCode{40}{12}% (
\TMP@EnsureCode{41}{12}% )
\TMP@EnsureCode{42}{12}% *
\TMP@EnsureCode{43}{12}% +
\TMP@EnsureCode{44}{12}% ,
\TMP@EnsureCode{45}{12}% -
\TMP@EnsureCode{46}{12}% .
\TMP@EnsureCode{47}{12}% /
\TMP@EnsureCode{58}{12}% :
\TMP@EnsureCode{60}{12}% <
\TMP@EnsureCode{62}{12}% >
\TMP@EnsureCode{91}{12}% [
\TMP@EnsureCode{93}{12}% ]
\TMP@EnsureCode{94}{7}% ^
\TMP@EnsureCode{96}{12}% `
\TMP@EnsureCode{126}\active % ~
\TMP@EnsureCode{128}{12}% ^^80
\TMP@EnsureCode{147}{12}% ^^93
\TMP@EnsureCode{148}{12}% ^^94
\TMP@EnsureCode{160}{12}% ^^a0
\TMP@EnsureCode{194}{12}% ^^c2
\TMP@EnsureCode{226}{12}% ^^e2
\edef\SOuL@AtEnd{\SOuL@AtEnd\noexpand\endinput}
%    \end{macrocode}
%
% \subsection{Loading packages}
%
%    Package \xpackage{soul} uses \cs{documentclass} to detect \LaTeX.
%    \begin{macrocode}
\ifx\documentclass\@undefined
%    \end{macrocode}
%
% \subsubsection{\plainTeX}
%
%    First we check, whether package \xpackage{soul} is already loaded.
%    \begin{macrocode}
  \expandafter\ifx\csname SOUL@\endcsname\relax
%    \end{macrocode}
%    In case of \plainTeX\ package \xpackage{soul} defines some
%    macros in a simple manner that will break the definitions of
%    \xfile{miniltx.tex}, for example. Therefore these macros
%    are first saved and restored afterwards.
%    \begin{macrocode}
    \let\SOuL@orgDeclareRobustCommand\DeclareRobustCommand
    \let\SOuL@orgnewcommand          \newcommand
    \let\SOuL@orgDeclareOption       \DeclareOption
    \let\SOuL@orgPackageError        \PackageError
    \def\SOuL@restorelatexcmds{%
      \let\DeclareRobustCommand\SOuL@orgDeclareRobustCommand
      \let\newcommand          \SOuL@orgnewcommand
      \let\DeclareOption       \SOuL@orgDeclareOption
      \let\PackageError        \SOuL@orgPackageError
    }%
    \input soul.sty\relax
    \SOuL@restorelatexcmds
  \fi
%    \end{macrocode}
%    \begin{macro}{\SOUL@error}
%     Package \xpackage{soul}'s use of \cs{PackageError} is replaced
%     by \cs{@PackageError} of package \xpackage{infwarerr}.
%    \begin{macrocode}
  \input infwarerr.sty\relax
  \let\SOuL@orgSOUL@error\SOUL@error
  \def\SOUL@error{%
    \begingroup
      \let\PackageError\@PackageError
      \SOuL@orgSOUL@error
    \endgroup
  }%
%    \end{macrocode}
%    \end{macro}
%    \begin{macrocode}
  \input etexcmds.sty\relax
%    \end{macrocode}
%    \begin{macro}{\@onelevel@sanitize}
%    Define \LaTeX's \cs{@onelevel@sanitize} if not already available.
%    \begin{macrocode}
  \expandafter\ifx\csname @onelevel@sanitize\endcsname\relax
    \def\@onelevel@sanitize#1{%
      \edef#1{%
        \expandafter\strip@prefix\meaning#1%
      }%
    }%
%    \end{macrocode}
%    \end{macro}
%    \begin{macro}{\strip@prefix}
%    \begin{macrocode}
    \def\strip@prefix#1>{}%
%    \end{macrocode}
%    \end{macro}
%    \begin{macrocode}
  \fi
\else
%    \end{macrocode}
%
% \subsubsection{\LaTeX}
%    \begin{macrocode}
  \DeclareOption*{\PassOptionsToPackage{\CurrentOption}{soul}}%
  \ProcessOptions\relax
  \RequirePackage{soul}[2003/11/17]%
  \RequirePackage{infwarerr}[2016/05/16]%
  \RequirePackage{etexcmds}[2016/05/16]%
\fi
%    \end{macrocode}
%
% \subsubsection{\eTeX}
%
%    In \plainTeX\ command \cs{+} is an \emph{outer} macro.
%    Therefore numbers are used to avoid problems.
%    \begin{macrocode}
\ifetex@unexpanded
  \catcode33=14 % '!': comment
  \catcode43=9  % '+': ignore
\else
  \catcode33=9  % '!': ignore
  \catcode43=14 % '+': comment
\fi
%    \end{macrocode}
%
% \subsection{Macro for redefinitions}
%
%    \begin{macro}{\SOuL@redefine}
%    \begin{macrocode}
\def\SOuL@redefine#1{%
  \begingroup
    \def\SOuL@cmd{#1}%
    \afterassignment\SOuL@cmdcheck
    \def\SOuL@temp
}
%    \end{macrocode}
%    \end{macro}
%    \begin{macro}{\SOuL@cmdcheck}
%    \begin{macrocode}
\def\SOuL@cmdcheck{%
    \expandafter\ifx\SOuL@cmd\SOuL@temp
    \else
      \edef\SOuL@temp*{\expandafter\string\SOuL@cmd}%
      \@PackageWarningNoLine{soulutf8}{%
        Command \SOuL@temp* has changed.\MessageBreak
        Supported versions of package `soul': 2003/11/17.\MessageBreak
        Depending on the unknown changes the redefinition\MessageBreak
        of \SOuL@temp* may not behave correctly%
      }%
    \fi
  \expandafter\endgroup
  \expandafter\def\SOuL@cmd
}
%    \end{macrocode}
%    \end{macro}
%
% \subsection{Redefinition of \cs{SOUL@eval}}
%
%    \begin{macro}{\SOUL@eval}
%    Macro \cs{SOUL@eval} is redefined to add detection of
%    the first byte of a UTF-8 sequence. Because \cs{SOUL@eval}
%    is overwritten, a warning is issued, if the contents of
%    \cs{SOUL@eval} is not as expected.
%    \begin{macrocode}
\SOuL@redefine\SOUL@eval{%
%    \end{macrocode}
%    First the expected definition.
%    \begin{macrocode}
  \def\SOUL@n*##1{\SOUL@scan}%
  \if\noexpand\SOUL@@\SOUL@spc
  \else
    \SOUL@ignorespacesfalse
  \fi
  \ifnum\SOUL@minus=\thr@@
    \SOUL@flushminus
  \else\ifnum\SOUL@comma=\tw@
    \SOUL@flushcomma
  \else\ifnum\SOUL@apo=\tw@
    \SOUL@flushapo
  \else\ifnum\SOUL@grave=\tw@
    \SOUL@flushgrave
  \fi\fi\fi\fi
  \ifx\SOUL@@-\else\SOUL@flushminus\fi
  \ifx\SOUL@@,\else\SOUL@flushcomma\fi
  \ifx\SOUL@@'\else\SOUL@flushapo\fi
  \ifx\SOUL@@`\else\SOUL@flushgrave\fi
  \ifx\SOUL@@-%
    \advance\SOUL@minus\@ne
  \else\ifx\SOUL@@,%
    \advance\SOUL@comma\@ne
  \else\ifx\SOUL@@'%
    \advance\SOUL@apo\@ne
  \else\ifx\SOUL@@`%
    \advance\SOUL@grave\@ne
  \else
    \SOUL@flushminus
    \SOUL@flushcomma
    \SOUL@flushapo
    \SOUL@flushgrave
    \ifx\SOUL@@\SOUL@stop
      \def\SOUL@n*{%
        \SOUL@doword
        \SOUL@eventuallyexhyphen\null
      }%
    \else\ifx\SOUL@@\par
      \def\SOUL@n*\par{\par\leavevmode\SOUL@scan}%
    \else\if\noexpand\SOUL@@\SOUL@spc
      \SOUL@doword
      \SOUL@eventuallyexhyphen\null
      \ifSOUL@ignorespaces
      \else
        \SOUL@everyspace{}%
      \fi
      \def\SOUL@n* {\SOUL@scan}%
    \else\ifx\SOUL@@\\%
      \SOUL@doword
      \SOUL@eventuallyexhyphen\null
      \SOUL@everyspace{\unskip\nobreak\hfil\break}%
      \SOUL@ignorespacestrue
    \else\ifx\SOUL@@~%
      \SOUL@doword
      \SOUL@eventuallyexhyphen\null
      \SOUL@everyspace{\nobreak}%
    \else\ifx\SOUL@@\slash
      \SOUL@doword
      \SOUL@eventuallyexhyphen{/}%
      \SOUL@exhyphen{/}%
    \else\ifx\SOUL@@\mbox
      \def\SOUL@n*{\SOUL@addprotect}%
    \else\ifx\SOUL@@\hbox
      \def\SOUL@n*{\SOUL@addprotect}%
    \else\ifx\SOUL@@\soulomit
      \def\SOUL@n*\soulomit##1{%
        \SOUL@doword
        {\spaceskip\SOUL@spaceskip##1}%
        \SOUL@scan
      }%
    \else\ifx\SOUL@@\break
      \SOUL@doword
      \break
    \else\ifx\SOUL@@\linebreak
      \SOUL@doword
      \SOUL@everyspace{\linebreak}%
    \else\ifcat\bgroup\noexpand\SOUL@@
      \def\SOUL@n*{\SOUL@addgroup{}}%
    \else\ifcat$\noexpand\SOUL@@
      \def\SOUL@n*{\SOUL@addmath}%
    \else
      \def\SOUL@n*{\SOUL@dotoken}%
    \fi\fi\fi\fi\fi\fi\fi\fi\fi\fi\fi\fi\fi
  \fi\fi\fi\fi
  \SOUL@n*%
}{%
%    \end{macrocode}
%    Now the redefined version follows.
%    \begin{macrocode}
  \def\SOUL@n*##1{\SOUL@scan}%
  \if\noexpand\SOUL@@\SOUL@spc
  \else
    \SOUL@ignorespacesfalse
  \fi
  \ifnum\SOUL@minus=\thr@@
    \SOUL@flushminus
  \else\ifnum\SOUL@comma=\tw@
    \SOUL@flushcomma
  \else\ifnum\SOUL@apo=\tw@
    \SOUL@flushapo
  \else\ifnum\SOUL@grave=\tw@
    \SOUL@flushgrave
  \fi\fi\fi\fi
  \ifx\SOUL@@-\else\SOUL@flushminus\fi
  \ifx\SOUL@@,\else\SOUL@flushcomma\fi
  \ifx\SOUL@@'\else\SOUL@flushapo\fi
  \ifx\SOUL@@`\else\SOUL@flushgrave\fi
  \ifx\SOUL@@-%
    \advance\SOUL@minus\@ne
  \else\ifx\SOUL@@,%
    \advance\SOUL@comma\@ne
  \else\ifx\SOUL@@'%
    \advance\SOUL@apo\@ne
  \else\ifx\SOUL@@`%
    \advance\SOUL@grave\@ne
  \else
    \SOUL@flushminus
    \SOUL@flushcomma
    \SOUL@flushapo
    \SOUL@flushgrave
    \ifx\SOUL@@\SOUL@stop
      \def\SOUL@n*{%
          \SOUL@doword
          \SOUL@eventuallyexhyphen\null
      }%
    \else\ifx\SOUL@@\par
      \def\SOUL@n*\par{\par\leavevmode\SOUL@scan}%
    \else\if\noexpand\SOUL@@\SOUL@spc
      \SOUL@doword
      \SOUL@eventuallyexhyphen\null
      \ifSOUL@ignorespaces
      \else
        \SOUL@everyspace{}%
      \fi
      \def\SOUL@n* {\SOUL@scan}%
    \else\ifx\SOUL@@\\%
      \SOUL@doword
      \SOUL@eventuallyexhyphen\null
      \SOUL@everyspace{\unskip\nobreak\hfil\break}%
      \SOUL@ignorespacestrue
    \else\ifx\SOUL@@~%
      \SOUL@doword
      \SOUL@eventuallyexhyphen\null
      \SOUL@everyspace{\nobreak}%
    \else\ifx\SOUL@@\slash
      \SOUL@doword
      \SOUL@eventuallyexhyphen{/}%
      \SOUL@exhyphen{/}%
    \else\ifx\SOUL@@\mbox
      \def\SOUL@n*{\SOUL@addprotect}%
    \else\ifx\SOUL@@\hbox
      \def\SOUL@n*{\SOUL@addprotect}%
    \else\ifx\SOUL@@\soulomit
      \def\SOUL@n*\soulomit##1{%
        \SOUL@doword
        {\spaceskip\SOUL@spaceskip##1}%
        \SOUL@scan
      }%
    \else\ifx\SOUL@@\break
      \SOUL@doword
      \break
    \else\ifx\SOUL@@\linebreak
      \SOUL@doword
      \SOUL@everyspace{\linebreak}%
    \else\ifcat\bgroup\noexpand\SOUL@@
      \def\SOUL@n*{\SOUL@addgroup{}}%
    \else\ifcat$\noexpand\SOUL@@
      \def\SOUL@n*{\SOUL@addmath}%
    \else
%    \end{macrocode}
%    The current token is examined to detect the start of
%    a UTF-8 sequence.
%    \begin{macrocode}
      \SOuL@analyzeutfviii
      \ifcase\SOuL@octets
        \SOuL@analyzeutfviiix
      \fi
      \ifcase\SOuL@octets
        \def\SOUL@n*{\SOUL@dotoken}%
      \or % 1
      \or % 2
        \def\SOUL@n*{\SOuL@addtwooctets}%
      \or % 3
        \def\SOUL@n*{\SOuL@addthreeoctets}%
      \or % 4
        \def\SOUL@n*{\SOuL@addfouroctets}%
      \fi
    \fi\fi\fi\fi\fi\fi\fi\fi\fi\fi\fi\fi\fi
  \fi\fi\fi\fi
  \SOUL@n*%
}
%    \end{macrocode}
%    \end{macro}
%
% \subsection{UTF-8 analysis}
%
% \subsubsection{Help strings}
%
%    \begin{macrocode}
\def\SOuL@defsanitizedstring#1#2{%
  \expandafter\def\csname SOuL@string#1\endcsname{#2}%
  \expandafter\@onelevel@sanitize\csname SOuL@string#1\endcsname
}
\SOuL@defsanitizedstring{UTFviii}{UTFviii@}
\SOuL@defsanitizedstring{octets}{@octets}
\SOuL@defsanitizedstring{two}{two}
\SOuL@defsanitizedstring{three}{three}
\SOuL@defsanitizedstring{four}{four}
\SOuL@defsanitizedstring{macrocolon}{macro:}
\SOuL@defsanitizedstring{csnameu}{csname u8-}
\SOuL@defsanitizedstring{undeferr}{utf@viii@undeferr}
\def\SOuL@stringendash{^^e2^^80^^93}
\def\SOuL@stringemdash{^^e2^^80^^94}
\def\SOuL@stringnobreakspace{^^c2^^a0}
\edef\SOuL@charhash{\string #}
\edef\SOuL@chartwo{\string 2}
\edef\SOuL@charthree{\string 3}
\def\SOuL@empty{}
%    \end{macrocode}
%
% \subsubsection{Support for \xfile{utf8.def}}
%
%    \begin{macro}{\SOuL@analyzeutfviii}
%    \begin{macrocode}
\begingroup
  \edef\x{\endgroup
    \def\noexpand\SOuL@analyzeutfviii{%
      \noexpand\expandafter\noexpand\SOuL@checkutfviii
      \noexpand\meaning\noexpand\SOUL@@
      \SOuL@stringUTFviii\SOuL@stringoctets
      \noexpand\@nil
    }%
    \def\noexpand\SOuL@checkutfviii
      ##1\SOuL@stringUTFviii##2\SOuL@stringoctets##3\noexpand\@nil
  }%
\x{%
  \def\SOuL@temp{#2}%
  \chardef\SOuL@octets=%
      \ifx\SOuL@temp\SOuL@stringtwo
        \tw@
      \else\ifx\SOuL@temp\SOuL@stringthree
        \thr@@
      \else\ifx\SOuL@temp\SOuL@stringfour
        4 %
      \else
        \z@
      \fi\fi\fi
}
%    \end{macrocode}
%    \end{macro}
%
% \subsubsection{Support for \xfile{utf8x.def}}
%
%    \begin{macro}{\SOuL@analyzeutfviiix}
%    \begin{macrocode}
\begingroup
  \edef\x{\endgroup
    \def\noexpand\SOuL@analyzeutfviiix{%
      \noexpand\expandafter\noexpand\SOuL@checkutfviiix
      \noexpand\meaning\noexpand\SOUL@@
      \SOuL@stringmacrocolon\SOuL@charhash1{}{}{}{}%
      \SOuL@stringcsnameu\SOuL@stringundeferr
      \noexpand\@nil
    }%
%    \end{macrocode}
%    \end{macro}
%    \begin{macro}{\SOuL@checkutfviiix}
%    \begin{macrocode}
    \def\noexpand\SOuL@checkutfviiix
      ##1\SOuL@stringmacrocolon\SOuL@charhash1##2##3##4##5##6%
      \SOuL@stringcsnameu##7\SOuL@stringundeferr##8\noexpand\@nil
  }%
\x{%
  \def\SOuL@temp{#7}%
  \ifx\SOuL@temp\SOuL@empty
    \chardef\SOuL@octets=\z@
  \else
    \def\SOuL@temp{#5}%
    \ifx\SOuL@temp\SOuL@charthree
      \chardef\SOuL@octets=4 %
    \else
      \def\SOuL@temp{#3}%
      \ifx\SOuL@temp\SOuL@chartwo
        \chardef\SOuL@octets=\thr@@
      \else
        \chardef\SOuL@octets=\tw@
      \fi
    \fi
  \fi
}
%    \end{macrocode}
%    \end{macro}
%
% \subsection{Actions for UTF-8 sequences}
%
%    \begin{macro}{\SOuL@addtwooctets}
%    \begin{macrocode}
\def\SOuL@addtwooctets#1#2{%
  \def\SOuL@temp{#1#2}%
  \@onelevel@sanitize\SOuL@temp
  \ifx\SOuL@temp\SOuL@stringnobreakspace
    \SOUL@doword
    \SOUL@eventuallyexhyphen\null
    \SOUL@everyspace{\nobreak}%
    \let\SOuL@next\SOUL@scan
  \else
    \def\SOuL@next{%
!     \SOUL@addtoken{{\noexpand#1\noexpand#2}}%
+     \SOUL@addtoken{{\etex@unexpanded{#1#2}}}%
    }%
  \fi
  \SOuL@next
}
%    \end{macrocode}
%    \end{macro}
%    \begin{macro}{\SOuL@addthreeoctets}
%    \begin{macrocode}
\def\SOuL@addthreeoctets#1#2#3{%
  \def\SOuL@temp{#1#2#3}%
  \@onelevel@sanitize\SOuL@temp
  \ifx\SOuL@temp\SOuL@stringendash
    \SOUL@doword
    \SOUL@eventuallyexhyphen{-}%
    \SOUL@exhyphen{--}%
    \let\SOuL@next\SOUL@scan
  \else
    \ifx\SOuL@temp\SOuL@stringemdash
      \SOUL@doword
      \SOUL@eventuallyexhyphen{-}%
      \SOUL@exhyphen{---}%
      \let\SOuL@next\SOUL@scan
    \else
      \def\SOuL@next{%
!       \SOUL@addtoken{{\noexpand#1\noexpand#2\noexpand#3}}%
+       \SOUL@addtoken{{\etex@unexpanded{#1#2#3}}}%
      }%
    \fi
  \fi
  \SOuL@next
}
%    \end{macrocode}
%    \end{macro}
%    \begin{macro}{\SOuL@addfouroctets}
%    \begin{macrocode}
\def\SOuL@addfouroctets#1#2#3#4{%
! \SOUL@addtoken{{\noexpand#1\noexpand#2\noexpand#3\noexpand#4}}%
+ \SOUL@addtoken{{\etex@unexpanded{#1#2#3#4}}}%
}
%    \end{macrocode}
%    \end{macro}
%
% \subsubsection{Redefinition of \cs{SOUL@splittoken}}
%
%    \begin{macro}{\SOUL@splittoken}
%    Macro \cs{SOUL@splittoken} separates the first token
%    or token group from a word and redefines the word to contain
%    the remaining tokens. However if the remaining tokens are
%    a token group, then the curly braces will be removed and
%    the token group is splitted by the next call of \cs{SOUL@splittoken}.
%    The redefinition avoids the removal of curly braces around the
%    remaining tokens.
%    \begin{macrocode}
\SOuL@redefine\SOUL@splittoken#1#2\SOUL@stop{%
  \global\SOUL@token={#1}%
  \global\SOUL@word={#2}%
}#1{%
  \global\SOUL@token={#1}%
  \SOuL@remainingtoken\relax
}
%    \end{macrocode}
%    \end{macro}
%    \begin{macro}{\SOuL@remainingtoken}
%    \begin{macrocode}
\def\SOuL@remainingtoken#1\SOUL@stop{%
  \global\SOUL@word=\expandafter{\@gobble#1}%
}
%    \end{macrocode}
%    \end{macro}
%
% \subsection{Patches}
%
%    The fixed \cs{SOUL@splittoken} allows to remove the double
%    sets of curly braces in other macros of package \xpackage{soul}.
%    The benefit is that implicite kernings are more often detected
%    and fixes a bug in package \xpackage{soul}. The disadvantage is
%    incompatibility. The width of the resulting strings may change.
%    \begin{macro}{\SOUL@flushcomma}
%    \begin{macrocode}
\SOuL@redefine\SOUL@flushcomma{%
  \ifcase\SOUL@comma
  \or
    \edef\x{\SOUL@word={\the\SOUL@word,}}\x
  \or
    \edef\x{\SOUL@word={\the\SOUL@word{{,,}}}}\x
  \fi
  \SOUL@comma\z@
}{%
  \ifcase\SOUL@comma
  \or
    \edef\x{\SOUL@word={\the\SOUL@word,}}\x
  \or
    \edef\x{\SOUL@word={\the\SOUL@word{,,}}}\x
  \fi
  \SOUL@comma\z@
}
%    \end{macrocode}
%    \end{macro}
%    \begin{macro}{\SOUL@flushapo}
%    \begin{macrocode}
\SOuL@redefine\SOUL@flushapo{%
  \ifcase\SOUL@apo
  \or
    \edef\x{\SOUL@word={\the\SOUL@word'}}\x
  \or
    \edef\x{\SOUL@word={\the\SOUL@word{{''}}}}\x
  \fi
  \SOUL@apo\z@
}{%
  \ifcase\SOUL@apo
  \or
    \edef\x{\SOUL@word={\the\SOUL@word'}}\x
  \or
    \edef\x{\SOUL@word={\the\SOUL@word{''}}}\x
  \fi
  \SOUL@apo\z@
}
%    \end{macrocode}
%    \end{macro}
%    \begin{macro}{\SOUL@flushgrave}
%    \begin{macrocode}
\SOuL@redefine\SOUL@flushgrave{%
  \ifcase\SOUL@grave
  \or
    \edef\x{\SOUL@word={\the\SOUL@word`}}\x
  \or
    \edef\x{\SOUL@word={\the\SOUL@word{{``}}}}\x
  \fi
  \SOUL@grave\z@
}{%
  \ifcase\SOUL@grave
  \or
    \edef\x{\SOUL@word={\the\SOUL@word`}}\x
  \or
    \edef\x{\SOUL@word={\the\SOUL@word{``}}}\x
  \fi
  \SOUL@grave\z@
}
%    \end{macrocode}
%    \end{macro}
%    \begin{macro}{\SOUL@addgroup}
%    \begin{macrocode}
\SOuL@redefine\SOUL@addgroup#1#2{%
  {%
    \let\protect\noexpand
    \edef\x{%
      \global\SOUL@word={%
        \the\SOUL@word
        {{\noexpand#1#2}}%
      }%
    }%
    \x
  }%
  \SOUL@scan
}#1#2{%
  \begingroup
    \let\protect\noexpand
    \edef\x{\endgroup
      \SOUL@word={%
        \the\SOUL@word
!       {\noexpand#1{#2}}%
+       {\etex@unexpanded{#1{#2}}}%
      }%
    }%
  \x
  \SOUL@scan
}
%    \end{macrocode}
%    \end{macro}
%    \begin{macro}{\SOUL@addmath}
%    \begin{macrocode}
\SOuL@redefine\SOUL@addmath$#1${%
  {%
    \let\protect\noexpand
    \edef\x{%
      \global\SOUL@word={%
        \the\SOUL@word
        {{\hbox{$#1$}}}%
      }%
    }%
    \x
  }%
  \SOUL@scan
}$#1${%
  \begingroup
    \let\protect\noexpand
    \edef\x{\endgroup
      \SOUL@word={%
        \the\SOUL@word
!       {\hbox{$#1$}}%
+       {\etex@unexpanded{\hbox{$#1$}}}%
      }%
    }%
  \x
  \SOUL@scan
}
%    \end{macrocode}
%    \end{macro}
%    \begin{macro}{\SOUL@addprotect}
%    \begin{macrocode}
\SOuL@redefine\SOUL@addprotect#1#2{%
  {%
    \let\protect\noexpand
    \edef\x{%
      \global\SOUL@word={%
        \the\SOUL@word
        {{\hbox{#2}}}%
      }%
    }%
    \x
  }%
  \SOUL@scan
}#1#2{%
  \begingroup
    \let\protect\noexpand
    \edef\x{\endgroup
      \SOUL@word={%
        \the\SOUL@word
!       {\hbox{#2}}%
+       {\etex@unexpanded{\hbox{#2}}}%
      }%
    }%
  \x
  \SOUL@scan
}
%    \end{macrocode}
%    \end{macro}
%
%    \begin{macro}{\SOUL@addtoken}
%    \begin{macrocode}
+ \SOuL@redefine\SOUL@addtoken#1{%
+   \edef\x{%
+     \SOUL@word={%
+       \the\SOUL@word
+       \noexpand#1%
+     }%
+   }%
+   \x
+   \SOUL@scan
+ }#1{%
+   \edef\x{%
+     \SOUL@word={%
+       \the\SOUL@word
+       \etex@unexpanded{#1}%
+     }%
+   }%
+   \x
+   \SOUL@scan
+ }%
%    \end{macrocode}
%    \end{macro}
%
%    \begin{macrocode}
\SOuL@AtEnd%
%    \end{macrocode}
%
%    \begin{macrocode}
%</package>
%    \end{macrocode}
%
% \section{Test}
%
% \subsection{Catcode checks for loading}
%
%    \begin{macrocode}
%<*test1>
%    \end{macrocode}
%    \begin{macrocode}
\catcode`\{=1 %
\catcode`\}=2 %
\catcode`\#=6 %
\catcode`\@=11 %
\expandafter\ifx\csname count@\endcsname\relax
  \countdef\count@=255 %
\fi
\expandafter\ifx\csname @gobble\endcsname\relax
  \long\def\@gobble#1{}%
\fi
\expandafter\ifx\csname @firstofone\endcsname\relax
  \long\def\@firstofone#1{#1}%
\fi
\expandafter\ifx\csname loop\endcsname\relax
  \expandafter\@firstofone
\else
  \expandafter\@gobble
\fi
{%
  \def\loop#1\repeat{%
    \def\body{#1}%
    \iterate
  }%
  \def\iterate{%
    \body
      \let\next\iterate
    \else
      \let\next\relax
    \fi
    \next
  }%
  \let\repeat=\fi
}%
\def\RestoreCatcodes{}
\count@=0 %
\loop
  \edef\RestoreCatcodes{%
    \RestoreCatcodes
    \catcode\the\count@=\the\catcode\count@\relax
  }%
\ifnum\count@<255 %
  \advance\count@ 1 %
\repeat

\def\RangeCatcodeInvalid#1#2{%
  \count@=#1\relax
  \loop
    \catcode\count@=15 %
  \ifnum\count@<#2\relax
    \advance\count@ 1 %
  \repeat
}
\def\RangeCatcodeCheck#1#2#3{%
  \count@=#1\relax
  \loop
    \ifnum#3=\catcode\count@
    \else
      \errmessage{%
        Character \the\count@\space
        with wrong catcode \the\catcode\count@\space
        instead of \number#3%
      }%
    \fi
  \ifnum\count@<#2\relax
    \advance\count@ 1 %
  \repeat
}
\def\space{ }
\expandafter\ifx\csname LoadCommand\endcsname\relax
  \def\LoadCommand{\input soulutf8.sty\relax}%
\fi
\def\Test{%
  \RangeCatcodeInvalid{0}{47}%
  \RangeCatcodeInvalid{58}{64}%
  \RangeCatcodeInvalid{91}{96}%
  \RangeCatcodeInvalid{123}{255}%
  \catcode`\@=12 %
  \catcode`\\=0 %
  \catcode`\%=14 %
  \LoadCommand
  \RangeCatcodeCheck{0}{36}{15}%
  \RangeCatcodeCheck{37}{37}{14}%
  \RangeCatcodeCheck{38}{47}{15}%
  \RangeCatcodeCheck{48}{57}{12}%
  \RangeCatcodeCheck{58}{63}{15}%
  \RangeCatcodeCheck{64}{64}{12}%
  \RangeCatcodeCheck{65}{90}{11}%
  \RangeCatcodeCheck{91}{91}{15}%
  \RangeCatcodeCheck{92}{92}{0}%
  \RangeCatcodeCheck{93}{96}{15}%
  \RangeCatcodeCheck{97}{122}{11}%
  \RangeCatcodeCheck{123}{255}{15}%
  \RestoreCatcodes
}
\Test
\csname @@end\endcsname
\end
%    \end{macrocode}
%    \begin{macrocode}
%</test1>
%    \end{macrocode}
%
% \subsection{\LaTeX\space tests}
%
%    \begin{macrocode}
%<*test2|test3|test4|test5>
%    \end{macrocode}
%    \begin{macrocode}
%<noetex>\let\SavedUnexpanded\unexpanded
%<noetex>\let\unexpanded\UNDEFINED
\NeedsTeXFormat{LaTeX2e}
\documentclass{article}
\usepackage[T1]{fontenc}
\usepackage{lmodern}
\usepackage{textcomp}
\usepackage[ngerman]{babel}
\expandafter\def\csname ver@hypdestopt.sty\endcsname{}
\usepackage[unicode]{hyperref}
\usepackage{bookmark}
\usepackage{color}
\usepackage{soul}[2003/11/17]
\makeatletter
\newcommand*{\@ifDefined}[1]{%
  \@ifundefined{#1}{%
    \@latex@error{%
      \expandafter\noexpand\csname#1\endcsname undefined%
    }\@ehc
  }%
}
\newcommand*{\SaveOrg}[1]{%
  \@ifdefinable{ORG@SOUL@#1}{%
    \@ifDefined{SOUL@#1}{%
      \expandafter\let\csname ORG@SOUL@#1\expandafter\endcsname
                      \csname SOUL@#1\endcsname
    }%
  }%
}
\newcommand*{\RestoreOrg}[1]{%
  \@ifDefined{ORG@SOUL@#1}{%
    \@ifDefined{SOUL@#1}{%
      \expandafter\let\csname SOUL@#1\expandafter\endcsname
                      \csname ORG@SOUL@#1\endcsname
    }%
  }%
}
\makeatother
\SaveOrg{eval}
\SaveOrg{splittoken}
\SaveOrg{flushcomma}
\SaveOrg{flushapo}
\SaveOrg{grave}
\SaveOrg{addgroup}
\SaveOrg{addmath}
\SaveOrg{addprotect}
\SaveOrg{addtoken}
\usepackage{soulutf8}[2016/05/16]
%    \end{macrocode}
%    \begin{macrocode}
%<noetex>\let\unexpanded\SavedUnexpanded
%    \end{macrocode}
%    Input encoding \xfile{utf8}.
%    \begin{macrocode}
%<*utf8>
\usepackage[utf8]{inputenc}
\DeclareUnicodeCharacter{00A0}{\nobreakspace}
%</utf8>
%    \end{macrocode}
%    Input encoding \xfile{utf8x}. Package \xpackage{ucs}
%    has a nasty bug, that it redefines \cs{show}.
%    \begin{macrocode}
%<*utf8x>
\let\origshow\show
\usepackage[utf8x]{inputenc}
\AtBeginDocument{%
  \let\show\origshow
}
%</utf8x>
%    \end{macrocode}
%
% \subsubsection{Driver definitions}
%
%    \begin{macrocode}
\makeatletter
%    \end{macrocode}
%    \begin{macro}{\an}
%    The definition for \cs{an} is taken from the documentation of
%    package \xpackage{soul}.
%    \begin{macrocode}
\DeclareRobustCommand*{\an}{%
  \def\SOUL@preamble{$^{^P}$}%
  \def\SOUL@everyspace##1{##1\texttt{\char`\ }\linebreak[2]}%
  \def\SOUL@postamble{$^{^E}$}%
  \def\SOUL@everyhyphen{$^{^-}$\linebreak[1]}%
  \def\SOUL@everyexhyphen##1{##1$^{^=}$}%
  \def\SOUL@everysyllable{$^{^S}$}%
  \def\SOUL@everytoken{\the\SOUL@token$^{^T}$}%
  \def\SOUL@everylowerthan{$^{^L}$}%
  \SOUL@
}
%    \end{macrocode}
%    \end{macro}
%
%    \begin{macro}{\sy}
%    The definition for \cs{sy} is taken from the documentation of
%    package \xpackage{soul}.
%    \begin{macrocode}
\DeclareRobustCommand*{\sy}{%
  \SOUL@setup
  \def\SOUL@preamble{%
    \lefthyphenmin=\z@
    \righthyphenmin=\z@
  }%
  \def\SOUL@everyspace##1{##1\space}%
  \def\SOUL@everysyllable{\the\SOUL@syllable}%
  \def\SOUL@everyhyphen{%
    \discretionary{%
      \SOUL@setkern\SOUL@hyphkern
      \SOUL@sethyphenchar
    }{}{%
      \hbox{\kern1pt$\cdot$}%
    }%
  }%
  \def\SOUL@everyexhyphen##1{%
    \SOUL@setkern\SOUL@hyphkern
    \hbox{##1}%
    \discretionary{}{}{%
      \SOUL@setkern\SOUL@charkern
    }%
  }%
  \SOUL@
}
%    \end{macrocode}
%    \end{macro}
%
%    \begin{macro}{\soulH}
%    \begin{macrocode}
\DeclareRobustCommand*{\soulH}{%
  \def\SOUL@preamble{%
    \gdef\SOULH{}%
    \soulHadd{[}%
  }%
  \def\SOUL@everyspace##1{%
    \soulHadd{Sp}%
  }%
  \def\SOUL@postamble{%
    \soulHadd{]}%
  }%
  \def\SOUL@everyhyphen{%
    \soulHadd{Hy}%
  }%
  \def\SOUL@everyexhyphen##1{%
    \soulHadd{Exhy}%
  }%
  \def\SOUL@everysyllable{%
    \soulHadd{Sy}%
  }%
  \def\SOUL@everytoken{%
    \soulHadd{T}%
  }%
  \def\SOUL@everylowerthan{%
    \soulHadd{<}%
  }%
  \SOUL@
}
\newcommand*{\soulHadd}[1]{%
  #1%
  \expandafter\gdef\expandafter\soulHstring\expandafter{%
    \soulHstring
    #1%
  }%
}
%    \end{macrocode}
%    \end{macro}
%
% \subsubsection{Test macros}
%
%    \begin{macrocode}
\usepackage{qstest}
\IncludeTests{*}
\LogTests{log}{*}{*}
%    \end{macrocode}
%    \begin{macrocode}
\newsavebox\boxA
\newsavebox\boxB
\newsavebox\boxC
%    \end{macrocode}
%    \begin{macro}{\TestDriver}
%    \begin{macrocode}
\newcommand*{\TestDriver}[4]{%
  \subsection{Driver #1}%
  \@PackageInfo{soulutf8-test}{* Driver #1}%
  \expandafter\let\expandafter\testdriver\csname #1\endcsname
  \def\teststringA{#3}%
  \def\teststringB{#4}%
  \begin{qstest}{string-#2-driver-#1}{string,#2,driver,#1}%
    \@PackageInfo{soulutf8-test}{* Version: soulutf8}%
    \global\let\soulHstring\@empty
    \setbox0=\hbox{\expandafter\testdriver\expandafter{\teststringA}}%
    \let\HstringA\soulHstring
    \SetTestVBox{\boxA}{\teststringA}%
    %
    \@PackageInfo{soulutf8-test}{* Version: soul + patch}%
    \global\let\soulHstring\@empty
    \RestoreOrg{eval}%
    \setbox2=\hbox{\expandafter\testdriver\expandafter{\teststringB}}%
    \let\HstringB\soulHstring
    \SetTestVBox{\boxB}{\teststringB}%
    %
    \@PackageInfo{soulutf8-test}{* Version: original soul}%
    \global\let\soulHstring\@empty
    \RestoreOrg{splittoken}%
    \RestoreOrg{flushcomma}%
    \RestoreOrg{flushapo}%
    \RestoreOrg{grave}%
    \RestoreOrg{addgroup}%
    \RestoreOrg{addmath}%
    \RestoreOrg{addprotect}%
    \RestoreOrg{addtoken}%
    \setbox4=\hbox{\expandafter\testdriver\expandafter{\teststringB}}%
    \let\HstringC\soulHstring
    \SetTestVBox{\boxC}{\teststringB}%
    %
    \TestInfo{0}{soulutf8}%
    \TestInfo{2}{soul + patch}%
    \TestInfo{4}{original soul}%
    \TestInfoVBox
    \showboxdepth=\maxdimen
    \showboxbreadth=\maxdimen
    \tracingonline=\@ne
    \Expect*{\HstringA}*{\HstringB}%
    \Expect*{\HstringA}*{\HstringC}%
    \Expect*{\the\wd0}*{\the\wd2}%
    \ifdim\wd0=\wd2 %
    \else
      \showbox0 %
      \showbox2 %
    \fi
    \Expect*{\the\ht0}*{\the\ht2}%
    \Expect*{\the\dp0}*{\the\dp2}%
    \Expect*{\the\wd\boxB}*{\the\wd\boxA}%
    \Expect*{\the\wd\boxC}*{\the\wd\boxA}%
    \Expect*{\the\ht\boxB}*{\the\ht\boxA}%
    \ifdim\ht\boxB=\ht\boxA
    \else
      \showbox\boxA
      \showbox\boxB
    \fi
    \Expect*{\the\ht\boxC}*{\the\ht\boxA}%
    \Expect*{\the\dp\boxB}*{\the\dp\boxA}%
    \Expect*{\the\dp\boxC}*{\the\dp\boxA}%
  \end{qstest}%
}
%    \end{macrocode}
%    \end{macro}
%    \begin{macro}{\SetTestVBox}
%    \begin{macrocode}
\newcommand*{\SetTestVBox}[2]{%
  \setbox#1=\vbox{%
    \hsize=0pt\relax
    \hfuzz=100mm\relax
    \expandafter\testdriver\expandafter{#2}%
  }%
}
%    \end{macrocode}
%    \end{macro}
%    \begin{macro}{\TestInfo}
%    \begin{macrocode}
\newcommand*{\TestInfo}[2]{%
  \par
  \noindent
  \copy#1\relax
  \hfill
  \begingroup
    \small
    \ifdim\wd0=\wd2\relax
      \ifdim\wd0=\wd4\relax
      \else
        \ifnum#1=2 %
          \raisebox{.5\baselineskip}[0pt][0pt]{%
            \llap{$\}$ }%
            \rlap{width = \the\wd#1\relax\space$\{$}%
            \phantom{width = 999.99999pt}%
            \qquad
          }%
        \else
          \ifnum#1=4 %
            \rlap{width = \the\wd#1}%
            \phantom{width = 999.99999pt}%
            \qquad
          \fi
        \fi
      \fi
    \else
      \begingroup
        \bfseries
        \rlap{width = \the\wd#1}%
        \phantom{width = 999.99999pt}%
      \endgroup
      \qquad
    \fi
    \rlap{(#2)}%
    \phantom{(original soul)}%
  \endgroup
  \par
}
%    \end{macrocode}
%    \end{macro}
%    \begin{macro}{\TestInfoVBox}
%    \begin{macrocode}
\newcommand*{\TestInfoVBox}{%
  \par
  \noindent
  \mbox{\vrule\relax\space\copy\boxA}%
  \ifdim\ht\boxA=\ht\boxB
  \else
    \hfill
    \begingroup
      \small
      \rlap{(soulutf8)}%
      \phantom{(soulutf8)}%
    \endgroup
    \par
    \noindent
    \mbox{\vrule\relax\space\copy\boxB}%
    \hfill
    \begingroup
      \small
      \rlap{(soul)}%
      \phantom{(soulutf8)}%
    \endgroup
  \fi
  \par
}
%    \end{macrocode}
%    \end{macro}
%    \begin{macro}{\Test}
%    \begin{macrocode}
\makeatletter
\newcommand*{\Test}[3]{%
  \clearpage
  \begingroup
    \def\linebreak{\textbackslash linebreak }%
    \def\\{\textbackslash\textbackslash}%
    \def\soulomit##1{%
      \textbackslash soulomit\textbraceleft##1\textbraceright
    }%
    \def\mbox##1{%
      \textbackslash mbox\textbraceleft##1\textbraceright
    }%
    \def\abc{\textbackslash abc }%
    \def\texttt##1{%
      \textbackslash texttt\textbraceleft##1\textbraceright
    }%
    \def\c##1{%
      \textbackslash c\textbraceleft##1\textbraceright
    }%
%<*utf8x>
    \count@=128 %
    \loop
    \ifnum\count@<256 %
      \lccode`\~=\count@
      \lowercase{%
        \def~{\noexpand~}%
      }%
      \advance\count@\@ne
    \repeat
%</utf8x>
    \csname protected@edef\endcsname\x{\endgroup
      \noexpand\section{String #1: #2}%
    }%
  \x
  \TestDriver{so}{#1}{#2}{#3}%
  \TestDriver{caps}{#1}{#2}{#3}%
  \TestDriver{ul}{#1}{#2}{#3}%
  \TestDriver{st}{#1}{#2}{#3}%
  \TestDriver{hl}{#1}{#2}{#3}%
  \TestDriver{an}{#1}{#2}{#3}%
  \TestDriver{sy}{#1}{#2}{#3}%
  \TestDriver{soulH}{#1}{#2}{#3}%
}
\makeatother
%    \end{macrocode}
%    \end{macro}
%
% \subsubsection{Testing}
%
%    \begin{macrocode}
\begin{document}
  \Test A{}{}%
  \Test B{Hello World}{Hello World}%
  \Test C{%
    Umlauts: ^^c3^^84^^c3^^96^^c3^^9c^^c3^^a4^^c3^^b6^^c3^^bc^^c3^^9f%
  }{%
    Umlauts: \"A\"O\"U\"a\"o\"u\ss
  }%
  \Test D{Euro: ^^e2^^82^^ac}{Euro: \texteuro}%
  \Test E{Stra^^c3^^9fe}{Stra\ss e}%
  \Test F{R^^c3^^b6mer}{R\"omer}%
  \Test G{A^^e2^^80^^93B^^e2^^80^^94C}{A--B---C}%
  \Test H{X^^c2^^a0Y}{X~Y}%
  \Test I{A^^c5^^a4}{A\v T}%
  \Test J{A,B,,C,,,D,,,,E}{A,B,,C,,,D,,,,E}%
  \Test K{A'B''C'''D''''E}{A'B''C'''D''''E}%
  \Test L{A`B``C```D````E}{A`B``C```D````E}%
  \Test M{A$x$B}{A$x$B}%
  \Test N{A\mbox{abc}B}{A\mbox{abc}B}%
  \Test O{,,TA''}{,,TA''}%
  \Test P{{!`}Hola{?`}}{{!`}Hola{?`}}%
  \Test Q{re-sent\slash x}{re-sent\slash x}%
  \Test R{new\\line}{new\\line}%
  \Test S{foo\linebreak bar}{foo\linebreak bar}%
  \Test T{Virchow{sche}}{Virchow{sche}}%
  \Test U{A\soulomit{foo}B}{A\soulomit{foo}B}%
  \Test V{A\texttt{xyz}B}{A\texttt{xyz}B}%
  \Test W{ffi}{ffi}%
  \Test X{f{}f{}i}{f{}f{}i}%
  \Test Y{f\mbox{}f\mbox{}i}{f\mbox{}f\mbox{}i}%
  \Test Z{\c{\"a}}{{\c{\"a}}}%
%<*test2|test3>
  \let\abcx\UNDEFINED
  \def\abc{%
    \begingroup
      \def\abcx{x}%
      \abcx
    \endgroup
  }%
  \Test{AA}{[{x\abc x}]}{[{xxx}]}%
%</test2|test3>
%    \end{macrocode}
%
%    \begin{macrocode}
\end{document}
%    \end{macrocode}
%    \begin{macrocode}
%</test2|test3|test4|test5>
%    \end{macrocode}
%
% \section{Installation}
%
% \subsection{Download}
%
% \paragraph{Package.} This package is available on
% CTAN\footnote{\url{http://ctan.org/pkg/soulutf8}}:
% \begin{description}
% \item[\CTAN{macros/latex/contrib/oberdiek/soulutf8.dtx}] The source file.
% \item[\CTAN{macros/latex/contrib/oberdiek/soulutf8.pdf}] Documentation.
% \end{description}
%
%
% \paragraph{Bundle.} All the packages of the bundle `oberdiek'
% are also available in a TDS compliant ZIP archive. There
% the packages are already unpacked and the documentation files
% are generated. The files and directories obey the TDS standard.
% \begin{description}
% \item[\CTAN{install/macros/latex/contrib/oberdiek.tds.zip}]
% \end{description}
% \emph{TDS} refers to the standard ``A Directory Structure
% for \TeX\ Files'' (\CTAN{tds/tds.pdf}). Directories
% with \xfile{texmf} in their name are usually organized this way.
%
% \subsection{Bundle installation}
%
% \paragraph{Unpacking.} Unpack the \xfile{oberdiek.tds.zip} in the
% TDS tree (also known as \xfile{texmf} tree) of your choice.
% Example (linux):
% \begin{quote}
%   |unzip oberdiek.tds.zip -d ~/texmf|
% \end{quote}
%
% \paragraph{Script installation.}
% Check the directory \xfile{TDS:scripts/oberdiek/} for
% scripts that need further installation steps.
% Package \xpackage{attachfile2} comes with the Perl script
% \xfile{pdfatfi.pl} that should be installed in such a way
% that it can be called as \texttt{pdfatfi}.
% Example (linux):
% \begin{quote}
%   |chmod +x scripts/oberdiek/pdfatfi.pl|\\
%   |cp scripts/oberdiek/pdfatfi.pl /usr/local/bin/|
% \end{quote}
%
% \subsection{Package installation}
%
% \paragraph{Unpacking.} The \xfile{.dtx} file is a self-extracting
% \docstrip\ archive. The files are extracted by running the
% \xfile{.dtx} through \plainTeX:
% \begin{quote}
%   \verb|tex soulutf8.dtx|
% \end{quote}
%
% \paragraph{TDS.} Now the different files must be moved into
% the different directories in your installation TDS tree
% (also known as \xfile{texmf} tree):
% \begin{quote}
% \def\t{^^A
% \begin{tabular}{@{}>{\ttfamily}l@{ $\rightarrow$ }>{\ttfamily}l@{}}
%   soulutf8.sty & tex/generic/oberdiek/soulutf8.sty\\
%   soulutf8.pdf & doc/latex/oberdiek/soulutf8.pdf\\
%   test/soulutf8-test1.tex & doc/latex/oberdiek/test/soulutf8-test1.tex\\
%   test/soulutf8-test2.tex & doc/latex/oberdiek/test/soulutf8-test2.tex\\
%   test/soulutf8-test3.tex & doc/latex/oberdiek/test/soulutf8-test3.tex\\
%   test/soulutf8-test4.tex & doc/latex/oberdiek/test/soulutf8-test4.tex\\
%   test/soulutf8-test5.tex & doc/latex/oberdiek/test/soulutf8-test5.tex\\
%   soulutf8.dtx & source/latex/oberdiek/soulutf8.dtx\\
% \end{tabular}^^A
% }^^A
% \sbox0{\t}^^A
% \ifdim\wd0>\linewidth
%   \begingroup
%     \advance\linewidth by\leftmargin
%     \advance\linewidth by\rightmargin
%   \edef\x{\endgroup
%     \def\noexpand\lw{\the\linewidth}^^A
%   }\x
%   \def\lwbox{^^A
%     \leavevmode
%     \hbox to \linewidth{^^A
%       \kern-\leftmargin\relax
%       \hss
%       \usebox0
%       \hss
%       \kern-\rightmargin\relax
%     }^^A
%   }^^A
%   \ifdim\wd0>\lw
%     \sbox0{\small\t}^^A
%     \ifdim\wd0>\linewidth
%       \ifdim\wd0>\lw
%         \sbox0{\footnotesize\t}^^A
%         \ifdim\wd0>\linewidth
%           \ifdim\wd0>\lw
%             \sbox0{\scriptsize\t}^^A
%             \ifdim\wd0>\linewidth
%               \ifdim\wd0>\lw
%                 \sbox0{\tiny\t}^^A
%                 \ifdim\wd0>\linewidth
%                   \lwbox
%                 \else
%                   \usebox0
%                 \fi
%               \else
%                 \lwbox
%               \fi
%             \else
%               \usebox0
%             \fi
%           \else
%             \lwbox
%           \fi
%         \else
%           \usebox0
%         \fi
%       \else
%         \lwbox
%       \fi
%     \else
%       \usebox0
%     \fi
%   \else
%     \lwbox
%   \fi
% \else
%   \usebox0
% \fi
% \end{quote}
% If you have a \xfile{docstrip.cfg} that configures and enables \docstrip's
% TDS installing feature, then some files can already be in the right
% place, see the documentation of \docstrip.
%
% \subsection{Refresh file name databases}
%
% If your \TeX~distribution
% (\teTeX, \mikTeX, \dots) relies on file name databases, you must refresh
% these. For example, \teTeX\ users run \verb|texhash| or
% \verb|mktexlsr|.
%
% \subsection{Some details for the interested}
%
% \paragraph{Attached source.}
%
% The PDF documentation on CTAN also includes the
% \xfile{.dtx} source file. It can be extracted by
% AcrobatReader 6 or higher. Another option is \textsf{pdftk},
% e.g. unpack the file into the current directory:
% \begin{quote}
%   \verb|pdftk soulutf8.pdf unpack_files output .|
% \end{quote}
%
% \paragraph{Unpacking with \LaTeX.}
% The \xfile{.dtx} chooses its action depending on the format:
% \begin{description}
% \item[\plainTeX:] Run \docstrip\ and extract the files.
% \item[\LaTeX:] Generate the documentation.
% \end{description}
% If you insist on using \LaTeX\ for \docstrip\ (really,
% \docstrip\ does not need \LaTeX), then inform the autodetect routine
% about your intention:
% \begin{quote}
%   \verb|latex \let\install=y% \iffalse meta-comment
%
% File: soulutf8.dtx
% Version: 2016/05/16 v1.1
% Info: Permit use of UTF-8 characters in soul
%
% Copyright (C) 2007 by
%    Heiko Oberdiek <heiko.oberdiek at googlemail.com>
%    2016
%    https://github.com/ho-tex/oberdiek/issues
%
% This work may be distributed and/or modified under the
% conditions of the LaTeX Project Public License, either
% version 1.3c of this license or (at your option) any later
% version. This version of this license is in
%    http://www.latex-project.org/lppl/lppl-1-3c.txt
% and the latest version of this license is in
%    http://www.latex-project.org/lppl.txt
% and version 1.3 or later is part of all distributions of
% LaTeX version 2005/12/01 or later.
%
% This work has the LPPL maintenance status "maintained".
%
% This Current Maintainer of this work is Heiko Oberdiek.
%
% The Base Interpreter refers to any `TeX-Format',
% because some files are installed in TDS:tex/generic//.
%
% This work consists of the main source file soulutf8.dtx
% and the derived files
%    soulutf8.sty, soulutf8.pdf, soulutf8.ins, soulutf8.drv,
%    soulutf8-test1.tex, soulutf8-test2.tex, soulutf8-test3.tex,
%    soulutf8-test4.tex, soulutf8-test5.tex.
%
% Distribution:
%    CTAN:macros/latex/contrib/oberdiek/soulutf8.dtx
%    CTAN:macros/latex/contrib/oberdiek/soulutf8.pdf
%
% Unpacking:
%    (a) If soulutf8.ins is present:
%           tex soulutf8.ins
%    (b) Without soulutf8.ins:
%           tex soulutf8.dtx
%    (c) If you insist on using LaTeX
%           latex \let\install=y\input{soulutf8.dtx}
%        (quote the arguments according to the demands of your shell)
%
% Documentation:
%    (a) If soulutf8.drv is present:
%           latex soulutf8.drv
%    (b) Without soulutf8.drv:
%           latex soulutf8.dtx; ...
%    The class ltxdoc loads the configuration file ltxdoc.cfg
%    if available. Here you can specify further options, e.g.
%    use A4 as paper format:
%       \PassOptionsToClass{a4paper}{article}
%
%    Programm calls to get the documentation (example):
%       pdflatex soulutf8.dtx
%       makeindex -s gind.ist soulutf8.idx
%       pdflatex soulutf8.dtx
%       makeindex -s gind.ist soulutf8.idx
%       pdflatex soulutf8.dtx
%
% Installation:
%    TDS:tex/generic/oberdiek/soulutf8.sty
%    TDS:doc/latex/oberdiek/soulutf8.pdf
%    TDS:doc/latex/oberdiek/test/soulutf8-test1.tex
%    TDS:doc/latex/oberdiek/test/soulutf8-test2.tex
%    TDS:doc/latex/oberdiek/test/soulutf8-test3.tex
%    TDS:doc/latex/oberdiek/test/soulutf8-test4.tex
%    TDS:doc/latex/oberdiek/test/soulutf8-test5.tex
%    TDS:source/latex/oberdiek/soulutf8.dtx
%
%<*ignore>
\begingroup
  \catcode123=1 %
  \catcode125=2 %
  \def\x{LaTeX2e}%
\expandafter\endgroup
\ifcase 0\ifx\install y1\fi\expandafter
         \ifx\csname processbatchFile\endcsname\relax\else1\fi
         \ifx\fmtname\x\else 1\fi\relax
\else\csname fi\endcsname
%</ignore>
%<*install>
\input docstrip.tex
\Msg{************************************************************************}
\Msg{* Installation}
\Msg{* Package: soulutf8 2016/05/16 v1.1 Permit use of UTF-8 characters in soul (HO)}
\Msg{************************************************************************}

\keepsilent
\askforoverwritefalse

\let\MetaPrefix\relax
\preamble

This is a generated file.

Project: soulutf8
Version: 2016/05/16 v1.1

Copyright (C) 2007 by
   Heiko Oberdiek <heiko.oberdiek at googlemail.com>

This work may be distributed and/or modified under the
conditions of the LaTeX Project Public License, either
version 1.3c of this license or (at your option) any later
version. This version of this license is in
   http://www.latex-project.org/lppl/lppl-1-3c.txt
and the latest version of this license is in
   http://www.latex-project.org/lppl.txt
and version 1.3 or later is part of all distributions of
LaTeX version 2005/12/01 or later.

This work has the LPPL maintenance status "maintained".

This Current Maintainer of this work is Heiko Oberdiek.

The Base Interpreter refers to any `TeX-Format',
because some files are installed in TDS:tex/generic//.

This work consists of the main source file soulutf8.dtx
and the derived files
   soulutf8.sty, soulutf8.pdf, soulutf8.ins, soulutf8.drv,
   soulutf8-test1.tex, soulutf8-test2.tex, soulutf8-test3.tex,
   soulutf8-test4.tex, soulutf8-test5.tex.

\endpreamble
\let\MetaPrefix\DoubleperCent

\generate{%
  \file{soulutf8.ins}{\from{soulutf8.dtx}{install}}%
  \file{soulutf8.drv}{\from{soulutf8.dtx}{driver}}%
  \usedir{tex/generic/oberdiek}%
  \file{soulutf8.sty}{\from{soulutf8.dtx}{package}}%
%  \usedir{doc/latex/oberdiek/test}%
%  \file{soulutf8-test1.tex}{\from{soulutf8.dtx}{test1}}%
%  \file{soulutf8-test2.tex}{\from{soulutf8.dtx}{test2,utf8}}%
%  \file{soulutf8-test3.tex}{\from{soulutf8.dtx}{test3,utf8x}}%
%  \file{soulutf8-test4.tex}{\from{soulutf8.dtx}{test4,utf8,noetex}}%
%  \file{soulutf8-test5.tex}{\from{soulutf8.dtx}{test5,utf8x,noetex}}%
  \nopreamble
  \nopostamble
%  \usedir{source/latex/oberdiek/catalogue}%
%  \file{soulutf8.xml}{\from{soulutf8.dtx}{catalogue}}%
}

\catcode32=13\relax% active space
\let =\space%
\Msg{************************************************************************}
\Msg{*}
\Msg{* To finish the installation you have to move the following}
\Msg{* file into a directory searched by TeX:}
\Msg{*}
\Msg{*     soulutf8.sty}
\Msg{*}
\Msg{* To produce the documentation run the file `soulutf8.drv'}
\Msg{* through LaTeX.}
\Msg{*}
\Msg{* Happy TeXing!}
\Msg{*}
\Msg{************************************************************************}

\endbatchfile
%</install>
%<*ignore>
\fi
%</ignore>
%<*driver>
\NeedsTeXFormat{LaTeX2e}
\ProvidesFile{soulutf8.drv}%
  [2016/05/16 v1.1 Permit use of UTF-8 characters in soul (HO)]%
\documentclass{ltxdoc}
\usepackage{holtxdoc}[2011/11/22]
\begin{document}
  \DocInput{soulutf8.dtx}%
\end{document}
%</driver>
% \fi
%
%
% \CharacterTable
%  {Upper-case    \A\B\C\D\E\F\G\H\I\J\K\L\M\N\O\P\Q\R\S\T\U\V\W\X\Y\Z
%   Lower-case    \a\b\c\d\e\f\g\h\i\j\k\l\m\n\o\p\q\r\s\t\u\v\w\x\y\z
%   Digits        \0\1\2\3\4\5\6\7\8\9
%   Exclamation   \!     Double quote  \"     Hash (number) \#
%   Dollar        \$     Percent       \%     Ampersand     \&
%   Acute accent  \'     Left paren    \(     Right paren   \)
%   Asterisk      \*     Plus          \+     Comma         \,
%   Minus         \-     Point         \.     Solidus       \/
%   Colon         \:     Semicolon     \;     Less than     \<
%   Equals        \=     Greater than  \>     Question mark \?
%   Commercial at \@     Left bracket  \[     Backslash     \\
%   Right bracket \]     Circumflex    \^     Underscore    \_
%   Grave accent  \`     Left brace    \{     Vertical bar  \|
%   Right brace   \}     Tilde         \~}
%
% \GetFileInfo{soulutf8.drv}
%
% \title{The \xpackage{soulutf8} package}
% \date{2016/05/16 v1.1}
% \author{Heiko Oberdiek\thanks
% {Please report any issues at https://github.com/ho-tex/oberdiek/issues}\\
% \xemail{heiko.oberdiek at googlemail.com}}
%
% \maketitle
%
% \begin{abstract}
% This package extends package \xpackage{soul} and adds some support
% for UTF-8. Namely the input encodings \xfile{utf8.def}
% from package \xpackage{inputenc} and
% package \xpackage{ucs}'s \xfile{utf8x.def} are supported.
% \end{abstract}
%
% \tableofcontents
%
% \section{Documentation}
%
% This package \xpackage{soulutf8} does not have own options and does not
% define new user commands.
% Any option is passed to package \xpackage{soul} \cite{soul}
% that is loaded first.
% Then some internal macros of \xpackage{soul} are redefined to add
% support for UTF-8.
% The following input encodings are supported:
% \begin{quote}
%   \begin{tabular}{@{}lll@{}}
%     \texttt{utf8} & \LaTeX\ base &
%        \xfile{TDS:tex/latex/base/utf8.def} \cite{utf8}\\
%     \texttt{utf8x} & Package \xpackage{ucs} &
%       \xfile{TDS:tex/latex/ucs/utf8x.def} \cite{ucs}\\
%   \end{tabular}
% \end{quote}
% UTF-8 byte sequences are added as token group to a word, even
% if these UTF-8 characters are some kind of hyphen or space.
% As exception the following three Unicode characters are handled specially:
% \begin{quote}
%   \begin{tabular}{lll}
%     Slot & Name & Action\\
%     \hline
%     \texttt{U+00A0} & NO-BREAK SPACE & like |~|\\
%     \texttt{U+2013} & EN DASH & |--|\\
%     \texttt{U+2014} & EM DASH & |---|
%   \end{tabular}
%
% \subsection{Patch}
%
% Also package \xpackage{soulutf8} tries to patch package \xpackage{soul}
% to improve its behaviour:
% \begin{itemize}
% \item
%   A problem with additional levels of curly braces is fixed.
%   As advantage more implicite kernings are detected.
%   However, the result may be incompatible with the
%   original behaviour of package \xpackage{soul} because
%   of these respected implicite kernings.
% \item
%   \eTeX\ , especially \cs{unexpanded} is supported.
%   This allows a better protection of token groups
%   (|\mbox{|\dots|}|, math, \dots).
% \end{itemize}
% \end{quote}
%
% \subsection{Future}
%
% Currently package \xpackage{soul} does not seem to be
% maintained. Nevertheless if there will be a new version that
% adds support for UTF-8, then this package may become
% obsolete.
%
% \StopEventually{
% }
%
% \section{Implementation}
%
%    \begin{macrocode}
%<*package>
%    \end{macrocode}
%
% \subsection{Reload check and package identification}
%    Reload check, especially if the package is not used with \LaTeX.
%    \begin{macrocode}
\begingroup\catcode61\catcode48\catcode32=10\relax%
  \catcode13=5 % ^^M
  \endlinechar=13 %
  \catcode35=6 % #
  \catcode39=12 % '
  \catcode44=12 % ,
  \catcode45=12 % -
  \catcode46=12 % .
  \catcode58=12 % :
  \catcode64=11 % @
  \catcode123=1 % {
  \catcode125=2 % }
  \expandafter\let\expandafter\x\csname ver@soulutf8.sty\endcsname
  \ifx\x\relax % plain-TeX, first loading
  \else
    \def\empty{}%
    \ifx\x\empty % LaTeX, first loading,
      % variable is initialized, but \ProvidesPackage not yet seen
    \else
      \expandafter\ifx\csname PackageInfo\endcsname\relax
        \def\x#1#2{%
          \immediate\write-1{Package #1 Info: #2.}%
        }%
      \else
        \def\x#1#2{\PackageInfo{#1}{#2, stopped}}%
      \fi
      \x{soulutf8}{The package is already loaded}%
      \aftergroup\endinput
    \fi
  \fi
\endgroup%
%    \end{macrocode}
%    Package identification:
%    \begin{macrocode}
\begingroup\catcode61\catcode48\catcode32=10\relax%
  \catcode13=5 % ^^M
  \endlinechar=13 %
  \catcode35=6 % #
  \catcode39=12 % '
  \catcode40=12 % (
  \catcode41=12 % )
  \catcode44=12 % ,
  \catcode45=12 % -
  \catcode46=12 % .
  \catcode47=12 % /
  \catcode58=12 % :
  \catcode64=11 % @
  \catcode91=12 % [
  \catcode93=12 % ]
  \catcode123=1 % {
  \catcode125=2 % }
  \expandafter\ifx\csname ProvidesPackage\endcsname\relax
    \def\x#1#2#3[#4]{\endgroup
      \immediate\write-1{Package: #3 #4}%
      \xdef#1{#4}%
    }%
  \else
    \def\x#1#2[#3]{\endgroup
      #2[{#3}]%
      \ifx#1\@undefined
        \xdef#1{#3}%
      \fi
      \ifx#1\relax
        \xdef#1{#3}%
      \fi
    }%
  \fi
\expandafter\x\csname ver@soulutf8.sty\endcsname
\ProvidesPackage{soulutf8}%
  [2016/05/16 v1.1 Permit use of UTF-8 characters in soul (HO)]%
%    \end{macrocode}
%
% \subsection{Catcodes}
%
%    \begin{macrocode}
\begingroup\catcode61\catcode48\catcode32=10\relax%
  \catcode13=5 % ^^M
  \endlinechar=13 %
  \catcode123=1 % {
  \catcode125=2 % }
  \catcode64=11 % @
  \def\x{\endgroup
    \expandafter\edef\csname SOuL@AtEnd\endcsname{%
      \endlinechar=\the\endlinechar\relax
      \catcode13=\the\catcode13\relax
      \catcode32=\the\catcode32\relax
      \catcode35=\the\catcode35\relax
      \catcode61=\the\catcode61\relax
      \catcode64=\the\catcode64\relax
      \catcode123=\the\catcode123\relax
      \catcode125=\the\catcode125\relax
    }%
  }%
\x\catcode61\catcode48\catcode32=10\relax%
\catcode13=5 % ^^M
\endlinechar=13 %
\catcode35=6 % #
\catcode64=11 % @
\catcode123=1 % {
\catcode125=2 % }
\def\TMP@EnsureCode#1#2{%
  \edef\SOuL@AtEnd{%
    \SOuL@AtEnd
    \catcode#1=\the\catcode#1\relax
  }%
  \catcode#1=#2\relax
}
\TMP@EnsureCode{10}{12}% ^^J
\TMP@EnsureCode{33}{12}% !
\TMP@EnsureCode{34}{12}% "
\TMP@EnsureCode{36}{3}% $
\TMP@EnsureCode{39}{12}% '
\TMP@EnsureCode{40}{12}% (
\TMP@EnsureCode{41}{12}% )
\TMP@EnsureCode{42}{12}% *
\TMP@EnsureCode{43}{12}% +
\TMP@EnsureCode{44}{12}% ,
\TMP@EnsureCode{45}{12}% -
\TMP@EnsureCode{46}{12}% .
\TMP@EnsureCode{47}{12}% /
\TMP@EnsureCode{58}{12}% :
\TMP@EnsureCode{60}{12}% <
\TMP@EnsureCode{62}{12}% >
\TMP@EnsureCode{91}{12}% [
\TMP@EnsureCode{93}{12}% ]
\TMP@EnsureCode{94}{7}% ^
\TMP@EnsureCode{96}{12}% `
\TMP@EnsureCode{126}\active % ~
\TMP@EnsureCode{128}{12}% ^^80
\TMP@EnsureCode{147}{12}% ^^93
\TMP@EnsureCode{148}{12}% ^^94
\TMP@EnsureCode{160}{12}% ^^a0
\TMP@EnsureCode{194}{12}% ^^c2
\TMP@EnsureCode{226}{12}% ^^e2
\edef\SOuL@AtEnd{\SOuL@AtEnd\noexpand\endinput}
%    \end{macrocode}
%
% \subsection{Loading packages}
%
%    Package \xpackage{soul} uses \cs{documentclass} to detect \LaTeX.
%    \begin{macrocode}
\ifx\documentclass\@undefined
%    \end{macrocode}
%
% \subsubsection{\plainTeX}
%
%    First we check, whether package \xpackage{soul} is already loaded.
%    \begin{macrocode}
  \expandafter\ifx\csname SOUL@\endcsname\relax
%    \end{macrocode}
%    In case of \plainTeX\ package \xpackage{soul} defines some
%    macros in a simple manner that will break the definitions of
%    \xfile{miniltx.tex}, for example. Therefore these macros
%    are first saved and restored afterwards.
%    \begin{macrocode}
    \let\SOuL@orgDeclareRobustCommand\DeclareRobustCommand
    \let\SOuL@orgnewcommand          \newcommand
    \let\SOuL@orgDeclareOption       \DeclareOption
    \let\SOuL@orgPackageError        \PackageError
    \def\SOuL@restorelatexcmds{%
      \let\DeclareRobustCommand\SOuL@orgDeclareRobustCommand
      \let\newcommand          \SOuL@orgnewcommand
      \let\DeclareOption       \SOuL@orgDeclareOption
      \let\PackageError        \SOuL@orgPackageError
    }%
    \input soul.sty\relax
    \SOuL@restorelatexcmds
  \fi
%    \end{macrocode}
%    \begin{macro}{\SOUL@error}
%     Package \xpackage{soul}'s use of \cs{PackageError} is replaced
%     by \cs{@PackageError} of package \xpackage{infwarerr}.
%    \begin{macrocode}
  \input infwarerr.sty\relax
  \let\SOuL@orgSOUL@error\SOUL@error
  \def\SOUL@error{%
    \begingroup
      \let\PackageError\@PackageError
      \SOuL@orgSOUL@error
    \endgroup
  }%
%    \end{macrocode}
%    \end{macro}
%    \begin{macrocode}
  \input etexcmds.sty\relax
%    \end{macrocode}
%    \begin{macro}{\@onelevel@sanitize}
%    Define \LaTeX's \cs{@onelevel@sanitize} if not already available.
%    \begin{macrocode}
  \expandafter\ifx\csname @onelevel@sanitize\endcsname\relax
    \def\@onelevel@sanitize#1{%
      \edef#1{%
        \expandafter\strip@prefix\meaning#1%
      }%
    }%
%    \end{macrocode}
%    \end{macro}
%    \begin{macro}{\strip@prefix}
%    \begin{macrocode}
    \def\strip@prefix#1>{}%
%    \end{macrocode}
%    \end{macro}
%    \begin{macrocode}
  \fi
\else
%    \end{macrocode}
%
% \subsubsection{\LaTeX}
%    \begin{macrocode}
  \DeclareOption*{\PassOptionsToPackage{\CurrentOption}{soul}}%
  \ProcessOptions\relax
  \RequirePackage{soul}[2003/11/17]%
  \RequirePackage{infwarerr}[2016/05/16]%
  \RequirePackage{etexcmds}[2016/05/16]%
\fi
%    \end{macrocode}
%
% \subsubsection{\eTeX}
%
%    In \plainTeX\ command \cs{+} is an \emph{outer} macro.
%    Therefore numbers are used to avoid problems.
%    \begin{macrocode}
\ifetex@unexpanded
  \catcode33=14 % '!': comment
  \catcode43=9  % '+': ignore
\else
  \catcode33=9  % '!': ignore
  \catcode43=14 % '+': comment
\fi
%    \end{macrocode}
%
% \subsection{Macro for redefinitions}
%
%    \begin{macro}{\SOuL@redefine}
%    \begin{macrocode}
\def\SOuL@redefine#1{%
  \begingroup
    \def\SOuL@cmd{#1}%
    \afterassignment\SOuL@cmdcheck
    \def\SOuL@temp
}
%    \end{macrocode}
%    \end{macro}
%    \begin{macro}{\SOuL@cmdcheck}
%    \begin{macrocode}
\def\SOuL@cmdcheck{%
    \expandafter\ifx\SOuL@cmd\SOuL@temp
    \else
      \edef\SOuL@temp*{\expandafter\string\SOuL@cmd}%
      \@PackageWarningNoLine{soulutf8}{%
        Command \SOuL@temp* has changed.\MessageBreak
        Supported versions of package `soul': 2003/11/17.\MessageBreak
        Depending on the unknown changes the redefinition\MessageBreak
        of \SOuL@temp* may not behave correctly%
      }%
    \fi
  \expandafter\endgroup
  \expandafter\def\SOuL@cmd
}
%    \end{macrocode}
%    \end{macro}
%
% \subsection{Redefinition of \cs{SOUL@eval}}
%
%    \begin{macro}{\SOUL@eval}
%    Macro \cs{SOUL@eval} is redefined to add detection of
%    the first byte of a UTF-8 sequence. Because \cs{SOUL@eval}
%    is overwritten, a warning is issued, if the contents of
%    \cs{SOUL@eval} is not as expected.
%    \begin{macrocode}
\SOuL@redefine\SOUL@eval{%
%    \end{macrocode}
%    First the expected definition.
%    \begin{macrocode}
  \def\SOUL@n*##1{\SOUL@scan}%
  \if\noexpand\SOUL@@\SOUL@spc
  \else
    \SOUL@ignorespacesfalse
  \fi
  \ifnum\SOUL@minus=\thr@@
    \SOUL@flushminus
  \else\ifnum\SOUL@comma=\tw@
    \SOUL@flushcomma
  \else\ifnum\SOUL@apo=\tw@
    \SOUL@flushapo
  \else\ifnum\SOUL@grave=\tw@
    \SOUL@flushgrave
  \fi\fi\fi\fi
  \ifx\SOUL@@-\else\SOUL@flushminus\fi
  \ifx\SOUL@@,\else\SOUL@flushcomma\fi
  \ifx\SOUL@@'\else\SOUL@flushapo\fi
  \ifx\SOUL@@`\else\SOUL@flushgrave\fi
  \ifx\SOUL@@-%
    \advance\SOUL@minus\@ne
  \else\ifx\SOUL@@,%
    \advance\SOUL@comma\@ne
  \else\ifx\SOUL@@'%
    \advance\SOUL@apo\@ne
  \else\ifx\SOUL@@`%
    \advance\SOUL@grave\@ne
  \else
    \SOUL@flushminus
    \SOUL@flushcomma
    \SOUL@flushapo
    \SOUL@flushgrave
    \ifx\SOUL@@\SOUL@stop
      \def\SOUL@n*{%
        \SOUL@doword
        \SOUL@eventuallyexhyphen\null
      }%
    \else\ifx\SOUL@@\par
      \def\SOUL@n*\par{\par\leavevmode\SOUL@scan}%
    \else\if\noexpand\SOUL@@\SOUL@spc
      \SOUL@doword
      \SOUL@eventuallyexhyphen\null
      \ifSOUL@ignorespaces
      \else
        \SOUL@everyspace{}%
      \fi
      \def\SOUL@n* {\SOUL@scan}%
    \else\ifx\SOUL@@\\%
      \SOUL@doword
      \SOUL@eventuallyexhyphen\null
      \SOUL@everyspace{\unskip\nobreak\hfil\break}%
      \SOUL@ignorespacestrue
    \else\ifx\SOUL@@~%
      \SOUL@doword
      \SOUL@eventuallyexhyphen\null
      \SOUL@everyspace{\nobreak}%
    \else\ifx\SOUL@@\slash
      \SOUL@doword
      \SOUL@eventuallyexhyphen{/}%
      \SOUL@exhyphen{/}%
    \else\ifx\SOUL@@\mbox
      \def\SOUL@n*{\SOUL@addprotect}%
    \else\ifx\SOUL@@\hbox
      \def\SOUL@n*{\SOUL@addprotect}%
    \else\ifx\SOUL@@\soulomit
      \def\SOUL@n*\soulomit##1{%
        \SOUL@doword
        {\spaceskip\SOUL@spaceskip##1}%
        \SOUL@scan
      }%
    \else\ifx\SOUL@@\break
      \SOUL@doword
      \break
    \else\ifx\SOUL@@\linebreak
      \SOUL@doword
      \SOUL@everyspace{\linebreak}%
    \else\ifcat\bgroup\noexpand\SOUL@@
      \def\SOUL@n*{\SOUL@addgroup{}}%
    \else\ifcat$\noexpand\SOUL@@
      \def\SOUL@n*{\SOUL@addmath}%
    \else
      \def\SOUL@n*{\SOUL@dotoken}%
    \fi\fi\fi\fi\fi\fi\fi\fi\fi\fi\fi\fi\fi
  \fi\fi\fi\fi
  \SOUL@n*%
}{%
%    \end{macrocode}
%    Now the redefined version follows.
%    \begin{macrocode}
  \def\SOUL@n*##1{\SOUL@scan}%
  \if\noexpand\SOUL@@\SOUL@spc
  \else
    \SOUL@ignorespacesfalse
  \fi
  \ifnum\SOUL@minus=\thr@@
    \SOUL@flushminus
  \else\ifnum\SOUL@comma=\tw@
    \SOUL@flushcomma
  \else\ifnum\SOUL@apo=\tw@
    \SOUL@flushapo
  \else\ifnum\SOUL@grave=\tw@
    \SOUL@flushgrave
  \fi\fi\fi\fi
  \ifx\SOUL@@-\else\SOUL@flushminus\fi
  \ifx\SOUL@@,\else\SOUL@flushcomma\fi
  \ifx\SOUL@@'\else\SOUL@flushapo\fi
  \ifx\SOUL@@`\else\SOUL@flushgrave\fi
  \ifx\SOUL@@-%
    \advance\SOUL@minus\@ne
  \else\ifx\SOUL@@,%
    \advance\SOUL@comma\@ne
  \else\ifx\SOUL@@'%
    \advance\SOUL@apo\@ne
  \else\ifx\SOUL@@`%
    \advance\SOUL@grave\@ne
  \else
    \SOUL@flushminus
    \SOUL@flushcomma
    \SOUL@flushapo
    \SOUL@flushgrave
    \ifx\SOUL@@\SOUL@stop
      \def\SOUL@n*{%
          \SOUL@doword
          \SOUL@eventuallyexhyphen\null
      }%
    \else\ifx\SOUL@@\par
      \def\SOUL@n*\par{\par\leavevmode\SOUL@scan}%
    \else\if\noexpand\SOUL@@\SOUL@spc
      \SOUL@doword
      \SOUL@eventuallyexhyphen\null
      \ifSOUL@ignorespaces
      \else
        \SOUL@everyspace{}%
      \fi
      \def\SOUL@n* {\SOUL@scan}%
    \else\ifx\SOUL@@\\%
      \SOUL@doword
      \SOUL@eventuallyexhyphen\null
      \SOUL@everyspace{\unskip\nobreak\hfil\break}%
      \SOUL@ignorespacestrue
    \else\ifx\SOUL@@~%
      \SOUL@doword
      \SOUL@eventuallyexhyphen\null
      \SOUL@everyspace{\nobreak}%
    \else\ifx\SOUL@@\slash
      \SOUL@doword
      \SOUL@eventuallyexhyphen{/}%
      \SOUL@exhyphen{/}%
    \else\ifx\SOUL@@\mbox
      \def\SOUL@n*{\SOUL@addprotect}%
    \else\ifx\SOUL@@\hbox
      \def\SOUL@n*{\SOUL@addprotect}%
    \else\ifx\SOUL@@\soulomit
      \def\SOUL@n*\soulomit##1{%
        \SOUL@doword
        {\spaceskip\SOUL@spaceskip##1}%
        \SOUL@scan
      }%
    \else\ifx\SOUL@@\break
      \SOUL@doword
      \break
    \else\ifx\SOUL@@\linebreak
      \SOUL@doword
      \SOUL@everyspace{\linebreak}%
    \else\ifcat\bgroup\noexpand\SOUL@@
      \def\SOUL@n*{\SOUL@addgroup{}}%
    \else\ifcat$\noexpand\SOUL@@
      \def\SOUL@n*{\SOUL@addmath}%
    \else
%    \end{macrocode}
%    The current token is examined to detect the start of
%    a UTF-8 sequence.
%    \begin{macrocode}
      \SOuL@analyzeutfviii
      \ifcase\SOuL@octets
        \SOuL@analyzeutfviiix
      \fi
      \ifcase\SOuL@octets
        \def\SOUL@n*{\SOUL@dotoken}%
      \or % 1
      \or % 2
        \def\SOUL@n*{\SOuL@addtwooctets}%
      \or % 3
        \def\SOUL@n*{\SOuL@addthreeoctets}%
      \or % 4
        \def\SOUL@n*{\SOuL@addfouroctets}%
      \fi
    \fi\fi\fi\fi\fi\fi\fi\fi\fi\fi\fi\fi\fi
  \fi\fi\fi\fi
  \SOUL@n*%
}
%    \end{macrocode}
%    \end{macro}
%
% \subsection{UTF-8 analysis}
%
% \subsubsection{Help strings}
%
%    \begin{macrocode}
\def\SOuL@defsanitizedstring#1#2{%
  \expandafter\def\csname SOuL@string#1\endcsname{#2}%
  \expandafter\@onelevel@sanitize\csname SOuL@string#1\endcsname
}
\SOuL@defsanitizedstring{UTFviii}{UTFviii@}
\SOuL@defsanitizedstring{octets}{@octets}
\SOuL@defsanitizedstring{two}{two}
\SOuL@defsanitizedstring{three}{three}
\SOuL@defsanitizedstring{four}{four}
\SOuL@defsanitizedstring{macrocolon}{macro:}
\SOuL@defsanitizedstring{csnameu}{csname u8-}
\SOuL@defsanitizedstring{undeferr}{utf@viii@undeferr}
\def\SOuL@stringendash{^^e2^^80^^93}
\def\SOuL@stringemdash{^^e2^^80^^94}
\def\SOuL@stringnobreakspace{^^c2^^a0}
\edef\SOuL@charhash{\string #}
\edef\SOuL@chartwo{\string 2}
\edef\SOuL@charthree{\string 3}
\def\SOuL@empty{}
%    \end{macrocode}
%
% \subsubsection{Support for \xfile{utf8.def}}
%
%    \begin{macro}{\SOuL@analyzeutfviii}
%    \begin{macrocode}
\begingroup
  \edef\x{\endgroup
    \def\noexpand\SOuL@analyzeutfviii{%
      \noexpand\expandafter\noexpand\SOuL@checkutfviii
      \noexpand\meaning\noexpand\SOUL@@
      \SOuL@stringUTFviii\SOuL@stringoctets
      \noexpand\@nil
    }%
    \def\noexpand\SOuL@checkutfviii
      ##1\SOuL@stringUTFviii##2\SOuL@stringoctets##3\noexpand\@nil
  }%
\x{%
  \def\SOuL@temp{#2}%
  \chardef\SOuL@octets=%
      \ifx\SOuL@temp\SOuL@stringtwo
        \tw@
      \else\ifx\SOuL@temp\SOuL@stringthree
        \thr@@
      \else\ifx\SOuL@temp\SOuL@stringfour
        4 %
      \else
        \z@
      \fi\fi\fi
}
%    \end{macrocode}
%    \end{macro}
%
% \subsubsection{Support for \xfile{utf8x.def}}
%
%    \begin{macro}{\SOuL@analyzeutfviiix}
%    \begin{macrocode}
\begingroup
  \edef\x{\endgroup
    \def\noexpand\SOuL@analyzeutfviiix{%
      \noexpand\expandafter\noexpand\SOuL@checkutfviiix
      \noexpand\meaning\noexpand\SOUL@@
      \SOuL@stringmacrocolon\SOuL@charhash1{}{}{}{}%
      \SOuL@stringcsnameu\SOuL@stringundeferr
      \noexpand\@nil
    }%
%    \end{macrocode}
%    \end{macro}
%    \begin{macro}{\SOuL@checkutfviiix}
%    \begin{macrocode}
    \def\noexpand\SOuL@checkutfviiix
      ##1\SOuL@stringmacrocolon\SOuL@charhash1##2##3##4##5##6%
      \SOuL@stringcsnameu##7\SOuL@stringundeferr##8\noexpand\@nil
  }%
\x{%
  \def\SOuL@temp{#7}%
  \ifx\SOuL@temp\SOuL@empty
    \chardef\SOuL@octets=\z@
  \else
    \def\SOuL@temp{#5}%
    \ifx\SOuL@temp\SOuL@charthree
      \chardef\SOuL@octets=4 %
    \else
      \def\SOuL@temp{#3}%
      \ifx\SOuL@temp\SOuL@chartwo
        \chardef\SOuL@octets=\thr@@
      \else
        \chardef\SOuL@octets=\tw@
      \fi
    \fi
  \fi
}
%    \end{macrocode}
%    \end{macro}
%
% \subsection{Actions for UTF-8 sequences}
%
%    \begin{macro}{\SOuL@addtwooctets}
%    \begin{macrocode}
\def\SOuL@addtwooctets#1#2{%
  \def\SOuL@temp{#1#2}%
  \@onelevel@sanitize\SOuL@temp
  \ifx\SOuL@temp\SOuL@stringnobreakspace
    \SOUL@doword
    \SOUL@eventuallyexhyphen\null
    \SOUL@everyspace{\nobreak}%
    \let\SOuL@next\SOUL@scan
  \else
    \def\SOuL@next{%
!     \SOUL@addtoken{{\noexpand#1\noexpand#2}}%
+     \SOUL@addtoken{{\etex@unexpanded{#1#2}}}%
    }%
  \fi
  \SOuL@next
}
%    \end{macrocode}
%    \end{macro}
%    \begin{macro}{\SOuL@addthreeoctets}
%    \begin{macrocode}
\def\SOuL@addthreeoctets#1#2#3{%
  \def\SOuL@temp{#1#2#3}%
  \@onelevel@sanitize\SOuL@temp
  \ifx\SOuL@temp\SOuL@stringendash
    \SOUL@doword
    \SOUL@eventuallyexhyphen{-}%
    \SOUL@exhyphen{--}%
    \let\SOuL@next\SOUL@scan
  \else
    \ifx\SOuL@temp\SOuL@stringemdash
      \SOUL@doword
      \SOUL@eventuallyexhyphen{-}%
      \SOUL@exhyphen{---}%
      \let\SOuL@next\SOUL@scan
    \else
      \def\SOuL@next{%
!       \SOUL@addtoken{{\noexpand#1\noexpand#2\noexpand#3}}%
+       \SOUL@addtoken{{\etex@unexpanded{#1#2#3}}}%
      }%
    \fi
  \fi
  \SOuL@next
}
%    \end{macrocode}
%    \end{macro}
%    \begin{macro}{\SOuL@addfouroctets}
%    \begin{macrocode}
\def\SOuL@addfouroctets#1#2#3#4{%
! \SOUL@addtoken{{\noexpand#1\noexpand#2\noexpand#3\noexpand#4}}%
+ \SOUL@addtoken{{\etex@unexpanded{#1#2#3#4}}}%
}
%    \end{macrocode}
%    \end{macro}
%
% \subsubsection{Redefinition of \cs{SOUL@splittoken}}
%
%    \begin{macro}{\SOUL@splittoken}
%    Macro \cs{SOUL@splittoken} separates the first token
%    or token group from a word and redefines the word to contain
%    the remaining tokens. However if the remaining tokens are
%    a token group, then the curly braces will be removed and
%    the token group is splitted by the next call of \cs{SOUL@splittoken}.
%    The redefinition avoids the removal of curly braces around the
%    remaining tokens.
%    \begin{macrocode}
\SOuL@redefine\SOUL@splittoken#1#2\SOUL@stop{%
  \global\SOUL@token={#1}%
  \global\SOUL@word={#2}%
}#1{%
  \global\SOUL@token={#1}%
  \SOuL@remainingtoken\relax
}
%    \end{macrocode}
%    \end{macro}
%    \begin{macro}{\SOuL@remainingtoken}
%    \begin{macrocode}
\def\SOuL@remainingtoken#1\SOUL@stop{%
  \global\SOUL@word=\expandafter{\@gobble#1}%
}
%    \end{macrocode}
%    \end{macro}
%
% \subsection{Patches}
%
%    The fixed \cs{SOUL@splittoken} allows to remove the double
%    sets of curly braces in other macros of package \xpackage{soul}.
%    The benefit is that implicite kernings are more often detected
%    and fixes a bug in package \xpackage{soul}. The disadvantage is
%    incompatibility. The width of the resulting strings may change.
%    \begin{macro}{\SOUL@flushcomma}
%    \begin{macrocode}
\SOuL@redefine\SOUL@flushcomma{%
  \ifcase\SOUL@comma
  \or
    \edef\x{\SOUL@word={\the\SOUL@word,}}\x
  \or
    \edef\x{\SOUL@word={\the\SOUL@word{{,,}}}}\x
  \fi
  \SOUL@comma\z@
}{%
  \ifcase\SOUL@comma
  \or
    \edef\x{\SOUL@word={\the\SOUL@word,}}\x
  \or
    \edef\x{\SOUL@word={\the\SOUL@word{,,}}}\x
  \fi
  \SOUL@comma\z@
}
%    \end{macrocode}
%    \end{macro}
%    \begin{macro}{\SOUL@flushapo}
%    \begin{macrocode}
\SOuL@redefine\SOUL@flushapo{%
  \ifcase\SOUL@apo
  \or
    \edef\x{\SOUL@word={\the\SOUL@word'}}\x
  \or
    \edef\x{\SOUL@word={\the\SOUL@word{{''}}}}\x
  \fi
  \SOUL@apo\z@
}{%
  \ifcase\SOUL@apo
  \or
    \edef\x{\SOUL@word={\the\SOUL@word'}}\x
  \or
    \edef\x{\SOUL@word={\the\SOUL@word{''}}}\x
  \fi
  \SOUL@apo\z@
}
%    \end{macrocode}
%    \end{macro}
%    \begin{macro}{\SOUL@flushgrave}
%    \begin{macrocode}
\SOuL@redefine\SOUL@flushgrave{%
  \ifcase\SOUL@grave
  \or
    \edef\x{\SOUL@word={\the\SOUL@word`}}\x
  \or
    \edef\x{\SOUL@word={\the\SOUL@word{{``}}}}\x
  \fi
  \SOUL@grave\z@
}{%
  \ifcase\SOUL@grave
  \or
    \edef\x{\SOUL@word={\the\SOUL@word`}}\x
  \or
    \edef\x{\SOUL@word={\the\SOUL@word{``}}}\x
  \fi
  \SOUL@grave\z@
}
%    \end{macrocode}
%    \end{macro}
%    \begin{macro}{\SOUL@addgroup}
%    \begin{macrocode}
\SOuL@redefine\SOUL@addgroup#1#2{%
  {%
    \let\protect\noexpand
    \edef\x{%
      \global\SOUL@word={%
        \the\SOUL@word
        {{\noexpand#1#2}}%
      }%
    }%
    \x
  }%
  \SOUL@scan
}#1#2{%
  \begingroup
    \let\protect\noexpand
    \edef\x{\endgroup
      \SOUL@word={%
        \the\SOUL@word
!       {\noexpand#1{#2}}%
+       {\etex@unexpanded{#1{#2}}}%
      }%
    }%
  \x
  \SOUL@scan
}
%    \end{macrocode}
%    \end{macro}
%    \begin{macro}{\SOUL@addmath}
%    \begin{macrocode}
\SOuL@redefine\SOUL@addmath$#1${%
  {%
    \let\protect\noexpand
    \edef\x{%
      \global\SOUL@word={%
        \the\SOUL@word
        {{\hbox{$#1$}}}%
      }%
    }%
    \x
  }%
  \SOUL@scan
}$#1${%
  \begingroup
    \let\protect\noexpand
    \edef\x{\endgroup
      \SOUL@word={%
        \the\SOUL@word
!       {\hbox{$#1$}}%
+       {\etex@unexpanded{\hbox{$#1$}}}%
      }%
    }%
  \x
  \SOUL@scan
}
%    \end{macrocode}
%    \end{macro}
%    \begin{macro}{\SOUL@addprotect}
%    \begin{macrocode}
\SOuL@redefine\SOUL@addprotect#1#2{%
  {%
    \let\protect\noexpand
    \edef\x{%
      \global\SOUL@word={%
        \the\SOUL@word
        {{\hbox{#2}}}%
      }%
    }%
    \x
  }%
  \SOUL@scan
}#1#2{%
  \begingroup
    \let\protect\noexpand
    \edef\x{\endgroup
      \SOUL@word={%
        \the\SOUL@word
!       {\hbox{#2}}%
+       {\etex@unexpanded{\hbox{#2}}}%
      }%
    }%
  \x
  \SOUL@scan
}
%    \end{macrocode}
%    \end{macro}
%
%    \begin{macro}{\SOUL@addtoken}
%    \begin{macrocode}
+ \SOuL@redefine\SOUL@addtoken#1{%
+   \edef\x{%
+     \SOUL@word={%
+       \the\SOUL@word
+       \noexpand#1%
+     }%
+   }%
+   \x
+   \SOUL@scan
+ }#1{%
+   \edef\x{%
+     \SOUL@word={%
+       \the\SOUL@word
+       \etex@unexpanded{#1}%
+     }%
+   }%
+   \x
+   \SOUL@scan
+ }%
%    \end{macrocode}
%    \end{macro}
%
%    \begin{macrocode}
\SOuL@AtEnd%
%    \end{macrocode}
%
%    \begin{macrocode}
%</package>
%    \end{macrocode}
%
% \section{Test}
%
% \subsection{Catcode checks for loading}
%
%    \begin{macrocode}
%<*test1>
%    \end{macrocode}
%    \begin{macrocode}
\catcode`\{=1 %
\catcode`\}=2 %
\catcode`\#=6 %
\catcode`\@=11 %
\expandafter\ifx\csname count@\endcsname\relax
  \countdef\count@=255 %
\fi
\expandafter\ifx\csname @gobble\endcsname\relax
  \long\def\@gobble#1{}%
\fi
\expandafter\ifx\csname @firstofone\endcsname\relax
  \long\def\@firstofone#1{#1}%
\fi
\expandafter\ifx\csname loop\endcsname\relax
  \expandafter\@firstofone
\else
  \expandafter\@gobble
\fi
{%
  \def\loop#1\repeat{%
    \def\body{#1}%
    \iterate
  }%
  \def\iterate{%
    \body
      \let\next\iterate
    \else
      \let\next\relax
    \fi
    \next
  }%
  \let\repeat=\fi
}%
\def\RestoreCatcodes{}
\count@=0 %
\loop
  \edef\RestoreCatcodes{%
    \RestoreCatcodes
    \catcode\the\count@=\the\catcode\count@\relax
  }%
\ifnum\count@<255 %
  \advance\count@ 1 %
\repeat

\def\RangeCatcodeInvalid#1#2{%
  \count@=#1\relax
  \loop
    \catcode\count@=15 %
  \ifnum\count@<#2\relax
    \advance\count@ 1 %
  \repeat
}
\def\RangeCatcodeCheck#1#2#3{%
  \count@=#1\relax
  \loop
    \ifnum#3=\catcode\count@
    \else
      \errmessage{%
        Character \the\count@\space
        with wrong catcode \the\catcode\count@\space
        instead of \number#3%
      }%
    \fi
  \ifnum\count@<#2\relax
    \advance\count@ 1 %
  \repeat
}
\def\space{ }
\expandafter\ifx\csname LoadCommand\endcsname\relax
  \def\LoadCommand{\input soulutf8.sty\relax}%
\fi
\def\Test{%
  \RangeCatcodeInvalid{0}{47}%
  \RangeCatcodeInvalid{58}{64}%
  \RangeCatcodeInvalid{91}{96}%
  \RangeCatcodeInvalid{123}{255}%
  \catcode`\@=12 %
  \catcode`\\=0 %
  \catcode`\%=14 %
  \LoadCommand
  \RangeCatcodeCheck{0}{36}{15}%
  \RangeCatcodeCheck{37}{37}{14}%
  \RangeCatcodeCheck{38}{47}{15}%
  \RangeCatcodeCheck{48}{57}{12}%
  \RangeCatcodeCheck{58}{63}{15}%
  \RangeCatcodeCheck{64}{64}{12}%
  \RangeCatcodeCheck{65}{90}{11}%
  \RangeCatcodeCheck{91}{91}{15}%
  \RangeCatcodeCheck{92}{92}{0}%
  \RangeCatcodeCheck{93}{96}{15}%
  \RangeCatcodeCheck{97}{122}{11}%
  \RangeCatcodeCheck{123}{255}{15}%
  \RestoreCatcodes
}
\Test
\csname @@end\endcsname
\end
%    \end{macrocode}
%    \begin{macrocode}
%</test1>
%    \end{macrocode}
%
% \subsection{\LaTeX\space tests}
%
%    \begin{macrocode}
%<*test2|test3|test4|test5>
%    \end{macrocode}
%    \begin{macrocode}
%<noetex>\let\SavedUnexpanded\unexpanded
%<noetex>\let\unexpanded\UNDEFINED
\NeedsTeXFormat{LaTeX2e}
\documentclass{article}
\usepackage[T1]{fontenc}
\usepackage{lmodern}
\usepackage{textcomp}
\usepackage[ngerman]{babel}
\expandafter\def\csname ver@hypdestopt.sty\endcsname{}
\usepackage[unicode]{hyperref}
\usepackage{bookmark}
\usepackage{color}
\usepackage{soul}[2003/11/17]
\makeatletter
\newcommand*{\@ifDefined}[1]{%
  \@ifundefined{#1}{%
    \@latex@error{%
      \expandafter\noexpand\csname#1\endcsname undefined%
    }\@ehc
  }%
}
\newcommand*{\SaveOrg}[1]{%
  \@ifdefinable{ORG@SOUL@#1}{%
    \@ifDefined{SOUL@#1}{%
      \expandafter\let\csname ORG@SOUL@#1\expandafter\endcsname
                      \csname SOUL@#1\endcsname
    }%
  }%
}
\newcommand*{\RestoreOrg}[1]{%
  \@ifDefined{ORG@SOUL@#1}{%
    \@ifDefined{SOUL@#1}{%
      \expandafter\let\csname SOUL@#1\expandafter\endcsname
                      \csname ORG@SOUL@#1\endcsname
    }%
  }%
}
\makeatother
\SaveOrg{eval}
\SaveOrg{splittoken}
\SaveOrg{flushcomma}
\SaveOrg{flushapo}
\SaveOrg{grave}
\SaveOrg{addgroup}
\SaveOrg{addmath}
\SaveOrg{addprotect}
\SaveOrg{addtoken}
\usepackage{soulutf8}[2016/05/16]
%    \end{macrocode}
%    \begin{macrocode}
%<noetex>\let\unexpanded\SavedUnexpanded
%    \end{macrocode}
%    Input encoding \xfile{utf8}.
%    \begin{macrocode}
%<*utf8>
\usepackage[utf8]{inputenc}
\DeclareUnicodeCharacter{00A0}{\nobreakspace}
%</utf8>
%    \end{macrocode}
%    Input encoding \xfile{utf8x}. Package \xpackage{ucs}
%    has a nasty bug, that it redefines \cs{show}.
%    \begin{macrocode}
%<*utf8x>
\let\origshow\show
\usepackage[utf8x]{inputenc}
\AtBeginDocument{%
  \let\show\origshow
}
%</utf8x>
%    \end{macrocode}
%
% \subsubsection{Driver definitions}
%
%    \begin{macrocode}
\makeatletter
%    \end{macrocode}
%    \begin{macro}{\an}
%    The definition for \cs{an} is taken from the documentation of
%    package \xpackage{soul}.
%    \begin{macrocode}
\DeclareRobustCommand*{\an}{%
  \def\SOUL@preamble{$^{^P}$}%
  \def\SOUL@everyspace##1{##1\texttt{\char`\ }\linebreak[2]}%
  \def\SOUL@postamble{$^{^E}$}%
  \def\SOUL@everyhyphen{$^{^-}$\linebreak[1]}%
  \def\SOUL@everyexhyphen##1{##1$^{^=}$}%
  \def\SOUL@everysyllable{$^{^S}$}%
  \def\SOUL@everytoken{\the\SOUL@token$^{^T}$}%
  \def\SOUL@everylowerthan{$^{^L}$}%
  \SOUL@
}
%    \end{macrocode}
%    \end{macro}
%
%    \begin{macro}{\sy}
%    The definition for \cs{sy} is taken from the documentation of
%    package \xpackage{soul}.
%    \begin{macrocode}
\DeclareRobustCommand*{\sy}{%
  \SOUL@setup
  \def\SOUL@preamble{%
    \lefthyphenmin=\z@
    \righthyphenmin=\z@
  }%
  \def\SOUL@everyspace##1{##1\space}%
  \def\SOUL@everysyllable{\the\SOUL@syllable}%
  \def\SOUL@everyhyphen{%
    \discretionary{%
      \SOUL@setkern\SOUL@hyphkern
      \SOUL@sethyphenchar
    }{}{%
      \hbox{\kern1pt$\cdot$}%
    }%
  }%
  \def\SOUL@everyexhyphen##1{%
    \SOUL@setkern\SOUL@hyphkern
    \hbox{##1}%
    \discretionary{}{}{%
      \SOUL@setkern\SOUL@charkern
    }%
  }%
  \SOUL@
}
%    \end{macrocode}
%    \end{macro}
%
%    \begin{macro}{\soulH}
%    \begin{macrocode}
\DeclareRobustCommand*{\soulH}{%
  \def\SOUL@preamble{%
    \gdef\SOULH{}%
    \soulHadd{[}%
  }%
  \def\SOUL@everyspace##1{%
    \soulHadd{Sp}%
  }%
  \def\SOUL@postamble{%
    \soulHadd{]}%
  }%
  \def\SOUL@everyhyphen{%
    \soulHadd{Hy}%
  }%
  \def\SOUL@everyexhyphen##1{%
    \soulHadd{Exhy}%
  }%
  \def\SOUL@everysyllable{%
    \soulHadd{Sy}%
  }%
  \def\SOUL@everytoken{%
    \soulHadd{T}%
  }%
  \def\SOUL@everylowerthan{%
    \soulHadd{<}%
  }%
  \SOUL@
}
\newcommand*{\soulHadd}[1]{%
  #1%
  \expandafter\gdef\expandafter\soulHstring\expandafter{%
    \soulHstring
    #1%
  }%
}
%    \end{macrocode}
%    \end{macro}
%
% \subsubsection{Test macros}
%
%    \begin{macrocode}
\usepackage{qstest}
\IncludeTests{*}
\LogTests{log}{*}{*}
%    \end{macrocode}
%    \begin{macrocode}
\newsavebox\boxA
\newsavebox\boxB
\newsavebox\boxC
%    \end{macrocode}
%    \begin{macro}{\TestDriver}
%    \begin{macrocode}
\newcommand*{\TestDriver}[4]{%
  \subsection{Driver #1}%
  \@PackageInfo{soulutf8-test}{* Driver #1}%
  \expandafter\let\expandafter\testdriver\csname #1\endcsname
  \def\teststringA{#3}%
  \def\teststringB{#4}%
  \begin{qstest}{string-#2-driver-#1}{string,#2,driver,#1}%
    \@PackageInfo{soulutf8-test}{* Version: soulutf8}%
    \global\let\soulHstring\@empty
    \setbox0=\hbox{\expandafter\testdriver\expandafter{\teststringA}}%
    \let\HstringA\soulHstring
    \SetTestVBox{\boxA}{\teststringA}%
    %
    \@PackageInfo{soulutf8-test}{* Version: soul + patch}%
    \global\let\soulHstring\@empty
    \RestoreOrg{eval}%
    \setbox2=\hbox{\expandafter\testdriver\expandafter{\teststringB}}%
    \let\HstringB\soulHstring
    \SetTestVBox{\boxB}{\teststringB}%
    %
    \@PackageInfo{soulutf8-test}{* Version: original soul}%
    \global\let\soulHstring\@empty
    \RestoreOrg{splittoken}%
    \RestoreOrg{flushcomma}%
    \RestoreOrg{flushapo}%
    \RestoreOrg{grave}%
    \RestoreOrg{addgroup}%
    \RestoreOrg{addmath}%
    \RestoreOrg{addprotect}%
    \RestoreOrg{addtoken}%
    \setbox4=\hbox{\expandafter\testdriver\expandafter{\teststringB}}%
    \let\HstringC\soulHstring
    \SetTestVBox{\boxC}{\teststringB}%
    %
    \TestInfo{0}{soulutf8}%
    \TestInfo{2}{soul + patch}%
    \TestInfo{4}{original soul}%
    \TestInfoVBox
    \showboxdepth=\maxdimen
    \showboxbreadth=\maxdimen
    \tracingonline=\@ne
    \Expect*{\HstringA}*{\HstringB}%
    \Expect*{\HstringA}*{\HstringC}%
    \Expect*{\the\wd0}*{\the\wd2}%
    \ifdim\wd0=\wd2 %
    \else
      \showbox0 %
      \showbox2 %
    \fi
    \Expect*{\the\ht0}*{\the\ht2}%
    \Expect*{\the\dp0}*{\the\dp2}%
    \Expect*{\the\wd\boxB}*{\the\wd\boxA}%
    \Expect*{\the\wd\boxC}*{\the\wd\boxA}%
    \Expect*{\the\ht\boxB}*{\the\ht\boxA}%
    \ifdim\ht\boxB=\ht\boxA
    \else
      \showbox\boxA
      \showbox\boxB
    \fi
    \Expect*{\the\ht\boxC}*{\the\ht\boxA}%
    \Expect*{\the\dp\boxB}*{\the\dp\boxA}%
    \Expect*{\the\dp\boxC}*{\the\dp\boxA}%
  \end{qstest}%
}
%    \end{macrocode}
%    \end{macro}
%    \begin{macro}{\SetTestVBox}
%    \begin{macrocode}
\newcommand*{\SetTestVBox}[2]{%
  \setbox#1=\vbox{%
    \hsize=0pt\relax
    \hfuzz=100mm\relax
    \expandafter\testdriver\expandafter{#2}%
  }%
}
%    \end{macrocode}
%    \end{macro}
%    \begin{macro}{\TestInfo}
%    \begin{macrocode}
\newcommand*{\TestInfo}[2]{%
  \par
  \noindent
  \copy#1\relax
  \hfill
  \begingroup
    \small
    \ifdim\wd0=\wd2\relax
      \ifdim\wd0=\wd4\relax
      \else
        \ifnum#1=2 %
          \raisebox{.5\baselineskip}[0pt][0pt]{%
            \llap{$\}$ }%
            \rlap{width = \the\wd#1\relax\space$\{$}%
            \phantom{width = 999.99999pt}%
            \qquad
          }%
        \else
          \ifnum#1=4 %
            \rlap{width = \the\wd#1}%
            \phantom{width = 999.99999pt}%
            \qquad
          \fi
        \fi
      \fi
    \else
      \begingroup
        \bfseries
        \rlap{width = \the\wd#1}%
        \phantom{width = 999.99999pt}%
      \endgroup
      \qquad
    \fi
    \rlap{(#2)}%
    \phantom{(original soul)}%
  \endgroup
  \par
}
%    \end{macrocode}
%    \end{macro}
%    \begin{macro}{\TestInfoVBox}
%    \begin{macrocode}
\newcommand*{\TestInfoVBox}{%
  \par
  \noindent
  \mbox{\vrule\relax\space\copy\boxA}%
  \ifdim\ht\boxA=\ht\boxB
  \else
    \hfill
    \begingroup
      \small
      \rlap{(soulutf8)}%
      \phantom{(soulutf8)}%
    \endgroup
    \par
    \noindent
    \mbox{\vrule\relax\space\copy\boxB}%
    \hfill
    \begingroup
      \small
      \rlap{(soul)}%
      \phantom{(soulutf8)}%
    \endgroup
  \fi
  \par
}
%    \end{macrocode}
%    \end{macro}
%    \begin{macro}{\Test}
%    \begin{macrocode}
\makeatletter
\newcommand*{\Test}[3]{%
  \clearpage
  \begingroup
    \def\linebreak{\textbackslash linebreak }%
    \def\\{\textbackslash\textbackslash}%
    \def\soulomit##1{%
      \textbackslash soulomit\textbraceleft##1\textbraceright
    }%
    \def\mbox##1{%
      \textbackslash mbox\textbraceleft##1\textbraceright
    }%
    \def\abc{\textbackslash abc }%
    \def\texttt##1{%
      \textbackslash texttt\textbraceleft##1\textbraceright
    }%
    \def\c##1{%
      \textbackslash c\textbraceleft##1\textbraceright
    }%
%<*utf8x>
    \count@=128 %
    \loop
    \ifnum\count@<256 %
      \lccode`\~=\count@
      \lowercase{%
        \def~{\noexpand~}%
      }%
      \advance\count@\@ne
    \repeat
%</utf8x>
    \csname protected@edef\endcsname\x{\endgroup
      \noexpand\section{String #1: #2}%
    }%
  \x
  \TestDriver{so}{#1}{#2}{#3}%
  \TestDriver{caps}{#1}{#2}{#3}%
  \TestDriver{ul}{#1}{#2}{#3}%
  \TestDriver{st}{#1}{#2}{#3}%
  \TestDriver{hl}{#1}{#2}{#3}%
  \TestDriver{an}{#1}{#2}{#3}%
  \TestDriver{sy}{#1}{#2}{#3}%
  \TestDriver{soulH}{#1}{#2}{#3}%
}
\makeatother
%    \end{macrocode}
%    \end{macro}
%
% \subsubsection{Testing}
%
%    \begin{macrocode}
\begin{document}
  \Test A{}{}%
  \Test B{Hello World}{Hello World}%
  \Test C{%
    Umlauts: ^^c3^^84^^c3^^96^^c3^^9c^^c3^^a4^^c3^^b6^^c3^^bc^^c3^^9f%
  }{%
    Umlauts: \"A\"O\"U\"a\"o\"u\ss
  }%
  \Test D{Euro: ^^e2^^82^^ac}{Euro: \texteuro}%
  \Test E{Stra^^c3^^9fe}{Stra\ss e}%
  \Test F{R^^c3^^b6mer}{R\"omer}%
  \Test G{A^^e2^^80^^93B^^e2^^80^^94C}{A--B---C}%
  \Test H{X^^c2^^a0Y}{X~Y}%
  \Test I{A^^c5^^a4}{A\v T}%
  \Test J{A,B,,C,,,D,,,,E}{A,B,,C,,,D,,,,E}%
  \Test K{A'B''C'''D''''E}{A'B''C'''D''''E}%
  \Test L{A`B``C```D````E}{A`B``C```D````E}%
  \Test M{A$x$B}{A$x$B}%
  \Test N{A\mbox{abc}B}{A\mbox{abc}B}%
  \Test O{,,TA''}{,,TA''}%
  \Test P{{!`}Hola{?`}}{{!`}Hola{?`}}%
  \Test Q{re-sent\slash x}{re-sent\slash x}%
  \Test R{new\\line}{new\\line}%
  \Test S{foo\linebreak bar}{foo\linebreak bar}%
  \Test T{Virchow{sche}}{Virchow{sche}}%
  \Test U{A\soulomit{foo}B}{A\soulomit{foo}B}%
  \Test V{A\texttt{xyz}B}{A\texttt{xyz}B}%
  \Test W{ffi}{ffi}%
  \Test X{f{}f{}i}{f{}f{}i}%
  \Test Y{f\mbox{}f\mbox{}i}{f\mbox{}f\mbox{}i}%
  \Test Z{\c{\"a}}{{\c{\"a}}}%
%<*test2|test3>
  \let\abcx\UNDEFINED
  \def\abc{%
    \begingroup
      \def\abcx{x}%
      \abcx
    \endgroup
  }%
  \Test{AA}{[{x\abc x}]}{[{xxx}]}%
%</test2|test3>
%    \end{macrocode}
%
%    \begin{macrocode}
\end{document}
%    \end{macrocode}
%    \begin{macrocode}
%</test2|test3|test4|test5>
%    \end{macrocode}
%
% \section{Installation}
%
% \subsection{Download}
%
% \paragraph{Package.} This package is available on
% CTAN\footnote{\url{http://ctan.org/pkg/soulutf8}}:
% \begin{description}
% \item[\CTAN{macros/latex/contrib/oberdiek/soulutf8.dtx}] The source file.
% \item[\CTAN{macros/latex/contrib/oberdiek/soulutf8.pdf}] Documentation.
% \end{description}
%
%
% \paragraph{Bundle.} All the packages of the bundle `oberdiek'
% are also available in a TDS compliant ZIP archive. There
% the packages are already unpacked and the documentation files
% are generated. The files and directories obey the TDS standard.
% \begin{description}
% \item[\CTAN{install/macros/latex/contrib/oberdiek.tds.zip}]
% \end{description}
% \emph{TDS} refers to the standard ``A Directory Structure
% for \TeX\ Files'' (\CTAN{tds/tds.pdf}). Directories
% with \xfile{texmf} in their name are usually organized this way.
%
% \subsection{Bundle installation}
%
% \paragraph{Unpacking.} Unpack the \xfile{oberdiek.tds.zip} in the
% TDS tree (also known as \xfile{texmf} tree) of your choice.
% Example (linux):
% \begin{quote}
%   |unzip oberdiek.tds.zip -d ~/texmf|
% \end{quote}
%
% \paragraph{Script installation.}
% Check the directory \xfile{TDS:scripts/oberdiek/} for
% scripts that need further installation steps.
% Package \xpackage{attachfile2} comes with the Perl script
% \xfile{pdfatfi.pl} that should be installed in such a way
% that it can be called as \texttt{pdfatfi}.
% Example (linux):
% \begin{quote}
%   |chmod +x scripts/oberdiek/pdfatfi.pl|\\
%   |cp scripts/oberdiek/pdfatfi.pl /usr/local/bin/|
% \end{quote}
%
% \subsection{Package installation}
%
% \paragraph{Unpacking.} The \xfile{.dtx} file is a self-extracting
% \docstrip\ archive. The files are extracted by running the
% \xfile{.dtx} through \plainTeX:
% \begin{quote}
%   \verb|tex soulutf8.dtx|
% \end{quote}
%
% \paragraph{TDS.} Now the different files must be moved into
% the different directories in your installation TDS tree
% (also known as \xfile{texmf} tree):
% \begin{quote}
% \def\t{^^A
% \begin{tabular}{@{}>{\ttfamily}l@{ $\rightarrow$ }>{\ttfamily}l@{}}
%   soulutf8.sty & tex/generic/oberdiek/soulutf8.sty\\
%   soulutf8.pdf & doc/latex/oberdiek/soulutf8.pdf\\
%   test/soulutf8-test1.tex & doc/latex/oberdiek/test/soulutf8-test1.tex\\
%   test/soulutf8-test2.tex & doc/latex/oberdiek/test/soulutf8-test2.tex\\
%   test/soulutf8-test3.tex & doc/latex/oberdiek/test/soulutf8-test3.tex\\
%   test/soulutf8-test4.tex & doc/latex/oberdiek/test/soulutf8-test4.tex\\
%   test/soulutf8-test5.tex & doc/latex/oberdiek/test/soulutf8-test5.tex\\
%   soulutf8.dtx & source/latex/oberdiek/soulutf8.dtx\\
% \end{tabular}^^A
% }^^A
% \sbox0{\t}^^A
% \ifdim\wd0>\linewidth
%   \begingroup
%     \advance\linewidth by\leftmargin
%     \advance\linewidth by\rightmargin
%   \edef\x{\endgroup
%     \def\noexpand\lw{\the\linewidth}^^A
%   }\x
%   \def\lwbox{^^A
%     \leavevmode
%     \hbox to \linewidth{^^A
%       \kern-\leftmargin\relax
%       \hss
%       \usebox0
%       \hss
%       \kern-\rightmargin\relax
%     }^^A
%   }^^A
%   \ifdim\wd0>\lw
%     \sbox0{\small\t}^^A
%     \ifdim\wd0>\linewidth
%       \ifdim\wd0>\lw
%         \sbox0{\footnotesize\t}^^A
%         \ifdim\wd0>\linewidth
%           \ifdim\wd0>\lw
%             \sbox0{\scriptsize\t}^^A
%             \ifdim\wd0>\linewidth
%               \ifdim\wd0>\lw
%                 \sbox0{\tiny\t}^^A
%                 \ifdim\wd0>\linewidth
%                   \lwbox
%                 \else
%                   \usebox0
%                 \fi
%               \else
%                 \lwbox
%               \fi
%             \else
%               \usebox0
%             \fi
%           \else
%             \lwbox
%           \fi
%         \else
%           \usebox0
%         \fi
%       \else
%         \lwbox
%       \fi
%     \else
%       \usebox0
%     \fi
%   \else
%     \lwbox
%   \fi
% \else
%   \usebox0
% \fi
% \end{quote}
% If you have a \xfile{docstrip.cfg} that configures and enables \docstrip's
% TDS installing feature, then some files can already be in the right
% place, see the documentation of \docstrip.
%
% \subsection{Refresh file name databases}
%
% If your \TeX~distribution
% (\teTeX, \mikTeX, \dots) relies on file name databases, you must refresh
% these. For example, \teTeX\ users run \verb|texhash| or
% \verb|mktexlsr|.
%
% \subsection{Some details for the interested}
%
% \paragraph{Attached source.}
%
% The PDF documentation on CTAN also includes the
% \xfile{.dtx} source file. It can be extracted by
% AcrobatReader 6 or higher. Another option is \textsf{pdftk},
% e.g. unpack the file into the current directory:
% \begin{quote}
%   \verb|pdftk soulutf8.pdf unpack_files output .|
% \end{quote}
%
% \paragraph{Unpacking with \LaTeX.}
% The \xfile{.dtx} chooses its action depending on the format:
% \begin{description}
% \item[\plainTeX:] Run \docstrip\ and extract the files.
% \item[\LaTeX:] Generate the documentation.
% \end{description}
% If you insist on using \LaTeX\ for \docstrip\ (really,
% \docstrip\ does not need \LaTeX), then inform the autodetect routine
% about your intention:
% \begin{quote}
%   \verb|latex \let\install=y\input{soulutf8.dtx}|
% \end{quote}
% Do not forget to quote the argument according to the demands
% of your shell.
%
% \paragraph{Generating the documentation.}
% You can use both the \xfile{.dtx} or the \xfile{.drv} to generate
% the documentation. The process can be configured by the
% configuration file \xfile{ltxdoc.cfg}. For instance, put this
% line into this file, if you want to have A4 as paper format:
% \begin{quote}
%   \verb|\PassOptionsToClass{a4paper}{article}|
% \end{quote}
% An example follows how to generate the
% documentation with pdf\LaTeX:
% \begin{quote}
%\begin{verbatim}
%pdflatex soulutf8.dtx
%makeindex -s gind.ist soulutf8.idx
%pdflatex soulutf8.dtx
%makeindex -s gind.ist soulutf8.idx
%pdflatex soulutf8.dtx
%\end{verbatim}
% \end{quote}
%
% \begin{thebibliography}{9}
%
% \bibitem{soul}
%   Melchior Franz: \textit{The \xpackage{soul} package};
%   2003/11/17;\\
%   \CTAN{macros/latex/contrib/soul/}.
%
% \bibitem{ucs}
%   Dominique P. G. Unruh: \textit{\xpackage{ucs.sty} -- Unicode Support};
%   2004/10/17;\\
%   \CTAN{macros/latex/contrib/unicode/}.
%
% \bibitem{utf8}
%   Frank Mittelbach, Chris Rowley:
%   \textit{Providing some UTF-8 support via \xpackage{inputenc}};
%   2006/03/30;\\
%   \CTAN{macros/latex/base/utf8ienc.dtx}.
%
% \end{thebibliography}
%
% \begin{History}
%   \begin{Version}{2007/09/09 v1.0}
%   \item
%     First version.
%   \end{Version}
%   \begin{Version}{2016/05/16 v1.1}
%   \item
%     Documentation updates.
%   \end{Version}
% \end{History}
%
% \PrintIndex
%
% \Finale
\endinput
|
% \end{quote}
% Do not forget to quote the argument according to the demands
% of your shell.
%
% \paragraph{Generating the documentation.}
% You can use both the \xfile{.dtx} or the \xfile{.drv} to generate
% the documentation. The process can be configured by the
% configuration file \xfile{ltxdoc.cfg}. For instance, put this
% line into this file, if you want to have A4 as paper format:
% \begin{quote}
%   \verb|\PassOptionsToClass{a4paper}{article}|
% \end{quote}
% An example follows how to generate the
% documentation with pdf\LaTeX:
% \begin{quote}
%\begin{verbatim}
%pdflatex soulutf8.dtx
%makeindex -s gind.ist soulutf8.idx
%pdflatex soulutf8.dtx
%makeindex -s gind.ist soulutf8.idx
%pdflatex soulutf8.dtx
%\end{verbatim}
% \end{quote}
%
% \begin{thebibliography}{9}
%
% \bibitem{soul}
%   Melchior Franz: \textit{The \xpackage{soul} package};
%   2003/11/17;\\
%   \CTAN{macros/latex/contrib/soul/}.
%
% \bibitem{ucs}
%   Dominique P. G. Unruh: \textit{\xpackage{ucs.sty} -- Unicode Support};
%   2004/10/17;\\
%   \CTAN{macros/latex/contrib/unicode/}.
%
% \bibitem{utf8}
%   Frank Mittelbach, Chris Rowley:
%   \textit{Providing some UTF-8 support via \xpackage{inputenc}};
%   2006/03/30;\\
%   \CTAN{macros/latex/base/utf8ienc.dtx}.
%
% \end{thebibliography}
%
% \begin{History}
%   \begin{Version}{2007/09/09 v1.0}
%   \item
%     First version.
%   \end{Version}
%   \begin{Version}{2016/05/16 v1.1}
%   \item
%     Documentation updates.
%   \end{Version}
% \end{History}
%
% \PrintIndex
%
% \Finale
\endinput

%        (quote the arguments according to the demands of your shell)
%
% Documentation:
%    (a) If soulutf8.drv is present:
%           latex soulutf8.drv
%    (b) Without soulutf8.drv:
%           latex soulutf8.dtx; ...
%    The class ltxdoc loads the configuration file ltxdoc.cfg
%    if available. Here you can specify further options, e.g.
%    use A4 as paper format:
%       \PassOptionsToClass{a4paper}{article}
%
%    Programm calls to get the documentation (example):
%       pdflatex soulutf8.dtx
%       makeindex -s gind.ist soulutf8.idx
%       pdflatex soulutf8.dtx
%       makeindex -s gind.ist soulutf8.idx
%       pdflatex soulutf8.dtx
%
% Installation:
%    TDS:tex/generic/oberdiek/soulutf8.sty
%    TDS:doc/latex/oberdiek/soulutf8.pdf
%    TDS:doc/latex/oberdiek/test/soulutf8-test1.tex
%    TDS:doc/latex/oberdiek/test/soulutf8-test2.tex
%    TDS:doc/latex/oberdiek/test/soulutf8-test3.tex
%    TDS:doc/latex/oberdiek/test/soulutf8-test4.tex
%    TDS:doc/latex/oberdiek/test/soulutf8-test5.tex
%    TDS:source/latex/oberdiek/soulutf8.dtx
%
%<*ignore>
\begingroup
  \catcode123=1 %
  \catcode125=2 %
  \def\x{LaTeX2e}%
\expandafter\endgroup
\ifcase 0\ifx\install y1\fi\expandafter
         \ifx\csname processbatchFile\endcsname\relax\else1\fi
         \ifx\fmtname\x\else 1\fi\relax
\else\csname fi\endcsname
%</ignore>
%<*install>
\input docstrip.tex
\Msg{************************************************************************}
\Msg{* Installation}
\Msg{* Package: soulutf8 2016/05/16 v1.1 Permit use of UTF-8 characters in soul (HO)}
\Msg{************************************************************************}

\keepsilent
\askforoverwritefalse

\let\MetaPrefix\relax
\preamble

This is a generated file.

Project: soulutf8
Version: 2016/05/16 v1.1

Copyright (C) 2007 by
   Heiko Oberdiek <heiko.oberdiek at googlemail.com>

This work may be distributed and/or modified under the
conditions of the LaTeX Project Public License, either
version 1.3c of this license or (at your option) any later
version. This version of this license is in
   http://www.latex-project.org/lppl/lppl-1-3c.txt
and the latest version of this license is in
   http://www.latex-project.org/lppl.txt
and version 1.3 or later is part of all distributions of
LaTeX version 2005/12/01 or later.

This work has the LPPL maintenance status "maintained".

This Current Maintainer of this work is Heiko Oberdiek.

The Base Interpreter refers to any `TeX-Format',
because some files are installed in TDS:tex/generic//.

This work consists of the main source file soulutf8.dtx
and the derived files
   soulutf8.sty, soulutf8.pdf, soulutf8.ins, soulutf8.drv,
   soulutf8-test1.tex, soulutf8-test2.tex, soulutf8-test3.tex,
   soulutf8-test4.tex, soulutf8-test5.tex.

\endpreamble
\let\MetaPrefix\DoubleperCent

\generate{%
  \file{soulutf8.ins}{\from{soulutf8.dtx}{install}}%
  \file{soulutf8.drv}{\from{soulutf8.dtx}{driver}}%
  \usedir{tex/generic/oberdiek}%
  \file{soulutf8.sty}{\from{soulutf8.dtx}{package}}%
%  \usedir{doc/latex/oberdiek/test}%
%  \file{soulutf8-test1.tex}{\from{soulutf8.dtx}{test1}}%
%  \file{soulutf8-test2.tex}{\from{soulutf8.dtx}{test2,utf8}}%
%  \file{soulutf8-test3.tex}{\from{soulutf8.dtx}{test3,utf8x}}%
%  \file{soulutf8-test4.tex}{\from{soulutf8.dtx}{test4,utf8,noetex}}%
%  \file{soulutf8-test5.tex}{\from{soulutf8.dtx}{test5,utf8x,noetex}}%
  \nopreamble
  \nopostamble
%  \usedir{source/latex/oberdiek/catalogue}%
%  \file{soulutf8.xml}{\from{soulutf8.dtx}{catalogue}}%
}

\catcode32=13\relax% active space
\let =\space%
\Msg{************************************************************************}
\Msg{*}
\Msg{* To finish the installation you have to move the following}
\Msg{* file into a directory searched by TeX:}
\Msg{*}
\Msg{*     soulutf8.sty}
\Msg{*}
\Msg{* To produce the documentation run the file `soulutf8.drv'}
\Msg{* through LaTeX.}
\Msg{*}
\Msg{* Happy TeXing!}
\Msg{*}
\Msg{************************************************************************}

\endbatchfile
%</install>
%<*ignore>
\fi
%</ignore>
%<*driver>
\NeedsTeXFormat{LaTeX2e}
\ProvidesFile{soulutf8.drv}%
  [2016/05/16 v1.1 Permit use of UTF-8 characters in soul (HO)]%
\documentclass{ltxdoc}
\usepackage{holtxdoc}[2011/11/22]
\begin{document}
  \DocInput{soulutf8.dtx}%
\end{document}
%</driver>
% \fi
%
%
% \CharacterTable
%  {Upper-case    \A\B\C\D\E\F\G\H\I\J\K\L\M\N\O\P\Q\R\S\T\U\V\W\X\Y\Z
%   Lower-case    \a\b\c\d\e\f\g\h\i\j\k\l\m\n\o\p\q\r\s\t\u\v\w\x\y\z
%   Digits        \0\1\2\3\4\5\6\7\8\9
%   Exclamation   \!     Double quote  \"     Hash (number) \#
%   Dollar        \$     Percent       \%     Ampersand     \&
%   Acute accent  \'     Left paren    \(     Right paren   \)
%   Asterisk      \*     Plus          \+     Comma         \,
%   Minus         \-     Point         \.     Solidus       \/
%   Colon         \:     Semicolon     \;     Less than     \<
%   Equals        \=     Greater than  \>     Question mark \?
%   Commercial at \@     Left bracket  \[     Backslash     \\
%   Right bracket \]     Circumflex    \^     Underscore    \_
%   Grave accent  \`     Left brace    \{     Vertical bar  \|
%   Right brace   \}     Tilde         \~}
%
% \GetFileInfo{soulutf8.drv}
%
% \title{The \xpackage{soulutf8} package}
% \date{2016/05/16 v1.1}
% \author{Heiko Oberdiek\thanks
% {Please report any issues at https://github.com/ho-tex/oberdiek/issues}\\
% \xemail{heiko.oberdiek at googlemail.com}}
%
% \maketitle
%
% \begin{abstract}
% This package extends package \xpackage{soul} and adds some support
% for UTF-8. Namely the input encodings \xfile{utf8.def}
% from package \xpackage{inputenc} and
% package \xpackage{ucs}'s \xfile{utf8x.def} are supported.
% \end{abstract}
%
% \tableofcontents
%
% \section{Documentation}
%
% This package \xpackage{soulutf8} does not have own options and does not
% define new user commands.
% Any option is passed to package \xpackage{soul} \cite{soul}
% that is loaded first.
% Then some internal macros of \xpackage{soul} are redefined to add
% support for UTF-8.
% The following input encodings are supported:
% \begin{quote}
%   \begin{tabular}{@{}lll@{}}
%     \texttt{utf8} & \LaTeX\ base &
%        \xfile{TDS:tex/latex/base/utf8.def} \cite{utf8}\\
%     \texttt{utf8x} & Package \xpackage{ucs} &
%       \xfile{TDS:tex/latex/ucs/utf8x.def} \cite{ucs}\\
%   \end{tabular}
% \end{quote}
% UTF-8 byte sequences are added as token group to a word, even
% if these UTF-8 characters are some kind of hyphen or space.
% As exception the following three Unicode characters are handled specially:
% \begin{quote}
%   \begin{tabular}{lll}
%     Slot & Name & Action\\
%     \hline
%     \texttt{U+00A0} & NO-BREAK SPACE & like |~|\\
%     \texttt{U+2013} & EN DASH & |--|\\
%     \texttt{U+2014} & EM DASH & |---|
%   \end{tabular}
%
% \subsection{Patch}
%
% Also package \xpackage{soulutf8} tries to patch package \xpackage{soul}
% to improve its behaviour:
% \begin{itemize}
% \item
%   A problem with additional levels of curly braces is fixed.
%   As advantage more implicite kernings are detected.
%   However, the result may be incompatible with the
%   original behaviour of package \xpackage{soul} because
%   of these respected implicite kernings.
% \item
%   \eTeX\ , especially \cs{unexpanded} is supported.
%   This allows a better protection of token groups
%   (|\mbox{|\dots|}|, math, \dots).
% \end{itemize}
% \end{quote}
%
% \subsection{Future}
%
% Currently package \xpackage{soul} does not seem to be
% maintained. Nevertheless if there will be a new version that
% adds support for UTF-8, then this package may become
% obsolete.
%
% \StopEventually{
% }
%
% \section{Implementation}
%
%    \begin{macrocode}
%<*package>
%    \end{macrocode}
%
% \subsection{Reload check and package identification}
%    Reload check, especially if the package is not used with \LaTeX.
%    \begin{macrocode}
\begingroup\catcode61\catcode48\catcode32=10\relax%
  \catcode13=5 % ^^M
  \endlinechar=13 %
  \catcode35=6 % #
  \catcode39=12 % '
  \catcode44=12 % ,
  \catcode45=12 % -
  \catcode46=12 % .
  \catcode58=12 % :
  \catcode64=11 % @
  \catcode123=1 % {
  \catcode125=2 % }
  \expandafter\let\expandafter\x\csname ver@soulutf8.sty\endcsname
  \ifx\x\relax % plain-TeX, first loading
  \else
    \def\empty{}%
    \ifx\x\empty % LaTeX, first loading,
      % variable is initialized, but \ProvidesPackage not yet seen
    \else
      \expandafter\ifx\csname PackageInfo\endcsname\relax
        \def\x#1#2{%
          \immediate\write-1{Package #1 Info: #2.}%
        }%
      \else
        \def\x#1#2{\PackageInfo{#1}{#2, stopped}}%
      \fi
      \x{soulutf8}{The package is already loaded}%
      \aftergroup\endinput
    \fi
  \fi
\endgroup%
%    \end{macrocode}
%    Package identification:
%    \begin{macrocode}
\begingroup\catcode61\catcode48\catcode32=10\relax%
  \catcode13=5 % ^^M
  \endlinechar=13 %
  \catcode35=6 % #
  \catcode39=12 % '
  \catcode40=12 % (
  \catcode41=12 % )
  \catcode44=12 % ,
  \catcode45=12 % -
  \catcode46=12 % .
  \catcode47=12 % /
  \catcode58=12 % :
  \catcode64=11 % @
  \catcode91=12 % [
  \catcode93=12 % ]
  \catcode123=1 % {
  \catcode125=2 % }
  \expandafter\ifx\csname ProvidesPackage\endcsname\relax
    \def\x#1#2#3[#4]{\endgroup
      \immediate\write-1{Package: #3 #4}%
      \xdef#1{#4}%
    }%
  \else
    \def\x#1#2[#3]{\endgroup
      #2[{#3}]%
      \ifx#1\@undefined
        \xdef#1{#3}%
      \fi
      \ifx#1\relax
        \xdef#1{#3}%
      \fi
    }%
  \fi
\expandafter\x\csname ver@soulutf8.sty\endcsname
\ProvidesPackage{soulutf8}%
  [2016/05/16 v1.1 Permit use of UTF-8 characters in soul (HO)]%
%    \end{macrocode}
%
% \subsection{Catcodes}
%
%    \begin{macrocode}
\begingroup\catcode61\catcode48\catcode32=10\relax%
  \catcode13=5 % ^^M
  \endlinechar=13 %
  \catcode123=1 % {
  \catcode125=2 % }
  \catcode64=11 % @
  \def\x{\endgroup
    \expandafter\edef\csname SOuL@AtEnd\endcsname{%
      \endlinechar=\the\endlinechar\relax
      \catcode13=\the\catcode13\relax
      \catcode32=\the\catcode32\relax
      \catcode35=\the\catcode35\relax
      \catcode61=\the\catcode61\relax
      \catcode64=\the\catcode64\relax
      \catcode123=\the\catcode123\relax
      \catcode125=\the\catcode125\relax
    }%
  }%
\x\catcode61\catcode48\catcode32=10\relax%
\catcode13=5 % ^^M
\endlinechar=13 %
\catcode35=6 % #
\catcode64=11 % @
\catcode123=1 % {
\catcode125=2 % }
\def\TMP@EnsureCode#1#2{%
  \edef\SOuL@AtEnd{%
    \SOuL@AtEnd
    \catcode#1=\the\catcode#1\relax
  }%
  \catcode#1=#2\relax
}
\TMP@EnsureCode{10}{12}% ^^J
\TMP@EnsureCode{33}{12}% !
\TMP@EnsureCode{34}{12}% "
\TMP@EnsureCode{36}{3}% $
\TMP@EnsureCode{39}{12}% '
\TMP@EnsureCode{40}{12}% (
\TMP@EnsureCode{41}{12}% )
\TMP@EnsureCode{42}{12}% *
\TMP@EnsureCode{43}{12}% +
\TMP@EnsureCode{44}{12}% ,
\TMP@EnsureCode{45}{12}% -
\TMP@EnsureCode{46}{12}% .
\TMP@EnsureCode{47}{12}% /
\TMP@EnsureCode{58}{12}% :
\TMP@EnsureCode{60}{12}% <
\TMP@EnsureCode{62}{12}% >
\TMP@EnsureCode{91}{12}% [
\TMP@EnsureCode{93}{12}% ]
\TMP@EnsureCode{94}{7}% ^
\TMP@EnsureCode{96}{12}% `
\TMP@EnsureCode{126}\active % ~
\TMP@EnsureCode{128}{12}% ^^80
\TMP@EnsureCode{147}{12}% ^^93
\TMP@EnsureCode{148}{12}% ^^94
\TMP@EnsureCode{160}{12}% ^^a0
\TMP@EnsureCode{194}{12}% ^^c2
\TMP@EnsureCode{226}{12}% ^^e2
\edef\SOuL@AtEnd{\SOuL@AtEnd\noexpand\endinput}
%    \end{macrocode}
%
% \subsection{Loading packages}
%
%    Package \xpackage{soul} uses \cs{documentclass} to detect \LaTeX.
%    \begin{macrocode}
\ifx\documentclass\@undefined
%    \end{macrocode}
%
% \subsubsection{\plainTeX}
%
%    First we check, whether package \xpackage{soul} is already loaded.
%    \begin{macrocode}
  \expandafter\ifx\csname SOUL@\endcsname\relax
%    \end{macrocode}
%    In case of \plainTeX\ package \xpackage{soul} defines some
%    macros in a simple manner that will break the definitions of
%    \xfile{miniltx.tex}, for example. Therefore these macros
%    are first saved and restored afterwards.
%    \begin{macrocode}
    \let\SOuL@orgDeclareRobustCommand\DeclareRobustCommand
    \let\SOuL@orgnewcommand          \newcommand
    \let\SOuL@orgDeclareOption       \DeclareOption
    \let\SOuL@orgPackageError        \PackageError
    \def\SOuL@restorelatexcmds{%
      \let\DeclareRobustCommand\SOuL@orgDeclareRobustCommand
      \let\newcommand          \SOuL@orgnewcommand
      \let\DeclareOption       \SOuL@orgDeclareOption
      \let\PackageError        \SOuL@orgPackageError
    }%
    \input soul.sty\relax
    \SOuL@restorelatexcmds
  \fi
%    \end{macrocode}
%    \begin{macro}{\SOUL@error}
%     Package \xpackage{soul}'s use of \cs{PackageError} is replaced
%     by \cs{@PackageError} of package \xpackage{infwarerr}.
%    \begin{macrocode}
  \input infwarerr.sty\relax
  \let\SOuL@orgSOUL@error\SOUL@error
  \def\SOUL@error{%
    \begingroup
      \let\PackageError\@PackageError
      \SOuL@orgSOUL@error
    \endgroup
  }%
%    \end{macrocode}
%    \end{macro}
%    \begin{macrocode}
  \input etexcmds.sty\relax
%    \end{macrocode}
%    \begin{macro}{\@onelevel@sanitize}
%    Define \LaTeX's \cs{@onelevel@sanitize} if not already available.
%    \begin{macrocode}
  \expandafter\ifx\csname @onelevel@sanitize\endcsname\relax
    \def\@onelevel@sanitize#1{%
      \edef#1{%
        \expandafter\strip@prefix\meaning#1%
      }%
    }%
%    \end{macrocode}
%    \end{macro}
%    \begin{macro}{\strip@prefix}
%    \begin{macrocode}
    \def\strip@prefix#1>{}%
%    \end{macrocode}
%    \end{macro}
%    \begin{macrocode}
  \fi
\else
%    \end{macrocode}
%
% \subsubsection{\LaTeX}
%    \begin{macrocode}
  \DeclareOption*{\PassOptionsToPackage{\CurrentOption}{soul}}%
  \ProcessOptions\relax
  \RequirePackage{soul}[2003/11/17]%
  \RequirePackage{infwarerr}[2016/05/16]%
  \RequirePackage{etexcmds}[2016/05/16]%
\fi
%    \end{macrocode}
%
% \subsubsection{\eTeX}
%
%    In \plainTeX\ command \cs{+} is an \emph{outer} macro.
%    Therefore numbers are used to avoid problems.
%    \begin{macrocode}
\ifetex@unexpanded
  \catcode33=14 % '!': comment
  \catcode43=9  % '+': ignore
\else
  \catcode33=9  % '!': ignore
  \catcode43=14 % '+': comment
\fi
%    \end{macrocode}
%
% \subsection{Macro for redefinitions}
%
%    \begin{macro}{\SOuL@redefine}
%    \begin{macrocode}
\def\SOuL@redefine#1{%
  \begingroup
    \def\SOuL@cmd{#1}%
    \afterassignment\SOuL@cmdcheck
    \def\SOuL@temp
}
%    \end{macrocode}
%    \end{macro}
%    \begin{macro}{\SOuL@cmdcheck}
%    \begin{macrocode}
\def\SOuL@cmdcheck{%
    \expandafter\ifx\SOuL@cmd\SOuL@temp
    \else
      \edef\SOuL@temp*{\expandafter\string\SOuL@cmd}%
      \@PackageWarningNoLine{soulutf8}{%
        Command \SOuL@temp* has changed.\MessageBreak
        Supported versions of package `soul': 2003/11/17.\MessageBreak
        Depending on the unknown changes the redefinition\MessageBreak
        of \SOuL@temp* may not behave correctly%
      }%
    \fi
  \expandafter\endgroup
  \expandafter\def\SOuL@cmd
}
%    \end{macrocode}
%    \end{macro}
%
% \subsection{Redefinition of \cs{SOUL@eval}}
%
%    \begin{macro}{\SOUL@eval}
%    Macro \cs{SOUL@eval} is redefined to add detection of
%    the first byte of a UTF-8 sequence. Because \cs{SOUL@eval}
%    is overwritten, a warning is issued, if the contents of
%    \cs{SOUL@eval} is not as expected.
%    \begin{macrocode}
\SOuL@redefine\SOUL@eval{%
%    \end{macrocode}
%    First the expected definition.
%    \begin{macrocode}
  \def\SOUL@n*##1{\SOUL@scan}%
  \if\noexpand\SOUL@@\SOUL@spc
  \else
    \SOUL@ignorespacesfalse
  \fi
  \ifnum\SOUL@minus=\thr@@
    \SOUL@flushminus
  \else\ifnum\SOUL@comma=\tw@
    \SOUL@flushcomma
  \else\ifnum\SOUL@apo=\tw@
    \SOUL@flushapo
  \else\ifnum\SOUL@grave=\tw@
    \SOUL@flushgrave
  \fi\fi\fi\fi
  \ifx\SOUL@@-\else\SOUL@flushminus\fi
  \ifx\SOUL@@,\else\SOUL@flushcomma\fi
  \ifx\SOUL@@'\else\SOUL@flushapo\fi
  \ifx\SOUL@@`\else\SOUL@flushgrave\fi
  \ifx\SOUL@@-%
    \advance\SOUL@minus\@ne
  \else\ifx\SOUL@@,%
    \advance\SOUL@comma\@ne
  \else\ifx\SOUL@@'%
    \advance\SOUL@apo\@ne
  \else\ifx\SOUL@@`%
    \advance\SOUL@grave\@ne
  \else
    \SOUL@flushminus
    \SOUL@flushcomma
    \SOUL@flushapo
    \SOUL@flushgrave
    \ifx\SOUL@@\SOUL@stop
      \def\SOUL@n*{%
        \SOUL@doword
        \SOUL@eventuallyexhyphen\null
      }%
    \else\ifx\SOUL@@\par
      \def\SOUL@n*\par{\par\leavevmode\SOUL@scan}%
    \else\if\noexpand\SOUL@@\SOUL@spc
      \SOUL@doword
      \SOUL@eventuallyexhyphen\null
      \ifSOUL@ignorespaces
      \else
        \SOUL@everyspace{}%
      \fi
      \def\SOUL@n* {\SOUL@scan}%
    \else\ifx\SOUL@@\\%
      \SOUL@doword
      \SOUL@eventuallyexhyphen\null
      \SOUL@everyspace{\unskip\nobreak\hfil\break}%
      \SOUL@ignorespacestrue
    \else\ifx\SOUL@@~%
      \SOUL@doword
      \SOUL@eventuallyexhyphen\null
      \SOUL@everyspace{\nobreak}%
    \else\ifx\SOUL@@\slash
      \SOUL@doword
      \SOUL@eventuallyexhyphen{/}%
      \SOUL@exhyphen{/}%
    \else\ifx\SOUL@@\mbox
      \def\SOUL@n*{\SOUL@addprotect}%
    \else\ifx\SOUL@@\hbox
      \def\SOUL@n*{\SOUL@addprotect}%
    \else\ifx\SOUL@@\soulomit
      \def\SOUL@n*\soulomit##1{%
        \SOUL@doword
        {\spaceskip\SOUL@spaceskip##1}%
        \SOUL@scan
      }%
    \else\ifx\SOUL@@\break
      \SOUL@doword
      \break
    \else\ifx\SOUL@@\linebreak
      \SOUL@doword
      \SOUL@everyspace{\linebreak}%
    \else\ifcat\bgroup\noexpand\SOUL@@
      \def\SOUL@n*{\SOUL@addgroup{}}%
    \else\ifcat$\noexpand\SOUL@@
      \def\SOUL@n*{\SOUL@addmath}%
    \else
      \def\SOUL@n*{\SOUL@dotoken}%
    \fi\fi\fi\fi\fi\fi\fi\fi\fi\fi\fi\fi\fi
  \fi\fi\fi\fi
  \SOUL@n*%
}{%
%    \end{macrocode}
%    Now the redefined version follows.
%    \begin{macrocode}
  \def\SOUL@n*##1{\SOUL@scan}%
  \if\noexpand\SOUL@@\SOUL@spc
  \else
    \SOUL@ignorespacesfalse
  \fi
  \ifnum\SOUL@minus=\thr@@
    \SOUL@flushminus
  \else\ifnum\SOUL@comma=\tw@
    \SOUL@flushcomma
  \else\ifnum\SOUL@apo=\tw@
    \SOUL@flushapo
  \else\ifnum\SOUL@grave=\tw@
    \SOUL@flushgrave
  \fi\fi\fi\fi
  \ifx\SOUL@@-\else\SOUL@flushminus\fi
  \ifx\SOUL@@,\else\SOUL@flushcomma\fi
  \ifx\SOUL@@'\else\SOUL@flushapo\fi
  \ifx\SOUL@@`\else\SOUL@flushgrave\fi
  \ifx\SOUL@@-%
    \advance\SOUL@minus\@ne
  \else\ifx\SOUL@@,%
    \advance\SOUL@comma\@ne
  \else\ifx\SOUL@@'%
    \advance\SOUL@apo\@ne
  \else\ifx\SOUL@@`%
    \advance\SOUL@grave\@ne
  \else
    \SOUL@flushminus
    \SOUL@flushcomma
    \SOUL@flushapo
    \SOUL@flushgrave
    \ifx\SOUL@@\SOUL@stop
      \def\SOUL@n*{%
          \SOUL@doword
          \SOUL@eventuallyexhyphen\null
      }%
    \else\ifx\SOUL@@\par
      \def\SOUL@n*\par{\par\leavevmode\SOUL@scan}%
    \else\if\noexpand\SOUL@@\SOUL@spc
      \SOUL@doword
      \SOUL@eventuallyexhyphen\null
      \ifSOUL@ignorespaces
      \else
        \SOUL@everyspace{}%
      \fi
      \def\SOUL@n* {\SOUL@scan}%
    \else\ifx\SOUL@@\\%
      \SOUL@doword
      \SOUL@eventuallyexhyphen\null
      \SOUL@everyspace{\unskip\nobreak\hfil\break}%
      \SOUL@ignorespacestrue
    \else\ifx\SOUL@@~%
      \SOUL@doword
      \SOUL@eventuallyexhyphen\null
      \SOUL@everyspace{\nobreak}%
    \else\ifx\SOUL@@\slash
      \SOUL@doword
      \SOUL@eventuallyexhyphen{/}%
      \SOUL@exhyphen{/}%
    \else\ifx\SOUL@@\mbox
      \def\SOUL@n*{\SOUL@addprotect}%
    \else\ifx\SOUL@@\hbox
      \def\SOUL@n*{\SOUL@addprotect}%
    \else\ifx\SOUL@@\soulomit
      \def\SOUL@n*\soulomit##1{%
        \SOUL@doword
        {\spaceskip\SOUL@spaceskip##1}%
        \SOUL@scan
      }%
    \else\ifx\SOUL@@\break
      \SOUL@doword
      \break
    \else\ifx\SOUL@@\linebreak
      \SOUL@doword
      \SOUL@everyspace{\linebreak}%
    \else\ifcat\bgroup\noexpand\SOUL@@
      \def\SOUL@n*{\SOUL@addgroup{}}%
    \else\ifcat$\noexpand\SOUL@@
      \def\SOUL@n*{\SOUL@addmath}%
    \else
%    \end{macrocode}
%    The current token is examined to detect the start of
%    a UTF-8 sequence.
%    \begin{macrocode}
      \SOuL@analyzeutfviii
      \ifcase\SOuL@octets
        \SOuL@analyzeutfviiix
      \fi
      \ifcase\SOuL@octets
        \def\SOUL@n*{\SOUL@dotoken}%
      \or % 1
      \or % 2
        \def\SOUL@n*{\SOuL@addtwooctets}%
      \or % 3
        \def\SOUL@n*{\SOuL@addthreeoctets}%
      \or % 4
        \def\SOUL@n*{\SOuL@addfouroctets}%
      \fi
    \fi\fi\fi\fi\fi\fi\fi\fi\fi\fi\fi\fi\fi
  \fi\fi\fi\fi
  \SOUL@n*%
}
%    \end{macrocode}
%    \end{macro}
%
% \subsection{UTF-8 analysis}
%
% \subsubsection{Help strings}
%
%    \begin{macrocode}
\def\SOuL@defsanitizedstring#1#2{%
  \expandafter\def\csname SOuL@string#1\endcsname{#2}%
  \expandafter\@onelevel@sanitize\csname SOuL@string#1\endcsname
}
\SOuL@defsanitizedstring{UTFviii}{UTFviii@}
\SOuL@defsanitizedstring{octets}{@octets}
\SOuL@defsanitizedstring{two}{two}
\SOuL@defsanitizedstring{three}{three}
\SOuL@defsanitizedstring{four}{four}
\SOuL@defsanitizedstring{macrocolon}{macro:}
\SOuL@defsanitizedstring{csnameu}{csname u8-}
\SOuL@defsanitizedstring{undeferr}{utf@viii@undeferr}
\def\SOuL@stringendash{^^e2^^80^^93}
\def\SOuL@stringemdash{^^e2^^80^^94}
\def\SOuL@stringnobreakspace{^^c2^^a0}
\edef\SOuL@charhash{\string #}
\edef\SOuL@chartwo{\string 2}
\edef\SOuL@charthree{\string 3}
\def\SOuL@empty{}
%    \end{macrocode}
%
% \subsubsection{Support for \xfile{utf8.def}}
%
%    \begin{macro}{\SOuL@analyzeutfviii}
%    \begin{macrocode}
\begingroup
  \edef\x{\endgroup
    \def\noexpand\SOuL@analyzeutfviii{%
      \noexpand\expandafter\noexpand\SOuL@checkutfviii
      \noexpand\meaning\noexpand\SOUL@@
      \SOuL@stringUTFviii\SOuL@stringoctets
      \noexpand\@nil
    }%
    \def\noexpand\SOuL@checkutfviii
      ##1\SOuL@stringUTFviii##2\SOuL@stringoctets##3\noexpand\@nil
  }%
\x{%
  \def\SOuL@temp{#2}%
  \chardef\SOuL@octets=%
      \ifx\SOuL@temp\SOuL@stringtwo
        \tw@
      \else\ifx\SOuL@temp\SOuL@stringthree
        \thr@@
      \else\ifx\SOuL@temp\SOuL@stringfour
        4 %
      \else
        \z@
      \fi\fi\fi
}
%    \end{macrocode}
%    \end{macro}
%
% \subsubsection{Support for \xfile{utf8x.def}}
%
%    \begin{macro}{\SOuL@analyzeutfviiix}
%    \begin{macrocode}
\begingroup
  \edef\x{\endgroup
    \def\noexpand\SOuL@analyzeutfviiix{%
      \noexpand\expandafter\noexpand\SOuL@checkutfviiix
      \noexpand\meaning\noexpand\SOUL@@
      \SOuL@stringmacrocolon\SOuL@charhash1{}{}{}{}%
      \SOuL@stringcsnameu\SOuL@stringundeferr
      \noexpand\@nil
    }%
%    \end{macrocode}
%    \end{macro}
%    \begin{macro}{\SOuL@checkutfviiix}
%    \begin{macrocode}
    \def\noexpand\SOuL@checkutfviiix
      ##1\SOuL@stringmacrocolon\SOuL@charhash1##2##3##4##5##6%
      \SOuL@stringcsnameu##7\SOuL@stringundeferr##8\noexpand\@nil
  }%
\x{%
  \def\SOuL@temp{#7}%
  \ifx\SOuL@temp\SOuL@empty
    \chardef\SOuL@octets=\z@
  \else
    \def\SOuL@temp{#5}%
    \ifx\SOuL@temp\SOuL@charthree
      \chardef\SOuL@octets=4 %
    \else
      \def\SOuL@temp{#3}%
      \ifx\SOuL@temp\SOuL@chartwo
        \chardef\SOuL@octets=\thr@@
      \else
        \chardef\SOuL@octets=\tw@
      \fi
    \fi
  \fi
}
%    \end{macrocode}
%    \end{macro}
%
% \subsection{Actions for UTF-8 sequences}
%
%    \begin{macro}{\SOuL@addtwooctets}
%    \begin{macrocode}
\def\SOuL@addtwooctets#1#2{%
  \def\SOuL@temp{#1#2}%
  \@onelevel@sanitize\SOuL@temp
  \ifx\SOuL@temp\SOuL@stringnobreakspace
    \SOUL@doword
    \SOUL@eventuallyexhyphen\null
    \SOUL@everyspace{\nobreak}%
    \let\SOuL@next\SOUL@scan
  \else
    \def\SOuL@next{%
!     \SOUL@addtoken{{\noexpand#1\noexpand#2}}%
+     \SOUL@addtoken{{\etex@unexpanded{#1#2}}}%
    }%
  \fi
  \SOuL@next
}
%    \end{macrocode}
%    \end{macro}
%    \begin{macro}{\SOuL@addthreeoctets}
%    \begin{macrocode}
\def\SOuL@addthreeoctets#1#2#3{%
  \def\SOuL@temp{#1#2#3}%
  \@onelevel@sanitize\SOuL@temp
  \ifx\SOuL@temp\SOuL@stringendash
    \SOUL@doword
    \SOUL@eventuallyexhyphen{-}%
    \SOUL@exhyphen{--}%
    \let\SOuL@next\SOUL@scan
  \else
    \ifx\SOuL@temp\SOuL@stringemdash
      \SOUL@doword
      \SOUL@eventuallyexhyphen{-}%
      \SOUL@exhyphen{---}%
      \let\SOuL@next\SOUL@scan
    \else
      \def\SOuL@next{%
!       \SOUL@addtoken{{\noexpand#1\noexpand#2\noexpand#3}}%
+       \SOUL@addtoken{{\etex@unexpanded{#1#2#3}}}%
      }%
    \fi
  \fi
  \SOuL@next
}
%    \end{macrocode}
%    \end{macro}
%    \begin{macro}{\SOuL@addfouroctets}
%    \begin{macrocode}
\def\SOuL@addfouroctets#1#2#3#4{%
! \SOUL@addtoken{{\noexpand#1\noexpand#2\noexpand#3\noexpand#4}}%
+ \SOUL@addtoken{{\etex@unexpanded{#1#2#3#4}}}%
}
%    \end{macrocode}
%    \end{macro}
%
% \subsubsection{Redefinition of \cs{SOUL@splittoken}}
%
%    \begin{macro}{\SOUL@splittoken}
%    Macro \cs{SOUL@splittoken} separates the first token
%    or token group from a word and redefines the word to contain
%    the remaining tokens. However if the remaining tokens are
%    a token group, then the curly braces will be removed and
%    the token group is splitted by the next call of \cs{SOUL@splittoken}.
%    The redefinition avoids the removal of curly braces around the
%    remaining tokens.
%    \begin{macrocode}
\SOuL@redefine\SOUL@splittoken#1#2\SOUL@stop{%
  \global\SOUL@token={#1}%
  \global\SOUL@word={#2}%
}#1{%
  \global\SOUL@token={#1}%
  \SOuL@remainingtoken\relax
}
%    \end{macrocode}
%    \end{macro}
%    \begin{macro}{\SOuL@remainingtoken}
%    \begin{macrocode}
\def\SOuL@remainingtoken#1\SOUL@stop{%
  \global\SOUL@word=\expandafter{\@gobble#1}%
}
%    \end{macrocode}
%    \end{macro}
%
% \subsection{Patches}
%
%    The fixed \cs{SOUL@splittoken} allows to remove the double
%    sets of curly braces in other macros of package \xpackage{soul}.
%    The benefit is that implicite kernings are more often detected
%    and fixes a bug in package \xpackage{soul}. The disadvantage is
%    incompatibility. The width of the resulting strings may change.
%    \begin{macro}{\SOUL@flushcomma}
%    \begin{macrocode}
\SOuL@redefine\SOUL@flushcomma{%
  \ifcase\SOUL@comma
  \or
    \edef\x{\SOUL@word={\the\SOUL@word,}}\x
  \or
    \edef\x{\SOUL@word={\the\SOUL@word{{,,}}}}\x
  \fi
  \SOUL@comma\z@
}{%
  \ifcase\SOUL@comma
  \or
    \edef\x{\SOUL@word={\the\SOUL@word,}}\x
  \or
    \edef\x{\SOUL@word={\the\SOUL@word{,,}}}\x
  \fi
  \SOUL@comma\z@
}
%    \end{macrocode}
%    \end{macro}
%    \begin{macro}{\SOUL@flushapo}
%    \begin{macrocode}
\SOuL@redefine\SOUL@flushapo{%
  \ifcase\SOUL@apo
  \or
    \edef\x{\SOUL@word={\the\SOUL@word'}}\x
  \or
    \edef\x{\SOUL@word={\the\SOUL@word{{''}}}}\x
  \fi
  \SOUL@apo\z@
}{%
  \ifcase\SOUL@apo
  \or
    \edef\x{\SOUL@word={\the\SOUL@word'}}\x
  \or
    \edef\x{\SOUL@word={\the\SOUL@word{''}}}\x
  \fi
  \SOUL@apo\z@
}
%    \end{macrocode}
%    \end{macro}
%    \begin{macro}{\SOUL@flushgrave}
%    \begin{macrocode}
\SOuL@redefine\SOUL@flushgrave{%
  \ifcase\SOUL@grave
  \or
    \edef\x{\SOUL@word={\the\SOUL@word`}}\x
  \or
    \edef\x{\SOUL@word={\the\SOUL@word{{``}}}}\x
  \fi
  \SOUL@grave\z@
}{%
  \ifcase\SOUL@grave
  \or
    \edef\x{\SOUL@word={\the\SOUL@word`}}\x
  \or
    \edef\x{\SOUL@word={\the\SOUL@word{``}}}\x
  \fi
  \SOUL@grave\z@
}
%    \end{macrocode}
%    \end{macro}
%    \begin{macro}{\SOUL@addgroup}
%    \begin{macrocode}
\SOuL@redefine\SOUL@addgroup#1#2{%
  {%
    \let\protect\noexpand
    \edef\x{%
      \global\SOUL@word={%
        \the\SOUL@word
        {{\noexpand#1#2}}%
      }%
    }%
    \x
  }%
  \SOUL@scan
}#1#2{%
  \begingroup
    \let\protect\noexpand
    \edef\x{\endgroup
      \SOUL@word={%
        \the\SOUL@word
!       {\noexpand#1{#2}}%
+       {\etex@unexpanded{#1{#2}}}%
      }%
    }%
  \x
  \SOUL@scan
}
%    \end{macrocode}
%    \end{macro}
%    \begin{macro}{\SOUL@addmath}
%    \begin{macrocode}
\SOuL@redefine\SOUL@addmath$#1${%
  {%
    \let\protect\noexpand
    \edef\x{%
      \global\SOUL@word={%
        \the\SOUL@word
        {{\hbox{$#1$}}}%
      }%
    }%
    \x
  }%
  \SOUL@scan
}$#1${%
  \begingroup
    \let\protect\noexpand
    \edef\x{\endgroup
      \SOUL@word={%
        \the\SOUL@word
!       {\hbox{$#1$}}%
+       {\etex@unexpanded{\hbox{$#1$}}}%
      }%
    }%
  \x
  \SOUL@scan
}
%    \end{macrocode}
%    \end{macro}
%    \begin{macro}{\SOUL@addprotect}
%    \begin{macrocode}
\SOuL@redefine\SOUL@addprotect#1#2{%
  {%
    \let\protect\noexpand
    \edef\x{%
      \global\SOUL@word={%
        \the\SOUL@word
        {{\hbox{#2}}}%
      }%
    }%
    \x
  }%
  \SOUL@scan
}#1#2{%
  \begingroup
    \let\protect\noexpand
    \edef\x{\endgroup
      \SOUL@word={%
        \the\SOUL@word
!       {\hbox{#2}}%
+       {\etex@unexpanded{\hbox{#2}}}%
      }%
    }%
  \x
  \SOUL@scan
}
%    \end{macrocode}
%    \end{macro}
%
%    \begin{macro}{\SOUL@addtoken}
%    \begin{macrocode}
+ \SOuL@redefine\SOUL@addtoken#1{%
+   \edef\x{%
+     \SOUL@word={%
+       \the\SOUL@word
+       \noexpand#1%
+     }%
+   }%
+   \x
+   \SOUL@scan
+ }#1{%
+   \edef\x{%
+     \SOUL@word={%
+       \the\SOUL@word
+       \etex@unexpanded{#1}%
+     }%
+   }%
+   \x
+   \SOUL@scan
+ }%
%    \end{macrocode}
%    \end{macro}
%
%    \begin{macrocode}
\SOuL@AtEnd%
%    \end{macrocode}
%
%    \begin{macrocode}
%</package>
%    \end{macrocode}
%
% \section{Test}
%
% \subsection{Catcode checks for loading}
%
%    \begin{macrocode}
%<*test1>
%    \end{macrocode}
%    \begin{macrocode}
\catcode`\{=1 %
\catcode`\}=2 %
\catcode`\#=6 %
\catcode`\@=11 %
\expandafter\ifx\csname count@\endcsname\relax
  \countdef\count@=255 %
\fi
\expandafter\ifx\csname @gobble\endcsname\relax
  \long\def\@gobble#1{}%
\fi
\expandafter\ifx\csname @firstofone\endcsname\relax
  \long\def\@firstofone#1{#1}%
\fi
\expandafter\ifx\csname loop\endcsname\relax
  \expandafter\@firstofone
\else
  \expandafter\@gobble
\fi
{%
  \def\loop#1\repeat{%
    \def\body{#1}%
    \iterate
  }%
  \def\iterate{%
    \body
      \let\next\iterate
    \else
      \let\next\relax
    \fi
    \next
  }%
  \let\repeat=\fi
}%
\def\RestoreCatcodes{}
\count@=0 %
\loop
  \edef\RestoreCatcodes{%
    \RestoreCatcodes
    \catcode\the\count@=\the\catcode\count@\relax
  }%
\ifnum\count@<255 %
  \advance\count@ 1 %
\repeat

\def\RangeCatcodeInvalid#1#2{%
  \count@=#1\relax
  \loop
    \catcode\count@=15 %
  \ifnum\count@<#2\relax
    \advance\count@ 1 %
  \repeat
}
\def\RangeCatcodeCheck#1#2#3{%
  \count@=#1\relax
  \loop
    \ifnum#3=\catcode\count@
    \else
      \errmessage{%
        Character \the\count@\space
        with wrong catcode \the\catcode\count@\space
        instead of \number#3%
      }%
    \fi
  \ifnum\count@<#2\relax
    \advance\count@ 1 %
  \repeat
}
\def\space{ }
\expandafter\ifx\csname LoadCommand\endcsname\relax
  \def\LoadCommand{\input soulutf8.sty\relax}%
\fi
\def\Test{%
  \RangeCatcodeInvalid{0}{47}%
  \RangeCatcodeInvalid{58}{64}%
  \RangeCatcodeInvalid{91}{96}%
  \RangeCatcodeInvalid{123}{255}%
  \catcode`\@=12 %
  \catcode`\\=0 %
  \catcode`\%=14 %
  \LoadCommand
  \RangeCatcodeCheck{0}{36}{15}%
  \RangeCatcodeCheck{37}{37}{14}%
  \RangeCatcodeCheck{38}{47}{15}%
  \RangeCatcodeCheck{48}{57}{12}%
  \RangeCatcodeCheck{58}{63}{15}%
  \RangeCatcodeCheck{64}{64}{12}%
  \RangeCatcodeCheck{65}{90}{11}%
  \RangeCatcodeCheck{91}{91}{15}%
  \RangeCatcodeCheck{92}{92}{0}%
  \RangeCatcodeCheck{93}{96}{15}%
  \RangeCatcodeCheck{97}{122}{11}%
  \RangeCatcodeCheck{123}{255}{15}%
  \RestoreCatcodes
}
\Test
\csname @@end\endcsname
\end
%    \end{macrocode}
%    \begin{macrocode}
%</test1>
%    \end{macrocode}
%
% \subsection{\LaTeX\space tests}
%
%    \begin{macrocode}
%<*test2|test3|test4|test5>
%    \end{macrocode}
%    \begin{macrocode}
%<noetex>\let\SavedUnexpanded\unexpanded
%<noetex>\let\unexpanded\UNDEFINED
\NeedsTeXFormat{LaTeX2e}
\documentclass{article}
\usepackage[T1]{fontenc}
\usepackage{lmodern}
\usepackage{textcomp}
\usepackage[ngerman]{babel}
\expandafter\def\csname ver@hypdestopt.sty\endcsname{}
\usepackage[unicode]{hyperref}
\usepackage{bookmark}
\usepackage{color}
\usepackage{soul}[2003/11/17]
\makeatletter
\newcommand*{\@ifDefined}[1]{%
  \@ifundefined{#1}{%
    \@latex@error{%
      \expandafter\noexpand\csname#1\endcsname undefined%
    }\@ehc
  }%
}
\newcommand*{\SaveOrg}[1]{%
  \@ifdefinable{ORG@SOUL@#1}{%
    \@ifDefined{SOUL@#1}{%
      \expandafter\let\csname ORG@SOUL@#1\expandafter\endcsname
                      \csname SOUL@#1\endcsname
    }%
  }%
}
\newcommand*{\RestoreOrg}[1]{%
  \@ifDefined{ORG@SOUL@#1}{%
    \@ifDefined{SOUL@#1}{%
      \expandafter\let\csname SOUL@#1\expandafter\endcsname
                      \csname ORG@SOUL@#1\endcsname
    }%
  }%
}
\makeatother
\SaveOrg{eval}
\SaveOrg{splittoken}
\SaveOrg{flushcomma}
\SaveOrg{flushapo}
\SaveOrg{grave}
\SaveOrg{addgroup}
\SaveOrg{addmath}
\SaveOrg{addprotect}
\SaveOrg{addtoken}
\usepackage{soulutf8}[2016/05/16]
%    \end{macrocode}
%    \begin{macrocode}
%<noetex>\let\unexpanded\SavedUnexpanded
%    \end{macrocode}
%    Input encoding \xfile{utf8}.
%    \begin{macrocode}
%<*utf8>
\usepackage[utf8]{inputenc}
\DeclareUnicodeCharacter{00A0}{\nobreakspace}
%</utf8>
%    \end{macrocode}
%    Input encoding \xfile{utf8x}. Package \xpackage{ucs}
%    has a nasty bug, that it redefines \cs{show}.
%    \begin{macrocode}
%<*utf8x>
\let\origshow\show
\usepackage[utf8x]{inputenc}
\AtBeginDocument{%
  \let\show\origshow
}
%</utf8x>
%    \end{macrocode}
%
% \subsubsection{Driver definitions}
%
%    \begin{macrocode}
\makeatletter
%    \end{macrocode}
%    \begin{macro}{\an}
%    The definition for \cs{an} is taken from the documentation of
%    package \xpackage{soul}.
%    \begin{macrocode}
\DeclareRobustCommand*{\an}{%
  \def\SOUL@preamble{$^{^P}$}%
  \def\SOUL@everyspace##1{##1\texttt{\char`\ }\linebreak[2]}%
  \def\SOUL@postamble{$^{^E}$}%
  \def\SOUL@everyhyphen{$^{^-}$\linebreak[1]}%
  \def\SOUL@everyexhyphen##1{##1$^{^=}$}%
  \def\SOUL@everysyllable{$^{^S}$}%
  \def\SOUL@everytoken{\the\SOUL@token$^{^T}$}%
  \def\SOUL@everylowerthan{$^{^L}$}%
  \SOUL@
}
%    \end{macrocode}
%    \end{macro}
%
%    \begin{macro}{\sy}
%    The definition for \cs{sy} is taken from the documentation of
%    package \xpackage{soul}.
%    \begin{macrocode}
\DeclareRobustCommand*{\sy}{%
  \SOUL@setup
  \def\SOUL@preamble{%
    \lefthyphenmin=\z@
    \righthyphenmin=\z@
  }%
  \def\SOUL@everyspace##1{##1\space}%
  \def\SOUL@everysyllable{\the\SOUL@syllable}%
  \def\SOUL@everyhyphen{%
    \discretionary{%
      \SOUL@setkern\SOUL@hyphkern
      \SOUL@sethyphenchar
    }{}{%
      \hbox{\kern1pt$\cdot$}%
    }%
  }%
  \def\SOUL@everyexhyphen##1{%
    \SOUL@setkern\SOUL@hyphkern
    \hbox{##1}%
    \discretionary{}{}{%
      \SOUL@setkern\SOUL@charkern
    }%
  }%
  \SOUL@
}
%    \end{macrocode}
%    \end{macro}
%
%    \begin{macro}{\soulH}
%    \begin{macrocode}
\DeclareRobustCommand*{\soulH}{%
  \def\SOUL@preamble{%
    \gdef\SOULH{}%
    \soulHadd{[}%
  }%
  \def\SOUL@everyspace##1{%
    \soulHadd{Sp}%
  }%
  \def\SOUL@postamble{%
    \soulHadd{]}%
  }%
  \def\SOUL@everyhyphen{%
    \soulHadd{Hy}%
  }%
  \def\SOUL@everyexhyphen##1{%
    \soulHadd{Exhy}%
  }%
  \def\SOUL@everysyllable{%
    \soulHadd{Sy}%
  }%
  \def\SOUL@everytoken{%
    \soulHadd{T}%
  }%
  \def\SOUL@everylowerthan{%
    \soulHadd{<}%
  }%
  \SOUL@
}
\newcommand*{\soulHadd}[1]{%
  #1%
  \expandafter\gdef\expandafter\soulHstring\expandafter{%
    \soulHstring
    #1%
  }%
}
%    \end{macrocode}
%    \end{macro}
%
% \subsubsection{Test macros}
%
%    \begin{macrocode}
\usepackage{qstest}
\IncludeTests{*}
\LogTests{log}{*}{*}
%    \end{macrocode}
%    \begin{macrocode}
\newsavebox\boxA
\newsavebox\boxB
\newsavebox\boxC
%    \end{macrocode}
%    \begin{macro}{\TestDriver}
%    \begin{macrocode}
\newcommand*{\TestDriver}[4]{%
  \subsection{Driver #1}%
  \@PackageInfo{soulutf8-test}{* Driver #1}%
  \expandafter\let\expandafter\testdriver\csname #1\endcsname
  \def\teststringA{#3}%
  \def\teststringB{#4}%
  \begin{qstest}{string-#2-driver-#1}{string,#2,driver,#1}%
    \@PackageInfo{soulutf8-test}{* Version: soulutf8}%
    \global\let\soulHstring\@empty
    \setbox0=\hbox{\expandafter\testdriver\expandafter{\teststringA}}%
    \let\HstringA\soulHstring
    \SetTestVBox{\boxA}{\teststringA}%
    %
    \@PackageInfo{soulutf8-test}{* Version: soul + patch}%
    \global\let\soulHstring\@empty
    \RestoreOrg{eval}%
    \setbox2=\hbox{\expandafter\testdriver\expandafter{\teststringB}}%
    \let\HstringB\soulHstring
    \SetTestVBox{\boxB}{\teststringB}%
    %
    \@PackageInfo{soulutf8-test}{* Version: original soul}%
    \global\let\soulHstring\@empty
    \RestoreOrg{splittoken}%
    \RestoreOrg{flushcomma}%
    \RestoreOrg{flushapo}%
    \RestoreOrg{grave}%
    \RestoreOrg{addgroup}%
    \RestoreOrg{addmath}%
    \RestoreOrg{addprotect}%
    \RestoreOrg{addtoken}%
    \setbox4=\hbox{\expandafter\testdriver\expandafter{\teststringB}}%
    \let\HstringC\soulHstring
    \SetTestVBox{\boxC}{\teststringB}%
    %
    \TestInfo{0}{soulutf8}%
    \TestInfo{2}{soul + patch}%
    \TestInfo{4}{original soul}%
    \TestInfoVBox
    \showboxdepth=\maxdimen
    \showboxbreadth=\maxdimen
    \tracingonline=\@ne
    \Expect*{\HstringA}*{\HstringB}%
    \Expect*{\HstringA}*{\HstringC}%
    \Expect*{\the\wd0}*{\the\wd2}%
    \ifdim\wd0=\wd2 %
    \else
      \showbox0 %
      \showbox2 %
    \fi
    \Expect*{\the\ht0}*{\the\ht2}%
    \Expect*{\the\dp0}*{\the\dp2}%
    \Expect*{\the\wd\boxB}*{\the\wd\boxA}%
    \Expect*{\the\wd\boxC}*{\the\wd\boxA}%
    \Expect*{\the\ht\boxB}*{\the\ht\boxA}%
    \ifdim\ht\boxB=\ht\boxA
    \else
      \showbox\boxA
      \showbox\boxB
    \fi
    \Expect*{\the\ht\boxC}*{\the\ht\boxA}%
    \Expect*{\the\dp\boxB}*{\the\dp\boxA}%
    \Expect*{\the\dp\boxC}*{\the\dp\boxA}%
  \end{qstest}%
}
%    \end{macrocode}
%    \end{macro}
%    \begin{macro}{\SetTestVBox}
%    \begin{macrocode}
\newcommand*{\SetTestVBox}[2]{%
  \setbox#1=\vbox{%
    \hsize=0pt\relax
    \hfuzz=100mm\relax
    \expandafter\testdriver\expandafter{#2}%
  }%
}
%    \end{macrocode}
%    \end{macro}
%    \begin{macro}{\TestInfo}
%    \begin{macrocode}
\newcommand*{\TestInfo}[2]{%
  \par
  \noindent
  \copy#1\relax
  \hfill
  \begingroup
    \small
    \ifdim\wd0=\wd2\relax
      \ifdim\wd0=\wd4\relax
      \else
        \ifnum#1=2 %
          \raisebox{.5\baselineskip}[0pt][0pt]{%
            \llap{$\}$ }%
            \rlap{width = \the\wd#1\relax\space$\{$}%
            \phantom{width = 999.99999pt}%
            \qquad
          }%
        \else
          \ifnum#1=4 %
            \rlap{width = \the\wd#1}%
            \phantom{width = 999.99999pt}%
            \qquad
          \fi
        \fi
      \fi
    \else
      \begingroup
        \bfseries
        \rlap{width = \the\wd#1}%
        \phantom{width = 999.99999pt}%
      \endgroup
      \qquad
    \fi
    \rlap{(#2)}%
    \phantom{(original soul)}%
  \endgroup
  \par
}
%    \end{macrocode}
%    \end{macro}
%    \begin{macro}{\TestInfoVBox}
%    \begin{macrocode}
\newcommand*{\TestInfoVBox}{%
  \par
  \noindent
  \mbox{\vrule\relax\space\copy\boxA}%
  \ifdim\ht\boxA=\ht\boxB
  \else
    \hfill
    \begingroup
      \small
      \rlap{(soulutf8)}%
      \phantom{(soulutf8)}%
    \endgroup
    \par
    \noindent
    \mbox{\vrule\relax\space\copy\boxB}%
    \hfill
    \begingroup
      \small
      \rlap{(soul)}%
      \phantom{(soulutf8)}%
    \endgroup
  \fi
  \par
}
%    \end{macrocode}
%    \end{macro}
%    \begin{macro}{\Test}
%    \begin{macrocode}
\makeatletter
\newcommand*{\Test}[3]{%
  \clearpage
  \begingroup
    \def\linebreak{\textbackslash linebreak }%
    \def\\{\textbackslash\textbackslash}%
    \def\soulomit##1{%
      \textbackslash soulomit\textbraceleft##1\textbraceright
    }%
    \def\mbox##1{%
      \textbackslash mbox\textbraceleft##1\textbraceright
    }%
    \def\abc{\textbackslash abc }%
    \def\texttt##1{%
      \textbackslash texttt\textbraceleft##1\textbraceright
    }%
    \def\c##1{%
      \textbackslash c\textbraceleft##1\textbraceright
    }%
%<*utf8x>
    \count@=128 %
    \loop
    \ifnum\count@<256 %
      \lccode`\~=\count@
      \lowercase{%
        \def~{\noexpand~}%
      }%
      \advance\count@\@ne
    \repeat
%</utf8x>
    \csname protected@edef\endcsname\x{\endgroup
      \noexpand\section{String #1: #2}%
    }%
  \x
  \TestDriver{so}{#1}{#2}{#3}%
  \TestDriver{caps}{#1}{#2}{#3}%
  \TestDriver{ul}{#1}{#2}{#3}%
  \TestDriver{st}{#1}{#2}{#3}%
  \TestDriver{hl}{#1}{#2}{#3}%
  \TestDriver{an}{#1}{#2}{#3}%
  \TestDriver{sy}{#1}{#2}{#3}%
  \TestDriver{soulH}{#1}{#2}{#3}%
}
\makeatother
%    \end{macrocode}
%    \end{macro}
%
% \subsubsection{Testing}
%
%    \begin{macrocode}
\begin{document}
  \Test A{}{}%
  \Test B{Hello World}{Hello World}%
  \Test C{%
    Umlauts: ^^c3^^84^^c3^^96^^c3^^9c^^c3^^a4^^c3^^b6^^c3^^bc^^c3^^9f%
  }{%
    Umlauts: \"A\"O\"U\"a\"o\"u\ss
  }%
  \Test D{Euro: ^^e2^^82^^ac}{Euro: \texteuro}%
  \Test E{Stra^^c3^^9fe}{Stra\ss e}%
  \Test F{R^^c3^^b6mer}{R\"omer}%
  \Test G{A^^e2^^80^^93B^^e2^^80^^94C}{A--B---C}%
  \Test H{X^^c2^^a0Y}{X~Y}%
  \Test I{A^^c5^^a4}{A\v T}%
  \Test J{A,B,,C,,,D,,,,E}{A,B,,C,,,D,,,,E}%
  \Test K{A'B''C'''D''''E}{A'B''C'''D''''E}%
  \Test L{A`B``C```D````E}{A`B``C```D````E}%
  \Test M{A$x$B}{A$x$B}%
  \Test N{A\mbox{abc}B}{A\mbox{abc}B}%
  \Test O{,,TA''}{,,TA''}%
  \Test P{{!`}Hola{?`}}{{!`}Hola{?`}}%
  \Test Q{re-sent\slash x}{re-sent\slash x}%
  \Test R{new\\line}{new\\line}%
  \Test S{foo\linebreak bar}{foo\linebreak bar}%
  \Test T{Virchow{sche}}{Virchow{sche}}%
  \Test U{A\soulomit{foo}B}{A\soulomit{foo}B}%
  \Test V{A\texttt{xyz}B}{A\texttt{xyz}B}%
  \Test W{ffi}{ffi}%
  \Test X{f{}f{}i}{f{}f{}i}%
  \Test Y{f\mbox{}f\mbox{}i}{f\mbox{}f\mbox{}i}%
  \Test Z{\c{\"a}}{{\c{\"a}}}%
%<*test2|test3>
  \let\abcx\UNDEFINED
  \def\abc{%
    \begingroup
      \def\abcx{x}%
      \abcx
    \endgroup
  }%
  \Test{AA}{[{x\abc x}]}{[{xxx}]}%
%</test2|test3>
%    \end{macrocode}
%
%    \begin{macrocode}
\end{document}
%    \end{macrocode}
%    \begin{macrocode}
%</test2|test3|test4|test5>
%    \end{macrocode}
%
% \section{Installation}
%
% \subsection{Download}
%
% \paragraph{Package.} This package is available on
% CTAN\footnote{\url{http://ctan.org/pkg/soulutf8}}:
% \begin{description}
% \item[\CTAN{macros/latex/contrib/oberdiek/soulutf8.dtx}] The source file.
% \item[\CTAN{macros/latex/contrib/oberdiek/soulutf8.pdf}] Documentation.
% \end{description}
%
%
% \paragraph{Bundle.} All the packages of the bundle `oberdiek'
% are also available in a TDS compliant ZIP archive. There
% the packages are already unpacked and the documentation files
% are generated. The files and directories obey the TDS standard.
% \begin{description}
% \item[\CTAN{install/macros/latex/contrib/oberdiek.tds.zip}]
% \end{description}
% \emph{TDS} refers to the standard ``A Directory Structure
% for \TeX\ Files'' (\CTAN{tds/tds.pdf}). Directories
% with \xfile{texmf} in their name are usually organized this way.
%
% \subsection{Bundle installation}
%
% \paragraph{Unpacking.} Unpack the \xfile{oberdiek.tds.zip} in the
% TDS tree (also known as \xfile{texmf} tree) of your choice.
% Example (linux):
% \begin{quote}
%   |unzip oberdiek.tds.zip -d ~/texmf|
% \end{quote}
%
% \paragraph{Script installation.}
% Check the directory \xfile{TDS:scripts/oberdiek/} for
% scripts that need further installation steps.
% Package \xpackage{attachfile2} comes with the Perl script
% \xfile{pdfatfi.pl} that should be installed in such a way
% that it can be called as \texttt{pdfatfi}.
% Example (linux):
% \begin{quote}
%   |chmod +x scripts/oberdiek/pdfatfi.pl|\\
%   |cp scripts/oberdiek/pdfatfi.pl /usr/local/bin/|
% \end{quote}
%
% \subsection{Package installation}
%
% \paragraph{Unpacking.} The \xfile{.dtx} file is a self-extracting
% \docstrip\ archive. The files are extracted by running the
% \xfile{.dtx} through \plainTeX:
% \begin{quote}
%   \verb|tex soulutf8.dtx|
% \end{quote}
%
% \paragraph{TDS.} Now the different files must be moved into
% the different directories in your installation TDS tree
% (also known as \xfile{texmf} tree):
% \begin{quote}
% \def\t{^^A
% \begin{tabular}{@{}>{\ttfamily}l@{ $\rightarrow$ }>{\ttfamily}l@{}}
%   soulutf8.sty & tex/generic/oberdiek/soulutf8.sty\\
%   soulutf8.pdf & doc/latex/oberdiek/soulutf8.pdf\\
%   test/soulutf8-test1.tex & doc/latex/oberdiek/test/soulutf8-test1.tex\\
%   test/soulutf8-test2.tex & doc/latex/oberdiek/test/soulutf8-test2.tex\\
%   test/soulutf8-test3.tex & doc/latex/oberdiek/test/soulutf8-test3.tex\\
%   test/soulutf8-test4.tex & doc/latex/oberdiek/test/soulutf8-test4.tex\\
%   test/soulutf8-test5.tex & doc/latex/oberdiek/test/soulutf8-test5.tex\\
%   soulutf8.dtx & source/latex/oberdiek/soulutf8.dtx\\
% \end{tabular}^^A
% }^^A
% \sbox0{\t}^^A
% \ifdim\wd0>\linewidth
%   \begingroup
%     \advance\linewidth by\leftmargin
%     \advance\linewidth by\rightmargin
%   \edef\x{\endgroup
%     \def\noexpand\lw{\the\linewidth}^^A
%   }\x
%   \def\lwbox{^^A
%     \leavevmode
%     \hbox to \linewidth{^^A
%       \kern-\leftmargin\relax
%       \hss
%       \usebox0
%       \hss
%       \kern-\rightmargin\relax
%     }^^A
%   }^^A
%   \ifdim\wd0>\lw
%     \sbox0{\small\t}^^A
%     \ifdim\wd0>\linewidth
%       \ifdim\wd0>\lw
%         \sbox0{\footnotesize\t}^^A
%         \ifdim\wd0>\linewidth
%           \ifdim\wd0>\lw
%             \sbox0{\scriptsize\t}^^A
%             \ifdim\wd0>\linewidth
%               \ifdim\wd0>\lw
%                 \sbox0{\tiny\t}^^A
%                 \ifdim\wd0>\linewidth
%                   \lwbox
%                 \else
%                   \usebox0
%                 \fi
%               \else
%                 \lwbox
%               \fi
%             \else
%               \usebox0
%             \fi
%           \else
%             \lwbox
%           \fi
%         \else
%           \usebox0
%         \fi
%       \else
%         \lwbox
%       \fi
%     \else
%       \usebox0
%     \fi
%   \else
%     \lwbox
%   \fi
% \else
%   \usebox0
% \fi
% \end{quote}
% If you have a \xfile{docstrip.cfg} that configures and enables \docstrip's
% TDS installing feature, then some files can already be in the right
% place, see the documentation of \docstrip.
%
% \subsection{Refresh file name databases}
%
% If your \TeX~distribution
% (\teTeX, \mikTeX, \dots) relies on file name databases, you must refresh
% these. For example, \teTeX\ users run \verb|texhash| or
% \verb|mktexlsr|.
%
% \subsection{Some details for the interested}
%
% \paragraph{Attached source.}
%
% The PDF documentation on CTAN also includes the
% \xfile{.dtx} source file. It can be extracted by
% AcrobatReader 6 or higher. Another option is \textsf{pdftk},
% e.g. unpack the file into the current directory:
% \begin{quote}
%   \verb|pdftk soulutf8.pdf unpack_files output .|
% \end{quote}
%
% \paragraph{Unpacking with \LaTeX.}
% The \xfile{.dtx} chooses its action depending on the format:
% \begin{description}
% \item[\plainTeX:] Run \docstrip\ and extract the files.
% \item[\LaTeX:] Generate the documentation.
% \end{description}
% If you insist on using \LaTeX\ for \docstrip\ (really,
% \docstrip\ does not need \LaTeX), then inform the autodetect routine
% about your intention:
% \begin{quote}
%   \verb|latex \let\install=y% \iffalse meta-comment
%
% File: soulutf8.dtx
% Version: 2016/05/16 v1.1
% Info: Permit use of UTF-8 characters in soul
%
% Copyright (C) 2007 by
%    Heiko Oberdiek <heiko.oberdiek at googlemail.com>
%    2016
%    https://github.com/ho-tex/oberdiek/issues
%
% This work may be distributed and/or modified under the
% conditions of the LaTeX Project Public License, either
% version 1.3c of this license or (at your option) any later
% version. This version of this license is in
%    http://www.latex-project.org/lppl/lppl-1-3c.txt
% and the latest version of this license is in
%    http://www.latex-project.org/lppl.txt
% and version 1.3 or later is part of all distributions of
% LaTeX version 2005/12/01 or later.
%
% This work has the LPPL maintenance status "maintained".
%
% This Current Maintainer of this work is Heiko Oberdiek.
%
% The Base Interpreter refers to any `TeX-Format',
% because some files are installed in TDS:tex/generic//.
%
% This work consists of the main source file soulutf8.dtx
% and the derived files
%    soulutf8.sty, soulutf8.pdf, soulutf8.ins, soulutf8.drv,
%    soulutf8-test1.tex, soulutf8-test2.tex, soulutf8-test3.tex,
%    soulutf8-test4.tex, soulutf8-test5.tex.
%
% Distribution:
%    CTAN:macros/latex/contrib/oberdiek/soulutf8.dtx
%    CTAN:macros/latex/contrib/oberdiek/soulutf8.pdf
%
% Unpacking:
%    (a) If soulutf8.ins is present:
%           tex soulutf8.ins
%    (b) Without soulutf8.ins:
%           tex soulutf8.dtx
%    (c) If you insist on using LaTeX
%           latex \let\install=y% \iffalse meta-comment
%
% File: soulutf8.dtx
% Version: 2016/05/16 v1.1
% Info: Permit use of UTF-8 characters in soul
%
% Copyright (C) 2007 by
%    Heiko Oberdiek <heiko.oberdiek at googlemail.com>
%    2016
%    https://github.com/ho-tex/oberdiek/issues
%
% This work may be distributed and/or modified under the
% conditions of the LaTeX Project Public License, either
% version 1.3c of this license or (at your option) any later
% version. This version of this license is in
%    http://www.latex-project.org/lppl/lppl-1-3c.txt
% and the latest version of this license is in
%    http://www.latex-project.org/lppl.txt
% and version 1.3 or later is part of all distributions of
% LaTeX version 2005/12/01 or later.
%
% This work has the LPPL maintenance status "maintained".
%
% This Current Maintainer of this work is Heiko Oberdiek.
%
% The Base Interpreter refers to any `TeX-Format',
% because some files are installed in TDS:tex/generic//.
%
% This work consists of the main source file soulutf8.dtx
% and the derived files
%    soulutf8.sty, soulutf8.pdf, soulutf8.ins, soulutf8.drv,
%    soulutf8-test1.tex, soulutf8-test2.tex, soulutf8-test3.tex,
%    soulutf8-test4.tex, soulutf8-test5.tex.
%
% Distribution:
%    CTAN:macros/latex/contrib/oberdiek/soulutf8.dtx
%    CTAN:macros/latex/contrib/oberdiek/soulutf8.pdf
%
% Unpacking:
%    (a) If soulutf8.ins is present:
%           tex soulutf8.ins
%    (b) Without soulutf8.ins:
%           tex soulutf8.dtx
%    (c) If you insist on using LaTeX
%           latex \let\install=y\input{soulutf8.dtx}
%        (quote the arguments according to the demands of your shell)
%
% Documentation:
%    (a) If soulutf8.drv is present:
%           latex soulutf8.drv
%    (b) Without soulutf8.drv:
%           latex soulutf8.dtx; ...
%    The class ltxdoc loads the configuration file ltxdoc.cfg
%    if available. Here you can specify further options, e.g.
%    use A4 as paper format:
%       \PassOptionsToClass{a4paper}{article}
%
%    Programm calls to get the documentation (example):
%       pdflatex soulutf8.dtx
%       makeindex -s gind.ist soulutf8.idx
%       pdflatex soulutf8.dtx
%       makeindex -s gind.ist soulutf8.idx
%       pdflatex soulutf8.dtx
%
% Installation:
%    TDS:tex/generic/oberdiek/soulutf8.sty
%    TDS:doc/latex/oberdiek/soulutf8.pdf
%    TDS:doc/latex/oberdiek/test/soulutf8-test1.tex
%    TDS:doc/latex/oberdiek/test/soulutf8-test2.tex
%    TDS:doc/latex/oberdiek/test/soulutf8-test3.tex
%    TDS:doc/latex/oberdiek/test/soulutf8-test4.tex
%    TDS:doc/latex/oberdiek/test/soulutf8-test5.tex
%    TDS:source/latex/oberdiek/soulutf8.dtx
%
%<*ignore>
\begingroup
  \catcode123=1 %
  \catcode125=2 %
  \def\x{LaTeX2e}%
\expandafter\endgroup
\ifcase 0\ifx\install y1\fi\expandafter
         \ifx\csname processbatchFile\endcsname\relax\else1\fi
         \ifx\fmtname\x\else 1\fi\relax
\else\csname fi\endcsname
%</ignore>
%<*install>
\input docstrip.tex
\Msg{************************************************************************}
\Msg{* Installation}
\Msg{* Package: soulutf8 2016/05/16 v1.1 Permit use of UTF-8 characters in soul (HO)}
\Msg{************************************************************************}

\keepsilent
\askforoverwritefalse

\let\MetaPrefix\relax
\preamble

This is a generated file.

Project: soulutf8
Version: 2016/05/16 v1.1

Copyright (C) 2007 by
   Heiko Oberdiek <heiko.oberdiek at googlemail.com>

This work may be distributed and/or modified under the
conditions of the LaTeX Project Public License, either
version 1.3c of this license or (at your option) any later
version. This version of this license is in
   http://www.latex-project.org/lppl/lppl-1-3c.txt
and the latest version of this license is in
   http://www.latex-project.org/lppl.txt
and version 1.3 or later is part of all distributions of
LaTeX version 2005/12/01 or later.

This work has the LPPL maintenance status "maintained".

This Current Maintainer of this work is Heiko Oberdiek.

The Base Interpreter refers to any `TeX-Format',
because some files are installed in TDS:tex/generic//.

This work consists of the main source file soulutf8.dtx
and the derived files
   soulutf8.sty, soulutf8.pdf, soulutf8.ins, soulutf8.drv,
   soulutf8-test1.tex, soulutf8-test2.tex, soulutf8-test3.tex,
   soulutf8-test4.tex, soulutf8-test5.tex.

\endpreamble
\let\MetaPrefix\DoubleperCent

\generate{%
  \file{soulutf8.ins}{\from{soulutf8.dtx}{install}}%
  \file{soulutf8.drv}{\from{soulutf8.dtx}{driver}}%
  \usedir{tex/generic/oberdiek}%
  \file{soulutf8.sty}{\from{soulutf8.dtx}{package}}%
%  \usedir{doc/latex/oberdiek/test}%
%  \file{soulutf8-test1.tex}{\from{soulutf8.dtx}{test1}}%
%  \file{soulutf8-test2.tex}{\from{soulutf8.dtx}{test2,utf8}}%
%  \file{soulutf8-test3.tex}{\from{soulutf8.dtx}{test3,utf8x}}%
%  \file{soulutf8-test4.tex}{\from{soulutf8.dtx}{test4,utf8,noetex}}%
%  \file{soulutf8-test5.tex}{\from{soulutf8.dtx}{test5,utf8x,noetex}}%
  \nopreamble
  \nopostamble
%  \usedir{source/latex/oberdiek/catalogue}%
%  \file{soulutf8.xml}{\from{soulutf8.dtx}{catalogue}}%
}

\catcode32=13\relax% active space
\let =\space%
\Msg{************************************************************************}
\Msg{*}
\Msg{* To finish the installation you have to move the following}
\Msg{* file into a directory searched by TeX:}
\Msg{*}
\Msg{*     soulutf8.sty}
\Msg{*}
\Msg{* To produce the documentation run the file `soulutf8.drv'}
\Msg{* through LaTeX.}
\Msg{*}
\Msg{* Happy TeXing!}
\Msg{*}
\Msg{************************************************************************}

\endbatchfile
%</install>
%<*ignore>
\fi
%</ignore>
%<*driver>
\NeedsTeXFormat{LaTeX2e}
\ProvidesFile{soulutf8.drv}%
  [2016/05/16 v1.1 Permit use of UTF-8 characters in soul (HO)]%
\documentclass{ltxdoc}
\usepackage{holtxdoc}[2011/11/22]
\begin{document}
  \DocInput{soulutf8.dtx}%
\end{document}
%</driver>
% \fi
%
%
% \CharacterTable
%  {Upper-case    \A\B\C\D\E\F\G\H\I\J\K\L\M\N\O\P\Q\R\S\T\U\V\W\X\Y\Z
%   Lower-case    \a\b\c\d\e\f\g\h\i\j\k\l\m\n\o\p\q\r\s\t\u\v\w\x\y\z
%   Digits        \0\1\2\3\4\5\6\7\8\9
%   Exclamation   \!     Double quote  \"     Hash (number) \#
%   Dollar        \$     Percent       \%     Ampersand     \&
%   Acute accent  \'     Left paren    \(     Right paren   \)
%   Asterisk      \*     Plus          \+     Comma         \,
%   Minus         \-     Point         \.     Solidus       \/
%   Colon         \:     Semicolon     \;     Less than     \<
%   Equals        \=     Greater than  \>     Question mark \?
%   Commercial at \@     Left bracket  \[     Backslash     \\
%   Right bracket \]     Circumflex    \^     Underscore    \_
%   Grave accent  \`     Left brace    \{     Vertical bar  \|
%   Right brace   \}     Tilde         \~}
%
% \GetFileInfo{soulutf8.drv}
%
% \title{The \xpackage{soulutf8} package}
% \date{2016/05/16 v1.1}
% \author{Heiko Oberdiek\thanks
% {Please report any issues at https://github.com/ho-tex/oberdiek/issues}\\
% \xemail{heiko.oberdiek at googlemail.com}}
%
% \maketitle
%
% \begin{abstract}
% This package extends package \xpackage{soul} and adds some support
% for UTF-8. Namely the input encodings \xfile{utf8.def}
% from package \xpackage{inputenc} and
% package \xpackage{ucs}'s \xfile{utf8x.def} are supported.
% \end{abstract}
%
% \tableofcontents
%
% \section{Documentation}
%
% This package \xpackage{soulutf8} does not have own options and does not
% define new user commands.
% Any option is passed to package \xpackage{soul} \cite{soul}
% that is loaded first.
% Then some internal macros of \xpackage{soul} are redefined to add
% support for UTF-8.
% The following input encodings are supported:
% \begin{quote}
%   \begin{tabular}{@{}lll@{}}
%     \texttt{utf8} & \LaTeX\ base &
%        \xfile{TDS:tex/latex/base/utf8.def} \cite{utf8}\\
%     \texttt{utf8x} & Package \xpackage{ucs} &
%       \xfile{TDS:tex/latex/ucs/utf8x.def} \cite{ucs}\\
%   \end{tabular}
% \end{quote}
% UTF-8 byte sequences are added as token group to a word, even
% if these UTF-8 characters are some kind of hyphen or space.
% As exception the following three Unicode characters are handled specially:
% \begin{quote}
%   \begin{tabular}{lll}
%     Slot & Name & Action\\
%     \hline
%     \texttt{U+00A0} & NO-BREAK SPACE & like |~|\\
%     \texttt{U+2013} & EN DASH & |--|\\
%     \texttt{U+2014} & EM DASH & |---|
%   \end{tabular}
%
% \subsection{Patch}
%
% Also package \xpackage{soulutf8} tries to patch package \xpackage{soul}
% to improve its behaviour:
% \begin{itemize}
% \item
%   A problem with additional levels of curly braces is fixed.
%   As advantage more implicite kernings are detected.
%   However, the result may be incompatible with the
%   original behaviour of package \xpackage{soul} because
%   of these respected implicite kernings.
% \item
%   \eTeX\ , especially \cs{unexpanded} is supported.
%   This allows a better protection of token groups
%   (|\mbox{|\dots|}|, math, \dots).
% \end{itemize}
% \end{quote}
%
% \subsection{Future}
%
% Currently package \xpackage{soul} does not seem to be
% maintained. Nevertheless if there will be a new version that
% adds support for UTF-8, then this package may become
% obsolete.
%
% \StopEventually{
% }
%
% \section{Implementation}
%
%    \begin{macrocode}
%<*package>
%    \end{macrocode}
%
% \subsection{Reload check and package identification}
%    Reload check, especially if the package is not used with \LaTeX.
%    \begin{macrocode}
\begingroup\catcode61\catcode48\catcode32=10\relax%
  \catcode13=5 % ^^M
  \endlinechar=13 %
  \catcode35=6 % #
  \catcode39=12 % '
  \catcode44=12 % ,
  \catcode45=12 % -
  \catcode46=12 % .
  \catcode58=12 % :
  \catcode64=11 % @
  \catcode123=1 % {
  \catcode125=2 % }
  \expandafter\let\expandafter\x\csname ver@soulutf8.sty\endcsname
  \ifx\x\relax % plain-TeX, first loading
  \else
    \def\empty{}%
    \ifx\x\empty % LaTeX, first loading,
      % variable is initialized, but \ProvidesPackage not yet seen
    \else
      \expandafter\ifx\csname PackageInfo\endcsname\relax
        \def\x#1#2{%
          \immediate\write-1{Package #1 Info: #2.}%
        }%
      \else
        \def\x#1#2{\PackageInfo{#1}{#2, stopped}}%
      \fi
      \x{soulutf8}{The package is already loaded}%
      \aftergroup\endinput
    \fi
  \fi
\endgroup%
%    \end{macrocode}
%    Package identification:
%    \begin{macrocode}
\begingroup\catcode61\catcode48\catcode32=10\relax%
  \catcode13=5 % ^^M
  \endlinechar=13 %
  \catcode35=6 % #
  \catcode39=12 % '
  \catcode40=12 % (
  \catcode41=12 % )
  \catcode44=12 % ,
  \catcode45=12 % -
  \catcode46=12 % .
  \catcode47=12 % /
  \catcode58=12 % :
  \catcode64=11 % @
  \catcode91=12 % [
  \catcode93=12 % ]
  \catcode123=1 % {
  \catcode125=2 % }
  \expandafter\ifx\csname ProvidesPackage\endcsname\relax
    \def\x#1#2#3[#4]{\endgroup
      \immediate\write-1{Package: #3 #4}%
      \xdef#1{#4}%
    }%
  \else
    \def\x#1#2[#3]{\endgroup
      #2[{#3}]%
      \ifx#1\@undefined
        \xdef#1{#3}%
      \fi
      \ifx#1\relax
        \xdef#1{#3}%
      \fi
    }%
  \fi
\expandafter\x\csname ver@soulutf8.sty\endcsname
\ProvidesPackage{soulutf8}%
  [2016/05/16 v1.1 Permit use of UTF-8 characters in soul (HO)]%
%    \end{macrocode}
%
% \subsection{Catcodes}
%
%    \begin{macrocode}
\begingroup\catcode61\catcode48\catcode32=10\relax%
  \catcode13=5 % ^^M
  \endlinechar=13 %
  \catcode123=1 % {
  \catcode125=2 % }
  \catcode64=11 % @
  \def\x{\endgroup
    \expandafter\edef\csname SOuL@AtEnd\endcsname{%
      \endlinechar=\the\endlinechar\relax
      \catcode13=\the\catcode13\relax
      \catcode32=\the\catcode32\relax
      \catcode35=\the\catcode35\relax
      \catcode61=\the\catcode61\relax
      \catcode64=\the\catcode64\relax
      \catcode123=\the\catcode123\relax
      \catcode125=\the\catcode125\relax
    }%
  }%
\x\catcode61\catcode48\catcode32=10\relax%
\catcode13=5 % ^^M
\endlinechar=13 %
\catcode35=6 % #
\catcode64=11 % @
\catcode123=1 % {
\catcode125=2 % }
\def\TMP@EnsureCode#1#2{%
  \edef\SOuL@AtEnd{%
    \SOuL@AtEnd
    \catcode#1=\the\catcode#1\relax
  }%
  \catcode#1=#2\relax
}
\TMP@EnsureCode{10}{12}% ^^J
\TMP@EnsureCode{33}{12}% !
\TMP@EnsureCode{34}{12}% "
\TMP@EnsureCode{36}{3}% $
\TMP@EnsureCode{39}{12}% '
\TMP@EnsureCode{40}{12}% (
\TMP@EnsureCode{41}{12}% )
\TMP@EnsureCode{42}{12}% *
\TMP@EnsureCode{43}{12}% +
\TMP@EnsureCode{44}{12}% ,
\TMP@EnsureCode{45}{12}% -
\TMP@EnsureCode{46}{12}% .
\TMP@EnsureCode{47}{12}% /
\TMP@EnsureCode{58}{12}% :
\TMP@EnsureCode{60}{12}% <
\TMP@EnsureCode{62}{12}% >
\TMP@EnsureCode{91}{12}% [
\TMP@EnsureCode{93}{12}% ]
\TMP@EnsureCode{94}{7}% ^
\TMP@EnsureCode{96}{12}% `
\TMP@EnsureCode{126}\active % ~
\TMP@EnsureCode{128}{12}% ^^80
\TMP@EnsureCode{147}{12}% ^^93
\TMP@EnsureCode{148}{12}% ^^94
\TMP@EnsureCode{160}{12}% ^^a0
\TMP@EnsureCode{194}{12}% ^^c2
\TMP@EnsureCode{226}{12}% ^^e2
\edef\SOuL@AtEnd{\SOuL@AtEnd\noexpand\endinput}
%    \end{macrocode}
%
% \subsection{Loading packages}
%
%    Package \xpackage{soul} uses \cs{documentclass} to detect \LaTeX.
%    \begin{macrocode}
\ifx\documentclass\@undefined
%    \end{macrocode}
%
% \subsubsection{\plainTeX}
%
%    First we check, whether package \xpackage{soul} is already loaded.
%    \begin{macrocode}
  \expandafter\ifx\csname SOUL@\endcsname\relax
%    \end{macrocode}
%    In case of \plainTeX\ package \xpackage{soul} defines some
%    macros in a simple manner that will break the definitions of
%    \xfile{miniltx.tex}, for example. Therefore these macros
%    are first saved and restored afterwards.
%    \begin{macrocode}
    \let\SOuL@orgDeclareRobustCommand\DeclareRobustCommand
    \let\SOuL@orgnewcommand          \newcommand
    \let\SOuL@orgDeclareOption       \DeclareOption
    \let\SOuL@orgPackageError        \PackageError
    \def\SOuL@restorelatexcmds{%
      \let\DeclareRobustCommand\SOuL@orgDeclareRobustCommand
      \let\newcommand          \SOuL@orgnewcommand
      \let\DeclareOption       \SOuL@orgDeclareOption
      \let\PackageError        \SOuL@orgPackageError
    }%
    \input soul.sty\relax
    \SOuL@restorelatexcmds
  \fi
%    \end{macrocode}
%    \begin{macro}{\SOUL@error}
%     Package \xpackage{soul}'s use of \cs{PackageError} is replaced
%     by \cs{@PackageError} of package \xpackage{infwarerr}.
%    \begin{macrocode}
  \input infwarerr.sty\relax
  \let\SOuL@orgSOUL@error\SOUL@error
  \def\SOUL@error{%
    \begingroup
      \let\PackageError\@PackageError
      \SOuL@orgSOUL@error
    \endgroup
  }%
%    \end{macrocode}
%    \end{macro}
%    \begin{macrocode}
  \input etexcmds.sty\relax
%    \end{macrocode}
%    \begin{macro}{\@onelevel@sanitize}
%    Define \LaTeX's \cs{@onelevel@sanitize} if not already available.
%    \begin{macrocode}
  \expandafter\ifx\csname @onelevel@sanitize\endcsname\relax
    \def\@onelevel@sanitize#1{%
      \edef#1{%
        \expandafter\strip@prefix\meaning#1%
      }%
    }%
%    \end{macrocode}
%    \end{macro}
%    \begin{macro}{\strip@prefix}
%    \begin{macrocode}
    \def\strip@prefix#1>{}%
%    \end{macrocode}
%    \end{macro}
%    \begin{macrocode}
  \fi
\else
%    \end{macrocode}
%
% \subsubsection{\LaTeX}
%    \begin{macrocode}
  \DeclareOption*{\PassOptionsToPackage{\CurrentOption}{soul}}%
  \ProcessOptions\relax
  \RequirePackage{soul}[2003/11/17]%
  \RequirePackage{infwarerr}[2016/05/16]%
  \RequirePackage{etexcmds}[2016/05/16]%
\fi
%    \end{macrocode}
%
% \subsubsection{\eTeX}
%
%    In \plainTeX\ command \cs{+} is an \emph{outer} macro.
%    Therefore numbers are used to avoid problems.
%    \begin{macrocode}
\ifetex@unexpanded
  \catcode33=14 % '!': comment
  \catcode43=9  % '+': ignore
\else
  \catcode33=9  % '!': ignore
  \catcode43=14 % '+': comment
\fi
%    \end{macrocode}
%
% \subsection{Macro for redefinitions}
%
%    \begin{macro}{\SOuL@redefine}
%    \begin{macrocode}
\def\SOuL@redefine#1{%
  \begingroup
    \def\SOuL@cmd{#1}%
    \afterassignment\SOuL@cmdcheck
    \def\SOuL@temp
}
%    \end{macrocode}
%    \end{macro}
%    \begin{macro}{\SOuL@cmdcheck}
%    \begin{macrocode}
\def\SOuL@cmdcheck{%
    \expandafter\ifx\SOuL@cmd\SOuL@temp
    \else
      \edef\SOuL@temp*{\expandafter\string\SOuL@cmd}%
      \@PackageWarningNoLine{soulutf8}{%
        Command \SOuL@temp* has changed.\MessageBreak
        Supported versions of package `soul': 2003/11/17.\MessageBreak
        Depending on the unknown changes the redefinition\MessageBreak
        of \SOuL@temp* may not behave correctly%
      }%
    \fi
  \expandafter\endgroup
  \expandafter\def\SOuL@cmd
}
%    \end{macrocode}
%    \end{macro}
%
% \subsection{Redefinition of \cs{SOUL@eval}}
%
%    \begin{macro}{\SOUL@eval}
%    Macro \cs{SOUL@eval} is redefined to add detection of
%    the first byte of a UTF-8 sequence. Because \cs{SOUL@eval}
%    is overwritten, a warning is issued, if the contents of
%    \cs{SOUL@eval} is not as expected.
%    \begin{macrocode}
\SOuL@redefine\SOUL@eval{%
%    \end{macrocode}
%    First the expected definition.
%    \begin{macrocode}
  \def\SOUL@n*##1{\SOUL@scan}%
  \if\noexpand\SOUL@@\SOUL@spc
  \else
    \SOUL@ignorespacesfalse
  \fi
  \ifnum\SOUL@minus=\thr@@
    \SOUL@flushminus
  \else\ifnum\SOUL@comma=\tw@
    \SOUL@flushcomma
  \else\ifnum\SOUL@apo=\tw@
    \SOUL@flushapo
  \else\ifnum\SOUL@grave=\tw@
    \SOUL@flushgrave
  \fi\fi\fi\fi
  \ifx\SOUL@@-\else\SOUL@flushminus\fi
  \ifx\SOUL@@,\else\SOUL@flushcomma\fi
  \ifx\SOUL@@'\else\SOUL@flushapo\fi
  \ifx\SOUL@@`\else\SOUL@flushgrave\fi
  \ifx\SOUL@@-%
    \advance\SOUL@minus\@ne
  \else\ifx\SOUL@@,%
    \advance\SOUL@comma\@ne
  \else\ifx\SOUL@@'%
    \advance\SOUL@apo\@ne
  \else\ifx\SOUL@@`%
    \advance\SOUL@grave\@ne
  \else
    \SOUL@flushminus
    \SOUL@flushcomma
    \SOUL@flushapo
    \SOUL@flushgrave
    \ifx\SOUL@@\SOUL@stop
      \def\SOUL@n*{%
        \SOUL@doword
        \SOUL@eventuallyexhyphen\null
      }%
    \else\ifx\SOUL@@\par
      \def\SOUL@n*\par{\par\leavevmode\SOUL@scan}%
    \else\if\noexpand\SOUL@@\SOUL@spc
      \SOUL@doword
      \SOUL@eventuallyexhyphen\null
      \ifSOUL@ignorespaces
      \else
        \SOUL@everyspace{}%
      \fi
      \def\SOUL@n* {\SOUL@scan}%
    \else\ifx\SOUL@@\\%
      \SOUL@doword
      \SOUL@eventuallyexhyphen\null
      \SOUL@everyspace{\unskip\nobreak\hfil\break}%
      \SOUL@ignorespacestrue
    \else\ifx\SOUL@@~%
      \SOUL@doword
      \SOUL@eventuallyexhyphen\null
      \SOUL@everyspace{\nobreak}%
    \else\ifx\SOUL@@\slash
      \SOUL@doword
      \SOUL@eventuallyexhyphen{/}%
      \SOUL@exhyphen{/}%
    \else\ifx\SOUL@@\mbox
      \def\SOUL@n*{\SOUL@addprotect}%
    \else\ifx\SOUL@@\hbox
      \def\SOUL@n*{\SOUL@addprotect}%
    \else\ifx\SOUL@@\soulomit
      \def\SOUL@n*\soulomit##1{%
        \SOUL@doword
        {\spaceskip\SOUL@spaceskip##1}%
        \SOUL@scan
      }%
    \else\ifx\SOUL@@\break
      \SOUL@doword
      \break
    \else\ifx\SOUL@@\linebreak
      \SOUL@doword
      \SOUL@everyspace{\linebreak}%
    \else\ifcat\bgroup\noexpand\SOUL@@
      \def\SOUL@n*{\SOUL@addgroup{}}%
    \else\ifcat$\noexpand\SOUL@@
      \def\SOUL@n*{\SOUL@addmath}%
    \else
      \def\SOUL@n*{\SOUL@dotoken}%
    \fi\fi\fi\fi\fi\fi\fi\fi\fi\fi\fi\fi\fi
  \fi\fi\fi\fi
  \SOUL@n*%
}{%
%    \end{macrocode}
%    Now the redefined version follows.
%    \begin{macrocode}
  \def\SOUL@n*##1{\SOUL@scan}%
  \if\noexpand\SOUL@@\SOUL@spc
  \else
    \SOUL@ignorespacesfalse
  \fi
  \ifnum\SOUL@minus=\thr@@
    \SOUL@flushminus
  \else\ifnum\SOUL@comma=\tw@
    \SOUL@flushcomma
  \else\ifnum\SOUL@apo=\tw@
    \SOUL@flushapo
  \else\ifnum\SOUL@grave=\tw@
    \SOUL@flushgrave
  \fi\fi\fi\fi
  \ifx\SOUL@@-\else\SOUL@flushminus\fi
  \ifx\SOUL@@,\else\SOUL@flushcomma\fi
  \ifx\SOUL@@'\else\SOUL@flushapo\fi
  \ifx\SOUL@@`\else\SOUL@flushgrave\fi
  \ifx\SOUL@@-%
    \advance\SOUL@minus\@ne
  \else\ifx\SOUL@@,%
    \advance\SOUL@comma\@ne
  \else\ifx\SOUL@@'%
    \advance\SOUL@apo\@ne
  \else\ifx\SOUL@@`%
    \advance\SOUL@grave\@ne
  \else
    \SOUL@flushminus
    \SOUL@flushcomma
    \SOUL@flushapo
    \SOUL@flushgrave
    \ifx\SOUL@@\SOUL@stop
      \def\SOUL@n*{%
          \SOUL@doword
          \SOUL@eventuallyexhyphen\null
      }%
    \else\ifx\SOUL@@\par
      \def\SOUL@n*\par{\par\leavevmode\SOUL@scan}%
    \else\if\noexpand\SOUL@@\SOUL@spc
      \SOUL@doword
      \SOUL@eventuallyexhyphen\null
      \ifSOUL@ignorespaces
      \else
        \SOUL@everyspace{}%
      \fi
      \def\SOUL@n* {\SOUL@scan}%
    \else\ifx\SOUL@@\\%
      \SOUL@doword
      \SOUL@eventuallyexhyphen\null
      \SOUL@everyspace{\unskip\nobreak\hfil\break}%
      \SOUL@ignorespacestrue
    \else\ifx\SOUL@@~%
      \SOUL@doword
      \SOUL@eventuallyexhyphen\null
      \SOUL@everyspace{\nobreak}%
    \else\ifx\SOUL@@\slash
      \SOUL@doword
      \SOUL@eventuallyexhyphen{/}%
      \SOUL@exhyphen{/}%
    \else\ifx\SOUL@@\mbox
      \def\SOUL@n*{\SOUL@addprotect}%
    \else\ifx\SOUL@@\hbox
      \def\SOUL@n*{\SOUL@addprotect}%
    \else\ifx\SOUL@@\soulomit
      \def\SOUL@n*\soulomit##1{%
        \SOUL@doword
        {\spaceskip\SOUL@spaceskip##1}%
        \SOUL@scan
      }%
    \else\ifx\SOUL@@\break
      \SOUL@doword
      \break
    \else\ifx\SOUL@@\linebreak
      \SOUL@doword
      \SOUL@everyspace{\linebreak}%
    \else\ifcat\bgroup\noexpand\SOUL@@
      \def\SOUL@n*{\SOUL@addgroup{}}%
    \else\ifcat$\noexpand\SOUL@@
      \def\SOUL@n*{\SOUL@addmath}%
    \else
%    \end{macrocode}
%    The current token is examined to detect the start of
%    a UTF-8 sequence.
%    \begin{macrocode}
      \SOuL@analyzeutfviii
      \ifcase\SOuL@octets
        \SOuL@analyzeutfviiix
      \fi
      \ifcase\SOuL@octets
        \def\SOUL@n*{\SOUL@dotoken}%
      \or % 1
      \or % 2
        \def\SOUL@n*{\SOuL@addtwooctets}%
      \or % 3
        \def\SOUL@n*{\SOuL@addthreeoctets}%
      \or % 4
        \def\SOUL@n*{\SOuL@addfouroctets}%
      \fi
    \fi\fi\fi\fi\fi\fi\fi\fi\fi\fi\fi\fi\fi
  \fi\fi\fi\fi
  \SOUL@n*%
}
%    \end{macrocode}
%    \end{macro}
%
% \subsection{UTF-8 analysis}
%
% \subsubsection{Help strings}
%
%    \begin{macrocode}
\def\SOuL@defsanitizedstring#1#2{%
  \expandafter\def\csname SOuL@string#1\endcsname{#2}%
  \expandafter\@onelevel@sanitize\csname SOuL@string#1\endcsname
}
\SOuL@defsanitizedstring{UTFviii}{UTFviii@}
\SOuL@defsanitizedstring{octets}{@octets}
\SOuL@defsanitizedstring{two}{two}
\SOuL@defsanitizedstring{three}{three}
\SOuL@defsanitizedstring{four}{four}
\SOuL@defsanitizedstring{macrocolon}{macro:}
\SOuL@defsanitizedstring{csnameu}{csname u8-}
\SOuL@defsanitizedstring{undeferr}{utf@viii@undeferr}
\def\SOuL@stringendash{^^e2^^80^^93}
\def\SOuL@stringemdash{^^e2^^80^^94}
\def\SOuL@stringnobreakspace{^^c2^^a0}
\edef\SOuL@charhash{\string #}
\edef\SOuL@chartwo{\string 2}
\edef\SOuL@charthree{\string 3}
\def\SOuL@empty{}
%    \end{macrocode}
%
% \subsubsection{Support for \xfile{utf8.def}}
%
%    \begin{macro}{\SOuL@analyzeutfviii}
%    \begin{macrocode}
\begingroup
  \edef\x{\endgroup
    \def\noexpand\SOuL@analyzeutfviii{%
      \noexpand\expandafter\noexpand\SOuL@checkutfviii
      \noexpand\meaning\noexpand\SOUL@@
      \SOuL@stringUTFviii\SOuL@stringoctets
      \noexpand\@nil
    }%
    \def\noexpand\SOuL@checkutfviii
      ##1\SOuL@stringUTFviii##2\SOuL@stringoctets##3\noexpand\@nil
  }%
\x{%
  \def\SOuL@temp{#2}%
  \chardef\SOuL@octets=%
      \ifx\SOuL@temp\SOuL@stringtwo
        \tw@
      \else\ifx\SOuL@temp\SOuL@stringthree
        \thr@@
      \else\ifx\SOuL@temp\SOuL@stringfour
        4 %
      \else
        \z@
      \fi\fi\fi
}
%    \end{macrocode}
%    \end{macro}
%
% \subsubsection{Support for \xfile{utf8x.def}}
%
%    \begin{macro}{\SOuL@analyzeutfviiix}
%    \begin{macrocode}
\begingroup
  \edef\x{\endgroup
    \def\noexpand\SOuL@analyzeutfviiix{%
      \noexpand\expandafter\noexpand\SOuL@checkutfviiix
      \noexpand\meaning\noexpand\SOUL@@
      \SOuL@stringmacrocolon\SOuL@charhash1{}{}{}{}%
      \SOuL@stringcsnameu\SOuL@stringundeferr
      \noexpand\@nil
    }%
%    \end{macrocode}
%    \end{macro}
%    \begin{macro}{\SOuL@checkutfviiix}
%    \begin{macrocode}
    \def\noexpand\SOuL@checkutfviiix
      ##1\SOuL@stringmacrocolon\SOuL@charhash1##2##3##4##5##6%
      \SOuL@stringcsnameu##7\SOuL@stringundeferr##8\noexpand\@nil
  }%
\x{%
  \def\SOuL@temp{#7}%
  \ifx\SOuL@temp\SOuL@empty
    \chardef\SOuL@octets=\z@
  \else
    \def\SOuL@temp{#5}%
    \ifx\SOuL@temp\SOuL@charthree
      \chardef\SOuL@octets=4 %
    \else
      \def\SOuL@temp{#3}%
      \ifx\SOuL@temp\SOuL@chartwo
        \chardef\SOuL@octets=\thr@@
      \else
        \chardef\SOuL@octets=\tw@
      \fi
    \fi
  \fi
}
%    \end{macrocode}
%    \end{macro}
%
% \subsection{Actions for UTF-8 sequences}
%
%    \begin{macro}{\SOuL@addtwooctets}
%    \begin{macrocode}
\def\SOuL@addtwooctets#1#2{%
  \def\SOuL@temp{#1#2}%
  \@onelevel@sanitize\SOuL@temp
  \ifx\SOuL@temp\SOuL@stringnobreakspace
    \SOUL@doword
    \SOUL@eventuallyexhyphen\null
    \SOUL@everyspace{\nobreak}%
    \let\SOuL@next\SOUL@scan
  \else
    \def\SOuL@next{%
!     \SOUL@addtoken{{\noexpand#1\noexpand#2}}%
+     \SOUL@addtoken{{\etex@unexpanded{#1#2}}}%
    }%
  \fi
  \SOuL@next
}
%    \end{macrocode}
%    \end{macro}
%    \begin{macro}{\SOuL@addthreeoctets}
%    \begin{macrocode}
\def\SOuL@addthreeoctets#1#2#3{%
  \def\SOuL@temp{#1#2#3}%
  \@onelevel@sanitize\SOuL@temp
  \ifx\SOuL@temp\SOuL@stringendash
    \SOUL@doword
    \SOUL@eventuallyexhyphen{-}%
    \SOUL@exhyphen{--}%
    \let\SOuL@next\SOUL@scan
  \else
    \ifx\SOuL@temp\SOuL@stringemdash
      \SOUL@doword
      \SOUL@eventuallyexhyphen{-}%
      \SOUL@exhyphen{---}%
      \let\SOuL@next\SOUL@scan
    \else
      \def\SOuL@next{%
!       \SOUL@addtoken{{\noexpand#1\noexpand#2\noexpand#3}}%
+       \SOUL@addtoken{{\etex@unexpanded{#1#2#3}}}%
      }%
    \fi
  \fi
  \SOuL@next
}
%    \end{macrocode}
%    \end{macro}
%    \begin{macro}{\SOuL@addfouroctets}
%    \begin{macrocode}
\def\SOuL@addfouroctets#1#2#3#4{%
! \SOUL@addtoken{{\noexpand#1\noexpand#2\noexpand#3\noexpand#4}}%
+ \SOUL@addtoken{{\etex@unexpanded{#1#2#3#4}}}%
}
%    \end{macrocode}
%    \end{macro}
%
% \subsubsection{Redefinition of \cs{SOUL@splittoken}}
%
%    \begin{macro}{\SOUL@splittoken}
%    Macro \cs{SOUL@splittoken} separates the first token
%    or token group from a word and redefines the word to contain
%    the remaining tokens. However if the remaining tokens are
%    a token group, then the curly braces will be removed and
%    the token group is splitted by the next call of \cs{SOUL@splittoken}.
%    The redefinition avoids the removal of curly braces around the
%    remaining tokens.
%    \begin{macrocode}
\SOuL@redefine\SOUL@splittoken#1#2\SOUL@stop{%
  \global\SOUL@token={#1}%
  \global\SOUL@word={#2}%
}#1{%
  \global\SOUL@token={#1}%
  \SOuL@remainingtoken\relax
}
%    \end{macrocode}
%    \end{macro}
%    \begin{macro}{\SOuL@remainingtoken}
%    \begin{macrocode}
\def\SOuL@remainingtoken#1\SOUL@stop{%
  \global\SOUL@word=\expandafter{\@gobble#1}%
}
%    \end{macrocode}
%    \end{macro}
%
% \subsection{Patches}
%
%    The fixed \cs{SOUL@splittoken} allows to remove the double
%    sets of curly braces in other macros of package \xpackage{soul}.
%    The benefit is that implicite kernings are more often detected
%    and fixes a bug in package \xpackage{soul}. The disadvantage is
%    incompatibility. The width of the resulting strings may change.
%    \begin{macro}{\SOUL@flushcomma}
%    \begin{macrocode}
\SOuL@redefine\SOUL@flushcomma{%
  \ifcase\SOUL@comma
  \or
    \edef\x{\SOUL@word={\the\SOUL@word,}}\x
  \or
    \edef\x{\SOUL@word={\the\SOUL@word{{,,}}}}\x
  \fi
  \SOUL@comma\z@
}{%
  \ifcase\SOUL@comma
  \or
    \edef\x{\SOUL@word={\the\SOUL@word,}}\x
  \or
    \edef\x{\SOUL@word={\the\SOUL@word{,,}}}\x
  \fi
  \SOUL@comma\z@
}
%    \end{macrocode}
%    \end{macro}
%    \begin{macro}{\SOUL@flushapo}
%    \begin{macrocode}
\SOuL@redefine\SOUL@flushapo{%
  \ifcase\SOUL@apo
  \or
    \edef\x{\SOUL@word={\the\SOUL@word'}}\x
  \or
    \edef\x{\SOUL@word={\the\SOUL@word{{''}}}}\x
  \fi
  \SOUL@apo\z@
}{%
  \ifcase\SOUL@apo
  \or
    \edef\x{\SOUL@word={\the\SOUL@word'}}\x
  \or
    \edef\x{\SOUL@word={\the\SOUL@word{''}}}\x
  \fi
  \SOUL@apo\z@
}
%    \end{macrocode}
%    \end{macro}
%    \begin{macro}{\SOUL@flushgrave}
%    \begin{macrocode}
\SOuL@redefine\SOUL@flushgrave{%
  \ifcase\SOUL@grave
  \or
    \edef\x{\SOUL@word={\the\SOUL@word`}}\x
  \or
    \edef\x{\SOUL@word={\the\SOUL@word{{``}}}}\x
  \fi
  \SOUL@grave\z@
}{%
  \ifcase\SOUL@grave
  \or
    \edef\x{\SOUL@word={\the\SOUL@word`}}\x
  \or
    \edef\x{\SOUL@word={\the\SOUL@word{``}}}\x
  \fi
  \SOUL@grave\z@
}
%    \end{macrocode}
%    \end{macro}
%    \begin{macro}{\SOUL@addgroup}
%    \begin{macrocode}
\SOuL@redefine\SOUL@addgroup#1#2{%
  {%
    \let\protect\noexpand
    \edef\x{%
      \global\SOUL@word={%
        \the\SOUL@word
        {{\noexpand#1#2}}%
      }%
    }%
    \x
  }%
  \SOUL@scan
}#1#2{%
  \begingroup
    \let\protect\noexpand
    \edef\x{\endgroup
      \SOUL@word={%
        \the\SOUL@word
!       {\noexpand#1{#2}}%
+       {\etex@unexpanded{#1{#2}}}%
      }%
    }%
  \x
  \SOUL@scan
}
%    \end{macrocode}
%    \end{macro}
%    \begin{macro}{\SOUL@addmath}
%    \begin{macrocode}
\SOuL@redefine\SOUL@addmath$#1${%
  {%
    \let\protect\noexpand
    \edef\x{%
      \global\SOUL@word={%
        \the\SOUL@word
        {{\hbox{$#1$}}}%
      }%
    }%
    \x
  }%
  \SOUL@scan
}$#1${%
  \begingroup
    \let\protect\noexpand
    \edef\x{\endgroup
      \SOUL@word={%
        \the\SOUL@word
!       {\hbox{$#1$}}%
+       {\etex@unexpanded{\hbox{$#1$}}}%
      }%
    }%
  \x
  \SOUL@scan
}
%    \end{macrocode}
%    \end{macro}
%    \begin{macro}{\SOUL@addprotect}
%    \begin{macrocode}
\SOuL@redefine\SOUL@addprotect#1#2{%
  {%
    \let\protect\noexpand
    \edef\x{%
      \global\SOUL@word={%
        \the\SOUL@word
        {{\hbox{#2}}}%
      }%
    }%
    \x
  }%
  \SOUL@scan
}#1#2{%
  \begingroup
    \let\protect\noexpand
    \edef\x{\endgroup
      \SOUL@word={%
        \the\SOUL@word
!       {\hbox{#2}}%
+       {\etex@unexpanded{\hbox{#2}}}%
      }%
    }%
  \x
  \SOUL@scan
}
%    \end{macrocode}
%    \end{macro}
%
%    \begin{macro}{\SOUL@addtoken}
%    \begin{macrocode}
+ \SOuL@redefine\SOUL@addtoken#1{%
+   \edef\x{%
+     \SOUL@word={%
+       \the\SOUL@word
+       \noexpand#1%
+     }%
+   }%
+   \x
+   \SOUL@scan
+ }#1{%
+   \edef\x{%
+     \SOUL@word={%
+       \the\SOUL@word
+       \etex@unexpanded{#1}%
+     }%
+   }%
+   \x
+   \SOUL@scan
+ }%
%    \end{macrocode}
%    \end{macro}
%
%    \begin{macrocode}
\SOuL@AtEnd%
%    \end{macrocode}
%
%    \begin{macrocode}
%</package>
%    \end{macrocode}
%
% \section{Test}
%
% \subsection{Catcode checks for loading}
%
%    \begin{macrocode}
%<*test1>
%    \end{macrocode}
%    \begin{macrocode}
\catcode`\{=1 %
\catcode`\}=2 %
\catcode`\#=6 %
\catcode`\@=11 %
\expandafter\ifx\csname count@\endcsname\relax
  \countdef\count@=255 %
\fi
\expandafter\ifx\csname @gobble\endcsname\relax
  \long\def\@gobble#1{}%
\fi
\expandafter\ifx\csname @firstofone\endcsname\relax
  \long\def\@firstofone#1{#1}%
\fi
\expandafter\ifx\csname loop\endcsname\relax
  \expandafter\@firstofone
\else
  \expandafter\@gobble
\fi
{%
  \def\loop#1\repeat{%
    \def\body{#1}%
    \iterate
  }%
  \def\iterate{%
    \body
      \let\next\iterate
    \else
      \let\next\relax
    \fi
    \next
  }%
  \let\repeat=\fi
}%
\def\RestoreCatcodes{}
\count@=0 %
\loop
  \edef\RestoreCatcodes{%
    \RestoreCatcodes
    \catcode\the\count@=\the\catcode\count@\relax
  }%
\ifnum\count@<255 %
  \advance\count@ 1 %
\repeat

\def\RangeCatcodeInvalid#1#2{%
  \count@=#1\relax
  \loop
    \catcode\count@=15 %
  \ifnum\count@<#2\relax
    \advance\count@ 1 %
  \repeat
}
\def\RangeCatcodeCheck#1#2#3{%
  \count@=#1\relax
  \loop
    \ifnum#3=\catcode\count@
    \else
      \errmessage{%
        Character \the\count@\space
        with wrong catcode \the\catcode\count@\space
        instead of \number#3%
      }%
    \fi
  \ifnum\count@<#2\relax
    \advance\count@ 1 %
  \repeat
}
\def\space{ }
\expandafter\ifx\csname LoadCommand\endcsname\relax
  \def\LoadCommand{\input soulutf8.sty\relax}%
\fi
\def\Test{%
  \RangeCatcodeInvalid{0}{47}%
  \RangeCatcodeInvalid{58}{64}%
  \RangeCatcodeInvalid{91}{96}%
  \RangeCatcodeInvalid{123}{255}%
  \catcode`\@=12 %
  \catcode`\\=0 %
  \catcode`\%=14 %
  \LoadCommand
  \RangeCatcodeCheck{0}{36}{15}%
  \RangeCatcodeCheck{37}{37}{14}%
  \RangeCatcodeCheck{38}{47}{15}%
  \RangeCatcodeCheck{48}{57}{12}%
  \RangeCatcodeCheck{58}{63}{15}%
  \RangeCatcodeCheck{64}{64}{12}%
  \RangeCatcodeCheck{65}{90}{11}%
  \RangeCatcodeCheck{91}{91}{15}%
  \RangeCatcodeCheck{92}{92}{0}%
  \RangeCatcodeCheck{93}{96}{15}%
  \RangeCatcodeCheck{97}{122}{11}%
  \RangeCatcodeCheck{123}{255}{15}%
  \RestoreCatcodes
}
\Test
\csname @@end\endcsname
\end
%    \end{macrocode}
%    \begin{macrocode}
%</test1>
%    \end{macrocode}
%
% \subsection{\LaTeX\space tests}
%
%    \begin{macrocode}
%<*test2|test3|test4|test5>
%    \end{macrocode}
%    \begin{macrocode}
%<noetex>\let\SavedUnexpanded\unexpanded
%<noetex>\let\unexpanded\UNDEFINED
\NeedsTeXFormat{LaTeX2e}
\documentclass{article}
\usepackage[T1]{fontenc}
\usepackage{lmodern}
\usepackage{textcomp}
\usepackage[ngerman]{babel}
\expandafter\def\csname ver@hypdestopt.sty\endcsname{}
\usepackage[unicode]{hyperref}
\usepackage{bookmark}
\usepackage{color}
\usepackage{soul}[2003/11/17]
\makeatletter
\newcommand*{\@ifDefined}[1]{%
  \@ifundefined{#1}{%
    \@latex@error{%
      \expandafter\noexpand\csname#1\endcsname undefined%
    }\@ehc
  }%
}
\newcommand*{\SaveOrg}[1]{%
  \@ifdefinable{ORG@SOUL@#1}{%
    \@ifDefined{SOUL@#1}{%
      \expandafter\let\csname ORG@SOUL@#1\expandafter\endcsname
                      \csname SOUL@#1\endcsname
    }%
  }%
}
\newcommand*{\RestoreOrg}[1]{%
  \@ifDefined{ORG@SOUL@#1}{%
    \@ifDefined{SOUL@#1}{%
      \expandafter\let\csname SOUL@#1\expandafter\endcsname
                      \csname ORG@SOUL@#1\endcsname
    }%
  }%
}
\makeatother
\SaveOrg{eval}
\SaveOrg{splittoken}
\SaveOrg{flushcomma}
\SaveOrg{flushapo}
\SaveOrg{grave}
\SaveOrg{addgroup}
\SaveOrg{addmath}
\SaveOrg{addprotect}
\SaveOrg{addtoken}
\usepackage{soulutf8}[2016/05/16]
%    \end{macrocode}
%    \begin{macrocode}
%<noetex>\let\unexpanded\SavedUnexpanded
%    \end{macrocode}
%    Input encoding \xfile{utf8}.
%    \begin{macrocode}
%<*utf8>
\usepackage[utf8]{inputenc}
\DeclareUnicodeCharacter{00A0}{\nobreakspace}
%</utf8>
%    \end{macrocode}
%    Input encoding \xfile{utf8x}. Package \xpackage{ucs}
%    has a nasty bug, that it redefines \cs{show}.
%    \begin{macrocode}
%<*utf8x>
\let\origshow\show
\usepackage[utf8x]{inputenc}
\AtBeginDocument{%
  \let\show\origshow
}
%</utf8x>
%    \end{macrocode}
%
% \subsubsection{Driver definitions}
%
%    \begin{macrocode}
\makeatletter
%    \end{macrocode}
%    \begin{macro}{\an}
%    The definition for \cs{an} is taken from the documentation of
%    package \xpackage{soul}.
%    \begin{macrocode}
\DeclareRobustCommand*{\an}{%
  \def\SOUL@preamble{$^{^P}$}%
  \def\SOUL@everyspace##1{##1\texttt{\char`\ }\linebreak[2]}%
  \def\SOUL@postamble{$^{^E}$}%
  \def\SOUL@everyhyphen{$^{^-}$\linebreak[1]}%
  \def\SOUL@everyexhyphen##1{##1$^{^=}$}%
  \def\SOUL@everysyllable{$^{^S}$}%
  \def\SOUL@everytoken{\the\SOUL@token$^{^T}$}%
  \def\SOUL@everylowerthan{$^{^L}$}%
  \SOUL@
}
%    \end{macrocode}
%    \end{macro}
%
%    \begin{macro}{\sy}
%    The definition for \cs{sy} is taken from the documentation of
%    package \xpackage{soul}.
%    \begin{macrocode}
\DeclareRobustCommand*{\sy}{%
  \SOUL@setup
  \def\SOUL@preamble{%
    \lefthyphenmin=\z@
    \righthyphenmin=\z@
  }%
  \def\SOUL@everyspace##1{##1\space}%
  \def\SOUL@everysyllable{\the\SOUL@syllable}%
  \def\SOUL@everyhyphen{%
    \discretionary{%
      \SOUL@setkern\SOUL@hyphkern
      \SOUL@sethyphenchar
    }{}{%
      \hbox{\kern1pt$\cdot$}%
    }%
  }%
  \def\SOUL@everyexhyphen##1{%
    \SOUL@setkern\SOUL@hyphkern
    \hbox{##1}%
    \discretionary{}{}{%
      \SOUL@setkern\SOUL@charkern
    }%
  }%
  \SOUL@
}
%    \end{macrocode}
%    \end{macro}
%
%    \begin{macro}{\soulH}
%    \begin{macrocode}
\DeclareRobustCommand*{\soulH}{%
  \def\SOUL@preamble{%
    \gdef\SOULH{}%
    \soulHadd{[}%
  }%
  \def\SOUL@everyspace##1{%
    \soulHadd{Sp}%
  }%
  \def\SOUL@postamble{%
    \soulHadd{]}%
  }%
  \def\SOUL@everyhyphen{%
    \soulHadd{Hy}%
  }%
  \def\SOUL@everyexhyphen##1{%
    \soulHadd{Exhy}%
  }%
  \def\SOUL@everysyllable{%
    \soulHadd{Sy}%
  }%
  \def\SOUL@everytoken{%
    \soulHadd{T}%
  }%
  \def\SOUL@everylowerthan{%
    \soulHadd{<}%
  }%
  \SOUL@
}
\newcommand*{\soulHadd}[1]{%
  #1%
  \expandafter\gdef\expandafter\soulHstring\expandafter{%
    \soulHstring
    #1%
  }%
}
%    \end{macrocode}
%    \end{macro}
%
% \subsubsection{Test macros}
%
%    \begin{macrocode}
\usepackage{qstest}
\IncludeTests{*}
\LogTests{log}{*}{*}
%    \end{macrocode}
%    \begin{macrocode}
\newsavebox\boxA
\newsavebox\boxB
\newsavebox\boxC
%    \end{macrocode}
%    \begin{macro}{\TestDriver}
%    \begin{macrocode}
\newcommand*{\TestDriver}[4]{%
  \subsection{Driver #1}%
  \@PackageInfo{soulutf8-test}{* Driver #1}%
  \expandafter\let\expandafter\testdriver\csname #1\endcsname
  \def\teststringA{#3}%
  \def\teststringB{#4}%
  \begin{qstest}{string-#2-driver-#1}{string,#2,driver,#1}%
    \@PackageInfo{soulutf8-test}{* Version: soulutf8}%
    \global\let\soulHstring\@empty
    \setbox0=\hbox{\expandafter\testdriver\expandafter{\teststringA}}%
    \let\HstringA\soulHstring
    \SetTestVBox{\boxA}{\teststringA}%
    %
    \@PackageInfo{soulutf8-test}{* Version: soul + patch}%
    \global\let\soulHstring\@empty
    \RestoreOrg{eval}%
    \setbox2=\hbox{\expandafter\testdriver\expandafter{\teststringB}}%
    \let\HstringB\soulHstring
    \SetTestVBox{\boxB}{\teststringB}%
    %
    \@PackageInfo{soulutf8-test}{* Version: original soul}%
    \global\let\soulHstring\@empty
    \RestoreOrg{splittoken}%
    \RestoreOrg{flushcomma}%
    \RestoreOrg{flushapo}%
    \RestoreOrg{grave}%
    \RestoreOrg{addgroup}%
    \RestoreOrg{addmath}%
    \RestoreOrg{addprotect}%
    \RestoreOrg{addtoken}%
    \setbox4=\hbox{\expandafter\testdriver\expandafter{\teststringB}}%
    \let\HstringC\soulHstring
    \SetTestVBox{\boxC}{\teststringB}%
    %
    \TestInfo{0}{soulutf8}%
    \TestInfo{2}{soul + patch}%
    \TestInfo{4}{original soul}%
    \TestInfoVBox
    \showboxdepth=\maxdimen
    \showboxbreadth=\maxdimen
    \tracingonline=\@ne
    \Expect*{\HstringA}*{\HstringB}%
    \Expect*{\HstringA}*{\HstringC}%
    \Expect*{\the\wd0}*{\the\wd2}%
    \ifdim\wd0=\wd2 %
    \else
      \showbox0 %
      \showbox2 %
    \fi
    \Expect*{\the\ht0}*{\the\ht2}%
    \Expect*{\the\dp0}*{\the\dp2}%
    \Expect*{\the\wd\boxB}*{\the\wd\boxA}%
    \Expect*{\the\wd\boxC}*{\the\wd\boxA}%
    \Expect*{\the\ht\boxB}*{\the\ht\boxA}%
    \ifdim\ht\boxB=\ht\boxA
    \else
      \showbox\boxA
      \showbox\boxB
    \fi
    \Expect*{\the\ht\boxC}*{\the\ht\boxA}%
    \Expect*{\the\dp\boxB}*{\the\dp\boxA}%
    \Expect*{\the\dp\boxC}*{\the\dp\boxA}%
  \end{qstest}%
}
%    \end{macrocode}
%    \end{macro}
%    \begin{macro}{\SetTestVBox}
%    \begin{macrocode}
\newcommand*{\SetTestVBox}[2]{%
  \setbox#1=\vbox{%
    \hsize=0pt\relax
    \hfuzz=100mm\relax
    \expandafter\testdriver\expandafter{#2}%
  }%
}
%    \end{macrocode}
%    \end{macro}
%    \begin{macro}{\TestInfo}
%    \begin{macrocode}
\newcommand*{\TestInfo}[2]{%
  \par
  \noindent
  \copy#1\relax
  \hfill
  \begingroup
    \small
    \ifdim\wd0=\wd2\relax
      \ifdim\wd0=\wd4\relax
      \else
        \ifnum#1=2 %
          \raisebox{.5\baselineskip}[0pt][0pt]{%
            \llap{$\}$ }%
            \rlap{width = \the\wd#1\relax\space$\{$}%
            \phantom{width = 999.99999pt}%
            \qquad
          }%
        \else
          \ifnum#1=4 %
            \rlap{width = \the\wd#1}%
            \phantom{width = 999.99999pt}%
            \qquad
          \fi
        \fi
      \fi
    \else
      \begingroup
        \bfseries
        \rlap{width = \the\wd#1}%
        \phantom{width = 999.99999pt}%
      \endgroup
      \qquad
    \fi
    \rlap{(#2)}%
    \phantom{(original soul)}%
  \endgroup
  \par
}
%    \end{macrocode}
%    \end{macro}
%    \begin{macro}{\TestInfoVBox}
%    \begin{macrocode}
\newcommand*{\TestInfoVBox}{%
  \par
  \noindent
  \mbox{\vrule\relax\space\copy\boxA}%
  \ifdim\ht\boxA=\ht\boxB
  \else
    \hfill
    \begingroup
      \small
      \rlap{(soulutf8)}%
      \phantom{(soulutf8)}%
    \endgroup
    \par
    \noindent
    \mbox{\vrule\relax\space\copy\boxB}%
    \hfill
    \begingroup
      \small
      \rlap{(soul)}%
      \phantom{(soulutf8)}%
    \endgroup
  \fi
  \par
}
%    \end{macrocode}
%    \end{macro}
%    \begin{macro}{\Test}
%    \begin{macrocode}
\makeatletter
\newcommand*{\Test}[3]{%
  \clearpage
  \begingroup
    \def\linebreak{\textbackslash linebreak }%
    \def\\{\textbackslash\textbackslash}%
    \def\soulomit##1{%
      \textbackslash soulomit\textbraceleft##1\textbraceright
    }%
    \def\mbox##1{%
      \textbackslash mbox\textbraceleft##1\textbraceright
    }%
    \def\abc{\textbackslash abc }%
    \def\texttt##1{%
      \textbackslash texttt\textbraceleft##1\textbraceright
    }%
    \def\c##1{%
      \textbackslash c\textbraceleft##1\textbraceright
    }%
%<*utf8x>
    \count@=128 %
    \loop
    \ifnum\count@<256 %
      \lccode`\~=\count@
      \lowercase{%
        \def~{\noexpand~}%
      }%
      \advance\count@\@ne
    \repeat
%</utf8x>
    \csname protected@edef\endcsname\x{\endgroup
      \noexpand\section{String #1: #2}%
    }%
  \x
  \TestDriver{so}{#1}{#2}{#3}%
  \TestDriver{caps}{#1}{#2}{#3}%
  \TestDriver{ul}{#1}{#2}{#3}%
  \TestDriver{st}{#1}{#2}{#3}%
  \TestDriver{hl}{#1}{#2}{#3}%
  \TestDriver{an}{#1}{#2}{#3}%
  \TestDriver{sy}{#1}{#2}{#3}%
  \TestDriver{soulH}{#1}{#2}{#3}%
}
\makeatother
%    \end{macrocode}
%    \end{macro}
%
% \subsubsection{Testing}
%
%    \begin{macrocode}
\begin{document}
  \Test A{}{}%
  \Test B{Hello World}{Hello World}%
  \Test C{%
    Umlauts: ^^c3^^84^^c3^^96^^c3^^9c^^c3^^a4^^c3^^b6^^c3^^bc^^c3^^9f%
  }{%
    Umlauts: \"A\"O\"U\"a\"o\"u\ss
  }%
  \Test D{Euro: ^^e2^^82^^ac}{Euro: \texteuro}%
  \Test E{Stra^^c3^^9fe}{Stra\ss e}%
  \Test F{R^^c3^^b6mer}{R\"omer}%
  \Test G{A^^e2^^80^^93B^^e2^^80^^94C}{A--B---C}%
  \Test H{X^^c2^^a0Y}{X~Y}%
  \Test I{A^^c5^^a4}{A\v T}%
  \Test J{A,B,,C,,,D,,,,E}{A,B,,C,,,D,,,,E}%
  \Test K{A'B''C'''D''''E}{A'B''C'''D''''E}%
  \Test L{A`B``C```D````E}{A`B``C```D````E}%
  \Test M{A$x$B}{A$x$B}%
  \Test N{A\mbox{abc}B}{A\mbox{abc}B}%
  \Test O{,,TA''}{,,TA''}%
  \Test P{{!`}Hola{?`}}{{!`}Hola{?`}}%
  \Test Q{re-sent\slash x}{re-sent\slash x}%
  \Test R{new\\line}{new\\line}%
  \Test S{foo\linebreak bar}{foo\linebreak bar}%
  \Test T{Virchow{sche}}{Virchow{sche}}%
  \Test U{A\soulomit{foo}B}{A\soulomit{foo}B}%
  \Test V{A\texttt{xyz}B}{A\texttt{xyz}B}%
  \Test W{ffi}{ffi}%
  \Test X{f{}f{}i}{f{}f{}i}%
  \Test Y{f\mbox{}f\mbox{}i}{f\mbox{}f\mbox{}i}%
  \Test Z{\c{\"a}}{{\c{\"a}}}%
%<*test2|test3>
  \let\abcx\UNDEFINED
  \def\abc{%
    \begingroup
      \def\abcx{x}%
      \abcx
    \endgroup
  }%
  \Test{AA}{[{x\abc x}]}{[{xxx}]}%
%</test2|test3>
%    \end{macrocode}
%
%    \begin{macrocode}
\end{document}
%    \end{macrocode}
%    \begin{macrocode}
%</test2|test3|test4|test5>
%    \end{macrocode}
%
% \section{Installation}
%
% \subsection{Download}
%
% \paragraph{Package.} This package is available on
% CTAN\footnote{\url{http://ctan.org/pkg/soulutf8}}:
% \begin{description}
% \item[\CTAN{macros/latex/contrib/oberdiek/soulutf8.dtx}] The source file.
% \item[\CTAN{macros/latex/contrib/oberdiek/soulutf8.pdf}] Documentation.
% \end{description}
%
%
% \paragraph{Bundle.} All the packages of the bundle `oberdiek'
% are also available in a TDS compliant ZIP archive. There
% the packages are already unpacked and the documentation files
% are generated. The files and directories obey the TDS standard.
% \begin{description}
% \item[\CTAN{install/macros/latex/contrib/oberdiek.tds.zip}]
% \end{description}
% \emph{TDS} refers to the standard ``A Directory Structure
% for \TeX\ Files'' (\CTAN{tds/tds.pdf}). Directories
% with \xfile{texmf} in their name are usually organized this way.
%
% \subsection{Bundle installation}
%
% \paragraph{Unpacking.} Unpack the \xfile{oberdiek.tds.zip} in the
% TDS tree (also known as \xfile{texmf} tree) of your choice.
% Example (linux):
% \begin{quote}
%   |unzip oberdiek.tds.zip -d ~/texmf|
% \end{quote}
%
% \paragraph{Script installation.}
% Check the directory \xfile{TDS:scripts/oberdiek/} for
% scripts that need further installation steps.
% Package \xpackage{attachfile2} comes with the Perl script
% \xfile{pdfatfi.pl} that should be installed in such a way
% that it can be called as \texttt{pdfatfi}.
% Example (linux):
% \begin{quote}
%   |chmod +x scripts/oberdiek/pdfatfi.pl|\\
%   |cp scripts/oberdiek/pdfatfi.pl /usr/local/bin/|
% \end{quote}
%
% \subsection{Package installation}
%
% \paragraph{Unpacking.} The \xfile{.dtx} file is a self-extracting
% \docstrip\ archive. The files are extracted by running the
% \xfile{.dtx} through \plainTeX:
% \begin{quote}
%   \verb|tex soulutf8.dtx|
% \end{quote}
%
% \paragraph{TDS.} Now the different files must be moved into
% the different directories in your installation TDS tree
% (also known as \xfile{texmf} tree):
% \begin{quote}
% \def\t{^^A
% \begin{tabular}{@{}>{\ttfamily}l@{ $\rightarrow$ }>{\ttfamily}l@{}}
%   soulutf8.sty & tex/generic/oberdiek/soulutf8.sty\\
%   soulutf8.pdf & doc/latex/oberdiek/soulutf8.pdf\\
%   test/soulutf8-test1.tex & doc/latex/oberdiek/test/soulutf8-test1.tex\\
%   test/soulutf8-test2.tex & doc/latex/oberdiek/test/soulutf8-test2.tex\\
%   test/soulutf8-test3.tex & doc/latex/oberdiek/test/soulutf8-test3.tex\\
%   test/soulutf8-test4.tex & doc/latex/oberdiek/test/soulutf8-test4.tex\\
%   test/soulutf8-test5.tex & doc/latex/oberdiek/test/soulutf8-test5.tex\\
%   soulutf8.dtx & source/latex/oberdiek/soulutf8.dtx\\
% \end{tabular}^^A
% }^^A
% \sbox0{\t}^^A
% \ifdim\wd0>\linewidth
%   \begingroup
%     \advance\linewidth by\leftmargin
%     \advance\linewidth by\rightmargin
%   \edef\x{\endgroup
%     \def\noexpand\lw{\the\linewidth}^^A
%   }\x
%   \def\lwbox{^^A
%     \leavevmode
%     \hbox to \linewidth{^^A
%       \kern-\leftmargin\relax
%       \hss
%       \usebox0
%       \hss
%       \kern-\rightmargin\relax
%     }^^A
%   }^^A
%   \ifdim\wd0>\lw
%     \sbox0{\small\t}^^A
%     \ifdim\wd0>\linewidth
%       \ifdim\wd0>\lw
%         \sbox0{\footnotesize\t}^^A
%         \ifdim\wd0>\linewidth
%           \ifdim\wd0>\lw
%             \sbox0{\scriptsize\t}^^A
%             \ifdim\wd0>\linewidth
%               \ifdim\wd0>\lw
%                 \sbox0{\tiny\t}^^A
%                 \ifdim\wd0>\linewidth
%                   \lwbox
%                 \else
%                   \usebox0
%                 \fi
%               \else
%                 \lwbox
%               \fi
%             \else
%               \usebox0
%             \fi
%           \else
%             \lwbox
%           \fi
%         \else
%           \usebox0
%         \fi
%       \else
%         \lwbox
%       \fi
%     \else
%       \usebox0
%     \fi
%   \else
%     \lwbox
%   \fi
% \else
%   \usebox0
% \fi
% \end{quote}
% If you have a \xfile{docstrip.cfg} that configures and enables \docstrip's
% TDS installing feature, then some files can already be in the right
% place, see the documentation of \docstrip.
%
% \subsection{Refresh file name databases}
%
% If your \TeX~distribution
% (\teTeX, \mikTeX, \dots) relies on file name databases, you must refresh
% these. For example, \teTeX\ users run \verb|texhash| or
% \verb|mktexlsr|.
%
% \subsection{Some details for the interested}
%
% \paragraph{Attached source.}
%
% The PDF documentation on CTAN also includes the
% \xfile{.dtx} source file. It can be extracted by
% AcrobatReader 6 or higher. Another option is \textsf{pdftk},
% e.g. unpack the file into the current directory:
% \begin{quote}
%   \verb|pdftk soulutf8.pdf unpack_files output .|
% \end{quote}
%
% \paragraph{Unpacking with \LaTeX.}
% The \xfile{.dtx} chooses its action depending on the format:
% \begin{description}
% \item[\plainTeX:] Run \docstrip\ and extract the files.
% \item[\LaTeX:] Generate the documentation.
% \end{description}
% If you insist on using \LaTeX\ for \docstrip\ (really,
% \docstrip\ does not need \LaTeX), then inform the autodetect routine
% about your intention:
% \begin{quote}
%   \verb|latex \let\install=y\input{soulutf8.dtx}|
% \end{quote}
% Do not forget to quote the argument according to the demands
% of your shell.
%
% \paragraph{Generating the documentation.}
% You can use both the \xfile{.dtx} or the \xfile{.drv} to generate
% the documentation. The process can be configured by the
% configuration file \xfile{ltxdoc.cfg}. For instance, put this
% line into this file, if you want to have A4 as paper format:
% \begin{quote}
%   \verb|\PassOptionsToClass{a4paper}{article}|
% \end{quote}
% An example follows how to generate the
% documentation with pdf\LaTeX:
% \begin{quote}
%\begin{verbatim}
%pdflatex soulutf8.dtx
%makeindex -s gind.ist soulutf8.idx
%pdflatex soulutf8.dtx
%makeindex -s gind.ist soulutf8.idx
%pdflatex soulutf8.dtx
%\end{verbatim}
% \end{quote}
%
% \begin{thebibliography}{9}
%
% \bibitem{soul}
%   Melchior Franz: \textit{The \xpackage{soul} package};
%   2003/11/17;\\
%   \CTAN{macros/latex/contrib/soul/}.
%
% \bibitem{ucs}
%   Dominique P. G. Unruh: \textit{\xpackage{ucs.sty} -- Unicode Support};
%   2004/10/17;\\
%   \CTAN{macros/latex/contrib/unicode/}.
%
% \bibitem{utf8}
%   Frank Mittelbach, Chris Rowley:
%   \textit{Providing some UTF-8 support via \xpackage{inputenc}};
%   2006/03/30;\\
%   \CTAN{macros/latex/base/utf8ienc.dtx}.
%
% \end{thebibliography}
%
% \begin{History}
%   \begin{Version}{2007/09/09 v1.0}
%   \item
%     First version.
%   \end{Version}
%   \begin{Version}{2016/05/16 v1.1}
%   \item
%     Documentation updates.
%   \end{Version}
% \end{History}
%
% \PrintIndex
%
% \Finale
\endinput

%        (quote the arguments according to the demands of your shell)
%
% Documentation:
%    (a) If soulutf8.drv is present:
%           latex soulutf8.drv
%    (b) Without soulutf8.drv:
%           latex soulutf8.dtx; ...
%    The class ltxdoc loads the configuration file ltxdoc.cfg
%    if available. Here you can specify further options, e.g.
%    use A4 as paper format:
%       \PassOptionsToClass{a4paper}{article}
%
%    Programm calls to get the documentation (example):
%       pdflatex soulutf8.dtx
%       makeindex -s gind.ist soulutf8.idx
%       pdflatex soulutf8.dtx
%       makeindex -s gind.ist soulutf8.idx
%       pdflatex soulutf8.dtx
%
% Installation:
%    TDS:tex/generic/oberdiek/soulutf8.sty
%    TDS:doc/latex/oberdiek/soulutf8.pdf
%    TDS:doc/latex/oberdiek/test/soulutf8-test1.tex
%    TDS:doc/latex/oberdiek/test/soulutf8-test2.tex
%    TDS:doc/latex/oberdiek/test/soulutf8-test3.tex
%    TDS:doc/latex/oberdiek/test/soulutf8-test4.tex
%    TDS:doc/latex/oberdiek/test/soulutf8-test5.tex
%    TDS:source/latex/oberdiek/soulutf8.dtx
%
%<*ignore>
\begingroup
  \catcode123=1 %
  \catcode125=2 %
  \def\x{LaTeX2e}%
\expandafter\endgroup
\ifcase 0\ifx\install y1\fi\expandafter
         \ifx\csname processbatchFile\endcsname\relax\else1\fi
         \ifx\fmtname\x\else 1\fi\relax
\else\csname fi\endcsname
%</ignore>
%<*install>
\input docstrip.tex
\Msg{************************************************************************}
\Msg{* Installation}
\Msg{* Package: soulutf8 2016/05/16 v1.1 Permit use of UTF-8 characters in soul (HO)}
\Msg{************************************************************************}

\keepsilent
\askforoverwritefalse

\let\MetaPrefix\relax
\preamble

This is a generated file.

Project: soulutf8
Version: 2016/05/16 v1.1

Copyright (C) 2007 by
   Heiko Oberdiek <heiko.oberdiek at googlemail.com>

This work may be distributed and/or modified under the
conditions of the LaTeX Project Public License, either
version 1.3c of this license or (at your option) any later
version. This version of this license is in
   http://www.latex-project.org/lppl/lppl-1-3c.txt
and the latest version of this license is in
   http://www.latex-project.org/lppl.txt
and version 1.3 or later is part of all distributions of
LaTeX version 2005/12/01 or later.

This work has the LPPL maintenance status "maintained".

This Current Maintainer of this work is Heiko Oberdiek.

The Base Interpreter refers to any `TeX-Format',
because some files are installed in TDS:tex/generic//.

This work consists of the main source file soulutf8.dtx
and the derived files
   soulutf8.sty, soulutf8.pdf, soulutf8.ins, soulutf8.drv,
   soulutf8-test1.tex, soulutf8-test2.tex, soulutf8-test3.tex,
   soulutf8-test4.tex, soulutf8-test5.tex.

\endpreamble
\let\MetaPrefix\DoubleperCent

\generate{%
  \file{soulutf8.ins}{\from{soulutf8.dtx}{install}}%
  \file{soulutf8.drv}{\from{soulutf8.dtx}{driver}}%
  \usedir{tex/generic/oberdiek}%
  \file{soulutf8.sty}{\from{soulutf8.dtx}{package}}%
%  \usedir{doc/latex/oberdiek/test}%
%  \file{soulutf8-test1.tex}{\from{soulutf8.dtx}{test1}}%
%  \file{soulutf8-test2.tex}{\from{soulutf8.dtx}{test2,utf8}}%
%  \file{soulutf8-test3.tex}{\from{soulutf8.dtx}{test3,utf8x}}%
%  \file{soulutf8-test4.tex}{\from{soulutf8.dtx}{test4,utf8,noetex}}%
%  \file{soulutf8-test5.tex}{\from{soulutf8.dtx}{test5,utf8x,noetex}}%
  \nopreamble
  \nopostamble
%  \usedir{source/latex/oberdiek/catalogue}%
%  \file{soulutf8.xml}{\from{soulutf8.dtx}{catalogue}}%
}

\catcode32=13\relax% active space
\let =\space%
\Msg{************************************************************************}
\Msg{*}
\Msg{* To finish the installation you have to move the following}
\Msg{* file into a directory searched by TeX:}
\Msg{*}
\Msg{*     soulutf8.sty}
\Msg{*}
\Msg{* To produce the documentation run the file `soulutf8.drv'}
\Msg{* through LaTeX.}
\Msg{*}
\Msg{* Happy TeXing!}
\Msg{*}
\Msg{************************************************************************}

\endbatchfile
%</install>
%<*ignore>
\fi
%</ignore>
%<*driver>
\NeedsTeXFormat{LaTeX2e}
\ProvidesFile{soulutf8.drv}%
  [2016/05/16 v1.1 Permit use of UTF-8 characters in soul (HO)]%
\documentclass{ltxdoc}
\usepackage{holtxdoc}[2011/11/22]
\begin{document}
  \DocInput{soulutf8.dtx}%
\end{document}
%</driver>
% \fi
%
%
% \CharacterTable
%  {Upper-case    \A\B\C\D\E\F\G\H\I\J\K\L\M\N\O\P\Q\R\S\T\U\V\W\X\Y\Z
%   Lower-case    \a\b\c\d\e\f\g\h\i\j\k\l\m\n\o\p\q\r\s\t\u\v\w\x\y\z
%   Digits        \0\1\2\3\4\5\6\7\8\9
%   Exclamation   \!     Double quote  \"     Hash (number) \#
%   Dollar        \$     Percent       \%     Ampersand     \&
%   Acute accent  \'     Left paren    \(     Right paren   \)
%   Asterisk      \*     Plus          \+     Comma         \,
%   Minus         \-     Point         \.     Solidus       \/
%   Colon         \:     Semicolon     \;     Less than     \<
%   Equals        \=     Greater than  \>     Question mark \?
%   Commercial at \@     Left bracket  \[     Backslash     \\
%   Right bracket \]     Circumflex    \^     Underscore    \_
%   Grave accent  \`     Left brace    \{     Vertical bar  \|
%   Right brace   \}     Tilde         \~}
%
% \GetFileInfo{soulutf8.drv}
%
% \title{The \xpackage{soulutf8} package}
% \date{2016/05/16 v1.1}
% \author{Heiko Oberdiek\thanks
% {Please report any issues at https://github.com/ho-tex/oberdiek/issues}\\
% \xemail{heiko.oberdiek at googlemail.com}}
%
% \maketitle
%
% \begin{abstract}
% This package extends package \xpackage{soul} and adds some support
% for UTF-8. Namely the input encodings \xfile{utf8.def}
% from package \xpackage{inputenc} and
% package \xpackage{ucs}'s \xfile{utf8x.def} are supported.
% \end{abstract}
%
% \tableofcontents
%
% \section{Documentation}
%
% This package \xpackage{soulutf8} does not have own options and does not
% define new user commands.
% Any option is passed to package \xpackage{soul} \cite{soul}
% that is loaded first.
% Then some internal macros of \xpackage{soul} are redefined to add
% support for UTF-8.
% The following input encodings are supported:
% \begin{quote}
%   \begin{tabular}{@{}lll@{}}
%     \texttt{utf8} & \LaTeX\ base &
%        \xfile{TDS:tex/latex/base/utf8.def} \cite{utf8}\\
%     \texttt{utf8x} & Package \xpackage{ucs} &
%       \xfile{TDS:tex/latex/ucs/utf8x.def} \cite{ucs}\\
%   \end{tabular}
% \end{quote}
% UTF-8 byte sequences are added as token group to a word, even
% if these UTF-8 characters are some kind of hyphen or space.
% As exception the following three Unicode characters are handled specially:
% \begin{quote}
%   \begin{tabular}{lll}
%     Slot & Name & Action\\
%     \hline
%     \texttt{U+00A0} & NO-BREAK SPACE & like |~|\\
%     \texttt{U+2013} & EN DASH & |--|\\
%     \texttt{U+2014} & EM DASH & |---|
%   \end{tabular}
%
% \subsection{Patch}
%
% Also package \xpackage{soulutf8} tries to patch package \xpackage{soul}
% to improve its behaviour:
% \begin{itemize}
% \item
%   A problem with additional levels of curly braces is fixed.
%   As advantage more implicite kernings are detected.
%   However, the result may be incompatible with the
%   original behaviour of package \xpackage{soul} because
%   of these respected implicite kernings.
% \item
%   \eTeX\ , especially \cs{unexpanded} is supported.
%   This allows a better protection of token groups
%   (|\mbox{|\dots|}|, math, \dots).
% \end{itemize}
% \end{quote}
%
% \subsection{Future}
%
% Currently package \xpackage{soul} does not seem to be
% maintained. Nevertheless if there will be a new version that
% adds support for UTF-8, then this package may become
% obsolete.
%
% \StopEventually{
% }
%
% \section{Implementation}
%
%    \begin{macrocode}
%<*package>
%    \end{macrocode}
%
% \subsection{Reload check and package identification}
%    Reload check, especially if the package is not used with \LaTeX.
%    \begin{macrocode}
\begingroup\catcode61\catcode48\catcode32=10\relax%
  \catcode13=5 % ^^M
  \endlinechar=13 %
  \catcode35=6 % #
  \catcode39=12 % '
  \catcode44=12 % ,
  \catcode45=12 % -
  \catcode46=12 % .
  \catcode58=12 % :
  \catcode64=11 % @
  \catcode123=1 % {
  \catcode125=2 % }
  \expandafter\let\expandafter\x\csname ver@soulutf8.sty\endcsname
  \ifx\x\relax % plain-TeX, first loading
  \else
    \def\empty{}%
    \ifx\x\empty % LaTeX, first loading,
      % variable is initialized, but \ProvidesPackage not yet seen
    \else
      \expandafter\ifx\csname PackageInfo\endcsname\relax
        \def\x#1#2{%
          \immediate\write-1{Package #1 Info: #2.}%
        }%
      \else
        \def\x#1#2{\PackageInfo{#1}{#2, stopped}}%
      \fi
      \x{soulutf8}{The package is already loaded}%
      \aftergroup\endinput
    \fi
  \fi
\endgroup%
%    \end{macrocode}
%    Package identification:
%    \begin{macrocode}
\begingroup\catcode61\catcode48\catcode32=10\relax%
  \catcode13=5 % ^^M
  \endlinechar=13 %
  \catcode35=6 % #
  \catcode39=12 % '
  \catcode40=12 % (
  \catcode41=12 % )
  \catcode44=12 % ,
  \catcode45=12 % -
  \catcode46=12 % .
  \catcode47=12 % /
  \catcode58=12 % :
  \catcode64=11 % @
  \catcode91=12 % [
  \catcode93=12 % ]
  \catcode123=1 % {
  \catcode125=2 % }
  \expandafter\ifx\csname ProvidesPackage\endcsname\relax
    \def\x#1#2#3[#4]{\endgroup
      \immediate\write-1{Package: #3 #4}%
      \xdef#1{#4}%
    }%
  \else
    \def\x#1#2[#3]{\endgroup
      #2[{#3}]%
      \ifx#1\@undefined
        \xdef#1{#3}%
      \fi
      \ifx#1\relax
        \xdef#1{#3}%
      \fi
    }%
  \fi
\expandafter\x\csname ver@soulutf8.sty\endcsname
\ProvidesPackage{soulutf8}%
  [2016/05/16 v1.1 Permit use of UTF-8 characters in soul (HO)]%
%    \end{macrocode}
%
% \subsection{Catcodes}
%
%    \begin{macrocode}
\begingroup\catcode61\catcode48\catcode32=10\relax%
  \catcode13=5 % ^^M
  \endlinechar=13 %
  \catcode123=1 % {
  \catcode125=2 % }
  \catcode64=11 % @
  \def\x{\endgroup
    \expandafter\edef\csname SOuL@AtEnd\endcsname{%
      \endlinechar=\the\endlinechar\relax
      \catcode13=\the\catcode13\relax
      \catcode32=\the\catcode32\relax
      \catcode35=\the\catcode35\relax
      \catcode61=\the\catcode61\relax
      \catcode64=\the\catcode64\relax
      \catcode123=\the\catcode123\relax
      \catcode125=\the\catcode125\relax
    }%
  }%
\x\catcode61\catcode48\catcode32=10\relax%
\catcode13=5 % ^^M
\endlinechar=13 %
\catcode35=6 % #
\catcode64=11 % @
\catcode123=1 % {
\catcode125=2 % }
\def\TMP@EnsureCode#1#2{%
  \edef\SOuL@AtEnd{%
    \SOuL@AtEnd
    \catcode#1=\the\catcode#1\relax
  }%
  \catcode#1=#2\relax
}
\TMP@EnsureCode{10}{12}% ^^J
\TMP@EnsureCode{33}{12}% !
\TMP@EnsureCode{34}{12}% "
\TMP@EnsureCode{36}{3}% $
\TMP@EnsureCode{39}{12}% '
\TMP@EnsureCode{40}{12}% (
\TMP@EnsureCode{41}{12}% )
\TMP@EnsureCode{42}{12}% *
\TMP@EnsureCode{43}{12}% +
\TMP@EnsureCode{44}{12}% ,
\TMP@EnsureCode{45}{12}% -
\TMP@EnsureCode{46}{12}% .
\TMP@EnsureCode{47}{12}% /
\TMP@EnsureCode{58}{12}% :
\TMP@EnsureCode{60}{12}% <
\TMP@EnsureCode{62}{12}% >
\TMP@EnsureCode{91}{12}% [
\TMP@EnsureCode{93}{12}% ]
\TMP@EnsureCode{94}{7}% ^
\TMP@EnsureCode{96}{12}% `
\TMP@EnsureCode{126}\active % ~
\TMP@EnsureCode{128}{12}% ^^80
\TMP@EnsureCode{147}{12}% ^^93
\TMP@EnsureCode{148}{12}% ^^94
\TMP@EnsureCode{160}{12}% ^^a0
\TMP@EnsureCode{194}{12}% ^^c2
\TMP@EnsureCode{226}{12}% ^^e2
\edef\SOuL@AtEnd{\SOuL@AtEnd\noexpand\endinput}
%    \end{macrocode}
%
% \subsection{Loading packages}
%
%    Package \xpackage{soul} uses \cs{documentclass} to detect \LaTeX.
%    \begin{macrocode}
\ifx\documentclass\@undefined
%    \end{macrocode}
%
% \subsubsection{\plainTeX}
%
%    First we check, whether package \xpackage{soul} is already loaded.
%    \begin{macrocode}
  \expandafter\ifx\csname SOUL@\endcsname\relax
%    \end{macrocode}
%    In case of \plainTeX\ package \xpackage{soul} defines some
%    macros in a simple manner that will break the definitions of
%    \xfile{miniltx.tex}, for example. Therefore these macros
%    are first saved and restored afterwards.
%    \begin{macrocode}
    \let\SOuL@orgDeclareRobustCommand\DeclareRobustCommand
    \let\SOuL@orgnewcommand          \newcommand
    \let\SOuL@orgDeclareOption       \DeclareOption
    \let\SOuL@orgPackageError        \PackageError
    \def\SOuL@restorelatexcmds{%
      \let\DeclareRobustCommand\SOuL@orgDeclareRobustCommand
      \let\newcommand          \SOuL@orgnewcommand
      \let\DeclareOption       \SOuL@orgDeclareOption
      \let\PackageError        \SOuL@orgPackageError
    }%
    \input soul.sty\relax
    \SOuL@restorelatexcmds
  \fi
%    \end{macrocode}
%    \begin{macro}{\SOUL@error}
%     Package \xpackage{soul}'s use of \cs{PackageError} is replaced
%     by \cs{@PackageError} of package \xpackage{infwarerr}.
%    \begin{macrocode}
  \input infwarerr.sty\relax
  \let\SOuL@orgSOUL@error\SOUL@error
  \def\SOUL@error{%
    \begingroup
      \let\PackageError\@PackageError
      \SOuL@orgSOUL@error
    \endgroup
  }%
%    \end{macrocode}
%    \end{macro}
%    \begin{macrocode}
  \input etexcmds.sty\relax
%    \end{macrocode}
%    \begin{macro}{\@onelevel@sanitize}
%    Define \LaTeX's \cs{@onelevel@sanitize} if not already available.
%    \begin{macrocode}
  \expandafter\ifx\csname @onelevel@sanitize\endcsname\relax
    \def\@onelevel@sanitize#1{%
      \edef#1{%
        \expandafter\strip@prefix\meaning#1%
      }%
    }%
%    \end{macrocode}
%    \end{macro}
%    \begin{macro}{\strip@prefix}
%    \begin{macrocode}
    \def\strip@prefix#1>{}%
%    \end{macrocode}
%    \end{macro}
%    \begin{macrocode}
  \fi
\else
%    \end{macrocode}
%
% \subsubsection{\LaTeX}
%    \begin{macrocode}
  \DeclareOption*{\PassOptionsToPackage{\CurrentOption}{soul}}%
  \ProcessOptions\relax
  \RequirePackage{soul}[2003/11/17]%
  \RequirePackage{infwarerr}[2016/05/16]%
  \RequirePackage{etexcmds}[2016/05/16]%
\fi
%    \end{macrocode}
%
% \subsubsection{\eTeX}
%
%    In \plainTeX\ command \cs{+} is an \emph{outer} macro.
%    Therefore numbers are used to avoid problems.
%    \begin{macrocode}
\ifetex@unexpanded
  \catcode33=14 % '!': comment
  \catcode43=9  % '+': ignore
\else
  \catcode33=9  % '!': ignore
  \catcode43=14 % '+': comment
\fi
%    \end{macrocode}
%
% \subsection{Macro for redefinitions}
%
%    \begin{macro}{\SOuL@redefine}
%    \begin{macrocode}
\def\SOuL@redefine#1{%
  \begingroup
    \def\SOuL@cmd{#1}%
    \afterassignment\SOuL@cmdcheck
    \def\SOuL@temp
}
%    \end{macrocode}
%    \end{macro}
%    \begin{macro}{\SOuL@cmdcheck}
%    \begin{macrocode}
\def\SOuL@cmdcheck{%
    \expandafter\ifx\SOuL@cmd\SOuL@temp
    \else
      \edef\SOuL@temp*{\expandafter\string\SOuL@cmd}%
      \@PackageWarningNoLine{soulutf8}{%
        Command \SOuL@temp* has changed.\MessageBreak
        Supported versions of package `soul': 2003/11/17.\MessageBreak
        Depending on the unknown changes the redefinition\MessageBreak
        of \SOuL@temp* may not behave correctly%
      }%
    \fi
  \expandafter\endgroup
  \expandafter\def\SOuL@cmd
}
%    \end{macrocode}
%    \end{macro}
%
% \subsection{Redefinition of \cs{SOUL@eval}}
%
%    \begin{macro}{\SOUL@eval}
%    Macro \cs{SOUL@eval} is redefined to add detection of
%    the first byte of a UTF-8 sequence. Because \cs{SOUL@eval}
%    is overwritten, a warning is issued, if the contents of
%    \cs{SOUL@eval} is not as expected.
%    \begin{macrocode}
\SOuL@redefine\SOUL@eval{%
%    \end{macrocode}
%    First the expected definition.
%    \begin{macrocode}
  \def\SOUL@n*##1{\SOUL@scan}%
  \if\noexpand\SOUL@@\SOUL@spc
  \else
    \SOUL@ignorespacesfalse
  \fi
  \ifnum\SOUL@minus=\thr@@
    \SOUL@flushminus
  \else\ifnum\SOUL@comma=\tw@
    \SOUL@flushcomma
  \else\ifnum\SOUL@apo=\tw@
    \SOUL@flushapo
  \else\ifnum\SOUL@grave=\tw@
    \SOUL@flushgrave
  \fi\fi\fi\fi
  \ifx\SOUL@@-\else\SOUL@flushminus\fi
  \ifx\SOUL@@,\else\SOUL@flushcomma\fi
  \ifx\SOUL@@'\else\SOUL@flushapo\fi
  \ifx\SOUL@@`\else\SOUL@flushgrave\fi
  \ifx\SOUL@@-%
    \advance\SOUL@minus\@ne
  \else\ifx\SOUL@@,%
    \advance\SOUL@comma\@ne
  \else\ifx\SOUL@@'%
    \advance\SOUL@apo\@ne
  \else\ifx\SOUL@@`%
    \advance\SOUL@grave\@ne
  \else
    \SOUL@flushminus
    \SOUL@flushcomma
    \SOUL@flushapo
    \SOUL@flushgrave
    \ifx\SOUL@@\SOUL@stop
      \def\SOUL@n*{%
        \SOUL@doword
        \SOUL@eventuallyexhyphen\null
      }%
    \else\ifx\SOUL@@\par
      \def\SOUL@n*\par{\par\leavevmode\SOUL@scan}%
    \else\if\noexpand\SOUL@@\SOUL@spc
      \SOUL@doword
      \SOUL@eventuallyexhyphen\null
      \ifSOUL@ignorespaces
      \else
        \SOUL@everyspace{}%
      \fi
      \def\SOUL@n* {\SOUL@scan}%
    \else\ifx\SOUL@@\\%
      \SOUL@doword
      \SOUL@eventuallyexhyphen\null
      \SOUL@everyspace{\unskip\nobreak\hfil\break}%
      \SOUL@ignorespacestrue
    \else\ifx\SOUL@@~%
      \SOUL@doword
      \SOUL@eventuallyexhyphen\null
      \SOUL@everyspace{\nobreak}%
    \else\ifx\SOUL@@\slash
      \SOUL@doword
      \SOUL@eventuallyexhyphen{/}%
      \SOUL@exhyphen{/}%
    \else\ifx\SOUL@@\mbox
      \def\SOUL@n*{\SOUL@addprotect}%
    \else\ifx\SOUL@@\hbox
      \def\SOUL@n*{\SOUL@addprotect}%
    \else\ifx\SOUL@@\soulomit
      \def\SOUL@n*\soulomit##1{%
        \SOUL@doword
        {\spaceskip\SOUL@spaceskip##1}%
        \SOUL@scan
      }%
    \else\ifx\SOUL@@\break
      \SOUL@doword
      \break
    \else\ifx\SOUL@@\linebreak
      \SOUL@doword
      \SOUL@everyspace{\linebreak}%
    \else\ifcat\bgroup\noexpand\SOUL@@
      \def\SOUL@n*{\SOUL@addgroup{}}%
    \else\ifcat$\noexpand\SOUL@@
      \def\SOUL@n*{\SOUL@addmath}%
    \else
      \def\SOUL@n*{\SOUL@dotoken}%
    \fi\fi\fi\fi\fi\fi\fi\fi\fi\fi\fi\fi\fi
  \fi\fi\fi\fi
  \SOUL@n*%
}{%
%    \end{macrocode}
%    Now the redefined version follows.
%    \begin{macrocode}
  \def\SOUL@n*##1{\SOUL@scan}%
  \if\noexpand\SOUL@@\SOUL@spc
  \else
    \SOUL@ignorespacesfalse
  \fi
  \ifnum\SOUL@minus=\thr@@
    \SOUL@flushminus
  \else\ifnum\SOUL@comma=\tw@
    \SOUL@flushcomma
  \else\ifnum\SOUL@apo=\tw@
    \SOUL@flushapo
  \else\ifnum\SOUL@grave=\tw@
    \SOUL@flushgrave
  \fi\fi\fi\fi
  \ifx\SOUL@@-\else\SOUL@flushminus\fi
  \ifx\SOUL@@,\else\SOUL@flushcomma\fi
  \ifx\SOUL@@'\else\SOUL@flushapo\fi
  \ifx\SOUL@@`\else\SOUL@flushgrave\fi
  \ifx\SOUL@@-%
    \advance\SOUL@minus\@ne
  \else\ifx\SOUL@@,%
    \advance\SOUL@comma\@ne
  \else\ifx\SOUL@@'%
    \advance\SOUL@apo\@ne
  \else\ifx\SOUL@@`%
    \advance\SOUL@grave\@ne
  \else
    \SOUL@flushminus
    \SOUL@flushcomma
    \SOUL@flushapo
    \SOUL@flushgrave
    \ifx\SOUL@@\SOUL@stop
      \def\SOUL@n*{%
          \SOUL@doword
          \SOUL@eventuallyexhyphen\null
      }%
    \else\ifx\SOUL@@\par
      \def\SOUL@n*\par{\par\leavevmode\SOUL@scan}%
    \else\if\noexpand\SOUL@@\SOUL@spc
      \SOUL@doword
      \SOUL@eventuallyexhyphen\null
      \ifSOUL@ignorespaces
      \else
        \SOUL@everyspace{}%
      \fi
      \def\SOUL@n* {\SOUL@scan}%
    \else\ifx\SOUL@@\\%
      \SOUL@doword
      \SOUL@eventuallyexhyphen\null
      \SOUL@everyspace{\unskip\nobreak\hfil\break}%
      \SOUL@ignorespacestrue
    \else\ifx\SOUL@@~%
      \SOUL@doword
      \SOUL@eventuallyexhyphen\null
      \SOUL@everyspace{\nobreak}%
    \else\ifx\SOUL@@\slash
      \SOUL@doword
      \SOUL@eventuallyexhyphen{/}%
      \SOUL@exhyphen{/}%
    \else\ifx\SOUL@@\mbox
      \def\SOUL@n*{\SOUL@addprotect}%
    \else\ifx\SOUL@@\hbox
      \def\SOUL@n*{\SOUL@addprotect}%
    \else\ifx\SOUL@@\soulomit
      \def\SOUL@n*\soulomit##1{%
        \SOUL@doword
        {\spaceskip\SOUL@spaceskip##1}%
        \SOUL@scan
      }%
    \else\ifx\SOUL@@\break
      \SOUL@doword
      \break
    \else\ifx\SOUL@@\linebreak
      \SOUL@doword
      \SOUL@everyspace{\linebreak}%
    \else\ifcat\bgroup\noexpand\SOUL@@
      \def\SOUL@n*{\SOUL@addgroup{}}%
    \else\ifcat$\noexpand\SOUL@@
      \def\SOUL@n*{\SOUL@addmath}%
    \else
%    \end{macrocode}
%    The current token is examined to detect the start of
%    a UTF-8 sequence.
%    \begin{macrocode}
      \SOuL@analyzeutfviii
      \ifcase\SOuL@octets
        \SOuL@analyzeutfviiix
      \fi
      \ifcase\SOuL@octets
        \def\SOUL@n*{\SOUL@dotoken}%
      \or % 1
      \or % 2
        \def\SOUL@n*{\SOuL@addtwooctets}%
      \or % 3
        \def\SOUL@n*{\SOuL@addthreeoctets}%
      \or % 4
        \def\SOUL@n*{\SOuL@addfouroctets}%
      \fi
    \fi\fi\fi\fi\fi\fi\fi\fi\fi\fi\fi\fi\fi
  \fi\fi\fi\fi
  \SOUL@n*%
}
%    \end{macrocode}
%    \end{macro}
%
% \subsection{UTF-8 analysis}
%
% \subsubsection{Help strings}
%
%    \begin{macrocode}
\def\SOuL@defsanitizedstring#1#2{%
  \expandafter\def\csname SOuL@string#1\endcsname{#2}%
  \expandafter\@onelevel@sanitize\csname SOuL@string#1\endcsname
}
\SOuL@defsanitizedstring{UTFviii}{UTFviii@}
\SOuL@defsanitizedstring{octets}{@octets}
\SOuL@defsanitizedstring{two}{two}
\SOuL@defsanitizedstring{three}{three}
\SOuL@defsanitizedstring{four}{four}
\SOuL@defsanitizedstring{macrocolon}{macro:}
\SOuL@defsanitizedstring{csnameu}{csname u8-}
\SOuL@defsanitizedstring{undeferr}{utf@viii@undeferr}
\def\SOuL@stringendash{^^e2^^80^^93}
\def\SOuL@stringemdash{^^e2^^80^^94}
\def\SOuL@stringnobreakspace{^^c2^^a0}
\edef\SOuL@charhash{\string #}
\edef\SOuL@chartwo{\string 2}
\edef\SOuL@charthree{\string 3}
\def\SOuL@empty{}
%    \end{macrocode}
%
% \subsubsection{Support for \xfile{utf8.def}}
%
%    \begin{macro}{\SOuL@analyzeutfviii}
%    \begin{macrocode}
\begingroup
  \edef\x{\endgroup
    \def\noexpand\SOuL@analyzeutfviii{%
      \noexpand\expandafter\noexpand\SOuL@checkutfviii
      \noexpand\meaning\noexpand\SOUL@@
      \SOuL@stringUTFviii\SOuL@stringoctets
      \noexpand\@nil
    }%
    \def\noexpand\SOuL@checkutfviii
      ##1\SOuL@stringUTFviii##2\SOuL@stringoctets##3\noexpand\@nil
  }%
\x{%
  \def\SOuL@temp{#2}%
  \chardef\SOuL@octets=%
      \ifx\SOuL@temp\SOuL@stringtwo
        \tw@
      \else\ifx\SOuL@temp\SOuL@stringthree
        \thr@@
      \else\ifx\SOuL@temp\SOuL@stringfour
        4 %
      \else
        \z@
      \fi\fi\fi
}
%    \end{macrocode}
%    \end{macro}
%
% \subsubsection{Support for \xfile{utf8x.def}}
%
%    \begin{macro}{\SOuL@analyzeutfviiix}
%    \begin{macrocode}
\begingroup
  \edef\x{\endgroup
    \def\noexpand\SOuL@analyzeutfviiix{%
      \noexpand\expandafter\noexpand\SOuL@checkutfviiix
      \noexpand\meaning\noexpand\SOUL@@
      \SOuL@stringmacrocolon\SOuL@charhash1{}{}{}{}%
      \SOuL@stringcsnameu\SOuL@stringundeferr
      \noexpand\@nil
    }%
%    \end{macrocode}
%    \end{macro}
%    \begin{macro}{\SOuL@checkutfviiix}
%    \begin{macrocode}
    \def\noexpand\SOuL@checkutfviiix
      ##1\SOuL@stringmacrocolon\SOuL@charhash1##2##3##4##5##6%
      \SOuL@stringcsnameu##7\SOuL@stringundeferr##8\noexpand\@nil
  }%
\x{%
  \def\SOuL@temp{#7}%
  \ifx\SOuL@temp\SOuL@empty
    \chardef\SOuL@octets=\z@
  \else
    \def\SOuL@temp{#5}%
    \ifx\SOuL@temp\SOuL@charthree
      \chardef\SOuL@octets=4 %
    \else
      \def\SOuL@temp{#3}%
      \ifx\SOuL@temp\SOuL@chartwo
        \chardef\SOuL@octets=\thr@@
      \else
        \chardef\SOuL@octets=\tw@
      \fi
    \fi
  \fi
}
%    \end{macrocode}
%    \end{macro}
%
% \subsection{Actions for UTF-8 sequences}
%
%    \begin{macro}{\SOuL@addtwooctets}
%    \begin{macrocode}
\def\SOuL@addtwooctets#1#2{%
  \def\SOuL@temp{#1#2}%
  \@onelevel@sanitize\SOuL@temp
  \ifx\SOuL@temp\SOuL@stringnobreakspace
    \SOUL@doword
    \SOUL@eventuallyexhyphen\null
    \SOUL@everyspace{\nobreak}%
    \let\SOuL@next\SOUL@scan
  \else
    \def\SOuL@next{%
!     \SOUL@addtoken{{\noexpand#1\noexpand#2}}%
+     \SOUL@addtoken{{\etex@unexpanded{#1#2}}}%
    }%
  \fi
  \SOuL@next
}
%    \end{macrocode}
%    \end{macro}
%    \begin{macro}{\SOuL@addthreeoctets}
%    \begin{macrocode}
\def\SOuL@addthreeoctets#1#2#3{%
  \def\SOuL@temp{#1#2#3}%
  \@onelevel@sanitize\SOuL@temp
  \ifx\SOuL@temp\SOuL@stringendash
    \SOUL@doword
    \SOUL@eventuallyexhyphen{-}%
    \SOUL@exhyphen{--}%
    \let\SOuL@next\SOUL@scan
  \else
    \ifx\SOuL@temp\SOuL@stringemdash
      \SOUL@doword
      \SOUL@eventuallyexhyphen{-}%
      \SOUL@exhyphen{---}%
      \let\SOuL@next\SOUL@scan
    \else
      \def\SOuL@next{%
!       \SOUL@addtoken{{\noexpand#1\noexpand#2\noexpand#3}}%
+       \SOUL@addtoken{{\etex@unexpanded{#1#2#3}}}%
      }%
    \fi
  \fi
  \SOuL@next
}
%    \end{macrocode}
%    \end{macro}
%    \begin{macro}{\SOuL@addfouroctets}
%    \begin{macrocode}
\def\SOuL@addfouroctets#1#2#3#4{%
! \SOUL@addtoken{{\noexpand#1\noexpand#2\noexpand#3\noexpand#4}}%
+ \SOUL@addtoken{{\etex@unexpanded{#1#2#3#4}}}%
}
%    \end{macrocode}
%    \end{macro}
%
% \subsubsection{Redefinition of \cs{SOUL@splittoken}}
%
%    \begin{macro}{\SOUL@splittoken}
%    Macro \cs{SOUL@splittoken} separates the first token
%    or token group from a word and redefines the word to contain
%    the remaining tokens. However if the remaining tokens are
%    a token group, then the curly braces will be removed and
%    the token group is splitted by the next call of \cs{SOUL@splittoken}.
%    The redefinition avoids the removal of curly braces around the
%    remaining tokens.
%    \begin{macrocode}
\SOuL@redefine\SOUL@splittoken#1#2\SOUL@stop{%
  \global\SOUL@token={#1}%
  \global\SOUL@word={#2}%
}#1{%
  \global\SOUL@token={#1}%
  \SOuL@remainingtoken\relax
}
%    \end{macrocode}
%    \end{macro}
%    \begin{macro}{\SOuL@remainingtoken}
%    \begin{macrocode}
\def\SOuL@remainingtoken#1\SOUL@stop{%
  \global\SOUL@word=\expandafter{\@gobble#1}%
}
%    \end{macrocode}
%    \end{macro}
%
% \subsection{Patches}
%
%    The fixed \cs{SOUL@splittoken} allows to remove the double
%    sets of curly braces in other macros of package \xpackage{soul}.
%    The benefit is that implicite kernings are more often detected
%    and fixes a bug in package \xpackage{soul}. The disadvantage is
%    incompatibility. The width of the resulting strings may change.
%    \begin{macro}{\SOUL@flushcomma}
%    \begin{macrocode}
\SOuL@redefine\SOUL@flushcomma{%
  \ifcase\SOUL@comma
  \or
    \edef\x{\SOUL@word={\the\SOUL@word,}}\x
  \or
    \edef\x{\SOUL@word={\the\SOUL@word{{,,}}}}\x
  \fi
  \SOUL@comma\z@
}{%
  \ifcase\SOUL@comma
  \or
    \edef\x{\SOUL@word={\the\SOUL@word,}}\x
  \or
    \edef\x{\SOUL@word={\the\SOUL@word{,,}}}\x
  \fi
  \SOUL@comma\z@
}
%    \end{macrocode}
%    \end{macro}
%    \begin{macro}{\SOUL@flushapo}
%    \begin{macrocode}
\SOuL@redefine\SOUL@flushapo{%
  \ifcase\SOUL@apo
  \or
    \edef\x{\SOUL@word={\the\SOUL@word'}}\x
  \or
    \edef\x{\SOUL@word={\the\SOUL@word{{''}}}}\x
  \fi
  \SOUL@apo\z@
}{%
  \ifcase\SOUL@apo
  \or
    \edef\x{\SOUL@word={\the\SOUL@word'}}\x
  \or
    \edef\x{\SOUL@word={\the\SOUL@word{''}}}\x
  \fi
  \SOUL@apo\z@
}
%    \end{macrocode}
%    \end{macro}
%    \begin{macro}{\SOUL@flushgrave}
%    \begin{macrocode}
\SOuL@redefine\SOUL@flushgrave{%
  \ifcase\SOUL@grave
  \or
    \edef\x{\SOUL@word={\the\SOUL@word`}}\x
  \or
    \edef\x{\SOUL@word={\the\SOUL@word{{``}}}}\x
  \fi
  \SOUL@grave\z@
}{%
  \ifcase\SOUL@grave
  \or
    \edef\x{\SOUL@word={\the\SOUL@word`}}\x
  \or
    \edef\x{\SOUL@word={\the\SOUL@word{``}}}\x
  \fi
  \SOUL@grave\z@
}
%    \end{macrocode}
%    \end{macro}
%    \begin{macro}{\SOUL@addgroup}
%    \begin{macrocode}
\SOuL@redefine\SOUL@addgroup#1#2{%
  {%
    \let\protect\noexpand
    \edef\x{%
      \global\SOUL@word={%
        \the\SOUL@word
        {{\noexpand#1#2}}%
      }%
    }%
    \x
  }%
  \SOUL@scan
}#1#2{%
  \begingroup
    \let\protect\noexpand
    \edef\x{\endgroup
      \SOUL@word={%
        \the\SOUL@word
!       {\noexpand#1{#2}}%
+       {\etex@unexpanded{#1{#2}}}%
      }%
    }%
  \x
  \SOUL@scan
}
%    \end{macrocode}
%    \end{macro}
%    \begin{macro}{\SOUL@addmath}
%    \begin{macrocode}
\SOuL@redefine\SOUL@addmath$#1${%
  {%
    \let\protect\noexpand
    \edef\x{%
      \global\SOUL@word={%
        \the\SOUL@word
        {{\hbox{$#1$}}}%
      }%
    }%
    \x
  }%
  \SOUL@scan
}$#1${%
  \begingroup
    \let\protect\noexpand
    \edef\x{\endgroup
      \SOUL@word={%
        \the\SOUL@word
!       {\hbox{$#1$}}%
+       {\etex@unexpanded{\hbox{$#1$}}}%
      }%
    }%
  \x
  \SOUL@scan
}
%    \end{macrocode}
%    \end{macro}
%    \begin{macro}{\SOUL@addprotect}
%    \begin{macrocode}
\SOuL@redefine\SOUL@addprotect#1#2{%
  {%
    \let\protect\noexpand
    \edef\x{%
      \global\SOUL@word={%
        \the\SOUL@word
        {{\hbox{#2}}}%
      }%
    }%
    \x
  }%
  \SOUL@scan
}#1#2{%
  \begingroup
    \let\protect\noexpand
    \edef\x{\endgroup
      \SOUL@word={%
        \the\SOUL@word
!       {\hbox{#2}}%
+       {\etex@unexpanded{\hbox{#2}}}%
      }%
    }%
  \x
  \SOUL@scan
}
%    \end{macrocode}
%    \end{macro}
%
%    \begin{macro}{\SOUL@addtoken}
%    \begin{macrocode}
+ \SOuL@redefine\SOUL@addtoken#1{%
+   \edef\x{%
+     \SOUL@word={%
+       \the\SOUL@word
+       \noexpand#1%
+     }%
+   }%
+   \x
+   \SOUL@scan
+ }#1{%
+   \edef\x{%
+     \SOUL@word={%
+       \the\SOUL@word
+       \etex@unexpanded{#1}%
+     }%
+   }%
+   \x
+   \SOUL@scan
+ }%
%    \end{macrocode}
%    \end{macro}
%
%    \begin{macrocode}
\SOuL@AtEnd%
%    \end{macrocode}
%
%    \begin{macrocode}
%</package>
%    \end{macrocode}
%
% \section{Test}
%
% \subsection{Catcode checks for loading}
%
%    \begin{macrocode}
%<*test1>
%    \end{macrocode}
%    \begin{macrocode}
\catcode`\{=1 %
\catcode`\}=2 %
\catcode`\#=6 %
\catcode`\@=11 %
\expandafter\ifx\csname count@\endcsname\relax
  \countdef\count@=255 %
\fi
\expandafter\ifx\csname @gobble\endcsname\relax
  \long\def\@gobble#1{}%
\fi
\expandafter\ifx\csname @firstofone\endcsname\relax
  \long\def\@firstofone#1{#1}%
\fi
\expandafter\ifx\csname loop\endcsname\relax
  \expandafter\@firstofone
\else
  \expandafter\@gobble
\fi
{%
  \def\loop#1\repeat{%
    \def\body{#1}%
    \iterate
  }%
  \def\iterate{%
    \body
      \let\next\iterate
    \else
      \let\next\relax
    \fi
    \next
  }%
  \let\repeat=\fi
}%
\def\RestoreCatcodes{}
\count@=0 %
\loop
  \edef\RestoreCatcodes{%
    \RestoreCatcodes
    \catcode\the\count@=\the\catcode\count@\relax
  }%
\ifnum\count@<255 %
  \advance\count@ 1 %
\repeat

\def\RangeCatcodeInvalid#1#2{%
  \count@=#1\relax
  \loop
    \catcode\count@=15 %
  \ifnum\count@<#2\relax
    \advance\count@ 1 %
  \repeat
}
\def\RangeCatcodeCheck#1#2#3{%
  \count@=#1\relax
  \loop
    \ifnum#3=\catcode\count@
    \else
      \errmessage{%
        Character \the\count@\space
        with wrong catcode \the\catcode\count@\space
        instead of \number#3%
      }%
    \fi
  \ifnum\count@<#2\relax
    \advance\count@ 1 %
  \repeat
}
\def\space{ }
\expandafter\ifx\csname LoadCommand\endcsname\relax
  \def\LoadCommand{\input soulutf8.sty\relax}%
\fi
\def\Test{%
  \RangeCatcodeInvalid{0}{47}%
  \RangeCatcodeInvalid{58}{64}%
  \RangeCatcodeInvalid{91}{96}%
  \RangeCatcodeInvalid{123}{255}%
  \catcode`\@=12 %
  \catcode`\\=0 %
  \catcode`\%=14 %
  \LoadCommand
  \RangeCatcodeCheck{0}{36}{15}%
  \RangeCatcodeCheck{37}{37}{14}%
  \RangeCatcodeCheck{38}{47}{15}%
  \RangeCatcodeCheck{48}{57}{12}%
  \RangeCatcodeCheck{58}{63}{15}%
  \RangeCatcodeCheck{64}{64}{12}%
  \RangeCatcodeCheck{65}{90}{11}%
  \RangeCatcodeCheck{91}{91}{15}%
  \RangeCatcodeCheck{92}{92}{0}%
  \RangeCatcodeCheck{93}{96}{15}%
  \RangeCatcodeCheck{97}{122}{11}%
  \RangeCatcodeCheck{123}{255}{15}%
  \RestoreCatcodes
}
\Test
\csname @@end\endcsname
\end
%    \end{macrocode}
%    \begin{macrocode}
%</test1>
%    \end{macrocode}
%
% \subsection{\LaTeX\space tests}
%
%    \begin{macrocode}
%<*test2|test3|test4|test5>
%    \end{macrocode}
%    \begin{macrocode}
%<noetex>\let\SavedUnexpanded\unexpanded
%<noetex>\let\unexpanded\UNDEFINED
\NeedsTeXFormat{LaTeX2e}
\documentclass{article}
\usepackage[T1]{fontenc}
\usepackage{lmodern}
\usepackage{textcomp}
\usepackage[ngerman]{babel}
\expandafter\def\csname ver@hypdestopt.sty\endcsname{}
\usepackage[unicode]{hyperref}
\usepackage{bookmark}
\usepackage{color}
\usepackage{soul}[2003/11/17]
\makeatletter
\newcommand*{\@ifDefined}[1]{%
  \@ifundefined{#1}{%
    \@latex@error{%
      \expandafter\noexpand\csname#1\endcsname undefined%
    }\@ehc
  }%
}
\newcommand*{\SaveOrg}[1]{%
  \@ifdefinable{ORG@SOUL@#1}{%
    \@ifDefined{SOUL@#1}{%
      \expandafter\let\csname ORG@SOUL@#1\expandafter\endcsname
                      \csname SOUL@#1\endcsname
    }%
  }%
}
\newcommand*{\RestoreOrg}[1]{%
  \@ifDefined{ORG@SOUL@#1}{%
    \@ifDefined{SOUL@#1}{%
      \expandafter\let\csname SOUL@#1\expandafter\endcsname
                      \csname ORG@SOUL@#1\endcsname
    }%
  }%
}
\makeatother
\SaveOrg{eval}
\SaveOrg{splittoken}
\SaveOrg{flushcomma}
\SaveOrg{flushapo}
\SaveOrg{grave}
\SaveOrg{addgroup}
\SaveOrg{addmath}
\SaveOrg{addprotect}
\SaveOrg{addtoken}
\usepackage{soulutf8}[2016/05/16]
%    \end{macrocode}
%    \begin{macrocode}
%<noetex>\let\unexpanded\SavedUnexpanded
%    \end{macrocode}
%    Input encoding \xfile{utf8}.
%    \begin{macrocode}
%<*utf8>
\usepackage[utf8]{inputenc}
\DeclareUnicodeCharacter{00A0}{\nobreakspace}
%</utf8>
%    \end{macrocode}
%    Input encoding \xfile{utf8x}. Package \xpackage{ucs}
%    has a nasty bug, that it redefines \cs{show}.
%    \begin{macrocode}
%<*utf8x>
\let\origshow\show
\usepackage[utf8x]{inputenc}
\AtBeginDocument{%
  \let\show\origshow
}
%</utf8x>
%    \end{macrocode}
%
% \subsubsection{Driver definitions}
%
%    \begin{macrocode}
\makeatletter
%    \end{macrocode}
%    \begin{macro}{\an}
%    The definition for \cs{an} is taken from the documentation of
%    package \xpackage{soul}.
%    \begin{macrocode}
\DeclareRobustCommand*{\an}{%
  \def\SOUL@preamble{$^{^P}$}%
  \def\SOUL@everyspace##1{##1\texttt{\char`\ }\linebreak[2]}%
  \def\SOUL@postamble{$^{^E}$}%
  \def\SOUL@everyhyphen{$^{^-}$\linebreak[1]}%
  \def\SOUL@everyexhyphen##1{##1$^{^=}$}%
  \def\SOUL@everysyllable{$^{^S}$}%
  \def\SOUL@everytoken{\the\SOUL@token$^{^T}$}%
  \def\SOUL@everylowerthan{$^{^L}$}%
  \SOUL@
}
%    \end{macrocode}
%    \end{macro}
%
%    \begin{macro}{\sy}
%    The definition for \cs{sy} is taken from the documentation of
%    package \xpackage{soul}.
%    \begin{macrocode}
\DeclareRobustCommand*{\sy}{%
  \SOUL@setup
  \def\SOUL@preamble{%
    \lefthyphenmin=\z@
    \righthyphenmin=\z@
  }%
  \def\SOUL@everyspace##1{##1\space}%
  \def\SOUL@everysyllable{\the\SOUL@syllable}%
  \def\SOUL@everyhyphen{%
    \discretionary{%
      \SOUL@setkern\SOUL@hyphkern
      \SOUL@sethyphenchar
    }{}{%
      \hbox{\kern1pt$\cdot$}%
    }%
  }%
  \def\SOUL@everyexhyphen##1{%
    \SOUL@setkern\SOUL@hyphkern
    \hbox{##1}%
    \discretionary{}{}{%
      \SOUL@setkern\SOUL@charkern
    }%
  }%
  \SOUL@
}
%    \end{macrocode}
%    \end{macro}
%
%    \begin{macro}{\soulH}
%    \begin{macrocode}
\DeclareRobustCommand*{\soulH}{%
  \def\SOUL@preamble{%
    \gdef\SOULH{}%
    \soulHadd{[}%
  }%
  \def\SOUL@everyspace##1{%
    \soulHadd{Sp}%
  }%
  \def\SOUL@postamble{%
    \soulHadd{]}%
  }%
  \def\SOUL@everyhyphen{%
    \soulHadd{Hy}%
  }%
  \def\SOUL@everyexhyphen##1{%
    \soulHadd{Exhy}%
  }%
  \def\SOUL@everysyllable{%
    \soulHadd{Sy}%
  }%
  \def\SOUL@everytoken{%
    \soulHadd{T}%
  }%
  \def\SOUL@everylowerthan{%
    \soulHadd{<}%
  }%
  \SOUL@
}
\newcommand*{\soulHadd}[1]{%
  #1%
  \expandafter\gdef\expandafter\soulHstring\expandafter{%
    \soulHstring
    #1%
  }%
}
%    \end{macrocode}
%    \end{macro}
%
% \subsubsection{Test macros}
%
%    \begin{macrocode}
\usepackage{qstest}
\IncludeTests{*}
\LogTests{log}{*}{*}
%    \end{macrocode}
%    \begin{macrocode}
\newsavebox\boxA
\newsavebox\boxB
\newsavebox\boxC
%    \end{macrocode}
%    \begin{macro}{\TestDriver}
%    \begin{macrocode}
\newcommand*{\TestDriver}[4]{%
  \subsection{Driver #1}%
  \@PackageInfo{soulutf8-test}{* Driver #1}%
  \expandafter\let\expandafter\testdriver\csname #1\endcsname
  \def\teststringA{#3}%
  \def\teststringB{#4}%
  \begin{qstest}{string-#2-driver-#1}{string,#2,driver,#1}%
    \@PackageInfo{soulutf8-test}{* Version: soulutf8}%
    \global\let\soulHstring\@empty
    \setbox0=\hbox{\expandafter\testdriver\expandafter{\teststringA}}%
    \let\HstringA\soulHstring
    \SetTestVBox{\boxA}{\teststringA}%
    %
    \@PackageInfo{soulutf8-test}{* Version: soul + patch}%
    \global\let\soulHstring\@empty
    \RestoreOrg{eval}%
    \setbox2=\hbox{\expandafter\testdriver\expandafter{\teststringB}}%
    \let\HstringB\soulHstring
    \SetTestVBox{\boxB}{\teststringB}%
    %
    \@PackageInfo{soulutf8-test}{* Version: original soul}%
    \global\let\soulHstring\@empty
    \RestoreOrg{splittoken}%
    \RestoreOrg{flushcomma}%
    \RestoreOrg{flushapo}%
    \RestoreOrg{grave}%
    \RestoreOrg{addgroup}%
    \RestoreOrg{addmath}%
    \RestoreOrg{addprotect}%
    \RestoreOrg{addtoken}%
    \setbox4=\hbox{\expandafter\testdriver\expandafter{\teststringB}}%
    \let\HstringC\soulHstring
    \SetTestVBox{\boxC}{\teststringB}%
    %
    \TestInfo{0}{soulutf8}%
    \TestInfo{2}{soul + patch}%
    \TestInfo{4}{original soul}%
    \TestInfoVBox
    \showboxdepth=\maxdimen
    \showboxbreadth=\maxdimen
    \tracingonline=\@ne
    \Expect*{\HstringA}*{\HstringB}%
    \Expect*{\HstringA}*{\HstringC}%
    \Expect*{\the\wd0}*{\the\wd2}%
    \ifdim\wd0=\wd2 %
    \else
      \showbox0 %
      \showbox2 %
    \fi
    \Expect*{\the\ht0}*{\the\ht2}%
    \Expect*{\the\dp0}*{\the\dp2}%
    \Expect*{\the\wd\boxB}*{\the\wd\boxA}%
    \Expect*{\the\wd\boxC}*{\the\wd\boxA}%
    \Expect*{\the\ht\boxB}*{\the\ht\boxA}%
    \ifdim\ht\boxB=\ht\boxA
    \else
      \showbox\boxA
      \showbox\boxB
    \fi
    \Expect*{\the\ht\boxC}*{\the\ht\boxA}%
    \Expect*{\the\dp\boxB}*{\the\dp\boxA}%
    \Expect*{\the\dp\boxC}*{\the\dp\boxA}%
  \end{qstest}%
}
%    \end{macrocode}
%    \end{macro}
%    \begin{macro}{\SetTestVBox}
%    \begin{macrocode}
\newcommand*{\SetTestVBox}[2]{%
  \setbox#1=\vbox{%
    \hsize=0pt\relax
    \hfuzz=100mm\relax
    \expandafter\testdriver\expandafter{#2}%
  }%
}
%    \end{macrocode}
%    \end{macro}
%    \begin{macro}{\TestInfo}
%    \begin{macrocode}
\newcommand*{\TestInfo}[2]{%
  \par
  \noindent
  \copy#1\relax
  \hfill
  \begingroup
    \small
    \ifdim\wd0=\wd2\relax
      \ifdim\wd0=\wd4\relax
      \else
        \ifnum#1=2 %
          \raisebox{.5\baselineskip}[0pt][0pt]{%
            \llap{$\}$ }%
            \rlap{width = \the\wd#1\relax\space$\{$}%
            \phantom{width = 999.99999pt}%
            \qquad
          }%
        \else
          \ifnum#1=4 %
            \rlap{width = \the\wd#1}%
            \phantom{width = 999.99999pt}%
            \qquad
          \fi
        \fi
      \fi
    \else
      \begingroup
        \bfseries
        \rlap{width = \the\wd#1}%
        \phantom{width = 999.99999pt}%
      \endgroup
      \qquad
    \fi
    \rlap{(#2)}%
    \phantom{(original soul)}%
  \endgroup
  \par
}
%    \end{macrocode}
%    \end{macro}
%    \begin{macro}{\TestInfoVBox}
%    \begin{macrocode}
\newcommand*{\TestInfoVBox}{%
  \par
  \noindent
  \mbox{\vrule\relax\space\copy\boxA}%
  \ifdim\ht\boxA=\ht\boxB
  \else
    \hfill
    \begingroup
      \small
      \rlap{(soulutf8)}%
      \phantom{(soulutf8)}%
    \endgroup
    \par
    \noindent
    \mbox{\vrule\relax\space\copy\boxB}%
    \hfill
    \begingroup
      \small
      \rlap{(soul)}%
      \phantom{(soulutf8)}%
    \endgroup
  \fi
  \par
}
%    \end{macrocode}
%    \end{macro}
%    \begin{macro}{\Test}
%    \begin{macrocode}
\makeatletter
\newcommand*{\Test}[3]{%
  \clearpage
  \begingroup
    \def\linebreak{\textbackslash linebreak }%
    \def\\{\textbackslash\textbackslash}%
    \def\soulomit##1{%
      \textbackslash soulomit\textbraceleft##1\textbraceright
    }%
    \def\mbox##1{%
      \textbackslash mbox\textbraceleft##1\textbraceright
    }%
    \def\abc{\textbackslash abc }%
    \def\texttt##1{%
      \textbackslash texttt\textbraceleft##1\textbraceright
    }%
    \def\c##1{%
      \textbackslash c\textbraceleft##1\textbraceright
    }%
%<*utf8x>
    \count@=128 %
    \loop
    \ifnum\count@<256 %
      \lccode`\~=\count@
      \lowercase{%
        \def~{\noexpand~}%
      }%
      \advance\count@\@ne
    \repeat
%</utf8x>
    \csname protected@edef\endcsname\x{\endgroup
      \noexpand\section{String #1: #2}%
    }%
  \x
  \TestDriver{so}{#1}{#2}{#3}%
  \TestDriver{caps}{#1}{#2}{#3}%
  \TestDriver{ul}{#1}{#2}{#3}%
  \TestDriver{st}{#1}{#2}{#3}%
  \TestDriver{hl}{#1}{#2}{#3}%
  \TestDriver{an}{#1}{#2}{#3}%
  \TestDriver{sy}{#1}{#2}{#3}%
  \TestDriver{soulH}{#1}{#2}{#3}%
}
\makeatother
%    \end{macrocode}
%    \end{macro}
%
% \subsubsection{Testing}
%
%    \begin{macrocode}
\begin{document}
  \Test A{}{}%
  \Test B{Hello World}{Hello World}%
  \Test C{%
    Umlauts: ^^c3^^84^^c3^^96^^c3^^9c^^c3^^a4^^c3^^b6^^c3^^bc^^c3^^9f%
  }{%
    Umlauts: \"A\"O\"U\"a\"o\"u\ss
  }%
  \Test D{Euro: ^^e2^^82^^ac}{Euro: \texteuro}%
  \Test E{Stra^^c3^^9fe}{Stra\ss e}%
  \Test F{R^^c3^^b6mer}{R\"omer}%
  \Test G{A^^e2^^80^^93B^^e2^^80^^94C}{A--B---C}%
  \Test H{X^^c2^^a0Y}{X~Y}%
  \Test I{A^^c5^^a4}{A\v T}%
  \Test J{A,B,,C,,,D,,,,E}{A,B,,C,,,D,,,,E}%
  \Test K{A'B''C'''D''''E}{A'B''C'''D''''E}%
  \Test L{A`B``C```D````E}{A`B``C```D````E}%
  \Test M{A$x$B}{A$x$B}%
  \Test N{A\mbox{abc}B}{A\mbox{abc}B}%
  \Test O{,,TA''}{,,TA''}%
  \Test P{{!`}Hola{?`}}{{!`}Hola{?`}}%
  \Test Q{re-sent\slash x}{re-sent\slash x}%
  \Test R{new\\line}{new\\line}%
  \Test S{foo\linebreak bar}{foo\linebreak bar}%
  \Test T{Virchow{sche}}{Virchow{sche}}%
  \Test U{A\soulomit{foo}B}{A\soulomit{foo}B}%
  \Test V{A\texttt{xyz}B}{A\texttt{xyz}B}%
  \Test W{ffi}{ffi}%
  \Test X{f{}f{}i}{f{}f{}i}%
  \Test Y{f\mbox{}f\mbox{}i}{f\mbox{}f\mbox{}i}%
  \Test Z{\c{\"a}}{{\c{\"a}}}%
%<*test2|test3>
  \let\abcx\UNDEFINED
  \def\abc{%
    \begingroup
      \def\abcx{x}%
      \abcx
    \endgroup
  }%
  \Test{AA}{[{x\abc x}]}{[{xxx}]}%
%</test2|test3>
%    \end{macrocode}
%
%    \begin{macrocode}
\end{document}
%    \end{macrocode}
%    \begin{macrocode}
%</test2|test3|test4|test5>
%    \end{macrocode}
%
% \section{Installation}
%
% \subsection{Download}
%
% \paragraph{Package.} This package is available on
% CTAN\footnote{\url{http://ctan.org/pkg/soulutf8}}:
% \begin{description}
% \item[\CTAN{macros/latex/contrib/oberdiek/soulutf8.dtx}] The source file.
% \item[\CTAN{macros/latex/contrib/oberdiek/soulutf8.pdf}] Documentation.
% \end{description}
%
%
% \paragraph{Bundle.} All the packages of the bundle `oberdiek'
% are also available in a TDS compliant ZIP archive. There
% the packages are already unpacked and the documentation files
% are generated. The files and directories obey the TDS standard.
% \begin{description}
% \item[\CTAN{install/macros/latex/contrib/oberdiek.tds.zip}]
% \end{description}
% \emph{TDS} refers to the standard ``A Directory Structure
% for \TeX\ Files'' (\CTAN{tds/tds.pdf}). Directories
% with \xfile{texmf} in their name are usually organized this way.
%
% \subsection{Bundle installation}
%
% \paragraph{Unpacking.} Unpack the \xfile{oberdiek.tds.zip} in the
% TDS tree (also known as \xfile{texmf} tree) of your choice.
% Example (linux):
% \begin{quote}
%   |unzip oberdiek.tds.zip -d ~/texmf|
% \end{quote}
%
% \paragraph{Script installation.}
% Check the directory \xfile{TDS:scripts/oberdiek/} for
% scripts that need further installation steps.
% Package \xpackage{attachfile2} comes with the Perl script
% \xfile{pdfatfi.pl} that should be installed in such a way
% that it can be called as \texttt{pdfatfi}.
% Example (linux):
% \begin{quote}
%   |chmod +x scripts/oberdiek/pdfatfi.pl|\\
%   |cp scripts/oberdiek/pdfatfi.pl /usr/local/bin/|
% \end{quote}
%
% \subsection{Package installation}
%
% \paragraph{Unpacking.} The \xfile{.dtx} file is a self-extracting
% \docstrip\ archive. The files are extracted by running the
% \xfile{.dtx} through \plainTeX:
% \begin{quote}
%   \verb|tex soulutf8.dtx|
% \end{quote}
%
% \paragraph{TDS.} Now the different files must be moved into
% the different directories in your installation TDS tree
% (also known as \xfile{texmf} tree):
% \begin{quote}
% \def\t{^^A
% \begin{tabular}{@{}>{\ttfamily}l@{ $\rightarrow$ }>{\ttfamily}l@{}}
%   soulutf8.sty & tex/generic/oberdiek/soulutf8.sty\\
%   soulutf8.pdf & doc/latex/oberdiek/soulutf8.pdf\\
%   test/soulutf8-test1.tex & doc/latex/oberdiek/test/soulutf8-test1.tex\\
%   test/soulutf8-test2.tex & doc/latex/oberdiek/test/soulutf8-test2.tex\\
%   test/soulutf8-test3.tex & doc/latex/oberdiek/test/soulutf8-test3.tex\\
%   test/soulutf8-test4.tex & doc/latex/oberdiek/test/soulutf8-test4.tex\\
%   test/soulutf8-test5.tex & doc/latex/oberdiek/test/soulutf8-test5.tex\\
%   soulutf8.dtx & source/latex/oberdiek/soulutf8.dtx\\
% \end{tabular}^^A
% }^^A
% \sbox0{\t}^^A
% \ifdim\wd0>\linewidth
%   \begingroup
%     \advance\linewidth by\leftmargin
%     \advance\linewidth by\rightmargin
%   \edef\x{\endgroup
%     \def\noexpand\lw{\the\linewidth}^^A
%   }\x
%   \def\lwbox{^^A
%     \leavevmode
%     \hbox to \linewidth{^^A
%       \kern-\leftmargin\relax
%       \hss
%       \usebox0
%       \hss
%       \kern-\rightmargin\relax
%     }^^A
%   }^^A
%   \ifdim\wd0>\lw
%     \sbox0{\small\t}^^A
%     \ifdim\wd0>\linewidth
%       \ifdim\wd0>\lw
%         \sbox0{\footnotesize\t}^^A
%         \ifdim\wd0>\linewidth
%           \ifdim\wd0>\lw
%             \sbox0{\scriptsize\t}^^A
%             \ifdim\wd0>\linewidth
%               \ifdim\wd0>\lw
%                 \sbox0{\tiny\t}^^A
%                 \ifdim\wd0>\linewidth
%                   \lwbox
%                 \else
%                   \usebox0
%                 \fi
%               \else
%                 \lwbox
%               \fi
%             \else
%               \usebox0
%             \fi
%           \else
%             \lwbox
%           \fi
%         \else
%           \usebox0
%         \fi
%       \else
%         \lwbox
%       \fi
%     \else
%       \usebox0
%     \fi
%   \else
%     \lwbox
%   \fi
% \else
%   \usebox0
% \fi
% \end{quote}
% If you have a \xfile{docstrip.cfg} that configures and enables \docstrip's
% TDS installing feature, then some files can already be in the right
% place, see the documentation of \docstrip.
%
% \subsection{Refresh file name databases}
%
% If your \TeX~distribution
% (\teTeX, \mikTeX, \dots) relies on file name databases, you must refresh
% these. For example, \teTeX\ users run \verb|texhash| or
% \verb|mktexlsr|.
%
% \subsection{Some details for the interested}
%
% \paragraph{Attached source.}
%
% The PDF documentation on CTAN also includes the
% \xfile{.dtx} source file. It can be extracted by
% AcrobatReader 6 or higher. Another option is \textsf{pdftk},
% e.g. unpack the file into the current directory:
% \begin{quote}
%   \verb|pdftk soulutf8.pdf unpack_files output .|
% \end{quote}
%
% \paragraph{Unpacking with \LaTeX.}
% The \xfile{.dtx} chooses its action depending on the format:
% \begin{description}
% \item[\plainTeX:] Run \docstrip\ and extract the files.
% \item[\LaTeX:] Generate the documentation.
% \end{description}
% If you insist on using \LaTeX\ for \docstrip\ (really,
% \docstrip\ does not need \LaTeX), then inform the autodetect routine
% about your intention:
% \begin{quote}
%   \verb|latex \let\install=y% \iffalse meta-comment
%
% File: soulutf8.dtx
% Version: 2016/05/16 v1.1
% Info: Permit use of UTF-8 characters in soul
%
% Copyright (C) 2007 by
%    Heiko Oberdiek <heiko.oberdiek at googlemail.com>
%    2016
%    https://github.com/ho-tex/oberdiek/issues
%
% This work may be distributed and/or modified under the
% conditions of the LaTeX Project Public License, either
% version 1.3c of this license or (at your option) any later
% version. This version of this license is in
%    http://www.latex-project.org/lppl/lppl-1-3c.txt
% and the latest version of this license is in
%    http://www.latex-project.org/lppl.txt
% and version 1.3 or later is part of all distributions of
% LaTeX version 2005/12/01 or later.
%
% This work has the LPPL maintenance status "maintained".
%
% This Current Maintainer of this work is Heiko Oberdiek.
%
% The Base Interpreter refers to any `TeX-Format',
% because some files are installed in TDS:tex/generic//.
%
% This work consists of the main source file soulutf8.dtx
% and the derived files
%    soulutf8.sty, soulutf8.pdf, soulutf8.ins, soulutf8.drv,
%    soulutf8-test1.tex, soulutf8-test2.tex, soulutf8-test3.tex,
%    soulutf8-test4.tex, soulutf8-test5.tex.
%
% Distribution:
%    CTAN:macros/latex/contrib/oberdiek/soulutf8.dtx
%    CTAN:macros/latex/contrib/oberdiek/soulutf8.pdf
%
% Unpacking:
%    (a) If soulutf8.ins is present:
%           tex soulutf8.ins
%    (b) Without soulutf8.ins:
%           tex soulutf8.dtx
%    (c) If you insist on using LaTeX
%           latex \let\install=y\input{soulutf8.dtx}
%        (quote the arguments according to the demands of your shell)
%
% Documentation:
%    (a) If soulutf8.drv is present:
%           latex soulutf8.drv
%    (b) Without soulutf8.drv:
%           latex soulutf8.dtx; ...
%    The class ltxdoc loads the configuration file ltxdoc.cfg
%    if available. Here you can specify further options, e.g.
%    use A4 as paper format:
%       \PassOptionsToClass{a4paper}{article}
%
%    Programm calls to get the documentation (example):
%       pdflatex soulutf8.dtx
%       makeindex -s gind.ist soulutf8.idx
%       pdflatex soulutf8.dtx
%       makeindex -s gind.ist soulutf8.idx
%       pdflatex soulutf8.dtx
%
% Installation:
%    TDS:tex/generic/oberdiek/soulutf8.sty
%    TDS:doc/latex/oberdiek/soulutf8.pdf
%    TDS:doc/latex/oberdiek/test/soulutf8-test1.tex
%    TDS:doc/latex/oberdiek/test/soulutf8-test2.tex
%    TDS:doc/latex/oberdiek/test/soulutf8-test3.tex
%    TDS:doc/latex/oberdiek/test/soulutf8-test4.tex
%    TDS:doc/latex/oberdiek/test/soulutf8-test5.tex
%    TDS:source/latex/oberdiek/soulutf8.dtx
%
%<*ignore>
\begingroup
  \catcode123=1 %
  \catcode125=2 %
  \def\x{LaTeX2e}%
\expandafter\endgroup
\ifcase 0\ifx\install y1\fi\expandafter
         \ifx\csname processbatchFile\endcsname\relax\else1\fi
         \ifx\fmtname\x\else 1\fi\relax
\else\csname fi\endcsname
%</ignore>
%<*install>
\input docstrip.tex
\Msg{************************************************************************}
\Msg{* Installation}
\Msg{* Package: soulutf8 2016/05/16 v1.1 Permit use of UTF-8 characters in soul (HO)}
\Msg{************************************************************************}

\keepsilent
\askforoverwritefalse

\let\MetaPrefix\relax
\preamble

This is a generated file.

Project: soulutf8
Version: 2016/05/16 v1.1

Copyright (C) 2007 by
   Heiko Oberdiek <heiko.oberdiek at googlemail.com>

This work may be distributed and/or modified under the
conditions of the LaTeX Project Public License, either
version 1.3c of this license or (at your option) any later
version. This version of this license is in
   http://www.latex-project.org/lppl/lppl-1-3c.txt
and the latest version of this license is in
   http://www.latex-project.org/lppl.txt
and version 1.3 or later is part of all distributions of
LaTeX version 2005/12/01 or later.

This work has the LPPL maintenance status "maintained".

This Current Maintainer of this work is Heiko Oberdiek.

The Base Interpreter refers to any `TeX-Format',
because some files are installed in TDS:tex/generic//.

This work consists of the main source file soulutf8.dtx
and the derived files
   soulutf8.sty, soulutf8.pdf, soulutf8.ins, soulutf8.drv,
   soulutf8-test1.tex, soulutf8-test2.tex, soulutf8-test3.tex,
   soulutf8-test4.tex, soulutf8-test5.tex.

\endpreamble
\let\MetaPrefix\DoubleperCent

\generate{%
  \file{soulutf8.ins}{\from{soulutf8.dtx}{install}}%
  \file{soulutf8.drv}{\from{soulutf8.dtx}{driver}}%
  \usedir{tex/generic/oberdiek}%
  \file{soulutf8.sty}{\from{soulutf8.dtx}{package}}%
%  \usedir{doc/latex/oberdiek/test}%
%  \file{soulutf8-test1.tex}{\from{soulutf8.dtx}{test1}}%
%  \file{soulutf8-test2.tex}{\from{soulutf8.dtx}{test2,utf8}}%
%  \file{soulutf8-test3.tex}{\from{soulutf8.dtx}{test3,utf8x}}%
%  \file{soulutf8-test4.tex}{\from{soulutf8.dtx}{test4,utf8,noetex}}%
%  \file{soulutf8-test5.tex}{\from{soulutf8.dtx}{test5,utf8x,noetex}}%
  \nopreamble
  \nopostamble
%  \usedir{source/latex/oberdiek/catalogue}%
%  \file{soulutf8.xml}{\from{soulutf8.dtx}{catalogue}}%
}

\catcode32=13\relax% active space
\let =\space%
\Msg{************************************************************************}
\Msg{*}
\Msg{* To finish the installation you have to move the following}
\Msg{* file into a directory searched by TeX:}
\Msg{*}
\Msg{*     soulutf8.sty}
\Msg{*}
\Msg{* To produce the documentation run the file `soulutf8.drv'}
\Msg{* through LaTeX.}
\Msg{*}
\Msg{* Happy TeXing!}
\Msg{*}
\Msg{************************************************************************}

\endbatchfile
%</install>
%<*ignore>
\fi
%</ignore>
%<*driver>
\NeedsTeXFormat{LaTeX2e}
\ProvidesFile{soulutf8.drv}%
  [2016/05/16 v1.1 Permit use of UTF-8 characters in soul (HO)]%
\documentclass{ltxdoc}
\usepackage{holtxdoc}[2011/11/22]
\begin{document}
  \DocInput{soulutf8.dtx}%
\end{document}
%</driver>
% \fi
%
%
% \CharacterTable
%  {Upper-case    \A\B\C\D\E\F\G\H\I\J\K\L\M\N\O\P\Q\R\S\T\U\V\W\X\Y\Z
%   Lower-case    \a\b\c\d\e\f\g\h\i\j\k\l\m\n\o\p\q\r\s\t\u\v\w\x\y\z
%   Digits        \0\1\2\3\4\5\6\7\8\9
%   Exclamation   \!     Double quote  \"     Hash (number) \#
%   Dollar        \$     Percent       \%     Ampersand     \&
%   Acute accent  \'     Left paren    \(     Right paren   \)
%   Asterisk      \*     Plus          \+     Comma         \,
%   Minus         \-     Point         \.     Solidus       \/
%   Colon         \:     Semicolon     \;     Less than     \<
%   Equals        \=     Greater than  \>     Question mark \?
%   Commercial at \@     Left bracket  \[     Backslash     \\
%   Right bracket \]     Circumflex    \^     Underscore    \_
%   Grave accent  \`     Left brace    \{     Vertical bar  \|
%   Right brace   \}     Tilde         \~}
%
% \GetFileInfo{soulutf8.drv}
%
% \title{The \xpackage{soulutf8} package}
% \date{2016/05/16 v1.1}
% \author{Heiko Oberdiek\thanks
% {Please report any issues at https://github.com/ho-tex/oberdiek/issues}\\
% \xemail{heiko.oberdiek at googlemail.com}}
%
% \maketitle
%
% \begin{abstract}
% This package extends package \xpackage{soul} and adds some support
% for UTF-8. Namely the input encodings \xfile{utf8.def}
% from package \xpackage{inputenc} and
% package \xpackage{ucs}'s \xfile{utf8x.def} are supported.
% \end{abstract}
%
% \tableofcontents
%
% \section{Documentation}
%
% This package \xpackage{soulutf8} does not have own options and does not
% define new user commands.
% Any option is passed to package \xpackage{soul} \cite{soul}
% that is loaded first.
% Then some internal macros of \xpackage{soul} are redefined to add
% support for UTF-8.
% The following input encodings are supported:
% \begin{quote}
%   \begin{tabular}{@{}lll@{}}
%     \texttt{utf8} & \LaTeX\ base &
%        \xfile{TDS:tex/latex/base/utf8.def} \cite{utf8}\\
%     \texttt{utf8x} & Package \xpackage{ucs} &
%       \xfile{TDS:tex/latex/ucs/utf8x.def} \cite{ucs}\\
%   \end{tabular}
% \end{quote}
% UTF-8 byte sequences are added as token group to a word, even
% if these UTF-8 characters are some kind of hyphen or space.
% As exception the following three Unicode characters are handled specially:
% \begin{quote}
%   \begin{tabular}{lll}
%     Slot & Name & Action\\
%     \hline
%     \texttt{U+00A0} & NO-BREAK SPACE & like |~|\\
%     \texttt{U+2013} & EN DASH & |--|\\
%     \texttt{U+2014} & EM DASH & |---|
%   \end{tabular}
%
% \subsection{Patch}
%
% Also package \xpackage{soulutf8} tries to patch package \xpackage{soul}
% to improve its behaviour:
% \begin{itemize}
% \item
%   A problem with additional levels of curly braces is fixed.
%   As advantage more implicite kernings are detected.
%   However, the result may be incompatible with the
%   original behaviour of package \xpackage{soul} because
%   of these respected implicite kernings.
% \item
%   \eTeX\ , especially \cs{unexpanded} is supported.
%   This allows a better protection of token groups
%   (|\mbox{|\dots|}|, math, \dots).
% \end{itemize}
% \end{quote}
%
% \subsection{Future}
%
% Currently package \xpackage{soul} does not seem to be
% maintained. Nevertheless if there will be a new version that
% adds support for UTF-8, then this package may become
% obsolete.
%
% \StopEventually{
% }
%
% \section{Implementation}
%
%    \begin{macrocode}
%<*package>
%    \end{macrocode}
%
% \subsection{Reload check and package identification}
%    Reload check, especially if the package is not used with \LaTeX.
%    \begin{macrocode}
\begingroup\catcode61\catcode48\catcode32=10\relax%
  \catcode13=5 % ^^M
  \endlinechar=13 %
  \catcode35=6 % #
  \catcode39=12 % '
  \catcode44=12 % ,
  \catcode45=12 % -
  \catcode46=12 % .
  \catcode58=12 % :
  \catcode64=11 % @
  \catcode123=1 % {
  \catcode125=2 % }
  \expandafter\let\expandafter\x\csname ver@soulutf8.sty\endcsname
  \ifx\x\relax % plain-TeX, first loading
  \else
    \def\empty{}%
    \ifx\x\empty % LaTeX, first loading,
      % variable is initialized, but \ProvidesPackage not yet seen
    \else
      \expandafter\ifx\csname PackageInfo\endcsname\relax
        \def\x#1#2{%
          \immediate\write-1{Package #1 Info: #2.}%
        }%
      \else
        \def\x#1#2{\PackageInfo{#1}{#2, stopped}}%
      \fi
      \x{soulutf8}{The package is already loaded}%
      \aftergroup\endinput
    \fi
  \fi
\endgroup%
%    \end{macrocode}
%    Package identification:
%    \begin{macrocode}
\begingroup\catcode61\catcode48\catcode32=10\relax%
  \catcode13=5 % ^^M
  \endlinechar=13 %
  \catcode35=6 % #
  \catcode39=12 % '
  \catcode40=12 % (
  \catcode41=12 % )
  \catcode44=12 % ,
  \catcode45=12 % -
  \catcode46=12 % .
  \catcode47=12 % /
  \catcode58=12 % :
  \catcode64=11 % @
  \catcode91=12 % [
  \catcode93=12 % ]
  \catcode123=1 % {
  \catcode125=2 % }
  \expandafter\ifx\csname ProvidesPackage\endcsname\relax
    \def\x#1#2#3[#4]{\endgroup
      \immediate\write-1{Package: #3 #4}%
      \xdef#1{#4}%
    }%
  \else
    \def\x#1#2[#3]{\endgroup
      #2[{#3}]%
      \ifx#1\@undefined
        \xdef#1{#3}%
      \fi
      \ifx#1\relax
        \xdef#1{#3}%
      \fi
    }%
  \fi
\expandafter\x\csname ver@soulutf8.sty\endcsname
\ProvidesPackage{soulutf8}%
  [2016/05/16 v1.1 Permit use of UTF-8 characters in soul (HO)]%
%    \end{macrocode}
%
% \subsection{Catcodes}
%
%    \begin{macrocode}
\begingroup\catcode61\catcode48\catcode32=10\relax%
  \catcode13=5 % ^^M
  \endlinechar=13 %
  \catcode123=1 % {
  \catcode125=2 % }
  \catcode64=11 % @
  \def\x{\endgroup
    \expandafter\edef\csname SOuL@AtEnd\endcsname{%
      \endlinechar=\the\endlinechar\relax
      \catcode13=\the\catcode13\relax
      \catcode32=\the\catcode32\relax
      \catcode35=\the\catcode35\relax
      \catcode61=\the\catcode61\relax
      \catcode64=\the\catcode64\relax
      \catcode123=\the\catcode123\relax
      \catcode125=\the\catcode125\relax
    }%
  }%
\x\catcode61\catcode48\catcode32=10\relax%
\catcode13=5 % ^^M
\endlinechar=13 %
\catcode35=6 % #
\catcode64=11 % @
\catcode123=1 % {
\catcode125=2 % }
\def\TMP@EnsureCode#1#2{%
  \edef\SOuL@AtEnd{%
    \SOuL@AtEnd
    \catcode#1=\the\catcode#1\relax
  }%
  \catcode#1=#2\relax
}
\TMP@EnsureCode{10}{12}% ^^J
\TMP@EnsureCode{33}{12}% !
\TMP@EnsureCode{34}{12}% "
\TMP@EnsureCode{36}{3}% $
\TMP@EnsureCode{39}{12}% '
\TMP@EnsureCode{40}{12}% (
\TMP@EnsureCode{41}{12}% )
\TMP@EnsureCode{42}{12}% *
\TMP@EnsureCode{43}{12}% +
\TMP@EnsureCode{44}{12}% ,
\TMP@EnsureCode{45}{12}% -
\TMP@EnsureCode{46}{12}% .
\TMP@EnsureCode{47}{12}% /
\TMP@EnsureCode{58}{12}% :
\TMP@EnsureCode{60}{12}% <
\TMP@EnsureCode{62}{12}% >
\TMP@EnsureCode{91}{12}% [
\TMP@EnsureCode{93}{12}% ]
\TMP@EnsureCode{94}{7}% ^
\TMP@EnsureCode{96}{12}% `
\TMP@EnsureCode{126}\active % ~
\TMP@EnsureCode{128}{12}% ^^80
\TMP@EnsureCode{147}{12}% ^^93
\TMP@EnsureCode{148}{12}% ^^94
\TMP@EnsureCode{160}{12}% ^^a0
\TMP@EnsureCode{194}{12}% ^^c2
\TMP@EnsureCode{226}{12}% ^^e2
\edef\SOuL@AtEnd{\SOuL@AtEnd\noexpand\endinput}
%    \end{macrocode}
%
% \subsection{Loading packages}
%
%    Package \xpackage{soul} uses \cs{documentclass} to detect \LaTeX.
%    \begin{macrocode}
\ifx\documentclass\@undefined
%    \end{macrocode}
%
% \subsubsection{\plainTeX}
%
%    First we check, whether package \xpackage{soul} is already loaded.
%    \begin{macrocode}
  \expandafter\ifx\csname SOUL@\endcsname\relax
%    \end{macrocode}
%    In case of \plainTeX\ package \xpackage{soul} defines some
%    macros in a simple manner that will break the definitions of
%    \xfile{miniltx.tex}, for example. Therefore these macros
%    are first saved and restored afterwards.
%    \begin{macrocode}
    \let\SOuL@orgDeclareRobustCommand\DeclareRobustCommand
    \let\SOuL@orgnewcommand          \newcommand
    \let\SOuL@orgDeclareOption       \DeclareOption
    \let\SOuL@orgPackageError        \PackageError
    \def\SOuL@restorelatexcmds{%
      \let\DeclareRobustCommand\SOuL@orgDeclareRobustCommand
      \let\newcommand          \SOuL@orgnewcommand
      \let\DeclareOption       \SOuL@orgDeclareOption
      \let\PackageError        \SOuL@orgPackageError
    }%
    \input soul.sty\relax
    \SOuL@restorelatexcmds
  \fi
%    \end{macrocode}
%    \begin{macro}{\SOUL@error}
%     Package \xpackage{soul}'s use of \cs{PackageError} is replaced
%     by \cs{@PackageError} of package \xpackage{infwarerr}.
%    \begin{macrocode}
  \input infwarerr.sty\relax
  \let\SOuL@orgSOUL@error\SOUL@error
  \def\SOUL@error{%
    \begingroup
      \let\PackageError\@PackageError
      \SOuL@orgSOUL@error
    \endgroup
  }%
%    \end{macrocode}
%    \end{macro}
%    \begin{macrocode}
  \input etexcmds.sty\relax
%    \end{macrocode}
%    \begin{macro}{\@onelevel@sanitize}
%    Define \LaTeX's \cs{@onelevel@sanitize} if not already available.
%    \begin{macrocode}
  \expandafter\ifx\csname @onelevel@sanitize\endcsname\relax
    \def\@onelevel@sanitize#1{%
      \edef#1{%
        \expandafter\strip@prefix\meaning#1%
      }%
    }%
%    \end{macrocode}
%    \end{macro}
%    \begin{macro}{\strip@prefix}
%    \begin{macrocode}
    \def\strip@prefix#1>{}%
%    \end{macrocode}
%    \end{macro}
%    \begin{macrocode}
  \fi
\else
%    \end{macrocode}
%
% \subsubsection{\LaTeX}
%    \begin{macrocode}
  \DeclareOption*{\PassOptionsToPackage{\CurrentOption}{soul}}%
  \ProcessOptions\relax
  \RequirePackage{soul}[2003/11/17]%
  \RequirePackage{infwarerr}[2016/05/16]%
  \RequirePackage{etexcmds}[2016/05/16]%
\fi
%    \end{macrocode}
%
% \subsubsection{\eTeX}
%
%    In \plainTeX\ command \cs{+} is an \emph{outer} macro.
%    Therefore numbers are used to avoid problems.
%    \begin{macrocode}
\ifetex@unexpanded
  \catcode33=14 % '!': comment
  \catcode43=9  % '+': ignore
\else
  \catcode33=9  % '!': ignore
  \catcode43=14 % '+': comment
\fi
%    \end{macrocode}
%
% \subsection{Macro for redefinitions}
%
%    \begin{macro}{\SOuL@redefine}
%    \begin{macrocode}
\def\SOuL@redefine#1{%
  \begingroup
    \def\SOuL@cmd{#1}%
    \afterassignment\SOuL@cmdcheck
    \def\SOuL@temp
}
%    \end{macrocode}
%    \end{macro}
%    \begin{macro}{\SOuL@cmdcheck}
%    \begin{macrocode}
\def\SOuL@cmdcheck{%
    \expandafter\ifx\SOuL@cmd\SOuL@temp
    \else
      \edef\SOuL@temp*{\expandafter\string\SOuL@cmd}%
      \@PackageWarningNoLine{soulutf8}{%
        Command \SOuL@temp* has changed.\MessageBreak
        Supported versions of package `soul': 2003/11/17.\MessageBreak
        Depending on the unknown changes the redefinition\MessageBreak
        of \SOuL@temp* may not behave correctly%
      }%
    \fi
  \expandafter\endgroup
  \expandafter\def\SOuL@cmd
}
%    \end{macrocode}
%    \end{macro}
%
% \subsection{Redefinition of \cs{SOUL@eval}}
%
%    \begin{macro}{\SOUL@eval}
%    Macro \cs{SOUL@eval} is redefined to add detection of
%    the first byte of a UTF-8 sequence. Because \cs{SOUL@eval}
%    is overwritten, a warning is issued, if the contents of
%    \cs{SOUL@eval} is not as expected.
%    \begin{macrocode}
\SOuL@redefine\SOUL@eval{%
%    \end{macrocode}
%    First the expected definition.
%    \begin{macrocode}
  \def\SOUL@n*##1{\SOUL@scan}%
  \if\noexpand\SOUL@@\SOUL@spc
  \else
    \SOUL@ignorespacesfalse
  \fi
  \ifnum\SOUL@minus=\thr@@
    \SOUL@flushminus
  \else\ifnum\SOUL@comma=\tw@
    \SOUL@flushcomma
  \else\ifnum\SOUL@apo=\tw@
    \SOUL@flushapo
  \else\ifnum\SOUL@grave=\tw@
    \SOUL@flushgrave
  \fi\fi\fi\fi
  \ifx\SOUL@@-\else\SOUL@flushminus\fi
  \ifx\SOUL@@,\else\SOUL@flushcomma\fi
  \ifx\SOUL@@'\else\SOUL@flushapo\fi
  \ifx\SOUL@@`\else\SOUL@flushgrave\fi
  \ifx\SOUL@@-%
    \advance\SOUL@minus\@ne
  \else\ifx\SOUL@@,%
    \advance\SOUL@comma\@ne
  \else\ifx\SOUL@@'%
    \advance\SOUL@apo\@ne
  \else\ifx\SOUL@@`%
    \advance\SOUL@grave\@ne
  \else
    \SOUL@flushminus
    \SOUL@flushcomma
    \SOUL@flushapo
    \SOUL@flushgrave
    \ifx\SOUL@@\SOUL@stop
      \def\SOUL@n*{%
        \SOUL@doword
        \SOUL@eventuallyexhyphen\null
      }%
    \else\ifx\SOUL@@\par
      \def\SOUL@n*\par{\par\leavevmode\SOUL@scan}%
    \else\if\noexpand\SOUL@@\SOUL@spc
      \SOUL@doword
      \SOUL@eventuallyexhyphen\null
      \ifSOUL@ignorespaces
      \else
        \SOUL@everyspace{}%
      \fi
      \def\SOUL@n* {\SOUL@scan}%
    \else\ifx\SOUL@@\\%
      \SOUL@doword
      \SOUL@eventuallyexhyphen\null
      \SOUL@everyspace{\unskip\nobreak\hfil\break}%
      \SOUL@ignorespacestrue
    \else\ifx\SOUL@@~%
      \SOUL@doword
      \SOUL@eventuallyexhyphen\null
      \SOUL@everyspace{\nobreak}%
    \else\ifx\SOUL@@\slash
      \SOUL@doword
      \SOUL@eventuallyexhyphen{/}%
      \SOUL@exhyphen{/}%
    \else\ifx\SOUL@@\mbox
      \def\SOUL@n*{\SOUL@addprotect}%
    \else\ifx\SOUL@@\hbox
      \def\SOUL@n*{\SOUL@addprotect}%
    \else\ifx\SOUL@@\soulomit
      \def\SOUL@n*\soulomit##1{%
        \SOUL@doword
        {\spaceskip\SOUL@spaceskip##1}%
        \SOUL@scan
      }%
    \else\ifx\SOUL@@\break
      \SOUL@doword
      \break
    \else\ifx\SOUL@@\linebreak
      \SOUL@doword
      \SOUL@everyspace{\linebreak}%
    \else\ifcat\bgroup\noexpand\SOUL@@
      \def\SOUL@n*{\SOUL@addgroup{}}%
    \else\ifcat$\noexpand\SOUL@@
      \def\SOUL@n*{\SOUL@addmath}%
    \else
      \def\SOUL@n*{\SOUL@dotoken}%
    \fi\fi\fi\fi\fi\fi\fi\fi\fi\fi\fi\fi\fi
  \fi\fi\fi\fi
  \SOUL@n*%
}{%
%    \end{macrocode}
%    Now the redefined version follows.
%    \begin{macrocode}
  \def\SOUL@n*##1{\SOUL@scan}%
  \if\noexpand\SOUL@@\SOUL@spc
  \else
    \SOUL@ignorespacesfalse
  \fi
  \ifnum\SOUL@minus=\thr@@
    \SOUL@flushminus
  \else\ifnum\SOUL@comma=\tw@
    \SOUL@flushcomma
  \else\ifnum\SOUL@apo=\tw@
    \SOUL@flushapo
  \else\ifnum\SOUL@grave=\tw@
    \SOUL@flushgrave
  \fi\fi\fi\fi
  \ifx\SOUL@@-\else\SOUL@flushminus\fi
  \ifx\SOUL@@,\else\SOUL@flushcomma\fi
  \ifx\SOUL@@'\else\SOUL@flushapo\fi
  \ifx\SOUL@@`\else\SOUL@flushgrave\fi
  \ifx\SOUL@@-%
    \advance\SOUL@minus\@ne
  \else\ifx\SOUL@@,%
    \advance\SOUL@comma\@ne
  \else\ifx\SOUL@@'%
    \advance\SOUL@apo\@ne
  \else\ifx\SOUL@@`%
    \advance\SOUL@grave\@ne
  \else
    \SOUL@flushminus
    \SOUL@flushcomma
    \SOUL@flushapo
    \SOUL@flushgrave
    \ifx\SOUL@@\SOUL@stop
      \def\SOUL@n*{%
          \SOUL@doword
          \SOUL@eventuallyexhyphen\null
      }%
    \else\ifx\SOUL@@\par
      \def\SOUL@n*\par{\par\leavevmode\SOUL@scan}%
    \else\if\noexpand\SOUL@@\SOUL@spc
      \SOUL@doword
      \SOUL@eventuallyexhyphen\null
      \ifSOUL@ignorespaces
      \else
        \SOUL@everyspace{}%
      \fi
      \def\SOUL@n* {\SOUL@scan}%
    \else\ifx\SOUL@@\\%
      \SOUL@doword
      \SOUL@eventuallyexhyphen\null
      \SOUL@everyspace{\unskip\nobreak\hfil\break}%
      \SOUL@ignorespacestrue
    \else\ifx\SOUL@@~%
      \SOUL@doword
      \SOUL@eventuallyexhyphen\null
      \SOUL@everyspace{\nobreak}%
    \else\ifx\SOUL@@\slash
      \SOUL@doword
      \SOUL@eventuallyexhyphen{/}%
      \SOUL@exhyphen{/}%
    \else\ifx\SOUL@@\mbox
      \def\SOUL@n*{\SOUL@addprotect}%
    \else\ifx\SOUL@@\hbox
      \def\SOUL@n*{\SOUL@addprotect}%
    \else\ifx\SOUL@@\soulomit
      \def\SOUL@n*\soulomit##1{%
        \SOUL@doword
        {\spaceskip\SOUL@spaceskip##1}%
        \SOUL@scan
      }%
    \else\ifx\SOUL@@\break
      \SOUL@doword
      \break
    \else\ifx\SOUL@@\linebreak
      \SOUL@doword
      \SOUL@everyspace{\linebreak}%
    \else\ifcat\bgroup\noexpand\SOUL@@
      \def\SOUL@n*{\SOUL@addgroup{}}%
    \else\ifcat$\noexpand\SOUL@@
      \def\SOUL@n*{\SOUL@addmath}%
    \else
%    \end{macrocode}
%    The current token is examined to detect the start of
%    a UTF-8 sequence.
%    \begin{macrocode}
      \SOuL@analyzeutfviii
      \ifcase\SOuL@octets
        \SOuL@analyzeutfviiix
      \fi
      \ifcase\SOuL@octets
        \def\SOUL@n*{\SOUL@dotoken}%
      \or % 1
      \or % 2
        \def\SOUL@n*{\SOuL@addtwooctets}%
      \or % 3
        \def\SOUL@n*{\SOuL@addthreeoctets}%
      \or % 4
        \def\SOUL@n*{\SOuL@addfouroctets}%
      \fi
    \fi\fi\fi\fi\fi\fi\fi\fi\fi\fi\fi\fi\fi
  \fi\fi\fi\fi
  \SOUL@n*%
}
%    \end{macrocode}
%    \end{macro}
%
% \subsection{UTF-8 analysis}
%
% \subsubsection{Help strings}
%
%    \begin{macrocode}
\def\SOuL@defsanitizedstring#1#2{%
  \expandafter\def\csname SOuL@string#1\endcsname{#2}%
  \expandafter\@onelevel@sanitize\csname SOuL@string#1\endcsname
}
\SOuL@defsanitizedstring{UTFviii}{UTFviii@}
\SOuL@defsanitizedstring{octets}{@octets}
\SOuL@defsanitizedstring{two}{two}
\SOuL@defsanitizedstring{three}{three}
\SOuL@defsanitizedstring{four}{four}
\SOuL@defsanitizedstring{macrocolon}{macro:}
\SOuL@defsanitizedstring{csnameu}{csname u8-}
\SOuL@defsanitizedstring{undeferr}{utf@viii@undeferr}
\def\SOuL@stringendash{^^e2^^80^^93}
\def\SOuL@stringemdash{^^e2^^80^^94}
\def\SOuL@stringnobreakspace{^^c2^^a0}
\edef\SOuL@charhash{\string #}
\edef\SOuL@chartwo{\string 2}
\edef\SOuL@charthree{\string 3}
\def\SOuL@empty{}
%    \end{macrocode}
%
% \subsubsection{Support for \xfile{utf8.def}}
%
%    \begin{macro}{\SOuL@analyzeutfviii}
%    \begin{macrocode}
\begingroup
  \edef\x{\endgroup
    \def\noexpand\SOuL@analyzeutfviii{%
      \noexpand\expandafter\noexpand\SOuL@checkutfviii
      \noexpand\meaning\noexpand\SOUL@@
      \SOuL@stringUTFviii\SOuL@stringoctets
      \noexpand\@nil
    }%
    \def\noexpand\SOuL@checkutfviii
      ##1\SOuL@stringUTFviii##2\SOuL@stringoctets##3\noexpand\@nil
  }%
\x{%
  \def\SOuL@temp{#2}%
  \chardef\SOuL@octets=%
      \ifx\SOuL@temp\SOuL@stringtwo
        \tw@
      \else\ifx\SOuL@temp\SOuL@stringthree
        \thr@@
      \else\ifx\SOuL@temp\SOuL@stringfour
        4 %
      \else
        \z@
      \fi\fi\fi
}
%    \end{macrocode}
%    \end{macro}
%
% \subsubsection{Support for \xfile{utf8x.def}}
%
%    \begin{macro}{\SOuL@analyzeutfviiix}
%    \begin{macrocode}
\begingroup
  \edef\x{\endgroup
    \def\noexpand\SOuL@analyzeutfviiix{%
      \noexpand\expandafter\noexpand\SOuL@checkutfviiix
      \noexpand\meaning\noexpand\SOUL@@
      \SOuL@stringmacrocolon\SOuL@charhash1{}{}{}{}%
      \SOuL@stringcsnameu\SOuL@stringundeferr
      \noexpand\@nil
    }%
%    \end{macrocode}
%    \end{macro}
%    \begin{macro}{\SOuL@checkutfviiix}
%    \begin{macrocode}
    \def\noexpand\SOuL@checkutfviiix
      ##1\SOuL@stringmacrocolon\SOuL@charhash1##2##3##4##5##6%
      \SOuL@stringcsnameu##7\SOuL@stringundeferr##8\noexpand\@nil
  }%
\x{%
  \def\SOuL@temp{#7}%
  \ifx\SOuL@temp\SOuL@empty
    \chardef\SOuL@octets=\z@
  \else
    \def\SOuL@temp{#5}%
    \ifx\SOuL@temp\SOuL@charthree
      \chardef\SOuL@octets=4 %
    \else
      \def\SOuL@temp{#3}%
      \ifx\SOuL@temp\SOuL@chartwo
        \chardef\SOuL@octets=\thr@@
      \else
        \chardef\SOuL@octets=\tw@
      \fi
    \fi
  \fi
}
%    \end{macrocode}
%    \end{macro}
%
% \subsection{Actions for UTF-8 sequences}
%
%    \begin{macro}{\SOuL@addtwooctets}
%    \begin{macrocode}
\def\SOuL@addtwooctets#1#2{%
  \def\SOuL@temp{#1#2}%
  \@onelevel@sanitize\SOuL@temp
  \ifx\SOuL@temp\SOuL@stringnobreakspace
    \SOUL@doword
    \SOUL@eventuallyexhyphen\null
    \SOUL@everyspace{\nobreak}%
    \let\SOuL@next\SOUL@scan
  \else
    \def\SOuL@next{%
!     \SOUL@addtoken{{\noexpand#1\noexpand#2}}%
+     \SOUL@addtoken{{\etex@unexpanded{#1#2}}}%
    }%
  \fi
  \SOuL@next
}
%    \end{macrocode}
%    \end{macro}
%    \begin{macro}{\SOuL@addthreeoctets}
%    \begin{macrocode}
\def\SOuL@addthreeoctets#1#2#3{%
  \def\SOuL@temp{#1#2#3}%
  \@onelevel@sanitize\SOuL@temp
  \ifx\SOuL@temp\SOuL@stringendash
    \SOUL@doword
    \SOUL@eventuallyexhyphen{-}%
    \SOUL@exhyphen{--}%
    \let\SOuL@next\SOUL@scan
  \else
    \ifx\SOuL@temp\SOuL@stringemdash
      \SOUL@doword
      \SOUL@eventuallyexhyphen{-}%
      \SOUL@exhyphen{---}%
      \let\SOuL@next\SOUL@scan
    \else
      \def\SOuL@next{%
!       \SOUL@addtoken{{\noexpand#1\noexpand#2\noexpand#3}}%
+       \SOUL@addtoken{{\etex@unexpanded{#1#2#3}}}%
      }%
    \fi
  \fi
  \SOuL@next
}
%    \end{macrocode}
%    \end{macro}
%    \begin{macro}{\SOuL@addfouroctets}
%    \begin{macrocode}
\def\SOuL@addfouroctets#1#2#3#4{%
! \SOUL@addtoken{{\noexpand#1\noexpand#2\noexpand#3\noexpand#4}}%
+ \SOUL@addtoken{{\etex@unexpanded{#1#2#3#4}}}%
}
%    \end{macrocode}
%    \end{macro}
%
% \subsubsection{Redefinition of \cs{SOUL@splittoken}}
%
%    \begin{macro}{\SOUL@splittoken}
%    Macro \cs{SOUL@splittoken} separates the first token
%    or token group from a word and redefines the word to contain
%    the remaining tokens. However if the remaining tokens are
%    a token group, then the curly braces will be removed and
%    the token group is splitted by the next call of \cs{SOUL@splittoken}.
%    The redefinition avoids the removal of curly braces around the
%    remaining tokens.
%    \begin{macrocode}
\SOuL@redefine\SOUL@splittoken#1#2\SOUL@stop{%
  \global\SOUL@token={#1}%
  \global\SOUL@word={#2}%
}#1{%
  \global\SOUL@token={#1}%
  \SOuL@remainingtoken\relax
}
%    \end{macrocode}
%    \end{macro}
%    \begin{macro}{\SOuL@remainingtoken}
%    \begin{macrocode}
\def\SOuL@remainingtoken#1\SOUL@stop{%
  \global\SOUL@word=\expandafter{\@gobble#1}%
}
%    \end{macrocode}
%    \end{macro}
%
% \subsection{Patches}
%
%    The fixed \cs{SOUL@splittoken} allows to remove the double
%    sets of curly braces in other macros of package \xpackage{soul}.
%    The benefit is that implicite kernings are more often detected
%    and fixes a bug in package \xpackage{soul}. The disadvantage is
%    incompatibility. The width of the resulting strings may change.
%    \begin{macro}{\SOUL@flushcomma}
%    \begin{macrocode}
\SOuL@redefine\SOUL@flushcomma{%
  \ifcase\SOUL@comma
  \or
    \edef\x{\SOUL@word={\the\SOUL@word,}}\x
  \or
    \edef\x{\SOUL@word={\the\SOUL@word{{,,}}}}\x
  \fi
  \SOUL@comma\z@
}{%
  \ifcase\SOUL@comma
  \or
    \edef\x{\SOUL@word={\the\SOUL@word,}}\x
  \or
    \edef\x{\SOUL@word={\the\SOUL@word{,,}}}\x
  \fi
  \SOUL@comma\z@
}
%    \end{macrocode}
%    \end{macro}
%    \begin{macro}{\SOUL@flushapo}
%    \begin{macrocode}
\SOuL@redefine\SOUL@flushapo{%
  \ifcase\SOUL@apo
  \or
    \edef\x{\SOUL@word={\the\SOUL@word'}}\x
  \or
    \edef\x{\SOUL@word={\the\SOUL@word{{''}}}}\x
  \fi
  \SOUL@apo\z@
}{%
  \ifcase\SOUL@apo
  \or
    \edef\x{\SOUL@word={\the\SOUL@word'}}\x
  \or
    \edef\x{\SOUL@word={\the\SOUL@word{''}}}\x
  \fi
  \SOUL@apo\z@
}
%    \end{macrocode}
%    \end{macro}
%    \begin{macro}{\SOUL@flushgrave}
%    \begin{macrocode}
\SOuL@redefine\SOUL@flushgrave{%
  \ifcase\SOUL@grave
  \or
    \edef\x{\SOUL@word={\the\SOUL@word`}}\x
  \or
    \edef\x{\SOUL@word={\the\SOUL@word{{``}}}}\x
  \fi
  \SOUL@grave\z@
}{%
  \ifcase\SOUL@grave
  \or
    \edef\x{\SOUL@word={\the\SOUL@word`}}\x
  \or
    \edef\x{\SOUL@word={\the\SOUL@word{``}}}\x
  \fi
  \SOUL@grave\z@
}
%    \end{macrocode}
%    \end{macro}
%    \begin{macro}{\SOUL@addgroup}
%    \begin{macrocode}
\SOuL@redefine\SOUL@addgroup#1#2{%
  {%
    \let\protect\noexpand
    \edef\x{%
      \global\SOUL@word={%
        \the\SOUL@word
        {{\noexpand#1#2}}%
      }%
    }%
    \x
  }%
  \SOUL@scan
}#1#2{%
  \begingroup
    \let\protect\noexpand
    \edef\x{\endgroup
      \SOUL@word={%
        \the\SOUL@word
!       {\noexpand#1{#2}}%
+       {\etex@unexpanded{#1{#2}}}%
      }%
    }%
  \x
  \SOUL@scan
}
%    \end{macrocode}
%    \end{macro}
%    \begin{macro}{\SOUL@addmath}
%    \begin{macrocode}
\SOuL@redefine\SOUL@addmath$#1${%
  {%
    \let\protect\noexpand
    \edef\x{%
      \global\SOUL@word={%
        \the\SOUL@word
        {{\hbox{$#1$}}}%
      }%
    }%
    \x
  }%
  \SOUL@scan
}$#1${%
  \begingroup
    \let\protect\noexpand
    \edef\x{\endgroup
      \SOUL@word={%
        \the\SOUL@word
!       {\hbox{$#1$}}%
+       {\etex@unexpanded{\hbox{$#1$}}}%
      }%
    }%
  \x
  \SOUL@scan
}
%    \end{macrocode}
%    \end{macro}
%    \begin{macro}{\SOUL@addprotect}
%    \begin{macrocode}
\SOuL@redefine\SOUL@addprotect#1#2{%
  {%
    \let\protect\noexpand
    \edef\x{%
      \global\SOUL@word={%
        \the\SOUL@word
        {{\hbox{#2}}}%
      }%
    }%
    \x
  }%
  \SOUL@scan
}#1#2{%
  \begingroup
    \let\protect\noexpand
    \edef\x{\endgroup
      \SOUL@word={%
        \the\SOUL@word
!       {\hbox{#2}}%
+       {\etex@unexpanded{\hbox{#2}}}%
      }%
    }%
  \x
  \SOUL@scan
}
%    \end{macrocode}
%    \end{macro}
%
%    \begin{macro}{\SOUL@addtoken}
%    \begin{macrocode}
+ \SOuL@redefine\SOUL@addtoken#1{%
+   \edef\x{%
+     \SOUL@word={%
+       \the\SOUL@word
+       \noexpand#1%
+     }%
+   }%
+   \x
+   \SOUL@scan
+ }#1{%
+   \edef\x{%
+     \SOUL@word={%
+       \the\SOUL@word
+       \etex@unexpanded{#1}%
+     }%
+   }%
+   \x
+   \SOUL@scan
+ }%
%    \end{macrocode}
%    \end{macro}
%
%    \begin{macrocode}
\SOuL@AtEnd%
%    \end{macrocode}
%
%    \begin{macrocode}
%</package>
%    \end{macrocode}
%
% \section{Test}
%
% \subsection{Catcode checks for loading}
%
%    \begin{macrocode}
%<*test1>
%    \end{macrocode}
%    \begin{macrocode}
\catcode`\{=1 %
\catcode`\}=2 %
\catcode`\#=6 %
\catcode`\@=11 %
\expandafter\ifx\csname count@\endcsname\relax
  \countdef\count@=255 %
\fi
\expandafter\ifx\csname @gobble\endcsname\relax
  \long\def\@gobble#1{}%
\fi
\expandafter\ifx\csname @firstofone\endcsname\relax
  \long\def\@firstofone#1{#1}%
\fi
\expandafter\ifx\csname loop\endcsname\relax
  \expandafter\@firstofone
\else
  \expandafter\@gobble
\fi
{%
  \def\loop#1\repeat{%
    \def\body{#1}%
    \iterate
  }%
  \def\iterate{%
    \body
      \let\next\iterate
    \else
      \let\next\relax
    \fi
    \next
  }%
  \let\repeat=\fi
}%
\def\RestoreCatcodes{}
\count@=0 %
\loop
  \edef\RestoreCatcodes{%
    \RestoreCatcodes
    \catcode\the\count@=\the\catcode\count@\relax
  }%
\ifnum\count@<255 %
  \advance\count@ 1 %
\repeat

\def\RangeCatcodeInvalid#1#2{%
  \count@=#1\relax
  \loop
    \catcode\count@=15 %
  \ifnum\count@<#2\relax
    \advance\count@ 1 %
  \repeat
}
\def\RangeCatcodeCheck#1#2#3{%
  \count@=#1\relax
  \loop
    \ifnum#3=\catcode\count@
    \else
      \errmessage{%
        Character \the\count@\space
        with wrong catcode \the\catcode\count@\space
        instead of \number#3%
      }%
    \fi
  \ifnum\count@<#2\relax
    \advance\count@ 1 %
  \repeat
}
\def\space{ }
\expandafter\ifx\csname LoadCommand\endcsname\relax
  \def\LoadCommand{\input soulutf8.sty\relax}%
\fi
\def\Test{%
  \RangeCatcodeInvalid{0}{47}%
  \RangeCatcodeInvalid{58}{64}%
  \RangeCatcodeInvalid{91}{96}%
  \RangeCatcodeInvalid{123}{255}%
  \catcode`\@=12 %
  \catcode`\\=0 %
  \catcode`\%=14 %
  \LoadCommand
  \RangeCatcodeCheck{0}{36}{15}%
  \RangeCatcodeCheck{37}{37}{14}%
  \RangeCatcodeCheck{38}{47}{15}%
  \RangeCatcodeCheck{48}{57}{12}%
  \RangeCatcodeCheck{58}{63}{15}%
  \RangeCatcodeCheck{64}{64}{12}%
  \RangeCatcodeCheck{65}{90}{11}%
  \RangeCatcodeCheck{91}{91}{15}%
  \RangeCatcodeCheck{92}{92}{0}%
  \RangeCatcodeCheck{93}{96}{15}%
  \RangeCatcodeCheck{97}{122}{11}%
  \RangeCatcodeCheck{123}{255}{15}%
  \RestoreCatcodes
}
\Test
\csname @@end\endcsname
\end
%    \end{macrocode}
%    \begin{macrocode}
%</test1>
%    \end{macrocode}
%
% \subsection{\LaTeX\space tests}
%
%    \begin{macrocode}
%<*test2|test3|test4|test5>
%    \end{macrocode}
%    \begin{macrocode}
%<noetex>\let\SavedUnexpanded\unexpanded
%<noetex>\let\unexpanded\UNDEFINED
\NeedsTeXFormat{LaTeX2e}
\documentclass{article}
\usepackage[T1]{fontenc}
\usepackage{lmodern}
\usepackage{textcomp}
\usepackage[ngerman]{babel}
\expandafter\def\csname ver@hypdestopt.sty\endcsname{}
\usepackage[unicode]{hyperref}
\usepackage{bookmark}
\usepackage{color}
\usepackage{soul}[2003/11/17]
\makeatletter
\newcommand*{\@ifDefined}[1]{%
  \@ifundefined{#1}{%
    \@latex@error{%
      \expandafter\noexpand\csname#1\endcsname undefined%
    }\@ehc
  }%
}
\newcommand*{\SaveOrg}[1]{%
  \@ifdefinable{ORG@SOUL@#1}{%
    \@ifDefined{SOUL@#1}{%
      \expandafter\let\csname ORG@SOUL@#1\expandafter\endcsname
                      \csname SOUL@#1\endcsname
    }%
  }%
}
\newcommand*{\RestoreOrg}[1]{%
  \@ifDefined{ORG@SOUL@#1}{%
    \@ifDefined{SOUL@#1}{%
      \expandafter\let\csname SOUL@#1\expandafter\endcsname
                      \csname ORG@SOUL@#1\endcsname
    }%
  }%
}
\makeatother
\SaveOrg{eval}
\SaveOrg{splittoken}
\SaveOrg{flushcomma}
\SaveOrg{flushapo}
\SaveOrg{grave}
\SaveOrg{addgroup}
\SaveOrg{addmath}
\SaveOrg{addprotect}
\SaveOrg{addtoken}
\usepackage{soulutf8}[2016/05/16]
%    \end{macrocode}
%    \begin{macrocode}
%<noetex>\let\unexpanded\SavedUnexpanded
%    \end{macrocode}
%    Input encoding \xfile{utf8}.
%    \begin{macrocode}
%<*utf8>
\usepackage[utf8]{inputenc}
\DeclareUnicodeCharacter{00A0}{\nobreakspace}
%</utf8>
%    \end{macrocode}
%    Input encoding \xfile{utf8x}. Package \xpackage{ucs}
%    has a nasty bug, that it redefines \cs{show}.
%    \begin{macrocode}
%<*utf8x>
\let\origshow\show
\usepackage[utf8x]{inputenc}
\AtBeginDocument{%
  \let\show\origshow
}
%</utf8x>
%    \end{macrocode}
%
% \subsubsection{Driver definitions}
%
%    \begin{macrocode}
\makeatletter
%    \end{macrocode}
%    \begin{macro}{\an}
%    The definition for \cs{an} is taken from the documentation of
%    package \xpackage{soul}.
%    \begin{macrocode}
\DeclareRobustCommand*{\an}{%
  \def\SOUL@preamble{$^{^P}$}%
  \def\SOUL@everyspace##1{##1\texttt{\char`\ }\linebreak[2]}%
  \def\SOUL@postamble{$^{^E}$}%
  \def\SOUL@everyhyphen{$^{^-}$\linebreak[1]}%
  \def\SOUL@everyexhyphen##1{##1$^{^=}$}%
  \def\SOUL@everysyllable{$^{^S}$}%
  \def\SOUL@everytoken{\the\SOUL@token$^{^T}$}%
  \def\SOUL@everylowerthan{$^{^L}$}%
  \SOUL@
}
%    \end{macrocode}
%    \end{macro}
%
%    \begin{macro}{\sy}
%    The definition for \cs{sy} is taken from the documentation of
%    package \xpackage{soul}.
%    \begin{macrocode}
\DeclareRobustCommand*{\sy}{%
  \SOUL@setup
  \def\SOUL@preamble{%
    \lefthyphenmin=\z@
    \righthyphenmin=\z@
  }%
  \def\SOUL@everyspace##1{##1\space}%
  \def\SOUL@everysyllable{\the\SOUL@syllable}%
  \def\SOUL@everyhyphen{%
    \discretionary{%
      \SOUL@setkern\SOUL@hyphkern
      \SOUL@sethyphenchar
    }{}{%
      \hbox{\kern1pt$\cdot$}%
    }%
  }%
  \def\SOUL@everyexhyphen##1{%
    \SOUL@setkern\SOUL@hyphkern
    \hbox{##1}%
    \discretionary{}{}{%
      \SOUL@setkern\SOUL@charkern
    }%
  }%
  \SOUL@
}
%    \end{macrocode}
%    \end{macro}
%
%    \begin{macro}{\soulH}
%    \begin{macrocode}
\DeclareRobustCommand*{\soulH}{%
  \def\SOUL@preamble{%
    \gdef\SOULH{}%
    \soulHadd{[}%
  }%
  \def\SOUL@everyspace##1{%
    \soulHadd{Sp}%
  }%
  \def\SOUL@postamble{%
    \soulHadd{]}%
  }%
  \def\SOUL@everyhyphen{%
    \soulHadd{Hy}%
  }%
  \def\SOUL@everyexhyphen##1{%
    \soulHadd{Exhy}%
  }%
  \def\SOUL@everysyllable{%
    \soulHadd{Sy}%
  }%
  \def\SOUL@everytoken{%
    \soulHadd{T}%
  }%
  \def\SOUL@everylowerthan{%
    \soulHadd{<}%
  }%
  \SOUL@
}
\newcommand*{\soulHadd}[1]{%
  #1%
  \expandafter\gdef\expandafter\soulHstring\expandafter{%
    \soulHstring
    #1%
  }%
}
%    \end{macrocode}
%    \end{macro}
%
% \subsubsection{Test macros}
%
%    \begin{macrocode}
\usepackage{qstest}
\IncludeTests{*}
\LogTests{log}{*}{*}
%    \end{macrocode}
%    \begin{macrocode}
\newsavebox\boxA
\newsavebox\boxB
\newsavebox\boxC
%    \end{macrocode}
%    \begin{macro}{\TestDriver}
%    \begin{macrocode}
\newcommand*{\TestDriver}[4]{%
  \subsection{Driver #1}%
  \@PackageInfo{soulutf8-test}{* Driver #1}%
  \expandafter\let\expandafter\testdriver\csname #1\endcsname
  \def\teststringA{#3}%
  \def\teststringB{#4}%
  \begin{qstest}{string-#2-driver-#1}{string,#2,driver,#1}%
    \@PackageInfo{soulutf8-test}{* Version: soulutf8}%
    \global\let\soulHstring\@empty
    \setbox0=\hbox{\expandafter\testdriver\expandafter{\teststringA}}%
    \let\HstringA\soulHstring
    \SetTestVBox{\boxA}{\teststringA}%
    %
    \@PackageInfo{soulutf8-test}{* Version: soul + patch}%
    \global\let\soulHstring\@empty
    \RestoreOrg{eval}%
    \setbox2=\hbox{\expandafter\testdriver\expandafter{\teststringB}}%
    \let\HstringB\soulHstring
    \SetTestVBox{\boxB}{\teststringB}%
    %
    \@PackageInfo{soulutf8-test}{* Version: original soul}%
    \global\let\soulHstring\@empty
    \RestoreOrg{splittoken}%
    \RestoreOrg{flushcomma}%
    \RestoreOrg{flushapo}%
    \RestoreOrg{grave}%
    \RestoreOrg{addgroup}%
    \RestoreOrg{addmath}%
    \RestoreOrg{addprotect}%
    \RestoreOrg{addtoken}%
    \setbox4=\hbox{\expandafter\testdriver\expandafter{\teststringB}}%
    \let\HstringC\soulHstring
    \SetTestVBox{\boxC}{\teststringB}%
    %
    \TestInfo{0}{soulutf8}%
    \TestInfo{2}{soul + patch}%
    \TestInfo{4}{original soul}%
    \TestInfoVBox
    \showboxdepth=\maxdimen
    \showboxbreadth=\maxdimen
    \tracingonline=\@ne
    \Expect*{\HstringA}*{\HstringB}%
    \Expect*{\HstringA}*{\HstringC}%
    \Expect*{\the\wd0}*{\the\wd2}%
    \ifdim\wd0=\wd2 %
    \else
      \showbox0 %
      \showbox2 %
    \fi
    \Expect*{\the\ht0}*{\the\ht2}%
    \Expect*{\the\dp0}*{\the\dp2}%
    \Expect*{\the\wd\boxB}*{\the\wd\boxA}%
    \Expect*{\the\wd\boxC}*{\the\wd\boxA}%
    \Expect*{\the\ht\boxB}*{\the\ht\boxA}%
    \ifdim\ht\boxB=\ht\boxA
    \else
      \showbox\boxA
      \showbox\boxB
    \fi
    \Expect*{\the\ht\boxC}*{\the\ht\boxA}%
    \Expect*{\the\dp\boxB}*{\the\dp\boxA}%
    \Expect*{\the\dp\boxC}*{\the\dp\boxA}%
  \end{qstest}%
}
%    \end{macrocode}
%    \end{macro}
%    \begin{macro}{\SetTestVBox}
%    \begin{macrocode}
\newcommand*{\SetTestVBox}[2]{%
  \setbox#1=\vbox{%
    \hsize=0pt\relax
    \hfuzz=100mm\relax
    \expandafter\testdriver\expandafter{#2}%
  }%
}
%    \end{macrocode}
%    \end{macro}
%    \begin{macro}{\TestInfo}
%    \begin{macrocode}
\newcommand*{\TestInfo}[2]{%
  \par
  \noindent
  \copy#1\relax
  \hfill
  \begingroup
    \small
    \ifdim\wd0=\wd2\relax
      \ifdim\wd0=\wd4\relax
      \else
        \ifnum#1=2 %
          \raisebox{.5\baselineskip}[0pt][0pt]{%
            \llap{$\}$ }%
            \rlap{width = \the\wd#1\relax\space$\{$}%
            \phantom{width = 999.99999pt}%
            \qquad
          }%
        \else
          \ifnum#1=4 %
            \rlap{width = \the\wd#1}%
            \phantom{width = 999.99999pt}%
            \qquad
          \fi
        \fi
      \fi
    \else
      \begingroup
        \bfseries
        \rlap{width = \the\wd#1}%
        \phantom{width = 999.99999pt}%
      \endgroup
      \qquad
    \fi
    \rlap{(#2)}%
    \phantom{(original soul)}%
  \endgroup
  \par
}
%    \end{macrocode}
%    \end{macro}
%    \begin{macro}{\TestInfoVBox}
%    \begin{macrocode}
\newcommand*{\TestInfoVBox}{%
  \par
  \noindent
  \mbox{\vrule\relax\space\copy\boxA}%
  \ifdim\ht\boxA=\ht\boxB
  \else
    \hfill
    \begingroup
      \small
      \rlap{(soulutf8)}%
      \phantom{(soulutf8)}%
    \endgroup
    \par
    \noindent
    \mbox{\vrule\relax\space\copy\boxB}%
    \hfill
    \begingroup
      \small
      \rlap{(soul)}%
      \phantom{(soulutf8)}%
    \endgroup
  \fi
  \par
}
%    \end{macrocode}
%    \end{macro}
%    \begin{macro}{\Test}
%    \begin{macrocode}
\makeatletter
\newcommand*{\Test}[3]{%
  \clearpage
  \begingroup
    \def\linebreak{\textbackslash linebreak }%
    \def\\{\textbackslash\textbackslash}%
    \def\soulomit##1{%
      \textbackslash soulomit\textbraceleft##1\textbraceright
    }%
    \def\mbox##1{%
      \textbackslash mbox\textbraceleft##1\textbraceright
    }%
    \def\abc{\textbackslash abc }%
    \def\texttt##1{%
      \textbackslash texttt\textbraceleft##1\textbraceright
    }%
    \def\c##1{%
      \textbackslash c\textbraceleft##1\textbraceright
    }%
%<*utf8x>
    \count@=128 %
    \loop
    \ifnum\count@<256 %
      \lccode`\~=\count@
      \lowercase{%
        \def~{\noexpand~}%
      }%
      \advance\count@\@ne
    \repeat
%</utf8x>
    \csname protected@edef\endcsname\x{\endgroup
      \noexpand\section{String #1: #2}%
    }%
  \x
  \TestDriver{so}{#1}{#2}{#3}%
  \TestDriver{caps}{#1}{#2}{#3}%
  \TestDriver{ul}{#1}{#2}{#3}%
  \TestDriver{st}{#1}{#2}{#3}%
  \TestDriver{hl}{#1}{#2}{#3}%
  \TestDriver{an}{#1}{#2}{#3}%
  \TestDriver{sy}{#1}{#2}{#3}%
  \TestDriver{soulH}{#1}{#2}{#3}%
}
\makeatother
%    \end{macrocode}
%    \end{macro}
%
% \subsubsection{Testing}
%
%    \begin{macrocode}
\begin{document}
  \Test A{}{}%
  \Test B{Hello World}{Hello World}%
  \Test C{%
    Umlauts: ^^c3^^84^^c3^^96^^c3^^9c^^c3^^a4^^c3^^b6^^c3^^bc^^c3^^9f%
  }{%
    Umlauts: \"A\"O\"U\"a\"o\"u\ss
  }%
  \Test D{Euro: ^^e2^^82^^ac}{Euro: \texteuro}%
  \Test E{Stra^^c3^^9fe}{Stra\ss e}%
  \Test F{R^^c3^^b6mer}{R\"omer}%
  \Test G{A^^e2^^80^^93B^^e2^^80^^94C}{A--B---C}%
  \Test H{X^^c2^^a0Y}{X~Y}%
  \Test I{A^^c5^^a4}{A\v T}%
  \Test J{A,B,,C,,,D,,,,E}{A,B,,C,,,D,,,,E}%
  \Test K{A'B''C'''D''''E}{A'B''C'''D''''E}%
  \Test L{A`B``C```D````E}{A`B``C```D````E}%
  \Test M{A$x$B}{A$x$B}%
  \Test N{A\mbox{abc}B}{A\mbox{abc}B}%
  \Test O{,,TA''}{,,TA''}%
  \Test P{{!`}Hola{?`}}{{!`}Hola{?`}}%
  \Test Q{re-sent\slash x}{re-sent\slash x}%
  \Test R{new\\line}{new\\line}%
  \Test S{foo\linebreak bar}{foo\linebreak bar}%
  \Test T{Virchow{sche}}{Virchow{sche}}%
  \Test U{A\soulomit{foo}B}{A\soulomit{foo}B}%
  \Test V{A\texttt{xyz}B}{A\texttt{xyz}B}%
  \Test W{ffi}{ffi}%
  \Test X{f{}f{}i}{f{}f{}i}%
  \Test Y{f\mbox{}f\mbox{}i}{f\mbox{}f\mbox{}i}%
  \Test Z{\c{\"a}}{{\c{\"a}}}%
%<*test2|test3>
  \let\abcx\UNDEFINED
  \def\abc{%
    \begingroup
      \def\abcx{x}%
      \abcx
    \endgroup
  }%
  \Test{AA}{[{x\abc x}]}{[{xxx}]}%
%</test2|test3>
%    \end{macrocode}
%
%    \begin{macrocode}
\end{document}
%    \end{macrocode}
%    \begin{macrocode}
%</test2|test3|test4|test5>
%    \end{macrocode}
%
% \section{Installation}
%
% \subsection{Download}
%
% \paragraph{Package.} This package is available on
% CTAN\footnote{\url{http://ctan.org/pkg/soulutf8}}:
% \begin{description}
% \item[\CTAN{macros/latex/contrib/oberdiek/soulutf8.dtx}] The source file.
% \item[\CTAN{macros/latex/contrib/oberdiek/soulutf8.pdf}] Documentation.
% \end{description}
%
%
% \paragraph{Bundle.} All the packages of the bundle `oberdiek'
% are also available in a TDS compliant ZIP archive. There
% the packages are already unpacked and the documentation files
% are generated. The files and directories obey the TDS standard.
% \begin{description}
% \item[\CTAN{install/macros/latex/contrib/oberdiek.tds.zip}]
% \end{description}
% \emph{TDS} refers to the standard ``A Directory Structure
% for \TeX\ Files'' (\CTAN{tds/tds.pdf}). Directories
% with \xfile{texmf} in their name are usually organized this way.
%
% \subsection{Bundle installation}
%
% \paragraph{Unpacking.} Unpack the \xfile{oberdiek.tds.zip} in the
% TDS tree (also known as \xfile{texmf} tree) of your choice.
% Example (linux):
% \begin{quote}
%   |unzip oberdiek.tds.zip -d ~/texmf|
% \end{quote}
%
% \paragraph{Script installation.}
% Check the directory \xfile{TDS:scripts/oberdiek/} for
% scripts that need further installation steps.
% Package \xpackage{attachfile2} comes with the Perl script
% \xfile{pdfatfi.pl} that should be installed in such a way
% that it can be called as \texttt{pdfatfi}.
% Example (linux):
% \begin{quote}
%   |chmod +x scripts/oberdiek/pdfatfi.pl|\\
%   |cp scripts/oberdiek/pdfatfi.pl /usr/local/bin/|
% \end{quote}
%
% \subsection{Package installation}
%
% \paragraph{Unpacking.} The \xfile{.dtx} file is a self-extracting
% \docstrip\ archive. The files are extracted by running the
% \xfile{.dtx} through \plainTeX:
% \begin{quote}
%   \verb|tex soulutf8.dtx|
% \end{quote}
%
% \paragraph{TDS.} Now the different files must be moved into
% the different directories in your installation TDS tree
% (also known as \xfile{texmf} tree):
% \begin{quote}
% \def\t{^^A
% \begin{tabular}{@{}>{\ttfamily}l@{ $\rightarrow$ }>{\ttfamily}l@{}}
%   soulutf8.sty & tex/generic/oberdiek/soulutf8.sty\\
%   soulutf8.pdf & doc/latex/oberdiek/soulutf8.pdf\\
%   test/soulutf8-test1.tex & doc/latex/oberdiek/test/soulutf8-test1.tex\\
%   test/soulutf8-test2.tex & doc/latex/oberdiek/test/soulutf8-test2.tex\\
%   test/soulutf8-test3.tex & doc/latex/oberdiek/test/soulutf8-test3.tex\\
%   test/soulutf8-test4.tex & doc/latex/oberdiek/test/soulutf8-test4.tex\\
%   test/soulutf8-test5.tex & doc/latex/oberdiek/test/soulutf8-test5.tex\\
%   soulutf8.dtx & source/latex/oberdiek/soulutf8.dtx\\
% \end{tabular}^^A
% }^^A
% \sbox0{\t}^^A
% \ifdim\wd0>\linewidth
%   \begingroup
%     \advance\linewidth by\leftmargin
%     \advance\linewidth by\rightmargin
%   \edef\x{\endgroup
%     \def\noexpand\lw{\the\linewidth}^^A
%   }\x
%   \def\lwbox{^^A
%     \leavevmode
%     \hbox to \linewidth{^^A
%       \kern-\leftmargin\relax
%       \hss
%       \usebox0
%       \hss
%       \kern-\rightmargin\relax
%     }^^A
%   }^^A
%   \ifdim\wd0>\lw
%     \sbox0{\small\t}^^A
%     \ifdim\wd0>\linewidth
%       \ifdim\wd0>\lw
%         \sbox0{\footnotesize\t}^^A
%         \ifdim\wd0>\linewidth
%           \ifdim\wd0>\lw
%             \sbox0{\scriptsize\t}^^A
%             \ifdim\wd0>\linewidth
%               \ifdim\wd0>\lw
%                 \sbox0{\tiny\t}^^A
%                 \ifdim\wd0>\linewidth
%                   \lwbox
%                 \else
%                   \usebox0
%                 \fi
%               \else
%                 \lwbox
%               \fi
%             \else
%               \usebox0
%             \fi
%           \else
%             \lwbox
%           \fi
%         \else
%           \usebox0
%         \fi
%       \else
%         \lwbox
%       \fi
%     \else
%       \usebox0
%     \fi
%   \else
%     \lwbox
%   \fi
% \else
%   \usebox0
% \fi
% \end{quote}
% If you have a \xfile{docstrip.cfg} that configures and enables \docstrip's
% TDS installing feature, then some files can already be in the right
% place, see the documentation of \docstrip.
%
% \subsection{Refresh file name databases}
%
% If your \TeX~distribution
% (\teTeX, \mikTeX, \dots) relies on file name databases, you must refresh
% these. For example, \teTeX\ users run \verb|texhash| or
% \verb|mktexlsr|.
%
% \subsection{Some details for the interested}
%
% \paragraph{Attached source.}
%
% The PDF documentation on CTAN also includes the
% \xfile{.dtx} source file. It can be extracted by
% AcrobatReader 6 or higher. Another option is \textsf{pdftk},
% e.g. unpack the file into the current directory:
% \begin{quote}
%   \verb|pdftk soulutf8.pdf unpack_files output .|
% \end{quote}
%
% \paragraph{Unpacking with \LaTeX.}
% The \xfile{.dtx} chooses its action depending on the format:
% \begin{description}
% \item[\plainTeX:] Run \docstrip\ and extract the files.
% \item[\LaTeX:] Generate the documentation.
% \end{description}
% If you insist on using \LaTeX\ for \docstrip\ (really,
% \docstrip\ does not need \LaTeX), then inform the autodetect routine
% about your intention:
% \begin{quote}
%   \verb|latex \let\install=y\input{soulutf8.dtx}|
% \end{quote}
% Do not forget to quote the argument according to the demands
% of your shell.
%
% \paragraph{Generating the documentation.}
% You can use both the \xfile{.dtx} or the \xfile{.drv} to generate
% the documentation. The process can be configured by the
% configuration file \xfile{ltxdoc.cfg}. For instance, put this
% line into this file, if you want to have A4 as paper format:
% \begin{quote}
%   \verb|\PassOptionsToClass{a4paper}{article}|
% \end{quote}
% An example follows how to generate the
% documentation with pdf\LaTeX:
% \begin{quote}
%\begin{verbatim}
%pdflatex soulutf8.dtx
%makeindex -s gind.ist soulutf8.idx
%pdflatex soulutf8.dtx
%makeindex -s gind.ist soulutf8.idx
%pdflatex soulutf8.dtx
%\end{verbatim}
% \end{quote}
%
% \begin{thebibliography}{9}
%
% \bibitem{soul}
%   Melchior Franz: \textit{The \xpackage{soul} package};
%   2003/11/17;\\
%   \CTAN{macros/latex/contrib/soul/}.
%
% \bibitem{ucs}
%   Dominique P. G. Unruh: \textit{\xpackage{ucs.sty} -- Unicode Support};
%   2004/10/17;\\
%   \CTAN{macros/latex/contrib/unicode/}.
%
% \bibitem{utf8}
%   Frank Mittelbach, Chris Rowley:
%   \textit{Providing some UTF-8 support via \xpackage{inputenc}};
%   2006/03/30;\\
%   \CTAN{macros/latex/base/utf8ienc.dtx}.
%
% \end{thebibliography}
%
% \begin{History}
%   \begin{Version}{2007/09/09 v1.0}
%   \item
%     First version.
%   \end{Version}
%   \begin{Version}{2016/05/16 v1.1}
%   \item
%     Documentation updates.
%   \end{Version}
% \end{History}
%
% \PrintIndex
%
% \Finale
\endinput
|
% \end{quote}
% Do not forget to quote the argument according to the demands
% of your shell.
%
% \paragraph{Generating the documentation.}
% You can use both the \xfile{.dtx} or the \xfile{.drv} to generate
% the documentation. The process can be configured by the
% configuration file \xfile{ltxdoc.cfg}. For instance, put this
% line into this file, if you want to have A4 as paper format:
% \begin{quote}
%   \verb|\PassOptionsToClass{a4paper}{article}|
% \end{quote}
% An example follows how to generate the
% documentation with pdf\LaTeX:
% \begin{quote}
%\begin{verbatim}
%pdflatex soulutf8.dtx
%makeindex -s gind.ist soulutf8.idx
%pdflatex soulutf8.dtx
%makeindex -s gind.ist soulutf8.idx
%pdflatex soulutf8.dtx
%\end{verbatim}
% \end{quote}
%
% \begin{thebibliography}{9}
%
% \bibitem{soul}
%   Melchior Franz: \textit{The \xpackage{soul} package};
%   2003/11/17;\\
%   \CTAN{macros/latex/contrib/soul/}.
%
% \bibitem{ucs}
%   Dominique P. G. Unruh: \textit{\xpackage{ucs.sty} -- Unicode Support};
%   2004/10/17;\\
%   \CTAN{macros/latex/contrib/unicode/}.
%
% \bibitem{utf8}
%   Frank Mittelbach, Chris Rowley:
%   \textit{Providing some UTF-8 support via \xpackage{inputenc}};
%   2006/03/30;\\
%   \CTAN{macros/latex/base/utf8ienc.dtx}.
%
% \end{thebibliography}
%
% \begin{History}
%   \begin{Version}{2007/09/09 v1.0}
%   \item
%     First version.
%   \end{Version}
%   \begin{Version}{2016/05/16 v1.1}
%   \item
%     Documentation updates.
%   \end{Version}
% \end{History}
%
% \PrintIndex
%
% \Finale
\endinput
|
% \end{quote}
% Do not forget to quote the argument according to the demands
% of your shell.
%
% \paragraph{Generating the documentation.}
% You can use both the \xfile{.dtx} or the \xfile{.drv} to generate
% the documentation. The process can be configured by the
% configuration file \xfile{ltxdoc.cfg}. For instance, put this
% line into this file, if you want to have A4 as paper format:
% \begin{quote}
%   \verb|\PassOptionsToClass{a4paper}{article}|
% \end{quote}
% An example follows how to generate the
% documentation with pdf\LaTeX:
% \begin{quote}
%\begin{verbatim}
%pdflatex soulutf8.dtx
%makeindex -s gind.ist soulutf8.idx
%pdflatex soulutf8.dtx
%makeindex -s gind.ist soulutf8.idx
%pdflatex soulutf8.dtx
%\end{verbatim}
% \end{quote}
%
% \begin{thebibliography}{9}
%
% \bibitem{soul}
%   Melchior Franz: \textit{The \xpackage{soul} package};
%   2003/11/17;\\
%   \CTAN{macros/latex/contrib/soul/}.
%
% \bibitem{ucs}
%   Dominique P. G. Unruh: \textit{\xpackage{ucs.sty} -- Unicode Support};
%   2004/10/17;\\
%   \CTAN{macros/latex/contrib/unicode/}.
%
% \bibitem{utf8}
%   Frank Mittelbach, Chris Rowley:
%   \textit{Providing some UTF-8 support via \xpackage{inputenc}};
%   2006/03/30;\\
%   \CTAN{macros/latex/base/utf8ienc.dtx}.
%
% \end{thebibliography}
%
% \begin{History}
%   \begin{Version}{2007/09/09 v1.0}
%   \item
%     First version.
%   \end{Version}
%   \begin{Version}{2016/05/16 v1.1}
%   \item
%     Documentation updates.
%   \end{Version}
% \end{History}
%
% \PrintIndex
%
% \Finale
\endinput

%        (quote the arguments according to the demands of your shell)
%
% Documentation:
%    (a) If soulutf8.drv is present:
%           latex soulutf8.drv
%    (b) Without soulutf8.drv:
%           latex soulutf8.dtx; ...
%    The class ltxdoc loads the configuration file ltxdoc.cfg
%    if available. Here you can specify further options, e.g.
%    use A4 as paper format:
%       \PassOptionsToClass{a4paper}{article}
%
%    Programm calls to get the documentation (example):
%       pdflatex soulutf8.dtx
%       makeindex -s gind.ist soulutf8.idx
%       pdflatex soulutf8.dtx
%       makeindex -s gind.ist soulutf8.idx
%       pdflatex soulutf8.dtx
%
% Installation:
%    TDS:tex/generic/oberdiek/soulutf8.sty
%    TDS:doc/latex/oberdiek/soulutf8.pdf
%    TDS:source/latex/oberdiek/soulutf8.dtx
%
%<*ignore>
\begingroup
  \catcode123=1 %
  \catcode125=2 %
  \def\x{LaTeX2e}%
\expandafter\endgroup
\ifcase 0\ifx\install y1\fi\expandafter
         \ifx\csname processbatchFile\endcsname\relax\else1\fi
         \ifx\fmtname\x\else 1\fi\relax
\else\csname fi\endcsname
%</ignore>
%<*install>
\input docstrip.tex
\Msg{************************************************************************}
\Msg{* Installation}
\Msg{* Package: soulutf8 2016/05/16 v1.1 Permit use of UTF-8 characters in soul (HO)}
\Msg{************************************************************************}

\keepsilent
\askforoverwritefalse

\let\MetaPrefix\relax
\preamble

This is a generated file.

Project: soulutf8
Version: 2016/05/16 v1.1

Copyright (C)
   2007 Heiko Oberdiek
   2016-2019 Oberdiek Package Support Group

This work may be distributed and/or modified under the
conditions of the LaTeX Project Public License, either
version 1.3c of this license or (at your option) any later
version. This version of this license is in
   https://www.latex-project.org/lppl/lppl-1-3c.txt
and the latest version of this license is in
   https://www.latex-project.org/lppl.txt
and version 1.3 or later is part of all distributions of
LaTeX version 2005/12/01 or later.

This work has the LPPL maintenance status "maintained".

The Current Maintainers of this work are
Heiko Oberdiek and the Oberdiek Package Support Group
https://github.com/ho-tex/oberdiek/issues


The Base Interpreter refers to any `TeX-Format',
because some files are installed in TDS:tex/generic//.

This work consists of the main source file soulutf8.dtx
and the derived files
   soulutf8.sty, soulutf8.pdf, soulutf8.ins, soulutf8.drv,
   soulutf8-test1.tex, soulutf8-test2.tex, soulutf8-test3.tex,
   soulutf8-test4.tex, soulutf8-test5.tex.

\endpreamble
\let\MetaPrefix\DoubleperCent

\generate{%
  \file{soulutf8.ins}{\from{soulutf8.dtx}{install}}%
  \file{soulutf8.drv}{\from{soulutf8.dtx}{driver}}%
  \usedir{tex/generic/oberdiek}%
  \file{soulutf8.sty}{\from{soulutf8.dtx}{package}}%
%  \usedir{doc/latex/oberdiek/test}%
%  \file{soulutf8-test1.tex}{\from{soulutf8.dtx}{test1}}%
%  \file{soulutf8-test2.tex}{\from{soulutf8.dtx}{test2,utf8}}%
%  \file{soulutf8-test3.tex}{\from{soulutf8.dtx}{test3,utf8x}}%
%  \file{soulutf8-test4.tex}{\from{soulutf8.dtx}{test4,utf8,noetex}}%
%  \file{soulutf8-test5.tex}{\from{soulutf8.dtx}{test5,utf8x,noetex}}%
}

\catcode32=13\relax% active space
\let =\space%
\Msg{************************************************************************}
\Msg{*}
\Msg{* To finish the installation you have to move the following}
\Msg{* file into a directory searched by TeX:}
\Msg{*}
\Msg{*     soulutf8.sty}
\Msg{*}
\Msg{* To produce the documentation run the file `soulutf8.drv'}
\Msg{* through LaTeX.}
\Msg{*}
\Msg{* Happy TeXing!}
\Msg{*}
\Msg{************************************************************************}

\endbatchfile
%</install>
%<*ignore>
\fi
%</ignore>
%<*driver>
\NeedsTeXFormat{LaTeX2e}
\ProvidesFile{soulutf8.drv}%
  [2016/05/16 v1.1 Permit use of UTF-8 characters in soul (HO)]%
\documentclass{ltxdoc}
\usepackage{holtxdoc}[2011/11/22]
\begin{document}
  \DocInput{soulutf8.dtx}%
\end{document}
%</driver>
% \fi
%
%
%
% \GetFileInfo{soulutf8.drv}
%
% \title{The \xpackage{soulutf8} package}
% \date{2016/05/16 v1.1}
% \author{Heiko Oberdiek\thanks
% {Please report any issues at \url{https://github.com/ho-tex/oberdiek/issues}}}
%
% \maketitle
%
% \begin{abstract}
% This package extends package \xpackage{soul} and adds some support
% for UTF-8. Namely the input encodings \xfile{utf8.def}
% from package \xpackage{inputenc} and
% package \xpackage{ucs}'s \xfile{utf8x.def} are supported.
% \end{abstract}
%
% \tableofcontents
%
% \section{Documentation}
%
% This package \xpackage{soulutf8} does not have own options and does not
% define new user commands.
% Any option is passed to package \xpackage{soul} \cite{soul}
% that is loaded first.
% Then some internal macros of \xpackage{soul} are redefined to add
% support for UTF-8.
% The following input encodings are supported:
% \begin{quote}
%   \begin{tabular}{@{}lll@{}}
%     \texttt{utf8} & \LaTeX\ base &
%        \xfile{TDS:tex/latex/base/utf8.def} \cite{utf8}\\
%     \texttt{utf8x} & Package \xpackage{ucs} &
%       \xfile{TDS:tex/latex/ucs/utf8x.def} \cite{ucs}\\
%   \end{tabular}
% \end{quote}
% UTF-8 byte sequences are added as token group to a word, even
% if these UTF-8 characters are some kind of hyphen or space.
% As exception the following three Unicode characters are handled specially:
% \begin{quote}
%   \begin{tabular}{lll}
%     Slot & Name & Action\\
%     \hline
%     \texttt{U+00A0} & NO-BREAK SPACE & like |~|\\
%     \texttt{U+2013} & EN DASH & |--|\\
%     \texttt{U+2014} & EM DASH & |---|
%   \end{tabular}
%
% \subsection{Patch}
%
% Also package \xpackage{soulutf8} tries to patch package \xpackage{soul}
% to improve its behaviour:
% \begin{itemize}
% \item
%   A problem with additional levels of curly braces is fixed.
%   As advantage more implicite kernings are detected.
%   However, the result may be incompatible with the
%   original behaviour of package \xpackage{soul} because
%   of these respected implicite kernings.
% \item
%   \eTeX\ , especially \cs{unexpanded} is supported.
%   This allows a better protection of token groups
%   (|\mbox{|\dots|}|, math, \dots).
% \end{itemize}
% \end{quote}
%
% \subsection{Future}
%
% Currently package \xpackage{soul} does not seem to be
% maintained. Nevertheless if there will be a new version that
% adds support for UTF-8, then this package may become
% obsolete.
%
% \StopEventually{
% }
%
% \section{Implementation}
%
%    \begin{macrocode}
%<*package>
%    \end{macrocode}
%
% \subsection{Reload check and package identification}
%    Reload check, especially if the package is not used with \LaTeX.
%    \begin{macrocode}
\begingroup\catcode61\catcode48\catcode32=10\relax%
  \catcode13=5 % ^^M
  \endlinechar=13 %
  \catcode35=6 % #
  \catcode39=12 % '
  \catcode44=12 % ,
  \catcode45=12 % -
  \catcode46=12 % .
  \catcode58=12 % :
  \catcode64=11 % @
  \catcode123=1 % {
  \catcode125=2 % }
  \expandafter\let\expandafter\x\csname ver@soulutf8.sty\endcsname
  \ifx\x\relax % plain-TeX, first loading
  \else
    \def\empty{}%
    \ifx\x\empty % LaTeX, first loading,
      % variable is initialized, but \ProvidesPackage not yet seen
    \else
      \expandafter\ifx\csname PackageInfo\endcsname\relax
        \def\x#1#2{%
          \immediate\write-1{Package #1 Info: #2.}%
        }%
      \else
        \def\x#1#2{\PackageInfo{#1}{#2, stopped}}%
      \fi
      \x{soulutf8}{The package is already loaded}%
      \aftergroup\endinput
    \fi
  \fi
\endgroup%
%    \end{macrocode}
%    Package identification:
%    \begin{macrocode}
\begingroup\catcode61\catcode48\catcode32=10\relax%
  \catcode13=5 % ^^M
  \endlinechar=13 %
  \catcode35=6 % #
  \catcode39=12 % '
  \catcode40=12 % (
  \catcode41=12 % )
  \catcode44=12 % ,
  \catcode45=12 % -
  \catcode46=12 % .
  \catcode47=12 % /
  \catcode58=12 % :
  \catcode64=11 % @
  \catcode91=12 % [
  \catcode93=12 % ]
  \catcode123=1 % {
  \catcode125=2 % }
  \expandafter\ifx\csname ProvidesPackage\endcsname\relax
    \def\x#1#2#3[#4]{\endgroup
      \immediate\write-1{Package: #3 #4}%
      \xdef#1{#4}%
    }%
  \else
    \def\x#1#2[#3]{\endgroup
      #2[{#3}]%
      \ifx#1\@undefined
        \xdef#1{#3}%
      \fi
      \ifx#1\relax
        \xdef#1{#3}%
      \fi
    }%
  \fi
\expandafter\x\csname ver@soulutf8.sty\endcsname
\ProvidesPackage{soulutf8}%
  [2016/05/16 v1.1 Permit use of UTF-8 characters in soul (HO)]%
%    \end{macrocode}
%
% \subsection{Catcodes}
%
%    \begin{macrocode}
\begingroup\catcode61\catcode48\catcode32=10\relax%
  \catcode13=5 % ^^M
  \endlinechar=13 %
  \catcode123=1 % {
  \catcode125=2 % }
  \catcode64=11 % @
  \def\x{\endgroup
    \expandafter\edef\csname SOuL@AtEnd\endcsname{%
      \endlinechar=\the\endlinechar\relax
      \catcode13=\the\catcode13\relax
      \catcode32=\the\catcode32\relax
      \catcode35=\the\catcode35\relax
      \catcode61=\the\catcode61\relax
      \catcode64=\the\catcode64\relax
      \catcode123=\the\catcode123\relax
      \catcode125=\the\catcode125\relax
    }%
  }%
\x\catcode61\catcode48\catcode32=10\relax%
\catcode13=5 % ^^M
\endlinechar=13 %
\catcode35=6 % #
\catcode64=11 % @
\catcode123=1 % {
\catcode125=2 % }
\def\TMP@EnsureCode#1#2{%
  \edef\SOuL@AtEnd{%
    \SOuL@AtEnd
    \catcode#1=\the\catcode#1\relax
  }%
  \catcode#1=#2\relax
}
\TMP@EnsureCode{10}{12}% ^^J
\TMP@EnsureCode{33}{12}% !
\TMP@EnsureCode{34}{12}% "
\TMP@EnsureCode{36}{3}% $
\TMP@EnsureCode{39}{12}% '
\TMP@EnsureCode{40}{12}% (
\TMP@EnsureCode{41}{12}% )
\TMP@EnsureCode{42}{12}% *
\TMP@EnsureCode{43}{12}% +
\TMP@EnsureCode{44}{12}% ,
\TMP@EnsureCode{45}{12}% -
\TMP@EnsureCode{46}{12}% .
\TMP@EnsureCode{47}{12}% /
\TMP@EnsureCode{58}{12}% :
\TMP@EnsureCode{60}{12}% <
\TMP@EnsureCode{62}{12}% >
\TMP@EnsureCode{91}{12}% [
\TMP@EnsureCode{93}{12}% ]
\TMP@EnsureCode{94}{7}% ^
\TMP@EnsureCode{96}{12}% `
\TMP@EnsureCode{126}\active % ~
\TMP@EnsureCode{128}{12}% ^^80
\TMP@EnsureCode{147}{12}% ^^93
\TMP@EnsureCode{148}{12}% ^^94
\TMP@EnsureCode{160}{12}% ^^a0
\TMP@EnsureCode{194}{12}% ^^c2
\TMP@EnsureCode{226}{12}% ^^e2
\edef\SOuL@AtEnd{\SOuL@AtEnd\noexpand\endinput}
%    \end{macrocode}
%
% \subsection{Loading packages}
%
%    Package \xpackage{soul} uses \cs{documentclass} to detect \LaTeX.
%    \begin{macrocode}
\ifx\documentclass\@undefined
%    \end{macrocode}
%
% \subsubsection{\plainTeX}
%
%    First we check, whether package \xpackage{soul} is already loaded.
%    \begin{macrocode}
  \expandafter\ifx\csname SOUL@\endcsname\relax
%    \end{macrocode}
%    In case of \plainTeX\ package \xpackage{soul} defines some
%    macros in a simple manner that will break the definitions of
%    \xfile{miniltx.tex}, for example. Therefore these macros
%    are first saved and restored afterwards.
%    \begin{macrocode}
    \let\SOuL@orgDeclareRobustCommand\DeclareRobustCommand
    \let\SOuL@orgnewcommand          \newcommand
    \let\SOuL@orgDeclareOption       \DeclareOption
    \let\SOuL@orgPackageError        \PackageError
    \def\SOuL@restorelatexcmds{%
      \let\DeclareRobustCommand\SOuL@orgDeclareRobustCommand
      \let\newcommand          \SOuL@orgnewcommand
      \let\DeclareOption       \SOuL@orgDeclareOption
      \let\PackageError        \SOuL@orgPackageError
    }%
    \input soul.sty\relax
    \SOuL@restorelatexcmds
  \fi
%    \end{macrocode}
%    \begin{macro}{\SOUL@error}
%     Package \xpackage{soul}'s use of \cs{PackageError} is replaced
%     by \cs{@PackageError} of package \xpackage{infwarerr}.
%    \begin{macrocode}
  \input infwarerr.sty\relax
  \let\SOuL@orgSOUL@error\SOUL@error
  \def\SOUL@error{%
    \begingroup
      \let\PackageError\@PackageError
      \SOuL@orgSOUL@error
    \endgroup
  }%
%    \end{macrocode}
%    \end{macro}
%    \begin{macrocode}
  \input etexcmds.sty\relax
%    \end{macrocode}
%    \begin{macro}{\@onelevel@sanitize}
%    Define \LaTeX's \cs{@onelevel@sanitize} if not already available.
%    \begin{macrocode}
  \expandafter\ifx\csname @onelevel@sanitize\endcsname\relax
    \def\@onelevel@sanitize#1{%
      \edef#1{%
        \expandafter\strip@prefix\meaning#1%
      }%
    }%
%    \end{macrocode}
%    \end{macro}
%    \begin{macro}{\strip@prefix}
%    \begin{macrocode}
    \def\strip@prefix#1>{}%
%    \end{macrocode}
%    \end{macro}
%    \begin{macrocode}
  \fi
\else
%    \end{macrocode}
%
% \subsubsection{\LaTeX}
%    \begin{macrocode}
  \DeclareOption*{\PassOptionsToPackage{\CurrentOption}{soul}}%
  \ProcessOptions\relax
  \RequirePackage{soul}[2003/11/17]%
  \RequirePackage{infwarerr}[2016/05/16]%
  \RequirePackage{etexcmds}[2016/05/16]%
\fi
%    \end{macrocode}
%
% \subsubsection{\eTeX}
%
%    In \plainTeX\ command \cs{+} is an \emph{outer} macro.
%    Therefore numbers are used to avoid problems.
%    \begin{macrocode}
\ifetex@unexpanded
  \catcode33=14 % '!': comment
  \catcode43=9  % '+': ignore
\else
  \catcode33=9  % '!': ignore
  \catcode43=14 % '+': comment
\fi
%    \end{macrocode}
%
% \subsection{Macro for redefinitions}
%
%    \begin{macro}{\SOuL@redefine}
%    \begin{macrocode}
\def\SOuL@redefine#1{%
  \begingroup
    \def\SOuL@cmd{#1}%
    \afterassignment\SOuL@cmdcheck
    \def\SOuL@temp
}
%    \end{macrocode}
%    \end{macro}
%    \begin{macro}{\SOuL@cmdcheck}
%    \begin{macrocode}
\def\SOuL@cmdcheck{%
    \expandafter\ifx\SOuL@cmd\SOuL@temp
    \else
      \edef\SOuL@temp*{\expandafter\string\SOuL@cmd}%
      \@PackageWarningNoLine{soulutf8}{%
        Command \SOuL@temp* has changed.\MessageBreak
        Supported versions of package `soul': 2003/11/17.\MessageBreak
        Depending on the unknown changes the redefinition\MessageBreak
        of \SOuL@temp* may not behave correctly%
      }%
    \fi
  \expandafter\endgroup
  \expandafter\def\SOuL@cmd
}
%    \end{macrocode}
%    \end{macro}
%
% \subsection{Redefinition of \cs{SOUL@eval}}
%
%    \begin{macro}{\SOUL@eval}
%    Macro \cs{SOUL@eval} is redefined to add detection of
%    the first byte of a UTF-8 sequence. Because \cs{SOUL@eval}
%    is overwritten, a warning is issued, if the contents of
%    \cs{SOUL@eval} is not as expected.
%    \begin{macrocode}
\SOuL@redefine\SOUL@eval{%
%    \end{macrocode}
%    First the expected definition.
%    \begin{macrocode}
  \def\SOUL@n*##1{\SOUL@scan}%
  \if\noexpand\SOUL@@\SOUL@spc
  \else
    \SOUL@ignorespacesfalse
  \fi
  \ifnum\SOUL@minus=\thr@@
    \SOUL@flushminus
  \else\ifnum\SOUL@comma=\tw@
    \SOUL@flushcomma
  \else\ifnum\SOUL@apo=\tw@
    \SOUL@flushapo
  \else\ifnum\SOUL@grave=\tw@
    \SOUL@flushgrave
  \fi\fi\fi\fi
  \ifx\SOUL@@-\else\SOUL@flushminus\fi
  \ifx\SOUL@@,\else\SOUL@flushcomma\fi
  \ifx\SOUL@@'\else\SOUL@flushapo\fi
  \ifx\SOUL@@`\else\SOUL@flushgrave\fi
  \ifx\SOUL@@-%
    \advance\SOUL@minus\@ne
  \else\ifx\SOUL@@,%
    \advance\SOUL@comma\@ne
  \else\ifx\SOUL@@'%
    \advance\SOUL@apo\@ne
  \else\ifx\SOUL@@`%
    \advance\SOUL@grave\@ne
  \else
    \SOUL@flushminus
    \SOUL@flushcomma
    \SOUL@flushapo
    \SOUL@flushgrave
    \ifx\SOUL@@\SOUL@stop
      \def\SOUL@n*{%
        \SOUL@doword
        \SOUL@eventuallyexhyphen\null
      }%
    \else\ifx\SOUL@@\par
      \def\SOUL@n*\par{\par\leavevmode\SOUL@scan}%
    \else\if\noexpand\SOUL@@\SOUL@spc
      \SOUL@doword
      \SOUL@eventuallyexhyphen\null
      \ifSOUL@ignorespaces
      \else
        \SOUL@everyspace{}%
      \fi
      \def\SOUL@n* {\SOUL@scan}%
    \else\ifx\SOUL@@\\%
      \SOUL@doword
      \SOUL@eventuallyexhyphen\null
      \SOUL@everyspace{\unskip\nobreak\hfil\break}%
      \SOUL@ignorespacestrue
    \else\ifx\SOUL@@~%
      \SOUL@doword
      \SOUL@eventuallyexhyphen\null
      \SOUL@everyspace{\nobreak}%
    \else\ifx\SOUL@@\slash
      \SOUL@doword
      \SOUL@eventuallyexhyphen{/}%
      \SOUL@exhyphen{/}%
    \else\ifx\SOUL@@\mbox
      \def\SOUL@n*{\SOUL@addprotect}%
    \else\ifx\SOUL@@\hbox
      \def\SOUL@n*{\SOUL@addprotect}%
    \else\ifx\SOUL@@\soulomit
      \def\SOUL@n*\soulomit##1{%
        \SOUL@doword
        {\spaceskip\SOUL@spaceskip##1}%
        \SOUL@scan
      }%
    \else\ifx\SOUL@@\break
      \SOUL@doword
      \break
    \else\ifx\SOUL@@\linebreak
      \SOUL@doword
      \SOUL@everyspace{\linebreak}%
    \else\ifcat\bgroup\noexpand\SOUL@@
      \def\SOUL@n*{\SOUL@addgroup{}}%
    \else\ifcat$\noexpand\SOUL@@
      \def\SOUL@n*{\SOUL@addmath}%
    \else
      \def\SOUL@n*{\SOUL@dotoken}%
    \fi\fi\fi\fi\fi\fi\fi\fi\fi\fi\fi\fi\fi
  \fi\fi\fi\fi
  \SOUL@n*%
}{%
%    \end{macrocode}
%    Now the redefined version follows.
%    \begin{macrocode}
  \def\SOUL@n*##1{\SOUL@scan}%
  \if\noexpand\SOUL@@\SOUL@spc
  \else
    \SOUL@ignorespacesfalse
  \fi
  \ifnum\SOUL@minus=\thr@@
    \SOUL@flushminus
  \else\ifnum\SOUL@comma=\tw@
    \SOUL@flushcomma
  \else\ifnum\SOUL@apo=\tw@
    \SOUL@flushapo
  \else\ifnum\SOUL@grave=\tw@
    \SOUL@flushgrave
  \fi\fi\fi\fi
  \ifx\SOUL@@-\else\SOUL@flushminus\fi
  \ifx\SOUL@@,\else\SOUL@flushcomma\fi
  \ifx\SOUL@@'\else\SOUL@flushapo\fi
  \ifx\SOUL@@`\else\SOUL@flushgrave\fi
  \ifx\SOUL@@-%
    \advance\SOUL@minus\@ne
  \else\ifx\SOUL@@,%
    \advance\SOUL@comma\@ne
  \else\ifx\SOUL@@'%
    \advance\SOUL@apo\@ne
  \else\ifx\SOUL@@`%
    \advance\SOUL@grave\@ne
  \else
    \SOUL@flushminus
    \SOUL@flushcomma
    \SOUL@flushapo
    \SOUL@flushgrave
    \ifx\SOUL@@\SOUL@stop
      \def\SOUL@n*{%
          \SOUL@doword
          \SOUL@eventuallyexhyphen\null
      }%
    \else\ifx\SOUL@@\par
      \def\SOUL@n*\par{\par\leavevmode\SOUL@scan}%
    \else\if\noexpand\SOUL@@\SOUL@spc
      \SOUL@doword
      \SOUL@eventuallyexhyphen\null
      \ifSOUL@ignorespaces
      \else
        \SOUL@everyspace{}%
      \fi
      \def\SOUL@n* {\SOUL@scan}%
    \else\ifx\SOUL@@\\%
      \SOUL@doword
      \SOUL@eventuallyexhyphen\null
      \SOUL@everyspace{\unskip\nobreak\hfil\break}%
      \SOUL@ignorespacestrue
    \else\ifx\SOUL@@~%
      \SOUL@doword
      \SOUL@eventuallyexhyphen\null
      \SOUL@everyspace{\nobreak}%
    \else\ifx\SOUL@@\slash
      \SOUL@doword
      \SOUL@eventuallyexhyphen{/}%
      \SOUL@exhyphen{/}%
    \else\ifx\SOUL@@\mbox
      \def\SOUL@n*{\SOUL@addprotect}%
    \else\ifx\SOUL@@\hbox
      \def\SOUL@n*{\SOUL@addprotect}%
    \else\ifx\SOUL@@\soulomit
      \def\SOUL@n*\soulomit##1{%
        \SOUL@doword
        {\spaceskip\SOUL@spaceskip##1}%
        \SOUL@scan
      }%
    \else\ifx\SOUL@@\break
      \SOUL@doword
      \break
    \else\ifx\SOUL@@\linebreak
      \SOUL@doword
      \SOUL@everyspace{\linebreak}%
    \else\ifcat\bgroup\noexpand\SOUL@@
      \def\SOUL@n*{\SOUL@addgroup{}}%
    \else\ifcat$\noexpand\SOUL@@
      \def\SOUL@n*{\SOUL@addmath}%
    \else
%    \end{macrocode}
%    The current token is examined to detect the start of
%    a UTF-8 sequence.
%    \begin{macrocode}
      \SOuL@analyzeutfviii
      \ifcase\SOuL@octets
        \SOuL@analyzeutfviiix
      \fi
      \ifcase\SOuL@octets
        \def\SOUL@n*{\SOUL@dotoken}%
      \or % 1
      \or % 2
        \def\SOUL@n*{\SOuL@addtwooctets}%
      \or % 3
        \def\SOUL@n*{\SOuL@addthreeoctets}%
      \or % 4
        \def\SOUL@n*{\SOuL@addfouroctets}%
      \fi
    \fi\fi\fi\fi\fi\fi\fi\fi\fi\fi\fi\fi\fi
  \fi\fi\fi\fi
  \SOUL@n*%
}
%    \end{macrocode}
%    \end{macro}
%
% \subsection{UTF-8 analysis}
%
% \subsubsection{Help strings}
%
%    \begin{macrocode}
\def\SOuL@defsanitizedstring#1#2{%
  \expandafter\def\csname SOuL@string#1\endcsname{#2}%
  \expandafter\@onelevel@sanitize\csname SOuL@string#1\endcsname
}
\SOuL@defsanitizedstring{UTFviii}{UTFviii@}
\SOuL@defsanitizedstring{octets}{@octets}
\SOuL@defsanitizedstring{two}{two}
\SOuL@defsanitizedstring{three}{three}
\SOuL@defsanitizedstring{four}{four}
\SOuL@defsanitizedstring{macrocolon}{macro:}
\SOuL@defsanitizedstring{csnameu}{csname u8-}
\SOuL@defsanitizedstring{undeferr}{utf@viii@undeferr}
\def\SOuL@stringendash{^^e2^^80^^93}
\def\SOuL@stringemdash{^^e2^^80^^94}
\def\SOuL@stringnobreakspace{^^c2^^a0}
\edef\SOuL@charhash{\string #}
\edef\SOuL@chartwo{\string 2}
\edef\SOuL@charthree{\string 3}
\def\SOuL@empty{}
%    \end{macrocode}
%
% \subsubsection{Support for \xfile{utf8.def}}
%
%    \begin{macro}{\SOuL@analyzeutfviii}
%    \begin{macrocode}
\begingroup
  \edef\x{\endgroup
    \def\noexpand\SOuL@analyzeutfviii{%
      \noexpand\expandafter\noexpand\SOuL@checkutfviii
      \noexpand\meaning\noexpand\SOUL@@
      \SOuL@stringUTFviii\SOuL@stringoctets
      \noexpand\@nil
    }%
    \def\noexpand\SOuL@checkutfviii
      ##1\SOuL@stringUTFviii##2\SOuL@stringoctets##3\noexpand\@nil
  }%
\x{%
  \def\SOuL@temp{#2}%
  \chardef\SOuL@octets=%
      \ifx\SOuL@temp\SOuL@stringtwo
        \tw@
      \else\ifx\SOuL@temp\SOuL@stringthree
        \thr@@
      \else\ifx\SOuL@temp\SOuL@stringfour
        4 %
      \else
        \z@
      \fi\fi\fi
}
%    \end{macrocode}
%    \end{macro}
%
% \subsubsection{Support for \xfile{utf8x.def}}
%
%    \begin{macro}{\SOuL@analyzeutfviiix}
%    \begin{macrocode}
\begingroup
  \edef\x{\endgroup
    \def\noexpand\SOuL@analyzeutfviiix{%
      \noexpand\expandafter\noexpand\SOuL@checkutfviiix
      \noexpand\meaning\noexpand\SOUL@@
      \SOuL@stringmacrocolon\SOuL@charhash1{}{}{}{}%
      \SOuL@stringcsnameu\SOuL@stringundeferr
      \noexpand\@nil
    }%
%    \end{macrocode}
%    \end{macro}
%    \begin{macro}{\SOuL@checkutfviiix}
%    \begin{macrocode}
    \def\noexpand\SOuL@checkutfviiix
      ##1\SOuL@stringmacrocolon\SOuL@charhash1##2##3##4##5##6%
      \SOuL@stringcsnameu##7\SOuL@stringundeferr##8\noexpand\@nil
  }%
\x{%
  \def\SOuL@temp{#7}%
  \ifx\SOuL@temp\SOuL@empty
    \chardef\SOuL@octets=\z@
  \else
    \def\SOuL@temp{#5}%
    \ifx\SOuL@temp\SOuL@charthree
      \chardef\SOuL@octets=4 %
    \else
      \def\SOuL@temp{#3}%
      \ifx\SOuL@temp\SOuL@chartwo
        \chardef\SOuL@octets=\thr@@
      \else
        \chardef\SOuL@octets=\tw@
      \fi
    \fi
  \fi
}
%    \end{macrocode}
%    \end{macro}
%
% \subsection{Actions for UTF-8 sequences}
%
%    \begin{macro}{\SOuL@addtwooctets}
%    \begin{macrocode}
\def\SOuL@addtwooctets#1#2{%
  \def\SOuL@temp{#1#2}%
  \@onelevel@sanitize\SOuL@temp
  \ifx\SOuL@temp\SOuL@stringnobreakspace
    \SOUL@doword
    \SOUL@eventuallyexhyphen\null
    \SOUL@everyspace{\nobreak}%
    \let\SOuL@next\SOUL@scan
  \else
    \def\SOuL@next{%
!     \SOUL@addtoken{{\noexpand#1\noexpand#2}}%
+     \SOUL@addtoken{{\etex@unexpanded{#1#2}}}%
    }%
  \fi
  \SOuL@next
}
%    \end{macrocode}
%    \end{macro}
%    \begin{macro}{\SOuL@addthreeoctets}
%    \begin{macrocode}
\def\SOuL@addthreeoctets#1#2#3{%
  \def\SOuL@temp{#1#2#3}%
  \@onelevel@sanitize\SOuL@temp
  \ifx\SOuL@temp\SOuL@stringendash
    \SOUL@doword
    \SOUL@eventuallyexhyphen{-}%
    \SOUL@exhyphen{--}%
    \let\SOuL@next\SOUL@scan
  \else
    \ifx\SOuL@temp\SOuL@stringemdash
      \SOUL@doword
      \SOUL@eventuallyexhyphen{-}%
      \SOUL@exhyphen{---}%
      \let\SOuL@next\SOUL@scan
    \else
      \def\SOuL@next{%
!       \SOUL@addtoken{{\noexpand#1\noexpand#2\noexpand#3}}%
+       \SOUL@addtoken{{\etex@unexpanded{#1#2#3}}}%
      }%
    \fi
  \fi
  \SOuL@next
}
%    \end{macrocode}
%    \end{macro}
%    \begin{macro}{\SOuL@addfouroctets}
%    \begin{macrocode}
\def\SOuL@addfouroctets#1#2#3#4{%
! \SOUL@addtoken{{\noexpand#1\noexpand#2\noexpand#3\noexpand#4}}%
+ \SOUL@addtoken{{\etex@unexpanded{#1#2#3#4}}}%
}
%    \end{macrocode}
%    \end{macro}
%
% \subsubsection{Redefinition of \cs{SOUL@splittoken}}
%
%    \begin{macro}{\SOUL@splittoken}
%    Macro \cs{SOUL@splittoken} separates the first token
%    or token group from a word and redefines the word to contain
%    the remaining tokens. However if the remaining tokens are
%    a token group, then the curly braces will be removed and
%    the token group is splitted by the next call of \cs{SOUL@splittoken}.
%    The redefinition avoids the removal of curly braces around the
%    remaining tokens.
%    \begin{macrocode}
\SOuL@redefine\SOUL@splittoken#1#2\SOUL@stop{%
  \global\SOUL@token={#1}%
  \global\SOUL@word={#2}%
}#1{%
  \global\SOUL@token={#1}%
  \SOuL@remainingtoken\relax
}
%    \end{macrocode}
%    \end{macro}
%    \begin{macro}{\SOuL@remainingtoken}
%    \begin{macrocode}
\def\SOuL@remainingtoken#1\SOUL@stop{%
  \global\SOUL@word=\expandafter{\@gobble#1}%
}
%    \end{macrocode}
%    \end{macro}
%
% \subsection{Patches}
%
%    The fixed \cs{SOUL@splittoken} allows to remove the double
%    sets of curly braces in other macros of package \xpackage{soul}.
%    The benefit is that implicite kernings are more often detected
%    and fixes a bug in package \xpackage{soul}. The disadvantage is
%    incompatibility. The width of the resulting strings may change.
%    \begin{macro}{\SOUL@flushcomma}
%    \begin{macrocode}
\SOuL@redefine\SOUL@flushcomma{%
  \ifcase\SOUL@comma
  \or
    \edef\x{\SOUL@word={\the\SOUL@word,}}\x
  \or
    \edef\x{\SOUL@word={\the\SOUL@word{{,,}}}}\x
  \fi
  \SOUL@comma\z@
}{%
  \ifcase\SOUL@comma
  \or
    \edef\x{\SOUL@word={\the\SOUL@word,}}\x
  \or
    \edef\x{\SOUL@word={\the\SOUL@word{,,}}}\x
  \fi
  \SOUL@comma\z@
}
%    \end{macrocode}
%    \end{macro}
%    \begin{macro}{\SOUL@flushapo}
%    \begin{macrocode}
\SOuL@redefine\SOUL@flushapo{%
  \ifcase\SOUL@apo
  \or
    \edef\x{\SOUL@word={\the\SOUL@word'}}\x
  \or
    \edef\x{\SOUL@word={\the\SOUL@word{{''}}}}\x
  \fi
  \SOUL@apo\z@
}{%
  \ifcase\SOUL@apo
  \or
    \edef\x{\SOUL@word={\the\SOUL@word'}}\x
  \or
    \edef\x{\SOUL@word={\the\SOUL@word{''}}}\x
  \fi
  \SOUL@apo\z@
}
%    \end{macrocode}
%    \end{macro}
%    \begin{macro}{\SOUL@flushgrave}
%    \begin{macrocode}
\SOuL@redefine\SOUL@flushgrave{%
  \ifcase\SOUL@grave
  \or
    \edef\x{\SOUL@word={\the\SOUL@word`}}\x
  \or
    \edef\x{\SOUL@word={\the\SOUL@word{{``}}}}\x
  \fi
  \SOUL@grave\z@
}{%
  \ifcase\SOUL@grave
  \or
    \edef\x{\SOUL@word={\the\SOUL@word`}}\x
  \or
    \edef\x{\SOUL@word={\the\SOUL@word{``}}}\x
  \fi
  \SOUL@grave\z@
}
%    \end{macrocode}
%    \end{macro}
%    \begin{macro}{\SOUL@addgroup}
%    \begin{macrocode}
\SOuL@redefine\SOUL@addgroup#1#2{%
  {%
    \let\protect\noexpand
    \edef\x{%
      \global\SOUL@word={%
        \the\SOUL@word
        {{\noexpand#1#2}}%
      }%
    }%
    \x
  }%
  \SOUL@scan
}#1#2{%
  \begingroup
    \let\protect\noexpand
    \edef\x{\endgroup
      \SOUL@word={%
        \the\SOUL@word
!       {\noexpand#1{#2}}%
+       {\etex@unexpanded{#1{#2}}}%
      }%
    }%
  \x
  \SOUL@scan
}
%    \end{macrocode}
%    \end{macro}
%    \begin{macro}{\SOUL@addmath}
%    \begin{macrocode}
\SOuL@redefine\SOUL@addmath$#1${%
  {%
    \let\protect\noexpand
    \edef\x{%
      \global\SOUL@word={%
        \the\SOUL@word
        {{\hbox{$#1$}}}%
      }%
    }%
    \x
  }%
  \SOUL@scan
}$#1${%
  \begingroup
    \let\protect\noexpand
    \edef\x{\endgroup
      \SOUL@word={%
        \the\SOUL@word
!       {\hbox{$#1$}}%
+       {\etex@unexpanded{\hbox{$#1$}}}%
      }%
    }%
  \x
  \SOUL@scan
}
%    \end{macrocode}
%    \end{macro}
%    \begin{macro}{\SOUL@addprotect}
%    \begin{macrocode}
\SOuL@redefine\SOUL@addprotect#1#2{%
  {%
    \let\protect\noexpand
    \edef\x{%
      \global\SOUL@word={%
        \the\SOUL@word
        {{\hbox{#2}}}%
      }%
    }%
    \x
  }%
  \SOUL@scan
}#1#2{%
  \begingroup
    \let\protect\noexpand
    \edef\x{\endgroup
      \SOUL@word={%
        \the\SOUL@word
!       {\hbox{#2}}%
+       {\etex@unexpanded{\hbox{#2}}}%
      }%
    }%
  \x
  \SOUL@scan
}
%    \end{macrocode}
%    \end{macro}
%
%    \begin{macro}{\SOUL@addtoken}
%    \begin{macrocode}
+ \SOuL@redefine\SOUL@addtoken#1{%
+   \edef\x{%
+     \SOUL@word={%
+       \the\SOUL@word
+       \noexpand#1%
+     }%
+   }%
+   \x
+   \SOUL@scan
+ }#1{%
+   \edef\x{%
+     \SOUL@word={%
+       \the\SOUL@word
+       \etex@unexpanded{#1}%
+     }%
+   }%
+   \x
+   \SOUL@scan
+ }%
%    \end{macrocode}
%    \end{macro}
%
%    \begin{macrocode}
\SOuL@AtEnd%
%    \end{macrocode}
%
%    \begin{macrocode}
%</package>
%    \end{macrocode}
%% \section{Installation}
%
% \subsection{Download}
%
% \paragraph{Package.} This package is available on
% CTAN\footnote{\CTANpkg{soulutf8}}:
% \begin{description}
% \item[\CTAN{macros/latex/contrib/oberdiek/soulutf8.dtx}] The source file.
% \item[\CTAN{macros/latex/contrib/oberdiek/soulutf8.pdf}] Documentation.
% \end{description}
%
%
% \paragraph{Bundle.} All the packages of the bundle `oberdiek'
% are also available in a TDS compliant ZIP archive. There
% the packages are already unpacked and the documentation files
% are generated. The files and directories obey the TDS standard.
% \begin{description}
% \item[\CTANinstall{install/macros/latex/contrib/oberdiek.tds.zip}]
% \end{description}
% \emph{TDS} refers to the standard ``A Directory Structure
% for \TeX\ Files'' (\CTANpkg{tds}). Directories
% with \xfile{texmf} in their name are usually organized this way.
%
% \subsection{Bundle installation}
%
% \paragraph{Unpacking.} Unpack the \xfile{oberdiek.tds.zip} in the
% TDS tree (also known as \xfile{texmf} tree) of your choice.
% Example (linux):
% \begin{quote}
%   |unzip oberdiek.tds.zip -d ~/texmf|
% \end{quote}
%
% \subsection{Package installation}
%
% \paragraph{Unpacking.} The \xfile{.dtx} file is a self-extracting
% \docstrip\ archive. The files are extracted by running the
% \xfile{.dtx} through \plainTeX:
% \begin{quote}
%   \verb|tex soulutf8.dtx|
% \end{quote}
%
% \paragraph{TDS.} Now the different files must be moved into
% the different directories in your installation TDS tree
% (also known as \xfile{texmf} tree):
% \begin{quote}
% \def\t{^^A
% \begin{tabular}{@{}>{\ttfamily}l@{ $\rightarrow$ }>{\ttfamily}l@{}}
%   soulutf8.sty & tex/generic/oberdiek/soulutf8.sty\\
%   soulutf8.pdf & doc/latex/oberdiek/soulutf8.pdf\\
%   soulutf8.dtx & source/latex/oberdiek/soulutf8.dtx\\
% \end{tabular}^^A
% }^^A
% \sbox0{\t}^^A
% \ifdim\wd0>\linewidth
%   \begingroup
%     \advance\linewidth by\leftmargin
%     \advance\linewidth by\rightmargin
%   \edef\x{\endgroup
%     \def\noexpand\lw{\the\linewidth}^^A
%   }\x
%   \def\lwbox{^^A
%     \leavevmode
%     \hbox to \linewidth{^^A
%       \kern-\leftmargin\relax
%       \hss
%       \usebox0
%       \hss
%       \kern-\rightmargin\relax
%     }^^A
%   }^^A
%   \ifdim\wd0>\lw
%     \sbox0{\small\t}^^A
%     \ifdim\wd0>\linewidth
%       \ifdim\wd0>\lw
%         \sbox0{\footnotesize\t}^^A
%         \ifdim\wd0>\linewidth
%           \ifdim\wd0>\lw
%             \sbox0{\scriptsize\t}^^A
%             \ifdim\wd0>\linewidth
%               \ifdim\wd0>\lw
%                 \sbox0{\tiny\t}^^A
%                 \ifdim\wd0>\linewidth
%                   \lwbox
%                 \else
%                   \usebox0
%                 \fi
%               \else
%                 \lwbox
%               \fi
%             \else
%               \usebox0
%             \fi
%           \else
%             \lwbox
%           \fi
%         \else
%           \usebox0
%         \fi
%       \else
%         \lwbox
%       \fi
%     \else
%       \usebox0
%     \fi
%   \else
%     \lwbox
%   \fi
% \else
%   \usebox0
% \fi
% \end{quote}
% If you have a \xfile{docstrip.cfg} that configures and enables \docstrip's
% TDS installing feature, then some files can already be in the right
% place, see the documentation of \docstrip.
%
% \subsection{Refresh file name databases}
%
% If your \TeX~distribution
% (\TeX\,Live, \mikTeX, \dots) relies on file name databases, you must refresh
% these. For example, \TeX\,Live\ users run \verb|texhash| or
% \verb|mktexlsr|.
%
% \subsection{Some details for the interested}
%
% \paragraph{Unpacking with \LaTeX.}
% The \xfile{.dtx} chooses its action depending on the format:
% \begin{description}
% \item[\plainTeX:] Run \docstrip\ and extract the files.
% \item[\LaTeX:] Generate the documentation.
% \end{description}
% If you insist on using \LaTeX\ for \docstrip\ (really,
% \docstrip\ does not need \LaTeX), then inform the autodetect routine
% about your intention:
% \begin{quote}
%   \verb|latex \let\install=y% \iffalse meta-comment
%
% File: soulutf8.dtx
% Version: 2016/05/16 v1.1
% Info: Permit use of UTF-8 characters in soul
%
% Copyright (C) 2007 by
%    Heiko Oberdiek <heiko.oberdiek at googlemail.com>
%    2016
%    https://github.com/ho-tex/oberdiek/issues
%
% This work may be distributed and/or modified under the
% conditions of the LaTeX Project Public License, either
% version 1.3c of this license or (at your option) any later
% version. This version of this license is in
%    http://www.latex-project.org/lppl/lppl-1-3c.txt
% and the latest version of this license is in
%    http://www.latex-project.org/lppl.txt
% and version 1.3 or later is part of all distributions of
% LaTeX version 2005/12/01 or later.
%
% This work has the LPPL maintenance status "maintained".
%
% This Current Maintainer of this work is Heiko Oberdiek.
%
% The Base Interpreter refers to any `TeX-Format',
% because some files are installed in TDS:tex/generic//.
%
% This work consists of the main source file soulutf8.dtx
% and the derived files
%    soulutf8.sty, soulutf8.pdf, soulutf8.ins, soulutf8.drv,
%    soulutf8-test1.tex, soulutf8-test2.tex, soulutf8-test3.tex,
%    soulutf8-test4.tex, soulutf8-test5.tex.
%
% Distribution:
%    CTAN:macros/latex/contrib/oberdiek/soulutf8.dtx
%    CTAN:macros/latex/contrib/oberdiek/soulutf8.pdf
%
% Unpacking:
%    (a) If soulutf8.ins is present:
%           tex soulutf8.ins
%    (b) Without soulutf8.ins:
%           tex soulutf8.dtx
%    (c) If you insist on using LaTeX
%           latex \let\install=y% \iffalse meta-comment
%
% File: soulutf8.dtx
% Version: 2016/05/16 v1.1
% Info: Permit use of UTF-8 characters in soul
%
% Copyright (C) 2007 by
%    Heiko Oberdiek <heiko.oberdiek at googlemail.com>
%    2016
%    https://github.com/ho-tex/oberdiek/issues
%
% This work may be distributed and/or modified under the
% conditions of the LaTeX Project Public License, either
% version 1.3c of this license or (at your option) any later
% version. This version of this license is in
%    http://www.latex-project.org/lppl/lppl-1-3c.txt
% and the latest version of this license is in
%    http://www.latex-project.org/lppl.txt
% and version 1.3 or later is part of all distributions of
% LaTeX version 2005/12/01 or later.
%
% This work has the LPPL maintenance status "maintained".
%
% This Current Maintainer of this work is Heiko Oberdiek.
%
% The Base Interpreter refers to any `TeX-Format',
% because some files are installed in TDS:tex/generic//.
%
% This work consists of the main source file soulutf8.dtx
% and the derived files
%    soulutf8.sty, soulutf8.pdf, soulutf8.ins, soulutf8.drv,
%    soulutf8-test1.tex, soulutf8-test2.tex, soulutf8-test3.tex,
%    soulutf8-test4.tex, soulutf8-test5.tex.
%
% Distribution:
%    CTAN:macros/latex/contrib/oberdiek/soulutf8.dtx
%    CTAN:macros/latex/contrib/oberdiek/soulutf8.pdf
%
% Unpacking:
%    (a) If soulutf8.ins is present:
%           tex soulutf8.ins
%    (b) Without soulutf8.ins:
%           tex soulutf8.dtx
%    (c) If you insist on using LaTeX
%           latex \let\install=y% \iffalse meta-comment
%
% File: soulutf8.dtx
% Version: 2016/05/16 v1.1
% Info: Permit use of UTF-8 characters in soul
%
% Copyright (C) 2007 by
%    Heiko Oberdiek <heiko.oberdiek at googlemail.com>
%    2016
%    https://github.com/ho-tex/oberdiek/issues
%
% This work may be distributed and/or modified under the
% conditions of the LaTeX Project Public License, either
% version 1.3c of this license or (at your option) any later
% version. This version of this license is in
%    http://www.latex-project.org/lppl/lppl-1-3c.txt
% and the latest version of this license is in
%    http://www.latex-project.org/lppl.txt
% and version 1.3 or later is part of all distributions of
% LaTeX version 2005/12/01 or later.
%
% This work has the LPPL maintenance status "maintained".
%
% This Current Maintainer of this work is Heiko Oberdiek.
%
% The Base Interpreter refers to any `TeX-Format',
% because some files are installed in TDS:tex/generic//.
%
% This work consists of the main source file soulutf8.dtx
% and the derived files
%    soulutf8.sty, soulutf8.pdf, soulutf8.ins, soulutf8.drv,
%    soulutf8-test1.tex, soulutf8-test2.tex, soulutf8-test3.tex,
%    soulutf8-test4.tex, soulutf8-test5.tex.
%
% Distribution:
%    CTAN:macros/latex/contrib/oberdiek/soulutf8.dtx
%    CTAN:macros/latex/contrib/oberdiek/soulutf8.pdf
%
% Unpacking:
%    (a) If soulutf8.ins is present:
%           tex soulutf8.ins
%    (b) Without soulutf8.ins:
%           tex soulutf8.dtx
%    (c) If you insist on using LaTeX
%           latex \let\install=y\input{soulutf8.dtx}
%        (quote the arguments according to the demands of your shell)
%
% Documentation:
%    (a) If soulutf8.drv is present:
%           latex soulutf8.drv
%    (b) Without soulutf8.drv:
%           latex soulutf8.dtx; ...
%    The class ltxdoc loads the configuration file ltxdoc.cfg
%    if available. Here you can specify further options, e.g.
%    use A4 as paper format:
%       \PassOptionsToClass{a4paper}{article}
%
%    Programm calls to get the documentation (example):
%       pdflatex soulutf8.dtx
%       makeindex -s gind.ist soulutf8.idx
%       pdflatex soulutf8.dtx
%       makeindex -s gind.ist soulutf8.idx
%       pdflatex soulutf8.dtx
%
% Installation:
%    TDS:tex/generic/oberdiek/soulutf8.sty
%    TDS:doc/latex/oberdiek/soulutf8.pdf
%    TDS:doc/latex/oberdiek/test/soulutf8-test1.tex
%    TDS:doc/latex/oberdiek/test/soulutf8-test2.tex
%    TDS:doc/latex/oberdiek/test/soulutf8-test3.tex
%    TDS:doc/latex/oberdiek/test/soulutf8-test4.tex
%    TDS:doc/latex/oberdiek/test/soulutf8-test5.tex
%    TDS:source/latex/oberdiek/soulutf8.dtx
%
%<*ignore>
\begingroup
  \catcode123=1 %
  \catcode125=2 %
  \def\x{LaTeX2e}%
\expandafter\endgroup
\ifcase 0\ifx\install y1\fi\expandafter
         \ifx\csname processbatchFile\endcsname\relax\else1\fi
         \ifx\fmtname\x\else 1\fi\relax
\else\csname fi\endcsname
%</ignore>
%<*install>
\input docstrip.tex
\Msg{************************************************************************}
\Msg{* Installation}
\Msg{* Package: soulutf8 2016/05/16 v1.1 Permit use of UTF-8 characters in soul (HO)}
\Msg{************************************************************************}

\keepsilent
\askforoverwritefalse

\let\MetaPrefix\relax
\preamble

This is a generated file.

Project: soulutf8
Version: 2016/05/16 v1.1

Copyright (C) 2007 by
   Heiko Oberdiek <heiko.oberdiek at googlemail.com>

This work may be distributed and/or modified under the
conditions of the LaTeX Project Public License, either
version 1.3c of this license or (at your option) any later
version. This version of this license is in
   http://www.latex-project.org/lppl/lppl-1-3c.txt
and the latest version of this license is in
   http://www.latex-project.org/lppl.txt
and version 1.3 or later is part of all distributions of
LaTeX version 2005/12/01 or later.

This work has the LPPL maintenance status "maintained".

This Current Maintainer of this work is Heiko Oberdiek.

The Base Interpreter refers to any `TeX-Format',
because some files are installed in TDS:tex/generic//.

This work consists of the main source file soulutf8.dtx
and the derived files
   soulutf8.sty, soulutf8.pdf, soulutf8.ins, soulutf8.drv,
   soulutf8-test1.tex, soulutf8-test2.tex, soulutf8-test3.tex,
   soulutf8-test4.tex, soulutf8-test5.tex.

\endpreamble
\let\MetaPrefix\DoubleperCent

\generate{%
  \file{soulutf8.ins}{\from{soulutf8.dtx}{install}}%
  \file{soulutf8.drv}{\from{soulutf8.dtx}{driver}}%
  \usedir{tex/generic/oberdiek}%
  \file{soulutf8.sty}{\from{soulutf8.dtx}{package}}%
%  \usedir{doc/latex/oberdiek/test}%
%  \file{soulutf8-test1.tex}{\from{soulutf8.dtx}{test1}}%
%  \file{soulutf8-test2.tex}{\from{soulutf8.dtx}{test2,utf8}}%
%  \file{soulutf8-test3.tex}{\from{soulutf8.dtx}{test3,utf8x}}%
%  \file{soulutf8-test4.tex}{\from{soulutf8.dtx}{test4,utf8,noetex}}%
%  \file{soulutf8-test5.tex}{\from{soulutf8.dtx}{test5,utf8x,noetex}}%
  \nopreamble
  \nopostamble
%  \usedir{source/latex/oberdiek/catalogue}%
%  \file{soulutf8.xml}{\from{soulutf8.dtx}{catalogue}}%
}

\catcode32=13\relax% active space
\let =\space%
\Msg{************************************************************************}
\Msg{*}
\Msg{* To finish the installation you have to move the following}
\Msg{* file into a directory searched by TeX:}
\Msg{*}
\Msg{*     soulutf8.sty}
\Msg{*}
\Msg{* To produce the documentation run the file `soulutf8.drv'}
\Msg{* through LaTeX.}
\Msg{*}
\Msg{* Happy TeXing!}
\Msg{*}
\Msg{************************************************************************}

\endbatchfile
%</install>
%<*ignore>
\fi
%</ignore>
%<*driver>
\NeedsTeXFormat{LaTeX2e}
\ProvidesFile{soulutf8.drv}%
  [2016/05/16 v1.1 Permit use of UTF-8 characters in soul (HO)]%
\documentclass{ltxdoc}
\usepackage{holtxdoc}[2011/11/22]
\begin{document}
  \DocInput{soulutf8.dtx}%
\end{document}
%</driver>
% \fi
%
%
% \CharacterTable
%  {Upper-case    \A\B\C\D\E\F\G\H\I\J\K\L\M\N\O\P\Q\R\S\T\U\V\W\X\Y\Z
%   Lower-case    \a\b\c\d\e\f\g\h\i\j\k\l\m\n\o\p\q\r\s\t\u\v\w\x\y\z
%   Digits        \0\1\2\3\4\5\6\7\8\9
%   Exclamation   \!     Double quote  \"     Hash (number) \#
%   Dollar        \$     Percent       \%     Ampersand     \&
%   Acute accent  \'     Left paren    \(     Right paren   \)
%   Asterisk      \*     Plus          \+     Comma         \,
%   Minus         \-     Point         \.     Solidus       \/
%   Colon         \:     Semicolon     \;     Less than     \<
%   Equals        \=     Greater than  \>     Question mark \?
%   Commercial at \@     Left bracket  \[     Backslash     \\
%   Right bracket \]     Circumflex    \^     Underscore    \_
%   Grave accent  \`     Left brace    \{     Vertical bar  \|
%   Right brace   \}     Tilde         \~}
%
% \GetFileInfo{soulutf8.drv}
%
% \title{The \xpackage{soulutf8} package}
% \date{2016/05/16 v1.1}
% \author{Heiko Oberdiek\thanks
% {Please report any issues at https://github.com/ho-tex/oberdiek/issues}\\
% \xemail{heiko.oberdiek at googlemail.com}}
%
% \maketitle
%
% \begin{abstract}
% This package extends package \xpackage{soul} and adds some support
% for UTF-8. Namely the input encodings \xfile{utf8.def}
% from package \xpackage{inputenc} and
% package \xpackage{ucs}'s \xfile{utf8x.def} are supported.
% \end{abstract}
%
% \tableofcontents
%
% \section{Documentation}
%
% This package \xpackage{soulutf8} does not have own options and does not
% define new user commands.
% Any option is passed to package \xpackage{soul} \cite{soul}
% that is loaded first.
% Then some internal macros of \xpackage{soul} are redefined to add
% support for UTF-8.
% The following input encodings are supported:
% \begin{quote}
%   \begin{tabular}{@{}lll@{}}
%     \texttt{utf8} & \LaTeX\ base &
%        \xfile{TDS:tex/latex/base/utf8.def} \cite{utf8}\\
%     \texttt{utf8x} & Package \xpackage{ucs} &
%       \xfile{TDS:tex/latex/ucs/utf8x.def} \cite{ucs}\\
%   \end{tabular}
% \end{quote}
% UTF-8 byte sequences are added as token group to a word, even
% if these UTF-8 characters are some kind of hyphen or space.
% As exception the following three Unicode characters are handled specially:
% \begin{quote}
%   \begin{tabular}{lll}
%     Slot & Name & Action\\
%     \hline
%     \texttt{U+00A0} & NO-BREAK SPACE & like |~|\\
%     \texttt{U+2013} & EN DASH & |--|\\
%     \texttt{U+2014} & EM DASH & |---|
%   \end{tabular}
%
% \subsection{Patch}
%
% Also package \xpackage{soulutf8} tries to patch package \xpackage{soul}
% to improve its behaviour:
% \begin{itemize}
% \item
%   A problem with additional levels of curly braces is fixed.
%   As advantage more implicite kernings are detected.
%   However, the result may be incompatible with the
%   original behaviour of package \xpackage{soul} because
%   of these respected implicite kernings.
% \item
%   \eTeX\ , especially \cs{unexpanded} is supported.
%   This allows a better protection of token groups
%   (|\mbox{|\dots|}|, math, \dots).
% \end{itemize}
% \end{quote}
%
% \subsection{Future}
%
% Currently package \xpackage{soul} does not seem to be
% maintained. Nevertheless if there will be a new version that
% adds support for UTF-8, then this package may become
% obsolete.
%
% \StopEventually{
% }
%
% \section{Implementation}
%
%    \begin{macrocode}
%<*package>
%    \end{macrocode}
%
% \subsection{Reload check and package identification}
%    Reload check, especially if the package is not used with \LaTeX.
%    \begin{macrocode}
\begingroup\catcode61\catcode48\catcode32=10\relax%
  \catcode13=5 % ^^M
  \endlinechar=13 %
  \catcode35=6 % #
  \catcode39=12 % '
  \catcode44=12 % ,
  \catcode45=12 % -
  \catcode46=12 % .
  \catcode58=12 % :
  \catcode64=11 % @
  \catcode123=1 % {
  \catcode125=2 % }
  \expandafter\let\expandafter\x\csname ver@soulutf8.sty\endcsname
  \ifx\x\relax % plain-TeX, first loading
  \else
    \def\empty{}%
    \ifx\x\empty % LaTeX, first loading,
      % variable is initialized, but \ProvidesPackage not yet seen
    \else
      \expandafter\ifx\csname PackageInfo\endcsname\relax
        \def\x#1#2{%
          \immediate\write-1{Package #1 Info: #2.}%
        }%
      \else
        \def\x#1#2{\PackageInfo{#1}{#2, stopped}}%
      \fi
      \x{soulutf8}{The package is already loaded}%
      \aftergroup\endinput
    \fi
  \fi
\endgroup%
%    \end{macrocode}
%    Package identification:
%    \begin{macrocode}
\begingroup\catcode61\catcode48\catcode32=10\relax%
  \catcode13=5 % ^^M
  \endlinechar=13 %
  \catcode35=6 % #
  \catcode39=12 % '
  \catcode40=12 % (
  \catcode41=12 % )
  \catcode44=12 % ,
  \catcode45=12 % -
  \catcode46=12 % .
  \catcode47=12 % /
  \catcode58=12 % :
  \catcode64=11 % @
  \catcode91=12 % [
  \catcode93=12 % ]
  \catcode123=1 % {
  \catcode125=2 % }
  \expandafter\ifx\csname ProvidesPackage\endcsname\relax
    \def\x#1#2#3[#4]{\endgroup
      \immediate\write-1{Package: #3 #4}%
      \xdef#1{#4}%
    }%
  \else
    \def\x#1#2[#3]{\endgroup
      #2[{#3}]%
      \ifx#1\@undefined
        \xdef#1{#3}%
      \fi
      \ifx#1\relax
        \xdef#1{#3}%
      \fi
    }%
  \fi
\expandafter\x\csname ver@soulutf8.sty\endcsname
\ProvidesPackage{soulutf8}%
  [2016/05/16 v1.1 Permit use of UTF-8 characters in soul (HO)]%
%    \end{macrocode}
%
% \subsection{Catcodes}
%
%    \begin{macrocode}
\begingroup\catcode61\catcode48\catcode32=10\relax%
  \catcode13=5 % ^^M
  \endlinechar=13 %
  \catcode123=1 % {
  \catcode125=2 % }
  \catcode64=11 % @
  \def\x{\endgroup
    \expandafter\edef\csname SOuL@AtEnd\endcsname{%
      \endlinechar=\the\endlinechar\relax
      \catcode13=\the\catcode13\relax
      \catcode32=\the\catcode32\relax
      \catcode35=\the\catcode35\relax
      \catcode61=\the\catcode61\relax
      \catcode64=\the\catcode64\relax
      \catcode123=\the\catcode123\relax
      \catcode125=\the\catcode125\relax
    }%
  }%
\x\catcode61\catcode48\catcode32=10\relax%
\catcode13=5 % ^^M
\endlinechar=13 %
\catcode35=6 % #
\catcode64=11 % @
\catcode123=1 % {
\catcode125=2 % }
\def\TMP@EnsureCode#1#2{%
  \edef\SOuL@AtEnd{%
    \SOuL@AtEnd
    \catcode#1=\the\catcode#1\relax
  }%
  \catcode#1=#2\relax
}
\TMP@EnsureCode{10}{12}% ^^J
\TMP@EnsureCode{33}{12}% !
\TMP@EnsureCode{34}{12}% "
\TMP@EnsureCode{36}{3}% $
\TMP@EnsureCode{39}{12}% '
\TMP@EnsureCode{40}{12}% (
\TMP@EnsureCode{41}{12}% )
\TMP@EnsureCode{42}{12}% *
\TMP@EnsureCode{43}{12}% +
\TMP@EnsureCode{44}{12}% ,
\TMP@EnsureCode{45}{12}% -
\TMP@EnsureCode{46}{12}% .
\TMP@EnsureCode{47}{12}% /
\TMP@EnsureCode{58}{12}% :
\TMP@EnsureCode{60}{12}% <
\TMP@EnsureCode{62}{12}% >
\TMP@EnsureCode{91}{12}% [
\TMP@EnsureCode{93}{12}% ]
\TMP@EnsureCode{94}{7}% ^
\TMP@EnsureCode{96}{12}% `
\TMP@EnsureCode{126}\active % ~
\TMP@EnsureCode{128}{12}% ^^80
\TMP@EnsureCode{147}{12}% ^^93
\TMP@EnsureCode{148}{12}% ^^94
\TMP@EnsureCode{160}{12}% ^^a0
\TMP@EnsureCode{194}{12}% ^^c2
\TMP@EnsureCode{226}{12}% ^^e2
\edef\SOuL@AtEnd{\SOuL@AtEnd\noexpand\endinput}
%    \end{macrocode}
%
% \subsection{Loading packages}
%
%    Package \xpackage{soul} uses \cs{documentclass} to detect \LaTeX.
%    \begin{macrocode}
\ifx\documentclass\@undefined
%    \end{macrocode}
%
% \subsubsection{\plainTeX}
%
%    First we check, whether package \xpackage{soul} is already loaded.
%    \begin{macrocode}
  \expandafter\ifx\csname SOUL@\endcsname\relax
%    \end{macrocode}
%    In case of \plainTeX\ package \xpackage{soul} defines some
%    macros in a simple manner that will break the definitions of
%    \xfile{miniltx.tex}, for example. Therefore these macros
%    are first saved and restored afterwards.
%    \begin{macrocode}
    \let\SOuL@orgDeclareRobustCommand\DeclareRobustCommand
    \let\SOuL@orgnewcommand          \newcommand
    \let\SOuL@orgDeclareOption       \DeclareOption
    \let\SOuL@orgPackageError        \PackageError
    \def\SOuL@restorelatexcmds{%
      \let\DeclareRobustCommand\SOuL@orgDeclareRobustCommand
      \let\newcommand          \SOuL@orgnewcommand
      \let\DeclareOption       \SOuL@orgDeclareOption
      \let\PackageError        \SOuL@orgPackageError
    }%
    \input soul.sty\relax
    \SOuL@restorelatexcmds
  \fi
%    \end{macrocode}
%    \begin{macro}{\SOUL@error}
%     Package \xpackage{soul}'s use of \cs{PackageError} is replaced
%     by \cs{@PackageError} of package \xpackage{infwarerr}.
%    \begin{macrocode}
  \input infwarerr.sty\relax
  \let\SOuL@orgSOUL@error\SOUL@error
  \def\SOUL@error{%
    \begingroup
      \let\PackageError\@PackageError
      \SOuL@orgSOUL@error
    \endgroup
  }%
%    \end{macrocode}
%    \end{macro}
%    \begin{macrocode}
  \input etexcmds.sty\relax
%    \end{macrocode}
%    \begin{macro}{\@onelevel@sanitize}
%    Define \LaTeX's \cs{@onelevel@sanitize} if not already available.
%    \begin{macrocode}
  \expandafter\ifx\csname @onelevel@sanitize\endcsname\relax
    \def\@onelevel@sanitize#1{%
      \edef#1{%
        \expandafter\strip@prefix\meaning#1%
      }%
    }%
%    \end{macrocode}
%    \end{macro}
%    \begin{macro}{\strip@prefix}
%    \begin{macrocode}
    \def\strip@prefix#1>{}%
%    \end{macrocode}
%    \end{macro}
%    \begin{macrocode}
  \fi
\else
%    \end{macrocode}
%
% \subsubsection{\LaTeX}
%    \begin{macrocode}
  \DeclareOption*{\PassOptionsToPackage{\CurrentOption}{soul}}%
  \ProcessOptions\relax
  \RequirePackage{soul}[2003/11/17]%
  \RequirePackage{infwarerr}[2016/05/16]%
  \RequirePackage{etexcmds}[2016/05/16]%
\fi
%    \end{macrocode}
%
% \subsubsection{\eTeX}
%
%    In \plainTeX\ command \cs{+} is an \emph{outer} macro.
%    Therefore numbers are used to avoid problems.
%    \begin{macrocode}
\ifetex@unexpanded
  \catcode33=14 % '!': comment
  \catcode43=9  % '+': ignore
\else
  \catcode33=9  % '!': ignore
  \catcode43=14 % '+': comment
\fi
%    \end{macrocode}
%
% \subsection{Macro for redefinitions}
%
%    \begin{macro}{\SOuL@redefine}
%    \begin{macrocode}
\def\SOuL@redefine#1{%
  \begingroup
    \def\SOuL@cmd{#1}%
    \afterassignment\SOuL@cmdcheck
    \def\SOuL@temp
}
%    \end{macrocode}
%    \end{macro}
%    \begin{macro}{\SOuL@cmdcheck}
%    \begin{macrocode}
\def\SOuL@cmdcheck{%
    \expandafter\ifx\SOuL@cmd\SOuL@temp
    \else
      \edef\SOuL@temp*{\expandafter\string\SOuL@cmd}%
      \@PackageWarningNoLine{soulutf8}{%
        Command \SOuL@temp* has changed.\MessageBreak
        Supported versions of package `soul': 2003/11/17.\MessageBreak
        Depending on the unknown changes the redefinition\MessageBreak
        of \SOuL@temp* may not behave correctly%
      }%
    \fi
  \expandafter\endgroup
  \expandafter\def\SOuL@cmd
}
%    \end{macrocode}
%    \end{macro}
%
% \subsection{Redefinition of \cs{SOUL@eval}}
%
%    \begin{macro}{\SOUL@eval}
%    Macro \cs{SOUL@eval} is redefined to add detection of
%    the first byte of a UTF-8 sequence. Because \cs{SOUL@eval}
%    is overwritten, a warning is issued, if the contents of
%    \cs{SOUL@eval} is not as expected.
%    \begin{macrocode}
\SOuL@redefine\SOUL@eval{%
%    \end{macrocode}
%    First the expected definition.
%    \begin{macrocode}
  \def\SOUL@n*##1{\SOUL@scan}%
  \if\noexpand\SOUL@@\SOUL@spc
  \else
    \SOUL@ignorespacesfalse
  \fi
  \ifnum\SOUL@minus=\thr@@
    \SOUL@flushminus
  \else\ifnum\SOUL@comma=\tw@
    \SOUL@flushcomma
  \else\ifnum\SOUL@apo=\tw@
    \SOUL@flushapo
  \else\ifnum\SOUL@grave=\tw@
    \SOUL@flushgrave
  \fi\fi\fi\fi
  \ifx\SOUL@@-\else\SOUL@flushminus\fi
  \ifx\SOUL@@,\else\SOUL@flushcomma\fi
  \ifx\SOUL@@'\else\SOUL@flushapo\fi
  \ifx\SOUL@@`\else\SOUL@flushgrave\fi
  \ifx\SOUL@@-%
    \advance\SOUL@minus\@ne
  \else\ifx\SOUL@@,%
    \advance\SOUL@comma\@ne
  \else\ifx\SOUL@@'%
    \advance\SOUL@apo\@ne
  \else\ifx\SOUL@@`%
    \advance\SOUL@grave\@ne
  \else
    \SOUL@flushminus
    \SOUL@flushcomma
    \SOUL@flushapo
    \SOUL@flushgrave
    \ifx\SOUL@@\SOUL@stop
      \def\SOUL@n*{%
        \SOUL@doword
        \SOUL@eventuallyexhyphen\null
      }%
    \else\ifx\SOUL@@\par
      \def\SOUL@n*\par{\par\leavevmode\SOUL@scan}%
    \else\if\noexpand\SOUL@@\SOUL@spc
      \SOUL@doword
      \SOUL@eventuallyexhyphen\null
      \ifSOUL@ignorespaces
      \else
        \SOUL@everyspace{}%
      \fi
      \def\SOUL@n* {\SOUL@scan}%
    \else\ifx\SOUL@@\\%
      \SOUL@doword
      \SOUL@eventuallyexhyphen\null
      \SOUL@everyspace{\unskip\nobreak\hfil\break}%
      \SOUL@ignorespacestrue
    \else\ifx\SOUL@@~%
      \SOUL@doword
      \SOUL@eventuallyexhyphen\null
      \SOUL@everyspace{\nobreak}%
    \else\ifx\SOUL@@\slash
      \SOUL@doword
      \SOUL@eventuallyexhyphen{/}%
      \SOUL@exhyphen{/}%
    \else\ifx\SOUL@@\mbox
      \def\SOUL@n*{\SOUL@addprotect}%
    \else\ifx\SOUL@@\hbox
      \def\SOUL@n*{\SOUL@addprotect}%
    \else\ifx\SOUL@@\soulomit
      \def\SOUL@n*\soulomit##1{%
        \SOUL@doword
        {\spaceskip\SOUL@spaceskip##1}%
        \SOUL@scan
      }%
    \else\ifx\SOUL@@\break
      \SOUL@doword
      \break
    \else\ifx\SOUL@@\linebreak
      \SOUL@doword
      \SOUL@everyspace{\linebreak}%
    \else\ifcat\bgroup\noexpand\SOUL@@
      \def\SOUL@n*{\SOUL@addgroup{}}%
    \else\ifcat$\noexpand\SOUL@@
      \def\SOUL@n*{\SOUL@addmath}%
    \else
      \def\SOUL@n*{\SOUL@dotoken}%
    \fi\fi\fi\fi\fi\fi\fi\fi\fi\fi\fi\fi\fi
  \fi\fi\fi\fi
  \SOUL@n*%
}{%
%    \end{macrocode}
%    Now the redefined version follows.
%    \begin{macrocode}
  \def\SOUL@n*##1{\SOUL@scan}%
  \if\noexpand\SOUL@@\SOUL@spc
  \else
    \SOUL@ignorespacesfalse
  \fi
  \ifnum\SOUL@minus=\thr@@
    \SOUL@flushminus
  \else\ifnum\SOUL@comma=\tw@
    \SOUL@flushcomma
  \else\ifnum\SOUL@apo=\tw@
    \SOUL@flushapo
  \else\ifnum\SOUL@grave=\tw@
    \SOUL@flushgrave
  \fi\fi\fi\fi
  \ifx\SOUL@@-\else\SOUL@flushminus\fi
  \ifx\SOUL@@,\else\SOUL@flushcomma\fi
  \ifx\SOUL@@'\else\SOUL@flushapo\fi
  \ifx\SOUL@@`\else\SOUL@flushgrave\fi
  \ifx\SOUL@@-%
    \advance\SOUL@minus\@ne
  \else\ifx\SOUL@@,%
    \advance\SOUL@comma\@ne
  \else\ifx\SOUL@@'%
    \advance\SOUL@apo\@ne
  \else\ifx\SOUL@@`%
    \advance\SOUL@grave\@ne
  \else
    \SOUL@flushminus
    \SOUL@flushcomma
    \SOUL@flushapo
    \SOUL@flushgrave
    \ifx\SOUL@@\SOUL@stop
      \def\SOUL@n*{%
          \SOUL@doword
          \SOUL@eventuallyexhyphen\null
      }%
    \else\ifx\SOUL@@\par
      \def\SOUL@n*\par{\par\leavevmode\SOUL@scan}%
    \else\if\noexpand\SOUL@@\SOUL@spc
      \SOUL@doword
      \SOUL@eventuallyexhyphen\null
      \ifSOUL@ignorespaces
      \else
        \SOUL@everyspace{}%
      \fi
      \def\SOUL@n* {\SOUL@scan}%
    \else\ifx\SOUL@@\\%
      \SOUL@doword
      \SOUL@eventuallyexhyphen\null
      \SOUL@everyspace{\unskip\nobreak\hfil\break}%
      \SOUL@ignorespacestrue
    \else\ifx\SOUL@@~%
      \SOUL@doword
      \SOUL@eventuallyexhyphen\null
      \SOUL@everyspace{\nobreak}%
    \else\ifx\SOUL@@\slash
      \SOUL@doword
      \SOUL@eventuallyexhyphen{/}%
      \SOUL@exhyphen{/}%
    \else\ifx\SOUL@@\mbox
      \def\SOUL@n*{\SOUL@addprotect}%
    \else\ifx\SOUL@@\hbox
      \def\SOUL@n*{\SOUL@addprotect}%
    \else\ifx\SOUL@@\soulomit
      \def\SOUL@n*\soulomit##1{%
        \SOUL@doword
        {\spaceskip\SOUL@spaceskip##1}%
        \SOUL@scan
      }%
    \else\ifx\SOUL@@\break
      \SOUL@doword
      \break
    \else\ifx\SOUL@@\linebreak
      \SOUL@doword
      \SOUL@everyspace{\linebreak}%
    \else\ifcat\bgroup\noexpand\SOUL@@
      \def\SOUL@n*{\SOUL@addgroup{}}%
    \else\ifcat$\noexpand\SOUL@@
      \def\SOUL@n*{\SOUL@addmath}%
    \else
%    \end{macrocode}
%    The current token is examined to detect the start of
%    a UTF-8 sequence.
%    \begin{macrocode}
      \SOuL@analyzeutfviii
      \ifcase\SOuL@octets
        \SOuL@analyzeutfviiix
      \fi
      \ifcase\SOuL@octets
        \def\SOUL@n*{\SOUL@dotoken}%
      \or % 1
      \or % 2
        \def\SOUL@n*{\SOuL@addtwooctets}%
      \or % 3
        \def\SOUL@n*{\SOuL@addthreeoctets}%
      \or % 4
        \def\SOUL@n*{\SOuL@addfouroctets}%
      \fi
    \fi\fi\fi\fi\fi\fi\fi\fi\fi\fi\fi\fi\fi
  \fi\fi\fi\fi
  \SOUL@n*%
}
%    \end{macrocode}
%    \end{macro}
%
% \subsection{UTF-8 analysis}
%
% \subsubsection{Help strings}
%
%    \begin{macrocode}
\def\SOuL@defsanitizedstring#1#2{%
  \expandafter\def\csname SOuL@string#1\endcsname{#2}%
  \expandafter\@onelevel@sanitize\csname SOuL@string#1\endcsname
}
\SOuL@defsanitizedstring{UTFviii}{UTFviii@}
\SOuL@defsanitizedstring{octets}{@octets}
\SOuL@defsanitizedstring{two}{two}
\SOuL@defsanitizedstring{three}{three}
\SOuL@defsanitizedstring{four}{four}
\SOuL@defsanitizedstring{macrocolon}{macro:}
\SOuL@defsanitizedstring{csnameu}{csname u8-}
\SOuL@defsanitizedstring{undeferr}{utf@viii@undeferr}
\def\SOuL@stringendash{^^e2^^80^^93}
\def\SOuL@stringemdash{^^e2^^80^^94}
\def\SOuL@stringnobreakspace{^^c2^^a0}
\edef\SOuL@charhash{\string #}
\edef\SOuL@chartwo{\string 2}
\edef\SOuL@charthree{\string 3}
\def\SOuL@empty{}
%    \end{macrocode}
%
% \subsubsection{Support for \xfile{utf8.def}}
%
%    \begin{macro}{\SOuL@analyzeutfviii}
%    \begin{macrocode}
\begingroup
  \edef\x{\endgroup
    \def\noexpand\SOuL@analyzeutfviii{%
      \noexpand\expandafter\noexpand\SOuL@checkutfviii
      \noexpand\meaning\noexpand\SOUL@@
      \SOuL@stringUTFviii\SOuL@stringoctets
      \noexpand\@nil
    }%
    \def\noexpand\SOuL@checkutfviii
      ##1\SOuL@stringUTFviii##2\SOuL@stringoctets##3\noexpand\@nil
  }%
\x{%
  \def\SOuL@temp{#2}%
  \chardef\SOuL@octets=%
      \ifx\SOuL@temp\SOuL@stringtwo
        \tw@
      \else\ifx\SOuL@temp\SOuL@stringthree
        \thr@@
      \else\ifx\SOuL@temp\SOuL@stringfour
        4 %
      \else
        \z@
      \fi\fi\fi
}
%    \end{macrocode}
%    \end{macro}
%
% \subsubsection{Support for \xfile{utf8x.def}}
%
%    \begin{macro}{\SOuL@analyzeutfviiix}
%    \begin{macrocode}
\begingroup
  \edef\x{\endgroup
    \def\noexpand\SOuL@analyzeutfviiix{%
      \noexpand\expandafter\noexpand\SOuL@checkutfviiix
      \noexpand\meaning\noexpand\SOUL@@
      \SOuL@stringmacrocolon\SOuL@charhash1{}{}{}{}%
      \SOuL@stringcsnameu\SOuL@stringundeferr
      \noexpand\@nil
    }%
%    \end{macrocode}
%    \end{macro}
%    \begin{macro}{\SOuL@checkutfviiix}
%    \begin{macrocode}
    \def\noexpand\SOuL@checkutfviiix
      ##1\SOuL@stringmacrocolon\SOuL@charhash1##2##3##4##5##6%
      \SOuL@stringcsnameu##7\SOuL@stringundeferr##8\noexpand\@nil
  }%
\x{%
  \def\SOuL@temp{#7}%
  \ifx\SOuL@temp\SOuL@empty
    \chardef\SOuL@octets=\z@
  \else
    \def\SOuL@temp{#5}%
    \ifx\SOuL@temp\SOuL@charthree
      \chardef\SOuL@octets=4 %
    \else
      \def\SOuL@temp{#3}%
      \ifx\SOuL@temp\SOuL@chartwo
        \chardef\SOuL@octets=\thr@@
      \else
        \chardef\SOuL@octets=\tw@
      \fi
    \fi
  \fi
}
%    \end{macrocode}
%    \end{macro}
%
% \subsection{Actions for UTF-8 sequences}
%
%    \begin{macro}{\SOuL@addtwooctets}
%    \begin{macrocode}
\def\SOuL@addtwooctets#1#2{%
  \def\SOuL@temp{#1#2}%
  \@onelevel@sanitize\SOuL@temp
  \ifx\SOuL@temp\SOuL@stringnobreakspace
    \SOUL@doword
    \SOUL@eventuallyexhyphen\null
    \SOUL@everyspace{\nobreak}%
    \let\SOuL@next\SOUL@scan
  \else
    \def\SOuL@next{%
!     \SOUL@addtoken{{\noexpand#1\noexpand#2}}%
+     \SOUL@addtoken{{\etex@unexpanded{#1#2}}}%
    }%
  \fi
  \SOuL@next
}
%    \end{macrocode}
%    \end{macro}
%    \begin{macro}{\SOuL@addthreeoctets}
%    \begin{macrocode}
\def\SOuL@addthreeoctets#1#2#3{%
  \def\SOuL@temp{#1#2#3}%
  \@onelevel@sanitize\SOuL@temp
  \ifx\SOuL@temp\SOuL@stringendash
    \SOUL@doword
    \SOUL@eventuallyexhyphen{-}%
    \SOUL@exhyphen{--}%
    \let\SOuL@next\SOUL@scan
  \else
    \ifx\SOuL@temp\SOuL@stringemdash
      \SOUL@doword
      \SOUL@eventuallyexhyphen{-}%
      \SOUL@exhyphen{---}%
      \let\SOuL@next\SOUL@scan
    \else
      \def\SOuL@next{%
!       \SOUL@addtoken{{\noexpand#1\noexpand#2\noexpand#3}}%
+       \SOUL@addtoken{{\etex@unexpanded{#1#2#3}}}%
      }%
    \fi
  \fi
  \SOuL@next
}
%    \end{macrocode}
%    \end{macro}
%    \begin{macro}{\SOuL@addfouroctets}
%    \begin{macrocode}
\def\SOuL@addfouroctets#1#2#3#4{%
! \SOUL@addtoken{{\noexpand#1\noexpand#2\noexpand#3\noexpand#4}}%
+ \SOUL@addtoken{{\etex@unexpanded{#1#2#3#4}}}%
}
%    \end{macrocode}
%    \end{macro}
%
% \subsubsection{Redefinition of \cs{SOUL@splittoken}}
%
%    \begin{macro}{\SOUL@splittoken}
%    Macro \cs{SOUL@splittoken} separates the first token
%    or token group from a word and redefines the word to contain
%    the remaining tokens. However if the remaining tokens are
%    a token group, then the curly braces will be removed and
%    the token group is splitted by the next call of \cs{SOUL@splittoken}.
%    The redefinition avoids the removal of curly braces around the
%    remaining tokens.
%    \begin{macrocode}
\SOuL@redefine\SOUL@splittoken#1#2\SOUL@stop{%
  \global\SOUL@token={#1}%
  \global\SOUL@word={#2}%
}#1{%
  \global\SOUL@token={#1}%
  \SOuL@remainingtoken\relax
}
%    \end{macrocode}
%    \end{macro}
%    \begin{macro}{\SOuL@remainingtoken}
%    \begin{macrocode}
\def\SOuL@remainingtoken#1\SOUL@stop{%
  \global\SOUL@word=\expandafter{\@gobble#1}%
}
%    \end{macrocode}
%    \end{macro}
%
% \subsection{Patches}
%
%    The fixed \cs{SOUL@splittoken} allows to remove the double
%    sets of curly braces in other macros of package \xpackage{soul}.
%    The benefit is that implicite kernings are more often detected
%    and fixes a bug in package \xpackage{soul}. The disadvantage is
%    incompatibility. The width of the resulting strings may change.
%    \begin{macro}{\SOUL@flushcomma}
%    \begin{macrocode}
\SOuL@redefine\SOUL@flushcomma{%
  \ifcase\SOUL@comma
  \or
    \edef\x{\SOUL@word={\the\SOUL@word,}}\x
  \or
    \edef\x{\SOUL@word={\the\SOUL@word{{,,}}}}\x
  \fi
  \SOUL@comma\z@
}{%
  \ifcase\SOUL@comma
  \or
    \edef\x{\SOUL@word={\the\SOUL@word,}}\x
  \or
    \edef\x{\SOUL@word={\the\SOUL@word{,,}}}\x
  \fi
  \SOUL@comma\z@
}
%    \end{macrocode}
%    \end{macro}
%    \begin{macro}{\SOUL@flushapo}
%    \begin{macrocode}
\SOuL@redefine\SOUL@flushapo{%
  \ifcase\SOUL@apo
  \or
    \edef\x{\SOUL@word={\the\SOUL@word'}}\x
  \or
    \edef\x{\SOUL@word={\the\SOUL@word{{''}}}}\x
  \fi
  \SOUL@apo\z@
}{%
  \ifcase\SOUL@apo
  \or
    \edef\x{\SOUL@word={\the\SOUL@word'}}\x
  \or
    \edef\x{\SOUL@word={\the\SOUL@word{''}}}\x
  \fi
  \SOUL@apo\z@
}
%    \end{macrocode}
%    \end{macro}
%    \begin{macro}{\SOUL@flushgrave}
%    \begin{macrocode}
\SOuL@redefine\SOUL@flushgrave{%
  \ifcase\SOUL@grave
  \or
    \edef\x{\SOUL@word={\the\SOUL@word`}}\x
  \or
    \edef\x{\SOUL@word={\the\SOUL@word{{``}}}}\x
  \fi
  \SOUL@grave\z@
}{%
  \ifcase\SOUL@grave
  \or
    \edef\x{\SOUL@word={\the\SOUL@word`}}\x
  \or
    \edef\x{\SOUL@word={\the\SOUL@word{``}}}\x
  \fi
  \SOUL@grave\z@
}
%    \end{macrocode}
%    \end{macro}
%    \begin{macro}{\SOUL@addgroup}
%    \begin{macrocode}
\SOuL@redefine\SOUL@addgroup#1#2{%
  {%
    \let\protect\noexpand
    \edef\x{%
      \global\SOUL@word={%
        \the\SOUL@word
        {{\noexpand#1#2}}%
      }%
    }%
    \x
  }%
  \SOUL@scan
}#1#2{%
  \begingroup
    \let\protect\noexpand
    \edef\x{\endgroup
      \SOUL@word={%
        \the\SOUL@word
!       {\noexpand#1{#2}}%
+       {\etex@unexpanded{#1{#2}}}%
      }%
    }%
  \x
  \SOUL@scan
}
%    \end{macrocode}
%    \end{macro}
%    \begin{macro}{\SOUL@addmath}
%    \begin{macrocode}
\SOuL@redefine\SOUL@addmath$#1${%
  {%
    \let\protect\noexpand
    \edef\x{%
      \global\SOUL@word={%
        \the\SOUL@word
        {{\hbox{$#1$}}}%
      }%
    }%
    \x
  }%
  \SOUL@scan
}$#1${%
  \begingroup
    \let\protect\noexpand
    \edef\x{\endgroup
      \SOUL@word={%
        \the\SOUL@word
!       {\hbox{$#1$}}%
+       {\etex@unexpanded{\hbox{$#1$}}}%
      }%
    }%
  \x
  \SOUL@scan
}
%    \end{macrocode}
%    \end{macro}
%    \begin{macro}{\SOUL@addprotect}
%    \begin{macrocode}
\SOuL@redefine\SOUL@addprotect#1#2{%
  {%
    \let\protect\noexpand
    \edef\x{%
      \global\SOUL@word={%
        \the\SOUL@word
        {{\hbox{#2}}}%
      }%
    }%
    \x
  }%
  \SOUL@scan
}#1#2{%
  \begingroup
    \let\protect\noexpand
    \edef\x{\endgroup
      \SOUL@word={%
        \the\SOUL@word
!       {\hbox{#2}}%
+       {\etex@unexpanded{\hbox{#2}}}%
      }%
    }%
  \x
  \SOUL@scan
}
%    \end{macrocode}
%    \end{macro}
%
%    \begin{macro}{\SOUL@addtoken}
%    \begin{macrocode}
+ \SOuL@redefine\SOUL@addtoken#1{%
+   \edef\x{%
+     \SOUL@word={%
+       \the\SOUL@word
+       \noexpand#1%
+     }%
+   }%
+   \x
+   \SOUL@scan
+ }#1{%
+   \edef\x{%
+     \SOUL@word={%
+       \the\SOUL@word
+       \etex@unexpanded{#1}%
+     }%
+   }%
+   \x
+   \SOUL@scan
+ }%
%    \end{macrocode}
%    \end{macro}
%
%    \begin{macrocode}
\SOuL@AtEnd%
%    \end{macrocode}
%
%    \begin{macrocode}
%</package>
%    \end{macrocode}
%
% \section{Test}
%
% \subsection{Catcode checks for loading}
%
%    \begin{macrocode}
%<*test1>
%    \end{macrocode}
%    \begin{macrocode}
\catcode`\{=1 %
\catcode`\}=2 %
\catcode`\#=6 %
\catcode`\@=11 %
\expandafter\ifx\csname count@\endcsname\relax
  \countdef\count@=255 %
\fi
\expandafter\ifx\csname @gobble\endcsname\relax
  \long\def\@gobble#1{}%
\fi
\expandafter\ifx\csname @firstofone\endcsname\relax
  \long\def\@firstofone#1{#1}%
\fi
\expandafter\ifx\csname loop\endcsname\relax
  \expandafter\@firstofone
\else
  \expandafter\@gobble
\fi
{%
  \def\loop#1\repeat{%
    \def\body{#1}%
    \iterate
  }%
  \def\iterate{%
    \body
      \let\next\iterate
    \else
      \let\next\relax
    \fi
    \next
  }%
  \let\repeat=\fi
}%
\def\RestoreCatcodes{}
\count@=0 %
\loop
  \edef\RestoreCatcodes{%
    \RestoreCatcodes
    \catcode\the\count@=\the\catcode\count@\relax
  }%
\ifnum\count@<255 %
  \advance\count@ 1 %
\repeat

\def\RangeCatcodeInvalid#1#2{%
  \count@=#1\relax
  \loop
    \catcode\count@=15 %
  \ifnum\count@<#2\relax
    \advance\count@ 1 %
  \repeat
}
\def\RangeCatcodeCheck#1#2#3{%
  \count@=#1\relax
  \loop
    \ifnum#3=\catcode\count@
    \else
      \errmessage{%
        Character \the\count@\space
        with wrong catcode \the\catcode\count@\space
        instead of \number#3%
      }%
    \fi
  \ifnum\count@<#2\relax
    \advance\count@ 1 %
  \repeat
}
\def\space{ }
\expandafter\ifx\csname LoadCommand\endcsname\relax
  \def\LoadCommand{\input soulutf8.sty\relax}%
\fi
\def\Test{%
  \RangeCatcodeInvalid{0}{47}%
  \RangeCatcodeInvalid{58}{64}%
  \RangeCatcodeInvalid{91}{96}%
  \RangeCatcodeInvalid{123}{255}%
  \catcode`\@=12 %
  \catcode`\\=0 %
  \catcode`\%=14 %
  \LoadCommand
  \RangeCatcodeCheck{0}{36}{15}%
  \RangeCatcodeCheck{37}{37}{14}%
  \RangeCatcodeCheck{38}{47}{15}%
  \RangeCatcodeCheck{48}{57}{12}%
  \RangeCatcodeCheck{58}{63}{15}%
  \RangeCatcodeCheck{64}{64}{12}%
  \RangeCatcodeCheck{65}{90}{11}%
  \RangeCatcodeCheck{91}{91}{15}%
  \RangeCatcodeCheck{92}{92}{0}%
  \RangeCatcodeCheck{93}{96}{15}%
  \RangeCatcodeCheck{97}{122}{11}%
  \RangeCatcodeCheck{123}{255}{15}%
  \RestoreCatcodes
}
\Test
\csname @@end\endcsname
\end
%    \end{macrocode}
%    \begin{macrocode}
%</test1>
%    \end{macrocode}
%
% \subsection{\LaTeX\space tests}
%
%    \begin{macrocode}
%<*test2|test3|test4|test5>
%    \end{macrocode}
%    \begin{macrocode}
%<noetex>\let\SavedUnexpanded\unexpanded
%<noetex>\let\unexpanded\UNDEFINED
\NeedsTeXFormat{LaTeX2e}
\documentclass{article}
\usepackage[T1]{fontenc}
\usepackage{lmodern}
\usepackage{textcomp}
\usepackage[ngerman]{babel}
\expandafter\def\csname ver@hypdestopt.sty\endcsname{}
\usepackage[unicode]{hyperref}
\usepackage{bookmark}
\usepackage{color}
\usepackage{soul}[2003/11/17]
\makeatletter
\newcommand*{\@ifDefined}[1]{%
  \@ifundefined{#1}{%
    \@latex@error{%
      \expandafter\noexpand\csname#1\endcsname undefined%
    }\@ehc
  }%
}
\newcommand*{\SaveOrg}[1]{%
  \@ifdefinable{ORG@SOUL@#1}{%
    \@ifDefined{SOUL@#1}{%
      \expandafter\let\csname ORG@SOUL@#1\expandafter\endcsname
                      \csname SOUL@#1\endcsname
    }%
  }%
}
\newcommand*{\RestoreOrg}[1]{%
  \@ifDefined{ORG@SOUL@#1}{%
    \@ifDefined{SOUL@#1}{%
      \expandafter\let\csname SOUL@#1\expandafter\endcsname
                      \csname ORG@SOUL@#1\endcsname
    }%
  }%
}
\makeatother
\SaveOrg{eval}
\SaveOrg{splittoken}
\SaveOrg{flushcomma}
\SaveOrg{flushapo}
\SaveOrg{grave}
\SaveOrg{addgroup}
\SaveOrg{addmath}
\SaveOrg{addprotect}
\SaveOrg{addtoken}
\usepackage{soulutf8}[2016/05/16]
%    \end{macrocode}
%    \begin{macrocode}
%<noetex>\let\unexpanded\SavedUnexpanded
%    \end{macrocode}
%    Input encoding \xfile{utf8}.
%    \begin{macrocode}
%<*utf8>
\usepackage[utf8]{inputenc}
\DeclareUnicodeCharacter{00A0}{\nobreakspace}
%</utf8>
%    \end{macrocode}
%    Input encoding \xfile{utf8x}. Package \xpackage{ucs}
%    has a nasty bug, that it redefines \cs{show}.
%    \begin{macrocode}
%<*utf8x>
\let\origshow\show
\usepackage[utf8x]{inputenc}
\AtBeginDocument{%
  \let\show\origshow
}
%</utf8x>
%    \end{macrocode}
%
% \subsubsection{Driver definitions}
%
%    \begin{macrocode}
\makeatletter
%    \end{macrocode}
%    \begin{macro}{\an}
%    The definition for \cs{an} is taken from the documentation of
%    package \xpackage{soul}.
%    \begin{macrocode}
\DeclareRobustCommand*{\an}{%
  \def\SOUL@preamble{$^{^P}$}%
  \def\SOUL@everyspace##1{##1\texttt{\char`\ }\linebreak[2]}%
  \def\SOUL@postamble{$^{^E}$}%
  \def\SOUL@everyhyphen{$^{^-}$\linebreak[1]}%
  \def\SOUL@everyexhyphen##1{##1$^{^=}$}%
  \def\SOUL@everysyllable{$^{^S}$}%
  \def\SOUL@everytoken{\the\SOUL@token$^{^T}$}%
  \def\SOUL@everylowerthan{$^{^L}$}%
  \SOUL@
}
%    \end{macrocode}
%    \end{macro}
%
%    \begin{macro}{\sy}
%    The definition for \cs{sy} is taken from the documentation of
%    package \xpackage{soul}.
%    \begin{macrocode}
\DeclareRobustCommand*{\sy}{%
  \SOUL@setup
  \def\SOUL@preamble{%
    \lefthyphenmin=\z@
    \righthyphenmin=\z@
  }%
  \def\SOUL@everyspace##1{##1\space}%
  \def\SOUL@everysyllable{\the\SOUL@syllable}%
  \def\SOUL@everyhyphen{%
    \discretionary{%
      \SOUL@setkern\SOUL@hyphkern
      \SOUL@sethyphenchar
    }{}{%
      \hbox{\kern1pt$\cdot$}%
    }%
  }%
  \def\SOUL@everyexhyphen##1{%
    \SOUL@setkern\SOUL@hyphkern
    \hbox{##1}%
    \discretionary{}{}{%
      \SOUL@setkern\SOUL@charkern
    }%
  }%
  \SOUL@
}
%    \end{macrocode}
%    \end{macro}
%
%    \begin{macro}{\soulH}
%    \begin{macrocode}
\DeclareRobustCommand*{\soulH}{%
  \def\SOUL@preamble{%
    \gdef\SOULH{}%
    \soulHadd{[}%
  }%
  \def\SOUL@everyspace##1{%
    \soulHadd{Sp}%
  }%
  \def\SOUL@postamble{%
    \soulHadd{]}%
  }%
  \def\SOUL@everyhyphen{%
    \soulHadd{Hy}%
  }%
  \def\SOUL@everyexhyphen##1{%
    \soulHadd{Exhy}%
  }%
  \def\SOUL@everysyllable{%
    \soulHadd{Sy}%
  }%
  \def\SOUL@everytoken{%
    \soulHadd{T}%
  }%
  \def\SOUL@everylowerthan{%
    \soulHadd{<}%
  }%
  \SOUL@
}
\newcommand*{\soulHadd}[1]{%
  #1%
  \expandafter\gdef\expandafter\soulHstring\expandafter{%
    \soulHstring
    #1%
  }%
}
%    \end{macrocode}
%    \end{macro}
%
% \subsubsection{Test macros}
%
%    \begin{macrocode}
\usepackage{qstest}
\IncludeTests{*}
\LogTests{log}{*}{*}
%    \end{macrocode}
%    \begin{macrocode}
\newsavebox\boxA
\newsavebox\boxB
\newsavebox\boxC
%    \end{macrocode}
%    \begin{macro}{\TestDriver}
%    \begin{macrocode}
\newcommand*{\TestDriver}[4]{%
  \subsection{Driver #1}%
  \@PackageInfo{soulutf8-test}{* Driver #1}%
  \expandafter\let\expandafter\testdriver\csname #1\endcsname
  \def\teststringA{#3}%
  \def\teststringB{#4}%
  \begin{qstest}{string-#2-driver-#1}{string,#2,driver,#1}%
    \@PackageInfo{soulutf8-test}{* Version: soulutf8}%
    \global\let\soulHstring\@empty
    \setbox0=\hbox{\expandafter\testdriver\expandafter{\teststringA}}%
    \let\HstringA\soulHstring
    \SetTestVBox{\boxA}{\teststringA}%
    %
    \@PackageInfo{soulutf8-test}{* Version: soul + patch}%
    \global\let\soulHstring\@empty
    \RestoreOrg{eval}%
    \setbox2=\hbox{\expandafter\testdriver\expandafter{\teststringB}}%
    \let\HstringB\soulHstring
    \SetTestVBox{\boxB}{\teststringB}%
    %
    \@PackageInfo{soulutf8-test}{* Version: original soul}%
    \global\let\soulHstring\@empty
    \RestoreOrg{splittoken}%
    \RestoreOrg{flushcomma}%
    \RestoreOrg{flushapo}%
    \RestoreOrg{grave}%
    \RestoreOrg{addgroup}%
    \RestoreOrg{addmath}%
    \RestoreOrg{addprotect}%
    \RestoreOrg{addtoken}%
    \setbox4=\hbox{\expandafter\testdriver\expandafter{\teststringB}}%
    \let\HstringC\soulHstring
    \SetTestVBox{\boxC}{\teststringB}%
    %
    \TestInfo{0}{soulutf8}%
    \TestInfo{2}{soul + patch}%
    \TestInfo{4}{original soul}%
    \TestInfoVBox
    \showboxdepth=\maxdimen
    \showboxbreadth=\maxdimen
    \tracingonline=\@ne
    \Expect*{\HstringA}*{\HstringB}%
    \Expect*{\HstringA}*{\HstringC}%
    \Expect*{\the\wd0}*{\the\wd2}%
    \ifdim\wd0=\wd2 %
    \else
      \showbox0 %
      \showbox2 %
    \fi
    \Expect*{\the\ht0}*{\the\ht2}%
    \Expect*{\the\dp0}*{\the\dp2}%
    \Expect*{\the\wd\boxB}*{\the\wd\boxA}%
    \Expect*{\the\wd\boxC}*{\the\wd\boxA}%
    \Expect*{\the\ht\boxB}*{\the\ht\boxA}%
    \ifdim\ht\boxB=\ht\boxA
    \else
      \showbox\boxA
      \showbox\boxB
    \fi
    \Expect*{\the\ht\boxC}*{\the\ht\boxA}%
    \Expect*{\the\dp\boxB}*{\the\dp\boxA}%
    \Expect*{\the\dp\boxC}*{\the\dp\boxA}%
  \end{qstest}%
}
%    \end{macrocode}
%    \end{macro}
%    \begin{macro}{\SetTestVBox}
%    \begin{macrocode}
\newcommand*{\SetTestVBox}[2]{%
  \setbox#1=\vbox{%
    \hsize=0pt\relax
    \hfuzz=100mm\relax
    \expandafter\testdriver\expandafter{#2}%
  }%
}
%    \end{macrocode}
%    \end{macro}
%    \begin{macro}{\TestInfo}
%    \begin{macrocode}
\newcommand*{\TestInfo}[2]{%
  \par
  \noindent
  \copy#1\relax
  \hfill
  \begingroup
    \small
    \ifdim\wd0=\wd2\relax
      \ifdim\wd0=\wd4\relax
      \else
        \ifnum#1=2 %
          \raisebox{.5\baselineskip}[0pt][0pt]{%
            \llap{$\}$ }%
            \rlap{width = \the\wd#1\relax\space$\{$}%
            \phantom{width = 999.99999pt}%
            \qquad
          }%
        \else
          \ifnum#1=4 %
            \rlap{width = \the\wd#1}%
            \phantom{width = 999.99999pt}%
            \qquad
          \fi
        \fi
      \fi
    \else
      \begingroup
        \bfseries
        \rlap{width = \the\wd#1}%
        \phantom{width = 999.99999pt}%
      \endgroup
      \qquad
    \fi
    \rlap{(#2)}%
    \phantom{(original soul)}%
  \endgroup
  \par
}
%    \end{macrocode}
%    \end{macro}
%    \begin{macro}{\TestInfoVBox}
%    \begin{macrocode}
\newcommand*{\TestInfoVBox}{%
  \par
  \noindent
  \mbox{\vrule\relax\space\copy\boxA}%
  \ifdim\ht\boxA=\ht\boxB
  \else
    \hfill
    \begingroup
      \small
      \rlap{(soulutf8)}%
      \phantom{(soulutf8)}%
    \endgroup
    \par
    \noindent
    \mbox{\vrule\relax\space\copy\boxB}%
    \hfill
    \begingroup
      \small
      \rlap{(soul)}%
      \phantom{(soulutf8)}%
    \endgroup
  \fi
  \par
}
%    \end{macrocode}
%    \end{macro}
%    \begin{macro}{\Test}
%    \begin{macrocode}
\makeatletter
\newcommand*{\Test}[3]{%
  \clearpage
  \begingroup
    \def\linebreak{\textbackslash linebreak }%
    \def\\{\textbackslash\textbackslash}%
    \def\soulomit##1{%
      \textbackslash soulomit\textbraceleft##1\textbraceright
    }%
    \def\mbox##1{%
      \textbackslash mbox\textbraceleft##1\textbraceright
    }%
    \def\abc{\textbackslash abc }%
    \def\texttt##1{%
      \textbackslash texttt\textbraceleft##1\textbraceright
    }%
    \def\c##1{%
      \textbackslash c\textbraceleft##1\textbraceright
    }%
%<*utf8x>
    \count@=128 %
    \loop
    \ifnum\count@<256 %
      \lccode`\~=\count@
      \lowercase{%
        \def~{\noexpand~}%
      }%
      \advance\count@\@ne
    \repeat
%</utf8x>
    \csname protected@edef\endcsname\x{\endgroup
      \noexpand\section{String #1: #2}%
    }%
  \x
  \TestDriver{so}{#1}{#2}{#3}%
  \TestDriver{caps}{#1}{#2}{#3}%
  \TestDriver{ul}{#1}{#2}{#3}%
  \TestDriver{st}{#1}{#2}{#3}%
  \TestDriver{hl}{#1}{#2}{#3}%
  \TestDriver{an}{#1}{#2}{#3}%
  \TestDriver{sy}{#1}{#2}{#3}%
  \TestDriver{soulH}{#1}{#2}{#3}%
}
\makeatother
%    \end{macrocode}
%    \end{macro}
%
% \subsubsection{Testing}
%
%    \begin{macrocode}
\begin{document}
  \Test A{}{}%
  \Test B{Hello World}{Hello World}%
  \Test C{%
    Umlauts: ^^c3^^84^^c3^^96^^c3^^9c^^c3^^a4^^c3^^b6^^c3^^bc^^c3^^9f%
  }{%
    Umlauts: \"A\"O\"U\"a\"o\"u\ss
  }%
  \Test D{Euro: ^^e2^^82^^ac}{Euro: \texteuro}%
  \Test E{Stra^^c3^^9fe}{Stra\ss e}%
  \Test F{R^^c3^^b6mer}{R\"omer}%
  \Test G{A^^e2^^80^^93B^^e2^^80^^94C}{A--B---C}%
  \Test H{X^^c2^^a0Y}{X~Y}%
  \Test I{A^^c5^^a4}{A\v T}%
  \Test J{A,B,,C,,,D,,,,E}{A,B,,C,,,D,,,,E}%
  \Test K{A'B''C'''D''''E}{A'B''C'''D''''E}%
  \Test L{A`B``C```D````E}{A`B``C```D````E}%
  \Test M{A$x$B}{A$x$B}%
  \Test N{A\mbox{abc}B}{A\mbox{abc}B}%
  \Test O{,,TA''}{,,TA''}%
  \Test P{{!`}Hola{?`}}{{!`}Hola{?`}}%
  \Test Q{re-sent\slash x}{re-sent\slash x}%
  \Test R{new\\line}{new\\line}%
  \Test S{foo\linebreak bar}{foo\linebreak bar}%
  \Test T{Virchow{sche}}{Virchow{sche}}%
  \Test U{A\soulomit{foo}B}{A\soulomit{foo}B}%
  \Test V{A\texttt{xyz}B}{A\texttt{xyz}B}%
  \Test W{ffi}{ffi}%
  \Test X{f{}f{}i}{f{}f{}i}%
  \Test Y{f\mbox{}f\mbox{}i}{f\mbox{}f\mbox{}i}%
  \Test Z{\c{\"a}}{{\c{\"a}}}%
%<*test2|test3>
  \let\abcx\UNDEFINED
  \def\abc{%
    \begingroup
      \def\abcx{x}%
      \abcx
    \endgroup
  }%
  \Test{AA}{[{x\abc x}]}{[{xxx}]}%
%</test2|test3>
%    \end{macrocode}
%
%    \begin{macrocode}
\end{document}
%    \end{macrocode}
%    \begin{macrocode}
%</test2|test3|test4|test5>
%    \end{macrocode}
%
% \section{Installation}
%
% \subsection{Download}
%
% \paragraph{Package.} This package is available on
% CTAN\footnote{\url{http://ctan.org/pkg/soulutf8}}:
% \begin{description}
% \item[\CTAN{macros/latex/contrib/oberdiek/soulutf8.dtx}] The source file.
% \item[\CTAN{macros/latex/contrib/oberdiek/soulutf8.pdf}] Documentation.
% \end{description}
%
%
% \paragraph{Bundle.} All the packages of the bundle `oberdiek'
% are also available in a TDS compliant ZIP archive. There
% the packages are already unpacked and the documentation files
% are generated. The files and directories obey the TDS standard.
% \begin{description}
% \item[\CTAN{install/macros/latex/contrib/oberdiek.tds.zip}]
% \end{description}
% \emph{TDS} refers to the standard ``A Directory Structure
% for \TeX\ Files'' (\CTAN{tds/tds.pdf}). Directories
% with \xfile{texmf} in their name are usually organized this way.
%
% \subsection{Bundle installation}
%
% \paragraph{Unpacking.} Unpack the \xfile{oberdiek.tds.zip} in the
% TDS tree (also known as \xfile{texmf} tree) of your choice.
% Example (linux):
% \begin{quote}
%   |unzip oberdiek.tds.zip -d ~/texmf|
% \end{quote}
%
% \paragraph{Script installation.}
% Check the directory \xfile{TDS:scripts/oberdiek/} for
% scripts that need further installation steps.
% Package \xpackage{attachfile2} comes with the Perl script
% \xfile{pdfatfi.pl} that should be installed in such a way
% that it can be called as \texttt{pdfatfi}.
% Example (linux):
% \begin{quote}
%   |chmod +x scripts/oberdiek/pdfatfi.pl|\\
%   |cp scripts/oberdiek/pdfatfi.pl /usr/local/bin/|
% \end{quote}
%
% \subsection{Package installation}
%
% \paragraph{Unpacking.} The \xfile{.dtx} file is a self-extracting
% \docstrip\ archive. The files are extracted by running the
% \xfile{.dtx} through \plainTeX:
% \begin{quote}
%   \verb|tex soulutf8.dtx|
% \end{quote}
%
% \paragraph{TDS.} Now the different files must be moved into
% the different directories in your installation TDS tree
% (also known as \xfile{texmf} tree):
% \begin{quote}
% \def\t{^^A
% \begin{tabular}{@{}>{\ttfamily}l@{ $\rightarrow$ }>{\ttfamily}l@{}}
%   soulutf8.sty & tex/generic/oberdiek/soulutf8.sty\\
%   soulutf8.pdf & doc/latex/oberdiek/soulutf8.pdf\\
%   test/soulutf8-test1.tex & doc/latex/oberdiek/test/soulutf8-test1.tex\\
%   test/soulutf8-test2.tex & doc/latex/oberdiek/test/soulutf8-test2.tex\\
%   test/soulutf8-test3.tex & doc/latex/oberdiek/test/soulutf8-test3.tex\\
%   test/soulutf8-test4.tex & doc/latex/oberdiek/test/soulutf8-test4.tex\\
%   test/soulutf8-test5.tex & doc/latex/oberdiek/test/soulutf8-test5.tex\\
%   soulutf8.dtx & source/latex/oberdiek/soulutf8.dtx\\
% \end{tabular}^^A
% }^^A
% \sbox0{\t}^^A
% \ifdim\wd0>\linewidth
%   \begingroup
%     \advance\linewidth by\leftmargin
%     \advance\linewidth by\rightmargin
%   \edef\x{\endgroup
%     \def\noexpand\lw{\the\linewidth}^^A
%   }\x
%   \def\lwbox{^^A
%     \leavevmode
%     \hbox to \linewidth{^^A
%       \kern-\leftmargin\relax
%       \hss
%       \usebox0
%       \hss
%       \kern-\rightmargin\relax
%     }^^A
%   }^^A
%   \ifdim\wd0>\lw
%     \sbox0{\small\t}^^A
%     \ifdim\wd0>\linewidth
%       \ifdim\wd0>\lw
%         \sbox0{\footnotesize\t}^^A
%         \ifdim\wd0>\linewidth
%           \ifdim\wd0>\lw
%             \sbox0{\scriptsize\t}^^A
%             \ifdim\wd0>\linewidth
%               \ifdim\wd0>\lw
%                 \sbox0{\tiny\t}^^A
%                 \ifdim\wd0>\linewidth
%                   \lwbox
%                 \else
%                   \usebox0
%                 \fi
%               \else
%                 \lwbox
%               \fi
%             \else
%               \usebox0
%             \fi
%           \else
%             \lwbox
%           \fi
%         \else
%           \usebox0
%         \fi
%       \else
%         \lwbox
%       \fi
%     \else
%       \usebox0
%     \fi
%   \else
%     \lwbox
%   \fi
% \else
%   \usebox0
% \fi
% \end{quote}
% If you have a \xfile{docstrip.cfg} that configures and enables \docstrip's
% TDS installing feature, then some files can already be in the right
% place, see the documentation of \docstrip.
%
% \subsection{Refresh file name databases}
%
% If your \TeX~distribution
% (\teTeX, \mikTeX, \dots) relies on file name databases, you must refresh
% these. For example, \teTeX\ users run \verb|texhash| or
% \verb|mktexlsr|.
%
% \subsection{Some details for the interested}
%
% \paragraph{Attached source.}
%
% The PDF documentation on CTAN also includes the
% \xfile{.dtx} source file. It can be extracted by
% AcrobatReader 6 or higher. Another option is \textsf{pdftk},
% e.g. unpack the file into the current directory:
% \begin{quote}
%   \verb|pdftk soulutf8.pdf unpack_files output .|
% \end{quote}
%
% \paragraph{Unpacking with \LaTeX.}
% The \xfile{.dtx} chooses its action depending on the format:
% \begin{description}
% \item[\plainTeX:] Run \docstrip\ and extract the files.
% \item[\LaTeX:] Generate the documentation.
% \end{description}
% If you insist on using \LaTeX\ for \docstrip\ (really,
% \docstrip\ does not need \LaTeX), then inform the autodetect routine
% about your intention:
% \begin{quote}
%   \verb|latex \let\install=y\input{soulutf8.dtx}|
% \end{quote}
% Do not forget to quote the argument according to the demands
% of your shell.
%
% \paragraph{Generating the documentation.}
% You can use both the \xfile{.dtx} or the \xfile{.drv} to generate
% the documentation. The process can be configured by the
% configuration file \xfile{ltxdoc.cfg}. For instance, put this
% line into this file, if you want to have A4 as paper format:
% \begin{quote}
%   \verb|\PassOptionsToClass{a4paper}{article}|
% \end{quote}
% An example follows how to generate the
% documentation with pdf\LaTeX:
% \begin{quote}
%\begin{verbatim}
%pdflatex soulutf8.dtx
%makeindex -s gind.ist soulutf8.idx
%pdflatex soulutf8.dtx
%makeindex -s gind.ist soulutf8.idx
%pdflatex soulutf8.dtx
%\end{verbatim}
% \end{quote}
%
% \begin{thebibliography}{9}
%
% \bibitem{soul}
%   Melchior Franz: \textit{The \xpackage{soul} package};
%   2003/11/17;\\
%   \CTAN{macros/latex/contrib/soul/}.
%
% \bibitem{ucs}
%   Dominique P. G. Unruh: \textit{\xpackage{ucs.sty} -- Unicode Support};
%   2004/10/17;\\
%   \CTAN{macros/latex/contrib/unicode/}.
%
% \bibitem{utf8}
%   Frank Mittelbach, Chris Rowley:
%   \textit{Providing some UTF-8 support via \xpackage{inputenc}};
%   2006/03/30;\\
%   \CTAN{macros/latex/base/utf8ienc.dtx}.
%
% \end{thebibliography}
%
% \begin{History}
%   \begin{Version}{2007/09/09 v1.0}
%   \item
%     First version.
%   \end{Version}
%   \begin{Version}{2016/05/16 v1.1}
%   \item
%     Documentation updates.
%   \end{Version}
% \end{History}
%
% \PrintIndex
%
% \Finale
\endinput

%        (quote the arguments according to the demands of your shell)
%
% Documentation:
%    (a) If soulutf8.drv is present:
%           latex soulutf8.drv
%    (b) Without soulutf8.drv:
%           latex soulutf8.dtx; ...
%    The class ltxdoc loads the configuration file ltxdoc.cfg
%    if available. Here you can specify further options, e.g.
%    use A4 as paper format:
%       \PassOptionsToClass{a4paper}{article}
%
%    Programm calls to get the documentation (example):
%       pdflatex soulutf8.dtx
%       makeindex -s gind.ist soulutf8.idx
%       pdflatex soulutf8.dtx
%       makeindex -s gind.ist soulutf8.idx
%       pdflatex soulutf8.dtx
%
% Installation:
%    TDS:tex/generic/oberdiek/soulutf8.sty
%    TDS:doc/latex/oberdiek/soulutf8.pdf
%    TDS:doc/latex/oberdiek/test/soulutf8-test1.tex
%    TDS:doc/latex/oberdiek/test/soulutf8-test2.tex
%    TDS:doc/latex/oberdiek/test/soulutf8-test3.tex
%    TDS:doc/latex/oberdiek/test/soulutf8-test4.tex
%    TDS:doc/latex/oberdiek/test/soulutf8-test5.tex
%    TDS:source/latex/oberdiek/soulutf8.dtx
%
%<*ignore>
\begingroup
  \catcode123=1 %
  \catcode125=2 %
  \def\x{LaTeX2e}%
\expandafter\endgroup
\ifcase 0\ifx\install y1\fi\expandafter
         \ifx\csname processbatchFile\endcsname\relax\else1\fi
         \ifx\fmtname\x\else 1\fi\relax
\else\csname fi\endcsname
%</ignore>
%<*install>
\input docstrip.tex
\Msg{************************************************************************}
\Msg{* Installation}
\Msg{* Package: soulutf8 2016/05/16 v1.1 Permit use of UTF-8 characters in soul (HO)}
\Msg{************************************************************************}

\keepsilent
\askforoverwritefalse

\let\MetaPrefix\relax
\preamble

This is a generated file.

Project: soulutf8
Version: 2016/05/16 v1.1

Copyright (C) 2007 by
   Heiko Oberdiek <heiko.oberdiek at googlemail.com>

This work may be distributed and/or modified under the
conditions of the LaTeX Project Public License, either
version 1.3c of this license or (at your option) any later
version. This version of this license is in
   http://www.latex-project.org/lppl/lppl-1-3c.txt
and the latest version of this license is in
   http://www.latex-project.org/lppl.txt
and version 1.3 or later is part of all distributions of
LaTeX version 2005/12/01 or later.

This work has the LPPL maintenance status "maintained".

This Current Maintainer of this work is Heiko Oberdiek.

The Base Interpreter refers to any `TeX-Format',
because some files are installed in TDS:tex/generic//.

This work consists of the main source file soulutf8.dtx
and the derived files
   soulutf8.sty, soulutf8.pdf, soulutf8.ins, soulutf8.drv,
   soulutf8-test1.tex, soulutf8-test2.tex, soulutf8-test3.tex,
   soulutf8-test4.tex, soulutf8-test5.tex.

\endpreamble
\let\MetaPrefix\DoubleperCent

\generate{%
  \file{soulutf8.ins}{\from{soulutf8.dtx}{install}}%
  \file{soulutf8.drv}{\from{soulutf8.dtx}{driver}}%
  \usedir{tex/generic/oberdiek}%
  \file{soulutf8.sty}{\from{soulutf8.dtx}{package}}%
%  \usedir{doc/latex/oberdiek/test}%
%  \file{soulutf8-test1.tex}{\from{soulutf8.dtx}{test1}}%
%  \file{soulutf8-test2.tex}{\from{soulutf8.dtx}{test2,utf8}}%
%  \file{soulutf8-test3.tex}{\from{soulutf8.dtx}{test3,utf8x}}%
%  \file{soulutf8-test4.tex}{\from{soulutf8.dtx}{test4,utf8,noetex}}%
%  \file{soulutf8-test5.tex}{\from{soulutf8.dtx}{test5,utf8x,noetex}}%
  \nopreamble
  \nopostamble
%  \usedir{source/latex/oberdiek/catalogue}%
%  \file{soulutf8.xml}{\from{soulutf8.dtx}{catalogue}}%
}

\catcode32=13\relax% active space
\let =\space%
\Msg{************************************************************************}
\Msg{*}
\Msg{* To finish the installation you have to move the following}
\Msg{* file into a directory searched by TeX:}
\Msg{*}
\Msg{*     soulutf8.sty}
\Msg{*}
\Msg{* To produce the documentation run the file `soulutf8.drv'}
\Msg{* through LaTeX.}
\Msg{*}
\Msg{* Happy TeXing!}
\Msg{*}
\Msg{************************************************************************}

\endbatchfile
%</install>
%<*ignore>
\fi
%</ignore>
%<*driver>
\NeedsTeXFormat{LaTeX2e}
\ProvidesFile{soulutf8.drv}%
  [2016/05/16 v1.1 Permit use of UTF-8 characters in soul (HO)]%
\documentclass{ltxdoc}
\usepackage{holtxdoc}[2011/11/22]
\begin{document}
  \DocInput{soulutf8.dtx}%
\end{document}
%</driver>
% \fi
%
%
% \CharacterTable
%  {Upper-case    \A\B\C\D\E\F\G\H\I\J\K\L\M\N\O\P\Q\R\S\T\U\V\W\X\Y\Z
%   Lower-case    \a\b\c\d\e\f\g\h\i\j\k\l\m\n\o\p\q\r\s\t\u\v\w\x\y\z
%   Digits        \0\1\2\3\4\5\6\7\8\9
%   Exclamation   \!     Double quote  \"     Hash (number) \#
%   Dollar        \$     Percent       \%     Ampersand     \&
%   Acute accent  \'     Left paren    \(     Right paren   \)
%   Asterisk      \*     Plus          \+     Comma         \,
%   Minus         \-     Point         \.     Solidus       \/
%   Colon         \:     Semicolon     \;     Less than     \<
%   Equals        \=     Greater than  \>     Question mark \?
%   Commercial at \@     Left bracket  \[     Backslash     \\
%   Right bracket \]     Circumflex    \^     Underscore    \_
%   Grave accent  \`     Left brace    \{     Vertical bar  \|
%   Right brace   \}     Tilde         \~}
%
% \GetFileInfo{soulutf8.drv}
%
% \title{The \xpackage{soulutf8} package}
% \date{2016/05/16 v1.1}
% \author{Heiko Oberdiek\thanks
% {Please report any issues at https://github.com/ho-tex/oberdiek/issues}\\
% \xemail{heiko.oberdiek at googlemail.com}}
%
% \maketitle
%
% \begin{abstract}
% This package extends package \xpackage{soul} and adds some support
% for UTF-8. Namely the input encodings \xfile{utf8.def}
% from package \xpackage{inputenc} and
% package \xpackage{ucs}'s \xfile{utf8x.def} are supported.
% \end{abstract}
%
% \tableofcontents
%
% \section{Documentation}
%
% This package \xpackage{soulutf8} does not have own options and does not
% define new user commands.
% Any option is passed to package \xpackage{soul} \cite{soul}
% that is loaded first.
% Then some internal macros of \xpackage{soul} are redefined to add
% support for UTF-8.
% The following input encodings are supported:
% \begin{quote}
%   \begin{tabular}{@{}lll@{}}
%     \texttt{utf8} & \LaTeX\ base &
%        \xfile{TDS:tex/latex/base/utf8.def} \cite{utf8}\\
%     \texttt{utf8x} & Package \xpackage{ucs} &
%       \xfile{TDS:tex/latex/ucs/utf8x.def} \cite{ucs}\\
%   \end{tabular}
% \end{quote}
% UTF-8 byte sequences are added as token group to a word, even
% if these UTF-8 characters are some kind of hyphen or space.
% As exception the following three Unicode characters are handled specially:
% \begin{quote}
%   \begin{tabular}{lll}
%     Slot & Name & Action\\
%     \hline
%     \texttt{U+00A0} & NO-BREAK SPACE & like |~|\\
%     \texttt{U+2013} & EN DASH & |--|\\
%     \texttt{U+2014} & EM DASH & |---|
%   \end{tabular}
%
% \subsection{Patch}
%
% Also package \xpackage{soulutf8} tries to patch package \xpackage{soul}
% to improve its behaviour:
% \begin{itemize}
% \item
%   A problem with additional levels of curly braces is fixed.
%   As advantage more implicite kernings are detected.
%   However, the result may be incompatible with the
%   original behaviour of package \xpackage{soul} because
%   of these respected implicite kernings.
% \item
%   \eTeX\ , especially \cs{unexpanded} is supported.
%   This allows a better protection of token groups
%   (|\mbox{|\dots|}|, math, \dots).
% \end{itemize}
% \end{quote}
%
% \subsection{Future}
%
% Currently package \xpackage{soul} does not seem to be
% maintained. Nevertheless if there will be a new version that
% adds support for UTF-8, then this package may become
% obsolete.
%
% \StopEventually{
% }
%
% \section{Implementation}
%
%    \begin{macrocode}
%<*package>
%    \end{macrocode}
%
% \subsection{Reload check and package identification}
%    Reload check, especially if the package is not used with \LaTeX.
%    \begin{macrocode}
\begingroup\catcode61\catcode48\catcode32=10\relax%
  \catcode13=5 % ^^M
  \endlinechar=13 %
  \catcode35=6 % #
  \catcode39=12 % '
  \catcode44=12 % ,
  \catcode45=12 % -
  \catcode46=12 % .
  \catcode58=12 % :
  \catcode64=11 % @
  \catcode123=1 % {
  \catcode125=2 % }
  \expandafter\let\expandafter\x\csname ver@soulutf8.sty\endcsname
  \ifx\x\relax % plain-TeX, first loading
  \else
    \def\empty{}%
    \ifx\x\empty % LaTeX, first loading,
      % variable is initialized, but \ProvidesPackage not yet seen
    \else
      \expandafter\ifx\csname PackageInfo\endcsname\relax
        \def\x#1#2{%
          \immediate\write-1{Package #1 Info: #2.}%
        }%
      \else
        \def\x#1#2{\PackageInfo{#1}{#2, stopped}}%
      \fi
      \x{soulutf8}{The package is already loaded}%
      \aftergroup\endinput
    \fi
  \fi
\endgroup%
%    \end{macrocode}
%    Package identification:
%    \begin{macrocode}
\begingroup\catcode61\catcode48\catcode32=10\relax%
  \catcode13=5 % ^^M
  \endlinechar=13 %
  \catcode35=6 % #
  \catcode39=12 % '
  \catcode40=12 % (
  \catcode41=12 % )
  \catcode44=12 % ,
  \catcode45=12 % -
  \catcode46=12 % .
  \catcode47=12 % /
  \catcode58=12 % :
  \catcode64=11 % @
  \catcode91=12 % [
  \catcode93=12 % ]
  \catcode123=1 % {
  \catcode125=2 % }
  \expandafter\ifx\csname ProvidesPackage\endcsname\relax
    \def\x#1#2#3[#4]{\endgroup
      \immediate\write-1{Package: #3 #4}%
      \xdef#1{#4}%
    }%
  \else
    \def\x#1#2[#3]{\endgroup
      #2[{#3}]%
      \ifx#1\@undefined
        \xdef#1{#3}%
      \fi
      \ifx#1\relax
        \xdef#1{#3}%
      \fi
    }%
  \fi
\expandafter\x\csname ver@soulutf8.sty\endcsname
\ProvidesPackage{soulutf8}%
  [2016/05/16 v1.1 Permit use of UTF-8 characters in soul (HO)]%
%    \end{macrocode}
%
% \subsection{Catcodes}
%
%    \begin{macrocode}
\begingroup\catcode61\catcode48\catcode32=10\relax%
  \catcode13=5 % ^^M
  \endlinechar=13 %
  \catcode123=1 % {
  \catcode125=2 % }
  \catcode64=11 % @
  \def\x{\endgroup
    \expandafter\edef\csname SOuL@AtEnd\endcsname{%
      \endlinechar=\the\endlinechar\relax
      \catcode13=\the\catcode13\relax
      \catcode32=\the\catcode32\relax
      \catcode35=\the\catcode35\relax
      \catcode61=\the\catcode61\relax
      \catcode64=\the\catcode64\relax
      \catcode123=\the\catcode123\relax
      \catcode125=\the\catcode125\relax
    }%
  }%
\x\catcode61\catcode48\catcode32=10\relax%
\catcode13=5 % ^^M
\endlinechar=13 %
\catcode35=6 % #
\catcode64=11 % @
\catcode123=1 % {
\catcode125=2 % }
\def\TMP@EnsureCode#1#2{%
  \edef\SOuL@AtEnd{%
    \SOuL@AtEnd
    \catcode#1=\the\catcode#1\relax
  }%
  \catcode#1=#2\relax
}
\TMP@EnsureCode{10}{12}% ^^J
\TMP@EnsureCode{33}{12}% !
\TMP@EnsureCode{34}{12}% "
\TMP@EnsureCode{36}{3}% $
\TMP@EnsureCode{39}{12}% '
\TMP@EnsureCode{40}{12}% (
\TMP@EnsureCode{41}{12}% )
\TMP@EnsureCode{42}{12}% *
\TMP@EnsureCode{43}{12}% +
\TMP@EnsureCode{44}{12}% ,
\TMP@EnsureCode{45}{12}% -
\TMP@EnsureCode{46}{12}% .
\TMP@EnsureCode{47}{12}% /
\TMP@EnsureCode{58}{12}% :
\TMP@EnsureCode{60}{12}% <
\TMP@EnsureCode{62}{12}% >
\TMP@EnsureCode{91}{12}% [
\TMP@EnsureCode{93}{12}% ]
\TMP@EnsureCode{94}{7}% ^
\TMP@EnsureCode{96}{12}% `
\TMP@EnsureCode{126}\active % ~
\TMP@EnsureCode{128}{12}% ^^80
\TMP@EnsureCode{147}{12}% ^^93
\TMP@EnsureCode{148}{12}% ^^94
\TMP@EnsureCode{160}{12}% ^^a0
\TMP@EnsureCode{194}{12}% ^^c2
\TMP@EnsureCode{226}{12}% ^^e2
\edef\SOuL@AtEnd{\SOuL@AtEnd\noexpand\endinput}
%    \end{macrocode}
%
% \subsection{Loading packages}
%
%    Package \xpackage{soul} uses \cs{documentclass} to detect \LaTeX.
%    \begin{macrocode}
\ifx\documentclass\@undefined
%    \end{macrocode}
%
% \subsubsection{\plainTeX}
%
%    First we check, whether package \xpackage{soul} is already loaded.
%    \begin{macrocode}
  \expandafter\ifx\csname SOUL@\endcsname\relax
%    \end{macrocode}
%    In case of \plainTeX\ package \xpackage{soul} defines some
%    macros in a simple manner that will break the definitions of
%    \xfile{miniltx.tex}, for example. Therefore these macros
%    are first saved and restored afterwards.
%    \begin{macrocode}
    \let\SOuL@orgDeclareRobustCommand\DeclareRobustCommand
    \let\SOuL@orgnewcommand          \newcommand
    \let\SOuL@orgDeclareOption       \DeclareOption
    \let\SOuL@orgPackageError        \PackageError
    \def\SOuL@restorelatexcmds{%
      \let\DeclareRobustCommand\SOuL@orgDeclareRobustCommand
      \let\newcommand          \SOuL@orgnewcommand
      \let\DeclareOption       \SOuL@orgDeclareOption
      \let\PackageError        \SOuL@orgPackageError
    }%
    \input soul.sty\relax
    \SOuL@restorelatexcmds
  \fi
%    \end{macrocode}
%    \begin{macro}{\SOUL@error}
%     Package \xpackage{soul}'s use of \cs{PackageError} is replaced
%     by \cs{@PackageError} of package \xpackage{infwarerr}.
%    \begin{macrocode}
  \input infwarerr.sty\relax
  \let\SOuL@orgSOUL@error\SOUL@error
  \def\SOUL@error{%
    \begingroup
      \let\PackageError\@PackageError
      \SOuL@orgSOUL@error
    \endgroup
  }%
%    \end{macrocode}
%    \end{macro}
%    \begin{macrocode}
  \input etexcmds.sty\relax
%    \end{macrocode}
%    \begin{macro}{\@onelevel@sanitize}
%    Define \LaTeX's \cs{@onelevel@sanitize} if not already available.
%    \begin{macrocode}
  \expandafter\ifx\csname @onelevel@sanitize\endcsname\relax
    \def\@onelevel@sanitize#1{%
      \edef#1{%
        \expandafter\strip@prefix\meaning#1%
      }%
    }%
%    \end{macrocode}
%    \end{macro}
%    \begin{macro}{\strip@prefix}
%    \begin{macrocode}
    \def\strip@prefix#1>{}%
%    \end{macrocode}
%    \end{macro}
%    \begin{macrocode}
  \fi
\else
%    \end{macrocode}
%
% \subsubsection{\LaTeX}
%    \begin{macrocode}
  \DeclareOption*{\PassOptionsToPackage{\CurrentOption}{soul}}%
  \ProcessOptions\relax
  \RequirePackage{soul}[2003/11/17]%
  \RequirePackage{infwarerr}[2016/05/16]%
  \RequirePackage{etexcmds}[2016/05/16]%
\fi
%    \end{macrocode}
%
% \subsubsection{\eTeX}
%
%    In \plainTeX\ command \cs{+} is an \emph{outer} macro.
%    Therefore numbers are used to avoid problems.
%    \begin{macrocode}
\ifetex@unexpanded
  \catcode33=14 % '!': comment
  \catcode43=9  % '+': ignore
\else
  \catcode33=9  % '!': ignore
  \catcode43=14 % '+': comment
\fi
%    \end{macrocode}
%
% \subsection{Macro for redefinitions}
%
%    \begin{macro}{\SOuL@redefine}
%    \begin{macrocode}
\def\SOuL@redefine#1{%
  \begingroup
    \def\SOuL@cmd{#1}%
    \afterassignment\SOuL@cmdcheck
    \def\SOuL@temp
}
%    \end{macrocode}
%    \end{macro}
%    \begin{macro}{\SOuL@cmdcheck}
%    \begin{macrocode}
\def\SOuL@cmdcheck{%
    \expandafter\ifx\SOuL@cmd\SOuL@temp
    \else
      \edef\SOuL@temp*{\expandafter\string\SOuL@cmd}%
      \@PackageWarningNoLine{soulutf8}{%
        Command \SOuL@temp* has changed.\MessageBreak
        Supported versions of package `soul': 2003/11/17.\MessageBreak
        Depending on the unknown changes the redefinition\MessageBreak
        of \SOuL@temp* may not behave correctly%
      }%
    \fi
  \expandafter\endgroup
  \expandafter\def\SOuL@cmd
}
%    \end{macrocode}
%    \end{macro}
%
% \subsection{Redefinition of \cs{SOUL@eval}}
%
%    \begin{macro}{\SOUL@eval}
%    Macro \cs{SOUL@eval} is redefined to add detection of
%    the first byte of a UTF-8 sequence. Because \cs{SOUL@eval}
%    is overwritten, a warning is issued, if the contents of
%    \cs{SOUL@eval} is not as expected.
%    \begin{macrocode}
\SOuL@redefine\SOUL@eval{%
%    \end{macrocode}
%    First the expected definition.
%    \begin{macrocode}
  \def\SOUL@n*##1{\SOUL@scan}%
  \if\noexpand\SOUL@@\SOUL@spc
  \else
    \SOUL@ignorespacesfalse
  \fi
  \ifnum\SOUL@minus=\thr@@
    \SOUL@flushminus
  \else\ifnum\SOUL@comma=\tw@
    \SOUL@flushcomma
  \else\ifnum\SOUL@apo=\tw@
    \SOUL@flushapo
  \else\ifnum\SOUL@grave=\tw@
    \SOUL@flushgrave
  \fi\fi\fi\fi
  \ifx\SOUL@@-\else\SOUL@flushminus\fi
  \ifx\SOUL@@,\else\SOUL@flushcomma\fi
  \ifx\SOUL@@'\else\SOUL@flushapo\fi
  \ifx\SOUL@@`\else\SOUL@flushgrave\fi
  \ifx\SOUL@@-%
    \advance\SOUL@minus\@ne
  \else\ifx\SOUL@@,%
    \advance\SOUL@comma\@ne
  \else\ifx\SOUL@@'%
    \advance\SOUL@apo\@ne
  \else\ifx\SOUL@@`%
    \advance\SOUL@grave\@ne
  \else
    \SOUL@flushminus
    \SOUL@flushcomma
    \SOUL@flushapo
    \SOUL@flushgrave
    \ifx\SOUL@@\SOUL@stop
      \def\SOUL@n*{%
        \SOUL@doword
        \SOUL@eventuallyexhyphen\null
      }%
    \else\ifx\SOUL@@\par
      \def\SOUL@n*\par{\par\leavevmode\SOUL@scan}%
    \else\if\noexpand\SOUL@@\SOUL@spc
      \SOUL@doword
      \SOUL@eventuallyexhyphen\null
      \ifSOUL@ignorespaces
      \else
        \SOUL@everyspace{}%
      \fi
      \def\SOUL@n* {\SOUL@scan}%
    \else\ifx\SOUL@@\\%
      \SOUL@doword
      \SOUL@eventuallyexhyphen\null
      \SOUL@everyspace{\unskip\nobreak\hfil\break}%
      \SOUL@ignorespacestrue
    \else\ifx\SOUL@@~%
      \SOUL@doword
      \SOUL@eventuallyexhyphen\null
      \SOUL@everyspace{\nobreak}%
    \else\ifx\SOUL@@\slash
      \SOUL@doword
      \SOUL@eventuallyexhyphen{/}%
      \SOUL@exhyphen{/}%
    \else\ifx\SOUL@@\mbox
      \def\SOUL@n*{\SOUL@addprotect}%
    \else\ifx\SOUL@@\hbox
      \def\SOUL@n*{\SOUL@addprotect}%
    \else\ifx\SOUL@@\soulomit
      \def\SOUL@n*\soulomit##1{%
        \SOUL@doword
        {\spaceskip\SOUL@spaceskip##1}%
        \SOUL@scan
      }%
    \else\ifx\SOUL@@\break
      \SOUL@doword
      \break
    \else\ifx\SOUL@@\linebreak
      \SOUL@doword
      \SOUL@everyspace{\linebreak}%
    \else\ifcat\bgroup\noexpand\SOUL@@
      \def\SOUL@n*{\SOUL@addgroup{}}%
    \else\ifcat$\noexpand\SOUL@@
      \def\SOUL@n*{\SOUL@addmath}%
    \else
      \def\SOUL@n*{\SOUL@dotoken}%
    \fi\fi\fi\fi\fi\fi\fi\fi\fi\fi\fi\fi\fi
  \fi\fi\fi\fi
  \SOUL@n*%
}{%
%    \end{macrocode}
%    Now the redefined version follows.
%    \begin{macrocode}
  \def\SOUL@n*##1{\SOUL@scan}%
  \if\noexpand\SOUL@@\SOUL@spc
  \else
    \SOUL@ignorespacesfalse
  \fi
  \ifnum\SOUL@minus=\thr@@
    \SOUL@flushminus
  \else\ifnum\SOUL@comma=\tw@
    \SOUL@flushcomma
  \else\ifnum\SOUL@apo=\tw@
    \SOUL@flushapo
  \else\ifnum\SOUL@grave=\tw@
    \SOUL@flushgrave
  \fi\fi\fi\fi
  \ifx\SOUL@@-\else\SOUL@flushminus\fi
  \ifx\SOUL@@,\else\SOUL@flushcomma\fi
  \ifx\SOUL@@'\else\SOUL@flushapo\fi
  \ifx\SOUL@@`\else\SOUL@flushgrave\fi
  \ifx\SOUL@@-%
    \advance\SOUL@minus\@ne
  \else\ifx\SOUL@@,%
    \advance\SOUL@comma\@ne
  \else\ifx\SOUL@@'%
    \advance\SOUL@apo\@ne
  \else\ifx\SOUL@@`%
    \advance\SOUL@grave\@ne
  \else
    \SOUL@flushminus
    \SOUL@flushcomma
    \SOUL@flushapo
    \SOUL@flushgrave
    \ifx\SOUL@@\SOUL@stop
      \def\SOUL@n*{%
          \SOUL@doword
          \SOUL@eventuallyexhyphen\null
      }%
    \else\ifx\SOUL@@\par
      \def\SOUL@n*\par{\par\leavevmode\SOUL@scan}%
    \else\if\noexpand\SOUL@@\SOUL@spc
      \SOUL@doword
      \SOUL@eventuallyexhyphen\null
      \ifSOUL@ignorespaces
      \else
        \SOUL@everyspace{}%
      \fi
      \def\SOUL@n* {\SOUL@scan}%
    \else\ifx\SOUL@@\\%
      \SOUL@doword
      \SOUL@eventuallyexhyphen\null
      \SOUL@everyspace{\unskip\nobreak\hfil\break}%
      \SOUL@ignorespacestrue
    \else\ifx\SOUL@@~%
      \SOUL@doword
      \SOUL@eventuallyexhyphen\null
      \SOUL@everyspace{\nobreak}%
    \else\ifx\SOUL@@\slash
      \SOUL@doword
      \SOUL@eventuallyexhyphen{/}%
      \SOUL@exhyphen{/}%
    \else\ifx\SOUL@@\mbox
      \def\SOUL@n*{\SOUL@addprotect}%
    \else\ifx\SOUL@@\hbox
      \def\SOUL@n*{\SOUL@addprotect}%
    \else\ifx\SOUL@@\soulomit
      \def\SOUL@n*\soulomit##1{%
        \SOUL@doword
        {\spaceskip\SOUL@spaceskip##1}%
        \SOUL@scan
      }%
    \else\ifx\SOUL@@\break
      \SOUL@doword
      \break
    \else\ifx\SOUL@@\linebreak
      \SOUL@doword
      \SOUL@everyspace{\linebreak}%
    \else\ifcat\bgroup\noexpand\SOUL@@
      \def\SOUL@n*{\SOUL@addgroup{}}%
    \else\ifcat$\noexpand\SOUL@@
      \def\SOUL@n*{\SOUL@addmath}%
    \else
%    \end{macrocode}
%    The current token is examined to detect the start of
%    a UTF-8 sequence.
%    \begin{macrocode}
      \SOuL@analyzeutfviii
      \ifcase\SOuL@octets
        \SOuL@analyzeutfviiix
      \fi
      \ifcase\SOuL@octets
        \def\SOUL@n*{\SOUL@dotoken}%
      \or % 1
      \or % 2
        \def\SOUL@n*{\SOuL@addtwooctets}%
      \or % 3
        \def\SOUL@n*{\SOuL@addthreeoctets}%
      \or % 4
        \def\SOUL@n*{\SOuL@addfouroctets}%
      \fi
    \fi\fi\fi\fi\fi\fi\fi\fi\fi\fi\fi\fi\fi
  \fi\fi\fi\fi
  \SOUL@n*%
}
%    \end{macrocode}
%    \end{macro}
%
% \subsection{UTF-8 analysis}
%
% \subsubsection{Help strings}
%
%    \begin{macrocode}
\def\SOuL@defsanitizedstring#1#2{%
  \expandafter\def\csname SOuL@string#1\endcsname{#2}%
  \expandafter\@onelevel@sanitize\csname SOuL@string#1\endcsname
}
\SOuL@defsanitizedstring{UTFviii}{UTFviii@}
\SOuL@defsanitizedstring{octets}{@octets}
\SOuL@defsanitizedstring{two}{two}
\SOuL@defsanitizedstring{three}{three}
\SOuL@defsanitizedstring{four}{four}
\SOuL@defsanitizedstring{macrocolon}{macro:}
\SOuL@defsanitizedstring{csnameu}{csname u8-}
\SOuL@defsanitizedstring{undeferr}{utf@viii@undeferr}
\def\SOuL@stringendash{^^e2^^80^^93}
\def\SOuL@stringemdash{^^e2^^80^^94}
\def\SOuL@stringnobreakspace{^^c2^^a0}
\edef\SOuL@charhash{\string #}
\edef\SOuL@chartwo{\string 2}
\edef\SOuL@charthree{\string 3}
\def\SOuL@empty{}
%    \end{macrocode}
%
% \subsubsection{Support for \xfile{utf8.def}}
%
%    \begin{macro}{\SOuL@analyzeutfviii}
%    \begin{macrocode}
\begingroup
  \edef\x{\endgroup
    \def\noexpand\SOuL@analyzeutfviii{%
      \noexpand\expandafter\noexpand\SOuL@checkutfviii
      \noexpand\meaning\noexpand\SOUL@@
      \SOuL@stringUTFviii\SOuL@stringoctets
      \noexpand\@nil
    }%
    \def\noexpand\SOuL@checkutfviii
      ##1\SOuL@stringUTFviii##2\SOuL@stringoctets##3\noexpand\@nil
  }%
\x{%
  \def\SOuL@temp{#2}%
  \chardef\SOuL@octets=%
      \ifx\SOuL@temp\SOuL@stringtwo
        \tw@
      \else\ifx\SOuL@temp\SOuL@stringthree
        \thr@@
      \else\ifx\SOuL@temp\SOuL@stringfour
        4 %
      \else
        \z@
      \fi\fi\fi
}
%    \end{macrocode}
%    \end{macro}
%
% \subsubsection{Support for \xfile{utf8x.def}}
%
%    \begin{macro}{\SOuL@analyzeutfviiix}
%    \begin{macrocode}
\begingroup
  \edef\x{\endgroup
    \def\noexpand\SOuL@analyzeutfviiix{%
      \noexpand\expandafter\noexpand\SOuL@checkutfviiix
      \noexpand\meaning\noexpand\SOUL@@
      \SOuL@stringmacrocolon\SOuL@charhash1{}{}{}{}%
      \SOuL@stringcsnameu\SOuL@stringundeferr
      \noexpand\@nil
    }%
%    \end{macrocode}
%    \end{macro}
%    \begin{macro}{\SOuL@checkutfviiix}
%    \begin{macrocode}
    \def\noexpand\SOuL@checkutfviiix
      ##1\SOuL@stringmacrocolon\SOuL@charhash1##2##3##4##5##6%
      \SOuL@stringcsnameu##7\SOuL@stringundeferr##8\noexpand\@nil
  }%
\x{%
  \def\SOuL@temp{#7}%
  \ifx\SOuL@temp\SOuL@empty
    \chardef\SOuL@octets=\z@
  \else
    \def\SOuL@temp{#5}%
    \ifx\SOuL@temp\SOuL@charthree
      \chardef\SOuL@octets=4 %
    \else
      \def\SOuL@temp{#3}%
      \ifx\SOuL@temp\SOuL@chartwo
        \chardef\SOuL@octets=\thr@@
      \else
        \chardef\SOuL@octets=\tw@
      \fi
    \fi
  \fi
}
%    \end{macrocode}
%    \end{macro}
%
% \subsection{Actions for UTF-8 sequences}
%
%    \begin{macro}{\SOuL@addtwooctets}
%    \begin{macrocode}
\def\SOuL@addtwooctets#1#2{%
  \def\SOuL@temp{#1#2}%
  \@onelevel@sanitize\SOuL@temp
  \ifx\SOuL@temp\SOuL@stringnobreakspace
    \SOUL@doword
    \SOUL@eventuallyexhyphen\null
    \SOUL@everyspace{\nobreak}%
    \let\SOuL@next\SOUL@scan
  \else
    \def\SOuL@next{%
!     \SOUL@addtoken{{\noexpand#1\noexpand#2}}%
+     \SOUL@addtoken{{\etex@unexpanded{#1#2}}}%
    }%
  \fi
  \SOuL@next
}
%    \end{macrocode}
%    \end{macro}
%    \begin{macro}{\SOuL@addthreeoctets}
%    \begin{macrocode}
\def\SOuL@addthreeoctets#1#2#3{%
  \def\SOuL@temp{#1#2#3}%
  \@onelevel@sanitize\SOuL@temp
  \ifx\SOuL@temp\SOuL@stringendash
    \SOUL@doword
    \SOUL@eventuallyexhyphen{-}%
    \SOUL@exhyphen{--}%
    \let\SOuL@next\SOUL@scan
  \else
    \ifx\SOuL@temp\SOuL@stringemdash
      \SOUL@doword
      \SOUL@eventuallyexhyphen{-}%
      \SOUL@exhyphen{---}%
      \let\SOuL@next\SOUL@scan
    \else
      \def\SOuL@next{%
!       \SOUL@addtoken{{\noexpand#1\noexpand#2\noexpand#3}}%
+       \SOUL@addtoken{{\etex@unexpanded{#1#2#3}}}%
      }%
    \fi
  \fi
  \SOuL@next
}
%    \end{macrocode}
%    \end{macro}
%    \begin{macro}{\SOuL@addfouroctets}
%    \begin{macrocode}
\def\SOuL@addfouroctets#1#2#3#4{%
! \SOUL@addtoken{{\noexpand#1\noexpand#2\noexpand#3\noexpand#4}}%
+ \SOUL@addtoken{{\etex@unexpanded{#1#2#3#4}}}%
}
%    \end{macrocode}
%    \end{macro}
%
% \subsubsection{Redefinition of \cs{SOUL@splittoken}}
%
%    \begin{macro}{\SOUL@splittoken}
%    Macro \cs{SOUL@splittoken} separates the first token
%    or token group from a word and redefines the word to contain
%    the remaining tokens. However if the remaining tokens are
%    a token group, then the curly braces will be removed and
%    the token group is splitted by the next call of \cs{SOUL@splittoken}.
%    The redefinition avoids the removal of curly braces around the
%    remaining tokens.
%    \begin{macrocode}
\SOuL@redefine\SOUL@splittoken#1#2\SOUL@stop{%
  \global\SOUL@token={#1}%
  \global\SOUL@word={#2}%
}#1{%
  \global\SOUL@token={#1}%
  \SOuL@remainingtoken\relax
}
%    \end{macrocode}
%    \end{macro}
%    \begin{macro}{\SOuL@remainingtoken}
%    \begin{macrocode}
\def\SOuL@remainingtoken#1\SOUL@stop{%
  \global\SOUL@word=\expandafter{\@gobble#1}%
}
%    \end{macrocode}
%    \end{macro}
%
% \subsection{Patches}
%
%    The fixed \cs{SOUL@splittoken} allows to remove the double
%    sets of curly braces in other macros of package \xpackage{soul}.
%    The benefit is that implicite kernings are more often detected
%    and fixes a bug in package \xpackage{soul}. The disadvantage is
%    incompatibility. The width of the resulting strings may change.
%    \begin{macro}{\SOUL@flushcomma}
%    \begin{macrocode}
\SOuL@redefine\SOUL@flushcomma{%
  \ifcase\SOUL@comma
  \or
    \edef\x{\SOUL@word={\the\SOUL@word,}}\x
  \or
    \edef\x{\SOUL@word={\the\SOUL@word{{,,}}}}\x
  \fi
  \SOUL@comma\z@
}{%
  \ifcase\SOUL@comma
  \or
    \edef\x{\SOUL@word={\the\SOUL@word,}}\x
  \or
    \edef\x{\SOUL@word={\the\SOUL@word{,,}}}\x
  \fi
  \SOUL@comma\z@
}
%    \end{macrocode}
%    \end{macro}
%    \begin{macro}{\SOUL@flushapo}
%    \begin{macrocode}
\SOuL@redefine\SOUL@flushapo{%
  \ifcase\SOUL@apo
  \or
    \edef\x{\SOUL@word={\the\SOUL@word'}}\x
  \or
    \edef\x{\SOUL@word={\the\SOUL@word{{''}}}}\x
  \fi
  \SOUL@apo\z@
}{%
  \ifcase\SOUL@apo
  \or
    \edef\x{\SOUL@word={\the\SOUL@word'}}\x
  \or
    \edef\x{\SOUL@word={\the\SOUL@word{''}}}\x
  \fi
  \SOUL@apo\z@
}
%    \end{macrocode}
%    \end{macro}
%    \begin{macro}{\SOUL@flushgrave}
%    \begin{macrocode}
\SOuL@redefine\SOUL@flushgrave{%
  \ifcase\SOUL@grave
  \or
    \edef\x{\SOUL@word={\the\SOUL@word`}}\x
  \or
    \edef\x{\SOUL@word={\the\SOUL@word{{``}}}}\x
  \fi
  \SOUL@grave\z@
}{%
  \ifcase\SOUL@grave
  \or
    \edef\x{\SOUL@word={\the\SOUL@word`}}\x
  \or
    \edef\x{\SOUL@word={\the\SOUL@word{``}}}\x
  \fi
  \SOUL@grave\z@
}
%    \end{macrocode}
%    \end{macro}
%    \begin{macro}{\SOUL@addgroup}
%    \begin{macrocode}
\SOuL@redefine\SOUL@addgroup#1#2{%
  {%
    \let\protect\noexpand
    \edef\x{%
      \global\SOUL@word={%
        \the\SOUL@word
        {{\noexpand#1#2}}%
      }%
    }%
    \x
  }%
  \SOUL@scan
}#1#2{%
  \begingroup
    \let\protect\noexpand
    \edef\x{\endgroup
      \SOUL@word={%
        \the\SOUL@word
!       {\noexpand#1{#2}}%
+       {\etex@unexpanded{#1{#2}}}%
      }%
    }%
  \x
  \SOUL@scan
}
%    \end{macrocode}
%    \end{macro}
%    \begin{macro}{\SOUL@addmath}
%    \begin{macrocode}
\SOuL@redefine\SOUL@addmath$#1${%
  {%
    \let\protect\noexpand
    \edef\x{%
      \global\SOUL@word={%
        \the\SOUL@word
        {{\hbox{$#1$}}}%
      }%
    }%
    \x
  }%
  \SOUL@scan
}$#1${%
  \begingroup
    \let\protect\noexpand
    \edef\x{\endgroup
      \SOUL@word={%
        \the\SOUL@word
!       {\hbox{$#1$}}%
+       {\etex@unexpanded{\hbox{$#1$}}}%
      }%
    }%
  \x
  \SOUL@scan
}
%    \end{macrocode}
%    \end{macro}
%    \begin{macro}{\SOUL@addprotect}
%    \begin{macrocode}
\SOuL@redefine\SOUL@addprotect#1#2{%
  {%
    \let\protect\noexpand
    \edef\x{%
      \global\SOUL@word={%
        \the\SOUL@word
        {{\hbox{#2}}}%
      }%
    }%
    \x
  }%
  \SOUL@scan
}#1#2{%
  \begingroup
    \let\protect\noexpand
    \edef\x{\endgroup
      \SOUL@word={%
        \the\SOUL@word
!       {\hbox{#2}}%
+       {\etex@unexpanded{\hbox{#2}}}%
      }%
    }%
  \x
  \SOUL@scan
}
%    \end{macrocode}
%    \end{macro}
%
%    \begin{macro}{\SOUL@addtoken}
%    \begin{macrocode}
+ \SOuL@redefine\SOUL@addtoken#1{%
+   \edef\x{%
+     \SOUL@word={%
+       \the\SOUL@word
+       \noexpand#1%
+     }%
+   }%
+   \x
+   \SOUL@scan
+ }#1{%
+   \edef\x{%
+     \SOUL@word={%
+       \the\SOUL@word
+       \etex@unexpanded{#1}%
+     }%
+   }%
+   \x
+   \SOUL@scan
+ }%
%    \end{macrocode}
%    \end{macro}
%
%    \begin{macrocode}
\SOuL@AtEnd%
%    \end{macrocode}
%
%    \begin{macrocode}
%</package>
%    \end{macrocode}
%
% \section{Test}
%
% \subsection{Catcode checks for loading}
%
%    \begin{macrocode}
%<*test1>
%    \end{macrocode}
%    \begin{macrocode}
\catcode`\{=1 %
\catcode`\}=2 %
\catcode`\#=6 %
\catcode`\@=11 %
\expandafter\ifx\csname count@\endcsname\relax
  \countdef\count@=255 %
\fi
\expandafter\ifx\csname @gobble\endcsname\relax
  \long\def\@gobble#1{}%
\fi
\expandafter\ifx\csname @firstofone\endcsname\relax
  \long\def\@firstofone#1{#1}%
\fi
\expandafter\ifx\csname loop\endcsname\relax
  \expandafter\@firstofone
\else
  \expandafter\@gobble
\fi
{%
  \def\loop#1\repeat{%
    \def\body{#1}%
    \iterate
  }%
  \def\iterate{%
    \body
      \let\next\iterate
    \else
      \let\next\relax
    \fi
    \next
  }%
  \let\repeat=\fi
}%
\def\RestoreCatcodes{}
\count@=0 %
\loop
  \edef\RestoreCatcodes{%
    \RestoreCatcodes
    \catcode\the\count@=\the\catcode\count@\relax
  }%
\ifnum\count@<255 %
  \advance\count@ 1 %
\repeat

\def\RangeCatcodeInvalid#1#2{%
  \count@=#1\relax
  \loop
    \catcode\count@=15 %
  \ifnum\count@<#2\relax
    \advance\count@ 1 %
  \repeat
}
\def\RangeCatcodeCheck#1#2#3{%
  \count@=#1\relax
  \loop
    \ifnum#3=\catcode\count@
    \else
      \errmessage{%
        Character \the\count@\space
        with wrong catcode \the\catcode\count@\space
        instead of \number#3%
      }%
    \fi
  \ifnum\count@<#2\relax
    \advance\count@ 1 %
  \repeat
}
\def\space{ }
\expandafter\ifx\csname LoadCommand\endcsname\relax
  \def\LoadCommand{\input soulutf8.sty\relax}%
\fi
\def\Test{%
  \RangeCatcodeInvalid{0}{47}%
  \RangeCatcodeInvalid{58}{64}%
  \RangeCatcodeInvalid{91}{96}%
  \RangeCatcodeInvalid{123}{255}%
  \catcode`\@=12 %
  \catcode`\\=0 %
  \catcode`\%=14 %
  \LoadCommand
  \RangeCatcodeCheck{0}{36}{15}%
  \RangeCatcodeCheck{37}{37}{14}%
  \RangeCatcodeCheck{38}{47}{15}%
  \RangeCatcodeCheck{48}{57}{12}%
  \RangeCatcodeCheck{58}{63}{15}%
  \RangeCatcodeCheck{64}{64}{12}%
  \RangeCatcodeCheck{65}{90}{11}%
  \RangeCatcodeCheck{91}{91}{15}%
  \RangeCatcodeCheck{92}{92}{0}%
  \RangeCatcodeCheck{93}{96}{15}%
  \RangeCatcodeCheck{97}{122}{11}%
  \RangeCatcodeCheck{123}{255}{15}%
  \RestoreCatcodes
}
\Test
\csname @@end\endcsname
\end
%    \end{macrocode}
%    \begin{macrocode}
%</test1>
%    \end{macrocode}
%
% \subsection{\LaTeX\space tests}
%
%    \begin{macrocode}
%<*test2|test3|test4|test5>
%    \end{macrocode}
%    \begin{macrocode}
%<noetex>\let\SavedUnexpanded\unexpanded
%<noetex>\let\unexpanded\UNDEFINED
\NeedsTeXFormat{LaTeX2e}
\documentclass{article}
\usepackage[T1]{fontenc}
\usepackage{lmodern}
\usepackage{textcomp}
\usepackage[ngerman]{babel}
\expandafter\def\csname ver@hypdestopt.sty\endcsname{}
\usepackage[unicode]{hyperref}
\usepackage{bookmark}
\usepackage{color}
\usepackage{soul}[2003/11/17]
\makeatletter
\newcommand*{\@ifDefined}[1]{%
  \@ifundefined{#1}{%
    \@latex@error{%
      \expandafter\noexpand\csname#1\endcsname undefined%
    }\@ehc
  }%
}
\newcommand*{\SaveOrg}[1]{%
  \@ifdefinable{ORG@SOUL@#1}{%
    \@ifDefined{SOUL@#1}{%
      \expandafter\let\csname ORG@SOUL@#1\expandafter\endcsname
                      \csname SOUL@#1\endcsname
    }%
  }%
}
\newcommand*{\RestoreOrg}[1]{%
  \@ifDefined{ORG@SOUL@#1}{%
    \@ifDefined{SOUL@#1}{%
      \expandafter\let\csname SOUL@#1\expandafter\endcsname
                      \csname ORG@SOUL@#1\endcsname
    }%
  }%
}
\makeatother
\SaveOrg{eval}
\SaveOrg{splittoken}
\SaveOrg{flushcomma}
\SaveOrg{flushapo}
\SaveOrg{grave}
\SaveOrg{addgroup}
\SaveOrg{addmath}
\SaveOrg{addprotect}
\SaveOrg{addtoken}
\usepackage{soulutf8}[2016/05/16]
%    \end{macrocode}
%    \begin{macrocode}
%<noetex>\let\unexpanded\SavedUnexpanded
%    \end{macrocode}
%    Input encoding \xfile{utf8}.
%    \begin{macrocode}
%<*utf8>
\usepackage[utf8]{inputenc}
\DeclareUnicodeCharacter{00A0}{\nobreakspace}
%</utf8>
%    \end{macrocode}
%    Input encoding \xfile{utf8x}. Package \xpackage{ucs}
%    has a nasty bug, that it redefines \cs{show}.
%    \begin{macrocode}
%<*utf8x>
\let\origshow\show
\usepackage[utf8x]{inputenc}
\AtBeginDocument{%
  \let\show\origshow
}
%</utf8x>
%    \end{macrocode}
%
% \subsubsection{Driver definitions}
%
%    \begin{macrocode}
\makeatletter
%    \end{macrocode}
%    \begin{macro}{\an}
%    The definition for \cs{an} is taken from the documentation of
%    package \xpackage{soul}.
%    \begin{macrocode}
\DeclareRobustCommand*{\an}{%
  \def\SOUL@preamble{$^{^P}$}%
  \def\SOUL@everyspace##1{##1\texttt{\char`\ }\linebreak[2]}%
  \def\SOUL@postamble{$^{^E}$}%
  \def\SOUL@everyhyphen{$^{^-}$\linebreak[1]}%
  \def\SOUL@everyexhyphen##1{##1$^{^=}$}%
  \def\SOUL@everysyllable{$^{^S}$}%
  \def\SOUL@everytoken{\the\SOUL@token$^{^T}$}%
  \def\SOUL@everylowerthan{$^{^L}$}%
  \SOUL@
}
%    \end{macrocode}
%    \end{macro}
%
%    \begin{macro}{\sy}
%    The definition for \cs{sy} is taken from the documentation of
%    package \xpackage{soul}.
%    \begin{macrocode}
\DeclareRobustCommand*{\sy}{%
  \SOUL@setup
  \def\SOUL@preamble{%
    \lefthyphenmin=\z@
    \righthyphenmin=\z@
  }%
  \def\SOUL@everyspace##1{##1\space}%
  \def\SOUL@everysyllable{\the\SOUL@syllable}%
  \def\SOUL@everyhyphen{%
    \discretionary{%
      \SOUL@setkern\SOUL@hyphkern
      \SOUL@sethyphenchar
    }{}{%
      \hbox{\kern1pt$\cdot$}%
    }%
  }%
  \def\SOUL@everyexhyphen##1{%
    \SOUL@setkern\SOUL@hyphkern
    \hbox{##1}%
    \discretionary{}{}{%
      \SOUL@setkern\SOUL@charkern
    }%
  }%
  \SOUL@
}
%    \end{macrocode}
%    \end{macro}
%
%    \begin{macro}{\soulH}
%    \begin{macrocode}
\DeclareRobustCommand*{\soulH}{%
  \def\SOUL@preamble{%
    \gdef\SOULH{}%
    \soulHadd{[}%
  }%
  \def\SOUL@everyspace##1{%
    \soulHadd{Sp}%
  }%
  \def\SOUL@postamble{%
    \soulHadd{]}%
  }%
  \def\SOUL@everyhyphen{%
    \soulHadd{Hy}%
  }%
  \def\SOUL@everyexhyphen##1{%
    \soulHadd{Exhy}%
  }%
  \def\SOUL@everysyllable{%
    \soulHadd{Sy}%
  }%
  \def\SOUL@everytoken{%
    \soulHadd{T}%
  }%
  \def\SOUL@everylowerthan{%
    \soulHadd{<}%
  }%
  \SOUL@
}
\newcommand*{\soulHadd}[1]{%
  #1%
  \expandafter\gdef\expandafter\soulHstring\expandafter{%
    \soulHstring
    #1%
  }%
}
%    \end{macrocode}
%    \end{macro}
%
% \subsubsection{Test macros}
%
%    \begin{macrocode}
\usepackage{qstest}
\IncludeTests{*}
\LogTests{log}{*}{*}
%    \end{macrocode}
%    \begin{macrocode}
\newsavebox\boxA
\newsavebox\boxB
\newsavebox\boxC
%    \end{macrocode}
%    \begin{macro}{\TestDriver}
%    \begin{macrocode}
\newcommand*{\TestDriver}[4]{%
  \subsection{Driver #1}%
  \@PackageInfo{soulutf8-test}{* Driver #1}%
  \expandafter\let\expandafter\testdriver\csname #1\endcsname
  \def\teststringA{#3}%
  \def\teststringB{#4}%
  \begin{qstest}{string-#2-driver-#1}{string,#2,driver,#1}%
    \@PackageInfo{soulutf8-test}{* Version: soulutf8}%
    \global\let\soulHstring\@empty
    \setbox0=\hbox{\expandafter\testdriver\expandafter{\teststringA}}%
    \let\HstringA\soulHstring
    \SetTestVBox{\boxA}{\teststringA}%
    %
    \@PackageInfo{soulutf8-test}{* Version: soul + patch}%
    \global\let\soulHstring\@empty
    \RestoreOrg{eval}%
    \setbox2=\hbox{\expandafter\testdriver\expandafter{\teststringB}}%
    \let\HstringB\soulHstring
    \SetTestVBox{\boxB}{\teststringB}%
    %
    \@PackageInfo{soulutf8-test}{* Version: original soul}%
    \global\let\soulHstring\@empty
    \RestoreOrg{splittoken}%
    \RestoreOrg{flushcomma}%
    \RestoreOrg{flushapo}%
    \RestoreOrg{grave}%
    \RestoreOrg{addgroup}%
    \RestoreOrg{addmath}%
    \RestoreOrg{addprotect}%
    \RestoreOrg{addtoken}%
    \setbox4=\hbox{\expandafter\testdriver\expandafter{\teststringB}}%
    \let\HstringC\soulHstring
    \SetTestVBox{\boxC}{\teststringB}%
    %
    \TestInfo{0}{soulutf8}%
    \TestInfo{2}{soul + patch}%
    \TestInfo{4}{original soul}%
    \TestInfoVBox
    \showboxdepth=\maxdimen
    \showboxbreadth=\maxdimen
    \tracingonline=\@ne
    \Expect*{\HstringA}*{\HstringB}%
    \Expect*{\HstringA}*{\HstringC}%
    \Expect*{\the\wd0}*{\the\wd2}%
    \ifdim\wd0=\wd2 %
    \else
      \showbox0 %
      \showbox2 %
    \fi
    \Expect*{\the\ht0}*{\the\ht2}%
    \Expect*{\the\dp0}*{\the\dp2}%
    \Expect*{\the\wd\boxB}*{\the\wd\boxA}%
    \Expect*{\the\wd\boxC}*{\the\wd\boxA}%
    \Expect*{\the\ht\boxB}*{\the\ht\boxA}%
    \ifdim\ht\boxB=\ht\boxA
    \else
      \showbox\boxA
      \showbox\boxB
    \fi
    \Expect*{\the\ht\boxC}*{\the\ht\boxA}%
    \Expect*{\the\dp\boxB}*{\the\dp\boxA}%
    \Expect*{\the\dp\boxC}*{\the\dp\boxA}%
  \end{qstest}%
}
%    \end{macrocode}
%    \end{macro}
%    \begin{macro}{\SetTestVBox}
%    \begin{macrocode}
\newcommand*{\SetTestVBox}[2]{%
  \setbox#1=\vbox{%
    \hsize=0pt\relax
    \hfuzz=100mm\relax
    \expandafter\testdriver\expandafter{#2}%
  }%
}
%    \end{macrocode}
%    \end{macro}
%    \begin{macro}{\TestInfo}
%    \begin{macrocode}
\newcommand*{\TestInfo}[2]{%
  \par
  \noindent
  \copy#1\relax
  \hfill
  \begingroup
    \small
    \ifdim\wd0=\wd2\relax
      \ifdim\wd0=\wd4\relax
      \else
        \ifnum#1=2 %
          \raisebox{.5\baselineskip}[0pt][0pt]{%
            \llap{$\}$ }%
            \rlap{width = \the\wd#1\relax\space$\{$}%
            \phantom{width = 999.99999pt}%
            \qquad
          }%
        \else
          \ifnum#1=4 %
            \rlap{width = \the\wd#1}%
            \phantom{width = 999.99999pt}%
            \qquad
          \fi
        \fi
      \fi
    \else
      \begingroup
        \bfseries
        \rlap{width = \the\wd#1}%
        \phantom{width = 999.99999pt}%
      \endgroup
      \qquad
    \fi
    \rlap{(#2)}%
    \phantom{(original soul)}%
  \endgroup
  \par
}
%    \end{macrocode}
%    \end{macro}
%    \begin{macro}{\TestInfoVBox}
%    \begin{macrocode}
\newcommand*{\TestInfoVBox}{%
  \par
  \noindent
  \mbox{\vrule\relax\space\copy\boxA}%
  \ifdim\ht\boxA=\ht\boxB
  \else
    \hfill
    \begingroup
      \small
      \rlap{(soulutf8)}%
      \phantom{(soulutf8)}%
    \endgroup
    \par
    \noindent
    \mbox{\vrule\relax\space\copy\boxB}%
    \hfill
    \begingroup
      \small
      \rlap{(soul)}%
      \phantom{(soulutf8)}%
    \endgroup
  \fi
  \par
}
%    \end{macrocode}
%    \end{macro}
%    \begin{macro}{\Test}
%    \begin{macrocode}
\makeatletter
\newcommand*{\Test}[3]{%
  \clearpage
  \begingroup
    \def\linebreak{\textbackslash linebreak }%
    \def\\{\textbackslash\textbackslash}%
    \def\soulomit##1{%
      \textbackslash soulomit\textbraceleft##1\textbraceright
    }%
    \def\mbox##1{%
      \textbackslash mbox\textbraceleft##1\textbraceright
    }%
    \def\abc{\textbackslash abc }%
    \def\texttt##1{%
      \textbackslash texttt\textbraceleft##1\textbraceright
    }%
    \def\c##1{%
      \textbackslash c\textbraceleft##1\textbraceright
    }%
%<*utf8x>
    \count@=128 %
    \loop
    \ifnum\count@<256 %
      \lccode`\~=\count@
      \lowercase{%
        \def~{\noexpand~}%
      }%
      \advance\count@\@ne
    \repeat
%</utf8x>
    \csname protected@edef\endcsname\x{\endgroup
      \noexpand\section{String #1: #2}%
    }%
  \x
  \TestDriver{so}{#1}{#2}{#3}%
  \TestDriver{caps}{#1}{#2}{#3}%
  \TestDriver{ul}{#1}{#2}{#3}%
  \TestDriver{st}{#1}{#2}{#3}%
  \TestDriver{hl}{#1}{#2}{#3}%
  \TestDriver{an}{#1}{#2}{#3}%
  \TestDriver{sy}{#1}{#2}{#3}%
  \TestDriver{soulH}{#1}{#2}{#3}%
}
\makeatother
%    \end{macrocode}
%    \end{macro}
%
% \subsubsection{Testing}
%
%    \begin{macrocode}
\begin{document}
  \Test A{}{}%
  \Test B{Hello World}{Hello World}%
  \Test C{%
    Umlauts: ^^c3^^84^^c3^^96^^c3^^9c^^c3^^a4^^c3^^b6^^c3^^bc^^c3^^9f%
  }{%
    Umlauts: \"A\"O\"U\"a\"o\"u\ss
  }%
  \Test D{Euro: ^^e2^^82^^ac}{Euro: \texteuro}%
  \Test E{Stra^^c3^^9fe}{Stra\ss e}%
  \Test F{R^^c3^^b6mer}{R\"omer}%
  \Test G{A^^e2^^80^^93B^^e2^^80^^94C}{A--B---C}%
  \Test H{X^^c2^^a0Y}{X~Y}%
  \Test I{A^^c5^^a4}{A\v T}%
  \Test J{A,B,,C,,,D,,,,E}{A,B,,C,,,D,,,,E}%
  \Test K{A'B''C'''D''''E}{A'B''C'''D''''E}%
  \Test L{A`B``C```D````E}{A`B``C```D````E}%
  \Test M{A$x$B}{A$x$B}%
  \Test N{A\mbox{abc}B}{A\mbox{abc}B}%
  \Test O{,,TA''}{,,TA''}%
  \Test P{{!`}Hola{?`}}{{!`}Hola{?`}}%
  \Test Q{re-sent\slash x}{re-sent\slash x}%
  \Test R{new\\line}{new\\line}%
  \Test S{foo\linebreak bar}{foo\linebreak bar}%
  \Test T{Virchow{sche}}{Virchow{sche}}%
  \Test U{A\soulomit{foo}B}{A\soulomit{foo}B}%
  \Test V{A\texttt{xyz}B}{A\texttt{xyz}B}%
  \Test W{ffi}{ffi}%
  \Test X{f{}f{}i}{f{}f{}i}%
  \Test Y{f\mbox{}f\mbox{}i}{f\mbox{}f\mbox{}i}%
  \Test Z{\c{\"a}}{{\c{\"a}}}%
%<*test2|test3>
  \let\abcx\UNDEFINED
  \def\abc{%
    \begingroup
      \def\abcx{x}%
      \abcx
    \endgroup
  }%
  \Test{AA}{[{x\abc x}]}{[{xxx}]}%
%</test2|test3>
%    \end{macrocode}
%
%    \begin{macrocode}
\end{document}
%    \end{macrocode}
%    \begin{macrocode}
%</test2|test3|test4|test5>
%    \end{macrocode}
%
% \section{Installation}
%
% \subsection{Download}
%
% \paragraph{Package.} This package is available on
% CTAN\footnote{\url{http://ctan.org/pkg/soulutf8}}:
% \begin{description}
% \item[\CTAN{macros/latex/contrib/oberdiek/soulutf8.dtx}] The source file.
% \item[\CTAN{macros/latex/contrib/oberdiek/soulutf8.pdf}] Documentation.
% \end{description}
%
%
% \paragraph{Bundle.} All the packages of the bundle `oberdiek'
% are also available in a TDS compliant ZIP archive. There
% the packages are already unpacked and the documentation files
% are generated. The files and directories obey the TDS standard.
% \begin{description}
% \item[\CTAN{install/macros/latex/contrib/oberdiek.tds.zip}]
% \end{description}
% \emph{TDS} refers to the standard ``A Directory Structure
% for \TeX\ Files'' (\CTAN{tds/tds.pdf}). Directories
% with \xfile{texmf} in their name are usually organized this way.
%
% \subsection{Bundle installation}
%
% \paragraph{Unpacking.} Unpack the \xfile{oberdiek.tds.zip} in the
% TDS tree (also known as \xfile{texmf} tree) of your choice.
% Example (linux):
% \begin{quote}
%   |unzip oberdiek.tds.zip -d ~/texmf|
% \end{quote}
%
% \paragraph{Script installation.}
% Check the directory \xfile{TDS:scripts/oberdiek/} for
% scripts that need further installation steps.
% Package \xpackage{attachfile2} comes with the Perl script
% \xfile{pdfatfi.pl} that should be installed in such a way
% that it can be called as \texttt{pdfatfi}.
% Example (linux):
% \begin{quote}
%   |chmod +x scripts/oberdiek/pdfatfi.pl|\\
%   |cp scripts/oberdiek/pdfatfi.pl /usr/local/bin/|
% \end{quote}
%
% \subsection{Package installation}
%
% \paragraph{Unpacking.} The \xfile{.dtx} file is a self-extracting
% \docstrip\ archive. The files are extracted by running the
% \xfile{.dtx} through \plainTeX:
% \begin{quote}
%   \verb|tex soulutf8.dtx|
% \end{quote}
%
% \paragraph{TDS.} Now the different files must be moved into
% the different directories in your installation TDS tree
% (also known as \xfile{texmf} tree):
% \begin{quote}
% \def\t{^^A
% \begin{tabular}{@{}>{\ttfamily}l@{ $\rightarrow$ }>{\ttfamily}l@{}}
%   soulutf8.sty & tex/generic/oberdiek/soulutf8.sty\\
%   soulutf8.pdf & doc/latex/oberdiek/soulutf8.pdf\\
%   test/soulutf8-test1.tex & doc/latex/oberdiek/test/soulutf8-test1.tex\\
%   test/soulutf8-test2.tex & doc/latex/oberdiek/test/soulutf8-test2.tex\\
%   test/soulutf8-test3.tex & doc/latex/oberdiek/test/soulutf8-test3.tex\\
%   test/soulutf8-test4.tex & doc/latex/oberdiek/test/soulutf8-test4.tex\\
%   test/soulutf8-test5.tex & doc/latex/oberdiek/test/soulutf8-test5.tex\\
%   soulutf8.dtx & source/latex/oberdiek/soulutf8.dtx\\
% \end{tabular}^^A
% }^^A
% \sbox0{\t}^^A
% \ifdim\wd0>\linewidth
%   \begingroup
%     \advance\linewidth by\leftmargin
%     \advance\linewidth by\rightmargin
%   \edef\x{\endgroup
%     \def\noexpand\lw{\the\linewidth}^^A
%   }\x
%   \def\lwbox{^^A
%     \leavevmode
%     \hbox to \linewidth{^^A
%       \kern-\leftmargin\relax
%       \hss
%       \usebox0
%       \hss
%       \kern-\rightmargin\relax
%     }^^A
%   }^^A
%   \ifdim\wd0>\lw
%     \sbox0{\small\t}^^A
%     \ifdim\wd0>\linewidth
%       \ifdim\wd0>\lw
%         \sbox0{\footnotesize\t}^^A
%         \ifdim\wd0>\linewidth
%           \ifdim\wd0>\lw
%             \sbox0{\scriptsize\t}^^A
%             \ifdim\wd0>\linewidth
%               \ifdim\wd0>\lw
%                 \sbox0{\tiny\t}^^A
%                 \ifdim\wd0>\linewidth
%                   \lwbox
%                 \else
%                   \usebox0
%                 \fi
%               \else
%                 \lwbox
%               \fi
%             \else
%               \usebox0
%             \fi
%           \else
%             \lwbox
%           \fi
%         \else
%           \usebox0
%         \fi
%       \else
%         \lwbox
%       \fi
%     \else
%       \usebox0
%     \fi
%   \else
%     \lwbox
%   \fi
% \else
%   \usebox0
% \fi
% \end{quote}
% If you have a \xfile{docstrip.cfg} that configures and enables \docstrip's
% TDS installing feature, then some files can already be in the right
% place, see the documentation of \docstrip.
%
% \subsection{Refresh file name databases}
%
% If your \TeX~distribution
% (\teTeX, \mikTeX, \dots) relies on file name databases, you must refresh
% these. For example, \teTeX\ users run \verb|texhash| or
% \verb|mktexlsr|.
%
% \subsection{Some details for the interested}
%
% \paragraph{Attached source.}
%
% The PDF documentation on CTAN also includes the
% \xfile{.dtx} source file. It can be extracted by
% AcrobatReader 6 or higher. Another option is \textsf{pdftk},
% e.g. unpack the file into the current directory:
% \begin{quote}
%   \verb|pdftk soulutf8.pdf unpack_files output .|
% \end{quote}
%
% \paragraph{Unpacking with \LaTeX.}
% The \xfile{.dtx} chooses its action depending on the format:
% \begin{description}
% \item[\plainTeX:] Run \docstrip\ and extract the files.
% \item[\LaTeX:] Generate the documentation.
% \end{description}
% If you insist on using \LaTeX\ for \docstrip\ (really,
% \docstrip\ does not need \LaTeX), then inform the autodetect routine
% about your intention:
% \begin{quote}
%   \verb|latex \let\install=y% \iffalse meta-comment
%
% File: soulutf8.dtx
% Version: 2016/05/16 v1.1
% Info: Permit use of UTF-8 characters in soul
%
% Copyright (C) 2007 by
%    Heiko Oberdiek <heiko.oberdiek at googlemail.com>
%    2016
%    https://github.com/ho-tex/oberdiek/issues
%
% This work may be distributed and/or modified under the
% conditions of the LaTeX Project Public License, either
% version 1.3c of this license or (at your option) any later
% version. This version of this license is in
%    http://www.latex-project.org/lppl/lppl-1-3c.txt
% and the latest version of this license is in
%    http://www.latex-project.org/lppl.txt
% and version 1.3 or later is part of all distributions of
% LaTeX version 2005/12/01 or later.
%
% This work has the LPPL maintenance status "maintained".
%
% This Current Maintainer of this work is Heiko Oberdiek.
%
% The Base Interpreter refers to any `TeX-Format',
% because some files are installed in TDS:tex/generic//.
%
% This work consists of the main source file soulutf8.dtx
% and the derived files
%    soulutf8.sty, soulutf8.pdf, soulutf8.ins, soulutf8.drv,
%    soulutf8-test1.tex, soulutf8-test2.tex, soulutf8-test3.tex,
%    soulutf8-test4.tex, soulutf8-test5.tex.
%
% Distribution:
%    CTAN:macros/latex/contrib/oberdiek/soulutf8.dtx
%    CTAN:macros/latex/contrib/oberdiek/soulutf8.pdf
%
% Unpacking:
%    (a) If soulutf8.ins is present:
%           tex soulutf8.ins
%    (b) Without soulutf8.ins:
%           tex soulutf8.dtx
%    (c) If you insist on using LaTeX
%           latex \let\install=y\input{soulutf8.dtx}
%        (quote the arguments according to the demands of your shell)
%
% Documentation:
%    (a) If soulutf8.drv is present:
%           latex soulutf8.drv
%    (b) Without soulutf8.drv:
%           latex soulutf8.dtx; ...
%    The class ltxdoc loads the configuration file ltxdoc.cfg
%    if available. Here you can specify further options, e.g.
%    use A4 as paper format:
%       \PassOptionsToClass{a4paper}{article}
%
%    Programm calls to get the documentation (example):
%       pdflatex soulutf8.dtx
%       makeindex -s gind.ist soulutf8.idx
%       pdflatex soulutf8.dtx
%       makeindex -s gind.ist soulutf8.idx
%       pdflatex soulutf8.dtx
%
% Installation:
%    TDS:tex/generic/oberdiek/soulutf8.sty
%    TDS:doc/latex/oberdiek/soulutf8.pdf
%    TDS:doc/latex/oberdiek/test/soulutf8-test1.tex
%    TDS:doc/latex/oberdiek/test/soulutf8-test2.tex
%    TDS:doc/latex/oberdiek/test/soulutf8-test3.tex
%    TDS:doc/latex/oberdiek/test/soulutf8-test4.tex
%    TDS:doc/latex/oberdiek/test/soulutf8-test5.tex
%    TDS:source/latex/oberdiek/soulutf8.dtx
%
%<*ignore>
\begingroup
  \catcode123=1 %
  \catcode125=2 %
  \def\x{LaTeX2e}%
\expandafter\endgroup
\ifcase 0\ifx\install y1\fi\expandafter
         \ifx\csname processbatchFile\endcsname\relax\else1\fi
         \ifx\fmtname\x\else 1\fi\relax
\else\csname fi\endcsname
%</ignore>
%<*install>
\input docstrip.tex
\Msg{************************************************************************}
\Msg{* Installation}
\Msg{* Package: soulutf8 2016/05/16 v1.1 Permit use of UTF-8 characters in soul (HO)}
\Msg{************************************************************************}

\keepsilent
\askforoverwritefalse

\let\MetaPrefix\relax
\preamble

This is a generated file.

Project: soulutf8
Version: 2016/05/16 v1.1

Copyright (C) 2007 by
   Heiko Oberdiek <heiko.oberdiek at googlemail.com>

This work may be distributed and/or modified under the
conditions of the LaTeX Project Public License, either
version 1.3c of this license or (at your option) any later
version. This version of this license is in
   http://www.latex-project.org/lppl/lppl-1-3c.txt
and the latest version of this license is in
   http://www.latex-project.org/lppl.txt
and version 1.3 or later is part of all distributions of
LaTeX version 2005/12/01 or later.

This work has the LPPL maintenance status "maintained".

This Current Maintainer of this work is Heiko Oberdiek.

The Base Interpreter refers to any `TeX-Format',
because some files are installed in TDS:tex/generic//.

This work consists of the main source file soulutf8.dtx
and the derived files
   soulutf8.sty, soulutf8.pdf, soulutf8.ins, soulutf8.drv,
   soulutf8-test1.tex, soulutf8-test2.tex, soulutf8-test3.tex,
   soulutf8-test4.tex, soulutf8-test5.tex.

\endpreamble
\let\MetaPrefix\DoubleperCent

\generate{%
  \file{soulutf8.ins}{\from{soulutf8.dtx}{install}}%
  \file{soulutf8.drv}{\from{soulutf8.dtx}{driver}}%
  \usedir{tex/generic/oberdiek}%
  \file{soulutf8.sty}{\from{soulutf8.dtx}{package}}%
%  \usedir{doc/latex/oberdiek/test}%
%  \file{soulutf8-test1.tex}{\from{soulutf8.dtx}{test1}}%
%  \file{soulutf8-test2.tex}{\from{soulutf8.dtx}{test2,utf8}}%
%  \file{soulutf8-test3.tex}{\from{soulutf8.dtx}{test3,utf8x}}%
%  \file{soulutf8-test4.tex}{\from{soulutf8.dtx}{test4,utf8,noetex}}%
%  \file{soulutf8-test5.tex}{\from{soulutf8.dtx}{test5,utf8x,noetex}}%
  \nopreamble
  \nopostamble
%  \usedir{source/latex/oberdiek/catalogue}%
%  \file{soulutf8.xml}{\from{soulutf8.dtx}{catalogue}}%
}

\catcode32=13\relax% active space
\let =\space%
\Msg{************************************************************************}
\Msg{*}
\Msg{* To finish the installation you have to move the following}
\Msg{* file into a directory searched by TeX:}
\Msg{*}
\Msg{*     soulutf8.sty}
\Msg{*}
\Msg{* To produce the documentation run the file `soulutf8.drv'}
\Msg{* through LaTeX.}
\Msg{*}
\Msg{* Happy TeXing!}
\Msg{*}
\Msg{************************************************************************}

\endbatchfile
%</install>
%<*ignore>
\fi
%</ignore>
%<*driver>
\NeedsTeXFormat{LaTeX2e}
\ProvidesFile{soulutf8.drv}%
  [2016/05/16 v1.1 Permit use of UTF-8 characters in soul (HO)]%
\documentclass{ltxdoc}
\usepackage{holtxdoc}[2011/11/22]
\begin{document}
  \DocInput{soulutf8.dtx}%
\end{document}
%</driver>
% \fi
%
%
% \CharacterTable
%  {Upper-case    \A\B\C\D\E\F\G\H\I\J\K\L\M\N\O\P\Q\R\S\T\U\V\W\X\Y\Z
%   Lower-case    \a\b\c\d\e\f\g\h\i\j\k\l\m\n\o\p\q\r\s\t\u\v\w\x\y\z
%   Digits        \0\1\2\3\4\5\6\7\8\9
%   Exclamation   \!     Double quote  \"     Hash (number) \#
%   Dollar        \$     Percent       \%     Ampersand     \&
%   Acute accent  \'     Left paren    \(     Right paren   \)
%   Asterisk      \*     Plus          \+     Comma         \,
%   Minus         \-     Point         \.     Solidus       \/
%   Colon         \:     Semicolon     \;     Less than     \<
%   Equals        \=     Greater than  \>     Question mark \?
%   Commercial at \@     Left bracket  \[     Backslash     \\
%   Right bracket \]     Circumflex    \^     Underscore    \_
%   Grave accent  \`     Left brace    \{     Vertical bar  \|
%   Right brace   \}     Tilde         \~}
%
% \GetFileInfo{soulutf8.drv}
%
% \title{The \xpackage{soulutf8} package}
% \date{2016/05/16 v1.1}
% \author{Heiko Oberdiek\thanks
% {Please report any issues at https://github.com/ho-tex/oberdiek/issues}\\
% \xemail{heiko.oberdiek at googlemail.com}}
%
% \maketitle
%
% \begin{abstract}
% This package extends package \xpackage{soul} and adds some support
% for UTF-8. Namely the input encodings \xfile{utf8.def}
% from package \xpackage{inputenc} and
% package \xpackage{ucs}'s \xfile{utf8x.def} are supported.
% \end{abstract}
%
% \tableofcontents
%
% \section{Documentation}
%
% This package \xpackage{soulutf8} does not have own options and does not
% define new user commands.
% Any option is passed to package \xpackage{soul} \cite{soul}
% that is loaded first.
% Then some internal macros of \xpackage{soul} are redefined to add
% support for UTF-8.
% The following input encodings are supported:
% \begin{quote}
%   \begin{tabular}{@{}lll@{}}
%     \texttt{utf8} & \LaTeX\ base &
%        \xfile{TDS:tex/latex/base/utf8.def} \cite{utf8}\\
%     \texttt{utf8x} & Package \xpackage{ucs} &
%       \xfile{TDS:tex/latex/ucs/utf8x.def} \cite{ucs}\\
%   \end{tabular}
% \end{quote}
% UTF-8 byte sequences are added as token group to a word, even
% if these UTF-8 characters are some kind of hyphen or space.
% As exception the following three Unicode characters are handled specially:
% \begin{quote}
%   \begin{tabular}{lll}
%     Slot & Name & Action\\
%     \hline
%     \texttt{U+00A0} & NO-BREAK SPACE & like |~|\\
%     \texttt{U+2013} & EN DASH & |--|\\
%     \texttt{U+2014} & EM DASH & |---|
%   \end{tabular}
%
% \subsection{Patch}
%
% Also package \xpackage{soulutf8} tries to patch package \xpackage{soul}
% to improve its behaviour:
% \begin{itemize}
% \item
%   A problem with additional levels of curly braces is fixed.
%   As advantage more implicite kernings are detected.
%   However, the result may be incompatible with the
%   original behaviour of package \xpackage{soul} because
%   of these respected implicite kernings.
% \item
%   \eTeX\ , especially \cs{unexpanded} is supported.
%   This allows a better protection of token groups
%   (|\mbox{|\dots|}|, math, \dots).
% \end{itemize}
% \end{quote}
%
% \subsection{Future}
%
% Currently package \xpackage{soul} does not seem to be
% maintained. Nevertheless if there will be a new version that
% adds support for UTF-8, then this package may become
% obsolete.
%
% \StopEventually{
% }
%
% \section{Implementation}
%
%    \begin{macrocode}
%<*package>
%    \end{macrocode}
%
% \subsection{Reload check and package identification}
%    Reload check, especially if the package is not used with \LaTeX.
%    \begin{macrocode}
\begingroup\catcode61\catcode48\catcode32=10\relax%
  \catcode13=5 % ^^M
  \endlinechar=13 %
  \catcode35=6 % #
  \catcode39=12 % '
  \catcode44=12 % ,
  \catcode45=12 % -
  \catcode46=12 % .
  \catcode58=12 % :
  \catcode64=11 % @
  \catcode123=1 % {
  \catcode125=2 % }
  \expandafter\let\expandafter\x\csname ver@soulutf8.sty\endcsname
  \ifx\x\relax % plain-TeX, first loading
  \else
    \def\empty{}%
    \ifx\x\empty % LaTeX, first loading,
      % variable is initialized, but \ProvidesPackage not yet seen
    \else
      \expandafter\ifx\csname PackageInfo\endcsname\relax
        \def\x#1#2{%
          \immediate\write-1{Package #1 Info: #2.}%
        }%
      \else
        \def\x#1#2{\PackageInfo{#1}{#2, stopped}}%
      \fi
      \x{soulutf8}{The package is already loaded}%
      \aftergroup\endinput
    \fi
  \fi
\endgroup%
%    \end{macrocode}
%    Package identification:
%    \begin{macrocode}
\begingroup\catcode61\catcode48\catcode32=10\relax%
  \catcode13=5 % ^^M
  \endlinechar=13 %
  \catcode35=6 % #
  \catcode39=12 % '
  \catcode40=12 % (
  \catcode41=12 % )
  \catcode44=12 % ,
  \catcode45=12 % -
  \catcode46=12 % .
  \catcode47=12 % /
  \catcode58=12 % :
  \catcode64=11 % @
  \catcode91=12 % [
  \catcode93=12 % ]
  \catcode123=1 % {
  \catcode125=2 % }
  \expandafter\ifx\csname ProvidesPackage\endcsname\relax
    \def\x#1#2#3[#4]{\endgroup
      \immediate\write-1{Package: #3 #4}%
      \xdef#1{#4}%
    }%
  \else
    \def\x#1#2[#3]{\endgroup
      #2[{#3}]%
      \ifx#1\@undefined
        \xdef#1{#3}%
      \fi
      \ifx#1\relax
        \xdef#1{#3}%
      \fi
    }%
  \fi
\expandafter\x\csname ver@soulutf8.sty\endcsname
\ProvidesPackage{soulutf8}%
  [2016/05/16 v1.1 Permit use of UTF-8 characters in soul (HO)]%
%    \end{macrocode}
%
% \subsection{Catcodes}
%
%    \begin{macrocode}
\begingroup\catcode61\catcode48\catcode32=10\relax%
  \catcode13=5 % ^^M
  \endlinechar=13 %
  \catcode123=1 % {
  \catcode125=2 % }
  \catcode64=11 % @
  \def\x{\endgroup
    \expandafter\edef\csname SOuL@AtEnd\endcsname{%
      \endlinechar=\the\endlinechar\relax
      \catcode13=\the\catcode13\relax
      \catcode32=\the\catcode32\relax
      \catcode35=\the\catcode35\relax
      \catcode61=\the\catcode61\relax
      \catcode64=\the\catcode64\relax
      \catcode123=\the\catcode123\relax
      \catcode125=\the\catcode125\relax
    }%
  }%
\x\catcode61\catcode48\catcode32=10\relax%
\catcode13=5 % ^^M
\endlinechar=13 %
\catcode35=6 % #
\catcode64=11 % @
\catcode123=1 % {
\catcode125=2 % }
\def\TMP@EnsureCode#1#2{%
  \edef\SOuL@AtEnd{%
    \SOuL@AtEnd
    \catcode#1=\the\catcode#1\relax
  }%
  \catcode#1=#2\relax
}
\TMP@EnsureCode{10}{12}% ^^J
\TMP@EnsureCode{33}{12}% !
\TMP@EnsureCode{34}{12}% "
\TMP@EnsureCode{36}{3}% $
\TMP@EnsureCode{39}{12}% '
\TMP@EnsureCode{40}{12}% (
\TMP@EnsureCode{41}{12}% )
\TMP@EnsureCode{42}{12}% *
\TMP@EnsureCode{43}{12}% +
\TMP@EnsureCode{44}{12}% ,
\TMP@EnsureCode{45}{12}% -
\TMP@EnsureCode{46}{12}% .
\TMP@EnsureCode{47}{12}% /
\TMP@EnsureCode{58}{12}% :
\TMP@EnsureCode{60}{12}% <
\TMP@EnsureCode{62}{12}% >
\TMP@EnsureCode{91}{12}% [
\TMP@EnsureCode{93}{12}% ]
\TMP@EnsureCode{94}{7}% ^
\TMP@EnsureCode{96}{12}% `
\TMP@EnsureCode{126}\active % ~
\TMP@EnsureCode{128}{12}% ^^80
\TMP@EnsureCode{147}{12}% ^^93
\TMP@EnsureCode{148}{12}% ^^94
\TMP@EnsureCode{160}{12}% ^^a0
\TMP@EnsureCode{194}{12}% ^^c2
\TMP@EnsureCode{226}{12}% ^^e2
\edef\SOuL@AtEnd{\SOuL@AtEnd\noexpand\endinput}
%    \end{macrocode}
%
% \subsection{Loading packages}
%
%    Package \xpackage{soul} uses \cs{documentclass} to detect \LaTeX.
%    \begin{macrocode}
\ifx\documentclass\@undefined
%    \end{macrocode}
%
% \subsubsection{\plainTeX}
%
%    First we check, whether package \xpackage{soul} is already loaded.
%    \begin{macrocode}
  \expandafter\ifx\csname SOUL@\endcsname\relax
%    \end{macrocode}
%    In case of \plainTeX\ package \xpackage{soul} defines some
%    macros in a simple manner that will break the definitions of
%    \xfile{miniltx.tex}, for example. Therefore these macros
%    are first saved and restored afterwards.
%    \begin{macrocode}
    \let\SOuL@orgDeclareRobustCommand\DeclareRobustCommand
    \let\SOuL@orgnewcommand          \newcommand
    \let\SOuL@orgDeclareOption       \DeclareOption
    \let\SOuL@orgPackageError        \PackageError
    \def\SOuL@restorelatexcmds{%
      \let\DeclareRobustCommand\SOuL@orgDeclareRobustCommand
      \let\newcommand          \SOuL@orgnewcommand
      \let\DeclareOption       \SOuL@orgDeclareOption
      \let\PackageError        \SOuL@orgPackageError
    }%
    \input soul.sty\relax
    \SOuL@restorelatexcmds
  \fi
%    \end{macrocode}
%    \begin{macro}{\SOUL@error}
%     Package \xpackage{soul}'s use of \cs{PackageError} is replaced
%     by \cs{@PackageError} of package \xpackage{infwarerr}.
%    \begin{macrocode}
  \input infwarerr.sty\relax
  \let\SOuL@orgSOUL@error\SOUL@error
  \def\SOUL@error{%
    \begingroup
      \let\PackageError\@PackageError
      \SOuL@orgSOUL@error
    \endgroup
  }%
%    \end{macrocode}
%    \end{macro}
%    \begin{macrocode}
  \input etexcmds.sty\relax
%    \end{macrocode}
%    \begin{macro}{\@onelevel@sanitize}
%    Define \LaTeX's \cs{@onelevel@sanitize} if not already available.
%    \begin{macrocode}
  \expandafter\ifx\csname @onelevel@sanitize\endcsname\relax
    \def\@onelevel@sanitize#1{%
      \edef#1{%
        \expandafter\strip@prefix\meaning#1%
      }%
    }%
%    \end{macrocode}
%    \end{macro}
%    \begin{macro}{\strip@prefix}
%    \begin{macrocode}
    \def\strip@prefix#1>{}%
%    \end{macrocode}
%    \end{macro}
%    \begin{macrocode}
  \fi
\else
%    \end{macrocode}
%
% \subsubsection{\LaTeX}
%    \begin{macrocode}
  \DeclareOption*{\PassOptionsToPackage{\CurrentOption}{soul}}%
  \ProcessOptions\relax
  \RequirePackage{soul}[2003/11/17]%
  \RequirePackage{infwarerr}[2016/05/16]%
  \RequirePackage{etexcmds}[2016/05/16]%
\fi
%    \end{macrocode}
%
% \subsubsection{\eTeX}
%
%    In \plainTeX\ command \cs{+} is an \emph{outer} macro.
%    Therefore numbers are used to avoid problems.
%    \begin{macrocode}
\ifetex@unexpanded
  \catcode33=14 % '!': comment
  \catcode43=9  % '+': ignore
\else
  \catcode33=9  % '!': ignore
  \catcode43=14 % '+': comment
\fi
%    \end{macrocode}
%
% \subsection{Macro for redefinitions}
%
%    \begin{macro}{\SOuL@redefine}
%    \begin{macrocode}
\def\SOuL@redefine#1{%
  \begingroup
    \def\SOuL@cmd{#1}%
    \afterassignment\SOuL@cmdcheck
    \def\SOuL@temp
}
%    \end{macrocode}
%    \end{macro}
%    \begin{macro}{\SOuL@cmdcheck}
%    \begin{macrocode}
\def\SOuL@cmdcheck{%
    \expandafter\ifx\SOuL@cmd\SOuL@temp
    \else
      \edef\SOuL@temp*{\expandafter\string\SOuL@cmd}%
      \@PackageWarningNoLine{soulutf8}{%
        Command \SOuL@temp* has changed.\MessageBreak
        Supported versions of package `soul': 2003/11/17.\MessageBreak
        Depending on the unknown changes the redefinition\MessageBreak
        of \SOuL@temp* may not behave correctly%
      }%
    \fi
  \expandafter\endgroup
  \expandafter\def\SOuL@cmd
}
%    \end{macrocode}
%    \end{macro}
%
% \subsection{Redefinition of \cs{SOUL@eval}}
%
%    \begin{macro}{\SOUL@eval}
%    Macro \cs{SOUL@eval} is redefined to add detection of
%    the first byte of a UTF-8 sequence. Because \cs{SOUL@eval}
%    is overwritten, a warning is issued, if the contents of
%    \cs{SOUL@eval} is not as expected.
%    \begin{macrocode}
\SOuL@redefine\SOUL@eval{%
%    \end{macrocode}
%    First the expected definition.
%    \begin{macrocode}
  \def\SOUL@n*##1{\SOUL@scan}%
  \if\noexpand\SOUL@@\SOUL@spc
  \else
    \SOUL@ignorespacesfalse
  \fi
  \ifnum\SOUL@minus=\thr@@
    \SOUL@flushminus
  \else\ifnum\SOUL@comma=\tw@
    \SOUL@flushcomma
  \else\ifnum\SOUL@apo=\tw@
    \SOUL@flushapo
  \else\ifnum\SOUL@grave=\tw@
    \SOUL@flushgrave
  \fi\fi\fi\fi
  \ifx\SOUL@@-\else\SOUL@flushminus\fi
  \ifx\SOUL@@,\else\SOUL@flushcomma\fi
  \ifx\SOUL@@'\else\SOUL@flushapo\fi
  \ifx\SOUL@@`\else\SOUL@flushgrave\fi
  \ifx\SOUL@@-%
    \advance\SOUL@minus\@ne
  \else\ifx\SOUL@@,%
    \advance\SOUL@comma\@ne
  \else\ifx\SOUL@@'%
    \advance\SOUL@apo\@ne
  \else\ifx\SOUL@@`%
    \advance\SOUL@grave\@ne
  \else
    \SOUL@flushminus
    \SOUL@flushcomma
    \SOUL@flushapo
    \SOUL@flushgrave
    \ifx\SOUL@@\SOUL@stop
      \def\SOUL@n*{%
        \SOUL@doword
        \SOUL@eventuallyexhyphen\null
      }%
    \else\ifx\SOUL@@\par
      \def\SOUL@n*\par{\par\leavevmode\SOUL@scan}%
    \else\if\noexpand\SOUL@@\SOUL@spc
      \SOUL@doword
      \SOUL@eventuallyexhyphen\null
      \ifSOUL@ignorespaces
      \else
        \SOUL@everyspace{}%
      \fi
      \def\SOUL@n* {\SOUL@scan}%
    \else\ifx\SOUL@@\\%
      \SOUL@doword
      \SOUL@eventuallyexhyphen\null
      \SOUL@everyspace{\unskip\nobreak\hfil\break}%
      \SOUL@ignorespacestrue
    \else\ifx\SOUL@@~%
      \SOUL@doword
      \SOUL@eventuallyexhyphen\null
      \SOUL@everyspace{\nobreak}%
    \else\ifx\SOUL@@\slash
      \SOUL@doword
      \SOUL@eventuallyexhyphen{/}%
      \SOUL@exhyphen{/}%
    \else\ifx\SOUL@@\mbox
      \def\SOUL@n*{\SOUL@addprotect}%
    \else\ifx\SOUL@@\hbox
      \def\SOUL@n*{\SOUL@addprotect}%
    \else\ifx\SOUL@@\soulomit
      \def\SOUL@n*\soulomit##1{%
        \SOUL@doword
        {\spaceskip\SOUL@spaceskip##1}%
        \SOUL@scan
      }%
    \else\ifx\SOUL@@\break
      \SOUL@doword
      \break
    \else\ifx\SOUL@@\linebreak
      \SOUL@doword
      \SOUL@everyspace{\linebreak}%
    \else\ifcat\bgroup\noexpand\SOUL@@
      \def\SOUL@n*{\SOUL@addgroup{}}%
    \else\ifcat$\noexpand\SOUL@@
      \def\SOUL@n*{\SOUL@addmath}%
    \else
      \def\SOUL@n*{\SOUL@dotoken}%
    \fi\fi\fi\fi\fi\fi\fi\fi\fi\fi\fi\fi\fi
  \fi\fi\fi\fi
  \SOUL@n*%
}{%
%    \end{macrocode}
%    Now the redefined version follows.
%    \begin{macrocode}
  \def\SOUL@n*##1{\SOUL@scan}%
  \if\noexpand\SOUL@@\SOUL@spc
  \else
    \SOUL@ignorespacesfalse
  \fi
  \ifnum\SOUL@minus=\thr@@
    \SOUL@flushminus
  \else\ifnum\SOUL@comma=\tw@
    \SOUL@flushcomma
  \else\ifnum\SOUL@apo=\tw@
    \SOUL@flushapo
  \else\ifnum\SOUL@grave=\tw@
    \SOUL@flushgrave
  \fi\fi\fi\fi
  \ifx\SOUL@@-\else\SOUL@flushminus\fi
  \ifx\SOUL@@,\else\SOUL@flushcomma\fi
  \ifx\SOUL@@'\else\SOUL@flushapo\fi
  \ifx\SOUL@@`\else\SOUL@flushgrave\fi
  \ifx\SOUL@@-%
    \advance\SOUL@minus\@ne
  \else\ifx\SOUL@@,%
    \advance\SOUL@comma\@ne
  \else\ifx\SOUL@@'%
    \advance\SOUL@apo\@ne
  \else\ifx\SOUL@@`%
    \advance\SOUL@grave\@ne
  \else
    \SOUL@flushminus
    \SOUL@flushcomma
    \SOUL@flushapo
    \SOUL@flushgrave
    \ifx\SOUL@@\SOUL@stop
      \def\SOUL@n*{%
          \SOUL@doword
          \SOUL@eventuallyexhyphen\null
      }%
    \else\ifx\SOUL@@\par
      \def\SOUL@n*\par{\par\leavevmode\SOUL@scan}%
    \else\if\noexpand\SOUL@@\SOUL@spc
      \SOUL@doword
      \SOUL@eventuallyexhyphen\null
      \ifSOUL@ignorespaces
      \else
        \SOUL@everyspace{}%
      \fi
      \def\SOUL@n* {\SOUL@scan}%
    \else\ifx\SOUL@@\\%
      \SOUL@doword
      \SOUL@eventuallyexhyphen\null
      \SOUL@everyspace{\unskip\nobreak\hfil\break}%
      \SOUL@ignorespacestrue
    \else\ifx\SOUL@@~%
      \SOUL@doword
      \SOUL@eventuallyexhyphen\null
      \SOUL@everyspace{\nobreak}%
    \else\ifx\SOUL@@\slash
      \SOUL@doword
      \SOUL@eventuallyexhyphen{/}%
      \SOUL@exhyphen{/}%
    \else\ifx\SOUL@@\mbox
      \def\SOUL@n*{\SOUL@addprotect}%
    \else\ifx\SOUL@@\hbox
      \def\SOUL@n*{\SOUL@addprotect}%
    \else\ifx\SOUL@@\soulomit
      \def\SOUL@n*\soulomit##1{%
        \SOUL@doword
        {\spaceskip\SOUL@spaceskip##1}%
        \SOUL@scan
      }%
    \else\ifx\SOUL@@\break
      \SOUL@doword
      \break
    \else\ifx\SOUL@@\linebreak
      \SOUL@doword
      \SOUL@everyspace{\linebreak}%
    \else\ifcat\bgroup\noexpand\SOUL@@
      \def\SOUL@n*{\SOUL@addgroup{}}%
    \else\ifcat$\noexpand\SOUL@@
      \def\SOUL@n*{\SOUL@addmath}%
    \else
%    \end{macrocode}
%    The current token is examined to detect the start of
%    a UTF-8 sequence.
%    \begin{macrocode}
      \SOuL@analyzeutfviii
      \ifcase\SOuL@octets
        \SOuL@analyzeutfviiix
      \fi
      \ifcase\SOuL@octets
        \def\SOUL@n*{\SOUL@dotoken}%
      \or % 1
      \or % 2
        \def\SOUL@n*{\SOuL@addtwooctets}%
      \or % 3
        \def\SOUL@n*{\SOuL@addthreeoctets}%
      \or % 4
        \def\SOUL@n*{\SOuL@addfouroctets}%
      \fi
    \fi\fi\fi\fi\fi\fi\fi\fi\fi\fi\fi\fi\fi
  \fi\fi\fi\fi
  \SOUL@n*%
}
%    \end{macrocode}
%    \end{macro}
%
% \subsection{UTF-8 analysis}
%
% \subsubsection{Help strings}
%
%    \begin{macrocode}
\def\SOuL@defsanitizedstring#1#2{%
  \expandafter\def\csname SOuL@string#1\endcsname{#2}%
  \expandafter\@onelevel@sanitize\csname SOuL@string#1\endcsname
}
\SOuL@defsanitizedstring{UTFviii}{UTFviii@}
\SOuL@defsanitizedstring{octets}{@octets}
\SOuL@defsanitizedstring{two}{two}
\SOuL@defsanitizedstring{three}{three}
\SOuL@defsanitizedstring{four}{four}
\SOuL@defsanitizedstring{macrocolon}{macro:}
\SOuL@defsanitizedstring{csnameu}{csname u8-}
\SOuL@defsanitizedstring{undeferr}{utf@viii@undeferr}
\def\SOuL@stringendash{^^e2^^80^^93}
\def\SOuL@stringemdash{^^e2^^80^^94}
\def\SOuL@stringnobreakspace{^^c2^^a0}
\edef\SOuL@charhash{\string #}
\edef\SOuL@chartwo{\string 2}
\edef\SOuL@charthree{\string 3}
\def\SOuL@empty{}
%    \end{macrocode}
%
% \subsubsection{Support for \xfile{utf8.def}}
%
%    \begin{macro}{\SOuL@analyzeutfviii}
%    \begin{macrocode}
\begingroup
  \edef\x{\endgroup
    \def\noexpand\SOuL@analyzeutfviii{%
      \noexpand\expandafter\noexpand\SOuL@checkutfviii
      \noexpand\meaning\noexpand\SOUL@@
      \SOuL@stringUTFviii\SOuL@stringoctets
      \noexpand\@nil
    }%
    \def\noexpand\SOuL@checkutfviii
      ##1\SOuL@stringUTFviii##2\SOuL@stringoctets##3\noexpand\@nil
  }%
\x{%
  \def\SOuL@temp{#2}%
  \chardef\SOuL@octets=%
      \ifx\SOuL@temp\SOuL@stringtwo
        \tw@
      \else\ifx\SOuL@temp\SOuL@stringthree
        \thr@@
      \else\ifx\SOuL@temp\SOuL@stringfour
        4 %
      \else
        \z@
      \fi\fi\fi
}
%    \end{macrocode}
%    \end{macro}
%
% \subsubsection{Support for \xfile{utf8x.def}}
%
%    \begin{macro}{\SOuL@analyzeutfviiix}
%    \begin{macrocode}
\begingroup
  \edef\x{\endgroup
    \def\noexpand\SOuL@analyzeutfviiix{%
      \noexpand\expandafter\noexpand\SOuL@checkutfviiix
      \noexpand\meaning\noexpand\SOUL@@
      \SOuL@stringmacrocolon\SOuL@charhash1{}{}{}{}%
      \SOuL@stringcsnameu\SOuL@stringundeferr
      \noexpand\@nil
    }%
%    \end{macrocode}
%    \end{macro}
%    \begin{macro}{\SOuL@checkutfviiix}
%    \begin{macrocode}
    \def\noexpand\SOuL@checkutfviiix
      ##1\SOuL@stringmacrocolon\SOuL@charhash1##2##3##4##5##6%
      \SOuL@stringcsnameu##7\SOuL@stringundeferr##8\noexpand\@nil
  }%
\x{%
  \def\SOuL@temp{#7}%
  \ifx\SOuL@temp\SOuL@empty
    \chardef\SOuL@octets=\z@
  \else
    \def\SOuL@temp{#5}%
    \ifx\SOuL@temp\SOuL@charthree
      \chardef\SOuL@octets=4 %
    \else
      \def\SOuL@temp{#3}%
      \ifx\SOuL@temp\SOuL@chartwo
        \chardef\SOuL@octets=\thr@@
      \else
        \chardef\SOuL@octets=\tw@
      \fi
    \fi
  \fi
}
%    \end{macrocode}
%    \end{macro}
%
% \subsection{Actions for UTF-8 sequences}
%
%    \begin{macro}{\SOuL@addtwooctets}
%    \begin{macrocode}
\def\SOuL@addtwooctets#1#2{%
  \def\SOuL@temp{#1#2}%
  \@onelevel@sanitize\SOuL@temp
  \ifx\SOuL@temp\SOuL@stringnobreakspace
    \SOUL@doword
    \SOUL@eventuallyexhyphen\null
    \SOUL@everyspace{\nobreak}%
    \let\SOuL@next\SOUL@scan
  \else
    \def\SOuL@next{%
!     \SOUL@addtoken{{\noexpand#1\noexpand#2}}%
+     \SOUL@addtoken{{\etex@unexpanded{#1#2}}}%
    }%
  \fi
  \SOuL@next
}
%    \end{macrocode}
%    \end{macro}
%    \begin{macro}{\SOuL@addthreeoctets}
%    \begin{macrocode}
\def\SOuL@addthreeoctets#1#2#3{%
  \def\SOuL@temp{#1#2#3}%
  \@onelevel@sanitize\SOuL@temp
  \ifx\SOuL@temp\SOuL@stringendash
    \SOUL@doword
    \SOUL@eventuallyexhyphen{-}%
    \SOUL@exhyphen{--}%
    \let\SOuL@next\SOUL@scan
  \else
    \ifx\SOuL@temp\SOuL@stringemdash
      \SOUL@doword
      \SOUL@eventuallyexhyphen{-}%
      \SOUL@exhyphen{---}%
      \let\SOuL@next\SOUL@scan
    \else
      \def\SOuL@next{%
!       \SOUL@addtoken{{\noexpand#1\noexpand#2\noexpand#3}}%
+       \SOUL@addtoken{{\etex@unexpanded{#1#2#3}}}%
      }%
    \fi
  \fi
  \SOuL@next
}
%    \end{macrocode}
%    \end{macro}
%    \begin{macro}{\SOuL@addfouroctets}
%    \begin{macrocode}
\def\SOuL@addfouroctets#1#2#3#4{%
! \SOUL@addtoken{{\noexpand#1\noexpand#2\noexpand#3\noexpand#4}}%
+ \SOUL@addtoken{{\etex@unexpanded{#1#2#3#4}}}%
}
%    \end{macrocode}
%    \end{macro}
%
% \subsubsection{Redefinition of \cs{SOUL@splittoken}}
%
%    \begin{macro}{\SOUL@splittoken}
%    Macro \cs{SOUL@splittoken} separates the first token
%    or token group from a word and redefines the word to contain
%    the remaining tokens. However if the remaining tokens are
%    a token group, then the curly braces will be removed and
%    the token group is splitted by the next call of \cs{SOUL@splittoken}.
%    The redefinition avoids the removal of curly braces around the
%    remaining tokens.
%    \begin{macrocode}
\SOuL@redefine\SOUL@splittoken#1#2\SOUL@stop{%
  \global\SOUL@token={#1}%
  \global\SOUL@word={#2}%
}#1{%
  \global\SOUL@token={#1}%
  \SOuL@remainingtoken\relax
}
%    \end{macrocode}
%    \end{macro}
%    \begin{macro}{\SOuL@remainingtoken}
%    \begin{macrocode}
\def\SOuL@remainingtoken#1\SOUL@stop{%
  \global\SOUL@word=\expandafter{\@gobble#1}%
}
%    \end{macrocode}
%    \end{macro}
%
% \subsection{Patches}
%
%    The fixed \cs{SOUL@splittoken} allows to remove the double
%    sets of curly braces in other macros of package \xpackage{soul}.
%    The benefit is that implicite kernings are more often detected
%    and fixes a bug in package \xpackage{soul}. The disadvantage is
%    incompatibility. The width of the resulting strings may change.
%    \begin{macro}{\SOUL@flushcomma}
%    \begin{macrocode}
\SOuL@redefine\SOUL@flushcomma{%
  \ifcase\SOUL@comma
  \or
    \edef\x{\SOUL@word={\the\SOUL@word,}}\x
  \or
    \edef\x{\SOUL@word={\the\SOUL@word{{,,}}}}\x
  \fi
  \SOUL@comma\z@
}{%
  \ifcase\SOUL@comma
  \or
    \edef\x{\SOUL@word={\the\SOUL@word,}}\x
  \or
    \edef\x{\SOUL@word={\the\SOUL@word{,,}}}\x
  \fi
  \SOUL@comma\z@
}
%    \end{macrocode}
%    \end{macro}
%    \begin{macro}{\SOUL@flushapo}
%    \begin{macrocode}
\SOuL@redefine\SOUL@flushapo{%
  \ifcase\SOUL@apo
  \or
    \edef\x{\SOUL@word={\the\SOUL@word'}}\x
  \or
    \edef\x{\SOUL@word={\the\SOUL@word{{''}}}}\x
  \fi
  \SOUL@apo\z@
}{%
  \ifcase\SOUL@apo
  \or
    \edef\x{\SOUL@word={\the\SOUL@word'}}\x
  \or
    \edef\x{\SOUL@word={\the\SOUL@word{''}}}\x
  \fi
  \SOUL@apo\z@
}
%    \end{macrocode}
%    \end{macro}
%    \begin{macro}{\SOUL@flushgrave}
%    \begin{macrocode}
\SOuL@redefine\SOUL@flushgrave{%
  \ifcase\SOUL@grave
  \or
    \edef\x{\SOUL@word={\the\SOUL@word`}}\x
  \or
    \edef\x{\SOUL@word={\the\SOUL@word{{``}}}}\x
  \fi
  \SOUL@grave\z@
}{%
  \ifcase\SOUL@grave
  \or
    \edef\x{\SOUL@word={\the\SOUL@word`}}\x
  \or
    \edef\x{\SOUL@word={\the\SOUL@word{``}}}\x
  \fi
  \SOUL@grave\z@
}
%    \end{macrocode}
%    \end{macro}
%    \begin{macro}{\SOUL@addgroup}
%    \begin{macrocode}
\SOuL@redefine\SOUL@addgroup#1#2{%
  {%
    \let\protect\noexpand
    \edef\x{%
      \global\SOUL@word={%
        \the\SOUL@word
        {{\noexpand#1#2}}%
      }%
    }%
    \x
  }%
  \SOUL@scan
}#1#2{%
  \begingroup
    \let\protect\noexpand
    \edef\x{\endgroup
      \SOUL@word={%
        \the\SOUL@word
!       {\noexpand#1{#2}}%
+       {\etex@unexpanded{#1{#2}}}%
      }%
    }%
  \x
  \SOUL@scan
}
%    \end{macrocode}
%    \end{macro}
%    \begin{macro}{\SOUL@addmath}
%    \begin{macrocode}
\SOuL@redefine\SOUL@addmath$#1${%
  {%
    \let\protect\noexpand
    \edef\x{%
      \global\SOUL@word={%
        \the\SOUL@word
        {{\hbox{$#1$}}}%
      }%
    }%
    \x
  }%
  \SOUL@scan
}$#1${%
  \begingroup
    \let\protect\noexpand
    \edef\x{\endgroup
      \SOUL@word={%
        \the\SOUL@word
!       {\hbox{$#1$}}%
+       {\etex@unexpanded{\hbox{$#1$}}}%
      }%
    }%
  \x
  \SOUL@scan
}
%    \end{macrocode}
%    \end{macro}
%    \begin{macro}{\SOUL@addprotect}
%    \begin{macrocode}
\SOuL@redefine\SOUL@addprotect#1#2{%
  {%
    \let\protect\noexpand
    \edef\x{%
      \global\SOUL@word={%
        \the\SOUL@word
        {{\hbox{#2}}}%
      }%
    }%
    \x
  }%
  \SOUL@scan
}#1#2{%
  \begingroup
    \let\protect\noexpand
    \edef\x{\endgroup
      \SOUL@word={%
        \the\SOUL@word
!       {\hbox{#2}}%
+       {\etex@unexpanded{\hbox{#2}}}%
      }%
    }%
  \x
  \SOUL@scan
}
%    \end{macrocode}
%    \end{macro}
%
%    \begin{macro}{\SOUL@addtoken}
%    \begin{macrocode}
+ \SOuL@redefine\SOUL@addtoken#1{%
+   \edef\x{%
+     \SOUL@word={%
+       \the\SOUL@word
+       \noexpand#1%
+     }%
+   }%
+   \x
+   \SOUL@scan
+ }#1{%
+   \edef\x{%
+     \SOUL@word={%
+       \the\SOUL@word
+       \etex@unexpanded{#1}%
+     }%
+   }%
+   \x
+   \SOUL@scan
+ }%
%    \end{macrocode}
%    \end{macro}
%
%    \begin{macrocode}
\SOuL@AtEnd%
%    \end{macrocode}
%
%    \begin{macrocode}
%</package>
%    \end{macrocode}
%
% \section{Test}
%
% \subsection{Catcode checks for loading}
%
%    \begin{macrocode}
%<*test1>
%    \end{macrocode}
%    \begin{macrocode}
\catcode`\{=1 %
\catcode`\}=2 %
\catcode`\#=6 %
\catcode`\@=11 %
\expandafter\ifx\csname count@\endcsname\relax
  \countdef\count@=255 %
\fi
\expandafter\ifx\csname @gobble\endcsname\relax
  \long\def\@gobble#1{}%
\fi
\expandafter\ifx\csname @firstofone\endcsname\relax
  \long\def\@firstofone#1{#1}%
\fi
\expandafter\ifx\csname loop\endcsname\relax
  \expandafter\@firstofone
\else
  \expandafter\@gobble
\fi
{%
  \def\loop#1\repeat{%
    \def\body{#1}%
    \iterate
  }%
  \def\iterate{%
    \body
      \let\next\iterate
    \else
      \let\next\relax
    \fi
    \next
  }%
  \let\repeat=\fi
}%
\def\RestoreCatcodes{}
\count@=0 %
\loop
  \edef\RestoreCatcodes{%
    \RestoreCatcodes
    \catcode\the\count@=\the\catcode\count@\relax
  }%
\ifnum\count@<255 %
  \advance\count@ 1 %
\repeat

\def\RangeCatcodeInvalid#1#2{%
  \count@=#1\relax
  \loop
    \catcode\count@=15 %
  \ifnum\count@<#2\relax
    \advance\count@ 1 %
  \repeat
}
\def\RangeCatcodeCheck#1#2#3{%
  \count@=#1\relax
  \loop
    \ifnum#3=\catcode\count@
    \else
      \errmessage{%
        Character \the\count@\space
        with wrong catcode \the\catcode\count@\space
        instead of \number#3%
      }%
    \fi
  \ifnum\count@<#2\relax
    \advance\count@ 1 %
  \repeat
}
\def\space{ }
\expandafter\ifx\csname LoadCommand\endcsname\relax
  \def\LoadCommand{\input soulutf8.sty\relax}%
\fi
\def\Test{%
  \RangeCatcodeInvalid{0}{47}%
  \RangeCatcodeInvalid{58}{64}%
  \RangeCatcodeInvalid{91}{96}%
  \RangeCatcodeInvalid{123}{255}%
  \catcode`\@=12 %
  \catcode`\\=0 %
  \catcode`\%=14 %
  \LoadCommand
  \RangeCatcodeCheck{0}{36}{15}%
  \RangeCatcodeCheck{37}{37}{14}%
  \RangeCatcodeCheck{38}{47}{15}%
  \RangeCatcodeCheck{48}{57}{12}%
  \RangeCatcodeCheck{58}{63}{15}%
  \RangeCatcodeCheck{64}{64}{12}%
  \RangeCatcodeCheck{65}{90}{11}%
  \RangeCatcodeCheck{91}{91}{15}%
  \RangeCatcodeCheck{92}{92}{0}%
  \RangeCatcodeCheck{93}{96}{15}%
  \RangeCatcodeCheck{97}{122}{11}%
  \RangeCatcodeCheck{123}{255}{15}%
  \RestoreCatcodes
}
\Test
\csname @@end\endcsname
\end
%    \end{macrocode}
%    \begin{macrocode}
%</test1>
%    \end{macrocode}
%
% \subsection{\LaTeX\space tests}
%
%    \begin{macrocode}
%<*test2|test3|test4|test5>
%    \end{macrocode}
%    \begin{macrocode}
%<noetex>\let\SavedUnexpanded\unexpanded
%<noetex>\let\unexpanded\UNDEFINED
\NeedsTeXFormat{LaTeX2e}
\documentclass{article}
\usepackage[T1]{fontenc}
\usepackage{lmodern}
\usepackage{textcomp}
\usepackage[ngerman]{babel}
\expandafter\def\csname ver@hypdestopt.sty\endcsname{}
\usepackage[unicode]{hyperref}
\usepackage{bookmark}
\usepackage{color}
\usepackage{soul}[2003/11/17]
\makeatletter
\newcommand*{\@ifDefined}[1]{%
  \@ifundefined{#1}{%
    \@latex@error{%
      \expandafter\noexpand\csname#1\endcsname undefined%
    }\@ehc
  }%
}
\newcommand*{\SaveOrg}[1]{%
  \@ifdefinable{ORG@SOUL@#1}{%
    \@ifDefined{SOUL@#1}{%
      \expandafter\let\csname ORG@SOUL@#1\expandafter\endcsname
                      \csname SOUL@#1\endcsname
    }%
  }%
}
\newcommand*{\RestoreOrg}[1]{%
  \@ifDefined{ORG@SOUL@#1}{%
    \@ifDefined{SOUL@#1}{%
      \expandafter\let\csname SOUL@#1\expandafter\endcsname
                      \csname ORG@SOUL@#1\endcsname
    }%
  }%
}
\makeatother
\SaveOrg{eval}
\SaveOrg{splittoken}
\SaveOrg{flushcomma}
\SaveOrg{flushapo}
\SaveOrg{grave}
\SaveOrg{addgroup}
\SaveOrg{addmath}
\SaveOrg{addprotect}
\SaveOrg{addtoken}
\usepackage{soulutf8}[2016/05/16]
%    \end{macrocode}
%    \begin{macrocode}
%<noetex>\let\unexpanded\SavedUnexpanded
%    \end{macrocode}
%    Input encoding \xfile{utf8}.
%    \begin{macrocode}
%<*utf8>
\usepackage[utf8]{inputenc}
\DeclareUnicodeCharacter{00A0}{\nobreakspace}
%</utf8>
%    \end{macrocode}
%    Input encoding \xfile{utf8x}. Package \xpackage{ucs}
%    has a nasty bug, that it redefines \cs{show}.
%    \begin{macrocode}
%<*utf8x>
\let\origshow\show
\usepackage[utf8x]{inputenc}
\AtBeginDocument{%
  \let\show\origshow
}
%</utf8x>
%    \end{macrocode}
%
% \subsubsection{Driver definitions}
%
%    \begin{macrocode}
\makeatletter
%    \end{macrocode}
%    \begin{macro}{\an}
%    The definition for \cs{an} is taken from the documentation of
%    package \xpackage{soul}.
%    \begin{macrocode}
\DeclareRobustCommand*{\an}{%
  \def\SOUL@preamble{$^{^P}$}%
  \def\SOUL@everyspace##1{##1\texttt{\char`\ }\linebreak[2]}%
  \def\SOUL@postamble{$^{^E}$}%
  \def\SOUL@everyhyphen{$^{^-}$\linebreak[1]}%
  \def\SOUL@everyexhyphen##1{##1$^{^=}$}%
  \def\SOUL@everysyllable{$^{^S}$}%
  \def\SOUL@everytoken{\the\SOUL@token$^{^T}$}%
  \def\SOUL@everylowerthan{$^{^L}$}%
  \SOUL@
}
%    \end{macrocode}
%    \end{macro}
%
%    \begin{macro}{\sy}
%    The definition for \cs{sy} is taken from the documentation of
%    package \xpackage{soul}.
%    \begin{macrocode}
\DeclareRobustCommand*{\sy}{%
  \SOUL@setup
  \def\SOUL@preamble{%
    \lefthyphenmin=\z@
    \righthyphenmin=\z@
  }%
  \def\SOUL@everyspace##1{##1\space}%
  \def\SOUL@everysyllable{\the\SOUL@syllable}%
  \def\SOUL@everyhyphen{%
    \discretionary{%
      \SOUL@setkern\SOUL@hyphkern
      \SOUL@sethyphenchar
    }{}{%
      \hbox{\kern1pt$\cdot$}%
    }%
  }%
  \def\SOUL@everyexhyphen##1{%
    \SOUL@setkern\SOUL@hyphkern
    \hbox{##1}%
    \discretionary{}{}{%
      \SOUL@setkern\SOUL@charkern
    }%
  }%
  \SOUL@
}
%    \end{macrocode}
%    \end{macro}
%
%    \begin{macro}{\soulH}
%    \begin{macrocode}
\DeclareRobustCommand*{\soulH}{%
  \def\SOUL@preamble{%
    \gdef\SOULH{}%
    \soulHadd{[}%
  }%
  \def\SOUL@everyspace##1{%
    \soulHadd{Sp}%
  }%
  \def\SOUL@postamble{%
    \soulHadd{]}%
  }%
  \def\SOUL@everyhyphen{%
    \soulHadd{Hy}%
  }%
  \def\SOUL@everyexhyphen##1{%
    \soulHadd{Exhy}%
  }%
  \def\SOUL@everysyllable{%
    \soulHadd{Sy}%
  }%
  \def\SOUL@everytoken{%
    \soulHadd{T}%
  }%
  \def\SOUL@everylowerthan{%
    \soulHadd{<}%
  }%
  \SOUL@
}
\newcommand*{\soulHadd}[1]{%
  #1%
  \expandafter\gdef\expandafter\soulHstring\expandafter{%
    \soulHstring
    #1%
  }%
}
%    \end{macrocode}
%    \end{macro}
%
% \subsubsection{Test macros}
%
%    \begin{macrocode}
\usepackage{qstest}
\IncludeTests{*}
\LogTests{log}{*}{*}
%    \end{macrocode}
%    \begin{macrocode}
\newsavebox\boxA
\newsavebox\boxB
\newsavebox\boxC
%    \end{macrocode}
%    \begin{macro}{\TestDriver}
%    \begin{macrocode}
\newcommand*{\TestDriver}[4]{%
  \subsection{Driver #1}%
  \@PackageInfo{soulutf8-test}{* Driver #1}%
  \expandafter\let\expandafter\testdriver\csname #1\endcsname
  \def\teststringA{#3}%
  \def\teststringB{#4}%
  \begin{qstest}{string-#2-driver-#1}{string,#2,driver,#1}%
    \@PackageInfo{soulutf8-test}{* Version: soulutf8}%
    \global\let\soulHstring\@empty
    \setbox0=\hbox{\expandafter\testdriver\expandafter{\teststringA}}%
    \let\HstringA\soulHstring
    \SetTestVBox{\boxA}{\teststringA}%
    %
    \@PackageInfo{soulutf8-test}{* Version: soul + patch}%
    \global\let\soulHstring\@empty
    \RestoreOrg{eval}%
    \setbox2=\hbox{\expandafter\testdriver\expandafter{\teststringB}}%
    \let\HstringB\soulHstring
    \SetTestVBox{\boxB}{\teststringB}%
    %
    \@PackageInfo{soulutf8-test}{* Version: original soul}%
    \global\let\soulHstring\@empty
    \RestoreOrg{splittoken}%
    \RestoreOrg{flushcomma}%
    \RestoreOrg{flushapo}%
    \RestoreOrg{grave}%
    \RestoreOrg{addgroup}%
    \RestoreOrg{addmath}%
    \RestoreOrg{addprotect}%
    \RestoreOrg{addtoken}%
    \setbox4=\hbox{\expandafter\testdriver\expandafter{\teststringB}}%
    \let\HstringC\soulHstring
    \SetTestVBox{\boxC}{\teststringB}%
    %
    \TestInfo{0}{soulutf8}%
    \TestInfo{2}{soul + patch}%
    \TestInfo{4}{original soul}%
    \TestInfoVBox
    \showboxdepth=\maxdimen
    \showboxbreadth=\maxdimen
    \tracingonline=\@ne
    \Expect*{\HstringA}*{\HstringB}%
    \Expect*{\HstringA}*{\HstringC}%
    \Expect*{\the\wd0}*{\the\wd2}%
    \ifdim\wd0=\wd2 %
    \else
      \showbox0 %
      \showbox2 %
    \fi
    \Expect*{\the\ht0}*{\the\ht2}%
    \Expect*{\the\dp0}*{\the\dp2}%
    \Expect*{\the\wd\boxB}*{\the\wd\boxA}%
    \Expect*{\the\wd\boxC}*{\the\wd\boxA}%
    \Expect*{\the\ht\boxB}*{\the\ht\boxA}%
    \ifdim\ht\boxB=\ht\boxA
    \else
      \showbox\boxA
      \showbox\boxB
    \fi
    \Expect*{\the\ht\boxC}*{\the\ht\boxA}%
    \Expect*{\the\dp\boxB}*{\the\dp\boxA}%
    \Expect*{\the\dp\boxC}*{\the\dp\boxA}%
  \end{qstest}%
}
%    \end{macrocode}
%    \end{macro}
%    \begin{macro}{\SetTestVBox}
%    \begin{macrocode}
\newcommand*{\SetTestVBox}[2]{%
  \setbox#1=\vbox{%
    \hsize=0pt\relax
    \hfuzz=100mm\relax
    \expandafter\testdriver\expandafter{#2}%
  }%
}
%    \end{macrocode}
%    \end{macro}
%    \begin{macro}{\TestInfo}
%    \begin{macrocode}
\newcommand*{\TestInfo}[2]{%
  \par
  \noindent
  \copy#1\relax
  \hfill
  \begingroup
    \small
    \ifdim\wd0=\wd2\relax
      \ifdim\wd0=\wd4\relax
      \else
        \ifnum#1=2 %
          \raisebox{.5\baselineskip}[0pt][0pt]{%
            \llap{$\}$ }%
            \rlap{width = \the\wd#1\relax\space$\{$}%
            \phantom{width = 999.99999pt}%
            \qquad
          }%
        \else
          \ifnum#1=4 %
            \rlap{width = \the\wd#1}%
            \phantom{width = 999.99999pt}%
            \qquad
          \fi
        \fi
      \fi
    \else
      \begingroup
        \bfseries
        \rlap{width = \the\wd#1}%
        \phantom{width = 999.99999pt}%
      \endgroup
      \qquad
    \fi
    \rlap{(#2)}%
    \phantom{(original soul)}%
  \endgroup
  \par
}
%    \end{macrocode}
%    \end{macro}
%    \begin{macro}{\TestInfoVBox}
%    \begin{macrocode}
\newcommand*{\TestInfoVBox}{%
  \par
  \noindent
  \mbox{\vrule\relax\space\copy\boxA}%
  \ifdim\ht\boxA=\ht\boxB
  \else
    \hfill
    \begingroup
      \small
      \rlap{(soulutf8)}%
      \phantom{(soulutf8)}%
    \endgroup
    \par
    \noindent
    \mbox{\vrule\relax\space\copy\boxB}%
    \hfill
    \begingroup
      \small
      \rlap{(soul)}%
      \phantom{(soulutf8)}%
    \endgroup
  \fi
  \par
}
%    \end{macrocode}
%    \end{macro}
%    \begin{macro}{\Test}
%    \begin{macrocode}
\makeatletter
\newcommand*{\Test}[3]{%
  \clearpage
  \begingroup
    \def\linebreak{\textbackslash linebreak }%
    \def\\{\textbackslash\textbackslash}%
    \def\soulomit##1{%
      \textbackslash soulomit\textbraceleft##1\textbraceright
    }%
    \def\mbox##1{%
      \textbackslash mbox\textbraceleft##1\textbraceright
    }%
    \def\abc{\textbackslash abc }%
    \def\texttt##1{%
      \textbackslash texttt\textbraceleft##1\textbraceright
    }%
    \def\c##1{%
      \textbackslash c\textbraceleft##1\textbraceright
    }%
%<*utf8x>
    \count@=128 %
    \loop
    \ifnum\count@<256 %
      \lccode`\~=\count@
      \lowercase{%
        \def~{\noexpand~}%
      }%
      \advance\count@\@ne
    \repeat
%</utf8x>
    \csname protected@edef\endcsname\x{\endgroup
      \noexpand\section{String #1: #2}%
    }%
  \x
  \TestDriver{so}{#1}{#2}{#3}%
  \TestDriver{caps}{#1}{#2}{#3}%
  \TestDriver{ul}{#1}{#2}{#3}%
  \TestDriver{st}{#1}{#2}{#3}%
  \TestDriver{hl}{#1}{#2}{#3}%
  \TestDriver{an}{#1}{#2}{#3}%
  \TestDriver{sy}{#1}{#2}{#3}%
  \TestDriver{soulH}{#1}{#2}{#3}%
}
\makeatother
%    \end{macrocode}
%    \end{macro}
%
% \subsubsection{Testing}
%
%    \begin{macrocode}
\begin{document}
  \Test A{}{}%
  \Test B{Hello World}{Hello World}%
  \Test C{%
    Umlauts: ^^c3^^84^^c3^^96^^c3^^9c^^c3^^a4^^c3^^b6^^c3^^bc^^c3^^9f%
  }{%
    Umlauts: \"A\"O\"U\"a\"o\"u\ss
  }%
  \Test D{Euro: ^^e2^^82^^ac}{Euro: \texteuro}%
  \Test E{Stra^^c3^^9fe}{Stra\ss e}%
  \Test F{R^^c3^^b6mer}{R\"omer}%
  \Test G{A^^e2^^80^^93B^^e2^^80^^94C}{A--B---C}%
  \Test H{X^^c2^^a0Y}{X~Y}%
  \Test I{A^^c5^^a4}{A\v T}%
  \Test J{A,B,,C,,,D,,,,E}{A,B,,C,,,D,,,,E}%
  \Test K{A'B''C'''D''''E}{A'B''C'''D''''E}%
  \Test L{A`B``C```D````E}{A`B``C```D````E}%
  \Test M{A$x$B}{A$x$B}%
  \Test N{A\mbox{abc}B}{A\mbox{abc}B}%
  \Test O{,,TA''}{,,TA''}%
  \Test P{{!`}Hola{?`}}{{!`}Hola{?`}}%
  \Test Q{re-sent\slash x}{re-sent\slash x}%
  \Test R{new\\line}{new\\line}%
  \Test S{foo\linebreak bar}{foo\linebreak bar}%
  \Test T{Virchow{sche}}{Virchow{sche}}%
  \Test U{A\soulomit{foo}B}{A\soulomit{foo}B}%
  \Test V{A\texttt{xyz}B}{A\texttt{xyz}B}%
  \Test W{ffi}{ffi}%
  \Test X{f{}f{}i}{f{}f{}i}%
  \Test Y{f\mbox{}f\mbox{}i}{f\mbox{}f\mbox{}i}%
  \Test Z{\c{\"a}}{{\c{\"a}}}%
%<*test2|test3>
  \let\abcx\UNDEFINED
  \def\abc{%
    \begingroup
      \def\abcx{x}%
      \abcx
    \endgroup
  }%
  \Test{AA}{[{x\abc x}]}{[{xxx}]}%
%</test2|test3>
%    \end{macrocode}
%
%    \begin{macrocode}
\end{document}
%    \end{macrocode}
%    \begin{macrocode}
%</test2|test3|test4|test5>
%    \end{macrocode}
%
% \section{Installation}
%
% \subsection{Download}
%
% \paragraph{Package.} This package is available on
% CTAN\footnote{\url{http://ctan.org/pkg/soulutf8}}:
% \begin{description}
% \item[\CTAN{macros/latex/contrib/oberdiek/soulutf8.dtx}] The source file.
% \item[\CTAN{macros/latex/contrib/oberdiek/soulutf8.pdf}] Documentation.
% \end{description}
%
%
% \paragraph{Bundle.} All the packages of the bundle `oberdiek'
% are also available in a TDS compliant ZIP archive. There
% the packages are already unpacked and the documentation files
% are generated. The files and directories obey the TDS standard.
% \begin{description}
% \item[\CTAN{install/macros/latex/contrib/oberdiek.tds.zip}]
% \end{description}
% \emph{TDS} refers to the standard ``A Directory Structure
% for \TeX\ Files'' (\CTAN{tds/tds.pdf}). Directories
% with \xfile{texmf} in their name are usually organized this way.
%
% \subsection{Bundle installation}
%
% \paragraph{Unpacking.} Unpack the \xfile{oberdiek.tds.zip} in the
% TDS tree (also known as \xfile{texmf} tree) of your choice.
% Example (linux):
% \begin{quote}
%   |unzip oberdiek.tds.zip -d ~/texmf|
% \end{quote}
%
% \paragraph{Script installation.}
% Check the directory \xfile{TDS:scripts/oberdiek/} for
% scripts that need further installation steps.
% Package \xpackage{attachfile2} comes with the Perl script
% \xfile{pdfatfi.pl} that should be installed in such a way
% that it can be called as \texttt{pdfatfi}.
% Example (linux):
% \begin{quote}
%   |chmod +x scripts/oberdiek/pdfatfi.pl|\\
%   |cp scripts/oberdiek/pdfatfi.pl /usr/local/bin/|
% \end{quote}
%
% \subsection{Package installation}
%
% \paragraph{Unpacking.} The \xfile{.dtx} file is a self-extracting
% \docstrip\ archive. The files are extracted by running the
% \xfile{.dtx} through \plainTeX:
% \begin{quote}
%   \verb|tex soulutf8.dtx|
% \end{quote}
%
% \paragraph{TDS.} Now the different files must be moved into
% the different directories in your installation TDS tree
% (also known as \xfile{texmf} tree):
% \begin{quote}
% \def\t{^^A
% \begin{tabular}{@{}>{\ttfamily}l@{ $\rightarrow$ }>{\ttfamily}l@{}}
%   soulutf8.sty & tex/generic/oberdiek/soulutf8.sty\\
%   soulutf8.pdf & doc/latex/oberdiek/soulutf8.pdf\\
%   test/soulutf8-test1.tex & doc/latex/oberdiek/test/soulutf8-test1.tex\\
%   test/soulutf8-test2.tex & doc/latex/oberdiek/test/soulutf8-test2.tex\\
%   test/soulutf8-test3.tex & doc/latex/oberdiek/test/soulutf8-test3.tex\\
%   test/soulutf8-test4.tex & doc/latex/oberdiek/test/soulutf8-test4.tex\\
%   test/soulutf8-test5.tex & doc/latex/oberdiek/test/soulutf8-test5.tex\\
%   soulutf8.dtx & source/latex/oberdiek/soulutf8.dtx\\
% \end{tabular}^^A
% }^^A
% \sbox0{\t}^^A
% \ifdim\wd0>\linewidth
%   \begingroup
%     \advance\linewidth by\leftmargin
%     \advance\linewidth by\rightmargin
%   \edef\x{\endgroup
%     \def\noexpand\lw{\the\linewidth}^^A
%   }\x
%   \def\lwbox{^^A
%     \leavevmode
%     \hbox to \linewidth{^^A
%       \kern-\leftmargin\relax
%       \hss
%       \usebox0
%       \hss
%       \kern-\rightmargin\relax
%     }^^A
%   }^^A
%   \ifdim\wd0>\lw
%     \sbox0{\small\t}^^A
%     \ifdim\wd0>\linewidth
%       \ifdim\wd0>\lw
%         \sbox0{\footnotesize\t}^^A
%         \ifdim\wd0>\linewidth
%           \ifdim\wd0>\lw
%             \sbox0{\scriptsize\t}^^A
%             \ifdim\wd0>\linewidth
%               \ifdim\wd0>\lw
%                 \sbox0{\tiny\t}^^A
%                 \ifdim\wd0>\linewidth
%                   \lwbox
%                 \else
%                   \usebox0
%                 \fi
%               \else
%                 \lwbox
%               \fi
%             \else
%               \usebox0
%             \fi
%           \else
%             \lwbox
%           \fi
%         \else
%           \usebox0
%         \fi
%       \else
%         \lwbox
%       \fi
%     \else
%       \usebox0
%     \fi
%   \else
%     \lwbox
%   \fi
% \else
%   \usebox0
% \fi
% \end{quote}
% If you have a \xfile{docstrip.cfg} that configures and enables \docstrip's
% TDS installing feature, then some files can already be in the right
% place, see the documentation of \docstrip.
%
% \subsection{Refresh file name databases}
%
% If your \TeX~distribution
% (\teTeX, \mikTeX, \dots) relies on file name databases, you must refresh
% these. For example, \teTeX\ users run \verb|texhash| or
% \verb|mktexlsr|.
%
% \subsection{Some details for the interested}
%
% \paragraph{Attached source.}
%
% The PDF documentation on CTAN also includes the
% \xfile{.dtx} source file. It can be extracted by
% AcrobatReader 6 or higher. Another option is \textsf{pdftk},
% e.g. unpack the file into the current directory:
% \begin{quote}
%   \verb|pdftk soulutf8.pdf unpack_files output .|
% \end{quote}
%
% \paragraph{Unpacking with \LaTeX.}
% The \xfile{.dtx} chooses its action depending on the format:
% \begin{description}
% \item[\plainTeX:] Run \docstrip\ and extract the files.
% \item[\LaTeX:] Generate the documentation.
% \end{description}
% If you insist on using \LaTeX\ for \docstrip\ (really,
% \docstrip\ does not need \LaTeX), then inform the autodetect routine
% about your intention:
% \begin{quote}
%   \verb|latex \let\install=y\input{soulutf8.dtx}|
% \end{quote}
% Do not forget to quote the argument according to the demands
% of your shell.
%
% \paragraph{Generating the documentation.}
% You can use both the \xfile{.dtx} or the \xfile{.drv} to generate
% the documentation. The process can be configured by the
% configuration file \xfile{ltxdoc.cfg}. For instance, put this
% line into this file, if you want to have A4 as paper format:
% \begin{quote}
%   \verb|\PassOptionsToClass{a4paper}{article}|
% \end{quote}
% An example follows how to generate the
% documentation with pdf\LaTeX:
% \begin{quote}
%\begin{verbatim}
%pdflatex soulutf8.dtx
%makeindex -s gind.ist soulutf8.idx
%pdflatex soulutf8.dtx
%makeindex -s gind.ist soulutf8.idx
%pdflatex soulutf8.dtx
%\end{verbatim}
% \end{quote}
%
% \begin{thebibliography}{9}
%
% \bibitem{soul}
%   Melchior Franz: \textit{The \xpackage{soul} package};
%   2003/11/17;\\
%   \CTAN{macros/latex/contrib/soul/}.
%
% \bibitem{ucs}
%   Dominique P. G. Unruh: \textit{\xpackage{ucs.sty} -- Unicode Support};
%   2004/10/17;\\
%   \CTAN{macros/latex/contrib/unicode/}.
%
% \bibitem{utf8}
%   Frank Mittelbach, Chris Rowley:
%   \textit{Providing some UTF-8 support via \xpackage{inputenc}};
%   2006/03/30;\\
%   \CTAN{macros/latex/base/utf8ienc.dtx}.
%
% \end{thebibliography}
%
% \begin{History}
%   \begin{Version}{2007/09/09 v1.0}
%   \item
%     First version.
%   \end{Version}
%   \begin{Version}{2016/05/16 v1.1}
%   \item
%     Documentation updates.
%   \end{Version}
% \end{History}
%
% \PrintIndex
%
% \Finale
\endinput
|
% \end{quote}
% Do not forget to quote the argument according to the demands
% of your shell.
%
% \paragraph{Generating the documentation.}
% You can use both the \xfile{.dtx} or the \xfile{.drv} to generate
% the documentation. The process can be configured by the
% configuration file \xfile{ltxdoc.cfg}. For instance, put this
% line into this file, if you want to have A4 as paper format:
% \begin{quote}
%   \verb|\PassOptionsToClass{a4paper}{article}|
% \end{quote}
% An example follows how to generate the
% documentation with pdf\LaTeX:
% \begin{quote}
%\begin{verbatim}
%pdflatex soulutf8.dtx
%makeindex -s gind.ist soulutf8.idx
%pdflatex soulutf8.dtx
%makeindex -s gind.ist soulutf8.idx
%pdflatex soulutf8.dtx
%\end{verbatim}
% \end{quote}
%
% \begin{thebibliography}{9}
%
% \bibitem{soul}
%   Melchior Franz: \textit{The \xpackage{soul} package};
%   2003/11/17;\\
%   \CTAN{macros/latex/contrib/soul/}.
%
% \bibitem{ucs}
%   Dominique P. G. Unruh: \textit{\xpackage{ucs.sty} -- Unicode Support};
%   2004/10/17;\\
%   \CTAN{macros/latex/contrib/unicode/}.
%
% \bibitem{utf8}
%   Frank Mittelbach, Chris Rowley:
%   \textit{Providing some UTF-8 support via \xpackage{inputenc}};
%   2006/03/30;\\
%   \CTAN{macros/latex/base/utf8ienc.dtx}.
%
% \end{thebibliography}
%
% \begin{History}
%   \begin{Version}{2007/09/09 v1.0}
%   \item
%     First version.
%   \end{Version}
%   \begin{Version}{2016/05/16 v1.1}
%   \item
%     Documentation updates.
%   \end{Version}
% \end{History}
%
% \PrintIndex
%
% \Finale
\endinput

%        (quote the arguments according to the demands of your shell)
%
% Documentation:
%    (a) If soulutf8.drv is present:
%           latex soulutf8.drv
%    (b) Without soulutf8.drv:
%           latex soulutf8.dtx; ...
%    The class ltxdoc loads the configuration file ltxdoc.cfg
%    if available. Here you can specify further options, e.g.
%    use A4 as paper format:
%       \PassOptionsToClass{a4paper}{article}
%
%    Programm calls to get the documentation (example):
%       pdflatex soulutf8.dtx
%       makeindex -s gind.ist soulutf8.idx
%       pdflatex soulutf8.dtx
%       makeindex -s gind.ist soulutf8.idx
%       pdflatex soulutf8.dtx
%
% Installation:
%    TDS:tex/generic/oberdiek/soulutf8.sty
%    TDS:doc/latex/oberdiek/soulutf8.pdf
%    TDS:doc/latex/oberdiek/test/soulutf8-test1.tex
%    TDS:doc/latex/oberdiek/test/soulutf8-test2.tex
%    TDS:doc/latex/oberdiek/test/soulutf8-test3.tex
%    TDS:doc/latex/oberdiek/test/soulutf8-test4.tex
%    TDS:doc/latex/oberdiek/test/soulutf8-test5.tex
%    TDS:source/latex/oberdiek/soulutf8.dtx
%
%<*ignore>
\begingroup
  \catcode123=1 %
  \catcode125=2 %
  \def\x{LaTeX2e}%
\expandafter\endgroup
\ifcase 0\ifx\install y1\fi\expandafter
         \ifx\csname processbatchFile\endcsname\relax\else1\fi
         \ifx\fmtname\x\else 1\fi\relax
\else\csname fi\endcsname
%</ignore>
%<*install>
\input docstrip.tex
\Msg{************************************************************************}
\Msg{* Installation}
\Msg{* Package: soulutf8 2016/05/16 v1.1 Permit use of UTF-8 characters in soul (HO)}
\Msg{************************************************************************}

\keepsilent
\askforoverwritefalse

\let\MetaPrefix\relax
\preamble

This is a generated file.

Project: soulutf8
Version: 2016/05/16 v1.1

Copyright (C) 2007 by
   Heiko Oberdiek <heiko.oberdiek at googlemail.com>

This work may be distributed and/or modified under the
conditions of the LaTeX Project Public License, either
version 1.3c of this license or (at your option) any later
version. This version of this license is in
   http://www.latex-project.org/lppl/lppl-1-3c.txt
and the latest version of this license is in
   http://www.latex-project.org/lppl.txt
and version 1.3 or later is part of all distributions of
LaTeX version 2005/12/01 or later.

This work has the LPPL maintenance status "maintained".

This Current Maintainer of this work is Heiko Oberdiek.

The Base Interpreter refers to any `TeX-Format',
because some files are installed in TDS:tex/generic//.

This work consists of the main source file soulutf8.dtx
and the derived files
   soulutf8.sty, soulutf8.pdf, soulutf8.ins, soulutf8.drv,
   soulutf8-test1.tex, soulutf8-test2.tex, soulutf8-test3.tex,
   soulutf8-test4.tex, soulutf8-test5.tex.

\endpreamble
\let\MetaPrefix\DoubleperCent

\generate{%
  \file{soulutf8.ins}{\from{soulutf8.dtx}{install}}%
  \file{soulutf8.drv}{\from{soulutf8.dtx}{driver}}%
  \usedir{tex/generic/oberdiek}%
  \file{soulutf8.sty}{\from{soulutf8.dtx}{package}}%
%  \usedir{doc/latex/oberdiek/test}%
%  \file{soulutf8-test1.tex}{\from{soulutf8.dtx}{test1}}%
%  \file{soulutf8-test2.tex}{\from{soulutf8.dtx}{test2,utf8}}%
%  \file{soulutf8-test3.tex}{\from{soulutf8.dtx}{test3,utf8x}}%
%  \file{soulutf8-test4.tex}{\from{soulutf8.dtx}{test4,utf8,noetex}}%
%  \file{soulutf8-test5.tex}{\from{soulutf8.dtx}{test5,utf8x,noetex}}%
  \nopreamble
  \nopostamble
%  \usedir{source/latex/oberdiek/catalogue}%
%  \file{soulutf8.xml}{\from{soulutf8.dtx}{catalogue}}%
}

\catcode32=13\relax% active space
\let =\space%
\Msg{************************************************************************}
\Msg{*}
\Msg{* To finish the installation you have to move the following}
\Msg{* file into a directory searched by TeX:}
\Msg{*}
\Msg{*     soulutf8.sty}
\Msg{*}
\Msg{* To produce the documentation run the file `soulutf8.drv'}
\Msg{* through LaTeX.}
\Msg{*}
\Msg{* Happy TeXing!}
\Msg{*}
\Msg{************************************************************************}

\endbatchfile
%</install>
%<*ignore>
\fi
%</ignore>
%<*driver>
\NeedsTeXFormat{LaTeX2e}
\ProvidesFile{soulutf8.drv}%
  [2016/05/16 v1.1 Permit use of UTF-8 characters in soul (HO)]%
\documentclass{ltxdoc}
\usepackage{holtxdoc}[2011/11/22]
\begin{document}
  \DocInput{soulutf8.dtx}%
\end{document}
%</driver>
% \fi
%
%
% \CharacterTable
%  {Upper-case    \A\B\C\D\E\F\G\H\I\J\K\L\M\N\O\P\Q\R\S\T\U\V\W\X\Y\Z
%   Lower-case    \a\b\c\d\e\f\g\h\i\j\k\l\m\n\o\p\q\r\s\t\u\v\w\x\y\z
%   Digits        \0\1\2\3\4\5\6\7\8\9
%   Exclamation   \!     Double quote  \"     Hash (number) \#
%   Dollar        \$     Percent       \%     Ampersand     \&
%   Acute accent  \'     Left paren    \(     Right paren   \)
%   Asterisk      \*     Plus          \+     Comma         \,
%   Minus         \-     Point         \.     Solidus       \/
%   Colon         \:     Semicolon     \;     Less than     \<
%   Equals        \=     Greater than  \>     Question mark \?
%   Commercial at \@     Left bracket  \[     Backslash     \\
%   Right bracket \]     Circumflex    \^     Underscore    \_
%   Grave accent  \`     Left brace    \{     Vertical bar  \|
%   Right brace   \}     Tilde         \~}
%
% \GetFileInfo{soulutf8.drv}
%
% \title{The \xpackage{soulutf8} package}
% \date{2016/05/16 v1.1}
% \author{Heiko Oberdiek\thanks
% {Please report any issues at https://github.com/ho-tex/oberdiek/issues}\\
% \xemail{heiko.oberdiek at googlemail.com}}
%
% \maketitle
%
% \begin{abstract}
% This package extends package \xpackage{soul} and adds some support
% for UTF-8. Namely the input encodings \xfile{utf8.def}
% from package \xpackage{inputenc} and
% package \xpackage{ucs}'s \xfile{utf8x.def} are supported.
% \end{abstract}
%
% \tableofcontents
%
% \section{Documentation}
%
% This package \xpackage{soulutf8} does not have own options and does not
% define new user commands.
% Any option is passed to package \xpackage{soul} \cite{soul}
% that is loaded first.
% Then some internal macros of \xpackage{soul} are redefined to add
% support for UTF-8.
% The following input encodings are supported:
% \begin{quote}
%   \begin{tabular}{@{}lll@{}}
%     \texttt{utf8} & \LaTeX\ base &
%        \xfile{TDS:tex/latex/base/utf8.def} \cite{utf8}\\
%     \texttt{utf8x} & Package \xpackage{ucs} &
%       \xfile{TDS:tex/latex/ucs/utf8x.def} \cite{ucs}\\
%   \end{tabular}
% \end{quote}
% UTF-8 byte sequences are added as token group to a word, even
% if these UTF-8 characters are some kind of hyphen or space.
% As exception the following three Unicode characters are handled specially:
% \begin{quote}
%   \begin{tabular}{lll}
%     Slot & Name & Action\\
%     \hline
%     \texttt{U+00A0} & NO-BREAK SPACE & like |~|\\
%     \texttt{U+2013} & EN DASH & |--|\\
%     \texttt{U+2014} & EM DASH & |---|
%   \end{tabular}
%
% \subsection{Patch}
%
% Also package \xpackage{soulutf8} tries to patch package \xpackage{soul}
% to improve its behaviour:
% \begin{itemize}
% \item
%   A problem with additional levels of curly braces is fixed.
%   As advantage more implicite kernings are detected.
%   However, the result may be incompatible with the
%   original behaviour of package \xpackage{soul} because
%   of these respected implicite kernings.
% \item
%   \eTeX\ , especially \cs{unexpanded} is supported.
%   This allows a better protection of token groups
%   (|\mbox{|\dots|}|, math, \dots).
% \end{itemize}
% \end{quote}
%
% \subsection{Future}
%
% Currently package \xpackage{soul} does not seem to be
% maintained. Nevertheless if there will be a new version that
% adds support for UTF-8, then this package may become
% obsolete.
%
% \StopEventually{
% }
%
% \section{Implementation}
%
%    \begin{macrocode}
%<*package>
%    \end{macrocode}
%
% \subsection{Reload check and package identification}
%    Reload check, especially if the package is not used with \LaTeX.
%    \begin{macrocode}
\begingroup\catcode61\catcode48\catcode32=10\relax%
  \catcode13=5 % ^^M
  \endlinechar=13 %
  \catcode35=6 % #
  \catcode39=12 % '
  \catcode44=12 % ,
  \catcode45=12 % -
  \catcode46=12 % .
  \catcode58=12 % :
  \catcode64=11 % @
  \catcode123=1 % {
  \catcode125=2 % }
  \expandafter\let\expandafter\x\csname ver@soulutf8.sty\endcsname
  \ifx\x\relax % plain-TeX, first loading
  \else
    \def\empty{}%
    \ifx\x\empty % LaTeX, first loading,
      % variable is initialized, but \ProvidesPackage not yet seen
    \else
      \expandafter\ifx\csname PackageInfo\endcsname\relax
        \def\x#1#2{%
          \immediate\write-1{Package #1 Info: #2.}%
        }%
      \else
        \def\x#1#2{\PackageInfo{#1}{#2, stopped}}%
      \fi
      \x{soulutf8}{The package is already loaded}%
      \aftergroup\endinput
    \fi
  \fi
\endgroup%
%    \end{macrocode}
%    Package identification:
%    \begin{macrocode}
\begingroup\catcode61\catcode48\catcode32=10\relax%
  \catcode13=5 % ^^M
  \endlinechar=13 %
  \catcode35=6 % #
  \catcode39=12 % '
  \catcode40=12 % (
  \catcode41=12 % )
  \catcode44=12 % ,
  \catcode45=12 % -
  \catcode46=12 % .
  \catcode47=12 % /
  \catcode58=12 % :
  \catcode64=11 % @
  \catcode91=12 % [
  \catcode93=12 % ]
  \catcode123=1 % {
  \catcode125=2 % }
  \expandafter\ifx\csname ProvidesPackage\endcsname\relax
    \def\x#1#2#3[#4]{\endgroup
      \immediate\write-1{Package: #3 #4}%
      \xdef#1{#4}%
    }%
  \else
    \def\x#1#2[#3]{\endgroup
      #2[{#3}]%
      \ifx#1\@undefined
        \xdef#1{#3}%
      \fi
      \ifx#1\relax
        \xdef#1{#3}%
      \fi
    }%
  \fi
\expandafter\x\csname ver@soulutf8.sty\endcsname
\ProvidesPackage{soulutf8}%
  [2016/05/16 v1.1 Permit use of UTF-8 characters in soul (HO)]%
%    \end{macrocode}
%
% \subsection{Catcodes}
%
%    \begin{macrocode}
\begingroup\catcode61\catcode48\catcode32=10\relax%
  \catcode13=5 % ^^M
  \endlinechar=13 %
  \catcode123=1 % {
  \catcode125=2 % }
  \catcode64=11 % @
  \def\x{\endgroup
    \expandafter\edef\csname SOuL@AtEnd\endcsname{%
      \endlinechar=\the\endlinechar\relax
      \catcode13=\the\catcode13\relax
      \catcode32=\the\catcode32\relax
      \catcode35=\the\catcode35\relax
      \catcode61=\the\catcode61\relax
      \catcode64=\the\catcode64\relax
      \catcode123=\the\catcode123\relax
      \catcode125=\the\catcode125\relax
    }%
  }%
\x\catcode61\catcode48\catcode32=10\relax%
\catcode13=5 % ^^M
\endlinechar=13 %
\catcode35=6 % #
\catcode64=11 % @
\catcode123=1 % {
\catcode125=2 % }
\def\TMP@EnsureCode#1#2{%
  \edef\SOuL@AtEnd{%
    \SOuL@AtEnd
    \catcode#1=\the\catcode#1\relax
  }%
  \catcode#1=#2\relax
}
\TMP@EnsureCode{10}{12}% ^^J
\TMP@EnsureCode{33}{12}% !
\TMP@EnsureCode{34}{12}% "
\TMP@EnsureCode{36}{3}% $
\TMP@EnsureCode{39}{12}% '
\TMP@EnsureCode{40}{12}% (
\TMP@EnsureCode{41}{12}% )
\TMP@EnsureCode{42}{12}% *
\TMP@EnsureCode{43}{12}% +
\TMP@EnsureCode{44}{12}% ,
\TMP@EnsureCode{45}{12}% -
\TMP@EnsureCode{46}{12}% .
\TMP@EnsureCode{47}{12}% /
\TMP@EnsureCode{58}{12}% :
\TMP@EnsureCode{60}{12}% <
\TMP@EnsureCode{62}{12}% >
\TMP@EnsureCode{91}{12}% [
\TMP@EnsureCode{93}{12}% ]
\TMP@EnsureCode{94}{7}% ^
\TMP@EnsureCode{96}{12}% `
\TMP@EnsureCode{126}\active % ~
\TMP@EnsureCode{128}{12}% ^^80
\TMP@EnsureCode{147}{12}% ^^93
\TMP@EnsureCode{148}{12}% ^^94
\TMP@EnsureCode{160}{12}% ^^a0
\TMP@EnsureCode{194}{12}% ^^c2
\TMP@EnsureCode{226}{12}% ^^e2
\edef\SOuL@AtEnd{\SOuL@AtEnd\noexpand\endinput}
%    \end{macrocode}
%
% \subsection{Loading packages}
%
%    Package \xpackage{soul} uses \cs{documentclass} to detect \LaTeX.
%    \begin{macrocode}
\ifx\documentclass\@undefined
%    \end{macrocode}
%
% \subsubsection{\plainTeX}
%
%    First we check, whether package \xpackage{soul} is already loaded.
%    \begin{macrocode}
  \expandafter\ifx\csname SOUL@\endcsname\relax
%    \end{macrocode}
%    In case of \plainTeX\ package \xpackage{soul} defines some
%    macros in a simple manner that will break the definitions of
%    \xfile{miniltx.tex}, for example. Therefore these macros
%    are first saved and restored afterwards.
%    \begin{macrocode}
    \let\SOuL@orgDeclareRobustCommand\DeclareRobustCommand
    \let\SOuL@orgnewcommand          \newcommand
    \let\SOuL@orgDeclareOption       \DeclareOption
    \let\SOuL@orgPackageError        \PackageError
    \def\SOuL@restorelatexcmds{%
      \let\DeclareRobustCommand\SOuL@orgDeclareRobustCommand
      \let\newcommand          \SOuL@orgnewcommand
      \let\DeclareOption       \SOuL@orgDeclareOption
      \let\PackageError        \SOuL@orgPackageError
    }%
    \input soul.sty\relax
    \SOuL@restorelatexcmds
  \fi
%    \end{macrocode}
%    \begin{macro}{\SOUL@error}
%     Package \xpackage{soul}'s use of \cs{PackageError} is replaced
%     by \cs{@PackageError} of package \xpackage{infwarerr}.
%    \begin{macrocode}
  \input infwarerr.sty\relax
  \let\SOuL@orgSOUL@error\SOUL@error
  \def\SOUL@error{%
    \begingroup
      \let\PackageError\@PackageError
      \SOuL@orgSOUL@error
    \endgroup
  }%
%    \end{macrocode}
%    \end{macro}
%    \begin{macrocode}
  \input etexcmds.sty\relax
%    \end{macrocode}
%    \begin{macro}{\@onelevel@sanitize}
%    Define \LaTeX's \cs{@onelevel@sanitize} if not already available.
%    \begin{macrocode}
  \expandafter\ifx\csname @onelevel@sanitize\endcsname\relax
    \def\@onelevel@sanitize#1{%
      \edef#1{%
        \expandafter\strip@prefix\meaning#1%
      }%
    }%
%    \end{macrocode}
%    \end{macro}
%    \begin{macro}{\strip@prefix}
%    \begin{macrocode}
    \def\strip@prefix#1>{}%
%    \end{macrocode}
%    \end{macro}
%    \begin{macrocode}
  \fi
\else
%    \end{macrocode}
%
% \subsubsection{\LaTeX}
%    \begin{macrocode}
  \DeclareOption*{\PassOptionsToPackage{\CurrentOption}{soul}}%
  \ProcessOptions\relax
  \RequirePackage{soul}[2003/11/17]%
  \RequirePackage{infwarerr}[2016/05/16]%
  \RequirePackage{etexcmds}[2016/05/16]%
\fi
%    \end{macrocode}
%
% \subsubsection{\eTeX}
%
%    In \plainTeX\ command \cs{+} is an \emph{outer} macro.
%    Therefore numbers are used to avoid problems.
%    \begin{macrocode}
\ifetex@unexpanded
  \catcode33=14 % '!': comment
  \catcode43=9  % '+': ignore
\else
  \catcode33=9  % '!': ignore
  \catcode43=14 % '+': comment
\fi
%    \end{macrocode}
%
% \subsection{Macro for redefinitions}
%
%    \begin{macro}{\SOuL@redefine}
%    \begin{macrocode}
\def\SOuL@redefine#1{%
  \begingroup
    \def\SOuL@cmd{#1}%
    \afterassignment\SOuL@cmdcheck
    \def\SOuL@temp
}
%    \end{macrocode}
%    \end{macro}
%    \begin{macro}{\SOuL@cmdcheck}
%    \begin{macrocode}
\def\SOuL@cmdcheck{%
    \expandafter\ifx\SOuL@cmd\SOuL@temp
    \else
      \edef\SOuL@temp*{\expandafter\string\SOuL@cmd}%
      \@PackageWarningNoLine{soulutf8}{%
        Command \SOuL@temp* has changed.\MessageBreak
        Supported versions of package `soul': 2003/11/17.\MessageBreak
        Depending on the unknown changes the redefinition\MessageBreak
        of \SOuL@temp* may not behave correctly%
      }%
    \fi
  \expandafter\endgroup
  \expandafter\def\SOuL@cmd
}
%    \end{macrocode}
%    \end{macro}
%
% \subsection{Redefinition of \cs{SOUL@eval}}
%
%    \begin{macro}{\SOUL@eval}
%    Macro \cs{SOUL@eval} is redefined to add detection of
%    the first byte of a UTF-8 sequence. Because \cs{SOUL@eval}
%    is overwritten, a warning is issued, if the contents of
%    \cs{SOUL@eval} is not as expected.
%    \begin{macrocode}
\SOuL@redefine\SOUL@eval{%
%    \end{macrocode}
%    First the expected definition.
%    \begin{macrocode}
  \def\SOUL@n*##1{\SOUL@scan}%
  \if\noexpand\SOUL@@\SOUL@spc
  \else
    \SOUL@ignorespacesfalse
  \fi
  \ifnum\SOUL@minus=\thr@@
    \SOUL@flushminus
  \else\ifnum\SOUL@comma=\tw@
    \SOUL@flushcomma
  \else\ifnum\SOUL@apo=\tw@
    \SOUL@flushapo
  \else\ifnum\SOUL@grave=\tw@
    \SOUL@flushgrave
  \fi\fi\fi\fi
  \ifx\SOUL@@-\else\SOUL@flushminus\fi
  \ifx\SOUL@@,\else\SOUL@flushcomma\fi
  \ifx\SOUL@@'\else\SOUL@flushapo\fi
  \ifx\SOUL@@`\else\SOUL@flushgrave\fi
  \ifx\SOUL@@-%
    \advance\SOUL@minus\@ne
  \else\ifx\SOUL@@,%
    \advance\SOUL@comma\@ne
  \else\ifx\SOUL@@'%
    \advance\SOUL@apo\@ne
  \else\ifx\SOUL@@`%
    \advance\SOUL@grave\@ne
  \else
    \SOUL@flushminus
    \SOUL@flushcomma
    \SOUL@flushapo
    \SOUL@flushgrave
    \ifx\SOUL@@\SOUL@stop
      \def\SOUL@n*{%
        \SOUL@doword
        \SOUL@eventuallyexhyphen\null
      }%
    \else\ifx\SOUL@@\par
      \def\SOUL@n*\par{\par\leavevmode\SOUL@scan}%
    \else\if\noexpand\SOUL@@\SOUL@spc
      \SOUL@doword
      \SOUL@eventuallyexhyphen\null
      \ifSOUL@ignorespaces
      \else
        \SOUL@everyspace{}%
      \fi
      \def\SOUL@n* {\SOUL@scan}%
    \else\ifx\SOUL@@\\%
      \SOUL@doword
      \SOUL@eventuallyexhyphen\null
      \SOUL@everyspace{\unskip\nobreak\hfil\break}%
      \SOUL@ignorespacestrue
    \else\ifx\SOUL@@~%
      \SOUL@doword
      \SOUL@eventuallyexhyphen\null
      \SOUL@everyspace{\nobreak}%
    \else\ifx\SOUL@@\slash
      \SOUL@doword
      \SOUL@eventuallyexhyphen{/}%
      \SOUL@exhyphen{/}%
    \else\ifx\SOUL@@\mbox
      \def\SOUL@n*{\SOUL@addprotect}%
    \else\ifx\SOUL@@\hbox
      \def\SOUL@n*{\SOUL@addprotect}%
    \else\ifx\SOUL@@\soulomit
      \def\SOUL@n*\soulomit##1{%
        \SOUL@doword
        {\spaceskip\SOUL@spaceskip##1}%
        \SOUL@scan
      }%
    \else\ifx\SOUL@@\break
      \SOUL@doword
      \break
    \else\ifx\SOUL@@\linebreak
      \SOUL@doword
      \SOUL@everyspace{\linebreak}%
    \else\ifcat\bgroup\noexpand\SOUL@@
      \def\SOUL@n*{\SOUL@addgroup{}}%
    \else\ifcat$\noexpand\SOUL@@
      \def\SOUL@n*{\SOUL@addmath}%
    \else
      \def\SOUL@n*{\SOUL@dotoken}%
    \fi\fi\fi\fi\fi\fi\fi\fi\fi\fi\fi\fi\fi
  \fi\fi\fi\fi
  \SOUL@n*%
}{%
%    \end{macrocode}
%    Now the redefined version follows.
%    \begin{macrocode}
  \def\SOUL@n*##1{\SOUL@scan}%
  \if\noexpand\SOUL@@\SOUL@spc
  \else
    \SOUL@ignorespacesfalse
  \fi
  \ifnum\SOUL@minus=\thr@@
    \SOUL@flushminus
  \else\ifnum\SOUL@comma=\tw@
    \SOUL@flushcomma
  \else\ifnum\SOUL@apo=\tw@
    \SOUL@flushapo
  \else\ifnum\SOUL@grave=\tw@
    \SOUL@flushgrave
  \fi\fi\fi\fi
  \ifx\SOUL@@-\else\SOUL@flushminus\fi
  \ifx\SOUL@@,\else\SOUL@flushcomma\fi
  \ifx\SOUL@@'\else\SOUL@flushapo\fi
  \ifx\SOUL@@`\else\SOUL@flushgrave\fi
  \ifx\SOUL@@-%
    \advance\SOUL@minus\@ne
  \else\ifx\SOUL@@,%
    \advance\SOUL@comma\@ne
  \else\ifx\SOUL@@'%
    \advance\SOUL@apo\@ne
  \else\ifx\SOUL@@`%
    \advance\SOUL@grave\@ne
  \else
    \SOUL@flushminus
    \SOUL@flushcomma
    \SOUL@flushapo
    \SOUL@flushgrave
    \ifx\SOUL@@\SOUL@stop
      \def\SOUL@n*{%
          \SOUL@doword
          \SOUL@eventuallyexhyphen\null
      }%
    \else\ifx\SOUL@@\par
      \def\SOUL@n*\par{\par\leavevmode\SOUL@scan}%
    \else\if\noexpand\SOUL@@\SOUL@spc
      \SOUL@doword
      \SOUL@eventuallyexhyphen\null
      \ifSOUL@ignorespaces
      \else
        \SOUL@everyspace{}%
      \fi
      \def\SOUL@n* {\SOUL@scan}%
    \else\ifx\SOUL@@\\%
      \SOUL@doword
      \SOUL@eventuallyexhyphen\null
      \SOUL@everyspace{\unskip\nobreak\hfil\break}%
      \SOUL@ignorespacestrue
    \else\ifx\SOUL@@~%
      \SOUL@doword
      \SOUL@eventuallyexhyphen\null
      \SOUL@everyspace{\nobreak}%
    \else\ifx\SOUL@@\slash
      \SOUL@doword
      \SOUL@eventuallyexhyphen{/}%
      \SOUL@exhyphen{/}%
    \else\ifx\SOUL@@\mbox
      \def\SOUL@n*{\SOUL@addprotect}%
    \else\ifx\SOUL@@\hbox
      \def\SOUL@n*{\SOUL@addprotect}%
    \else\ifx\SOUL@@\soulomit
      \def\SOUL@n*\soulomit##1{%
        \SOUL@doword
        {\spaceskip\SOUL@spaceskip##1}%
        \SOUL@scan
      }%
    \else\ifx\SOUL@@\break
      \SOUL@doword
      \break
    \else\ifx\SOUL@@\linebreak
      \SOUL@doword
      \SOUL@everyspace{\linebreak}%
    \else\ifcat\bgroup\noexpand\SOUL@@
      \def\SOUL@n*{\SOUL@addgroup{}}%
    \else\ifcat$\noexpand\SOUL@@
      \def\SOUL@n*{\SOUL@addmath}%
    \else
%    \end{macrocode}
%    The current token is examined to detect the start of
%    a UTF-8 sequence.
%    \begin{macrocode}
      \SOuL@analyzeutfviii
      \ifcase\SOuL@octets
        \SOuL@analyzeutfviiix
      \fi
      \ifcase\SOuL@octets
        \def\SOUL@n*{\SOUL@dotoken}%
      \or % 1
      \or % 2
        \def\SOUL@n*{\SOuL@addtwooctets}%
      \or % 3
        \def\SOUL@n*{\SOuL@addthreeoctets}%
      \or % 4
        \def\SOUL@n*{\SOuL@addfouroctets}%
      \fi
    \fi\fi\fi\fi\fi\fi\fi\fi\fi\fi\fi\fi\fi
  \fi\fi\fi\fi
  \SOUL@n*%
}
%    \end{macrocode}
%    \end{macro}
%
% \subsection{UTF-8 analysis}
%
% \subsubsection{Help strings}
%
%    \begin{macrocode}
\def\SOuL@defsanitizedstring#1#2{%
  \expandafter\def\csname SOuL@string#1\endcsname{#2}%
  \expandafter\@onelevel@sanitize\csname SOuL@string#1\endcsname
}
\SOuL@defsanitizedstring{UTFviii}{UTFviii@}
\SOuL@defsanitizedstring{octets}{@octets}
\SOuL@defsanitizedstring{two}{two}
\SOuL@defsanitizedstring{three}{three}
\SOuL@defsanitizedstring{four}{four}
\SOuL@defsanitizedstring{macrocolon}{macro:}
\SOuL@defsanitizedstring{csnameu}{csname u8-}
\SOuL@defsanitizedstring{undeferr}{utf@viii@undeferr}
\def\SOuL@stringendash{^^e2^^80^^93}
\def\SOuL@stringemdash{^^e2^^80^^94}
\def\SOuL@stringnobreakspace{^^c2^^a0}
\edef\SOuL@charhash{\string #}
\edef\SOuL@chartwo{\string 2}
\edef\SOuL@charthree{\string 3}
\def\SOuL@empty{}
%    \end{macrocode}
%
% \subsubsection{Support for \xfile{utf8.def}}
%
%    \begin{macro}{\SOuL@analyzeutfviii}
%    \begin{macrocode}
\begingroup
  \edef\x{\endgroup
    \def\noexpand\SOuL@analyzeutfviii{%
      \noexpand\expandafter\noexpand\SOuL@checkutfviii
      \noexpand\meaning\noexpand\SOUL@@
      \SOuL@stringUTFviii\SOuL@stringoctets
      \noexpand\@nil
    }%
    \def\noexpand\SOuL@checkutfviii
      ##1\SOuL@stringUTFviii##2\SOuL@stringoctets##3\noexpand\@nil
  }%
\x{%
  \def\SOuL@temp{#2}%
  \chardef\SOuL@octets=%
      \ifx\SOuL@temp\SOuL@stringtwo
        \tw@
      \else\ifx\SOuL@temp\SOuL@stringthree
        \thr@@
      \else\ifx\SOuL@temp\SOuL@stringfour
        4 %
      \else
        \z@
      \fi\fi\fi
}
%    \end{macrocode}
%    \end{macro}
%
% \subsubsection{Support for \xfile{utf8x.def}}
%
%    \begin{macro}{\SOuL@analyzeutfviiix}
%    \begin{macrocode}
\begingroup
  \edef\x{\endgroup
    \def\noexpand\SOuL@analyzeutfviiix{%
      \noexpand\expandafter\noexpand\SOuL@checkutfviiix
      \noexpand\meaning\noexpand\SOUL@@
      \SOuL@stringmacrocolon\SOuL@charhash1{}{}{}{}%
      \SOuL@stringcsnameu\SOuL@stringundeferr
      \noexpand\@nil
    }%
%    \end{macrocode}
%    \end{macro}
%    \begin{macro}{\SOuL@checkutfviiix}
%    \begin{macrocode}
    \def\noexpand\SOuL@checkutfviiix
      ##1\SOuL@stringmacrocolon\SOuL@charhash1##2##3##4##5##6%
      \SOuL@stringcsnameu##7\SOuL@stringundeferr##8\noexpand\@nil
  }%
\x{%
  \def\SOuL@temp{#7}%
  \ifx\SOuL@temp\SOuL@empty
    \chardef\SOuL@octets=\z@
  \else
    \def\SOuL@temp{#5}%
    \ifx\SOuL@temp\SOuL@charthree
      \chardef\SOuL@octets=4 %
    \else
      \def\SOuL@temp{#3}%
      \ifx\SOuL@temp\SOuL@chartwo
        \chardef\SOuL@octets=\thr@@
      \else
        \chardef\SOuL@octets=\tw@
      \fi
    \fi
  \fi
}
%    \end{macrocode}
%    \end{macro}
%
% \subsection{Actions for UTF-8 sequences}
%
%    \begin{macro}{\SOuL@addtwooctets}
%    \begin{macrocode}
\def\SOuL@addtwooctets#1#2{%
  \def\SOuL@temp{#1#2}%
  \@onelevel@sanitize\SOuL@temp
  \ifx\SOuL@temp\SOuL@stringnobreakspace
    \SOUL@doword
    \SOUL@eventuallyexhyphen\null
    \SOUL@everyspace{\nobreak}%
    \let\SOuL@next\SOUL@scan
  \else
    \def\SOuL@next{%
!     \SOUL@addtoken{{\noexpand#1\noexpand#2}}%
+     \SOUL@addtoken{{\etex@unexpanded{#1#2}}}%
    }%
  \fi
  \SOuL@next
}
%    \end{macrocode}
%    \end{macro}
%    \begin{macro}{\SOuL@addthreeoctets}
%    \begin{macrocode}
\def\SOuL@addthreeoctets#1#2#3{%
  \def\SOuL@temp{#1#2#3}%
  \@onelevel@sanitize\SOuL@temp
  \ifx\SOuL@temp\SOuL@stringendash
    \SOUL@doword
    \SOUL@eventuallyexhyphen{-}%
    \SOUL@exhyphen{--}%
    \let\SOuL@next\SOUL@scan
  \else
    \ifx\SOuL@temp\SOuL@stringemdash
      \SOUL@doword
      \SOUL@eventuallyexhyphen{-}%
      \SOUL@exhyphen{---}%
      \let\SOuL@next\SOUL@scan
    \else
      \def\SOuL@next{%
!       \SOUL@addtoken{{\noexpand#1\noexpand#2\noexpand#3}}%
+       \SOUL@addtoken{{\etex@unexpanded{#1#2#3}}}%
      }%
    \fi
  \fi
  \SOuL@next
}
%    \end{macrocode}
%    \end{macro}
%    \begin{macro}{\SOuL@addfouroctets}
%    \begin{macrocode}
\def\SOuL@addfouroctets#1#2#3#4{%
! \SOUL@addtoken{{\noexpand#1\noexpand#2\noexpand#3\noexpand#4}}%
+ \SOUL@addtoken{{\etex@unexpanded{#1#2#3#4}}}%
}
%    \end{macrocode}
%    \end{macro}
%
% \subsubsection{Redefinition of \cs{SOUL@splittoken}}
%
%    \begin{macro}{\SOUL@splittoken}
%    Macro \cs{SOUL@splittoken} separates the first token
%    or token group from a word and redefines the word to contain
%    the remaining tokens. However if the remaining tokens are
%    a token group, then the curly braces will be removed and
%    the token group is splitted by the next call of \cs{SOUL@splittoken}.
%    The redefinition avoids the removal of curly braces around the
%    remaining tokens.
%    \begin{macrocode}
\SOuL@redefine\SOUL@splittoken#1#2\SOUL@stop{%
  \global\SOUL@token={#1}%
  \global\SOUL@word={#2}%
}#1{%
  \global\SOUL@token={#1}%
  \SOuL@remainingtoken\relax
}
%    \end{macrocode}
%    \end{macro}
%    \begin{macro}{\SOuL@remainingtoken}
%    \begin{macrocode}
\def\SOuL@remainingtoken#1\SOUL@stop{%
  \global\SOUL@word=\expandafter{\@gobble#1}%
}
%    \end{macrocode}
%    \end{macro}
%
% \subsection{Patches}
%
%    The fixed \cs{SOUL@splittoken} allows to remove the double
%    sets of curly braces in other macros of package \xpackage{soul}.
%    The benefit is that implicite kernings are more often detected
%    and fixes a bug in package \xpackage{soul}. The disadvantage is
%    incompatibility. The width of the resulting strings may change.
%    \begin{macro}{\SOUL@flushcomma}
%    \begin{macrocode}
\SOuL@redefine\SOUL@flushcomma{%
  \ifcase\SOUL@comma
  \or
    \edef\x{\SOUL@word={\the\SOUL@word,}}\x
  \or
    \edef\x{\SOUL@word={\the\SOUL@word{{,,}}}}\x
  \fi
  \SOUL@comma\z@
}{%
  \ifcase\SOUL@comma
  \or
    \edef\x{\SOUL@word={\the\SOUL@word,}}\x
  \or
    \edef\x{\SOUL@word={\the\SOUL@word{,,}}}\x
  \fi
  \SOUL@comma\z@
}
%    \end{macrocode}
%    \end{macro}
%    \begin{macro}{\SOUL@flushapo}
%    \begin{macrocode}
\SOuL@redefine\SOUL@flushapo{%
  \ifcase\SOUL@apo
  \or
    \edef\x{\SOUL@word={\the\SOUL@word'}}\x
  \or
    \edef\x{\SOUL@word={\the\SOUL@word{{''}}}}\x
  \fi
  \SOUL@apo\z@
}{%
  \ifcase\SOUL@apo
  \or
    \edef\x{\SOUL@word={\the\SOUL@word'}}\x
  \or
    \edef\x{\SOUL@word={\the\SOUL@word{''}}}\x
  \fi
  \SOUL@apo\z@
}
%    \end{macrocode}
%    \end{macro}
%    \begin{macro}{\SOUL@flushgrave}
%    \begin{macrocode}
\SOuL@redefine\SOUL@flushgrave{%
  \ifcase\SOUL@grave
  \or
    \edef\x{\SOUL@word={\the\SOUL@word`}}\x
  \or
    \edef\x{\SOUL@word={\the\SOUL@word{{``}}}}\x
  \fi
  \SOUL@grave\z@
}{%
  \ifcase\SOUL@grave
  \or
    \edef\x{\SOUL@word={\the\SOUL@word`}}\x
  \or
    \edef\x{\SOUL@word={\the\SOUL@word{``}}}\x
  \fi
  \SOUL@grave\z@
}
%    \end{macrocode}
%    \end{macro}
%    \begin{macro}{\SOUL@addgroup}
%    \begin{macrocode}
\SOuL@redefine\SOUL@addgroup#1#2{%
  {%
    \let\protect\noexpand
    \edef\x{%
      \global\SOUL@word={%
        \the\SOUL@word
        {{\noexpand#1#2}}%
      }%
    }%
    \x
  }%
  \SOUL@scan
}#1#2{%
  \begingroup
    \let\protect\noexpand
    \edef\x{\endgroup
      \SOUL@word={%
        \the\SOUL@word
!       {\noexpand#1{#2}}%
+       {\etex@unexpanded{#1{#2}}}%
      }%
    }%
  \x
  \SOUL@scan
}
%    \end{macrocode}
%    \end{macro}
%    \begin{macro}{\SOUL@addmath}
%    \begin{macrocode}
\SOuL@redefine\SOUL@addmath$#1${%
  {%
    \let\protect\noexpand
    \edef\x{%
      \global\SOUL@word={%
        \the\SOUL@word
        {{\hbox{$#1$}}}%
      }%
    }%
    \x
  }%
  \SOUL@scan
}$#1${%
  \begingroup
    \let\protect\noexpand
    \edef\x{\endgroup
      \SOUL@word={%
        \the\SOUL@word
!       {\hbox{$#1$}}%
+       {\etex@unexpanded{\hbox{$#1$}}}%
      }%
    }%
  \x
  \SOUL@scan
}
%    \end{macrocode}
%    \end{macro}
%    \begin{macro}{\SOUL@addprotect}
%    \begin{macrocode}
\SOuL@redefine\SOUL@addprotect#1#2{%
  {%
    \let\protect\noexpand
    \edef\x{%
      \global\SOUL@word={%
        \the\SOUL@word
        {{\hbox{#2}}}%
      }%
    }%
    \x
  }%
  \SOUL@scan
}#1#2{%
  \begingroup
    \let\protect\noexpand
    \edef\x{\endgroup
      \SOUL@word={%
        \the\SOUL@word
!       {\hbox{#2}}%
+       {\etex@unexpanded{\hbox{#2}}}%
      }%
    }%
  \x
  \SOUL@scan
}
%    \end{macrocode}
%    \end{macro}
%
%    \begin{macro}{\SOUL@addtoken}
%    \begin{macrocode}
+ \SOuL@redefine\SOUL@addtoken#1{%
+   \edef\x{%
+     \SOUL@word={%
+       \the\SOUL@word
+       \noexpand#1%
+     }%
+   }%
+   \x
+   \SOUL@scan
+ }#1{%
+   \edef\x{%
+     \SOUL@word={%
+       \the\SOUL@word
+       \etex@unexpanded{#1}%
+     }%
+   }%
+   \x
+   \SOUL@scan
+ }%
%    \end{macrocode}
%    \end{macro}
%
%    \begin{macrocode}
\SOuL@AtEnd%
%    \end{macrocode}
%
%    \begin{macrocode}
%</package>
%    \end{macrocode}
%
% \section{Test}
%
% \subsection{Catcode checks for loading}
%
%    \begin{macrocode}
%<*test1>
%    \end{macrocode}
%    \begin{macrocode}
\catcode`\{=1 %
\catcode`\}=2 %
\catcode`\#=6 %
\catcode`\@=11 %
\expandafter\ifx\csname count@\endcsname\relax
  \countdef\count@=255 %
\fi
\expandafter\ifx\csname @gobble\endcsname\relax
  \long\def\@gobble#1{}%
\fi
\expandafter\ifx\csname @firstofone\endcsname\relax
  \long\def\@firstofone#1{#1}%
\fi
\expandafter\ifx\csname loop\endcsname\relax
  \expandafter\@firstofone
\else
  \expandafter\@gobble
\fi
{%
  \def\loop#1\repeat{%
    \def\body{#1}%
    \iterate
  }%
  \def\iterate{%
    \body
      \let\next\iterate
    \else
      \let\next\relax
    \fi
    \next
  }%
  \let\repeat=\fi
}%
\def\RestoreCatcodes{}
\count@=0 %
\loop
  \edef\RestoreCatcodes{%
    \RestoreCatcodes
    \catcode\the\count@=\the\catcode\count@\relax
  }%
\ifnum\count@<255 %
  \advance\count@ 1 %
\repeat

\def\RangeCatcodeInvalid#1#2{%
  \count@=#1\relax
  \loop
    \catcode\count@=15 %
  \ifnum\count@<#2\relax
    \advance\count@ 1 %
  \repeat
}
\def\RangeCatcodeCheck#1#2#3{%
  \count@=#1\relax
  \loop
    \ifnum#3=\catcode\count@
    \else
      \errmessage{%
        Character \the\count@\space
        with wrong catcode \the\catcode\count@\space
        instead of \number#3%
      }%
    \fi
  \ifnum\count@<#2\relax
    \advance\count@ 1 %
  \repeat
}
\def\space{ }
\expandafter\ifx\csname LoadCommand\endcsname\relax
  \def\LoadCommand{\input soulutf8.sty\relax}%
\fi
\def\Test{%
  \RangeCatcodeInvalid{0}{47}%
  \RangeCatcodeInvalid{58}{64}%
  \RangeCatcodeInvalid{91}{96}%
  \RangeCatcodeInvalid{123}{255}%
  \catcode`\@=12 %
  \catcode`\\=0 %
  \catcode`\%=14 %
  \LoadCommand
  \RangeCatcodeCheck{0}{36}{15}%
  \RangeCatcodeCheck{37}{37}{14}%
  \RangeCatcodeCheck{38}{47}{15}%
  \RangeCatcodeCheck{48}{57}{12}%
  \RangeCatcodeCheck{58}{63}{15}%
  \RangeCatcodeCheck{64}{64}{12}%
  \RangeCatcodeCheck{65}{90}{11}%
  \RangeCatcodeCheck{91}{91}{15}%
  \RangeCatcodeCheck{92}{92}{0}%
  \RangeCatcodeCheck{93}{96}{15}%
  \RangeCatcodeCheck{97}{122}{11}%
  \RangeCatcodeCheck{123}{255}{15}%
  \RestoreCatcodes
}
\Test
\csname @@end\endcsname
\end
%    \end{macrocode}
%    \begin{macrocode}
%</test1>
%    \end{macrocode}
%
% \subsection{\LaTeX\space tests}
%
%    \begin{macrocode}
%<*test2|test3|test4|test5>
%    \end{macrocode}
%    \begin{macrocode}
%<noetex>\let\SavedUnexpanded\unexpanded
%<noetex>\let\unexpanded\UNDEFINED
\NeedsTeXFormat{LaTeX2e}
\documentclass{article}
\usepackage[T1]{fontenc}
\usepackage{lmodern}
\usepackage{textcomp}
\usepackage[ngerman]{babel}
\expandafter\def\csname ver@hypdestopt.sty\endcsname{}
\usepackage[unicode]{hyperref}
\usepackage{bookmark}
\usepackage{color}
\usepackage{soul}[2003/11/17]
\makeatletter
\newcommand*{\@ifDefined}[1]{%
  \@ifundefined{#1}{%
    \@latex@error{%
      \expandafter\noexpand\csname#1\endcsname undefined%
    }\@ehc
  }%
}
\newcommand*{\SaveOrg}[1]{%
  \@ifdefinable{ORG@SOUL@#1}{%
    \@ifDefined{SOUL@#1}{%
      \expandafter\let\csname ORG@SOUL@#1\expandafter\endcsname
                      \csname SOUL@#1\endcsname
    }%
  }%
}
\newcommand*{\RestoreOrg}[1]{%
  \@ifDefined{ORG@SOUL@#1}{%
    \@ifDefined{SOUL@#1}{%
      \expandafter\let\csname SOUL@#1\expandafter\endcsname
                      \csname ORG@SOUL@#1\endcsname
    }%
  }%
}
\makeatother
\SaveOrg{eval}
\SaveOrg{splittoken}
\SaveOrg{flushcomma}
\SaveOrg{flushapo}
\SaveOrg{grave}
\SaveOrg{addgroup}
\SaveOrg{addmath}
\SaveOrg{addprotect}
\SaveOrg{addtoken}
\usepackage{soulutf8}[2016/05/16]
%    \end{macrocode}
%    \begin{macrocode}
%<noetex>\let\unexpanded\SavedUnexpanded
%    \end{macrocode}
%    Input encoding \xfile{utf8}.
%    \begin{macrocode}
%<*utf8>
\usepackage[utf8]{inputenc}
\DeclareUnicodeCharacter{00A0}{\nobreakspace}
%</utf8>
%    \end{macrocode}
%    Input encoding \xfile{utf8x}. Package \xpackage{ucs}
%    has a nasty bug, that it redefines \cs{show}.
%    \begin{macrocode}
%<*utf8x>
\let\origshow\show
\usepackage[utf8x]{inputenc}
\AtBeginDocument{%
  \let\show\origshow
}
%</utf8x>
%    \end{macrocode}
%
% \subsubsection{Driver definitions}
%
%    \begin{macrocode}
\makeatletter
%    \end{macrocode}
%    \begin{macro}{\an}
%    The definition for \cs{an} is taken from the documentation of
%    package \xpackage{soul}.
%    \begin{macrocode}
\DeclareRobustCommand*{\an}{%
  \def\SOUL@preamble{$^{^P}$}%
  \def\SOUL@everyspace##1{##1\texttt{\char`\ }\linebreak[2]}%
  \def\SOUL@postamble{$^{^E}$}%
  \def\SOUL@everyhyphen{$^{^-}$\linebreak[1]}%
  \def\SOUL@everyexhyphen##1{##1$^{^=}$}%
  \def\SOUL@everysyllable{$^{^S}$}%
  \def\SOUL@everytoken{\the\SOUL@token$^{^T}$}%
  \def\SOUL@everylowerthan{$^{^L}$}%
  \SOUL@
}
%    \end{macrocode}
%    \end{macro}
%
%    \begin{macro}{\sy}
%    The definition for \cs{sy} is taken from the documentation of
%    package \xpackage{soul}.
%    \begin{macrocode}
\DeclareRobustCommand*{\sy}{%
  \SOUL@setup
  \def\SOUL@preamble{%
    \lefthyphenmin=\z@
    \righthyphenmin=\z@
  }%
  \def\SOUL@everyspace##1{##1\space}%
  \def\SOUL@everysyllable{\the\SOUL@syllable}%
  \def\SOUL@everyhyphen{%
    \discretionary{%
      \SOUL@setkern\SOUL@hyphkern
      \SOUL@sethyphenchar
    }{}{%
      \hbox{\kern1pt$\cdot$}%
    }%
  }%
  \def\SOUL@everyexhyphen##1{%
    \SOUL@setkern\SOUL@hyphkern
    \hbox{##1}%
    \discretionary{}{}{%
      \SOUL@setkern\SOUL@charkern
    }%
  }%
  \SOUL@
}
%    \end{macrocode}
%    \end{macro}
%
%    \begin{macro}{\soulH}
%    \begin{macrocode}
\DeclareRobustCommand*{\soulH}{%
  \def\SOUL@preamble{%
    \gdef\SOULH{}%
    \soulHadd{[}%
  }%
  \def\SOUL@everyspace##1{%
    \soulHadd{Sp}%
  }%
  \def\SOUL@postamble{%
    \soulHadd{]}%
  }%
  \def\SOUL@everyhyphen{%
    \soulHadd{Hy}%
  }%
  \def\SOUL@everyexhyphen##1{%
    \soulHadd{Exhy}%
  }%
  \def\SOUL@everysyllable{%
    \soulHadd{Sy}%
  }%
  \def\SOUL@everytoken{%
    \soulHadd{T}%
  }%
  \def\SOUL@everylowerthan{%
    \soulHadd{<}%
  }%
  \SOUL@
}
\newcommand*{\soulHadd}[1]{%
  #1%
  \expandafter\gdef\expandafter\soulHstring\expandafter{%
    \soulHstring
    #1%
  }%
}
%    \end{macrocode}
%    \end{macro}
%
% \subsubsection{Test macros}
%
%    \begin{macrocode}
\usepackage{qstest}
\IncludeTests{*}
\LogTests{log}{*}{*}
%    \end{macrocode}
%    \begin{macrocode}
\newsavebox\boxA
\newsavebox\boxB
\newsavebox\boxC
%    \end{macrocode}
%    \begin{macro}{\TestDriver}
%    \begin{macrocode}
\newcommand*{\TestDriver}[4]{%
  \subsection{Driver #1}%
  \@PackageInfo{soulutf8-test}{* Driver #1}%
  \expandafter\let\expandafter\testdriver\csname #1\endcsname
  \def\teststringA{#3}%
  \def\teststringB{#4}%
  \begin{qstest}{string-#2-driver-#1}{string,#2,driver,#1}%
    \@PackageInfo{soulutf8-test}{* Version: soulutf8}%
    \global\let\soulHstring\@empty
    \setbox0=\hbox{\expandafter\testdriver\expandafter{\teststringA}}%
    \let\HstringA\soulHstring
    \SetTestVBox{\boxA}{\teststringA}%
    %
    \@PackageInfo{soulutf8-test}{* Version: soul + patch}%
    \global\let\soulHstring\@empty
    \RestoreOrg{eval}%
    \setbox2=\hbox{\expandafter\testdriver\expandafter{\teststringB}}%
    \let\HstringB\soulHstring
    \SetTestVBox{\boxB}{\teststringB}%
    %
    \@PackageInfo{soulutf8-test}{* Version: original soul}%
    \global\let\soulHstring\@empty
    \RestoreOrg{splittoken}%
    \RestoreOrg{flushcomma}%
    \RestoreOrg{flushapo}%
    \RestoreOrg{grave}%
    \RestoreOrg{addgroup}%
    \RestoreOrg{addmath}%
    \RestoreOrg{addprotect}%
    \RestoreOrg{addtoken}%
    \setbox4=\hbox{\expandafter\testdriver\expandafter{\teststringB}}%
    \let\HstringC\soulHstring
    \SetTestVBox{\boxC}{\teststringB}%
    %
    \TestInfo{0}{soulutf8}%
    \TestInfo{2}{soul + patch}%
    \TestInfo{4}{original soul}%
    \TestInfoVBox
    \showboxdepth=\maxdimen
    \showboxbreadth=\maxdimen
    \tracingonline=\@ne
    \Expect*{\HstringA}*{\HstringB}%
    \Expect*{\HstringA}*{\HstringC}%
    \Expect*{\the\wd0}*{\the\wd2}%
    \ifdim\wd0=\wd2 %
    \else
      \showbox0 %
      \showbox2 %
    \fi
    \Expect*{\the\ht0}*{\the\ht2}%
    \Expect*{\the\dp0}*{\the\dp2}%
    \Expect*{\the\wd\boxB}*{\the\wd\boxA}%
    \Expect*{\the\wd\boxC}*{\the\wd\boxA}%
    \Expect*{\the\ht\boxB}*{\the\ht\boxA}%
    \ifdim\ht\boxB=\ht\boxA
    \else
      \showbox\boxA
      \showbox\boxB
    \fi
    \Expect*{\the\ht\boxC}*{\the\ht\boxA}%
    \Expect*{\the\dp\boxB}*{\the\dp\boxA}%
    \Expect*{\the\dp\boxC}*{\the\dp\boxA}%
  \end{qstest}%
}
%    \end{macrocode}
%    \end{macro}
%    \begin{macro}{\SetTestVBox}
%    \begin{macrocode}
\newcommand*{\SetTestVBox}[2]{%
  \setbox#1=\vbox{%
    \hsize=0pt\relax
    \hfuzz=100mm\relax
    \expandafter\testdriver\expandafter{#2}%
  }%
}
%    \end{macrocode}
%    \end{macro}
%    \begin{macro}{\TestInfo}
%    \begin{macrocode}
\newcommand*{\TestInfo}[2]{%
  \par
  \noindent
  \copy#1\relax
  \hfill
  \begingroup
    \small
    \ifdim\wd0=\wd2\relax
      \ifdim\wd0=\wd4\relax
      \else
        \ifnum#1=2 %
          \raisebox{.5\baselineskip}[0pt][0pt]{%
            \llap{$\}$ }%
            \rlap{width = \the\wd#1\relax\space$\{$}%
            \phantom{width = 999.99999pt}%
            \qquad
          }%
        \else
          \ifnum#1=4 %
            \rlap{width = \the\wd#1}%
            \phantom{width = 999.99999pt}%
            \qquad
          \fi
        \fi
      \fi
    \else
      \begingroup
        \bfseries
        \rlap{width = \the\wd#1}%
        \phantom{width = 999.99999pt}%
      \endgroup
      \qquad
    \fi
    \rlap{(#2)}%
    \phantom{(original soul)}%
  \endgroup
  \par
}
%    \end{macrocode}
%    \end{macro}
%    \begin{macro}{\TestInfoVBox}
%    \begin{macrocode}
\newcommand*{\TestInfoVBox}{%
  \par
  \noindent
  \mbox{\vrule\relax\space\copy\boxA}%
  \ifdim\ht\boxA=\ht\boxB
  \else
    \hfill
    \begingroup
      \small
      \rlap{(soulutf8)}%
      \phantom{(soulutf8)}%
    \endgroup
    \par
    \noindent
    \mbox{\vrule\relax\space\copy\boxB}%
    \hfill
    \begingroup
      \small
      \rlap{(soul)}%
      \phantom{(soulutf8)}%
    \endgroup
  \fi
  \par
}
%    \end{macrocode}
%    \end{macro}
%    \begin{macro}{\Test}
%    \begin{macrocode}
\makeatletter
\newcommand*{\Test}[3]{%
  \clearpage
  \begingroup
    \def\linebreak{\textbackslash linebreak }%
    \def\\{\textbackslash\textbackslash}%
    \def\soulomit##1{%
      \textbackslash soulomit\textbraceleft##1\textbraceright
    }%
    \def\mbox##1{%
      \textbackslash mbox\textbraceleft##1\textbraceright
    }%
    \def\abc{\textbackslash abc }%
    \def\texttt##1{%
      \textbackslash texttt\textbraceleft##1\textbraceright
    }%
    \def\c##1{%
      \textbackslash c\textbraceleft##1\textbraceright
    }%
%<*utf8x>
    \count@=128 %
    \loop
    \ifnum\count@<256 %
      \lccode`\~=\count@
      \lowercase{%
        \def~{\noexpand~}%
      }%
      \advance\count@\@ne
    \repeat
%</utf8x>
    \csname protected@edef\endcsname\x{\endgroup
      \noexpand\section{String #1: #2}%
    }%
  \x
  \TestDriver{so}{#1}{#2}{#3}%
  \TestDriver{caps}{#1}{#2}{#3}%
  \TestDriver{ul}{#1}{#2}{#3}%
  \TestDriver{st}{#1}{#2}{#3}%
  \TestDriver{hl}{#1}{#2}{#3}%
  \TestDriver{an}{#1}{#2}{#3}%
  \TestDriver{sy}{#1}{#2}{#3}%
  \TestDriver{soulH}{#1}{#2}{#3}%
}
\makeatother
%    \end{macrocode}
%    \end{macro}
%
% \subsubsection{Testing}
%
%    \begin{macrocode}
\begin{document}
  \Test A{}{}%
  \Test B{Hello World}{Hello World}%
  \Test C{%
    Umlauts: ^^c3^^84^^c3^^96^^c3^^9c^^c3^^a4^^c3^^b6^^c3^^bc^^c3^^9f%
  }{%
    Umlauts: \"A\"O\"U\"a\"o\"u\ss
  }%
  \Test D{Euro: ^^e2^^82^^ac}{Euro: \texteuro}%
  \Test E{Stra^^c3^^9fe}{Stra\ss e}%
  \Test F{R^^c3^^b6mer}{R\"omer}%
  \Test G{A^^e2^^80^^93B^^e2^^80^^94C}{A--B---C}%
  \Test H{X^^c2^^a0Y}{X~Y}%
  \Test I{A^^c5^^a4}{A\v T}%
  \Test J{A,B,,C,,,D,,,,E}{A,B,,C,,,D,,,,E}%
  \Test K{A'B''C'''D''''E}{A'B''C'''D''''E}%
  \Test L{A`B``C```D````E}{A`B``C```D````E}%
  \Test M{A$x$B}{A$x$B}%
  \Test N{A\mbox{abc}B}{A\mbox{abc}B}%
  \Test O{,,TA''}{,,TA''}%
  \Test P{{!`}Hola{?`}}{{!`}Hola{?`}}%
  \Test Q{re-sent\slash x}{re-sent\slash x}%
  \Test R{new\\line}{new\\line}%
  \Test S{foo\linebreak bar}{foo\linebreak bar}%
  \Test T{Virchow{sche}}{Virchow{sche}}%
  \Test U{A\soulomit{foo}B}{A\soulomit{foo}B}%
  \Test V{A\texttt{xyz}B}{A\texttt{xyz}B}%
  \Test W{ffi}{ffi}%
  \Test X{f{}f{}i}{f{}f{}i}%
  \Test Y{f\mbox{}f\mbox{}i}{f\mbox{}f\mbox{}i}%
  \Test Z{\c{\"a}}{{\c{\"a}}}%
%<*test2|test3>
  \let\abcx\UNDEFINED
  \def\abc{%
    \begingroup
      \def\abcx{x}%
      \abcx
    \endgroup
  }%
  \Test{AA}{[{x\abc x}]}{[{xxx}]}%
%</test2|test3>
%    \end{macrocode}
%
%    \begin{macrocode}
\end{document}
%    \end{macrocode}
%    \begin{macrocode}
%</test2|test3|test4|test5>
%    \end{macrocode}
%
% \section{Installation}
%
% \subsection{Download}
%
% \paragraph{Package.} This package is available on
% CTAN\footnote{\url{http://ctan.org/pkg/soulutf8}}:
% \begin{description}
% \item[\CTAN{macros/latex/contrib/oberdiek/soulutf8.dtx}] The source file.
% \item[\CTAN{macros/latex/contrib/oberdiek/soulutf8.pdf}] Documentation.
% \end{description}
%
%
% \paragraph{Bundle.} All the packages of the bundle `oberdiek'
% are also available in a TDS compliant ZIP archive. There
% the packages are already unpacked and the documentation files
% are generated. The files and directories obey the TDS standard.
% \begin{description}
% \item[\CTAN{install/macros/latex/contrib/oberdiek.tds.zip}]
% \end{description}
% \emph{TDS} refers to the standard ``A Directory Structure
% for \TeX\ Files'' (\CTAN{tds/tds.pdf}). Directories
% with \xfile{texmf} in their name are usually organized this way.
%
% \subsection{Bundle installation}
%
% \paragraph{Unpacking.} Unpack the \xfile{oberdiek.tds.zip} in the
% TDS tree (also known as \xfile{texmf} tree) of your choice.
% Example (linux):
% \begin{quote}
%   |unzip oberdiek.tds.zip -d ~/texmf|
% \end{quote}
%
% \paragraph{Script installation.}
% Check the directory \xfile{TDS:scripts/oberdiek/} for
% scripts that need further installation steps.
% Package \xpackage{attachfile2} comes with the Perl script
% \xfile{pdfatfi.pl} that should be installed in such a way
% that it can be called as \texttt{pdfatfi}.
% Example (linux):
% \begin{quote}
%   |chmod +x scripts/oberdiek/pdfatfi.pl|\\
%   |cp scripts/oberdiek/pdfatfi.pl /usr/local/bin/|
% \end{quote}
%
% \subsection{Package installation}
%
% \paragraph{Unpacking.} The \xfile{.dtx} file is a self-extracting
% \docstrip\ archive. The files are extracted by running the
% \xfile{.dtx} through \plainTeX:
% \begin{quote}
%   \verb|tex soulutf8.dtx|
% \end{quote}
%
% \paragraph{TDS.} Now the different files must be moved into
% the different directories in your installation TDS tree
% (also known as \xfile{texmf} tree):
% \begin{quote}
% \def\t{^^A
% \begin{tabular}{@{}>{\ttfamily}l@{ $\rightarrow$ }>{\ttfamily}l@{}}
%   soulutf8.sty & tex/generic/oberdiek/soulutf8.sty\\
%   soulutf8.pdf & doc/latex/oberdiek/soulutf8.pdf\\
%   test/soulutf8-test1.tex & doc/latex/oberdiek/test/soulutf8-test1.tex\\
%   test/soulutf8-test2.tex & doc/latex/oberdiek/test/soulutf8-test2.tex\\
%   test/soulutf8-test3.tex & doc/latex/oberdiek/test/soulutf8-test3.tex\\
%   test/soulutf8-test4.tex & doc/latex/oberdiek/test/soulutf8-test4.tex\\
%   test/soulutf8-test5.tex & doc/latex/oberdiek/test/soulutf8-test5.tex\\
%   soulutf8.dtx & source/latex/oberdiek/soulutf8.dtx\\
% \end{tabular}^^A
% }^^A
% \sbox0{\t}^^A
% \ifdim\wd0>\linewidth
%   \begingroup
%     \advance\linewidth by\leftmargin
%     \advance\linewidth by\rightmargin
%   \edef\x{\endgroup
%     \def\noexpand\lw{\the\linewidth}^^A
%   }\x
%   \def\lwbox{^^A
%     \leavevmode
%     \hbox to \linewidth{^^A
%       \kern-\leftmargin\relax
%       \hss
%       \usebox0
%       \hss
%       \kern-\rightmargin\relax
%     }^^A
%   }^^A
%   \ifdim\wd0>\lw
%     \sbox0{\small\t}^^A
%     \ifdim\wd0>\linewidth
%       \ifdim\wd0>\lw
%         \sbox0{\footnotesize\t}^^A
%         \ifdim\wd0>\linewidth
%           \ifdim\wd0>\lw
%             \sbox0{\scriptsize\t}^^A
%             \ifdim\wd0>\linewidth
%               \ifdim\wd0>\lw
%                 \sbox0{\tiny\t}^^A
%                 \ifdim\wd0>\linewidth
%                   \lwbox
%                 \else
%                   \usebox0
%                 \fi
%               \else
%                 \lwbox
%               \fi
%             \else
%               \usebox0
%             \fi
%           \else
%             \lwbox
%           \fi
%         \else
%           \usebox0
%         \fi
%       \else
%         \lwbox
%       \fi
%     \else
%       \usebox0
%     \fi
%   \else
%     \lwbox
%   \fi
% \else
%   \usebox0
% \fi
% \end{quote}
% If you have a \xfile{docstrip.cfg} that configures and enables \docstrip's
% TDS installing feature, then some files can already be in the right
% place, see the documentation of \docstrip.
%
% \subsection{Refresh file name databases}
%
% If your \TeX~distribution
% (\teTeX, \mikTeX, \dots) relies on file name databases, you must refresh
% these. For example, \teTeX\ users run \verb|texhash| or
% \verb|mktexlsr|.
%
% \subsection{Some details for the interested}
%
% \paragraph{Attached source.}
%
% The PDF documentation on CTAN also includes the
% \xfile{.dtx} source file. It can be extracted by
% AcrobatReader 6 or higher. Another option is \textsf{pdftk},
% e.g. unpack the file into the current directory:
% \begin{quote}
%   \verb|pdftk soulutf8.pdf unpack_files output .|
% \end{quote}
%
% \paragraph{Unpacking with \LaTeX.}
% The \xfile{.dtx} chooses its action depending on the format:
% \begin{description}
% \item[\plainTeX:] Run \docstrip\ and extract the files.
% \item[\LaTeX:] Generate the documentation.
% \end{description}
% If you insist on using \LaTeX\ for \docstrip\ (really,
% \docstrip\ does not need \LaTeX), then inform the autodetect routine
% about your intention:
% \begin{quote}
%   \verb|latex \let\install=y% \iffalse meta-comment
%
% File: soulutf8.dtx
% Version: 2016/05/16 v1.1
% Info: Permit use of UTF-8 characters in soul
%
% Copyright (C) 2007 by
%    Heiko Oberdiek <heiko.oberdiek at googlemail.com>
%    2016
%    https://github.com/ho-tex/oberdiek/issues
%
% This work may be distributed and/or modified under the
% conditions of the LaTeX Project Public License, either
% version 1.3c of this license or (at your option) any later
% version. This version of this license is in
%    http://www.latex-project.org/lppl/lppl-1-3c.txt
% and the latest version of this license is in
%    http://www.latex-project.org/lppl.txt
% and version 1.3 or later is part of all distributions of
% LaTeX version 2005/12/01 or later.
%
% This work has the LPPL maintenance status "maintained".
%
% This Current Maintainer of this work is Heiko Oberdiek.
%
% The Base Interpreter refers to any `TeX-Format',
% because some files are installed in TDS:tex/generic//.
%
% This work consists of the main source file soulutf8.dtx
% and the derived files
%    soulutf8.sty, soulutf8.pdf, soulutf8.ins, soulutf8.drv,
%    soulutf8-test1.tex, soulutf8-test2.tex, soulutf8-test3.tex,
%    soulutf8-test4.tex, soulutf8-test5.tex.
%
% Distribution:
%    CTAN:macros/latex/contrib/oberdiek/soulutf8.dtx
%    CTAN:macros/latex/contrib/oberdiek/soulutf8.pdf
%
% Unpacking:
%    (a) If soulutf8.ins is present:
%           tex soulutf8.ins
%    (b) Without soulutf8.ins:
%           tex soulutf8.dtx
%    (c) If you insist on using LaTeX
%           latex \let\install=y% \iffalse meta-comment
%
% File: soulutf8.dtx
% Version: 2016/05/16 v1.1
% Info: Permit use of UTF-8 characters in soul
%
% Copyright (C) 2007 by
%    Heiko Oberdiek <heiko.oberdiek at googlemail.com>
%    2016
%    https://github.com/ho-tex/oberdiek/issues
%
% This work may be distributed and/or modified under the
% conditions of the LaTeX Project Public License, either
% version 1.3c of this license or (at your option) any later
% version. This version of this license is in
%    http://www.latex-project.org/lppl/lppl-1-3c.txt
% and the latest version of this license is in
%    http://www.latex-project.org/lppl.txt
% and version 1.3 or later is part of all distributions of
% LaTeX version 2005/12/01 or later.
%
% This work has the LPPL maintenance status "maintained".
%
% This Current Maintainer of this work is Heiko Oberdiek.
%
% The Base Interpreter refers to any `TeX-Format',
% because some files are installed in TDS:tex/generic//.
%
% This work consists of the main source file soulutf8.dtx
% and the derived files
%    soulutf8.sty, soulutf8.pdf, soulutf8.ins, soulutf8.drv,
%    soulutf8-test1.tex, soulutf8-test2.tex, soulutf8-test3.tex,
%    soulutf8-test4.tex, soulutf8-test5.tex.
%
% Distribution:
%    CTAN:macros/latex/contrib/oberdiek/soulutf8.dtx
%    CTAN:macros/latex/contrib/oberdiek/soulutf8.pdf
%
% Unpacking:
%    (a) If soulutf8.ins is present:
%           tex soulutf8.ins
%    (b) Without soulutf8.ins:
%           tex soulutf8.dtx
%    (c) If you insist on using LaTeX
%           latex \let\install=y\input{soulutf8.dtx}
%        (quote the arguments according to the demands of your shell)
%
% Documentation:
%    (a) If soulutf8.drv is present:
%           latex soulutf8.drv
%    (b) Without soulutf8.drv:
%           latex soulutf8.dtx; ...
%    The class ltxdoc loads the configuration file ltxdoc.cfg
%    if available. Here you can specify further options, e.g.
%    use A4 as paper format:
%       \PassOptionsToClass{a4paper}{article}
%
%    Programm calls to get the documentation (example):
%       pdflatex soulutf8.dtx
%       makeindex -s gind.ist soulutf8.idx
%       pdflatex soulutf8.dtx
%       makeindex -s gind.ist soulutf8.idx
%       pdflatex soulutf8.dtx
%
% Installation:
%    TDS:tex/generic/oberdiek/soulutf8.sty
%    TDS:doc/latex/oberdiek/soulutf8.pdf
%    TDS:doc/latex/oberdiek/test/soulutf8-test1.tex
%    TDS:doc/latex/oberdiek/test/soulutf8-test2.tex
%    TDS:doc/latex/oberdiek/test/soulutf8-test3.tex
%    TDS:doc/latex/oberdiek/test/soulutf8-test4.tex
%    TDS:doc/latex/oberdiek/test/soulutf8-test5.tex
%    TDS:source/latex/oberdiek/soulutf8.dtx
%
%<*ignore>
\begingroup
  \catcode123=1 %
  \catcode125=2 %
  \def\x{LaTeX2e}%
\expandafter\endgroup
\ifcase 0\ifx\install y1\fi\expandafter
         \ifx\csname processbatchFile\endcsname\relax\else1\fi
         \ifx\fmtname\x\else 1\fi\relax
\else\csname fi\endcsname
%</ignore>
%<*install>
\input docstrip.tex
\Msg{************************************************************************}
\Msg{* Installation}
\Msg{* Package: soulutf8 2016/05/16 v1.1 Permit use of UTF-8 characters in soul (HO)}
\Msg{************************************************************************}

\keepsilent
\askforoverwritefalse

\let\MetaPrefix\relax
\preamble

This is a generated file.

Project: soulutf8
Version: 2016/05/16 v1.1

Copyright (C) 2007 by
   Heiko Oberdiek <heiko.oberdiek at googlemail.com>

This work may be distributed and/or modified under the
conditions of the LaTeX Project Public License, either
version 1.3c of this license or (at your option) any later
version. This version of this license is in
   http://www.latex-project.org/lppl/lppl-1-3c.txt
and the latest version of this license is in
   http://www.latex-project.org/lppl.txt
and version 1.3 or later is part of all distributions of
LaTeX version 2005/12/01 or later.

This work has the LPPL maintenance status "maintained".

This Current Maintainer of this work is Heiko Oberdiek.

The Base Interpreter refers to any `TeX-Format',
because some files are installed in TDS:tex/generic//.

This work consists of the main source file soulutf8.dtx
and the derived files
   soulutf8.sty, soulutf8.pdf, soulutf8.ins, soulutf8.drv,
   soulutf8-test1.tex, soulutf8-test2.tex, soulutf8-test3.tex,
   soulutf8-test4.tex, soulutf8-test5.tex.

\endpreamble
\let\MetaPrefix\DoubleperCent

\generate{%
  \file{soulutf8.ins}{\from{soulutf8.dtx}{install}}%
  \file{soulutf8.drv}{\from{soulutf8.dtx}{driver}}%
  \usedir{tex/generic/oberdiek}%
  \file{soulutf8.sty}{\from{soulutf8.dtx}{package}}%
%  \usedir{doc/latex/oberdiek/test}%
%  \file{soulutf8-test1.tex}{\from{soulutf8.dtx}{test1}}%
%  \file{soulutf8-test2.tex}{\from{soulutf8.dtx}{test2,utf8}}%
%  \file{soulutf8-test3.tex}{\from{soulutf8.dtx}{test3,utf8x}}%
%  \file{soulutf8-test4.tex}{\from{soulutf8.dtx}{test4,utf8,noetex}}%
%  \file{soulutf8-test5.tex}{\from{soulutf8.dtx}{test5,utf8x,noetex}}%
  \nopreamble
  \nopostamble
%  \usedir{source/latex/oberdiek/catalogue}%
%  \file{soulutf8.xml}{\from{soulutf8.dtx}{catalogue}}%
}

\catcode32=13\relax% active space
\let =\space%
\Msg{************************************************************************}
\Msg{*}
\Msg{* To finish the installation you have to move the following}
\Msg{* file into a directory searched by TeX:}
\Msg{*}
\Msg{*     soulutf8.sty}
\Msg{*}
\Msg{* To produce the documentation run the file `soulutf8.drv'}
\Msg{* through LaTeX.}
\Msg{*}
\Msg{* Happy TeXing!}
\Msg{*}
\Msg{************************************************************************}

\endbatchfile
%</install>
%<*ignore>
\fi
%</ignore>
%<*driver>
\NeedsTeXFormat{LaTeX2e}
\ProvidesFile{soulutf8.drv}%
  [2016/05/16 v1.1 Permit use of UTF-8 characters in soul (HO)]%
\documentclass{ltxdoc}
\usepackage{holtxdoc}[2011/11/22]
\begin{document}
  \DocInput{soulutf8.dtx}%
\end{document}
%</driver>
% \fi
%
%
% \CharacterTable
%  {Upper-case    \A\B\C\D\E\F\G\H\I\J\K\L\M\N\O\P\Q\R\S\T\U\V\W\X\Y\Z
%   Lower-case    \a\b\c\d\e\f\g\h\i\j\k\l\m\n\o\p\q\r\s\t\u\v\w\x\y\z
%   Digits        \0\1\2\3\4\5\6\7\8\9
%   Exclamation   \!     Double quote  \"     Hash (number) \#
%   Dollar        \$     Percent       \%     Ampersand     \&
%   Acute accent  \'     Left paren    \(     Right paren   \)
%   Asterisk      \*     Plus          \+     Comma         \,
%   Minus         \-     Point         \.     Solidus       \/
%   Colon         \:     Semicolon     \;     Less than     \<
%   Equals        \=     Greater than  \>     Question mark \?
%   Commercial at \@     Left bracket  \[     Backslash     \\
%   Right bracket \]     Circumflex    \^     Underscore    \_
%   Grave accent  \`     Left brace    \{     Vertical bar  \|
%   Right brace   \}     Tilde         \~}
%
% \GetFileInfo{soulutf8.drv}
%
% \title{The \xpackage{soulutf8} package}
% \date{2016/05/16 v1.1}
% \author{Heiko Oberdiek\thanks
% {Please report any issues at https://github.com/ho-tex/oberdiek/issues}\\
% \xemail{heiko.oberdiek at googlemail.com}}
%
% \maketitle
%
% \begin{abstract}
% This package extends package \xpackage{soul} and adds some support
% for UTF-8. Namely the input encodings \xfile{utf8.def}
% from package \xpackage{inputenc} and
% package \xpackage{ucs}'s \xfile{utf8x.def} are supported.
% \end{abstract}
%
% \tableofcontents
%
% \section{Documentation}
%
% This package \xpackage{soulutf8} does not have own options and does not
% define new user commands.
% Any option is passed to package \xpackage{soul} \cite{soul}
% that is loaded first.
% Then some internal macros of \xpackage{soul} are redefined to add
% support for UTF-8.
% The following input encodings are supported:
% \begin{quote}
%   \begin{tabular}{@{}lll@{}}
%     \texttt{utf8} & \LaTeX\ base &
%        \xfile{TDS:tex/latex/base/utf8.def} \cite{utf8}\\
%     \texttt{utf8x} & Package \xpackage{ucs} &
%       \xfile{TDS:tex/latex/ucs/utf8x.def} \cite{ucs}\\
%   \end{tabular}
% \end{quote}
% UTF-8 byte sequences are added as token group to a word, even
% if these UTF-8 characters are some kind of hyphen or space.
% As exception the following three Unicode characters are handled specially:
% \begin{quote}
%   \begin{tabular}{lll}
%     Slot & Name & Action\\
%     \hline
%     \texttt{U+00A0} & NO-BREAK SPACE & like |~|\\
%     \texttt{U+2013} & EN DASH & |--|\\
%     \texttt{U+2014} & EM DASH & |---|
%   \end{tabular}
%
% \subsection{Patch}
%
% Also package \xpackage{soulutf8} tries to patch package \xpackage{soul}
% to improve its behaviour:
% \begin{itemize}
% \item
%   A problem with additional levels of curly braces is fixed.
%   As advantage more implicite kernings are detected.
%   However, the result may be incompatible with the
%   original behaviour of package \xpackage{soul} because
%   of these respected implicite kernings.
% \item
%   \eTeX\ , especially \cs{unexpanded} is supported.
%   This allows a better protection of token groups
%   (|\mbox{|\dots|}|, math, \dots).
% \end{itemize}
% \end{quote}
%
% \subsection{Future}
%
% Currently package \xpackage{soul} does not seem to be
% maintained. Nevertheless if there will be a new version that
% adds support for UTF-8, then this package may become
% obsolete.
%
% \StopEventually{
% }
%
% \section{Implementation}
%
%    \begin{macrocode}
%<*package>
%    \end{macrocode}
%
% \subsection{Reload check and package identification}
%    Reload check, especially if the package is not used with \LaTeX.
%    \begin{macrocode}
\begingroup\catcode61\catcode48\catcode32=10\relax%
  \catcode13=5 % ^^M
  \endlinechar=13 %
  \catcode35=6 % #
  \catcode39=12 % '
  \catcode44=12 % ,
  \catcode45=12 % -
  \catcode46=12 % .
  \catcode58=12 % :
  \catcode64=11 % @
  \catcode123=1 % {
  \catcode125=2 % }
  \expandafter\let\expandafter\x\csname ver@soulutf8.sty\endcsname
  \ifx\x\relax % plain-TeX, first loading
  \else
    \def\empty{}%
    \ifx\x\empty % LaTeX, first loading,
      % variable is initialized, but \ProvidesPackage not yet seen
    \else
      \expandafter\ifx\csname PackageInfo\endcsname\relax
        \def\x#1#2{%
          \immediate\write-1{Package #1 Info: #2.}%
        }%
      \else
        \def\x#1#2{\PackageInfo{#1}{#2, stopped}}%
      \fi
      \x{soulutf8}{The package is already loaded}%
      \aftergroup\endinput
    \fi
  \fi
\endgroup%
%    \end{macrocode}
%    Package identification:
%    \begin{macrocode}
\begingroup\catcode61\catcode48\catcode32=10\relax%
  \catcode13=5 % ^^M
  \endlinechar=13 %
  \catcode35=6 % #
  \catcode39=12 % '
  \catcode40=12 % (
  \catcode41=12 % )
  \catcode44=12 % ,
  \catcode45=12 % -
  \catcode46=12 % .
  \catcode47=12 % /
  \catcode58=12 % :
  \catcode64=11 % @
  \catcode91=12 % [
  \catcode93=12 % ]
  \catcode123=1 % {
  \catcode125=2 % }
  \expandafter\ifx\csname ProvidesPackage\endcsname\relax
    \def\x#1#2#3[#4]{\endgroup
      \immediate\write-1{Package: #3 #4}%
      \xdef#1{#4}%
    }%
  \else
    \def\x#1#2[#3]{\endgroup
      #2[{#3}]%
      \ifx#1\@undefined
        \xdef#1{#3}%
      \fi
      \ifx#1\relax
        \xdef#1{#3}%
      \fi
    }%
  \fi
\expandafter\x\csname ver@soulutf8.sty\endcsname
\ProvidesPackage{soulutf8}%
  [2016/05/16 v1.1 Permit use of UTF-8 characters in soul (HO)]%
%    \end{macrocode}
%
% \subsection{Catcodes}
%
%    \begin{macrocode}
\begingroup\catcode61\catcode48\catcode32=10\relax%
  \catcode13=5 % ^^M
  \endlinechar=13 %
  \catcode123=1 % {
  \catcode125=2 % }
  \catcode64=11 % @
  \def\x{\endgroup
    \expandafter\edef\csname SOuL@AtEnd\endcsname{%
      \endlinechar=\the\endlinechar\relax
      \catcode13=\the\catcode13\relax
      \catcode32=\the\catcode32\relax
      \catcode35=\the\catcode35\relax
      \catcode61=\the\catcode61\relax
      \catcode64=\the\catcode64\relax
      \catcode123=\the\catcode123\relax
      \catcode125=\the\catcode125\relax
    }%
  }%
\x\catcode61\catcode48\catcode32=10\relax%
\catcode13=5 % ^^M
\endlinechar=13 %
\catcode35=6 % #
\catcode64=11 % @
\catcode123=1 % {
\catcode125=2 % }
\def\TMP@EnsureCode#1#2{%
  \edef\SOuL@AtEnd{%
    \SOuL@AtEnd
    \catcode#1=\the\catcode#1\relax
  }%
  \catcode#1=#2\relax
}
\TMP@EnsureCode{10}{12}% ^^J
\TMP@EnsureCode{33}{12}% !
\TMP@EnsureCode{34}{12}% "
\TMP@EnsureCode{36}{3}% $
\TMP@EnsureCode{39}{12}% '
\TMP@EnsureCode{40}{12}% (
\TMP@EnsureCode{41}{12}% )
\TMP@EnsureCode{42}{12}% *
\TMP@EnsureCode{43}{12}% +
\TMP@EnsureCode{44}{12}% ,
\TMP@EnsureCode{45}{12}% -
\TMP@EnsureCode{46}{12}% .
\TMP@EnsureCode{47}{12}% /
\TMP@EnsureCode{58}{12}% :
\TMP@EnsureCode{60}{12}% <
\TMP@EnsureCode{62}{12}% >
\TMP@EnsureCode{91}{12}% [
\TMP@EnsureCode{93}{12}% ]
\TMP@EnsureCode{94}{7}% ^
\TMP@EnsureCode{96}{12}% `
\TMP@EnsureCode{126}\active % ~
\TMP@EnsureCode{128}{12}% ^^80
\TMP@EnsureCode{147}{12}% ^^93
\TMP@EnsureCode{148}{12}% ^^94
\TMP@EnsureCode{160}{12}% ^^a0
\TMP@EnsureCode{194}{12}% ^^c2
\TMP@EnsureCode{226}{12}% ^^e2
\edef\SOuL@AtEnd{\SOuL@AtEnd\noexpand\endinput}
%    \end{macrocode}
%
% \subsection{Loading packages}
%
%    Package \xpackage{soul} uses \cs{documentclass} to detect \LaTeX.
%    \begin{macrocode}
\ifx\documentclass\@undefined
%    \end{macrocode}
%
% \subsubsection{\plainTeX}
%
%    First we check, whether package \xpackage{soul} is already loaded.
%    \begin{macrocode}
  \expandafter\ifx\csname SOUL@\endcsname\relax
%    \end{macrocode}
%    In case of \plainTeX\ package \xpackage{soul} defines some
%    macros in a simple manner that will break the definitions of
%    \xfile{miniltx.tex}, for example. Therefore these macros
%    are first saved and restored afterwards.
%    \begin{macrocode}
    \let\SOuL@orgDeclareRobustCommand\DeclareRobustCommand
    \let\SOuL@orgnewcommand          \newcommand
    \let\SOuL@orgDeclareOption       \DeclareOption
    \let\SOuL@orgPackageError        \PackageError
    \def\SOuL@restorelatexcmds{%
      \let\DeclareRobustCommand\SOuL@orgDeclareRobustCommand
      \let\newcommand          \SOuL@orgnewcommand
      \let\DeclareOption       \SOuL@orgDeclareOption
      \let\PackageError        \SOuL@orgPackageError
    }%
    \input soul.sty\relax
    \SOuL@restorelatexcmds
  \fi
%    \end{macrocode}
%    \begin{macro}{\SOUL@error}
%     Package \xpackage{soul}'s use of \cs{PackageError} is replaced
%     by \cs{@PackageError} of package \xpackage{infwarerr}.
%    \begin{macrocode}
  \input infwarerr.sty\relax
  \let\SOuL@orgSOUL@error\SOUL@error
  \def\SOUL@error{%
    \begingroup
      \let\PackageError\@PackageError
      \SOuL@orgSOUL@error
    \endgroup
  }%
%    \end{macrocode}
%    \end{macro}
%    \begin{macrocode}
  \input etexcmds.sty\relax
%    \end{macrocode}
%    \begin{macro}{\@onelevel@sanitize}
%    Define \LaTeX's \cs{@onelevel@sanitize} if not already available.
%    \begin{macrocode}
  \expandafter\ifx\csname @onelevel@sanitize\endcsname\relax
    \def\@onelevel@sanitize#1{%
      \edef#1{%
        \expandafter\strip@prefix\meaning#1%
      }%
    }%
%    \end{macrocode}
%    \end{macro}
%    \begin{macro}{\strip@prefix}
%    \begin{macrocode}
    \def\strip@prefix#1>{}%
%    \end{macrocode}
%    \end{macro}
%    \begin{macrocode}
  \fi
\else
%    \end{macrocode}
%
% \subsubsection{\LaTeX}
%    \begin{macrocode}
  \DeclareOption*{\PassOptionsToPackage{\CurrentOption}{soul}}%
  \ProcessOptions\relax
  \RequirePackage{soul}[2003/11/17]%
  \RequirePackage{infwarerr}[2016/05/16]%
  \RequirePackage{etexcmds}[2016/05/16]%
\fi
%    \end{macrocode}
%
% \subsubsection{\eTeX}
%
%    In \plainTeX\ command \cs{+} is an \emph{outer} macro.
%    Therefore numbers are used to avoid problems.
%    \begin{macrocode}
\ifetex@unexpanded
  \catcode33=14 % '!': comment
  \catcode43=9  % '+': ignore
\else
  \catcode33=9  % '!': ignore
  \catcode43=14 % '+': comment
\fi
%    \end{macrocode}
%
% \subsection{Macro for redefinitions}
%
%    \begin{macro}{\SOuL@redefine}
%    \begin{macrocode}
\def\SOuL@redefine#1{%
  \begingroup
    \def\SOuL@cmd{#1}%
    \afterassignment\SOuL@cmdcheck
    \def\SOuL@temp
}
%    \end{macrocode}
%    \end{macro}
%    \begin{macro}{\SOuL@cmdcheck}
%    \begin{macrocode}
\def\SOuL@cmdcheck{%
    \expandafter\ifx\SOuL@cmd\SOuL@temp
    \else
      \edef\SOuL@temp*{\expandafter\string\SOuL@cmd}%
      \@PackageWarningNoLine{soulutf8}{%
        Command \SOuL@temp* has changed.\MessageBreak
        Supported versions of package `soul': 2003/11/17.\MessageBreak
        Depending on the unknown changes the redefinition\MessageBreak
        of \SOuL@temp* may not behave correctly%
      }%
    \fi
  \expandafter\endgroup
  \expandafter\def\SOuL@cmd
}
%    \end{macrocode}
%    \end{macro}
%
% \subsection{Redefinition of \cs{SOUL@eval}}
%
%    \begin{macro}{\SOUL@eval}
%    Macro \cs{SOUL@eval} is redefined to add detection of
%    the first byte of a UTF-8 sequence. Because \cs{SOUL@eval}
%    is overwritten, a warning is issued, if the contents of
%    \cs{SOUL@eval} is not as expected.
%    \begin{macrocode}
\SOuL@redefine\SOUL@eval{%
%    \end{macrocode}
%    First the expected definition.
%    \begin{macrocode}
  \def\SOUL@n*##1{\SOUL@scan}%
  \if\noexpand\SOUL@@\SOUL@spc
  \else
    \SOUL@ignorespacesfalse
  \fi
  \ifnum\SOUL@minus=\thr@@
    \SOUL@flushminus
  \else\ifnum\SOUL@comma=\tw@
    \SOUL@flushcomma
  \else\ifnum\SOUL@apo=\tw@
    \SOUL@flushapo
  \else\ifnum\SOUL@grave=\tw@
    \SOUL@flushgrave
  \fi\fi\fi\fi
  \ifx\SOUL@@-\else\SOUL@flushminus\fi
  \ifx\SOUL@@,\else\SOUL@flushcomma\fi
  \ifx\SOUL@@'\else\SOUL@flushapo\fi
  \ifx\SOUL@@`\else\SOUL@flushgrave\fi
  \ifx\SOUL@@-%
    \advance\SOUL@minus\@ne
  \else\ifx\SOUL@@,%
    \advance\SOUL@comma\@ne
  \else\ifx\SOUL@@'%
    \advance\SOUL@apo\@ne
  \else\ifx\SOUL@@`%
    \advance\SOUL@grave\@ne
  \else
    \SOUL@flushminus
    \SOUL@flushcomma
    \SOUL@flushapo
    \SOUL@flushgrave
    \ifx\SOUL@@\SOUL@stop
      \def\SOUL@n*{%
        \SOUL@doword
        \SOUL@eventuallyexhyphen\null
      }%
    \else\ifx\SOUL@@\par
      \def\SOUL@n*\par{\par\leavevmode\SOUL@scan}%
    \else\if\noexpand\SOUL@@\SOUL@spc
      \SOUL@doword
      \SOUL@eventuallyexhyphen\null
      \ifSOUL@ignorespaces
      \else
        \SOUL@everyspace{}%
      \fi
      \def\SOUL@n* {\SOUL@scan}%
    \else\ifx\SOUL@@\\%
      \SOUL@doword
      \SOUL@eventuallyexhyphen\null
      \SOUL@everyspace{\unskip\nobreak\hfil\break}%
      \SOUL@ignorespacestrue
    \else\ifx\SOUL@@~%
      \SOUL@doword
      \SOUL@eventuallyexhyphen\null
      \SOUL@everyspace{\nobreak}%
    \else\ifx\SOUL@@\slash
      \SOUL@doword
      \SOUL@eventuallyexhyphen{/}%
      \SOUL@exhyphen{/}%
    \else\ifx\SOUL@@\mbox
      \def\SOUL@n*{\SOUL@addprotect}%
    \else\ifx\SOUL@@\hbox
      \def\SOUL@n*{\SOUL@addprotect}%
    \else\ifx\SOUL@@\soulomit
      \def\SOUL@n*\soulomit##1{%
        \SOUL@doword
        {\spaceskip\SOUL@spaceskip##1}%
        \SOUL@scan
      }%
    \else\ifx\SOUL@@\break
      \SOUL@doword
      \break
    \else\ifx\SOUL@@\linebreak
      \SOUL@doword
      \SOUL@everyspace{\linebreak}%
    \else\ifcat\bgroup\noexpand\SOUL@@
      \def\SOUL@n*{\SOUL@addgroup{}}%
    \else\ifcat$\noexpand\SOUL@@
      \def\SOUL@n*{\SOUL@addmath}%
    \else
      \def\SOUL@n*{\SOUL@dotoken}%
    \fi\fi\fi\fi\fi\fi\fi\fi\fi\fi\fi\fi\fi
  \fi\fi\fi\fi
  \SOUL@n*%
}{%
%    \end{macrocode}
%    Now the redefined version follows.
%    \begin{macrocode}
  \def\SOUL@n*##1{\SOUL@scan}%
  \if\noexpand\SOUL@@\SOUL@spc
  \else
    \SOUL@ignorespacesfalse
  \fi
  \ifnum\SOUL@minus=\thr@@
    \SOUL@flushminus
  \else\ifnum\SOUL@comma=\tw@
    \SOUL@flushcomma
  \else\ifnum\SOUL@apo=\tw@
    \SOUL@flushapo
  \else\ifnum\SOUL@grave=\tw@
    \SOUL@flushgrave
  \fi\fi\fi\fi
  \ifx\SOUL@@-\else\SOUL@flushminus\fi
  \ifx\SOUL@@,\else\SOUL@flushcomma\fi
  \ifx\SOUL@@'\else\SOUL@flushapo\fi
  \ifx\SOUL@@`\else\SOUL@flushgrave\fi
  \ifx\SOUL@@-%
    \advance\SOUL@minus\@ne
  \else\ifx\SOUL@@,%
    \advance\SOUL@comma\@ne
  \else\ifx\SOUL@@'%
    \advance\SOUL@apo\@ne
  \else\ifx\SOUL@@`%
    \advance\SOUL@grave\@ne
  \else
    \SOUL@flushminus
    \SOUL@flushcomma
    \SOUL@flushapo
    \SOUL@flushgrave
    \ifx\SOUL@@\SOUL@stop
      \def\SOUL@n*{%
          \SOUL@doword
          \SOUL@eventuallyexhyphen\null
      }%
    \else\ifx\SOUL@@\par
      \def\SOUL@n*\par{\par\leavevmode\SOUL@scan}%
    \else\if\noexpand\SOUL@@\SOUL@spc
      \SOUL@doword
      \SOUL@eventuallyexhyphen\null
      \ifSOUL@ignorespaces
      \else
        \SOUL@everyspace{}%
      \fi
      \def\SOUL@n* {\SOUL@scan}%
    \else\ifx\SOUL@@\\%
      \SOUL@doword
      \SOUL@eventuallyexhyphen\null
      \SOUL@everyspace{\unskip\nobreak\hfil\break}%
      \SOUL@ignorespacestrue
    \else\ifx\SOUL@@~%
      \SOUL@doword
      \SOUL@eventuallyexhyphen\null
      \SOUL@everyspace{\nobreak}%
    \else\ifx\SOUL@@\slash
      \SOUL@doword
      \SOUL@eventuallyexhyphen{/}%
      \SOUL@exhyphen{/}%
    \else\ifx\SOUL@@\mbox
      \def\SOUL@n*{\SOUL@addprotect}%
    \else\ifx\SOUL@@\hbox
      \def\SOUL@n*{\SOUL@addprotect}%
    \else\ifx\SOUL@@\soulomit
      \def\SOUL@n*\soulomit##1{%
        \SOUL@doword
        {\spaceskip\SOUL@spaceskip##1}%
        \SOUL@scan
      }%
    \else\ifx\SOUL@@\break
      \SOUL@doword
      \break
    \else\ifx\SOUL@@\linebreak
      \SOUL@doword
      \SOUL@everyspace{\linebreak}%
    \else\ifcat\bgroup\noexpand\SOUL@@
      \def\SOUL@n*{\SOUL@addgroup{}}%
    \else\ifcat$\noexpand\SOUL@@
      \def\SOUL@n*{\SOUL@addmath}%
    \else
%    \end{macrocode}
%    The current token is examined to detect the start of
%    a UTF-8 sequence.
%    \begin{macrocode}
      \SOuL@analyzeutfviii
      \ifcase\SOuL@octets
        \SOuL@analyzeutfviiix
      \fi
      \ifcase\SOuL@octets
        \def\SOUL@n*{\SOUL@dotoken}%
      \or % 1
      \or % 2
        \def\SOUL@n*{\SOuL@addtwooctets}%
      \or % 3
        \def\SOUL@n*{\SOuL@addthreeoctets}%
      \or % 4
        \def\SOUL@n*{\SOuL@addfouroctets}%
      \fi
    \fi\fi\fi\fi\fi\fi\fi\fi\fi\fi\fi\fi\fi
  \fi\fi\fi\fi
  \SOUL@n*%
}
%    \end{macrocode}
%    \end{macro}
%
% \subsection{UTF-8 analysis}
%
% \subsubsection{Help strings}
%
%    \begin{macrocode}
\def\SOuL@defsanitizedstring#1#2{%
  \expandafter\def\csname SOuL@string#1\endcsname{#2}%
  \expandafter\@onelevel@sanitize\csname SOuL@string#1\endcsname
}
\SOuL@defsanitizedstring{UTFviii}{UTFviii@}
\SOuL@defsanitizedstring{octets}{@octets}
\SOuL@defsanitizedstring{two}{two}
\SOuL@defsanitizedstring{three}{three}
\SOuL@defsanitizedstring{four}{four}
\SOuL@defsanitizedstring{macrocolon}{macro:}
\SOuL@defsanitizedstring{csnameu}{csname u8-}
\SOuL@defsanitizedstring{undeferr}{utf@viii@undeferr}
\def\SOuL@stringendash{^^e2^^80^^93}
\def\SOuL@stringemdash{^^e2^^80^^94}
\def\SOuL@stringnobreakspace{^^c2^^a0}
\edef\SOuL@charhash{\string #}
\edef\SOuL@chartwo{\string 2}
\edef\SOuL@charthree{\string 3}
\def\SOuL@empty{}
%    \end{macrocode}
%
% \subsubsection{Support for \xfile{utf8.def}}
%
%    \begin{macro}{\SOuL@analyzeutfviii}
%    \begin{macrocode}
\begingroup
  \edef\x{\endgroup
    \def\noexpand\SOuL@analyzeutfviii{%
      \noexpand\expandafter\noexpand\SOuL@checkutfviii
      \noexpand\meaning\noexpand\SOUL@@
      \SOuL@stringUTFviii\SOuL@stringoctets
      \noexpand\@nil
    }%
    \def\noexpand\SOuL@checkutfviii
      ##1\SOuL@stringUTFviii##2\SOuL@stringoctets##3\noexpand\@nil
  }%
\x{%
  \def\SOuL@temp{#2}%
  \chardef\SOuL@octets=%
      \ifx\SOuL@temp\SOuL@stringtwo
        \tw@
      \else\ifx\SOuL@temp\SOuL@stringthree
        \thr@@
      \else\ifx\SOuL@temp\SOuL@stringfour
        4 %
      \else
        \z@
      \fi\fi\fi
}
%    \end{macrocode}
%    \end{macro}
%
% \subsubsection{Support for \xfile{utf8x.def}}
%
%    \begin{macro}{\SOuL@analyzeutfviiix}
%    \begin{macrocode}
\begingroup
  \edef\x{\endgroup
    \def\noexpand\SOuL@analyzeutfviiix{%
      \noexpand\expandafter\noexpand\SOuL@checkutfviiix
      \noexpand\meaning\noexpand\SOUL@@
      \SOuL@stringmacrocolon\SOuL@charhash1{}{}{}{}%
      \SOuL@stringcsnameu\SOuL@stringundeferr
      \noexpand\@nil
    }%
%    \end{macrocode}
%    \end{macro}
%    \begin{macro}{\SOuL@checkutfviiix}
%    \begin{macrocode}
    \def\noexpand\SOuL@checkutfviiix
      ##1\SOuL@stringmacrocolon\SOuL@charhash1##2##3##4##5##6%
      \SOuL@stringcsnameu##7\SOuL@stringundeferr##8\noexpand\@nil
  }%
\x{%
  \def\SOuL@temp{#7}%
  \ifx\SOuL@temp\SOuL@empty
    \chardef\SOuL@octets=\z@
  \else
    \def\SOuL@temp{#5}%
    \ifx\SOuL@temp\SOuL@charthree
      \chardef\SOuL@octets=4 %
    \else
      \def\SOuL@temp{#3}%
      \ifx\SOuL@temp\SOuL@chartwo
        \chardef\SOuL@octets=\thr@@
      \else
        \chardef\SOuL@octets=\tw@
      \fi
    \fi
  \fi
}
%    \end{macrocode}
%    \end{macro}
%
% \subsection{Actions for UTF-8 sequences}
%
%    \begin{macro}{\SOuL@addtwooctets}
%    \begin{macrocode}
\def\SOuL@addtwooctets#1#2{%
  \def\SOuL@temp{#1#2}%
  \@onelevel@sanitize\SOuL@temp
  \ifx\SOuL@temp\SOuL@stringnobreakspace
    \SOUL@doword
    \SOUL@eventuallyexhyphen\null
    \SOUL@everyspace{\nobreak}%
    \let\SOuL@next\SOUL@scan
  \else
    \def\SOuL@next{%
!     \SOUL@addtoken{{\noexpand#1\noexpand#2}}%
+     \SOUL@addtoken{{\etex@unexpanded{#1#2}}}%
    }%
  \fi
  \SOuL@next
}
%    \end{macrocode}
%    \end{macro}
%    \begin{macro}{\SOuL@addthreeoctets}
%    \begin{macrocode}
\def\SOuL@addthreeoctets#1#2#3{%
  \def\SOuL@temp{#1#2#3}%
  \@onelevel@sanitize\SOuL@temp
  \ifx\SOuL@temp\SOuL@stringendash
    \SOUL@doword
    \SOUL@eventuallyexhyphen{-}%
    \SOUL@exhyphen{--}%
    \let\SOuL@next\SOUL@scan
  \else
    \ifx\SOuL@temp\SOuL@stringemdash
      \SOUL@doword
      \SOUL@eventuallyexhyphen{-}%
      \SOUL@exhyphen{---}%
      \let\SOuL@next\SOUL@scan
    \else
      \def\SOuL@next{%
!       \SOUL@addtoken{{\noexpand#1\noexpand#2\noexpand#3}}%
+       \SOUL@addtoken{{\etex@unexpanded{#1#2#3}}}%
      }%
    \fi
  \fi
  \SOuL@next
}
%    \end{macrocode}
%    \end{macro}
%    \begin{macro}{\SOuL@addfouroctets}
%    \begin{macrocode}
\def\SOuL@addfouroctets#1#2#3#4{%
! \SOUL@addtoken{{\noexpand#1\noexpand#2\noexpand#3\noexpand#4}}%
+ \SOUL@addtoken{{\etex@unexpanded{#1#2#3#4}}}%
}
%    \end{macrocode}
%    \end{macro}
%
% \subsubsection{Redefinition of \cs{SOUL@splittoken}}
%
%    \begin{macro}{\SOUL@splittoken}
%    Macro \cs{SOUL@splittoken} separates the first token
%    or token group from a word and redefines the word to contain
%    the remaining tokens. However if the remaining tokens are
%    a token group, then the curly braces will be removed and
%    the token group is splitted by the next call of \cs{SOUL@splittoken}.
%    The redefinition avoids the removal of curly braces around the
%    remaining tokens.
%    \begin{macrocode}
\SOuL@redefine\SOUL@splittoken#1#2\SOUL@stop{%
  \global\SOUL@token={#1}%
  \global\SOUL@word={#2}%
}#1{%
  \global\SOUL@token={#1}%
  \SOuL@remainingtoken\relax
}
%    \end{macrocode}
%    \end{macro}
%    \begin{macro}{\SOuL@remainingtoken}
%    \begin{macrocode}
\def\SOuL@remainingtoken#1\SOUL@stop{%
  \global\SOUL@word=\expandafter{\@gobble#1}%
}
%    \end{macrocode}
%    \end{macro}
%
% \subsection{Patches}
%
%    The fixed \cs{SOUL@splittoken} allows to remove the double
%    sets of curly braces in other macros of package \xpackage{soul}.
%    The benefit is that implicite kernings are more often detected
%    and fixes a bug in package \xpackage{soul}. The disadvantage is
%    incompatibility. The width of the resulting strings may change.
%    \begin{macro}{\SOUL@flushcomma}
%    \begin{macrocode}
\SOuL@redefine\SOUL@flushcomma{%
  \ifcase\SOUL@comma
  \or
    \edef\x{\SOUL@word={\the\SOUL@word,}}\x
  \or
    \edef\x{\SOUL@word={\the\SOUL@word{{,,}}}}\x
  \fi
  \SOUL@comma\z@
}{%
  \ifcase\SOUL@comma
  \or
    \edef\x{\SOUL@word={\the\SOUL@word,}}\x
  \or
    \edef\x{\SOUL@word={\the\SOUL@word{,,}}}\x
  \fi
  \SOUL@comma\z@
}
%    \end{macrocode}
%    \end{macro}
%    \begin{macro}{\SOUL@flushapo}
%    \begin{macrocode}
\SOuL@redefine\SOUL@flushapo{%
  \ifcase\SOUL@apo
  \or
    \edef\x{\SOUL@word={\the\SOUL@word'}}\x
  \or
    \edef\x{\SOUL@word={\the\SOUL@word{{''}}}}\x
  \fi
  \SOUL@apo\z@
}{%
  \ifcase\SOUL@apo
  \or
    \edef\x{\SOUL@word={\the\SOUL@word'}}\x
  \or
    \edef\x{\SOUL@word={\the\SOUL@word{''}}}\x
  \fi
  \SOUL@apo\z@
}
%    \end{macrocode}
%    \end{macro}
%    \begin{macro}{\SOUL@flushgrave}
%    \begin{macrocode}
\SOuL@redefine\SOUL@flushgrave{%
  \ifcase\SOUL@grave
  \or
    \edef\x{\SOUL@word={\the\SOUL@word`}}\x
  \or
    \edef\x{\SOUL@word={\the\SOUL@word{{``}}}}\x
  \fi
  \SOUL@grave\z@
}{%
  \ifcase\SOUL@grave
  \or
    \edef\x{\SOUL@word={\the\SOUL@word`}}\x
  \or
    \edef\x{\SOUL@word={\the\SOUL@word{``}}}\x
  \fi
  \SOUL@grave\z@
}
%    \end{macrocode}
%    \end{macro}
%    \begin{macro}{\SOUL@addgroup}
%    \begin{macrocode}
\SOuL@redefine\SOUL@addgroup#1#2{%
  {%
    \let\protect\noexpand
    \edef\x{%
      \global\SOUL@word={%
        \the\SOUL@word
        {{\noexpand#1#2}}%
      }%
    }%
    \x
  }%
  \SOUL@scan
}#1#2{%
  \begingroup
    \let\protect\noexpand
    \edef\x{\endgroup
      \SOUL@word={%
        \the\SOUL@word
!       {\noexpand#1{#2}}%
+       {\etex@unexpanded{#1{#2}}}%
      }%
    }%
  \x
  \SOUL@scan
}
%    \end{macrocode}
%    \end{macro}
%    \begin{macro}{\SOUL@addmath}
%    \begin{macrocode}
\SOuL@redefine\SOUL@addmath$#1${%
  {%
    \let\protect\noexpand
    \edef\x{%
      \global\SOUL@word={%
        \the\SOUL@word
        {{\hbox{$#1$}}}%
      }%
    }%
    \x
  }%
  \SOUL@scan
}$#1${%
  \begingroup
    \let\protect\noexpand
    \edef\x{\endgroup
      \SOUL@word={%
        \the\SOUL@word
!       {\hbox{$#1$}}%
+       {\etex@unexpanded{\hbox{$#1$}}}%
      }%
    }%
  \x
  \SOUL@scan
}
%    \end{macrocode}
%    \end{macro}
%    \begin{macro}{\SOUL@addprotect}
%    \begin{macrocode}
\SOuL@redefine\SOUL@addprotect#1#2{%
  {%
    \let\protect\noexpand
    \edef\x{%
      \global\SOUL@word={%
        \the\SOUL@word
        {{\hbox{#2}}}%
      }%
    }%
    \x
  }%
  \SOUL@scan
}#1#2{%
  \begingroup
    \let\protect\noexpand
    \edef\x{\endgroup
      \SOUL@word={%
        \the\SOUL@word
!       {\hbox{#2}}%
+       {\etex@unexpanded{\hbox{#2}}}%
      }%
    }%
  \x
  \SOUL@scan
}
%    \end{macrocode}
%    \end{macro}
%
%    \begin{macro}{\SOUL@addtoken}
%    \begin{macrocode}
+ \SOuL@redefine\SOUL@addtoken#1{%
+   \edef\x{%
+     \SOUL@word={%
+       \the\SOUL@word
+       \noexpand#1%
+     }%
+   }%
+   \x
+   \SOUL@scan
+ }#1{%
+   \edef\x{%
+     \SOUL@word={%
+       \the\SOUL@word
+       \etex@unexpanded{#1}%
+     }%
+   }%
+   \x
+   \SOUL@scan
+ }%
%    \end{macrocode}
%    \end{macro}
%
%    \begin{macrocode}
\SOuL@AtEnd%
%    \end{macrocode}
%
%    \begin{macrocode}
%</package>
%    \end{macrocode}
%
% \section{Test}
%
% \subsection{Catcode checks for loading}
%
%    \begin{macrocode}
%<*test1>
%    \end{macrocode}
%    \begin{macrocode}
\catcode`\{=1 %
\catcode`\}=2 %
\catcode`\#=6 %
\catcode`\@=11 %
\expandafter\ifx\csname count@\endcsname\relax
  \countdef\count@=255 %
\fi
\expandafter\ifx\csname @gobble\endcsname\relax
  \long\def\@gobble#1{}%
\fi
\expandafter\ifx\csname @firstofone\endcsname\relax
  \long\def\@firstofone#1{#1}%
\fi
\expandafter\ifx\csname loop\endcsname\relax
  \expandafter\@firstofone
\else
  \expandafter\@gobble
\fi
{%
  \def\loop#1\repeat{%
    \def\body{#1}%
    \iterate
  }%
  \def\iterate{%
    \body
      \let\next\iterate
    \else
      \let\next\relax
    \fi
    \next
  }%
  \let\repeat=\fi
}%
\def\RestoreCatcodes{}
\count@=0 %
\loop
  \edef\RestoreCatcodes{%
    \RestoreCatcodes
    \catcode\the\count@=\the\catcode\count@\relax
  }%
\ifnum\count@<255 %
  \advance\count@ 1 %
\repeat

\def\RangeCatcodeInvalid#1#2{%
  \count@=#1\relax
  \loop
    \catcode\count@=15 %
  \ifnum\count@<#2\relax
    \advance\count@ 1 %
  \repeat
}
\def\RangeCatcodeCheck#1#2#3{%
  \count@=#1\relax
  \loop
    \ifnum#3=\catcode\count@
    \else
      \errmessage{%
        Character \the\count@\space
        with wrong catcode \the\catcode\count@\space
        instead of \number#3%
      }%
    \fi
  \ifnum\count@<#2\relax
    \advance\count@ 1 %
  \repeat
}
\def\space{ }
\expandafter\ifx\csname LoadCommand\endcsname\relax
  \def\LoadCommand{\input soulutf8.sty\relax}%
\fi
\def\Test{%
  \RangeCatcodeInvalid{0}{47}%
  \RangeCatcodeInvalid{58}{64}%
  \RangeCatcodeInvalid{91}{96}%
  \RangeCatcodeInvalid{123}{255}%
  \catcode`\@=12 %
  \catcode`\\=0 %
  \catcode`\%=14 %
  \LoadCommand
  \RangeCatcodeCheck{0}{36}{15}%
  \RangeCatcodeCheck{37}{37}{14}%
  \RangeCatcodeCheck{38}{47}{15}%
  \RangeCatcodeCheck{48}{57}{12}%
  \RangeCatcodeCheck{58}{63}{15}%
  \RangeCatcodeCheck{64}{64}{12}%
  \RangeCatcodeCheck{65}{90}{11}%
  \RangeCatcodeCheck{91}{91}{15}%
  \RangeCatcodeCheck{92}{92}{0}%
  \RangeCatcodeCheck{93}{96}{15}%
  \RangeCatcodeCheck{97}{122}{11}%
  \RangeCatcodeCheck{123}{255}{15}%
  \RestoreCatcodes
}
\Test
\csname @@end\endcsname
\end
%    \end{macrocode}
%    \begin{macrocode}
%</test1>
%    \end{macrocode}
%
% \subsection{\LaTeX\space tests}
%
%    \begin{macrocode}
%<*test2|test3|test4|test5>
%    \end{macrocode}
%    \begin{macrocode}
%<noetex>\let\SavedUnexpanded\unexpanded
%<noetex>\let\unexpanded\UNDEFINED
\NeedsTeXFormat{LaTeX2e}
\documentclass{article}
\usepackage[T1]{fontenc}
\usepackage{lmodern}
\usepackage{textcomp}
\usepackage[ngerman]{babel}
\expandafter\def\csname ver@hypdestopt.sty\endcsname{}
\usepackage[unicode]{hyperref}
\usepackage{bookmark}
\usepackage{color}
\usepackage{soul}[2003/11/17]
\makeatletter
\newcommand*{\@ifDefined}[1]{%
  \@ifundefined{#1}{%
    \@latex@error{%
      \expandafter\noexpand\csname#1\endcsname undefined%
    }\@ehc
  }%
}
\newcommand*{\SaveOrg}[1]{%
  \@ifdefinable{ORG@SOUL@#1}{%
    \@ifDefined{SOUL@#1}{%
      \expandafter\let\csname ORG@SOUL@#1\expandafter\endcsname
                      \csname SOUL@#1\endcsname
    }%
  }%
}
\newcommand*{\RestoreOrg}[1]{%
  \@ifDefined{ORG@SOUL@#1}{%
    \@ifDefined{SOUL@#1}{%
      \expandafter\let\csname SOUL@#1\expandafter\endcsname
                      \csname ORG@SOUL@#1\endcsname
    }%
  }%
}
\makeatother
\SaveOrg{eval}
\SaveOrg{splittoken}
\SaveOrg{flushcomma}
\SaveOrg{flushapo}
\SaveOrg{grave}
\SaveOrg{addgroup}
\SaveOrg{addmath}
\SaveOrg{addprotect}
\SaveOrg{addtoken}
\usepackage{soulutf8}[2016/05/16]
%    \end{macrocode}
%    \begin{macrocode}
%<noetex>\let\unexpanded\SavedUnexpanded
%    \end{macrocode}
%    Input encoding \xfile{utf8}.
%    \begin{macrocode}
%<*utf8>
\usepackage[utf8]{inputenc}
\DeclareUnicodeCharacter{00A0}{\nobreakspace}
%</utf8>
%    \end{macrocode}
%    Input encoding \xfile{utf8x}. Package \xpackage{ucs}
%    has a nasty bug, that it redefines \cs{show}.
%    \begin{macrocode}
%<*utf8x>
\let\origshow\show
\usepackage[utf8x]{inputenc}
\AtBeginDocument{%
  \let\show\origshow
}
%</utf8x>
%    \end{macrocode}
%
% \subsubsection{Driver definitions}
%
%    \begin{macrocode}
\makeatletter
%    \end{macrocode}
%    \begin{macro}{\an}
%    The definition for \cs{an} is taken from the documentation of
%    package \xpackage{soul}.
%    \begin{macrocode}
\DeclareRobustCommand*{\an}{%
  \def\SOUL@preamble{$^{^P}$}%
  \def\SOUL@everyspace##1{##1\texttt{\char`\ }\linebreak[2]}%
  \def\SOUL@postamble{$^{^E}$}%
  \def\SOUL@everyhyphen{$^{^-}$\linebreak[1]}%
  \def\SOUL@everyexhyphen##1{##1$^{^=}$}%
  \def\SOUL@everysyllable{$^{^S}$}%
  \def\SOUL@everytoken{\the\SOUL@token$^{^T}$}%
  \def\SOUL@everylowerthan{$^{^L}$}%
  \SOUL@
}
%    \end{macrocode}
%    \end{macro}
%
%    \begin{macro}{\sy}
%    The definition for \cs{sy} is taken from the documentation of
%    package \xpackage{soul}.
%    \begin{macrocode}
\DeclareRobustCommand*{\sy}{%
  \SOUL@setup
  \def\SOUL@preamble{%
    \lefthyphenmin=\z@
    \righthyphenmin=\z@
  }%
  \def\SOUL@everyspace##1{##1\space}%
  \def\SOUL@everysyllable{\the\SOUL@syllable}%
  \def\SOUL@everyhyphen{%
    \discretionary{%
      \SOUL@setkern\SOUL@hyphkern
      \SOUL@sethyphenchar
    }{}{%
      \hbox{\kern1pt$\cdot$}%
    }%
  }%
  \def\SOUL@everyexhyphen##1{%
    \SOUL@setkern\SOUL@hyphkern
    \hbox{##1}%
    \discretionary{}{}{%
      \SOUL@setkern\SOUL@charkern
    }%
  }%
  \SOUL@
}
%    \end{macrocode}
%    \end{macro}
%
%    \begin{macro}{\soulH}
%    \begin{macrocode}
\DeclareRobustCommand*{\soulH}{%
  \def\SOUL@preamble{%
    \gdef\SOULH{}%
    \soulHadd{[}%
  }%
  \def\SOUL@everyspace##1{%
    \soulHadd{Sp}%
  }%
  \def\SOUL@postamble{%
    \soulHadd{]}%
  }%
  \def\SOUL@everyhyphen{%
    \soulHadd{Hy}%
  }%
  \def\SOUL@everyexhyphen##1{%
    \soulHadd{Exhy}%
  }%
  \def\SOUL@everysyllable{%
    \soulHadd{Sy}%
  }%
  \def\SOUL@everytoken{%
    \soulHadd{T}%
  }%
  \def\SOUL@everylowerthan{%
    \soulHadd{<}%
  }%
  \SOUL@
}
\newcommand*{\soulHadd}[1]{%
  #1%
  \expandafter\gdef\expandafter\soulHstring\expandafter{%
    \soulHstring
    #1%
  }%
}
%    \end{macrocode}
%    \end{macro}
%
% \subsubsection{Test macros}
%
%    \begin{macrocode}
\usepackage{qstest}
\IncludeTests{*}
\LogTests{log}{*}{*}
%    \end{macrocode}
%    \begin{macrocode}
\newsavebox\boxA
\newsavebox\boxB
\newsavebox\boxC
%    \end{macrocode}
%    \begin{macro}{\TestDriver}
%    \begin{macrocode}
\newcommand*{\TestDriver}[4]{%
  \subsection{Driver #1}%
  \@PackageInfo{soulutf8-test}{* Driver #1}%
  \expandafter\let\expandafter\testdriver\csname #1\endcsname
  \def\teststringA{#3}%
  \def\teststringB{#4}%
  \begin{qstest}{string-#2-driver-#1}{string,#2,driver,#1}%
    \@PackageInfo{soulutf8-test}{* Version: soulutf8}%
    \global\let\soulHstring\@empty
    \setbox0=\hbox{\expandafter\testdriver\expandafter{\teststringA}}%
    \let\HstringA\soulHstring
    \SetTestVBox{\boxA}{\teststringA}%
    %
    \@PackageInfo{soulutf8-test}{* Version: soul + patch}%
    \global\let\soulHstring\@empty
    \RestoreOrg{eval}%
    \setbox2=\hbox{\expandafter\testdriver\expandafter{\teststringB}}%
    \let\HstringB\soulHstring
    \SetTestVBox{\boxB}{\teststringB}%
    %
    \@PackageInfo{soulutf8-test}{* Version: original soul}%
    \global\let\soulHstring\@empty
    \RestoreOrg{splittoken}%
    \RestoreOrg{flushcomma}%
    \RestoreOrg{flushapo}%
    \RestoreOrg{grave}%
    \RestoreOrg{addgroup}%
    \RestoreOrg{addmath}%
    \RestoreOrg{addprotect}%
    \RestoreOrg{addtoken}%
    \setbox4=\hbox{\expandafter\testdriver\expandafter{\teststringB}}%
    \let\HstringC\soulHstring
    \SetTestVBox{\boxC}{\teststringB}%
    %
    \TestInfo{0}{soulutf8}%
    \TestInfo{2}{soul + patch}%
    \TestInfo{4}{original soul}%
    \TestInfoVBox
    \showboxdepth=\maxdimen
    \showboxbreadth=\maxdimen
    \tracingonline=\@ne
    \Expect*{\HstringA}*{\HstringB}%
    \Expect*{\HstringA}*{\HstringC}%
    \Expect*{\the\wd0}*{\the\wd2}%
    \ifdim\wd0=\wd2 %
    \else
      \showbox0 %
      \showbox2 %
    \fi
    \Expect*{\the\ht0}*{\the\ht2}%
    \Expect*{\the\dp0}*{\the\dp2}%
    \Expect*{\the\wd\boxB}*{\the\wd\boxA}%
    \Expect*{\the\wd\boxC}*{\the\wd\boxA}%
    \Expect*{\the\ht\boxB}*{\the\ht\boxA}%
    \ifdim\ht\boxB=\ht\boxA
    \else
      \showbox\boxA
      \showbox\boxB
    \fi
    \Expect*{\the\ht\boxC}*{\the\ht\boxA}%
    \Expect*{\the\dp\boxB}*{\the\dp\boxA}%
    \Expect*{\the\dp\boxC}*{\the\dp\boxA}%
  \end{qstest}%
}
%    \end{macrocode}
%    \end{macro}
%    \begin{macro}{\SetTestVBox}
%    \begin{macrocode}
\newcommand*{\SetTestVBox}[2]{%
  \setbox#1=\vbox{%
    \hsize=0pt\relax
    \hfuzz=100mm\relax
    \expandafter\testdriver\expandafter{#2}%
  }%
}
%    \end{macrocode}
%    \end{macro}
%    \begin{macro}{\TestInfo}
%    \begin{macrocode}
\newcommand*{\TestInfo}[2]{%
  \par
  \noindent
  \copy#1\relax
  \hfill
  \begingroup
    \small
    \ifdim\wd0=\wd2\relax
      \ifdim\wd0=\wd4\relax
      \else
        \ifnum#1=2 %
          \raisebox{.5\baselineskip}[0pt][0pt]{%
            \llap{$\}$ }%
            \rlap{width = \the\wd#1\relax\space$\{$}%
            \phantom{width = 999.99999pt}%
            \qquad
          }%
        \else
          \ifnum#1=4 %
            \rlap{width = \the\wd#1}%
            \phantom{width = 999.99999pt}%
            \qquad
          \fi
        \fi
      \fi
    \else
      \begingroup
        \bfseries
        \rlap{width = \the\wd#1}%
        \phantom{width = 999.99999pt}%
      \endgroup
      \qquad
    \fi
    \rlap{(#2)}%
    \phantom{(original soul)}%
  \endgroup
  \par
}
%    \end{macrocode}
%    \end{macro}
%    \begin{macro}{\TestInfoVBox}
%    \begin{macrocode}
\newcommand*{\TestInfoVBox}{%
  \par
  \noindent
  \mbox{\vrule\relax\space\copy\boxA}%
  \ifdim\ht\boxA=\ht\boxB
  \else
    \hfill
    \begingroup
      \small
      \rlap{(soulutf8)}%
      \phantom{(soulutf8)}%
    \endgroup
    \par
    \noindent
    \mbox{\vrule\relax\space\copy\boxB}%
    \hfill
    \begingroup
      \small
      \rlap{(soul)}%
      \phantom{(soulutf8)}%
    \endgroup
  \fi
  \par
}
%    \end{macrocode}
%    \end{macro}
%    \begin{macro}{\Test}
%    \begin{macrocode}
\makeatletter
\newcommand*{\Test}[3]{%
  \clearpage
  \begingroup
    \def\linebreak{\textbackslash linebreak }%
    \def\\{\textbackslash\textbackslash}%
    \def\soulomit##1{%
      \textbackslash soulomit\textbraceleft##1\textbraceright
    }%
    \def\mbox##1{%
      \textbackslash mbox\textbraceleft##1\textbraceright
    }%
    \def\abc{\textbackslash abc }%
    \def\texttt##1{%
      \textbackslash texttt\textbraceleft##1\textbraceright
    }%
    \def\c##1{%
      \textbackslash c\textbraceleft##1\textbraceright
    }%
%<*utf8x>
    \count@=128 %
    \loop
    \ifnum\count@<256 %
      \lccode`\~=\count@
      \lowercase{%
        \def~{\noexpand~}%
      }%
      \advance\count@\@ne
    \repeat
%</utf8x>
    \csname protected@edef\endcsname\x{\endgroup
      \noexpand\section{String #1: #2}%
    }%
  \x
  \TestDriver{so}{#1}{#2}{#3}%
  \TestDriver{caps}{#1}{#2}{#3}%
  \TestDriver{ul}{#1}{#2}{#3}%
  \TestDriver{st}{#1}{#2}{#3}%
  \TestDriver{hl}{#1}{#2}{#3}%
  \TestDriver{an}{#1}{#2}{#3}%
  \TestDriver{sy}{#1}{#2}{#3}%
  \TestDriver{soulH}{#1}{#2}{#3}%
}
\makeatother
%    \end{macrocode}
%    \end{macro}
%
% \subsubsection{Testing}
%
%    \begin{macrocode}
\begin{document}
  \Test A{}{}%
  \Test B{Hello World}{Hello World}%
  \Test C{%
    Umlauts: ^^c3^^84^^c3^^96^^c3^^9c^^c3^^a4^^c3^^b6^^c3^^bc^^c3^^9f%
  }{%
    Umlauts: \"A\"O\"U\"a\"o\"u\ss
  }%
  \Test D{Euro: ^^e2^^82^^ac}{Euro: \texteuro}%
  \Test E{Stra^^c3^^9fe}{Stra\ss e}%
  \Test F{R^^c3^^b6mer}{R\"omer}%
  \Test G{A^^e2^^80^^93B^^e2^^80^^94C}{A--B---C}%
  \Test H{X^^c2^^a0Y}{X~Y}%
  \Test I{A^^c5^^a4}{A\v T}%
  \Test J{A,B,,C,,,D,,,,E}{A,B,,C,,,D,,,,E}%
  \Test K{A'B''C'''D''''E}{A'B''C'''D''''E}%
  \Test L{A`B``C```D````E}{A`B``C```D````E}%
  \Test M{A$x$B}{A$x$B}%
  \Test N{A\mbox{abc}B}{A\mbox{abc}B}%
  \Test O{,,TA''}{,,TA''}%
  \Test P{{!`}Hola{?`}}{{!`}Hola{?`}}%
  \Test Q{re-sent\slash x}{re-sent\slash x}%
  \Test R{new\\line}{new\\line}%
  \Test S{foo\linebreak bar}{foo\linebreak bar}%
  \Test T{Virchow{sche}}{Virchow{sche}}%
  \Test U{A\soulomit{foo}B}{A\soulomit{foo}B}%
  \Test V{A\texttt{xyz}B}{A\texttt{xyz}B}%
  \Test W{ffi}{ffi}%
  \Test X{f{}f{}i}{f{}f{}i}%
  \Test Y{f\mbox{}f\mbox{}i}{f\mbox{}f\mbox{}i}%
  \Test Z{\c{\"a}}{{\c{\"a}}}%
%<*test2|test3>
  \let\abcx\UNDEFINED
  \def\abc{%
    \begingroup
      \def\abcx{x}%
      \abcx
    \endgroup
  }%
  \Test{AA}{[{x\abc x}]}{[{xxx}]}%
%</test2|test3>
%    \end{macrocode}
%
%    \begin{macrocode}
\end{document}
%    \end{macrocode}
%    \begin{macrocode}
%</test2|test3|test4|test5>
%    \end{macrocode}
%
% \section{Installation}
%
% \subsection{Download}
%
% \paragraph{Package.} This package is available on
% CTAN\footnote{\url{http://ctan.org/pkg/soulutf8}}:
% \begin{description}
% \item[\CTAN{macros/latex/contrib/oberdiek/soulutf8.dtx}] The source file.
% \item[\CTAN{macros/latex/contrib/oberdiek/soulutf8.pdf}] Documentation.
% \end{description}
%
%
% \paragraph{Bundle.} All the packages of the bundle `oberdiek'
% are also available in a TDS compliant ZIP archive. There
% the packages are already unpacked and the documentation files
% are generated. The files and directories obey the TDS standard.
% \begin{description}
% \item[\CTAN{install/macros/latex/contrib/oberdiek.tds.zip}]
% \end{description}
% \emph{TDS} refers to the standard ``A Directory Structure
% for \TeX\ Files'' (\CTAN{tds/tds.pdf}). Directories
% with \xfile{texmf} in their name are usually organized this way.
%
% \subsection{Bundle installation}
%
% \paragraph{Unpacking.} Unpack the \xfile{oberdiek.tds.zip} in the
% TDS tree (also known as \xfile{texmf} tree) of your choice.
% Example (linux):
% \begin{quote}
%   |unzip oberdiek.tds.zip -d ~/texmf|
% \end{quote}
%
% \paragraph{Script installation.}
% Check the directory \xfile{TDS:scripts/oberdiek/} for
% scripts that need further installation steps.
% Package \xpackage{attachfile2} comes with the Perl script
% \xfile{pdfatfi.pl} that should be installed in such a way
% that it can be called as \texttt{pdfatfi}.
% Example (linux):
% \begin{quote}
%   |chmod +x scripts/oberdiek/pdfatfi.pl|\\
%   |cp scripts/oberdiek/pdfatfi.pl /usr/local/bin/|
% \end{quote}
%
% \subsection{Package installation}
%
% \paragraph{Unpacking.} The \xfile{.dtx} file is a self-extracting
% \docstrip\ archive. The files are extracted by running the
% \xfile{.dtx} through \plainTeX:
% \begin{quote}
%   \verb|tex soulutf8.dtx|
% \end{quote}
%
% \paragraph{TDS.} Now the different files must be moved into
% the different directories in your installation TDS tree
% (also known as \xfile{texmf} tree):
% \begin{quote}
% \def\t{^^A
% \begin{tabular}{@{}>{\ttfamily}l@{ $\rightarrow$ }>{\ttfamily}l@{}}
%   soulutf8.sty & tex/generic/oberdiek/soulutf8.sty\\
%   soulutf8.pdf & doc/latex/oberdiek/soulutf8.pdf\\
%   test/soulutf8-test1.tex & doc/latex/oberdiek/test/soulutf8-test1.tex\\
%   test/soulutf8-test2.tex & doc/latex/oberdiek/test/soulutf8-test2.tex\\
%   test/soulutf8-test3.tex & doc/latex/oberdiek/test/soulutf8-test3.tex\\
%   test/soulutf8-test4.tex & doc/latex/oberdiek/test/soulutf8-test4.tex\\
%   test/soulutf8-test5.tex & doc/latex/oberdiek/test/soulutf8-test5.tex\\
%   soulutf8.dtx & source/latex/oberdiek/soulutf8.dtx\\
% \end{tabular}^^A
% }^^A
% \sbox0{\t}^^A
% \ifdim\wd0>\linewidth
%   \begingroup
%     \advance\linewidth by\leftmargin
%     \advance\linewidth by\rightmargin
%   \edef\x{\endgroup
%     \def\noexpand\lw{\the\linewidth}^^A
%   }\x
%   \def\lwbox{^^A
%     \leavevmode
%     \hbox to \linewidth{^^A
%       \kern-\leftmargin\relax
%       \hss
%       \usebox0
%       \hss
%       \kern-\rightmargin\relax
%     }^^A
%   }^^A
%   \ifdim\wd0>\lw
%     \sbox0{\small\t}^^A
%     \ifdim\wd0>\linewidth
%       \ifdim\wd0>\lw
%         \sbox0{\footnotesize\t}^^A
%         \ifdim\wd0>\linewidth
%           \ifdim\wd0>\lw
%             \sbox0{\scriptsize\t}^^A
%             \ifdim\wd0>\linewidth
%               \ifdim\wd0>\lw
%                 \sbox0{\tiny\t}^^A
%                 \ifdim\wd0>\linewidth
%                   \lwbox
%                 \else
%                   \usebox0
%                 \fi
%               \else
%                 \lwbox
%               \fi
%             \else
%               \usebox0
%             \fi
%           \else
%             \lwbox
%           \fi
%         \else
%           \usebox0
%         \fi
%       \else
%         \lwbox
%       \fi
%     \else
%       \usebox0
%     \fi
%   \else
%     \lwbox
%   \fi
% \else
%   \usebox0
% \fi
% \end{quote}
% If you have a \xfile{docstrip.cfg} that configures and enables \docstrip's
% TDS installing feature, then some files can already be in the right
% place, see the documentation of \docstrip.
%
% \subsection{Refresh file name databases}
%
% If your \TeX~distribution
% (\teTeX, \mikTeX, \dots) relies on file name databases, you must refresh
% these. For example, \teTeX\ users run \verb|texhash| or
% \verb|mktexlsr|.
%
% \subsection{Some details for the interested}
%
% \paragraph{Attached source.}
%
% The PDF documentation on CTAN also includes the
% \xfile{.dtx} source file. It can be extracted by
% AcrobatReader 6 or higher. Another option is \textsf{pdftk},
% e.g. unpack the file into the current directory:
% \begin{quote}
%   \verb|pdftk soulutf8.pdf unpack_files output .|
% \end{quote}
%
% \paragraph{Unpacking with \LaTeX.}
% The \xfile{.dtx} chooses its action depending on the format:
% \begin{description}
% \item[\plainTeX:] Run \docstrip\ and extract the files.
% \item[\LaTeX:] Generate the documentation.
% \end{description}
% If you insist on using \LaTeX\ for \docstrip\ (really,
% \docstrip\ does not need \LaTeX), then inform the autodetect routine
% about your intention:
% \begin{quote}
%   \verb|latex \let\install=y\input{soulutf8.dtx}|
% \end{quote}
% Do not forget to quote the argument according to the demands
% of your shell.
%
% \paragraph{Generating the documentation.}
% You can use both the \xfile{.dtx} or the \xfile{.drv} to generate
% the documentation. The process can be configured by the
% configuration file \xfile{ltxdoc.cfg}. For instance, put this
% line into this file, if you want to have A4 as paper format:
% \begin{quote}
%   \verb|\PassOptionsToClass{a4paper}{article}|
% \end{quote}
% An example follows how to generate the
% documentation with pdf\LaTeX:
% \begin{quote}
%\begin{verbatim}
%pdflatex soulutf8.dtx
%makeindex -s gind.ist soulutf8.idx
%pdflatex soulutf8.dtx
%makeindex -s gind.ist soulutf8.idx
%pdflatex soulutf8.dtx
%\end{verbatim}
% \end{quote}
%
% \begin{thebibliography}{9}
%
% \bibitem{soul}
%   Melchior Franz: \textit{The \xpackage{soul} package};
%   2003/11/17;\\
%   \CTAN{macros/latex/contrib/soul/}.
%
% \bibitem{ucs}
%   Dominique P. G. Unruh: \textit{\xpackage{ucs.sty} -- Unicode Support};
%   2004/10/17;\\
%   \CTAN{macros/latex/contrib/unicode/}.
%
% \bibitem{utf8}
%   Frank Mittelbach, Chris Rowley:
%   \textit{Providing some UTF-8 support via \xpackage{inputenc}};
%   2006/03/30;\\
%   \CTAN{macros/latex/base/utf8ienc.dtx}.
%
% \end{thebibliography}
%
% \begin{History}
%   \begin{Version}{2007/09/09 v1.0}
%   \item
%     First version.
%   \end{Version}
%   \begin{Version}{2016/05/16 v1.1}
%   \item
%     Documentation updates.
%   \end{Version}
% \end{History}
%
% \PrintIndex
%
% \Finale
\endinput

%        (quote the arguments according to the demands of your shell)
%
% Documentation:
%    (a) If soulutf8.drv is present:
%           latex soulutf8.drv
%    (b) Without soulutf8.drv:
%           latex soulutf8.dtx; ...
%    The class ltxdoc loads the configuration file ltxdoc.cfg
%    if available. Here you can specify further options, e.g.
%    use A4 as paper format:
%       \PassOptionsToClass{a4paper}{article}
%
%    Programm calls to get the documentation (example):
%       pdflatex soulutf8.dtx
%       makeindex -s gind.ist soulutf8.idx
%       pdflatex soulutf8.dtx
%       makeindex -s gind.ist soulutf8.idx
%       pdflatex soulutf8.dtx
%
% Installation:
%    TDS:tex/generic/oberdiek/soulutf8.sty
%    TDS:doc/latex/oberdiek/soulutf8.pdf
%    TDS:doc/latex/oberdiek/test/soulutf8-test1.tex
%    TDS:doc/latex/oberdiek/test/soulutf8-test2.tex
%    TDS:doc/latex/oberdiek/test/soulutf8-test3.tex
%    TDS:doc/latex/oberdiek/test/soulutf8-test4.tex
%    TDS:doc/latex/oberdiek/test/soulutf8-test5.tex
%    TDS:source/latex/oberdiek/soulutf8.dtx
%
%<*ignore>
\begingroup
  \catcode123=1 %
  \catcode125=2 %
  \def\x{LaTeX2e}%
\expandafter\endgroup
\ifcase 0\ifx\install y1\fi\expandafter
         \ifx\csname processbatchFile\endcsname\relax\else1\fi
         \ifx\fmtname\x\else 1\fi\relax
\else\csname fi\endcsname
%</ignore>
%<*install>
\input docstrip.tex
\Msg{************************************************************************}
\Msg{* Installation}
\Msg{* Package: soulutf8 2016/05/16 v1.1 Permit use of UTF-8 characters in soul (HO)}
\Msg{************************************************************************}

\keepsilent
\askforoverwritefalse

\let\MetaPrefix\relax
\preamble

This is a generated file.

Project: soulutf8
Version: 2016/05/16 v1.1

Copyright (C) 2007 by
   Heiko Oberdiek <heiko.oberdiek at googlemail.com>

This work may be distributed and/or modified under the
conditions of the LaTeX Project Public License, either
version 1.3c of this license or (at your option) any later
version. This version of this license is in
   http://www.latex-project.org/lppl/lppl-1-3c.txt
and the latest version of this license is in
   http://www.latex-project.org/lppl.txt
and version 1.3 or later is part of all distributions of
LaTeX version 2005/12/01 or later.

This work has the LPPL maintenance status "maintained".

This Current Maintainer of this work is Heiko Oberdiek.

The Base Interpreter refers to any `TeX-Format',
because some files are installed in TDS:tex/generic//.

This work consists of the main source file soulutf8.dtx
and the derived files
   soulutf8.sty, soulutf8.pdf, soulutf8.ins, soulutf8.drv,
   soulutf8-test1.tex, soulutf8-test2.tex, soulutf8-test3.tex,
   soulutf8-test4.tex, soulutf8-test5.tex.

\endpreamble
\let\MetaPrefix\DoubleperCent

\generate{%
  \file{soulutf8.ins}{\from{soulutf8.dtx}{install}}%
  \file{soulutf8.drv}{\from{soulutf8.dtx}{driver}}%
  \usedir{tex/generic/oberdiek}%
  \file{soulutf8.sty}{\from{soulutf8.dtx}{package}}%
%  \usedir{doc/latex/oberdiek/test}%
%  \file{soulutf8-test1.tex}{\from{soulutf8.dtx}{test1}}%
%  \file{soulutf8-test2.tex}{\from{soulutf8.dtx}{test2,utf8}}%
%  \file{soulutf8-test3.tex}{\from{soulutf8.dtx}{test3,utf8x}}%
%  \file{soulutf8-test4.tex}{\from{soulutf8.dtx}{test4,utf8,noetex}}%
%  \file{soulutf8-test5.tex}{\from{soulutf8.dtx}{test5,utf8x,noetex}}%
  \nopreamble
  \nopostamble
%  \usedir{source/latex/oberdiek/catalogue}%
%  \file{soulutf8.xml}{\from{soulutf8.dtx}{catalogue}}%
}

\catcode32=13\relax% active space
\let =\space%
\Msg{************************************************************************}
\Msg{*}
\Msg{* To finish the installation you have to move the following}
\Msg{* file into a directory searched by TeX:}
\Msg{*}
\Msg{*     soulutf8.sty}
\Msg{*}
\Msg{* To produce the documentation run the file `soulutf8.drv'}
\Msg{* through LaTeX.}
\Msg{*}
\Msg{* Happy TeXing!}
\Msg{*}
\Msg{************************************************************************}

\endbatchfile
%</install>
%<*ignore>
\fi
%</ignore>
%<*driver>
\NeedsTeXFormat{LaTeX2e}
\ProvidesFile{soulutf8.drv}%
  [2016/05/16 v1.1 Permit use of UTF-8 characters in soul (HO)]%
\documentclass{ltxdoc}
\usepackage{holtxdoc}[2011/11/22]
\begin{document}
  \DocInput{soulutf8.dtx}%
\end{document}
%</driver>
% \fi
%
%
% \CharacterTable
%  {Upper-case    \A\B\C\D\E\F\G\H\I\J\K\L\M\N\O\P\Q\R\S\T\U\V\W\X\Y\Z
%   Lower-case    \a\b\c\d\e\f\g\h\i\j\k\l\m\n\o\p\q\r\s\t\u\v\w\x\y\z
%   Digits        \0\1\2\3\4\5\6\7\8\9
%   Exclamation   \!     Double quote  \"     Hash (number) \#
%   Dollar        \$     Percent       \%     Ampersand     \&
%   Acute accent  \'     Left paren    \(     Right paren   \)
%   Asterisk      \*     Plus          \+     Comma         \,
%   Minus         \-     Point         \.     Solidus       \/
%   Colon         \:     Semicolon     \;     Less than     \<
%   Equals        \=     Greater than  \>     Question mark \?
%   Commercial at \@     Left bracket  \[     Backslash     \\
%   Right bracket \]     Circumflex    \^     Underscore    \_
%   Grave accent  \`     Left brace    \{     Vertical bar  \|
%   Right brace   \}     Tilde         \~}
%
% \GetFileInfo{soulutf8.drv}
%
% \title{The \xpackage{soulutf8} package}
% \date{2016/05/16 v1.1}
% \author{Heiko Oberdiek\thanks
% {Please report any issues at https://github.com/ho-tex/oberdiek/issues}\\
% \xemail{heiko.oberdiek at googlemail.com}}
%
% \maketitle
%
% \begin{abstract}
% This package extends package \xpackage{soul} and adds some support
% for UTF-8. Namely the input encodings \xfile{utf8.def}
% from package \xpackage{inputenc} and
% package \xpackage{ucs}'s \xfile{utf8x.def} are supported.
% \end{abstract}
%
% \tableofcontents
%
% \section{Documentation}
%
% This package \xpackage{soulutf8} does not have own options and does not
% define new user commands.
% Any option is passed to package \xpackage{soul} \cite{soul}
% that is loaded first.
% Then some internal macros of \xpackage{soul} are redefined to add
% support for UTF-8.
% The following input encodings are supported:
% \begin{quote}
%   \begin{tabular}{@{}lll@{}}
%     \texttt{utf8} & \LaTeX\ base &
%        \xfile{TDS:tex/latex/base/utf8.def} \cite{utf8}\\
%     \texttt{utf8x} & Package \xpackage{ucs} &
%       \xfile{TDS:tex/latex/ucs/utf8x.def} \cite{ucs}\\
%   \end{tabular}
% \end{quote}
% UTF-8 byte sequences are added as token group to a word, even
% if these UTF-8 characters are some kind of hyphen or space.
% As exception the following three Unicode characters are handled specially:
% \begin{quote}
%   \begin{tabular}{lll}
%     Slot & Name & Action\\
%     \hline
%     \texttt{U+00A0} & NO-BREAK SPACE & like |~|\\
%     \texttt{U+2013} & EN DASH & |--|\\
%     \texttt{U+2014} & EM DASH & |---|
%   \end{tabular}
%
% \subsection{Patch}
%
% Also package \xpackage{soulutf8} tries to patch package \xpackage{soul}
% to improve its behaviour:
% \begin{itemize}
% \item
%   A problem with additional levels of curly braces is fixed.
%   As advantage more implicite kernings are detected.
%   However, the result may be incompatible with the
%   original behaviour of package \xpackage{soul} because
%   of these respected implicite kernings.
% \item
%   \eTeX\ , especially \cs{unexpanded} is supported.
%   This allows a better protection of token groups
%   (|\mbox{|\dots|}|, math, \dots).
% \end{itemize}
% \end{quote}
%
% \subsection{Future}
%
% Currently package \xpackage{soul} does not seem to be
% maintained. Nevertheless if there will be a new version that
% adds support for UTF-8, then this package may become
% obsolete.
%
% \StopEventually{
% }
%
% \section{Implementation}
%
%    \begin{macrocode}
%<*package>
%    \end{macrocode}
%
% \subsection{Reload check and package identification}
%    Reload check, especially if the package is not used with \LaTeX.
%    \begin{macrocode}
\begingroup\catcode61\catcode48\catcode32=10\relax%
  \catcode13=5 % ^^M
  \endlinechar=13 %
  \catcode35=6 % #
  \catcode39=12 % '
  \catcode44=12 % ,
  \catcode45=12 % -
  \catcode46=12 % .
  \catcode58=12 % :
  \catcode64=11 % @
  \catcode123=1 % {
  \catcode125=2 % }
  \expandafter\let\expandafter\x\csname ver@soulutf8.sty\endcsname
  \ifx\x\relax % plain-TeX, first loading
  \else
    \def\empty{}%
    \ifx\x\empty % LaTeX, first loading,
      % variable is initialized, but \ProvidesPackage not yet seen
    \else
      \expandafter\ifx\csname PackageInfo\endcsname\relax
        \def\x#1#2{%
          \immediate\write-1{Package #1 Info: #2.}%
        }%
      \else
        \def\x#1#2{\PackageInfo{#1}{#2, stopped}}%
      \fi
      \x{soulutf8}{The package is already loaded}%
      \aftergroup\endinput
    \fi
  \fi
\endgroup%
%    \end{macrocode}
%    Package identification:
%    \begin{macrocode}
\begingroup\catcode61\catcode48\catcode32=10\relax%
  \catcode13=5 % ^^M
  \endlinechar=13 %
  \catcode35=6 % #
  \catcode39=12 % '
  \catcode40=12 % (
  \catcode41=12 % )
  \catcode44=12 % ,
  \catcode45=12 % -
  \catcode46=12 % .
  \catcode47=12 % /
  \catcode58=12 % :
  \catcode64=11 % @
  \catcode91=12 % [
  \catcode93=12 % ]
  \catcode123=1 % {
  \catcode125=2 % }
  \expandafter\ifx\csname ProvidesPackage\endcsname\relax
    \def\x#1#2#3[#4]{\endgroup
      \immediate\write-1{Package: #3 #4}%
      \xdef#1{#4}%
    }%
  \else
    \def\x#1#2[#3]{\endgroup
      #2[{#3}]%
      \ifx#1\@undefined
        \xdef#1{#3}%
      \fi
      \ifx#1\relax
        \xdef#1{#3}%
      \fi
    }%
  \fi
\expandafter\x\csname ver@soulutf8.sty\endcsname
\ProvidesPackage{soulutf8}%
  [2016/05/16 v1.1 Permit use of UTF-8 characters in soul (HO)]%
%    \end{macrocode}
%
% \subsection{Catcodes}
%
%    \begin{macrocode}
\begingroup\catcode61\catcode48\catcode32=10\relax%
  \catcode13=5 % ^^M
  \endlinechar=13 %
  \catcode123=1 % {
  \catcode125=2 % }
  \catcode64=11 % @
  \def\x{\endgroup
    \expandafter\edef\csname SOuL@AtEnd\endcsname{%
      \endlinechar=\the\endlinechar\relax
      \catcode13=\the\catcode13\relax
      \catcode32=\the\catcode32\relax
      \catcode35=\the\catcode35\relax
      \catcode61=\the\catcode61\relax
      \catcode64=\the\catcode64\relax
      \catcode123=\the\catcode123\relax
      \catcode125=\the\catcode125\relax
    }%
  }%
\x\catcode61\catcode48\catcode32=10\relax%
\catcode13=5 % ^^M
\endlinechar=13 %
\catcode35=6 % #
\catcode64=11 % @
\catcode123=1 % {
\catcode125=2 % }
\def\TMP@EnsureCode#1#2{%
  \edef\SOuL@AtEnd{%
    \SOuL@AtEnd
    \catcode#1=\the\catcode#1\relax
  }%
  \catcode#1=#2\relax
}
\TMP@EnsureCode{10}{12}% ^^J
\TMP@EnsureCode{33}{12}% !
\TMP@EnsureCode{34}{12}% "
\TMP@EnsureCode{36}{3}% $
\TMP@EnsureCode{39}{12}% '
\TMP@EnsureCode{40}{12}% (
\TMP@EnsureCode{41}{12}% )
\TMP@EnsureCode{42}{12}% *
\TMP@EnsureCode{43}{12}% +
\TMP@EnsureCode{44}{12}% ,
\TMP@EnsureCode{45}{12}% -
\TMP@EnsureCode{46}{12}% .
\TMP@EnsureCode{47}{12}% /
\TMP@EnsureCode{58}{12}% :
\TMP@EnsureCode{60}{12}% <
\TMP@EnsureCode{62}{12}% >
\TMP@EnsureCode{91}{12}% [
\TMP@EnsureCode{93}{12}% ]
\TMP@EnsureCode{94}{7}% ^
\TMP@EnsureCode{96}{12}% `
\TMP@EnsureCode{126}\active % ~
\TMP@EnsureCode{128}{12}% ^^80
\TMP@EnsureCode{147}{12}% ^^93
\TMP@EnsureCode{148}{12}% ^^94
\TMP@EnsureCode{160}{12}% ^^a0
\TMP@EnsureCode{194}{12}% ^^c2
\TMP@EnsureCode{226}{12}% ^^e2
\edef\SOuL@AtEnd{\SOuL@AtEnd\noexpand\endinput}
%    \end{macrocode}
%
% \subsection{Loading packages}
%
%    Package \xpackage{soul} uses \cs{documentclass} to detect \LaTeX.
%    \begin{macrocode}
\ifx\documentclass\@undefined
%    \end{macrocode}
%
% \subsubsection{\plainTeX}
%
%    First we check, whether package \xpackage{soul} is already loaded.
%    \begin{macrocode}
  \expandafter\ifx\csname SOUL@\endcsname\relax
%    \end{macrocode}
%    In case of \plainTeX\ package \xpackage{soul} defines some
%    macros in a simple manner that will break the definitions of
%    \xfile{miniltx.tex}, for example. Therefore these macros
%    are first saved and restored afterwards.
%    \begin{macrocode}
    \let\SOuL@orgDeclareRobustCommand\DeclareRobustCommand
    \let\SOuL@orgnewcommand          \newcommand
    \let\SOuL@orgDeclareOption       \DeclareOption
    \let\SOuL@orgPackageError        \PackageError
    \def\SOuL@restorelatexcmds{%
      \let\DeclareRobustCommand\SOuL@orgDeclareRobustCommand
      \let\newcommand          \SOuL@orgnewcommand
      \let\DeclareOption       \SOuL@orgDeclareOption
      \let\PackageError        \SOuL@orgPackageError
    }%
    \input soul.sty\relax
    \SOuL@restorelatexcmds
  \fi
%    \end{macrocode}
%    \begin{macro}{\SOUL@error}
%     Package \xpackage{soul}'s use of \cs{PackageError} is replaced
%     by \cs{@PackageError} of package \xpackage{infwarerr}.
%    \begin{macrocode}
  \input infwarerr.sty\relax
  \let\SOuL@orgSOUL@error\SOUL@error
  \def\SOUL@error{%
    \begingroup
      \let\PackageError\@PackageError
      \SOuL@orgSOUL@error
    \endgroup
  }%
%    \end{macrocode}
%    \end{macro}
%    \begin{macrocode}
  \input etexcmds.sty\relax
%    \end{macrocode}
%    \begin{macro}{\@onelevel@sanitize}
%    Define \LaTeX's \cs{@onelevel@sanitize} if not already available.
%    \begin{macrocode}
  \expandafter\ifx\csname @onelevel@sanitize\endcsname\relax
    \def\@onelevel@sanitize#1{%
      \edef#1{%
        \expandafter\strip@prefix\meaning#1%
      }%
    }%
%    \end{macrocode}
%    \end{macro}
%    \begin{macro}{\strip@prefix}
%    \begin{macrocode}
    \def\strip@prefix#1>{}%
%    \end{macrocode}
%    \end{macro}
%    \begin{macrocode}
  \fi
\else
%    \end{macrocode}
%
% \subsubsection{\LaTeX}
%    \begin{macrocode}
  \DeclareOption*{\PassOptionsToPackage{\CurrentOption}{soul}}%
  \ProcessOptions\relax
  \RequirePackage{soul}[2003/11/17]%
  \RequirePackage{infwarerr}[2016/05/16]%
  \RequirePackage{etexcmds}[2016/05/16]%
\fi
%    \end{macrocode}
%
% \subsubsection{\eTeX}
%
%    In \plainTeX\ command \cs{+} is an \emph{outer} macro.
%    Therefore numbers are used to avoid problems.
%    \begin{macrocode}
\ifetex@unexpanded
  \catcode33=14 % '!': comment
  \catcode43=9  % '+': ignore
\else
  \catcode33=9  % '!': ignore
  \catcode43=14 % '+': comment
\fi
%    \end{macrocode}
%
% \subsection{Macro for redefinitions}
%
%    \begin{macro}{\SOuL@redefine}
%    \begin{macrocode}
\def\SOuL@redefine#1{%
  \begingroup
    \def\SOuL@cmd{#1}%
    \afterassignment\SOuL@cmdcheck
    \def\SOuL@temp
}
%    \end{macrocode}
%    \end{macro}
%    \begin{macro}{\SOuL@cmdcheck}
%    \begin{macrocode}
\def\SOuL@cmdcheck{%
    \expandafter\ifx\SOuL@cmd\SOuL@temp
    \else
      \edef\SOuL@temp*{\expandafter\string\SOuL@cmd}%
      \@PackageWarningNoLine{soulutf8}{%
        Command \SOuL@temp* has changed.\MessageBreak
        Supported versions of package `soul': 2003/11/17.\MessageBreak
        Depending on the unknown changes the redefinition\MessageBreak
        of \SOuL@temp* may not behave correctly%
      }%
    \fi
  \expandafter\endgroup
  \expandafter\def\SOuL@cmd
}
%    \end{macrocode}
%    \end{macro}
%
% \subsection{Redefinition of \cs{SOUL@eval}}
%
%    \begin{macro}{\SOUL@eval}
%    Macro \cs{SOUL@eval} is redefined to add detection of
%    the first byte of a UTF-8 sequence. Because \cs{SOUL@eval}
%    is overwritten, a warning is issued, if the contents of
%    \cs{SOUL@eval} is not as expected.
%    \begin{macrocode}
\SOuL@redefine\SOUL@eval{%
%    \end{macrocode}
%    First the expected definition.
%    \begin{macrocode}
  \def\SOUL@n*##1{\SOUL@scan}%
  \if\noexpand\SOUL@@\SOUL@spc
  \else
    \SOUL@ignorespacesfalse
  \fi
  \ifnum\SOUL@minus=\thr@@
    \SOUL@flushminus
  \else\ifnum\SOUL@comma=\tw@
    \SOUL@flushcomma
  \else\ifnum\SOUL@apo=\tw@
    \SOUL@flushapo
  \else\ifnum\SOUL@grave=\tw@
    \SOUL@flushgrave
  \fi\fi\fi\fi
  \ifx\SOUL@@-\else\SOUL@flushminus\fi
  \ifx\SOUL@@,\else\SOUL@flushcomma\fi
  \ifx\SOUL@@'\else\SOUL@flushapo\fi
  \ifx\SOUL@@`\else\SOUL@flushgrave\fi
  \ifx\SOUL@@-%
    \advance\SOUL@minus\@ne
  \else\ifx\SOUL@@,%
    \advance\SOUL@comma\@ne
  \else\ifx\SOUL@@'%
    \advance\SOUL@apo\@ne
  \else\ifx\SOUL@@`%
    \advance\SOUL@grave\@ne
  \else
    \SOUL@flushminus
    \SOUL@flushcomma
    \SOUL@flushapo
    \SOUL@flushgrave
    \ifx\SOUL@@\SOUL@stop
      \def\SOUL@n*{%
        \SOUL@doword
        \SOUL@eventuallyexhyphen\null
      }%
    \else\ifx\SOUL@@\par
      \def\SOUL@n*\par{\par\leavevmode\SOUL@scan}%
    \else\if\noexpand\SOUL@@\SOUL@spc
      \SOUL@doword
      \SOUL@eventuallyexhyphen\null
      \ifSOUL@ignorespaces
      \else
        \SOUL@everyspace{}%
      \fi
      \def\SOUL@n* {\SOUL@scan}%
    \else\ifx\SOUL@@\\%
      \SOUL@doword
      \SOUL@eventuallyexhyphen\null
      \SOUL@everyspace{\unskip\nobreak\hfil\break}%
      \SOUL@ignorespacestrue
    \else\ifx\SOUL@@~%
      \SOUL@doword
      \SOUL@eventuallyexhyphen\null
      \SOUL@everyspace{\nobreak}%
    \else\ifx\SOUL@@\slash
      \SOUL@doword
      \SOUL@eventuallyexhyphen{/}%
      \SOUL@exhyphen{/}%
    \else\ifx\SOUL@@\mbox
      \def\SOUL@n*{\SOUL@addprotect}%
    \else\ifx\SOUL@@\hbox
      \def\SOUL@n*{\SOUL@addprotect}%
    \else\ifx\SOUL@@\soulomit
      \def\SOUL@n*\soulomit##1{%
        \SOUL@doword
        {\spaceskip\SOUL@spaceskip##1}%
        \SOUL@scan
      }%
    \else\ifx\SOUL@@\break
      \SOUL@doword
      \break
    \else\ifx\SOUL@@\linebreak
      \SOUL@doword
      \SOUL@everyspace{\linebreak}%
    \else\ifcat\bgroup\noexpand\SOUL@@
      \def\SOUL@n*{\SOUL@addgroup{}}%
    \else\ifcat$\noexpand\SOUL@@
      \def\SOUL@n*{\SOUL@addmath}%
    \else
      \def\SOUL@n*{\SOUL@dotoken}%
    \fi\fi\fi\fi\fi\fi\fi\fi\fi\fi\fi\fi\fi
  \fi\fi\fi\fi
  \SOUL@n*%
}{%
%    \end{macrocode}
%    Now the redefined version follows.
%    \begin{macrocode}
  \def\SOUL@n*##1{\SOUL@scan}%
  \if\noexpand\SOUL@@\SOUL@spc
  \else
    \SOUL@ignorespacesfalse
  \fi
  \ifnum\SOUL@minus=\thr@@
    \SOUL@flushminus
  \else\ifnum\SOUL@comma=\tw@
    \SOUL@flushcomma
  \else\ifnum\SOUL@apo=\tw@
    \SOUL@flushapo
  \else\ifnum\SOUL@grave=\tw@
    \SOUL@flushgrave
  \fi\fi\fi\fi
  \ifx\SOUL@@-\else\SOUL@flushminus\fi
  \ifx\SOUL@@,\else\SOUL@flushcomma\fi
  \ifx\SOUL@@'\else\SOUL@flushapo\fi
  \ifx\SOUL@@`\else\SOUL@flushgrave\fi
  \ifx\SOUL@@-%
    \advance\SOUL@minus\@ne
  \else\ifx\SOUL@@,%
    \advance\SOUL@comma\@ne
  \else\ifx\SOUL@@'%
    \advance\SOUL@apo\@ne
  \else\ifx\SOUL@@`%
    \advance\SOUL@grave\@ne
  \else
    \SOUL@flushminus
    \SOUL@flushcomma
    \SOUL@flushapo
    \SOUL@flushgrave
    \ifx\SOUL@@\SOUL@stop
      \def\SOUL@n*{%
          \SOUL@doword
          \SOUL@eventuallyexhyphen\null
      }%
    \else\ifx\SOUL@@\par
      \def\SOUL@n*\par{\par\leavevmode\SOUL@scan}%
    \else\if\noexpand\SOUL@@\SOUL@spc
      \SOUL@doword
      \SOUL@eventuallyexhyphen\null
      \ifSOUL@ignorespaces
      \else
        \SOUL@everyspace{}%
      \fi
      \def\SOUL@n* {\SOUL@scan}%
    \else\ifx\SOUL@@\\%
      \SOUL@doword
      \SOUL@eventuallyexhyphen\null
      \SOUL@everyspace{\unskip\nobreak\hfil\break}%
      \SOUL@ignorespacestrue
    \else\ifx\SOUL@@~%
      \SOUL@doword
      \SOUL@eventuallyexhyphen\null
      \SOUL@everyspace{\nobreak}%
    \else\ifx\SOUL@@\slash
      \SOUL@doword
      \SOUL@eventuallyexhyphen{/}%
      \SOUL@exhyphen{/}%
    \else\ifx\SOUL@@\mbox
      \def\SOUL@n*{\SOUL@addprotect}%
    \else\ifx\SOUL@@\hbox
      \def\SOUL@n*{\SOUL@addprotect}%
    \else\ifx\SOUL@@\soulomit
      \def\SOUL@n*\soulomit##1{%
        \SOUL@doword
        {\spaceskip\SOUL@spaceskip##1}%
        \SOUL@scan
      }%
    \else\ifx\SOUL@@\break
      \SOUL@doword
      \break
    \else\ifx\SOUL@@\linebreak
      \SOUL@doword
      \SOUL@everyspace{\linebreak}%
    \else\ifcat\bgroup\noexpand\SOUL@@
      \def\SOUL@n*{\SOUL@addgroup{}}%
    \else\ifcat$\noexpand\SOUL@@
      \def\SOUL@n*{\SOUL@addmath}%
    \else
%    \end{macrocode}
%    The current token is examined to detect the start of
%    a UTF-8 sequence.
%    \begin{macrocode}
      \SOuL@analyzeutfviii
      \ifcase\SOuL@octets
        \SOuL@analyzeutfviiix
      \fi
      \ifcase\SOuL@octets
        \def\SOUL@n*{\SOUL@dotoken}%
      \or % 1
      \or % 2
        \def\SOUL@n*{\SOuL@addtwooctets}%
      \or % 3
        \def\SOUL@n*{\SOuL@addthreeoctets}%
      \or % 4
        \def\SOUL@n*{\SOuL@addfouroctets}%
      \fi
    \fi\fi\fi\fi\fi\fi\fi\fi\fi\fi\fi\fi\fi
  \fi\fi\fi\fi
  \SOUL@n*%
}
%    \end{macrocode}
%    \end{macro}
%
% \subsection{UTF-8 analysis}
%
% \subsubsection{Help strings}
%
%    \begin{macrocode}
\def\SOuL@defsanitizedstring#1#2{%
  \expandafter\def\csname SOuL@string#1\endcsname{#2}%
  \expandafter\@onelevel@sanitize\csname SOuL@string#1\endcsname
}
\SOuL@defsanitizedstring{UTFviii}{UTFviii@}
\SOuL@defsanitizedstring{octets}{@octets}
\SOuL@defsanitizedstring{two}{two}
\SOuL@defsanitizedstring{three}{three}
\SOuL@defsanitizedstring{four}{four}
\SOuL@defsanitizedstring{macrocolon}{macro:}
\SOuL@defsanitizedstring{csnameu}{csname u8-}
\SOuL@defsanitizedstring{undeferr}{utf@viii@undeferr}
\def\SOuL@stringendash{^^e2^^80^^93}
\def\SOuL@stringemdash{^^e2^^80^^94}
\def\SOuL@stringnobreakspace{^^c2^^a0}
\edef\SOuL@charhash{\string #}
\edef\SOuL@chartwo{\string 2}
\edef\SOuL@charthree{\string 3}
\def\SOuL@empty{}
%    \end{macrocode}
%
% \subsubsection{Support for \xfile{utf8.def}}
%
%    \begin{macro}{\SOuL@analyzeutfviii}
%    \begin{macrocode}
\begingroup
  \edef\x{\endgroup
    \def\noexpand\SOuL@analyzeutfviii{%
      \noexpand\expandafter\noexpand\SOuL@checkutfviii
      \noexpand\meaning\noexpand\SOUL@@
      \SOuL@stringUTFviii\SOuL@stringoctets
      \noexpand\@nil
    }%
    \def\noexpand\SOuL@checkutfviii
      ##1\SOuL@stringUTFviii##2\SOuL@stringoctets##3\noexpand\@nil
  }%
\x{%
  \def\SOuL@temp{#2}%
  \chardef\SOuL@octets=%
      \ifx\SOuL@temp\SOuL@stringtwo
        \tw@
      \else\ifx\SOuL@temp\SOuL@stringthree
        \thr@@
      \else\ifx\SOuL@temp\SOuL@stringfour
        4 %
      \else
        \z@
      \fi\fi\fi
}
%    \end{macrocode}
%    \end{macro}
%
% \subsubsection{Support for \xfile{utf8x.def}}
%
%    \begin{macro}{\SOuL@analyzeutfviiix}
%    \begin{macrocode}
\begingroup
  \edef\x{\endgroup
    \def\noexpand\SOuL@analyzeutfviiix{%
      \noexpand\expandafter\noexpand\SOuL@checkutfviiix
      \noexpand\meaning\noexpand\SOUL@@
      \SOuL@stringmacrocolon\SOuL@charhash1{}{}{}{}%
      \SOuL@stringcsnameu\SOuL@stringundeferr
      \noexpand\@nil
    }%
%    \end{macrocode}
%    \end{macro}
%    \begin{macro}{\SOuL@checkutfviiix}
%    \begin{macrocode}
    \def\noexpand\SOuL@checkutfviiix
      ##1\SOuL@stringmacrocolon\SOuL@charhash1##2##3##4##5##6%
      \SOuL@stringcsnameu##7\SOuL@stringundeferr##8\noexpand\@nil
  }%
\x{%
  \def\SOuL@temp{#7}%
  \ifx\SOuL@temp\SOuL@empty
    \chardef\SOuL@octets=\z@
  \else
    \def\SOuL@temp{#5}%
    \ifx\SOuL@temp\SOuL@charthree
      \chardef\SOuL@octets=4 %
    \else
      \def\SOuL@temp{#3}%
      \ifx\SOuL@temp\SOuL@chartwo
        \chardef\SOuL@octets=\thr@@
      \else
        \chardef\SOuL@octets=\tw@
      \fi
    \fi
  \fi
}
%    \end{macrocode}
%    \end{macro}
%
% \subsection{Actions for UTF-8 sequences}
%
%    \begin{macro}{\SOuL@addtwooctets}
%    \begin{macrocode}
\def\SOuL@addtwooctets#1#2{%
  \def\SOuL@temp{#1#2}%
  \@onelevel@sanitize\SOuL@temp
  \ifx\SOuL@temp\SOuL@stringnobreakspace
    \SOUL@doword
    \SOUL@eventuallyexhyphen\null
    \SOUL@everyspace{\nobreak}%
    \let\SOuL@next\SOUL@scan
  \else
    \def\SOuL@next{%
!     \SOUL@addtoken{{\noexpand#1\noexpand#2}}%
+     \SOUL@addtoken{{\etex@unexpanded{#1#2}}}%
    }%
  \fi
  \SOuL@next
}
%    \end{macrocode}
%    \end{macro}
%    \begin{macro}{\SOuL@addthreeoctets}
%    \begin{macrocode}
\def\SOuL@addthreeoctets#1#2#3{%
  \def\SOuL@temp{#1#2#3}%
  \@onelevel@sanitize\SOuL@temp
  \ifx\SOuL@temp\SOuL@stringendash
    \SOUL@doword
    \SOUL@eventuallyexhyphen{-}%
    \SOUL@exhyphen{--}%
    \let\SOuL@next\SOUL@scan
  \else
    \ifx\SOuL@temp\SOuL@stringemdash
      \SOUL@doword
      \SOUL@eventuallyexhyphen{-}%
      \SOUL@exhyphen{---}%
      \let\SOuL@next\SOUL@scan
    \else
      \def\SOuL@next{%
!       \SOUL@addtoken{{\noexpand#1\noexpand#2\noexpand#3}}%
+       \SOUL@addtoken{{\etex@unexpanded{#1#2#3}}}%
      }%
    \fi
  \fi
  \SOuL@next
}
%    \end{macrocode}
%    \end{macro}
%    \begin{macro}{\SOuL@addfouroctets}
%    \begin{macrocode}
\def\SOuL@addfouroctets#1#2#3#4{%
! \SOUL@addtoken{{\noexpand#1\noexpand#2\noexpand#3\noexpand#4}}%
+ \SOUL@addtoken{{\etex@unexpanded{#1#2#3#4}}}%
}
%    \end{macrocode}
%    \end{macro}
%
% \subsubsection{Redefinition of \cs{SOUL@splittoken}}
%
%    \begin{macro}{\SOUL@splittoken}
%    Macro \cs{SOUL@splittoken} separates the first token
%    or token group from a word and redefines the word to contain
%    the remaining tokens. However if the remaining tokens are
%    a token group, then the curly braces will be removed and
%    the token group is splitted by the next call of \cs{SOUL@splittoken}.
%    The redefinition avoids the removal of curly braces around the
%    remaining tokens.
%    \begin{macrocode}
\SOuL@redefine\SOUL@splittoken#1#2\SOUL@stop{%
  \global\SOUL@token={#1}%
  \global\SOUL@word={#2}%
}#1{%
  \global\SOUL@token={#1}%
  \SOuL@remainingtoken\relax
}
%    \end{macrocode}
%    \end{macro}
%    \begin{macro}{\SOuL@remainingtoken}
%    \begin{macrocode}
\def\SOuL@remainingtoken#1\SOUL@stop{%
  \global\SOUL@word=\expandafter{\@gobble#1}%
}
%    \end{macrocode}
%    \end{macro}
%
% \subsection{Patches}
%
%    The fixed \cs{SOUL@splittoken} allows to remove the double
%    sets of curly braces in other macros of package \xpackage{soul}.
%    The benefit is that implicite kernings are more often detected
%    and fixes a bug in package \xpackage{soul}. The disadvantage is
%    incompatibility. The width of the resulting strings may change.
%    \begin{macro}{\SOUL@flushcomma}
%    \begin{macrocode}
\SOuL@redefine\SOUL@flushcomma{%
  \ifcase\SOUL@comma
  \or
    \edef\x{\SOUL@word={\the\SOUL@word,}}\x
  \or
    \edef\x{\SOUL@word={\the\SOUL@word{{,,}}}}\x
  \fi
  \SOUL@comma\z@
}{%
  \ifcase\SOUL@comma
  \or
    \edef\x{\SOUL@word={\the\SOUL@word,}}\x
  \or
    \edef\x{\SOUL@word={\the\SOUL@word{,,}}}\x
  \fi
  \SOUL@comma\z@
}
%    \end{macrocode}
%    \end{macro}
%    \begin{macro}{\SOUL@flushapo}
%    \begin{macrocode}
\SOuL@redefine\SOUL@flushapo{%
  \ifcase\SOUL@apo
  \or
    \edef\x{\SOUL@word={\the\SOUL@word'}}\x
  \or
    \edef\x{\SOUL@word={\the\SOUL@word{{''}}}}\x
  \fi
  \SOUL@apo\z@
}{%
  \ifcase\SOUL@apo
  \or
    \edef\x{\SOUL@word={\the\SOUL@word'}}\x
  \or
    \edef\x{\SOUL@word={\the\SOUL@word{''}}}\x
  \fi
  \SOUL@apo\z@
}
%    \end{macrocode}
%    \end{macro}
%    \begin{macro}{\SOUL@flushgrave}
%    \begin{macrocode}
\SOuL@redefine\SOUL@flushgrave{%
  \ifcase\SOUL@grave
  \or
    \edef\x{\SOUL@word={\the\SOUL@word`}}\x
  \or
    \edef\x{\SOUL@word={\the\SOUL@word{{``}}}}\x
  \fi
  \SOUL@grave\z@
}{%
  \ifcase\SOUL@grave
  \or
    \edef\x{\SOUL@word={\the\SOUL@word`}}\x
  \or
    \edef\x{\SOUL@word={\the\SOUL@word{``}}}\x
  \fi
  \SOUL@grave\z@
}
%    \end{macrocode}
%    \end{macro}
%    \begin{macro}{\SOUL@addgroup}
%    \begin{macrocode}
\SOuL@redefine\SOUL@addgroup#1#2{%
  {%
    \let\protect\noexpand
    \edef\x{%
      \global\SOUL@word={%
        \the\SOUL@word
        {{\noexpand#1#2}}%
      }%
    }%
    \x
  }%
  \SOUL@scan
}#1#2{%
  \begingroup
    \let\protect\noexpand
    \edef\x{\endgroup
      \SOUL@word={%
        \the\SOUL@word
!       {\noexpand#1{#2}}%
+       {\etex@unexpanded{#1{#2}}}%
      }%
    }%
  \x
  \SOUL@scan
}
%    \end{macrocode}
%    \end{macro}
%    \begin{macro}{\SOUL@addmath}
%    \begin{macrocode}
\SOuL@redefine\SOUL@addmath$#1${%
  {%
    \let\protect\noexpand
    \edef\x{%
      \global\SOUL@word={%
        \the\SOUL@word
        {{\hbox{$#1$}}}%
      }%
    }%
    \x
  }%
  \SOUL@scan
}$#1${%
  \begingroup
    \let\protect\noexpand
    \edef\x{\endgroup
      \SOUL@word={%
        \the\SOUL@word
!       {\hbox{$#1$}}%
+       {\etex@unexpanded{\hbox{$#1$}}}%
      }%
    }%
  \x
  \SOUL@scan
}
%    \end{macrocode}
%    \end{macro}
%    \begin{macro}{\SOUL@addprotect}
%    \begin{macrocode}
\SOuL@redefine\SOUL@addprotect#1#2{%
  {%
    \let\protect\noexpand
    \edef\x{%
      \global\SOUL@word={%
        \the\SOUL@word
        {{\hbox{#2}}}%
      }%
    }%
    \x
  }%
  \SOUL@scan
}#1#2{%
  \begingroup
    \let\protect\noexpand
    \edef\x{\endgroup
      \SOUL@word={%
        \the\SOUL@word
!       {\hbox{#2}}%
+       {\etex@unexpanded{\hbox{#2}}}%
      }%
    }%
  \x
  \SOUL@scan
}
%    \end{macrocode}
%    \end{macro}
%
%    \begin{macro}{\SOUL@addtoken}
%    \begin{macrocode}
+ \SOuL@redefine\SOUL@addtoken#1{%
+   \edef\x{%
+     \SOUL@word={%
+       \the\SOUL@word
+       \noexpand#1%
+     }%
+   }%
+   \x
+   \SOUL@scan
+ }#1{%
+   \edef\x{%
+     \SOUL@word={%
+       \the\SOUL@word
+       \etex@unexpanded{#1}%
+     }%
+   }%
+   \x
+   \SOUL@scan
+ }%
%    \end{macrocode}
%    \end{macro}
%
%    \begin{macrocode}
\SOuL@AtEnd%
%    \end{macrocode}
%
%    \begin{macrocode}
%</package>
%    \end{macrocode}
%
% \section{Test}
%
% \subsection{Catcode checks for loading}
%
%    \begin{macrocode}
%<*test1>
%    \end{macrocode}
%    \begin{macrocode}
\catcode`\{=1 %
\catcode`\}=2 %
\catcode`\#=6 %
\catcode`\@=11 %
\expandafter\ifx\csname count@\endcsname\relax
  \countdef\count@=255 %
\fi
\expandafter\ifx\csname @gobble\endcsname\relax
  \long\def\@gobble#1{}%
\fi
\expandafter\ifx\csname @firstofone\endcsname\relax
  \long\def\@firstofone#1{#1}%
\fi
\expandafter\ifx\csname loop\endcsname\relax
  \expandafter\@firstofone
\else
  \expandafter\@gobble
\fi
{%
  \def\loop#1\repeat{%
    \def\body{#1}%
    \iterate
  }%
  \def\iterate{%
    \body
      \let\next\iterate
    \else
      \let\next\relax
    \fi
    \next
  }%
  \let\repeat=\fi
}%
\def\RestoreCatcodes{}
\count@=0 %
\loop
  \edef\RestoreCatcodes{%
    \RestoreCatcodes
    \catcode\the\count@=\the\catcode\count@\relax
  }%
\ifnum\count@<255 %
  \advance\count@ 1 %
\repeat

\def\RangeCatcodeInvalid#1#2{%
  \count@=#1\relax
  \loop
    \catcode\count@=15 %
  \ifnum\count@<#2\relax
    \advance\count@ 1 %
  \repeat
}
\def\RangeCatcodeCheck#1#2#3{%
  \count@=#1\relax
  \loop
    \ifnum#3=\catcode\count@
    \else
      \errmessage{%
        Character \the\count@\space
        with wrong catcode \the\catcode\count@\space
        instead of \number#3%
      }%
    \fi
  \ifnum\count@<#2\relax
    \advance\count@ 1 %
  \repeat
}
\def\space{ }
\expandafter\ifx\csname LoadCommand\endcsname\relax
  \def\LoadCommand{\input soulutf8.sty\relax}%
\fi
\def\Test{%
  \RangeCatcodeInvalid{0}{47}%
  \RangeCatcodeInvalid{58}{64}%
  \RangeCatcodeInvalid{91}{96}%
  \RangeCatcodeInvalid{123}{255}%
  \catcode`\@=12 %
  \catcode`\\=0 %
  \catcode`\%=14 %
  \LoadCommand
  \RangeCatcodeCheck{0}{36}{15}%
  \RangeCatcodeCheck{37}{37}{14}%
  \RangeCatcodeCheck{38}{47}{15}%
  \RangeCatcodeCheck{48}{57}{12}%
  \RangeCatcodeCheck{58}{63}{15}%
  \RangeCatcodeCheck{64}{64}{12}%
  \RangeCatcodeCheck{65}{90}{11}%
  \RangeCatcodeCheck{91}{91}{15}%
  \RangeCatcodeCheck{92}{92}{0}%
  \RangeCatcodeCheck{93}{96}{15}%
  \RangeCatcodeCheck{97}{122}{11}%
  \RangeCatcodeCheck{123}{255}{15}%
  \RestoreCatcodes
}
\Test
\csname @@end\endcsname
\end
%    \end{macrocode}
%    \begin{macrocode}
%</test1>
%    \end{macrocode}
%
% \subsection{\LaTeX\space tests}
%
%    \begin{macrocode}
%<*test2|test3|test4|test5>
%    \end{macrocode}
%    \begin{macrocode}
%<noetex>\let\SavedUnexpanded\unexpanded
%<noetex>\let\unexpanded\UNDEFINED
\NeedsTeXFormat{LaTeX2e}
\documentclass{article}
\usepackage[T1]{fontenc}
\usepackage{lmodern}
\usepackage{textcomp}
\usepackage[ngerman]{babel}
\expandafter\def\csname ver@hypdestopt.sty\endcsname{}
\usepackage[unicode]{hyperref}
\usepackage{bookmark}
\usepackage{color}
\usepackage{soul}[2003/11/17]
\makeatletter
\newcommand*{\@ifDefined}[1]{%
  \@ifundefined{#1}{%
    \@latex@error{%
      \expandafter\noexpand\csname#1\endcsname undefined%
    }\@ehc
  }%
}
\newcommand*{\SaveOrg}[1]{%
  \@ifdefinable{ORG@SOUL@#1}{%
    \@ifDefined{SOUL@#1}{%
      \expandafter\let\csname ORG@SOUL@#1\expandafter\endcsname
                      \csname SOUL@#1\endcsname
    }%
  }%
}
\newcommand*{\RestoreOrg}[1]{%
  \@ifDefined{ORG@SOUL@#1}{%
    \@ifDefined{SOUL@#1}{%
      \expandafter\let\csname SOUL@#1\expandafter\endcsname
                      \csname ORG@SOUL@#1\endcsname
    }%
  }%
}
\makeatother
\SaveOrg{eval}
\SaveOrg{splittoken}
\SaveOrg{flushcomma}
\SaveOrg{flushapo}
\SaveOrg{grave}
\SaveOrg{addgroup}
\SaveOrg{addmath}
\SaveOrg{addprotect}
\SaveOrg{addtoken}
\usepackage{soulutf8}[2016/05/16]
%    \end{macrocode}
%    \begin{macrocode}
%<noetex>\let\unexpanded\SavedUnexpanded
%    \end{macrocode}
%    Input encoding \xfile{utf8}.
%    \begin{macrocode}
%<*utf8>
\usepackage[utf8]{inputenc}
\DeclareUnicodeCharacter{00A0}{\nobreakspace}
%</utf8>
%    \end{macrocode}
%    Input encoding \xfile{utf8x}. Package \xpackage{ucs}
%    has a nasty bug, that it redefines \cs{show}.
%    \begin{macrocode}
%<*utf8x>
\let\origshow\show
\usepackage[utf8x]{inputenc}
\AtBeginDocument{%
  \let\show\origshow
}
%</utf8x>
%    \end{macrocode}
%
% \subsubsection{Driver definitions}
%
%    \begin{macrocode}
\makeatletter
%    \end{macrocode}
%    \begin{macro}{\an}
%    The definition for \cs{an} is taken from the documentation of
%    package \xpackage{soul}.
%    \begin{macrocode}
\DeclareRobustCommand*{\an}{%
  \def\SOUL@preamble{$^{^P}$}%
  \def\SOUL@everyspace##1{##1\texttt{\char`\ }\linebreak[2]}%
  \def\SOUL@postamble{$^{^E}$}%
  \def\SOUL@everyhyphen{$^{^-}$\linebreak[1]}%
  \def\SOUL@everyexhyphen##1{##1$^{^=}$}%
  \def\SOUL@everysyllable{$^{^S}$}%
  \def\SOUL@everytoken{\the\SOUL@token$^{^T}$}%
  \def\SOUL@everylowerthan{$^{^L}$}%
  \SOUL@
}
%    \end{macrocode}
%    \end{macro}
%
%    \begin{macro}{\sy}
%    The definition for \cs{sy} is taken from the documentation of
%    package \xpackage{soul}.
%    \begin{macrocode}
\DeclareRobustCommand*{\sy}{%
  \SOUL@setup
  \def\SOUL@preamble{%
    \lefthyphenmin=\z@
    \righthyphenmin=\z@
  }%
  \def\SOUL@everyspace##1{##1\space}%
  \def\SOUL@everysyllable{\the\SOUL@syllable}%
  \def\SOUL@everyhyphen{%
    \discretionary{%
      \SOUL@setkern\SOUL@hyphkern
      \SOUL@sethyphenchar
    }{}{%
      \hbox{\kern1pt$\cdot$}%
    }%
  }%
  \def\SOUL@everyexhyphen##1{%
    \SOUL@setkern\SOUL@hyphkern
    \hbox{##1}%
    \discretionary{}{}{%
      \SOUL@setkern\SOUL@charkern
    }%
  }%
  \SOUL@
}
%    \end{macrocode}
%    \end{macro}
%
%    \begin{macro}{\soulH}
%    \begin{macrocode}
\DeclareRobustCommand*{\soulH}{%
  \def\SOUL@preamble{%
    \gdef\SOULH{}%
    \soulHadd{[}%
  }%
  \def\SOUL@everyspace##1{%
    \soulHadd{Sp}%
  }%
  \def\SOUL@postamble{%
    \soulHadd{]}%
  }%
  \def\SOUL@everyhyphen{%
    \soulHadd{Hy}%
  }%
  \def\SOUL@everyexhyphen##1{%
    \soulHadd{Exhy}%
  }%
  \def\SOUL@everysyllable{%
    \soulHadd{Sy}%
  }%
  \def\SOUL@everytoken{%
    \soulHadd{T}%
  }%
  \def\SOUL@everylowerthan{%
    \soulHadd{<}%
  }%
  \SOUL@
}
\newcommand*{\soulHadd}[1]{%
  #1%
  \expandafter\gdef\expandafter\soulHstring\expandafter{%
    \soulHstring
    #1%
  }%
}
%    \end{macrocode}
%    \end{macro}
%
% \subsubsection{Test macros}
%
%    \begin{macrocode}
\usepackage{qstest}
\IncludeTests{*}
\LogTests{log}{*}{*}
%    \end{macrocode}
%    \begin{macrocode}
\newsavebox\boxA
\newsavebox\boxB
\newsavebox\boxC
%    \end{macrocode}
%    \begin{macro}{\TestDriver}
%    \begin{macrocode}
\newcommand*{\TestDriver}[4]{%
  \subsection{Driver #1}%
  \@PackageInfo{soulutf8-test}{* Driver #1}%
  \expandafter\let\expandafter\testdriver\csname #1\endcsname
  \def\teststringA{#3}%
  \def\teststringB{#4}%
  \begin{qstest}{string-#2-driver-#1}{string,#2,driver,#1}%
    \@PackageInfo{soulutf8-test}{* Version: soulutf8}%
    \global\let\soulHstring\@empty
    \setbox0=\hbox{\expandafter\testdriver\expandafter{\teststringA}}%
    \let\HstringA\soulHstring
    \SetTestVBox{\boxA}{\teststringA}%
    %
    \@PackageInfo{soulutf8-test}{* Version: soul + patch}%
    \global\let\soulHstring\@empty
    \RestoreOrg{eval}%
    \setbox2=\hbox{\expandafter\testdriver\expandafter{\teststringB}}%
    \let\HstringB\soulHstring
    \SetTestVBox{\boxB}{\teststringB}%
    %
    \@PackageInfo{soulutf8-test}{* Version: original soul}%
    \global\let\soulHstring\@empty
    \RestoreOrg{splittoken}%
    \RestoreOrg{flushcomma}%
    \RestoreOrg{flushapo}%
    \RestoreOrg{grave}%
    \RestoreOrg{addgroup}%
    \RestoreOrg{addmath}%
    \RestoreOrg{addprotect}%
    \RestoreOrg{addtoken}%
    \setbox4=\hbox{\expandafter\testdriver\expandafter{\teststringB}}%
    \let\HstringC\soulHstring
    \SetTestVBox{\boxC}{\teststringB}%
    %
    \TestInfo{0}{soulutf8}%
    \TestInfo{2}{soul + patch}%
    \TestInfo{4}{original soul}%
    \TestInfoVBox
    \showboxdepth=\maxdimen
    \showboxbreadth=\maxdimen
    \tracingonline=\@ne
    \Expect*{\HstringA}*{\HstringB}%
    \Expect*{\HstringA}*{\HstringC}%
    \Expect*{\the\wd0}*{\the\wd2}%
    \ifdim\wd0=\wd2 %
    \else
      \showbox0 %
      \showbox2 %
    \fi
    \Expect*{\the\ht0}*{\the\ht2}%
    \Expect*{\the\dp0}*{\the\dp2}%
    \Expect*{\the\wd\boxB}*{\the\wd\boxA}%
    \Expect*{\the\wd\boxC}*{\the\wd\boxA}%
    \Expect*{\the\ht\boxB}*{\the\ht\boxA}%
    \ifdim\ht\boxB=\ht\boxA
    \else
      \showbox\boxA
      \showbox\boxB
    \fi
    \Expect*{\the\ht\boxC}*{\the\ht\boxA}%
    \Expect*{\the\dp\boxB}*{\the\dp\boxA}%
    \Expect*{\the\dp\boxC}*{\the\dp\boxA}%
  \end{qstest}%
}
%    \end{macrocode}
%    \end{macro}
%    \begin{macro}{\SetTestVBox}
%    \begin{macrocode}
\newcommand*{\SetTestVBox}[2]{%
  \setbox#1=\vbox{%
    \hsize=0pt\relax
    \hfuzz=100mm\relax
    \expandafter\testdriver\expandafter{#2}%
  }%
}
%    \end{macrocode}
%    \end{macro}
%    \begin{macro}{\TestInfo}
%    \begin{macrocode}
\newcommand*{\TestInfo}[2]{%
  \par
  \noindent
  \copy#1\relax
  \hfill
  \begingroup
    \small
    \ifdim\wd0=\wd2\relax
      \ifdim\wd0=\wd4\relax
      \else
        \ifnum#1=2 %
          \raisebox{.5\baselineskip}[0pt][0pt]{%
            \llap{$\}$ }%
            \rlap{width = \the\wd#1\relax\space$\{$}%
            \phantom{width = 999.99999pt}%
            \qquad
          }%
        \else
          \ifnum#1=4 %
            \rlap{width = \the\wd#1}%
            \phantom{width = 999.99999pt}%
            \qquad
          \fi
        \fi
      \fi
    \else
      \begingroup
        \bfseries
        \rlap{width = \the\wd#1}%
        \phantom{width = 999.99999pt}%
      \endgroup
      \qquad
    \fi
    \rlap{(#2)}%
    \phantom{(original soul)}%
  \endgroup
  \par
}
%    \end{macrocode}
%    \end{macro}
%    \begin{macro}{\TestInfoVBox}
%    \begin{macrocode}
\newcommand*{\TestInfoVBox}{%
  \par
  \noindent
  \mbox{\vrule\relax\space\copy\boxA}%
  \ifdim\ht\boxA=\ht\boxB
  \else
    \hfill
    \begingroup
      \small
      \rlap{(soulutf8)}%
      \phantom{(soulutf8)}%
    \endgroup
    \par
    \noindent
    \mbox{\vrule\relax\space\copy\boxB}%
    \hfill
    \begingroup
      \small
      \rlap{(soul)}%
      \phantom{(soulutf8)}%
    \endgroup
  \fi
  \par
}
%    \end{macrocode}
%    \end{macro}
%    \begin{macro}{\Test}
%    \begin{macrocode}
\makeatletter
\newcommand*{\Test}[3]{%
  \clearpage
  \begingroup
    \def\linebreak{\textbackslash linebreak }%
    \def\\{\textbackslash\textbackslash}%
    \def\soulomit##1{%
      \textbackslash soulomit\textbraceleft##1\textbraceright
    }%
    \def\mbox##1{%
      \textbackslash mbox\textbraceleft##1\textbraceright
    }%
    \def\abc{\textbackslash abc }%
    \def\texttt##1{%
      \textbackslash texttt\textbraceleft##1\textbraceright
    }%
    \def\c##1{%
      \textbackslash c\textbraceleft##1\textbraceright
    }%
%<*utf8x>
    \count@=128 %
    \loop
    \ifnum\count@<256 %
      \lccode`\~=\count@
      \lowercase{%
        \def~{\noexpand~}%
      }%
      \advance\count@\@ne
    \repeat
%</utf8x>
    \csname protected@edef\endcsname\x{\endgroup
      \noexpand\section{String #1: #2}%
    }%
  \x
  \TestDriver{so}{#1}{#2}{#3}%
  \TestDriver{caps}{#1}{#2}{#3}%
  \TestDriver{ul}{#1}{#2}{#3}%
  \TestDriver{st}{#1}{#2}{#3}%
  \TestDriver{hl}{#1}{#2}{#3}%
  \TestDriver{an}{#1}{#2}{#3}%
  \TestDriver{sy}{#1}{#2}{#3}%
  \TestDriver{soulH}{#1}{#2}{#3}%
}
\makeatother
%    \end{macrocode}
%    \end{macro}
%
% \subsubsection{Testing}
%
%    \begin{macrocode}
\begin{document}
  \Test A{}{}%
  \Test B{Hello World}{Hello World}%
  \Test C{%
    Umlauts: ^^c3^^84^^c3^^96^^c3^^9c^^c3^^a4^^c3^^b6^^c3^^bc^^c3^^9f%
  }{%
    Umlauts: \"A\"O\"U\"a\"o\"u\ss
  }%
  \Test D{Euro: ^^e2^^82^^ac}{Euro: \texteuro}%
  \Test E{Stra^^c3^^9fe}{Stra\ss e}%
  \Test F{R^^c3^^b6mer}{R\"omer}%
  \Test G{A^^e2^^80^^93B^^e2^^80^^94C}{A--B---C}%
  \Test H{X^^c2^^a0Y}{X~Y}%
  \Test I{A^^c5^^a4}{A\v T}%
  \Test J{A,B,,C,,,D,,,,E}{A,B,,C,,,D,,,,E}%
  \Test K{A'B''C'''D''''E}{A'B''C'''D''''E}%
  \Test L{A`B``C```D````E}{A`B``C```D````E}%
  \Test M{A$x$B}{A$x$B}%
  \Test N{A\mbox{abc}B}{A\mbox{abc}B}%
  \Test O{,,TA''}{,,TA''}%
  \Test P{{!`}Hola{?`}}{{!`}Hola{?`}}%
  \Test Q{re-sent\slash x}{re-sent\slash x}%
  \Test R{new\\line}{new\\line}%
  \Test S{foo\linebreak bar}{foo\linebreak bar}%
  \Test T{Virchow{sche}}{Virchow{sche}}%
  \Test U{A\soulomit{foo}B}{A\soulomit{foo}B}%
  \Test V{A\texttt{xyz}B}{A\texttt{xyz}B}%
  \Test W{ffi}{ffi}%
  \Test X{f{}f{}i}{f{}f{}i}%
  \Test Y{f\mbox{}f\mbox{}i}{f\mbox{}f\mbox{}i}%
  \Test Z{\c{\"a}}{{\c{\"a}}}%
%<*test2|test3>
  \let\abcx\UNDEFINED
  \def\abc{%
    \begingroup
      \def\abcx{x}%
      \abcx
    \endgroup
  }%
  \Test{AA}{[{x\abc x}]}{[{xxx}]}%
%</test2|test3>
%    \end{macrocode}
%
%    \begin{macrocode}
\end{document}
%    \end{macrocode}
%    \begin{macrocode}
%</test2|test3|test4|test5>
%    \end{macrocode}
%
% \section{Installation}
%
% \subsection{Download}
%
% \paragraph{Package.} This package is available on
% CTAN\footnote{\url{http://ctan.org/pkg/soulutf8}}:
% \begin{description}
% \item[\CTAN{macros/latex/contrib/oberdiek/soulutf8.dtx}] The source file.
% \item[\CTAN{macros/latex/contrib/oberdiek/soulutf8.pdf}] Documentation.
% \end{description}
%
%
% \paragraph{Bundle.} All the packages of the bundle `oberdiek'
% are also available in a TDS compliant ZIP archive. There
% the packages are already unpacked and the documentation files
% are generated. The files and directories obey the TDS standard.
% \begin{description}
% \item[\CTAN{install/macros/latex/contrib/oberdiek.tds.zip}]
% \end{description}
% \emph{TDS} refers to the standard ``A Directory Structure
% for \TeX\ Files'' (\CTAN{tds/tds.pdf}). Directories
% with \xfile{texmf} in their name are usually organized this way.
%
% \subsection{Bundle installation}
%
% \paragraph{Unpacking.} Unpack the \xfile{oberdiek.tds.zip} in the
% TDS tree (also known as \xfile{texmf} tree) of your choice.
% Example (linux):
% \begin{quote}
%   |unzip oberdiek.tds.zip -d ~/texmf|
% \end{quote}
%
% \paragraph{Script installation.}
% Check the directory \xfile{TDS:scripts/oberdiek/} for
% scripts that need further installation steps.
% Package \xpackage{attachfile2} comes with the Perl script
% \xfile{pdfatfi.pl} that should be installed in such a way
% that it can be called as \texttt{pdfatfi}.
% Example (linux):
% \begin{quote}
%   |chmod +x scripts/oberdiek/pdfatfi.pl|\\
%   |cp scripts/oberdiek/pdfatfi.pl /usr/local/bin/|
% \end{quote}
%
% \subsection{Package installation}
%
% \paragraph{Unpacking.} The \xfile{.dtx} file is a self-extracting
% \docstrip\ archive. The files are extracted by running the
% \xfile{.dtx} through \plainTeX:
% \begin{quote}
%   \verb|tex soulutf8.dtx|
% \end{quote}
%
% \paragraph{TDS.} Now the different files must be moved into
% the different directories in your installation TDS tree
% (also known as \xfile{texmf} tree):
% \begin{quote}
% \def\t{^^A
% \begin{tabular}{@{}>{\ttfamily}l@{ $\rightarrow$ }>{\ttfamily}l@{}}
%   soulutf8.sty & tex/generic/oberdiek/soulutf8.sty\\
%   soulutf8.pdf & doc/latex/oberdiek/soulutf8.pdf\\
%   test/soulutf8-test1.tex & doc/latex/oberdiek/test/soulutf8-test1.tex\\
%   test/soulutf8-test2.tex & doc/latex/oberdiek/test/soulutf8-test2.tex\\
%   test/soulutf8-test3.tex & doc/latex/oberdiek/test/soulutf8-test3.tex\\
%   test/soulutf8-test4.tex & doc/latex/oberdiek/test/soulutf8-test4.tex\\
%   test/soulutf8-test5.tex & doc/latex/oberdiek/test/soulutf8-test5.tex\\
%   soulutf8.dtx & source/latex/oberdiek/soulutf8.dtx\\
% \end{tabular}^^A
% }^^A
% \sbox0{\t}^^A
% \ifdim\wd0>\linewidth
%   \begingroup
%     \advance\linewidth by\leftmargin
%     \advance\linewidth by\rightmargin
%   \edef\x{\endgroup
%     \def\noexpand\lw{\the\linewidth}^^A
%   }\x
%   \def\lwbox{^^A
%     \leavevmode
%     \hbox to \linewidth{^^A
%       \kern-\leftmargin\relax
%       \hss
%       \usebox0
%       \hss
%       \kern-\rightmargin\relax
%     }^^A
%   }^^A
%   \ifdim\wd0>\lw
%     \sbox0{\small\t}^^A
%     \ifdim\wd0>\linewidth
%       \ifdim\wd0>\lw
%         \sbox0{\footnotesize\t}^^A
%         \ifdim\wd0>\linewidth
%           \ifdim\wd0>\lw
%             \sbox0{\scriptsize\t}^^A
%             \ifdim\wd0>\linewidth
%               \ifdim\wd0>\lw
%                 \sbox0{\tiny\t}^^A
%                 \ifdim\wd0>\linewidth
%                   \lwbox
%                 \else
%                   \usebox0
%                 \fi
%               \else
%                 \lwbox
%               \fi
%             \else
%               \usebox0
%             \fi
%           \else
%             \lwbox
%           \fi
%         \else
%           \usebox0
%         \fi
%       \else
%         \lwbox
%       \fi
%     \else
%       \usebox0
%     \fi
%   \else
%     \lwbox
%   \fi
% \else
%   \usebox0
% \fi
% \end{quote}
% If you have a \xfile{docstrip.cfg} that configures and enables \docstrip's
% TDS installing feature, then some files can already be in the right
% place, see the documentation of \docstrip.
%
% \subsection{Refresh file name databases}
%
% If your \TeX~distribution
% (\teTeX, \mikTeX, \dots) relies on file name databases, you must refresh
% these. For example, \teTeX\ users run \verb|texhash| or
% \verb|mktexlsr|.
%
% \subsection{Some details for the interested}
%
% \paragraph{Attached source.}
%
% The PDF documentation on CTAN also includes the
% \xfile{.dtx} source file. It can be extracted by
% AcrobatReader 6 or higher. Another option is \textsf{pdftk},
% e.g. unpack the file into the current directory:
% \begin{quote}
%   \verb|pdftk soulutf8.pdf unpack_files output .|
% \end{quote}
%
% \paragraph{Unpacking with \LaTeX.}
% The \xfile{.dtx} chooses its action depending on the format:
% \begin{description}
% \item[\plainTeX:] Run \docstrip\ and extract the files.
% \item[\LaTeX:] Generate the documentation.
% \end{description}
% If you insist on using \LaTeX\ for \docstrip\ (really,
% \docstrip\ does not need \LaTeX), then inform the autodetect routine
% about your intention:
% \begin{quote}
%   \verb|latex \let\install=y% \iffalse meta-comment
%
% File: soulutf8.dtx
% Version: 2016/05/16 v1.1
% Info: Permit use of UTF-8 characters in soul
%
% Copyright (C) 2007 by
%    Heiko Oberdiek <heiko.oberdiek at googlemail.com>
%    2016
%    https://github.com/ho-tex/oberdiek/issues
%
% This work may be distributed and/or modified under the
% conditions of the LaTeX Project Public License, either
% version 1.3c of this license or (at your option) any later
% version. This version of this license is in
%    http://www.latex-project.org/lppl/lppl-1-3c.txt
% and the latest version of this license is in
%    http://www.latex-project.org/lppl.txt
% and version 1.3 or later is part of all distributions of
% LaTeX version 2005/12/01 or later.
%
% This work has the LPPL maintenance status "maintained".
%
% This Current Maintainer of this work is Heiko Oberdiek.
%
% The Base Interpreter refers to any `TeX-Format',
% because some files are installed in TDS:tex/generic//.
%
% This work consists of the main source file soulutf8.dtx
% and the derived files
%    soulutf8.sty, soulutf8.pdf, soulutf8.ins, soulutf8.drv,
%    soulutf8-test1.tex, soulutf8-test2.tex, soulutf8-test3.tex,
%    soulutf8-test4.tex, soulutf8-test5.tex.
%
% Distribution:
%    CTAN:macros/latex/contrib/oberdiek/soulutf8.dtx
%    CTAN:macros/latex/contrib/oberdiek/soulutf8.pdf
%
% Unpacking:
%    (a) If soulutf8.ins is present:
%           tex soulutf8.ins
%    (b) Without soulutf8.ins:
%           tex soulutf8.dtx
%    (c) If you insist on using LaTeX
%           latex \let\install=y\input{soulutf8.dtx}
%        (quote the arguments according to the demands of your shell)
%
% Documentation:
%    (a) If soulutf8.drv is present:
%           latex soulutf8.drv
%    (b) Without soulutf8.drv:
%           latex soulutf8.dtx; ...
%    The class ltxdoc loads the configuration file ltxdoc.cfg
%    if available. Here you can specify further options, e.g.
%    use A4 as paper format:
%       \PassOptionsToClass{a4paper}{article}
%
%    Programm calls to get the documentation (example):
%       pdflatex soulutf8.dtx
%       makeindex -s gind.ist soulutf8.idx
%       pdflatex soulutf8.dtx
%       makeindex -s gind.ist soulutf8.idx
%       pdflatex soulutf8.dtx
%
% Installation:
%    TDS:tex/generic/oberdiek/soulutf8.sty
%    TDS:doc/latex/oberdiek/soulutf8.pdf
%    TDS:doc/latex/oberdiek/test/soulutf8-test1.tex
%    TDS:doc/latex/oberdiek/test/soulutf8-test2.tex
%    TDS:doc/latex/oberdiek/test/soulutf8-test3.tex
%    TDS:doc/latex/oberdiek/test/soulutf8-test4.tex
%    TDS:doc/latex/oberdiek/test/soulutf8-test5.tex
%    TDS:source/latex/oberdiek/soulutf8.dtx
%
%<*ignore>
\begingroup
  \catcode123=1 %
  \catcode125=2 %
  \def\x{LaTeX2e}%
\expandafter\endgroup
\ifcase 0\ifx\install y1\fi\expandafter
         \ifx\csname processbatchFile\endcsname\relax\else1\fi
         \ifx\fmtname\x\else 1\fi\relax
\else\csname fi\endcsname
%</ignore>
%<*install>
\input docstrip.tex
\Msg{************************************************************************}
\Msg{* Installation}
\Msg{* Package: soulutf8 2016/05/16 v1.1 Permit use of UTF-8 characters in soul (HO)}
\Msg{************************************************************************}

\keepsilent
\askforoverwritefalse

\let\MetaPrefix\relax
\preamble

This is a generated file.

Project: soulutf8
Version: 2016/05/16 v1.1

Copyright (C) 2007 by
   Heiko Oberdiek <heiko.oberdiek at googlemail.com>

This work may be distributed and/or modified under the
conditions of the LaTeX Project Public License, either
version 1.3c of this license or (at your option) any later
version. This version of this license is in
   http://www.latex-project.org/lppl/lppl-1-3c.txt
and the latest version of this license is in
   http://www.latex-project.org/lppl.txt
and version 1.3 or later is part of all distributions of
LaTeX version 2005/12/01 or later.

This work has the LPPL maintenance status "maintained".

This Current Maintainer of this work is Heiko Oberdiek.

The Base Interpreter refers to any `TeX-Format',
because some files are installed in TDS:tex/generic//.

This work consists of the main source file soulutf8.dtx
and the derived files
   soulutf8.sty, soulutf8.pdf, soulutf8.ins, soulutf8.drv,
   soulutf8-test1.tex, soulutf8-test2.tex, soulutf8-test3.tex,
   soulutf8-test4.tex, soulutf8-test5.tex.

\endpreamble
\let\MetaPrefix\DoubleperCent

\generate{%
  \file{soulutf8.ins}{\from{soulutf8.dtx}{install}}%
  \file{soulutf8.drv}{\from{soulutf8.dtx}{driver}}%
  \usedir{tex/generic/oberdiek}%
  \file{soulutf8.sty}{\from{soulutf8.dtx}{package}}%
%  \usedir{doc/latex/oberdiek/test}%
%  \file{soulutf8-test1.tex}{\from{soulutf8.dtx}{test1}}%
%  \file{soulutf8-test2.tex}{\from{soulutf8.dtx}{test2,utf8}}%
%  \file{soulutf8-test3.tex}{\from{soulutf8.dtx}{test3,utf8x}}%
%  \file{soulutf8-test4.tex}{\from{soulutf8.dtx}{test4,utf8,noetex}}%
%  \file{soulutf8-test5.tex}{\from{soulutf8.dtx}{test5,utf8x,noetex}}%
  \nopreamble
  \nopostamble
%  \usedir{source/latex/oberdiek/catalogue}%
%  \file{soulutf8.xml}{\from{soulutf8.dtx}{catalogue}}%
}

\catcode32=13\relax% active space
\let =\space%
\Msg{************************************************************************}
\Msg{*}
\Msg{* To finish the installation you have to move the following}
\Msg{* file into a directory searched by TeX:}
\Msg{*}
\Msg{*     soulutf8.sty}
\Msg{*}
\Msg{* To produce the documentation run the file `soulutf8.drv'}
\Msg{* through LaTeX.}
\Msg{*}
\Msg{* Happy TeXing!}
\Msg{*}
\Msg{************************************************************************}

\endbatchfile
%</install>
%<*ignore>
\fi
%</ignore>
%<*driver>
\NeedsTeXFormat{LaTeX2e}
\ProvidesFile{soulutf8.drv}%
  [2016/05/16 v1.1 Permit use of UTF-8 characters in soul (HO)]%
\documentclass{ltxdoc}
\usepackage{holtxdoc}[2011/11/22]
\begin{document}
  \DocInput{soulutf8.dtx}%
\end{document}
%</driver>
% \fi
%
%
% \CharacterTable
%  {Upper-case    \A\B\C\D\E\F\G\H\I\J\K\L\M\N\O\P\Q\R\S\T\U\V\W\X\Y\Z
%   Lower-case    \a\b\c\d\e\f\g\h\i\j\k\l\m\n\o\p\q\r\s\t\u\v\w\x\y\z
%   Digits        \0\1\2\3\4\5\6\7\8\9
%   Exclamation   \!     Double quote  \"     Hash (number) \#
%   Dollar        \$     Percent       \%     Ampersand     \&
%   Acute accent  \'     Left paren    \(     Right paren   \)
%   Asterisk      \*     Plus          \+     Comma         \,
%   Minus         \-     Point         \.     Solidus       \/
%   Colon         \:     Semicolon     \;     Less than     \<
%   Equals        \=     Greater than  \>     Question mark \?
%   Commercial at \@     Left bracket  \[     Backslash     \\
%   Right bracket \]     Circumflex    \^     Underscore    \_
%   Grave accent  \`     Left brace    \{     Vertical bar  \|
%   Right brace   \}     Tilde         \~}
%
% \GetFileInfo{soulutf8.drv}
%
% \title{The \xpackage{soulutf8} package}
% \date{2016/05/16 v1.1}
% \author{Heiko Oberdiek\thanks
% {Please report any issues at https://github.com/ho-tex/oberdiek/issues}\\
% \xemail{heiko.oberdiek at googlemail.com}}
%
% \maketitle
%
% \begin{abstract}
% This package extends package \xpackage{soul} and adds some support
% for UTF-8. Namely the input encodings \xfile{utf8.def}
% from package \xpackage{inputenc} and
% package \xpackage{ucs}'s \xfile{utf8x.def} are supported.
% \end{abstract}
%
% \tableofcontents
%
% \section{Documentation}
%
% This package \xpackage{soulutf8} does not have own options and does not
% define new user commands.
% Any option is passed to package \xpackage{soul} \cite{soul}
% that is loaded first.
% Then some internal macros of \xpackage{soul} are redefined to add
% support for UTF-8.
% The following input encodings are supported:
% \begin{quote}
%   \begin{tabular}{@{}lll@{}}
%     \texttt{utf8} & \LaTeX\ base &
%        \xfile{TDS:tex/latex/base/utf8.def} \cite{utf8}\\
%     \texttt{utf8x} & Package \xpackage{ucs} &
%       \xfile{TDS:tex/latex/ucs/utf8x.def} \cite{ucs}\\
%   \end{tabular}
% \end{quote}
% UTF-8 byte sequences are added as token group to a word, even
% if these UTF-8 characters are some kind of hyphen or space.
% As exception the following three Unicode characters are handled specially:
% \begin{quote}
%   \begin{tabular}{lll}
%     Slot & Name & Action\\
%     \hline
%     \texttt{U+00A0} & NO-BREAK SPACE & like |~|\\
%     \texttt{U+2013} & EN DASH & |--|\\
%     \texttt{U+2014} & EM DASH & |---|
%   \end{tabular}
%
% \subsection{Patch}
%
% Also package \xpackage{soulutf8} tries to patch package \xpackage{soul}
% to improve its behaviour:
% \begin{itemize}
% \item
%   A problem with additional levels of curly braces is fixed.
%   As advantage more implicite kernings are detected.
%   However, the result may be incompatible with the
%   original behaviour of package \xpackage{soul} because
%   of these respected implicite kernings.
% \item
%   \eTeX\ , especially \cs{unexpanded} is supported.
%   This allows a better protection of token groups
%   (|\mbox{|\dots|}|, math, \dots).
% \end{itemize}
% \end{quote}
%
% \subsection{Future}
%
% Currently package \xpackage{soul} does not seem to be
% maintained. Nevertheless if there will be a new version that
% adds support for UTF-8, then this package may become
% obsolete.
%
% \StopEventually{
% }
%
% \section{Implementation}
%
%    \begin{macrocode}
%<*package>
%    \end{macrocode}
%
% \subsection{Reload check and package identification}
%    Reload check, especially if the package is not used with \LaTeX.
%    \begin{macrocode}
\begingroup\catcode61\catcode48\catcode32=10\relax%
  \catcode13=5 % ^^M
  \endlinechar=13 %
  \catcode35=6 % #
  \catcode39=12 % '
  \catcode44=12 % ,
  \catcode45=12 % -
  \catcode46=12 % .
  \catcode58=12 % :
  \catcode64=11 % @
  \catcode123=1 % {
  \catcode125=2 % }
  \expandafter\let\expandafter\x\csname ver@soulutf8.sty\endcsname
  \ifx\x\relax % plain-TeX, first loading
  \else
    \def\empty{}%
    \ifx\x\empty % LaTeX, first loading,
      % variable is initialized, but \ProvidesPackage not yet seen
    \else
      \expandafter\ifx\csname PackageInfo\endcsname\relax
        \def\x#1#2{%
          \immediate\write-1{Package #1 Info: #2.}%
        }%
      \else
        \def\x#1#2{\PackageInfo{#1}{#2, stopped}}%
      \fi
      \x{soulutf8}{The package is already loaded}%
      \aftergroup\endinput
    \fi
  \fi
\endgroup%
%    \end{macrocode}
%    Package identification:
%    \begin{macrocode}
\begingroup\catcode61\catcode48\catcode32=10\relax%
  \catcode13=5 % ^^M
  \endlinechar=13 %
  \catcode35=6 % #
  \catcode39=12 % '
  \catcode40=12 % (
  \catcode41=12 % )
  \catcode44=12 % ,
  \catcode45=12 % -
  \catcode46=12 % .
  \catcode47=12 % /
  \catcode58=12 % :
  \catcode64=11 % @
  \catcode91=12 % [
  \catcode93=12 % ]
  \catcode123=1 % {
  \catcode125=2 % }
  \expandafter\ifx\csname ProvidesPackage\endcsname\relax
    \def\x#1#2#3[#4]{\endgroup
      \immediate\write-1{Package: #3 #4}%
      \xdef#1{#4}%
    }%
  \else
    \def\x#1#2[#3]{\endgroup
      #2[{#3}]%
      \ifx#1\@undefined
        \xdef#1{#3}%
      \fi
      \ifx#1\relax
        \xdef#1{#3}%
      \fi
    }%
  \fi
\expandafter\x\csname ver@soulutf8.sty\endcsname
\ProvidesPackage{soulutf8}%
  [2016/05/16 v1.1 Permit use of UTF-8 characters in soul (HO)]%
%    \end{macrocode}
%
% \subsection{Catcodes}
%
%    \begin{macrocode}
\begingroup\catcode61\catcode48\catcode32=10\relax%
  \catcode13=5 % ^^M
  \endlinechar=13 %
  \catcode123=1 % {
  \catcode125=2 % }
  \catcode64=11 % @
  \def\x{\endgroup
    \expandafter\edef\csname SOuL@AtEnd\endcsname{%
      \endlinechar=\the\endlinechar\relax
      \catcode13=\the\catcode13\relax
      \catcode32=\the\catcode32\relax
      \catcode35=\the\catcode35\relax
      \catcode61=\the\catcode61\relax
      \catcode64=\the\catcode64\relax
      \catcode123=\the\catcode123\relax
      \catcode125=\the\catcode125\relax
    }%
  }%
\x\catcode61\catcode48\catcode32=10\relax%
\catcode13=5 % ^^M
\endlinechar=13 %
\catcode35=6 % #
\catcode64=11 % @
\catcode123=1 % {
\catcode125=2 % }
\def\TMP@EnsureCode#1#2{%
  \edef\SOuL@AtEnd{%
    \SOuL@AtEnd
    \catcode#1=\the\catcode#1\relax
  }%
  \catcode#1=#2\relax
}
\TMP@EnsureCode{10}{12}% ^^J
\TMP@EnsureCode{33}{12}% !
\TMP@EnsureCode{34}{12}% "
\TMP@EnsureCode{36}{3}% $
\TMP@EnsureCode{39}{12}% '
\TMP@EnsureCode{40}{12}% (
\TMP@EnsureCode{41}{12}% )
\TMP@EnsureCode{42}{12}% *
\TMP@EnsureCode{43}{12}% +
\TMP@EnsureCode{44}{12}% ,
\TMP@EnsureCode{45}{12}% -
\TMP@EnsureCode{46}{12}% .
\TMP@EnsureCode{47}{12}% /
\TMP@EnsureCode{58}{12}% :
\TMP@EnsureCode{60}{12}% <
\TMP@EnsureCode{62}{12}% >
\TMP@EnsureCode{91}{12}% [
\TMP@EnsureCode{93}{12}% ]
\TMP@EnsureCode{94}{7}% ^
\TMP@EnsureCode{96}{12}% `
\TMP@EnsureCode{126}\active % ~
\TMP@EnsureCode{128}{12}% ^^80
\TMP@EnsureCode{147}{12}% ^^93
\TMP@EnsureCode{148}{12}% ^^94
\TMP@EnsureCode{160}{12}% ^^a0
\TMP@EnsureCode{194}{12}% ^^c2
\TMP@EnsureCode{226}{12}% ^^e2
\edef\SOuL@AtEnd{\SOuL@AtEnd\noexpand\endinput}
%    \end{macrocode}
%
% \subsection{Loading packages}
%
%    Package \xpackage{soul} uses \cs{documentclass} to detect \LaTeX.
%    \begin{macrocode}
\ifx\documentclass\@undefined
%    \end{macrocode}
%
% \subsubsection{\plainTeX}
%
%    First we check, whether package \xpackage{soul} is already loaded.
%    \begin{macrocode}
  \expandafter\ifx\csname SOUL@\endcsname\relax
%    \end{macrocode}
%    In case of \plainTeX\ package \xpackage{soul} defines some
%    macros in a simple manner that will break the definitions of
%    \xfile{miniltx.tex}, for example. Therefore these macros
%    are first saved and restored afterwards.
%    \begin{macrocode}
    \let\SOuL@orgDeclareRobustCommand\DeclareRobustCommand
    \let\SOuL@orgnewcommand          \newcommand
    \let\SOuL@orgDeclareOption       \DeclareOption
    \let\SOuL@orgPackageError        \PackageError
    \def\SOuL@restorelatexcmds{%
      \let\DeclareRobustCommand\SOuL@orgDeclareRobustCommand
      \let\newcommand          \SOuL@orgnewcommand
      \let\DeclareOption       \SOuL@orgDeclareOption
      \let\PackageError        \SOuL@orgPackageError
    }%
    \input soul.sty\relax
    \SOuL@restorelatexcmds
  \fi
%    \end{macrocode}
%    \begin{macro}{\SOUL@error}
%     Package \xpackage{soul}'s use of \cs{PackageError} is replaced
%     by \cs{@PackageError} of package \xpackage{infwarerr}.
%    \begin{macrocode}
  \input infwarerr.sty\relax
  \let\SOuL@orgSOUL@error\SOUL@error
  \def\SOUL@error{%
    \begingroup
      \let\PackageError\@PackageError
      \SOuL@orgSOUL@error
    \endgroup
  }%
%    \end{macrocode}
%    \end{macro}
%    \begin{macrocode}
  \input etexcmds.sty\relax
%    \end{macrocode}
%    \begin{macro}{\@onelevel@sanitize}
%    Define \LaTeX's \cs{@onelevel@sanitize} if not already available.
%    \begin{macrocode}
  \expandafter\ifx\csname @onelevel@sanitize\endcsname\relax
    \def\@onelevel@sanitize#1{%
      \edef#1{%
        \expandafter\strip@prefix\meaning#1%
      }%
    }%
%    \end{macrocode}
%    \end{macro}
%    \begin{macro}{\strip@prefix}
%    \begin{macrocode}
    \def\strip@prefix#1>{}%
%    \end{macrocode}
%    \end{macro}
%    \begin{macrocode}
  \fi
\else
%    \end{macrocode}
%
% \subsubsection{\LaTeX}
%    \begin{macrocode}
  \DeclareOption*{\PassOptionsToPackage{\CurrentOption}{soul}}%
  \ProcessOptions\relax
  \RequirePackage{soul}[2003/11/17]%
  \RequirePackage{infwarerr}[2016/05/16]%
  \RequirePackage{etexcmds}[2016/05/16]%
\fi
%    \end{macrocode}
%
% \subsubsection{\eTeX}
%
%    In \plainTeX\ command \cs{+} is an \emph{outer} macro.
%    Therefore numbers are used to avoid problems.
%    \begin{macrocode}
\ifetex@unexpanded
  \catcode33=14 % '!': comment
  \catcode43=9  % '+': ignore
\else
  \catcode33=9  % '!': ignore
  \catcode43=14 % '+': comment
\fi
%    \end{macrocode}
%
% \subsection{Macro for redefinitions}
%
%    \begin{macro}{\SOuL@redefine}
%    \begin{macrocode}
\def\SOuL@redefine#1{%
  \begingroup
    \def\SOuL@cmd{#1}%
    \afterassignment\SOuL@cmdcheck
    \def\SOuL@temp
}
%    \end{macrocode}
%    \end{macro}
%    \begin{macro}{\SOuL@cmdcheck}
%    \begin{macrocode}
\def\SOuL@cmdcheck{%
    \expandafter\ifx\SOuL@cmd\SOuL@temp
    \else
      \edef\SOuL@temp*{\expandafter\string\SOuL@cmd}%
      \@PackageWarningNoLine{soulutf8}{%
        Command \SOuL@temp* has changed.\MessageBreak
        Supported versions of package `soul': 2003/11/17.\MessageBreak
        Depending on the unknown changes the redefinition\MessageBreak
        of \SOuL@temp* may not behave correctly%
      }%
    \fi
  \expandafter\endgroup
  \expandafter\def\SOuL@cmd
}
%    \end{macrocode}
%    \end{macro}
%
% \subsection{Redefinition of \cs{SOUL@eval}}
%
%    \begin{macro}{\SOUL@eval}
%    Macro \cs{SOUL@eval} is redefined to add detection of
%    the first byte of a UTF-8 sequence. Because \cs{SOUL@eval}
%    is overwritten, a warning is issued, if the contents of
%    \cs{SOUL@eval} is not as expected.
%    \begin{macrocode}
\SOuL@redefine\SOUL@eval{%
%    \end{macrocode}
%    First the expected definition.
%    \begin{macrocode}
  \def\SOUL@n*##1{\SOUL@scan}%
  \if\noexpand\SOUL@@\SOUL@spc
  \else
    \SOUL@ignorespacesfalse
  \fi
  \ifnum\SOUL@minus=\thr@@
    \SOUL@flushminus
  \else\ifnum\SOUL@comma=\tw@
    \SOUL@flushcomma
  \else\ifnum\SOUL@apo=\tw@
    \SOUL@flushapo
  \else\ifnum\SOUL@grave=\tw@
    \SOUL@flushgrave
  \fi\fi\fi\fi
  \ifx\SOUL@@-\else\SOUL@flushminus\fi
  \ifx\SOUL@@,\else\SOUL@flushcomma\fi
  \ifx\SOUL@@'\else\SOUL@flushapo\fi
  \ifx\SOUL@@`\else\SOUL@flushgrave\fi
  \ifx\SOUL@@-%
    \advance\SOUL@minus\@ne
  \else\ifx\SOUL@@,%
    \advance\SOUL@comma\@ne
  \else\ifx\SOUL@@'%
    \advance\SOUL@apo\@ne
  \else\ifx\SOUL@@`%
    \advance\SOUL@grave\@ne
  \else
    \SOUL@flushminus
    \SOUL@flushcomma
    \SOUL@flushapo
    \SOUL@flushgrave
    \ifx\SOUL@@\SOUL@stop
      \def\SOUL@n*{%
        \SOUL@doword
        \SOUL@eventuallyexhyphen\null
      }%
    \else\ifx\SOUL@@\par
      \def\SOUL@n*\par{\par\leavevmode\SOUL@scan}%
    \else\if\noexpand\SOUL@@\SOUL@spc
      \SOUL@doword
      \SOUL@eventuallyexhyphen\null
      \ifSOUL@ignorespaces
      \else
        \SOUL@everyspace{}%
      \fi
      \def\SOUL@n* {\SOUL@scan}%
    \else\ifx\SOUL@@\\%
      \SOUL@doword
      \SOUL@eventuallyexhyphen\null
      \SOUL@everyspace{\unskip\nobreak\hfil\break}%
      \SOUL@ignorespacestrue
    \else\ifx\SOUL@@~%
      \SOUL@doword
      \SOUL@eventuallyexhyphen\null
      \SOUL@everyspace{\nobreak}%
    \else\ifx\SOUL@@\slash
      \SOUL@doword
      \SOUL@eventuallyexhyphen{/}%
      \SOUL@exhyphen{/}%
    \else\ifx\SOUL@@\mbox
      \def\SOUL@n*{\SOUL@addprotect}%
    \else\ifx\SOUL@@\hbox
      \def\SOUL@n*{\SOUL@addprotect}%
    \else\ifx\SOUL@@\soulomit
      \def\SOUL@n*\soulomit##1{%
        \SOUL@doword
        {\spaceskip\SOUL@spaceskip##1}%
        \SOUL@scan
      }%
    \else\ifx\SOUL@@\break
      \SOUL@doword
      \break
    \else\ifx\SOUL@@\linebreak
      \SOUL@doword
      \SOUL@everyspace{\linebreak}%
    \else\ifcat\bgroup\noexpand\SOUL@@
      \def\SOUL@n*{\SOUL@addgroup{}}%
    \else\ifcat$\noexpand\SOUL@@
      \def\SOUL@n*{\SOUL@addmath}%
    \else
      \def\SOUL@n*{\SOUL@dotoken}%
    \fi\fi\fi\fi\fi\fi\fi\fi\fi\fi\fi\fi\fi
  \fi\fi\fi\fi
  \SOUL@n*%
}{%
%    \end{macrocode}
%    Now the redefined version follows.
%    \begin{macrocode}
  \def\SOUL@n*##1{\SOUL@scan}%
  \if\noexpand\SOUL@@\SOUL@spc
  \else
    \SOUL@ignorespacesfalse
  \fi
  \ifnum\SOUL@minus=\thr@@
    \SOUL@flushminus
  \else\ifnum\SOUL@comma=\tw@
    \SOUL@flushcomma
  \else\ifnum\SOUL@apo=\tw@
    \SOUL@flushapo
  \else\ifnum\SOUL@grave=\tw@
    \SOUL@flushgrave
  \fi\fi\fi\fi
  \ifx\SOUL@@-\else\SOUL@flushminus\fi
  \ifx\SOUL@@,\else\SOUL@flushcomma\fi
  \ifx\SOUL@@'\else\SOUL@flushapo\fi
  \ifx\SOUL@@`\else\SOUL@flushgrave\fi
  \ifx\SOUL@@-%
    \advance\SOUL@minus\@ne
  \else\ifx\SOUL@@,%
    \advance\SOUL@comma\@ne
  \else\ifx\SOUL@@'%
    \advance\SOUL@apo\@ne
  \else\ifx\SOUL@@`%
    \advance\SOUL@grave\@ne
  \else
    \SOUL@flushminus
    \SOUL@flushcomma
    \SOUL@flushapo
    \SOUL@flushgrave
    \ifx\SOUL@@\SOUL@stop
      \def\SOUL@n*{%
          \SOUL@doword
          \SOUL@eventuallyexhyphen\null
      }%
    \else\ifx\SOUL@@\par
      \def\SOUL@n*\par{\par\leavevmode\SOUL@scan}%
    \else\if\noexpand\SOUL@@\SOUL@spc
      \SOUL@doword
      \SOUL@eventuallyexhyphen\null
      \ifSOUL@ignorespaces
      \else
        \SOUL@everyspace{}%
      \fi
      \def\SOUL@n* {\SOUL@scan}%
    \else\ifx\SOUL@@\\%
      \SOUL@doword
      \SOUL@eventuallyexhyphen\null
      \SOUL@everyspace{\unskip\nobreak\hfil\break}%
      \SOUL@ignorespacestrue
    \else\ifx\SOUL@@~%
      \SOUL@doword
      \SOUL@eventuallyexhyphen\null
      \SOUL@everyspace{\nobreak}%
    \else\ifx\SOUL@@\slash
      \SOUL@doword
      \SOUL@eventuallyexhyphen{/}%
      \SOUL@exhyphen{/}%
    \else\ifx\SOUL@@\mbox
      \def\SOUL@n*{\SOUL@addprotect}%
    \else\ifx\SOUL@@\hbox
      \def\SOUL@n*{\SOUL@addprotect}%
    \else\ifx\SOUL@@\soulomit
      \def\SOUL@n*\soulomit##1{%
        \SOUL@doword
        {\spaceskip\SOUL@spaceskip##1}%
        \SOUL@scan
      }%
    \else\ifx\SOUL@@\break
      \SOUL@doword
      \break
    \else\ifx\SOUL@@\linebreak
      \SOUL@doword
      \SOUL@everyspace{\linebreak}%
    \else\ifcat\bgroup\noexpand\SOUL@@
      \def\SOUL@n*{\SOUL@addgroup{}}%
    \else\ifcat$\noexpand\SOUL@@
      \def\SOUL@n*{\SOUL@addmath}%
    \else
%    \end{macrocode}
%    The current token is examined to detect the start of
%    a UTF-8 sequence.
%    \begin{macrocode}
      \SOuL@analyzeutfviii
      \ifcase\SOuL@octets
        \SOuL@analyzeutfviiix
      \fi
      \ifcase\SOuL@octets
        \def\SOUL@n*{\SOUL@dotoken}%
      \or % 1
      \or % 2
        \def\SOUL@n*{\SOuL@addtwooctets}%
      \or % 3
        \def\SOUL@n*{\SOuL@addthreeoctets}%
      \or % 4
        \def\SOUL@n*{\SOuL@addfouroctets}%
      \fi
    \fi\fi\fi\fi\fi\fi\fi\fi\fi\fi\fi\fi\fi
  \fi\fi\fi\fi
  \SOUL@n*%
}
%    \end{macrocode}
%    \end{macro}
%
% \subsection{UTF-8 analysis}
%
% \subsubsection{Help strings}
%
%    \begin{macrocode}
\def\SOuL@defsanitizedstring#1#2{%
  \expandafter\def\csname SOuL@string#1\endcsname{#2}%
  \expandafter\@onelevel@sanitize\csname SOuL@string#1\endcsname
}
\SOuL@defsanitizedstring{UTFviii}{UTFviii@}
\SOuL@defsanitizedstring{octets}{@octets}
\SOuL@defsanitizedstring{two}{two}
\SOuL@defsanitizedstring{three}{three}
\SOuL@defsanitizedstring{four}{four}
\SOuL@defsanitizedstring{macrocolon}{macro:}
\SOuL@defsanitizedstring{csnameu}{csname u8-}
\SOuL@defsanitizedstring{undeferr}{utf@viii@undeferr}
\def\SOuL@stringendash{^^e2^^80^^93}
\def\SOuL@stringemdash{^^e2^^80^^94}
\def\SOuL@stringnobreakspace{^^c2^^a0}
\edef\SOuL@charhash{\string #}
\edef\SOuL@chartwo{\string 2}
\edef\SOuL@charthree{\string 3}
\def\SOuL@empty{}
%    \end{macrocode}
%
% \subsubsection{Support for \xfile{utf8.def}}
%
%    \begin{macro}{\SOuL@analyzeutfviii}
%    \begin{macrocode}
\begingroup
  \edef\x{\endgroup
    \def\noexpand\SOuL@analyzeutfviii{%
      \noexpand\expandafter\noexpand\SOuL@checkutfviii
      \noexpand\meaning\noexpand\SOUL@@
      \SOuL@stringUTFviii\SOuL@stringoctets
      \noexpand\@nil
    }%
    \def\noexpand\SOuL@checkutfviii
      ##1\SOuL@stringUTFviii##2\SOuL@stringoctets##3\noexpand\@nil
  }%
\x{%
  \def\SOuL@temp{#2}%
  \chardef\SOuL@octets=%
      \ifx\SOuL@temp\SOuL@stringtwo
        \tw@
      \else\ifx\SOuL@temp\SOuL@stringthree
        \thr@@
      \else\ifx\SOuL@temp\SOuL@stringfour
        4 %
      \else
        \z@
      \fi\fi\fi
}
%    \end{macrocode}
%    \end{macro}
%
% \subsubsection{Support for \xfile{utf8x.def}}
%
%    \begin{macro}{\SOuL@analyzeutfviiix}
%    \begin{macrocode}
\begingroup
  \edef\x{\endgroup
    \def\noexpand\SOuL@analyzeutfviiix{%
      \noexpand\expandafter\noexpand\SOuL@checkutfviiix
      \noexpand\meaning\noexpand\SOUL@@
      \SOuL@stringmacrocolon\SOuL@charhash1{}{}{}{}%
      \SOuL@stringcsnameu\SOuL@stringundeferr
      \noexpand\@nil
    }%
%    \end{macrocode}
%    \end{macro}
%    \begin{macro}{\SOuL@checkutfviiix}
%    \begin{macrocode}
    \def\noexpand\SOuL@checkutfviiix
      ##1\SOuL@stringmacrocolon\SOuL@charhash1##2##3##4##5##6%
      \SOuL@stringcsnameu##7\SOuL@stringundeferr##8\noexpand\@nil
  }%
\x{%
  \def\SOuL@temp{#7}%
  \ifx\SOuL@temp\SOuL@empty
    \chardef\SOuL@octets=\z@
  \else
    \def\SOuL@temp{#5}%
    \ifx\SOuL@temp\SOuL@charthree
      \chardef\SOuL@octets=4 %
    \else
      \def\SOuL@temp{#3}%
      \ifx\SOuL@temp\SOuL@chartwo
        \chardef\SOuL@octets=\thr@@
      \else
        \chardef\SOuL@octets=\tw@
      \fi
    \fi
  \fi
}
%    \end{macrocode}
%    \end{macro}
%
% \subsection{Actions for UTF-8 sequences}
%
%    \begin{macro}{\SOuL@addtwooctets}
%    \begin{macrocode}
\def\SOuL@addtwooctets#1#2{%
  \def\SOuL@temp{#1#2}%
  \@onelevel@sanitize\SOuL@temp
  \ifx\SOuL@temp\SOuL@stringnobreakspace
    \SOUL@doword
    \SOUL@eventuallyexhyphen\null
    \SOUL@everyspace{\nobreak}%
    \let\SOuL@next\SOUL@scan
  \else
    \def\SOuL@next{%
!     \SOUL@addtoken{{\noexpand#1\noexpand#2}}%
+     \SOUL@addtoken{{\etex@unexpanded{#1#2}}}%
    }%
  \fi
  \SOuL@next
}
%    \end{macrocode}
%    \end{macro}
%    \begin{macro}{\SOuL@addthreeoctets}
%    \begin{macrocode}
\def\SOuL@addthreeoctets#1#2#3{%
  \def\SOuL@temp{#1#2#3}%
  \@onelevel@sanitize\SOuL@temp
  \ifx\SOuL@temp\SOuL@stringendash
    \SOUL@doword
    \SOUL@eventuallyexhyphen{-}%
    \SOUL@exhyphen{--}%
    \let\SOuL@next\SOUL@scan
  \else
    \ifx\SOuL@temp\SOuL@stringemdash
      \SOUL@doword
      \SOUL@eventuallyexhyphen{-}%
      \SOUL@exhyphen{---}%
      \let\SOuL@next\SOUL@scan
    \else
      \def\SOuL@next{%
!       \SOUL@addtoken{{\noexpand#1\noexpand#2\noexpand#3}}%
+       \SOUL@addtoken{{\etex@unexpanded{#1#2#3}}}%
      }%
    \fi
  \fi
  \SOuL@next
}
%    \end{macrocode}
%    \end{macro}
%    \begin{macro}{\SOuL@addfouroctets}
%    \begin{macrocode}
\def\SOuL@addfouroctets#1#2#3#4{%
! \SOUL@addtoken{{\noexpand#1\noexpand#2\noexpand#3\noexpand#4}}%
+ \SOUL@addtoken{{\etex@unexpanded{#1#2#3#4}}}%
}
%    \end{macrocode}
%    \end{macro}
%
% \subsubsection{Redefinition of \cs{SOUL@splittoken}}
%
%    \begin{macro}{\SOUL@splittoken}
%    Macro \cs{SOUL@splittoken} separates the first token
%    or token group from a word and redefines the word to contain
%    the remaining tokens. However if the remaining tokens are
%    a token group, then the curly braces will be removed and
%    the token group is splitted by the next call of \cs{SOUL@splittoken}.
%    The redefinition avoids the removal of curly braces around the
%    remaining tokens.
%    \begin{macrocode}
\SOuL@redefine\SOUL@splittoken#1#2\SOUL@stop{%
  \global\SOUL@token={#1}%
  \global\SOUL@word={#2}%
}#1{%
  \global\SOUL@token={#1}%
  \SOuL@remainingtoken\relax
}
%    \end{macrocode}
%    \end{macro}
%    \begin{macro}{\SOuL@remainingtoken}
%    \begin{macrocode}
\def\SOuL@remainingtoken#1\SOUL@stop{%
  \global\SOUL@word=\expandafter{\@gobble#1}%
}
%    \end{macrocode}
%    \end{macro}
%
% \subsection{Patches}
%
%    The fixed \cs{SOUL@splittoken} allows to remove the double
%    sets of curly braces in other macros of package \xpackage{soul}.
%    The benefit is that implicite kernings are more often detected
%    and fixes a bug in package \xpackage{soul}. The disadvantage is
%    incompatibility. The width of the resulting strings may change.
%    \begin{macro}{\SOUL@flushcomma}
%    \begin{macrocode}
\SOuL@redefine\SOUL@flushcomma{%
  \ifcase\SOUL@comma
  \or
    \edef\x{\SOUL@word={\the\SOUL@word,}}\x
  \or
    \edef\x{\SOUL@word={\the\SOUL@word{{,,}}}}\x
  \fi
  \SOUL@comma\z@
}{%
  \ifcase\SOUL@comma
  \or
    \edef\x{\SOUL@word={\the\SOUL@word,}}\x
  \or
    \edef\x{\SOUL@word={\the\SOUL@word{,,}}}\x
  \fi
  \SOUL@comma\z@
}
%    \end{macrocode}
%    \end{macro}
%    \begin{macro}{\SOUL@flushapo}
%    \begin{macrocode}
\SOuL@redefine\SOUL@flushapo{%
  \ifcase\SOUL@apo
  \or
    \edef\x{\SOUL@word={\the\SOUL@word'}}\x
  \or
    \edef\x{\SOUL@word={\the\SOUL@word{{''}}}}\x
  \fi
  \SOUL@apo\z@
}{%
  \ifcase\SOUL@apo
  \or
    \edef\x{\SOUL@word={\the\SOUL@word'}}\x
  \or
    \edef\x{\SOUL@word={\the\SOUL@word{''}}}\x
  \fi
  \SOUL@apo\z@
}
%    \end{macrocode}
%    \end{macro}
%    \begin{macro}{\SOUL@flushgrave}
%    \begin{macrocode}
\SOuL@redefine\SOUL@flushgrave{%
  \ifcase\SOUL@grave
  \or
    \edef\x{\SOUL@word={\the\SOUL@word`}}\x
  \or
    \edef\x{\SOUL@word={\the\SOUL@word{{``}}}}\x
  \fi
  \SOUL@grave\z@
}{%
  \ifcase\SOUL@grave
  \or
    \edef\x{\SOUL@word={\the\SOUL@word`}}\x
  \or
    \edef\x{\SOUL@word={\the\SOUL@word{``}}}\x
  \fi
  \SOUL@grave\z@
}
%    \end{macrocode}
%    \end{macro}
%    \begin{macro}{\SOUL@addgroup}
%    \begin{macrocode}
\SOuL@redefine\SOUL@addgroup#1#2{%
  {%
    \let\protect\noexpand
    \edef\x{%
      \global\SOUL@word={%
        \the\SOUL@word
        {{\noexpand#1#2}}%
      }%
    }%
    \x
  }%
  \SOUL@scan
}#1#2{%
  \begingroup
    \let\protect\noexpand
    \edef\x{\endgroup
      \SOUL@word={%
        \the\SOUL@word
!       {\noexpand#1{#2}}%
+       {\etex@unexpanded{#1{#2}}}%
      }%
    }%
  \x
  \SOUL@scan
}
%    \end{macrocode}
%    \end{macro}
%    \begin{macro}{\SOUL@addmath}
%    \begin{macrocode}
\SOuL@redefine\SOUL@addmath$#1${%
  {%
    \let\protect\noexpand
    \edef\x{%
      \global\SOUL@word={%
        \the\SOUL@word
        {{\hbox{$#1$}}}%
      }%
    }%
    \x
  }%
  \SOUL@scan
}$#1${%
  \begingroup
    \let\protect\noexpand
    \edef\x{\endgroup
      \SOUL@word={%
        \the\SOUL@word
!       {\hbox{$#1$}}%
+       {\etex@unexpanded{\hbox{$#1$}}}%
      }%
    }%
  \x
  \SOUL@scan
}
%    \end{macrocode}
%    \end{macro}
%    \begin{macro}{\SOUL@addprotect}
%    \begin{macrocode}
\SOuL@redefine\SOUL@addprotect#1#2{%
  {%
    \let\protect\noexpand
    \edef\x{%
      \global\SOUL@word={%
        \the\SOUL@word
        {{\hbox{#2}}}%
      }%
    }%
    \x
  }%
  \SOUL@scan
}#1#2{%
  \begingroup
    \let\protect\noexpand
    \edef\x{\endgroup
      \SOUL@word={%
        \the\SOUL@word
!       {\hbox{#2}}%
+       {\etex@unexpanded{\hbox{#2}}}%
      }%
    }%
  \x
  \SOUL@scan
}
%    \end{macrocode}
%    \end{macro}
%
%    \begin{macro}{\SOUL@addtoken}
%    \begin{macrocode}
+ \SOuL@redefine\SOUL@addtoken#1{%
+   \edef\x{%
+     \SOUL@word={%
+       \the\SOUL@word
+       \noexpand#1%
+     }%
+   }%
+   \x
+   \SOUL@scan
+ }#1{%
+   \edef\x{%
+     \SOUL@word={%
+       \the\SOUL@word
+       \etex@unexpanded{#1}%
+     }%
+   }%
+   \x
+   \SOUL@scan
+ }%
%    \end{macrocode}
%    \end{macro}
%
%    \begin{macrocode}
\SOuL@AtEnd%
%    \end{macrocode}
%
%    \begin{macrocode}
%</package>
%    \end{macrocode}
%
% \section{Test}
%
% \subsection{Catcode checks for loading}
%
%    \begin{macrocode}
%<*test1>
%    \end{macrocode}
%    \begin{macrocode}
\catcode`\{=1 %
\catcode`\}=2 %
\catcode`\#=6 %
\catcode`\@=11 %
\expandafter\ifx\csname count@\endcsname\relax
  \countdef\count@=255 %
\fi
\expandafter\ifx\csname @gobble\endcsname\relax
  \long\def\@gobble#1{}%
\fi
\expandafter\ifx\csname @firstofone\endcsname\relax
  \long\def\@firstofone#1{#1}%
\fi
\expandafter\ifx\csname loop\endcsname\relax
  \expandafter\@firstofone
\else
  \expandafter\@gobble
\fi
{%
  \def\loop#1\repeat{%
    \def\body{#1}%
    \iterate
  }%
  \def\iterate{%
    \body
      \let\next\iterate
    \else
      \let\next\relax
    \fi
    \next
  }%
  \let\repeat=\fi
}%
\def\RestoreCatcodes{}
\count@=0 %
\loop
  \edef\RestoreCatcodes{%
    \RestoreCatcodes
    \catcode\the\count@=\the\catcode\count@\relax
  }%
\ifnum\count@<255 %
  \advance\count@ 1 %
\repeat

\def\RangeCatcodeInvalid#1#2{%
  \count@=#1\relax
  \loop
    \catcode\count@=15 %
  \ifnum\count@<#2\relax
    \advance\count@ 1 %
  \repeat
}
\def\RangeCatcodeCheck#1#2#3{%
  \count@=#1\relax
  \loop
    \ifnum#3=\catcode\count@
    \else
      \errmessage{%
        Character \the\count@\space
        with wrong catcode \the\catcode\count@\space
        instead of \number#3%
      }%
    \fi
  \ifnum\count@<#2\relax
    \advance\count@ 1 %
  \repeat
}
\def\space{ }
\expandafter\ifx\csname LoadCommand\endcsname\relax
  \def\LoadCommand{\input soulutf8.sty\relax}%
\fi
\def\Test{%
  \RangeCatcodeInvalid{0}{47}%
  \RangeCatcodeInvalid{58}{64}%
  \RangeCatcodeInvalid{91}{96}%
  \RangeCatcodeInvalid{123}{255}%
  \catcode`\@=12 %
  \catcode`\\=0 %
  \catcode`\%=14 %
  \LoadCommand
  \RangeCatcodeCheck{0}{36}{15}%
  \RangeCatcodeCheck{37}{37}{14}%
  \RangeCatcodeCheck{38}{47}{15}%
  \RangeCatcodeCheck{48}{57}{12}%
  \RangeCatcodeCheck{58}{63}{15}%
  \RangeCatcodeCheck{64}{64}{12}%
  \RangeCatcodeCheck{65}{90}{11}%
  \RangeCatcodeCheck{91}{91}{15}%
  \RangeCatcodeCheck{92}{92}{0}%
  \RangeCatcodeCheck{93}{96}{15}%
  \RangeCatcodeCheck{97}{122}{11}%
  \RangeCatcodeCheck{123}{255}{15}%
  \RestoreCatcodes
}
\Test
\csname @@end\endcsname
\end
%    \end{macrocode}
%    \begin{macrocode}
%</test1>
%    \end{macrocode}
%
% \subsection{\LaTeX\space tests}
%
%    \begin{macrocode}
%<*test2|test3|test4|test5>
%    \end{macrocode}
%    \begin{macrocode}
%<noetex>\let\SavedUnexpanded\unexpanded
%<noetex>\let\unexpanded\UNDEFINED
\NeedsTeXFormat{LaTeX2e}
\documentclass{article}
\usepackage[T1]{fontenc}
\usepackage{lmodern}
\usepackage{textcomp}
\usepackage[ngerman]{babel}
\expandafter\def\csname ver@hypdestopt.sty\endcsname{}
\usepackage[unicode]{hyperref}
\usepackage{bookmark}
\usepackage{color}
\usepackage{soul}[2003/11/17]
\makeatletter
\newcommand*{\@ifDefined}[1]{%
  \@ifundefined{#1}{%
    \@latex@error{%
      \expandafter\noexpand\csname#1\endcsname undefined%
    }\@ehc
  }%
}
\newcommand*{\SaveOrg}[1]{%
  \@ifdefinable{ORG@SOUL@#1}{%
    \@ifDefined{SOUL@#1}{%
      \expandafter\let\csname ORG@SOUL@#1\expandafter\endcsname
                      \csname SOUL@#1\endcsname
    }%
  }%
}
\newcommand*{\RestoreOrg}[1]{%
  \@ifDefined{ORG@SOUL@#1}{%
    \@ifDefined{SOUL@#1}{%
      \expandafter\let\csname SOUL@#1\expandafter\endcsname
                      \csname ORG@SOUL@#1\endcsname
    }%
  }%
}
\makeatother
\SaveOrg{eval}
\SaveOrg{splittoken}
\SaveOrg{flushcomma}
\SaveOrg{flushapo}
\SaveOrg{grave}
\SaveOrg{addgroup}
\SaveOrg{addmath}
\SaveOrg{addprotect}
\SaveOrg{addtoken}
\usepackage{soulutf8}[2016/05/16]
%    \end{macrocode}
%    \begin{macrocode}
%<noetex>\let\unexpanded\SavedUnexpanded
%    \end{macrocode}
%    Input encoding \xfile{utf8}.
%    \begin{macrocode}
%<*utf8>
\usepackage[utf8]{inputenc}
\DeclareUnicodeCharacter{00A0}{\nobreakspace}
%</utf8>
%    \end{macrocode}
%    Input encoding \xfile{utf8x}. Package \xpackage{ucs}
%    has a nasty bug, that it redefines \cs{show}.
%    \begin{macrocode}
%<*utf8x>
\let\origshow\show
\usepackage[utf8x]{inputenc}
\AtBeginDocument{%
  \let\show\origshow
}
%</utf8x>
%    \end{macrocode}
%
% \subsubsection{Driver definitions}
%
%    \begin{macrocode}
\makeatletter
%    \end{macrocode}
%    \begin{macro}{\an}
%    The definition for \cs{an} is taken from the documentation of
%    package \xpackage{soul}.
%    \begin{macrocode}
\DeclareRobustCommand*{\an}{%
  \def\SOUL@preamble{$^{^P}$}%
  \def\SOUL@everyspace##1{##1\texttt{\char`\ }\linebreak[2]}%
  \def\SOUL@postamble{$^{^E}$}%
  \def\SOUL@everyhyphen{$^{^-}$\linebreak[1]}%
  \def\SOUL@everyexhyphen##1{##1$^{^=}$}%
  \def\SOUL@everysyllable{$^{^S}$}%
  \def\SOUL@everytoken{\the\SOUL@token$^{^T}$}%
  \def\SOUL@everylowerthan{$^{^L}$}%
  \SOUL@
}
%    \end{macrocode}
%    \end{macro}
%
%    \begin{macro}{\sy}
%    The definition for \cs{sy} is taken from the documentation of
%    package \xpackage{soul}.
%    \begin{macrocode}
\DeclareRobustCommand*{\sy}{%
  \SOUL@setup
  \def\SOUL@preamble{%
    \lefthyphenmin=\z@
    \righthyphenmin=\z@
  }%
  \def\SOUL@everyspace##1{##1\space}%
  \def\SOUL@everysyllable{\the\SOUL@syllable}%
  \def\SOUL@everyhyphen{%
    \discretionary{%
      \SOUL@setkern\SOUL@hyphkern
      \SOUL@sethyphenchar
    }{}{%
      \hbox{\kern1pt$\cdot$}%
    }%
  }%
  \def\SOUL@everyexhyphen##1{%
    \SOUL@setkern\SOUL@hyphkern
    \hbox{##1}%
    \discretionary{}{}{%
      \SOUL@setkern\SOUL@charkern
    }%
  }%
  \SOUL@
}
%    \end{macrocode}
%    \end{macro}
%
%    \begin{macro}{\soulH}
%    \begin{macrocode}
\DeclareRobustCommand*{\soulH}{%
  \def\SOUL@preamble{%
    \gdef\SOULH{}%
    \soulHadd{[}%
  }%
  \def\SOUL@everyspace##1{%
    \soulHadd{Sp}%
  }%
  \def\SOUL@postamble{%
    \soulHadd{]}%
  }%
  \def\SOUL@everyhyphen{%
    \soulHadd{Hy}%
  }%
  \def\SOUL@everyexhyphen##1{%
    \soulHadd{Exhy}%
  }%
  \def\SOUL@everysyllable{%
    \soulHadd{Sy}%
  }%
  \def\SOUL@everytoken{%
    \soulHadd{T}%
  }%
  \def\SOUL@everylowerthan{%
    \soulHadd{<}%
  }%
  \SOUL@
}
\newcommand*{\soulHadd}[1]{%
  #1%
  \expandafter\gdef\expandafter\soulHstring\expandafter{%
    \soulHstring
    #1%
  }%
}
%    \end{macrocode}
%    \end{macro}
%
% \subsubsection{Test macros}
%
%    \begin{macrocode}
\usepackage{qstest}
\IncludeTests{*}
\LogTests{log}{*}{*}
%    \end{macrocode}
%    \begin{macrocode}
\newsavebox\boxA
\newsavebox\boxB
\newsavebox\boxC
%    \end{macrocode}
%    \begin{macro}{\TestDriver}
%    \begin{macrocode}
\newcommand*{\TestDriver}[4]{%
  \subsection{Driver #1}%
  \@PackageInfo{soulutf8-test}{* Driver #1}%
  \expandafter\let\expandafter\testdriver\csname #1\endcsname
  \def\teststringA{#3}%
  \def\teststringB{#4}%
  \begin{qstest}{string-#2-driver-#1}{string,#2,driver,#1}%
    \@PackageInfo{soulutf8-test}{* Version: soulutf8}%
    \global\let\soulHstring\@empty
    \setbox0=\hbox{\expandafter\testdriver\expandafter{\teststringA}}%
    \let\HstringA\soulHstring
    \SetTestVBox{\boxA}{\teststringA}%
    %
    \@PackageInfo{soulutf8-test}{* Version: soul + patch}%
    \global\let\soulHstring\@empty
    \RestoreOrg{eval}%
    \setbox2=\hbox{\expandafter\testdriver\expandafter{\teststringB}}%
    \let\HstringB\soulHstring
    \SetTestVBox{\boxB}{\teststringB}%
    %
    \@PackageInfo{soulutf8-test}{* Version: original soul}%
    \global\let\soulHstring\@empty
    \RestoreOrg{splittoken}%
    \RestoreOrg{flushcomma}%
    \RestoreOrg{flushapo}%
    \RestoreOrg{grave}%
    \RestoreOrg{addgroup}%
    \RestoreOrg{addmath}%
    \RestoreOrg{addprotect}%
    \RestoreOrg{addtoken}%
    \setbox4=\hbox{\expandafter\testdriver\expandafter{\teststringB}}%
    \let\HstringC\soulHstring
    \SetTestVBox{\boxC}{\teststringB}%
    %
    \TestInfo{0}{soulutf8}%
    \TestInfo{2}{soul + patch}%
    \TestInfo{4}{original soul}%
    \TestInfoVBox
    \showboxdepth=\maxdimen
    \showboxbreadth=\maxdimen
    \tracingonline=\@ne
    \Expect*{\HstringA}*{\HstringB}%
    \Expect*{\HstringA}*{\HstringC}%
    \Expect*{\the\wd0}*{\the\wd2}%
    \ifdim\wd0=\wd2 %
    \else
      \showbox0 %
      \showbox2 %
    \fi
    \Expect*{\the\ht0}*{\the\ht2}%
    \Expect*{\the\dp0}*{\the\dp2}%
    \Expect*{\the\wd\boxB}*{\the\wd\boxA}%
    \Expect*{\the\wd\boxC}*{\the\wd\boxA}%
    \Expect*{\the\ht\boxB}*{\the\ht\boxA}%
    \ifdim\ht\boxB=\ht\boxA
    \else
      \showbox\boxA
      \showbox\boxB
    \fi
    \Expect*{\the\ht\boxC}*{\the\ht\boxA}%
    \Expect*{\the\dp\boxB}*{\the\dp\boxA}%
    \Expect*{\the\dp\boxC}*{\the\dp\boxA}%
  \end{qstest}%
}
%    \end{macrocode}
%    \end{macro}
%    \begin{macro}{\SetTestVBox}
%    \begin{macrocode}
\newcommand*{\SetTestVBox}[2]{%
  \setbox#1=\vbox{%
    \hsize=0pt\relax
    \hfuzz=100mm\relax
    \expandafter\testdriver\expandafter{#2}%
  }%
}
%    \end{macrocode}
%    \end{macro}
%    \begin{macro}{\TestInfo}
%    \begin{macrocode}
\newcommand*{\TestInfo}[2]{%
  \par
  \noindent
  \copy#1\relax
  \hfill
  \begingroup
    \small
    \ifdim\wd0=\wd2\relax
      \ifdim\wd0=\wd4\relax
      \else
        \ifnum#1=2 %
          \raisebox{.5\baselineskip}[0pt][0pt]{%
            \llap{$\}$ }%
            \rlap{width = \the\wd#1\relax\space$\{$}%
            \phantom{width = 999.99999pt}%
            \qquad
          }%
        \else
          \ifnum#1=4 %
            \rlap{width = \the\wd#1}%
            \phantom{width = 999.99999pt}%
            \qquad
          \fi
        \fi
      \fi
    \else
      \begingroup
        \bfseries
        \rlap{width = \the\wd#1}%
        \phantom{width = 999.99999pt}%
      \endgroup
      \qquad
    \fi
    \rlap{(#2)}%
    \phantom{(original soul)}%
  \endgroup
  \par
}
%    \end{macrocode}
%    \end{macro}
%    \begin{macro}{\TestInfoVBox}
%    \begin{macrocode}
\newcommand*{\TestInfoVBox}{%
  \par
  \noindent
  \mbox{\vrule\relax\space\copy\boxA}%
  \ifdim\ht\boxA=\ht\boxB
  \else
    \hfill
    \begingroup
      \small
      \rlap{(soulutf8)}%
      \phantom{(soulutf8)}%
    \endgroup
    \par
    \noindent
    \mbox{\vrule\relax\space\copy\boxB}%
    \hfill
    \begingroup
      \small
      \rlap{(soul)}%
      \phantom{(soulutf8)}%
    \endgroup
  \fi
  \par
}
%    \end{macrocode}
%    \end{macro}
%    \begin{macro}{\Test}
%    \begin{macrocode}
\makeatletter
\newcommand*{\Test}[3]{%
  \clearpage
  \begingroup
    \def\linebreak{\textbackslash linebreak }%
    \def\\{\textbackslash\textbackslash}%
    \def\soulomit##1{%
      \textbackslash soulomit\textbraceleft##1\textbraceright
    }%
    \def\mbox##1{%
      \textbackslash mbox\textbraceleft##1\textbraceright
    }%
    \def\abc{\textbackslash abc }%
    \def\texttt##1{%
      \textbackslash texttt\textbraceleft##1\textbraceright
    }%
    \def\c##1{%
      \textbackslash c\textbraceleft##1\textbraceright
    }%
%<*utf8x>
    \count@=128 %
    \loop
    \ifnum\count@<256 %
      \lccode`\~=\count@
      \lowercase{%
        \def~{\noexpand~}%
      }%
      \advance\count@\@ne
    \repeat
%</utf8x>
    \csname protected@edef\endcsname\x{\endgroup
      \noexpand\section{String #1: #2}%
    }%
  \x
  \TestDriver{so}{#1}{#2}{#3}%
  \TestDriver{caps}{#1}{#2}{#3}%
  \TestDriver{ul}{#1}{#2}{#3}%
  \TestDriver{st}{#1}{#2}{#3}%
  \TestDriver{hl}{#1}{#2}{#3}%
  \TestDriver{an}{#1}{#2}{#3}%
  \TestDriver{sy}{#1}{#2}{#3}%
  \TestDriver{soulH}{#1}{#2}{#3}%
}
\makeatother
%    \end{macrocode}
%    \end{macro}
%
% \subsubsection{Testing}
%
%    \begin{macrocode}
\begin{document}
  \Test A{}{}%
  \Test B{Hello World}{Hello World}%
  \Test C{%
    Umlauts: ^^c3^^84^^c3^^96^^c3^^9c^^c3^^a4^^c3^^b6^^c3^^bc^^c3^^9f%
  }{%
    Umlauts: \"A\"O\"U\"a\"o\"u\ss
  }%
  \Test D{Euro: ^^e2^^82^^ac}{Euro: \texteuro}%
  \Test E{Stra^^c3^^9fe}{Stra\ss e}%
  \Test F{R^^c3^^b6mer}{R\"omer}%
  \Test G{A^^e2^^80^^93B^^e2^^80^^94C}{A--B---C}%
  \Test H{X^^c2^^a0Y}{X~Y}%
  \Test I{A^^c5^^a4}{A\v T}%
  \Test J{A,B,,C,,,D,,,,E}{A,B,,C,,,D,,,,E}%
  \Test K{A'B''C'''D''''E}{A'B''C'''D''''E}%
  \Test L{A`B``C```D````E}{A`B``C```D````E}%
  \Test M{A$x$B}{A$x$B}%
  \Test N{A\mbox{abc}B}{A\mbox{abc}B}%
  \Test O{,,TA''}{,,TA''}%
  \Test P{{!`}Hola{?`}}{{!`}Hola{?`}}%
  \Test Q{re-sent\slash x}{re-sent\slash x}%
  \Test R{new\\line}{new\\line}%
  \Test S{foo\linebreak bar}{foo\linebreak bar}%
  \Test T{Virchow{sche}}{Virchow{sche}}%
  \Test U{A\soulomit{foo}B}{A\soulomit{foo}B}%
  \Test V{A\texttt{xyz}B}{A\texttt{xyz}B}%
  \Test W{ffi}{ffi}%
  \Test X{f{}f{}i}{f{}f{}i}%
  \Test Y{f\mbox{}f\mbox{}i}{f\mbox{}f\mbox{}i}%
  \Test Z{\c{\"a}}{{\c{\"a}}}%
%<*test2|test3>
  \let\abcx\UNDEFINED
  \def\abc{%
    \begingroup
      \def\abcx{x}%
      \abcx
    \endgroup
  }%
  \Test{AA}{[{x\abc x}]}{[{xxx}]}%
%</test2|test3>
%    \end{macrocode}
%
%    \begin{macrocode}
\end{document}
%    \end{macrocode}
%    \begin{macrocode}
%</test2|test3|test4|test5>
%    \end{macrocode}
%
% \section{Installation}
%
% \subsection{Download}
%
% \paragraph{Package.} This package is available on
% CTAN\footnote{\url{http://ctan.org/pkg/soulutf8}}:
% \begin{description}
% \item[\CTAN{macros/latex/contrib/oberdiek/soulutf8.dtx}] The source file.
% \item[\CTAN{macros/latex/contrib/oberdiek/soulutf8.pdf}] Documentation.
% \end{description}
%
%
% \paragraph{Bundle.} All the packages of the bundle `oberdiek'
% are also available in a TDS compliant ZIP archive. There
% the packages are already unpacked and the documentation files
% are generated. The files and directories obey the TDS standard.
% \begin{description}
% \item[\CTAN{install/macros/latex/contrib/oberdiek.tds.zip}]
% \end{description}
% \emph{TDS} refers to the standard ``A Directory Structure
% for \TeX\ Files'' (\CTAN{tds/tds.pdf}). Directories
% with \xfile{texmf} in their name are usually organized this way.
%
% \subsection{Bundle installation}
%
% \paragraph{Unpacking.} Unpack the \xfile{oberdiek.tds.zip} in the
% TDS tree (also known as \xfile{texmf} tree) of your choice.
% Example (linux):
% \begin{quote}
%   |unzip oberdiek.tds.zip -d ~/texmf|
% \end{quote}
%
% \paragraph{Script installation.}
% Check the directory \xfile{TDS:scripts/oberdiek/} for
% scripts that need further installation steps.
% Package \xpackage{attachfile2} comes with the Perl script
% \xfile{pdfatfi.pl} that should be installed in such a way
% that it can be called as \texttt{pdfatfi}.
% Example (linux):
% \begin{quote}
%   |chmod +x scripts/oberdiek/pdfatfi.pl|\\
%   |cp scripts/oberdiek/pdfatfi.pl /usr/local/bin/|
% \end{quote}
%
% \subsection{Package installation}
%
% \paragraph{Unpacking.} The \xfile{.dtx} file is a self-extracting
% \docstrip\ archive. The files are extracted by running the
% \xfile{.dtx} through \plainTeX:
% \begin{quote}
%   \verb|tex soulutf8.dtx|
% \end{quote}
%
% \paragraph{TDS.} Now the different files must be moved into
% the different directories in your installation TDS tree
% (also known as \xfile{texmf} tree):
% \begin{quote}
% \def\t{^^A
% \begin{tabular}{@{}>{\ttfamily}l@{ $\rightarrow$ }>{\ttfamily}l@{}}
%   soulutf8.sty & tex/generic/oberdiek/soulutf8.sty\\
%   soulutf8.pdf & doc/latex/oberdiek/soulutf8.pdf\\
%   test/soulutf8-test1.tex & doc/latex/oberdiek/test/soulutf8-test1.tex\\
%   test/soulutf8-test2.tex & doc/latex/oberdiek/test/soulutf8-test2.tex\\
%   test/soulutf8-test3.tex & doc/latex/oberdiek/test/soulutf8-test3.tex\\
%   test/soulutf8-test4.tex & doc/latex/oberdiek/test/soulutf8-test4.tex\\
%   test/soulutf8-test5.tex & doc/latex/oberdiek/test/soulutf8-test5.tex\\
%   soulutf8.dtx & source/latex/oberdiek/soulutf8.dtx\\
% \end{tabular}^^A
% }^^A
% \sbox0{\t}^^A
% \ifdim\wd0>\linewidth
%   \begingroup
%     \advance\linewidth by\leftmargin
%     \advance\linewidth by\rightmargin
%   \edef\x{\endgroup
%     \def\noexpand\lw{\the\linewidth}^^A
%   }\x
%   \def\lwbox{^^A
%     \leavevmode
%     \hbox to \linewidth{^^A
%       \kern-\leftmargin\relax
%       \hss
%       \usebox0
%       \hss
%       \kern-\rightmargin\relax
%     }^^A
%   }^^A
%   \ifdim\wd0>\lw
%     \sbox0{\small\t}^^A
%     \ifdim\wd0>\linewidth
%       \ifdim\wd0>\lw
%         \sbox0{\footnotesize\t}^^A
%         \ifdim\wd0>\linewidth
%           \ifdim\wd0>\lw
%             \sbox0{\scriptsize\t}^^A
%             \ifdim\wd0>\linewidth
%               \ifdim\wd0>\lw
%                 \sbox0{\tiny\t}^^A
%                 \ifdim\wd0>\linewidth
%                   \lwbox
%                 \else
%                   \usebox0
%                 \fi
%               \else
%                 \lwbox
%               \fi
%             \else
%               \usebox0
%             \fi
%           \else
%             \lwbox
%           \fi
%         \else
%           \usebox0
%         \fi
%       \else
%         \lwbox
%       \fi
%     \else
%       \usebox0
%     \fi
%   \else
%     \lwbox
%   \fi
% \else
%   \usebox0
% \fi
% \end{quote}
% If you have a \xfile{docstrip.cfg} that configures and enables \docstrip's
% TDS installing feature, then some files can already be in the right
% place, see the documentation of \docstrip.
%
% \subsection{Refresh file name databases}
%
% If your \TeX~distribution
% (\teTeX, \mikTeX, \dots) relies on file name databases, you must refresh
% these. For example, \teTeX\ users run \verb|texhash| or
% \verb|mktexlsr|.
%
% \subsection{Some details for the interested}
%
% \paragraph{Attached source.}
%
% The PDF documentation on CTAN also includes the
% \xfile{.dtx} source file. It can be extracted by
% AcrobatReader 6 or higher. Another option is \textsf{pdftk},
% e.g. unpack the file into the current directory:
% \begin{quote}
%   \verb|pdftk soulutf8.pdf unpack_files output .|
% \end{quote}
%
% \paragraph{Unpacking with \LaTeX.}
% The \xfile{.dtx} chooses its action depending on the format:
% \begin{description}
% \item[\plainTeX:] Run \docstrip\ and extract the files.
% \item[\LaTeX:] Generate the documentation.
% \end{description}
% If you insist on using \LaTeX\ for \docstrip\ (really,
% \docstrip\ does not need \LaTeX), then inform the autodetect routine
% about your intention:
% \begin{quote}
%   \verb|latex \let\install=y\input{soulutf8.dtx}|
% \end{quote}
% Do not forget to quote the argument according to the demands
% of your shell.
%
% \paragraph{Generating the documentation.}
% You can use both the \xfile{.dtx} or the \xfile{.drv} to generate
% the documentation. The process can be configured by the
% configuration file \xfile{ltxdoc.cfg}. For instance, put this
% line into this file, if you want to have A4 as paper format:
% \begin{quote}
%   \verb|\PassOptionsToClass{a4paper}{article}|
% \end{quote}
% An example follows how to generate the
% documentation with pdf\LaTeX:
% \begin{quote}
%\begin{verbatim}
%pdflatex soulutf8.dtx
%makeindex -s gind.ist soulutf8.idx
%pdflatex soulutf8.dtx
%makeindex -s gind.ist soulutf8.idx
%pdflatex soulutf8.dtx
%\end{verbatim}
% \end{quote}
%
% \begin{thebibliography}{9}
%
% \bibitem{soul}
%   Melchior Franz: \textit{The \xpackage{soul} package};
%   2003/11/17;\\
%   \CTAN{macros/latex/contrib/soul/}.
%
% \bibitem{ucs}
%   Dominique P. G. Unruh: \textit{\xpackage{ucs.sty} -- Unicode Support};
%   2004/10/17;\\
%   \CTAN{macros/latex/contrib/unicode/}.
%
% \bibitem{utf8}
%   Frank Mittelbach, Chris Rowley:
%   \textit{Providing some UTF-8 support via \xpackage{inputenc}};
%   2006/03/30;\\
%   \CTAN{macros/latex/base/utf8ienc.dtx}.
%
% \end{thebibliography}
%
% \begin{History}
%   \begin{Version}{2007/09/09 v1.0}
%   \item
%     First version.
%   \end{Version}
%   \begin{Version}{2016/05/16 v1.1}
%   \item
%     Documentation updates.
%   \end{Version}
% \end{History}
%
% \PrintIndex
%
% \Finale
\endinput
|
% \end{quote}
% Do not forget to quote the argument according to the demands
% of your shell.
%
% \paragraph{Generating the documentation.}
% You can use both the \xfile{.dtx} or the \xfile{.drv} to generate
% the documentation. The process can be configured by the
% configuration file \xfile{ltxdoc.cfg}. For instance, put this
% line into this file, if you want to have A4 as paper format:
% \begin{quote}
%   \verb|\PassOptionsToClass{a4paper}{article}|
% \end{quote}
% An example follows how to generate the
% documentation with pdf\LaTeX:
% \begin{quote}
%\begin{verbatim}
%pdflatex soulutf8.dtx
%makeindex -s gind.ist soulutf8.idx
%pdflatex soulutf8.dtx
%makeindex -s gind.ist soulutf8.idx
%pdflatex soulutf8.dtx
%\end{verbatim}
% \end{quote}
%
% \begin{thebibliography}{9}
%
% \bibitem{soul}
%   Melchior Franz: \textit{The \xpackage{soul} package};
%   2003/11/17;\\
%   \CTAN{macros/latex/contrib/soul/}.
%
% \bibitem{ucs}
%   Dominique P. G. Unruh: \textit{\xpackage{ucs.sty} -- Unicode Support};
%   2004/10/17;\\
%   \CTAN{macros/latex/contrib/unicode/}.
%
% \bibitem{utf8}
%   Frank Mittelbach, Chris Rowley:
%   \textit{Providing some UTF-8 support via \xpackage{inputenc}};
%   2006/03/30;\\
%   \CTAN{macros/latex/base/utf8ienc.dtx}.
%
% \end{thebibliography}
%
% \begin{History}
%   \begin{Version}{2007/09/09 v1.0}
%   \item
%     First version.
%   \end{Version}
%   \begin{Version}{2016/05/16 v1.1}
%   \item
%     Documentation updates.
%   \end{Version}
% \end{History}
%
% \PrintIndex
%
% \Finale
\endinput
|
% \end{quote}
% Do not forget to quote the argument according to the demands
% of your shell.
%
% \paragraph{Generating the documentation.}
% You can use both the \xfile{.dtx} or the \xfile{.drv} to generate
% the documentation. The process can be configured by the
% configuration file \xfile{ltxdoc.cfg}. For instance, put this
% line into this file, if you want to have A4 as paper format:
% \begin{quote}
%   \verb|\PassOptionsToClass{a4paper}{article}|
% \end{quote}
% An example follows how to generate the
% documentation with pdf\LaTeX:
% \begin{quote}
%\begin{verbatim}
%pdflatex soulutf8.dtx
%makeindex -s gind.ist soulutf8.idx
%pdflatex soulutf8.dtx
%makeindex -s gind.ist soulutf8.idx
%pdflatex soulutf8.dtx
%\end{verbatim}
% \end{quote}
%
% \begin{thebibliography}{9}
%
% \bibitem{soul}
%   Melchior Franz: \textit{The \xpackage{soul} package};
%   2003/11/17;\\
%   \CTAN{macros/latex/contrib/soul/}.
%
% \bibitem{ucs}
%   Dominique P. G. Unruh: \textit{\xpackage{ucs.sty} -- Unicode Support};
%   2004/10/17;\\
%   \CTAN{macros/latex/contrib/unicode/}.
%
% \bibitem{utf8}
%   Frank Mittelbach, Chris Rowley:
%   \textit{Providing some UTF-8 support via \xpackage{inputenc}};
%   2006/03/30;\\
%   \CTAN{macros/latex/base/utf8ienc.dtx}.
%
% \end{thebibliography}
%
% \begin{History}
%   \begin{Version}{2007/09/09 v1.0}
%   \item
%     First version.
%   \end{Version}
%   \begin{Version}{2016/05/16 v1.1}
%   \item
%     Documentation updates.
%   \end{Version}
% \end{History}
%
% \PrintIndex
%
% \Finale
\endinput
|
% \end{quote}
% Do not forget to quote the argument according to the demands
% of your shell.
%
% \paragraph{Generating the documentation.}
% You can use both the \xfile{.dtx} or the \xfile{.drv} to generate
% the documentation. The process can be configured by the
% configuration file \xfile{ltxdoc.cfg}. For instance, put this
% line into this file, if you want to have A4 as paper format:
% \begin{quote}
%   \verb|\PassOptionsToClass{a4paper}{article}|
% \end{quote}
% An example follows how to generate the
% documentation with pdf\LaTeX:
% \begin{quote}
%\begin{verbatim}
%pdflatex soulutf8.dtx
%makeindex -s gind.ist soulutf8.idx
%pdflatex soulutf8.dtx
%makeindex -s gind.ist soulutf8.idx
%pdflatex soulutf8.dtx
%\end{verbatim}
% \end{quote}
%
% \begin{thebibliography}{9}
%
% \bibitem{soul}
%   Melchior Franz: \textit{The \xpackage{soul} package};
%   2003/11/17;\\
%   \CTANpkg{soul}.
%
% \bibitem{ucs}
%   Dominique P. G. Unruh: \textit{\xpackage{ucs.sty} -- Unicode Support};
%   2004/10/17;\\
%   \CTANpkg{unicode}.
%
% \bibitem{utf8}
%   Frank Mittelbach, Chris Rowley:
%   \textit{Providing some UTF-8 support via \xpackage{inputenc}};
%   2006/03/30;\\
%   \CTAN{macros/latex/base/utf8ienc.dtx}.
%
% \end{thebibliography}
%
% \begin{History}
%   \begin{Version}{2007/09/09 v1.0}
%   \item
%     First version.
%   \end{Version}
%   \begin{Version}{2016/05/16 v1.1}
%   \item
%     Documentation updates.
%   \end{Version}
% \end{History}
%
% \PrintIndex
%
% \Finale
\endinput
