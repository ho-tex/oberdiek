% \iffalse meta-comment
%
% File: protecteddef.dtx
% Version: 2016/05/16 v1.1
% Info: Define protected commands
%
% Copyright (C) 2011 by
%    Heiko Oberdiek <heiko.oberdiek at googlemail.com>
%    2016
%    https://github.com/ho-tex/oberdiek/issues
%
% This work may be distributed and/or modified under the
% conditions of the LaTeX Project Public License, either
% version 1.3c of this license or (at your option) any later
% version. This version of this license is in
%    https://www.latex-project.org/lppl/lppl-1-3c.txt
% and the latest version of this license is in
%    https://www.latex-project.org/lppl.txt
% and version 1.3 or later is part of all distributions of
% LaTeX version 2005/12/01 or later.
%
% This work has the LPPL maintenance status "maintained".
%
% The Current Maintainers of this work are
% Heiko Oberdiek and the Oberdiek Package Support Group
% https://github.com/ho-tex/oberdiek/issues
%
% The Base Interpreter refers to any `TeX-Format',
% because some files are installed in TDS:tex/generic//.
%
% This work consists of the main source file protecteddef.dtx
% and the derived files
%    protecteddef.sty, protecteddef.pdf, protecteddef.ins, protecteddef.drv,
%    protecteddef-test1.tex, protecteddef-test2.tex.
%
% Distribution:
%    CTAN:macros/latex/contrib/oberdiek/protecteddef.dtx
%    CTAN:macros/latex/contrib/oberdiek/protecteddef.pdf
%
% Unpacking:
%    (a) If protecteddef.ins is present:
%           tex protecteddef.ins
%    (b) Without protecteddef.ins:
%           tex protecteddef.dtx
%    (c) If you insist on using LaTeX
%           latex \let\install=y% \iffalse meta-comment
%
% File: protecteddef.dtx
% Version: 2016/05/16 v1.1
% Info: Define protected commands
%
% Copyright (C)
%    2011 Heiko Oberdiek
%    2016-2019 Oberdiek Package Support Group
%    https://github.com/ho-tex/oberdiek/issues
%
% This work may be distributed and/or modified under the
% conditions of the LaTeX Project Public License, either
% version 1.3c of this license or (at your option) any later
% version. This version of this license is in
%    https://www.latex-project.org/lppl/lppl-1-3c.txt
% and the latest version of this license is in
%    https://www.latex-project.org/lppl.txt
% and version 1.3 or later is part of all distributions of
% LaTeX version 2005/12/01 or later.
%
% This work has the LPPL maintenance status "maintained".
%
% The Current Maintainers of this work are
% Heiko Oberdiek and the Oberdiek Package Support Group
% https://github.com/ho-tex/oberdiek/issues
%
% The Base Interpreter refers to any `TeX-Format',
% because some files are installed in TDS:tex/generic//.
%
% This work consists of the main source file protecteddef.dtx
% and the derived files
%    protecteddef.sty, protecteddef.pdf, protecteddef.ins, protecteddef.drv,
%    protecteddef-test1.tex, protecteddef-test2.tex.
%
% Distribution:
%    CTAN:macros/latex/contrib/oberdiek/protecteddef.dtx
%    CTAN:macros/latex/contrib/oberdiek/protecteddef.pdf
%
% Unpacking:
%    (a) If protecteddef.ins is present:
%           tex protecteddef.ins
%    (b) Without protecteddef.ins:
%           tex protecteddef.dtx
%    (c) If you insist on using LaTeX
%           latex \let\install=y% \iffalse meta-comment
%
% File: protecteddef.dtx
% Version: 2016/05/16 v1.1
% Info: Define protected commands
%
% Copyright (C)
%    2011 Heiko Oberdiek
%    2016-2019 Oberdiek Package Support Group
%    https://github.com/ho-tex/oberdiek/issues
%
% This work may be distributed and/or modified under the
% conditions of the LaTeX Project Public License, either
% version 1.3c of this license or (at your option) any later
% version. This version of this license is in
%    https://www.latex-project.org/lppl/lppl-1-3c.txt
% and the latest version of this license is in
%    https://www.latex-project.org/lppl.txt
% and version 1.3 or later is part of all distributions of
% LaTeX version 2005/12/01 or later.
%
% This work has the LPPL maintenance status "maintained".
%
% The Current Maintainers of this work are
% Heiko Oberdiek and the Oberdiek Package Support Group
% https://github.com/ho-tex/oberdiek/issues
%
% The Base Interpreter refers to any `TeX-Format',
% because some files are installed in TDS:tex/generic//.
%
% This work consists of the main source file protecteddef.dtx
% and the derived files
%    protecteddef.sty, protecteddef.pdf, protecteddef.ins, protecteddef.drv,
%    protecteddef-test1.tex, protecteddef-test2.tex.
%
% Distribution:
%    CTAN:macros/latex/contrib/oberdiek/protecteddef.dtx
%    CTAN:macros/latex/contrib/oberdiek/protecteddef.pdf
%
% Unpacking:
%    (a) If protecteddef.ins is present:
%           tex protecteddef.ins
%    (b) Without protecteddef.ins:
%           tex protecteddef.dtx
%    (c) If you insist on using LaTeX
%           latex \let\install=y% \iffalse meta-comment
%
% File: protecteddef.dtx
% Version: 2016/05/16 v1.1
% Info: Define protected commands
%
% Copyright (C)
%    2011 Heiko Oberdiek
%    2016-2019 Oberdiek Package Support Group
%    https://github.com/ho-tex/oberdiek/issues
%
% This work may be distributed and/or modified under the
% conditions of the LaTeX Project Public License, either
% version 1.3c of this license or (at your option) any later
% version. This version of this license is in
%    https://www.latex-project.org/lppl/lppl-1-3c.txt
% and the latest version of this license is in
%    https://www.latex-project.org/lppl.txt
% and version 1.3 or later is part of all distributions of
% LaTeX version 2005/12/01 or later.
%
% This work has the LPPL maintenance status "maintained".
%
% The Current Maintainers of this work are
% Heiko Oberdiek and the Oberdiek Package Support Group
% https://github.com/ho-tex/oberdiek/issues
%
% The Base Interpreter refers to any `TeX-Format',
% because some files are installed in TDS:tex/generic//.
%
% This work consists of the main source file protecteddef.dtx
% and the derived files
%    protecteddef.sty, protecteddef.pdf, protecteddef.ins, protecteddef.drv,
%    protecteddef-test1.tex, protecteddef-test2.tex.
%
% Distribution:
%    CTAN:macros/latex/contrib/oberdiek/protecteddef.dtx
%    CTAN:macros/latex/contrib/oberdiek/protecteddef.pdf
%
% Unpacking:
%    (a) If protecteddef.ins is present:
%           tex protecteddef.ins
%    (b) Without protecteddef.ins:
%           tex protecteddef.dtx
%    (c) If you insist on using LaTeX
%           latex \let\install=y\input{protecteddef.dtx}
%        (quote the arguments according to the demands of your shell)
%
% Documentation:
%    (a) If protecteddef.drv is present:
%           latex protecteddef.drv
%    (b) Without protecteddef.drv:
%           latex protecteddef.dtx; ...
%    The class ltxdoc loads the configuration file ltxdoc.cfg
%    if available. Here you can specify further options, e.g.
%    use A4 as paper format:
%       \PassOptionsToClass{a4paper}{article}
%
%    Programm calls to get the documentation (example):
%       pdflatex protecteddef.dtx
%       makeindex -s gind.ist protecteddef.idx
%       pdflatex protecteddef.dtx
%       makeindex -s gind.ist protecteddef.idx
%       pdflatex protecteddef.dtx
%
% Installation:
%    TDS:tex/generic/oberdiek/protecteddef.sty
%    TDS:doc/latex/oberdiek/protecteddef.pdf
%    TDS:doc/latex/oberdiek/test/protecteddef-test1.tex
%    TDS:doc/latex/oberdiek/test/protecteddef-test2.tex
%    TDS:source/latex/oberdiek/protecteddef.dtx
%
%<*ignore>
\begingroup
  \catcode123=1 %
  \catcode125=2 %
  \def\x{LaTeX2e}%
\expandafter\endgroup
\ifcase 0\ifx\install y1\fi\expandafter
         \ifx\csname processbatchFile\endcsname\relax\else1\fi
         \ifx\fmtname\x\else 1\fi\relax
\else\csname fi\endcsname
%</ignore>
%<*install>
\input docstrip.tex
\Msg{************************************************************************}
\Msg{* Installation}
\Msg{* Package: protecteddef 2016/05/16 v1.1 Define protected commands (HO)}
\Msg{************************************************************************}

\keepsilent
\askforoverwritefalse

\let\MetaPrefix\relax
\preamble

This is a generated file.

Project: protecteddef
Version: 2016/05/16 v1.1

Copyright (C)
   2011 Heiko Oberdiek
   2016-2019 Oberdiek Package Support Group

This work may be distributed and/or modified under the
conditions of the LaTeX Project Public License, either
version 1.3c of this license or (at your option) any later
version. This version of this license is in
   https://www.latex-project.org/lppl/lppl-1-3c.txt
and the latest version of this license is in
   https://www.latex-project.org/lppl.txt
and version 1.3 or later is part of all distributions of
LaTeX version 2005/12/01 or later.

This work has the LPPL maintenance status "maintained".

The Current Maintainers of this work are
Heiko Oberdiek and the Oberdiek Package Support Group
https://github.com/ho-tex/oberdiek/issues


The Base Interpreter refers to any `TeX-Format',
because some files are installed in TDS:tex/generic//.

This work consists of the main source file protecteddef.dtx
and the derived files
   protecteddef.sty, protecteddef.pdf, protecteddef.ins, protecteddef.drv,
   protecteddef-test1.tex, protecteddef-test2.tex.

\endpreamble
\let\MetaPrefix\DoubleperCent

\generate{%
  \file{protecteddef.ins}{\from{protecteddef.dtx}{install}}%
  \file{protecteddef.drv}{\from{protecteddef.dtx}{driver}}%
  \usedir{tex/generic/oberdiek}%
  \file{protecteddef.sty}{\from{protecteddef.dtx}{package}}%
%  \usedir{doc/latex/oberdiek/test}%
%  \file{protecteddef-test1.tex}{\from{protecteddef.dtx}{test1}}%
%  \file{protecteddef-test2.tex}{\from{protecteddef.dtx}{test2}}%
  \nopreamble
  \nopostamble
%  \usedir{source/latex/oberdiek/catalogue}%
%  \file{protecteddef.xml}{\from{protecteddef.dtx}{catalogue}}%
}

\catcode32=13\relax% active space
\let =\space%
\Msg{************************************************************************}
\Msg{*}
\Msg{* To finish the installation you have to move the following}
\Msg{* file into a directory searched by TeX:}
\Msg{*}
\Msg{*     protecteddef.sty}
\Msg{*}
\Msg{* To produce the documentation run the file `protecteddef.drv'}
\Msg{* through LaTeX.}
\Msg{*}
\Msg{* Happy TeXing!}
\Msg{*}
\Msg{************************************************************************}

\endbatchfile
%</install>
%<*ignore>
\fi
%</ignore>
%<*driver>
\NeedsTeXFormat{LaTeX2e}
\ProvidesFile{protecteddef.drv}%
  [2016/05/16 v1.1 Define protected commands (HO)]%
\documentclass{ltxdoc}
\usepackage{holtxdoc}[2011/11/22]
\begin{document}
  \DocInput{protecteddef.dtx}%
\end{document}
%</driver>
% \fi
%
%
% \CharacterTable
%  {Upper-case    \A\B\C\D\E\F\G\H\I\J\K\L\M\N\O\P\Q\R\S\T\U\V\W\X\Y\Z
%   Lower-case    \a\b\c\d\e\f\g\h\i\j\k\l\m\n\o\p\q\r\s\t\u\v\w\x\y\z
%   Digits        \0\1\2\3\4\5\6\7\8\9
%   Exclamation   \!     Double quote  \"     Hash (number) \#
%   Dollar        \$     Percent       \%     Ampersand     \&
%   Acute accent  \'     Left paren    \(     Right paren   \)
%   Asterisk      \*     Plus          \+     Comma         \,
%   Minus         \-     Point         \.     Solidus       \/
%   Colon         \:     Semicolon     \;     Less than     \<
%   Equals        \=     Greater than  \>     Question mark \?
%   Commercial at \@     Left bracket  \[     Backslash     \\
%   Right bracket \]     Circumflex    \^     Underscore    \_
%   Grave accent  \`     Left brace    \{     Vertical bar  \|
%   Right brace   \}     Tilde         \~}
%
% \GetFileInfo{protecteddef.drv}
%
% \title{The \xpackage{protecteddef} package}
% \date{2016/05/16 v1.1}
% \author{Heiko Oberdiek\thanks
% {Please report any issues at \url{https://github.com/ho-tex/oberdiek/issues}}}
%
% \maketitle
%
% \begin{abstract}
% This packages provides \cs{ProtectedDef} for defining
% robust macros for both \hologo{plainTeX} and \hologo{LaTeX}.
% First \hologo{eTeX}'s \cs{protected} is tried, then
% \hologo{LaTeX}'s \cs{DeclareRobustCommand} is used.
% Otherwise the macro is not made robust.
% \end{abstract}
%
% \tableofcontents
%
% \section{Documentation}
%
% Many of my packages work for both formats \hologo{plainTeX}
% and \hologo{LaTeX}, even \hologo{iniTeX} is often supported.
% It would be nice if fragile macros could be protected and
% made robust. However the different format worlds offer different
% solutions.
%
% \subsection{The \hologo{LaTeX}'s way}
%
% Usually \cs{newcommand} is used to define macros. It provides
% a check if the command to be defined is already defined or
% cannot be defined for other reasons.
%
% For making robust macros \hologo{LaTeX} provides \cs{DeclareRobustCommand}.
% It shares the syntax with \cs{newcommand}. However it does not provide
% latters check. Internally the check is available via \cs{@ifdefinable}.
%
% Internally the robust macro is using \cs{protect} with a nested
% macro definition. The \cs{protect} infrastructure is a feature of
% \cs{LaTeX} and usually not available in other formats.
%
% \subsection{The \hologo{eTeX}'s way}
%
% The need for robust macros is addressed in \cs{eTeX}. It provides
% \cs{protected} that modifies the behaviour of \cs{def} in a similar
% way as \cs{long}. A protected macro does not expand in
% some expandable contexts like writing to a file or \cs{edef}.
%
% \subsection{The way of this package}
%
% The package tries to find the available protection mechanism.
% First it looks for \cs{eTeX}'s \cs{protected}, then
% it uses \hologo{LaTeX}'s \cs{DeclareRobustCommand}.
% If both fails, then the macro remains unprotected.
%
% Additionally, \cs{LaTeX}'s check, if a macro is already defined
% is added in all cases. First \hologo{LaTeX}'s \cs{@ifdefinable}
% is tried to be compatible with \hologo{LaTeX}.
% If \cs{@ifdefinable} is not available, then the test is implemented
% by asserting that the macro is undefined or has the meaning of \cs{relax}.
% If the test fails, then in all cases the macro is not defined
% and an error is thrown.
%
% \subsection{Usage}
%
% \begin{declcs}{ProtectedDef} \* \M{cmd} [\meta{num}] \M{definition text}
% \end{declcs}
% Macro \cs{ProtectedDef} follows the syntax of \hologo{LaTeX}'s
% \cs{newcommand} with the exception that an optional argument
% is not supported. Macro \meta{cmd} is to be defined as
% \cs{long} macro without star with \meta{num} arguments.
%
% The number of arguments \meta{num} must be given as explicite digit
% |0| upto |9|. Otherwise the part between the argument \meta{cmd} and
% the \meta{definition text} is taken as parameter text in the syntax
% of vanilla \hologo{TeX}. Examples (with \cs{protected}):
% \begin{quote}
%   \def\to{$\Rightarrow$ }%
%   |\ProtectedDef*{\foo}[1]{\message{#1}}|\\
%   \to |\protected\def\foo#1{\message#1}}|\\[1ex]
%   |\ProtectedDef\foo{abc}|\\
%   \to |\protected\def\foo{abc}|\\[1ex]
%   |\ProtectedDef*\foo(#1)<#2>{#1/#2}|\\
%   \to |\protected\def\foo(#1)<#2>{#1/#2}|
% \end{quote}
%
% \StopEventually{
% }
%
% \section{Implementation}
%
%    \begin{macrocode}
%<*package>
%    \end{macrocode}
%
% \subsection{Reload check and package identification}
%    Reload check, especially if the package is not used with \LaTeX.
%    \begin{macrocode}
\begingroup\catcode61\catcode48\catcode32=10\relax%
  \catcode13=5 % ^^M
  \endlinechar=13 %
  \catcode35=6 % #
  \catcode39=12 % '
  \catcode44=12 % ,
  \catcode45=12 % -
  \catcode46=12 % .
  \catcode58=12 % :
  \catcode64=11 % @
  \catcode123=1 % {
  \catcode125=2 % }
  \expandafter\let\expandafter\x\csname ver@protecteddef.sty\endcsname
  \ifx\x\relax % plain-TeX, first loading
  \else
    \def\empty{}%
    \ifx\x\empty % LaTeX, first loading,
      % variable is initialized, but \ProvidesPackage not yet seen
    \else
      \expandafter\ifx\csname PackageInfo\endcsname\relax
        \def\x#1#2{%
          \immediate\write-1{Package #1 Info: #2.}%
        }%
      \else
        \def\x#1#2{\PackageInfo{#1}{#2, stopped}}%
      \fi
      \x{protecteddef}{The package is already loaded}%
      \aftergroup\endinput
    \fi
  \fi
\endgroup%
%    \end{macrocode}
%    Package identification:
%    \begin{macrocode}
\begingroup\catcode61\catcode48\catcode32=10\relax%
  \catcode13=5 % ^^M
  \endlinechar=13 %
  \catcode35=6 % #
  \catcode39=12 % '
  \catcode40=12 % (
  \catcode41=12 % )
  \catcode44=12 % ,
  \catcode45=12 % -
  \catcode46=12 % .
  \catcode47=12 % /
  \catcode58=12 % :
  \catcode64=11 % @
  \catcode91=12 % [
  \catcode93=12 % ]
  \catcode123=1 % {
  \catcode125=2 % }
  \expandafter\ifx\csname ProvidesPackage\endcsname\relax
    \def\x#1#2#3[#4]{\endgroup
      \immediate\write-1{Package: #3 #4}%
      \xdef#1{#4}%
    }%
  \else
    \def\x#1#2[#3]{\endgroup
      #2[{#3}]%
      \ifx#1\@undefined
        \xdef#1{#3}%
      \fi
      \ifx#1\relax
        \xdef#1{#3}%
      \fi
    }%
  \fi
\expandafter\x\csname ver@protecteddef.sty\endcsname
\ProvidesPackage{protecteddef}%
  [2016/05/16 v1.1 Define protected commands (HO)]%
%    \end{macrocode}
%
% \subsection{Catcodes}
%
%    \begin{macrocode}
\begingroup\catcode61\catcode48\catcode32=10\relax%
  \catcode13=5 % ^^M
  \endlinechar=13 %
  \catcode123=1 % {
  \catcode125=2 % }
  \catcode64=11 % @
  \def\x{\endgroup
    \expandafter\edef\csname ProDef@AtEnd\endcsname{%
      \endlinechar=\the\endlinechar\relax
      \catcode13=\the\catcode13\relax
      \catcode32=\the\catcode32\relax
      \catcode35=\the\catcode35\relax
      \catcode61=\the\catcode61\relax
      \catcode64=\the\catcode64\relax
      \catcode123=\the\catcode123\relax
      \catcode125=\the\catcode125\relax
    }%
  }%
\x\catcode61\catcode48\catcode32=10\relax%
\catcode13=5 % ^^M
\endlinechar=13 %
\catcode35=6 % #
\catcode64=11 % @
\catcode123=1 % {
\catcode125=2 % }
\def\TMP@EnsureCode#1#2{%
  \edef\ProDef@AtEnd{%
    \ProDef@AtEnd
    \catcode#1=\the\catcode#1\relax
  }%
  \catcode#1=#2\relax
}
\TMP@EnsureCode{38}{4}% &
\TMP@EnsureCode{40}{12}% (
\TMP@EnsureCode{41}{12}% )
\TMP@EnsureCode{42}{12}% *
\TMP@EnsureCode{45}{12}% -
\TMP@EnsureCode{46}{12}% .
\TMP@EnsureCode{47}{12}% /
\TMP@EnsureCode{91}{12}% [
\TMP@EnsureCode{93}{12}% ]
\TMP@EnsureCode{96}{12}% `
\edef\ProDef@AtEnd{\ProDef@AtEnd\noexpand\endinput}
%    \end{macrocode}
%
% \subsection{Resources}
%
%    \begin{macrocode}
\begingroup\expandafter\expandafter\expandafter\endgroup
\expandafter\ifx\csname RequirePackage\endcsname\relax
  \def\TMP@RequirePackage#1[#2]{%
    \begingroup\expandafter\expandafter\expandafter\endgroup
    \expandafter\ifx\csname ver@#1.sty\endcsname\relax
      \input #1.sty\relax
    \fi
  }%
\else
  \let\TMP@RequirePackage\RequirePackage
\fi
\TMP@RequirePackage{ltxcmds}[2010/12/12]%
\TMP@RequirePackage{infwarerr}[2010/04/08]%
%    \end{macrocode}
%
%    \begin{macrocode}
\def\ProDef@temp#1{%
  \expandafter\def\csname ProDef@param[#1]\endcsname % hash-ok
}
\expandafter\def\csname ProDef@param\endcsname{}
\ProDef@temp0{}
\ProDef@temp1{##1}
\ProDef@temp2{##1##2}
\ProDef@temp3{##1##2##3}
\ProDef@temp4{##1##2##3##4}
\ProDef@temp5{##1##2##3##4##5}
\ProDef@temp6{##1##2##3##4##5##6}
\ProDef@temp7{##1##2##3##4##5##7}
\ProDef@temp8{##1##2##3##4##5##7##8}
\ProDef@temp9{##1##2##3##4##5##7##8##9}
%    \end{macrocode}
%
%    \begin{macro}{\ProDef@IfDefinable}
%    \begin{macrocode}
\ltx@IfUndefined{@ifdefinable}{%
  \long\def\ProDef@IfDefinable#1{%
    \begingroup
      \escapechar=-1 %
    \ltx@ifundefined{\string#1}{%
      \endgroup
      \ltx@firstofone
    }{%
      \expandafter\endgroup
      \expandafter
      \edef\expandafter\ProDef@temp\expandafter{\string#1 }%
      \@PackageError{protecteddef}{%
        Command \ltx@backslashchar\ProDef@temp already defined%
      }\@ehc
      \ltx@gobbletwo
    }%
  }%
}{%
  \long\def\ProDef@IfDefinable#1{%
    \let\ProDef@next\ltx@gobbletwo
    \@ifdefinable{#1}{%
      \let\ProDef@next\ltx@firstofone
    }%
    \ProDef@next
  }%
}
%    \end{macrocode}
%    \end{macro}
%
%    \begin{macrocode}
\begingroup\expandafter\expandafter\expandafter\endgroup
\expandafter\ifx\csname protected\endcsname\relax
  \begingroup\expandafter\expandafter\expandafter\endgroup
  \expandafter\ifx\csname DeclareRobustCommand\endcsname\relax
    \catcode`\&=14 % comment
  \else
    \newcommand*{\ProtectedDef}{%
      \ltx@ifnextchar*{%
        \ProDef@ProtectedDef
      }{%
        \ProDef@ProtectedDef{}%
      }%
    }%
    \long\def\ProDef@ProtectedDef#1#2#3#{%
      \ProDef@IfDefinable{#2}{%
        \ltx@IfUndefined{ProDef@param#3}{%
          \DeclareRobustCommand*{#2}{}%
          \begingroup
            \escapechar=-1 %
            \def\ProDef@temp{#1}%
          \edef\x{\endgroup
            \ifx\ProDef@temp\ltx@empty
              \noexpand\long
            \fi
            \noexpand\def
            \expandafter\noexpand\csname\string#2 \endcsname
          }%
          \x#3%
        }{%
          \DeclareRobustCommand#1{#2}#3%
        }%
      }%
    }%
    \expandafter\expandafter\expandafter\ProDef@AtEnd
  \fi
\else
  \catcode`\&=9 % ignore
\fi%
\ProDef@IfDefinable\ProtectedDef{%
&  \protected
  \def\ProtectedDef%
}{%
  \ltx@ifnextchar*{%
    \let\ProDef@long\ltx@empty
    \expandafter\ProDef@ProtectedDef\ltx@gobble
  }{%
    \let\ProDef@long\long
    \ProDef@ProtectedDef
  }%
}
\long\def\ProDef@ProtectedDef#1#2#{%
  \ProDef@IfDefinable{#1}{%
    \ltx@IfUndefined{ProDef@param#2}{%
&     \protected
      \ProDef@long
      \def#1#2%
    }{%
&     \protected
      \ProDef@long
      \expandafter\expandafter\expandafter\def
      \expandafter\expandafter\expandafter#1%
      \csname ProDef@param#2\endcsname
    }%
  }%
}
%    \end{macrocode}
%
%    \begin{macrocode}
\ProDef@AtEnd%
%    \end{macrocode}
%    \begin{macrocode}
%</package>
%    \end{macrocode}
%
% \section{Test}
%
% \subsection{Catcode checks for loading}
%
%    \begin{macrocode}
%<*test1>
%    \end{macrocode}
%    \begin{macrocode}
\catcode`\{=1 %
\catcode`\}=2 %
\catcode`\#=6 %
\catcode`\@=11 %
\expandafter\ifx\csname count@\endcsname\relax
  \countdef\count@=255 %
\fi
\expandafter\ifx\csname @gobble\endcsname\relax
  \long\def\@gobble#1{}%
\fi
\expandafter\ifx\csname @firstofone\endcsname\relax
  \long\def\@firstofone#1{#1}%
\fi
\expandafter\ifx\csname loop\endcsname\relax
  \expandafter\@firstofone
\else
  \expandafter\@gobble
\fi
{%
  \def\loop#1\repeat{%
    \def\body{#1}%
    \iterate
  }%
  \def\iterate{%
    \body
      \let\next\iterate
    \else
      \let\next\relax
    \fi
    \next
  }%
  \let\repeat=\fi
}%
\def\RestoreCatcodes{}
\count@=0 %
\loop
  \edef\RestoreCatcodes{%
    \RestoreCatcodes
    \catcode\the\count@=\the\catcode\count@\relax
  }%
\ifnum\count@<255 %
  \advance\count@ 1 %
\repeat

\def\RangeCatcodeInvalid#1#2{%
  \count@=#1\relax
  \loop
    \catcode\count@=15 %
  \ifnum\count@<#2\relax
    \advance\count@ 1 %
  \repeat
}
\def\RangeCatcodeCheck#1#2#3{%
  \count@=#1\relax
  \loop
    \ifnum#3=\catcode\count@
    \else
      \errmessage{%
        Character \the\count@\space
        with wrong catcode \the\catcode\count@\space
        instead of \number#3%
      }%
    \fi
  \ifnum\count@<#2\relax
    \advance\count@ 1 %
  \repeat
}
\def\space{ }
\expandafter\ifx\csname LoadCommand\endcsname\relax
  \def\LoadCommand{\input protecteddef.sty\relax}%
\fi
\def\Test{%
  \RangeCatcodeInvalid{0}{47}%
  \RangeCatcodeInvalid{58}{64}%
  \RangeCatcodeInvalid{91}{96}%
  \RangeCatcodeInvalid{123}{255}%
  \catcode`\@=12 %
  \catcode`\\=0 %
  \catcode`\%=14 %
  \LoadCommand
  \RangeCatcodeCheck{0}{36}{15}%
  \RangeCatcodeCheck{37}{37}{14}%
  \RangeCatcodeCheck{38}{47}{15}%
  \RangeCatcodeCheck{48}{57}{12}%
  \RangeCatcodeCheck{58}{63}{15}%
  \RangeCatcodeCheck{64}{64}{12}%
  \RangeCatcodeCheck{65}{90}{11}%
  \RangeCatcodeCheck{91}{91}{15}%
  \RangeCatcodeCheck{92}{92}{0}%
  \RangeCatcodeCheck{93}{96}{15}%
  \RangeCatcodeCheck{97}{122}{11}%
  \RangeCatcodeCheck{123}{255}{15}%
  \RestoreCatcodes
}
\Test
\csname @@end\endcsname
\end
%    \end{macrocode}
%    \begin{macrocode}
%</test1>
%    \end{macrocode}
%
% \subsection{Test without \hologo{LaTeX} and \cs{protected}}
%
%    \begin{macrocode}
%<*test2>
\errorcontextlines=10000 %
\begingroup\expandafter\expandafter\expandafter\endgroup
\expandafter\ifx\csname RequirePackage\endcsname\relax
  \input protecteddef.sty\relax
  \catcode`\{=1 %
  \catcode`\}=2 %
  \catcode`\#=6 %
\else
  \RequirePackage{protecteddef}[2016/05/16]%
\fi
\begingroup\expandafter\expandafter\expandafter\endgroup
\expandafter\ifx\csname protected\endcsname\relax
  \let\pdef\def
\else
  \def\pdef{\protected\def}%
\fi
\def\msg#{\immediate\write16}
\countdef\errcount=2 %
\long\def\BeginCheck#1\ProtectedDef#2\EndCheck{%
  \begingroup
    \toks0={\ProtectedDef#2}%
    \msg{<<\the\toks0>>}%
  \endgroup
  \setbox0=\hbox{%
    #1%
    \ProtectedDef#2%
    \check\foo
  }%
  \ifdim\wd0=0pt\relax
  \else
    \errmessage{[Definition] Unwanted spaces?!}%
  \fi
  \setbox0=\hbox{%
    \def\fooinitial{XYZ}%
    \let\foo\fooinitial
    \errcount=0 %
    \expandafter\def\csname @PackageError\endcsname##1##2##3{%
      \advance\errcount by 1 %
    }%
    \expandafter\def\csname @notdefinable\endcsname{%
      \advance\errcount by 1 %
    }%
    \ProtectedDef#2%
    \ifnum\errcount=1 %
    \else
      \errmessage{1 error expected, but found: \the\errcount}%
    \fi
    \ifx\foo\fooinitial
    \else
      \def\space{ }%
      \errmessage{\string\foo\space is overwritten}%
    \fi
  }%
  \ifdim\wd0=0pt\relax
  \else
    \errmessage{[Error] Unwanted spaces?!}%
  \fi
}
\chardef\DeclareVersion=0 %
\begingroup\expandafter\expandafter\expandafter\endgroup
\expandafter\ifx\csname protected\endcsname\relax
  \begingroup\expandafter\expandafter\expandafter\endgroup
  \expandafter\ifx\csname DeclareRobustCommand\endcsname\relax
  \else
    \chardef\DeclareVersion=1 %
  \fi
\fi
\ifnum\DeclareVersion=0 %
  \def\check#1{%
    \ifx\cmp#1%
      \msg{* Test passed.}%
    \else
      \msg{}%
      \msg{[\meaning#1]}%
      \msg{[\meaning\cmp]}%
      \errmessage{Test failed!}%
    \fi
  }%
\else
  \def\check#1{%
    \begingroup
      \escapechar=-1 %
    \edef\x{\endgroup
      \def\noexpand\cs/{\string#1}%
    }\x
    \edef\CMP{%
      \noexpand\protect
      \expandafter\noexpand\csname\cs/ \endcsname
    }%
    \ifx\CMP#1%
      \expandafter\ifx\csname\cs/ \endcsname\cmp
        \msg{Test passed.}%
      \else
        \msg{}%
        \msg{[\expandafter\meaning\csname\cs/ \endcsname]}%
        \msg{[\meaning\cmp]}%
        \errmessage{Test failed!}%
      \fi
    \else
      \msg{}%
      \msg{[\meaning#1]}%
      \msg{[\meaning\CMP]}%
      \errmessage{Test failed!}%
    \fi
  }%
\fi

\tracingmacros=1

\BeginCheck
  \pdef\cmp{}%
  \ProtectedDef*\foo{}%
\EndCheck

\BeginCheck
  \pdef\cmp{}%
  \ProtectedDef*\foo[0]{}%
\EndCheck

\BeginCheck
  \pdef\cmp#1{<#1>}%
  \ProtectedDef*\foo[1]{<#1>}%
\EndCheck

\BeginCheck
  \pdef\cmp(#1){<#1>}%
  \ProtectedDef*\foo(#1){<#1>}%
\EndCheck

\BeginCheck
  \long\pdef\cmp{}%
  \ProtectedDef\foo{}%
\EndCheck

\BeginCheck
  \long\pdef\cmp{}%
  \ProtectedDef\foo[0]{}%
\EndCheck

\BeginCheck
  \long\pdef\cmp#1{<#1>}%
  \ProtectedDef\foo[1]{<#1>}%
\EndCheck

\BeginCheck
  \long\pdef\cmp(#1){<#1>}%
  \ProtectedDef\foo(#1){<#1>}%
\EndCheck

\csname @@end\endcsname\end
%</test2>
%    \end{macrocode}
%
% \section{Installation}
%
% \subsection{Download}
%
% \paragraph{Package.} This package is available on
% CTAN\footnote{\CTANpkg{protecteddef}}:
% \begin{description}
% \item[\CTAN{macros/latex/contrib/oberdiek/protecteddef.dtx}] The source file.
% \item[\CTAN{macros/latex/contrib/oberdiek/protecteddef.pdf}] Documentation.
% \end{description}
%
%
% \paragraph{Bundle.} All the packages of the bundle `oberdiek'
% are also available in a TDS compliant ZIP archive. There
% the packages are already unpacked and the documentation files
% are generated. The files and directories obey the TDS standard.
% \begin{description}
% \item[\CTANinstall{install/macros/latex/contrib/oberdiek.tds.zip}]
% \end{description}
% \emph{TDS} refers to the standard ``A Directory Structure
% for \TeX\ Files'' (\CTAN{tds/tds.pdf}). Directories
% with \xfile{texmf} in their name are usually organized this way.
%
% \subsection{Bundle installation}
%
% \paragraph{Unpacking.} Unpack the \xfile{oberdiek.tds.zip} in the
% TDS tree (also known as \xfile{texmf} tree) of your choice.
% Example (linux):
% \begin{quote}
%   |unzip oberdiek.tds.zip -d ~/texmf|
% \end{quote}
%
% \paragraph{Script installation.}
% Check the directory \xfile{TDS:scripts/oberdiek/} for
% scripts that need further installation steps.
%
% \subsection{Package installation}
%
% \paragraph{Unpacking.} The \xfile{.dtx} file is a self-extracting
% \docstrip\ archive. The files are extracted by running the
% \xfile{.dtx} through \plainTeX:
% \begin{quote}
%   \verb|tex protecteddef.dtx|
% \end{quote}
%
% \paragraph{TDS.} Now the different files must be moved into
% the different directories in your installation TDS tree
% (also known as \xfile{texmf} tree):
% \begin{quote}
% \def\t{^^A
% \begin{tabular}{@{}>{\ttfamily}l@{ $\rightarrow$ }>{\ttfamily}l@{}}
%   protecteddef.sty & tex/generic/oberdiek/protecteddef.sty\\
%   protecteddef.pdf & doc/latex/oberdiek/protecteddef.pdf\\
%   test/protecteddef-test1.tex & doc/latex/oberdiek/test/protecteddef-test1.tex\\
%   test/protecteddef-test2.tex & doc/latex/oberdiek/test/protecteddef-test2.tex\\
%   protecteddef.dtx & source/latex/oberdiek/protecteddef.dtx\\
% \end{tabular}^^A
% }^^A
% \sbox0{\t}^^A
% \ifdim\wd0>\linewidth
%   \begingroup
%     \advance\linewidth by\leftmargin
%     \advance\linewidth by\rightmargin
%   \edef\x{\endgroup
%     \def\noexpand\lw{\the\linewidth}^^A
%   }\x
%   \def\lwbox{^^A
%     \leavevmode
%     \hbox to \linewidth{^^A
%       \kern-\leftmargin\relax
%       \hss
%       \usebox0
%       \hss
%       \kern-\rightmargin\relax
%     }^^A
%   }^^A
%   \ifdim\wd0>\lw
%     \sbox0{\small\t}^^A
%     \ifdim\wd0>\linewidth
%       \ifdim\wd0>\lw
%         \sbox0{\footnotesize\t}^^A
%         \ifdim\wd0>\linewidth
%           \ifdim\wd0>\lw
%             \sbox0{\scriptsize\t}^^A
%             \ifdim\wd0>\linewidth
%               \ifdim\wd0>\lw
%                 \sbox0{\tiny\t}^^A
%                 \ifdim\wd0>\linewidth
%                   \lwbox
%                 \else
%                   \usebox0
%                 \fi
%               \else
%                 \lwbox
%               \fi
%             \else
%               \usebox0
%             \fi
%           \else
%             \lwbox
%           \fi
%         \else
%           \usebox0
%         \fi
%       \else
%         \lwbox
%       \fi
%     \else
%       \usebox0
%     \fi
%   \else
%     \lwbox
%   \fi
% \else
%   \usebox0
% \fi
% \end{quote}
% If you have a \xfile{docstrip.cfg} that configures and enables \docstrip's
% TDS installing feature, then some files can already be in the right
% place, see the documentation of \docstrip.
%
% \subsection{Refresh file name databases}
%
% If your \TeX~distribution
% (\TeX\,Live, \mikTeX, \dots) relies on file name databases, you must refresh
% these. For example, \TeX\,Live\ users run \verb|texhash| or
% \verb|mktexlsr|.
%
% \subsection{Some details for the interested}
%
% \paragraph{Unpacking with \LaTeX.}
% The \xfile{.dtx} chooses its action depending on the format:
% \begin{description}
% \item[\plainTeX:] Run \docstrip\ and extract the files.
% \item[\LaTeX:] Generate the documentation.
% \end{description}
% If you insist on using \LaTeX\ for \docstrip\ (really,
% \docstrip\ does not need \LaTeX), then inform the autodetect routine
% about your intention:
% \begin{quote}
%   \verb|latex \let\install=y\input{protecteddef.dtx}|
% \end{quote}
% Do not forget to quote the argument according to the demands
% of your shell.
%
% \paragraph{Generating the documentation.}
% You can use both the \xfile{.dtx} or the \xfile{.drv} to generate
% the documentation. The process can be configured by the
% configuration file \xfile{ltxdoc.cfg}. For instance, put this
% line into this file, if you want to have A4 as paper format:
% \begin{quote}
%   \verb|\PassOptionsToClass{a4paper}{article}|
% \end{quote}
% An example follows how to generate the
% documentation with pdf\LaTeX:
% \begin{quote}
%\begin{verbatim}
%pdflatex protecteddef.dtx
%makeindex -s gind.ist protecteddef.idx
%pdflatex protecteddef.dtx
%makeindex -s gind.ist protecteddef.idx
%pdflatex protecteddef.dtx
%\end{verbatim}
% \end{quote}
%
% \begin{History}
%   \begin{Version}{2011/01/31 v1.0}
%   \item
%     First public version.
%   \end{Version}
%   \begin{Version}{2016/05/16 v1.1}
%   \item
%     Documentation updates.
%   \end{Version}
% \end{History}
%
% \PrintIndex
%
% \Finale
\endinput

%        (quote the arguments according to the demands of your shell)
%
% Documentation:
%    (a) If protecteddef.drv is present:
%           latex protecteddef.drv
%    (b) Without protecteddef.drv:
%           latex protecteddef.dtx; ...
%    The class ltxdoc loads the configuration file ltxdoc.cfg
%    if available. Here you can specify further options, e.g.
%    use A4 as paper format:
%       \PassOptionsToClass{a4paper}{article}
%
%    Programm calls to get the documentation (example):
%       pdflatex protecteddef.dtx
%       makeindex -s gind.ist protecteddef.idx
%       pdflatex protecteddef.dtx
%       makeindex -s gind.ist protecteddef.idx
%       pdflatex protecteddef.dtx
%
% Installation:
%    TDS:tex/generic/oberdiek/protecteddef.sty
%    TDS:doc/latex/oberdiek/protecteddef.pdf
%    TDS:doc/latex/oberdiek/test/protecteddef-test1.tex
%    TDS:doc/latex/oberdiek/test/protecteddef-test2.tex
%    TDS:source/latex/oberdiek/protecteddef.dtx
%
%<*ignore>
\begingroup
  \catcode123=1 %
  \catcode125=2 %
  \def\x{LaTeX2e}%
\expandafter\endgroup
\ifcase 0\ifx\install y1\fi\expandafter
         \ifx\csname processbatchFile\endcsname\relax\else1\fi
         \ifx\fmtname\x\else 1\fi\relax
\else\csname fi\endcsname
%</ignore>
%<*install>
\input docstrip.tex
\Msg{************************************************************************}
\Msg{* Installation}
\Msg{* Package: protecteddef 2016/05/16 v1.1 Define protected commands (HO)}
\Msg{************************************************************************}

\keepsilent
\askforoverwritefalse

\let\MetaPrefix\relax
\preamble

This is a generated file.

Project: protecteddef
Version: 2016/05/16 v1.1

Copyright (C)
   2011 Heiko Oberdiek
   2016-2019 Oberdiek Package Support Group

This work may be distributed and/or modified under the
conditions of the LaTeX Project Public License, either
version 1.3c of this license or (at your option) any later
version. This version of this license is in
   https://www.latex-project.org/lppl/lppl-1-3c.txt
and the latest version of this license is in
   https://www.latex-project.org/lppl.txt
and version 1.3 or later is part of all distributions of
LaTeX version 2005/12/01 or later.

This work has the LPPL maintenance status "maintained".

The Current Maintainers of this work are
Heiko Oberdiek and the Oberdiek Package Support Group
https://github.com/ho-tex/oberdiek/issues


The Base Interpreter refers to any `TeX-Format',
because some files are installed in TDS:tex/generic//.

This work consists of the main source file protecteddef.dtx
and the derived files
   protecteddef.sty, protecteddef.pdf, protecteddef.ins, protecteddef.drv,
   protecteddef-test1.tex, protecteddef-test2.tex.

\endpreamble
\let\MetaPrefix\DoubleperCent

\generate{%
  \file{protecteddef.ins}{\from{protecteddef.dtx}{install}}%
  \file{protecteddef.drv}{\from{protecteddef.dtx}{driver}}%
  \usedir{tex/generic/oberdiek}%
  \file{protecteddef.sty}{\from{protecteddef.dtx}{package}}%
%  \usedir{doc/latex/oberdiek/test}%
%  \file{protecteddef-test1.tex}{\from{protecteddef.dtx}{test1}}%
%  \file{protecteddef-test2.tex}{\from{protecteddef.dtx}{test2}}%
  \nopreamble
  \nopostamble
%  \usedir{source/latex/oberdiek/catalogue}%
%  \file{protecteddef.xml}{\from{protecteddef.dtx}{catalogue}}%
}

\catcode32=13\relax% active space
\let =\space%
\Msg{************************************************************************}
\Msg{*}
\Msg{* To finish the installation you have to move the following}
\Msg{* file into a directory searched by TeX:}
\Msg{*}
\Msg{*     protecteddef.sty}
\Msg{*}
\Msg{* To produce the documentation run the file `protecteddef.drv'}
\Msg{* through LaTeX.}
\Msg{*}
\Msg{* Happy TeXing!}
\Msg{*}
\Msg{************************************************************************}

\endbatchfile
%</install>
%<*ignore>
\fi
%</ignore>
%<*driver>
\NeedsTeXFormat{LaTeX2e}
\ProvidesFile{protecteddef.drv}%
  [2016/05/16 v1.1 Define protected commands (HO)]%
\documentclass{ltxdoc}
\usepackage{holtxdoc}[2011/11/22]
\begin{document}
  \DocInput{protecteddef.dtx}%
\end{document}
%</driver>
% \fi
%
%
% \CharacterTable
%  {Upper-case    \A\B\C\D\E\F\G\H\I\J\K\L\M\N\O\P\Q\R\S\T\U\V\W\X\Y\Z
%   Lower-case    \a\b\c\d\e\f\g\h\i\j\k\l\m\n\o\p\q\r\s\t\u\v\w\x\y\z
%   Digits        \0\1\2\3\4\5\6\7\8\9
%   Exclamation   \!     Double quote  \"     Hash (number) \#
%   Dollar        \$     Percent       \%     Ampersand     \&
%   Acute accent  \'     Left paren    \(     Right paren   \)
%   Asterisk      \*     Plus          \+     Comma         \,
%   Minus         \-     Point         \.     Solidus       \/
%   Colon         \:     Semicolon     \;     Less than     \<
%   Equals        \=     Greater than  \>     Question mark \?
%   Commercial at \@     Left bracket  \[     Backslash     \\
%   Right bracket \]     Circumflex    \^     Underscore    \_
%   Grave accent  \`     Left brace    \{     Vertical bar  \|
%   Right brace   \}     Tilde         \~}
%
% \GetFileInfo{protecteddef.drv}
%
% \title{The \xpackage{protecteddef} package}
% \date{2016/05/16 v1.1}
% \author{Heiko Oberdiek\thanks
% {Please report any issues at \url{https://github.com/ho-tex/oberdiek/issues}}}
%
% \maketitle
%
% \begin{abstract}
% This packages provides \cs{ProtectedDef} for defining
% robust macros for both \hologo{plainTeX} and \hologo{LaTeX}.
% First \hologo{eTeX}'s \cs{protected} is tried, then
% \hologo{LaTeX}'s \cs{DeclareRobustCommand} is used.
% Otherwise the macro is not made robust.
% \end{abstract}
%
% \tableofcontents
%
% \section{Documentation}
%
% Many of my packages work for both formats \hologo{plainTeX}
% and \hologo{LaTeX}, even \hologo{iniTeX} is often supported.
% It would be nice if fragile macros could be protected and
% made robust. However the different format worlds offer different
% solutions.
%
% \subsection{The \hologo{LaTeX}'s way}
%
% Usually \cs{newcommand} is used to define macros. It provides
% a check if the command to be defined is already defined or
% cannot be defined for other reasons.
%
% For making robust macros \hologo{LaTeX} provides \cs{DeclareRobustCommand}.
% It shares the syntax with \cs{newcommand}. However it does not provide
% latters check. Internally the check is available via \cs{@ifdefinable}.
%
% Internally the robust macro is using \cs{protect} with a nested
% macro definition. The \cs{protect} infrastructure is a feature of
% \cs{LaTeX} and usually not available in other formats.
%
% \subsection{The \hologo{eTeX}'s way}
%
% The need for robust macros is addressed in \cs{eTeX}. It provides
% \cs{protected} that modifies the behaviour of \cs{def} in a similar
% way as \cs{long}. A protected macro does not expand in
% some expandable contexts like writing to a file or \cs{edef}.
%
% \subsection{The way of this package}
%
% The package tries to find the available protection mechanism.
% First it looks for \cs{eTeX}'s \cs{protected}, then
% it uses \hologo{LaTeX}'s \cs{DeclareRobustCommand}.
% If both fails, then the macro remains unprotected.
%
% Additionally, \cs{LaTeX}'s check, if a macro is already defined
% is added in all cases. First \hologo{LaTeX}'s \cs{@ifdefinable}
% is tried to be compatible with \hologo{LaTeX}.
% If \cs{@ifdefinable} is not available, then the test is implemented
% by asserting that the macro is undefined or has the meaning of \cs{relax}.
% If the test fails, then in all cases the macro is not defined
% and an error is thrown.
%
% \subsection{Usage}
%
% \begin{declcs}{ProtectedDef} \* \M{cmd} [\meta{num}] \M{definition text}
% \end{declcs}
% Macro \cs{ProtectedDef} follows the syntax of \hologo{LaTeX}'s
% \cs{newcommand} with the exception that an optional argument
% is not supported. Macro \meta{cmd} is to be defined as
% \cs{long} macro without star with \meta{num} arguments.
%
% The number of arguments \meta{num} must be given as explicite digit
% |0| upto |9|. Otherwise the part between the argument \meta{cmd} and
% the \meta{definition text} is taken as parameter text in the syntax
% of vanilla \hologo{TeX}. Examples (with \cs{protected}):
% \begin{quote}
%   \def\to{$\Rightarrow$ }%
%   |\ProtectedDef*{\foo}[1]{\message{#1}}|\\
%   \to |\protected\def\foo#1{\message#1}}|\\[1ex]
%   |\ProtectedDef\foo{abc}|\\
%   \to |\protected\def\foo{abc}|\\[1ex]
%   |\ProtectedDef*\foo(#1)<#2>{#1/#2}|\\
%   \to |\protected\def\foo(#1)<#2>{#1/#2}|
% \end{quote}
%
% \StopEventually{
% }
%
% \section{Implementation}
%
%    \begin{macrocode}
%<*package>
%    \end{macrocode}
%
% \subsection{Reload check and package identification}
%    Reload check, especially if the package is not used with \LaTeX.
%    \begin{macrocode}
\begingroup\catcode61\catcode48\catcode32=10\relax%
  \catcode13=5 % ^^M
  \endlinechar=13 %
  \catcode35=6 % #
  \catcode39=12 % '
  \catcode44=12 % ,
  \catcode45=12 % -
  \catcode46=12 % .
  \catcode58=12 % :
  \catcode64=11 % @
  \catcode123=1 % {
  \catcode125=2 % }
  \expandafter\let\expandafter\x\csname ver@protecteddef.sty\endcsname
  \ifx\x\relax % plain-TeX, first loading
  \else
    \def\empty{}%
    \ifx\x\empty % LaTeX, first loading,
      % variable is initialized, but \ProvidesPackage not yet seen
    \else
      \expandafter\ifx\csname PackageInfo\endcsname\relax
        \def\x#1#2{%
          \immediate\write-1{Package #1 Info: #2.}%
        }%
      \else
        \def\x#1#2{\PackageInfo{#1}{#2, stopped}}%
      \fi
      \x{protecteddef}{The package is already loaded}%
      \aftergroup\endinput
    \fi
  \fi
\endgroup%
%    \end{macrocode}
%    Package identification:
%    \begin{macrocode}
\begingroup\catcode61\catcode48\catcode32=10\relax%
  \catcode13=5 % ^^M
  \endlinechar=13 %
  \catcode35=6 % #
  \catcode39=12 % '
  \catcode40=12 % (
  \catcode41=12 % )
  \catcode44=12 % ,
  \catcode45=12 % -
  \catcode46=12 % .
  \catcode47=12 % /
  \catcode58=12 % :
  \catcode64=11 % @
  \catcode91=12 % [
  \catcode93=12 % ]
  \catcode123=1 % {
  \catcode125=2 % }
  \expandafter\ifx\csname ProvidesPackage\endcsname\relax
    \def\x#1#2#3[#4]{\endgroup
      \immediate\write-1{Package: #3 #4}%
      \xdef#1{#4}%
    }%
  \else
    \def\x#1#2[#3]{\endgroup
      #2[{#3}]%
      \ifx#1\@undefined
        \xdef#1{#3}%
      \fi
      \ifx#1\relax
        \xdef#1{#3}%
      \fi
    }%
  \fi
\expandafter\x\csname ver@protecteddef.sty\endcsname
\ProvidesPackage{protecteddef}%
  [2016/05/16 v1.1 Define protected commands (HO)]%
%    \end{macrocode}
%
% \subsection{Catcodes}
%
%    \begin{macrocode}
\begingroup\catcode61\catcode48\catcode32=10\relax%
  \catcode13=5 % ^^M
  \endlinechar=13 %
  \catcode123=1 % {
  \catcode125=2 % }
  \catcode64=11 % @
  \def\x{\endgroup
    \expandafter\edef\csname ProDef@AtEnd\endcsname{%
      \endlinechar=\the\endlinechar\relax
      \catcode13=\the\catcode13\relax
      \catcode32=\the\catcode32\relax
      \catcode35=\the\catcode35\relax
      \catcode61=\the\catcode61\relax
      \catcode64=\the\catcode64\relax
      \catcode123=\the\catcode123\relax
      \catcode125=\the\catcode125\relax
    }%
  }%
\x\catcode61\catcode48\catcode32=10\relax%
\catcode13=5 % ^^M
\endlinechar=13 %
\catcode35=6 % #
\catcode64=11 % @
\catcode123=1 % {
\catcode125=2 % }
\def\TMP@EnsureCode#1#2{%
  \edef\ProDef@AtEnd{%
    \ProDef@AtEnd
    \catcode#1=\the\catcode#1\relax
  }%
  \catcode#1=#2\relax
}
\TMP@EnsureCode{38}{4}% &
\TMP@EnsureCode{40}{12}% (
\TMP@EnsureCode{41}{12}% )
\TMP@EnsureCode{42}{12}% *
\TMP@EnsureCode{45}{12}% -
\TMP@EnsureCode{46}{12}% .
\TMP@EnsureCode{47}{12}% /
\TMP@EnsureCode{91}{12}% [
\TMP@EnsureCode{93}{12}% ]
\TMP@EnsureCode{96}{12}% `
\edef\ProDef@AtEnd{\ProDef@AtEnd\noexpand\endinput}
%    \end{macrocode}
%
% \subsection{Resources}
%
%    \begin{macrocode}
\begingroup\expandafter\expandafter\expandafter\endgroup
\expandafter\ifx\csname RequirePackage\endcsname\relax
  \def\TMP@RequirePackage#1[#2]{%
    \begingroup\expandafter\expandafter\expandafter\endgroup
    \expandafter\ifx\csname ver@#1.sty\endcsname\relax
      \input #1.sty\relax
    \fi
  }%
\else
  \let\TMP@RequirePackage\RequirePackage
\fi
\TMP@RequirePackage{ltxcmds}[2010/12/12]%
\TMP@RequirePackage{infwarerr}[2010/04/08]%
%    \end{macrocode}
%
%    \begin{macrocode}
\def\ProDef@temp#1{%
  \expandafter\def\csname ProDef@param[#1]\endcsname % hash-ok
}
\expandafter\def\csname ProDef@param\endcsname{}
\ProDef@temp0{}
\ProDef@temp1{##1}
\ProDef@temp2{##1##2}
\ProDef@temp3{##1##2##3}
\ProDef@temp4{##1##2##3##4}
\ProDef@temp5{##1##2##3##4##5}
\ProDef@temp6{##1##2##3##4##5##6}
\ProDef@temp7{##1##2##3##4##5##7}
\ProDef@temp8{##1##2##3##4##5##7##8}
\ProDef@temp9{##1##2##3##4##5##7##8##9}
%    \end{macrocode}
%
%    \begin{macro}{\ProDef@IfDefinable}
%    \begin{macrocode}
\ltx@IfUndefined{@ifdefinable}{%
  \long\def\ProDef@IfDefinable#1{%
    \begingroup
      \escapechar=-1 %
    \ltx@ifundefined{\string#1}{%
      \endgroup
      \ltx@firstofone
    }{%
      \expandafter\endgroup
      \expandafter
      \edef\expandafter\ProDef@temp\expandafter{\string#1 }%
      \@PackageError{protecteddef}{%
        Command \ltx@backslashchar\ProDef@temp already defined%
      }\@ehc
      \ltx@gobbletwo
    }%
  }%
}{%
  \long\def\ProDef@IfDefinable#1{%
    \let\ProDef@next\ltx@gobbletwo
    \@ifdefinable{#1}{%
      \let\ProDef@next\ltx@firstofone
    }%
    \ProDef@next
  }%
}
%    \end{macrocode}
%    \end{macro}
%
%    \begin{macrocode}
\begingroup\expandafter\expandafter\expandafter\endgroup
\expandafter\ifx\csname protected\endcsname\relax
  \begingroup\expandafter\expandafter\expandafter\endgroup
  \expandafter\ifx\csname DeclareRobustCommand\endcsname\relax
    \catcode`\&=14 % comment
  \else
    \newcommand*{\ProtectedDef}{%
      \ltx@ifnextchar*{%
        \ProDef@ProtectedDef
      }{%
        \ProDef@ProtectedDef{}%
      }%
    }%
    \long\def\ProDef@ProtectedDef#1#2#3#{%
      \ProDef@IfDefinable{#2}{%
        \ltx@IfUndefined{ProDef@param#3}{%
          \DeclareRobustCommand*{#2}{}%
          \begingroup
            \escapechar=-1 %
            \def\ProDef@temp{#1}%
          \edef\x{\endgroup
            \ifx\ProDef@temp\ltx@empty
              \noexpand\long
            \fi
            \noexpand\def
            \expandafter\noexpand\csname\string#2 \endcsname
          }%
          \x#3%
        }{%
          \DeclareRobustCommand#1{#2}#3%
        }%
      }%
    }%
    \expandafter\expandafter\expandafter\ProDef@AtEnd
  \fi
\else
  \catcode`\&=9 % ignore
\fi%
\ProDef@IfDefinable\ProtectedDef{%
&  \protected
  \def\ProtectedDef%
}{%
  \ltx@ifnextchar*{%
    \let\ProDef@long\ltx@empty
    \expandafter\ProDef@ProtectedDef\ltx@gobble
  }{%
    \let\ProDef@long\long
    \ProDef@ProtectedDef
  }%
}
\long\def\ProDef@ProtectedDef#1#2#{%
  \ProDef@IfDefinable{#1}{%
    \ltx@IfUndefined{ProDef@param#2}{%
&     \protected
      \ProDef@long
      \def#1#2%
    }{%
&     \protected
      \ProDef@long
      \expandafter\expandafter\expandafter\def
      \expandafter\expandafter\expandafter#1%
      \csname ProDef@param#2\endcsname
    }%
  }%
}
%    \end{macrocode}
%
%    \begin{macrocode}
\ProDef@AtEnd%
%    \end{macrocode}
%    \begin{macrocode}
%</package>
%    \end{macrocode}
%
% \section{Test}
%
% \subsection{Catcode checks for loading}
%
%    \begin{macrocode}
%<*test1>
%    \end{macrocode}
%    \begin{macrocode}
\catcode`\{=1 %
\catcode`\}=2 %
\catcode`\#=6 %
\catcode`\@=11 %
\expandafter\ifx\csname count@\endcsname\relax
  \countdef\count@=255 %
\fi
\expandafter\ifx\csname @gobble\endcsname\relax
  \long\def\@gobble#1{}%
\fi
\expandafter\ifx\csname @firstofone\endcsname\relax
  \long\def\@firstofone#1{#1}%
\fi
\expandafter\ifx\csname loop\endcsname\relax
  \expandafter\@firstofone
\else
  \expandafter\@gobble
\fi
{%
  \def\loop#1\repeat{%
    \def\body{#1}%
    \iterate
  }%
  \def\iterate{%
    \body
      \let\next\iterate
    \else
      \let\next\relax
    \fi
    \next
  }%
  \let\repeat=\fi
}%
\def\RestoreCatcodes{}
\count@=0 %
\loop
  \edef\RestoreCatcodes{%
    \RestoreCatcodes
    \catcode\the\count@=\the\catcode\count@\relax
  }%
\ifnum\count@<255 %
  \advance\count@ 1 %
\repeat

\def\RangeCatcodeInvalid#1#2{%
  \count@=#1\relax
  \loop
    \catcode\count@=15 %
  \ifnum\count@<#2\relax
    \advance\count@ 1 %
  \repeat
}
\def\RangeCatcodeCheck#1#2#3{%
  \count@=#1\relax
  \loop
    \ifnum#3=\catcode\count@
    \else
      \errmessage{%
        Character \the\count@\space
        with wrong catcode \the\catcode\count@\space
        instead of \number#3%
      }%
    \fi
  \ifnum\count@<#2\relax
    \advance\count@ 1 %
  \repeat
}
\def\space{ }
\expandafter\ifx\csname LoadCommand\endcsname\relax
  \def\LoadCommand{\input protecteddef.sty\relax}%
\fi
\def\Test{%
  \RangeCatcodeInvalid{0}{47}%
  \RangeCatcodeInvalid{58}{64}%
  \RangeCatcodeInvalid{91}{96}%
  \RangeCatcodeInvalid{123}{255}%
  \catcode`\@=12 %
  \catcode`\\=0 %
  \catcode`\%=14 %
  \LoadCommand
  \RangeCatcodeCheck{0}{36}{15}%
  \RangeCatcodeCheck{37}{37}{14}%
  \RangeCatcodeCheck{38}{47}{15}%
  \RangeCatcodeCheck{48}{57}{12}%
  \RangeCatcodeCheck{58}{63}{15}%
  \RangeCatcodeCheck{64}{64}{12}%
  \RangeCatcodeCheck{65}{90}{11}%
  \RangeCatcodeCheck{91}{91}{15}%
  \RangeCatcodeCheck{92}{92}{0}%
  \RangeCatcodeCheck{93}{96}{15}%
  \RangeCatcodeCheck{97}{122}{11}%
  \RangeCatcodeCheck{123}{255}{15}%
  \RestoreCatcodes
}
\Test
\csname @@end\endcsname
\end
%    \end{macrocode}
%    \begin{macrocode}
%</test1>
%    \end{macrocode}
%
% \subsection{Test without \hologo{LaTeX} and \cs{protected}}
%
%    \begin{macrocode}
%<*test2>
\errorcontextlines=10000 %
\begingroup\expandafter\expandafter\expandafter\endgroup
\expandafter\ifx\csname RequirePackage\endcsname\relax
  \input protecteddef.sty\relax
  \catcode`\{=1 %
  \catcode`\}=2 %
  \catcode`\#=6 %
\else
  \RequirePackage{protecteddef}[2016/05/16]%
\fi
\begingroup\expandafter\expandafter\expandafter\endgroup
\expandafter\ifx\csname protected\endcsname\relax
  \let\pdef\def
\else
  \def\pdef{\protected\def}%
\fi
\def\msg#{\immediate\write16}
\countdef\errcount=2 %
\long\def\BeginCheck#1\ProtectedDef#2\EndCheck{%
  \begingroup
    \toks0={\ProtectedDef#2}%
    \msg{<<\the\toks0>>}%
  \endgroup
  \setbox0=\hbox{%
    #1%
    \ProtectedDef#2%
    \check\foo
  }%
  \ifdim\wd0=0pt\relax
  \else
    \errmessage{[Definition] Unwanted spaces?!}%
  \fi
  \setbox0=\hbox{%
    \def\fooinitial{XYZ}%
    \let\foo\fooinitial
    \errcount=0 %
    \expandafter\def\csname @PackageError\endcsname##1##2##3{%
      \advance\errcount by 1 %
    }%
    \expandafter\def\csname @notdefinable\endcsname{%
      \advance\errcount by 1 %
    }%
    \ProtectedDef#2%
    \ifnum\errcount=1 %
    \else
      \errmessage{1 error expected, but found: \the\errcount}%
    \fi
    \ifx\foo\fooinitial
    \else
      \def\space{ }%
      \errmessage{\string\foo\space is overwritten}%
    \fi
  }%
  \ifdim\wd0=0pt\relax
  \else
    \errmessage{[Error] Unwanted spaces?!}%
  \fi
}
\chardef\DeclareVersion=0 %
\begingroup\expandafter\expandafter\expandafter\endgroup
\expandafter\ifx\csname protected\endcsname\relax
  \begingroup\expandafter\expandafter\expandafter\endgroup
  \expandafter\ifx\csname DeclareRobustCommand\endcsname\relax
  \else
    \chardef\DeclareVersion=1 %
  \fi
\fi
\ifnum\DeclareVersion=0 %
  \def\check#1{%
    \ifx\cmp#1%
      \msg{* Test passed.}%
    \else
      \msg{}%
      \msg{[\meaning#1]}%
      \msg{[\meaning\cmp]}%
      \errmessage{Test failed!}%
    \fi
  }%
\else
  \def\check#1{%
    \begingroup
      \escapechar=-1 %
    \edef\x{\endgroup
      \def\noexpand\cs/{\string#1}%
    }\x
    \edef\CMP{%
      \noexpand\protect
      \expandafter\noexpand\csname\cs/ \endcsname
    }%
    \ifx\CMP#1%
      \expandafter\ifx\csname\cs/ \endcsname\cmp
        \msg{Test passed.}%
      \else
        \msg{}%
        \msg{[\expandafter\meaning\csname\cs/ \endcsname]}%
        \msg{[\meaning\cmp]}%
        \errmessage{Test failed!}%
      \fi
    \else
      \msg{}%
      \msg{[\meaning#1]}%
      \msg{[\meaning\CMP]}%
      \errmessage{Test failed!}%
    \fi
  }%
\fi

\tracingmacros=1

\BeginCheck
  \pdef\cmp{}%
  \ProtectedDef*\foo{}%
\EndCheck

\BeginCheck
  \pdef\cmp{}%
  \ProtectedDef*\foo[0]{}%
\EndCheck

\BeginCheck
  \pdef\cmp#1{<#1>}%
  \ProtectedDef*\foo[1]{<#1>}%
\EndCheck

\BeginCheck
  \pdef\cmp(#1){<#1>}%
  \ProtectedDef*\foo(#1){<#1>}%
\EndCheck

\BeginCheck
  \long\pdef\cmp{}%
  \ProtectedDef\foo{}%
\EndCheck

\BeginCheck
  \long\pdef\cmp{}%
  \ProtectedDef\foo[0]{}%
\EndCheck

\BeginCheck
  \long\pdef\cmp#1{<#1>}%
  \ProtectedDef\foo[1]{<#1>}%
\EndCheck

\BeginCheck
  \long\pdef\cmp(#1){<#1>}%
  \ProtectedDef\foo(#1){<#1>}%
\EndCheck

\csname @@end\endcsname\end
%</test2>
%    \end{macrocode}
%
% \section{Installation}
%
% \subsection{Download}
%
% \paragraph{Package.} This package is available on
% CTAN\footnote{\CTANpkg{protecteddef}}:
% \begin{description}
% \item[\CTAN{macros/latex/contrib/oberdiek/protecteddef.dtx}] The source file.
% \item[\CTAN{macros/latex/contrib/oberdiek/protecteddef.pdf}] Documentation.
% \end{description}
%
%
% \paragraph{Bundle.} All the packages of the bundle `oberdiek'
% are also available in a TDS compliant ZIP archive. There
% the packages are already unpacked and the documentation files
% are generated. The files and directories obey the TDS standard.
% \begin{description}
% \item[\CTANinstall{install/macros/latex/contrib/oberdiek.tds.zip}]
% \end{description}
% \emph{TDS} refers to the standard ``A Directory Structure
% for \TeX\ Files'' (\CTAN{tds/tds.pdf}). Directories
% with \xfile{texmf} in their name are usually organized this way.
%
% \subsection{Bundle installation}
%
% \paragraph{Unpacking.} Unpack the \xfile{oberdiek.tds.zip} in the
% TDS tree (also known as \xfile{texmf} tree) of your choice.
% Example (linux):
% \begin{quote}
%   |unzip oberdiek.tds.zip -d ~/texmf|
% \end{quote}
%
% \paragraph{Script installation.}
% Check the directory \xfile{TDS:scripts/oberdiek/} for
% scripts that need further installation steps.
%
% \subsection{Package installation}
%
% \paragraph{Unpacking.} The \xfile{.dtx} file is a self-extracting
% \docstrip\ archive. The files are extracted by running the
% \xfile{.dtx} through \plainTeX:
% \begin{quote}
%   \verb|tex protecteddef.dtx|
% \end{quote}
%
% \paragraph{TDS.} Now the different files must be moved into
% the different directories in your installation TDS tree
% (also known as \xfile{texmf} tree):
% \begin{quote}
% \def\t{^^A
% \begin{tabular}{@{}>{\ttfamily}l@{ $\rightarrow$ }>{\ttfamily}l@{}}
%   protecteddef.sty & tex/generic/oberdiek/protecteddef.sty\\
%   protecteddef.pdf & doc/latex/oberdiek/protecteddef.pdf\\
%   test/protecteddef-test1.tex & doc/latex/oberdiek/test/protecteddef-test1.tex\\
%   test/protecteddef-test2.tex & doc/latex/oberdiek/test/protecteddef-test2.tex\\
%   protecteddef.dtx & source/latex/oberdiek/protecteddef.dtx\\
% \end{tabular}^^A
% }^^A
% \sbox0{\t}^^A
% \ifdim\wd0>\linewidth
%   \begingroup
%     \advance\linewidth by\leftmargin
%     \advance\linewidth by\rightmargin
%   \edef\x{\endgroup
%     \def\noexpand\lw{\the\linewidth}^^A
%   }\x
%   \def\lwbox{^^A
%     \leavevmode
%     \hbox to \linewidth{^^A
%       \kern-\leftmargin\relax
%       \hss
%       \usebox0
%       \hss
%       \kern-\rightmargin\relax
%     }^^A
%   }^^A
%   \ifdim\wd0>\lw
%     \sbox0{\small\t}^^A
%     \ifdim\wd0>\linewidth
%       \ifdim\wd0>\lw
%         \sbox0{\footnotesize\t}^^A
%         \ifdim\wd0>\linewidth
%           \ifdim\wd0>\lw
%             \sbox0{\scriptsize\t}^^A
%             \ifdim\wd0>\linewidth
%               \ifdim\wd0>\lw
%                 \sbox0{\tiny\t}^^A
%                 \ifdim\wd0>\linewidth
%                   \lwbox
%                 \else
%                   \usebox0
%                 \fi
%               \else
%                 \lwbox
%               \fi
%             \else
%               \usebox0
%             \fi
%           \else
%             \lwbox
%           \fi
%         \else
%           \usebox0
%         \fi
%       \else
%         \lwbox
%       \fi
%     \else
%       \usebox0
%     \fi
%   \else
%     \lwbox
%   \fi
% \else
%   \usebox0
% \fi
% \end{quote}
% If you have a \xfile{docstrip.cfg} that configures and enables \docstrip's
% TDS installing feature, then some files can already be in the right
% place, see the documentation of \docstrip.
%
% \subsection{Refresh file name databases}
%
% If your \TeX~distribution
% (\TeX\,Live, \mikTeX, \dots) relies on file name databases, you must refresh
% these. For example, \TeX\,Live\ users run \verb|texhash| or
% \verb|mktexlsr|.
%
% \subsection{Some details for the interested}
%
% \paragraph{Unpacking with \LaTeX.}
% The \xfile{.dtx} chooses its action depending on the format:
% \begin{description}
% \item[\plainTeX:] Run \docstrip\ and extract the files.
% \item[\LaTeX:] Generate the documentation.
% \end{description}
% If you insist on using \LaTeX\ for \docstrip\ (really,
% \docstrip\ does not need \LaTeX), then inform the autodetect routine
% about your intention:
% \begin{quote}
%   \verb|latex \let\install=y% \iffalse meta-comment
%
% File: protecteddef.dtx
% Version: 2016/05/16 v1.1
% Info: Define protected commands
%
% Copyright (C)
%    2011 Heiko Oberdiek
%    2016-2019 Oberdiek Package Support Group
%    https://github.com/ho-tex/oberdiek/issues
%
% This work may be distributed and/or modified under the
% conditions of the LaTeX Project Public License, either
% version 1.3c of this license or (at your option) any later
% version. This version of this license is in
%    https://www.latex-project.org/lppl/lppl-1-3c.txt
% and the latest version of this license is in
%    https://www.latex-project.org/lppl.txt
% and version 1.3 or later is part of all distributions of
% LaTeX version 2005/12/01 or later.
%
% This work has the LPPL maintenance status "maintained".
%
% The Current Maintainers of this work are
% Heiko Oberdiek and the Oberdiek Package Support Group
% https://github.com/ho-tex/oberdiek/issues
%
% The Base Interpreter refers to any `TeX-Format',
% because some files are installed in TDS:tex/generic//.
%
% This work consists of the main source file protecteddef.dtx
% and the derived files
%    protecteddef.sty, protecteddef.pdf, protecteddef.ins, protecteddef.drv,
%    protecteddef-test1.tex, protecteddef-test2.tex.
%
% Distribution:
%    CTAN:macros/latex/contrib/oberdiek/protecteddef.dtx
%    CTAN:macros/latex/contrib/oberdiek/protecteddef.pdf
%
% Unpacking:
%    (a) If protecteddef.ins is present:
%           tex protecteddef.ins
%    (b) Without protecteddef.ins:
%           tex protecteddef.dtx
%    (c) If you insist on using LaTeX
%           latex \let\install=y\input{protecteddef.dtx}
%        (quote the arguments according to the demands of your shell)
%
% Documentation:
%    (a) If protecteddef.drv is present:
%           latex protecteddef.drv
%    (b) Without protecteddef.drv:
%           latex protecteddef.dtx; ...
%    The class ltxdoc loads the configuration file ltxdoc.cfg
%    if available. Here you can specify further options, e.g.
%    use A4 as paper format:
%       \PassOptionsToClass{a4paper}{article}
%
%    Programm calls to get the documentation (example):
%       pdflatex protecteddef.dtx
%       makeindex -s gind.ist protecteddef.idx
%       pdflatex protecteddef.dtx
%       makeindex -s gind.ist protecteddef.idx
%       pdflatex protecteddef.dtx
%
% Installation:
%    TDS:tex/generic/oberdiek/protecteddef.sty
%    TDS:doc/latex/oberdiek/protecteddef.pdf
%    TDS:doc/latex/oberdiek/test/protecteddef-test1.tex
%    TDS:doc/latex/oberdiek/test/protecteddef-test2.tex
%    TDS:source/latex/oberdiek/protecteddef.dtx
%
%<*ignore>
\begingroup
  \catcode123=1 %
  \catcode125=2 %
  \def\x{LaTeX2e}%
\expandafter\endgroup
\ifcase 0\ifx\install y1\fi\expandafter
         \ifx\csname processbatchFile\endcsname\relax\else1\fi
         \ifx\fmtname\x\else 1\fi\relax
\else\csname fi\endcsname
%</ignore>
%<*install>
\input docstrip.tex
\Msg{************************************************************************}
\Msg{* Installation}
\Msg{* Package: protecteddef 2016/05/16 v1.1 Define protected commands (HO)}
\Msg{************************************************************************}

\keepsilent
\askforoverwritefalse

\let\MetaPrefix\relax
\preamble

This is a generated file.

Project: protecteddef
Version: 2016/05/16 v1.1

Copyright (C)
   2011 Heiko Oberdiek
   2016-2019 Oberdiek Package Support Group

This work may be distributed and/or modified under the
conditions of the LaTeX Project Public License, either
version 1.3c of this license or (at your option) any later
version. This version of this license is in
   https://www.latex-project.org/lppl/lppl-1-3c.txt
and the latest version of this license is in
   https://www.latex-project.org/lppl.txt
and version 1.3 or later is part of all distributions of
LaTeX version 2005/12/01 or later.

This work has the LPPL maintenance status "maintained".

The Current Maintainers of this work are
Heiko Oberdiek and the Oberdiek Package Support Group
https://github.com/ho-tex/oberdiek/issues


The Base Interpreter refers to any `TeX-Format',
because some files are installed in TDS:tex/generic//.

This work consists of the main source file protecteddef.dtx
and the derived files
   protecteddef.sty, protecteddef.pdf, protecteddef.ins, protecteddef.drv,
   protecteddef-test1.tex, protecteddef-test2.tex.

\endpreamble
\let\MetaPrefix\DoubleperCent

\generate{%
  \file{protecteddef.ins}{\from{protecteddef.dtx}{install}}%
  \file{protecteddef.drv}{\from{protecteddef.dtx}{driver}}%
  \usedir{tex/generic/oberdiek}%
  \file{protecteddef.sty}{\from{protecteddef.dtx}{package}}%
%  \usedir{doc/latex/oberdiek/test}%
%  \file{protecteddef-test1.tex}{\from{protecteddef.dtx}{test1}}%
%  \file{protecteddef-test2.tex}{\from{protecteddef.dtx}{test2}}%
  \nopreamble
  \nopostamble
%  \usedir{source/latex/oberdiek/catalogue}%
%  \file{protecteddef.xml}{\from{protecteddef.dtx}{catalogue}}%
}

\catcode32=13\relax% active space
\let =\space%
\Msg{************************************************************************}
\Msg{*}
\Msg{* To finish the installation you have to move the following}
\Msg{* file into a directory searched by TeX:}
\Msg{*}
\Msg{*     protecteddef.sty}
\Msg{*}
\Msg{* To produce the documentation run the file `protecteddef.drv'}
\Msg{* through LaTeX.}
\Msg{*}
\Msg{* Happy TeXing!}
\Msg{*}
\Msg{************************************************************************}

\endbatchfile
%</install>
%<*ignore>
\fi
%</ignore>
%<*driver>
\NeedsTeXFormat{LaTeX2e}
\ProvidesFile{protecteddef.drv}%
  [2016/05/16 v1.1 Define protected commands (HO)]%
\documentclass{ltxdoc}
\usepackage{holtxdoc}[2011/11/22]
\begin{document}
  \DocInput{protecteddef.dtx}%
\end{document}
%</driver>
% \fi
%
%
% \CharacterTable
%  {Upper-case    \A\B\C\D\E\F\G\H\I\J\K\L\M\N\O\P\Q\R\S\T\U\V\W\X\Y\Z
%   Lower-case    \a\b\c\d\e\f\g\h\i\j\k\l\m\n\o\p\q\r\s\t\u\v\w\x\y\z
%   Digits        \0\1\2\3\4\5\6\7\8\9
%   Exclamation   \!     Double quote  \"     Hash (number) \#
%   Dollar        \$     Percent       \%     Ampersand     \&
%   Acute accent  \'     Left paren    \(     Right paren   \)
%   Asterisk      \*     Plus          \+     Comma         \,
%   Minus         \-     Point         \.     Solidus       \/
%   Colon         \:     Semicolon     \;     Less than     \<
%   Equals        \=     Greater than  \>     Question mark \?
%   Commercial at \@     Left bracket  \[     Backslash     \\
%   Right bracket \]     Circumflex    \^     Underscore    \_
%   Grave accent  \`     Left brace    \{     Vertical bar  \|
%   Right brace   \}     Tilde         \~}
%
% \GetFileInfo{protecteddef.drv}
%
% \title{The \xpackage{protecteddef} package}
% \date{2016/05/16 v1.1}
% \author{Heiko Oberdiek\thanks
% {Please report any issues at \url{https://github.com/ho-tex/oberdiek/issues}}}
%
% \maketitle
%
% \begin{abstract}
% This packages provides \cs{ProtectedDef} for defining
% robust macros for both \hologo{plainTeX} and \hologo{LaTeX}.
% First \hologo{eTeX}'s \cs{protected} is tried, then
% \hologo{LaTeX}'s \cs{DeclareRobustCommand} is used.
% Otherwise the macro is not made robust.
% \end{abstract}
%
% \tableofcontents
%
% \section{Documentation}
%
% Many of my packages work for both formats \hologo{plainTeX}
% and \hologo{LaTeX}, even \hologo{iniTeX} is often supported.
% It would be nice if fragile macros could be protected and
% made robust. However the different format worlds offer different
% solutions.
%
% \subsection{The \hologo{LaTeX}'s way}
%
% Usually \cs{newcommand} is used to define macros. It provides
% a check if the command to be defined is already defined or
% cannot be defined for other reasons.
%
% For making robust macros \hologo{LaTeX} provides \cs{DeclareRobustCommand}.
% It shares the syntax with \cs{newcommand}. However it does not provide
% latters check. Internally the check is available via \cs{@ifdefinable}.
%
% Internally the robust macro is using \cs{protect} with a nested
% macro definition. The \cs{protect} infrastructure is a feature of
% \cs{LaTeX} and usually not available in other formats.
%
% \subsection{The \hologo{eTeX}'s way}
%
% The need for robust macros is addressed in \cs{eTeX}. It provides
% \cs{protected} that modifies the behaviour of \cs{def} in a similar
% way as \cs{long}. A protected macro does not expand in
% some expandable contexts like writing to a file or \cs{edef}.
%
% \subsection{The way of this package}
%
% The package tries to find the available protection mechanism.
% First it looks for \cs{eTeX}'s \cs{protected}, then
% it uses \hologo{LaTeX}'s \cs{DeclareRobustCommand}.
% If both fails, then the macro remains unprotected.
%
% Additionally, \cs{LaTeX}'s check, if a macro is already defined
% is added in all cases. First \hologo{LaTeX}'s \cs{@ifdefinable}
% is tried to be compatible with \hologo{LaTeX}.
% If \cs{@ifdefinable} is not available, then the test is implemented
% by asserting that the macro is undefined or has the meaning of \cs{relax}.
% If the test fails, then in all cases the macro is not defined
% and an error is thrown.
%
% \subsection{Usage}
%
% \begin{declcs}{ProtectedDef} \* \M{cmd} [\meta{num}] \M{definition text}
% \end{declcs}
% Macro \cs{ProtectedDef} follows the syntax of \hologo{LaTeX}'s
% \cs{newcommand} with the exception that an optional argument
% is not supported. Macro \meta{cmd} is to be defined as
% \cs{long} macro without star with \meta{num} arguments.
%
% The number of arguments \meta{num} must be given as explicite digit
% |0| upto |9|. Otherwise the part between the argument \meta{cmd} and
% the \meta{definition text} is taken as parameter text in the syntax
% of vanilla \hologo{TeX}. Examples (with \cs{protected}):
% \begin{quote}
%   \def\to{$\Rightarrow$ }%
%   |\ProtectedDef*{\foo}[1]{\message{#1}}|\\
%   \to |\protected\def\foo#1{\message#1}}|\\[1ex]
%   |\ProtectedDef\foo{abc}|\\
%   \to |\protected\def\foo{abc}|\\[1ex]
%   |\ProtectedDef*\foo(#1)<#2>{#1/#2}|\\
%   \to |\protected\def\foo(#1)<#2>{#1/#2}|
% \end{quote}
%
% \StopEventually{
% }
%
% \section{Implementation}
%
%    \begin{macrocode}
%<*package>
%    \end{macrocode}
%
% \subsection{Reload check and package identification}
%    Reload check, especially if the package is not used with \LaTeX.
%    \begin{macrocode}
\begingroup\catcode61\catcode48\catcode32=10\relax%
  \catcode13=5 % ^^M
  \endlinechar=13 %
  \catcode35=6 % #
  \catcode39=12 % '
  \catcode44=12 % ,
  \catcode45=12 % -
  \catcode46=12 % .
  \catcode58=12 % :
  \catcode64=11 % @
  \catcode123=1 % {
  \catcode125=2 % }
  \expandafter\let\expandafter\x\csname ver@protecteddef.sty\endcsname
  \ifx\x\relax % plain-TeX, first loading
  \else
    \def\empty{}%
    \ifx\x\empty % LaTeX, first loading,
      % variable is initialized, but \ProvidesPackage not yet seen
    \else
      \expandafter\ifx\csname PackageInfo\endcsname\relax
        \def\x#1#2{%
          \immediate\write-1{Package #1 Info: #2.}%
        }%
      \else
        \def\x#1#2{\PackageInfo{#1}{#2, stopped}}%
      \fi
      \x{protecteddef}{The package is already loaded}%
      \aftergroup\endinput
    \fi
  \fi
\endgroup%
%    \end{macrocode}
%    Package identification:
%    \begin{macrocode}
\begingroup\catcode61\catcode48\catcode32=10\relax%
  \catcode13=5 % ^^M
  \endlinechar=13 %
  \catcode35=6 % #
  \catcode39=12 % '
  \catcode40=12 % (
  \catcode41=12 % )
  \catcode44=12 % ,
  \catcode45=12 % -
  \catcode46=12 % .
  \catcode47=12 % /
  \catcode58=12 % :
  \catcode64=11 % @
  \catcode91=12 % [
  \catcode93=12 % ]
  \catcode123=1 % {
  \catcode125=2 % }
  \expandafter\ifx\csname ProvidesPackage\endcsname\relax
    \def\x#1#2#3[#4]{\endgroup
      \immediate\write-1{Package: #3 #4}%
      \xdef#1{#4}%
    }%
  \else
    \def\x#1#2[#3]{\endgroup
      #2[{#3}]%
      \ifx#1\@undefined
        \xdef#1{#3}%
      \fi
      \ifx#1\relax
        \xdef#1{#3}%
      \fi
    }%
  \fi
\expandafter\x\csname ver@protecteddef.sty\endcsname
\ProvidesPackage{protecteddef}%
  [2016/05/16 v1.1 Define protected commands (HO)]%
%    \end{macrocode}
%
% \subsection{Catcodes}
%
%    \begin{macrocode}
\begingroup\catcode61\catcode48\catcode32=10\relax%
  \catcode13=5 % ^^M
  \endlinechar=13 %
  \catcode123=1 % {
  \catcode125=2 % }
  \catcode64=11 % @
  \def\x{\endgroup
    \expandafter\edef\csname ProDef@AtEnd\endcsname{%
      \endlinechar=\the\endlinechar\relax
      \catcode13=\the\catcode13\relax
      \catcode32=\the\catcode32\relax
      \catcode35=\the\catcode35\relax
      \catcode61=\the\catcode61\relax
      \catcode64=\the\catcode64\relax
      \catcode123=\the\catcode123\relax
      \catcode125=\the\catcode125\relax
    }%
  }%
\x\catcode61\catcode48\catcode32=10\relax%
\catcode13=5 % ^^M
\endlinechar=13 %
\catcode35=6 % #
\catcode64=11 % @
\catcode123=1 % {
\catcode125=2 % }
\def\TMP@EnsureCode#1#2{%
  \edef\ProDef@AtEnd{%
    \ProDef@AtEnd
    \catcode#1=\the\catcode#1\relax
  }%
  \catcode#1=#2\relax
}
\TMP@EnsureCode{38}{4}% &
\TMP@EnsureCode{40}{12}% (
\TMP@EnsureCode{41}{12}% )
\TMP@EnsureCode{42}{12}% *
\TMP@EnsureCode{45}{12}% -
\TMP@EnsureCode{46}{12}% .
\TMP@EnsureCode{47}{12}% /
\TMP@EnsureCode{91}{12}% [
\TMP@EnsureCode{93}{12}% ]
\TMP@EnsureCode{96}{12}% `
\edef\ProDef@AtEnd{\ProDef@AtEnd\noexpand\endinput}
%    \end{macrocode}
%
% \subsection{Resources}
%
%    \begin{macrocode}
\begingroup\expandafter\expandafter\expandafter\endgroup
\expandafter\ifx\csname RequirePackage\endcsname\relax
  \def\TMP@RequirePackage#1[#2]{%
    \begingroup\expandafter\expandafter\expandafter\endgroup
    \expandafter\ifx\csname ver@#1.sty\endcsname\relax
      \input #1.sty\relax
    \fi
  }%
\else
  \let\TMP@RequirePackage\RequirePackage
\fi
\TMP@RequirePackage{ltxcmds}[2010/12/12]%
\TMP@RequirePackage{infwarerr}[2010/04/08]%
%    \end{macrocode}
%
%    \begin{macrocode}
\def\ProDef@temp#1{%
  \expandafter\def\csname ProDef@param[#1]\endcsname % hash-ok
}
\expandafter\def\csname ProDef@param\endcsname{}
\ProDef@temp0{}
\ProDef@temp1{##1}
\ProDef@temp2{##1##2}
\ProDef@temp3{##1##2##3}
\ProDef@temp4{##1##2##3##4}
\ProDef@temp5{##1##2##3##4##5}
\ProDef@temp6{##1##2##3##4##5##6}
\ProDef@temp7{##1##2##3##4##5##7}
\ProDef@temp8{##1##2##3##4##5##7##8}
\ProDef@temp9{##1##2##3##4##5##7##8##9}
%    \end{macrocode}
%
%    \begin{macro}{\ProDef@IfDefinable}
%    \begin{macrocode}
\ltx@IfUndefined{@ifdefinable}{%
  \long\def\ProDef@IfDefinable#1{%
    \begingroup
      \escapechar=-1 %
    \ltx@ifundefined{\string#1}{%
      \endgroup
      \ltx@firstofone
    }{%
      \expandafter\endgroup
      \expandafter
      \edef\expandafter\ProDef@temp\expandafter{\string#1 }%
      \@PackageError{protecteddef}{%
        Command \ltx@backslashchar\ProDef@temp already defined%
      }\@ehc
      \ltx@gobbletwo
    }%
  }%
}{%
  \long\def\ProDef@IfDefinable#1{%
    \let\ProDef@next\ltx@gobbletwo
    \@ifdefinable{#1}{%
      \let\ProDef@next\ltx@firstofone
    }%
    \ProDef@next
  }%
}
%    \end{macrocode}
%    \end{macro}
%
%    \begin{macrocode}
\begingroup\expandafter\expandafter\expandafter\endgroup
\expandafter\ifx\csname protected\endcsname\relax
  \begingroup\expandafter\expandafter\expandafter\endgroup
  \expandafter\ifx\csname DeclareRobustCommand\endcsname\relax
    \catcode`\&=14 % comment
  \else
    \newcommand*{\ProtectedDef}{%
      \ltx@ifnextchar*{%
        \ProDef@ProtectedDef
      }{%
        \ProDef@ProtectedDef{}%
      }%
    }%
    \long\def\ProDef@ProtectedDef#1#2#3#{%
      \ProDef@IfDefinable{#2}{%
        \ltx@IfUndefined{ProDef@param#3}{%
          \DeclareRobustCommand*{#2}{}%
          \begingroup
            \escapechar=-1 %
            \def\ProDef@temp{#1}%
          \edef\x{\endgroup
            \ifx\ProDef@temp\ltx@empty
              \noexpand\long
            \fi
            \noexpand\def
            \expandafter\noexpand\csname\string#2 \endcsname
          }%
          \x#3%
        }{%
          \DeclareRobustCommand#1{#2}#3%
        }%
      }%
    }%
    \expandafter\expandafter\expandafter\ProDef@AtEnd
  \fi
\else
  \catcode`\&=9 % ignore
\fi%
\ProDef@IfDefinable\ProtectedDef{%
&  \protected
  \def\ProtectedDef%
}{%
  \ltx@ifnextchar*{%
    \let\ProDef@long\ltx@empty
    \expandafter\ProDef@ProtectedDef\ltx@gobble
  }{%
    \let\ProDef@long\long
    \ProDef@ProtectedDef
  }%
}
\long\def\ProDef@ProtectedDef#1#2#{%
  \ProDef@IfDefinable{#1}{%
    \ltx@IfUndefined{ProDef@param#2}{%
&     \protected
      \ProDef@long
      \def#1#2%
    }{%
&     \protected
      \ProDef@long
      \expandafter\expandafter\expandafter\def
      \expandafter\expandafter\expandafter#1%
      \csname ProDef@param#2\endcsname
    }%
  }%
}
%    \end{macrocode}
%
%    \begin{macrocode}
\ProDef@AtEnd%
%    \end{macrocode}
%    \begin{macrocode}
%</package>
%    \end{macrocode}
%
% \section{Test}
%
% \subsection{Catcode checks for loading}
%
%    \begin{macrocode}
%<*test1>
%    \end{macrocode}
%    \begin{macrocode}
\catcode`\{=1 %
\catcode`\}=2 %
\catcode`\#=6 %
\catcode`\@=11 %
\expandafter\ifx\csname count@\endcsname\relax
  \countdef\count@=255 %
\fi
\expandafter\ifx\csname @gobble\endcsname\relax
  \long\def\@gobble#1{}%
\fi
\expandafter\ifx\csname @firstofone\endcsname\relax
  \long\def\@firstofone#1{#1}%
\fi
\expandafter\ifx\csname loop\endcsname\relax
  \expandafter\@firstofone
\else
  \expandafter\@gobble
\fi
{%
  \def\loop#1\repeat{%
    \def\body{#1}%
    \iterate
  }%
  \def\iterate{%
    \body
      \let\next\iterate
    \else
      \let\next\relax
    \fi
    \next
  }%
  \let\repeat=\fi
}%
\def\RestoreCatcodes{}
\count@=0 %
\loop
  \edef\RestoreCatcodes{%
    \RestoreCatcodes
    \catcode\the\count@=\the\catcode\count@\relax
  }%
\ifnum\count@<255 %
  \advance\count@ 1 %
\repeat

\def\RangeCatcodeInvalid#1#2{%
  \count@=#1\relax
  \loop
    \catcode\count@=15 %
  \ifnum\count@<#2\relax
    \advance\count@ 1 %
  \repeat
}
\def\RangeCatcodeCheck#1#2#3{%
  \count@=#1\relax
  \loop
    \ifnum#3=\catcode\count@
    \else
      \errmessage{%
        Character \the\count@\space
        with wrong catcode \the\catcode\count@\space
        instead of \number#3%
      }%
    \fi
  \ifnum\count@<#2\relax
    \advance\count@ 1 %
  \repeat
}
\def\space{ }
\expandafter\ifx\csname LoadCommand\endcsname\relax
  \def\LoadCommand{\input protecteddef.sty\relax}%
\fi
\def\Test{%
  \RangeCatcodeInvalid{0}{47}%
  \RangeCatcodeInvalid{58}{64}%
  \RangeCatcodeInvalid{91}{96}%
  \RangeCatcodeInvalid{123}{255}%
  \catcode`\@=12 %
  \catcode`\\=0 %
  \catcode`\%=14 %
  \LoadCommand
  \RangeCatcodeCheck{0}{36}{15}%
  \RangeCatcodeCheck{37}{37}{14}%
  \RangeCatcodeCheck{38}{47}{15}%
  \RangeCatcodeCheck{48}{57}{12}%
  \RangeCatcodeCheck{58}{63}{15}%
  \RangeCatcodeCheck{64}{64}{12}%
  \RangeCatcodeCheck{65}{90}{11}%
  \RangeCatcodeCheck{91}{91}{15}%
  \RangeCatcodeCheck{92}{92}{0}%
  \RangeCatcodeCheck{93}{96}{15}%
  \RangeCatcodeCheck{97}{122}{11}%
  \RangeCatcodeCheck{123}{255}{15}%
  \RestoreCatcodes
}
\Test
\csname @@end\endcsname
\end
%    \end{macrocode}
%    \begin{macrocode}
%</test1>
%    \end{macrocode}
%
% \subsection{Test without \hologo{LaTeX} and \cs{protected}}
%
%    \begin{macrocode}
%<*test2>
\errorcontextlines=10000 %
\begingroup\expandafter\expandafter\expandafter\endgroup
\expandafter\ifx\csname RequirePackage\endcsname\relax
  \input protecteddef.sty\relax
  \catcode`\{=1 %
  \catcode`\}=2 %
  \catcode`\#=6 %
\else
  \RequirePackage{protecteddef}[2016/05/16]%
\fi
\begingroup\expandafter\expandafter\expandafter\endgroup
\expandafter\ifx\csname protected\endcsname\relax
  \let\pdef\def
\else
  \def\pdef{\protected\def}%
\fi
\def\msg#{\immediate\write16}
\countdef\errcount=2 %
\long\def\BeginCheck#1\ProtectedDef#2\EndCheck{%
  \begingroup
    \toks0={\ProtectedDef#2}%
    \msg{<<\the\toks0>>}%
  \endgroup
  \setbox0=\hbox{%
    #1%
    \ProtectedDef#2%
    \check\foo
  }%
  \ifdim\wd0=0pt\relax
  \else
    \errmessage{[Definition] Unwanted spaces?!}%
  \fi
  \setbox0=\hbox{%
    \def\fooinitial{XYZ}%
    \let\foo\fooinitial
    \errcount=0 %
    \expandafter\def\csname @PackageError\endcsname##1##2##3{%
      \advance\errcount by 1 %
    }%
    \expandafter\def\csname @notdefinable\endcsname{%
      \advance\errcount by 1 %
    }%
    \ProtectedDef#2%
    \ifnum\errcount=1 %
    \else
      \errmessage{1 error expected, but found: \the\errcount}%
    \fi
    \ifx\foo\fooinitial
    \else
      \def\space{ }%
      \errmessage{\string\foo\space is overwritten}%
    \fi
  }%
  \ifdim\wd0=0pt\relax
  \else
    \errmessage{[Error] Unwanted spaces?!}%
  \fi
}
\chardef\DeclareVersion=0 %
\begingroup\expandafter\expandafter\expandafter\endgroup
\expandafter\ifx\csname protected\endcsname\relax
  \begingroup\expandafter\expandafter\expandafter\endgroup
  \expandafter\ifx\csname DeclareRobustCommand\endcsname\relax
  \else
    \chardef\DeclareVersion=1 %
  \fi
\fi
\ifnum\DeclareVersion=0 %
  \def\check#1{%
    \ifx\cmp#1%
      \msg{* Test passed.}%
    \else
      \msg{}%
      \msg{[\meaning#1]}%
      \msg{[\meaning\cmp]}%
      \errmessage{Test failed!}%
    \fi
  }%
\else
  \def\check#1{%
    \begingroup
      \escapechar=-1 %
    \edef\x{\endgroup
      \def\noexpand\cs/{\string#1}%
    }\x
    \edef\CMP{%
      \noexpand\protect
      \expandafter\noexpand\csname\cs/ \endcsname
    }%
    \ifx\CMP#1%
      \expandafter\ifx\csname\cs/ \endcsname\cmp
        \msg{Test passed.}%
      \else
        \msg{}%
        \msg{[\expandafter\meaning\csname\cs/ \endcsname]}%
        \msg{[\meaning\cmp]}%
        \errmessage{Test failed!}%
      \fi
    \else
      \msg{}%
      \msg{[\meaning#1]}%
      \msg{[\meaning\CMP]}%
      \errmessage{Test failed!}%
    \fi
  }%
\fi

\tracingmacros=1

\BeginCheck
  \pdef\cmp{}%
  \ProtectedDef*\foo{}%
\EndCheck

\BeginCheck
  \pdef\cmp{}%
  \ProtectedDef*\foo[0]{}%
\EndCheck

\BeginCheck
  \pdef\cmp#1{<#1>}%
  \ProtectedDef*\foo[1]{<#1>}%
\EndCheck

\BeginCheck
  \pdef\cmp(#1){<#1>}%
  \ProtectedDef*\foo(#1){<#1>}%
\EndCheck

\BeginCheck
  \long\pdef\cmp{}%
  \ProtectedDef\foo{}%
\EndCheck

\BeginCheck
  \long\pdef\cmp{}%
  \ProtectedDef\foo[0]{}%
\EndCheck

\BeginCheck
  \long\pdef\cmp#1{<#1>}%
  \ProtectedDef\foo[1]{<#1>}%
\EndCheck

\BeginCheck
  \long\pdef\cmp(#1){<#1>}%
  \ProtectedDef\foo(#1){<#1>}%
\EndCheck

\csname @@end\endcsname\end
%</test2>
%    \end{macrocode}
%
% \section{Installation}
%
% \subsection{Download}
%
% \paragraph{Package.} This package is available on
% CTAN\footnote{\CTANpkg{protecteddef}}:
% \begin{description}
% \item[\CTAN{macros/latex/contrib/oberdiek/protecteddef.dtx}] The source file.
% \item[\CTAN{macros/latex/contrib/oberdiek/protecteddef.pdf}] Documentation.
% \end{description}
%
%
% \paragraph{Bundle.} All the packages of the bundle `oberdiek'
% are also available in a TDS compliant ZIP archive. There
% the packages are already unpacked and the documentation files
% are generated. The files and directories obey the TDS standard.
% \begin{description}
% \item[\CTANinstall{install/macros/latex/contrib/oberdiek.tds.zip}]
% \end{description}
% \emph{TDS} refers to the standard ``A Directory Structure
% for \TeX\ Files'' (\CTAN{tds/tds.pdf}). Directories
% with \xfile{texmf} in their name are usually organized this way.
%
% \subsection{Bundle installation}
%
% \paragraph{Unpacking.} Unpack the \xfile{oberdiek.tds.zip} in the
% TDS tree (also known as \xfile{texmf} tree) of your choice.
% Example (linux):
% \begin{quote}
%   |unzip oberdiek.tds.zip -d ~/texmf|
% \end{quote}
%
% \paragraph{Script installation.}
% Check the directory \xfile{TDS:scripts/oberdiek/} for
% scripts that need further installation steps.
%
% \subsection{Package installation}
%
% \paragraph{Unpacking.} The \xfile{.dtx} file is a self-extracting
% \docstrip\ archive. The files are extracted by running the
% \xfile{.dtx} through \plainTeX:
% \begin{quote}
%   \verb|tex protecteddef.dtx|
% \end{quote}
%
% \paragraph{TDS.} Now the different files must be moved into
% the different directories in your installation TDS tree
% (also known as \xfile{texmf} tree):
% \begin{quote}
% \def\t{^^A
% \begin{tabular}{@{}>{\ttfamily}l@{ $\rightarrow$ }>{\ttfamily}l@{}}
%   protecteddef.sty & tex/generic/oberdiek/protecteddef.sty\\
%   protecteddef.pdf & doc/latex/oberdiek/protecteddef.pdf\\
%   test/protecteddef-test1.tex & doc/latex/oberdiek/test/protecteddef-test1.tex\\
%   test/protecteddef-test2.tex & doc/latex/oberdiek/test/protecteddef-test2.tex\\
%   protecteddef.dtx & source/latex/oberdiek/protecteddef.dtx\\
% \end{tabular}^^A
% }^^A
% \sbox0{\t}^^A
% \ifdim\wd0>\linewidth
%   \begingroup
%     \advance\linewidth by\leftmargin
%     \advance\linewidth by\rightmargin
%   \edef\x{\endgroup
%     \def\noexpand\lw{\the\linewidth}^^A
%   }\x
%   \def\lwbox{^^A
%     \leavevmode
%     \hbox to \linewidth{^^A
%       \kern-\leftmargin\relax
%       \hss
%       \usebox0
%       \hss
%       \kern-\rightmargin\relax
%     }^^A
%   }^^A
%   \ifdim\wd0>\lw
%     \sbox0{\small\t}^^A
%     \ifdim\wd0>\linewidth
%       \ifdim\wd0>\lw
%         \sbox0{\footnotesize\t}^^A
%         \ifdim\wd0>\linewidth
%           \ifdim\wd0>\lw
%             \sbox0{\scriptsize\t}^^A
%             \ifdim\wd0>\linewidth
%               \ifdim\wd0>\lw
%                 \sbox0{\tiny\t}^^A
%                 \ifdim\wd0>\linewidth
%                   \lwbox
%                 \else
%                   \usebox0
%                 \fi
%               \else
%                 \lwbox
%               \fi
%             \else
%               \usebox0
%             \fi
%           \else
%             \lwbox
%           \fi
%         \else
%           \usebox0
%         \fi
%       \else
%         \lwbox
%       \fi
%     \else
%       \usebox0
%     \fi
%   \else
%     \lwbox
%   \fi
% \else
%   \usebox0
% \fi
% \end{quote}
% If you have a \xfile{docstrip.cfg} that configures and enables \docstrip's
% TDS installing feature, then some files can already be in the right
% place, see the documentation of \docstrip.
%
% \subsection{Refresh file name databases}
%
% If your \TeX~distribution
% (\TeX\,Live, \mikTeX, \dots) relies on file name databases, you must refresh
% these. For example, \TeX\,Live\ users run \verb|texhash| or
% \verb|mktexlsr|.
%
% \subsection{Some details for the interested}
%
% \paragraph{Unpacking with \LaTeX.}
% The \xfile{.dtx} chooses its action depending on the format:
% \begin{description}
% \item[\plainTeX:] Run \docstrip\ and extract the files.
% \item[\LaTeX:] Generate the documentation.
% \end{description}
% If you insist on using \LaTeX\ for \docstrip\ (really,
% \docstrip\ does not need \LaTeX), then inform the autodetect routine
% about your intention:
% \begin{quote}
%   \verb|latex \let\install=y\input{protecteddef.dtx}|
% \end{quote}
% Do not forget to quote the argument according to the demands
% of your shell.
%
% \paragraph{Generating the documentation.}
% You can use both the \xfile{.dtx} or the \xfile{.drv} to generate
% the documentation. The process can be configured by the
% configuration file \xfile{ltxdoc.cfg}. For instance, put this
% line into this file, if you want to have A4 as paper format:
% \begin{quote}
%   \verb|\PassOptionsToClass{a4paper}{article}|
% \end{quote}
% An example follows how to generate the
% documentation with pdf\LaTeX:
% \begin{quote}
%\begin{verbatim}
%pdflatex protecteddef.dtx
%makeindex -s gind.ist protecteddef.idx
%pdflatex protecteddef.dtx
%makeindex -s gind.ist protecteddef.idx
%pdflatex protecteddef.dtx
%\end{verbatim}
% \end{quote}
%
% \begin{History}
%   \begin{Version}{2011/01/31 v1.0}
%   \item
%     First public version.
%   \end{Version}
%   \begin{Version}{2016/05/16 v1.1}
%   \item
%     Documentation updates.
%   \end{Version}
% \end{History}
%
% \PrintIndex
%
% \Finale
\endinput
|
% \end{quote}
% Do not forget to quote the argument according to the demands
% of your shell.
%
% \paragraph{Generating the documentation.}
% You can use both the \xfile{.dtx} or the \xfile{.drv} to generate
% the documentation. The process can be configured by the
% configuration file \xfile{ltxdoc.cfg}. For instance, put this
% line into this file, if you want to have A4 as paper format:
% \begin{quote}
%   \verb|\PassOptionsToClass{a4paper}{article}|
% \end{quote}
% An example follows how to generate the
% documentation with pdf\LaTeX:
% \begin{quote}
%\begin{verbatim}
%pdflatex protecteddef.dtx
%makeindex -s gind.ist protecteddef.idx
%pdflatex protecteddef.dtx
%makeindex -s gind.ist protecteddef.idx
%pdflatex protecteddef.dtx
%\end{verbatim}
% \end{quote}
%
% \begin{History}
%   \begin{Version}{2011/01/31 v1.0}
%   \item
%     First public version.
%   \end{Version}
%   \begin{Version}{2016/05/16 v1.1}
%   \item
%     Documentation updates.
%   \end{Version}
% \end{History}
%
% \PrintIndex
%
% \Finale
\endinput

%        (quote the arguments according to the demands of your shell)
%
% Documentation:
%    (a) If protecteddef.drv is present:
%           latex protecteddef.drv
%    (b) Without protecteddef.drv:
%           latex protecteddef.dtx; ...
%    The class ltxdoc loads the configuration file ltxdoc.cfg
%    if available. Here you can specify further options, e.g.
%    use A4 as paper format:
%       \PassOptionsToClass{a4paper}{article}
%
%    Programm calls to get the documentation (example):
%       pdflatex protecteddef.dtx
%       makeindex -s gind.ist protecteddef.idx
%       pdflatex protecteddef.dtx
%       makeindex -s gind.ist protecteddef.idx
%       pdflatex protecteddef.dtx
%
% Installation:
%    TDS:tex/generic/oberdiek/protecteddef.sty
%    TDS:doc/latex/oberdiek/protecteddef.pdf
%    TDS:doc/latex/oberdiek/test/protecteddef-test1.tex
%    TDS:doc/latex/oberdiek/test/protecteddef-test2.tex
%    TDS:source/latex/oberdiek/protecteddef.dtx
%
%<*ignore>
\begingroup
  \catcode123=1 %
  \catcode125=2 %
  \def\x{LaTeX2e}%
\expandafter\endgroup
\ifcase 0\ifx\install y1\fi\expandafter
         \ifx\csname processbatchFile\endcsname\relax\else1\fi
         \ifx\fmtname\x\else 1\fi\relax
\else\csname fi\endcsname
%</ignore>
%<*install>
\input docstrip.tex
\Msg{************************************************************************}
\Msg{* Installation}
\Msg{* Package: protecteddef 2016/05/16 v1.1 Define protected commands (HO)}
\Msg{************************************************************************}

\keepsilent
\askforoverwritefalse

\let\MetaPrefix\relax
\preamble

This is a generated file.

Project: protecteddef
Version: 2016/05/16 v1.1

Copyright (C)
   2011 Heiko Oberdiek
   2016-2019 Oberdiek Package Support Group

This work may be distributed and/or modified under the
conditions of the LaTeX Project Public License, either
version 1.3c of this license or (at your option) any later
version. This version of this license is in
   https://www.latex-project.org/lppl/lppl-1-3c.txt
and the latest version of this license is in
   https://www.latex-project.org/lppl.txt
and version 1.3 or later is part of all distributions of
LaTeX version 2005/12/01 or later.

This work has the LPPL maintenance status "maintained".

The Current Maintainers of this work are
Heiko Oberdiek and the Oberdiek Package Support Group
https://github.com/ho-tex/oberdiek/issues


The Base Interpreter refers to any `TeX-Format',
because some files are installed in TDS:tex/generic//.

This work consists of the main source file protecteddef.dtx
and the derived files
   protecteddef.sty, protecteddef.pdf, protecteddef.ins, protecteddef.drv,
   protecteddef-test1.tex, protecteddef-test2.tex.

\endpreamble
\let\MetaPrefix\DoubleperCent

\generate{%
  \file{protecteddef.ins}{\from{protecteddef.dtx}{install}}%
  \file{protecteddef.drv}{\from{protecteddef.dtx}{driver}}%
  \usedir{tex/generic/oberdiek}%
  \file{protecteddef.sty}{\from{protecteddef.dtx}{package}}%
%  \usedir{doc/latex/oberdiek/test}%
%  \file{protecteddef-test1.tex}{\from{protecteddef.dtx}{test1}}%
%  \file{protecteddef-test2.tex}{\from{protecteddef.dtx}{test2}}%
  \nopreamble
  \nopostamble
%  \usedir{source/latex/oberdiek/catalogue}%
%  \file{protecteddef.xml}{\from{protecteddef.dtx}{catalogue}}%
}

\catcode32=13\relax% active space
\let =\space%
\Msg{************************************************************************}
\Msg{*}
\Msg{* To finish the installation you have to move the following}
\Msg{* file into a directory searched by TeX:}
\Msg{*}
\Msg{*     protecteddef.sty}
\Msg{*}
\Msg{* To produce the documentation run the file `protecteddef.drv'}
\Msg{* through LaTeX.}
\Msg{*}
\Msg{* Happy TeXing!}
\Msg{*}
\Msg{************************************************************************}

\endbatchfile
%</install>
%<*ignore>
\fi
%</ignore>
%<*driver>
\NeedsTeXFormat{LaTeX2e}
\ProvidesFile{protecteddef.drv}%
  [2016/05/16 v1.1 Define protected commands (HO)]%
\documentclass{ltxdoc}
\usepackage{holtxdoc}[2011/11/22]
\begin{document}
  \DocInput{protecteddef.dtx}%
\end{document}
%</driver>
% \fi
%
%
% \CharacterTable
%  {Upper-case    \A\B\C\D\E\F\G\H\I\J\K\L\M\N\O\P\Q\R\S\T\U\V\W\X\Y\Z
%   Lower-case    \a\b\c\d\e\f\g\h\i\j\k\l\m\n\o\p\q\r\s\t\u\v\w\x\y\z
%   Digits        \0\1\2\3\4\5\6\7\8\9
%   Exclamation   \!     Double quote  \"     Hash (number) \#
%   Dollar        \$     Percent       \%     Ampersand     \&
%   Acute accent  \'     Left paren    \(     Right paren   \)
%   Asterisk      \*     Plus          \+     Comma         \,
%   Minus         \-     Point         \.     Solidus       \/
%   Colon         \:     Semicolon     \;     Less than     \<
%   Equals        \=     Greater than  \>     Question mark \?
%   Commercial at \@     Left bracket  \[     Backslash     \\
%   Right bracket \]     Circumflex    \^     Underscore    \_
%   Grave accent  \`     Left brace    \{     Vertical bar  \|
%   Right brace   \}     Tilde         \~}
%
% \GetFileInfo{protecteddef.drv}
%
% \title{The \xpackage{protecteddef} package}
% \date{2016/05/16 v1.1}
% \author{Heiko Oberdiek\thanks
% {Please report any issues at \url{https://github.com/ho-tex/oberdiek/issues}}}
%
% \maketitle
%
% \begin{abstract}
% This packages provides \cs{ProtectedDef} for defining
% robust macros for both \hologo{plainTeX} and \hologo{LaTeX}.
% First \hologo{eTeX}'s \cs{protected} is tried, then
% \hologo{LaTeX}'s \cs{DeclareRobustCommand} is used.
% Otherwise the macro is not made robust.
% \end{abstract}
%
% \tableofcontents
%
% \section{Documentation}
%
% Many of my packages work for both formats \hologo{plainTeX}
% and \hologo{LaTeX}, even \hologo{iniTeX} is often supported.
% It would be nice if fragile macros could be protected and
% made robust. However the different format worlds offer different
% solutions.
%
% \subsection{The \hologo{LaTeX}'s way}
%
% Usually \cs{newcommand} is used to define macros. It provides
% a check if the command to be defined is already defined or
% cannot be defined for other reasons.
%
% For making robust macros \hologo{LaTeX} provides \cs{DeclareRobustCommand}.
% It shares the syntax with \cs{newcommand}. However it does not provide
% latters check. Internally the check is available via \cs{@ifdefinable}.
%
% Internally the robust macro is using \cs{protect} with a nested
% macro definition. The \cs{protect} infrastructure is a feature of
% \cs{LaTeX} and usually not available in other formats.
%
% \subsection{The \hologo{eTeX}'s way}
%
% The need for robust macros is addressed in \cs{eTeX}. It provides
% \cs{protected} that modifies the behaviour of \cs{def} in a similar
% way as \cs{long}. A protected macro does not expand in
% some expandable contexts like writing to a file or \cs{edef}.
%
% \subsection{The way of this package}
%
% The package tries to find the available protection mechanism.
% First it looks for \cs{eTeX}'s \cs{protected}, then
% it uses \hologo{LaTeX}'s \cs{DeclareRobustCommand}.
% If both fails, then the macro remains unprotected.
%
% Additionally, \cs{LaTeX}'s check, if a macro is already defined
% is added in all cases. First \hologo{LaTeX}'s \cs{@ifdefinable}
% is tried to be compatible with \hologo{LaTeX}.
% If \cs{@ifdefinable} is not available, then the test is implemented
% by asserting that the macro is undefined or has the meaning of \cs{relax}.
% If the test fails, then in all cases the macro is not defined
% and an error is thrown.
%
% \subsection{Usage}
%
% \begin{declcs}{ProtectedDef} \* \M{cmd} [\meta{num}] \M{definition text}
% \end{declcs}
% Macro \cs{ProtectedDef} follows the syntax of \hologo{LaTeX}'s
% \cs{newcommand} with the exception that an optional argument
% is not supported. Macro \meta{cmd} is to be defined as
% \cs{long} macro without star with \meta{num} arguments.
%
% The number of arguments \meta{num} must be given as explicite digit
% |0| upto |9|. Otherwise the part between the argument \meta{cmd} and
% the \meta{definition text} is taken as parameter text in the syntax
% of vanilla \hologo{TeX}. Examples (with \cs{protected}):
% \begin{quote}
%   \def\to{$\Rightarrow$ }%
%   |\ProtectedDef*{\foo}[1]{\message{#1}}|\\
%   \to |\protected\def\foo#1{\message#1}}|\\[1ex]
%   |\ProtectedDef\foo{abc}|\\
%   \to |\protected\def\foo{abc}|\\[1ex]
%   |\ProtectedDef*\foo(#1)<#2>{#1/#2}|\\
%   \to |\protected\def\foo(#1)<#2>{#1/#2}|
% \end{quote}
%
% \StopEventually{
% }
%
% \section{Implementation}
%
%    \begin{macrocode}
%<*package>
%    \end{macrocode}
%
% \subsection{Reload check and package identification}
%    Reload check, especially if the package is not used with \LaTeX.
%    \begin{macrocode}
\begingroup\catcode61\catcode48\catcode32=10\relax%
  \catcode13=5 % ^^M
  \endlinechar=13 %
  \catcode35=6 % #
  \catcode39=12 % '
  \catcode44=12 % ,
  \catcode45=12 % -
  \catcode46=12 % .
  \catcode58=12 % :
  \catcode64=11 % @
  \catcode123=1 % {
  \catcode125=2 % }
  \expandafter\let\expandafter\x\csname ver@protecteddef.sty\endcsname
  \ifx\x\relax % plain-TeX, first loading
  \else
    \def\empty{}%
    \ifx\x\empty % LaTeX, first loading,
      % variable is initialized, but \ProvidesPackage not yet seen
    \else
      \expandafter\ifx\csname PackageInfo\endcsname\relax
        \def\x#1#2{%
          \immediate\write-1{Package #1 Info: #2.}%
        }%
      \else
        \def\x#1#2{\PackageInfo{#1}{#2, stopped}}%
      \fi
      \x{protecteddef}{The package is already loaded}%
      \aftergroup\endinput
    \fi
  \fi
\endgroup%
%    \end{macrocode}
%    Package identification:
%    \begin{macrocode}
\begingroup\catcode61\catcode48\catcode32=10\relax%
  \catcode13=5 % ^^M
  \endlinechar=13 %
  \catcode35=6 % #
  \catcode39=12 % '
  \catcode40=12 % (
  \catcode41=12 % )
  \catcode44=12 % ,
  \catcode45=12 % -
  \catcode46=12 % .
  \catcode47=12 % /
  \catcode58=12 % :
  \catcode64=11 % @
  \catcode91=12 % [
  \catcode93=12 % ]
  \catcode123=1 % {
  \catcode125=2 % }
  \expandafter\ifx\csname ProvidesPackage\endcsname\relax
    \def\x#1#2#3[#4]{\endgroup
      \immediate\write-1{Package: #3 #4}%
      \xdef#1{#4}%
    }%
  \else
    \def\x#1#2[#3]{\endgroup
      #2[{#3}]%
      \ifx#1\@undefined
        \xdef#1{#3}%
      \fi
      \ifx#1\relax
        \xdef#1{#3}%
      \fi
    }%
  \fi
\expandafter\x\csname ver@protecteddef.sty\endcsname
\ProvidesPackage{protecteddef}%
  [2016/05/16 v1.1 Define protected commands (HO)]%
%    \end{macrocode}
%
% \subsection{Catcodes}
%
%    \begin{macrocode}
\begingroup\catcode61\catcode48\catcode32=10\relax%
  \catcode13=5 % ^^M
  \endlinechar=13 %
  \catcode123=1 % {
  \catcode125=2 % }
  \catcode64=11 % @
  \def\x{\endgroup
    \expandafter\edef\csname ProDef@AtEnd\endcsname{%
      \endlinechar=\the\endlinechar\relax
      \catcode13=\the\catcode13\relax
      \catcode32=\the\catcode32\relax
      \catcode35=\the\catcode35\relax
      \catcode61=\the\catcode61\relax
      \catcode64=\the\catcode64\relax
      \catcode123=\the\catcode123\relax
      \catcode125=\the\catcode125\relax
    }%
  }%
\x\catcode61\catcode48\catcode32=10\relax%
\catcode13=5 % ^^M
\endlinechar=13 %
\catcode35=6 % #
\catcode64=11 % @
\catcode123=1 % {
\catcode125=2 % }
\def\TMP@EnsureCode#1#2{%
  \edef\ProDef@AtEnd{%
    \ProDef@AtEnd
    \catcode#1=\the\catcode#1\relax
  }%
  \catcode#1=#2\relax
}
\TMP@EnsureCode{38}{4}% &
\TMP@EnsureCode{40}{12}% (
\TMP@EnsureCode{41}{12}% )
\TMP@EnsureCode{42}{12}% *
\TMP@EnsureCode{45}{12}% -
\TMP@EnsureCode{46}{12}% .
\TMP@EnsureCode{47}{12}% /
\TMP@EnsureCode{91}{12}% [
\TMP@EnsureCode{93}{12}% ]
\TMP@EnsureCode{96}{12}% `
\edef\ProDef@AtEnd{\ProDef@AtEnd\noexpand\endinput}
%    \end{macrocode}
%
% \subsection{Resources}
%
%    \begin{macrocode}
\begingroup\expandafter\expandafter\expandafter\endgroup
\expandafter\ifx\csname RequirePackage\endcsname\relax
  \def\TMP@RequirePackage#1[#2]{%
    \begingroup\expandafter\expandafter\expandafter\endgroup
    \expandafter\ifx\csname ver@#1.sty\endcsname\relax
      \input #1.sty\relax
    \fi
  }%
\else
  \let\TMP@RequirePackage\RequirePackage
\fi
\TMP@RequirePackage{ltxcmds}[2010/12/12]%
\TMP@RequirePackage{infwarerr}[2010/04/08]%
%    \end{macrocode}
%
%    \begin{macrocode}
\def\ProDef@temp#1{%
  \expandafter\def\csname ProDef@param[#1]\endcsname % hash-ok
}
\expandafter\def\csname ProDef@param\endcsname{}
\ProDef@temp0{}
\ProDef@temp1{##1}
\ProDef@temp2{##1##2}
\ProDef@temp3{##1##2##3}
\ProDef@temp4{##1##2##3##4}
\ProDef@temp5{##1##2##3##4##5}
\ProDef@temp6{##1##2##3##4##5##6}
\ProDef@temp7{##1##2##3##4##5##7}
\ProDef@temp8{##1##2##3##4##5##7##8}
\ProDef@temp9{##1##2##3##4##5##7##8##9}
%    \end{macrocode}
%
%    \begin{macro}{\ProDef@IfDefinable}
%    \begin{macrocode}
\ltx@IfUndefined{@ifdefinable}{%
  \long\def\ProDef@IfDefinable#1{%
    \begingroup
      \escapechar=-1 %
    \ltx@ifundefined{\string#1}{%
      \endgroup
      \ltx@firstofone
    }{%
      \expandafter\endgroup
      \expandafter
      \edef\expandafter\ProDef@temp\expandafter{\string#1 }%
      \@PackageError{protecteddef}{%
        Command \ltx@backslashchar\ProDef@temp already defined%
      }\@ehc
      \ltx@gobbletwo
    }%
  }%
}{%
  \long\def\ProDef@IfDefinable#1{%
    \let\ProDef@next\ltx@gobbletwo
    \@ifdefinable{#1}{%
      \let\ProDef@next\ltx@firstofone
    }%
    \ProDef@next
  }%
}
%    \end{macrocode}
%    \end{macro}
%
%    \begin{macrocode}
\begingroup\expandafter\expandafter\expandafter\endgroup
\expandafter\ifx\csname protected\endcsname\relax
  \begingroup\expandafter\expandafter\expandafter\endgroup
  \expandafter\ifx\csname DeclareRobustCommand\endcsname\relax
    \catcode`\&=14 % comment
  \else
    \newcommand*{\ProtectedDef}{%
      \ltx@ifnextchar*{%
        \ProDef@ProtectedDef
      }{%
        \ProDef@ProtectedDef{}%
      }%
    }%
    \long\def\ProDef@ProtectedDef#1#2#3#{%
      \ProDef@IfDefinable{#2}{%
        \ltx@IfUndefined{ProDef@param#3}{%
          \DeclareRobustCommand*{#2}{}%
          \begingroup
            \escapechar=-1 %
            \def\ProDef@temp{#1}%
          \edef\x{\endgroup
            \ifx\ProDef@temp\ltx@empty
              \noexpand\long
            \fi
            \noexpand\def
            \expandafter\noexpand\csname\string#2 \endcsname
          }%
          \x#3%
        }{%
          \DeclareRobustCommand#1{#2}#3%
        }%
      }%
    }%
    \expandafter\expandafter\expandafter\ProDef@AtEnd
  \fi
\else
  \catcode`\&=9 % ignore
\fi%
\ProDef@IfDefinable\ProtectedDef{%
&  \protected
  \def\ProtectedDef%
}{%
  \ltx@ifnextchar*{%
    \let\ProDef@long\ltx@empty
    \expandafter\ProDef@ProtectedDef\ltx@gobble
  }{%
    \let\ProDef@long\long
    \ProDef@ProtectedDef
  }%
}
\long\def\ProDef@ProtectedDef#1#2#{%
  \ProDef@IfDefinable{#1}{%
    \ltx@IfUndefined{ProDef@param#2}{%
&     \protected
      \ProDef@long
      \def#1#2%
    }{%
&     \protected
      \ProDef@long
      \expandafter\expandafter\expandafter\def
      \expandafter\expandafter\expandafter#1%
      \csname ProDef@param#2\endcsname
    }%
  }%
}
%    \end{macrocode}
%
%    \begin{macrocode}
\ProDef@AtEnd%
%    \end{macrocode}
%    \begin{macrocode}
%</package>
%    \end{macrocode}
%
% \section{Test}
%
% \subsection{Catcode checks for loading}
%
%    \begin{macrocode}
%<*test1>
%    \end{macrocode}
%    \begin{macrocode}
\catcode`\{=1 %
\catcode`\}=2 %
\catcode`\#=6 %
\catcode`\@=11 %
\expandafter\ifx\csname count@\endcsname\relax
  \countdef\count@=255 %
\fi
\expandafter\ifx\csname @gobble\endcsname\relax
  \long\def\@gobble#1{}%
\fi
\expandafter\ifx\csname @firstofone\endcsname\relax
  \long\def\@firstofone#1{#1}%
\fi
\expandafter\ifx\csname loop\endcsname\relax
  \expandafter\@firstofone
\else
  \expandafter\@gobble
\fi
{%
  \def\loop#1\repeat{%
    \def\body{#1}%
    \iterate
  }%
  \def\iterate{%
    \body
      \let\next\iterate
    \else
      \let\next\relax
    \fi
    \next
  }%
  \let\repeat=\fi
}%
\def\RestoreCatcodes{}
\count@=0 %
\loop
  \edef\RestoreCatcodes{%
    \RestoreCatcodes
    \catcode\the\count@=\the\catcode\count@\relax
  }%
\ifnum\count@<255 %
  \advance\count@ 1 %
\repeat

\def\RangeCatcodeInvalid#1#2{%
  \count@=#1\relax
  \loop
    \catcode\count@=15 %
  \ifnum\count@<#2\relax
    \advance\count@ 1 %
  \repeat
}
\def\RangeCatcodeCheck#1#2#3{%
  \count@=#1\relax
  \loop
    \ifnum#3=\catcode\count@
    \else
      \errmessage{%
        Character \the\count@\space
        with wrong catcode \the\catcode\count@\space
        instead of \number#3%
      }%
    \fi
  \ifnum\count@<#2\relax
    \advance\count@ 1 %
  \repeat
}
\def\space{ }
\expandafter\ifx\csname LoadCommand\endcsname\relax
  \def\LoadCommand{\input protecteddef.sty\relax}%
\fi
\def\Test{%
  \RangeCatcodeInvalid{0}{47}%
  \RangeCatcodeInvalid{58}{64}%
  \RangeCatcodeInvalid{91}{96}%
  \RangeCatcodeInvalid{123}{255}%
  \catcode`\@=12 %
  \catcode`\\=0 %
  \catcode`\%=14 %
  \LoadCommand
  \RangeCatcodeCheck{0}{36}{15}%
  \RangeCatcodeCheck{37}{37}{14}%
  \RangeCatcodeCheck{38}{47}{15}%
  \RangeCatcodeCheck{48}{57}{12}%
  \RangeCatcodeCheck{58}{63}{15}%
  \RangeCatcodeCheck{64}{64}{12}%
  \RangeCatcodeCheck{65}{90}{11}%
  \RangeCatcodeCheck{91}{91}{15}%
  \RangeCatcodeCheck{92}{92}{0}%
  \RangeCatcodeCheck{93}{96}{15}%
  \RangeCatcodeCheck{97}{122}{11}%
  \RangeCatcodeCheck{123}{255}{15}%
  \RestoreCatcodes
}
\Test
\csname @@end\endcsname
\end
%    \end{macrocode}
%    \begin{macrocode}
%</test1>
%    \end{macrocode}
%
% \subsection{Test without \hologo{LaTeX} and \cs{protected}}
%
%    \begin{macrocode}
%<*test2>
\errorcontextlines=10000 %
\begingroup\expandafter\expandafter\expandafter\endgroup
\expandafter\ifx\csname RequirePackage\endcsname\relax
  \input protecteddef.sty\relax
  \catcode`\{=1 %
  \catcode`\}=2 %
  \catcode`\#=6 %
\else
  \RequirePackage{protecteddef}[2016/05/16]%
\fi
\begingroup\expandafter\expandafter\expandafter\endgroup
\expandafter\ifx\csname protected\endcsname\relax
  \let\pdef\def
\else
  \def\pdef{\protected\def}%
\fi
\def\msg#{\immediate\write16}
\countdef\errcount=2 %
\long\def\BeginCheck#1\ProtectedDef#2\EndCheck{%
  \begingroup
    \toks0={\ProtectedDef#2}%
    \msg{<<\the\toks0>>}%
  \endgroup
  \setbox0=\hbox{%
    #1%
    \ProtectedDef#2%
    \check\foo
  }%
  \ifdim\wd0=0pt\relax
  \else
    \errmessage{[Definition] Unwanted spaces?!}%
  \fi
  \setbox0=\hbox{%
    \def\fooinitial{XYZ}%
    \let\foo\fooinitial
    \errcount=0 %
    \expandafter\def\csname @PackageError\endcsname##1##2##3{%
      \advance\errcount by 1 %
    }%
    \expandafter\def\csname @notdefinable\endcsname{%
      \advance\errcount by 1 %
    }%
    \ProtectedDef#2%
    \ifnum\errcount=1 %
    \else
      \errmessage{1 error expected, but found: \the\errcount}%
    \fi
    \ifx\foo\fooinitial
    \else
      \def\space{ }%
      \errmessage{\string\foo\space is overwritten}%
    \fi
  }%
  \ifdim\wd0=0pt\relax
  \else
    \errmessage{[Error] Unwanted spaces?!}%
  \fi
}
\chardef\DeclareVersion=0 %
\begingroup\expandafter\expandafter\expandafter\endgroup
\expandafter\ifx\csname protected\endcsname\relax
  \begingroup\expandafter\expandafter\expandafter\endgroup
  \expandafter\ifx\csname DeclareRobustCommand\endcsname\relax
  \else
    \chardef\DeclareVersion=1 %
  \fi
\fi
\ifnum\DeclareVersion=0 %
  \def\check#1{%
    \ifx\cmp#1%
      \msg{* Test passed.}%
    \else
      \msg{}%
      \msg{[\meaning#1]}%
      \msg{[\meaning\cmp]}%
      \errmessage{Test failed!}%
    \fi
  }%
\else
  \def\check#1{%
    \begingroup
      \escapechar=-1 %
    \edef\x{\endgroup
      \def\noexpand\cs/{\string#1}%
    }\x
    \edef\CMP{%
      \noexpand\protect
      \expandafter\noexpand\csname\cs/ \endcsname
    }%
    \ifx\CMP#1%
      \expandafter\ifx\csname\cs/ \endcsname\cmp
        \msg{Test passed.}%
      \else
        \msg{}%
        \msg{[\expandafter\meaning\csname\cs/ \endcsname]}%
        \msg{[\meaning\cmp]}%
        \errmessage{Test failed!}%
      \fi
    \else
      \msg{}%
      \msg{[\meaning#1]}%
      \msg{[\meaning\CMP]}%
      \errmessage{Test failed!}%
    \fi
  }%
\fi

\tracingmacros=1

\BeginCheck
  \pdef\cmp{}%
  \ProtectedDef*\foo{}%
\EndCheck

\BeginCheck
  \pdef\cmp{}%
  \ProtectedDef*\foo[0]{}%
\EndCheck

\BeginCheck
  \pdef\cmp#1{<#1>}%
  \ProtectedDef*\foo[1]{<#1>}%
\EndCheck

\BeginCheck
  \pdef\cmp(#1){<#1>}%
  \ProtectedDef*\foo(#1){<#1>}%
\EndCheck

\BeginCheck
  \long\pdef\cmp{}%
  \ProtectedDef\foo{}%
\EndCheck

\BeginCheck
  \long\pdef\cmp{}%
  \ProtectedDef\foo[0]{}%
\EndCheck

\BeginCheck
  \long\pdef\cmp#1{<#1>}%
  \ProtectedDef\foo[1]{<#1>}%
\EndCheck

\BeginCheck
  \long\pdef\cmp(#1){<#1>}%
  \ProtectedDef\foo(#1){<#1>}%
\EndCheck

\csname @@end\endcsname\end
%</test2>
%    \end{macrocode}
%
% \section{Installation}
%
% \subsection{Download}
%
% \paragraph{Package.} This package is available on
% CTAN\footnote{\CTANpkg{protecteddef}}:
% \begin{description}
% \item[\CTAN{macros/latex/contrib/oberdiek/protecteddef.dtx}] The source file.
% \item[\CTAN{macros/latex/contrib/oberdiek/protecteddef.pdf}] Documentation.
% \end{description}
%
%
% \paragraph{Bundle.} All the packages of the bundle `oberdiek'
% are also available in a TDS compliant ZIP archive. There
% the packages are already unpacked and the documentation files
% are generated. The files and directories obey the TDS standard.
% \begin{description}
% \item[\CTANinstall{install/macros/latex/contrib/oberdiek.tds.zip}]
% \end{description}
% \emph{TDS} refers to the standard ``A Directory Structure
% for \TeX\ Files'' (\CTAN{tds/tds.pdf}). Directories
% with \xfile{texmf} in their name are usually organized this way.
%
% \subsection{Bundle installation}
%
% \paragraph{Unpacking.} Unpack the \xfile{oberdiek.tds.zip} in the
% TDS tree (also known as \xfile{texmf} tree) of your choice.
% Example (linux):
% \begin{quote}
%   |unzip oberdiek.tds.zip -d ~/texmf|
% \end{quote}
%
% \paragraph{Script installation.}
% Check the directory \xfile{TDS:scripts/oberdiek/} for
% scripts that need further installation steps.
%
% \subsection{Package installation}
%
% \paragraph{Unpacking.} The \xfile{.dtx} file is a self-extracting
% \docstrip\ archive. The files are extracted by running the
% \xfile{.dtx} through \plainTeX:
% \begin{quote}
%   \verb|tex protecteddef.dtx|
% \end{quote}
%
% \paragraph{TDS.} Now the different files must be moved into
% the different directories in your installation TDS tree
% (also known as \xfile{texmf} tree):
% \begin{quote}
% \def\t{^^A
% \begin{tabular}{@{}>{\ttfamily}l@{ $\rightarrow$ }>{\ttfamily}l@{}}
%   protecteddef.sty & tex/generic/oberdiek/protecteddef.sty\\
%   protecteddef.pdf & doc/latex/oberdiek/protecteddef.pdf\\
%   test/protecteddef-test1.tex & doc/latex/oberdiek/test/protecteddef-test1.tex\\
%   test/protecteddef-test2.tex & doc/latex/oberdiek/test/protecteddef-test2.tex\\
%   protecteddef.dtx & source/latex/oberdiek/protecteddef.dtx\\
% \end{tabular}^^A
% }^^A
% \sbox0{\t}^^A
% \ifdim\wd0>\linewidth
%   \begingroup
%     \advance\linewidth by\leftmargin
%     \advance\linewidth by\rightmargin
%   \edef\x{\endgroup
%     \def\noexpand\lw{\the\linewidth}^^A
%   }\x
%   \def\lwbox{^^A
%     \leavevmode
%     \hbox to \linewidth{^^A
%       \kern-\leftmargin\relax
%       \hss
%       \usebox0
%       \hss
%       \kern-\rightmargin\relax
%     }^^A
%   }^^A
%   \ifdim\wd0>\lw
%     \sbox0{\small\t}^^A
%     \ifdim\wd0>\linewidth
%       \ifdim\wd0>\lw
%         \sbox0{\footnotesize\t}^^A
%         \ifdim\wd0>\linewidth
%           \ifdim\wd0>\lw
%             \sbox0{\scriptsize\t}^^A
%             \ifdim\wd0>\linewidth
%               \ifdim\wd0>\lw
%                 \sbox0{\tiny\t}^^A
%                 \ifdim\wd0>\linewidth
%                   \lwbox
%                 \else
%                   \usebox0
%                 \fi
%               \else
%                 \lwbox
%               \fi
%             \else
%               \usebox0
%             \fi
%           \else
%             \lwbox
%           \fi
%         \else
%           \usebox0
%         \fi
%       \else
%         \lwbox
%       \fi
%     \else
%       \usebox0
%     \fi
%   \else
%     \lwbox
%   \fi
% \else
%   \usebox0
% \fi
% \end{quote}
% If you have a \xfile{docstrip.cfg} that configures and enables \docstrip's
% TDS installing feature, then some files can already be in the right
% place, see the documentation of \docstrip.
%
% \subsection{Refresh file name databases}
%
% If your \TeX~distribution
% (\TeX\,Live, \mikTeX, \dots) relies on file name databases, you must refresh
% these. For example, \TeX\,Live\ users run \verb|texhash| or
% \verb|mktexlsr|.
%
% \subsection{Some details for the interested}
%
% \paragraph{Unpacking with \LaTeX.}
% The \xfile{.dtx} chooses its action depending on the format:
% \begin{description}
% \item[\plainTeX:] Run \docstrip\ and extract the files.
% \item[\LaTeX:] Generate the documentation.
% \end{description}
% If you insist on using \LaTeX\ for \docstrip\ (really,
% \docstrip\ does not need \LaTeX), then inform the autodetect routine
% about your intention:
% \begin{quote}
%   \verb|latex \let\install=y% \iffalse meta-comment
%
% File: protecteddef.dtx
% Version: 2016/05/16 v1.1
% Info: Define protected commands
%
% Copyright (C)
%    2011 Heiko Oberdiek
%    2016-2019 Oberdiek Package Support Group
%    https://github.com/ho-tex/oberdiek/issues
%
% This work may be distributed and/or modified under the
% conditions of the LaTeX Project Public License, either
% version 1.3c of this license or (at your option) any later
% version. This version of this license is in
%    https://www.latex-project.org/lppl/lppl-1-3c.txt
% and the latest version of this license is in
%    https://www.latex-project.org/lppl.txt
% and version 1.3 or later is part of all distributions of
% LaTeX version 2005/12/01 or later.
%
% This work has the LPPL maintenance status "maintained".
%
% The Current Maintainers of this work are
% Heiko Oberdiek and the Oberdiek Package Support Group
% https://github.com/ho-tex/oberdiek/issues
%
% The Base Interpreter refers to any `TeX-Format',
% because some files are installed in TDS:tex/generic//.
%
% This work consists of the main source file protecteddef.dtx
% and the derived files
%    protecteddef.sty, protecteddef.pdf, protecteddef.ins, protecteddef.drv,
%    protecteddef-test1.tex, protecteddef-test2.tex.
%
% Distribution:
%    CTAN:macros/latex/contrib/oberdiek/protecteddef.dtx
%    CTAN:macros/latex/contrib/oberdiek/protecteddef.pdf
%
% Unpacking:
%    (a) If protecteddef.ins is present:
%           tex protecteddef.ins
%    (b) Without protecteddef.ins:
%           tex protecteddef.dtx
%    (c) If you insist on using LaTeX
%           latex \let\install=y% \iffalse meta-comment
%
% File: protecteddef.dtx
% Version: 2016/05/16 v1.1
% Info: Define protected commands
%
% Copyright (C)
%    2011 Heiko Oberdiek
%    2016-2019 Oberdiek Package Support Group
%    https://github.com/ho-tex/oberdiek/issues
%
% This work may be distributed and/or modified under the
% conditions of the LaTeX Project Public License, either
% version 1.3c of this license or (at your option) any later
% version. This version of this license is in
%    https://www.latex-project.org/lppl/lppl-1-3c.txt
% and the latest version of this license is in
%    https://www.latex-project.org/lppl.txt
% and version 1.3 or later is part of all distributions of
% LaTeX version 2005/12/01 or later.
%
% This work has the LPPL maintenance status "maintained".
%
% The Current Maintainers of this work are
% Heiko Oberdiek and the Oberdiek Package Support Group
% https://github.com/ho-tex/oberdiek/issues
%
% The Base Interpreter refers to any `TeX-Format',
% because some files are installed in TDS:tex/generic//.
%
% This work consists of the main source file protecteddef.dtx
% and the derived files
%    protecteddef.sty, protecteddef.pdf, protecteddef.ins, protecteddef.drv,
%    protecteddef-test1.tex, protecteddef-test2.tex.
%
% Distribution:
%    CTAN:macros/latex/contrib/oberdiek/protecteddef.dtx
%    CTAN:macros/latex/contrib/oberdiek/protecteddef.pdf
%
% Unpacking:
%    (a) If protecteddef.ins is present:
%           tex protecteddef.ins
%    (b) Without protecteddef.ins:
%           tex protecteddef.dtx
%    (c) If you insist on using LaTeX
%           latex \let\install=y\input{protecteddef.dtx}
%        (quote the arguments according to the demands of your shell)
%
% Documentation:
%    (a) If protecteddef.drv is present:
%           latex protecteddef.drv
%    (b) Without protecteddef.drv:
%           latex protecteddef.dtx; ...
%    The class ltxdoc loads the configuration file ltxdoc.cfg
%    if available. Here you can specify further options, e.g.
%    use A4 as paper format:
%       \PassOptionsToClass{a4paper}{article}
%
%    Programm calls to get the documentation (example):
%       pdflatex protecteddef.dtx
%       makeindex -s gind.ist protecteddef.idx
%       pdflatex protecteddef.dtx
%       makeindex -s gind.ist protecteddef.idx
%       pdflatex protecteddef.dtx
%
% Installation:
%    TDS:tex/generic/oberdiek/protecteddef.sty
%    TDS:doc/latex/oberdiek/protecteddef.pdf
%    TDS:doc/latex/oberdiek/test/protecteddef-test1.tex
%    TDS:doc/latex/oberdiek/test/protecteddef-test2.tex
%    TDS:source/latex/oberdiek/protecteddef.dtx
%
%<*ignore>
\begingroup
  \catcode123=1 %
  \catcode125=2 %
  \def\x{LaTeX2e}%
\expandafter\endgroup
\ifcase 0\ifx\install y1\fi\expandafter
         \ifx\csname processbatchFile\endcsname\relax\else1\fi
         \ifx\fmtname\x\else 1\fi\relax
\else\csname fi\endcsname
%</ignore>
%<*install>
\input docstrip.tex
\Msg{************************************************************************}
\Msg{* Installation}
\Msg{* Package: protecteddef 2016/05/16 v1.1 Define protected commands (HO)}
\Msg{************************************************************************}

\keepsilent
\askforoverwritefalse

\let\MetaPrefix\relax
\preamble

This is a generated file.

Project: protecteddef
Version: 2016/05/16 v1.1

Copyright (C)
   2011 Heiko Oberdiek
   2016-2019 Oberdiek Package Support Group

This work may be distributed and/or modified under the
conditions of the LaTeX Project Public License, either
version 1.3c of this license or (at your option) any later
version. This version of this license is in
   https://www.latex-project.org/lppl/lppl-1-3c.txt
and the latest version of this license is in
   https://www.latex-project.org/lppl.txt
and version 1.3 or later is part of all distributions of
LaTeX version 2005/12/01 or later.

This work has the LPPL maintenance status "maintained".

The Current Maintainers of this work are
Heiko Oberdiek and the Oberdiek Package Support Group
https://github.com/ho-tex/oberdiek/issues


The Base Interpreter refers to any `TeX-Format',
because some files are installed in TDS:tex/generic//.

This work consists of the main source file protecteddef.dtx
and the derived files
   protecteddef.sty, protecteddef.pdf, protecteddef.ins, protecteddef.drv,
   protecteddef-test1.tex, protecteddef-test2.tex.

\endpreamble
\let\MetaPrefix\DoubleperCent

\generate{%
  \file{protecteddef.ins}{\from{protecteddef.dtx}{install}}%
  \file{protecteddef.drv}{\from{protecteddef.dtx}{driver}}%
  \usedir{tex/generic/oberdiek}%
  \file{protecteddef.sty}{\from{protecteddef.dtx}{package}}%
%  \usedir{doc/latex/oberdiek/test}%
%  \file{protecteddef-test1.tex}{\from{protecteddef.dtx}{test1}}%
%  \file{protecteddef-test2.tex}{\from{protecteddef.dtx}{test2}}%
  \nopreamble
  \nopostamble
%  \usedir{source/latex/oberdiek/catalogue}%
%  \file{protecteddef.xml}{\from{protecteddef.dtx}{catalogue}}%
}

\catcode32=13\relax% active space
\let =\space%
\Msg{************************************************************************}
\Msg{*}
\Msg{* To finish the installation you have to move the following}
\Msg{* file into a directory searched by TeX:}
\Msg{*}
\Msg{*     protecteddef.sty}
\Msg{*}
\Msg{* To produce the documentation run the file `protecteddef.drv'}
\Msg{* through LaTeX.}
\Msg{*}
\Msg{* Happy TeXing!}
\Msg{*}
\Msg{************************************************************************}

\endbatchfile
%</install>
%<*ignore>
\fi
%</ignore>
%<*driver>
\NeedsTeXFormat{LaTeX2e}
\ProvidesFile{protecteddef.drv}%
  [2016/05/16 v1.1 Define protected commands (HO)]%
\documentclass{ltxdoc}
\usepackage{holtxdoc}[2011/11/22]
\begin{document}
  \DocInput{protecteddef.dtx}%
\end{document}
%</driver>
% \fi
%
%
% \CharacterTable
%  {Upper-case    \A\B\C\D\E\F\G\H\I\J\K\L\M\N\O\P\Q\R\S\T\U\V\W\X\Y\Z
%   Lower-case    \a\b\c\d\e\f\g\h\i\j\k\l\m\n\o\p\q\r\s\t\u\v\w\x\y\z
%   Digits        \0\1\2\3\4\5\6\7\8\9
%   Exclamation   \!     Double quote  \"     Hash (number) \#
%   Dollar        \$     Percent       \%     Ampersand     \&
%   Acute accent  \'     Left paren    \(     Right paren   \)
%   Asterisk      \*     Plus          \+     Comma         \,
%   Minus         \-     Point         \.     Solidus       \/
%   Colon         \:     Semicolon     \;     Less than     \<
%   Equals        \=     Greater than  \>     Question mark \?
%   Commercial at \@     Left bracket  \[     Backslash     \\
%   Right bracket \]     Circumflex    \^     Underscore    \_
%   Grave accent  \`     Left brace    \{     Vertical bar  \|
%   Right brace   \}     Tilde         \~}
%
% \GetFileInfo{protecteddef.drv}
%
% \title{The \xpackage{protecteddef} package}
% \date{2016/05/16 v1.1}
% \author{Heiko Oberdiek\thanks
% {Please report any issues at \url{https://github.com/ho-tex/oberdiek/issues}}}
%
% \maketitle
%
% \begin{abstract}
% This packages provides \cs{ProtectedDef} for defining
% robust macros for both \hologo{plainTeX} and \hologo{LaTeX}.
% First \hologo{eTeX}'s \cs{protected} is tried, then
% \hologo{LaTeX}'s \cs{DeclareRobustCommand} is used.
% Otherwise the macro is not made robust.
% \end{abstract}
%
% \tableofcontents
%
% \section{Documentation}
%
% Many of my packages work for both formats \hologo{plainTeX}
% and \hologo{LaTeX}, even \hologo{iniTeX} is often supported.
% It would be nice if fragile macros could be protected and
% made robust. However the different format worlds offer different
% solutions.
%
% \subsection{The \hologo{LaTeX}'s way}
%
% Usually \cs{newcommand} is used to define macros. It provides
% a check if the command to be defined is already defined or
% cannot be defined for other reasons.
%
% For making robust macros \hologo{LaTeX} provides \cs{DeclareRobustCommand}.
% It shares the syntax with \cs{newcommand}. However it does not provide
% latters check. Internally the check is available via \cs{@ifdefinable}.
%
% Internally the robust macro is using \cs{protect} with a nested
% macro definition. The \cs{protect} infrastructure is a feature of
% \cs{LaTeX} and usually not available in other formats.
%
% \subsection{The \hologo{eTeX}'s way}
%
% The need for robust macros is addressed in \cs{eTeX}. It provides
% \cs{protected} that modifies the behaviour of \cs{def} in a similar
% way as \cs{long}. A protected macro does not expand in
% some expandable contexts like writing to a file or \cs{edef}.
%
% \subsection{The way of this package}
%
% The package tries to find the available protection mechanism.
% First it looks for \cs{eTeX}'s \cs{protected}, then
% it uses \hologo{LaTeX}'s \cs{DeclareRobustCommand}.
% If both fails, then the macro remains unprotected.
%
% Additionally, \cs{LaTeX}'s check, if a macro is already defined
% is added in all cases. First \hologo{LaTeX}'s \cs{@ifdefinable}
% is tried to be compatible with \hologo{LaTeX}.
% If \cs{@ifdefinable} is not available, then the test is implemented
% by asserting that the macro is undefined or has the meaning of \cs{relax}.
% If the test fails, then in all cases the macro is not defined
% and an error is thrown.
%
% \subsection{Usage}
%
% \begin{declcs}{ProtectedDef} \* \M{cmd} [\meta{num}] \M{definition text}
% \end{declcs}
% Macro \cs{ProtectedDef} follows the syntax of \hologo{LaTeX}'s
% \cs{newcommand} with the exception that an optional argument
% is not supported. Macro \meta{cmd} is to be defined as
% \cs{long} macro without star with \meta{num} arguments.
%
% The number of arguments \meta{num} must be given as explicite digit
% |0| upto |9|. Otherwise the part between the argument \meta{cmd} and
% the \meta{definition text} is taken as parameter text in the syntax
% of vanilla \hologo{TeX}. Examples (with \cs{protected}):
% \begin{quote}
%   \def\to{$\Rightarrow$ }%
%   |\ProtectedDef*{\foo}[1]{\message{#1}}|\\
%   \to |\protected\def\foo#1{\message#1}}|\\[1ex]
%   |\ProtectedDef\foo{abc}|\\
%   \to |\protected\def\foo{abc}|\\[1ex]
%   |\ProtectedDef*\foo(#1)<#2>{#1/#2}|\\
%   \to |\protected\def\foo(#1)<#2>{#1/#2}|
% \end{quote}
%
% \StopEventually{
% }
%
% \section{Implementation}
%
%    \begin{macrocode}
%<*package>
%    \end{macrocode}
%
% \subsection{Reload check and package identification}
%    Reload check, especially if the package is not used with \LaTeX.
%    \begin{macrocode}
\begingroup\catcode61\catcode48\catcode32=10\relax%
  \catcode13=5 % ^^M
  \endlinechar=13 %
  \catcode35=6 % #
  \catcode39=12 % '
  \catcode44=12 % ,
  \catcode45=12 % -
  \catcode46=12 % .
  \catcode58=12 % :
  \catcode64=11 % @
  \catcode123=1 % {
  \catcode125=2 % }
  \expandafter\let\expandafter\x\csname ver@protecteddef.sty\endcsname
  \ifx\x\relax % plain-TeX, first loading
  \else
    \def\empty{}%
    \ifx\x\empty % LaTeX, first loading,
      % variable is initialized, but \ProvidesPackage not yet seen
    \else
      \expandafter\ifx\csname PackageInfo\endcsname\relax
        \def\x#1#2{%
          \immediate\write-1{Package #1 Info: #2.}%
        }%
      \else
        \def\x#1#2{\PackageInfo{#1}{#2, stopped}}%
      \fi
      \x{protecteddef}{The package is already loaded}%
      \aftergroup\endinput
    \fi
  \fi
\endgroup%
%    \end{macrocode}
%    Package identification:
%    \begin{macrocode}
\begingroup\catcode61\catcode48\catcode32=10\relax%
  \catcode13=5 % ^^M
  \endlinechar=13 %
  \catcode35=6 % #
  \catcode39=12 % '
  \catcode40=12 % (
  \catcode41=12 % )
  \catcode44=12 % ,
  \catcode45=12 % -
  \catcode46=12 % .
  \catcode47=12 % /
  \catcode58=12 % :
  \catcode64=11 % @
  \catcode91=12 % [
  \catcode93=12 % ]
  \catcode123=1 % {
  \catcode125=2 % }
  \expandafter\ifx\csname ProvidesPackage\endcsname\relax
    \def\x#1#2#3[#4]{\endgroup
      \immediate\write-1{Package: #3 #4}%
      \xdef#1{#4}%
    }%
  \else
    \def\x#1#2[#3]{\endgroup
      #2[{#3}]%
      \ifx#1\@undefined
        \xdef#1{#3}%
      \fi
      \ifx#1\relax
        \xdef#1{#3}%
      \fi
    }%
  \fi
\expandafter\x\csname ver@protecteddef.sty\endcsname
\ProvidesPackage{protecteddef}%
  [2016/05/16 v1.1 Define protected commands (HO)]%
%    \end{macrocode}
%
% \subsection{Catcodes}
%
%    \begin{macrocode}
\begingroup\catcode61\catcode48\catcode32=10\relax%
  \catcode13=5 % ^^M
  \endlinechar=13 %
  \catcode123=1 % {
  \catcode125=2 % }
  \catcode64=11 % @
  \def\x{\endgroup
    \expandafter\edef\csname ProDef@AtEnd\endcsname{%
      \endlinechar=\the\endlinechar\relax
      \catcode13=\the\catcode13\relax
      \catcode32=\the\catcode32\relax
      \catcode35=\the\catcode35\relax
      \catcode61=\the\catcode61\relax
      \catcode64=\the\catcode64\relax
      \catcode123=\the\catcode123\relax
      \catcode125=\the\catcode125\relax
    }%
  }%
\x\catcode61\catcode48\catcode32=10\relax%
\catcode13=5 % ^^M
\endlinechar=13 %
\catcode35=6 % #
\catcode64=11 % @
\catcode123=1 % {
\catcode125=2 % }
\def\TMP@EnsureCode#1#2{%
  \edef\ProDef@AtEnd{%
    \ProDef@AtEnd
    \catcode#1=\the\catcode#1\relax
  }%
  \catcode#1=#2\relax
}
\TMP@EnsureCode{38}{4}% &
\TMP@EnsureCode{40}{12}% (
\TMP@EnsureCode{41}{12}% )
\TMP@EnsureCode{42}{12}% *
\TMP@EnsureCode{45}{12}% -
\TMP@EnsureCode{46}{12}% .
\TMP@EnsureCode{47}{12}% /
\TMP@EnsureCode{91}{12}% [
\TMP@EnsureCode{93}{12}% ]
\TMP@EnsureCode{96}{12}% `
\edef\ProDef@AtEnd{\ProDef@AtEnd\noexpand\endinput}
%    \end{macrocode}
%
% \subsection{Resources}
%
%    \begin{macrocode}
\begingroup\expandafter\expandafter\expandafter\endgroup
\expandafter\ifx\csname RequirePackage\endcsname\relax
  \def\TMP@RequirePackage#1[#2]{%
    \begingroup\expandafter\expandafter\expandafter\endgroup
    \expandafter\ifx\csname ver@#1.sty\endcsname\relax
      \input #1.sty\relax
    \fi
  }%
\else
  \let\TMP@RequirePackage\RequirePackage
\fi
\TMP@RequirePackage{ltxcmds}[2010/12/12]%
\TMP@RequirePackage{infwarerr}[2010/04/08]%
%    \end{macrocode}
%
%    \begin{macrocode}
\def\ProDef@temp#1{%
  \expandafter\def\csname ProDef@param[#1]\endcsname % hash-ok
}
\expandafter\def\csname ProDef@param\endcsname{}
\ProDef@temp0{}
\ProDef@temp1{##1}
\ProDef@temp2{##1##2}
\ProDef@temp3{##1##2##3}
\ProDef@temp4{##1##2##3##4}
\ProDef@temp5{##1##2##3##4##5}
\ProDef@temp6{##1##2##3##4##5##6}
\ProDef@temp7{##1##2##3##4##5##7}
\ProDef@temp8{##1##2##3##4##5##7##8}
\ProDef@temp9{##1##2##3##4##5##7##8##9}
%    \end{macrocode}
%
%    \begin{macro}{\ProDef@IfDefinable}
%    \begin{macrocode}
\ltx@IfUndefined{@ifdefinable}{%
  \long\def\ProDef@IfDefinable#1{%
    \begingroup
      \escapechar=-1 %
    \ltx@ifundefined{\string#1}{%
      \endgroup
      \ltx@firstofone
    }{%
      \expandafter\endgroup
      \expandafter
      \edef\expandafter\ProDef@temp\expandafter{\string#1 }%
      \@PackageError{protecteddef}{%
        Command \ltx@backslashchar\ProDef@temp already defined%
      }\@ehc
      \ltx@gobbletwo
    }%
  }%
}{%
  \long\def\ProDef@IfDefinable#1{%
    \let\ProDef@next\ltx@gobbletwo
    \@ifdefinable{#1}{%
      \let\ProDef@next\ltx@firstofone
    }%
    \ProDef@next
  }%
}
%    \end{macrocode}
%    \end{macro}
%
%    \begin{macrocode}
\begingroup\expandafter\expandafter\expandafter\endgroup
\expandafter\ifx\csname protected\endcsname\relax
  \begingroup\expandafter\expandafter\expandafter\endgroup
  \expandafter\ifx\csname DeclareRobustCommand\endcsname\relax
    \catcode`\&=14 % comment
  \else
    \newcommand*{\ProtectedDef}{%
      \ltx@ifnextchar*{%
        \ProDef@ProtectedDef
      }{%
        \ProDef@ProtectedDef{}%
      }%
    }%
    \long\def\ProDef@ProtectedDef#1#2#3#{%
      \ProDef@IfDefinable{#2}{%
        \ltx@IfUndefined{ProDef@param#3}{%
          \DeclareRobustCommand*{#2}{}%
          \begingroup
            \escapechar=-1 %
            \def\ProDef@temp{#1}%
          \edef\x{\endgroup
            \ifx\ProDef@temp\ltx@empty
              \noexpand\long
            \fi
            \noexpand\def
            \expandafter\noexpand\csname\string#2 \endcsname
          }%
          \x#3%
        }{%
          \DeclareRobustCommand#1{#2}#3%
        }%
      }%
    }%
    \expandafter\expandafter\expandafter\ProDef@AtEnd
  \fi
\else
  \catcode`\&=9 % ignore
\fi%
\ProDef@IfDefinable\ProtectedDef{%
&  \protected
  \def\ProtectedDef%
}{%
  \ltx@ifnextchar*{%
    \let\ProDef@long\ltx@empty
    \expandafter\ProDef@ProtectedDef\ltx@gobble
  }{%
    \let\ProDef@long\long
    \ProDef@ProtectedDef
  }%
}
\long\def\ProDef@ProtectedDef#1#2#{%
  \ProDef@IfDefinable{#1}{%
    \ltx@IfUndefined{ProDef@param#2}{%
&     \protected
      \ProDef@long
      \def#1#2%
    }{%
&     \protected
      \ProDef@long
      \expandafter\expandafter\expandafter\def
      \expandafter\expandafter\expandafter#1%
      \csname ProDef@param#2\endcsname
    }%
  }%
}
%    \end{macrocode}
%
%    \begin{macrocode}
\ProDef@AtEnd%
%    \end{macrocode}
%    \begin{macrocode}
%</package>
%    \end{macrocode}
%
% \section{Test}
%
% \subsection{Catcode checks for loading}
%
%    \begin{macrocode}
%<*test1>
%    \end{macrocode}
%    \begin{macrocode}
\catcode`\{=1 %
\catcode`\}=2 %
\catcode`\#=6 %
\catcode`\@=11 %
\expandafter\ifx\csname count@\endcsname\relax
  \countdef\count@=255 %
\fi
\expandafter\ifx\csname @gobble\endcsname\relax
  \long\def\@gobble#1{}%
\fi
\expandafter\ifx\csname @firstofone\endcsname\relax
  \long\def\@firstofone#1{#1}%
\fi
\expandafter\ifx\csname loop\endcsname\relax
  \expandafter\@firstofone
\else
  \expandafter\@gobble
\fi
{%
  \def\loop#1\repeat{%
    \def\body{#1}%
    \iterate
  }%
  \def\iterate{%
    \body
      \let\next\iterate
    \else
      \let\next\relax
    \fi
    \next
  }%
  \let\repeat=\fi
}%
\def\RestoreCatcodes{}
\count@=0 %
\loop
  \edef\RestoreCatcodes{%
    \RestoreCatcodes
    \catcode\the\count@=\the\catcode\count@\relax
  }%
\ifnum\count@<255 %
  \advance\count@ 1 %
\repeat

\def\RangeCatcodeInvalid#1#2{%
  \count@=#1\relax
  \loop
    \catcode\count@=15 %
  \ifnum\count@<#2\relax
    \advance\count@ 1 %
  \repeat
}
\def\RangeCatcodeCheck#1#2#3{%
  \count@=#1\relax
  \loop
    \ifnum#3=\catcode\count@
    \else
      \errmessage{%
        Character \the\count@\space
        with wrong catcode \the\catcode\count@\space
        instead of \number#3%
      }%
    \fi
  \ifnum\count@<#2\relax
    \advance\count@ 1 %
  \repeat
}
\def\space{ }
\expandafter\ifx\csname LoadCommand\endcsname\relax
  \def\LoadCommand{\input protecteddef.sty\relax}%
\fi
\def\Test{%
  \RangeCatcodeInvalid{0}{47}%
  \RangeCatcodeInvalid{58}{64}%
  \RangeCatcodeInvalid{91}{96}%
  \RangeCatcodeInvalid{123}{255}%
  \catcode`\@=12 %
  \catcode`\\=0 %
  \catcode`\%=14 %
  \LoadCommand
  \RangeCatcodeCheck{0}{36}{15}%
  \RangeCatcodeCheck{37}{37}{14}%
  \RangeCatcodeCheck{38}{47}{15}%
  \RangeCatcodeCheck{48}{57}{12}%
  \RangeCatcodeCheck{58}{63}{15}%
  \RangeCatcodeCheck{64}{64}{12}%
  \RangeCatcodeCheck{65}{90}{11}%
  \RangeCatcodeCheck{91}{91}{15}%
  \RangeCatcodeCheck{92}{92}{0}%
  \RangeCatcodeCheck{93}{96}{15}%
  \RangeCatcodeCheck{97}{122}{11}%
  \RangeCatcodeCheck{123}{255}{15}%
  \RestoreCatcodes
}
\Test
\csname @@end\endcsname
\end
%    \end{macrocode}
%    \begin{macrocode}
%</test1>
%    \end{macrocode}
%
% \subsection{Test without \hologo{LaTeX} and \cs{protected}}
%
%    \begin{macrocode}
%<*test2>
\errorcontextlines=10000 %
\begingroup\expandafter\expandafter\expandafter\endgroup
\expandafter\ifx\csname RequirePackage\endcsname\relax
  \input protecteddef.sty\relax
  \catcode`\{=1 %
  \catcode`\}=2 %
  \catcode`\#=6 %
\else
  \RequirePackage{protecteddef}[2016/05/16]%
\fi
\begingroup\expandafter\expandafter\expandafter\endgroup
\expandafter\ifx\csname protected\endcsname\relax
  \let\pdef\def
\else
  \def\pdef{\protected\def}%
\fi
\def\msg#{\immediate\write16}
\countdef\errcount=2 %
\long\def\BeginCheck#1\ProtectedDef#2\EndCheck{%
  \begingroup
    \toks0={\ProtectedDef#2}%
    \msg{<<\the\toks0>>}%
  \endgroup
  \setbox0=\hbox{%
    #1%
    \ProtectedDef#2%
    \check\foo
  }%
  \ifdim\wd0=0pt\relax
  \else
    \errmessage{[Definition] Unwanted spaces?!}%
  \fi
  \setbox0=\hbox{%
    \def\fooinitial{XYZ}%
    \let\foo\fooinitial
    \errcount=0 %
    \expandafter\def\csname @PackageError\endcsname##1##2##3{%
      \advance\errcount by 1 %
    }%
    \expandafter\def\csname @notdefinable\endcsname{%
      \advance\errcount by 1 %
    }%
    \ProtectedDef#2%
    \ifnum\errcount=1 %
    \else
      \errmessage{1 error expected, but found: \the\errcount}%
    \fi
    \ifx\foo\fooinitial
    \else
      \def\space{ }%
      \errmessage{\string\foo\space is overwritten}%
    \fi
  }%
  \ifdim\wd0=0pt\relax
  \else
    \errmessage{[Error] Unwanted spaces?!}%
  \fi
}
\chardef\DeclareVersion=0 %
\begingroup\expandafter\expandafter\expandafter\endgroup
\expandafter\ifx\csname protected\endcsname\relax
  \begingroup\expandafter\expandafter\expandafter\endgroup
  \expandafter\ifx\csname DeclareRobustCommand\endcsname\relax
  \else
    \chardef\DeclareVersion=1 %
  \fi
\fi
\ifnum\DeclareVersion=0 %
  \def\check#1{%
    \ifx\cmp#1%
      \msg{* Test passed.}%
    \else
      \msg{}%
      \msg{[\meaning#1]}%
      \msg{[\meaning\cmp]}%
      \errmessage{Test failed!}%
    \fi
  }%
\else
  \def\check#1{%
    \begingroup
      \escapechar=-1 %
    \edef\x{\endgroup
      \def\noexpand\cs/{\string#1}%
    }\x
    \edef\CMP{%
      \noexpand\protect
      \expandafter\noexpand\csname\cs/ \endcsname
    }%
    \ifx\CMP#1%
      \expandafter\ifx\csname\cs/ \endcsname\cmp
        \msg{Test passed.}%
      \else
        \msg{}%
        \msg{[\expandafter\meaning\csname\cs/ \endcsname]}%
        \msg{[\meaning\cmp]}%
        \errmessage{Test failed!}%
      \fi
    \else
      \msg{}%
      \msg{[\meaning#1]}%
      \msg{[\meaning\CMP]}%
      \errmessage{Test failed!}%
    \fi
  }%
\fi

\tracingmacros=1

\BeginCheck
  \pdef\cmp{}%
  \ProtectedDef*\foo{}%
\EndCheck

\BeginCheck
  \pdef\cmp{}%
  \ProtectedDef*\foo[0]{}%
\EndCheck

\BeginCheck
  \pdef\cmp#1{<#1>}%
  \ProtectedDef*\foo[1]{<#1>}%
\EndCheck

\BeginCheck
  \pdef\cmp(#1){<#1>}%
  \ProtectedDef*\foo(#1){<#1>}%
\EndCheck

\BeginCheck
  \long\pdef\cmp{}%
  \ProtectedDef\foo{}%
\EndCheck

\BeginCheck
  \long\pdef\cmp{}%
  \ProtectedDef\foo[0]{}%
\EndCheck

\BeginCheck
  \long\pdef\cmp#1{<#1>}%
  \ProtectedDef\foo[1]{<#1>}%
\EndCheck

\BeginCheck
  \long\pdef\cmp(#1){<#1>}%
  \ProtectedDef\foo(#1){<#1>}%
\EndCheck

\csname @@end\endcsname\end
%</test2>
%    \end{macrocode}
%
% \section{Installation}
%
% \subsection{Download}
%
% \paragraph{Package.} This package is available on
% CTAN\footnote{\CTANpkg{protecteddef}}:
% \begin{description}
% \item[\CTAN{macros/latex/contrib/oberdiek/protecteddef.dtx}] The source file.
% \item[\CTAN{macros/latex/contrib/oberdiek/protecteddef.pdf}] Documentation.
% \end{description}
%
%
% \paragraph{Bundle.} All the packages of the bundle `oberdiek'
% are also available in a TDS compliant ZIP archive. There
% the packages are already unpacked and the documentation files
% are generated. The files and directories obey the TDS standard.
% \begin{description}
% \item[\CTANinstall{install/macros/latex/contrib/oberdiek.tds.zip}]
% \end{description}
% \emph{TDS} refers to the standard ``A Directory Structure
% for \TeX\ Files'' (\CTAN{tds/tds.pdf}). Directories
% with \xfile{texmf} in their name are usually organized this way.
%
% \subsection{Bundle installation}
%
% \paragraph{Unpacking.} Unpack the \xfile{oberdiek.tds.zip} in the
% TDS tree (also known as \xfile{texmf} tree) of your choice.
% Example (linux):
% \begin{quote}
%   |unzip oberdiek.tds.zip -d ~/texmf|
% \end{quote}
%
% \paragraph{Script installation.}
% Check the directory \xfile{TDS:scripts/oberdiek/} for
% scripts that need further installation steps.
%
% \subsection{Package installation}
%
% \paragraph{Unpacking.} The \xfile{.dtx} file is a self-extracting
% \docstrip\ archive. The files are extracted by running the
% \xfile{.dtx} through \plainTeX:
% \begin{quote}
%   \verb|tex protecteddef.dtx|
% \end{quote}
%
% \paragraph{TDS.} Now the different files must be moved into
% the different directories in your installation TDS tree
% (also known as \xfile{texmf} tree):
% \begin{quote}
% \def\t{^^A
% \begin{tabular}{@{}>{\ttfamily}l@{ $\rightarrow$ }>{\ttfamily}l@{}}
%   protecteddef.sty & tex/generic/oberdiek/protecteddef.sty\\
%   protecteddef.pdf & doc/latex/oberdiek/protecteddef.pdf\\
%   test/protecteddef-test1.tex & doc/latex/oberdiek/test/protecteddef-test1.tex\\
%   test/protecteddef-test2.tex & doc/latex/oberdiek/test/protecteddef-test2.tex\\
%   protecteddef.dtx & source/latex/oberdiek/protecteddef.dtx\\
% \end{tabular}^^A
% }^^A
% \sbox0{\t}^^A
% \ifdim\wd0>\linewidth
%   \begingroup
%     \advance\linewidth by\leftmargin
%     \advance\linewidth by\rightmargin
%   \edef\x{\endgroup
%     \def\noexpand\lw{\the\linewidth}^^A
%   }\x
%   \def\lwbox{^^A
%     \leavevmode
%     \hbox to \linewidth{^^A
%       \kern-\leftmargin\relax
%       \hss
%       \usebox0
%       \hss
%       \kern-\rightmargin\relax
%     }^^A
%   }^^A
%   \ifdim\wd0>\lw
%     \sbox0{\small\t}^^A
%     \ifdim\wd0>\linewidth
%       \ifdim\wd0>\lw
%         \sbox0{\footnotesize\t}^^A
%         \ifdim\wd0>\linewidth
%           \ifdim\wd0>\lw
%             \sbox0{\scriptsize\t}^^A
%             \ifdim\wd0>\linewidth
%               \ifdim\wd0>\lw
%                 \sbox0{\tiny\t}^^A
%                 \ifdim\wd0>\linewidth
%                   \lwbox
%                 \else
%                   \usebox0
%                 \fi
%               \else
%                 \lwbox
%               \fi
%             \else
%               \usebox0
%             \fi
%           \else
%             \lwbox
%           \fi
%         \else
%           \usebox0
%         \fi
%       \else
%         \lwbox
%       \fi
%     \else
%       \usebox0
%     \fi
%   \else
%     \lwbox
%   \fi
% \else
%   \usebox0
% \fi
% \end{quote}
% If you have a \xfile{docstrip.cfg} that configures and enables \docstrip's
% TDS installing feature, then some files can already be in the right
% place, see the documentation of \docstrip.
%
% \subsection{Refresh file name databases}
%
% If your \TeX~distribution
% (\TeX\,Live, \mikTeX, \dots) relies on file name databases, you must refresh
% these. For example, \TeX\,Live\ users run \verb|texhash| or
% \verb|mktexlsr|.
%
% \subsection{Some details for the interested}
%
% \paragraph{Unpacking with \LaTeX.}
% The \xfile{.dtx} chooses its action depending on the format:
% \begin{description}
% \item[\plainTeX:] Run \docstrip\ and extract the files.
% \item[\LaTeX:] Generate the documentation.
% \end{description}
% If you insist on using \LaTeX\ for \docstrip\ (really,
% \docstrip\ does not need \LaTeX), then inform the autodetect routine
% about your intention:
% \begin{quote}
%   \verb|latex \let\install=y\input{protecteddef.dtx}|
% \end{quote}
% Do not forget to quote the argument according to the demands
% of your shell.
%
% \paragraph{Generating the documentation.}
% You can use both the \xfile{.dtx} or the \xfile{.drv} to generate
% the documentation. The process can be configured by the
% configuration file \xfile{ltxdoc.cfg}. For instance, put this
% line into this file, if you want to have A4 as paper format:
% \begin{quote}
%   \verb|\PassOptionsToClass{a4paper}{article}|
% \end{quote}
% An example follows how to generate the
% documentation with pdf\LaTeX:
% \begin{quote}
%\begin{verbatim}
%pdflatex protecteddef.dtx
%makeindex -s gind.ist protecteddef.idx
%pdflatex protecteddef.dtx
%makeindex -s gind.ist protecteddef.idx
%pdflatex protecteddef.dtx
%\end{verbatim}
% \end{quote}
%
% \begin{History}
%   \begin{Version}{2011/01/31 v1.0}
%   \item
%     First public version.
%   \end{Version}
%   \begin{Version}{2016/05/16 v1.1}
%   \item
%     Documentation updates.
%   \end{Version}
% \end{History}
%
% \PrintIndex
%
% \Finale
\endinput

%        (quote the arguments according to the demands of your shell)
%
% Documentation:
%    (a) If protecteddef.drv is present:
%           latex protecteddef.drv
%    (b) Without protecteddef.drv:
%           latex protecteddef.dtx; ...
%    The class ltxdoc loads the configuration file ltxdoc.cfg
%    if available. Here you can specify further options, e.g.
%    use A4 as paper format:
%       \PassOptionsToClass{a4paper}{article}
%
%    Programm calls to get the documentation (example):
%       pdflatex protecteddef.dtx
%       makeindex -s gind.ist protecteddef.idx
%       pdflatex protecteddef.dtx
%       makeindex -s gind.ist protecteddef.idx
%       pdflatex protecteddef.dtx
%
% Installation:
%    TDS:tex/generic/oberdiek/protecteddef.sty
%    TDS:doc/latex/oberdiek/protecteddef.pdf
%    TDS:doc/latex/oberdiek/test/protecteddef-test1.tex
%    TDS:doc/latex/oberdiek/test/protecteddef-test2.tex
%    TDS:source/latex/oberdiek/protecteddef.dtx
%
%<*ignore>
\begingroup
  \catcode123=1 %
  \catcode125=2 %
  \def\x{LaTeX2e}%
\expandafter\endgroup
\ifcase 0\ifx\install y1\fi\expandafter
         \ifx\csname processbatchFile\endcsname\relax\else1\fi
         \ifx\fmtname\x\else 1\fi\relax
\else\csname fi\endcsname
%</ignore>
%<*install>
\input docstrip.tex
\Msg{************************************************************************}
\Msg{* Installation}
\Msg{* Package: protecteddef 2016/05/16 v1.1 Define protected commands (HO)}
\Msg{************************************************************************}

\keepsilent
\askforoverwritefalse

\let\MetaPrefix\relax
\preamble

This is a generated file.

Project: protecteddef
Version: 2016/05/16 v1.1

Copyright (C)
   2011 Heiko Oberdiek
   2016-2019 Oberdiek Package Support Group

This work may be distributed and/or modified under the
conditions of the LaTeX Project Public License, either
version 1.3c of this license or (at your option) any later
version. This version of this license is in
   https://www.latex-project.org/lppl/lppl-1-3c.txt
and the latest version of this license is in
   https://www.latex-project.org/lppl.txt
and version 1.3 or later is part of all distributions of
LaTeX version 2005/12/01 or later.

This work has the LPPL maintenance status "maintained".

The Current Maintainers of this work are
Heiko Oberdiek and the Oberdiek Package Support Group
https://github.com/ho-tex/oberdiek/issues


The Base Interpreter refers to any `TeX-Format',
because some files are installed in TDS:tex/generic//.

This work consists of the main source file protecteddef.dtx
and the derived files
   protecteddef.sty, protecteddef.pdf, protecteddef.ins, protecteddef.drv,
   protecteddef-test1.tex, protecteddef-test2.tex.

\endpreamble
\let\MetaPrefix\DoubleperCent

\generate{%
  \file{protecteddef.ins}{\from{protecteddef.dtx}{install}}%
  \file{protecteddef.drv}{\from{protecteddef.dtx}{driver}}%
  \usedir{tex/generic/oberdiek}%
  \file{protecteddef.sty}{\from{protecteddef.dtx}{package}}%
%  \usedir{doc/latex/oberdiek/test}%
%  \file{protecteddef-test1.tex}{\from{protecteddef.dtx}{test1}}%
%  \file{protecteddef-test2.tex}{\from{protecteddef.dtx}{test2}}%
  \nopreamble
  \nopostamble
%  \usedir{source/latex/oberdiek/catalogue}%
%  \file{protecteddef.xml}{\from{protecteddef.dtx}{catalogue}}%
}

\catcode32=13\relax% active space
\let =\space%
\Msg{************************************************************************}
\Msg{*}
\Msg{* To finish the installation you have to move the following}
\Msg{* file into a directory searched by TeX:}
\Msg{*}
\Msg{*     protecteddef.sty}
\Msg{*}
\Msg{* To produce the documentation run the file `protecteddef.drv'}
\Msg{* through LaTeX.}
\Msg{*}
\Msg{* Happy TeXing!}
\Msg{*}
\Msg{************************************************************************}

\endbatchfile
%</install>
%<*ignore>
\fi
%</ignore>
%<*driver>
\NeedsTeXFormat{LaTeX2e}
\ProvidesFile{protecteddef.drv}%
  [2016/05/16 v1.1 Define protected commands (HO)]%
\documentclass{ltxdoc}
\usepackage{holtxdoc}[2011/11/22]
\begin{document}
  \DocInput{protecteddef.dtx}%
\end{document}
%</driver>
% \fi
%
%
% \CharacterTable
%  {Upper-case    \A\B\C\D\E\F\G\H\I\J\K\L\M\N\O\P\Q\R\S\T\U\V\W\X\Y\Z
%   Lower-case    \a\b\c\d\e\f\g\h\i\j\k\l\m\n\o\p\q\r\s\t\u\v\w\x\y\z
%   Digits        \0\1\2\3\4\5\6\7\8\9
%   Exclamation   \!     Double quote  \"     Hash (number) \#
%   Dollar        \$     Percent       \%     Ampersand     \&
%   Acute accent  \'     Left paren    \(     Right paren   \)
%   Asterisk      \*     Plus          \+     Comma         \,
%   Minus         \-     Point         \.     Solidus       \/
%   Colon         \:     Semicolon     \;     Less than     \<
%   Equals        \=     Greater than  \>     Question mark \?
%   Commercial at \@     Left bracket  \[     Backslash     \\
%   Right bracket \]     Circumflex    \^     Underscore    \_
%   Grave accent  \`     Left brace    \{     Vertical bar  \|
%   Right brace   \}     Tilde         \~}
%
% \GetFileInfo{protecteddef.drv}
%
% \title{The \xpackage{protecteddef} package}
% \date{2016/05/16 v1.1}
% \author{Heiko Oberdiek\thanks
% {Please report any issues at \url{https://github.com/ho-tex/oberdiek/issues}}}
%
% \maketitle
%
% \begin{abstract}
% This packages provides \cs{ProtectedDef} for defining
% robust macros for both \hologo{plainTeX} and \hologo{LaTeX}.
% First \hologo{eTeX}'s \cs{protected} is tried, then
% \hologo{LaTeX}'s \cs{DeclareRobustCommand} is used.
% Otherwise the macro is not made robust.
% \end{abstract}
%
% \tableofcontents
%
% \section{Documentation}
%
% Many of my packages work for both formats \hologo{plainTeX}
% and \hologo{LaTeX}, even \hologo{iniTeX} is often supported.
% It would be nice if fragile macros could be protected and
% made robust. However the different format worlds offer different
% solutions.
%
% \subsection{The \hologo{LaTeX}'s way}
%
% Usually \cs{newcommand} is used to define macros. It provides
% a check if the command to be defined is already defined or
% cannot be defined for other reasons.
%
% For making robust macros \hologo{LaTeX} provides \cs{DeclareRobustCommand}.
% It shares the syntax with \cs{newcommand}. However it does not provide
% latters check. Internally the check is available via \cs{@ifdefinable}.
%
% Internally the robust macro is using \cs{protect} with a nested
% macro definition. The \cs{protect} infrastructure is a feature of
% \cs{LaTeX} and usually not available in other formats.
%
% \subsection{The \hologo{eTeX}'s way}
%
% The need for robust macros is addressed in \cs{eTeX}. It provides
% \cs{protected} that modifies the behaviour of \cs{def} in a similar
% way as \cs{long}. A protected macro does not expand in
% some expandable contexts like writing to a file or \cs{edef}.
%
% \subsection{The way of this package}
%
% The package tries to find the available protection mechanism.
% First it looks for \cs{eTeX}'s \cs{protected}, then
% it uses \hologo{LaTeX}'s \cs{DeclareRobustCommand}.
% If both fails, then the macro remains unprotected.
%
% Additionally, \cs{LaTeX}'s check, if a macro is already defined
% is added in all cases. First \hologo{LaTeX}'s \cs{@ifdefinable}
% is tried to be compatible with \hologo{LaTeX}.
% If \cs{@ifdefinable} is not available, then the test is implemented
% by asserting that the macro is undefined or has the meaning of \cs{relax}.
% If the test fails, then in all cases the macro is not defined
% and an error is thrown.
%
% \subsection{Usage}
%
% \begin{declcs}{ProtectedDef} \* \M{cmd} [\meta{num}] \M{definition text}
% \end{declcs}
% Macro \cs{ProtectedDef} follows the syntax of \hologo{LaTeX}'s
% \cs{newcommand} with the exception that an optional argument
% is not supported. Macro \meta{cmd} is to be defined as
% \cs{long} macro without star with \meta{num} arguments.
%
% The number of arguments \meta{num} must be given as explicite digit
% |0| upto |9|. Otherwise the part between the argument \meta{cmd} and
% the \meta{definition text} is taken as parameter text in the syntax
% of vanilla \hologo{TeX}. Examples (with \cs{protected}):
% \begin{quote}
%   \def\to{$\Rightarrow$ }%
%   |\ProtectedDef*{\foo}[1]{\message{#1}}|\\
%   \to |\protected\def\foo#1{\message#1}}|\\[1ex]
%   |\ProtectedDef\foo{abc}|\\
%   \to |\protected\def\foo{abc}|\\[1ex]
%   |\ProtectedDef*\foo(#1)<#2>{#1/#2}|\\
%   \to |\protected\def\foo(#1)<#2>{#1/#2}|
% \end{quote}
%
% \StopEventually{
% }
%
% \section{Implementation}
%
%    \begin{macrocode}
%<*package>
%    \end{macrocode}
%
% \subsection{Reload check and package identification}
%    Reload check, especially if the package is not used with \LaTeX.
%    \begin{macrocode}
\begingroup\catcode61\catcode48\catcode32=10\relax%
  \catcode13=5 % ^^M
  \endlinechar=13 %
  \catcode35=6 % #
  \catcode39=12 % '
  \catcode44=12 % ,
  \catcode45=12 % -
  \catcode46=12 % .
  \catcode58=12 % :
  \catcode64=11 % @
  \catcode123=1 % {
  \catcode125=2 % }
  \expandafter\let\expandafter\x\csname ver@protecteddef.sty\endcsname
  \ifx\x\relax % plain-TeX, first loading
  \else
    \def\empty{}%
    \ifx\x\empty % LaTeX, first loading,
      % variable is initialized, but \ProvidesPackage not yet seen
    \else
      \expandafter\ifx\csname PackageInfo\endcsname\relax
        \def\x#1#2{%
          \immediate\write-1{Package #1 Info: #2.}%
        }%
      \else
        \def\x#1#2{\PackageInfo{#1}{#2, stopped}}%
      \fi
      \x{protecteddef}{The package is already loaded}%
      \aftergroup\endinput
    \fi
  \fi
\endgroup%
%    \end{macrocode}
%    Package identification:
%    \begin{macrocode}
\begingroup\catcode61\catcode48\catcode32=10\relax%
  \catcode13=5 % ^^M
  \endlinechar=13 %
  \catcode35=6 % #
  \catcode39=12 % '
  \catcode40=12 % (
  \catcode41=12 % )
  \catcode44=12 % ,
  \catcode45=12 % -
  \catcode46=12 % .
  \catcode47=12 % /
  \catcode58=12 % :
  \catcode64=11 % @
  \catcode91=12 % [
  \catcode93=12 % ]
  \catcode123=1 % {
  \catcode125=2 % }
  \expandafter\ifx\csname ProvidesPackage\endcsname\relax
    \def\x#1#2#3[#4]{\endgroup
      \immediate\write-1{Package: #3 #4}%
      \xdef#1{#4}%
    }%
  \else
    \def\x#1#2[#3]{\endgroup
      #2[{#3}]%
      \ifx#1\@undefined
        \xdef#1{#3}%
      \fi
      \ifx#1\relax
        \xdef#1{#3}%
      \fi
    }%
  \fi
\expandafter\x\csname ver@protecteddef.sty\endcsname
\ProvidesPackage{protecteddef}%
  [2016/05/16 v1.1 Define protected commands (HO)]%
%    \end{macrocode}
%
% \subsection{Catcodes}
%
%    \begin{macrocode}
\begingroup\catcode61\catcode48\catcode32=10\relax%
  \catcode13=5 % ^^M
  \endlinechar=13 %
  \catcode123=1 % {
  \catcode125=2 % }
  \catcode64=11 % @
  \def\x{\endgroup
    \expandafter\edef\csname ProDef@AtEnd\endcsname{%
      \endlinechar=\the\endlinechar\relax
      \catcode13=\the\catcode13\relax
      \catcode32=\the\catcode32\relax
      \catcode35=\the\catcode35\relax
      \catcode61=\the\catcode61\relax
      \catcode64=\the\catcode64\relax
      \catcode123=\the\catcode123\relax
      \catcode125=\the\catcode125\relax
    }%
  }%
\x\catcode61\catcode48\catcode32=10\relax%
\catcode13=5 % ^^M
\endlinechar=13 %
\catcode35=6 % #
\catcode64=11 % @
\catcode123=1 % {
\catcode125=2 % }
\def\TMP@EnsureCode#1#2{%
  \edef\ProDef@AtEnd{%
    \ProDef@AtEnd
    \catcode#1=\the\catcode#1\relax
  }%
  \catcode#1=#2\relax
}
\TMP@EnsureCode{38}{4}% &
\TMP@EnsureCode{40}{12}% (
\TMP@EnsureCode{41}{12}% )
\TMP@EnsureCode{42}{12}% *
\TMP@EnsureCode{45}{12}% -
\TMP@EnsureCode{46}{12}% .
\TMP@EnsureCode{47}{12}% /
\TMP@EnsureCode{91}{12}% [
\TMP@EnsureCode{93}{12}% ]
\TMP@EnsureCode{96}{12}% `
\edef\ProDef@AtEnd{\ProDef@AtEnd\noexpand\endinput}
%    \end{macrocode}
%
% \subsection{Resources}
%
%    \begin{macrocode}
\begingroup\expandafter\expandafter\expandafter\endgroup
\expandafter\ifx\csname RequirePackage\endcsname\relax
  \def\TMP@RequirePackage#1[#2]{%
    \begingroup\expandafter\expandafter\expandafter\endgroup
    \expandafter\ifx\csname ver@#1.sty\endcsname\relax
      \input #1.sty\relax
    \fi
  }%
\else
  \let\TMP@RequirePackage\RequirePackage
\fi
\TMP@RequirePackage{ltxcmds}[2010/12/12]%
\TMP@RequirePackage{infwarerr}[2010/04/08]%
%    \end{macrocode}
%
%    \begin{macrocode}
\def\ProDef@temp#1{%
  \expandafter\def\csname ProDef@param[#1]\endcsname % hash-ok
}
\expandafter\def\csname ProDef@param\endcsname{}
\ProDef@temp0{}
\ProDef@temp1{##1}
\ProDef@temp2{##1##2}
\ProDef@temp3{##1##2##3}
\ProDef@temp4{##1##2##3##4}
\ProDef@temp5{##1##2##3##4##5}
\ProDef@temp6{##1##2##3##4##5##6}
\ProDef@temp7{##1##2##3##4##5##7}
\ProDef@temp8{##1##2##3##4##5##7##8}
\ProDef@temp9{##1##2##3##4##5##7##8##9}
%    \end{macrocode}
%
%    \begin{macro}{\ProDef@IfDefinable}
%    \begin{macrocode}
\ltx@IfUndefined{@ifdefinable}{%
  \long\def\ProDef@IfDefinable#1{%
    \begingroup
      \escapechar=-1 %
    \ltx@ifundefined{\string#1}{%
      \endgroup
      \ltx@firstofone
    }{%
      \expandafter\endgroup
      \expandafter
      \edef\expandafter\ProDef@temp\expandafter{\string#1 }%
      \@PackageError{protecteddef}{%
        Command \ltx@backslashchar\ProDef@temp already defined%
      }\@ehc
      \ltx@gobbletwo
    }%
  }%
}{%
  \long\def\ProDef@IfDefinable#1{%
    \let\ProDef@next\ltx@gobbletwo
    \@ifdefinable{#1}{%
      \let\ProDef@next\ltx@firstofone
    }%
    \ProDef@next
  }%
}
%    \end{macrocode}
%    \end{macro}
%
%    \begin{macrocode}
\begingroup\expandafter\expandafter\expandafter\endgroup
\expandafter\ifx\csname protected\endcsname\relax
  \begingroup\expandafter\expandafter\expandafter\endgroup
  \expandafter\ifx\csname DeclareRobustCommand\endcsname\relax
    \catcode`\&=14 % comment
  \else
    \newcommand*{\ProtectedDef}{%
      \ltx@ifnextchar*{%
        \ProDef@ProtectedDef
      }{%
        \ProDef@ProtectedDef{}%
      }%
    }%
    \long\def\ProDef@ProtectedDef#1#2#3#{%
      \ProDef@IfDefinable{#2}{%
        \ltx@IfUndefined{ProDef@param#3}{%
          \DeclareRobustCommand*{#2}{}%
          \begingroup
            \escapechar=-1 %
            \def\ProDef@temp{#1}%
          \edef\x{\endgroup
            \ifx\ProDef@temp\ltx@empty
              \noexpand\long
            \fi
            \noexpand\def
            \expandafter\noexpand\csname\string#2 \endcsname
          }%
          \x#3%
        }{%
          \DeclareRobustCommand#1{#2}#3%
        }%
      }%
    }%
    \expandafter\expandafter\expandafter\ProDef@AtEnd
  \fi
\else
  \catcode`\&=9 % ignore
\fi%
\ProDef@IfDefinable\ProtectedDef{%
&  \protected
  \def\ProtectedDef%
}{%
  \ltx@ifnextchar*{%
    \let\ProDef@long\ltx@empty
    \expandafter\ProDef@ProtectedDef\ltx@gobble
  }{%
    \let\ProDef@long\long
    \ProDef@ProtectedDef
  }%
}
\long\def\ProDef@ProtectedDef#1#2#{%
  \ProDef@IfDefinable{#1}{%
    \ltx@IfUndefined{ProDef@param#2}{%
&     \protected
      \ProDef@long
      \def#1#2%
    }{%
&     \protected
      \ProDef@long
      \expandafter\expandafter\expandafter\def
      \expandafter\expandafter\expandafter#1%
      \csname ProDef@param#2\endcsname
    }%
  }%
}
%    \end{macrocode}
%
%    \begin{macrocode}
\ProDef@AtEnd%
%    \end{macrocode}
%    \begin{macrocode}
%</package>
%    \end{macrocode}
%
% \section{Test}
%
% \subsection{Catcode checks for loading}
%
%    \begin{macrocode}
%<*test1>
%    \end{macrocode}
%    \begin{macrocode}
\catcode`\{=1 %
\catcode`\}=2 %
\catcode`\#=6 %
\catcode`\@=11 %
\expandafter\ifx\csname count@\endcsname\relax
  \countdef\count@=255 %
\fi
\expandafter\ifx\csname @gobble\endcsname\relax
  \long\def\@gobble#1{}%
\fi
\expandafter\ifx\csname @firstofone\endcsname\relax
  \long\def\@firstofone#1{#1}%
\fi
\expandafter\ifx\csname loop\endcsname\relax
  \expandafter\@firstofone
\else
  \expandafter\@gobble
\fi
{%
  \def\loop#1\repeat{%
    \def\body{#1}%
    \iterate
  }%
  \def\iterate{%
    \body
      \let\next\iterate
    \else
      \let\next\relax
    \fi
    \next
  }%
  \let\repeat=\fi
}%
\def\RestoreCatcodes{}
\count@=0 %
\loop
  \edef\RestoreCatcodes{%
    \RestoreCatcodes
    \catcode\the\count@=\the\catcode\count@\relax
  }%
\ifnum\count@<255 %
  \advance\count@ 1 %
\repeat

\def\RangeCatcodeInvalid#1#2{%
  \count@=#1\relax
  \loop
    \catcode\count@=15 %
  \ifnum\count@<#2\relax
    \advance\count@ 1 %
  \repeat
}
\def\RangeCatcodeCheck#1#2#3{%
  \count@=#1\relax
  \loop
    \ifnum#3=\catcode\count@
    \else
      \errmessage{%
        Character \the\count@\space
        with wrong catcode \the\catcode\count@\space
        instead of \number#3%
      }%
    \fi
  \ifnum\count@<#2\relax
    \advance\count@ 1 %
  \repeat
}
\def\space{ }
\expandafter\ifx\csname LoadCommand\endcsname\relax
  \def\LoadCommand{\input protecteddef.sty\relax}%
\fi
\def\Test{%
  \RangeCatcodeInvalid{0}{47}%
  \RangeCatcodeInvalid{58}{64}%
  \RangeCatcodeInvalid{91}{96}%
  \RangeCatcodeInvalid{123}{255}%
  \catcode`\@=12 %
  \catcode`\\=0 %
  \catcode`\%=14 %
  \LoadCommand
  \RangeCatcodeCheck{0}{36}{15}%
  \RangeCatcodeCheck{37}{37}{14}%
  \RangeCatcodeCheck{38}{47}{15}%
  \RangeCatcodeCheck{48}{57}{12}%
  \RangeCatcodeCheck{58}{63}{15}%
  \RangeCatcodeCheck{64}{64}{12}%
  \RangeCatcodeCheck{65}{90}{11}%
  \RangeCatcodeCheck{91}{91}{15}%
  \RangeCatcodeCheck{92}{92}{0}%
  \RangeCatcodeCheck{93}{96}{15}%
  \RangeCatcodeCheck{97}{122}{11}%
  \RangeCatcodeCheck{123}{255}{15}%
  \RestoreCatcodes
}
\Test
\csname @@end\endcsname
\end
%    \end{macrocode}
%    \begin{macrocode}
%</test1>
%    \end{macrocode}
%
% \subsection{Test without \hologo{LaTeX} and \cs{protected}}
%
%    \begin{macrocode}
%<*test2>
\errorcontextlines=10000 %
\begingroup\expandafter\expandafter\expandafter\endgroup
\expandafter\ifx\csname RequirePackage\endcsname\relax
  \input protecteddef.sty\relax
  \catcode`\{=1 %
  \catcode`\}=2 %
  \catcode`\#=6 %
\else
  \RequirePackage{protecteddef}[2016/05/16]%
\fi
\begingroup\expandafter\expandafter\expandafter\endgroup
\expandafter\ifx\csname protected\endcsname\relax
  \let\pdef\def
\else
  \def\pdef{\protected\def}%
\fi
\def\msg#{\immediate\write16}
\countdef\errcount=2 %
\long\def\BeginCheck#1\ProtectedDef#2\EndCheck{%
  \begingroup
    \toks0={\ProtectedDef#2}%
    \msg{<<\the\toks0>>}%
  \endgroup
  \setbox0=\hbox{%
    #1%
    \ProtectedDef#2%
    \check\foo
  }%
  \ifdim\wd0=0pt\relax
  \else
    \errmessage{[Definition] Unwanted spaces?!}%
  \fi
  \setbox0=\hbox{%
    \def\fooinitial{XYZ}%
    \let\foo\fooinitial
    \errcount=0 %
    \expandafter\def\csname @PackageError\endcsname##1##2##3{%
      \advance\errcount by 1 %
    }%
    \expandafter\def\csname @notdefinable\endcsname{%
      \advance\errcount by 1 %
    }%
    \ProtectedDef#2%
    \ifnum\errcount=1 %
    \else
      \errmessage{1 error expected, but found: \the\errcount}%
    \fi
    \ifx\foo\fooinitial
    \else
      \def\space{ }%
      \errmessage{\string\foo\space is overwritten}%
    \fi
  }%
  \ifdim\wd0=0pt\relax
  \else
    \errmessage{[Error] Unwanted spaces?!}%
  \fi
}
\chardef\DeclareVersion=0 %
\begingroup\expandafter\expandafter\expandafter\endgroup
\expandafter\ifx\csname protected\endcsname\relax
  \begingroup\expandafter\expandafter\expandafter\endgroup
  \expandafter\ifx\csname DeclareRobustCommand\endcsname\relax
  \else
    \chardef\DeclareVersion=1 %
  \fi
\fi
\ifnum\DeclareVersion=0 %
  \def\check#1{%
    \ifx\cmp#1%
      \msg{* Test passed.}%
    \else
      \msg{}%
      \msg{[\meaning#1]}%
      \msg{[\meaning\cmp]}%
      \errmessage{Test failed!}%
    \fi
  }%
\else
  \def\check#1{%
    \begingroup
      \escapechar=-1 %
    \edef\x{\endgroup
      \def\noexpand\cs/{\string#1}%
    }\x
    \edef\CMP{%
      \noexpand\protect
      \expandafter\noexpand\csname\cs/ \endcsname
    }%
    \ifx\CMP#1%
      \expandafter\ifx\csname\cs/ \endcsname\cmp
        \msg{Test passed.}%
      \else
        \msg{}%
        \msg{[\expandafter\meaning\csname\cs/ \endcsname]}%
        \msg{[\meaning\cmp]}%
        \errmessage{Test failed!}%
      \fi
    \else
      \msg{}%
      \msg{[\meaning#1]}%
      \msg{[\meaning\CMP]}%
      \errmessage{Test failed!}%
    \fi
  }%
\fi

\tracingmacros=1

\BeginCheck
  \pdef\cmp{}%
  \ProtectedDef*\foo{}%
\EndCheck

\BeginCheck
  \pdef\cmp{}%
  \ProtectedDef*\foo[0]{}%
\EndCheck

\BeginCheck
  \pdef\cmp#1{<#1>}%
  \ProtectedDef*\foo[1]{<#1>}%
\EndCheck

\BeginCheck
  \pdef\cmp(#1){<#1>}%
  \ProtectedDef*\foo(#1){<#1>}%
\EndCheck

\BeginCheck
  \long\pdef\cmp{}%
  \ProtectedDef\foo{}%
\EndCheck

\BeginCheck
  \long\pdef\cmp{}%
  \ProtectedDef\foo[0]{}%
\EndCheck

\BeginCheck
  \long\pdef\cmp#1{<#1>}%
  \ProtectedDef\foo[1]{<#1>}%
\EndCheck

\BeginCheck
  \long\pdef\cmp(#1){<#1>}%
  \ProtectedDef\foo(#1){<#1>}%
\EndCheck

\csname @@end\endcsname\end
%</test2>
%    \end{macrocode}
%
% \section{Installation}
%
% \subsection{Download}
%
% \paragraph{Package.} This package is available on
% CTAN\footnote{\CTANpkg{protecteddef}}:
% \begin{description}
% \item[\CTAN{macros/latex/contrib/oberdiek/protecteddef.dtx}] The source file.
% \item[\CTAN{macros/latex/contrib/oberdiek/protecteddef.pdf}] Documentation.
% \end{description}
%
%
% \paragraph{Bundle.} All the packages of the bundle `oberdiek'
% are also available in a TDS compliant ZIP archive. There
% the packages are already unpacked and the documentation files
% are generated. The files and directories obey the TDS standard.
% \begin{description}
% \item[\CTANinstall{install/macros/latex/contrib/oberdiek.tds.zip}]
% \end{description}
% \emph{TDS} refers to the standard ``A Directory Structure
% for \TeX\ Files'' (\CTAN{tds/tds.pdf}). Directories
% with \xfile{texmf} in their name are usually organized this way.
%
% \subsection{Bundle installation}
%
% \paragraph{Unpacking.} Unpack the \xfile{oberdiek.tds.zip} in the
% TDS tree (also known as \xfile{texmf} tree) of your choice.
% Example (linux):
% \begin{quote}
%   |unzip oberdiek.tds.zip -d ~/texmf|
% \end{quote}
%
% \paragraph{Script installation.}
% Check the directory \xfile{TDS:scripts/oberdiek/} for
% scripts that need further installation steps.
%
% \subsection{Package installation}
%
% \paragraph{Unpacking.} The \xfile{.dtx} file is a self-extracting
% \docstrip\ archive. The files are extracted by running the
% \xfile{.dtx} through \plainTeX:
% \begin{quote}
%   \verb|tex protecteddef.dtx|
% \end{quote}
%
% \paragraph{TDS.} Now the different files must be moved into
% the different directories in your installation TDS tree
% (also known as \xfile{texmf} tree):
% \begin{quote}
% \def\t{^^A
% \begin{tabular}{@{}>{\ttfamily}l@{ $\rightarrow$ }>{\ttfamily}l@{}}
%   protecteddef.sty & tex/generic/oberdiek/protecteddef.sty\\
%   protecteddef.pdf & doc/latex/oberdiek/protecteddef.pdf\\
%   test/protecteddef-test1.tex & doc/latex/oberdiek/test/protecteddef-test1.tex\\
%   test/protecteddef-test2.tex & doc/latex/oberdiek/test/protecteddef-test2.tex\\
%   protecteddef.dtx & source/latex/oberdiek/protecteddef.dtx\\
% \end{tabular}^^A
% }^^A
% \sbox0{\t}^^A
% \ifdim\wd0>\linewidth
%   \begingroup
%     \advance\linewidth by\leftmargin
%     \advance\linewidth by\rightmargin
%   \edef\x{\endgroup
%     \def\noexpand\lw{\the\linewidth}^^A
%   }\x
%   \def\lwbox{^^A
%     \leavevmode
%     \hbox to \linewidth{^^A
%       \kern-\leftmargin\relax
%       \hss
%       \usebox0
%       \hss
%       \kern-\rightmargin\relax
%     }^^A
%   }^^A
%   \ifdim\wd0>\lw
%     \sbox0{\small\t}^^A
%     \ifdim\wd0>\linewidth
%       \ifdim\wd0>\lw
%         \sbox0{\footnotesize\t}^^A
%         \ifdim\wd0>\linewidth
%           \ifdim\wd0>\lw
%             \sbox0{\scriptsize\t}^^A
%             \ifdim\wd0>\linewidth
%               \ifdim\wd0>\lw
%                 \sbox0{\tiny\t}^^A
%                 \ifdim\wd0>\linewidth
%                   \lwbox
%                 \else
%                   \usebox0
%                 \fi
%               \else
%                 \lwbox
%               \fi
%             \else
%               \usebox0
%             \fi
%           \else
%             \lwbox
%           \fi
%         \else
%           \usebox0
%         \fi
%       \else
%         \lwbox
%       \fi
%     \else
%       \usebox0
%     \fi
%   \else
%     \lwbox
%   \fi
% \else
%   \usebox0
% \fi
% \end{quote}
% If you have a \xfile{docstrip.cfg} that configures and enables \docstrip's
% TDS installing feature, then some files can already be in the right
% place, see the documentation of \docstrip.
%
% \subsection{Refresh file name databases}
%
% If your \TeX~distribution
% (\TeX\,Live, \mikTeX, \dots) relies on file name databases, you must refresh
% these. For example, \TeX\,Live\ users run \verb|texhash| or
% \verb|mktexlsr|.
%
% \subsection{Some details for the interested}
%
% \paragraph{Unpacking with \LaTeX.}
% The \xfile{.dtx} chooses its action depending on the format:
% \begin{description}
% \item[\plainTeX:] Run \docstrip\ and extract the files.
% \item[\LaTeX:] Generate the documentation.
% \end{description}
% If you insist on using \LaTeX\ for \docstrip\ (really,
% \docstrip\ does not need \LaTeX), then inform the autodetect routine
% about your intention:
% \begin{quote}
%   \verb|latex \let\install=y% \iffalse meta-comment
%
% File: protecteddef.dtx
% Version: 2016/05/16 v1.1
% Info: Define protected commands
%
% Copyright (C)
%    2011 Heiko Oberdiek
%    2016-2019 Oberdiek Package Support Group
%    https://github.com/ho-tex/oberdiek/issues
%
% This work may be distributed and/or modified under the
% conditions of the LaTeX Project Public License, either
% version 1.3c of this license or (at your option) any later
% version. This version of this license is in
%    https://www.latex-project.org/lppl/lppl-1-3c.txt
% and the latest version of this license is in
%    https://www.latex-project.org/lppl.txt
% and version 1.3 or later is part of all distributions of
% LaTeX version 2005/12/01 or later.
%
% This work has the LPPL maintenance status "maintained".
%
% The Current Maintainers of this work are
% Heiko Oberdiek and the Oberdiek Package Support Group
% https://github.com/ho-tex/oberdiek/issues
%
% The Base Interpreter refers to any `TeX-Format',
% because some files are installed in TDS:tex/generic//.
%
% This work consists of the main source file protecteddef.dtx
% and the derived files
%    protecteddef.sty, protecteddef.pdf, protecteddef.ins, protecteddef.drv,
%    protecteddef-test1.tex, protecteddef-test2.tex.
%
% Distribution:
%    CTAN:macros/latex/contrib/oberdiek/protecteddef.dtx
%    CTAN:macros/latex/contrib/oberdiek/protecteddef.pdf
%
% Unpacking:
%    (a) If protecteddef.ins is present:
%           tex protecteddef.ins
%    (b) Without protecteddef.ins:
%           tex protecteddef.dtx
%    (c) If you insist on using LaTeX
%           latex \let\install=y\input{protecteddef.dtx}
%        (quote the arguments according to the demands of your shell)
%
% Documentation:
%    (a) If protecteddef.drv is present:
%           latex protecteddef.drv
%    (b) Without protecteddef.drv:
%           latex protecteddef.dtx; ...
%    The class ltxdoc loads the configuration file ltxdoc.cfg
%    if available. Here you can specify further options, e.g.
%    use A4 as paper format:
%       \PassOptionsToClass{a4paper}{article}
%
%    Programm calls to get the documentation (example):
%       pdflatex protecteddef.dtx
%       makeindex -s gind.ist protecteddef.idx
%       pdflatex protecteddef.dtx
%       makeindex -s gind.ist protecteddef.idx
%       pdflatex protecteddef.dtx
%
% Installation:
%    TDS:tex/generic/oberdiek/protecteddef.sty
%    TDS:doc/latex/oberdiek/protecteddef.pdf
%    TDS:doc/latex/oberdiek/test/protecteddef-test1.tex
%    TDS:doc/latex/oberdiek/test/protecteddef-test2.tex
%    TDS:source/latex/oberdiek/protecteddef.dtx
%
%<*ignore>
\begingroup
  \catcode123=1 %
  \catcode125=2 %
  \def\x{LaTeX2e}%
\expandafter\endgroup
\ifcase 0\ifx\install y1\fi\expandafter
         \ifx\csname processbatchFile\endcsname\relax\else1\fi
         \ifx\fmtname\x\else 1\fi\relax
\else\csname fi\endcsname
%</ignore>
%<*install>
\input docstrip.tex
\Msg{************************************************************************}
\Msg{* Installation}
\Msg{* Package: protecteddef 2016/05/16 v1.1 Define protected commands (HO)}
\Msg{************************************************************************}

\keepsilent
\askforoverwritefalse

\let\MetaPrefix\relax
\preamble

This is a generated file.

Project: protecteddef
Version: 2016/05/16 v1.1

Copyright (C)
   2011 Heiko Oberdiek
   2016-2019 Oberdiek Package Support Group

This work may be distributed and/or modified under the
conditions of the LaTeX Project Public License, either
version 1.3c of this license or (at your option) any later
version. This version of this license is in
   https://www.latex-project.org/lppl/lppl-1-3c.txt
and the latest version of this license is in
   https://www.latex-project.org/lppl.txt
and version 1.3 or later is part of all distributions of
LaTeX version 2005/12/01 or later.

This work has the LPPL maintenance status "maintained".

The Current Maintainers of this work are
Heiko Oberdiek and the Oberdiek Package Support Group
https://github.com/ho-tex/oberdiek/issues


The Base Interpreter refers to any `TeX-Format',
because some files are installed in TDS:tex/generic//.

This work consists of the main source file protecteddef.dtx
and the derived files
   protecteddef.sty, protecteddef.pdf, protecteddef.ins, protecteddef.drv,
   protecteddef-test1.tex, protecteddef-test2.tex.

\endpreamble
\let\MetaPrefix\DoubleperCent

\generate{%
  \file{protecteddef.ins}{\from{protecteddef.dtx}{install}}%
  \file{protecteddef.drv}{\from{protecteddef.dtx}{driver}}%
  \usedir{tex/generic/oberdiek}%
  \file{protecteddef.sty}{\from{protecteddef.dtx}{package}}%
%  \usedir{doc/latex/oberdiek/test}%
%  \file{protecteddef-test1.tex}{\from{protecteddef.dtx}{test1}}%
%  \file{protecteddef-test2.tex}{\from{protecteddef.dtx}{test2}}%
  \nopreamble
  \nopostamble
%  \usedir{source/latex/oberdiek/catalogue}%
%  \file{protecteddef.xml}{\from{protecteddef.dtx}{catalogue}}%
}

\catcode32=13\relax% active space
\let =\space%
\Msg{************************************************************************}
\Msg{*}
\Msg{* To finish the installation you have to move the following}
\Msg{* file into a directory searched by TeX:}
\Msg{*}
\Msg{*     protecteddef.sty}
\Msg{*}
\Msg{* To produce the documentation run the file `protecteddef.drv'}
\Msg{* through LaTeX.}
\Msg{*}
\Msg{* Happy TeXing!}
\Msg{*}
\Msg{************************************************************************}

\endbatchfile
%</install>
%<*ignore>
\fi
%</ignore>
%<*driver>
\NeedsTeXFormat{LaTeX2e}
\ProvidesFile{protecteddef.drv}%
  [2016/05/16 v1.1 Define protected commands (HO)]%
\documentclass{ltxdoc}
\usepackage{holtxdoc}[2011/11/22]
\begin{document}
  \DocInput{protecteddef.dtx}%
\end{document}
%</driver>
% \fi
%
%
% \CharacterTable
%  {Upper-case    \A\B\C\D\E\F\G\H\I\J\K\L\M\N\O\P\Q\R\S\T\U\V\W\X\Y\Z
%   Lower-case    \a\b\c\d\e\f\g\h\i\j\k\l\m\n\o\p\q\r\s\t\u\v\w\x\y\z
%   Digits        \0\1\2\3\4\5\6\7\8\9
%   Exclamation   \!     Double quote  \"     Hash (number) \#
%   Dollar        \$     Percent       \%     Ampersand     \&
%   Acute accent  \'     Left paren    \(     Right paren   \)
%   Asterisk      \*     Plus          \+     Comma         \,
%   Minus         \-     Point         \.     Solidus       \/
%   Colon         \:     Semicolon     \;     Less than     \<
%   Equals        \=     Greater than  \>     Question mark \?
%   Commercial at \@     Left bracket  \[     Backslash     \\
%   Right bracket \]     Circumflex    \^     Underscore    \_
%   Grave accent  \`     Left brace    \{     Vertical bar  \|
%   Right brace   \}     Tilde         \~}
%
% \GetFileInfo{protecteddef.drv}
%
% \title{The \xpackage{protecteddef} package}
% \date{2016/05/16 v1.1}
% \author{Heiko Oberdiek\thanks
% {Please report any issues at \url{https://github.com/ho-tex/oberdiek/issues}}}
%
% \maketitle
%
% \begin{abstract}
% This packages provides \cs{ProtectedDef} for defining
% robust macros for both \hologo{plainTeX} and \hologo{LaTeX}.
% First \hologo{eTeX}'s \cs{protected} is tried, then
% \hologo{LaTeX}'s \cs{DeclareRobustCommand} is used.
% Otherwise the macro is not made robust.
% \end{abstract}
%
% \tableofcontents
%
% \section{Documentation}
%
% Many of my packages work for both formats \hologo{plainTeX}
% and \hologo{LaTeX}, even \hologo{iniTeX} is often supported.
% It would be nice if fragile macros could be protected and
% made robust. However the different format worlds offer different
% solutions.
%
% \subsection{The \hologo{LaTeX}'s way}
%
% Usually \cs{newcommand} is used to define macros. It provides
% a check if the command to be defined is already defined or
% cannot be defined for other reasons.
%
% For making robust macros \hologo{LaTeX} provides \cs{DeclareRobustCommand}.
% It shares the syntax with \cs{newcommand}. However it does not provide
% latters check. Internally the check is available via \cs{@ifdefinable}.
%
% Internally the robust macro is using \cs{protect} with a nested
% macro definition. The \cs{protect} infrastructure is a feature of
% \cs{LaTeX} and usually not available in other formats.
%
% \subsection{The \hologo{eTeX}'s way}
%
% The need for robust macros is addressed in \cs{eTeX}. It provides
% \cs{protected} that modifies the behaviour of \cs{def} in a similar
% way as \cs{long}. A protected macro does not expand in
% some expandable contexts like writing to a file or \cs{edef}.
%
% \subsection{The way of this package}
%
% The package tries to find the available protection mechanism.
% First it looks for \cs{eTeX}'s \cs{protected}, then
% it uses \hologo{LaTeX}'s \cs{DeclareRobustCommand}.
% If both fails, then the macro remains unprotected.
%
% Additionally, \cs{LaTeX}'s check, if a macro is already defined
% is added in all cases. First \hologo{LaTeX}'s \cs{@ifdefinable}
% is tried to be compatible with \hologo{LaTeX}.
% If \cs{@ifdefinable} is not available, then the test is implemented
% by asserting that the macro is undefined or has the meaning of \cs{relax}.
% If the test fails, then in all cases the macro is not defined
% and an error is thrown.
%
% \subsection{Usage}
%
% \begin{declcs}{ProtectedDef} \* \M{cmd} [\meta{num}] \M{definition text}
% \end{declcs}
% Macro \cs{ProtectedDef} follows the syntax of \hologo{LaTeX}'s
% \cs{newcommand} with the exception that an optional argument
% is not supported. Macro \meta{cmd} is to be defined as
% \cs{long} macro without star with \meta{num} arguments.
%
% The number of arguments \meta{num} must be given as explicite digit
% |0| upto |9|. Otherwise the part between the argument \meta{cmd} and
% the \meta{definition text} is taken as parameter text in the syntax
% of vanilla \hologo{TeX}. Examples (with \cs{protected}):
% \begin{quote}
%   \def\to{$\Rightarrow$ }%
%   |\ProtectedDef*{\foo}[1]{\message{#1}}|\\
%   \to |\protected\def\foo#1{\message#1}}|\\[1ex]
%   |\ProtectedDef\foo{abc}|\\
%   \to |\protected\def\foo{abc}|\\[1ex]
%   |\ProtectedDef*\foo(#1)<#2>{#1/#2}|\\
%   \to |\protected\def\foo(#1)<#2>{#1/#2}|
% \end{quote}
%
% \StopEventually{
% }
%
% \section{Implementation}
%
%    \begin{macrocode}
%<*package>
%    \end{macrocode}
%
% \subsection{Reload check and package identification}
%    Reload check, especially if the package is not used with \LaTeX.
%    \begin{macrocode}
\begingroup\catcode61\catcode48\catcode32=10\relax%
  \catcode13=5 % ^^M
  \endlinechar=13 %
  \catcode35=6 % #
  \catcode39=12 % '
  \catcode44=12 % ,
  \catcode45=12 % -
  \catcode46=12 % .
  \catcode58=12 % :
  \catcode64=11 % @
  \catcode123=1 % {
  \catcode125=2 % }
  \expandafter\let\expandafter\x\csname ver@protecteddef.sty\endcsname
  \ifx\x\relax % plain-TeX, first loading
  \else
    \def\empty{}%
    \ifx\x\empty % LaTeX, first loading,
      % variable is initialized, but \ProvidesPackage not yet seen
    \else
      \expandafter\ifx\csname PackageInfo\endcsname\relax
        \def\x#1#2{%
          \immediate\write-1{Package #1 Info: #2.}%
        }%
      \else
        \def\x#1#2{\PackageInfo{#1}{#2, stopped}}%
      \fi
      \x{protecteddef}{The package is already loaded}%
      \aftergroup\endinput
    \fi
  \fi
\endgroup%
%    \end{macrocode}
%    Package identification:
%    \begin{macrocode}
\begingroup\catcode61\catcode48\catcode32=10\relax%
  \catcode13=5 % ^^M
  \endlinechar=13 %
  \catcode35=6 % #
  \catcode39=12 % '
  \catcode40=12 % (
  \catcode41=12 % )
  \catcode44=12 % ,
  \catcode45=12 % -
  \catcode46=12 % .
  \catcode47=12 % /
  \catcode58=12 % :
  \catcode64=11 % @
  \catcode91=12 % [
  \catcode93=12 % ]
  \catcode123=1 % {
  \catcode125=2 % }
  \expandafter\ifx\csname ProvidesPackage\endcsname\relax
    \def\x#1#2#3[#4]{\endgroup
      \immediate\write-1{Package: #3 #4}%
      \xdef#1{#4}%
    }%
  \else
    \def\x#1#2[#3]{\endgroup
      #2[{#3}]%
      \ifx#1\@undefined
        \xdef#1{#3}%
      \fi
      \ifx#1\relax
        \xdef#1{#3}%
      \fi
    }%
  \fi
\expandafter\x\csname ver@protecteddef.sty\endcsname
\ProvidesPackage{protecteddef}%
  [2016/05/16 v1.1 Define protected commands (HO)]%
%    \end{macrocode}
%
% \subsection{Catcodes}
%
%    \begin{macrocode}
\begingroup\catcode61\catcode48\catcode32=10\relax%
  \catcode13=5 % ^^M
  \endlinechar=13 %
  \catcode123=1 % {
  \catcode125=2 % }
  \catcode64=11 % @
  \def\x{\endgroup
    \expandafter\edef\csname ProDef@AtEnd\endcsname{%
      \endlinechar=\the\endlinechar\relax
      \catcode13=\the\catcode13\relax
      \catcode32=\the\catcode32\relax
      \catcode35=\the\catcode35\relax
      \catcode61=\the\catcode61\relax
      \catcode64=\the\catcode64\relax
      \catcode123=\the\catcode123\relax
      \catcode125=\the\catcode125\relax
    }%
  }%
\x\catcode61\catcode48\catcode32=10\relax%
\catcode13=5 % ^^M
\endlinechar=13 %
\catcode35=6 % #
\catcode64=11 % @
\catcode123=1 % {
\catcode125=2 % }
\def\TMP@EnsureCode#1#2{%
  \edef\ProDef@AtEnd{%
    \ProDef@AtEnd
    \catcode#1=\the\catcode#1\relax
  }%
  \catcode#1=#2\relax
}
\TMP@EnsureCode{38}{4}% &
\TMP@EnsureCode{40}{12}% (
\TMP@EnsureCode{41}{12}% )
\TMP@EnsureCode{42}{12}% *
\TMP@EnsureCode{45}{12}% -
\TMP@EnsureCode{46}{12}% .
\TMP@EnsureCode{47}{12}% /
\TMP@EnsureCode{91}{12}% [
\TMP@EnsureCode{93}{12}% ]
\TMP@EnsureCode{96}{12}% `
\edef\ProDef@AtEnd{\ProDef@AtEnd\noexpand\endinput}
%    \end{macrocode}
%
% \subsection{Resources}
%
%    \begin{macrocode}
\begingroup\expandafter\expandafter\expandafter\endgroup
\expandafter\ifx\csname RequirePackage\endcsname\relax
  \def\TMP@RequirePackage#1[#2]{%
    \begingroup\expandafter\expandafter\expandafter\endgroup
    \expandafter\ifx\csname ver@#1.sty\endcsname\relax
      \input #1.sty\relax
    \fi
  }%
\else
  \let\TMP@RequirePackage\RequirePackage
\fi
\TMP@RequirePackage{ltxcmds}[2010/12/12]%
\TMP@RequirePackage{infwarerr}[2010/04/08]%
%    \end{macrocode}
%
%    \begin{macrocode}
\def\ProDef@temp#1{%
  \expandafter\def\csname ProDef@param[#1]\endcsname % hash-ok
}
\expandafter\def\csname ProDef@param\endcsname{}
\ProDef@temp0{}
\ProDef@temp1{##1}
\ProDef@temp2{##1##2}
\ProDef@temp3{##1##2##3}
\ProDef@temp4{##1##2##3##4}
\ProDef@temp5{##1##2##3##4##5}
\ProDef@temp6{##1##2##3##4##5##6}
\ProDef@temp7{##1##2##3##4##5##7}
\ProDef@temp8{##1##2##3##4##5##7##8}
\ProDef@temp9{##1##2##3##4##5##7##8##9}
%    \end{macrocode}
%
%    \begin{macro}{\ProDef@IfDefinable}
%    \begin{macrocode}
\ltx@IfUndefined{@ifdefinable}{%
  \long\def\ProDef@IfDefinable#1{%
    \begingroup
      \escapechar=-1 %
    \ltx@ifundefined{\string#1}{%
      \endgroup
      \ltx@firstofone
    }{%
      \expandafter\endgroup
      \expandafter
      \edef\expandafter\ProDef@temp\expandafter{\string#1 }%
      \@PackageError{protecteddef}{%
        Command \ltx@backslashchar\ProDef@temp already defined%
      }\@ehc
      \ltx@gobbletwo
    }%
  }%
}{%
  \long\def\ProDef@IfDefinable#1{%
    \let\ProDef@next\ltx@gobbletwo
    \@ifdefinable{#1}{%
      \let\ProDef@next\ltx@firstofone
    }%
    \ProDef@next
  }%
}
%    \end{macrocode}
%    \end{macro}
%
%    \begin{macrocode}
\begingroup\expandafter\expandafter\expandafter\endgroup
\expandafter\ifx\csname protected\endcsname\relax
  \begingroup\expandafter\expandafter\expandafter\endgroup
  \expandafter\ifx\csname DeclareRobustCommand\endcsname\relax
    \catcode`\&=14 % comment
  \else
    \newcommand*{\ProtectedDef}{%
      \ltx@ifnextchar*{%
        \ProDef@ProtectedDef
      }{%
        \ProDef@ProtectedDef{}%
      }%
    }%
    \long\def\ProDef@ProtectedDef#1#2#3#{%
      \ProDef@IfDefinable{#2}{%
        \ltx@IfUndefined{ProDef@param#3}{%
          \DeclareRobustCommand*{#2}{}%
          \begingroup
            \escapechar=-1 %
            \def\ProDef@temp{#1}%
          \edef\x{\endgroup
            \ifx\ProDef@temp\ltx@empty
              \noexpand\long
            \fi
            \noexpand\def
            \expandafter\noexpand\csname\string#2 \endcsname
          }%
          \x#3%
        }{%
          \DeclareRobustCommand#1{#2}#3%
        }%
      }%
    }%
    \expandafter\expandafter\expandafter\ProDef@AtEnd
  \fi
\else
  \catcode`\&=9 % ignore
\fi%
\ProDef@IfDefinable\ProtectedDef{%
&  \protected
  \def\ProtectedDef%
}{%
  \ltx@ifnextchar*{%
    \let\ProDef@long\ltx@empty
    \expandafter\ProDef@ProtectedDef\ltx@gobble
  }{%
    \let\ProDef@long\long
    \ProDef@ProtectedDef
  }%
}
\long\def\ProDef@ProtectedDef#1#2#{%
  \ProDef@IfDefinable{#1}{%
    \ltx@IfUndefined{ProDef@param#2}{%
&     \protected
      \ProDef@long
      \def#1#2%
    }{%
&     \protected
      \ProDef@long
      \expandafter\expandafter\expandafter\def
      \expandafter\expandafter\expandafter#1%
      \csname ProDef@param#2\endcsname
    }%
  }%
}
%    \end{macrocode}
%
%    \begin{macrocode}
\ProDef@AtEnd%
%    \end{macrocode}
%    \begin{macrocode}
%</package>
%    \end{macrocode}
%
% \section{Test}
%
% \subsection{Catcode checks for loading}
%
%    \begin{macrocode}
%<*test1>
%    \end{macrocode}
%    \begin{macrocode}
\catcode`\{=1 %
\catcode`\}=2 %
\catcode`\#=6 %
\catcode`\@=11 %
\expandafter\ifx\csname count@\endcsname\relax
  \countdef\count@=255 %
\fi
\expandafter\ifx\csname @gobble\endcsname\relax
  \long\def\@gobble#1{}%
\fi
\expandafter\ifx\csname @firstofone\endcsname\relax
  \long\def\@firstofone#1{#1}%
\fi
\expandafter\ifx\csname loop\endcsname\relax
  \expandafter\@firstofone
\else
  \expandafter\@gobble
\fi
{%
  \def\loop#1\repeat{%
    \def\body{#1}%
    \iterate
  }%
  \def\iterate{%
    \body
      \let\next\iterate
    \else
      \let\next\relax
    \fi
    \next
  }%
  \let\repeat=\fi
}%
\def\RestoreCatcodes{}
\count@=0 %
\loop
  \edef\RestoreCatcodes{%
    \RestoreCatcodes
    \catcode\the\count@=\the\catcode\count@\relax
  }%
\ifnum\count@<255 %
  \advance\count@ 1 %
\repeat

\def\RangeCatcodeInvalid#1#2{%
  \count@=#1\relax
  \loop
    \catcode\count@=15 %
  \ifnum\count@<#2\relax
    \advance\count@ 1 %
  \repeat
}
\def\RangeCatcodeCheck#1#2#3{%
  \count@=#1\relax
  \loop
    \ifnum#3=\catcode\count@
    \else
      \errmessage{%
        Character \the\count@\space
        with wrong catcode \the\catcode\count@\space
        instead of \number#3%
      }%
    \fi
  \ifnum\count@<#2\relax
    \advance\count@ 1 %
  \repeat
}
\def\space{ }
\expandafter\ifx\csname LoadCommand\endcsname\relax
  \def\LoadCommand{\input protecteddef.sty\relax}%
\fi
\def\Test{%
  \RangeCatcodeInvalid{0}{47}%
  \RangeCatcodeInvalid{58}{64}%
  \RangeCatcodeInvalid{91}{96}%
  \RangeCatcodeInvalid{123}{255}%
  \catcode`\@=12 %
  \catcode`\\=0 %
  \catcode`\%=14 %
  \LoadCommand
  \RangeCatcodeCheck{0}{36}{15}%
  \RangeCatcodeCheck{37}{37}{14}%
  \RangeCatcodeCheck{38}{47}{15}%
  \RangeCatcodeCheck{48}{57}{12}%
  \RangeCatcodeCheck{58}{63}{15}%
  \RangeCatcodeCheck{64}{64}{12}%
  \RangeCatcodeCheck{65}{90}{11}%
  \RangeCatcodeCheck{91}{91}{15}%
  \RangeCatcodeCheck{92}{92}{0}%
  \RangeCatcodeCheck{93}{96}{15}%
  \RangeCatcodeCheck{97}{122}{11}%
  \RangeCatcodeCheck{123}{255}{15}%
  \RestoreCatcodes
}
\Test
\csname @@end\endcsname
\end
%    \end{macrocode}
%    \begin{macrocode}
%</test1>
%    \end{macrocode}
%
% \subsection{Test without \hologo{LaTeX} and \cs{protected}}
%
%    \begin{macrocode}
%<*test2>
\errorcontextlines=10000 %
\begingroup\expandafter\expandafter\expandafter\endgroup
\expandafter\ifx\csname RequirePackage\endcsname\relax
  \input protecteddef.sty\relax
  \catcode`\{=1 %
  \catcode`\}=2 %
  \catcode`\#=6 %
\else
  \RequirePackage{protecteddef}[2016/05/16]%
\fi
\begingroup\expandafter\expandafter\expandafter\endgroup
\expandafter\ifx\csname protected\endcsname\relax
  \let\pdef\def
\else
  \def\pdef{\protected\def}%
\fi
\def\msg#{\immediate\write16}
\countdef\errcount=2 %
\long\def\BeginCheck#1\ProtectedDef#2\EndCheck{%
  \begingroup
    \toks0={\ProtectedDef#2}%
    \msg{<<\the\toks0>>}%
  \endgroup
  \setbox0=\hbox{%
    #1%
    \ProtectedDef#2%
    \check\foo
  }%
  \ifdim\wd0=0pt\relax
  \else
    \errmessage{[Definition] Unwanted spaces?!}%
  \fi
  \setbox0=\hbox{%
    \def\fooinitial{XYZ}%
    \let\foo\fooinitial
    \errcount=0 %
    \expandafter\def\csname @PackageError\endcsname##1##2##3{%
      \advance\errcount by 1 %
    }%
    \expandafter\def\csname @notdefinable\endcsname{%
      \advance\errcount by 1 %
    }%
    \ProtectedDef#2%
    \ifnum\errcount=1 %
    \else
      \errmessage{1 error expected, but found: \the\errcount}%
    \fi
    \ifx\foo\fooinitial
    \else
      \def\space{ }%
      \errmessage{\string\foo\space is overwritten}%
    \fi
  }%
  \ifdim\wd0=0pt\relax
  \else
    \errmessage{[Error] Unwanted spaces?!}%
  \fi
}
\chardef\DeclareVersion=0 %
\begingroup\expandafter\expandafter\expandafter\endgroup
\expandafter\ifx\csname protected\endcsname\relax
  \begingroup\expandafter\expandafter\expandafter\endgroup
  \expandafter\ifx\csname DeclareRobustCommand\endcsname\relax
  \else
    \chardef\DeclareVersion=1 %
  \fi
\fi
\ifnum\DeclareVersion=0 %
  \def\check#1{%
    \ifx\cmp#1%
      \msg{* Test passed.}%
    \else
      \msg{}%
      \msg{[\meaning#1]}%
      \msg{[\meaning\cmp]}%
      \errmessage{Test failed!}%
    \fi
  }%
\else
  \def\check#1{%
    \begingroup
      \escapechar=-1 %
    \edef\x{\endgroup
      \def\noexpand\cs/{\string#1}%
    }\x
    \edef\CMP{%
      \noexpand\protect
      \expandafter\noexpand\csname\cs/ \endcsname
    }%
    \ifx\CMP#1%
      \expandafter\ifx\csname\cs/ \endcsname\cmp
        \msg{Test passed.}%
      \else
        \msg{}%
        \msg{[\expandafter\meaning\csname\cs/ \endcsname]}%
        \msg{[\meaning\cmp]}%
        \errmessage{Test failed!}%
      \fi
    \else
      \msg{}%
      \msg{[\meaning#1]}%
      \msg{[\meaning\CMP]}%
      \errmessage{Test failed!}%
    \fi
  }%
\fi

\tracingmacros=1

\BeginCheck
  \pdef\cmp{}%
  \ProtectedDef*\foo{}%
\EndCheck

\BeginCheck
  \pdef\cmp{}%
  \ProtectedDef*\foo[0]{}%
\EndCheck

\BeginCheck
  \pdef\cmp#1{<#1>}%
  \ProtectedDef*\foo[1]{<#1>}%
\EndCheck

\BeginCheck
  \pdef\cmp(#1){<#1>}%
  \ProtectedDef*\foo(#1){<#1>}%
\EndCheck

\BeginCheck
  \long\pdef\cmp{}%
  \ProtectedDef\foo{}%
\EndCheck

\BeginCheck
  \long\pdef\cmp{}%
  \ProtectedDef\foo[0]{}%
\EndCheck

\BeginCheck
  \long\pdef\cmp#1{<#1>}%
  \ProtectedDef\foo[1]{<#1>}%
\EndCheck

\BeginCheck
  \long\pdef\cmp(#1){<#1>}%
  \ProtectedDef\foo(#1){<#1>}%
\EndCheck

\csname @@end\endcsname\end
%</test2>
%    \end{macrocode}
%
% \section{Installation}
%
% \subsection{Download}
%
% \paragraph{Package.} This package is available on
% CTAN\footnote{\CTANpkg{protecteddef}}:
% \begin{description}
% \item[\CTAN{macros/latex/contrib/oberdiek/protecteddef.dtx}] The source file.
% \item[\CTAN{macros/latex/contrib/oberdiek/protecteddef.pdf}] Documentation.
% \end{description}
%
%
% \paragraph{Bundle.} All the packages of the bundle `oberdiek'
% are also available in a TDS compliant ZIP archive. There
% the packages are already unpacked and the documentation files
% are generated. The files and directories obey the TDS standard.
% \begin{description}
% \item[\CTANinstall{install/macros/latex/contrib/oberdiek.tds.zip}]
% \end{description}
% \emph{TDS} refers to the standard ``A Directory Structure
% for \TeX\ Files'' (\CTAN{tds/tds.pdf}). Directories
% with \xfile{texmf} in their name are usually organized this way.
%
% \subsection{Bundle installation}
%
% \paragraph{Unpacking.} Unpack the \xfile{oberdiek.tds.zip} in the
% TDS tree (also known as \xfile{texmf} tree) of your choice.
% Example (linux):
% \begin{quote}
%   |unzip oberdiek.tds.zip -d ~/texmf|
% \end{quote}
%
% \paragraph{Script installation.}
% Check the directory \xfile{TDS:scripts/oberdiek/} for
% scripts that need further installation steps.
%
% \subsection{Package installation}
%
% \paragraph{Unpacking.} The \xfile{.dtx} file is a self-extracting
% \docstrip\ archive. The files are extracted by running the
% \xfile{.dtx} through \plainTeX:
% \begin{quote}
%   \verb|tex protecteddef.dtx|
% \end{quote}
%
% \paragraph{TDS.} Now the different files must be moved into
% the different directories in your installation TDS tree
% (also known as \xfile{texmf} tree):
% \begin{quote}
% \def\t{^^A
% \begin{tabular}{@{}>{\ttfamily}l@{ $\rightarrow$ }>{\ttfamily}l@{}}
%   protecteddef.sty & tex/generic/oberdiek/protecteddef.sty\\
%   protecteddef.pdf & doc/latex/oberdiek/protecteddef.pdf\\
%   test/protecteddef-test1.tex & doc/latex/oberdiek/test/protecteddef-test1.tex\\
%   test/protecteddef-test2.tex & doc/latex/oberdiek/test/protecteddef-test2.tex\\
%   protecteddef.dtx & source/latex/oberdiek/protecteddef.dtx\\
% \end{tabular}^^A
% }^^A
% \sbox0{\t}^^A
% \ifdim\wd0>\linewidth
%   \begingroup
%     \advance\linewidth by\leftmargin
%     \advance\linewidth by\rightmargin
%   \edef\x{\endgroup
%     \def\noexpand\lw{\the\linewidth}^^A
%   }\x
%   \def\lwbox{^^A
%     \leavevmode
%     \hbox to \linewidth{^^A
%       \kern-\leftmargin\relax
%       \hss
%       \usebox0
%       \hss
%       \kern-\rightmargin\relax
%     }^^A
%   }^^A
%   \ifdim\wd0>\lw
%     \sbox0{\small\t}^^A
%     \ifdim\wd0>\linewidth
%       \ifdim\wd0>\lw
%         \sbox0{\footnotesize\t}^^A
%         \ifdim\wd0>\linewidth
%           \ifdim\wd0>\lw
%             \sbox0{\scriptsize\t}^^A
%             \ifdim\wd0>\linewidth
%               \ifdim\wd0>\lw
%                 \sbox0{\tiny\t}^^A
%                 \ifdim\wd0>\linewidth
%                   \lwbox
%                 \else
%                   \usebox0
%                 \fi
%               \else
%                 \lwbox
%               \fi
%             \else
%               \usebox0
%             \fi
%           \else
%             \lwbox
%           \fi
%         \else
%           \usebox0
%         \fi
%       \else
%         \lwbox
%       \fi
%     \else
%       \usebox0
%     \fi
%   \else
%     \lwbox
%   \fi
% \else
%   \usebox0
% \fi
% \end{quote}
% If you have a \xfile{docstrip.cfg} that configures and enables \docstrip's
% TDS installing feature, then some files can already be in the right
% place, see the documentation of \docstrip.
%
% \subsection{Refresh file name databases}
%
% If your \TeX~distribution
% (\TeX\,Live, \mikTeX, \dots) relies on file name databases, you must refresh
% these. For example, \TeX\,Live\ users run \verb|texhash| or
% \verb|mktexlsr|.
%
% \subsection{Some details for the interested}
%
% \paragraph{Unpacking with \LaTeX.}
% The \xfile{.dtx} chooses its action depending on the format:
% \begin{description}
% \item[\plainTeX:] Run \docstrip\ and extract the files.
% \item[\LaTeX:] Generate the documentation.
% \end{description}
% If you insist on using \LaTeX\ for \docstrip\ (really,
% \docstrip\ does not need \LaTeX), then inform the autodetect routine
% about your intention:
% \begin{quote}
%   \verb|latex \let\install=y\input{protecteddef.dtx}|
% \end{quote}
% Do not forget to quote the argument according to the demands
% of your shell.
%
% \paragraph{Generating the documentation.}
% You can use both the \xfile{.dtx} or the \xfile{.drv} to generate
% the documentation. The process can be configured by the
% configuration file \xfile{ltxdoc.cfg}. For instance, put this
% line into this file, if you want to have A4 as paper format:
% \begin{quote}
%   \verb|\PassOptionsToClass{a4paper}{article}|
% \end{quote}
% An example follows how to generate the
% documentation with pdf\LaTeX:
% \begin{quote}
%\begin{verbatim}
%pdflatex protecteddef.dtx
%makeindex -s gind.ist protecteddef.idx
%pdflatex protecteddef.dtx
%makeindex -s gind.ist protecteddef.idx
%pdflatex protecteddef.dtx
%\end{verbatim}
% \end{quote}
%
% \begin{History}
%   \begin{Version}{2011/01/31 v1.0}
%   \item
%     First public version.
%   \end{Version}
%   \begin{Version}{2016/05/16 v1.1}
%   \item
%     Documentation updates.
%   \end{Version}
% \end{History}
%
% \PrintIndex
%
% \Finale
\endinput
|
% \end{quote}
% Do not forget to quote the argument according to the demands
% of your shell.
%
% \paragraph{Generating the documentation.}
% You can use both the \xfile{.dtx} or the \xfile{.drv} to generate
% the documentation. The process can be configured by the
% configuration file \xfile{ltxdoc.cfg}. For instance, put this
% line into this file, if you want to have A4 as paper format:
% \begin{quote}
%   \verb|\PassOptionsToClass{a4paper}{article}|
% \end{quote}
% An example follows how to generate the
% documentation with pdf\LaTeX:
% \begin{quote}
%\begin{verbatim}
%pdflatex protecteddef.dtx
%makeindex -s gind.ist protecteddef.idx
%pdflatex protecteddef.dtx
%makeindex -s gind.ist protecteddef.idx
%pdflatex protecteddef.dtx
%\end{verbatim}
% \end{quote}
%
% \begin{History}
%   \begin{Version}{2011/01/31 v1.0}
%   \item
%     First public version.
%   \end{Version}
%   \begin{Version}{2016/05/16 v1.1}
%   \item
%     Documentation updates.
%   \end{Version}
% \end{History}
%
% \PrintIndex
%
% \Finale
\endinput
|
% \end{quote}
% Do not forget to quote the argument according to the demands
% of your shell.
%
% \paragraph{Generating the documentation.}
% You can use both the \xfile{.dtx} or the \xfile{.drv} to generate
% the documentation. The process can be configured by the
% configuration file \xfile{ltxdoc.cfg}. For instance, put this
% line into this file, if you want to have A4 as paper format:
% \begin{quote}
%   \verb|\PassOptionsToClass{a4paper}{article}|
% \end{quote}
% An example follows how to generate the
% documentation with pdf\LaTeX:
% \begin{quote}
%\begin{verbatim}
%pdflatex protecteddef.dtx
%makeindex -s gind.ist protecteddef.idx
%pdflatex protecteddef.dtx
%makeindex -s gind.ist protecteddef.idx
%pdflatex protecteddef.dtx
%\end{verbatim}
% \end{quote}
%
% \begin{History}
%   \begin{Version}{2011/01/31 v1.0}
%   \item
%     First public version.
%   \end{Version}
%   \begin{Version}{2016/05/16 v1.1}
%   \item
%     Documentation updates.
%   \end{Version}
% \end{History}
%
% \PrintIndex
%
% \Finale
\endinput

%        (quote the arguments according to the demands of your shell)
%
% Documentation:
%    (a) If protecteddef.drv is present:
%           latex protecteddef.drv
%    (b) Without protecteddef.drv:
%           latex protecteddef.dtx; ...
%    The class ltxdoc loads the configuration file ltxdoc.cfg
%    if available. Here you can specify further options, e.g.
%    use A4 as paper format:
%       \PassOptionsToClass{a4paper}{article}
%
%    Programm calls to get the documentation (example):
%       pdflatex protecteddef.dtx
%       makeindex -s gind.ist protecteddef.idx
%       pdflatex protecteddef.dtx
%       makeindex -s gind.ist protecteddef.idx
%       pdflatex protecteddef.dtx
%
% Installation:
%    TDS:tex/generic/oberdiek/protecteddef.sty
%    TDS:doc/latex/oberdiek/protecteddef.pdf
%    TDS:doc/latex/oberdiek/test/protecteddef-test1.tex
%    TDS:doc/latex/oberdiek/test/protecteddef-test2.tex
%    TDS:source/latex/oberdiek/protecteddef.dtx
%
%<*ignore>
\begingroup
  \catcode123=1 %
  \catcode125=2 %
  \def\x{LaTeX2e}%
\expandafter\endgroup
\ifcase 0\ifx\install y1\fi\expandafter
         \ifx\csname processbatchFile\endcsname\relax\else1\fi
         \ifx\fmtname\x\else 1\fi\relax
\else\csname fi\endcsname
%</ignore>
%<*install>
\input docstrip.tex
\Msg{************************************************************************}
\Msg{* Installation}
\Msg{* Package: protecteddef 2016/05/16 v1.1 Define protected commands (HO)}
\Msg{************************************************************************}

\keepsilent
\askforoverwritefalse

\let\MetaPrefix\relax
\preamble

This is a generated file.

Project: protecteddef
Version: 2016/05/16 v1.1

Copyright (C) 2011 by
   Heiko Oberdiek <heiko.oberdiek at googlemail.com>

This work may be distributed and/or modified under the
conditions of the LaTeX Project Public License, either
version 1.3c of this license or (at your option) any later
version. This version of this license is in
   https://www.latex-project.org/lppl/lppl-1-3c.txt
and the latest version of this license is in
   https://www.latex-project.org/lppl.txt
and version 1.3 or later is part of all distributions of
LaTeX version 2005/12/01 or later.

This work has the LPPL maintenance status "maintained".

The Current Maintainers of this work are
Heiko Oberdiek and the Oberdiek Package Support Group
https://github.com/ho-tex/oberdiek/issues


The Base Interpreter refers to any `TeX-Format',
because some files are installed in TDS:tex/generic//.

This work consists of the main source file protecteddef.dtx
and the derived files
   protecteddef.sty, protecteddef.pdf, protecteddef.ins, protecteddef.drv,
   protecteddef-test1.tex, protecteddef-test2.tex.

\endpreamble
\let\MetaPrefix\DoubleperCent

\generate{%
  \file{protecteddef.ins}{\from{protecteddef.dtx}{install}}%
  \file{protecteddef.drv}{\from{protecteddef.dtx}{driver}}%
  \usedir{tex/generic/oberdiek}%
  \file{protecteddef.sty}{\from{protecteddef.dtx}{package}}%
%  \usedir{doc/latex/oberdiek/test}%
%  \file{protecteddef-test1.tex}{\from{protecteddef.dtx}{test1}}%
%  \file{protecteddef-test2.tex}{\from{protecteddef.dtx}{test2}}%
  \nopreamble
  \nopostamble
%  \usedir{source/latex/oberdiek/catalogue}%
%  \file{protecteddef.xml}{\from{protecteddef.dtx}{catalogue}}%
}

\catcode32=13\relax% active space
\let =\space%
\Msg{************************************************************************}
\Msg{*}
\Msg{* To finish the installation you have to move the following}
\Msg{* file into a directory searched by TeX:}
\Msg{*}
\Msg{*     protecteddef.sty}
\Msg{*}
\Msg{* To produce the documentation run the file `protecteddef.drv'}
\Msg{* through LaTeX.}
\Msg{*}
\Msg{* Happy TeXing!}
\Msg{*}
\Msg{************************************************************************}

\endbatchfile
%</install>
%<*ignore>
\fi
%</ignore>
%<*driver>
\NeedsTeXFormat{LaTeX2e}
\ProvidesFile{protecteddef.drv}%
  [2016/05/16 v1.1 Define protected commands (HO)]%
\documentclass{ltxdoc}
\usepackage{holtxdoc}[2011/11/22]
\begin{document}
  \DocInput{protecteddef.dtx}%
\end{document}
%</driver>
% \fi
%
%
% \CharacterTable
%  {Upper-case    \A\B\C\D\E\F\G\H\I\J\K\L\M\N\O\P\Q\R\S\T\U\V\W\X\Y\Z
%   Lower-case    \a\b\c\d\e\f\g\h\i\j\k\l\m\n\o\p\q\r\s\t\u\v\w\x\y\z
%   Digits        \0\1\2\3\4\5\6\7\8\9
%   Exclamation   \!     Double quote  \"     Hash (number) \#
%   Dollar        \$     Percent       \%     Ampersand     \&
%   Acute accent  \'     Left paren    \(     Right paren   \)
%   Asterisk      \*     Plus          \+     Comma         \,
%   Minus         \-     Point         \.     Solidus       \/
%   Colon         \:     Semicolon     \;     Less than     \<
%   Equals        \=     Greater than  \>     Question mark \?
%   Commercial at \@     Left bracket  \[     Backslash     \\
%   Right bracket \]     Circumflex    \^     Underscore    \_
%   Grave accent  \`     Left brace    \{     Vertical bar  \|
%   Right brace   \}     Tilde         \~}
%
% \GetFileInfo{protecteddef.drv}
%
% \title{The \xpackage{protecteddef} package}
% \date{2016/05/16 v1.1}
% \author{Heiko Oberdiek\thanks
% {Please report any issues at \url{https://github.com/ho-tex/oberdiek/issues}}}
%
% \maketitle
%
% \begin{abstract}
% This packages provides \cs{ProtectedDef} for defining
% robust macros for both \hologo{plainTeX} and \hologo{LaTeX}.
% First \hologo{eTeX}'s \cs{protected} is tried, then
% \hologo{LaTeX}'s \cs{DeclareRobustCommand} is used.
% Otherwise the macro is not made robust.
% \end{abstract}
%
% \tableofcontents
%
% \section{Documentation}
%
% Many of my packages work for both formats \hologo{plainTeX}
% and \hologo{LaTeX}, even \hologo{iniTeX} is often supported.
% It would be nice if fragile macros could be protected and
% made robust. However the different format worlds offer different
% solutions.
%
% \subsection{The \hologo{LaTeX}'s way}
%
% Usually \cs{newcommand} is used to define macros. It provides
% a check if the command to be defined is already defined or
% cannot be defined for other reasons.
%
% For making robust macros \hologo{LaTeX} provides \cs{DeclareRobustCommand}.
% It shares the syntax with \cs{newcommand}. However it does not provide
% latters check. Internally the check is available via \cs{@ifdefinable}.
%
% Internally the robust macro is using \cs{protect} with a nested
% macro definition. The \cs{protect} infrastructure is a feature of
% \cs{LaTeX} and usually not available in other formats.
%
% \subsection{The \hologo{eTeX}'s way}
%
% The need for robust macros is addressed in \cs{eTeX}. It provides
% \cs{protected} that modifies the behaviour of \cs{def} in a similar
% way as \cs{long}. A protected macro does not expand in
% some expandable contexts like writing to a file or \cs{edef}.
%
% \subsection{The way of this package}
%
% The package tries to find the available protection mechanism.
% First it looks for \cs{eTeX}'s \cs{protected}, then
% it uses \hologo{LaTeX}'s \cs{DeclareRobustCommand}.
% If both fails, then the macro remains unprotected.
%
% Additionally, \cs{LaTeX}'s check, if a macro is already defined
% is added in all cases. First \hologo{LaTeX}'s \cs{@ifdefinable}
% is tried to be compatible with \hologo{LaTeX}.
% If \cs{@ifdefinable} is not available, then the test is implemented
% by asserting that the macro is undefined or has the meaning of \cs{relax}.
% If the test fails, then in all cases the macro is not defined
% and an error is thrown.
%
% \subsection{Usage}
%
% \begin{declcs}{ProtectedDef} \* \M{cmd} [\meta{num}] \M{definition text}
% \end{declcs}
% Macro \cs{ProtectedDef} follows the syntax of \hologo{LaTeX}'s
% \cs{newcommand} with the exception that an optional argument
% is not supported. Macro \meta{cmd} is to be defined as
% \cs{long} macro without star with \meta{num} arguments.
%
% The number of arguments \meta{num} must be given as explicite digit
% |0| upto |9|. Otherwise the part between the argument \meta{cmd} and
% the \meta{definition text} is taken as parameter text in the syntax
% of vanilla \hologo{TeX}. Examples (with \cs{protected}):
% \begin{quote}
%   \def\to{$\Rightarrow$ }%
%   |\ProtectedDef*{\foo}[1]{\message{#1}}|\\
%   \to |\protected\def\foo#1{\message#1}}|\\[1ex]
%   |\ProtectedDef\foo{abc}|\\
%   \to |\protected\def\foo{abc}|\\[1ex]
%   |\ProtectedDef*\foo(#1)<#2>{#1/#2}|\\
%   \to |\protected\def\foo(#1)<#2>{#1/#2}|
% \end{quote}
%
% \StopEventually{
% }
%
% \section{Implementation}
%
%    \begin{macrocode}
%<*package>
%    \end{macrocode}
%
% \subsection{Reload check and package identification}
%    Reload check, especially if the package is not used with \LaTeX.
%    \begin{macrocode}
\begingroup\catcode61\catcode48\catcode32=10\relax%
  \catcode13=5 % ^^M
  \endlinechar=13 %
  \catcode35=6 % #
  \catcode39=12 % '
  \catcode44=12 % ,
  \catcode45=12 % -
  \catcode46=12 % .
  \catcode58=12 % :
  \catcode64=11 % @
  \catcode123=1 % {
  \catcode125=2 % }
  \expandafter\let\expandafter\x\csname ver@protecteddef.sty\endcsname
  \ifx\x\relax % plain-TeX, first loading
  \else
    \def\empty{}%
    \ifx\x\empty % LaTeX, first loading,
      % variable is initialized, but \ProvidesPackage not yet seen
    \else
      \expandafter\ifx\csname PackageInfo\endcsname\relax
        \def\x#1#2{%
          \immediate\write-1{Package #1 Info: #2.}%
        }%
      \else
        \def\x#1#2{\PackageInfo{#1}{#2, stopped}}%
      \fi
      \x{protecteddef}{The package is already loaded}%
      \aftergroup\endinput
    \fi
  \fi
\endgroup%
%    \end{macrocode}
%    Package identification:
%    \begin{macrocode}
\begingroup\catcode61\catcode48\catcode32=10\relax%
  \catcode13=5 % ^^M
  \endlinechar=13 %
  \catcode35=6 % #
  \catcode39=12 % '
  \catcode40=12 % (
  \catcode41=12 % )
  \catcode44=12 % ,
  \catcode45=12 % -
  \catcode46=12 % .
  \catcode47=12 % /
  \catcode58=12 % :
  \catcode64=11 % @
  \catcode91=12 % [
  \catcode93=12 % ]
  \catcode123=1 % {
  \catcode125=2 % }
  \expandafter\ifx\csname ProvidesPackage\endcsname\relax
    \def\x#1#2#3[#4]{\endgroup
      \immediate\write-1{Package: #3 #4}%
      \xdef#1{#4}%
    }%
  \else
    \def\x#1#2[#3]{\endgroup
      #2[{#3}]%
      \ifx#1\@undefined
        \xdef#1{#3}%
      \fi
      \ifx#1\relax
        \xdef#1{#3}%
      \fi
    }%
  \fi
\expandafter\x\csname ver@protecteddef.sty\endcsname
\ProvidesPackage{protecteddef}%
  [2016/05/16 v1.1 Define protected commands (HO)]%
%    \end{macrocode}
%
% \subsection{Catcodes}
%
%    \begin{macrocode}
\begingroup\catcode61\catcode48\catcode32=10\relax%
  \catcode13=5 % ^^M
  \endlinechar=13 %
  \catcode123=1 % {
  \catcode125=2 % }
  \catcode64=11 % @
  \def\x{\endgroup
    \expandafter\edef\csname ProDef@AtEnd\endcsname{%
      \endlinechar=\the\endlinechar\relax
      \catcode13=\the\catcode13\relax
      \catcode32=\the\catcode32\relax
      \catcode35=\the\catcode35\relax
      \catcode61=\the\catcode61\relax
      \catcode64=\the\catcode64\relax
      \catcode123=\the\catcode123\relax
      \catcode125=\the\catcode125\relax
    }%
  }%
\x\catcode61\catcode48\catcode32=10\relax%
\catcode13=5 % ^^M
\endlinechar=13 %
\catcode35=6 % #
\catcode64=11 % @
\catcode123=1 % {
\catcode125=2 % }
\def\TMP@EnsureCode#1#2{%
  \edef\ProDef@AtEnd{%
    \ProDef@AtEnd
    \catcode#1=\the\catcode#1\relax
  }%
  \catcode#1=#2\relax
}
\TMP@EnsureCode{38}{4}% &
\TMP@EnsureCode{40}{12}% (
\TMP@EnsureCode{41}{12}% )
\TMP@EnsureCode{42}{12}% *
\TMP@EnsureCode{45}{12}% -
\TMP@EnsureCode{46}{12}% .
\TMP@EnsureCode{47}{12}% /
\TMP@EnsureCode{91}{12}% [
\TMP@EnsureCode{93}{12}% ]
\TMP@EnsureCode{96}{12}% `
\edef\ProDef@AtEnd{\ProDef@AtEnd\noexpand\endinput}
%    \end{macrocode}
%
% \subsection{Resources}
%
%    \begin{macrocode}
\begingroup\expandafter\expandafter\expandafter\endgroup
\expandafter\ifx\csname RequirePackage\endcsname\relax
  \def\TMP@RequirePackage#1[#2]{%
    \begingroup\expandafter\expandafter\expandafter\endgroup
    \expandafter\ifx\csname ver@#1.sty\endcsname\relax
      \input #1.sty\relax
    \fi
  }%
\else
  \let\TMP@RequirePackage\RequirePackage
\fi
\TMP@RequirePackage{ltxcmds}[2010/12/12]%
\TMP@RequirePackage{infwarerr}[2010/04/08]%
%    \end{macrocode}
%
%    \begin{macrocode}
\def\ProDef@temp#1{%
  \expandafter\def\csname ProDef@param[#1]\endcsname % hash-ok
}
\expandafter\def\csname ProDef@param\endcsname{}
\ProDef@temp0{}
\ProDef@temp1{##1}
\ProDef@temp2{##1##2}
\ProDef@temp3{##1##2##3}
\ProDef@temp4{##1##2##3##4}
\ProDef@temp5{##1##2##3##4##5}
\ProDef@temp6{##1##2##3##4##5##6}
\ProDef@temp7{##1##2##3##4##5##7}
\ProDef@temp8{##1##2##3##4##5##7##8}
\ProDef@temp9{##1##2##3##4##5##7##8##9}
%    \end{macrocode}
%
%    \begin{macro}{\ProDef@IfDefinable}
%    \begin{macrocode}
\ltx@IfUndefined{@ifdefinable}{%
  \long\def\ProDef@IfDefinable#1{%
    \begingroup
      \escapechar=-1 %
    \ltx@ifundefined{\string#1}{%
      \endgroup
      \ltx@firstofone
    }{%
      \expandafter\endgroup
      \expandafter
      \edef\expandafter\ProDef@temp\expandafter{\string#1 }%
      \@PackageError{protecteddef}{%
        Command \ltx@backslashchar\ProDef@temp already defined%
      }\@ehc
      \ltx@gobbletwo
    }%
  }%
}{%
  \long\def\ProDef@IfDefinable#1{%
    \let\ProDef@next\ltx@gobbletwo
    \@ifdefinable{#1}{%
      \let\ProDef@next\ltx@firstofone
    }%
    \ProDef@next
  }%
}
%    \end{macrocode}
%    \end{macro}
%
%    \begin{macrocode}
\begingroup\expandafter\expandafter\expandafter\endgroup
\expandafter\ifx\csname protected\endcsname\relax
  \begingroup\expandafter\expandafter\expandafter\endgroup
  \expandafter\ifx\csname DeclareRobustCommand\endcsname\relax
    \catcode`\&=14 % comment
  \else
    \newcommand*{\ProtectedDef}{%
      \ltx@ifnextchar*{%
        \ProDef@ProtectedDef
      }{%
        \ProDef@ProtectedDef{}%
      }%
    }%
    \long\def\ProDef@ProtectedDef#1#2#3#{%
      \ProDef@IfDefinable{#2}{%
        \ltx@IfUndefined{ProDef@param#3}{%
          \DeclareRobustCommand*{#2}{}%
          \begingroup
            \escapechar=-1 %
            \def\ProDef@temp{#1}%
          \edef\x{\endgroup
            \ifx\ProDef@temp\ltx@empty
              \noexpand\long
            \fi
            \noexpand\def
            \expandafter\noexpand\csname\string#2 \endcsname
          }%
          \x#3%
        }{%
          \DeclareRobustCommand#1{#2}#3%
        }%
      }%
    }%
    \expandafter\expandafter\expandafter\ProDef@AtEnd
  \fi
\else
  \catcode`\&=9 % ignore
\fi%
\ProDef@IfDefinable\ProtectedDef{%
&  \protected
  \def\ProtectedDef%
}{%
  \ltx@ifnextchar*{%
    \let\ProDef@long\ltx@empty
    \expandafter\ProDef@ProtectedDef\ltx@gobble
  }{%
    \let\ProDef@long\long
    \ProDef@ProtectedDef
  }%
}
\long\def\ProDef@ProtectedDef#1#2#{%
  \ProDef@IfDefinable{#1}{%
    \ltx@IfUndefined{ProDef@param#2}{%
&     \protected
      \ProDef@long
      \def#1#2%
    }{%
&     \protected
      \ProDef@long
      \expandafter\expandafter\expandafter\def
      \expandafter\expandafter\expandafter#1%
      \csname ProDef@param#2\endcsname
    }%
  }%
}
%    \end{macrocode}
%
%    \begin{macrocode}
\ProDef@AtEnd%
%    \end{macrocode}
%    \begin{macrocode}
%</package>
%    \end{macrocode}
%
% \section{Test}
%
% \subsection{Catcode checks for loading}
%
%    \begin{macrocode}
%<*test1>
%    \end{macrocode}
%    \begin{macrocode}
\catcode`\{=1 %
\catcode`\}=2 %
\catcode`\#=6 %
\catcode`\@=11 %
\expandafter\ifx\csname count@\endcsname\relax
  \countdef\count@=255 %
\fi
\expandafter\ifx\csname @gobble\endcsname\relax
  \long\def\@gobble#1{}%
\fi
\expandafter\ifx\csname @firstofone\endcsname\relax
  \long\def\@firstofone#1{#1}%
\fi
\expandafter\ifx\csname loop\endcsname\relax
  \expandafter\@firstofone
\else
  \expandafter\@gobble
\fi
{%
  \def\loop#1\repeat{%
    \def\body{#1}%
    \iterate
  }%
  \def\iterate{%
    \body
      \let\next\iterate
    \else
      \let\next\relax
    \fi
    \next
  }%
  \let\repeat=\fi
}%
\def\RestoreCatcodes{}
\count@=0 %
\loop
  \edef\RestoreCatcodes{%
    \RestoreCatcodes
    \catcode\the\count@=\the\catcode\count@\relax
  }%
\ifnum\count@<255 %
  \advance\count@ 1 %
\repeat

\def\RangeCatcodeInvalid#1#2{%
  \count@=#1\relax
  \loop
    \catcode\count@=15 %
  \ifnum\count@<#2\relax
    \advance\count@ 1 %
  \repeat
}
\def\RangeCatcodeCheck#1#2#3{%
  \count@=#1\relax
  \loop
    \ifnum#3=\catcode\count@
    \else
      \errmessage{%
        Character \the\count@\space
        with wrong catcode \the\catcode\count@\space
        instead of \number#3%
      }%
    \fi
  \ifnum\count@<#2\relax
    \advance\count@ 1 %
  \repeat
}
\def\space{ }
\expandafter\ifx\csname LoadCommand\endcsname\relax
  \def\LoadCommand{\input protecteddef.sty\relax}%
\fi
\def\Test{%
  \RangeCatcodeInvalid{0}{47}%
  \RangeCatcodeInvalid{58}{64}%
  \RangeCatcodeInvalid{91}{96}%
  \RangeCatcodeInvalid{123}{255}%
  \catcode`\@=12 %
  \catcode`\\=0 %
  \catcode`\%=14 %
  \LoadCommand
  \RangeCatcodeCheck{0}{36}{15}%
  \RangeCatcodeCheck{37}{37}{14}%
  \RangeCatcodeCheck{38}{47}{15}%
  \RangeCatcodeCheck{48}{57}{12}%
  \RangeCatcodeCheck{58}{63}{15}%
  \RangeCatcodeCheck{64}{64}{12}%
  \RangeCatcodeCheck{65}{90}{11}%
  \RangeCatcodeCheck{91}{91}{15}%
  \RangeCatcodeCheck{92}{92}{0}%
  \RangeCatcodeCheck{93}{96}{15}%
  \RangeCatcodeCheck{97}{122}{11}%
  \RangeCatcodeCheck{123}{255}{15}%
  \RestoreCatcodes
}
\Test
\csname @@end\endcsname
\end
%    \end{macrocode}
%    \begin{macrocode}
%</test1>
%    \end{macrocode}
%
% \subsection{Test without \hologo{LaTeX} and \cs{protected}}
%
%    \begin{macrocode}
%<*test2>
\errorcontextlines=10000 %
\begingroup\expandafter\expandafter\expandafter\endgroup
\expandafter\ifx\csname RequirePackage\endcsname\relax
  \input protecteddef.sty\relax
  \catcode`\{=1 %
  \catcode`\}=2 %
  \catcode`\#=6 %
\else
  \RequirePackage{protecteddef}[2016/05/16]%
\fi
\begingroup\expandafter\expandafter\expandafter\endgroup
\expandafter\ifx\csname protected\endcsname\relax
  \let\pdef\def
\else
  \def\pdef{\protected\def}%
\fi
\def\msg#{\immediate\write16}
\countdef\errcount=2 %
\long\def\BeginCheck#1\ProtectedDef#2\EndCheck{%
  \begingroup
    \toks0={\ProtectedDef#2}%
    \msg{<<\the\toks0>>}%
  \endgroup
  \setbox0=\hbox{%
    #1%
    \ProtectedDef#2%
    \check\foo
  }%
  \ifdim\wd0=0pt\relax
  \else
    \errmessage{[Definition] Unwanted spaces?!}%
  \fi
  \setbox0=\hbox{%
    \def\fooinitial{XYZ}%
    \let\foo\fooinitial
    \errcount=0 %
    \expandafter\def\csname @PackageError\endcsname##1##2##3{%
      \advance\errcount by 1 %
    }%
    \expandafter\def\csname @notdefinable\endcsname{%
      \advance\errcount by 1 %
    }%
    \ProtectedDef#2%
    \ifnum\errcount=1 %
    \else
      \errmessage{1 error expected, but found: \the\errcount}%
    \fi
    \ifx\foo\fooinitial
    \else
      \def\space{ }%
      \errmessage{\string\foo\space is overwritten}%
    \fi
  }%
  \ifdim\wd0=0pt\relax
  \else
    \errmessage{[Error] Unwanted spaces?!}%
  \fi
}
\chardef\DeclareVersion=0 %
\begingroup\expandafter\expandafter\expandafter\endgroup
\expandafter\ifx\csname protected\endcsname\relax
  \begingroup\expandafter\expandafter\expandafter\endgroup
  \expandafter\ifx\csname DeclareRobustCommand\endcsname\relax
  \else
    \chardef\DeclareVersion=1 %
  \fi
\fi
\ifnum\DeclareVersion=0 %
  \def\check#1{%
    \ifx\cmp#1%
      \msg{* Test passed.}%
    \else
      \msg{}%
      \msg{[\meaning#1]}%
      \msg{[\meaning\cmp]}%
      \errmessage{Test failed!}%
    \fi
  }%
\else
  \def\check#1{%
    \begingroup
      \escapechar=-1 %
    \edef\x{\endgroup
      \def\noexpand\cs/{\string#1}%
    }\x
    \edef\CMP{%
      \noexpand\protect
      \expandafter\noexpand\csname\cs/ \endcsname
    }%
    \ifx\CMP#1%
      \expandafter\ifx\csname\cs/ \endcsname\cmp
        \msg{Test passed.}%
      \else
        \msg{}%
        \msg{[\expandafter\meaning\csname\cs/ \endcsname]}%
        \msg{[\meaning\cmp]}%
        \errmessage{Test failed!}%
      \fi
    \else
      \msg{}%
      \msg{[\meaning#1]}%
      \msg{[\meaning\CMP]}%
      \errmessage{Test failed!}%
    \fi
  }%
\fi

\tracingmacros=1

\BeginCheck
  \pdef\cmp{}%
  \ProtectedDef*\foo{}%
\EndCheck

\BeginCheck
  \pdef\cmp{}%
  \ProtectedDef*\foo[0]{}%
\EndCheck

\BeginCheck
  \pdef\cmp#1{<#1>}%
  \ProtectedDef*\foo[1]{<#1>}%
\EndCheck

\BeginCheck
  \pdef\cmp(#1){<#1>}%
  \ProtectedDef*\foo(#1){<#1>}%
\EndCheck

\BeginCheck
  \long\pdef\cmp{}%
  \ProtectedDef\foo{}%
\EndCheck

\BeginCheck
  \long\pdef\cmp{}%
  \ProtectedDef\foo[0]{}%
\EndCheck

\BeginCheck
  \long\pdef\cmp#1{<#1>}%
  \ProtectedDef\foo[1]{<#1>}%
\EndCheck

\BeginCheck
  \long\pdef\cmp(#1){<#1>}%
  \ProtectedDef\foo(#1){<#1>}%
\EndCheck

\csname @@end\endcsname\end
%</test2>
%    \end{macrocode}
%
% \section{Installation}
%
% \subsection{Download}
%
% \paragraph{Package.} This package is available on
% CTAN\footnote{\CTANpkg{protecteddef}}:
% \begin{description}
% \item[\CTAN{macros/latex/contrib/oberdiek/protecteddef.dtx}] The source file.
% \item[\CTAN{macros/latex/contrib/oberdiek/protecteddef.pdf}] Documentation.
% \end{description}
%
%
% \paragraph{Bundle.} All the packages of the bundle `oberdiek'
% are also available in a TDS compliant ZIP archive. There
% the packages are already unpacked and the documentation files
% are generated. The files and directories obey the TDS standard.
% \begin{description}
% \item[\CTANinstall{install/macros/latex/contrib/oberdiek.tds.zip}]
% \end{description}
% \emph{TDS} refers to the standard ``A Directory Structure
% for \TeX\ Files'' (\CTAN{tds/tds.pdf}). Directories
% with \xfile{texmf} in their name are usually organized this way.
%
% \subsection{Bundle installation}
%
% \paragraph{Unpacking.} Unpack the \xfile{oberdiek.tds.zip} in the
% TDS tree (also known as \xfile{texmf} tree) of your choice.
% Example (linux):
% \begin{quote}
%   |unzip oberdiek.tds.zip -d ~/texmf|
% \end{quote}
%
% \paragraph{Script installation.}
% Check the directory \xfile{TDS:scripts/oberdiek/} for
% scripts that need further installation steps.
%
% \subsection{Package installation}
%
% \paragraph{Unpacking.} The \xfile{.dtx} file is a self-extracting
% \docstrip\ archive. The files are extracted by running the
% \xfile{.dtx} through \plainTeX:
% \begin{quote}
%   \verb|tex protecteddef.dtx|
% \end{quote}
%
% \paragraph{TDS.} Now the different files must be moved into
% the different directories in your installation TDS tree
% (also known as \xfile{texmf} tree):
% \begin{quote}
% \def\t{^^A
% \begin{tabular}{@{}>{\ttfamily}l@{ $\rightarrow$ }>{\ttfamily}l@{}}
%   protecteddef.sty & tex/generic/oberdiek/protecteddef.sty\\
%   protecteddef.pdf & doc/latex/oberdiek/protecteddef.pdf\\
%   test/protecteddef-test1.tex & doc/latex/oberdiek/test/protecteddef-test1.tex\\
%   test/protecteddef-test2.tex & doc/latex/oberdiek/test/protecteddef-test2.tex\\
%   protecteddef.dtx & source/latex/oberdiek/protecteddef.dtx\\
% \end{tabular}^^A
% }^^A
% \sbox0{\t}^^A
% \ifdim\wd0>\linewidth
%   \begingroup
%     \advance\linewidth by\leftmargin
%     \advance\linewidth by\rightmargin
%   \edef\x{\endgroup
%     \def\noexpand\lw{\the\linewidth}^^A
%   }\x
%   \def\lwbox{^^A
%     \leavevmode
%     \hbox to \linewidth{^^A
%       \kern-\leftmargin\relax
%       \hss
%       \usebox0
%       \hss
%       \kern-\rightmargin\relax
%     }^^A
%   }^^A
%   \ifdim\wd0>\lw
%     \sbox0{\small\t}^^A
%     \ifdim\wd0>\linewidth
%       \ifdim\wd0>\lw
%         \sbox0{\footnotesize\t}^^A
%         \ifdim\wd0>\linewidth
%           \ifdim\wd0>\lw
%             \sbox0{\scriptsize\t}^^A
%             \ifdim\wd0>\linewidth
%               \ifdim\wd0>\lw
%                 \sbox0{\tiny\t}^^A
%                 \ifdim\wd0>\linewidth
%                   \lwbox
%                 \else
%                   \usebox0
%                 \fi
%               \else
%                 \lwbox
%               \fi
%             \else
%               \usebox0
%             \fi
%           \else
%             \lwbox
%           \fi
%         \else
%           \usebox0
%         \fi
%       \else
%         \lwbox
%       \fi
%     \else
%       \usebox0
%     \fi
%   \else
%     \lwbox
%   \fi
% \else
%   \usebox0
% \fi
% \end{quote}
% If you have a \xfile{docstrip.cfg} that configures and enables \docstrip's
% TDS installing feature, then some files can already be in the right
% place, see the documentation of \docstrip.
%
% \subsection{Refresh file name databases}
%
% If your \TeX~distribution
% (\TeX\,Live, \mikTeX, \dots) relies on file name databases, you must refresh
% these. For example, \TeX\,Live\ users run \verb|texhash| or
% \verb|mktexlsr|.
%
% \subsection{Some details for the interested}
%
% \paragraph{Unpacking with \LaTeX.}
% The \xfile{.dtx} chooses its action depending on the format:
% \begin{description}
% \item[\plainTeX:] Run \docstrip\ and extract the files.
% \item[\LaTeX:] Generate the documentation.
% \end{description}
% If you insist on using \LaTeX\ for \docstrip\ (really,
% \docstrip\ does not need \LaTeX), then inform the autodetect routine
% about your intention:
% \begin{quote}
%   \verb|latex \let\install=y% \iffalse meta-comment
%
% File: protecteddef.dtx
% Version: 2016/05/16 v1.1
% Info: Define protected commands
%
% Copyright (C)
%    2011 Heiko Oberdiek
%    2016-2019 Oberdiek Package Support Group
%    https://github.com/ho-tex/oberdiek/issues
%
% This work may be distributed and/or modified under the
% conditions of the LaTeX Project Public License, either
% version 1.3c of this license or (at your option) any later
% version. This version of this license is in
%    https://www.latex-project.org/lppl/lppl-1-3c.txt
% and the latest version of this license is in
%    https://www.latex-project.org/lppl.txt
% and version 1.3 or later is part of all distributions of
% LaTeX version 2005/12/01 or later.
%
% This work has the LPPL maintenance status "maintained".
%
% The Current Maintainers of this work are
% Heiko Oberdiek and the Oberdiek Package Support Group
% https://github.com/ho-tex/oberdiek/issues
%
% The Base Interpreter refers to any `TeX-Format',
% because some files are installed in TDS:tex/generic//.
%
% This work consists of the main source file protecteddef.dtx
% and the derived files
%    protecteddef.sty, protecteddef.pdf, protecteddef.ins, protecteddef.drv,
%    protecteddef-test1.tex, protecteddef-test2.tex.
%
% Distribution:
%    CTAN:macros/latex/contrib/oberdiek/protecteddef.dtx
%    CTAN:macros/latex/contrib/oberdiek/protecteddef.pdf
%
% Unpacking:
%    (a) If protecteddef.ins is present:
%           tex protecteddef.ins
%    (b) Without protecteddef.ins:
%           tex protecteddef.dtx
%    (c) If you insist on using LaTeX
%           latex \let\install=y% \iffalse meta-comment
%
% File: protecteddef.dtx
% Version: 2016/05/16 v1.1
% Info: Define protected commands
%
% Copyright (C)
%    2011 Heiko Oberdiek
%    2016-2019 Oberdiek Package Support Group
%    https://github.com/ho-tex/oberdiek/issues
%
% This work may be distributed and/or modified under the
% conditions of the LaTeX Project Public License, either
% version 1.3c of this license or (at your option) any later
% version. This version of this license is in
%    https://www.latex-project.org/lppl/lppl-1-3c.txt
% and the latest version of this license is in
%    https://www.latex-project.org/lppl.txt
% and version 1.3 or later is part of all distributions of
% LaTeX version 2005/12/01 or later.
%
% This work has the LPPL maintenance status "maintained".
%
% The Current Maintainers of this work are
% Heiko Oberdiek and the Oberdiek Package Support Group
% https://github.com/ho-tex/oberdiek/issues
%
% The Base Interpreter refers to any `TeX-Format',
% because some files are installed in TDS:tex/generic//.
%
% This work consists of the main source file protecteddef.dtx
% and the derived files
%    protecteddef.sty, protecteddef.pdf, protecteddef.ins, protecteddef.drv,
%    protecteddef-test1.tex, protecteddef-test2.tex.
%
% Distribution:
%    CTAN:macros/latex/contrib/oberdiek/protecteddef.dtx
%    CTAN:macros/latex/contrib/oberdiek/protecteddef.pdf
%
% Unpacking:
%    (a) If protecteddef.ins is present:
%           tex protecteddef.ins
%    (b) Without protecteddef.ins:
%           tex protecteddef.dtx
%    (c) If you insist on using LaTeX
%           latex \let\install=y% \iffalse meta-comment
%
% File: protecteddef.dtx
% Version: 2016/05/16 v1.1
% Info: Define protected commands
%
% Copyright (C)
%    2011 Heiko Oberdiek
%    2016-2019 Oberdiek Package Support Group
%    https://github.com/ho-tex/oberdiek/issues
%
% This work may be distributed and/or modified under the
% conditions of the LaTeX Project Public License, either
% version 1.3c of this license or (at your option) any later
% version. This version of this license is in
%    https://www.latex-project.org/lppl/lppl-1-3c.txt
% and the latest version of this license is in
%    https://www.latex-project.org/lppl.txt
% and version 1.3 or later is part of all distributions of
% LaTeX version 2005/12/01 or later.
%
% This work has the LPPL maintenance status "maintained".
%
% The Current Maintainers of this work are
% Heiko Oberdiek and the Oberdiek Package Support Group
% https://github.com/ho-tex/oberdiek/issues
%
% The Base Interpreter refers to any `TeX-Format',
% because some files are installed in TDS:tex/generic//.
%
% This work consists of the main source file protecteddef.dtx
% and the derived files
%    protecteddef.sty, protecteddef.pdf, protecteddef.ins, protecteddef.drv,
%    protecteddef-test1.tex, protecteddef-test2.tex.
%
% Distribution:
%    CTAN:macros/latex/contrib/oberdiek/protecteddef.dtx
%    CTAN:macros/latex/contrib/oberdiek/protecteddef.pdf
%
% Unpacking:
%    (a) If protecteddef.ins is present:
%           tex protecteddef.ins
%    (b) Without protecteddef.ins:
%           tex protecteddef.dtx
%    (c) If you insist on using LaTeX
%           latex \let\install=y\input{protecteddef.dtx}
%        (quote the arguments according to the demands of your shell)
%
% Documentation:
%    (a) If protecteddef.drv is present:
%           latex protecteddef.drv
%    (b) Without protecteddef.drv:
%           latex protecteddef.dtx; ...
%    The class ltxdoc loads the configuration file ltxdoc.cfg
%    if available. Here you can specify further options, e.g.
%    use A4 as paper format:
%       \PassOptionsToClass{a4paper}{article}
%
%    Programm calls to get the documentation (example):
%       pdflatex protecteddef.dtx
%       makeindex -s gind.ist protecteddef.idx
%       pdflatex protecteddef.dtx
%       makeindex -s gind.ist protecteddef.idx
%       pdflatex protecteddef.dtx
%
% Installation:
%    TDS:tex/generic/oberdiek/protecteddef.sty
%    TDS:doc/latex/oberdiek/protecteddef.pdf
%    TDS:doc/latex/oberdiek/test/protecteddef-test1.tex
%    TDS:doc/latex/oberdiek/test/protecteddef-test2.tex
%    TDS:source/latex/oberdiek/protecteddef.dtx
%
%<*ignore>
\begingroup
  \catcode123=1 %
  \catcode125=2 %
  \def\x{LaTeX2e}%
\expandafter\endgroup
\ifcase 0\ifx\install y1\fi\expandafter
         \ifx\csname processbatchFile\endcsname\relax\else1\fi
         \ifx\fmtname\x\else 1\fi\relax
\else\csname fi\endcsname
%</ignore>
%<*install>
\input docstrip.tex
\Msg{************************************************************************}
\Msg{* Installation}
\Msg{* Package: protecteddef 2016/05/16 v1.1 Define protected commands (HO)}
\Msg{************************************************************************}

\keepsilent
\askforoverwritefalse

\let\MetaPrefix\relax
\preamble

This is a generated file.

Project: protecteddef
Version: 2016/05/16 v1.1

Copyright (C)
   2011 Heiko Oberdiek
   2016-2019 Oberdiek Package Support Group

This work may be distributed and/or modified under the
conditions of the LaTeX Project Public License, either
version 1.3c of this license or (at your option) any later
version. This version of this license is in
   https://www.latex-project.org/lppl/lppl-1-3c.txt
and the latest version of this license is in
   https://www.latex-project.org/lppl.txt
and version 1.3 or later is part of all distributions of
LaTeX version 2005/12/01 or later.

This work has the LPPL maintenance status "maintained".

The Current Maintainers of this work are
Heiko Oberdiek and the Oberdiek Package Support Group
https://github.com/ho-tex/oberdiek/issues


The Base Interpreter refers to any `TeX-Format',
because some files are installed in TDS:tex/generic//.

This work consists of the main source file protecteddef.dtx
and the derived files
   protecteddef.sty, protecteddef.pdf, protecteddef.ins, protecteddef.drv,
   protecteddef-test1.tex, protecteddef-test2.tex.

\endpreamble
\let\MetaPrefix\DoubleperCent

\generate{%
  \file{protecteddef.ins}{\from{protecteddef.dtx}{install}}%
  \file{protecteddef.drv}{\from{protecteddef.dtx}{driver}}%
  \usedir{tex/generic/oberdiek}%
  \file{protecteddef.sty}{\from{protecteddef.dtx}{package}}%
%  \usedir{doc/latex/oberdiek/test}%
%  \file{protecteddef-test1.tex}{\from{protecteddef.dtx}{test1}}%
%  \file{protecteddef-test2.tex}{\from{protecteddef.dtx}{test2}}%
  \nopreamble
  \nopostamble
%  \usedir{source/latex/oberdiek/catalogue}%
%  \file{protecteddef.xml}{\from{protecteddef.dtx}{catalogue}}%
}

\catcode32=13\relax% active space
\let =\space%
\Msg{************************************************************************}
\Msg{*}
\Msg{* To finish the installation you have to move the following}
\Msg{* file into a directory searched by TeX:}
\Msg{*}
\Msg{*     protecteddef.sty}
\Msg{*}
\Msg{* To produce the documentation run the file `protecteddef.drv'}
\Msg{* through LaTeX.}
\Msg{*}
\Msg{* Happy TeXing!}
\Msg{*}
\Msg{************************************************************************}

\endbatchfile
%</install>
%<*ignore>
\fi
%</ignore>
%<*driver>
\NeedsTeXFormat{LaTeX2e}
\ProvidesFile{protecteddef.drv}%
  [2016/05/16 v1.1 Define protected commands (HO)]%
\documentclass{ltxdoc}
\usepackage{holtxdoc}[2011/11/22]
\begin{document}
  \DocInput{protecteddef.dtx}%
\end{document}
%</driver>
% \fi
%
%
% \CharacterTable
%  {Upper-case    \A\B\C\D\E\F\G\H\I\J\K\L\M\N\O\P\Q\R\S\T\U\V\W\X\Y\Z
%   Lower-case    \a\b\c\d\e\f\g\h\i\j\k\l\m\n\o\p\q\r\s\t\u\v\w\x\y\z
%   Digits        \0\1\2\3\4\5\6\7\8\9
%   Exclamation   \!     Double quote  \"     Hash (number) \#
%   Dollar        \$     Percent       \%     Ampersand     \&
%   Acute accent  \'     Left paren    \(     Right paren   \)
%   Asterisk      \*     Plus          \+     Comma         \,
%   Minus         \-     Point         \.     Solidus       \/
%   Colon         \:     Semicolon     \;     Less than     \<
%   Equals        \=     Greater than  \>     Question mark \?
%   Commercial at \@     Left bracket  \[     Backslash     \\
%   Right bracket \]     Circumflex    \^     Underscore    \_
%   Grave accent  \`     Left brace    \{     Vertical bar  \|
%   Right brace   \}     Tilde         \~}
%
% \GetFileInfo{protecteddef.drv}
%
% \title{The \xpackage{protecteddef} package}
% \date{2016/05/16 v1.1}
% \author{Heiko Oberdiek\thanks
% {Please report any issues at \url{https://github.com/ho-tex/oberdiek/issues}}}
%
% \maketitle
%
% \begin{abstract}
% This packages provides \cs{ProtectedDef} for defining
% robust macros for both \hologo{plainTeX} and \hologo{LaTeX}.
% First \hologo{eTeX}'s \cs{protected} is tried, then
% \hologo{LaTeX}'s \cs{DeclareRobustCommand} is used.
% Otherwise the macro is not made robust.
% \end{abstract}
%
% \tableofcontents
%
% \section{Documentation}
%
% Many of my packages work for both formats \hologo{plainTeX}
% and \hologo{LaTeX}, even \hologo{iniTeX} is often supported.
% It would be nice if fragile macros could be protected and
% made robust. However the different format worlds offer different
% solutions.
%
% \subsection{The \hologo{LaTeX}'s way}
%
% Usually \cs{newcommand} is used to define macros. It provides
% a check if the command to be defined is already defined or
% cannot be defined for other reasons.
%
% For making robust macros \hologo{LaTeX} provides \cs{DeclareRobustCommand}.
% It shares the syntax with \cs{newcommand}. However it does not provide
% latters check. Internally the check is available via \cs{@ifdefinable}.
%
% Internally the robust macro is using \cs{protect} with a nested
% macro definition. The \cs{protect} infrastructure is a feature of
% \cs{LaTeX} and usually not available in other formats.
%
% \subsection{The \hologo{eTeX}'s way}
%
% The need for robust macros is addressed in \cs{eTeX}. It provides
% \cs{protected} that modifies the behaviour of \cs{def} in a similar
% way as \cs{long}. A protected macro does not expand in
% some expandable contexts like writing to a file or \cs{edef}.
%
% \subsection{The way of this package}
%
% The package tries to find the available protection mechanism.
% First it looks for \cs{eTeX}'s \cs{protected}, then
% it uses \hologo{LaTeX}'s \cs{DeclareRobustCommand}.
% If both fails, then the macro remains unprotected.
%
% Additionally, \cs{LaTeX}'s check, if a macro is already defined
% is added in all cases. First \hologo{LaTeX}'s \cs{@ifdefinable}
% is tried to be compatible with \hologo{LaTeX}.
% If \cs{@ifdefinable} is not available, then the test is implemented
% by asserting that the macro is undefined or has the meaning of \cs{relax}.
% If the test fails, then in all cases the macro is not defined
% and an error is thrown.
%
% \subsection{Usage}
%
% \begin{declcs}{ProtectedDef} \* \M{cmd} [\meta{num}] \M{definition text}
% \end{declcs}
% Macro \cs{ProtectedDef} follows the syntax of \hologo{LaTeX}'s
% \cs{newcommand} with the exception that an optional argument
% is not supported. Macro \meta{cmd} is to be defined as
% \cs{long} macro without star with \meta{num} arguments.
%
% The number of arguments \meta{num} must be given as explicite digit
% |0| upto |9|. Otherwise the part between the argument \meta{cmd} and
% the \meta{definition text} is taken as parameter text in the syntax
% of vanilla \hologo{TeX}. Examples (with \cs{protected}):
% \begin{quote}
%   \def\to{$\Rightarrow$ }%
%   |\ProtectedDef*{\foo}[1]{\message{#1}}|\\
%   \to |\protected\def\foo#1{\message#1}}|\\[1ex]
%   |\ProtectedDef\foo{abc}|\\
%   \to |\protected\def\foo{abc}|\\[1ex]
%   |\ProtectedDef*\foo(#1)<#2>{#1/#2}|\\
%   \to |\protected\def\foo(#1)<#2>{#1/#2}|
% \end{quote}
%
% \StopEventually{
% }
%
% \section{Implementation}
%
%    \begin{macrocode}
%<*package>
%    \end{macrocode}
%
% \subsection{Reload check and package identification}
%    Reload check, especially if the package is not used with \LaTeX.
%    \begin{macrocode}
\begingroup\catcode61\catcode48\catcode32=10\relax%
  \catcode13=5 % ^^M
  \endlinechar=13 %
  \catcode35=6 % #
  \catcode39=12 % '
  \catcode44=12 % ,
  \catcode45=12 % -
  \catcode46=12 % .
  \catcode58=12 % :
  \catcode64=11 % @
  \catcode123=1 % {
  \catcode125=2 % }
  \expandafter\let\expandafter\x\csname ver@protecteddef.sty\endcsname
  \ifx\x\relax % plain-TeX, first loading
  \else
    \def\empty{}%
    \ifx\x\empty % LaTeX, first loading,
      % variable is initialized, but \ProvidesPackage not yet seen
    \else
      \expandafter\ifx\csname PackageInfo\endcsname\relax
        \def\x#1#2{%
          \immediate\write-1{Package #1 Info: #2.}%
        }%
      \else
        \def\x#1#2{\PackageInfo{#1}{#2, stopped}}%
      \fi
      \x{protecteddef}{The package is already loaded}%
      \aftergroup\endinput
    \fi
  \fi
\endgroup%
%    \end{macrocode}
%    Package identification:
%    \begin{macrocode}
\begingroup\catcode61\catcode48\catcode32=10\relax%
  \catcode13=5 % ^^M
  \endlinechar=13 %
  \catcode35=6 % #
  \catcode39=12 % '
  \catcode40=12 % (
  \catcode41=12 % )
  \catcode44=12 % ,
  \catcode45=12 % -
  \catcode46=12 % .
  \catcode47=12 % /
  \catcode58=12 % :
  \catcode64=11 % @
  \catcode91=12 % [
  \catcode93=12 % ]
  \catcode123=1 % {
  \catcode125=2 % }
  \expandafter\ifx\csname ProvidesPackage\endcsname\relax
    \def\x#1#2#3[#4]{\endgroup
      \immediate\write-1{Package: #3 #4}%
      \xdef#1{#4}%
    }%
  \else
    \def\x#1#2[#3]{\endgroup
      #2[{#3}]%
      \ifx#1\@undefined
        \xdef#1{#3}%
      \fi
      \ifx#1\relax
        \xdef#1{#3}%
      \fi
    }%
  \fi
\expandafter\x\csname ver@protecteddef.sty\endcsname
\ProvidesPackage{protecteddef}%
  [2016/05/16 v1.1 Define protected commands (HO)]%
%    \end{macrocode}
%
% \subsection{Catcodes}
%
%    \begin{macrocode}
\begingroup\catcode61\catcode48\catcode32=10\relax%
  \catcode13=5 % ^^M
  \endlinechar=13 %
  \catcode123=1 % {
  \catcode125=2 % }
  \catcode64=11 % @
  \def\x{\endgroup
    \expandafter\edef\csname ProDef@AtEnd\endcsname{%
      \endlinechar=\the\endlinechar\relax
      \catcode13=\the\catcode13\relax
      \catcode32=\the\catcode32\relax
      \catcode35=\the\catcode35\relax
      \catcode61=\the\catcode61\relax
      \catcode64=\the\catcode64\relax
      \catcode123=\the\catcode123\relax
      \catcode125=\the\catcode125\relax
    }%
  }%
\x\catcode61\catcode48\catcode32=10\relax%
\catcode13=5 % ^^M
\endlinechar=13 %
\catcode35=6 % #
\catcode64=11 % @
\catcode123=1 % {
\catcode125=2 % }
\def\TMP@EnsureCode#1#2{%
  \edef\ProDef@AtEnd{%
    \ProDef@AtEnd
    \catcode#1=\the\catcode#1\relax
  }%
  \catcode#1=#2\relax
}
\TMP@EnsureCode{38}{4}% &
\TMP@EnsureCode{40}{12}% (
\TMP@EnsureCode{41}{12}% )
\TMP@EnsureCode{42}{12}% *
\TMP@EnsureCode{45}{12}% -
\TMP@EnsureCode{46}{12}% .
\TMP@EnsureCode{47}{12}% /
\TMP@EnsureCode{91}{12}% [
\TMP@EnsureCode{93}{12}% ]
\TMP@EnsureCode{96}{12}% `
\edef\ProDef@AtEnd{\ProDef@AtEnd\noexpand\endinput}
%    \end{macrocode}
%
% \subsection{Resources}
%
%    \begin{macrocode}
\begingroup\expandafter\expandafter\expandafter\endgroup
\expandafter\ifx\csname RequirePackage\endcsname\relax
  \def\TMP@RequirePackage#1[#2]{%
    \begingroup\expandafter\expandafter\expandafter\endgroup
    \expandafter\ifx\csname ver@#1.sty\endcsname\relax
      \input #1.sty\relax
    \fi
  }%
\else
  \let\TMP@RequirePackage\RequirePackage
\fi
\TMP@RequirePackage{ltxcmds}[2010/12/12]%
\TMP@RequirePackage{infwarerr}[2010/04/08]%
%    \end{macrocode}
%
%    \begin{macrocode}
\def\ProDef@temp#1{%
  \expandafter\def\csname ProDef@param[#1]\endcsname % hash-ok
}
\expandafter\def\csname ProDef@param\endcsname{}
\ProDef@temp0{}
\ProDef@temp1{##1}
\ProDef@temp2{##1##2}
\ProDef@temp3{##1##2##3}
\ProDef@temp4{##1##2##3##4}
\ProDef@temp5{##1##2##3##4##5}
\ProDef@temp6{##1##2##3##4##5##6}
\ProDef@temp7{##1##2##3##4##5##7}
\ProDef@temp8{##1##2##3##4##5##7##8}
\ProDef@temp9{##1##2##3##4##5##7##8##9}
%    \end{macrocode}
%
%    \begin{macro}{\ProDef@IfDefinable}
%    \begin{macrocode}
\ltx@IfUndefined{@ifdefinable}{%
  \long\def\ProDef@IfDefinable#1{%
    \begingroup
      \escapechar=-1 %
    \ltx@ifundefined{\string#1}{%
      \endgroup
      \ltx@firstofone
    }{%
      \expandafter\endgroup
      \expandafter
      \edef\expandafter\ProDef@temp\expandafter{\string#1 }%
      \@PackageError{protecteddef}{%
        Command \ltx@backslashchar\ProDef@temp already defined%
      }\@ehc
      \ltx@gobbletwo
    }%
  }%
}{%
  \long\def\ProDef@IfDefinable#1{%
    \let\ProDef@next\ltx@gobbletwo
    \@ifdefinable{#1}{%
      \let\ProDef@next\ltx@firstofone
    }%
    \ProDef@next
  }%
}
%    \end{macrocode}
%    \end{macro}
%
%    \begin{macrocode}
\begingroup\expandafter\expandafter\expandafter\endgroup
\expandafter\ifx\csname protected\endcsname\relax
  \begingroup\expandafter\expandafter\expandafter\endgroup
  \expandafter\ifx\csname DeclareRobustCommand\endcsname\relax
    \catcode`\&=14 % comment
  \else
    \newcommand*{\ProtectedDef}{%
      \ltx@ifnextchar*{%
        \ProDef@ProtectedDef
      }{%
        \ProDef@ProtectedDef{}%
      }%
    }%
    \long\def\ProDef@ProtectedDef#1#2#3#{%
      \ProDef@IfDefinable{#2}{%
        \ltx@IfUndefined{ProDef@param#3}{%
          \DeclareRobustCommand*{#2}{}%
          \begingroup
            \escapechar=-1 %
            \def\ProDef@temp{#1}%
          \edef\x{\endgroup
            \ifx\ProDef@temp\ltx@empty
              \noexpand\long
            \fi
            \noexpand\def
            \expandafter\noexpand\csname\string#2 \endcsname
          }%
          \x#3%
        }{%
          \DeclareRobustCommand#1{#2}#3%
        }%
      }%
    }%
    \expandafter\expandafter\expandafter\ProDef@AtEnd
  \fi
\else
  \catcode`\&=9 % ignore
\fi%
\ProDef@IfDefinable\ProtectedDef{%
&  \protected
  \def\ProtectedDef%
}{%
  \ltx@ifnextchar*{%
    \let\ProDef@long\ltx@empty
    \expandafter\ProDef@ProtectedDef\ltx@gobble
  }{%
    \let\ProDef@long\long
    \ProDef@ProtectedDef
  }%
}
\long\def\ProDef@ProtectedDef#1#2#{%
  \ProDef@IfDefinable{#1}{%
    \ltx@IfUndefined{ProDef@param#2}{%
&     \protected
      \ProDef@long
      \def#1#2%
    }{%
&     \protected
      \ProDef@long
      \expandafter\expandafter\expandafter\def
      \expandafter\expandafter\expandafter#1%
      \csname ProDef@param#2\endcsname
    }%
  }%
}
%    \end{macrocode}
%
%    \begin{macrocode}
\ProDef@AtEnd%
%    \end{macrocode}
%    \begin{macrocode}
%</package>
%    \end{macrocode}
%
% \section{Test}
%
% \subsection{Catcode checks for loading}
%
%    \begin{macrocode}
%<*test1>
%    \end{macrocode}
%    \begin{macrocode}
\catcode`\{=1 %
\catcode`\}=2 %
\catcode`\#=6 %
\catcode`\@=11 %
\expandafter\ifx\csname count@\endcsname\relax
  \countdef\count@=255 %
\fi
\expandafter\ifx\csname @gobble\endcsname\relax
  \long\def\@gobble#1{}%
\fi
\expandafter\ifx\csname @firstofone\endcsname\relax
  \long\def\@firstofone#1{#1}%
\fi
\expandafter\ifx\csname loop\endcsname\relax
  \expandafter\@firstofone
\else
  \expandafter\@gobble
\fi
{%
  \def\loop#1\repeat{%
    \def\body{#1}%
    \iterate
  }%
  \def\iterate{%
    \body
      \let\next\iterate
    \else
      \let\next\relax
    \fi
    \next
  }%
  \let\repeat=\fi
}%
\def\RestoreCatcodes{}
\count@=0 %
\loop
  \edef\RestoreCatcodes{%
    \RestoreCatcodes
    \catcode\the\count@=\the\catcode\count@\relax
  }%
\ifnum\count@<255 %
  \advance\count@ 1 %
\repeat

\def\RangeCatcodeInvalid#1#2{%
  \count@=#1\relax
  \loop
    \catcode\count@=15 %
  \ifnum\count@<#2\relax
    \advance\count@ 1 %
  \repeat
}
\def\RangeCatcodeCheck#1#2#3{%
  \count@=#1\relax
  \loop
    \ifnum#3=\catcode\count@
    \else
      \errmessage{%
        Character \the\count@\space
        with wrong catcode \the\catcode\count@\space
        instead of \number#3%
      }%
    \fi
  \ifnum\count@<#2\relax
    \advance\count@ 1 %
  \repeat
}
\def\space{ }
\expandafter\ifx\csname LoadCommand\endcsname\relax
  \def\LoadCommand{\input protecteddef.sty\relax}%
\fi
\def\Test{%
  \RangeCatcodeInvalid{0}{47}%
  \RangeCatcodeInvalid{58}{64}%
  \RangeCatcodeInvalid{91}{96}%
  \RangeCatcodeInvalid{123}{255}%
  \catcode`\@=12 %
  \catcode`\\=0 %
  \catcode`\%=14 %
  \LoadCommand
  \RangeCatcodeCheck{0}{36}{15}%
  \RangeCatcodeCheck{37}{37}{14}%
  \RangeCatcodeCheck{38}{47}{15}%
  \RangeCatcodeCheck{48}{57}{12}%
  \RangeCatcodeCheck{58}{63}{15}%
  \RangeCatcodeCheck{64}{64}{12}%
  \RangeCatcodeCheck{65}{90}{11}%
  \RangeCatcodeCheck{91}{91}{15}%
  \RangeCatcodeCheck{92}{92}{0}%
  \RangeCatcodeCheck{93}{96}{15}%
  \RangeCatcodeCheck{97}{122}{11}%
  \RangeCatcodeCheck{123}{255}{15}%
  \RestoreCatcodes
}
\Test
\csname @@end\endcsname
\end
%    \end{macrocode}
%    \begin{macrocode}
%</test1>
%    \end{macrocode}
%
% \subsection{Test without \hologo{LaTeX} and \cs{protected}}
%
%    \begin{macrocode}
%<*test2>
\errorcontextlines=10000 %
\begingroup\expandafter\expandafter\expandafter\endgroup
\expandafter\ifx\csname RequirePackage\endcsname\relax
  \input protecteddef.sty\relax
  \catcode`\{=1 %
  \catcode`\}=2 %
  \catcode`\#=6 %
\else
  \RequirePackage{protecteddef}[2016/05/16]%
\fi
\begingroup\expandafter\expandafter\expandafter\endgroup
\expandafter\ifx\csname protected\endcsname\relax
  \let\pdef\def
\else
  \def\pdef{\protected\def}%
\fi
\def\msg#{\immediate\write16}
\countdef\errcount=2 %
\long\def\BeginCheck#1\ProtectedDef#2\EndCheck{%
  \begingroup
    \toks0={\ProtectedDef#2}%
    \msg{<<\the\toks0>>}%
  \endgroup
  \setbox0=\hbox{%
    #1%
    \ProtectedDef#2%
    \check\foo
  }%
  \ifdim\wd0=0pt\relax
  \else
    \errmessage{[Definition] Unwanted spaces?!}%
  \fi
  \setbox0=\hbox{%
    \def\fooinitial{XYZ}%
    \let\foo\fooinitial
    \errcount=0 %
    \expandafter\def\csname @PackageError\endcsname##1##2##3{%
      \advance\errcount by 1 %
    }%
    \expandafter\def\csname @notdefinable\endcsname{%
      \advance\errcount by 1 %
    }%
    \ProtectedDef#2%
    \ifnum\errcount=1 %
    \else
      \errmessage{1 error expected, but found: \the\errcount}%
    \fi
    \ifx\foo\fooinitial
    \else
      \def\space{ }%
      \errmessage{\string\foo\space is overwritten}%
    \fi
  }%
  \ifdim\wd0=0pt\relax
  \else
    \errmessage{[Error] Unwanted spaces?!}%
  \fi
}
\chardef\DeclareVersion=0 %
\begingroup\expandafter\expandafter\expandafter\endgroup
\expandafter\ifx\csname protected\endcsname\relax
  \begingroup\expandafter\expandafter\expandafter\endgroup
  \expandafter\ifx\csname DeclareRobustCommand\endcsname\relax
  \else
    \chardef\DeclareVersion=1 %
  \fi
\fi
\ifnum\DeclareVersion=0 %
  \def\check#1{%
    \ifx\cmp#1%
      \msg{* Test passed.}%
    \else
      \msg{}%
      \msg{[\meaning#1]}%
      \msg{[\meaning\cmp]}%
      \errmessage{Test failed!}%
    \fi
  }%
\else
  \def\check#1{%
    \begingroup
      \escapechar=-1 %
    \edef\x{\endgroup
      \def\noexpand\cs/{\string#1}%
    }\x
    \edef\CMP{%
      \noexpand\protect
      \expandafter\noexpand\csname\cs/ \endcsname
    }%
    \ifx\CMP#1%
      \expandafter\ifx\csname\cs/ \endcsname\cmp
        \msg{Test passed.}%
      \else
        \msg{}%
        \msg{[\expandafter\meaning\csname\cs/ \endcsname]}%
        \msg{[\meaning\cmp]}%
        \errmessage{Test failed!}%
      \fi
    \else
      \msg{}%
      \msg{[\meaning#1]}%
      \msg{[\meaning\CMP]}%
      \errmessage{Test failed!}%
    \fi
  }%
\fi

\tracingmacros=1

\BeginCheck
  \pdef\cmp{}%
  \ProtectedDef*\foo{}%
\EndCheck

\BeginCheck
  \pdef\cmp{}%
  \ProtectedDef*\foo[0]{}%
\EndCheck

\BeginCheck
  \pdef\cmp#1{<#1>}%
  \ProtectedDef*\foo[1]{<#1>}%
\EndCheck

\BeginCheck
  \pdef\cmp(#1){<#1>}%
  \ProtectedDef*\foo(#1){<#1>}%
\EndCheck

\BeginCheck
  \long\pdef\cmp{}%
  \ProtectedDef\foo{}%
\EndCheck

\BeginCheck
  \long\pdef\cmp{}%
  \ProtectedDef\foo[0]{}%
\EndCheck

\BeginCheck
  \long\pdef\cmp#1{<#1>}%
  \ProtectedDef\foo[1]{<#1>}%
\EndCheck

\BeginCheck
  \long\pdef\cmp(#1){<#1>}%
  \ProtectedDef\foo(#1){<#1>}%
\EndCheck

\csname @@end\endcsname\end
%</test2>
%    \end{macrocode}
%
% \section{Installation}
%
% \subsection{Download}
%
% \paragraph{Package.} This package is available on
% CTAN\footnote{\CTANpkg{protecteddef}}:
% \begin{description}
% \item[\CTAN{macros/latex/contrib/oberdiek/protecteddef.dtx}] The source file.
% \item[\CTAN{macros/latex/contrib/oberdiek/protecteddef.pdf}] Documentation.
% \end{description}
%
%
% \paragraph{Bundle.} All the packages of the bundle `oberdiek'
% are also available in a TDS compliant ZIP archive. There
% the packages are already unpacked and the documentation files
% are generated. The files and directories obey the TDS standard.
% \begin{description}
% \item[\CTANinstall{install/macros/latex/contrib/oberdiek.tds.zip}]
% \end{description}
% \emph{TDS} refers to the standard ``A Directory Structure
% for \TeX\ Files'' (\CTAN{tds/tds.pdf}). Directories
% with \xfile{texmf} in their name are usually organized this way.
%
% \subsection{Bundle installation}
%
% \paragraph{Unpacking.} Unpack the \xfile{oberdiek.tds.zip} in the
% TDS tree (also known as \xfile{texmf} tree) of your choice.
% Example (linux):
% \begin{quote}
%   |unzip oberdiek.tds.zip -d ~/texmf|
% \end{quote}
%
% \paragraph{Script installation.}
% Check the directory \xfile{TDS:scripts/oberdiek/} for
% scripts that need further installation steps.
%
% \subsection{Package installation}
%
% \paragraph{Unpacking.} The \xfile{.dtx} file is a self-extracting
% \docstrip\ archive. The files are extracted by running the
% \xfile{.dtx} through \plainTeX:
% \begin{quote}
%   \verb|tex protecteddef.dtx|
% \end{quote}
%
% \paragraph{TDS.} Now the different files must be moved into
% the different directories in your installation TDS tree
% (also known as \xfile{texmf} tree):
% \begin{quote}
% \def\t{^^A
% \begin{tabular}{@{}>{\ttfamily}l@{ $\rightarrow$ }>{\ttfamily}l@{}}
%   protecteddef.sty & tex/generic/oberdiek/protecteddef.sty\\
%   protecteddef.pdf & doc/latex/oberdiek/protecteddef.pdf\\
%   test/protecteddef-test1.tex & doc/latex/oberdiek/test/protecteddef-test1.tex\\
%   test/protecteddef-test2.tex & doc/latex/oberdiek/test/protecteddef-test2.tex\\
%   protecteddef.dtx & source/latex/oberdiek/protecteddef.dtx\\
% \end{tabular}^^A
% }^^A
% \sbox0{\t}^^A
% \ifdim\wd0>\linewidth
%   \begingroup
%     \advance\linewidth by\leftmargin
%     \advance\linewidth by\rightmargin
%   \edef\x{\endgroup
%     \def\noexpand\lw{\the\linewidth}^^A
%   }\x
%   \def\lwbox{^^A
%     \leavevmode
%     \hbox to \linewidth{^^A
%       \kern-\leftmargin\relax
%       \hss
%       \usebox0
%       \hss
%       \kern-\rightmargin\relax
%     }^^A
%   }^^A
%   \ifdim\wd0>\lw
%     \sbox0{\small\t}^^A
%     \ifdim\wd0>\linewidth
%       \ifdim\wd0>\lw
%         \sbox0{\footnotesize\t}^^A
%         \ifdim\wd0>\linewidth
%           \ifdim\wd0>\lw
%             \sbox0{\scriptsize\t}^^A
%             \ifdim\wd0>\linewidth
%               \ifdim\wd0>\lw
%                 \sbox0{\tiny\t}^^A
%                 \ifdim\wd0>\linewidth
%                   \lwbox
%                 \else
%                   \usebox0
%                 \fi
%               \else
%                 \lwbox
%               \fi
%             \else
%               \usebox0
%             \fi
%           \else
%             \lwbox
%           \fi
%         \else
%           \usebox0
%         \fi
%       \else
%         \lwbox
%       \fi
%     \else
%       \usebox0
%     \fi
%   \else
%     \lwbox
%   \fi
% \else
%   \usebox0
% \fi
% \end{quote}
% If you have a \xfile{docstrip.cfg} that configures and enables \docstrip's
% TDS installing feature, then some files can already be in the right
% place, see the documentation of \docstrip.
%
% \subsection{Refresh file name databases}
%
% If your \TeX~distribution
% (\TeX\,Live, \mikTeX, \dots) relies on file name databases, you must refresh
% these. For example, \TeX\,Live\ users run \verb|texhash| or
% \verb|mktexlsr|.
%
% \subsection{Some details for the interested}
%
% \paragraph{Unpacking with \LaTeX.}
% The \xfile{.dtx} chooses its action depending on the format:
% \begin{description}
% \item[\plainTeX:] Run \docstrip\ and extract the files.
% \item[\LaTeX:] Generate the documentation.
% \end{description}
% If you insist on using \LaTeX\ for \docstrip\ (really,
% \docstrip\ does not need \LaTeX), then inform the autodetect routine
% about your intention:
% \begin{quote}
%   \verb|latex \let\install=y\input{protecteddef.dtx}|
% \end{quote}
% Do not forget to quote the argument according to the demands
% of your shell.
%
% \paragraph{Generating the documentation.}
% You can use both the \xfile{.dtx} or the \xfile{.drv} to generate
% the documentation. The process can be configured by the
% configuration file \xfile{ltxdoc.cfg}. For instance, put this
% line into this file, if you want to have A4 as paper format:
% \begin{quote}
%   \verb|\PassOptionsToClass{a4paper}{article}|
% \end{quote}
% An example follows how to generate the
% documentation with pdf\LaTeX:
% \begin{quote}
%\begin{verbatim}
%pdflatex protecteddef.dtx
%makeindex -s gind.ist protecteddef.idx
%pdflatex protecteddef.dtx
%makeindex -s gind.ist protecteddef.idx
%pdflatex protecteddef.dtx
%\end{verbatim}
% \end{quote}
%
% \begin{History}
%   \begin{Version}{2011/01/31 v1.0}
%   \item
%     First public version.
%   \end{Version}
%   \begin{Version}{2016/05/16 v1.1}
%   \item
%     Documentation updates.
%   \end{Version}
% \end{History}
%
% \PrintIndex
%
% \Finale
\endinput

%        (quote the arguments according to the demands of your shell)
%
% Documentation:
%    (a) If protecteddef.drv is present:
%           latex protecteddef.drv
%    (b) Without protecteddef.drv:
%           latex protecteddef.dtx; ...
%    The class ltxdoc loads the configuration file ltxdoc.cfg
%    if available. Here you can specify further options, e.g.
%    use A4 as paper format:
%       \PassOptionsToClass{a4paper}{article}
%
%    Programm calls to get the documentation (example):
%       pdflatex protecteddef.dtx
%       makeindex -s gind.ist protecteddef.idx
%       pdflatex protecteddef.dtx
%       makeindex -s gind.ist protecteddef.idx
%       pdflatex protecteddef.dtx
%
% Installation:
%    TDS:tex/generic/oberdiek/protecteddef.sty
%    TDS:doc/latex/oberdiek/protecteddef.pdf
%    TDS:doc/latex/oberdiek/test/protecteddef-test1.tex
%    TDS:doc/latex/oberdiek/test/protecteddef-test2.tex
%    TDS:source/latex/oberdiek/protecteddef.dtx
%
%<*ignore>
\begingroup
  \catcode123=1 %
  \catcode125=2 %
  \def\x{LaTeX2e}%
\expandafter\endgroup
\ifcase 0\ifx\install y1\fi\expandafter
         \ifx\csname processbatchFile\endcsname\relax\else1\fi
         \ifx\fmtname\x\else 1\fi\relax
\else\csname fi\endcsname
%</ignore>
%<*install>
\input docstrip.tex
\Msg{************************************************************************}
\Msg{* Installation}
\Msg{* Package: protecteddef 2016/05/16 v1.1 Define protected commands (HO)}
\Msg{************************************************************************}

\keepsilent
\askforoverwritefalse

\let\MetaPrefix\relax
\preamble

This is a generated file.

Project: protecteddef
Version: 2016/05/16 v1.1

Copyright (C)
   2011 Heiko Oberdiek
   2016-2019 Oberdiek Package Support Group

This work may be distributed and/or modified under the
conditions of the LaTeX Project Public License, either
version 1.3c of this license or (at your option) any later
version. This version of this license is in
   https://www.latex-project.org/lppl/lppl-1-3c.txt
and the latest version of this license is in
   https://www.latex-project.org/lppl.txt
and version 1.3 or later is part of all distributions of
LaTeX version 2005/12/01 or later.

This work has the LPPL maintenance status "maintained".

The Current Maintainers of this work are
Heiko Oberdiek and the Oberdiek Package Support Group
https://github.com/ho-tex/oberdiek/issues


The Base Interpreter refers to any `TeX-Format',
because some files are installed in TDS:tex/generic//.

This work consists of the main source file protecteddef.dtx
and the derived files
   protecteddef.sty, protecteddef.pdf, protecteddef.ins, protecteddef.drv,
   protecteddef-test1.tex, protecteddef-test2.tex.

\endpreamble
\let\MetaPrefix\DoubleperCent

\generate{%
  \file{protecteddef.ins}{\from{protecteddef.dtx}{install}}%
  \file{protecteddef.drv}{\from{protecteddef.dtx}{driver}}%
  \usedir{tex/generic/oberdiek}%
  \file{protecteddef.sty}{\from{protecteddef.dtx}{package}}%
%  \usedir{doc/latex/oberdiek/test}%
%  \file{protecteddef-test1.tex}{\from{protecteddef.dtx}{test1}}%
%  \file{protecteddef-test2.tex}{\from{protecteddef.dtx}{test2}}%
  \nopreamble
  \nopostamble
%  \usedir{source/latex/oberdiek/catalogue}%
%  \file{protecteddef.xml}{\from{protecteddef.dtx}{catalogue}}%
}

\catcode32=13\relax% active space
\let =\space%
\Msg{************************************************************************}
\Msg{*}
\Msg{* To finish the installation you have to move the following}
\Msg{* file into a directory searched by TeX:}
\Msg{*}
\Msg{*     protecteddef.sty}
\Msg{*}
\Msg{* To produce the documentation run the file `protecteddef.drv'}
\Msg{* through LaTeX.}
\Msg{*}
\Msg{* Happy TeXing!}
\Msg{*}
\Msg{************************************************************************}

\endbatchfile
%</install>
%<*ignore>
\fi
%</ignore>
%<*driver>
\NeedsTeXFormat{LaTeX2e}
\ProvidesFile{protecteddef.drv}%
  [2016/05/16 v1.1 Define protected commands (HO)]%
\documentclass{ltxdoc}
\usepackage{holtxdoc}[2011/11/22]
\begin{document}
  \DocInput{protecteddef.dtx}%
\end{document}
%</driver>
% \fi
%
%
% \CharacterTable
%  {Upper-case    \A\B\C\D\E\F\G\H\I\J\K\L\M\N\O\P\Q\R\S\T\U\V\W\X\Y\Z
%   Lower-case    \a\b\c\d\e\f\g\h\i\j\k\l\m\n\o\p\q\r\s\t\u\v\w\x\y\z
%   Digits        \0\1\2\3\4\5\6\7\8\9
%   Exclamation   \!     Double quote  \"     Hash (number) \#
%   Dollar        \$     Percent       \%     Ampersand     \&
%   Acute accent  \'     Left paren    \(     Right paren   \)
%   Asterisk      \*     Plus          \+     Comma         \,
%   Minus         \-     Point         \.     Solidus       \/
%   Colon         \:     Semicolon     \;     Less than     \<
%   Equals        \=     Greater than  \>     Question mark \?
%   Commercial at \@     Left bracket  \[     Backslash     \\
%   Right bracket \]     Circumflex    \^     Underscore    \_
%   Grave accent  \`     Left brace    \{     Vertical bar  \|
%   Right brace   \}     Tilde         \~}
%
% \GetFileInfo{protecteddef.drv}
%
% \title{The \xpackage{protecteddef} package}
% \date{2016/05/16 v1.1}
% \author{Heiko Oberdiek\thanks
% {Please report any issues at \url{https://github.com/ho-tex/oberdiek/issues}}}
%
% \maketitle
%
% \begin{abstract}
% This packages provides \cs{ProtectedDef} for defining
% robust macros for both \hologo{plainTeX} and \hologo{LaTeX}.
% First \hologo{eTeX}'s \cs{protected} is tried, then
% \hologo{LaTeX}'s \cs{DeclareRobustCommand} is used.
% Otherwise the macro is not made robust.
% \end{abstract}
%
% \tableofcontents
%
% \section{Documentation}
%
% Many of my packages work for both formats \hologo{plainTeX}
% and \hologo{LaTeX}, even \hologo{iniTeX} is often supported.
% It would be nice if fragile macros could be protected and
% made robust. However the different format worlds offer different
% solutions.
%
% \subsection{The \hologo{LaTeX}'s way}
%
% Usually \cs{newcommand} is used to define macros. It provides
% a check if the command to be defined is already defined or
% cannot be defined for other reasons.
%
% For making robust macros \hologo{LaTeX} provides \cs{DeclareRobustCommand}.
% It shares the syntax with \cs{newcommand}. However it does not provide
% latters check. Internally the check is available via \cs{@ifdefinable}.
%
% Internally the robust macro is using \cs{protect} with a nested
% macro definition. The \cs{protect} infrastructure is a feature of
% \cs{LaTeX} and usually not available in other formats.
%
% \subsection{The \hologo{eTeX}'s way}
%
% The need for robust macros is addressed in \cs{eTeX}. It provides
% \cs{protected} that modifies the behaviour of \cs{def} in a similar
% way as \cs{long}. A protected macro does not expand in
% some expandable contexts like writing to a file or \cs{edef}.
%
% \subsection{The way of this package}
%
% The package tries to find the available protection mechanism.
% First it looks for \cs{eTeX}'s \cs{protected}, then
% it uses \hologo{LaTeX}'s \cs{DeclareRobustCommand}.
% If both fails, then the macro remains unprotected.
%
% Additionally, \cs{LaTeX}'s check, if a macro is already defined
% is added in all cases. First \hologo{LaTeX}'s \cs{@ifdefinable}
% is tried to be compatible with \hologo{LaTeX}.
% If \cs{@ifdefinable} is not available, then the test is implemented
% by asserting that the macro is undefined or has the meaning of \cs{relax}.
% If the test fails, then in all cases the macro is not defined
% and an error is thrown.
%
% \subsection{Usage}
%
% \begin{declcs}{ProtectedDef} \* \M{cmd} [\meta{num}] \M{definition text}
% \end{declcs}
% Macro \cs{ProtectedDef} follows the syntax of \hologo{LaTeX}'s
% \cs{newcommand} with the exception that an optional argument
% is not supported. Macro \meta{cmd} is to be defined as
% \cs{long} macro without star with \meta{num} arguments.
%
% The number of arguments \meta{num} must be given as explicite digit
% |0| upto |9|. Otherwise the part between the argument \meta{cmd} and
% the \meta{definition text} is taken as parameter text in the syntax
% of vanilla \hologo{TeX}. Examples (with \cs{protected}):
% \begin{quote}
%   \def\to{$\Rightarrow$ }%
%   |\ProtectedDef*{\foo}[1]{\message{#1}}|\\
%   \to |\protected\def\foo#1{\message#1}}|\\[1ex]
%   |\ProtectedDef\foo{abc}|\\
%   \to |\protected\def\foo{abc}|\\[1ex]
%   |\ProtectedDef*\foo(#1)<#2>{#1/#2}|\\
%   \to |\protected\def\foo(#1)<#2>{#1/#2}|
% \end{quote}
%
% \StopEventually{
% }
%
% \section{Implementation}
%
%    \begin{macrocode}
%<*package>
%    \end{macrocode}
%
% \subsection{Reload check and package identification}
%    Reload check, especially if the package is not used with \LaTeX.
%    \begin{macrocode}
\begingroup\catcode61\catcode48\catcode32=10\relax%
  \catcode13=5 % ^^M
  \endlinechar=13 %
  \catcode35=6 % #
  \catcode39=12 % '
  \catcode44=12 % ,
  \catcode45=12 % -
  \catcode46=12 % .
  \catcode58=12 % :
  \catcode64=11 % @
  \catcode123=1 % {
  \catcode125=2 % }
  \expandafter\let\expandafter\x\csname ver@protecteddef.sty\endcsname
  \ifx\x\relax % plain-TeX, first loading
  \else
    \def\empty{}%
    \ifx\x\empty % LaTeX, first loading,
      % variable is initialized, but \ProvidesPackage not yet seen
    \else
      \expandafter\ifx\csname PackageInfo\endcsname\relax
        \def\x#1#2{%
          \immediate\write-1{Package #1 Info: #2.}%
        }%
      \else
        \def\x#1#2{\PackageInfo{#1}{#2, stopped}}%
      \fi
      \x{protecteddef}{The package is already loaded}%
      \aftergroup\endinput
    \fi
  \fi
\endgroup%
%    \end{macrocode}
%    Package identification:
%    \begin{macrocode}
\begingroup\catcode61\catcode48\catcode32=10\relax%
  \catcode13=5 % ^^M
  \endlinechar=13 %
  \catcode35=6 % #
  \catcode39=12 % '
  \catcode40=12 % (
  \catcode41=12 % )
  \catcode44=12 % ,
  \catcode45=12 % -
  \catcode46=12 % .
  \catcode47=12 % /
  \catcode58=12 % :
  \catcode64=11 % @
  \catcode91=12 % [
  \catcode93=12 % ]
  \catcode123=1 % {
  \catcode125=2 % }
  \expandafter\ifx\csname ProvidesPackage\endcsname\relax
    \def\x#1#2#3[#4]{\endgroup
      \immediate\write-1{Package: #3 #4}%
      \xdef#1{#4}%
    }%
  \else
    \def\x#1#2[#3]{\endgroup
      #2[{#3}]%
      \ifx#1\@undefined
        \xdef#1{#3}%
      \fi
      \ifx#1\relax
        \xdef#1{#3}%
      \fi
    }%
  \fi
\expandafter\x\csname ver@protecteddef.sty\endcsname
\ProvidesPackage{protecteddef}%
  [2016/05/16 v1.1 Define protected commands (HO)]%
%    \end{macrocode}
%
% \subsection{Catcodes}
%
%    \begin{macrocode}
\begingroup\catcode61\catcode48\catcode32=10\relax%
  \catcode13=5 % ^^M
  \endlinechar=13 %
  \catcode123=1 % {
  \catcode125=2 % }
  \catcode64=11 % @
  \def\x{\endgroup
    \expandafter\edef\csname ProDef@AtEnd\endcsname{%
      \endlinechar=\the\endlinechar\relax
      \catcode13=\the\catcode13\relax
      \catcode32=\the\catcode32\relax
      \catcode35=\the\catcode35\relax
      \catcode61=\the\catcode61\relax
      \catcode64=\the\catcode64\relax
      \catcode123=\the\catcode123\relax
      \catcode125=\the\catcode125\relax
    }%
  }%
\x\catcode61\catcode48\catcode32=10\relax%
\catcode13=5 % ^^M
\endlinechar=13 %
\catcode35=6 % #
\catcode64=11 % @
\catcode123=1 % {
\catcode125=2 % }
\def\TMP@EnsureCode#1#2{%
  \edef\ProDef@AtEnd{%
    \ProDef@AtEnd
    \catcode#1=\the\catcode#1\relax
  }%
  \catcode#1=#2\relax
}
\TMP@EnsureCode{38}{4}% &
\TMP@EnsureCode{40}{12}% (
\TMP@EnsureCode{41}{12}% )
\TMP@EnsureCode{42}{12}% *
\TMP@EnsureCode{45}{12}% -
\TMP@EnsureCode{46}{12}% .
\TMP@EnsureCode{47}{12}% /
\TMP@EnsureCode{91}{12}% [
\TMP@EnsureCode{93}{12}% ]
\TMP@EnsureCode{96}{12}% `
\edef\ProDef@AtEnd{\ProDef@AtEnd\noexpand\endinput}
%    \end{macrocode}
%
% \subsection{Resources}
%
%    \begin{macrocode}
\begingroup\expandafter\expandafter\expandafter\endgroup
\expandafter\ifx\csname RequirePackage\endcsname\relax
  \def\TMP@RequirePackage#1[#2]{%
    \begingroup\expandafter\expandafter\expandafter\endgroup
    \expandafter\ifx\csname ver@#1.sty\endcsname\relax
      \input #1.sty\relax
    \fi
  }%
\else
  \let\TMP@RequirePackage\RequirePackage
\fi
\TMP@RequirePackage{ltxcmds}[2010/12/12]%
\TMP@RequirePackage{infwarerr}[2010/04/08]%
%    \end{macrocode}
%
%    \begin{macrocode}
\def\ProDef@temp#1{%
  \expandafter\def\csname ProDef@param[#1]\endcsname % hash-ok
}
\expandafter\def\csname ProDef@param\endcsname{}
\ProDef@temp0{}
\ProDef@temp1{##1}
\ProDef@temp2{##1##2}
\ProDef@temp3{##1##2##3}
\ProDef@temp4{##1##2##3##4}
\ProDef@temp5{##1##2##3##4##5}
\ProDef@temp6{##1##2##3##4##5##6}
\ProDef@temp7{##1##2##3##4##5##7}
\ProDef@temp8{##1##2##3##4##5##7##8}
\ProDef@temp9{##1##2##3##4##5##7##8##9}
%    \end{macrocode}
%
%    \begin{macro}{\ProDef@IfDefinable}
%    \begin{macrocode}
\ltx@IfUndefined{@ifdefinable}{%
  \long\def\ProDef@IfDefinable#1{%
    \begingroup
      \escapechar=-1 %
    \ltx@ifundefined{\string#1}{%
      \endgroup
      \ltx@firstofone
    }{%
      \expandafter\endgroup
      \expandafter
      \edef\expandafter\ProDef@temp\expandafter{\string#1 }%
      \@PackageError{protecteddef}{%
        Command \ltx@backslashchar\ProDef@temp already defined%
      }\@ehc
      \ltx@gobbletwo
    }%
  }%
}{%
  \long\def\ProDef@IfDefinable#1{%
    \let\ProDef@next\ltx@gobbletwo
    \@ifdefinable{#1}{%
      \let\ProDef@next\ltx@firstofone
    }%
    \ProDef@next
  }%
}
%    \end{macrocode}
%    \end{macro}
%
%    \begin{macrocode}
\begingroup\expandafter\expandafter\expandafter\endgroup
\expandafter\ifx\csname protected\endcsname\relax
  \begingroup\expandafter\expandafter\expandafter\endgroup
  \expandafter\ifx\csname DeclareRobustCommand\endcsname\relax
    \catcode`\&=14 % comment
  \else
    \newcommand*{\ProtectedDef}{%
      \ltx@ifnextchar*{%
        \ProDef@ProtectedDef
      }{%
        \ProDef@ProtectedDef{}%
      }%
    }%
    \long\def\ProDef@ProtectedDef#1#2#3#{%
      \ProDef@IfDefinable{#2}{%
        \ltx@IfUndefined{ProDef@param#3}{%
          \DeclareRobustCommand*{#2}{}%
          \begingroup
            \escapechar=-1 %
            \def\ProDef@temp{#1}%
          \edef\x{\endgroup
            \ifx\ProDef@temp\ltx@empty
              \noexpand\long
            \fi
            \noexpand\def
            \expandafter\noexpand\csname\string#2 \endcsname
          }%
          \x#3%
        }{%
          \DeclareRobustCommand#1{#2}#3%
        }%
      }%
    }%
    \expandafter\expandafter\expandafter\ProDef@AtEnd
  \fi
\else
  \catcode`\&=9 % ignore
\fi%
\ProDef@IfDefinable\ProtectedDef{%
&  \protected
  \def\ProtectedDef%
}{%
  \ltx@ifnextchar*{%
    \let\ProDef@long\ltx@empty
    \expandafter\ProDef@ProtectedDef\ltx@gobble
  }{%
    \let\ProDef@long\long
    \ProDef@ProtectedDef
  }%
}
\long\def\ProDef@ProtectedDef#1#2#{%
  \ProDef@IfDefinable{#1}{%
    \ltx@IfUndefined{ProDef@param#2}{%
&     \protected
      \ProDef@long
      \def#1#2%
    }{%
&     \protected
      \ProDef@long
      \expandafter\expandafter\expandafter\def
      \expandafter\expandafter\expandafter#1%
      \csname ProDef@param#2\endcsname
    }%
  }%
}
%    \end{macrocode}
%
%    \begin{macrocode}
\ProDef@AtEnd%
%    \end{macrocode}
%    \begin{macrocode}
%</package>
%    \end{macrocode}
%
% \section{Test}
%
% \subsection{Catcode checks for loading}
%
%    \begin{macrocode}
%<*test1>
%    \end{macrocode}
%    \begin{macrocode}
\catcode`\{=1 %
\catcode`\}=2 %
\catcode`\#=6 %
\catcode`\@=11 %
\expandafter\ifx\csname count@\endcsname\relax
  \countdef\count@=255 %
\fi
\expandafter\ifx\csname @gobble\endcsname\relax
  \long\def\@gobble#1{}%
\fi
\expandafter\ifx\csname @firstofone\endcsname\relax
  \long\def\@firstofone#1{#1}%
\fi
\expandafter\ifx\csname loop\endcsname\relax
  \expandafter\@firstofone
\else
  \expandafter\@gobble
\fi
{%
  \def\loop#1\repeat{%
    \def\body{#1}%
    \iterate
  }%
  \def\iterate{%
    \body
      \let\next\iterate
    \else
      \let\next\relax
    \fi
    \next
  }%
  \let\repeat=\fi
}%
\def\RestoreCatcodes{}
\count@=0 %
\loop
  \edef\RestoreCatcodes{%
    \RestoreCatcodes
    \catcode\the\count@=\the\catcode\count@\relax
  }%
\ifnum\count@<255 %
  \advance\count@ 1 %
\repeat

\def\RangeCatcodeInvalid#1#2{%
  \count@=#1\relax
  \loop
    \catcode\count@=15 %
  \ifnum\count@<#2\relax
    \advance\count@ 1 %
  \repeat
}
\def\RangeCatcodeCheck#1#2#3{%
  \count@=#1\relax
  \loop
    \ifnum#3=\catcode\count@
    \else
      \errmessage{%
        Character \the\count@\space
        with wrong catcode \the\catcode\count@\space
        instead of \number#3%
      }%
    \fi
  \ifnum\count@<#2\relax
    \advance\count@ 1 %
  \repeat
}
\def\space{ }
\expandafter\ifx\csname LoadCommand\endcsname\relax
  \def\LoadCommand{\input protecteddef.sty\relax}%
\fi
\def\Test{%
  \RangeCatcodeInvalid{0}{47}%
  \RangeCatcodeInvalid{58}{64}%
  \RangeCatcodeInvalid{91}{96}%
  \RangeCatcodeInvalid{123}{255}%
  \catcode`\@=12 %
  \catcode`\\=0 %
  \catcode`\%=14 %
  \LoadCommand
  \RangeCatcodeCheck{0}{36}{15}%
  \RangeCatcodeCheck{37}{37}{14}%
  \RangeCatcodeCheck{38}{47}{15}%
  \RangeCatcodeCheck{48}{57}{12}%
  \RangeCatcodeCheck{58}{63}{15}%
  \RangeCatcodeCheck{64}{64}{12}%
  \RangeCatcodeCheck{65}{90}{11}%
  \RangeCatcodeCheck{91}{91}{15}%
  \RangeCatcodeCheck{92}{92}{0}%
  \RangeCatcodeCheck{93}{96}{15}%
  \RangeCatcodeCheck{97}{122}{11}%
  \RangeCatcodeCheck{123}{255}{15}%
  \RestoreCatcodes
}
\Test
\csname @@end\endcsname
\end
%    \end{macrocode}
%    \begin{macrocode}
%</test1>
%    \end{macrocode}
%
% \subsection{Test without \hologo{LaTeX} and \cs{protected}}
%
%    \begin{macrocode}
%<*test2>
\errorcontextlines=10000 %
\begingroup\expandafter\expandafter\expandafter\endgroup
\expandafter\ifx\csname RequirePackage\endcsname\relax
  \input protecteddef.sty\relax
  \catcode`\{=1 %
  \catcode`\}=2 %
  \catcode`\#=6 %
\else
  \RequirePackage{protecteddef}[2016/05/16]%
\fi
\begingroup\expandafter\expandafter\expandafter\endgroup
\expandafter\ifx\csname protected\endcsname\relax
  \let\pdef\def
\else
  \def\pdef{\protected\def}%
\fi
\def\msg#{\immediate\write16}
\countdef\errcount=2 %
\long\def\BeginCheck#1\ProtectedDef#2\EndCheck{%
  \begingroup
    \toks0={\ProtectedDef#2}%
    \msg{<<\the\toks0>>}%
  \endgroup
  \setbox0=\hbox{%
    #1%
    \ProtectedDef#2%
    \check\foo
  }%
  \ifdim\wd0=0pt\relax
  \else
    \errmessage{[Definition] Unwanted spaces?!}%
  \fi
  \setbox0=\hbox{%
    \def\fooinitial{XYZ}%
    \let\foo\fooinitial
    \errcount=0 %
    \expandafter\def\csname @PackageError\endcsname##1##2##3{%
      \advance\errcount by 1 %
    }%
    \expandafter\def\csname @notdefinable\endcsname{%
      \advance\errcount by 1 %
    }%
    \ProtectedDef#2%
    \ifnum\errcount=1 %
    \else
      \errmessage{1 error expected, but found: \the\errcount}%
    \fi
    \ifx\foo\fooinitial
    \else
      \def\space{ }%
      \errmessage{\string\foo\space is overwritten}%
    \fi
  }%
  \ifdim\wd0=0pt\relax
  \else
    \errmessage{[Error] Unwanted spaces?!}%
  \fi
}
\chardef\DeclareVersion=0 %
\begingroup\expandafter\expandafter\expandafter\endgroup
\expandafter\ifx\csname protected\endcsname\relax
  \begingroup\expandafter\expandafter\expandafter\endgroup
  \expandafter\ifx\csname DeclareRobustCommand\endcsname\relax
  \else
    \chardef\DeclareVersion=1 %
  \fi
\fi
\ifnum\DeclareVersion=0 %
  \def\check#1{%
    \ifx\cmp#1%
      \msg{* Test passed.}%
    \else
      \msg{}%
      \msg{[\meaning#1]}%
      \msg{[\meaning\cmp]}%
      \errmessage{Test failed!}%
    \fi
  }%
\else
  \def\check#1{%
    \begingroup
      \escapechar=-1 %
    \edef\x{\endgroup
      \def\noexpand\cs/{\string#1}%
    }\x
    \edef\CMP{%
      \noexpand\protect
      \expandafter\noexpand\csname\cs/ \endcsname
    }%
    \ifx\CMP#1%
      \expandafter\ifx\csname\cs/ \endcsname\cmp
        \msg{Test passed.}%
      \else
        \msg{}%
        \msg{[\expandafter\meaning\csname\cs/ \endcsname]}%
        \msg{[\meaning\cmp]}%
        \errmessage{Test failed!}%
      \fi
    \else
      \msg{}%
      \msg{[\meaning#1]}%
      \msg{[\meaning\CMP]}%
      \errmessage{Test failed!}%
    \fi
  }%
\fi

\tracingmacros=1

\BeginCheck
  \pdef\cmp{}%
  \ProtectedDef*\foo{}%
\EndCheck

\BeginCheck
  \pdef\cmp{}%
  \ProtectedDef*\foo[0]{}%
\EndCheck

\BeginCheck
  \pdef\cmp#1{<#1>}%
  \ProtectedDef*\foo[1]{<#1>}%
\EndCheck

\BeginCheck
  \pdef\cmp(#1){<#1>}%
  \ProtectedDef*\foo(#1){<#1>}%
\EndCheck

\BeginCheck
  \long\pdef\cmp{}%
  \ProtectedDef\foo{}%
\EndCheck

\BeginCheck
  \long\pdef\cmp{}%
  \ProtectedDef\foo[0]{}%
\EndCheck

\BeginCheck
  \long\pdef\cmp#1{<#1>}%
  \ProtectedDef\foo[1]{<#1>}%
\EndCheck

\BeginCheck
  \long\pdef\cmp(#1){<#1>}%
  \ProtectedDef\foo(#1){<#1>}%
\EndCheck

\csname @@end\endcsname\end
%</test2>
%    \end{macrocode}
%
% \section{Installation}
%
% \subsection{Download}
%
% \paragraph{Package.} This package is available on
% CTAN\footnote{\CTANpkg{protecteddef}}:
% \begin{description}
% \item[\CTAN{macros/latex/contrib/oberdiek/protecteddef.dtx}] The source file.
% \item[\CTAN{macros/latex/contrib/oberdiek/protecteddef.pdf}] Documentation.
% \end{description}
%
%
% \paragraph{Bundle.} All the packages of the bundle `oberdiek'
% are also available in a TDS compliant ZIP archive. There
% the packages are already unpacked and the documentation files
% are generated. The files and directories obey the TDS standard.
% \begin{description}
% \item[\CTANinstall{install/macros/latex/contrib/oberdiek.tds.zip}]
% \end{description}
% \emph{TDS} refers to the standard ``A Directory Structure
% for \TeX\ Files'' (\CTAN{tds/tds.pdf}). Directories
% with \xfile{texmf} in their name are usually organized this way.
%
% \subsection{Bundle installation}
%
% \paragraph{Unpacking.} Unpack the \xfile{oberdiek.tds.zip} in the
% TDS tree (also known as \xfile{texmf} tree) of your choice.
% Example (linux):
% \begin{quote}
%   |unzip oberdiek.tds.zip -d ~/texmf|
% \end{quote}
%
% \paragraph{Script installation.}
% Check the directory \xfile{TDS:scripts/oberdiek/} for
% scripts that need further installation steps.
%
% \subsection{Package installation}
%
% \paragraph{Unpacking.} The \xfile{.dtx} file is a self-extracting
% \docstrip\ archive. The files are extracted by running the
% \xfile{.dtx} through \plainTeX:
% \begin{quote}
%   \verb|tex protecteddef.dtx|
% \end{quote}
%
% \paragraph{TDS.} Now the different files must be moved into
% the different directories in your installation TDS tree
% (also known as \xfile{texmf} tree):
% \begin{quote}
% \def\t{^^A
% \begin{tabular}{@{}>{\ttfamily}l@{ $\rightarrow$ }>{\ttfamily}l@{}}
%   protecteddef.sty & tex/generic/oberdiek/protecteddef.sty\\
%   protecteddef.pdf & doc/latex/oberdiek/protecteddef.pdf\\
%   test/protecteddef-test1.tex & doc/latex/oberdiek/test/protecteddef-test1.tex\\
%   test/protecteddef-test2.tex & doc/latex/oberdiek/test/protecteddef-test2.tex\\
%   protecteddef.dtx & source/latex/oberdiek/protecteddef.dtx\\
% \end{tabular}^^A
% }^^A
% \sbox0{\t}^^A
% \ifdim\wd0>\linewidth
%   \begingroup
%     \advance\linewidth by\leftmargin
%     \advance\linewidth by\rightmargin
%   \edef\x{\endgroup
%     \def\noexpand\lw{\the\linewidth}^^A
%   }\x
%   \def\lwbox{^^A
%     \leavevmode
%     \hbox to \linewidth{^^A
%       \kern-\leftmargin\relax
%       \hss
%       \usebox0
%       \hss
%       \kern-\rightmargin\relax
%     }^^A
%   }^^A
%   \ifdim\wd0>\lw
%     \sbox0{\small\t}^^A
%     \ifdim\wd0>\linewidth
%       \ifdim\wd0>\lw
%         \sbox0{\footnotesize\t}^^A
%         \ifdim\wd0>\linewidth
%           \ifdim\wd0>\lw
%             \sbox0{\scriptsize\t}^^A
%             \ifdim\wd0>\linewidth
%               \ifdim\wd0>\lw
%                 \sbox0{\tiny\t}^^A
%                 \ifdim\wd0>\linewidth
%                   \lwbox
%                 \else
%                   \usebox0
%                 \fi
%               \else
%                 \lwbox
%               \fi
%             \else
%               \usebox0
%             \fi
%           \else
%             \lwbox
%           \fi
%         \else
%           \usebox0
%         \fi
%       \else
%         \lwbox
%       \fi
%     \else
%       \usebox0
%     \fi
%   \else
%     \lwbox
%   \fi
% \else
%   \usebox0
% \fi
% \end{quote}
% If you have a \xfile{docstrip.cfg} that configures and enables \docstrip's
% TDS installing feature, then some files can already be in the right
% place, see the documentation of \docstrip.
%
% \subsection{Refresh file name databases}
%
% If your \TeX~distribution
% (\TeX\,Live, \mikTeX, \dots) relies on file name databases, you must refresh
% these. For example, \TeX\,Live\ users run \verb|texhash| or
% \verb|mktexlsr|.
%
% \subsection{Some details for the interested}
%
% \paragraph{Unpacking with \LaTeX.}
% The \xfile{.dtx} chooses its action depending on the format:
% \begin{description}
% \item[\plainTeX:] Run \docstrip\ and extract the files.
% \item[\LaTeX:] Generate the documentation.
% \end{description}
% If you insist on using \LaTeX\ for \docstrip\ (really,
% \docstrip\ does not need \LaTeX), then inform the autodetect routine
% about your intention:
% \begin{quote}
%   \verb|latex \let\install=y% \iffalse meta-comment
%
% File: protecteddef.dtx
% Version: 2016/05/16 v1.1
% Info: Define protected commands
%
% Copyright (C)
%    2011 Heiko Oberdiek
%    2016-2019 Oberdiek Package Support Group
%    https://github.com/ho-tex/oberdiek/issues
%
% This work may be distributed and/or modified under the
% conditions of the LaTeX Project Public License, either
% version 1.3c of this license or (at your option) any later
% version. This version of this license is in
%    https://www.latex-project.org/lppl/lppl-1-3c.txt
% and the latest version of this license is in
%    https://www.latex-project.org/lppl.txt
% and version 1.3 or later is part of all distributions of
% LaTeX version 2005/12/01 or later.
%
% This work has the LPPL maintenance status "maintained".
%
% The Current Maintainers of this work are
% Heiko Oberdiek and the Oberdiek Package Support Group
% https://github.com/ho-tex/oberdiek/issues
%
% The Base Interpreter refers to any `TeX-Format',
% because some files are installed in TDS:tex/generic//.
%
% This work consists of the main source file protecteddef.dtx
% and the derived files
%    protecteddef.sty, protecteddef.pdf, protecteddef.ins, protecteddef.drv,
%    protecteddef-test1.tex, protecteddef-test2.tex.
%
% Distribution:
%    CTAN:macros/latex/contrib/oberdiek/protecteddef.dtx
%    CTAN:macros/latex/contrib/oberdiek/protecteddef.pdf
%
% Unpacking:
%    (a) If protecteddef.ins is present:
%           tex protecteddef.ins
%    (b) Without protecteddef.ins:
%           tex protecteddef.dtx
%    (c) If you insist on using LaTeX
%           latex \let\install=y\input{protecteddef.dtx}
%        (quote the arguments according to the demands of your shell)
%
% Documentation:
%    (a) If protecteddef.drv is present:
%           latex protecteddef.drv
%    (b) Without protecteddef.drv:
%           latex protecteddef.dtx; ...
%    The class ltxdoc loads the configuration file ltxdoc.cfg
%    if available. Here you can specify further options, e.g.
%    use A4 as paper format:
%       \PassOptionsToClass{a4paper}{article}
%
%    Programm calls to get the documentation (example):
%       pdflatex protecteddef.dtx
%       makeindex -s gind.ist protecteddef.idx
%       pdflatex protecteddef.dtx
%       makeindex -s gind.ist protecteddef.idx
%       pdflatex protecteddef.dtx
%
% Installation:
%    TDS:tex/generic/oberdiek/protecteddef.sty
%    TDS:doc/latex/oberdiek/protecteddef.pdf
%    TDS:doc/latex/oberdiek/test/protecteddef-test1.tex
%    TDS:doc/latex/oberdiek/test/protecteddef-test2.tex
%    TDS:source/latex/oberdiek/protecteddef.dtx
%
%<*ignore>
\begingroup
  \catcode123=1 %
  \catcode125=2 %
  \def\x{LaTeX2e}%
\expandafter\endgroup
\ifcase 0\ifx\install y1\fi\expandafter
         \ifx\csname processbatchFile\endcsname\relax\else1\fi
         \ifx\fmtname\x\else 1\fi\relax
\else\csname fi\endcsname
%</ignore>
%<*install>
\input docstrip.tex
\Msg{************************************************************************}
\Msg{* Installation}
\Msg{* Package: protecteddef 2016/05/16 v1.1 Define protected commands (HO)}
\Msg{************************************************************************}

\keepsilent
\askforoverwritefalse

\let\MetaPrefix\relax
\preamble

This is a generated file.

Project: protecteddef
Version: 2016/05/16 v1.1

Copyright (C)
   2011 Heiko Oberdiek
   2016-2019 Oberdiek Package Support Group

This work may be distributed and/or modified under the
conditions of the LaTeX Project Public License, either
version 1.3c of this license or (at your option) any later
version. This version of this license is in
   https://www.latex-project.org/lppl/lppl-1-3c.txt
and the latest version of this license is in
   https://www.latex-project.org/lppl.txt
and version 1.3 or later is part of all distributions of
LaTeX version 2005/12/01 or later.

This work has the LPPL maintenance status "maintained".

The Current Maintainers of this work are
Heiko Oberdiek and the Oberdiek Package Support Group
https://github.com/ho-tex/oberdiek/issues


The Base Interpreter refers to any `TeX-Format',
because some files are installed in TDS:tex/generic//.

This work consists of the main source file protecteddef.dtx
and the derived files
   protecteddef.sty, protecteddef.pdf, protecteddef.ins, protecteddef.drv,
   protecteddef-test1.tex, protecteddef-test2.tex.

\endpreamble
\let\MetaPrefix\DoubleperCent

\generate{%
  \file{protecteddef.ins}{\from{protecteddef.dtx}{install}}%
  \file{protecteddef.drv}{\from{protecteddef.dtx}{driver}}%
  \usedir{tex/generic/oberdiek}%
  \file{protecteddef.sty}{\from{protecteddef.dtx}{package}}%
%  \usedir{doc/latex/oberdiek/test}%
%  \file{protecteddef-test1.tex}{\from{protecteddef.dtx}{test1}}%
%  \file{protecteddef-test2.tex}{\from{protecteddef.dtx}{test2}}%
  \nopreamble
  \nopostamble
%  \usedir{source/latex/oberdiek/catalogue}%
%  \file{protecteddef.xml}{\from{protecteddef.dtx}{catalogue}}%
}

\catcode32=13\relax% active space
\let =\space%
\Msg{************************************************************************}
\Msg{*}
\Msg{* To finish the installation you have to move the following}
\Msg{* file into a directory searched by TeX:}
\Msg{*}
\Msg{*     protecteddef.sty}
\Msg{*}
\Msg{* To produce the documentation run the file `protecteddef.drv'}
\Msg{* through LaTeX.}
\Msg{*}
\Msg{* Happy TeXing!}
\Msg{*}
\Msg{************************************************************************}

\endbatchfile
%</install>
%<*ignore>
\fi
%</ignore>
%<*driver>
\NeedsTeXFormat{LaTeX2e}
\ProvidesFile{protecteddef.drv}%
  [2016/05/16 v1.1 Define protected commands (HO)]%
\documentclass{ltxdoc}
\usepackage{holtxdoc}[2011/11/22]
\begin{document}
  \DocInput{protecteddef.dtx}%
\end{document}
%</driver>
% \fi
%
%
% \CharacterTable
%  {Upper-case    \A\B\C\D\E\F\G\H\I\J\K\L\M\N\O\P\Q\R\S\T\U\V\W\X\Y\Z
%   Lower-case    \a\b\c\d\e\f\g\h\i\j\k\l\m\n\o\p\q\r\s\t\u\v\w\x\y\z
%   Digits        \0\1\2\3\4\5\6\7\8\9
%   Exclamation   \!     Double quote  \"     Hash (number) \#
%   Dollar        \$     Percent       \%     Ampersand     \&
%   Acute accent  \'     Left paren    \(     Right paren   \)
%   Asterisk      \*     Plus          \+     Comma         \,
%   Minus         \-     Point         \.     Solidus       \/
%   Colon         \:     Semicolon     \;     Less than     \<
%   Equals        \=     Greater than  \>     Question mark \?
%   Commercial at \@     Left bracket  \[     Backslash     \\
%   Right bracket \]     Circumflex    \^     Underscore    \_
%   Grave accent  \`     Left brace    \{     Vertical bar  \|
%   Right brace   \}     Tilde         \~}
%
% \GetFileInfo{protecteddef.drv}
%
% \title{The \xpackage{protecteddef} package}
% \date{2016/05/16 v1.1}
% \author{Heiko Oberdiek\thanks
% {Please report any issues at \url{https://github.com/ho-tex/oberdiek/issues}}}
%
% \maketitle
%
% \begin{abstract}
% This packages provides \cs{ProtectedDef} for defining
% robust macros for both \hologo{plainTeX} and \hologo{LaTeX}.
% First \hologo{eTeX}'s \cs{protected} is tried, then
% \hologo{LaTeX}'s \cs{DeclareRobustCommand} is used.
% Otherwise the macro is not made robust.
% \end{abstract}
%
% \tableofcontents
%
% \section{Documentation}
%
% Many of my packages work for both formats \hologo{plainTeX}
% and \hologo{LaTeX}, even \hologo{iniTeX} is often supported.
% It would be nice if fragile macros could be protected and
% made robust. However the different format worlds offer different
% solutions.
%
% \subsection{The \hologo{LaTeX}'s way}
%
% Usually \cs{newcommand} is used to define macros. It provides
% a check if the command to be defined is already defined or
% cannot be defined for other reasons.
%
% For making robust macros \hologo{LaTeX} provides \cs{DeclareRobustCommand}.
% It shares the syntax with \cs{newcommand}. However it does not provide
% latters check. Internally the check is available via \cs{@ifdefinable}.
%
% Internally the robust macro is using \cs{protect} with a nested
% macro definition. The \cs{protect} infrastructure is a feature of
% \cs{LaTeX} and usually not available in other formats.
%
% \subsection{The \hologo{eTeX}'s way}
%
% The need for robust macros is addressed in \cs{eTeX}. It provides
% \cs{protected} that modifies the behaviour of \cs{def} in a similar
% way as \cs{long}. A protected macro does not expand in
% some expandable contexts like writing to a file or \cs{edef}.
%
% \subsection{The way of this package}
%
% The package tries to find the available protection mechanism.
% First it looks for \cs{eTeX}'s \cs{protected}, then
% it uses \hologo{LaTeX}'s \cs{DeclareRobustCommand}.
% If both fails, then the macro remains unprotected.
%
% Additionally, \cs{LaTeX}'s check, if a macro is already defined
% is added in all cases. First \hologo{LaTeX}'s \cs{@ifdefinable}
% is tried to be compatible with \hologo{LaTeX}.
% If \cs{@ifdefinable} is not available, then the test is implemented
% by asserting that the macro is undefined or has the meaning of \cs{relax}.
% If the test fails, then in all cases the macro is not defined
% and an error is thrown.
%
% \subsection{Usage}
%
% \begin{declcs}{ProtectedDef} \* \M{cmd} [\meta{num}] \M{definition text}
% \end{declcs}
% Macro \cs{ProtectedDef} follows the syntax of \hologo{LaTeX}'s
% \cs{newcommand} with the exception that an optional argument
% is not supported. Macro \meta{cmd} is to be defined as
% \cs{long} macro without star with \meta{num} arguments.
%
% The number of arguments \meta{num} must be given as explicite digit
% |0| upto |9|. Otherwise the part between the argument \meta{cmd} and
% the \meta{definition text} is taken as parameter text in the syntax
% of vanilla \hologo{TeX}. Examples (with \cs{protected}):
% \begin{quote}
%   \def\to{$\Rightarrow$ }%
%   |\ProtectedDef*{\foo}[1]{\message{#1}}|\\
%   \to |\protected\def\foo#1{\message#1}}|\\[1ex]
%   |\ProtectedDef\foo{abc}|\\
%   \to |\protected\def\foo{abc}|\\[1ex]
%   |\ProtectedDef*\foo(#1)<#2>{#1/#2}|\\
%   \to |\protected\def\foo(#1)<#2>{#1/#2}|
% \end{quote}
%
% \StopEventually{
% }
%
% \section{Implementation}
%
%    \begin{macrocode}
%<*package>
%    \end{macrocode}
%
% \subsection{Reload check and package identification}
%    Reload check, especially if the package is not used with \LaTeX.
%    \begin{macrocode}
\begingroup\catcode61\catcode48\catcode32=10\relax%
  \catcode13=5 % ^^M
  \endlinechar=13 %
  \catcode35=6 % #
  \catcode39=12 % '
  \catcode44=12 % ,
  \catcode45=12 % -
  \catcode46=12 % .
  \catcode58=12 % :
  \catcode64=11 % @
  \catcode123=1 % {
  \catcode125=2 % }
  \expandafter\let\expandafter\x\csname ver@protecteddef.sty\endcsname
  \ifx\x\relax % plain-TeX, first loading
  \else
    \def\empty{}%
    \ifx\x\empty % LaTeX, first loading,
      % variable is initialized, but \ProvidesPackage not yet seen
    \else
      \expandafter\ifx\csname PackageInfo\endcsname\relax
        \def\x#1#2{%
          \immediate\write-1{Package #1 Info: #2.}%
        }%
      \else
        \def\x#1#2{\PackageInfo{#1}{#2, stopped}}%
      \fi
      \x{protecteddef}{The package is already loaded}%
      \aftergroup\endinput
    \fi
  \fi
\endgroup%
%    \end{macrocode}
%    Package identification:
%    \begin{macrocode}
\begingroup\catcode61\catcode48\catcode32=10\relax%
  \catcode13=5 % ^^M
  \endlinechar=13 %
  \catcode35=6 % #
  \catcode39=12 % '
  \catcode40=12 % (
  \catcode41=12 % )
  \catcode44=12 % ,
  \catcode45=12 % -
  \catcode46=12 % .
  \catcode47=12 % /
  \catcode58=12 % :
  \catcode64=11 % @
  \catcode91=12 % [
  \catcode93=12 % ]
  \catcode123=1 % {
  \catcode125=2 % }
  \expandafter\ifx\csname ProvidesPackage\endcsname\relax
    \def\x#1#2#3[#4]{\endgroup
      \immediate\write-1{Package: #3 #4}%
      \xdef#1{#4}%
    }%
  \else
    \def\x#1#2[#3]{\endgroup
      #2[{#3}]%
      \ifx#1\@undefined
        \xdef#1{#3}%
      \fi
      \ifx#1\relax
        \xdef#1{#3}%
      \fi
    }%
  \fi
\expandafter\x\csname ver@protecteddef.sty\endcsname
\ProvidesPackage{protecteddef}%
  [2016/05/16 v1.1 Define protected commands (HO)]%
%    \end{macrocode}
%
% \subsection{Catcodes}
%
%    \begin{macrocode}
\begingroup\catcode61\catcode48\catcode32=10\relax%
  \catcode13=5 % ^^M
  \endlinechar=13 %
  \catcode123=1 % {
  \catcode125=2 % }
  \catcode64=11 % @
  \def\x{\endgroup
    \expandafter\edef\csname ProDef@AtEnd\endcsname{%
      \endlinechar=\the\endlinechar\relax
      \catcode13=\the\catcode13\relax
      \catcode32=\the\catcode32\relax
      \catcode35=\the\catcode35\relax
      \catcode61=\the\catcode61\relax
      \catcode64=\the\catcode64\relax
      \catcode123=\the\catcode123\relax
      \catcode125=\the\catcode125\relax
    }%
  }%
\x\catcode61\catcode48\catcode32=10\relax%
\catcode13=5 % ^^M
\endlinechar=13 %
\catcode35=6 % #
\catcode64=11 % @
\catcode123=1 % {
\catcode125=2 % }
\def\TMP@EnsureCode#1#2{%
  \edef\ProDef@AtEnd{%
    \ProDef@AtEnd
    \catcode#1=\the\catcode#1\relax
  }%
  \catcode#1=#2\relax
}
\TMP@EnsureCode{38}{4}% &
\TMP@EnsureCode{40}{12}% (
\TMP@EnsureCode{41}{12}% )
\TMP@EnsureCode{42}{12}% *
\TMP@EnsureCode{45}{12}% -
\TMP@EnsureCode{46}{12}% .
\TMP@EnsureCode{47}{12}% /
\TMP@EnsureCode{91}{12}% [
\TMP@EnsureCode{93}{12}% ]
\TMP@EnsureCode{96}{12}% `
\edef\ProDef@AtEnd{\ProDef@AtEnd\noexpand\endinput}
%    \end{macrocode}
%
% \subsection{Resources}
%
%    \begin{macrocode}
\begingroup\expandafter\expandafter\expandafter\endgroup
\expandafter\ifx\csname RequirePackage\endcsname\relax
  \def\TMP@RequirePackage#1[#2]{%
    \begingroup\expandafter\expandafter\expandafter\endgroup
    \expandafter\ifx\csname ver@#1.sty\endcsname\relax
      \input #1.sty\relax
    \fi
  }%
\else
  \let\TMP@RequirePackage\RequirePackage
\fi
\TMP@RequirePackage{ltxcmds}[2010/12/12]%
\TMP@RequirePackage{infwarerr}[2010/04/08]%
%    \end{macrocode}
%
%    \begin{macrocode}
\def\ProDef@temp#1{%
  \expandafter\def\csname ProDef@param[#1]\endcsname % hash-ok
}
\expandafter\def\csname ProDef@param\endcsname{}
\ProDef@temp0{}
\ProDef@temp1{##1}
\ProDef@temp2{##1##2}
\ProDef@temp3{##1##2##3}
\ProDef@temp4{##1##2##3##4}
\ProDef@temp5{##1##2##3##4##5}
\ProDef@temp6{##1##2##3##4##5##6}
\ProDef@temp7{##1##2##3##4##5##7}
\ProDef@temp8{##1##2##3##4##5##7##8}
\ProDef@temp9{##1##2##3##4##5##7##8##9}
%    \end{macrocode}
%
%    \begin{macro}{\ProDef@IfDefinable}
%    \begin{macrocode}
\ltx@IfUndefined{@ifdefinable}{%
  \long\def\ProDef@IfDefinable#1{%
    \begingroup
      \escapechar=-1 %
    \ltx@ifundefined{\string#1}{%
      \endgroup
      \ltx@firstofone
    }{%
      \expandafter\endgroup
      \expandafter
      \edef\expandafter\ProDef@temp\expandafter{\string#1 }%
      \@PackageError{protecteddef}{%
        Command \ltx@backslashchar\ProDef@temp already defined%
      }\@ehc
      \ltx@gobbletwo
    }%
  }%
}{%
  \long\def\ProDef@IfDefinable#1{%
    \let\ProDef@next\ltx@gobbletwo
    \@ifdefinable{#1}{%
      \let\ProDef@next\ltx@firstofone
    }%
    \ProDef@next
  }%
}
%    \end{macrocode}
%    \end{macro}
%
%    \begin{macrocode}
\begingroup\expandafter\expandafter\expandafter\endgroup
\expandafter\ifx\csname protected\endcsname\relax
  \begingroup\expandafter\expandafter\expandafter\endgroup
  \expandafter\ifx\csname DeclareRobustCommand\endcsname\relax
    \catcode`\&=14 % comment
  \else
    \newcommand*{\ProtectedDef}{%
      \ltx@ifnextchar*{%
        \ProDef@ProtectedDef
      }{%
        \ProDef@ProtectedDef{}%
      }%
    }%
    \long\def\ProDef@ProtectedDef#1#2#3#{%
      \ProDef@IfDefinable{#2}{%
        \ltx@IfUndefined{ProDef@param#3}{%
          \DeclareRobustCommand*{#2}{}%
          \begingroup
            \escapechar=-1 %
            \def\ProDef@temp{#1}%
          \edef\x{\endgroup
            \ifx\ProDef@temp\ltx@empty
              \noexpand\long
            \fi
            \noexpand\def
            \expandafter\noexpand\csname\string#2 \endcsname
          }%
          \x#3%
        }{%
          \DeclareRobustCommand#1{#2}#3%
        }%
      }%
    }%
    \expandafter\expandafter\expandafter\ProDef@AtEnd
  \fi
\else
  \catcode`\&=9 % ignore
\fi%
\ProDef@IfDefinable\ProtectedDef{%
&  \protected
  \def\ProtectedDef%
}{%
  \ltx@ifnextchar*{%
    \let\ProDef@long\ltx@empty
    \expandafter\ProDef@ProtectedDef\ltx@gobble
  }{%
    \let\ProDef@long\long
    \ProDef@ProtectedDef
  }%
}
\long\def\ProDef@ProtectedDef#1#2#{%
  \ProDef@IfDefinable{#1}{%
    \ltx@IfUndefined{ProDef@param#2}{%
&     \protected
      \ProDef@long
      \def#1#2%
    }{%
&     \protected
      \ProDef@long
      \expandafter\expandafter\expandafter\def
      \expandafter\expandafter\expandafter#1%
      \csname ProDef@param#2\endcsname
    }%
  }%
}
%    \end{macrocode}
%
%    \begin{macrocode}
\ProDef@AtEnd%
%    \end{macrocode}
%    \begin{macrocode}
%</package>
%    \end{macrocode}
%
% \section{Test}
%
% \subsection{Catcode checks for loading}
%
%    \begin{macrocode}
%<*test1>
%    \end{macrocode}
%    \begin{macrocode}
\catcode`\{=1 %
\catcode`\}=2 %
\catcode`\#=6 %
\catcode`\@=11 %
\expandafter\ifx\csname count@\endcsname\relax
  \countdef\count@=255 %
\fi
\expandafter\ifx\csname @gobble\endcsname\relax
  \long\def\@gobble#1{}%
\fi
\expandafter\ifx\csname @firstofone\endcsname\relax
  \long\def\@firstofone#1{#1}%
\fi
\expandafter\ifx\csname loop\endcsname\relax
  \expandafter\@firstofone
\else
  \expandafter\@gobble
\fi
{%
  \def\loop#1\repeat{%
    \def\body{#1}%
    \iterate
  }%
  \def\iterate{%
    \body
      \let\next\iterate
    \else
      \let\next\relax
    \fi
    \next
  }%
  \let\repeat=\fi
}%
\def\RestoreCatcodes{}
\count@=0 %
\loop
  \edef\RestoreCatcodes{%
    \RestoreCatcodes
    \catcode\the\count@=\the\catcode\count@\relax
  }%
\ifnum\count@<255 %
  \advance\count@ 1 %
\repeat

\def\RangeCatcodeInvalid#1#2{%
  \count@=#1\relax
  \loop
    \catcode\count@=15 %
  \ifnum\count@<#2\relax
    \advance\count@ 1 %
  \repeat
}
\def\RangeCatcodeCheck#1#2#3{%
  \count@=#1\relax
  \loop
    \ifnum#3=\catcode\count@
    \else
      \errmessage{%
        Character \the\count@\space
        with wrong catcode \the\catcode\count@\space
        instead of \number#3%
      }%
    \fi
  \ifnum\count@<#2\relax
    \advance\count@ 1 %
  \repeat
}
\def\space{ }
\expandafter\ifx\csname LoadCommand\endcsname\relax
  \def\LoadCommand{\input protecteddef.sty\relax}%
\fi
\def\Test{%
  \RangeCatcodeInvalid{0}{47}%
  \RangeCatcodeInvalid{58}{64}%
  \RangeCatcodeInvalid{91}{96}%
  \RangeCatcodeInvalid{123}{255}%
  \catcode`\@=12 %
  \catcode`\\=0 %
  \catcode`\%=14 %
  \LoadCommand
  \RangeCatcodeCheck{0}{36}{15}%
  \RangeCatcodeCheck{37}{37}{14}%
  \RangeCatcodeCheck{38}{47}{15}%
  \RangeCatcodeCheck{48}{57}{12}%
  \RangeCatcodeCheck{58}{63}{15}%
  \RangeCatcodeCheck{64}{64}{12}%
  \RangeCatcodeCheck{65}{90}{11}%
  \RangeCatcodeCheck{91}{91}{15}%
  \RangeCatcodeCheck{92}{92}{0}%
  \RangeCatcodeCheck{93}{96}{15}%
  \RangeCatcodeCheck{97}{122}{11}%
  \RangeCatcodeCheck{123}{255}{15}%
  \RestoreCatcodes
}
\Test
\csname @@end\endcsname
\end
%    \end{macrocode}
%    \begin{macrocode}
%</test1>
%    \end{macrocode}
%
% \subsection{Test without \hologo{LaTeX} and \cs{protected}}
%
%    \begin{macrocode}
%<*test2>
\errorcontextlines=10000 %
\begingroup\expandafter\expandafter\expandafter\endgroup
\expandafter\ifx\csname RequirePackage\endcsname\relax
  \input protecteddef.sty\relax
  \catcode`\{=1 %
  \catcode`\}=2 %
  \catcode`\#=6 %
\else
  \RequirePackage{protecteddef}[2016/05/16]%
\fi
\begingroup\expandafter\expandafter\expandafter\endgroup
\expandafter\ifx\csname protected\endcsname\relax
  \let\pdef\def
\else
  \def\pdef{\protected\def}%
\fi
\def\msg#{\immediate\write16}
\countdef\errcount=2 %
\long\def\BeginCheck#1\ProtectedDef#2\EndCheck{%
  \begingroup
    \toks0={\ProtectedDef#2}%
    \msg{<<\the\toks0>>}%
  \endgroup
  \setbox0=\hbox{%
    #1%
    \ProtectedDef#2%
    \check\foo
  }%
  \ifdim\wd0=0pt\relax
  \else
    \errmessage{[Definition] Unwanted spaces?!}%
  \fi
  \setbox0=\hbox{%
    \def\fooinitial{XYZ}%
    \let\foo\fooinitial
    \errcount=0 %
    \expandafter\def\csname @PackageError\endcsname##1##2##3{%
      \advance\errcount by 1 %
    }%
    \expandafter\def\csname @notdefinable\endcsname{%
      \advance\errcount by 1 %
    }%
    \ProtectedDef#2%
    \ifnum\errcount=1 %
    \else
      \errmessage{1 error expected, but found: \the\errcount}%
    \fi
    \ifx\foo\fooinitial
    \else
      \def\space{ }%
      \errmessage{\string\foo\space is overwritten}%
    \fi
  }%
  \ifdim\wd0=0pt\relax
  \else
    \errmessage{[Error] Unwanted spaces?!}%
  \fi
}
\chardef\DeclareVersion=0 %
\begingroup\expandafter\expandafter\expandafter\endgroup
\expandafter\ifx\csname protected\endcsname\relax
  \begingroup\expandafter\expandafter\expandafter\endgroup
  \expandafter\ifx\csname DeclareRobustCommand\endcsname\relax
  \else
    \chardef\DeclareVersion=1 %
  \fi
\fi
\ifnum\DeclareVersion=0 %
  \def\check#1{%
    \ifx\cmp#1%
      \msg{* Test passed.}%
    \else
      \msg{}%
      \msg{[\meaning#1]}%
      \msg{[\meaning\cmp]}%
      \errmessage{Test failed!}%
    \fi
  }%
\else
  \def\check#1{%
    \begingroup
      \escapechar=-1 %
    \edef\x{\endgroup
      \def\noexpand\cs/{\string#1}%
    }\x
    \edef\CMP{%
      \noexpand\protect
      \expandafter\noexpand\csname\cs/ \endcsname
    }%
    \ifx\CMP#1%
      \expandafter\ifx\csname\cs/ \endcsname\cmp
        \msg{Test passed.}%
      \else
        \msg{}%
        \msg{[\expandafter\meaning\csname\cs/ \endcsname]}%
        \msg{[\meaning\cmp]}%
        \errmessage{Test failed!}%
      \fi
    \else
      \msg{}%
      \msg{[\meaning#1]}%
      \msg{[\meaning\CMP]}%
      \errmessage{Test failed!}%
    \fi
  }%
\fi

\tracingmacros=1

\BeginCheck
  \pdef\cmp{}%
  \ProtectedDef*\foo{}%
\EndCheck

\BeginCheck
  \pdef\cmp{}%
  \ProtectedDef*\foo[0]{}%
\EndCheck

\BeginCheck
  \pdef\cmp#1{<#1>}%
  \ProtectedDef*\foo[1]{<#1>}%
\EndCheck

\BeginCheck
  \pdef\cmp(#1){<#1>}%
  \ProtectedDef*\foo(#1){<#1>}%
\EndCheck

\BeginCheck
  \long\pdef\cmp{}%
  \ProtectedDef\foo{}%
\EndCheck

\BeginCheck
  \long\pdef\cmp{}%
  \ProtectedDef\foo[0]{}%
\EndCheck

\BeginCheck
  \long\pdef\cmp#1{<#1>}%
  \ProtectedDef\foo[1]{<#1>}%
\EndCheck

\BeginCheck
  \long\pdef\cmp(#1){<#1>}%
  \ProtectedDef\foo(#1){<#1>}%
\EndCheck

\csname @@end\endcsname\end
%</test2>
%    \end{macrocode}
%
% \section{Installation}
%
% \subsection{Download}
%
% \paragraph{Package.} This package is available on
% CTAN\footnote{\CTANpkg{protecteddef}}:
% \begin{description}
% \item[\CTAN{macros/latex/contrib/oberdiek/protecteddef.dtx}] The source file.
% \item[\CTAN{macros/latex/contrib/oberdiek/protecteddef.pdf}] Documentation.
% \end{description}
%
%
% \paragraph{Bundle.} All the packages of the bundle `oberdiek'
% are also available in a TDS compliant ZIP archive. There
% the packages are already unpacked and the documentation files
% are generated. The files and directories obey the TDS standard.
% \begin{description}
% \item[\CTANinstall{install/macros/latex/contrib/oberdiek.tds.zip}]
% \end{description}
% \emph{TDS} refers to the standard ``A Directory Structure
% for \TeX\ Files'' (\CTAN{tds/tds.pdf}). Directories
% with \xfile{texmf} in their name are usually organized this way.
%
% \subsection{Bundle installation}
%
% \paragraph{Unpacking.} Unpack the \xfile{oberdiek.tds.zip} in the
% TDS tree (also known as \xfile{texmf} tree) of your choice.
% Example (linux):
% \begin{quote}
%   |unzip oberdiek.tds.zip -d ~/texmf|
% \end{quote}
%
% \paragraph{Script installation.}
% Check the directory \xfile{TDS:scripts/oberdiek/} for
% scripts that need further installation steps.
%
% \subsection{Package installation}
%
% \paragraph{Unpacking.} The \xfile{.dtx} file is a self-extracting
% \docstrip\ archive. The files are extracted by running the
% \xfile{.dtx} through \plainTeX:
% \begin{quote}
%   \verb|tex protecteddef.dtx|
% \end{quote}
%
% \paragraph{TDS.} Now the different files must be moved into
% the different directories in your installation TDS tree
% (also known as \xfile{texmf} tree):
% \begin{quote}
% \def\t{^^A
% \begin{tabular}{@{}>{\ttfamily}l@{ $\rightarrow$ }>{\ttfamily}l@{}}
%   protecteddef.sty & tex/generic/oberdiek/protecteddef.sty\\
%   protecteddef.pdf & doc/latex/oberdiek/protecteddef.pdf\\
%   test/protecteddef-test1.tex & doc/latex/oberdiek/test/protecteddef-test1.tex\\
%   test/protecteddef-test2.tex & doc/latex/oberdiek/test/protecteddef-test2.tex\\
%   protecteddef.dtx & source/latex/oberdiek/protecteddef.dtx\\
% \end{tabular}^^A
% }^^A
% \sbox0{\t}^^A
% \ifdim\wd0>\linewidth
%   \begingroup
%     \advance\linewidth by\leftmargin
%     \advance\linewidth by\rightmargin
%   \edef\x{\endgroup
%     \def\noexpand\lw{\the\linewidth}^^A
%   }\x
%   \def\lwbox{^^A
%     \leavevmode
%     \hbox to \linewidth{^^A
%       \kern-\leftmargin\relax
%       \hss
%       \usebox0
%       \hss
%       \kern-\rightmargin\relax
%     }^^A
%   }^^A
%   \ifdim\wd0>\lw
%     \sbox0{\small\t}^^A
%     \ifdim\wd0>\linewidth
%       \ifdim\wd0>\lw
%         \sbox0{\footnotesize\t}^^A
%         \ifdim\wd0>\linewidth
%           \ifdim\wd0>\lw
%             \sbox0{\scriptsize\t}^^A
%             \ifdim\wd0>\linewidth
%               \ifdim\wd0>\lw
%                 \sbox0{\tiny\t}^^A
%                 \ifdim\wd0>\linewidth
%                   \lwbox
%                 \else
%                   \usebox0
%                 \fi
%               \else
%                 \lwbox
%               \fi
%             \else
%               \usebox0
%             \fi
%           \else
%             \lwbox
%           \fi
%         \else
%           \usebox0
%         \fi
%       \else
%         \lwbox
%       \fi
%     \else
%       \usebox0
%     \fi
%   \else
%     \lwbox
%   \fi
% \else
%   \usebox0
% \fi
% \end{quote}
% If you have a \xfile{docstrip.cfg} that configures and enables \docstrip's
% TDS installing feature, then some files can already be in the right
% place, see the documentation of \docstrip.
%
% \subsection{Refresh file name databases}
%
% If your \TeX~distribution
% (\TeX\,Live, \mikTeX, \dots) relies on file name databases, you must refresh
% these. For example, \TeX\,Live\ users run \verb|texhash| or
% \verb|mktexlsr|.
%
% \subsection{Some details for the interested}
%
% \paragraph{Unpacking with \LaTeX.}
% The \xfile{.dtx} chooses its action depending on the format:
% \begin{description}
% \item[\plainTeX:] Run \docstrip\ and extract the files.
% \item[\LaTeX:] Generate the documentation.
% \end{description}
% If you insist on using \LaTeX\ for \docstrip\ (really,
% \docstrip\ does not need \LaTeX), then inform the autodetect routine
% about your intention:
% \begin{quote}
%   \verb|latex \let\install=y\input{protecteddef.dtx}|
% \end{quote}
% Do not forget to quote the argument according to the demands
% of your shell.
%
% \paragraph{Generating the documentation.}
% You can use both the \xfile{.dtx} or the \xfile{.drv} to generate
% the documentation. The process can be configured by the
% configuration file \xfile{ltxdoc.cfg}. For instance, put this
% line into this file, if you want to have A4 as paper format:
% \begin{quote}
%   \verb|\PassOptionsToClass{a4paper}{article}|
% \end{quote}
% An example follows how to generate the
% documentation with pdf\LaTeX:
% \begin{quote}
%\begin{verbatim}
%pdflatex protecteddef.dtx
%makeindex -s gind.ist protecteddef.idx
%pdflatex protecteddef.dtx
%makeindex -s gind.ist protecteddef.idx
%pdflatex protecteddef.dtx
%\end{verbatim}
% \end{quote}
%
% \begin{History}
%   \begin{Version}{2011/01/31 v1.0}
%   \item
%     First public version.
%   \end{Version}
%   \begin{Version}{2016/05/16 v1.1}
%   \item
%     Documentation updates.
%   \end{Version}
% \end{History}
%
% \PrintIndex
%
% \Finale
\endinput
|
% \end{quote}
% Do not forget to quote the argument according to the demands
% of your shell.
%
% \paragraph{Generating the documentation.}
% You can use both the \xfile{.dtx} or the \xfile{.drv} to generate
% the documentation. The process can be configured by the
% configuration file \xfile{ltxdoc.cfg}. For instance, put this
% line into this file, if you want to have A4 as paper format:
% \begin{quote}
%   \verb|\PassOptionsToClass{a4paper}{article}|
% \end{quote}
% An example follows how to generate the
% documentation with pdf\LaTeX:
% \begin{quote}
%\begin{verbatim}
%pdflatex protecteddef.dtx
%makeindex -s gind.ist protecteddef.idx
%pdflatex protecteddef.dtx
%makeindex -s gind.ist protecteddef.idx
%pdflatex protecteddef.dtx
%\end{verbatim}
% \end{quote}
%
% \begin{History}
%   \begin{Version}{2011/01/31 v1.0}
%   \item
%     First public version.
%   \end{Version}
%   \begin{Version}{2016/05/16 v1.1}
%   \item
%     Documentation updates.
%   \end{Version}
% \end{History}
%
% \PrintIndex
%
% \Finale
\endinput

%        (quote the arguments according to the demands of your shell)
%
% Documentation:
%    (a) If protecteddef.drv is present:
%           latex protecteddef.drv
%    (b) Without protecteddef.drv:
%           latex protecteddef.dtx; ...
%    The class ltxdoc loads the configuration file ltxdoc.cfg
%    if available. Here you can specify further options, e.g.
%    use A4 as paper format:
%       \PassOptionsToClass{a4paper}{article}
%
%    Programm calls to get the documentation (example):
%       pdflatex protecteddef.dtx
%       makeindex -s gind.ist protecteddef.idx
%       pdflatex protecteddef.dtx
%       makeindex -s gind.ist protecteddef.idx
%       pdflatex protecteddef.dtx
%
% Installation:
%    TDS:tex/generic/oberdiek/protecteddef.sty
%    TDS:doc/latex/oberdiek/protecteddef.pdf
%    TDS:doc/latex/oberdiek/test/protecteddef-test1.tex
%    TDS:doc/latex/oberdiek/test/protecteddef-test2.tex
%    TDS:source/latex/oberdiek/protecteddef.dtx
%
%<*ignore>
\begingroup
  \catcode123=1 %
  \catcode125=2 %
  \def\x{LaTeX2e}%
\expandafter\endgroup
\ifcase 0\ifx\install y1\fi\expandafter
         \ifx\csname processbatchFile\endcsname\relax\else1\fi
         \ifx\fmtname\x\else 1\fi\relax
\else\csname fi\endcsname
%</ignore>
%<*install>
\input docstrip.tex
\Msg{************************************************************************}
\Msg{* Installation}
\Msg{* Package: protecteddef 2016/05/16 v1.1 Define protected commands (HO)}
\Msg{************************************************************************}

\keepsilent
\askforoverwritefalse

\let\MetaPrefix\relax
\preamble

This is a generated file.

Project: protecteddef
Version: 2016/05/16 v1.1

Copyright (C)
   2011 Heiko Oberdiek
   2016-2019 Oberdiek Package Support Group

This work may be distributed and/or modified under the
conditions of the LaTeX Project Public License, either
version 1.3c of this license or (at your option) any later
version. This version of this license is in
   https://www.latex-project.org/lppl/lppl-1-3c.txt
and the latest version of this license is in
   https://www.latex-project.org/lppl.txt
and version 1.3 or later is part of all distributions of
LaTeX version 2005/12/01 or later.

This work has the LPPL maintenance status "maintained".

The Current Maintainers of this work are
Heiko Oberdiek and the Oberdiek Package Support Group
https://github.com/ho-tex/oberdiek/issues


The Base Interpreter refers to any `TeX-Format',
because some files are installed in TDS:tex/generic//.

This work consists of the main source file protecteddef.dtx
and the derived files
   protecteddef.sty, protecteddef.pdf, protecteddef.ins, protecteddef.drv,
   protecteddef-test1.tex, protecteddef-test2.tex.

\endpreamble
\let\MetaPrefix\DoubleperCent

\generate{%
  \file{protecteddef.ins}{\from{protecteddef.dtx}{install}}%
  \file{protecteddef.drv}{\from{protecteddef.dtx}{driver}}%
  \usedir{tex/generic/oberdiek}%
  \file{protecteddef.sty}{\from{protecteddef.dtx}{package}}%
%  \usedir{doc/latex/oberdiek/test}%
%  \file{protecteddef-test1.tex}{\from{protecteddef.dtx}{test1}}%
%  \file{protecteddef-test2.tex}{\from{protecteddef.dtx}{test2}}%
  \nopreamble
  \nopostamble
%  \usedir{source/latex/oberdiek/catalogue}%
%  \file{protecteddef.xml}{\from{protecteddef.dtx}{catalogue}}%
}

\catcode32=13\relax% active space
\let =\space%
\Msg{************************************************************************}
\Msg{*}
\Msg{* To finish the installation you have to move the following}
\Msg{* file into a directory searched by TeX:}
\Msg{*}
\Msg{*     protecteddef.sty}
\Msg{*}
\Msg{* To produce the documentation run the file `protecteddef.drv'}
\Msg{* through LaTeX.}
\Msg{*}
\Msg{* Happy TeXing!}
\Msg{*}
\Msg{************************************************************************}

\endbatchfile
%</install>
%<*ignore>
\fi
%</ignore>
%<*driver>
\NeedsTeXFormat{LaTeX2e}
\ProvidesFile{protecteddef.drv}%
  [2016/05/16 v1.1 Define protected commands (HO)]%
\documentclass{ltxdoc}
\usepackage{holtxdoc}[2011/11/22]
\begin{document}
  \DocInput{protecteddef.dtx}%
\end{document}
%</driver>
% \fi
%
%
% \CharacterTable
%  {Upper-case    \A\B\C\D\E\F\G\H\I\J\K\L\M\N\O\P\Q\R\S\T\U\V\W\X\Y\Z
%   Lower-case    \a\b\c\d\e\f\g\h\i\j\k\l\m\n\o\p\q\r\s\t\u\v\w\x\y\z
%   Digits        \0\1\2\3\4\5\6\7\8\9
%   Exclamation   \!     Double quote  \"     Hash (number) \#
%   Dollar        \$     Percent       \%     Ampersand     \&
%   Acute accent  \'     Left paren    \(     Right paren   \)
%   Asterisk      \*     Plus          \+     Comma         \,
%   Minus         \-     Point         \.     Solidus       \/
%   Colon         \:     Semicolon     \;     Less than     \<
%   Equals        \=     Greater than  \>     Question mark \?
%   Commercial at \@     Left bracket  \[     Backslash     \\
%   Right bracket \]     Circumflex    \^     Underscore    \_
%   Grave accent  \`     Left brace    \{     Vertical bar  \|
%   Right brace   \}     Tilde         \~}
%
% \GetFileInfo{protecteddef.drv}
%
% \title{The \xpackage{protecteddef} package}
% \date{2016/05/16 v1.1}
% \author{Heiko Oberdiek\thanks
% {Please report any issues at \url{https://github.com/ho-tex/oberdiek/issues}}}
%
% \maketitle
%
% \begin{abstract}
% This packages provides \cs{ProtectedDef} for defining
% robust macros for both \hologo{plainTeX} and \hologo{LaTeX}.
% First \hologo{eTeX}'s \cs{protected} is tried, then
% \hologo{LaTeX}'s \cs{DeclareRobustCommand} is used.
% Otherwise the macro is not made robust.
% \end{abstract}
%
% \tableofcontents
%
% \section{Documentation}
%
% Many of my packages work for both formats \hologo{plainTeX}
% and \hologo{LaTeX}, even \hologo{iniTeX} is often supported.
% It would be nice if fragile macros could be protected and
% made robust. However the different format worlds offer different
% solutions.
%
% \subsection{The \hologo{LaTeX}'s way}
%
% Usually \cs{newcommand} is used to define macros. It provides
% a check if the command to be defined is already defined or
% cannot be defined for other reasons.
%
% For making robust macros \hologo{LaTeX} provides \cs{DeclareRobustCommand}.
% It shares the syntax with \cs{newcommand}. However it does not provide
% latters check. Internally the check is available via \cs{@ifdefinable}.
%
% Internally the robust macro is using \cs{protect} with a nested
% macro definition. The \cs{protect} infrastructure is a feature of
% \cs{LaTeX} and usually not available in other formats.
%
% \subsection{The \hologo{eTeX}'s way}
%
% The need for robust macros is addressed in \cs{eTeX}. It provides
% \cs{protected} that modifies the behaviour of \cs{def} in a similar
% way as \cs{long}. A protected macro does not expand in
% some expandable contexts like writing to a file or \cs{edef}.
%
% \subsection{The way of this package}
%
% The package tries to find the available protection mechanism.
% First it looks for \cs{eTeX}'s \cs{protected}, then
% it uses \hologo{LaTeX}'s \cs{DeclareRobustCommand}.
% If both fails, then the macro remains unprotected.
%
% Additionally, \cs{LaTeX}'s check, if a macro is already defined
% is added in all cases. First \hologo{LaTeX}'s \cs{@ifdefinable}
% is tried to be compatible with \hologo{LaTeX}.
% If \cs{@ifdefinable} is not available, then the test is implemented
% by asserting that the macro is undefined or has the meaning of \cs{relax}.
% If the test fails, then in all cases the macro is not defined
% and an error is thrown.
%
% \subsection{Usage}
%
% \begin{declcs}{ProtectedDef} \* \M{cmd} [\meta{num}] \M{definition text}
% \end{declcs}
% Macro \cs{ProtectedDef} follows the syntax of \hologo{LaTeX}'s
% \cs{newcommand} with the exception that an optional argument
% is not supported. Macro \meta{cmd} is to be defined as
% \cs{long} macro without star with \meta{num} arguments.
%
% The number of arguments \meta{num} must be given as explicite digit
% |0| upto |9|. Otherwise the part between the argument \meta{cmd} and
% the \meta{definition text} is taken as parameter text in the syntax
% of vanilla \hologo{TeX}. Examples (with \cs{protected}):
% \begin{quote}
%   \def\to{$\Rightarrow$ }%
%   |\ProtectedDef*{\foo}[1]{\message{#1}}|\\
%   \to |\protected\def\foo#1{\message#1}}|\\[1ex]
%   |\ProtectedDef\foo{abc}|\\
%   \to |\protected\def\foo{abc}|\\[1ex]
%   |\ProtectedDef*\foo(#1)<#2>{#1/#2}|\\
%   \to |\protected\def\foo(#1)<#2>{#1/#2}|
% \end{quote}
%
% \StopEventually{
% }
%
% \section{Implementation}
%
%    \begin{macrocode}
%<*package>
%    \end{macrocode}
%
% \subsection{Reload check and package identification}
%    Reload check, especially if the package is not used with \LaTeX.
%    \begin{macrocode}
\begingroup\catcode61\catcode48\catcode32=10\relax%
  \catcode13=5 % ^^M
  \endlinechar=13 %
  \catcode35=6 % #
  \catcode39=12 % '
  \catcode44=12 % ,
  \catcode45=12 % -
  \catcode46=12 % .
  \catcode58=12 % :
  \catcode64=11 % @
  \catcode123=1 % {
  \catcode125=2 % }
  \expandafter\let\expandafter\x\csname ver@protecteddef.sty\endcsname
  \ifx\x\relax % plain-TeX, first loading
  \else
    \def\empty{}%
    \ifx\x\empty % LaTeX, first loading,
      % variable is initialized, but \ProvidesPackage not yet seen
    \else
      \expandafter\ifx\csname PackageInfo\endcsname\relax
        \def\x#1#2{%
          \immediate\write-1{Package #1 Info: #2.}%
        }%
      \else
        \def\x#1#2{\PackageInfo{#1}{#2, stopped}}%
      \fi
      \x{protecteddef}{The package is already loaded}%
      \aftergroup\endinput
    \fi
  \fi
\endgroup%
%    \end{macrocode}
%    Package identification:
%    \begin{macrocode}
\begingroup\catcode61\catcode48\catcode32=10\relax%
  \catcode13=5 % ^^M
  \endlinechar=13 %
  \catcode35=6 % #
  \catcode39=12 % '
  \catcode40=12 % (
  \catcode41=12 % )
  \catcode44=12 % ,
  \catcode45=12 % -
  \catcode46=12 % .
  \catcode47=12 % /
  \catcode58=12 % :
  \catcode64=11 % @
  \catcode91=12 % [
  \catcode93=12 % ]
  \catcode123=1 % {
  \catcode125=2 % }
  \expandafter\ifx\csname ProvidesPackage\endcsname\relax
    \def\x#1#2#3[#4]{\endgroup
      \immediate\write-1{Package: #3 #4}%
      \xdef#1{#4}%
    }%
  \else
    \def\x#1#2[#3]{\endgroup
      #2[{#3}]%
      \ifx#1\@undefined
        \xdef#1{#3}%
      \fi
      \ifx#1\relax
        \xdef#1{#3}%
      \fi
    }%
  \fi
\expandafter\x\csname ver@protecteddef.sty\endcsname
\ProvidesPackage{protecteddef}%
  [2016/05/16 v1.1 Define protected commands (HO)]%
%    \end{macrocode}
%
% \subsection{Catcodes}
%
%    \begin{macrocode}
\begingroup\catcode61\catcode48\catcode32=10\relax%
  \catcode13=5 % ^^M
  \endlinechar=13 %
  \catcode123=1 % {
  \catcode125=2 % }
  \catcode64=11 % @
  \def\x{\endgroup
    \expandafter\edef\csname ProDef@AtEnd\endcsname{%
      \endlinechar=\the\endlinechar\relax
      \catcode13=\the\catcode13\relax
      \catcode32=\the\catcode32\relax
      \catcode35=\the\catcode35\relax
      \catcode61=\the\catcode61\relax
      \catcode64=\the\catcode64\relax
      \catcode123=\the\catcode123\relax
      \catcode125=\the\catcode125\relax
    }%
  }%
\x\catcode61\catcode48\catcode32=10\relax%
\catcode13=5 % ^^M
\endlinechar=13 %
\catcode35=6 % #
\catcode64=11 % @
\catcode123=1 % {
\catcode125=2 % }
\def\TMP@EnsureCode#1#2{%
  \edef\ProDef@AtEnd{%
    \ProDef@AtEnd
    \catcode#1=\the\catcode#1\relax
  }%
  \catcode#1=#2\relax
}
\TMP@EnsureCode{38}{4}% &
\TMP@EnsureCode{40}{12}% (
\TMP@EnsureCode{41}{12}% )
\TMP@EnsureCode{42}{12}% *
\TMP@EnsureCode{45}{12}% -
\TMP@EnsureCode{46}{12}% .
\TMP@EnsureCode{47}{12}% /
\TMP@EnsureCode{91}{12}% [
\TMP@EnsureCode{93}{12}% ]
\TMP@EnsureCode{96}{12}% `
\edef\ProDef@AtEnd{\ProDef@AtEnd\noexpand\endinput}
%    \end{macrocode}
%
% \subsection{Resources}
%
%    \begin{macrocode}
\begingroup\expandafter\expandafter\expandafter\endgroup
\expandafter\ifx\csname RequirePackage\endcsname\relax
  \def\TMP@RequirePackage#1[#2]{%
    \begingroup\expandafter\expandafter\expandafter\endgroup
    \expandafter\ifx\csname ver@#1.sty\endcsname\relax
      \input #1.sty\relax
    \fi
  }%
\else
  \let\TMP@RequirePackage\RequirePackage
\fi
\TMP@RequirePackage{ltxcmds}[2010/12/12]%
\TMP@RequirePackage{infwarerr}[2010/04/08]%
%    \end{macrocode}
%
%    \begin{macrocode}
\def\ProDef@temp#1{%
  \expandafter\def\csname ProDef@param[#1]\endcsname % hash-ok
}
\expandafter\def\csname ProDef@param\endcsname{}
\ProDef@temp0{}
\ProDef@temp1{##1}
\ProDef@temp2{##1##2}
\ProDef@temp3{##1##2##3}
\ProDef@temp4{##1##2##3##4}
\ProDef@temp5{##1##2##3##4##5}
\ProDef@temp6{##1##2##3##4##5##6}
\ProDef@temp7{##1##2##3##4##5##7}
\ProDef@temp8{##1##2##3##4##5##7##8}
\ProDef@temp9{##1##2##3##4##5##7##8##9}
%    \end{macrocode}
%
%    \begin{macro}{\ProDef@IfDefinable}
%    \begin{macrocode}
\ltx@IfUndefined{@ifdefinable}{%
  \long\def\ProDef@IfDefinable#1{%
    \begingroup
      \escapechar=-1 %
    \ltx@ifundefined{\string#1}{%
      \endgroup
      \ltx@firstofone
    }{%
      \expandafter\endgroup
      \expandafter
      \edef\expandafter\ProDef@temp\expandafter{\string#1 }%
      \@PackageError{protecteddef}{%
        Command \ltx@backslashchar\ProDef@temp already defined%
      }\@ehc
      \ltx@gobbletwo
    }%
  }%
}{%
  \long\def\ProDef@IfDefinable#1{%
    \let\ProDef@next\ltx@gobbletwo
    \@ifdefinable{#1}{%
      \let\ProDef@next\ltx@firstofone
    }%
    \ProDef@next
  }%
}
%    \end{macrocode}
%    \end{macro}
%
%    \begin{macrocode}
\begingroup\expandafter\expandafter\expandafter\endgroup
\expandafter\ifx\csname protected\endcsname\relax
  \begingroup\expandafter\expandafter\expandafter\endgroup
  \expandafter\ifx\csname DeclareRobustCommand\endcsname\relax
    \catcode`\&=14 % comment
  \else
    \newcommand*{\ProtectedDef}{%
      \ltx@ifnextchar*{%
        \ProDef@ProtectedDef
      }{%
        \ProDef@ProtectedDef{}%
      }%
    }%
    \long\def\ProDef@ProtectedDef#1#2#3#{%
      \ProDef@IfDefinable{#2}{%
        \ltx@IfUndefined{ProDef@param#3}{%
          \DeclareRobustCommand*{#2}{}%
          \begingroup
            \escapechar=-1 %
            \def\ProDef@temp{#1}%
          \edef\x{\endgroup
            \ifx\ProDef@temp\ltx@empty
              \noexpand\long
            \fi
            \noexpand\def
            \expandafter\noexpand\csname\string#2 \endcsname
          }%
          \x#3%
        }{%
          \DeclareRobustCommand#1{#2}#3%
        }%
      }%
    }%
    \expandafter\expandafter\expandafter\ProDef@AtEnd
  \fi
\else
  \catcode`\&=9 % ignore
\fi%
\ProDef@IfDefinable\ProtectedDef{%
&  \protected
  \def\ProtectedDef%
}{%
  \ltx@ifnextchar*{%
    \let\ProDef@long\ltx@empty
    \expandafter\ProDef@ProtectedDef\ltx@gobble
  }{%
    \let\ProDef@long\long
    \ProDef@ProtectedDef
  }%
}
\long\def\ProDef@ProtectedDef#1#2#{%
  \ProDef@IfDefinable{#1}{%
    \ltx@IfUndefined{ProDef@param#2}{%
&     \protected
      \ProDef@long
      \def#1#2%
    }{%
&     \protected
      \ProDef@long
      \expandafter\expandafter\expandafter\def
      \expandafter\expandafter\expandafter#1%
      \csname ProDef@param#2\endcsname
    }%
  }%
}
%    \end{macrocode}
%
%    \begin{macrocode}
\ProDef@AtEnd%
%    \end{macrocode}
%    \begin{macrocode}
%</package>
%    \end{macrocode}
%
% \section{Test}
%
% \subsection{Catcode checks for loading}
%
%    \begin{macrocode}
%<*test1>
%    \end{macrocode}
%    \begin{macrocode}
\catcode`\{=1 %
\catcode`\}=2 %
\catcode`\#=6 %
\catcode`\@=11 %
\expandafter\ifx\csname count@\endcsname\relax
  \countdef\count@=255 %
\fi
\expandafter\ifx\csname @gobble\endcsname\relax
  \long\def\@gobble#1{}%
\fi
\expandafter\ifx\csname @firstofone\endcsname\relax
  \long\def\@firstofone#1{#1}%
\fi
\expandafter\ifx\csname loop\endcsname\relax
  \expandafter\@firstofone
\else
  \expandafter\@gobble
\fi
{%
  \def\loop#1\repeat{%
    \def\body{#1}%
    \iterate
  }%
  \def\iterate{%
    \body
      \let\next\iterate
    \else
      \let\next\relax
    \fi
    \next
  }%
  \let\repeat=\fi
}%
\def\RestoreCatcodes{}
\count@=0 %
\loop
  \edef\RestoreCatcodes{%
    \RestoreCatcodes
    \catcode\the\count@=\the\catcode\count@\relax
  }%
\ifnum\count@<255 %
  \advance\count@ 1 %
\repeat

\def\RangeCatcodeInvalid#1#2{%
  \count@=#1\relax
  \loop
    \catcode\count@=15 %
  \ifnum\count@<#2\relax
    \advance\count@ 1 %
  \repeat
}
\def\RangeCatcodeCheck#1#2#3{%
  \count@=#1\relax
  \loop
    \ifnum#3=\catcode\count@
    \else
      \errmessage{%
        Character \the\count@\space
        with wrong catcode \the\catcode\count@\space
        instead of \number#3%
      }%
    \fi
  \ifnum\count@<#2\relax
    \advance\count@ 1 %
  \repeat
}
\def\space{ }
\expandafter\ifx\csname LoadCommand\endcsname\relax
  \def\LoadCommand{\input protecteddef.sty\relax}%
\fi
\def\Test{%
  \RangeCatcodeInvalid{0}{47}%
  \RangeCatcodeInvalid{58}{64}%
  \RangeCatcodeInvalid{91}{96}%
  \RangeCatcodeInvalid{123}{255}%
  \catcode`\@=12 %
  \catcode`\\=0 %
  \catcode`\%=14 %
  \LoadCommand
  \RangeCatcodeCheck{0}{36}{15}%
  \RangeCatcodeCheck{37}{37}{14}%
  \RangeCatcodeCheck{38}{47}{15}%
  \RangeCatcodeCheck{48}{57}{12}%
  \RangeCatcodeCheck{58}{63}{15}%
  \RangeCatcodeCheck{64}{64}{12}%
  \RangeCatcodeCheck{65}{90}{11}%
  \RangeCatcodeCheck{91}{91}{15}%
  \RangeCatcodeCheck{92}{92}{0}%
  \RangeCatcodeCheck{93}{96}{15}%
  \RangeCatcodeCheck{97}{122}{11}%
  \RangeCatcodeCheck{123}{255}{15}%
  \RestoreCatcodes
}
\Test
\csname @@end\endcsname
\end
%    \end{macrocode}
%    \begin{macrocode}
%</test1>
%    \end{macrocode}
%
% \subsection{Test without \hologo{LaTeX} and \cs{protected}}
%
%    \begin{macrocode}
%<*test2>
\errorcontextlines=10000 %
\begingroup\expandafter\expandafter\expandafter\endgroup
\expandafter\ifx\csname RequirePackage\endcsname\relax
  \input protecteddef.sty\relax
  \catcode`\{=1 %
  \catcode`\}=2 %
  \catcode`\#=6 %
\else
  \RequirePackage{protecteddef}[2016/05/16]%
\fi
\begingroup\expandafter\expandafter\expandafter\endgroup
\expandafter\ifx\csname protected\endcsname\relax
  \let\pdef\def
\else
  \def\pdef{\protected\def}%
\fi
\def\msg#{\immediate\write16}
\countdef\errcount=2 %
\long\def\BeginCheck#1\ProtectedDef#2\EndCheck{%
  \begingroup
    \toks0={\ProtectedDef#2}%
    \msg{<<\the\toks0>>}%
  \endgroup
  \setbox0=\hbox{%
    #1%
    \ProtectedDef#2%
    \check\foo
  }%
  \ifdim\wd0=0pt\relax
  \else
    \errmessage{[Definition] Unwanted spaces?!}%
  \fi
  \setbox0=\hbox{%
    \def\fooinitial{XYZ}%
    \let\foo\fooinitial
    \errcount=0 %
    \expandafter\def\csname @PackageError\endcsname##1##2##3{%
      \advance\errcount by 1 %
    }%
    \expandafter\def\csname @notdefinable\endcsname{%
      \advance\errcount by 1 %
    }%
    \ProtectedDef#2%
    \ifnum\errcount=1 %
    \else
      \errmessage{1 error expected, but found: \the\errcount}%
    \fi
    \ifx\foo\fooinitial
    \else
      \def\space{ }%
      \errmessage{\string\foo\space is overwritten}%
    \fi
  }%
  \ifdim\wd0=0pt\relax
  \else
    \errmessage{[Error] Unwanted spaces?!}%
  \fi
}
\chardef\DeclareVersion=0 %
\begingroup\expandafter\expandafter\expandafter\endgroup
\expandafter\ifx\csname protected\endcsname\relax
  \begingroup\expandafter\expandafter\expandafter\endgroup
  \expandafter\ifx\csname DeclareRobustCommand\endcsname\relax
  \else
    \chardef\DeclareVersion=1 %
  \fi
\fi
\ifnum\DeclareVersion=0 %
  \def\check#1{%
    \ifx\cmp#1%
      \msg{* Test passed.}%
    \else
      \msg{}%
      \msg{[\meaning#1]}%
      \msg{[\meaning\cmp]}%
      \errmessage{Test failed!}%
    \fi
  }%
\else
  \def\check#1{%
    \begingroup
      \escapechar=-1 %
    \edef\x{\endgroup
      \def\noexpand\cs/{\string#1}%
    }\x
    \edef\CMP{%
      \noexpand\protect
      \expandafter\noexpand\csname\cs/ \endcsname
    }%
    \ifx\CMP#1%
      \expandafter\ifx\csname\cs/ \endcsname\cmp
        \msg{Test passed.}%
      \else
        \msg{}%
        \msg{[\expandafter\meaning\csname\cs/ \endcsname]}%
        \msg{[\meaning\cmp]}%
        \errmessage{Test failed!}%
      \fi
    \else
      \msg{}%
      \msg{[\meaning#1]}%
      \msg{[\meaning\CMP]}%
      \errmessage{Test failed!}%
    \fi
  }%
\fi

\tracingmacros=1

\BeginCheck
  \pdef\cmp{}%
  \ProtectedDef*\foo{}%
\EndCheck

\BeginCheck
  \pdef\cmp{}%
  \ProtectedDef*\foo[0]{}%
\EndCheck

\BeginCheck
  \pdef\cmp#1{<#1>}%
  \ProtectedDef*\foo[1]{<#1>}%
\EndCheck

\BeginCheck
  \pdef\cmp(#1){<#1>}%
  \ProtectedDef*\foo(#1){<#1>}%
\EndCheck

\BeginCheck
  \long\pdef\cmp{}%
  \ProtectedDef\foo{}%
\EndCheck

\BeginCheck
  \long\pdef\cmp{}%
  \ProtectedDef\foo[0]{}%
\EndCheck

\BeginCheck
  \long\pdef\cmp#1{<#1>}%
  \ProtectedDef\foo[1]{<#1>}%
\EndCheck

\BeginCheck
  \long\pdef\cmp(#1){<#1>}%
  \ProtectedDef\foo(#1){<#1>}%
\EndCheck

\csname @@end\endcsname\end
%</test2>
%    \end{macrocode}
%
% \section{Installation}
%
% \subsection{Download}
%
% \paragraph{Package.} This package is available on
% CTAN\footnote{\CTANpkg{protecteddef}}:
% \begin{description}
% \item[\CTAN{macros/latex/contrib/oberdiek/protecteddef.dtx}] The source file.
% \item[\CTAN{macros/latex/contrib/oberdiek/protecteddef.pdf}] Documentation.
% \end{description}
%
%
% \paragraph{Bundle.} All the packages of the bundle `oberdiek'
% are also available in a TDS compliant ZIP archive. There
% the packages are already unpacked and the documentation files
% are generated. The files and directories obey the TDS standard.
% \begin{description}
% \item[\CTANinstall{install/macros/latex/contrib/oberdiek.tds.zip}]
% \end{description}
% \emph{TDS} refers to the standard ``A Directory Structure
% for \TeX\ Files'' (\CTAN{tds/tds.pdf}). Directories
% with \xfile{texmf} in their name are usually organized this way.
%
% \subsection{Bundle installation}
%
% \paragraph{Unpacking.} Unpack the \xfile{oberdiek.tds.zip} in the
% TDS tree (also known as \xfile{texmf} tree) of your choice.
% Example (linux):
% \begin{quote}
%   |unzip oberdiek.tds.zip -d ~/texmf|
% \end{quote}
%
% \paragraph{Script installation.}
% Check the directory \xfile{TDS:scripts/oberdiek/} for
% scripts that need further installation steps.
%
% \subsection{Package installation}
%
% \paragraph{Unpacking.} The \xfile{.dtx} file is a self-extracting
% \docstrip\ archive. The files are extracted by running the
% \xfile{.dtx} through \plainTeX:
% \begin{quote}
%   \verb|tex protecteddef.dtx|
% \end{quote}
%
% \paragraph{TDS.} Now the different files must be moved into
% the different directories in your installation TDS tree
% (also known as \xfile{texmf} tree):
% \begin{quote}
% \def\t{^^A
% \begin{tabular}{@{}>{\ttfamily}l@{ $\rightarrow$ }>{\ttfamily}l@{}}
%   protecteddef.sty & tex/generic/oberdiek/protecteddef.sty\\
%   protecteddef.pdf & doc/latex/oberdiek/protecteddef.pdf\\
%   test/protecteddef-test1.tex & doc/latex/oberdiek/test/protecteddef-test1.tex\\
%   test/protecteddef-test2.tex & doc/latex/oberdiek/test/protecteddef-test2.tex\\
%   protecteddef.dtx & source/latex/oberdiek/protecteddef.dtx\\
% \end{tabular}^^A
% }^^A
% \sbox0{\t}^^A
% \ifdim\wd0>\linewidth
%   \begingroup
%     \advance\linewidth by\leftmargin
%     \advance\linewidth by\rightmargin
%   \edef\x{\endgroup
%     \def\noexpand\lw{\the\linewidth}^^A
%   }\x
%   \def\lwbox{^^A
%     \leavevmode
%     \hbox to \linewidth{^^A
%       \kern-\leftmargin\relax
%       \hss
%       \usebox0
%       \hss
%       \kern-\rightmargin\relax
%     }^^A
%   }^^A
%   \ifdim\wd0>\lw
%     \sbox0{\small\t}^^A
%     \ifdim\wd0>\linewidth
%       \ifdim\wd0>\lw
%         \sbox0{\footnotesize\t}^^A
%         \ifdim\wd0>\linewidth
%           \ifdim\wd0>\lw
%             \sbox0{\scriptsize\t}^^A
%             \ifdim\wd0>\linewidth
%               \ifdim\wd0>\lw
%                 \sbox0{\tiny\t}^^A
%                 \ifdim\wd0>\linewidth
%                   \lwbox
%                 \else
%                   \usebox0
%                 \fi
%               \else
%                 \lwbox
%               \fi
%             \else
%               \usebox0
%             \fi
%           \else
%             \lwbox
%           \fi
%         \else
%           \usebox0
%         \fi
%       \else
%         \lwbox
%       \fi
%     \else
%       \usebox0
%     \fi
%   \else
%     \lwbox
%   \fi
% \else
%   \usebox0
% \fi
% \end{quote}
% If you have a \xfile{docstrip.cfg} that configures and enables \docstrip's
% TDS installing feature, then some files can already be in the right
% place, see the documentation of \docstrip.
%
% \subsection{Refresh file name databases}
%
% If your \TeX~distribution
% (\TeX\,Live, \mikTeX, \dots) relies on file name databases, you must refresh
% these. For example, \TeX\,Live\ users run \verb|texhash| or
% \verb|mktexlsr|.
%
% \subsection{Some details for the interested}
%
% \paragraph{Unpacking with \LaTeX.}
% The \xfile{.dtx} chooses its action depending on the format:
% \begin{description}
% \item[\plainTeX:] Run \docstrip\ and extract the files.
% \item[\LaTeX:] Generate the documentation.
% \end{description}
% If you insist on using \LaTeX\ for \docstrip\ (really,
% \docstrip\ does not need \LaTeX), then inform the autodetect routine
% about your intention:
% \begin{quote}
%   \verb|latex \let\install=y% \iffalse meta-comment
%
% File: protecteddef.dtx
% Version: 2016/05/16 v1.1
% Info: Define protected commands
%
% Copyright (C)
%    2011 Heiko Oberdiek
%    2016-2019 Oberdiek Package Support Group
%    https://github.com/ho-tex/oberdiek/issues
%
% This work may be distributed and/or modified under the
% conditions of the LaTeX Project Public License, either
% version 1.3c of this license or (at your option) any later
% version. This version of this license is in
%    https://www.latex-project.org/lppl/lppl-1-3c.txt
% and the latest version of this license is in
%    https://www.latex-project.org/lppl.txt
% and version 1.3 or later is part of all distributions of
% LaTeX version 2005/12/01 or later.
%
% This work has the LPPL maintenance status "maintained".
%
% The Current Maintainers of this work are
% Heiko Oberdiek and the Oberdiek Package Support Group
% https://github.com/ho-tex/oberdiek/issues
%
% The Base Interpreter refers to any `TeX-Format',
% because some files are installed in TDS:tex/generic//.
%
% This work consists of the main source file protecteddef.dtx
% and the derived files
%    protecteddef.sty, protecteddef.pdf, protecteddef.ins, protecteddef.drv,
%    protecteddef-test1.tex, protecteddef-test2.tex.
%
% Distribution:
%    CTAN:macros/latex/contrib/oberdiek/protecteddef.dtx
%    CTAN:macros/latex/contrib/oberdiek/protecteddef.pdf
%
% Unpacking:
%    (a) If protecteddef.ins is present:
%           tex protecteddef.ins
%    (b) Without protecteddef.ins:
%           tex protecteddef.dtx
%    (c) If you insist on using LaTeX
%           latex \let\install=y% \iffalse meta-comment
%
% File: protecteddef.dtx
% Version: 2016/05/16 v1.1
% Info: Define protected commands
%
% Copyright (C)
%    2011 Heiko Oberdiek
%    2016-2019 Oberdiek Package Support Group
%    https://github.com/ho-tex/oberdiek/issues
%
% This work may be distributed and/or modified under the
% conditions of the LaTeX Project Public License, either
% version 1.3c of this license or (at your option) any later
% version. This version of this license is in
%    https://www.latex-project.org/lppl/lppl-1-3c.txt
% and the latest version of this license is in
%    https://www.latex-project.org/lppl.txt
% and version 1.3 or later is part of all distributions of
% LaTeX version 2005/12/01 or later.
%
% This work has the LPPL maintenance status "maintained".
%
% The Current Maintainers of this work are
% Heiko Oberdiek and the Oberdiek Package Support Group
% https://github.com/ho-tex/oberdiek/issues
%
% The Base Interpreter refers to any `TeX-Format',
% because some files are installed in TDS:tex/generic//.
%
% This work consists of the main source file protecteddef.dtx
% and the derived files
%    protecteddef.sty, protecteddef.pdf, protecteddef.ins, protecteddef.drv,
%    protecteddef-test1.tex, protecteddef-test2.tex.
%
% Distribution:
%    CTAN:macros/latex/contrib/oberdiek/protecteddef.dtx
%    CTAN:macros/latex/contrib/oberdiek/protecteddef.pdf
%
% Unpacking:
%    (a) If protecteddef.ins is present:
%           tex protecteddef.ins
%    (b) Without protecteddef.ins:
%           tex protecteddef.dtx
%    (c) If you insist on using LaTeX
%           latex \let\install=y\input{protecteddef.dtx}
%        (quote the arguments according to the demands of your shell)
%
% Documentation:
%    (a) If protecteddef.drv is present:
%           latex protecteddef.drv
%    (b) Without protecteddef.drv:
%           latex protecteddef.dtx; ...
%    The class ltxdoc loads the configuration file ltxdoc.cfg
%    if available. Here you can specify further options, e.g.
%    use A4 as paper format:
%       \PassOptionsToClass{a4paper}{article}
%
%    Programm calls to get the documentation (example):
%       pdflatex protecteddef.dtx
%       makeindex -s gind.ist protecteddef.idx
%       pdflatex protecteddef.dtx
%       makeindex -s gind.ist protecteddef.idx
%       pdflatex protecteddef.dtx
%
% Installation:
%    TDS:tex/generic/oberdiek/protecteddef.sty
%    TDS:doc/latex/oberdiek/protecteddef.pdf
%    TDS:doc/latex/oberdiek/test/protecteddef-test1.tex
%    TDS:doc/latex/oberdiek/test/protecteddef-test2.tex
%    TDS:source/latex/oberdiek/protecteddef.dtx
%
%<*ignore>
\begingroup
  \catcode123=1 %
  \catcode125=2 %
  \def\x{LaTeX2e}%
\expandafter\endgroup
\ifcase 0\ifx\install y1\fi\expandafter
         \ifx\csname processbatchFile\endcsname\relax\else1\fi
         \ifx\fmtname\x\else 1\fi\relax
\else\csname fi\endcsname
%</ignore>
%<*install>
\input docstrip.tex
\Msg{************************************************************************}
\Msg{* Installation}
\Msg{* Package: protecteddef 2016/05/16 v1.1 Define protected commands (HO)}
\Msg{************************************************************************}

\keepsilent
\askforoverwritefalse

\let\MetaPrefix\relax
\preamble

This is a generated file.

Project: protecteddef
Version: 2016/05/16 v1.1

Copyright (C)
   2011 Heiko Oberdiek
   2016-2019 Oberdiek Package Support Group

This work may be distributed and/or modified under the
conditions of the LaTeX Project Public License, either
version 1.3c of this license or (at your option) any later
version. This version of this license is in
   https://www.latex-project.org/lppl/lppl-1-3c.txt
and the latest version of this license is in
   https://www.latex-project.org/lppl.txt
and version 1.3 or later is part of all distributions of
LaTeX version 2005/12/01 or later.

This work has the LPPL maintenance status "maintained".

The Current Maintainers of this work are
Heiko Oberdiek and the Oberdiek Package Support Group
https://github.com/ho-tex/oberdiek/issues


The Base Interpreter refers to any `TeX-Format',
because some files are installed in TDS:tex/generic//.

This work consists of the main source file protecteddef.dtx
and the derived files
   protecteddef.sty, protecteddef.pdf, protecteddef.ins, protecteddef.drv,
   protecteddef-test1.tex, protecteddef-test2.tex.

\endpreamble
\let\MetaPrefix\DoubleperCent

\generate{%
  \file{protecteddef.ins}{\from{protecteddef.dtx}{install}}%
  \file{protecteddef.drv}{\from{protecteddef.dtx}{driver}}%
  \usedir{tex/generic/oberdiek}%
  \file{protecteddef.sty}{\from{protecteddef.dtx}{package}}%
%  \usedir{doc/latex/oberdiek/test}%
%  \file{protecteddef-test1.tex}{\from{protecteddef.dtx}{test1}}%
%  \file{protecteddef-test2.tex}{\from{protecteddef.dtx}{test2}}%
  \nopreamble
  \nopostamble
%  \usedir{source/latex/oberdiek/catalogue}%
%  \file{protecteddef.xml}{\from{protecteddef.dtx}{catalogue}}%
}

\catcode32=13\relax% active space
\let =\space%
\Msg{************************************************************************}
\Msg{*}
\Msg{* To finish the installation you have to move the following}
\Msg{* file into a directory searched by TeX:}
\Msg{*}
\Msg{*     protecteddef.sty}
\Msg{*}
\Msg{* To produce the documentation run the file `protecteddef.drv'}
\Msg{* through LaTeX.}
\Msg{*}
\Msg{* Happy TeXing!}
\Msg{*}
\Msg{************************************************************************}

\endbatchfile
%</install>
%<*ignore>
\fi
%</ignore>
%<*driver>
\NeedsTeXFormat{LaTeX2e}
\ProvidesFile{protecteddef.drv}%
  [2016/05/16 v1.1 Define protected commands (HO)]%
\documentclass{ltxdoc}
\usepackage{holtxdoc}[2011/11/22]
\begin{document}
  \DocInput{protecteddef.dtx}%
\end{document}
%</driver>
% \fi
%
%
% \CharacterTable
%  {Upper-case    \A\B\C\D\E\F\G\H\I\J\K\L\M\N\O\P\Q\R\S\T\U\V\W\X\Y\Z
%   Lower-case    \a\b\c\d\e\f\g\h\i\j\k\l\m\n\o\p\q\r\s\t\u\v\w\x\y\z
%   Digits        \0\1\2\3\4\5\6\7\8\9
%   Exclamation   \!     Double quote  \"     Hash (number) \#
%   Dollar        \$     Percent       \%     Ampersand     \&
%   Acute accent  \'     Left paren    \(     Right paren   \)
%   Asterisk      \*     Plus          \+     Comma         \,
%   Minus         \-     Point         \.     Solidus       \/
%   Colon         \:     Semicolon     \;     Less than     \<
%   Equals        \=     Greater than  \>     Question mark \?
%   Commercial at \@     Left bracket  \[     Backslash     \\
%   Right bracket \]     Circumflex    \^     Underscore    \_
%   Grave accent  \`     Left brace    \{     Vertical bar  \|
%   Right brace   \}     Tilde         \~}
%
% \GetFileInfo{protecteddef.drv}
%
% \title{The \xpackage{protecteddef} package}
% \date{2016/05/16 v1.1}
% \author{Heiko Oberdiek\thanks
% {Please report any issues at \url{https://github.com/ho-tex/oberdiek/issues}}}
%
% \maketitle
%
% \begin{abstract}
% This packages provides \cs{ProtectedDef} for defining
% robust macros for both \hologo{plainTeX} and \hologo{LaTeX}.
% First \hologo{eTeX}'s \cs{protected} is tried, then
% \hologo{LaTeX}'s \cs{DeclareRobustCommand} is used.
% Otherwise the macro is not made robust.
% \end{abstract}
%
% \tableofcontents
%
% \section{Documentation}
%
% Many of my packages work for both formats \hologo{plainTeX}
% and \hologo{LaTeX}, even \hologo{iniTeX} is often supported.
% It would be nice if fragile macros could be protected and
% made robust. However the different format worlds offer different
% solutions.
%
% \subsection{The \hologo{LaTeX}'s way}
%
% Usually \cs{newcommand} is used to define macros. It provides
% a check if the command to be defined is already defined or
% cannot be defined for other reasons.
%
% For making robust macros \hologo{LaTeX} provides \cs{DeclareRobustCommand}.
% It shares the syntax with \cs{newcommand}. However it does not provide
% latters check. Internally the check is available via \cs{@ifdefinable}.
%
% Internally the robust macro is using \cs{protect} with a nested
% macro definition. The \cs{protect} infrastructure is a feature of
% \cs{LaTeX} and usually not available in other formats.
%
% \subsection{The \hologo{eTeX}'s way}
%
% The need for robust macros is addressed in \cs{eTeX}. It provides
% \cs{protected} that modifies the behaviour of \cs{def} in a similar
% way as \cs{long}. A protected macro does not expand in
% some expandable contexts like writing to a file or \cs{edef}.
%
% \subsection{The way of this package}
%
% The package tries to find the available protection mechanism.
% First it looks for \cs{eTeX}'s \cs{protected}, then
% it uses \hologo{LaTeX}'s \cs{DeclareRobustCommand}.
% If both fails, then the macro remains unprotected.
%
% Additionally, \cs{LaTeX}'s check, if a macro is already defined
% is added in all cases. First \hologo{LaTeX}'s \cs{@ifdefinable}
% is tried to be compatible with \hologo{LaTeX}.
% If \cs{@ifdefinable} is not available, then the test is implemented
% by asserting that the macro is undefined or has the meaning of \cs{relax}.
% If the test fails, then in all cases the macro is not defined
% and an error is thrown.
%
% \subsection{Usage}
%
% \begin{declcs}{ProtectedDef} \* \M{cmd} [\meta{num}] \M{definition text}
% \end{declcs}
% Macro \cs{ProtectedDef} follows the syntax of \hologo{LaTeX}'s
% \cs{newcommand} with the exception that an optional argument
% is not supported. Macro \meta{cmd} is to be defined as
% \cs{long} macro without star with \meta{num} arguments.
%
% The number of arguments \meta{num} must be given as explicite digit
% |0| upto |9|. Otherwise the part between the argument \meta{cmd} and
% the \meta{definition text} is taken as parameter text in the syntax
% of vanilla \hologo{TeX}. Examples (with \cs{protected}):
% \begin{quote}
%   \def\to{$\Rightarrow$ }%
%   |\ProtectedDef*{\foo}[1]{\message{#1}}|\\
%   \to |\protected\def\foo#1{\message#1}}|\\[1ex]
%   |\ProtectedDef\foo{abc}|\\
%   \to |\protected\def\foo{abc}|\\[1ex]
%   |\ProtectedDef*\foo(#1)<#2>{#1/#2}|\\
%   \to |\protected\def\foo(#1)<#2>{#1/#2}|
% \end{quote}
%
% \StopEventually{
% }
%
% \section{Implementation}
%
%    \begin{macrocode}
%<*package>
%    \end{macrocode}
%
% \subsection{Reload check and package identification}
%    Reload check, especially if the package is not used with \LaTeX.
%    \begin{macrocode}
\begingroup\catcode61\catcode48\catcode32=10\relax%
  \catcode13=5 % ^^M
  \endlinechar=13 %
  \catcode35=6 % #
  \catcode39=12 % '
  \catcode44=12 % ,
  \catcode45=12 % -
  \catcode46=12 % .
  \catcode58=12 % :
  \catcode64=11 % @
  \catcode123=1 % {
  \catcode125=2 % }
  \expandafter\let\expandafter\x\csname ver@protecteddef.sty\endcsname
  \ifx\x\relax % plain-TeX, first loading
  \else
    \def\empty{}%
    \ifx\x\empty % LaTeX, first loading,
      % variable is initialized, but \ProvidesPackage not yet seen
    \else
      \expandafter\ifx\csname PackageInfo\endcsname\relax
        \def\x#1#2{%
          \immediate\write-1{Package #1 Info: #2.}%
        }%
      \else
        \def\x#1#2{\PackageInfo{#1}{#2, stopped}}%
      \fi
      \x{protecteddef}{The package is already loaded}%
      \aftergroup\endinput
    \fi
  \fi
\endgroup%
%    \end{macrocode}
%    Package identification:
%    \begin{macrocode}
\begingroup\catcode61\catcode48\catcode32=10\relax%
  \catcode13=5 % ^^M
  \endlinechar=13 %
  \catcode35=6 % #
  \catcode39=12 % '
  \catcode40=12 % (
  \catcode41=12 % )
  \catcode44=12 % ,
  \catcode45=12 % -
  \catcode46=12 % .
  \catcode47=12 % /
  \catcode58=12 % :
  \catcode64=11 % @
  \catcode91=12 % [
  \catcode93=12 % ]
  \catcode123=1 % {
  \catcode125=2 % }
  \expandafter\ifx\csname ProvidesPackage\endcsname\relax
    \def\x#1#2#3[#4]{\endgroup
      \immediate\write-1{Package: #3 #4}%
      \xdef#1{#4}%
    }%
  \else
    \def\x#1#2[#3]{\endgroup
      #2[{#3}]%
      \ifx#1\@undefined
        \xdef#1{#3}%
      \fi
      \ifx#1\relax
        \xdef#1{#3}%
      \fi
    }%
  \fi
\expandafter\x\csname ver@protecteddef.sty\endcsname
\ProvidesPackage{protecteddef}%
  [2016/05/16 v1.1 Define protected commands (HO)]%
%    \end{macrocode}
%
% \subsection{Catcodes}
%
%    \begin{macrocode}
\begingroup\catcode61\catcode48\catcode32=10\relax%
  \catcode13=5 % ^^M
  \endlinechar=13 %
  \catcode123=1 % {
  \catcode125=2 % }
  \catcode64=11 % @
  \def\x{\endgroup
    \expandafter\edef\csname ProDef@AtEnd\endcsname{%
      \endlinechar=\the\endlinechar\relax
      \catcode13=\the\catcode13\relax
      \catcode32=\the\catcode32\relax
      \catcode35=\the\catcode35\relax
      \catcode61=\the\catcode61\relax
      \catcode64=\the\catcode64\relax
      \catcode123=\the\catcode123\relax
      \catcode125=\the\catcode125\relax
    }%
  }%
\x\catcode61\catcode48\catcode32=10\relax%
\catcode13=5 % ^^M
\endlinechar=13 %
\catcode35=6 % #
\catcode64=11 % @
\catcode123=1 % {
\catcode125=2 % }
\def\TMP@EnsureCode#1#2{%
  \edef\ProDef@AtEnd{%
    \ProDef@AtEnd
    \catcode#1=\the\catcode#1\relax
  }%
  \catcode#1=#2\relax
}
\TMP@EnsureCode{38}{4}% &
\TMP@EnsureCode{40}{12}% (
\TMP@EnsureCode{41}{12}% )
\TMP@EnsureCode{42}{12}% *
\TMP@EnsureCode{45}{12}% -
\TMP@EnsureCode{46}{12}% .
\TMP@EnsureCode{47}{12}% /
\TMP@EnsureCode{91}{12}% [
\TMP@EnsureCode{93}{12}% ]
\TMP@EnsureCode{96}{12}% `
\edef\ProDef@AtEnd{\ProDef@AtEnd\noexpand\endinput}
%    \end{macrocode}
%
% \subsection{Resources}
%
%    \begin{macrocode}
\begingroup\expandafter\expandafter\expandafter\endgroup
\expandafter\ifx\csname RequirePackage\endcsname\relax
  \def\TMP@RequirePackage#1[#2]{%
    \begingroup\expandafter\expandafter\expandafter\endgroup
    \expandafter\ifx\csname ver@#1.sty\endcsname\relax
      \input #1.sty\relax
    \fi
  }%
\else
  \let\TMP@RequirePackage\RequirePackage
\fi
\TMP@RequirePackage{ltxcmds}[2010/12/12]%
\TMP@RequirePackage{infwarerr}[2010/04/08]%
%    \end{macrocode}
%
%    \begin{macrocode}
\def\ProDef@temp#1{%
  \expandafter\def\csname ProDef@param[#1]\endcsname % hash-ok
}
\expandafter\def\csname ProDef@param\endcsname{}
\ProDef@temp0{}
\ProDef@temp1{##1}
\ProDef@temp2{##1##2}
\ProDef@temp3{##1##2##3}
\ProDef@temp4{##1##2##3##4}
\ProDef@temp5{##1##2##3##4##5}
\ProDef@temp6{##1##2##3##4##5##6}
\ProDef@temp7{##1##2##3##4##5##7}
\ProDef@temp8{##1##2##3##4##5##7##8}
\ProDef@temp9{##1##2##3##4##5##7##8##9}
%    \end{macrocode}
%
%    \begin{macro}{\ProDef@IfDefinable}
%    \begin{macrocode}
\ltx@IfUndefined{@ifdefinable}{%
  \long\def\ProDef@IfDefinable#1{%
    \begingroup
      \escapechar=-1 %
    \ltx@ifundefined{\string#1}{%
      \endgroup
      \ltx@firstofone
    }{%
      \expandafter\endgroup
      \expandafter
      \edef\expandafter\ProDef@temp\expandafter{\string#1 }%
      \@PackageError{protecteddef}{%
        Command \ltx@backslashchar\ProDef@temp already defined%
      }\@ehc
      \ltx@gobbletwo
    }%
  }%
}{%
  \long\def\ProDef@IfDefinable#1{%
    \let\ProDef@next\ltx@gobbletwo
    \@ifdefinable{#1}{%
      \let\ProDef@next\ltx@firstofone
    }%
    \ProDef@next
  }%
}
%    \end{macrocode}
%    \end{macro}
%
%    \begin{macrocode}
\begingroup\expandafter\expandafter\expandafter\endgroup
\expandafter\ifx\csname protected\endcsname\relax
  \begingroup\expandafter\expandafter\expandafter\endgroup
  \expandafter\ifx\csname DeclareRobustCommand\endcsname\relax
    \catcode`\&=14 % comment
  \else
    \newcommand*{\ProtectedDef}{%
      \ltx@ifnextchar*{%
        \ProDef@ProtectedDef
      }{%
        \ProDef@ProtectedDef{}%
      }%
    }%
    \long\def\ProDef@ProtectedDef#1#2#3#{%
      \ProDef@IfDefinable{#2}{%
        \ltx@IfUndefined{ProDef@param#3}{%
          \DeclareRobustCommand*{#2}{}%
          \begingroup
            \escapechar=-1 %
            \def\ProDef@temp{#1}%
          \edef\x{\endgroup
            \ifx\ProDef@temp\ltx@empty
              \noexpand\long
            \fi
            \noexpand\def
            \expandafter\noexpand\csname\string#2 \endcsname
          }%
          \x#3%
        }{%
          \DeclareRobustCommand#1{#2}#3%
        }%
      }%
    }%
    \expandafter\expandafter\expandafter\ProDef@AtEnd
  \fi
\else
  \catcode`\&=9 % ignore
\fi%
\ProDef@IfDefinable\ProtectedDef{%
&  \protected
  \def\ProtectedDef%
}{%
  \ltx@ifnextchar*{%
    \let\ProDef@long\ltx@empty
    \expandafter\ProDef@ProtectedDef\ltx@gobble
  }{%
    \let\ProDef@long\long
    \ProDef@ProtectedDef
  }%
}
\long\def\ProDef@ProtectedDef#1#2#{%
  \ProDef@IfDefinable{#1}{%
    \ltx@IfUndefined{ProDef@param#2}{%
&     \protected
      \ProDef@long
      \def#1#2%
    }{%
&     \protected
      \ProDef@long
      \expandafter\expandafter\expandafter\def
      \expandafter\expandafter\expandafter#1%
      \csname ProDef@param#2\endcsname
    }%
  }%
}
%    \end{macrocode}
%
%    \begin{macrocode}
\ProDef@AtEnd%
%    \end{macrocode}
%    \begin{macrocode}
%</package>
%    \end{macrocode}
%
% \section{Test}
%
% \subsection{Catcode checks for loading}
%
%    \begin{macrocode}
%<*test1>
%    \end{macrocode}
%    \begin{macrocode}
\catcode`\{=1 %
\catcode`\}=2 %
\catcode`\#=6 %
\catcode`\@=11 %
\expandafter\ifx\csname count@\endcsname\relax
  \countdef\count@=255 %
\fi
\expandafter\ifx\csname @gobble\endcsname\relax
  \long\def\@gobble#1{}%
\fi
\expandafter\ifx\csname @firstofone\endcsname\relax
  \long\def\@firstofone#1{#1}%
\fi
\expandafter\ifx\csname loop\endcsname\relax
  \expandafter\@firstofone
\else
  \expandafter\@gobble
\fi
{%
  \def\loop#1\repeat{%
    \def\body{#1}%
    \iterate
  }%
  \def\iterate{%
    \body
      \let\next\iterate
    \else
      \let\next\relax
    \fi
    \next
  }%
  \let\repeat=\fi
}%
\def\RestoreCatcodes{}
\count@=0 %
\loop
  \edef\RestoreCatcodes{%
    \RestoreCatcodes
    \catcode\the\count@=\the\catcode\count@\relax
  }%
\ifnum\count@<255 %
  \advance\count@ 1 %
\repeat

\def\RangeCatcodeInvalid#1#2{%
  \count@=#1\relax
  \loop
    \catcode\count@=15 %
  \ifnum\count@<#2\relax
    \advance\count@ 1 %
  \repeat
}
\def\RangeCatcodeCheck#1#2#3{%
  \count@=#1\relax
  \loop
    \ifnum#3=\catcode\count@
    \else
      \errmessage{%
        Character \the\count@\space
        with wrong catcode \the\catcode\count@\space
        instead of \number#3%
      }%
    \fi
  \ifnum\count@<#2\relax
    \advance\count@ 1 %
  \repeat
}
\def\space{ }
\expandafter\ifx\csname LoadCommand\endcsname\relax
  \def\LoadCommand{\input protecteddef.sty\relax}%
\fi
\def\Test{%
  \RangeCatcodeInvalid{0}{47}%
  \RangeCatcodeInvalid{58}{64}%
  \RangeCatcodeInvalid{91}{96}%
  \RangeCatcodeInvalid{123}{255}%
  \catcode`\@=12 %
  \catcode`\\=0 %
  \catcode`\%=14 %
  \LoadCommand
  \RangeCatcodeCheck{0}{36}{15}%
  \RangeCatcodeCheck{37}{37}{14}%
  \RangeCatcodeCheck{38}{47}{15}%
  \RangeCatcodeCheck{48}{57}{12}%
  \RangeCatcodeCheck{58}{63}{15}%
  \RangeCatcodeCheck{64}{64}{12}%
  \RangeCatcodeCheck{65}{90}{11}%
  \RangeCatcodeCheck{91}{91}{15}%
  \RangeCatcodeCheck{92}{92}{0}%
  \RangeCatcodeCheck{93}{96}{15}%
  \RangeCatcodeCheck{97}{122}{11}%
  \RangeCatcodeCheck{123}{255}{15}%
  \RestoreCatcodes
}
\Test
\csname @@end\endcsname
\end
%    \end{macrocode}
%    \begin{macrocode}
%</test1>
%    \end{macrocode}
%
% \subsection{Test without \hologo{LaTeX} and \cs{protected}}
%
%    \begin{macrocode}
%<*test2>
\errorcontextlines=10000 %
\begingroup\expandafter\expandafter\expandafter\endgroup
\expandafter\ifx\csname RequirePackage\endcsname\relax
  \input protecteddef.sty\relax
  \catcode`\{=1 %
  \catcode`\}=2 %
  \catcode`\#=6 %
\else
  \RequirePackage{protecteddef}[2016/05/16]%
\fi
\begingroup\expandafter\expandafter\expandafter\endgroup
\expandafter\ifx\csname protected\endcsname\relax
  \let\pdef\def
\else
  \def\pdef{\protected\def}%
\fi
\def\msg#{\immediate\write16}
\countdef\errcount=2 %
\long\def\BeginCheck#1\ProtectedDef#2\EndCheck{%
  \begingroup
    \toks0={\ProtectedDef#2}%
    \msg{<<\the\toks0>>}%
  \endgroup
  \setbox0=\hbox{%
    #1%
    \ProtectedDef#2%
    \check\foo
  }%
  \ifdim\wd0=0pt\relax
  \else
    \errmessage{[Definition] Unwanted spaces?!}%
  \fi
  \setbox0=\hbox{%
    \def\fooinitial{XYZ}%
    \let\foo\fooinitial
    \errcount=0 %
    \expandafter\def\csname @PackageError\endcsname##1##2##3{%
      \advance\errcount by 1 %
    }%
    \expandafter\def\csname @notdefinable\endcsname{%
      \advance\errcount by 1 %
    }%
    \ProtectedDef#2%
    \ifnum\errcount=1 %
    \else
      \errmessage{1 error expected, but found: \the\errcount}%
    \fi
    \ifx\foo\fooinitial
    \else
      \def\space{ }%
      \errmessage{\string\foo\space is overwritten}%
    \fi
  }%
  \ifdim\wd0=0pt\relax
  \else
    \errmessage{[Error] Unwanted spaces?!}%
  \fi
}
\chardef\DeclareVersion=0 %
\begingroup\expandafter\expandafter\expandafter\endgroup
\expandafter\ifx\csname protected\endcsname\relax
  \begingroup\expandafter\expandafter\expandafter\endgroup
  \expandafter\ifx\csname DeclareRobustCommand\endcsname\relax
  \else
    \chardef\DeclareVersion=1 %
  \fi
\fi
\ifnum\DeclareVersion=0 %
  \def\check#1{%
    \ifx\cmp#1%
      \msg{* Test passed.}%
    \else
      \msg{}%
      \msg{[\meaning#1]}%
      \msg{[\meaning\cmp]}%
      \errmessage{Test failed!}%
    \fi
  }%
\else
  \def\check#1{%
    \begingroup
      \escapechar=-1 %
    \edef\x{\endgroup
      \def\noexpand\cs/{\string#1}%
    }\x
    \edef\CMP{%
      \noexpand\protect
      \expandafter\noexpand\csname\cs/ \endcsname
    }%
    \ifx\CMP#1%
      \expandafter\ifx\csname\cs/ \endcsname\cmp
        \msg{Test passed.}%
      \else
        \msg{}%
        \msg{[\expandafter\meaning\csname\cs/ \endcsname]}%
        \msg{[\meaning\cmp]}%
        \errmessage{Test failed!}%
      \fi
    \else
      \msg{}%
      \msg{[\meaning#1]}%
      \msg{[\meaning\CMP]}%
      \errmessage{Test failed!}%
    \fi
  }%
\fi

\tracingmacros=1

\BeginCheck
  \pdef\cmp{}%
  \ProtectedDef*\foo{}%
\EndCheck

\BeginCheck
  \pdef\cmp{}%
  \ProtectedDef*\foo[0]{}%
\EndCheck

\BeginCheck
  \pdef\cmp#1{<#1>}%
  \ProtectedDef*\foo[1]{<#1>}%
\EndCheck

\BeginCheck
  \pdef\cmp(#1){<#1>}%
  \ProtectedDef*\foo(#1){<#1>}%
\EndCheck

\BeginCheck
  \long\pdef\cmp{}%
  \ProtectedDef\foo{}%
\EndCheck

\BeginCheck
  \long\pdef\cmp{}%
  \ProtectedDef\foo[0]{}%
\EndCheck

\BeginCheck
  \long\pdef\cmp#1{<#1>}%
  \ProtectedDef\foo[1]{<#1>}%
\EndCheck

\BeginCheck
  \long\pdef\cmp(#1){<#1>}%
  \ProtectedDef\foo(#1){<#1>}%
\EndCheck

\csname @@end\endcsname\end
%</test2>
%    \end{macrocode}
%
% \section{Installation}
%
% \subsection{Download}
%
% \paragraph{Package.} This package is available on
% CTAN\footnote{\CTANpkg{protecteddef}}:
% \begin{description}
% \item[\CTAN{macros/latex/contrib/oberdiek/protecteddef.dtx}] The source file.
% \item[\CTAN{macros/latex/contrib/oberdiek/protecteddef.pdf}] Documentation.
% \end{description}
%
%
% \paragraph{Bundle.} All the packages of the bundle `oberdiek'
% are also available in a TDS compliant ZIP archive. There
% the packages are already unpacked and the documentation files
% are generated. The files and directories obey the TDS standard.
% \begin{description}
% \item[\CTANinstall{install/macros/latex/contrib/oberdiek.tds.zip}]
% \end{description}
% \emph{TDS} refers to the standard ``A Directory Structure
% for \TeX\ Files'' (\CTAN{tds/tds.pdf}). Directories
% with \xfile{texmf} in their name are usually organized this way.
%
% \subsection{Bundle installation}
%
% \paragraph{Unpacking.} Unpack the \xfile{oberdiek.tds.zip} in the
% TDS tree (also known as \xfile{texmf} tree) of your choice.
% Example (linux):
% \begin{quote}
%   |unzip oberdiek.tds.zip -d ~/texmf|
% \end{quote}
%
% \paragraph{Script installation.}
% Check the directory \xfile{TDS:scripts/oberdiek/} for
% scripts that need further installation steps.
%
% \subsection{Package installation}
%
% \paragraph{Unpacking.} The \xfile{.dtx} file is a self-extracting
% \docstrip\ archive. The files are extracted by running the
% \xfile{.dtx} through \plainTeX:
% \begin{quote}
%   \verb|tex protecteddef.dtx|
% \end{quote}
%
% \paragraph{TDS.} Now the different files must be moved into
% the different directories in your installation TDS tree
% (also known as \xfile{texmf} tree):
% \begin{quote}
% \def\t{^^A
% \begin{tabular}{@{}>{\ttfamily}l@{ $\rightarrow$ }>{\ttfamily}l@{}}
%   protecteddef.sty & tex/generic/oberdiek/protecteddef.sty\\
%   protecteddef.pdf & doc/latex/oberdiek/protecteddef.pdf\\
%   test/protecteddef-test1.tex & doc/latex/oberdiek/test/protecteddef-test1.tex\\
%   test/protecteddef-test2.tex & doc/latex/oberdiek/test/protecteddef-test2.tex\\
%   protecteddef.dtx & source/latex/oberdiek/protecteddef.dtx\\
% \end{tabular}^^A
% }^^A
% \sbox0{\t}^^A
% \ifdim\wd0>\linewidth
%   \begingroup
%     \advance\linewidth by\leftmargin
%     \advance\linewidth by\rightmargin
%   \edef\x{\endgroup
%     \def\noexpand\lw{\the\linewidth}^^A
%   }\x
%   \def\lwbox{^^A
%     \leavevmode
%     \hbox to \linewidth{^^A
%       \kern-\leftmargin\relax
%       \hss
%       \usebox0
%       \hss
%       \kern-\rightmargin\relax
%     }^^A
%   }^^A
%   \ifdim\wd0>\lw
%     \sbox0{\small\t}^^A
%     \ifdim\wd0>\linewidth
%       \ifdim\wd0>\lw
%         \sbox0{\footnotesize\t}^^A
%         \ifdim\wd0>\linewidth
%           \ifdim\wd0>\lw
%             \sbox0{\scriptsize\t}^^A
%             \ifdim\wd0>\linewidth
%               \ifdim\wd0>\lw
%                 \sbox0{\tiny\t}^^A
%                 \ifdim\wd0>\linewidth
%                   \lwbox
%                 \else
%                   \usebox0
%                 \fi
%               \else
%                 \lwbox
%               \fi
%             \else
%               \usebox0
%             \fi
%           \else
%             \lwbox
%           \fi
%         \else
%           \usebox0
%         \fi
%       \else
%         \lwbox
%       \fi
%     \else
%       \usebox0
%     \fi
%   \else
%     \lwbox
%   \fi
% \else
%   \usebox0
% \fi
% \end{quote}
% If you have a \xfile{docstrip.cfg} that configures and enables \docstrip's
% TDS installing feature, then some files can already be in the right
% place, see the documentation of \docstrip.
%
% \subsection{Refresh file name databases}
%
% If your \TeX~distribution
% (\TeX\,Live, \mikTeX, \dots) relies on file name databases, you must refresh
% these. For example, \TeX\,Live\ users run \verb|texhash| or
% \verb|mktexlsr|.
%
% \subsection{Some details for the interested}
%
% \paragraph{Unpacking with \LaTeX.}
% The \xfile{.dtx} chooses its action depending on the format:
% \begin{description}
% \item[\plainTeX:] Run \docstrip\ and extract the files.
% \item[\LaTeX:] Generate the documentation.
% \end{description}
% If you insist on using \LaTeX\ for \docstrip\ (really,
% \docstrip\ does not need \LaTeX), then inform the autodetect routine
% about your intention:
% \begin{quote}
%   \verb|latex \let\install=y\input{protecteddef.dtx}|
% \end{quote}
% Do not forget to quote the argument according to the demands
% of your shell.
%
% \paragraph{Generating the documentation.}
% You can use both the \xfile{.dtx} or the \xfile{.drv} to generate
% the documentation. The process can be configured by the
% configuration file \xfile{ltxdoc.cfg}. For instance, put this
% line into this file, if you want to have A4 as paper format:
% \begin{quote}
%   \verb|\PassOptionsToClass{a4paper}{article}|
% \end{quote}
% An example follows how to generate the
% documentation with pdf\LaTeX:
% \begin{quote}
%\begin{verbatim}
%pdflatex protecteddef.dtx
%makeindex -s gind.ist protecteddef.idx
%pdflatex protecteddef.dtx
%makeindex -s gind.ist protecteddef.idx
%pdflatex protecteddef.dtx
%\end{verbatim}
% \end{quote}
%
% \begin{History}
%   \begin{Version}{2011/01/31 v1.0}
%   \item
%     First public version.
%   \end{Version}
%   \begin{Version}{2016/05/16 v1.1}
%   \item
%     Documentation updates.
%   \end{Version}
% \end{History}
%
% \PrintIndex
%
% \Finale
\endinput

%        (quote the arguments according to the demands of your shell)
%
% Documentation:
%    (a) If protecteddef.drv is present:
%           latex protecteddef.drv
%    (b) Without protecteddef.drv:
%           latex protecteddef.dtx; ...
%    The class ltxdoc loads the configuration file ltxdoc.cfg
%    if available. Here you can specify further options, e.g.
%    use A4 as paper format:
%       \PassOptionsToClass{a4paper}{article}
%
%    Programm calls to get the documentation (example):
%       pdflatex protecteddef.dtx
%       makeindex -s gind.ist protecteddef.idx
%       pdflatex protecteddef.dtx
%       makeindex -s gind.ist protecteddef.idx
%       pdflatex protecteddef.dtx
%
% Installation:
%    TDS:tex/generic/oberdiek/protecteddef.sty
%    TDS:doc/latex/oberdiek/protecteddef.pdf
%    TDS:doc/latex/oberdiek/test/protecteddef-test1.tex
%    TDS:doc/latex/oberdiek/test/protecteddef-test2.tex
%    TDS:source/latex/oberdiek/protecteddef.dtx
%
%<*ignore>
\begingroup
  \catcode123=1 %
  \catcode125=2 %
  \def\x{LaTeX2e}%
\expandafter\endgroup
\ifcase 0\ifx\install y1\fi\expandafter
         \ifx\csname processbatchFile\endcsname\relax\else1\fi
         \ifx\fmtname\x\else 1\fi\relax
\else\csname fi\endcsname
%</ignore>
%<*install>
\input docstrip.tex
\Msg{************************************************************************}
\Msg{* Installation}
\Msg{* Package: protecteddef 2016/05/16 v1.1 Define protected commands (HO)}
\Msg{************************************************************************}

\keepsilent
\askforoverwritefalse

\let\MetaPrefix\relax
\preamble

This is a generated file.

Project: protecteddef
Version: 2016/05/16 v1.1

Copyright (C)
   2011 Heiko Oberdiek
   2016-2019 Oberdiek Package Support Group

This work may be distributed and/or modified under the
conditions of the LaTeX Project Public License, either
version 1.3c of this license or (at your option) any later
version. This version of this license is in
   https://www.latex-project.org/lppl/lppl-1-3c.txt
and the latest version of this license is in
   https://www.latex-project.org/lppl.txt
and version 1.3 or later is part of all distributions of
LaTeX version 2005/12/01 or later.

This work has the LPPL maintenance status "maintained".

The Current Maintainers of this work are
Heiko Oberdiek and the Oberdiek Package Support Group
https://github.com/ho-tex/oberdiek/issues


The Base Interpreter refers to any `TeX-Format',
because some files are installed in TDS:tex/generic//.

This work consists of the main source file protecteddef.dtx
and the derived files
   protecteddef.sty, protecteddef.pdf, protecteddef.ins, protecteddef.drv,
   protecteddef-test1.tex, protecteddef-test2.tex.

\endpreamble
\let\MetaPrefix\DoubleperCent

\generate{%
  \file{protecteddef.ins}{\from{protecteddef.dtx}{install}}%
  \file{protecteddef.drv}{\from{protecteddef.dtx}{driver}}%
  \usedir{tex/generic/oberdiek}%
  \file{protecteddef.sty}{\from{protecteddef.dtx}{package}}%
%  \usedir{doc/latex/oberdiek/test}%
%  \file{protecteddef-test1.tex}{\from{protecteddef.dtx}{test1}}%
%  \file{protecteddef-test2.tex}{\from{protecteddef.dtx}{test2}}%
  \nopreamble
  \nopostamble
%  \usedir{source/latex/oberdiek/catalogue}%
%  \file{protecteddef.xml}{\from{protecteddef.dtx}{catalogue}}%
}

\catcode32=13\relax% active space
\let =\space%
\Msg{************************************************************************}
\Msg{*}
\Msg{* To finish the installation you have to move the following}
\Msg{* file into a directory searched by TeX:}
\Msg{*}
\Msg{*     protecteddef.sty}
\Msg{*}
\Msg{* To produce the documentation run the file `protecteddef.drv'}
\Msg{* through LaTeX.}
\Msg{*}
\Msg{* Happy TeXing!}
\Msg{*}
\Msg{************************************************************************}

\endbatchfile
%</install>
%<*ignore>
\fi
%</ignore>
%<*driver>
\NeedsTeXFormat{LaTeX2e}
\ProvidesFile{protecteddef.drv}%
  [2016/05/16 v1.1 Define protected commands (HO)]%
\documentclass{ltxdoc}
\usepackage{holtxdoc}[2011/11/22]
\begin{document}
  \DocInput{protecteddef.dtx}%
\end{document}
%</driver>
% \fi
%
%
% \CharacterTable
%  {Upper-case    \A\B\C\D\E\F\G\H\I\J\K\L\M\N\O\P\Q\R\S\T\U\V\W\X\Y\Z
%   Lower-case    \a\b\c\d\e\f\g\h\i\j\k\l\m\n\o\p\q\r\s\t\u\v\w\x\y\z
%   Digits        \0\1\2\3\4\5\6\7\8\9
%   Exclamation   \!     Double quote  \"     Hash (number) \#
%   Dollar        \$     Percent       \%     Ampersand     \&
%   Acute accent  \'     Left paren    \(     Right paren   \)
%   Asterisk      \*     Plus          \+     Comma         \,
%   Minus         \-     Point         \.     Solidus       \/
%   Colon         \:     Semicolon     \;     Less than     \<
%   Equals        \=     Greater than  \>     Question mark \?
%   Commercial at \@     Left bracket  \[     Backslash     \\
%   Right bracket \]     Circumflex    \^     Underscore    \_
%   Grave accent  \`     Left brace    \{     Vertical bar  \|
%   Right brace   \}     Tilde         \~}
%
% \GetFileInfo{protecteddef.drv}
%
% \title{The \xpackage{protecteddef} package}
% \date{2016/05/16 v1.1}
% \author{Heiko Oberdiek\thanks
% {Please report any issues at \url{https://github.com/ho-tex/oberdiek/issues}}}
%
% \maketitle
%
% \begin{abstract}
% This packages provides \cs{ProtectedDef} for defining
% robust macros for both \hologo{plainTeX} and \hologo{LaTeX}.
% First \hologo{eTeX}'s \cs{protected} is tried, then
% \hologo{LaTeX}'s \cs{DeclareRobustCommand} is used.
% Otherwise the macro is not made robust.
% \end{abstract}
%
% \tableofcontents
%
% \section{Documentation}
%
% Many of my packages work for both formats \hologo{plainTeX}
% and \hologo{LaTeX}, even \hologo{iniTeX} is often supported.
% It would be nice if fragile macros could be protected and
% made robust. However the different format worlds offer different
% solutions.
%
% \subsection{The \hologo{LaTeX}'s way}
%
% Usually \cs{newcommand} is used to define macros. It provides
% a check if the command to be defined is already defined or
% cannot be defined for other reasons.
%
% For making robust macros \hologo{LaTeX} provides \cs{DeclareRobustCommand}.
% It shares the syntax with \cs{newcommand}. However it does not provide
% latters check. Internally the check is available via \cs{@ifdefinable}.
%
% Internally the robust macro is using \cs{protect} with a nested
% macro definition. The \cs{protect} infrastructure is a feature of
% \cs{LaTeX} and usually not available in other formats.
%
% \subsection{The \hologo{eTeX}'s way}
%
% The need for robust macros is addressed in \cs{eTeX}. It provides
% \cs{protected} that modifies the behaviour of \cs{def} in a similar
% way as \cs{long}. A protected macro does not expand in
% some expandable contexts like writing to a file or \cs{edef}.
%
% \subsection{The way of this package}
%
% The package tries to find the available protection mechanism.
% First it looks for \cs{eTeX}'s \cs{protected}, then
% it uses \hologo{LaTeX}'s \cs{DeclareRobustCommand}.
% If both fails, then the macro remains unprotected.
%
% Additionally, \cs{LaTeX}'s check, if a macro is already defined
% is added in all cases. First \hologo{LaTeX}'s \cs{@ifdefinable}
% is tried to be compatible with \hologo{LaTeX}.
% If \cs{@ifdefinable} is not available, then the test is implemented
% by asserting that the macro is undefined or has the meaning of \cs{relax}.
% If the test fails, then in all cases the macro is not defined
% and an error is thrown.
%
% \subsection{Usage}
%
% \begin{declcs}{ProtectedDef} \* \M{cmd} [\meta{num}] \M{definition text}
% \end{declcs}
% Macro \cs{ProtectedDef} follows the syntax of \hologo{LaTeX}'s
% \cs{newcommand} with the exception that an optional argument
% is not supported. Macro \meta{cmd} is to be defined as
% \cs{long} macro without star with \meta{num} arguments.
%
% The number of arguments \meta{num} must be given as explicite digit
% |0| upto |9|. Otherwise the part between the argument \meta{cmd} and
% the \meta{definition text} is taken as parameter text in the syntax
% of vanilla \hologo{TeX}. Examples (with \cs{protected}):
% \begin{quote}
%   \def\to{$\Rightarrow$ }%
%   |\ProtectedDef*{\foo}[1]{\message{#1}}|\\
%   \to |\protected\def\foo#1{\message#1}}|\\[1ex]
%   |\ProtectedDef\foo{abc}|\\
%   \to |\protected\def\foo{abc}|\\[1ex]
%   |\ProtectedDef*\foo(#1)<#2>{#1/#2}|\\
%   \to |\protected\def\foo(#1)<#2>{#1/#2}|
% \end{quote}
%
% \StopEventually{
% }
%
% \section{Implementation}
%
%    \begin{macrocode}
%<*package>
%    \end{macrocode}
%
% \subsection{Reload check and package identification}
%    Reload check, especially if the package is not used with \LaTeX.
%    \begin{macrocode}
\begingroup\catcode61\catcode48\catcode32=10\relax%
  \catcode13=5 % ^^M
  \endlinechar=13 %
  \catcode35=6 % #
  \catcode39=12 % '
  \catcode44=12 % ,
  \catcode45=12 % -
  \catcode46=12 % .
  \catcode58=12 % :
  \catcode64=11 % @
  \catcode123=1 % {
  \catcode125=2 % }
  \expandafter\let\expandafter\x\csname ver@protecteddef.sty\endcsname
  \ifx\x\relax % plain-TeX, first loading
  \else
    \def\empty{}%
    \ifx\x\empty % LaTeX, first loading,
      % variable is initialized, but \ProvidesPackage not yet seen
    \else
      \expandafter\ifx\csname PackageInfo\endcsname\relax
        \def\x#1#2{%
          \immediate\write-1{Package #1 Info: #2.}%
        }%
      \else
        \def\x#1#2{\PackageInfo{#1}{#2, stopped}}%
      \fi
      \x{protecteddef}{The package is already loaded}%
      \aftergroup\endinput
    \fi
  \fi
\endgroup%
%    \end{macrocode}
%    Package identification:
%    \begin{macrocode}
\begingroup\catcode61\catcode48\catcode32=10\relax%
  \catcode13=5 % ^^M
  \endlinechar=13 %
  \catcode35=6 % #
  \catcode39=12 % '
  \catcode40=12 % (
  \catcode41=12 % )
  \catcode44=12 % ,
  \catcode45=12 % -
  \catcode46=12 % .
  \catcode47=12 % /
  \catcode58=12 % :
  \catcode64=11 % @
  \catcode91=12 % [
  \catcode93=12 % ]
  \catcode123=1 % {
  \catcode125=2 % }
  \expandafter\ifx\csname ProvidesPackage\endcsname\relax
    \def\x#1#2#3[#4]{\endgroup
      \immediate\write-1{Package: #3 #4}%
      \xdef#1{#4}%
    }%
  \else
    \def\x#1#2[#3]{\endgroup
      #2[{#3}]%
      \ifx#1\@undefined
        \xdef#1{#3}%
      \fi
      \ifx#1\relax
        \xdef#1{#3}%
      \fi
    }%
  \fi
\expandafter\x\csname ver@protecteddef.sty\endcsname
\ProvidesPackage{protecteddef}%
  [2016/05/16 v1.1 Define protected commands (HO)]%
%    \end{macrocode}
%
% \subsection{Catcodes}
%
%    \begin{macrocode}
\begingroup\catcode61\catcode48\catcode32=10\relax%
  \catcode13=5 % ^^M
  \endlinechar=13 %
  \catcode123=1 % {
  \catcode125=2 % }
  \catcode64=11 % @
  \def\x{\endgroup
    \expandafter\edef\csname ProDef@AtEnd\endcsname{%
      \endlinechar=\the\endlinechar\relax
      \catcode13=\the\catcode13\relax
      \catcode32=\the\catcode32\relax
      \catcode35=\the\catcode35\relax
      \catcode61=\the\catcode61\relax
      \catcode64=\the\catcode64\relax
      \catcode123=\the\catcode123\relax
      \catcode125=\the\catcode125\relax
    }%
  }%
\x\catcode61\catcode48\catcode32=10\relax%
\catcode13=5 % ^^M
\endlinechar=13 %
\catcode35=6 % #
\catcode64=11 % @
\catcode123=1 % {
\catcode125=2 % }
\def\TMP@EnsureCode#1#2{%
  \edef\ProDef@AtEnd{%
    \ProDef@AtEnd
    \catcode#1=\the\catcode#1\relax
  }%
  \catcode#1=#2\relax
}
\TMP@EnsureCode{38}{4}% &
\TMP@EnsureCode{40}{12}% (
\TMP@EnsureCode{41}{12}% )
\TMP@EnsureCode{42}{12}% *
\TMP@EnsureCode{45}{12}% -
\TMP@EnsureCode{46}{12}% .
\TMP@EnsureCode{47}{12}% /
\TMP@EnsureCode{91}{12}% [
\TMP@EnsureCode{93}{12}% ]
\TMP@EnsureCode{96}{12}% `
\edef\ProDef@AtEnd{\ProDef@AtEnd\noexpand\endinput}
%    \end{macrocode}
%
% \subsection{Resources}
%
%    \begin{macrocode}
\begingroup\expandafter\expandafter\expandafter\endgroup
\expandafter\ifx\csname RequirePackage\endcsname\relax
  \def\TMP@RequirePackage#1[#2]{%
    \begingroup\expandafter\expandafter\expandafter\endgroup
    \expandafter\ifx\csname ver@#1.sty\endcsname\relax
      \input #1.sty\relax
    \fi
  }%
\else
  \let\TMP@RequirePackage\RequirePackage
\fi
\TMP@RequirePackage{ltxcmds}[2010/12/12]%
\TMP@RequirePackage{infwarerr}[2010/04/08]%
%    \end{macrocode}
%
%    \begin{macrocode}
\def\ProDef@temp#1{%
  \expandafter\def\csname ProDef@param[#1]\endcsname % hash-ok
}
\expandafter\def\csname ProDef@param\endcsname{}
\ProDef@temp0{}
\ProDef@temp1{##1}
\ProDef@temp2{##1##2}
\ProDef@temp3{##1##2##3}
\ProDef@temp4{##1##2##3##4}
\ProDef@temp5{##1##2##3##4##5}
\ProDef@temp6{##1##2##3##4##5##6}
\ProDef@temp7{##1##2##3##4##5##7}
\ProDef@temp8{##1##2##3##4##5##7##8}
\ProDef@temp9{##1##2##3##4##5##7##8##9}
%    \end{macrocode}
%
%    \begin{macro}{\ProDef@IfDefinable}
%    \begin{macrocode}
\ltx@IfUndefined{@ifdefinable}{%
  \long\def\ProDef@IfDefinable#1{%
    \begingroup
      \escapechar=-1 %
    \ltx@ifundefined{\string#1}{%
      \endgroup
      \ltx@firstofone
    }{%
      \expandafter\endgroup
      \expandafter
      \edef\expandafter\ProDef@temp\expandafter{\string#1 }%
      \@PackageError{protecteddef}{%
        Command \ltx@backslashchar\ProDef@temp already defined%
      }\@ehc
      \ltx@gobbletwo
    }%
  }%
}{%
  \long\def\ProDef@IfDefinable#1{%
    \let\ProDef@next\ltx@gobbletwo
    \@ifdefinable{#1}{%
      \let\ProDef@next\ltx@firstofone
    }%
    \ProDef@next
  }%
}
%    \end{macrocode}
%    \end{macro}
%
%    \begin{macrocode}
\begingroup\expandafter\expandafter\expandafter\endgroup
\expandafter\ifx\csname protected\endcsname\relax
  \begingroup\expandafter\expandafter\expandafter\endgroup
  \expandafter\ifx\csname DeclareRobustCommand\endcsname\relax
    \catcode`\&=14 % comment
  \else
    \newcommand*{\ProtectedDef}{%
      \ltx@ifnextchar*{%
        \ProDef@ProtectedDef
      }{%
        \ProDef@ProtectedDef{}%
      }%
    }%
    \long\def\ProDef@ProtectedDef#1#2#3#{%
      \ProDef@IfDefinable{#2}{%
        \ltx@IfUndefined{ProDef@param#3}{%
          \DeclareRobustCommand*{#2}{}%
          \begingroup
            \escapechar=-1 %
            \def\ProDef@temp{#1}%
          \edef\x{\endgroup
            \ifx\ProDef@temp\ltx@empty
              \noexpand\long
            \fi
            \noexpand\def
            \expandafter\noexpand\csname\string#2 \endcsname
          }%
          \x#3%
        }{%
          \DeclareRobustCommand#1{#2}#3%
        }%
      }%
    }%
    \expandafter\expandafter\expandafter\ProDef@AtEnd
  \fi
\else
  \catcode`\&=9 % ignore
\fi%
\ProDef@IfDefinable\ProtectedDef{%
&  \protected
  \def\ProtectedDef%
}{%
  \ltx@ifnextchar*{%
    \let\ProDef@long\ltx@empty
    \expandafter\ProDef@ProtectedDef\ltx@gobble
  }{%
    \let\ProDef@long\long
    \ProDef@ProtectedDef
  }%
}
\long\def\ProDef@ProtectedDef#1#2#{%
  \ProDef@IfDefinable{#1}{%
    \ltx@IfUndefined{ProDef@param#2}{%
&     \protected
      \ProDef@long
      \def#1#2%
    }{%
&     \protected
      \ProDef@long
      \expandafter\expandafter\expandafter\def
      \expandafter\expandafter\expandafter#1%
      \csname ProDef@param#2\endcsname
    }%
  }%
}
%    \end{macrocode}
%
%    \begin{macrocode}
\ProDef@AtEnd%
%    \end{macrocode}
%    \begin{macrocode}
%</package>
%    \end{macrocode}
%
% \section{Test}
%
% \subsection{Catcode checks for loading}
%
%    \begin{macrocode}
%<*test1>
%    \end{macrocode}
%    \begin{macrocode}
\catcode`\{=1 %
\catcode`\}=2 %
\catcode`\#=6 %
\catcode`\@=11 %
\expandafter\ifx\csname count@\endcsname\relax
  \countdef\count@=255 %
\fi
\expandafter\ifx\csname @gobble\endcsname\relax
  \long\def\@gobble#1{}%
\fi
\expandafter\ifx\csname @firstofone\endcsname\relax
  \long\def\@firstofone#1{#1}%
\fi
\expandafter\ifx\csname loop\endcsname\relax
  \expandafter\@firstofone
\else
  \expandafter\@gobble
\fi
{%
  \def\loop#1\repeat{%
    \def\body{#1}%
    \iterate
  }%
  \def\iterate{%
    \body
      \let\next\iterate
    \else
      \let\next\relax
    \fi
    \next
  }%
  \let\repeat=\fi
}%
\def\RestoreCatcodes{}
\count@=0 %
\loop
  \edef\RestoreCatcodes{%
    \RestoreCatcodes
    \catcode\the\count@=\the\catcode\count@\relax
  }%
\ifnum\count@<255 %
  \advance\count@ 1 %
\repeat

\def\RangeCatcodeInvalid#1#2{%
  \count@=#1\relax
  \loop
    \catcode\count@=15 %
  \ifnum\count@<#2\relax
    \advance\count@ 1 %
  \repeat
}
\def\RangeCatcodeCheck#1#2#3{%
  \count@=#1\relax
  \loop
    \ifnum#3=\catcode\count@
    \else
      \errmessage{%
        Character \the\count@\space
        with wrong catcode \the\catcode\count@\space
        instead of \number#3%
      }%
    \fi
  \ifnum\count@<#2\relax
    \advance\count@ 1 %
  \repeat
}
\def\space{ }
\expandafter\ifx\csname LoadCommand\endcsname\relax
  \def\LoadCommand{\input protecteddef.sty\relax}%
\fi
\def\Test{%
  \RangeCatcodeInvalid{0}{47}%
  \RangeCatcodeInvalid{58}{64}%
  \RangeCatcodeInvalid{91}{96}%
  \RangeCatcodeInvalid{123}{255}%
  \catcode`\@=12 %
  \catcode`\\=0 %
  \catcode`\%=14 %
  \LoadCommand
  \RangeCatcodeCheck{0}{36}{15}%
  \RangeCatcodeCheck{37}{37}{14}%
  \RangeCatcodeCheck{38}{47}{15}%
  \RangeCatcodeCheck{48}{57}{12}%
  \RangeCatcodeCheck{58}{63}{15}%
  \RangeCatcodeCheck{64}{64}{12}%
  \RangeCatcodeCheck{65}{90}{11}%
  \RangeCatcodeCheck{91}{91}{15}%
  \RangeCatcodeCheck{92}{92}{0}%
  \RangeCatcodeCheck{93}{96}{15}%
  \RangeCatcodeCheck{97}{122}{11}%
  \RangeCatcodeCheck{123}{255}{15}%
  \RestoreCatcodes
}
\Test
\csname @@end\endcsname
\end
%    \end{macrocode}
%    \begin{macrocode}
%</test1>
%    \end{macrocode}
%
% \subsection{Test without \hologo{LaTeX} and \cs{protected}}
%
%    \begin{macrocode}
%<*test2>
\errorcontextlines=10000 %
\begingroup\expandafter\expandafter\expandafter\endgroup
\expandafter\ifx\csname RequirePackage\endcsname\relax
  \input protecteddef.sty\relax
  \catcode`\{=1 %
  \catcode`\}=2 %
  \catcode`\#=6 %
\else
  \RequirePackage{protecteddef}[2016/05/16]%
\fi
\begingroup\expandafter\expandafter\expandafter\endgroup
\expandafter\ifx\csname protected\endcsname\relax
  \let\pdef\def
\else
  \def\pdef{\protected\def}%
\fi
\def\msg#{\immediate\write16}
\countdef\errcount=2 %
\long\def\BeginCheck#1\ProtectedDef#2\EndCheck{%
  \begingroup
    \toks0={\ProtectedDef#2}%
    \msg{<<\the\toks0>>}%
  \endgroup
  \setbox0=\hbox{%
    #1%
    \ProtectedDef#2%
    \check\foo
  }%
  \ifdim\wd0=0pt\relax
  \else
    \errmessage{[Definition] Unwanted spaces?!}%
  \fi
  \setbox0=\hbox{%
    \def\fooinitial{XYZ}%
    \let\foo\fooinitial
    \errcount=0 %
    \expandafter\def\csname @PackageError\endcsname##1##2##3{%
      \advance\errcount by 1 %
    }%
    \expandafter\def\csname @notdefinable\endcsname{%
      \advance\errcount by 1 %
    }%
    \ProtectedDef#2%
    \ifnum\errcount=1 %
    \else
      \errmessage{1 error expected, but found: \the\errcount}%
    \fi
    \ifx\foo\fooinitial
    \else
      \def\space{ }%
      \errmessage{\string\foo\space is overwritten}%
    \fi
  }%
  \ifdim\wd0=0pt\relax
  \else
    \errmessage{[Error] Unwanted spaces?!}%
  \fi
}
\chardef\DeclareVersion=0 %
\begingroup\expandafter\expandafter\expandafter\endgroup
\expandafter\ifx\csname protected\endcsname\relax
  \begingroup\expandafter\expandafter\expandafter\endgroup
  \expandafter\ifx\csname DeclareRobustCommand\endcsname\relax
  \else
    \chardef\DeclareVersion=1 %
  \fi
\fi
\ifnum\DeclareVersion=0 %
  \def\check#1{%
    \ifx\cmp#1%
      \msg{* Test passed.}%
    \else
      \msg{}%
      \msg{[\meaning#1]}%
      \msg{[\meaning\cmp]}%
      \errmessage{Test failed!}%
    \fi
  }%
\else
  \def\check#1{%
    \begingroup
      \escapechar=-1 %
    \edef\x{\endgroup
      \def\noexpand\cs/{\string#1}%
    }\x
    \edef\CMP{%
      \noexpand\protect
      \expandafter\noexpand\csname\cs/ \endcsname
    }%
    \ifx\CMP#1%
      \expandafter\ifx\csname\cs/ \endcsname\cmp
        \msg{Test passed.}%
      \else
        \msg{}%
        \msg{[\expandafter\meaning\csname\cs/ \endcsname]}%
        \msg{[\meaning\cmp]}%
        \errmessage{Test failed!}%
      \fi
    \else
      \msg{}%
      \msg{[\meaning#1]}%
      \msg{[\meaning\CMP]}%
      \errmessage{Test failed!}%
    \fi
  }%
\fi

\tracingmacros=1

\BeginCheck
  \pdef\cmp{}%
  \ProtectedDef*\foo{}%
\EndCheck

\BeginCheck
  \pdef\cmp{}%
  \ProtectedDef*\foo[0]{}%
\EndCheck

\BeginCheck
  \pdef\cmp#1{<#1>}%
  \ProtectedDef*\foo[1]{<#1>}%
\EndCheck

\BeginCheck
  \pdef\cmp(#1){<#1>}%
  \ProtectedDef*\foo(#1){<#1>}%
\EndCheck

\BeginCheck
  \long\pdef\cmp{}%
  \ProtectedDef\foo{}%
\EndCheck

\BeginCheck
  \long\pdef\cmp{}%
  \ProtectedDef\foo[0]{}%
\EndCheck

\BeginCheck
  \long\pdef\cmp#1{<#1>}%
  \ProtectedDef\foo[1]{<#1>}%
\EndCheck

\BeginCheck
  \long\pdef\cmp(#1){<#1>}%
  \ProtectedDef\foo(#1){<#1>}%
\EndCheck

\csname @@end\endcsname\end
%</test2>
%    \end{macrocode}
%
% \section{Installation}
%
% \subsection{Download}
%
% \paragraph{Package.} This package is available on
% CTAN\footnote{\CTANpkg{protecteddef}}:
% \begin{description}
% \item[\CTAN{macros/latex/contrib/oberdiek/protecteddef.dtx}] The source file.
% \item[\CTAN{macros/latex/contrib/oberdiek/protecteddef.pdf}] Documentation.
% \end{description}
%
%
% \paragraph{Bundle.} All the packages of the bundle `oberdiek'
% are also available in a TDS compliant ZIP archive. There
% the packages are already unpacked and the documentation files
% are generated. The files and directories obey the TDS standard.
% \begin{description}
% \item[\CTANinstall{install/macros/latex/contrib/oberdiek.tds.zip}]
% \end{description}
% \emph{TDS} refers to the standard ``A Directory Structure
% for \TeX\ Files'' (\CTAN{tds/tds.pdf}). Directories
% with \xfile{texmf} in their name are usually organized this way.
%
% \subsection{Bundle installation}
%
% \paragraph{Unpacking.} Unpack the \xfile{oberdiek.tds.zip} in the
% TDS tree (also known as \xfile{texmf} tree) of your choice.
% Example (linux):
% \begin{quote}
%   |unzip oberdiek.tds.zip -d ~/texmf|
% \end{quote}
%
% \paragraph{Script installation.}
% Check the directory \xfile{TDS:scripts/oberdiek/} for
% scripts that need further installation steps.
%
% \subsection{Package installation}
%
% \paragraph{Unpacking.} The \xfile{.dtx} file is a self-extracting
% \docstrip\ archive. The files are extracted by running the
% \xfile{.dtx} through \plainTeX:
% \begin{quote}
%   \verb|tex protecteddef.dtx|
% \end{quote}
%
% \paragraph{TDS.} Now the different files must be moved into
% the different directories in your installation TDS tree
% (also known as \xfile{texmf} tree):
% \begin{quote}
% \def\t{^^A
% \begin{tabular}{@{}>{\ttfamily}l@{ $\rightarrow$ }>{\ttfamily}l@{}}
%   protecteddef.sty & tex/generic/oberdiek/protecteddef.sty\\
%   protecteddef.pdf & doc/latex/oberdiek/protecteddef.pdf\\
%   test/protecteddef-test1.tex & doc/latex/oberdiek/test/protecteddef-test1.tex\\
%   test/protecteddef-test2.tex & doc/latex/oberdiek/test/protecteddef-test2.tex\\
%   protecteddef.dtx & source/latex/oberdiek/protecteddef.dtx\\
% \end{tabular}^^A
% }^^A
% \sbox0{\t}^^A
% \ifdim\wd0>\linewidth
%   \begingroup
%     \advance\linewidth by\leftmargin
%     \advance\linewidth by\rightmargin
%   \edef\x{\endgroup
%     \def\noexpand\lw{\the\linewidth}^^A
%   }\x
%   \def\lwbox{^^A
%     \leavevmode
%     \hbox to \linewidth{^^A
%       \kern-\leftmargin\relax
%       \hss
%       \usebox0
%       \hss
%       \kern-\rightmargin\relax
%     }^^A
%   }^^A
%   \ifdim\wd0>\lw
%     \sbox0{\small\t}^^A
%     \ifdim\wd0>\linewidth
%       \ifdim\wd0>\lw
%         \sbox0{\footnotesize\t}^^A
%         \ifdim\wd0>\linewidth
%           \ifdim\wd0>\lw
%             \sbox0{\scriptsize\t}^^A
%             \ifdim\wd0>\linewidth
%               \ifdim\wd0>\lw
%                 \sbox0{\tiny\t}^^A
%                 \ifdim\wd0>\linewidth
%                   \lwbox
%                 \else
%                   \usebox0
%                 \fi
%               \else
%                 \lwbox
%               \fi
%             \else
%               \usebox0
%             \fi
%           \else
%             \lwbox
%           \fi
%         \else
%           \usebox0
%         \fi
%       \else
%         \lwbox
%       \fi
%     \else
%       \usebox0
%     \fi
%   \else
%     \lwbox
%   \fi
% \else
%   \usebox0
% \fi
% \end{quote}
% If you have a \xfile{docstrip.cfg} that configures and enables \docstrip's
% TDS installing feature, then some files can already be in the right
% place, see the documentation of \docstrip.
%
% \subsection{Refresh file name databases}
%
% If your \TeX~distribution
% (\TeX\,Live, \mikTeX, \dots) relies on file name databases, you must refresh
% these. For example, \TeX\,Live\ users run \verb|texhash| or
% \verb|mktexlsr|.
%
% \subsection{Some details for the interested}
%
% \paragraph{Unpacking with \LaTeX.}
% The \xfile{.dtx} chooses its action depending on the format:
% \begin{description}
% \item[\plainTeX:] Run \docstrip\ and extract the files.
% \item[\LaTeX:] Generate the documentation.
% \end{description}
% If you insist on using \LaTeX\ for \docstrip\ (really,
% \docstrip\ does not need \LaTeX), then inform the autodetect routine
% about your intention:
% \begin{quote}
%   \verb|latex \let\install=y% \iffalse meta-comment
%
% File: protecteddef.dtx
% Version: 2016/05/16 v1.1
% Info: Define protected commands
%
% Copyright (C)
%    2011 Heiko Oberdiek
%    2016-2019 Oberdiek Package Support Group
%    https://github.com/ho-tex/oberdiek/issues
%
% This work may be distributed and/or modified under the
% conditions of the LaTeX Project Public License, either
% version 1.3c of this license or (at your option) any later
% version. This version of this license is in
%    https://www.latex-project.org/lppl/lppl-1-3c.txt
% and the latest version of this license is in
%    https://www.latex-project.org/lppl.txt
% and version 1.3 or later is part of all distributions of
% LaTeX version 2005/12/01 or later.
%
% This work has the LPPL maintenance status "maintained".
%
% The Current Maintainers of this work are
% Heiko Oberdiek and the Oberdiek Package Support Group
% https://github.com/ho-tex/oberdiek/issues
%
% The Base Interpreter refers to any `TeX-Format',
% because some files are installed in TDS:tex/generic//.
%
% This work consists of the main source file protecteddef.dtx
% and the derived files
%    protecteddef.sty, protecteddef.pdf, protecteddef.ins, protecteddef.drv,
%    protecteddef-test1.tex, protecteddef-test2.tex.
%
% Distribution:
%    CTAN:macros/latex/contrib/oberdiek/protecteddef.dtx
%    CTAN:macros/latex/contrib/oberdiek/protecteddef.pdf
%
% Unpacking:
%    (a) If protecteddef.ins is present:
%           tex protecteddef.ins
%    (b) Without protecteddef.ins:
%           tex protecteddef.dtx
%    (c) If you insist on using LaTeX
%           latex \let\install=y\input{protecteddef.dtx}
%        (quote the arguments according to the demands of your shell)
%
% Documentation:
%    (a) If protecteddef.drv is present:
%           latex protecteddef.drv
%    (b) Without protecteddef.drv:
%           latex protecteddef.dtx; ...
%    The class ltxdoc loads the configuration file ltxdoc.cfg
%    if available. Here you can specify further options, e.g.
%    use A4 as paper format:
%       \PassOptionsToClass{a4paper}{article}
%
%    Programm calls to get the documentation (example):
%       pdflatex protecteddef.dtx
%       makeindex -s gind.ist protecteddef.idx
%       pdflatex protecteddef.dtx
%       makeindex -s gind.ist protecteddef.idx
%       pdflatex protecteddef.dtx
%
% Installation:
%    TDS:tex/generic/oberdiek/protecteddef.sty
%    TDS:doc/latex/oberdiek/protecteddef.pdf
%    TDS:doc/latex/oberdiek/test/protecteddef-test1.tex
%    TDS:doc/latex/oberdiek/test/protecteddef-test2.tex
%    TDS:source/latex/oberdiek/protecteddef.dtx
%
%<*ignore>
\begingroup
  \catcode123=1 %
  \catcode125=2 %
  \def\x{LaTeX2e}%
\expandafter\endgroup
\ifcase 0\ifx\install y1\fi\expandafter
         \ifx\csname processbatchFile\endcsname\relax\else1\fi
         \ifx\fmtname\x\else 1\fi\relax
\else\csname fi\endcsname
%</ignore>
%<*install>
\input docstrip.tex
\Msg{************************************************************************}
\Msg{* Installation}
\Msg{* Package: protecteddef 2016/05/16 v1.1 Define protected commands (HO)}
\Msg{************************************************************************}

\keepsilent
\askforoverwritefalse

\let\MetaPrefix\relax
\preamble

This is a generated file.

Project: protecteddef
Version: 2016/05/16 v1.1

Copyright (C)
   2011 Heiko Oberdiek
   2016-2019 Oberdiek Package Support Group

This work may be distributed and/or modified under the
conditions of the LaTeX Project Public License, either
version 1.3c of this license or (at your option) any later
version. This version of this license is in
   https://www.latex-project.org/lppl/lppl-1-3c.txt
and the latest version of this license is in
   https://www.latex-project.org/lppl.txt
and version 1.3 or later is part of all distributions of
LaTeX version 2005/12/01 or later.

This work has the LPPL maintenance status "maintained".

The Current Maintainers of this work are
Heiko Oberdiek and the Oberdiek Package Support Group
https://github.com/ho-tex/oberdiek/issues


The Base Interpreter refers to any `TeX-Format',
because some files are installed in TDS:tex/generic//.

This work consists of the main source file protecteddef.dtx
and the derived files
   protecteddef.sty, protecteddef.pdf, protecteddef.ins, protecteddef.drv,
   protecteddef-test1.tex, protecteddef-test2.tex.

\endpreamble
\let\MetaPrefix\DoubleperCent

\generate{%
  \file{protecteddef.ins}{\from{protecteddef.dtx}{install}}%
  \file{protecteddef.drv}{\from{protecteddef.dtx}{driver}}%
  \usedir{tex/generic/oberdiek}%
  \file{protecteddef.sty}{\from{protecteddef.dtx}{package}}%
%  \usedir{doc/latex/oberdiek/test}%
%  \file{protecteddef-test1.tex}{\from{protecteddef.dtx}{test1}}%
%  \file{protecteddef-test2.tex}{\from{protecteddef.dtx}{test2}}%
  \nopreamble
  \nopostamble
%  \usedir{source/latex/oberdiek/catalogue}%
%  \file{protecteddef.xml}{\from{protecteddef.dtx}{catalogue}}%
}

\catcode32=13\relax% active space
\let =\space%
\Msg{************************************************************************}
\Msg{*}
\Msg{* To finish the installation you have to move the following}
\Msg{* file into a directory searched by TeX:}
\Msg{*}
\Msg{*     protecteddef.sty}
\Msg{*}
\Msg{* To produce the documentation run the file `protecteddef.drv'}
\Msg{* through LaTeX.}
\Msg{*}
\Msg{* Happy TeXing!}
\Msg{*}
\Msg{************************************************************************}

\endbatchfile
%</install>
%<*ignore>
\fi
%</ignore>
%<*driver>
\NeedsTeXFormat{LaTeX2e}
\ProvidesFile{protecteddef.drv}%
  [2016/05/16 v1.1 Define protected commands (HO)]%
\documentclass{ltxdoc}
\usepackage{holtxdoc}[2011/11/22]
\begin{document}
  \DocInput{protecteddef.dtx}%
\end{document}
%</driver>
% \fi
%
%
% \CharacterTable
%  {Upper-case    \A\B\C\D\E\F\G\H\I\J\K\L\M\N\O\P\Q\R\S\T\U\V\W\X\Y\Z
%   Lower-case    \a\b\c\d\e\f\g\h\i\j\k\l\m\n\o\p\q\r\s\t\u\v\w\x\y\z
%   Digits        \0\1\2\3\4\5\6\7\8\9
%   Exclamation   \!     Double quote  \"     Hash (number) \#
%   Dollar        \$     Percent       \%     Ampersand     \&
%   Acute accent  \'     Left paren    \(     Right paren   \)
%   Asterisk      \*     Plus          \+     Comma         \,
%   Minus         \-     Point         \.     Solidus       \/
%   Colon         \:     Semicolon     \;     Less than     \<
%   Equals        \=     Greater than  \>     Question mark \?
%   Commercial at \@     Left bracket  \[     Backslash     \\
%   Right bracket \]     Circumflex    \^     Underscore    \_
%   Grave accent  \`     Left brace    \{     Vertical bar  \|
%   Right brace   \}     Tilde         \~}
%
% \GetFileInfo{protecteddef.drv}
%
% \title{The \xpackage{protecteddef} package}
% \date{2016/05/16 v1.1}
% \author{Heiko Oberdiek\thanks
% {Please report any issues at \url{https://github.com/ho-tex/oberdiek/issues}}}
%
% \maketitle
%
% \begin{abstract}
% This packages provides \cs{ProtectedDef} for defining
% robust macros for both \hologo{plainTeX} and \hologo{LaTeX}.
% First \hologo{eTeX}'s \cs{protected} is tried, then
% \hologo{LaTeX}'s \cs{DeclareRobustCommand} is used.
% Otherwise the macro is not made robust.
% \end{abstract}
%
% \tableofcontents
%
% \section{Documentation}
%
% Many of my packages work for both formats \hologo{plainTeX}
% and \hologo{LaTeX}, even \hologo{iniTeX} is often supported.
% It would be nice if fragile macros could be protected and
% made robust. However the different format worlds offer different
% solutions.
%
% \subsection{The \hologo{LaTeX}'s way}
%
% Usually \cs{newcommand} is used to define macros. It provides
% a check if the command to be defined is already defined or
% cannot be defined for other reasons.
%
% For making robust macros \hologo{LaTeX} provides \cs{DeclareRobustCommand}.
% It shares the syntax with \cs{newcommand}. However it does not provide
% latters check. Internally the check is available via \cs{@ifdefinable}.
%
% Internally the robust macro is using \cs{protect} with a nested
% macro definition. The \cs{protect} infrastructure is a feature of
% \cs{LaTeX} and usually not available in other formats.
%
% \subsection{The \hologo{eTeX}'s way}
%
% The need for robust macros is addressed in \cs{eTeX}. It provides
% \cs{protected} that modifies the behaviour of \cs{def} in a similar
% way as \cs{long}. A protected macro does not expand in
% some expandable contexts like writing to a file or \cs{edef}.
%
% \subsection{The way of this package}
%
% The package tries to find the available protection mechanism.
% First it looks for \cs{eTeX}'s \cs{protected}, then
% it uses \hologo{LaTeX}'s \cs{DeclareRobustCommand}.
% If both fails, then the macro remains unprotected.
%
% Additionally, \cs{LaTeX}'s check, if a macro is already defined
% is added in all cases. First \hologo{LaTeX}'s \cs{@ifdefinable}
% is tried to be compatible with \hologo{LaTeX}.
% If \cs{@ifdefinable} is not available, then the test is implemented
% by asserting that the macro is undefined or has the meaning of \cs{relax}.
% If the test fails, then in all cases the macro is not defined
% and an error is thrown.
%
% \subsection{Usage}
%
% \begin{declcs}{ProtectedDef} \* \M{cmd} [\meta{num}] \M{definition text}
% \end{declcs}
% Macro \cs{ProtectedDef} follows the syntax of \hologo{LaTeX}'s
% \cs{newcommand} with the exception that an optional argument
% is not supported. Macro \meta{cmd} is to be defined as
% \cs{long} macro without star with \meta{num} arguments.
%
% The number of arguments \meta{num} must be given as explicite digit
% |0| upto |9|. Otherwise the part between the argument \meta{cmd} and
% the \meta{definition text} is taken as parameter text in the syntax
% of vanilla \hologo{TeX}. Examples (with \cs{protected}):
% \begin{quote}
%   \def\to{$\Rightarrow$ }%
%   |\ProtectedDef*{\foo}[1]{\message{#1}}|\\
%   \to |\protected\def\foo#1{\message#1}}|\\[1ex]
%   |\ProtectedDef\foo{abc}|\\
%   \to |\protected\def\foo{abc}|\\[1ex]
%   |\ProtectedDef*\foo(#1)<#2>{#1/#2}|\\
%   \to |\protected\def\foo(#1)<#2>{#1/#2}|
% \end{quote}
%
% \StopEventually{
% }
%
% \section{Implementation}
%
%    \begin{macrocode}
%<*package>
%    \end{macrocode}
%
% \subsection{Reload check and package identification}
%    Reload check, especially if the package is not used with \LaTeX.
%    \begin{macrocode}
\begingroup\catcode61\catcode48\catcode32=10\relax%
  \catcode13=5 % ^^M
  \endlinechar=13 %
  \catcode35=6 % #
  \catcode39=12 % '
  \catcode44=12 % ,
  \catcode45=12 % -
  \catcode46=12 % .
  \catcode58=12 % :
  \catcode64=11 % @
  \catcode123=1 % {
  \catcode125=2 % }
  \expandafter\let\expandafter\x\csname ver@protecteddef.sty\endcsname
  \ifx\x\relax % plain-TeX, first loading
  \else
    \def\empty{}%
    \ifx\x\empty % LaTeX, first loading,
      % variable is initialized, but \ProvidesPackage not yet seen
    \else
      \expandafter\ifx\csname PackageInfo\endcsname\relax
        \def\x#1#2{%
          \immediate\write-1{Package #1 Info: #2.}%
        }%
      \else
        \def\x#1#2{\PackageInfo{#1}{#2, stopped}}%
      \fi
      \x{protecteddef}{The package is already loaded}%
      \aftergroup\endinput
    \fi
  \fi
\endgroup%
%    \end{macrocode}
%    Package identification:
%    \begin{macrocode}
\begingroup\catcode61\catcode48\catcode32=10\relax%
  \catcode13=5 % ^^M
  \endlinechar=13 %
  \catcode35=6 % #
  \catcode39=12 % '
  \catcode40=12 % (
  \catcode41=12 % )
  \catcode44=12 % ,
  \catcode45=12 % -
  \catcode46=12 % .
  \catcode47=12 % /
  \catcode58=12 % :
  \catcode64=11 % @
  \catcode91=12 % [
  \catcode93=12 % ]
  \catcode123=1 % {
  \catcode125=2 % }
  \expandafter\ifx\csname ProvidesPackage\endcsname\relax
    \def\x#1#2#3[#4]{\endgroup
      \immediate\write-1{Package: #3 #4}%
      \xdef#1{#4}%
    }%
  \else
    \def\x#1#2[#3]{\endgroup
      #2[{#3}]%
      \ifx#1\@undefined
        \xdef#1{#3}%
      \fi
      \ifx#1\relax
        \xdef#1{#3}%
      \fi
    }%
  \fi
\expandafter\x\csname ver@protecteddef.sty\endcsname
\ProvidesPackage{protecteddef}%
  [2016/05/16 v1.1 Define protected commands (HO)]%
%    \end{macrocode}
%
% \subsection{Catcodes}
%
%    \begin{macrocode}
\begingroup\catcode61\catcode48\catcode32=10\relax%
  \catcode13=5 % ^^M
  \endlinechar=13 %
  \catcode123=1 % {
  \catcode125=2 % }
  \catcode64=11 % @
  \def\x{\endgroup
    \expandafter\edef\csname ProDef@AtEnd\endcsname{%
      \endlinechar=\the\endlinechar\relax
      \catcode13=\the\catcode13\relax
      \catcode32=\the\catcode32\relax
      \catcode35=\the\catcode35\relax
      \catcode61=\the\catcode61\relax
      \catcode64=\the\catcode64\relax
      \catcode123=\the\catcode123\relax
      \catcode125=\the\catcode125\relax
    }%
  }%
\x\catcode61\catcode48\catcode32=10\relax%
\catcode13=5 % ^^M
\endlinechar=13 %
\catcode35=6 % #
\catcode64=11 % @
\catcode123=1 % {
\catcode125=2 % }
\def\TMP@EnsureCode#1#2{%
  \edef\ProDef@AtEnd{%
    \ProDef@AtEnd
    \catcode#1=\the\catcode#1\relax
  }%
  \catcode#1=#2\relax
}
\TMP@EnsureCode{38}{4}% &
\TMP@EnsureCode{40}{12}% (
\TMP@EnsureCode{41}{12}% )
\TMP@EnsureCode{42}{12}% *
\TMP@EnsureCode{45}{12}% -
\TMP@EnsureCode{46}{12}% .
\TMP@EnsureCode{47}{12}% /
\TMP@EnsureCode{91}{12}% [
\TMP@EnsureCode{93}{12}% ]
\TMP@EnsureCode{96}{12}% `
\edef\ProDef@AtEnd{\ProDef@AtEnd\noexpand\endinput}
%    \end{macrocode}
%
% \subsection{Resources}
%
%    \begin{macrocode}
\begingroup\expandafter\expandafter\expandafter\endgroup
\expandafter\ifx\csname RequirePackage\endcsname\relax
  \def\TMP@RequirePackage#1[#2]{%
    \begingroup\expandafter\expandafter\expandafter\endgroup
    \expandafter\ifx\csname ver@#1.sty\endcsname\relax
      \input #1.sty\relax
    \fi
  }%
\else
  \let\TMP@RequirePackage\RequirePackage
\fi
\TMP@RequirePackage{ltxcmds}[2010/12/12]%
\TMP@RequirePackage{infwarerr}[2010/04/08]%
%    \end{macrocode}
%
%    \begin{macrocode}
\def\ProDef@temp#1{%
  \expandafter\def\csname ProDef@param[#1]\endcsname % hash-ok
}
\expandafter\def\csname ProDef@param\endcsname{}
\ProDef@temp0{}
\ProDef@temp1{##1}
\ProDef@temp2{##1##2}
\ProDef@temp3{##1##2##3}
\ProDef@temp4{##1##2##3##4}
\ProDef@temp5{##1##2##3##4##5}
\ProDef@temp6{##1##2##3##4##5##6}
\ProDef@temp7{##1##2##3##4##5##7}
\ProDef@temp8{##1##2##3##4##5##7##8}
\ProDef@temp9{##1##2##3##4##5##7##8##9}
%    \end{macrocode}
%
%    \begin{macro}{\ProDef@IfDefinable}
%    \begin{macrocode}
\ltx@IfUndefined{@ifdefinable}{%
  \long\def\ProDef@IfDefinable#1{%
    \begingroup
      \escapechar=-1 %
    \ltx@ifundefined{\string#1}{%
      \endgroup
      \ltx@firstofone
    }{%
      \expandafter\endgroup
      \expandafter
      \edef\expandafter\ProDef@temp\expandafter{\string#1 }%
      \@PackageError{protecteddef}{%
        Command \ltx@backslashchar\ProDef@temp already defined%
      }\@ehc
      \ltx@gobbletwo
    }%
  }%
}{%
  \long\def\ProDef@IfDefinable#1{%
    \let\ProDef@next\ltx@gobbletwo
    \@ifdefinable{#1}{%
      \let\ProDef@next\ltx@firstofone
    }%
    \ProDef@next
  }%
}
%    \end{macrocode}
%    \end{macro}
%
%    \begin{macrocode}
\begingroup\expandafter\expandafter\expandafter\endgroup
\expandafter\ifx\csname protected\endcsname\relax
  \begingroup\expandafter\expandafter\expandafter\endgroup
  \expandafter\ifx\csname DeclareRobustCommand\endcsname\relax
    \catcode`\&=14 % comment
  \else
    \newcommand*{\ProtectedDef}{%
      \ltx@ifnextchar*{%
        \ProDef@ProtectedDef
      }{%
        \ProDef@ProtectedDef{}%
      }%
    }%
    \long\def\ProDef@ProtectedDef#1#2#3#{%
      \ProDef@IfDefinable{#2}{%
        \ltx@IfUndefined{ProDef@param#3}{%
          \DeclareRobustCommand*{#2}{}%
          \begingroup
            \escapechar=-1 %
            \def\ProDef@temp{#1}%
          \edef\x{\endgroup
            \ifx\ProDef@temp\ltx@empty
              \noexpand\long
            \fi
            \noexpand\def
            \expandafter\noexpand\csname\string#2 \endcsname
          }%
          \x#3%
        }{%
          \DeclareRobustCommand#1{#2}#3%
        }%
      }%
    }%
    \expandafter\expandafter\expandafter\ProDef@AtEnd
  \fi
\else
  \catcode`\&=9 % ignore
\fi%
\ProDef@IfDefinable\ProtectedDef{%
&  \protected
  \def\ProtectedDef%
}{%
  \ltx@ifnextchar*{%
    \let\ProDef@long\ltx@empty
    \expandafter\ProDef@ProtectedDef\ltx@gobble
  }{%
    \let\ProDef@long\long
    \ProDef@ProtectedDef
  }%
}
\long\def\ProDef@ProtectedDef#1#2#{%
  \ProDef@IfDefinable{#1}{%
    \ltx@IfUndefined{ProDef@param#2}{%
&     \protected
      \ProDef@long
      \def#1#2%
    }{%
&     \protected
      \ProDef@long
      \expandafter\expandafter\expandafter\def
      \expandafter\expandafter\expandafter#1%
      \csname ProDef@param#2\endcsname
    }%
  }%
}
%    \end{macrocode}
%
%    \begin{macrocode}
\ProDef@AtEnd%
%    \end{macrocode}
%    \begin{macrocode}
%</package>
%    \end{macrocode}
%
% \section{Test}
%
% \subsection{Catcode checks for loading}
%
%    \begin{macrocode}
%<*test1>
%    \end{macrocode}
%    \begin{macrocode}
\catcode`\{=1 %
\catcode`\}=2 %
\catcode`\#=6 %
\catcode`\@=11 %
\expandafter\ifx\csname count@\endcsname\relax
  \countdef\count@=255 %
\fi
\expandafter\ifx\csname @gobble\endcsname\relax
  \long\def\@gobble#1{}%
\fi
\expandafter\ifx\csname @firstofone\endcsname\relax
  \long\def\@firstofone#1{#1}%
\fi
\expandafter\ifx\csname loop\endcsname\relax
  \expandafter\@firstofone
\else
  \expandafter\@gobble
\fi
{%
  \def\loop#1\repeat{%
    \def\body{#1}%
    \iterate
  }%
  \def\iterate{%
    \body
      \let\next\iterate
    \else
      \let\next\relax
    \fi
    \next
  }%
  \let\repeat=\fi
}%
\def\RestoreCatcodes{}
\count@=0 %
\loop
  \edef\RestoreCatcodes{%
    \RestoreCatcodes
    \catcode\the\count@=\the\catcode\count@\relax
  }%
\ifnum\count@<255 %
  \advance\count@ 1 %
\repeat

\def\RangeCatcodeInvalid#1#2{%
  \count@=#1\relax
  \loop
    \catcode\count@=15 %
  \ifnum\count@<#2\relax
    \advance\count@ 1 %
  \repeat
}
\def\RangeCatcodeCheck#1#2#3{%
  \count@=#1\relax
  \loop
    \ifnum#3=\catcode\count@
    \else
      \errmessage{%
        Character \the\count@\space
        with wrong catcode \the\catcode\count@\space
        instead of \number#3%
      }%
    \fi
  \ifnum\count@<#2\relax
    \advance\count@ 1 %
  \repeat
}
\def\space{ }
\expandafter\ifx\csname LoadCommand\endcsname\relax
  \def\LoadCommand{\input protecteddef.sty\relax}%
\fi
\def\Test{%
  \RangeCatcodeInvalid{0}{47}%
  \RangeCatcodeInvalid{58}{64}%
  \RangeCatcodeInvalid{91}{96}%
  \RangeCatcodeInvalid{123}{255}%
  \catcode`\@=12 %
  \catcode`\\=0 %
  \catcode`\%=14 %
  \LoadCommand
  \RangeCatcodeCheck{0}{36}{15}%
  \RangeCatcodeCheck{37}{37}{14}%
  \RangeCatcodeCheck{38}{47}{15}%
  \RangeCatcodeCheck{48}{57}{12}%
  \RangeCatcodeCheck{58}{63}{15}%
  \RangeCatcodeCheck{64}{64}{12}%
  \RangeCatcodeCheck{65}{90}{11}%
  \RangeCatcodeCheck{91}{91}{15}%
  \RangeCatcodeCheck{92}{92}{0}%
  \RangeCatcodeCheck{93}{96}{15}%
  \RangeCatcodeCheck{97}{122}{11}%
  \RangeCatcodeCheck{123}{255}{15}%
  \RestoreCatcodes
}
\Test
\csname @@end\endcsname
\end
%    \end{macrocode}
%    \begin{macrocode}
%</test1>
%    \end{macrocode}
%
% \subsection{Test without \hologo{LaTeX} and \cs{protected}}
%
%    \begin{macrocode}
%<*test2>
\errorcontextlines=10000 %
\begingroup\expandafter\expandafter\expandafter\endgroup
\expandafter\ifx\csname RequirePackage\endcsname\relax
  \input protecteddef.sty\relax
  \catcode`\{=1 %
  \catcode`\}=2 %
  \catcode`\#=6 %
\else
  \RequirePackage{protecteddef}[2016/05/16]%
\fi
\begingroup\expandafter\expandafter\expandafter\endgroup
\expandafter\ifx\csname protected\endcsname\relax
  \let\pdef\def
\else
  \def\pdef{\protected\def}%
\fi
\def\msg#{\immediate\write16}
\countdef\errcount=2 %
\long\def\BeginCheck#1\ProtectedDef#2\EndCheck{%
  \begingroup
    \toks0={\ProtectedDef#2}%
    \msg{<<\the\toks0>>}%
  \endgroup
  \setbox0=\hbox{%
    #1%
    \ProtectedDef#2%
    \check\foo
  }%
  \ifdim\wd0=0pt\relax
  \else
    \errmessage{[Definition] Unwanted spaces?!}%
  \fi
  \setbox0=\hbox{%
    \def\fooinitial{XYZ}%
    \let\foo\fooinitial
    \errcount=0 %
    \expandafter\def\csname @PackageError\endcsname##1##2##3{%
      \advance\errcount by 1 %
    }%
    \expandafter\def\csname @notdefinable\endcsname{%
      \advance\errcount by 1 %
    }%
    \ProtectedDef#2%
    \ifnum\errcount=1 %
    \else
      \errmessage{1 error expected, but found: \the\errcount}%
    \fi
    \ifx\foo\fooinitial
    \else
      \def\space{ }%
      \errmessage{\string\foo\space is overwritten}%
    \fi
  }%
  \ifdim\wd0=0pt\relax
  \else
    \errmessage{[Error] Unwanted spaces?!}%
  \fi
}
\chardef\DeclareVersion=0 %
\begingroup\expandafter\expandafter\expandafter\endgroup
\expandafter\ifx\csname protected\endcsname\relax
  \begingroup\expandafter\expandafter\expandafter\endgroup
  \expandafter\ifx\csname DeclareRobustCommand\endcsname\relax
  \else
    \chardef\DeclareVersion=1 %
  \fi
\fi
\ifnum\DeclareVersion=0 %
  \def\check#1{%
    \ifx\cmp#1%
      \msg{* Test passed.}%
    \else
      \msg{}%
      \msg{[\meaning#1]}%
      \msg{[\meaning\cmp]}%
      \errmessage{Test failed!}%
    \fi
  }%
\else
  \def\check#1{%
    \begingroup
      \escapechar=-1 %
    \edef\x{\endgroup
      \def\noexpand\cs/{\string#1}%
    }\x
    \edef\CMP{%
      \noexpand\protect
      \expandafter\noexpand\csname\cs/ \endcsname
    }%
    \ifx\CMP#1%
      \expandafter\ifx\csname\cs/ \endcsname\cmp
        \msg{Test passed.}%
      \else
        \msg{}%
        \msg{[\expandafter\meaning\csname\cs/ \endcsname]}%
        \msg{[\meaning\cmp]}%
        \errmessage{Test failed!}%
      \fi
    \else
      \msg{}%
      \msg{[\meaning#1]}%
      \msg{[\meaning\CMP]}%
      \errmessage{Test failed!}%
    \fi
  }%
\fi

\tracingmacros=1

\BeginCheck
  \pdef\cmp{}%
  \ProtectedDef*\foo{}%
\EndCheck

\BeginCheck
  \pdef\cmp{}%
  \ProtectedDef*\foo[0]{}%
\EndCheck

\BeginCheck
  \pdef\cmp#1{<#1>}%
  \ProtectedDef*\foo[1]{<#1>}%
\EndCheck

\BeginCheck
  \pdef\cmp(#1){<#1>}%
  \ProtectedDef*\foo(#1){<#1>}%
\EndCheck

\BeginCheck
  \long\pdef\cmp{}%
  \ProtectedDef\foo{}%
\EndCheck

\BeginCheck
  \long\pdef\cmp{}%
  \ProtectedDef\foo[0]{}%
\EndCheck

\BeginCheck
  \long\pdef\cmp#1{<#1>}%
  \ProtectedDef\foo[1]{<#1>}%
\EndCheck

\BeginCheck
  \long\pdef\cmp(#1){<#1>}%
  \ProtectedDef\foo(#1){<#1>}%
\EndCheck

\csname @@end\endcsname\end
%</test2>
%    \end{macrocode}
%
% \section{Installation}
%
% \subsection{Download}
%
% \paragraph{Package.} This package is available on
% CTAN\footnote{\CTANpkg{protecteddef}}:
% \begin{description}
% \item[\CTAN{macros/latex/contrib/oberdiek/protecteddef.dtx}] The source file.
% \item[\CTAN{macros/latex/contrib/oberdiek/protecteddef.pdf}] Documentation.
% \end{description}
%
%
% \paragraph{Bundle.} All the packages of the bundle `oberdiek'
% are also available in a TDS compliant ZIP archive. There
% the packages are already unpacked and the documentation files
% are generated. The files and directories obey the TDS standard.
% \begin{description}
% \item[\CTANinstall{install/macros/latex/contrib/oberdiek.tds.zip}]
% \end{description}
% \emph{TDS} refers to the standard ``A Directory Structure
% for \TeX\ Files'' (\CTAN{tds/tds.pdf}). Directories
% with \xfile{texmf} in their name are usually organized this way.
%
% \subsection{Bundle installation}
%
% \paragraph{Unpacking.} Unpack the \xfile{oberdiek.tds.zip} in the
% TDS tree (also known as \xfile{texmf} tree) of your choice.
% Example (linux):
% \begin{quote}
%   |unzip oberdiek.tds.zip -d ~/texmf|
% \end{quote}
%
% \paragraph{Script installation.}
% Check the directory \xfile{TDS:scripts/oberdiek/} for
% scripts that need further installation steps.
%
% \subsection{Package installation}
%
% \paragraph{Unpacking.} The \xfile{.dtx} file is a self-extracting
% \docstrip\ archive. The files are extracted by running the
% \xfile{.dtx} through \plainTeX:
% \begin{quote}
%   \verb|tex protecteddef.dtx|
% \end{quote}
%
% \paragraph{TDS.} Now the different files must be moved into
% the different directories in your installation TDS tree
% (also known as \xfile{texmf} tree):
% \begin{quote}
% \def\t{^^A
% \begin{tabular}{@{}>{\ttfamily}l@{ $\rightarrow$ }>{\ttfamily}l@{}}
%   protecteddef.sty & tex/generic/oberdiek/protecteddef.sty\\
%   protecteddef.pdf & doc/latex/oberdiek/protecteddef.pdf\\
%   test/protecteddef-test1.tex & doc/latex/oberdiek/test/protecteddef-test1.tex\\
%   test/protecteddef-test2.tex & doc/latex/oberdiek/test/protecteddef-test2.tex\\
%   protecteddef.dtx & source/latex/oberdiek/protecteddef.dtx\\
% \end{tabular}^^A
% }^^A
% \sbox0{\t}^^A
% \ifdim\wd0>\linewidth
%   \begingroup
%     \advance\linewidth by\leftmargin
%     \advance\linewidth by\rightmargin
%   \edef\x{\endgroup
%     \def\noexpand\lw{\the\linewidth}^^A
%   }\x
%   \def\lwbox{^^A
%     \leavevmode
%     \hbox to \linewidth{^^A
%       \kern-\leftmargin\relax
%       \hss
%       \usebox0
%       \hss
%       \kern-\rightmargin\relax
%     }^^A
%   }^^A
%   \ifdim\wd0>\lw
%     \sbox0{\small\t}^^A
%     \ifdim\wd0>\linewidth
%       \ifdim\wd0>\lw
%         \sbox0{\footnotesize\t}^^A
%         \ifdim\wd0>\linewidth
%           \ifdim\wd0>\lw
%             \sbox0{\scriptsize\t}^^A
%             \ifdim\wd0>\linewidth
%               \ifdim\wd0>\lw
%                 \sbox0{\tiny\t}^^A
%                 \ifdim\wd0>\linewidth
%                   \lwbox
%                 \else
%                   \usebox0
%                 \fi
%               \else
%                 \lwbox
%               \fi
%             \else
%               \usebox0
%             \fi
%           \else
%             \lwbox
%           \fi
%         \else
%           \usebox0
%         \fi
%       \else
%         \lwbox
%       \fi
%     \else
%       \usebox0
%     \fi
%   \else
%     \lwbox
%   \fi
% \else
%   \usebox0
% \fi
% \end{quote}
% If you have a \xfile{docstrip.cfg} that configures and enables \docstrip's
% TDS installing feature, then some files can already be in the right
% place, see the documentation of \docstrip.
%
% \subsection{Refresh file name databases}
%
% If your \TeX~distribution
% (\TeX\,Live, \mikTeX, \dots) relies on file name databases, you must refresh
% these. For example, \TeX\,Live\ users run \verb|texhash| or
% \verb|mktexlsr|.
%
% \subsection{Some details for the interested}
%
% \paragraph{Unpacking with \LaTeX.}
% The \xfile{.dtx} chooses its action depending on the format:
% \begin{description}
% \item[\plainTeX:] Run \docstrip\ and extract the files.
% \item[\LaTeX:] Generate the documentation.
% \end{description}
% If you insist on using \LaTeX\ for \docstrip\ (really,
% \docstrip\ does not need \LaTeX), then inform the autodetect routine
% about your intention:
% \begin{quote}
%   \verb|latex \let\install=y\input{protecteddef.dtx}|
% \end{quote}
% Do not forget to quote the argument according to the demands
% of your shell.
%
% \paragraph{Generating the documentation.}
% You can use both the \xfile{.dtx} or the \xfile{.drv} to generate
% the documentation. The process can be configured by the
% configuration file \xfile{ltxdoc.cfg}. For instance, put this
% line into this file, if you want to have A4 as paper format:
% \begin{quote}
%   \verb|\PassOptionsToClass{a4paper}{article}|
% \end{quote}
% An example follows how to generate the
% documentation with pdf\LaTeX:
% \begin{quote}
%\begin{verbatim}
%pdflatex protecteddef.dtx
%makeindex -s gind.ist protecteddef.idx
%pdflatex protecteddef.dtx
%makeindex -s gind.ist protecteddef.idx
%pdflatex protecteddef.dtx
%\end{verbatim}
% \end{quote}
%
% \begin{History}
%   \begin{Version}{2011/01/31 v1.0}
%   \item
%     First public version.
%   \end{Version}
%   \begin{Version}{2016/05/16 v1.1}
%   \item
%     Documentation updates.
%   \end{Version}
% \end{History}
%
% \PrintIndex
%
% \Finale
\endinput
|
% \end{quote}
% Do not forget to quote the argument according to the demands
% of your shell.
%
% \paragraph{Generating the documentation.}
% You can use both the \xfile{.dtx} or the \xfile{.drv} to generate
% the documentation. The process can be configured by the
% configuration file \xfile{ltxdoc.cfg}. For instance, put this
% line into this file, if you want to have A4 as paper format:
% \begin{quote}
%   \verb|\PassOptionsToClass{a4paper}{article}|
% \end{quote}
% An example follows how to generate the
% documentation with pdf\LaTeX:
% \begin{quote}
%\begin{verbatim}
%pdflatex protecteddef.dtx
%makeindex -s gind.ist protecteddef.idx
%pdflatex protecteddef.dtx
%makeindex -s gind.ist protecteddef.idx
%pdflatex protecteddef.dtx
%\end{verbatim}
% \end{quote}
%
% \begin{History}
%   \begin{Version}{2011/01/31 v1.0}
%   \item
%     First public version.
%   \end{Version}
%   \begin{Version}{2016/05/16 v1.1}
%   \item
%     Documentation updates.
%   \end{Version}
% \end{History}
%
% \PrintIndex
%
% \Finale
\endinput
|
% \end{quote}
% Do not forget to quote the argument according to the demands
% of your shell.
%
% \paragraph{Generating the documentation.}
% You can use both the \xfile{.dtx} or the \xfile{.drv} to generate
% the documentation. The process can be configured by the
% configuration file \xfile{ltxdoc.cfg}. For instance, put this
% line into this file, if you want to have A4 as paper format:
% \begin{quote}
%   \verb|\PassOptionsToClass{a4paper}{article}|
% \end{quote}
% An example follows how to generate the
% documentation with pdf\LaTeX:
% \begin{quote}
%\begin{verbatim}
%pdflatex protecteddef.dtx
%makeindex -s gind.ist protecteddef.idx
%pdflatex protecteddef.dtx
%makeindex -s gind.ist protecteddef.idx
%pdflatex protecteddef.dtx
%\end{verbatim}
% \end{quote}
%
% \begin{History}
%   \begin{Version}{2011/01/31 v1.0}
%   \item
%     First public version.
%   \end{Version}
%   \begin{Version}{2016/05/16 v1.1}
%   \item
%     Documentation updates.
%   \end{Version}
% \end{History}
%
% \PrintIndex
%
% \Finale
\endinput
|
% \end{quote}
% Do not forget to quote the argument according to the demands
% of your shell.
%
% \paragraph{Generating the documentation.}
% You can use both the \xfile{.dtx} or the \xfile{.drv} to generate
% the documentation. The process can be configured by the
% configuration file \xfile{ltxdoc.cfg}. For instance, put this
% line into this file, if you want to have A4 as paper format:
% \begin{quote}
%   \verb|\PassOptionsToClass{a4paper}{article}|
% \end{quote}
% An example follows how to generate the
% documentation with pdf\LaTeX:
% \begin{quote}
%\begin{verbatim}
%pdflatex protecteddef.dtx
%makeindex -s gind.ist protecteddef.idx
%pdflatex protecteddef.dtx
%makeindex -s gind.ist protecteddef.idx
%pdflatex protecteddef.dtx
%\end{verbatim}
% \end{quote}
%
% \begin{History}
%   \begin{Version}{2011/01/31 v1.0}
%   \item
%     First public version.
%   \end{Version}
%   \begin{Version}{2016/05/16 v1.1}
%   \item
%     Documentation updates.
%   \end{Version}
% \end{History}
%
% \PrintIndex
%
% \Finale
\endinput
