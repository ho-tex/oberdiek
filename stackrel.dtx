% \iffalse meta-comment
%
% File: stackrel.dtx
% Version: 2016/05/16 v1.3
% Info: Adding subscript option to stackrel
%
% Copyright (C) 2006, 2007 by
%    Heiko Oberdiek <heiko.oberdiek at googlemail.com>
%    2016
%    https://github.com/ho-tex/oberdiek/issues
%
% This work may be distributed and/or modified under the
% conditions of the LaTeX Project Public License, either
% version 1.3c of this license or (at your option) any later
% version. This version of this license is in
%    https://www.latex-project.org/lppl/lppl-1-3c.txt
% and the latest version of this license is in
%    https://www.latex-project.org/lppl.txt
% and version 1.3 or later is part of all distributions of
% LaTeX version 2005/12/01 or later.
%
% This work has the LPPL maintenance status "maintained".
%
% The Current Maintainers of this work are
% Heiko Oberdiek and the Oberdiek Package Support Group
% https://github.com/ho-tex/oberdiek/issues
%
% This work consists of the main source file stackrel.dtx
% and the derived files
%    stackrel.sty, stackrel.pdf, stackrel.ins, stackrel.drv.
%
% Distribution:
%    CTAN:macros/latex/contrib/oberdiek/stackrel.dtx
%    CTAN:macros/latex/contrib/oberdiek/stackrel.pdf
%
% Unpacking:
%    (a) If stackrel.ins is present:
%           tex stackrel.ins
%    (b) Without stackrel.ins:
%           tex stackrel.dtx
%    (c) If you insist on using LaTeX
%           latex \let\install=y% \iffalse meta-comment
%
% File: stackrel.dtx
% Version: 2016/05/16 v1.3
% Info: Adding subscript option to stackrel
%
% Copyright (C) 2006, 2007 by
%    Heiko Oberdiek <heiko.oberdiek at googlemail.com>
%    2016
%    https://github.com/ho-tex/oberdiek/issues
%
% This work may be distributed and/or modified under the
% conditions of the LaTeX Project Public License, either
% version 1.3c of this license or (at your option) any later
% version. This version of this license is in
%    http://www.latex-project.org/lppl/lppl-1-3c.txt
% and the latest version of this license is in
%    http://www.latex-project.org/lppl.txt
% and version 1.3 or later is part of all distributions of
% LaTeX version 2005/12/01 or later.
%
% This work has the LPPL maintenance status "maintained".
%
% This Current Maintainer of this work is Heiko Oberdiek.
%
% This work consists of the main source file stackrel.dtx
% and the derived files
%    stackrel.sty, stackrel.pdf, stackrel.ins, stackrel.drv.
%
% Distribution:
%    CTAN:macros/latex/contrib/oberdiek/stackrel.dtx
%    CTAN:macros/latex/contrib/oberdiek/stackrel.pdf
%
% Unpacking:
%    (a) If stackrel.ins is present:
%           tex stackrel.ins
%    (b) Without stackrel.ins:
%           tex stackrel.dtx
%    (c) If you insist on using LaTeX
%           latex \let\install=y% \iffalse meta-comment
%
% File: stackrel.dtx
% Version: 2016/05/16 v1.3
% Info: Adding subscript option to stackrel
%
% Copyright (C) 2006, 2007 by
%    Heiko Oberdiek <heiko.oberdiek at googlemail.com>
%    2016
%    https://github.com/ho-tex/oberdiek/issues
%
% This work may be distributed and/or modified under the
% conditions of the LaTeX Project Public License, either
% version 1.3c of this license or (at your option) any later
% version. This version of this license is in
%    http://www.latex-project.org/lppl/lppl-1-3c.txt
% and the latest version of this license is in
%    http://www.latex-project.org/lppl.txt
% and version 1.3 or later is part of all distributions of
% LaTeX version 2005/12/01 or later.
%
% This work has the LPPL maintenance status "maintained".
%
% This Current Maintainer of this work is Heiko Oberdiek.
%
% This work consists of the main source file stackrel.dtx
% and the derived files
%    stackrel.sty, stackrel.pdf, stackrel.ins, stackrel.drv.
%
% Distribution:
%    CTAN:macros/latex/contrib/oberdiek/stackrel.dtx
%    CTAN:macros/latex/contrib/oberdiek/stackrel.pdf
%
% Unpacking:
%    (a) If stackrel.ins is present:
%           tex stackrel.ins
%    (b) Without stackrel.ins:
%           tex stackrel.dtx
%    (c) If you insist on using LaTeX
%           latex \let\install=y% \iffalse meta-comment
%
% File: stackrel.dtx
% Version: 2016/05/16 v1.3
% Info: Adding subscript option to stackrel
%
% Copyright (C) 2006, 2007 by
%    Heiko Oberdiek <heiko.oberdiek at googlemail.com>
%    2016
%    https://github.com/ho-tex/oberdiek/issues
%
% This work may be distributed and/or modified under the
% conditions of the LaTeX Project Public License, either
% version 1.3c of this license or (at your option) any later
% version. This version of this license is in
%    http://www.latex-project.org/lppl/lppl-1-3c.txt
% and the latest version of this license is in
%    http://www.latex-project.org/lppl.txt
% and version 1.3 or later is part of all distributions of
% LaTeX version 2005/12/01 or later.
%
% This work has the LPPL maintenance status "maintained".
%
% This Current Maintainer of this work is Heiko Oberdiek.
%
% This work consists of the main source file stackrel.dtx
% and the derived files
%    stackrel.sty, stackrel.pdf, stackrel.ins, stackrel.drv.
%
% Distribution:
%    CTAN:macros/latex/contrib/oberdiek/stackrel.dtx
%    CTAN:macros/latex/contrib/oberdiek/stackrel.pdf
%
% Unpacking:
%    (a) If stackrel.ins is present:
%           tex stackrel.ins
%    (b) Without stackrel.ins:
%           tex stackrel.dtx
%    (c) If you insist on using LaTeX
%           latex \let\install=y\input{stackrel.dtx}
%        (quote the arguments according to the demands of your shell)
%
% Documentation:
%    (a) If stackrel.drv is present:
%           latex stackrel.drv
%    (b) Without stackrel.drv:
%           latex stackrel.dtx; ...
%    The class ltxdoc loads the configuration file ltxdoc.cfg
%    if available. Here you can specify further options, e.g.
%    use A4 as paper format:
%       \PassOptionsToClass{a4paper}{article}
%
%    Programm calls to get the documentation (example):
%       pdflatex stackrel.dtx
%       makeindex -s gind.ist stackrel.idx
%       pdflatex stackrel.dtx
%       makeindex -s gind.ist stackrel.idx
%       pdflatex stackrel.dtx
%
% Installation:
%    TDS:tex/latex/oberdiek/stackrel.sty
%    TDS:doc/latex/oberdiek/stackrel.pdf
%    TDS:source/latex/oberdiek/stackrel.dtx
%
%<*ignore>
\begingroup
  \catcode123=1 %
  \catcode125=2 %
  \def\x{LaTeX2e}%
\expandafter\endgroup
\ifcase 0\ifx\install y1\fi\expandafter
         \ifx\csname processbatchFile\endcsname\relax\else1\fi
         \ifx\fmtname\x\else 1\fi\relax
\else\csname fi\endcsname
%</ignore>
%<*install>
\input docstrip.tex
\Msg{************************************************************************}
\Msg{* Installation}
\Msg{* Package: stackrel 2016/05/16 v1.3 Adding subscript option to stackrel (HO)}
\Msg{************************************************************************}

\keepsilent
\askforoverwritefalse

\let\MetaPrefix\relax
\preamble

This is a generated file.

Project: stackrel
Version: 2016/05/16 v1.3

Copyright (C) 2006, 2007 by
   Heiko Oberdiek <heiko.oberdiek at googlemail.com>

This work may be distributed and/or modified under the
conditions of the LaTeX Project Public License, either
version 1.3c of this license or (at your option) any later
version. This version of this license is in
   http://www.latex-project.org/lppl/lppl-1-3c.txt
and the latest version of this license is in
   http://www.latex-project.org/lppl.txt
and version 1.3 or later is part of all distributions of
LaTeX version 2005/12/01 or later.

This work has the LPPL maintenance status "maintained".

This Current Maintainer of this work is Heiko Oberdiek.

This work consists of the main source file stackrel.dtx
and the derived files
   stackrel.sty, stackrel.pdf, stackrel.ins, stackrel.drv.

\endpreamble
\let\MetaPrefix\DoubleperCent

\generate{%
  \file{stackrel.ins}{\from{stackrel.dtx}{install}}%
  \file{stackrel.drv}{\from{stackrel.dtx}{driver}}%
  \usedir{tex/latex/oberdiek}%
  \file{stackrel.sty}{\from{stackrel.dtx}{package}}%
  \nopreamble
  \nopostamble
%  \usedir{source/latex/oberdiek/catalogue}%
%  \file{stackrel.xml}{\from{stackrel.dtx}{catalogue}}%
}

\catcode32=13\relax% active space
\let =\space%
\Msg{************************************************************************}
\Msg{*}
\Msg{* To finish the installation you have to move the following}
\Msg{* file into a directory searched by TeX:}
\Msg{*}
\Msg{*     stackrel.sty}
\Msg{*}
\Msg{* To produce the documentation run the file `stackrel.drv'}
\Msg{* through LaTeX.}
\Msg{*}
\Msg{* Happy TeXing!}
\Msg{*}
\Msg{************************************************************************}

\endbatchfile
%</install>
%<*ignore>
\fi
%</ignore>
%<*driver>
\NeedsTeXFormat{LaTeX2e}
\ProvidesFile{stackrel.drv}%
  [2016/05/16 v1.3 Adding subscript option to stackrel (HO)]%
\documentclass{ltxdoc}
\usepackage{amsmath}
\usepackage{holtxdoc}[2011/11/22]
\usepackage{stackrel}[2016/05/16]
\begin{document}
  \DocInput{stackrel.dtx}%
\end{document}
%</driver>
% \fi
%
%
% \CharacterTable
%  {Upper-case    \A\B\C\D\E\F\G\H\I\J\K\L\M\N\O\P\Q\R\S\T\U\V\W\X\Y\Z
%   Lower-case    \a\b\c\d\e\f\g\h\i\j\k\l\m\n\o\p\q\r\s\t\u\v\w\x\y\z
%   Digits        \0\1\2\3\4\5\6\7\8\9
%   Exclamation   \!     Double quote  \"     Hash (number) \#
%   Dollar        \$     Percent       \%     Ampersand     \&
%   Acute accent  \'     Left paren    \(     Right paren   \)
%   Asterisk      \*     Plus          \+     Comma         \,
%   Minus         \-     Point         \.     Solidus       \/
%   Colon         \:     Semicolon     \;     Less than     \<
%   Equals        \=     Greater than  \>     Question mark \?
%   Commercial at \@     Left bracket  \[     Backslash     \\
%   Right bracket \]     Circumflex    \^     Underscore    \_
%   Grave accent  \`     Left brace    \{     Vertical bar  \|
%   Right brace   \}     Tilde         \~}
%
% \GetFileInfo{stackrel.drv}
%
% \title{The \xpackage{stackrel} package}
% \date{2016/05/16 v1.3}
% \author{Heiko Oberdiek\thanks
% {Please report any issues at \url{https://github.com/ho-tex/oberdiek/issues}}\\
% \xemail{heiko.oberdiek at googlemail.com}}
%
% \maketitle
%
% \begin{abstract}
% This package adds an optional argument to \cs{stackrel} for
% putting something below the relational symbol and defines
% \cs{stackbin} for binary symbols.
% \end{abstract}
%
% \tableofcontents
%
% \section{User interface}
%
% \LaTeX's \cs{stackrel} allows a superscript above a relational symbol,
% but pure \LaTeX\ does not provide a macro for putting a subscript
% below the symbol. This is supported by \AmS\LaTeX's \cs{underset}
% macro that works on both relational and binary symbols. A combination
% of \cs{underset} and \cs{overset} can be used to put \mbox{sub-} and
% superscripts to the same symbol.
%
% This package \xpackage{stackrel} extends the syntax of \cs{stackrel}
% by adding an optional argument for the subscript position.
% It follows the syntax of extensible arrows of packages
% \xpackage{amsmath} and \xpackage{mathtools}.
%
% \begin{declcs}{stackrel}
%   |[|\meta{subscript}|]| \M{superscript} \M{rel}\\
%   \cs{stackbin}
%   |[|\meta{subscript}|]| \M{superscript} \M{bin}
% \end{declcs}
% Example:
% \begin{quote}
% |A \stackbin[\text{and}]{}{+} B \stackrel[x]{!}{=} C|\\
% $A \stackbin[\text{and}]{}{+} B \stackrel[x]{!}{=} C$
% \end{quote}
%
% \StopEventually{
% }
%
% \section{Implementation}
%
%    \begin{macrocode}
%<*package>
\NeedsTeXFormat{LaTeX2e}
\ProvidesPackage{stackrel}
  [2016/05/16 v1.3 Adding subscript option to stackrel (HO)]%
%    \end{macrocode}
%
%    Given the original definition of \cs{stackrel} the addition
%    of the optional argument is straightforward. If an argument
%    is empty, then the corresponding sub- or superscript is
%    suppressed.
%
%    Depending on the available resources (\eTeX, \pdfTeX)
%    three methods are given for testing emptyness. All tests
%    allow the hash to be used inside the arguments without
%    doubling (for the unlikely case that someone wants to
%    define macros with arguments).
%    \begin{macro}{\stack@relbin}
%    \begin{macrocode}
\RequirePackage{etexcmds}[2007/09/09]
\ifetex@unexpanded
  \RequirePackage{pdftexcmds}[2016/05/16]%
  \begingroup\expandafter\expandafter\expandafter\endgroup
  \expandafter\ifx\csname pdf@strcmp\endcsname\relax
    \newcommand*{\stack@relbin}[3][]{%
      \mathop{#3}\limits
      \edef\reserved@a{\etex@unexpanded{#1}}%
      \ifx\reserved@a\@empty\else_{#1}\fi
      \edef\reserved@a{\etex@unexpanded{#2}}%
      \ifx\reserved@a\@empty\else^{#2}\fi
      \egroup
    }%
  \else
    \newcommand*{\stack@relbin}[3][]{%
      \mathop{#3}\limits
      \ifcase\pdf@strcmp{\detokenize{#1}}{}\else_{#1}\fi
      \ifcase\pdf@strcmp{\detokenize{#2}}{}\else^{#2}\fi
      \egroup
    }%
  \fi
\else
  \newcommand*{\stack@relbin}[3][]{%
    \mathop{#3}\limits
    \toks@{#1}%
    \edef\reserved@a{\the\toks@}%
    \ifx\reserved@a\@empty\else_{#1}\fi
    \toks@{#2}%
    \edef\reserved@a{\the\toks@}%
    \ifx\reserved@a\@empty\else^{#2}\fi
    \egroup
  }%
\fi
%    \end{macrocode}
%    \end{macro}
%    \begin{macro}{\stackrel}
%    \begin{macrocode}
\renewcommand*{\stackrel}{%
  \mathrel\bgroup\stack@relbin
}
%    \end{macrocode}
%    \end{macro}
%    \begin{macro}{\stackbin}
%    \begin{macrocode}
\newcommand*{\stackbin}{%
  \mathbin\bgroup\stack@relbin
}
%    \end{macrocode}
%    \end{macro}
%
%    \begin{macrocode}
%</package>
%    \end{macrocode}
%
% \section{Installation}
%
% \subsection{Download}
%
% \paragraph{Package.} This package is available on
% CTAN\footnote{\CTANpkg{stackrel}}:
% \begin{description}
% \item[\CTAN{macros/latex/contrib/oberdiek/stackrel.dtx}] The source file.
% \item[\CTAN{macros/latex/contrib/oberdiek/stackrel.pdf}] Documentation.
% \end{description}
%
%
% \paragraph{Bundle.} All the packages of the bundle `oberdiek'
% are also available in a TDS compliant ZIP archive. There
% the packages are already unpacked and the documentation files
% are generated. The files and directories obey the TDS standard.
% \begin{description}
% \item[\CTANinstall{install/macros/latex/contrib/oberdiek.tds.zip}]
% \end{description}
% \emph{TDS} refers to the standard ``A Directory Structure
% for \TeX\ Files'' (\CTAN{tds/tds.pdf}). Directories
% with \xfile{texmf} in their name are usually organized this way.
%
% \subsection{Bundle installation}
%
% \paragraph{Unpacking.} Unpack the \xfile{oberdiek.tds.zip} in the
% TDS tree (also known as \xfile{texmf} tree) of your choice.
% Example (linux):
% \begin{quote}
%   |unzip oberdiek.tds.zip -d ~/texmf|
% \end{quote}
%
% \paragraph{Script installation.}
% Check the directory \xfile{TDS:scripts/oberdiek/} for
% scripts that need further installation steps.
% Package \xpackage{attachfile2} comes with the Perl script
% \xfile{pdfatfi.pl} that should be installed in such a way
% that it can be called as \texttt{pdfatfi}.
% Example (linux):
% \begin{quote}
%   |chmod +x scripts/oberdiek/pdfatfi.pl|\\
%   |cp scripts/oberdiek/pdfatfi.pl /usr/local/bin/|
% \end{quote}
%
% \subsection{Package installation}
%
% \paragraph{Unpacking.} The \xfile{.dtx} file is a self-extracting
% \docstrip\ archive. The files are extracted by running the
% \xfile{.dtx} through \plainTeX:
% \begin{quote}
%   \verb|tex stackrel.dtx|
% \end{quote}
%
% \paragraph{TDS.} Now the different files must be moved into
% the different directories in your installation TDS tree
% (also known as \xfile{texmf} tree):
% \begin{quote}
% \def\t{^^A
% \begin{tabular}{@{}>{\ttfamily}l@{ $\rightarrow$ }>{\ttfamily}l@{}}
%   stackrel.sty & tex/latex/oberdiek/stackrel.sty\\
%   stackrel.pdf & doc/latex/oberdiek/stackrel.pdf\\
%   stackrel.dtx & source/latex/oberdiek/stackrel.dtx\\
% \end{tabular}^^A
% }^^A
% \sbox0{\t}^^A
% \ifdim\wd0>\linewidth
%   \begingroup
%     \advance\linewidth by\leftmargin
%     \advance\linewidth by\rightmargin
%   \edef\x{\endgroup
%     \def\noexpand\lw{\the\linewidth}^^A
%   }\x
%   \def\lwbox{^^A
%     \leavevmode
%     \hbox to \linewidth{^^A
%       \kern-\leftmargin\relax
%       \hss
%       \usebox0
%       \hss
%       \kern-\rightmargin\relax
%     }^^A
%   }^^A
%   \ifdim\wd0>\lw
%     \sbox0{\small\t}^^A
%     \ifdim\wd0>\linewidth
%       \ifdim\wd0>\lw
%         \sbox0{\footnotesize\t}^^A
%         \ifdim\wd0>\linewidth
%           \ifdim\wd0>\lw
%             \sbox0{\scriptsize\t}^^A
%             \ifdim\wd0>\linewidth
%               \ifdim\wd0>\lw
%                 \sbox0{\tiny\t}^^A
%                 \ifdim\wd0>\linewidth
%                   \lwbox
%                 \else
%                   \usebox0
%                 \fi
%               \else
%                 \lwbox
%               \fi
%             \else
%               \usebox0
%             \fi
%           \else
%             \lwbox
%           \fi
%         \else
%           \usebox0
%         \fi
%       \else
%         \lwbox
%       \fi
%     \else
%       \usebox0
%     \fi
%   \else
%     \lwbox
%   \fi
% \else
%   \usebox0
% \fi
% \end{quote}
% If you have a \xfile{docstrip.cfg} that configures and enables \docstrip's
% TDS installing feature, then some files can already be in the right
% place, see the documentation of \docstrip.
%
% \subsection{Refresh file name databases}
%
% If your \TeX~distribution
% (\teTeX, \mikTeX, \dots) relies on file name databases, you must refresh
% these. For example, \teTeX\ users run \verb|texhash| or
% \verb|mktexlsr|.
%
% \subsection{Some details for the interested}
%
% \paragraph{Attached source.}
%
% The PDF documentation on CTAN also includes the
% \xfile{.dtx} source file. It can be extracted by
% AcrobatReader 6 or higher. Another option is \textsf{pdftk},
% e.g. unpack the file into the current directory:
% \begin{quote}
%   \verb|pdftk stackrel.pdf unpack_files output .|
% \end{quote}
%
% \paragraph{Unpacking with \LaTeX.}
% The \xfile{.dtx} chooses its action depending on the format:
% \begin{description}
% \item[\plainTeX:] Run \docstrip\ and extract the files.
% \item[\LaTeX:] Generate the documentation.
% \end{description}
% If you insist on using \LaTeX\ for \docstrip\ (really,
% \docstrip\ does not need \LaTeX), then inform the autodetect routine
% about your intention:
% \begin{quote}
%   \verb|latex \let\install=y\input{stackrel.dtx}|
% \end{quote}
% Do not forget to quote the argument according to the demands
% of your shell.
%
% \paragraph{Generating the documentation.}
% You can use both the \xfile{.dtx} or the \xfile{.drv} to generate
% the documentation. The process can be configured by the
% configuration file \xfile{ltxdoc.cfg}. For instance, put this
% line into this file, if you want to have A4 as paper format:
% \begin{quote}
%   \verb|\PassOptionsToClass{a4paper}{article}|
% \end{quote}
% An example follows how to generate the
% documentation with pdf\LaTeX:
% \begin{quote}
%\begin{verbatim}
%pdflatex stackrel.dtx
%makeindex -s gind.ist stackrel.idx
%pdflatex stackrel.dtx
%makeindex -s gind.ist stackrel.idx
%pdflatex stackrel.dtx
%\end{verbatim}
% \end{quote}
%
% \begin{History}
%   \begin{Version}{2006/12/02 v1.0}
%   \item
%     First version.
%   \end{Version}
%   \begin{Version}{2007/05/06 v1.1}
%   \item
%     Uses package \xpackage{etexcmds}.
%   \end{Version}
%   \begin{Version}{2007/11/11 v1.2}
%   \item
%     Use of package \xpackage{pdftexcmds} for \LuaTeX\ support.
%   \end{Version}
%   \begin{Version}{2016/05/16 v1.3}
%   \item
%     Documentation updates.
%   \end{Version}
% \end{History}
%
% \clearpage
% \PrintIndex
%
% \Finale
\endinput

%        (quote the arguments according to the demands of your shell)
%
% Documentation:
%    (a) If stackrel.drv is present:
%           latex stackrel.drv
%    (b) Without stackrel.drv:
%           latex stackrel.dtx; ...
%    The class ltxdoc loads the configuration file ltxdoc.cfg
%    if available. Here you can specify further options, e.g.
%    use A4 as paper format:
%       \PassOptionsToClass{a4paper}{article}
%
%    Programm calls to get the documentation (example):
%       pdflatex stackrel.dtx
%       makeindex -s gind.ist stackrel.idx
%       pdflatex stackrel.dtx
%       makeindex -s gind.ist stackrel.idx
%       pdflatex stackrel.dtx
%
% Installation:
%    TDS:tex/latex/oberdiek/stackrel.sty
%    TDS:doc/latex/oberdiek/stackrel.pdf
%    TDS:source/latex/oberdiek/stackrel.dtx
%
%<*ignore>
\begingroup
  \catcode123=1 %
  \catcode125=2 %
  \def\x{LaTeX2e}%
\expandafter\endgroup
\ifcase 0\ifx\install y1\fi\expandafter
         \ifx\csname processbatchFile\endcsname\relax\else1\fi
         \ifx\fmtname\x\else 1\fi\relax
\else\csname fi\endcsname
%</ignore>
%<*install>
\input docstrip.tex
\Msg{************************************************************************}
\Msg{* Installation}
\Msg{* Package: stackrel 2016/05/16 v1.3 Adding subscript option to stackrel (HO)}
\Msg{************************************************************************}

\keepsilent
\askforoverwritefalse

\let\MetaPrefix\relax
\preamble

This is a generated file.

Project: stackrel
Version: 2016/05/16 v1.3

Copyright (C) 2006, 2007 by
   Heiko Oberdiek <heiko.oberdiek at googlemail.com>

This work may be distributed and/or modified under the
conditions of the LaTeX Project Public License, either
version 1.3c of this license or (at your option) any later
version. This version of this license is in
   http://www.latex-project.org/lppl/lppl-1-3c.txt
and the latest version of this license is in
   http://www.latex-project.org/lppl.txt
and version 1.3 or later is part of all distributions of
LaTeX version 2005/12/01 or later.

This work has the LPPL maintenance status "maintained".

This Current Maintainer of this work is Heiko Oberdiek.

This work consists of the main source file stackrel.dtx
and the derived files
   stackrel.sty, stackrel.pdf, stackrel.ins, stackrel.drv.

\endpreamble
\let\MetaPrefix\DoubleperCent

\generate{%
  \file{stackrel.ins}{\from{stackrel.dtx}{install}}%
  \file{stackrel.drv}{\from{stackrel.dtx}{driver}}%
  \usedir{tex/latex/oberdiek}%
  \file{stackrel.sty}{\from{stackrel.dtx}{package}}%
  \nopreamble
  \nopostamble
%  \usedir{source/latex/oberdiek/catalogue}%
%  \file{stackrel.xml}{\from{stackrel.dtx}{catalogue}}%
}

\catcode32=13\relax% active space
\let =\space%
\Msg{************************************************************************}
\Msg{*}
\Msg{* To finish the installation you have to move the following}
\Msg{* file into a directory searched by TeX:}
\Msg{*}
\Msg{*     stackrel.sty}
\Msg{*}
\Msg{* To produce the documentation run the file `stackrel.drv'}
\Msg{* through LaTeX.}
\Msg{*}
\Msg{* Happy TeXing!}
\Msg{*}
\Msg{************************************************************************}

\endbatchfile
%</install>
%<*ignore>
\fi
%</ignore>
%<*driver>
\NeedsTeXFormat{LaTeX2e}
\ProvidesFile{stackrel.drv}%
  [2016/05/16 v1.3 Adding subscript option to stackrel (HO)]%
\documentclass{ltxdoc}
\usepackage{amsmath}
\usepackage{holtxdoc}[2011/11/22]
\usepackage{stackrel}[2016/05/16]
\begin{document}
  \DocInput{stackrel.dtx}%
\end{document}
%</driver>
% \fi
%
%
% \CharacterTable
%  {Upper-case    \A\B\C\D\E\F\G\H\I\J\K\L\M\N\O\P\Q\R\S\T\U\V\W\X\Y\Z
%   Lower-case    \a\b\c\d\e\f\g\h\i\j\k\l\m\n\o\p\q\r\s\t\u\v\w\x\y\z
%   Digits        \0\1\2\3\4\5\6\7\8\9
%   Exclamation   \!     Double quote  \"     Hash (number) \#
%   Dollar        \$     Percent       \%     Ampersand     \&
%   Acute accent  \'     Left paren    \(     Right paren   \)
%   Asterisk      \*     Plus          \+     Comma         \,
%   Minus         \-     Point         \.     Solidus       \/
%   Colon         \:     Semicolon     \;     Less than     \<
%   Equals        \=     Greater than  \>     Question mark \?
%   Commercial at \@     Left bracket  \[     Backslash     \\
%   Right bracket \]     Circumflex    \^     Underscore    \_
%   Grave accent  \`     Left brace    \{     Vertical bar  \|
%   Right brace   \}     Tilde         \~}
%
% \GetFileInfo{stackrel.drv}
%
% \title{The \xpackage{stackrel} package}
% \date{2016/05/16 v1.3}
% \author{Heiko Oberdiek\thanks
% {Please report any issues at \url{https://github.com/ho-tex/oberdiek/issues}}\\
% \xemail{heiko.oberdiek at googlemail.com}}
%
% \maketitle
%
% \begin{abstract}
% This package adds an optional argument to \cs{stackrel} for
% putting something below the relational symbol and defines
% \cs{stackbin} for binary symbols.
% \end{abstract}
%
% \tableofcontents
%
% \section{User interface}
%
% \LaTeX's \cs{stackrel} allows a superscript above a relational symbol,
% but pure \LaTeX\ does not provide a macro for putting a subscript
% below the symbol. This is supported by \AmS\LaTeX's \cs{underset}
% macro that works on both relational and binary symbols. A combination
% of \cs{underset} and \cs{overset} can be used to put \mbox{sub-} and
% superscripts to the same symbol.
%
% This package \xpackage{stackrel} extends the syntax of \cs{stackrel}
% by adding an optional argument for the subscript position.
% It follows the syntax of extensible arrows of packages
% \xpackage{amsmath} and \xpackage{mathtools}.
%
% \begin{declcs}{stackrel}
%   |[|\meta{subscript}|]| \M{superscript} \M{rel}\\
%   \cs{stackbin}
%   |[|\meta{subscript}|]| \M{superscript} \M{bin}
% \end{declcs}
% Example:
% \begin{quote}
% |A \stackbin[\text{and}]{}{+} B \stackrel[x]{!}{=} C|\\
% $A \stackbin[\text{and}]{}{+} B \stackrel[x]{!}{=} C$
% \end{quote}
%
% \StopEventually{
% }
%
% \section{Implementation}
%
%    \begin{macrocode}
%<*package>
\NeedsTeXFormat{LaTeX2e}
\ProvidesPackage{stackrel}
  [2016/05/16 v1.3 Adding subscript option to stackrel (HO)]%
%    \end{macrocode}
%
%    Given the original definition of \cs{stackrel} the addition
%    of the optional argument is straightforward. If an argument
%    is empty, then the corresponding sub- or superscript is
%    suppressed.
%
%    Depending on the available resources (\eTeX, \pdfTeX)
%    three methods are given for testing emptyness. All tests
%    allow the hash to be used inside the arguments without
%    doubling (for the unlikely case that someone wants to
%    define macros with arguments).
%    \begin{macro}{\stack@relbin}
%    \begin{macrocode}
\RequirePackage{etexcmds}[2007/09/09]
\ifetex@unexpanded
  \RequirePackage{pdftexcmds}[2016/05/16]%
  \begingroup\expandafter\expandafter\expandafter\endgroup
  \expandafter\ifx\csname pdf@strcmp\endcsname\relax
    \newcommand*{\stack@relbin}[3][]{%
      \mathop{#3}\limits
      \edef\reserved@a{\etex@unexpanded{#1}}%
      \ifx\reserved@a\@empty\else_{#1}\fi
      \edef\reserved@a{\etex@unexpanded{#2}}%
      \ifx\reserved@a\@empty\else^{#2}\fi
      \egroup
    }%
  \else
    \newcommand*{\stack@relbin}[3][]{%
      \mathop{#3}\limits
      \ifcase\pdf@strcmp{\detokenize{#1}}{}\else_{#1}\fi
      \ifcase\pdf@strcmp{\detokenize{#2}}{}\else^{#2}\fi
      \egroup
    }%
  \fi
\else
  \newcommand*{\stack@relbin}[3][]{%
    \mathop{#3}\limits
    \toks@{#1}%
    \edef\reserved@a{\the\toks@}%
    \ifx\reserved@a\@empty\else_{#1}\fi
    \toks@{#2}%
    \edef\reserved@a{\the\toks@}%
    \ifx\reserved@a\@empty\else^{#2}\fi
    \egroup
  }%
\fi
%    \end{macrocode}
%    \end{macro}
%    \begin{macro}{\stackrel}
%    \begin{macrocode}
\renewcommand*{\stackrel}{%
  \mathrel\bgroup\stack@relbin
}
%    \end{macrocode}
%    \end{macro}
%    \begin{macro}{\stackbin}
%    \begin{macrocode}
\newcommand*{\stackbin}{%
  \mathbin\bgroup\stack@relbin
}
%    \end{macrocode}
%    \end{macro}
%
%    \begin{macrocode}
%</package>
%    \end{macrocode}
%
% \section{Installation}
%
% \subsection{Download}
%
% \paragraph{Package.} This package is available on
% CTAN\footnote{\CTANpkg{stackrel}}:
% \begin{description}
% \item[\CTAN{macros/latex/contrib/oberdiek/stackrel.dtx}] The source file.
% \item[\CTAN{macros/latex/contrib/oberdiek/stackrel.pdf}] Documentation.
% \end{description}
%
%
% \paragraph{Bundle.} All the packages of the bundle `oberdiek'
% are also available in a TDS compliant ZIP archive. There
% the packages are already unpacked and the documentation files
% are generated. The files and directories obey the TDS standard.
% \begin{description}
% \item[\CTANinstall{install/macros/latex/contrib/oberdiek.tds.zip}]
% \end{description}
% \emph{TDS} refers to the standard ``A Directory Structure
% for \TeX\ Files'' (\CTAN{tds/tds.pdf}). Directories
% with \xfile{texmf} in their name are usually organized this way.
%
% \subsection{Bundle installation}
%
% \paragraph{Unpacking.} Unpack the \xfile{oberdiek.tds.zip} in the
% TDS tree (also known as \xfile{texmf} tree) of your choice.
% Example (linux):
% \begin{quote}
%   |unzip oberdiek.tds.zip -d ~/texmf|
% \end{quote}
%
% \paragraph{Script installation.}
% Check the directory \xfile{TDS:scripts/oberdiek/} for
% scripts that need further installation steps.
% Package \xpackage{attachfile2} comes with the Perl script
% \xfile{pdfatfi.pl} that should be installed in such a way
% that it can be called as \texttt{pdfatfi}.
% Example (linux):
% \begin{quote}
%   |chmod +x scripts/oberdiek/pdfatfi.pl|\\
%   |cp scripts/oberdiek/pdfatfi.pl /usr/local/bin/|
% \end{quote}
%
% \subsection{Package installation}
%
% \paragraph{Unpacking.} The \xfile{.dtx} file is a self-extracting
% \docstrip\ archive. The files are extracted by running the
% \xfile{.dtx} through \plainTeX:
% \begin{quote}
%   \verb|tex stackrel.dtx|
% \end{quote}
%
% \paragraph{TDS.} Now the different files must be moved into
% the different directories in your installation TDS tree
% (also known as \xfile{texmf} tree):
% \begin{quote}
% \def\t{^^A
% \begin{tabular}{@{}>{\ttfamily}l@{ $\rightarrow$ }>{\ttfamily}l@{}}
%   stackrel.sty & tex/latex/oberdiek/stackrel.sty\\
%   stackrel.pdf & doc/latex/oberdiek/stackrel.pdf\\
%   stackrel.dtx & source/latex/oberdiek/stackrel.dtx\\
% \end{tabular}^^A
% }^^A
% \sbox0{\t}^^A
% \ifdim\wd0>\linewidth
%   \begingroup
%     \advance\linewidth by\leftmargin
%     \advance\linewidth by\rightmargin
%   \edef\x{\endgroup
%     \def\noexpand\lw{\the\linewidth}^^A
%   }\x
%   \def\lwbox{^^A
%     \leavevmode
%     \hbox to \linewidth{^^A
%       \kern-\leftmargin\relax
%       \hss
%       \usebox0
%       \hss
%       \kern-\rightmargin\relax
%     }^^A
%   }^^A
%   \ifdim\wd0>\lw
%     \sbox0{\small\t}^^A
%     \ifdim\wd0>\linewidth
%       \ifdim\wd0>\lw
%         \sbox0{\footnotesize\t}^^A
%         \ifdim\wd0>\linewidth
%           \ifdim\wd0>\lw
%             \sbox0{\scriptsize\t}^^A
%             \ifdim\wd0>\linewidth
%               \ifdim\wd0>\lw
%                 \sbox0{\tiny\t}^^A
%                 \ifdim\wd0>\linewidth
%                   \lwbox
%                 \else
%                   \usebox0
%                 \fi
%               \else
%                 \lwbox
%               \fi
%             \else
%               \usebox0
%             \fi
%           \else
%             \lwbox
%           \fi
%         \else
%           \usebox0
%         \fi
%       \else
%         \lwbox
%       \fi
%     \else
%       \usebox0
%     \fi
%   \else
%     \lwbox
%   \fi
% \else
%   \usebox0
% \fi
% \end{quote}
% If you have a \xfile{docstrip.cfg} that configures and enables \docstrip's
% TDS installing feature, then some files can already be in the right
% place, see the documentation of \docstrip.
%
% \subsection{Refresh file name databases}
%
% If your \TeX~distribution
% (\teTeX, \mikTeX, \dots) relies on file name databases, you must refresh
% these. For example, \teTeX\ users run \verb|texhash| or
% \verb|mktexlsr|.
%
% \subsection{Some details for the interested}
%
% \paragraph{Attached source.}
%
% The PDF documentation on CTAN also includes the
% \xfile{.dtx} source file. It can be extracted by
% AcrobatReader 6 or higher. Another option is \textsf{pdftk},
% e.g. unpack the file into the current directory:
% \begin{quote}
%   \verb|pdftk stackrel.pdf unpack_files output .|
% \end{quote}
%
% \paragraph{Unpacking with \LaTeX.}
% The \xfile{.dtx} chooses its action depending on the format:
% \begin{description}
% \item[\plainTeX:] Run \docstrip\ and extract the files.
% \item[\LaTeX:] Generate the documentation.
% \end{description}
% If you insist on using \LaTeX\ for \docstrip\ (really,
% \docstrip\ does not need \LaTeX), then inform the autodetect routine
% about your intention:
% \begin{quote}
%   \verb|latex \let\install=y% \iffalse meta-comment
%
% File: stackrel.dtx
% Version: 2016/05/16 v1.3
% Info: Adding subscript option to stackrel
%
% Copyright (C) 2006, 2007 by
%    Heiko Oberdiek <heiko.oberdiek at googlemail.com>
%    2016
%    https://github.com/ho-tex/oberdiek/issues
%
% This work may be distributed and/or modified under the
% conditions of the LaTeX Project Public License, either
% version 1.3c of this license or (at your option) any later
% version. This version of this license is in
%    http://www.latex-project.org/lppl/lppl-1-3c.txt
% and the latest version of this license is in
%    http://www.latex-project.org/lppl.txt
% and version 1.3 or later is part of all distributions of
% LaTeX version 2005/12/01 or later.
%
% This work has the LPPL maintenance status "maintained".
%
% This Current Maintainer of this work is Heiko Oberdiek.
%
% This work consists of the main source file stackrel.dtx
% and the derived files
%    stackrel.sty, stackrel.pdf, stackrel.ins, stackrel.drv.
%
% Distribution:
%    CTAN:macros/latex/contrib/oberdiek/stackrel.dtx
%    CTAN:macros/latex/contrib/oberdiek/stackrel.pdf
%
% Unpacking:
%    (a) If stackrel.ins is present:
%           tex stackrel.ins
%    (b) Without stackrel.ins:
%           tex stackrel.dtx
%    (c) If you insist on using LaTeX
%           latex \let\install=y\input{stackrel.dtx}
%        (quote the arguments according to the demands of your shell)
%
% Documentation:
%    (a) If stackrel.drv is present:
%           latex stackrel.drv
%    (b) Without stackrel.drv:
%           latex stackrel.dtx; ...
%    The class ltxdoc loads the configuration file ltxdoc.cfg
%    if available. Here you can specify further options, e.g.
%    use A4 as paper format:
%       \PassOptionsToClass{a4paper}{article}
%
%    Programm calls to get the documentation (example):
%       pdflatex stackrel.dtx
%       makeindex -s gind.ist stackrel.idx
%       pdflatex stackrel.dtx
%       makeindex -s gind.ist stackrel.idx
%       pdflatex stackrel.dtx
%
% Installation:
%    TDS:tex/latex/oberdiek/stackrel.sty
%    TDS:doc/latex/oberdiek/stackrel.pdf
%    TDS:source/latex/oberdiek/stackrel.dtx
%
%<*ignore>
\begingroup
  \catcode123=1 %
  \catcode125=2 %
  \def\x{LaTeX2e}%
\expandafter\endgroup
\ifcase 0\ifx\install y1\fi\expandafter
         \ifx\csname processbatchFile\endcsname\relax\else1\fi
         \ifx\fmtname\x\else 1\fi\relax
\else\csname fi\endcsname
%</ignore>
%<*install>
\input docstrip.tex
\Msg{************************************************************************}
\Msg{* Installation}
\Msg{* Package: stackrel 2016/05/16 v1.3 Adding subscript option to stackrel (HO)}
\Msg{************************************************************************}

\keepsilent
\askforoverwritefalse

\let\MetaPrefix\relax
\preamble

This is a generated file.

Project: stackrel
Version: 2016/05/16 v1.3

Copyright (C) 2006, 2007 by
   Heiko Oberdiek <heiko.oberdiek at googlemail.com>

This work may be distributed and/or modified under the
conditions of the LaTeX Project Public License, either
version 1.3c of this license or (at your option) any later
version. This version of this license is in
   http://www.latex-project.org/lppl/lppl-1-3c.txt
and the latest version of this license is in
   http://www.latex-project.org/lppl.txt
and version 1.3 or later is part of all distributions of
LaTeX version 2005/12/01 or later.

This work has the LPPL maintenance status "maintained".

This Current Maintainer of this work is Heiko Oberdiek.

This work consists of the main source file stackrel.dtx
and the derived files
   stackrel.sty, stackrel.pdf, stackrel.ins, stackrel.drv.

\endpreamble
\let\MetaPrefix\DoubleperCent

\generate{%
  \file{stackrel.ins}{\from{stackrel.dtx}{install}}%
  \file{stackrel.drv}{\from{stackrel.dtx}{driver}}%
  \usedir{tex/latex/oberdiek}%
  \file{stackrel.sty}{\from{stackrel.dtx}{package}}%
  \nopreamble
  \nopostamble
%  \usedir{source/latex/oberdiek/catalogue}%
%  \file{stackrel.xml}{\from{stackrel.dtx}{catalogue}}%
}

\catcode32=13\relax% active space
\let =\space%
\Msg{************************************************************************}
\Msg{*}
\Msg{* To finish the installation you have to move the following}
\Msg{* file into a directory searched by TeX:}
\Msg{*}
\Msg{*     stackrel.sty}
\Msg{*}
\Msg{* To produce the documentation run the file `stackrel.drv'}
\Msg{* through LaTeX.}
\Msg{*}
\Msg{* Happy TeXing!}
\Msg{*}
\Msg{************************************************************************}

\endbatchfile
%</install>
%<*ignore>
\fi
%</ignore>
%<*driver>
\NeedsTeXFormat{LaTeX2e}
\ProvidesFile{stackrel.drv}%
  [2016/05/16 v1.3 Adding subscript option to stackrel (HO)]%
\documentclass{ltxdoc}
\usepackage{amsmath}
\usepackage{holtxdoc}[2011/11/22]
\usepackage{stackrel}[2016/05/16]
\begin{document}
  \DocInput{stackrel.dtx}%
\end{document}
%</driver>
% \fi
%
%
% \CharacterTable
%  {Upper-case    \A\B\C\D\E\F\G\H\I\J\K\L\M\N\O\P\Q\R\S\T\U\V\W\X\Y\Z
%   Lower-case    \a\b\c\d\e\f\g\h\i\j\k\l\m\n\o\p\q\r\s\t\u\v\w\x\y\z
%   Digits        \0\1\2\3\4\5\6\7\8\9
%   Exclamation   \!     Double quote  \"     Hash (number) \#
%   Dollar        \$     Percent       \%     Ampersand     \&
%   Acute accent  \'     Left paren    \(     Right paren   \)
%   Asterisk      \*     Plus          \+     Comma         \,
%   Minus         \-     Point         \.     Solidus       \/
%   Colon         \:     Semicolon     \;     Less than     \<
%   Equals        \=     Greater than  \>     Question mark \?
%   Commercial at \@     Left bracket  \[     Backslash     \\
%   Right bracket \]     Circumflex    \^     Underscore    \_
%   Grave accent  \`     Left brace    \{     Vertical bar  \|
%   Right brace   \}     Tilde         \~}
%
% \GetFileInfo{stackrel.drv}
%
% \title{The \xpackage{stackrel} package}
% \date{2016/05/16 v1.3}
% \author{Heiko Oberdiek\thanks
% {Please report any issues at \url{https://github.com/ho-tex/oberdiek/issues}}\\
% \xemail{heiko.oberdiek at googlemail.com}}
%
% \maketitle
%
% \begin{abstract}
% This package adds an optional argument to \cs{stackrel} for
% putting something below the relational symbol and defines
% \cs{stackbin} for binary symbols.
% \end{abstract}
%
% \tableofcontents
%
% \section{User interface}
%
% \LaTeX's \cs{stackrel} allows a superscript above a relational symbol,
% but pure \LaTeX\ does not provide a macro for putting a subscript
% below the symbol. This is supported by \AmS\LaTeX's \cs{underset}
% macro that works on both relational and binary symbols. A combination
% of \cs{underset} and \cs{overset} can be used to put \mbox{sub-} and
% superscripts to the same symbol.
%
% This package \xpackage{stackrel} extends the syntax of \cs{stackrel}
% by adding an optional argument for the subscript position.
% It follows the syntax of extensible arrows of packages
% \xpackage{amsmath} and \xpackage{mathtools}.
%
% \begin{declcs}{stackrel}
%   |[|\meta{subscript}|]| \M{superscript} \M{rel}\\
%   \cs{stackbin}
%   |[|\meta{subscript}|]| \M{superscript} \M{bin}
% \end{declcs}
% Example:
% \begin{quote}
% |A \stackbin[\text{and}]{}{+} B \stackrel[x]{!}{=} C|\\
% $A \stackbin[\text{and}]{}{+} B \stackrel[x]{!}{=} C$
% \end{quote}
%
% \StopEventually{
% }
%
% \section{Implementation}
%
%    \begin{macrocode}
%<*package>
\NeedsTeXFormat{LaTeX2e}
\ProvidesPackage{stackrel}
  [2016/05/16 v1.3 Adding subscript option to stackrel (HO)]%
%    \end{macrocode}
%
%    Given the original definition of \cs{stackrel} the addition
%    of the optional argument is straightforward. If an argument
%    is empty, then the corresponding sub- or superscript is
%    suppressed.
%
%    Depending on the available resources (\eTeX, \pdfTeX)
%    three methods are given for testing emptyness. All tests
%    allow the hash to be used inside the arguments without
%    doubling (for the unlikely case that someone wants to
%    define macros with arguments).
%    \begin{macro}{\stack@relbin}
%    \begin{macrocode}
\RequirePackage{etexcmds}[2007/09/09]
\ifetex@unexpanded
  \RequirePackage{pdftexcmds}[2016/05/16]%
  \begingroup\expandafter\expandafter\expandafter\endgroup
  \expandafter\ifx\csname pdf@strcmp\endcsname\relax
    \newcommand*{\stack@relbin}[3][]{%
      \mathop{#3}\limits
      \edef\reserved@a{\etex@unexpanded{#1}}%
      \ifx\reserved@a\@empty\else_{#1}\fi
      \edef\reserved@a{\etex@unexpanded{#2}}%
      \ifx\reserved@a\@empty\else^{#2}\fi
      \egroup
    }%
  \else
    \newcommand*{\stack@relbin}[3][]{%
      \mathop{#3}\limits
      \ifcase\pdf@strcmp{\detokenize{#1}}{}\else_{#1}\fi
      \ifcase\pdf@strcmp{\detokenize{#2}}{}\else^{#2}\fi
      \egroup
    }%
  \fi
\else
  \newcommand*{\stack@relbin}[3][]{%
    \mathop{#3}\limits
    \toks@{#1}%
    \edef\reserved@a{\the\toks@}%
    \ifx\reserved@a\@empty\else_{#1}\fi
    \toks@{#2}%
    \edef\reserved@a{\the\toks@}%
    \ifx\reserved@a\@empty\else^{#2}\fi
    \egroup
  }%
\fi
%    \end{macrocode}
%    \end{macro}
%    \begin{macro}{\stackrel}
%    \begin{macrocode}
\renewcommand*{\stackrel}{%
  \mathrel\bgroup\stack@relbin
}
%    \end{macrocode}
%    \end{macro}
%    \begin{macro}{\stackbin}
%    \begin{macrocode}
\newcommand*{\stackbin}{%
  \mathbin\bgroup\stack@relbin
}
%    \end{macrocode}
%    \end{macro}
%
%    \begin{macrocode}
%</package>
%    \end{macrocode}
%
% \section{Installation}
%
% \subsection{Download}
%
% \paragraph{Package.} This package is available on
% CTAN\footnote{\CTANpkg{stackrel}}:
% \begin{description}
% \item[\CTAN{macros/latex/contrib/oberdiek/stackrel.dtx}] The source file.
% \item[\CTAN{macros/latex/contrib/oberdiek/stackrel.pdf}] Documentation.
% \end{description}
%
%
% \paragraph{Bundle.} All the packages of the bundle `oberdiek'
% are also available in a TDS compliant ZIP archive. There
% the packages are already unpacked and the documentation files
% are generated. The files and directories obey the TDS standard.
% \begin{description}
% \item[\CTANinstall{install/macros/latex/contrib/oberdiek.tds.zip}]
% \end{description}
% \emph{TDS} refers to the standard ``A Directory Structure
% for \TeX\ Files'' (\CTAN{tds/tds.pdf}). Directories
% with \xfile{texmf} in their name are usually organized this way.
%
% \subsection{Bundle installation}
%
% \paragraph{Unpacking.} Unpack the \xfile{oberdiek.tds.zip} in the
% TDS tree (also known as \xfile{texmf} tree) of your choice.
% Example (linux):
% \begin{quote}
%   |unzip oberdiek.tds.zip -d ~/texmf|
% \end{quote}
%
% \paragraph{Script installation.}
% Check the directory \xfile{TDS:scripts/oberdiek/} for
% scripts that need further installation steps.
% Package \xpackage{attachfile2} comes with the Perl script
% \xfile{pdfatfi.pl} that should be installed in such a way
% that it can be called as \texttt{pdfatfi}.
% Example (linux):
% \begin{quote}
%   |chmod +x scripts/oberdiek/pdfatfi.pl|\\
%   |cp scripts/oberdiek/pdfatfi.pl /usr/local/bin/|
% \end{quote}
%
% \subsection{Package installation}
%
% \paragraph{Unpacking.} The \xfile{.dtx} file is a self-extracting
% \docstrip\ archive. The files are extracted by running the
% \xfile{.dtx} through \plainTeX:
% \begin{quote}
%   \verb|tex stackrel.dtx|
% \end{quote}
%
% \paragraph{TDS.} Now the different files must be moved into
% the different directories in your installation TDS tree
% (also known as \xfile{texmf} tree):
% \begin{quote}
% \def\t{^^A
% \begin{tabular}{@{}>{\ttfamily}l@{ $\rightarrow$ }>{\ttfamily}l@{}}
%   stackrel.sty & tex/latex/oberdiek/stackrel.sty\\
%   stackrel.pdf & doc/latex/oberdiek/stackrel.pdf\\
%   stackrel.dtx & source/latex/oberdiek/stackrel.dtx\\
% \end{tabular}^^A
% }^^A
% \sbox0{\t}^^A
% \ifdim\wd0>\linewidth
%   \begingroup
%     \advance\linewidth by\leftmargin
%     \advance\linewidth by\rightmargin
%   \edef\x{\endgroup
%     \def\noexpand\lw{\the\linewidth}^^A
%   }\x
%   \def\lwbox{^^A
%     \leavevmode
%     \hbox to \linewidth{^^A
%       \kern-\leftmargin\relax
%       \hss
%       \usebox0
%       \hss
%       \kern-\rightmargin\relax
%     }^^A
%   }^^A
%   \ifdim\wd0>\lw
%     \sbox0{\small\t}^^A
%     \ifdim\wd0>\linewidth
%       \ifdim\wd0>\lw
%         \sbox0{\footnotesize\t}^^A
%         \ifdim\wd0>\linewidth
%           \ifdim\wd0>\lw
%             \sbox0{\scriptsize\t}^^A
%             \ifdim\wd0>\linewidth
%               \ifdim\wd0>\lw
%                 \sbox0{\tiny\t}^^A
%                 \ifdim\wd0>\linewidth
%                   \lwbox
%                 \else
%                   \usebox0
%                 \fi
%               \else
%                 \lwbox
%               \fi
%             \else
%               \usebox0
%             \fi
%           \else
%             \lwbox
%           \fi
%         \else
%           \usebox0
%         \fi
%       \else
%         \lwbox
%       \fi
%     \else
%       \usebox0
%     \fi
%   \else
%     \lwbox
%   \fi
% \else
%   \usebox0
% \fi
% \end{quote}
% If you have a \xfile{docstrip.cfg} that configures and enables \docstrip's
% TDS installing feature, then some files can already be in the right
% place, see the documentation of \docstrip.
%
% \subsection{Refresh file name databases}
%
% If your \TeX~distribution
% (\teTeX, \mikTeX, \dots) relies on file name databases, you must refresh
% these. For example, \teTeX\ users run \verb|texhash| or
% \verb|mktexlsr|.
%
% \subsection{Some details for the interested}
%
% \paragraph{Attached source.}
%
% The PDF documentation on CTAN also includes the
% \xfile{.dtx} source file. It can be extracted by
% AcrobatReader 6 or higher. Another option is \textsf{pdftk},
% e.g. unpack the file into the current directory:
% \begin{quote}
%   \verb|pdftk stackrel.pdf unpack_files output .|
% \end{quote}
%
% \paragraph{Unpacking with \LaTeX.}
% The \xfile{.dtx} chooses its action depending on the format:
% \begin{description}
% \item[\plainTeX:] Run \docstrip\ and extract the files.
% \item[\LaTeX:] Generate the documentation.
% \end{description}
% If you insist on using \LaTeX\ for \docstrip\ (really,
% \docstrip\ does not need \LaTeX), then inform the autodetect routine
% about your intention:
% \begin{quote}
%   \verb|latex \let\install=y\input{stackrel.dtx}|
% \end{quote}
% Do not forget to quote the argument according to the demands
% of your shell.
%
% \paragraph{Generating the documentation.}
% You can use both the \xfile{.dtx} or the \xfile{.drv} to generate
% the documentation. The process can be configured by the
% configuration file \xfile{ltxdoc.cfg}. For instance, put this
% line into this file, if you want to have A4 as paper format:
% \begin{quote}
%   \verb|\PassOptionsToClass{a4paper}{article}|
% \end{quote}
% An example follows how to generate the
% documentation with pdf\LaTeX:
% \begin{quote}
%\begin{verbatim}
%pdflatex stackrel.dtx
%makeindex -s gind.ist stackrel.idx
%pdflatex stackrel.dtx
%makeindex -s gind.ist stackrel.idx
%pdflatex stackrel.dtx
%\end{verbatim}
% \end{quote}
%
% \begin{History}
%   \begin{Version}{2006/12/02 v1.0}
%   \item
%     First version.
%   \end{Version}
%   \begin{Version}{2007/05/06 v1.1}
%   \item
%     Uses package \xpackage{etexcmds}.
%   \end{Version}
%   \begin{Version}{2007/11/11 v1.2}
%   \item
%     Use of package \xpackage{pdftexcmds} for \LuaTeX\ support.
%   \end{Version}
%   \begin{Version}{2016/05/16 v1.3}
%   \item
%     Documentation updates.
%   \end{Version}
% \end{History}
%
% \clearpage
% \PrintIndex
%
% \Finale
\endinput
|
% \end{quote}
% Do not forget to quote the argument according to the demands
% of your shell.
%
% \paragraph{Generating the documentation.}
% You can use both the \xfile{.dtx} or the \xfile{.drv} to generate
% the documentation. The process can be configured by the
% configuration file \xfile{ltxdoc.cfg}. For instance, put this
% line into this file, if you want to have A4 as paper format:
% \begin{quote}
%   \verb|\PassOptionsToClass{a4paper}{article}|
% \end{quote}
% An example follows how to generate the
% documentation with pdf\LaTeX:
% \begin{quote}
%\begin{verbatim}
%pdflatex stackrel.dtx
%makeindex -s gind.ist stackrel.idx
%pdflatex stackrel.dtx
%makeindex -s gind.ist stackrel.idx
%pdflatex stackrel.dtx
%\end{verbatim}
% \end{quote}
%
% \begin{History}
%   \begin{Version}{2006/12/02 v1.0}
%   \item
%     First version.
%   \end{Version}
%   \begin{Version}{2007/05/06 v1.1}
%   \item
%     Uses package \xpackage{etexcmds}.
%   \end{Version}
%   \begin{Version}{2007/11/11 v1.2}
%   \item
%     Use of package \xpackage{pdftexcmds} for \LuaTeX\ support.
%   \end{Version}
%   \begin{Version}{2016/05/16 v1.3}
%   \item
%     Documentation updates.
%   \end{Version}
% \end{History}
%
% \clearpage
% \PrintIndex
%
% \Finale
\endinput

%        (quote the arguments according to the demands of your shell)
%
% Documentation:
%    (a) If stackrel.drv is present:
%           latex stackrel.drv
%    (b) Without stackrel.drv:
%           latex stackrel.dtx; ...
%    The class ltxdoc loads the configuration file ltxdoc.cfg
%    if available. Here you can specify further options, e.g.
%    use A4 as paper format:
%       \PassOptionsToClass{a4paper}{article}
%
%    Programm calls to get the documentation (example):
%       pdflatex stackrel.dtx
%       makeindex -s gind.ist stackrel.idx
%       pdflatex stackrel.dtx
%       makeindex -s gind.ist stackrel.idx
%       pdflatex stackrel.dtx
%
% Installation:
%    TDS:tex/latex/oberdiek/stackrel.sty
%    TDS:doc/latex/oberdiek/stackrel.pdf
%    TDS:source/latex/oberdiek/stackrel.dtx
%
%<*ignore>
\begingroup
  \catcode123=1 %
  \catcode125=2 %
  \def\x{LaTeX2e}%
\expandafter\endgroup
\ifcase 0\ifx\install y1\fi\expandafter
         \ifx\csname processbatchFile\endcsname\relax\else1\fi
         \ifx\fmtname\x\else 1\fi\relax
\else\csname fi\endcsname
%</ignore>
%<*install>
\input docstrip.tex
\Msg{************************************************************************}
\Msg{* Installation}
\Msg{* Package: stackrel 2016/05/16 v1.3 Adding subscript option to stackrel (HO)}
\Msg{************************************************************************}

\keepsilent
\askforoverwritefalse

\let\MetaPrefix\relax
\preamble

This is a generated file.

Project: stackrel
Version: 2016/05/16 v1.3

Copyright (C) 2006, 2007 by
   Heiko Oberdiek <heiko.oberdiek at googlemail.com>

This work may be distributed and/or modified under the
conditions of the LaTeX Project Public License, either
version 1.3c of this license or (at your option) any later
version. This version of this license is in
   http://www.latex-project.org/lppl/lppl-1-3c.txt
and the latest version of this license is in
   http://www.latex-project.org/lppl.txt
and version 1.3 or later is part of all distributions of
LaTeX version 2005/12/01 or later.

This work has the LPPL maintenance status "maintained".

This Current Maintainer of this work is Heiko Oberdiek.

This work consists of the main source file stackrel.dtx
and the derived files
   stackrel.sty, stackrel.pdf, stackrel.ins, stackrel.drv.

\endpreamble
\let\MetaPrefix\DoubleperCent

\generate{%
  \file{stackrel.ins}{\from{stackrel.dtx}{install}}%
  \file{stackrel.drv}{\from{stackrel.dtx}{driver}}%
  \usedir{tex/latex/oberdiek}%
  \file{stackrel.sty}{\from{stackrel.dtx}{package}}%
  \nopreamble
  \nopostamble
%  \usedir{source/latex/oberdiek/catalogue}%
%  \file{stackrel.xml}{\from{stackrel.dtx}{catalogue}}%
}

\catcode32=13\relax% active space
\let =\space%
\Msg{************************************************************************}
\Msg{*}
\Msg{* To finish the installation you have to move the following}
\Msg{* file into a directory searched by TeX:}
\Msg{*}
\Msg{*     stackrel.sty}
\Msg{*}
\Msg{* To produce the documentation run the file `stackrel.drv'}
\Msg{* through LaTeX.}
\Msg{*}
\Msg{* Happy TeXing!}
\Msg{*}
\Msg{************************************************************************}

\endbatchfile
%</install>
%<*ignore>
\fi
%</ignore>
%<*driver>
\NeedsTeXFormat{LaTeX2e}
\ProvidesFile{stackrel.drv}%
  [2016/05/16 v1.3 Adding subscript option to stackrel (HO)]%
\documentclass{ltxdoc}
\usepackage{amsmath}
\usepackage{holtxdoc}[2011/11/22]
\usepackage{stackrel}[2016/05/16]
\begin{document}
  \DocInput{stackrel.dtx}%
\end{document}
%</driver>
% \fi
%
%
% \CharacterTable
%  {Upper-case    \A\B\C\D\E\F\G\H\I\J\K\L\M\N\O\P\Q\R\S\T\U\V\W\X\Y\Z
%   Lower-case    \a\b\c\d\e\f\g\h\i\j\k\l\m\n\o\p\q\r\s\t\u\v\w\x\y\z
%   Digits        \0\1\2\3\4\5\6\7\8\9
%   Exclamation   \!     Double quote  \"     Hash (number) \#
%   Dollar        \$     Percent       \%     Ampersand     \&
%   Acute accent  \'     Left paren    \(     Right paren   \)
%   Asterisk      \*     Plus          \+     Comma         \,
%   Minus         \-     Point         \.     Solidus       \/
%   Colon         \:     Semicolon     \;     Less than     \<
%   Equals        \=     Greater than  \>     Question mark \?
%   Commercial at \@     Left bracket  \[     Backslash     \\
%   Right bracket \]     Circumflex    \^     Underscore    \_
%   Grave accent  \`     Left brace    \{     Vertical bar  \|
%   Right brace   \}     Tilde         \~}
%
% \GetFileInfo{stackrel.drv}
%
% \title{The \xpackage{stackrel} package}
% \date{2016/05/16 v1.3}
% \author{Heiko Oberdiek\thanks
% {Please report any issues at \url{https://github.com/ho-tex/oberdiek/issues}}\\
% \xemail{heiko.oberdiek at googlemail.com}}
%
% \maketitle
%
% \begin{abstract}
% This package adds an optional argument to \cs{stackrel} for
% putting something below the relational symbol and defines
% \cs{stackbin} for binary symbols.
% \end{abstract}
%
% \tableofcontents
%
% \section{User interface}
%
% \LaTeX's \cs{stackrel} allows a superscript above a relational symbol,
% but pure \LaTeX\ does not provide a macro for putting a subscript
% below the symbol. This is supported by \AmS\LaTeX's \cs{underset}
% macro that works on both relational and binary symbols. A combination
% of \cs{underset} and \cs{overset} can be used to put \mbox{sub-} and
% superscripts to the same symbol.
%
% This package \xpackage{stackrel} extends the syntax of \cs{stackrel}
% by adding an optional argument for the subscript position.
% It follows the syntax of extensible arrows of packages
% \xpackage{amsmath} and \xpackage{mathtools}.
%
% \begin{declcs}{stackrel}
%   |[|\meta{subscript}|]| \M{superscript} \M{rel}\\
%   \cs{stackbin}
%   |[|\meta{subscript}|]| \M{superscript} \M{bin}
% \end{declcs}
% Example:
% \begin{quote}
% |A \stackbin[\text{and}]{}{+} B \stackrel[x]{!}{=} C|\\
% $A \stackbin[\text{and}]{}{+} B \stackrel[x]{!}{=} C$
% \end{quote}
%
% \StopEventually{
% }
%
% \section{Implementation}
%
%    \begin{macrocode}
%<*package>
\NeedsTeXFormat{LaTeX2e}
\ProvidesPackage{stackrel}
  [2016/05/16 v1.3 Adding subscript option to stackrel (HO)]%
%    \end{macrocode}
%
%    Given the original definition of \cs{stackrel} the addition
%    of the optional argument is straightforward. If an argument
%    is empty, then the corresponding sub- or superscript is
%    suppressed.
%
%    Depending on the available resources (\eTeX, \pdfTeX)
%    three methods are given for testing emptyness. All tests
%    allow the hash to be used inside the arguments without
%    doubling (for the unlikely case that someone wants to
%    define macros with arguments).
%    \begin{macro}{\stack@relbin}
%    \begin{macrocode}
\RequirePackage{etexcmds}[2007/09/09]
\ifetex@unexpanded
  \RequirePackage{pdftexcmds}[2016/05/16]%
  \begingroup\expandafter\expandafter\expandafter\endgroup
  \expandafter\ifx\csname pdf@strcmp\endcsname\relax
    \newcommand*{\stack@relbin}[3][]{%
      \mathop{#3}\limits
      \edef\reserved@a{\etex@unexpanded{#1}}%
      \ifx\reserved@a\@empty\else_{#1}\fi
      \edef\reserved@a{\etex@unexpanded{#2}}%
      \ifx\reserved@a\@empty\else^{#2}\fi
      \egroup
    }%
  \else
    \newcommand*{\stack@relbin}[3][]{%
      \mathop{#3}\limits
      \ifcase\pdf@strcmp{\detokenize{#1}}{}\else_{#1}\fi
      \ifcase\pdf@strcmp{\detokenize{#2}}{}\else^{#2}\fi
      \egroup
    }%
  \fi
\else
  \newcommand*{\stack@relbin}[3][]{%
    \mathop{#3}\limits
    \toks@{#1}%
    \edef\reserved@a{\the\toks@}%
    \ifx\reserved@a\@empty\else_{#1}\fi
    \toks@{#2}%
    \edef\reserved@a{\the\toks@}%
    \ifx\reserved@a\@empty\else^{#2}\fi
    \egroup
  }%
\fi
%    \end{macrocode}
%    \end{macro}
%    \begin{macro}{\stackrel}
%    \begin{macrocode}
\renewcommand*{\stackrel}{%
  \mathrel\bgroup\stack@relbin
}
%    \end{macrocode}
%    \end{macro}
%    \begin{macro}{\stackbin}
%    \begin{macrocode}
\newcommand*{\stackbin}{%
  \mathbin\bgroup\stack@relbin
}
%    \end{macrocode}
%    \end{macro}
%
%    \begin{macrocode}
%</package>
%    \end{macrocode}
%
% \section{Installation}
%
% \subsection{Download}
%
% \paragraph{Package.} This package is available on
% CTAN\footnote{\CTANpkg{stackrel}}:
% \begin{description}
% \item[\CTAN{macros/latex/contrib/oberdiek/stackrel.dtx}] The source file.
% \item[\CTAN{macros/latex/contrib/oberdiek/stackrel.pdf}] Documentation.
% \end{description}
%
%
% \paragraph{Bundle.} All the packages of the bundle `oberdiek'
% are also available in a TDS compliant ZIP archive. There
% the packages are already unpacked and the documentation files
% are generated. The files and directories obey the TDS standard.
% \begin{description}
% \item[\CTANinstall{install/macros/latex/contrib/oberdiek.tds.zip}]
% \end{description}
% \emph{TDS} refers to the standard ``A Directory Structure
% for \TeX\ Files'' (\CTAN{tds/tds.pdf}). Directories
% with \xfile{texmf} in their name are usually organized this way.
%
% \subsection{Bundle installation}
%
% \paragraph{Unpacking.} Unpack the \xfile{oberdiek.tds.zip} in the
% TDS tree (also known as \xfile{texmf} tree) of your choice.
% Example (linux):
% \begin{quote}
%   |unzip oberdiek.tds.zip -d ~/texmf|
% \end{quote}
%
% \paragraph{Script installation.}
% Check the directory \xfile{TDS:scripts/oberdiek/} for
% scripts that need further installation steps.
% Package \xpackage{attachfile2} comes with the Perl script
% \xfile{pdfatfi.pl} that should be installed in such a way
% that it can be called as \texttt{pdfatfi}.
% Example (linux):
% \begin{quote}
%   |chmod +x scripts/oberdiek/pdfatfi.pl|\\
%   |cp scripts/oberdiek/pdfatfi.pl /usr/local/bin/|
% \end{quote}
%
% \subsection{Package installation}
%
% \paragraph{Unpacking.} The \xfile{.dtx} file is a self-extracting
% \docstrip\ archive. The files are extracted by running the
% \xfile{.dtx} through \plainTeX:
% \begin{quote}
%   \verb|tex stackrel.dtx|
% \end{quote}
%
% \paragraph{TDS.} Now the different files must be moved into
% the different directories in your installation TDS tree
% (also known as \xfile{texmf} tree):
% \begin{quote}
% \def\t{^^A
% \begin{tabular}{@{}>{\ttfamily}l@{ $\rightarrow$ }>{\ttfamily}l@{}}
%   stackrel.sty & tex/latex/oberdiek/stackrel.sty\\
%   stackrel.pdf & doc/latex/oberdiek/stackrel.pdf\\
%   stackrel.dtx & source/latex/oberdiek/stackrel.dtx\\
% \end{tabular}^^A
% }^^A
% \sbox0{\t}^^A
% \ifdim\wd0>\linewidth
%   \begingroup
%     \advance\linewidth by\leftmargin
%     \advance\linewidth by\rightmargin
%   \edef\x{\endgroup
%     \def\noexpand\lw{\the\linewidth}^^A
%   }\x
%   \def\lwbox{^^A
%     \leavevmode
%     \hbox to \linewidth{^^A
%       \kern-\leftmargin\relax
%       \hss
%       \usebox0
%       \hss
%       \kern-\rightmargin\relax
%     }^^A
%   }^^A
%   \ifdim\wd0>\lw
%     \sbox0{\small\t}^^A
%     \ifdim\wd0>\linewidth
%       \ifdim\wd0>\lw
%         \sbox0{\footnotesize\t}^^A
%         \ifdim\wd0>\linewidth
%           \ifdim\wd0>\lw
%             \sbox0{\scriptsize\t}^^A
%             \ifdim\wd0>\linewidth
%               \ifdim\wd0>\lw
%                 \sbox0{\tiny\t}^^A
%                 \ifdim\wd0>\linewidth
%                   \lwbox
%                 \else
%                   \usebox0
%                 \fi
%               \else
%                 \lwbox
%               \fi
%             \else
%               \usebox0
%             \fi
%           \else
%             \lwbox
%           \fi
%         \else
%           \usebox0
%         \fi
%       \else
%         \lwbox
%       \fi
%     \else
%       \usebox0
%     \fi
%   \else
%     \lwbox
%   \fi
% \else
%   \usebox0
% \fi
% \end{quote}
% If you have a \xfile{docstrip.cfg} that configures and enables \docstrip's
% TDS installing feature, then some files can already be in the right
% place, see the documentation of \docstrip.
%
% \subsection{Refresh file name databases}
%
% If your \TeX~distribution
% (\teTeX, \mikTeX, \dots) relies on file name databases, you must refresh
% these. For example, \teTeX\ users run \verb|texhash| or
% \verb|mktexlsr|.
%
% \subsection{Some details for the interested}
%
% \paragraph{Attached source.}
%
% The PDF documentation on CTAN also includes the
% \xfile{.dtx} source file. It can be extracted by
% AcrobatReader 6 or higher. Another option is \textsf{pdftk},
% e.g. unpack the file into the current directory:
% \begin{quote}
%   \verb|pdftk stackrel.pdf unpack_files output .|
% \end{quote}
%
% \paragraph{Unpacking with \LaTeX.}
% The \xfile{.dtx} chooses its action depending on the format:
% \begin{description}
% \item[\plainTeX:] Run \docstrip\ and extract the files.
% \item[\LaTeX:] Generate the documentation.
% \end{description}
% If you insist on using \LaTeX\ for \docstrip\ (really,
% \docstrip\ does not need \LaTeX), then inform the autodetect routine
% about your intention:
% \begin{quote}
%   \verb|latex \let\install=y% \iffalse meta-comment
%
% File: stackrel.dtx
% Version: 2016/05/16 v1.3
% Info: Adding subscript option to stackrel
%
% Copyright (C) 2006, 2007 by
%    Heiko Oberdiek <heiko.oberdiek at googlemail.com>
%    2016
%    https://github.com/ho-tex/oberdiek/issues
%
% This work may be distributed and/or modified under the
% conditions of the LaTeX Project Public License, either
% version 1.3c of this license or (at your option) any later
% version. This version of this license is in
%    http://www.latex-project.org/lppl/lppl-1-3c.txt
% and the latest version of this license is in
%    http://www.latex-project.org/lppl.txt
% and version 1.3 or later is part of all distributions of
% LaTeX version 2005/12/01 or later.
%
% This work has the LPPL maintenance status "maintained".
%
% This Current Maintainer of this work is Heiko Oberdiek.
%
% This work consists of the main source file stackrel.dtx
% and the derived files
%    stackrel.sty, stackrel.pdf, stackrel.ins, stackrel.drv.
%
% Distribution:
%    CTAN:macros/latex/contrib/oberdiek/stackrel.dtx
%    CTAN:macros/latex/contrib/oberdiek/stackrel.pdf
%
% Unpacking:
%    (a) If stackrel.ins is present:
%           tex stackrel.ins
%    (b) Without stackrel.ins:
%           tex stackrel.dtx
%    (c) If you insist on using LaTeX
%           latex \let\install=y% \iffalse meta-comment
%
% File: stackrel.dtx
% Version: 2016/05/16 v1.3
% Info: Adding subscript option to stackrel
%
% Copyright (C) 2006, 2007 by
%    Heiko Oberdiek <heiko.oberdiek at googlemail.com>
%    2016
%    https://github.com/ho-tex/oberdiek/issues
%
% This work may be distributed and/or modified under the
% conditions of the LaTeX Project Public License, either
% version 1.3c of this license or (at your option) any later
% version. This version of this license is in
%    http://www.latex-project.org/lppl/lppl-1-3c.txt
% and the latest version of this license is in
%    http://www.latex-project.org/lppl.txt
% and version 1.3 or later is part of all distributions of
% LaTeX version 2005/12/01 or later.
%
% This work has the LPPL maintenance status "maintained".
%
% This Current Maintainer of this work is Heiko Oberdiek.
%
% This work consists of the main source file stackrel.dtx
% and the derived files
%    stackrel.sty, stackrel.pdf, stackrel.ins, stackrel.drv.
%
% Distribution:
%    CTAN:macros/latex/contrib/oberdiek/stackrel.dtx
%    CTAN:macros/latex/contrib/oberdiek/stackrel.pdf
%
% Unpacking:
%    (a) If stackrel.ins is present:
%           tex stackrel.ins
%    (b) Without stackrel.ins:
%           tex stackrel.dtx
%    (c) If you insist on using LaTeX
%           latex \let\install=y\input{stackrel.dtx}
%        (quote the arguments according to the demands of your shell)
%
% Documentation:
%    (a) If stackrel.drv is present:
%           latex stackrel.drv
%    (b) Without stackrel.drv:
%           latex stackrel.dtx; ...
%    The class ltxdoc loads the configuration file ltxdoc.cfg
%    if available. Here you can specify further options, e.g.
%    use A4 as paper format:
%       \PassOptionsToClass{a4paper}{article}
%
%    Programm calls to get the documentation (example):
%       pdflatex stackrel.dtx
%       makeindex -s gind.ist stackrel.idx
%       pdflatex stackrel.dtx
%       makeindex -s gind.ist stackrel.idx
%       pdflatex stackrel.dtx
%
% Installation:
%    TDS:tex/latex/oberdiek/stackrel.sty
%    TDS:doc/latex/oberdiek/stackrel.pdf
%    TDS:source/latex/oberdiek/stackrel.dtx
%
%<*ignore>
\begingroup
  \catcode123=1 %
  \catcode125=2 %
  \def\x{LaTeX2e}%
\expandafter\endgroup
\ifcase 0\ifx\install y1\fi\expandafter
         \ifx\csname processbatchFile\endcsname\relax\else1\fi
         \ifx\fmtname\x\else 1\fi\relax
\else\csname fi\endcsname
%</ignore>
%<*install>
\input docstrip.tex
\Msg{************************************************************************}
\Msg{* Installation}
\Msg{* Package: stackrel 2016/05/16 v1.3 Adding subscript option to stackrel (HO)}
\Msg{************************************************************************}

\keepsilent
\askforoverwritefalse

\let\MetaPrefix\relax
\preamble

This is a generated file.

Project: stackrel
Version: 2016/05/16 v1.3

Copyright (C) 2006, 2007 by
   Heiko Oberdiek <heiko.oberdiek at googlemail.com>

This work may be distributed and/or modified under the
conditions of the LaTeX Project Public License, either
version 1.3c of this license or (at your option) any later
version. This version of this license is in
   http://www.latex-project.org/lppl/lppl-1-3c.txt
and the latest version of this license is in
   http://www.latex-project.org/lppl.txt
and version 1.3 or later is part of all distributions of
LaTeX version 2005/12/01 or later.

This work has the LPPL maintenance status "maintained".

This Current Maintainer of this work is Heiko Oberdiek.

This work consists of the main source file stackrel.dtx
and the derived files
   stackrel.sty, stackrel.pdf, stackrel.ins, stackrel.drv.

\endpreamble
\let\MetaPrefix\DoubleperCent

\generate{%
  \file{stackrel.ins}{\from{stackrel.dtx}{install}}%
  \file{stackrel.drv}{\from{stackrel.dtx}{driver}}%
  \usedir{tex/latex/oberdiek}%
  \file{stackrel.sty}{\from{stackrel.dtx}{package}}%
  \nopreamble
  \nopostamble
%  \usedir{source/latex/oberdiek/catalogue}%
%  \file{stackrel.xml}{\from{stackrel.dtx}{catalogue}}%
}

\catcode32=13\relax% active space
\let =\space%
\Msg{************************************************************************}
\Msg{*}
\Msg{* To finish the installation you have to move the following}
\Msg{* file into a directory searched by TeX:}
\Msg{*}
\Msg{*     stackrel.sty}
\Msg{*}
\Msg{* To produce the documentation run the file `stackrel.drv'}
\Msg{* through LaTeX.}
\Msg{*}
\Msg{* Happy TeXing!}
\Msg{*}
\Msg{************************************************************************}

\endbatchfile
%</install>
%<*ignore>
\fi
%</ignore>
%<*driver>
\NeedsTeXFormat{LaTeX2e}
\ProvidesFile{stackrel.drv}%
  [2016/05/16 v1.3 Adding subscript option to stackrel (HO)]%
\documentclass{ltxdoc}
\usepackage{amsmath}
\usepackage{holtxdoc}[2011/11/22]
\usepackage{stackrel}[2016/05/16]
\begin{document}
  \DocInput{stackrel.dtx}%
\end{document}
%</driver>
% \fi
%
%
% \CharacterTable
%  {Upper-case    \A\B\C\D\E\F\G\H\I\J\K\L\M\N\O\P\Q\R\S\T\U\V\W\X\Y\Z
%   Lower-case    \a\b\c\d\e\f\g\h\i\j\k\l\m\n\o\p\q\r\s\t\u\v\w\x\y\z
%   Digits        \0\1\2\3\4\5\6\7\8\9
%   Exclamation   \!     Double quote  \"     Hash (number) \#
%   Dollar        \$     Percent       \%     Ampersand     \&
%   Acute accent  \'     Left paren    \(     Right paren   \)
%   Asterisk      \*     Plus          \+     Comma         \,
%   Minus         \-     Point         \.     Solidus       \/
%   Colon         \:     Semicolon     \;     Less than     \<
%   Equals        \=     Greater than  \>     Question mark \?
%   Commercial at \@     Left bracket  \[     Backslash     \\
%   Right bracket \]     Circumflex    \^     Underscore    \_
%   Grave accent  \`     Left brace    \{     Vertical bar  \|
%   Right brace   \}     Tilde         \~}
%
% \GetFileInfo{stackrel.drv}
%
% \title{The \xpackage{stackrel} package}
% \date{2016/05/16 v1.3}
% \author{Heiko Oberdiek\thanks
% {Please report any issues at \url{https://github.com/ho-tex/oberdiek/issues}}\\
% \xemail{heiko.oberdiek at googlemail.com}}
%
% \maketitle
%
% \begin{abstract}
% This package adds an optional argument to \cs{stackrel} for
% putting something below the relational symbol and defines
% \cs{stackbin} for binary symbols.
% \end{abstract}
%
% \tableofcontents
%
% \section{User interface}
%
% \LaTeX's \cs{stackrel} allows a superscript above a relational symbol,
% but pure \LaTeX\ does not provide a macro for putting a subscript
% below the symbol. This is supported by \AmS\LaTeX's \cs{underset}
% macro that works on both relational and binary symbols. A combination
% of \cs{underset} and \cs{overset} can be used to put \mbox{sub-} and
% superscripts to the same symbol.
%
% This package \xpackage{stackrel} extends the syntax of \cs{stackrel}
% by adding an optional argument for the subscript position.
% It follows the syntax of extensible arrows of packages
% \xpackage{amsmath} and \xpackage{mathtools}.
%
% \begin{declcs}{stackrel}
%   |[|\meta{subscript}|]| \M{superscript} \M{rel}\\
%   \cs{stackbin}
%   |[|\meta{subscript}|]| \M{superscript} \M{bin}
% \end{declcs}
% Example:
% \begin{quote}
% |A \stackbin[\text{and}]{}{+} B \stackrel[x]{!}{=} C|\\
% $A \stackbin[\text{and}]{}{+} B \stackrel[x]{!}{=} C$
% \end{quote}
%
% \StopEventually{
% }
%
% \section{Implementation}
%
%    \begin{macrocode}
%<*package>
\NeedsTeXFormat{LaTeX2e}
\ProvidesPackage{stackrel}
  [2016/05/16 v1.3 Adding subscript option to stackrel (HO)]%
%    \end{macrocode}
%
%    Given the original definition of \cs{stackrel} the addition
%    of the optional argument is straightforward. If an argument
%    is empty, then the corresponding sub- or superscript is
%    suppressed.
%
%    Depending on the available resources (\eTeX, \pdfTeX)
%    three methods are given for testing emptyness. All tests
%    allow the hash to be used inside the arguments without
%    doubling (for the unlikely case that someone wants to
%    define macros with arguments).
%    \begin{macro}{\stack@relbin}
%    \begin{macrocode}
\RequirePackage{etexcmds}[2007/09/09]
\ifetex@unexpanded
  \RequirePackage{pdftexcmds}[2016/05/16]%
  \begingroup\expandafter\expandafter\expandafter\endgroup
  \expandafter\ifx\csname pdf@strcmp\endcsname\relax
    \newcommand*{\stack@relbin}[3][]{%
      \mathop{#3}\limits
      \edef\reserved@a{\etex@unexpanded{#1}}%
      \ifx\reserved@a\@empty\else_{#1}\fi
      \edef\reserved@a{\etex@unexpanded{#2}}%
      \ifx\reserved@a\@empty\else^{#2}\fi
      \egroup
    }%
  \else
    \newcommand*{\stack@relbin}[3][]{%
      \mathop{#3}\limits
      \ifcase\pdf@strcmp{\detokenize{#1}}{}\else_{#1}\fi
      \ifcase\pdf@strcmp{\detokenize{#2}}{}\else^{#2}\fi
      \egroup
    }%
  \fi
\else
  \newcommand*{\stack@relbin}[3][]{%
    \mathop{#3}\limits
    \toks@{#1}%
    \edef\reserved@a{\the\toks@}%
    \ifx\reserved@a\@empty\else_{#1}\fi
    \toks@{#2}%
    \edef\reserved@a{\the\toks@}%
    \ifx\reserved@a\@empty\else^{#2}\fi
    \egroup
  }%
\fi
%    \end{macrocode}
%    \end{macro}
%    \begin{macro}{\stackrel}
%    \begin{macrocode}
\renewcommand*{\stackrel}{%
  \mathrel\bgroup\stack@relbin
}
%    \end{macrocode}
%    \end{macro}
%    \begin{macro}{\stackbin}
%    \begin{macrocode}
\newcommand*{\stackbin}{%
  \mathbin\bgroup\stack@relbin
}
%    \end{macrocode}
%    \end{macro}
%
%    \begin{macrocode}
%</package>
%    \end{macrocode}
%
% \section{Installation}
%
% \subsection{Download}
%
% \paragraph{Package.} This package is available on
% CTAN\footnote{\CTANpkg{stackrel}}:
% \begin{description}
% \item[\CTAN{macros/latex/contrib/oberdiek/stackrel.dtx}] The source file.
% \item[\CTAN{macros/latex/contrib/oberdiek/stackrel.pdf}] Documentation.
% \end{description}
%
%
% \paragraph{Bundle.} All the packages of the bundle `oberdiek'
% are also available in a TDS compliant ZIP archive. There
% the packages are already unpacked and the documentation files
% are generated. The files and directories obey the TDS standard.
% \begin{description}
% \item[\CTANinstall{install/macros/latex/contrib/oberdiek.tds.zip}]
% \end{description}
% \emph{TDS} refers to the standard ``A Directory Structure
% for \TeX\ Files'' (\CTAN{tds/tds.pdf}). Directories
% with \xfile{texmf} in their name are usually organized this way.
%
% \subsection{Bundle installation}
%
% \paragraph{Unpacking.} Unpack the \xfile{oberdiek.tds.zip} in the
% TDS tree (also known as \xfile{texmf} tree) of your choice.
% Example (linux):
% \begin{quote}
%   |unzip oberdiek.tds.zip -d ~/texmf|
% \end{quote}
%
% \paragraph{Script installation.}
% Check the directory \xfile{TDS:scripts/oberdiek/} for
% scripts that need further installation steps.
% Package \xpackage{attachfile2} comes with the Perl script
% \xfile{pdfatfi.pl} that should be installed in such a way
% that it can be called as \texttt{pdfatfi}.
% Example (linux):
% \begin{quote}
%   |chmod +x scripts/oberdiek/pdfatfi.pl|\\
%   |cp scripts/oberdiek/pdfatfi.pl /usr/local/bin/|
% \end{quote}
%
% \subsection{Package installation}
%
% \paragraph{Unpacking.} The \xfile{.dtx} file is a self-extracting
% \docstrip\ archive. The files are extracted by running the
% \xfile{.dtx} through \plainTeX:
% \begin{quote}
%   \verb|tex stackrel.dtx|
% \end{quote}
%
% \paragraph{TDS.} Now the different files must be moved into
% the different directories in your installation TDS tree
% (also known as \xfile{texmf} tree):
% \begin{quote}
% \def\t{^^A
% \begin{tabular}{@{}>{\ttfamily}l@{ $\rightarrow$ }>{\ttfamily}l@{}}
%   stackrel.sty & tex/latex/oberdiek/stackrel.sty\\
%   stackrel.pdf & doc/latex/oberdiek/stackrel.pdf\\
%   stackrel.dtx & source/latex/oberdiek/stackrel.dtx\\
% \end{tabular}^^A
% }^^A
% \sbox0{\t}^^A
% \ifdim\wd0>\linewidth
%   \begingroup
%     \advance\linewidth by\leftmargin
%     \advance\linewidth by\rightmargin
%   \edef\x{\endgroup
%     \def\noexpand\lw{\the\linewidth}^^A
%   }\x
%   \def\lwbox{^^A
%     \leavevmode
%     \hbox to \linewidth{^^A
%       \kern-\leftmargin\relax
%       \hss
%       \usebox0
%       \hss
%       \kern-\rightmargin\relax
%     }^^A
%   }^^A
%   \ifdim\wd0>\lw
%     \sbox0{\small\t}^^A
%     \ifdim\wd0>\linewidth
%       \ifdim\wd0>\lw
%         \sbox0{\footnotesize\t}^^A
%         \ifdim\wd0>\linewidth
%           \ifdim\wd0>\lw
%             \sbox0{\scriptsize\t}^^A
%             \ifdim\wd0>\linewidth
%               \ifdim\wd0>\lw
%                 \sbox0{\tiny\t}^^A
%                 \ifdim\wd0>\linewidth
%                   \lwbox
%                 \else
%                   \usebox0
%                 \fi
%               \else
%                 \lwbox
%               \fi
%             \else
%               \usebox0
%             \fi
%           \else
%             \lwbox
%           \fi
%         \else
%           \usebox0
%         \fi
%       \else
%         \lwbox
%       \fi
%     \else
%       \usebox0
%     \fi
%   \else
%     \lwbox
%   \fi
% \else
%   \usebox0
% \fi
% \end{quote}
% If you have a \xfile{docstrip.cfg} that configures and enables \docstrip's
% TDS installing feature, then some files can already be in the right
% place, see the documentation of \docstrip.
%
% \subsection{Refresh file name databases}
%
% If your \TeX~distribution
% (\teTeX, \mikTeX, \dots) relies on file name databases, you must refresh
% these. For example, \teTeX\ users run \verb|texhash| or
% \verb|mktexlsr|.
%
% \subsection{Some details for the interested}
%
% \paragraph{Attached source.}
%
% The PDF documentation on CTAN also includes the
% \xfile{.dtx} source file. It can be extracted by
% AcrobatReader 6 or higher. Another option is \textsf{pdftk},
% e.g. unpack the file into the current directory:
% \begin{quote}
%   \verb|pdftk stackrel.pdf unpack_files output .|
% \end{quote}
%
% \paragraph{Unpacking with \LaTeX.}
% The \xfile{.dtx} chooses its action depending on the format:
% \begin{description}
% \item[\plainTeX:] Run \docstrip\ and extract the files.
% \item[\LaTeX:] Generate the documentation.
% \end{description}
% If you insist on using \LaTeX\ for \docstrip\ (really,
% \docstrip\ does not need \LaTeX), then inform the autodetect routine
% about your intention:
% \begin{quote}
%   \verb|latex \let\install=y\input{stackrel.dtx}|
% \end{quote}
% Do not forget to quote the argument according to the demands
% of your shell.
%
% \paragraph{Generating the documentation.}
% You can use both the \xfile{.dtx} or the \xfile{.drv} to generate
% the documentation. The process can be configured by the
% configuration file \xfile{ltxdoc.cfg}. For instance, put this
% line into this file, if you want to have A4 as paper format:
% \begin{quote}
%   \verb|\PassOptionsToClass{a4paper}{article}|
% \end{quote}
% An example follows how to generate the
% documentation with pdf\LaTeX:
% \begin{quote}
%\begin{verbatim}
%pdflatex stackrel.dtx
%makeindex -s gind.ist stackrel.idx
%pdflatex stackrel.dtx
%makeindex -s gind.ist stackrel.idx
%pdflatex stackrel.dtx
%\end{verbatim}
% \end{quote}
%
% \begin{History}
%   \begin{Version}{2006/12/02 v1.0}
%   \item
%     First version.
%   \end{Version}
%   \begin{Version}{2007/05/06 v1.1}
%   \item
%     Uses package \xpackage{etexcmds}.
%   \end{Version}
%   \begin{Version}{2007/11/11 v1.2}
%   \item
%     Use of package \xpackage{pdftexcmds} for \LuaTeX\ support.
%   \end{Version}
%   \begin{Version}{2016/05/16 v1.3}
%   \item
%     Documentation updates.
%   \end{Version}
% \end{History}
%
% \clearpage
% \PrintIndex
%
% \Finale
\endinput

%        (quote the arguments according to the demands of your shell)
%
% Documentation:
%    (a) If stackrel.drv is present:
%           latex stackrel.drv
%    (b) Without stackrel.drv:
%           latex stackrel.dtx; ...
%    The class ltxdoc loads the configuration file ltxdoc.cfg
%    if available. Here you can specify further options, e.g.
%    use A4 as paper format:
%       \PassOptionsToClass{a4paper}{article}
%
%    Programm calls to get the documentation (example):
%       pdflatex stackrel.dtx
%       makeindex -s gind.ist stackrel.idx
%       pdflatex stackrel.dtx
%       makeindex -s gind.ist stackrel.idx
%       pdflatex stackrel.dtx
%
% Installation:
%    TDS:tex/latex/oberdiek/stackrel.sty
%    TDS:doc/latex/oberdiek/stackrel.pdf
%    TDS:source/latex/oberdiek/stackrel.dtx
%
%<*ignore>
\begingroup
  \catcode123=1 %
  \catcode125=2 %
  \def\x{LaTeX2e}%
\expandafter\endgroup
\ifcase 0\ifx\install y1\fi\expandafter
         \ifx\csname processbatchFile\endcsname\relax\else1\fi
         \ifx\fmtname\x\else 1\fi\relax
\else\csname fi\endcsname
%</ignore>
%<*install>
\input docstrip.tex
\Msg{************************************************************************}
\Msg{* Installation}
\Msg{* Package: stackrel 2016/05/16 v1.3 Adding subscript option to stackrel (HO)}
\Msg{************************************************************************}

\keepsilent
\askforoverwritefalse

\let\MetaPrefix\relax
\preamble

This is a generated file.

Project: stackrel
Version: 2016/05/16 v1.3

Copyright (C) 2006, 2007 by
   Heiko Oberdiek <heiko.oberdiek at googlemail.com>

This work may be distributed and/or modified under the
conditions of the LaTeX Project Public License, either
version 1.3c of this license or (at your option) any later
version. This version of this license is in
   http://www.latex-project.org/lppl/lppl-1-3c.txt
and the latest version of this license is in
   http://www.latex-project.org/lppl.txt
and version 1.3 or later is part of all distributions of
LaTeX version 2005/12/01 or later.

This work has the LPPL maintenance status "maintained".

This Current Maintainer of this work is Heiko Oberdiek.

This work consists of the main source file stackrel.dtx
and the derived files
   stackrel.sty, stackrel.pdf, stackrel.ins, stackrel.drv.

\endpreamble
\let\MetaPrefix\DoubleperCent

\generate{%
  \file{stackrel.ins}{\from{stackrel.dtx}{install}}%
  \file{stackrel.drv}{\from{stackrel.dtx}{driver}}%
  \usedir{tex/latex/oberdiek}%
  \file{stackrel.sty}{\from{stackrel.dtx}{package}}%
  \nopreamble
  \nopostamble
%  \usedir{source/latex/oberdiek/catalogue}%
%  \file{stackrel.xml}{\from{stackrel.dtx}{catalogue}}%
}

\catcode32=13\relax% active space
\let =\space%
\Msg{************************************************************************}
\Msg{*}
\Msg{* To finish the installation you have to move the following}
\Msg{* file into a directory searched by TeX:}
\Msg{*}
\Msg{*     stackrel.sty}
\Msg{*}
\Msg{* To produce the documentation run the file `stackrel.drv'}
\Msg{* through LaTeX.}
\Msg{*}
\Msg{* Happy TeXing!}
\Msg{*}
\Msg{************************************************************************}

\endbatchfile
%</install>
%<*ignore>
\fi
%</ignore>
%<*driver>
\NeedsTeXFormat{LaTeX2e}
\ProvidesFile{stackrel.drv}%
  [2016/05/16 v1.3 Adding subscript option to stackrel (HO)]%
\documentclass{ltxdoc}
\usepackage{amsmath}
\usepackage{holtxdoc}[2011/11/22]
\usepackage{stackrel}[2016/05/16]
\begin{document}
  \DocInput{stackrel.dtx}%
\end{document}
%</driver>
% \fi
%
%
% \CharacterTable
%  {Upper-case    \A\B\C\D\E\F\G\H\I\J\K\L\M\N\O\P\Q\R\S\T\U\V\W\X\Y\Z
%   Lower-case    \a\b\c\d\e\f\g\h\i\j\k\l\m\n\o\p\q\r\s\t\u\v\w\x\y\z
%   Digits        \0\1\2\3\4\5\6\7\8\9
%   Exclamation   \!     Double quote  \"     Hash (number) \#
%   Dollar        \$     Percent       \%     Ampersand     \&
%   Acute accent  \'     Left paren    \(     Right paren   \)
%   Asterisk      \*     Plus          \+     Comma         \,
%   Minus         \-     Point         \.     Solidus       \/
%   Colon         \:     Semicolon     \;     Less than     \<
%   Equals        \=     Greater than  \>     Question mark \?
%   Commercial at \@     Left bracket  \[     Backslash     \\
%   Right bracket \]     Circumflex    \^     Underscore    \_
%   Grave accent  \`     Left brace    \{     Vertical bar  \|
%   Right brace   \}     Tilde         \~}
%
% \GetFileInfo{stackrel.drv}
%
% \title{The \xpackage{stackrel} package}
% \date{2016/05/16 v1.3}
% \author{Heiko Oberdiek\thanks
% {Please report any issues at \url{https://github.com/ho-tex/oberdiek/issues}}\\
% \xemail{heiko.oberdiek at googlemail.com}}
%
% \maketitle
%
% \begin{abstract}
% This package adds an optional argument to \cs{stackrel} for
% putting something below the relational symbol and defines
% \cs{stackbin} for binary symbols.
% \end{abstract}
%
% \tableofcontents
%
% \section{User interface}
%
% \LaTeX's \cs{stackrel} allows a superscript above a relational symbol,
% but pure \LaTeX\ does not provide a macro for putting a subscript
% below the symbol. This is supported by \AmS\LaTeX's \cs{underset}
% macro that works on both relational and binary symbols. A combination
% of \cs{underset} and \cs{overset} can be used to put \mbox{sub-} and
% superscripts to the same symbol.
%
% This package \xpackage{stackrel} extends the syntax of \cs{stackrel}
% by adding an optional argument for the subscript position.
% It follows the syntax of extensible arrows of packages
% \xpackage{amsmath} and \xpackage{mathtools}.
%
% \begin{declcs}{stackrel}
%   |[|\meta{subscript}|]| \M{superscript} \M{rel}\\
%   \cs{stackbin}
%   |[|\meta{subscript}|]| \M{superscript} \M{bin}
% \end{declcs}
% Example:
% \begin{quote}
% |A \stackbin[\text{and}]{}{+} B \stackrel[x]{!}{=} C|\\
% $A \stackbin[\text{and}]{}{+} B \stackrel[x]{!}{=} C$
% \end{quote}
%
% \StopEventually{
% }
%
% \section{Implementation}
%
%    \begin{macrocode}
%<*package>
\NeedsTeXFormat{LaTeX2e}
\ProvidesPackage{stackrel}
  [2016/05/16 v1.3 Adding subscript option to stackrel (HO)]%
%    \end{macrocode}
%
%    Given the original definition of \cs{stackrel} the addition
%    of the optional argument is straightforward. If an argument
%    is empty, then the corresponding sub- or superscript is
%    suppressed.
%
%    Depending on the available resources (\eTeX, \pdfTeX)
%    three methods are given for testing emptyness. All tests
%    allow the hash to be used inside the arguments without
%    doubling (for the unlikely case that someone wants to
%    define macros with arguments).
%    \begin{macro}{\stack@relbin}
%    \begin{macrocode}
\RequirePackage{etexcmds}[2007/09/09]
\ifetex@unexpanded
  \RequirePackage{pdftexcmds}[2016/05/16]%
  \begingroup\expandafter\expandafter\expandafter\endgroup
  \expandafter\ifx\csname pdf@strcmp\endcsname\relax
    \newcommand*{\stack@relbin}[3][]{%
      \mathop{#3}\limits
      \edef\reserved@a{\etex@unexpanded{#1}}%
      \ifx\reserved@a\@empty\else_{#1}\fi
      \edef\reserved@a{\etex@unexpanded{#2}}%
      \ifx\reserved@a\@empty\else^{#2}\fi
      \egroup
    }%
  \else
    \newcommand*{\stack@relbin}[3][]{%
      \mathop{#3}\limits
      \ifcase\pdf@strcmp{\detokenize{#1}}{}\else_{#1}\fi
      \ifcase\pdf@strcmp{\detokenize{#2}}{}\else^{#2}\fi
      \egroup
    }%
  \fi
\else
  \newcommand*{\stack@relbin}[3][]{%
    \mathop{#3}\limits
    \toks@{#1}%
    \edef\reserved@a{\the\toks@}%
    \ifx\reserved@a\@empty\else_{#1}\fi
    \toks@{#2}%
    \edef\reserved@a{\the\toks@}%
    \ifx\reserved@a\@empty\else^{#2}\fi
    \egroup
  }%
\fi
%    \end{macrocode}
%    \end{macro}
%    \begin{macro}{\stackrel}
%    \begin{macrocode}
\renewcommand*{\stackrel}{%
  \mathrel\bgroup\stack@relbin
}
%    \end{macrocode}
%    \end{macro}
%    \begin{macro}{\stackbin}
%    \begin{macrocode}
\newcommand*{\stackbin}{%
  \mathbin\bgroup\stack@relbin
}
%    \end{macrocode}
%    \end{macro}
%
%    \begin{macrocode}
%</package>
%    \end{macrocode}
%
% \section{Installation}
%
% \subsection{Download}
%
% \paragraph{Package.} This package is available on
% CTAN\footnote{\CTANpkg{stackrel}}:
% \begin{description}
% \item[\CTAN{macros/latex/contrib/oberdiek/stackrel.dtx}] The source file.
% \item[\CTAN{macros/latex/contrib/oberdiek/stackrel.pdf}] Documentation.
% \end{description}
%
%
% \paragraph{Bundle.} All the packages of the bundle `oberdiek'
% are also available in a TDS compliant ZIP archive. There
% the packages are already unpacked and the documentation files
% are generated. The files and directories obey the TDS standard.
% \begin{description}
% \item[\CTANinstall{install/macros/latex/contrib/oberdiek.tds.zip}]
% \end{description}
% \emph{TDS} refers to the standard ``A Directory Structure
% for \TeX\ Files'' (\CTAN{tds/tds.pdf}). Directories
% with \xfile{texmf} in their name are usually organized this way.
%
% \subsection{Bundle installation}
%
% \paragraph{Unpacking.} Unpack the \xfile{oberdiek.tds.zip} in the
% TDS tree (also known as \xfile{texmf} tree) of your choice.
% Example (linux):
% \begin{quote}
%   |unzip oberdiek.tds.zip -d ~/texmf|
% \end{quote}
%
% \paragraph{Script installation.}
% Check the directory \xfile{TDS:scripts/oberdiek/} for
% scripts that need further installation steps.
% Package \xpackage{attachfile2} comes with the Perl script
% \xfile{pdfatfi.pl} that should be installed in such a way
% that it can be called as \texttt{pdfatfi}.
% Example (linux):
% \begin{quote}
%   |chmod +x scripts/oberdiek/pdfatfi.pl|\\
%   |cp scripts/oberdiek/pdfatfi.pl /usr/local/bin/|
% \end{quote}
%
% \subsection{Package installation}
%
% \paragraph{Unpacking.} The \xfile{.dtx} file is a self-extracting
% \docstrip\ archive. The files are extracted by running the
% \xfile{.dtx} through \plainTeX:
% \begin{quote}
%   \verb|tex stackrel.dtx|
% \end{quote}
%
% \paragraph{TDS.} Now the different files must be moved into
% the different directories in your installation TDS tree
% (also known as \xfile{texmf} tree):
% \begin{quote}
% \def\t{^^A
% \begin{tabular}{@{}>{\ttfamily}l@{ $\rightarrow$ }>{\ttfamily}l@{}}
%   stackrel.sty & tex/latex/oberdiek/stackrel.sty\\
%   stackrel.pdf & doc/latex/oberdiek/stackrel.pdf\\
%   stackrel.dtx & source/latex/oberdiek/stackrel.dtx\\
% \end{tabular}^^A
% }^^A
% \sbox0{\t}^^A
% \ifdim\wd0>\linewidth
%   \begingroup
%     \advance\linewidth by\leftmargin
%     \advance\linewidth by\rightmargin
%   \edef\x{\endgroup
%     \def\noexpand\lw{\the\linewidth}^^A
%   }\x
%   \def\lwbox{^^A
%     \leavevmode
%     \hbox to \linewidth{^^A
%       \kern-\leftmargin\relax
%       \hss
%       \usebox0
%       \hss
%       \kern-\rightmargin\relax
%     }^^A
%   }^^A
%   \ifdim\wd0>\lw
%     \sbox0{\small\t}^^A
%     \ifdim\wd0>\linewidth
%       \ifdim\wd0>\lw
%         \sbox0{\footnotesize\t}^^A
%         \ifdim\wd0>\linewidth
%           \ifdim\wd0>\lw
%             \sbox0{\scriptsize\t}^^A
%             \ifdim\wd0>\linewidth
%               \ifdim\wd0>\lw
%                 \sbox0{\tiny\t}^^A
%                 \ifdim\wd0>\linewidth
%                   \lwbox
%                 \else
%                   \usebox0
%                 \fi
%               \else
%                 \lwbox
%               \fi
%             \else
%               \usebox0
%             \fi
%           \else
%             \lwbox
%           \fi
%         \else
%           \usebox0
%         \fi
%       \else
%         \lwbox
%       \fi
%     \else
%       \usebox0
%     \fi
%   \else
%     \lwbox
%   \fi
% \else
%   \usebox0
% \fi
% \end{quote}
% If you have a \xfile{docstrip.cfg} that configures and enables \docstrip's
% TDS installing feature, then some files can already be in the right
% place, see the documentation of \docstrip.
%
% \subsection{Refresh file name databases}
%
% If your \TeX~distribution
% (\teTeX, \mikTeX, \dots) relies on file name databases, you must refresh
% these. For example, \teTeX\ users run \verb|texhash| or
% \verb|mktexlsr|.
%
% \subsection{Some details for the interested}
%
% \paragraph{Attached source.}
%
% The PDF documentation on CTAN also includes the
% \xfile{.dtx} source file. It can be extracted by
% AcrobatReader 6 or higher. Another option is \textsf{pdftk},
% e.g. unpack the file into the current directory:
% \begin{quote}
%   \verb|pdftk stackrel.pdf unpack_files output .|
% \end{quote}
%
% \paragraph{Unpacking with \LaTeX.}
% The \xfile{.dtx} chooses its action depending on the format:
% \begin{description}
% \item[\plainTeX:] Run \docstrip\ and extract the files.
% \item[\LaTeX:] Generate the documentation.
% \end{description}
% If you insist on using \LaTeX\ for \docstrip\ (really,
% \docstrip\ does not need \LaTeX), then inform the autodetect routine
% about your intention:
% \begin{quote}
%   \verb|latex \let\install=y% \iffalse meta-comment
%
% File: stackrel.dtx
% Version: 2016/05/16 v1.3
% Info: Adding subscript option to stackrel
%
% Copyright (C) 2006, 2007 by
%    Heiko Oberdiek <heiko.oberdiek at googlemail.com>
%    2016
%    https://github.com/ho-tex/oberdiek/issues
%
% This work may be distributed and/or modified under the
% conditions of the LaTeX Project Public License, either
% version 1.3c of this license or (at your option) any later
% version. This version of this license is in
%    http://www.latex-project.org/lppl/lppl-1-3c.txt
% and the latest version of this license is in
%    http://www.latex-project.org/lppl.txt
% and version 1.3 or later is part of all distributions of
% LaTeX version 2005/12/01 or later.
%
% This work has the LPPL maintenance status "maintained".
%
% This Current Maintainer of this work is Heiko Oberdiek.
%
% This work consists of the main source file stackrel.dtx
% and the derived files
%    stackrel.sty, stackrel.pdf, stackrel.ins, stackrel.drv.
%
% Distribution:
%    CTAN:macros/latex/contrib/oberdiek/stackrel.dtx
%    CTAN:macros/latex/contrib/oberdiek/stackrel.pdf
%
% Unpacking:
%    (a) If stackrel.ins is present:
%           tex stackrel.ins
%    (b) Without stackrel.ins:
%           tex stackrel.dtx
%    (c) If you insist on using LaTeX
%           latex \let\install=y\input{stackrel.dtx}
%        (quote the arguments according to the demands of your shell)
%
% Documentation:
%    (a) If stackrel.drv is present:
%           latex stackrel.drv
%    (b) Without stackrel.drv:
%           latex stackrel.dtx; ...
%    The class ltxdoc loads the configuration file ltxdoc.cfg
%    if available. Here you can specify further options, e.g.
%    use A4 as paper format:
%       \PassOptionsToClass{a4paper}{article}
%
%    Programm calls to get the documentation (example):
%       pdflatex stackrel.dtx
%       makeindex -s gind.ist stackrel.idx
%       pdflatex stackrel.dtx
%       makeindex -s gind.ist stackrel.idx
%       pdflatex stackrel.dtx
%
% Installation:
%    TDS:tex/latex/oberdiek/stackrel.sty
%    TDS:doc/latex/oberdiek/stackrel.pdf
%    TDS:source/latex/oberdiek/stackrel.dtx
%
%<*ignore>
\begingroup
  \catcode123=1 %
  \catcode125=2 %
  \def\x{LaTeX2e}%
\expandafter\endgroup
\ifcase 0\ifx\install y1\fi\expandafter
         \ifx\csname processbatchFile\endcsname\relax\else1\fi
         \ifx\fmtname\x\else 1\fi\relax
\else\csname fi\endcsname
%</ignore>
%<*install>
\input docstrip.tex
\Msg{************************************************************************}
\Msg{* Installation}
\Msg{* Package: stackrel 2016/05/16 v1.3 Adding subscript option to stackrel (HO)}
\Msg{************************************************************************}

\keepsilent
\askforoverwritefalse

\let\MetaPrefix\relax
\preamble

This is a generated file.

Project: stackrel
Version: 2016/05/16 v1.3

Copyright (C) 2006, 2007 by
   Heiko Oberdiek <heiko.oberdiek at googlemail.com>

This work may be distributed and/or modified under the
conditions of the LaTeX Project Public License, either
version 1.3c of this license or (at your option) any later
version. This version of this license is in
   http://www.latex-project.org/lppl/lppl-1-3c.txt
and the latest version of this license is in
   http://www.latex-project.org/lppl.txt
and version 1.3 or later is part of all distributions of
LaTeX version 2005/12/01 or later.

This work has the LPPL maintenance status "maintained".

This Current Maintainer of this work is Heiko Oberdiek.

This work consists of the main source file stackrel.dtx
and the derived files
   stackrel.sty, stackrel.pdf, stackrel.ins, stackrel.drv.

\endpreamble
\let\MetaPrefix\DoubleperCent

\generate{%
  \file{stackrel.ins}{\from{stackrel.dtx}{install}}%
  \file{stackrel.drv}{\from{stackrel.dtx}{driver}}%
  \usedir{tex/latex/oberdiek}%
  \file{stackrel.sty}{\from{stackrel.dtx}{package}}%
  \nopreamble
  \nopostamble
%  \usedir{source/latex/oberdiek/catalogue}%
%  \file{stackrel.xml}{\from{stackrel.dtx}{catalogue}}%
}

\catcode32=13\relax% active space
\let =\space%
\Msg{************************************************************************}
\Msg{*}
\Msg{* To finish the installation you have to move the following}
\Msg{* file into a directory searched by TeX:}
\Msg{*}
\Msg{*     stackrel.sty}
\Msg{*}
\Msg{* To produce the documentation run the file `stackrel.drv'}
\Msg{* through LaTeX.}
\Msg{*}
\Msg{* Happy TeXing!}
\Msg{*}
\Msg{************************************************************************}

\endbatchfile
%</install>
%<*ignore>
\fi
%</ignore>
%<*driver>
\NeedsTeXFormat{LaTeX2e}
\ProvidesFile{stackrel.drv}%
  [2016/05/16 v1.3 Adding subscript option to stackrel (HO)]%
\documentclass{ltxdoc}
\usepackage{amsmath}
\usepackage{holtxdoc}[2011/11/22]
\usepackage{stackrel}[2016/05/16]
\begin{document}
  \DocInput{stackrel.dtx}%
\end{document}
%</driver>
% \fi
%
%
% \CharacterTable
%  {Upper-case    \A\B\C\D\E\F\G\H\I\J\K\L\M\N\O\P\Q\R\S\T\U\V\W\X\Y\Z
%   Lower-case    \a\b\c\d\e\f\g\h\i\j\k\l\m\n\o\p\q\r\s\t\u\v\w\x\y\z
%   Digits        \0\1\2\3\4\5\6\7\8\9
%   Exclamation   \!     Double quote  \"     Hash (number) \#
%   Dollar        \$     Percent       \%     Ampersand     \&
%   Acute accent  \'     Left paren    \(     Right paren   \)
%   Asterisk      \*     Plus          \+     Comma         \,
%   Minus         \-     Point         \.     Solidus       \/
%   Colon         \:     Semicolon     \;     Less than     \<
%   Equals        \=     Greater than  \>     Question mark \?
%   Commercial at \@     Left bracket  \[     Backslash     \\
%   Right bracket \]     Circumflex    \^     Underscore    \_
%   Grave accent  \`     Left brace    \{     Vertical bar  \|
%   Right brace   \}     Tilde         \~}
%
% \GetFileInfo{stackrel.drv}
%
% \title{The \xpackage{stackrel} package}
% \date{2016/05/16 v1.3}
% \author{Heiko Oberdiek\thanks
% {Please report any issues at \url{https://github.com/ho-tex/oberdiek/issues}}\\
% \xemail{heiko.oberdiek at googlemail.com}}
%
% \maketitle
%
% \begin{abstract}
% This package adds an optional argument to \cs{stackrel} for
% putting something below the relational symbol and defines
% \cs{stackbin} for binary symbols.
% \end{abstract}
%
% \tableofcontents
%
% \section{User interface}
%
% \LaTeX's \cs{stackrel} allows a superscript above a relational symbol,
% but pure \LaTeX\ does not provide a macro for putting a subscript
% below the symbol. This is supported by \AmS\LaTeX's \cs{underset}
% macro that works on both relational and binary symbols. A combination
% of \cs{underset} and \cs{overset} can be used to put \mbox{sub-} and
% superscripts to the same symbol.
%
% This package \xpackage{stackrel} extends the syntax of \cs{stackrel}
% by adding an optional argument for the subscript position.
% It follows the syntax of extensible arrows of packages
% \xpackage{amsmath} and \xpackage{mathtools}.
%
% \begin{declcs}{stackrel}
%   |[|\meta{subscript}|]| \M{superscript} \M{rel}\\
%   \cs{stackbin}
%   |[|\meta{subscript}|]| \M{superscript} \M{bin}
% \end{declcs}
% Example:
% \begin{quote}
% |A \stackbin[\text{and}]{}{+} B \stackrel[x]{!}{=} C|\\
% $A \stackbin[\text{and}]{}{+} B \stackrel[x]{!}{=} C$
% \end{quote}
%
% \StopEventually{
% }
%
% \section{Implementation}
%
%    \begin{macrocode}
%<*package>
\NeedsTeXFormat{LaTeX2e}
\ProvidesPackage{stackrel}
  [2016/05/16 v1.3 Adding subscript option to stackrel (HO)]%
%    \end{macrocode}
%
%    Given the original definition of \cs{stackrel} the addition
%    of the optional argument is straightforward. If an argument
%    is empty, then the corresponding sub- or superscript is
%    suppressed.
%
%    Depending on the available resources (\eTeX, \pdfTeX)
%    three methods are given for testing emptyness. All tests
%    allow the hash to be used inside the arguments without
%    doubling (for the unlikely case that someone wants to
%    define macros with arguments).
%    \begin{macro}{\stack@relbin}
%    \begin{macrocode}
\RequirePackage{etexcmds}[2007/09/09]
\ifetex@unexpanded
  \RequirePackage{pdftexcmds}[2016/05/16]%
  \begingroup\expandafter\expandafter\expandafter\endgroup
  \expandafter\ifx\csname pdf@strcmp\endcsname\relax
    \newcommand*{\stack@relbin}[3][]{%
      \mathop{#3}\limits
      \edef\reserved@a{\etex@unexpanded{#1}}%
      \ifx\reserved@a\@empty\else_{#1}\fi
      \edef\reserved@a{\etex@unexpanded{#2}}%
      \ifx\reserved@a\@empty\else^{#2}\fi
      \egroup
    }%
  \else
    \newcommand*{\stack@relbin}[3][]{%
      \mathop{#3}\limits
      \ifcase\pdf@strcmp{\detokenize{#1}}{}\else_{#1}\fi
      \ifcase\pdf@strcmp{\detokenize{#2}}{}\else^{#2}\fi
      \egroup
    }%
  \fi
\else
  \newcommand*{\stack@relbin}[3][]{%
    \mathop{#3}\limits
    \toks@{#1}%
    \edef\reserved@a{\the\toks@}%
    \ifx\reserved@a\@empty\else_{#1}\fi
    \toks@{#2}%
    \edef\reserved@a{\the\toks@}%
    \ifx\reserved@a\@empty\else^{#2}\fi
    \egroup
  }%
\fi
%    \end{macrocode}
%    \end{macro}
%    \begin{macro}{\stackrel}
%    \begin{macrocode}
\renewcommand*{\stackrel}{%
  \mathrel\bgroup\stack@relbin
}
%    \end{macrocode}
%    \end{macro}
%    \begin{macro}{\stackbin}
%    \begin{macrocode}
\newcommand*{\stackbin}{%
  \mathbin\bgroup\stack@relbin
}
%    \end{macrocode}
%    \end{macro}
%
%    \begin{macrocode}
%</package>
%    \end{macrocode}
%
% \section{Installation}
%
% \subsection{Download}
%
% \paragraph{Package.} This package is available on
% CTAN\footnote{\CTANpkg{stackrel}}:
% \begin{description}
% \item[\CTAN{macros/latex/contrib/oberdiek/stackrel.dtx}] The source file.
% \item[\CTAN{macros/latex/contrib/oberdiek/stackrel.pdf}] Documentation.
% \end{description}
%
%
% \paragraph{Bundle.} All the packages of the bundle `oberdiek'
% are also available in a TDS compliant ZIP archive. There
% the packages are already unpacked and the documentation files
% are generated. The files and directories obey the TDS standard.
% \begin{description}
% \item[\CTANinstall{install/macros/latex/contrib/oberdiek.tds.zip}]
% \end{description}
% \emph{TDS} refers to the standard ``A Directory Structure
% for \TeX\ Files'' (\CTAN{tds/tds.pdf}). Directories
% with \xfile{texmf} in their name are usually organized this way.
%
% \subsection{Bundle installation}
%
% \paragraph{Unpacking.} Unpack the \xfile{oberdiek.tds.zip} in the
% TDS tree (also known as \xfile{texmf} tree) of your choice.
% Example (linux):
% \begin{quote}
%   |unzip oberdiek.tds.zip -d ~/texmf|
% \end{quote}
%
% \paragraph{Script installation.}
% Check the directory \xfile{TDS:scripts/oberdiek/} for
% scripts that need further installation steps.
% Package \xpackage{attachfile2} comes with the Perl script
% \xfile{pdfatfi.pl} that should be installed in such a way
% that it can be called as \texttt{pdfatfi}.
% Example (linux):
% \begin{quote}
%   |chmod +x scripts/oberdiek/pdfatfi.pl|\\
%   |cp scripts/oberdiek/pdfatfi.pl /usr/local/bin/|
% \end{quote}
%
% \subsection{Package installation}
%
% \paragraph{Unpacking.} The \xfile{.dtx} file is a self-extracting
% \docstrip\ archive. The files are extracted by running the
% \xfile{.dtx} through \plainTeX:
% \begin{quote}
%   \verb|tex stackrel.dtx|
% \end{quote}
%
% \paragraph{TDS.} Now the different files must be moved into
% the different directories in your installation TDS tree
% (also known as \xfile{texmf} tree):
% \begin{quote}
% \def\t{^^A
% \begin{tabular}{@{}>{\ttfamily}l@{ $\rightarrow$ }>{\ttfamily}l@{}}
%   stackrel.sty & tex/latex/oberdiek/stackrel.sty\\
%   stackrel.pdf & doc/latex/oberdiek/stackrel.pdf\\
%   stackrel.dtx & source/latex/oberdiek/stackrel.dtx\\
% \end{tabular}^^A
% }^^A
% \sbox0{\t}^^A
% \ifdim\wd0>\linewidth
%   \begingroup
%     \advance\linewidth by\leftmargin
%     \advance\linewidth by\rightmargin
%   \edef\x{\endgroup
%     \def\noexpand\lw{\the\linewidth}^^A
%   }\x
%   \def\lwbox{^^A
%     \leavevmode
%     \hbox to \linewidth{^^A
%       \kern-\leftmargin\relax
%       \hss
%       \usebox0
%       \hss
%       \kern-\rightmargin\relax
%     }^^A
%   }^^A
%   \ifdim\wd0>\lw
%     \sbox0{\small\t}^^A
%     \ifdim\wd0>\linewidth
%       \ifdim\wd0>\lw
%         \sbox0{\footnotesize\t}^^A
%         \ifdim\wd0>\linewidth
%           \ifdim\wd0>\lw
%             \sbox0{\scriptsize\t}^^A
%             \ifdim\wd0>\linewidth
%               \ifdim\wd0>\lw
%                 \sbox0{\tiny\t}^^A
%                 \ifdim\wd0>\linewidth
%                   \lwbox
%                 \else
%                   \usebox0
%                 \fi
%               \else
%                 \lwbox
%               \fi
%             \else
%               \usebox0
%             \fi
%           \else
%             \lwbox
%           \fi
%         \else
%           \usebox0
%         \fi
%       \else
%         \lwbox
%       \fi
%     \else
%       \usebox0
%     \fi
%   \else
%     \lwbox
%   \fi
% \else
%   \usebox0
% \fi
% \end{quote}
% If you have a \xfile{docstrip.cfg} that configures and enables \docstrip's
% TDS installing feature, then some files can already be in the right
% place, see the documentation of \docstrip.
%
% \subsection{Refresh file name databases}
%
% If your \TeX~distribution
% (\teTeX, \mikTeX, \dots) relies on file name databases, you must refresh
% these. For example, \teTeX\ users run \verb|texhash| or
% \verb|mktexlsr|.
%
% \subsection{Some details for the interested}
%
% \paragraph{Attached source.}
%
% The PDF documentation on CTAN also includes the
% \xfile{.dtx} source file. It can be extracted by
% AcrobatReader 6 or higher. Another option is \textsf{pdftk},
% e.g. unpack the file into the current directory:
% \begin{quote}
%   \verb|pdftk stackrel.pdf unpack_files output .|
% \end{quote}
%
% \paragraph{Unpacking with \LaTeX.}
% The \xfile{.dtx} chooses its action depending on the format:
% \begin{description}
% \item[\plainTeX:] Run \docstrip\ and extract the files.
% \item[\LaTeX:] Generate the documentation.
% \end{description}
% If you insist on using \LaTeX\ for \docstrip\ (really,
% \docstrip\ does not need \LaTeX), then inform the autodetect routine
% about your intention:
% \begin{quote}
%   \verb|latex \let\install=y\input{stackrel.dtx}|
% \end{quote}
% Do not forget to quote the argument according to the demands
% of your shell.
%
% \paragraph{Generating the documentation.}
% You can use both the \xfile{.dtx} or the \xfile{.drv} to generate
% the documentation. The process can be configured by the
% configuration file \xfile{ltxdoc.cfg}. For instance, put this
% line into this file, if you want to have A4 as paper format:
% \begin{quote}
%   \verb|\PassOptionsToClass{a4paper}{article}|
% \end{quote}
% An example follows how to generate the
% documentation with pdf\LaTeX:
% \begin{quote}
%\begin{verbatim}
%pdflatex stackrel.dtx
%makeindex -s gind.ist stackrel.idx
%pdflatex stackrel.dtx
%makeindex -s gind.ist stackrel.idx
%pdflatex stackrel.dtx
%\end{verbatim}
% \end{quote}
%
% \begin{History}
%   \begin{Version}{2006/12/02 v1.0}
%   \item
%     First version.
%   \end{Version}
%   \begin{Version}{2007/05/06 v1.1}
%   \item
%     Uses package \xpackage{etexcmds}.
%   \end{Version}
%   \begin{Version}{2007/11/11 v1.2}
%   \item
%     Use of package \xpackage{pdftexcmds} for \LuaTeX\ support.
%   \end{Version}
%   \begin{Version}{2016/05/16 v1.3}
%   \item
%     Documentation updates.
%   \end{Version}
% \end{History}
%
% \clearpage
% \PrintIndex
%
% \Finale
\endinput
|
% \end{quote}
% Do not forget to quote the argument according to the demands
% of your shell.
%
% \paragraph{Generating the documentation.}
% You can use both the \xfile{.dtx} or the \xfile{.drv} to generate
% the documentation. The process can be configured by the
% configuration file \xfile{ltxdoc.cfg}. For instance, put this
% line into this file, if you want to have A4 as paper format:
% \begin{quote}
%   \verb|\PassOptionsToClass{a4paper}{article}|
% \end{quote}
% An example follows how to generate the
% documentation with pdf\LaTeX:
% \begin{quote}
%\begin{verbatim}
%pdflatex stackrel.dtx
%makeindex -s gind.ist stackrel.idx
%pdflatex stackrel.dtx
%makeindex -s gind.ist stackrel.idx
%pdflatex stackrel.dtx
%\end{verbatim}
% \end{quote}
%
% \begin{History}
%   \begin{Version}{2006/12/02 v1.0}
%   \item
%     First version.
%   \end{Version}
%   \begin{Version}{2007/05/06 v1.1}
%   \item
%     Uses package \xpackage{etexcmds}.
%   \end{Version}
%   \begin{Version}{2007/11/11 v1.2}
%   \item
%     Use of package \xpackage{pdftexcmds} for \LuaTeX\ support.
%   \end{Version}
%   \begin{Version}{2016/05/16 v1.3}
%   \item
%     Documentation updates.
%   \end{Version}
% \end{History}
%
% \clearpage
% \PrintIndex
%
% \Finale
\endinput
|
% \end{quote}
% Do not forget to quote the argument according to the demands
% of your shell.
%
% \paragraph{Generating the documentation.}
% You can use both the \xfile{.dtx} or the \xfile{.drv} to generate
% the documentation. The process can be configured by the
% configuration file \xfile{ltxdoc.cfg}. For instance, put this
% line into this file, if you want to have A4 as paper format:
% \begin{quote}
%   \verb|\PassOptionsToClass{a4paper}{article}|
% \end{quote}
% An example follows how to generate the
% documentation with pdf\LaTeX:
% \begin{quote}
%\begin{verbatim}
%pdflatex stackrel.dtx
%makeindex -s gind.ist stackrel.idx
%pdflatex stackrel.dtx
%makeindex -s gind.ist stackrel.idx
%pdflatex stackrel.dtx
%\end{verbatim}
% \end{quote}
%
% \begin{History}
%   \begin{Version}{2006/12/02 v1.0}
%   \item
%     First version.
%   \end{Version}
%   \begin{Version}{2007/05/06 v1.1}
%   \item
%     Uses package \xpackage{etexcmds}.
%   \end{Version}
%   \begin{Version}{2007/11/11 v1.2}
%   \item
%     Use of package \xpackage{pdftexcmds} for \LuaTeX\ support.
%   \end{Version}
%   \begin{Version}{2016/05/16 v1.3}
%   \item
%     Documentation updates.
%   \end{Version}
% \end{History}
%
% \clearpage
% \PrintIndex
%
% \Finale
\endinput

%        (quote the arguments according to the demands of your shell)
%
% Documentation:
%    (a) If stackrel.drv is present:
%           latex stackrel.drv
%    (b) Without stackrel.drv:
%           latex stackrel.dtx; ...
%    The class ltxdoc loads the configuration file ltxdoc.cfg
%    if available. Here you can specify further options, e.g.
%    use A4 as paper format:
%       \PassOptionsToClass{a4paper}{article}
%
%    Programm calls to get the documentation (example):
%       pdflatex stackrel.dtx
%       makeindex -s gind.ist stackrel.idx
%       pdflatex stackrel.dtx
%       makeindex -s gind.ist stackrel.idx
%       pdflatex stackrel.dtx
%
% Installation:
%    TDS:tex/latex/oberdiek/stackrel.sty
%    TDS:doc/latex/oberdiek/stackrel.pdf
%    TDS:source/latex/oberdiek/stackrel.dtx
%
%<*ignore>
\begingroup
  \catcode123=1 %
  \catcode125=2 %
  \def\x{LaTeX2e}%
\expandafter\endgroup
\ifcase 0\ifx\install y1\fi\expandafter
         \ifx\csname processbatchFile\endcsname\relax\else1\fi
         \ifx\fmtname\x\else 1\fi\relax
\else\csname fi\endcsname
%</ignore>
%<*install>
\input docstrip.tex
\Msg{************************************************************************}
\Msg{* Installation}
\Msg{* Package: stackrel 2016/05/16 v1.3 Adding subscript option to stackrel (HO)}
\Msg{************************************************************************}

\keepsilent
\askforoverwritefalse

\let\MetaPrefix\relax
\preamble

This is a generated file.

Project: stackrel
Version: 2016/05/16 v1.3

Copyright (C) 2006, 2007 by
   Heiko Oberdiek <heiko.oberdiek at googlemail.com>

This work may be distributed and/or modified under the
conditions of the LaTeX Project Public License, either
version 1.3c of this license or (at your option) any later
version. This version of this license is in
   https://www.latex-project.org/lppl/lppl-1-3c.txt
and the latest version of this license is in
   https://www.latex-project.org/lppl.txt
and version 1.3 or later is part of all distributions of
LaTeX version 2005/12/01 or later.

This work has the LPPL maintenance status "maintained".

The Current Maintainers of this work are
Heiko Oberdiek and the Oberdiek Package Support Group
https://github.com/ho-tex/oberdiek/issues


This work consists of the main source file stackrel.dtx
and the derived files
   stackrel.sty, stackrel.pdf, stackrel.ins, stackrel.drv.

\endpreamble
\let\MetaPrefix\DoubleperCent

\generate{%
  \file{stackrel.ins}{\from{stackrel.dtx}{install}}%
  \file{stackrel.drv}{\from{stackrel.dtx}{driver}}%
  \usedir{tex/latex/oberdiek}%
  \file{stackrel.sty}{\from{stackrel.dtx}{package}}%
  \nopreamble
  \nopostamble
%  \usedir{source/latex/oberdiek/catalogue}%
%  \file{stackrel.xml}{\from{stackrel.dtx}{catalogue}}%
}

\catcode32=13\relax% active space
\let =\space%
\Msg{************************************************************************}
\Msg{*}
\Msg{* To finish the installation you have to move the following}
\Msg{* file into a directory searched by TeX:}
\Msg{*}
\Msg{*     stackrel.sty}
\Msg{*}
\Msg{* To produce the documentation run the file `stackrel.drv'}
\Msg{* through LaTeX.}
\Msg{*}
\Msg{* Happy TeXing!}
\Msg{*}
\Msg{************************************************************************}

\endbatchfile
%</install>
%<*ignore>
\fi
%</ignore>
%<*driver>
\NeedsTeXFormat{LaTeX2e}
\ProvidesFile{stackrel.drv}%
  [2016/05/16 v1.3 Adding subscript option to stackrel (HO)]%
\documentclass{ltxdoc}
\usepackage{amsmath}
\usepackage{holtxdoc}[2011/11/22]
\usepackage{stackrel}[2016/05/16]
\begin{document}
  \DocInput{stackrel.dtx}%
\end{document}
%</driver>
% \fi
%
%
% \CharacterTable
%  {Upper-case    \A\B\C\D\E\F\G\H\I\J\K\L\M\N\O\P\Q\R\S\T\U\V\W\X\Y\Z
%   Lower-case    \a\b\c\d\e\f\g\h\i\j\k\l\m\n\o\p\q\r\s\t\u\v\w\x\y\z
%   Digits        \0\1\2\3\4\5\6\7\8\9
%   Exclamation   \!     Double quote  \"     Hash (number) \#
%   Dollar        \$     Percent       \%     Ampersand     \&
%   Acute accent  \'     Left paren    \(     Right paren   \)
%   Asterisk      \*     Plus          \+     Comma         \,
%   Minus         \-     Point         \.     Solidus       \/
%   Colon         \:     Semicolon     \;     Less than     \<
%   Equals        \=     Greater than  \>     Question mark \?
%   Commercial at \@     Left bracket  \[     Backslash     \\
%   Right bracket \]     Circumflex    \^     Underscore    \_
%   Grave accent  \`     Left brace    \{     Vertical bar  \|
%   Right brace   \}     Tilde         \~}
%
% \GetFileInfo{stackrel.drv}
%
% \title{The \xpackage{stackrel} package}
% \date{2016/05/16 v1.3}
% \author{Heiko Oberdiek\thanks
% {Please report any issues at \url{https://github.com/ho-tex/oberdiek/issues}}}
%
% \maketitle
%
% \begin{abstract}
% This package adds an optional argument to \cs{stackrel} for
% putting something below the relational symbol and defines
% \cs{stackbin} for binary symbols.
% \end{abstract}
%
% \tableofcontents
%
% \section{User interface}
%
% \LaTeX's \cs{stackrel} allows a superscript above a relational symbol,
% but pure \LaTeX\ does not provide a macro for putting a subscript
% below the symbol. This is supported by \AmS\LaTeX's \cs{underset}
% macro that works on both relational and binary symbols. A combination
% of \cs{underset} and \cs{overset} can be used to put \mbox{sub-} and
% superscripts to the same symbol.
%
% This package \xpackage{stackrel} extends the syntax of \cs{stackrel}
% by adding an optional argument for the subscript position.
% It follows the syntax of extensible arrows of packages
% \xpackage{amsmath} and \xpackage{mathtools}.
%
% \begin{declcs}{stackrel}
%   |[|\meta{subscript}|]| \M{superscript} \M{rel}\\
%   \cs{stackbin}
%   |[|\meta{subscript}|]| \M{superscript} \M{bin}
% \end{declcs}
% Example:
% \begin{quote}
% |A \stackbin[\text{and}]{}{+} B \stackrel[x]{!}{=} C|\\
% $A \stackbin[\text{and}]{}{+} B \stackrel[x]{!}{=} C$
% \end{quote}
%
% \StopEventually{
% }
%
% \section{Implementation}
%
%    \begin{macrocode}
%<*package>
\NeedsTeXFormat{LaTeX2e}
\ProvidesPackage{stackrel}
  [2016/05/16 v1.3 Adding subscript option to stackrel (HO)]%
%    \end{macrocode}
%
%    Given the original definition of \cs{stackrel} the addition
%    of the optional argument is straightforward. If an argument
%    is empty, then the corresponding sub- or superscript is
%    suppressed.
%
%    Depending on the available resources (\eTeX, \pdfTeX)
%    three methods are given for testing emptyness. All tests
%    allow the hash to be used inside the arguments without
%    doubling (for the unlikely case that someone wants to
%    define macros with arguments).
%    \begin{macro}{\stack@relbin}
%    \begin{macrocode}
\RequirePackage{etexcmds}[2007/09/09]
\ifetex@unexpanded
  \RequirePackage{pdftexcmds}[2016/05/16]%
  \begingroup\expandafter\expandafter\expandafter\endgroup
  \expandafter\ifx\csname pdf@strcmp\endcsname\relax
    \newcommand*{\stack@relbin}[3][]{%
      \mathop{#3}\limits
      \edef\reserved@a{\etex@unexpanded{#1}}%
      \ifx\reserved@a\@empty\else_{#1}\fi
      \edef\reserved@a{\etex@unexpanded{#2}}%
      \ifx\reserved@a\@empty\else^{#2}\fi
      \egroup
    }%
  \else
    \newcommand*{\stack@relbin}[3][]{%
      \mathop{#3}\limits
      \ifcase\pdf@strcmp{\detokenize{#1}}{}\else_{#1}\fi
      \ifcase\pdf@strcmp{\detokenize{#2}}{}\else^{#2}\fi
      \egroup
    }%
  \fi
\else
  \newcommand*{\stack@relbin}[3][]{%
    \mathop{#3}\limits
    \toks@{#1}%
    \edef\reserved@a{\the\toks@}%
    \ifx\reserved@a\@empty\else_{#1}\fi
    \toks@{#2}%
    \edef\reserved@a{\the\toks@}%
    \ifx\reserved@a\@empty\else^{#2}\fi
    \egroup
  }%
\fi
%    \end{macrocode}
%    \end{macro}
%    \begin{macro}{\stackrel}
%    \begin{macrocode}
\renewcommand*{\stackrel}{%
  \mathrel\bgroup\stack@relbin
}
%    \end{macrocode}
%    \end{macro}
%    \begin{macro}{\stackbin}
%    \begin{macrocode}
\newcommand*{\stackbin}{%
  \mathbin\bgroup\stack@relbin
}
%    \end{macrocode}
%    \end{macro}
%
%    \begin{macrocode}
%</package>
%    \end{macrocode}
%
% \section{Installation}
%
% \subsection{Download}
%
% \paragraph{Package.} This package is available on
% CTAN\footnote{\CTANpkg{stackrel}}:
% \begin{description}
% \item[\CTAN{macros/latex/contrib/oberdiek/stackrel.dtx}] The source file.
% \item[\CTAN{macros/latex/contrib/oberdiek/stackrel.pdf}] Documentation.
% \end{description}
%
%
% \paragraph{Bundle.} All the packages of the bundle `oberdiek'
% are also available in a TDS compliant ZIP archive. There
% the packages are already unpacked and the documentation files
% are generated. The files and directories obey the TDS standard.
% \begin{description}
% \item[\CTANinstall{install/macros/latex/contrib/oberdiek.tds.zip}]
% \end{description}
% \emph{TDS} refers to the standard ``A Directory Structure
% for \TeX\ Files'' (\CTAN{tds/tds.pdf}). Directories
% with \xfile{texmf} in their name are usually organized this way.
%
% \subsection{Bundle installation}
%
% \paragraph{Unpacking.} Unpack the \xfile{oberdiek.tds.zip} in the
% TDS tree (also known as \xfile{texmf} tree) of your choice.
% Example (linux):
% \begin{quote}
%   |unzip oberdiek.tds.zip -d ~/texmf|
% \end{quote}
%
% \paragraph{Script installation.}
% Check the directory \xfile{TDS:scripts/oberdiek/} for
% scripts that need further installation steps.
% Package \xpackage{attachfile2} comes with the Perl script
% \xfile{pdfatfi.pl} that should be installed in such a way
% that it can be called as \texttt{pdfatfi}.
% Example (linux):
% \begin{quote}
%   |chmod +x scripts/oberdiek/pdfatfi.pl|\\
%   |cp scripts/oberdiek/pdfatfi.pl /usr/local/bin/|
% \end{quote}
%
% \subsection{Package installation}
%
% \paragraph{Unpacking.} The \xfile{.dtx} file is a self-extracting
% \docstrip\ archive. The files are extracted by running the
% \xfile{.dtx} through \plainTeX:
% \begin{quote}
%   \verb|tex stackrel.dtx|
% \end{quote}
%
% \paragraph{TDS.} Now the different files must be moved into
% the different directories in your installation TDS tree
% (also known as \xfile{texmf} tree):
% \begin{quote}
% \def\t{^^A
% \begin{tabular}{@{}>{\ttfamily}l@{ $\rightarrow$ }>{\ttfamily}l@{}}
%   stackrel.sty & tex/latex/oberdiek/stackrel.sty\\
%   stackrel.pdf & doc/latex/oberdiek/stackrel.pdf\\
%   stackrel.dtx & source/latex/oberdiek/stackrel.dtx\\
% \end{tabular}^^A
% }^^A
% \sbox0{\t}^^A
% \ifdim\wd0>\linewidth
%   \begingroup
%     \advance\linewidth by\leftmargin
%     \advance\linewidth by\rightmargin
%   \edef\x{\endgroup
%     \def\noexpand\lw{\the\linewidth}^^A
%   }\x
%   \def\lwbox{^^A
%     \leavevmode
%     \hbox to \linewidth{^^A
%       \kern-\leftmargin\relax
%       \hss
%       \usebox0
%       \hss
%       \kern-\rightmargin\relax
%     }^^A
%   }^^A
%   \ifdim\wd0>\lw
%     \sbox0{\small\t}^^A
%     \ifdim\wd0>\linewidth
%       \ifdim\wd0>\lw
%         \sbox0{\footnotesize\t}^^A
%         \ifdim\wd0>\linewidth
%           \ifdim\wd0>\lw
%             \sbox0{\scriptsize\t}^^A
%             \ifdim\wd0>\linewidth
%               \ifdim\wd0>\lw
%                 \sbox0{\tiny\t}^^A
%                 \ifdim\wd0>\linewidth
%                   \lwbox
%                 \else
%                   \usebox0
%                 \fi
%               \else
%                 \lwbox
%               \fi
%             \else
%               \usebox0
%             \fi
%           \else
%             \lwbox
%           \fi
%         \else
%           \usebox0
%         \fi
%       \else
%         \lwbox
%       \fi
%     \else
%       \usebox0
%     \fi
%   \else
%     \lwbox
%   \fi
% \else
%   \usebox0
% \fi
% \end{quote}
% If you have a \xfile{docstrip.cfg} that configures and enables \docstrip's
% TDS installing feature, then some files can already be in the right
% place, see the documentation of \docstrip.
%
% \subsection{Refresh file name databases}
%
% If your \TeX~distribution
% (\teTeX, \mikTeX, \dots) relies on file name databases, you must refresh
% these. For example, \teTeX\ users run \verb|texhash| or
% \verb|mktexlsr|.
%
% \subsection{Some details for the interested}
%
% \paragraph{Attached source.}
%
% The PDF documentation on CTAN also includes the
% \xfile{.dtx} source file. It can be extracted by
% AcrobatReader 6 or higher. Another option is \textsf{pdftk},
% e.g. unpack the file into the current directory:
% \begin{quote}
%   \verb|pdftk stackrel.pdf unpack_files output .|
% \end{quote}
%
% \paragraph{Unpacking with \LaTeX.}
% The \xfile{.dtx} chooses its action depending on the format:
% \begin{description}
% \item[\plainTeX:] Run \docstrip\ and extract the files.
% \item[\LaTeX:] Generate the documentation.
% \end{description}
% If you insist on using \LaTeX\ for \docstrip\ (really,
% \docstrip\ does not need \LaTeX), then inform the autodetect routine
% about your intention:
% \begin{quote}
%   \verb|latex \let\install=y% \iffalse meta-comment
%
% File: stackrel.dtx
% Version: 2016/05/16 v1.3
% Info: Adding subscript option to stackrel
%
% Copyright (C) 2006, 2007 by
%    Heiko Oberdiek <heiko.oberdiek at googlemail.com>
%    2016
%    https://github.com/ho-tex/oberdiek/issues
%
% This work may be distributed and/or modified under the
% conditions of the LaTeX Project Public License, either
% version 1.3c of this license or (at your option) any later
% version. This version of this license is in
%    http://www.latex-project.org/lppl/lppl-1-3c.txt
% and the latest version of this license is in
%    http://www.latex-project.org/lppl.txt
% and version 1.3 or later is part of all distributions of
% LaTeX version 2005/12/01 or later.
%
% This work has the LPPL maintenance status "maintained".
%
% This Current Maintainer of this work is Heiko Oberdiek.
%
% This work consists of the main source file stackrel.dtx
% and the derived files
%    stackrel.sty, stackrel.pdf, stackrel.ins, stackrel.drv.
%
% Distribution:
%    CTAN:macros/latex/contrib/oberdiek/stackrel.dtx
%    CTAN:macros/latex/contrib/oberdiek/stackrel.pdf
%
% Unpacking:
%    (a) If stackrel.ins is present:
%           tex stackrel.ins
%    (b) Without stackrel.ins:
%           tex stackrel.dtx
%    (c) If you insist on using LaTeX
%           latex \let\install=y% \iffalse meta-comment
%
% File: stackrel.dtx
% Version: 2016/05/16 v1.3
% Info: Adding subscript option to stackrel
%
% Copyright (C) 2006, 2007 by
%    Heiko Oberdiek <heiko.oberdiek at googlemail.com>
%    2016
%    https://github.com/ho-tex/oberdiek/issues
%
% This work may be distributed and/or modified under the
% conditions of the LaTeX Project Public License, either
% version 1.3c of this license or (at your option) any later
% version. This version of this license is in
%    http://www.latex-project.org/lppl/lppl-1-3c.txt
% and the latest version of this license is in
%    http://www.latex-project.org/lppl.txt
% and version 1.3 or later is part of all distributions of
% LaTeX version 2005/12/01 or later.
%
% This work has the LPPL maintenance status "maintained".
%
% This Current Maintainer of this work is Heiko Oberdiek.
%
% This work consists of the main source file stackrel.dtx
% and the derived files
%    stackrel.sty, stackrel.pdf, stackrel.ins, stackrel.drv.
%
% Distribution:
%    CTAN:macros/latex/contrib/oberdiek/stackrel.dtx
%    CTAN:macros/latex/contrib/oberdiek/stackrel.pdf
%
% Unpacking:
%    (a) If stackrel.ins is present:
%           tex stackrel.ins
%    (b) Without stackrel.ins:
%           tex stackrel.dtx
%    (c) If you insist on using LaTeX
%           latex \let\install=y% \iffalse meta-comment
%
% File: stackrel.dtx
% Version: 2016/05/16 v1.3
% Info: Adding subscript option to stackrel
%
% Copyright (C) 2006, 2007 by
%    Heiko Oberdiek <heiko.oberdiek at googlemail.com>
%    2016
%    https://github.com/ho-tex/oberdiek/issues
%
% This work may be distributed and/or modified under the
% conditions of the LaTeX Project Public License, either
% version 1.3c of this license or (at your option) any later
% version. This version of this license is in
%    http://www.latex-project.org/lppl/lppl-1-3c.txt
% and the latest version of this license is in
%    http://www.latex-project.org/lppl.txt
% and version 1.3 or later is part of all distributions of
% LaTeX version 2005/12/01 or later.
%
% This work has the LPPL maintenance status "maintained".
%
% This Current Maintainer of this work is Heiko Oberdiek.
%
% This work consists of the main source file stackrel.dtx
% and the derived files
%    stackrel.sty, stackrel.pdf, stackrel.ins, stackrel.drv.
%
% Distribution:
%    CTAN:macros/latex/contrib/oberdiek/stackrel.dtx
%    CTAN:macros/latex/contrib/oberdiek/stackrel.pdf
%
% Unpacking:
%    (a) If stackrel.ins is present:
%           tex stackrel.ins
%    (b) Without stackrel.ins:
%           tex stackrel.dtx
%    (c) If you insist on using LaTeX
%           latex \let\install=y\input{stackrel.dtx}
%        (quote the arguments according to the demands of your shell)
%
% Documentation:
%    (a) If stackrel.drv is present:
%           latex stackrel.drv
%    (b) Without stackrel.drv:
%           latex stackrel.dtx; ...
%    The class ltxdoc loads the configuration file ltxdoc.cfg
%    if available. Here you can specify further options, e.g.
%    use A4 as paper format:
%       \PassOptionsToClass{a4paper}{article}
%
%    Programm calls to get the documentation (example):
%       pdflatex stackrel.dtx
%       makeindex -s gind.ist stackrel.idx
%       pdflatex stackrel.dtx
%       makeindex -s gind.ist stackrel.idx
%       pdflatex stackrel.dtx
%
% Installation:
%    TDS:tex/latex/oberdiek/stackrel.sty
%    TDS:doc/latex/oberdiek/stackrel.pdf
%    TDS:source/latex/oberdiek/stackrel.dtx
%
%<*ignore>
\begingroup
  \catcode123=1 %
  \catcode125=2 %
  \def\x{LaTeX2e}%
\expandafter\endgroup
\ifcase 0\ifx\install y1\fi\expandafter
         \ifx\csname processbatchFile\endcsname\relax\else1\fi
         \ifx\fmtname\x\else 1\fi\relax
\else\csname fi\endcsname
%</ignore>
%<*install>
\input docstrip.tex
\Msg{************************************************************************}
\Msg{* Installation}
\Msg{* Package: stackrel 2016/05/16 v1.3 Adding subscript option to stackrel (HO)}
\Msg{************************************************************************}

\keepsilent
\askforoverwritefalse

\let\MetaPrefix\relax
\preamble

This is a generated file.

Project: stackrel
Version: 2016/05/16 v1.3

Copyright (C) 2006, 2007 by
   Heiko Oberdiek <heiko.oberdiek at googlemail.com>

This work may be distributed and/or modified under the
conditions of the LaTeX Project Public License, either
version 1.3c of this license or (at your option) any later
version. This version of this license is in
   http://www.latex-project.org/lppl/lppl-1-3c.txt
and the latest version of this license is in
   http://www.latex-project.org/lppl.txt
and version 1.3 or later is part of all distributions of
LaTeX version 2005/12/01 or later.

This work has the LPPL maintenance status "maintained".

This Current Maintainer of this work is Heiko Oberdiek.

This work consists of the main source file stackrel.dtx
and the derived files
   stackrel.sty, stackrel.pdf, stackrel.ins, stackrel.drv.

\endpreamble
\let\MetaPrefix\DoubleperCent

\generate{%
  \file{stackrel.ins}{\from{stackrel.dtx}{install}}%
  \file{stackrel.drv}{\from{stackrel.dtx}{driver}}%
  \usedir{tex/latex/oberdiek}%
  \file{stackrel.sty}{\from{stackrel.dtx}{package}}%
  \nopreamble
  \nopostamble
%  \usedir{source/latex/oberdiek/catalogue}%
%  \file{stackrel.xml}{\from{stackrel.dtx}{catalogue}}%
}

\catcode32=13\relax% active space
\let =\space%
\Msg{************************************************************************}
\Msg{*}
\Msg{* To finish the installation you have to move the following}
\Msg{* file into a directory searched by TeX:}
\Msg{*}
\Msg{*     stackrel.sty}
\Msg{*}
\Msg{* To produce the documentation run the file `stackrel.drv'}
\Msg{* through LaTeX.}
\Msg{*}
\Msg{* Happy TeXing!}
\Msg{*}
\Msg{************************************************************************}

\endbatchfile
%</install>
%<*ignore>
\fi
%</ignore>
%<*driver>
\NeedsTeXFormat{LaTeX2e}
\ProvidesFile{stackrel.drv}%
  [2016/05/16 v1.3 Adding subscript option to stackrel (HO)]%
\documentclass{ltxdoc}
\usepackage{amsmath}
\usepackage{holtxdoc}[2011/11/22]
\usepackage{stackrel}[2016/05/16]
\begin{document}
  \DocInput{stackrel.dtx}%
\end{document}
%</driver>
% \fi
%
%
% \CharacterTable
%  {Upper-case    \A\B\C\D\E\F\G\H\I\J\K\L\M\N\O\P\Q\R\S\T\U\V\W\X\Y\Z
%   Lower-case    \a\b\c\d\e\f\g\h\i\j\k\l\m\n\o\p\q\r\s\t\u\v\w\x\y\z
%   Digits        \0\1\2\3\4\5\6\7\8\9
%   Exclamation   \!     Double quote  \"     Hash (number) \#
%   Dollar        \$     Percent       \%     Ampersand     \&
%   Acute accent  \'     Left paren    \(     Right paren   \)
%   Asterisk      \*     Plus          \+     Comma         \,
%   Minus         \-     Point         \.     Solidus       \/
%   Colon         \:     Semicolon     \;     Less than     \<
%   Equals        \=     Greater than  \>     Question mark \?
%   Commercial at \@     Left bracket  \[     Backslash     \\
%   Right bracket \]     Circumflex    \^     Underscore    \_
%   Grave accent  \`     Left brace    \{     Vertical bar  \|
%   Right brace   \}     Tilde         \~}
%
% \GetFileInfo{stackrel.drv}
%
% \title{The \xpackage{stackrel} package}
% \date{2016/05/16 v1.3}
% \author{Heiko Oberdiek\thanks
% {Please report any issues at \url{https://github.com/ho-tex/oberdiek/issues}}\\
% \xemail{heiko.oberdiek at googlemail.com}}
%
% \maketitle
%
% \begin{abstract}
% This package adds an optional argument to \cs{stackrel} for
% putting something below the relational symbol and defines
% \cs{stackbin} for binary symbols.
% \end{abstract}
%
% \tableofcontents
%
% \section{User interface}
%
% \LaTeX's \cs{stackrel} allows a superscript above a relational symbol,
% but pure \LaTeX\ does not provide a macro for putting a subscript
% below the symbol. This is supported by \AmS\LaTeX's \cs{underset}
% macro that works on both relational and binary symbols. A combination
% of \cs{underset} and \cs{overset} can be used to put \mbox{sub-} and
% superscripts to the same symbol.
%
% This package \xpackage{stackrel} extends the syntax of \cs{stackrel}
% by adding an optional argument for the subscript position.
% It follows the syntax of extensible arrows of packages
% \xpackage{amsmath} and \xpackage{mathtools}.
%
% \begin{declcs}{stackrel}
%   |[|\meta{subscript}|]| \M{superscript} \M{rel}\\
%   \cs{stackbin}
%   |[|\meta{subscript}|]| \M{superscript} \M{bin}
% \end{declcs}
% Example:
% \begin{quote}
% |A \stackbin[\text{and}]{}{+} B \stackrel[x]{!}{=} C|\\
% $A \stackbin[\text{and}]{}{+} B \stackrel[x]{!}{=} C$
% \end{quote}
%
% \StopEventually{
% }
%
% \section{Implementation}
%
%    \begin{macrocode}
%<*package>
\NeedsTeXFormat{LaTeX2e}
\ProvidesPackage{stackrel}
  [2016/05/16 v1.3 Adding subscript option to stackrel (HO)]%
%    \end{macrocode}
%
%    Given the original definition of \cs{stackrel} the addition
%    of the optional argument is straightforward. If an argument
%    is empty, then the corresponding sub- or superscript is
%    suppressed.
%
%    Depending on the available resources (\eTeX, \pdfTeX)
%    three methods are given for testing emptyness. All tests
%    allow the hash to be used inside the arguments without
%    doubling (for the unlikely case that someone wants to
%    define macros with arguments).
%    \begin{macro}{\stack@relbin}
%    \begin{macrocode}
\RequirePackage{etexcmds}[2007/09/09]
\ifetex@unexpanded
  \RequirePackage{pdftexcmds}[2016/05/16]%
  \begingroup\expandafter\expandafter\expandafter\endgroup
  \expandafter\ifx\csname pdf@strcmp\endcsname\relax
    \newcommand*{\stack@relbin}[3][]{%
      \mathop{#3}\limits
      \edef\reserved@a{\etex@unexpanded{#1}}%
      \ifx\reserved@a\@empty\else_{#1}\fi
      \edef\reserved@a{\etex@unexpanded{#2}}%
      \ifx\reserved@a\@empty\else^{#2}\fi
      \egroup
    }%
  \else
    \newcommand*{\stack@relbin}[3][]{%
      \mathop{#3}\limits
      \ifcase\pdf@strcmp{\detokenize{#1}}{}\else_{#1}\fi
      \ifcase\pdf@strcmp{\detokenize{#2}}{}\else^{#2}\fi
      \egroup
    }%
  \fi
\else
  \newcommand*{\stack@relbin}[3][]{%
    \mathop{#3}\limits
    \toks@{#1}%
    \edef\reserved@a{\the\toks@}%
    \ifx\reserved@a\@empty\else_{#1}\fi
    \toks@{#2}%
    \edef\reserved@a{\the\toks@}%
    \ifx\reserved@a\@empty\else^{#2}\fi
    \egroup
  }%
\fi
%    \end{macrocode}
%    \end{macro}
%    \begin{macro}{\stackrel}
%    \begin{macrocode}
\renewcommand*{\stackrel}{%
  \mathrel\bgroup\stack@relbin
}
%    \end{macrocode}
%    \end{macro}
%    \begin{macro}{\stackbin}
%    \begin{macrocode}
\newcommand*{\stackbin}{%
  \mathbin\bgroup\stack@relbin
}
%    \end{macrocode}
%    \end{macro}
%
%    \begin{macrocode}
%</package>
%    \end{macrocode}
%
% \section{Installation}
%
% \subsection{Download}
%
% \paragraph{Package.} This package is available on
% CTAN\footnote{\CTANpkg{stackrel}}:
% \begin{description}
% \item[\CTAN{macros/latex/contrib/oberdiek/stackrel.dtx}] The source file.
% \item[\CTAN{macros/latex/contrib/oberdiek/stackrel.pdf}] Documentation.
% \end{description}
%
%
% \paragraph{Bundle.} All the packages of the bundle `oberdiek'
% are also available in a TDS compliant ZIP archive. There
% the packages are already unpacked and the documentation files
% are generated. The files and directories obey the TDS standard.
% \begin{description}
% \item[\CTANinstall{install/macros/latex/contrib/oberdiek.tds.zip}]
% \end{description}
% \emph{TDS} refers to the standard ``A Directory Structure
% for \TeX\ Files'' (\CTAN{tds/tds.pdf}). Directories
% with \xfile{texmf} in their name are usually organized this way.
%
% \subsection{Bundle installation}
%
% \paragraph{Unpacking.} Unpack the \xfile{oberdiek.tds.zip} in the
% TDS tree (also known as \xfile{texmf} tree) of your choice.
% Example (linux):
% \begin{quote}
%   |unzip oberdiek.tds.zip -d ~/texmf|
% \end{quote}
%
% \paragraph{Script installation.}
% Check the directory \xfile{TDS:scripts/oberdiek/} for
% scripts that need further installation steps.
% Package \xpackage{attachfile2} comes with the Perl script
% \xfile{pdfatfi.pl} that should be installed in such a way
% that it can be called as \texttt{pdfatfi}.
% Example (linux):
% \begin{quote}
%   |chmod +x scripts/oberdiek/pdfatfi.pl|\\
%   |cp scripts/oberdiek/pdfatfi.pl /usr/local/bin/|
% \end{quote}
%
% \subsection{Package installation}
%
% \paragraph{Unpacking.} The \xfile{.dtx} file is a self-extracting
% \docstrip\ archive. The files are extracted by running the
% \xfile{.dtx} through \plainTeX:
% \begin{quote}
%   \verb|tex stackrel.dtx|
% \end{quote}
%
% \paragraph{TDS.} Now the different files must be moved into
% the different directories in your installation TDS tree
% (also known as \xfile{texmf} tree):
% \begin{quote}
% \def\t{^^A
% \begin{tabular}{@{}>{\ttfamily}l@{ $\rightarrow$ }>{\ttfamily}l@{}}
%   stackrel.sty & tex/latex/oberdiek/stackrel.sty\\
%   stackrel.pdf & doc/latex/oberdiek/stackrel.pdf\\
%   stackrel.dtx & source/latex/oberdiek/stackrel.dtx\\
% \end{tabular}^^A
% }^^A
% \sbox0{\t}^^A
% \ifdim\wd0>\linewidth
%   \begingroup
%     \advance\linewidth by\leftmargin
%     \advance\linewidth by\rightmargin
%   \edef\x{\endgroup
%     \def\noexpand\lw{\the\linewidth}^^A
%   }\x
%   \def\lwbox{^^A
%     \leavevmode
%     \hbox to \linewidth{^^A
%       \kern-\leftmargin\relax
%       \hss
%       \usebox0
%       \hss
%       \kern-\rightmargin\relax
%     }^^A
%   }^^A
%   \ifdim\wd0>\lw
%     \sbox0{\small\t}^^A
%     \ifdim\wd0>\linewidth
%       \ifdim\wd0>\lw
%         \sbox0{\footnotesize\t}^^A
%         \ifdim\wd0>\linewidth
%           \ifdim\wd0>\lw
%             \sbox0{\scriptsize\t}^^A
%             \ifdim\wd0>\linewidth
%               \ifdim\wd0>\lw
%                 \sbox0{\tiny\t}^^A
%                 \ifdim\wd0>\linewidth
%                   \lwbox
%                 \else
%                   \usebox0
%                 \fi
%               \else
%                 \lwbox
%               \fi
%             \else
%               \usebox0
%             \fi
%           \else
%             \lwbox
%           \fi
%         \else
%           \usebox0
%         \fi
%       \else
%         \lwbox
%       \fi
%     \else
%       \usebox0
%     \fi
%   \else
%     \lwbox
%   \fi
% \else
%   \usebox0
% \fi
% \end{quote}
% If you have a \xfile{docstrip.cfg} that configures and enables \docstrip's
% TDS installing feature, then some files can already be in the right
% place, see the documentation of \docstrip.
%
% \subsection{Refresh file name databases}
%
% If your \TeX~distribution
% (\teTeX, \mikTeX, \dots) relies on file name databases, you must refresh
% these. For example, \teTeX\ users run \verb|texhash| or
% \verb|mktexlsr|.
%
% \subsection{Some details for the interested}
%
% \paragraph{Attached source.}
%
% The PDF documentation on CTAN also includes the
% \xfile{.dtx} source file. It can be extracted by
% AcrobatReader 6 or higher. Another option is \textsf{pdftk},
% e.g. unpack the file into the current directory:
% \begin{quote}
%   \verb|pdftk stackrel.pdf unpack_files output .|
% \end{quote}
%
% \paragraph{Unpacking with \LaTeX.}
% The \xfile{.dtx} chooses its action depending on the format:
% \begin{description}
% \item[\plainTeX:] Run \docstrip\ and extract the files.
% \item[\LaTeX:] Generate the documentation.
% \end{description}
% If you insist on using \LaTeX\ for \docstrip\ (really,
% \docstrip\ does not need \LaTeX), then inform the autodetect routine
% about your intention:
% \begin{quote}
%   \verb|latex \let\install=y\input{stackrel.dtx}|
% \end{quote}
% Do not forget to quote the argument according to the demands
% of your shell.
%
% \paragraph{Generating the documentation.}
% You can use both the \xfile{.dtx} or the \xfile{.drv} to generate
% the documentation. The process can be configured by the
% configuration file \xfile{ltxdoc.cfg}. For instance, put this
% line into this file, if you want to have A4 as paper format:
% \begin{quote}
%   \verb|\PassOptionsToClass{a4paper}{article}|
% \end{quote}
% An example follows how to generate the
% documentation with pdf\LaTeX:
% \begin{quote}
%\begin{verbatim}
%pdflatex stackrel.dtx
%makeindex -s gind.ist stackrel.idx
%pdflatex stackrel.dtx
%makeindex -s gind.ist stackrel.idx
%pdflatex stackrel.dtx
%\end{verbatim}
% \end{quote}
%
% \begin{History}
%   \begin{Version}{2006/12/02 v1.0}
%   \item
%     First version.
%   \end{Version}
%   \begin{Version}{2007/05/06 v1.1}
%   \item
%     Uses package \xpackage{etexcmds}.
%   \end{Version}
%   \begin{Version}{2007/11/11 v1.2}
%   \item
%     Use of package \xpackage{pdftexcmds} for \LuaTeX\ support.
%   \end{Version}
%   \begin{Version}{2016/05/16 v1.3}
%   \item
%     Documentation updates.
%   \end{Version}
% \end{History}
%
% \clearpage
% \PrintIndex
%
% \Finale
\endinput

%        (quote the arguments according to the demands of your shell)
%
% Documentation:
%    (a) If stackrel.drv is present:
%           latex stackrel.drv
%    (b) Without stackrel.drv:
%           latex stackrel.dtx; ...
%    The class ltxdoc loads the configuration file ltxdoc.cfg
%    if available. Here you can specify further options, e.g.
%    use A4 as paper format:
%       \PassOptionsToClass{a4paper}{article}
%
%    Programm calls to get the documentation (example):
%       pdflatex stackrel.dtx
%       makeindex -s gind.ist stackrel.idx
%       pdflatex stackrel.dtx
%       makeindex -s gind.ist stackrel.idx
%       pdflatex stackrel.dtx
%
% Installation:
%    TDS:tex/latex/oberdiek/stackrel.sty
%    TDS:doc/latex/oberdiek/stackrel.pdf
%    TDS:source/latex/oberdiek/stackrel.dtx
%
%<*ignore>
\begingroup
  \catcode123=1 %
  \catcode125=2 %
  \def\x{LaTeX2e}%
\expandafter\endgroup
\ifcase 0\ifx\install y1\fi\expandafter
         \ifx\csname processbatchFile\endcsname\relax\else1\fi
         \ifx\fmtname\x\else 1\fi\relax
\else\csname fi\endcsname
%</ignore>
%<*install>
\input docstrip.tex
\Msg{************************************************************************}
\Msg{* Installation}
\Msg{* Package: stackrel 2016/05/16 v1.3 Adding subscript option to stackrel (HO)}
\Msg{************************************************************************}

\keepsilent
\askforoverwritefalse

\let\MetaPrefix\relax
\preamble

This is a generated file.

Project: stackrel
Version: 2016/05/16 v1.3

Copyright (C) 2006, 2007 by
   Heiko Oberdiek <heiko.oberdiek at googlemail.com>

This work may be distributed and/or modified under the
conditions of the LaTeX Project Public License, either
version 1.3c of this license or (at your option) any later
version. This version of this license is in
   http://www.latex-project.org/lppl/lppl-1-3c.txt
and the latest version of this license is in
   http://www.latex-project.org/lppl.txt
and version 1.3 or later is part of all distributions of
LaTeX version 2005/12/01 or later.

This work has the LPPL maintenance status "maintained".

This Current Maintainer of this work is Heiko Oberdiek.

This work consists of the main source file stackrel.dtx
and the derived files
   stackrel.sty, stackrel.pdf, stackrel.ins, stackrel.drv.

\endpreamble
\let\MetaPrefix\DoubleperCent

\generate{%
  \file{stackrel.ins}{\from{stackrel.dtx}{install}}%
  \file{stackrel.drv}{\from{stackrel.dtx}{driver}}%
  \usedir{tex/latex/oberdiek}%
  \file{stackrel.sty}{\from{stackrel.dtx}{package}}%
  \nopreamble
  \nopostamble
%  \usedir{source/latex/oberdiek/catalogue}%
%  \file{stackrel.xml}{\from{stackrel.dtx}{catalogue}}%
}

\catcode32=13\relax% active space
\let =\space%
\Msg{************************************************************************}
\Msg{*}
\Msg{* To finish the installation you have to move the following}
\Msg{* file into a directory searched by TeX:}
\Msg{*}
\Msg{*     stackrel.sty}
\Msg{*}
\Msg{* To produce the documentation run the file `stackrel.drv'}
\Msg{* through LaTeX.}
\Msg{*}
\Msg{* Happy TeXing!}
\Msg{*}
\Msg{************************************************************************}

\endbatchfile
%</install>
%<*ignore>
\fi
%</ignore>
%<*driver>
\NeedsTeXFormat{LaTeX2e}
\ProvidesFile{stackrel.drv}%
  [2016/05/16 v1.3 Adding subscript option to stackrel (HO)]%
\documentclass{ltxdoc}
\usepackage{amsmath}
\usepackage{holtxdoc}[2011/11/22]
\usepackage{stackrel}[2016/05/16]
\begin{document}
  \DocInput{stackrel.dtx}%
\end{document}
%</driver>
% \fi
%
%
% \CharacterTable
%  {Upper-case    \A\B\C\D\E\F\G\H\I\J\K\L\M\N\O\P\Q\R\S\T\U\V\W\X\Y\Z
%   Lower-case    \a\b\c\d\e\f\g\h\i\j\k\l\m\n\o\p\q\r\s\t\u\v\w\x\y\z
%   Digits        \0\1\2\3\4\5\6\7\8\9
%   Exclamation   \!     Double quote  \"     Hash (number) \#
%   Dollar        \$     Percent       \%     Ampersand     \&
%   Acute accent  \'     Left paren    \(     Right paren   \)
%   Asterisk      \*     Plus          \+     Comma         \,
%   Minus         \-     Point         \.     Solidus       \/
%   Colon         \:     Semicolon     \;     Less than     \<
%   Equals        \=     Greater than  \>     Question mark \?
%   Commercial at \@     Left bracket  \[     Backslash     \\
%   Right bracket \]     Circumflex    \^     Underscore    \_
%   Grave accent  \`     Left brace    \{     Vertical bar  \|
%   Right brace   \}     Tilde         \~}
%
% \GetFileInfo{stackrel.drv}
%
% \title{The \xpackage{stackrel} package}
% \date{2016/05/16 v1.3}
% \author{Heiko Oberdiek\thanks
% {Please report any issues at \url{https://github.com/ho-tex/oberdiek/issues}}\\
% \xemail{heiko.oberdiek at googlemail.com}}
%
% \maketitle
%
% \begin{abstract}
% This package adds an optional argument to \cs{stackrel} for
% putting something below the relational symbol and defines
% \cs{stackbin} for binary symbols.
% \end{abstract}
%
% \tableofcontents
%
% \section{User interface}
%
% \LaTeX's \cs{stackrel} allows a superscript above a relational symbol,
% but pure \LaTeX\ does not provide a macro for putting a subscript
% below the symbol. This is supported by \AmS\LaTeX's \cs{underset}
% macro that works on both relational and binary symbols. A combination
% of \cs{underset} and \cs{overset} can be used to put \mbox{sub-} and
% superscripts to the same symbol.
%
% This package \xpackage{stackrel} extends the syntax of \cs{stackrel}
% by adding an optional argument for the subscript position.
% It follows the syntax of extensible arrows of packages
% \xpackage{amsmath} and \xpackage{mathtools}.
%
% \begin{declcs}{stackrel}
%   |[|\meta{subscript}|]| \M{superscript} \M{rel}\\
%   \cs{stackbin}
%   |[|\meta{subscript}|]| \M{superscript} \M{bin}
% \end{declcs}
% Example:
% \begin{quote}
% |A \stackbin[\text{and}]{}{+} B \stackrel[x]{!}{=} C|\\
% $A \stackbin[\text{and}]{}{+} B \stackrel[x]{!}{=} C$
% \end{quote}
%
% \StopEventually{
% }
%
% \section{Implementation}
%
%    \begin{macrocode}
%<*package>
\NeedsTeXFormat{LaTeX2e}
\ProvidesPackage{stackrel}
  [2016/05/16 v1.3 Adding subscript option to stackrel (HO)]%
%    \end{macrocode}
%
%    Given the original definition of \cs{stackrel} the addition
%    of the optional argument is straightforward. If an argument
%    is empty, then the corresponding sub- or superscript is
%    suppressed.
%
%    Depending on the available resources (\eTeX, \pdfTeX)
%    three methods are given for testing emptyness. All tests
%    allow the hash to be used inside the arguments without
%    doubling (for the unlikely case that someone wants to
%    define macros with arguments).
%    \begin{macro}{\stack@relbin}
%    \begin{macrocode}
\RequirePackage{etexcmds}[2007/09/09]
\ifetex@unexpanded
  \RequirePackage{pdftexcmds}[2016/05/16]%
  \begingroup\expandafter\expandafter\expandafter\endgroup
  \expandafter\ifx\csname pdf@strcmp\endcsname\relax
    \newcommand*{\stack@relbin}[3][]{%
      \mathop{#3}\limits
      \edef\reserved@a{\etex@unexpanded{#1}}%
      \ifx\reserved@a\@empty\else_{#1}\fi
      \edef\reserved@a{\etex@unexpanded{#2}}%
      \ifx\reserved@a\@empty\else^{#2}\fi
      \egroup
    }%
  \else
    \newcommand*{\stack@relbin}[3][]{%
      \mathop{#3}\limits
      \ifcase\pdf@strcmp{\detokenize{#1}}{}\else_{#1}\fi
      \ifcase\pdf@strcmp{\detokenize{#2}}{}\else^{#2}\fi
      \egroup
    }%
  \fi
\else
  \newcommand*{\stack@relbin}[3][]{%
    \mathop{#3}\limits
    \toks@{#1}%
    \edef\reserved@a{\the\toks@}%
    \ifx\reserved@a\@empty\else_{#1}\fi
    \toks@{#2}%
    \edef\reserved@a{\the\toks@}%
    \ifx\reserved@a\@empty\else^{#2}\fi
    \egroup
  }%
\fi
%    \end{macrocode}
%    \end{macro}
%    \begin{macro}{\stackrel}
%    \begin{macrocode}
\renewcommand*{\stackrel}{%
  \mathrel\bgroup\stack@relbin
}
%    \end{macrocode}
%    \end{macro}
%    \begin{macro}{\stackbin}
%    \begin{macrocode}
\newcommand*{\stackbin}{%
  \mathbin\bgroup\stack@relbin
}
%    \end{macrocode}
%    \end{macro}
%
%    \begin{macrocode}
%</package>
%    \end{macrocode}
%
% \section{Installation}
%
% \subsection{Download}
%
% \paragraph{Package.} This package is available on
% CTAN\footnote{\CTANpkg{stackrel}}:
% \begin{description}
% \item[\CTAN{macros/latex/contrib/oberdiek/stackrel.dtx}] The source file.
% \item[\CTAN{macros/latex/contrib/oberdiek/stackrel.pdf}] Documentation.
% \end{description}
%
%
% \paragraph{Bundle.} All the packages of the bundle `oberdiek'
% are also available in a TDS compliant ZIP archive. There
% the packages are already unpacked and the documentation files
% are generated. The files and directories obey the TDS standard.
% \begin{description}
% \item[\CTANinstall{install/macros/latex/contrib/oberdiek.tds.zip}]
% \end{description}
% \emph{TDS} refers to the standard ``A Directory Structure
% for \TeX\ Files'' (\CTAN{tds/tds.pdf}). Directories
% with \xfile{texmf} in their name are usually organized this way.
%
% \subsection{Bundle installation}
%
% \paragraph{Unpacking.} Unpack the \xfile{oberdiek.tds.zip} in the
% TDS tree (also known as \xfile{texmf} tree) of your choice.
% Example (linux):
% \begin{quote}
%   |unzip oberdiek.tds.zip -d ~/texmf|
% \end{quote}
%
% \paragraph{Script installation.}
% Check the directory \xfile{TDS:scripts/oberdiek/} for
% scripts that need further installation steps.
% Package \xpackage{attachfile2} comes with the Perl script
% \xfile{pdfatfi.pl} that should be installed in such a way
% that it can be called as \texttt{pdfatfi}.
% Example (linux):
% \begin{quote}
%   |chmod +x scripts/oberdiek/pdfatfi.pl|\\
%   |cp scripts/oberdiek/pdfatfi.pl /usr/local/bin/|
% \end{quote}
%
% \subsection{Package installation}
%
% \paragraph{Unpacking.} The \xfile{.dtx} file is a self-extracting
% \docstrip\ archive. The files are extracted by running the
% \xfile{.dtx} through \plainTeX:
% \begin{quote}
%   \verb|tex stackrel.dtx|
% \end{quote}
%
% \paragraph{TDS.} Now the different files must be moved into
% the different directories in your installation TDS tree
% (also known as \xfile{texmf} tree):
% \begin{quote}
% \def\t{^^A
% \begin{tabular}{@{}>{\ttfamily}l@{ $\rightarrow$ }>{\ttfamily}l@{}}
%   stackrel.sty & tex/latex/oberdiek/stackrel.sty\\
%   stackrel.pdf & doc/latex/oberdiek/stackrel.pdf\\
%   stackrel.dtx & source/latex/oberdiek/stackrel.dtx\\
% \end{tabular}^^A
% }^^A
% \sbox0{\t}^^A
% \ifdim\wd0>\linewidth
%   \begingroup
%     \advance\linewidth by\leftmargin
%     \advance\linewidth by\rightmargin
%   \edef\x{\endgroup
%     \def\noexpand\lw{\the\linewidth}^^A
%   }\x
%   \def\lwbox{^^A
%     \leavevmode
%     \hbox to \linewidth{^^A
%       \kern-\leftmargin\relax
%       \hss
%       \usebox0
%       \hss
%       \kern-\rightmargin\relax
%     }^^A
%   }^^A
%   \ifdim\wd0>\lw
%     \sbox0{\small\t}^^A
%     \ifdim\wd0>\linewidth
%       \ifdim\wd0>\lw
%         \sbox0{\footnotesize\t}^^A
%         \ifdim\wd0>\linewidth
%           \ifdim\wd0>\lw
%             \sbox0{\scriptsize\t}^^A
%             \ifdim\wd0>\linewidth
%               \ifdim\wd0>\lw
%                 \sbox0{\tiny\t}^^A
%                 \ifdim\wd0>\linewidth
%                   \lwbox
%                 \else
%                   \usebox0
%                 \fi
%               \else
%                 \lwbox
%               \fi
%             \else
%               \usebox0
%             \fi
%           \else
%             \lwbox
%           \fi
%         \else
%           \usebox0
%         \fi
%       \else
%         \lwbox
%       \fi
%     \else
%       \usebox0
%     \fi
%   \else
%     \lwbox
%   \fi
% \else
%   \usebox0
% \fi
% \end{quote}
% If you have a \xfile{docstrip.cfg} that configures and enables \docstrip's
% TDS installing feature, then some files can already be in the right
% place, see the documentation of \docstrip.
%
% \subsection{Refresh file name databases}
%
% If your \TeX~distribution
% (\teTeX, \mikTeX, \dots) relies on file name databases, you must refresh
% these. For example, \teTeX\ users run \verb|texhash| or
% \verb|mktexlsr|.
%
% \subsection{Some details for the interested}
%
% \paragraph{Attached source.}
%
% The PDF documentation on CTAN also includes the
% \xfile{.dtx} source file. It can be extracted by
% AcrobatReader 6 or higher. Another option is \textsf{pdftk},
% e.g. unpack the file into the current directory:
% \begin{quote}
%   \verb|pdftk stackrel.pdf unpack_files output .|
% \end{quote}
%
% \paragraph{Unpacking with \LaTeX.}
% The \xfile{.dtx} chooses its action depending on the format:
% \begin{description}
% \item[\plainTeX:] Run \docstrip\ and extract the files.
% \item[\LaTeX:] Generate the documentation.
% \end{description}
% If you insist on using \LaTeX\ for \docstrip\ (really,
% \docstrip\ does not need \LaTeX), then inform the autodetect routine
% about your intention:
% \begin{quote}
%   \verb|latex \let\install=y% \iffalse meta-comment
%
% File: stackrel.dtx
% Version: 2016/05/16 v1.3
% Info: Adding subscript option to stackrel
%
% Copyright (C) 2006, 2007 by
%    Heiko Oberdiek <heiko.oberdiek at googlemail.com>
%    2016
%    https://github.com/ho-tex/oberdiek/issues
%
% This work may be distributed and/or modified under the
% conditions of the LaTeX Project Public License, either
% version 1.3c of this license or (at your option) any later
% version. This version of this license is in
%    http://www.latex-project.org/lppl/lppl-1-3c.txt
% and the latest version of this license is in
%    http://www.latex-project.org/lppl.txt
% and version 1.3 or later is part of all distributions of
% LaTeX version 2005/12/01 or later.
%
% This work has the LPPL maintenance status "maintained".
%
% This Current Maintainer of this work is Heiko Oberdiek.
%
% This work consists of the main source file stackrel.dtx
% and the derived files
%    stackrel.sty, stackrel.pdf, stackrel.ins, stackrel.drv.
%
% Distribution:
%    CTAN:macros/latex/contrib/oberdiek/stackrel.dtx
%    CTAN:macros/latex/contrib/oberdiek/stackrel.pdf
%
% Unpacking:
%    (a) If stackrel.ins is present:
%           tex stackrel.ins
%    (b) Without stackrel.ins:
%           tex stackrel.dtx
%    (c) If you insist on using LaTeX
%           latex \let\install=y\input{stackrel.dtx}
%        (quote the arguments according to the demands of your shell)
%
% Documentation:
%    (a) If stackrel.drv is present:
%           latex stackrel.drv
%    (b) Without stackrel.drv:
%           latex stackrel.dtx; ...
%    The class ltxdoc loads the configuration file ltxdoc.cfg
%    if available. Here you can specify further options, e.g.
%    use A4 as paper format:
%       \PassOptionsToClass{a4paper}{article}
%
%    Programm calls to get the documentation (example):
%       pdflatex stackrel.dtx
%       makeindex -s gind.ist stackrel.idx
%       pdflatex stackrel.dtx
%       makeindex -s gind.ist stackrel.idx
%       pdflatex stackrel.dtx
%
% Installation:
%    TDS:tex/latex/oberdiek/stackrel.sty
%    TDS:doc/latex/oberdiek/stackrel.pdf
%    TDS:source/latex/oberdiek/stackrel.dtx
%
%<*ignore>
\begingroup
  \catcode123=1 %
  \catcode125=2 %
  \def\x{LaTeX2e}%
\expandafter\endgroup
\ifcase 0\ifx\install y1\fi\expandafter
         \ifx\csname processbatchFile\endcsname\relax\else1\fi
         \ifx\fmtname\x\else 1\fi\relax
\else\csname fi\endcsname
%</ignore>
%<*install>
\input docstrip.tex
\Msg{************************************************************************}
\Msg{* Installation}
\Msg{* Package: stackrel 2016/05/16 v1.3 Adding subscript option to stackrel (HO)}
\Msg{************************************************************************}

\keepsilent
\askforoverwritefalse

\let\MetaPrefix\relax
\preamble

This is a generated file.

Project: stackrel
Version: 2016/05/16 v1.3

Copyright (C) 2006, 2007 by
   Heiko Oberdiek <heiko.oberdiek at googlemail.com>

This work may be distributed and/or modified under the
conditions of the LaTeX Project Public License, either
version 1.3c of this license or (at your option) any later
version. This version of this license is in
   http://www.latex-project.org/lppl/lppl-1-3c.txt
and the latest version of this license is in
   http://www.latex-project.org/lppl.txt
and version 1.3 or later is part of all distributions of
LaTeX version 2005/12/01 or later.

This work has the LPPL maintenance status "maintained".

This Current Maintainer of this work is Heiko Oberdiek.

This work consists of the main source file stackrel.dtx
and the derived files
   stackrel.sty, stackrel.pdf, stackrel.ins, stackrel.drv.

\endpreamble
\let\MetaPrefix\DoubleperCent

\generate{%
  \file{stackrel.ins}{\from{stackrel.dtx}{install}}%
  \file{stackrel.drv}{\from{stackrel.dtx}{driver}}%
  \usedir{tex/latex/oberdiek}%
  \file{stackrel.sty}{\from{stackrel.dtx}{package}}%
  \nopreamble
  \nopostamble
%  \usedir{source/latex/oberdiek/catalogue}%
%  \file{stackrel.xml}{\from{stackrel.dtx}{catalogue}}%
}

\catcode32=13\relax% active space
\let =\space%
\Msg{************************************************************************}
\Msg{*}
\Msg{* To finish the installation you have to move the following}
\Msg{* file into a directory searched by TeX:}
\Msg{*}
\Msg{*     stackrel.sty}
\Msg{*}
\Msg{* To produce the documentation run the file `stackrel.drv'}
\Msg{* through LaTeX.}
\Msg{*}
\Msg{* Happy TeXing!}
\Msg{*}
\Msg{************************************************************************}

\endbatchfile
%</install>
%<*ignore>
\fi
%</ignore>
%<*driver>
\NeedsTeXFormat{LaTeX2e}
\ProvidesFile{stackrel.drv}%
  [2016/05/16 v1.3 Adding subscript option to stackrel (HO)]%
\documentclass{ltxdoc}
\usepackage{amsmath}
\usepackage{holtxdoc}[2011/11/22]
\usepackage{stackrel}[2016/05/16]
\begin{document}
  \DocInput{stackrel.dtx}%
\end{document}
%</driver>
% \fi
%
%
% \CharacterTable
%  {Upper-case    \A\B\C\D\E\F\G\H\I\J\K\L\M\N\O\P\Q\R\S\T\U\V\W\X\Y\Z
%   Lower-case    \a\b\c\d\e\f\g\h\i\j\k\l\m\n\o\p\q\r\s\t\u\v\w\x\y\z
%   Digits        \0\1\2\3\4\5\6\7\8\9
%   Exclamation   \!     Double quote  \"     Hash (number) \#
%   Dollar        \$     Percent       \%     Ampersand     \&
%   Acute accent  \'     Left paren    \(     Right paren   \)
%   Asterisk      \*     Plus          \+     Comma         \,
%   Minus         \-     Point         \.     Solidus       \/
%   Colon         \:     Semicolon     \;     Less than     \<
%   Equals        \=     Greater than  \>     Question mark \?
%   Commercial at \@     Left bracket  \[     Backslash     \\
%   Right bracket \]     Circumflex    \^     Underscore    \_
%   Grave accent  \`     Left brace    \{     Vertical bar  \|
%   Right brace   \}     Tilde         \~}
%
% \GetFileInfo{stackrel.drv}
%
% \title{The \xpackage{stackrel} package}
% \date{2016/05/16 v1.3}
% \author{Heiko Oberdiek\thanks
% {Please report any issues at \url{https://github.com/ho-tex/oberdiek/issues}}\\
% \xemail{heiko.oberdiek at googlemail.com}}
%
% \maketitle
%
% \begin{abstract}
% This package adds an optional argument to \cs{stackrel} for
% putting something below the relational symbol and defines
% \cs{stackbin} for binary symbols.
% \end{abstract}
%
% \tableofcontents
%
% \section{User interface}
%
% \LaTeX's \cs{stackrel} allows a superscript above a relational symbol,
% but pure \LaTeX\ does not provide a macro for putting a subscript
% below the symbol. This is supported by \AmS\LaTeX's \cs{underset}
% macro that works on both relational and binary symbols. A combination
% of \cs{underset} and \cs{overset} can be used to put \mbox{sub-} and
% superscripts to the same symbol.
%
% This package \xpackage{stackrel} extends the syntax of \cs{stackrel}
% by adding an optional argument for the subscript position.
% It follows the syntax of extensible arrows of packages
% \xpackage{amsmath} and \xpackage{mathtools}.
%
% \begin{declcs}{stackrel}
%   |[|\meta{subscript}|]| \M{superscript} \M{rel}\\
%   \cs{stackbin}
%   |[|\meta{subscript}|]| \M{superscript} \M{bin}
% \end{declcs}
% Example:
% \begin{quote}
% |A \stackbin[\text{and}]{}{+} B \stackrel[x]{!}{=} C|\\
% $A \stackbin[\text{and}]{}{+} B \stackrel[x]{!}{=} C$
% \end{quote}
%
% \StopEventually{
% }
%
% \section{Implementation}
%
%    \begin{macrocode}
%<*package>
\NeedsTeXFormat{LaTeX2e}
\ProvidesPackage{stackrel}
  [2016/05/16 v1.3 Adding subscript option to stackrel (HO)]%
%    \end{macrocode}
%
%    Given the original definition of \cs{stackrel} the addition
%    of the optional argument is straightforward. If an argument
%    is empty, then the corresponding sub- or superscript is
%    suppressed.
%
%    Depending on the available resources (\eTeX, \pdfTeX)
%    three methods are given for testing emptyness. All tests
%    allow the hash to be used inside the arguments without
%    doubling (for the unlikely case that someone wants to
%    define macros with arguments).
%    \begin{macro}{\stack@relbin}
%    \begin{macrocode}
\RequirePackage{etexcmds}[2007/09/09]
\ifetex@unexpanded
  \RequirePackage{pdftexcmds}[2016/05/16]%
  \begingroup\expandafter\expandafter\expandafter\endgroup
  \expandafter\ifx\csname pdf@strcmp\endcsname\relax
    \newcommand*{\stack@relbin}[3][]{%
      \mathop{#3}\limits
      \edef\reserved@a{\etex@unexpanded{#1}}%
      \ifx\reserved@a\@empty\else_{#1}\fi
      \edef\reserved@a{\etex@unexpanded{#2}}%
      \ifx\reserved@a\@empty\else^{#2}\fi
      \egroup
    }%
  \else
    \newcommand*{\stack@relbin}[3][]{%
      \mathop{#3}\limits
      \ifcase\pdf@strcmp{\detokenize{#1}}{}\else_{#1}\fi
      \ifcase\pdf@strcmp{\detokenize{#2}}{}\else^{#2}\fi
      \egroup
    }%
  \fi
\else
  \newcommand*{\stack@relbin}[3][]{%
    \mathop{#3}\limits
    \toks@{#1}%
    \edef\reserved@a{\the\toks@}%
    \ifx\reserved@a\@empty\else_{#1}\fi
    \toks@{#2}%
    \edef\reserved@a{\the\toks@}%
    \ifx\reserved@a\@empty\else^{#2}\fi
    \egroup
  }%
\fi
%    \end{macrocode}
%    \end{macro}
%    \begin{macro}{\stackrel}
%    \begin{macrocode}
\renewcommand*{\stackrel}{%
  \mathrel\bgroup\stack@relbin
}
%    \end{macrocode}
%    \end{macro}
%    \begin{macro}{\stackbin}
%    \begin{macrocode}
\newcommand*{\stackbin}{%
  \mathbin\bgroup\stack@relbin
}
%    \end{macrocode}
%    \end{macro}
%
%    \begin{macrocode}
%</package>
%    \end{macrocode}
%
% \section{Installation}
%
% \subsection{Download}
%
% \paragraph{Package.} This package is available on
% CTAN\footnote{\CTANpkg{stackrel}}:
% \begin{description}
% \item[\CTAN{macros/latex/contrib/oberdiek/stackrel.dtx}] The source file.
% \item[\CTAN{macros/latex/contrib/oberdiek/stackrel.pdf}] Documentation.
% \end{description}
%
%
% \paragraph{Bundle.} All the packages of the bundle `oberdiek'
% are also available in a TDS compliant ZIP archive. There
% the packages are already unpacked and the documentation files
% are generated. The files and directories obey the TDS standard.
% \begin{description}
% \item[\CTANinstall{install/macros/latex/contrib/oberdiek.tds.zip}]
% \end{description}
% \emph{TDS} refers to the standard ``A Directory Structure
% for \TeX\ Files'' (\CTAN{tds/tds.pdf}). Directories
% with \xfile{texmf} in their name are usually organized this way.
%
% \subsection{Bundle installation}
%
% \paragraph{Unpacking.} Unpack the \xfile{oberdiek.tds.zip} in the
% TDS tree (also known as \xfile{texmf} tree) of your choice.
% Example (linux):
% \begin{quote}
%   |unzip oberdiek.tds.zip -d ~/texmf|
% \end{quote}
%
% \paragraph{Script installation.}
% Check the directory \xfile{TDS:scripts/oberdiek/} for
% scripts that need further installation steps.
% Package \xpackage{attachfile2} comes with the Perl script
% \xfile{pdfatfi.pl} that should be installed in such a way
% that it can be called as \texttt{pdfatfi}.
% Example (linux):
% \begin{quote}
%   |chmod +x scripts/oberdiek/pdfatfi.pl|\\
%   |cp scripts/oberdiek/pdfatfi.pl /usr/local/bin/|
% \end{quote}
%
% \subsection{Package installation}
%
% \paragraph{Unpacking.} The \xfile{.dtx} file is a self-extracting
% \docstrip\ archive. The files are extracted by running the
% \xfile{.dtx} through \plainTeX:
% \begin{quote}
%   \verb|tex stackrel.dtx|
% \end{quote}
%
% \paragraph{TDS.} Now the different files must be moved into
% the different directories in your installation TDS tree
% (also known as \xfile{texmf} tree):
% \begin{quote}
% \def\t{^^A
% \begin{tabular}{@{}>{\ttfamily}l@{ $\rightarrow$ }>{\ttfamily}l@{}}
%   stackrel.sty & tex/latex/oberdiek/stackrel.sty\\
%   stackrel.pdf & doc/latex/oberdiek/stackrel.pdf\\
%   stackrel.dtx & source/latex/oberdiek/stackrel.dtx\\
% \end{tabular}^^A
% }^^A
% \sbox0{\t}^^A
% \ifdim\wd0>\linewidth
%   \begingroup
%     \advance\linewidth by\leftmargin
%     \advance\linewidth by\rightmargin
%   \edef\x{\endgroup
%     \def\noexpand\lw{\the\linewidth}^^A
%   }\x
%   \def\lwbox{^^A
%     \leavevmode
%     \hbox to \linewidth{^^A
%       \kern-\leftmargin\relax
%       \hss
%       \usebox0
%       \hss
%       \kern-\rightmargin\relax
%     }^^A
%   }^^A
%   \ifdim\wd0>\lw
%     \sbox0{\small\t}^^A
%     \ifdim\wd0>\linewidth
%       \ifdim\wd0>\lw
%         \sbox0{\footnotesize\t}^^A
%         \ifdim\wd0>\linewidth
%           \ifdim\wd0>\lw
%             \sbox0{\scriptsize\t}^^A
%             \ifdim\wd0>\linewidth
%               \ifdim\wd0>\lw
%                 \sbox0{\tiny\t}^^A
%                 \ifdim\wd0>\linewidth
%                   \lwbox
%                 \else
%                   \usebox0
%                 \fi
%               \else
%                 \lwbox
%               \fi
%             \else
%               \usebox0
%             \fi
%           \else
%             \lwbox
%           \fi
%         \else
%           \usebox0
%         \fi
%       \else
%         \lwbox
%       \fi
%     \else
%       \usebox0
%     \fi
%   \else
%     \lwbox
%   \fi
% \else
%   \usebox0
% \fi
% \end{quote}
% If you have a \xfile{docstrip.cfg} that configures and enables \docstrip's
% TDS installing feature, then some files can already be in the right
% place, see the documentation of \docstrip.
%
% \subsection{Refresh file name databases}
%
% If your \TeX~distribution
% (\teTeX, \mikTeX, \dots) relies on file name databases, you must refresh
% these. For example, \teTeX\ users run \verb|texhash| or
% \verb|mktexlsr|.
%
% \subsection{Some details for the interested}
%
% \paragraph{Attached source.}
%
% The PDF documentation on CTAN also includes the
% \xfile{.dtx} source file. It can be extracted by
% AcrobatReader 6 or higher. Another option is \textsf{pdftk},
% e.g. unpack the file into the current directory:
% \begin{quote}
%   \verb|pdftk stackrel.pdf unpack_files output .|
% \end{quote}
%
% \paragraph{Unpacking with \LaTeX.}
% The \xfile{.dtx} chooses its action depending on the format:
% \begin{description}
% \item[\plainTeX:] Run \docstrip\ and extract the files.
% \item[\LaTeX:] Generate the documentation.
% \end{description}
% If you insist on using \LaTeX\ for \docstrip\ (really,
% \docstrip\ does not need \LaTeX), then inform the autodetect routine
% about your intention:
% \begin{quote}
%   \verb|latex \let\install=y\input{stackrel.dtx}|
% \end{quote}
% Do not forget to quote the argument according to the demands
% of your shell.
%
% \paragraph{Generating the documentation.}
% You can use both the \xfile{.dtx} or the \xfile{.drv} to generate
% the documentation. The process can be configured by the
% configuration file \xfile{ltxdoc.cfg}. For instance, put this
% line into this file, if you want to have A4 as paper format:
% \begin{quote}
%   \verb|\PassOptionsToClass{a4paper}{article}|
% \end{quote}
% An example follows how to generate the
% documentation with pdf\LaTeX:
% \begin{quote}
%\begin{verbatim}
%pdflatex stackrel.dtx
%makeindex -s gind.ist stackrel.idx
%pdflatex stackrel.dtx
%makeindex -s gind.ist stackrel.idx
%pdflatex stackrel.dtx
%\end{verbatim}
% \end{quote}
%
% \begin{History}
%   \begin{Version}{2006/12/02 v1.0}
%   \item
%     First version.
%   \end{Version}
%   \begin{Version}{2007/05/06 v1.1}
%   \item
%     Uses package \xpackage{etexcmds}.
%   \end{Version}
%   \begin{Version}{2007/11/11 v1.2}
%   \item
%     Use of package \xpackage{pdftexcmds} for \LuaTeX\ support.
%   \end{Version}
%   \begin{Version}{2016/05/16 v1.3}
%   \item
%     Documentation updates.
%   \end{Version}
% \end{History}
%
% \clearpage
% \PrintIndex
%
% \Finale
\endinput
|
% \end{quote}
% Do not forget to quote the argument according to the demands
% of your shell.
%
% \paragraph{Generating the documentation.}
% You can use both the \xfile{.dtx} or the \xfile{.drv} to generate
% the documentation. The process can be configured by the
% configuration file \xfile{ltxdoc.cfg}. For instance, put this
% line into this file, if you want to have A4 as paper format:
% \begin{quote}
%   \verb|\PassOptionsToClass{a4paper}{article}|
% \end{quote}
% An example follows how to generate the
% documentation with pdf\LaTeX:
% \begin{quote}
%\begin{verbatim}
%pdflatex stackrel.dtx
%makeindex -s gind.ist stackrel.idx
%pdflatex stackrel.dtx
%makeindex -s gind.ist stackrel.idx
%pdflatex stackrel.dtx
%\end{verbatim}
% \end{quote}
%
% \begin{History}
%   \begin{Version}{2006/12/02 v1.0}
%   \item
%     First version.
%   \end{Version}
%   \begin{Version}{2007/05/06 v1.1}
%   \item
%     Uses package \xpackage{etexcmds}.
%   \end{Version}
%   \begin{Version}{2007/11/11 v1.2}
%   \item
%     Use of package \xpackage{pdftexcmds} for \LuaTeX\ support.
%   \end{Version}
%   \begin{Version}{2016/05/16 v1.3}
%   \item
%     Documentation updates.
%   \end{Version}
% \end{History}
%
% \clearpage
% \PrintIndex
%
% \Finale
\endinput

%        (quote the arguments according to the demands of your shell)
%
% Documentation:
%    (a) If stackrel.drv is present:
%           latex stackrel.drv
%    (b) Without stackrel.drv:
%           latex stackrel.dtx; ...
%    The class ltxdoc loads the configuration file ltxdoc.cfg
%    if available. Here you can specify further options, e.g.
%    use A4 as paper format:
%       \PassOptionsToClass{a4paper}{article}
%
%    Programm calls to get the documentation (example):
%       pdflatex stackrel.dtx
%       makeindex -s gind.ist stackrel.idx
%       pdflatex stackrel.dtx
%       makeindex -s gind.ist stackrel.idx
%       pdflatex stackrel.dtx
%
% Installation:
%    TDS:tex/latex/oberdiek/stackrel.sty
%    TDS:doc/latex/oberdiek/stackrel.pdf
%    TDS:source/latex/oberdiek/stackrel.dtx
%
%<*ignore>
\begingroup
  \catcode123=1 %
  \catcode125=2 %
  \def\x{LaTeX2e}%
\expandafter\endgroup
\ifcase 0\ifx\install y1\fi\expandafter
         \ifx\csname processbatchFile\endcsname\relax\else1\fi
         \ifx\fmtname\x\else 1\fi\relax
\else\csname fi\endcsname
%</ignore>
%<*install>
\input docstrip.tex
\Msg{************************************************************************}
\Msg{* Installation}
\Msg{* Package: stackrel 2016/05/16 v1.3 Adding subscript option to stackrel (HO)}
\Msg{************************************************************************}

\keepsilent
\askforoverwritefalse

\let\MetaPrefix\relax
\preamble

This is a generated file.

Project: stackrel
Version: 2016/05/16 v1.3

Copyright (C) 2006, 2007 by
   Heiko Oberdiek <heiko.oberdiek at googlemail.com>

This work may be distributed and/or modified under the
conditions of the LaTeX Project Public License, either
version 1.3c of this license or (at your option) any later
version. This version of this license is in
   http://www.latex-project.org/lppl/lppl-1-3c.txt
and the latest version of this license is in
   http://www.latex-project.org/lppl.txt
and version 1.3 or later is part of all distributions of
LaTeX version 2005/12/01 or later.

This work has the LPPL maintenance status "maintained".

This Current Maintainer of this work is Heiko Oberdiek.

This work consists of the main source file stackrel.dtx
and the derived files
   stackrel.sty, stackrel.pdf, stackrel.ins, stackrel.drv.

\endpreamble
\let\MetaPrefix\DoubleperCent

\generate{%
  \file{stackrel.ins}{\from{stackrel.dtx}{install}}%
  \file{stackrel.drv}{\from{stackrel.dtx}{driver}}%
  \usedir{tex/latex/oberdiek}%
  \file{stackrel.sty}{\from{stackrel.dtx}{package}}%
  \nopreamble
  \nopostamble
%  \usedir{source/latex/oberdiek/catalogue}%
%  \file{stackrel.xml}{\from{stackrel.dtx}{catalogue}}%
}

\catcode32=13\relax% active space
\let =\space%
\Msg{************************************************************************}
\Msg{*}
\Msg{* To finish the installation you have to move the following}
\Msg{* file into a directory searched by TeX:}
\Msg{*}
\Msg{*     stackrel.sty}
\Msg{*}
\Msg{* To produce the documentation run the file `stackrel.drv'}
\Msg{* through LaTeX.}
\Msg{*}
\Msg{* Happy TeXing!}
\Msg{*}
\Msg{************************************************************************}

\endbatchfile
%</install>
%<*ignore>
\fi
%</ignore>
%<*driver>
\NeedsTeXFormat{LaTeX2e}
\ProvidesFile{stackrel.drv}%
  [2016/05/16 v1.3 Adding subscript option to stackrel (HO)]%
\documentclass{ltxdoc}
\usepackage{amsmath}
\usepackage{holtxdoc}[2011/11/22]
\usepackage{stackrel}[2016/05/16]
\begin{document}
  \DocInput{stackrel.dtx}%
\end{document}
%</driver>
% \fi
%
%
% \CharacterTable
%  {Upper-case    \A\B\C\D\E\F\G\H\I\J\K\L\M\N\O\P\Q\R\S\T\U\V\W\X\Y\Z
%   Lower-case    \a\b\c\d\e\f\g\h\i\j\k\l\m\n\o\p\q\r\s\t\u\v\w\x\y\z
%   Digits        \0\1\2\3\4\5\6\7\8\9
%   Exclamation   \!     Double quote  \"     Hash (number) \#
%   Dollar        \$     Percent       \%     Ampersand     \&
%   Acute accent  \'     Left paren    \(     Right paren   \)
%   Asterisk      \*     Plus          \+     Comma         \,
%   Minus         \-     Point         \.     Solidus       \/
%   Colon         \:     Semicolon     \;     Less than     \<
%   Equals        \=     Greater than  \>     Question mark \?
%   Commercial at \@     Left bracket  \[     Backslash     \\
%   Right bracket \]     Circumflex    \^     Underscore    \_
%   Grave accent  \`     Left brace    \{     Vertical bar  \|
%   Right brace   \}     Tilde         \~}
%
% \GetFileInfo{stackrel.drv}
%
% \title{The \xpackage{stackrel} package}
% \date{2016/05/16 v1.3}
% \author{Heiko Oberdiek\thanks
% {Please report any issues at \url{https://github.com/ho-tex/oberdiek/issues}}\\
% \xemail{heiko.oberdiek at googlemail.com}}
%
% \maketitle
%
% \begin{abstract}
% This package adds an optional argument to \cs{stackrel} for
% putting something below the relational symbol and defines
% \cs{stackbin} for binary symbols.
% \end{abstract}
%
% \tableofcontents
%
% \section{User interface}
%
% \LaTeX's \cs{stackrel} allows a superscript above a relational symbol,
% but pure \LaTeX\ does not provide a macro for putting a subscript
% below the symbol. This is supported by \AmS\LaTeX's \cs{underset}
% macro that works on both relational and binary symbols. A combination
% of \cs{underset} and \cs{overset} can be used to put \mbox{sub-} and
% superscripts to the same symbol.
%
% This package \xpackage{stackrel} extends the syntax of \cs{stackrel}
% by adding an optional argument for the subscript position.
% It follows the syntax of extensible arrows of packages
% \xpackage{amsmath} and \xpackage{mathtools}.
%
% \begin{declcs}{stackrel}
%   |[|\meta{subscript}|]| \M{superscript} \M{rel}\\
%   \cs{stackbin}
%   |[|\meta{subscript}|]| \M{superscript} \M{bin}
% \end{declcs}
% Example:
% \begin{quote}
% |A \stackbin[\text{and}]{}{+} B \stackrel[x]{!}{=} C|\\
% $A \stackbin[\text{and}]{}{+} B \stackrel[x]{!}{=} C$
% \end{quote}
%
% \StopEventually{
% }
%
% \section{Implementation}
%
%    \begin{macrocode}
%<*package>
\NeedsTeXFormat{LaTeX2e}
\ProvidesPackage{stackrel}
  [2016/05/16 v1.3 Adding subscript option to stackrel (HO)]%
%    \end{macrocode}
%
%    Given the original definition of \cs{stackrel} the addition
%    of the optional argument is straightforward. If an argument
%    is empty, then the corresponding sub- or superscript is
%    suppressed.
%
%    Depending on the available resources (\eTeX, \pdfTeX)
%    three methods are given for testing emptyness. All tests
%    allow the hash to be used inside the arguments without
%    doubling (for the unlikely case that someone wants to
%    define macros with arguments).
%    \begin{macro}{\stack@relbin}
%    \begin{macrocode}
\RequirePackage{etexcmds}[2007/09/09]
\ifetex@unexpanded
  \RequirePackage{pdftexcmds}[2016/05/16]%
  \begingroup\expandafter\expandafter\expandafter\endgroup
  \expandafter\ifx\csname pdf@strcmp\endcsname\relax
    \newcommand*{\stack@relbin}[3][]{%
      \mathop{#3}\limits
      \edef\reserved@a{\etex@unexpanded{#1}}%
      \ifx\reserved@a\@empty\else_{#1}\fi
      \edef\reserved@a{\etex@unexpanded{#2}}%
      \ifx\reserved@a\@empty\else^{#2}\fi
      \egroup
    }%
  \else
    \newcommand*{\stack@relbin}[3][]{%
      \mathop{#3}\limits
      \ifcase\pdf@strcmp{\detokenize{#1}}{}\else_{#1}\fi
      \ifcase\pdf@strcmp{\detokenize{#2}}{}\else^{#2}\fi
      \egroup
    }%
  \fi
\else
  \newcommand*{\stack@relbin}[3][]{%
    \mathop{#3}\limits
    \toks@{#1}%
    \edef\reserved@a{\the\toks@}%
    \ifx\reserved@a\@empty\else_{#1}\fi
    \toks@{#2}%
    \edef\reserved@a{\the\toks@}%
    \ifx\reserved@a\@empty\else^{#2}\fi
    \egroup
  }%
\fi
%    \end{macrocode}
%    \end{macro}
%    \begin{macro}{\stackrel}
%    \begin{macrocode}
\renewcommand*{\stackrel}{%
  \mathrel\bgroup\stack@relbin
}
%    \end{macrocode}
%    \end{macro}
%    \begin{macro}{\stackbin}
%    \begin{macrocode}
\newcommand*{\stackbin}{%
  \mathbin\bgroup\stack@relbin
}
%    \end{macrocode}
%    \end{macro}
%
%    \begin{macrocode}
%</package>
%    \end{macrocode}
%
% \section{Installation}
%
% \subsection{Download}
%
% \paragraph{Package.} This package is available on
% CTAN\footnote{\CTANpkg{stackrel}}:
% \begin{description}
% \item[\CTAN{macros/latex/contrib/oberdiek/stackrel.dtx}] The source file.
% \item[\CTAN{macros/latex/contrib/oberdiek/stackrel.pdf}] Documentation.
% \end{description}
%
%
% \paragraph{Bundle.} All the packages of the bundle `oberdiek'
% are also available in a TDS compliant ZIP archive. There
% the packages are already unpacked and the documentation files
% are generated. The files and directories obey the TDS standard.
% \begin{description}
% \item[\CTANinstall{install/macros/latex/contrib/oberdiek.tds.zip}]
% \end{description}
% \emph{TDS} refers to the standard ``A Directory Structure
% for \TeX\ Files'' (\CTAN{tds/tds.pdf}). Directories
% with \xfile{texmf} in their name are usually organized this way.
%
% \subsection{Bundle installation}
%
% \paragraph{Unpacking.} Unpack the \xfile{oberdiek.tds.zip} in the
% TDS tree (also known as \xfile{texmf} tree) of your choice.
% Example (linux):
% \begin{quote}
%   |unzip oberdiek.tds.zip -d ~/texmf|
% \end{quote}
%
% \paragraph{Script installation.}
% Check the directory \xfile{TDS:scripts/oberdiek/} for
% scripts that need further installation steps.
% Package \xpackage{attachfile2} comes with the Perl script
% \xfile{pdfatfi.pl} that should be installed in such a way
% that it can be called as \texttt{pdfatfi}.
% Example (linux):
% \begin{quote}
%   |chmod +x scripts/oberdiek/pdfatfi.pl|\\
%   |cp scripts/oberdiek/pdfatfi.pl /usr/local/bin/|
% \end{quote}
%
% \subsection{Package installation}
%
% \paragraph{Unpacking.} The \xfile{.dtx} file is a self-extracting
% \docstrip\ archive. The files are extracted by running the
% \xfile{.dtx} through \plainTeX:
% \begin{quote}
%   \verb|tex stackrel.dtx|
% \end{quote}
%
% \paragraph{TDS.} Now the different files must be moved into
% the different directories in your installation TDS tree
% (also known as \xfile{texmf} tree):
% \begin{quote}
% \def\t{^^A
% \begin{tabular}{@{}>{\ttfamily}l@{ $\rightarrow$ }>{\ttfamily}l@{}}
%   stackrel.sty & tex/latex/oberdiek/stackrel.sty\\
%   stackrel.pdf & doc/latex/oberdiek/stackrel.pdf\\
%   stackrel.dtx & source/latex/oberdiek/stackrel.dtx\\
% \end{tabular}^^A
% }^^A
% \sbox0{\t}^^A
% \ifdim\wd0>\linewidth
%   \begingroup
%     \advance\linewidth by\leftmargin
%     \advance\linewidth by\rightmargin
%   \edef\x{\endgroup
%     \def\noexpand\lw{\the\linewidth}^^A
%   }\x
%   \def\lwbox{^^A
%     \leavevmode
%     \hbox to \linewidth{^^A
%       \kern-\leftmargin\relax
%       \hss
%       \usebox0
%       \hss
%       \kern-\rightmargin\relax
%     }^^A
%   }^^A
%   \ifdim\wd0>\lw
%     \sbox0{\small\t}^^A
%     \ifdim\wd0>\linewidth
%       \ifdim\wd0>\lw
%         \sbox0{\footnotesize\t}^^A
%         \ifdim\wd0>\linewidth
%           \ifdim\wd0>\lw
%             \sbox0{\scriptsize\t}^^A
%             \ifdim\wd0>\linewidth
%               \ifdim\wd0>\lw
%                 \sbox0{\tiny\t}^^A
%                 \ifdim\wd0>\linewidth
%                   \lwbox
%                 \else
%                   \usebox0
%                 \fi
%               \else
%                 \lwbox
%               \fi
%             \else
%               \usebox0
%             \fi
%           \else
%             \lwbox
%           \fi
%         \else
%           \usebox0
%         \fi
%       \else
%         \lwbox
%       \fi
%     \else
%       \usebox0
%     \fi
%   \else
%     \lwbox
%   \fi
% \else
%   \usebox0
% \fi
% \end{quote}
% If you have a \xfile{docstrip.cfg} that configures and enables \docstrip's
% TDS installing feature, then some files can already be in the right
% place, see the documentation of \docstrip.
%
% \subsection{Refresh file name databases}
%
% If your \TeX~distribution
% (\teTeX, \mikTeX, \dots) relies on file name databases, you must refresh
% these. For example, \teTeX\ users run \verb|texhash| or
% \verb|mktexlsr|.
%
% \subsection{Some details for the interested}
%
% \paragraph{Attached source.}
%
% The PDF documentation on CTAN also includes the
% \xfile{.dtx} source file. It can be extracted by
% AcrobatReader 6 or higher. Another option is \textsf{pdftk},
% e.g. unpack the file into the current directory:
% \begin{quote}
%   \verb|pdftk stackrel.pdf unpack_files output .|
% \end{quote}
%
% \paragraph{Unpacking with \LaTeX.}
% The \xfile{.dtx} chooses its action depending on the format:
% \begin{description}
% \item[\plainTeX:] Run \docstrip\ and extract the files.
% \item[\LaTeX:] Generate the documentation.
% \end{description}
% If you insist on using \LaTeX\ for \docstrip\ (really,
% \docstrip\ does not need \LaTeX), then inform the autodetect routine
% about your intention:
% \begin{quote}
%   \verb|latex \let\install=y% \iffalse meta-comment
%
% File: stackrel.dtx
% Version: 2016/05/16 v1.3
% Info: Adding subscript option to stackrel
%
% Copyright (C) 2006, 2007 by
%    Heiko Oberdiek <heiko.oberdiek at googlemail.com>
%    2016
%    https://github.com/ho-tex/oberdiek/issues
%
% This work may be distributed and/or modified under the
% conditions of the LaTeX Project Public License, either
% version 1.3c of this license or (at your option) any later
% version. This version of this license is in
%    http://www.latex-project.org/lppl/lppl-1-3c.txt
% and the latest version of this license is in
%    http://www.latex-project.org/lppl.txt
% and version 1.3 or later is part of all distributions of
% LaTeX version 2005/12/01 or later.
%
% This work has the LPPL maintenance status "maintained".
%
% This Current Maintainer of this work is Heiko Oberdiek.
%
% This work consists of the main source file stackrel.dtx
% and the derived files
%    stackrel.sty, stackrel.pdf, stackrel.ins, stackrel.drv.
%
% Distribution:
%    CTAN:macros/latex/contrib/oberdiek/stackrel.dtx
%    CTAN:macros/latex/contrib/oberdiek/stackrel.pdf
%
% Unpacking:
%    (a) If stackrel.ins is present:
%           tex stackrel.ins
%    (b) Without stackrel.ins:
%           tex stackrel.dtx
%    (c) If you insist on using LaTeX
%           latex \let\install=y% \iffalse meta-comment
%
% File: stackrel.dtx
% Version: 2016/05/16 v1.3
% Info: Adding subscript option to stackrel
%
% Copyright (C) 2006, 2007 by
%    Heiko Oberdiek <heiko.oberdiek at googlemail.com>
%    2016
%    https://github.com/ho-tex/oberdiek/issues
%
% This work may be distributed and/or modified under the
% conditions of the LaTeX Project Public License, either
% version 1.3c of this license or (at your option) any later
% version. This version of this license is in
%    http://www.latex-project.org/lppl/lppl-1-3c.txt
% and the latest version of this license is in
%    http://www.latex-project.org/lppl.txt
% and version 1.3 or later is part of all distributions of
% LaTeX version 2005/12/01 or later.
%
% This work has the LPPL maintenance status "maintained".
%
% This Current Maintainer of this work is Heiko Oberdiek.
%
% This work consists of the main source file stackrel.dtx
% and the derived files
%    stackrel.sty, stackrel.pdf, stackrel.ins, stackrel.drv.
%
% Distribution:
%    CTAN:macros/latex/contrib/oberdiek/stackrel.dtx
%    CTAN:macros/latex/contrib/oberdiek/stackrel.pdf
%
% Unpacking:
%    (a) If stackrel.ins is present:
%           tex stackrel.ins
%    (b) Without stackrel.ins:
%           tex stackrel.dtx
%    (c) If you insist on using LaTeX
%           latex \let\install=y\input{stackrel.dtx}
%        (quote the arguments according to the demands of your shell)
%
% Documentation:
%    (a) If stackrel.drv is present:
%           latex stackrel.drv
%    (b) Without stackrel.drv:
%           latex stackrel.dtx; ...
%    The class ltxdoc loads the configuration file ltxdoc.cfg
%    if available. Here you can specify further options, e.g.
%    use A4 as paper format:
%       \PassOptionsToClass{a4paper}{article}
%
%    Programm calls to get the documentation (example):
%       pdflatex stackrel.dtx
%       makeindex -s gind.ist stackrel.idx
%       pdflatex stackrel.dtx
%       makeindex -s gind.ist stackrel.idx
%       pdflatex stackrel.dtx
%
% Installation:
%    TDS:tex/latex/oberdiek/stackrel.sty
%    TDS:doc/latex/oberdiek/stackrel.pdf
%    TDS:source/latex/oberdiek/stackrel.dtx
%
%<*ignore>
\begingroup
  \catcode123=1 %
  \catcode125=2 %
  \def\x{LaTeX2e}%
\expandafter\endgroup
\ifcase 0\ifx\install y1\fi\expandafter
         \ifx\csname processbatchFile\endcsname\relax\else1\fi
         \ifx\fmtname\x\else 1\fi\relax
\else\csname fi\endcsname
%</ignore>
%<*install>
\input docstrip.tex
\Msg{************************************************************************}
\Msg{* Installation}
\Msg{* Package: stackrel 2016/05/16 v1.3 Adding subscript option to stackrel (HO)}
\Msg{************************************************************************}

\keepsilent
\askforoverwritefalse

\let\MetaPrefix\relax
\preamble

This is a generated file.

Project: stackrel
Version: 2016/05/16 v1.3

Copyright (C) 2006, 2007 by
   Heiko Oberdiek <heiko.oberdiek at googlemail.com>

This work may be distributed and/or modified under the
conditions of the LaTeX Project Public License, either
version 1.3c of this license or (at your option) any later
version. This version of this license is in
   http://www.latex-project.org/lppl/lppl-1-3c.txt
and the latest version of this license is in
   http://www.latex-project.org/lppl.txt
and version 1.3 or later is part of all distributions of
LaTeX version 2005/12/01 or later.

This work has the LPPL maintenance status "maintained".

This Current Maintainer of this work is Heiko Oberdiek.

This work consists of the main source file stackrel.dtx
and the derived files
   stackrel.sty, stackrel.pdf, stackrel.ins, stackrel.drv.

\endpreamble
\let\MetaPrefix\DoubleperCent

\generate{%
  \file{stackrel.ins}{\from{stackrel.dtx}{install}}%
  \file{stackrel.drv}{\from{stackrel.dtx}{driver}}%
  \usedir{tex/latex/oberdiek}%
  \file{stackrel.sty}{\from{stackrel.dtx}{package}}%
  \nopreamble
  \nopostamble
%  \usedir{source/latex/oberdiek/catalogue}%
%  \file{stackrel.xml}{\from{stackrel.dtx}{catalogue}}%
}

\catcode32=13\relax% active space
\let =\space%
\Msg{************************************************************************}
\Msg{*}
\Msg{* To finish the installation you have to move the following}
\Msg{* file into a directory searched by TeX:}
\Msg{*}
\Msg{*     stackrel.sty}
\Msg{*}
\Msg{* To produce the documentation run the file `stackrel.drv'}
\Msg{* through LaTeX.}
\Msg{*}
\Msg{* Happy TeXing!}
\Msg{*}
\Msg{************************************************************************}

\endbatchfile
%</install>
%<*ignore>
\fi
%</ignore>
%<*driver>
\NeedsTeXFormat{LaTeX2e}
\ProvidesFile{stackrel.drv}%
  [2016/05/16 v1.3 Adding subscript option to stackrel (HO)]%
\documentclass{ltxdoc}
\usepackage{amsmath}
\usepackage{holtxdoc}[2011/11/22]
\usepackage{stackrel}[2016/05/16]
\begin{document}
  \DocInput{stackrel.dtx}%
\end{document}
%</driver>
% \fi
%
%
% \CharacterTable
%  {Upper-case    \A\B\C\D\E\F\G\H\I\J\K\L\M\N\O\P\Q\R\S\T\U\V\W\X\Y\Z
%   Lower-case    \a\b\c\d\e\f\g\h\i\j\k\l\m\n\o\p\q\r\s\t\u\v\w\x\y\z
%   Digits        \0\1\2\3\4\5\6\7\8\9
%   Exclamation   \!     Double quote  \"     Hash (number) \#
%   Dollar        \$     Percent       \%     Ampersand     \&
%   Acute accent  \'     Left paren    \(     Right paren   \)
%   Asterisk      \*     Plus          \+     Comma         \,
%   Minus         \-     Point         \.     Solidus       \/
%   Colon         \:     Semicolon     \;     Less than     \<
%   Equals        \=     Greater than  \>     Question mark \?
%   Commercial at \@     Left bracket  \[     Backslash     \\
%   Right bracket \]     Circumflex    \^     Underscore    \_
%   Grave accent  \`     Left brace    \{     Vertical bar  \|
%   Right brace   \}     Tilde         \~}
%
% \GetFileInfo{stackrel.drv}
%
% \title{The \xpackage{stackrel} package}
% \date{2016/05/16 v1.3}
% \author{Heiko Oberdiek\thanks
% {Please report any issues at \url{https://github.com/ho-tex/oberdiek/issues}}\\
% \xemail{heiko.oberdiek at googlemail.com}}
%
% \maketitle
%
% \begin{abstract}
% This package adds an optional argument to \cs{stackrel} for
% putting something below the relational symbol and defines
% \cs{stackbin} for binary symbols.
% \end{abstract}
%
% \tableofcontents
%
% \section{User interface}
%
% \LaTeX's \cs{stackrel} allows a superscript above a relational symbol,
% but pure \LaTeX\ does not provide a macro for putting a subscript
% below the symbol. This is supported by \AmS\LaTeX's \cs{underset}
% macro that works on both relational and binary symbols. A combination
% of \cs{underset} and \cs{overset} can be used to put \mbox{sub-} and
% superscripts to the same symbol.
%
% This package \xpackage{stackrel} extends the syntax of \cs{stackrel}
% by adding an optional argument for the subscript position.
% It follows the syntax of extensible arrows of packages
% \xpackage{amsmath} and \xpackage{mathtools}.
%
% \begin{declcs}{stackrel}
%   |[|\meta{subscript}|]| \M{superscript} \M{rel}\\
%   \cs{stackbin}
%   |[|\meta{subscript}|]| \M{superscript} \M{bin}
% \end{declcs}
% Example:
% \begin{quote}
% |A \stackbin[\text{and}]{}{+} B \stackrel[x]{!}{=} C|\\
% $A \stackbin[\text{and}]{}{+} B \stackrel[x]{!}{=} C$
% \end{quote}
%
% \StopEventually{
% }
%
% \section{Implementation}
%
%    \begin{macrocode}
%<*package>
\NeedsTeXFormat{LaTeX2e}
\ProvidesPackage{stackrel}
  [2016/05/16 v1.3 Adding subscript option to stackrel (HO)]%
%    \end{macrocode}
%
%    Given the original definition of \cs{stackrel} the addition
%    of the optional argument is straightforward. If an argument
%    is empty, then the corresponding sub- or superscript is
%    suppressed.
%
%    Depending on the available resources (\eTeX, \pdfTeX)
%    three methods are given for testing emptyness. All tests
%    allow the hash to be used inside the arguments without
%    doubling (for the unlikely case that someone wants to
%    define macros with arguments).
%    \begin{macro}{\stack@relbin}
%    \begin{macrocode}
\RequirePackage{etexcmds}[2007/09/09]
\ifetex@unexpanded
  \RequirePackage{pdftexcmds}[2016/05/16]%
  \begingroup\expandafter\expandafter\expandafter\endgroup
  \expandafter\ifx\csname pdf@strcmp\endcsname\relax
    \newcommand*{\stack@relbin}[3][]{%
      \mathop{#3}\limits
      \edef\reserved@a{\etex@unexpanded{#1}}%
      \ifx\reserved@a\@empty\else_{#1}\fi
      \edef\reserved@a{\etex@unexpanded{#2}}%
      \ifx\reserved@a\@empty\else^{#2}\fi
      \egroup
    }%
  \else
    \newcommand*{\stack@relbin}[3][]{%
      \mathop{#3}\limits
      \ifcase\pdf@strcmp{\detokenize{#1}}{}\else_{#1}\fi
      \ifcase\pdf@strcmp{\detokenize{#2}}{}\else^{#2}\fi
      \egroup
    }%
  \fi
\else
  \newcommand*{\stack@relbin}[3][]{%
    \mathop{#3}\limits
    \toks@{#1}%
    \edef\reserved@a{\the\toks@}%
    \ifx\reserved@a\@empty\else_{#1}\fi
    \toks@{#2}%
    \edef\reserved@a{\the\toks@}%
    \ifx\reserved@a\@empty\else^{#2}\fi
    \egroup
  }%
\fi
%    \end{macrocode}
%    \end{macro}
%    \begin{macro}{\stackrel}
%    \begin{macrocode}
\renewcommand*{\stackrel}{%
  \mathrel\bgroup\stack@relbin
}
%    \end{macrocode}
%    \end{macro}
%    \begin{macro}{\stackbin}
%    \begin{macrocode}
\newcommand*{\stackbin}{%
  \mathbin\bgroup\stack@relbin
}
%    \end{macrocode}
%    \end{macro}
%
%    \begin{macrocode}
%</package>
%    \end{macrocode}
%
% \section{Installation}
%
% \subsection{Download}
%
% \paragraph{Package.} This package is available on
% CTAN\footnote{\CTANpkg{stackrel}}:
% \begin{description}
% \item[\CTAN{macros/latex/contrib/oberdiek/stackrel.dtx}] The source file.
% \item[\CTAN{macros/latex/contrib/oberdiek/stackrel.pdf}] Documentation.
% \end{description}
%
%
% \paragraph{Bundle.} All the packages of the bundle `oberdiek'
% are also available in a TDS compliant ZIP archive. There
% the packages are already unpacked and the documentation files
% are generated. The files and directories obey the TDS standard.
% \begin{description}
% \item[\CTANinstall{install/macros/latex/contrib/oberdiek.tds.zip}]
% \end{description}
% \emph{TDS} refers to the standard ``A Directory Structure
% for \TeX\ Files'' (\CTAN{tds/tds.pdf}). Directories
% with \xfile{texmf} in their name are usually organized this way.
%
% \subsection{Bundle installation}
%
% \paragraph{Unpacking.} Unpack the \xfile{oberdiek.tds.zip} in the
% TDS tree (also known as \xfile{texmf} tree) of your choice.
% Example (linux):
% \begin{quote}
%   |unzip oberdiek.tds.zip -d ~/texmf|
% \end{quote}
%
% \paragraph{Script installation.}
% Check the directory \xfile{TDS:scripts/oberdiek/} for
% scripts that need further installation steps.
% Package \xpackage{attachfile2} comes with the Perl script
% \xfile{pdfatfi.pl} that should be installed in such a way
% that it can be called as \texttt{pdfatfi}.
% Example (linux):
% \begin{quote}
%   |chmod +x scripts/oberdiek/pdfatfi.pl|\\
%   |cp scripts/oberdiek/pdfatfi.pl /usr/local/bin/|
% \end{quote}
%
% \subsection{Package installation}
%
% \paragraph{Unpacking.} The \xfile{.dtx} file is a self-extracting
% \docstrip\ archive. The files are extracted by running the
% \xfile{.dtx} through \plainTeX:
% \begin{quote}
%   \verb|tex stackrel.dtx|
% \end{quote}
%
% \paragraph{TDS.} Now the different files must be moved into
% the different directories in your installation TDS tree
% (also known as \xfile{texmf} tree):
% \begin{quote}
% \def\t{^^A
% \begin{tabular}{@{}>{\ttfamily}l@{ $\rightarrow$ }>{\ttfamily}l@{}}
%   stackrel.sty & tex/latex/oberdiek/stackrel.sty\\
%   stackrel.pdf & doc/latex/oberdiek/stackrel.pdf\\
%   stackrel.dtx & source/latex/oberdiek/stackrel.dtx\\
% \end{tabular}^^A
% }^^A
% \sbox0{\t}^^A
% \ifdim\wd0>\linewidth
%   \begingroup
%     \advance\linewidth by\leftmargin
%     \advance\linewidth by\rightmargin
%   \edef\x{\endgroup
%     \def\noexpand\lw{\the\linewidth}^^A
%   }\x
%   \def\lwbox{^^A
%     \leavevmode
%     \hbox to \linewidth{^^A
%       \kern-\leftmargin\relax
%       \hss
%       \usebox0
%       \hss
%       \kern-\rightmargin\relax
%     }^^A
%   }^^A
%   \ifdim\wd0>\lw
%     \sbox0{\small\t}^^A
%     \ifdim\wd0>\linewidth
%       \ifdim\wd0>\lw
%         \sbox0{\footnotesize\t}^^A
%         \ifdim\wd0>\linewidth
%           \ifdim\wd0>\lw
%             \sbox0{\scriptsize\t}^^A
%             \ifdim\wd0>\linewidth
%               \ifdim\wd0>\lw
%                 \sbox0{\tiny\t}^^A
%                 \ifdim\wd0>\linewidth
%                   \lwbox
%                 \else
%                   \usebox0
%                 \fi
%               \else
%                 \lwbox
%               \fi
%             \else
%               \usebox0
%             \fi
%           \else
%             \lwbox
%           \fi
%         \else
%           \usebox0
%         \fi
%       \else
%         \lwbox
%       \fi
%     \else
%       \usebox0
%     \fi
%   \else
%     \lwbox
%   \fi
% \else
%   \usebox0
% \fi
% \end{quote}
% If you have a \xfile{docstrip.cfg} that configures and enables \docstrip's
% TDS installing feature, then some files can already be in the right
% place, see the documentation of \docstrip.
%
% \subsection{Refresh file name databases}
%
% If your \TeX~distribution
% (\teTeX, \mikTeX, \dots) relies on file name databases, you must refresh
% these. For example, \teTeX\ users run \verb|texhash| or
% \verb|mktexlsr|.
%
% \subsection{Some details for the interested}
%
% \paragraph{Attached source.}
%
% The PDF documentation on CTAN also includes the
% \xfile{.dtx} source file. It can be extracted by
% AcrobatReader 6 or higher. Another option is \textsf{pdftk},
% e.g. unpack the file into the current directory:
% \begin{quote}
%   \verb|pdftk stackrel.pdf unpack_files output .|
% \end{quote}
%
% \paragraph{Unpacking with \LaTeX.}
% The \xfile{.dtx} chooses its action depending on the format:
% \begin{description}
% \item[\plainTeX:] Run \docstrip\ and extract the files.
% \item[\LaTeX:] Generate the documentation.
% \end{description}
% If you insist on using \LaTeX\ for \docstrip\ (really,
% \docstrip\ does not need \LaTeX), then inform the autodetect routine
% about your intention:
% \begin{quote}
%   \verb|latex \let\install=y\input{stackrel.dtx}|
% \end{quote}
% Do not forget to quote the argument according to the demands
% of your shell.
%
% \paragraph{Generating the documentation.}
% You can use both the \xfile{.dtx} or the \xfile{.drv} to generate
% the documentation. The process can be configured by the
% configuration file \xfile{ltxdoc.cfg}. For instance, put this
% line into this file, if you want to have A4 as paper format:
% \begin{quote}
%   \verb|\PassOptionsToClass{a4paper}{article}|
% \end{quote}
% An example follows how to generate the
% documentation with pdf\LaTeX:
% \begin{quote}
%\begin{verbatim}
%pdflatex stackrel.dtx
%makeindex -s gind.ist stackrel.idx
%pdflatex stackrel.dtx
%makeindex -s gind.ist stackrel.idx
%pdflatex stackrel.dtx
%\end{verbatim}
% \end{quote}
%
% \begin{History}
%   \begin{Version}{2006/12/02 v1.0}
%   \item
%     First version.
%   \end{Version}
%   \begin{Version}{2007/05/06 v1.1}
%   \item
%     Uses package \xpackage{etexcmds}.
%   \end{Version}
%   \begin{Version}{2007/11/11 v1.2}
%   \item
%     Use of package \xpackage{pdftexcmds} for \LuaTeX\ support.
%   \end{Version}
%   \begin{Version}{2016/05/16 v1.3}
%   \item
%     Documentation updates.
%   \end{Version}
% \end{History}
%
% \clearpage
% \PrintIndex
%
% \Finale
\endinput

%        (quote the arguments according to the demands of your shell)
%
% Documentation:
%    (a) If stackrel.drv is present:
%           latex stackrel.drv
%    (b) Without stackrel.drv:
%           latex stackrel.dtx; ...
%    The class ltxdoc loads the configuration file ltxdoc.cfg
%    if available. Here you can specify further options, e.g.
%    use A4 as paper format:
%       \PassOptionsToClass{a4paper}{article}
%
%    Programm calls to get the documentation (example):
%       pdflatex stackrel.dtx
%       makeindex -s gind.ist stackrel.idx
%       pdflatex stackrel.dtx
%       makeindex -s gind.ist stackrel.idx
%       pdflatex stackrel.dtx
%
% Installation:
%    TDS:tex/latex/oberdiek/stackrel.sty
%    TDS:doc/latex/oberdiek/stackrel.pdf
%    TDS:source/latex/oberdiek/stackrel.dtx
%
%<*ignore>
\begingroup
  \catcode123=1 %
  \catcode125=2 %
  \def\x{LaTeX2e}%
\expandafter\endgroup
\ifcase 0\ifx\install y1\fi\expandafter
         \ifx\csname processbatchFile\endcsname\relax\else1\fi
         \ifx\fmtname\x\else 1\fi\relax
\else\csname fi\endcsname
%</ignore>
%<*install>
\input docstrip.tex
\Msg{************************************************************************}
\Msg{* Installation}
\Msg{* Package: stackrel 2016/05/16 v1.3 Adding subscript option to stackrel (HO)}
\Msg{************************************************************************}

\keepsilent
\askforoverwritefalse

\let\MetaPrefix\relax
\preamble

This is a generated file.

Project: stackrel
Version: 2016/05/16 v1.3

Copyright (C) 2006, 2007 by
   Heiko Oberdiek <heiko.oberdiek at googlemail.com>

This work may be distributed and/or modified under the
conditions of the LaTeX Project Public License, either
version 1.3c of this license or (at your option) any later
version. This version of this license is in
   http://www.latex-project.org/lppl/lppl-1-3c.txt
and the latest version of this license is in
   http://www.latex-project.org/lppl.txt
and version 1.3 or later is part of all distributions of
LaTeX version 2005/12/01 or later.

This work has the LPPL maintenance status "maintained".

This Current Maintainer of this work is Heiko Oberdiek.

This work consists of the main source file stackrel.dtx
and the derived files
   stackrel.sty, stackrel.pdf, stackrel.ins, stackrel.drv.

\endpreamble
\let\MetaPrefix\DoubleperCent

\generate{%
  \file{stackrel.ins}{\from{stackrel.dtx}{install}}%
  \file{stackrel.drv}{\from{stackrel.dtx}{driver}}%
  \usedir{tex/latex/oberdiek}%
  \file{stackrel.sty}{\from{stackrel.dtx}{package}}%
  \nopreamble
  \nopostamble
%  \usedir{source/latex/oberdiek/catalogue}%
%  \file{stackrel.xml}{\from{stackrel.dtx}{catalogue}}%
}

\catcode32=13\relax% active space
\let =\space%
\Msg{************************************************************************}
\Msg{*}
\Msg{* To finish the installation you have to move the following}
\Msg{* file into a directory searched by TeX:}
\Msg{*}
\Msg{*     stackrel.sty}
\Msg{*}
\Msg{* To produce the documentation run the file `stackrel.drv'}
\Msg{* through LaTeX.}
\Msg{*}
\Msg{* Happy TeXing!}
\Msg{*}
\Msg{************************************************************************}

\endbatchfile
%</install>
%<*ignore>
\fi
%</ignore>
%<*driver>
\NeedsTeXFormat{LaTeX2e}
\ProvidesFile{stackrel.drv}%
  [2016/05/16 v1.3 Adding subscript option to stackrel (HO)]%
\documentclass{ltxdoc}
\usepackage{amsmath}
\usepackage{holtxdoc}[2011/11/22]
\usepackage{stackrel}[2016/05/16]
\begin{document}
  \DocInput{stackrel.dtx}%
\end{document}
%</driver>
% \fi
%
%
% \CharacterTable
%  {Upper-case    \A\B\C\D\E\F\G\H\I\J\K\L\M\N\O\P\Q\R\S\T\U\V\W\X\Y\Z
%   Lower-case    \a\b\c\d\e\f\g\h\i\j\k\l\m\n\o\p\q\r\s\t\u\v\w\x\y\z
%   Digits        \0\1\2\3\4\5\6\7\8\9
%   Exclamation   \!     Double quote  \"     Hash (number) \#
%   Dollar        \$     Percent       \%     Ampersand     \&
%   Acute accent  \'     Left paren    \(     Right paren   \)
%   Asterisk      \*     Plus          \+     Comma         \,
%   Minus         \-     Point         \.     Solidus       \/
%   Colon         \:     Semicolon     \;     Less than     \<
%   Equals        \=     Greater than  \>     Question mark \?
%   Commercial at \@     Left bracket  \[     Backslash     \\
%   Right bracket \]     Circumflex    \^     Underscore    \_
%   Grave accent  \`     Left brace    \{     Vertical bar  \|
%   Right brace   \}     Tilde         \~}
%
% \GetFileInfo{stackrel.drv}
%
% \title{The \xpackage{stackrel} package}
% \date{2016/05/16 v1.3}
% \author{Heiko Oberdiek\thanks
% {Please report any issues at \url{https://github.com/ho-tex/oberdiek/issues}}\\
% \xemail{heiko.oberdiek at googlemail.com}}
%
% \maketitle
%
% \begin{abstract}
% This package adds an optional argument to \cs{stackrel} for
% putting something below the relational symbol and defines
% \cs{stackbin} for binary symbols.
% \end{abstract}
%
% \tableofcontents
%
% \section{User interface}
%
% \LaTeX's \cs{stackrel} allows a superscript above a relational symbol,
% but pure \LaTeX\ does not provide a macro for putting a subscript
% below the symbol. This is supported by \AmS\LaTeX's \cs{underset}
% macro that works on both relational and binary symbols. A combination
% of \cs{underset} and \cs{overset} can be used to put \mbox{sub-} and
% superscripts to the same symbol.
%
% This package \xpackage{stackrel} extends the syntax of \cs{stackrel}
% by adding an optional argument for the subscript position.
% It follows the syntax of extensible arrows of packages
% \xpackage{amsmath} and \xpackage{mathtools}.
%
% \begin{declcs}{stackrel}
%   |[|\meta{subscript}|]| \M{superscript} \M{rel}\\
%   \cs{stackbin}
%   |[|\meta{subscript}|]| \M{superscript} \M{bin}
% \end{declcs}
% Example:
% \begin{quote}
% |A \stackbin[\text{and}]{}{+} B \stackrel[x]{!}{=} C|\\
% $A \stackbin[\text{and}]{}{+} B \stackrel[x]{!}{=} C$
% \end{quote}
%
% \StopEventually{
% }
%
% \section{Implementation}
%
%    \begin{macrocode}
%<*package>
\NeedsTeXFormat{LaTeX2e}
\ProvidesPackage{stackrel}
  [2016/05/16 v1.3 Adding subscript option to stackrel (HO)]%
%    \end{macrocode}
%
%    Given the original definition of \cs{stackrel} the addition
%    of the optional argument is straightforward. If an argument
%    is empty, then the corresponding sub- or superscript is
%    suppressed.
%
%    Depending on the available resources (\eTeX, \pdfTeX)
%    three methods are given for testing emptyness. All tests
%    allow the hash to be used inside the arguments without
%    doubling (for the unlikely case that someone wants to
%    define macros with arguments).
%    \begin{macro}{\stack@relbin}
%    \begin{macrocode}
\RequirePackage{etexcmds}[2007/09/09]
\ifetex@unexpanded
  \RequirePackage{pdftexcmds}[2016/05/16]%
  \begingroup\expandafter\expandafter\expandafter\endgroup
  \expandafter\ifx\csname pdf@strcmp\endcsname\relax
    \newcommand*{\stack@relbin}[3][]{%
      \mathop{#3}\limits
      \edef\reserved@a{\etex@unexpanded{#1}}%
      \ifx\reserved@a\@empty\else_{#1}\fi
      \edef\reserved@a{\etex@unexpanded{#2}}%
      \ifx\reserved@a\@empty\else^{#2}\fi
      \egroup
    }%
  \else
    \newcommand*{\stack@relbin}[3][]{%
      \mathop{#3}\limits
      \ifcase\pdf@strcmp{\detokenize{#1}}{}\else_{#1}\fi
      \ifcase\pdf@strcmp{\detokenize{#2}}{}\else^{#2}\fi
      \egroup
    }%
  \fi
\else
  \newcommand*{\stack@relbin}[3][]{%
    \mathop{#3}\limits
    \toks@{#1}%
    \edef\reserved@a{\the\toks@}%
    \ifx\reserved@a\@empty\else_{#1}\fi
    \toks@{#2}%
    \edef\reserved@a{\the\toks@}%
    \ifx\reserved@a\@empty\else^{#2}\fi
    \egroup
  }%
\fi
%    \end{macrocode}
%    \end{macro}
%    \begin{macro}{\stackrel}
%    \begin{macrocode}
\renewcommand*{\stackrel}{%
  \mathrel\bgroup\stack@relbin
}
%    \end{macrocode}
%    \end{macro}
%    \begin{macro}{\stackbin}
%    \begin{macrocode}
\newcommand*{\stackbin}{%
  \mathbin\bgroup\stack@relbin
}
%    \end{macrocode}
%    \end{macro}
%
%    \begin{macrocode}
%</package>
%    \end{macrocode}
%
% \section{Installation}
%
% \subsection{Download}
%
% \paragraph{Package.} This package is available on
% CTAN\footnote{\CTANpkg{stackrel}}:
% \begin{description}
% \item[\CTAN{macros/latex/contrib/oberdiek/stackrel.dtx}] The source file.
% \item[\CTAN{macros/latex/contrib/oberdiek/stackrel.pdf}] Documentation.
% \end{description}
%
%
% \paragraph{Bundle.} All the packages of the bundle `oberdiek'
% are also available in a TDS compliant ZIP archive. There
% the packages are already unpacked and the documentation files
% are generated. The files and directories obey the TDS standard.
% \begin{description}
% \item[\CTANinstall{install/macros/latex/contrib/oberdiek.tds.zip}]
% \end{description}
% \emph{TDS} refers to the standard ``A Directory Structure
% for \TeX\ Files'' (\CTAN{tds/tds.pdf}). Directories
% with \xfile{texmf} in their name are usually organized this way.
%
% \subsection{Bundle installation}
%
% \paragraph{Unpacking.} Unpack the \xfile{oberdiek.tds.zip} in the
% TDS tree (also known as \xfile{texmf} tree) of your choice.
% Example (linux):
% \begin{quote}
%   |unzip oberdiek.tds.zip -d ~/texmf|
% \end{quote}
%
% \paragraph{Script installation.}
% Check the directory \xfile{TDS:scripts/oberdiek/} for
% scripts that need further installation steps.
% Package \xpackage{attachfile2} comes with the Perl script
% \xfile{pdfatfi.pl} that should be installed in such a way
% that it can be called as \texttt{pdfatfi}.
% Example (linux):
% \begin{quote}
%   |chmod +x scripts/oberdiek/pdfatfi.pl|\\
%   |cp scripts/oberdiek/pdfatfi.pl /usr/local/bin/|
% \end{quote}
%
% \subsection{Package installation}
%
% \paragraph{Unpacking.} The \xfile{.dtx} file is a self-extracting
% \docstrip\ archive. The files are extracted by running the
% \xfile{.dtx} through \plainTeX:
% \begin{quote}
%   \verb|tex stackrel.dtx|
% \end{quote}
%
% \paragraph{TDS.} Now the different files must be moved into
% the different directories in your installation TDS tree
% (also known as \xfile{texmf} tree):
% \begin{quote}
% \def\t{^^A
% \begin{tabular}{@{}>{\ttfamily}l@{ $\rightarrow$ }>{\ttfamily}l@{}}
%   stackrel.sty & tex/latex/oberdiek/stackrel.sty\\
%   stackrel.pdf & doc/latex/oberdiek/stackrel.pdf\\
%   stackrel.dtx & source/latex/oberdiek/stackrel.dtx\\
% \end{tabular}^^A
% }^^A
% \sbox0{\t}^^A
% \ifdim\wd0>\linewidth
%   \begingroup
%     \advance\linewidth by\leftmargin
%     \advance\linewidth by\rightmargin
%   \edef\x{\endgroup
%     \def\noexpand\lw{\the\linewidth}^^A
%   }\x
%   \def\lwbox{^^A
%     \leavevmode
%     \hbox to \linewidth{^^A
%       \kern-\leftmargin\relax
%       \hss
%       \usebox0
%       \hss
%       \kern-\rightmargin\relax
%     }^^A
%   }^^A
%   \ifdim\wd0>\lw
%     \sbox0{\small\t}^^A
%     \ifdim\wd0>\linewidth
%       \ifdim\wd0>\lw
%         \sbox0{\footnotesize\t}^^A
%         \ifdim\wd0>\linewidth
%           \ifdim\wd0>\lw
%             \sbox0{\scriptsize\t}^^A
%             \ifdim\wd0>\linewidth
%               \ifdim\wd0>\lw
%                 \sbox0{\tiny\t}^^A
%                 \ifdim\wd0>\linewidth
%                   \lwbox
%                 \else
%                   \usebox0
%                 \fi
%               \else
%                 \lwbox
%               \fi
%             \else
%               \usebox0
%             \fi
%           \else
%             \lwbox
%           \fi
%         \else
%           \usebox0
%         \fi
%       \else
%         \lwbox
%       \fi
%     \else
%       \usebox0
%     \fi
%   \else
%     \lwbox
%   \fi
% \else
%   \usebox0
% \fi
% \end{quote}
% If you have a \xfile{docstrip.cfg} that configures and enables \docstrip's
% TDS installing feature, then some files can already be in the right
% place, see the documentation of \docstrip.
%
% \subsection{Refresh file name databases}
%
% If your \TeX~distribution
% (\teTeX, \mikTeX, \dots) relies on file name databases, you must refresh
% these. For example, \teTeX\ users run \verb|texhash| or
% \verb|mktexlsr|.
%
% \subsection{Some details for the interested}
%
% \paragraph{Attached source.}
%
% The PDF documentation on CTAN also includes the
% \xfile{.dtx} source file. It can be extracted by
% AcrobatReader 6 or higher. Another option is \textsf{pdftk},
% e.g. unpack the file into the current directory:
% \begin{quote}
%   \verb|pdftk stackrel.pdf unpack_files output .|
% \end{quote}
%
% \paragraph{Unpacking with \LaTeX.}
% The \xfile{.dtx} chooses its action depending on the format:
% \begin{description}
% \item[\plainTeX:] Run \docstrip\ and extract the files.
% \item[\LaTeX:] Generate the documentation.
% \end{description}
% If you insist on using \LaTeX\ for \docstrip\ (really,
% \docstrip\ does not need \LaTeX), then inform the autodetect routine
% about your intention:
% \begin{quote}
%   \verb|latex \let\install=y% \iffalse meta-comment
%
% File: stackrel.dtx
% Version: 2016/05/16 v1.3
% Info: Adding subscript option to stackrel
%
% Copyright (C) 2006, 2007 by
%    Heiko Oberdiek <heiko.oberdiek at googlemail.com>
%    2016
%    https://github.com/ho-tex/oberdiek/issues
%
% This work may be distributed and/or modified under the
% conditions of the LaTeX Project Public License, either
% version 1.3c of this license or (at your option) any later
% version. This version of this license is in
%    http://www.latex-project.org/lppl/lppl-1-3c.txt
% and the latest version of this license is in
%    http://www.latex-project.org/lppl.txt
% and version 1.3 or later is part of all distributions of
% LaTeX version 2005/12/01 or later.
%
% This work has the LPPL maintenance status "maintained".
%
% This Current Maintainer of this work is Heiko Oberdiek.
%
% This work consists of the main source file stackrel.dtx
% and the derived files
%    stackrel.sty, stackrel.pdf, stackrel.ins, stackrel.drv.
%
% Distribution:
%    CTAN:macros/latex/contrib/oberdiek/stackrel.dtx
%    CTAN:macros/latex/contrib/oberdiek/stackrel.pdf
%
% Unpacking:
%    (a) If stackrel.ins is present:
%           tex stackrel.ins
%    (b) Without stackrel.ins:
%           tex stackrel.dtx
%    (c) If you insist on using LaTeX
%           latex \let\install=y\input{stackrel.dtx}
%        (quote the arguments according to the demands of your shell)
%
% Documentation:
%    (a) If stackrel.drv is present:
%           latex stackrel.drv
%    (b) Without stackrel.drv:
%           latex stackrel.dtx; ...
%    The class ltxdoc loads the configuration file ltxdoc.cfg
%    if available. Here you can specify further options, e.g.
%    use A4 as paper format:
%       \PassOptionsToClass{a4paper}{article}
%
%    Programm calls to get the documentation (example):
%       pdflatex stackrel.dtx
%       makeindex -s gind.ist stackrel.idx
%       pdflatex stackrel.dtx
%       makeindex -s gind.ist stackrel.idx
%       pdflatex stackrel.dtx
%
% Installation:
%    TDS:tex/latex/oberdiek/stackrel.sty
%    TDS:doc/latex/oberdiek/stackrel.pdf
%    TDS:source/latex/oberdiek/stackrel.dtx
%
%<*ignore>
\begingroup
  \catcode123=1 %
  \catcode125=2 %
  \def\x{LaTeX2e}%
\expandafter\endgroup
\ifcase 0\ifx\install y1\fi\expandafter
         \ifx\csname processbatchFile\endcsname\relax\else1\fi
         \ifx\fmtname\x\else 1\fi\relax
\else\csname fi\endcsname
%</ignore>
%<*install>
\input docstrip.tex
\Msg{************************************************************************}
\Msg{* Installation}
\Msg{* Package: stackrel 2016/05/16 v1.3 Adding subscript option to stackrel (HO)}
\Msg{************************************************************************}

\keepsilent
\askforoverwritefalse

\let\MetaPrefix\relax
\preamble

This is a generated file.

Project: stackrel
Version: 2016/05/16 v1.3

Copyright (C) 2006, 2007 by
   Heiko Oberdiek <heiko.oberdiek at googlemail.com>

This work may be distributed and/or modified under the
conditions of the LaTeX Project Public License, either
version 1.3c of this license or (at your option) any later
version. This version of this license is in
   http://www.latex-project.org/lppl/lppl-1-3c.txt
and the latest version of this license is in
   http://www.latex-project.org/lppl.txt
and version 1.3 or later is part of all distributions of
LaTeX version 2005/12/01 or later.

This work has the LPPL maintenance status "maintained".

This Current Maintainer of this work is Heiko Oberdiek.

This work consists of the main source file stackrel.dtx
and the derived files
   stackrel.sty, stackrel.pdf, stackrel.ins, stackrel.drv.

\endpreamble
\let\MetaPrefix\DoubleperCent

\generate{%
  \file{stackrel.ins}{\from{stackrel.dtx}{install}}%
  \file{stackrel.drv}{\from{stackrel.dtx}{driver}}%
  \usedir{tex/latex/oberdiek}%
  \file{stackrel.sty}{\from{stackrel.dtx}{package}}%
  \nopreamble
  \nopostamble
%  \usedir{source/latex/oberdiek/catalogue}%
%  \file{stackrel.xml}{\from{stackrel.dtx}{catalogue}}%
}

\catcode32=13\relax% active space
\let =\space%
\Msg{************************************************************************}
\Msg{*}
\Msg{* To finish the installation you have to move the following}
\Msg{* file into a directory searched by TeX:}
\Msg{*}
\Msg{*     stackrel.sty}
\Msg{*}
\Msg{* To produce the documentation run the file `stackrel.drv'}
\Msg{* through LaTeX.}
\Msg{*}
\Msg{* Happy TeXing!}
\Msg{*}
\Msg{************************************************************************}

\endbatchfile
%</install>
%<*ignore>
\fi
%</ignore>
%<*driver>
\NeedsTeXFormat{LaTeX2e}
\ProvidesFile{stackrel.drv}%
  [2016/05/16 v1.3 Adding subscript option to stackrel (HO)]%
\documentclass{ltxdoc}
\usepackage{amsmath}
\usepackage{holtxdoc}[2011/11/22]
\usepackage{stackrel}[2016/05/16]
\begin{document}
  \DocInput{stackrel.dtx}%
\end{document}
%</driver>
% \fi
%
%
% \CharacterTable
%  {Upper-case    \A\B\C\D\E\F\G\H\I\J\K\L\M\N\O\P\Q\R\S\T\U\V\W\X\Y\Z
%   Lower-case    \a\b\c\d\e\f\g\h\i\j\k\l\m\n\o\p\q\r\s\t\u\v\w\x\y\z
%   Digits        \0\1\2\3\4\5\6\7\8\9
%   Exclamation   \!     Double quote  \"     Hash (number) \#
%   Dollar        \$     Percent       \%     Ampersand     \&
%   Acute accent  \'     Left paren    \(     Right paren   \)
%   Asterisk      \*     Plus          \+     Comma         \,
%   Minus         \-     Point         \.     Solidus       \/
%   Colon         \:     Semicolon     \;     Less than     \<
%   Equals        \=     Greater than  \>     Question mark \?
%   Commercial at \@     Left bracket  \[     Backslash     \\
%   Right bracket \]     Circumflex    \^     Underscore    \_
%   Grave accent  \`     Left brace    \{     Vertical bar  \|
%   Right brace   \}     Tilde         \~}
%
% \GetFileInfo{stackrel.drv}
%
% \title{The \xpackage{stackrel} package}
% \date{2016/05/16 v1.3}
% \author{Heiko Oberdiek\thanks
% {Please report any issues at \url{https://github.com/ho-tex/oberdiek/issues}}\\
% \xemail{heiko.oberdiek at googlemail.com}}
%
% \maketitle
%
% \begin{abstract}
% This package adds an optional argument to \cs{stackrel} for
% putting something below the relational symbol and defines
% \cs{stackbin} for binary symbols.
% \end{abstract}
%
% \tableofcontents
%
% \section{User interface}
%
% \LaTeX's \cs{stackrel} allows a superscript above a relational symbol,
% but pure \LaTeX\ does not provide a macro for putting a subscript
% below the symbol. This is supported by \AmS\LaTeX's \cs{underset}
% macro that works on both relational and binary symbols. A combination
% of \cs{underset} and \cs{overset} can be used to put \mbox{sub-} and
% superscripts to the same symbol.
%
% This package \xpackage{stackrel} extends the syntax of \cs{stackrel}
% by adding an optional argument for the subscript position.
% It follows the syntax of extensible arrows of packages
% \xpackage{amsmath} and \xpackage{mathtools}.
%
% \begin{declcs}{stackrel}
%   |[|\meta{subscript}|]| \M{superscript} \M{rel}\\
%   \cs{stackbin}
%   |[|\meta{subscript}|]| \M{superscript} \M{bin}
% \end{declcs}
% Example:
% \begin{quote}
% |A \stackbin[\text{and}]{}{+} B \stackrel[x]{!}{=} C|\\
% $A \stackbin[\text{and}]{}{+} B \stackrel[x]{!}{=} C$
% \end{quote}
%
% \StopEventually{
% }
%
% \section{Implementation}
%
%    \begin{macrocode}
%<*package>
\NeedsTeXFormat{LaTeX2e}
\ProvidesPackage{stackrel}
  [2016/05/16 v1.3 Adding subscript option to stackrel (HO)]%
%    \end{macrocode}
%
%    Given the original definition of \cs{stackrel} the addition
%    of the optional argument is straightforward. If an argument
%    is empty, then the corresponding sub- or superscript is
%    suppressed.
%
%    Depending on the available resources (\eTeX, \pdfTeX)
%    three methods are given for testing emptyness. All tests
%    allow the hash to be used inside the arguments without
%    doubling (for the unlikely case that someone wants to
%    define macros with arguments).
%    \begin{macro}{\stack@relbin}
%    \begin{macrocode}
\RequirePackage{etexcmds}[2007/09/09]
\ifetex@unexpanded
  \RequirePackage{pdftexcmds}[2016/05/16]%
  \begingroup\expandafter\expandafter\expandafter\endgroup
  \expandafter\ifx\csname pdf@strcmp\endcsname\relax
    \newcommand*{\stack@relbin}[3][]{%
      \mathop{#3}\limits
      \edef\reserved@a{\etex@unexpanded{#1}}%
      \ifx\reserved@a\@empty\else_{#1}\fi
      \edef\reserved@a{\etex@unexpanded{#2}}%
      \ifx\reserved@a\@empty\else^{#2}\fi
      \egroup
    }%
  \else
    \newcommand*{\stack@relbin}[3][]{%
      \mathop{#3}\limits
      \ifcase\pdf@strcmp{\detokenize{#1}}{}\else_{#1}\fi
      \ifcase\pdf@strcmp{\detokenize{#2}}{}\else^{#2}\fi
      \egroup
    }%
  \fi
\else
  \newcommand*{\stack@relbin}[3][]{%
    \mathop{#3}\limits
    \toks@{#1}%
    \edef\reserved@a{\the\toks@}%
    \ifx\reserved@a\@empty\else_{#1}\fi
    \toks@{#2}%
    \edef\reserved@a{\the\toks@}%
    \ifx\reserved@a\@empty\else^{#2}\fi
    \egroup
  }%
\fi
%    \end{macrocode}
%    \end{macro}
%    \begin{macro}{\stackrel}
%    \begin{macrocode}
\renewcommand*{\stackrel}{%
  \mathrel\bgroup\stack@relbin
}
%    \end{macrocode}
%    \end{macro}
%    \begin{macro}{\stackbin}
%    \begin{macrocode}
\newcommand*{\stackbin}{%
  \mathbin\bgroup\stack@relbin
}
%    \end{macrocode}
%    \end{macro}
%
%    \begin{macrocode}
%</package>
%    \end{macrocode}
%
% \section{Installation}
%
% \subsection{Download}
%
% \paragraph{Package.} This package is available on
% CTAN\footnote{\CTANpkg{stackrel}}:
% \begin{description}
% \item[\CTAN{macros/latex/contrib/oberdiek/stackrel.dtx}] The source file.
% \item[\CTAN{macros/latex/contrib/oberdiek/stackrel.pdf}] Documentation.
% \end{description}
%
%
% \paragraph{Bundle.} All the packages of the bundle `oberdiek'
% are also available in a TDS compliant ZIP archive. There
% the packages are already unpacked and the documentation files
% are generated. The files and directories obey the TDS standard.
% \begin{description}
% \item[\CTANinstall{install/macros/latex/contrib/oberdiek.tds.zip}]
% \end{description}
% \emph{TDS} refers to the standard ``A Directory Structure
% for \TeX\ Files'' (\CTAN{tds/tds.pdf}). Directories
% with \xfile{texmf} in their name are usually organized this way.
%
% \subsection{Bundle installation}
%
% \paragraph{Unpacking.} Unpack the \xfile{oberdiek.tds.zip} in the
% TDS tree (also known as \xfile{texmf} tree) of your choice.
% Example (linux):
% \begin{quote}
%   |unzip oberdiek.tds.zip -d ~/texmf|
% \end{quote}
%
% \paragraph{Script installation.}
% Check the directory \xfile{TDS:scripts/oberdiek/} for
% scripts that need further installation steps.
% Package \xpackage{attachfile2} comes with the Perl script
% \xfile{pdfatfi.pl} that should be installed in such a way
% that it can be called as \texttt{pdfatfi}.
% Example (linux):
% \begin{quote}
%   |chmod +x scripts/oberdiek/pdfatfi.pl|\\
%   |cp scripts/oberdiek/pdfatfi.pl /usr/local/bin/|
% \end{quote}
%
% \subsection{Package installation}
%
% \paragraph{Unpacking.} The \xfile{.dtx} file is a self-extracting
% \docstrip\ archive. The files are extracted by running the
% \xfile{.dtx} through \plainTeX:
% \begin{quote}
%   \verb|tex stackrel.dtx|
% \end{quote}
%
% \paragraph{TDS.} Now the different files must be moved into
% the different directories in your installation TDS tree
% (also known as \xfile{texmf} tree):
% \begin{quote}
% \def\t{^^A
% \begin{tabular}{@{}>{\ttfamily}l@{ $\rightarrow$ }>{\ttfamily}l@{}}
%   stackrel.sty & tex/latex/oberdiek/stackrel.sty\\
%   stackrel.pdf & doc/latex/oberdiek/stackrel.pdf\\
%   stackrel.dtx & source/latex/oberdiek/stackrel.dtx\\
% \end{tabular}^^A
% }^^A
% \sbox0{\t}^^A
% \ifdim\wd0>\linewidth
%   \begingroup
%     \advance\linewidth by\leftmargin
%     \advance\linewidth by\rightmargin
%   \edef\x{\endgroup
%     \def\noexpand\lw{\the\linewidth}^^A
%   }\x
%   \def\lwbox{^^A
%     \leavevmode
%     \hbox to \linewidth{^^A
%       \kern-\leftmargin\relax
%       \hss
%       \usebox0
%       \hss
%       \kern-\rightmargin\relax
%     }^^A
%   }^^A
%   \ifdim\wd0>\lw
%     \sbox0{\small\t}^^A
%     \ifdim\wd0>\linewidth
%       \ifdim\wd0>\lw
%         \sbox0{\footnotesize\t}^^A
%         \ifdim\wd0>\linewidth
%           \ifdim\wd0>\lw
%             \sbox0{\scriptsize\t}^^A
%             \ifdim\wd0>\linewidth
%               \ifdim\wd0>\lw
%                 \sbox0{\tiny\t}^^A
%                 \ifdim\wd0>\linewidth
%                   \lwbox
%                 \else
%                   \usebox0
%                 \fi
%               \else
%                 \lwbox
%               \fi
%             \else
%               \usebox0
%             \fi
%           \else
%             \lwbox
%           \fi
%         \else
%           \usebox0
%         \fi
%       \else
%         \lwbox
%       \fi
%     \else
%       \usebox0
%     \fi
%   \else
%     \lwbox
%   \fi
% \else
%   \usebox0
% \fi
% \end{quote}
% If you have a \xfile{docstrip.cfg} that configures and enables \docstrip's
% TDS installing feature, then some files can already be in the right
% place, see the documentation of \docstrip.
%
% \subsection{Refresh file name databases}
%
% If your \TeX~distribution
% (\teTeX, \mikTeX, \dots) relies on file name databases, you must refresh
% these. For example, \teTeX\ users run \verb|texhash| or
% \verb|mktexlsr|.
%
% \subsection{Some details for the interested}
%
% \paragraph{Attached source.}
%
% The PDF documentation on CTAN also includes the
% \xfile{.dtx} source file. It can be extracted by
% AcrobatReader 6 or higher. Another option is \textsf{pdftk},
% e.g. unpack the file into the current directory:
% \begin{quote}
%   \verb|pdftk stackrel.pdf unpack_files output .|
% \end{quote}
%
% \paragraph{Unpacking with \LaTeX.}
% The \xfile{.dtx} chooses its action depending on the format:
% \begin{description}
% \item[\plainTeX:] Run \docstrip\ and extract the files.
% \item[\LaTeX:] Generate the documentation.
% \end{description}
% If you insist on using \LaTeX\ for \docstrip\ (really,
% \docstrip\ does not need \LaTeX), then inform the autodetect routine
% about your intention:
% \begin{quote}
%   \verb|latex \let\install=y\input{stackrel.dtx}|
% \end{quote}
% Do not forget to quote the argument according to the demands
% of your shell.
%
% \paragraph{Generating the documentation.}
% You can use both the \xfile{.dtx} or the \xfile{.drv} to generate
% the documentation. The process can be configured by the
% configuration file \xfile{ltxdoc.cfg}. For instance, put this
% line into this file, if you want to have A4 as paper format:
% \begin{quote}
%   \verb|\PassOptionsToClass{a4paper}{article}|
% \end{quote}
% An example follows how to generate the
% documentation with pdf\LaTeX:
% \begin{quote}
%\begin{verbatim}
%pdflatex stackrel.dtx
%makeindex -s gind.ist stackrel.idx
%pdflatex stackrel.dtx
%makeindex -s gind.ist stackrel.idx
%pdflatex stackrel.dtx
%\end{verbatim}
% \end{quote}
%
% \begin{History}
%   \begin{Version}{2006/12/02 v1.0}
%   \item
%     First version.
%   \end{Version}
%   \begin{Version}{2007/05/06 v1.1}
%   \item
%     Uses package \xpackage{etexcmds}.
%   \end{Version}
%   \begin{Version}{2007/11/11 v1.2}
%   \item
%     Use of package \xpackage{pdftexcmds} for \LuaTeX\ support.
%   \end{Version}
%   \begin{Version}{2016/05/16 v1.3}
%   \item
%     Documentation updates.
%   \end{Version}
% \end{History}
%
% \clearpage
% \PrintIndex
%
% \Finale
\endinput
|
% \end{quote}
% Do not forget to quote the argument according to the demands
% of your shell.
%
% \paragraph{Generating the documentation.}
% You can use both the \xfile{.dtx} or the \xfile{.drv} to generate
% the documentation. The process can be configured by the
% configuration file \xfile{ltxdoc.cfg}. For instance, put this
% line into this file, if you want to have A4 as paper format:
% \begin{quote}
%   \verb|\PassOptionsToClass{a4paper}{article}|
% \end{quote}
% An example follows how to generate the
% documentation with pdf\LaTeX:
% \begin{quote}
%\begin{verbatim}
%pdflatex stackrel.dtx
%makeindex -s gind.ist stackrel.idx
%pdflatex stackrel.dtx
%makeindex -s gind.ist stackrel.idx
%pdflatex stackrel.dtx
%\end{verbatim}
% \end{quote}
%
% \begin{History}
%   \begin{Version}{2006/12/02 v1.0}
%   \item
%     First version.
%   \end{Version}
%   \begin{Version}{2007/05/06 v1.1}
%   \item
%     Uses package \xpackage{etexcmds}.
%   \end{Version}
%   \begin{Version}{2007/11/11 v1.2}
%   \item
%     Use of package \xpackage{pdftexcmds} for \LuaTeX\ support.
%   \end{Version}
%   \begin{Version}{2016/05/16 v1.3}
%   \item
%     Documentation updates.
%   \end{Version}
% \end{History}
%
% \clearpage
% \PrintIndex
%
% \Finale
\endinput
|
% \end{quote}
% Do not forget to quote the argument according to the demands
% of your shell.
%
% \paragraph{Generating the documentation.}
% You can use both the \xfile{.dtx} or the \xfile{.drv} to generate
% the documentation. The process can be configured by the
% configuration file \xfile{ltxdoc.cfg}. For instance, put this
% line into this file, if you want to have A4 as paper format:
% \begin{quote}
%   \verb|\PassOptionsToClass{a4paper}{article}|
% \end{quote}
% An example follows how to generate the
% documentation with pdf\LaTeX:
% \begin{quote}
%\begin{verbatim}
%pdflatex stackrel.dtx
%makeindex -s gind.ist stackrel.idx
%pdflatex stackrel.dtx
%makeindex -s gind.ist stackrel.idx
%pdflatex stackrel.dtx
%\end{verbatim}
% \end{quote}
%
% \begin{History}
%   \begin{Version}{2006/12/02 v1.0}
%   \item
%     First version.
%   \end{Version}
%   \begin{Version}{2007/05/06 v1.1}
%   \item
%     Uses package \xpackage{etexcmds}.
%   \end{Version}
%   \begin{Version}{2007/11/11 v1.2}
%   \item
%     Use of package \xpackage{pdftexcmds} for \LuaTeX\ support.
%   \end{Version}
%   \begin{Version}{2016/05/16 v1.3}
%   \item
%     Documentation updates.
%   \end{Version}
% \end{History}
%
% \clearpage
% \PrintIndex
%
% \Finale
\endinput
|
% \end{quote}
% Do not forget to quote the argument according to the demands
% of your shell.
%
% \paragraph{Generating the documentation.}
% You can use both the \xfile{.dtx} or the \xfile{.drv} to generate
% the documentation. The process can be configured by the
% configuration file \xfile{ltxdoc.cfg}. For instance, put this
% line into this file, if you want to have A4 as paper format:
% \begin{quote}
%   \verb|\PassOptionsToClass{a4paper}{article}|
% \end{quote}
% An example follows how to generate the
% documentation with pdf\LaTeX:
% \begin{quote}
%\begin{verbatim}
%pdflatex stackrel.dtx
%makeindex -s gind.ist stackrel.idx
%pdflatex stackrel.dtx
%makeindex -s gind.ist stackrel.idx
%pdflatex stackrel.dtx
%\end{verbatim}
% \end{quote}
%
% \begin{History}
%   \begin{Version}{2006/12/02 v1.0}
%   \item
%     First version.
%   \end{Version}
%   \begin{Version}{2007/05/06 v1.1}
%   \item
%     Uses package \xpackage{etexcmds}.
%   \end{Version}
%   \begin{Version}{2007/11/11 v1.2}
%   \item
%     Use of package \xpackage{pdftexcmds} for \LuaTeX\ support.
%   \end{Version}
%   \begin{Version}{2016/05/16 v1.3}
%   \item
%     Documentation updates.
%   \end{Version}
% \end{History}
%
% \clearpage
% \PrintIndex
%
% \Finale
\endinput
