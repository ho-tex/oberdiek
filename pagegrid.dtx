% \iffalse meta-comment
%
% File: pagegrid.dtx
% Version: 2016/05/16 v1.5
% Info: Print page grid in background
%
% Copyright (C) 2009 by
%    Heiko Oberdiek <heiko.oberdiek at googlemail.com>
%    2016
%    https://github.com/ho-tex/oberdiek/issues
%
% This work may be distributed and/or modified under the
% conditions of the LaTeX Project Public License, either
% version 1.3c of this license or (at your option) any later
% version. This version of this license is in
%    http://www.latex-project.org/lppl/lppl-1-3c.txt
% and the latest version of this license is in
%    http://www.latex-project.org/lppl.txt
% and version 1.3 or later is part of all distributions of
% LaTeX version 2005/12/01 or later.
%
% This work has the LPPL maintenance status "maintained".
%
% This Current Maintainer of this work is Heiko Oberdiek.
%
% This work consists of the main source file pagegrid.dtx
% and the derived files
%    pagegrid.sty, pagegrid.pdf, pagegrid.ins, pagegrid.drv,
%    pagegrid-test1.tex.
%
% Distribution:
%    CTAN:macros/latex/contrib/oberdiek/pagegrid.dtx
%    CTAN:macros/latex/contrib/oberdiek/pagegrid.pdf
%
% Unpacking:
%    (a) If pagegrid.ins is present:
%           tex pagegrid.ins
%    (b) Without pagegrid.ins:
%           tex pagegrid.dtx
%    (c) If you insist on using LaTeX
%           latex \let\install=y% \iffalse meta-comment
%
% File: pagegrid.dtx
% Version: 2016/05/16 v1.5
% Info: Print page grid in background
%
% Copyright (C) 2009 by
%    Heiko Oberdiek <heiko.oberdiek at googlemail.com>
%    2016
%    https://github.com/ho-tex/oberdiek/issues
%
% This work may be distributed and/or modified under the
% conditions of the LaTeX Project Public License, either
% version 1.3c of this license or (at your option) any later
% version. This version of this license is in
%    https://www.latex-project.org/lppl/lppl-1-3c.txt
% and the latest version of this license is in
%    https://www.latex-project.org/lppl.txt
% and version 1.3 or later is part of all distributions of
% LaTeX version 2005/12/01 or later.
%
% This work has the LPPL maintenance status "maintained".
%
% The Current Maintainers of this work are
% Heiko Oberdiek and the Oberdiek Package Support Group
% https://github.com/ho-tex/oberdiek/issues
%
% This work consists of the main source file pagegrid.dtx
% and the derived files
%    pagegrid.sty, pagegrid.pdf, pagegrid.ins, pagegrid.drv,
%    pagegrid-test1.tex.
%
% Distribution:
%    CTAN:macros/latex/contrib/oberdiek/pagegrid.dtx
%    CTAN:macros/latex/contrib/oberdiek/pagegrid.pdf
%
% Unpacking:
%    (a) If pagegrid.ins is present:
%           tex pagegrid.ins
%    (b) Without pagegrid.ins:
%           tex pagegrid.dtx
%    (c) If you insist on using LaTeX
%           latex \let\install=y% \iffalse meta-comment
%
% File: pagegrid.dtx
% Version: 2016/05/16 v1.5
% Info: Print page grid in background
%
% Copyright (C) 2009 by
%    Heiko Oberdiek <heiko.oberdiek at googlemail.com>
%    2016
%    https://github.com/ho-tex/oberdiek/issues
%
% This work may be distributed and/or modified under the
% conditions of the LaTeX Project Public License, either
% version 1.3c of this license or (at your option) any later
% version. This version of this license is in
%    https://www.latex-project.org/lppl/lppl-1-3c.txt
% and the latest version of this license is in
%    https://www.latex-project.org/lppl.txt
% and version 1.3 or later is part of all distributions of
% LaTeX version 2005/12/01 or later.
%
% This work has the LPPL maintenance status "maintained".
%
% The Current Maintainers of this work are
% Heiko Oberdiek and the Oberdiek Package Support Group
% https://github.com/ho-tex/oberdiek/issues
%
% This work consists of the main source file pagegrid.dtx
% and the derived files
%    pagegrid.sty, pagegrid.pdf, pagegrid.ins, pagegrid.drv,
%    pagegrid-test1.tex.
%
% Distribution:
%    CTAN:macros/latex/contrib/oberdiek/pagegrid.dtx
%    CTAN:macros/latex/contrib/oberdiek/pagegrid.pdf
%
% Unpacking:
%    (a) If pagegrid.ins is present:
%           tex pagegrid.ins
%    (b) Without pagegrid.ins:
%           tex pagegrid.dtx
%    (c) If you insist on using LaTeX
%           latex \let\install=y% \iffalse meta-comment
%
% File: pagegrid.dtx
% Version: 2016/05/16 v1.5
% Info: Print page grid in background
%
% Copyright (C) 2009 by
%    Heiko Oberdiek <heiko.oberdiek at googlemail.com>
%    2016
%    https://github.com/ho-tex/oberdiek/issues
%
% This work may be distributed and/or modified under the
% conditions of the LaTeX Project Public License, either
% version 1.3c of this license or (at your option) any later
% version. This version of this license is in
%    https://www.latex-project.org/lppl/lppl-1-3c.txt
% and the latest version of this license is in
%    https://www.latex-project.org/lppl.txt
% and version 1.3 or later is part of all distributions of
% LaTeX version 2005/12/01 or later.
%
% This work has the LPPL maintenance status "maintained".
%
% The Current Maintainers of this work are
% Heiko Oberdiek and the Oberdiek Package Support Group
% https://github.com/ho-tex/oberdiek/issues
%
% This work consists of the main source file pagegrid.dtx
% and the derived files
%    pagegrid.sty, pagegrid.pdf, pagegrid.ins, pagegrid.drv,
%    pagegrid-test1.tex.
%
% Distribution:
%    CTAN:macros/latex/contrib/oberdiek/pagegrid.dtx
%    CTAN:macros/latex/contrib/oberdiek/pagegrid.pdf
%
% Unpacking:
%    (a) If pagegrid.ins is present:
%           tex pagegrid.ins
%    (b) Without pagegrid.ins:
%           tex pagegrid.dtx
%    (c) If you insist on using LaTeX
%           latex \let\install=y\input{pagegrid.dtx}
%        (quote the arguments according to the demands of your shell)
%
% Documentation:
%    (a) If pagegrid.drv is present:
%           latex pagegrid.drv
%    (b) Without pagegrid.drv:
%           latex pagegrid.dtx; ...
%    The class ltxdoc loads the configuration file ltxdoc.cfg
%    if available. Here you can specify further options, e.g.
%    use A4 as paper format:
%       \PassOptionsToClass{a4paper}{article}
%
%    Programm calls to get the documentation (example):
%       pdflatex pagegrid.dtx
%       makeindex -s gind.ist pagegrid.idx
%       pdflatex pagegrid.dtx
%       makeindex -s gind.ist pagegrid.idx
%       pdflatex pagegrid.dtx
%
% Installation:
%    TDS:tex/latex/oberdiek/pagegrid.sty
%    TDS:doc/latex/oberdiek/pagegrid.pdf
%    TDS:doc/latex/oberdiek/test/pagegrid-test1.tex
%    TDS:source/latex/oberdiek/pagegrid.dtx
%
%<*ignore>
\begingroup
  \catcode123=1 %
  \catcode125=2 %
  \def\x{LaTeX2e}%
\expandafter\endgroup
\ifcase 0\ifx\install y1\fi\expandafter
         \ifx\csname processbatchFile\endcsname\relax\else1\fi
         \ifx\fmtname\x\else 1\fi\relax
\else\csname fi\endcsname
%</ignore>
%<*install>
\input docstrip.tex
\Msg{************************************************************************}
\Msg{* Installation}
\Msg{* Package: pagegrid 2016/05/16 v1.5 Print page grid in background (HO)}
\Msg{************************************************************************}

\keepsilent
\askforoverwritefalse

\let\MetaPrefix\relax
\preamble

This is a generated file.

Project: pagegrid
Version: 2016/05/16 v1.5

Copyright (C) 2009 by
   Heiko Oberdiek <heiko.oberdiek at googlemail.com>

This work may be distributed and/or modified under the
conditions of the LaTeX Project Public License, either
version 1.3c of this license or (at your option) any later
version. This version of this license is in
   https://www.latex-project.org/lppl/lppl-1-3c.txt
and the latest version of this license is in
   https://www.latex-project.org/lppl.txt
and version 1.3 or later is part of all distributions of
LaTeX version 2005/12/01 or later.

This work has the LPPL maintenance status "maintained".

The Current Maintainers of this work are
Heiko Oberdiek and the Oberdiek Package Support Group
https://github.com/ho-tex/oberdiek/issues


This work consists of the main source file pagegrid.dtx
and the derived files
   pagegrid.sty, pagegrid.pdf, pagegrid.ins, pagegrid.drv,
   pagegrid-test1.tex.

\endpreamble
\let\MetaPrefix\DoubleperCent

\generate{%
  \file{pagegrid.ins}{\from{pagegrid.dtx}{install}}%
  \file{pagegrid.drv}{\from{pagegrid.dtx}{driver}}%
  \usedir{tex/latex/oberdiek}%
  \file{pagegrid.sty}{\from{pagegrid.dtx}{package}}%
%  \usedir{doc/latex/oberdiek/test}%
%  \file{pagegrid-test1.tex}{\from{pagegrid.dtx}{test1}}%
  \nopreamble
  \nopostamble
%  \usedir{source/latex/oberdiek/catalogue}%
%  \file{pagegrid.xml}{\from{pagegrid.dtx}{catalogue}}%
}

\catcode32=13\relax% active space
\let =\space%
\Msg{************************************************************************}
\Msg{*}
\Msg{* To finish the installation you have to move the following}
\Msg{* file into a directory searched by TeX:}
\Msg{*}
\Msg{*     pagegrid.sty}
\Msg{*}
\Msg{* To produce the documentation run the file `pagegrid.drv'}
\Msg{* through LaTeX.}
\Msg{*}
\Msg{* Happy TeXing!}
\Msg{*}
\Msg{************************************************************************}

\endbatchfile
%</install>
%<*ignore>
\fi
%</ignore>
%<*driver>
\NeedsTeXFormat{LaTeX2e}
\ProvidesFile{pagegrid.drv}%
  [2016/05/16 v1.5 Print page grid in background (HO)]%
\documentclass{ltxdoc}
\usepackage{holtxdoc}[2011/11/22]
\begin{document}
  \DocInput{pagegrid.dtx}%
\end{document}
%</driver>
% \fi
%
%
% \CharacterTable
%  {Upper-case    \A\B\C\D\E\F\G\H\I\J\K\L\M\N\O\P\Q\R\S\T\U\V\W\X\Y\Z
%   Lower-case    \a\b\c\d\e\f\g\h\i\j\k\l\m\n\o\p\q\r\s\t\u\v\w\x\y\z
%   Digits        \0\1\2\3\4\5\6\7\8\9
%   Exclamation   \!     Double quote  \"     Hash (number) \#
%   Dollar        \$     Percent       \%     Ampersand     \&
%   Acute accent  \'     Left paren    \(     Right paren   \)
%   Asterisk      \*     Plus          \+     Comma         \,
%   Minus         \-     Point         \.     Solidus       \/
%   Colon         \:     Semicolon     \;     Less than     \<
%   Equals        \=     Greater than  \>     Question mark \?
%   Commercial at \@     Left bracket  \[     Backslash     \\
%   Right bracket \]     Circumflex    \^     Underscore    \_
%   Grave accent  \`     Left brace    \{     Vertical bar  \|
%   Right brace   \}     Tilde         \~}
%
% \GetFileInfo{pagegrid.drv}
%
% \title{The \xpackage{pagegrid} package}
% \date{2016/05/16 v1.5}
% \author{Heiko Oberdiek\thanks
% {Please report any issues at \url{https://github.com/ho-tex/oberdiek/issues}}}
%
% \maketitle
%
% \begin{abstract}
% The \LaTeX\ package prints a page grid in the background.
% \end{abstract}
%
% \tableofcontents
%
% \section{Documentation}
%
% The package puts a grid on the paper. It was written for
% developers of a class or package
% who have to put elements on definite locations on a page
% (e.g. letter class). The grid allows a faster optical check,
% whether the positions are correct. If the previewer already
% offers features for measuring, the package might be obsolete.
% Otherwise it saves the developer from printing the page and
% measuring by hand.
%
% \subsection{Options}
%
% Options are evaluated in the following order:
% \begin{enumerate}
% \item
%  Configuration file \xfile{pagegrid.cfg} using \cs{pagegridsetup}
%  if the file exists.
%  \item
%  Package options given for \cs{usepackage}.
%  \item
%  Later calls of \cs{pagegridsetup}.
% \end{enumerate}
% \begin{declcs}{pagegridsetup}\M{option list}
% \end{declcs}
% The options are key value options. Boolean options are enabled by
% default (without value) or by using the explicit value \texttt{true}.
% Value \texttt{false} disable the option.
%
% \subsubsection{Options \xoption{enable}, \xoption{disable}}
%
% \begin{description}
% \item[\xoption{enable}:] This boolean option controls whether the page grid
%   is drawn. As default the page grid drawing is activated.
% \item[\xoption{disable}:] It is the opposite
%   of option \xoption{enable}. It was added for convenience and
%   allows the abbreviation \texttt{disable} for \texttt{enable=false}.
% \end{description}
%
% \subsubsection{Grid origins}
%
% The package supports up to two grids on a page allowing
% measurement from opposite directions. As default two grids are drawn,
% the first from bottom left to top right. The origin of the second
% grid is at the opposite top right corner.
% The origins are controlled by the following options.
% The number of grids (one or two) depend on the number of these options
% in one call of \cs{pagegridsetup}.
% The following frame shows a paper and in its corners are the
% corresponding options. At the left and right side alias names
% are given for the options inside the paper.
% \begin{quote}
% \begin{tabular}{@{}r|@{\,}l@{\qquad}r@{\,}|l@{}}
% \cline{2-3}
% \xoption{left-top}, \xoption{lt}, \xoption{top-left}
% & \vphantom{\"U}\xoption{tl} & \xoption{tr}
% & \xoption{top-right}, \xoption{rt}, \xoption{right-top}\\
% &&&\\
% \xoption{left-bottom}, \xoption{lb}, \xoption{bottom-left}
% & \xoption{bl} & \xoption{br}
% & \xoption{bottom-right}, \xoption{rb}, \xoption{right-bottom}\\
% \cline{2-3}
% \end{tabular}
% \end{quote}
% Examples:
% \begin{quote}
% |\pagegridsetup{bl,tr}|
% \end{quote}
% This is the default setting with two grids as described previously.
% The following setups one grid only. Its origin is the upper left
% corner:
% \begin{quote}
% |\pagegridsetup{top-left}|
% \end{quote}
%
% \subsubsection{Grid unit}
%
% \begin{description}
% \item[\xoption{step}] This option takes a length and
% setups the unit for the grid. The page width and page height
% should be multiples of this unit.
% Currently the default is \texttt{1mm}. But this might change
% later by a heuristic based on the paper size.
% \end{description}
%
% \subsubsection{Color options}
%
% The basic grid lines are drawn as ultra thin help lines and is only
% drawn for the first grid.
% Each tenth and fiftyth line of the basic net is drawn thicker in a special
% color for the two grids.
% \begin{description}
% \item[\xoption{firstcolor}:] Color for the thicker lines and the arrows
% of the first grid. Default value is \texttt{red}.
% \item[\xoption{secondcolor}:] Color for the thicker lines and the arrows
% of the second grid. Default value is \texttt{blue}.
% \end{description}
% Use a color specification that package \xpackage{tikz} understands.
% (The grid is drawn with \xpackage{pgf}/\xpackage{tikz}.)
%
% \subsubsection{Arrow options}
%
% Arrows are put at the origin at the grid to show the grid start
% and the direction of the grid.
% \begin{description}
% \item[\xoption{arrows}:] This boolean option turns the arrows on or off.
% As default arrows are enabled.
% \item[\xoption{arrowlength}:] The length given as value is the
% length of the edge of a square at the origin within the
% arrow is put as diagonal. Default is 10 times the grid unit (10\,mm).
% The real arrow length is this length multiplied by $\sqrt2$.
% \end{description}
%
% \subsubsection{Miscellaneous options}
%
% \begin{description}
% \item[\xoption{double}:] The output page is doubled, one without page
% grid and the other with page grid. Possible values are shown in the
% following table:
% \begin{quote}
% \begin{tabular}{ll}
% Option & Meaning\\
% \hline
% |false| & Turns option off.\\
% |first| & Grid page comes first.\\
% |last| & Grid page comes after the page without grid.\\
% |true| & Same as |last|.\\
% \meta{no value} & Same as |true|.\\
% \end{tabular}
% \end{quote}
% \textbf{Note:}
% The double output of the page has side effects.
% All whatits are executed twice, for example: file writing
% and anchor setting. Some unwanted actions are catched such
% as multiple \cs{label} definitions, duplicate entries in
% the table of contents. For bookmarks, use package \xpackage{bookmarks}.
% \item[\xoption{foreground}:] Boolean option, default is \texttt{false}.
% Sometimes there might be elements on the page (e.g. large images)
% that hide the grid. Then option \xoption{foreground} puts the grids
% over the current output page.
% \end{description}
%
% \StopEventually{
% }
%
% \section{Implementation}
%    \begin{macrocode}
%<*package>
%    \end{macrocode}
%    Reload check, especially if the package is not used with \LaTeX.
%    \begin{macrocode}
\begingroup\catcode61\catcode48\catcode32=10\relax%
  \catcode13=5 % ^^M
  \endlinechar=13 %
  \catcode35=6 % #
  \catcode39=12 % '
  \catcode44=12 % ,
  \catcode45=12 % -
  \catcode46=12 % .
  \catcode58=12 % :
  \catcode64=11 % @
  \catcode123=1 % {
  \catcode125=2 % }
  \expandafter\let\expandafter\x\csname ver@pagegrid.sty\endcsname
  \ifx\x\relax % plain-TeX, first loading
  \else
    \def\empty{}%
    \ifx\x\empty % LaTeX, first loading,
      % variable is initialized, but \ProvidesPackage not yet seen
    \else
      \expandafter\ifx\csname PackageInfo\endcsname\relax
        \def\x#1#2{%
          \immediate\write-1{Package #1 Info: #2.}%
        }%
      \else
        \def\x#1#2{\PackageInfo{#1}{#2, stopped}}%
      \fi
      \x{pagegrid}{The package is already loaded}%
      \aftergroup\endinput
    \fi
  \fi
\endgroup%
%    \end{macrocode}
%    Package identification:
%    \begin{macrocode}
\begingroup\catcode61\catcode48\catcode32=10\relax%
  \catcode13=5 % ^^M
  \endlinechar=13 %
  \catcode35=6 % #
  \catcode39=12 % '
  \catcode40=12 % (
  \catcode41=12 % )
  \catcode44=12 % ,
  \catcode45=12 % -
  \catcode46=12 % .
  \catcode47=12 % /
  \catcode58=12 % :
  \catcode64=11 % @
  \catcode91=12 % [
  \catcode93=12 % ]
  \catcode123=1 % {
  \catcode125=2 % }
  \expandafter\ifx\csname ProvidesPackage\endcsname\relax
    \def\x#1#2#3[#4]{\endgroup
      \immediate\write-1{Package: #3 #4}%
      \xdef#1{#4}%
    }%
  \else
    \def\x#1#2[#3]{\endgroup
      #2[{#3}]%
      \ifx#1\@undefined
        \xdef#1{#3}%
      \fi
      \ifx#1\relax
        \xdef#1{#3}%
      \fi
    }%
  \fi
\expandafter\x\csname ver@pagegrid.sty\endcsname
\ProvidesPackage{pagegrid}%
  [2016/05/16 v1.5 Print page grid in background (HO)]%
%    \end{macrocode}
%
%    \begin{macrocode}
\begingroup\catcode61\catcode48\catcode32=10\relax%
  \catcode13=5 % ^^M
  \endlinechar=13 %
  \catcode123=1 % {
  \catcode125=2 % }
  \catcode64=11 % @
  \def\x{\endgroup
    \expandafter\edef\csname pagegrid@AtEnd\endcsname{%
      \endlinechar=\the\endlinechar\relax
      \catcode13=\the\catcode13\relax
      \catcode32=\the\catcode32\relax
      \catcode35=\the\catcode35\relax
      \catcode61=\the\catcode61\relax
      \catcode64=\the\catcode64\relax
      \catcode123=\the\catcode123\relax
      \catcode125=\the\catcode125\relax
    }%
  }%
\x\catcode61\catcode48\catcode32=10\relax%
\catcode13=5 % ^^M
\endlinechar=13 %
\catcode35=6 % #
\catcode64=11 % @
\catcode123=1 % {
\catcode125=2 % }
\def\TMP@EnsureCode#1#2{%
  \edef\pagegrid@AtEnd{%
    \pagegrid@AtEnd
    \catcode#1=\the\catcode#1\relax
  }%
  \catcode#1=#2\relax
}
\TMP@EnsureCode{9}{10}% (tab)
\TMP@EnsureCode{10}{12}% ^^J
\TMP@EnsureCode{33}{12}% !
\TMP@EnsureCode{34}{12}% "
\TMP@EnsureCode{36}{3}% $
\TMP@EnsureCode{38}{4}% &
\TMP@EnsureCode{39}{12}% '
\TMP@EnsureCode{40}{12}% (
\TMP@EnsureCode{41}{12}% )
\TMP@EnsureCode{42}{12}% *
\TMP@EnsureCode{43}{12}% +
\TMP@EnsureCode{44}{12}% ,
\TMP@EnsureCode{45}{12}% -
\TMP@EnsureCode{46}{12}% .
\TMP@EnsureCode{47}{12}% /
\TMP@EnsureCode{58}{12}% :
\TMP@EnsureCode{59}{12}% ;
\TMP@EnsureCode{60}{12}% <
\TMP@EnsureCode{62}{12}% >
\TMP@EnsureCode{63}{12}% ?
\TMP@EnsureCode{91}{12}% [
\TMP@EnsureCode{93}{12}% ]
\TMP@EnsureCode{94}{7}% ^ (superscript)
\TMP@EnsureCode{95}{8}% _ (subscript)
\TMP@EnsureCode{96}{12}% `
\TMP@EnsureCode{124}{12}% |
\edef\pagegrid@AtEnd{\pagegrid@AtEnd\noexpand\endinput}
%    \end{macrocode}
%
%    \begin{macrocode}
\RequirePackage{tikz}
\RequirePackage{atbegshi}[2009/12/02]
\RequirePackage{kvoptions}[2009/07/17]
%    \end{macrocode}
%    \begin{macrocode}
\begingroup\expandafter\expandafter\expandafter\endgroup
\expandafter\ifx\csname stockwidth\endcsname\relax
  \def\pagegrid@width{\paperwidth}%
  \def\pagegrid@height{\paperheight}%
\else
  \def\pagegrid@width{\stockwidth}%
  \def\pagegrid@height{\stockheight}%
\fi
%    \end{macrocode}
%
%    \begin{macrocode}
\SetupKeyvalOptions{%
  family=pagegrid,%
  prefix=pagegrid@,%
}
\def\pagegrid@init{%
  \let\pagegrid@origin@a\@empty
  \let\pagegrid@origin@b\@empty
  \let\pagegrid@init\relax
}
\let\pagegrid@@init\pagegrid@init
\def\pagegrid@origin@a{bl}
\def\pagegrid@origin@b{tr}
\def\pagegrid@SetOrigin#1{%
  \pagegrid@init
  \ifx\pagegrid@origin@a\@empty
    \def\pagegrid@origin@a{#1}%
  \else
    \ifx\pagegrid@origin@b\@empty
    \else
      \let\pagegrid@origin@a\pagegrid@origin@b
    \fi
    \def\pagegrid@origin@b{#1}%
  \fi
}
\def\pagegrid@temp#1{%
  \DeclareVoidOption{#1}{\pagegrid@SetOrigin{#1}}%
  \@namedef{pagegrid@N@#1}{#1}%
}
\pagegrid@temp{bl}
\pagegrid@temp{br}
\pagegrid@temp{tl}
\pagegrid@temp{tr}
\def\pagegrid@temp#1#2{%
  \DeclareVoidOption{#2}{\pagegrid@SetOrigin{#1}}%
}%
\pagegrid@temp{bl}{lb}
\pagegrid@temp{br}{rb}
\pagegrid@temp{tl}{lt}
\pagegrid@temp{tr}{rt}
\pagegrid@temp{bl}{bottom-left}
\pagegrid@temp{br}{bottom-right}
\pagegrid@temp{tl}{top-left}
\pagegrid@temp{tr}{top-right}
\pagegrid@temp{bl}{left-bottom}
\pagegrid@temp{br}{right-bottom}
\pagegrid@temp{tl}{left-top}
\pagegrid@temp{tr}{right-top}
%    \end{macrocode}
%    \begin{macrocode}
\DeclareBoolOption[true]{enable}
\DeclareComplementaryOption{disable}{enable}
%    \end{macrocode}
%    \begin{macrocode}
\DeclareBoolOption{foreground}
%    \end{macrocode}
%    \begin{macrocode}
\newlength{\pagegrid@step}
\define@key{pagegrid}{step}{%
  \setlength{\pagegrid@step}{#1}%
}
%    \end{macrocode}
%    \begin{macrocode}
\DeclareStringOption[red]{firstcolor}
\DeclareStringOption[blue]{secondcolor}
%    \end{macrocode}
%    \begin{macrocode}
\DeclareBoolOption[true]{arrows}
\newlength\pagegrid@arrowlength
\pagegrid@arrowlength=\z@
\define@key{pagegrid}{arrowlength}{%
  \setlength{\pagegrid@arrowlength}{#1}%
}
%    \end{macrocode}
%    \begin{macrocode}
\define@key{pagegrid}{double}[true]{%
  \@ifundefined{pagegrid@double@#1}{%
    \PackageWarning{pagegrid}{%
      Unsupported value `#1' for option `double'.\MessageBreak
      Known values are:\MessageBreak
      `false', `first', `last', `true'.\MessageBreak
      Now `false' is used%
    }%
    \chardef\pagegrid@double\z@
  }{%
    \chardef\pagegrid@double\csname pagegrid@double@#1\endcsname\relax
  }%
}
\@namedef{pagegrid@double@false}{0}
\@namedef{pagegrid@double@first}{1}
\@namedef{pagegrid@double@last}{2}
\@namedef{pagegrid@double@true}{2}
\chardef\pagegrid@double\z@
%    \end{macrocode}
%    \begin{macrocode}
\newcommand*{\pagegridsetup}{%
  \let\pagegrid@init\pagegrid@@init
  \setkeys{pagegrid}%
}
%    \end{macrocode}
%    \begin{macrocode}
\pagegridsetup{%
  step=1mm%
}
\InputIfFileExists{pagegrid.cfg}{}%
\ProcessKeyvalOptions*\relax
\AtBeginDocument{%
  \ifdim\pagegrid@arrowlength>\z@
  \else
    \pagegrid@arrowlength=10\pagegrid@step
  \fi
}
%    \end{macrocode}
%
%    \begin{macrocode}
\def\pagegridShipoutDoubleBegin{%
  \begingroup
  \let\newlabel\@gobbletwo
  \let\zref@newlabel\@gobbletwo
  \let\@writefile\@gobbletwo
  \let\select@language\@gobble
}
\def\pagegridShipoutDoubleEnd{%
  \endgroup
}
\def\pagegrid@WriteDouble#1#2{%
  \immediate\write#1{%
    \@backslashchar csname %
    pagegridShipoutDouble#2%
    \@backslashchar endcsname%
  }%
}
\def\pagegrid@ShipoutDouble#1{%
  \begingroup
    \if@filesw
      \pagegrid@WriteDouble\@mainaux{Begin}%
      \ifx\@auxout\@partaux
        \pagegrid@WriteDouble\@partaux{Begin}%
        \def\pagegrid@temp{%
          \pagegrid@WriteDouble\@mainaux{End}%
          \pagegrid@WriteDouble\@partaux{End}%
        }%
      \else
        \def\pagegrid@temp{%
          \pagegrid@WriteDouble\@mainaux{End}%
        }%
      \fi
    \else
      \def\pagegrid@temp{}%
    \fi
    \let\protect\noexpand
    \AtBeginShipoutOriginalShipout\copy#1\relax
    \pagegrid@temp
  \endgroup
}
%    \end{macrocode}
%
%    \begin{macrocode}
\AtBeginShipout{%
  \ifdim\pagegrid@step>\z@
  \else
    \pagegrid@enablefalse
  \fi
  \ifpagegrid@enable
    \ifnum\pagegrid@double=\@ne
      \pagegrid@ShipoutDouble\AtBeginShipoutBox
    \else
      \ifnum\pagegrid@double=\tw@
        \@ifundefined{pagegrid@DoubleBox}{%
          \newbox\pagegrid@DoubleBox
        }{}%
        \setbox\pagegrid@DoubleBox=\copy\AtBeginShipoutBox
      \fi
    \fi
    \ifpagegrid@foreground
      \expandafter\AtBeginShipoutUpperLeftForeground
    \else
      \expandafter\AtBeginShipoutUpperLeft
    \fi
    {%
      \put(0,0){%
        \makebox(0,0)[lt]{%
          \begin{tikzpicture}[%
            bl/.style={},%
            br/.style={xshift=\pagegrid@width,xscale=-1},%
            tl/.style={yshift=\pagegrid@height,yscale=-1},%
            tr/.style={xshift=\pagegrid@width,%
                       yshift=\pagegrid@height,scale=-1}%
          ]%
            \useasboundingbox
              (0mm,\pagegrid@height) rectangle (0mm,\pagegrid@height);%
            \draw[%
              \pagegrid@origin@a,%
              step=\pagegrid@step,%
              style=help lines,%
              ultra thin%
            ] (0mm,0mm) grid (\pagegrid@width,\pagegrid@height);%
            \ifx\pagegrid@origin@b\@empty
            \else
              \draw[%
                \pagegrid@origin@b,%
                step=10\pagegrid@step,%
                {\pagegrid@secondcolor},%
                very thin%
              ] (0mm,0mm) grid (\pagegrid@width,\pagegrid@height);%
            \fi
            \draw[%
               \pagegrid@origin@a,%
               step=10\pagegrid@step,%
               {\pagegrid@firstcolor},%
               very thin%
            ] (0mm,0mm) grid (\pagegrid@width,\pagegrid@height);%
            \ifx\pagegrid@origin@b\@empty
            \else
              \draw[%
                \pagegrid@origin@b,%
                step=50\pagegrid@step,%
                {\pagegrid@secondcolor},%
                thick%
              ] (0mm,0mm) grid (\pagegrid@width,\pagegrid@height);%
            \fi
            \draw[%
              \pagegrid@origin@a,%
              step=50\pagegrid@step,%
              {\pagegrid@firstcolor},%
              thick%
            ] (0mm,0mm) grid (\pagegrid@width,\pagegrid@height);%
            \ifpagegrid@arrows
              \ifx\pagegrid@origin@b\@empty
              \else
                \draw[%
                  \pagegrid@origin@b,%
                  {\pagegrid@secondcolor},%
                  stroke,%
                  line width=1pt,%
                  line cap=round%
                ] (0mm,0mm) %
                -- (\pagegrid@arrowlength,\pagegrid@arrowlength) %
                   (\pagegrid@arrowlength,.5\pagegrid@arrowlength) %
                -- (\pagegrid@arrowlength,\pagegrid@arrowlength) %
                -- (.5\pagegrid@arrowlength,\pagegrid@arrowlength);%
              \fi
              \draw[%
                \pagegrid@origin@a,%
                {\pagegrid@firstcolor},%
                stroke,%
                line width=1pt,%
                line cap=round%
              ] (0mm,0mm) %
              -- (\pagegrid@arrowlength,\pagegrid@arrowlength) %
                 (\pagegrid@arrowlength,.5\pagegrid@arrowlength) %
              -- (\pagegrid@arrowlength,\pagegrid@arrowlength) %
              -- (.5\pagegrid@arrowlength,\pagegrid@arrowlength);%
            \fi
          \end{tikzpicture}%
        }%
      }%
    }%
    \ifnum\pagegrid@double=\tw@
      \pagegrid@ShipoutDouble\pagegrid@DoubleBox
    \fi
  \fi
}
%    \end{macrocode}
%
%    \begin{macrocode}
\pagegrid@AtEnd%
%</package>
%    \end{macrocode}
%
% \section{Test}
%
% \subsection{Catcode checks for loading}
%
%    \begin{macrocode}
%<*test1>
%    \end{macrocode}
%    \begin{macrocode}
\catcode`\{=1 %
\catcode`\}=2 %
\catcode`\#=6 %
\catcode`\@=11 %
\expandafter\ifx\csname count@\endcsname\relax
  \countdef\count@=255 %
\fi
\expandafter\ifx\csname @gobble\endcsname\relax
  \long\def\@gobble#1{}%
\fi
\expandafter\ifx\csname @firstofone\endcsname\relax
  \long\def\@firstofone#1{#1}%
\fi
\expandafter\ifx\csname loop\endcsname\relax
  \expandafter\@firstofone
\else
  \expandafter\@gobble
\fi
{%
  \def\loop#1\repeat{%
    \def\body{#1}%
    \iterate
  }%
  \def\iterate{%
    \body
      \let\next\iterate
    \else
      \let\next\relax
    \fi
    \next
  }%
  \let\repeat=\fi
}%
\def\RestoreCatcodes{}
\count@=0 %
\loop
  \edef\RestoreCatcodes{%
    \RestoreCatcodes
    \catcode\the\count@=\the\catcode\count@\relax
  }%
\ifnum\count@<255 %
  \advance\count@ 1 %
\repeat

\def\RangeCatcodeInvalid#1#2{%
  \count@=#1\relax
  \loop
    \catcode\count@=15 %
  \ifnum\count@<#2\relax
    \advance\count@ 1 %
  \repeat
}
\def\RangeCatcodeCheck#1#2#3{%
  \count@=#1\relax
  \loop
    \ifnum#3=\catcode\count@
    \else
      \errmessage{%
        Character \the\count@\space
        with wrong catcode \the\catcode\count@\space
        instead of \number#3%
      }%
    \fi
  \ifnum\count@<#2\relax
    \advance\count@ 1 %
  \repeat
}
\def\space{ }
\expandafter\ifx\csname LoadCommand\endcsname\relax
  \def\LoadCommand{\input pagegrid.sty\relax}%
\fi
\def\Test{%
  \RangeCatcodeInvalid{0}{47}%
  \RangeCatcodeInvalid{58}{64}%
  \RangeCatcodeInvalid{91}{96}%
  \RangeCatcodeInvalid{123}{255}%
  \catcode`\@=12 %
  \catcode`\\=0 %
  \catcode`\%=14 %
  \LoadCommand
  \RangeCatcodeCheck{0}{36}{15}%
  \RangeCatcodeCheck{37}{37}{14}%
  \RangeCatcodeCheck{38}{47}{15}%
  \RangeCatcodeCheck{48}{57}{12}%
  \RangeCatcodeCheck{58}{63}{15}%
  \RangeCatcodeCheck{64}{64}{12}%
  \RangeCatcodeCheck{65}{90}{11}%
  \RangeCatcodeCheck{91}{91}{15}%
  \RangeCatcodeCheck{92}{92}{0}%
  \RangeCatcodeCheck{93}{96}{15}%
  \RangeCatcodeCheck{97}{122}{11}%
  \RangeCatcodeCheck{123}{255}{15}%
  \RestoreCatcodes
}
\Test
\csname @@end\endcsname
\end
%    \end{macrocode}
%    \begin{macrocode}
%</test1>
%    \end{macrocode}
%
% \section{Installation}
%
% \subsection{Download}
%
% \paragraph{Package.} This package is available on
% CTAN\footnote{\CTANpkg{pagegrid}}:
% \begin{description}
% \item[\CTAN{macros/latex/contrib/oberdiek/pagegrid.dtx}] The source file.
% \item[\CTAN{macros/latex/contrib/oberdiek/pagegrid.pdf}] Documentation.
% \end{description}
%
%
% \paragraph{Bundle.} All the packages of the bundle `oberdiek'
% are also available in a TDS compliant ZIP archive. There
% the packages are already unpacked and the documentation files
% are generated. The files and directories obey the TDS standard.
% \begin{description}
% \item[\CTANinstall{install/macros/latex/contrib/oberdiek.tds.zip}]
% \end{description}
% \emph{TDS} refers to the standard ``A Directory Structure
% for \TeX\ Files'' (\CTAN{tds/tds.pdf}). Directories
% with \xfile{texmf} in their name are usually organized this way.
%
% \subsection{Bundle installation}
%
% \paragraph{Unpacking.} Unpack the \xfile{oberdiek.tds.zip} in the
% TDS tree (also known as \xfile{texmf} tree) of your choice.
% Example (linux):
% \begin{quote}
%   |unzip oberdiek.tds.zip -d ~/texmf|
% \end{quote}
%
% \paragraph{Script installation.}
% Check the directory \xfile{TDS:scripts/oberdiek/} for
% scripts that need further installation steps.
%
% \subsection{Package installation}
%
% \paragraph{Unpacking.} The \xfile{.dtx} file is a self-extracting
% \docstrip\ archive. The files are extracted by running the
% \xfile{.dtx} through \plainTeX:
% \begin{quote}
%   \verb|tex pagegrid.dtx|
% \end{quote}
%
% \paragraph{TDS.} Now the different files must be moved into
% the different directories in your installation TDS tree
% (also known as \xfile{texmf} tree):
% \begin{quote}
% \def\t{^^A
% \begin{tabular}{@{}>{\ttfamily}l@{ $\rightarrow$ }>{\ttfamily}l@{}}
%   pagegrid.sty & tex/latex/oberdiek/pagegrid.sty\\
%   pagegrid.pdf & doc/latex/oberdiek/pagegrid.pdf\\
%   test/pagegrid-test1.tex & doc/latex/oberdiek/test/pagegrid-test1.tex\\
%   pagegrid.dtx & source/latex/oberdiek/pagegrid.dtx\\
% \end{tabular}^^A
% }^^A
% \sbox0{\t}^^A
% \ifdim\wd0>\linewidth
%   \begingroup
%     \advance\linewidth by\leftmargin
%     \advance\linewidth by\rightmargin
%   \edef\x{\endgroup
%     \def\noexpand\lw{\the\linewidth}^^A
%   }\x
%   \def\lwbox{^^A
%     \leavevmode
%     \hbox to \linewidth{^^A
%       \kern-\leftmargin\relax
%       \hss
%       \usebox0
%       \hss
%       \kern-\rightmargin\relax
%     }^^A
%   }^^A
%   \ifdim\wd0>\lw
%     \sbox0{\small\t}^^A
%     \ifdim\wd0>\linewidth
%       \ifdim\wd0>\lw
%         \sbox0{\footnotesize\t}^^A
%         \ifdim\wd0>\linewidth
%           \ifdim\wd0>\lw
%             \sbox0{\scriptsize\t}^^A
%             \ifdim\wd0>\linewidth
%               \ifdim\wd0>\lw
%                 \sbox0{\tiny\t}^^A
%                 \ifdim\wd0>\linewidth
%                   \lwbox
%                 \else
%                   \usebox0
%                 \fi
%               \else
%                 \lwbox
%               \fi
%             \else
%               \usebox0
%             \fi
%           \else
%             \lwbox
%           \fi
%         \else
%           \usebox0
%         \fi
%       \else
%         \lwbox
%       \fi
%     \else
%       \usebox0
%     \fi
%   \else
%     \lwbox
%   \fi
% \else
%   \usebox0
% \fi
% \end{quote}
% If you have a \xfile{docstrip.cfg} that configures and enables \docstrip's
% TDS installing feature, then some files can already be in the right
% place, see the documentation of \docstrip.
%
% \subsection{Refresh file name databases}
%
% If your \TeX~distribution
% (\TeX\,Live, \mikTeX, \dots) relies on file name databases, you must refresh
% these. For example, \TeX\,Live\ users run \verb|texhash| or
% \verb|mktexlsr|.
%
% \subsection{Some details for the interested}
%
% \paragraph{Unpacking with \LaTeX.}
% The \xfile{.dtx} chooses its action depending on the format:
% \begin{description}
% \item[\plainTeX:] Run \docstrip\ and extract the files.
% \item[\LaTeX:] Generate the documentation.
% \end{description}
% If you insist on using \LaTeX\ for \docstrip\ (really,
% \docstrip\ does not need \LaTeX), then inform the autodetect routine
% about your intention:
% \begin{quote}
%   \verb|latex \let\install=y\input{pagegrid.dtx}|
% \end{quote}
% Do not forget to quote the argument according to the demands
% of your shell.
%
% \paragraph{Generating the documentation.}
% You can use both the \xfile{.dtx} or the \xfile{.drv} to generate
% the documentation. The process can be configured by the
% configuration file \xfile{ltxdoc.cfg}. For instance, put this
% line into this file, if you want to have A4 as paper format:
% \begin{quote}
%   \verb|\PassOptionsToClass{a4paper}{article}|
% \end{quote}
% An example follows how to generate the
% documentation with pdf\LaTeX:
% \begin{quote}
%\begin{verbatim}
%pdflatex pagegrid.dtx
%makeindex -s gind.ist pagegrid.idx
%pdflatex pagegrid.dtx
%makeindex -s gind.ist pagegrid.idx
%pdflatex pagegrid.dtx
%\end{verbatim}
% \end{quote}
%
% \section{Acknowledgement}
%
% \begin{description}
% \item[Klaus Braune:]
%  He provided the idea and the first \xpackage{tikz} code.
% \end{description}
%
% \begin{History}
%   \begin{Version}{2009/11/06 v1.0}
%   \item
%     The first version.
%   \end{Version}
%   \begin{Version}{2009/11/06 v1.1}
%   \item
%     Option \xoption{foreground} added.
%   \end{Version}
%   \begin{Version}{2009/12/02 v1.2}
%   \item
%     Color options, arrow options added.
%   \item
%     Names for origin options changed.
%   \end{Version}
%   \begin{Version}{2009/12/03 v1.3}
%   \item
%     Option \xoption{double} added.
%   \item
%     First CTAN release.
%   \end{Version}
%   \begin{Version}{2009/12/04 v1.4}
%   \item
%     Option \xoption{double}: Some unwanted side effects removed.
%   \end{Version}
%   \begin{Version}{2016/05/16 v1.5}
%   \item
%     Documentation updates.
%   \end{Version}
% \end{History}
%
% \PrintIndex
%
% \Finale
\endinput

%        (quote the arguments according to the demands of your shell)
%
% Documentation:
%    (a) If pagegrid.drv is present:
%           latex pagegrid.drv
%    (b) Without pagegrid.drv:
%           latex pagegrid.dtx; ...
%    The class ltxdoc loads the configuration file ltxdoc.cfg
%    if available. Here you can specify further options, e.g.
%    use A4 as paper format:
%       \PassOptionsToClass{a4paper}{article}
%
%    Programm calls to get the documentation (example):
%       pdflatex pagegrid.dtx
%       makeindex -s gind.ist pagegrid.idx
%       pdflatex pagegrid.dtx
%       makeindex -s gind.ist pagegrid.idx
%       pdflatex pagegrid.dtx
%
% Installation:
%    TDS:tex/latex/oberdiek/pagegrid.sty
%    TDS:doc/latex/oberdiek/pagegrid.pdf
%    TDS:doc/latex/oberdiek/test/pagegrid-test1.tex
%    TDS:source/latex/oberdiek/pagegrid.dtx
%
%<*ignore>
\begingroup
  \catcode123=1 %
  \catcode125=2 %
  \def\x{LaTeX2e}%
\expandafter\endgroup
\ifcase 0\ifx\install y1\fi\expandafter
         \ifx\csname processbatchFile\endcsname\relax\else1\fi
         \ifx\fmtname\x\else 1\fi\relax
\else\csname fi\endcsname
%</ignore>
%<*install>
\input docstrip.tex
\Msg{************************************************************************}
\Msg{* Installation}
\Msg{* Package: pagegrid 2016/05/16 v1.5 Print page grid in background (HO)}
\Msg{************************************************************************}

\keepsilent
\askforoverwritefalse

\let\MetaPrefix\relax
\preamble

This is a generated file.

Project: pagegrid
Version: 2016/05/16 v1.5

Copyright (C) 2009 by
   Heiko Oberdiek <heiko.oberdiek at googlemail.com>

This work may be distributed and/or modified under the
conditions of the LaTeX Project Public License, either
version 1.3c of this license or (at your option) any later
version. This version of this license is in
   https://www.latex-project.org/lppl/lppl-1-3c.txt
and the latest version of this license is in
   https://www.latex-project.org/lppl.txt
and version 1.3 or later is part of all distributions of
LaTeX version 2005/12/01 or later.

This work has the LPPL maintenance status "maintained".

The Current Maintainers of this work are
Heiko Oberdiek and the Oberdiek Package Support Group
https://github.com/ho-tex/oberdiek/issues


This work consists of the main source file pagegrid.dtx
and the derived files
   pagegrid.sty, pagegrid.pdf, pagegrid.ins, pagegrid.drv,
   pagegrid-test1.tex.

\endpreamble
\let\MetaPrefix\DoubleperCent

\generate{%
  \file{pagegrid.ins}{\from{pagegrid.dtx}{install}}%
  \file{pagegrid.drv}{\from{pagegrid.dtx}{driver}}%
  \usedir{tex/latex/oberdiek}%
  \file{pagegrid.sty}{\from{pagegrid.dtx}{package}}%
%  \usedir{doc/latex/oberdiek/test}%
%  \file{pagegrid-test1.tex}{\from{pagegrid.dtx}{test1}}%
  \nopreamble
  \nopostamble
%  \usedir{source/latex/oberdiek/catalogue}%
%  \file{pagegrid.xml}{\from{pagegrid.dtx}{catalogue}}%
}

\catcode32=13\relax% active space
\let =\space%
\Msg{************************************************************************}
\Msg{*}
\Msg{* To finish the installation you have to move the following}
\Msg{* file into a directory searched by TeX:}
\Msg{*}
\Msg{*     pagegrid.sty}
\Msg{*}
\Msg{* To produce the documentation run the file `pagegrid.drv'}
\Msg{* through LaTeX.}
\Msg{*}
\Msg{* Happy TeXing!}
\Msg{*}
\Msg{************************************************************************}

\endbatchfile
%</install>
%<*ignore>
\fi
%</ignore>
%<*driver>
\NeedsTeXFormat{LaTeX2e}
\ProvidesFile{pagegrid.drv}%
  [2016/05/16 v1.5 Print page grid in background (HO)]%
\documentclass{ltxdoc}
\usepackage{holtxdoc}[2011/11/22]
\begin{document}
  \DocInput{pagegrid.dtx}%
\end{document}
%</driver>
% \fi
%
%
% \CharacterTable
%  {Upper-case    \A\B\C\D\E\F\G\H\I\J\K\L\M\N\O\P\Q\R\S\T\U\V\W\X\Y\Z
%   Lower-case    \a\b\c\d\e\f\g\h\i\j\k\l\m\n\o\p\q\r\s\t\u\v\w\x\y\z
%   Digits        \0\1\2\3\4\5\6\7\8\9
%   Exclamation   \!     Double quote  \"     Hash (number) \#
%   Dollar        \$     Percent       \%     Ampersand     \&
%   Acute accent  \'     Left paren    \(     Right paren   \)
%   Asterisk      \*     Plus          \+     Comma         \,
%   Minus         \-     Point         \.     Solidus       \/
%   Colon         \:     Semicolon     \;     Less than     \<
%   Equals        \=     Greater than  \>     Question mark \?
%   Commercial at \@     Left bracket  \[     Backslash     \\
%   Right bracket \]     Circumflex    \^     Underscore    \_
%   Grave accent  \`     Left brace    \{     Vertical bar  \|
%   Right brace   \}     Tilde         \~}
%
% \GetFileInfo{pagegrid.drv}
%
% \title{The \xpackage{pagegrid} package}
% \date{2016/05/16 v1.5}
% \author{Heiko Oberdiek\thanks
% {Please report any issues at \url{https://github.com/ho-tex/oberdiek/issues}}}
%
% \maketitle
%
% \begin{abstract}
% The \LaTeX\ package prints a page grid in the background.
% \end{abstract}
%
% \tableofcontents
%
% \section{Documentation}
%
% The package puts a grid on the paper. It was written for
% developers of a class or package
% who have to put elements on definite locations on a page
% (e.g. letter class). The grid allows a faster optical check,
% whether the positions are correct. If the previewer already
% offers features for measuring, the package might be obsolete.
% Otherwise it saves the developer from printing the page and
% measuring by hand.
%
% \subsection{Options}
%
% Options are evaluated in the following order:
% \begin{enumerate}
% \item
%  Configuration file \xfile{pagegrid.cfg} using \cs{pagegridsetup}
%  if the file exists.
%  \item
%  Package options given for \cs{usepackage}.
%  \item
%  Later calls of \cs{pagegridsetup}.
% \end{enumerate}
% \begin{declcs}{pagegridsetup}\M{option list}
% \end{declcs}
% The options are key value options. Boolean options are enabled by
% default (without value) or by using the explicit value \texttt{true}.
% Value \texttt{false} disable the option.
%
% \subsubsection{Options \xoption{enable}, \xoption{disable}}
%
% \begin{description}
% \item[\xoption{enable}:] This boolean option controls whether the page grid
%   is drawn. As default the page grid drawing is activated.
% \item[\xoption{disable}:] It is the opposite
%   of option \xoption{enable}. It was added for convenience and
%   allows the abbreviation \texttt{disable} for \texttt{enable=false}.
% \end{description}
%
% \subsubsection{Grid origins}
%
% The package supports up to two grids on a page allowing
% measurement from opposite directions. As default two grids are drawn,
% the first from bottom left to top right. The origin of the second
% grid is at the opposite top right corner.
% The origins are controlled by the following options.
% The number of grids (one or two) depend on the number of these options
% in one call of \cs{pagegridsetup}.
% The following frame shows a paper and in its corners are the
% corresponding options. At the left and right side alias names
% are given for the options inside the paper.
% \begin{quote}
% \begin{tabular}{@{}r|@{\,}l@{\qquad}r@{\,}|l@{}}
% \cline{2-3}
% \xoption{left-top}, \xoption{lt}, \xoption{top-left}
% & \vphantom{\"U}\xoption{tl} & \xoption{tr}
% & \xoption{top-right}, \xoption{rt}, \xoption{right-top}\\
% &&&\\
% \xoption{left-bottom}, \xoption{lb}, \xoption{bottom-left}
% & \xoption{bl} & \xoption{br}
% & \xoption{bottom-right}, \xoption{rb}, \xoption{right-bottom}\\
% \cline{2-3}
% \end{tabular}
% \end{quote}
% Examples:
% \begin{quote}
% |\pagegridsetup{bl,tr}|
% \end{quote}
% This is the default setting with two grids as described previously.
% The following setups one grid only. Its origin is the upper left
% corner:
% \begin{quote}
% |\pagegridsetup{top-left}|
% \end{quote}
%
% \subsubsection{Grid unit}
%
% \begin{description}
% \item[\xoption{step}] This option takes a length and
% setups the unit for the grid. The page width and page height
% should be multiples of this unit.
% Currently the default is \texttt{1mm}. But this might change
% later by a heuristic based on the paper size.
% \end{description}
%
% \subsubsection{Color options}
%
% The basic grid lines are drawn as ultra thin help lines and is only
% drawn for the first grid.
% Each tenth and fiftyth line of the basic net is drawn thicker in a special
% color for the two grids.
% \begin{description}
% \item[\xoption{firstcolor}:] Color for the thicker lines and the arrows
% of the first grid. Default value is \texttt{red}.
% \item[\xoption{secondcolor}:] Color for the thicker lines and the arrows
% of the second grid. Default value is \texttt{blue}.
% \end{description}
% Use a color specification that package \xpackage{tikz} understands.
% (The grid is drawn with \xpackage{pgf}/\xpackage{tikz}.)
%
% \subsubsection{Arrow options}
%
% Arrows are put at the origin at the grid to show the grid start
% and the direction of the grid.
% \begin{description}
% \item[\xoption{arrows}:] This boolean option turns the arrows on or off.
% As default arrows are enabled.
% \item[\xoption{arrowlength}:] The length given as value is the
% length of the edge of a square at the origin within the
% arrow is put as diagonal. Default is 10 times the grid unit (10\,mm).
% The real arrow length is this length multiplied by $\sqrt2$.
% \end{description}
%
% \subsubsection{Miscellaneous options}
%
% \begin{description}
% \item[\xoption{double}:] The output page is doubled, one without page
% grid and the other with page grid. Possible values are shown in the
% following table:
% \begin{quote}
% \begin{tabular}{ll}
% Option & Meaning\\
% \hline
% |false| & Turns option off.\\
% |first| & Grid page comes first.\\
% |last| & Grid page comes after the page without grid.\\
% |true| & Same as |last|.\\
% \meta{no value} & Same as |true|.\\
% \end{tabular}
% \end{quote}
% \textbf{Note:}
% The double output of the page has side effects.
% All whatits are executed twice, for example: file writing
% and anchor setting. Some unwanted actions are catched such
% as multiple \cs{label} definitions, duplicate entries in
% the table of contents. For bookmarks, use package \xpackage{bookmarks}.
% \item[\xoption{foreground}:] Boolean option, default is \texttt{false}.
% Sometimes there might be elements on the page (e.g. large images)
% that hide the grid. Then option \xoption{foreground} puts the grids
% over the current output page.
% \end{description}
%
% \StopEventually{
% }
%
% \section{Implementation}
%    \begin{macrocode}
%<*package>
%    \end{macrocode}
%    Reload check, especially if the package is not used with \LaTeX.
%    \begin{macrocode}
\begingroup\catcode61\catcode48\catcode32=10\relax%
  \catcode13=5 % ^^M
  \endlinechar=13 %
  \catcode35=6 % #
  \catcode39=12 % '
  \catcode44=12 % ,
  \catcode45=12 % -
  \catcode46=12 % .
  \catcode58=12 % :
  \catcode64=11 % @
  \catcode123=1 % {
  \catcode125=2 % }
  \expandafter\let\expandafter\x\csname ver@pagegrid.sty\endcsname
  \ifx\x\relax % plain-TeX, first loading
  \else
    \def\empty{}%
    \ifx\x\empty % LaTeX, first loading,
      % variable is initialized, but \ProvidesPackage not yet seen
    \else
      \expandafter\ifx\csname PackageInfo\endcsname\relax
        \def\x#1#2{%
          \immediate\write-1{Package #1 Info: #2.}%
        }%
      \else
        \def\x#1#2{\PackageInfo{#1}{#2, stopped}}%
      \fi
      \x{pagegrid}{The package is already loaded}%
      \aftergroup\endinput
    \fi
  \fi
\endgroup%
%    \end{macrocode}
%    Package identification:
%    \begin{macrocode}
\begingroup\catcode61\catcode48\catcode32=10\relax%
  \catcode13=5 % ^^M
  \endlinechar=13 %
  \catcode35=6 % #
  \catcode39=12 % '
  \catcode40=12 % (
  \catcode41=12 % )
  \catcode44=12 % ,
  \catcode45=12 % -
  \catcode46=12 % .
  \catcode47=12 % /
  \catcode58=12 % :
  \catcode64=11 % @
  \catcode91=12 % [
  \catcode93=12 % ]
  \catcode123=1 % {
  \catcode125=2 % }
  \expandafter\ifx\csname ProvidesPackage\endcsname\relax
    \def\x#1#2#3[#4]{\endgroup
      \immediate\write-1{Package: #3 #4}%
      \xdef#1{#4}%
    }%
  \else
    \def\x#1#2[#3]{\endgroup
      #2[{#3}]%
      \ifx#1\@undefined
        \xdef#1{#3}%
      \fi
      \ifx#1\relax
        \xdef#1{#3}%
      \fi
    }%
  \fi
\expandafter\x\csname ver@pagegrid.sty\endcsname
\ProvidesPackage{pagegrid}%
  [2016/05/16 v1.5 Print page grid in background (HO)]%
%    \end{macrocode}
%
%    \begin{macrocode}
\begingroup\catcode61\catcode48\catcode32=10\relax%
  \catcode13=5 % ^^M
  \endlinechar=13 %
  \catcode123=1 % {
  \catcode125=2 % }
  \catcode64=11 % @
  \def\x{\endgroup
    \expandafter\edef\csname pagegrid@AtEnd\endcsname{%
      \endlinechar=\the\endlinechar\relax
      \catcode13=\the\catcode13\relax
      \catcode32=\the\catcode32\relax
      \catcode35=\the\catcode35\relax
      \catcode61=\the\catcode61\relax
      \catcode64=\the\catcode64\relax
      \catcode123=\the\catcode123\relax
      \catcode125=\the\catcode125\relax
    }%
  }%
\x\catcode61\catcode48\catcode32=10\relax%
\catcode13=5 % ^^M
\endlinechar=13 %
\catcode35=6 % #
\catcode64=11 % @
\catcode123=1 % {
\catcode125=2 % }
\def\TMP@EnsureCode#1#2{%
  \edef\pagegrid@AtEnd{%
    \pagegrid@AtEnd
    \catcode#1=\the\catcode#1\relax
  }%
  \catcode#1=#2\relax
}
\TMP@EnsureCode{9}{10}% (tab)
\TMP@EnsureCode{10}{12}% ^^J
\TMP@EnsureCode{33}{12}% !
\TMP@EnsureCode{34}{12}% "
\TMP@EnsureCode{36}{3}% $
\TMP@EnsureCode{38}{4}% &
\TMP@EnsureCode{39}{12}% '
\TMP@EnsureCode{40}{12}% (
\TMP@EnsureCode{41}{12}% )
\TMP@EnsureCode{42}{12}% *
\TMP@EnsureCode{43}{12}% +
\TMP@EnsureCode{44}{12}% ,
\TMP@EnsureCode{45}{12}% -
\TMP@EnsureCode{46}{12}% .
\TMP@EnsureCode{47}{12}% /
\TMP@EnsureCode{58}{12}% :
\TMP@EnsureCode{59}{12}% ;
\TMP@EnsureCode{60}{12}% <
\TMP@EnsureCode{62}{12}% >
\TMP@EnsureCode{63}{12}% ?
\TMP@EnsureCode{91}{12}% [
\TMP@EnsureCode{93}{12}% ]
\TMP@EnsureCode{94}{7}% ^ (superscript)
\TMP@EnsureCode{95}{8}% _ (subscript)
\TMP@EnsureCode{96}{12}% `
\TMP@EnsureCode{124}{12}% |
\edef\pagegrid@AtEnd{\pagegrid@AtEnd\noexpand\endinput}
%    \end{macrocode}
%
%    \begin{macrocode}
\RequirePackage{tikz}
\RequirePackage{atbegshi}[2009/12/02]
\RequirePackage{kvoptions}[2009/07/17]
%    \end{macrocode}
%    \begin{macrocode}
\begingroup\expandafter\expandafter\expandafter\endgroup
\expandafter\ifx\csname stockwidth\endcsname\relax
  \def\pagegrid@width{\paperwidth}%
  \def\pagegrid@height{\paperheight}%
\else
  \def\pagegrid@width{\stockwidth}%
  \def\pagegrid@height{\stockheight}%
\fi
%    \end{macrocode}
%
%    \begin{macrocode}
\SetupKeyvalOptions{%
  family=pagegrid,%
  prefix=pagegrid@,%
}
\def\pagegrid@init{%
  \let\pagegrid@origin@a\@empty
  \let\pagegrid@origin@b\@empty
  \let\pagegrid@init\relax
}
\let\pagegrid@@init\pagegrid@init
\def\pagegrid@origin@a{bl}
\def\pagegrid@origin@b{tr}
\def\pagegrid@SetOrigin#1{%
  \pagegrid@init
  \ifx\pagegrid@origin@a\@empty
    \def\pagegrid@origin@a{#1}%
  \else
    \ifx\pagegrid@origin@b\@empty
    \else
      \let\pagegrid@origin@a\pagegrid@origin@b
    \fi
    \def\pagegrid@origin@b{#1}%
  \fi
}
\def\pagegrid@temp#1{%
  \DeclareVoidOption{#1}{\pagegrid@SetOrigin{#1}}%
  \@namedef{pagegrid@N@#1}{#1}%
}
\pagegrid@temp{bl}
\pagegrid@temp{br}
\pagegrid@temp{tl}
\pagegrid@temp{tr}
\def\pagegrid@temp#1#2{%
  \DeclareVoidOption{#2}{\pagegrid@SetOrigin{#1}}%
}%
\pagegrid@temp{bl}{lb}
\pagegrid@temp{br}{rb}
\pagegrid@temp{tl}{lt}
\pagegrid@temp{tr}{rt}
\pagegrid@temp{bl}{bottom-left}
\pagegrid@temp{br}{bottom-right}
\pagegrid@temp{tl}{top-left}
\pagegrid@temp{tr}{top-right}
\pagegrid@temp{bl}{left-bottom}
\pagegrid@temp{br}{right-bottom}
\pagegrid@temp{tl}{left-top}
\pagegrid@temp{tr}{right-top}
%    \end{macrocode}
%    \begin{macrocode}
\DeclareBoolOption[true]{enable}
\DeclareComplementaryOption{disable}{enable}
%    \end{macrocode}
%    \begin{macrocode}
\DeclareBoolOption{foreground}
%    \end{macrocode}
%    \begin{macrocode}
\newlength{\pagegrid@step}
\define@key{pagegrid}{step}{%
  \setlength{\pagegrid@step}{#1}%
}
%    \end{macrocode}
%    \begin{macrocode}
\DeclareStringOption[red]{firstcolor}
\DeclareStringOption[blue]{secondcolor}
%    \end{macrocode}
%    \begin{macrocode}
\DeclareBoolOption[true]{arrows}
\newlength\pagegrid@arrowlength
\pagegrid@arrowlength=\z@
\define@key{pagegrid}{arrowlength}{%
  \setlength{\pagegrid@arrowlength}{#1}%
}
%    \end{macrocode}
%    \begin{macrocode}
\define@key{pagegrid}{double}[true]{%
  \@ifundefined{pagegrid@double@#1}{%
    \PackageWarning{pagegrid}{%
      Unsupported value `#1' for option `double'.\MessageBreak
      Known values are:\MessageBreak
      `false', `first', `last', `true'.\MessageBreak
      Now `false' is used%
    }%
    \chardef\pagegrid@double\z@
  }{%
    \chardef\pagegrid@double\csname pagegrid@double@#1\endcsname\relax
  }%
}
\@namedef{pagegrid@double@false}{0}
\@namedef{pagegrid@double@first}{1}
\@namedef{pagegrid@double@last}{2}
\@namedef{pagegrid@double@true}{2}
\chardef\pagegrid@double\z@
%    \end{macrocode}
%    \begin{macrocode}
\newcommand*{\pagegridsetup}{%
  \let\pagegrid@init\pagegrid@@init
  \setkeys{pagegrid}%
}
%    \end{macrocode}
%    \begin{macrocode}
\pagegridsetup{%
  step=1mm%
}
\InputIfFileExists{pagegrid.cfg}{}%
\ProcessKeyvalOptions*\relax
\AtBeginDocument{%
  \ifdim\pagegrid@arrowlength>\z@
  \else
    \pagegrid@arrowlength=10\pagegrid@step
  \fi
}
%    \end{macrocode}
%
%    \begin{macrocode}
\def\pagegridShipoutDoubleBegin{%
  \begingroup
  \let\newlabel\@gobbletwo
  \let\zref@newlabel\@gobbletwo
  \let\@writefile\@gobbletwo
  \let\select@language\@gobble
}
\def\pagegridShipoutDoubleEnd{%
  \endgroup
}
\def\pagegrid@WriteDouble#1#2{%
  \immediate\write#1{%
    \@backslashchar csname %
    pagegridShipoutDouble#2%
    \@backslashchar endcsname%
  }%
}
\def\pagegrid@ShipoutDouble#1{%
  \begingroup
    \if@filesw
      \pagegrid@WriteDouble\@mainaux{Begin}%
      \ifx\@auxout\@partaux
        \pagegrid@WriteDouble\@partaux{Begin}%
        \def\pagegrid@temp{%
          \pagegrid@WriteDouble\@mainaux{End}%
          \pagegrid@WriteDouble\@partaux{End}%
        }%
      \else
        \def\pagegrid@temp{%
          \pagegrid@WriteDouble\@mainaux{End}%
        }%
      \fi
    \else
      \def\pagegrid@temp{}%
    \fi
    \let\protect\noexpand
    \AtBeginShipoutOriginalShipout\copy#1\relax
    \pagegrid@temp
  \endgroup
}
%    \end{macrocode}
%
%    \begin{macrocode}
\AtBeginShipout{%
  \ifdim\pagegrid@step>\z@
  \else
    \pagegrid@enablefalse
  \fi
  \ifpagegrid@enable
    \ifnum\pagegrid@double=\@ne
      \pagegrid@ShipoutDouble\AtBeginShipoutBox
    \else
      \ifnum\pagegrid@double=\tw@
        \@ifundefined{pagegrid@DoubleBox}{%
          \newbox\pagegrid@DoubleBox
        }{}%
        \setbox\pagegrid@DoubleBox=\copy\AtBeginShipoutBox
      \fi
    \fi
    \ifpagegrid@foreground
      \expandafter\AtBeginShipoutUpperLeftForeground
    \else
      \expandafter\AtBeginShipoutUpperLeft
    \fi
    {%
      \put(0,0){%
        \makebox(0,0)[lt]{%
          \begin{tikzpicture}[%
            bl/.style={},%
            br/.style={xshift=\pagegrid@width,xscale=-1},%
            tl/.style={yshift=\pagegrid@height,yscale=-1},%
            tr/.style={xshift=\pagegrid@width,%
                       yshift=\pagegrid@height,scale=-1}%
          ]%
            \useasboundingbox
              (0mm,\pagegrid@height) rectangle (0mm,\pagegrid@height);%
            \draw[%
              \pagegrid@origin@a,%
              step=\pagegrid@step,%
              style=help lines,%
              ultra thin%
            ] (0mm,0mm) grid (\pagegrid@width,\pagegrid@height);%
            \ifx\pagegrid@origin@b\@empty
            \else
              \draw[%
                \pagegrid@origin@b,%
                step=10\pagegrid@step,%
                {\pagegrid@secondcolor},%
                very thin%
              ] (0mm,0mm) grid (\pagegrid@width,\pagegrid@height);%
            \fi
            \draw[%
               \pagegrid@origin@a,%
               step=10\pagegrid@step,%
               {\pagegrid@firstcolor},%
               very thin%
            ] (0mm,0mm) grid (\pagegrid@width,\pagegrid@height);%
            \ifx\pagegrid@origin@b\@empty
            \else
              \draw[%
                \pagegrid@origin@b,%
                step=50\pagegrid@step,%
                {\pagegrid@secondcolor},%
                thick%
              ] (0mm,0mm) grid (\pagegrid@width,\pagegrid@height);%
            \fi
            \draw[%
              \pagegrid@origin@a,%
              step=50\pagegrid@step,%
              {\pagegrid@firstcolor},%
              thick%
            ] (0mm,0mm) grid (\pagegrid@width,\pagegrid@height);%
            \ifpagegrid@arrows
              \ifx\pagegrid@origin@b\@empty
              \else
                \draw[%
                  \pagegrid@origin@b,%
                  {\pagegrid@secondcolor},%
                  stroke,%
                  line width=1pt,%
                  line cap=round%
                ] (0mm,0mm) %
                -- (\pagegrid@arrowlength,\pagegrid@arrowlength) %
                   (\pagegrid@arrowlength,.5\pagegrid@arrowlength) %
                -- (\pagegrid@arrowlength,\pagegrid@arrowlength) %
                -- (.5\pagegrid@arrowlength,\pagegrid@arrowlength);%
              \fi
              \draw[%
                \pagegrid@origin@a,%
                {\pagegrid@firstcolor},%
                stroke,%
                line width=1pt,%
                line cap=round%
              ] (0mm,0mm) %
              -- (\pagegrid@arrowlength,\pagegrid@arrowlength) %
                 (\pagegrid@arrowlength,.5\pagegrid@arrowlength) %
              -- (\pagegrid@arrowlength,\pagegrid@arrowlength) %
              -- (.5\pagegrid@arrowlength,\pagegrid@arrowlength);%
            \fi
          \end{tikzpicture}%
        }%
      }%
    }%
    \ifnum\pagegrid@double=\tw@
      \pagegrid@ShipoutDouble\pagegrid@DoubleBox
    \fi
  \fi
}
%    \end{macrocode}
%
%    \begin{macrocode}
\pagegrid@AtEnd%
%</package>
%    \end{macrocode}
%
% \section{Test}
%
% \subsection{Catcode checks for loading}
%
%    \begin{macrocode}
%<*test1>
%    \end{macrocode}
%    \begin{macrocode}
\catcode`\{=1 %
\catcode`\}=2 %
\catcode`\#=6 %
\catcode`\@=11 %
\expandafter\ifx\csname count@\endcsname\relax
  \countdef\count@=255 %
\fi
\expandafter\ifx\csname @gobble\endcsname\relax
  \long\def\@gobble#1{}%
\fi
\expandafter\ifx\csname @firstofone\endcsname\relax
  \long\def\@firstofone#1{#1}%
\fi
\expandafter\ifx\csname loop\endcsname\relax
  \expandafter\@firstofone
\else
  \expandafter\@gobble
\fi
{%
  \def\loop#1\repeat{%
    \def\body{#1}%
    \iterate
  }%
  \def\iterate{%
    \body
      \let\next\iterate
    \else
      \let\next\relax
    \fi
    \next
  }%
  \let\repeat=\fi
}%
\def\RestoreCatcodes{}
\count@=0 %
\loop
  \edef\RestoreCatcodes{%
    \RestoreCatcodes
    \catcode\the\count@=\the\catcode\count@\relax
  }%
\ifnum\count@<255 %
  \advance\count@ 1 %
\repeat

\def\RangeCatcodeInvalid#1#2{%
  \count@=#1\relax
  \loop
    \catcode\count@=15 %
  \ifnum\count@<#2\relax
    \advance\count@ 1 %
  \repeat
}
\def\RangeCatcodeCheck#1#2#3{%
  \count@=#1\relax
  \loop
    \ifnum#3=\catcode\count@
    \else
      \errmessage{%
        Character \the\count@\space
        with wrong catcode \the\catcode\count@\space
        instead of \number#3%
      }%
    \fi
  \ifnum\count@<#2\relax
    \advance\count@ 1 %
  \repeat
}
\def\space{ }
\expandafter\ifx\csname LoadCommand\endcsname\relax
  \def\LoadCommand{\input pagegrid.sty\relax}%
\fi
\def\Test{%
  \RangeCatcodeInvalid{0}{47}%
  \RangeCatcodeInvalid{58}{64}%
  \RangeCatcodeInvalid{91}{96}%
  \RangeCatcodeInvalid{123}{255}%
  \catcode`\@=12 %
  \catcode`\\=0 %
  \catcode`\%=14 %
  \LoadCommand
  \RangeCatcodeCheck{0}{36}{15}%
  \RangeCatcodeCheck{37}{37}{14}%
  \RangeCatcodeCheck{38}{47}{15}%
  \RangeCatcodeCheck{48}{57}{12}%
  \RangeCatcodeCheck{58}{63}{15}%
  \RangeCatcodeCheck{64}{64}{12}%
  \RangeCatcodeCheck{65}{90}{11}%
  \RangeCatcodeCheck{91}{91}{15}%
  \RangeCatcodeCheck{92}{92}{0}%
  \RangeCatcodeCheck{93}{96}{15}%
  \RangeCatcodeCheck{97}{122}{11}%
  \RangeCatcodeCheck{123}{255}{15}%
  \RestoreCatcodes
}
\Test
\csname @@end\endcsname
\end
%    \end{macrocode}
%    \begin{macrocode}
%</test1>
%    \end{macrocode}
%
% \section{Installation}
%
% \subsection{Download}
%
% \paragraph{Package.} This package is available on
% CTAN\footnote{\CTANpkg{pagegrid}}:
% \begin{description}
% \item[\CTAN{macros/latex/contrib/oberdiek/pagegrid.dtx}] The source file.
% \item[\CTAN{macros/latex/contrib/oberdiek/pagegrid.pdf}] Documentation.
% \end{description}
%
%
% \paragraph{Bundle.} All the packages of the bundle `oberdiek'
% are also available in a TDS compliant ZIP archive. There
% the packages are already unpacked and the documentation files
% are generated. The files and directories obey the TDS standard.
% \begin{description}
% \item[\CTANinstall{install/macros/latex/contrib/oberdiek.tds.zip}]
% \end{description}
% \emph{TDS} refers to the standard ``A Directory Structure
% for \TeX\ Files'' (\CTAN{tds/tds.pdf}). Directories
% with \xfile{texmf} in their name are usually organized this way.
%
% \subsection{Bundle installation}
%
% \paragraph{Unpacking.} Unpack the \xfile{oberdiek.tds.zip} in the
% TDS tree (also known as \xfile{texmf} tree) of your choice.
% Example (linux):
% \begin{quote}
%   |unzip oberdiek.tds.zip -d ~/texmf|
% \end{quote}
%
% \paragraph{Script installation.}
% Check the directory \xfile{TDS:scripts/oberdiek/} for
% scripts that need further installation steps.
%
% \subsection{Package installation}
%
% \paragraph{Unpacking.} The \xfile{.dtx} file is a self-extracting
% \docstrip\ archive. The files are extracted by running the
% \xfile{.dtx} through \plainTeX:
% \begin{quote}
%   \verb|tex pagegrid.dtx|
% \end{quote}
%
% \paragraph{TDS.} Now the different files must be moved into
% the different directories in your installation TDS tree
% (also known as \xfile{texmf} tree):
% \begin{quote}
% \def\t{^^A
% \begin{tabular}{@{}>{\ttfamily}l@{ $\rightarrow$ }>{\ttfamily}l@{}}
%   pagegrid.sty & tex/latex/oberdiek/pagegrid.sty\\
%   pagegrid.pdf & doc/latex/oberdiek/pagegrid.pdf\\
%   test/pagegrid-test1.tex & doc/latex/oberdiek/test/pagegrid-test1.tex\\
%   pagegrid.dtx & source/latex/oberdiek/pagegrid.dtx\\
% \end{tabular}^^A
% }^^A
% \sbox0{\t}^^A
% \ifdim\wd0>\linewidth
%   \begingroup
%     \advance\linewidth by\leftmargin
%     \advance\linewidth by\rightmargin
%   \edef\x{\endgroup
%     \def\noexpand\lw{\the\linewidth}^^A
%   }\x
%   \def\lwbox{^^A
%     \leavevmode
%     \hbox to \linewidth{^^A
%       \kern-\leftmargin\relax
%       \hss
%       \usebox0
%       \hss
%       \kern-\rightmargin\relax
%     }^^A
%   }^^A
%   \ifdim\wd0>\lw
%     \sbox0{\small\t}^^A
%     \ifdim\wd0>\linewidth
%       \ifdim\wd0>\lw
%         \sbox0{\footnotesize\t}^^A
%         \ifdim\wd0>\linewidth
%           \ifdim\wd0>\lw
%             \sbox0{\scriptsize\t}^^A
%             \ifdim\wd0>\linewidth
%               \ifdim\wd0>\lw
%                 \sbox0{\tiny\t}^^A
%                 \ifdim\wd0>\linewidth
%                   \lwbox
%                 \else
%                   \usebox0
%                 \fi
%               \else
%                 \lwbox
%               \fi
%             \else
%               \usebox0
%             \fi
%           \else
%             \lwbox
%           \fi
%         \else
%           \usebox0
%         \fi
%       \else
%         \lwbox
%       \fi
%     \else
%       \usebox0
%     \fi
%   \else
%     \lwbox
%   \fi
% \else
%   \usebox0
% \fi
% \end{quote}
% If you have a \xfile{docstrip.cfg} that configures and enables \docstrip's
% TDS installing feature, then some files can already be in the right
% place, see the documentation of \docstrip.
%
% \subsection{Refresh file name databases}
%
% If your \TeX~distribution
% (\TeX\,Live, \mikTeX, \dots) relies on file name databases, you must refresh
% these. For example, \TeX\,Live\ users run \verb|texhash| or
% \verb|mktexlsr|.
%
% \subsection{Some details for the interested}
%
% \paragraph{Unpacking with \LaTeX.}
% The \xfile{.dtx} chooses its action depending on the format:
% \begin{description}
% \item[\plainTeX:] Run \docstrip\ and extract the files.
% \item[\LaTeX:] Generate the documentation.
% \end{description}
% If you insist on using \LaTeX\ for \docstrip\ (really,
% \docstrip\ does not need \LaTeX), then inform the autodetect routine
% about your intention:
% \begin{quote}
%   \verb|latex \let\install=y% \iffalse meta-comment
%
% File: pagegrid.dtx
% Version: 2016/05/16 v1.5
% Info: Print page grid in background
%
% Copyright (C) 2009 by
%    Heiko Oberdiek <heiko.oberdiek at googlemail.com>
%    2016
%    https://github.com/ho-tex/oberdiek/issues
%
% This work may be distributed and/or modified under the
% conditions of the LaTeX Project Public License, either
% version 1.3c of this license or (at your option) any later
% version. This version of this license is in
%    https://www.latex-project.org/lppl/lppl-1-3c.txt
% and the latest version of this license is in
%    https://www.latex-project.org/lppl.txt
% and version 1.3 or later is part of all distributions of
% LaTeX version 2005/12/01 or later.
%
% This work has the LPPL maintenance status "maintained".
%
% The Current Maintainers of this work are
% Heiko Oberdiek and the Oberdiek Package Support Group
% https://github.com/ho-tex/oberdiek/issues
%
% This work consists of the main source file pagegrid.dtx
% and the derived files
%    pagegrid.sty, pagegrid.pdf, pagegrid.ins, pagegrid.drv,
%    pagegrid-test1.tex.
%
% Distribution:
%    CTAN:macros/latex/contrib/oberdiek/pagegrid.dtx
%    CTAN:macros/latex/contrib/oberdiek/pagegrid.pdf
%
% Unpacking:
%    (a) If pagegrid.ins is present:
%           tex pagegrid.ins
%    (b) Without pagegrid.ins:
%           tex pagegrid.dtx
%    (c) If you insist on using LaTeX
%           latex \let\install=y\input{pagegrid.dtx}
%        (quote the arguments according to the demands of your shell)
%
% Documentation:
%    (a) If pagegrid.drv is present:
%           latex pagegrid.drv
%    (b) Without pagegrid.drv:
%           latex pagegrid.dtx; ...
%    The class ltxdoc loads the configuration file ltxdoc.cfg
%    if available. Here you can specify further options, e.g.
%    use A4 as paper format:
%       \PassOptionsToClass{a4paper}{article}
%
%    Programm calls to get the documentation (example):
%       pdflatex pagegrid.dtx
%       makeindex -s gind.ist pagegrid.idx
%       pdflatex pagegrid.dtx
%       makeindex -s gind.ist pagegrid.idx
%       pdflatex pagegrid.dtx
%
% Installation:
%    TDS:tex/latex/oberdiek/pagegrid.sty
%    TDS:doc/latex/oberdiek/pagegrid.pdf
%    TDS:doc/latex/oberdiek/test/pagegrid-test1.tex
%    TDS:source/latex/oberdiek/pagegrid.dtx
%
%<*ignore>
\begingroup
  \catcode123=1 %
  \catcode125=2 %
  \def\x{LaTeX2e}%
\expandafter\endgroup
\ifcase 0\ifx\install y1\fi\expandafter
         \ifx\csname processbatchFile\endcsname\relax\else1\fi
         \ifx\fmtname\x\else 1\fi\relax
\else\csname fi\endcsname
%</ignore>
%<*install>
\input docstrip.tex
\Msg{************************************************************************}
\Msg{* Installation}
\Msg{* Package: pagegrid 2016/05/16 v1.5 Print page grid in background (HO)}
\Msg{************************************************************************}

\keepsilent
\askforoverwritefalse

\let\MetaPrefix\relax
\preamble

This is a generated file.

Project: pagegrid
Version: 2016/05/16 v1.5

Copyright (C) 2009 by
   Heiko Oberdiek <heiko.oberdiek at googlemail.com>

This work may be distributed and/or modified under the
conditions of the LaTeX Project Public License, either
version 1.3c of this license or (at your option) any later
version. This version of this license is in
   https://www.latex-project.org/lppl/lppl-1-3c.txt
and the latest version of this license is in
   https://www.latex-project.org/lppl.txt
and version 1.3 or later is part of all distributions of
LaTeX version 2005/12/01 or later.

This work has the LPPL maintenance status "maintained".

The Current Maintainers of this work are
Heiko Oberdiek and the Oberdiek Package Support Group
https://github.com/ho-tex/oberdiek/issues


This work consists of the main source file pagegrid.dtx
and the derived files
   pagegrid.sty, pagegrid.pdf, pagegrid.ins, pagegrid.drv,
   pagegrid-test1.tex.

\endpreamble
\let\MetaPrefix\DoubleperCent

\generate{%
  \file{pagegrid.ins}{\from{pagegrid.dtx}{install}}%
  \file{pagegrid.drv}{\from{pagegrid.dtx}{driver}}%
  \usedir{tex/latex/oberdiek}%
  \file{pagegrid.sty}{\from{pagegrid.dtx}{package}}%
%  \usedir{doc/latex/oberdiek/test}%
%  \file{pagegrid-test1.tex}{\from{pagegrid.dtx}{test1}}%
  \nopreamble
  \nopostamble
%  \usedir{source/latex/oberdiek/catalogue}%
%  \file{pagegrid.xml}{\from{pagegrid.dtx}{catalogue}}%
}

\catcode32=13\relax% active space
\let =\space%
\Msg{************************************************************************}
\Msg{*}
\Msg{* To finish the installation you have to move the following}
\Msg{* file into a directory searched by TeX:}
\Msg{*}
\Msg{*     pagegrid.sty}
\Msg{*}
\Msg{* To produce the documentation run the file `pagegrid.drv'}
\Msg{* through LaTeX.}
\Msg{*}
\Msg{* Happy TeXing!}
\Msg{*}
\Msg{************************************************************************}

\endbatchfile
%</install>
%<*ignore>
\fi
%</ignore>
%<*driver>
\NeedsTeXFormat{LaTeX2e}
\ProvidesFile{pagegrid.drv}%
  [2016/05/16 v1.5 Print page grid in background (HO)]%
\documentclass{ltxdoc}
\usepackage{holtxdoc}[2011/11/22]
\begin{document}
  \DocInput{pagegrid.dtx}%
\end{document}
%</driver>
% \fi
%
%
% \CharacterTable
%  {Upper-case    \A\B\C\D\E\F\G\H\I\J\K\L\M\N\O\P\Q\R\S\T\U\V\W\X\Y\Z
%   Lower-case    \a\b\c\d\e\f\g\h\i\j\k\l\m\n\o\p\q\r\s\t\u\v\w\x\y\z
%   Digits        \0\1\2\3\4\5\6\7\8\9
%   Exclamation   \!     Double quote  \"     Hash (number) \#
%   Dollar        \$     Percent       \%     Ampersand     \&
%   Acute accent  \'     Left paren    \(     Right paren   \)
%   Asterisk      \*     Plus          \+     Comma         \,
%   Minus         \-     Point         \.     Solidus       \/
%   Colon         \:     Semicolon     \;     Less than     \<
%   Equals        \=     Greater than  \>     Question mark \?
%   Commercial at \@     Left bracket  \[     Backslash     \\
%   Right bracket \]     Circumflex    \^     Underscore    \_
%   Grave accent  \`     Left brace    \{     Vertical bar  \|
%   Right brace   \}     Tilde         \~}
%
% \GetFileInfo{pagegrid.drv}
%
% \title{The \xpackage{pagegrid} package}
% \date{2016/05/16 v1.5}
% \author{Heiko Oberdiek\thanks
% {Please report any issues at \url{https://github.com/ho-tex/oberdiek/issues}}}
%
% \maketitle
%
% \begin{abstract}
% The \LaTeX\ package prints a page grid in the background.
% \end{abstract}
%
% \tableofcontents
%
% \section{Documentation}
%
% The package puts a grid on the paper. It was written for
% developers of a class or package
% who have to put elements on definite locations on a page
% (e.g. letter class). The grid allows a faster optical check,
% whether the positions are correct. If the previewer already
% offers features for measuring, the package might be obsolete.
% Otherwise it saves the developer from printing the page and
% measuring by hand.
%
% \subsection{Options}
%
% Options are evaluated in the following order:
% \begin{enumerate}
% \item
%  Configuration file \xfile{pagegrid.cfg} using \cs{pagegridsetup}
%  if the file exists.
%  \item
%  Package options given for \cs{usepackage}.
%  \item
%  Later calls of \cs{pagegridsetup}.
% \end{enumerate}
% \begin{declcs}{pagegridsetup}\M{option list}
% \end{declcs}
% The options are key value options. Boolean options are enabled by
% default (without value) or by using the explicit value \texttt{true}.
% Value \texttt{false} disable the option.
%
% \subsubsection{Options \xoption{enable}, \xoption{disable}}
%
% \begin{description}
% \item[\xoption{enable}:] This boolean option controls whether the page grid
%   is drawn. As default the page grid drawing is activated.
% \item[\xoption{disable}:] It is the opposite
%   of option \xoption{enable}. It was added for convenience and
%   allows the abbreviation \texttt{disable} for \texttt{enable=false}.
% \end{description}
%
% \subsubsection{Grid origins}
%
% The package supports up to two grids on a page allowing
% measurement from opposite directions. As default two grids are drawn,
% the first from bottom left to top right. The origin of the second
% grid is at the opposite top right corner.
% The origins are controlled by the following options.
% The number of grids (one or two) depend on the number of these options
% in one call of \cs{pagegridsetup}.
% The following frame shows a paper and in its corners are the
% corresponding options. At the left and right side alias names
% are given for the options inside the paper.
% \begin{quote}
% \begin{tabular}{@{}r|@{\,}l@{\qquad}r@{\,}|l@{}}
% \cline{2-3}
% \xoption{left-top}, \xoption{lt}, \xoption{top-left}
% & \vphantom{\"U}\xoption{tl} & \xoption{tr}
% & \xoption{top-right}, \xoption{rt}, \xoption{right-top}\\
% &&&\\
% \xoption{left-bottom}, \xoption{lb}, \xoption{bottom-left}
% & \xoption{bl} & \xoption{br}
% & \xoption{bottom-right}, \xoption{rb}, \xoption{right-bottom}\\
% \cline{2-3}
% \end{tabular}
% \end{quote}
% Examples:
% \begin{quote}
% |\pagegridsetup{bl,tr}|
% \end{quote}
% This is the default setting with two grids as described previously.
% The following setups one grid only. Its origin is the upper left
% corner:
% \begin{quote}
% |\pagegridsetup{top-left}|
% \end{quote}
%
% \subsubsection{Grid unit}
%
% \begin{description}
% \item[\xoption{step}] This option takes a length and
% setups the unit for the grid. The page width and page height
% should be multiples of this unit.
% Currently the default is \texttt{1mm}. But this might change
% later by a heuristic based on the paper size.
% \end{description}
%
% \subsubsection{Color options}
%
% The basic grid lines are drawn as ultra thin help lines and is only
% drawn for the first grid.
% Each tenth and fiftyth line of the basic net is drawn thicker in a special
% color for the two grids.
% \begin{description}
% \item[\xoption{firstcolor}:] Color for the thicker lines and the arrows
% of the first grid. Default value is \texttt{red}.
% \item[\xoption{secondcolor}:] Color for the thicker lines and the arrows
% of the second grid. Default value is \texttt{blue}.
% \end{description}
% Use a color specification that package \xpackage{tikz} understands.
% (The grid is drawn with \xpackage{pgf}/\xpackage{tikz}.)
%
% \subsubsection{Arrow options}
%
% Arrows are put at the origin at the grid to show the grid start
% and the direction of the grid.
% \begin{description}
% \item[\xoption{arrows}:] This boolean option turns the arrows on or off.
% As default arrows are enabled.
% \item[\xoption{arrowlength}:] The length given as value is the
% length of the edge of a square at the origin within the
% arrow is put as diagonal. Default is 10 times the grid unit (10\,mm).
% The real arrow length is this length multiplied by $\sqrt2$.
% \end{description}
%
% \subsubsection{Miscellaneous options}
%
% \begin{description}
% \item[\xoption{double}:] The output page is doubled, one without page
% grid and the other with page grid. Possible values are shown in the
% following table:
% \begin{quote}
% \begin{tabular}{ll}
% Option & Meaning\\
% \hline
% |false| & Turns option off.\\
% |first| & Grid page comes first.\\
% |last| & Grid page comes after the page without grid.\\
% |true| & Same as |last|.\\
% \meta{no value} & Same as |true|.\\
% \end{tabular}
% \end{quote}
% \textbf{Note:}
% The double output of the page has side effects.
% All whatits are executed twice, for example: file writing
% and anchor setting. Some unwanted actions are catched such
% as multiple \cs{label} definitions, duplicate entries in
% the table of contents. For bookmarks, use package \xpackage{bookmarks}.
% \item[\xoption{foreground}:] Boolean option, default is \texttt{false}.
% Sometimes there might be elements on the page (e.g. large images)
% that hide the grid. Then option \xoption{foreground} puts the grids
% over the current output page.
% \end{description}
%
% \StopEventually{
% }
%
% \section{Implementation}
%    \begin{macrocode}
%<*package>
%    \end{macrocode}
%    Reload check, especially if the package is not used with \LaTeX.
%    \begin{macrocode}
\begingroup\catcode61\catcode48\catcode32=10\relax%
  \catcode13=5 % ^^M
  \endlinechar=13 %
  \catcode35=6 % #
  \catcode39=12 % '
  \catcode44=12 % ,
  \catcode45=12 % -
  \catcode46=12 % .
  \catcode58=12 % :
  \catcode64=11 % @
  \catcode123=1 % {
  \catcode125=2 % }
  \expandafter\let\expandafter\x\csname ver@pagegrid.sty\endcsname
  \ifx\x\relax % plain-TeX, first loading
  \else
    \def\empty{}%
    \ifx\x\empty % LaTeX, first loading,
      % variable is initialized, but \ProvidesPackage not yet seen
    \else
      \expandafter\ifx\csname PackageInfo\endcsname\relax
        \def\x#1#2{%
          \immediate\write-1{Package #1 Info: #2.}%
        }%
      \else
        \def\x#1#2{\PackageInfo{#1}{#2, stopped}}%
      \fi
      \x{pagegrid}{The package is already loaded}%
      \aftergroup\endinput
    \fi
  \fi
\endgroup%
%    \end{macrocode}
%    Package identification:
%    \begin{macrocode}
\begingroup\catcode61\catcode48\catcode32=10\relax%
  \catcode13=5 % ^^M
  \endlinechar=13 %
  \catcode35=6 % #
  \catcode39=12 % '
  \catcode40=12 % (
  \catcode41=12 % )
  \catcode44=12 % ,
  \catcode45=12 % -
  \catcode46=12 % .
  \catcode47=12 % /
  \catcode58=12 % :
  \catcode64=11 % @
  \catcode91=12 % [
  \catcode93=12 % ]
  \catcode123=1 % {
  \catcode125=2 % }
  \expandafter\ifx\csname ProvidesPackage\endcsname\relax
    \def\x#1#2#3[#4]{\endgroup
      \immediate\write-1{Package: #3 #4}%
      \xdef#1{#4}%
    }%
  \else
    \def\x#1#2[#3]{\endgroup
      #2[{#3}]%
      \ifx#1\@undefined
        \xdef#1{#3}%
      \fi
      \ifx#1\relax
        \xdef#1{#3}%
      \fi
    }%
  \fi
\expandafter\x\csname ver@pagegrid.sty\endcsname
\ProvidesPackage{pagegrid}%
  [2016/05/16 v1.5 Print page grid in background (HO)]%
%    \end{macrocode}
%
%    \begin{macrocode}
\begingroup\catcode61\catcode48\catcode32=10\relax%
  \catcode13=5 % ^^M
  \endlinechar=13 %
  \catcode123=1 % {
  \catcode125=2 % }
  \catcode64=11 % @
  \def\x{\endgroup
    \expandafter\edef\csname pagegrid@AtEnd\endcsname{%
      \endlinechar=\the\endlinechar\relax
      \catcode13=\the\catcode13\relax
      \catcode32=\the\catcode32\relax
      \catcode35=\the\catcode35\relax
      \catcode61=\the\catcode61\relax
      \catcode64=\the\catcode64\relax
      \catcode123=\the\catcode123\relax
      \catcode125=\the\catcode125\relax
    }%
  }%
\x\catcode61\catcode48\catcode32=10\relax%
\catcode13=5 % ^^M
\endlinechar=13 %
\catcode35=6 % #
\catcode64=11 % @
\catcode123=1 % {
\catcode125=2 % }
\def\TMP@EnsureCode#1#2{%
  \edef\pagegrid@AtEnd{%
    \pagegrid@AtEnd
    \catcode#1=\the\catcode#1\relax
  }%
  \catcode#1=#2\relax
}
\TMP@EnsureCode{9}{10}% (tab)
\TMP@EnsureCode{10}{12}% ^^J
\TMP@EnsureCode{33}{12}% !
\TMP@EnsureCode{34}{12}% "
\TMP@EnsureCode{36}{3}% $
\TMP@EnsureCode{38}{4}% &
\TMP@EnsureCode{39}{12}% '
\TMP@EnsureCode{40}{12}% (
\TMP@EnsureCode{41}{12}% )
\TMP@EnsureCode{42}{12}% *
\TMP@EnsureCode{43}{12}% +
\TMP@EnsureCode{44}{12}% ,
\TMP@EnsureCode{45}{12}% -
\TMP@EnsureCode{46}{12}% .
\TMP@EnsureCode{47}{12}% /
\TMP@EnsureCode{58}{12}% :
\TMP@EnsureCode{59}{12}% ;
\TMP@EnsureCode{60}{12}% <
\TMP@EnsureCode{62}{12}% >
\TMP@EnsureCode{63}{12}% ?
\TMP@EnsureCode{91}{12}% [
\TMP@EnsureCode{93}{12}% ]
\TMP@EnsureCode{94}{7}% ^ (superscript)
\TMP@EnsureCode{95}{8}% _ (subscript)
\TMP@EnsureCode{96}{12}% `
\TMP@EnsureCode{124}{12}% |
\edef\pagegrid@AtEnd{\pagegrid@AtEnd\noexpand\endinput}
%    \end{macrocode}
%
%    \begin{macrocode}
\RequirePackage{tikz}
\RequirePackage{atbegshi}[2009/12/02]
\RequirePackage{kvoptions}[2009/07/17]
%    \end{macrocode}
%    \begin{macrocode}
\begingroup\expandafter\expandafter\expandafter\endgroup
\expandafter\ifx\csname stockwidth\endcsname\relax
  \def\pagegrid@width{\paperwidth}%
  \def\pagegrid@height{\paperheight}%
\else
  \def\pagegrid@width{\stockwidth}%
  \def\pagegrid@height{\stockheight}%
\fi
%    \end{macrocode}
%
%    \begin{macrocode}
\SetupKeyvalOptions{%
  family=pagegrid,%
  prefix=pagegrid@,%
}
\def\pagegrid@init{%
  \let\pagegrid@origin@a\@empty
  \let\pagegrid@origin@b\@empty
  \let\pagegrid@init\relax
}
\let\pagegrid@@init\pagegrid@init
\def\pagegrid@origin@a{bl}
\def\pagegrid@origin@b{tr}
\def\pagegrid@SetOrigin#1{%
  \pagegrid@init
  \ifx\pagegrid@origin@a\@empty
    \def\pagegrid@origin@a{#1}%
  \else
    \ifx\pagegrid@origin@b\@empty
    \else
      \let\pagegrid@origin@a\pagegrid@origin@b
    \fi
    \def\pagegrid@origin@b{#1}%
  \fi
}
\def\pagegrid@temp#1{%
  \DeclareVoidOption{#1}{\pagegrid@SetOrigin{#1}}%
  \@namedef{pagegrid@N@#1}{#1}%
}
\pagegrid@temp{bl}
\pagegrid@temp{br}
\pagegrid@temp{tl}
\pagegrid@temp{tr}
\def\pagegrid@temp#1#2{%
  \DeclareVoidOption{#2}{\pagegrid@SetOrigin{#1}}%
}%
\pagegrid@temp{bl}{lb}
\pagegrid@temp{br}{rb}
\pagegrid@temp{tl}{lt}
\pagegrid@temp{tr}{rt}
\pagegrid@temp{bl}{bottom-left}
\pagegrid@temp{br}{bottom-right}
\pagegrid@temp{tl}{top-left}
\pagegrid@temp{tr}{top-right}
\pagegrid@temp{bl}{left-bottom}
\pagegrid@temp{br}{right-bottom}
\pagegrid@temp{tl}{left-top}
\pagegrid@temp{tr}{right-top}
%    \end{macrocode}
%    \begin{macrocode}
\DeclareBoolOption[true]{enable}
\DeclareComplementaryOption{disable}{enable}
%    \end{macrocode}
%    \begin{macrocode}
\DeclareBoolOption{foreground}
%    \end{macrocode}
%    \begin{macrocode}
\newlength{\pagegrid@step}
\define@key{pagegrid}{step}{%
  \setlength{\pagegrid@step}{#1}%
}
%    \end{macrocode}
%    \begin{macrocode}
\DeclareStringOption[red]{firstcolor}
\DeclareStringOption[blue]{secondcolor}
%    \end{macrocode}
%    \begin{macrocode}
\DeclareBoolOption[true]{arrows}
\newlength\pagegrid@arrowlength
\pagegrid@arrowlength=\z@
\define@key{pagegrid}{arrowlength}{%
  \setlength{\pagegrid@arrowlength}{#1}%
}
%    \end{macrocode}
%    \begin{macrocode}
\define@key{pagegrid}{double}[true]{%
  \@ifundefined{pagegrid@double@#1}{%
    \PackageWarning{pagegrid}{%
      Unsupported value `#1' for option `double'.\MessageBreak
      Known values are:\MessageBreak
      `false', `first', `last', `true'.\MessageBreak
      Now `false' is used%
    }%
    \chardef\pagegrid@double\z@
  }{%
    \chardef\pagegrid@double\csname pagegrid@double@#1\endcsname\relax
  }%
}
\@namedef{pagegrid@double@false}{0}
\@namedef{pagegrid@double@first}{1}
\@namedef{pagegrid@double@last}{2}
\@namedef{pagegrid@double@true}{2}
\chardef\pagegrid@double\z@
%    \end{macrocode}
%    \begin{macrocode}
\newcommand*{\pagegridsetup}{%
  \let\pagegrid@init\pagegrid@@init
  \setkeys{pagegrid}%
}
%    \end{macrocode}
%    \begin{macrocode}
\pagegridsetup{%
  step=1mm%
}
\InputIfFileExists{pagegrid.cfg}{}%
\ProcessKeyvalOptions*\relax
\AtBeginDocument{%
  \ifdim\pagegrid@arrowlength>\z@
  \else
    \pagegrid@arrowlength=10\pagegrid@step
  \fi
}
%    \end{macrocode}
%
%    \begin{macrocode}
\def\pagegridShipoutDoubleBegin{%
  \begingroup
  \let\newlabel\@gobbletwo
  \let\zref@newlabel\@gobbletwo
  \let\@writefile\@gobbletwo
  \let\select@language\@gobble
}
\def\pagegridShipoutDoubleEnd{%
  \endgroup
}
\def\pagegrid@WriteDouble#1#2{%
  \immediate\write#1{%
    \@backslashchar csname %
    pagegridShipoutDouble#2%
    \@backslashchar endcsname%
  }%
}
\def\pagegrid@ShipoutDouble#1{%
  \begingroup
    \if@filesw
      \pagegrid@WriteDouble\@mainaux{Begin}%
      \ifx\@auxout\@partaux
        \pagegrid@WriteDouble\@partaux{Begin}%
        \def\pagegrid@temp{%
          \pagegrid@WriteDouble\@mainaux{End}%
          \pagegrid@WriteDouble\@partaux{End}%
        }%
      \else
        \def\pagegrid@temp{%
          \pagegrid@WriteDouble\@mainaux{End}%
        }%
      \fi
    \else
      \def\pagegrid@temp{}%
    \fi
    \let\protect\noexpand
    \AtBeginShipoutOriginalShipout\copy#1\relax
    \pagegrid@temp
  \endgroup
}
%    \end{macrocode}
%
%    \begin{macrocode}
\AtBeginShipout{%
  \ifdim\pagegrid@step>\z@
  \else
    \pagegrid@enablefalse
  \fi
  \ifpagegrid@enable
    \ifnum\pagegrid@double=\@ne
      \pagegrid@ShipoutDouble\AtBeginShipoutBox
    \else
      \ifnum\pagegrid@double=\tw@
        \@ifundefined{pagegrid@DoubleBox}{%
          \newbox\pagegrid@DoubleBox
        }{}%
        \setbox\pagegrid@DoubleBox=\copy\AtBeginShipoutBox
      \fi
    \fi
    \ifpagegrid@foreground
      \expandafter\AtBeginShipoutUpperLeftForeground
    \else
      \expandafter\AtBeginShipoutUpperLeft
    \fi
    {%
      \put(0,0){%
        \makebox(0,0)[lt]{%
          \begin{tikzpicture}[%
            bl/.style={},%
            br/.style={xshift=\pagegrid@width,xscale=-1},%
            tl/.style={yshift=\pagegrid@height,yscale=-1},%
            tr/.style={xshift=\pagegrid@width,%
                       yshift=\pagegrid@height,scale=-1}%
          ]%
            \useasboundingbox
              (0mm,\pagegrid@height) rectangle (0mm,\pagegrid@height);%
            \draw[%
              \pagegrid@origin@a,%
              step=\pagegrid@step,%
              style=help lines,%
              ultra thin%
            ] (0mm,0mm) grid (\pagegrid@width,\pagegrid@height);%
            \ifx\pagegrid@origin@b\@empty
            \else
              \draw[%
                \pagegrid@origin@b,%
                step=10\pagegrid@step,%
                {\pagegrid@secondcolor},%
                very thin%
              ] (0mm,0mm) grid (\pagegrid@width,\pagegrid@height);%
            \fi
            \draw[%
               \pagegrid@origin@a,%
               step=10\pagegrid@step,%
               {\pagegrid@firstcolor},%
               very thin%
            ] (0mm,0mm) grid (\pagegrid@width,\pagegrid@height);%
            \ifx\pagegrid@origin@b\@empty
            \else
              \draw[%
                \pagegrid@origin@b,%
                step=50\pagegrid@step,%
                {\pagegrid@secondcolor},%
                thick%
              ] (0mm,0mm) grid (\pagegrid@width,\pagegrid@height);%
            \fi
            \draw[%
              \pagegrid@origin@a,%
              step=50\pagegrid@step,%
              {\pagegrid@firstcolor},%
              thick%
            ] (0mm,0mm) grid (\pagegrid@width,\pagegrid@height);%
            \ifpagegrid@arrows
              \ifx\pagegrid@origin@b\@empty
              \else
                \draw[%
                  \pagegrid@origin@b,%
                  {\pagegrid@secondcolor},%
                  stroke,%
                  line width=1pt,%
                  line cap=round%
                ] (0mm,0mm) %
                -- (\pagegrid@arrowlength,\pagegrid@arrowlength) %
                   (\pagegrid@arrowlength,.5\pagegrid@arrowlength) %
                -- (\pagegrid@arrowlength,\pagegrid@arrowlength) %
                -- (.5\pagegrid@arrowlength,\pagegrid@arrowlength);%
              \fi
              \draw[%
                \pagegrid@origin@a,%
                {\pagegrid@firstcolor},%
                stroke,%
                line width=1pt,%
                line cap=round%
              ] (0mm,0mm) %
              -- (\pagegrid@arrowlength,\pagegrid@arrowlength) %
                 (\pagegrid@arrowlength,.5\pagegrid@arrowlength) %
              -- (\pagegrid@arrowlength,\pagegrid@arrowlength) %
              -- (.5\pagegrid@arrowlength,\pagegrid@arrowlength);%
            \fi
          \end{tikzpicture}%
        }%
      }%
    }%
    \ifnum\pagegrid@double=\tw@
      \pagegrid@ShipoutDouble\pagegrid@DoubleBox
    \fi
  \fi
}
%    \end{macrocode}
%
%    \begin{macrocode}
\pagegrid@AtEnd%
%</package>
%    \end{macrocode}
%
% \section{Test}
%
% \subsection{Catcode checks for loading}
%
%    \begin{macrocode}
%<*test1>
%    \end{macrocode}
%    \begin{macrocode}
\catcode`\{=1 %
\catcode`\}=2 %
\catcode`\#=6 %
\catcode`\@=11 %
\expandafter\ifx\csname count@\endcsname\relax
  \countdef\count@=255 %
\fi
\expandafter\ifx\csname @gobble\endcsname\relax
  \long\def\@gobble#1{}%
\fi
\expandafter\ifx\csname @firstofone\endcsname\relax
  \long\def\@firstofone#1{#1}%
\fi
\expandafter\ifx\csname loop\endcsname\relax
  \expandafter\@firstofone
\else
  \expandafter\@gobble
\fi
{%
  \def\loop#1\repeat{%
    \def\body{#1}%
    \iterate
  }%
  \def\iterate{%
    \body
      \let\next\iterate
    \else
      \let\next\relax
    \fi
    \next
  }%
  \let\repeat=\fi
}%
\def\RestoreCatcodes{}
\count@=0 %
\loop
  \edef\RestoreCatcodes{%
    \RestoreCatcodes
    \catcode\the\count@=\the\catcode\count@\relax
  }%
\ifnum\count@<255 %
  \advance\count@ 1 %
\repeat

\def\RangeCatcodeInvalid#1#2{%
  \count@=#1\relax
  \loop
    \catcode\count@=15 %
  \ifnum\count@<#2\relax
    \advance\count@ 1 %
  \repeat
}
\def\RangeCatcodeCheck#1#2#3{%
  \count@=#1\relax
  \loop
    \ifnum#3=\catcode\count@
    \else
      \errmessage{%
        Character \the\count@\space
        with wrong catcode \the\catcode\count@\space
        instead of \number#3%
      }%
    \fi
  \ifnum\count@<#2\relax
    \advance\count@ 1 %
  \repeat
}
\def\space{ }
\expandafter\ifx\csname LoadCommand\endcsname\relax
  \def\LoadCommand{\input pagegrid.sty\relax}%
\fi
\def\Test{%
  \RangeCatcodeInvalid{0}{47}%
  \RangeCatcodeInvalid{58}{64}%
  \RangeCatcodeInvalid{91}{96}%
  \RangeCatcodeInvalid{123}{255}%
  \catcode`\@=12 %
  \catcode`\\=0 %
  \catcode`\%=14 %
  \LoadCommand
  \RangeCatcodeCheck{0}{36}{15}%
  \RangeCatcodeCheck{37}{37}{14}%
  \RangeCatcodeCheck{38}{47}{15}%
  \RangeCatcodeCheck{48}{57}{12}%
  \RangeCatcodeCheck{58}{63}{15}%
  \RangeCatcodeCheck{64}{64}{12}%
  \RangeCatcodeCheck{65}{90}{11}%
  \RangeCatcodeCheck{91}{91}{15}%
  \RangeCatcodeCheck{92}{92}{0}%
  \RangeCatcodeCheck{93}{96}{15}%
  \RangeCatcodeCheck{97}{122}{11}%
  \RangeCatcodeCheck{123}{255}{15}%
  \RestoreCatcodes
}
\Test
\csname @@end\endcsname
\end
%    \end{macrocode}
%    \begin{macrocode}
%</test1>
%    \end{macrocode}
%
% \section{Installation}
%
% \subsection{Download}
%
% \paragraph{Package.} This package is available on
% CTAN\footnote{\CTANpkg{pagegrid}}:
% \begin{description}
% \item[\CTAN{macros/latex/contrib/oberdiek/pagegrid.dtx}] The source file.
% \item[\CTAN{macros/latex/contrib/oberdiek/pagegrid.pdf}] Documentation.
% \end{description}
%
%
% \paragraph{Bundle.} All the packages of the bundle `oberdiek'
% are also available in a TDS compliant ZIP archive. There
% the packages are already unpacked and the documentation files
% are generated. The files and directories obey the TDS standard.
% \begin{description}
% \item[\CTANinstall{install/macros/latex/contrib/oberdiek.tds.zip}]
% \end{description}
% \emph{TDS} refers to the standard ``A Directory Structure
% for \TeX\ Files'' (\CTAN{tds/tds.pdf}). Directories
% with \xfile{texmf} in their name are usually organized this way.
%
% \subsection{Bundle installation}
%
% \paragraph{Unpacking.} Unpack the \xfile{oberdiek.tds.zip} in the
% TDS tree (also known as \xfile{texmf} tree) of your choice.
% Example (linux):
% \begin{quote}
%   |unzip oberdiek.tds.zip -d ~/texmf|
% \end{quote}
%
% \paragraph{Script installation.}
% Check the directory \xfile{TDS:scripts/oberdiek/} for
% scripts that need further installation steps.
%
% \subsection{Package installation}
%
% \paragraph{Unpacking.} The \xfile{.dtx} file is a self-extracting
% \docstrip\ archive. The files are extracted by running the
% \xfile{.dtx} through \plainTeX:
% \begin{quote}
%   \verb|tex pagegrid.dtx|
% \end{quote}
%
% \paragraph{TDS.} Now the different files must be moved into
% the different directories in your installation TDS tree
% (also known as \xfile{texmf} tree):
% \begin{quote}
% \def\t{^^A
% \begin{tabular}{@{}>{\ttfamily}l@{ $\rightarrow$ }>{\ttfamily}l@{}}
%   pagegrid.sty & tex/latex/oberdiek/pagegrid.sty\\
%   pagegrid.pdf & doc/latex/oberdiek/pagegrid.pdf\\
%   test/pagegrid-test1.tex & doc/latex/oberdiek/test/pagegrid-test1.tex\\
%   pagegrid.dtx & source/latex/oberdiek/pagegrid.dtx\\
% \end{tabular}^^A
% }^^A
% \sbox0{\t}^^A
% \ifdim\wd0>\linewidth
%   \begingroup
%     \advance\linewidth by\leftmargin
%     \advance\linewidth by\rightmargin
%   \edef\x{\endgroup
%     \def\noexpand\lw{\the\linewidth}^^A
%   }\x
%   \def\lwbox{^^A
%     \leavevmode
%     \hbox to \linewidth{^^A
%       \kern-\leftmargin\relax
%       \hss
%       \usebox0
%       \hss
%       \kern-\rightmargin\relax
%     }^^A
%   }^^A
%   \ifdim\wd0>\lw
%     \sbox0{\small\t}^^A
%     \ifdim\wd0>\linewidth
%       \ifdim\wd0>\lw
%         \sbox0{\footnotesize\t}^^A
%         \ifdim\wd0>\linewidth
%           \ifdim\wd0>\lw
%             \sbox0{\scriptsize\t}^^A
%             \ifdim\wd0>\linewidth
%               \ifdim\wd0>\lw
%                 \sbox0{\tiny\t}^^A
%                 \ifdim\wd0>\linewidth
%                   \lwbox
%                 \else
%                   \usebox0
%                 \fi
%               \else
%                 \lwbox
%               \fi
%             \else
%               \usebox0
%             \fi
%           \else
%             \lwbox
%           \fi
%         \else
%           \usebox0
%         \fi
%       \else
%         \lwbox
%       \fi
%     \else
%       \usebox0
%     \fi
%   \else
%     \lwbox
%   \fi
% \else
%   \usebox0
% \fi
% \end{quote}
% If you have a \xfile{docstrip.cfg} that configures and enables \docstrip's
% TDS installing feature, then some files can already be in the right
% place, see the documentation of \docstrip.
%
% \subsection{Refresh file name databases}
%
% If your \TeX~distribution
% (\TeX\,Live, \mikTeX, \dots) relies on file name databases, you must refresh
% these. For example, \TeX\,Live\ users run \verb|texhash| or
% \verb|mktexlsr|.
%
% \subsection{Some details for the interested}
%
% \paragraph{Unpacking with \LaTeX.}
% The \xfile{.dtx} chooses its action depending on the format:
% \begin{description}
% \item[\plainTeX:] Run \docstrip\ and extract the files.
% \item[\LaTeX:] Generate the documentation.
% \end{description}
% If you insist on using \LaTeX\ for \docstrip\ (really,
% \docstrip\ does not need \LaTeX), then inform the autodetect routine
% about your intention:
% \begin{quote}
%   \verb|latex \let\install=y\input{pagegrid.dtx}|
% \end{quote}
% Do not forget to quote the argument according to the demands
% of your shell.
%
% \paragraph{Generating the documentation.}
% You can use both the \xfile{.dtx} or the \xfile{.drv} to generate
% the documentation. The process can be configured by the
% configuration file \xfile{ltxdoc.cfg}. For instance, put this
% line into this file, if you want to have A4 as paper format:
% \begin{quote}
%   \verb|\PassOptionsToClass{a4paper}{article}|
% \end{quote}
% An example follows how to generate the
% documentation with pdf\LaTeX:
% \begin{quote}
%\begin{verbatim}
%pdflatex pagegrid.dtx
%makeindex -s gind.ist pagegrid.idx
%pdflatex pagegrid.dtx
%makeindex -s gind.ist pagegrid.idx
%pdflatex pagegrid.dtx
%\end{verbatim}
% \end{quote}
%
% \section{Acknowledgement}
%
% \begin{description}
% \item[Klaus Braune:]
%  He provided the idea and the first \xpackage{tikz} code.
% \end{description}
%
% \begin{History}
%   \begin{Version}{2009/11/06 v1.0}
%   \item
%     The first version.
%   \end{Version}
%   \begin{Version}{2009/11/06 v1.1}
%   \item
%     Option \xoption{foreground} added.
%   \end{Version}
%   \begin{Version}{2009/12/02 v1.2}
%   \item
%     Color options, arrow options added.
%   \item
%     Names for origin options changed.
%   \end{Version}
%   \begin{Version}{2009/12/03 v1.3}
%   \item
%     Option \xoption{double} added.
%   \item
%     First CTAN release.
%   \end{Version}
%   \begin{Version}{2009/12/04 v1.4}
%   \item
%     Option \xoption{double}: Some unwanted side effects removed.
%   \end{Version}
%   \begin{Version}{2016/05/16 v1.5}
%   \item
%     Documentation updates.
%   \end{Version}
% \end{History}
%
% \PrintIndex
%
% \Finale
\endinput
|
% \end{quote}
% Do not forget to quote the argument according to the demands
% of your shell.
%
% \paragraph{Generating the documentation.}
% You can use both the \xfile{.dtx} or the \xfile{.drv} to generate
% the documentation. The process can be configured by the
% configuration file \xfile{ltxdoc.cfg}. For instance, put this
% line into this file, if you want to have A4 as paper format:
% \begin{quote}
%   \verb|\PassOptionsToClass{a4paper}{article}|
% \end{quote}
% An example follows how to generate the
% documentation with pdf\LaTeX:
% \begin{quote}
%\begin{verbatim}
%pdflatex pagegrid.dtx
%makeindex -s gind.ist pagegrid.idx
%pdflatex pagegrid.dtx
%makeindex -s gind.ist pagegrid.idx
%pdflatex pagegrid.dtx
%\end{verbatim}
% \end{quote}
%
% \section{Acknowledgement}
%
% \begin{description}
% \item[Klaus Braune:]
%  He provided the idea and the first \xpackage{tikz} code.
% \end{description}
%
% \begin{History}
%   \begin{Version}{2009/11/06 v1.0}
%   \item
%     The first version.
%   \end{Version}
%   \begin{Version}{2009/11/06 v1.1}
%   \item
%     Option \xoption{foreground} added.
%   \end{Version}
%   \begin{Version}{2009/12/02 v1.2}
%   \item
%     Color options, arrow options added.
%   \item
%     Names for origin options changed.
%   \end{Version}
%   \begin{Version}{2009/12/03 v1.3}
%   \item
%     Option \xoption{double} added.
%   \item
%     First CTAN release.
%   \end{Version}
%   \begin{Version}{2009/12/04 v1.4}
%   \item
%     Option \xoption{double}: Some unwanted side effects removed.
%   \end{Version}
%   \begin{Version}{2016/05/16 v1.5}
%   \item
%     Documentation updates.
%   \end{Version}
% \end{History}
%
% \PrintIndex
%
% \Finale
\endinput

%        (quote the arguments according to the demands of your shell)
%
% Documentation:
%    (a) If pagegrid.drv is present:
%           latex pagegrid.drv
%    (b) Without pagegrid.drv:
%           latex pagegrid.dtx; ...
%    The class ltxdoc loads the configuration file ltxdoc.cfg
%    if available. Here you can specify further options, e.g.
%    use A4 as paper format:
%       \PassOptionsToClass{a4paper}{article}
%
%    Programm calls to get the documentation (example):
%       pdflatex pagegrid.dtx
%       makeindex -s gind.ist pagegrid.idx
%       pdflatex pagegrid.dtx
%       makeindex -s gind.ist pagegrid.idx
%       pdflatex pagegrid.dtx
%
% Installation:
%    TDS:tex/latex/oberdiek/pagegrid.sty
%    TDS:doc/latex/oberdiek/pagegrid.pdf
%    TDS:doc/latex/oberdiek/test/pagegrid-test1.tex
%    TDS:source/latex/oberdiek/pagegrid.dtx
%
%<*ignore>
\begingroup
  \catcode123=1 %
  \catcode125=2 %
  \def\x{LaTeX2e}%
\expandafter\endgroup
\ifcase 0\ifx\install y1\fi\expandafter
         \ifx\csname processbatchFile\endcsname\relax\else1\fi
         \ifx\fmtname\x\else 1\fi\relax
\else\csname fi\endcsname
%</ignore>
%<*install>
\input docstrip.tex
\Msg{************************************************************************}
\Msg{* Installation}
\Msg{* Package: pagegrid 2016/05/16 v1.5 Print page grid in background (HO)}
\Msg{************************************************************************}

\keepsilent
\askforoverwritefalse

\let\MetaPrefix\relax
\preamble

This is a generated file.

Project: pagegrid
Version: 2016/05/16 v1.5

Copyright (C) 2009 by
   Heiko Oberdiek <heiko.oberdiek at googlemail.com>

This work may be distributed and/or modified under the
conditions of the LaTeX Project Public License, either
version 1.3c of this license or (at your option) any later
version. This version of this license is in
   https://www.latex-project.org/lppl/lppl-1-3c.txt
and the latest version of this license is in
   https://www.latex-project.org/lppl.txt
and version 1.3 or later is part of all distributions of
LaTeX version 2005/12/01 or later.

This work has the LPPL maintenance status "maintained".

The Current Maintainers of this work are
Heiko Oberdiek and the Oberdiek Package Support Group
https://github.com/ho-tex/oberdiek/issues


This work consists of the main source file pagegrid.dtx
and the derived files
   pagegrid.sty, pagegrid.pdf, pagegrid.ins, pagegrid.drv,
   pagegrid-test1.tex.

\endpreamble
\let\MetaPrefix\DoubleperCent

\generate{%
  \file{pagegrid.ins}{\from{pagegrid.dtx}{install}}%
  \file{pagegrid.drv}{\from{pagegrid.dtx}{driver}}%
  \usedir{tex/latex/oberdiek}%
  \file{pagegrid.sty}{\from{pagegrid.dtx}{package}}%
%  \usedir{doc/latex/oberdiek/test}%
%  \file{pagegrid-test1.tex}{\from{pagegrid.dtx}{test1}}%
  \nopreamble
  \nopostamble
%  \usedir{source/latex/oberdiek/catalogue}%
%  \file{pagegrid.xml}{\from{pagegrid.dtx}{catalogue}}%
}

\catcode32=13\relax% active space
\let =\space%
\Msg{************************************************************************}
\Msg{*}
\Msg{* To finish the installation you have to move the following}
\Msg{* file into a directory searched by TeX:}
\Msg{*}
\Msg{*     pagegrid.sty}
\Msg{*}
\Msg{* To produce the documentation run the file `pagegrid.drv'}
\Msg{* through LaTeX.}
\Msg{*}
\Msg{* Happy TeXing!}
\Msg{*}
\Msg{************************************************************************}

\endbatchfile
%</install>
%<*ignore>
\fi
%</ignore>
%<*driver>
\NeedsTeXFormat{LaTeX2e}
\ProvidesFile{pagegrid.drv}%
  [2016/05/16 v1.5 Print page grid in background (HO)]%
\documentclass{ltxdoc}
\usepackage{holtxdoc}[2011/11/22]
\begin{document}
  \DocInput{pagegrid.dtx}%
\end{document}
%</driver>
% \fi
%
%
% \CharacterTable
%  {Upper-case    \A\B\C\D\E\F\G\H\I\J\K\L\M\N\O\P\Q\R\S\T\U\V\W\X\Y\Z
%   Lower-case    \a\b\c\d\e\f\g\h\i\j\k\l\m\n\o\p\q\r\s\t\u\v\w\x\y\z
%   Digits        \0\1\2\3\4\5\6\7\8\9
%   Exclamation   \!     Double quote  \"     Hash (number) \#
%   Dollar        \$     Percent       \%     Ampersand     \&
%   Acute accent  \'     Left paren    \(     Right paren   \)
%   Asterisk      \*     Plus          \+     Comma         \,
%   Minus         \-     Point         \.     Solidus       \/
%   Colon         \:     Semicolon     \;     Less than     \<
%   Equals        \=     Greater than  \>     Question mark \?
%   Commercial at \@     Left bracket  \[     Backslash     \\
%   Right bracket \]     Circumflex    \^     Underscore    \_
%   Grave accent  \`     Left brace    \{     Vertical bar  \|
%   Right brace   \}     Tilde         \~}
%
% \GetFileInfo{pagegrid.drv}
%
% \title{The \xpackage{pagegrid} package}
% \date{2016/05/16 v1.5}
% \author{Heiko Oberdiek\thanks
% {Please report any issues at \url{https://github.com/ho-tex/oberdiek/issues}}}
%
% \maketitle
%
% \begin{abstract}
% The \LaTeX\ package prints a page grid in the background.
% \end{abstract}
%
% \tableofcontents
%
% \section{Documentation}
%
% The package puts a grid on the paper. It was written for
% developers of a class or package
% who have to put elements on definite locations on a page
% (e.g. letter class). The grid allows a faster optical check,
% whether the positions are correct. If the previewer already
% offers features for measuring, the package might be obsolete.
% Otherwise it saves the developer from printing the page and
% measuring by hand.
%
% \subsection{Options}
%
% Options are evaluated in the following order:
% \begin{enumerate}
% \item
%  Configuration file \xfile{pagegrid.cfg} using \cs{pagegridsetup}
%  if the file exists.
%  \item
%  Package options given for \cs{usepackage}.
%  \item
%  Later calls of \cs{pagegridsetup}.
% \end{enumerate}
% \begin{declcs}{pagegridsetup}\M{option list}
% \end{declcs}
% The options are key value options. Boolean options are enabled by
% default (without value) or by using the explicit value \texttt{true}.
% Value \texttt{false} disable the option.
%
% \subsubsection{Options \xoption{enable}, \xoption{disable}}
%
% \begin{description}
% \item[\xoption{enable}:] This boolean option controls whether the page grid
%   is drawn. As default the page grid drawing is activated.
% \item[\xoption{disable}:] It is the opposite
%   of option \xoption{enable}. It was added for convenience and
%   allows the abbreviation \texttt{disable} for \texttt{enable=false}.
% \end{description}
%
% \subsubsection{Grid origins}
%
% The package supports up to two grids on a page allowing
% measurement from opposite directions. As default two grids are drawn,
% the first from bottom left to top right. The origin of the second
% grid is at the opposite top right corner.
% The origins are controlled by the following options.
% The number of grids (one or two) depend on the number of these options
% in one call of \cs{pagegridsetup}.
% The following frame shows a paper and in its corners are the
% corresponding options. At the left and right side alias names
% are given for the options inside the paper.
% \begin{quote}
% \begin{tabular}{@{}r|@{\,}l@{\qquad}r@{\,}|l@{}}
% \cline{2-3}
% \xoption{left-top}, \xoption{lt}, \xoption{top-left}
% & \vphantom{\"U}\xoption{tl} & \xoption{tr}
% & \xoption{top-right}, \xoption{rt}, \xoption{right-top}\\
% &&&\\
% \xoption{left-bottom}, \xoption{lb}, \xoption{bottom-left}
% & \xoption{bl} & \xoption{br}
% & \xoption{bottom-right}, \xoption{rb}, \xoption{right-bottom}\\
% \cline{2-3}
% \end{tabular}
% \end{quote}
% Examples:
% \begin{quote}
% |\pagegridsetup{bl,tr}|
% \end{quote}
% This is the default setting with two grids as described previously.
% The following setups one grid only. Its origin is the upper left
% corner:
% \begin{quote}
% |\pagegridsetup{top-left}|
% \end{quote}
%
% \subsubsection{Grid unit}
%
% \begin{description}
% \item[\xoption{step}] This option takes a length and
% setups the unit for the grid. The page width and page height
% should be multiples of this unit.
% Currently the default is \texttt{1mm}. But this might change
% later by a heuristic based on the paper size.
% \end{description}
%
% \subsubsection{Color options}
%
% The basic grid lines are drawn as ultra thin help lines and is only
% drawn for the first grid.
% Each tenth and fiftyth line of the basic net is drawn thicker in a special
% color for the two grids.
% \begin{description}
% \item[\xoption{firstcolor}:] Color for the thicker lines and the arrows
% of the first grid. Default value is \texttt{red}.
% \item[\xoption{secondcolor}:] Color for the thicker lines and the arrows
% of the second grid. Default value is \texttt{blue}.
% \end{description}
% Use a color specification that package \xpackage{tikz} understands.
% (The grid is drawn with \xpackage{pgf}/\xpackage{tikz}.)
%
% \subsubsection{Arrow options}
%
% Arrows are put at the origin at the grid to show the grid start
% and the direction of the grid.
% \begin{description}
% \item[\xoption{arrows}:] This boolean option turns the arrows on or off.
% As default arrows are enabled.
% \item[\xoption{arrowlength}:] The length given as value is the
% length of the edge of a square at the origin within the
% arrow is put as diagonal. Default is 10 times the grid unit (10\,mm).
% The real arrow length is this length multiplied by $\sqrt2$.
% \end{description}
%
% \subsubsection{Miscellaneous options}
%
% \begin{description}
% \item[\xoption{double}:] The output page is doubled, one without page
% grid and the other with page grid. Possible values are shown in the
% following table:
% \begin{quote}
% \begin{tabular}{ll}
% Option & Meaning\\
% \hline
% |false| & Turns option off.\\
% |first| & Grid page comes first.\\
% |last| & Grid page comes after the page without grid.\\
% |true| & Same as |last|.\\
% \meta{no value} & Same as |true|.\\
% \end{tabular}
% \end{quote}
% \textbf{Note:}
% The double output of the page has side effects.
% All whatits are executed twice, for example: file writing
% and anchor setting. Some unwanted actions are catched such
% as multiple \cs{label} definitions, duplicate entries in
% the table of contents. For bookmarks, use package \xpackage{bookmarks}.
% \item[\xoption{foreground}:] Boolean option, default is \texttt{false}.
% Sometimes there might be elements on the page (e.g. large images)
% that hide the grid. Then option \xoption{foreground} puts the grids
% over the current output page.
% \end{description}
%
% \StopEventually{
% }
%
% \section{Implementation}
%    \begin{macrocode}
%<*package>
%    \end{macrocode}
%    Reload check, especially if the package is not used with \LaTeX.
%    \begin{macrocode}
\begingroup\catcode61\catcode48\catcode32=10\relax%
  \catcode13=5 % ^^M
  \endlinechar=13 %
  \catcode35=6 % #
  \catcode39=12 % '
  \catcode44=12 % ,
  \catcode45=12 % -
  \catcode46=12 % .
  \catcode58=12 % :
  \catcode64=11 % @
  \catcode123=1 % {
  \catcode125=2 % }
  \expandafter\let\expandafter\x\csname ver@pagegrid.sty\endcsname
  \ifx\x\relax % plain-TeX, first loading
  \else
    \def\empty{}%
    \ifx\x\empty % LaTeX, first loading,
      % variable is initialized, but \ProvidesPackage not yet seen
    \else
      \expandafter\ifx\csname PackageInfo\endcsname\relax
        \def\x#1#2{%
          \immediate\write-1{Package #1 Info: #2.}%
        }%
      \else
        \def\x#1#2{\PackageInfo{#1}{#2, stopped}}%
      \fi
      \x{pagegrid}{The package is already loaded}%
      \aftergroup\endinput
    \fi
  \fi
\endgroup%
%    \end{macrocode}
%    Package identification:
%    \begin{macrocode}
\begingroup\catcode61\catcode48\catcode32=10\relax%
  \catcode13=5 % ^^M
  \endlinechar=13 %
  \catcode35=6 % #
  \catcode39=12 % '
  \catcode40=12 % (
  \catcode41=12 % )
  \catcode44=12 % ,
  \catcode45=12 % -
  \catcode46=12 % .
  \catcode47=12 % /
  \catcode58=12 % :
  \catcode64=11 % @
  \catcode91=12 % [
  \catcode93=12 % ]
  \catcode123=1 % {
  \catcode125=2 % }
  \expandafter\ifx\csname ProvidesPackage\endcsname\relax
    \def\x#1#2#3[#4]{\endgroup
      \immediate\write-1{Package: #3 #4}%
      \xdef#1{#4}%
    }%
  \else
    \def\x#1#2[#3]{\endgroup
      #2[{#3}]%
      \ifx#1\@undefined
        \xdef#1{#3}%
      \fi
      \ifx#1\relax
        \xdef#1{#3}%
      \fi
    }%
  \fi
\expandafter\x\csname ver@pagegrid.sty\endcsname
\ProvidesPackage{pagegrid}%
  [2016/05/16 v1.5 Print page grid in background (HO)]%
%    \end{macrocode}
%
%    \begin{macrocode}
\begingroup\catcode61\catcode48\catcode32=10\relax%
  \catcode13=5 % ^^M
  \endlinechar=13 %
  \catcode123=1 % {
  \catcode125=2 % }
  \catcode64=11 % @
  \def\x{\endgroup
    \expandafter\edef\csname pagegrid@AtEnd\endcsname{%
      \endlinechar=\the\endlinechar\relax
      \catcode13=\the\catcode13\relax
      \catcode32=\the\catcode32\relax
      \catcode35=\the\catcode35\relax
      \catcode61=\the\catcode61\relax
      \catcode64=\the\catcode64\relax
      \catcode123=\the\catcode123\relax
      \catcode125=\the\catcode125\relax
    }%
  }%
\x\catcode61\catcode48\catcode32=10\relax%
\catcode13=5 % ^^M
\endlinechar=13 %
\catcode35=6 % #
\catcode64=11 % @
\catcode123=1 % {
\catcode125=2 % }
\def\TMP@EnsureCode#1#2{%
  \edef\pagegrid@AtEnd{%
    \pagegrid@AtEnd
    \catcode#1=\the\catcode#1\relax
  }%
  \catcode#1=#2\relax
}
\TMP@EnsureCode{9}{10}% (tab)
\TMP@EnsureCode{10}{12}% ^^J
\TMP@EnsureCode{33}{12}% !
\TMP@EnsureCode{34}{12}% "
\TMP@EnsureCode{36}{3}% $
\TMP@EnsureCode{38}{4}% &
\TMP@EnsureCode{39}{12}% '
\TMP@EnsureCode{40}{12}% (
\TMP@EnsureCode{41}{12}% )
\TMP@EnsureCode{42}{12}% *
\TMP@EnsureCode{43}{12}% +
\TMP@EnsureCode{44}{12}% ,
\TMP@EnsureCode{45}{12}% -
\TMP@EnsureCode{46}{12}% .
\TMP@EnsureCode{47}{12}% /
\TMP@EnsureCode{58}{12}% :
\TMP@EnsureCode{59}{12}% ;
\TMP@EnsureCode{60}{12}% <
\TMP@EnsureCode{62}{12}% >
\TMP@EnsureCode{63}{12}% ?
\TMP@EnsureCode{91}{12}% [
\TMP@EnsureCode{93}{12}% ]
\TMP@EnsureCode{94}{7}% ^ (superscript)
\TMP@EnsureCode{95}{8}% _ (subscript)
\TMP@EnsureCode{96}{12}% `
\TMP@EnsureCode{124}{12}% |
\edef\pagegrid@AtEnd{\pagegrid@AtEnd\noexpand\endinput}
%    \end{macrocode}
%
%    \begin{macrocode}
\RequirePackage{tikz}
\RequirePackage{atbegshi}[2009/12/02]
\RequirePackage{kvoptions}[2009/07/17]
%    \end{macrocode}
%    \begin{macrocode}
\begingroup\expandafter\expandafter\expandafter\endgroup
\expandafter\ifx\csname stockwidth\endcsname\relax
  \def\pagegrid@width{\paperwidth}%
  \def\pagegrid@height{\paperheight}%
\else
  \def\pagegrid@width{\stockwidth}%
  \def\pagegrid@height{\stockheight}%
\fi
%    \end{macrocode}
%
%    \begin{macrocode}
\SetupKeyvalOptions{%
  family=pagegrid,%
  prefix=pagegrid@,%
}
\def\pagegrid@init{%
  \let\pagegrid@origin@a\@empty
  \let\pagegrid@origin@b\@empty
  \let\pagegrid@init\relax
}
\let\pagegrid@@init\pagegrid@init
\def\pagegrid@origin@a{bl}
\def\pagegrid@origin@b{tr}
\def\pagegrid@SetOrigin#1{%
  \pagegrid@init
  \ifx\pagegrid@origin@a\@empty
    \def\pagegrid@origin@a{#1}%
  \else
    \ifx\pagegrid@origin@b\@empty
    \else
      \let\pagegrid@origin@a\pagegrid@origin@b
    \fi
    \def\pagegrid@origin@b{#1}%
  \fi
}
\def\pagegrid@temp#1{%
  \DeclareVoidOption{#1}{\pagegrid@SetOrigin{#1}}%
  \@namedef{pagegrid@N@#1}{#1}%
}
\pagegrid@temp{bl}
\pagegrid@temp{br}
\pagegrid@temp{tl}
\pagegrid@temp{tr}
\def\pagegrid@temp#1#2{%
  \DeclareVoidOption{#2}{\pagegrid@SetOrigin{#1}}%
}%
\pagegrid@temp{bl}{lb}
\pagegrid@temp{br}{rb}
\pagegrid@temp{tl}{lt}
\pagegrid@temp{tr}{rt}
\pagegrid@temp{bl}{bottom-left}
\pagegrid@temp{br}{bottom-right}
\pagegrid@temp{tl}{top-left}
\pagegrid@temp{tr}{top-right}
\pagegrid@temp{bl}{left-bottom}
\pagegrid@temp{br}{right-bottom}
\pagegrid@temp{tl}{left-top}
\pagegrid@temp{tr}{right-top}
%    \end{macrocode}
%    \begin{macrocode}
\DeclareBoolOption[true]{enable}
\DeclareComplementaryOption{disable}{enable}
%    \end{macrocode}
%    \begin{macrocode}
\DeclareBoolOption{foreground}
%    \end{macrocode}
%    \begin{macrocode}
\newlength{\pagegrid@step}
\define@key{pagegrid}{step}{%
  \setlength{\pagegrid@step}{#1}%
}
%    \end{macrocode}
%    \begin{macrocode}
\DeclareStringOption[red]{firstcolor}
\DeclareStringOption[blue]{secondcolor}
%    \end{macrocode}
%    \begin{macrocode}
\DeclareBoolOption[true]{arrows}
\newlength\pagegrid@arrowlength
\pagegrid@arrowlength=\z@
\define@key{pagegrid}{arrowlength}{%
  \setlength{\pagegrid@arrowlength}{#1}%
}
%    \end{macrocode}
%    \begin{macrocode}
\define@key{pagegrid}{double}[true]{%
  \@ifundefined{pagegrid@double@#1}{%
    \PackageWarning{pagegrid}{%
      Unsupported value `#1' for option `double'.\MessageBreak
      Known values are:\MessageBreak
      `false', `first', `last', `true'.\MessageBreak
      Now `false' is used%
    }%
    \chardef\pagegrid@double\z@
  }{%
    \chardef\pagegrid@double\csname pagegrid@double@#1\endcsname\relax
  }%
}
\@namedef{pagegrid@double@false}{0}
\@namedef{pagegrid@double@first}{1}
\@namedef{pagegrid@double@last}{2}
\@namedef{pagegrid@double@true}{2}
\chardef\pagegrid@double\z@
%    \end{macrocode}
%    \begin{macrocode}
\newcommand*{\pagegridsetup}{%
  \let\pagegrid@init\pagegrid@@init
  \setkeys{pagegrid}%
}
%    \end{macrocode}
%    \begin{macrocode}
\pagegridsetup{%
  step=1mm%
}
\InputIfFileExists{pagegrid.cfg}{}%
\ProcessKeyvalOptions*\relax
\AtBeginDocument{%
  \ifdim\pagegrid@arrowlength>\z@
  \else
    \pagegrid@arrowlength=10\pagegrid@step
  \fi
}
%    \end{macrocode}
%
%    \begin{macrocode}
\def\pagegridShipoutDoubleBegin{%
  \begingroup
  \let\newlabel\@gobbletwo
  \let\zref@newlabel\@gobbletwo
  \let\@writefile\@gobbletwo
  \let\select@language\@gobble
}
\def\pagegridShipoutDoubleEnd{%
  \endgroup
}
\def\pagegrid@WriteDouble#1#2{%
  \immediate\write#1{%
    \@backslashchar csname %
    pagegridShipoutDouble#2%
    \@backslashchar endcsname%
  }%
}
\def\pagegrid@ShipoutDouble#1{%
  \begingroup
    \if@filesw
      \pagegrid@WriteDouble\@mainaux{Begin}%
      \ifx\@auxout\@partaux
        \pagegrid@WriteDouble\@partaux{Begin}%
        \def\pagegrid@temp{%
          \pagegrid@WriteDouble\@mainaux{End}%
          \pagegrid@WriteDouble\@partaux{End}%
        }%
      \else
        \def\pagegrid@temp{%
          \pagegrid@WriteDouble\@mainaux{End}%
        }%
      \fi
    \else
      \def\pagegrid@temp{}%
    \fi
    \let\protect\noexpand
    \AtBeginShipoutOriginalShipout\copy#1\relax
    \pagegrid@temp
  \endgroup
}
%    \end{macrocode}
%
%    \begin{macrocode}
\AtBeginShipout{%
  \ifdim\pagegrid@step>\z@
  \else
    \pagegrid@enablefalse
  \fi
  \ifpagegrid@enable
    \ifnum\pagegrid@double=\@ne
      \pagegrid@ShipoutDouble\AtBeginShipoutBox
    \else
      \ifnum\pagegrid@double=\tw@
        \@ifundefined{pagegrid@DoubleBox}{%
          \newbox\pagegrid@DoubleBox
        }{}%
        \setbox\pagegrid@DoubleBox=\copy\AtBeginShipoutBox
      \fi
    \fi
    \ifpagegrid@foreground
      \expandafter\AtBeginShipoutUpperLeftForeground
    \else
      \expandafter\AtBeginShipoutUpperLeft
    \fi
    {%
      \put(0,0){%
        \makebox(0,0)[lt]{%
          \begin{tikzpicture}[%
            bl/.style={},%
            br/.style={xshift=\pagegrid@width,xscale=-1},%
            tl/.style={yshift=\pagegrid@height,yscale=-1},%
            tr/.style={xshift=\pagegrid@width,%
                       yshift=\pagegrid@height,scale=-1}%
          ]%
            \useasboundingbox
              (0mm,\pagegrid@height) rectangle (0mm,\pagegrid@height);%
            \draw[%
              \pagegrid@origin@a,%
              step=\pagegrid@step,%
              style=help lines,%
              ultra thin%
            ] (0mm,0mm) grid (\pagegrid@width,\pagegrid@height);%
            \ifx\pagegrid@origin@b\@empty
            \else
              \draw[%
                \pagegrid@origin@b,%
                step=10\pagegrid@step,%
                {\pagegrid@secondcolor},%
                very thin%
              ] (0mm,0mm) grid (\pagegrid@width,\pagegrid@height);%
            \fi
            \draw[%
               \pagegrid@origin@a,%
               step=10\pagegrid@step,%
               {\pagegrid@firstcolor},%
               very thin%
            ] (0mm,0mm) grid (\pagegrid@width,\pagegrid@height);%
            \ifx\pagegrid@origin@b\@empty
            \else
              \draw[%
                \pagegrid@origin@b,%
                step=50\pagegrid@step,%
                {\pagegrid@secondcolor},%
                thick%
              ] (0mm,0mm) grid (\pagegrid@width,\pagegrid@height);%
            \fi
            \draw[%
              \pagegrid@origin@a,%
              step=50\pagegrid@step,%
              {\pagegrid@firstcolor},%
              thick%
            ] (0mm,0mm) grid (\pagegrid@width,\pagegrid@height);%
            \ifpagegrid@arrows
              \ifx\pagegrid@origin@b\@empty
              \else
                \draw[%
                  \pagegrid@origin@b,%
                  {\pagegrid@secondcolor},%
                  stroke,%
                  line width=1pt,%
                  line cap=round%
                ] (0mm,0mm) %
                -- (\pagegrid@arrowlength,\pagegrid@arrowlength) %
                   (\pagegrid@arrowlength,.5\pagegrid@arrowlength) %
                -- (\pagegrid@arrowlength,\pagegrid@arrowlength) %
                -- (.5\pagegrid@arrowlength,\pagegrid@arrowlength);%
              \fi
              \draw[%
                \pagegrid@origin@a,%
                {\pagegrid@firstcolor},%
                stroke,%
                line width=1pt,%
                line cap=round%
              ] (0mm,0mm) %
              -- (\pagegrid@arrowlength,\pagegrid@arrowlength) %
                 (\pagegrid@arrowlength,.5\pagegrid@arrowlength) %
              -- (\pagegrid@arrowlength,\pagegrid@arrowlength) %
              -- (.5\pagegrid@arrowlength,\pagegrid@arrowlength);%
            \fi
          \end{tikzpicture}%
        }%
      }%
    }%
    \ifnum\pagegrid@double=\tw@
      \pagegrid@ShipoutDouble\pagegrid@DoubleBox
    \fi
  \fi
}
%    \end{macrocode}
%
%    \begin{macrocode}
\pagegrid@AtEnd%
%</package>
%    \end{macrocode}
%
% \section{Test}
%
% \subsection{Catcode checks for loading}
%
%    \begin{macrocode}
%<*test1>
%    \end{macrocode}
%    \begin{macrocode}
\catcode`\{=1 %
\catcode`\}=2 %
\catcode`\#=6 %
\catcode`\@=11 %
\expandafter\ifx\csname count@\endcsname\relax
  \countdef\count@=255 %
\fi
\expandafter\ifx\csname @gobble\endcsname\relax
  \long\def\@gobble#1{}%
\fi
\expandafter\ifx\csname @firstofone\endcsname\relax
  \long\def\@firstofone#1{#1}%
\fi
\expandafter\ifx\csname loop\endcsname\relax
  \expandafter\@firstofone
\else
  \expandafter\@gobble
\fi
{%
  \def\loop#1\repeat{%
    \def\body{#1}%
    \iterate
  }%
  \def\iterate{%
    \body
      \let\next\iterate
    \else
      \let\next\relax
    \fi
    \next
  }%
  \let\repeat=\fi
}%
\def\RestoreCatcodes{}
\count@=0 %
\loop
  \edef\RestoreCatcodes{%
    \RestoreCatcodes
    \catcode\the\count@=\the\catcode\count@\relax
  }%
\ifnum\count@<255 %
  \advance\count@ 1 %
\repeat

\def\RangeCatcodeInvalid#1#2{%
  \count@=#1\relax
  \loop
    \catcode\count@=15 %
  \ifnum\count@<#2\relax
    \advance\count@ 1 %
  \repeat
}
\def\RangeCatcodeCheck#1#2#3{%
  \count@=#1\relax
  \loop
    \ifnum#3=\catcode\count@
    \else
      \errmessage{%
        Character \the\count@\space
        with wrong catcode \the\catcode\count@\space
        instead of \number#3%
      }%
    \fi
  \ifnum\count@<#2\relax
    \advance\count@ 1 %
  \repeat
}
\def\space{ }
\expandafter\ifx\csname LoadCommand\endcsname\relax
  \def\LoadCommand{\input pagegrid.sty\relax}%
\fi
\def\Test{%
  \RangeCatcodeInvalid{0}{47}%
  \RangeCatcodeInvalid{58}{64}%
  \RangeCatcodeInvalid{91}{96}%
  \RangeCatcodeInvalid{123}{255}%
  \catcode`\@=12 %
  \catcode`\\=0 %
  \catcode`\%=14 %
  \LoadCommand
  \RangeCatcodeCheck{0}{36}{15}%
  \RangeCatcodeCheck{37}{37}{14}%
  \RangeCatcodeCheck{38}{47}{15}%
  \RangeCatcodeCheck{48}{57}{12}%
  \RangeCatcodeCheck{58}{63}{15}%
  \RangeCatcodeCheck{64}{64}{12}%
  \RangeCatcodeCheck{65}{90}{11}%
  \RangeCatcodeCheck{91}{91}{15}%
  \RangeCatcodeCheck{92}{92}{0}%
  \RangeCatcodeCheck{93}{96}{15}%
  \RangeCatcodeCheck{97}{122}{11}%
  \RangeCatcodeCheck{123}{255}{15}%
  \RestoreCatcodes
}
\Test
\csname @@end\endcsname
\end
%    \end{macrocode}
%    \begin{macrocode}
%</test1>
%    \end{macrocode}
%
% \section{Installation}
%
% \subsection{Download}
%
% \paragraph{Package.} This package is available on
% CTAN\footnote{\CTANpkg{pagegrid}}:
% \begin{description}
% \item[\CTAN{macros/latex/contrib/oberdiek/pagegrid.dtx}] The source file.
% \item[\CTAN{macros/latex/contrib/oberdiek/pagegrid.pdf}] Documentation.
% \end{description}
%
%
% \paragraph{Bundle.} All the packages of the bundle `oberdiek'
% are also available in a TDS compliant ZIP archive. There
% the packages are already unpacked and the documentation files
% are generated. The files and directories obey the TDS standard.
% \begin{description}
% \item[\CTANinstall{install/macros/latex/contrib/oberdiek.tds.zip}]
% \end{description}
% \emph{TDS} refers to the standard ``A Directory Structure
% for \TeX\ Files'' (\CTAN{tds/tds.pdf}). Directories
% with \xfile{texmf} in their name are usually organized this way.
%
% \subsection{Bundle installation}
%
% \paragraph{Unpacking.} Unpack the \xfile{oberdiek.tds.zip} in the
% TDS tree (also known as \xfile{texmf} tree) of your choice.
% Example (linux):
% \begin{quote}
%   |unzip oberdiek.tds.zip -d ~/texmf|
% \end{quote}
%
% \paragraph{Script installation.}
% Check the directory \xfile{TDS:scripts/oberdiek/} for
% scripts that need further installation steps.
%
% \subsection{Package installation}
%
% \paragraph{Unpacking.} The \xfile{.dtx} file is a self-extracting
% \docstrip\ archive. The files are extracted by running the
% \xfile{.dtx} through \plainTeX:
% \begin{quote}
%   \verb|tex pagegrid.dtx|
% \end{quote}
%
% \paragraph{TDS.} Now the different files must be moved into
% the different directories in your installation TDS tree
% (also known as \xfile{texmf} tree):
% \begin{quote}
% \def\t{^^A
% \begin{tabular}{@{}>{\ttfamily}l@{ $\rightarrow$ }>{\ttfamily}l@{}}
%   pagegrid.sty & tex/latex/oberdiek/pagegrid.sty\\
%   pagegrid.pdf & doc/latex/oberdiek/pagegrid.pdf\\
%   test/pagegrid-test1.tex & doc/latex/oberdiek/test/pagegrid-test1.tex\\
%   pagegrid.dtx & source/latex/oberdiek/pagegrid.dtx\\
% \end{tabular}^^A
% }^^A
% \sbox0{\t}^^A
% \ifdim\wd0>\linewidth
%   \begingroup
%     \advance\linewidth by\leftmargin
%     \advance\linewidth by\rightmargin
%   \edef\x{\endgroup
%     \def\noexpand\lw{\the\linewidth}^^A
%   }\x
%   \def\lwbox{^^A
%     \leavevmode
%     \hbox to \linewidth{^^A
%       \kern-\leftmargin\relax
%       \hss
%       \usebox0
%       \hss
%       \kern-\rightmargin\relax
%     }^^A
%   }^^A
%   \ifdim\wd0>\lw
%     \sbox0{\small\t}^^A
%     \ifdim\wd0>\linewidth
%       \ifdim\wd0>\lw
%         \sbox0{\footnotesize\t}^^A
%         \ifdim\wd0>\linewidth
%           \ifdim\wd0>\lw
%             \sbox0{\scriptsize\t}^^A
%             \ifdim\wd0>\linewidth
%               \ifdim\wd0>\lw
%                 \sbox0{\tiny\t}^^A
%                 \ifdim\wd0>\linewidth
%                   \lwbox
%                 \else
%                   \usebox0
%                 \fi
%               \else
%                 \lwbox
%               \fi
%             \else
%               \usebox0
%             \fi
%           \else
%             \lwbox
%           \fi
%         \else
%           \usebox0
%         \fi
%       \else
%         \lwbox
%       \fi
%     \else
%       \usebox0
%     \fi
%   \else
%     \lwbox
%   \fi
% \else
%   \usebox0
% \fi
% \end{quote}
% If you have a \xfile{docstrip.cfg} that configures and enables \docstrip's
% TDS installing feature, then some files can already be in the right
% place, see the documentation of \docstrip.
%
% \subsection{Refresh file name databases}
%
% If your \TeX~distribution
% (\TeX\,Live, \mikTeX, \dots) relies on file name databases, you must refresh
% these. For example, \TeX\,Live\ users run \verb|texhash| or
% \verb|mktexlsr|.
%
% \subsection{Some details for the interested}
%
% \paragraph{Unpacking with \LaTeX.}
% The \xfile{.dtx} chooses its action depending on the format:
% \begin{description}
% \item[\plainTeX:] Run \docstrip\ and extract the files.
% \item[\LaTeX:] Generate the documentation.
% \end{description}
% If you insist on using \LaTeX\ for \docstrip\ (really,
% \docstrip\ does not need \LaTeX), then inform the autodetect routine
% about your intention:
% \begin{quote}
%   \verb|latex \let\install=y% \iffalse meta-comment
%
% File: pagegrid.dtx
% Version: 2016/05/16 v1.5
% Info: Print page grid in background
%
% Copyright (C) 2009 by
%    Heiko Oberdiek <heiko.oberdiek at googlemail.com>
%    2016
%    https://github.com/ho-tex/oberdiek/issues
%
% This work may be distributed and/or modified under the
% conditions of the LaTeX Project Public License, either
% version 1.3c of this license or (at your option) any later
% version. This version of this license is in
%    https://www.latex-project.org/lppl/lppl-1-3c.txt
% and the latest version of this license is in
%    https://www.latex-project.org/lppl.txt
% and version 1.3 or later is part of all distributions of
% LaTeX version 2005/12/01 or later.
%
% This work has the LPPL maintenance status "maintained".
%
% The Current Maintainers of this work are
% Heiko Oberdiek and the Oberdiek Package Support Group
% https://github.com/ho-tex/oberdiek/issues
%
% This work consists of the main source file pagegrid.dtx
% and the derived files
%    pagegrid.sty, pagegrid.pdf, pagegrid.ins, pagegrid.drv,
%    pagegrid-test1.tex.
%
% Distribution:
%    CTAN:macros/latex/contrib/oberdiek/pagegrid.dtx
%    CTAN:macros/latex/contrib/oberdiek/pagegrid.pdf
%
% Unpacking:
%    (a) If pagegrid.ins is present:
%           tex pagegrid.ins
%    (b) Without pagegrid.ins:
%           tex pagegrid.dtx
%    (c) If you insist on using LaTeX
%           latex \let\install=y% \iffalse meta-comment
%
% File: pagegrid.dtx
% Version: 2016/05/16 v1.5
% Info: Print page grid in background
%
% Copyright (C) 2009 by
%    Heiko Oberdiek <heiko.oberdiek at googlemail.com>
%    2016
%    https://github.com/ho-tex/oberdiek/issues
%
% This work may be distributed and/or modified under the
% conditions of the LaTeX Project Public License, either
% version 1.3c of this license or (at your option) any later
% version. This version of this license is in
%    https://www.latex-project.org/lppl/lppl-1-3c.txt
% and the latest version of this license is in
%    https://www.latex-project.org/lppl.txt
% and version 1.3 or later is part of all distributions of
% LaTeX version 2005/12/01 or later.
%
% This work has the LPPL maintenance status "maintained".
%
% The Current Maintainers of this work are
% Heiko Oberdiek and the Oberdiek Package Support Group
% https://github.com/ho-tex/oberdiek/issues
%
% This work consists of the main source file pagegrid.dtx
% and the derived files
%    pagegrid.sty, pagegrid.pdf, pagegrid.ins, pagegrid.drv,
%    pagegrid-test1.tex.
%
% Distribution:
%    CTAN:macros/latex/contrib/oberdiek/pagegrid.dtx
%    CTAN:macros/latex/contrib/oberdiek/pagegrid.pdf
%
% Unpacking:
%    (a) If pagegrid.ins is present:
%           tex pagegrid.ins
%    (b) Without pagegrid.ins:
%           tex pagegrid.dtx
%    (c) If you insist on using LaTeX
%           latex \let\install=y\input{pagegrid.dtx}
%        (quote the arguments according to the demands of your shell)
%
% Documentation:
%    (a) If pagegrid.drv is present:
%           latex pagegrid.drv
%    (b) Without pagegrid.drv:
%           latex pagegrid.dtx; ...
%    The class ltxdoc loads the configuration file ltxdoc.cfg
%    if available. Here you can specify further options, e.g.
%    use A4 as paper format:
%       \PassOptionsToClass{a4paper}{article}
%
%    Programm calls to get the documentation (example):
%       pdflatex pagegrid.dtx
%       makeindex -s gind.ist pagegrid.idx
%       pdflatex pagegrid.dtx
%       makeindex -s gind.ist pagegrid.idx
%       pdflatex pagegrid.dtx
%
% Installation:
%    TDS:tex/latex/oberdiek/pagegrid.sty
%    TDS:doc/latex/oberdiek/pagegrid.pdf
%    TDS:doc/latex/oberdiek/test/pagegrid-test1.tex
%    TDS:source/latex/oberdiek/pagegrid.dtx
%
%<*ignore>
\begingroup
  \catcode123=1 %
  \catcode125=2 %
  \def\x{LaTeX2e}%
\expandafter\endgroup
\ifcase 0\ifx\install y1\fi\expandafter
         \ifx\csname processbatchFile\endcsname\relax\else1\fi
         \ifx\fmtname\x\else 1\fi\relax
\else\csname fi\endcsname
%</ignore>
%<*install>
\input docstrip.tex
\Msg{************************************************************************}
\Msg{* Installation}
\Msg{* Package: pagegrid 2016/05/16 v1.5 Print page grid in background (HO)}
\Msg{************************************************************************}

\keepsilent
\askforoverwritefalse

\let\MetaPrefix\relax
\preamble

This is a generated file.

Project: pagegrid
Version: 2016/05/16 v1.5

Copyright (C) 2009 by
   Heiko Oberdiek <heiko.oberdiek at googlemail.com>

This work may be distributed and/or modified under the
conditions of the LaTeX Project Public License, either
version 1.3c of this license or (at your option) any later
version. This version of this license is in
   https://www.latex-project.org/lppl/lppl-1-3c.txt
and the latest version of this license is in
   https://www.latex-project.org/lppl.txt
and version 1.3 or later is part of all distributions of
LaTeX version 2005/12/01 or later.

This work has the LPPL maintenance status "maintained".

The Current Maintainers of this work are
Heiko Oberdiek and the Oberdiek Package Support Group
https://github.com/ho-tex/oberdiek/issues


This work consists of the main source file pagegrid.dtx
and the derived files
   pagegrid.sty, pagegrid.pdf, pagegrid.ins, pagegrid.drv,
   pagegrid-test1.tex.

\endpreamble
\let\MetaPrefix\DoubleperCent

\generate{%
  \file{pagegrid.ins}{\from{pagegrid.dtx}{install}}%
  \file{pagegrid.drv}{\from{pagegrid.dtx}{driver}}%
  \usedir{tex/latex/oberdiek}%
  \file{pagegrid.sty}{\from{pagegrid.dtx}{package}}%
%  \usedir{doc/latex/oberdiek/test}%
%  \file{pagegrid-test1.tex}{\from{pagegrid.dtx}{test1}}%
  \nopreamble
  \nopostamble
%  \usedir{source/latex/oberdiek/catalogue}%
%  \file{pagegrid.xml}{\from{pagegrid.dtx}{catalogue}}%
}

\catcode32=13\relax% active space
\let =\space%
\Msg{************************************************************************}
\Msg{*}
\Msg{* To finish the installation you have to move the following}
\Msg{* file into a directory searched by TeX:}
\Msg{*}
\Msg{*     pagegrid.sty}
\Msg{*}
\Msg{* To produce the documentation run the file `pagegrid.drv'}
\Msg{* through LaTeX.}
\Msg{*}
\Msg{* Happy TeXing!}
\Msg{*}
\Msg{************************************************************************}

\endbatchfile
%</install>
%<*ignore>
\fi
%</ignore>
%<*driver>
\NeedsTeXFormat{LaTeX2e}
\ProvidesFile{pagegrid.drv}%
  [2016/05/16 v1.5 Print page grid in background (HO)]%
\documentclass{ltxdoc}
\usepackage{holtxdoc}[2011/11/22]
\begin{document}
  \DocInput{pagegrid.dtx}%
\end{document}
%</driver>
% \fi
%
%
% \CharacterTable
%  {Upper-case    \A\B\C\D\E\F\G\H\I\J\K\L\M\N\O\P\Q\R\S\T\U\V\W\X\Y\Z
%   Lower-case    \a\b\c\d\e\f\g\h\i\j\k\l\m\n\o\p\q\r\s\t\u\v\w\x\y\z
%   Digits        \0\1\2\3\4\5\6\7\8\9
%   Exclamation   \!     Double quote  \"     Hash (number) \#
%   Dollar        \$     Percent       \%     Ampersand     \&
%   Acute accent  \'     Left paren    \(     Right paren   \)
%   Asterisk      \*     Plus          \+     Comma         \,
%   Minus         \-     Point         \.     Solidus       \/
%   Colon         \:     Semicolon     \;     Less than     \<
%   Equals        \=     Greater than  \>     Question mark \?
%   Commercial at \@     Left bracket  \[     Backslash     \\
%   Right bracket \]     Circumflex    \^     Underscore    \_
%   Grave accent  \`     Left brace    \{     Vertical bar  \|
%   Right brace   \}     Tilde         \~}
%
% \GetFileInfo{pagegrid.drv}
%
% \title{The \xpackage{pagegrid} package}
% \date{2016/05/16 v1.5}
% \author{Heiko Oberdiek\thanks
% {Please report any issues at \url{https://github.com/ho-tex/oberdiek/issues}}}
%
% \maketitle
%
% \begin{abstract}
% The \LaTeX\ package prints a page grid in the background.
% \end{abstract}
%
% \tableofcontents
%
% \section{Documentation}
%
% The package puts a grid on the paper. It was written for
% developers of a class or package
% who have to put elements on definite locations on a page
% (e.g. letter class). The grid allows a faster optical check,
% whether the positions are correct. If the previewer already
% offers features for measuring, the package might be obsolete.
% Otherwise it saves the developer from printing the page and
% measuring by hand.
%
% \subsection{Options}
%
% Options are evaluated in the following order:
% \begin{enumerate}
% \item
%  Configuration file \xfile{pagegrid.cfg} using \cs{pagegridsetup}
%  if the file exists.
%  \item
%  Package options given for \cs{usepackage}.
%  \item
%  Later calls of \cs{pagegridsetup}.
% \end{enumerate}
% \begin{declcs}{pagegridsetup}\M{option list}
% \end{declcs}
% The options are key value options. Boolean options are enabled by
% default (without value) or by using the explicit value \texttt{true}.
% Value \texttt{false} disable the option.
%
% \subsubsection{Options \xoption{enable}, \xoption{disable}}
%
% \begin{description}
% \item[\xoption{enable}:] This boolean option controls whether the page grid
%   is drawn. As default the page grid drawing is activated.
% \item[\xoption{disable}:] It is the opposite
%   of option \xoption{enable}. It was added for convenience and
%   allows the abbreviation \texttt{disable} for \texttt{enable=false}.
% \end{description}
%
% \subsubsection{Grid origins}
%
% The package supports up to two grids on a page allowing
% measurement from opposite directions. As default two grids are drawn,
% the first from bottom left to top right. The origin of the second
% grid is at the opposite top right corner.
% The origins are controlled by the following options.
% The number of grids (one or two) depend on the number of these options
% in one call of \cs{pagegridsetup}.
% The following frame shows a paper and in its corners are the
% corresponding options. At the left and right side alias names
% are given for the options inside the paper.
% \begin{quote}
% \begin{tabular}{@{}r|@{\,}l@{\qquad}r@{\,}|l@{}}
% \cline{2-3}
% \xoption{left-top}, \xoption{lt}, \xoption{top-left}
% & \vphantom{\"U}\xoption{tl} & \xoption{tr}
% & \xoption{top-right}, \xoption{rt}, \xoption{right-top}\\
% &&&\\
% \xoption{left-bottom}, \xoption{lb}, \xoption{bottom-left}
% & \xoption{bl} & \xoption{br}
% & \xoption{bottom-right}, \xoption{rb}, \xoption{right-bottom}\\
% \cline{2-3}
% \end{tabular}
% \end{quote}
% Examples:
% \begin{quote}
% |\pagegridsetup{bl,tr}|
% \end{quote}
% This is the default setting with two grids as described previously.
% The following setups one grid only. Its origin is the upper left
% corner:
% \begin{quote}
% |\pagegridsetup{top-left}|
% \end{quote}
%
% \subsubsection{Grid unit}
%
% \begin{description}
% \item[\xoption{step}] This option takes a length and
% setups the unit for the grid. The page width and page height
% should be multiples of this unit.
% Currently the default is \texttt{1mm}. But this might change
% later by a heuristic based on the paper size.
% \end{description}
%
% \subsubsection{Color options}
%
% The basic grid lines are drawn as ultra thin help lines and is only
% drawn for the first grid.
% Each tenth and fiftyth line of the basic net is drawn thicker in a special
% color for the two grids.
% \begin{description}
% \item[\xoption{firstcolor}:] Color for the thicker lines and the arrows
% of the first grid. Default value is \texttt{red}.
% \item[\xoption{secondcolor}:] Color for the thicker lines and the arrows
% of the second grid. Default value is \texttt{blue}.
% \end{description}
% Use a color specification that package \xpackage{tikz} understands.
% (The grid is drawn with \xpackage{pgf}/\xpackage{tikz}.)
%
% \subsubsection{Arrow options}
%
% Arrows are put at the origin at the grid to show the grid start
% and the direction of the grid.
% \begin{description}
% \item[\xoption{arrows}:] This boolean option turns the arrows on or off.
% As default arrows are enabled.
% \item[\xoption{arrowlength}:] The length given as value is the
% length of the edge of a square at the origin within the
% arrow is put as diagonal. Default is 10 times the grid unit (10\,mm).
% The real arrow length is this length multiplied by $\sqrt2$.
% \end{description}
%
% \subsubsection{Miscellaneous options}
%
% \begin{description}
% \item[\xoption{double}:] The output page is doubled, one without page
% grid and the other with page grid. Possible values are shown in the
% following table:
% \begin{quote}
% \begin{tabular}{ll}
% Option & Meaning\\
% \hline
% |false| & Turns option off.\\
% |first| & Grid page comes first.\\
% |last| & Grid page comes after the page without grid.\\
% |true| & Same as |last|.\\
% \meta{no value} & Same as |true|.\\
% \end{tabular}
% \end{quote}
% \textbf{Note:}
% The double output of the page has side effects.
% All whatits are executed twice, for example: file writing
% and anchor setting. Some unwanted actions are catched such
% as multiple \cs{label} definitions, duplicate entries in
% the table of contents. For bookmarks, use package \xpackage{bookmarks}.
% \item[\xoption{foreground}:] Boolean option, default is \texttt{false}.
% Sometimes there might be elements on the page (e.g. large images)
% that hide the grid. Then option \xoption{foreground} puts the grids
% over the current output page.
% \end{description}
%
% \StopEventually{
% }
%
% \section{Implementation}
%    \begin{macrocode}
%<*package>
%    \end{macrocode}
%    Reload check, especially if the package is not used with \LaTeX.
%    \begin{macrocode}
\begingroup\catcode61\catcode48\catcode32=10\relax%
  \catcode13=5 % ^^M
  \endlinechar=13 %
  \catcode35=6 % #
  \catcode39=12 % '
  \catcode44=12 % ,
  \catcode45=12 % -
  \catcode46=12 % .
  \catcode58=12 % :
  \catcode64=11 % @
  \catcode123=1 % {
  \catcode125=2 % }
  \expandafter\let\expandafter\x\csname ver@pagegrid.sty\endcsname
  \ifx\x\relax % plain-TeX, first loading
  \else
    \def\empty{}%
    \ifx\x\empty % LaTeX, first loading,
      % variable is initialized, but \ProvidesPackage not yet seen
    \else
      \expandafter\ifx\csname PackageInfo\endcsname\relax
        \def\x#1#2{%
          \immediate\write-1{Package #1 Info: #2.}%
        }%
      \else
        \def\x#1#2{\PackageInfo{#1}{#2, stopped}}%
      \fi
      \x{pagegrid}{The package is already loaded}%
      \aftergroup\endinput
    \fi
  \fi
\endgroup%
%    \end{macrocode}
%    Package identification:
%    \begin{macrocode}
\begingroup\catcode61\catcode48\catcode32=10\relax%
  \catcode13=5 % ^^M
  \endlinechar=13 %
  \catcode35=6 % #
  \catcode39=12 % '
  \catcode40=12 % (
  \catcode41=12 % )
  \catcode44=12 % ,
  \catcode45=12 % -
  \catcode46=12 % .
  \catcode47=12 % /
  \catcode58=12 % :
  \catcode64=11 % @
  \catcode91=12 % [
  \catcode93=12 % ]
  \catcode123=1 % {
  \catcode125=2 % }
  \expandafter\ifx\csname ProvidesPackage\endcsname\relax
    \def\x#1#2#3[#4]{\endgroup
      \immediate\write-1{Package: #3 #4}%
      \xdef#1{#4}%
    }%
  \else
    \def\x#1#2[#3]{\endgroup
      #2[{#3}]%
      \ifx#1\@undefined
        \xdef#1{#3}%
      \fi
      \ifx#1\relax
        \xdef#1{#3}%
      \fi
    }%
  \fi
\expandafter\x\csname ver@pagegrid.sty\endcsname
\ProvidesPackage{pagegrid}%
  [2016/05/16 v1.5 Print page grid in background (HO)]%
%    \end{macrocode}
%
%    \begin{macrocode}
\begingroup\catcode61\catcode48\catcode32=10\relax%
  \catcode13=5 % ^^M
  \endlinechar=13 %
  \catcode123=1 % {
  \catcode125=2 % }
  \catcode64=11 % @
  \def\x{\endgroup
    \expandafter\edef\csname pagegrid@AtEnd\endcsname{%
      \endlinechar=\the\endlinechar\relax
      \catcode13=\the\catcode13\relax
      \catcode32=\the\catcode32\relax
      \catcode35=\the\catcode35\relax
      \catcode61=\the\catcode61\relax
      \catcode64=\the\catcode64\relax
      \catcode123=\the\catcode123\relax
      \catcode125=\the\catcode125\relax
    }%
  }%
\x\catcode61\catcode48\catcode32=10\relax%
\catcode13=5 % ^^M
\endlinechar=13 %
\catcode35=6 % #
\catcode64=11 % @
\catcode123=1 % {
\catcode125=2 % }
\def\TMP@EnsureCode#1#2{%
  \edef\pagegrid@AtEnd{%
    \pagegrid@AtEnd
    \catcode#1=\the\catcode#1\relax
  }%
  \catcode#1=#2\relax
}
\TMP@EnsureCode{9}{10}% (tab)
\TMP@EnsureCode{10}{12}% ^^J
\TMP@EnsureCode{33}{12}% !
\TMP@EnsureCode{34}{12}% "
\TMP@EnsureCode{36}{3}% $
\TMP@EnsureCode{38}{4}% &
\TMP@EnsureCode{39}{12}% '
\TMP@EnsureCode{40}{12}% (
\TMP@EnsureCode{41}{12}% )
\TMP@EnsureCode{42}{12}% *
\TMP@EnsureCode{43}{12}% +
\TMP@EnsureCode{44}{12}% ,
\TMP@EnsureCode{45}{12}% -
\TMP@EnsureCode{46}{12}% .
\TMP@EnsureCode{47}{12}% /
\TMP@EnsureCode{58}{12}% :
\TMP@EnsureCode{59}{12}% ;
\TMP@EnsureCode{60}{12}% <
\TMP@EnsureCode{62}{12}% >
\TMP@EnsureCode{63}{12}% ?
\TMP@EnsureCode{91}{12}% [
\TMP@EnsureCode{93}{12}% ]
\TMP@EnsureCode{94}{7}% ^ (superscript)
\TMP@EnsureCode{95}{8}% _ (subscript)
\TMP@EnsureCode{96}{12}% `
\TMP@EnsureCode{124}{12}% |
\edef\pagegrid@AtEnd{\pagegrid@AtEnd\noexpand\endinput}
%    \end{macrocode}
%
%    \begin{macrocode}
\RequirePackage{tikz}
\RequirePackage{atbegshi}[2009/12/02]
\RequirePackage{kvoptions}[2009/07/17]
%    \end{macrocode}
%    \begin{macrocode}
\begingroup\expandafter\expandafter\expandafter\endgroup
\expandafter\ifx\csname stockwidth\endcsname\relax
  \def\pagegrid@width{\paperwidth}%
  \def\pagegrid@height{\paperheight}%
\else
  \def\pagegrid@width{\stockwidth}%
  \def\pagegrid@height{\stockheight}%
\fi
%    \end{macrocode}
%
%    \begin{macrocode}
\SetupKeyvalOptions{%
  family=pagegrid,%
  prefix=pagegrid@,%
}
\def\pagegrid@init{%
  \let\pagegrid@origin@a\@empty
  \let\pagegrid@origin@b\@empty
  \let\pagegrid@init\relax
}
\let\pagegrid@@init\pagegrid@init
\def\pagegrid@origin@a{bl}
\def\pagegrid@origin@b{tr}
\def\pagegrid@SetOrigin#1{%
  \pagegrid@init
  \ifx\pagegrid@origin@a\@empty
    \def\pagegrid@origin@a{#1}%
  \else
    \ifx\pagegrid@origin@b\@empty
    \else
      \let\pagegrid@origin@a\pagegrid@origin@b
    \fi
    \def\pagegrid@origin@b{#1}%
  \fi
}
\def\pagegrid@temp#1{%
  \DeclareVoidOption{#1}{\pagegrid@SetOrigin{#1}}%
  \@namedef{pagegrid@N@#1}{#1}%
}
\pagegrid@temp{bl}
\pagegrid@temp{br}
\pagegrid@temp{tl}
\pagegrid@temp{tr}
\def\pagegrid@temp#1#2{%
  \DeclareVoidOption{#2}{\pagegrid@SetOrigin{#1}}%
}%
\pagegrid@temp{bl}{lb}
\pagegrid@temp{br}{rb}
\pagegrid@temp{tl}{lt}
\pagegrid@temp{tr}{rt}
\pagegrid@temp{bl}{bottom-left}
\pagegrid@temp{br}{bottom-right}
\pagegrid@temp{tl}{top-left}
\pagegrid@temp{tr}{top-right}
\pagegrid@temp{bl}{left-bottom}
\pagegrid@temp{br}{right-bottom}
\pagegrid@temp{tl}{left-top}
\pagegrid@temp{tr}{right-top}
%    \end{macrocode}
%    \begin{macrocode}
\DeclareBoolOption[true]{enable}
\DeclareComplementaryOption{disable}{enable}
%    \end{macrocode}
%    \begin{macrocode}
\DeclareBoolOption{foreground}
%    \end{macrocode}
%    \begin{macrocode}
\newlength{\pagegrid@step}
\define@key{pagegrid}{step}{%
  \setlength{\pagegrid@step}{#1}%
}
%    \end{macrocode}
%    \begin{macrocode}
\DeclareStringOption[red]{firstcolor}
\DeclareStringOption[blue]{secondcolor}
%    \end{macrocode}
%    \begin{macrocode}
\DeclareBoolOption[true]{arrows}
\newlength\pagegrid@arrowlength
\pagegrid@arrowlength=\z@
\define@key{pagegrid}{arrowlength}{%
  \setlength{\pagegrid@arrowlength}{#1}%
}
%    \end{macrocode}
%    \begin{macrocode}
\define@key{pagegrid}{double}[true]{%
  \@ifundefined{pagegrid@double@#1}{%
    \PackageWarning{pagegrid}{%
      Unsupported value `#1' for option `double'.\MessageBreak
      Known values are:\MessageBreak
      `false', `first', `last', `true'.\MessageBreak
      Now `false' is used%
    }%
    \chardef\pagegrid@double\z@
  }{%
    \chardef\pagegrid@double\csname pagegrid@double@#1\endcsname\relax
  }%
}
\@namedef{pagegrid@double@false}{0}
\@namedef{pagegrid@double@first}{1}
\@namedef{pagegrid@double@last}{2}
\@namedef{pagegrid@double@true}{2}
\chardef\pagegrid@double\z@
%    \end{macrocode}
%    \begin{macrocode}
\newcommand*{\pagegridsetup}{%
  \let\pagegrid@init\pagegrid@@init
  \setkeys{pagegrid}%
}
%    \end{macrocode}
%    \begin{macrocode}
\pagegridsetup{%
  step=1mm%
}
\InputIfFileExists{pagegrid.cfg}{}%
\ProcessKeyvalOptions*\relax
\AtBeginDocument{%
  \ifdim\pagegrid@arrowlength>\z@
  \else
    \pagegrid@arrowlength=10\pagegrid@step
  \fi
}
%    \end{macrocode}
%
%    \begin{macrocode}
\def\pagegridShipoutDoubleBegin{%
  \begingroup
  \let\newlabel\@gobbletwo
  \let\zref@newlabel\@gobbletwo
  \let\@writefile\@gobbletwo
  \let\select@language\@gobble
}
\def\pagegridShipoutDoubleEnd{%
  \endgroup
}
\def\pagegrid@WriteDouble#1#2{%
  \immediate\write#1{%
    \@backslashchar csname %
    pagegridShipoutDouble#2%
    \@backslashchar endcsname%
  }%
}
\def\pagegrid@ShipoutDouble#1{%
  \begingroup
    \if@filesw
      \pagegrid@WriteDouble\@mainaux{Begin}%
      \ifx\@auxout\@partaux
        \pagegrid@WriteDouble\@partaux{Begin}%
        \def\pagegrid@temp{%
          \pagegrid@WriteDouble\@mainaux{End}%
          \pagegrid@WriteDouble\@partaux{End}%
        }%
      \else
        \def\pagegrid@temp{%
          \pagegrid@WriteDouble\@mainaux{End}%
        }%
      \fi
    \else
      \def\pagegrid@temp{}%
    \fi
    \let\protect\noexpand
    \AtBeginShipoutOriginalShipout\copy#1\relax
    \pagegrid@temp
  \endgroup
}
%    \end{macrocode}
%
%    \begin{macrocode}
\AtBeginShipout{%
  \ifdim\pagegrid@step>\z@
  \else
    \pagegrid@enablefalse
  \fi
  \ifpagegrid@enable
    \ifnum\pagegrid@double=\@ne
      \pagegrid@ShipoutDouble\AtBeginShipoutBox
    \else
      \ifnum\pagegrid@double=\tw@
        \@ifundefined{pagegrid@DoubleBox}{%
          \newbox\pagegrid@DoubleBox
        }{}%
        \setbox\pagegrid@DoubleBox=\copy\AtBeginShipoutBox
      \fi
    \fi
    \ifpagegrid@foreground
      \expandafter\AtBeginShipoutUpperLeftForeground
    \else
      \expandafter\AtBeginShipoutUpperLeft
    \fi
    {%
      \put(0,0){%
        \makebox(0,0)[lt]{%
          \begin{tikzpicture}[%
            bl/.style={},%
            br/.style={xshift=\pagegrid@width,xscale=-1},%
            tl/.style={yshift=\pagegrid@height,yscale=-1},%
            tr/.style={xshift=\pagegrid@width,%
                       yshift=\pagegrid@height,scale=-1}%
          ]%
            \useasboundingbox
              (0mm,\pagegrid@height) rectangle (0mm,\pagegrid@height);%
            \draw[%
              \pagegrid@origin@a,%
              step=\pagegrid@step,%
              style=help lines,%
              ultra thin%
            ] (0mm,0mm) grid (\pagegrid@width,\pagegrid@height);%
            \ifx\pagegrid@origin@b\@empty
            \else
              \draw[%
                \pagegrid@origin@b,%
                step=10\pagegrid@step,%
                {\pagegrid@secondcolor},%
                very thin%
              ] (0mm,0mm) grid (\pagegrid@width,\pagegrid@height);%
            \fi
            \draw[%
               \pagegrid@origin@a,%
               step=10\pagegrid@step,%
               {\pagegrid@firstcolor},%
               very thin%
            ] (0mm,0mm) grid (\pagegrid@width,\pagegrid@height);%
            \ifx\pagegrid@origin@b\@empty
            \else
              \draw[%
                \pagegrid@origin@b,%
                step=50\pagegrid@step,%
                {\pagegrid@secondcolor},%
                thick%
              ] (0mm,0mm) grid (\pagegrid@width,\pagegrid@height);%
            \fi
            \draw[%
              \pagegrid@origin@a,%
              step=50\pagegrid@step,%
              {\pagegrid@firstcolor},%
              thick%
            ] (0mm,0mm) grid (\pagegrid@width,\pagegrid@height);%
            \ifpagegrid@arrows
              \ifx\pagegrid@origin@b\@empty
              \else
                \draw[%
                  \pagegrid@origin@b,%
                  {\pagegrid@secondcolor},%
                  stroke,%
                  line width=1pt,%
                  line cap=round%
                ] (0mm,0mm) %
                -- (\pagegrid@arrowlength,\pagegrid@arrowlength) %
                   (\pagegrid@arrowlength,.5\pagegrid@arrowlength) %
                -- (\pagegrid@arrowlength,\pagegrid@arrowlength) %
                -- (.5\pagegrid@arrowlength,\pagegrid@arrowlength);%
              \fi
              \draw[%
                \pagegrid@origin@a,%
                {\pagegrid@firstcolor},%
                stroke,%
                line width=1pt,%
                line cap=round%
              ] (0mm,0mm) %
              -- (\pagegrid@arrowlength,\pagegrid@arrowlength) %
                 (\pagegrid@arrowlength,.5\pagegrid@arrowlength) %
              -- (\pagegrid@arrowlength,\pagegrid@arrowlength) %
              -- (.5\pagegrid@arrowlength,\pagegrid@arrowlength);%
            \fi
          \end{tikzpicture}%
        }%
      }%
    }%
    \ifnum\pagegrid@double=\tw@
      \pagegrid@ShipoutDouble\pagegrid@DoubleBox
    \fi
  \fi
}
%    \end{macrocode}
%
%    \begin{macrocode}
\pagegrid@AtEnd%
%</package>
%    \end{macrocode}
%
% \section{Test}
%
% \subsection{Catcode checks for loading}
%
%    \begin{macrocode}
%<*test1>
%    \end{macrocode}
%    \begin{macrocode}
\catcode`\{=1 %
\catcode`\}=2 %
\catcode`\#=6 %
\catcode`\@=11 %
\expandafter\ifx\csname count@\endcsname\relax
  \countdef\count@=255 %
\fi
\expandafter\ifx\csname @gobble\endcsname\relax
  \long\def\@gobble#1{}%
\fi
\expandafter\ifx\csname @firstofone\endcsname\relax
  \long\def\@firstofone#1{#1}%
\fi
\expandafter\ifx\csname loop\endcsname\relax
  \expandafter\@firstofone
\else
  \expandafter\@gobble
\fi
{%
  \def\loop#1\repeat{%
    \def\body{#1}%
    \iterate
  }%
  \def\iterate{%
    \body
      \let\next\iterate
    \else
      \let\next\relax
    \fi
    \next
  }%
  \let\repeat=\fi
}%
\def\RestoreCatcodes{}
\count@=0 %
\loop
  \edef\RestoreCatcodes{%
    \RestoreCatcodes
    \catcode\the\count@=\the\catcode\count@\relax
  }%
\ifnum\count@<255 %
  \advance\count@ 1 %
\repeat

\def\RangeCatcodeInvalid#1#2{%
  \count@=#1\relax
  \loop
    \catcode\count@=15 %
  \ifnum\count@<#2\relax
    \advance\count@ 1 %
  \repeat
}
\def\RangeCatcodeCheck#1#2#3{%
  \count@=#1\relax
  \loop
    \ifnum#3=\catcode\count@
    \else
      \errmessage{%
        Character \the\count@\space
        with wrong catcode \the\catcode\count@\space
        instead of \number#3%
      }%
    \fi
  \ifnum\count@<#2\relax
    \advance\count@ 1 %
  \repeat
}
\def\space{ }
\expandafter\ifx\csname LoadCommand\endcsname\relax
  \def\LoadCommand{\input pagegrid.sty\relax}%
\fi
\def\Test{%
  \RangeCatcodeInvalid{0}{47}%
  \RangeCatcodeInvalid{58}{64}%
  \RangeCatcodeInvalid{91}{96}%
  \RangeCatcodeInvalid{123}{255}%
  \catcode`\@=12 %
  \catcode`\\=0 %
  \catcode`\%=14 %
  \LoadCommand
  \RangeCatcodeCheck{0}{36}{15}%
  \RangeCatcodeCheck{37}{37}{14}%
  \RangeCatcodeCheck{38}{47}{15}%
  \RangeCatcodeCheck{48}{57}{12}%
  \RangeCatcodeCheck{58}{63}{15}%
  \RangeCatcodeCheck{64}{64}{12}%
  \RangeCatcodeCheck{65}{90}{11}%
  \RangeCatcodeCheck{91}{91}{15}%
  \RangeCatcodeCheck{92}{92}{0}%
  \RangeCatcodeCheck{93}{96}{15}%
  \RangeCatcodeCheck{97}{122}{11}%
  \RangeCatcodeCheck{123}{255}{15}%
  \RestoreCatcodes
}
\Test
\csname @@end\endcsname
\end
%    \end{macrocode}
%    \begin{macrocode}
%</test1>
%    \end{macrocode}
%
% \section{Installation}
%
% \subsection{Download}
%
% \paragraph{Package.} This package is available on
% CTAN\footnote{\CTANpkg{pagegrid}}:
% \begin{description}
% \item[\CTAN{macros/latex/contrib/oberdiek/pagegrid.dtx}] The source file.
% \item[\CTAN{macros/latex/contrib/oberdiek/pagegrid.pdf}] Documentation.
% \end{description}
%
%
% \paragraph{Bundle.} All the packages of the bundle `oberdiek'
% are also available in a TDS compliant ZIP archive. There
% the packages are already unpacked and the documentation files
% are generated. The files and directories obey the TDS standard.
% \begin{description}
% \item[\CTANinstall{install/macros/latex/contrib/oberdiek.tds.zip}]
% \end{description}
% \emph{TDS} refers to the standard ``A Directory Structure
% for \TeX\ Files'' (\CTAN{tds/tds.pdf}). Directories
% with \xfile{texmf} in their name are usually organized this way.
%
% \subsection{Bundle installation}
%
% \paragraph{Unpacking.} Unpack the \xfile{oberdiek.tds.zip} in the
% TDS tree (also known as \xfile{texmf} tree) of your choice.
% Example (linux):
% \begin{quote}
%   |unzip oberdiek.tds.zip -d ~/texmf|
% \end{quote}
%
% \paragraph{Script installation.}
% Check the directory \xfile{TDS:scripts/oberdiek/} for
% scripts that need further installation steps.
%
% \subsection{Package installation}
%
% \paragraph{Unpacking.} The \xfile{.dtx} file is a self-extracting
% \docstrip\ archive. The files are extracted by running the
% \xfile{.dtx} through \plainTeX:
% \begin{quote}
%   \verb|tex pagegrid.dtx|
% \end{quote}
%
% \paragraph{TDS.} Now the different files must be moved into
% the different directories in your installation TDS tree
% (also known as \xfile{texmf} tree):
% \begin{quote}
% \def\t{^^A
% \begin{tabular}{@{}>{\ttfamily}l@{ $\rightarrow$ }>{\ttfamily}l@{}}
%   pagegrid.sty & tex/latex/oberdiek/pagegrid.sty\\
%   pagegrid.pdf & doc/latex/oberdiek/pagegrid.pdf\\
%   test/pagegrid-test1.tex & doc/latex/oberdiek/test/pagegrid-test1.tex\\
%   pagegrid.dtx & source/latex/oberdiek/pagegrid.dtx\\
% \end{tabular}^^A
% }^^A
% \sbox0{\t}^^A
% \ifdim\wd0>\linewidth
%   \begingroup
%     \advance\linewidth by\leftmargin
%     \advance\linewidth by\rightmargin
%   \edef\x{\endgroup
%     \def\noexpand\lw{\the\linewidth}^^A
%   }\x
%   \def\lwbox{^^A
%     \leavevmode
%     \hbox to \linewidth{^^A
%       \kern-\leftmargin\relax
%       \hss
%       \usebox0
%       \hss
%       \kern-\rightmargin\relax
%     }^^A
%   }^^A
%   \ifdim\wd0>\lw
%     \sbox0{\small\t}^^A
%     \ifdim\wd0>\linewidth
%       \ifdim\wd0>\lw
%         \sbox0{\footnotesize\t}^^A
%         \ifdim\wd0>\linewidth
%           \ifdim\wd0>\lw
%             \sbox0{\scriptsize\t}^^A
%             \ifdim\wd0>\linewidth
%               \ifdim\wd0>\lw
%                 \sbox0{\tiny\t}^^A
%                 \ifdim\wd0>\linewidth
%                   \lwbox
%                 \else
%                   \usebox0
%                 \fi
%               \else
%                 \lwbox
%               \fi
%             \else
%               \usebox0
%             \fi
%           \else
%             \lwbox
%           \fi
%         \else
%           \usebox0
%         \fi
%       \else
%         \lwbox
%       \fi
%     \else
%       \usebox0
%     \fi
%   \else
%     \lwbox
%   \fi
% \else
%   \usebox0
% \fi
% \end{quote}
% If you have a \xfile{docstrip.cfg} that configures and enables \docstrip's
% TDS installing feature, then some files can already be in the right
% place, see the documentation of \docstrip.
%
% \subsection{Refresh file name databases}
%
% If your \TeX~distribution
% (\TeX\,Live, \mikTeX, \dots) relies on file name databases, you must refresh
% these. For example, \TeX\,Live\ users run \verb|texhash| or
% \verb|mktexlsr|.
%
% \subsection{Some details for the interested}
%
% \paragraph{Unpacking with \LaTeX.}
% The \xfile{.dtx} chooses its action depending on the format:
% \begin{description}
% \item[\plainTeX:] Run \docstrip\ and extract the files.
% \item[\LaTeX:] Generate the documentation.
% \end{description}
% If you insist on using \LaTeX\ for \docstrip\ (really,
% \docstrip\ does not need \LaTeX), then inform the autodetect routine
% about your intention:
% \begin{quote}
%   \verb|latex \let\install=y\input{pagegrid.dtx}|
% \end{quote}
% Do not forget to quote the argument according to the demands
% of your shell.
%
% \paragraph{Generating the documentation.}
% You can use both the \xfile{.dtx} or the \xfile{.drv} to generate
% the documentation. The process can be configured by the
% configuration file \xfile{ltxdoc.cfg}. For instance, put this
% line into this file, if you want to have A4 as paper format:
% \begin{quote}
%   \verb|\PassOptionsToClass{a4paper}{article}|
% \end{quote}
% An example follows how to generate the
% documentation with pdf\LaTeX:
% \begin{quote}
%\begin{verbatim}
%pdflatex pagegrid.dtx
%makeindex -s gind.ist pagegrid.idx
%pdflatex pagegrid.dtx
%makeindex -s gind.ist pagegrid.idx
%pdflatex pagegrid.dtx
%\end{verbatim}
% \end{quote}
%
% \section{Acknowledgement}
%
% \begin{description}
% \item[Klaus Braune:]
%  He provided the idea and the first \xpackage{tikz} code.
% \end{description}
%
% \begin{History}
%   \begin{Version}{2009/11/06 v1.0}
%   \item
%     The first version.
%   \end{Version}
%   \begin{Version}{2009/11/06 v1.1}
%   \item
%     Option \xoption{foreground} added.
%   \end{Version}
%   \begin{Version}{2009/12/02 v1.2}
%   \item
%     Color options, arrow options added.
%   \item
%     Names for origin options changed.
%   \end{Version}
%   \begin{Version}{2009/12/03 v1.3}
%   \item
%     Option \xoption{double} added.
%   \item
%     First CTAN release.
%   \end{Version}
%   \begin{Version}{2009/12/04 v1.4}
%   \item
%     Option \xoption{double}: Some unwanted side effects removed.
%   \end{Version}
%   \begin{Version}{2016/05/16 v1.5}
%   \item
%     Documentation updates.
%   \end{Version}
% \end{History}
%
% \PrintIndex
%
% \Finale
\endinput

%        (quote the arguments according to the demands of your shell)
%
% Documentation:
%    (a) If pagegrid.drv is present:
%           latex pagegrid.drv
%    (b) Without pagegrid.drv:
%           latex pagegrid.dtx; ...
%    The class ltxdoc loads the configuration file ltxdoc.cfg
%    if available. Here you can specify further options, e.g.
%    use A4 as paper format:
%       \PassOptionsToClass{a4paper}{article}
%
%    Programm calls to get the documentation (example):
%       pdflatex pagegrid.dtx
%       makeindex -s gind.ist pagegrid.idx
%       pdflatex pagegrid.dtx
%       makeindex -s gind.ist pagegrid.idx
%       pdflatex pagegrid.dtx
%
% Installation:
%    TDS:tex/latex/oberdiek/pagegrid.sty
%    TDS:doc/latex/oberdiek/pagegrid.pdf
%    TDS:doc/latex/oberdiek/test/pagegrid-test1.tex
%    TDS:source/latex/oberdiek/pagegrid.dtx
%
%<*ignore>
\begingroup
  \catcode123=1 %
  \catcode125=2 %
  \def\x{LaTeX2e}%
\expandafter\endgroup
\ifcase 0\ifx\install y1\fi\expandafter
         \ifx\csname processbatchFile\endcsname\relax\else1\fi
         \ifx\fmtname\x\else 1\fi\relax
\else\csname fi\endcsname
%</ignore>
%<*install>
\input docstrip.tex
\Msg{************************************************************************}
\Msg{* Installation}
\Msg{* Package: pagegrid 2016/05/16 v1.5 Print page grid in background (HO)}
\Msg{************************************************************************}

\keepsilent
\askforoverwritefalse

\let\MetaPrefix\relax
\preamble

This is a generated file.

Project: pagegrid
Version: 2016/05/16 v1.5

Copyright (C) 2009 by
   Heiko Oberdiek <heiko.oberdiek at googlemail.com>

This work may be distributed and/or modified under the
conditions of the LaTeX Project Public License, either
version 1.3c of this license or (at your option) any later
version. This version of this license is in
   https://www.latex-project.org/lppl/lppl-1-3c.txt
and the latest version of this license is in
   https://www.latex-project.org/lppl.txt
and version 1.3 or later is part of all distributions of
LaTeX version 2005/12/01 or later.

This work has the LPPL maintenance status "maintained".

The Current Maintainers of this work are
Heiko Oberdiek and the Oberdiek Package Support Group
https://github.com/ho-tex/oberdiek/issues


This work consists of the main source file pagegrid.dtx
and the derived files
   pagegrid.sty, pagegrid.pdf, pagegrid.ins, pagegrid.drv,
   pagegrid-test1.tex.

\endpreamble
\let\MetaPrefix\DoubleperCent

\generate{%
  \file{pagegrid.ins}{\from{pagegrid.dtx}{install}}%
  \file{pagegrid.drv}{\from{pagegrid.dtx}{driver}}%
  \usedir{tex/latex/oberdiek}%
  \file{pagegrid.sty}{\from{pagegrid.dtx}{package}}%
%  \usedir{doc/latex/oberdiek/test}%
%  \file{pagegrid-test1.tex}{\from{pagegrid.dtx}{test1}}%
  \nopreamble
  \nopostamble
%  \usedir{source/latex/oberdiek/catalogue}%
%  \file{pagegrid.xml}{\from{pagegrid.dtx}{catalogue}}%
}

\catcode32=13\relax% active space
\let =\space%
\Msg{************************************************************************}
\Msg{*}
\Msg{* To finish the installation you have to move the following}
\Msg{* file into a directory searched by TeX:}
\Msg{*}
\Msg{*     pagegrid.sty}
\Msg{*}
\Msg{* To produce the documentation run the file `pagegrid.drv'}
\Msg{* through LaTeX.}
\Msg{*}
\Msg{* Happy TeXing!}
\Msg{*}
\Msg{************************************************************************}

\endbatchfile
%</install>
%<*ignore>
\fi
%</ignore>
%<*driver>
\NeedsTeXFormat{LaTeX2e}
\ProvidesFile{pagegrid.drv}%
  [2016/05/16 v1.5 Print page grid in background (HO)]%
\documentclass{ltxdoc}
\usepackage{holtxdoc}[2011/11/22]
\begin{document}
  \DocInput{pagegrid.dtx}%
\end{document}
%</driver>
% \fi
%
%
% \CharacterTable
%  {Upper-case    \A\B\C\D\E\F\G\H\I\J\K\L\M\N\O\P\Q\R\S\T\U\V\W\X\Y\Z
%   Lower-case    \a\b\c\d\e\f\g\h\i\j\k\l\m\n\o\p\q\r\s\t\u\v\w\x\y\z
%   Digits        \0\1\2\3\4\5\6\7\8\9
%   Exclamation   \!     Double quote  \"     Hash (number) \#
%   Dollar        \$     Percent       \%     Ampersand     \&
%   Acute accent  \'     Left paren    \(     Right paren   \)
%   Asterisk      \*     Plus          \+     Comma         \,
%   Minus         \-     Point         \.     Solidus       \/
%   Colon         \:     Semicolon     \;     Less than     \<
%   Equals        \=     Greater than  \>     Question mark \?
%   Commercial at \@     Left bracket  \[     Backslash     \\
%   Right bracket \]     Circumflex    \^     Underscore    \_
%   Grave accent  \`     Left brace    \{     Vertical bar  \|
%   Right brace   \}     Tilde         \~}
%
% \GetFileInfo{pagegrid.drv}
%
% \title{The \xpackage{pagegrid} package}
% \date{2016/05/16 v1.5}
% \author{Heiko Oberdiek\thanks
% {Please report any issues at \url{https://github.com/ho-tex/oberdiek/issues}}}
%
% \maketitle
%
% \begin{abstract}
% The \LaTeX\ package prints a page grid in the background.
% \end{abstract}
%
% \tableofcontents
%
% \section{Documentation}
%
% The package puts a grid on the paper. It was written for
% developers of a class or package
% who have to put elements on definite locations on a page
% (e.g. letter class). The grid allows a faster optical check,
% whether the positions are correct. If the previewer already
% offers features for measuring, the package might be obsolete.
% Otherwise it saves the developer from printing the page and
% measuring by hand.
%
% \subsection{Options}
%
% Options are evaluated in the following order:
% \begin{enumerate}
% \item
%  Configuration file \xfile{pagegrid.cfg} using \cs{pagegridsetup}
%  if the file exists.
%  \item
%  Package options given for \cs{usepackage}.
%  \item
%  Later calls of \cs{pagegridsetup}.
% \end{enumerate}
% \begin{declcs}{pagegridsetup}\M{option list}
% \end{declcs}
% The options are key value options. Boolean options are enabled by
% default (without value) or by using the explicit value \texttt{true}.
% Value \texttt{false} disable the option.
%
% \subsubsection{Options \xoption{enable}, \xoption{disable}}
%
% \begin{description}
% \item[\xoption{enable}:] This boolean option controls whether the page grid
%   is drawn. As default the page grid drawing is activated.
% \item[\xoption{disable}:] It is the opposite
%   of option \xoption{enable}. It was added for convenience and
%   allows the abbreviation \texttt{disable} for \texttt{enable=false}.
% \end{description}
%
% \subsubsection{Grid origins}
%
% The package supports up to two grids on a page allowing
% measurement from opposite directions. As default two grids are drawn,
% the first from bottom left to top right. The origin of the second
% grid is at the opposite top right corner.
% The origins are controlled by the following options.
% The number of grids (one or two) depend on the number of these options
% in one call of \cs{pagegridsetup}.
% The following frame shows a paper and in its corners are the
% corresponding options. At the left and right side alias names
% are given for the options inside the paper.
% \begin{quote}
% \begin{tabular}{@{}r|@{\,}l@{\qquad}r@{\,}|l@{}}
% \cline{2-3}
% \xoption{left-top}, \xoption{lt}, \xoption{top-left}
% & \vphantom{\"U}\xoption{tl} & \xoption{tr}
% & \xoption{top-right}, \xoption{rt}, \xoption{right-top}\\
% &&&\\
% \xoption{left-bottom}, \xoption{lb}, \xoption{bottom-left}
% & \xoption{bl} & \xoption{br}
% & \xoption{bottom-right}, \xoption{rb}, \xoption{right-bottom}\\
% \cline{2-3}
% \end{tabular}
% \end{quote}
% Examples:
% \begin{quote}
% |\pagegridsetup{bl,tr}|
% \end{quote}
% This is the default setting with two grids as described previously.
% The following setups one grid only. Its origin is the upper left
% corner:
% \begin{quote}
% |\pagegridsetup{top-left}|
% \end{quote}
%
% \subsubsection{Grid unit}
%
% \begin{description}
% \item[\xoption{step}] This option takes a length and
% setups the unit for the grid. The page width and page height
% should be multiples of this unit.
% Currently the default is \texttt{1mm}. But this might change
% later by a heuristic based on the paper size.
% \end{description}
%
% \subsubsection{Color options}
%
% The basic grid lines are drawn as ultra thin help lines and is only
% drawn for the first grid.
% Each tenth and fiftyth line of the basic net is drawn thicker in a special
% color for the two grids.
% \begin{description}
% \item[\xoption{firstcolor}:] Color for the thicker lines and the arrows
% of the first grid. Default value is \texttt{red}.
% \item[\xoption{secondcolor}:] Color for the thicker lines and the arrows
% of the second grid. Default value is \texttt{blue}.
% \end{description}
% Use a color specification that package \xpackage{tikz} understands.
% (The grid is drawn with \xpackage{pgf}/\xpackage{tikz}.)
%
% \subsubsection{Arrow options}
%
% Arrows are put at the origin at the grid to show the grid start
% and the direction of the grid.
% \begin{description}
% \item[\xoption{arrows}:] This boolean option turns the arrows on or off.
% As default arrows are enabled.
% \item[\xoption{arrowlength}:] The length given as value is the
% length of the edge of a square at the origin within the
% arrow is put as diagonal. Default is 10 times the grid unit (10\,mm).
% The real arrow length is this length multiplied by $\sqrt2$.
% \end{description}
%
% \subsubsection{Miscellaneous options}
%
% \begin{description}
% \item[\xoption{double}:] The output page is doubled, one without page
% grid and the other with page grid. Possible values are shown in the
% following table:
% \begin{quote}
% \begin{tabular}{ll}
% Option & Meaning\\
% \hline
% |false| & Turns option off.\\
% |first| & Grid page comes first.\\
% |last| & Grid page comes after the page without grid.\\
% |true| & Same as |last|.\\
% \meta{no value} & Same as |true|.\\
% \end{tabular}
% \end{quote}
% \textbf{Note:}
% The double output of the page has side effects.
% All whatits are executed twice, for example: file writing
% and anchor setting. Some unwanted actions are catched such
% as multiple \cs{label} definitions, duplicate entries in
% the table of contents. For bookmarks, use package \xpackage{bookmarks}.
% \item[\xoption{foreground}:] Boolean option, default is \texttt{false}.
% Sometimes there might be elements on the page (e.g. large images)
% that hide the grid. Then option \xoption{foreground} puts the grids
% over the current output page.
% \end{description}
%
% \StopEventually{
% }
%
% \section{Implementation}
%    \begin{macrocode}
%<*package>
%    \end{macrocode}
%    Reload check, especially if the package is not used with \LaTeX.
%    \begin{macrocode}
\begingroup\catcode61\catcode48\catcode32=10\relax%
  \catcode13=5 % ^^M
  \endlinechar=13 %
  \catcode35=6 % #
  \catcode39=12 % '
  \catcode44=12 % ,
  \catcode45=12 % -
  \catcode46=12 % .
  \catcode58=12 % :
  \catcode64=11 % @
  \catcode123=1 % {
  \catcode125=2 % }
  \expandafter\let\expandafter\x\csname ver@pagegrid.sty\endcsname
  \ifx\x\relax % plain-TeX, first loading
  \else
    \def\empty{}%
    \ifx\x\empty % LaTeX, first loading,
      % variable is initialized, but \ProvidesPackage not yet seen
    \else
      \expandafter\ifx\csname PackageInfo\endcsname\relax
        \def\x#1#2{%
          \immediate\write-1{Package #1 Info: #2.}%
        }%
      \else
        \def\x#1#2{\PackageInfo{#1}{#2, stopped}}%
      \fi
      \x{pagegrid}{The package is already loaded}%
      \aftergroup\endinput
    \fi
  \fi
\endgroup%
%    \end{macrocode}
%    Package identification:
%    \begin{macrocode}
\begingroup\catcode61\catcode48\catcode32=10\relax%
  \catcode13=5 % ^^M
  \endlinechar=13 %
  \catcode35=6 % #
  \catcode39=12 % '
  \catcode40=12 % (
  \catcode41=12 % )
  \catcode44=12 % ,
  \catcode45=12 % -
  \catcode46=12 % .
  \catcode47=12 % /
  \catcode58=12 % :
  \catcode64=11 % @
  \catcode91=12 % [
  \catcode93=12 % ]
  \catcode123=1 % {
  \catcode125=2 % }
  \expandafter\ifx\csname ProvidesPackage\endcsname\relax
    \def\x#1#2#3[#4]{\endgroup
      \immediate\write-1{Package: #3 #4}%
      \xdef#1{#4}%
    }%
  \else
    \def\x#1#2[#3]{\endgroup
      #2[{#3}]%
      \ifx#1\@undefined
        \xdef#1{#3}%
      \fi
      \ifx#1\relax
        \xdef#1{#3}%
      \fi
    }%
  \fi
\expandafter\x\csname ver@pagegrid.sty\endcsname
\ProvidesPackage{pagegrid}%
  [2016/05/16 v1.5 Print page grid in background (HO)]%
%    \end{macrocode}
%
%    \begin{macrocode}
\begingroup\catcode61\catcode48\catcode32=10\relax%
  \catcode13=5 % ^^M
  \endlinechar=13 %
  \catcode123=1 % {
  \catcode125=2 % }
  \catcode64=11 % @
  \def\x{\endgroup
    \expandafter\edef\csname pagegrid@AtEnd\endcsname{%
      \endlinechar=\the\endlinechar\relax
      \catcode13=\the\catcode13\relax
      \catcode32=\the\catcode32\relax
      \catcode35=\the\catcode35\relax
      \catcode61=\the\catcode61\relax
      \catcode64=\the\catcode64\relax
      \catcode123=\the\catcode123\relax
      \catcode125=\the\catcode125\relax
    }%
  }%
\x\catcode61\catcode48\catcode32=10\relax%
\catcode13=5 % ^^M
\endlinechar=13 %
\catcode35=6 % #
\catcode64=11 % @
\catcode123=1 % {
\catcode125=2 % }
\def\TMP@EnsureCode#1#2{%
  \edef\pagegrid@AtEnd{%
    \pagegrid@AtEnd
    \catcode#1=\the\catcode#1\relax
  }%
  \catcode#1=#2\relax
}
\TMP@EnsureCode{9}{10}% (tab)
\TMP@EnsureCode{10}{12}% ^^J
\TMP@EnsureCode{33}{12}% !
\TMP@EnsureCode{34}{12}% "
\TMP@EnsureCode{36}{3}% $
\TMP@EnsureCode{38}{4}% &
\TMP@EnsureCode{39}{12}% '
\TMP@EnsureCode{40}{12}% (
\TMP@EnsureCode{41}{12}% )
\TMP@EnsureCode{42}{12}% *
\TMP@EnsureCode{43}{12}% +
\TMP@EnsureCode{44}{12}% ,
\TMP@EnsureCode{45}{12}% -
\TMP@EnsureCode{46}{12}% .
\TMP@EnsureCode{47}{12}% /
\TMP@EnsureCode{58}{12}% :
\TMP@EnsureCode{59}{12}% ;
\TMP@EnsureCode{60}{12}% <
\TMP@EnsureCode{62}{12}% >
\TMP@EnsureCode{63}{12}% ?
\TMP@EnsureCode{91}{12}% [
\TMP@EnsureCode{93}{12}% ]
\TMP@EnsureCode{94}{7}% ^ (superscript)
\TMP@EnsureCode{95}{8}% _ (subscript)
\TMP@EnsureCode{96}{12}% `
\TMP@EnsureCode{124}{12}% |
\edef\pagegrid@AtEnd{\pagegrid@AtEnd\noexpand\endinput}
%    \end{macrocode}
%
%    \begin{macrocode}
\RequirePackage{tikz}
\RequirePackage{atbegshi}[2009/12/02]
\RequirePackage{kvoptions}[2009/07/17]
%    \end{macrocode}
%    \begin{macrocode}
\begingroup\expandafter\expandafter\expandafter\endgroup
\expandafter\ifx\csname stockwidth\endcsname\relax
  \def\pagegrid@width{\paperwidth}%
  \def\pagegrid@height{\paperheight}%
\else
  \def\pagegrid@width{\stockwidth}%
  \def\pagegrid@height{\stockheight}%
\fi
%    \end{macrocode}
%
%    \begin{macrocode}
\SetupKeyvalOptions{%
  family=pagegrid,%
  prefix=pagegrid@,%
}
\def\pagegrid@init{%
  \let\pagegrid@origin@a\@empty
  \let\pagegrid@origin@b\@empty
  \let\pagegrid@init\relax
}
\let\pagegrid@@init\pagegrid@init
\def\pagegrid@origin@a{bl}
\def\pagegrid@origin@b{tr}
\def\pagegrid@SetOrigin#1{%
  \pagegrid@init
  \ifx\pagegrid@origin@a\@empty
    \def\pagegrid@origin@a{#1}%
  \else
    \ifx\pagegrid@origin@b\@empty
    \else
      \let\pagegrid@origin@a\pagegrid@origin@b
    \fi
    \def\pagegrid@origin@b{#1}%
  \fi
}
\def\pagegrid@temp#1{%
  \DeclareVoidOption{#1}{\pagegrid@SetOrigin{#1}}%
  \@namedef{pagegrid@N@#1}{#1}%
}
\pagegrid@temp{bl}
\pagegrid@temp{br}
\pagegrid@temp{tl}
\pagegrid@temp{tr}
\def\pagegrid@temp#1#2{%
  \DeclareVoidOption{#2}{\pagegrid@SetOrigin{#1}}%
}%
\pagegrid@temp{bl}{lb}
\pagegrid@temp{br}{rb}
\pagegrid@temp{tl}{lt}
\pagegrid@temp{tr}{rt}
\pagegrid@temp{bl}{bottom-left}
\pagegrid@temp{br}{bottom-right}
\pagegrid@temp{tl}{top-left}
\pagegrid@temp{tr}{top-right}
\pagegrid@temp{bl}{left-bottom}
\pagegrid@temp{br}{right-bottom}
\pagegrid@temp{tl}{left-top}
\pagegrid@temp{tr}{right-top}
%    \end{macrocode}
%    \begin{macrocode}
\DeclareBoolOption[true]{enable}
\DeclareComplementaryOption{disable}{enable}
%    \end{macrocode}
%    \begin{macrocode}
\DeclareBoolOption{foreground}
%    \end{macrocode}
%    \begin{macrocode}
\newlength{\pagegrid@step}
\define@key{pagegrid}{step}{%
  \setlength{\pagegrid@step}{#1}%
}
%    \end{macrocode}
%    \begin{macrocode}
\DeclareStringOption[red]{firstcolor}
\DeclareStringOption[blue]{secondcolor}
%    \end{macrocode}
%    \begin{macrocode}
\DeclareBoolOption[true]{arrows}
\newlength\pagegrid@arrowlength
\pagegrid@arrowlength=\z@
\define@key{pagegrid}{arrowlength}{%
  \setlength{\pagegrid@arrowlength}{#1}%
}
%    \end{macrocode}
%    \begin{macrocode}
\define@key{pagegrid}{double}[true]{%
  \@ifundefined{pagegrid@double@#1}{%
    \PackageWarning{pagegrid}{%
      Unsupported value `#1' for option `double'.\MessageBreak
      Known values are:\MessageBreak
      `false', `first', `last', `true'.\MessageBreak
      Now `false' is used%
    }%
    \chardef\pagegrid@double\z@
  }{%
    \chardef\pagegrid@double\csname pagegrid@double@#1\endcsname\relax
  }%
}
\@namedef{pagegrid@double@false}{0}
\@namedef{pagegrid@double@first}{1}
\@namedef{pagegrid@double@last}{2}
\@namedef{pagegrid@double@true}{2}
\chardef\pagegrid@double\z@
%    \end{macrocode}
%    \begin{macrocode}
\newcommand*{\pagegridsetup}{%
  \let\pagegrid@init\pagegrid@@init
  \setkeys{pagegrid}%
}
%    \end{macrocode}
%    \begin{macrocode}
\pagegridsetup{%
  step=1mm%
}
\InputIfFileExists{pagegrid.cfg}{}%
\ProcessKeyvalOptions*\relax
\AtBeginDocument{%
  \ifdim\pagegrid@arrowlength>\z@
  \else
    \pagegrid@arrowlength=10\pagegrid@step
  \fi
}
%    \end{macrocode}
%
%    \begin{macrocode}
\def\pagegridShipoutDoubleBegin{%
  \begingroup
  \let\newlabel\@gobbletwo
  \let\zref@newlabel\@gobbletwo
  \let\@writefile\@gobbletwo
  \let\select@language\@gobble
}
\def\pagegridShipoutDoubleEnd{%
  \endgroup
}
\def\pagegrid@WriteDouble#1#2{%
  \immediate\write#1{%
    \@backslashchar csname %
    pagegridShipoutDouble#2%
    \@backslashchar endcsname%
  }%
}
\def\pagegrid@ShipoutDouble#1{%
  \begingroup
    \if@filesw
      \pagegrid@WriteDouble\@mainaux{Begin}%
      \ifx\@auxout\@partaux
        \pagegrid@WriteDouble\@partaux{Begin}%
        \def\pagegrid@temp{%
          \pagegrid@WriteDouble\@mainaux{End}%
          \pagegrid@WriteDouble\@partaux{End}%
        }%
      \else
        \def\pagegrid@temp{%
          \pagegrid@WriteDouble\@mainaux{End}%
        }%
      \fi
    \else
      \def\pagegrid@temp{}%
    \fi
    \let\protect\noexpand
    \AtBeginShipoutOriginalShipout\copy#1\relax
    \pagegrid@temp
  \endgroup
}
%    \end{macrocode}
%
%    \begin{macrocode}
\AtBeginShipout{%
  \ifdim\pagegrid@step>\z@
  \else
    \pagegrid@enablefalse
  \fi
  \ifpagegrid@enable
    \ifnum\pagegrid@double=\@ne
      \pagegrid@ShipoutDouble\AtBeginShipoutBox
    \else
      \ifnum\pagegrid@double=\tw@
        \@ifundefined{pagegrid@DoubleBox}{%
          \newbox\pagegrid@DoubleBox
        }{}%
        \setbox\pagegrid@DoubleBox=\copy\AtBeginShipoutBox
      \fi
    \fi
    \ifpagegrid@foreground
      \expandafter\AtBeginShipoutUpperLeftForeground
    \else
      \expandafter\AtBeginShipoutUpperLeft
    \fi
    {%
      \put(0,0){%
        \makebox(0,0)[lt]{%
          \begin{tikzpicture}[%
            bl/.style={},%
            br/.style={xshift=\pagegrid@width,xscale=-1},%
            tl/.style={yshift=\pagegrid@height,yscale=-1},%
            tr/.style={xshift=\pagegrid@width,%
                       yshift=\pagegrid@height,scale=-1}%
          ]%
            \useasboundingbox
              (0mm,\pagegrid@height) rectangle (0mm,\pagegrid@height);%
            \draw[%
              \pagegrid@origin@a,%
              step=\pagegrid@step,%
              style=help lines,%
              ultra thin%
            ] (0mm,0mm) grid (\pagegrid@width,\pagegrid@height);%
            \ifx\pagegrid@origin@b\@empty
            \else
              \draw[%
                \pagegrid@origin@b,%
                step=10\pagegrid@step,%
                {\pagegrid@secondcolor},%
                very thin%
              ] (0mm,0mm) grid (\pagegrid@width,\pagegrid@height);%
            \fi
            \draw[%
               \pagegrid@origin@a,%
               step=10\pagegrid@step,%
               {\pagegrid@firstcolor},%
               very thin%
            ] (0mm,0mm) grid (\pagegrid@width,\pagegrid@height);%
            \ifx\pagegrid@origin@b\@empty
            \else
              \draw[%
                \pagegrid@origin@b,%
                step=50\pagegrid@step,%
                {\pagegrid@secondcolor},%
                thick%
              ] (0mm,0mm) grid (\pagegrid@width,\pagegrid@height);%
            \fi
            \draw[%
              \pagegrid@origin@a,%
              step=50\pagegrid@step,%
              {\pagegrid@firstcolor},%
              thick%
            ] (0mm,0mm) grid (\pagegrid@width,\pagegrid@height);%
            \ifpagegrid@arrows
              \ifx\pagegrid@origin@b\@empty
              \else
                \draw[%
                  \pagegrid@origin@b,%
                  {\pagegrid@secondcolor},%
                  stroke,%
                  line width=1pt,%
                  line cap=round%
                ] (0mm,0mm) %
                -- (\pagegrid@arrowlength,\pagegrid@arrowlength) %
                   (\pagegrid@arrowlength,.5\pagegrid@arrowlength) %
                -- (\pagegrid@arrowlength,\pagegrid@arrowlength) %
                -- (.5\pagegrid@arrowlength,\pagegrid@arrowlength);%
              \fi
              \draw[%
                \pagegrid@origin@a,%
                {\pagegrid@firstcolor},%
                stroke,%
                line width=1pt,%
                line cap=round%
              ] (0mm,0mm) %
              -- (\pagegrid@arrowlength,\pagegrid@arrowlength) %
                 (\pagegrid@arrowlength,.5\pagegrid@arrowlength) %
              -- (\pagegrid@arrowlength,\pagegrid@arrowlength) %
              -- (.5\pagegrid@arrowlength,\pagegrid@arrowlength);%
            \fi
          \end{tikzpicture}%
        }%
      }%
    }%
    \ifnum\pagegrid@double=\tw@
      \pagegrid@ShipoutDouble\pagegrid@DoubleBox
    \fi
  \fi
}
%    \end{macrocode}
%
%    \begin{macrocode}
\pagegrid@AtEnd%
%</package>
%    \end{macrocode}
%
% \section{Test}
%
% \subsection{Catcode checks for loading}
%
%    \begin{macrocode}
%<*test1>
%    \end{macrocode}
%    \begin{macrocode}
\catcode`\{=1 %
\catcode`\}=2 %
\catcode`\#=6 %
\catcode`\@=11 %
\expandafter\ifx\csname count@\endcsname\relax
  \countdef\count@=255 %
\fi
\expandafter\ifx\csname @gobble\endcsname\relax
  \long\def\@gobble#1{}%
\fi
\expandafter\ifx\csname @firstofone\endcsname\relax
  \long\def\@firstofone#1{#1}%
\fi
\expandafter\ifx\csname loop\endcsname\relax
  \expandafter\@firstofone
\else
  \expandafter\@gobble
\fi
{%
  \def\loop#1\repeat{%
    \def\body{#1}%
    \iterate
  }%
  \def\iterate{%
    \body
      \let\next\iterate
    \else
      \let\next\relax
    \fi
    \next
  }%
  \let\repeat=\fi
}%
\def\RestoreCatcodes{}
\count@=0 %
\loop
  \edef\RestoreCatcodes{%
    \RestoreCatcodes
    \catcode\the\count@=\the\catcode\count@\relax
  }%
\ifnum\count@<255 %
  \advance\count@ 1 %
\repeat

\def\RangeCatcodeInvalid#1#2{%
  \count@=#1\relax
  \loop
    \catcode\count@=15 %
  \ifnum\count@<#2\relax
    \advance\count@ 1 %
  \repeat
}
\def\RangeCatcodeCheck#1#2#3{%
  \count@=#1\relax
  \loop
    \ifnum#3=\catcode\count@
    \else
      \errmessage{%
        Character \the\count@\space
        with wrong catcode \the\catcode\count@\space
        instead of \number#3%
      }%
    \fi
  \ifnum\count@<#2\relax
    \advance\count@ 1 %
  \repeat
}
\def\space{ }
\expandafter\ifx\csname LoadCommand\endcsname\relax
  \def\LoadCommand{\input pagegrid.sty\relax}%
\fi
\def\Test{%
  \RangeCatcodeInvalid{0}{47}%
  \RangeCatcodeInvalid{58}{64}%
  \RangeCatcodeInvalid{91}{96}%
  \RangeCatcodeInvalid{123}{255}%
  \catcode`\@=12 %
  \catcode`\\=0 %
  \catcode`\%=14 %
  \LoadCommand
  \RangeCatcodeCheck{0}{36}{15}%
  \RangeCatcodeCheck{37}{37}{14}%
  \RangeCatcodeCheck{38}{47}{15}%
  \RangeCatcodeCheck{48}{57}{12}%
  \RangeCatcodeCheck{58}{63}{15}%
  \RangeCatcodeCheck{64}{64}{12}%
  \RangeCatcodeCheck{65}{90}{11}%
  \RangeCatcodeCheck{91}{91}{15}%
  \RangeCatcodeCheck{92}{92}{0}%
  \RangeCatcodeCheck{93}{96}{15}%
  \RangeCatcodeCheck{97}{122}{11}%
  \RangeCatcodeCheck{123}{255}{15}%
  \RestoreCatcodes
}
\Test
\csname @@end\endcsname
\end
%    \end{macrocode}
%    \begin{macrocode}
%</test1>
%    \end{macrocode}
%
% \section{Installation}
%
% \subsection{Download}
%
% \paragraph{Package.} This package is available on
% CTAN\footnote{\CTANpkg{pagegrid}}:
% \begin{description}
% \item[\CTAN{macros/latex/contrib/oberdiek/pagegrid.dtx}] The source file.
% \item[\CTAN{macros/latex/contrib/oberdiek/pagegrid.pdf}] Documentation.
% \end{description}
%
%
% \paragraph{Bundle.} All the packages of the bundle `oberdiek'
% are also available in a TDS compliant ZIP archive. There
% the packages are already unpacked and the documentation files
% are generated. The files and directories obey the TDS standard.
% \begin{description}
% \item[\CTANinstall{install/macros/latex/contrib/oberdiek.tds.zip}]
% \end{description}
% \emph{TDS} refers to the standard ``A Directory Structure
% for \TeX\ Files'' (\CTAN{tds/tds.pdf}). Directories
% with \xfile{texmf} in their name are usually organized this way.
%
% \subsection{Bundle installation}
%
% \paragraph{Unpacking.} Unpack the \xfile{oberdiek.tds.zip} in the
% TDS tree (also known as \xfile{texmf} tree) of your choice.
% Example (linux):
% \begin{quote}
%   |unzip oberdiek.tds.zip -d ~/texmf|
% \end{quote}
%
% \paragraph{Script installation.}
% Check the directory \xfile{TDS:scripts/oberdiek/} for
% scripts that need further installation steps.
%
% \subsection{Package installation}
%
% \paragraph{Unpacking.} The \xfile{.dtx} file is a self-extracting
% \docstrip\ archive. The files are extracted by running the
% \xfile{.dtx} through \plainTeX:
% \begin{quote}
%   \verb|tex pagegrid.dtx|
% \end{quote}
%
% \paragraph{TDS.} Now the different files must be moved into
% the different directories in your installation TDS tree
% (also known as \xfile{texmf} tree):
% \begin{quote}
% \def\t{^^A
% \begin{tabular}{@{}>{\ttfamily}l@{ $\rightarrow$ }>{\ttfamily}l@{}}
%   pagegrid.sty & tex/latex/oberdiek/pagegrid.sty\\
%   pagegrid.pdf & doc/latex/oberdiek/pagegrid.pdf\\
%   test/pagegrid-test1.tex & doc/latex/oberdiek/test/pagegrid-test1.tex\\
%   pagegrid.dtx & source/latex/oberdiek/pagegrid.dtx\\
% \end{tabular}^^A
% }^^A
% \sbox0{\t}^^A
% \ifdim\wd0>\linewidth
%   \begingroup
%     \advance\linewidth by\leftmargin
%     \advance\linewidth by\rightmargin
%   \edef\x{\endgroup
%     \def\noexpand\lw{\the\linewidth}^^A
%   }\x
%   \def\lwbox{^^A
%     \leavevmode
%     \hbox to \linewidth{^^A
%       \kern-\leftmargin\relax
%       \hss
%       \usebox0
%       \hss
%       \kern-\rightmargin\relax
%     }^^A
%   }^^A
%   \ifdim\wd0>\lw
%     \sbox0{\small\t}^^A
%     \ifdim\wd0>\linewidth
%       \ifdim\wd0>\lw
%         \sbox0{\footnotesize\t}^^A
%         \ifdim\wd0>\linewidth
%           \ifdim\wd0>\lw
%             \sbox0{\scriptsize\t}^^A
%             \ifdim\wd0>\linewidth
%               \ifdim\wd0>\lw
%                 \sbox0{\tiny\t}^^A
%                 \ifdim\wd0>\linewidth
%                   \lwbox
%                 \else
%                   \usebox0
%                 \fi
%               \else
%                 \lwbox
%               \fi
%             \else
%               \usebox0
%             \fi
%           \else
%             \lwbox
%           \fi
%         \else
%           \usebox0
%         \fi
%       \else
%         \lwbox
%       \fi
%     \else
%       \usebox0
%     \fi
%   \else
%     \lwbox
%   \fi
% \else
%   \usebox0
% \fi
% \end{quote}
% If you have a \xfile{docstrip.cfg} that configures and enables \docstrip's
% TDS installing feature, then some files can already be in the right
% place, see the documentation of \docstrip.
%
% \subsection{Refresh file name databases}
%
% If your \TeX~distribution
% (\TeX\,Live, \mikTeX, \dots) relies on file name databases, you must refresh
% these. For example, \TeX\,Live\ users run \verb|texhash| or
% \verb|mktexlsr|.
%
% \subsection{Some details for the interested}
%
% \paragraph{Unpacking with \LaTeX.}
% The \xfile{.dtx} chooses its action depending on the format:
% \begin{description}
% \item[\plainTeX:] Run \docstrip\ and extract the files.
% \item[\LaTeX:] Generate the documentation.
% \end{description}
% If you insist on using \LaTeX\ for \docstrip\ (really,
% \docstrip\ does not need \LaTeX), then inform the autodetect routine
% about your intention:
% \begin{quote}
%   \verb|latex \let\install=y% \iffalse meta-comment
%
% File: pagegrid.dtx
% Version: 2016/05/16 v1.5
% Info: Print page grid in background
%
% Copyright (C) 2009 by
%    Heiko Oberdiek <heiko.oberdiek at googlemail.com>
%    2016
%    https://github.com/ho-tex/oberdiek/issues
%
% This work may be distributed and/or modified under the
% conditions of the LaTeX Project Public License, either
% version 1.3c of this license or (at your option) any later
% version. This version of this license is in
%    https://www.latex-project.org/lppl/lppl-1-3c.txt
% and the latest version of this license is in
%    https://www.latex-project.org/lppl.txt
% and version 1.3 or later is part of all distributions of
% LaTeX version 2005/12/01 or later.
%
% This work has the LPPL maintenance status "maintained".
%
% The Current Maintainers of this work are
% Heiko Oberdiek and the Oberdiek Package Support Group
% https://github.com/ho-tex/oberdiek/issues
%
% This work consists of the main source file pagegrid.dtx
% and the derived files
%    pagegrid.sty, pagegrid.pdf, pagegrid.ins, pagegrid.drv,
%    pagegrid-test1.tex.
%
% Distribution:
%    CTAN:macros/latex/contrib/oberdiek/pagegrid.dtx
%    CTAN:macros/latex/contrib/oberdiek/pagegrid.pdf
%
% Unpacking:
%    (a) If pagegrid.ins is present:
%           tex pagegrid.ins
%    (b) Without pagegrid.ins:
%           tex pagegrid.dtx
%    (c) If you insist on using LaTeX
%           latex \let\install=y\input{pagegrid.dtx}
%        (quote the arguments according to the demands of your shell)
%
% Documentation:
%    (a) If pagegrid.drv is present:
%           latex pagegrid.drv
%    (b) Without pagegrid.drv:
%           latex pagegrid.dtx; ...
%    The class ltxdoc loads the configuration file ltxdoc.cfg
%    if available. Here you can specify further options, e.g.
%    use A4 as paper format:
%       \PassOptionsToClass{a4paper}{article}
%
%    Programm calls to get the documentation (example):
%       pdflatex pagegrid.dtx
%       makeindex -s gind.ist pagegrid.idx
%       pdflatex pagegrid.dtx
%       makeindex -s gind.ist pagegrid.idx
%       pdflatex pagegrid.dtx
%
% Installation:
%    TDS:tex/latex/oberdiek/pagegrid.sty
%    TDS:doc/latex/oberdiek/pagegrid.pdf
%    TDS:doc/latex/oberdiek/test/pagegrid-test1.tex
%    TDS:source/latex/oberdiek/pagegrid.dtx
%
%<*ignore>
\begingroup
  \catcode123=1 %
  \catcode125=2 %
  \def\x{LaTeX2e}%
\expandafter\endgroup
\ifcase 0\ifx\install y1\fi\expandafter
         \ifx\csname processbatchFile\endcsname\relax\else1\fi
         \ifx\fmtname\x\else 1\fi\relax
\else\csname fi\endcsname
%</ignore>
%<*install>
\input docstrip.tex
\Msg{************************************************************************}
\Msg{* Installation}
\Msg{* Package: pagegrid 2016/05/16 v1.5 Print page grid in background (HO)}
\Msg{************************************************************************}

\keepsilent
\askforoverwritefalse

\let\MetaPrefix\relax
\preamble

This is a generated file.

Project: pagegrid
Version: 2016/05/16 v1.5

Copyright (C) 2009 by
   Heiko Oberdiek <heiko.oberdiek at googlemail.com>

This work may be distributed and/or modified under the
conditions of the LaTeX Project Public License, either
version 1.3c of this license or (at your option) any later
version. This version of this license is in
   https://www.latex-project.org/lppl/lppl-1-3c.txt
and the latest version of this license is in
   https://www.latex-project.org/lppl.txt
and version 1.3 or later is part of all distributions of
LaTeX version 2005/12/01 or later.

This work has the LPPL maintenance status "maintained".

The Current Maintainers of this work are
Heiko Oberdiek and the Oberdiek Package Support Group
https://github.com/ho-tex/oberdiek/issues


This work consists of the main source file pagegrid.dtx
and the derived files
   pagegrid.sty, pagegrid.pdf, pagegrid.ins, pagegrid.drv,
   pagegrid-test1.tex.

\endpreamble
\let\MetaPrefix\DoubleperCent

\generate{%
  \file{pagegrid.ins}{\from{pagegrid.dtx}{install}}%
  \file{pagegrid.drv}{\from{pagegrid.dtx}{driver}}%
  \usedir{tex/latex/oberdiek}%
  \file{pagegrid.sty}{\from{pagegrid.dtx}{package}}%
%  \usedir{doc/latex/oberdiek/test}%
%  \file{pagegrid-test1.tex}{\from{pagegrid.dtx}{test1}}%
  \nopreamble
  \nopostamble
%  \usedir{source/latex/oberdiek/catalogue}%
%  \file{pagegrid.xml}{\from{pagegrid.dtx}{catalogue}}%
}

\catcode32=13\relax% active space
\let =\space%
\Msg{************************************************************************}
\Msg{*}
\Msg{* To finish the installation you have to move the following}
\Msg{* file into a directory searched by TeX:}
\Msg{*}
\Msg{*     pagegrid.sty}
\Msg{*}
\Msg{* To produce the documentation run the file `pagegrid.drv'}
\Msg{* through LaTeX.}
\Msg{*}
\Msg{* Happy TeXing!}
\Msg{*}
\Msg{************************************************************************}

\endbatchfile
%</install>
%<*ignore>
\fi
%</ignore>
%<*driver>
\NeedsTeXFormat{LaTeX2e}
\ProvidesFile{pagegrid.drv}%
  [2016/05/16 v1.5 Print page grid in background (HO)]%
\documentclass{ltxdoc}
\usepackage{holtxdoc}[2011/11/22]
\begin{document}
  \DocInput{pagegrid.dtx}%
\end{document}
%</driver>
% \fi
%
%
% \CharacterTable
%  {Upper-case    \A\B\C\D\E\F\G\H\I\J\K\L\M\N\O\P\Q\R\S\T\U\V\W\X\Y\Z
%   Lower-case    \a\b\c\d\e\f\g\h\i\j\k\l\m\n\o\p\q\r\s\t\u\v\w\x\y\z
%   Digits        \0\1\2\3\4\5\6\7\8\9
%   Exclamation   \!     Double quote  \"     Hash (number) \#
%   Dollar        \$     Percent       \%     Ampersand     \&
%   Acute accent  \'     Left paren    \(     Right paren   \)
%   Asterisk      \*     Plus          \+     Comma         \,
%   Minus         \-     Point         \.     Solidus       \/
%   Colon         \:     Semicolon     \;     Less than     \<
%   Equals        \=     Greater than  \>     Question mark \?
%   Commercial at \@     Left bracket  \[     Backslash     \\
%   Right bracket \]     Circumflex    \^     Underscore    \_
%   Grave accent  \`     Left brace    \{     Vertical bar  \|
%   Right brace   \}     Tilde         \~}
%
% \GetFileInfo{pagegrid.drv}
%
% \title{The \xpackage{pagegrid} package}
% \date{2016/05/16 v1.5}
% \author{Heiko Oberdiek\thanks
% {Please report any issues at \url{https://github.com/ho-tex/oberdiek/issues}}}
%
% \maketitle
%
% \begin{abstract}
% The \LaTeX\ package prints a page grid in the background.
% \end{abstract}
%
% \tableofcontents
%
% \section{Documentation}
%
% The package puts a grid on the paper. It was written for
% developers of a class or package
% who have to put elements on definite locations on a page
% (e.g. letter class). The grid allows a faster optical check,
% whether the positions are correct. If the previewer already
% offers features for measuring, the package might be obsolete.
% Otherwise it saves the developer from printing the page and
% measuring by hand.
%
% \subsection{Options}
%
% Options are evaluated in the following order:
% \begin{enumerate}
% \item
%  Configuration file \xfile{pagegrid.cfg} using \cs{pagegridsetup}
%  if the file exists.
%  \item
%  Package options given for \cs{usepackage}.
%  \item
%  Later calls of \cs{pagegridsetup}.
% \end{enumerate}
% \begin{declcs}{pagegridsetup}\M{option list}
% \end{declcs}
% The options are key value options. Boolean options are enabled by
% default (without value) or by using the explicit value \texttt{true}.
% Value \texttt{false} disable the option.
%
% \subsubsection{Options \xoption{enable}, \xoption{disable}}
%
% \begin{description}
% \item[\xoption{enable}:] This boolean option controls whether the page grid
%   is drawn. As default the page grid drawing is activated.
% \item[\xoption{disable}:] It is the opposite
%   of option \xoption{enable}. It was added for convenience and
%   allows the abbreviation \texttt{disable} for \texttt{enable=false}.
% \end{description}
%
% \subsubsection{Grid origins}
%
% The package supports up to two grids on a page allowing
% measurement from opposite directions. As default two grids are drawn,
% the first from bottom left to top right. The origin of the second
% grid is at the opposite top right corner.
% The origins are controlled by the following options.
% The number of grids (one or two) depend on the number of these options
% in one call of \cs{pagegridsetup}.
% The following frame shows a paper and in its corners are the
% corresponding options. At the left and right side alias names
% are given for the options inside the paper.
% \begin{quote}
% \begin{tabular}{@{}r|@{\,}l@{\qquad}r@{\,}|l@{}}
% \cline{2-3}
% \xoption{left-top}, \xoption{lt}, \xoption{top-left}
% & \vphantom{\"U}\xoption{tl} & \xoption{tr}
% & \xoption{top-right}, \xoption{rt}, \xoption{right-top}\\
% &&&\\
% \xoption{left-bottom}, \xoption{lb}, \xoption{bottom-left}
% & \xoption{bl} & \xoption{br}
% & \xoption{bottom-right}, \xoption{rb}, \xoption{right-bottom}\\
% \cline{2-3}
% \end{tabular}
% \end{quote}
% Examples:
% \begin{quote}
% |\pagegridsetup{bl,tr}|
% \end{quote}
% This is the default setting with two grids as described previously.
% The following setups one grid only. Its origin is the upper left
% corner:
% \begin{quote}
% |\pagegridsetup{top-left}|
% \end{quote}
%
% \subsubsection{Grid unit}
%
% \begin{description}
% \item[\xoption{step}] This option takes a length and
% setups the unit for the grid. The page width and page height
% should be multiples of this unit.
% Currently the default is \texttt{1mm}. But this might change
% later by a heuristic based on the paper size.
% \end{description}
%
% \subsubsection{Color options}
%
% The basic grid lines are drawn as ultra thin help lines and is only
% drawn for the first grid.
% Each tenth and fiftyth line of the basic net is drawn thicker in a special
% color for the two grids.
% \begin{description}
% \item[\xoption{firstcolor}:] Color for the thicker lines and the arrows
% of the first grid. Default value is \texttt{red}.
% \item[\xoption{secondcolor}:] Color for the thicker lines and the arrows
% of the second grid. Default value is \texttt{blue}.
% \end{description}
% Use a color specification that package \xpackage{tikz} understands.
% (The grid is drawn with \xpackage{pgf}/\xpackage{tikz}.)
%
% \subsubsection{Arrow options}
%
% Arrows are put at the origin at the grid to show the grid start
% and the direction of the grid.
% \begin{description}
% \item[\xoption{arrows}:] This boolean option turns the arrows on or off.
% As default arrows are enabled.
% \item[\xoption{arrowlength}:] The length given as value is the
% length of the edge of a square at the origin within the
% arrow is put as diagonal. Default is 10 times the grid unit (10\,mm).
% The real arrow length is this length multiplied by $\sqrt2$.
% \end{description}
%
% \subsubsection{Miscellaneous options}
%
% \begin{description}
% \item[\xoption{double}:] The output page is doubled, one without page
% grid and the other with page grid. Possible values are shown in the
% following table:
% \begin{quote}
% \begin{tabular}{ll}
% Option & Meaning\\
% \hline
% |false| & Turns option off.\\
% |first| & Grid page comes first.\\
% |last| & Grid page comes after the page without grid.\\
% |true| & Same as |last|.\\
% \meta{no value} & Same as |true|.\\
% \end{tabular}
% \end{quote}
% \textbf{Note:}
% The double output of the page has side effects.
% All whatits are executed twice, for example: file writing
% and anchor setting. Some unwanted actions are catched such
% as multiple \cs{label} definitions, duplicate entries in
% the table of contents. For bookmarks, use package \xpackage{bookmarks}.
% \item[\xoption{foreground}:] Boolean option, default is \texttt{false}.
% Sometimes there might be elements on the page (e.g. large images)
% that hide the grid. Then option \xoption{foreground} puts the grids
% over the current output page.
% \end{description}
%
% \StopEventually{
% }
%
% \section{Implementation}
%    \begin{macrocode}
%<*package>
%    \end{macrocode}
%    Reload check, especially if the package is not used with \LaTeX.
%    \begin{macrocode}
\begingroup\catcode61\catcode48\catcode32=10\relax%
  \catcode13=5 % ^^M
  \endlinechar=13 %
  \catcode35=6 % #
  \catcode39=12 % '
  \catcode44=12 % ,
  \catcode45=12 % -
  \catcode46=12 % .
  \catcode58=12 % :
  \catcode64=11 % @
  \catcode123=1 % {
  \catcode125=2 % }
  \expandafter\let\expandafter\x\csname ver@pagegrid.sty\endcsname
  \ifx\x\relax % plain-TeX, first loading
  \else
    \def\empty{}%
    \ifx\x\empty % LaTeX, first loading,
      % variable is initialized, but \ProvidesPackage not yet seen
    \else
      \expandafter\ifx\csname PackageInfo\endcsname\relax
        \def\x#1#2{%
          \immediate\write-1{Package #1 Info: #2.}%
        }%
      \else
        \def\x#1#2{\PackageInfo{#1}{#2, stopped}}%
      \fi
      \x{pagegrid}{The package is already loaded}%
      \aftergroup\endinput
    \fi
  \fi
\endgroup%
%    \end{macrocode}
%    Package identification:
%    \begin{macrocode}
\begingroup\catcode61\catcode48\catcode32=10\relax%
  \catcode13=5 % ^^M
  \endlinechar=13 %
  \catcode35=6 % #
  \catcode39=12 % '
  \catcode40=12 % (
  \catcode41=12 % )
  \catcode44=12 % ,
  \catcode45=12 % -
  \catcode46=12 % .
  \catcode47=12 % /
  \catcode58=12 % :
  \catcode64=11 % @
  \catcode91=12 % [
  \catcode93=12 % ]
  \catcode123=1 % {
  \catcode125=2 % }
  \expandafter\ifx\csname ProvidesPackage\endcsname\relax
    \def\x#1#2#3[#4]{\endgroup
      \immediate\write-1{Package: #3 #4}%
      \xdef#1{#4}%
    }%
  \else
    \def\x#1#2[#3]{\endgroup
      #2[{#3}]%
      \ifx#1\@undefined
        \xdef#1{#3}%
      \fi
      \ifx#1\relax
        \xdef#1{#3}%
      \fi
    }%
  \fi
\expandafter\x\csname ver@pagegrid.sty\endcsname
\ProvidesPackage{pagegrid}%
  [2016/05/16 v1.5 Print page grid in background (HO)]%
%    \end{macrocode}
%
%    \begin{macrocode}
\begingroup\catcode61\catcode48\catcode32=10\relax%
  \catcode13=5 % ^^M
  \endlinechar=13 %
  \catcode123=1 % {
  \catcode125=2 % }
  \catcode64=11 % @
  \def\x{\endgroup
    \expandafter\edef\csname pagegrid@AtEnd\endcsname{%
      \endlinechar=\the\endlinechar\relax
      \catcode13=\the\catcode13\relax
      \catcode32=\the\catcode32\relax
      \catcode35=\the\catcode35\relax
      \catcode61=\the\catcode61\relax
      \catcode64=\the\catcode64\relax
      \catcode123=\the\catcode123\relax
      \catcode125=\the\catcode125\relax
    }%
  }%
\x\catcode61\catcode48\catcode32=10\relax%
\catcode13=5 % ^^M
\endlinechar=13 %
\catcode35=6 % #
\catcode64=11 % @
\catcode123=1 % {
\catcode125=2 % }
\def\TMP@EnsureCode#1#2{%
  \edef\pagegrid@AtEnd{%
    \pagegrid@AtEnd
    \catcode#1=\the\catcode#1\relax
  }%
  \catcode#1=#2\relax
}
\TMP@EnsureCode{9}{10}% (tab)
\TMP@EnsureCode{10}{12}% ^^J
\TMP@EnsureCode{33}{12}% !
\TMP@EnsureCode{34}{12}% "
\TMP@EnsureCode{36}{3}% $
\TMP@EnsureCode{38}{4}% &
\TMP@EnsureCode{39}{12}% '
\TMP@EnsureCode{40}{12}% (
\TMP@EnsureCode{41}{12}% )
\TMP@EnsureCode{42}{12}% *
\TMP@EnsureCode{43}{12}% +
\TMP@EnsureCode{44}{12}% ,
\TMP@EnsureCode{45}{12}% -
\TMP@EnsureCode{46}{12}% .
\TMP@EnsureCode{47}{12}% /
\TMP@EnsureCode{58}{12}% :
\TMP@EnsureCode{59}{12}% ;
\TMP@EnsureCode{60}{12}% <
\TMP@EnsureCode{62}{12}% >
\TMP@EnsureCode{63}{12}% ?
\TMP@EnsureCode{91}{12}% [
\TMP@EnsureCode{93}{12}% ]
\TMP@EnsureCode{94}{7}% ^ (superscript)
\TMP@EnsureCode{95}{8}% _ (subscript)
\TMP@EnsureCode{96}{12}% `
\TMP@EnsureCode{124}{12}% |
\edef\pagegrid@AtEnd{\pagegrid@AtEnd\noexpand\endinput}
%    \end{macrocode}
%
%    \begin{macrocode}
\RequirePackage{tikz}
\RequirePackage{atbegshi}[2009/12/02]
\RequirePackage{kvoptions}[2009/07/17]
%    \end{macrocode}
%    \begin{macrocode}
\begingroup\expandafter\expandafter\expandafter\endgroup
\expandafter\ifx\csname stockwidth\endcsname\relax
  \def\pagegrid@width{\paperwidth}%
  \def\pagegrid@height{\paperheight}%
\else
  \def\pagegrid@width{\stockwidth}%
  \def\pagegrid@height{\stockheight}%
\fi
%    \end{macrocode}
%
%    \begin{macrocode}
\SetupKeyvalOptions{%
  family=pagegrid,%
  prefix=pagegrid@,%
}
\def\pagegrid@init{%
  \let\pagegrid@origin@a\@empty
  \let\pagegrid@origin@b\@empty
  \let\pagegrid@init\relax
}
\let\pagegrid@@init\pagegrid@init
\def\pagegrid@origin@a{bl}
\def\pagegrid@origin@b{tr}
\def\pagegrid@SetOrigin#1{%
  \pagegrid@init
  \ifx\pagegrid@origin@a\@empty
    \def\pagegrid@origin@a{#1}%
  \else
    \ifx\pagegrid@origin@b\@empty
    \else
      \let\pagegrid@origin@a\pagegrid@origin@b
    \fi
    \def\pagegrid@origin@b{#1}%
  \fi
}
\def\pagegrid@temp#1{%
  \DeclareVoidOption{#1}{\pagegrid@SetOrigin{#1}}%
  \@namedef{pagegrid@N@#1}{#1}%
}
\pagegrid@temp{bl}
\pagegrid@temp{br}
\pagegrid@temp{tl}
\pagegrid@temp{tr}
\def\pagegrid@temp#1#2{%
  \DeclareVoidOption{#2}{\pagegrid@SetOrigin{#1}}%
}%
\pagegrid@temp{bl}{lb}
\pagegrid@temp{br}{rb}
\pagegrid@temp{tl}{lt}
\pagegrid@temp{tr}{rt}
\pagegrid@temp{bl}{bottom-left}
\pagegrid@temp{br}{bottom-right}
\pagegrid@temp{tl}{top-left}
\pagegrid@temp{tr}{top-right}
\pagegrid@temp{bl}{left-bottom}
\pagegrid@temp{br}{right-bottom}
\pagegrid@temp{tl}{left-top}
\pagegrid@temp{tr}{right-top}
%    \end{macrocode}
%    \begin{macrocode}
\DeclareBoolOption[true]{enable}
\DeclareComplementaryOption{disable}{enable}
%    \end{macrocode}
%    \begin{macrocode}
\DeclareBoolOption{foreground}
%    \end{macrocode}
%    \begin{macrocode}
\newlength{\pagegrid@step}
\define@key{pagegrid}{step}{%
  \setlength{\pagegrid@step}{#1}%
}
%    \end{macrocode}
%    \begin{macrocode}
\DeclareStringOption[red]{firstcolor}
\DeclareStringOption[blue]{secondcolor}
%    \end{macrocode}
%    \begin{macrocode}
\DeclareBoolOption[true]{arrows}
\newlength\pagegrid@arrowlength
\pagegrid@arrowlength=\z@
\define@key{pagegrid}{arrowlength}{%
  \setlength{\pagegrid@arrowlength}{#1}%
}
%    \end{macrocode}
%    \begin{macrocode}
\define@key{pagegrid}{double}[true]{%
  \@ifundefined{pagegrid@double@#1}{%
    \PackageWarning{pagegrid}{%
      Unsupported value `#1' for option `double'.\MessageBreak
      Known values are:\MessageBreak
      `false', `first', `last', `true'.\MessageBreak
      Now `false' is used%
    }%
    \chardef\pagegrid@double\z@
  }{%
    \chardef\pagegrid@double\csname pagegrid@double@#1\endcsname\relax
  }%
}
\@namedef{pagegrid@double@false}{0}
\@namedef{pagegrid@double@first}{1}
\@namedef{pagegrid@double@last}{2}
\@namedef{pagegrid@double@true}{2}
\chardef\pagegrid@double\z@
%    \end{macrocode}
%    \begin{macrocode}
\newcommand*{\pagegridsetup}{%
  \let\pagegrid@init\pagegrid@@init
  \setkeys{pagegrid}%
}
%    \end{macrocode}
%    \begin{macrocode}
\pagegridsetup{%
  step=1mm%
}
\InputIfFileExists{pagegrid.cfg}{}%
\ProcessKeyvalOptions*\relax
\AtBeginDocument{%
  \ifdim\pagegrid@arrowlength>\z@
  \else
    \pagegrid@arrowlength=10\pagegrid@step
  \fi
}
%    \end{macrocode}
%
%    \begin{macrocode}
\def\pagegridShipoutDoubleBegin{%
  \begingroup
  \let\newlabel\@gobbletwo
  \let\zref@newlabel\@gobbletwo
  \let\@writefile\@gobbletwo
  \let\select@language\@gobble
}
\def\pagegridShipoutDoubleEnd{%
  \endgroup
}
\def\pagegrid@WriteDouble#1#2{%
  \immediate\write#1{%
    \@backslashchar csname %
    pagegridShipoutDouble#2%
    \@backslashchar endcsname%
  }%
}
\def\pagegrid@ShipoutDouble#1{%
  \begingroup
    \if@filesw
      \pagegrid@WriteDouble\@mainaux{Begin}%
      \ifx\@auxout\@partaux
        \pagegrid@WriteDouble\@partaux{Begin}%
        \def\pagegrid@temp{%
          \pagegrid@WriteDouble\@mainaux{End}%
          \pagegrid@WriteDouble\@partaux{End}%
        }%
      \else
        \def\pagegrid@temp{%
          \pagegrid@WriteDouble\@mainaux{End}%
        }%
      \fi
    \else
      \def\pagegrid@temp{}%
    \fi
    \let\protect\noexpand
    \AtBeginShipoutOriginalShipout\copy#1\relax
    \pagegrid@temp
  \endgroup
}
%    \end{macrocode}
%
%    \begin{macrocode}
\AtBeginShipout{%
  \ifdim\pagegrid@step>\z@
  \else
    \pagegrid@enablefalse
  \fi
  \ifpagegrid@enable
    \ifnum\pagegrid@double=\@ne
      \pagegrid@ShipoutDouble\AtBeginShipoutBox
    \else
      \ifnum\pagegrid@double=\tw@
        \@ifundefined{pagegrid@DoubleBox}{%
          \newbox\pagegrid@DoubleBox
        }{}%
        \setbox\pagegrid@DoubleBox=\copy\AtBeginShipoutBox
      \fi
    \fi
    \ifpagegrid@foreground
      \expandafter\AtBeginShipoutUpperLeftForeground
    \else
      \expandafter\AtBeginShipoutUpperLeft
    \fi
    {%
      \put(0,0){%
        \makebox(0,0)[lt]{%
          \begin{tikzpicture}[%
            bl/.style={},%
            br/.style={xshift=\pagegrid@width,xscale=-1},%
            tl/.style={yshift=\pagegrid@height,yscale=-1},%
            tr/.style={xshift=\pagegrid@width,%
                       yshift=\pagegrid@height,scale=-1}%
          ]%
            \useasboundingbox
              (0mm,\pagegrid@height) rectangle (0mm,\pagegrid@height);%
            \draw[%
              \pagegrid@origin@a,%
              step=\pagegrid@step,%
              style=help lines,%
              ultra thin%
            ] (0mm,0mm) grid (\pagegrid@width,\pagegrid@height);%
            \ifx\pagegrid@origin@b\@empty
            \else
              \draw[%
                \pagegrid@origin@b,%
                step=10\pagegrid@step,%
                {\pagegrid@secondcolor},%
                very thin%
              ] (0mm,0mm) grid (\pagegrid@width,\pagegrid@height);%
            \fi
            \draw[%
               \pagegrid@origin@a,%
               step=10\pagegrid@step,%
               {\pagegrid@firstcolor},%
               very thin%
            ] (0mm,0mm) grid (\pagegrid@width,\pagegrid@height);%
            \ifx\pagegrid@origin@b\@empty
            \else
              \draw[%
                \pagegrid@origin@b,%
                step=50\pagegrid@step,%
                {\pagegrid@secondcolor},%
                thick%
              ] (0mm,0mm) grid (\pagegrid@width,\pagegrid@height);%
            \fi
            \draw[%
              \pagegrid@origin@a,%
              step=50\pagegrid@step,%
              {\pagegrid@firstcolor},%
              thick%
            ] (0mm,0mm) grid (\pagegrid@width,\pagegrid@height);%
            \ifpagegrid@arrows
              \ifx\pagegrid@origin@b\@empty
              \else
                \draw[%
                  \pagegrid@origin@b,%
                  {\pagegrid@secondcolor},%
                  stroke,%
                  line width=1pt,%
                  line cap=round%
                ] (0mm,0mm) %
                -- (\pagegrid@arrowlength,\pagegrid@arrowlength) %
                   (\pagegrid@arrowlength,.5\pagegrid@arrowlength) %
                -- (\pagegrid@arrowlength,\pagegrid@arrowlength) %
                -- (.5\pagegrid@arrowlength,\pagegrid@arrowlength);%
              \fi
              \draw[%
                \pagegrid@origin@a,%
                {\pagegrid@firstcolor},%
                stroke,%
                line width=1pt,%
                line cap=round%
              ] (0mm,0mm) %
              -- (\pagegrid@arrowlength,\pagegrid@arrowlength) %
                 (\pagegrid@arrowlength,.5\pagegrid@arrowlength) %
              -- (\pagegrid@arrowlength,\pagegrid@arrowlength) %
              -- (.5\pagegrid@arrowlength,\pagegrid@arrowlength);%
            \fi
          \end{tikzpicture}%
        }%
      }%
    }%
    \ifnum\pagegrid@double=\tw@
      \pagegrid@ShipoutDouble\pagegrid@DoubleBox
    \fi
  \fi
}
%    \end{macrocode}
%
%    \begin{macrocode}
\pagegrid@AtEnd%
%</package>
%    \end{macrocode}
%
% \section{Test}
%
% \subsection{Catcode checks for loading}
%
%    \begin{macrocode}
%<*test1>
%    \end{macrocode}
%    \begin{macrocode}
\catcode`\{=1 %
\catcode`\}=2 %
\catcode`\#=6 %
\catcode`\@=11 %
\expandafter\ifx\csname count@\endcsname\relax
  \countdef\count@=255 %
\fi
\expandafter\ifx\csname @gobble\endcsname\relax
  \long\def\@gobble#1{}%
\fi
\expandafter\ifx\csname @firstofone\endcsname\relax
  \long\def\@firstofone#1{#1}%
\fi
\expandafter\ifx\csname loop\endcsname\relax
  \expandafter\@firstofone
\else
  \expandafter\@gobble
\fi
{%
  \def\loop#1\repeat{%
    \def\body{#1}%
    \iterate
  }%
  \def\iterate{%
    \body
      \let\next\iterate
    \else
      \let\next\relax
    \fi
    \next
  }%
  \let\repeat=\fi
}%
\def\RestoreCatcodes{}
\count@=0 %
\loop
  \edef\RestoreCatcodes{%
    \RestoreCatcodes
    \catcode\the\count@=\the\catcode\count@\relax
  }%
\ifnum\count@<255 %
  \advance\count@ 1 %
\repeat

\def\RangeCatcodeInvalid#1#2{%
  \count@=#1\relax
  \loop
    \catcode\count@=15 %
  \ifnum\count@<#2\relax
    \advance\count@ 1 %
  \repeat
}
\def\RangeCatcodeCheck#1#2#3{%
  \count@=#1\relax
  \loop
    \ifnum#3=\catcode\count@
    \else
      \errmessage{%
        Character \the\count@\space
        with wrong catcode \the\catcode\count@\space
        instead of \number#3%
      }%
    \fi
  \ifnum\count@<#2\relax
    \advance\count@ 1 %
  \repeat
}
\def\space{ }
\expandafter\ifx\csname LoadCommand\endcsname\relax
  \def\LoadCommand{\input pagegrid.sty\relax}%
\fi
\def\Test{%
  \RangeCatcodeInvalid{0}{47}%
  \RangeCatcodeInvalid{58}{64}%
  \RangeCatcodeInvalid{91}{96}%
  \RangeCatcodeInvalid{123}{255}%
  \catcode`\@=12 %
  \catcode`\\=0 %
  \catcode`\%=14 %
  \LoadCommand
  \RangeCatcodeCheck{0}{36}{15}%
  \RangeCatcodeCheck{37}{37}{14}%
  \RangeCatcodeCheck{38}{47}{15}%
  \RangeCatcodeCheck{48}{57}{12}%
  \RangeCatcodeCheck{58}{63}{15}%
  \RangeCatcodeCheck{64}{64}{12}%
  \RangeCatcodeCheck{65}{90}{11}%
  \RangeCatcodeCheck{91}{91}{15}%
  \RangeCatcodeCheck{92}{92}{0}%
  \RangeCatcodeCheck{93}{96}{15}%
  \RangeCatcodeCheck{97}{122}{11}%
  \RangeCatcodeCheck{123}{255}{15}%
  \RestoreCatcodes
}
\Test
\csname @@end\endcsname
\end
%    \end{macrocode}
%    \begin{macrocode}
%</test1>
%    \end{macrocode}
%
% \section{Installation}
%
% \subsection{Download}
%
% \paragraph{Package.} This package is available on
% CTAN\footnote{\CTANpkg{pagegrid}}:
% \begin{description}
% \item[\CTAN{macros/latex/contrib/oberdiek/pagegrid.dtx}] The source file.
% \item[\CTAN{macros/latex/contrib/oberdiek/pagegrid.pdf}] Documentation.
% \end{description}
%
%
% \paragraph{Bundle.} All the packages of the bundle `oberdiek'
% are also available in a TDS compliant ZIP archive. There
% the packages are already unpacked and the documentation files
% are generated. The files and directories obey the TDS standard.
% \begin{description}
% \item[\CTANinstall{install/macros/latex/contrib/oberdiek.tds.zip}]
% \end{description}
% \emph{TDS} refers to the standard ``A Directory Structure
% for \TeX\ Files'' (\CTAN{tds/tds.pdf}). Directories
% with \xfile{texmf} in their name are usually organized this way.
%
% \subsection{Bundle installation}
%
% \paragraph{Unpacking.} Unpack the \xfile{oberdiek.tds.zip} in the
% TDS tree (also known as \xfile{texmf} tree) of your choice.
% Example (linux):
% \begin{quote}
%   |unzip oberdiek.tds.zip -d ~/texmf|
% \end{quote}
%
% \paragraph{Script installation.}
% Check the directory \xfile{TDS:scripts/oberdiek/} for
% scripts that need further installation steps.
%
% \subsection{Package installation}
%
% \paragraph{Unpacking.} The \xfile{.dtx} file is a self-extracting
% \docstrip\ archive. The files are extracted by running the
% \xfile{.dtx} through \plainTeX:
% \begin{quote}
%   \verb|tex pagegrid.dtx|
% \end{quote}
%
% \paragraph{TDS.} Now the different files must be moved into
% the different directories in your installation TDS tree
% (also known as \xfile{texmf} tree):
% \begin{quote}
% \def\t{^^A
% \begin{tabular}{@{}>{\ttfamily}l@{ $\rightarrow$ }>{\ttfamily}l@{}}
%   pagegrid.sty & tex/latex/oberdiek/pagegrid.sty\\
%   pagegrid.pdf & doc/latex/oberdiek/pagegrid.pdf\\
%   test/pagegrid-test1.tex & doc/latex/oberdiek/test/pagegrid-test1.tex\\
%   pagegrid.dtx & source/latex/oberdiek/pagegrid.dtx\\
% \end{tabular}^^A
% }^^A
% \sbox0{\t}^^A
% \ifdim\wd0>\linewidth
%   \begingroup
%     \advance\linewidth by\leftmargin
%     \advance\linewidth by\rightmargin
%   \edef\x{\endgroup
%     \def\noexpand\lw{\the\linewidth}^^A
%   }\x
%   \def\lwbox{^^A
%     \leavevmode
%     \hbox to \linewidth{^^A
%       \kern-\leftmargin\relax
%       \hss
%       \usebox0
%       \hss
%       \kern-\rightmargin\relax
%     }^^A
%   }^^A
%   \ifdim\wd0>\lw
%     \sbox0{\small\t}^^A
%     \ifdim\wd0>\linewidth
%       \ifdim\wd0>\lw
%         \sbox0{\footnotesize\t}^^A
%         \ifdim\wd0>\linewidth
%           \ifdim\wd0>\lw
%             \sbox0{\scriptsize\t}^^A
%             \ifdim\wd0>\linewidth
%               \ifdim\wd0>\lw
%                 \sbox0{\tiny\t}^^A
%                 \ifdim\wd0>\linewidth
%                   \lwbox
%                 \else
%                   \usebox0
%                 \fi
%               \else
%                 \lwbox
%               \fi
%             \else
%               \usebox0
%             \fi
%           \else
%             \lwbox
%           \fi
%         \else
%           \usebox0
%         \fi
%       \else
%         \lwbox
%       \fi
%     \else
%       \usebox0
%     \fi
%   \else
%     \lwbox
%   \fi
% \else
%   \usebox0
% \fi
% \end{quote}
% If you have a \xfile{docstrip.cfg} that configures and enables \docstrip's
% TDS installing feature, then some files can already be in the right
% place, see the documentation of \docstrip.
%
% \subsection{Refresh file name databases}
%
% If your \TeX~distribution
% (\TeX\,Live, \mikTeX, \dots) relies on file name databases, you must refresh
% these. For example, \TeX\,Live\ users run \verb|texhash| or
% \verb|mktexlsr|.
%
% \subsection{Some details for the interested}
%
% \paragraph{Unpacking with \LaTeX.}
% The \xfile{.dtx} chooses its action depending on the format:
% \begin{description}
% \item[\plainTeX:] Run \docstrip\ and extract the files.
% \item[\LaTeX:] Generate the documentation.
% \end{description}
% If you insist on using \LaTeX\ for \docstrip\ (really,
% \docstrip\ does not need \LaTeX), then inform the autodetect routine
% about your intention:
% \begin{quote}
%   \verb|latex \let\install=y\input{pagegrid.dtx}|
% \end{quote}
% Do not forget to quote the argument according to the demands
% of your shell.
%
% \paragraph{Generating the documentation.}
% You can use both the \xfile{.dtx} or the \xfile{.drv} to generate
% the documentation. The process can be configured by the
% configuration file \xfile{ltxdoc.cfg}. For instance, put this
% line into this file, if you want to have A4 as paper format:
% \begin{quote}
%   \verb|\PassOptionsToClass{a4paper}{article}|
% \end{quote}
% An example follows how to generate the
% documentation with pdf\LaTeX:
% \begin{quote}
%\begin{verbatim}
%pdflatex pagegrid.dtx
%makeindex -s gind.ist pagegrid.idx
%pdflatex pagegrid.dtx
%makeindex -s gind.ist pagegrid.idx
%pdflatex pagegrid.dtx
%\end{verbatim}
% \end{quote}
%
% \section{Acknowledgement}
%
% \begin{description}
% \item[Klaus Braune:]
%  He provided the idea and the first \xpackage{tikz} code.
% \end{description}
%
% \begin{History}
%   \begin{Version}{2009/11/06 v1.0}
%   \item
%     The first version.
%   \end{Version}
%   \begin{Version}{2009/11/06 v1.1}
%   \item
%     Option \xoption{foreground} added.
%   \end{Version}
%   \begin{Version}{2009/12/02 v1.2}
%   \item
%     Color options, arrow options added.
%   \item
%     Names for origin options changed.
%   \end{Version}
%   \begin{Version}{2009/12/03 v1.3}
%   \item
%     Option \xoption{double} added.
%   \item
%     First CTAN release.
%   \end{Version}
%   \begin{Version}{2009/12/04 v1.4}
%   \item
%     Option \xoption{double}: Some unwanted side effects removed.
%   \end{Version}
%   \begin{Version}{2016/05/16 v1.5}
%   \item
%     Documentation updates.
%   \end{Version}
% \end{History}
%
% \PrintIndex
%
% \Finale
\endinput
|
% \end{quote}
% Do not forget to quote the argument according to the demands
% of your shell.
%
% \paragraph{Generating the documentation.}
% You can use both the \xfile{.dtx} or the \xfile{.drv} to generate
% the documentation. The process can be configured by the
% configuration file \xfile{ltxdoc.cfg}. For instance, put this
% line into this file, if you want to have A4 as paper format:
% \begin{quote}
%   \verb|\PassOptionsToClass{a4paper}{article}|
% \end{quote}
% An example follows how to generate the
% documentation with pdf\LaTeX:
% \begin{quote}
%\begin{verbatim}
%pdflatex pagegrid.dtx
%makeindex -s gind.ist pagegrid.idx
%pdflatex pagegrid.dtx
%makeindex -s gind.ist pagegrid.idx
%pdflatex pagegrid.dtx
%\end{verbatim}
% \end{quote}
%
% \section{Acknowledgement}
%
% \begin{description}
% \item[Klaus Braune:]
%  He provided the idea and the first \xpackage{tikz} code.
% \end{description}
%
% \begin{History}
%   \begin{Version}{2009/11/06 v1.0}
%   \item
%     The first version.
%   \end{Version}
%   \begin{Version}{2009/11/06 v1.1}
%   \item
%     Option \xoption{foreground} added.
%   \end{Version}
%   \begin{Version}{2009/12/02 v1.2}
%   \item
%     Color options, arrow options added.
%   \item
%     Names for origin options changed.
%   \end{Version}
%   \begin{Version}{2009/12/03 v1.3}
%   \item
%     Option \xoption{double} added.
%   \item
%     First CTAN release.
%   \end{Version}
%   \begin{Version}{2009/12/04 v1.4}
%   \item
%     Option \xoption{double}: Some unwanted side effects removed.
%   \end{Version}
%   \begin{Version}{2016/05/16 v1.5}
%   \item
%     Documentation updates.
%   \end{Version}
% \end{History}
%
% \PrintIndex
%
% \Finale
\endinput
|
% \end{quote}
% Do not forget to quote the argument according to the demands
% of your shell.
%
% \paragraph{Generating the documentation.}
% You can use both the \xfile{.dtx} or the \xfile{.drv} to generate
% the documentation. The process can be configured by the
% configuration file \xfile{ltxdoc.cfg}. For instance, put this
% line into this file, if you want to have A4 as paper format:
% \begin{quote}
%   \verb|\PassOptionsToClass{a4paper}{article}|
% \end{quote}
% An example follows how to generate the
% documentation with pdf\LaTeX:
% \begin{quote}
%\begin{verbatim}
%pdflatex pagegrid.dtx
%makeindex -s gind.ist pagegrid.idx
%pdflatex pagegrid.dtx
%makeindex -s gind.ist pagegrid.idx
%pdflatex pagegrid.dtx
%\end{verbatim}
% \end{quote}
%
% \section{Acknowledgement}
%
% \begin{description}
% \item[Klaus Braune:]
%  He provided the idea and the first \xpackage{tikz} code.
% \end{description}
%
% \begin{History}
%   \begin{Version}{2009/11/06 v1.0}
%   \item
%     The first version.
%   \end{Version}
%   \begin{Version}{2009/11/06 v1.1}
%   \item
%     Option \xoption{foreground} added.
%   \end{Version}
%   \begin{Version}{2009/12/02 v1.2}
%   \item
%     Color options, arrow options added.
%   \item
%     Names for origin options changed.
%   \end{Version}
%   \begin{Version}{2009/12/03 v1.3}
%   \item
%     Option \xoption{double} added.
%   \item
%     First CTAN release.
%   \end{Version}
%   \begin{Version}{2009/12/04 v1.4}
%   \item
%     Option \xoption{double}: Some unwanted side effects removed.
%   \end{Version}
%   \begin{Version}{2016/05/16 v1.5}
%   \item
%     Documentation updates.
%   \end{Version}
% \end{History}
%
% \PrintIndex
%
% \Finale
\endinput

%        (quote the arguments according to the demands of your shell)
%
% Documentation:
%    (a) If pagegrid.drv is present:
%           latex pagegrid.drv
%    (b) Without pagegrid.drv:
%           latex pagegrid.dtx; ...
%    The class ltxdoc loads the configuration file ltxdoc.cfg
%    if available. Here you can specify further options, e.g.
%    use A4 as paper format:
%       \PassOptionsToClass{a4paper}{article}
%
%    Programm calls to get the documentation (example):
%       pdflatex pagegrid.dtx
%       makeindex -s gind.ist pagegrid.idx
%       pdflatex pagegrid.dtx
%       makeindex -s gind.ist pagegrid.idx
%       pdflatex pagegrid.dtx
%
% Installation:
%    TDS:tex/latex/oberdiek/pagegrid.sty
%    TDS:doc/latex/oberdiek/pagegrid.pdf
%    TDS:doc/latex/oberdiek/test/pagegrid-test1.tex
%    TDS:source/latex/oberdiek/pagegrid.dtx
%
%<*ignore>
\begingroup
  \catcode123=1 %
  \catcode125=2 %
  \def\x{LaTeX2e}%
\expandafter\endgroup
\ifcase 0\ifx\install y1\fi\expandafter
         \ifx\csname processbatchFile\endcsname\relax\else1\fi
         \ifx\fmtname\x\else 1\fi\relax
\else\csname fi\endcsname
%</ignore>
%<*install>
\input docstrip.tex
\Msg{************************************************************************}
\Msg{* Installation}
\Msg{* Package: pagegrid 2016/05/16 v1.5 Print page grid in background (HO)}
\Msg{************************************************************************}

\keepsilent
\askforoverwritefalse

\let\MetaPrefix\relax
\preamble

This is a generated file.

Project: pagegrid
Version: 2016/05/16 v1.5

Copyright (C) 2009 by
   Heiko Oberdiek <heiko.oberdiek at googlemail.com>

This work may be distributed and/or modified under the
conditions of the LaTeX Project Public License, either
version 1.3c of this license or (at your option) any later
version. This version of this license is in
   http://www.latex-project.org/lppl/lppl-1-3c.txt
and the latest version of this license is in
   http://www.latex-project.org/lppl.txt
and version 1.3 or later is part of all distributions of
LaTeX version 2005/12/01 or later.

This work has the LPPL maintenance status "maintained".

This Current Maintainer of this work is Heiko Oberdiek.

This work consists of the main source file pagegrid.dtx
and the derived files
   pagegrid.sty, pagegrid.pdf, pagegrid.ins, pagegrid.drv,
   pagegrid-test1.tex.

\endpreamble
\let\MetaPrefix\DoubleperCent

\generate{%
  \file{pagegrid.ins}{\from{pagegrid.dtx}{install}}%
  \file{pagegrid.drv}{\from{pagegrid.dtx}{driver}}%
  \usedir{tex/latex/oberdiek}%
  \file{pagegrid.sty}{\from{pagegrid.dtx}{package}}%
%  \usedir{doc/latex/oberdiek/test}%
%  \file{pagegrid-test1.tex}{\from{pagegrid.dtx}{test1}}%
  \nopreamble
  \nopostamble
%  \usedir{source/latex/oberdiek/catalogue}%
%  \file{pagegrid.xml}{\from{pagegrid.dtx}{catalogue}}%
}

\catcode32=13\relax% active space
\let =\space%
\Msg{************************************************************************}
\Msg{*}
\Msg{* To finish the installation you have to move the following}
\Msg{* file into a directory searched by TeX:}
\Msg{*}
\Msg{*     pagegrid.sty}
\Msg{*}
\Msg{* To produce the documentation run the file `pagegrid.drv'}
\Msg{* through LaTeX.}
\Msg{*}
\Msg{* Happy TeXing!}
\Msg{*}
\Msg{************************************************************************}

\endbatchfile
%</install>
%<*ignore>
\fi
%</ignore>
%<*driver>
\NeedsTeXFormat{LaTeX2e}
\ProvidesFile{pagegrid.drv}%
  [2016/05/16 v1.5 Print page grid in background (HO)]%
\documentclass{ltxdoc}
\usepackage{holtxdoc}[2011/11/22]
\begin{document}
  \DocInput{pagegrid.dtx}%
\end{document}
%</driver>
% \fi
%
%
% \CharacterTable
%  {Upper-case    \A\B\C\D\E\F\G\H\I\J\K\L\M\N\O\P\Q\R\S\T\U\V\W\X\Y\Z
%   Lower-case    \a\b\c\d\e\f\g\h\i\j\k\l\m\n\o\p\q\r\s\t\u\v\w\x\y\z
%   Digits        \0\1\2\3\4\5\6\7\8\9
%   Exclamation   \!     Double quote  \"     Hash (number) \#
%   Dollar        \$     Percent       \%     Ampersand     \&
%   Acute accent  \'     Left paren    \(     Right paren   \)
%   Asterisk      \*     Plus          \+     Comma         \,
%   Minus         \-     Point         \.     Solidus       \/
%   Colon         \:     Semicolon     \;     Less than     \<
%   Equals        \=     Greater than  \>     Question mark \?
%   Commercial at \@     Left bracket  \[     Backslash     \\
%   Right bracket \]     Circumflex    \^     Underscore    \_
%   Grave accent  \`     Left brace    \{     Vertical bar  \|
%   Right brace   \}     Tilde         \~}
%
% \GetFileInfo{pagegrid.drv}
%
% \title{The \xpackage{pagegrid} package}
% \date{2016/05/16 v1.5}
% \author{Heiko Oberdiek\thanks
% {Please report any issues at https://github.com/ho-tex/oberdiek/issues}\\
% \xemail{heiko.oberdiek at googlemail.com}}
%
% \maketitle
%
% \begin{abstract}
% The \LaTeX\ package prints a page grid in the background.
% \end{abstract}
%
% \tableofcontents
%
% \section{Documentation}
%
% The package puts a grid on the paper. It was written for
% developers of a class or package
% who have to put elements on definite locations on a page
% (e.g. letter class). The grid allows a faster optical check,
% whether the positions are correct. If the previewer already
% offers features for measuring, the package might be obsolete.
% Otherwise it saves the developer from printing the page and
% measuring by hand.
%
% \subsection{Options}
%
% Options are evaluated in the following order:
% \begin{enumerate}
% \item
%  Configuration file \xfile{pagegrid.cfg} using \cs{pagegridsetup}
%  if the file exists.
%  \item
%  Package options given for \cs{usepackage}.
%  \item
%  Later calls of \cs{pagegridsetup}.
% \end{enumerate}
% \begin{declcs}{pagegridsetup}\M{option list}
% \end{declcs}
% The options are key value options. Boolean options are enabled by
% default (without value) or by using the explicit value \texttt{true}.
% Value \texttt{false} disable the option.
%
% \subsubsection{Options \xoption{enable}, \xoption{disable}}
%
% \begin{description}
% \item[\xoption{enable}:] This boolean option controls whether the page grid
%   is drawn. As default the page grid drawing is activated.
% \item[\xoption{disable}:] It is the opposite
%   of option \xoption{enable}. It was added for convenience and
%   allows the abbreviation \texttt{disable} for \texttt{enable=false}.
% \end{description}
%
% \subsubsection{Grid origins}
%
% The package supports up to two grids on a page allowing
% measurement from opposite directions. As default two grids are drawn,
% the first from bottom left to top right. The origin of the second
% grid is at the opposite top right corner.
% The origins are controlled by the following options.
% The number of grids (one or two) depend on the number of these options
% in one call of \cs{pagegridsetup}.
% The following frame shows a paper and in its corners are the
% corresponding options. At the left and right side alias names
% are given for the options inside the paper.
% \begin{quote}
% \begin{tabular}{@{}r|@{\,}l@{\qquad}r@{\,}|l@{}}
% \cline{2-3}
% \xoption{left-top}, \xoption{lt}, \xoption{top-left}
% & \vphantom{\"U}\xoption{tl} & \xoption{tr}
% & \xoption{top-right}, \xoption{rt}, \xoption{right-top}\\
% &&&\\
% \xoption{left-bottom}, \xoption{lb}, \xoption{bottom-left}
% & \xoption{bl} & \xoption{br}
% & \xoption{bottom-right}, \xoption{rb}, \xoption{right-bottom}\\
% \cline{2-3}
% \end{tabular}
% \end{quote}
% Examples:
% \begin{quote}
% |\pagegridsetup{bl,tr}|
% \end{quote}
% This is the default setting with two grids as described previously.
% The following setups one grid only. Its origin is the upper left
% corner:
% \begin{quote}
% |\pagegridsetup{top-left}|
% \end{quote}
%
% \subsubsection{Grid unit}
%
% \begin{description}
% \item[\xoption{step}] This option takes a length and
% setups the unit for the grid. The page width and page height
% should be multiples of this unit.
% Currently the default is \texttt{1mm}. But this might change
% later by a heuristic based on the paper size.
% \end{description}
%
% \subsubsection{Color options}
%
% The basic grid lines are drawn as ultra thin help lines and is only
% drawn for the first grid.
% Each tenth and fiftyth line of the basic net is drawn thicker in a special
% color for the two grids.
% \begin{description}
% \item[\xoption{firstcolor}:] Color for the thicker lines and the arrows
% of the first grid. Default value is \texttt{red}.
% \item[\xoption{secondcolor}:] Color for the thicker lines and the arrows
% of the second grid. Default value is \texttt{blue}.
% \end{description}
% Use a color specification that package \xpackage{tikz} understands.
% (The grid is drawn with \xpackage{pgf}/\xpackage{tikz}.)
%
% \subsubsection{Arrow options}
%
% Arrows are put at the origin at the grid to show the grid start
% and the direction of the grid.
% \begin{description}
% \item[\xoption{arrows}:] This boolean option turns the arrows on or off.
% As default arrows are enabled.
% \item[\xoption{arrowlength}:] The length given as value is the
% length of the edge of a square at the origin within the
% arrow is put as diagonal. Default is 10 times the grid unit (10\,mm).
% The real arrow length is this length multiplied by $\sqrt2$.
% \end{description}
%
% \subsubsection{Miscellaneous options}
%
% \begin{description}
% \item[\xoption{double}:] The output page is doubled, one without page
% grid and the other with page grid. Possible values are shown in the
% following table:
% \begin{quote}
% \begin{tabular}{ll}
% Option & Meaning\\
% \hline
% |false| & Turns option off.\\
% |first| & Grid page comes first.\\
% |last| & Grid page comes after the page without grid.\\
% |true| & Same as |last|.\\
% \meta{no value} & Same as |true|.\\
% \end{tabular}
% \end{quote}
% \textbf{Note:}
% The double output of the page has side effects.
% All whatits are executed twice, for example: file writing
% and anchor setting. Some unwanted actions are catched such
% as multiple \cs{label} definitions, duplicate entries in
% the table of contents. For bookmarks, use package \xpackage{bookmarks}.
% \item[\xoption{foreground}:] Boolean option, default is \texttt{false}.
% Sometimes there might be elements on the page (e.g. large images)
% that hide the grid. Then option \xoption{foreground} puts the grids
% over the current output page.
% \end{description}
%
% \StopEventually{
% }
%
% \section{Implementation}
%    \begin{macrocode}
%<*package>
%    \end{macrocode}
%    Reload check, especially if the package is not used with \LaTeX.
%    \begin{macrocode}
\begingroup\catcode61\catcode48\catcode32=10\relax%
  \catcode13=5 % ^^M
  \endlinechar=13 %
  \catcode35=6 % #
  \catcode39=12 % '
  \catcode44=12 % ,
  \catcode45=12 % -
  \catcode46=12 % .
  \catcode58=12 % :
  \catcode64=11 % @
  \catcode123=1 % {
  \catcode125=2 % }
  \expandafter\let\expandafter\x\csname ver@pagegrid.sty\endcsname
  \ifx\x\relax % plain-TeX, first loading
  \else
    \def\empty{}%
    \ifx\x\empty % LaTeX, first loading,
      % variable is initialized, but \ProvidesPackage not yet seen
    \else
      \expandafter\ifx\csname PackageInfo\endcsname\relax
        \def\x#1#2{%
          \immediate\write-1{Package #1 Info: #2.}%
        }%
      \else
        \def\x#1#2{\PackageInfo{#1}{#2, stopped}}%
      \fi
      \x{pagegrid}{The package is already loaded}%
      \aftergroup\endinput
    \fi
  \fi
\endgroup%
%    \end{macrocode}
%    Package identification:
%    \begin{macrocode}
\begingroup\catcode61\catcode48\catcode32=10\relax%
  \catcode13=5 % ^^M
  \endlinechar=13 %
  \catcode35=6 % #
  \catcode39=12 % '
  \catcode40=12 % (
  \catcode41=12 % )
  \catcode44=12 % ,
  \catcode45=12 % -
  \catcode46=12 % .
  \catcode47=12 % /
  \catcode58=12 % :
  \catcode64=11 % @
  \catcode91=12 % [
  \catcode93=12 % ]
  \catcode123=1 % {
  \catcode125=2 % }
  \expandafter\ifx\csname ProvidesPackage\endcsname\relax
    \def\x#1#2#3[#4]{\endgroup
      \immediate\write-1{Package: #3 #4}%
      \xdef#1{#4}%
    }%
  \else
    \def\x#1#2[#3]{\endgroup
      #2[{#3}]%
      \ifx#1\@undefined
        \xdef#1{#3}%
      \fi
      \ifx#1\relax
        \xdef#1{#3}%
      \fi
    }%
  \fi
\expandafter\x\csname ver@pagegrid.sty\endcsname
\ProvidesPackage{pagegrid}%
  [2016/05/16 v1.5 Print page grid in background (HO)]%
%    \end{macrocode}
%
%    \begin{macrocode}
\begingroup\catcode61\catcode48\catcode32=10\relax%
  \catcode13=5 % ^^M
  \endlinechar=13 %
  \catcode123=1 % {
  \catcode125=2 % }
  \catcode64=11 % @
  \def\x{\endgroup
    \expandafter\edef\csname pagegrid@AtEnd\endcsname{%
      \endlinechar=\the\endlinechar\relax
      \catcode13=\the\catcode13\relax
      \catcode32=\the\catcode32\relax
      \catcode35=\the\catcode35\relax
      \catcode61=\the\catcode61\relax
      \catcode64=\the\catcode64\relax
      \catcode123=\the\catcode123\relax
      \catcode125=\the\catcode125\relax
    }%
  }%
\x\catcode61\catcode48\catcode32=10\relax%
\catcode13=5 % ^^M
\endlinechar=13 %
\catcode35=6 % #
\catcode64=11 % @
\catcode123=1 % {
\catcode125=2 % }
\def\TMP@EnsureCode#1#2{%
  \edef\pagegrid@AtEnd{%
    \pagegrid@AtEnd
    \catcode#1=\the\catcode#1\relax
  }%
  \catcode#1=#2\relax
}
\TMP@EnsureCode{9}{10}% (tab)
\TMP@EnsureCode{10}{12}% ^^J
\TMP@EnsureCode{33}{12}% !
\TMP@EnsureCode{34}{12}% "
\TMP@EnsureCode{36}{3}% $
\TMP@EnsureCode{38}{4}% &
\TMP@EnsureCode{39}{12}% '
\TMP@EnsureCode{40}{12}% (
\TMP@EnsureCode{41}{12}% )
\TMP@EnsureCode{42}{12}% *
\TMP@EnsureCode{43}{12}% +
\TMP@EnsureCode{44}{12}% ,
\TMP@EnsureCode{45}{12}% -
\TMP@EnsureCode{46}{12}% .
\TMP@EnsureCode{47}{12}% /
\TMP@EnsureCode{58}{12}% :
\TMP@EnsureCode{59}{12}% ;
\TMP@EnsureCode{60}{12}% <
\TMP@EnsureCode{62}{12}% >
\TMP@EnsureCode{63}{12}% ?
\TMP@EnsureCode{91}{12}% [
\TMP@EnsureCode{93}{12}% ]
\TMP@EnsureCode{94}{7}% ^ (superscript)
\TMP@EnsureCode{95}{8}% _ (subscript)
\TMP@EnsureCode{96}{12}% `
\TMP@EnsureCode{124}{12}% |
\edef\pagegrid@AtEnd{\pagegrid@AtEnd\noexpand\endinput}
%    \end{macrocode}
%
%    \begin{macrocode}
\RequirePackage{tikz}
\RequirePackage{atbegshi}[2009/12/02]
\RequirePackage{kvoptions}[2009/07/17]
%    \end{macrocode}
%    \begin{macrocode}
\begingroup\expandafter\expandafter\expandafter\endgroup
\expandafter\ifx\csname stockwidth\endcsname\relax
  \def\pagegrid@width{\paperwidth}%
  \def\pagegrid@height{\paperheight}%
\else
  \def\pagegrid@width{\stockwidth}%
  \def\pagegrid@height{\stockheight}%
\fi
%    \end{macrocode}
%
%    \begin{macrocode}
\SetupKeyvalOptions{%
  family=pagegrid,%
  prefix=pagegrid@,%
}
\def\pagegrid@init{%
  \let\pagegrid@origin@a\@empty
  \let\pagegrid@origin@b\@empty
  \let\pagegrid@init\relax
}
\let\pagegrid@@init\pagegrid@init
\def\pagegrid@origin@a{bl}
\def\pagegrid@origin@b{tr}
\def\pagegrid@SetOrigin#1{%
  \pagegrid@init
  \ifx\pagegrid@origin@a\@empty
    \def\pagegrid@origin@a{#1}%
  \else
    \ifx\pagegrid@origin@b\@empty
    \else
      \let\pagegrid@origin@a\pagegrid@origin@b
    \fi
    \def\pagegrid@origin@b{#1}%
  \fi
}
\def\pagegrid@temp#1{%
  \DeclareVoidOption{#1}{\pagegrid@SetOrigin{#1}}%
  \@namedef{pagegrid@N@#1}{#1}%
}
\pagegrid@temp{bl}
\pagegrid@temp{br}
\pagegrid@temp{tl}
\pagegrid@temp{tr}
\def\pagegrid@temp#1#2{%
  \DeclareVoidOption{#2}{\pagegrid@SetOrigin{#1}}%
}%
\pagegrid@temp{bl}{lb}
\pagegrid@temp{br}{rb}
\pagegrid@temp{tl}{lt}
\pagegrid@temp{tr}{rt}
\pagegrid@temp{bl}{bottom-left}
\pagegrid@temp{br}{bottom-right}
\pagegrid@temp{tl}{top-left}
\pagegrid@temp{tr}{top-right}
\pagegrid@temp{bl}{left-bottom}
\pagegrid@temp{br}{right-bottom}
\pagegrid@temp{tl}{left-top}
\pagegrid@temp{tr}{right-top}
%    \end{macrocode}
%    \begin{macrocode}
\DeclareBoolOption[true]{enable}
\DeclareComplementaryOption{disable}{enable}
%    \end{macrocode}
%    \begin{macrocode}
\DeclareBoolOption{foreground}
%    \end{macrocode}
%    \begin{macrocode}
\newlength{\pagegrid@step}
\define@key{pagegrid}{step}{%
  \setlength{\pagegrid@step}{#1}%
}
%    \end{macrocode}
%    \begin{macrocode}
\DeclareStringOption[red]{firstcolor}
\DeclareStringOption[blue]{secondcolor}
%    \end{macrocode}
%    \begin{macrocode}
\DeclareBoolOption[true]{arrows}
\newlength\pagegrid@arrowlength
\pagegrid@arrowlength=\z@
\define@key{pagegrid}{arrowlength}{%
  \setlength{\pagegrid@arrowlength}{#1}%
}
%    \end{macrocode}
%    \begin{macrocode}
\define@key{pagegrid}{double}[true]{%
  \@ifundefined{pagegrid@double@#1}{%
    \PackageWarning{pagegrid}{%
      Unsupported value `#1' for option `double'.\MessageBreak
      Known values are:\MessageBreak
      `false', `first', `last', `true'.\MessageBreak
      Now `false' is used%
    }%
    \chardef\pagegrid@double\z@
  }{%
    \chardef\pagegrid@double\csname pagegrid@double@#1\endcsname\relax
  }%
}
\@namedef{pagegrid@double@false}{0}
\@namedef{pagegrid@double@first}{1}
\@namedef{pagegrid@double@last}{2}
\@namedef{pagegrid@double@true}{2}
\chardef\pagegrid@double\z@
%    \end{macrocode}
%    \begin{macrocode}
\newcommand*{\pagegridsetup}{%
  \let\pagegrid@init\pagegrid@@init
  \setkeys{pagegrid}%
}
%    \end{macrocode}
%    \begin{macrocode}
\pagegridsetup{%
  step=1mm%
}
\InputIfFileExists{pagegrid.cfg}{}%
\ProcessKeyvalOptions*\relax
\AtBeginDocument{%
  \ifdim\pagegrid@arrowlength>\z@
  \else
    \pagegrid@arrowlength=10\pagegrid@step
  \fi
}
%    \end{macrocode}
%
%    \begin{macrocode}
\def\pagegridShipoutDoubleBegin{%
  \begingroup
  \let\newlabel\@gobbletwo
  \let\zref@newlabel\@gobbletwo
  \let\@writefile\@gobbletwo
  \let\select@language\@gobble
}
\def\pagegridShipoutDoubleEnd{%
  \endgroup
}
\def\pagegrid@WriteDouble#1#2{%
  \immediate\write#1{%
    \@backslashchar csname %
    pagegridShipoutDouble#2%
    \@backslashchar endcsname%
  }%
}
\def\pagegrid@ShipoutDouble#1{%
  \begingroup
    \if@filesw
      \pagegrid@WriteDouble\@mainaux{Begin}%
      \ifx\@auxout\@partaux
        \pagegrid@WriteDouble\@partaux{Begin}%
        \def\pagegrid@temp{%
          \pagegrid@WriteDouble\@mainaux{End}%
          \pagegrid@WriteDouble\@partaux{End}%
        }%
      \else
        \def\pagegrid@temp{%
          \pagegrid@WriteDouble\@mainaux{End}%
        }%
      \fi
    \else
      \def\pagegrid@temp{}%
    \fi
    \let\protect\noexpand
    \AtBeginShipoutOriginalShipout\copy#1\relax
    \pagegrid@temp
  \endgroup
}
%    \end{macrocode}
%
%    \begin{macrocode}
\AtBeginShipout{%
  \ifdim\pagegrid@step>\z@
  \else
    \pagegrid@enablefalse
  \fi
  \ifpagegrid@enable
    \ifnum\pagegrid@double=\@ne
      \pagegrid@ShipoutDouble\AtBeginShipoutBox
    \else
      \ifnum\pagegrid@double=\tw@
        \@ifundefined{pagegrid@DoubleBox}{%
          \newbox\pagegrid@DoubleBox
        }{}%
        \setbox\pagegrid@DoubleBox=\copy\AtBeginShipoutBox
      \fi
    \fi
    \ifpagegrid@foreground
      \expandafter\AtBeginShipoutUpperLeftForeground
    \else
      \expandafter\AtBeginShipoutUpperLeft
    \fi
    {%
      \put(0,0){%
        \makebox(0,0)[lt]{%
          \begin{tikzpicture}[%
            bl/.style={},%
            br/.style={xshift=\pagegrid@width,xscale=-1},%
            tl/.style={yshift=\pagegrid@height,yscale=-1},%
            tr/.style={xshift=\pagegrid@width,%
                       yshift=\pagegrid@height,scale=-1}%
          ]%
            \useasboundingbox
              (0mm,\pagegrid@height) rectangle (0mm,\pagegrid@height);%
            \draw[%
              \pagegrid@origin@a,%
              step=\pagegrid@step,%
              style=help lines,%
              ultra thin%
            ] (0mm,0mm) grid (\pagegrid@width,\pagegrid@height);%
            \ifx\pagegrid@origin@b\@empty
            \else
              \draw[%
                \pagegrid@origin@b,%
                step=10\pagegrid@step,%
                {\pagegrid@secondcolor},%
                very thin%
              ] (0mm,0mm) grid (\pagegrid@width,\pagegrid@height);%
            \fi
            \draw[%
               \pagegrid@origin@a,%
               step=10\pagegrid@step,%
               {\pagegrid@firstcolor},%
               very thin%
            ] (0mm,0mm) grid (\pagegrid@width,\pagegrid@height);%
            \ifx\pagegrid@origin@b\@empty
            \else
              \draw[%
                \pagegrid@origin@b,%
                step=50\pagegrid@step,%
                {\pagegrid@secondcolor},%
                thick%
              ] (0mm,0mm) grid (\pagegrid@width,\pagegrid@height);%
            \fi
            \draw[%
              \pagegrid@origin@a,%
              step=50\pagegrid@step,%
              {\pagegrid@firstcolor},%
              thick%
            ] (0mm,0mm) grid (\pagegrid@width,\pagegrid@height);%
            \ifpagegrid@arrows
              \ifx\pagegrid@origin@b\@empty
              \else
                \draw[%
                  \pagegrid@origin@b,%
                  {\pagegrid@secondcolor},%
                  stroke,%
                  line width=1pt,%
                  line cap=round%
                ] (0mm,0mm) %
                -- (\pagegrid@arrowlength,\pagegrid@arrowlength) %
                   (\pagegrid@arrowlength,.5\pagegrid@arrowlength) %
                -- (\pagegrid@arrowlength,\pagegrid@arrowlength) %
                -- (.5\pagegrid@arrowlength,\pagegrid@arrowlength);%
              \fi
              \draw[%
                \pagegrid@origin@a,%
                {\pagegrid@firstcolor},%
                stroke,%
                line width=1pt,%
                line cap=round%
              ] (0mm,0mm) %
              -- (\pagegrid@arrowlength,\pagegrid@arrowlength) %
                 (\pagegrid@arrowlength,.5\pagegrid@arrowlength) %
              -- (\pagegrid@arrowlength,\pagegrid@arrowlength) %
              -- (.5\pagegrid@arrowlength,\pagegrid@arrowlength);%
            \fi
          \end{tikzpicture}%
        }%
      }%
    }%
    \ifnum\pagegrid@double=\tw@
      \pagegrid@ShipoutDouble\pagegrid@DoubleBox
    \fi
  \fi
}
%    \end{macrocode}
%
%    \begin{macrocode}
\pagegrid@AtEnd%
%</package>
%    \end{macrocode}
%
% \section{Test}
%
% \subsection{Catcode checks for loading}
%
%    \begin{macrocode}
%<*test1>
%    \end{macrocode}
%    \begin{macrocode}
\catcode`\{=1 %
\catcode`\}=2 %
\catcode`\#=6 %
\catcode`\@=11 %
\expandafter\ifx\csname count@\endcsname\relax
  \countdef\count@=255 %
\fi
\expandafter\ifx\csname @gobble\endcsname\relax
  \long\def\@gobble#1{}%
\fi
\expandafter\ifx\csname @firstofone\endcsname\relax
  \long\def\@firstofone#1{#1}%
\fi
\expandafter\ifx\csname loop\endcsname\relax
  \expandafter\@firstofone
\else
  \expandafter\@gobble
\fi
{%
  \def\loop#1\repeat{%
    \def\body{#1}%
    \iterate
  }%
  \def\iterate{%
    \body
      \let\next\iterate
    \else
      \let\next\relax
    \fi
    \next
  }%
  \let\repeat=\fi
}%
\def\RestoreCatcodes{}
\count@=0 %
\loop
  \edef\RestoreCatcodes{%
    \RestoreCatcodes
    \catcode\the\count@=\the\catcode\count@\relax
  }%
\ifnum\count@<255 %
  \advance\count@ 1 %
\repeat

\def\RangeCatcodeInvalid#1#2{%
  \count@=#1\relax
  \loop
    \catcode\count@=15 %
  \ifnum\count@<#2\relax
    \advance\count@ 1 %
  \repeat
}
\def\RangeCatcodeCheck#1#2#3{%
  \count@=#1\relax
  \loop
    \ifnum#3=\catcode\count@
    \else
      \errmessage{%
        Character \the\count@\space
        with wrong catcode \the\catcode\count@\space
        instead of \number#3%
      }%
    \fi
  \ifnum\count@<#2\relax
    \advance\count@ 1 %
  \repeat
}
\def\space{ }
\expandafter\ifx\csname LoadCommand\endcsname\relax
  \def\LoadCommand{\input pagegrid.sty\relax}%
\fi
\def\Test{%
  \RangeCatcodeInvalid{0}{47}%
  \RangeCatcodeInvalid{58}{64}%
  \RangeCatcodeInvalid{91}{96}%
  \RangeCatcodeInvalid{123}{255}%
  \catcode`\@=12 %
  \catcode`\\=0 %
  \catcode`\%=14 %
  \LoadCommand
  \RangeCatcodeCheck{0}{36}{15}%
  \RangeCatcodeCheck{37}{37}{14}%
  \RangeCatcodeCheck{38}{47}{15}%
  \RangeCatcodeCheck{48}{57}{12}%
  \RangeCatcodeCheck{58}{63}{15}%
  \RangeCatcodeCheck{64}{64}{12}%
  \RangeCatcodeCheck{65}{90}{11}%
  \RangeCatcodeCheck{91}{91}{15}%
  \RangeCatcodeCheck{92}{92}{0}%
  \RangeCatcodeCheck{93}{96}{15}%
  \RangeCatcodeCheck{97}{122}{11}%
  \RangeCatcodeCheck{123}{255}{15}%
  \RestoreCatcodes
}
\Test
\csname @@end\endcsname
\end
%    \end{macrocode}
%    \begin{macrocode}
%</test1>
%    \end{macrocode}
%
% \section{Installation}
%
% \subsection{Download}
%
% \paragraph{Package.} This package is available on
% CTAN\footnote{\url{https://ctan.org/pkg/pagegrid}}:
% \begin{description}
% \item[\CTAN{macros/latex/contrib/oberdiek/pagegrid.dtx}] The source file.
% \item[\CTAN{macros/latex/contrib/oberdiek/pagegrid.pdf}] Documentation.
% \end{description}
%
%
% \paragraph{Bundle.} All the packages of the bundle `oberdiek'
% are also available in a TDS compliant ZIP archive. There
% the packages are already unpacked and the documentation files
% are generated. The files and directories obey the TDS standard.
% \begin{description}
% \item[\CTANinstall{install/macros/latex/contrib/oberdiek.tds.zip}]
% \end{description}
% \emph{TDS} refers to the standard ``A Directory Structure
% for \TeX\ Files'' (\CTAN{tds/tds.pdf}). Directories
% with \xfile{texmf} in their name are usually organized this way.
%
% \subsection{Bundle installation}
%
% \paragraph{Unpacking.} Unpack the \xfile{oberdiek.tds.zip} in the
% TDS tree (also known as \xfile{texmf} tree) of your choice.
% Example (linux):
% \begin{quote}
%   |unzip oberdiek.tds.zip -d ~/texmf|
% \end{quote}
%
% \paragraph{Script installation.}
% Check the directory \xfile{TDS:scripts/oberdiek/} for
% scripts that need further installation steps.
% Package \xpackage{attachfile2} comes with the Perl script
% \xfile{pdfatfi.pl} that should be installed in such a way
% that it can be called as \texttt{pdfatfi}.
% Example (linux):
% \begin{quote}
%   |chmod +x scripts/oberdiek/pdfatfi.pl|\\
%   |cp scripts/oberdiek/pdfatfi.pl /usr/local/bin/|
% \end{quote}
%
% \subsection{Package installation}
%
% \paragraph{Unpacking.} The \xfile{.dtx} file is a self-extracting
% \docstrip\ archive. The files are extracted by running the
% \xfile{.dtx} through \plainTeX:
% \begin{quote}
%   \verb|tex pagegrid.dtx|
% \end{quote}
%
% \paragraph{TDS.} Now the different files must be moved into
% the different directories in your installation TDS tree
% (also known as \xfile{texmf} tree):
% \begin{quote}
% \def\t{^^A
% \begin{tabular}{@{}>{\ttfamily}l@{ $\rightarrow$ }>{\ttfamily}l@{}}
%   pagegrid.sty & tex/latex/oberdiek/pagegrid.sty\\
%   pagegrid.pdf & doc/latex/oberdiek/pagegrid.pdf\\
%   test/pagegrid-test1.tex & doc/latex/oberdiek/test/pagegrid-test1.tex\\
%   pagegrid.dtx & source/latex/oberdiek/pagegrid.dtx\\
% \end{tabular}^^A
% }^^A
% \sbox0{\t}^^A
% \ifdim\wd0>\linewidth
%   \begingroup
%     \advance\linewidth by\leftmargin
%     \advance\linewidth by\rightmargin
%   \edef\x{\endgroup
%     \def\noexpand\lw{\the\linewidth}^^A
%   }\x
%   \def\lwbox{^^A
%     \leavevmode
%     \hbox to \linewidth{^^A
%       \kern-\leftmargin\relax
%       \hss
%       \usebox0
%       \hss
%       \kern-\rightmargin\relax
%     }^^A
%   }^^A
%   \ifdim\wd0>\lw
%     \sbox0{\small\t}^^A
%     \ifdim\wd0>\linewidth
%       \ifdim\wd0>\lw
%         \sbox0{\footnotesize\t}^^A
%         \ifdim\wd0>\linewidth
%           \ifdim\wd0>\lw
%             \sbox0{\scriptsize\t}^^A
%             \ifdim\wd0>\linewidth
%               \ifdim\wd0>\lw
%                 \sbox0{\tiny\t}^^A
%                 \ifdim\wd0>\linewidth
%                   \lwbox
%                 \else
%                   \usebox0
%                 \fi
%               \else
%                 \lwbox
%               \fi
%             \else
%               \usebox0
%             \fi
%           \else
%             \lwbox
%           \fi
%         \else
%           \usebox0
%         \fi
%       \else
%         \lwbox
%       \fi
%     \else
%       \usebox0
%     \fi
%   \else
%     \lwbox
%   \fi
% \else
%   \usebox0
% \fi
% \end{quote}
% If you have a \xfile{docstrip.cfg} that configures and enables \docstrip's
% TDS installing feature, then some files can already be in the right
% place, see the documentation of \docstrip.
%
% \subsection{Refresh file name databases}
%
% If your \TeX~distribution
% (\teTeX, \mikTeX, \dots) relies on file name databases, you must refresh
% these. For example, \teTeX\ users run \verb|texhash| or
% \verb|mktexlsr|.
%
% \subsection{Some details for the interested}
%
% \paragraph{Attached source.}
%
% The PDF documentation on CTAN also includes the
% \xfile{.dtx} source file. It can be extracted by
% AcrobatReader 6 or higher. Another option is \textsf{pdftk},
% e.g. unpack the file into the current directory:
% \begin{quote}
%   \verb|pdftk pagegrid.pdf unpack_files output .|
% \end{quote}
%
% \paragraph{Unpacking with \LaTeX.}
% The \xfile{.dtx} chooses its action depending on the format:
% \begin{description}
% \item[\plainTeX:] Run \docstrip\ and extract the files.
% \item[\LaTeX:] Generate the documentation.
% \end{description}
% If you insist on using \LaTeX\ for \docstrip\ (really,
% \docstrip\ does not need \LaTeX), then inform the autodetect routine
% about your intention:
% \begin{quote}
%   \verb|latex \let\install=y% \iffalse meta-comment
%
% File: pagegrid.dtx
% Version: 2016/05/16 v1.5
% Info: Print page grid in background
%
% Copyright (C) 2009 by
%    Heiko Oberdiek <heiko.oberdiek at googlemail.com>
%    2016
%    https://github.com/ho-tex/oberdiek/issues
%
% This work may be distributed and/or modified under the
% conditions of the LaTeX Project Public License, either
% version 1.3c of this license or (at your option) any later
% version. This version of this license is in
%    https://www.latex-project.org/lppl/lppl-1-3c.txt
% and the latest version of this license is in
%    https://www.latex-project.org/lppl.txt
% and version 1.3 or later is part of all distributions of
% LaTeX version 2005/12/01 or later.
%
% This work has the LPPL maintenance status "maintained".
%
% The Current Maintainers of this work are
% Heiko Oberdiek and the Oberdiek Package Support Group
% https://github.com/ho-tex/oberdiek/issues
%
% This work consists of the main source file pagegrid.dtx
% and the derived files
%    pagegrid.sty, pagegrid.pdf, pagegrid.ins, pagegrid.drv,
%    pagegrid-test1.tex.
%
% Distribution:
%    CTAN:macros/latex/contrib/oberdiek/pagegrid.dtx
%    CTAN:macros/latex/contrib/oberdiek/pagegrid.pdf
%
% Unpacking:
%    (a) If pagegrid.ins is present:
%           tex pagegrid.ins
%    (b) Without pagegrid.ins:
%           tex pagegrid.dtx
%    (c) If you insist on using LaTeX
%           latex \let\install=y% \iffalse meta-comment
%
% File: pagegrid.dtx
% Version: 2016/05/16 v1.5
% Info: Print page grid in background
%
% Copyright (C) 2009 by
%    Heiko Oberdiek <heiko.oberdiek at googlemail.com>
%    2016
%    https://github.com/ho-tex/oberdiek/issues
%
% This work may be distributed and/or modified under the
% conditions of the LaTeX Project Public License, either
% version 1.3c of this license or (at your option) any later
% version. This version of this license is in
%    https://www.latex-project.org/lppl/lppl-1-3c.txt
% and the latest version of this license is in
%    https://www.latex-project.org/lppl.txt
% and version 1.3 or later is part of all distributions of
% LaTeX version 2005/12/01 or later.
%
% This work has the LPPL maintenance status "maintained".
%
% The Current Maintainers of this work are
% Heiko Oberdiek and the Oberdiek Package Support Group
% https://github.com/ho-tex/oberdiek/issues
%
% This work consists of the main source file pagegrid.dtx
% and the derived files
%    pagegrid.sty, pagegrid.pdf, pagegrid.ins, pagegrid.drv,
%    pagegrid-test1.tex.
%
% Distribution:
%    CTAN:macros/latex/contrib/oberdiek/pagegrid.dtx
%    CTAN:macros/latex/contrib/oberdiek/pagegrid.pdf
%
% Unpacking:
%    (a) If pagegrid.ins is present:
%           tex pagegrid.ins
%    (b) Without pagegrid.ins:
%           tex pagegrid.dtx
%    (c) If you insist on using LaTeX
%           latex \let\install=y% \iffalse meta-comment
%
% File: pagegrid.dtx
% Version: 2016/05/16 v1.5
% Info: Print page grid in background
%
% Copyright (C) 2009 by
%    Heiko Oberdiek <heiko.oberdiek at googlemail.com>
%    2016
%    https://github.com/ho-tex/oberdiek/issues
%
% This work may be distributed and/or modified under the
% conditions of the LaTeX Project Public License, either
% version 1.3c of this license or (at your option) any later
% version. This version of this license is in
%    https://www.latex-project.org/lppl/lppl-1-3c.txt
% and the latest version of this license is in
%    https://www.latex-project.org/lppl.txt
% and version 1.3 or later is part of all distributions of
% LaTeX version 2005/12/01 or later.
%
% This work has the LPPL maintenance status "maintained".
%
% The Current Maintainers of this work are
% Heiko Oberdiek and the Oberdiek Package Support Group
% https://github.com/ho-tex/oberdiek/issues
%
% This work consists of the main source file pagegrid.dtx
% and the derived files
%    pagegrid.sty, pagegrid.pdf, pagegrid.ins, pagegrid.drv,
%    pagegrid-test1.tex.
%
% Distribution:
%    CTAN:macros/latex/contrib/oberdiek/pagegrid.dtx
%    CTAN:macros/latex/contrib/oberdiek/pagegrid.pdf
%
% Unpacking:
%    (a) If pagegrid.ins is present:
%           tex pagegrid.ins
%    (b) Without pagegrid.ins:
%           tex pagegrid.dtx
%    (c) If you insist on using LaTeX
%           latex \let\install=y\input{pagegrid.dtx}
%        (quote the arguments according to the demands of your shell)
%
% Documentation:
%    (a) If pagegrid.drv is present:
%           latex pagegrid.drv
%    (b) Without pagegrid.drv:
%           latex pagegrid.dtx; ...
%    The class ltxdoc loads the configuration file ltxdoc.cfg
%    if available. Here you can specify further options, e.g.
%    use A4 as paper format:
%       \PassOptionsToClass{a4paper}{article}
%
%    Programm calls to get the documentation (example):
%       pdflatex pagegrid.dtx
%       makeindex -s gind.ist pagegrid.idx
%       pdflatex pagegrid.dtx
%       makeindex -s gind.ist pagegrid.idx
%       pdflatex pagegrid.dtx
%
% Installation:
%    TDS:tex/latex/oberdiek/pagegrid.sty
%    TDS:doc/latex/oberdiek/pagegrid.pdf
%    TDS:doc/latex/oberdiek/test/pagegrid-test1.tex
%    TDS:source/latex/oberdiek/pagegrid.dtx
%
%<*ignore>
\begingroup
  \catcode123=1 %
  \catcode125=2 %
  \def\x{LaTeX2e}%
\expandafter\endgroup
\ifcase 0\ifx\install y1\fi\expandafter
         \ifx\csname processbatchFile\endcsname\relax\else1\fi
         \ifx\fmtname\x\else 1\fi\relax
\else\csname fi\endcsname
%</ignore>
%<*install>
\input docstrip.tex
\Msg{************************************************************************}
\Msg{* Installation}
\Msg{* Package: pagegrid 2016/05/16 v1.5 Print page grid in background (HO)}
\Msg{************************************************************************}

\keepsilent
\askforoverwritefalse

\let\MetaPrefix\relax
\preamble

This is a generated file.

Project: pagegrid
Version: 2016/05/16 v1.5

Copyright (C) 2009 by
   Heiko Oberdiek <heiko.oberdiek at googlemail.com>

This work may be distributed and/or modified under the
conditions of the LaTeX Project Public License, either
version 1.3c of this license or (at your option) any later
version. This version of this license is in
   https://www.latex-project.org/lppl/lppl-1-3c.txt
and the latest version of this license is in
   https://www.latex-project.org/lppl.txt
and version 1.3 or later is part of all distributions of
LaTeX version 2005/12/01 or later.

This work has the LPPL maintenance status "maintained".

The Current Maintainers of this work are
Heiko Oberdiek and the Oberdiek Package Support Group
https://github.com/ho-tex/oberdiek/issues


This work consists of the main source file pagegrid.dtx
and the derived files
   pagegrid.sty, pagegrid.pdf, pagegrid.ins, pagegrid.drv,
   pagegrid-test1.tex.

\endpreamble
\let\MetaPrefix\DoubleperCent

\generate{%
  \file{pagegrid.ins}{\from{pagegrid.dtx}{install}}%
  \file{pagegrid.drv}{\from{pagegrid.dtx}{driver}}%
  \usedir{tex/latex/oberdiek}%
  \file{pagegrid.sty}{\from{pagegrid.dtx}{package}}%
%  \usedir{doc/latex/oberdiek/test}%
%  \file{pagegrid-test1.tex}{\from{pagegrid.dtx}{test1}}%
  \nopreamble
  \nopostamble
%  \usedir{source/latex/oberdiek/catalogue}%
%  \file{pagegrid.xml}{\from{pagegrid.dtx}{catalogue}}%
}

\catcode32=13\relax% active space
\let =\space%
\Msg{************************************************************************}
\Msg{*}
\Msg{* To finish the installation you have to move the following}
\Msg{* file into a directory searched by TeX:}
\Msg{*}
\Msg{*     pagegrid.sty}
\Msg{*}
\Msg{* To produce the documentation run the file `pagegrid.drv'}
\Msg{* through LaTeX.}
\Msg{*}
\Msg{* Happy TeXing!}
\Msg{*}
\Msg{************************************************************************}

\endbatchfile
%</install>
%<*ignore>
\fi
%</ignore>
%<*driver>
\NeedsTeXFormat{LaTeX2e}
\ProvidesFile{pagegrid.drv}%
  [2016/05/16 v1.5 Print page grid in background (HO)]%
\documentclass{ltxdoc}
\usepackage{holtxdoc}[2011/11/22]
\begin{document}
  \DocInput{pagegrid.dtx}%
\end{document}
%</driver>
% \fi
%
%
% \CharacterTable
%  {Upper-case    \A\B\C\D\E\F\G\H\I\J\K\L\M\N\O\P\Q\R\S\T\U\V\W\X\Y\Z
%   Lower-case    \a\b\c\d\e\f\g\h\i\j\k\l\m\n\o\p\q\r\s\t\u\v\w\x\y\z
%   Digits        \0\1\2\3\4\5\6\7\8\9
%   Exclamation   \!     Double quote  \"     Hash (number) \#
%   Dollar        \$     Percent       \%     Ampersand     \&
%   Acute accent  \'     Left paren    \(     Right paren   \)
%   Asterisk      \*     Plus          \+     Comma         \,
%   Minus         \-     Point         \.     Solidus       \/
%   Colon         \:     Semicolon     \;     Less than     \<
%   Equals        \=     Greater than  \>     Question mark \?
%   Commercial at \@     Left bracket  \[     Backslash     \\
%   Right bracket \]     Circumflex    \^     Underscore    \_
%   Grave accent  \`     Left brace    \{     Vertical bar  \|
%   Right brace   \}     Tilde         \~}
%
% \GetFileInfo{pagegrid.drv}
%
% \title{The \xpackage{pagegrid} package}
% \date{2016/05/16 v1.5}
% \author{Heiko Oberdiek\thanks
% {Please report any issues at \url{https://github.com/ho-tex/oberdiek/issues}}}
%
% \maketitle
%
% \begin{abstract}
% The \LaTeX\ package prints a page grid in the background.
% \end{abstract}
%
% \tableofcontents
%
% \section{Documentation}
%
% The package puts a grid on the paper. It was written for
% developers of a class or package
% who have to put elements on definite locations on a page
% (e.g. letter class). The grid allows a faster optical check,
% whether the positions are correct. If the previewer already
% offers features for measuring, the package might be obsolete.
% Otherwise it saves the developer from printing the page and
% measuring by hand.
%
% \subsection{Options}
%
% Options are evaluated in the following order:
% \begin{enumerate}
% \item
%  Configuration file \xfile{pagegrid.cfg} using \cs{pagegridsetup}
%  if the file exists.
%  \item
%  Package options given for \cs{usepackage}.
%  \item
%  Later calls of \cs{pagegridsetup}.
% \end{enumerate}
% \begin{declcs}{pagegridsetup}\M{option list}
% \end{declcs}
% The options are key value options. Boolean options are enabled by
% default (without value) or by using the explicit value \texttt{true}.
% Value \texttt{false} disable the option.
%
% \subsubsection{Options \xoption{enable}, \xoption{disable}}
%
% \begin{description}
% \item[\xoption{enable}:] This boolean option controls whether the page grid
%   is drawn. As default the page grid drawing is activated.
% \item[\xoption{disable}:] It is the opposite
%   of option \xoption{enable}. It was added for convenience and
%   allows the abbreviation \texttt{disable} for \texttt{enable=false}.
% \end{description}
%
% \subsubsection{Grid origins}
%
% The package supports up to two grids on a page allowing
% measurement from opposite directions. As default two grids are drawn,
% the first from bottom left to top right. The origin of the second
% grid is at the opposite top right corner.
% The origins are controlled by the following options.
% The number of grids (one or two) depend on the number of these options
% in one call of \cs{pagegridsetup}.
% The following frame shows a paper and in its corners are the
% corresponding options. At the left and right side alias names
% are given for the options inside the paper.
% \begin{quote}
% \begin{tabular}{@{}r|@{\,}l@{\qquad}r@{\,}|l@{}}
% \cline{2-3}
% \xoption{left-top}, \xoption{lt}, \xoption{top-left}
% & \vphantom{\"U}\xoption{tl} & \xoption{tr}
% & \xoption{top-right}, \xoption{rt}, \xoption{right-top}\\
% &&&\\
% \xoption{left-bottom}, \xoption{lb}, \xoption{bottom-left}
% & \xoption{bl} & \xoption{br}
% & \xoption{bottom-right}, \xoption{rb}, \xoption{right-bottom}\\
% \cline{2-3}
% \end{tabular}
% \end{quote}
% Examples:
% \begin{quote}
% |\pagegridsetup{bl,tr}|
% \end{quote}
% This is the default setting with two grids as described previously.
% The following setups one grid only. Its origin is the upper left
% corner:
% \begin{quote}
% |\pagegridsetup{top-left}|
% \end{quote}
%
% \subsubsection{Grid unit}
%
% \begin{description}
% \item[\xoption{step}] This option takes a length and
% setups the unit for the grid. The page width and page height
% should be multiples of this unit.
% Currently the default is \texttt{1mm}. But this might change
% later by a heuristic based on the paper size.
% \end{description}
%
% \subsubsection{Color options}
%
% The basic grid lines are drawn as ultra thin help lines and is only
% drawn for the first grid.
% Each tenth and fiftyth line of the basic net is drawn thicker in a special
% color for the two grids.
% \begin{description}
% \item[\xoption{firstcolor}:] Color for the thicker lines and the arrows
% of the first grid. Default value is \texttt{red}.
% \item[\xoption{secondcolor}:] Color for the thicker lines and the arrows
% of the second grid. Default value is \texttt{blue}.
% \end{description}
% Use a color specification that package \xpackage{tikz} understands.
% (The grid is drawn with \xpackage{pgf}/\xpackage{tikz}.)
%
% \subsubsection{Arrow options}
%
% Arrows are put at the origin at the grid to show the grid start
% and the direction of the grid.
% \begin{description}
% \item[\xoption{arrows}:] This boolean option turns the arrows on or off.
% As default arrows are enabled.
% \item[\xoption{arrowlength}:] The length given as value is the
% length of the edge of a square at the origin within the
% arrow is put as diagonal. Default is 10 times the grid unit (10\,mm).
% The real arrow length is this length multiplied by $\sqrt2$.
% \end{description}
%
% \subsubsection{Miscellaneous options}
%
% \begin{description}
% \item[\xoption{double}:] The output page is doubled, one without page
% grid and the other with page grid. Possible values are shown in the
% following table:
% \begin{quote}
% \begin{tabular}{ll}
% Option & Meaning\\
% \hline
% |false| & Turns option off.\\
% |first| & Grid page comes first.\\
% |last| & Grid page comes after the page without grid.\\
% |true| & Same as |last|.\\
% \meta{no value} & Same as |true|.\\
% \end{tabular}
% \end{quote}
% \textbf{Note:}
% The double output of the page has side effects.
% All whatits are executed twice, for example: file writing
% and anchor setting. Some unwanted actions are catched such
% as multiple \cs{label} definitions, duplicate entries in
% the table of contents. For bookmarks, use package \xpackage{bookmarks}.
% \item[\xoption{foreground}:] Boolean option, default is \texttt{false}.
% Sometimes there might be elements on the page (e.g. large images)
% that hide the grid. Then option \xoption{foreground} puts the grids
% over the current output page.
% \end{description}
%
% \StopEventually{
% }
%
% \section{Implementation}
%    \begin{macrocode}
%<*package>
%    \end{macrocode}
%    Reload check, especially if the package is not used with \LaTeX.
%    \begin{macrocode}
\begingroup\catcode61\catcode48\catcode32=10\relax%
  \catcode13=5 % ^^M
  \endlinechar=13 %
  \catcode35=6 % #
  \catcode39=12 % '
  \catcode44=12 % ,
  \catcode45=12 % -
  \catcode46=12 % .
  \catcode58=12 % :
  \catcode64=11 % @
  \catcode123=1 % {
  \catcode125=2 % }
  \expandafter\let\expandafter\x\csname ver@pagegrid.sty\endcsname
  \ifx\x\relax % plain-TeX, first loading
  \else
    \def\empty{}%
    \ifx\x\empty % LaTeX, first loading,
      % variable is initialized, but \ProvidesPackage not yet seen
    \else
      \expandafter\ifx\csname PackageInfo\endcsname\relax
        \def\x#1#2{%
          \immediate\write-1{Package #1 Info: #2.}%
        }%
      \else
        \def\x#1#2{\PackageInfo{#1}{#2, stopped}}%
      \fi
      \x{pagegrid}{The package is already loaded}%
      \aftergroup\endinput
    \fi
  \fi
\endgroup%
%    \end{macrocode}
%    Package identification:
%    \begin{macrocode}
\begingroup\catcode61\catcode48\catcode32=10\relax%
  \catcode13=5 % ^^M
  \endlinechar=13 %
  \catcode35=6 % #
  \catcode39=12 % '
  \catcode40=12 % (
  \catcode41=12 % )
  \catcode44=12 % ,
  \catcode45=12 % -
  \catcode46=12 % .
  \catcode47=12 % /
  \catcode58=12 % :
  \catcode64=11 % @
  \catcode91=12 % [
  \catcode93=12 % ]
  \catcode123=1 % {
  \catcode125=2 % }
  \expandafter\ifx\csname ProvidesPackage\endcsname\relax
    \def\x#1#2#3[#4]{\endgroup
      \immediate\write-1{Package: #3 #4}%
      \xdef#1{#4}%
    }%
  \else
    \def\x#1#2[#3]{\endgroup
      #2[{#3}]%
      \ifx#1\@undefined
        \xdef#1{#3}%
      \fi
      \ifx#1\relax
        \xdef#1{#3}%
      \fi
    }%
  \fi
\expandafter\x\csname ver@pagegrid.sty\endcsname
\ProvidesPackage{pagegrid}%
  [2016/05/16 v1.5 Print page grid in background (HO)]%
%    \end{macrocode}
%
%    \begin{macrocode}
\begingroup\catcode61\catcode48\catcode32=10\relax%
  \catcode13=5 % ^^M
  \endlinechar=13 %
  \catcode123=1 % {
  \catcode125=2 % }
  \catcode64=11 % @
  \def\x{\endgroup
    \expandafter\edef\csname pagegrid@AtEnd\endcsname{%
      \endlinechar=\the\endlinechar\relax
      \catcode13=\the\catcode13\relax
      \catcode32=\the\catcode32\relax
      \catcode35=\the\catcode35\relax
      \catcode61=\the\catcode61\relax
      \catcode64=\the\catcode64\relax
      \catcode123=\the\catcode123\relax
      \catcode125=\the\catcode125\relax
    }%
  }%
\x\catcode61\catcode48\catcode32=10\relax%
\catcode13=5 % ^^M
\endlinechar=13 %
\catcode35=6 % #
\catcode64=11 % @
\catcode123=1 % {
\catcode125=2 % }
\def\TMP@EnsureCode#1#2{%
  \edef\pagegrid@AtEnd{%
    \pagegrid@AtEnd
    \catcode#1=\the\catcode#1\relax
  }%
  \catcode#1=#2\relax
}
\TMP@EnsureCode{9}{10}% (tab)
\TMP@EnsureCode{10}{12}% ^^J
\TMP@EnsureCode{33}{12}% !
\TMP@EnsureCode{34}{12}% "
\TMP@EnsureCode{36}{3}% $
\TMP@EnsureCode{38}{4}% &
\TMP@EnsureCode{39}{12}% '
\TMP@EnsureCode{40}{12}% (
\TMP@EnsureCode{41}{12}% )
\TMP@EnsureCode{42}{12}% *
\TMP@EnsureCode{43}{12}% +
\TMP@EnsureCode{44}{12}% ,
\TMP@EnsureCode{45}{12}% -
\TMP@EnsureCode{46}{12}% .
\TMP@EnsureCode{47}{12}% /
\TMP@EnsureCode{58}{12}% :
\TMP@EnsureCode{59}{12}% ;
\TMP@EnsureCode{60}{12}% <
\TMP@EnsureCode{62}{12}% >
\TMP@EnsureCode{63}{12}% ?
\TMP@EnsureCode{91}{12}% [
\TMP@EnsureCode{93}{12}% ]
\TMP@EnsureCode{94}{7}% ^ (superscript)
\TMP@EnsureCode{95}{8}% _ (subscript)
\TMP@EnsureCode{96}{12}% `
\TMP@EnsureCode{124}{12}% |
\edef\pagegrid@AtEnd{\pagegrid@AtEnd\noexpand\endinput}
%    \end{macrocode}
%
%    \begin{macrocode}
\RequirePackage{tikz}
\RequirePackage{atbegshi}[2009/12/02]
\RequirePackage{kvoptions}[2009/07/17]
%    \end{macrocode}
%    \begin{macrocode}
\begingroup\expandafter\expandafter\expandafter\endgroup
\expandafter\ifx\csname stockwidth\endcsname\relax
  \def\pagegrid@width{\paperwidth}%
  \def\pagegrid@height{\paperheight}%
\else
  \def\pagegrid@width{\stockwidth}%
  \def\pagegrid@height{\stockheight}%
\fi
%    \end{macrocode}
%
%    \begin{macrocode}
\SetupKeyvalOptions{%
  family=pagegrid,%
  prefix=pagegrid@,%
}
\def\pagegrid@init{%
  \let\pagegrid@origin@a\@empty
  \let\pagegrid@origin@b\@empty
  \let\pagegrid@init\relax
}
\let\pagegrid@@init\pagegrid@init
\def\pagegrid@origin@a{bl}
\def\pagegrid@origin@b{tr}
\def\pagegrid@SetOrigin#1{%
  \pagegrid@init
  \ifx\pagegrid@origin@a\@empty
    \def\pagegrid@origin@a{#1}%
  \else
    \ifx\pagegrid@origin@b\@empty
    \else
      \let\pagegrid@origin@a\pagegrid@origin@b
    \fi
    \def\pagegrid@origin@b{#1}%
  \fi
}
\def\pagegrid@temp#1{%
  \DeclareVoidOption{#1}{\pagegrid@SetOrigin{#1}}%
  \@namedef{pagegrid@N@#1}{#1}%
}
\pagegrid@temp{bl}
\pagegrid@temp{br}
\pagegrid@temp{tl}
\pagegrid@temp{tr}
\def\pagegrid@temp#1#2{%
  \DeclareVoidOption{#2}{\pagegrid@SetOrigin{#1}}%
}%
\pagegrid@temp{bl}{lb}
\pagegrid@temp{br}{rb}
\pagegrid@temp{tl}{lt}
\pagegrid@temp{tr}{rt}
\pagegrid@temp{bl}{bottom-left}
\pagegrid@temp{br}{bottom-right}
\pagegrid@temp{tl}{top-left}
\pagegrid@temp{tr}{top-right}
\pagegrid@temp{bl}{left-bottom}
\pagegrid@temp{br}{right-bottom}
\pagegrid@temp{tl}{left-top}
\pagegrid@temp{tr}{right-top}
%    \end{macrocode}
%    \begin{macrocode}
\DeclareBoolOption[true]{enable}
\DeclareComplementaryOption{disable}{enable}
%    \end{macrocode}
%    \begin{macrocode}
\DeclareBoolOption{foreground}
%    \end{macrocode}
%    \begin{macrocode}
\newlength{\pagegrid@step}
\define@key{pagegrid}{step}{%
  \setlength{\pagegrid@step}{#1}%
}
%    \end{macrocode}
%    \begin{macrocode}
\DeclareStringOption[red]{firstcolor}
\DeclareStringOption[blue]{secondcolor}
%    \end{macrocode}
%    \begin{macrocode}
\DeclareBoolOption[true]{arrows}
\newlength\pagegrid@arrowlength
\pagegrid@arrowlength=\z@
\define@key{pagegrid}{arrowlength}{%
  \setlength{\pagegrid@arrowlength}{#1}%
}
%    \end{macrocode}
%    \begin{macrocode}
\define@key{pagegrid}{double}[true]{%
  \@ifundefined{pagegrid@double@#1}{%
    \PackageWarning{pagegrid}{%
      Unsupported value `#1' for option `double'.\MessageBreak
      Known values are:\MessageBreak
      `false', `first', `last', `true'.\MessageBreak
      Now `false' is used%
    }%
    \chardef\pagegrid@double\z@
  }{%
    \chardef\pagegrid@double\csname pagegrid@double@#1\endcsname\relax
  }%
}
\@namedef{pagegrid@double@false}{0}
\@namedef{pagegrid@double@first}{1}
\@namedef{pagegrid@double@last}{2}
\@namedef{pagegrid@double@true}{2}
\chardef\pagegrid@double\z@
%    \end{macrocode}
%    \begin{macrocode}
\newcommand*{\pagegridsetup}{%
  \let\pagegrid@init\pagegrid@@init
  \setkeys{pagegrid}%
}
%    \end{macrocode}
%    \begin{macrocode}
\pagegridsetup{%
  step=1mm%
}
\InputIfFileExists{pagegrid.cfg}{}%
\ProcessKeyvalOptions*\relax
\AtBeginDocument{%
  \ifdim\pagegrid@arrowlength>\z@
  \else
    \pagegrid@arrowlength=10\pagegrid@step
  \fi
}
%    \end{macrocode}
%
%    \begin{macrocode}
\def\pagegridShipoutDoubleBegin{%
  \begingroup
  \let\newlabel\@gobbletwo
  \let\zref@newlabel\@gobbletwo
  \let\@writefile\@gobbletwo
  \let\select@language\@gobble
}
\def\pagegridShipoutDoubleEnd{%
  \endgroup
}
\def\pagegrid@WriteDouble#1#2{%
  \immediate\write#1{%
    \@backslashchar csname %
    pagegridShipoutDouble#2%
    \@backslashchar endcsname%
  }%
}
\def\pagegrid@ShipoutDouble#1{%
  \begingroup
    \if@filesw
      \pagegrid@WriteDouble\@mainaux{Begin}%
      \ifx\@auxout\@partaux
        \pagegrid@WriteDouble\@partaux{Begin}%
        \def\pagegrid@temp{%
          \pagegrid@WriteDouble\@mainaux{End}%
          \pagegrid@WriteDouble\@partaux{End}%
        }%
      \else
        \def\pagegrid@temp{%
          \pagegrid@WriteDouble\@mainaux{End}%
        }%
      \fi
    \else
      \def\pagegrid@temp{}%
    \fi
    \let\protect\noexpand
    \AtBeginShipoutOriginalShipout\copy#1\relax
    \pagegrid@temp
  \endgroup
}
%    \end{macrocode}
%
%    \begin{macrocode}
\AtBeginShipout{%
  \ifdim\pagegrid@step>\z@
  \else
    \pagegrid@enablefalse
  \fi
  \ifpagegrid@enable
    \ifnum\pagegrid@double=\@ne
      \pagegrid@ShipoutDouble\AtBeginShipoutBox
    \else
      \ifnum\pagegrid@double=\tw@
        \@ifundefined{pagegrid@DoubleBox}{%
          \newbox\pagegrid@DoubleBox
        }{}%
        \setbox\pagegrid@DoubleBox=\copy\AtBeginShipoutBox
      \fi
    \fi
    \ifpagegrid@foreground
      \expandafter\AtBeginShipoutUpperLeftForeground
    \else
      \expandafter\AtBeginShipoutUpperLeft
    \fi
    {%
      \put(0,0){%
        \makebox(0,0)[lt]{%
          \begin{tikzpicture}[%
            bl/.style={},%
            br/.style={xshift=\pagegrid@width,xscale=-1},%
            tl/.style={yshift=\pagegrid@height,yscale=-1},%
            tr/.style={xshift=\pagegrid@width,%
                       yshift=\pagegrid@height,scale=-1}%
          ]%
            \useasboundingbox
              (0mm,\pagegrid@height) rectangle (0mm,\pagegrid@height);%
            \draw[%
              \pagegrid@origin@a,%
              step=\pagegrid@step,%
              style=help lines,%
              ultra thin%
            ] (0mm,0mm) grid (\pagegrid@width,\pagegrid@height);%
            \ifx\pagegrid@origin@b\@empty
            \else
              \draw[%
                \pagegrid@origin@b,%
                step=10\pagegrid@step,%
                {\pagegrid@secondcolor},%
                very thin%
              ] (0mm,0mm) grid (\pagegrid@width,\pagegrid@height);%
            \fi
            \draw[%
               \pagegrid@origin@a,%
               step=10\pagegrid@step,%
               {\pagegrid@firstcolor},%
               very thin%
            ] (0mm,0mm) grid (\pagegrid@width,\pagegrid@height);%
            \ifx\pagegrid@origin@b\@empty
            \else
              \draw[%
                \pagegrid@origin@b,%
                step=50\pagegrid@step,%
                {\pagegrid@secondcolor},%
                thick%
              ] (0mm,0mm) grid (\pagegrid@width,\pagegrid@height);%
            \fi
            \draw[%
              \pagegrid@origin@a,%
              step=50\pagegrid@step,%
              {\pagegrid@firstcolor},%
              thick%
            ] (0mm,0mm) grid (\pagegrid@width,\pagegrid@height);%
            \ifpagegrid@arrows
              \ifx\pagegrid@origin@b\@empty
              \else
                \draw[%
                  \pagegrid@origin@b,%
                  {\pagegrid@secondcolor},%
                  stroke,%
                  line width=1pt,%
                  line cap=round%
                ] (0mm,0mm) %
                -- (\pagegrid@arrowlength,\pagegrid@arrowlength) %
                   (\pagegrid@arrowlength,.5\pagegrid@arrowlength) %
                -- (\pagegrid@arrowlength,\pagegrid@arrowlength) %
                -- (.5\pagegrid@arrowlength,\pagegrid@arrowlength);%
              \fi
              \draw[%
                \pagegrid@origin@a,%
                {\pagegrid@firstcolor},%
                stroke,%
                line width=1pt,%
                line cap=round%
              ] (0mm,0mm) %
              -- (\pagegrid@arrowlength,\pagegrid@arrowlength) %
                 (\pagegrid@arrowlength,.5\pagegrid@arrowlength) %
              -- (\pagegrid@arrowlength,\pagegrid@arrowlength) %
              -- (.5\pagegrid@arrowlength,\pagegrid@arrowlength);%
            \fi
          \end{tikzpicture}%
        }%
      }%
    }%
    \ifnum\pagegrid@double=\tw@
      \pagegrid@ShipoutDouble\pagegrid@DoubleBox
    \fi
  \fi
}
%    \end{macrocode}
%
%    \begin{macrocode}
\pagegrid@AtEnd%
%</package>
%    \end{macrocode}
%
% \section{Test}
%
% \subsection{Catcode checks for loading}
%
%    \begin{macrocode}
%<*test1>
%    \end{macrocode}
%    \begin{macrocode}
\catcode`\{=1 %
\catcode`\}=2 %
\catcode`\#=6 %
\catcode`\@=11 %
\expandafter\ifx\csname count@\endcsname\relax
  \countdef\count@=255 %
\fi
\expandafter\ifx\csname @gobble\endcsname\relax
  \long\def\@gobble#1{}%
\fi
\expandafter\ifx\csname @firstofone\endcsname\relax
  \long\def\@firstofone#1{#1}%
\fi
\expandafter\ifx\csname loop\endcsname\relax
  \expandafter\@firstofone
\else
  \expandafter\@gobble
\fi
{%
  \def\loop#1\repeat{%
    \def\body{#1}%
    \iterate
  }%
  \def\iterate{%
    \body
      \let\next\iterate
    \else
      \let\next\relax
    \fi
    \next
  }%
  \let\repeat=\fi
}%
\def\RestoreCatcodes{}
\count@=0 %
\loop
  \edef\RestoreCatcodes{%
    \RestoreCatcodes
    \catcode\the\count@=\the\catcode\count@\relax
  }%
\ifnum\count@<255 %
  \advance\count@ 1 %
\repeat

\def\RangeCatcodeInvalid#1#2{%
  \count@=#1\relax
  \loop
    \catcode\count@=15 %
  \ifnum\count@<#2\relax
    \advance\count@ 1 %
  \repeat
}
\def\RangeCatcodeCheck#1#2#3{%
  \count@=#1\relax
  \loop
    \ifnum#3=\catcode\count@
    \else
      \errmessage{%
        Character \the\count@\space
        with wrong catcode \the\catcode\count@\space
        instead of \number#3%
      }%
    \fi
  \ifnum\count@<#2\relax
    \advance\count@ 1 %
  \repeat
}
\def\space{ }
\expandafter\ifx\csname LoadCommand\endcsname\relax
  \def\LoadCommand{\input pagegrid.sty\relax}%
\fi
\def\Test{%
  \RangeCatcodeInvalid{0}{47}%
  \RangeCatcodeInvalid{58}{64}%
  \RangeCatcodeInvalid{91}{96}%
  \RangeCatcodeInvalid{123}{255}%
  \catcode`\@=12 %
  \catcode`\\=0 %
  \catcode`\%=14 %
  \LoadCommand
  \RangeCatcodeCheck{0}{36}{15}%
  \RangeCatcodeCheck{37}{37}{14}%
  \RangeCatcodeCheck{38}{47}{15}%
  \RangeCatcodeCheck{48}{57}{12}%
  \RangeCatcodeCheck{58}{63}{15}%
  \RangeCatcodeCheck{64}{64}{12}%
  \RangeCatcodeCheck{65}{90}{11}%
  \RangeCatcodeCheck{91}{91}{15}%
  \RangeCatcodeCheck{92}{92}{0}%
  \RangeCatcodeCheck{93}{96}{15}%
  \RangeCatcodeCheck{97}{122}{11}%
  \RangeCatcodeCheck{123}{255}{15}%
  \RestoreCatcodes
}
\Test
\csname @@end\endcsname
\end
%    \end{macrocode}
%    \begin{macrocode}
%</test1>
%    \end{macrocode}
%
% \section{Installation}
%
% \subsection{Download}
%
% \paragraph{Package.} This package is available on
% CTAN\footnote{\CTANpkg{pagegrid}}:
% \begin{description}
% \item[\CTAN{macros/latex/contrib/oberdiek/pagegrid.dtx}] The source file.
% \item[\CTAN{macros/latex/contrib/oberdiek/pagegrid.pdf}] Documentation.
% \end{description}
%
%
% \paragraph{Bundle.} All the packages of the bundle `oberdiek'
% are also available in a TDS compliant ZIP archive. There
% the packages are already unpacked and the documentation files
% are generated. The files and directories obey the TDS standard.
% \begin{description}
% \item[\CTANinstall{install/macros/latex/contrib/oberdiek.tds.zip}]
% \end{description}
% \emph{TDS} refers to the standard ``A Directory Structure
% for \TeX\ Files'' (\CTAN{tds/tds.pdf}). Directories
% with \xfile{texmf} in their name are usually organized this way.
%
% \subsection{Bundle installation}
%
% \paragraph{Unpacking.} Unpack the \xfile{oberdiek.tds.zip} in the
% TDS tree (also known as \xfile{texmf} tree) of your choice.
% Example (linux):
% \begin{quote}
%   |unzip oberdiek.tds.zip -d ~/texmf|
% \end{quote}
%
% \paragraph{Script installation.}
% Check the directory \xfile{TDS:scripts/oberdiek/} for
% scripts that need further installation steps.
%
% \subsection{Package installation}
%
% \paragraph{Unpacking.} The \xfile{.dtx} file is a self-extracting
% \docstrip\ archive. The files are extracted by running the
% \xfile{.dtx} through \plainTeX:
% \begin{quote}
%   \verb|tex pagegrid.dtx|
% \end{quote}
%
% \paragraph{TDS.} Now the different files must be moved into
% the different directories in your installation TDS tree
% (also known as \xfile{texmf} tree):
% \begin{quote}
% \def\t{^^A
% \begin{tabular}{@{}>{\ttfamily}l@{ $\rightarrow$ }>{\ttfamily}l@{}}
%   pagegrid.sty & tex/latex/oberdiek/pagegrid.sty\\
%   pagegrid.pdf & doc/latex/oberdiek/pagegrid.pdf\\
%   test/pagegrid-test1.tex & doc/latex/oberdiek/test/pagegrid-test1.tex\\
%   pagegrid.dtx & source/latex/oberdiek/pagegrid.dtx\\
% \end{tabular}^^A
% }^^A
% \sbox0{\t}^^A
% \ifdim\wd0>\linewidth
%   \begingroup
%     \advance\linewidth by\leftmargin
%     \advance\linewidth by\rightmargin
%   \edef\x{\endgroup
%     \def\noexpand\lw{\the\linewidth}^^A
%   }\x
%   \def\lwbox{^^A
%     \leavevmode
%     \hbox to \linewidth{^^A
%       \kern-\leftmargin\relax
%       \hss
%       \usebox0
%       \hss
%       \kern-\rightmargin\relax
%     }^^A
%   }^^A
%   \ifdim\wd0>\lw
%     \sbox0{\small\t}^^A
%     \ifdim\wd0>\linewidth
%       \ifdim\wd0>\lw
%         \sbox0{\footnotesize\t}^^A
%         \ifdim\wd0>\linewidth
%           \ifdim\wd0>\lw
%             \sbox0{\scriptsize\t}^^A
%             \ifdim\wd0>\linewidth
%               \ifdim\wd0>\lw
%                 \sbox0{\tiny\t}^^A
%                 \ifdim\wd0>\linewidth
%                   \lwbox
%                 \else
%                   \usebox0
%                 \fi
%               \else
%                 \lwbox
%               \fi
%             \else
%               \usebox0
%             \fi
%           \else
%             \lwbox
%           \fi
%         \else
%           \usebox0
%         \fi
%       \else
%         \lwbox
%       \fi
%     \else
%       \usebox0
%     \fi
%   \else
%     \lwbox
%   \fi
% \else
%   \usebox0
% \fi
% \end{quote}
% If you have a \xfile{docstrip.cfg} that configures and enables \docstrip's
% TDS installing feature, then some files can already be in the right
% place, see the documentation of \docstrip.
%
% \subsection{Refresh file name databases}
%
% If your \TeX~distribution
% (\TeX\,Live, \mikTeX, \dots) relies on file name databases, you must refresh
% these. For example, \TeX\,Live\ users run \verb|texhash| or
% \verb|mktexlsr|.
%
% \subsection{Some details for the interested}
%
% \paragraph{Unpacking with \LaTeX.}
% The \xfile{.dtx} chooses its action depending on the format:
% \begin{description}
% \item[\plainTeX:] Run \docstrip\ and extract the files.
% \item[\LaTeX:] Generate the documentation.
% \end{description}
% If you insist on using \LaTeX\ for \docstrip\ (really,
% \docstrip\ does not need \LaTeX), then inform the autodetect routine
% about your intention:
% \begin{quote}
%   \verb|latex \let\install=y\input{pagegrid.dtx}|
% \end{quote}
% Do not forget to quote the argument according to the demands
% of your shell.
%
% \paragraph{Generating the documentation.}
% You can use both the \xfile{.dtx} or the \xfile{.drv} to generate
% the documentation. The process can be configured by the
% configuration file \xfile{ltxdoc.cfg}. For instance, put this
% line into this file, if you want to have A4 as paper format:
% \begin{quote}
%   \verb|\PassOptionsToClass{a4paper}{article}|
% \end{quote}
% An example follows how to generate the
% documentation with pdf\LaTeX:
% \begin{quote}
%\begin{verbatim}
%pdflatex pagegrid.dtx
%makeindex -s gind.ist pagegrid.idx
%pdflatex pagegrid.dtx
%makeindex -s gind.ist pagegrid.idx
%pdflatex pagegrid.dtx
%\end{verbatim}
% \end{quote}
%
% \section{Acknowledgement}
%
% \begin{description}
% \item[Klaus Braune:]
%  He provided the idea and the first \xpackage{tikz} code.
% \end{description}
%
% \begin{History}
%   \begin{Version}{2009/11/06 v1.0}
%   \item
%     The first version.
%   \end{Version}
%   \begin{Version}{2009/11/06 v1.1}
%   \item
%     Option \xoption{foreground} added.
%   \end{Version}
%   \begin{Version}{2009/12/02 v1.2}
%   \item
%     Color options, arrow options added.
%   \item
%     Names for origin options changed.
%   \end{Version}
%   \begin{Version}{2009/12/03 v1.3}
%   \item
%     Option \xoption{double} added.
%   \item
%     First CTAN release.
%   \end{Version}
%   \begin{Version}{2009/12/04 v1.4}
%   \item
%     Option \xoption{double}: Some unwanted side effects removed.
%   \end{Version}
%   \begin{Version}{2016/05/16 v1.5}
%   \item
%     Documentation updates.
%   \end{Version}
% \end{History}
%
% \PrintIndex
%
% \Finale
\endinput

%        (quote the arguments according to the demands of your shell)
%
% Documentation:
%    (a) If pagegrid.drv is present:
%           latex pagegrid.drv
%    (b) Without pagegrid.drv:
%           latex pagegrid.dtx; ...
%    The class ltxdoc loads the configuration file ltxdoc.cfg
%    if available. Here you can specify further options, e.g.
%    use A4 as paper format:
%       \PassOptionsToClass{a4paper}{article}
%
%    Programm calls to get the documentation (example):
%       pdflatex pagegrid.dtx
%       makeindex -s gind.ist pagegrid.idx
%       pdflatex pagegrid.dtx
%       makeindex -s gind.ist pagegrid.idx
%       pdflatex pagegrid.dtx
%
% Installation:
%    TDS:tex/latex/oberdiek/pagegrid.sty
%    TDS:doc/latex/oberdiek/pagegrid.pdf
%    TDS:doc/latex/oberdiek/test/pagegrid-test1.tex
%    TDS:source/latex/oberdiek/pagegrid.dtx
%
%<*ignore>
\begingroup
  \catcode123=1 %
  \catcode125=2 %
  \def\x{LaTeX2e}%
\expandafter\endgroup
\ifcase 0\ifx\install y1\fi\expandafter
         \ifx\csname processbatchFile\endcsname\relax\else1\fi
         \ifx\fmtname\x\else 1\fi\relax
\else\csname fi\endcsname
%</ignore>
%<*install>
\input docstrip.tex
\Msg{************************************************************************}
\Msg{* Installation}
\Msg{* Package: pagegrid 2016/05/16 v1.5 Print page grid in background (HO)}
\Msg{************************************************************************}

\keepsilent
\askforoverwritefalse

\let\MetaPrefix\relax
\preamble

This is a generated file.

Project: pagegrid
Version: 2016/05/16 v1.5

Copyright (C) 2009 by
   Heiko Oberdiek <heiko.oberdiek at googlemail.com>

This work may be distributed and/or modified under the
conditions of the LaTeX Project Public License, either
version 1.3c of this license or (at your option) any later
version. This version of this license is in
   https://www.latex-project.org/lppl/lppl-1-3c.txt
and the latest version of this license is in
   https://www.latex-project.org/lppl.txt
and version 1.3 or later is part of all distributions of
LaTeX version 2005/12/01 or later.

This work has the LPPL maintenance status "maintained".

The Current Maintainers of this work are
Heiko Oberdiek and the Oberdiek Package Support Group
https://github.com/ho-tex/oberdiek/issues


This work consists of the main source file pagegrid.dtx
and the derived files
   pagegrid.sty, pagegrid.pdf, pagegrid.ins, pagegrid.drv,
   pagegrid-test1.tex.

\endpreamble
\let\MetaPrefix\DoubleperCent

\generate{%
  \file{pagegrid.ins}{\from{pagegrid.dtx}{install}}%
  \file{pagegrid.drv}{\from{pagegrid.dtx}{driver}}%
  \usedir{tex/latex/oberdiek}%
  \file{pagegrid.sty}{\from{pagegrid.dtx}{package}}%
%  \usedir{doc/latex/oberdiek/test}%
%  \file{pagegrid-test1.tex}{\from{pagegrid.dtx}{test1}}%
  \nopreamble
  \nopostamble
%  \usedir{source/latex/oberdiek/catalogue}%
%  \file{pagegrid.xml}{\from{pagegrid.dtx}{catalogue}}%
}

\catcode32=13\relax% active space
\let =\space%
\Msg{************************************************************************}
\Msg{*}
\Msg{* To finish the installation you have to move the following}
\Msg{* file into a directory searched by TeX:}
\Msg{*}
\Msg{*     pagegrid.sty}
\Msg{*}
\Msg{* To produce the documentation run the file `pagegrid.drv'}
\Msg{* through LaTeX.}
\Msg{*}
\Msg{* Happy TeXing!}
\Msg{*}
\Msg{************************************************************************}

\endbatchfile
%</install>
%<*ignore>
\fi
%</ignore>
%<*driver>
\NeedsTeXFormat{LaTeX2e}
\ProvidesFile{pagegrid.drv}%
  [2016/05/16 v1.5 Print page grid in background (HO)]%
\documentclass{ltxdoc}
\usepackage{holtxdoc}[2011/11/22]
\begin{document}
  \DocInput{pagegrid.dtx}%
\end{document}
%</driver>
% \fi
%
%
% \CharacterTable
%  {Upper-case    \A\B\C\D\E\F\G\H\I\J\K\L\M\N\O\P\Q\R\S\T\U\V\W\X\Y\Z
%   Lower-case    \a\b\c\d\e\f\g\h\i\j\k\l\m\n\o\p\q\r\s\t\u\v\w\x\y\z
%   Digits        \0\1\2\3\4\5\6\7\8\9
%   Exclamation   \!     Double quote  \"     Hash (number) \#
%   Dollar        \$     Percent       \%     Ampersand     \&
%   Acute accent  \'     Left paren    \(     Right paren   \)
%   Asterisk      \*     Plus          \+     Comma         \,
%   Minus         \-     Point         \.     Solidus       \/
%   Colon         \:     Semicolon     \;     Less than     \<
%   Equals        \=     Greater than  \>     Question mark \?
%   Commercial at \@     Left bracket  \[     Backslash     \\
%   Right bracket \]     Circumflex    \^     Underscore    \_
%   Grave accent  \`     Left brace    \{     Vertical bar  \|
%   Right brace   \}     Tilde         \~}
%
% \GetFileInfo{pagegrid.drv}
%
% \title{The \xpackage{pagegrid} package}
% \date{2016/05/16 v1.5}
% \author{Heiko Oberdiek\thanks
% {Please report any issues at \url{https://github.com/ho-tex/oberdiek/issues}}}
%
% \maketitle
%
% \begin{abstract}
% The \LaTeX\ package prints a page grid in the background.
% \end{abstract}
%
% \tableofcontents
%
% \section{Documentation}
%
% The package puts a grid on the paper. It was written for
% developers of a class or package
% who have to put elements on definite locations on a page
% (e.g. letter class). The grid allows a faster optical check,
% whether the positions are correct. If the previewer already
% offers features for measuring, the package might be obsolete.
% Otherwise it saves the developer from printing the page and
% measuring by hand.
%
% \subsection{Options}
%
% Options are evaluated in the following order:
% \begin{enumerate}
% \item
%  Configuration file \xfile{pagegrid.cfg} using \cs{pagegridsetup}
%  if the file exists.
%  \item
%  Package options given for \cs{usepackage}.
%  \item
%  Later calls of \cs{pagegridsetup}.
% \end{enumerate}
% \begin{declcs}{pagegridsetup}\M{option list}
% \end{declcs}
% The options are key value options. Boolean options are enabled by
% default (without value) or by using the explicit value \texttt{true}.
% Value \texttt{false} disable the option.
%
% \subsubsection{Options \xoption{enable}, \xoption{disable}}
%
% \begin{description}
% \item[\xoption{enable}:] This boolean option controls whether the page grid
%   is drawn. As default the page grid drawing is activated.
% \item[\xoption{disable}:] It is the opposite
%   of option \xoption{enable}. It was added for convenience and
%   allows the abbreviation \texttt{disable} for \texttt{enable=false}.
% \end{description}
%
% \subsubsection{Grid origins}
%
% The package supports up to two grids on a page allowing
% measurement from opposite directions. As default two grids are drawn,
% the first from bottom left to top right. The origin of the second
% grid is at the opposite top right corner.
% The origins are controlled by the following options.
% The number of grids (one or two) depend on the number of these options
% in one call of \cs{pagegridsetup}.
% The following frame shows a paper and in its corners are the
% corresponding options. At the left and right side alias names
% are given for the options inside the paper.
% \begin{quote}
% \begin{tabular}{@{}r|@{\,}l@{\qquad}r@{\,}|l@{}}
% \cline{2-3}
% \xoption{left-top}, \xoption{lt}, \xoption{top-left}
% & \vphantom{\"U}\xoption{tl} & \xoption{tr}
% & \xoption{top-right}, \xoption{rt}, \xoption{right-top}\\
% &&&\\
% \xoption{left-bottom}, \xoption{lb}, \xoption{bottom-left}
% & \xoption{bl} & \xoption{br}
% & \xoption{bottom-right}, \xoption{rb}, \xoption{right-bottom}\\
% \cline{2-3}
% \end{tabular}
% \end{quote}
% Examples:
% \begin{quote}
% |\pagegridsetup{bl,tr}|
% \end{quote}
% This is the default setting with two grids as described previously.
% The following setups one grid only. Its origin is the upper left
% corner:
% \begin{quote}
% |\pagegridsetup{top-left}|
% \end{quote}
%
% \subsubsection{Grid unit}
%
% \begin{description}
% \item[\xoption{step}] This option takes a length and
% setups the unit for the grid. The page width and page height
% should be multiples of this unit.
% Currently the default is \texttt{1mm}. But this might change
% later by a heuristic based on the paper size.
% \end{description}
%
% \subsubsection{Color options}
%
% The basic grid lines are drawn as ultra thin help lines and is only
% drawn for the first grid.
% Each tenth and fiftyth line of the basic net is drawn thicker in a special
% color for the two grids.
% \begin{description}
% \item[\xoption{firstcolor}:] Color for the thicker lines and the arrows
% of the first grid. Default value is \texttt{red}.
% \item[\xoption{secondcolor}:] Color for the thicker lines and the arrows
% of the second grid. Default value is \texttt{blue}.
% \end{description}
% Use a color specification that package \xpackage{tikz} understands.
% (The grid is drawn with \xpackage{pgf}/\xpackage{tikz}.)
%
% \subsubsection{Arrow options}
%
% Arrows are put at the origin at the grid to show the grid start
% and the direction of the grid.
% \begin{description}
% \item[\xoption{arrows}:] This boolean option turns the arrows on or off.
% As default arrows are enabled.
% \item[\xoption{arrowlength}:] The length given as value is the
% length of the edge of a square at the origin within the
% arrow is put as diagonal. Default is 10 times the grid unit (10\,mm).
% The real arrow length is this length multiplied by $\sqrt2$.
% \end{description}
%
% \subsubsection{Miscellaneous options}
%
% \begin{description}
% \item[\xoption{double}:] The output page is doubled, one without page
% grid and the other with page grid. Possible values are shown in the
% following table:
% \begin{quote}
% \begin{tabular}{ll}
% Option & Meaning\\
% \hline
% |false| & Turns option off.\\
% |first| & Grid page comes first.\\
% |last| & Grid page comes after the page without grid.\\
% |true| & Same as |last|.\\
% \meta{no value} & Same as |true|.\\
% \end{tabular}
% \end{quote}
% \textbf{Note:}
% The double output of the page has side effects.
% All whatits are executed twice, for example: file writing
% and anchor setting. Some unwanted actions are catched such
% as multiple \cs{label} definitions, duplicate entries in
% the table of contents. For bookmarks, use package \xpackage{bookmarks}.
% \item[\xoption{foreground}:] Boolean option, default is \texttt{false}.
% Sometimes there might be elements on the page (e.g. large images)
% that hide the grid. Then option \xoption{foreground} puts the grids
% over the current output page.
% \end{description}
%
% \StopEventually{
% }
%
% \section{Implementation}
%    \begin{macrocode}
%<*package>
%    \end{macrocode}
%    Reload check, especially if the package is not used with \LaTeX.
%    \begin{macrocode}
\begingroup\catcode61\catcode48\catcode32=10\relax%
  \catcode13=5 % ^^M
  \endlinechar=13 %
  \catcode35=6 % #
  \catcode39=12 % '
  \catcode44=12 % ,
  \catcode45=12 % -
  \catcode46=12 % .
  \catcode58=12 % :
  \catcode64=11 % @
  \catcode123=1 % {
  \catcode125=2 % }
  \expandafter\let\expandafter\x\csname ver@pagegrid.sty\endcsname
  \ifx\x\relax % plain-TeX, first loading
  \else
    \def\empty{}%
    \ifx\x\empty % LaTeX, first loading,
      % variable is initialized, but \ProvidesPackage not yet seen
    \else
      \expandafter\ifx\csname PackageInfo\endcsname\relax
        \def\x#1#2{%
          \immediate\write-1{Package #1 Info: #2.}%
        }%
      \else
        \def\x#1#2{\PackageInfo{#1}{#2, stopped}}%
      \fi
      \x{pagegrid}{The package is already loaded}%
      \aftergroup\endinput
    \fi
  \fi
\endgroup%
%    \end{macrocode}
%    Package identification:
%    \begin{macrocode}
\begingroup\catcode61\catcode48\catcode32=10\relax%
  \catcode13=5 % ^^M
  \endlinechar=13 %
  \catcode35=6 % #
  \catcode39=12 % '
  \catcode40=12 % (
  \catcode41=12 % )
  \catcode44=12 % ,
  \catcode45=12 % -
  \catcode46=12 % .
  \catcode47=12 % /
  \catcode58=12 % :
  \catcode64=11 % @
  \catcode91=12 % [
  \catcode93=12 % ]
  \catcode123=1 % {
  \catcode125=2 % }
  \expandafter\ifx\csname ProvidesPackage\endcsname\relax
    \def\x#1#2#3[#4]{\endgroup
      \immediate\write-1{Package: #3 #4}%
      \xdef#1{#4}%
    }%
  \else
    \def\x#1#2[#3]{\endgroup
      #2[{#3}]%
      \ifx#1\@undefined
        \xdef#1{#3}%
      \fi
      \ifx#1\relax
        \xdef#1{#3}%
      \fi
    }%
  \fi
\expandafter\x\csname ver@pagegrid.sty\endcsname
\ProvidesPackage{pagegrid}%
  [2016/05/16 v1.5 Print page grid in background (HO)]%
%    \end{macrocode}
%
%    \begin{macrocode}
\begingroup\catcode61\catcode48\catcode32=10\relax%
  \catcode13=5 % ^^M
  \endlinechar=13 %
  \catcode123=1 % {
  \catcode125=2 % }
  \catcode64=11 % @
  \def\x{\endgroup
    \expandafter\edef\csname pagegrid@AtEnd\endcsname{%
      \endlinechar=\the\endlinechar\relax
      \catcode13=\the\catcode13\relax
      \catcode32=\the\catcode32\relax
      \catcode35=\the\catcode35\relax
      \catcode61=\the\catcode61\relax
      \catcode64=\the\catcode64\relax
      \catcode123=\the\catcode123\relax
      \catcode125=\the\catcode125\relax
    }%
  }%
\x\catcode61\catcode48\catcode32=10\relax%
\catcode13=5 % ^^M
\endlinechar=13 %
\catcode35=6 % #
\catcode64=11 % @
\catcode123=1 % {
\catcode125=2 % }
\def\TMP@EnsureCode#1#2{%
  \edef\pagegrid@AtEnd{%
    \pagegrid@AtEnd
    \catcode#1=\the\catcode#1\relax
  }%
  \catcode#1=#2\relax
}
\TMP@EnsureCode{9}{10}% (tab)
\TMP@EnsureCode{10}{12}% ^^J
\TMP@EnsureCode{33}{12}% !
\TMP@EnsureCode{34}{12}% "
\TMP@EnsureCode{36}{3}% $
\TMP@EnsureCode{38}{4}% &
\TMP@EnsureCode{39}{12}% '
\TMP@EnsureCode{40}{12}% (
\TMP@EnsureCode{41}{12}% )
\TMP@EnsureCode{42}{12}% *
\TMP@EnsureCode{43}{12}% +
\TMP@EnsureCode{44}{12}% ,
\TMP@EnsureCode{45}{12}% -
\TMP@EnsureCode{46}{12}% .
\TMP@EnsureCode{47}{12}% /
\TMP@EnsureCode{58}{12}% :
\TMP@EnsureCode{59}{12}% ;
\TMP@EnsureCode{60}{12}% <
\TMP@EnsureCode{62}{12}% >
\TMP@EnsureCode{63}{12}% ?
\TMP@EnsureCode{91}{12}% [
\TMP@EnsureCode{93}{12}% ]
\TMP@EnsureCode{94}{7}% ^ (superscript)
\TMP@EnsureCode{95}{8}% _ (subscript)
\TMP@EnsureCode{96}{12}% `
\TMP@EnsureCode{124}{12}% |
\edef\pagegrid@AtEnd{\pagegrid@AtEnd\noexpand\endinput}
%    \end{macrocode}
%
%    \begin{macrocode}
\RequirePackage{tikz}
\RequirePackage{atbegshi}[2009/12/02]
\RequirePackage{kvoptions}[2009/07/17]
%    \end{macrocode}
%    \begin{macrocode}
\begingroup\expandafter\expandafter\expandafter\endgroup
\expandafter\ifx\csname stockwidth\endcsname\relax
  \def\pagegrid@width{\paperwidth}%
  \def\pagegrid@height{\paperheight}%
\else
  \def\pagegrid@width{\stockwidth}%
  \def\pagegrid@height{\stockheight}%
\fi
%    \end{macrocode}
%
%    \begin{macrocode}
\SetupKeyvalOptions{%
  family=pagegrid,%
  prefix=pagegrid@,%
}
\def\pagegrid@init{%
  \let\pagegrid@origin@a\@empty
  \let\pagegrid@origin@b\@empty
  \let\pagegrid@init\relax
}
\let\pagegrid@@init\pagegrid@init
\def\pagegrid@origin@a{bl}
\def\pagegrid@origin@b{tr}
\def\pagegrid@SetOrigin#1{%
  \pagegrid@init
  \ifx\pagegrid@origin@a\@empty
    \def\pagegrid@origin@a{#1}%
  \else
    \ifx\pagegrid@origin@b\@empty
    \else
      \let\pagegrid@origin@a\pagegrid@origin@b
    \fi
    \def\pagegrid@origin@b{#1}%
  \fi
}
\def\pagegrid@temp#1{%
  \DeclareVoidOption{#1}{\pagegrid@SetOrigin{#1}}%
  \@namedef{pagegrid@N@#1}{#1}%
}
\pagegrid@temp{bl}
\pagegrid@temp{br}
\pagegrid@temp{tl}
\pagegrid@temp{tr}
\def\pagegrid@temp#1#2{%
  \DeclareVoidOption{#2}{\pagegrid@SetOrigin{#1}}%
}%
\pagegrid@temp{bl}{lb}
\pagegrid@temp{br}{rb}
\pagegrid@temp{tl}{lt}
\pagegrid@temp{tr}{rt}
\pagegrid@temp{bl}{bottom-left}
\pagegrid@temp{br}{bottom-right}
\pagegrid@temp{tl}{top-left}
\pagegrid@temp{tr}{top-right}
\pagegrid@temp{bl}{left-bottom}
\pagegrid@temp{br}{right-bottom}
\pagegrid@temp{tl}{left-top}
\pagegrid@temp{tr}{right-top}
%    \end{macrocode}
%    \begin{macrocode}
\DeclareBoolOption[true]{enable}
\DeclareComplementaryOption{disable}{enable}
%    \end{macrocode}
%    \begin{macrocode}
\DeclareBoolOption{foreground}
%    \end{macrocode}
%    \begin{macrocode}
\newlength{\pagegrid@step}
\define@key{pagegrid}{step}{%
  \setlength{\pagegrid@step}{#1}%
}
%    \end{macrocode}
%    \begin{macrocode}
\DeclareStringOption[red]{firstcolor}
\DeclareStringOption[blue]{secondcolor}
%    \end{macrocode}
%    \begin{macrocode}
\DeclareBoolOption[true]{arrows}
\newlength\pagegrid@arrowlength
\pagegrid@arrowlength=\z@
\define@key{pagegrid}{arrowlength}{%
  \setlength{\pagegrid@arrowlength}{#1}%
}
%    \end{macrocode}
%    \begin{macrocode}
\define@key{pagegrid}{double}[true]{%
  \@ifundefined{pagegrid@double@#1}{%
    \PackageWarning{pagegrid}{%
      Unsupported value `#1' for option `double'.\MessageBreak
      Known values are:\MessageBreak
      `false', `first', `last', `true'.\MessageBreak
      Now `false' is used%
    }%
    \chardef\pagegrid@double\z@
  }{%
    \chardef\pagegrid@double\csname pagegrid@double@#1\endcsname\relax
  }%
}
\@namedef{pagegrid@double@false}{0}
\@namedef{pagegrid@double@first}{1}
\@namedef{pagegrid@double@last}{2}
\@namedef{pagegrid@double@true}{2}
\chardef\pagegrid@double\z@
%    \end{macrocode}
%    \begin{macrocode}
\newcommand*{\pagegridsetup}{%
  \let\pagegrid@init\pagegrid@@init
  \setkeys{pagegrid}%
}
%    \end{macrocode}
%    \begin{macrocode}
\pagegridsetup{%
  step=1mm%
}
\InputIfFileExists{pagegrid.cfg}{}%
\ProcessKeyvalOptions*\relax
\AtBeginDocument{%
  \ifdim\pagegrid@arrowlength>\z@
  \else
    \pagegrid@arrowlength=10\pagegrid@step
  \fi
}
%    \end{macrocode}
%
%    \begin{macrocode}
\def\pagegridShipoutDoubleBegin{%
  \begingroup
  \let\newlabel\@gobbletwo
  \let\zref@newlabel\@gobbletwo
  \let\@writefile\@gobbletwo
  \let\select@language\@gobble
}
\def\pagegridShipoutDoubleEnd{%
  \endgroup
}
\def\pagegrid@WriteDouble#1#2{%
  \immediate\write#1{%
    \@backslashchar csname %
    pagegridShipoutDouble#2%
    \@backslashchar endcsname%
  }%
}
\def\pagegrid@ShipoutDouble#1{%
  \begingroup
    \if@filesw
      \pagegrid@WriteDouble\@mainaux{Begin}%
      \ifx\@auxout\@partaux
        \pagegrid@WriteDouble\@partaux{Begin}%
        \def\pagegrid@temp{%
          \pagegrid@WriteDouble\@mainaux{End}%
          \pagegrid@WriteDouble\@partaux{End}%
        }%
      \else
        \def\pagegrid@temp{%
          \pagegrid@WriteDouble\@mainaux{End}%
        }%
      \fi
    \else
      \def\pagegrid@temp{}%
    \fi
    \let\protect\noexpand
    \AtBeginShipoutOriginalShipout\copy#1\relax
    \pagegrid@temp
  \endgroup
}
%    \end{macrocode}
%
%    \begin{macrocode}
\AtBeginShipout{%
  \ifdim\pagegrid@step>\z@
  \else
    \pagegrid@enablefalse
  \fi
  \ifpagegrid@enable
    \ifnum\pagegrid@double=\@ne
      \pagegrid@ShipoutDouble\AtBeginShipoutBox
    \else
      \ifnum\pagegrid@double=\tw@
        \@ifundefined{pagegrid@DoubleBox}{%
          \newbox\pagegrid@DoubleBox
        }{}%
        \setbox\pagegrid@DoubleBox=\copy\AtBeginShipoutBox
      \fi
    \fi
    \ifpagegrid@foreground
      \expandafter\AtBeginShipoutUpperLeftForeground
    \else
      \expandafter\AtBeginShipoutUpperLeft
    \fi
    {%
      \put(0,0){%
        \makebox(0,0)[lt]{%
          \begin{tikzpicture}[%
            bl/.style={},%
            br/.style={xshift=\pagegrid@width,xscale=-1},%
            tl/.style={yshift=\pagegrid@height,yscale=-1},%
            tr/.style={xshift=\pagegrid@width,%
                       yshift=\pagegrid@height,scale=-1}%
          ]%
            \useasboundingbox
              (0mm,\pagegrid@height) rectangle (0mm,\pagegrid@height);%
            \draw[%
              \pagegrid@origin@a,%
              step=\pagegrid@step,%
              style=help lines,%
              ultra thin%
            ] (0mm,0mm) grid (\pagegrid@width,\pagegrid@height);%
            \ifx\pagegrid@origin@b\@empty
            \else
              \draw[%
                \pagegrid@origin@b,%
                step=10\pagegrid@step,%
                {\pagegrid@secondcolor},%
                very thin%
              ] (0mm,0mm) grid (\pagegrid@width,\pagegrid@height);%
            \fi
            \draw[%
               \pagegrid@origin@a,%
               step=10\pagegrid@step,%
               {\pagegrid@firstcolor},%
               very thin%
            ] (0mm,0mm) grid (\pagegrid@width,\pagegrid@height);%
            \ifx\pagegrid@origin@b\@empty
            \else
              \draw[%
                \pagegrid@origin@b,%
                step=50\pagegrid@step,%
                {\pagegrid@secondcolor},%
                thick%
              ] (0mm,0mm) grid (\pagegrid@width,\pagegrid@height);%
            \fi
            \draw[%
              \pagegrid@origin@a,%
              step=50\pagegrid@step,%
              {\pagegrid@firstcolor},%
              thick%
            ] (0mm,0mm) grid (\pagegrid@width,\pagegrid@height);%
            \ifpagegrid@arrows
              \ifx\pagegrid@origin@b\@empty
              \else
                \draw[%
                  \pagegrid@origin@b,%
                  {\pagegrid@secondcolor},%
                  stroke,%
                  line width=1pt,%
                  line cap=round%
                ] (0mm,0mm) %
                -- (\pagegrid@arrowlength,\pagegrid@arrowlength) %
                   (\pagegrid@arrowlength,.5\pagegrid@arrowlength) %
                -- (\pagegrid@arrowlength,\pagegrid@arrowlength) %
                -- (.5\pagegrid@arrowlength,\pagegrid@arrowlength);%
              \fi
              \draw[%
                \pagegrid@origin@a,%
                {\pagegrid@firstcolor},%
                stroke,%
                line width=1pt,%
                line cap=round%
              ] (0mm,0mm) %
              -- (\pagegrid@arrowlength,\pagegrid@arrowlength) %
                 (\pagegrid@arrowlength,.5\pagegrid@arrowlength) %
              -- (\pagegrid@arrowlength,\pagegrid@arrowlength) %
              -- (.5\pagegrid@arrowlength,\pagegrid@arrowlength);%
            \fi
          \end{tikzpicture}%
        }%
      }%
    }%
    \ifnum\pagegrid@double=\tw@
      \pagegrid@ShipoutDouble\pagegrid@DoubleBox
    \fi
  \fi
}
%    \end{macrocode}
%
%    \begin{macrocode}
\pagegrid@AtEnd%
%</package>
%    \end{macrocode}
%
% \section{Test}
%
% \subsection{Catcode checks for loading}
%
%    \begin{macrocode}
%<*test1>
%    \end{macrocode}
%    \begin{macrocode}
\catcode`\{=1 %
\catcode`\}=2 %
\catcode`\#=6 %
\catcode`\@=11 %
\expandafter\ifx\csname count@\endcsname\relax
  \countdef\count@=255 %
\fi
\expandafter\ifx\csname @gobble\endcsname\relax
  \long\def\@gobble#1{}%
\fi
\expandafter\ifx\csname @firstofone\endcsname\relax
  \long\def\@firstofone#1{#1}%
\fi
\expandafter\ifx\csname loop\endcsname\relax
  \expandafter\@firstofone
\else
  \expandafter\@gobble
\fi
{%
  \def\loop#1\repeat{%
    \def\body{#1}%
    \iterate
  }%
  \def\iterate{%
    \body
      \let\next\iterate
    \else
      \let\next\relax
    \fi
    \next
  }%
  \let\repeat=\fi
}%
\def\RestoreCatcodes{}
\count@=0 %
\loop
  \edef\RestoreCatcodes{%
    \RestoreCatcodes
    \catcode\the\count@=\the\catcode\count@\relax
  }%
\ifnum\count@<255 %
  \advance\count@ 1 %
\repeat

\def\RangeCatcodeInvalid#1#2{%
  \count@=#1\relax
  \loop
    \catcode\count@=15 %
  \ifnum\count@<#2\relax
    \advance\count@ 1 %
  \repeat
}
\def\RangeCatcodeCheck#1#2#3{%
  \count@=#1\relax
  \loop
    \ifnum#3=\catcode\count@
    \else
      \errmessage{%
        Character \the\count@\space
        with wrong catcode \the\catcode\count@\space
        instead of \number#3%
      }%
    \fi
  \ifnum\count@<#2\relax
    \advance\count@ 1 %
  \repeat
}
\def\space{ }
\expandafter\ifx\csname LoadCommand\endcsname\relax
  \def\LoadCommand{\input pagegrid.sty\relax}%
\fi
\def\Test{%
  \RangeCatcodeInvalid{0}{47}%
  \RangeCatcodeInvalid{58}{64}%
  \RangeCatcodeInvalid{91}{96}%
  \RangeCatcodeInvalid{123}{255}%
  \catcode`\@=12 %
  \catcode`\\=0 %
  \catcode`\%=14 %
  \LoadCommand
  \RangeCatcodeCheck{0}{36}{15}%
  \RangeCatcodeCheck{37}{37}{14}%
  \RangeCatcodeCheck{38}{47}{15}%
  \RangeCatcodeCheck{48}{57}{12}%
  \RangeCatcodeCheck{58}{63}{15}%
  \RangeCatcodeCheck{64}{64}{12}%
  \RangeCatcodeCheck{65}{90}{11}%
  \RangeCatcodeCheck{91}{91}{15}%
  \RangeCatcodeCheck{92}{92}{0}%
  \RangeCatcodeCheck{93}{96}{15}%
  \RangeCatcodeCheck{97}{122}{11}%
  \RangeCatcodeCheck{123}{255}{15}%
  \RestoreCatcodes
}
\Test
\csname @@end\endcsname
\end
%    \end{macrocode}
%    \begin{macrocode}
%</test1>
%    \end{macrocode}
%
% \section{Installation}
%
% \subsection{Download}
%
% \paragraph{Package.} This package is available on
% CTAN\footnote{\CTANpkg{pagegrid}}:
% \begin{description}
% \item[\CTAN{macros/latex/contrib/oberdiek/pagegrid.dtx}] The source file.
% \item[\CTAN{macros/latex/contrib/oberdiek/pagegrid.pdf}] Documentation.
% \end{description}
%
%
% \paragraph{Bundle.} All the packages of the bundle `oberdiek'
% are also available in a TDS compliant ZIP archive. There
% the packages are already unpacked and the documentation files
% are generated. The files and directories obey the TDS standard.
% \begin{description}
% \item[\CTANinstall{install/macros/latex/contrib/oberdiek.tds.zip}]
% \end{description}
% \emph{TDS} refers to the standard ``A Directory Structure
% for \TeX\ Files'' (\CTAN{tds/tds.pdf}). Directories
% with \xfile{texmf} in their name are usually organized this way.
%
% \subsection{Bundle installation}
%
% \paragraph{Unpacking.} Unpack the \xfile{oberdiek.tds.zip} in the
% TDS tree (also known as \xfile{texmf} tree) of your choice.
% Example (linux):
% \begin{quote}
%   |unzip oberdiek.tds.zip -d ~/texmf|
% \end{quote}
%
% \paragraph{Script installation.}
% Check the directory \xfile{TDS:scripts/oberdiek/} for
% scripts that need further installation steps.
%
% \subsection{Package installation}
%
% \paragraph{Unpacking.} The \xfile{.dtx} file is a self-extracting
% \docstrip\ archive. The files are extracted by running the
% \xfile{.dtx} through \plainTeX:
% \begin{quote}
%   \verb|tex pagegrid.dtx|
% \end{quote}
%
% \paragraph{TDS.} Now the different files must be moved into
% the different directories in your installation TDS tree
% (also known as \xfile{texmf} tree):
% \begin{quote}
% \def\t{^^A
% \begin{tabular}{@{}>{\ttfamily}l@{ $\rightarrow$ }>{\ttfamily}l@{}}
%   pagegrid.sty & tex/latex/oberdiek/pagegrid.sty\\
%   pagegrid.pdf & doc/latex/oberdiek/pagegrid.pdf\\
%   test/pagegrid-test1.tex & doc/latex/oberdiek/test/pagegrid-test1.tex\\
%   pagegrid.dtx & source/latex/oberdiek/pagegrid.dtx\\
% \end{tabular}^^A
% }^^A
% \sbox0{\t}^^A
% \ifdim\wd0>\linewidth
%   \begingroup
%     \advance\linewidth by\leftmargin
%     \advance\linewidth by\rightmargin
%   \edef\x{\endgroup
%     \def\noexpand\lw{\the\linewidth}^^A
%   }\x
%   \def\lwbox{^^A
%     \leavevmode
%     \hbox to \linewidth{^^A
%       \kern-\leftmargin\relax
%       \hss
%       \usebox0
%       \hss
%       \kern-\rightmargin\relax
%     }^^A
%   }^^A
%   \ifdim\wd0>\lw
%     \sbox0{\small\t}^^A
%     \ifdim\wd0>\linewidth
%       \ifdim\wd0>\lw
%         \sbox0{\footnotesize\t}^^A
%         \ifdim\wd0>\linewidth
%           \ifdim\wd0>\lw
%             \sbox0{\scriptsize\t}^^A
%             \ifdim\wd0>\linewidth
%               \ifdim\wd0>\lw
%                 \sbox0{\tiny\t}^^A
%                 \ifdim\wd0>\linewidth
%                   \lwbox
%                 \else
%                   \usebox0
%                 \fi
%               \else
%                 \lwbox
%               \fi
%             \else
%               \usebox0
%             \fi
%           \else
%             \lwbox
%           \fi
%         \else
%           \usebox0
%         \fi
%       \else
%         \lwbox
%       \fi
%     \else
%       \usebox0
%     \fi
%   \else
%     \lwbox
%   \fi
% \else
%   \usebox0
% \fi
% \end{quote}
% If you have a \xfile{docstrip.cfg} that configures and enables \docstrip's
% TDS installing feature, then some files can already be in the right
% place, see the documentation of \docstrip.
%
% \subsection{Refresh file name databases}
%
% If your \TeX~distribution
% (\TeX\,Live, \mikTeX, \dots) relies on file name databases, you must refresh
% these. For example, \TeX\,Live\ users run \verb|texhash| or
% \verb|mktexlsr|.
%
% \subsection{Some details for the interested}
%
% \paragraph{Unpacking with \LaTeX.}
% The \xfile{.dtx} chooses its action depending on the format:
% \begin{description}
% \item[\plainTeX:] Run \docstrip\ and extract the files.
% \item[\LaTeX:] Generate the documentation.
% \end{description}
% If you insist on using \LaTeX\ for \docstrip\ (really,
% \docstrip\ does not need \LaTeX), then inform the autodetect routine
% about your intention:
% \begin{quote}
%   \verb|latex \let\install=y% \iffalse meta-comment
%
% File: pagegrid.dtx
% Version: 2016/05/16 v1.5
% Info: Print page grid in background
%
% Copyright (C) 2009 by
%    Heiko Oberdiek <heiko.oberdiek at googlemail.com>
%    2016
%    https://github.com/ho-tex/oberdiek/issues
%
% This work may be distributed and/or modified under the
% conditions of the LaTeX Project Public License, either
% version 1.3c of this license or (at your option) any later
% version. This version of this license is in
%    https://www.latex-project.org/lppl/lppl-1-3c.txt
% and the latest version of this license is in
%    https://www.latex-project.org/lppl.txt
% and version 1.3 or later is part of all distributions of
% LaTeX version 2005/12/01 or later.
%
% This work has the LPPL maintenance status "maintained".
%
% The Current Maintainers of this work are
% Heiko Oberdiek and the Oberdiek Package Support Group
% https://github.com/ho-tex/oberdiek/issues
%
% This work consists of the main source file pagegrid.dtx
% and the derived files
%    pagegrid.sty, pagegrid.pdf, pagegrid.ins, pagegrid.drv,
%    pagegrid-test1.tex.
%
% Distribution:
%    CTAN:macros/latex/contrib/oberdiek/pagegrid.dtx
%    CTAN:macros/latex/contrib/oberdiek/pagegrid.pdf
%
% Unpacking:
%    (a) If pagegrid.ins is present:
%           tex pagegrid.ins
%    (b) Without pagegrid.ins:
%           tex pagegrid.dtx
%    (c) If you insist on using LaTeX
%           latex \let\install=y\input{pagegrid.dtx}
%        (quote the arguments according to the demands of your shell)
%
% Documentation:
%    (a) If pagegrid.drv is present:
%           latex pagegrid.drv
%    (b) Without pagegrid.drv:
%           latex pagegrid.dtx; ...
%    The class ltxdoc loads the configuration file ltxdoc.cfg
%    if available. Here you can specify further options, e.g.
%    use A4 as paper format:
%       \PassOptionsToClass{a4paper}{article}
%
%    Programm calls to get the documentation (example):
%       pdflatex pagegrid.dtx
%       makeindex -s gind.ist pagegrid.idx
%       pdflatex pagegrid.dtx
%       makeindex -s gind.ist pagegrid.idx
%       pdflatex pagegrid.dtx
%
% Installation:
%    TDS:tex/latex/oberdiek/pagegrid.sty
%    TDS:doc/latex/oberdiek/pagegrid.pdf
%    TDS:doc/latex/oberdiek/test/pagegrid-test1.tex
%    TDS:source/latex/oberdiek/pagegrid.dtx
%
%<*ignore>
\begingroup
  \catcode123=1 %
  \catcode125=2 %
  \def\x{LaTeX2e}%
\expandafter\endgroup
\ifcase 0\ifx\install y1\fi\expandafter
         \ifx\csname processbatchFile\endcsname\relax\else1\fi
         \ifx\fmtname\x\else 1\fi\relax
\else\csname fi\endcsname
%</ignore>
%<*install>
\input docstrip.tex
\Msg{************************************************************************}
\Msg{* Installation}
\Msg{* Package: pagegrid 2016/05/16 v1.5 Print page grid in background (HO)}
\Msg{************************************************************************}

\keepsilent
\askforoverwritefalse

\let\MetaPrefix\relax
\preamble

This is a generated file.

Project: pagegrid
Version: 2016/05/16 v1.5

Copyright (C) 2009 by
   Heiko Oberdiek <heiko.oberdiek at googlemail.com>

This work may be distributed and/or modified under the
conditions of the LaTeX Project Public License, either
version 1.3c of this license or (at your option) any later
version. This version of this license is in
   https://www.latex-project.org/lppl/lppl-1-3c.txt
and the latest version of this license is in
   https://www.latex-project.org/lppl.txt
and version 1.3 or later is part of all distributions of
LaTeX version 2005/12/01 or later.

This work has the LPPL maintenance status "maintained".

The Current Maintainers of this work are
Heiko Oberdiek and the Oberdiek Package Support Group
https://github.com/ho-tex/oberdiek/issues


This work consists of the main source file pagegrid.dtx
and the derived files
   pagegrid.sty, pagegrid.pdf, pagegrid.ins, pagegrid.drv,
   pagegrid-test1.tex.

\endpreamble
\let\MetaPrefix\DoubleperCent

\generate{%
  \file{pagegrid.ins}{\from{pagegrid.dtx}{install}}%
  \file{pagegrid.drv}{\from{pagegrid.dtx}{driver}}%
  \usedir{tex/latex/oberdiek}%
  \file{pagegrid.sty}{\from{pagegrid.dtx}{package}}%
%  \usedir{doc/latex/oberdiek/test}%
%  \file{pagegrid-test1.tex}{\from{pagegrid.dtx}{test1}}%
  \nopreamble
  \nopostamble
%  \usedir{source/latex/oberdiek/catalogue}%
%  \file{pagegrid.xml}{\from{pagegrid.dtx}{catalogue}}%
}

\catcode32=13\relax% active space
\let =\space%
\Msg{************************************************************************}
\Msg{*}
\Msg{* To finish the installation you have to move the following}
\Msg{* file into a directory searched by TeX:}
\Msg{*}
\Msg{*     pagegrid.sty}
\Msg{*}
\Msg{* To produce the documentation run the file `pagegrid.drv'}
\Msg{* through LaTeX.}
\Msg{*}
\Msg{* Happy TeXing!}
\Msg{*}
\Msg{************************************************************************}

\endbatchfile
%</install>
%<*ignore>
\fi
%</ignore>
%<*driver>
\NeedsTeXFormat{LaTeX2e}
\ProvidesFile{pagegrid.drv}%
  [2016/05/16 v1.5 Print page grid in background (HO)]%
\documentclass{ltxdoc}
\usepackage{holtxdoc}[2011/11/22]
\begin{document}
  \DocInput{pagegrid.dtx}%
\end{document}
%</driver>
% \fi
%
%
% \CharacterTable
%  {Upper-case    \A\B\C\D\E\F\G\H\I\J\K\L\M\N\O\P\Q\R\S\T\U\V\W\X\Y\Z
%   Lower-case    \a\b\c\d\e\f\g\h\i\j\k\l\m\n\o\p\q\r\s\t\u\v\w\x\y\z
%   Digits        \0\1\2\3\4\5\6\7\8\9
%   Exclamation   \!     Double quote  \"     Hash (number) \#
%   Dollar        \$     Percent       \%     Ampersand     \&
%   Acute accent  \'     Left paren    \(     Right paren   \)
%   Asterisk      \*     Plus          \+     Comma         \,
%   Minus         \-     Point         \.     Solidus       \/
%   Colon         \:     Semicolon     \;     Less than     \<
%   Equals        \=     Greater than  \>     Question mark \?
%   Commercial at \@     Left bracket  \[     Backslash     \\
%   Right bracket \]     Circumflex    \^     Underscore    \_
%   Grave accent  \`     Left brace    \{     Vertical bar  \|
%   Right brace   \}     Tilde         \~}
%
% \GetFileInfo{pagegrid.drv}
%
% \title{The \xpackage{pagegrid} package}
% \date{2016/05/16 v1.5}
% \author{Heiko Oberdiek\thanks
% {Please report any issues at \url{https://github.com/ho-tex/oberdiek/issues}}}
%
% \maketitle
%
% \begin{abstract}
% The \LaTeX\ package prints a page grid in the background.
% \end{abstract}
%
% \tableofcontents
%
% \section{Documentation}
%
% The package puts a grid on the paper. It was written for
% developers of a class or package
% who have to put elements on definite locations on a page
% (e.g. letter class). The grid allows a faster optical check,
% whether the positions are correct. If the previewer already
% offers features for measuring, the package might be obsolete.
% Otherwise it saves the developer from printing the page and
% measuring by hand.
%
% \subsection{Options}
%
% Options are evaluated in the following order:
% \begin{enumerate}
% \item
%  Configuration file \xfile{pagegrid.cfg} using \cs{pagegridsetup}
%  if the file exists.
%  \item
%  Package options given for \cs{usepackage}.
%  \item
%  Later calls of \cs{pagegridsetup}.
% \end{enumerate}
% \begin{declcs}{pagegridsetup}\M{option list}
% \end{declcs}
% The options are key value options. Boolean options are enabled by
% default (without value) or by using the explicit value \texttt{true}.
% Value \texttt{false} disable the option.
%
% \subsubsection{Options \xoption{enable}, \xoption{disable}}
%
% \begin{description}
% \item[\xoption{enable}:] This boolean option controls whether the page grid
%   is drawn. As default the page grid drawing is activated.
% \item[\xoption{disable}:] It is the opposite
%   of option \xoption{enable}. It was added for convenience and
%   allows the abbreviation \texttt{disable} for \texttt{enable=false}.
% \end{description}
%
% \subsubsection{Grid origins}
%
% The package supports up to two grids on a page allowing
% measurement from opposite directions. As default two grids are drawn,
% the first from bottom left to top right. The origin of the second
% grid is at the opposite top right corner.
% The origins are controlled by the following options.
% The number of grids (one or two) depend on the number of these options
% in one call of \cs{pagegridsetup}.
% The following frame shows a paper and in its corners are the
% corresponding options. At the left and right side alias names
% are given for the options inside the paper.
% \begin{quote}
% \begin{tabular}{@{}r|@{\,}l@{\qquad}r@{\,}|l@{}}
% \cline{2-3}
% \xoption{left-top}, \xoption{lt}, \xoption{top-left}
% & \vphantom{\"U}\xoption{tl} & \xoption{tr}
% & \xoption{top-right}, \xoption{rt}, \xoption{right-top}\\
% &&&\\
% \xoption{left-bottom}, \xoption{lb}, \xoption{bottom-left}
% & \xoption{bl} & \xoption{br}
% & \xoption{bottom-right}, \xoption{rb}, \xoption{right-bottom}\\
% \cline{2-3}
% \end{tabular}
% \end{quote}
% Examples:
% \begin{quote}
% |\pagegridsetup{bl,tr}|
% \end{quote}
% This is the default setting with two grids as described previously.
% The following setups one grid only. Its origin is the upper left
% corner:
% \begin{quote}
% |\pagegridsetup{top-left}|
% \end{quote}
%
% \subsubsection{Grid unit}
%
% \begin{description}
% \item[\xoption{step}] This option takes a length and
% setups the unit for the grid. The page width and page height
% should be multiples of this unit.
% Currently the default is \texttt{1mm}. But this might change
% later by a heuristic based on the paper size.
% \end{description}
%
% \subsubsection{Color options}
%
% The basic grid lines are drawn as ultra thin help lines and is only
% drawn for the first grid.
% Each tenth and fiftyth line of the basic net is drawn thicker in a special
% color for the two grids.
% \begin{description}
% \item[\xoption{firstcolor}:] Color for the thicker lines and the arrows
% of the first grid. Default value is \texttt{red}.
% \item[\xoption{secondcolor}:] Color for the thicker lines and the arrows
% of the second grid. Default value is \texttt{blue}.
% \end{description}
% Use a color specification that package \xpackage{tikz} understands.
% (The grid is drawn with \xpackage{pgf}/\xpackage{tikz}.)
%
% \subsubsection{Arrow options}
%
% Arrows are put at the origin at the grid to show the grid start
% and the direction of the grid.
% \begin{description}
% \item[\xoption{arrows}:] This boolean option turns the arrows on or off.
% As default arrows are enabled.
% \item[\xoption{arrowlength}:] The length given as value is the
% length of the edge of a square at the origin within the
% arrow is put as diagonal. Default is 10 times the grid unit (10\,mm).
% The real arrow length is this length multiplied by $\sqrt2$.
% \end{description}
%
% \subsubsection{Miscellaneous options}
%
% \begin{description}
% \item[\xoption{double}:] The output page is doubled, one without page
% grid and the other with page grid. Possible values are shown in the
% following table:
% \begin{quote}
% \begin{tabular}{ll}
% Option & Meaning\\
% \hline
% |false| & Turns option off.\\
% |first| & Grid page comes first.\\
% |last| & Grid page comes after the page without grid.\\
% |true| & Same as |last|.\\
% \meta{no value} & Same as |true|.\\
% \end{tabular}
% \end{quote}
% \textbf{Note:}
% The double output of the page has side effects.
% All whatits are executed twice, for example: file writing
% and anchor setting. Some unwanted actions are catched such
% as multiple \cs{label} definitions, duplicate entries in
% the table of contents. For bookmarks, use package \xpackage{bookmarks}.
% \item[\xoption{foreground}:] Boolean option, default is \texttt{false}.
% Sometimes there might be elements on the page (e.g. large images)
% that hide the grid. Then option \xoption{foreground} puts the grids
% over the current output page.
% \end{description}
%
% \StopEventually{
% }
%
% \section{Implementation}
%    \begin{macrocode}
%<*package>
%    \end{macrocode}
%    Reload check, especially if the package is not used with \LaTeX.
%    \begin{macrocode}
\begingroup\catcode61\catcode48\catcode32=10\relax%
  \catcode13=5 % ^^M
  \endlinechar=13 %
  \catcode35=6 % #
  \catcode39=12 % '
  \catcode44=12 % ,
  \catcode45=12 % -
  \catcode46=12 % .
  \catcode58=12 % :
  \catcode64=11 % @
  \catcode123=1 % {
  \catcode125=2 % }
  \expandafter\let\expandafter\x\csname ver@pagegrid.sty\endcsname
  \ifx\x\relax % plain-TeX, first loading
  \else
    \def\empty{}%
    \ifx\x\empty % LaTeX, first loading,
      % variable is initialized, but \ProvidesPackage not yet seen
    \else
      \expandafter\ifx\csname PackageInfo\endcsname\relax
        \def\x#1#2{%
          \immediate\write-1{Package #1 Info: #2.}%
        }%
      \else
        \def\x#1#2{\PackageInfo{#1}{#2, stopped}}%
      \fi
      \x{pagegrid}{The package is already loaded}%
      \aftergroup\endinput
    \fi
  \fi
\endgroup%
%    \end{macrocode}
%    Package identification:
%    \begin{macrocode}
\begingroup\catcode61\catcode48\catcode32=10\relax%
  \catcode13=5 % ^^M
  \endlinechar=13 %
  \catcode35=6 % #
  \catcode39=12 % '
  \catcode40=12 % (
  \catcode41=12 % )
  \catcode44=12 % ,
  \catcode45=12 % -
  \catcode46=12 % .
  \catcode47=12 % /
  \catcode58=12 % :
  \catcode64=11 % @
  \catcode91=12 % [
  \catcode93=12 % ]
  \catcode123=1 % {
  \catcode125=2 % }
  \expandafter\ifx\csname ProvidesPackage\endcsname\relax
    \def\x#1#2#3[#4]{\endgroup
      \immediate\write-1{Package: #3 #4}%
      \xdef#1{#4}%
    }%
  \else
    \def\x#1#2[#3]{\endgroup
      #2[{#3}]%
      \ifx#1\@undefined
        \xdef#1{#3}%
      \fi
      \ifx#1\relax
        \xdef#1{#3}%
      \fi
    }%
  \fi
\expandafter\x\csname ver@pagegrid.sty\endcsname
\ProvidesPackage{pagegrid}%
  [2016/05/16 v1.5 Print page grid in background (HO)]%
%    \end{macrocode}
%
%    \begin{macrocode}
\begingroup\catcode61\catcode48\catcode32=10\relax%
  \catcode13=5 % ^^M
  \endlinechar=13 %
  \catcode123=1 % {
  \catcode125=2 % }
  \catcode64=11 % @
  \def\x{\endgroup
    \expandafter\edef\csname pagegrid@AtEnd\endcsname{%
      \endlinechar=\the\endlinechar\relax
      \catcode13=\the\catcode13\relax
      \catcode32=\the\catcode32\relax
      \catcode35=\the\catcode35\relax
      \catcode61=\the\catcode61\relax
      \catcode64=\the\catcode64\relax
      \catcode123=\the\catcode123\relax
      \catcode125=\the\catcode125\relax
    }%
  }%
\x\catcode61\catcode48\catcode32=10\relax%
\catcode13=5 % ^^M
\endlinechar=13 %
\catcode35=6 % #
\catcode64=11 % @
\catcode123=1 % {
\catcode125=2 % }
\def\TMP@EnsureCode#1#2{%
  \edef\pagegrid@AtEnd{%
    \pagegrid@AtEnd
    \catcode#1=\the\catcode#1\relax
  }%
  \catcode#1=#2\relax
}
\TMP@EnsureCode{9}{10}% (tab)
\TMP@EnsureCode{10}{12}% ^^J
\TMP@EnsureCode{33}{12}% !
\TMP@EnsureCode{34}{12}% "
\TMP@EnsureCode{36}{3}% $
\TMP@EnsureCode{38}{4}% &
\TMP@EnsureCode{39}{12}% '
\TMP@EnsureCode{40}{12}% (
\TMP@EnsureCode{41}{12}% )
\TMP@EnsureCode{42}{12}% *
\TMP@EnsureCode{43}{12}% +
\TMP@EnsureCode{44}{12}% ,
\TMP@EnsureCode{45}{12}% -
\TMP@EnsureCode{46}{12}% .
\TMP@EnsureCode{47}{12}% /
\TMP@EnsureCode{58}{12}% :
\TMP@EnsureCode{59}{12}% ;
\TMP@EnsureCode{60}{12}% <
\TMP@EnsureCode{62}{12}% >
\TMP@EnsureCode{63}{12}% ?
\TMP@EnsureCode{91}{12}% [
\TMP@EnsureCode{93}{12}% ]
\TMP@EnsureCode{94}{7}% ^ (superscript)
\TMP@EnsureCode{95}{8}% _ (subscript)
\TMP@EnsureCode{96}{12}% `
\TMP@EnsureCode{124}{12}% |
\edef\pagegrid@AtEnd{\pagegrid@AtEnd\noexpand\endinput}
%    \end{macrocode}
%
%    \begin{macrocode}
\RequirePackage{tikz}
\RequirePackage{atbegshi}[2009/12/02]
\RequirePackage{kvoptions}[2009/07/17]
%    \end{macrocode}
%    \begin{macrocode}
\begingroup\expandafter\expandafter\expandafter\endgroup
\expandafter\ifx\csname stockwidth\endcsname\relax
  \def\pagegrid@width{\paperwidth}%
  \def\pagegrid@height{\paperheight}%
\else
  \def\pagegrid@width{\stockwidth}%
  \def\pagegrid@height{\stockheight}%
\fi
%    \end{macrocode}
%
%    \begin{macrocode}
\SetupKeyvalOptions{%
  family=pagegrid,%
  prefix=pagegrid@,%
}
\def\pagegrid@init{%
  \let\pagegrid@origin@a\@empty
  \let\pagegrid@origin@b\@empty
  \let\pagegrid@init\relax
}
\let\pagegrid@@init\pagegrid@init
\def\pagegrid@origin@a{bl}
\def\pagegrid@origin@b{tr}
\def\pagegrid@SetOrigin#1{%
  \pagegrid@init
  \ifx\pagegrid@origin@a\@empty
    \def\pagegrid@origin@a{#1}%
  \else
    \ifx\pagegrid@origin@b\@empty
    \else
      \let\pagegrid@origin@a\pagegrid@origin@b
    \fi
    \def\pagegrid@origin@b{#1}%
  \fi
}
\def\pagegrid@temp#1{%
  \DeclareVoidOption{#1}{\pagegrid@SetOrigin{#1}}%
  \@namedef{pagegrid@N@#1}{#1}%
}
\pagegrid@temp{bl}
\pagegrid@temp{br}
\pagegrid@temp{tl}
\pagegrid@temp{tr}
\def\pagegrid@temp#1#2{%
  \DeclareVoidOption{#2}{\pagegrid@SetOrigin{#1}}%
}%
\pagegrid@temp{bl}{lb}
\pagegrid@temp{br}{rb}
\pagegrid@temp{tl}{lt}
\pagegrid@temp{tr}{rt}
\pagegrid@temp{bl}{bottom-left}
\pagegrid@temp{br}{bottom-right}
\pagegrid@temp{tl}{top-left}
\pagegrid@temp{tr}{top-right}
\pagegrid@temp{bl}{left-bottom}
\pagegrid@temp{br}{right-bottom}
\pagegrid@temp{tl}{left-top}
\pagegrid@temp{tr}{right-top}
%    \end{macrocode}
%    \begin{macrocode}
\DeclareBoolOption[true]{enable}
\DeclareComplementaryOption{disable}{enable}
%    \end{macrocode}
%    \begin{macrocode}
\DeclareBoolOption{foreground}
%    \end{macrocode}
%    \begin{macrocode}
\newlength{\pagegrid@step}
\define@key{pagegrid}{step}{%
  \setlength{\pagegrid@step}{#1}%
}
%    \end{macrocode}
%    \begin{macrocode}
\DeclareStringOption[red]{firstcolor}
\DeclareStringOption[blue]{secondcolor}
%    \end{macrocode}
%    \begin{macrocode}
\DeclareBoolOption[true]{arrows}
\newlength\pagegrid@arrowlength
\pagegrid@arrowlength=\z@
\define@key{pagegrid}{arrowlength}{%
  \setlength{\pagegrid@arrowlength}{#1}%
}
%    \end{macrocode}
%    \begin{macrocode}
\define@key{pagegrid}{double}[true]{%
  \@ifundefined{pagegrid@double@#1}{%
    \PackageWarning{pagegrid}{%
      Unsupported value `#1' for option `double'.\MessageBreak
      Known values are:\MessageBreak
      `false', `first', `last', `true'.\MessageBreak
      Now `false' is used%
    }%
    \chardef\pagegrid@double\z@
  }{%
    \chardef\pagegrid@double\csname pagegrid@double@#1\endcsname\relax
  }%
}
\@namedef{pagegrid@double@false}{0}
\@namedef{pagegrid@double@first}{1}
\@namedef{pagegrid@double@last}{2}
\@namedef{pagegrid@double@true}{2}
\chardef\pagegrid@double\z@
%    \end{macrocode}
%    \begin{macrocode}
\newcommand*{\pagegridsetup}{%
  \let\pagegrid@init\pagegrid@@init
  \setkeys{pagegrid}%
}
%    \end{macrocode}
%    \begin{macrocode}
\pagegridsetup{%
  step=1mm%
}
\InputIfFileExists{pagegrid.cfg}{}%
\ProcessKeyvalOptions*\relax
\AtBeginDocument{%
  \ifdim\pagegrid@arrowlength>\z@
  \else
    \pagegrid@arrowlength=10\pagegrid@step
  \fi
}
%    \end{macrocode}
%
%    \begin{macrocode}
\def\pagegridShipoutDoubleBegin{%
  \begingroup
  \let\newlabel\@gobbletwo
  \let\zref@newlabel\@gobbletwo
  \let\@writefile\@gobbletwo
  \let\select@language\@gobble
}
\def\pagegridShipoutDoubleEnd{%
  \endgroup
}
\def\pagegrid@WriteDouble#1#2{%
  \immediate\write#1{%
    \@backslashchar csname %
    pagegridShipoutDouble#2%
    \@backslashchar endcsname%
  }%
}
\def\pagegrid@ShipoutDouble#1{%
  \begingroup
    \if@filesw
      \pagegrid@WriteDouble\@mainaux{Begin}%
      \ifx\@auxout\@partaux
        \pagegrid@WriteDouble\@partaux{Begin}%
        \def\pagegrid@temp{%
          \pagegrid@WriteDouble\@mainaux{End}%
          \pagegrid@WriteDouble\@partaux{End}%
        }%
      \else
        \def\pagegrid@temp{%
          \pagegrid@WriteDouble\@mainaux{End}%
        }%
      \fi
    \else
      \def\pagegrid@temp{}%
    \fi
    \let\protect\noexpand
    \AtBeginShipoutOriginalShipout\copy#1\relax
    \pagegrid@temp
  \endgroup
}
%    \end{macrocode}
%
%    \begin{macrocode}
\AtBeginShipout{%
  \ifdim\pagegrid@step>\z@
  \else
    \pagegrid@enablefalse
  \fi
  \ifpagegrid@enable
    \ifnum\pagegrid@double=\@ne
      \pagegrid@ShipoutDouble\AtBeginShipoutBox
    \else
      \ifnum\pagegrid@double=\tw@
        \@ifundefined{pagegrid@DoubleBox}{%
          \newbox\pagegrid@DoubleBox
        }{}%
        \setbox\pagegrid@DoubleBox=\copy\AtBeginShipoutBox
      \fi
    \fi
    \ifpagegrid@foreground
      \expandafter\AtBeginShipoutUpperLeftForeground
    \else
      \expandafter\AtBeginShipoutUpperLeft
    \fi
    {%
      \put(0,0){%
        \makebox(0,0)[lt]{%
          \begin{tikzpicture}[%
            bl/.style={},%
            br/.style={xshift=\pagegrid@width,xscale=-1},%
            tl/.style={yshift=\pagegrid@height,yscale=-1},%
            tr/.style={xshift=\pagegrid@width,%
                       yshift=\pagegrid@height,scale=-1}%
          ]%
            \useasboundingbox
              (0mm,\pagegrid@height) rectangle (0mm,\pagegrid@height);%
            \draw[%
              \pagegrid@origin@a,%
              step=\pagegrid@step,%
              style=help lines,%
              ultra thin%
            ] (0mm,0mm) grid (\pagegrid@width,\pagegrid@height);%
            \ifx\pagegrid@origin@b\@empty
            \else
              \draw[%
                \pagegrid@origin@b,%
                step=10\pagegrid@step,%
                {\pagegrid@secondcolor},%
                very thin%
              ] (0mm,0mm) grid (\pagegrid@width,\pagegrid@height);%
            \fi
            \draw[%
               \pagegrid@origin@a,%
               step=10\pagegrid@step,%
               {\pagegrid@firstcolor},%
               very thin%
            ] (0mm,0mm) grid (\pagegrid@width,\pagegrid@height);%
            \ifx\pagegrid@origin@b\@empty
            \else
              \draw[%
                \pagegrid@origin@b,%
                step=50\pagegrid@step,%
                {\pagegrid@secondcolor},%
                thick%
              ] (0mm,0mm) grid (\pagegrid@width,\pagegrid@height);%
            \fi
            \draw[%
              \pagegrid@origin@a,%
              step=50\pagegrid@step,%
              {\pagegrid@firstcolor},%
              thick%
            ] (0mm,0mm) grid (\pagegrid@width,\pagegrid@height);%
            \ifpagegrid@arrows
              \ifx\pagegrid@origin@b\@empty
              \else
                \draw[%
                  \pagegrid@origin@b,%
                  {\pagegrid@secondcolor},%
                  stroke,%
                  line width=1pt,%
                  line cap=round%
                ] (0mm,0mm) %
                -- (\pagegrid@arrowlength,\pagegrid@arrowlength) %
                   (\pagegrid@arrowlength,.5\pagegrid@arrowlength) %
                -- (\pagegrid@arrowlength,\pagegrid@arrowlength) %
                -- (.5\pagegrid@arrowlength,\pagegrid@arrowlength);%
              \fi
              \draw[%
                \pagegrid@origin@a,%
                {\pagegrid@firstcolor},%
                stroke,%
                line width=1pt,%
                line cap=round%
              ] (0mm,0mm) %
              -- (\pagegrid@arrowlength,\pagegrid@arrowlength) %
                 (\pagegrid@arrowlength,.5\pagegrid@arrowlength) %
              -- (\pagegrid@arrowlength,\pagegrid@arrowlength) %
              -- (.5\pagegrid@arrowlength,\pagegrid@arrowlength);%
            \fi
          \end{tikzpicture}%
        }%
      }%
    }%
    \ifnum\pagegrid@double=\tw@
      \pagegrid@ShipoutDouble\pagegrid@DoubleBox
    \fi
  \fi
}
%    \end{macrocode}
%
%    \begin{macrocode}
\pagegrid@AtEnd%
%</package>
%    \end{macrocode}
%
% \section{Test}
%
% \subsection{Catcode checks for loading}
%
%    \begin{macrocode}
%<*test1>
%    \end{macrocode}
%    \begin{macrocode}
\catcode`\{=1 %
\catcode`\}=2 %
\catcode`\#=6 %
\catcode`\@=11 %
\expandafter\ifx\csname count@\endcsname\relax
  \countdef\count@=255 %
\fi
\expandafter\ifx\csname @gobble\endcsname\relax
  \long\def\@gobble#1{}%
\fi
\expandafter\ifx\csname @firstofone\endcsname\relax
  \long\def\@firstofone#1{#1}%
\fi
\expandafter\ifx\csname loop\endcsname\relax
  \expandafter\@firstofone
\else
  \expandafter\@gobble
\fi
{%
  \def\loop#1\repeat{%
    \def\body{#1}%
    \iterate
  }%
  \def\iterate{%
    \body
      \let\next\iterate
    \else
      \let\next\relax
    \fi
    \next
  }%
  \let\repeat=\fi
}%
\def\RestoreCatcodes{}
\count@=0 %
\loop
  \edef\RestoreCatcodes{%
    \RestoreCatcodes
    \catcode\the\count@=\the\catcode\count@\relax
  }%
\ifnum\count@<255 %
  \advance\count@ 1 %
\repeat

\def\RangeCatcodeInvalid#1#2{%
  \count@=#1\relax
  \loop
    \catcode\count@=15 %
  \ifnum\count@<#2\relax
    \advance\count@ 1 %
  \repeat
}
\def\RangeCatcodeCheck#1#2#3{%
  \count@=#1\relax
  \loop
    \ifnum#3=\catcode\count@
    \else
      \errmessage{%
        Character \the\count@\space
        with wrong catcode \the\catcode\count@\space
        instead of \number#3%
      }%
    \fi
  \ifnum\count@<#2\relax
    \advance\count@ 1 %
  \repeat
}
\def\space{ }
\expandafter\ifx\csname LoadCommand\endcsname\relax
  \def\LoadCommand{\input pagegrid.sty\relax}%
\fi
\def\Test{%
  \RangeCatcodeInvalid{0}{47}%
  \RangeCatcodeInvalid{58}{64}%
  \RangeCatcodeInvalid{91}{96}%
  \RangeCatcodeInvalid{123}{255}%
  \catcode`\@=12 %
  \catcode`\\=0 %
  \catcode`\%=14 %
  \LoadCommand
  \RangeCatcodeCheck{0}{36}{15}%
  \RangeCatcodeCheck{37}{37}{14}%
  \RangeCatcodeCheck{38}{47}{15}%
  \RangeCatcodeCheck{48}{57}{12}%
  \RangeCatcodeCheck{58}{63}{15}%
  \RangeCatcodeCheck{64}{64}{12}%
  \RangeCatcodeCheck{65}{90}{11}%
  \RangeCatcodeCheck{91}{91}{15}%
  \RangeCatcodeCheck{92}{92}{0}%
  \RangeCatcodeCheck{93}{96}{15}%
  \RangeCatcodeCheck{97}{122}{11}%
  \RangeCatcodeCheck{123}{255}{15}%
  \RestoreCatcodes
}
\Test
\csname @@end\endcsname
\end
%    \end{macrocode}
%    \begin{macrocode}
%</test1>
%    \end{macrocode}
%
% \section{Installation}
%
% \subsection{Download}
%
% \paragraph{Package.} This package is available on
% CTAN\footnote{\CTANpkg{pagegrid}}:
% \begin{description}
% \item[\CTAN{macros/latex/contrib/oberdiek/pagegrid.dtx}] The source file.
% \item[\CTAN{macros/latex/contrib/oberdiek/pagegrid.pdf}] Documentation.
% \end{description}
%
%
% \paragraph{Bundle.} All the packages of the bundle `oberdiek'
% are also available in a TDS compliant ZIP archive. There
% the packages are already unpacked and the documentation files
% are generated. The files and directories obey the TDS standard.
% \begin{description}
% \item[\CTANinstall{install/macros/latex/contrib/oberdiek.tds.zip}]
% \end{description}
% \emph{TDS} refers to the standard ``A Directory Structure
% for \TeX\ Files'' (\CTAN{tds/tds.pdf}). Directories
% with \xfile{texmf} in their name are usually organized this way.
%
% \subsection{Bundle installation}
%
% \paragraph{Unpacking.} Unpack the \xfile{oberdiek.tds.zip} in the
% TDS tree (also known as \xfile{texmf} tree) of your choice.
% Example (linux):
% \begin{quote}
%   |unzip oberdiek.tds.zip -d ~/texmf|
% \end{quote}
%
% \paragraph{Script installation.}
% Check the directory \xfile{TDS:scripts/oberdiek/} for
% scripts that need further installation steps.
%
% \subsection{Package installation}
%
% \paragraph{Unpacking.} The \xfile{.dtx} file is a self-extracting
% \docstrip\ archive. The files are extracted by running the
% \xfile{.dtx} through \plainTeX:
% \begin{quote}
%   \verb|tex pagegrid.dtx|
% \end{quote}
%
% \paragraph{TDS.} Now the different files must be moved into
% the different directories in your installation TDS tree
% (also known as \xfile{texmf} tree):
% \begin{quote}
% \def\t{^^A
% \begin{tabular}{@{}>{\ttfamily}l@{ $\rightarrow$ }>{\ttfamily}l@{}}
%   pagegrid.sty & tex/latex/oberdiek/pagegrid.sty\\
%   pagegrid.pdf & doc/latex/oberdiek/pagegrid.pdf\\
%   test/pagegrid-test1.tex & doc/latex/oberdiek/test/pagegrid-test1.tex\\
%   pagegrid.dtx & source/latex/oberdiek/pagegrid.dtx\\
% \end{tabular}^^A
% }^^A
% \sbox0{\t}^^A
% \ifdim\wd0>\linewidth
%   \begingroup
%     \advance\linewidth by\leftmargin
%     \advance\linewidth by\rightmargin
%   \edef\x{\endgroup
%     \def\noexpand\lw{\the\linewidth}^^A
%   }\x
%   \def\lwbox{^^A
%     \leavevmode
%     \hbox to \linewidth{^^A
%       \kern-\leftmargin\relax
%       \hss
%       \usebox0
%       \hss
%       \kern-\rightmargin\relax
%     }^^A
%   }^^A
%   \ifdim\wd0>\lw
%     \sbox0{\small\t}^^A
%     \ifdim\wd0>\linewidth
%       \ifdim\wd0>\lw
%         \sbox0{\footnotesize\t}^^A
%         \ifdim\wd0>\linewidth
%           \ifdim\wd0>\lw
%             \sbox0{\scriptsize\t}^^A
%             \ifdim\wd0>\linewidth
%               \ifdim\wd0>\lw
%                 \sbox0{\tiny\t}^^A
%                 \ifdim\wd0>\linewidth
%                   \lwbox
%                 \else
%                   \usebox0
%                 \fi
%               \else
%                 \lwbox
%               \fi
%             \else
%               \usebox0
%             \fi
%           \else
%             \lwbox
%           \fi
%         \else
%           \usebox0
%         \fi
%       \else
%         \lwbox
%       \fi
%     \else
%       \usebox0
%     \fi
%   \else
%     \lwbox
%   \fi
% \else
%   \usebox0
% \fi
% \end{quote}
% If you have a \xfile{docstrip.cfg} that configures and enables \docstrip's
% TDS installing feature, then some files can already be in the right
% place, see the documentation of \docstrip.
%
% \subsection{Refresh file name databases}
%
% If your \TeX~distribution
% (\TeX\,Live, \mikTeX, \dots) relies on file name databases, you must refresh
% these. For example, \TeX\,Live\ users run \verb|texhash| or
% \verb|mktexlsr|.
%
% \subsection{Some details for the interested}
%
% \paragraph{Unpacking with \LaTeX.}
% The \xfile{.dtx} chooses its action depending on the format:
% \begin{description}
% \item[\plainTeX:] Run \docstrip\ and extract the files.
% \item[\LaTeX:] Generate the documentation.
% \end{description}
% If you insist on using \LaTeX\ for \docstrip\ (really,
% \docstrip\ does not need \LaTeX), then inform the autodetect routine
% about your intention:
% \begin{quote}
%   \verb|latex \let\install=y\input{pagegrid.dtx}|
% \end{quote}
% Do not forget to quote the argument according to the demands
% of your shell.
%
% \paragraph{Generating the documentation.}
% You can use both the \xfile{.dtx} or the \xfile{.drv} to generate
% the documentation. The process can be configured by the
% configuration file \xfile{ltxdoc.cfg}. For instance, put this
% line into this file, if you want to have A4 as paper format:
% \begin{quote}
%   \verb|\PassOptionsToClass{a4paper}{article}|
% \end{quote}
% An example follows how to generate the
% documentation with pdf\LaTeX:
% \begin{quote}
%\begin{verbatim}
%pdflatex pagegrid.dtx
%makeindex -s gind.ist pagegrid.idx
%pdflatex pagegrid.dtx
%makeindex -s gind.ist pagegrid.idx
%pdflatex pagegrid.dtx
%\end{verbatim}
% \end{quote}
%
% \section{Acknowledgement}
%
% \begin{description}
% \item[Klaus Braune:]
%  He provided the idea and the first \xpackage{tikz} code.
% \end{description}
%
% \begin{History}
%   \begin{Version}{2009/11/06 v1.0}
%   \item
%     The first version.
%   \end{Version}
%   \begin{Version}{2009/11/06 v1.1}
%   \item
%     Option \xoption{foreground} added.
%   \end{Version}
%   \begin{Version}{2009/12/02 v1.2}
%   \item
%     Color options, arrow options added.
%   \item
%     Names for origin options changed.
%   \end{Version}
%   \begin{Version}{2009/12/03 v1.3}
%   \item
%     Option \xoption{double} added.
%   \item
%     First CTAN release.
%   \end{Version}
%   \begin{Version}{2009/12/04 v1.4}
%   \item
%     Option \xoption{double}: Some unwanted side effects removed.
%   \end{Version}
%   \begin{Version}{2016/05/16 v1.5}
%   \item
%     Documentation updates.
%   \end{Version}
% \end{History}
%
% \PrintIndex
%
% \Finale
\endinput
|
% \end{quote}
% Do not forget to quote the argument according to the demands
% of your shell.
%
% \paragraph{Generating the documentation.}
% You can use both the \xfile{.dtx} or the \xfile{.drv} to generate
% the documentation. The process can be configured by the
% configuration file \xfile{ltxdoc.cfg}. For instance, put this
% line into this file, if you want to have A4 as paper format:
% \begin{quote}
%   \verb|\PassOptionsToClass{a4paper}{article}|
% \end{quote}
% An example follows how to generate the
% documentation with pdf\LaTeX:
% \begin{quote}
%\begin{verbatim}
%pdflatex pagegrid.dtx
%makeindex -s gind.ist pagegrid.idx
%pdflatex pagegrid.dtx
%makeindex -s gind.ist pagegrid.idx
%pdflatex pagegrid.dtx
%\end{verbatim}
% \end{quote}
%
% \section{Acknowledgement}
%
% \begin{description}
% \item[Klaus Braune:]
%  He provided the idea and the first \xpackage{tikz} code.
% \end{description}
%
% \begin{History}
%   \begin{Version}{2009/11/06 v1.0}
%   \item
%     The first version.
%   \end{Version}
%   \begin{Version}{2009/11/06 v1.1}
%   \item
%     Option \xoption{foreground} added.
%   \end{Version}
%   \begin{Version}{2009/12/02 v1.2}
%   \item
%     Color options, arrow options added.
%   \item
%     Names for origin options changed.
%   \end{Version}
%   \begin{Version}{2009/12/03 v1.3}
%   \item
%     Option \xoption{double} added.
%   \item
%     First CTAN release.
%   \end{Version}
%   \begin{Version}{2009/12/04 v1.4}
%   \item
%     Option \xoption{double}: Some unwanted side effects removed.
%   \end{Version}
%   \begin{Version}{2016/05/16 v1.5}
%   \item
%     Documentation updates.
%   \end{Version}
% \end{History}
%
% \PrintIndex
%
% \Finale
\endinput

%        (quote the arguments according to the demands of your shell)
%
% Documentation:
%    (a) If pagegrid.drv is present:
%           latex pagegrid.drv
%    (b) Without pagegrid.drv:
%           latex pagegrid.dtx; ...
%    The class ltxdoc loads the configuration file ltxdoc.cfg
%    if available. Here you can specify further options, e.g.
%    use A4 as paper format:
%       \PassOptionsToClass{a4paper}{article}
%
%    Programm calls to get the documentation (example):
%       pdflatex pagegrid.dtx
%       makeindex -s gind.ist pagegrid.idx
%       pdflatex pagegrid.dtx
%       makeindex -s gind.ist pagegrid.idx
%       pdflatex pagegrid.dtx
%
% Installation:
%    TDS:tex/latex/oberdiek/pagegrid.sty
%    TDS:doc/latex/oberdiek/pagegrid.pdf
%    TDS:doc/latex/oberdiek/test/pagegrid-test1.tex
%    TDS:source/latex/oberdiek/pagegrid.dtx
%
%<*ignore>
\begingroup
  \catcode123=1 %
  \catcode125=2 %
  \def\x{LaTeX2e}%
\expandafter\endgroup
\ifcase 0\ifx\install y1\fi\expandafter
         \ifx\csname processbatchFile\endcsname\relax\else1\fi
         \ifx\fmtname\x\else 1\fi\relax
\else\csname fi\endcsname
%</ignore>
%<*install>
\input docstrip.tex
\Msg{************************************************************************}
\Msg{* Installation}
\Msg{* Package: pagegrid 2016/05/16 v1.5 Print page grid in background (HO)}
\Msg{************************************************************************}

\keepsilent
\askforoverwritefalse

\let\MetaPrefix\relax
\preamble

This is a generated file.

Project: pagegrid
Version: 2016/05/16 v1.5

Copyright (C) 2009 by
   Heiko Oberdiek <heiko.oberdiek at googlemail.com>

This work may be distributed and/or modified under the
conditions of the LaTeX Project Public License, either
version 1.3c of this license or (at your option) any later
version. This version of this license is in
   https://www.latex-project.org/lppl/lppl-1-3c.txt
and the latest version of this license is in
   https://www.latex-project.org/lppl.txt
and version 1.3 or later is part of all distributions of
LaTeX version 2005/12/01 or later.

This work has the LPPL maintenance status "maintained".

The Current Maintainers of this work are
Heiko Oberdiek and the Oberdiek Package Support Group
https://github.com/ho-tex/oberdiek/issues


This work consists of the main source file pagegrid.dtx
and the derived files
   pagegrid.sty, pagegrid.pdf, pagegrid.ins, pagegrid.drv,
   pagegrid-test1.tex.

\endpreamble
\let\MetaPrefix\DoubleperCent

\generate{%
  \file{pagegrid.ins}{\from{pagegrid.dtx}{install}}%
  \file{pagegrid.drv}{\from{pagegrid.dtx}{driver}}%
  \usedir{tex/latex/oberdiek}%
  \file{pagegrid.sty}{\from{pagegrid.dtx}{package}}%
%  \usedir{doc/latex/oberdiek/test}%
%  \file{pagegrid-test1.tex}{\from{pagegrid.dtx}{test1}}%
  \nopreamble
  \nopostamble
%  \usedir{source/latex/oberdiek/catalogue}%
%  \file{pagegrid.xml}{\from{pagegrid.dtx}{catalogue}}%
}

\catcode32=13\relax% active space
\let =\space%
\Msg{************************************************************************}
\Msg{*}
\Msg{* To finish the installation you have to move the following}
\Msg{* file into a directory searched by TeX:}
\Msg{*}
\Msg{*     pagegrid.sty}
\Msg{*}
\Msg{* To produce the documentation run the file `pagegrid.drv'}
\Msg{* through LaTeX.}
\Msg{*}
\Msg{* Happy TeXing!}
\Msg{*}
\Msg{************************************************************************}

\endbatchfile
%</install>
%<*ignore>
\fi
%</ignore>
%<*driver>
\NeedsTeXFormat{LaTeX2e}
\ProvidesFile{pagegrid.drv}%
  [2016/05/16 v1.5 Print page grid in background (HO)]%
\documentclass{ltxdoc}
\usepackage{holtxdoc}[2011/11/22]
\begin{document}
  \DocInput{pagegrid.dtx}%
\end{document}
%</driver>
% \fi
%
%
% \CharacterTable
%  {Upper-case    \A\B\C\D\E\F\G\H\I\J\K\L\M\N\O\P\Q\R\S\T\U\V\W\X\Y\Z
%   Lower-case    \a\b\c\d\e\f\g\h\i\j\k\l\m\n\o\p\q\r\s\t\u\v\w\x\y\z
%   Digits        \0\1\2\3\4\5\6\7\8\9
%   Exclamation   \!     Double quote  \"     Hash (number) \#
%   Dollar        \$     Percent       \%     Ampersand     \&
%   Acute accent  \'     Left paren    \(     Right paren   \)
%   Asterisk      \*     Plus          \+     Comma         \,
%   Minus         \-     Point         \.     Solidus       \/
%   Colon         \:     Semicolon     \;     Less than     \<
%   Equals        \=     Greater than  \>     Question mark \?
%   Commercial at \@     Left bracket  \[     Backslash     \\
%   Right bracket \]     Circumflex    \^     Underscore    \_
%   Grave accent  \`     Left brace    \{     Vertical bar  \|
%   Right brace   \}     Tilde         \~}
%
% \GetFileInfo{pagegrid.drv}
%
% \title{The \xpackage{pagegrid} package}
% \date{2016/05/16 v1.5}
% \author{Heiko Oberdiek\thanks
% {Please report any issues at \url{https://github.com/ho-tex/oberdiek/issues}}}
%
% \maketitle
%
% \begin{abstract}
% The \LaTeX\ package prints a page grid in the background.
% \end{abstract}
%
% \tableofcontents
%
% \section{Documentation}
%
% The package puts a grid on the paper. It was written for
% developers of a class or package
% who have to put elements on definite locations on a page
% (e.g. letter class). The grid allows a faster optical check,
% whether the positions are correct. If the previewer already
% offers features for measuring, the package might be obsolete.
% Otherwise it saves the developer from printing the page and
% measuring by hand.
%
% \subsection{Options}
%
% Options are evaluated in the following order:
% \begin{enumerate}
% \item
%  Configuration file \xfile{pagegrid.cfg} using \cs{pagegridsetup}
%  if the file exists.
%  \item
%  Package options given for \cs{usepackage}.
%  \item
%  Later calls of \cs{pagegridsetup}.
% \end{enumerate}
% \begin{declcs}{pagegridsetup}\M{option list}
% \end{declcs}
% The options are key value options. Boolean options are enabled by
% default (without value) or by using the explicit value \texttt{true}.
% Value \texttt{false} disable the option.
%
% \subsubsection{Options \xoption{enable}, \xoption{disable}}
%
% \begin{description}
% \item[\xoption{enable}:] This boolean option controls whether the page grid
%   is drawn. As default the page grid drawing is activated.
% \item[\xoption{disable}:] It is the opposite
%   of option \xoption{enable}. It was added for convenience and
%   allows the abbreviation \texttt{disable} for \texttt{enable=false}.
% \end{description}
%
% \subsubsection{Grid origins}
%
% The package supports up to two grids on a page allowing
% measurement from opposite directions. As default two grids are drawn,
% the first from bottom left to top right. The origin of the second
% grid is at the opposite top right corner.
% The origins are controlled by the following options.
% The number of grids (one or two) depend on the number of these options
% in one call of \cs{pagegridsetup}.
% The following frame shows a paper and in its corners are the
% corresponding options. At the left and right side alias names
% are given for the options inside the paper.
% \begin{quote}
% \begin{tabular}{@{}r|@{\,}l@{\qquad}r@{\,}|l@{}}
% \cline{2-3}
% \xoption{left-top}, \xoption{lt}, \xoption{top-left}
% & \vphantom{\"U}\xoption{tl} & \xoption{tr}
% & \xoption{top-right}, \xoption{rt}, \xoption{right-top}\\
% &&&\\
% \xoption{left-bottom}, \xoption{lb}, \xoption{bottom-left}
% & \xoption{bl} & \xoption{br}
% & \xoption{bottom-right}, \xoption{rb}, \xoption{right-bottom}\\
% \cline{2-3}
% \end{tabular}
% \end{quote}
% Examples:
% \begin{quote}
% |\pagegridsetup{bl,tr}|
% \end{quote}
% This is the default setting with two grids as described previously.
% The following setups one grid only. Its origin is the upper left
% corner:
% \begin{quote}
% |\pagegridsetup{top-left}|
% \end{quote}
%
% \subsubsection{Grid unit}
%
% \begin{description}
% \item[\xoption{step}] This option takes a length and
% setups the unit for the grid. The page width and page height
% should be multiples of this unit.
% Currently the default is \texttt{1mm}. But this might change
% later by a heuristic based on the paper size.
% \end{description}
%
% \subsubsection{Color options}
%
% The basic grid lines are drawn as ultra thin help lines and is only
% drawn for the first grid.
% Each tenth and fiftyth line of the basic net is drawn thicker in a special
% color for the two grids.
% \begin{description}
% \item[\xoption{firstcolor}:] Color for the thicker lines and the arrows
% of the first grid. Default value is \texttt{red}.
% \item[\xoption{secondcolor}:] Color for the thicker lines and the arrows
% of the second grid. Default value is \texttt{blue}.
% \end{description}
% Use a color specification that package \xpackage{tikz} understands.
% (The grid is drawn with \xpackage{pgf}/\xpackage{tikz}.)
%
% \subsubsection{Arrow options}
%
% Arrows are put at the origin at the grid to show the grid start
% and the direction of the grid.
% \begin{description}
% \item[\xoption{arrows}:] This boolean option turns the arrows on or off.
% As default arrows are enabled.
% \item[\xoption{arrowlength}:] The length given as value is the
% length of the edge of a square at the origin within the
% arrow is put as diagonal. Default is 10 times the grid unit (10\,mm).
% The real arrow length is this length multiplied by $\sqrt2$.
% \end{description}
%
% \subsubsection{Miscellaneous options}
%
% \begin{description}
% \item[\xoption{double}:] The output page is doubled, one without page
% grid and the other with page grid. Possible values are shown in the
% following table:
% \begin{quote}
% \begin{tabular}{ll}
% Option & Meaning\\
% \hline
% |false| & Turns option off.\\
% |first| & Grid page comes first.\\
% |last| & Grid page comes after the page without grid.\\
% |true| & Same as |last|.\\
% \meta{no value} & Same as |true|.\\
% \end{tabular}
% \end{quote}
% \textbf{Note:}
% The double output of the page has side effects.
% All whatits are executed twice, for example: file writing
% and anchor setting. Some unwanted actions are catched such
% as multiple \cs{label} definitions, duplicate entries in
% the table of contents. For bookmarks, use package \xpackage{bookmarks}.
% \item[\xoption{foreground}:] Boolean option, default is \texttt{false}.
% Sometimes there might be elements on the page (e.g. large images)
% that hide the grid. Then option \xoption{foreground} puts the grids
% over the current output page.
% \end{description}
%
% \StopEventually{
% }
%
% \section{Implementation}
%    \begin{macrocode}
%<*package>
%    \end{macrocode}
%    Reload check, especially if the package is not used with \LaTeX.
%    \begin{macrocode}
\begingroup\catcode61\catcode48\catcode32=10\relax%
  \catcode13=5 % ^^M
  \endlinechar=13 %
  \catcode35=6 % #
  \catcode39=12 % '
  \catcode44=12 % ,
  \catcode45=12 % -
  \catcode46=12 % .
  \catcode58=12 % :
  \catcode64=11 % @
  \catcode123=1 % {
  \catcode125=2 % }
  \expandafter\let\expandafter\x\csname ver@pagegrid.sty\endcsname
  \ifx\x\relax % plain-TeX, first loading
  \else
    \def\empty{}%
    \ifx\x\empty % LaTeX, first loading,
      % variable is initialized, but \ProvidesPackage not yet seen
    \else
      \expandafter\ifx\csname PackageInfo\endcsname\relax
        \def\x#1#2{%
          \immediate\write-1{Package #1 Info: #2.}%
        }%
      \else
        \def\x#1#2{\PackageInfo{#1}{#2, stopped}}%
      \fi
      \x{pagegrid}{The package is already loaded}%
      \aftergroup\endinput
    \fi
  \fi
\endgroup%
%    \end{macrocode}
%    Package identification:
%    \begin{macrocode}
\begingroup\catcode61\catcode48\catcode32=10\relax%
  \catcode13=5 % ^^M
  \endlinechar=13 %
  \catcode35=6 % #
  \catcode39=12 % '
  \catcode40=12 % (
  \catcode41=12 % )
  \catcode44=12 % ,
  \catcode45=12 % -
  \catcode46=12 % .
  \catcode47=12 % /
  \catcode58=12 % :
  \catcode64=11 % @
  \catcode91=12 % [
  \catcode93=12 % ]
  \catcode123=1 % {
  \catcode125=2 % }
  \expandafter\ifx\csname ProvidesPackage\endcsname\relax
    \def\x#1#2#3[#4]{\endgroup
      \immediate\write-1{Package: #3 #4}%
      \xdef#1{#4}%
    }%
  \else
    \def\x#1#2[#3]{\endgroup
      #2[{#3}]%
      \ifx#1\@undefined
        \xdef#1{#3}%
      \fi
      \ifx#1\relax
        \xdef#1{#3}%
      \fi
    }%
  \fi
\expandafter\x\csname ver@pagegrid.sty\endcsname
\ProvidesPackage{pagegrid}%
  [2016/05/16 v1.5 Print page grid in background (HO)]%
%    \end{macrocode}
%
%    \begin{macrocode}
\begingroup\catcode61\catcode48\catcode32=10\relax%
  \catcode13=5 % ^^M
  \endlinechar=13 %
  \catcode123=1 % {
  \catcode125=2 % }
  \catcode64=11 % @
  \def\x{\endgroup
    \expandafter\edef\csname pagegrid@AtEnd\endcsname{%
      \endlinechar=\the\endlinechar\relax
      \catcode13=\the\catcode13\relax
      \catcode32=\the\catcode32\relax
      \catcode35=\the\catcode35\relax
      \catcode61=\the\catcode61\relax
      \catcode64=\the\catcode64\relax
      \catcode123=\the\catcode123\relax
      \catcode125=\the\catcode125\relax
    }%
  }%
\x\catcode61\catcode48\catcode32=10\relax%
\catcode13=5 % ^^M
\endlinechar=13 %
\catcode35=6 % #
\catcode64=11 % @
\catcode123=1 % {
\catcode125=2 % }
\def\TMP@EnsureCode#1#2{%
  \edef\pagegrid@AtEnd{%
    \pagegrid@AtEnd
    \catcode#1=\the\catcode#1\relax
  }%
  \catcode#1=#2\relax
}
\TMP@EnsureCode{9}{10}% (tab)
\TMP@EnsureCode{10}{12}% ^^J
\TMP@EnsureCode{33}{12}% !
\TMP@EnsureCode{34}{12}% "
\TMP@EnsureCode{36}{3}% $
\TMP@EnsureCode{38}{4}% &
\TMP@EnsureCode{39}{12}% '
\TMP@EnsureCode{40}{12}% (
\TMP@EnsureCode{41}{12}% )
\TMP@EnsureCode{42}{12}% *
\TMP@EnsureCode{43}{12}% +
\TMP@EnsureCode{44}{12}% ,
\TMP@EnsureCode{45}{12}% -
\TMP@EnsureCode{46}{12}% .
\TMP@EnsureCode{47}{12}% /
\TMP@EnsureCode{58}{12}% :
\TMP@EnsureCode{59}{12}% ;
\TMP@EnsureCode{60}{12}% <
\TMP@EnsureCode{62}{12}% >
\TMP@EnsureCode{63}{12}% ?
\TMP@EnsureCode{91}{12}% [
\TMP@EnsureCode{93}{12}% ]
\TMP@EnsureCode{94}{7}% ^ (superscript)
\TMP@EnsureCode{95}{8}% _ (subscript)
\TMP@EnsureCode{96}{12}% `
\TMP@EnsureCode{124}{12}% |
\edef\pagegrid@AtEnd{\pagegrid@AtEnd\noexpand\endinput}
%    \end{macrocode}
%
%    \begin{macrocode}
\RequirePackage{tikz}
\RequirePackage{atbegshi}[2009/12/02]
\RequirePackage{kvoptions}[2009/07/17]
%    \end{macrocode}
%    \begin{macrocode}
\begingroup\expandafter\expandafter\expandafter\endgroup
\expandafter\ifx\csname stockwidth\endcsname\relax
  \def\pagegrid@width{\paperwidth}%
  \def\pagegrid@height{\paperheight}%
\else
  \def\pagegrid@width{\stockwidth}%
  \def\pagegrid@height{\stockheight}%
\fi
%    \end{macrocode}
%
%    \begin{macrocode}
\SetupKeyvalOptions{%
  family=pagegrid,%
  prefix=pagegrid@,%
}
\def\pagegrid@init{%
  \let\pagegrid@origin@a\@empty
  \let\pagegrid@origin@b\@empty
  \let\pagegrid@init\relax
}
\let\pagegrid@@init\pagegrid@init
\def\pagegrid@origin@a{bl}
\def\pagegrid@origin@b{tr}
\def\pagegrid@SetOrigin#1{%
  \pagegrid@init
  \ifx\pagegrid@origin@a\@empty
    \def\pagegrid@origin@a{#1}%
  \else
    \ifx\pagegrid@origin@b\@empty
    \else
      \let\pagegrid@origin@a\pagegrid@origin@b
    \fi
    \def\pagegrid@origin@b{#1}%
  \fi
}
\def\pagegrid@temp#1{%
  \DeclareVoidOption{#1}{\pagegrid@SetOrigin{#1}}%
  \@namedef{pagegrid@N@#1}{#1}%
}
\pagegrid@temp{bl}
\pagegrid@temp{br}
\pagegrid@temp{tl}
\pagegrid@temp{tr}
\def\pagegrid@temp#1#2{%
  \DeclareVoidOption{#2}{\pagegrid@SetOrigin{#1}}%
}%
\pagegrid@temp{bl}{lb}
\pagegrid@temp{br}{rb}
\pagegrid@temp{tl}{lt}
\pagegrid@temp{tr}{rt}
\pagegrid@temp{bl}{bottom-left}
\pagegrid@temp{br}{bottom-right}
\pagegrid@temp{tl}{top-left}
\pagegrid@temp{tr}{top-right}
\pagegrid@temp{bl}{left-bottom}
\pagegrid@temp{br}{right-bottom}
\pagegrid@temp{tl}{left-top}
\pagegrid@temp{tr}{right-top}
%    \end{macrocode}
%    \begin{macrocode}
\DeclareBoolOption[true]{enable}
\DeclareComplementaryOption{disable}{enable}
%    \end{macrocode}
%    \begin{macrocode}
\DeclareBoolOption{foreground}
%    \end{macrocode}
%    \begin{macrocode}
\newlength{\pagegrid@step}
\define@key{pagegrid}{step}{%
  \setlength{\pagegrid@step}{#1}%
}
%    \end{macrocode}
%    \begin{macrocode}
\DeclareStringOption[red]{firstcolor}
\DeclareStringOption[blue]{secondcolor}
%    \end{macrocode}
%    \begin{macrocode}
\DeclareBoolOption[true]{arrows}
\newlength\pagegrid@arrowlength
\pagegrid@arrowlength=\z@
\define@key{pagegrid}{arrowlength}{%
  \setlength{\pagegrid@arrowlength}{#1}%
}
%    \end{macrocode}
%    \begin{macrocode}
\define@key{pagegrid}{double}[true]{%
  \@ifundefined{pagegrid@double@#1}{%
    \PackageWarning{pagegrid}{%
      Unsupported value `#1' for option `double'.\MessageBreak
      Known values are:\MessageBreak
      `false', `first', `last', `true'.\MessageBreak
      Now `false' is used%
    }%
    \chardef\pagegrid@double\z@
  }{%
    \chardef\pagegrid@double\csname pagegrid@double@#1\endcsname\relax
  }%
}
\@namedef{pagegrid@double@false}{0}
\@namedef{pagegrid@double@first}{1}
\@namedef{pagegrid@double@last}{2}
\@namedef{pagegrid@double@true}{2}
\chardef\pagegrid@double\z@
%    \end{macrocode}
%    \begin{macrocode}
\newcommand*{\pagegridsetup}{%
  \let\pagegrid@init\pagegrid@@init
  \setkeys{pagegrid}%
}
%    \end{macrocode}
%    \begin{macrocode}
\pagegridsetup{%
  step=1mm%
}
\InputIfFileExists{pagegrid.cfg}{}%
\ProcessKeyvalOptions*\relax
\AtBeginDocument{%
  \ifdim\pagegrid@arrowlength>\z@
  \else
    \pagegrid@arrowlength=10\pagegrid@step
  \fi
}
%    \end{macrocode}
%
%    \begin{macrocode}
\def\pagegridShipoutDoubleBegin{%
  \begingroup
  \let\newlabel\@gobbletwo
  \let\zref@newlabel\@gobbletwo
  \let\@writefile\@gobbletwo
  \let\select@language\@gobble
}
\def\pagegridShipoutDoubleEnd{%
  \endgroup
}
\def\pagegrid@WriteDouble#1#2{%
  \immediate\write#1{%
    \@backslashchar csname %
    pagegridShipoutDouble#2%
    \@backslashchar endcsname%
  }%
}
\def\pagegrid@ShipoutDouble#1{%
  \begingroup
    \if@filesw
      \pagegrid@WriteDouble\@mainaux{Begin}%
      \ifx\@auxout\@partaux
        \pagegrid@WriteDouble\@partaux{Begin}%
        \def\pagegrid@temp{%
          \pagegrid@WriteDouble\@mainaux{End}%
          \pagegrid@WriteDouble\@partaux{End}%
        }%
      \else
        \def\pagegrid@temp{%
          \pagegrid@WriteDouble\@mainaux{End}%
        }%
      \fi
    \else
      \def\pagegrid@temp{}%
    \fi
    \let\protect\noexpand
    \AtBeginShipoutOriginalShipout\copy#1\relax
    \pagegrid@temp
  \endgroup
}
%    \end{macrocode}
%
%    \begin{macrocode}
\AtBeginShipout{%
  \ifdim\pagegrid@step>\z@
  \else
    \pagegrid@enablefalse
  \fi
  \ifpagegrid@enable
    \ifnum\pagegrid@double=\@ne
      \pagegrid@ShipoutDouble\AtBeginShipoutBox
    \else
      \ifnum\pagegrid@double=\tw@
        \@ifundefined{pagegrid@DoubleBox}{%
          \newbox\pagegrid@DoubleBox
        }{}%
        \setbox\pagegrid@DoubleBox=\copy\AtBeginShipoutBox
      \fi
    \fi
    \ifpagegrid@foreground
      \expandafter\AtBeginShipoutUpperLeftForeground
    \else
      \expandafter\AtBeginShipoutUpperLeft
    \fi
    {%
      \put(0,0){%
        \makebox(0,0)[lt]{%
          \begin{tikzpicture}[%
            bl/.style={},%
            br/.style={xshift=\pagegrid@width,xscale=-1},%
            tl/.style={yshift=\pagegrid@height,yscale=-1},%
            tr/.style={xshift=\pagegrid@width,%
                       yshift=\pagegrid@height,scale=-1}%
          ]%
            \useasboundingbox
              (0mm,\pagegrid@height) rectangle (0mm,\pagegrid@height);%
            \draw[%
              \pagegrid@origin@a,%
              step=\pagegrid@step,%
              style=help lines,%
              ultra thin%
            ] (0mm,0mm) grid (\pagegrid@width,\pagegrid@height);%
            \ifx\pagegrid@origin@b\@empty
            \else
              \draw[%
                \pagegrid@origin@b,%
                step=10\pagegrid@step,%
                {\pagegrid@secondcolor},%
                very thin%
              ] (0mm,0mm) grid (\pagegrid@width,\pagegrid@height);%
            \fi
            \draw[%
               \pagegrid@origin@a,%
               step=10\pagegrid@step,%
               {\pagegrid@firstcolor},%
               very thin%
            ] (0mm,0mm) grid (\pagegrid@width,\pagegrid@height);%
            \ifx\pagegrid@origin@b\@empty
            \else
              \draw[%
                \pagegrid@origin@b,%
                step=50\pagegrid@step,%
                {\pagegrid@secondcolor},%
                thick%
              ] (0mm,0mm) grid (\pagegrid@width,\pagegrid@height);%
            \fi
            \draw[%
              \pagegrid@origin@a,%
              step=50\pagegrid@step,%
              {\pagegrid@firstcolor},%
              thick%
            ] (0mm,0mm) grid (\pagegrid@width,\pagegrid@height);%
            \ifpagegrid@arrows
              \ifx\pagegrid@origin@b\@empty
              \else
                \draw[%
                  \pagegrid@origin@b,%
                  {\pagegrid@secondcolor},%
                  stroke,%
                  line width=1pt,%
                  line cap=round%
                ] (0mm,0mm) %
                -- (\pagegrid@arrowlength,\pagegrid@arrowlength) %
                   (\pagegrid@arrowlength,.5\pagegrid@arrowlength) %
                -- (\pagegrid@arrowlength,\pagegrid@arrowlength) %
                -- (.5\pagegrid@arrowlength,\pagegrid@arrowlength);%
              \fi
              \draw[%
                \pagegrid@origin@a,%
                {\pagegrid@firstcolor},%
                stroke,%
                line width=1pt,%
                line cap=round%
              ] (0mm,0mm) %
              -- (\pagegrid@arrowlength,\pagegrid@arrowlength) %
                 (\pagegrid@arrowlength,.5\pagegrid@arrowlength) %
              -- (\pagegrid@arrowlength,\pagegrid@arrowlength) %
              -- (.5\pagegrid@arrowlength,\pagegrid@arrowlength);%
            \fi
          \end{tikzpicture}%
        }%
      }%
    }%
    \ifnum\pagegrid@double=\tw@
      \pagegrid@ShipoutDouble\pagegrid@DoubleBox
    \fi
  \fi
}
%    \end{macrocode}
%
%    \begin{macrocode}
\pagegrid@AtEnd%
%</package>
%    \end{macrocode}
%
% \section{Test}
%
% \subsection{Catcode checks for loading}
%
%    \begin{macrocode}
%<*test1>
%    \end{macrocode}
%    \begin{macrocode}
\catcode`\{=1 %
\catcode`\}=2 %
\catcode`\#=6 %
\catcode`\@=11 %
\expandafter\ifx\csname count@\endcsname\relax
  \countdef\count@=255 %
\fi
\expandafter\ifx\csname @gobble\endcsname\relax
  \long\def\@gobble#1{}%
\fi
\expandafter\ifx\csname @firstofone\endcsname\relax
  \long\def\@firstofone#1{#1}%
\fi
\expandafter\ifx\csname loop\endcsname\relax
  \expandafter\@firstofone
\else
  \expandafter\@gobble
\fi
{%
  \def\loop#1\repeat{%
    \def\body{#1}%
    \iterate
  }%
  \def\iterate{%
    \body
      \let\next\iterate
    \else
      \let\next\relax
    \fi
    \next
  }%
  \let\repeat=\fi
}%
\def\RestoreCatcodes{}
\count@=0 %
\loop
  \edef\RestoreCatcodes{%
    \RestoreCatcodes
    \catcode\the\count@=\the\catcode\count@\relax
  }%
\ifnum\count@<255 %
  \advance\count@ 1 %
\repeat

\def\RangeCatcodeInvalid#1#2{%
  \count@=#1\relax
  \loop
    \catcode\count@=15 %
  \ifnum\count@<#2\relax
    \advance\count@ 1 %
  \repeat
}
\def\RangeCatcodeCheck#1#2#3{%
  \count@=#1\relax
  \loop
    \ifnum#3=\catcode\count@
    \else
      \errmessage{%
        Character \the\count@\space
        with wrong catcode \the\catcode\count@\space
        instead of \number#3%
      }%
    \fi
  \ifnum\count@<#2\relax
    \advance\count@ 1 %
  \repeat
}
\def\space{ }
\expandafter\ifx\csname LoadCommand\endcsname\relax
  \def\LoadCommand{\input pagegrid.sty\relax}%
\fi
\def\Test{%
  \RangeCatcodeInvalid{0}{47}%
  \RangeCatcodeInvalid{58}{64}%
  \RangeCatcodeInvalid{91}{96}%
  \RangeCatcodeInvalid{123}{255}%
  \catcode`\@=12 %
  \catcode`\\=0 %
  \catcode`\%=14 %
  \LoadCommand
  \RangeCatcodeCheck{0}{36}{15}%
  \RangeCatcodeCheck{37}{37}{14}%
  \RangeCatcodeCheck{38}{47}{15}%
  \RangeCatcodeCheck{48}{57}{12}%
  \RangeCatcodeCheck{58}{63}{15}%
  \RangeCatcodeCheck{64}{64}{12}%
  \RangeCatcodeCheck{65}{90}{11}%
  \RangeCatcodeCheck{91}{91}{15}%
  \RangeCatcodeCheck{92}{92}{0}%
  \RangeCatcodeCheck{93}{96}{15}%
  \RangeCatcodeCheck{97}{122}{11}%
  \RangeCatcodeCheck{123}{255}{15}%
  \RestoreCatcodes
}
\Test
\csname @@end\endcsname
\end
%    \end{macrocode}
%    \begin{macrocode}
%</test1>
%    \end{macrocode}
%
% \section{Installation}
%
% \subsection{Download}
%
% \paragraph{Package.} This package is available on
% CTAN\footnote{\CTANpkg{pagegrid}}:
% \begin{description}
% \item[\CTAN{macros/latex/contrib/oberdiek/pagegrid.dtx}] The source file.
% \item[\CTAN{macros/latex/contrib/oberdiek/pagegrid.pdf}] Documentation.
% \end{description}
%
%
% \paragraph{Bundle.} All the packages of the bundle `oberdiek'
% are also available in a TDS compliant ZIP archive. There
% the packages are already unpacked and the documentation files
% are generated. The files and directories obey the TDS standard.
% \begin{description}
% \item[\CTANinstall{install/macros/latex/contrib/oberdiek.tds.zip}]
% \end{description}
% \emph{TDS} refers to the standard ``A Directory Structure
% for \TeX\ Files'' (\CTAN{tds/tds.pdf}). Directories
% with \xfile{texmf} in their name are usually organized this way.
%
% \subsection{Bundle installation}
%
% \paragraph{Unpacking.} Unpack the \xfile{oberdiek.tds.zip} in the
% TDS tree (also known as \xfile{texmf} tree) of your choice.
% Example (linux):
% \begin{quote}
%   |unzip oberdiek.tds.zip -d ~/texmf|
% \end{quote}
%
% \paragraph{Script installation.}
% Check the directory \xfile{TDS:scripts/oberdiek/} for
% scripts that need further installation steps.
%
% \subsection{Package installation}
%
% \paragraph{Unpacking.} The \xfile{.dtx} file is a self-extracting
% \docstrip\ archive. The files are extracted by running the
% \xfile{.dtx} through \plainTeX:
% \begin{quote}
%   \verb|tex pagegrid.dtx|
% \end{quote}
%
% \paragraph{TDS.} Now the different files must be moved into
% the different directories in your installation TDS tree
% (also known as \xfile{texmf} tree):
% \begin{quote}
% \def\t{^^A
% \begin{tabular}{@{}>{\ttfamily}l@{ $\rightarrow$ }>{\ttfamily}l@{}}
%   pagegrid.sty & tex/latex/oberdiek/pagegrid.sty\\
%   pagegrid.pdf & doc/latex/oberdiek/pagegrid.pdf\\
%   test/pagegrid-test1.tex & doc/latex/oberdiek/test/pagegrid-test1.tex\\
%   pagegrid.dtx & source/latex/oberdiek/pagegrid.dtx\\
% \end{tabular}^^A
% }^^A
% \sbox0{\t}^^A
% \ifdim\wd0>\linewidth
%   \begingroup
%     \advance\linewidth by\leftmargin
%     \advance\linewidth by\rightmargin
%   \edef\x{\endgroup
%     \def\noexpand\lw{\the\linewidth}^^A
%   }\x
%   \def\lwbox{^^A
%     \leavevmode
%     \hbox to \linewidth{^^A
%       \kern-\leftmargin\relax
%       \hss
%       \usebox0
%       \hss
%       \kern-\rightmargin\relax
%     }^^A
%   }^^A
%   \ifdim\wd0>\lw
%     \sbox0{\small\t}^^A
%     \ifdim\wd0>\linewidth
%       \ifdim\wd0>\lw
%         \sbox0{\footnotesize\t}^^A
%         \ifdim\wd0>\linewidth
%           \ifdim\wd0>\lw
%             \sbox0{\scriptsize\t}^^A
%             \ifdim\wd0>\linewidth
%               \ifdim\wd0>\lw
%                 \sbox0{\tiny\t}^^A
%                 \ifdim\wd0>\linewidth
%                   \lwbox
%                 \else
%                   \usebox0
%                 \fi
%               \else
%                 \lwbox
%               \fi
%             \else
%               \usebox0
%             \fi
%           \else
%             \lwbox
%           \fi
%         \else
%           \usebox0
%         \fi
%       \else
%         \lwbox
%       \fi
%     \else
%       \usebox0
%     \fi
%   \else
%     \lwbox
%   \fi
% \else
%   \usebox0
% \fi
% \end{quote}
% If you have a \xfile{docstrip.cfg} that configures and enables \docstrip's
% TDS installing feature, then some files can already be in the right
% place, see the documentation of \docstrip.
%
% \subsection{Refresh file name databases}
%
% If your \TeX~distribution
% (\TeX\,Live, \mikTeX, \dots) relies on file name databases, you must refresh
% these. For example, \TeX\,Live\ users run \verb|texhash| or
% \verb|mktexlsr|.
%
% \subsection{Some details for the interested}
%
% \paragraph{Unpacking with \LaTeX.}
% The \xfile{.dtx} chooses its action depending on the format:
% \begin{description}
% \item[\plainTeX:] Run \docstrip\ and extract the files.
% \item[\LaTeX:] Generate the documentation.
% \end{description}
% If you insist on using \LaTeX\ for \docstrip\ (really,
% \docstrip\ does not need \LaTeX), then inform the autodetect routine
% about your intention:
% \begin{quote}
%   \verb|latex \let\install=y% \iffalse meta-comment
%
% File: pagegrid.dtx
% Version: 2016/05/16 v1.5
% Info: Print page grid in background
%
% Copyright (C) 2009 by
%    Heiko Oberdiek <heiko.oberdiek at googlemail.com>
%    2016
%    https://github.com/ho-tex/oberdiek/issues
%
% This work may be distributed and/or modified under the
% conditions of the LaTeX Project Public License, either
% version 1.3c of this license or (at your option) any later
% version. This version of this license is in
%    https://www.latex-project.org/lppl/lppl-1-3c.txt
% and the latest version of this license is in
%    https://www.latex-project.org/lppl.txt
% and version 1.3 or later is part of all distributions of
% LaTeX version 2005/12/01 or later.
%
% This work has the LPPL maintenance status "maintained".
%
% The Current Maintainers of this work are
% Heiko Oberdiek and the Oberdiek Package Support Group
% https://github.com/ho-tex/oberdiek/issues
%
% This work consists of the main source file pagegrid.dtx
% and the derived files
%    pagegrid.sty, pagegrid.pdf, pagegrid.ins, pagegrid.drv,
%    pagegrid-test1.tex.
%
% Distribution:
%    CTAN:macros/latex/contrib/oberdiek/pagegrid.dtx
%    CTAN:macros/latex/contrib/oberdiek/pagegrid.pdf
%
% Unpacking:
%    (a) If pagegrid.ins is present:
%           tex pagegrid.ins
%    (b) Without pagegrid.ins:
%           tex pagegrid.dtx
%    (c) If you insist on using LaTeX
%           latex \let\install=y% \iffalse meta-comment
%
% File: pagegrid.dtx
% Version: 2016/05/16 v1.5
% Info: Print page grid in background
%
% Copyright (C) 2009 by
%    Heiko Oberdiek <heiko.oberdiek at googlemail.com>
%    2016
%    https://github.com/ho-tex/oberdiek/issues
%
% This work may be distributed and/or modified under the
% conditions of the LaTeX Project Public License, either
% version 1.3c of this license or (at your option) any later
% version. This version of this license is in
%    https://www.latex-project.org/lppl/lppl-1-3c.txt
% and the latest version of this license is in
%    https://www.latex-project.org/lppl.txt
% and version 1.3 or later is part of all distributions of
% LaTeX version 2005/12/01 or later.
%
% This work has the LPPL maintenance status "maintained".
%
% The Current Maintainers of this work are
% Heiko Oberdiek and the Oberdiek Package Support Group
% https://github.com/ho-tex/oberdiek/issues
%
% This work consists of the main source file pagegrid.dtx
% and the derived files
%    pagegrid.sty, pagegrid.pdf, pagegrid.ins, pagegrid.drv,
%    pagegrid-test1.tex.
%
% Distribution:
%    CTAN:macros/latex/contrib/oberdiek/pagegrid.dtx
%    CTAN:macros/latex/contrib/oberdiek/pagegrid.pdf
%
% Unpacking:
%    (a) If pagegrid.ins is present:
%           tex pagegrid.ins
%    (b) Without pagegrid.ins:
%           tex pagegrid.dtx
%    (c) If you insist on using LaTeX
%           latex \let\install=y\input{pagegrid.dtx}
%        (quote the arguments according to the demands of your shell)
%
% Documentation:
%    (a) If pagegrid.drv is present:
%           latex pagegrid.drv
%    (b) Without pagegrid.drv:
%           latex pagegrid.dtx; ...
%    The class ltxdoc loads the configuration file ltxdoc.cfg
%    if available. Here you can specify further options, e.g.
%    use A4 as paper format:
%       \PassOptionsToClass{a4paper}{article}
%
%    Programm calls to get the documentation (example):
%       pdflatex pagegrid.dtx
%       makeindex -s gind.ist pagegrid.idx
%       pdflatex pagegrid.dtx
%       makeindex -s gind.ist pagegrid.idx
%       pdflatex pagegrid.dtx
%
% Installation:
%    TDS:tex/latex/oberdiek/pagegrid.sty
%    TDS:doc/latex/oberdiek/pagegrid.pdf
%    TDS:doc/latex/oberdiek/test/pagegrid-test1.tex
%    TDS:source/latex/oberdiek/pagegrid.dtx
%
%<*ignore>
\begingroup
  \catcode123=1 %
  \catcode125=2 %
  \def\x{LaTeX2e}%
\expandafter\endgroup
\ifcase 0\ifx\install y1\fi\expandafter
         \ifx\csname processbatchFile\endcsname\relax\else1\fi
         \ifx\fmtname\x\else 1\fi\relax
\else\csname fi\endcsname
%</ignore>
%<*install>
\input docstrip.tex
\Msg{************************************************************************}
\Msg{* Installation}
\Msg{* Package: pagegrid 2016/05/16 v1.5 Print page grid in background (HO)}
\Msg{************************************************************************}

\keepsilent
\askforoverwritefalse

\let\MetaPrefix\relax
\preamble

This is a generated file.

Project: pagegrid
Version: 2016/05/16 v1.5

Copyright (C) 2009 by
   Heiko Oberdiek <heiko.oberdiek at googlemail.com>

This work may be distributed and/or modified under the
conditions of the LaTeX Project Public License, either
version 1.3c of this license or (at your option) any later
version. This version of this license is in
   https://www.latex-project.org/lppl/lppl-1-3c.txt
and the latest version of this license is in
   https://www.latex-project.org/lppl.txt
and version 1.3 or later is part of all distributions of
LaTeX version 2005/12/01 or later.

This work has the LPPL maintenance status "maintained".

The Current Maintainers of this work are
Heiko Oberdiek and the Oberdiek Package Support Group
https://github.com/ho-tex/oberdiek/issues


This work consists of the main source file pagegrid.dtx
and the derived files
   pagegrid.sty, pagegrid.pdf, pagegrid.ins, pagegrid.drv,
   pagegrid-test1.tex.

\endpreamble
\let\MetaPrefix\DoubleperCent

\generate{%
  \file{pagegrid.ins}{\from{pagegrid.dtx}{install}}%
  \file{pagegrid.drv}{\from{pagegrid.dtx}{driver}}%
  \usedir{tex/latex/oberdiek}%
  \file{pagegrid.sty}{\from{pagegrid.dtx}{package}}%
%  \usedir{doc/latex/oberdiek/test}%
%  \file{pagegrid-test1.tex}{\from{pagegrid.dtx}{test1}}%
  \nopreamble
  \nopostamble
%  \usedir{source/latex/oberdiek/catalogue}%
%  \file{pagegrid.xml}{\from{pagegrid.dtx}{catalogue}}%
}

\catcode32=13\relax% active space
\let =\space%
\Msg{************************************************************************}
\Msg{*}
\Msg{* To finish the installation you have to move the following}
\Msg{* file into a directory searched by TeX:}
\Msg{*}
\Msg{*     pagegrid.sty}
\Msg{*}
\Msg{* To produce the documentation run the file `pagegrid.drv'}
\Msg{* through LaTeX.}
\Msg{*}
\Msg{* Happy TeXing!}
\Msg{*}
\Msg{************************************************************************}

\endbatchfile
%</install>
%<*ignore>
\fi
%</ignore>
%<*driver>
\NeedsTeXFormat{LaTeX2e}
\ProvidesFile{pagegrid.drv}%
  [2016/05/16 v1.5 Print page grid in background (HO)]%
\documentclass{ltxdoc}
\usepackage{holtxdoc}[2011/11/22]
\begin{document}
  \DocInput{pagegrid.dtx}%
\end{document}
%</driver>
% \fi
%
%
% \CharacterTable
%  {Upper-case    \A\B\C\D\E\F\G\H\I\J\K\L\M\N\O\P\Q\R\S\T\U\V\W\X\Y\Z
%   Lower-case    \a\b\c\d\e\f\g\h\i\j\k\l\m\n\o\p\q\r\s\t\u\v\w\x\y\z
%   Digits        \0\1\2\3\4\5\6\7\8\9
%   Exclamation   \!     Double quote  \"     Hash (number) \#
%   Dollar        \$     Percent       \%     Ampersand     \&
%   Acute accent  \'     Left paren    \(     Right paren   \)
%   Asterisk      \*     Plus          \+     Comma         \,
%   Minus         \-     Point         \.     Solidus       \/
%   Colon         \:     Semicolon     \;     Less than     \<
%   Equals        \=     Greater than  \>     Question mark \?
%   Commercial at \@     Left bracket  \[     Backslash     \\
%   Right bracket \]     Circumflex    \^     Underscore    \_
%   Grave accent  \`     Left brace    \{     Vertical bar  \|
%   Right brace   \}     Tilde         \~}
%
% \GetFileInfo{pagegrid.drv}
%
% \title{The \xpackage{pagegrid} package}
% \date{2016/05/16 v1.5}
% \author{Heiko Oberdiek\thanks
% {Please report any issues at \url{https://github.com/ho-tex/oberdiek/issues}}}
%
% \maketitle
%
% \begin{abstract}
% The \LaTeX\ package prints a page grid in the background.
% \end{abstract}
%
% \tableofcontents
%
% \section{Documentation}
%
% The package puts a grid on the paper. It was written for
% developers of a class or package
% who have to put elements on definite locations on a page
% (e.g. letter class). The grid allows a faster optical check,
% whether the positions are correct. If the previewer already
% offers features for measuring, the package might be obsolete.
% Otherwise it saves the developer from printing the page and
% measuring by hand.
%
% \subsection{Options}
%
% Options are evaluated in the following order:
% \begin{enumerate}
% \item
%  Configuration file \xfile{pagegrid.cfg} using \cs{pagegridsetup}
%  if the file exists.
%  \item
%  Package options given for \cs{usepackage}.
%  \item
%  Later calls of \cs{pagegridsetup}.
% \end{enumerate}
% \begin{declcs}{pagegridsetup}\M{option list}
% \end{declcs}
% The options are key value options. Boolean options are enabled by
% default (without value) or by using the explicit value \texttt{true}.
% Value \texttt{false} disable the option.
%
% \subsubsection{Options \xoption{enable}, \xoption{disable}}
%
% \begin{description}
% \item[\xoption{enable}:] This boolean option controls whether the page grid
%   is drawn. As default the page grid drawing is activated.
% \item[\xoption{disable}:] It is the opposite
%   of option \xoption{enable}. It was added for convenience and
%   allows the abbreviation \texttt{disable} for \texttt{enable=false}.
% \end{description}
%
% \subsubsection{Grid origins}
%
% The package supports up to two grids on a page allowing
% measurement from opposite directions. As default two grids are drawn,
% the first from bottom left to top right. The origin of the second
% grid is at the opposite top right corner.
% The origins are controlled by the following options.
% The number of grids (one or two) depend on the number of these options
% in one call of \cs{pagegridsetup}.
% The following frame shows a paper and in its corners are the
% corresponding options. At the left and right side alias names
% are given for the options inside the paper.
% \begin{quote}
% \begin{tabular}{@{}r|@{\,}l@{\qquad}r@{\,}|l@{}}
% \cline{2-3}
% \xoption{left-top}, \xoption{lt}, \xoption{top-left}
% & \vphantom{\"U}\xoption{tl} & \xoption{tr}
% & \xoption{top-right}, \xoption{rt}, \xoption{right-top}\\
% &&&\\
% \xoption{left-bottom}, \xoption{lb}, \xoption{bottom-left}
% & \xoption{bl} & \xoption{br}
% & \xoption{bottom-right}, \xoption{rb}, \xoption{right-bottom}\\
% \cline{2-3}
% \end{tabular}
% \end{quote}
% Examples:
% \begin{quote}
% |\pagegridsetup{bl,tr}|
% \end{quote}
% This is the default setting with two grids as described previously.
% The following setups one grid only. Its origin is the upper left
% corner:
% \begin{quote}
% |\pagegridsetup{top-left}|
% \end{quote}
%
% \subsubsection{Grid unit}
%
% \begin{description}
% \item[\xoption{step}] This option takes a length and
% setups the unit for the grid. The page width and page height
% should be multiples of this unit.
% Currently the default is \texttt{1mm}. But this might change
% later by a heuristic based on the paper size.
% \end{description}
%
% \subsubsection{Color options}
%
% The basic grid lines are drawn as ultra thin help lines and is only
% drawn for the first grid.
% Each tenth and fiftyth line of the basic net is drawn thicker in a special
% color for the two grids.
% \begin{description}
% \item[\xoption{firstcolor}:] Color for the thicker lines and the arrows
% of the first grid. Default value is \texttt{red}.
% \item[\xoption{secondcolor}:] Color for the thicker lines and the arrows
% of the second grid. Default value is \texttt{blue}.
% \end{description}
% Use a color specification that package \xpackage{tikz} understands.
% (The grid is drawn with \xpackage{pgf}/\xpackage{tikz}.)
%
% \subsubsection{Arrow options}
%
% Arrows are put at the origin at the grid to show the grid start
% and the direction of the grid.
% \begin{description}
% \item[\xoption{arrows}:] This boolean option turns the arrows on or off.
% As default arrows are enabled.
% \item[\xoption{arrowlength}:] The length given as value is the
% length of the edge of a square at the origin within the
% arrow is put as diagonal. Default is 10 times the grid unit (10\,mm).
% The real arrow length is this length multiplied by $\sqrt2$.
% \end{description}
%
% \subsubsection{Miscellaneous options}
%
% \begin{description}
% \item[\xoption{double}:] The output page is doubled, one without page
% grid and the other with page grid. Possible values are shown in the
% following table:
% \begin{quote}
% \begin{tabular}{ll}
% Option & Meaning\\
% \hline
% |false| & Turns option off.\\
% |first| & Grid page comes first.\\
% |last| & Grid page comes after the page without grid.\\
% |true| & Same as |last|.\\
% \meta{no value} & Same as |true|.\\
% \end{tabular}
% \end{quote}
% \textbf{Note:}
% The double output of the page has side effects.
% All whatits are executed twice, for example: file writing
% and anchor setting. Some unwanted actions are catched such
% as multiple \cs{label} definitions, duplicate entries in
% the table of contents. For bookmarks, use package \xpackage{bookmarks}.
% \item[\xoption{foreground}:] Boolean option, default is \texttt{false}.
% Sometimes there might be elements on the page (e.g. large images)
% that hide the grid. Then option \xoption{foreground} puts the grids
% over the current output page.
% \end{description}
%
% \StopEventually{
% }
%
% \section{Implementation}
%    \begin{macrocode}
%<*package>
%    \end{macrocode}
%    Reload check, especially if the package is not used with \LaTeX.
%    \begin{macrocode}
\begingroup\catcode61\catcode48\catcode32=10\relax%
  \catcode13=5 % ^^M
  \endlinechar=13 %
  \catcode35=6 % #
  \catcode39=12 % '
  \catcode44=12 % ,
  \catcode45=12 % -
  \catcode46=12 % .
  \catcode58=12 % :
  \catcode64=11 % @
  \catcode123=1 % {
  \catcode125=2 % }
  \expandafter\let\expandafter\x\csname ver@pagegrid.sty\endcsname
  \ifx\x\relax % plain-TeX, first loading
  \else
    \def\empty{}%
    \ifx\x\empty % LaTeX, first loading,
      % variable is initialized, but \ProvidesPackage not yet seen
    \else
      \expandafter\ifx\csname PackageInfo\endcsname\relax
        \def\x#1#2{%
          \immediate\write-1{Package #1 Info: #2.}%
        }%
      \else
        \def\x#1#2{\PackageInfo{#1}{#2, stopped}}%
      \fi
      \x{pagegrid}{The package is already loaded}%
      \aftergroup\endinput
    \fi
  \fi
\endgroup%
%    \end{macrocode}
%    Package identification:
%    \begin{macrocode}
\begingroup\catcode61\catcode48\catcode32=10\relax%
  \catcode13=5 % ^^M
  \endlinechar=13 %
  \catcode35=6 % #
  \catcode39=12 % '
  \catcode40=12 % (
  \catcode41=12 % )
  \catcode44=12 % ,
  \catcode45=12 % -
  \catcode46=12 % .
  \catcode47=12 % /
  \catcode58=12 % :
  \catcode64=11 % @
  \catcode91=12 % [
  \catcode93=12 % ]
  \catcode123=1 % {
  \catcode125=2 % }
  \expandafter\ifx\csname ProvidesPackage\endcsname\relax
    \def\x#1#2#3[#4]{\endgroup
      \immediate\write-1{Package: #3 #4}%
      \xdef#1{#4}%
    }%
  \else
    \def\x#1#2[#3]{\endgroup
      #2[{#3}]%
      \ifx#1\@undefined
        \xdef#1{#3}%
      \fi
      \ifx#1\relax
        \xdef#1{#3}%
      \fi
    }%
  \fi
\expandafter\x\csname ver@pagegrid.sty\endcsname
\ProvidesPackage{pagegrid}%
  [2016/05/16 v1.5 Print page grid in background (HO)]%
%    \end{macrocode}
%
%    \begin{macrocode}
\begingroup\catcode61\catcode48\catcode32=10\relax%
  \catcode13=5 % ^^M
  \endlinechar=13 %
  \catcode123=1 % {
  \catcode125=2 % }
  \catcode64=11 % @
  \def\x{\endgroup
    \expandafter\edef\csname pagegrid@AtEnd\endcsname{%
      \endlinechar=\the\endlinechar\relax
      \catcode13=\the\catcode13\relax
      \catcode32=\the\catcode32\relax
      \catcode35=\the\catcode35\relax
      \catcode61=\the\catcode61\relax
      \catcode64=\the\catcode64\relax
      \catcode123=\the\catcode123\relax
      \catcode125=\the\catcode125\relax
    }%
  }%
\x\catcode61\catcode48\catcode32=10\relax%
\catcode13=5 % ^^M
\endlinechar=13 %
\catcode35=6 % #
\catcode64=11 % @
\catcode123=1 % {
\catcode125=2 % }
\def\TMP@EnsureCode#1#2{%
  \edef\pagegrid@AtEnd{%
    \pagegrid@AtEnd
    \catcode#1=\the\catcode#1\relax
  }%
  \catcode#1=#2\relax
}
\TMP@EnsureCode{9}{10}% (tab)
\TMP@EnsureCode{10}{12}% ^^J
\TMP@EnsureCode{33}{12}% !
\TMP@EnsureCode{34}{12}% "
\TMP@EnsureCode{36}{3}% $
\TMP@EnsureCode{38}{4}% &
\TMP@EnsureCode{39}{12}% '
\TMP@EnsureCode{40}{12}% (
\TMP@EnsureCode{41}{12}% )
\TMP@EnsureCode{42}{12}% *
\TMP@EnsureCode{43}{12}% +
\TMP@EnsureCode{44}{12}% ,
\TMP@EnsureCode{45}{12}% -
\TMP@EnsureCode{46}{12}% .
\TMP@EnsureCode{47}{12}% /
\TMP@EnsureCode{58}{12}% :
\TMP@EnsureCode{59}{12}% ;
\TMP@EnsureCode{60}{12}% <
\TMP@EnsureCode{62}{12}% >
\TMP@EnsureCode{63}{12}% ?
\TMP@EnsureCode{91}{12}% [
\TMP@EnsureCode{93}{12}% ]
\TMP@EnsureCode{94}{7}% ^ (superscript)
\TMP@EnsureCode{95}{8}% _ (subscript)
\TMP@EnsureCode{96}{12}% `
\TMP@EnsureCode{124}{12}% |
\edef\pagegrid@AtEnd{\pagegrid@AtEnd\noexpand\endinput}
%    \end{macrocode}
%
%    \begin{macrocode}
\RequirePackage{tikz}
\RequirePackage{atbegshi}[2009/12/02]
\RequirePackage{kvoptions}[2009/07/17]
%    \end{macrocode}
%    \begin{macrocode}
\begingroup\expandafter\expandafter\expandafter\endgroup
\expandafter\ifx\csname stockwidth\endcsname\relax
  \def\pagegrid@width{\paperwidth}%
  \def\pagegrid@height{\paperheight}%
\else
  \def\pagegrid@width{\stockwidth}%
  \def\pagegrid@height{\stockheight}%
\fi
%    \end{macrocode}
%
%    \begin{macrocode}
\SetupKeyvalOptions{%
  family=pagegrid,%
  prefix=pagegrid@,%
}
\def\pagegrid@init{%
  \let\pagegrid@origin@a\@empty
  \let\pagegrid@origin@b\@empty
  \let\pagegrid@init\relax
}
\let\pagegrid@@init\pagegrid@init
\def\pagegrid@origin@a{bl}
\def\pagegrid@origin@b{tr}
\def\pagegrid@SetOrigin#1{%
  \pagegrid@init
  \ifx\pagegrid@origin@a\@empty
    \def\pagegrid@origin@a{#1}%
  \else
    \ifx\pagegrid@origin@b\@empty
    \else
      \let\pagegrid@origin@a\pagegrid@origin@b
    \fi
    \def\pagegrid@origin@b{#1}%
  \fi
}
\def\pagegrid@temp#1{%
  \DeclareVoidOption{#1}{\pagegrid@SetOrigin{#1}}%
  \@namedef{pagegrid@N@#1}{#1}%
}
\pagegrid@temp{bl}
\pagegrid@temp{br}
\pagegrid@temp{tl}
\pagegrid@temp{tr}
\def\pagegrid@temp#1#2{%
  \DeclareVoidOption{#2}{\pagegrid@SetOrigin{#1}}%
}%
\pagegrid@temp{bl}{lb}
\pagegrid@temp{br}{rb}
\pagegrid@temp{tl}{lt}
\pagegrid@temp{tr}{rt}
\pagegrid@temp{bl}{bottom-left}
\pagegrid@temp{br}{bottom-right}
\pagegrid@temp{tl}{top-left}
\pagegrid@temp{tr}{top-right}
\pagegrid@temp{bl}{left-bottom}
\pagegrid@temp{br}{right-bottom}
\pagegrid@temp{tl}{left-top}
\pagegrid@temp{tr}{right-top}
%    \end{macrocode}
%    \begin{macrocode}
\DeclareBoolOption[true]{enable}
\DeclareComplementaryOption{disable}{enable}
%    \end{macrocode}
%    \begin{macrocode}
\DeclareBoolOption{foreground}
%    \end{macrocode}
%    \begin{macrocode}
\newlength{\pagegrid@step}
\define@key{pagegrid}{step}{%
  \setlength{\pagegrid@step}{#1}%
}
%    \end{macrocode}
%    \begin{macrocode}
\DeclareStringOption[red]{firstcolor}
\DeclareStringOption[blue]{secondcolor}
%    \end{macrocode}
%    \begin{macrocode}
\DeclareBoolOption[true]{arrows}
\newlength\pagegrid@arrowlength
\pagegrid@arrowlength=\z@
\define@key{pagegrid}{arrowlength}{%
  \setlength{\pagegrid@arrowlength}{#1}%
}
%    \end{macrocode}
%    \begin{macrocode}
\define@key{pagegrid}{double}[true]{%
  \@ifundefined{pagegrid@double@#1}{%
    \PackageWarning{pagegrid}{%
      Unsupported value `#1' for option `double'.\MessageBreak
      Known values are:\MessageBreak
      `false', `first', `last', `true'.\MessageBreak
      Now `false' is used%
    }%
    \chardef\pagegrid@double\z@
  }{%
    \chardef\pagegrid@double\csname pagegrid@double@#1\endcsname\relax
  }%
}
\@namedef{pagegrid@double@false}{0}
\@namedef{pagegrid@double@first}{1}
\@namedef{pagegrid@double@last}{2}
\@namedef{pagegrid@double@true}{2}
\chardef\pagegrid@double\z@
%    \end{macrocode}
%    \begin{macrocode}
\newcommand*{\pagegridsetup}{%
  \let\pagegrid@init\pagegrid@@init
  \setkeys{pagegrid}%
}
%    \end{macrocode}
%    \begin{macrocode}
\pagegridsetup{%
  step=1mm%
}
\InputIfFileExists{pagegrid.cfg}{}%
\ProcessKeyvalOptions*\relax
\AtBeginDocument{%
  \ifdim\pagegrid@arrowlength>\z@
  \else
    \pagegrid@arrowlength=10\pagegrid@step
  \fi
}
%    \end{macrocode}
%
%    \begin{macrocode}
\def\pagegridShipoutDoubleBegin{%
  \begingroup
  \let\newlabel\@gobbletwo
  \let\zref@newlabel\@gobbletwo
  \let\@writefile\@gobbletwo
  \let\select@language\@gobble
}
\def\pagegridShipoutDoubleEnd{%
  \endgroup
}
\def\pagegrid@WriteDouble#1#2{%
  \immediate\write#1{%
    \@backslashchar csname %
    pagegridShipoutDouble#2%
    \@backslashchar endcsname%
  }%
}
\def\pagegrid@ShipoutDouble#1{%
  \begingroup
    \if@filesw
      \pagegrid@WriteDouble\@mainaux{Begin}%
      \ifx\@auxout\@partaux
        \pagegrid@WriteDouble\@partaux{Begin}%
        \def\pagegrid@temp{%
          \pagegrid@WriteDouble\@mainaux{End}%
          \pagegrid@WriteDouble\@partaux{End}%
        }%
      \else
        \def\pagegrid@temp{%
          \pagegrid@WriteDouble\@mainaux{End}%
        }%
      \fi
    \else
      \def\pagegrid@temp{}%
    \fi
    \let\protect\noexpand
    \AtBeginShipoutOriginalShipout\copy#1\relax
    \pagegrid@temp
  \endgroup
}
%    \end{macrocode}
%
%    \begin{macrocode}
\AtBeginShipout{%
  \ifdim\pagegrid@step>\z@
  \else
    \pagegrid@enablefalse
  \fi
  \ifpagegrid@enable
    \ifnum\pagegrid@double=\@ne
      \pagegrid@ShipoutDouble\AtBeginShipoutBox
    \else
      \ifnum\pagegrid@double=\tw@
        \@ifundefined{pagegrid@DoubleBox}{%
          \newbox\pagegrid@DoubleBox
        }{}%
        \setbox\pagegrid@DoubleBox=\copy\AtBeginShipoutBox
      \fi
    \fi
    \ifpagegrid@foreground
      \expandafter\AtBeginShipoutUpperLeftForeground
    \else
      \expandafter\AtBeginShipoutUpperLeft
    \fi
    {%
      \put(0,0){%
        \makebox(0,0)[lt]{%
          \begin{tikzpicture}[%
            bl/.style={},%
            br/.style={xshift=\pagegrid@width,xscale=-1},%
            tl/.style={yshift=\pagegrid@height,yscale=-1},%
            tr/.style={xshift=\pagegrid@width,%
                       yshift=\pagegrid@height,scale=-1}%
          ]%
            \useasboundingbox
              (0mm,\pagegrid@height) rectangle (0mm,\pagegrid@height);%
            \draw[%
              \pagegrid@origin@a,%
              step=\pagegrid@step,%
              style=help lines,%
              ultra thin%
            ] (0mm,0mm) grid (\pagegrid@width,\pagegrid@height);%
            \ifx\pagegrid@origin@b\@empty
            \else
              \draw[%
                \pagegrid@origin@b,%
                step=10\pagegrid@step,%
                {\pagegrid@secondcolor},%
                very thin%
              ] (0mm,0mm) grid (\pagegrid@width,\pagegrid@height);%
            \fi
            \draw[%
               \pagegrid@origin@a,%
               step=10\pagegrid@step,%
               {\pagegrid@firstcolor},%
               very thin%
            ] (0mm,0mm) grid (\pagegrid@width,\pagegrid@height);%
            \ifx\pagegrid@origin@b\@empty
            \else
              \draw[%
                \pagegrid@origin@b,%
                step=50\pagegrid@step,%
                {\pagegrid@secondcolor},%
                thick%
              ] (0mm,0mm) grid (\pagegrid@width,\pagegrid@height);%
            \fi
            \draw[%
              \pagegrid@origin@a,%
              step=50\pagegrid@step,%
              {\pagegrid@firstcolor},%
              thick%
            ] (0mm,0mm) grid (\pagegrid@width,\pagegrid@height);%
            \ifpagegrid@arrows
              \ifx\pagegrid@origin@b\@empty
              \else
                \draw[%
                  \pagegrid@origin@b,%
                  {\pagegrid@secondcolor},%
                  stroke,%
                  line width=1pt,%
                  line cap=round%
                ] (0mm,0mm) %
                -- (\pagegrid@arrowlength,\pagegrid@arrowlength) %
                   (\pagegrid@arrowlength,.5\pagegrid@arrowlength) %
                -- (\pagegrid@arrowlength,\pagegrid@arrowlength) %
                -- (.5\pagegrid@arrowlength,\pagegrid@arrowlength);%
              \fi
              \draw[%
                \pagegrid@origin@a,%
                {\pagegrid@firstcolor},%
                stroke,%
                line width=1pt,%
                line cap=round%
              ] (0mm,0mm) %
              -- (\pagegrid@arrowlength,\pagegrid@arrowlength) %
                 (\pagegrid@arrowlength,.5\pagegrid@arrowlength) %
              -- (\pagegrid@arrowlength,\pagegrid@arrowlength) %
              -- (.5\pagegrid@arrowlength,\pagegrid@arrowlength);%
            \fi
          \end{tikzpicture}%
        }%
      }%
    }%
    \ifnum\pagegrid@double=\tw@
      \pagegrid@ShipoutDouble\pagegrid@DoubleBox
    \fi
  \fi
}
%    \end{macrocode}
%
%    \begin{macrocode}
\pagegrid@AtEnd%
%</package>
%    \end{macrocode}
%
% \section{Test}
%
% \subsection{Catcode checks for loading}
%
%    \begin{macrocode}
%<*test1>
%    \end{macrocode}
%    \begin{macrocode}
\catcode`\{=1 %
\catcode`\}=2 %
\catcode`\#=6 %
\catcode`\@=11 %
\expandafter\ifx\csname count@\endcsname\relax
  \countdef\count@=255 %
\fi
\expandafter\ifx\csname @gobble\endcsname\relax
  \long\def\@gobble#1{}%
\fi
\expandafter\ifx\csname @firstofone\endcsname\relax
  \long\def\@firstofone#1{#1}%
\fi
\expandafter\ifx\csname loop\endcsname\relax
  \expandafter\@firstofone
\else
  \expandafter\@gobble
\fi
{%
  \def\loop#1\repeat{%
    \def\body{#1}%
    \iterate
  }%
  \def\iterate{%
    \body
      \let\next\iterate
    \else
      \let\next\relax
    \fi
    \next
  }%
  \let\repeat=\fi
}%
\def\RestoreCatcodes{}
\count@=0 %
\loop
  \edef\RestoreCatcodes{%
    \RestoreCatcodes
    \catcode\the\count@=\the\catcode\count@\relax
  }%
\ifnum\count@<255 %
  \advance\count@ 1 %
\repeat

\def\RangeCatcodeInvalid#1#2{%
  \count@=#1\relax
  \loop
    \catcode\count@=15 %
  \ifnum\count@<#2\relax
    \advance\count@ 1 %
  \repeat
}
\def\RangeCatcodeCheck#1#2#3{%
  \count@=#1\relax
  \loop
    \ifnum#3=\catcode\count@
    \else
      \errmessage{%
        Character \the\count@\space
        with wrong catcode \the\catcode\count@\space
        instead of \number#3%
      }%
    \fi
  \ifnum\count@<#2\relax
    \advance\count@ 1 %
  \repeat
}
\def\space{ }
\expandafter\ifx\csname LoadCommand\endcsname\relax
  \def\LoadCommand{\input pagegrid.sty\relax}%
\fi
\def\Test{%
  \RangeCatcodeInvalid{0}{47}%
  \RangeCatcodeInvalid{58}{64}%
  \RangeCatcodeInvalid{91}{96}%
  \RangeCatcodeInvalid{123}{255}%
  \catcode`\@=12 %
  \catcode`\\=0 %
  \catcode`\%=14 %
  \LoadCommand
  \RangeCatcodeCheck{0}{36}{15}%
  \RangeCatcodeCheck{37}{37}{14}%
  \RangeCatcodeCheck{38}{47}{15}%
  \RangeCatcodeCheck{48}{57}{12}%
  \RangeCatcodeCheck{58}{63}{15}%
  \RangeCatcodeCheck{64}{64}{12}%
  \RangeCatcodeCheck{65}{90}{11}%
  \RangeCatcodeCheck{91}{91}{15}%
  \RangeCatcodeCheck{92}{92}{0}%
  \RangeCatcodeCheck{93}{96}{15}%
  \RangeCatcodeCheck{97}{122}{11}%
  \RangeCatcodeCheck{123}{255}{15}%
  \RestoreCatcodes
}
\Test
\csname @@end\endcsname
\end
%    \end{macrocode}
%    \begin{macrocode}
%</test1>
%    \end{macrocode}
%
% \section{Installation}
%
% \subsection{Download}
%
% \paragraph{Package.} This package is available on
% CTAN\footnote{\CTANpkg{pagegrid}}:
% \begin{description}
% \item[\CTAN{macros/latex/contrib/oberdiek/pagegrid.dtx}] The source file.
% \item[\CTAN{macros/latex/contrib/oberdiek/pagegrid.pdf}] Documentation.
% \end{description}
%
%
% \paragraph{Bundle.} All the packages of the bundle `oberdiek'
% are also available in a TDS compliant ZIP archive. There
% the packages are already unpacked and the documentation files
% are generated. The files and directories obey the TDS standard.
% \begin{description}
% \item[\CTANinstall{install/macros/latex/contrib/oberdiek.tds.zip}]
% \end{description}
% \emph{TDS} refers to the standard ``A Directory Structure
% for \TeX\ Files'' (\CTAN{tds/tds.pdf}). Directories
% with \xfile{texmf} in their name are usually organized this way.
%
% \subsection{Bundle installation}
%
% \paragraph{Unpacking.} Unpack the \xfile{oberdiek.tds.zip} in the
% TDS tree (also known as \xfile{texmf} tree) of your choice.
% Example (linux):
% \begin{quote}
%   |unzip oberdiek.tds.zip -d ~/texmf|
% \end{quote}
%
% \paragraph{Script installation.}
% Check the directory \xfile{TDS:scripts/oberdiek/} for
% scripts that need further installation steps.
%
% \subsection{Package installation}
%
% \paragraph{Unpacking.} The \xfile{.dtx} file is a self-extracting
% \docstrip\ archive. The files are extracted by running the
% \xfile{.dtx} through \plainTeX:
% \begin{quote}
%   \verb|tex pagegrid.dtx|
% \end{quote}
%
% \paragraph{TDS.} Now the different files must be moved into
% the different directories in your installation TDS tree
% (also known as \xfile{texmf} tree):
% \begin{quote}
% \def\t{^^A
% \begin{tabular}{@{}>{\ttfamily}l@{ $\rightarrow$ }>{\ttfamily}l@{}}
%   pagegrid.sty & tex/latex/oberdiek/pagegrid.sty\\
%   pagegrid.pdf & doc/latex/oberdiek/pagegrid.pdf\\
%   test/pagegrid-test1.tex & doc/latex/oberdiek/test/pagegrid-test1.tex\\
%   pagegrid.dtx & source/latex/oberdiek/pagegrid.dtx\\
% \end{tabular}^^A
% }^^A
% \sbox0{\t}^^A
% \ifdim\wd0>\linewidth
%   \begingroup
%     \advance\linewidth by\leftmargin
%     \advance\linewidth by\rightmargin
%   \edef\x{\endgroup
%     \def\noexpand\lw{\the\linewidth}^^A
%   }\x
%   \def\lwbox{^^A
%     \leavevmode
%     \hbox to \linewidth{^^A
%       \kern-\leftmargin\relax
%       \hss
%       \usebox0
%       \hss
%       \kern-\rightmargin\relax
%     }^^A
%   }^^A
%   \ifdim\wd0>\lw
%     \sbox0{\small\t}^^A
%     \ifdim\wd0>\linewidth
%       \ifdim\wd0>\lw
%         \sbox0{\footnotesize\t}^^A
%         \ifdim\wd0>\linewidth
%           \ifdim\wd0>\lw
%             \sbox0{\scriptsize\t}^^A
%             \ifdim\wd0>\linewidth
%               \ifdim\wd0>\lw
%                 \sbox0{\tiny\t}^^A
%                 \ifdim\wd0>\linewidth
%                   \lwbox
%                 \else
%                   \usebox0
%                 \fi
%               \else
%                 \lwbox
%               \fi
%             \else
%               \usebox0
%             \fi
%           \else
%             \lwbox
%           \fi
%         \else
%           \usebox0
%         \fi
%       \else
%         \lwbox
%       \fi
%     \else
%       \usebox0
%     \fi
%   \else
%     \lwbox
%   \fi
% \else
%   \usebox0
% \fi
% \end{quote}
% If you have a \xfile{docstrip.cfg} that configures and enables \docstrip's
% TDS installing feature, then some files can already be in the right
% place, see the documentation of \docstrip.
%
% \subsection{Refresh file name databases}
%
% If your \TeX~distribution
% (\TeX\,Live, \mikTeX, \dots) relies on file name databases, you must refresh
% these. For example, \TeX\,Live\ users run \verb|texhash| or
% \verb|mktexlsr|.
%
% \subsection{Some details for the interested}
%
% \paragraph{Unpacking with \LaTeX.}
% The \xfile{.dtx} chooses its action depending on the format:
% \begin{description}
% \item[\plainTeX:] Run \docstrip\ and extract the files.
% \item[\LaTeX:] Generate the documentation.
% \end{description}
% If you insist on using \LaTeX\ for \docstrip\ (really,
% \docstrip\ does not need \LaTeX), then inform the autodetect routine
% about your intention:
% \begin{quote}
%   \verb|latex \let\install=y\input{pagegrid.dtx}|
% \end{quote}
% Do not forget to quote the argument according to the demands
% of your shell.
%
% \paragraph{Generating the documentation.}
% You can use both the \xfile{.dtx} or the \xfile{.drv} to generate
% the documentation. The process can be configured by the
% configuration file \xfile{ltxdoc.cfg}. For instance, put this
% line into this file, if you want to have A4 as paper format:
% \begin{quote}
%   \verb|\PassOptionsToClass{a4paper}{article}|
% \end{quote}
% An example follows how to generate the
% documentation with pdf\LaTeX:
% \begin{quote}
%\begin{verbatim}
%pdflatex pagegrid.dtx
%makeindex -s gind.ist pagegrid.idx
%pdflatex pagegrid.dtx
%makeindex -s gind.ist pagegrid.idx
%pdflatex pagegrid.dtx
%\end{verbatim}
% \end{quote}
%
% \section{Acknowledgement}
%
% \begin{description}
% \item[Klaus Braune:]
%  He provided the idea and the first \xpackage{tikz} code.
% \end{description}
%
% \begin{History}
%   \begin{Version}{2009/11/06 v1.0}
%   \item
%     The first version.
%   \end{Version}
%   \begin{Version}{2009/11/06 v1.1}
%   \item
%     Option \xoption{foreground} added.
%   \end{Version}
%   \begin{Version}{2009/12/02 v1.2}
%   \item
%     Color options, arrow options added.
%   \item
%     Names for origin options changed.
%   \end{Version}
%   \begin{Version}{2009/12/03 v1.3}
%   \item
%     Option \xoption{double} added.
%   \item
%     First CTAN release.
%   \end{Version}
%   \begin{Version}{2009/12/04 v1.4}
%   \item
%     Option \xoption{double}: Some unwanted side effects removed.
%   \end{Version}
%   \begin{Version}{2016/05/16 v1.5}
%   \item
%     Documentation updates.
%   \end{Version}
% \end{History}
%
% \PrintIndex
%
% \Finale
\endinput

%        (quote the arguments according to the demands of your shell)
%
% Documentation:
%    (a) If pagegrid.drv is present:
%           latex pagegrid.drv
%    (b) Without pagegrid.drv:
%           latex pagegrid.dtx; ...
%    The class ltxdoc loads the configuration file ltxdoc.cfg
%    if available. Here you can specify further options, e.g.
%    use A4 as paper format:
%       \PassOptionsToClass{a4paper}{article}
%
%    Programm calls to get the documentation (example):
%       pdflatex pagegrid.dtx
%       makeindex -s gind.ist pagegrid.idx
%       pdflatex pagegrid.dtx
%       makeindex -s gind.ist pagegrid.idx
%       pdflatex pagegrid.dtx
%
% Installation:
%    TDS:tex/latex/oberdiek/pagegrid.sty
%    TDS:doc/latex/oberdiek/pagegrid.pdf
%    TDS:doc/latex/oberdiek/test/pagegrid-test1.tex
%    TDS:source/latex/oberdiek/pagegrid.dtx
%
%<*ignore>
\begingroup
  \catcode123=1 %
  \catcode125=2 %
  \def\x{LaTeX2e}%
\expandafter\endgroup
\ifcase 0\ifx\install y1\fi\expandafter
         \ifx\csname processbatchFile\endcsname\relax\else1\fi
         \ifx\fmtname\x\else 1\fi\relax
\else\csname fi\endcsname
%</ignore>
%<*install>
\input docstrip.tex
\Msg{************************************************************************}
\Msg{* Installation}
\Msg{* Package: pagegrid 2016/05/16 v1.5 Print page grid in background (HO)}
\Msg{************************************************************************}

\keepsilent
\askforoverwritefalse

\let\MetaPrefix\relax
\preamble

This is a generated file.

Project: pagegrid
Version: 2016/05/16 v1.5

Copyright (C) 2009 by
   Heiko Oberdiek <heiko.oberdiek at googlemail.com>

This work may be distributed and/or modified under the
conditions of the LaTeX Project Public License, either
version 1.3c of this license or (at your option) any later
version. This version of this license is in
   https://www.latex-project.org/lppl/lppl-1-3c.txt
and the latest version of this license is in
   https://www.latex-project.org/lppl.txt
and version 1.3 or later is part of all distributions of
LaTeX version 2005/12/01 or later.

This work has the LPPL maintenance status "maintained".

The Current Maintainers of this work are
Heiko Oberdiek and the Oberdiek Package Support Group
https://github.com/ho-tex/oberdiek/issues


This work consists of the main source file pagegrid.dtx
and the derived files
   pagegrid.sty, pagegrid.pdf, pagegrid.ins, pagegrid.drv,
   pagegrid-test1.tex.

\endpreamble
\let\MetaPrefix\DoubleperCent

\generate{%
  \file{pagegrid.ins}{\from{pagegrid.dtx}{install}}%
  \file{pagegrid.drv}{\from{pagegrid.dtx}{driver}}%
  \usedir{tex/latex/oberdiek}%
  \file{pagegrid.sty}{\from{pagegrid.dtx}{package}}%
%  \usedir{doc/latex/oberdiek/test}%
%  \file{pagegrid-test1.tex}{\from{pagegrid.dtx}{test1}}%
  \nopreamble
  \nopostamble
%  \usedir{source/latex/oberdiek/catalogue}%
%  \file{pagegrid.xml}{\from{pagegrid.dtx}{catalogue}}%
}

\catcode32=13\relax% active space
\let =\space%
\Msg{************************************************************************}
\Msg{*}
\Msg{* To finish the installation you have to move the following}
\Msg{* file into a directory searched by TeX:}
\Msg{*}
\Msg{*     pagegrid.sty}
\Msg{*}
\Msg{* To produce the documentation run the file `pagegrid.drv'}
\Msg{* through LaTeX.}
\Msg{*}
\Msg{* Happy TeXing!}
\Msg{*}
\Msg{************************************************************************}

\endbatchfile
%</install>
%<*ignore>
\fi
%</ignore>
%<*driver>
\NeedsTeXFormat{LaTeX2e}
\ProvidesFile{pagegrid.drv}%
  [2016/05/16 v1.5 Print page grid in background (HO)]%
\documentclass{ltxdoc}
\usepackage{holtxdoc}[2011/11/22]
\begin{document}
  \DocInput{pagegrid.dtx}%
\end{document}
%</driver>
% \fi
%
%
% \CharacterTable
%  {Upper-case    \A\B\C\D\E\F\G\H\I\J\K\L\M\N\O\P\Q\R\S\T\U\V\W\X\Y\Z
%   Lower-case    \a\b\c\d\e\f\g\h\i\j\k\l\m\n\o\p\q\r\s\t\u\v\w\x\y\z
%   Digits        \0\1\2\3\4\5\6\7\8\9
%   Exclamation   \!     Double quote  \"     Hash (number) \#
%   Dollar        \$     Percent       \%     Ampersand     \&
%   Acute accent  \'     Left paren    \(     Right paren   \)
%   Asterisk      \*     Plus          \+     Comma         \,
%   Minus         \-     Point         \.     Solidus       \/
%   Colon         \:     Semicolon     \;     Less than     \<
%   Equals        \=     Greater than  \>     Question mark \?
%   Commercial at \@     Left bracket  \[     Backslash     \\
%   Right bracket \]     Circumflex    \^     Underscore    \_
%   Grave accent  \`     Left brace    \{     Vertical bar  \|
%   Right brace   \}     Tilde         \~}
%
% \GetFileInfo{pagegrid.drv}
%
% \title{The \xpackage{pagegrid} package}
% \date{2016/05/16 v1.5}
% \author{Heiko Oberdiek\thanks
% {Please report any issues at \url{https://github.com/ho-tex/oberdiek/issues}}}
%
% \maketitle
%
% \begin{abstract}
% The \LaTeX\ package prints a page grid in the background.
% \end{abstract}
%
% \tableofcontents
%
% \section{Documentation}
%
% The package puts a grid on the paper. It was written for
% developers of a class or package
% who have to put elements on definite locations on a page
% (e.g. letter class). The grid allows a faster optical check,
% whether the positions are correct. If the previewer already
% offers features for measuring, the package might be obsolete.
% Otherwise it saves the developer from printing the page and
% measuring by hand.
%
% \subsection{Options}
%
% Options are evaluated in the following order:
% \begin{enumerate}
% \item
%  Configuration file \xfile{pagegrid.cfg} using \cs{pagegridsetup}
%  if the file exists.
%  \item
%  Package options given for \cs{usepackage}.
%  \item
%  Later calls of \cs{pagegridsetup}.
% \end{enumerate}
% \begin{declcs}{pagegridsetup}\M{option list}
% \end{declcs}
% The options are key value options. Boolean options are enabled by
% default (without value) or by using the explicit value \texttt{true}.
% Value \texttt{false} disable the option.
%
% \subsubsection{Options \xoption{enable}, \xoption{disable}}
%
% \begin{description}
% \item[\xoption{enable}:] This boolean option controls whether the page grid
%   is drawn. As default the page grid drawing is activated.
% \item[\xoption{disable}:] It is the opposite
%   of option \xoption{enable}. It was added for convenience and
%   allows the abbreviation \texttt{disable} for \texttt{enable=false}.
% \end{description}
%
% \subsubsection{Grid origins}
%
% The package supports up to two grids on a page allowing
% measurement from opposite directions. As default two grids are drawn,
% the first from bottom left to top right. The origin of the second
% grid is at the opposite top right corner.
% The origins are controlled by the following options.
% The number of grids (one or two) depend on the number of these options
% in one call of \cs{pagegridsetup}.
% The following frame shows a paper and in its corners are the
% corresponding options. At the left and right side alias names
% are given for the options inside the paper.
% \begin{quote}
% \begin{tabular}{@{}r|@{\,}l@{\qquad}r@{\,}|l@{}}
% \cline{2-3}
% \xoption{left-top}, \xoption{lt}, \xoption{top-left}
% & \vphantom{\"U}\xoption{tl} & \xoption{tr}
% & \xoption{top-right}, \xoption{rt}, \xoption{right-top}\\
% &&&\\
% \xoption{left-bottom}, \xoption{lb}, \xoption{bottom-left}
% & \xoption{bl} & \xoption{br}
% & \xoption{bottom-right}, \xoption{rb}, \xoption{right-bottom}\\
% \cline{2-3}
% \end{tabular}
% \end{quote}
% Examples:
% \begin{quote}
% |\pagegridsetup{bl,tr}|
% \end{quote}
% This is the default setting with two grids as described previously.
% The following setups one grid only. Its origin is the upper left
% corner:
% \begin{quote}
% |\pagegridsetup{top-left}|
% \end{quote}
%
% \subsubsection{Grid unit}
%
% \begin{description}
% \item[\xoption{step}] This option takes a length and
% setups the unit for the grid. The page width and page height
% should be multiples of this unit.
% Currently the default is \texttt{1mm}. But this might change
% later by a heuristic based on the paper size.
% \end{description}
%
% \subsubsection{Color options}
%
% The basic grid lines are drawn as ultra thin help lines and is only
% drawn for the first grid.
% Each tenth and fiftyth line of the basic net is drawn thicker in a special
% color for the two grids.
% \begin{description}
% \item[\xoption{firstcolor}:] Color for the thicker lines and the arrows
% of the first grid. Default value is \texttt{red}.
% \item[\xoption{secondcolor}:] Color for the thicker lines and the arrows
% of the second grid. Default value is \texttt{blue}.
% \end{description}
% Use a color specification that package \xpackage{tikz} understands.
% (The grid is drawn with \xpackage{pgf}/\xpackage{tikz}.)
%
% \subsubsection{Arrow options}
%
% Arrows are put at the origin at the grid to show the grid start
% and the direction of the grid.
% \begin{description}
% \item[\xoption{arrows}:] This boolean option turns the arrows on or off.
% As default arrows are enabled.
% \item[\xoption{arrowlength}:] The length given as value is the
% length of the edge of a square at the origin within the
% arrow is put as diagonal. Default is 10 times the grid unit (10\,mm).
% The real arrow length is this length multiplied by $\sqrt2$.
% \end{description}
%
% \subsubsection{Miscellaneous options}
%
% \begin{description}
% \item[\xoption{double}:] The output page is doubled, one without page
% grid and the other with page grid. Possible values are shown in the
% following table:
% \begin{quote}
% \begin{tabular}{ll}
% Option & Meaning\\
% \hline
% |false| & Turns option off.\\
% |first| & Grid page comes first.\\
% |last| & Grid page comes after the page without grid.\\
% |true| & Same as |last|.\\
% \meta{no value} & Same as |true|.\\
% \end{tabular}
% \end{quote}
% \textbf{Note:}
% The double output of the page has side effects.
% All whatits are executed twice, for example: file writing
% and anchor setting. Some unwanted actions are catched such
% as multiple \cs{label} definitions, duplicate entries in
% the table of contents. For bookmarks, use package \xpackage{bookmarks}.
% \item[\xoption{foreground}:] Boolean option, default is \texttt{false}.
% Sometimes there might be elements on the page (e.g. large images)
% that hide the grid. Then option \xoption{foreground} puts the grids
% over the current output page.
% \end{description}
%
% \StopEventually{
% }
%
% \section{Implementation}
%    \begin{macrocode}
%<*package>
%    \end{macrocode}
%    Reload check, especially if the package is not used with \LaTeX.
%    \begin{macrocode}
\begingroup\catcode61\catcode48\catcode32=10\relax%
  \catcode13=5 % ^^M
  \endlinechar=13 %
  \catcode35=6 % #
  \catcode39=12 % '
  \catcode44=12 % ,
  \catcode45=12 % -
  \catcode46=12 % .
  \catcode58=12 % :
  \catcode64=11 % @
  \catcode123=1 % {
  \catcode125=2 % }
  \expandafter\let\expandafter\x\csname ver@pagegrid.sty\endcsname
  \ifx\x\relax % plain-TeX, first loading
  \else
    \def\empty{}%
    \ifx\x\empty % LaTeX, first loading,
      % variable is initialized, but \ProvidesPackage not yet seen
    \else
      \expandafter\ifx\csname PackageInfo\endcsname\relax
        \def\x#1#2{%
          \immediate\write-1{Package #1 Info: #2.}%
        }%
      \else
        \def\x#1#2{\PackageInfo{#1}{#2, stopped}}%
      \fi
      \x{pagegrid}{The package is already loaded}%
      \aftergroup\endinput
    \fi
  \fi
\endgroup%
%    \end{macrocode}
%    Package identification:
%    \begin{macrocode}
\begingroup\catcode61\catcode48\catcode32=10\relax%
  \catcode13=5 % ^^M
  \endlinechar=13 %
  \catcode35=6 % #
  \catcode39=12 % '
  \catcode40=12 % (
  \catcode41=12 % )
  \catcode44=12 % ,
  \catcode45=12 % -
  \catcode46=12 % .
  \catcode47=12 % /
  \catcode58=12 % :
  \catcode64=11 % @
  \catcode91=12 % [
  \catcode93=12 % ]
  \catcode123=1 % {
  \catcode125=2 % }
  \expandafter\ifx\csname ProvidesPackage\endcsname\relax
    \def\x#1#2#3[#4]{\endgroup
      \immediate\write-1{Package: #3 #4}%
      \xdef#1{#4}%
    }%
  \else
    \def\x#1#2[#3]{\endgroup
      #2[{#3}]%
      \ifx#1\@undefined
        \xdef#1{#3}%
      \fi
      \ifx#1\relax
        \xdef#1{#3}%
      \fi
    }%
  \fi
\expandafter\x\csname ver@pagegrid.sty\endcsname
\ProvidesPackage{pagegrid}%
  [2016/05/16 v1.5 Print page grid in background (HO)]%
%    \end{macrocode}
%
%    \begin{macrocode}
\begingroup\catcode61\catcode48\catcode32=10\relax%
  \catcode13=5 % ^^M
  \endlinechar=13 %
  \catcode123=1 % {
  \catcode125=2 % }
  \catcode64=11 % @
  \def\x{\endgroup
    \expandafter\edef\csname pagegrid@AtEnd\endcsname{%
      \endlinechar=\the\endlinechar\relax
      \catcode13=\the\catcode13\relax
      \catcode32=\the\catcode32\relax
      \catcode35=\the\catcode35\relax
      \catcode61=\the\catcode61\relax
      \catcode64=\the\catcode64\relax
      \catcode123=\the\catcode123\relax
      \catcode125=\the\catcode125\relax
    }%
  }%
\x\catcode61\catcode48\catcode32=10\relax%
\catcode13=5 % ^^M
\endlinechar=13 %
\catcode35=6 % #
\catcode64=11 % @
\catcode123=1 % {
\catcode125=2 % }
\def\TMP@EnsureCode#1#2{%
  \edef\pagegrid@AtEnd{%
    \pagegrid@AtEnd
    \catcode#1=\the\catcode#1\relax
  }%
  \catcode#1=#2\relax
}
\TMP@EnsureCode{9}{10}% (tab)
\TMP@EnsureCode{10}{12}% ^^J
\TMP@EnsureCode{33}{12}% !
\TMP@EnsureCode{34}{12}% "
\TMP@EnsureCode{36}{3}% $
\TMP@EnsureCode{38}{4}% &
\TMP@EnsureCode{39}{12}% '
\TMP@EnsureCode{40}{12}% (
\TMP@EnsureCode{41}{12}% )
\TMP@EnsureCode{42}{12}% *
\TMP@EnsureCode{43}{12}% +
\TMP@EnsureCode{44}{12}% ,
\TMP@EnsureCode{45}{12}% -
\TMP@EnsureCode{46}{12}% .
\TMP@EnsureCode{47}{12}% /
\TMP@EnsureCode{58}{12}% :
\TMP@EnsureCode{59}{12}% ;
\TMP@EnsureCode{60}{12}% <
\TMP@EnsureCode{62}{12}% >
\TMP@EnsureCode{63}{12}% ?
\TMP@EnsureCode{91}{12}% [
\TMP@EnsureCode{93}{12}% ]
\TMP@EnsureCode{94}{7}% ^ (superscript)
\TMP@EnsureCode{95}{8}% _ (subscript)
\TMP@EnsureCode{96}{12}% `
\TMP@EnsureCode{124}{12}% |
\edef\pagegrid@AtEnd{\pagegrid@AtEnd\noexpand\endinput}
%    \end{macrocode}
%
%    \begin{macrocode}
\RequirePackage{tikz}
\RequirePackage{atbegshi}[2009/12/02]
\RequirePackage{kvoptions}[2009/07/17]
%    \end{macrocode}
%    \begin{macrocode}
\begingroup\expandafter\expandafter\expandafter\endgroup
\expandafter\ifx\csname stockwidth\endcsname\relax
  \def\pagegrid@width{\paperwidth}%
  \def\pagegrid@height{\paperheight}%
\else
  \def\pagegrid@width{\stockwidth}%
  \def\pagegrid@height{\stockheight}%
\fi
%    \end{macrocode}
%
%    \begin{macrocode}
\SetupKeyvalOptions{%
  family=pagegrid,%
  prefix=pagegrid@,%
}
\def\pagegrid@init{%
  \let\pagegrid@origin@a\@empty
  \let\pagegrid@origin@b\@empty
  \let\pagegrid@init\relax
}
\let\pagegrid@@init\pagegrid@init
\def\pagegrid@origin@a{bl}
\def\pagegrid@origin@b{tr}
\def\pagegrid@SetOrigin#1{%
  \pagegrid@init
  \ifx\pagegrid@origin@a\@empty
    \def\pagegrid@origin@a{#1}%
  \else
    \ifx\pagegrid@origin@b\@empty
    \else
      \let\pagegrid@origin@a\pagegrid@origin@b
    \fi
    \def\pagegrid@origin@b{#1}%
  \fi
}
\def\pagegrid@temp#1{%
  \DeclareVoidOption{#1}{\pagegrid@SetOrigin{#1}}%
  \@namedef{pagegrid@N@#1}{#1}%
}
\pagegrid@temp{bl}
\pagegrid@temp{br}
\pagegrid@temp{tl}
\pagegrid@temp{tr}
\def\pagegrid@temp#1#2{%
  \DeclareVoidOption{#2}{\pagegrid@SetOrigin{#1}}%
}%
\pagegrid@temp{bl}{lb}
\pagegrid@temp{br}{rb}
\pagegrid@temp{tl}{lt}
\pagegrid@temp{tr}{rt}
\pagegrid@temp{bl}{bottom-left}
\pagegrid@temp{br}{bottom-right}
\pagegrid@temp{tl}{top-left}
\pagegrid@temp{tr}{top-right}
\pagegrid@temp{bl}{left-bottom}
\pagegrid@temp{br}{right-bottom}
\pagegrid@temp{tl}{left-top}
\pagegrid@temp{tr}{right-top}
%    \end{macrocode}
%    \begin{macrocode}
\DeclareBoolOption[true]{enable}
\DeclareComplementaryOption{disable}{enable}
%    \end{macrocode}
%    \begin{macrocode}
\DeclareBoolOption{foreground}
%    \end{macrocode}
%    \begin{macrocode}
\newlength{\pagegrid@step}
\define@key{pagegrid}{step}{%
  \setlength{\pagegrid@step}{#1}%
}
%    \end{macrocode}
%    \begin{macrocode}
\DeclareStringOption[red]{firstcolor}
\DeclareStringOption[blue]{secondcolor}
%    \end{macrocode}
%    \begin{macrocode}
\DeclareBoolOption[true]{arrows}
\newlength\pagegrid@arrowlength
\pagegrid@arrowlength=\z@
\define@key{pagegrid}{arrowlength}{%
  \setlength{\pagegrid@arrowlength}{#1}%
}
%    \end{macrocode}
%    \begin{macrocode}
\define@key{pagegrid}{double}[true]{%
  \@ifundefined{pagegrid@double@#1}{%
    \PackageWarning{pagegrid}{%
      Unsupported value `#1' for option `double'.\MessageBreak
      Known values are:\MessageBreak
      `false', `first', `last', `true'.\MessageBreak
      Now `false' is used%
    }%
    \chardef\pagegrid@double\z@
  }{%
    \chardef\pagegrid@double\csname pagegrid@double@#1\endcsname\relax
  }%
}
\@namedef{pagegrid@double@false}{0}
\@namedef{pagegrid@double@first}{1}
\@namedef{pagegrid@double@last}{2}
\@namedef{pagegrid@double@true}{2}
\chardef\pagegrid@double\z@
%    \end{macrocode}
%    \begin{macrocode}
\newcommand*{\pagegridsetup}{%
  \let\pagegrid@init\pagegrid@@init
  \setkeys{pagegrid}%
}
%    \end{macrocode}
%    \begin{macrocode}
\pagegridsetup{%
  step=1mm%
}
\InputIfFileExists{pagegrid.cfg}{}%
\ProcessKeyvalOptions*\relax
\AtBeginDocument{%
  \ifdim\pagegrid@arrowlength>\z@
  \else
    \pagegrid@arrowlength=10\pagegrid@step
  \fi
}
%    \end{macrocode}
%
%    \begin{macrocode}
\def\pagegridShipoutDoubleBegin{%
  \begingroup
  \let\newlabel\@gobbletwo
  \let\zref@newlabel\@gobbletwo
  \let\@writefile\@gobbletwo
  \let\select@language\@gobble
}
\def\pagegridShipoutDoubleEnd{%
  \endgroup
}
\def\pagegrid@WriteDouble#1#2{%
  \immediate\write#1{%
    \@backslashchar csname %
    pagegridShipoutDouble#2%
    \@backslashchar endcsname%
  }%
}
\def\pagegrid@ShipoutDouble#1{%
  \begingroup
    \if@filesw
      \pagegrid@WriteDouble\@mainaux{Begin}%
      \ifx\@auxout\@partaux
        \pagegrid@WriteDouble\@partaux{Begin}%
        \def\pagegrid@temp{%
          \pagegrid@WriteDouble\@mainaux{End}%
          \pagegrid@WriteDouble\@partaux{End}%
        }%
      \else
        \def\pagegrid@temp{%
          \pagegrid@WriteDouble\@mainaux{End}%
        }%
      \fi
    \else
      \def\pagegrid@temp{}%
    \fi
    \let\protect\noexpand
    \AtBeginShipoutOriginalShipout\copy#1\relax
    \pagegrid@temp
  \endgroup
}
%    \end{macrocode}
%
%    \begin{macrocode}
\AtBeginShipout{%
  \ifdim\pagegrid@step>\z@
  \else
    \pagegrid@enablefalse
  \fi
  \ifpagegrid@enable
    \ifnum\pagegrid@double=\@ne
      \pagegrid@ShipoutDouble\AtBeginShipoutBox
    \else
      \ifnum\pagegrid@double=\tw@
        \@ifundefined{pagegrid@DoubleBox}{%
          \newbox\pagegrid@DoubleBox
        }{}%
        \setbox\pagegrid@DoubleBox=\copy\AtBeginShipoutBox
      \fi
    \fi
    \ifpagegrid@foreground
      \expandafter\AtBeginShipoutUpperLeftForeground
    \else
      \expandafter\AtBeginShipoutUpperLeft
    \fi
    {%
      \put(0,0){%
        \makebox(0,0)[lt]{%
          \begin{tikzpicture}[%
            bl/.style={},%
            br/.style={xshift=\pagegrid@width,xscale=-1},%
            tl/.style={yshift=\pagegrid@height,yscale=-1},%
            tr/.style={xshift=\pagegrid@width,%
                       yshift=\pagegrid@height,scale=-1}%
          ]%
            \useasboundingbox
              (0mm,\pagegrid@height) rectangle (0mm,\pagegrid@height);%
            \draw[%
              \pagegrid@origin@a,%
              step=\pagegrid@step,%
              style=help lines,%
              ultra thin%
            ] (0mm,0mm) grid (\pagegrid@width,\pagegrid@height);%
            \ifx\pagegrid@origin@b\@empty
            \else
              \draw[%
                \pagegrid@origin@b,%
                step=10\pagegrid@step,%
                {\pagegrid@secondcolor},%
                very thin%
              ] (0mm,0mm) grid (\pagegrid@width,\pagegrid@height);%
            \fi
            \draw[%
               \pagegrid@origin@a,%
               step=10\pagegrid@step,%
               {\pagegrid@firstcolor},%
               very thin%
            ] (0mm,0mm) grid (\pagegrid@width,\pagegrid@height);%
            \ifx\pagegrid@origin@b\@empty
            \else
              \draw[%
                \pagegrid@origin@b,%
                step=50\pagegrid@step,%
                {\pagegrid@secondcolor},%
                thick%
              ] (0mm,0mm) grid (\pagegrid@width,\pagegrid@height);%
            \fi
            \draw[%
              \pagegrid@origin@a,%
              step=50\pagegrid@step,%
              {\pagegrid@firstcolor},%
              thick%
            ] (0mm,0mm) grid (\pagegrid@width,\pagegrid@height);%
            \ifpagegrid@arrows
              \ifx\pagegrid@origin@b\@empty
              \else
                \draw[%
                  \pagegrid@origin@b,%
                  {\pagegrid@secondcolor},%
                  stroke,%
                  line width=1pt,%
                  line cap=round%
                ] (0mm,0mm) %
                -- (\pagegrid@arrowlength,\pagegrid@arrowlength) %
                   (\pagegrid@arrowlength,.5\pagegrid@arrowlength) %
                -- (\pagegrid@arrowlength,\pagegrid@arrowlength) %
                -- (.5\pagegrid@arrowlength,\pagegrid@arrowlength);%
              \fi
              \draw[%
                \pagegrid@origin@a,%
                {\pagegrid@firstcolor},%
                stroke,%
                line width=1pt,%
                line cap=round%
              ] (0mm,0mm) %
              -- (\pagegrid@arrowlength,\pagegrid@arrowlength) %
                 (\pagegrid@arrowlength,.5\pagegrid@arrowlength) %
              -- (\pagegrid@arrowlength,\pagegrid@arrowlength) %
              -- (.5\pagegrid@arrowlength,\pagegrid@arrowlength);%
            \fi
          \end{tikzpicture}%
        }%
      }%
    }%
    \ifnum\pagegrid@double=\tw@
      \pagegrid@ShipoutDouble\pagegrid@DoubleBox
    \fi
  \fi
}
%    \end{macrocode}
%
%    \begin{macrocode}
\pagegrid@AtEnd%
%</package>
%    \end{macrocode}
%
% \section{Test}
%
% \subsection{Catcode checks for loading}
%
%    \begin{macrocode}
%<*test1>
%    \end{macrocode}
%    \begin{macrocode}
\catcode`\{=1 %
\catcode`\}=2 %
\catcode`\#=6 %
\catcode`\@=11 %
\expandafter\ifx\csname count@\endcsname\relax
  \countdef\count@=255 %
\fi
\expandafter\ifx\csname @gobble\endcsname\relax
  \long\def\@gobble#1{}%
\fi
\expandafter\ifx\csname @firstofone\endcsname\relax
  \long\def\@firstofone#1{#1}%
\fi
\expandafter\ifx\csname loop\endcsname\relax
  \expandafter\@firstofone
\else
  \expandafter\@gobble
\fi
{%
  \def\loop#1\repeat{%
    \def\body{#1}%
    \iterate
  }%
  \def\iterate{%
    \body
      \let\next\iterate
    \else
      \let\next\relax
    \fi
    \next
  }%
  \let\repeat=\fi
}%
\def\RestoreCatcodes{}
\count@=0 %
\loop
  \edef\RestoreCatcodes{%
    \RestoreCatcodes
    \catcode\the\count@=\the\catcode\count@\relax
  }%
\ifnum\count@<255 %
  \advance\count@ 1 %
\repeat

\def\RangeCatcodeInvalid#1#2{%
  \count@=#1\relax
  \loop
    \catcode\count@=15 %
  \ifnum\count@<#2\relax
    \advance\count@ 1 %
  \repeat
}
\def\RangeCatcodeCheck#1#2#3{%
  \count@=#1\relax
  \loop
    \ifnum#3=\catcode\count@
    \else
      \errmessage{%
        Character \the\count@\space
        with wrong catcode \the\catcode\count@\space
        instead of \number#3%
      }%
    \fi
  \ifnum\count@<#2\relax
    \advance\count@ 1 %
  \repeat
}
\def\space{ }
\expandafter\ifx\csname LoadCommand\endcsname\relax
  \def\LoadCommand{\input pagegrid.sty\relax}%
\fi
\def\Test{%
  \RangeCatcodeInvalid{0}{47}%
  \RangeCatcodeInvalid{58}{64}%
  \RangeCatcodeInvalid{91}{96}%
  \RangeCatcodeInvalid{123}{255}%
  \catcode`\@=12 %
  \catcode`\\=0 %
  \catcode`\%=14 %
  \LoadCommand
  \RangeCatcodeCheck{0}{36}{15}%
  \RangeCatcodeCheck{37}{37}{14}%
  \RangeCatcodeCheck{38}{47}{15}%
  \RangeCatcodeCheck{48}{57}{12}%
  \RangeCatcodeCheck{58}{63}{15}%
  \RangeCatcodeCheck{64}{64}{12}%
  \RangeCatcodeCheck{65}{90}{11}%
  \RangeCatcodeCheck{91}{91}{15}%
  \RangeCatcodeCheck{92}{92}{0}%
  \RangeCatcodeCheck{93}{96}{15}%
  \RangeCatcodeCheck{97}{122}{11}%
  \RangeCatcodeCheck{123}{255}{15}%
  \RestoreCatcodes
}
\Test
\csname @@end\endcsname
\end
%    \end{macrocode}
%    \begin{macrocode}
%</test1>
%    \end{macrocode}
%
% \section{Installation}
%
% \subsection{Download}
%
% \paragraph{Package.} This package is available on
% CTAN\footnote{\CTANpkg{pagegrid}}:
% \begin{description}
% \item[\CTAN{macros/latex/contrib/oberdiek/pagegrid.dtx}] The source file.
% \item[\CTAN{macros/latex/contrib/oberdiek/pagegrid.pdf}] Documentation.
% \end{description}
%
%
% \paragraph{Bundle.} All the packages of the bundle `oberdiek'
% are also available in a TDS compliant ZIP archive. There
% the packages are already unpacked and the documentation files
% are generated. The files and directories obey the TDS standard.
% \begin{description}
% \item[\CTANinstall{install/macros/latex/contrib/oberdiek.tds.zip}]
% \end{description}
% \emph{TDS} refers to the standard ``A Directory Structure
% for \TeX\ Files'' (\CTAN{tds/tds.pdf}). Directories
% with \xfile{texmf} in their name are usually organized this way.
%
% \subsection{Bundle installation}
%
% \paragraph{Unpacking.} Unpack the \xfile{oberdiek.tds.zip} in the
% TDS tree (also known as \xfile{texmf} tree) of your choice.
% Example (linux):
% \begin{quote}
%   |unzip oberdiek.tds.zip -d ~/texmf|
% \end{quote}
%
% \paragraph{Script installation.}
% Check the directory \xfile{TDS:scripts/oberdiek/} for
% scripts that need further installation steps.
%
% \subsection{Package installation}
%
% \paragraph{Unpacking.} The \xfile{.dtx} file is a self-extracting
% \docstrip\ archive. The files are extracted by running the
% \xfile{.dtx} through \plainTeX:
% \begin{quote}
%   \verb|tex pagegrid.dtx|
% \end{quote}
%
% \paragraph{TDS.} Now the different files must be moved into
% the different directories in your installation TDS tree
% (also known as \xfile{texmf} tree):
% \begin{quote}
% \def\t{^^A
% \begin{tabular}{@{}>{\ttfamily}l@{ $\rightarrow$ }>{\ttfamily}l@{}}
%   pagegrid.sty & tex/latex/oberdiek/pagegrid.sty\\
%   pagegrid.pdf & doc/latex/oberdiek/pagegrid.pdf\\
%   test/pagegrid-test1.tex & doc/latex/oberdiek/test/pagegrid-test1.tex\\
%   pagegrid.dtx & source/latex/oberdiek/pagegrid.dtx\\
% \end{tabular}^^A
% }^^A
% \sbox0{\t}^^A
% \ifdim\wd0>\linewidth
%   \begingroup
%     \advance\linewidth by\leftmargin
%     \advance\linewidth by\rightmargin
%   \edef\x{\endgroup
%     \def\noexpand\lw{\the\linewidth}^^A
%   }\x
%   \def\lwbox{^^A
%     \leavevmode
%     \hbox to \linewidth{^^A
%       \kern-\leftmargin\relax
%       \hss
%       \usebox0
%       \hss
%       \kern-\rightmargin\relax
%     }^^A
%   }^^A
%   \ifdim\wd0>\lw
%     \sbox0{\small\t}^^A
%     \ifdim\wd0>\linewidth
%       \ifdim\wd0>\lw
%         \sbox0{\footnotesize\t}^^A
%         \ifdim\wd0>\linewidth
%           \ifdim\wd0>\lw
%             \sbox0{\scriptsize\t}^^A
%             \ifdim\wd0>\linewidth
%               \ifdim\wd0>\lw
%                 \sbox0{\tiny\t}^^A
%                 \ifdim\wd0>\linewidth
%                   \lwbox
%                 \else
%                   \usebox0
%                 \fi
%               \else
%                 \lwbox
%               \fi
%             \else
%               \usebox0
%             \fi
%           \else
%             \lwbox
%           \fi
%         \else
%           \usebox0
%         \fi
%       \else
%         \lwbox
%       \fi
%     \else
%       \usebox0
%     \fi
%   \else
%     \lwbox
%   \fi
% \else
%   \usebox0
% \fi
% \end{quote}
% If you have a \xfile{docstrip.cfg} that configures and enables \docstrip's
% TDS installing feature, then some files can already be in the right
% place, see the documentation of \docstrip.
%
% \subsection{Refresh file name databases}
%
% If your \TeX~distribution
% (\TeX\,Live, \mikTeX, \dots) relies on file name databases, you must refresh
% these. For example, \TeX\,Live\ users run \verb|texhash| or
% \verb|mktexlsr|.
%
% \subsection{Some details for the interested}
%
% \paragraph{Unpacking with \LaTeX.}
% The \xfile{.dtx} chooses its action depending on the format:
% \begin{description}
% \item[\plainTeX:] Run \docstrip\ and extract the files.
% \item[\LaTeX:] Generate the documentation.
% \end{description}
% If you insist on using \LaTeX\ for \docstrip\ (really,
% \docstrip\ does not need \LaTeX), then inform the autodetect routine
% about your intention:
% \begin{quote}
%   \verb|latex \let\install=y% \iffalse meta-comment
%
% File: pagegrid.dtx
% Version: 2016/05/16 v1.5
% Info: Print page grid in background
%
% Copyright (C) 2009 by
%    Heiko Oberdiek <heiko.oberdiek at googlemail.com>
%    2016
%    https://github.com/ho-tex/oberdiek/issues
%
% This work may be distributed and/or modified under the
% conditions of the LaTeX Project Public License, either
% version 1.3c of this license or (at your option) any later
% version. This version of this license is in
%    https://www.latex-project.org/lppl/lppl-1-3c.txt
% and the latest version of this license is in
%    https://www.latex-project.org/lppl.txt
% and version 1.3 or later is part of all distributions of
% LaTeX version 2005/12/01 or later.
%
% This work has the LPPL maintenance status "maintained".
%
% The Current Maintainers of this work are
% Heiko Oberdiek and the Oberdiek Package Support Group
% https://github.com/ho-tex/oberdiek/issues
%
% This work consists of the main source file pagegrid.dtx
% and the derived files
%    pagegrid.sty, pagegrid.pdf, pagegrid.ins, pagegrid.drv,
%    pagegrid-test1.tex.
%
% Distribution:
%    CTAN:macros/latex/contrib/oberdiek/pagegrid.dtx
%    CTAN:macros/latex/contrib/oberdiek/pagegrid.pdf
%
% Unpacking:
%    (a) If pagegrid.ins is present:
%           tex pagegrid.ins
%    (b) Without pagegrid.ins:
%           tex pagegrid.dtx
%    (c) If you insist on using LaTeX
%           latex \let\install=y\input{pagegrid.dtx}
%        (quote the arguments according to the demands of your shell)
%
% Documentation:
%    (a) If pagegrid.drv is present:
%           latex pagegrid.drv
%    (b) Without pagegrid.drv:
%           latex pagegrid.dtx; ...
%    The class ltxdoc loads the configuration file ltxdoc.cfg
%    if available. Here you can specify further options, e.g.
%    use A4 as paper format:
%       \PassOptionsToClass{a4paper}{article}
%
%    Programm calls to get the documentation (example):
%       pdflatex pagegrid.dtx
%       makeindex -s gind.ist pagegrid.idx
%       pdflatex pagegrid.dtx
%       makeindex -s gind.ist pagegrid.idx
%       pdflatex pagegrid.dtx
%
% Installation:
%    TDS:tex/latex/oberdiek/pagegrid.sty
%    TDS:doc/latex/oberdiek/pagegrid.pdf
%    TDS:doc/latex/oberdiek/test/pagegrid-test1.tex
%    TDS:source/latex/oberdiek/pagegrid.dtx
%
%<*ignore>
\begingroup
  \catcode123=1 %
  \catcode125=2 %
  \def\x{LaTeX2e}%
\expandafter\endgroup
\ifcase 0\ifx\install y1\fi\expandafter
         \ifx\csname processbatchFile\endcsname\relax\else1\fi
         \ifx\fmtname\x\else 1\fi\relax
\else\csname fi\endcsname
%</ignore>
%<*install>
\input docstrip.tex
\Msg{************************************************************************}
\Msg{* Installation}
\Msg{* Package: pagegrid 2016/05/16 v1.5 Print page grid in background (HO)}
\Msg{************************************************************************}

\keepsilent
\askforoverwritefalse

\let\MetaPrefix\relax
\preamble

This is a generated file.

Project: pagegrid
Version: 2016/05/16 v1.5

Copyright (C) 2009 by
   Heiko Oberdiek <heiko.oberdiek at googlemail.com>

This work may be distributed and/or modified under the
conditions of the LaTeX Project Public License, either
version 1.3c of this license or (at your option) any later
version. This version of this license is in
   https://www.latex-project.org/lppl/lppl-1-3c.txt
and the latest version of this license is in
   https://www.latex-project.org/lppl.txt
and version 1.3 or later is part of all distributions of
LaTeX version 2005/12/01 or later.

This work has the LPPL maintenance status "maintained".

The Current Maintainers of this work are
Heiko Oberdiek and the Oberdiek Package Support Group
https://github.com/ho-tex/oberdiek/issues


This work consists of the main source file pagegrid.dtx
and the derived files
   pagegrid.sty, pagegrid.pdf, pagegrid.ins, pagegrid.drv,
   pagegrid-test1.tex.

\endpreamble
\let\MetaPrefix\DoubleperCent

\generate{%
  \file{pagegrid.ins}{\from{pagegrid.dtx}{install}}%
  \file{pagegrid.drv}{\from{pagegrid.dtx}{driver}}%
  \usedir{tex/latex/oberdiek}%
  \file{pagegrid.sty}{\from{pagegrid.dtx}{package}}%
%  \usedir{doc/latex/oberdiek/test}%
%  \file{pagegrid-test1.tex}{\from{pagegrid.dtx}{test1}}%
  \nopreamble
  \nopostamble
%  \usedir{source/latex/oberdiek/catalogue}%
%  \file{pagegrid.xml}{\from{pagegrid.dtx}{catalogue}}%
}

\catcode32=13\relax% active space
\let =\space%
\Msg{************************************************************************}
\Msg{*}
\Msg{* To finish the installation you have to move the following}
\Msg{* file into a directory searched by TeX:}
\Msg{*}
\Msg{*     pagegrid.sty}
\Msg{*}
\Msg{* To produce the documentation run the file `pagegrid.drv'}
\Msg{* through LaTeX.}
\Msg{*}
\Msg{* Happy TeXing!}
\Msg{*}
\Msg{************************************************************************}

\endbatchfile
%</install>
%<*ignore>
\fi
%</ignore>
%<*driver>
\NeedsTeXFormat{LaTeX2e}
\ProvidesFile{pagegrid.drv}%
  [2016/05/16 v1.5 Print page grid in background (HO)]%
\documentclass{ltxdoc}
\usepackage{holtxdoc}[2011/11/22]
\begin{document}
  \DocInput{pagegrid.dtx}%
\end{document}
%</driver>
% \fi
%
%
% \CharacterTable
%  {Upper-case    \A\B\C\D\E\F\G\H\I\J\K\L\M\N\O\P\Q\R\S\T\U\V\W\X\Y\Z
%   Lower-case    \a\b\c\d\e\f\g\h\i\j\k\l\m\n\o\p\q\r\s\t\u\v\w\x\y\z
%   Digits        \0\1\2\3\4\5\6\7\8\9
%   Exclamation   \!     Double quote  \"     Hash (number) \#
%   Dollar        \$     Percent       \%     Ampersand     \&
%   Acute accent  \'     Left paren    \(     Right paren   \)
%   Asterisk      \*     Plus          \+     Comma         \,
%   Minus         \-     Point         \.     Solidus       \/
%   Colon         \:     Semicolon     \;     Less than     \<
%   Equals        \=     Greater than  \>     Question mark \?
%   Commercial at \@     Left bracket  \[     Backslash     \\
%   Right bracket \]     Circumflex    \^     Underscore    \_
%   Grave accent  \`     Left brace    \{     Vertical bar  \|
%   Right brace   \}     Tilde         \~}
%
% \GetFileInfo{pagegrid.drv}
%
% \title{The \xpackage{pagegrid} package}
% \date{2016/05/16 v1.5}
% \author{Heiko Oberdiek\thanks
% {Please report any issues at \url{https://github.com/ho-tex/oberdiek/issues}}}
%
% \maketitle
%
% \begin{abstract}
% The \LaTeX\ package prints a page grid in the background.
% \end{abstract}
%
% \tableofcontents
%
% \section{Documentation}
%
% The package puts a grid on the paper. It was written for
% developers of a class or package
% who have to put elements on definite locations on a page
% (e.g. letter class). The grid allows a faster optical check,
% whether the positions are correct. If the previewer already
% offers features for measuring, the package might be obsolete.
% Otherwise it saves the developer from printing the page and
% measuring by hand.
%
% \subsection{Options}
%
% Options are evaluated in the following order:
% \begin{enumerate}
% \item
%  Configuration file \xfile{pagegrid.cfg} using \cs{pagegridsetup}
%  if the file exists.
%  \item
%  Package options given for \cs{usepackage}.
%  \item
%  Later calls of \cs{pagegridsetup}.
% \end{enumerate}
% \begin{declcs}{pagegridsetup}\M{option list}
% \end{declcs}
% The options are key value options. Boolean options are enabled by
% default (without value) or by using the explicit value \texttt{true}.
% Value \texttt{false} disable the option.
%
% \subsubsection{Options \xoption{enable}, \xoption{disable}}
%
% \begin{description}
% \item[\xoption{enable}:] This boolean option controls whether the page grid
%   is drawn. As default the page grid drawing is activated.
% \item[\xoption{disable}:] It is the opposite
%   of option \xoption{enable}. It was added for convenience and
%   allows the abbreviation \texttt{disable} for \texttt{enable=false}.
% \end{description}
%
% \subsubsection{Grid origins}
%
% The package supports up to two grids on a page allowing
% measurement from opposite directions. As default two grids are drawn,
% the first from bottom left to top right. The origin of the second
% grid is at the opposite top right corner.
% The origins are controlled by the following options.
% The number of grids (one or two) depend on the number of these options
% in one call of \cs{pagegridsetup}.
% The following frame shows a paper and in its corners are the
% corresponding options. At the left and right side alias names
% are given for the options inside the paper.
% \begin{quote}
% \begin{tabular}{@{}r|@{\,}l@{\qquad}r@{\,}|l@{}}
% \cline{2-3}
% \xoption{left-top}, \xoption{lt}, \xoption{top-left}
% & \vphantom{\"U}\xoption{tl} & \xoption{tr}
% & \xoption{top-right}, \xoption{rt}, \xoption{right-top}\\
% &&&\\
% \xoption{left-bottom}, \xoption{lb}, \xoption{bottom-left}
% & \xoption{bl} & \xoption{br}
% & \xoption{bottom-right}, \xoption{rb}, \xoption{right-bottom}\\
% \cline{2-3}
% \end{tabular}
% \end{quote}
% Examples:
% \begin{quote}
% |\pagegridsetup{bl,tr}|
% \end{quote}
% This is the default setting with two grids as described previously.
% The following setups one grid only. Its origin is the upper left
% corner:
% \begin{quote}
% |\pagegridsetup{top-left}|
% \end{quote}
%
% \subsubsection{Grid unit}
%
% \begin{description}
% \item[\xoption{step}] This option takes a length and
% setups the unit for the grid. The page width and page height
% should be multiples of this unit.
% Currently the default is \texttt{1mm}. But this might change
% later by a heuristic based on the paper size.
% \end{description}
%
% \subsubsection{Color options}
%
% The basic grid lines are drawn as ultra thin help lines and is only
% drawn for the first grid.
% Each tenth and fiftyth line of the basic net is drawn thicker in a special
% color for the two grids.
% \begin{description}
% \item[\xoption{firstcolor}:] Color for the thicker lines and the arrows
% of the first grid. Default value is \texttt{red}.
% \item[\xoption{secondcolor}:] Color for the thicker lines and the arrows
% of the second grid. Default value is \texttt{blue}.
% \end{description}
% Use a color specification that package \xpackage{tikz} understands.
% (The grid is drawn with \xpackage{pgf}/\xpackage{tikz}.)
%
% \subsubsection{Arrow options}
%
% Arrows are put at the origin at the grid to show the grid start
% and the direction of the grid.
% \begin{description}
% \item[\xoption{arrows}:] This boolean option turns the arrows on or off.
% As default arrows are enabled.
% \item[\xoption{arrowlength}:] The length given as value is the
% length of the edge of a square at the origin within the
% arrow is put as diagonal. Default is 10 times the grid unit (10\,mm).
% The real arrow length is this length multiplied by $\sqrt2$.
% \end{description}
%
% \subsubsection{Miscellaneous options}
%
% \begin{description}
% \item[\xoption{double}:] The output page is doubled, one without page
% grid and the other with page grid. Possible values are shown in the
% following table:
% \begin{quote}
% \begin{tabular}{ll}
% Option & Meaning\\
% \hline
% |false| & Turns option off.\\
% |first| & Grid page comes first.\\
% |last| & Grid page comes after the page without grid.\\
% |true| & Same as |last|.\\
% \meta{no value} & Same as |true|.\\
% \end{tabular}
% \end{quote}
% \textbf{Note:}
% The double output of the page has side effects.
% All whatits are executed twice, for example: file writing
% and anchor setting. Some unwanted actions are catched such
% as multiple \cs{label} definitions, duplicate entries in
% the table of contents. For bookmarks, use package \xpackage{bookmarks}.
% \item[\xoption{foreground}:] Boolean option, default is \texttt{false}.
% Sometimes there might be elements on the page (e.g. large images)
% that hide the grid. Then option \xoption{foreground} puts the grids
% over the current output page.
% \end{description}
%
% \StopEventually{
% }
%
% \section{Implementation}
%    \begin{macrocode}
%<*package>
%    \end{macrocode}
%    Reload check, especially if the package is not used with \LaTeX.
%    \begin{macrocode}
\begingroup\catcode61\catcode48\catcode32=10\relax%
  \catcode13=5 % ^^M
  \endlinechar=13 %
  \catcode35=6 % #
  \catcode39=12 % '
  \catcode44=12 % ,
  \catcode45=12 % -
  \catcode46=12 % .
  \catcode58=12 % :
  \catcode64=11 % @
  \catcode123=1 % {
  \catcode125=2 % }
  \expandafter\let\expandafter\x\csname ver@pagegrid.sty\endcsname
  \ifx\x\relax % plain-TeX, first loading
  \else
    \def\empty{}%
    \ifx\x\empty % LaTeX, first loading,
      % variable is initialized, but \ProvidesPackage not yet seen
    \else
      \expandafter\ifx\csname PackageInfo\endcsname\relax
        \def\x#1#2{%
          \immediate\write-1{Package #1 Info: #2.}%
        }%
      \else
        \def\x#1#2{\PackageInfo{#1}{#2, stopped}}%
      \fi
      \x{pagegrid}{The package is already loaded}%
      \aftergroup\endinput
    \fi
  \fi
\endgroup%
%    \end{macrocode}
%    Package identification:
%    \begin{macrocode}
\begingroup\catcode61\catcode48\catcode32=10\relax%
  \catcode13=5 % ^^M
  \endlinechar=13 %
  \catcode35=6 % #
  \catcode39=12 % '
  \catcode40=12 % (
  \catcode41=12 % )
  \catcode44=12 % ,
  \catcode45=12 % -
  \catcode46=12 % .
  \catcode47=12 % /
  \catcode58=12 % :
  \catcode64=11 % @
  \catcode91=12 % [
  \catcode93=12 % ]
  \catcode123=1 % {
  \catcode125=2 % }
  \expandafter\ifx\csname ProvidesPackage\endcsname\relax
    \def\x#1#2#3[#4]{\endgroup
      \immediate\write-1{Package: #3 #4}%
      \xdef#1{#4}%
    }%
  \else
    \def\x#1#2[#3]{\endgroup
      #2[{#3}]%
      \ifx#1\@undefined
        \xdef#1{#3}%
      \fi
      \ifx#1\relax
        \xdef#1{#3}%
      \fi
    }%
  \fi
\expandafter\x\csname ver@pagegrid.sty\endcsname
\ProvidesPackage{pagegrid}%
  [2016/05/16 v1.5 Print page grid in background (HO)]%
%    \end{macrocode}
%
%    \begin{macrocode}
\begingroup\catcode61\catcode48\catcode32=10\relax%
  \catcode13=5 % ^^M
  \endlinechar=13 %
  \catcode123=1 % {
  \catcode125=2 % }
  \catcode64=11 % @
  \def\x{\endgroup
    \expandafter\edef\csname pagegrid@AtEnd\endcsname{%
      \endlinechar=\the\endlinechar\relax
      \catcode13=\the\catcode13\relax
      \catcode32=\the\catcode32\relax
      \catcode35=\the\catcode35\relax
      \catcode61=\the\catcode61\relax
      \catcode64=\the\catcode64\relax
      \catcode123=\the\catcode123\relax
      \catcode125=\the\catcode125\relax
    }%
  }%
\x\catcode61\catcode48\catcode32=10\relax%
\catcode13=5 % ^^M
\endlinechar=13 %
\catcode35=6 % #
\catcode64=11 % @
\catcode123=1 % {
\catcode125=2 % }
\def\TMP@EnsureCode#1#2{%
  \edef\pagegrid@AtEnd{%
    \pagegrid@AtEnd
    \catcode#1=\the\catcode#1\relax
  }%
  \catcode#1=#2\relax
}
\TMP@EnsureCode{9}{10}% (tab)
\TMP@EnsureCode{10}{12}% ^^J
\TMP@EnsureCode{33}{12}% !
\TMP@EnsureCode{34}{12}% "
\TMP@EnsureCode{36}{3}% $
\TMP@EnsureCode{38}{4}% &
\TMP@EnsureCode{39}{12}% '
\TMP@EnsureCode{40}{12}% (
\TMP@EnsureCode{41}{12}% )
\TMP@EnsureCode{42}{12}% *
\TMP@EnsureCode{43}{12}% +
\TMP@EnsureCode{44}{12}% ,
\TMP@EnsureCode{45}{12}% -
\TMP@EnsureCode{46}{12}% .
\TMP@EnsureCode{47}{12}% /
\TMP@EnsureCode{58}{12}% :
\TMP@EnsureCode{59}{12}% ;
\TMP@EnsureCode{60}{12}% <
\TMP@EnsureCode{62}{12}% >
\TMP@EnsureCode{63}{12}% ?
\TMP@EnsureCode{91}{12}% [
\TMP@EnsureCode{93}{12}% ]
\TMP@EnsureCode{94}{7}% ^ (superscript)
\TMP@EnsureCode{95}{8}% _ (subscript)
\TMP@EnsureCode{96}{12}% `
\TMP@EnsureCode{124}{12}% |
\edef\pagegrid@AtEnd{\pagegrid@AtEnd\noexpand\endinput}
%    \end{macrocode}
%
%    \begin{macrocode}
\RequirePackage{tikz}
\RequirePackage{atbegshi}[2009/12/02]
\RequirePackage{kvoptions}[2009/07/17]
%    \end{macrocode}
%    \begin{macrocode}
\begingroup\expandafter\expandafter\expandafter\endgroup
\expandafter\ifx\csname stockwidth\endcsname\relax
  \def\pagegrid@width{\paperwidth}%
  \def\pagegrid@height{\paperheight}%
\else
  \def\pagegrid@width{\stockwidth}%
  \def\pagegrid@height{\stockheight}%
\fi
%    \end{macrocode}
%
%    \begin{macrocode}
\SetupKeyvalOptions{%
  family=pagegrid,%
  prefix=pagegrid@,%
}
\def\pagegrid@init{%
  \let\pagegrid@origin@a\@empty
  \let\pagegrid@origin@b\@empty
  \let\pagegrid@init\relax
}
\let\pagegrid@@init\pagegrid@init
\def\pagegrid@origin@a{bl}
\def\pagegrid@origin@b{tr}
\def\pagegrid@SetOrigin#1{%
  \pagegrid@init
  \ifx\pagegrid@origin@a\@empty
    \def\pagegrid@origin@a{#1}%
  \else
    \ifx\pagegrid@origin@b\@empty
    \else
      \let\pagegrid@origin@a\pagegrid@origin@b
    \fi
    \def\pagegrid@origin@b{#1}%
  \fi
}
\def\pagegrid@temp#1{%
  \DeclareVoidOption{#1}{\pagegrid@SetOrigin{#1}}%
  \@namedef{pagegrid@N@#1}{#1}%
}
\pagegrid@temp{bl}
\pagegrid@temp{br}
\pagegrid@temp{tl}
\pagegrid@temp{tr}
\def\pagegrid@temp#1#2{%
  \DeclareVoidOption{#2}{\pagegrid@SetOrigin{#1}}%
}%
\pagegrid@temp{bl}{lb}
\pagegrid@temp{br}{rb}
\pagegrid@temp{tl}{lt}
\pagegrid@temp{tr}{rt}
\pagegrid@temp{bl}{bottom-left}
\pagegrid@temp{br}{bottom-right}
\pagegrid@temp{tl}{top-left}
\pagegrid@temp{tr}{top-right}
\pagegrid@temp{bl}{left-bottom}
\pagegrid@temp{br}{right-bottom}
\pagegrid@temp{tl}{left-top}
\pagegrid@temp{tr}{right-top}
%    \end{macrocode}
%    \begin{macrocode}
\DeclareBoolOption[true]{enable}
\DeclareComplementaryOption{disable}{enable}
%    \end{macrocode}
%    \begin{macrocode}
\DeclareBoolOption{foreground}
%    \end{macrocode}
%    \begin{macrocode}
\newlength{\pagegrid@step}
\define@key{pagegrid}{step}{%
  \setlength{\pagegrid@step}{#1}%
}
%    \end{macrocode}
%    \begin{macrocode}
\DeclareStringOption[red]{firstcolor}
\DeclareStringOption[blue]{secondcolor}
%    \end{macrocode}
%    \begin{macrocode}
\DeclareBoolOption[true]{arrows}
\newlength\pagegrid@arrowlength
\pagegrid@arrowlength=\z@
\define@key{pagegrid}{arrowlength}{%
  \setlength{\pagegrid@arrowlength}{#1}%
}
%    \end{macrocode}
%    \begin{macrocode}
\define@key{pagegrid}{double}[true]{%
  \@ifundefined{pagegrid@double@#1}{%
    \PackageWarning{pagegrid}{%
      Unsupported value `#1' for option `double'.\MessageBreak
      Known values are:\MessageBreak
      `false', `first', `last', `true'.\MessageBreak
      Now `false' is used%
    }%
    \chardef\pagegrid@double\z@
  }{%
    \chardef\pagegrid@double\csname pagegrid@double@#1\endcsname\relax
  }%
}
\@namedef{pagegrid@double@false}{0}
\@namedef{pagegrid@double@first}{1}
\@namedef{pagegrid@double@last}{2}
\@namedef{pagegrid@double@true}{2}
\chardef\pagegrid@double\z@
%    \end{macrocode}
%    \begin{macrocode}
\newcommand*{\pagegridsetup}{%
  \let\pagegrid@init\pagegrid@@init
  \setkeys{pagegrid}%
}
%    \end{macrocode}
%    \begin{macrocode}
\pagegridsetup{%
  step=1mm%
}
\InputIfFileExists{pagegrid.cfg}{}%
\ProcessKeyvalOptions*\relax
\AtBeginDocument{%
  \ifdim\pagegrid@arrowlength>\z@
  \else
    \pagegrid@arrowlength=10\pagegrid@step
  \fi
}
%    \end{macrocode}
%
%    \begin{macrocode}
\def\pagegridShipoutDoubleBegin{%
  \begingroup
  \let\newlabel\@gobbletwo
  \let\zref@newlabel\@gobbletwo
  \let\@writefile\@gobbletwo
  \let\select@language\@gobble
}
\def\pagegridShipoutDoubleEnd{%
  \endgroup
}
\def\pagegrid@WriteDouble#1#2{%
  \immediate\write#1{%
    \@backslashchar csname %
    pagegridShipoutDouble#2%
    \@backslashchar endcsname%
  }%
}
\def\pagegrid@ShipoutDouble#1{%
  \begingroup
    \if@filesw
      \pagegrid@WriteDouble\@mainaux{Begin}%
      \ifx\@auxout\@partaux
        \pagegrid@WriteDouble\@partaux{Begin}%
        \def\pagegrid@temp{%
          \pagegrid@WriteDouble\@mainaux{End}%
          \pagegrid@WriteDouble\@partaux{End}%
        }%
      \else
        \def\pagegrid@temp{%
          \pagegrid@WriteDouble\@mainaux{End}%
        }%
      \fi
    \else
      \def\pagegrid@temp{}%
    \fi
    \let\protect\noexpand
    \AtBeginShipoutOriginalShipout\copy#1\relax
    \pagegrid@temp
  \endgroup
}
%    \end{macrocode}
%
%    \begin{macrocode}
\AtBeginShipout{%
  \ifdim\pagegrid@step>\z@
  \else
    \pagegrid@enablefalse
  \fi
  \ifpagegrid@enable
    \ifnum\pagegrid@double=\@ne
      \pagegrid@ShipoutDouble\AtBeginShipoutBox
    \else
      \ifnum\pagegrid@double=\tw@
        \@ifundefined{pagegrid@DoubleBox}{%
          \newbox\pagegrid@DoubleBox
        }{}%
        \setbox\pagegrid@DoubleBox=\copy\AtBeginShipoutBox
      \fi
    \fi
    \ifpagegrid@foreground
      \expandafter\AtBeginShipoutUpperLeftForeground
    \else
      \expandafter\AtBeginShipoutUpperLeft
    \fi
    {%
      \put(0,0){%
        \makebox(0,0)[lt]{%
          \begin{tikzpicture}[%
            bl/.style={},%
            br/.style={xshift=\pagegrid@width,xscale=-1},%
            tl/.style={yshift=\pagegrid@height,yscale=-1},%
            tr/.style={xshift=\pagegrid@width,%
                       yshift=\pagegrid@height,scale=-1}%
          ]%
            \useasboundingbox
              (0mm,\pagegrid@height) rectangle (0mm,\pagegrid@height);%
            \draw[%
              \pagegrid@origin@a,%
              step=\pagegrid@step,%
              style=help lines,%
              ultra thin%
            ] (0mm,0mm) grid (\pagegrid@width,\pagegrid@height);%
            \ifx\pagegrid@origin@b\@empty
            \else
              \draw[%
                \pagegrid@origin@b,%
                step=10\pagegrid@step,%
                {\pagegrid@secondcolor},%
                very thin%
              ] (0mm,0mm) grid (\pagegrid@width,\pagegrid@height);%
            \fi
            \draw[%
               \pagegrid@origin@a,%
               step=10\pagegrid@step,%
               {\pagegrid@firstcolor},%
               very thin%
            ] (0mm,0mm) grid (\pagegrid@width,\pagegrid@height);%
            \ifx\pagegrid@origin@b\@empty
            \else
              \draw[%
                \pagegrid@origin@b,%
                step=50\pagegrid@step,%
                {\pagegrid@secondcolor},%
                thick%
              ] (0mm,0mm) grid (\pagegrid@width,\pagegrid@height);%
            \fi
            \draw[%
              \pagegrid@origin@a,%
              step=50\pagegrid@step,%
              {\pagegrid@firstcolor},%
              thick%
            ] (0mm,0mm) grid (\pagegrid@width,\pagegrid@height);%
            \ifpagegrid@arrows
              \ifx\pagegrid@origin@b\@empty
              \else
                \draw[%
                  \pagegrid@origin@b,%
                  {\pagegrid@secondcolor},%
                  stroke,%
                  line width=1pt,%
                  line cap=round%
                ] (0mm,0mm) %
                -- (\pagegrid@arrowlength,\pagegrid@arrowlength) %
                   (\pagegrid@arrowlength,.5\pagegrid@arrowlength) %
                -- (\pagegrid@arrowlength,\pagegrid@arrowlength) %
                -- (.5\pagegrid@arrowlength,\pagegrid@arrowlength);%
              \fi
              \draw[%
                \pagegrid@origin@a,%
                {\pagegrid@firstcolor},%
                stroke,%
                line width=1pt,%
                line cap=round%
              ] (0mm,0mm) %
              -- (\pagegrid@arrowlength,\pagegrid@arrowlength) %
                 (\pagegrid@arrowlength,.5\pagegrid@arrowlength) %
              -- (\pagegrid@arrowlength,\pagegrid@arrowlength) %
              -- (.5\pagegrid@arrowlength,\pagegrid@arrowlength);%
            \fi
          \end{tikzpicture}%
        }%
      }%
    }%
    \ifnum\pagegrid@double=\tw@
      \pagegrid@ShipoutDouble\pagegrid@DoubleBox
    \fi
  \fi
}
%    \end{macrocode}
%
%    \begin{macrocode}
\pagegrid@AtEnd%
%</package>
%    \end{macrocode}
%
% \section{Test}
%
% \subsection{Catcode checks for loading}
%
%    \begin{macrocode}
%<*test1>
%    \end{macrocode}
%    \begin{macrocode}
\catcode`\{=1 %
\catcode`\}=2 %
\catcode`\#=6 %
\catcode`\@=11 %
\expandafter\ifx\csname count@\endcsname\relax
  \countdef\count@=255 %
\fi
\expandafter\ifx\csname @gobble\endcsname\relax
  \long\def\@gobble#1{}%
\fi
\expandafter\ifx\csname @firstofone\endcsname\relax
  \long\def\@firstofone#1{#1}%
\fi
\expandafter\ifx\csname loop\endcsname\relax
  \expandafter\@firstofone
\else
  \expandafter\@gobble
\fi
{%
  \def\loop#1\repeat{%
    \def\body{#1}%
    \iterate
  }%
  \def\iterate{%
    \body
      \let\next\iterate
    \else
      \let\next\relax
    \fi
    \next
  }%
  \let\repeat=\fi
}%
\def\RestoreCatcodes{}
\count@=0 %
\loop
  \edef\RestoreCatcodes{%
    \RestoreCatcodes
    \catcode\the\count@=\the\catcode\count@\relax
  }%
\ifnum\count@<255 %
  \advance\count@ 1 %
\repeat

\def\RangeCatcodeInvalid#1#2{%
  \count@=#1\relax
  \loop
    \catcode\count@=15 %
  \ifnum\count@<#2\relax
    \advance\count@ 1 %
  \repeat
}
\def\RangeCatcodeCheck#1#2#3{%
  \count@=#1\relax
  \loop
    \ifnum#3=\catcode\count@
    \else
      \errmessage{%
        Character \the\count@\space
        with wrong catcode \the\catcode\count@\space
        instead of \number#3%
      }%
    \fi
  \ifnum\count@<#2\relax
    \advance\count@ 1 %
  \repeat
}
\def\space{ }
\expandafter\ifx\csname LoadCommand\endcsname\relax
  \def\LoadCommand{\input pagegrid.sty\relax}%
\fi
\def\Test{%
  \RangeCatcodeInvalid{0}{47}%
  \RangeCatcodeInvalid{58}{64}%
  \RangeCatcodeInvalid{91}{96}%
  \RangeCatcodeInvalid{123}{255}%
  \catcode`\@=12 %
  \catcode`\\=0 %
  \catcode`\%=14 %
  \LoadCommand
  \RangeCatcodeCheck{0}{36}{15}%
  \RangeCatcodeCheck{37}{37}{14}%
  \RangeCatcodeCheck{38}{47}{15}%
  \RangeCatcodeCheck{48}{57}{12}%
  \RangeCatcodeCheck{58}{63}{15}%
  \RangeCatcodeCheck{64}{64}{12}%
  \RangeCatcodeCheck{65}{90}{11}%
  \RangeCatcodeCheck{91}{91}{15}%
  \RangeCatcodeCheck{92}{92}{0}%
  \RangeCatcodeCheck{93}{96}{15}%
  \RangeCatcodeCheck{97}{122}{11}%
  \RangeCatcodeCheck{123}{255}{15}%
  \RestoreCatcodes
}
\Test
\csname @@end\endcsname
\end
%    \end{macrocode}
%    \begin{macrocode}
%</test1>
%    \end{macrocode}
%
% \section{Installation}
%
% \subsection{Download}
%
% \paragraph{Package.} This package is available on
% CTAN\footnote{\CTANpkg{pagegrid}}:
% \begin{description}
% \item[\CTAN{macros/latex/contrib/oberdiek/pagegrid.dtx}] The source file.
% \item[\CTAN{macros/latex/contrib/oberdiek/pagegrid.pdf}] Documentation.
% \end{description}
%
%
% \paragraph{Bundle.} All the packages of the bundle `oberdiek'
% are also available in a TDS compliant ZIP archive. There
% the packages are already unpacked and the documentation files
% are generated. The files and directories obey the TDS standard.
% \begin{description}
% \item[\CTANinstall{install/macros/latex/contrib/oberdiek.tds.zip}]
% \end{description}
% \emph{TDS} refers to the standard ``A Directory Structure
% for \TeX\ Files'' (\CTAN{tds/tds.pdf}). Directories
% with \xfile{texmf} in their name are usually organized this way.
%
% \subsection{Bundle installation}
%
% \paragraph{Unpacking.} Unpack the \xfile{oberdiek.tds.zip} in the
% TDS tree (also known as \xfile{texmf} tree) of your choice.
% Example (linux):
% \begin{quote}
%   |unzip oberdiek.tds.zip -d ~/texmf|
% \end{quote}
%
% \paragraph{Script installation.}
% Check the directory \xfile{TDS:scripts/oberdiek/} for
% scripts that need further installation steps.
%
% \subsection{Package installation}
%
% \paragraph{Unpacking.} The \xfile{.dtx} file is a self-extracting
% \docstrip\ archive. The files are extracted by running the
% \xfile{.dtx} through \plainTeX:
% \begin{quote}
%   \verb|tex pagegrid.dtx|
% \end{quote}
%
% \paragraph{TDS.} Now the different files must be moved into
% the different directories in your installation TDS tree
% (also known as \xfile{texmf} tree):
% \begin{quote}
% \def\t{^^A
% \begin{tabular}{@{}>{\ttfamily}l@{ $\rightarrow$ }>{\ttfamily}l@{}}
%   pagegrid.sty & tex/latex/oberdiek/pagegrid.sty\\
%   pagegrid.pdf & doc/latex/oberdiek/pagegrid.pdf\\
%   test/pagegrid-test1.tex & doc/latex/oberdiek/test/pagegrid-test1.tex\\
%   pagegrid.dtx & source/latex/oberdiek/pagegrid.dtx\\
% \end{tabular}^^A
% }^^A
% \sbox0{\t}^^A
% \ifdim\wd0>\linewidth
%   \begingroup
%     \advance\linewidth by\leftmargin
%     \advance\linewidth by\rightmargin
%   \edef\x{\endgroup
%     \def\noexpand\lw{\the\linewidth}^^A
%   }\x
%   \def\lwbox{^^A
%     \leavevmode
%     \hbox to \linewidth{^^A
%       \kern-\leftmargin\relax
%       \hss
%       \usebox0
%       \hss
%       \kern-\rightmargin\relax
%     }^^A
%   }^^A
%   \ifdim\wd0>\lw
%     \sbox0{\small\t}^^A
%     \ifdim\wd0>\linewidth
%       \ifdim\wd0>\lw
%         \sbox0{\footnotesize\t}^^A
%         \ifdim\wd0>\linewidth
%           \ifdim\wd0>\lw
%             \sbox0{\scriptsize\t}^^A
%             \ifdim\wd0>\linewidth
%               \ifdim\wd0>\lw
%                 \sbox0{\tiny\t}^^A
%                 \ifdim\wd0>\linewidth
%                   \lwbox
%                 \else
%                   \usebox0
%                 \fi
%               \else
%                 \lwbox
%               \fi
%             \else
%               \usebox0
%             \fi
%           \else
%             \lwbox
%           \fi
%         \else
%           \usebox0
%         \fi
%       \else
%         \lwbox
%       \fi
%     \else
%       \usebox0
%     \fi
%   \else
%     \lwbox
%   \fi
% \else
%   \usebox0
% \fi
% \end{quote}
% If you have a \xfile{docstrip.cfg} that configures and enables \docstrip's
% TDS installing feature, then some files can already be in the right
% place, see the documentation of \docstrip.
%
% \subsection{Refresh file name databases}
%
% If your \TeX~distribution
% (\TeX\,Live, \mikTeX, \dots) relies on file name databases, you must refresh
% these. For example, \TeX\,Live\ users run \verb|texhash| or
% \verb|mktexlsr|.
%
% \subsection{Some details for the interested}
%
% \paragraph{Unpacking with \LaTeX.}
% The \xfile{.dtx} chooses its action depending on the format:
% \begin{description}
% \item[\plainTeX:] Run \docstrip\ and extract the files.
% \item[\LaTeX:] Generate the documentation.
% \end{description}
% If you insist on using \LaTeX\ for \docstrip\ (really,
% \docstrip\ does not need \LaTeX), then inform the autodetect routine
% about your intention:
% \begin{quote}
%   \verb|latex \let\install=y\input{pagegrid.dtx}|
% \end{quote}
% Do not forget to quote the argument according to the demands
% of your shell.
%
% \paragraph{Generating the documentation.}
% You can use both the \xfile{.dtx} or the \xfile{.drv} to generate
% the documentation. The process can be configured by the
% configuration file \xfile{ltxdoc.cfg}. For instance, put this
% line into this file, if you want to have A4 as paper format:
% \begin{quote}
%   \verb|\PassOptionsToClass{a4paper}{article}|
% \end{quote}
% An example follows how to generate the
% documentation with pdf\LaTeX:
% \begin{quote}
%\begin{verbatim}
%pdflatex pagegrid.dtx
%makeindex -s gind.ist pagegrid.idx
%pdflatex pagegrid.dtx
%makeindex -s gind.ist pagegrid.idx
%pdflatex pagegrid.dtx
%\end{verbatim}
% \end{quote}
%
% \section{Acknowledgement}
%
% \begin{description}
% \item[Klaus Braune:]
%  He provided the idea and the first \xpackage{tikz} code.
% \end{description}
%
% \begin{History}
%   \begin{Version}{2009/11/06 v1.0}
%   \item
%     The first version.
%   \end{Version}
%   \begin{Version}{2009/11/06 v1.1}
%   \item
%     Option \xoption{foreground} added.
%   \end{Version}
%   \begin{Version}{2009/12/02 v1.2}
%   \item
%     Color options, arrow options added.
%   \item
%     Names for origin options changed.
%   \end{Version}
%   \begin{Version}{2009/12/03 v1.3}
%   \item
%     Option \xoption{double} added.
%   \item
%     First CTAN release.
%   \end{Version}
%   \begin{Version}{2009/12/04 v1.4}
%   \item
%     Option \xoption{double}: Some unwanted side effects removed.
%   \end{Version}
%   \begin{Version}{2016/05/16 v1.5}
%   \item
%     Documentation updates.
%   \end{Version}
% \end{History}
%
% \PrintIndex
%
% \Finale
\endinput
|
% \end{quote}
% Do not forget to quote the argument according to the demands
% of your shell.
%
% \paragraph{Generating the documentation.}
% You can use both the \xfile{.dtx} or the \xfile{.drv} to generate
% the documentation. The process can be configured by the
% configuration file \xfile{ltxdoc.cfg}. For instance, put this
% line into this file, if you want to have A4 as paper format:
% \begin{quote}
%   \verb|\PassOptionsToClass{a4paper}{article}|
% \end{quote}
% An example follows how to generate the
% documentation with pdf\LaTeX:
% \begin{quote}
%\begin{verbatim}
%pdflatex pagegrid.dtx
%makeindex -s gind.ist pagegrid.idx
%pdflatex pagegrid.dtx
%makeindex -s gind.ist pagegrid.idx
%pdflatex pagegrid.dtx
%\end{verbatim}
% \end{quote}
%
% \section{Acknowledgement}
%
% \begin{description}
% \item[Klaus Braune:]
%  He provided the idea and the first \xpackage{tikz} code.
% \end{description}
%
% \begin{History}
%   \begin{Version}{2009/11/06 v1.0}
%   \item
%     The first version.
%   \end{Version}
%   \begin{Version}{2009/11/06 v1.1}
%   \item
%     Option \xoption{foreground} added.
%   \end{Version}
%   \begin{Version}{2009/12/02 v1.2}
%   \item
%     Color options, arrow options added.
%   \item
%     Names for origin options changed.
%   \end{Version}
%   \begin{Version}{2009/12/03 v1.3}
%   \item
%     Option \xoption{double} added.
%   \item
%     First CTAN release.
%   \end{Version}
%   \begin{Version}{2009/12/04 v1.4}
%   \item
%     Option \xoption{double}: Some unwanted side effects removed.
%   \end{Version}
%   \begin{Version}{2016/05/16 v1.5}
%   \item
%     Documentation updates.
%   \end{Version}
% \end{History}
%
% \PrintIndex
%
% \Finale
\endinput
|
% \end{quote}
% Do not forget to quote the argument according to the demands
% of your shell.
%
% \paragraph{Generating the documentation.}
% You can use both the \xfile{.dtx} or the \xfile{.drv} to generate
% the documentation. The process can be configured by the
% configuration file \xfile{ltxdoc.cfg}. For instance, put this
% line into this file, if you want to have A4 as paper format:
% \begin{quote}
%   \verb|\PassOptionsToClass{a4paper}{article}|
% \end{quote}
% An example follows how to generate the
% documentation with pdf\LaTeX:
% \begin{quote}
%\begin{verbatim}
%pdflatex pagegrid.dtx
%makeindex -s gind.ist pagegrid.idx
%pdflatex pagegrid.dtx
%makeindex -s gind.ist pagegrid.idx
%pdflatex pagegrid.dtx
%\end{verbatim}
% \end{quote}
%
% \section{Acknowledgement}
%
% \begin{description}
% \item[Klaus Braune:]
%  He provided the idea and the first \xpackage{tikz} code.
% \end{description}
%
% \begin{History}
%   \begin{Version}{2009/11/06 v1.0}
%   \item
%     The first version.
%   \end{Version}
%   \begin{Version}{2009/11/06 v1.1}
%   \item
%     Option \xoption{foreground} added.
%   \end{Version}
%   \begin{Version}{2009/12/02 v1.2}
%   \item
%     Color options, arrow options added.
%   \item
%     Names for origin options changed.
%   \end{Version}
%   \begin{Version}{2009/12/03 v1.3}
%   \item
%     Option \xoption{double} added.
%   \item
%     First CTAN release.
%   \end{Version}
%   \begin{Version}{2009/12/04 v1.4}
%   \item
%     Option \xoption{double}: Some unwanted side effects removed.
%   \end{Version}
%   \begin{Version}{2016/05/16 v1.5}
%   \item
%     Documentation updates.
%   \end{Version}
% \end{History}
%
% \PrintIndex
%
% \Finale
\endinput
|
% \end{quote}
% Do not forget to quote the argument according to the demands
% of your shell.
%
% \paragraph{Generating the documentation.}
% You can use both the \xfile{.dtx} or the \xfile{.drv} to generate
% the documentation. The process can be configured by the
% configuration file \xfile{ltxdoc.cfg}. For instance, put this
% line into this file, if you want to have A4 as paper format:
% \begin{quote}
%   \verb|\PassOptionsToClass{a4paper}{article}|
% \end{quote}
% An example follows how to generate the
% documentation with pdf\LaTeX:
% \begin{quote}
%\begin{verbatim}
%pdflatex pagegrid.dtx
%makeindex -s gind.ist pagegrid.idx
%pdflatex pagegrid.dtx
%makeindex -s gind.ist pagegrid.idx
%pdflatex pagegrid.dtx
%\end{verbatim}
% \end{quote}
%
% \section{Acknowledgement}
%
% \begin{description}
% \item[Klaus Braune:]
%  He provided the idea and the first \xpackage{tikz} code.
% \end{description}
%
% \begin{History}
%   \begin{Version}{2009/11/06 v1.0}
%   \item
%     The first version.
%   \end{Version}
%   \begin{Version}{2009/11/06 v1.1}
%   \item
%     Option \xoption{foreground} added.
%   \end{Version}
%   \begin{Version}{2009/12/02 v1.2}
%   \item
%     Color options, arrow options added.
%   \item
%     Names for origin options changed.
%   \end{Version}
%   \begin{Version}{2009/12/03 v1.3}
%   \item
%     Option \xoption{double} added.
%   \item
%     First CTAN release.
%   \end{Version}
%   \begin{Version}{2009/12/04 v1.4}
%   \item
%     Option \xoption{double}: Some unwanted side effects removed.
%   \end{Version}
%   \begin{Version}{2016/05/16 v1.5}
%   \item
%     Documentation updates.
%   \end{Version}
% \end{History}
%
% \PrintIndex
%
% \Finale
\endinput
