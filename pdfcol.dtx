% \iffalse meta-comment
%
% File: pdfcol.dtx
% Version: 2018/11/01 v1.5
% Info: Handle new color stacks for pdfTeX
%
% Copyright (C)
%    2007 Heiko Oberdiek
%    2016-2019 Oberdiek Package Support Group
%    https://github.com/ho-tex/oberdiek/issues
%
% This work may be distributed and/or modified under the
% conditions of the LaTeX Project Public License, either
% version 1.3c of this license or (at your option) any later
% version. This version of this license is in
%    https://www.latex-project.org/lppl/lppl-1-3c.txt
% and the latest version of this license is in
%    https://www.latex-project.org/lppl.txt
% and version 1.3 or later is part of all distributions of
% LaTeX version 2005/12/01 or later.
%
% This work has the LPPL maintenance status "maintained".
%
% The Current Maintainers of this work are
% Heiko Oberdiek and the Oberdiek Package Support Group
% https://github.com/ho-tex/oberdiek/issues
%
% The Base Interpreter refers to any `TeX-Format',
% because some files are installed in TDS:tex/generic//.
%
% This work consists of the main source file pdfcol.dtx
% and the derived files
%    pdfcol.sty, pdfcol.pdf, pdfcol.ins, pdfcol.drv, pdfcol-test1.tex,
%    pdfcol-test2.tex, pdfcol-test3.tex, pdfcol-test4.tex.
%
% Distribution:
%    CTAN:macros/latex/contrib/oberdiek/pdfcol.dtx
%    CTAN:macros/latex/contrib/oberdiek/pdfcol.pdf
%
% Unpacking:
%    (a) If pdfcol.ins is present:
%           tex pdfcol.ins
%    (b) Without pdfcol.ins:
%           tex pdfcol.dtx
%    (c) If you insist on using LaTeX
%           latex \let\install=y% \iffalse meta-comment
%
% File: pdfcol.dtx
% Version: 2018/11/01 v1.5
% Info: Handle new color stacks for pdfTeX
%
% Copyright (C)
%    2007 Heiko Oberdiek
%    2016-2019 Oberdiek Package Support Group
%    https://github.com/ho-tex/oberdiek/issues
%
% This work may be distributed and/or modified under the
% conditions of the LaTeX Project Public License, either
% version 1.3c of this license or (at your option) any later
% version. This version of this license is in
%    https://www.latex-project.org/lppl/lppl-1-3c.txt
% and the latest version of this license is in
%    https://www.latex-project.org/lppl.txt
% and version 1.3 or later is part of all distributions of
% LaTeX version 2005/12/01 or later.
%
% This work has the LPPL maintenance status "maintained".
%
% The Current Maintainers of this work are
% Heiko Oberdiek and the Oberdiek Package Support Group
% https://github.com/ho-tex/oberdiek/issues
%
% The Base Interpreter refers to any `TeX-Format',
% because some files are installed in TDS:tex/generic//.
%
% This work consists of the main source file pdfcol.dtx
% and the derived files
%    pdfcol.sty, pdfcol.pdf, pdfcol.ins, pdfcol.drv, pdfcol-test1.tex,
%    pdfcol-test2.tex, pdfcol-test3.tex, pdfcol-test4.tex.
%
% Distribution:
%    CTAN:macros/latex/contrib/oberdiek/pdfcol.dtx
%    CTAN:macros/latex/contrib/oberdiek/pdfcol.pdf
%
% Unpacking:
%    (a) If pdfcol.ins is present:
%           tex pdfcol.ins
%    (b) Without pdfcol.ins:
%           tex pdfcol.dtx
%    (c) If you insist on using LaTeX
%           latex \let\install=y% \iffalse meta-comment
%
% File: pdfcol.dtx
% Version: 2018/11/01 v1.5
% Info: Handle new color stacks for pdfTeX
%
% Copyright (C)
%    2007 Heiko Oberdiek
%    2016-2019 Oberdiek Package Support Group
%    https://github.com/ho-tex/oberdiek/issues
%
% This work may be distributed and/or modified under the
% conditions of the LaTeX Project Public License, either
% version 1.3c of this license or (at your option) any later
% version. This version of this license is in
%    https://www.latex-project.org/lppl/lppl-1-3c.txt
% and the latest version of this license is in
%    https://www.latex-project.org/lppl.txt
% and version 1.3 or later is part of all distributions of
% LaTeX version 2005/12/01 or later.
%
% This work has the LPPL maintenance status "maintained".
%
% The Current Maintainers of this work are
% Heiko Oberdiek and the Oberdiek Package Support Group
% https://github.com/ho-tex/oberdiek/issues
%
% The Base Interpreter refers to any `TeX-Format',
% because some files are installed in TDS:tex/generic//.
%
% This work consists of the main source file pdfcol.dtx
% and the derived files
%    pdfcol.sty, pdfcol.pdf, pdfcol.ins, pdfcol.drv, pdfcol-test1.tex,
%    pdfcol-test2.tex, pdfcol-test3.tex, pdfcol-test4.tex.
%
% Distribution:
%    CTAN:macros/latex/contrib/oberdiek/pdfcol.dtx
%    CTAN:macros/latex/contrib/oberdiek/pdfcol.pdf
%
% Unpacking:
%    (a) If pdfcol.ins is present:
%           tex pdfcol.ins
%    (b) Without pdfcol.ins:
%           tex pdfcol.dtx
%    (c) If you insist on using LaTeX
%           latex \let\install=y% \iffalse meta-comment
%
% File: pdfcol.dtx
% Version: 2018/11/01 v1.5
% Info: Handle new color stacks for pdfTeX
%
% Copyright (C)
%    2007 Heiko Oberdiek
%    2016-2019 Oberdiek Package Support Group
%    https://github.com/ho-tex/oberdiek/issues
%
% This work may be distributed and/or modified under the
% conditions of the LaTeX Project Public License, either
% version 1.3c of this license or (at your option) any later
% version. This version of this license is in
%    https://www.latex-project.org/lppl/lppl-1-3c.txt
% and the latest version of this license is in
%    https://www.latex-project.org/lppl.txt
% and version 1.3 or later is part of all distributions of
% LaTeX version 2005/12/01 or later.
%
% This work has the LPPL maintenance status "maintained".
%
% The Current Maintainers of this work are
% Heiko Oberdiek and the Oberdiek Package Support Group
% https://github.com/ho-tex/oberdiek/issues
%
% The Base Interpreter refers to any `TeX-Format',
% because some files are installed in TDS:tex/generic//.
%
% This work consists of the main source file pdfcol.dtx
% and the derived files
%    pdfcol.sty, pdfcol.pdf, pdfcol.ins, pdfcol.drv, pdfcol-test1.tex,
%    pdfcol-test2.tex, pdfcol-test3.tex, pdfcol-test4.tex.
%
% Distribution:
%    CTAN:macros/latex/contrib/oberdiek/pdfcol.dtx
%    CTAN:macros/latex/contrib/oberdiek/pdfcol.pdf
%
% Unpacking:
%    (a) If pdfcol.ins is present:
%           tex pdfcol.ins
%    (b) Without pdfcol.ins:
%           tex pdfcol.dtx
%    (c) If you insist on using LaTeX
%           latex \let\install=y\input{pdfcol.dtx}
%        (quote the arguments according to the demands of your shell)
%
% Documentation:
%    (a) If pdfcol.drv is present:
%           latex pdfcol.drv
%    (b) Without pdfcol.drv:
%           latex pdfcol.dtx; ...
%    The class ltxdoc loads the configuration file ltxdoc.cfg
%    if available. Here you can specify further options, e.g.
%    use A4 as paper format:
%       \PassOptionsToClass{a4paper}{article}
%
%    Programm calls to get the documentation (example):
%       pdflatex pdfcol.dtx
%       makeindex -s gind.ist pdfcol.idx
%       pdflatex pdfcol.dtx
%       makeindex -s gind.ist pdfcol.idx
%       pdflatex pdfcol.dtx
%
% Installation:
%    TDS:tex/generic/oberdiek/pdfcol.sty
%    TDS:doc/latex/oberdiek/pdfcol.pdf
%    TDS:source/latex/oberdiek/pdfcol.dtx
%
%<*ignore>
\begingroup
  \catcode123=1 %
  \catcode125=2 %
  \def\x{LaTeX2e}%
\expandafter\endgroup
\ifcase 0\ifx\install y1\fi\expandafter
         \ifx\csname processbatchFile\endcsname\relax\else1\fi
         \ifx\fmtname\x\else 1\fi\relax
\else\csname fi\endcsname
%</ignore>
%<*install>
\input docstrip.tex
\Msg{************************************************************************}
\Msg{* Installation}
\Msg{* Package: pdfcol 2018/11/01 v1.5 Handle new color stacks for pdfTeX (HO)}
\Msg{************************************************************************}

\keepsilent
\askforoverwritefalse

\let\MetaPrefix\relax
\preamble

This is a generated file.

Project: pdfcol
Version: 2018/11/01 v1.5

Copyright (C)
   2007 Heiko Oberdiek
   2016-2019 Oberdiek Package Support Group

This work may be distributed and/or modified under the
conditions of the LaTeX Project Public License, either
version 1.3c of this license or (at your option) any later
version. This version of this license is in
   https://www.latex-project.org/lppl/lppl-1-3c.txt
and the latest version of this license is in
   https://www.latex-project.org/lppl.txt
and version 1.3 or later is part of all distributions of
LaTeX version 2005/12/01 or later.

This work has the LPPL maintenance status "maintained".

The Current Maintainers of this work are
Heiko Oberdiek and the Oberdiek Package Support Group
https://github.com/ho-tex/oberdiek/issues


The Base Interpreter refers to any `TeX-Format',
because some files are installed in TDS:tex/generic//.

This work consists of the main source file pdfcol.dtx
and the derived files
   pdfcol.sty, pdfcol.pdf, pdfcol.ins, pdfcol.drv, pdfcol-test1.tex,
   pdfcol-test2.tex, pdfcol-test3.tex, pdfcol-test4.tex.

\endpreamble
\let\MetaPrefix\DoubleperCent

\generate{%
  \file{pdfcol.ins}{\from{pdfcol.dtx}{install}}%
  \file{pdfcol.drv}{\from{pdfcol.dtx}{driver}}%
  \usedir{tex/generic/oberdiek}%
  \file{pdfcol.sty}{\from{pdfcol.dtx}{package}}%
%  \usedir{doc/latex/oberdiek/test}%
%  \file{pdfcol-test1.tex}{\from{pdfcol.dtx}{test1}}%
%  \file{pdfcol-test2.tex}{\from{pdfcol.dtx}{test2}}%
%  \file{pdfcol-test3.tex}{\from{pdfcol.dtx}{test3}}%
%  \file{pdfcol-test4.tex}{\from{pdfcol.dtx}{test4}}%
}

\catcode32=13\relax% active space
\let =\space%
\Msg{************************************************************************}
\Msg{*}
\Msg{* To finish the installation you have to move the following}
\Msg{* file into a directory searched by TeX:}
\Msg{*}
\Msg{*     pdfcol.sty}
\Msg{*}
\Msg{* To produce the documentation run the file `pdfcol.drv'}
\Msg{* through LaTeX.}
\Msg{*}
\Msg{* Happy TeXing!}
\Msg{*}
\Msg{************************************************************************}

\endbatchfile
%</install>
%<*ignore>
\fi
%</ignore>
%<*driver>
\NeedsTeXFormat{LaTeX2e}
\ProvidesFile{pdfcol.drv}%
  [2018/11/01 v1.5 Handle new color stacks for pdfTeX (HO)]%
\documentclass{ltxdoc}
\usepackage{holtxdoc}[2011/11/22]
\begin{document}
  \DocInput{pdfcol.dtx}%
\end{document}
%</driver>
% \fi
%
%
%
% \GetFileInfo{pdfcol.drv}
%
% \title{The \xpackage{pdfcol} package}
% \date{2018/11/01 v1.5}
% \author{Heiko Oberdiek\thanks
% {Please report any issues at \url{https://github.com/ho-tex/oberdiek/issues}}}
%
% \maketitle
%
% \begin{abstract}
% Since version 1.40 \pdfTeX\ supports color stacks.
% The driver file \xfile{pdftex.def} for package \xpackage{color}
% defines and uses a main color stack since version v0.04b.
% Package \xpackage{pdfcol} is intended for package writers.
% It defines macros for setting and maintaining new color stacks.
% \end{abstract}
%
% \tableofcontents
%
% \section{Documentation}
%
% Version 1.40 of \pdfTeX\ adds new primitives \cs{pdfcolorstackinit}
% and \cs{pdfcolorstack}. Now color stacks can be defined and used.
% A main color stack is maintained by the driver file \xfile{pdftex.def}
% similar to dvips or dvipdfm. However the number of color stacks
% is not limited to one in \pdfTeX. Thus further color problems
% can now be solved, such as footnotes across pages or text
% that is set in parallel columns (e.g. packages \xpackage{parallel}
% or \xpackage{parcolumn}). Unlike the main color stack,
% the support by additional color stacks cannot be done in
% a transparent manner.
%
% This package \xpackage{pdfcol} provides an easier interface to
% additional color stacks without the need to use the
% low level primitives.
%
% \subsection{Requirements}
% \label{sec:req}
%
% \begin{itemize}
% \item
%   \pdfTeX\ 1.40 or greater.
% \item
%   \pdfTeX in PDF mode. (I don't know a DVI driver that
%   support several color stacks.)
% \item
%   \xfile{pdftex.def} 2007/01/02 v0.04b.
% \end{itemize}
% Package \xpackage{pdfcol} checks the requirements and
% sets switch \cs{ifpdfcolAvailable} accordingly.
%
% \subsection{Interface}
%
% \begin{declcs}{ifpdfcolAvailable}
% \end{declcs}
% If the requirements of section \ref{sec:req} are met the
% switch \cs{ifpdfcolAvailable} behaves as \cs{iftrue}.
% Otherwise the other interface macros in this section will
% be disabled with a message. Also the first use of such a
% macro will print a message. The messages are print to
% the \xext{log} file only if \pdfTeX\ is not used in PDF mode.
%
% \begin{declcs}{pdfcolErrorNoStacks}
% \end{declcs}
% The first call of \cs{pdfcolErrorNoStacks} prints an error
% message, if color stacks are not available.
%
% \begin{declcs}{pdfcolInitStack} \M{name}
% \end{declcs}
% A new color stack is initialized by \cs{pdfcolInitStack}.
% The \meta{name} is used for indentifying the stack. It usually
% consists of letters and digits. (The name must survive a \cs{csname}.)
%
% The intension of the macro is the definition of an additional
% color stack. Thus the stack is not page bounded like the
% main color stack. Black (\texttt{0 g 0 G}) is used as initial
% color value. And colors are written with modifier \texttt{direct}
% that means without setting the current transfer matrix and changing
% the current point (see documentation of \pdfTeX\ for
% |\pdfliteral direct{...}|).
%
% \begin{declcs}{pdfcolIfStackExists} \M{name} \M{then} \M{else}
% \end{declcs}
% Macro \cs{pdfcolIfStackExists} checks whether color stack \meta{name}
% exists. In case of success argument \meta{then} is executed
% and \meta{else} otherwise.
%
% \begin{declcs}{pdfcolSwitchStack} \M{name}
% \end{declcs}
% Macro \cs{pdfcolSwitchStack} switches the color stack. The color macros
% of package \xpackage{color} (or \xpackage{xcolor}) now uses the
% new color stack with name \meta{name}.
%
% \begin{declcs}{pdfcolSetCurrentColor}
% \end{declcs}
% Macro \cs{pdfcolSetCurrentColor} replaces the topmost
% entry of the stack by the current color (\cs{current@color}).
%
% \begin{declcs}{pdfcolSetCurrent} \M{name}
% \end{declcs}
% Macro \cs{pdfcolSetCurrent} sets the color that is read in
% the top-most entry of color stack \meta{name}. If \meta{name}
% is empty, the default color stack is used.
%
% \StopEventually{
% }
%
% \section{Implementation}
%
%    \begin{macrocode}
%<*package>
%    \end{macrocode}
%
% \subsection{Reload check and package identification}
%    Reload check, especially if the package is not used with \LaTeX.
%    \begin{macrocode}
\begingroup\catcode61\catcode48\catcode32=10\relax%
  \catcode13=5 % ^^M
  \endlinechar=13 %
  \catcode35=6 % #
  \catcode39=12 % '
  \catcode44=12 % ,
  \catcode45=12 % -
  \catcode46=12 % .
  \catcode58=12 % :
  \catcode64=11 % @
  \catcode123=1 % {
  \catcode125=2 % }
  \expandafter\let\expandafter\x\csname ver@pdfcol.sty\endcsname
  \ifx\x\relax % plain-TeX, first loading
  \else
    \def\empty{}%
    \ifx\x\empty % LaTeX, first loading,
      % variable is initialized, but \ProvidesPackage not yet seen
    \else
      \expandafter\ifx\csname PackageInfo\endcsname\relax
        \def\x#1#2{%
          \immediate\write-1{Package #1 Info: #2.}%
        }%
      \else
        \def\x#1#2{\PackageInfo{#1}{#2, stopped}}%
      \fi
      \x{pdfcol}{The package is already loaded}%
      \aftergroup\endinput
    \fi
  \fi
\endgroup%
%    \end{macrocode}
%    Package identification:
%    \begin{macrocode}
\begingroup\catcode61\catcode48\catcode32=10\relax%
  \catcode13=5 % ^^M
  \endlinechar=13 %
  \catcode35=6 % #
  \catcode39=12 % '
  \catcode40=12 % (
  \catcode41=12 % )
  \catcode44=12 % ,
  \catcode45=12 % -
  \catcode46=12 % .
  \catcode47=12 % /
  \catcode58=12 % :
  \catcode64=11 % @
  \catcode91=12 % [
  \catcode93=12 % ]
  \catcode123=1 % {
  \catcode125=2 % }
  \expandafter\ifx\csname ProvidesPackage\endcsname\relax
    \def\x#1#2#3[#4]{\endgroup
      \immediate\write-1{Package: #3 #4}%
      \xdef#1{#4}%
    }%
  \else
    \def\x#1#2[#3]{\endgroup
      #2[{#3}]%
      \ifx#1\@undefined
        \xdef#1{#3}%
      \fi
      \ifx#1\relax
        \xdef#1{#3}%
      \fi
    }%
  \fi
\expandafter\x\csname ver@pdfcol.sty\endcsname
\ProvidesPackage{pdfcol}%
  [2018/11/01 v1.5 Handle new color stacks for pdfTeX (HO)]%
%    \end{macrocode}
%
% \subsection{Catcodes}
%
%    \begin{macrocode}
\begingroup\catcode61\catcode48\catcode32=10\relax%
  \catcode13=5 % ^^M
  \endlinechar=13 %
  \catcode123=1 % {
  \catcode125=2 % }
  \catcode64=11 % @
  \def\x{\endgroup
    \expandafter\edef\csname PDFCOL@AtEnd\endcsname{%
      \endlinechar=\the\endlinechar\relax
      \catcode13=\the\catcode13\relax
      \catcode32=\the\catcode32\relax
      \catcode35=\the\catcode35\relax
      \catcode61=\the\catcode61\relax
      \catcode64=\the\catcode64\relax
      \catcode123=\the\catcode123\relax
      \catcode125=\the\catcode125\relax
    }%
  }%
\x\catcode61\catcode48\catcode32=10\relax%
\catcode13=5 % ^^M
\endlinechar=13 %
\catcode35=6 % #
\catcode64=11 % @
\catcode123=1 % {
\catcode125=2 % }
\def\TMP@EnsureCode#1#2{%
  \edef\PDFCOL@AtEnd{%
    \PDFCOL@AtEnd
    \catcode#1=\the\catcode#1\relax
  }%
  \catcode#1=#2\relax
}
\TMP@EnsureCode{39}{12}% '
\TMP@EnsureCode{40}{12}% (
\TMP@EnsureCode{41}{12}% )
\TMP@EnsureCode{43}{12}% +
\TMP@EnsureCode{44}{12}% ,
\TMP@EnsureCode{46}{12}% .
\TMP@EnsureCode{47}{12}% /
\TMP@EnsureCode{91}{12}% [
\TMP@EnsureCode{93}{12}% ]
\TMP@EnsureCode{96}{12}% `
\edef\PDFCOL@AtEnd{\PDFCOL@AtEnd\noexpand\endinput}
%    \end{macrocode}
%
% \subsection{Check requirements}
%
%    \begin{macro}{\PDFCOL@RequirePackage}
%    \begin{macrocode}
\begingroup\expandafter\expandafter\expandafter\endgroup
\expandafter\ifx\csname RequirePackage\endcsname\relax
  \def\PDFCOL@RequirePackage#1[#2]{\input #1.sty\relax}%
\else
  \def\PDFCOL@RequirePackage#1[#2]{%
    \RequirePackage{#1}[{#2}]%
  }%
\fi
%    \end{macrocode}
%    \end{macro}
%
% LuaTeX Compatability
%    \begin{macrocode}
\ifx\pdfextension\@undefined\else
 \def\pdfcolorstackinit {\pdffeedback colorstackinit}
 \protected\def\pdfcolorstack     {\pdfextension colorstack}
\fi
%    \end{macrocode}
%
%    \begin{macrocode}
\PDFCOL@RequirePackage{ltxcmds}[2010/03/01]
%    \end{macrocode}
%
%    \begin{macro}{ifpdfcolAvailable}
%    \begin{macrocode}
\ltx@newif\ifpdfcolAvailable
\pdfcolAvailabletrue
%    \end{macrocode}
%    \end{macro}
%
% \subsubsection{Check package \xpackage{luacolor}}
%
%    \begin{macrocode}
\ltx@newif\ifPDFCOL@luacolor
\begingroup\expandafter\expandafter\expandafter\endgroup
\expandafter\ifx\csname ver@luacolor.sty\endcsname\relax
  \PDFCOL@luacolorfalse
\else
  \PDFCOL@luacolortrue
\fi
%    \end{macrocode}
%
% \subsubsection{Check PDF mode}
%
%    \begin{macrocode}
\PDFCOL@RequirePackage{infwarerr}[2007/09/09]
\PDFCOL@RequirePackage{ifpdf}[2007/09/09]
\ifcase\ifpdf\ifPDFCOL@luacolor 1\fi\else 1\fi0 %
  \def\PDFCOL@Message{%
    \@PackageWarningNoLine{pdfcol}%
  }%
\else
  \pdfcolAvailablefalse
  \def\PDFCOL@Message{%
    \@PackageInfoNoLine{pdfcol}%
  }%
  \PDFCOL@Message{%
    Interface disabled because of %
    \ifPDFCOL@luacolor
      package `luacolor'%
    \else
      missing PDF mode of pdfTeX%
    \fi
  }%
\fi
%    \end{macrocode}
%
% \subsubsection{Check version of \pdfTeX}
%
%    \begin{macrocode}
\ifpdfcolAvailable
  \begingroup\expandafter\expandafter\expandafter\endgroup
  \expandafter\ifx\csname pdfcolorstack\endcsname\relax
    \pdfcolAvailablefalse
    \PDFCOL@Message{%
      Interface disabled because of too old pdfTeX.\MessageBreak
      Required is version 1.40+ for \string\pdfcolorstack
    }%
  \fi
\fi
\ifpdfcolAvailable
  \begingroup\expandafter\expandafter\expandafter\endgroup
  \expandafter\ifx\csname pdfcolorstack\endcsname\relax
    \pdfcolAvailablefalse
    \PDFCOL@Message{%
      Interface disabled because of too old pdfTeX.\MessageBreak
      Required is version 1.40+ for \string\pdfcolorstackinit
    }%
  \fi
\fi
%    \end{macrocode}
%
% \subsubsection{Check \xfile{pdftex.def}}
%
%    \begin{macrocode}
\ifpdfcolAvailable
  \begingroup\expandafter\expandafter\expandafter\endgroup
  \expandafter\ifx\csname @pdfcolorstack\endcsname\relax
%    \end{macrocode}
%    Try to load package color if it is not yet loaded (\LaTeX\ case).
%    \begin{macrocode}
    \begingroup\expandafter\expandafter\expandafter\endgroup
    \expandafter\ifx\csname ver@color.sty\endcsname\relax
      \begingroup\expandafter\expandafter\expandafter\endgroup
      \expandafter\ifx\csname documentclass\endcsname\relax
      \else
        \RequirePackage[pdftex]{color}\relax
      \fi
    \fi
    \begingroup\expandafter\expandafter\expandafter\endgroup
    \expandafter\ifx\csname @pdfcolorstack\endcsname\relax
      \pdfcolAvailablefalse
      \PDFCOL@Message{%
        Interface disabled because `pdftex.def'\MessageBreak
        is not loaded or it is too old.\MessageBreak
        Required is version 0.04b or greater%
      }%
    \fi
  \fi
\fi
%    \end{macrocode}
%
%    \begin{macrocode}
\let\pdfcolAvailabletrue\relax
\let\pdfcolAvailablefalse\relax
%    \end{macrocode}
%
% \subsection{Enabled interface macros}
%
%    \begin{macrocode}
\ifpdfcolAvailable
%    \end{macrocode}
%
%    \begin{macro}{\pdfcolErrorNoStacks}
%    \begin{macrocode}
  \let\pdfcolErrorNoStacks\relax
%    \end{macrocode}
%    \end{macro}
%
%    \begin{macro}{\pdfcol@Value}
%    \begin{macrocode}
  \expandafter\ifx\csname pdfcol@Value\endcsname\relax
    \def\pdfcol@Value{0 g 0 G}%
  \fi
%    \end{macrocode}
%    \end{macro}
%
%    \begin{macro}{\pdfcol@LiteralModifier}
%    \begin{macrocode}
  \expandafter\ifx\csname pdfcol@LiteralModifier\endcsname\relax
    \def\pdfcol@LiteralModifier{direct}%
  \fi
%    \end{macrocode}
%    \end{macro}
%
%    \begin{macro}{\pdfcolInitStack}
%    \begin{macrocode}
  \def\pdfcolInitStack#1{%
    \expandafter\ifx\csname pdfcol@Stack@#1\endcsname\relax
      \global\expandafter\chardef\csname pdfcol@Stack@#1\endcsname=%
          \pdfcolorstackinit\pdfcol@LiteralModifier{\pdfcol@Value}%
          \relax
      \@PackageInfo{pdfcol}{%
        New color stack `#1' = \number\csname pdfcol@Stack@#1\endcsname
      }%
    \else
      \@PackageError{pdfcol}{%
        Stack `#1' is already defined%
      }\@ehc
    \fi
  }%
%    \end{macrocode}
%    \end{macro}
%
%    \begin{macro}{\pdfcolIfStackExists}
%    \begin{macrocode}
  \def\pdfcolIfStackExists#1{%
    \expandafter\ifx\csname pdfcol@Stack@#1\endcsname\relax
      \expandafter\@secondoftwo
    \else
      \expandafter\@firstoftwo
    \fi
  }%
%    \end{macrocode}
%    \end{macro}
%    \begin{macro}{\@firstoftwo}
%    \begin{macrocode}
  \expandafter\ifx\csname @firstoftwo\endcsname\relax
    \long\def\@firstoftwo#1#2{#1}%
  \fi
%    \end{macrocode}
%    \end{macro}
%    \begin{macro}{\@secondoftwo}
%    \begin{macrocode}
  \expandafter\ifx\csname @secondoftwo\endcsname\relax
    \long\def\@secondoftwo#1#2{#2}%
  \fi
%    \end{macrocode}
%    \end{macro}
%
%    \begin{macro}{\pdfcolSwitchStack}
%    \begin{macrocode}
  \def\pdfcolSwitchStack#1{%
    \pdfcolIfStackExists{#1}{%
      \expandafter\let\expandafter\@pdfcolorstack
                      \csname pdfcol@Stack@#1\endcsname
    }{%
      \pdfcol@ErrorNoStack{#1}%
    }%
  }%
%    \end{macrocode}
%    \end{macro}
%
%    \begin{macro}{\pdfcolSetCurrentColor}
%    \begin{macrocode}
  \def\pdfcolSetCurrentColor{%
    \pdfcolorstack\@pdfcolorstack set{\current@color}%
  }%
%    \end{macrocode}
%    \end{macro}
%
%    \begin{macro}{\pdfcolSetCurrent}
%    \begin{macrocode}
  \def\pdfcolSetCurrent#1{%
    \ifx\\#1\\%
      \pdfcolorstack\@pdfcolorstack current\relax
    \else
      \pdfcolIfStackExists{#1}{%
        \pdfcolorstack\csname pdfcol@Stack@#1\endcsname current\relax
      }{%
        \pdfcol@ErrorNoStack{#1}%
      }%
    \fi
  }%
%    \end{macrocode}
%    \end{macro}
%
%    \begin{macro}{\pdfcol@ErrorNoStack}
%    \begin{macrocode}
  \def\pdfcol@ErrorNoStack#1{%
    \@PackageError{pdfcol}{Stack `#1' does not exists}\@ehc
  }%
%    \end{macrocode}
%    \end{macro}
%
% \subsection{Disabled interface macros}
%
%    \begin{macrocode}
\else
%    \end{macrocode}
%
%    \begin{macro}{\pdfcolErrorNoStacks}
%    \begin{macrocode}
  \def\pdfcolErrorNoStacks{%
    \@PackageError{pdfcol}{%
      Color stacks are not available%
    }{%
      Update pdfTeX (1.40) and `pdftex.def' (0.04b) %
          if necessary.\MessageBreak
      Ensure that `pdftex.def' is loaded %
          (package `color' or `xcolor').\MessageBreak
      Further messages can be found in TeX's %
          protocol file `\jobname.log'.\MessageBreak
      \MessageBreak
      \@ehc
    }%
    \global\let\pdfcolErrorNoStacks\relax
  }%
%    \end{macrocode}
%    \end{macro}
%
%    \begin{macro}{\PDFCOL@Disabled}
%    \begin{macrocode}
  \def\PDFCOL@Disabled{%
    \PDFCOL@Message{%
      pdfTeX's color stacks are not available%
    }%
    \global\let\PDFCOL@Disabled\relax
  }%
%    \end{macrocode}
%    \end{macro}
%
%    \begin{macro}{\pdfcolInitStack}
%    \begin{macrocode}
  \def\pdfcolInitStack#1{%
    \PDFCOL@Disabled
  }%
%    \end{macrocode}
%    \end{macro}
%
%    \begin{macro}{\pdfcolIfStackExists}
%    \begin{macrocode}
  \long\def\pdfcolIfStackExists#1#2#3{#3}%
%    \end{macrocode}
%    \end{macro}
%
%    \begin{macro}{\pdfcolSwitchStack}
%    \begin{macrocode}
  \def\pdfcolSwitchStack#1{%
    \PDFCOL@Disabled
  }%
%    \end{macrocode}
%    \end{macro}
%
%    \begin{macro}{\pdfcolSetCurrentColor}
%    \begin{macrocode}
  \def\pdfcolSetCurrentColor{%
    \PDFCOL@Disabled
  }%
%    \end{macrocode}
%    \end{macro}
%
%    \begin{macro}{\pdfcolSetCurrent}
%    \begin{macrocode}
  \def\pdfcolSetCurrent#1{%
    \PDFCOL@Disabled
  }%
%    \end{macrocode}
%    \end{macro}
%    \begin{macrocode}
\fi
%    \end{macrocode}
%
%    \begin{macrocode}
\PDFCOL@AtEnd%
%</package>
%    \end{macrocode}
%% \section{Installation}
%
% \subsection{Download}
%
% \paragraph{Package.} This package is available on
% CTAN\footnote{\CTANpkg{pdfcol}}:
% \begin{description}
% \item[\CTAN{macros/latex/contrib/oberdiek/pdfcol.dtx}] The source file.
% \item[\CTAN{macros/latex/contrib/oberdiek/pdfcol.pdf}] Documentation.
% \end{description}
%
%
% \paragraph{Bundle.} All the packages of the bundle `oberdiek'
% are also available in a TDS compliant ZIP archive. There
% the packages are already unpacked and the documentation files
% are generated. The files and directories obey the TDS standard.
% \begin{description}
% \item[\CTANinstall{install/macros/latex/contrib/oberdiek.tds.zip}]
% \end{description}
% \emph{TDS} refers to the standard ``A Directory Structure
% for \TeX\ Files'' (\CTANpkg{tds}). Directories
% with \xfile{texmf} in their name are usually organized this way.
%
% \subsection{Bundle installation}
%
% \paragraph{Unpacking.} Unpack the \xfile{oberdiek.tds.zip} in the
% TDS tree (also known as \xfile{texmf} tree) of your choice.
% Example (linux):
% \begin{quote}
%   |unzip oberdiek.tds.zip -d ~/texmf|
% \end{quote}
%
% \subsection{Package installation}
%
% \paragraph{Unpacking.} The \xfile{.dtx} file is a self-extracting
% \docstrip\ archive. The files are extracted by running the
% \xfile{.dtx} through \plainTeX:
% \begin{quote}
%   \verb|tex pdfcol.dtx|
% \end{quote}
%
% \paragraph{TDS.} Now the different files must be moved into
% the different directories in your installation TDS tree
% (also known as \xfile{texmf} tree):
% \begin{quote}
% \def\t{^^A
% \begin{tabular}{@{}>{\ttfamily}l@{ $\rightarrow$ }>{\ttfamily}l@{}}
%   pdfcol.sty & tex/generic/oberdiek/pdfcol.sty\\
%   pdfcol.pdf & doc/latex/oberdiek/pdfcol.pdf\\
%   pdfcol.dtx & source/latex/oberdiek/pdfcol.dtx\\
% \end{tabular}^^A
% }^^A
% \sbox0{\t}^^A
% \ifdim\wd0>\linewidth
%   \begingroup
%     \advance\linewidth by\leftmargin
%     \advance\linewidth by\rightmargin
%   \edef\x{\endgroup
%     \def\noexpand\lw{\the\linewidth}^^A
%   }\x
%   \def\lwbox{^^A
%     \leavevmode
%     \hbox to \linewidth{^^A
%       \kern-\leftmargin\relax
%       \hss
%       \usebox0
%       \hss
%       \kern-\rightmargin\relax
%     }^^A
%   }^^A
%   \ifdim\wd0>\lw
%     \sbox0{\small\t}^^A
%     \ifdim\wd0>\linewidth
%       \ifdim\wd0>\lw
%         \sbox0{\footnotesize\t}^^A
%         \ifdim\wd0>\linewidth
%           \ifdim\wd0>\lw
%             \sbox0{\scriptsize\t}^^A
%             \ifdim\wd0>\linewidth
%               \ifdim\wd0>\lw
%                 \sbox0{\tiny\t}^^A
%                 \ifdim\wd0>\linewidth
%                   \lwbox
%                 \else
%                   \usebox0
%                 \fi
%               \else
%                 \lwbox
%               \fi
%             \else
%               \usebox0
%             \fi
%           \else
%             \lwbox
%           \fi
%         \else
%           \usebox0
%         \fi
%       \else
%         \lwbox
%       \fi
%     \else
%       \usebox0
%     \fi
%   \else
%     \lwbox
%   \fi
% \else
%   \usebox0
% \fi
% \end{quote}
% If you have a \xfile{docstrip.cfg} that configures and enables \docstrip's
% TDS installing feature, then some files can already be in the right
% place, see the documentation of \docstrip.
%
% \subsection{Refresh file name databases}
%
% If your \TeX~distribution
% (\TeX\,Live, \mikTeX, \dots) relies on file name databases, you must refresh
% these. For example, \TeX\,Live\ users run \verb|texhash| or
% \verb|mktexlsr|.
%
% \subsection{Some details for the interested}
%
% \paragraph{Unpacking with \LaTeX.}
% The \xfile{.dtx} chooses its action depending on the format:
% \begin{description}
% \item[\plainTeX:] Run \docstrip\ and extract the files.
% \item[\LaTeX:] Generate the documentation.
% \end{description}
% If you insist on using \LaTeX\ for \docstrip\ (really,
% \docstrip\ does not need \LaTeX), then inform the autodetect routine
% about your intention:
% \begin{quote}
%   \verb|latex \let\install=y\input{pdfcol.dtx}|
% \end{quote}
% Do not forget to quote the argument according to the demands
% of your shell.
%
% \paragraph{Generating the documentation.}
% You can use both the \xfile{.dtx} or the \xfile{.drv} to generate
% the documentation. The process can be configured by the
% configuration file \xfile{ltxdoc.cfg}. For instance, put this
% line into this file, if you want to have A4 as paper format:
% \begin{quote}
%   \verb|\PassOptionsToClass{a4paper}{article}|
% \end{quote}
% An example follows how to generate the
% documentation with pdf\LaTeX:
% \begin{quote}
%\begin{verbatim}
%pdflatex pdfcol.dtx
%makeindex -s gind.ist pdfcol.idx
%pdflatex pdfcol.dtx
%makeindex -s gind.ist pdfcol.idx
%pdflatex pdfcol.dtx
%\end{verbatim}
% \end{quote}
%
% \begin{History}
%   \begin{Version}{2007/09/09 v1.0}
%   \item
%     First version.
%   \end{Version}
%   \begin{Version}{2007/12/09 v1.1}
%   \item
%     \cs{pdfcolSetCurrentColor} added.
%   \end{Version}
%   \begin{Version}{2007/12/12 v1.2}
%   \item
%     Detection for package \xpackage{luacolor} added.
%   \end{Version}
%   \begin{Version}{2016/05/16 v1.3}
%   \item
%     Documentation updates.
%   \end{Version}
%   \begin{Version}{2016/05/17 v1.4}
%   \item
%     Use luatex85 package for new luatex compatibility
%   \end{Version}
%   \begin{Version}{2018/11/01 v1.5}
%   \item
%     Remove luatex85 dependency
%   \end{Version}
% \end{History}
%
% \PrintIndex
%
% \Finale
\endinput

%        (quote the arguments according to the demands of your shell)
%
% Documentation:
%    (a) If pdfcol.drv is present:
%           latex pdfcol.drv
%    (b) Without pdfcol.drv:
%           latex pdfcol.dtx; ...
%    The class ltxdoc loads the configuration file ltxdoc.cfg
%    if available. Here you can specify further options, e.g.
%    use A4 as paper format:
%       \PassOptionsToClass{a4paper}{article}
%
%    Programm calls to get the documentation (example):
%       pdflatex pdfcol.dtx
%       makeindex -s gind.ist pdfcol.idx
%       pdflatex pdfcol.dtx
%       makeindex -s gind.ist pdfcol.idx
%       pdflatex pdfcol.dtx
%
% Installation:
%    TDS:tex/generic/oberdiek/pdfcol.sty
%    TDS:doc/latex/oberdiek/pdfcol.pdf
%    TDS:source/latex/oberdiek/pdfcol.dtx
%
%<*ignore>
\begingroup
  \catcode123=1 %
  \catcode125=2 %
  \def\x{LaTeX2e}%
\expandafter\endgroup
\ifcase 0\ifx\install y1\fi\expandafter
         \ifx\csname processbatchFile\endcsname\relax\else1\fi
         \ifx\fmtname\x\else 1\fi\relax
\else\csname fi\endcsname
%</ignore>
%<*install>
\input docstrip.tex
\Msg{************************************************************************}
\Msg{* Installation}
\Msg{* Package: pdfcol 2018/11/01 v1.5 Handle new color stacks for pdfTeX (HO)}
\Msg{************************************************************************}

\keepsilent
\askforoverwritefalse

\let\MetaPrefix\relax
\preamble

This is a generated file.

Project: pdfcol
Version: 2018/11/01 v1.5

Copyright (C)
   2007 Heiko Oberdiek
   2016-2019 Oberdiek Package Support Group

This work may be distributed and/or modified under the
conditions of the LaTeX Project Public License, either
version 1.3c of this license or (at your option) any later
version. This version of this license is in
   https://www.latex-project.org/lppl/lppl-1-3c.txt
and the latest version of this license is in
   https://www.latex-project.org/lppl.txt
and version 1.3 or later is part of all distributions of
LaTeX version 2005/12/01 or later.

This work has the LPPL maintenance status "maintained".

The Current Maintainers of this work are
Heiko Oberdiek and the Oberdiek Package Support Group
https://github.com/ho-tex/oberdiek/issues


The Base Interpreter refers to any `TeX-Format',
because some files are installed in TDS:tex/generic//.

This work consists of the main source file pdfcol.dtx
and the derived files
   pdfcol.sty, pdfcol.pdf, pdfcol.ins, pdfcol.drv, pdfcol-test1.tex,
   pdfcol-test2.tex, pdfcol-test3.tex, pdfcol-test4.tex.

\endpreamble
\let\MetaPrefix\DoubleperCent

\generate{%
  \file{pdfcol.ins}{\from{pdfcol.dtx}{install}}%
  \file{pdfcol.drv}{\from{pdfcol.dtx}{driver}}%
  \usedir{tex/generic/oberdiek}%
  \file{pdfcol.sty}{\from{pdfcol.dtx}{package}}%
%  \usedir{doc/latex/oberdiek/test}%
%  \file{pdfcol-test1.tex}{\from{pdfcol.dtx}{test1}}%
%  \file{pdfcol-test2.tex}{\from{pdfcol.dtx}{test2}}%
%  \file{pdfcol-test3.tex}{\from{pdfcol.dtx}{test3}}%
%  \file{pdfcol-test4.tex}{\from{pdfcol.dtx}{test4}}%
}

\catcode32=13\relax% active space
\let =\space%
\Msg{************************************************************************}
\Msg{*}
\Msg{* To finish the installation you have to move the following}
\Msg{* file into a directory searched by TeX:}
\Msg{*}
\Msg{*     pdfcol.sty}
\Msg{*}
\Msg{* To produce the documentation run the file `pdfcol.drv'}
\Msg{* through LaTeX.}
\Msg{*}
\Msg{* Happy TeXing!}
\Msg{*}
\Msg{************************************************************************}

\endbatchfile
%</install>
%<*ignore>
\fi
%</ignore>
%<*driver>
\NeedsTeXFormat{LaTeX2e}
\ProvidesFile{pdfcol.drv}%
  [2018/11/01 v1.5 Handle new color stacks for pdfTeX (HO)]%
\documentclass{ltxdoc}
\usepackage{holtxdoc}[2011/11/22]
\begin{document}
  \DocInput{pdfcol.dtx}%
\end{document}
%</driver>
% \fi
%
%
%
% \GetFileInfo{pdfcol.drv}
%
% \title{The \xpackage{pdfcol} package}
% \date{2018/11/01 v1.5}
% \author{Heiko Oberdiek\thanks
% {Please report any issues at \url{https://github.com/ho-tex/oberdiek/issues}}}
%
% \maketitle
%
% \begin{abstract}
% Since version 1.40 \pdfTeX\ supports color stacks.
% The driver file \xfile{pdftex.def} for package \xpackage{color}
% defines and uses a main color stack since version v0.04b.
% Package \xpackage{pdfcol} is intended for package writers.
% It defines macros for setting and maintaining new color stacks.
% \end{abstract}
%
% \tableofcontents
%
% \section{Documentation}
%
% Version 1.40 of \pdfTeX\ adds new primitives \cs{pdfcolorstackinit}
% and \cs{pdfcolorstack}. Now color stacks can be defined and used.
% A main color stack is maintained by the driver file \xfile{pdftex.def}
% similar to dvips or dvipdfm. However the number of color stacks
% is not limited to one in \pdfTeX. Thus further color problems
% can now be solved, such as footnotes across pages or text
% that is set in parallel columns (e.g. packages \xpackage{parallel}
% or \xpackage{parcolumn}). Unlike the main color stack,
% the support by additional color stacks cannot be done in
% a transparent manner.
%
% This package \xpackage{pdfcol} provides an easier interface to
% additional color stacks without the need to use the
% low level primitives.
%
% \subsection{Requirements}
% \label{sec:req}
%
% \begin{itemize}
% \item
%   \pdfTeX\ 1.40 or greater.
% \item
%   \pdfTeX in PDF mode. (I don't know a DVI driver that
%   support several color stacks.)
% \item
%   \xfile{pdftex.def} 2007/01/02 v0.04b.
% \end{itemize}
% Package \xpackage{pdfcol} checks the requirements and
% sets switch \cs{ifpdfcolAvailable} accordingly.
%
% \subsection{Interface}
%
% \begin{declcs}{ifpdfcolAvailable}
% \end{declcs}
% If the requirements of section \ref{sec:req} are met the
% switch \cs{ifpdfcolAvailable} behaves as \cs{iftrue}.
% Otherwise the other interface macros in this section will
% be disabled with a message. Also the first use of such a
% macro will print a message. The messages are print to
% the \xext{log} file only if \pdfTeX\ is not used in PDF mode.
%
% \begin{declcs}{pdfcolErrorNoStacks}
% \end{declcs}
% The first call of \cs{pdfcolErrorNoStacks} prints an error
% message, if color stacks are not available.
%
% \begin{declcs}{pdfcolInitStack} \M{name}
% \end{declcs}
% A new color stack is initialized by \cs{pdfcolInitStack}.
% The \meta{name} is used for indentifying the stack. It usually
% consists of letters and digits. (The name must survive a \cs{csname}.)
%
% The intension of the macro is the definition of an additional
% color stack. Thus the stack is not page bounded like the
% main color stack. Black (\texttt{0 g 0 G}) is used as initial
% color value. And colors are written with modifier \texttt{direct}
% that means without setting the current transfer matrix and changing
% the current point (see documentation of \pdfTeX\ for
% |\pdfliteral direct{...}|).
%
% \begin{declcs}{pdfcolIfStackExists} \M{name} \M{then} \M{else}
% \end{declcs}
% Macro \cs{pdfcolIfStackExists} checks whether color stack \meta{name}
% exists. In case of success argument \meta{then} is executed
% and \meta{else} otherwise.
%
% \begin{declcs}{pdfcolSwitchStack} \M{name}
% \end{declcs}
% Macro \cs{pdfcolSwitchStack} switches the color stack. The color macros
% of package \xpackage{color} (or \xpackage{xcolor}) now uses the
% new color stack with name \meta{name}.
%
% \begin{declcs}{pdfcolSetCurrentColor}
% \end{declcs}
% Macro \cs{pdfcolSetCurrentColor} replaces the topmost
% entry of the stack by the current color (\cs{current@color}).
%
% \begin{declcs}{pdfcolSetCurrent} \M{name}
% \end{declcs}
% Macro \cs{pdfcolSetCurrent} sets the color that is read in
% the top-most entry of color stack \meta{name}. If \meta{name}
% is empty, the default color stack is used.
%
% \StopEventually{
% }
%
% \section{Implementation}
%
%    \begin{macrocode}
%<*package>
%    \end{macrocode}
%
% \subsection{Reload check and package identification}
%    Reload check, especially if the package is not used with \LaTeX.
%    \begin{macrocode}
\begingroup\catcode61\catcode48\catcode32=10\relax%
  \catcode13=5 % ^^M
  \endlinechar=13 %
  \catcode35=6 % #
  \catcode39=12 % '
  \catcode44=12 % ,
  \catcode45=12 % -
  \catcode46=12 % .
  \catcode58=12 % :
  \catcode64=11 % @
  \catcode123=1 % {
  \catcode125=2 % }
  \expandafter\let\expandafter\x\csname ver@pdfcol.sty\endcsname
  \ifx\x\relax % plain-TeX, first loading
  \else
    \def\empty{}%
    \ifx\x\empty % LaTeX, first loading,
      % variable is initialized, but \ProvidesPackage not yet seen
    \else
      \expandafter\ifx\csname PackageInfo\endcsname\relax
        \def\x#1#2{%
          \immediate\write-1{Package #1 Info: #2.}%
        }%
      \else
        \def\x#1#2{\PackageInfo{#1}{#2, stopped}}%
      \fi
      \x{pdfcol}{The package is already loaded}%
      \aftergroup\endinput
    \fi
  \fi
\endgroup%
%    \end{macrocode}
%    Package identification:
%    \begin{macrocode}
\begingroup\catcode61\catcode48\catcode32=10\relax%
  \catcode13=5 % ^^M
  \endlinechar=13 %
  \catcode35=6 % #
  \catcode39=12 % '
  \catcode40=12 % (
  \catcode41=12 % )
  \catcode44=12 % ,
  \catcode45=12 % -
  \catcode46=12 % .
  \catcode47=12 % /
  \catcode58=12 % :
  \catcode64=11 % @
  \catcode91=12 % [
  \catcode93=12 % ]
  \catcode123=1 % {
  \catcode125=2 % }
  \expandafter\ifx\csname ProvidesPackage\endcsname\relax
    \def\x#1#2#3[#4]{\endgroup
      \immediate\write-1{Package: #3 #4}%
      \xdef#1{#4}%
    }%
  \else
    \def\x#1#2[#3]{\endgroup
      #2[{#3}]%
      \ifx#1\@undefined
        \xdef#1{#3}%
      \fi
      \ifx#1\relax
        \xdef#1{#3}%
      \fi
    }%
  \fi
\expandafter\x\csname ver@pdfcol.sty\endcsname
\ProvidesPackage{pdfcol}%
  [2018/11/01 v1.5 Handle new color stacks for pdfTeX (HO)]%
%    \end{macrocode}
%
% \subsection{Catcodes}
%
%    \begin{macrocode}
\begingroup\catcode61\catcode48\catcode32=10\relax%
  \catcode13=5 % ^^M
  \endlinechar=13 %
  \catcode123=1 % {
  \catcode125=2 % }
  \catcode64=11 % @
  \def\x{\endgroup
    \expandafter\edef\csname PDFCOL@AtEnd\endcsname{%
      \endlinechar=\the\endlinechar\relax
      \catcode13=\the\catcode13\relax
      \catcode32=\the\catcode32\relax
      \catcode35=\the\catcode35\relax
      \catcode61=\the\catcode61\relax
      \catcode64=\the\catcode64\relax
      \catcode123=\the\catcode123\relax
      \catcode125=\the\catcode125\relax
    }%
  }%
\x\catcode61\catcode48\catcode32=10\relax%
\catcode13=5 % ^^M
\endlinechar=13 %
\catcode35=6 % #
\catcode64=11 % @
\catcode123=1 % {
\catcode125=2 % }
\def\TMP@EnsureCode#1#2{%
  \edef\PDFCOL@AtEnd{%
    \PDFCOL@AtEnd
    \catcode#1=\the\catcode#1\relax
  }%
  \catcode#1=#2\relax
}
\TMP@EnsureCode{39}{12}% '
\TMP@EnsureCode{40}{12}% (
\TMP@EnsureCode{41}{12}% )
\TMP@EnsureCode{43}{12}% +
\TMP@EnsureCode{44}{12}% ,
\TMP@EnsureCode{46}{12}% .
\TMP@EnsureCode{47}{12}% /
\TMP@EnsureCode{91}{12}% [
\TMP@EnsureCode{93}{12}% ]
\TMP@EnsureCode{96}{12}% `
\edef\PDFCOL@AtEnd{\PDFCOL@AtEnd\noexpand\endinput}
%    \end{macrocode}
%
% \subsection{Check requirements}
%
%    \begin{macro}{\PDFCOL@RequirePackage}
%    \begin{macrocode}
\begingroup\expandafter\expandafter\expandafter\endgroup
\expandafter\ifx\csname RequirePackage\endcsname\relax
  \def\PDFCOL@RequirePackage#1[#2]{\input #1.sty\relax}%
\else
  \def\PDFCOL@RequirePackage#1[#2]{%
    \RequirePackage{#1}[{#2}]%
  }%
\fi
%    \end{macrocode}
%    \end{macro}
%
% LuaTeX Compatability
%    \begin{macrocode}
\ifx\pdfextension\@undefined\else
 \def\pdfcolorstackinit {\pdffeedback colorstackinit}
 \protected\def\pdfcolorstack     {\pdfextension colorstack}
\fi
%    \end{macrocode}
%
%    \begin{macrocode}
\PDFCOL@RequirePackage{ltxcmds}[2010/03/01]
%    \end{macrocode}
%
%    \begin{macro}{ifpdfcolAvailable}
%    \begin{macrocode}
\ltx@newif\ifpdfcolAvailable
\pdfcolAvailabletrue
%    \end{macrocode}
%    \end{macro}
%
% \subsubsection{Check package \xpackage{luacolor}}
%
%    \begin{macrocode}
\ltx@newif\ifPDFCOL@luacolor
\begingroup\expandafter\expandafter\expandafter\endgroup
\expandafter\ifx\csname ver@luacolor.sty\endcsname\relax
  \PDFCOL@luacolorfalse
\else
  \PDFCOL@luacolortrue
\fi
%    \end{macrocode}
%
% \subsubsection{Check PDF mode}
%
%    \begin{macrocode}
\PDFCOL@RequirePackage{infwarerr}[2007/09/09]
\PDFCOL@RequirePackage{ifpdf}[2007/09/09]
\ifcase\ifpdf\ifPDFCOL@luacolor 1\fi\else 1\fi0 %
  \def\PDFCOL@Message{%
    \@PackageWarningNoLine{pdfcol}%
  }%
\else
  \pdfcolAvailablefalse
  \def\PDFCOL@Message{%
    \@PackageInfoNoLine{pdfcol}%
  }%
  \PDFCOL@Message{%
    Interface disabled because of %
    \ifPDFCOL@luacolor
      package `luacolor'%
    \else
      missing PDF mode of pdfTeX%
    \fi
  }%
\fi
%    \end{macrocode}
%
% \subsubsection{Check version of \pdfTeX}
%
%    \begin{macrocode}
\ifpdfcolAvailable
  \begingroup\expandafter\expandafter\expandafter\endgroup
  \expandafter\ifx\csname pdfcolorstack\endcsname\relax
    \pdfcolAvailablefalse
    \PDFCOL@Message{%
      Interface disabled because of too old pdfTeX.\MessageBreak
      Required is version 1.40+ for \string\pdfcolorstack
    }%
  \fi
\fi
\ifpdfcolAvailable
  \begingroup\expandafter\expandafter\expandafter\endgroup
  \expandafter\ifx\csname pdfcolorstack\endcsname\relax
    \pdfcolAvailablefalse
    \PDFCOL@Message{%
      Interface disabled because of too old pdfTeX.\MessageBreak
      Required is version 1.40+ for \string\pdfcolorstackinit
    }%
  \fi
\fi
%    \end{macrocode}
%
% \subsubsection{Check \xfile{pdftex.def}}
%
%    \begin{macrocode}
\ifpdfcolAvailable
  \begingroup\expandafter\expandafter\expandafter\endgroup
  \expandafter\ifx\csname @pdfcolorstack\endcsname\relax
%    \end{macrocode}
%    Try to load package color if it is not yet loaded (\LaTeX\ case).
%    \begin{macrocode}
    \begingroup\expandafter\expandafter\expandafter\endgroup
    \expandafter\ifx\csname ver@color.sty\endcsname\relax
      \begingroup\expandafter\expandafter\expandafter\endgroup
      \expandafter\ifx\csname documentclass\endcsname\relax
      \else
        \RequirePackage[pdftex]{color}\relax
      \fi
    \fi
    \begingroup\expandafter\expandafter\expandafter\endgroup
    \expandafter\ifx\csname @pdfcolorstack\endcsname\relax
      \pdfcolAvailablefalse
      \PDFCOL@Message{%
        Interface disabled because `pdftex.def'\MessageBreak
        is not loaded or it is too old.\MessageBreak
        Required is version 0.04b or greater%
      }%
    \fi
  \fi
\fi
%    \end{macrocode}
%
%    \begin{macrocode}
\let\pdfcolAvailabletrue\relax
\let\pdfcolAvailablefalse\relax
%    \end{macrocode}
%
% \subsection{Enabled interface macros}
%
%    \begin{macrocode}
\ifpdfcolAvailable
%    \end{macrocode}
%
%    \begin{macro}{\pdfcolErrorNoStacks}
%    \begin{macrocode}
  \let\pdfcolErrorNoStacks\relax
%    \end{macrocode}
%    \end{macro}
%
%    \begin{macro}{\pdfcol@Value}
%    \begin{macrocode}
  \expandafter\ifx\csname pdfcol@Value\endcsname\relax
    \def\pdfcol@Value{0 g 0 G}%
  \fi
%    \end{macrocode}
%    \end{macro}
%
%    \begin{macro}{\pdfcol@LiteralModifier}
%    \begin{macrocode}
  \expandafter\ifx\csname pdfcol@LiteralModifier\endcsname\relax
    \def\pdfcol@LiteralModifier{direct}%
  \fi
%    \end{macrocode}
%    \end{macro}
%
%    \begin{macro}{\pdfcolInitStack}
%    \begin{macrocode}
  \def\pdfcolInitStack#1{%
    \expandafter\ifx\csname pdfcol@Stack@#1\endcsname\relax
      \global\expandafter\chardef\csname pdfcol@Stack@#1\endcsname=%
          \pdfcolorstackinit\pdfcol@LiteralModifier{\pdfcol@Value}%
          \relax
      \@PackageInfo{pdfcol}{%
        New color stack `#1' = \number\csname pdfcol@Stack@#1\endcsname
      }%
    \else
      \@PackageError{pdfcol}{%
        Stack `#1' is already defined%
      }\@ehc
    \fi
  }%
%    \end{macrocode}
%    \end{macro}
%
%    \begin{macro}{\pdfcolIfStackExists}
%    \begin{macrocode}
  \def\pdfcolIfStackExists#1{%
    \expandafter\ifx\csname pdfcol@Stack@#1\endcsname\relax
      \expandafter\@secondoftwo
    \else
      \expandafter\@firstoftwo
    \fi
  }%
%    \end{macrocode}
%    \end{macro}
%    \begin{macro}{\@firstoftwo}
%    \begin{macrocode}
  \expandafter\ifx\csname @firstoftwo\endcsname\relax
    \long\def\@firstoftwo#1#2{#1}%
  \fi
%    \end{macrocode}
%    \end{macro}
%    \begin{macro}{\@secondoftwo}
%    \begin{macrocode}
  \expandafter\ifx\csname @secondoftwo\endcsname\relax
    \long\def\@secondoftwo#1#2{#2}%
  \fi
%    \end{macrocode}
%    \end{macro}
%
%    \begin{macro}{\pdfcolSwitchStack}
%    \begin{macrocode}
  \def\pdfcolSwitchStack#1{%
    \pdfcolIfStackExists{#1}{%
      \expandafter\let\expandafter\@pdfcolorstack
                      \csname pdfcol@Stack@#1\endcsname
    }{%
      \pdfcol@ErrorNoStack{#1}%
    }%
  }%
%    \end{macrocode}
%    \end{macro}
%
%    \begin{macro}{\pdfcolSetCurrentColor}
%    \begin{macrocode}
  \def\pdfcolSetCurrentColor{%
    \pdfcolorstack\@pdfcolorstack set{\current@color}%
  }%
%    \end{macrocode}
%    \end{macro}
%
%    \begin{macro}{\pdfcolSetCurrent}
%    \begin{macrocode}
  \def\pdfcolSetCurrent#1{%
    \ifx\\#1\\%
      \pdfcolorstack\@pdfcolorstack current\relax
    \else
      \pdfcolIfStackExists{#1}{%
        \pdfcolorstack\csname pdfcol@Stack@#1\endcsname current\relax
      }{%
        \pdfcol@ErrorNoStack{#1}%
      }%
    \fi
  }%
%    \end{macrocode}
%    \end{macro}
%
%    \begin{macro}{\pdfcol@ErrorNoStack}
%    \begin{macrocode}
  \def\pdfcol@ErrorNoStack#1{%
    \@PackageError{pdfcol}{Stack `#1' does not exists}\@ehc
  }%
%    \end{macrocode}
%    \end{macro}
%
% \subsection{Disabled interface macros}
%
%    \begin{macrocode}
\else
%    \end{macrocode}
%
%    \begin{macro}{\pdfcolErrorNoStacks}
%    \begin{macrocode}
  \def\pdfcolErrorNoStacks{%
    \@PackageError{pdfcol}{%
      Color stacks are not available%
    }{%
      Update pdfTeX (1.40) and `pdftex.def' (0.04b) %
          if necessary.\MessageBreak
      Ensure that `pdftex.def' is loaded %
          (package `color' or `xcolor').\MessageBreak
      Further messages can be found in TeX's %
          protocol file `\jobname.log'.\MessageBreak
      \MessageBreak
      \@ehc
    }%
    \global\let\pdfcolErrorNoStacks\relax
  }%
%    \end{macrocode}
%    \end{macro}
%
%    \begin{macro}{\PDFCOL@Disabled}
%    \begin{macrocode}
  \def\PDFCOL@Disabled{%
    \PDFCOL@Message{%
      pdfTeX's color stacks are not available%
    }%
    \global\let\PDFCOL@Disabled\relax
  }%
%    \end{macrocode}
%    \end{macro}
%
%    \begin{macro}{\pdfcolInitStack}
%    \begin{macrocode}
  \def\pdfcolInitStack#1{%
    \PDFCOL@Disabled
  }%
%    \end{macrocode}
%    \end{macro}
%
%    \begin{macro}{\pdfcolIfStackExists}
%    \begin{macrocode}
  \long\def\pdfcolIfStackExists#1#2#3{#3}%
%    \end{macrocode}
%    \end{macro}
%
%    \begin{macro}{\pdfcolSwitchStack}
%    \begin{macrocode}
  \def\pdfcolSwitchStack#1{%
    \PDFCOL@Disabled
  }%
%    \end{macrocode}
%    \end{macro}
%
%    \begin{macro}{\pdfcolSetCurrentColor}
%    \begin{macrocode}
  \def\pdfcolSetCurrentColor{%
    \PDFCOL@Disabled
  }%
%    \end{macrocode}
%    \end{macro}
%
%    \begin{macro}{\pdfcolSetCurrent}
%    \begin{macrocode}
  \def\pdfcolSetCurrent#1{%
    \PDFCOL@Disabled
  }%
%    \end{macrocode}
%    \end{macro}
%    \begin{macrocode}
\fi
%    \end{macrocode}
%
%    \begin{macrocode}
\PDFCOL@AtEnd%
%</package>
%    \end{macrocode}
%% \section{Installation}
%
% \subsection{Download}
%
% \paragraph{Package.} This package is available on
% CTAN\footnote{\CTANpkg{pdfcol}}:
% \begin{description}
% \item[\CTAN{macros/latex/contrib/oberdiek/pdfcol.dtx}] The source file.
% \item[\CTAN{macros/latex/contrib/oberdiek/pdfcol.pdf}] Documentation.
% \end{description}
%
%
% \paragraph{Bundle.} All the packages of the bundle `oberdiek'
% are also available in a TDS compliant ZIP archive. There
% the packages are already unpacked and the documentation files
% are generated. The files and directories obey the TDS standard.
% \begin{description}
% \item[\CTANinstall{install/macros/latex/contrib/oberdiek.tds.zip}]
% \end{description}
% \emph{TDS} refers to the standard ``A Directory Structure
% for \TeX\ Files'' (\CTANpkg{tds}). Directories
% with \xfile{texmf} in their name are usually organized this way.
%
% \subsection{Bundle installation}
%
% \paragraph{Unpacking.} Unpack the \xfile{oberdiek.tds.zip} in the
% TDS tree (also known as \xfile{texmf} tree) of your choice.
% Example (linux):
% \begin{quote}
%   |unzip oberdiek.tds.zip -d ~/texmf|
% \end{quote}
%
% \subsection{Package installation}
%
% \paragraph{Unpacking.} The \xfile{.dtx} file is a self-extracting
% \docstrip\ archive. The files are extracted by running the
% \xfile{.dtx} through \plainTeX:
% \begin{quote}
%   \verb|tex pdfcol.dtx|
% \end{quote}
%
% \paragraph{TDS.} Now the different files must be moved into
% the different directories in your installation TDS tree
% (also known as \xfile{texmf} tree):
% \begin{quote}
% \def\t{^^A
% \begin{tabular}{@{}>{\ttfamily}l@{ $\rightarrow$ }>{\ttfamily}l@{}}
%   pdfcol.sty & tex/generic/oberdiek/pdfcol.sty\\
%   pdfcol.pdf & doc/latex/oberdiek/pdfcol.pdf\\
%   pdfcol.dtx & source/latex/oberdiek/pdfcol.dtx\\
% \end{tabular}^^A
% }^^A
% \sbox0{\t}^^A
% \ifdim\wd0>\linewidth
%   \begingroup
%     \advance\linewidth by\leftmargin
%     \advance\linewidth by\rightmargin
%   \edef\x{\endgroup
%     \def\noexpand\lw{\the\linewidth}^^A
%   }\x
%   \def\lwbox{^^A
%     \leavevmode
%     \hbox to \linewidth{^^A
%       \kern-\leftmargin\relax
%       \hss
%       \usebox0
%       \hss
%       \kern-\rightmargin\relax
%     }^^A
%   }^^A
%   \ifdim\wd0>\lw
%     \sbox0{\small\t}^^A
%     \ifdim\wd0>\linewidth
%       \ifdim\wd0>\lw
%         \sbox0{\footnotesize\t}^^A
%         \ifdim\wd0>\linewidth
%           \ifdim\wd0>\lw
%             \sbox0{\scriptsize\t}^^A
%             \ifdim\wd0>\linewidth
%               \ifdim\wd0>\lw
%                 \sbox0{\tiny\t}^^A
%                 \ifdim\wd0>\linewidth
%                   \lwbox
%                 \else
%                   \usebox0
%                 \fi
%               \else
%                 \lwbox
%               \fi
%             \else
%               \usebox0
%             \fi
%           \else
%             \lwbox
%           \fi
%         \else
%           \usebox0
%         \fi
%       \else
%         \lwbox
%       \fi
%     \else
%       \usebox0
%     \fi
%   \else
%     \lwbox
%   \fi
% \else
%   \usebox0
% \fi
% \end{quote}
% If you have a \xfile{docstrip.cfg} that configures and enables \docstrip's
% TDS installing feature, then some files can already be in the right
% place, see the documentation of \docstrip.
%
% \subsection{Refresh file name databases}
%
% If your \TeX~distribution
% (\TeX\,Live, \mikTeX, \dots) relies on file name databases, you must refresh
% these. For example, \TeX\,Live\ users run \verb|texhash| or
% \verb|mktexlsr|.
%
% \subsection{Some details for the interested}
%
% \paragraph{Unpacking with \LaTeX.}
% The \xfile{.dtx} chooses its action depending on the format:
% \begin{description}
% \item[\plainTeX:] Run \docstrip\ and extract the files.
% \item[\LaTeX:] Generate the documentation.
% \end{description}
% If you insist on using \LaTeX\ for \docstrip\ (really,
% \docstrip\ does not need \LaTeX), then inform the autodetect routine
% about your intention:
% \begin{quote}
%   \verb|latex \let\install=y% \iffalse meta-comment
%
% File: pdfcol.dtx
% Version: 2018/11/01 v1.5
% Info: Handle new color stacks for pdfTeX
%
% Copyright (C)
%    2007 Heiko Oberdiek
%    2016-2019 Oberdiek Package Support Group
%    https://github.com/ho-tex/oberdiek/issues
%
% This work may be distributed and/or modified under the
% conditions of the LaTeX Project Public License, either
% version 1.3c of this license or (at your option) any later
% version. This version of this license is in
%    https://www.latex-project.org/lppl/lppl-1-3c.txt
% and the latest version of this license is in
%    https://www.latex-project.org/lppl.txt
% and version 1.3 or later is part of all distributions of
% LaTeX version 2005/12/01 or later.
%
% This work has the LPPL maintenance status "maintained".
%
% The Current Maintainers of this work are
% Heiko Oberdiek and the Oberdiek Package Support Group
% https://github.com/ho-tex/oberdiek/issues
%
% The Base Interpreter refers to any `TeX-Format',
% because some files are installed in TDS:tex/generic//.
%
% This work consists of the main source file pdfcol.dtx
% and the derived files
%    pdfcol.sty, pdfcol.pdf, pdfcol.ins, pdfcol.drv, pdfcol-test1.tex,
%    pdfcol-test2.tex, pdfcol-test3.tex, pdfcol-test4.tex.
%
% Distribution:
%    CTAN:macros/latex/contrib/oberdiek/pdfcol.dtx
%    CTAN:macros/latex/contrib/oberdiek/pdfcol.pdf
%
% Unpacking:
%    (a) If pdfcol.ins is present:
%           tex pdfcol.ins
%    (b) Without pdfcol.ins:
%           tex pdfcol.dtx
%    (c) If you insist on using LaTeX
%           latex \let\install=y\input{pdfcol.dtx}
%        (quote the arguments according to the demands of your shell)
%
% Documentation:
%    (a) If pdfcol.drv is present:
%           latex pdfcol.drv
%    (b) Without pdfcol.drv:
%           latex pdfcol.dtx; ...
%    The class ltxdoc loads the configuration file ltxdoc.cfg
%    if available. Here you can specify further options, e.g.
%    use A4 as paper format:
%       \PassOptionsToClass{a4paper}{article}
%
%    Programm calls to get the documentation (example):
%       pdflatex pdfcol.dtx
%       makeindex -s gind.ist pdfcol.idx
%       pdflatex pdfcol.dtx
%       makeindex -s gind.ist pdfcol.idx
%       pdflatex pdfcol.dtx
%
% Installation:
%    TDS:tex/generic/oberdiek/pdfcol.sty
%    TDS:doc/latex/oberdiek/pdfcol.pdf
%    TDS:source/latex/oberdiek/pdfcol.dtx
%
%<*ignore>
\begingroup
  \catcode123=1 %
  \catcode125=2 %
  \def\x{LaTeX2e}%
\expandafter\endgroup
\ifcase 0\ifx\install y1\fi\expandafter
         \ifx\csname processbatchFile\endcsname\relax\else1\fi
         \ifx\fmtname\x\else 1\fi\relax
\else\csname fi\endcsname
%</ignore>
%<*install>
\input docstrip.tex
\Msg{************************************************************************}
\Msg{* Installation}
\Msg{* Package: pdfcol 2018/11/01 v1.5 Handle new color stacks for pdfTeX (HO)}
\Msg{************************************************************************}

\keepsilent
\askforoverwritefalse

\let\MetaPrefix\relax
\preamble

This is a generated file.

Project: pdfcol
Version: 2018/11/01 v1.5

Copyright (C)
   2007 Heiko Oberdiek
   2016-2019 Oberdiek Package Support Group

This work may be distributed and/or modified under the
conditions of the LaTeX Project Public License, either
version 1.3c of this license or (at your option) any later
version. This version of this license is in
   https://www.latex-project.org/lppl/lppl-1-3c.txt
and the latest version of this license is in
   https://www.latex-project.org/lppl.txt
and version 1.3 or later is part of all distributions of
LaTeX version 2005/12/01 or later.

This work has the LPPL maintenance status "maintained".

The Current Maintainers of this work are
Heiko Oberdiek and the Oberdiek Package Support Group
https://github.com/ho-tex/oberdiek/issues


The Base Interpreter refers to any `TeX-Format',
because some files are installed in TDS:tex/generic//.

This work consists of the main source file pdfcol.dtx
and the derived files
   pdfcol.sty, pdfcol.pdf, pdfcol.ins, pdfcol.drv, pdfcol-test1.tex,
   pdfcol-test2.tex, pdfcol-test3.tex, pdfcol-test4.tex.

\endpreamble
\let\MetaPrefix\DoubleperCent

\generate{%
  \file{pdfcol.ins}{\from{pdfcol.dtx}{install}}%
  \file{pdfcol.drv}{\from{pdfcol.dtx}{driver}}%
  \usedir{tex/generic/oberdiek}%
  \file{pdfcol.sty}{\from{pdfcol.dtx}{package}}%
%  \usedir{doc/latex/oberdiek/test}%
%  \file{pdfcol-test1.tex}{\from{pdfcol.dtx}{test1}}%
%  \file{pdfcol-test2.tex}{\from{pdfcol.dtx}{test2}}%
%  \file{pdfcol-test3.tex}{\from{pdfcol.dtx}{test3}}%
%  \file{pdfcol-test4.tex}{\from{pdfcol.dtx}{test4}}%
}

\catcode32=13\relax% active space
\let =\space%
\Msg{************************************************************************}
\Msg{*}
\Msg{* To finish the installation you have to move the following}
\Msg{* file into a directory searched by TeX:}
\Msg{*}
\Msg{*     pdfcol.sty}
\Msg{*}
\Msg{* To produce the documentation run the file `pdfcol.drv'}
\Msg{* through LaTeX.}
\Msg{*}
\Msg{* Happy TeXing!}
\Msg{*}
\Msg{************************************************************************}

\endbatchfile
%</install>
%<*ignore>
\fi
%</ignore>
%<*driver>
\NeedsTeXFormat{LaTeX2e}
\ProvidesFile{pdfcol.drv}%
  [2018/11/01 v1.5 Handle new color stacks for pdfTeX (HO)]%
\documentclass{ltxdoc}
\usepackage{holtxdoc}[2011/11/22]
\begin{document}
  \DocInput{pdfcol.dtx}%
\end{document}
%</driver>
% \fi
%
%
%
% \GetFileInfo{pdfcol.drv}
%
% \title{The \xpackage{pdfcol} package}
% \date{2018/11/01 v1.5}
% \author{Heiko Oberdiek\thanks
% {Please report any issues at \url{https://github.com/ho-tex/oberdiek/issues}}}
%
% \maketitle
%
% \begin{abstract}
% Since version 1.40 \pdfTeX\ supports color stacks.
% The driver file \xfile{pdftex.def} for package \xpackage{color}
% defines and uses a main color stack since version v0.04b.
% Package \xpackage{pdfcol} is intended for package writers.
% It defines macros for setting and maintaining new color stacks.
% \end{abstract}
%
% \tableofcontents
%
% \section{Documentation}
%
% Version 1.40 of \pdfTeX\ adds new primitives \cs{pdfcolorstackinit}
% and \cs{pdfcolorstack}. Now color stacks can be defined and used.
% A main color stack is maintained by the driver file \xfile{pdftex.def}
% similar to dvips or dvipdfm. However the number of color stacks
% is not limited to one in \pdfTeX. Thus further color problems
% can now be solved, such as footnotes across pages or text
% that is set in parallel columns (e.g. packages \xpackage{parallel}
% or \xpackage{parcolumn}). Unlike the main color stack,
% the support by additional color stacks cannot be done in
% a transparent manner.
%
% This package \xpackage{pdfcol} provides an easier interface to
% additional color stacks without the need to use the
% low level primitives.
%
% \subsection{Requirements}
% \label{sec:req}
%
% \begin{itemize}
% \item
%   \pdfTeX\ 1.40 or greater.
% \item
%   \pdfTeX in PDF mode. (I don't know a DVI driver that
%   support several color stacks.)
% \item
%   \xfile{pdftex.def} 2007/01/02 v0.04b.
% \end{itemize}
% Package \xpackage{pdfcol} checks the requirements and
% sets switch \cs{ifpdfcolAvailable} accordingly.
%
% \subsection{Interface}
%
% \begin{declcs}{ifpdfcolAvailable}
% \end{declcs}
% If the requirements of section \ref{sec:req} are met the
% switch \cs{ifpdfcolAvailable} behaves as \cs{iftrue}.
% Otherwise the other interface macros in this section will
% be disabled with a message. Also the first use of such a
% macro will print a message. The messages are print to
% the \xext{log} file only if \pdfTeX\ is not used in PDF mode.
%
% \begin{declcs}{pdfcolErrorNoStacks}
% \end{declcs}
% The first call of \cs{pdfcolErrorNoStacks} prints an error
% message, if color stacks are not available.
%
% \begin{declcs}{pdfcolInitStack} \M{name}
% \end{declcs}
% A new color stack is initialized by \cs{pdfcolInitStack}.
% The \meta{name} is used for indentifying the stack. It usually
% consists of letters and digits. (The name must survive a \cs{csname}.)
%
% The intension of the macro is the definition of an additional
% color stack. Thus the stack is not page bounded like the
% main color stack. Black (\texttt{0 g 0 G}) is used as initial
% color value. And colors are written with modifier \texttt{direct}
% that means without setting the current transfer matrix and changing
% the current point (see documentation of \pdfTeX\ for
% |\pdfliteral direct{...}|).
%
% \begin{declcs}{pdfcolIfStackExists} \M{name} \M{then} \M{else}
% \end{declcs}
% Macro \cs{pdfcolIfStackExists} checks whether color stack \meta{name}
% exists. In case of success argument \meta{then} is executed
% and \meta{else} otherwise.
%
% \begin{declcs}{pdfcolSwitchStack} \M{name}
% \end{declcs}
% Macro \cs{pdfcolSwitchStack} switches the color stack. The color macros
% of package \xpackage{color} (or \xpackage{xcolor}) now uses the
% new color stack with name \meta{name}.
%
% \begin{declcs}{pdfcolSetCurrentColor}
% \end{declcs}
% Macro \cs{pdfcolSetCurrentColor} replaces the topmost
% entry of the stack by the current color (\cs{current@color}).
%
% \begin{declcs}{pdfcolSetCurrent} \M{name}
% \end{declcs}
% Macro \cs{pdfcolSetCurrent} sets the color that is read in
% the top-most entry of color stack \meta{name}. If \meta{name}
% is empty, the default color stack is used.
%
% \StopEventually{
% }
%
% \section{Implementation}
%
%    \begin{macrocode}
%<*package>
%    \end{macrocode}
%
% \subsection{Reload check and package identification}
%    Reload check, especially if the package is not used with \LaTeX.
%    \begin{macrocode}
\begingroup\catcode61\catcode48\catcode32=10\relax%
  \catcode13=5 % ^^M
  \endlinechar=13 %
  \catcode35=6 % #
  \catcode39=12 % '
  \catcode44=12 % ,
  \catcode45=12 % -
  \catcode46=12 % .
  \catcode58=12 % :
  \catcode64=11 % @
  \catcode123=1 % {
  \catcode125=2 % }
  \expandafter\let\expandafter\x\csname ver@pdfcol.sty\endcsname
  \ifx\x\relax % plain-TeX, first loading
  \else
    \def\empty{}%
    \ifx\x\empty % LaTeX, first loading,
      % variable is initialized, but \ProvidesPackage not yet seen
    \else
      \expandafter\ifx\csname PackageInfo\endcsname\relax
        \def\x#1#2{%
          \immediate\write-1{Package #1 Info: #2.}%
        }%
      \else
        \def\x#1#2{\PackageInfo{#1}{#2, stopped}}%
      \fi
      \x{pdfcol}{The package is already loaded}%
      \aftergroup\endinput
    \fi
  \fi
\endgroup%
%    \end{macrocode}
%    Package identification:
%    \begin{macrocode}
\begingroup\catcode61\catcode48\catcode32=10\relax%
  \catcode13=5 % ^^M
  \endlinechar=13 %
  \catcode35=6 % #
  \catcode39=12 % '
  \catcode40=12 % (
  \catcode41=12 % )
  \catcode44=12 % ,
  \catcode45=12 % -
  \catcode46=12 % .
  \catcode47=12 % /
  \catcode58=12 % :
  \catcode64=11 % @
  \catcode91=12 % [
  \catcode93=12 % ]
  \catcode123=1 % {
  \catcode125=2 % }
  \expandafter\ifx\csname ProvidesPackage\endcsname\relax
    \def\x#1#2#3[#4]{\endgroup
      \immediate\write-1{Package: #3 #4}%
      \xdef#1{#4}%
    }%
  \else
    \def\x#1#2[#3]{\endgroup
      #2[{#3}]%
      \ifx#1\@undefined
        \xdef#1{#3}%
      \fi
      \ifx#1\relax
        \xdef#1{#3}%
      \fi
    }%
  \fi
\expandafter\x\csname ver@pdfcol.sty\endcsname
\ProvidesPackage{pdfcol}%
  [2018/11/01 v1.5 Handle new color stacks for pdfTeX (HO)]%
%    \end{macrocode}
%
% \subsection{Catcodes}
%
%    \begin{macrocode}
\begingroup\catcode61\catcode48\catcode32=10\relax%
  \catcode13=5 % ^^M
  \endlinechar=13 %
  \catcode123=1 % {
  \catcode125=2 % }
  \catcode64=11 % @
  \def\x{\endgroup
    \expandafter\edef\csname PDFCOL@AtEnd\endcsname{%
      \endlinechar=\the\endlinechar\relax
      \catcode13=\the\catcode13\relax
      \catcode32=\the\catcode32\relax
      \catcode35=\the\catcode35\relax
      \catcode61=\the\catcode61\relax
      \catcode64=\the\catcode64\relax
      \catcode123=\the\catcode123\relax
      \catcode125=\the\catcode125\relax
    }%
  }%
\x\catcode61\catcode48\catcode32=10\relax%
\catcode13=5 % ^^M
\endlinechar=13 %
\catcode35=6 % #
\catcode64=11 % @
\catcode123=1 % {
\catcode125=2 % }
\def\TMP@EnsureCode#1#2{%
  \edef\PDFCOL@AtEnd{%
    \PDFCOL@AtEnd
    \catcode#1=\the\catcode#1\relax
  }%
  \catcode#1=#2\relax
}
\TMP@EnsureCode{39}{12}% '
\TMP@EnsureCode{40}{12}% (
\TMP@EnsureCode{41}{12}% )
\TMP@EnsureCode{43}{12}% +
\TMP@EnsureCode{44}{12}% ,
\TMP@EnsureCode{46}{12}% .
\TMP@EnsureCode{47}{12}% /
\TMP@EnsureCode{91}{12}% [
\TMP@EnsureCode{93}{12}% ]
\TMP@EnsureCode{96}{12}% `
\edef\PDFCOL@AtEnd{\PDFCOL@AtEnd\noexpand\endinput}
%    \end{macrocode}
%
% \subsection{Check requirements}
%
%    \begin{macro}{\PDFCOL@RequirePackage}
%    \begin{macrocode}
\begingroup\expandafter\expandafter\expandafter\endgroup
\expandafter\ifx\csname RequirePackage\endcsname\relax
  \def\PDFCOL@RequirePackage#1[#2]{\input #1.sty\relax}%
\else
  \def\PDFCOL@RequirePackage#1[#2]{%
    \RequirePackage{#1}[{#2}]%
  }%
\fi
%    \end{macrocode}
%    \end{macro}
%
% LuaTeX Compatability
%    \begin{macrocode}
\ifx\pdfextension\@undefined\else
 \def\pdfcolorstackinit {\pdffeedback colorstackinit}
 \protected\def\pdfcolorstack     {\pdfextension colorstack}
\fi
%    \end{macrocode}
%
%    \begin{macrocode}
\PDFCOL@RequirePackage{ltxcmds}[2010/03/01]
%    \end{macrocode}
%
%    \begin{macro}{ifpdfcolAvailable}
%    \begin{macrocode}
\ltx@newif\ifpdfcolAvailable
\pdfcolAvailabletrue
%    \end{macrocode}
%    \end{macro}
%
% \subsubsection{Check package \xpackage{luacolor}}
%
%    \begin{macrocode}
\ltx@newif\ifPDFCOL@luacolor
\begingroup\expandafter\expandafter\expandafter\endgroup
\expandafter\ifx\csname ver@luacolor.sty\endcsname\relax
  \PDFCOL@luacolorfalse
\else
  \PDFCOL@luacolortrue
\fi
%    \end{macrocode}
%
% \subsubsection{Check PDF mode}
%
%    \begin{macrocode}
\PDFCOL@RequirePackage{infwarerr}[2007/09/09]
\PDFCOL@RequirePackage{ifpdf}[2007/09/09]
\ifcase\ifpdf\ifPDFCOL@luacolor 1\fi\else 1\fi0 %
  \def\PDFCOL@Message{%
    \@PackageWarningNoLine{pdfcol}%
  }%
\else
  \pdfcolAvailablefalse
  \def\PDFCOL@Message{%
    \@PackageInfoNoLine{pdfcol}%
  }%
  \PDFCOL@Message{%
    Interface disabled because of %
    \ifPDFCOL@luacolor
      package `luacolor'%
    \else
      missing PDF mode of pdfTeX%
    \fi
  }%
\fi
%    \end{macrocode}
%
% \subsubsection{Check version of \pdfTeX}
%
%    \begin{macrocode}
\ifpdfcolAvailable
  \begingroup\expandafter\expandafter\expandafter\endgroup
  \expandafter\ifx\csname pdfcolorstack\endcsname\relax
    \pdfcolAvailablefalse
    \PDFCOL@Message{%
      Interface disabled because of too old pdfTeX.\MessageBreak
      Required is version 1.40+ for \string\pdfcolorstack
    }%
  \fi
\fi
\ifpdfcolAvailable
  \begingroup\expandafter\expandafter\expandafter\endgroup
  \expandafter\ifx\csname pdfcolorstack\endcsname\relax
    \pdfcolAvailablefalse
    \PDFCOL@Message{%
      Interface disabled because of too old pdfTeX.\MessageBreak
      Required is version 1.40+ for \string\pdfcolorstackinit
    }%
  \fi
\fi
%    \end{macrocode}
%
% \subsubsection{Check \xfile{pdftex.def}}
%
%    \begin{macrocode}
\ifpdfcolAvailable
  \begingroup\expandafter\expandafter\expandafter\endgroup
  \expandafter\ifx\csname @pdfcolorstack\endcsname\relax
%    \end{macrocode}
%    Try to load package color if it is not yet loaded (\LaTeX\ case).
%    \begin{macrocode}
    \begingroup\expandafter\expandafter\expandafter\endgroup
    \expandafter\ifx\csname ver@color.sty\endcsname\relax
      \begingroup\expandafter\expandafter\expandafter\endgroup
      \expandafter\ifx\csname documentclass\endcsname\relax
      \else
        \RequirePackage[pdftex]{color}\relax
      \fi
    \fi
    \begingroup\expandafter\expandafter\expandafter\endgroup
    \expandafter\ifx\csname @pdfcolorstack\endcsname\relax
      \pdfcolAvailablefalse
      \PDFCOL@Message{%
        Interface disabled because `pdftex.def'\MessageBreak
        is not loaded or it is too old.\MessageBreak
        Required is version 0.04b or greater%
      }%
    \fi
  \fi
\fi
%    \end{macrocode}
%
%    \begin{macrocode}
\let\pdfcolAvailabletrue\relax
\let\pdfcolAvailablefalse\relax
%    \end{macrocode}
%
% \subsection{Enabled interface macros}
%
%    \begin{macrocode}
\ifpdfcolAvailable
%    \end{macrocode}
%
%    \begin{macro}{\pdfcolErrorNoStacks}
%    \begin{macrocode}
  \let\pdfcolErrorNoStacks\relax
%    \end{macrocode}
%    \end{macro}
%
%    \begin{macro}{\pdfcol@Value}
%    \begin{macrocode}
  \expandafter\ifx\csname pdfcol@Value\endcsname\relax
    \def\pdfcol@Value{0 g 0 G}%
  \fi
%    \end{macrocode}
%    \end{macro}
%
%    \begin{macro}{\pdfcol@LiteralModifier}
%    \begin{macrocode}
  \expandafter\ifx\csname pdfcol@LiteralModifier\endcsname\relax
    \def\pdfcol@LiteralModifier{direct}%
  \fi
%    \end{macrocode}
%    \end{macro}
%
%    \begin{macro}{\pdfcolInitStack}
%    \begin{macrocode}
  \def\pdfcolInitStack#1{%
    \expandafter\ifx\csname pdfcol@Stack@#1\endcsname\relax
      \global\expandafter\chardef\csname pdfcol@Stack@#1\endcsname=%
          \pdfcolorstackinit\pdfcol@LiteralModifier{\pdfcol@Value}%
          \relax
      \@PackageInfo{pdfcol}{%
        New color stack `#1' = \number\csname pdfcol@Stack@#1\endcsname
      }%
    \else
      \@PackageError{pdfcol}{%
        Stack `#1' is already defined%
      }\@ehc
    \fi
  }%
%    \end{macrocode}
%    \end{macro}
%
%    \begin{macro}{\pdfcolIfStackExists}
%    \begin{macrocode}
  \def\pdfcolIfStackExists#1{%
    \expandafter\ifx\csname pdfcol@Stack@#1\endcsname\relax
      \expandafter\@secondoftwo
    \else
      \expandafter\@firstoftwo
    \fi
  }%
%    \end{macrocode}
%    \end{macro}
%    \begin{macro}{\@firstoftwo}
%    \begin{macrocode}
  \expandafter\ifx\csname @firstoftwo\endcsname\relax
    \long\def\@firstoftwo#1#2{#1}%
  \fi
%    \end{macrocode}
%    \end{macro}
%    \begin{macro}{\@secondoftwo}
%    \begin{macrocode}
  \expandafter\ifx\csname @secondoftwo\endcsname\relax
    \long\def\@secondoftwo#1#2{#2}%
  \fi
%    \end{macrocode}
%    \end{macro}
%
%    \begin{macro}{\pdfcolSwitchStack}
%    \begin{macrocode}
  \def\pdfcolSwitchStack#1{%
    \pdfcolIfStackExists{#1}{%
      \expandafter\let\expandafter\@pdfcolorstack
                      \csname pdfcol@Stack@#1\endcsname
    }{%
      \pdfcol@ErrorNoStack{#1}%
    }%
  }%
%    \end{macrocode}
%    \end{macro}
%
%    \begin{macro}{\pdfcolSetCurrentColor}
%    \begin{macrocode}
  \def\pdfcolSetCurrentColor{%
    \pdfcolorstack\@pdfcolorstack set{\current@color}%
  }%
%    \end{macrocode}
%    \end{macro}
%
%    \begin{macro}{\pdfcolSetCurrent}
%    \begin{macrocode}
  \def\pdfcolSetCurrent#1{%
    \ifx\\#1\\%
      \pdfcolorstack\@pdfcolorstack current\relax
    \else
      \pdfcolIfStackExists{#1}{%
        \pdfcolorstack\csname pdfcol@Stack@#1\endcsname current\relax
      }{%
        \pdfcol@ErrorNoStack{#1}%
      }%
    \fi
  }%
%    \end{macrocode}
%    \end{macro}
%
%    \begin{macro}{\pdfcol@ErrorNoStack}
%    \begin{macrocode}
  \def\pdfcol@ErrorNoStack#1{%
    \@PackageError{pdfcol}{Stack `#1' does not exists}\@ehc
  }%
%    \end{macrocode}
%    \end{macro}
%
% \subsection{Disabled interface macros}
%
%    \begin{macrocode}
\else
%    \end{macrocode}
%
%    \begin{macro}{\pdfcolErrorNoStacks}
%    \begin{macrocode}
  \def\pdfcolErrorNoStacks{%
    \@PackageError{pdfcol}{%
      Color stacks are not available%
    }{%
      Update pdfTeX (1.40) and `pdftex.def' (0.04b) %
          if necessary.\MessageBreak
      Ensure that `pdftex.def' is loaded %
          (package `color' or `xcolor').\MessageBreak
      Further messages can be found in TeX's %
          protocol file `\jobname.log'.\MessageBreak
      \MessageBreak
      \@ehc
    }%
    \global\let\pdfcolErrorNoStacks\relax
  }%
%    \end{macrocode}
%    \end{macro}
%
%    \begin{macro}{\PDFCOL@Disabled}
%    \begin{macrocode}
  \def\PDFCOL@Disabled{%
    \PDFCOL@Message{%
      pdfTeX's color stacks are not available%
    }%
    \global\let\PDFCOL@Disabled\relax
  }%
%    \end{macrocode}
%    \end{macro}
%
%    \begin{macro}{\pdfcolInitStack}
%    \begin{macrocode}
  \def\pdfcolInitStack#1{%
    \PDFCOL@Disabled
  }%
%    \end{macrocode}
%    \end{macro}
%
%    \begin{macro}{\pdfcolIfStackExists}
%    \begin{macrocode}
  \long\def\pdfcolIfStackExists#1#2#3{#3}%
%    \end{macrocode}
%    \end{macro}
%
%    \begin{macro}{\pdfcolSwitchStack}
%    \begin{macrocode}
  \def\pdfcolSwitchStack#1{%
    \PDFCOL@Disabled
  }%
%    \end{macrocode}
%    \end{macro}
%
%    \begin{macro}{\pdfcolSetCurrentColor}
%    \begin{macrocode}
  \def\pdfcolSetCurrentColor{%
    \PDFCOL@Disabled
  }%
%    \end{macrocode}
%    \end{macro}
%
%    \begin{macro}{\pdfcolSetCurrent}
%    \begin{macrocode}
  \def\pdfcolSetCurrent#1{%
    \PDFCOL@Disabled
  }%
%    \end{macrocode}
%    \end{macro}
%    \begin{macrocode}
\fi
%    \end{macrocode}
%
%    \begin{macrocode}
\PDFCOL@AtEnd%
%</package>
%    \end{macrocode}
%% \section{Installation}
%
% \subsection{Download}
%
% \paragraph{Package.} This package is available on
% CTAN\footnote{\CTANpkg{pdfcol}}:
% \begin{description}
% \item[\CTAN{macros/latex/contrib/oberdiek/pdfcol.dtx}] The source file.
% \item[\CTAN{macros/latex/contrib/oberdiek/pdfcol.pdf}] Documentation.
% \end{description}
%
%
% \paragraph{Bundle.} All the packages of the bundle `oberdiek'
% are also available in a TDS compliant ZIP archive. There
% the packages are already unpacked and the documentation files
% are generated. The files and directories obey the TDS standard.
% \begin{description}
% \item[\CTANinstall{install/macros/latex/contrib/oberdiek.tds.zip}]
% \end{description}
% \emph{TDS} refers to the standard ``A Directory Structure
% for \TeX\ Files'' (\CTANpkg{tds}). Directories
% with \xfile{texmf} in their name are usually organized this way.
%
% \subsection{Bundle installation}
%
% \paragraph{Unpacking.} Unpack the \xfile{oberdiek.tds.zip} in the
% TDS tree (also known as \xfile{texmf} tree) of your choice.
% Example (linux):
% \begin{quote}
%   |unzip oberdiek.tds.zip -d ~/texmf|
% \end{quote}
%
% \subsection{Package installation}
%
% \paragraph{Unpacking.} The \xfile{.dtx} file is a self-extracting
% \docstrip\ archive. The files are extracted by running the
% \xfile{.dtx} through \plainTeX:
% \begin{quote}
%   \verb|tex pdfcol.dtx|
% \end{quote}
%
% \paragraph{TDS.} Now the different files must be moved into
% the different directories in your installation TDS tree
% (also known as \xfile{texmf} tree):
% \begin{quote}
% \def\t{^^A
% \begin{tabular}{@{}>{\ttfamily}l@{ $\rightarrow$ }>{\ttfamily}l@{}}
%   pdfcol.sty & tex/generic/oberdiek/pdfcol.sty\\
%   pdfcol.pdf & doc/latex/oberdiek/pdfcol.pdf\\
%   pdfcol.dtx & source/latex/oberdiek/pdfcol.dtx\\
% \end{tabular}^^A
% }^^A
% \sbox0{\t}^^A
% \ifdim\wd0>\linewidth
%   \begingroup
%     \advance\linewidth by\leftmargin
%     \advance\linewidth by\rightmargin
%   \edef\x{\endgroup
%     \def\noexpand\lw{\the\linewidth}^^A
%   }\x
%   \def\lwbox{^^A
%     \leavevmode
%     \hbox to \linewidth{^^A
%       \kern-\leftmargin\relax
%       \hss
%       \usebox0
%       \hss
%       \kern-\rightmargin\relax
%     }^^A
%   }^^A
%   \ifdim\wd0>\lw
%     \sbox0{\small\t}^^A
%     \ifdim\wd0>\linewidth
%       \ifdim\wd0>\lw
%         \sbox0{\footnotesize\t}^^A
%         \ifdim\wd0>\linewidth
%           \ifdim\wd0>\lw
%             \sbox0{\scriptsize\t}^^A
%             \ifdim\wd0>\linewidth
%               \ifdim\wd0>\lw
%                 \sbox0{\tiny\t}^^A
%                 \ifdim\wd0>\linewidth
%                   \lwbox
%                 \else
%                   \usebox0
%                 \fi
%               \else
%                 \lwbox
%               \fi
%             \else
%               \usebox0
%             \fi
%           \else
%             \lwbox
%           \fi
%         \else
%           \usebox0
%         \fi
%       \else
%         \lwbox
%       \fi
%     \else
%       \usebox0
%     \fi
%   \else
%     \lwbox
%   \fi
% \else
%   \usebox0
% \fi
% \end{quote}
% If you have a \xfile{docstrip.cfg} that configures and enables \docstrip's
% TDS installing feature, then some files can already be in the right
% place, see the documentation of \docstrip.
%
% \subsection{Refresh file name databases}
%
% If your \TeX~distribution
% (\TeX\,Live, \mikTeX, \dots) relies on file name databases, you must refresh
% these. For example, \TeX\,Live\ users run \verb|texhash| or
% \verb|mktexlsr|.
%
% \subsection{Some details for the interested}
%
% \paragraph{Unpacking with \LaTeX.}
% The \xfile{.dtx} chooses its action depending on the format:
% \begin{description}
% \item[\plainTeX:] Run \docstrip\ and extract the files.
% \item[\LaTeX:] Generate the documentation.
% \end{description}
% If you insist on using \LaTeX\ for \docstrip\ (really,
% \docstrip\ does not need \LaTeX), then inform the autodetect routine
% about your intention:
% \begin{quote}
%   \verb|latex \let\install=y\input{pdfcol.dtx}|
% \end{quote}
% Do not forget to quote the argument according to the demands
% of your shell.
%
% \paragraph{Generating the documentation.}
% You can use both the \xfile{.dtx} or the \xfile{.drv} to generate
% the documentation. The process can be configured by the
% configuration file \xfile{ltxdoc.cfg}. For instance, put this
% line into this file, if you want to have A4 as paper format:
% \begin{quote}
%   \verb|\PassOptionsToClass{a4paper}{article}|
% \end{quote}
% An example follows how to generate the
% documentation with pdf\LaTeX:
% \begin{quote}
%\begin{verbatim}
%pdflatex pdfcol.dtx
%makeindex -s gind.ist pdfcol.idx
%pdflatex pdfcol.dtx
%makeindex -s gind.ist pdfcol.idx
%pdflatex pdfcol.dtx
%\end{verbatim}
% \end{quote}
%
% \begin{History}
%   \begin{Version}{2007/09/09 v1.0}
%   \item
%     First version.
%   \end{Version}
%   \begin{Version}{2007/12/09 v1.1}
%   \item
%     \cs{pdfcolSetCurrentColor} added.
%   \end{Version}
%   \begin{Version}{2007/12/12 v1.2}
%   \item
%     Detection for package \xpackage{luacolor} added.
%   \end{Version}
%   \begin{Version}{2016/05/16 v1.3}
%   \item
%     Documentation updates.
%   \end{Version}
%   \begin{Version}{2016/05/17 v1.4}
%   \item
%     Use luatex85 package for new luatex compatibility
%   \end{Version}
%   \begin{Version}{2018/11/01 v1.5}
%   \item
%     Remove luatex85 dependency
%   \end{Version}
% \end{History}
%
% \PrintIndex
%
% \Finale
\endinput
|
% \end{quote}
% Do not forget to quote the argument according to the demands
% of your shell.
%
% \paragraph{Generating the documentation.}
% You can use both the \xfile{.dtx} or the \xfile{.drv} to generate
% the documentation. The process can be configured by the
% configuration file \xfile{ltxdoc.cfg}. For instance, put this
% line into this file, if you want to have A4 as paper format:
% \begin{quote}
%   \verb|\PassOptionsToClass{a4paper}{article}|
% \end{quote}
% An example follows how to generate the
% documentation with pdf\LaTeX:
% \begin{quote}
%\begin{verbatim}
%pdflatex pdfcol.dtx
%makeindex -s gind.ist pdfcol.idx
%pdflatex pdfcol.dtx
%makeindex -s gind.ist pdfcol.idx
%pdflatex pdfcol.dtx
%\end{verbatim}
% \end{quote}
%
% \begin{History}
%   \begin{Version}{2007/09/09 v1.0}
%   \item
%     First version.
%   \end{Version}
%   \begin{Version}{2007/12/09 v1.1}
%   \item
%     \cs{pdfcolSetCurrentColor} added.
%   \end{Version}
%   \begin{Version}{2007/12/12 v1.2}
%   \item
%     Detection for package \xpackage{luacolor} added.
%   \end{Version}
%   \begin{Version}{2016/05/16 v1.3}
%   \item
%     Documentation updates.
%   \end{Version}
%   \begin{Version}{2016/05/17 v1.4}
%   \item
%     Use luatex85 package for new luatex compatibility
%   \end{Version}
%   \begin{Version}{2018/11/01 v1.5}
%   \item
%     Remove luatex85 dependency
%   \end{Version}
% \end{History}
%
% \PrintIndex
%
% \Finale
\endinput

%        (quote the arguments according to the demands of your shell)
%
% Documentation:
%    (a) If pdfcol.drv is present:
%           latex pdfcol.drv
%    (b) Without pdfcol.drv:
%           latex pdfcol.dtx; ...
%    The class ltxdoc loads the configuration file ltxdoc.cfg
%    if available. Here you can specify further options, e.g.
%    use A4 as paper format:
%       \PassOptionsToClass{a4paper}{article}
%
%    Programm calls to get the documentation (example):
%       pdflatex pdfcol.dtx
%       makeindex -s gind.ist pdfcol.idx
%       pdflatex pdfcol.dtx
%       makeindex -s gind.ist pdfcol.idx
%       pdflatex pdfcol.dtx
%
% Installation:
%    TDS:tex/generic/oberdiek/pdfcol.sty
%    TDS:doc/latex/oberdiek/pdfcol.pdf
%    TDS:source/latex/oberdiek/pdfcol.dtx
%
%<*ignore>
\begingroup
  \catcode123=1 %
  \catcode125=2 %
  \def\x{LaTeX2e}%
\expandafter\endgroup
\ifcase 0\ifx\install y1\fi\expandafter
         \ifx\csname processbatchFile\endcsname\relax\else1\fi
         \ifx\fmtname\x\else 1\fi\relax
\else\csname fi\endcsname
%</ignore>
%<*install>
\input docstrip.tex
\Msg{************************************************************************}
\Msg{* Installation}
\Msg{* Package: pdfcol 2018/11/01 v1.5 Handle new color stacks for pdfTeX (HO)}
\Msg{************************************************************************}

\keepsilent
\askforoverwritefalse

\let\MetaPrefix\relax
\preamble

This is a generated file.

Project: pdfcol
Version: 2018/11/01 v1.5

Copyright (C)
   2007 Heiko Oberdiek
   2016-2019 Oberdiek Package Support Group

This work may be distributed and/or modified under the
conditions of the LaTeX Project Public License, either
version 1.3c of this license or (at your option) any later
version. This version of this license is in
   https://www.latex-project.org/lppl/lppl-1-3c.txt
and the latest version of this license is in
   https://www.latex-project.org/lppl.txt
and version 1.3 or later is part of all distributions of
LaTeX version 2005/12/01 or later.

This work has the LPPL maintenance status "maintained".

The Current Maintainers of this work are
Heiko Oberdiek and the Oberdiek Package Support Group
https://github.com/ho-tex/oberdiek/issues


The Base Interpreter refers to any `TeX-Format',
because some files are installed in TDS:tex/generic//.

This work consists of the main source file pdfcol.dtx
and the derived files
   pdfcol.sty, pdfcol.pdf, pdfcol.ins, pdfcol.drv, pdfcol-test1.tex,
   pdfcol-test2.tex, pdfcol-test3.tex, pdfcol-test4.tex.

\endpreamble
\let\MetaPrefix\DoubleperCent

\generate{%
  \file{pdfcol.ins}{\from{pdfcol.dtx}{install}}%
  \file{pdfcol.drv}{\from{pdfcol.dtx}{driver}}%
  \usedir{tex/generic/oberdiek}%
  \file{pdfcol.sty}{\from{pdfcol.dtx}{package}}%
%  \usedir{doc/latex/oberdiek/test}%
%  \file{pdfcol-test1.tex}{\from{pdfcol.dtx}{test1}}%
%  \file{pdfcol-test2.tex}{\from{pdfcol.dtx}{test2}}%
%  \file{pdfcol-test3.tex}{\from{pdfcol.dtx}{test3}}%
%  \file{pdfcol-test4.tex}{\from{pdfcol.dtx}{test4}}%
}

\catcode32=13\relax% active space
\let =\space%
\Msg{************************************************************************}
\Msg{*}
\Msg{* To finish the installation you have to move the following}
\Msg{* file into a directory searched by TeX:}
\Msg{*}
\Msg{*     pdfcol.sty}
\Msg{*}
\Msg{* To produce the documentation run the file `pdfcol.drv'}
\Msg{* through LaTeX.}
\Msg{*}
\Msg{* Happy TeXing!}
\Msg{*}
\Msg{************************************************************************}

\endbatchfile
%</install>
%<*ignore>
\fi
%</ignore>
%<*driver>
\NeedsTeXFormat{LaTeX2e}
\ProvidesFile{pdfcol.drv}%
  [2018/11/01 v1.5 Handle new color stacks for pdfTeX (HO)]%
\documentclass{ltxdoc}
\usepackage{holtxdoc}[2011/11/22]
\begin{document}
  \DocInput{pdfcol.dtx}%
\end{document}
%</driver>
% \fi
%
%
%
% \GetFileInfo{pdfcol.drv}
%
% \title{The \xpackage{pdfcol} package}
% \date{2018/11/01 v1.5}
% \author{Heiko Oberdiek\thanks
% {Please report any issues at \url{https://github.com/ho-tex/oberdiek/issues}}}
%
% \maketitle
%
% \begin{abstract}
% Since version 1.40 \pdfTeX\ supports color stacks.
% The driver file \xfile{pdftex.def} for package \xpackage{color}
% defines and uses a main color stack since version v0.04b.
% Package \xpackage{pdfcol} is intended for package writers.
% It defines macros for setting and maintaining new color stacks.
% \end{abstract}
%
% \tableofcontents
%
% \section{Documentation}
%
% Version 1.40 of \pdfTeX\ adds new primitives \cs{pdfcolorstackinit}
% and \cs{pdfcolorstack}. Now color stacks can be defined and used.
% A main color stack is maintained by the driver file \xfile{pdftex.def}
% similar to dvips or dvipdfm. However the number of color stacks
% is not limited to one in \pdfTeX. Thus further color problems
% can now be solved, such as footnotes across pages or text
% that is set in parallel columns (e.g. packages \xpackage{parallel}
% or \xpackage{parcolumn}). Unlike the main color stack,
% the support by additional color stacks cannot be done in
% a transparent manner.
%
% This package \xpackage{pdfcol} provides an easier interface to
% additional color stacks without the need to use the
% low level primitives.
%
% \subsection{Requirements}
% \label{sec:req}
%
% \begin{itemize}
% \item
%   \pdfTeX\ 1.40 or greater.
% \item
%   \pdfTeX in PDF mode. (I don't know a DVI driver that
%   support several color stacks.)
% \item
%   \xfile{pdftex.def} 2007/01/02 v0.04b.
% \end{itemize}
% Package \xpackage{pdfcol} checks the requirements and
% sets switch \cs{ifpdfcolAvailable} accordingly.
%
% \subsection{Interface}
%
% \begin{declcs}{ifpdfcolAvailable}
% \end{declcs}
% If the requirements of section \ref{sec:req} are met the
% switch \cs{ifpdfcolAvailable} behaves as \cs{iftrue}.
% Otherwise the other interface macros in this section will
% be disabled with a message. Also the first use of such a
% macro will print a message. The messages are print to
% the \xext{log} file only if \pdfTeX\ is not used in PDF mode.
%
% \begin{declcs}{pdfcolErrorNoStacks}
% \end{declcs}
% The first call of \cs{pdfcolErrorNoStacks} prints an error
% message, if color stacks are not available.
%
% \begin{declcs}{pdfcolInitStack} \M{name}
% \end{declcs}
% A new color stack is initialized by \cs{pdfcolInitStack}.
% The \meta{name} is used for indentifying the stack. It usually
% consists of letters and digits. (The name must survive a \cs{csname}.)
%
% The intension of the macro is the definition of an additional
% color stack. Thus the stack is not page bounded like the
% main color stack. Black (\texttt{0 g 0 G}) is used as initial
% color value. And colors are written with modifier \texttt{direct}
% that means without setting the current transfer matrix and changing
% the current point (see documentation of \pdfTeX\ for
% |\pdfliteral direct{...}|).
%
% \begin{declcs}{pdfcolIfStackExists} \M{name} \M{then} \M{else}
% \end{declcs}
% Macro \cs{pdfcolIfStackExists} checks whether color stack \meta{name}
% exists. In case of success argument \meta{then} is executed
% and \meta{else} otherwise.
%
% \begin{declcs}{pdfcolSwitchStack} \M{name}
% \end{declcs}
% Macro \cs{pdfcolSwitchStack} switches the color stack. The color macros
% of package \xpackage{color} (or \xpackage{xcolor}) now uses the
% new color stack with name \meta{name}.
%
% \begin{declcs}{pdfcolSetCurrentColor}
% \end{declcs}
% Macro \cs{pdfcolSetCurrentColor} replaces the topmost
% entry of the stack by the current color (\cs{current@color}).
%
% \begin{declcs}{pdfcolSetCurrent} \M{name}
% \end{declcs}
% Macro \cs{pdfcolSetCurrent} sets the color that is read in
% the top-most entry of color stack \meta{name}. If \meta{name}
% is empty, the default color stack is used.
%
% \StopEventually{
% }
%
% \section{Implementation}
%
%    \begin{macrocode}
%<*package>
%    \end{macrocode}
%
% \subsection{Reload check and package identification}
%    Reload check, especially if the package is not used with \LaTeX.
%    \begin{macrocode}
\begingroup\catcode61\catcode48\catcode32=10\relax%
  \catcode13=5 % ^^M
  \endlinechar=13 %
  \catcode35=6 % #
  \catcode39=12 % '
  \catcode44=12 % ,
  \catcode45=12 % -
  \catcode46=12 % .
  \catcode58=12 % :
  \catcode64=11 % @
  \catcode123=1 % {
  \catcode125=2 % }
  \expandafter\let\expandafter\x\csname ver@pdfcol.sty\endcsname
  \ifx\x\relax % plain-TeX, first loading
  \else
    \def\empty{}%
    \ifx\x\empty % LaTeX, first loading,
      % variable is initialized, but \ProvidesPackage not yet seen
    \else
      \expandafter\ifx\csname PackageInfo\endcsname\relax
        \def\x#1#2{%
          \immediate\write-1{Package #1 Info: #2.}%
        }%
      \else
        \def\x#1#2{\PackageInfo{#1}{#2, stopped}}%
      \fi
      \x{pdfcol}{The package is already loaded}%
      \aftergroup\endinput
    \fi
  \fi
\endgroup%
%    \end{macrocode}
%    Package identification:
%    \begin{macrocode}
\begingroup\catcode61\catcode48\catcode32=10\relax%
  \catcode13=5 % ^^M
  \endlinechar=13 %
  \catcode35=6 % #
  \catcode39=12 % '
  \catcode40=12 % (
  \catcode41=12 % )
  \catcode44=12 % ,
  \catcode45=12 % -
  \catcode46=12 % .
  \catcode47=12 % /
  \catcode58=12 % :
  \catcode64=11 % @
  \catcode91=12 % [
  \catcode93=12 % ]
  \catcode123=1 % {
  \catcode125=2 % }
  \expandafter\ifx\csname ProvidesPackage\endcsname\relax
    \def\x#1#2#3[#4]{\endgroup
      \immediate\write-1{Package: #3 #4}%
      \xdef#1{#4}%
    }%
  \else
    \def\x#1#2[#3]{\endgroup
      #2[{#3}]%
      \ifx#1\@undefined
        \xdef#1{#3}%
      \fi
      \ifx#1\relax
        \xdef#1{#3}%
      \fi
    }%
  \fi
\expandafter\x\csname ver@pdfcol.sty\endcsname
\ProvidesPackage{pdfcol}%
  [2018/11/01 v1.5 Handle new color stacks for pdfTeX (HO)]%
%    \end{macrocode}
%
% \subsection{Catcodes}
%
%    \begin{macrocode}
\begingroup\catcode61\catcode48\catcode32=10\relax%
  \catcode13=5 % ^^M
  \endlinechar=13 %
  \catcode123=1 % {
  \catcode125=2 % }
  \catcode64=11 % @
  \def\x{\endgroup
    \expandafter\edef\csname PDFCOL@AtEnd\endcsname{%
      \endlinechar=\the\endlinechar\relax
      \catcode13=\the\catcode13\relax
      \catcode32=\the\catcode32\relax
      \catcode35=\the\catcode35\relax
      \catcode61=\the\catcode61\relax
      \catcode64=\the\catcode64\relax
      \catcode123=\the\catcode123\relax
      \catcode125=\the\catcode125\relax
    }%
  }%
\x\catcode61\catcode48\catcode32=10\relax%
\catcode13=5 % ^^M
\endlinechar=13 %
\catcode35=6 % #
\catcode64=11 % @
\catcode123=1 % {
\catcode125=2 % }
\def\TMP@EnsureCode#1#2{%
  \edef\PDFCOL@AtEnd{%
    \PDFCOL@AtEnd
    \catcode#1=\the\catcode#1\relax
  }%
  \catcode#1=#2\relax
}
\TMP@EnsureCode{39}{12}% '
\TMP@EnsureCode{40}{12}% (
\TMP@EnsureCode{41}{12}% )
\TMP@EnsureCode{43}{12}% +
\TMP@EnsureCode{44}{12}% ,
\TMP@EnsureCode{46}{12}% .
\TMP@EnsureCode{47}{12}% /
\TMP@EnsureCode{91}{12}% [
\TMP@EnsureCode{93}{12}% ]
\TMP@EnsureCode{96}{12}% `
\edef\PDFCOL@AtEnd{\PDFCOL@AtEnd\noexpand\endinput}
%    \end{macrocode}
%
% \subsection{Check requirements}
%
%    \begin{macro}{\PDFCOL@RequirePackage}
%    \begin{macrocode}
\begingroup\expandafter\expandafter\expandafter\endgroup
\expandafter\ifx\csname RequirePackage\endcsname\relax
  \def\PDFCOL@RequirePackage#1[#2]{\input #1.sty\relax}%
\else
  \def\PDFCOL@RequirePackage#1[#2]{%
    \RequirePackage{#1}[{#2}]%
  }%
\fi
%    \end{macrocode}
%    \end{macro}
%
% LuaTeX Compatability
%    \begin{macrocode}
\ifx\pdfextension\@undefined\else
 \def\pdfcolorstackinit {\pdffeedback colorstackinit}
 \protected\def\pdfcolorstack     {\pdfextension colorstack}
\fi
%    \end{macrocode}
%
%    \begin{macrocode}
\PDFCOL@RequirePackage{ltxcmds}[2010/03/01]
%    \end{macrocode}
%
%    \begin{macro}{ifpdfcolAvailable}
%    \begin{macrocode}
\ltx@newif\ifpdfcolAvailable
\pdfcolAvailabletrue
%    \end{macrocode}
%    \end{macro}
%
% \subsubsection{Check package \xpackage{luacolor}}
%
%    \begin{macrocode}
\ltx@newif\ifPDFCOL@luacolor
\begingroup\expandafter\expandafter\expandafter\endgroup
\expandafter\ifx\csname ver@luacolor.sty\endcsname\relax
  \PDFCOL@luacolorfalse
\else
  \PDFCOL@luacolortrue
\fi
%    \end{macrocode}
%
% \subsubsection{Check PDF mode}
%
%    \begin{macrocode}
\PDFCOL@RequirePackage{infwarerr}[2007/09/09]
\PDFCOL@RequirePackage{ifpdf}[2007/09/09]
\ifcase\ifpdf\ifPDFCOL@luacolor 1\fi\else 1\fi0 %
  \def\PDFCOL@Message{%
    \@PackageWarningNoLine{pdfcol}%
  }%
\else
  \pdfcolAvailablefalse
  \def\PDFCOL@Message{%
    \@PackageInfoNoLine{pdfcol}%
  }%
  \PDFCOL@Message{%
    Interface disabled because of %
    \ifPDFCOL@luacolor
      package `luacolor'%
    \else
      missing PDF mode of pdfTeX%
    \fi
  }%
\fi
%    \end{macrocode}
%
% \subsubsection{Check version of \pdfTeX}
%
%    \begin{macrocode}
\ifpdfcolAvailable
  \begingroup\expandafter\expandafter\expandafter\endgroup
  \expandafter\ifx\csname pdfcolorstack\endcsname\relax
    \pdfcolAvailablefalse
    \PDFCOL@Message{%
      Interface disabled because of too old pdfTeX.\MessageBreak
      Required is version 1.40+ for \string\pdfcolorstack
    }%
  \fi
\fi
\ifpdfcolAvailable
  \begingroup\expandafter\expandafter\expandafter\endgroup
  \expandafter\ifx\csname pdfcolorstack\endcsname\relax
    \pdfcolAvailablefalse
    \PDFCOL@Message{%
      Interface disabled because of too old pdfTeX.\MessageBreak
      Required is version 1.40+ for \string\pdfcolorstackinit
    }%
  \fi
\fi
%    \end{macrocode}
%
% \subsubsection{Check \xfile{pdftex.def}}
%
%    \begin{macrocode}
\ifpdfcolAvailable
  \begingroup\expandafter\expandafter\expandafter\endgroup
  \expandafter\ifx\csname @pdfcolorstack\endcsname\relax
%    \end{macrocode}
%    Try to load package color if it is not yet loaded (\LaTeX\ case).
%    \begin{macrocode}
    \begingroup\expandafter\expandafter\expandafter\endgroup
    \expandafter\ifx\csname ver@color.sty\endcsname\relax
      \begingroup\expandafter\expandafter\expandafter\endgroup
      \expandafter\ifx\csname documentclass\endcsname\relax
      \else
        \RequirePackage[pdftex]{color}\relax
      \fi
    \fi
    \begingroup\expandafter\expandafter\expandafter\endgroup
    \expandafter\ifx\csname @pdfcolorstack\endcsname\relax
      \pdfcolAvailablefalse
      \PDFCOL@Message{%
        Interface disabled because `pdftex.def'\MessageBreak
        is not loaded or it is too old.\MessageBreak
        Required is version 0.04b or greater%
      }%
    \fi
  \fi
\fi
%    \end{macrocode}
%
%    \begin{macrocode}
\let\pdfcolAvailabletrue\relax
\let\pdfcolAvailablefalse\relax
%    \end{macrocode}
%
% \subsection{Enabled interface macros}
%
%    \begin{macrocode}
\ifpdfcolAvailable
%    \end{macrocode}
%
%    \begin{macro}{\pdfcolErrorNoStacks}
%    \begin{macrocode}
  \let\pdfcolErrorNoStacks\relax
%    \end{macrocode}
%    \end{macro}
%
%    \begin{macro}{\pdfcol@Value}
%    \begin{macrocode}
  \expandafter\ifx\csname pdfcol@Value\endcsname\relax
    \def\pdfcol@Value{0 g 0 G}%
  \fi
%    \end{macrocode}
%    \end{macro}
%
%    \begin{macro}{\pdfcol@LiteralModifier}
%    \begin{macrocode}
  \expandafter\ifx\csname pdfcol@LiteralModifier\endcsname\relax
    \def\pdfcol@LiteralModifier{direct}%
  \fi
%    \end{macrocode}
%    \end{macro}
%
%    \begin{macro}{\pdfcolInitStack}
%    \begin{macrocode}
  \def\pdfcolInitStack#1{%
    \expandafter\ifx\csname pdfcol@Stack@#1\endcsname\relax
      \global\expandafter\chardef\csname pdfcol@Stack@#1\endcsname=%
          \pdfcolorstackinit\pdfcol@LiteralModifier{\pdfcol@Value}%
          \relax
      \@PackageInfo{pdfcol}{%
        New color stack `#1' = \number\csname pdfcol@Stack@#1\endcsname
      }%
    \else
      \@PackageError{pdfcol}{%
        Stack `#1' is already defined%
      }\@ehc
    \fi
  }%
%    \end{macrocode}
%    \end{macro}
%
%    \begin{macro}{\pdfcolIfStackExists}
%    \begin{macrocode}
  \def\pdfcolIfStackExists#1{%
    \expandafter\ifx\csname pdfcol@Stack@#1\endcsname\relax
      \expandafter\@secondoftwo
    \else
      \expandafter\@firstoftwo
    \fi
  }%
%    \end{macrocode}
%    \end{macro}
%    \begin{macro}{\@firstoftwo}
%    \begin{macrocode}
  \expandafter\ifx\csname @firstoftwo\endcsname\relax
    \long\def\@firstoftwo#1#2{#1}%
  \fi
%    \end{macrocode}
%    \end{macro}
%    \begin{macro}{\@secondoftwo}
%    \begin{macrocode}
  \expandafter\ifx\csname @secondoftwo\endcsname\relax
    \long\def\@secondoftwo#1#2{#2}%
  \fi
%    \end{macrocode}
%    \end{macro}
%
%    \begin{macro}{\pdfcolSwitchStack}
%    \begin{macrocode}
  \def\pdfcolSwitchStack#1{%
    \pdfcolIfStackExists{#1}{%
      \expandafter\let\expandafter\@pdfcolorstack
                      \csname pdfcol@Stack@#1\endcsname
    }{%
      \pdfcol@ErrorNoStack{#1}%
    }%
  }%
%    \end{macrocode}
%    \end{macro}
%
%    \begin{macro}{\pdfcolSetCurrentColor}
%    \begin{macrocode}
  \def\pdfcolSetCurrentColor{%
    \pdfcolorstack\@pdfcolorstack set{\current@color}%
  }%
%    \end{macrocode}
%    \end{macro}
%
%    \begin{macro}{\pdfcolSetCurrent}
%    \begin{macrocode}
  \def\pdfcolSetCurrent#1{%
    \ifx\\#1\\%
      \pdfcolorstack\@pdfcolorstack current\relax
    \else
      \pdfcolIfStackExists{#1}{%
        \pdfcolorstack\csname pdfcol@Stack@#1\endcsname current\relax
      }{%
        \pdfcol@ErrorNoStack{#1}%
      }%
    \fi
  }%
%    \end{macrocode}
%    \end{macro}
%
%    \begin{macro}{\pdfcol@ErrorNoStack}
%    \begin{macrocode}
  \def\pdfcol@ErrorNoStack#1{%
    \@PackageError{pdfcol}{Stack `#1' does not exists}\@ehc
  }%
%    \end{macrocode}
%    \end{macro}
%
% \subsection{Disabled interface macros}
%
%    \begin{macrocode}
\else
%    \end{macrocode}
%
%    \begin{macro}{\pdfcolErrorNoStacks}
%    \begin{macrocode}
  \def\pdfcolErrorNoStacks{%
    \@PackageError{pdfcol}{%
      Color stacks are not available%
    }{%
      Update pdfTeX (1.40) and `pdftex.def' (0.04b) %
          if necessary.\MessageBreak
      Ensure that `pdftex.def' is loaded %
          (package `color' or `xcolor').\MessageBreak
      Further messages can be found in TeX's %
          protocol file `\jobname.log'.\MessageBreak
      \MessageBreak
      \@ehc
    }%
    \global\let\pdfcolErrorNoStacks\relax
  }%
%    \end{macrocode}
%    \end{macro}
%
%    \begin{macro}{\PDFCOL@Disabled}
%    \begin{macrocode}
  \def\PDFCOL@Disabled{%
    \PDFCOL@Message{%
      pdfTeX's color stacks are not available%
    }%
    \global\let\PDFCOL@Disabled\relax
  }%
%    \end{macrocode}
%    \end{macro}
%
%    \begin{macro}{\pdfcolInitStack}
%    \begin{macrocode}
  \def\pdfcolInitStack#1{%
    \PDFCOL@Disabled
  }%
%    \end{macrocode}
%    \end{macro}
%
%    \begin{macro}{\pdfcolIfStackExists}
%    \begin{macrocode}
  \long\def\pdfcolIfStackExists#1#2#3{#3}%
%    \end{macrocode}
%    \end{macro}
%
%    \begin{macro}{\pdfcolSwitchStack}
%    \begin{macrocode}
  \def\pdfcolSwitchStack#1{%
    \PDFCOL@Disabled
  }%
%    \end{macrocode}
%    \end{macro}
%
%    \begin{macro}{\pdfcolSetCurrentColor}
%    \begin{macrocode}
  \def\pdfcolSetCurrentColor{%
    \PDFCOL@Disabled
  }%
%    \end{macrocode}
%    \end{macro}
%
%    \begin{macro}{\pdfcolSetCurrent}
%    \begin{macrocode}
  \def\pdfcolSetCurrent#1{%
    \PDFCOL@Disabled
  }%
%    \end{macrocode}
%    \end{macro}
%    \begin{macrocode}
\fi
%    \end{macrocode}
%
%    \begin{macrocode}
\PDFCOL@AtEnd%
%</package>
%    \end{macrocode}
%% \section{Installation}
%
% \subsection{Download}
%
% \paragraph{Package.} This package is available on
% CTAN\footnote{\CTANpkg{pdfcol}}:
% \begin{description}
% \item[\CTAN{macros/latex/contrib/oberdiek/pdfcol.dtx}] The source file.
% \item[\CTAN{macros/latex/contrib/oberdiek/pdfcol.pdf}] Documentation.
% \end{description}
%
%
% \paragraph{Bundle.} All the packages of the bundle `oberdiek'
% are also available in a TDS compliant ZIP archive. There
% the packages are already unpacked and the documentation files
% are generated. The files and directories obey the TDS standard.
% \begin{description}
% \item[\CTANinstall{install/macros/latex/contrib/oberdiek.tds.zip}]
% \end{description}
% \emph{TDS} refers to the standard ``A Directory Structure
% for \TeX\ Files'' (\CTANpkg{tds}). Directories
% with \xfile{texmf} in their name are usually organized this way.
%
% \subsection{Bundle installation}
%
% \paragraph{Unpacking.} Unpack the \xfile{oberdiek.tds.zip} in the
% TDS tree (also known as \xfile{texmf} tree) of your choice.
% Example (linux):
% \begin{quote}
%   |unzip oberdiek.tds.zip -d ~/texmf|
% \end{quote}
%
% \subsection{Package installation}
%
% \paragraph{Unpacking.} The \xfile{.dtx} file is a self-extracting
% \docstrip\ archive. The files are extracted by running the
% \xfile{.dtx} through \plainTeX:
% \begin{quote}
%   \verb|tex pdfcol.dtx|
% \end{quote}
%
% \paragraph{TDS.} Now the different files must be moved into
% the different directories in your installation TDS tree
% (also known as \xfile{texmf} tree):
% \begin{quote}
% \def\t{^^A
% \begin{tabular}{@{}>{\ttfamily}l@{ $\rightarrow$ }>{\ttfamily}l@{}}
%   pdfcol.sty & tex/generic/oberdiek/pdfcol.sty\\
%   pdfcol.pdf & doc/latex/oberdiek/pdfcol.pdf\\
%   pdfcol.dtx & source/latex/oberdiek/pdfcol.dtx\\
% \end{tabular}^^A
% }^^A
% \sbox0{\t}^^A
% \ifdim\wd0>\linewidth
%   \begingroup
%     \advance\linewidth by\leftmargin
%     \advance\linewidth by\rightmargin
%   \edef\x{\endgroup
%     \def\noexpand\lw{\the\linewidth}^^A
%   }\x
%   \def\lwbox{^^A
%     \leavevmode
%     \hbox to \linewidth{^^A
%       \kern-\leftmargin\relax
%       \hss
%       \usebox0
%       \hss
%       \kern-\rightmargin\relax
%     }^^A
%   }^^A
%   \ifdim\wd0>\lw
%     \sbox0{\small\t}^^A
%     \ifdim\wd0>\linewidth
%       \ifdim\wd0>\lw
%         \sbox0{\footnotesize\t}^^A
%         \ifdim\wd0>\linewidth
%           \ifdim\wd0>\lw
%             \sbox0{\scriptsize\t}^^A
%             \ifdim\wd0>\linewidth
%               \ifdim\wd0>\lw
%                 \sbox0{\tiny\t}^^A
%                 \ifdim\wd0>\linewidth
%                   \lwbox
%                 \else
%                   \usebox0
%                 \fi
%               \else
%                 \lwbox
%               \fi
%             \else
%               \usebox0
%             \fi
%           \else
%             \lwbox
%           \fi
%         \else
%           \usebox0
%         \fi
%       \else
%         \lwbox
%       \fi
%     \else
%       \usebox0
%     \fi
%   \else
%     \lwbox
%   \fi
% \else
%   \usebox0
% \fi
% \end{quote}
% If you have a \xfile{docstrip.cfg} that configures and enables \docstrip's
% TDS installing feature, then some files can already be in the right
% place, see the documentation of \docstrip.
%
% \subsection{Refresh file name databases}
%
% If your \TeX~distribution
% (\TeX\,Live, \mikTeX, \dots) relies on file name databases, you must refresh
% these. For example, \TeX\,Live\ users run \verb|texhash| or
% \verb|mktexlsr|.
%
% \subsection{Some details for the interested}
%
% \paragraph{Unpacking with \LaTeX.}
% The \xfile{.dtx} chooses its action depending on the format:
% \begin{description}
% \item[\plainTeX:] Run \docstrip\ and extract the files.
% \item[\LaTeX:] Generate the documentation.
% \end{description}
% If you insist on using \LaTeX\ for \docstrip\ (really,
% \docstrip\ does not need \LaTeX), then inform the autodetect routine
% about your intention:
% \begin{quote}
%   \verb|latex \let\install=y% \iffalse meta-comment
%
% File: pdfcol.dtx
% Version: 2018/11/01 v1.5
% Info: Handle new color stacks for pdfTeX
%
% Copyright (C)
%    2007 Heiko Oberdiek
%    2016-2019 Oberdiek Package Support Group
%    https://github.com/ho-tex/oberdiek/issues
%
% This work may be distributed and/or modified under the
% conditions of the LaTeX Project Public License, either
% version 1.3c of this license or (at your option) any later
% version. This version of this license is in
%    https://www.latex-project.org/lppl/lppl-1-3c.txt
% and the latest version of this license is in
%    https://www.latex-project.org/lppl.txt
% and version 1.3 or later is part of all distributions of
% LaTeX version 2005/12/01 or later.
%
% This work has the LPPL maintenance status "maintained".
%
% The Current Maintainers of this work are
% Heiko Oberdiek and the Oberdiek Package Support Group
% https://github.com/ho-tex/oberdiek/issues
%
% The Base Interpreter refers to any `TeX-Format',
% because some files are installed in TDS:tex/generic//.
%
% This work consists of the main source file pdfcol.dtx
% and the derived files
%    pdfcol.sty, pdfcol.pdf, pdfcol.ins, pdfcol.drv, pdfcol-test1.tex,
%    pdfcol-test2.tex, pdfcol-test3.tex, pdfcol-test4.tex.
%
% Distribution:
%    CTAN:macros/latex/contrib/oberdiek/pdfcol.dtx
%    CTAN:macros/latex/contrib/oberdiek/pdfcol.pdf
%
% Unpacking:
%    (a) If pdfcol.ins is present:
%           tex pdfcol.ins
%    (b) Without pdfcol.ins:
%           tex pdfcol.dtx
%    (c) If you insist on using LaTeX
%           latex \let\install=y% \iffalse meta-comment
%
% File: pdfcol.dtx
% Version: 2018/11/01 v1.5
% Info: Handle new color stacks for pdfTeX
%
% Copyright (C)
%    2007 Heiko Oberdiek
%    2016-2019 Oberdiek Package Support Group
%    https://github.com/ho-tex/oberdiek/issues
%
% This work may be distributed and/or modified under the
% conditions of the LaTeX Project Public License, either
% version 1.3c of this license or (at your option) any later
% version. This version of this license is in
%    https://www.latex-project.org/lppl/lppl-1-3c.txt
% and the latest version of this license is in
%    https://www.latex-project.org/lppl.txt
% and version 1.3 or later is part of all distributions of
% LaTeX version 2005/12/01 or later.
%
% This work has the LPPL maintenance status "maintained".
%
% The Current Maintainers of this work are
% Heiko Oberdiek and the Oberdiek Package Support Group
% https://github.com/ho-tex/oberdiek/issues
%
% The Base Interpreter refers to any `TeX-Format',
% because some files are installed in TDS:tex/generic//.
%
% This work consists of the main source file pdfcol.dtx
% and the derived files
%    pdfcol.sty, pdfcol.pdf, pdfcol.ins, pdfcol.drv, pdfcol-test1.tex,
%    pdfcol-test2.tex, pdfcol-test3.tex, pdfcol-test4.tex.
%
% Distribution:
%    CTAN:macros/latex/contrib/oberdiek/pdfcol.dtx
%    CTAN:macros/latex/contrib/oberdiek/pdfcol.pdf
%
% Unpacking:
%    (a) If pdfcol.ins is present:
%           tex pdfcol.ins
%    (b) Without pdfcol.ins:
%           tex pdfcol.dtx
%    (c) If you insist on using LaTeX
%           latex \let\install=y\input{pdfcol.dtx}
%        (quote the arguments according to the demands of your shell)
%
% Documentation:
%    (a) If pdfcol.drv is present:
%           latex pdfcol.drv
%    (b) Without pdfcol.drv:
%           latex pdfcol.dtx; ...
%    The class ltxdoc loads the configuration file ltxdoc.cfg
%    if available. Here you can specify further options, e.g.
%    use A4 as paper format:
%       \PassOptionsToClass{a4paper}{article}
%
%    Programm calls to get the documentation (example):
%       pdflatex pdfcol.dtx
%       makeindex -s gind.ist pdfcol.idx
%       pdflatex pdfcol.dtx
%       makeindex -s gind.ist pdfcol.idx
%       pdflatex pdfcol.dtx
%
% Installation:
%    TDS:tex/generic/oberdiek/pdfcol.sty
%    TDS:doc/latex/oberdiek/pdfcol.pdf
%    TDS:source/latex/oberdiek/pdfcol.dtx
%
%<*ignore>
\begingroup
  \catcode123=1 %
  \catcode125=2 %
  \def\x{LaTeX2e}%
\expandafter\endgroup
\ifcase 0\ifx\install y1\fi\expandafter
         \ifx\csname processbatchFile\endcsname\relax\else1\fi
         \ifx\fmtname\x\else 1\fi\relax
\else\csname fi\endcsname
%</ignore>
%<*install>
\input docstrip.tex
\Msg{************************************************************************}
\Msg{* Installation}
\Msg{* Package: pdfcol 2018/11/01 v1.5 Handle new color stacks for pdfTeX (HO)}
\Msg{************************************************************************}

\keepsilent
\askforoverwritefalse

\let\MetaPrefix\relax
\preamble

This is a generated file.

Project: pdfcol
Version: 2018/11/01 v1.5

Copyright (C)
   2007 Heiko Oberdiek
   2016-2019 Oberdiek Package Support Group

This work may be distributed and/or modified under the
conditions of the LaTeX Project Public License, either
version 1.3c of this license or (at your option) any later
version. This version of this license is in
   https://www.latex-project.org/lppl/lppl-1-3c.txt
and the latest version of this license is in
   https://www.latex-project.org/lppl.txt
and version 1.3 or later is part of all distributions of
LaTeX version 2005/12/01 or later.

This work has the LPPL maintenance status "maintained".

The Current Maintainers of this work are
Heiko Oberdiek and the Oberdiek Package Support Group
https://github.com/ho-tex/oberdiek/issues


The Base Interpreter refers to any `TeX-Format',
because some files are installed in TDS:tex/generic//.

This work consists of the main source file pdfcol.dtx
and the derived files
   pdfcol.sty, pdfcol.pdf, pdfcol.ins, pdfcol.drv, pdfcol-test1.tex,
   pdfcol-test2.tex, pdfcol-test3.tex, pdfcol-test4.tex.

\endpreamble
\let\MetaPrefix\DoubleperCent

\generate{%
  \file{pdfcol.ins}{\from{pdfcol.dtx}{install}}%
  \file{pdfcol.drv}{\from{pdfcol.dtx}{driver}}%
  \usedir{tex/generic/oberdiek}%
  \file{pdfcol.sty}{\from{pdfcol.dtx}{package}}%
%  \usedir{doc/latex/oberdiek/test}%
%  \file{pdfcol-test1.tex}{\from{pdfcol.dtx}{test1}}%
%  \file{pdfcol-test2.tex}{\from{pdfcol.dtx}{test2}}%
%  \file{pdfcol-test3.tex}{\from{pdfcol.dtx}{test3}}%
%  \file{pdfcol-test4.tex}{\from{pdfcol.dtx}{test4}}%
}

\catcode32=13\relax% active space
\let =\space%
\Msg{************************************************************************}
\Msg{*}
\Msg{* To finish the installation you have to move the following}
\Msg{* file into a directory searched by TeX:}
\Msg{*}
\Msg{*     pdfcol.sty}
\Msg{*}
\Msg{* To produce the documentation run the file `pdfcol.drv'}
\Msg{* through LaTeX.}
\Msg{*}
\Msg{* Happy TeXing!}
\Msg{*}
\Msg{************************************************************************}

\endbatchfile
%</install>
%<*ignore>
\fi
%</ignore>
%<*driver>
\NeedsTeXFormat{LaTeX2e}
\ProvidesFile{pdfcol.drv}%
  [2018/11/01 v1.5 Handle new color stacks for pdfTeX (HO)]%
\documentclass{ltxdoc}
\usepackage{holtxdoc}[2011/11/22]
\begin{document}
  \DocInput{pdfcol.dtx}%
\end{document}
%</driver>
% \fi
%
%
%
% \GetFileInfo{pdfcol.drv}
%
% \title{The \xpackage{pdfcol} package}
% \date{2018/11/01 v1.5}
% \author{Heiko Oberdiek\thanks
% {Please report any issues at \url{https://github.com/ho-tex/oberdiek/issues}}}
%
% \maketitle
%
% \begin{abstract}
% Since version 1.40 \pdfTeX\ supports color stacks.
% The driver file \xfile{pdftex.def} for package \xpackage{color}
% defines and uses a main color stack since version v0.04b.
% Package \xpackage{pdfcol} is intended for package writers.
% It defines macros for setting and maintaining new color stacks.
% \end{abstract}
%
% \tableofcontents
%
% \section{Documentation}
%
% Version 1.40 of \pdfTeX\ adds new primitives \cs{pdfcolorstackinit}
% and \cs{pdfcolorstack}. Now color stacks can be defined and used.
% A main color stack is maintained by the driver file \xfile{pdftex.def}
% similar to dvips or dvipdfm. However the number of color stacks
% is not limited to one in \pdfTeX. Thus further color problems
% can now be solved, such as footnotes across pages or text
% that is set in parallel columns (e.g. packages \xpackage{parallel}
% or \xpackage{parcolumn}). Unlike the main color stack,
% the support by additional color stacks cannot be done in
% a transparent manner.
%
% This package \xpackage{pdfcol} provides an easier interface to
% additional color stacks without the need to use the
% low level primitives.
%
% \subsection{Requirements}
% \label{sec:req}
%
% \begin{itemize}
% \item
%   \pdfTeX\ 1.40 or greater.
% \item
%   \pdfTeX in PDF mode. (I don't know a DVI driver that
%   support several color stacks.)
% \item
%   \xfile{pdftex.def} 2007/01/02 v0.04b.
% \end{itemize}
% Package \xpackage{pdfcol} checks the requirements and
% sets switch \cs{ifpdfcolAvailable} accordingly.
%
% \subsection{Interface}
%
% \begin{declcs}{ifpdfcolAvailable}
% \end{declcs}
% If the requirements of section \ref{sec:req} are met the
% switch \cs{ifpdfcolAvailable} behaves as \cs{iftrue}.
% Otherwise the other interface macros in this section will
% be disabled with a message. Also the first use of such a
% macro will print a message. The messages are print to
% the \xext{log} file only if \pdfTeX\ is not used in PDF mode.
%
% \begin{declcs}{pdfcolErrorNoStacks}
% \end{declcs}
% The first call of \cs{pdfcolErrorNoStacks} prints an error
% message, if color stacks are not available.
%
% \begin{declcs}{pdfcolInitStack} \M{name}
% \end{declcs}
% A new color stack is initialized by \cs{pdfcolInitStack}.
% The \meta{name} is used for indentifying the stack. It usually
% consists of letters and digits. (The name must survive a \cs{csname}.)
%
% The intension of the macro is the definition of an additional
% color stack. Thus the stack is not page bounded like the
% main color stack. Black (\texttt{0 g 0 G}) is used as initial
% color value. And colors are written with modifier \texttt{direct}
% that means without setting the current transfer matrix and changing
% the current point (see documentation of \pdfTeX\ for
% |\pdfliteral direct{...}|).
%
% \begin{declcs}{pdfcolIfStackExists} \M{name} \M{then} \M{else}
% \end{declcs}
% Macro \cs{pdfcolIfStackExists} checks whether color stack \meta{name}
% exists. In case of success argument \meta{then} is executed
% and \meta{else} otherwise.
%
% \begin{declcs}{pdfcolSwitchStack} \M{name}
% \end{declcs}
% Macro \cs{pdfcolSwitchStack} switches the color stack. The color macros
% of package \xpackage{color} (or \xpackage{xcolor}) now uses the
% new color stack with name \meta{name}.
%
% \begin{declcs}{pdfcolSetCurrentColor}
% \end{declcs}
% Macro \cs{pdfcolSetCurrentColor} replaces the topmost
% entry of the stack by the current color (\cs{current@color}).
%
% \begin{declcs}{pdfcolSetCurrent} \M{name}
% \end{declcs}
% Macro \cs{pdfcolSetCurrent} sets the color that is read in
% the top-most entry of color stack \meta{name}. If \meta{name}
% is empty, the default color stack is used.
%
% \StopEventually{
% }
%
% \section{Implementation}
%
%    \begin{macrocode}
%<*package>
%    \end{macrocode}
%
% \subsection{Reload check and package identification}
%    Reload check, especially if the package is not used with \LaTeX.
%    \begin{macrocode}
\begingroup\catcode61\catcode48\catcode32=10\relax%
  \catcode13=5 % ^^M
  \endlinechar=13 %
  \catcode35=6 % #
  \catcode39=12 % '
  \catcode44=12 % ,
  \catcode45=12 % -
  \catcode46=12 % .
  \catcode58=12 % :
  \catcode64=11 % @
  \catcode123=1 % {
  \catcode125=2 % }
  \expandafter\let\expandafter\x\csname ver@pdfcol.sty\endcsname
  \ifx\x\relax % plain-TeX, first loading
  \else
    \def\empty{}%
    \ifx\x\empty % LaTeX, first loading,
      % variable is initialized, but \ProvidesPackage not yet seen
    \else
      \expandafter\ifx\csname PackageInfo\endcsname\relax
        \def\x#1#2{%
          \immediate\write-1{Package #1 Info: #2.}%
        }%
      \else
        \def\x#1#2{\PackageInfo{#1}{#2, stopped}}%
      \fi
      \x{pdfcol}{The package is already loaded}%
      \aftergroup\endinput
    \fi
  \fi
\endgroup%
%    \end{macrocode}
%    Package identification:
%    \begin{macrocode}
\begingroup\catcode61\catcode48\catcode32=10\relax%
  \catcode13=5 % ^^M
  \endlinechar=13 %
  \catcode35=6 % #
  \catcode39=12 % '
  \catcode40=12 % (
  \catcode41=12 % )
  \catcode44=12 % ,
  \catcode45=12 % -
  \catcode46=12 % .
  \catcode47=12 % /
  \catcode58=12 % :
  \catcode64=11 % @
  \catcode91=12 % [
  \catcode93=12 % ]
  \catcode123=1 % {
  \catcode125=2 % }
  \expandafter\ifx\csname ProvidesPackage\endcsname\relax
    \def\x#1#2#3[#4]{\endgroup
      \immediate\write-1{Package: #3 #4}%
      \xdef#1{#4}%
    }%
  \else
    \def\x#1#2[#3]{\endgroup
      #2[{#3}]%
      \ifx#1\@undefined
        \xdef#1{#3}%
      \fi
      \ifx#1\relax
        \xdef#1{#3}%
      \fi
    }%
  \fi
\expandafter\x\csname ver@pdfcol.sty\endcsname
\ProvidesPackage{pdfcol}%
  [2018/11/01 v1.5 Handle new color stacks for pdfTeX (HO)]%
%    \end{macrocode}
%
% \subsection{Catcodes}
%
%    \begin{macrocode}
\begingroup\catcode61\catcode48\catcode32=10\relax%
  \catcode13=5 % ^^M
  \endlinechar=13 %
  \catcode123=1 % {
  \catcode125=2 % }
  \catcode64=11 % @
  \def\x{\endgroup
    \expandafter\edef\csname PDFCOL@AtEnd\endcsname{%
      \endlinechar=\the\endlinechar\relax
      \catcode13=\the\catcode13\relax
      \catcode32=\the\catcode32\relax
      \catcode35=\the\catcode35\relax
      \catcode61=\the\catcode61\relax
      \catcode64=\the\catcode64\relax
      \catcode123=\the\catcode123\relax
      \catcode125=\the\catcode125\relax
    }%
  }%
\x\catcode61\catcode48\catcode32=10\relax%
\catcode13=5 % ^^M
\endlinechar=13 %
\catcode35=6 % #
\catcode64=11 % @
\catcode123=1 % {
\catcode125=2 % }
\def\TMP@EnsureCode#1#2{%
  \edef\PDFCOL@AtEnd{%
    \PDFCOL@AtEnd
    \catcode#1=\the\catcode#1\relax
  }%
  \catcode#1=#2\relax
}
\TMP@EnsureCode{39}{12}% '
\TMP@EnsureCode{40}{12}% (
\TMP@EnsureCode{41}{12}% )
\TMP@EnsureCode{43}{12}% +
\TMP@EnsureCode{44}{12}% ,
\TMP@EnsureCode{46}{12}% .
\TMP@EnsureCode{47}{12}% /
\TMP@EnsureCode{91}{12}% [
\TMP@EnsureCode{93}{12}% ]
\TMP@EnsureCode{96}{12}% `
\edef\PDFCOL@AtEnd{\PDFCOL@AtEnd\noexpand\endinput}
%    \end{macrocode}
%
% \subsection{Check requirements}
%
%    \begin{macro}{\PDFCOL@RequirePackage}
%    \begin{macrocode}
\begingroup\expandafter\expandafter\expandafter\endgroup
\expandafter\ifx\csname RequirePackage\endcsname\relax
  \def\PDFCOL@RequirePackage#1[#2]{\input #1.sty\relax}%
\else
  \def\PDFCOL@RequirePackage#1[#2]{%
    \RequirePackage{#1}[{#2}]%
  }%
\fi
%    \end{macrocode}
%    \end{macro}
%
% LuaTeX Compatability
%    \begin{macrocode}
\ifx\pdfextension\@undefined\else
 \def\pdfcolorstackinit {\pdffeedback colorstackinit}
 \protected\def\pdfcolorstack     {\pdfextension colorstack}
\fi
%    \end{macrocode}
%
%    \begin{macrocode}
\PDFCOL@RequirePackage{ltxcmds}[2010/03/01]
%    \end{macrocode}
%
%    \begin{macro}{ifpdfcolAvailable}
%    \begin{macrocode}
\ltx@newif\ifpdfcolAvailable
\pdfcolAvailabletrue
%    \end{macrocode}
%    \end{macro}
%
% \subsubsection{Check package \xpackage{luacolor}}
%
%    \begin{macrocode}
\ltx@newif\ifPDFCOL@luacolor
\begingroup\expandafter\expandafter\expandafter\endgroup
\expandafter\ifx\csname ver@luacolor.sty\endcsname\relax
  \PDFCOL@luacolorfalse
\else
  \PDFCOL@luacolortrue
\fi
%    \end{macrocode}
%
% \subsubsection{Check PDF mode}
%
%    \begin{macrocode}
\PDFCOL@RequirePackage{infwarerr}[2007/09/09]
\PDFCOL@RequirePackage{ifpdf}[2007/09/09]
\ifcase\ifpdf\ifPDFCOL@luacolor 1\fi\else 1\fi0 %
  \def\PDFCOL@Message{%
    \@PackageWarningNoLine{pdfcol}%
  }%
\else
  \pdfcolAvailablefalse
  \def\PDFCOL@Message{%
    \@PackageInfoNoLine{pdfcol}%
  }%
  \PDFCOL@Message{%
    Interface disabled because of %
    \ifPDFCOL@luacolor
      package `luacolor'%
    \else
      missing PDF mode of pdfTeX%
    \fi
  }%
\fi
%    \end{macrocode}
%
% \subsubsection{Check version of \pdfTeX}
%
%    \begin{macrocode}
\ifpdfcolAvailable
  \begingroup\expandafter\expandafter\expandafter\endgroup
  \expandafter\ifx\csname pdfcolorstack\endcsname\relax
    \pdfcolAvailablefalse
    \PDFCOL@Message{%
      Interface disabled because of too old pdfTeX.\MessageBreak
      Required is version 1.40+ for \string\pdfcolorstack
    }%
  \fi
\fi
\ifpdfcolAvailable
  \begingroup\expandafter\expandafter\expandafter\endgroup
  \expandafter\ifx\csname pdfcolorstack\endcsname\relax
    \pdfcolAvailablefalse
    \PDFCOL@Message{%
      Interface disabled because of too old pdfTeX.\MessageBreak
      Required is version 1.40+ for \string\pdfcolorstackinit
    }%
  \fi
\fi
%    \end{macrocode}
%
% \subsubsection{Check \xfile{pdftex.def}}
%
%    \begin{macrocode}
\ifpdfcolAvailable
  \begingroup\expandafter\expandafter\expandafter\endgroup
  \expandafter\ifx\csname @pdfcolorstack\endcsname\relax
%    \end{macrocode}
%    Try to load package color if it is not yet loaded (\LaTeX\ case).
%    \begin{macrocode}
    \begingroup\expandafter\expandafter\expandafter\endgroup
    \expandafter\ifx\csname ver@color.sty\endcsname\relax
      \begingroup\expandafter\expandafter\expandafter\endgroup
      \expandafter\ifx\csname documentclass\endcsname\relax
      \else
        \RequirePackage[pdftex]{color}\relax
      \fi
    \fi
    \begingroup\expandafter\expandafter\expandafter\endgroup
    \expandafter\ifx\csname @pdfcolorstack\endcsname\relax
      \pdfcolAvailablefalse
      \PDFCOL@Message{%
        Interface disabled because `pdftex.def'\MessageBreak
        is not loaded or it is too old.\MessageBreak
        Required is version 0.04b or greater%
      }%
    \fi
  \fi
\fi
%    \end{macrocode}
%
%    \begin{macrocode}
\let\pdfcolAvailabletrue\relax
\let\pdfcolAvailablefalse\relax
%    \end{macrocode}
%
% \subsection{Enabled interface macros}
%
%    \begin{macrocode}
\ifpdfcolAvailable
%    \end{macrocode}
%
%    \begin{macro}{\pdfcolErrorNoStacks}
%    \begin{macrocode}
  \let\pdfcolErrorNoStacks\relax
%    \end{macrocode}
%    \end{macro}
%
%    \begin{macro}{\pdfcol@Value}
%    \begin{macrocode}
  \expandafter\ifx\csname pdfcol@Value\endcsname\relax
    \def\pdfcol@Value{0 g 0 G}%
  \fi
%    \end{macrocode}
%    \end{macro}
%
%    \begin{macro}{\pdfcol@LiteralModifier}
%    \begin{macrocode}
  \expandafter\ifx\csname pdfcol@LiteralModifier\endcsname\relax
    \def\pdfcol@LiteralModifier{direct}%
  \fi
%    \end{macrocode}
%    \end{macro}
%
%    \begin{macro}{\pdfcolInitStack}
%    \begin{macrocode}
  \def\pdfcolInitStack#1{%
    \expandafter\ifx\csname pdfcol@Stack@#1\endcsname\relax
      \global\expandafter\chardef\csname pdfcol@Stack@#1\endcsname=%
          \pdfcolorstackinit\pdfcol@LiteralModifier{\pdfcol@Value}%
          \relax
      \@PackageInfo{pdfcol}{%
        New color stack `#1' = \number\csname pdfcol@Stack@#1\endcsname
      }%
    \else
      \@PackageError{pdfcol}{%
        Stack `#1' is already defined%
      }\@ehc
    \fi
  }%
%    \end{macrocode}
%    \end{macro}
%
%    \begin{macro}{\pdfcolIfStackExists}
%    \begin{macrocode}
  \def\pdfcolIfStackExists#1{%
    \expandafter\ifx\csname pdfcol@Stack@#1\endcsname\relax
      \expandafter\@secondoftwo
    \else
      \expandafter\@firstoftwo
    \fi
  }%
%    \end{macrocode}
%    \end{macro}
%    \begin{macro}{\@firstoftwo}
%    \begin{macrocode}
  \expandafter\ifx\csname @firstoftwo\endcsname\relax
    \long\def\@firstoftwo#1#2{#1}%
  \fi
%    \end{macrocode}
%    \end{macro}
%    \begin{macro}{\@secondoftwo}
%    \begin{macrocode}
  \expandafter\ifx\csname @secondoftwo\endcsname\relax
    \long\def\@secondoftwo#1#2{#2}%
  \fi
%    \end{macrocode}
%    \end{macro}
%
%    \begin{macro}{\pdfcolSwitchStack}
%    \begin{macrocode}
  \def\pdfcolSwitchStack#1{%
    \pdfcolIfStackExists{#1}{%
      \expandafter\let\expandafter\@pdfcolorstack
                      \csname pdfcol@Stack@#1\endcsname
    }{%
      \pdfcol@ErrorNoStack{#1}%
    }%
  }%
%    \end{macrocode}
%    \end{macro}
%
%    \begin{macro}{\pdfcolSetCurrentColor}
%    \begin{macrocode}
  \def\pdfcolSetCurrentColor{%
    \pdfcolorstack\@pdfcolorstack set{\current@color}%
  }%
%    \end{macrocode}
%    \end{macro}
%
%    \begin{macro}{\pdfcolSetCurrent}
%    \begin{macrocode}
  \def\pdfcolSetCurrent#1{%
    \ifx\\#1\\%
      \pdfcolorstack\@pdfcolorstack current\relax
    \else
      \pdfcolIfStackExists{#1}{%
        \pdfcolorstack\csname pdfcol@Stack@#1\endcsname current\relax
      }{%
        \pdfcol@ErrorNoStack{#1}%
      }%
    \fi
  }%
%    \end{macrocode}
%    \end{macro}
%
%    \begin{macro}{\pdfcol@ErrorNoStack}
%    \begin{macrocode}
  \def\pdfcol@ErrorNoStack#1{%
    \@PackageError{pdfcol}{Stack `#1' does not exists}\@ehc
  }%
%    \end{macrocode}
%    \end{macro}
%
% \subsection{Disabled interface macros}
%
%    \begin{macrocode}
\else
%    \end{macrocode}
%
%    \begin{macro}{\pdfcolErrorNoStacks}
%    \begin{macrocode}
  \def\pdfcolErrorNoStacks{%
    \@PackageError{pdfcol}{%
      Color stacks are not available%
    }{%
      Update pdfTeX (1.40) and `pdftex.def' (0.04b) %
          if necessary.\MessageBreak
      Ensure that `pdftex.def' is loaded %
          (package `color' or `xcolor').\MessageBreak
      Further messages can be found in TeX's %
          protocol file `\jobname.log'.\MessageBreak
      \MessageBreak
      \@ehc
    }%
    \global\let\pdfcolErrorNoStacks\relax
  }%
%    \end{macrocode}
%    \end{macro}
%
%    \begin{macro}{\PDFCOL@Disabled}
%    \begin{macrocode}
  \def\PDFCOL@Disabled{%
    \PDFCOL@Message{%
      pdfTeX's color stacks are not available%
    }%
    \global\let\PDFCOL@Disabled\relax
  }%
%    \end{macrocode}
%    \end{macro}
%
%    \begin{macro}{\pdfcolInitStack}
%    \begin{macrocode}
  \def\pdfcolInitStack#1{%
    \PDFCOL@Disabled
  }%
%    \end{macrocode}
%    \end{macro}
%
%    \begin{macro}{\pdfcolIfStackExists}
%    \begin{macrocode}
  \long\def\pdfcolIfStackExists#1#2#3{#3}%
%    \end{macrocode}
%    \end{macro}
%
%    \begin{macro}{\pdfcolSwitchStack}
%    \begin{macrocode}
  \def\pdfcolSwitchStack#1{%
    \PDFCOL@Disabled
  }%
%    \end{macrocode}
%    \end{macro}
%
%    \begin{macro}{\pdfcolSetCurrentColor}
%    \begin{macrocode}
  \def\pdfcolSetCurrentColor{%
    \PDFCOL@Disabled
  }%
%    \end{macrocode}
%    \end{macro}
%
%    \begin{macro}{\pdfcolSetCurrent}
%    \begin{macrocode}
  \def\pdfcolSetCurrent#1{%
    \PDFCOL@Disabled
  }%
%    \end{macrocode}
%    \end{macro}
%    \begin{macrocode}
\fi
%    \end{macrocode}
%
%    \begin{macrocode}
\PDFCOL@AtEnd%
%</package>
%    \end{macrocode}
%% \section{Installation}
%
% \subsection{Download}
%
% \paragraph{Package.} This package is available on
% CTAN\footnote{\CTANpkg{pdfcol}}:
% \begin{description}
% \item[\CTAN{macros/latex/contrib/oberdiek/pdfcol.dtx}] The source file.
% \item[\CTAN{macros/latex/contrib/oberdiek/pdfcol.pdf}] Documentation.
% \end{description}
%
%
% \paragraph{Bundle.} All the packages of the bundle `oberdiek'
% are also available in a TDS compliant ZIP archive. There
% the packages are already unpacked and the documentation files
% are generated. The files and directories obey the TDS standard.
% \begin{description}
% \item[\CTANinstall{install/macros/latex/contrib/oberdiek.tds.zip}]
% \end{description}
% \emph{TDS} refers to the standard ``A Directory Structure
% for \TeX\ Files'' (\CTANpkg{tds}). Directories
% with \xfile{texmf} in their name are usually organized this way.
%
% \subsection{Bundle installation}
%
% \paragraph{Unpacking.} Unpack the \xfile{oberdiek.tds.zip} in the
% TDS tree (also known as \xfile{texmf} tree) of your choice.
% Example (linux):
% \begin{quote}
%   |unzip oberdiek.tds.zip -d ~/texmf|
% \end{quote}
%
% \subsection{Package installation}
%
% \paragraph{Unpacking.} The \xfile{.dtx} file is a self-extracting
% \docstrip\ archive. The files are extracted by running the
% \xfile{.dtx} through \plainTeX:
% \begin{quote}
%   \verb|tex pdfcol.dtx|
% \end{quote}
%
% \paragraph{TDS.} Now the different files must be moved into
% the different directories in your installation TDS tree
% (also known as \xfile{texmf} tree):
% \begin{quote}
% \def\t{^^A
% \begin{tabular}{@{}>{\ttfamily}l@{ $\rightarrow$ }>{\ttfamily}l@{}}
%   pdfcol.sty & tex/generic/oberdiek/pdfcol.sty\\
%   pdfcol.pdf & doc/latex/oberdiek/pdfcol.pdf\\
%   pdfcol.dtx & source/latex/oberdiek/pdfcol.dtx\\
% \end{tabular}^^A
% }^^A
% \sbox0{\t}^^A
% \ifdim\wd0>\linewidth
%   \begingroup
%     \advance\linewidth by\leftmargin
%     \advance\linewidth by\rightmargin
%   \edef\x{\endgroup
%     \def\noexpand\lw{\the\linewidth}^^A
%   }\x
%   \def\lwbox{^^A
%     \leavevmode
%     \hbox to \linewidth{^^A
%       \kern-\leftmargin\relax
%       \hss
%       \usebox0
%       \hss
%       \kern-\rightmargin\relax
%     }^^A
%   }^^A
%   \ifdim\wd0>\lw
%     \sbox0{\small\t}^^A
%     \ifdim\wd0>\linewidth
%       \ifdim\wd0>\lw
%         \sbox0{\footnotesize\t}^^A
%         \ifdim\wd0>\linewidth
%           \ifdim\wd0>\lw
%             \sbox0{\scriptsize\t}^^A
%             \ifdim\wd0>\linewidth
%               \ifdim\wd0>\lw
%                 \sbox0{\tiny\t}^^A
%                 \ifdim\wd0>\linewidth
%                   \lwbox
%                 \else
%                   \usebox0
%                 \fi
%               \else
%                 \lwbox
%               \fi
%             \else
%               \usebox0
%             \fi
%           \else
%             \lwbox
%           \fi
%         \else
%           \usebox0
%         \fi
%       \else
%         \lwbox
%       \fi
%     \else
%       \usebox0
%     \fi
%   \else
%     \lwbox
%   \fi
% \else
%   \usebox0
% \fi
% \end{quote}
% If you have a \xfile{docstrip.cfg} that configures and enables \docstrip's
% TDS installing feature, then some files can already be in the right
% place, see the documentation of \docstrip.
%
% \subsection{Refresh file name databases}
%
% If your \TeX~distribution
% (\TeX\,Live, \mikTeX, \dots) relies on file name databases, you must refresh
% these. For example, \TeX\,Live\ users run \verb|texhash| or
% \verb|mktexlsr|.
%
% \subsection{Some details for the interested}
%
% \paragraph{Unpacking with \LaTeX.}
% The \xfile{.dtx} chooses its action depending on the format:
% \begin{description}
% \item[\plainTeX:] Run \docstrip\ and extract the files.
% \item[\LaTeX:] Generate the documentation.
% \end{description}
% If you insist on using \LaTeX\ for \docstrip\ (really,
% \docstrip\ does not need \LaTeX), then inform the autodetect routine
% about your intention:
% \begin{quote}
%   \verb|latex \let\install=y\input{pdfcol.dtx}|
% \end{quote}
% Do not forget to quote the argument according to the demands
% of your shell.
%
% \paragraph{Generating the documentation.}
% You can use both the \xfile{.dtx} or the \xfile{.drv} to generate
% the documentation. The process can be configured by the
% configuration file \xfile{ltxdoc.cfg}. For instance, put this
% line into this file, if you want to have A4 as paper format:
% \begin{quote}
%   \verb|\PassOptionsToClass{a4paper}{article}|
% \end{quote}
% An example follows how to generate the
% documentation with pdf\LaTeX:
% \begin{quote}
%\begin{verbatim}
%pdflatex pdfcol.dtx
%makeindex -s gind.ist pdfcol.idx
%pdflatex pdfcol.dtx
%makeindex -s gind.ist pdfcol.idx
%pdflatex pdfcol.dtx
%\end{verbatim}
% \end{quote}
%
% \begin{History}
%   \begin{Version}{2007/09/09 v1.0}
%   \item
%     First version.
%   \end{Version}
%   \begin{Version}{2007/12/09 v1.1}
%   \item
%     \cs{pdfcolSetCurrentColor} added.
%   \end{Version}
%   \begin{Version}{2007/12/12 v1.2}
%   \item
%     Detection for package \xpackage{luacolor} added.
%   \end{Version}
%   \begin{Version}{2016/05/16 v1.3}
%   \item
%     Documentation updates.
%   \end{Version}
%   \begin{Version}{2016/05/17 v1.4}
%   \item
%     Use luatex85 package for new luatex compatibility
%   \end{Version}
%   \begin{Version}{2018/11/01 v1.5}
%   \item
%     Remove luatex85 dependency
%   \end{Version}
% \end{History}
%
% \PrintIndex
%
% \Finale
\endinput

%        (quote the arguments according to the demands of your shell)
%
% Documentation:
%    (a) If pdfcol.drv is present:
%           latex pdfcol.drv
%    (b) Without pdfcol.drv:
%           latex pdfcol.dtx; ...
%    The class ltxdoc loads the configuration file ltxdoc.cfg
%    if available. Here you can specify further options, e.g.
%    use A4 as paper format:
%       \PassOptionsToClass{a4paper}{article}
%
%    Programm calls to get the documentation (example):
%       pdflatex pdfcol.dtx
%       makeindex -s gind.ist pdfcol.idx
%       pdflatex pdfcol.dtx
%       makeindex -s gind.ist pdfcol.idx
%       pdflatex pdfcol.dtx
%
% Installation:
%    TDS:tex/generic/oberdiek/pdfcol.sty
%    TDS:doc/latex/oberdiek/pdfcol.pdf
%    TDS:source/latex/oberdiek/pdfcol.dtx
%
%<*ignore>
\begingroup
  \catcode123=1 %
  \catcode125=2 %
  \def\x{LaTeX2e}%
\expandafter\endgroup
\ifcase 0\ifx\install y1\fi\expandafter
         \ifx\csname processbatchFile\endcsname\relax\else1\fi
         \ifx\fmtname\x\else 1\fi\relax
\else\csname fi\endcsname
%</ignore>
%<*install>
\input docstrip.tex
\Msg{************************************************************************}
\Msg{* Installation}
\Msg{* Package: pdfcol 2018/11/01 v1.5 Handle new color stacks for pdfTeX (HO)}
\Msg{************************************************************************}

\keepsilent
\askforoverwritefalse

\let\MetaPrefix\relax
\preamble

This is a generated file.

Project: pdfcol
Version: 2018/11/01 v1.5

Copyright (C)
   2007 Heiko Oberdiek
   2016-2019 Oberdiek Package Support Group

This work may be distributed and/or modified under the
conditions of the LaTeX Project Public License, either
version 1.3c of this license or (at your option) any later
version. This version of this license is in
   https://www.latex-project.org/lppl/lppl-1-3c.txt
and the latest version of this license is in
   https://www.latex-project.org/lppl.txt
and version 1.3 or later is part of all distributions of
LaTeX version 2005/12/01 or later.

This work has the LPPL maintenance status "maintained".

The Current Maintainers of this work are
Heiko Oberdiek and the Oberdiek Package Support Group
https://github.com/ho-tex/oberdiek/issues


The Base Interpreter refers to any `TeX-Format',
because some files are installed in TDS:tex/generic//.

This work consists of the main source file pdfcol.dtx
and the derived files
   pdfcol.sty, pdfcol.pdf, pdfcol.ins, pdfcol.drv, pdfcol-test1.tex,
   pdfcol-test2.tex, pdfcol-test3.tex, pdfcol-test4.tex.

\endpreamble
\let\MetaPrefix\DoubleperCent

\generate{%
  \file{pdfcol.ins}{\from{pdfcol.dtx}{install}}%
  \file{pdfcol.drv}{\from{pdfcol.dtx}{driver}}%
  \usedir{tex/generic/oberdiek}%
  \file{pdfcol.sty}{\from{pdfcol.dtx}{package}}%
%  \usedir{doc/latex/oberdiek/test}%
%  \file{pdfcol-test1.tex}{\from{pdfcol.dtx}{test1}}%
%  \file{pdfcol-test2.tex}{\from{pdfcol.dtx}{test2}}%
%  \file{pdfcol-test3.tex}{\from{pdfcol.dtx}{test3}}%
%  \file{pdfcol-test4.tex}{\from{pdfcol.dtx}{test4}}%
}

\catcode32=13\relax% active space
\let =\space%
\Msg{************************************************************************}
\Msg{*}
\Msg{* To finish the installation you have to move the following}
\Msg{* file into a directory searched by TeX:}
\Msg{*}
\Msg{*     pdfcol.sty}
\Msg{*}
\Msg{* To produce the documentation run the file `pdfcol.drv'}
\Msg{* through LaTeX.}
\Msg{*}
\Msg{* Happy TeXing!}
\Msg{*}
\Msg{************************************************************************}

\endbatchfile
%</install>
%<*ignore>
\fi
%</ignore>
%<*driver>
\NeedsTeXFormat{LaTeX2e}
\ProvidesFile{pdfcol.drv}%
  [2018/11/01 v1.5 Handle new color stacks for pdfTeX (HO)]%
\documentclass{ltxdoc}
\usepackage{holtxdoc}[2011/11/22]
\begin{document}
  \DocInput{pdfcol.dtx}%
\end{document}
%</driver>
% \fi
%
%
%
% \GetFileInfo{pdfcol.drv}
%
% \title{The \xpackage{pdfcol} package}
% \date{2018/11/01 v1.5}
% \author{Heiko Oberdiek\thanks
% {Please report any issues at \url{https://github.com/ho-tex/oberdiek/issues}}}
%
% \maketitle
%
% \begin{abstract}
% Since version 1.40 \pdfTeX\ supports color stacks.
% The driver file \xfile{pdftex.def} for package \xpackage{color}
% defines and uses a main color stack since version v0.04b.
% Package \xpackage{pdfcol} is intended for package writers.
% It defines macros for setting and maintaining new color stacks.
% \end{abstract}
%
% \tableofcontents
%
% \section{Documentation}
%
% Version 1.40 of \pdfTeX\ adds new primitives \cs{pdfcolorstackinit}
% and \cs{pdfcolorstack}. Now color stacks can be defined and used.
% A main color stack is maintained by the driver file \xfile{pdftex.def}
% similar to dvips or dvipdfm. However the number of color stacks
% is not limited to one in \pdfTeX. Thus further color problems
% can now be solved, such as footnotes across pages or text
% that is set in parallel columns (e.g. packages \xpackage{parallel}
% or \xpackage{parcolumn}). Unlike the main color stack,
% the support by additional color stacks cannot be done in
% a transparent manner.
%
% This package \xpackage{pdfcol} provides an easier interface to
% additional color stacks without the need to use the
% low level primitives.
%
% \subsection{Requirements}
% \label{sec:req}
%
% \begin{itemize}
% \item
%   \pdfTeX\ 1.40 or greater.
% \item
%   \pdfTeX in PDF mode. (I don't know a DVI driver that
%   support several color stacks.)
% \item
%   \xfile{pdftex.def} 2007/01/02 v0.04b.
% \end{itemize}
% Package \xpackage{pdfcol} checks the requirements and
% sets switch \cs{ifpdfcolAvailable} accordingly.
%
% \subsection{Interface}
%
% \begin{declcs}{ifpdfcolAvailable}
% \end{declcs}
% If the requirements of section \ref{sec:req} are met the
% switch \cs{ifpdfcolAvailable} behaves as \cs{iftrue}.
% Otherwise the other interface macros in this section will
% be disabled with a message. Also the first use of such a
% macro will print a message. The messages are print to
% the \xext{log} file only if \pdfTeX\ is not used in PDF mode.
%
% \begin{declcs}{pdfcolErrorNoStacks}
% \end{declcs}
% The first call of \cs{pdfcolErrorNoStacks} prints an error
% message, if color stacks are not available.
%
% \begin{declcs}{pdfcolInitStack} \M{name}
% \end{declcs}
% A new color stack is initialized by \cs{pdfcolInitStack}.
% The \meta{name} is used for indentifying the stack. It usually
% consists of letters and digits. (The name must survive a \cs{csname}.)
%
% The intension of the macro is the definition of an additional
% color stack. Thus the stack is not page bounded like the
% main color stack. Black (\texttt{0 g 0 G}) is used as initial
% color value. And colors are written with modifier \texttt{direct}
% that means without setting the current transfer matrix and changing
% the current point (see documentation of \pdfTeX\ for
% |\pdfliteral direct{...}|).
%
% \begin{declcs}{pdfcolIfStackExists} \M{name} \M{then} \M{else}
% \end{declcs}
% Macro \cs{pdfcolIfStackExists} checks whether color stack \meta{name}
% exists. In case of success argument \meta{then} is executed
% and \meta{else} otherwise.
%
% \begin{declcs}{pdfcolSwitchStack} \M{name}
% \end{declcs}
% Macro \cs{pdfcolSwitchStack} switches the color stack. The color macros
% of package \xpackage{color} (or \xpackage{xcolor}) now uses the
% new color stack with name \meta{name}.
%
% \begin{declcs}{pdfcolSetCurrentColor}
% \end{declcs}
% Macro \cs{pdfcolSetCurrentColor} replaces the topmost
% entry of the stack by the current color (\cs{current@color}).
%
% \begin{declcs}{pdfcolSetCurrent} \M{name}
% \end{declcs}
% Macro \cs{pdfcolSetCurrent} sets the color that is read in
% the top-most entry of color stack \meta{name}. If \meta{name}
% is empty, the default color stack is used.
%
% \StopEventually{
% }
%
% \section{Implementation}
%
%    \begin{macrocode}
%<*package>
%    \end{macrocode}
%
% \subsection{Reload check and package identification}
%    Reload check, especially if the package is not used with \LaTeX.
%    \begin{macrocode}
\begingroup\catcode61\catcode48\catcode32=10\relax%
  \catcode13=5 % ^^M
  \endlinechar=13 %
  \catcode35=6 % #
  \catcode39=12 % '
  \catcode44=12 % ,
  \catcode45=12 % -
  \catcode46=12 % .
  \catcode58=12 % :
  \catcode64=11 % @
  \catcode123=1 % {
  \catcode125=2 % }
  \expandafter\let\expandafter\x\csname ver@pdfcol.sty\endcsname
  \ifx\x\relax % plain-TeX, first loading
  \else
    \def\empty{}%
    \ifx\x\empty % LaTeX, first loading,
      % variable is initialized, but \ProvidesPackage not yet seen
    \else
      \expandafter\ifx\csname PackageInfo\endcsname\relax
        \def\x#1#2{%
          \immediate\write-1{Package #1 Info: #2.}%
        }%
      \else
        \def\x#1#2{\PackageInfo{#1}{#2, stopped}}%
      \fi
      \x{pdfcol}{The package is already loaded}%
      \aftergroup\endinput
    \fi
  \fi
\endgroup%
%    \end{macrocode}
%    Package identification:
%    \begin{macrocode}
\begingroup\catcode61\catcode48\catcode32=10\relax%
  \catcode13=5 % ^^M
  \endlinechar=13 %
  \catcode35=6 % #
  \catcode39=12 % '
  \catcode40=12 % (
  \catcode41=12 % )
  \catcode44=12 % ,
  \catcode45=12 % -
  \catcode46=12 % .
  \catcode47=12 % /
  \catcode58=12 % :
  \catcode64=11 % @
  \catcode91=12 % [
  \catcode93=12 % ]
  \catcode123=1 % {
  \catcode125=2 % }
  \expandafter\ifx\csname ProvidesPackage\endcsname\relax
    \def\x#1#2#3[#4]{\endgroup
      \immediate\write-1{Package: #3 #4}%
      \xdef#1{#4}%
    }%
  \else
    \def\x#1#2[#3]{\endgroup
      #2[{#3}]%
      \ifx#1\@undefined
        \xdef#1{#3}%
      \fi
      \ifx#1\relax
        \xdef#1{#3}%
      \fi
    }%
  \fi
\expandafter\x\csname ver@pdfcol.sty\endcsname
\ProvidesPackage{pdfcol}%
  [2018/11/01 v1.5 Handle new color stacks for pdfTeX (HO)]%
%    \end{macrocode}
%
% \subsection{Catcodes}
%
%    \begin{macrocode}
\begingroup\catcode61\catcode48\catcode32=10\relax%
  \catcode13=5 % ^^M
  \endlinechar=13 %
  \catcode123=1 % {
  \catcode125=2 % }
  \catcode64=11 % @
  \def\x{\endgroup
    \expandafter\edef\csname PDFCOL@AtEnd\endcsname{%
      \endlinechar=\the\endlinechar\relax
      \catcode13=\the\catcode13\relax
      \catcode32=\the\catcode32\relax
      \catcode35=\the\catcode35\relax
      \catcode61=\the\catcode61\relax
      \catcode64=\the\catcode64\relax
      \catcode123=\the\catcode123\relax
      \catcode125=\the\catcode125\relax
    }%
  }%
\x\catcode61\catcode48\catcode32=10\relax%
\catcode13=5 % ^^M
\endlinechar=13 %
\catcode35=6 % #
\catcode64=11 % @
\catcode123=1 % {
\catcode125=2 % }
\def\TMP@EnsureCode#1#2{%
  \edef\PDFCOL@AtEnd{%
    \PDFCOL@AtEnd
    \catcode#1=\the\catcode#1\relax
  }%
  \catcode#1=#2\relax
}
\TMP@EnsureCode{39}{12}% '
\TMP@EnsureCode{40}{12}% (
\TMP@EnsureCode{41}{12}% )
\TMP@EnsureCode{43}{12}% +
\TMP@EnsureCode{44}{12}% ,
\TMP@EnsureCode{46}{12}% .
\TMP@EnsureCode{47}{12}% /
\TMP@EnsureCode{91}{12}% [
\TMP@EnsureCode{93}{12}% ]
\TMP@EnsureCode{96}{12}% `
\edef\PDFCOL@AtEnd{\PDFCOL@AtEnd\noexpand\endinput}
%    \end{macrocode}
%
% \subsection{Check requirements}
%
%    \begin{macro}{\PDFCOL@RequirePackage}
%    \begin{macrocode}
\begingroup\expandafter\expandafter\expandafter\endgroup
\expandafter\ifx\csname RequirePackage\endcsname\relax
  \def\PDFCOL@RequirePackage#1[#2]{\input #1.sty\relax}%
\else
  \def\PDFCOL@RequirePackage#1[#2]{%
    \RequirePackage{#1}[{#2}]%
  }%
\fi
%    \end{macrocode}
%    \end{macro}
%
% LuaTeX Compatability
%    \begin{macrocode}
\ifx\pdfextension\@undefined\else
 \def\pdfcolorstackinit {\pdffeedback colorstackinit}
 \protected\def\pdfcolorstack     {\pdfextension colorstack}
\fi
%    \end{macrocode}
%
%    \begin{macrocode}
\PDFCOL@RequirePackage{ltxcmds}[2010/03/01]
%    \end{macrocode}
%
%    \begin{macro}{ifpdfcolAvailable}
%    \begin{macrocode}
\ltx@newif\ifpdfcolAvailable
\pdfcolAvailabletrue
%    \end{macrocode}
%    \end{macro}
%
% \subsubsection{Check package \xpackage{luacolor}}
%
%    \begin{macrocode}
\ltx@newif\ifPDFCOL@luacolor
\begingroup\expandafter\expandafter\expandafter\endgroup
\expandafter\ifx\csname ver@luacolor.sty\endcsname\relax
  \PDFCOL@luacolorfalse
\else
  \PDFCOL@luacolortrue
\fi
%    \end{macrocode}
%
% \subsubsection{Check PDF mode}
%
%    \begin{macrocode}
\PDFCOL@RequirePackage{infwarerr}[2007/09/09]
\PDFCOL@RequirePackage{ifpdf}[2007/09/09]
\ifcase\ifpdf\ifPDFCOL@luacolor 1\fi\else 1\fi0 %
  \def\PDFCOL@Message{%
    \@PackageWarningNoLine{pdfcol}%
  }%
\else
  \pdfcolAvailablefalse
  \def\PDFCOL@Message{%
    \@PackageInfoNoLine{pdfcol}%
  }%
  \PDFCOL@Message{%
    Interface disabled because of %
    \ifPDFCOL@luacolor
      package `luacolor'%
    \else
      missing PDF mode of pdfTeX%
    \fi
  }%
\fi
%    \end{macrocode}
%
% \subsubsection{Check version of \pdfTeX}
%
%    \begin{macrocode}
\ifpdfcolAvailable
  \begingroup\expandafter\expandafter\expandafter\endgroup
  \expandafter\ifx\csname pdfcolorstack\endcsname\relax
    \pdfcolAvailablefalse
    \PDFCOL@Message{%
      Interface disabled because of too old pdfTeX.\MessageBreak
      Required is version 1.40+ for \string\pdfcolorstack
    }%
  \fi
\fi
\ifpdfcolAvailable
  \begingroup\expandafter\expandafter\expandafter\endgroup
  \expandafter\ifx\csname pdfcolorstack\endcsname\relax
    \pdfcolAvailablefalse
    \PDFCOL@Message{%
      Interface disabled because of too old pdfTeX.\MessageBreak
      Required is version 1.40+ for \string\pdfcolorstackinit
    }%
  \fi
\fi
%    \end{macrocode}
%
% \subsubsection{Check \xfile{pdftex.def}}
%
%    \begin{macrocode}
\ifpdfcolAvailable
  \begingroup\expandafter\expandafter\expandafter\endgroup
  \expandafter\ifx\csname @pdfcolorstack\endcsname\relax
%    \end{macrocode}
%    Try to load package color if it is not yet loaded (\LaTeX\ case).
%    \begin{macrocode}
    \begingroup\expandafter\expandafter\expandafter\endgroup
    \expandafter\ifx\csname ver@color.sty\endcsname\relax
      \begingroup\expandafter\expandafter\expandafter\endgroup
      \expandafter\ifx\csname documentclass\endcsname\relax
      \else
        \RequirePackage[pdftex]{color}\relax
      \fi
    \fi
    \begingroup\expandafter\expandafter\expandafter\endgroup
    \expandafter\ifx\csname @pdfcolorstack\endcsname\relax
      \pdfcolAvailablefalse
      \PDFCOL@Message{%
        Interface disabled because `pdftex.def'\MessageBreak
        is not loaded or it is too old.\MessageBreak
        Required is version 0.04b or greater%
      }%
    \fi
  \fi
\fi
%    \end{macrocode}
%
%    \begin{macrocode}
\let\pdfcolAvailabletrue\relax
\let\pdfcolAvailablefalse\relax
%    \end{macrocode}
%
% \subsection{Enabled interface macros}
%
%    \begin{macrocode}
\ifpdfcolAvailable
%    \end{macrocode}
%
%    \begin{macro}{\pdfcolErrorNoStacks}
%    \begin{macrocode}
  \let\pdfcolErrorNoStacks\relax
%    \end{macrocode}
%    \end{macro}
%
%    \begin{macro}{\pdfcol@Value}
%    \begin{macrocode}
  \expandafter\ifx\csname pdfcol@Value\endcsname\relax
    \def\pdfcol@Value{0 g 0 G}%
  \fi
%    \end{macrocode}
%    \end{macro}
%
%    \begin{macro}{\pdfcol@LiteralModifier}
%    \begin{macrocode}
  \expandafter\ifx\csname pdfcol@LiteralModifier\endcsname\relax
    \def\pdfcol@LiteralModifier{direct}%
  \fi
%    \end{macrocode}
%    \end{macro}
%
%    \begin{macro}{\pdfcolInitStack}
%    \begin{macrocode}
  \def\pdfcolInitStack#1{%
    \expandafter\ifx\csname pdfcol@Stack@#1\endcsname\relax
      \global\expandafter\chardef\csname pdfcol@Stack@#1\endcsname=%
          \pdfcolorstackinit\pdfcol@LiteralModifier{\pdfcol@Value}%
          \relax
      \@PackageInfo{pdfcol}{%
        New color stack `#1' = \number\csname pdfcol@Stack@#1\endcsname
      }%
    \else
      \@PackageError{pdfcol}{%
        Stack `#1' is already defined%
      }\@ehc
    \fi
  }%
%    \end{macrocode}
%    \end{macro}
%
%    \begin{macro}{\pdfcolIfStackExists}
%    \begin{macrocode}
  \def\pdfcolIfStackExists#1{%
    \expandafter\ifx\csname pdfcol@Stack@#1\endcsname\relax
      \expandafter\@secondoftwo
    \else
      \expandafter\@firstoftwo
    \fi
  }%
%    \end{macrocode}
%    \end{macro}
%    \begin{macro}{\@firstoftwo}
%    \begin{macrocode}
  \expandafter\ifx\csname @firstoftwo\endcsname\relax
    \long\def\@firstoftwo#1#2{#1}%
  \fi
%    \end{macrocode}
%    \end{macro}
%    \begin{macro}{\@secondoftwo}
%    \begin{macrocode}
  \expandafter\ifx\csname @secondoftwo\endcsname\relax
    \long\def\@secondoftwo#1#2{#2}%
  \fi
%    \end{macrocode}
%    \end{macro}
%
%    \begin{macro}{\pdfcolSwitchStack}
%    \begin{macrocode}
  \def\pdfcolSwitchStack#1{%
    \pdfcolIfStackExists{#1}{%
      \expandafter\let\expandafter\@pdfcolorstack
                      \csname pdfcol@Stack@#1\endcsname
    }{%
      \pdfcol@ErrorNoStack{#1}%
    }%
  }%
%    \end{macrocode}
%    \end{macro}
%
%    \begin{macro}{\pdfcolSetCurrentColor}
%    \begin{macrocode}
  \def\pdfcolSetCurrentColor{%
    \pdfcolorstack\@pdfcolorstack set{\current@color}%
  }%
%    \end{macrocode}
%    \end{macro}
%
%    \begin{macro}{\pdfcolSetCurrent}
%    \begin{macrocode}
  \def\pdfcolSetCurrent#1{%
    \ifx\\#1\\%
      \pdfcolorstack\@pdfcolorstack current\relax
    \else
      \pdfcolIfStackExists{#1}{%
        \pdfcolorstack\csname pdfcol@Stack@#1\endcsname current\relax
      }{%
        \pdfcol@ErrorNoStack{#1}%
      }%
    \fi
  }%
%    \end{macrocode}
%    \end{macro}
%
%    \begin{macro}{\pdfcol@ErrorNoStack}
%    \begin{macrocode}
  \def\pdfcol@ErrorNoStack#1{%
    \@PackageError{pdfcol}{Stack `#1' does not exists}\@ehc
  }%
%    \end{macrocode}
%    \end{macro}
%
% \subsection{Disabled interface macros}
%
%    \begin{macrocode}
\else
%    \end{macrocode}
%
%    \begin{macro}{\pdfcolErrorNoStacks}
%    \begin{macrocode}
  \def\pdfcolErrorNoStacks{%
    \@PackageError{pdfcol}{%
      Color stacks are not available%
    }{%
      Update pdfTeX (1.40) and `pdftex.def' (0.04b) %
          if necessary.\MessageBreak
      Ensure that `pdftex.def' is loaded %
          (package `color' or `xcolor').\MessageBreak
      Further messages can be found in TeX's %
          protocol file `\jobname.log'.\MessageBreak
      \MessageBreak
      \@ehc
    }%
    \global\let\pdfcolErrorNoStacks\relax
  }%
%    \end{macrocode}
%    \end{macro}
%
%    \begin{macro}{\PDFCOL@Disabled}
%    \begin{macrocode}
  \def\PDFCOL@Disabled{%
    \PDFCOL@Message{%
      pdfTeX's color stacks are not available%
    }%
    \global\let\PDFCOL@Disabled\relax
  }%
%    \end{macrocode}
%    \end{macro}
%
%    \begin{macro}{\pdfcolInitStack}
%    \begin{macrocode}
  \def\pdfcolInitStack#1{%
    \PDFCOL@Disabled
  }%
%    \end{macrocode}
%    \end{macro}
%
%    \begin{macro}{\pdfcolIfStackExists}
%    \begin{macrocode}
  \long\def\pdfcolIfStackExists#1#2#3{#3}%
%    \end{macrocode}
%    \end{macro}
%
%    \begin{macro}{\pdfcolSwitchStack}
%    \begin{macrocode}
  \def\pdfcolSwitchStack#1{%
    \PDFCOL@Disabled
  }%
%    \end{macrocode}
%    \end{macro}
%
%    \begin{macro}{\pdfcolSetCurrentColor}
%    \begin{macrocode}
  \def\pdfcolSetCurrentColor{%
    \PDFCOL@Disabled
  }%
%    \end{macrocode}
%    \end{macro}
%
%    \begin{macro}{\pdfcolSetCurrent}
%    \begin{macrocode}
  \def\pdfcolSetCurrent#1{%
    \PDFCOL@Disabled
  }%
%    \end{macrocode}
%    \end{macro}
%    \begin{macrocode}
\fi
%    \end{macrocode}
%
%    \begin{macrocode}
\PDFCOL@AtEnd%
%</package>
%    \end{macrocode}
%% \section{Installation}
%
% \subsection{Download}
%
% \paragraph{Package.} This package is available on
% CTAN\footnote{\CTANpkg{pdfcol}}:
% \begin{description}
% \item[\CTAN{macros/latex/contrib/oberdiek/pdfcol.dtx}] The source file.
% \item[\CTAN{macros/latex/contrib/oberdiek/pdfcol.pdf}] Documentation.
% \end{description}
%
%
% \paragraph{Bundle.} All the packages of the bundle `oberdiek'
% are also available in a TDS compliant ZIP archive. There
% the packages are already unpacked and the documentation files
% are generated. The files and directories obey the TDS standard.
% \begin{description}
% \item[\CTANinstall{install/macros/latex/contrib/oberdiek.tds.zip}]
% \end{description}
% \emph{TDS} refers to the standard ``A Directory Structure
% for \TeX\ Files'' (\CTANpkg{tds}). Directories
% with \xfile{texmf} in their name are usually organized this way.
%
% \subsection{Bundle installation}
%
% \paragraph{Unpacking.} Unpack the \xfile{oberdiek.tds.zip} in the
% TDS tree (also known as \xfile{texmf} tree) of your choice.
% Example (linux):
% \begin{quote}
%   |unzip oberdiek.tds.zip -d ~/texmf|
% \end{quote}
%
% \subsection{Package installation}
%
% \paragraph{Unpacking.} The \xfile{.dtx} file is a self-extracting
% \docstrip\ archive. The files are extracted by running the
% \xfile{.dtx} through \plainTeX:
% \begin{quote}
%   \verb|tex pdfcol.dtx|
% \end{quote}
%
% \paragraph{TDS.} Now the different files must be moved into
% the different directories in your installation TDS tree
% (also known as \xfile{texmf} tree):
% \begin{quote}
% \def\t{^^A
% \begin{tabular}{@{}>{\ttfamily}l@{ $\rightarrow$ }>{\ttfamily}l@{}}
%   pdfcol.sty & tex/generic/oberdiek/pdfcol.sty\\
%   pdfcol.pdf & doc/latex/oberdiek/pdfcol.pdf\\
%   pdfcol.dtx & source/latex/oberdiek/pdfcol.dtx\\
% \end{tabular}^^A
% }^^A
% \sbox0{\t}^^A
% \ifdim\wd0>\linewidth
%   \begingroup
%     \advance\linewidth by\leftmargin
%     \advance\linewidth by\rightmargin
%   \edef\x{\endgroup
%     \def\noexpand\lw{\the\linewidth}^^A
%   }\x
%   \def\lwbox{^^A
%     \leavevmode
%     \hbox to \linewidth{^^A
%       \kern-\leftmargin\relax
%       \hss
%       \usebox0
%       \hss
%       \kern-\rightmargin\relax
%     }^^A
%   }^^A
%   \ifdim\wd0>\lw
%     \sbox0{\small\t}^^A
%     \ifdim\wd0>\linewidth
%       \ifdim\wd0>\lw
%         \sbox0{\footnotesize\t}^^A
%         \ifdim\wd0>\linewidth
%           \ifdim\wd0>\lw
%             \sbox0{\scriptsize\t}^^A
%             \ifdim\wd0>\linewidth
%               \ifdim\wd0>\lw
%                 \sbox0{\tiny\t}^^A
%                 \ifdim\wd0>\linewidth
%                   \lwbox
%                 \else
%                   \usebox0
%                 \fi
%               \else
%                 \lwbox
%               \fi
%             \else
%               \usebox0
%             \fi
%           \else
%             \lwbox
%           \fi
%         \else
%           \usebox0
%         \fi
%       \else
%         \lwbox
%       \fi
%     \else
%       \usebox0
%     \fi
%   \else
%     \lwbox
%   \fi
% \else
%   \usebox0
% \fi
% \end{quote}
% If you have a \xfile{docstrip.cfg} that configures and enables \docstrip's
% TDS installing feature, then some files can already be in the right
% place, see the documentation of \docstrip.
%
% \subsection{Refresh file name databases}
%
% If your \TeX~distribution
% (\TeX\,Live, \mikTeX, \dots) relies on file name databases, you must refresh
% these. For example, \TeX\,Live\ users run \verb|texhash| or
% \verb|mktexlsr|.
%
% \subsection{Some details for the interested}
%
% \paragraph{Unpacking with \LaTeX.}
% The \xfile{.dtx} chooses its action depending on the format:
% \begin{description}
% \item[\plainTeX:] Run \docstrip\ and extract the files.
% \item[\LaTeX:] Generate the documentation.
% \end{description}
% If you insist on using \LaTeX\ for \docstrip\ (really,
% \docstrip\ does not need \LaTeX), then inform the autodetect routine
% about your intention:
% \begin{quote}
%   \verb|latex \let\install=y% \iffalse meta-comment
%
% File: pdfcol.dtx
% Version: 2018/11/01 v1.5
% Info: Handle new color stacks for pdfTeX
%
% Copyright (C)
%    2007 Heiko Oberdiek
%    2016-2019 Oberdiek Package Support Group
%    https://github.com/ho-tex/oberdiek/issues
%
% This work may be distributed and/or modified under the
% conditions of the LaTeX Project Public License, either
% version 1.3c of this license or (at your option) any later
% version. This version of this license is in
%    https://www.latex-project.org/lppl/lppl-1-3c.txt
% and the latest version of this license is in
%    https://www.latex-project.org/lppl.txt
% and version 1.3 or later is part of all distributions of
% LaTeX version 2005/12/01 or later.
%
% This work has the LPPL maintenance status "maintained".
%
% The Current Maintainers of this work are
% Heiko Oberdiek and the Oberdiek Package Support Group
% https://github.com/ho-tex/oberdiek/issues
%
% The Base Interpreter refers to any `TeX-Format',
% because some files are installed in TDS:tex/generic//.
%
% This work consists of the main source file pdfcol.dtx
% and the derived files
%    pdfcol.sty, pdfcol.pdf, pdfcol.ins, pdfcol.drv, pdfcol-test1.tex,
%    pdfcol-test2.tex, pdfcol-test3.tex, pdfcol-test4.tex.
%
% Distribution:
%    CTAN:macros/latex/contrib/oberdiek/pdfcol.dtx
%    CTAN:macros/latex/contrib/oberdiek/pdfcol.pdf
%
% Unpacking:
%    (a) If pdfcol.ins is present:
%           tex pdfcol.ins
%    (b) Without pdfcol.ins:
%           tex pdfcol.dtx
%    (c) If you insist on using LaTeX
%           latex \let\install=y\input{pdfcol.dtx}
%        (quote the arguments according to the demands of your shell)
%
% Documentation:
%    (a) If pdfcol.drv is present:
%           latex pdfcol.drv
%    (b) Without pdfcol.drv:
%           latex pdfcol.dtx; ...
%    The class ltxdoc loads the configuration file ltxdoc.cfg
%    if available. Here you can specify further options, e.g.
%    use A4 as paper format:
%       \PassOptionsToClass{a4paper}{article}
%
%    Programm calls to get the documentation (example):
%       pdflatex pdfcol.dtx
%       makeindex -s gind.ist pdfcol.idx
%       pdflatex pdfcol.dtx
%       makeindex -s gind.ist pdfcol.idx
%       pdflatex pdfcol.dtx
%
% Installation:
%    TDS:tex/generic/oberdiek/pdfcol.sty
%    TDS:doc/latex/oberdiek/pdfcol.pdf
%    TDS:source/latex/oberdiek/pdfcol.dtx
%
%<*ignore>
\begingroup
  \catcode123=1 %
  \catcode125=2 %
  \def\x{LaTeX2e}%
\expandafter\endgroup
\ifcase 0\ifx\install y1\fi\expandafter
         \ifx\csname processbatchFile\endcsname\relax\else1\fi
         \ifx\fmtname\x\else 1\fi\relax
\else\csname fi\endcsname
%</ignore>
%<*install>
\input docstrip.tex
\Msg{************************************************************************}
\Msg{* Installation}
\Msg{* Package: pdfcol 2018/11/01 v1.5 Handle new color stacks for pdfTeX (HO)}
\Msg{************************************************************************}

\keepsilent
\askforoverwritefalse

\let\MetaPrefix\relax
\preamble

This is a generated file.

Project: pdfcol
Version: 2018/11/01 v1.5

Copyright (C)
   2007 Heiko Oberdiek
   2016-2019 Oberdiek Package Support Group

This work may be distributed and/or modified under the
conditions of the LaTeX Project Public License, either
version 1.3c of this license or (at your option) any later
version. This version of this license is in
   https://www.latex-project.org/lppl/lppl-1-3c.txt
and the latest version of this license is in
   https://www.latex-project.org/lppl.txt
and version 1.3 or later is part of all distributions of
LaTeX version 2005/12/01 or later.

This work has the LPPL maintenance status "maintained".

The Current Maintainers of this work are
Heiko Oberdiek and the Oberdiek Package Support Group
https://github.com/ho-tex/oberdiek/issues


The Base Interpreter refers to any `TeX-Format',
because some files are installed in TDS:tex/generic//.

This work consists of the main source file pdfcol.dtx
and the derived files
   pdfcol.sty, pdfcol.pdf, pdfcol.ins, pdfcol.drv, pdfcol-test1.tex,
   pdfcol-test2.tex, pdfcol-test3.tex, pdfcol-test4.tex.

\endpreamble
\let\MetaPrefix\DoubleperCent

\generate{%
  \file{pdfcol.ins}{\from{pdfcol.dtx}{install}}%
  \file{pdfcol.drv}{\from{pdfcol.dtx}{driver}}%
  \usedir{tex/generic/oberdiek}%
  \file{pdfcol.sty}{\from{pdfcol.dtx}{package}}%
%  \usedir{doc/latex/oberdiek/test}%
%  \file{pdfcol-test1.tex}{\from{pdfcol.dtx}{test1}}%
%  \file{pdfcol-test2.tex}{\from{pdfcol.dtx}{test2}}%
%  \file{pdfcol-test3.tex}{\from{pdfcol.dtx}{test3}}%
%  \file{pdfcol-test4.tex}{\from{pdfcol.dtx}{test4}}%
}

\catcode32=13\relax% active space
\let =\space%
\Msg{************************************************************************}
\Msg{*}
\Msg{* To finish the installation you have to move the following}
\Msg{* file into a directory searched by TeX:}
\Msg{*}
\Msg{*     pdfcol.sty}
\Msg{*}
\Msg{* To produce the documentation run the file `pdfcol.drv'}
\Msg{* through LaTeX.}
\Msg{*}
\Msg{* Happy TeXing!}
\Msg{*}
\Msg{************************************************************************}

\endbatchfile
%</install>
%<*ignore>
\fi
%</ignore>
%<*driver>
\NeedsTeXFormat{LaTeX2e}
\ProvidesFile{pdfcol.drv}%
  [2018/11/01 v1.5 Handle new color stacks for pdfTeX (HO)]%
\documentclass{ltxdoc}
\usepackage{holtxdoc}[2011/11/22]
\begin{document}
  \DocInput{pdfcol.dtx}%
\end{document}
%</driver>
% \fi
%
%
%
% \GetFileInfo{pdfcol.drv}
%
% \title{The \xpackage{pdfcol} package}
% \date{2018/11/01 v1.5}
% \author{Heiko Oberdiek\thanks
% {Please report any issues at \url{https://github.com/ho-tex/oberdiek/issues}}}
%
% \maketitle
%
% \begin{abstract}
% Since version 1.40 \pdfTeX\ supports color stacks.
% The driver file \xfile{pdftex.def} for package \xpackage{color}
% defines and uses a main color stack since version v0.04b.
% Package \xpackage{pdfcol} is intended for package writers.
% It defines macros for setting and maintaining new color stacks.
% \end{abstract}
%
% \tableofcontents
%
% \section{Documentation}
%
% Version 1.40 of \pdfTeX\ adds new primitives \cs{pdfcolorstackinit}
% and \cs{pdfcolorstack}. Now color stacks can be defined and used.
% A main color stack is maintained by the driver file \xfile{pdftex.def}
% similar to dvips or dvipdfm. However the number of color stacks
% is not limited to one in \pdfTeX. Thus further color problems
% can now be solved, such as footnotes across pages or text
% that is set in parallel columns (e.g. packages \xpackage{parallel}
% or \xpackage{parcolumn}). Unlike the main color stack,
% the support by additional color stacks cannot be done in
% a transparent manner.
%
% This package \xpackage{pdfcol} provides an easier interface to
% additional color stacks without the need to use the
% low level primitives.
%
% \subsection{Requirements}
% \label{sec:req}
%
% \begin{itemize}
% \item
%   \pdfTeX\ 1.40 or greater.
% \item
%   \pdfTeX in PDF mode. (I don't know a DVI driver that
%   support several color stacks.)
% \item
%   \xfile{pdftex.def} 2007/01/02 v0.04b.
% \end{itemize}
% Package \xpackage{pdfcol} checks the requirements and
% sets switch \cs{ifpdfcolAvailable} accordingly.
%
% \subsection{Interface}
%
% \begin{declcs}{ifpdfcolAvailable}
% \end{declcs}
% If the requirements of section \ref{sec:req} are met the
% switch \cs{ifpdfcolAvailable} behaves as \cs{iftrue}.
% Otherwise the other interface macros in this section will
% be disabled with a message. Also the first use of such a
% macro will print a message. The messages are print to
% the \xext{log} file only if \pdfTeX\ is not used in PDF mode.
%
% \begin{declcs}{pdfcolErrorNoStacks}
% \end{declcs}
% The first call of \cs{pdfcolErrorNoStacks} prints an error
% message, if color stacks are not available.
%
% \begin{declcs}{pdfcolInitStack} \M{name}
% \end{declcs}
% A new color stack is initialized by \cs{pdfcolInitStack}.
% The \meta{name} is used for indentifying the stack. It usually
% consists of letters and digits. (The name must survive a \cs{csname}.)
%
% The intension of the macro is the definition of an additional
% color stack. Thus the stack is not page bounded like the
% main color stack. Black (\texttt{0 g 0 G}) is used as initial
% color value. And colors are written with modifier \texttt{direct}
% that means without setting the current transfer matrix and changing
% the current point (see documentation of \pdfTeX\ for
% |\pdfliteral direct{...}|).
%
% \begin{declcs}{pdfcolIfStackExists} \M{name} \M{then} \M{else}
% \end{declcs}
% Macro \cs{pdfcolIfStackExists} checks whether color stack \meta{name}
% exists. In case of success argument \meta{then} is executed
% and \meta{else} otherwise.
%
% \begin{declcs}{pdfcolSwitchStack} \M{name}
% \end{declcs}
% Macro \cs{pdfcolSwitchStack} switches the color stack. The color macros
% of package \xpackage{color} (or \xpackage{xcolor}) now uses the
% new color stack with name \meta{name}.
%
% \begin{declcs}{pdfcolSetCurrentColor}
% \end{declcs}
% Macro \cs{pdfcolSetCurrentColor} replaces the topmost
% entry of the stack by the current color (\cs{current@color}).
%
% \begin{declcs}{pdfcolSetCurrent} \M{name}
% \end{declcs}
% Macro \cs{pdfcolSetCurrent} sets the color that is read in
% the top-most entry of color stack \meta{name}. If \meta{name}
% is empty, the default color stack is used.
%
% \StopEventually{
% }
%
% \section{Implementation}
%
%    \begin{macrocode}
%<*package>
%    \end{macrocode}
%
% \subsection{Reload check and package identification}
%    Reload check, especially if the package is not used with \LaTeX.
%    \begin{macrocode}
\begingroup\catcode61\catcode48\catcode32=10\relax%
  \catcode13=5 % ^^M
  \endlinechar=13 %
  \catcode35=6 % #
  \catcode39=12 % '
  \catcode44=12 % ,
  \catcode45=12 % -
  \catcode46=12 % .
  \catcode58=12 % :
  \catcode64=11 % @
  \catcode123=1 % {
  \catcode125=2 % }
  \expandafter\let\expandafter\x\csname ver@pdfcol.sty\endcsname
  \ifx\x\relax % plain-TeX, first loading
  \else
    \def\empty{}%
    \ifx\x\empty % LaTeX, first loading,
      % variable is initialized, but \ProvidesPackage not yet seen
    \else
      \expandafter\ifx\csname PackageInfo\endcsname\relax
        \def\x#1#2{%
          \immediate\write-1{Package #1 Info: #2.}%
        }%
      \else
        \def\x#1#2{\PackageInfo{#1}{#2, stopped}}%
      \fi
      \x{pdfcol}{The package is already loaded}%
      \aftergroup\endinput
    \fi
  \fi
\endgroup%
%    \end{macrocode}
%    Package identification:
%    \begin{macrocode}
\begingroup\catcode61\catcode48\catcode32=10\relax%
  \catcode13=5 % ^^M
  \endlinechar=13 %
  \catcode35=6 % #
  \catcode39=12 % '
  \catcode40=12 % (
  \catcode41=12 % )
  \catcode44=12 % ,
  \catcode45=12 % -
  \catcode46=12 % .
  \catcode47=12 % /
  \catcode58=12 % :
  \catcode64=11 % @
  \catcode91=12 % [
  \catcode93=12 % ]
  \catcode123=1 % {
  \catcode125=2 % }
  \expandafter\ifx\csname ProvidesPackage\endcsname\relax
    \def\x#1#2#3[#4]{\endgroup
      \immediate\write-1{Package: #3 #4}%
      \xdef#1{#4}%
    }%
  \else
    \def\x#1#2[#3]{\endgroup
      #2[{#3}]%
      \ifx#1\@undefined
        \xdef#1{#3}%
      \fi
      \ifx#1\relax
        \xdef#1{#3}%
      \fi
    }%
  \fi
\expandafter\x\csname ver@pdfcol.sty\endcsname
\ProvidesPackage{pdfcol}%
  [2018/11/01 v1.5 Handle new color stacks for pdfTeX (HO)]%
%    \end{macrocode}
%
% \subsection{Catcodes}
%
%    \begin{macrocode}
\begingroup\catcode61\catcode48\catcode32=10\relax%
  \catcode13=5 % ^^M
  \endlinechar=13 %
  \catcode123=1 % {
  \catcode125=2 % }
  \catcode64=11 % @
  \def\x{\endgroup
    \expandafter\edef\csname PDFCOL@AtEnd\endcsname{%
      \endlinechar=\the\endlinechar\relax
      \catcode13=\the\catcode13\relax
      \catcode32=\the\catcode32\relax
      \catcode35=\the\catcode35\relax
      \catcode61=\the\catcode61\relax
      \catcode64=\the\catcode64\relax
      \catcode123=\the\catcode123\relax
      \catcode125=\the\catcode125\relax
    }%
  }%
\x\catcode61\catcode48\catcode32=10\relax%
\catcode13=5 % ^^M
\endlinechar=13 %
\catcode35=6 % #
\catcode64=11 % @
\catcode123=1 % {
\catcode125=2 % }
\def\TMP@EnsureCode#1#2{%
  \edef\PDFCOL@AtEnd{%
    \PDFCOL@AtEnd
    \catcode#1=\the\catcode#1\relax
  }%
  \catcode#1=#2\relax
}
\TMP@EnsureCode{39}{12}% '
\TMP@EnsureCode{40}{12}% (
\TMP@EnsureCode{41}{12}% )
\TMP@EnsureCode{43}{12}% +
\TMP@EnsureCode{44}{12}% ,
\TMP@EnsureCode{46}{12}% .
\TMP@EnsureCode{47}{12}% /
\TMP@EnsureCode{91}{12}% [
\TMP@EnsureCode{93}{12}% ]
\TMP@EnsureCode{96}{12}% `
\edef\PDFCOL@AtEnd{\PDFCOL@AtEnd\noexpand\endinput}
%    \end{macrocode}
%
% \subsection{Check requirements}
%
%    \begin{macro}{\PDFCOL@RequirePackage}
%    \begin{macrocode}
\begingroup\expandafter\expandafter\expandafter\endgroup
\expandafter\ifx\csname RequirePackage\endcsname\relax
  \def\PDFCOL@RequirePackage#1[#2]{\input #1.sty\relax}%
\else
  \def\PDFCOL@RequirePackage#1[#2]{%
    \RequirePackage{#1}[{#2}]%
  }%
\fi
%    \end{macrocode}
%    \end{macro}
%
% LuaTeX Compatability
%    \begin{macrocode}
\ifx\pdfextension\@undefined\else
 \def\pdfcolorstackinit {\pdffeedback colorstackinit}
 \protected\def\pdfcolorstack     {\pdfextension colorstack}
\fi
%    \end{macrocode}
%
%    \begin{macrocode}
\PDFCOL@RequirePackage{ltxcmds}[2010/03/01]
%    \end{macrocode}
%
%    \begin{macro}{ifpdfcolAvailable}
%    \begin{macrocode}
\ltx@newif\ifpdfcolAvailable
\pdfcolAvailabletrue
%    \end{macrocode}
%    \end{macro}
%
% \subsubsection{Check package \xpackage{luacolor}}
%
%    \begin{macrocode}
\ltx@newif\ifPDFCOL@luacolor
\begingroup\expandafter\expandafter\expandafter\endgroup
\expandafter\ifx\csname ver@luacolor.sty\endcsname\relax
  \PDFCOL@luacolorfalse
\else
  \PDFCOL@luacolortrue
\fi
%    \end{macrocode}
%
% \subsubsection{Check PDF mode}
%
%    \begin{macrocode}
\PDFCOL@RequirePackage{infwarerr}[2007/09/09]
\PDFCOL@RequirePackage{ifpdf}[2007/09/09]
\ifcase\ifpdf\ifPDFCOL@luacolor 1\fi\else 1\fi0 %
  \def\PDFCOL@Message{%
    \@PackageWarningNoLine{pdfcol}%
  }%
\else
  \pdfcolAvailablefalse
  \def\PDFCOL@Message{%
    \@PackageInfoNoLine{pdfcol}%
  }%
  \PDFCOL@Message{%
    Interface disabled because of %
    \ifPDFCOL@luacolor
      package `luacolor'%
    \else
      missing PDF mode of pdfTeX%
    \fi
  }%
\fi
%    \end{macrocode}
%
% \subsubsection{Check version of \pdfTeX}
%
%    \begin{macrocode}
\ifpdfcolAvailable
  \begingroup\expandafter\expandafter\expandafter\endgroup
  \expandafter\ifx\csname pdfcolorstack\endcsname\relax
    \pdfcolAvailablefalse
    \PDFCOL@Message{%
      Interface disabled because of too old pdfTeX.\MessageBreak
      Required is version 1.40+ for \string\pdfcolorstack
    }%
  \fi
\fi
\ifpdfcolAvailable
  \begingroup\expandafter\expandafter\expandafter\endgroup
  \expandafter\ifx\csname pdfcolorstack\endcsname\relax
    \pdfcolAvailablefalse
    \PDFCOL@Message{%
      Interface disabled because of too old pdfTeX.\MessageBreak
      Required is version 1.40+ for \string\pdfcolorstackinit
    }%
  \fi
\fi
%    \end{macrocode}
%
% \subsubsection{Check \xfile{pdftex.def}}
%
%    \begin{macrocode}
\ifpdfcolAvailable
  \begingroup\expandafter\expandafter\expandafter\endgroup
  \expandafter\ifx\csname @pdfcolorstack\endcsname\relax
%    \end{macrocode}
%    Try to load package color if it is not yet loaded (\LaTeX\ case).
%    \begin{macrocode}
    \begingroup\expandafter\expandafter\expandafter\endgroup
    \expandafter\ifx\csname ver@color.sty\endcsname\relax
      \begingroup\expandafter\expandafter\expandafter\endgroup
      \expandafter\ifx\csname documentclass\endcsname\relax
      \else
        \RequirePackage[pdftex]{color}\relax
      \fi
    \fi
    \begingroup\expandafter\expandafter\expandafter\endgroup
    \expandafter\ifx\csname @pdfcolorstack\endcsname\relax
      \pdfcolAvailablefalse
      \PDFCOL@Message{%
        Interface disabled because `pdftex.def'\MessageBreak
        is not loaded or it is too old.\MessageBreak
        Required is version 0.04b or greater%
      }%
    \fi
  \fi
\fi
%    \end{macrocode}
%
%    \begin{macrocode}
\let\pdfcolAvailabletrue\relax
\let\pdfcolAvailablefalse\relax
%    \end{macrocode}
%
% \subsection{Enabled interface macros}
%
%    \begin{macrocode}
\ifpdfcolAvailable
%    \end{macrocode}
%
%    \begin{macro}{\pdfcolErrorNoStacks}
%    \begin{macrocode}
  \let\pdfcolErrorNoStacks\relax
%    \end{macrocode}
%    \end{macro}
%
%    \begin{macro}{\pdfcol@Value}
%    \begin{macrocode}
  \expandafter\ifx\csname pdfcol@Value\endcsname\relax
    \def\pdfcol@Value{0 g 0 G}%
  \fi
%    \end{macrocode}
%    \end{macro}
%
%    \begin{macro}{\pdfcol@LiteralModifier}
%    \begin{macrocode}
  \expandafter\ifx\csname pdfcol@LiteralModifier\endcsname\relax
    \def\pdfcol@LiteralModifier{direct}%
  \fi
%    \end{macrocode}
%    \end{macro}
%
%    \begin{macro}{\pdfcolInitStack}
%    \begin{macrocode}
  \def\pdfcolInitStack#1{%
    \expandafter\ifx\csname pdfcol@Stack@#1\endcsname\relax
      \global\expandafter\chardef\csname pdfcol@Stack@#1\endcsname=%
          \pdfcolorstackinit\pdfcol@LiteralModifier{\pdfcol@Value}%
          \relax
      \@PackageInfo{pdfcol}{%
        New color stack `#1' = \number\csname pdfcol@Stack@#1\endcsname
      }%
    \else
      \@PackageError{pdfcol}{%
        Stack `#1' is already defined%
      }\@ehc
    \fi
  }%
%    \end{macrocode}
%    \end{macro}
%
%    \begin{macro}{\pdfcolIfStackExists}
%    \begin{macrocode}
  \def\pdfcolIfStackExists#1{%
    \expandafter\ifx\csname pdfcol@Stack@#1\endcsname\relax
      \expandafter\@secondoftwo
    \else
      \expandafter\@firstoftwo
    \fi
  }%
%    \end{macrocode}
%    \end{macro}
%    \begin{macro}{\@firstoftwo}
%    \begin{macrocode}
  \expandafter\ifx\csname @firstoftwo\endcsname\relax
    \long\def\@firstoftwo#1#2{#1}%
  \fi
%    \end{macrocode}
%    \end{macro}
%    \begin{macro}{\@secondoftwo}
%    \begin{macrocode}
  \expandafter\ifx\csname @secondoftwo\endcsname\relax
    \long\def\@secondoftwo#1#2{#2}%
  \fi
%    \end{macrocode}
%    \end{macro}
%
%    \begin{macro}{\pdfcolSwitchStack}
%    \begin{macrocode}
  \def\pdfcolSwitchStack#1{%
    \pdfcolIfStackExists{#1}{%
      \expandafter\let\expandafter\@pdfcolorstack
                      \csname pdfcol@Stack@#1\endcsname
    }{%
      \pdfcol@ErrorNoStack{#1}%
    }%
  }%
%    \end{macrocode}
%    \end{macro}
%
%    \begin{macro}{\pdfcolSetCurrentColor}
%    \begin{macrocode}
  \def\pdfcolSetCurrentColor{%
    \pdfcolorstack\@pdfcolorstack set{\current@color}%
  }%
%    \end{macrocode}
%    \end{macro}
%
%    \begin{macro}{\pdfcolSetCurrent}
%    \begin{macrocode}
  \def\pdfcolSetCurrent#1{%
    \ifx\\#1\\%
      \pdfcolorstack\@pdfcolorstack current\relax
    \else
      \pdfcolIfStackExists{#1}{%
        \pdfcolorstack\csname pdfcol@Stack@#1\endcsname current\relax
      }{%
        \pdfcol@ErrorNoStack{#1}%
      }%
    \fi
  }%
%    \end{macrocode}
%    \end{macro}
%
%    \begin{macro}{\pdfcol@ErrorNoStack}
%    \begin{macrocode}
  \def\pdfcol@ErrorNoStack#1{%
    \@PackageError{pdfcol}{Stack `#1' does not exists}\@ehc
  }%
%    \end{macrocode}
%    \end{macro}
%
% \subsection{Disabled interface macros}
%
%    \begin{macrocode}
\else
%    \end{macrocode}
%
%    \begin{macro}{\pdfcolErrorNoStacks}
%    \begin{macrocode}
  \def\pdfcolErrorNoStacks{%
    \@PackageError{pdfcol}{%
      Color stacks are not available%
    }{%
      Update pdfTeX (1.40) and `pdftex.def' (0.04b) %
          if necessary.\MessageBreak
      Ensure that `pdftex.def' is loaded %
          (package `color' or `xcolor').\MessageBreak
      Further messages can be found in TeX's %
          protocol file `\jobname.log'.\MessageBreak
      \MessageBreak
      \@ehc
    }%
    \global\let\pdfcolErrorNoStacks\relax
  }%
%    \end{macrocode}
%    \end{macro}
%
%    \begin{macro}{\PDFCOL@Disabled}
%    \begin{macrocode}
  \def\PDFCOL@Disabled{%
    \PDFCOL@Message{%
      pdfTeX's color stacks are not available%
    }%
    \global\let\PDFCOL@Disabled\relax
  }%
%    \end{macrocode}
%    \end{macro}
%
%    \begin{macro}{\pdfcolInitStack}
%    \begin{macrocode}
  \def\pdfcolInitStack#1{%
    \PDFCOL@Disabled
  }%
%    \end{macrocode}
%    \end{macro}
%
%    \begin{macro}{\pdfcolIfStackExists}
%    \begin{macrocode}
  \long\def\pdfcolIfStackExists#1#2#3{#3}%
%    \end{macrocode}
%    \end{macro}
%
%    \begin{macro}{\pdfcolSwitchStack}
%    \begin{macrocode}
  \def\pdfcolSwitchStack#1{%
    \PDFCOL@Disabled
  }%
%    \end{macrocode}
%    \end{macro}
%
%    \begin{macro}{\pdfcolSetCurrentColor}
%    \begin{macrocode}
  \def\pdfcolSetCurrentColor{%
    \PDFCOL@Disabled
  }%
%    \end{macrocode}
%    \end{macro}
%
%    \begin{macro}{\pdfcolSetCurrent}
%    \begin{macrocode}
  \def\pdfcolSetCurrent#1{%
    \PDFCOL@Disabled
  }%
%    \end{macrocode}
%    \end{macro}
%    \begin{macrocode}
\fi
%    \end{macrocode}
%
%    \begin{macrocode}
\PDFCOL@AtEnd%
%</package>
%    \end{macrocode}
%% \section{Installation}
%
% \subsection{Download}
%
% \paragraph{Package.} This package is available on
% CTAN\footnote{\CTANpkg{pdfcol}}:
% \begin{description}
% \item[\CTAN{macros/latex/contrib/oberdiek/pdfcol.dtx}] The source file.
% \item[\CTAN{macros/latex/contrib/oberdiek/pdfcol.pdf}] Documentation.
% \end{description}
%
%
% \paragraph{Bundle.} All the packages of the bundle `oberdiek'
% are also available in a TDS compliant ZIP archive. There
% the packages are already unpacked and the documentation files
% are generated. The files and directories obey the TDS standard.
% \begin{description}
% \item[\CTANinstall{install/macros/latex/contrib/oberdiek.tds.zip}]
% \end{description}
% \emph{TDS} refers to the standard ``A Directory Structure
% for \TeX\ Files'' (\CTANpkg{tds}). Directories
% with \xfile{texmf} in their name are usually organized this way.
%
% \subsection{Bundle installation}
%
% \paragraph{Unpacking.} Unpack the \xfile{oberdiek.tds.zip} in the
% TDS tree (also known as \xfile{texmf} tree) of your choice.
% Example (linux):
% \begin{quote}
%   |unzip oberdiek.tds.zip -d ~/texmf|
% \end{quote}
%
% \subsection{Package installation}
%
% \paragraph{Unpacking.} The \xfile{.dtx} file is a self-extracting
% \docstrip\ archive. The files are extracted by running the
% \xfile{.dtx} through \plainTeX:
% \begin{quote}
%   \verb|tex pdfcol.dtx|
% \end{quote}
%
% \paragraph{TDS.} Now the different files must be moved into
% the different directories in your installation TDS tree
% (also known as \xfile{texmf} tree):
% \begin{quote}
% \def\t{^^A
% \begin{tabular}{@{}>{\ttfamily}l@{ $\rightarrow$ }>{\ttfamily}l@{}}
%   pdfcol.sty & tex/generic/oberdiek/pdfcol.sty\\
%   pdfcol.pdf & doc/latex/oberdiek/pdfcol.pdf\\
%   pdfcol.dtx & source/latex/oberdiek/pdfcol.dtx\\
% \end{tabular}^^A
% }^^A
% \sbox0{\t}^^A
% \ifdim\wd0>\linewidth
%   \begingroup
%     \advance\linewidth by\leftmargin
%     \advance\linewidth by\rightmargin
%   \edef\x{\endgroup
%     \def\noexpand\lw{\the\linewidth}^^A
%   }\x
%   \def\lwbox{^^A
%     \leavevmode
%     \hbox to \linewidth{^^A
%       \kern-\leftmargin\relax
%       \hss
%       \usebox0
%       \hss
%       \kern-\rightmargin\relax
%     }^^A
%   }^^A
%   \ifdim\wd0>\lw
%     \sbox0{\small\t}^^A
%     \ifdim\wd0>\linewidth
%       \ifdim\wd0>\lw
%         \sbox0{\footnotesize\t}^^A
%         \ifdim\wd0>\linewidth
%           \ifdim\wd0>\lw
%             \sbox0{\scriptsize\t}^^A
%             \ifdim\wd0>\linewidth
%               \ifdim\wd0>\lw
%                 \sbox0{\tiny\t}^^A
%                 \ifdim\wd0>\linewidth
%                   \lwbox
%                 \else
%                   \usebox0
%                 \fi
%               \else
%                 \lwbox
%               \fi
%             \else
%               \usebox0
%             \fi
%           \else
%             \lwbox
%           \fi
%         \else
%           \usebox0
%         \fi
%       \else
%         \lwbox
%       \fi
%     \else
%       \usebox0
%     \fi
%   \else
%     \lwbox
%   \fi
% \else
%   \usebox0
% \fi
% \end{quote}
% If you have a \xfile{docstrip.cfg} that configures and enables \docstrip's
% TDS installing feature, then some files can already be in the right
% place, see the documentation of \docstrip.
%
% \subsection{Refresh file name databases}
%
% If your \TeX~distribution
% (\TeX\,Live, \mikTeX, \dots) relies on file name databases, you must refresh
% these. For example, \TeX\,Live\ users run \verb|texhash| or
% \verb|mktexlsr|.
%
% \subsection{Some details for the interested}
%
% \paragraph{Unpacking with \LaTeX.}
% The \xfile{.dtx} chooses its action depending on the format:
% \begin{description}
% \item[\plainTeX:] Run \docstrip\ and extract the files.
% \item[\LaTeX:] Generate the documentation.
% \end{description}
% If you insist on using \LaTeX\ for \docstrip\ (really,
% \docstrip\ does not need \LaTeX), then inform the autodetect routine
% about your intention:
% \begin{quote}
%   \verb|latex \let\install=y\input{pdfcol.dtx}|
% \end{quote}
% Do not forget to quote the argument according to the demands
% of your shell.
%
% \paragraph{Generating the documentation.}
% You can use both the \xfile{.dtx} or the \xfile{.drv} to generate
% the documentation. The process can be configured by the
% configuration file \xfile{ltxdoc.cfg}. For instance, put this
% line into this file, if you want to have A4 as paper format:
% \begin{quote}
%   \verb|\PassOptionsToClass{a4paper}{article}|
% \end{quote}
% An example follows how to generate the
% documentation with pdf\LaTeX:
% \begin{quote}
%\begin{verbatim}
%pdflatex pdfcol.dtx
%makeindex -s gind.ist pdfcol.idx
%pdflatex pdfcol.dtx
%makeindex -s gind.ist pdfcol.idx
%pdflatex pdfcol.dtx
%\end{verbatim}
% \end{quote}
%
% \begin{History}
%   \begin{Version}{2007/09/09 v1.0}
%   \item
%     First version.
%   \end{Version}
%   \begin{Version}{2007/12/09 v1.1}
%   \item
%     \cs{pdfcolSetCurrentColor} added.
%   \end{Version}
%   \begin{Version}{2007/12/12 v1.2}
%   \item
%     Detection for package \xpackage{luacolor} added.
%   \end{Version}
%   \begin{Version}{2016/05/16 v1.3}
%   \item
%     Documentation updates.
%   \end{Version}
%   \begin{Version}{2016/05/17 v1.4}
%   \item
%     Use luatex85 package for new luatex compatibility
%   \end{Version}
%   \begin{Version}{2018/11/01 v1.5}
%   \item
%     Remove luatex85 dependency
%   \end{Version}
% \end{History}
%
% \PrintIndex
%
% \Finale
\endinput
|
% \end{quote}
% Do not forget to quote the argument according to the demands
% of your shell.
%
% \paragraph{Generating the documentation.}
% You can use both the \xfile{.dtx} or the \xfile{.drv} to generate
% the documentation. The process can be configured by the
% configuration file \xfile{ltxdoc.cfg}. For instance, put this
% line into this file, if you want to have A4 as paper format:
% \begin{quote}
%   \verb|\PassOptionsToClass{a4paper}{article}|
% \end{quote}
% An example follows how to generate the
% documentation with pdf\LaTeX:
% \begin{quote}
%\begin{verbatim}
%pdflatex pdfcol.dtx
%makeindex -s gind.ist pdfcol.idx
%pdflatex pdfcol.dtx
%makeindex -s gind.ist pdfcol.idx
%pdflatex pdfcol.dtx
%\end{verbatim}
% \end{quote}
%
% \begin{History}
%   \begin{Version}{2007/09/09 v1.0}
%   \item
%     First version.
%   \end{Version}
%   \begin{Version}{2007/12/09 v1.1}
%   \item
%     \cs{pdfcolSetCurrentColor} added.
%   \end{Version}
%   \begin{Version}{2007/12/12 v1.2}
%   \item
%     Detection for package \xpackage{luacolor} added.
%   \end{Version}
%   \begin{Version}{2016/05/16 v1.3}
%   \item
%     Documentation updates.
%   \end{Version}
%   \begin{Version}{2016/05/17 v1.4}
%   \item
%     Use luatex85 package for new luatex compatibility
%   \end{Version}
%   \begin{Version}{2018/11/01 v1.5}
%   \item
%     Remove luatex85 dependency
%   \end{Version}
% \end{History}
%
% \PrintIndex
%
% \Finale
\endinput
|
% \end{quote}
% Do not forget to quote the argument according to the demands
% of your shell.
%
% \paragraph{Generating the documentation.}
% You can use both the \xfile{.dtx} or the \xfile{.drv} to generate
% the documentation. The process can be configured by the
% configuration file \xfile{ltxdoc.cfg}. For instance, put this
% line into this file, if you want to have A4 as paper format:
% \begin{quote}
%   \verb|\PassOptionsToClass{a4paper}{article}|
% \end{quote}
% An example follows how to generate the
% documentation with pdf\LaTeX:
% \begin{quote}
%\begin{verbatim}
%pdflatex pdfcol.dtx
%makeindex -s gind.ist pdfcol.idx
%pdflatex pdfcol.dtx
%makeindex -s gind.ist pdfcol.idx
%pdflatex pdfcol.dtx
%\end{verbatim}
% \end{quote}
%
% \begin{History}
%   \begin{Version}{2007/09/09 v1.0}
%   \item
%     First version.
%   \end{Version}
%   \begin{Version}{2007/12/09 v1.1}
%   \item
%     \cs{pdfcolSetCurrentColor} added.
%   \end{Version}
%   \begin{Version}{2007/12/12 v1.2}
%   \item
%     Detection for package \xpackage{luacolor} added.
%   \end{Version}
%   \begin{Version}{2016/05/16 v1.3}
%   \item
%     Documentation updates.
%   \end{Version}
%   \begin{Version}{2016/05/17 v1.4}
%   \item
%     Use luatex85 package for new luatex compatibility
%   \end{Version}
%   \begin{Version}{2018/11/01 v1.5}
%   \item
%     Remove luatex85 dependency
%   \end{Version}
% \end{History}
%
% \PrintIndex
%
% \Finale
\endinput

%        (quote the arguments according to the demands of your shell)
%
% Documentation:
%    (a) If pdfcol.drv is present:
%           latex pdfcol.drv
%    (b) Without pdfcol.drv:
%           latex pdfcol.dtx; ...
%    The class ltxdoc loads the configuration file ltxdoc.cfg
%    if available. Here you can specify further options, e.g.
%    use A4 as paper format:
%       \PassOptionsToClass{a4paper}{article}
%
%    Programm calls to get the documentation (example):
%       pdflatex pdfcol.dtx
%       makeindex -s gind.ist pdfcol.idx
%       pdflatex pdfcol.dtx
%       makeindex -s gind.ist pdfcol.idx
%       pdflatex pdfcol.dtx
%
% Installation:
%    TDS:tex/generic/oberdiek/pdfcol.sty
%    TDS:doc/latex/oberdiek/pdfcol.pdf
%    TDS:source/latex/oberdiek/pdfcol.dtx
%
%<*ignore>
\begingroup
  \catcode123=1 %
  \catcode125=2 %
  \def\x{LaTeX2e}%
\expandafter\endgroup
\ifcase 0\ifx\install y1\fi\expandafter
         \ifx\csname processbatchFile\endcsname\relax\else1\fi
         \ifx\fmtname\x\else 1\fi\relax
\else\csname fi\endcsname
%</ignore>
%<*install>
\input docstrip.tex
\Msg{************************************************************************}
\Msg{* Installation}
\Msg{* Package: pdfcol 2018/11/01 v1.5 Handle new color stacks for pdfTeX (HO)}
\Msg{************************************************************************}

\keepsilent
\askforoverwritefalse

\let\MetaPrefix\relax
\preamble

This is a generated file.

Project: pdfcol
Version: 2018/11/01 v1.5

Copyright (C)
   2007 Heiko Oberdiek
   2016-2019 Oberdiek Package Support Group

This work may be distributed and/or modified under the
conditions of the LaTeX Project Public License, either
version 1.3c of this license or (at your option) any later
version. This version of this license is in
   https://www.latex-project.org/lppl/lppl-1-3c.txt
and the latest version of this license is in
   https://www.latex-project.org/lppl.txt
and version 1.3 or later is part of all distributions of
LaTeX version 2005/12/01 or later.

This work has the LPPL maintenance status "maintained".

The Current Maintainers of this work are
Heiko Oberdiek and the Oberdiek Package Support Group
https://github.com/ho-tex/oberdiek/issues


The Base Interpreter refers to any `TeX-Format',
because some files are installed in TDS:tex/generic//.

This work consists of the main source file pdfcol.dtx
and the derived files
   pdfcol.sty, pdfcol.pdf, pdfcol.ins, pdfcol.drv, pdfcol-test1.tex,
   pdfcol-test2.tex, pdfcol-test3.tex, pdfcol-test4.tex.

\endpreamble
\let\MetaPrefix\DoubleperCent

\generate{%
  \file{pdfcol.ins}{\from{pdfcol.dtx}{install}}%
  \file{pdfcol.drv}{\from{pdfcol.dtx}{driver}}%
  \usedir{tex/generic/oberdiek}%
  \file{pdfcol.sty}{\from{pdfcol.dtx}{package}}%
%  \usedir{doc/latex/oberdiek/test}%
%  \file{pdfcol-test1.tex}{\from{pdfcol.dtx}{test1}}%
%  \file{pdfcol-test2.tex}{\from{pdfcol.dtx}{test2}}%
%  \file{pdfcol-test3.tex}{\from{pdfcol.dtx}{test3}}%
%  \file{pdfcol-test4.tex}{\from{pdfcol.dtx}{test4}}%
}

\catcode32=13\relax% active space
\let =\space%
\Msg{************************************************************************}
\Msg{*}
\Msg{* To finish the installation you have to move the following}
\Msg{* file into a directory searched by TeX:}
\Msg{*}
\Msg{*     pdfcol.sty}
\Msg{*}
\Msg{* To produce the documentation run the file `pdfcol.drv'}
\Msg{* through LaTeX.}
\Msg{*}
\Msg{* Happy TeXing!}
\Msg{*}
\Msg{************************************************************************}

\endbatchfile
%</install>
%<*ignore>
\fi
%</ignore>
%<*driver>
\NeedsTeXFormat{LaTeX2e}
\ProvidesFile{pdfcol.drv}%
  [2018/11/01 v1.5 Handle new color stacks for pdfTeX (HO)]%
\documentclass{ltxdoc}
\usepackage{holtxdoc}[2011/11/22]
\begin{document}
  \DocInput{pdfcol.dtx}%
\end{document}
%</driver>
% \fi
%
%
%
% \GetFileInfo{pdfcol.drv}
%
% \title{The \xpackage{pdfcol} package}
% \date{2018/11/01 v1.5}
% \author{Heiko Oberdiek\thanks
% {Please report any issues at \url{https://github.com/ho-tex/oberdiek/issues}}}
%
% \maketitle
%
% \begin{abstract}
% Since version 1.40 \pdfTeX\ supports color stacks.
% The driver file \xfile{pdftex.def} for package \xpackage{color}
% defines and uses a main color stack since version v0.04b.
% Package \xpackage{pdfcol} is intended for package writers.
% It defines macros for setting and maintaining new color stacks.
% \end{abstract}
%
% \tableofcontents
%
% \section{Documentation}
%
% Version 1.40 of \pdfTeX\ adds new primitives \cs{pdfcolorstackinit}
% and \cs{pdfcolorstack}. Now color stacks can be defined and used.
% A main color stack is maintained by the driver file \xfile{pdftex.def}
% similar to dvips or dvipdfm. However the number of color stacks
% is not limited to one in \pdfTeX. Thus further color problems
% can now be solved, such as footnotes across pages or text
% that is set in parallel columns (e.g. packages \xpackage{parallel}
% or \xpackage{parcolumn}). Unlike the main color stack,
% the support by additional color stacks cannot be done in
% a transparent manner.
%
% This package \xpackage{pdfcol} provides an easier interface to
% additional color stacks without the need to use the
% low level primitives.
%
% \subsection{Requirements}
% \label{sec:req}
%
% \begin{itemize}
% \item
%   \pdfTeX\ 1.40 or greater.
% \item
%   \pdfTeX in PDF mode. (I don't know a DVI driver that
%   support several color stacks.)
% \item
%   \xfile{pdftex.def} 2007/01/02 v0.04b.
% \end{itemize}
% Package \xpackage{pdfcol} checks the requirements and
% sets switch \cs{ifpdfcolAvailable} accordingly.
%
% \subsection{Interface}
%
% \begin{declcs}{ifpdfcolAvailable}
% \end{declcs}
% If the requirements of section \ref{sec:req} are met the
% switch \cs{ifpdfcolAvailable} behaves as \cs{iftrue}.
% Otherwise the other interface macros in this section will
% be disabled with a message. Also the first use of such a
% macro will print a message. The messages are print to
% the \xext{log} file only if \pdfTeX\ is not used in PDF mode.
%
% \begin{declcs}{pdfcolErrorNoStacks}
% \end{declcs}
% The first call of \cs{pdfcolErrorNoStacks} prints an error
% message, if color stacks are not available.
%
% \begin{declcs}{pdfcolInitStack} \M{name}
% \end{declcs}
% A new color stack is initialized by \cs{pdfcolInitStack}.
% The \meta{name} is used for indentifying the stack. It usually
% consists of letters and digits. (The name must survive a \cs{csname}.)
%
% The intension of the macro is the definition of an additional
% color stack. Thus the stack is not page bounded like the
% main color stack. Black (\texttt{0 g 0 G}) is used as initial
% color value. And colors are written with modifier \texttt{direct}
% that means without setting the current transfer matrix and changing
% the current point (see documentation of \pdfTeX\ for
% |\pdfliteral direct{...}|).
%
% \begin{declcs}{pdfcolIfStackExists} \M{name} \M{then} \M{else}
% \end{declcs}
% Macro \cs{pdfcolIfStackExists} checks whether color stack \meta{name}
% exists. In case of success argument \meta{then} is executed
% and \meta{else} otherwise.
%
% \begin{declcs}{pdfcolSwitchStack} \M{name}
% \end{declcs}
% Macro \cs{pdfcolSwitchStack} switches the color stack. The color macros
% of package \xpackage{color} (or \xpackage{xcolor}) now uses the
% new color stack with name \meta{name}.
%
% \begin{declcs}{pdfcolSetCurrentColor}
% \end{declcs}
% Macro \cs{pdfcolSetCurrentColor} replaces the topmost
% entry of the stack by the current color (\cs{current@color}).
%
% \begin{declcs}{pdfcolSetCurrent} \M{name}
% \end{declcs}
% Macro \cs{pdfcolSetCurrent} sets the color that is read in
% the top-most entry of color stack \meta{name}. If \meta{name}
% is empty, the default color stack is used.
%
% \StopEventually{
% }
%
% \section{Implementation}
%
%    \begin{macrocode}
%<*package>
%    \end{macrocode}
%
% \subsection{Reload check and package identification}
%    Reload check, especially if the package is not used with \LaTeX.
%    \begin{macrocode}
\begingroup\catcode61\catcode48\catcode32=10\relax%
  \catcode13=5 % ^^M
  \endlinechar=13 %
  \catcode35=6 % #
  \catcode39=12 % '
  \catcode44=12 % ,
  \catcode45=12 % -
  \catcode46=12 % .
  \catcode58=12 % :
  \catcode64=11 % @
  \catcode123=1 % {
  \catcode125=2 % }
  \expandafter\let\expandafter\x\csname ver@pdfcol.sty\endcsname
  \ifx\x\relax % plain-TeX, first loading
  \else
    \def\empty{}%
    \ifx\x\empty % LaTeX, first loading,
      % variable is initialized, but \ProvidesPackage not yet seen
    \else
      \expandafter\ifx\csname PackageInfo\endcsname\relax
        \def\x#1#2{%
          \immediate\write-1{Package #1 Info: #2.}%
        }%
      \else
        \def\x#1#2{\PackageInfo{#1}{#2, stopped}}%
      \fi
      \x{pdfcol}{The package is already loaded}%
      \aftergroup\endinput
    \fi
  \fi
\endgroup%
%    \end{macrocode}
%    Package identification:
%    \begin{macrocode}
\begingroup\catcode61\catcode48\catcode32=10\relax%
  \catcode13=5 % ^^M
  \endlinechar=13 %
  \catcode35=6 % #
  \catcode39=12 % '
  \catcode40=12 % (
  \catcode41=12 % )
  \catcode44=12 % ,
  \catcode45=12 % -
  \catcode46=12 % .
  \catcode47=12 % /
  \catcode58=12 % :
  \catcode64=11 % @
  \catcode91=12 % [
  \catcode93=12 % ]
  \catcode123=1 % {
  \catcode125=2 % }
  \expandafter\ifx\csname ProvidesPackage\endcsname\relax
    \def\x#1#2#3[#4]{\endgroup
      \immediate\write-1{Package: #3 #4}%
      \xdef#1{#4}%
    }%
  \else
    \def\x#1#2[#3]{\endgroup
      #2[{#3}]%
      \ifx#1\@undefined
        \xdef#1{#3}%
      \fi
      \ifx#1\relax
        \xdef#1{#3}%
      \fi
    }%
  \fi
\expandafter\x\csname ver@pdfcol.sty\endcsname
\ProvidesPackage{pdfcol}%
  [2018/11/01 v1.5 Handle new color stacks for pdfTeX (HO)]%
%    \end{macrocode}
%
% \subsection{Catcodes}
%
%    \begin{macrocode}
\begingroup\catcode61\catcode48\catcode32=10\relax%
  \catcode13=5 % ^^M
  \endlinechar=13 %
  \catcode123=1 % {
  \catcode125=2 % }
  \catcode64=11 % @
  \def\x{\endgroup
    \expandafter\edef\csname PDFCOL@AtEnd\endcsname{%
      \endlinechar=\the\endlinechar\relax
      \catcode13=\the\catcode13\relax
      \catcode32=\the\catcode32\relax
      \catcode35=\the\catcode35\relax
      \catcode61=\the\catcode61\relax
      \catcode64=\the\catcode64\relax
      \catcode123=\the\catcode123\relax
      \catcode125=\the\catcode125\relax
    }%
  }%
\x\catcode61\catcode48\catcode32=10\relax%
\catcode13=5 % ^^M
\endlinechar=13 %
\catcode35=6 % #
\catcode64=11 % @
\catcode123=1 % {
\catcode125=2 % }
\def\TMP@EnsureCode#1#2{%
  \edef\PDFCOL@AtEnd{%
    \PDFCOL@AtEnd
    \catcode#1=\the\catcode#1\relax
  }%
  \catcode#1=#2\relax
}
\TMP@EnsureCode{39}{12}% '
\TMP@EnsureCode{40}{12}% (
\TMP@EnsureCode{41}{12}% )
\TMP@EnsureCode{43}{12}% +
\TMP@EnsureCode{44}{12}% ,
\TMP@EnsureCode{46}{12}% .
\TMP@EnsureCode{47}{12}% /
\TMP@EnsureCode{91}{12}% [
\TMP@EnsureCode{93}{12}% ]
\TMP@EnsureCode{96}{12}% `
\edef\PDFCOL@AtEnd{\PDFCOL@AtEnd\noexpand\endinput}
%    \end{macrocode}
%
% \subsection{Check requirements}
%
%    \begin{macro}{\PDFCOL@RequirePackage}
%    \begin{macrocode}
\begingroup\expandafter\expandafter\expandafter\endgroup
\expandafter\ifx\csname RequirePackage\endcsname\relax
  \def\PDFCOL@RequirePackage#1[#2]{\input #1.sty\relax}%
\else
  \def\PDFCOL@RequirePackage#1[#2]{%
    \RequirePackage{#1}[{#2}]%
  }%
\fi
%    \end{macrocode}
%    \end{macro}
%
% LuaTeX Compatability
%    \begin{macrocode}
\ifx\pdfextension\@undefined\else
 \def\pdfcolorstackinit {\pdffeedback colorstackinit}
 \protected\def\pdfcolorstack     {\pdfextension colorstack}
\fi
%    \end{macrocode}
%
%    \begin{macrocode}
\PDFCOL@RequirePackage{ltxcmds}[2010/03/01]
%    \end{macrocode}
%
%    \begin{macro}{ifpdfcolAvailable}
%    \begin{macrocode}
\ltx@newif\ifpdfcolAvailable
\pdfcolAvailabletrue
%    \end{macrocode}
%    \end{macro}
%
% \subsubsection{Check package \xpackage{luacolor}}
%
%    \begin{macrocode}
\ltx@newif\ifPDFCOL@luacolor
\begingroup\expandafter\expandafter\expandafter\endgroup
\expandafter\ifx\csname ver@luacolor.sty\endcsname\relax
  \PDFCOL@luacolorfalse
\else
  \PDFCOL@luacolortrue
\fi
%    \end{macrocode}
%
% \subsubsection{Check PDF mode}
%
%    \begin{macrocode}
\PDFCOL@RequirePackage{infwarerr}[2007/09/09]
\PDFCOL@RequirePackage{ifpdf}[2007/09/09]
\ifcase\ifpdf\ifPDFCOL@luacolor 1\fi\else 1\fi0 %
  \def\PDFCOL@Message{%
    \@PackageWarningNoLine{pdfcol}%
  }%
\else
  \pdfcolAvailablefalse
  \def\PDFCOL@Message{%
    \@PackageInfoNoLine{pdfcol}%
  }%
  \PDFCOL@Message{%
    Interface disabled because of %
    \ifPDFCOL@luacolor
      package `luacolor'%
    \else
      missing PDF mode of pdfTeX%
    \fi
  }%
\fi
%    \end{macrocode}
%
% \subsubsection{Check version of \pdfTeX}
%
%    \begin{macrocode}
\ifpdfcolAvailable
  \begingroup\expandafter\expandafter\expandafter\endgroup
  \expandafter\ifx\csname pdfcolorstack\endcsname\relax
    \pdfcolAvailablefalse
    \PDFCOL@Message{%
      Interface disabled because of too old pdfTeX.\MessageBreak
      Required is version 1.40+ for \string\pdfcolorstack
    }%
  \fi
\fi
\ifpdfcolAvailable
  \begingroup\expandafter\expandafter\expandafter\endgroup
  \expandafter\ifx\csname pdfcolorstack\endcsname\relax
    \pdfcolAvailablefalse
    \PDFCOL@Message{%
      Interface disabled because of too old pdfTeX.\MessageBreak
      Required is version 1.40+ for \string\pdfcolorstackinit
    }%
  \fi
\fi
%    \end{macrocode}
%
% \subsubsection{Check \xfile{pdftex.def}}
%
%    \begin{macrocode}
\ifpdfcolAvailable
  \begingroup\expandafter\expandafter\expandafter\endgroup
  \expandafter\ifx\csname @pdfcolorstack\endcsname\relax
%    \end{macrocode}
%    Try to load package color if it is not yet loaded (\LaTeX\ case).
%    \begin{macrocode}
    \begingroup\expandafter\expandafter\expandafter\endgroup
    \expandafter\ifx\csname ver@color.sty\endcsname\relax
      \begingroup\expandafter\expandafter\expandafter\endgroup
      \expandafter\ifx\csname documentclass\endcsname\relax
      \else
        \RequirePackage[pdftex]{color}\relax
      \fi
    \fi
    \begingroup\expandafter\expandafter\expandafter\endgroup
    \expandafter\ifx\csname @pdfcolorstack\endcsname\relax
      \pdfcolAvailablefalse
      \PDFCOL@Message{%
        Interface disabled because `pdftex.def'\MessageBreak
        is not loaded or it is too old.\MessageBreak
        Required is version 0.04b or greater%
      }%
    \fi
  \fi
\fi
%    \end{macrocode}
%
%    \begin{macrocode}
\let\pdfcolAvailabletrue\relax
\let\pdfcolAvailablefalse\relax
%    \end{macrocode}
%
% \subsection{Enabled interface macros}
%
%    \begin{macrocode}
\ifpdfcolAvailable
%    \end{macrocode}
%
%    \begin{macro}{\pdfcolErrorNoStacks}
%    \begin{macrocode}
  \let\pdfcolErrorNoStacks\relax
%    \end{macrocode}
%    \end{macro}
%
%    \begin{macro}{\pdfcol@Value}
%    \begin{macrocode}
  \expandafter\ifx\csname pdfcol@Value\endcsname\relax
    \def\pdfcol@Value{0 g 0 G}%
  \fi
%    \end{macrocode}
%    \end{macro}
%
%    \begin{macro}{\pdfcol@LiteralModifier}
%    \begin{macrocode}
  \expandafter\ifx\csname pdfcol@LiteralModifier\endcsname\relax
    \def\pdfcol@LiteralModifier{direct}%
  \fi
%    \end{macrocode}
%    \end{macro}
%
%    \begin{macro}{\pdfcolInitStack}
%    \begin{macrocode}
  \def\pdfcolInitStack#1{%
    \expandafter\ifx\csname pdfcol@Stack@#1\endcsname\relax
      \global\expandafter\chardef\csname pdfcol@Stack@#1\endcsname=%
          \pdfcolorstackinit\pdfcol@LiteralModifier{\pdfcol@Value}%
          \relax
      \@PackageInfo{pdfcol}{%
        New color stack `#1' = \number\csname pdfcol@Stack@#1\endcsname
      }%
    \else
      \@PackageError{pdfcol}{%
        Stack `#1' is already defined%
      }\@ehc
    \fi
  }%
%    \end{macrocode}
%    \end{macro}
%
%    \begin{macro}{\pdfcolIfStackExists}
%    \begin{macrocode}
  \def\pdfcolIfStackExists#1{%
    \expandafter\ifx\csname pdfcol@Stack@#1\endcsname\relax
      \expandafter\@secondoftwo
    \else
      \expandafter\@firstoftwo
    \fi
  }%
%    \end{macrocode}
%    \end{macro}
%    \begin{macro}{\@firstoftwo}
%    \begin{macrocode}
  \expandafter\ifx\csname @firstoftwo\endcsname\relax
    \long\def\@firstoftwo#1#2{#1}%
  \fi
%    \end{macrocode}
%    \end{macro}
%    \begin{macro}{\@secondoftwo}
%    \begin{macrocode}
  \expandafter\ifx\csname @secondoftwo\endcsname\relax
    \long\def\@secondoftwo#1#2{#2}%
  \fi
%    \end{macrocode}
%    \end{macro}
%
%    \begin{macro}{\pdfcolSwitchStack}
%    \begin{macrocode}
  \def\pdfcolSwitchStack#1{%
    \pdfcolIfStackExists{#1}{%
      \expandafter\let\expandafter\@pdfcolorstack
                      \csname pdfcol@Stack@#1\endcsname
    }{%
      \pdfcol@ErrorNoStack{#1}%
    }%
  }%
%    \end{macrocode}
%    \end{macro}
%
%    \begin{macro}{\pdfcolSetCurrentColor}
%    \begin{macrocode}
  \def\pdfcolSetCurrentColor{%
    \pdfcolorstack\@pdfcolorstack set{\current@color}%
  }%
%    \end{macrocode}
%    \end{macro}
%
%    \begin{macro}{\pdfcolSetCurrent}
%    \begin{macrocode}
  \def\pdfcolSetCurrent#1{%
    \ifx\\#1\\%
      \pdfcolorstack\@pdfcolorstack current\relax
    \else
      \pdfcolIfStackExists{#1}{%
        \pdfcolorstack\csname pdfcol@Stack@#1\endcsname current\relax
      }{%
        \pdfcol@ErrorNoStack{#1}%
      }%
    \fi
  }%
%    \end{macrocode}
%    \end{macro}
%
%    \begin{macro}{\pdfcol@ErrorNoStack}
%    \begin{macrocode}
  \def\pdfcol@ErrorNoStack#1{%
    \@PackageError{pdfcol}{Stack `#1' does not exists}\@ehc
  }%
%    \end{macrocode}
%    \end{macro}
%
% \subsection{Disabled interface macros}
%
%    \begin{macrocode}
\else
%    \end{macrocode}
%
%    \begin{macro}{\pdfcolErrorNoStacks}
%    \begin{macrocode}
  \def\pdfcolErrorNoStacks{%
    \@PackageError{pdfcol}{%
      Color stacks are not available%
    }{%
      Update pdfTeX (1.40) and `pdftex.def' (0.04b) %
          if necessary.\MessageBreak
      Ensure that `pdftex.def' is loaded %
          (package `color' or `xcolor').\MessageBreak
      Further messages can be found in TeX's %
          protocol file `\jobname.log'.\MessageBreak
      \MessageBreak
      \@ehc
    }%
    \global\let\pdfcolErrorNoStacks\relax
  }%
%    \end{macrocode}
%    \end{macro}
%
%    \begin{macro}{\PDFCOL@Disabled}
%    \begin{macrocode}
  \def\PDFCOL@Disabled{%
    \PDFCOL@Message{%
      pdfTeX's color stacks are not available%
    }%
    \global\let\PDFCOL@Disabled\relax
  }%
%    \end{macrocode}
%    \end{macro}
%
%    \begin{macro}{\pdfcolInitStack}
%    \begin{macrocode}
  \def\pdfcolInitStack#1{%
    \PDFCOL@Disabled
  }%
%    \end{macrocode}
%    \end{macro}
%
%    \begin{macro}{\pdfcolIfStackExists}
%    \begin{macrocode}
  \long\def\pdfcolIfStackExists#1#2#3{#3}%
%    \end{macrocode}
%    \end{macro}
%
%    \begin{macro}{\pdfcolSwitchStack}
%    \begin{macrocode}
  \def\pdfcolSwitchStack#1{%
    \PDFCOL@Disabled
  }%
%    \end{macrocode}
%    \end{macro}
%
%    \begin{macro}{\pdfcolSetCurrentColor}
%    \begin{macrocode}
  \def\pdfcolSetCurrentColor{%
    \PDFCOL@Disabled
  }%
%    \end{macrocode}
%    \end{macro}
%
%    \begin{macro}{\pdfcolSetCurrent}
%    \begin{macrocode}
  \def\pdfcolSetCurrent#1{%
    \PDFCOL@Disabled
  }%
%    \end{macrocode}
%    \end{macro}
%    \begin{macrocode}
\fi
%    \end{macrocode}
%
%    \begin{macrocode}
\PDFCOL@AtEnd%
%</package>
%    \end{macrocode}
%% \section{Installation}
%
% \subsection{Download}
%
% \paragraph{Package.} This package is available on
% CTAN\footnote{\CTANpkg{pdfcol}}:
% \begin{description}
% \item[\CTAN{macros/latex/contrib/oberdiek/pdfcol.dtx}] The source file.
% \item[\CTAN{macros/latex/contrib/oberdiek/pdfcol.pdf}] Documentation.
% \end{description}
%
%
% \paragraph{Bundle.} All the packages of the bundle `oberdiek'
% are also available in a TDS compliant ZIP archive. There
% the packages are already unpacked and the documentation files
% are generated. The files and directories obey the TDS standard.
% \begin{description}
% \item[\CTANinstall{install/macros/latex/contrib/oberdiek.tds.zip}]
% \end{description}
% \emph{TDS} refers to the standard ``A Directory Structure
% for \TeX\ Files'' (\CTANpkg{tds}). Directories
% with \xfile{texmf} in their name are usually organized this way.
%
% \subsection{Bundle installation}
%
% \paragraph{Unpacking.} Unpack the \xfile{oberdiek.tds.zip} in the
% TDS tree (also known as \xfile{texmf} tree) of your choice.
% Example (linux):
% \begin{quote}
%   |unzip oberdiek.tds.zip -d ~/texmf|
% \end{quote}
%
% \subsection{Package installation}
%
% \paragraph{Unpacking.} The \xfile{.dtx} file is a self-extracting
% \docstrip\ archive. The files are extracted by running the
% \xfile{.dtx} through \plainTeX:
% \begin{quote}
%   \verb|tex pdfcol.dtx|
% \end{quote}
%
% \paragraph{TDS.} Now the different files must be moved into
% the different directories in your installation TDS tree
% (also known as \xfile{texmf} tree):
% \begin{quote}
% \def\t{^^A
% \begin{tabular}{@{}>{\ttfamily}l@{ $\rightarrow$ }>{\ttfamily}l@{}}
%   pdfcol.sty & tex/generic/oberdiek/pdfcol.sty\\
%   pdfcol.pdf & doc/latex/oberdiek/pdfcol.pdf\\
%   pdfcol.dtx & source/latex/oberdiek/pdfcol.dtx\\
% \end{tabular}^^A
% }^^A
% \sbox0{\t}^^A
% \ifdim\wd0>\linewidth
%   \begingroup
%     \advance\linewidth by\leftmargin
%     \advance\linewidth by\rightmargin
%   \edef\x{\endgroup
%     \def\noexpand\lw{\the\linewidth}^^A
%   }\x
%   \def\lwbox{^^A
%     \leavevmode
%     \hbox to \linewidth{^^A
%       \kern-\leftmargin\relax
%       \hss
%       \usebox0
%       \hss
%       \kern-\rightmargin\relax
%     }^^A
%   }^^A
%   \ifdim\wd0>\lw
%     \sbox0{\small\t}^^A
%     \ifdim\wd0>\linewidth
%       \ifdim\wd0>\lw
%         \sbox0{\footnotesize\t}^^A
%         \ifdim\wd0>\linewidth
%           \ifdim\wd0>\lw
%             \sbox0{\scriptsize\t}^^A
%             \ifdim\wd0>\linewidth
%               \ifdim\wd0>\lw
%                 \sbox0{\tiny\t}^^A
%                 \ifdim\wd0>\linewidth
%                   \lwbox
%                 \else
%                   \usebox0
%                 \fi
%               \else
%                 \lwbox
%               \fi
%             \else
%               \usebox0
%             \fi
%           \else
%             \lwbox
%           \fi
%         \else
%           \usebox0
%         \fi
%       \else
%         \lwbox
%       \fi
%     \else
%       \usebox0
%     \fi
%   \else
%     \lwbox
%   \fi
% \else
%   \usebox0
% \fi
% \end{quote}
% If you have a \xfile{docstrip.cfg} that configures and enables \docstrip's
% TDS installing feature, then some files can already be in the right
% place, see the documentation of \docstrip.
%
% \subsection{Refresh file name databases}
%
% If your \TeX~distribution
% (\TeX\,Live, \mikTeX, \dots) relies on file name databases, you must refresh
% these. For example, \TeX\,Live\ users run \verb|texhash| or
% \verb|mktexlsr|.
%
% \subsection{Some details for the interested}
%
% \paragraph{Unpacking with \LaTeX.}
% The \xfile{.dtx} chooses its action depending on the format:
% \begin{description}
% \item[\plainTeX:] Run \docstrip\ and extract the files.
% \item[\LaTeX:] Generate the documentation.
% \end{description}
% If you insist on using \LaTeX\ for \docstrip\ (really,
% \docstrip\ does not need \LaTeX), then inform the autodetect routine
% about your intention:
% \begin{quote}
%   \verb|latex \let\install=y% \iffalse meta-comment
%
% File: pdfcol.dtx
% Version: 2018/11/01 v1.5
% Info: Handle new color stacks for pdfTeX
%
% Copyright (C)
%    2007 Heiko Oberdiek
%    2016-2019 Oberdiek Package Support Group
%    https://github.com/ho-tex/oberdiek/issues
%
% This work may be distributed and/or modified under the
% conditions of the LaTeX Project Public License, either
% version 1.3c of this license or (at your option) any later
% version. This version of this license is in
%    https://www.latex-project.org/lppl/lppl-1-3c.txt
% and the latest version of this license is in
%    https://www.latex-project.org/lppl.txt
% and version 1.3 or later is part of all distributions of
% LaTeX version 2005/12/01 or later.
%
% This work has the LPPL maintenance status "maintained".
%
% The Current Maintainers of this work are
% Heiko Oberdiek and the Oberdiek Package Support Group
% https://github.com/ho-tex/oberdiek/issues
%
% The Base Interpreter refers to any `TeX-Format',
% because some files are installed in TDS:tex/generic//.
%
% This work consists of the main source file pdfcol.dtx
% and the derived files
%    pdfcol.sty, pdfcol.pdf, pdfcol.ins, pdfcol.drv, pdfcol-test1.tex,
%    pdfcol-test2.tex, pdfcol-test3.tex, pdfcol-test4.tex.
%
% Distribution:
%    CTAN:macros/latex/contrib/oberdiek/pdfcol.dtx
%    CTAN:macros/latex/contrib/oberdiek/pdfcol.pdf
%
% Unpacking:
%    (a) If pdfcol.ins is present:
%           tex pdfcol.ins
%    (b) Without pdfcol.ins:
%           tex pdfcol.dtx
%    (c) If you insist on using LaTeX
%           latex \let\install=y% \iffalse meta-comment
%
% File: pdfcol.dtx
% Version: 2018/11/01 v1.5
% Info: Handle new color stacks for pdfTeX
%
% Copyright (C)
%    2007 Heiko Oberdiek
%    2016-2019 Oberdiek Package Support Group
%    https://github.com/ho-tex/oberdiek/issues
%
% This work may be distributed and/or modified under the
% conditions of the LaTeX Project Public License, either
% version 1.3c of this license or (at your option) any later
% version. This version of this license is in
%    https://www.latex-project.org/lppl/lppl-1-3c.txt
% and the latest version of this license is in
%    https://www.latex-project.org/lppl.txt
% and version 1.3 or later is part of all distributions of
% LaTeX version 2005/12/01 or later.
%
% This work has the LPPL maintenance status "maintained".
%
% The Current Maintainers of this work are
% Heiko Oberdiek and the Oberdiek Package Support Group
% https://github.com/ho-tex/oberdiek/issues
%
% The Base Interpreter refers to any `TeX-Format',
% because some files are installed in TDS:tex/generic//.
%
% This work consists of the main source file pdfcol.dtx
% and the derived files
%    pdfcol.sty, pdfcol.pdf, pdfcol.ins, pdfcol.drv, pdfcol-test1.tex,
%    pdfcol-test2.tex, pdfcol-test3.tex, pdfcol-test4.tex.
%
% Distribution:
%    CTAN:macros/latex/contrib/oberdiek/pdfcol.dtx
%    CTAN:macros/latex/contrib/oberdiek/pdfcol.pdf
%
% Unpacking:
%    (a) If pdfcol.ins is present:
%           tex pdfcol.ins
%    (b) Without pdfcol.ins:
%           tex pdfcol.dtx
%    (c) If you insist on using LaTeX
%           latex \let\install=y% \iffalse meta-comment
%
% File: pdfcol.dtx
% Version: 2018/11/01 v1.5
% Info: Handle new color stacks for pdfTeX
%
% Copyright (C)
%    2007 Heiko Oberdiek
%    2016-2019 Oberdiek Package Support Group
%    https://github.com/ho-tex/oberdiek/issues
%
% This work may be distributed and/or modified under the
% conditions of the LaTeX Project Public License, either
% version 1.3c of this license or (at your option) any later
% version. This version of this license is in
%    https://www.latex-project.org/lppl/lppl-1-3c.txt
% and the latest version of this license is in
%    https://www.latex-project.org/lppl.txt
% and version 1.3 or later is part of all distributions of
% LaTeX version 2005/12/01 or later.
%
% This work has the LPPL maintenance status "maintained".
%
% The Current Maintainers of this work are
% Heiko Oberdiek and the Oberdiek Package Support Group
% https://github.com/ho-tex/oberdiek/issues
%
% The Base Interpreter refers to any `TeX-Format',
% because some files are installed in TDS:tex/generic//.
%
% This work consists of the main source file pdfcol.dtx
% and the derived files
%    pdfcol.sty, pdfcol.pdf, pdfcol.ins, pdfcol.drv, pdfcol-test1.tex,
%    pdfcol-test2.tex, pdfcol-test3.tex, pdfcol-test4.tex.
%
% Distribution:
%    CTAN:macros/latex/contrib/oberdiek/pdfcol.dtx
%    CTAN:macros/latex/contrib/oberdiek/pdfcol.pdf
%
% Unpacking:
%    (a) If pdfcol.ins is present:
%           tex pdfcol.ins
%    (b) Without pdfcol.ins:
%           tex pdfcol.dtx
%    (c) If you insist on using LaTeX
%           latex \let\install=y\input{pdfcol.dtx}
%        (quote the arguments according to the demands of your shell)
%
% Documentation:
%    (a) If pdfcol.drv is present:
%           latex pdfcol.drv
%    (b) Without pdfcol.drv:
%           latex pdfcol.dtx; ...
%    The class ltxdoc loads the configuration file ltxdoc.cfg
%    if available. Here you can specify further options, e.g.
%    use A4 as paper format:
%       \PassOptionsToClass{a4paper}{article}
%
%    Programm calls to get the documentation (example):
%       pdflatex pdfcol.dtx
%       makeindex -s gind.ist pdfcol.idx
%       pdflatex pdfcol.dtx
%       makeindex -s gind.ist pdfcol.idx
%       pdflatex pdfcol.dtx
%
% Installation:
%    TDS:tex/generic/oberdiek/pdfcol.sty
%    TDS:doc/latex/oberdiek/pdfcol.pdf
%    TDS:source/latex/oberdiek/pdfcol.dtx
%
%<*ignore>
\begingroup
  \catcode123=1 %
  \catcode125=2 %
  \def\x{LaTeX2e}%
\expandafter\endgroup
\ifcase 0\ifx\install y1\fi\expandafter
         \ifx\csname processbatchFile\endcsname\relax\else1\fi
         \ifx\fmtname\x\else 1\fi\relax
\else\csname fi\endcsname
%</ignore>
%<*install>
\input docstrip.tex
\Msg{************************************************************************}
\Msg{* Installation}
\Msg{* Package: pdfcol 2018/11/01 v1.5 Handle new color stacks for pdfTeX (HO)}
\Msg{************************************************************************}

\keepsilent
\askforoverwritefalse

\let\MetaPrefix\relax
\preamble

This is a generated file.

Project: pdfcol
Version: 2018/11/01 v1.5

Copyright (C)
   2007 Heiko Oberdiek
   2016-2019 Oberdiek Package Support Group

This work may be distributed and/or modified under the
conditions of the LaTeX Project Public License, either
version 1.3c of this license or (at your option) any later
version. This version of this license is in
   https://www.latex-project.org/lppl/lppl-1-3c.txt
and the latest version of this license is in
   https://www.latex-project.org/lppl.txt
and version 1.3 or later is part of all distributions of
LaTeX version 2005/12/01 or later.

This work has the LPPL maintenance status "maintained".

The Current Maintainers of this work are
Heiko Oberdiek and the Oberdiek Package Support Group
https://github.com/ho-tex/oberdiek/issues


The Base Interpreter refers to any `TeX-Format',
because some files are installed in TDS:tex/generic//.

This work consists of the main source file pdfcol.dtx
and the derived files
   pdfcol.sty, pdfcol.pdf, pdfcol.ins, pdfcol.drv, pdfcol-test1.tex,
   pdfcol-test2.tex, pdfcol-test3.tex, pdfcol-test4.tex.

\endpreamble
\let\MetaPrefix\DoubleperCent

\generate{%
  \file{pdfcol.ins}{\from{pdfcol.dtx}{install}}%
  \file{pdfcol.drv}{\from{pdfcol.dtx}{driver}}%
  \usedir{tex/generic/oberdiek}%
  \file{pdfcol.sty}{\from{pdfcol.dtx}{package}}%
%  \usedir{doc/latex/oberdiek/test}%
%  \file{pdfcol-test1.tex}{\from{pdfcol.dtx}{test1}}%
%  \file{pdfcol-test2.tex}{\from{pdfcol.dtx}{test2}}%
%  \file{pdfcol-test3.tex}{\from{pdfcol.dtx}{test3}}%
%  \file{pdfcol-test4.tex}{\from{pdfcol.dtx}{test4}}%
}

\catcode32=13\relax% active space
\let =\space%
\Msg{************************************************************************}
\Msg{*}
\Msg{* To finish the installation you have to move the following}
\Msg{* file into a directory searched by TeX:}
\Msg{*}
\Msg{*     pdfcol.sty}
\Msg{*}
\Msg{* To produce the documentation run the file `pdfcol.drv'}
\Msg{* through LaTeX.}
\Msg{*}
\Msg{* Happy TeXing!}
\Msg{*}
\Msg{************************************************************************}

\endbatchfile
%</install>
%<*ignore>
\fi
%</ignore>
%<*driver>
\NeedsTeXFormat{LaTeX2e}
\ProvidesFile{pdfcol.drv}%
  [2018/11/01 v1.5 Handle new color stacks for pdfTeX (HO)]%
\documentclass{ltxdoc}
\usepackage{holtxdoc}[2011/11/22]
\begin{document}
  \DocInput{pdfcol.dtx}%
\end{document}
%</driver>
% \fi
%
%
%
% \GetFileInfo{pdfcol.drv}
%
% \title{The \xpackage{pdfcol} package}
% \date{2018/11/01 v1.5}
% \author{Heiko Oberdiek\thanks
% {Please report any issues at \url{https://github.com/ho-tex/oberdiek/issues}}}
%
% \maketitle
%
% \begin{abstract}
% Since version 1.40 \pdfTeX\ supports color stacks.
% The driver file \xfile{pdftex.def} for package \xpackage{color}
% defines and uses a main color stack since version v0.04b.
% Package \xpackage{pdfcol} is intended for package writers.
% It defines macros for setting and maintaining new color stacks.
% \end{abstract}
%
% \tableofcontents
%
% \section{Documentation}
%
% Version 1.40 of \pdfTeX\ adds new primitives \cs{pdfcolorstackinit}
% and \cs{pdfcolorstack}. Now color stacks can be defined and used.
% A main color stack is maintained by the driver file \xfile{pdftex.def}
% similar to dvips or dvipdfm. However the number of color stacks
% is not limited to one in \pdfTeX. Thus further color problems
% can now be solved, such as footnotes across pages or text
% that is set in parallel columns (e.g. packages \xpackage{parallel}
% or \xpackage{parcolumn}). Unlike the main color stack,
% the support by additional color stacks cannot be done in
% a transparent manner.
%
% This package \xpackage{pdfcol} provides an easier interface to
% additional color stacks without the need to use the
% low level primitives.
%
% \subsection{Requirements}
% \label{sec:req}
%
% \begin{itemize}
% \item
%   \pdfTeX\ 1.40 or greater.
% \item
%   \pdfTeX in PDF mode. (I don't know a DVI driver that
%   support several color stacks.)
% \item
%   \xfile{pdftex.def} 2007/01/02 v0.04b.
% \end{itemize}
% Package \xpackage{pdfcol} checks the requirements and
% sets switch \cs{ifpdfcolAvailable} accordingly.
%
% \subsection{Interface}
%
% \begin{declcs}{ifpdfcolAvailable}
% \end{declcs}
% If the requirements of section \ref{sec:req} are met the
% switch \cs{ifpdfcolAvailable} behaves as \cs{iftrue}.
% Otherwise the other interface macros in this section will
% be disabled with a message. Also the first use of such a
% macro will print a message. The messages are print to
% the \xext{log} file only if \pdfTeX\ is not used in PDF mode.
%
% \begin{declcs}{pdfcolErrorNoStacks}
% \end{declcs}
% The first call of \cs{pdfcolErrorNoStacks} prints an error
% message, if color stacks are not available.
%
% \begin{declcs}{pdfcolInitStack} \M{name}
% \end{declcs}
% A new color stack is initialized by \cs{pdfcolInitStack}.
% The \meta{name} is used for indentifying the stack. It usually
% consists of letters and digits. (The name must survive a \cs{csname}.)
%
% The intension of the macro is the definition of an additional
% color stack. Thus the stack is not page bounded like the
% main color stack. Black (\texttt{0 g 0 G}) is used as initial
% color value. And colors are written with modifier \texttt{direct}
% that means without setting the current transfer matrix and changing
% the current point (see documentation of \pdfTeX\ for
% |\pdfliteral direct{...}|).
%
% \begin{declcs}{pdfcolIfStackExists} \M{name} \M{then} \M{else}
% \end{declcs}
% Macro \cs{pdfcolIfStackExists} checks whether color stack \meta{name}
% exists. In case of success argument \meta{then} is executed
% and \meta{else} otherwise.
%
% \begin{declcs}{pdfcolSwitchStack} \M{name}
% \end{declcs}
% Macro \cs{pdfcolSwitchStack} switches the color stack. The color macros
% of package \xpackage{color} (or \xpackage{xcolor}) now uses the
% new color stack with name \meta{name}.
%
% \begin{declcs}{pdfcolSetCurrentColor}
% \end{declcs}
% Macro \cs{pdfcolSetCurrentColor} replaces the topmost
% entry of the stack by the current color (\cs{current@color}).
%
% \begin{declcs}{pdfcolSetCurrent} \M{name}
% \end{declcs}
% Macro \cs{pdfcolSetCurrent} sets the color that is read in
% the top-most entry of color stack \meta{name}. If \meta{name}
% is empty, the default color stack is used.
%
% \StopEventually{
% }
%
% \section{Implementation}
%
%    \begin{macrocode}
%<*package>
%    \end{macrocode}
%
% \subsection{Reload check and package identification}
%    Reload check, especially if the package is not used with \LaTeX.
%    \begin{macrocode}
\begingroup\catcode61\catcode48\catcode32=10\relax%
  \catcode13=5 % ^^M
  \endlinechar=13 %
  \catcode35=6 % #
  \catcode39=12 % '
  \catcode44=12 % ,
  \catcode45=12 % -
  \catcode46=12 % .
  \catcode58=12 % :
  \catcode64=11 % @
  \catcode123=1 % {
  \catcode125=2 % }
  \expandafter\let\expandafter\x\csname ver@pdfcol.sty\endcsname
  \ifx\x\relax % plain-TeX, first loading
  \else
    \def\empty{}%
    \ifx\x\empty % LaTeX, first loading,
      % variable is initialized, but \ProvidesPackage not yet seen
    \else
      \expandafter\ifx\csname PackageInfo\endcsname\relax
        \def\x#1#2{%
          \immediate\write-1{Package #1 Info: #2.}%
        }%
      \else
        \def\x#1#2{\PackageInfo{#1}{#2, stopped}}%
      \fi
      \x{pdfcol}{The package is already loaded}%
      \aftergroup\endinput
    \fi
  \fi
\endgroup%
%    \end{macrocode}
%    Package identification:
%    \begin{macrocode}
\begingroup\catcode61\catcode48\catcode32=10\relax%
  \catcode13=5 % ^^M
  \endlinechar=13 %
  \catcode35=6 % #
  \catcode39=12 % '
  \catcode40=12 % (
  \catcode41=12 % )
  \catcode44=12 % ,
  \catcode45=12 % -
  \catcode46=12 % .
  \catcode47=12 % /
  \catcode58=12 % :
  \catcode64=11 % @
  \catcode91=12 % [
  \catcode93=12 % ]
  \catcode123=1 % {
  \catcode125=2 % }
  \expandafter\ifx\csname ProvidesPackage\endcsname\relax
    \def\x#1#2#3[#4]{\endgroup
      \immediate\write-1{Package: #3 #4}%
      \xdef#1{#4}%
    }%
  \else
    \def\x#1#2[#3]{\endgroup
      #2[{#3}]%
      \ifx#1\@undefined
        \xdef#1{#3}%
      \fi
      \ifx#1\relax
        \xdef#1{#3}%
      \fi
    }%
  \fi
\expandafter\x\csname ver@pdfcol.sty\endcsname
\ProvidesPackage{pdfcol}%
  [2018/11/01 v1.5 Handle new color stacks for pdfTeX (HO)]%
%    \end{macrocode}
%
% \subsection{Catcodes}
%
%    \begin{macrocode}
\begingroup\catcode61\catcode48\catcode32=10\relax%
  \catcode13=5 % ^^M
  \endlinechar=13 %
  \catcode123=1 % {
  \catcode125=2 % }
  \catcode64=11 % @
  \def\x{\endgroup
    \expandafter\edef\csname PDFCOL@AtEnd\endcsname{%
      \endlinechar=\the\endlinechar\relax
      \catcode13=\the\catcode13\relax
      \catcode32=\the\catcode32\relax
      \catcode35=\the\catcode35\relax
      \catcode61=\the\catcode61\relax
      \catcode64=\the\catcode64\relax
      \catcode123=\the\catcode123\relax
      \catcode125=\the\catcode125\relax
    }%
  }%
\x\catcode61\catcode48\catcode32=10\relax%
\catcode13=5 % ^^M
\endlinechar=13 %
\catcode35=6 % #
\catcode64=11 % @
\catcode123=1 % {
\catcode125=2 % }
\def\TMP@EnsureCode#1#2{%
  \edef\PDFCOL@AtEnd{%
    \PDFCOL@AtEnd
    \catcode#1=\the\catcode#1\relax
  }%
  \catcode#1=#2\relax
}
\TMP@EnsureCode{39}{12}% '
\TMP@EnsureCode{40}{12}% (
\TMP@EnsureCode{41}{12}% )
\TMP@EnsureCode{43}{12}% +
\TMP@EnsureCode{44}{12}% ,
\TMP@EnsureCode{46}{12}% .
\TMP@EnsureCode{47}{12}% /
\TMP@EnsureCode{91}{12}% [
\TMP@EnsureCode{93}{12}% ]
\TMP@EnsureCode{96}{12}% `
\edef\PDFCOL@AtEnd{\PDFCOL@AtEnd\noexpand\endinput}
%    \end{macrocode}
%
% \subsection{Check requirements}
%
%    \begin{macro}{\PDFCOL@RequirePackage}
%    \begin{macrocode}
\begingroup\expandafter\expandafter\expandafter\endgroup
\expandafter\ifx\csname RequirePackage\endcsname\relax
  \def\PDFCOL@RequirePackage#1[#2]{\input #1.sty\relax}%
\else
  \def\PDFCOL@RequirePackage#1[#2]{%
    \RequirePackage{#1}[{#2}]%
  }%
\fi
%    \end{macrocode}
%    \end{macro}
%
% LuaTeX Compatability
%    \begin{macrocode}
\ifx\pdfextension\@undefined\else
 \def\pdfcolorstackinit {\pdffeedback colorstackinit}
 \protected\def\pdfcolorstack     {\pdfextension colorstack}
\fi
%    \end{macrocode}
%
%    \begin{macrocode}
\PDFCOL@RequirePackage{ltxcmds}[2010/03/01]
%    \end{macrocode}
%
%    \begin{macro}{ifpdfcolAvailable}
%    \begin{macrocode}
\ltx@newif\ifpdfcolAvailable
\pdfcolAvailabletrue
%    \end{macrocode}
%    \end{macro}
%
% \subsubsection{Check package \xpackage{luacolor}}
%
%    \begin{macrocode}
\ltx@newif\ifPDFCOL@luacolor
\begingroup\expandafter\expandafter\expandafter\endgroup
\expandafter\ifx\csname ver@luacolor.sty\endcsname\relax
  \PDFCOL@luacolorfalse
\else
  \PDFCOL@luacolortrue
\fi
%    \end{macrocode}
%
% \subsubsection{Check PDF mode}
%
%    \begin{macrocode}
\PDFCOL@RequirePackage{infwarerr}[2007/09/09]
\PDFCOL@RequirePackage{ifpdf}[2007/09/09]
\ifcase\ifpdf\ifPDFCOL@luacolor 1\fi\else 1\fi0 %
  \def\PDFCOL@Message{%
    \@PackageWarningNoLine{pdfcol}%
  }%
\else
  \pdfcolAvailablefalse
  \def\PDFCOL@Message{%
    \@PackageInfoNoLine{pdfcol}%
  }%
  \PDFCOL@Message{%
    Interface disabled because of %
    \ifPDFCOL@luacolor
      package `luacolor'%
    \else
      missing PDF mode of pdfTeX%
    \fi
  }%
\fi
%    \end{macrocode}
%
% \subsubsection{Check version of \pdfTeX}
%
%    \begin{macrocode}
\ifpdfcolAvailable
  \begingroup\expandafter\expandafter\expandafter\endgroup
  \expandafter\ifx\csname pdfcolorstack\endcsname\relax
    \pdfcolAvailablefalse
    \PDFCOL@Message{%
      Interface disabled because of too old pdfTeX.\MessageBreak
      Required is version 1.40+ for \string\pdfcolorstack
    }%
  \fi
\fi
\ifpdfcolAvailable
  \begingroup\expandafter\expandafter\expandafter\endgroup
  \expandafter\ifx\csname pdfcolorstack\endcsname\relax
    \pdfcolAvailablefalse
    \PDFCOL@Message{%
      Interface disabled because of too old pdfTeX.\MessageBreak
      Required is version 1.40+ for \string\pdfcolorstackinit
    }%
  \fi
\fi
%    \end{macrocode}
%
% \subsubsection{Check \xfile{pdftex.def}}
%
%    \begin{macrocode}
\ifpdfcolAvailable
  \begingroup\expandafter\expandafter\expandafter\endgroup
  \expandafter\ifx\csname @pdfcolorstack\endcsname\relax
%    \end{macrocode}
%    Try to load package color if it is not yet loaded (\LaTeX\ case).
%    \begin{macrocode}
    \begingroup\expandafter\expandafter\expandafter\endgroup
    \expandafter\ifx\csname ver@color.sty\endcsname\relax
      \begingroup\expandafter\expandafter\expandafter\endgroup
      \expandafter\ifx\csname documentclass\endcsname\relax
      \else
        \RequirePackage[pdftex]{color}\relax
      \fi
    \fi
    \begingroup\expandafter\expandafter\expandafter\endgroup
    \expandafter\ifx\csname @pdfcolorstack\endcsname\relax
      \pdfcolAvailablefalse
      \PDFCOL@Message{%
        Interface disabled because `pdftex.def'\MessageBreak
        is not loaded or it is too old.\MessageBreak
        Required is version 0.04b or greater%
      }%
    \fi
  \fi
\fi
%    \end{macrocode}
%
%    \begin{macrocode}
\let\pdfcolAvailabletrue\relax
\let\pdfcolAvailablefalse\relax
%    \end{macrocode}
%
% \subsection{Enabled interface macros}
%
%    \begin{macrocode}
\ifpdfcolAvailable
%    \end{macrocode}
%
%    \begin{macro}{\pdfcolErrorNoStacks}
%    \begin{macrocode}
  \let\pdfcolErrorNoStacks\relax
%    \end{macrocode}
%    \end{macro}
%
%    \begin{macro}{\pdfcol@Value}
%    \begin{macrocode}
  \expandafter\ifx\csname pdfcol@Value\endcsname\relax
    \def\pdfcol@Value{0 g 0 G}%
  \fi
%    \end{macrocode}
%    \end{macro}
%
%    \begin{macro}{\pdfcol@LiteralModifier}
%    \begin{macrocode}
  \expandafter\ifx\csname pdfcol@LiteralModifier\endcsname\relax
    \def\pdfcol@LiteralModifier{direct}%
  \fi
%    \end{macrocode}
%    \end{macro}
%
%    \begin{macro}{\pdfcolInitStack}
%    \begin{macrocode}
  \def\pdfcolInitStack#1{%
    \expandafter\ifx\csname pdfcol@Stack@#1\endcsname\relax
      \global\expandafter\chardef\csname pdfcol@Stack@#1\endcsname=%
          \pdfcolorstackinit\pdfcol@LiteralModifier{\pdfcol@Value}%
          \relax
      \@PackageInfo{pdfcol}{%
        New color stack `#1' = \number\csname pdfcol@Stack@#1\endcsname
      }%
    \else
      \@PackageError{pdfcol}{%
        Stack `#1' is already defined%
      }\@ehc
    \fi
  }%
%    \end{macrocode}
%    \end{macro}
%
%    \begin{macro}{\pdfcolIfStackExists}
%    \begin{macrocode}
  \def\pdfcolIfStackExists#1{%
    \expandafter\ifx\csname pdfcol@Stack@#1\endcsname\relax
      \expandafter\@secondoftwo
    \else
      \expandafter\@firstoftwo
    \fi
  }%
%    \end{macrocode}
%    \end{macro}
%    \begin{macro}{\@firstoftwo}
%    \begin{macrocode}
  \expandafter\ifx\csname @firstoftwo\endcsname\relax
    \long\def\@firstoftwo#1#2{#1}%
  \fi
%    \end{macrocode}
%    \end{macro}
%    \begin{macro}{\@secondoftwo}
%    \begin{macrocode}
  \expandafter\ifx\csname @secondoftwo\endcsname\relax
    \long\def\@secondoftwo#1#2{#2}%
  \fi
%    \end{macrocode}
%    \end{macro}
%
%    \begin{macro}{\pdfcolSwitchStack}
%    \begin{macrocode}
  \def\pdfcolSwitchStack#1{%
    \pdfcolIfStackExists{#1}{%
      \expandafter\let\expandafter\@pdfcolorstack
                      \csname pdfcol@Stack@#1\endcsname
    }{%
      \pdfcol@ErrorNoStack{#1}%
    }%
  }%
%    \end{macrocode}
%    \end{macro}
%
%    \begin{macro}{\pdfcolSetCurrentColor}
%    \begin{macrocode}
  \def\pdfcolSetCurrentColor{%
    \pdfcolorstack\@pdfcolorstack set{\current@color}%
  }%
%    \end{macrocode}
%    \end{macro}
%
%    \begin{macro}{\pdfcolSetCurrent}
%    \begin{macrocode}
  \def\pdfcolSetCurrent#1{%
    \ifx\\#1\\%
      \pdfcolorstack\@pdfcolorstack current\relax
    \else
      \pdfcolIfStackExists{#1}{%
        \pdfcolorstack\csname pdfcol@Stack@#1\endcsname current\relax
      }{%
        \pdfcol@ErrorNoStack{#1}%
      }%
    \fi
  }%
%    \end{macrocode}
%    \end{macro}
%
%    \begin{macro}{\pdfcol@ErrorNoStack}
%    \begin{macrocode}
  \def\pdfcol@ErrorNoStack#1{%
    \@PackageError{pdfcol}{Stack `#1' does not exists}\@ehc
  }%
%    \end{macrocode}
%    \end{macro}
%
% \subsection{Disabled interface macros}
%
%    \begin{macrocode}
\else
%    \end{macrocode}
%
%    \begin{macro}{\pdfcolErrorNoStacks}
%    \begin{macrocode}
  \def\pdfcolErrorNoStacks{%
    \@PackageError{pdfcol}{%
      Color stacks are not available%
    }{%
      Update pdfTeX (1.40) and `pdftex.def' (0.04b) %
          if necessary.\MessageBreak
      Ensure that `pdftex.def' is loaded %
          (package `color' or `xcolor').\MessageBreak
      Further messages can be found in TeX's %
          protocol file `\jobname.log'.\MessageBreak
      \MessageBreak
      \@ehc
    }%
    \global\let\pdfcolErrorNoStacks\relax
  }%
%    \end{macrocode}
%    \end{macro}
%
%    \begin{macro}{\PDFCOL@Disabled}
%    \begin{macrocode}
  \def\PDFCOL@Disabled{%
    \PDFCOL@Message{%
      pdfTeX's color stacks are not available%
    }%
    \global\let\PDFCOL@Disabled\relax
  }%
%    \end{macrocode}
%    \end{macro}
%
%    \begin{macro}{\pdfcolInitStack}
%    \begin{macrocode}
  \def\pdfcolInitStack#1{%
    \PDFCOL@Disabled
  }%
%    \end{macrocode}
%    \end{macro}
%
%    \begin{macro}{\pdfcolIfStackExists}
%    \begin{macrocode}
  \long\def\pdfcolIfStackExists#1#2#3{#3}%
%    \end{macrocode}
%    \end{macro}
%
%    \begin{macro}{\pdfcolSwitchStack}
%    \begin{macrocode}
  \def\pdfcolSwitchStack#1{%
    \PDFCOL@Disabled
  }%
%    \end{macrocode}
%    \end{macro}
%
%    \begin{macro}{\pdfcolSetCurrentColor}
%    \begin{macrocode}
  \def\pdfcolSetCurrentColor{%
    \PDFCOL@Disabled
  }%
%    \end{macrocode}
%    \end{macro}
%
%    \begin{macro}{\pdfcolSetCurrent}
%    \begin{macrocode}
  \def\pdfcolSetCurrent#1{%
    \PDFCOL@Disabled
  }%
%    \end{macrocode}
%    \end{macro}
%    \begin{macrocode}
\fi
%    \end{macrocode}
%
%    \begin{macrocode}
\PDFCOL@AtEnd%
%</package>
%    \end{macrocode}
%% \section{Installation}
%
% \subsection{Download}
%
% \paragraph{Package.} This package is available on
% CTAN\footnote{\CTANpkg{pdfcol}}:
% \begin{description}
% \item[\CTAN{macros/latex/contrib/oberdiek/pdfcol.dtx}] The source file.
% \item[\CTAN{macros/latex/contrib/oberdiek/pdfcol.pdf}] Documentation.
% \end{description}
%
%
% \paragraph{Bundle.} All the packages of the bundle `oberdiek'
% are also available in a TDS compliant ZIP archive. There
% the packages are already unpacked and the documentation files
% are generated. The files and directories obey the TDS standard.
% \begin{description}
% \item[\CTANinstall{install/macros/latex/contrib/oberdiek.tds.zip}]
% \end{description}
% \emph{TDS} refers to the standard ``A Directory Structure
% for \TeX\ Files'' (\CTANpkg{tds}). Directories
% with \xfile{texmf} in their name are usually organized this way.
%
% \subsection{Bundle installation}
%
% \paragraph{Unpacking.} Unpack the \xfile{oberdiek.tds.zip} in the
% TDS tree (also known as \xfile{texmf} tree) of your choice.
% Example (linux):
% \begin{quote}
%   |unzip oberdiek.tds.zip -d ~/texmf|
% \end{quote}
%
% \subsection{Package installation}
%
% \paragraph{Unpacking.} The \xfile{.dtx} file is a self-extracting
% \docstrip\ archive. The files are extracted by running the
% \xfile{.dtx} through \plainTeX:
% \begin{quote}
%   \verb|tex pdfcol.dtx|
% \end{quote}
%
% \paragraph{TDS.} Now the different files must be moved into
% the different directories in your installation TDS tree
% (also known as \xfile{texmf} tree):
% \begin{quote}
% \def\t{^^A
% \begin{tabular}{@{}>{\ttfamily}l@{ $\rightarrow$ }>{\ttfamily}l@{}}
%   pdfcol.sty & tex/generic/oberdiek/pdfcol.sty\\
%   pdfcol.pdf & doc/latex/oberdiek/pdfcol.pdf\\
%   pdfcol.dtx & source/latex/oberdiek/pdfcol.dtx\\
% \end{tabular}^^A
% }^^A
% \sbox0{\t}^^A
% \ifdim\wd0>\linewidth
%   \begingroup
%     \advance\linewidth by\leftmargin
%     \advance\linewidth by\rightmargin
%   \edef\x{\endgroup
%     \def\noexpand\lw{\the\linewidth}^^A
%   }\x
%   \def\lwbox{^^A
%     \leavevmode
%     \hbox to \linewidth{^^A
%       \kern-\leftmargin\relax
%       \hss
%       \usebox0
%       \hss
%       \kern-\rightmargin\relax
%     }^^A
%   }^^A
%   \ifdim\wd0>\lw
%     \sbox0{\small\t}^^A
%     \ifdim\wd0>\linewidth
%       \ifdim\wd0>\lw
%         \sbox0{\footnotesize\t}^^A
%         \ifdim\wd0>\linewidth
%           \ifdim\wd0>\lw
%             \sbox0{\scriptsize\t}^^A
%             \ifdim\wd0>\linewidth
%               \ifdim\wd0>\lw
%                 \sbox0{\tiny\t}^^A
%                 \ifdim\wd0>\linewidth
%                   \lwbox
%                 \else
%                   \usebox0
%                 \fi
%               \else
%                 \lwbox
%               \fi
%             \else
%               \usebox0
%             \fi
%           \else
%             \lwbox
%           \fi
%         \else
%           \usebox0
%         \fi
%       \else
%         \lwbox
%       \fi
%     \else
%       \usebox0
%     \fi
%   \else
%     \lwbox
%   \fi
% \else
%   \usebox0
% \fi
% \end{quote}
% If you have a \xfile{docstrip.cfg} that configures and enables \docstrip's
% TDS installing feature, then some files can already be in the right
% place, see the documentation of \docstrip.
%
% \subsection{Refresh file name databases}
%
% If your \TeX~distribution
% (\TeX\,Live, \mikTeX, \dots) relies on file name databases, you must refresh
% these. For example, \TeX\,Live\ users run \verb|texhash| or
% \verb|mktexlsr|.
%
% \subsection{Some details for the interested}
%
% \paragraph{Unpacking with \LaTeX.}
% The \xfile{.dtx} chooses its action depending on the format:
% \begin{description}
% \item[\plainTeX:] Run \docstrip\ and extract the files.
% \item[\LaTeX:] Generate the documentation.
% \end{description}
% If you insist on using \LaTeX\ for \docstrip\ (really,
% \docstrip\ does not need \LaTeX), then inform the autodetect routine
% about your intention:
% \begin{quote}
%   \verb|latex \let\install=y\input{pdfcol.dtx}|
% \end{quote}
% Do not forget to quote the argument according to the demands
% of your shell.
%
% \paragraph{Generating the documentation.}
% You can use both the \xfile{.dtx} or the \xfile{.drv} to generate
% the documentation. The process can be configured by the
% configuration file \xfile{ltxdoc.cfg}. For instance, put this
% line into this file, if you want to have A4 as paper format:
% \begin{quote}
%   \verb|\PassOptionsToClass{a4paper}{article}|
% \end{quote}
% An example follows how to generate the
% documentation with pdf\LaTeX:
% \begin{quote}
%\begin{verbatim}
%pdflatex pdfcol.dtx
%makeindex -s gind.ist pdfcol.idx
%pdflatex pdfcol.dtx
%makeindex -s gind.ist pdfcol.idx
%pdflatex pdfcol.dtx
%\end{verbatim}
% \end{quote}
%
% \begin{History}
%   \begin{Version}{2007/09/09 v1.0}
%   \item
%     First version.
%   \end{Version}
%   \begin{Version}{2007/12/09 v1.1}
%   \item
%     \cs{pdfcolSetCurrentColor} added.
%   \end{Version}
%   \begin{Version}{2007/12/12 v1.2}
%   \item
%     Detection for package \xpackage{luacolor} added.
%   \end{Version}
%   \begin{Version}{2016/05/16 v1.3}
%   \item
%     Documentation updates.
%   \end{Version}
%   \begin{Version}{2016/05/17 v1.4}
%   \item
%     Use luatex85 package for new luatex compatibility
%   \end{Version}
%   \begin{Version}{2018/11/01 v1.5}
%   \item
%     Remove luatex85 dependency
%   \end{Version}
% \end{History}
%
% \PrintIndex
%
% \Finale
\endinput

%        (quote the arguments according to the demands of your shell)
%
% Documentation:
%    (a) If pdfcol.drv is present:
%           latex pdfcol.drv
%    (b) Without pdfcol.drv:
%           latex pdfcol.dtx; ...
%    The class ltxdoc loads the configuration file ltxdoc.cfg
%    if available. Here you can specify further options, e.g.
%    use A4 as paper format:
%       \PassOptionsToClass{a4paper}{article}
%
%    Programm calls to get the documentation (example):
%       pdflatex pdfcol.dtx
%       makeindex -s gind.ist pdfcol.idx
%       pdflatex pdfcol.dtx
%       makeindex -s gind.ist pdfcol.idx
%       pdflatex pdfcol.dtx
%
% Installation:
%    TDS:tex/generic/oberdiek/pdfcol.sty
%    TDS:doc/latex/oberdiek/pdfcol.pdf
%    TDS:source/latex/oberdiek/pdfcol.dtx
%
%<*ignore>
\begingroup
  \catcode123=1 %
  \catcode125=2 %
  \def\x{LaTeX2e}%
\expandafter\endgroup
\ifcase 0\ifx\install y1\fi\expandafter
         \ifx\csname processbatchFile\endcsname\relax\else1\fi
         \ifx\fmtname\x\else 1\fi\relax
\else\csname fi\endcsname
%</ignore>
%<*install>
\input docstrip.tex
\Msg{************************************************************************}
\Msg{* Installation}
\Msg{* Package: pdfcol 2018/11/01 v1.5 Handle new color stacks for pdfTeX (HO)}
\Msg{************************************************************************}

\keepsilent
\askforoverwritefalse

\let\MetaPrefix\relax
\preamble

This is a generated file.

Project: pdfcol
Version: 2018/11/01 v1.5

Copyright (C)
   2007 Heiko Oberdiek
   2016-2019 Oberdiek Package Support Group

This work may be distributed and/or modified under the
conditions of the LaTeX Project Public License, either
version 1.3c of this license or (at your option) any later
version. This version of this license is in
   https://www.latex-project.org/lppl/lppl-1-3c.txt
and the latest version of this license is in
   https://www.latex-project.org/lppl.txt
and version 1.3 or later is part of all distributions of
LaTeX version 2005/12/01 or later.

This work has the LPPL maintenance status "maintained".

The Current Maintainers of this work are
Heiko Oberdiek and the Oberdiek Package Support Group
https://github.com/ho-tex/oberdiek/issues


The Base Interpreter refers to any `TeX-Format',
because some files are installed in TDS:tex/generic//.

This work consists of the main source file pdfcol.dtx
and the derived files
   pdfcol.sty, pdfcol.pdf, pdfcol.ins, pdfcol.drv, pdfcol-test1.tex,
   pdfcol-test2.tex, pdfcol-test3.tex, pdfcol-test4.tex.

\endpreamble
\let\MetaPrefix\DoubleperCent

\generate{%
  \file{pdfcol.ins}{\from{pdfcol.dtx}{install}}%
  \file{pdfcol.drv}{\from{pdfcol.dtx}{driver}}%
  \usedir{tex/generic/oberdiek}%
  \file{pdfcol.sty}{\from{pdfcol.dtx}{package}}%
%  \usedir{doc/latex/oberdiek/test}%
%  \file{pdfcol-test1.tex}{\from{pdfcol.dtx}{test1}}%
%  \file{pdfcol-test2.tex}{\from{pdfcol.dtx}{test2}}%
%  \file{pdfcol-test3.tex}{\from{pdfcol.dtx}{test3}}%
%  \file{pdfcol-test4.tex}{\from{pdfcol.dtx}{test4}}%
}

\catcode32=13\relax% active space
\let =\space%
\Msg{************************************************************************}
\Msg{*}
\Msg{* To finish the installation you have to move the following}
\Msg{* file into a directory searched by TeX:}
\Msg{*}
\Msg{*     pdfcol.sty}
\Msg{*}
\Msg{* To produce the documentation run the file `pdfcol.drv'}
\Msg{* through LaTeX.}
\Msg{*}
\Msg{* Happy TeXing!}
\Msg{*}
\Msg{************************************************************************}

\endbatchfile
%</install>
%<*ignore>
\fi
%</ignore>
%<*driver>
\NeedsTeXFormat{LaTeX2e}
\ProvidesFile{pdfcol.drv}%
  [2018/11/01 v1.5 Handle new color stacks for pdfTeX (HO)]%
\documentclass{ltxdoc}
\usepackage{holtxdoc}[2011/11/22]
\begin{document}
  \DocInput{pdfcol.dtx}%
\end{document}
%</driver>
% \fi
%
%
%
% \GetFileInfo{pdfcol.drv}
%
% \title{The \xpackage{pdfcol} package}
% \date{2018/11/01 v1.5}
% \author{Heiko Oberdiek\thanks
% {Please report any issues at \url{https://github.com/ho-tex/oberdiek/issues}}}
%
% \maketitle
%
% \begin{abstract}
% Since version 1.40 \pdfTeX\ supports color stacks.
% The driver file \xfile{pdftex.def} for package \xpackage{color}
% defines and uses a main color stack since version v0.04b.
% Package \xpackage{pdfcol} is intended for package writers.
% It defines macros for setting and maintaining new color stacks.
% \end{abstract}
%
% \tableofcontents
%
% \section{Documentation}
%
% Version 1.40 of \pdfTeX\ adds new primitives \cs{pdfcolorstackinit}
% and \cs{pdfcolorstack}. Now color stacks can be defined and used.
% A main color stack is maintained by the driver file \xfile{pdftex.def}
% similar to dvips or dvipdfm. However the number of color stacks
% is not limited to one in \pdfTeX. Thus further color problems
% can now be solved, such as footnotes across pages or text
% that is set in parallel columns (e.g. packages \xpackage{parallel}
% or \xpackage{parcolumn}). Unlike the main color stack,
% the support by additional color stacks cannot be done in
% a transparent manner.
%
% This package \xpackage{pdfcol} provides an easier interface to
% additional color stacks without the need to use the
% low level primitives.
%
% \subsection{Requirements}
% \label{sec:req}
%
% \begin{itemize}
% \item
%   \pdfTeX\ 1.40 or greater.
% \item
%   \pdfTeX in PDF mode. (I don't know a DVI driver that
%   support several color stacks.)
% \item
%   \xfile{pdftex.def} 2007/01/02 v0.04b.
% \end{itemize}
% Package \xpackage{pdfcol} checks the requirements and
% sets switch \cs{ifpdfcolAvailable} accordingly.
%
% \subsection{Interface}
%
% \begin{declcs}{ifpdfcolAvailable}
% \end{declcs}
% If the requirements of section \ref{sec:req} are met the
% switch \cs{ifpdfcolAvailable} behaves as \cs{iftrue}.
% Otherwise the other interface macros in this section will
% be disabled with a message. Also the first use of such a
% macro will print a message. The messages are print to
% the \xext{log} file only if \pdfTeX\ is not used in PDF mode.
%
% \begin{declcs}{pdfcolErrorNoStacks}
% \end{declcs}
% The first call of \cs{pdfcolErrorNoStacks} prints an error
% message, if color stacks are not available.
%
% \begin{declcs}{pdfcolInitStack} \M{name}
% \end{declcs}
% A new color stack is initialized by \cs{pdfcolInitStack}.
% The \meta{name} is used for indentifying the stack. It usually
% consists of letters and digits. (The name must survive a \cs{csname}.)
%
% The intension of the macro is the definition of an additional
% color stack. Thus the stack is not page bounded like the
% main color stack. Black (\texttt{0 g 0 G}) is used as initial
% color value. And colors are written with modifier \texttt{direct}
% that means without setting the current transfer matrix and changing
% the current point (see documentation of \pdfTeX\ for
% |\pdfliteral direct{...}|).
%
% \begin{declcs}{pdfcolIfStackExists} \M{name} \M{then} \M{else}
% \end{declcs}
% Macro \cs{pdfcolIfStackExists} checks whether color stack \meta{name}
% exists. In case of success argument \meta{then} is executed
% and \meta{else} otherwise.
%
% \begin{declcs}{pdfcolSwitchStack} \M{name}
% \end{declcs}
% Macro \cs{pdfcolSwitchStack} switches the color stack. The color macros
% of package \xpackage{color} (or \xpackage{xcolor}) now uses the
% new color stack with name \meta{name}.
%
% \begin{declcs}{pdfcolSetCurrentColor}
% \end{declcs}
% Macro \cs{pdfcolSetCurrentColor} replaces the topmost
% entry of the stack by the current color (\cs{current@color}).
%
% \begin{declcs}{pdfcolSetCurrent} \M{name}
% \end{declcs}
% Macro \cs{pdfcolSetCurrent} sets the color that is read in
% the top-most entry of color stack \meta{name}. If \meta{name}
% is empty, the default color stack is used.
%
% \StopEventually{
% }
%
% \section{Implementation}
%
%    \begin{macrocode}
%<*package>
%    \end{macrocode}
%
% \subsection{Reload check and package identification}
%    Reload check, especially if the package is not used with \LaTeX.
%    \begin{macrocode}
\begingroup\catcode61\catcode48\catcode32=10\relax%
  \catcode13=5 % ^^M
  \endlinechar=13 %
  \catcode35=6 % #
  \catcode39=12 % '
  \catcode44=12 % ,
  \catcode45=12 % -
  \catcode46=12 % .
  \catcode58=12 % :
  \catcode64=11 % @
  \catcode123=1 % {
  \catcode125=2 % }
  \expandafter\let\expandafter\x\csname ver@pdfcol.sty\endcsname
  \ifx\x\relax % plain-TeX, first loading
  \else
    \def\empty{}%
    \ifx\x\empty % LaTeX, first loading,
      % variable is initialized, but \ProvidesPackage not yet seen
    \else
      \expandafter\ifx\csname PackageInfo\endcsname\relax
        \def\x#1#2{%
          \immediate\write-1{Package #1 Info: #2.}%
        }%
      \else
        \def\x#1#2{\PackageInfo{#1}{#2, stopped}}%
      \fi
      \x{pdfcol}{The package is already loaded}%
      \aftergroup\endinput
    \fi
  \fi
\endgroup%
%    \end{macrocode}
%    Package identification:
%    \begin{macrocode}
\begingroup\catcode61\catcode48\catcode32=10\relax%
  \catcode13=5 % ^^M
  \endlinechar=13 %
  \catcode35=6 % #
  \catcode39=12 % '
  \catcode40=12 % (
  \catcode41=12 % )
  \catcode44=12 % ,
  \catcode45=12 % -
  \catcode46=12 % .
  \catcode47=12 % /
  \catcode58=12 % :
  \catcode64=11 % @
  \catcode91=12 % [
  \catcode93=12 % ]
  \catcode123=1 % {
  \catcode125=2 % }
  \expandafter\ifx\csname ProvidesPackage\endcsname\relax
    \def\x#1#2#3[#4]{\endgroup
      \immediate\write-1{Package: #3 #4}%
      \xdef#1{#4}%
    }%
  \else
    \def\x#1#2[#3]{\endgroup
      #2[{#3}]%
      \ifx#1\@undefined
        \xdef#1{#3}%
      \fi
      \ifx#1\relax
        \xdef#1{#3}%
      \fi
    }%
  \fi
\expandafter\x\csname ver@pdfcol.sty\endcsname
\ProvidesPackage{pdfcol}%
  [2018/11/01 v1.5 Handle new color stacks for pdfTeX (HO)]%
%    \end{macrocode}
%
% \subsection{Catcodes}
%
%    \begin{macrocode}
\begingroup\catcode61\catcode48\catcode32=10\relax%
  \catcode13=5 % ^^M
  \endlinechar=13 %
  \catcode123=1 % {
  \catcode125=2 % }
  \catcode64=11 % @
  \def\x{\endgroup
    \expandafter\edef\csname PDFCOL@AtEnd\endcsname{%
      \endlinechar=\the\endlinechar\relax
      \catcode13=\the\catcode13\relax
      \catcode32=\the\catcode32\relax
      \catcode35=\the\catcode35\relax
      \catcode61=\the\catcode61\relax
      \catcode64=\the\catcode64\relax
      \catcode123=\the\catcode123\relax
      \catcode125=\the\catcode125\relax
    }%
  }%
\x\catcode61\catcode48\catcode32=10\relax%
\catcode13=5 % ^^M
\endlinechar=13 %
\catcode35=6 % #
\catcode64=11 % @
\catcode123=1 % {
\catcode125=2 % }
\def\TMP@EnsureCode#1#2{%
  \edef\PDFCOL@AtEnd{%
    \PDFCOL@AtEnd
    \catcode#1=\the\catcode#1\relax
  }%
  \catcode#1=#2\relax
}
\TMP@EnsureCode{39}{12}% '
\TMP@EnsureCode{40}{12}% (
\TMP@EnsureCode{41}{12}% )
\TMP@EnsureCode{43}{12}% +
\TMP@EnsureCode{44}{12}% ,
\TMP@EnsureCode{46}{12}% .
\TMP@EnsureCode{47}{12}% /
\TMP@EnsureCode{91}{12}% [
\TMP@EnsureCode{93}{12}% ]
\TMP@EnsureCode{96}{12}% `
\edef\PDFCOL@AtEnd{\PDFCOL@AtEnd\noexpand\endinput}
%    \end{macrocode}
%
% \subsection{Check requirements}
%
%    \begin{macro}{\PDFCOL@RequirePackage}
%    \begin{macrocode}
\begingroup\expandafter\expandafter\expandafter\endgroup
\expandafter\ifx\csname RequirePackage\endcsname\relax
  \def\PDFCOL@RequirePackage#1[#2]{\input #1.sty\relax}%
\else
  \def\PDFCOL@RequirePackage#1[#2]{%
    \RequirePackage{#1}[{#2}]%
  }%
\fi
%    \end{macrocode}
%    \end{macro}
%
% LuaTeX Compatability
%    \begin{macrocode}
\ifx\pdfextension\@undefined\else
 \def\pdfcolorstackinit {\pdffeedback colorstackinit}
 \protected\def\pdfcolorstack     {\pdfextension colorstack}
\fi
%    \end{macrocode}
%
%    \begin{macrocode}
\PDFCOL@RequirePackage{ltxcmds}[2010/03/01]
%    \end{macrocode}
%
%    \begin{macro}{ifpdfcolAvailable}
%    \begin{macrocode}
\ltx@newif\ifpdfcolAvailable
\pdfcolAvailabletrue
%    \end{macrocode}
%    \end{macro}
%
% \subsubsection{Check package \xpackage{luacolor}}
%
%    \begin{macrocode}
\ltx@newif\ifPDFCOL@luacolor
\begingroup\expandafter\expandafter\expandafter\endgroup
\expandafter\ifx\csname ver@luacolor.sty\endcsname\relax
  \PDFCOL@luacolorfalse
\else
  \PDFCOL@luacolortrue
\fi
%    \end{macrocode}
%
% \subsubsection{Check PDF mode}
%
%    \begin{macrocode}
\PDFCOL@RequirePackage{infwarerr}[2007/09/09]
\PDFCOL@RequirePackage{ifpdf}[2007/09/09]
\ifcase\ifpdf\ifPDFCOL@luacolor 1\fi\else 1\fi0 %
  \def\PDFCOL@Message{%
    \@PackageWarningNoLine{pdfcol}%
  }%
\else
  \pdfcolAvailablefalse
  \def\PDFCOL@Message{%
    \@PackageInfoNoLine{pdfcol}%
  }%
  \PDFCOL@Message{%
    Interface disabled because of %
    \ifPDFCOL@luacolor
      package `luacolor'%
    \else
      missing PDF mode of pdfTeX%
    \fi
  }%
\fi
%    \end{macrocode}
%
% \subsubsection{Check version of \pdfTeX}
%
%    \begin{macrocode}
\ifpdfcolAvailable
  \begingroup\expandafter\expandafter\expandafter\endgroup
  \expandafter\ifx\csname pdfcolorstack\endcsname\relax
    \pdfcolAvailablefalse
    \PDFCOL@Message{%
      Interface disabled because of too old pdfTeX.\MessageBreak
      Required is version 1.40+ for \string\pdfcolorstack
    }%
  \fi
\fi
\ifpdfcolAvailable
  \begingroup\expandafter\expandafter\expandafter\endgroup
  \expandafter\ifx\csname pdfcolorstack\endcsname\relax
    \pdfcolAvailablefalse
    \PDFCOL@Message{%
      Interface disabled because of too old pdfTeX.\MessageBreak
      Required is version 1.40+ for \string\pdfcolorstackinit
    }%
  \fi
\fi
%    \end{macrocode}
%
% \subsubsection{Check \xfile{pdftex.def}}
%
%    \begin{macrocode}
\ifpdfcolAvailable
  \begingroup\expandafter\expandafter\expandafter\endgroup
  \expandafter\ifx\csname @pdfcolorstack\endcsname\relax
%    \end{macrocode}
%    Try to load package color if it is not yet loaded (\LaTeX\ case).
%    \begin{macrocode}
    \begingroup\expandafter\expandafter\expandafter\endgroup
    \expandafter\ifx\csname ver@color.sty\endcsname\relax
      \begingroup\expandafter\expandafter\expandafter\endgroup
      \expandafter\ifx\csname documentclass\endcsname\relax
      \else
        \RequirePackage[pdftex]{color}\relax
      \fi
    \fi
    \begingroup\expandafter\expandafter\expandafter\endgroup
    \expandafter\ifx\csname @pdfcolorstack\endcsname\relax
      \pdfcolAvailablefalse
      \PDFCOL@Message{%
        Interface disabled because `pdftex.def'\MessageBreak
        is not loaded or it is too old.\MessageBreak
        Required is version 0.04b or greater%
      }%
    \fi
  \fi
\fi
%    \end{macrocode}
%
%    \begin{macrocode}
\let\pdfcolAvailabletrue\relax
\let\pdfcolAvailablefalse\relax
%    \end{macrocode}
%
% \subsection{Enabled interface macros}
%
%    \begin{macrocode}
\ifpdfcolAvailable
%    \end{macrocode}
%
%    \begin{macro}{\pdfcolErrorNoStacks}
%    \begin{macrocode}
  \let\pdfcolErrorNoStacks\relax
%    \end{macrocode}
%    \end{macro}
%
%    \begin{macro}{\pdfcol@Value}
%    \begin{macrocode}
  \expandafter\ifx\csname pdfcol@Value\endcsname\relax
    \def\pdfcol@Value{0 g 0 G}%
  \fi
%    \end{macrocode}
%    \end{macro}
%
%    \begin{macro}{\pdfcol@LiteralModifier}
%    \begin{macrocode}
  \expandafter\ifx\csname pdfcol@LiteralModifier\endcsname\relax
    \def\pdfcol@LiteralModifier{direct}%
  \fi
%    \end{macrocode}
%    \end{macro}
%
%    \begin{macro}{\pdfcolInitStack}
%    \begin{macrocode}
  \def\pdfcolInitStack#1{%
    \expandafter\ifx\csname pdfcol@Stack@#1\endcsname\relax
      \global\expandafter\chardef\csname pdfcol@Stack@#1\endcsname=%
          \pdfcolorstackinit\pdfcol@LiteralModifier{\pdfcol@Value}%
          \relax
      \@PackageInfo{pdfcol}{%
        New color stack `#1' = \number\csname pdfcol@Stack@#1\endcsname
      }%
    \else
      \@PackageError{pdfcol}{%
        Stack `#1' is already defined%
      }\@ehc
    \fi
  }%
%    \end{macrocode}
%    \end{macro}
%
%    \begin{macro}{\pdfcolIfStackExists}
%    \begin{macrocode}
  \def\pdfcolIfStackExists#1{%
    \expandafter\ifx\csname pdfcol@Stack@#1\endcsname\relax
      \expandafter\@secondoftwo
    \else
      \expandafter\@firstoftwo
    \fi
  }%
%    \end{macrocode}
%    \end{macro}
%    \begin{macro}{\@firstoftwo}
%    \begin{macrocode}
  \expandafter\ifx\csname @firstoftwo\endcsname\relax
    \long\def\@firstoftwo#1#2{#1}%
  \fi
%    \end{macrocode}
%    \end{macro}
%    \begin{macro}{\@secondoftwo}
%    \begin{macrocode}
  \expandafter\ifx\csname @secondoftwo\endcsname\relax
    \long\def\@secondoftwo#1#2{#2}%
  \fi
%    \end{macrocode}
%    \end{macro}
%
%    \begin{macro}{\pdfcolSwitchStack}
%    \begin{macrocode}
  \def\pdfcolSwitchStack#1{%
    \pdfcolIfStackExists{#1}{%
      \expandafter\let\expandafter\@pdfcolorstack
                      \csname pdfcol@Stack@#1\endcsname
    }{%
      \pdfcol@ErrorNoStack{#1}%
    }%
  }%
%    \end{macrocode}
%    \end{macro}
%
%    \begin{macro}{\pdfcolSetCurrentColor}
%    \begin{macrocode}
  \def\pdfcolSetCurrentColor{%
    \pdfcolorstack\@pdfcolorstack set{\current@color}%
  }%
%    \end{macrocode}
%    \end{macro}
%
%    \begin{macro}{\pdfcolSetCurrent}
%    \begin{macrocode}
  \def\pdfcolSetCurrent#1{%
    \ifx\\#1\\%
      \pdfcolorstack\@pdfcolorstack current\relax
    \else
      \pdfcolIfStackExists{#1}{%
        \pdfcolorstack\csname pdfcol@Stack@#1\endcsname current\relax
      }{%
        \pdfcol@ErrorNoStack{#1}%
      }%
    \fi
  }%
%    \end{macrocode}
%    \end{macro}
%
%    \begin{macro}{\pdfcol@ErrorNoStack}
%    \begin{macrocode}
  \def\pdfcol@ErrorNoStack#1{%
    \@PackageError{pdfcol}{Stack `#1' does not exists}\@ehc
  }%
%    \end{macrocode}
%    \end{macro}
%
% \subsection{Disabled interface macros}
%
%    \begin{macrocode}
\else
%    \end{macrocode}
%
%    \begin{macro}{\pdfcolErrorNoStacks}
%    \begin{macrocode}
  \def\pdfcolErrorNoStacks{%
    \@PackageError{pdfcol}{%
      Color stacks are not available%
    }{%
      Update pdfTeX (1.40) and `pdftex.def' (0.04b) %
          if necessary.\MessageBreak
      Ensure that `pdftex.def' is loaded %
          (package `color' or `xcolor').\MessageBreak
      Further messages can be found in TeX's %
          protocol file `\jobname.log'.\MessageBreak
      \MessageBreak
      \@ehc
    }%
    \global\let\pdfcolErrorNoStacks\relax
  }%
%    \end{macrocode}
%    \end{macro}
%
%    \begin{macro}{\PDFCOL@Disabled}
%    \begin{macrocode}
  \def\PDFCOL@Disabled{%
    \PDFCOL@Message{%
      pdfTeX's color stacks are not available%
    }%
    \global\let\PDFCOL@Disabled\relax
  }%
%    \end{macrocode}
%    \end{macro}
%
%    \begin{macro}{\pdfcolInitStack}
%    \begin{macrocode}
  \def\pdfcolInitStack#1{%
    \PDFCOL@Disabled
  }%
%    \end{macrocode}
%    \end{macro}
%
%    \begin{macro}{\pdfcolIfStackExists}
%    \begin{macrocode}
  \long\def\pdfcolIfStackExists#1#2#3{#3}%
%    \end{macrocode}
%    \end{macro}
%
%    \begin{macro}{\pdfcolSwitchStack}
%    \begin{macrocode}
  \def\pdfcolSwitchStack#1{%
    \PDFCOL@Disabled
  }%
%    \end{macrocode}
%    \end{macro}
%
%    \begin{macro}{\pdfcolSetCurrentColor}
%    \begin{macrocode}
  \def\pdfcolSetCurrentColor{%
    \PDFCOL@Disabled
  }%
%    \end{macrocode}
%    \end{macro}
%
%    \begin{macro}{\pdfcolSetCurrent}
%    \begin{macrocode}
  \def\pdfcolSetCurrent#1{%
    \PDFCOL@Disabled
  }%
%    \end{macrocode}
%    \end{macro}
%    \begin{macrocode}
\fi
%    \end{macrocode}
%
%    \begin{macrocode}
\PDFCOL@AtEnd%
%</package>
%    \end{macrocode}
%% \section{Installation}
%
% \subsection{Download}
%
% \paragraph{Package.} This package is available on
% CTAN\footnote{\CTANpkg{pdfcol}}:
% \begin{description}
% \item[\CTAN{macros/latex/contrib/oberdiek/pdfcol.dtx}] The source file.
% \item[\CTAN{macros/latex/contrib/oberdiek/pdfcol.pdf}] Documentation.
% \end{description}
%
%
% \paragraph{Bundle.} All the packages of the bundle `oberdiek'
% are also available in a TDS compliant ZIP archive. There
% the packages are already unpacked and the documentation files
% are generated. The files and directories obey the TDS standard.
% \begin{description}
% \item[\CTANinstall{install/macros/latex/contrib/oberdiek.tds.zip}]
% \end{description}
% \emph{TDS} refers to the standard ``A Directory Structure
% for \TeX\ Files'' (\CTANpkg{tds}). Directories
% with \xfile{texmf} in their name are usually organized this way.
%
% \subsection{Bundle installation}
%
% \paragraph{Unpacking.} Unpack the \xfile{oberdiek.tds.zip} in the
% TDS tree (also known as \xfile{texmf} tree) of your choice.
% Example (linux):
% \begin{quote}
%   |unzip oberdiek.tds.zip -d ~/texmf|
% \end{quote}
%
% \subsection{Package installation}
%
% \paragraph{Unpacking.} The \xfile{.dtx} file is a self-extracting
% \docstrip\ archive. The files are extracted by running the
% \xfile{.dtx} through \plainTeX:
% \begin{quote}
%   \verb|tex pdfcol.dtx|
% \end{quote}
%
% \paragraph{TDS.} Now the different files must be moved into
% the different directories in your installation TDS tree
% (also known as \xfile{texmf} tree):
% \begin{quote}
% \def\t{^^A
% \begin{tabular}{@{}>{\ttfamily}l@{ $\rightarrow$ }>{\ttfamily}l@{}}
%   pdfcol.sty & tex/generic/oberdiek/pdfcol.sty\\
%   pdfcol.pdf & doc/latex/oberdiek/pdfcol.pdf\\
%   pdfcol.dtx & source/latex/oberdiek/pdfcol.dtx\\
% \end{tabular}^^A
% }^^A
% \sbox0{\t}^^A
% \ifdim\wd0>\linewidth
%   \begingroup
%     \advance\linewidth by\leftmargin
%     \advance\linewidth by\rightmargin
%   \edef\x{\endgroup
%     \def\noexpand\lw{\the\linewidth}^^A
%   }\x
%   \def\lwbox{^^A
%     \leavevmode
%     \hbox to \linewidth{^^A
%       \kern-\leftmargin\relax
%       \hss
%       \usebox0
%       \hss
%       \kern-\rightmargin\relax
%     }^^A
%   }^^A
%   \ifdim\wd0>\lw
%     \sbox0{\small\t}^^A
%     \ifdim\wd0>\linewidth
%       \ifdim\wd0>\lw
%         \sbox0{\footnotesize\t}^^A
%         \ifdim\wd0>\linewidth
%           \ifdim\wd0>\lw
%             \sbox0{\scriptsize\t}^^A
%             \ifdim\wd0>\linewidth
%               \ifdim\wd0>\lw
%                 \sbox0{\tiny\t}^^A
%                 \ifdim\wd0>\linewidth
%                   \lwbox
%                 \else
%                   \usebox0
%                 \fi
%               \else
%                 \lwbox
%               \fi
%             \else
%               \usebox0
%             \fi
%           \else
%             \lwbox
%           \fi
%         \else
%           \usebox0
%         \fi
%       \else
%         \lwbox
%       \fi
%     \else
%       \usebox0
%     \fi
%   \else
%     \lwbox
%   \fi
% \else
%   \usebox0
% \fi
% \end{quote}
% If you have a \xfile{docstrip.cfg} that configures and enables \docstrip's
% TDS installing feature, then some files can already be in the right
% place, see the documentation of \docstrip.
%
% \subsection{Refresh file name databases}
%
% If your \TeX~distribution
% (\TeX\,Live, \mikTeX, \dots) relies on file name databases, you must refresh
% these. For example, \TeX\,Live\ users run \verb|texhash| or
% \verb|mktexlsr|.
%
% \subsection{Some details for the interested}
%
% \paragraph{Unpacking with \LaTeX.}
% The \xfile{.dtx} chooses its action depending on the format:
% \begin{description}
% \item[\plainTeX:] Run \docstrip\ and extract the files.
% \item[\LaTeX:] Generate the documentation.
% \end{description}
% If you insist on using \LaTeX\ for \docstrip\ (really,
% \docstrip\ does not need \LaTeX), then inform the autodetect routine
% about your intention:
% \begin{quote}
%   \verb|latex \let\install=y% \iffalse meta-comment
%
% File: pdfcol.dtx
% Version: 2018/11/01 v1.5
% Info: Handle new color stacks for pdfTeX
%
% Copyright (C)
%    2007 Heiko Oberdiek
%    2016-2019 Oberdiek Package Support Group
%    https://github.com/ho-tex/oberdiek/issues
%
% This work may be distributed and/or modified under the
% conditions of the LaTeX Project Public License, either
% version 1.3c of this license or (at your option) any later
% version. This version of this license is in
%    https://www.latex-project.org/lppl/lppl-1-3c.txt
% and the latest version of this license is in
%    https://www.latex-project.org/lppl.txt
% and version 1.3 or later is part of all distributions of
% LaTeX version 2005/12/01 or later.
%
% This work has the LPPL maintenance status "maintained".
%
% The Current Maintainers of this work are
% Heiko Oberdiek and the Oberdiek Package Support Group
% https://github.com/ho-tex/oberdiek/issues
%
% The Base Interpreter refers to any `TeX-Format',
% because some files are installed in TDS:tex/generic//.
%
% This work consists of the main source file pdfcol.dtx
% and the derived files
%    pdfcol.sty, pdfcol.pdf, pdfcol.ins, pdfcol.drv, pdfcol-test1.tex,
%    pdfcol-test2.tex, pdfcol-test3.tex, pdfcol-test4.tex.
%
% Distribution:
%    CTAN:macros/latex/contrib/oberdiek/pdfcol.dtx
%    CTAN:macros/latex/contrib/oberdiek/pdfcol.pdf
%
% Unpacking:
%    (a) If pdfcol.ins is present:
%           tex pdfcol.ins
%    (b) Without pdfcol.ins:
%           tex pdfcol.dtx
%    (c) If you insist on using LaTeX
%           latex \let\install=y\input{pdfcol.dtx}
%        (quote the arguments according to the demands of your shell)
%
% Documentation:
%    (a) If pdfcol.drv is present:
%           latex pdfcol.drv
%    (b) Without pdfcol.drv:
%           latex pdfcol.dtx; ...
%    The class ltxdoc loads the configuration file ltxdoc.cfg
%    if available. Here you can specify further options, e.g.
%    use A4 as paper format:
%       \PassOptionsToClass{a4paper}{article}
%
%    Programm calls to get the documentation (example):
%       pdflatex pdfcol.dtx
%       makeindex -s gind.ist pdfcol.idx
%       pdflatex pdfcol.dtx
%       makeindex -s gind.ist pdfcol.idx
%       pdflatex pdfcol.dtx
%
% Installation:
%    TDS:tex/generic/oberdiek/pdfcol.sty
%    TDS:doc/latex/oberdiek/pdfcol.pdf
%    TDS:source/latex/oberdiek/pdfcol.dtx
%
%<*ignore>
\begingroup
  \catcode123=1 %
  \catcode125=2 %
  \def\x{LaTeX2e}%
\expandafter\endgroup
\ifcase 0\ifx\install y1\fi\expandafter
         \ifx\csname processbatchFile\endcsname\relax\else1\fi
         \ifx\fmtname\x\else 1\fi\relax
\else\csname fi\endcsname
%</ignore>
%<*install>
\input docstrip.tex
\Msg{************************************************************************}
\Msg{* Installation}
\Msg{* Package: pdfcol 2018/11/01 v1.5 Handle new color stacks for pdfTeX (HO)}
\Msg{************************************************************************}

\keepsilent
\askforoverwritefalse

\let\MetaPrefix\relax
\preamble

This is a generated file.

Project: pdfcol
Version: 2018/11/01 v1.5

Copyright (C)
   2007 Heiko Oberdiek
   2016-2019 Oberdiek Package Support Group

This work may be distributed and/or modified under the
conditions of the LaTeX Project Public License, either
version 1.3c of this license or (at your option) any later
version. This version of this license is in
   https://www.latex-project.org/lppl/lppl-1-3c.txt
and the latest version of this license is in
   https://www.latex-project.org/lppl.txt
and version 1.3 or later is part of all distributions of
LaTeX version 2005/12/01 or later.

This work has the LPPL maintenance status "maintained".

The Current Maintainers of this work are
Heiko Oberdiek and the Oberdiek Package Support Group
https://github.com/ho-tex/oberdiek/issues


The Base Interpreter refers to any `TeX-Format',
because some files are installed in TDS:tex/generic//.

This work consists of the main source file pdfcol.dtx
and the derived files
   pdfcol.sty, pdfcol.pdf, pdfcol.ins, pdfcol.drv, pdfcol-test1.tex,
   pdfcol-test2.tex, pdfcol-test3.tex, pdfcol-test4.tex.

\endpreamble
\let\MetaPrefix\DoubleperCent

\generate{%
  \file{pdfcol.ins}{\from{pdfcol.dtx}{install}}%
  \file{pdfcol.drv}{\from{pdfcol.dtx}{driver}}%
  \usedir{tex/generic/oberdiek}%
  \file{pdfcol.sty}{\from{pdfcol.dtx}{package}}%
%  \usedir{doc/latex/oberdiek/test}%
%  \file{pdfcol-test1.tex}{\from{pdfcol.dtx}{test1}}%
%  \file{pdfcol-test2.tex}{\from{pdfcol.dtx}{test2}}%
%  \file{pdfcol-test3.tex}{\from{pdfcol.dtx}{test3}}%
%  \file{pdfcol-test4.tex}{\from{pdfcol.dtx}{test4}}%
}

\catcode32=13\relax% active space
\let =\space%
\Msg{************************************************************************}
\Msg{*}
\Msg{* To finish the installation you have to move the following}
\Msg{* file into a directory searched by TeX:}
\Msg{*}
\Msg{*     pdfcol.sty}
\Msg{*}
\Msg{* To produce the documentation run the file `pdfcol.drv'}
\Msg{* through LaTeX.}
\Msg{*}
\Msg{* Happy TeXing!}
\Msg{*}
\Msg{************************************************************************}

\endbatchfile
%</install>
%<*ignore>
\fi
%</ignore>
%<*driver>
\NeedsTeXFormat{LaTeX2e}
\ProvidesFile{pdfcol.drv}%
  [2018/11/01 v1.5 Handle new color stacks for pdfTeX (HO)]%
\documentclass{ltxdoc}
\usepackage{holtxdoc}[2011/11/22]
\begin{document}
  \DocInput{pdfcol.dtx}%
\end{document}
%</driver>
% \fi
%
%
%
% \GetFileInfo{pdfcol.drv}
%
% \title{The \xpackage{pdfcol} package}
% \date{2018/11/01 v1.5}
% \author{Heiko Oberdiek\thanks
% {Please report any issues at \url{https://github.com/ho-tex/oberdiek/issues}}}
%
% \maketitle
%
% \begin{abstract}
% Since version 1.40 \pdfTeX\ supports color stacks.
% The driver file \xfile{pdftex.def} for package \xpackage{color}
% defines and uses a main color stack since version v0.04b.
% Package \xpackage{pdfcol} is intended for package writers.
% It defines macros for setting and maintaining new color stacks.
% \end{abstract}
%
% \tableofcontents
%
% \section{Documentation}
%
% Version 1.40 of \pdfTeX\ adds new primitives \cs{pdfcolorstackinit}
% and \cs{pdfcolorstack}. Now color stacks can be defined and used.
% A main color stack is maintained by the driver file \xfile{pdftex.def}
% similar to dvips or dvipdfm. However the number of color stacks
% is not limited to one in \pdfTeX. Thus further color problems
% can now be solved, such as footnotes across pages or text
% that is set in parallel columns (e.g. packages \xpackage{parallel}
% or \xpackage{parcolumn}). Unlike the main color stack,
% the support by additional color stacks cannot be done in
% a transparent manner.
%
% This package \xpackage{pdfcol} provides an easier interface to
% additional color stacks without the need to use the
% low level primitives.
%
% \subsection{Requirements}
% \label{sec:req}
%
% \begin{itemize}
% \item
%   \pdfTeX\ 1.40 or greater.
% \item
%   \pdfTeX in PDF mode. (I don't know a DVI driver that
%   support several color stacks.)
% \item
%   \xfile{pdftex.def} 2007/01/02 v0.04b.
% \end{itemize}
% Package \xpackage{pdfcol} checks the requirements and
% sets switch \cs{ifpdfcolAvailable} accordingly.
%
% \subsection{Interface}
%
% \begin{declcs}{ifpdfcolAvailable}
% \end{declcs}
% If the requirements of section \ref{sec:req} are met the
% switch \cs{ifpdfcolAvailable} behaves as \cs{iftrue}.
% Otherwise the other interface macros in this section will
% be disabled with a message. Also the first use of such a
% macro will print a message. The messages are print to
% the \xext{log} file only if \pdfTeX\ is not used in PDF mode.
%
% \begin{declcs}{pdfcolErrorNoStacks}
% \end{declcs}
% The first call of \cs{pdfcolErrorNoStacks} prints an error
% message, if color stacks are not available.
%
% \begin{declcs}{pdfcolInitStack} \M{name}
% \end{declcs}
% A new color stack is initialized by \cs{pdfcolInitStack}.
% The \meta{name} is used for indentifying the stack. It usually
% consists of letters and digits. (The name must survive a \cs{csname}.)
%
% The intension of the macro is the definition of an additional
% color stack. Thus the stack is not page bounded like the
% main color stack. Black (\texttt{0 g 0 G}) is used as initial
% color value. And colors are written with modifier \texttt{direct}
% that means without setting the current transfer matrix and changing
% the current point (see documentation of \pdfTeX\ for
% |\pdfliteral direct{...}|).
%
% \begin{declcs}{pdfcolIfStackExists} \M{name} \M{then} \M{else}
% \end{declcs}
% Macro \cs{pdfcolIfStackExists} checks whether color stack \meta{name}
% exists. In case of success argument \meta{then} is executed
% and \meta{else} otherwise.
%
% \begin{declcs}{pdfcolSwitchStack} \M{name}
% \end{declcs}
% Macro \cs{pdfcolSwitchStack} switches the color stack. The color macros
% of package \xpackage{color} (or \xpackage{xcolor}) now uses the
% new color stack with name \meta{name}.
%
% \begin{declcs}{pdfcolSetCurrentColor}
% \end{declcs}
% Macro \cs{pdfcolSetCurrentColor} replaces the topmost
% entry of the stack by the current color (\cs{current@color}).
%
% \begin{declcs}{pdfcolSetCurrent} \M{name}
% \end{declcs}
% Macro \cs{pdfcolSetCurrent} sets the color that is read in
% the top-most entry of color stack \meta{name}. If \meta{name}
% is empty, the default color stack is used.
%
% \StopEventually{
% }
%
% \section{Implementation}
%
%    \begin{macrocode}
%<*package>
%    \end{macrocode}
%
% \subsection{Reload check and package identification}
%    Reload check, especially if the package is not used with \LaTeX.
%    \begin{macrocode}
\begingroup\catcode61\catcode48\catcode32=10\relax%
  \catcode13=5 % ^^M
  \endlinechar=13 %
  \catcode35=6 % #
  \catcode39=12 % '
  \catcode44=12 % ,
  \catcode45=12 % -
  \catcode46=12 % .
  \catcode58=12 % :
  \catcode64=11 % @
  \catcode123=1 % {
  \catcode125=2 % }
  \expandafter\let\expandafter\x\csname ver@pdfcol.sty\endcsname
  \ifx\x\relax % plain-TeX, first loading
  \else
    \def\empty{}%
    \ifx\x\empty % LaTeX, first loading,
      % variable is initialized, but \ProvidesPackage not yet seen
    \else
      \expandafter\ifx\csname PackageInfo\endcsname\relax
        \def\x#1#2{%
          \immediate\write-1{Package #1 Info: #2.}%
        }%
      \else
        \def\x#1#2{\PackageInfo{#1}{#2, stopped}}%
      \fi
      \x{pdfcol}{The package is already loaded}%
      \aftergroup\endinput
    \fi
  \fi
\endgroup%
%    \end{macrocode}
%    Package identification:
%    \begin{macrocode}
\begingroup\catcode61\catcode48\catcode32=10\relax%
  \catcode13=5 % ^^M
  \endlinechar=13 %
  \catcode35=6 % #
  \catcode39=12 % '
  \catcode40=12 % (
  \catcode41=12 % )
  \catcode44=12 % ,
  \catcode45=12 % -
  \catcode46=12 % .
  \catcode47=12 % /
  \catcode58=12 % :
  \catcode64=11 % @
  \catcode91=12 % [
  \catcode93=12 % ]
  \catcode123=1 % {
  \catcode125=2 % }
  \expandafter\ifx\csname ProvidesPackage\endcsname\relax
    \def\x#1#2#3[#4]{\endgroup
      \immediate\write-1{Package: #3 #4}%
      \xdef#1{#4}%
    }%
  \else
    \def\x#1#2[#3]{\endgroup
      #2[{#3}]%
      \ifx#1\@undefined
        \xdef#1{#3}%
      \fi
      \ifx#1\relax
        \xdef#1{#3}%
      \fi
    }%
  \fi
\expandafter\x\csname ver@pdfcol.sty\endcsname
\ProvidesPackage{pdfcol}%
  [2018/11/01 v1.5 Handle new color stacks for pdfTeX (HO)]%
%    \end{macrocode}
%
% \subsection{Catcodes}
%
%    \begin{macrocode}
\begingroup\catcode61\catcode48\catcode32=10\relax%
  \catcode13=5 % ^^M
  \endlinechar=13 %
  \catcode123=1 % {
  \catcode125=2 % }
  \catcode64=11 % @
  \def\x{\endgroup
    \expandafter\edef\csname PDFCOL@AtEnd\endcsname{%
      \endlinechar=\the\endlinechar\relax
      \catcode13=\the\catcode13\relax
      \catcode32=\the\catcode32\relax
      \catcode35=\the\catcode35\relax
      \catcode61=\the\catcode61\relax
      \catcode64=\the\catcode64\relax
      \catcode123=\the\catcode123\relax
      \catcode125=\the\catcode125\relax
    }%
  }%
\x\catcode61\catcode48\catcode32=10\relax%
\catcode13=5 % ^^M
\endlinechar=13 %
\catcode35=6 % #
\catcode64=11 % @
\catcode123=1 % {
\catcode125=2 % }
\def\TMP@EnsureCode#1#2{%
  \edef\PDFCOL@AtEnd{%
    \PDFCOL@AtEnd
    \catcode#1=\the\catcode#1\relax
  }%
  \catcode#1=#2\relax
}
\TMP@EnsureCode{39}{12}% '
\TMP@EnsureCode{40}{12}% (
\TMP@EnsureCode{41}{12}% )
\TMP@EnsureCode{43}{12}% +
\TMP@EnsureCode{44}{12}% ,
\TMP@EnsureCode{46}{12}% .
\TMP@EnsureCode{47}{12}% /
\TMP@EnsureCode{91}{12}% [
\TMP@EnsureCode{93}{12}% ]
\TMP@EnsureCode{96}{12}% `
\edef\PDFCOL@AtEnd{\PDFCOL@AtEnd\noexpand\endinput}
%    \end{macrocode}
%
% \subsection{Check requirements}
%
%    \begin{macro}{\PDFCOL@RequirePackage}
%    \begin{macrocode}
\begingroup\expandafter\expandafter\expandafter\endgroup
\expandafter\ifx\csname RequirePackage\endcsname\relax
  \def\PDFCOL@RequirePackage#1[#2]{\input #1.sty\relax}%
\else
  \def\PDFCOL@RequirePackage#1[#2]{%
    \RequirePackage{#1}[{#2}]%
  }%
\fi
%    \end{macrocode}
%    \end{macro}
%
% LuaTeX Compatability
%    \begin{macrocode}
\ifx\pdfextension\@undefined\else
 \def\pdfcolorstackinit {\pdffeedback colorstackinit}
 \protected\def\pdfcolorstack     {\pdfextension colorstack}
\fi
%    \end{macrocode}
%
%    \begin{macrocode}
\PDFCOL@RequirePackage{ltxcmds}[2010/03/01]
%    \end{macrocode}
%
%    \begin{macro}{ifpdfcolAvailable}
%    \begin{macrocode}
\ltx@newif\ifpdfcolAvailable
\pdfcolAvailabletrue
%    \end{macrocode}
%    \end{macro}
%
% \subsubsection{Check package \xpackage{luacolor}}
%
%    \begin{macrocode}
\ltx@newif\ifPDFCOL@luacolor
\begingroup\expandafter\expandafter\expandafter\endgroup
\expandafter\ifx\csname ver@luacolor.sty\endcsname\relax
  \PDFCOL@luacolorfalse
\else
  \PDFCOL@luacolortrue
\fi
%    \end{macrocode}
%
% \subsubsection{Check PDF mode}
%
%    \begin{macrocode}
\PDFCOL@RequirePackage{infwarerr}[2007/09/09]
\PDFCOL@RequirePackage{ifpdf}[2007/09/09]
\ifcase\ifpdf\ifPDFCOL@luacolor 1\fi\else 1\fi0 %
  \def\PDFCOL@Message{%
    \@PackageWarningNoLine{pdfcol}%
  }%
\else
  \pdfcolAvailablefalse
  \def\PDFCOL@Message{%
    \@PackageInfoNoLine{pdfcol}%
  }%
  \PDFCOL@Message{%
    Interface disabled because of %
    \ifPDFCOL@luacolor
      package `luacolor'%
    \else
      missing PDF mode of pdfTeX%
    \fi
  }%
\fi
%    \end{macrocode}
%
% \subsubsection{Check version of \pdfTeX}
%
%    \begin{macrocode}
\ifpdfcolAvailable
  \begingroup\expandafter\expandafter\expandafter\endgroup
  \expandafter\ifx\csname pdfcolorstack\endcsname\relax
    \pdfcolAvailablefalse
    \PDFCOL@Message{%
      Interface disabled because of too old pdfTeX.\MessageBreak
      Required is version 1.40+ for \string\pdfcolorstack
    }%
  \fi
\fi
\ifpdfcolAvailable
  \begingroup\expandafter\expandafter\expandafter\endgroup
  \expandafter\ifx\csname pdfcolorstack\endcsname\relax
    \pdfcolAvailablefalse
    \PDFCOL@Message{%
      Interface disabled because of too old pdfTeX.\MessageBreak
      Required is version 1.40+ for \string\pdfcolorstackinit
    }%
  \fi
\fi
%    \end{macrocode}
%
% \subsubsection{Check \xfile{pdftex.def}}
%
%    \begin{macrocode}
\ifpdfcolAvailable
  \begingroup\expandafter\expandafter\expandafter\endgroup
  \expandafter\ifx\csname @pdfcolorstack\endcsname\relax
%    \end{macrocode}
%    Try to load package color if it is not yet loaded (\LaTeX\ case).
%    \begin{macrocode}
    \begingroup\expandafter\expandafter\expandafter\endgroup
    \expandafter\ifx\csname ver@color.sty\endcsname\relax
      \begingroup\expandafter\expandafter\expandafter\endgroup
      \expandafter\ifx\csname documentclass\endcsname\relax
      \else
        \RequirePackage[pdftex]{color}\relax
      \fi
    \fi
    \begingroup\expandafter\expandafter\expandafter\endgroup
    \expandafter\ifx\csname @pdfcolorstack\endcsname\relax
      \pdfcolAvailablefalse
      \PDFCOL@Message{%
        Interface disabled because `pdftex.def'\MessageBreak
        is not loaded or it is too old.\MessageBreak
        Required is version 0.04b or greater%
      }%
    \fi
  \fi
\fi
%    \end{macrocode}
%
%    \begin{macrocode}
\let\pdfcolAvailabletrue\relax
\let\pdfcolAvailablefalse\relax
%    \end{macrocode}
%
% \subsection{Enabled interface macros}
%
%    \begin{macrocode}
\ifpdfcolAvailable
%    \end{macrocode}
%
%    \begin{macro}{\pdfcolErrorNoStacks}
%    \begin{macrocode}
  \let\pdfcolErrorNoStacks\relax
%    \end{macrocode}
%    \end{macro}
%
%    \begin{macro}{\pdfcol@Value}
%    \begin{macrocode}
  \expandafter\ifx\csname pdfcol@Value\endcsname\relax
    \def\pdfcol@Value{0 g 0 G}%
  \fi
%    \end{macrocode}
%    \end{macro}
%
%    \begin{macro}{\pdfcol@LiteralModifier}
%    \begin{macrocode}
  \expandafter\ifx\csname pdfcol@LiteralModifier\endcsname\relax
    \def\pdfcol@LiteralModifier{direct}%
  \fi
%    \end{macrocode}
%    \end{macro}
%
%    \begin{macro}{\pdfcolInitStack}
%    \begin{macrocode}
  \def\pdfcolInitStack#1{%
    \expandafter\ifx\csname pdfcol@Stack@#1\endcsname\relax
      \global\expandafter\chardef\csname pdfcol@Stack@#1\endcsname=%
          \pdfcolorstackinit\pdfcol@LiteralModifier{\pdfcol@Value}%
          \relax
      \@PackageInfo{pdfcol}{%
        New color stack `#1' = \number\csname pdfcol@Stack@#1\endcsname
      }%
    \else
      \@PackageError{pdfcol}{%
        Stack `#1' is already defined%
      }\@ehc
    \fi
  }%
%    \end{macrocode}
%    \end{macro}
%
%    \begin{macro}{\pdfcolIfStackExists}
%    \begin{macrocode}
  \def\pdfcolIfStackExists#1{%
    \expandafter\ifx\csname pdfcol@Stack@#1\endcsname\relax
      \expandafter\@secondoftwo
    \else
      \expandafter\@firstoftwo
    \fi
  }%
%    \end{macrocode}
%    \end{macro}
%    \begin{macro}{\@firstoftwo}
%    \begin{macrocode}
  \expandafter\ifx\csname @firstoftwo\endcsname\relax
    \long\def\@firstoftwo#1#2{#1}%
  \fi
%    \end{macrocode}
%    \end{macro}
%    \begin{macro}{\@secondoftwo}
%    \begin{macrocode}
  \expandafter\ifx\csname @secondoftwo\endcsname\relax
    \long\def\@secondoftwo#1#2{#2}%
  \fi
%    \end{macrocode}
%    \end{macro}
%
%    \begin{macro}{\pdfcolSwitchStack}
%    \begin{macrocode}
  \def\pdfcolSwitchStack#1{%
    \pdfcolIfStackExists{#1}{%
      \expandafter\let\expandafter\@pdfcolorstack
                      \csname pdfcol@Stack@#1\endcsname
    }{%
      \pdfcol@ErrorNoStack{#1}%
    }%
  }%
%    \end{macrocode}
%    \end{macro}
%
%    \begin{macro}{\pdfcolSetCurrentColor}
%    \begin{macrocode}
  \def\pdfcolSetCurrentColor{%
    \pdfcolorstack\@pdfcolorstack set{\current@color}%
  }%
%    \end{macrocode}
%    \end{macro}
%
%    \begin{macro}{\pdfcolSetCurrent}
%    \begin{macrocode}
  \def\pdfcolSetCurrent#1{%
    \ifx\\#1\\%
      \pdfcolorstack\@pdfcolorstack current\relax
    \else
      \pdfcolIfStackExists{#1}{%
        \pdfcolorstack\csname pdfcol@Stack@#1\endcsname current\relax
      }{%
        \pdfcol@ErrorNoStack{#1}%
      }%
    \fi
  }%
%    \end{macrocode}
%    \end{macro}
%
%    \begin{macro}{\pdfcol@ErrorNoStack}
%    \begin{macrocode}
  \def\pdfcol@ErrorNoStack#1{%
    \@PackageError{pdfcol}{Stack `#1' does not exists}\@ehc
  }%
%    \end{macrocode}
%    \end{macro}
%
% \subsection{Disabled interface macros}
%
%    \begin{macrocode}
\else
%    \end{macrocode}
%
%    \begin{macro}{\pdfcolErrorNoStacks}
%    \begin{macrocode}
  \def\pdfcolErrorNoStacks{%
    \@PackageError{pdfcol}{%
      Color stacks are not available%
    }{%
      Update pdfTeX (1.40) and `pdftex.def' (0.04b) %
          if necessary.\MessageBreak
      Ensure that `pdftex.def' is loaded %
          (package `color' or `xcolor').\MessageBreak
      Further messages can be found in TeX's %
          protocol file `\jobname.log'.\MessageBreak
      \MessageBreak
      \@ehc
    }%
    \global\let\pdfcolErrorNoStacks\relax
  }%
%    \end{macrocode}
%    \end{macro}
%
%    \begin{macro}{\PDFCOL@Disabled}
%    \begin{macrocode}
  \def\PDFCOL@Disabled{%
    \PDFCOL@Message{%
      pdfTeX's color stacks are not available%
    }%
    \global\let\PDFCOL@Disabled\relax
  }%
%    \end{macrocode}
%    \end{macro}
%
%    \begin{macro}{\pdfcolInitStack}
%    \begin{macrocode}
  \def\pdfcolInitStack#1{%
    \PDFCOL@Disabled
  }%
%    \end{macrocode}
%    \end{macro}
%
%    \begin{macro}{\pdfcolIfStackExists}
%    \begin{macrocode}
  \long\def\pdfcolIfStackExists#1#2#3{#3}%
%    \end{macrocode}
%    \end{macro}
%
%    \begin{macro}{\pdfcolSwitchStack}
%    \begin{macrocode}
  \def\pdfcolSwitchStack#1{%
    \PDFCOL@Disabled
  }%
%    \end{macrocode}
%    \end{macro}
%
%    \begin{macro}{\pdfcolSetCurrentColor}
%    \begin{macrocode}
  \def\pdfcolSetCurrentColor{%
    \PDFCOL@Disabled
  }%
%    \end{macrocode}
%    \end{macro}
%
%    \begin{macro}{\pdfcolSetCurrent}
%    \begin{macrocode}
  \def\pdfcolSetCurrent#1{%
    \PDFCOL@Disabled
  }%
%    \end{macrocode}
%    \end{macro}
%    \begin{macrocode}
\fi
%    \end{macrocode}
%
%    \begin{macrocode}
\PDFCOL@AtEnd%
%</package>
%    \end{macrocode}
%% \section{Installation}
%
% \subsection{Download}
%
% \paragraph{Package.} This package is available on
% CTAN\footnote{\CTANpkg{pdfcol}}:
% \begin{description}
% \item[\CTAN{macros/latex/contrib/oberdiek/pdfcol.dtx}] The source file.
% \item[\CTAN{macros/latex/contrib/oberdiek/pdfcol.pdf}] Documentation.
% \end{description}
%
%
% \paragraph{Bundle.} All the packages of the bundle `oberdiek'
% are also available in a TDS compliant ZIP archive. There
% the packages are already unpacked and the documentation files
% are generated. The files and directories obey the TDS standard.
% \begin{description}
% \item[\CTANinstall{install/macros/latex/contrib/oberdiek.tds.zip}]
% \end{description}
% \emph{TDS} refers to the standard ``A Directory Structure
% for \TeX\ Files'' (\CTANpkg{tds}). Directories
% with \xfile{texmf} in their name are usually organized this way.
%
% \subsection{Bundle installation}
%
% \paragraph{Unpacking.} Unpack the \xfile{oberdiek.tds.zip} in the
% TDS tree (also known as \xfile{texmf} tree) of your choice.
% Example (linux):
% \begin{quote}
%   |unzip oberdiek.tds.zip -d ~/texmf|
% \end{quote}
%
% \subsection{Package installation}
%
% \paragraph{Unpacking.} The \xfile{.dtx} file is a self-extracting
% \docstrip\ archive. The files are extracted by running the
% \xfile{.dtx} through \plainTeX:
% \begin{quote}
%   \verb|tex pdfcol.dtx|
% \end{quote}
%
% \paragraph{TDS.} Now the different files must be moved into
% the different directories in your installation TDS tree
% (also known as \xfile{texmf} tree):
% \begin{quote}
% \def\t{^^A
% \begin{tabular}{@{}>{\ttfamily}l@{ $\rightarrow$ }>{\ttfamily}l@{}}
%   pdfcol.sty & tex/generic/oberdiek/pdfcol.sty\\
%   pdfcol.pdf & doc/latex/oberdiek/pdfcol.pdf\\
%   pdfcol.dtx & source/latex/oberdiek/pdfcol.dtx\\
% \end{tabular}^^A
% }^^A
% \sbox0{\t}^^A
% \ifdim\wd0>\linewidth
%   \begingroup
%     \advance\linewidth by\leftmargin
%     \advance\linewidth by\rightmargin
%   \edef\x{\endgroup
%     \def\noexpand\lw{\the\linewidth}^^A
%   }\x
%   \def\lwbox{^^A
%     \leavevmode
%     \hbox to \linewidth{^^A
%       \kern-\leftmargin\relax
%       \hss
%       \usebox0
%       \hss
%       \kern-\rightmargin\relax
%     }^^A
%   }^^A
%   \ifdim\wd0>\lw
%     \sbox0{\small\t}^^A
%     \ifdim\wd0>\linewidth
%       \ifdim\wd0>\lw
%         \sbox0{\footnotesize\t}^^A
%         \ifdim\wd0>\linewidth
%           \ifdim\wd0>\lw
%             \sbox0{\scriptsize\t}^^A
%             \ifdim\wd0>\linewidth
%               \ifdim\wd0>\lw
%                 \sbox0{\tiny\t}^^A
%                 \ifdim\wd0>\linewidth
%                   \lwbox
%                 \else
%                   \usebox0
%                 \fi
%               \else
%                 \lwbox
%               \fi
%             \else
%               \usebox0
%             \fi
%           \else
%             \lwbox
%           \fi
%         \else
%           \usebox0
%         \fi
%       \else
%         \lwbox
%       \fi
%     \else
%       \usebox0
%     \fi
%   \else
%     \lwbox
%   \fi
% \else
%   \usebox0
% \fi
% \end{quote}
% If you have a \xfile{docstrip.cfg} that configures and enables \docstrip's
% TDS installing feature, then some files can already be in the right
% place, see the documentation of \docstrip.
%
% \subsection{Refresh file name databases}
%
% If your \TeX~distribution
% (\TeX\,Live, \mikTeX, \dots) relies on file name databases, you must refresh
% these. For example, \TeX\,Live\ users run \verb|texhash| or
% \verb|mktexlsr|.
%
% \subsection{Some details for the interested}
%
% \paragraph{Unpacking with \LaTeX.}
% The \xfile{.dtx} chooses its action depending on the format:
% \begin{description}
% \item[\plainTeX:] Run \docstrip\ and extract the files.
% \item[\LaTeX:] Generate the documentation.
% \end{description}
% If you insist on using \LaTeX\ for \docstrip\ (really,
% \docstrip\ does not need \LaTeX), then inform the autodetect routine
% about your intention:
% \begin{quote}
%   \verb|latex \let\install=y\input{pdfcol.dtx}|
% \end{quote}
% Do not forget to quote the argument according to the demands
% of your shell.
%
% \paragraph{Generating the documentation.}
% You can use both the \xfile{.dtx} or the \xfile{.drv} to generate
% the documentation. The process can be configured by the
% configuration file \xfile{ltxdoc.cfg}. For instance, put this
% line into this file, if you want to have A4 as paper format:
% \begin{quote}
%   \verb|\PassOptionsToClass{a4paper}{article}|
% \end{quote}
% An example follows how to generate the
% documentation with pdf\LaTeX:
% \begin{quote}
%\begin{verbatim}
%pdflatex pdfcol.dtx
%makeindex -s gind.ist pdfcol.idx
%pdflatex pdfcol.dtx
%makeindex -s gind.ist pdfcol.idx
%pdflatex pdfcol.dtx
%\end{verbatim}
% \end{quote}
%
% \begin{History}
%   \begin{Version}{2007/09/09 v1.0}
%   \item
%     First version.
%   \end{Version}
%   \begin{Version}{2007/12/09 v1.1}
%   \item
%     \cs{pdfcolSetCurrentColor} added.
%   \end{Version}
%   \begin{Version}{2007/12/12 v1.2}
%   \item
%     Detection for package \xpackage{luacolor} added.
%   \end{Version}
%   \begin{Version}{2016/05/16 v1.3}
%   \item
%     Documentation updates.
%   \end{Version}
%   \begin{Version}{2016/05/17 v1.4}
%   \item
%     Use luatex85 package for new luatex compatibility
%   \end{Version}
%   \begin{Version}{2018/11/01 v1.5}
%   \item
%     Remove luatex85 dependency
%   \end{Version}
% \end{History}
%
% \PrintIndex
%
% \Finale
\endinput
|
% \end{quote}
% Do not forget to quote the argument according to the demands
% of your shell.
%
% \paragraph{Generating the documentation.}
% You can use both the \xfile{.dtx} or the \xfile{.drv} to generate
% the documentation. The process can be configured by the
% configuration file \xfile{ltxdoc.cfg}. For instance, put this
% line into this file, if you want to have A4 as paper format:
% \begin{quote}
%   \verb|\PassOptionsToClass{a4paper}{article}|
% \end{quote}
% An example follows how to generate the
% documentation with pdf\LaTeX:
% \begin{quote}
%\begin{verbatim}
%pdflatex pdfcol.dtx
%makeindex -s gind.ist pdfcol.idx
%pdflatex pdfcol.dtx
%makeindex -s gind.ist pdfcol.idx
%pdflatex pdfcol.dtx
%\end{verbatim}
% \end{quote}
%
% \begin{History}
%   \begin{Version}{2007/09/09 v1.0}
%   \item
%     First version.
%   \end{Version}
%   \begin{Version}{2007/12/09 v1.1}
%   \item
%     \cs{pdfcolSetCurrentColor} added.
%   \end{Version}
%   \begin{Version}{2007/12/12 v1.2}
%   \item
%     Detection for package \xpackage{luacolor} added.
%   \end{Version}
%   \begin{Version}{2016/05/16 v1.3}
%   \item
%     Documentation updates.
%   \end{Version}
%   \begin{Version}{2016/05/17 v1.4}
%   \item
%     Use luatex85 package for new luatex compatibility
%   \end{Version}
%   \begin{Version}{2018/11/01 v1.5}
%   \item
%     Remove luatex85 dependency
%   \end{Version}
% \end{History}
%
% \PrintIndex
%
% \Finale
\endinput

%        (quote the arguments according to the demands of your shell)
%
% Documentation:
%    (a) If pdfcol.drv is present:
%           latex pdfcol.drv
%    (b) Without pdfcol.drv:
%           latex pdfcol.dtx; ...
%    The class ltxdoc loads the configuration file ltxdoc.cfg
%    if available. Here you can specify further options, e.g.
%    use A4 as paper format:
%       \PassOptionsToClass{a4paper}{article}
%
%    Programm calls to get the documentation (example):
%       pdflatex pdfcol.dtx
%       makeindex -s gind.ist pdfcol.idx
%       pdflatex pdfcol.dtx
%       makeindex -s gind.ist pdfcol.idx
%       pdflatex pdfcol.dtx
%
% Installation:
%    TDS:tex/generic/oberdiek/pdfcol.sty
%    TDS:doc/latex/oberdiek/pdfcol.pdf
%    TDS:source/latex/oberdiek/pdfcol.dtx
%
%<*ignore>
\begingroup
  \catcode123=1 %
  \catcode125=2 %
  \def\x{LaTeX2e}%
\expandafter\endgroup
\ifcase 0\ifx\install y1\fi\expandafter
         \ifx\csname processbatchFile\endcsname\relax\else1\fi
         \ifx\fmtname\x\else 1\fi\relax
\else\csname fi\endcsname
%</ignore>
%<*install>
\input docstrip.tex
\Msg{************************************************************************}
\Msg{* Installation}
\Msg{* Package: pdfcol 2018/11/01 v1.5 Handle new color stacks for pdfTeX (HO)}
\Msg{************************************************************************}

\keepsilent
\askforoverwritefalse

\let\MetaPrefix\relax
\preamble

This is a generated file.

Project: pdfcol
Version: 2018/11/01 v1.5

Copyright (C)
   2007 Heiko Oberdiek
   2016-2019 Oberdiek Package Support Group

This work may be distributed and/or modified under the
conditions of the LaTeX Project Public License, either
version 1.3c of this license or (at your option) any later
version. This version of this license is in
   https://www.latex-project.org/lppl/lppl-1-3c.txt
and the latest version of this license is in
   https://www.latex-project.org/lppl.txt
and version 1.3 or later is part of all distributions of
LaTeX version 2005/12/01 or later.

This work has the LPPL maintenance status "maintained".

The Current Maintainers of this work are
Heiko Oberdiek and the Oberdiek Package Support Group
https://github.com/ho-tex/oberdiek/issues


The Base Interpreter refers to any `TeX-Format',
because some files are installed in TDS:tex/generic//.

This work consists of the main source file pdfcol.dtx
and the derived files
   pdfcol.sty, pdfcol.pdf, pdfcol.ins, pdfcol.drv, pdfcol-test1.tex,
   pdfcol-test2.tex, pdfcol-test3.tex, pdfcol-test4.tex.

\endpreamble
\let\MetaPrefix\DoubleperCent

\generate{%
  \file{pdfcol.ins}{\from{pdfcol.dtx}{install}}%
  \file{pdfcol.drv}{\from{pdfcol.dtx}{driver}}%
  \usedir{tex/generic/oberdiek}%
  \file{pdfcol.sty}{\from{pdfcol.dtx}{package}}%
%  \usedir{doc/latex/oberdiek/test}%
%  \file{pdfcol-test1.tex}{\from{pdfcol.dtx}{test1}}%
%  \file{pdfcol-test2.tex}{\from{pdfcol.dtx}{test2}}%
%  \file{pdfcol-test3.tex}{\from{pdfcol.dtx}{test3}}%
%  \file{pdfcol-test4.tex}{\from{pdfcol.dtx}{test4}}%
}

\catcode32=13\relax% active space
\let =\space%
\Msg{************************************************************************}
\Msg{*}
\Msg{* To finish the installation you have to move the following}
\Msg{* file into a directory searched by TeX:}
\Msg{*}
\Msg{*     pdfcol.sty}
\Msg{*}
\Msg{* To produce the documentation run the file `pdfcol.drv'}
\Msg{* through LaTeX.}
\Msg{*}
\Msg{* Happy TeXing!}
\Msg{*}
\Msg{************************************************************************}

\endbatchfile
%</install>
%<*ignore>
\fi
%</ignore>
%<*driver>
\NeedsTeXFormat{LaTeX2e}
\ProvidesFile{pdfcol.drv}%
  [2018/11/01 v1.5 Handle new color stacks for pdfTeX (HO)]%
\documentclass{ltxdoc}
\usepackage{holtxdoc}[2011/11/22]
\begin{document}
  \DocInput{pdfcol.dtx}%
\end{document}
%</driver>
% \fi
%
%
%
% \GetFileInfo{pdfcol.drv}
%
% \title{The \xpackage{pdfcol} package}
% \date{2018/11/01 v1.5}
% \author{Heiko Oberdiek\thanks
% {Please report any issues at \url{https://github.com/ho-tex/oberdiek/issues}}}
%
% \maketitle
%
% \begin{abstract}
% Since version 1.40 \pdfTeX\ supports color stacks.
% The driver file \xfile{pdftex.def} for package \xpackage{color}
% defines and uses a main color stack since version v0.04b.
% Package \xpackage{pdfcol} is intended for package writers.
% It defines macros for setting and maintaining new color stacks.
% \end{abstract}
%
% \tableofcontents
%
% \section{Documentation}
%
% Version 1.40 of \pdfTeX\ adds new primitives \cs{pdfcolorstackinit}
% and \cs{pdfcolorstack}. Now color stacks can be defined and used.
% A main color stack is maintained by the driver file \xfile{pdftex.def}
% similar to dvips or dvipdfm. However the number of color stacks
% is not limited to one in \pdfTeX. Thus further color problems
% can now be solved, such as footnotes across pages or text
% that is set in parallel columns (e.g. packages \xpackage{parallel}
% or \xpackage{parcolumn}). Unlike the main color stack,
% the support by additional color stacks cannot be done in
% a transparent manner.
%
% This package \xpackage{pdfcol} provides an easier interface to
% additional color stacks without the need to use the
% low level primitives.
%
% \subsection{Requirements}
% \label{sec:req}
%
% \begin{itemize}
% \item
%   \pdfTeX\ 1.40 or greater.
% \item
%   \pdfTeX in PDF mode. (I don't know a DVI driver that
%   support several color stacks.)
% \item
%   \xfile{pdftex.def} 2007/01/02 v0.04b.
% \end{itemize}
% Package \xpackage{pdfcol} checks the requirements and
% sets switch \cs{ifpdfcolAvailable} accordingly.
%
% \subsection{Interface}
%
% \begin{declcs}{ifpdfcolAvailable}
% \end{declcs}
% If the requirements of section \ref{sec:req} are met the
% switch \cs{ifpdfcolAvailable} behaves as \cs{iftrue}.
% Otherwise the other interface macros in this section will
% be disabled with a message. Also the first use of such a
% macro will print a message. The messages are print to
% the \xext{log} file only if \pdfTeX\ is not used in PDF mode.
%
% \begin{declcs}{pdfcolErrorNoStacks}
% \end{declcs}
% The first call of \cs{pdfcolErrorNoStacks} prints an error
% message, if color stacks are not available.
%
% \begin{declcs}{pdfcolInitStack} \M{name}
% \end{declcs}
% A new color stack is initialized by \cs{pdfcolInitStack}.
% The \meta{name} is used for indentifying the stack. It usually
% consists of letters and digits. (The name must survive a \cs{csname}.)
%
% The intension of the macro is the definition of an additional
% color stack. Thus the stack is not page bounded like the
% main color stack. Black (\texttt{0 g 0 G}) is used as initial
% color value. And colors are written with modifier \texttt{direct}
% that means without setting the current transfer matrix and changing
% the current point (see documentation of \pdfTeX\ for
% |\pdfliteral direct{...}|).
%
% \begin{declcs}{pdfcolIfStackExists} \M{name} \M{then} \M{else}
% \end{declcs}
% Macro \cs{pdfcolIfStackExists} checks whether color stack \meta{name}
% exists. In case of success argument \meta{then} is executed
% and \meta{else} otherwise.
%
% \begin{declcs}{pdfcolSwitchStack} \M{name}
% \end{declcs}
% Macro \cs{pdfcolSwitchStack} switches the color stack. The color macros
% of package \xpackage{color} (or \xpackage{xcolor}) now uses the
% new color stack with name \meta{name}.
%
% \begin{declcs}{pdfcolSetCurrentColor}
% \end{declcs}
% Macro \cs{pdfcolSetCurrentColor} replaces the topmost
% entry of the stack by the current color (\cs{current@color}).
%
% \begin{declcs}{pdfcolSetCurrent} \M{name}
% \end{declcs}
% Macro \cs{pdfcolSetCurrent} sets the color that is read in
% the top-most entry of color stack \meta{name}. If \meta{name}
% is empty, the default color stack is used.
%
% \StopEventually{
% }
%
% \section{Implementation}
%
%    \begin{macrocode}
%<*package>
%    \end{macrocode}
%
% \subsection{Reload check and package identification}
%    Reload check, especially if the package is not used with \LaTeX.
%    \begin{macrocode}
\begingroup\catcode61\catcode48\catcode32=10\relax%
  \catcode13=5 % ^^M
  \endlinechar=13 %
  \catcode35=6 % #
  \catcode39=12 % '
  \catcode44=12 % ,
  \catcode45=12 % -
  \catcode46=12 % .
  \catcode58=12 % :
  \catcode64=11 % @
  \catcode123=1 % {
  \catcode125=2 % }
  \expandafter\let\expandafter\x\csname ver@pdfcol.sty\endcsname
  \ifx\x\relax % plain-TeX, first loading
  \else
    \def\empty{}%
    \ifx\x\empty % LaTeX, first loading,
      % variable is initialized, but \ProvidesPackage not yet seen
    \else
      \expandafter\ifx\csname PackageInfo\endcsname\relax
        \def\x#1#2{%
          \immediate\write-1{Package #1 Info: #2.}%
        }%
      \else
        \def\x#1#2{\PackageInfo{#1}{#2, stopped}}%
      \fi
      \x{pdfcol}{The package is already loaded}%
      \aftergroup\endinput
    \fi
  \fi
\endgroup%
%    \end{macrocode}
%    Package identification:
%    \begin{macrocode}
\begingroup\catcode61\catcode48\catcode32=10\relax%
  \catcode13=5 % ^^M
  \endlinechar=13 %
  \catcode35=6 % #
  \catcode39=12 % '
  \catcode40=12 % (
  \catcode41=12 % )
  \catcode44=12 % ,
  \catcode45=12 % -
  \catcode46=12 % .
  \catcode47=12 % /
  \catcode58=12 % :
  \catcode64=11 % @
  \catcode91=12 % [
  \catcode93=12 % ]
  \catcode123=1 % {
  \catcode125=2 % }
  \expandafter\ifx\csname ProvidesPackage\endcsname\relax
    \def\x#1#2#3[#4]{\endgroup
      \immediate\write-1{Package: #3 #4}%
      \xdef#1{#4}%
    }%
  \else
    \def\x#1#2[#3]{\endgroup
      #2[{#3}]%
      \ifx#1\@undefined
        \xdef#1{#3}%
      \fi
      \ifx#1\relax
        \xdef#1{#3}%
      \fi
    }%
  \fi
\expandafter\x\csname ver@pdfcol.sty\endcsname
\ProvidesPackage{pdfcol}%
  [2018/11/01 v1.5 Handle new color stacks for pdfTeX (HO)]%
%    \end{macrocode}
%
% \subsection{Catcodes}
%
%    \begin{macrocode}
\begingroup\catcode61\catcode48\catcode32=10\relax%
  \catcode13=5 % ^^M
  \endlinechar=13 %
  \catcode123=1 % {
  \catcode125=2 % }
  \catcode64=11 % @
  \def\x{\endgroup
    \expandafter\edef\csname PDFCOL@AtEnd\endcsname{%
      \endlinechar=\the\endlinechar\relax
      \catcode13=\the\catcode13\relax
      \catcode32=\the\catcode32\relax
      \catcode35=\the\catcode35\relax
      \catcode61=\the\catcode61\relax
      \catcode64=\the\catcode64\relax
      \catcode123=\the\catcode123\relax
      \catcode125=\the\catcode125\relax
    }%
  }%
\x\catcode61\catcode48\catcode32=10\relax%
\catcode13=5 % ^^M
\endlinechar=13 %
\catcode35=6 % #
\catcode64=11 % @
\catcode123=1 % {
\catcode125=2 % }
\def\TMP@EnsureCode#1#2{%
  \edef\PDFCOL@AtEnd{%
    \PDFCOL@AtEnd
    \catcode#1=\the\catcode#1\relax
  }%
  \catcode#1=#2\relax
}
\TMP@EnsureCode{39}{12}% '
\TMP@EnsureCode{40}{12}% (
\TMP@EnsureCode{41}{12}% )
\TMP@EnsureCode{43}{12}% +
\TMP@EnsureCode{44}{12}% ,
\TMP@EnsureCode{46}{12}% .
\TMP@EnsureCode{47}{12}% /
\TMP@EnsureCode{91}{12}% [
\TMP@EnsureCode{93}{12}% ]
\TMP@EnsureCode{96}{12}% `
\edef\PDFCOL@AtEnd{\PDFCOL@AtEnd\noexpand\endinput}
%    \end{macrocode}
%
% \subsection{Check requirements}
%
%    \begin{macro}{\PDFCOL@RequirePackage}
%    \begin{macrocode}
\begingroup\expandafter\expandafter\expandafter\endgroup
\expandafter\ifx\csname RequirePackage\endcsname\relax
  \def\PDFCOL@RequirePackage#1[#2]{\input #1.sty\relax}%
\else
  \def\PDFCOL@RequirePackage#1[#2]{%
    \RequirePackage{#1}[{#2}]%
  }%
\fi
%    \end{macrocode}
%    \end{macro}
%
% LuaTeX Compatability
%    \begin{macrocode}
\ifx\pdfextension\@undefined\else
 \def\pdfcolorstackinit {\pdffeedback colorstackinit}
 \protected\def\pdfcolorstack     {\pdfextension colorstack}
\fi
%    \end{macrocode}
%
%    \begin{macrocode}
\PDFCOL@RequirePackage{ltxcmds}[2010/03/01]
%    \end{macrocode}
%
%    \begin{macro}{ifpdfcolAvailable}
%    \begin{macrocode}
\ltx@newif\ifpdfcolAvailable
\pdfcolAvailabletrue
%    \end{macrocode}
%    \end{macro}
%
% \subsubsection{Check package \xpackage{luacolor}}
%
%    \begin{macrocode}
\ltx@newif\ifPDFCOL@luacolor
\begingroup\expandafter\expandafter\expandafter\endgroup
\expandafter\ifx\csname ver@luacolor.sty\endcsname\relax
  \PDFCOL@luacolorfalse
\else
  \PDFCOL@luacolortrue
\fi
%    \end{macrocode}
%
% \subsubsection{Check PDF mode}
%
%    \begin{macrocode}
\PDFCOL@RequirePackage{infwarerr}[2007/09/09]
\PDFCOL@RequirePackage{ifpdf}[2007/09/09]
\ifcase\ifpdf\ifPDFCOL@luacolor 1\fi\else 1\fi0 %
  \def\PDFCOL@Message{%
    \@PackageWarningNoLine{pdfcol}%
  }%
\else
  \pdfcolAvailablefalse
  \def\PDFCOL@Message{%
    \@PackageInfoNoLine{pdfcol}%
  }%
  \PDFCOL@Message{%
    Interface disabled because of %
    \ifPDFCOL@luacolor
      package `luacolor'%
    \else
      missing PDF mode of pdfTeX%
    \fi
  }%
\fi
%    \end{macrocode}
%
% \subsubsection{Check version of \pdfTeX}
%
%    \begin{macrocode}
\ifpdfcolAvailable
  \begingroup\expandafter\expandafter\expandafter\endgroup
  \expandafter\ifx\csname pdfcolorstack\endcsname\relax
    \pdfcolAvailablefalse
    \PDFCOL@Message{%
      Interface disabled because of too old pdfTeX.\MessageBreak
      Required is version 1.40+ for \string\pdfcolorstack
    }%
  \fi
\fi
\ifpdfcolAvailable
  \begingroup\expandafter\expandafter\expandafter\endgroup
  \expandafter\ifx\csname pdfcolorstack\endcsname\relax
    \pdfcolAvailablefalse
    \PDFCOL@Message{%
      Interface disabled because of too old pdfTeX.\MessageBreak
      Required is version 1.40+ for \string\pdfcolorstackinit
    }%
  \fi
\fi
%    \end{macrocode}
%
% \subsubsection{Check \xfile{pdftex.def}}
%
%    \begin{macrocode}
\ifpdfcolAvailable
  \begingroup\expandafter\expandafter\expandafter\endgroup
  \expandafter\ifx\csname @pdfcolorstack\endcsname\relax
%    \end{macrocode}
%    Try to load package color if it is not yet loaded (\LaTeX\ case).
%    \begin{macrocode}
    \begingroup\expandafter\expandafter\expandafter\endgroup
    \expandafter\ifx\csname ver@color.sty\endcsname\relax
      \begingroup\expandafter\expandafter\expandafter\endgroup
      \expandafter\ifx\csname documentclass\endcsname\relax
      \else
        \RequirePackage[pdftex]{color}\relax
      \fi
    \fi
    \begingroup\expandafter\expandafter\expandafter\endgroup
    \expandafter\ifx\csname @pdfcolorstack\endcsname\relax
      \pdfcolAvailablefalse
      \PDFCOL@Message{%
        Interface disabled because `pdftex.def'\MessageBreak
        is not loaded or it is too old.\MessageBreak
        Required is version 0.04b or greater%
      }%
    \fi
  \fi
\fi
%    \end{macrocode}
%
%    \begin{macrocode}
\let\pdfcolAvailabletrue\relax
\let\pdfcolAvailablefalse\relax
%    \end{macrocode}
%
% \subsection{Enabled interface macros}
%
%    \begin{macrocode}
\ifpdfcolAvailable
%    \end{macrocode}
%
%    \begin{macro}{\pdfcolErrorNoStacks}
%    \begin{macrocode}
  \let\pdfcolErrorNoStacks\relax
%    \end{macrocode}
%    \end{macro}
%
%    \begin{macro}{\pdfcol@Value}
%    \begin{macrocode}
  \expandafter\ifx\csname pdfcol@Value\endcsname\relax
    \def\pdfcol@Value{0 g 0 G}%
  \fi
%    \end{macrocode}
%    \end{macro}
%
%    \begin{macro}{\pdfcol@LiteralModifier}
%    \begin{macrocode}
  \expandafter\ifx\csname pdfcol@LiteralModifier\endcsname\relax
    \def\pdfcol@LiteralModifier{direct}%
  \fi
%    \end{macrocode}
%    \end{macro}
%
%    \begin{macro}{\pdfcolInitStack}
%    \begin{macrocode}
  \def\pdfcolInitStack#1{%
    \expandafter\ifx\csname pdfcol@Stack@#1\endcsname\relax
      \global\expandafter\chardef\csname pdfcol@Stack@#1\endcsname=%
          \pdfcolorstackinit\pdfcol@LiteralModifier{\pdfcol@Value}%
          \relax
      \@PackageInfo{pdfcol}{%
        New color stack `#1' = \number\csname pdfcol@Stack@#1\endcsname
      }%
    \else
      \@PackageError{pdfcol}{%
        Stack `#1' is already defined%
      }\@ehc
    \fi
  }%
%    \end{macrocode}
%    \end{macro}
%
%    \begin{macro}{\pdfcolIfStackExists}
%    \begin{macrocode}
  \def\pdfcolIfStackExists#1{%
    \expandafter\ifx\csname pdfcol@Stack@#1\endcsname\relax
      \expandafter\@secondoftwo
    \else
      \expandafter\@firstoftwo
    \fi
  }%
%    \end{macrocode}
%    \end{macro}
%    \begin{macro}{\@firstoftwo}
%    \begin{macrocode}
  \expandafter\ifx\csname @firstoftwo\endcsname\relax
    \long\def\@firstoftwo#1#2{#1}%
  \fi
%    \end{macrocode}
%    \end{macro}
%    \begin{macro}{\@secondoftwo}
%    \begin{macrocode}
  \expandafter\ifx\csname @secondoftwo\endcsname\relax
    \long\def\@secondoftwo#1#2{#2}%
  \fi
%    \end{macrocode}
%    \end{macro}
%
%    \begin{macro}{\pdfcolSwitchStack}
%    \begin{macrocode}
  \def\pdfcolSwitchStack#1{%
    \pdfcolIfStackExists{#1}{%
      \expandafter\let\expandafter\@pdfcolorstack
                      \csname pdfcol@Stack@#1\endcsname
    }{%
      \pdfcol@ErrorNoStack{#1}%
    }%
  }%
%    \end{macrocode}
%    \end{macro}
%
%    \begin{macro}{\pdfcolSetCurrentColor}
%    \begin{macrocode}
  \def\pdfcolSetCurrentColor{%
    \pdfcolorstack\@pdfcolorstack set{\current@color}%
  }%
%    \end{macrocode}
%    \end{macro}
%
%    \begin{macro}{\pdfcolSetCurrent}
%    \begin{macrocode}
  \def\pdfcolSetCurrent#1{%
    \ifx\\#1\\%
      \pdfcolorstack\@pdfcolorstack current\relax
    \else
      \pdfcolIfStackExists{#1}{%
        \pdfcolorstack\csname pdfcol@Stack@#1\endcsname current\relax
      }{%
        \pdfcol@ErrorNoStack{#1}%
      }%
    \fi
  }%
%    \end{macrocode}
%    \end{macro}
%
%    \begin{macro}{\pdfcol@ErrorNoStack}
%    \begin{macrocode}
  \def\pdfcol@ErrorNoStack#1{%
    \@PackageError{pdfcol}{Stack `#1' does not exists}\@ehc
  }%
%    \end{macrocode}
%    \end{macro}
%
% \subsection{Disabled interface macros}
%
%    \begin{macrocode}
\else
%    \end{macrocode}
%
%    \begin{macro}{\pdfcolErrorNoStacks}
%    \begin{macrocode}
  \def\pdfcolErrorNoStacks{%
    \@PackageError{pdfcol}{%
      Color stacks are not available%
    }{%
      Update pdfTeX (1.40) and `pdftex.def' (0.04b) %
          if necessary.\MessageBreak
      Ensure that `pdftex.def' is loaded %
          (package `color' or `xcolor').\MessageBreak
      Further messages can be found in TeX's %
          protocol file `\jobname.log'.\MessageBreak
      \MessageBreak
      \@ehc
    }%
    \global\let\pdfcolErrorNoStacks\relax
  }%
%    \end{macrocode}
%    \end{macro}
%
%    \begin{macro}{\PDFCOL@Disabled}
%    \begin{macrocode}
  \def\PDFCOL@Disabled{%
    \PDFCOL@Message{%
      pdfTeX's color stacks are not available%
    }%
    \global\let\PDFCOL@Disabled\relax
  }%
%    \end{macrocode}
%    \end{macro}
%
%    \begin{macro}{\pdfcolInitStack}
%    \begin{macrocode}
  \def\pdfcolInitStack#1{%
    \PDFCOL@Disabled
  }%
%    \end{macrocode}
%    \end{macro}
%
%    \begin{macro}{\pdfcolIfStackExists}
%    \begin{macrocode}
  \long\def\pdfcolIfStackExists#1#2#3{#3}%
%    \end{macrocode}
%    \end{macro}
%
%    \begin{macro}{\pdfcolSwitchStack}
%    \begin{macrocode}
  \def\pdfcolSwitchStack#1{%
    \PDFCOL@Disabled
  }%
%    \end{macrocode}
%    \end{macro}
%
%    \begin{macro}{\pdfcolSetCurrentColor}
%    \begin{macrocode}
  \def\pdfcolSetCurrentColor{%
    \PDFCOL@Disabled
  }%
%    \end{macrocode}
%    \end{macro}
%
%    \begin{macro}{\pdfcolSetCurrent}
%    \begin{macrocode}
  \def\pdfcolSetCurrent#1{%
    \PDFCOL@Disabled
  }%
%    \end{macrocode}
%    \end{macro}
%    \begin{macrocode}
\fi
%    \end{macrocode}
%
%    \begin{macrocode}
\PDFCOL@AtEnd%
%</package>
%    \end{macrocode}
%% \section{Installation}
%
% \subsection{Download}
%
% \paragraph{Package.} This package is available on
% CTAN\footnote{\CTANpkg{pdfcol}}:
% \begin{description}
% \item[\CTAN{macros/latex/contrib/oberdiek/pdfcol.dtx}] The source file.
% \item[\CTAN{macros/latex/contrib/oberdiek/pdfcol.pdf}] Documentation.
% \end{description}
%
%
% \paragraph{Bundle.} All the packages of the bundle `oberdiek'
% are also available in a TDS compliant ZIP archive. There
% the packages are already unpacked and the documentation files
% are generated. The files and directories obey the TDS standard.
% \begin{description}
% \item[\CTANinstall{install/macros/latex/contrib/oberdiek.tds.zip}]
% \end{description}
% \emph{TDS} refers to the standard ``A Directory Structure
% for \TeX\ Files'' (\CTANpkg{tds}). Directories
% with \xfile{texmf} in their name are usually organized this way.
%
% \subsection{Bundle installation}
%
% \paragraph{Unpacking.} Unpack the \xfile{oberdiek.tds.zip} in the
% TDS tree (also known as \xfile{texmf} tree) of your choice.
% Example (linux):
% \begin{quote}
%   |unzip oberdiek.tds.zip -d ~/texmf|
% \end{quote}
%
% \subsection{Package installation}
%
% \paragraph{Unpacking.} The \xfile{.dtx} file is a self-extracting
% \docstrip\ archive. The files are extracted by running the
% \xfile{.dtx} through \plainTeX:
% \begin{quote}
%   \verb|tex pdfcol.dtx|
% \end{quote}
%
% \paragraph{TDS.} Now the different files must be moved into
% the different directories in your installation TDS tree
% (also known as \xfile{texmf} tree):
% \begin{quote}
% \def\t{^^A
% \begin{tabular}{@{}>{\ttfamily}l@{ $\rightarrow$ }>{\ttfamily}l@{}}
%   pdfcol.sty & tex/generic/oberdiek/pdfcol.sty\\
%   pdfcol.pdf & doc/latex/oberdiek/pdfcol.pdf\\
%   pdfcol.dtx & source/latex/oberdiek/pdfcol.dtx\\
% \end{tabular}^^A
% }^^A
% \sbox0{\t}^^A
% \ifdim\wd0>\linewidth
%   \begingroup
%     \advance\linewidth by\leftmargin
%     \advance\linewidth by\rightmargin
%   \edef\x{\endgroup
%     \def\noexpand\lw{\the\linewidth}^^A
%   }\x
%   \def\lwbox{^^A
%     \leavevmode
%     \hbox to \linewidth{^^A
%       \kern-\leftmargin\relax
%       \hss
%       \usebox0
%       \hss
%       \kern-\rightmargin\relax
%     }^^A
%   }^^A
%   \ifdim\wd0>\lw
%     \sbox0{\small\t}^^A
%     \ifdim\wd0>\linewidth
%       \ifdim\wd0>\lw
%         \sbox0{\footnotesize\t}^^A
%         \ifdim\wd0>\linewidth
%           \ifdim\wd0>\lw
%             \sbox0{\scriptsize\t}^^A
%             \ifdim\wd0>\linewidth
%               \ifdim\wd0>\lw
%                 \sbox0{\tiny\t}^^A
%                 \ifdim\wd0>\linewidth
%                   \lwbox
%                 \else
%                   \usebox0
%                 \fi
%               \else
%                 \lwbox
%               \fi
%             \else
%               \usebox0
%             \fi
%           \else
%             \lwbox
%           \fi
%         \else
%           \usebox0
%         \fi
%       \else
%         \lwbox
%       \fi
%     \else
%       \usebox0
%     \fi
%   \else
%     \lwbox
%   \fi
% \else
%   \usebox0
% \fi
% \end{quote}
% If you have a \xfile{docstrip.cfg} that configures and enables \docstrip's
% TDS installing feature, then some files can already be in the right
% place, see the documentation of \docstrip.
%
% \subsection{Refresh file name databases}
%
% If your \TeX~distribution
% (\TeX\,Live, \mikTeX, \dots) relies on file name databases, you must refresh
% these. For example, \TeX\,Live\ users run \verb|texhash| or
% \verb|mktexlsr|.
%
% \subsection{Some details for the interested}
%
% \paragraph{Unpacking with \LaTeX.}
% The \xfile{.dtx} chooses its action depending on the format:
% \begin{description}
% \item[\plainTeX:] Run \docstrip\ and extract the files.
% \item[\LaTeX:] Generate the documentation.
% \end{description}
% If you insist on using \LaTeX\ for \docstrip\ (really,
% \docstrip\ does not need \LaTeX), then inform the autodetect routine
% about your intention:
% \begin{quote}
%   \verb|latex \let\install=y% \iffalse meta-comment
%
% File: pdfcol.dtx
% Version: 2018/11/01 v1.5
% Info: Handle new color stacks for pdfTeX
%
% Copyright (C)
%    2007 Heiko Oberdiek
%    2016-2019 Oberdiek Package Support Group
%    https://github.com/ho-tex/oberdiek/issues
%
% This work may be distributed and/or modified under the
% conditions of the LaTeX Project Public License, either
% version 1.3c of this license or (at your option) any later
% version. This version of this license is in
%    https://www.latex-project.org/lppl/lppl-1-3c.txt
% and the latest version of this license is in
%    https://www.latex-project.org/lppl.txt
% and version 1.3 or later is part of all distributions of
% LaTeX version 2005/12/01 or later.
%
% This work has the LPPL maintenance status "maintained".
%
% The Current Maintainers of this work are
% Heiko Oberdiek and the Oberdiek Package Support Group
% https://github.com/ho-tex/oberdiek/issues
%
% The Base Interpreter refers to any `TeX-Format',
% because some files are installed in TDS:tex/generic//.
%
% This work consists of the main source file pdfcol.dtx
% and the derived files
%    pdfcol.sty, pdfcol.pdf, pdfcol.ins, pdfcol.drv, pdfcol-test1.tex,
%    pdfcol-test2.tex, pdfcol-test3.tex, pdfcol-test4.tex.
%
% Distribution:
%    CTAN:macros/latex/contrib/oberdiek/pdfcol.dtx
%    CTAN:macros/latex/contrib/oberdiek/pdfcol.pdf
%
% Unpacking:
%    (a) If pdfcol.ins is present:
%           tex pdfcol.ins
%    (b) Without pdfcol.ins:
%           tex pdfcol.dtx
%    (c) If you insist on using LaTeX
%           latex \let\install=y% \iffalse meta-comment
%
% File: pdfcol.dtx
% Version: 2018/11/01 v1.5
% Info: Handle new color stacks for pdfTeX
%
% Copyright (C)
%    2007 Heiko Oberdiek
%    2016-2019 Oberdiek Package Support Group
%    https://github.com/ho-tex/oberdiek/issues
%
% This work may be distributed and/or modified under the
% conditions of the LaTeX Project Public License, either
% version 1.3c of this license or (at your option) any later
% version. This version of this license is in
%    https://www.latex-project.org/lppl/lppl-1-3c.txt
% and the latest version of this license is in
%    https://www.latex-project.org/lppl.txt
% and version 1.3 or later is part of all distributions of
% LaTeX version 2005/12/01 or later.
%
% This work has the LPPL maintenance status "maintained".
%
% The Current Maintainers of this work are
% Heiko Oberdiek and the Oberdiek Package Support Group
% https://github.com/ho-tex/oberdiek/issues
%
% The Base Interpreter refers to any `TeX-Format',
% because some files are installed in TDS:tex/generic//.
%
% This work consists of the main source file pdfcol.dtx
% and the derived files
%    pdfcol.sty, pdfcol.pdf, pdfcol.ins, pdfcol.drv, pdfcol-test1.tex,
%    pdfcol-test2.tex, pdfcol-test3.tex, pdfcol-test4.tex.
%
% Distribution:
%    CTAN:macros/latex/contrib/oberdiek/pdfcol.dtx
%    CTAN:macros/latex/contrib/oberdiek/pdfcol.pdf
%
% Unpacking:
%    (a) If pdfcol.ins is present:
%           tex pdfcol.ins
%    (b) Without pdfcol.ins:
%           tex pdfcol.dtx
%    (c) If you insist on using LaTeX
%           latex \let\install=y\input{pdfcol.dtx}
%        (quote the arguments according to the demands of your shell)
%
% Documentation:
%    (a) If pdfcol.drv is present:
%           latex pdfcol.drv
%    (b) Without pdfcol.drv:
%           latex pdfcol.dtx; ...
%    The class ltxdoc loads the configuration file ltxdoc.cfg
%    if available. Here you can specify further options, e.g.
%    use A4 as paper format:
%       \PassOptionsToClass{a4paper}{article}
%
%    Programm calls to get the documentation (example):
%       pdflatex pdfcol.dtx
%       makeindex -s gind.ist pdfcol.idx
%       pdflatex pdfcol.dtx
%       makeindex -s gind.ist pdfcol.idx
%       pdflatex pdfcol.dtx
%
% Installation:
%    TDS:tex/generic/oberdiek/pdfcol.sty
%    TDS:doc/latex/oberdiek/pdfcol.pdf
%    TDS:source/latex/oberdiek/pdfcol.dtx
%
%<*ignore>
\begingroup
  \catcode123=1 %
  \catcode125=2 %
  \def\x{LaTeX2e}%
\expandafter\endgroup
\ifcase 0\ifx\install y1\fi\expandafter
         \ifx\csname processbatchFile\endcsname\relax\else1\fi
         \ifx\fmtname\x\else 1\fi\relax
\else\csname fi\endcsname
%</ignore>
%<*install>
\input docstrip.tex
\Msg{************************************************************************}
\Msg{* Installation}
\Msg{* Package: pdfcol 2018/11/01 v1.5 Handle new color stacks for pdfTeX (HO)}
\Msg{************************************************************************}

\keepsilent
\askforoverwritefalse

\let\MetaPrefix\relax
\preamble

This is a generated file.

Project: pdfcol
Version: 2018/11/01 v1.5

Copyright (C)
   2007 Heiko Oberdiek
   2016-2019 Oberdiek Package Support Group

This work may be distributed and/or modified under the
conditions of the LaTeX Project Public License, either
version 1.3c of this license or (at your option) any later
version. This version of this license is in
   https://www.latex-project.org/lppl/lppl-1-3c.txt
and the latest version of this license is in
   https://www.latex-project.org/lppl.txt
and version 1.3 or later is part of all distributions of
LaTeX version 2005/12/01 or later.

This work has the LPPL maintenance status "maintained".

The Current Maintainers of this work are
Heiko Oberdiek and the Oberdiek Package Support Group
https://github.com/ho-tex/oberdiek/issues


The Base Interpreter refers to any `TeX-Format',
because some files are installed in TDS:tex/generic//.

This work consists of the main source file pdfcol.dtx
and the derived files
   pdfcol.sty, pdfcol.pdf, pdfcol.ins, pdfcol.drv, pdfcol-test1.tex,
   pdfcol-test2.tex, pdfcol-test3.tex, pdfcol-test4.tex.

\endpreamble
\let\MetaPrefix\DoubleperCent

\generate{%
  \file{pdfcol.ins}{\from{pdfcol.dtx}{install}}%
  \file{pdfcol.drv}{\from{pdfcol.dtx}{driver}}%
  \usedir{tex/generic/oberdiek}%
  \file{pdfcol.sty}{\from{pdfcol.dtx}{package}}%
%  \usedir{doc/latex/oberdiek/test}%
%  \file{pdfcol-test1.tex}{\from{pdfcol.dtx}{test1}}%
%  \file{pdfcol-test2.tex}{\from{pdfcol.dtx}{test2}}%
%  \file{pdfcol-test3.tex}{\from{pdfcol.dtx}{test3}}%
%  \file{pdfcol-test4.tex}{\from{pdfcol.dtx}{test4}}%
}

\catcode32=13\relax% active space
\let =\space%
\Msg{************************************************************************}
\Msg{*}
\Msg{* To finish the installation you have to move the following}
\Msg{* file into a directory searched by TeX:}
\Msg{*}
\Msg{*     pdfcol.sty}
\Msg{*}
\Msg{* To produce the documentation run the file `pdfcol.drv'}
\Msg{* through LaTeX.}
\Msg{*}
\Msg{* Happy TeXing!}
\Msg{*}
\Msg{************************************************************************}

\endbatchfile
%</install>
%<*ignore>
\fi
%</ignore>
%<*driver>
\NeedsTeXFormat{LaTeX2e}
\ProvidesFile{pdfcol.drv}%
  [2018/11/01 v1.5 Handle new color stacks for pdfTeX (HO)]%
\documentclass{ltxdoc}
\usepackage{holtxdoc}[2011/11/22]
\begin{document}
  \DocInput{pdfcol.dtx}%
\end{document}
%</driver>
% \fi
%
%
%
% \GetFileInfo{pdfcol.drv}
%
% \title{The \xpackage{pdfcol} package}
% \date{2018/11/01 v1.5}
% \author{Heiko Oberdiek\thanks
% {Please report any issues at \url{https://github.com/ho-tex/oberdiek/issues}}}
%
% \maketitle
%
% \begin{abstract}
% Since version 1.40 \pdfTeX\ supports color stacks.
% The driver file \xfile{pdftex.def} for package \xpackage{color}
% defines and uses a main color stack since version v0.04b.
% Package \xpackage{pdfcol} is intended for package writers.
% It defines macros for setting and maintaining new color stacks.
% \end{abstract}
%
% \tableofcontents
%
% \section{Documentation}
%
% Version 1.40 of \pdfTeX\ adds new primitives \cs{pdfcolorstackinit}
% and \cs{pdfcolorstack}. Now color stacks can be defined and used.
% A main color stack is maintained by the driver file \xfile{pdftex.def}
% similar to dvips or dvipdfm. However the number of color stacks
% is not limited to one in \pdfTeX. Thus further color problems
% can now be solved, such as footnotes across pages or text
% that is set in parallel columns (e.g. packages \xpackage{parallel}
% or \xpackage{parcolumn}). Unlike the main color stack,
% the support by additional color stacks cannot be done in
% a transparent manner.
%
% This package \xpackage{pdfcol} provides an easier interface to
% additional color stacks without the need to use the
% low level primitives.
%
% \subsection{Requirements}
% \label{sec:req}
%
% \begin{itemize}
% \item
%   \pdfTeX\ 1.40 or greater.
% \item
%   \pdfTeX in PDF mode. (I don't know a DVI driver that
%   support several color stacks.)
% \item
%   \xfile{pdftex.def} 2007/01/02 v0.04b.
% \end{itemize}
% Package \xpackage{pdfcol} checks the requirements and
% sets switch \cs{ifpdfcolAvailable} accordingly.
%
% \subsection{Interface}
%
% \begin{declcs}{ifpdfcolAvailable}
% \end{declcs}
% If the requirements of section \ref{sec:req} are met the
% switch \cs{ifpdfcolAvailable} behaves as \cs{iftrue}.
% Otherwise the other interface macros in this section will
% be disabled with a message. Also the first use of such a
% macro will print a message. The messages are print to
% the \xext{log} file only if \pdfTeX\ is not used in PDF mode.
%
% \begin{declcs}{pdfcolErrorNoStacks}
% \end{declcs}
% The first call of \cs{pdfcolErrorNoStacks} prints an error
% message, if color stacks are not available.
%
% \begin{declcs}{pdfcolInitStack} \M{name}
% \end{declcs}
% A new color stack is initialized by \cs{pdfcolInitStack}.
% The \meta{name} is used for indentifying the stack. It usually
% consists of letters and digits. (The name must survive a \cs{csname}.)
%
% The intension of the macro is the definition of an additional
% color stack. Thus the stack is not page bounded like the
% main color stack. Black (\texttt{0 g 0 G}) is used as initial
% color value. And colors are written with modifier \texttt{direct}
% that means without setting the current transfer matrix and changing
% the current point (see documentation of \pdfTeX\ for
% |\pdfliteral direct{...}|).
%
% \begin{declcs}{pdfcolIfStackExists} \M{name} \M{then} \M{else}
% \end{declcs}
% Macro \cs{pdfcolIfStackExists} checks whether color stack \meta{name}
% exists. In case of success argument \meta{then} is executed
% and \meta{else} otherwise.
%
% \begin{declcs}{pdfcolSwitchStack} \M{name}
% \end{declcs}
% Macro \cs{pdfcolSwitchStack} switches the color stack. The color macros
% of package \xpackage{color} (or \xpackage{xcolor}) now uses the
% new color stack with name \meta{name}.
%
% \begin{declcs}{pdfcolSetCurrentColor}
% \end{declcs}
% Macro \cs{pdfcolSetCurrentColor} replaces the topmost
% entry of the stack by the current color (\cs{current@color}).
%
% \begin{declcs}{pdfcolSetCurrent} \M{name}
% \end{declcs}
% Macro \cs{pdfcolSetCurrent} sets the color that is read in
% the top-most entry of color stack \meta{name}. If \meta{name}
% is empty, the default color stack is used.
%
% \StopEventually{
% }
%
% \section{Implementation}
%
%    \begin{macrocode}
%<*package>
%    \end{macrocode}
%
% \subsection{Reload check and package identification}
%    Reload check, especially if the package is not used with \LaTeX.
%    \begin{macrocode}
\begingroup\catcode61\catcode48\catcode32=10\relax%
  \catcode13=5 % ^^M
  \endlinechar=13 %
  \catcode35=6 % #
  \catcode39=12 % '
  \catcode44=12 % ,
  \catcode45=12 % -
  \catcode46=12 % .
  \catcode58=12 % :
  \catcode64=11 % @
  \catcode123=1 % {
  \catcode125=2 % }
  \expandafter\let\expandafter\x\csname ver@pdfcol.sty\endcsname
  \ifx\x\relax % plain-TeX, first loading
  \else
    \def\empty{}%
    \ifx\x\empty % LaTeX, first loading,
      % variable is initialized, but \ProvidesPackage not yet seen
    \else
      \expandafter\ifx\csname PackageInfo\endcsname\relax
        \def\x#1#2{%
          \immediate\write-1{Package #1 Info: #2.}%
        }%
      \else
        \def\x#1#2{\PackageInfo{#1}{#2, stopped}}%
      \fi
      \x{pdfcol}{The package is already loaded}%
      \aftergroup\endinput
    \fi
  \fi
\endgroup%
%    \end{macrocode}
%    Package identification:
%    \begin{macrocode}
\begingroup\catcode61\catcode48\catcode32=10\relax%
  \catcode13=5 % ^^M
  \endlinechar=13 %
  \catcode35=6 % #
  \catcode39=12 % '
  \catcode40=12 % (
  \catcode41=12 % )
  \catcode44=12 % ,
  \catcode45=12 % -
  \catcode46=12 % .
  \catcode47=12 % /
  \catcode58=12 % :
  \catcode64=11 % @
  \catcode91=12 % [
  \catcode93=12 % ]
  \catcode123=1 % {
  \catcode125=2 % }
  \expandafter\ifx\csname ProvidesPackage\endcsname\relax
    \def\x#1#2#3[#4]{\endgroup
      \immediate\write-1{Package: #3 #4}%
      \xdef#1{#4}%
    }%
  \else
    \def\x#1#2[#3]{\endgroup
      #2[{#3}]%
      \ifx#1\@undefined
        \xdef#1{#3}%
      \fi
      \ifx#1\relax
        \xdef#1{#3}%
      \fi
    }%
  \fi
\expandafter\x\csname ver@pdfcol.sty\endcsname
\ProvidesPackage{pdfcol}%
  [2018/11/01 v1.5 Handle new color stacks for pdfTeX (HO)]%
%    \end{macrocode}
%
% \subsection{Catcodes}
%
%    \begin{macrocode}
\begingroup\catcode61\catcode48\catcode32=10\relax%
  \catcode13=5 % ^^M
  \endlinechar=13 %
  \catcode123=1 % {
  \catcode125=2 % }
  \catcode64=11 % @
  \def\x{\endgroup
    \expandafter\edef\csname PDFCOL@AtEnd\endcsname{%
      \endlinechar=\the\endlinechar\relax
      \catcode13=\the\catcode13\relax
      \catcode32=\the\catcode32\relax
      \catcode35=\the\catcode35\relax
      \catcode61=\the\catcode61\relax
      \catcode64=\the\catcode64\relax
      \catcode123=\the\catcode123\relax
      \catcode125=\the\catcode125\relax
    }%
  }%
\x\catcode61\catcode48\catcode32=10\relax%
\catcode13=5 % ^^M
\endlinechar=13 %
\catcode35=6 % #
\catcode64=11 % @
\catcode123=1 % {
\catcode125=2 % }
\def\TMP@EnsureCode#1#2{%
  \edef\PDFCOL@AtEnd{%
    \PDFCOL@AtEnd
    \catcode#1=\the\catcode#1\relax
  }%
  \catcode#1=#2\relax
}
\TMP@EnsureCode{39}{12}% '
\TMP@EnsureCode{40}{12}% (
\TMP@EnsureCode{41}{12}% )
\TMP@EnsureCode{43}{12}% +
\TMP@EnsureCode{44}{12}% ,
\TMP@EnsureCode{46}{12}% .
\TMP@EnsureCode{47}{12}% /
\TMP@EnsureCode{91}{12}% [
\TMP@EnsureCode{93}{12}% ]
\TMP@EnsureCode{96}{12}% `
\edef\PDFCOL@AtEnd{\PDFCOL@AtEnd\noexpand\endinput}
%    \end{macrocode}
%
% \subsection{Check requirements}
%
%    \begin{macro}{\PDFCOL@RequirePackage}
%    \begin{macrocode}
\begingroup\expandafter\expandafter\expandafter\endgroup
\expandafter\ifx\csname RequirePackage\endcsname\relax
  \def\PDFCOL@RequirePackage#1[#2]{\input #1.sty\relax}%
\else
  \def\PDFCOL@RequirePackage#1[#2]{%
    \RequirePackage{#1}[{#2}]%
  }%
\fi
%    \end{macrocode}
%    \end{macro}
%
% LuaTeX Compatability
%    \begin{macrocode}
\ifx\pdfextension\@undefined\else
 \def\pdfcolorstackinit {\pdffeedback colorstackinit}
 \protected\def\pdfcolorstack     {\pdfextension colorstack}
\fi
%    \end{macrocode}
%
%    \begin{macrocode}
\PDFCOL@RequirePackage{ltxcmds}[2010/03/01]
%    \end{macrocode}
%
%    \begin{macro}{ifpdfcolAvailable}
%    \begin{macrocode}
\ltx@newif\ifpdfcolAvailable
\pdfcolAvailabletrue
%    \end{macrocode}
%    \end{macro}
%
% \subsubsection{Check package \xpackage{luacolor}}
%
%    \begin{macrocode}
\ltx@newif\ifPDFCOL@luacolor
\begingroup\expandafter\expandafter\expandafter\endgroup
\expandafter\ifx\csname ver@luacolor.sty\endcsname\relax
  \PDFCOL@luacolorfalse
\else
  \PDFCOL@luacolortrue
\fi
%    \end{macrocode}
%
% \subsubsection{Check PDF mode}
%
%    \begin{macrocode}
\PDFCOL@RequirePackage{infwarerr}[2007/09/09]
\PDFCOL@RequirePackage{ifpdf}[2007/09/09]
\ifcase\ifpdf\ifPDFCOL@luacolor 1\fi\else 1\fi0 %
  \def\PDFCOL@Message{%
    \@PackageWarningNoLine{pdfcol}%
  }%
\else
  \pdfcolAvailablefalse
  \def\PDFCOL@Message{%
    \@PackageInfoNoLine{pdfcol}%
  }%
  \PDFCOL@Message{%
    Interface disabled because of %
    \ifPDFCOL@luacolor
      package `luacolor'%
    \else
      missing PDF mode of pdfTeX%
    \fi
  }%
\fi
%    \end{macrocode}
%
% \subsubsection{Check version of \pdfTeX}
%
%    \begin{macrocode}
\ifpdfcolAvailable
  \begingroup\expandafter\expandafter\expandafter\endgroup
  \expandafter\ifx\csname pdfcolorstack\endcsname\relax
    \pdfcolAvailablefalse
    \PDFCOL@Message{%
      Interface disabled because of too old pdfTeX.\MessageBreak
      Required is version 1.40+ for \string\pdfcolorstack
    }%
  \fi
\fi
\ifpdfcolAvailable
  \begingroup\expandafter\expandafter\expandafter\endgroup
  \expandafter\ifx\csname pdfcolorstack\endcsname\relax
    \pdfcolAvailablefalse
    \PDFCOL@Message{%
      Interface disabled because of too old pdfTeX.\MessageBreak
      Required is version 1.40+ for \string\pdfcolorstackinit
    }%
  \fi
\fi
%    \end{macrocode}
%
% \subsubsection{Check \xfile{pdftex.def}}
%
%    \begin{macrocode}
\ifpdfcolAvailable
  \begingroup\expandafter\expandafter\expandafter\endgroup
  \expandafter\ifx\csname @pdfcolorstack\endcsname\relax
%    \end{macrocode}
%    Try to load package color if it is not yet loaded (\LaTeX\ case).
%    \begin{macrocode}
    \begingroup\expandafter\expandafter\expandafter\endgroup
    \expandafter\ifx\csname ver@color.sty\endcsname\relax
      \begingroup\expandafter\expandafter\expandafter\endgroup
      \expandafter\ifx\csname documentclass\endcsname\relax
      \else
        \RequirePackage[pdftex]{color}\relax
      \fi
    \fi
    \begingroup\expandafter\expandafter\expandafter\endgroup
    \expandafter\ifx\csname @pdfcolorstack\endcsname\relax
      \pdfcolAvailablefalse
      \PDFCOL@Message{%
        Interface disabled because `pdftex.def'\MessageBreak
        is not loaded or it is too old.\MessageBreak
        Required is version 0.04b or greater%
      }%
    \fi
  \fi
\fi
%    \end{macrocode}
%
%    \begin{macrocode}
\let\pdfcolAvailabletrue\relax
\let\pdfcolAvailablefalse\relax
%    \end{macrocode}
%
% \subsection{Enabled interface macros}
%
%    \begin{macrocode}
\ifpdfcolAvailable
%    \end{macrocode}
%
%    \begin{macro}{\pdfcolErrorNoStacks}
%    \begin{macrocode}
  \let\pdfcolErrorNoStacks\relax
%    \end{macrocode}
%    \end{macro}
%
%    \begin{macro}{\pdfcol@Value}
%    \begin{macrocode}
  \expandafter\ifx\csname pdfcol@Value\endcsname\relax
    \def\pdfcol@Value{0 g 0 G}%
  \fi
%    \end{macrocode}
%    \end{macro}
%
%    \begin{macro}{\pdfcol@LiteralModifier}
%    \begin{macrocode}
  \expandafter\ifx\csname pdfcol@LiteralModifier\endcsname\relax
    \def\pdfcol@LiteralModifier{direct}%
  \fi
%    \end{macrocode}
%    \end{macro}
%
%    \begin{macro}{\pdfcolInitStack}
%    \begin{macrocode}
  \def\pdfcolInitStack#1{%
    \expandafter\ifx\csname pdfcol@Stack@#1\endcsname\relax
      \global\expandafter\chardef\csname pdfcol@Stack@#1\endcsname=%
          \pdfcolorstackinit\pdfcol@LiteralModifier{\pdfcol@Value}%
          \relax
      \@PackageInfo{pdfcol}{%
        New color stack `#1' = \number\csname pdfcol@Stack@#1\endcsname
      }%
    \else
      \@PackageError{pdfcol}{%
        Stack `#1' is already defined%
      }\@ehc
    \fi
  }%
%    \end{macrocode}
%    \end{macro}
%
%    \begin{macro}{\pdfcolIfStackExists}
%    \begin{macrocode}
  \def\pdfcolIfStackExists#1{%
    \expandafter\ifx\csname pdfcol@Stack@#1\endcsname\relax
      \expandafter\@secondoftwo
    \else
      \expandafter\@firstoftwo
    \fi
  }%
%    \end{macrocode}
%    \end{macro}
%    \begin{macro}{\@firstoftwo}
%    \begin{macrocode}
  \expandafter\ifx\csname @firstoftwo\endcsname\relax
    \long\def\@firstoftwo#1#2{#1}%
  \fi
%    \end{macrocode}
%    \end{macro}
%    \begin{macro}{\@secondoftwo}
%    \begin{macrocode}
  \expandafter\ifx\csname @secondoftwo\endcsname\relax
    \long\def\@secondoftwo#1#2{#2}%
  \fi
%    \end{macrocode}
%    \end{macro}
%
%    \begin{macro}{\pdfcolSwitchStack}
%    \begin{macrocode}
  \def\pdfcolSwitchStack#1{%
    \pdfcolIfStackExists{#1}{%
      \expandafter\let\expandafter\@pdfcolorstack
                      \csname pdfcol@Stack@#1\endcsname
    }{%
      \pdfcol@ErrorNoStack{#1}%
    }%
  }%
%    \end{macrocode}
%    \end{macro}
%
%    \begin{macro}{\pdfcolSetCurrentColor}
%    \begin{macrocode}
  \def\pdfcolSetCurrentColor{%
    \pdfcolorstack\@pdfcolorstack set{\current@color}%
  }%
%    \end{macrocode}
%    \end{macro}
%
%    \begin{macro}{\pdfcolSetCurrent}
%    \begin{macrocode}
  \def\pdfcolSetCurrent#1{%
    \ifx\\#1\\%
      \pdfcolorstack\@pdfcolorstack current\relax
    \else
      \pdfcolIfStackExists{#1}{%
        \pdfcolorstack\csname pdfcol@Stack@#1\endcsname current\relax
      }{%
        \pdfcol@ErrorNoStack{#1}%
      }%
    \fi
  }%
%    \end{macrocode}
%    \end{macro}
%
%    \begin{macro}{\pdfcol@ErrorNoStack}
%    \begin{macrocode}
  \def\pdfcol@ErrorNoStack#1{%
    \@PackageError{pdfcol}{Stack `#1' does not exists}\@ehc
  }%
%    \end{macrocode}
%    \end{macro}
%
% \subsection{Disabled interface macros}
%
%    \begin{macrocode}
\else
%    \end{macrocode}
%
%    \begin{macro}{\pdfcolErrorNoStacks}
%    \begin{macrocode}
  \def\pdfcolErrorNoStacks{%
    \@PackageError{pdfcol}{%
      Color stacks are not available%
    }{%
      Update pdfTeX (1.40) and `pdftex.def' (0.04b) %
          if necessary.\MessageBreak
      Ensure that `pdftex.def' is loaded %
          (package `color' or `xcolor').\MessageBreak
      Further messages can be found in TeX's %
          protocol file `\jobname.log'.\MessageBreak
      \MessageBreak
      \@ehc
    }%
    \global\let\pdfcolErrorNoStacks\relax
  }%
%    \end{macrocode}
%    \end{macro}
%
%    \begin{macro}{\PDFCOL@Disabled}
%    \begin{macrocode}
  \def\PDFCOL@Disabled{%
    \PDFCOL@Message{%
      pdfTeX's color stacks are not available%
    }%
    \global\let\PDFCOL@Disabled\relax
  }%
%    \end{macrocode}
%    \end{macro}
%
%    \begin{macro}{\pdfcolInitStack}
%    \begin{macrocode}
  \def\pdfcolInitStack#1{%
    \PDFCOL@Disabled
  }%
%    \end{macrocode}
%    \end{macro}
%
%    \begin{macro}{\pdfcolIfStackExists}
%    \begin{macrocode}
  \long\def\pdfcolIfStackExists#1#2#3{#3}%
%    \end{macrocode}
%    \end{macro}
%
%    \begin{macro}{\pdfcolSwitchStack}
%    \begin{macrocode}
  \def\pdfcolSwitchStack#1{%
    \PDFCOL@Disabled
  }%
%    \end{macrocode}
%    \end{macro}
%
%    \begin{macro}{\pdfcolSetCurrentColor}
%    \begin{macrocode}
  \def\pdfcolSetCurrentColor{%
    \PDFCOL@Disabled
  }%
%    \end{macrocode}
%    \end{macro}
%
%    \begin{macro}{\pdfcolSetCurrent}
%    \begin{macrocode}
  \def\pdfcolSetCurrent#1{%
    \PDFCOL@Disabled
  }%
%    \end{macrocode}
%    \end{macro}
%    \begin{macrocode}
\fi
%    \end{macrocode}
%
%    \begin{macrocode}
\PDFCOL@AtEnd%
%</package>
%    \end{macrocode}
%% \section{Installation}
%
% \subsection{Download}
%
% \paragraph{Package.} This package is available on
% CTAN\footnote{\CTANpkg{pdfcol}}:
% \begin{description}
% \item[\CTAN{macros/latex/contrib/oberdiek/pdfcol.dtx}] The source file.
% \item[\CTAN{macros/latex/contrib/oberdiek/pdfcol.pdf}] Documentation.
% \end{description}
%
%
% \paragraph{Bundle.} All the packages of the bundle `oberdiek'
% are also available in a TDS compliant ZIP archive. There
% the packages are already unpacked and the documentation files
% are generated. The files and directories obey the TDS standard.
% \begin{description}
% \item[\CTANinstall{install/macros/latex/contrib/oberdiek.tds.zip}]
% \end{description}
% \emph{TDS} refers to the standard ``A Directory Structure
% for \TeX\ Files'' (\CTANpkg{tds}). Directories
% with \xfile{texmf} in their name are usually organized this way.
%
% \subsection{Bundle installation}
%
% \paragraph{Unpacking.} Unpack the \xfile{oberdiek.tds.zip} in the
% TDS tree (also known as \xfile{texmf} tree) of your choice.
% Example (linux):
% \begin{quote}
%   |unzip oberdiek.tds.zip -d ~/texmf|
% \end{quote}
%
% \subsection{Package installation}
%
% \paragraph{Unpacking.} The \xfile{.dtx} file is a self-extracting
% \docstrip\ archive. The files are extracted by running the
% \xfile{.dtx} through \plainTeX:
% \begin{quote}
%   \verb|tex pdfcol.dtx|
% \end{quote}
%
% \paragraph{TDS.} Now the different files must be moved into
% the different directories in your installation TDS tree
% (also known as \xfile{texmf} tree):
% \begin{quote}
% \def\t{^^A
% \begin{tabular}{@{}>{\ttfamily}l@{ $\rightarrow$ }>{\ttfamily}l@{}}
%   pdfcol.sty & tex/generic/oberdiek/pdfcol.sty\\
%   pdfcol.pdf & doc/latex/oberdiek/pdfcol.pdf\\
%   pdfcol.dtx & source/latex/oberdiek/pdfcol.dtx\\
% \end{tabular}^^A
% }^^A
% \sbox0{\t}^^A
% \ifdim\wd0>\linewidth
%   \begingroup
%     \advance\linewidth by\leftmargin
%     \advance\linewidth by\rightmargin
%   \edef\x{\endgroup
%     \def\noexpand\lw{\the\linewidth}^^A
%   }\x
%   \def\lwbox{^^A
%     \leavevmode
%     \hbox to \linewidth{^^A
%       \kern-\leftmargin\relax
%       \hss
%       \usebox0
%       \hss
%       \kern-\rightmargin\relax
%     }^^A
%   }^^A
%   \ifdim\wd0>\lw
%     \sbox0{\small\t}^^A
%     \ifdim\wd0>\linewidth
%       \ifdim\wd0>\lw
%         \sbox0{\footnotesize\t}^^A
%         \ifdim\wd0>\linewidth
%           \ifdim\wd0>\lw
%             \sbox0{\scriptsize\t}^^A
%             \ifdim\wd0>\linewidth
%               \ifdim\wd0>\lw
%                 \sbox0{\tiny\t}^^A
%                 \ifdim\wd0>\linewidth
%                   \lwbox
%                 \else
%                   \usebox0
%                 \fi
%               \else
%                 \lwbox
%               \fi
%             \else
%               \usebox0
%             \fi
%           \else
%             \lwbox
%           \fi
%         \else
%           \usebox0
%         \fi
%       \else
%         \lwbox
%       \fi
%     \else
%       \usebox0
%     \fi
%   \else
%     \lwbox
%   \fi
% \else
%   \usebox0
% \fi
% \end{quote}
% If you have a \xfile{docstrip.cfg} that configures and enables \docstrip's
% TDS installing feature, then some files can already be in the right
% place, see the documentation of \docstrip.
%
% \subsection{Refresh file name databases}
%
% If your \TeX~distribution
% (\TeX\,Live, \mikTeX, \dots) relies on file name databases, you must refresh
% these. For example, \TeX\,Live\ users run \verb|texhash| or
% \verb|mktexlsr|.
%
% \subsection{Some details for the interested}
%
% \paragraph{Unpacking with \LaTeX.}
% The \xfile{.dtx} chooses its action depending on the format:
% \begin{description}
% \item[\plainTeX:] Run \docstrip\ and extract the files.
% \item[\LaTeX:] Generate the documentation.
% \end{description}
% If you insist on using \LaTeX\ for \docstrip\ (really,
% \docstrip\ does not need \LaTeX), then inform the autodetect routine
% about your intention:
% \begin{quote}
%   \verb|latex \let\install=y\input{pdfcol.dtx}|
% \end{quote}
% Do not forget to quote the argument according to the demands
% of your shell.
%
% \paragraph{Generating the documentation.}
% You can use both the \xfile{.dtx} or the \xfile{.drv} to generate
% the documentation. The process can be configured by the
% configuration file \xfile{ltxdoc.cfg}. For instance, put this
% line into this file, if you want to have A4 as paper format:
% \begin{quote}
%   \verb|\PassOptionsToClass{a4paper}{article}|
% \end{quote}
% An example follows how to generate the
% documentation with pdf\LaTeX:
% \begin{quote}
%\begin{verbatim}
%pdflatex pdfcol.dtx
%makeindex -s gind.ist pdfcol.idx
%pdflatex pdfcol.dtx
%makeindex -s gind.ist pdfcol.idx
%pdflatex pdfcol.dtx
%\end{verbatim}
% \end{quote}
%
% \begin{History}
%   \begin{Version}{2007/09/09 v1.0}
%   \item
%     First version.
%   \end{Version}
%   \begin{Version}{2007/12/09 v1.1}
%   \item
%     \cs{pdfcolSetCurrentColor} added.
%   \end{Version}
%   \begin{Version}{2007/12/12 v1.2}
%   \item
%     Detection for package \xpackage{luacolor} added.
%   \end{Version}
%   \begin{Version}{2016/05/16 v1.3}
%   \item
%     Documentation updates.
%   \end{Version}
%   \begin{Version}{2016/05/17 v1.4}
%   \item
%     Use luatex85 package for new luatex compatibility
%   \end{Version}
%   \begin{Version}{2018/11/01 v1.5}
%   \item
%     Remove luatex85 dependency
%   \end{Version}
% \end{History}
%
% \PrintIndex
%
% \Finale
\endinput

%        (quote the arguments according to the demands of your shell)
%
% Documentation:
%    (a) If pdfcol.drv is present:
%           latex pdfcol.drv
%    (b) Without pdfcol.drv:
%           latex pdfcol.dtx; ...
%    The class ltxdoc loads the configuration file ltxdoc.cfg
%    if available. Here you can specify further options, e.g.
%    use A4 as paper format:
%       \PassOptionsToClass{a4paper}{article}
%
%    Programm calls to get the documentation (example):
%       pdflatex pdfcol.dtx
%       makeindex -s gind.ist pdfcol.idx
%       pdflatex pdfcol.dtx
%       makeindex -s gind.ist pdfcol.idx
%       pdflatex pdfcol.dtx
%
% Installation:
%    TDS:tex/generic/oberdiek/pdfcol.sty
%    TDS:doc/latex/oberdiek/pdfcol.pdf
%    TDS:source/latex/oberdiek/pdfcol.dtx
%
%<*ignore>
\begingroup
  \catcode123=1 %
  \catcode125=2 %
  \def\x{LaTeX2e}%
\expandafter\endgroup
\ifcase 0\ifx\install y1\fi\expandafter
         \ifx\csname processbatchFile\endcsname\relax\else1\fi
         \ifx\fmtname\x\else 1\fi\relax
\else\csname fi\endcsname
%</ignore>
%<*install>
\input docstrip.tex
\Msg{************************************************************************}
\Msg{* Installation}
\Msg{* Package: pdfcol 2018/11/01 v1.5 Handle new color stacks for pdfTeX (HO)}
\Msg{************************************************************************}

\keepsilent
\askforoverwritefalse

\let\MetaPrefix\relax
\preamble

This is a generated file.

Project: pdfcol
Version: 2018/11/01 v1.5

Copyright (C)
   2007 Heiko Oberdiek
   2016-2019 Oberdiek Package Support Group

This work may be distributed and/or modified under the
conditions of the LaTeX Project Public License, either
version 1.3c of this license or (at your option) any later
version. This version of this license is in
   https://www.latex-project.org/lppl/lppl-1-3c.txt
and the latest version of this license is in
   https://www.latex-project.org/lppl.txt
and version 1.3 or later is part of all distributions of
LaTeX version 2005/12/01 or later.

This work has the LPPL maintenance status "maintained".

The Current Maintainers of this work are
Heiko Oberdiek and the Oberdiek Package Support Group
https://github.com/ho-tex/oberdiek/issues


The Base Interpreter refers to any `TeX-Format',
because some files are installed in TDS:tex/generic//.

This work consists of the main source file pdfcol.dtx
and the derived files
   pdfcol.sty, pdfcol.pdf, pdfcol.ins, pdfcol.drv, pdfcol-test1.tex,
   pdfcol-test2.tex, pdfcol-test3.tex, pdfcol-test4.tex.

\endpreamble
\let\MetaPrefix\DoubleperCent

\generate{%
  \file{pdfcol.ins}{\from{pdfcol.dtx}{install}}%
  \file{pdfcol.drv}{\from{pdfcol.dtx}{driver}}%
  \usedir{tex/generic/oberdiek}%
  \file{pdfcol.sty}{\from{pdfcol.dtx}{package}}%
%  \usedir{doc/latex/oberdiek/test}%
%  \file{pdfcol-test1.tex}{\from{pdfcol.dtx}{test1}}%
%  \file{pdfcol-test2.tex}{\from{pdfcol.dtx}{test2}}%
%  \file{pdfcol-test3.tex}{\from{pdfcol.dtx}{test3}}%
%  \file{pdfcol-test4.tex}{\from{pdfcol.dtx}{test4}}%
}

\catcode32=13\relax% active space
\let =\space%
\Msg{************************************************************************}
\Msg{*}
\Msg{* To finish the installation you have to move the following}
\Msg{* file into a directory searched by TeX:}
\Msg{*}
\Msg{*     pdfcol.sty}
\Msg{*}
\Msg{* To produce the documentation run the file `pdfcol.drv'}
\Msg{* through LaTeX.}
\Msg{*}
\Msg{* Happy TeXing!}
\Msg{*}
\Msg{************************************************************************}

\endbatchfile
%</install>
%<*ignore>
\fi
%</ignore>
%<*driver>
\NeedsTeXFormat{LaTeX2e}
\ProvidesFile{pdfcol.drv}%
  [2018/11/01 v1.5 Handle new color stacks for pdfTeX (HO)]%
\documentclass{ltxdoc}
\usepackage{holtxdoc}[2011/11/22]
\begin{document}
  \DocInput{pdfcol.dtx}%
\end{document}
%</driver>
% \fi
%
%
%
% \GetFileInfo{pdfcol.drv}
%
% \title{The \xpackage{pdfcol} package}
% \date{2018/11/01 v1.5}
% \author{Heiko Oberdiek\thanks
% {Please report any issues at \url{https://github.com/ho-tex/oberdiek/issues}}}
%
% \maketitle
%
% \begin{abstract}
% Since version 1.40 \pdfTeX\ supports color stacks.
% The driver file \xfile{pdftex.def} for package \xpackage{color}
% defines and uses a main color stack since version v0.04b.
% Package \xpackage{pdfcol} is intended for package writers.
% It defines macros for setting and maintaining new color stacks.
% \end{abstract}
%
% \tableofcontents
%
% \section{Documentation}
%
% Version 1.40 of \pdfTeX\ adds new primitives \cs{pdfcolorstackinit}
% and \cs{pdfcolorstack}. Now color stacks can be defined and used.
% A main color stack is maintained by the driver file \xfile{pdftex.def}
% similar to dvips or dvipdfm. However the number of color stacks
% is not limited to one in \pdfTeX. Thus further color problems
% can now be solved, such as footnotes across pages or text
% that is set in parallel columns (e.g. packages \xpackage{parallel}
% or \xpackage{parcolumn}). Unlike the main color stack,
% the support by additional color stacks cannot be done in
% a transparent manner.
%
% This package \xpackage{pdfcol} provides an easier interface to
% additional color stacks without the need to use the
% low level primitives.
%
% \subsection{Requirements}
% \label{sec:req}
%
% \begin{itemize}
% \item
%   \pdfTeX\ 1.40 or greater.
% \item
%   \pdfTeX in PDF mode. (I don't know a DVI driver that
%   support several color stacks.)
% \item
%   \xfile{pdftex.def} 2007/01/02 v0.04b.
% \end{itemize}
% Package \xpackage{pdfcol} checks the requirements and
% sets switch \cs{ifpdfcolAvailable} accordingly.
%
% \subsection{Interface}
%
% \begin{declcs}{ifpdfcolAvailable}
% \end{declcs}
% If the requirements of section \ref{sec:req} are met the
% switch \cs{ifpdfcolAvailable} behaves as \cs{iftrue}.
% Otherwise the other interface macros in this section will
% be disabled with a message. Also the first use of such a
% macro will print a message. The messages are print to
% the \xext{log} file only if \pdfTeX\ is not used in PDF mode.
%
% \begin{declcs}{pdfcolErrorNoStacks}
% \end{declcs}
% The first call of \cs{pdfcolErrorNoStacks} prints an error
% message, if color stacks are not available.
%
% \begin{declcs}{pdfcolInitStack} \M{name}
% \end{declcs}
% A new color stack is initialized by \cs{pdfcolInitStack}.
% The \meta{name} is used for indentifying the stack. It usually
% consists of letters and digits. (The name must survive a \cs{csname}.)
%
% The intension of the macro is the definition of an additional
% color stack. Thus the stack is not page bounded like the
% main color stack. Black (\texttt{0 g 0 G}) is used as initial
% color value. And colors are written with modifier \texttt{direct}
% that means without setting the current transfer matrix and changing
% the current point (see documentation of \pdfTeX\ for
% |\pdfliteral direct{...}|).
%
% \begin{declcs}{pdfcolIfStackExists} \M{name} \M{then} \M{else}
% \end{declcs}
% Macro \cs{pdfcolIfStackExists} checks whether color stack \meta{name}
% exists. In case of success argument \meta{then} is executed
% and \meta{else} otherwise.
%
% \begin{declcs}{pdfcolSwitchStack} \M{name}
% \end{declcs}
% Macro \cs{pdfcolSwitchStack} switches the color stack. The color macros
% of package \xpackage{color} (or \xpackage{xcolor}) now uses the
% new color stack with name \meta{name}.
%
% \begin{declcs}{pdfcolSetCurrentColor}
% \end{declcs}
% Macro \cs{pdfcolSetCurrentColor} replaces the topmost
% entry of the stack by the current color (\cs{current@color}).
%
% \begin{declcs}{pdfcolSetCurrent} \M{name}
% \end{declcs}
% Macro \cs{pdfcolSetCurrent} sets the color that is read in
% the top-most entry of color stack \meta{name}. If \meta{name}
% is empty, the default color stack is used.
%
% \StopEventually{
% }
%
% \section{Implementation}
%
%    \begin{macrocode}
%<*package>
%    \end{macrocode}
%
% \subsection{Reload check and package identification}
%    Reload check, especially if the package is not used with \LaTeX.
%    \begin{macrocode}
\begingroup\catcode61\catcode48\catcode32=10\relax%
  \catcode13=5 % ^^M
  \endlinechar=13 %
  \catcode35=6 % #
  \catcode39=12 % '
  \catcode44=12 % ,
  \catcode45=12 % -
  \catcode46=12 % .
  \catcode58=12 % :
  \catcode64=11 % @
  \catcode123=1 % {
  \catcode125=2 % }
  \expandafter\let\expandafter\x\csname ver@pdfcol.sty\endcsname
  \ifx\x\relax % plain-TeX, first loading
  \else
    \def\empty{}%
    \ifx\x\empty % LaTeX, first loading,
      % variable is initialized, but \ProvidesPackage not yet seen
    \else
      \expandafter\ifx\csname PackageInfo\endcsname\relax
        \def\x#1#2{%
          \immediate\write-1{Package #1 Info: #2.}%
        }%
      \else
        \def\x#1#2{\PackageInfo{#1}{#2, stopped}}%
      \fi
      \x{pdfcol}{The package is already loaded}%
      \aftergroup\endinput
    \fi
  \fi
\endgroup%
%    \end{macrocode}
%    Package identification:
%    \begin{macrocode}
\begingroup\catcode61\catcode48\catcode32=10\relax%
  \catcode13=5 % ^^M
  \endlinechar=13 %
  \catcode35=6 % #
  \catcode39=12 % '
  \catcode40=12 % (
  \catcode41=12 % )
  \catcode44=12 % ,
  \catcode45=12 % -
  \catcode46=12 % .
  \catcode47=12 % /
  \catcode58=12 % :
  \catcode64=11 % @
  \catcode91=12 % [
  \catcode93=12 % ]
  \catcode123=1 % {
  \catcode125=2 % }
  \expandafter\ifx\csname ProvidesPackage\endcsname\relax
    \def\x#1#2#3[#4]{\endgroup
      \immediate\write-1{Package: #3 #4}%
      \xdef#1{#4}%
    }%
  \else
    \def\x#1#2[#3]{\endgroup
      #2[{#3}]%
      \ifx#1\@undefined
        \xdef#1{#3}%
      \fi
      \ifx#1\relax
        \xdef#1{#3}%
      \fi
    }%
  \fi
\expandafter\x\csname ver@pdfcol.sty\endcsname
\ProvidesPackage{pdfcol}%
  [2018/11/01 v1.5 Handle new color stacks for pdfTeX (HO)]%
%    \end{macrocode}
%
% \subsection{Catcodes}
%
%    \begin{macrocode}
\begingroup\catcode61\catcode48\catcode32=10\relax%
  \catcode13=5 % ^^M
  \endlinechar=13 %
  \catcode123=1 % {
  \catcode125=2 % }
  \catcode64=11 % @
  \def\x{\endgroup
    \expandafter\edef\csname PDFCOL@AtEnd\endcsname{%
      \endlinechar=\the\endlinechar\relax
      \catcode13=\the\catcode13\relax
      \catcode32=\the\catcode32\relax
      \catcode35=\the\catcode35\relax
      \catcode61=\the\catcode61\relax
      \catcode64=\the\catcode64\relax
      \catcode123=\the\catcode123\relax
      \catcode125=\the\catcode125\relax
    }%
  }%
\x\catcode61\catcode48\catcode32=10\relax%
\catcode13=5 % ^^M
\endlinechar=13 %
\catcode35=6 % #
\catcode64=11 % @
\catcode123=1 % {
\catcode125=2 % }
\def\TMP@EnsureCode#1#2{%
  \edef\PDFCOL@AtEnd{%
    \PDFCOL@AtEnd
    \catcode#1=\the\catcode#1\relax
  }%
  \catcode#1=#2\relax
}
\TMP@EnsureCode{39}{12}% '
\TMP@EnsureCode{40}{12}% (
\TMP@EnsureCode{41}{12}% )
\TMP@EnsureCode{43}{12}% +
\TMP@EnsureCode{44}{12}% ,
\TMP@EnsureCode{46}{12}% .
\TMP@EnsureCode{47}{12}% /
\TMP@EnsureCode{91}{12}% [
\TMP@EnsureCode{93}{12}% ]
\TMP@EnsureCode{96}{12}% `
\edef\PDFCOL@AtEnd{\PDFCOL@AtEnd\noexpand\endinput}
%    \end{macrocode}
%
% \subsection{Check requirements}
%
%    \begin{macro}{\PDFCOL@RequirePackage}
%    \begin{macrocode}
\begingroup\expandafter\expandafter\expandafter\endgroup
\expandafter\ifx\csname RequirePackage\endcsname\relax
  \def\PDFCOL@RequirePackage#1[#2]{\input #1.sty\relax}%
\else
  \def\PDFCOL@RequirePackage#1[#2]{%
    \RequirePackage{#1}[{#2}]%
  }%
\fi
%    \end{macrocode}
%    \end{macro}
%
% LuaTeX Compatability
%    \begin{macrocode}
\ifx\pdfextension\@undefined\else
 \def\pdfcolorstackinit {\pdffeedback colorstackinit}
 \protected\def\pdfcolorstack     {\pdfextension colorstack}
\fi
%    \end{macrocode}
%
%    \begin{macrocode}
\PDFCOL@RequirePackage{ltxcmds}[2010/03/01]
%    \end{macrocode}
%
%    \begin{macro}{ifpdfcolAvailable}
%    \begin{macrocode}
\ltx@newif\ifpdfcolAvailable
\pdfcolAvailabletrue
%    \end{macrocode}
%    \end{macro}
%
% \subsubsection{Check package \xpackage{luacolor}}
%
%    \begin{macrocode}
\ltx@newif\ifPDFCOL@luacolor
\begingroup\expandafter\expandafter\expandafter\endgroup
\expandafter\ifx\csname ver@luacolor.sty\endcsname\relax
  \PDFCOL@luacolorfalse
\else
  \PDFCOL@luacolortrue
\fi
%    \end{macrocode}
%
% \subsubsection{Check PDF mode}
%
%    \begin{macrocode}
\PDFCOL@RequirePackage{infwarerr}[2007/09/09]
\PDFCOL@RequirePackage{ifpdf}[2007/09/09]
\ifcase\ifpdf\ifPDFCOL@luacolor 1\fi\else 1\fi0 %
  \def\PDFCOL@Message{%
    \@PackageWarningNoLine{pdfcol}%
  }%
\else
  \pdfcolAvailablefalse
  \def\PDFCOL@Message{%
    \@PackageInfoNoLine{pdfcol}%
  }%
  \PDFCOL@Message{%
    Interface disabled because of %
    \ifPDFCOL@luacolor
      package `luacolor'%
    \else
      missing PDF mode of pdfTeX%
    \fi
  }%
\fi
%    \end{macrocode}
%
% \subsubsection{Check version of \pdfTeX}
%
%    \begin{macrocode}
\ifpdfcolAvailable
  \begingroup\expandafter\expandafter\expandafter\endgroup
  \expandafter\ifx\csname pdfcolorstack\endcsname\relax
    \pdfcolAvailablefalse
    \PDFCOL@Message{%
      Interface disabled because of too old pdfTeX.\MessageBreak
      Required is version 1.40+ for \string\pdfcolorstack
    }%
  \fi
\fi
\ifpdfcolAvailable
  \begingroup\expandafter\expandafter\expandafter\endgroup
  \expandafter\ifx\csname pdfcolorstack\endcsname\relax
    \pdfcolAvailablefalse
    \PDFCOL@Message{%
      Interface disabled because of too old pdfTeX.\MessageBreak
      Required is version 1.40+ for \string\pdfcolorstackinit
    }%
  \fi
\fi
%    \end{macrocode}
%
% \subsubsection{Check \xfile{pdftex.def}}
%
%    \begin{macrocode}
\ifpdfcolAvailable
  \begingroup\expandafter\expandafter\expandafter\endgroup
  \expandafter\ifx\csname @pdfcolorstack\endcsname\relax
%    \end{macrocode}
%    Try to load package color if it is not yet loaded (\LaTeX\ case).
%    \begin{macrocode}
    \begingroup\expandafter\expandafter\expandafter\endgroup
    \expandafter\ifx\csname ver@color.sty\endcsname\relax
      \begingroup\expandafter\expandafter\expandafter\endgroup
      \expandafter\ifx\csname documentclass\endcsname\relax
      \else
        \RequirePackage[pdftex]{color}\relax
      \fi
    \fi
    \begingroup\expandafter\expandafter\expandafter\endgroup
    \expandafter\ifx\csname @pdfcolorstack\endcsname\relax
      \pdfcolAvailablefalse
      \PDFCOL@Message{%
        Interface disabled because `pdftex.def'\MessageBreak
        is not loaded or it is too old.\MessageBreak
        Required is version 0.04b or greater%
      }%
    \fi
  \fi
\fi
%    \end{macrocode}
%
%    \begin{macrocode}
\let\pdfcolAvailabletrue\relax
\let\pdfcolAvailablefalse\relax
%    \end{macrocode}
%
% \subsection{Enabled interface macros}
%
%    \begin{macrocode}
\ifpdfcolAvailable
%    \end{macrocode}
%
%    \begin{macro}{\pdfcolErrorNoStacks}
%    \begin{macrocode}
  \let\pdfcolErrorNoStacks\relax
%    \end{macrocode}
%    \end{macro}
%
%    \begin{macro}{\pdfcol@Value}
%    \begin{macrocode}
  \expandafter\ifx\csname pdfcol@Value\endcsname\relax
    \def\pdfcol@Value{0 g 0 G}%
  \fi
%    \end{macrocode}
%    \end{macro}
%
%    \begin{macro}{\pdfcol@LiteralModifier}
%    \begin{macrocode}
  \expandafter\ifx\csname pdfcol@LiteralModifier\endcsname\relax
    \def\pdfcol@LiteralModifier{direct}%
  \fi
%    \end{macrocode}
%    \end{macro}
%
%    \begin{macro}{\pdfcolInitStack}
%    \begin{macrocode}
  \def\pdfcolInitStack#1{%
    \expandafter\ifx\csname pdfcol@Stack@#1\endcsname\relax
      \global\expandafter\chardef\csname pdfcol@Stack@#1\endcsname=%
          \pdfcolorstackinit\pdfcol@LiteralModifier{\pdfcol@Value}%
          \relax
      \@PackageInfo{pdfcol}{%
        New color stack `#1' = \number\csname pdfcol@Stack@#1\endcsname
      }%
    \else
      \@PackageError{pdfcol}{%
        Stack `#1' is already defined%
      }\@ehc
    \fi
  }%
%    \end{macrocode}
%    \end{macro}
%
%    \begin{macro}{\pdfcolIfStackExists}
%    \begin{macrocode}
  \def\pdfcolIfStackExists#1{%
    \expandafter\ifx\csname pdfcol@Stack@#1\endcsname\relax
      \expandafter\@secondoftwo
    \else
      \expandafter\@firstoftwo
    \fi
  }%
%    \end{macrocode}
%    \end{macro}
%    \begin{macro}{\@firstoftwo}
%    \begin{macrocode}
  \expandafter\ifx\csname @firstoftwo\endcsname\relax
    \long\def\@firstoftwo#1#2{#1}%
  \fi
%    \end{macrocode}
%    \end{macro}
%    \begin{macro}{\@secondoftwo}
%    \begin{macrocode}
  \expandafter\ifx\csname @secondoftwo\endcsname\relax
    \long\def\@secondoftwo#1#2{#2}%
  \fi
%    \end{macrocode}
%    \end{macro}
%
%    \begin{macro}{\pdfcolSwitchStack}
%    \begin{macrocode}
  \def\pdfcolSwitchStack#1{%
    \pdfcolIfStackExists{#1}{%
      \expandafter\let\expandafter\@pdfcolorstack
                      \csname pdfcol@Stack@#1\endcsname
    }{%
      \pdfcol@ErrorNoStack{#1}%
    }%
  }%
%    \end{macrocode}
%    \end{macro}
%
%    \begin{macro}{\pdfcolSetCurrentColor}
%    \begin{macrocode}
  \def\pdfcolSetCurrentColor{%
    \pdfcolorstack\@pdfcolorstack set{\current@color}%
  }%
%    \end{macrocode}
%    \end{macro}
%
%    \begin{macro}{\pdfcolSetCurrent}
%    \begin{macrocode}
  \def\pdfcolSetCurrent#1{%
    \ifx\\#1\\%
      \pdfcolorstack\@pdfcolorstack current\relax
    \else
      \pdfcolIfStackExists{#1}{%
        \pdfcolorstack\csname pdfcol@Stack@#1\endcsname current\relax
      }{%
        \pdfcol@ErrorNoStack{#1}%
      }%
    \fi
  }%
%    \end{macrocode}
%    \end{macro}
%
%    \begin{macro}{\pdfcol@ErrorNoStack}
%    \begin{macrocode}
  \def\pdfcol@ErrorNoStack#1{%
    \@PackageError{pdfcol}{Stack `#1' does not exists}\@ehc
  }%
%    \end{macrocode}
%    \end{macro}
%
% \subsection{Disabled interface macros}
%
%    \begin{macrocode}
\else
%    \end{macrocode}
%
%    \begin{macro}{\pdfcolErrorNoStacks}
%    \begin{macrocode}
  \def\pdfcolErrorNoStacks{%
    \@PackageError{pdfcol}{%
      Color stacks are not available%
    }{%
      Update pdfTeX (1.40) and `pdftex.def' (0.04b) %
          if necessary.\MessageBreak
      Ensure that `pdftex.def' is loaded %
          (package `color' or `xcolor').\MessageBreak
      Further messages can be found in TeX's %
          protocol file `\jobname.log'.\MessageBreak
      \MessageBreak
      \@ehc
    }%
    \global\let\pdfcolErrorNoStacks\relax
  }%
%    \end{macrocode}
%    \end{macro}
%
%    \begin{macro}{\PDFCOL@Disabled}
%    \begin{macrocode}
  \def\PDFCOL@Disabled{%
    \PDFCOL@Message{%
      pdfTeX's color stacks are not available%
    }%
    \global\let\PDFCOL@Disabled\relax
  }%
%    \end{macrocode}
%    \end{macro}
%
%    \begin{macro}{\pdfcolInitStack}
%    \begin{macrocode}
  \def\pdfcolInitStack#1{%
    \PDFCOL@Disabled
  }%
%    \end{macrocode}
%    \end{macro}
%
%    \begin{macro}{\pdfcolIfStackExists}
%    \begin{macrocode}
  \long\def\pdfcolIfStackExists#1#2#3{#3}%
%    \end{macrocode}
%    \end{macro}
%
%    \begin{macro}{\pdfcolSwitchStack}
%    \begin{macrocode}
  \def\pdfcolSwitchStack#1{%
    \PDFCOL@Disabled
  }%
%    \end{macrocode}
%    \end{macro}
%
%    \begin{macro}{\pdfcolSetCurrentColor}
%    \begin{macrocode}
  \def\pdfcolSetCurrentColor{%
    \PDFCOL@Disabled
  }%
%    \end{macrocode}
%    \end{macro}
%
%    \begin{macro}{\pdfcolSetCurrent}
%    \begin{macrocode}
  \def\pdfcolSetCurrent#1{%
    \PDFCOL@Disabled
  }%
%    \end{macrocode}
%    \end{macro}
%    \begin{macrocode}
\fi
%    \end{macrocode}
%
%    \begin{macrocode}
\PDFCOL@AtEnd%
%</package>
%    \end{macrocode}
%% \section{Installation}
%
% \subsection{Download}
%
% \paragraph{Package.} This package is available on
% CTAN\footnote{\CTANpkg{pdfcol}}:
% \begin{description}
% \item[\CTAN{macros/latex/contrib/oberdiek/pdfcol.dtx}] The source file.
% \item[\CTAN{macros/latex/contrib/oberdiek/pdfcol.pdf}] Documentation.
% \end{description}
%
%
% \paragraph{Bundle.} All the packages of the bundle `oberdiek'
% are also available in a TDS compliant ZIP archive. There
% the packages are already unpacked and the documentation files
% are generated. The files and directories obey the TDS standard.
% \begin{description}
% \item[\CTANinstall{install/macros/latex/contrib/oberdiek.tds.zip}]
% \end{description}
% \emph{TDS} refers to the standard ``A Directory Structure
% for \TeX\ Files'' (\CTANpkg{tds}). Directories
% with \xfile{texmf} in their name are usually organized this way.
%
% \subsection{Bundle installation}
%
% \paragraph{Unpacking.} Unpack the \xfile{oberdiek.tds.zip} in the
% TDS tree (also known as \xfile{texmf} tree) of your choice.
% Example (linux):
% \begin{quote}
%   |unzip oberdiek.tds.zip -d ~/texmf|
% \end{quote}
%
% \subsection{Package installation}
%
% \paragraph{Unpacking.} The \xfile{.dtx} file is a self-extracting
% \docstrip\ archive. The files are extracted by running the
% \xfile{.dtx} through \plainTeX:
% \begin{quote}
%   \verb|tex pdfcol.dtx|
% \end{quote}
%
% \paragraph{TDS.} Now the different files must be moved into
% the different directories in your installation TDS tree
% (also known as \xfile{texmf} tree):
% \begin{quote}
% \def\t{^^A
% \begin{tabular}{@{}>{\ttfamily}l@{ $\rightarrow$ }>{\ttfamily}l@{}}
%   pdfcol.sty & tex/generic/oberdiek/pdfcol.sty\\
%   pdfcol.pdf & doc/latex/oberdiek/pdfcol.pdf\\
%   pdfcol.dtx & source/latex/oberdiek/pdfcol.dtx\\
% \end{tabular}^^A
% }^^A
% \sbox0{\t}^^A
% \ifdim\wd0>\linewidth
%   \begingroup
%     \advance\linewidth by\leftmargin
%     \advance\linewidth by\rightmargin
%   \edef\x{\endgroup
%     \def\noexpand\lw{\the\linewidth}^^A
%   }\x
%   \def\lwbox{^^A
%     \leavevmode
%     \hbox to \linewidth{^^A
%       \kern-\leftmargin\relax
%       \hss
%       \usebox0
%       \hss
%       \kern-\rightmargin\relax
%     }^^A
%   }^^A
%   \ifdim\wd0>\lw
%     \sbox0{\small\t}^^A
%     \ifdim\wd0>\linewidth
%       \ifdim\wd0>\lw
%         \sbox0{\footnotesize\t}^^A
%         \ifdim\wd0>\linewidth
%           \ifdim\wd0>\lw
%             \sbox0{\scriptsize\t}^^A
%             \ifdim\wd0>\linewidth
%               \ifdim\wd0>\lw
%                 \sbox0{\tiny\t}^^A
%                 \ifdim\wd0>\linewidth
%                   \lwbox
%                 \else
%                   \usebox0
%                 \fi
%               \else
%                 \lwbox
%               \fi
%             \else
%               \usebox0
%             \fi
%           \else
%             \lwbox
%           \fi
%         \else
%           \usebox0
%         \fi
%       \else
%         \lwbox
%       \fi
%     \else
%       \usebox0
%     \fi
%   \else
%     \lwbox
%   \fi
% \else
%   \usebox0
% \fi
% \end{quote}
% If you have a \xfile{docstrip.cfg} that configures and enables \docstrip's
% TDS installing feature, then some files can already be in the right
% place, see the documentation of \docstrip.
%
% \subsection{Refresh file name databases}
%
% If your \TeX~distribution
% (\TeX\,Live, \mikTeX, \dots) relies on file name databases, you must refresh
% these. For example, \TeX\,Live\ users run \verb|texhash| or
% \verb|mktexlsr|.
%
% \subsection{Some details for the interested}
%
% \paragraph{Unpacking with \LaTeX.}
% The \xfile{.dtx} chooses its action depending on the format:
% \begin{description}
% \item[\plainTeX:] Run \docstrip\ and extract the files.
% \item[\LaTeX:] Generate the documentation.
% \end{description}
% If you insist on using \LaTeX\ for \docstrip\ (really,
% \docstrip\ does not need \LaTeX), then inform the autodetect routine
% about your intention:
% \begin{quote}
%   \verb|latex \let\install=y% \iffalse meta-comment
%
% File: pdfcol.dtx
% Version: 2018/11/01 v1.5
% Info: Handle new color stacks for pdfTeX
%
% Copyright (C)
%    2007 Heiko Oberdiek
%    2016-2019 Oberdiek Package Support Group
%    https://github.com/ho-tex/oberdiek/issues
%
% This work may be distributed and/or modified under the
% conditions of the LaTeX Project Public License, either
% version 1.3c of this license or (at your option) any later
% version. This version of this license is in
%    https://www.latex-project.org/lppl/lppl-1-3c.txt
% and the latest version of this license is in
%    https://www.latex-project.org/lppl.txt
% and version 1.3 or later is part of all distributions of
% LaTeX version 2005/12/01 or later.
%
% This work has the LPPL maintenance status "maintained".
%
% The Current Maintainers of this work are
% Heiko Oberdiek and the Oberdiek Package Support Group
% https://github.com/ho-tex/oberdiek/issues
%
% The Base Interpreter refers to any `TeX-Format',
% because some files are installed in TDS:tex/generic//.
%
% This work consists of the main source file pdfcol.dtx
% and the derived files
%    pdfcol.sty, pdfcol.pdf, pdfcol.ins, pdfcol.drv, pdfcol-test1.tex,
%    pdfcol-test2.tex, pdfcol-test3.tex, pdfcol-test4.tex.
%
% Distribution:
%    CTAN:macros/latex/contrib/oberdiek/pdfcol.dtx
%    CTAN:macros/latex/contrib/oberdiek/pdfcol.pdf
%
% Unpacking:
%    (a) If pdfcol.ins is present:
%           tex pdfcol.ins
%    (b) Without pdfcol.ins:
%           tex pdfcol.dtx
%    (c) If you insist on using LaTeX
%           latex \let\install=y\input{pdfcol.dtx}
%        (quote the arguments according to the demands of your shell)
%
% Documentation:
%    (a) If pdfcol.drv is present:
%           latex pdfcol.drv
%    (b) Without pdfcol.drv:
%           latex pdfcol.dtx; ...
%    The class ltxdoc loads the configuration file ltxdoc.cfg
%    if available. Here you can specify further options, e.g.
%    use A4 as paper format:
%       \PassOptionsToClass{a4paper}{article}
%
%    Programm calls to get the documentation (example):
%       pdflatex pdfcol.dtx
%       makeindex -s gind.ist pdfcol.idx
%       pdflatex pdfcol.dtx
%       makeindex -s gind.ist pdfcol.idx
%       pdflatex pdfcol.dtx
%
% Installation:
%    TDS:tex/generic/oberdiek/pdfcol.sty
%    TDS:doc/latex/oberdiek/pdfcol.pdf
%    TDS:source/latex/oberdiek/pdfcol.dtx
%
%<*ignore>
\begingroup
  \catcode123=1 %
  \catcode125=2 %
  \def\x{LaTeX2e}%
\expandafter\endgroup
\ifcase 0\ifx\install y1\fi\expandafter
         \ifx\csname processbatchFile\endcsname\relax\else1\fi
         \ifx\fmtname\x\else 1\fi\relax
\else\csname fi\endcsname
%</ignore>
%<*install>
\input docstrip.tex
\Msg{************************************************************************}
\Msg{* Installation}
\Msg{* Package: pdfcol 2018/11/01 v1.5 Handle new color stacks for pdfTeX (HO)}
\Msg{************************************************************************}

\keepsilent
\askforoverwritefalse

\let\MetaPrefix\relax
\preamble

This is a generated file.

Project: pdfcol
Version: 2018/11/01 v1.5

Copyright (C)
   2007 Heiko Oberdiek
   2016-2019 Oberdiek Package Support Group

This work may be distributed and/or modified under the
conditions of the LaTeX Project Public License, either
version 1.3c of this license or (at your option) any later
version. This version of this license is in
   https://www.latex-project.org/lppl/lppl-1-3c.txt
and the latest version of this license is in
   https://www.latex-project.org/lppl.txt
and version 1.3 or later is part of all distributions of
LaTeX version 2005/12/01 or later.

This work has the LPPL maintenance status "maintained".

The Current Maintainers of this work are
Heiko Oberdiek and the Oberdiek Package Support Group
https://github.com/ho-tex/oberdiek/issues


The Base Interpreter refers to any `TeX-Format',
because some files are installed in TDS:tex/generic//.

This work consists of the main source file pdfcol.dtx
and the derived files
   pdfcol.sty, pdfcol.pdf, pdfcol.ins, pdfcol.drv, pdfcol-test1.tex,
   pdfcol-test2.tex, pdfcol-test3.tex, pdfcol-test4.tex.

\endpreamble
\let\MetaPrefix\DoubleperCent

\generate{%
  \file{pdfcol.ins}{\from{pdfcol.dtx}{install}}%
  \file{pdfcol.drv}{\from{pdfcol.dtx}{driver}}%
  \usedir{tex/generic/oberdiek}%
  \file{pdfcol.sty}{\from{pdfcol.dtx}{package}}%
%  \usedir{doc/latex/oberdiek/test}%
%  \file{pdfcol-test1.tex}{\from{pdfcol.dtx}{test1}}%
%  \file{pdfcol-test2.tex}{\from{pdfcol.dtx}{test2}}%
%  \file{pdfcol-test3.tex}{\from{pdfcol.dtx}{test3}}%
%  \file{pdfcol-test4.tex}{\from{pdfcol.dtx}{test4}}%
}

\catcode32=13\relax% active space
\let =\space%
\Msg{************************************************************************}
\Msg{*}
\Msg{* To finish the installation you have to move the following}
\Msg{* file into a directory searched by TeX:}
\Msg{*}
\Msg{*     pdfcol.sty}
\Msg{*}
\Msg{* To produce the documentation run the file `pdfcol.drv'}
\Msg{* through LaTeX.}
\Msg{*}
\Msg{* Happy TeXing!}
\Msg{*}
\Msg{************************************************************************}

\endbatchfile
%</install>
%<*ignore>
\fi
%</ignore>
%<*driver>
\NeedsTeXFormat{LaTeX2e}
\ProvidesFile{pdfcol.drv}%
  [2018/11/01 v1.5 Handle new color stacks for pdfTeX (HO)]%
\documentclass{ltxdoc}
\usepackage{holtxdoc}[2011/11/22]
\begin{document}
  \DocInput{pdfcol.dtx}%
\end{document}
%</driver>
% \fi
%
%
%
% \GetFileInfo{pdfcol.drv}
%
% \title{The \xpackage{pdfcol} package}
% \date{2018/11/01 v1.5}
% \author{Heiko Oberdiek\thanks
% {Please report any issues at \url{https://github.com/ho-tex/oberdiek/issues}}}
%
% \maketitle
%
% \begin{abstract}
% Since version 1.40 \pdfTeX\ supports color stacks.
% The driver file \xfile{pdftex.def} for package \xpackage{color}
% defines and uses a main color stack since version v0.04b.
% Package \xpackage{pdfcol} is intended for package writers.
% It defines macros for setting and maintaining new color stacks.
% \end{abstract}
%
% \tableofcontents
%
% \section{Documentation}
%
% Version 1.40 of \pdfTeX\ adds new primitives \cs{pdfcolorstackinit}
% and \cs{pdfcolorstack}. Now color stacks can be defined and used.
% A main color stack is maintained by the driver file \xfile{pdftex.def}
% similar to dvips or dvipdfm. However the number of color stacks
% is not limited to one in \pdfTeX. Thus further color problems
% can now be solved, such as footnotes across pages or text
% that is set in parallel columns (e.g. packages \xpackage{parallel}
% or \xpackage{parcolumn}). Unlike the main color stack,
% the support by additional color stacks cannot be done in
% a transparent manner.
%
% This package \xpackage{pdfcol} provides an easier interface to
% additional color stacks without the need to use the
% low level primitives.
%
% \subsection{Requirements}
% \label{sec:req}
%
% \begin{itemize}
% \item
%   \pdfTeX\ 1.40 or greater.
% \item
%   \pdfTeX in PDF mode. (I don't know a DVI driver that
%   support several color stacks.)
% \item
%   \xfile{pdftex.def} 2007/01/02 v0.04b.
% \end{itemize}
% Package \xpackage{pdfcol} checks the requirements and
% sets switch \cs{ifpdfcolAvailable} accordingly.
%
% \subsection{Interface}
%
% \begin{declcs}{ifpdfcolAvailable}
% \end{declcs}
% If the requirements of section \ref{sec:req} are met the
% switch \cs{ifpdfcolAvailable} behaves as \cs{iftrue}.
% Otherwise the other interface macros in this section will
% be disabled with a message. Also the first use of such a
% macro will print a message. The messages are print to
% the \xext{log} file only if \pdfTeX\ is not used in PDF mode.
%
% \begin{declcs}{pdfcolErrorNoStacks}
% \end{declcs}
% The first call of \cs{pdfcolErrorNoStacks} prints an error
% message, if color stacks are not available.
%
% \begin{declcs}{pdfcolInitStack} \M{name}
% \end{declcs}
% A new color stack is initialized by \cs{pdfcolInitStack}.
% The \meta{name} is used for indentifying the stack. It usually
% consists of letters and digits. (The name must survive a \cs{csname}.)
%
% The intension of the macro is the definition of an additional
% color stack. Thus the stack is not page bounded like the
% main color stack. Black (\texttt{0 g 0 G}) is used as initial
% color value. And colors are written with modifier \texttt{direct}
% that means without setting the current transfer matrix and changing
% the current point (see documentation of \pdfTeX\ for
% |\pdfliteral direct{...}|).
%
% \begin{declcs}{pdfcolIfStackExists} \M{name} \M{then} \M{else}
% \end{declcs}
% Macro \cs{pdfcolIfStackExists} checks whether color stack \meta{name}
% exists. In case of success argument \meta{then} is executed
% and \meta{else} otherwise.
%
% \begin{declcs}{pdfcolSwitchStack} \M{name}
% \end{declcs}
% Macro \cs{pdfcolSwitchStack} switches the color stack. The color macros
% of package \xpackage{color} (or \xpackage{xcolor}) now uses the
% new color stack with name \meta{name}.
%
% \begin{declcs}{pdfcolSetCurrentColor}
% \end{declcs}
% Macro \cs{pdfcolSetCurrentColor} replaces the topmost
% entry of the stack by the current color (\cs{current@color}).
%
% \begin{declcs}{pdfcolSetCurrent} \M{name}
% \end{declcs}
% Macro \cs{pdfcolSetCurrent} sets the color that is read in
% the top-most entry of color stack \meta{name}. If \meta{name}
% is empty, the default color stack is used.
%
% \StopEventually{
% }
%
% \section{Implementation}
%
%    \begin{macrocode}
%<*package>
%    \end{macrocode}
%
% \subsection{Reload check and package identification}
%    Reload check, especially if the package is not used with \LaTeX.
%    \begin{macrocode}
\begingroup\catcode61\catcode48\catcode32=10\relax%
  \catcode13=5 % ^^M
  \endlinechar=13 %
  \catcode35=6 % #
  \catcode39=12 % '
  \catcode44=12 % ,
  \catcode45=12 % -
  \catcode46=12 % .
  \catcode58=12 % :
  \catcode64=11 % @
  \catcode123=1 % {
  \catcode125=2 % }
  \expandafter\let\expandafter\x\csname ver@pdfcol.sty\endcsname
  \ifx\x\relax % plain-TeX, first loading
  \else
    \def\empty{}%
    \ifx\x\empty % LaTeX, first loading,
      % variable is initialized, but \ProvidesPackage not yet seen
    \else
      \expandafter\ifx\csname PackageInfo\endcsname\relax
        \def\x#1#2{%
          \immediate\write-1{Package #1 Info: #2.}%
        }%
      \else
        \def\x#1#2{\PackageInfo{#1}{#2, stopped}}%
      \fi
      \x{pdfcol}{The package is already loaded}%
      \aftergroup\endinput
    \fi
  \fi
\endgroup%
%    \end{macrocode}
%    Package identification:
%    \begin{macrocode}
\begingroup\catcode61\catcode48\catcode32=10\relax%
  \catcode13=5 % ^^M
  \endlinechar=13 %
  \catcode35=6 % #
  \catcode39=12 % '
  \catcode40=12 % (
  \catcode41=12 % )
  \catcode44=12 % ,
  \catcode45=12 % -
  \catcode46=12 % .
  \catcode47=12 % /
  \catcode58=12 % :
  \catcode64=11 % @
  \catcode91=12 % [
  \catcode93=12 % ]
  \catcode123=1 % {
  \catcode125=2 % }
  \expandafter\ifx\csname ProvidesPackage\endcsname\relax
    \def\x#1#2#3[#4]{\endgroup
      \immediate\write-1{Package: #3 #4}%
      \xdef#1{#4}%
    }%
  \else
    \def\x#1#2[#3]{\endgroup
      #2[{#3}]%
      \ifx#1\@undefined
        \xdef#1{#3}%
      \fi
      \ifx#1\relax
        \xdef#1{#3}%
      \fi
    }%
  \fi
\expandafter\x\csname ver@pdfcol.sty\endcsname
\ProvidesPackage{pdfcol}%
  [2018/11/01 v1.5 Handle new color stacks for pdfTeX (HO)]%
%    \end{macrocode}
%
% \subsection{Catcodes}
%
%    \begin{macrocode}
\begingroup\catcode61\catcode48\catcode32=10\relax%
  \catcode13=5 % ^^M
  \endlinechar=13 %
  \catcode123=1 % {
  \catcode125=2 % }
  \catcode64=11 % @
  \def\x{\endgroup
    \expandafter\edef\csname PDFCOL@AtEnd\endcsname{%
      \endlinechar=\the\endlinechar\relax
      \catcode13=\the\catcode13\relax
      \catcode32=\the\catcode32\relax
      \catcode35=\the\catcode35\relax
      \catcode61=\the\catcode61\relax
      \catcode64=\the\catcode64\relax
      \catcode123=\the\catcode123\relax
      \catcode125=\the\catcode125\relax
    }%
  }%
\x\catcode61\catcode48\catcode32=10\relax%
\catcode13=5 % ^^M
\endlinechar=13 %
\catcode35=6 % #
\catcode64=11 % @
\catcode123=1 % {
\catcode125=2 % }
\def\TMP@EnsureCode#1#2{%
  \edef\PDFCOL@AtEnd{%
    \PDFCOL@AtEnd
    \catcode#1=\the\catcode#1\relax
  }%
  \catcode#1=#2\relax
}
\TMP@EnsureCode{39}{12}% '
\TMP@EnsureCode{40}{12}% (
\TMP@EnsureCode{41}{12}% )
\TMP@EnsureCode{43}{12}% +
\TMP@EnsureCode{44}{12}% ,
\TMP@EnsureCode{46}{12}% .
\TMP@EnsureCode{47}{12}% /
\TMP@EnsureCode{91}{12}% [
\TMP@EnsureCode{93}{12}% ]
\TMP@EnsureCode{96}{12}% `
\edef\PDFCOL@AtEnd{\PDFCOL@AtEnd\noexpand\endinput}
%    \end{macrocode}
%
% \subsection{Check requirements}
%
%    \begin{macro}{\PDFCOL@RequirePackage}
%    \begin{macrocode}
\begingroup\expandafter\expandafter\expandafter\endgroup
\expandafter\ifx\csname RequirePackage\endcsname\relax
  \def\PDFCOL@RequirePackage#1[#2]{\input #1.sty\relax}%
\else
  \def\PDFCOL@RequirePackage#1[#2]{%
    \RequirePackage{#1}[{#2}]%
  }%
\fi
%    \end{macrocode}
%    \end{macro}
%
% LuaTeX Compatability
%    \begin{macrocode}
\ifx\pdfextension\@undefined\else
 \def\pdfcolorstackinit {\pdffeedback colorstackinit}
 \protected\def\pdfcolorstack     {\pdfextension colorstack}
\fi
%    \end{macrocode}
%
%    \begin{macrocode}
\PDFCOL@RequirePackage{ltxcmds}[2010/03/01]
%    \end{macrocode}
%
%    \begin{macro}{ifpdfcolAvailable}
%    \begin{macrocode}
\ltx@newif\ifpdfcolAvailable
\pdfcolAvailabletrue
%    \end{macrocode}
%    \end{macro}
%
% \subsubsection{Check package \xpackage{luacolor}}
%
%    \begin{macrocode}
\ltx@newif\ifPDFCOL@luacolor
\begingroup\expandafter\expandafter\expandafter\endgroup
\expandafter\ifx\csname ver@luacolor.sty\endcsname\relax
  \PDFCOL@luacolorfalse
\else
  \PDFCOL@luacolortrue
\fi
%    \end{macrocode}
%
% \subsubsection{Check PDF mode}
%
%    \begin{macrocode}
\PDFCOL@RequirePackage{infwarerr}[2007/09/09]
\PDFCOL@RequirePackage{ifpdf}[2007/09/09]
\ifcase\ifpdf\ifPDFCOL@luacolor 1\fi\else 1\fi0 %
  \def\PDFCOL@Message{%
    \@PackageWarningNoLine{pdfcol}%
  }%
\else
  \pdfcolAvailablefalse
  \def\PDFCOL@Message{%
    \@PackageInfoNoLine{pdfcol}%
  }%
  \PDFCOL@Message{%
    Interface disabled because of %
    \ifPDFCOL@luacolor
      package `luacolor'%
    \else
      missing PDF mode of pdfTeX%
    \fi
  }%
\fi
%    \end{macrocode}
%
% \subsubsection{Check version of \pdfTeX}
%
%    \begin{macrocode}
\ifpdfcolAvailable
  \begingroup\expandafter\expandafter\expandafter\endgroup
  \expandafter\ifx\csname pdfcolorstack\endcsname\relax
    \pdfcolAvailablefalse
    \PDFCOL@Message{%
      Interface disabled because of too old pdfTeX.\MessageBreak
      Required is version 1.40+ for \string\pdfcolorstack
    }%
  \fi
\fi
\ifpdfcolAvailable
  \begingroup\expandafter\expandafter\expandafter\endgroup
  \expandafter\ifx\csname pdfcolorstack\endcsname\relax
    \pdfcolAvailablefalse
    \PDFCOL@Message{%
      Interface disabled because of too old pdfTeX.\MessageBreak
      Required is version 1.40+ for \string\pdfcolorstackinit
    }%
  \fi
\fi
%    \end{macrocode}
%
% \subsubsection{Check \xfile{pdftex.def}}
%
%    \begin{macrocode}
\ifpdfcolAvailable
  \begingroup\expandafter\expandafter\expandafter\endgroup
  \expandafter\ifx\csname @pdfcolorstack\endcsname\relax
%    \end{macrocode}
%    Try to load package color if it is not yet loaded (\LaTeX\ case).
%    \begin{macrocode}
    \begingroup\expandafter\expandafter\expandafter\endgroup
    \expandafter\ifx\csname ver@color.sty\endcsname\relax
      \begingroup\expandafter\expandafter\expandafter\endgroup
      \expandafter\ifx\csname documentclass\endcsname\relax
      \else
        \RequirePackage[pdftex]{color}\relax
      \fi
    \fi
    \begingroup\expandafter\expandafter\expandafter\endgroup
    \expandafter\ifx\csname @pdfcolorstack\endcsname\relax
      \pdfcolAvailablefalse
      \PDFCOL@Message{%
        Interface disabled because `pdftex.def'\MessageBreak
        is not loaded or it is too old.\MessageBreak
        Required is version 0.04b or greater%
      }%
    \fi
  \fi
\fi
%    \end{macrocode}
%
%    \begin{macrocode}
\let\pdfcolAvailabletrue\relax
\let\pdfcolAvailablefalse\relax
%    \end{macrocode}
%
% \subsection{Enabled interface macros}
%
%    \begin{macrocode}
\ifpdfcolAvailable
%    \end{macrocode}
%
%    \begin{macro}{\pdfcolErrorNoStacks}
%    \begin{macrocode}
  \let\pdfcolErrorNoStacks\relax
%    \end{macrocode}
%    \end{macro}
%
%    \begin{macro}{\pdfcol@Value}
%    \begin{macrocode}
  \expandafter\ifx\csname pdfcol@Value\endcsname\relax
    \def\pdfcol@Value{0 g 0 G}%
  \fi
%    \end{macrocode}
%    \end{macro}
%
%    \begin{macro}{\pdfcol@LiteralModifier}
%    \begin{macrocode}
  \expandafter\ifx\csname pdfcol@LiteralModifier\endcsname\relax
    \def\pdfcol@LiteralModifier{direct}%
  \fi
%    \end{macrocode}
%    \end{macro}
%
%    \begin{macro}{\pdfcolInitStack}
%    \begin{macrocode}
  \def\pdfcolInitStack#1{%
    \expandafter\ifx\csname pdfcol@Stack@#1\endcsname\relax
      \global\expandafter\chardef\csname pdfcol@Stack@#1\endcsname=%
          \pdfcolorstackinit\pdfcol@LiteralModifier{\pdfcol@Value}%
          \relax
      \@PackageInfo{pdfcol}{%
        New color stack `#1' = \number\csname pdfcol@Stack@#1\endcsname
      }%
    \else
      \@PackageError{pdfcol}{%
        Stack `#1' is already defined%
      }\@ehc
    \fi
  }%
%    \end{macrocode}
%    \end{macro}
%
%    \begin{macro}{\pdfcolIfStackExists}
%    \begin{macrocode}
  \def\pdfcolIfStackExists#1{%
    \expandafter\ifx\csname pdfcol@Stack@#1\endcsname\relax
      \expandafter\@secondoftwo
    \else
      \expandafter\@firstoftwo
    \fi
  }%
%    \end{macrocode}
%    \end{macro}
%    \begin{macro}{\@firstoftwo}
%    \begin{macrocode}
  \expandafter\ifx\csname @firstoftwo\endcsname\relax
    \long\def\@firstoftwo#1#2{#1}%
  \fi
%    \end{macrocode}
%    \end{macro}
%    \begin{macro}{\@secondoftwo}
%    \begin{macrocode}
  \expandafter\ifx\csname @secondoftwo\endcsname\relax
    \long\def\@secondoftwo#1#2{#2}%
  \fi
%    \end{macrocode}
%    \end{macro}
%
%    \begin{macro}{\pdfcolSwitchStack}
%    \begin{macrocode}
  \def\pdfcolSwitchStack#1{%
    \pdfcolIfStackExists{#1}{%
      \expandafter\let\expandafter\@pdfcolorstack
                      \csname pdfcol@Stack@#1\endcsname
    }{%
      \pdfcol@ErrorNoStack{#1}%
    }%
  }%
%    \end{macrocode}
%    \end{macro}
%
%    \begin{macro}{\pdfcolSetCurrentColor}
%    \begin{macrocode}
  \def\pdfcolSetCurrentColor{%
    \pdfcolorstack\@pdfcolorstack set{\current@color}%
  }%
%    \end{macrocode}
%    \end{macro}
%
%    \begin{macro}{\pdfcolSetCurrent}
%    \begin{macrocode}
  \def\pdfcolSetCurrent#1{%
    \ifx\\#1\\%
      \pdfcolorstack\@pdfcolorstack current\relax
    \else
      \pdfcolIfStackExists{#1}{%
        \pdfcolorstack\csname pdfcol@Stack@#1\endcsname current\relax
      }{%
        \pdfcol@ErrorNoStack{#1}%
      }%
    \fi
  }%
%    \end{macrocode}
%    \end{macro}
%
%    \begin{macro}{\pdfcol@ErrorNoStack}
%    \begin{macrocode}
  \def\pdfcol@ErrorNoStack#1{%
    \@PackageError{pdfcol}{Stack `#1' does not exists}\@ehc
  }%
%    \end{macrocode}
%    \end{macro}
%
% \subsection{Disabled interface macros}
%
%    \begin{macrocode}
\else
%    \end{macrocode}
%
%    \begin{macro}{\pdfcolErrorNoStacks}
%    \begin{macrocode}
  \def\pdfcolErrorNoStacks{%
    \@PackageError{pdfcol}{%
      Color stacks are not available%
    }{%
      Update pdfTeX (1.40) and `pdftex.def' (0.04b) %
          if necessary.\MessageBreak
      Ensure that `pdftex.def' is loaded %
          (package `color' or `xcolor').\MessageBreak
      Further messages can be found in TeX's %
          protocol file `\jobname.log'.\MessageBreak
      \MessageBreak
      \@ehc
    }%
    \global\let\pdfcolErrorNoStacks\relax
  }%
%    \end{macrocode}
%    \end{macro}
%
%    \begin{macro}{\PDFCOL@Disabled}
%    \begin{macrocode}
  \def\PDFCOL@Disabled{%
    \PDFCOL@Message{%
      pdfTeX's color stacks are not available%
    }%
    \global\let\PDFCOL@Disabled\relax
  }%
%    \end{macrocode}
%    \end{macro}
%
%    \begin{macro}{\pdfcolInitStack}
%    \begin{macrocode}
  \def\pdfcolInitStack#1{%
    \PDFCOL@Disabled
  }%
%    \end{macrocode}
%    \end{macro}
%
%    \begin{macro}{\pdfcolIfStackExists}
%    \begin{macrocode}
  \long\def\pdfcolIfStackExists#1#2#3{#3}%
%    \end{macrocode}
%    \end{macro}
%
%    \begin{macro}{\pdfcolSwitchStack}
%    \begin{macrocode}
  \def\pdfcolSwitchStack#1{%
    \PDFCOL@Disabled
  }%
%    \end{macrocode}
%    \end{macro}
%
%    \begin{macro}{\pdfcolSetCurrentColor}
%    \begin{macrocode}
  \def\pdfcolSetCurrentColor{%
    \PDFCOL@Disabled
  }%
%    \end{macrocode}
%    \end{macro}
%
%    \begin{macro}{\pdfcolSetCurrent}
%    \begin{macrocode}
  \def\pdfcolSetCurrent#1{%
    \PDFCOL@Disabled
  }%
%    \end{macrocode}
%    \end{macro}
%    \begin{macrocode}
\fi
%    \end{macrocode}
%
%    \begin{macrocode}
\PDFCOL@AtEnd%
%</package>
%    \end{macrocode}
%% \section{Installation}
%
% \subsection{Download}
%
% \paragraph{Package.} This package is available on
% CTAN\footnote{\CTANpkg{pdfcol}}:
% \begin{description}
% \item[\CTAN{macros/latex/contrib/oberdiek/pdfcol.dtx}] The source file.
% \item[\CTAN{macros/latex/contrib/oberdiek/pdfcol.pdf}] Documentation.
% \end{description}
%
%
% \paragraph{Bundle.} All the packages of the bundle `oberdiek'
% are also available in a TDS compliant ZIP archive. There
% the packages are already unpacked and the documentation files
% are generated. The files and directories obey the TDS standard.
% \begin{description}
% \item[\CTANinstall{install/macros/latex/contrib/oberdiek.tds.zip}]
% \end{description}
% \emph{TDS} refers to the standard ``A Directory Structure
% for \TeX\ Files'' (\CTANpkg{tds}). Directories
% with \xfile{texmf} in their name are usually organized this way.
%
% \subsection{Bundle installation}
%
% \paragraph{Unpacking.} Unpack the \xfile{oberdiek.tds.zip} in the
% TDS tree (also known as \xfile{texmf} tree) of your choice.
% Example (linux):
% \begin{quote}
%   |unzip oberdiek.tds.zip -d ~/texmf|
% \end{quote}
%
% \subsection{Package installation}
%
% \paragraph{Unpacking.} The \xfile{.dtx} file is a self-extracting
% \docstrip\ archive. The files are extracted by running the
% \xfile{.dtx} through \plainTeX:
% \begin{quote}
%   \verb|tex pdfcol.dtx|
% \end{quote}
%
% \paragraph{TDS.} Now the different files must be moved into
% the different directories in your installation TDS tree
% (also known as \xfile{texmf} tree):
% \begin{quote}
% \def\t{^^A
% \begin{tabular}{@{}>{\ttfamily}l@{ $\rightarrow$ }>{\ttfamily}l@{}}
%   pdfcol.sty & tex/generic/oberdiek/pdfcol.sty\\
%   pdfcol.pdf & doc/latex/oberdiek/pdfcol.pdf\\
%   pdfcol.dtx & source/latex/oberdiek/pdfcol.dtx\\
% \end{tabular}^^A
% }^^A
% \sbox0{\t}^^A
% \ifdim\wd0>\linewidth
%   \begingroup
%     \advance\linewidth by\leftmargin
%     \advance\linewidth by\rightmargin
%   \edef\x{\endgroup
%     \def\noexpand\lw{\the\linewidth}^^A
%   }\x
%   \def\lwbox{^^A
%     \leavevmode
%     \hbox to \linewidth{^^A
%       \kern-\leftmargin\relax
%       \hss
%       \usebox0
%       \hss
%       \kern-\rightmargin\relax
%     }^^A
%   }^^A
%   \ifdim\wd0>\lw
%     \sbox0{\small\t}^^A
%     \ifdim\wd0>\linewidth
%       \ifdim\wd0>\lw
%         \sbox0{\footnotesize\t}^^A
%         \ifdim\wd0>\linewidth
%           \ifdim\wd0>\lw
%             \sbox0{\scriptsize\t}^^A
%             \ifdim\wd0>\linewidth
%               \ifdim\wd0>\lw
%                 \sbox0{\tiny\t}^^A
%                 \ifdim\wd0>\linewidth
%                   \lwbox
%                 \else
%                   \usebox0
%                 \fi
%               \else
%                 \lwbox
%               \fi
%             \else
%               \usebox0
%             \fi
%           \else
%             \lwbox
%           \fi
%         \else
%           \usebox0
%         \fi
%       \else
%         \lwbox
%       \fi
%     \else
%       \usebox0
%     \fi
%   \else
%     \lwbox
%   \fi
% \else
%   \usebox0
% \fi
% \end{quote}
% If you have a \xfile{docstrip.cfg} that configures and enables \docstrip's
% TDS installing feature, then some files can already be in the right
% place, see the documentation of \docstrip.
%
% \subsection{Refresh file name databases}
%
% If your \TeX~distribution
% (\TeX\,Live, \mikTeX, \dots) relies on file name databases, you must refresh
% these. For example, \TeX\,Live\ users run \verb|texhash| or
% \verb|mktexlsr|.
%
% \subsection{Some details for the interested}
%
% \paragraph{Unpacking with \LaTeX.}
% The \xfile{.dtx} chooses its action depending on the format:
% \begin{description}
% \item[\plainTeX:] Run \docstrip\ and extract the files.
% \item[\LaTeX:] Generate the documentation.
% \end{description}
% If you insist on using \LaTeX\ for \docstrip\ (really,
% \docstrip\ does not need \LaTeX), then inform the autodetect routine
% about your intention:
% \begin{quote}
%   \verb|latex \let\install=y\input{pdfcol.dtx}|
% \end{quote}
% Do not forget to quote the argument according to the demands
% of your shell.
%
% \paragraph{Generating the documentation.}
% You can use both the \xfile{.dtx} or the \xfile{.drv} to generate
% the documentation. The process can be configured by the
% configuration file \xfile{ltxdoc.cfg}. For instance, put this
% line into this file, if you want to have A4 as paper format:
% \begin{quote}
%   \verb|\PassOptionsToClass{a4paper}{article}|
% \end{quote}
% An example follows how to generate the
% documentation with pdf\LaTeX:
% \begin{quote}
%\begin{verbatim}
%pdflatex pdfcol.dtx
%makeindex -s gind.ist pdfcol.idx
%pdflatex pdfcol.dtx
%makeindex -s gind.ist pdfcol.idx
%pdflatex pdfcol.dtx
%\end{verbatim}
% \end{quote}
%
% \begin{History}
%   \begin{Version}{2007/09/09 v1.0}
%   \item
%     First version.
%   \end{Version}
%   \begin{Version}{2007/12/09 v1.1}
%   \item
%     \cs{pdfcolSetCurrentColor} added.
%   \end{Version}
%   \begin{Version}{2007/12/12 v1.2}
%   \item
%     Detection for package \xpackage{luacolor} added.
%   \end{Version}
%   \begin{Version}{2016/05/16 v1.3}
%   \item
%     Documentation updates.
%   \end{Version}
%   \begin{Version}{2016/05/17 v1.4}
%   \item
%     Use luatex85 package for new luatex compatibility
%   \end{Version}
%   \begin{Version}{2018/11/01 v1.5}
%   \item
%     Remove luatex85 dependency
%   \end{Version}
% \end{History}
%
% \PrintIndex
%
% \Finale
\endinput
|
% \end{quote}
% Do not forget to quote the argument according to the demands
% of your shell.
%
% \paragraph{Generating the documentation.}
% You can use both the \xfile{.dtx} or the \xfile{.drv} to generate
% the documentation. The process can be configured by the
% configuration file \xfile{ltxdoc.cfg}. For instance, put this
% line into this file, if you want to have A4 as paper format:
% \begin{quote}
%   \verb|\PassOptionsToClass{a4paper}{article}|
% \end{quote}
% An example follows how to generate the
% documentation with pdf\LaTeX:
% \begin{quote}
%\begin{verbatim}
%pdflatex pdfcol.dtx
%makeindex -s gind.ist pdfcol.idx
%pdflatex pdfcol.dtx
%makeindex -s gind.ist pdfcol.idx
%pdflatex pdfcol.dtx
%\end{verbatim}
% \end{quote}
%
% \begin{History}
%   \begin{Version}{2007/09/09 v1.0}
%   \item
%     First version.
%   \end{Version}
%   \begin{Version}{2007/12/09 v1.1}
%   \item
%     \cs{pdfcolSetCurrentColor} added.
%   \end{Version}
%   \begin{Version}{2007/12/12 v1.2}
%   \item
%     Detection for package \xpackage{luacolor} added.
%   \end{Version}
%   \begin{Version}{2016/05/16 v1.3}
%   \item
%     Documentation updates.
%   \end{Version}
%   \begin{Version}{2016/05/17 v1.4}
%   \item
%     Use luatex85 package for new luatex compatibility
%   \end{Version}
%   \begin{Version}{2018/11/01 v1.5}
%   \item
%     Remove luatex85 dependency
%   \end{Version}
% \end{History}
%
% \PrintIndex
%
% \Finale
\endinput
|
% \end{quote}
% Do not forget to quote the argument according to the demands
% of your shell.
%
% \paragraph{Generating the documentation.}
% You can use both the \xfile{.dtx} or the \xfile{.drv} to generate
% the documentation. The process can be configured by the
% configuration file \xfile{ltxdoc.cfg}. For instance, put this
% line into this file, if you want to have A4 as paper format:
% \begin{quote}
%   \verb|\PassOptionsToClass{a4paper}{article}|
% \end{quote}
% An example follows how to generate the
% documentation with pdf\LaTeX:
% \begin{quote}
%\begin{verbatim}
%pdflatex pdfcol.dtx
%makeindex -s gind.ist pdfcol.idx
%pdflatex pdfcol.dtx
%makeindex -s gind.ist pdfcol.idx
%pdflatex pdfcol.dtx
%\end{verbatim}
% \end{quote}
%
% \begin{History}
%   \begin{Version}{2007/09/09 v1.0}
%   \item
%     First version.
%   \end{Version}
%   \begin{Version}{2007/12/09 v1.1}
%   \item
%     \cs{pdfcolSetCurrentColor} added.
%   \end{Version}
%   \begin{Version}{2007/12/12 v1.2}
%   \item
%     Detection for package \xpackage{luacolor} added.
%   \end{Version}
%   \begin{Version}{2016/05/16 v1.3}
%   \item
%     Documentation updates.
%   \end{Version}
%   \begin{Version}{2016/05/17 v1.4}
%   \item
%     Use luatex85 package for new luatex compatibility
%   \end{Version}
%   \begin{Version}{2018/11/01 v1.5}
%   \item
%     Remove luatex85 dependency
%   \end{Version}
% \end{History}
%
% \PrintIndex
%
% \Finale
\endinput
|
% \end{quote}
% Do not forget to quote the argument according to the demands
% of your shell.
%
% \paragraph{Generating the documentation.}
% You can use both the \xfile{.dtx} or the \xfile{.drv} to generate
% the documentation. The process can be configured by the
% configuration file \xfile{ltxdoc.cfg}. For instance, put this
% line into this file, if you want to have A4 as paper format:
% \begin{quote}
%   \verb|\PassOptionsToClass{a4paper}{article}|
% \end{quote}
% An example follows how to generate the
% documentation with pdf\LaTeX:
% \begin{quote}
%\begin{verbatim}
%pdflatex pdfcol.dtx
%makeindex -s gind.ist pdfcol.idx
%pdflatex pdfcol.dtx
%makeindex -s gind.ist pdfcol.idx
%pdflatex pdfcol.dtx
%\end{verbatim}
% \end{quote}
%
% \begin{History}
%   \begin{Version}{2007/09/09 v1.0}
%   \item
%     First version.
%   \end{Version}
%   \begin{Version}{2007/12/09 v1.1}
%   \item
%     \cs{pdfcolSetCurrentColor} added.
%   \end{Version}
%   \begin{Version}{2007/12/12 v1.2}
%   \item
%     Detection for package \xpackage{luacolor} added.
%   \end{Version}
%   \begin{Version}{2016/05/16 v1.3}
%   \item
%     Documentation updates.
%   \end{Version}
%   \begin{Version}{2016/05/17 v1.4}
%   \item
%     Use luatex85 package for new luatex compatibility
%   \end{Version}
%   \begin{Version}{2018/11/01 v1.5}
%   \item
%     Remove luatex85 dependency
%   \end{Version}
% \end{History}
%
% \PrintIndex
%
% \Finale
\endinput
