% \iffalse meta-comment
%
% File: atenddvi.dtx
% Version: 2016/05/16 v1.2
% Info: At end DVI hook
%
% Copyright (C) 2007 by
%    Heiko Oberdiek <heiko.oberdiek at googlemail.com>
%    2016
%    https://github.com/ho-tex/oberdiek/issues
%
% This work may be distributed and/or modified under the
% conditions of the LaTeX Project Public License, either
% version 1.3c of this license or (at your option) any later
% version. This version of this license is in
%    http://www.latex-project.org/lppl/lppl-1-3c.txt
% and the latest version of this license is in
%    http://www.latex-project.org/lppl.txt
% and version 1.3 or later is part of all distributions of
% LaTeX version 2005/12/01 or later.
%
% This work has the LPPL maintenance status "maintained".
%
% This Current Maintainer of this work is Heiko Oberdiek.
%
% This work consists of the main source file atenddvi.dtx
% and the derived files
%    atenddvi.sty, atenddvi.pdf, atenddvi.ins, atenddvi.drv.
%
% Distribution:
%    CTAN:macros/latex/contrib/oberdiek/atenddvi.dtx
%    CTAN:macros/latex/contrib/oberdiek/atenddvi.pdf
%
% Unpacking:
%    (a) If atenddvi.ins is present:
%           tex atenddvi.ins
%    (b) Without atenddvi.ins:
%           tex atenddvi.dtx
%    (c) If you insist on using LaTeX
%           latex \let\install=y% \iffalse meta-comment
%
% File: atenddvi.dtx
% Version: 2016/05/16 v1.2
% Info: At end DVI hook
%
% Copyright (C)
%    2007 Heiko Oberdiek
%    2016-2019 Oberdiek Package Support Group
%    https://github.com/ho-tex/oberdiek/issues
%
% This work may be distributed and/or modified under the
% conditions of the LaTeX Project Public License, either
% version 1.3c of this license or (at your option) any later
% version. This version of this license is in
%    https://www.latex-project.org/lppl/lppl-1-3c.txt
% and the latest version of this license is in
%    https://www.latex-project.org/lppl.txt
% and version 1.3 or later is part of all distributions of
% LaTeX version 2005/12/01 or later.
%
% This work has the LPPL maintenance status "maintained".
%
% The Current Maintainers of this work are
% Heiko Oberdiek and the Oberdiek Package Support Group
% https://github.com/ho-tex/oberdiek/issues
%
% This work consists of the main source file atenddvi.dtx
% and the derived files
%    atenddvi.sty, atenddvi.pdf, atenddvi.ins, atenddvi.drv.
%
% Distribution:
%    CTAN:macros/latex/contrib/oberdiek/atenddvi.dtx
%    CTAN:macros/latex/contrib/oberdiek/atenddvi.pdf
%
% Unpacking:
%    (a) If atenddvi.ins is present:
%           tex atenddvi.ins
%    (b) Without atenddvi.ins:
%           tex atenddvi.dtx
%    (c) If you insist on using LaTeX
%           latex \let\install=y% \iffalse meta-comment
%
% File: atenddvi.dtx
% Version: 2016/05/16 v1.2
% Info: At end DVI hook
%
% Copyright (C)
%    2007 Heiko Oberdiek
%    2016-2019 Oberdiek Package Support Group
%    https://github.com/ho-tex/oberdiek/issues
%
% This work may be distributed and/or modified under the
% conditions of the LaTeX Project Public License, either
% version 1.3c of this license or (at your option) any later
% version. This version of this license is in
%    https://www.latex-project.org/lppl/lppl-1-3c.txt
% and the latest version of this license is in
%    https://www.latex-project.org/lppl.txt
% and version 1.3 or later is part of all distributions of
% LaTeX version 2005/12/01 or later.
%
% This work has the LPPL maintenance status "maintained".
%
% The Current Maintainers of this work are
% Heiko Oberdiek and the Oberdiek Package Support Group
% https://github.com/ho-tex/oberdiek/issues
%
% This work consists of the main source file atenddvi.dtx
% and the derived files
%    atenddvi.sty, atenddvi.pdf, atenddvi.ins, atenddvi.drv.
%
% Distribution:
%    CTAN:macros/latex/contrib/oberdiek/atenddvi.dtx
%    CTAN:macros/latex/contrib/oberdiek/atenddvi.pdf
%
% Unpacking:
%    (a) If atenddvi.ins is present:
%           tex atenddvi.ins
%    (b) Without atenddvi.ins:
%           tex atenddvi.dtx
%    (c) If you insist on using LaTeX
%           latex \let\install=y% \iffalse meta-comment
%
% File: atenddvi.dtx
% Version: 2016/05/16 v1.2
% Info: At end DVI hook
%
% Copyright (C)
%    2007 Heiko Oberdiek
%    2016-2019 Oberdiek Package Support Group
%    https://github.com/ho-tex/oberdiek/issues
%
% This work may be distributed and/or modified under the
% conditions of the LaTeX Project Public License, either
% version 1.3c of this license or (at your option) any later
% version. This version of this license is in
%    https://www.latex-project.org/lppl/lppl-1-3c.txt
% and the latest version of this license is in
%    https://www.latex-project.org/lppl.txt
% and version 1.3 or later is part of all distributions of
% LaTeX version 2005/12/01 or later.
%
% This work has the LPPL maintenance status "maintained".
%
% The Current Maintainers of this work are
% Heiko Oberdiek and the Oberdiek Package Support Group
% https://github.com/ho-tex/oberdiek/issues
%
% This work consists of the main source file atenddvi.dtx
% and the derived files
%    atenddvi.sty, atenddvi.pdf, atenddvi.ins, atenddvi.drv.
%
% Distribution:
%    CTAN:macros/latex/contrib/oberdiek/atenddvi.dtx
%    CTAN:macros/latex/contrib/oberdiek/atenddvi.pdf
%
% Unpacking:
%    (a) If atenddvi.ins is present:
%           tex atenddvi.ins
%    (b) Without atenddvi.ins:
%           tex atenddvi.dtx
%    (c) If you insist on using LaTeX
%           latex \let\install=y\input{atenddvi.dtx}
%        (quote the arguments according to the demands of your shell)
%
% Documentation:
%    (a) If atenddvi.drv is present:
%           latex atenddvi.drv
%    (b) Without atenddvi.drv:
%           latex atenddvi.dtx; ...
%    The class ltxdoc loads the configuration file ltxdoc.cfg
%    if available. Here you can specify further options, e.g.
%    use A4 as paper format:
%       \PassOptionsToClass{a4paper}{article}
%
%    Programm calls to get the documentation (example):
%       pdflatex atenddvi.dtx
%       makeindex -s gind.ist atenddvi.idx
%       pdflatex atenddvi.dtx
%       makeindex -s gind.ist atenddvi.idx
%       pdflatex atenddvi.dtx
%
% Installation:
%    TDS:tex/latex/oberdiek/atenddvi.sty
%    TDS:doc/latex/oberdiek/atenddvi.pdf
%    TDS:source/latex/oberdiek/atenddvi.dtx
%
%<*ignore>
\begingroup
  \catcode123=1 %
  \catcode125=2 %
  \def\x{LaTeX2e}%
\expandafter\endgroup
\ifcase 0\ifx\install y1\fi\expandafter
         \ifx\csname processbatchFile\endcsname\relax\else1\fi
         \ifx\fmtname\x\else 1\fi\relax
\else\csname fi\endcsname
%</ignore>
%<*install>
\input docstrip.tex
\Msg{************************************************************************}
\Msg{* Installation}
\Msg{* Package: atenddvi 2016/05/16 v1.2 At end DVI hook (HO)}
\Msg{************************************************************************}

\keepsilent
\askforoverwritefalse

\let\MetaPrefix\relax
\preamble

This is a generated file.

Project: atenddvi
Version: 2016/05/16 v1.2

Copyright (C)
   2007 Heiko Oberdiek
   2016-2019 Oberdiek Package Support Group

This work may be distributed and/or modified under the
conditions of the LaTeX Project Public License, either
version 1.3c of this license or (at your option) any later
version. This version of this license is in
   https://www.latex-project.org/lppl/lppl-1-3c.txt
and the latest version of this license is in
   https://www.latex-project.org/lppl.txt
and version 1.3 or later is part of all distributions of
LaTeX version 2005/12/01 or later.

This work has the LPPL maintenance status "maintained".

The Current Maintainers of this work are
Heiko Oberdiek and the Oberdiek Package Support Group
https://github.com/ho-tex/oberdiek/issues


This work consists of the main source file atenddvi.dtx
and the derived files
   atenddvi.sty, atenddvi.pdf, atenddvi.ins, atenddvi.drv.

\endpreamble
\let\MetaPrefix\DoubleperCent

\generate{%
  \file{atenddvi.ins}{\from{atenddvi.dtx}{install}}%
  \file{atenddvi.drv}{\from{atenddvi.dtx}{driver}}%
  \usedir{tex/latex/oberdiek}%
  \file{atenddvi.sty}{\from{atenddvi.dtx}{package}}%
  \nopreamble
  \nopostamble
%  \usedir{source/latex/oberdiek/catalogue}%
%  \file{atenddvi.xml}{\from{atenddvi.dtx}{catalogue}}%
}

\catcode32=13\relax% active space
\let =\space%
\Msg{************************************************************************}
\Msg{*}
\Msg{* To finish the installation you have to move the following}
\Msg{* file into a directory searched by TeX:}
\Msg{*}
\Msg{*     atenddvi.sty}
\Msg{*}
\Msg{* To produce the documentation run the file `atenddvi.drv'}
\Msg{* through LaTeX.}
\Msg{*}
\Msg{* Happy TeXing!}
\Msg{*}
\Msg{************************************************************************}

\endbatchfile
%</install>
%<*ignore>
\fi
%</ignore>
%<*driver>
\NeedsTeXFormat{LaTeX2e}
\ProvidesFile{atenddvi.drv}%
  [2016/05/16 v1.2 At end DVI hook (HO)]%
\documentclass{ltxdoc}
\usepackage{holtxdoc}[2011/11/22]
\begin{document}
  \DocInput{atenddvi.dtx}%
\end{document}
%</driver>
% \fi
%
%
% \CharacterTable
%  {Upper-case    \A\B\C\D\E\F\G\H\I\J\K\L\M\N\O\P\Q\R\S\T\U\V\W\X\Y\Z
%   Lower-case    \a\b\c\d\e\f\g\h\i\j\k\l\m\n\o\p\q\r\s\t\u\v\w\x\y\z
%   Digits        \0\1\2\3\4\5\6\7\8\9
%   Exclamation   \!     Double quote  \"     Hash (number) \#
%   Dollar        \$     Percent       \%     Ampersand     \&
%   Acute accent  \'     Left paren    \(     Right paren   \)
%   Asterisk      \*     Plus          \+     Comma         \,
%   Minus         \-     Point         \.     Solidus       \/
%   Colon         \:     Semicolon     \;     Less than     \<
%   Equals        \=     Greater than  \>     Question mark \?
%   Commercial at \@     Left bracket  \[     Backslash     \\
%   Right bracket \]     Circumflex    \^     Underscore    \_
%   Grave accent  \`     Left brace    \{     Vertical bar  \|
%   Right brace   \}     Tilde         \~}
%
% \GetFileInfo{atenddvi.drv}
%
% \title{The \xpackage{atenddvi} package}
% \date{2016/05/16 v1.2}
% \author{Heiko Oberdiek\thanks
% {Please report any issues at \url{https://github.com/ho-tex/oberdiek/issues}}}
%
% \maketitle
%
% \begin{abstract}
% \LaTeX\ offers \cs{AtBeginDvi}. This package \xpackage{atenddvi}
% provides the counterpart \cs{AtEndDvi}. The execution of its
% argument is delayed to the end of the document at the end of the
% last page. Thus \cs{special} and \cs{write} remain effective, because
% they are put into the last page. This is the main difference
% to \cs{AtEndDocument}.
% \end{abstract}
%
% \tableofcontents
%
% \section{Documentation}
%
% \begin{declcs}{AtEndDvi} \M{code}
% \end{declcs}
% Macro \cs{AtEndDvi} provides a hook mechanism to put \meta{code}
% at the end of the last output page. It is the logical counterpart
% to \cs{AtBeginDvi}. Despite the name the output type DVI, PDF or whatever
% does not matter.
%
% Unlike \cs{AtBeginDvi} the \meta{code} is not put in a box and
% therefore executed immediately. The hook for \cs{AtEndDvi} is based on
% a macro similar to \cs{AtBeginDocument} or \cs{AtEndDocument}. The
% execution of \meta{code} is delayed until the hook is executed on
% the last page.
%
% Commands such as \cs{special} or \cs{write} (not the \cs{immediate}
% variant) must go as nodes into the contents of a page to have the
% desired effect.
% When the hook for \cs{AtEndDocument} is executed, the last intended
% page may already be shipped out. Therefore \cs{special} or \cs{write}
% cannot be used in a reliable way without generating new page.
%
% This gap is closed by \cs{AtEndDvi} of this package \xpackage{atenddvi}.
% If the document is compiled the first time, the package remembers
% the last page in a reference. In the sceond run, it puts the hook
% on the page that has been detected in the previous run as last page.
% The package detectes if the number of pages has changed, and then
% generates a warning to rerun \LaTeX.
%
% \StopEventually{
% }
%
% \section{Implementation}
%
%    \begin{macrocode}
%<*package>
\NeedsTeXFormat{LaTeX2e}
\ProvidesPackage{atenddvi}%
  [2016/05/16 v1.2 At end DVI hook (HO)]%
%    \end{macrocode}
%
%    Load the required packages
%    \begin{macrocode}
\RequirePackage{zref-abspage,zref-lastpage}[2007/03/19]
\RequirePackage{atbegshi}
%    \end{macrocode}
%
%    \begin{macro}{\AtEndDvi@Hook}
%    Macro \cs{AtEndDvi@Hook} is the data storage macro
%    for the code that is executed later at end of the last page.
%    \begin{macrocode}
\let\AtEndDvi@Hook\@empty
%    \end{macrocode}
%    \end{macro}
%    \begin{macro}{\AtEndDvi}
%    Macro \cs{AtEndDvi} is called in the same way as
%    \cs{AtBeginDocument}. The argument is added to the hook macro.
%    \begin{macrocode}
\newcommand*{\AtEndDvi}{%
  \g@addto@macro\AtEndDvi@Hook
}
%    \end{macrocode}
%    \end{macro}
%
%    \begin{macro}{\AtEndDvi@AtBeginShipout}
%    \begin{macrocode}
\def\AtEndDvi@AtBeginShipout{%
  \begingroup
%    \end{macrocode}
%    The reference `LastPage' is marked used. If the reference
%    is not yet defined, then the user gets the warning because of
%    the undefined reference and the rerun warning at the end of
%    the compile run. However, we do not need a warning each page,
%    the first page is enough.
%    \begin{macrocode}
    \ifnum\value{abspage}=1 %
      \zref@refused{LastPage}%
    \fi
%    \end{macrocode}
%    The current absolute page number is compared with the
%    absolute page number of the reference `LastPage'.
%    \begin{macrocode}
    \ifnum\zref@extractdefault{LastPage}{abspage}{0}=\value{abspage}%
%    \end{macrocode}
%    \begin{macro}{\AtEndDvi@LastPage}
%    We found the right page and remember it in a macro.
%    \begin{macrocode}
      \xdef\AtEndDvi@LastPage{\number\value{abspage}}%
%    \end{macrocode}
%    \end{macro}
%    The hook of \cs{AtEndDvi} is now put on the last page
%    after the contents of the page.
%    \begin{macrocode}
      \global\setbox\AtBeginShipoutBox=\vbox{%
        \hbox{%
          \box\AtBeginShipoutBox
          \setbox\AtBeginShipoutBox=\hbox{%
            \begingroup
              \AtEndDvi@Hook
            \endgroup
          }%
          \wd\AtBeginShipoutBox=\z@
          \ht\AtBeginShipoutBox=\z@
          \dp\AtBeginShipoutBox=\z@
          \box\AtBeginShipoutBox
        }%
      }%
%    \end{macrocode}
%    We do not need the every page hook.
%    \begin{macrocode}
      \global\let\AtEndDvi@AtBeginShipout\@empty
%    \end{macrocode}
%    The hook is consumed, \cs{AtEndDvi} does not have an effect.
%    \begin{macrocode}
      \global\let\AtEndDvi\@gobble
%    \end{macrocode}
%    Make a protocol entry, which page is used by this package
%    as last page.
%    \begin{macrocode}
      \let\on@line\@empty
      \PackageInfo{atenddvi}{Last page = \AtEndDvi@LastPage}%
    \fi
  \endgroup
}
%    \end{macrocode}
%    \end{macro}
%
%    \begin{macro}{\AtEndDvi@AtBeginDocument}
%    In order to get as late as possible in the chain of the
%    every shipout hook, the call of \cs{AtBeginShipout} is delayed.
%    \begin{macrocode}
\def\AtEndDvi@AtBeginDocument{%
  \AtBeginShipout{\AtEndDvi@AtBeginShipout}%
%    \end{macrocode}
%    \begin{macro}{\AtEndDvi@Check}
%    After \cs{AtEndDocument} \LaTeX\ reads its \xfile{.aux} files
%    again. Code in \cs{AtEndDocument} could generate additional
%    pages. This is unlikely by code in the \xfile{.aux} file,
%    thus we use the \xfile{.aux} file to run macro
%    \cs{AtEndDvi@Check} for checking the last page.
%
%    During the first reading of the \xfile{.aux} file,
%    \cs{AtEndDvi@Check} is disabled, its real meaning
%    is assigned afterwards.
%    \begin{macrocode}
  \if@filesw
    \immediate\write\@mainaux{%
      \string\providecommand\string\AtEndDvi@Check{}%
    }%
    \immediate\write\@mainaux{%
      \string\AtEndDvi@Check
    }%
  \fi
  \let\AtEndDvi@Check\AtEndDvi@CheckImpl
}
%    \end{macrocode}
%    \end{macro}
%    \begin{macrocode}
\AtBeginDocument{\AtEndDvi@AtBeginDocument}
%    \end{macrocode}
%    \end{macro}
%
%    \begin{macro}{\AtEndDvi@CheckImpl}
%    First check is whether a last page was found at all.
%    Secondly the found last page is compared with the real last page.
%    \begin{macrocode}
\def\AtEndDvi@CheckImpl{%
  \@ifundefined{AtEndDvi@LastPage}{%
    \PackageWarningNoLine{atenddvi}{%
      Rerun LaTeX, last page not yet found%
    }%
  }{%
    \ifnum\AtEndDvi@LastPage=\value{abspage}%
    \else
      \PackageWarningNoLine{atenddvi}{%
        Rerun LaTeX, last page has changed%
      }%
    \fi
  }%
}
%    \end{macrocode}
%    \end{macro}
%
%    \begin{macrocode}
%</package>
%    \end{macrocode}
%
% \section{Installation}
%
% \subsection{Download}
%
% \paragraph{Package.} This package is available on
% CTAN\footnote{\CTANpkg{atenddvi}}:
% \begin{description}
% \item[\CTAN{macros/latex/contrib/oberdiek/atenddvi.dtx}] The source file.
% \item[\CTAN{macros/latex/contrib/oberdiek/atenddvi.pdf}] Documentation.
% \end{description}
%
%
% \paragraph{Bundle.} All the packages of the bundle `oberdiek'
% are also available in a TDS compliant ZIP archive. There
% the packages are already unpacked and the documentation files
% are generated. The files and directories obey the TDS standard.
% \begin{description}
% \item[\CTANinstall{install/macros/latex/contrib/oberdiek.tds.zip}]
% \end{description}
% \emph{TDS} refers to the standard ``A Directory Structure
% for \TeX\ Files'' (\CTAN{tds/tds.pdf}). Directories
% with \xfile{texmf} in their name are usually organized this way.
%
% \subsection{Bundle installation}
%
% \paragraph{Unpacking.} Unpack the \xfile{oberdiek.tds.zip} in the
% TDS tree (also known as \xfile{texmf} tree) of your choice.
% Example (linux):
% \begin{quote}
%   |unzip oberdiek.tds.zip -d ~/texmf|
% \end{quote}
%
% \subsection{Package installation}
%
% \paragraph{Unpacking.} The \xfile{.dtx} file is a self-extracting
% \docstrip\ archive. The files are extracted by running the
% \xfile{.dtx} through \plainTeX:
% \begin{quote}
%   \verb|tex atenddvi.dtx|
% \end{quote}
%
% \paragraph{TDS.} Now the different files must be moved into
% the different directories in your installation TDS tree
% (also known as \xfile{texmf} tree):
% \begin{quote}
% \def\t{^^A
% \begin{tabular}{@{}>{\ttfamily}l@{ $\rightarrow$ }>{\ttfamily}l@{}}
%   atenddvi.sty & tex/latex/oberdiek/atenddvi.sty\\
%   atenddvi.pdf & doc/latex/oberdiek/atenddvi.pdf\\
%   atenddvi.dtx & source/latex/oberdiek/atenddvi.dtx\\
% \end{tabular}^^A
% }^^A
% \sbox0{\t}^^A
% \ifdim\wd0>\linewidth
%   \begingroup
%     \advance\linewidth by\leftmargin
%     \advance\linewidth by\rightmargin
%   \edef\x{\endgroup
%     \def\noexpand\lw{\the\linewidth}^^A
%   }\x
%   \def\lwbox{^^A
%     \leavevmode
%     \hbox to \linewidth{^^A
%       \kern-\leftmargin\relax
%       \hss
%       \usebox0
%       \hss
%       \kern-\rightmargin\relax
%     }^^A
%   }^^A
%   \ifdim\wd0>\lw
%     \sbox0{\small\t}^^A
%     \ifdim\wd0>\linewidth
%       \ifdim\wd0>\lw
%         \sbox0{\footnotesize\t}^^A
%         \ifdim\wd0>\linewidth
%           \ifdim\wd0>\lw
%             \sbox0{\scriptsize\t}^^A
%             \ifdim\wd0>\linewidth
%               \ifdim\wd0>\lw
%                 \sbox0{\tiny\t}^^A
%                 \ifdim\wd0>\linewidth
%                   \lwbox
%                 \else
%                   \usebox0
%                 \fi
%               \else
%                 \lwbox
%               \fi
%             \else
%               \usebox0
%             \fi
%           \else
%             \lwbox
%           \fi
%         \else
%           \usebox0
%         \fi
%       \else
%         \lwbox
%       \fi
%     \else
%       \usebox0
%     \fi
%   \else
%     \lwbox
%   \fi
% \else
%   \usebox0
% \fi
% \end{quote}
% If you have a \xfile{docstrip.cfg} that configures and enables \docstrip's
% TDS installing feature, then some files can already be in the right
% place, see the documentation of \docstrip.
%
% \subsection{Refresh file name databases}
%
% If your \TeX~distribution
% (\TeX\,Live, \mikTeX, \dots) relies on file name databases, you must refresh
% these. For example, \TeX\,Live\ users run \verb|texhash| or
% \verb|mktexlsr|.
%
% \subsection{Some details for the interested}
%
% \paragraph{Unpacking with \LaTeX.}
% The \xfile{.dtx} chooses its action depending on the format:
% \begin{description}
% \item[\plainTeX:] Run \docstrip\ and extract the files.
% \item[\LaTeX:] Generate the documentation.
% \end{description}
% If you insist on using \LaTeX\ for \docstrip\ (really,
% \docstrip\ does not need \LaTeX), then inform the autodetect routine
% about your intention:
% \begin{quote}
%   \verb|latex \let\install=y\input{atenddvi.dtx}|
% \end{quote}
% Do not forget to quote the argument according to the demands
% of your shell.
%
% \paragraph{Generating the documentation.}
% You can use both the \xfile{.dtx} or the \xfile{.drv} to generate
% the documentation. The process can be configured by the
% configuration file \xfile{ltxdoc.cfg}. For instance, put this
% line into this file, if you want to have A4 as paper format:
% \begin{quote}
%   \verb|\PassOptionsToClass{a4paper}{article}|
% \end{quote}
% An example follows how to generate the
% documentation with pdf\LaTeX:
% \begin{quote}
%\begin{verbatim}
%pdflatex atenddvi.dtx
%makeindex -s gind.ist atenddvi.idx
%pdflatex atenddvi.dtx
%makeindex -s gind.ist atenddvi.idx
%pdflatex atenddvi.dtx
%\end{verbatim}
% \end{quote}
%
% \begin{History}
%   \begin{Version}{2007/03/20 v1.0}
%   \item
%     First version.
%   \end{Version}
%   \begin{Version}{2007/04/17 v1.1}
%   \item
%     Package \xpackage{atbegshi} replaces package \xpackage{everyshi}.
%   \end{Version}
%   \begin{Version}{2016/05/16 v1.2}
%   \item
%     Documentation updates.
%   \end{Version}
% \end{History}
%
% \PrintIndex
%
% \Finale
\endinput

%        (quote the arguments according to the demands of your shell)
%
% Documentation:
%    (a) If atenddvi.drv is present:
%           latex atenddvi.drv
%    (b) Without atenddvi.drv:
%           latex atenddvi.dtx; ...
%    The class ltxdoc loads the configuration file ltxdoc.cfg
%    if available. Here you can specify further options, e.g.
%    use A4 as paper format:
%       \PassOptionsToClass{a4paper}{article}
%
%    Programm calls to get the documentation (example):
%       pdflatex atenddvi.dtx
%       makeindex -s gind.ist atenddvi.idx
%       pdflatex atenddvi.dtx
%       makeindex -s gind.ist atenddvi.idx
%       pdflatex atenddvi.dtx
%
% Installation:
%    TDS:tex/latex/oberdiek/atenddvi.sty
%    TDS:doc/latex/oberdiek/atenddvi.pdf
%    TDS:source/latex/oberdiek/atenddvi.dtx
%
%<*ignore>
\begingroup
  \catcode123=1 %
  \catcode125=2 %
  \def\x{LaTeX2e}%
\expandafter\endgroup
\ifcase 0\ifx\install y1\fi\expandafter
         \ifx\csname processbatchFile\endcsname\relax\else1\fi
         \ifx\fmtname\x\else 1\fi\relax
\else\csname fi\endcsname
%</ignore>
%<*install>
\input docstrip.tex
\Msg{************************************************************************}
\Msg{* Installation}
\Msg{* Package: atenddvi 2016/05/16 v1.2 At end DVI hook (HO)}
\Msg{************************************************************************}

\keepsilent
\askforoverwritefalse

\let\MetaPrefix\relax
\preamble

This is a generated file.

Project: atenddvi
Version: 2016/05/16 v1.2

Copyright (C)
   2007 Heiko Oberdiek
   2016-2019 Oberdiek Package Support Group

This work may be distributed and/or modified under the
conditions of the LaTeX Project Public License, either
version 1.3c of this license or (at your option) any later
version. This version of this license is in
   https://www.latex-project.org/lppl/lppl-1-3c.txt
and the latest version of this license is in
   https://www.latex-project.org/lppl.txt
and version 1.3 or later is part of all distributions of
LaTeX version 2005/12/01 or later.

This work has the LPPL maintenance status "maintained".

The Current Maintainers of this work are
Heiko Oberdiek and the Oberdiek Package Support Group
https://github.com/ho-tex/oberdiek/issues


This work consists of the main source file atenddvi.dtx
and the derived files
   atenddvi.sty, atenddvi.pdf, atenddvi.ins, atenddvi.drv.

\endpreamble
\let\MetaPrefix\DoubleperCent

\generate{%
  \file{atenddvi.ins}{\from{atenddvi.dtx}{install}}%
  \file{atenddvi.drv}{\from{atenddvi.dtx}{driver}}%
  \usedir{tex/latex/oberdiek}%
  \file{atenddvi.sty}{\from{atenddvi.dtx}{package}}%
  \nopreamble
  \nopostamble
%  \usedir{source/latex/oberdiek/catalogue}%
%  \file{atenddvi.xml}{\from{atenddvi.dtx}{catalogue}}%
}

\catcode32=13\relax% active space
\let =\space%
\Msg{************************************************************************}
\Msg{*}
\Msg{* To finish the installation you have to move the following}
\Msg{* file into a directory searched by TeX:}
\Msg{*}
\Msg{*     atenddvi.sty}
\Msg{*}
\Msg{* To produce the documentation run the file `atenddvi.drv'}
\Msg{* through LaTeX.}
\Msg{*}
\Msg{* Happy TeXing!}
\Msg{*}
\Msg{************************************************************************}

\endbatchfile
%</install>
%<*ignore>
\fi
%</ignore>
%<*driver>
\NeedsTeXFormat{LaTeX2e}
\ProvidesFile{atenddvi.drv}%
  [2016/05/16 v1.2 At end DVI hook (HO)]%
\documentclass{ltxdoc}
\usepackage{holtxdoc}[2011/11/22]
\begin{document}
  \DocInput{atenddvi.dtx}%
\end{document}
%</driver>
% \fi
%
%
% \CharacterTable
%  {Upper-case    \A\B\C\D\E\F\G\H\I\J\K\L\M\N\O\P\Q\R\S\T\U\V\W\X\Y\Z
%   Lower-case    \a\b\c\d\e\f\g\h\i\j\k\l\m\n\o\p\q\r\s\t\u\v\w\x\y\z
%   Digits        \0\1\2\3\4\5\6\7\8\9
%   Exclamation   \!     Double quote  \"     Hash (number) \#
%   Dollar        \$     Percent       \%     Ampersand     \&
%   Acute accent  \'     Left paren    \(     Right paren   \)
%   Asterisk      \*     Plus          \+     Comma         \,
%   Minus         \-     Point         \.     Solidus       \/
%   Colon         \:     Semicolon     \;     Less than     \<
%   Equals        \=     Greater than  \>     Question mark \?
%   Commercial at \@     Left bracket  \[     Backslash     \\
%   Right bracket \]     Circumflex    \^     Underscore    \_
%   Grave accent  \`     Left brace    \{     Vertical bar  \|
%   Right brace   \}     Tilde         \~}
%
% \GetFileInfo{atenddvi.drv}
%
% \title{The \xpackage{atenddvi} package}
% \date{2016/05/16 v1.2}
% \author{Heiko Oberdiek\thanks
% {Please report any issues at \url{https://github.com/ho-tex/oberdiek/issues}}}
%
% \maketitle
%
% \begin{abstract}
% \LaTeX\ offers \cs{AtBeginDvi}. This package \xpackage{atenddvi}
% provides the counterpart \cs{AtEndDvi}. The execution of its
% argument is delayed to the end of the document at the end of the
% last page. Thus \cs{special} and \cs{write} remain effective, because
% they are put into the last page. This is the main difference
% to \cs{AtEndDocument}.
% \end{abstract}
%
% \tableofcontents
%
% \section{Documentation}
%
% \begin{declcs}{AtEndDvi} \M{code}
% \end{declcs}
% Macro \cs{AtEndDvi} provides a hook mechanism to put \meta{code}
% at the end of the last output page. It is the logical counterpart
% to \cs{AtBeginDvi}. Despite the name the output type DVI, PDF or whatever
% does not matter.
%
% Unlike \cs{AtBeginDvi} the \meta{code} is not put in a box and
% therefore executed immediately. The hook for \cs{AtEndDvi} is based on
% a macro similar to \cs{AtBeginDocument} or \cs{AtEndDocument}. The
% execution of \meta{code} is delayed until the hook is executed on
% the last page.
%
% Commands such as \cs{special} or \cs{write} (not the \cs{immediate}
% variant) must go as nodes into the contents of a page to have the
% desired effect.
% When the hook for \cs{AtEndDocument} is executed, the last intended
% page may already be shipped out. Therefore \cs{special} or \cs{write}
% cannot be used in a reliable way without generating new page.
%
% This gap is closed by \cs{AtEndDvi} of this package \xpackage{atenddvi}.
% If the document is compiled the first time, the package remembers
% the last page in a reference. In the sceond run, it puts the hook
% on the page that has been detected in the previous run as last page.
% The package detectes if the number of pages has changed, and then
% generates a warning to rerun \LaTeX.
%
% \StopEventually{
% }
%
% \section{Implementation}
%
%    \begin{macrocode}
%<*package>
\NeedsTeXFormat{LaTeX2e}
\ProvidesPackage{atenddvi}%
  [2016/05/16 v1.2 At end DVI hook (HO)]%
%    \end{macrocode}
%
%    Load the required packages
%    \begin{macrocode}
\RequirePackage{zref-abspage,zref-lastpage}[2007/03/19]
\RequirePackage{atbegshi}
%    \end{macrocode}
%
%    \begin{macro}{\AtEndDvi@Hook}
%    Macro \cs{AtEndDvi@Hook} is the data storage macro
%    for the code that is executed later at end of the last page.
%    \begin{macrocode}
\let\AtEndDvi@Hook\@empty
%    \end{macrocode}
%    \end{macro}
%    \begin{macro}{\AtEndDvi}
%    Macro \cs{AtEndDvi} is called in the same way as
%    \cs{AtBeginDocument}. The argument is added to the hook macro.
%    \begin{macrocode}
\newcommand*{\AtEndDvi}{%
  \g@addto@macro\AtEndDvi@Hook
}
%    \end{macrocode}
%    \end{macro}
%
%    \begin{macro}{\AtEndDvi@AtBeginShipout}
%    \begin{macrocode}
\def\AtEndDvi@AtBeginShipout{%
  \begingroup
%    \end{macrocode}
%    The reference `LastPage' is marked used. If the reference
%    is not yet defined, then the user gets the warning because of
%    the undefined reference and the rerun warning at the end of
%    the compile run. However, we do not need a warning each page,
%    the first page is enough.
%    \begin{macrocode}
    \ifnum\value{abspage}=1 %
      \zref@refused{LastPage}%
    \fi
%    \end{macrocode}
%    The current absolute page number is compared with the
%    absolute page number of the reference `LastPage'.
%    \begin{macrocode}
    \ifnum\zref@extractdefault{LastPage}{abspage}{0}=\value{abspage}%
%    \end{macrocode}
%    \begin{macro}{\AtEndDvi@LastPage}
%    We found the right page and remember it in a macro.
%    \begin{macrocode}
      \xdef\AtEndDvi@LastPage{\number\value{abspage}}%
%    \end{macrocode}
%    \end{macro}
%    The hook of \cs{AtEndDvi} is now put on the last page
%    after the contents of the page.
%    \begin{macrocode}
      \global\setbox\AtBeginShipoutBox=\vbox{%
        \hbox{%
          \box\AtBeginShipoutBox
          \setbox\AtBeginShipoutBox=\hbox{%
            \begingroup
              \AtEndDvi@Hook
            \endgroup
          }%
          \wd\AtBeginShipoutBox=\z@
          \ht\AtBeginShipoutBox=\z@
          \dp\AtBeginShipoutBox=\z@
          \box\AtBeginShipoutBox
        }%
      }%
%    \end{macrocode}
%    We do not need the every page hook.
%    \begin{macrocode}
      \global\let\AtEndDvi@AtBeginShipout\@empty
%    \end{macrocode}
%    The hook is consumed, \cs{AtEndDvi} does not have an effect.
%    \begin{macrocode}
      \global\let\AtEndDvi\@gobble
%    \end{macrocode}
%    Make a protocol entry, which page is used by this package
%    as last page.
%    \begin{macrocode}
      \let\on@line\@empty
      \PackageInfo{atenddvi}{Last page = \AtEndDvi@LastPage}%
    \fi
  \endgroup
}
%    \end{macrocode}
%    \end{macro}
%
%    \begin{macro}{\AtEndDvi@AtBeginDocument}
%    In order to get as late as possible in the chain of the
%    every shipout hook, the call of \cs{AtBeginShipout} is delayed.
%    \begin{macrocode}
\def\AtEndDvi@AtBeginDocument{%
  \AtBeginShipout{\AtEndDvi@AtBeginShipout}%
%    \end{macrocode}
%    \begin{macro}{\AtEndDvi@Check}
%    After \cs{AtEndDocument} \LaTeX\ reads its \xfile{.aux} files
%    again. Code in \cs{AtEndDocument} could generate additional
%    pages. This is unlikely by code in the \xfile{.aux} file,
%    thus we use the \xfile{.aux} file to run macro
%    \cs{AtEndDvi@Check} for checking the last page.
%
%    During the first reading of the \xfile{.aux} file,
%    \cs{AtEndDvi@Check} is disabled, its real meaning
%    is assigned afterwards.
%    \begin{macrocode}
  \if@filesw
    \immediate\write\@mainaux{%
      \string\providecommand\string\AtEndDvi@Check{}%
    }%
    \immediate\write\@mainaux{%
      \string\AtEndDvi@Check
    }%
  \fi
  \let\AtEndDvi@Check\AtEndDvi@CheckImpl
}
%    \end{macrocode}
%    \end{macro}
%    \begin{macrocode}
\AtBeginDocument{\AtEndDvi@AtBeginDocument}
%    \end{macrocode}
%    \end{macro}
%
%    \begin{macro}{\AtEndDvi@CheckImpl}
%    First check is whether a last page was found at all.
%    Secondly the found last page is compared with the real last page.
%    \begin{macrocode}
\def\AtEndDvi@CheckImpl{%
  \@ifundefined{AtEndDvi@LastPage}{%
    \PackageWarningNoLine{atenddvi}{%
      Rerun LaTeX, last page not yet found%
    }%
  }{%
    \ifnum\AtEndDvi@LastPage=\value{abspage}%
    \else
      \PackageWarningNoLine{atenddvi}{%
        Rerun LaTeX, last page has changed%
      }%
    \fi
  }%
}
%    \end{macrocode}
%    \end{macro}
%
%    \begin{macrocode}
%</package>
%    \end{macrocode}
%
% \section{Installation}
%
% \subsection{Download}
%
% \paragraph{Package.} This package is available on
% CTAN\footnote{\CTANpkg{atenddvi}}:
% \begin{description}
% \item[\CTAN{macros/latex/contrib/oberdiek/atenddvi.dtx}] The source file.
% \item[\CTAN{macros/latex/contrib/oberdiek/atenddvi.pdf}] Documentation.
% \end{description}
%
%
% \paragraph{Bundle.} All the packages of the bundle `oberdiek'
% are also available in a TDS compliant ZIP archive. There
% the packages are already unpacked and the documentation files
% are generated. The files and directories obey the TDS standard.
% \begin{description}
% \item[\CTANinstall{install/macros/latex/contrib/oberdiek.tds.zip}]
% \end{description}
% \emph{TDS} refers to the standard ``A Directory Structure
% for \TeX\ Files'' (\CTAN{tds/tds.pdf}). Directories
% with \xfile{texmf} in their name are usually organized this way.
%
% \subsection{Bundle installation}
%
% \paragraph{Unpacking.} Unpack the \xfile{oberdiek.tds.zip} in the
% TDS tree (also known as \xfile{texmf} tree) of your choice.
% Example (linux):
% \begin{quote}
%   |unzip oberdiek.tds.zip -d ~/texmf|
% \end{quote}
%
% \subsection{Package installation}
%
% \paragraph{Unpacking.} The \xfile{.dtx} file is a self-extracting
% \docstrip\ archive. The files are extracted by running the
% \xfile{.dtx} through \plainTeX:
% \begin{quote}
%   \verb|tex atenddvi.dtx|
% \end{quote}
%
% \paragraph{TDS.} Now the different files must be moved into
% the different directories in your installation TDS tree
% (also known as \xfile{texmf} tree):
% \begin{quote}
% \def\t{^^A
% \begin{tabular}{@{}>{\ttfamily}l@{ $\rightarrow$ }>{\ttfamily}l@{}}
%   atenddvi.sty & tex/latex/oberdiek/atenddvi.sty\\
%   atenddvi.pdf & doc/latex/oberdiek/atenddvi.pdf\\
%   atenddvi.dtx & source/latex/oberdiek/atenddvi.dtx\\
% \end{tabular}^^A
% }^^A
% \sbox0{\t}^^A
% \ifdim\wd0>\linewidth
%   \begingroup
%     \advance\linewidth by\leftmargin
%     \advance\linewidth by\rightmargin
%   \edef\x{\endgroup
%     \def\noexpand\lw{\the\linewidth}^^A
%   }\x
%   \def\lwbox{^^A
%     \leavevmode
%     \hbox to \linewidth{^^A
%       \kern-\leftmargin\relax
%       \hss
%       \usebox0
%       \hss
%       \kern-\rightmargin\relax
%     }^^A
%   }^^A
%   \ifdim\wd0>\lw
%     \sbox0{\small\t}^^A
%     \ifdim\wd0>\linewidth
%       \ifdim\wd0>\lw
%         \sbox0{\footnotesize\t}^^A
%         \ifdim\wd0>\linewidth
%           \ifdim\wd0>\lw
%             \sbox0{\scriptsize\t}^^A
%             \ifdim\wd0>\linewidth
%               \ifdim\wd0>\lw
%                 \sbox0{\tiny\t}^^A
%                 \ifdim\wd0>\linewidth
%                   \lwbox
%                 \else
%                   \usebox0
%                 \fi
%               \else
%                 \lwbox
%               \fi
%             \else
%               \usebox0
%             \fi
%           \else
%             \lwbox
%           \fi
%         \else
%           \usebox0
%         \fi
%       \else
%         \lwbox
%       \fi
%     \else
%       \usebox0
%     \fi
%   \else
%     \lwbox
%   \fi
% \else
%   \usebox0
% \fi
% \end{quote}
% If you have a \xfile{docstrip.cfg} that configures and enables \docstrip's
% TDS installing feature, then some files can already be in the right
% place, see the documentation of \docstrip.
%
% \subsection{Refresh file name databases}
%
% If your \TeX~distribution
% (\TeX\,Live, \mikTeX, \dots) relies on file name databases, you must refresh
% these. For example, \TeX\,Live\ users run \verb|texhash| or
% \verb|mktexlsr|.
%
% \subsection{Some details for the interested}
%
% \paragraph{Unpacking with \LaTeX.}
% The \xfile{.dtx} chooses its action depending on the format:
% \begin{description}
% \item[\plainTeX:] Run \docstrip\ and extract the files.
% \item[\LaTeX:] Generate the documentation.
% \end{description}
% If you insist on using \LaTeX\ for \docstrip\ (really,
% \docstrip\ does not need \LaTeX), then inform the autodetect routine
% about your intention:
% \begin{quote}
%   \verb|latex \let\install=y% \iffalse meta-comment
%
% File: atenddvi.dtx
% Version: 2016/05/16 v1.2
% Info: At end DVI hook
%
% Copyright (C)
%    2007 Heiko Oberdiek
%    2016-2019 Oberdiek Package Support Group
%    https://github.com/ho-tex/oberdiek/issues
%
% This work may be distributed and/or modified under the
% conditions of the LaTeX Project Public License, either
% version 1.3c of this license or (at your option) any later
% version. This version of this license is in
%    https://www.latex-project.org/lppl/lppl-1-3c.txt
% and the latest version of this license is in
%    https://www.latex-project.org/lppl.txt
% and version 1.3 or later is part of all distributions of
% LaTeX version 2005/12/01 or later.
%
% This work has the LPPL maintenance status "maintained".
%
% The Current Maintainers of this work are
% Heiko Oberdiek and the Oberdiek Package Support Group
% https://github.com/ho-tex/oberdiek/issues
%
% This work consists of the main source file atenddvi.dtx
% and the derived files
%    atenddvi.sty, atenddvi.pdf, atenddvi.ins, atenddvi.drv.
%
% Distribution:
%    CTAN:macros/latex/contrib/oberdiek/atenddvi.dtx
%    CTAN:macros/latex/contrib/oberdiek/atenddvi.pdf
%
% Unpacking:
%    (a) If atenddvi.ins is present:
%           tex atenddvi.ins
%    (b) Without atenddvi.ins:
%           tex atenddvi.dtx
%    (c) If you insist on using LaTeX
%           latex \let\install=y\input{atenddvi.dtx}
%        (quote the arguments according to the demands of your shell)
%
% Documentation:
%    (a) If atenddvi.drv is present:
%           latex atenddvi.drv
%    (b) Without atenddvi.drv:
%           latex atenddvi.dtx; ...
%    The class ltxdoc loads the configuration file ltxdoc.cfg
%    if available. Here you can specify further options, e.g.
%    use A4 as paper format:
%       \PassOptionsToClass{a4paper}{article}
%
%    Programm calls to get the documentation (example):
%       pdflatex atenddvi.dtx
%       makeindex -s gind.ist atenddvi.idx
%       pdflatex atenddvi.dtx
%       makeindex -s gind.ist atenddvi.idx
%       pdflatex atenddvi.dtx
%
% Installation:
%    TDS:tex/latex/oberdiek/atenddvi.sty
%    TDS:doc/latex/oberdiek/atenddvi.pdf
%    TDS:source/latex/oberdiek/atenddvi.dtx
%
%<*ignore>
\begingroup
  \catcode123=1 %
  \catcode125=2 %
  \def\x{LaTeX2e}%
\expandafter\endgroup
\ifcase 0\ifx\install y1\fi\expandafter
         \ifx\csname processbatchFile\endcsname\relax\else1\fi
         \ifx\fmtname\x\else 1\fi\relax
\else\csname fi\endcsname
%</ignore>
%<*install>
\input docstrip.tex
\Msg{************************************************************************}
\Msg{* Installation}
\Msg{* Package: atenddvi 2016/05/16 v1.2 At end DVI hook (HO)}
\Msg{************************************************************************}

\keepsilent
\askforoverwritefalse

\let\MetaPrefix\relax
\preamble

This is a generated file.

Project: atenddvi
Version: 2016/05/16 v1.2

Copyright (C)
   2007 Heiko Oberdiek
   2016-2019 Oberdiek Package Support Group

This work may be distributed and/or modified under the
conditions of the LaTeX Project Public License, either
version 1.3c of this license or (at your option) any later
version. This version of this license is in
   https://www.latex-project.org/lppl/lppl-1-3c.txt
and the latest version of this license is in
   https://www.latex-project.org/lppl.txt
and version 1.3 or later is part of all distributions of
LaTeX version 2005/12/01 or later.

This work has the LPPL maintenance status "maintained".

The Current Maintainers of this work are
Heiko Oberdiek and the Oberdiek Package Support Group
https://github.com/ho-tex/oberdiek/issues


This work consists of the main source file atenddvi.dtx
and the derived files
   atenddvi.sty, atenddvi.pdf, atenddvi.ins, atenddvi.drv.

\endpreamble
\let\MetaPrefix\DoubleperCent

\generate{%
  \file{atenddvi.ins}{\from{atenddvi.dtx}{install}}%
  \file{atenddvi.drv}{\from{atenddvi.dtx}{driver}}%
  \usedir{tex/latex/oberdiek}%
  \file{atenddvi.sty}{\from{atenddvi.dtx}{package}}%
  \nopreamble
  \nopostamble
%  \usedir{source/latex/oberdiek/catalogue}%
%  \file{atenddvi.xml}{\from{atenddvi.dtx}{catalogue}}%
}

\catcode32=13\relax% active space
\let =\space%
\Msg{************************************************************************}
\Msg{*}
\Msg{* To finish the installation you have to move the following}
\Msg{* file into a directory searched by TeX:}
\Msg{*}
\Msg{*     atenddvi.sty}
\Msg{*}
\Msg{* To produce the documentation run the file `atenddvi.drv'}
\Msg{* through LaTeX.}
\Msg{*}
\Msg{* Happy TeXing!}
\Msg{*}
\Msg{************************************************************************}

\endbatchfile
%</install>
%<*ignore>
\fi
%</ignore>
%<*driver>
\NeedsTeXFormat{LaTeX2e}
\ProvidesFile{atenddvi.drv}%
  [2016/05/16 v1.2 At end DVI hook (HO)]%
\documentclass{ltxdoc}
\usepackage{holtxdoc}[2011/11/22]
\begin{document}
  \DocInput{atenddvi.dtx}%
\end{document}
%</driver>
% \fi
%
%
% \CharacterTable
%  {Upper-case    \A\B\C\D\E\F\G\H\I\J\K\L\M\N\O\P\Q\R\S\T\U\V\W\X\Y\Z
%   Lower-case    \a\b\c\d\e\f\g\h\i\j\k\l\m\n\o\p\q\r\s\t\u\v\w\x\y\z
%   Digits        \0\1\2\3\4\5\6\7\8\9
%   Exclamation   \!     Double quote  \"     Hash (number) \#
%   Dollar        \$     Percent       \%     Ampersand     \&
%   Acute accent  \'     Left paren    \(     Right paren   \)
%   Asterisk      \*     Plus          \+     Comma         \,
%   Minus         \-     Point         \.     Solidus       \/
%   Colon         \:     Semicolon     \;     Less than     \<
%   Equals        \=     Greater than  \>     Question mark \?
%   Commercial at \@     Left bracket  \[     Backslash     \\
%   Right bracket \]     Circumflex    \^     Underscore    \_
%   Grave accent  \`     Left brace    \{     Vertical bar  \|
%   Right brace   \}     Tilde         \~}
%
% \GetFileInfo{atenddvi.drv}
%
% \title{The \xpackage{atenddvi} package}
% \date{2016/05/16 v1.2}
% \author{Heiko Oberdiek\thanks
% {Please report any issues at \url{https://github.com/ho-tex/oberdiek/issues}}}
%
% \maketitle
%
% \begin{abstract}
% \LaTeX\ offers \cs{AtBeginDvi}. This package \xpackage{atenddvi}
% provides the counterpart \cs{AtEndDvi}. The execution of its
% argument is delayed to the end of the document at the end of the
% last page. Thus \cs{special} and \cs{write} remain effective, because
% they are put into the last page. This is the main difference
% to \cs{AtEndDocument}.
% \end{abstract}
%
% \tableofcontents
%
% \section{Documentation}
%
% \begin{declcs}{AtEndDvi} \M{code}
% \end{declcs}
% Macro \cs{AtEndDvi} provides a hook mechanism to put \meta{code}
% at the end of the last output page. It is the logical counterpart
% to \cs{AtBeginDvi}. Despite the name the output type DVI, PDF or whatever
% does not matter.
%
% Unlike \cs{AtBeginDvi} the \meta{code} is not put in a box and
% therefore executed immediately. The hook for \cs{AtEndDvi} is based on
% a macro similar to \cs{AtBeginDocument} or \cs{AtEndDocument}. The
% execution of \meta{code} is delayed until the hook is executed on
% the last page.
%
% Commands such as \cs{special} or \cs{write} (not the \cs{immediate}
% variant) must go as nodes into the contents of a page to have the
% desired effect.
% When the hook for \cs{AtEndDocument} is executed, the last intended
% page may already be shipped out. Therefore \cs{special} or \cs{write}
% cannot be used in a reliable way without generating new page.
%
% This gap is closed by \cs{AtEndDvi} of this package \xpackage{atenddvi}.
% If the document is compiled the first time, the package remembers
% the last page in a reference. In the sceond run, it puts the hook
% on the page that has been detected in the previous run as last page.
% The package detectes if the number of pages has changed, and then
% generates a warning to rerun \LaTeX.
%
% \StopEventually{
% }
%
% \section{Implementation}
%
%    \begin{macrocode}
%<*package>
\NeedsTeXFormat{LaTeX2e}
\ProvidesPackage{atenddvi}%
  [2016/05/16 v1.2 At end DVI hook (HO)]%
%    \end{macrocode}
%
%    Load the required packages
%    \begin{macrocode}
\RequirePackage{zref-abspage,zref-lastpage}[2007/03/19]
\RequirePackage{atbegshi}
%    \end{macrocode}
%
%    \begin{macro}{\AtEndDvi@Hook}
%    Macro \cs{AtEndDvi@Hook} is the data storage macro
%    for the code that is executed later at end of the last page.
%    \begin{macrocode}
\let\AtEndDvi@Hook\@empty
%    \end{macrocode}
%    \end{macro}
%    \begin{macro}{\AtEndDvi}
%    Macro \cs{AtEndDvi} is called in the same way as
%    \cs{AtBeginDocument}. The argument is added to the hook macro.
%    \begin{macrocode}
\newcommand*{\AtEndDvi}{%
  \g@addto@macro\AtEndDvi@Hook
}
%    \end{macrocode}
%    \end{macro}
%
%    \begin{macro}{\AtEndDvi@AtBeginShipout}
%    \begin{macrocode}
\def\AtEndDvi@AtBeginShipout{%
  \begingroup
%    \end{macrocode}
%    The reference `LastPage' is marked used. If the reference
%    is not yet defined, then the user gets the warning because of
%    the undefined reference and the rerun warning at the end of
%    the compile run. However, we do not need a warning each page,
%    the first page is enough.
%    \begin{macrocode}
    \ifnum\value{abspage}=1 %
      \zref@refused{LastPage}%
    \fi
%    \end{macrocode}
%    The current absolute page number is compared with the
%    absolute page number of the reference `LastPage'.
%    \begin{macrocode}
    \ifnum\zref@extractdefault{LastPage}{abspage}{0}=\value{abspage}%
%    \end{macrocode}
%    \begin{macro}{\AtEndDvi@LastPage}
%    We found the right page and remember it in a macro.
%    \begin{macrocode}
      \xdef\AtEndDvi@LastPage{\number\value{abspage}}%
%    \end{macrocode}
%    \end{macro}
%    The hook of \cs{AtEndDvi} is now put on the last page
%    after the contents of the page.
%    \begin{macrocode}
      \global\setbox\AtBeginShipoutBox=\vbox{%
        \hbox{%
          \box\AtBeginShipoutBox
          \setbox\AtBeginShipoutBox=\hbox{%
            \begingroup
              \AtEndDvi@Hook
            \endgroup
          }%
          \wd\AtBeginShipoutBox=\z@
          \ht\AtBeginShipoutBox=\z@
          \dp\AtBeginShipoutBox=\z@
          \box\AtBeginShipoutBox
        }%
      }%
%    \end{macrocode}
%    We do not need the every page hook.
%    \begin{macrocode}
      \global\let\AtEndDvi@AtBeginShipout\@empty
%    \end{macrocode}
%    The hook is consumed, \cs{AtEndDvi} does not have an effect.
%    \begin{macrocode}
      \global\let\AtEndDvi\@gobble
%    \end{macrocode}
%    Make a protocol entry, which page is used by this package
%    as last page.
%    \begin{macrocode}
      \let\on@line\@empty
      \PackageInfo{atenddvi}{Last page = \AtEndDvi@LastPage}%
    \fi
  \endgroup
}
%    \end{macrocode}
%    \end{macro}
%
%    \begin{macro}{\AtEndDvi@AtBeginDocument}
%    In order to get as late as possible in the chain of the
%    every shipout hook, the call of \cs{AtBeginShipout} is delayed.
%    \begin{macrocode}
\def\AtEndDvi@AtBeginDocument{%
  \AtBeginShipout{\AtEndDvi@AtBeginShipout}%
%    \end{macrocode}
%    \begin{macro}{\AtEndDvi@Check}
%    After \cs{AtEndDocument} \LaTeX\ reads its \xfile{.aux} files
%    again. Code in \cs{AtEndDocument} could generate additional
%    pages. This is unlikely by code in the \xfile{.aux} file,
%    thus we use the \xfile{.aux} file to run macro
%    \cs{AtEndDvi@Check} for checking the last page.
%
%    During the first reading of the \xfile{.aux} file,
%    \cs{AtEndDvi@Check} is disabled, its real meaning
%    is assigned afterwards.
%    \begin{macrocode}
  \if@filesw
    \immediate\write\@mainaux{%
      \string\providecommand\string\AtEndDvi@Check{}%
    }%
    \immediate\write\@mainaux{%
      \string\AtEndDvi@Check
    }%
  \fi
  \let\AtEndDvi@Check\AtEndDvi@CheckImpl
}
%    \end{macrocode}
%    \end{macro}
%    \begin{macrocode}
\AtBeginDocument{\AtEndDvi@AtBeginDocument}
%    \end{macrocode}
%    \end{macro}
%
%    \begin{macro}{\AtEndDvi@CheckImpl}
%    First check is whether a last page was found at all.
%    Secondly the found last page is compared with the real last page.
%    \begin{macrocode}
\def\AtEndDvi@CheckImpl{%
  \@ifundefined{AtEndDvi@LastPage}{%
    \PackageWarningNoLine{atenddvi}{%
      Rerun LaTeX, last page not yet found%
    }%
  }{%
    \ifnum\AtEndDvi@LastPage=\value{abspage}%
    \else
      \PackageWarningNoLine{atenddvi}{%
        Rerun LaTeX, last page has changed%
      }%
    \fi
  }%
}
%    \end{macrocode}
%    \end{macro}
%
%    \begin{macrocode}
%</package>
%    \end{macrocode}
%
% \section{Installation}
%
% \subsection{Download}
%
% \paragraph{Package.} This package is available on
% CTAN\footnote{\CTANpkg{atenddvi}}:
% \begin{description}
% \item[\CTAN{macros/latex/contrib/oberdiek/atenddvi.dtx}] The source file.
% \item[\CTAN{macros/latex/contrib/oberdiek/atenddvi.pdf}] Documentation.
% \end{description}
%
%
% \paragraph{Bundle.} All the packages of the bundle `oberdiek'
% are also available in a TDS compliant ZIP archive. There
% the packages are already unpacked and the documentation files
% are generated. The files and directories obey the TDS standard.
% \begin{description}
% \item[\CTANinstall{install/macros/latex/contrib/oberdiek.tds.zip}]
% \end{description}
% \emph{TDS} refers to the standard ``A Directory Structure
% for \TeX\ Files'' (\CTAN{tds/tds.pdf}). Directories
% with \xfile{texmf} in their name are usually organized this way.
%
% \subsection{Bundle installation}
%
% \paragraph{Unpacking.} Unpack the \xfile{oberdiek.tds.zip} in the
% TDS tree (also known as \xfile{texmf} tree) of your choice.
% Example (linux):
% \begin{quote}
%   |unzip oberdiek.tds.zip -d ~/texmf|
% \end{quote}
%
% \subsection{Package installation}
%
% \paragraph{Unpacking.} The \xfile{.dtx} file is a self-extracting
% \docstrip\ archive. The files are extracted by running the
% \xfile{.dtx} through \plainTeX:
% \begin{quote}
%   \verb|tex atenddvi.dtx|
% \end{quote}
%
% \paragraph{TDS.} Now the different files must be moved into
% the different directories in your installation TDS tree
% (also known as \xfile{texmf} tree):
% \begin{quote}
% \def\t{^^A
% \begin{tabular}{@{}>{\ttfamily}l@{ $\rightarrow$ }>{\ttfamily}l@{}}
%   atenddvi.sty & tex/latex/oberdiek/atenddvi.sty\\
%   atenddvi.pdf & doc/latex/oberdiek/atenddvi.pdf\\
%   atenddvi.dtx & source/latex/oberdiek/atenddvi.dtx\\
% \end{tabular}^^A
% }^^A
% \sbox0{\t}^^A
% \ifdim\wd0>\linewidth
%   \begingroup
%     \advance\linewidth by\leftmargin
%     \advance\linewidth by\rightmargin
%   \edef\x{\endgroup
%     \def\noexpand\lw{\the\linewidth}^^A
%   }\x
%   \def\lwbox{^^A
%     \leavevmode
%     \hbox to \linewidth{^^A
%       \kern-\leftmargin\relax
%       \hss
%       \usebox0
%       \hss
%       \kern-\rightmargin\relax
%     }^^A
%   }^^A
%   \ifdim\wd0>\lw
%     \sbox0{\small\t}^^A
%     \ifdim\wd0>\linewidth
%       \ifdim\wd0>\lw
%         \sbox0{\footnotesize\t}^^A
%         \ifdim\wd0>\linewidth
%           \ifdim\wd0>\lw
%             \sbox0{\scriptsize\t}^^A
%             \ifdim\wd0>\linewidth
%               \ifdim\wd0>\lw
%                 \sbox0{\tiny\t}^^A
%                 \ifdim\wd0>\linewidth
%                   \lwbox
%                 \else
%                   \usebox0
%                 \fi
%               \else
%                 \lwbox
%               \fi
%             \else
%               \usebox0
%             \fi
%           \else
%             \lwbox
%           \fi
%         \else
%           \usebox0
%         \fi
%       \else
%         \lwbox
%       \fi
%     \else
%       \usebox0
%     \fi
%   \else
%     \lwbox
%   \fi
% \else
%   \usebox0
% \fi
% \end{quote}
% If you have a \xfile{docstrip.cfg} that configures and enables \docstrip's
% TDS installing feature, then some files can already be in the right
% place, see the documentation of \docstrip.
%
% \subsection{Refresh file name databases}
%
% If your \TeX~distribution
% (\TeX\,Live, \mikTeX, \dots) relies on file name databases, you must refresh
% these. For example, \TeX\,Live\ users run \verb|texhash| or
% \verb|mktexlsr|.
%
% \subsection{Some details for the interested}
%
% \paragraph{Unpacking with \LaTeX.}
% The \xfile{.dtx} chooses its action depending on the format:
% \begin{description}
% \item[\plainTeX:] Run \docstrip\ and extract the files.
% \item[\LaTeX:] Generate the documentation.
% \end{description}
% If you insist on using \LaTeX\ for \docstrip\ (really,
% \docstrip\ does not need \LaTeX), then inform the autodetect routine
% about your intention:
% \begin{quote}
%   \verb|latex \let\install=y\input{atenddvi.dtx}|
% \end{quote}
% Do not forget to quote the argument according to the demands
% of your shell.
%
% \paragraph{Generating the documentation.}
% You can use both the \xfile{.dtx} or the \xfile{.drv} to generate
% the documentation. The process can be configured by the
% configuration file \xfile{ltxdoc.cfg}. For instance, put this
% line into this file, if you want to have A4 as paper format:
% \begin{quote}
%   \verb|\PassOptionsToClass{a4paper}{article}|
% \end{quote}
% An example follows how to generate the
% documentation with pdf\LaTeX:
% \begin{quote}
%\begin{verbatim}
%pdflatex atenddvi.dtx
%makeindex -s gind.ist atenddvi.idx
%pdflatex atenddvi.dtx
%makeindex -s gind.ist atenddvi.idx
%pdflatex atenddvi.dtx
%\end{verbatim}
% \end{quote}
%
% \begin{History}
%   \begin{Version}{2007/03/20 v1.0}
%   \item
%     First version.
%   \end{Version}
%   \begin{Version}{2007/04/17 v1.1}
%   \item
%     Package \xpackage{atbegshi} replaces package \xpackage{everyshi}.
%   \end{Version}
%   \begin{Version}{2016/05/16 v1.2}
%   \item
%     Documentation updates.
%   \end{Version}
% \end{History}
%
% \PrintIndex
%
% \Finale
\endinput
|
% \end{quote}
% Do not forget to quote the argument according to the demands
% of your shell.
%
% \paragraph{Generating the documentation.}
% You can use both the \xfile{.dtx} or the \xfile{.drv} to generate
% the documentation. The process can be configured by the
% configuration file \xfile{ltxdoc.cfg}. For instance, put this
% line into this file, if you want to have A4 as paper format:
% \begin{quote}
%   \verb|\PassOptionsToClass{a4paper}{article}|
% \end{quote}
% An example follows how to generate the
% documentation with pdf\LaTeX:
% \begin{quote}
%\begin{verbatim}
%pdflatex atenddvi.dtx
%makeindex -s gind.ist atenddvi.idx
%pdflatex atenddvi.dtx
%makeindex -s gind.ist atenddvi.idx
%pdflatex atenddvi.dtx
%\end{verbatim}
% \end{quote}
%
% \begin{History}
%   \begin{Version}{2007/03/20 v1.0}
%   \item
%     First version.
%   \end{Version}
%   \begin{Version}{2007/04/17 v1.1}
%   \item
%     Package \xpackage{atbegshi} replaces package \xpackage{everyshi}.
%   \end{Version}
%   \begin{Version}{2016/05/16 v1.2}
%   \item
%     Documentation updates.
%   \end{Version}
% \end{History}
%
% \PrintIndex
%
% \Finale
\endinput

%        (quote the arguments according to the demands of your shell)
%
% Documentation:
%    (a) If atenddvi.drv is present:
%           latex atenddvi.drv
%    (b) Without atenddvi.drv:
%           latex atenddvi.dtx; ...
%    The class ltxdoc loads the configuration file ltxdoc.cfg
%    if available. Here you can specify further options, e.g.
%    use A4 as paper format:
%       \PassOptionsToClass{a4paper}{article}
%
%    Programm calls to get the documentation (example):
%       pdflatex atenddvi.dtx
%       makeindex -s gind.ist atenddvi.idx
%       pdflatex atenddvi.dtx
%       makeindex -s gind.ist atenddvi.idx
%       pdflatex atenddvi.dtx
%
% Installation:
%    TDS:tex/latex/oberdiek/atenddvi.sty
%    TDS:doc/latex/oberdiek/atenddvi.pdf
%    TDS:source/latex/oberdiek/atenddvi.dtx
%
%<*ignore>
\begingroup
  \catcode123=1 %
  \catcode125=2 %
  \def\x{LaTeX2e}%
\expandafter\endgroup
\ifcase 0\ifx\install y1\fi\expandafter
         \ifx\csname processbatchFile\endcsname\relax\else1\fi
         \ifx\fmtname\x\else 1\fi\relax
\else\csname fi\endcsname
%</ignore>
%<*install>
\input docstrip.tex
\Msg{************************************************************************}
\Msg{* Installation}
\Msg{* Package: atenddvi 2016/05/16 v1.2 At end DVI hook (HO)}
\Msg{************************************************************************}

\keepsilent
\askforoverwritefalse

\let\MetaPrefix\relax
\preamble

This is a generated file.

Project: atenddvi
Version: 2016/05/16 v1.2

Copyright (C)
   2007 Heiko Oberdiek
   2016-2019 Oberdiek Package Support Group

This work may be distributed and/or modified under the
conditions of the LaTeX Project Public License, either
version 1.3c of this license or (at your option) any later
version. This version of this license is in
   https://www.latex-project.org/lppl/lppl-1-3c.txt
and the latest version of this license is in
   https://www.latex-project.org/lppl.txt
and version 1.3 or later is part of all distributions of
LaTeX version 2005/12/01 or later.

This work has the LPPL maintenance status "maintained".

The Current Maintainers of this work are
Heiko Oberdiek and the Oberdiek Package Support Group
https://github.com/ho-tex/oberdiek/issues


This work consists of the main source file atenddvi.dtx
and the derived files
   atenddvi.sty, atenddvi.pdf, atenddvi.ins, atenddvi.drv.

\endpreamble
\let\MetaPrefix\DoubleperCent

\generate{%
  \file{atenddvi.ins}{\from{atenddvi.dtx}{install}}%
  \file{atenddvi.drv}{\from{atenddvi.dtx}{driver}}%
  \usedir{tex/latex/oberdiek}%
  \file{atenddvi.sty}{\from{atenddvi.dtx}{package}}%
  \nopreamble
  \nopostamble
%  \usedir{source/latex/oberdiek/catalogue}%
%  \file{atenddvi.xml}{\from{atenddvi.dtx}{catalogue}}%
}

\catcode32=13\relax% active space
\let =\space%
\Msg{************************************************************************}
\Msg{*}
\Msg{* To finish the installation you have to move the following}
\Msg{* file into a directory searched by TeX:}
\Msg{*}
\Msg{*     atenddvi.sty}
\Msg{*}
\Msg{* To produce the documentation run the file `atenddvi.drv'}
\Msg{* through LaTeX.}
\Msg{*}
\Msg{* Happy TeXing!}
\Msg{*}
\Msg{************************************************************************}

\endbatchfile
%</install>
%<*ignore>
\fi
%</ignore>
%<*driver>
\NeedsTeXFormat{LaTeX2e}
\ProvidesFile{atenddvi.drv}%
  [2016/05/16 v1.2 At end DVI hook (HO)]%
\documentclass{ltxdoc}
\usepackage{holtxdoc}[2011/11/22]
\begin{document}
  \DocInput{atenddvi.dtx}%
\end{document}
%</driver>
% \fi
%
%
% \CharacterTable
%  {Upper-case    \A\B\C\D\E\F\G\H\I\J\K\L\M\N\O\P\Q\R\S\T\U\V\W\X\Y\Z
%   Lower-case    \a\b\c\d\e\f\g\h\i\j\k\l\m\n\o\p\q\r\s\t\u\v\w\x\y\z
%   Digits        \0\1\2\3\4\5\6\7\8\9
%   Exclamation   \!     Double quote  \"     Hash (number) \#
%   Dollar        \$     Percent       \%     Ampersand     \&
%   Acute accent  \'     Left paren    \(     Right paren   \)
%   Asterisk      \*     Plus          \+     Comma         \,
%   Minus         \-     Point         \.     Solidus       \/
%   Colon         \:     Semicolon     \;     Less than     \<
%   Equals        \=     Greater than  \>     Question mark \?
%   Commercial at \@     Left bracket  \[     Backslash     \\
%   Right bracket \]     Circumflex    \^     Underscore    \_
%   Grave accent  \`     Left brace    \{     Vertical bar  \|
%   Right brace   \}     Tilde         \~}
%
% \GetFileInfo{atenddvi.drv}
%
% \title{The \xpackage{atenddvi} package}
% \date{2016/05/16 v1.2}
% \author{Heiko Oberdiek\thanks
% {Please report any issues at \url{https://github.com/ho-tex/oberdiek/issues}}}
%
% \maketitle
%
% \begin{abstract}
% \LaTeX\ offers \cs{AtBeginDvi}. This package \xpackage{atenddvi}
% provides the counterpart \cs{AtEndDvi}. The execution of its
% argument is delayed to the end of the document at the end of the
% last page. Thus \cs{special} and \cs{write} remain effective, because
% they are put into the last page. This is the main difference
% to \cs{AtEndDocument}.
% \end{abstract}
%
% \tableofcontents
%
% \section{Documentation}
%
% \begin{declcs}{AtEndDvi} \M{code}
% \end{declcs}
% Macro \cs{AtEndDvi} provides a hook mechanism to put \meta{code}
% at the end of the last output page. It is the logical counterpart
% to \cs{AtBeginDvi}. Despite the name the output type DVI, PDF or whatever
% does not matter.
%
% Unlike \cs{AtBeginDvi} the \meta{code} is not put in a box and
% therefore executed immediately. The hook for \cs{AtEndDvi} is based on
% a macro similar to \cs{AtBeginDocument} or \cs{AtEndDocument}. The
% execution of \meta{code} is delayed until the hook is executed on
% the last page.
%
% Commands such as \cs{special} or \cs{write} (not the \cs{immediate}
% variant) must go as nodes into the contents of a page to have the
% desired effect.
% When the hook for \cs{AtEndDocument} is executed, the last intended
% page may already be shipped out. Therefore \cs{special} or \cs{write}
% cannot be used in a reliable way without generating new page.
%
% This gap is closed by \cs{AtEndDvi} of this package \xpackage{atenddvi}.
% If the document is compiled the first time, the package remembers
% the last page in a reference. In the sceond run, it puts the hook
% on the page that has been detected in the previous run as last page.
% The package detectes if the number of pages has changed, and then
% generates a warning to rerun \LaTeX.
%
% \StopEventually{
% }
%
% \section{Implementation}
%
%    \begin{macrocode}
%<*package>
\NeedsTeXFormat{LaTeX2e}
\ProvidesPackage{atenddvi}%
  [2016/05/16 v1.2 At end DVI hook (HO)]%
%    \end{macrocode}
%
%    Load the required packages
%    \begin{macrocode}
\RequirePackage{zref-abspage,zref-lastpage}[2007/03/19]
\RequirePackage{atbegshi}
%    \end{macrocode}
%
%    \begin{macro}{\AtEndDvi@Hook}
%    Macro \cs{AtEndDvi@Hook} is the data storage macro
%    for the code that is executed later at end of the last page.
%    \begin{macrocode}
\let\AtEndDvi@Hook\@empty
%    \end{macrocode}
%    \end{macro}
%    \begin{macro}{\AtEndDvi}
%    Macro \cs{AtEndDvi} is called in the same way as
%    \cs{AtBeginDocument}. The argument is added to the hook macro.
%    \begin{macrocode}
\newcommand*{\AtEndDvi}{%
  \g@addto@macro\AtEndDvi@Hook
}
%    \end{macrocode}
%    \end{macro}
%
%    \begin{macro}{\AtEndDvi@AtBeginShipout}
%    \begin{macrocode}
\def\AtEndDvi@AtBeginShipout{%
  \begingroup
%    \end{macrocode}
%    The reference `LastPage' is marked used. If the reference
%    is not yet defined, then the user gets the warning because of
%    the undefined reference and the rerun warning at the end of
%    the compile run. However, we do not need a warning each page,
%    the first page is enough.
%    \begin{macrocode}
    \ifnum\value{abspage}=1 %
      \zref@refused{LastPage}%
    \fi
%    \end{macrocode}
%    The current absolute page number is compared with the
%    absolute page number of the reference `LastPage'.
%    \begin{macrocode}
    \ifnum\zref@extractdefault{LastPage}{abspage}{0}=\value{abspage}%
%    \end{macrocode}
%    \begin{macro}{\AtEndDvi@LastPage}
%    We found the right page and remember it in a macro.
%    \begin{macrocode}
      \xdef\AtEndDvi@LastPage{\number\value{abspage}}%
%    \end{macrocode}
%    \end{macro}
%    The hook of \cs{AtEndDvi} is now put on the last page
%    after the contents of the page.
%    \begin{macrocode}
      \global\setbox\AtBeginShipoutBox=\vbox{%
        \hbox{%
          \box\AtBeginShipoutBox
          \setbox\AtBeginShipoutBox=\hbox{%
            \begingroup
              \AtEndDvi@Hook
            \endgroup
          }%
          \wd\AtBeginShipoutBox=\z@
          \ht\AtBeginShipoutBox=\z@
          \dp\AtBeginShipoutBox=\z@
          \box\AtBeginShipoutBox
        }%
      }%
%    \end{macrocode}
%    We do not need the every page hook.
%    \begin{macrocode}
      \global\let\AtEndDvi@AtBeginShipout\@empty
%    \end{macrocode}
%    The hook is consumed, \cs{AtEndDvi} does not have an effect.
%    \begin{macrocode}
      \global\let\AtEndDvi\@gobble
%    \end{macrocode}
%    Make a protocol entry, which page is used by this package
%    as last page.
%    \begin{macrocode}
      \let\on@line\@empty
      \PackageInfo{atenddvi}{Last page = \AtEndDvi@LastPage}%
    \fi
  \endgroup
}
%    \end{macrocode}
%    \end{macro}
%
%    \begin{macro}{\AtEndDvi@AtBeginDocument}
%    In order to get as late as possible in the chain of the
%    every shipout hook, the call of \cs{AtBeginShipout} is delayed.
%    \begin{macrocode}
\def\AtEndDvi@AtBeginDocument{%
  \AtBeginShipout{\AtEndDvi@AtBeginShipout}%
%    \end{macrocode}
%    \begin{macro}{\AtEndDvi@Check}
%    After \cs{AtEndDocument} \LaTeX\ reads its \xfile{.aux} files
%    again. Code in \cs{AtEndDocument} could generate additional
%    pages. This is unlikely by code in the \xfile{.aux} file,
%    thus we use the \xfile{.aux} file to run macro
%    \cs{AtEndDvi@Check} for checking the last page.
%
%    During the first reading of the \xfile{.aux} file,
%    \cs{AtEndDvi@Check} is disabled, its real meaning
%    is assigned afterwards.
%    \begin{macrocode}
  \if@filesw
    \immediate\write\@mainaux{%
      \string\providecommand\string\AtEndDvi@Check{}%
    }%
    \immediate\write\@mainaux{%
      \string\AtEndDvi@Check
    }%
  \fi
  \let\AtEndDvi@Check\AtEndDvi@CheckImpl
}
%    \end{macrocode}
%    \end{macro}
%    \begin{macrocode}
\AtBeginDocument{\AtEndDvi@AtBeginDocument}
%    \end{macrocode}
%    \end{macro}
%
%    \begin{macro}{\AtEndDvi@CheckImpl}
%    First check is whether a last page was found at all.
%    Secondly the found last page is compared with the real last page.
%    \begin{macrocode}
\def\AtEndDvi@CheckImpl{%
  \@ifundefined{AtEndDvi@LastPage}{%
    \PackageWarningNoLine{atenddvi}{%
      Rerun LaTeX, last page not yet found%
    }%
  }{%
    \ifnum\AtEndDvi@LastPage=\value{abspage}%
    \else
      \PackageWarningNoLine{atenddvi}{%
        Rerun LaTeX, last page has changed%
      }%
    \fi
  }%
}
%    \end{macrocode}
%    \end{macro}
%
%    \begin{macrocode}
%</package>
%    \end{macrocode}
%
% \section{Installation}
%
% \subsection{Download}
%
% \paragraph{Package.} This package is available on
% CTAN\footnote{\CTANpkg{atenddvi}}:
% \begin{description}
% \item[\CTAN{macros/latex/contrib/oberdiek/atenddvi.dtx}] The source file.
% \item[\CTAN{macros/latex/contrib/oberdiek/atenddvi.pdf}] Documentation.
% \end{description}
%
%
% \paragraph{Bundle.} All the packages of the bundle `oberdiek'
% are also available in a TDS compliant ZIP archive. There
% the packages are already unpacked and the documentation files
% are generated. The files and directories obey the TDS standard.
% \begin{description}
% \item[\CTANinstall{install/macros/latex/contrib/oberdiek.tds.zip}]
% \end{description}
% \emph{TDS} refers to the standard ``A Directory Structure
% for \TeX\ Files'' (\CTAN{tds/tds.pdf}). Directories
% with \xfile{texmf} in their name are usually organized this way.
%
% \subsection{Bundle installation}
%
% \paragraph{Unpacking.} Unpack the \xfile{oberdiek.tds.zip} in the
% TDS tree (also known as \xfile{texmf} tree) of your choice.
% Example (linux):
% \begin{quote}
%   |unzip oberdiek.tds.zip -d ~/texmf|
% \end{quote}
%
% \subsection{Package installation}
%
% \paragraph{Unpacking.} The \xfile{.dtx} file is a self-extracting
% \docstrip\ archive. The files are extracted by running the
% \xfile{.dtx} through \plainTeX:
% \begin{quote}
%   \verb|tex atenddvi.dtx|
% \end{quote}
%
% \paragraph{TDS.} Now the different files must be moved into
% the different directories in your installation TDS tree
% (also known as \xfile{texmf} tree):
% \begin{quote}
% \def\t{^^A
% \begin{tabular}{@{}>{\ttfamily}l@{ $\rightarrow$ }>{\ttfamily}l@{}}
%   atenddvi.sty & tex/latex/oberdiek/atenddvi.sty\\
%   atenddvi.pdf & doc/latex/oberdiek/atenddvi.pdf\\
%   atenddvi.dtx & source/latex/oberdiek/atenddvi.dtx\\
% \end{tabular}^^A
% }^^A
% \sbox0{\t}^^A
% \ifdim\wd0>\linewidth
%   \begingroup
%     \advance\linewidth by\leftmargin
%     \advance\linewidth by\rightmargin
%   \edef\x{\endgroup
%     \def\noexpand\lw{\the\linewidth}^^A
%   }\x
%   \def\lwbox{^^A
%     \leavevmode
%     \hbox to \linewidth{^^A
%       \kern-\leftmargin\relax
%       \hss
%       \usebox0
%       \hss
%       \kern-\rightmargin\relax
%     }^^A
%   }^^A
%   \ifdim\wd0>\lw
%     \sbox0{\small\t}^^A
%     \ifdim\wd0>\linewidth
%       \ifdim\wd0>\lw
%         \sbox0{\footnotesize\t}^^A
%         \ifdim\wd0>\linewidth
%           \ifdim\wd0>\lw
%             \sbox0{\scriptsize\t}^^A
%             \ifdim\wd0>\linewidth
%               \ifdim\wd0>\lw
%                 \sbox0{\tiny\t}^^A
%                 \ifdim\wd0>\linewidth
%                   \lwbox
%                 \else
%                   \usebox0
%                 \fi
%               \else
%                 \lwbox
%               \fi
%             \else
%               \usebox0
%             \fi
%           \else
%             \lwbox
%           \fi
%         \else
%           \usebox0
%         \fi
%       \else
%         \lwbox
%       \fi
%     \else
%       \usebox0
%     \fi
%   \else
%     \lwbox
%   \fi
% \else
%   \usebox0
% \fi
% \end{quote}
% If you have a \xfile{docstrip.cfg} that configures and enables \docstrip's
% TDS installing feature, then some files can already be in the right
% place, see the documentation of \docstrip.
%
% \subsection{Refresh file name databases}
%
% If your \TeX~distribution
% (\TeX\,Live, \mikTeX, \dots) relies on file name databases, you must refresh
% these. For example, \TeX\,Live\ users run \verb|texhash| or
% \verb|mktexlsr|.
%
% \subsection{Some details for the interested}
%
% \paragraph{Unpacking with \LaTeX.}
% The \xfile{.dtx} chooses its action depending on the format:
% \begin{description}
% \item[\plainTeX:] Run \docstrip\ and extract the files.
% \item[\LaTeX:] Generate the documentation.
% \end{description}
% If you insist on using \LaTeX\ for \docstrip\ (really,
% \docstrip\ does not need \LaTeX), then inform the autodetect routine
% about your intention:
% \begin{quote}
%   \verb|latex \let\install=y% \iffalse meta-comment
%
% File: atenddvi.dtx
% Version: 2016/05/16 v1.2
% Info: At end DVI hook
%
% Copyright (C)
%    2007 Heiko Oberdiek
%    2016-2019 Oberdiek Package Support Group
%    https://github.com/ho-tex/oberdiek/issues
%
% This work may be distributed and/or modified under the
% conditions of the LaTeX Project Public License, either
% version 1.3c of this license or (at your option) any later
% version. This version of this license is in
%    https://www.latex-project.org/lppl/lppl-1-3c.txt
% and the latest version of this license is in
%    https://www.latex-project.org/lppl.txt
% and version 1.3 or later is part of all distributions of
% LaTeX version 2005/12/01 or later.
%
% This work has the LPPL maintenance status "maintained".
%
% The Current Maintainers of this work are
% Heiko Oberdiek and the Oberdiek Package Support Group
% https://github.com/ho-tex/oberdiek/issues
%
% This work consists of the main source file atenddvi.dtx
% and the derived files
%    atenddvi.sty, atenddvi.pdf, atenddvi.ins, atenddvi.drv.
%
% Distribution:
%    CTAN:macros/latex/contrib/oberdiek/atenddvi.dtx
%    CTAN:macros/latex/contrib/oberdiek/atenddvi.pdf
%
% Unpacking:
%    (a) If atenddvi.ins is present:
%           tex atenddvi.ins
%    (b) Without atenddvi.ins:
%           tex atenddvi.dtx
%    (c) If you insist on using LaTeX
%           latex \let\install=y% \iffalse meta-comment
%
% File: atenddvi.dtx
% Version: 2016/05/16 v1.2
% Info: At end DVI hook
%
% Copyright (C)
%    2007 Heiko Oberdiek
%    2016-2019 Oberdiek Package Support Group
%    https://github.com/ho-tex/oberdiek/issues
%
% This work may be distributed and/or modified under the
% conditions of the LaTeX Project Public License, either
% version 1.3c of this license or (at your option) any later
% version. This version of this license is in
%    https://www.latex-project.org/lppl/lppl-1-3c.txt
% and the latest version of this license is in
%    https://www.latex-project.org/lppl.txt
% and version 1.3 or later is part of all distributions of
% LaTeX version 2005/12/01 or later.
%
% This work has the LPPL maintenance status "maintained".
%
% The Current Maintainers of this work are
% Heiko Oberdiek and the Oberdiek Package Support Group
% https://github.com/ho-tex/oberdiek/issues
%
% This work consists of the main source file atenddvi.dtx
% and the derived files
%    atenddvi.sty, atenddvi.pdf, atenddvi.ins, atenddvi.drv.
%
% Distribution:
%    CTAN:macros/latex/contrib/oberdiek/atenddvi.dtx
%    CTAN:macros/latex/contrib/oberdiek/atenddvi.pdf
%
% Unpacking:
%    (a) If atenddvi.ins is present:
%           tex atenddvi.ins
%    (b) Without atenddvi.ins:
%           tex atenddvi.dtx
%    (c) If you insist on using LaTeX
%           latex \let\install=y\input{atenddvi.dtx}
%        (quote the arguments according to the demands of your shell)
%
% Documentation:
%    (a) If atenddvi.drv is present:
%           latex atenddvi.drv
%    (b) Without atenddvi.drv:
%           latex atenddvi.dtx; ...
%    The class ltxdoc loads the configuration file ltxdoc.cfg
%    if available. Here you can specify further options, e.g.
%    use A4 as paper format:
%       \PassOptionsToClass{a4paper}{article}
%
%    Programm calls to get the documentation (example):
%       pdflatex atenddvi.dtx
%       makeindex -s gind.ist atenddvi.idx
%       pdflatex atenddvi.dtx
%       makeindex -s gind.ist atenddvi.idx
%       pdflatex atenddvi.dtx
%
% Installation:
%    TDS:tex/latex/oberdiek/atenddvi.sty
%    TDS:doc/latex/oberdiek/atenddvi.pdf
%    TDS:source/latex/oberdiek/atenddvi.dtx
%
%<*ignore>
\begingroup
  \catcode123=1 %
  \catcode125=2 %
  \def\x{LaTeX2e}%
\expandafter\endgroup
\ifcase 0\ifx\install y1\fi\expandafter
         \ifx\csname processbatchFile\endcsname\relax\else1\fi
         \ifx\fmtname\x\else 1\fi\relax
\else\csname fi\endcsname
%</ignore>
%<*install>
\input docstrip.tex
\Msg{************************************************************************}
\Msg{* Installation}
\Msg{* Package: atenddvi 2016/05/16 v1.2 At end DVI hook (HO)}
\Msg{************************************************************************}

\keepsilent
\askforoverwritefalse

\let\MetaPrefix\relax
\preamble

This is a generated file.

Project: atenddvi
Version: 2016/05/16 v1.2

Copyright (C)
   2007 Heiko Oberdiek
   2016-2019 Oberdiek Package Support Group

This work may be distributed and/or modified under the
conditions of the LaTeX Project Public License, either
version 1.3c of this license or (at your option) any later
version. This version of this license is in
   https://www.latex-project.org/lppl/lppl-1-3c.txt
and the latest version of this license is in
   https://www.latex-project.org/lppl.txt
and version 1.3 or later is part of all distributions of
LaTeX version 2005/12/01 or later.

This work has the LPPL maintenance status "maintained".

The Current Maintainers of this work are
Heiko Oberdiek and the Oberdiek Package Support Group
https://github.com/ho-tex/oberdiek/issues


This work consists of the main source file atenddvi.dtx
and the derived files
   atenddvi.sty, atenddvi.pdf, atenddvi.ins, atenddvi.drv.

\endpreamble
\let\MetaPrefix\DoubleperCent

\generate{%
  \file{atenddvi.ins}{\from{atenddvi.dtx}{install}}%
  \file{atenddvi.drv}{\from{atenddvi.dtx}{driver}}%
  \usedir{tex/latex/oberdiek}%
  \file{atenddvi.sty}{\from{atenddvi.dtx}{package}}%
  \nopreamble
  \nopostamble
%  \usedir{source/latex/oberdiek/catalogue}%
%  \file{atenddvi.xml}{\from{atenddvi.dtx}{catalogue}}%
}

\catcode32=13\relax% active space
\let =\space%
\Msg{************************************************************************}
\Msg{*}
\Msg{* To finish the installation you have to move the following}
\Msg{* file into a directory searched by TeX:}
\Msg{*}
\Msg{*     atenddvi.sty}
\Msg{*}
\Msg{* To produce the documentation run the file `atenddvi.drv'}
\Msg{* through LaTeX.}
\Msg{*}
\Msg{* Happy TeXing!}
\Msg{*}
\Msg{************************************************************************}

\endbatchfile
%</install>
%<*ignore>
\fi
%</ignore>
%<*driver>
\NeedsTeXFormat{LaTeX2e}
\ProvidesFile{atenddvi.drv}%
  [2016/05/16 v1.2 At end DVI hook (HO)]%
\documentclass{ltxdoc}
\usepackage{holtxdoc}[2011/11/22]
\begin{document}
  \DocInput{atenddvi.dtx}%
\end{document}
%</driver>
% \fi
%
%
% \CharacterTable
%  {Upper-case    \A\B\C\D\E\F\G\H\I\J\K\L\M\N\O\P\Q\R\S\T\U\V\W\X\Y\Z
%   Lower-case    \a\b\c\d\e\f\g\h\i\j\k\l\m\n\o\p\q\r\s\t\u\v\w\x\y\z
%   Digits        \0\1\2\3\4\5\6\7\8\9
%   Exclamation   \!     Double quote  \"     Hash (number) \#
%   Dollar        \$     Percent       \%     Ampersand     \&
%   Acute accent  \'     Left paren    \(     Right paren   \)
%   Asterisk      \*     Plus          \+     Comma         \,
%   Minus         \-     Point         \.     Solidus       \/
%   Colon         \:     Semicolon     \;     Less than     \<
%   Equals        \=     Greater than  \>     Question mark \?
%   Commercial at \@     Left bracket  \[     Backslash     \\
%   Right bracket \]     Circumflex    \^     Underscore    \_
%   Grave accent  \`     Left brace    \{     Vertical bar  \|
%   Right brace   \}     Tilde         \~}
%
% \GetFileInfo{atenddvi.drv}
%
% \title{The \xpackage{atenddvi} package}
% \date{2016/05/16 v1.2}
% \author{Heiko Oberdiek\thanks
% {Please report any issues at \url{https://github.com/ho-tex/oberdiek/issues}}}
%
% \maketitle
%
% \begin{abstract}
% \LaTeX\ offers \cs{AtBeginDvi}. This package \xpackage{atenddvi}
% provides the counterpart \cs{AtEndDvi}. The execution of its
% argument is delayed to the end of the document at the end of the
% last page. Thus \cs{special} and \cs{write} remain effective, because
% they are put into the last page. This is the main difference
% to \cs{AtEndDocument}.
% \end{abstract}
%
% \tableofcontents
%
% \section{Documentation}
%
% \begin{declcs}{AtEndDvi} \M{code}
% \end{declcs}
% Macro \cs{AtEndDvi} provides a hook mechanism to put \meta{code}
% at the end of the last output page. It is the logical counterpart
% to \cs{AtBeginDvi}. Despite the name the output type DVI, PDF or whatever
% does not matter.
%
% Unlike \cs{AtBeginDvi} the \meta{code} is not put in a box and
% therefore executed immediately. The hook for \cs{AtEndDvi} is based on
% a macro similar to \cs{AtBeginDocument} or \cs{AtEndDocument}. The
% execution of \meta{code} is delayed until the hook is executed on
% the last page.
%
% Commands such as \cs{special} or \cs{write} (not the \cs{immediate}
% variant) must go as nodes into the contents of a page to have the
% desired effect.
% When the hook for \cs{AtEndDocument} is executed, the last intended
% page may already be shipped out. Therefore \cs{special} or \cs{write}
% cannot be used in a reliable way without generating new page.
%
% This gap is closed by \cs{AtEndDvi} of this package \xpackage{atenddvi}.
% If the document is compiled the first time, the package remembers
% the last page in a reference. In the sceond run, it puts the hook
% on the page that has been detected in the previous run as last page.
% The package detectes if the number of pages has changed, and then
% generates a warning to rerun \LaTeX.
%
% \StopEventually{
% }
%
% \section{Implementation}
%
%    \begin{macrocode}
%<*package>
\NeedsTeXFormat{LaTeX2e}
\ProvidesPackage{atenddvi}%
  [2016/05/16 v1.2 At end DVI hook (HO)]%
%    \end{macrocode}
%
%    Load the required packages
%    \begin{macrocode}
\RequirePackage{zref-abspage,zref-lastpage}[2007/03/19]
\RequirePackage{atbegshi}
%    \end{macrocode}
%
%    \begin{macro}{\AtEndDvi@Hook}
%    Macro \cs{AtEndDvi@Hook} is the data storage macro
%    for the code that is executed later at end of the last page.
%    \begin{macrocode}
\let\AtEndDvi@Hook\@empty
%    \end{macrocode}
%    \end{macro}
%    \begin{macro}{\AtEndDvi}
%    Macro \cs{AtEndDvi} is called in the same way as
%    \cs{AtBeginDocument}. The argument is added to the hook macro.
%    \begin{macrocode}
\newcommand*{\AtEndDvi}{%
  \g@addto@macro\AtEndDvi@Hook
}
%    \end{macrocode}
%    \end{macro}
%
%    \begin{macro}{\AtEndDvi@AtBeginShipout}
%    \begin{macrocode}
\def\AtEndDvi@AtBeginShipout{%
  \begingroup
%    \end{macrocode}
%    The reference `LastPage' is marked used. If the reference
%    is not yet defined, then the user gets the warning because of
%    the undefined reference and the rerun warning at the end of
%    the compile run. However, we do not need a warning each page,
%    the first page is enough.
%    \begin{macrocode}
    \ifnum\value{abspage}=1 %
      \zref@refused{LastPage}%
    \fi
%    \end{macrocode}
%    The current absolute page number is compared with the
%    absolute page number of the reference `LastPage'.
%    \begin{macrocode}
    \ifnum\zref@extractdefault{LastPage}{abspage}{0}=\value{abspage}%
%    \end{macrocode}
%    \begin{macro}{\AtEndDvi@LastPage}
%    We found the right page and remember it in a macro.
%    \begin{macrocode}
      \xdef\AtEndDvi@LastPage{\number\value{abspage}}%
%    \end{macrocode}
%    \end{macro}
%    The hook of \cs{AtEndDvi} is now put on the last page
%    after the contents of the page.
%    \begin{macrocode}
      \global\setbox\AtBeginShipoutBox=\vbox{%
        \hbox{%
          \box\AtBeginShipoutBox
          \setbox\AtBeginShipoutBox=\hbox{%
            \begingroup
              \AtEndDvi@Hook
            \endgroup
          }%
          \wd\AtBeginShipoutBox=\z@
          \ht\AtBeginShipoutBox=\z@
          \dp\AtBeginShipoutBox=\z@
          \box\AtBeginShipoutBox
        }%
      }%
%    \end{macrocode}
%    We do not need the every page hook.
%    \begin{macrocode}
      \global\let\AtEndDvi@AtBeginShipout\@empty
%    \end{macrocode}
%    The hook is consumed, \cs{AtEndDvi} does not have an effect.
%    \begin{macrocode}
      \global\let\AtEndDvi\@gobble
%    \end{macrocode}
%    Make a protocol entry, which page is used by this package
%    as last page.
%    \begin{macrocode}
      \let\on@line\@empty
      \PackageInfo{atenddvi}{Last page = \AtEndDvi@LastPage}%
    \fi
  \endgroup
}
%    \end{macrocode}
%    \end{macro}
%
%    \begin{macro}{\AtEndDvi@AtBeginDocument}
%    In order to get as late as possible in the chain of the
%    every shipout hook, the call of \cs{AtBeginShipout} is delayed.
%    \begin{macrocode}
\def\AtEndDvi@AtBeginDocument{%
  \AtBeginShipout{\AtEndDvi@AtBeginShipout}%
%    \end{macrocode}
%    \begin{macro}{\AtEndDvi@Check}
%    After \cs{AtEndDocument} \LaTeX\ reads its \xfile{.aux} files
%    again. Code in \cs{AtEndDocument} could generate additional
%    pages. This is unlikely by code in the \xfile{.aux} file,
%    thus we use the \xfile{.aux} file to run macro
%    \cs{AtEndDvi@Check} for checking the last page.
%
%    During the first reading of the \xfile{.aux} file,
%    \cs{AtEndDvi@Check} is disabled, its real meaning
%    is assigned afterwards.
%    \begin{macrocode}
  \if@filesw
    \immediate\write\@mainaux{%
      \string\providecommand\string\AtEndDvi@Check{}%
    }%
    \immediate\write\@mainaux{%
      \string\AtEndDvi@Check
    }%
  \fi
  \let\AtEndDvi@Check\AtEndDvi@CheckImpl
}
%    \end{macrocode}
%    \end{macro}
%    \begin{macrocode}
\AtBeginDocument{\AtEndDvi@AtBeginDocument}
%    \end{macrocode}
%    \end{macro}
%
%    \begin{macro}{\AtEndDvi@CheckImpl}
%    First check is whether a last page was found at all.
%    Secondly the found last page is compared with the real last page.
%    \begin{macrocode}
\def\AtEndDvi@CheckImpl{%
  \@ifundefined{AtEndDvi@LastPage}{%
    \PackageWarningNoLine{atenddvi}{%
      Rerun LaTeX, last page not yet found%
    }%
  }{%
    \ifnum\AtEndDvi@LastPage=\value{abspage}%
    \else
      \PackageWarningNoLine{atenddvi}{%
        Rerun LaTeX, last page has changed%
      }%
    \fi
  }%
}
%    \end{macrocode}
%    \end{macro}
%
%    \begin{macrocode}
%</package>
%    \end{macrocode}
%
% \section{Installation}
%
% \subsection{Download}
%
% \paragraph{Package.} This package is available on
% CTAN\footnote{\CTANpkg{atenddvi}}:
% \begin{description}
% \item[\CTAN{macros/latex/contrib/oberdiek/atenddvi.dtx}] The source file.
% \item[\CTAN{macros/latex/contrib/oberdiek/atenddvi.pdf}] Documentation.
% \end{description}
%
%
% \paragraph{Bundle.} All the packages of the bundle `oberdiek'
% are also available in a TDS compliant ZIP archive. There
% the packages are already unpacked and the documentation files
% are generated. The files and directories obey the TDS standard.
% \begin{description}
% \item[\CTANinstall{install/macros/latex/contrib/oberdiek.tds.zip}]
% \end{description}
% \emph{TDS} refers to the standard ``A Directory Structure
% for \TeX\ Files'' (\CTAN{tds/tds.pdf}). Directories
% with \xfile{texmf} in their name are usually organized this way.
%
% \subsection{Bundle installation}
%
% \paragraph{Unpacking.} Unpack the \xfile{oberdiek.tds.zip} in the
% TDS tree (also known as \xfile{texmf} tree) of your choice.
% Example (linux):
% \begin{quote}
%   |unzip oberdiek.tds.zip -d ~/texmf|
% \end{quote}
%
% \subsection{Package installation}
%
% \paragraph{Unpacking.} The \xfile{.dtx} file is a self-extracting
% \docstrip\ archive. The files are extracted by running the
% \xfile{.dtx} through \plainTeX:
% \begin{quote}
%   \verb|tex atenddvi.dtx|
% \end{quote}
%
% \paragraph{TDS.} Now the different files must be moved into
% the different directories in your installation TDS tree
% (also known as \xfile{texmf} tree):
% \begin{quote}
% \def\t{^^A
% \begin{tabular}{@{}>{\ttfamily}l@{ $\rightarrow$ }>{\ttfamily}l@{}}
%   atenddvi.sty & tex/latex/oberdiek/atenddvi.sty\\
%   atenddvi.pdf & doc/latex/oberdiek/atenddvi.pdf\\
%   atenddvi.dtx & source/latex/oberdiek/atenddvi.dtx\\
% \end{tabular}^^A
% }^^A
% \sbox0{\t}^^A
% \ifdim\wd0>\linewidth
%   \begingroup
%     \advance\linewidth by\leftmargin
%     \advance\linewidth by\rightmargin
%   \edef\x{\endgroup
%     \def\noexpand\lw{\the\linewidth}^^A
%   }\x
%   \def\lwbox{^^A
%     \leavevmode
%     \hbox to \linewidth{^^A
%       \kern-\leftmargin\relax
%       \hss
%       \usebox0
%       \hss
%       \kern-\rightmargin\relax
%     }^^A
%   }^^A
%   \ifdim\wd0>\lw
%     \sbox0{\small\t}^^A
%     \ifdim\wd0>\linewidth
%       \ifdim\wd0>\lw
%         \sbox0{\footnotesize\t}^^A
%         \ifdim\wd0>\linewidth
%           \ifdim\wd0>\lw
%             \sbox0{\scriptsize\t}^^A
%             \ifdim\wd0>\linewidth
%               \ifdim\wd0>\lw
%                 \sbox0{\tiny\t}^^A
%                 \ifdim\wd0>\linewidth
%                   \lwbox
%                 \else
%                   \usebox0
%                 \fi
%               \else
%                 \lwbox
%               \fi
%             \else
%               \usebox0
%             \fi
%           \else
%             \lwbox
%           \fi
%         \else
%           \usebox0
%         \fi
%       \else
%         \lwbox
%       \fi
%     \else
%       \usebox0
%     \fi
%   \else
%     \lwbox
%   \fi
% \else
%   \usebox0
% \fi
% \end{quote}
% If you have a \xfile{docstrip.cfg} that configures and enables \docstrip's
% TDS installing feature, then some files can already be in the right
% place, see the documentation of \docstrip.
%
% \subsection{Refresh file name databases}
%
% If your \TeX~distribution
% (\TeX\,Live, \mikTeX, \dots) relies on file name databases, you must refresh
% these. For example, \TeX\,Live\ users run \verb|texhash| or
% \verb|mktexlsr|.
%
% \subsection{Some details for the interested}
%
% \paragraph{Unpacking with \LaTeX.}
% The \xfile{.dtx} chooses its action depending on the format:
% \begin{description}
% \item[\plainTeX:] Run \docstrip\ and extract the files.
% \item[\LaTeX:] Generate the documentation.
% \end{description}
% If you insist on using \LaTeX\ for \docstrip\ (really,
% \docstrip\ does not need \LaTeX), then inform the autodetect routine
% about your intention:
% \begin{quote}
%   \verb|latex \let\install=y\input{atenddvi.dtx}|
% \end{quote}
% Do not forget to quote the argument according to the demands
% of your shell.
%
% \paragraph{Generating the documentation.}
% You can use both the \xfile{.dtx} or the \xfile{.drv} to generate
% the documentation. The process can be configured by the
% configuration file \xfile{ltxdoc.cfg}. For instance, put this
% line into this file, if you want to have A4 as paper format:
% \begin{quote}
%   \verb|\PassOptionsToClass{a4paper}{article}|
% \end{quote}
% An example follows how to generate the
% documentation with pdf\LaTeX:
% \begin{quote}
%\begin{verbatim}
%pdflatex atenddvi.dtx
%makeindex -s gind.ist atenddvi.idx
%pdflatex atenddvi.dtx
%makeindex -s gind.ist atenddvi.idx
%pdflatex atenddvi.dtx
%\end{verbatim}
% \end{quote}
%
% \begin{History}
%   \begin{Version}{2007/03/20 v1.0}
%   \item
%     First version.
%   \end{Version}
%   \begin{Version}{2007/04/17 v1.1}
%   \item
%     Package \xpackage{atbegshi} replaces package \xpackage{everyshi}.
%   \end{Version}
%   \begin{Version}{2016/05/16 v1.2}
%   \item
%     Documentation updates.
%   \end{Version}
% \end{History}
%
% \PrintIndex
%
% \Finale
\endinput

%        (quote the arguments according to the demands of your shell)
%
% Documentation:
%    (a) If atenddvi.drv is present:
%           latex atenddvi.drv
%    (b) Without atenddvi.drv:
%           latex atenddvi.dtx; ...
%    The class ltxdoc loads the configuration file ltxdoc.cfg
%    if available. Here you can specify further options, e.g.
%    use A4 as paper format:
%       \PassOptionsToClass{a4paper}{article}
%
%    Programm calls to get the documentation (example):
%       pdflatex atenddvi.dtx
%       makeindex -s gind.ist atenddvi.idx
%       pdflatex atenddvi.dtx
%       makeindex -s gind.ist atenddvi.idx
%       pdflatex atenddvi.dtx
%
% Installation:
%    TDS:tex/latex/oberdiek/atenddvi.sty
%    TDS:doc/latex/oberdiek/atenddvi.pdf
%    TDS:source/latex/oberdiek/atenddvi.dtx
%
%<*ignore>
\begingroup
  \catcode123=1 %
  \catcode125=2 %
  \def\x{LaTeX2e}%
\expandafter\endgroup
\ifcase 0\ifx\install y1\fi\expandafter
         \ifx\csname processbatchFile\endcsname\relax\else1\fi
         \ifx\fmtname\x\else 1\fi\relax
\else\csname fi\endcsname
%</ignore>
%<*install>
\input docstrip.tex
\Msg{************************************************************************}
\Msg{* Installation}
\Msg{* Package: atenddvi 2016/05/16 v1.2 At end DVI hook (HO)}
\Msg{************************************************************************}

\keepsilent
\askforoverwritefalse

\let\MetaPrefix\relax
\preamble

This is a generated file.

Project: atenddvi
Version: 2016/05/16 v1.2

Copyright (C)
   2007 Heiko Oberdiek
   2016-2019 Oberdiek Package Support Group

This work may be distributed and/or modified under the
conditions of the LaTeX Project Public License, either
version 1.3c of this license or (at your option) any later
version. This version of this license is in
   https://www.latex-project.org/lppl/lppl-1-3c.txt
and the latest version of this license is in
   https://www.latex-project.org/lppl.txt
and version 1.3 or later is part of all distributions of
LaTeX version 2005/12/01 or later.

This work has the LPPL maintenance status "maintained".

The Current Maintainers of this work are
Heiko Oberdiek and the Oberdiek Package Support Group
https://github.com/ho-tex/oberdiek/issues


This work consists of the main source file atenddvi.dtx
and the derived files
   atenddvi.sty, atenddvi.pdf, atenddvi.ins, atenddvi.drv.

\endpreamble
\let\MetaPrefix\DoubleperCent

\generate{%
  \file{atenddvi.ins}{\from{atenddvi.dtx}{install}}%
  \file{atenddvi.drv}{\from{atenddvi.dtx}{driver}}%
  \usedir{tex/latex/oberdiek}%
  \file{atenddvi.sty}{\from{atenddvi.dtx}{package}}%
  \nopreamble
  \nopostamble
%  \usedir{source/latex/oberdiek/catalogue}%
%  \file{atenddvi.xml}{\from{atenddvi.dtx}{catalogue}}%
}

\catcode32=13\relax% active space
\let =\space%
\Msg{************************************************************************}
\Msg{*}
\Msg{* To finish the installation you have to move the following}
\Msg{* file into a directory searched by TeX:}
\Msg{*}
\Msg{*     atenddvi.sty}
\Msg{*}
\Msg{* To produce the documentation run the file `atenddvi.drv'}
\Msg{* through LaTeX.}
\Msg{*}
\Msg{* Happy TeXing!}
\Msg{*}
\Msg{************************************************************************}

\endbatchfile
%</install>
%<*ignore>
\fi
%</ignore>
%<*driver>
\NeedsTeXFormat{LaTeX2e}
\ProvidesFile{atenddvi.drv}%
  [2016/05/16 v1.2 At end DVI hook (HO)]%
\documentclass{ltxdoc}
\usepackage{holtxdoc}[2011/11/22]
\begin{document}
  \DocInput{atenddvi.dtx}%
\end{document}
%</driver>
% \fi
%
%
% \CharacterTable
%  {Upper-case    \A\B\C\D\E\F\G\H\I\J\K\L\M\N\O\P\Q\R\S\T\U\V\W\X\Y\Z
%   Lower-case    \a\b\c\d\e\f\g\h\i\j\k\l\m\n\o\p\q\r\s\t\u\v\w\x\y\z
%   Digits        \0\1\2\3\4\5\6\7\8\9
%   Exclamation   \!     Double quote  \"     Hash (number) \#
%   Dollar        \$     Percent       \%     Ampersand     \&
%   Acute accent  \'     Left paren    \(     Right paren   \)
%   Asterisk      \*     Plus          \+     Comma         \,
%   Minus         \-     Point         \.     Solidus       \/
%   Colon         \:     Semicolon     \;     Less than     \<
%   Equals        \=     Greater than  \>     Question mark \?
%   Commercial at \@     Left bracket  \[     Backslash     \\
%   Right bracket \]     Circumflex    \^     Underscore    \_
%   Grave accent  \`     Left brace    \{     Vertical bar  \|
%   Right brace   \}     Tilde         \~}
%
% \GetFileInfo{atenddvi.drv}
%
% \title{The \xpackage{atenddvi} package}
% \date{2016/05/16 v1.2}
% \author{Heiko Oberdiek\thanks
% {Please report any issues at \url{https://github.com/ho-tex/oberdiek/issues}}}
%
% \maketitle
%
% \begin{abstract}
% \LaTeX\ offers \cs{AtBeginDvi}. This package \xpackage{atenddvi}
% provides the counterpart \cs{AtEndDvi}. The execution of its
% argument is delayed to the end of the document at the end of the
% last page. Thus \cs{special} and \cs{write} remain effective, because
% they are put into the last page. This is the main difference
% to \cs{AtEndDocument}.
% \end{abstract}
%
% \tableofcontents
%
% \section{Documentation}
%
% \begin{declcs}{AtEndDvi} \M{code}
% \end{declcs}
% Macro \cs{AtEndDvi} provides a hook mechanism to put \meta{code}
% at the end of the last output page. It is the logical counterpart
% to \cs{AtBeginDvi}. Despite the name the output type DVI, PDF or whatever
% does not matter.
%
% Unlike \cs{AtBeginDvi} the \meta{code} is not put in a box and
% therefore executed immediately. The hook for \cs{AtEndDvi} is based on
% a macro similar to \cs{AtBeginDocument} or \cs{AtEndDocument}. The
% execution of \meta{code} is delayed until the hook is executed on
% the last page.
%
% Commands such as \cs{special} or \cs{write} (not the \cs{immediate}
% variant) must go as nodes into the contents of a page to have the
% desired effect.
% When the hook for \cs{AtEndDocument} is executed, the last intended
% page may already be shipped out. Therefore \cs{special} or \cs{write}
% cannot be used in a reliable way without generating new page.
%
% This gap is closed by \cs{AtEndDvi} of this package \xpackage{atenddvi}.
% If the document is compiled the first time, the package remembers
% the last page in a reference. In the sceond run, it puts the hook
% on the page that has been detected in the previous run as last page.
% The package detectes if the number of pages has changed, and then
% generates a warning to rerun \LaTeX.
%
% \StopEventually{
% }
%
% \section{Implementation}
%
%    \begin{macrocode}
%<*package>
\NeedsTeXFormat{LaTeX2e}
\ProvidesPackage{atenddvi}%
  [2016/05/16 v1.2 At end DVI hook (HO)]%
%    \end{macrocode}
%
%    Load the required packages
%    \begin{macrocode}
\RequirePackage{zref-abspage,zref-lastpage}[2007/03/19]
\RequirePackage{atbegshi}
%    \end{macrocode}
%
%    \begin{macro}{\AtEndDvi@Hook}
%    Macro \cs{AtEndDvi@Hook} is the data storage macro
%    for the code that is executed later at end of the last page.
%    \begin{macrocode}
\let\AtEndDvi@Hook\@empty
%    \end{macrocode}
%    \end{macro}
%    \begin{macro}{\AtEndDvi}
%    Macro \cs{AtEndDvi} is called in the same way as
%    \cs{AtBeginDocument}. The argument is added to the hook macro.
%    \begin{macrocode}
\newcommand*{\AtEndDvi}{%
  \g@addto@macro\AtEndDvi@Hook
}
%    \end{macrocode}
%    \end{macro}
%
%    \begin{macro}{\AtEndDvi@AtBeginShipout}
%    \begin{macrocode}
\def\AtEndDvi@AtBeginShipout{%
  \begingroup
%    \end{macrocode}
%    The reference `LastPage' is marked used. If the reference
%    is not yet defined, then the user gets the warning because of
%    the undefined reference and the rerun warning at the end of
%    the compile run. However, we do not need a warning each page,
%    the first page is enough.
%    \begin{macrocode}
    \ifnum\value{abspage}=1 %
      \zref@refused{LastPage}%
    \fi
%    \end{macrocode}
%    The current absolute page number is compared with the
%    absolute page number of the reference `LastPage'.
%    \begin{macrocode}
    \ifnum\zref@extractdefault{LastPage}{abspage}{0}=\value{abspage}%
%    \end{macrocode}
%    \begin{macro}{\AtEndDvi@LastPage}
%    We found the right page and remember it in a macro.
%    \begin{macrocode}
      \xdef\AtEndDvi@LastPage{\number\value{abspage}}%
%    \end{macrocode}
%    \end{macro}
%    The hook of \cs{AtEndDvi} is now put on the last page
%    after the contents of the page.
%    \begin{macrocode}
      \global\setbox\AtBeginShipoutBox=\vbox{%
        \hbox{%
          \box\AtBeginShipoutBox
          \setbox\AtBeginShipoutBox=\hbox{%
            \begingroup
              \AtEndDvi@Hook
            \endgroup
          }%
          \wd\AtBeginShipoutBox=\z@
          \ht\AtBeginShipoutBox=\z@
          \dp\AtBeginShipoutBox=\z@
          \box\AtBeginShipoutBox
        }%
      }%
%    \end{macrocode}
%    We do not need the every page hook.
%    \begin{macrocode}
      \global\let\AtEndDvi@AtBeginShipout\@empty
%    \end{macrocode}
%    The hook is consumed, \cs{AtEndDvi} does not have an effect.
%    \begin{macrocode}
      \global\let\AtEndDvi\@gobble
%    \end{macrocode}
%    Make a protocol entry, which page is used by this package
%    as last page.
%    \begin{macrocode}
      \let\on@line\@empty
      \PackageInfo{atenddvi}{Last page = \AtEndDvi@LastPage}%
    \fi
  \endgroup
}
%    \end{macrocode}
%    \end{macro}
%
%    \begin{macro}{\AtEndDvi@AtBeginDocument}
%    In order to get as late as possible in the chain of the
%    every shipout hook, the call of \cs{AtBeginShipout} is delayed.
%    \begin{macrocode}
\def\AtEndDvi@AtBeginDocument{%
  \AtBeginShipout{\AtEndDvi@AtBeginShipout}%
%    \end{macrocode}
%    \begin{macro}{\AtEndDvi@Check}
%    After \cs{AtEndDocument} \LaTeX\ reads its \xfile{.aux} files
%    again. Code in \cs{AtEndDocument} could generate additional
%    pages. This is unlikely by code in the \xfile{.aux} file,
%    thus we use the \xfile{.aux} file to run macro
%    \cs{AtEndDvi@Check} for checking the last page.
%
%    During the first reading of the \xfile{.aux} file,
%    \cs{AtEndDvi@Check} is disabled, its real meaning
%    is assigned afterwards.
%    \begin{macrocode}
  \if@filesw
    \immediate\write\@mainaux{%
      \string\providecommand\string\AtEndDvi@Check{}%
    }%
    \immediate\write\@mainaux{%
      \string\AtEndDvi@Check
    }%
  \fi
  \let\AtEndDvi@Check\AtEndDvi@CheckImpl
}
%    \end{macrocode}
%    \end{macro}
%    \begin{macrocode}
\AtBeginDocument{\AtEndDvi@AtBeginDocument}
%    \end{macrocode}
%    \end{macro}
%
%    \begin{macro}{\AtEndDvi@CheckImpl}
%    First check is whether a last page was found at all.
%    Secondly the found last page is compared with the real last page.
%    \begin{macrocode}
\def\AtEndDvi@CheckImpl{%
  \@ifundefined{AtEndDvi@LastPage}{%
    \PackageWarningNoLine{atenddvi}{%
      Rerun LaTeX, last page not yet found%
    }%
  }{%
    \ifnum\AtEndDvi@LastPage=\value{abspage}%
    \else
      \PackageWarningNoLine{atenddvi}{%
        Rerun LaTeX, last page has changed%
      }%
    \fi
  }%
}
%    \end{macrocode}
%    \end{macro}
%
%    \begin{macrocode}
%</package>
%    \end{macrocode}
%
% \section{Installation}
%
% \subsection{Download}
%
% \paragraph{Package.} This package is available on
% CTAN\footnote{\CTANpkg{atenddvi}}:
% \begin{description}
% \item[\CTAN{macros/latex/contrib/oberdiek/atenddvi.dtx}] The source file.
% \item[\CTAN{macros/latex/contrib/oberdiek/atenddvi.pdf}] Documentation.
% \end{description}
%
%
% \paragraph{Bundle.} All the packages of the bundle `oberdiek'
% are also available in a TDS compliant ZIP archive. There
% the packages are already unpacked and the documentation files
% are generated. The files and directories obey the TDS standard.
% \begin{description}
% \item[\CTANinstall{install/macros/latex/contrib/oberdiek.tds.zip}]
% \end{description}
% \emph{TDS} refers to the standard ``A Directory Structure
% for \TeX\ Files'' (\CTAN{tds/tds.pdf}). Directories
% with \xfile{texmf} in their name are usually organized this way.
%
% \subsection{Bundle installation}
%
% \paragraph{Unpacking.} Unpack the \xfile{oberdiek.tds.zip} in the
% TDS tree (also known as \xfile{texmf} tree) of your choice.
% Example (linux):
% \begin{quote}
%   |unzip oberdiek.tds.zip -d ~/texmf|
% \end{quote}
%
% \subsection{Package installation}
%
% \paragraph{Unpacking.} The \xfile{.dtx} file is a self-extracting
% \docstrip\ archive. The files are extracted by running the
% \xfile{.dtx} through \plainTeX:
% \begin{quote}
%   \verb|tex atenddvi.dtx|
% \end{quote}
%
% \paragraph{TDS.} Now the different files must be moved into
% the different directories in your installation TDS tree
% (also known as \xfile{texmf} tree):
% \begin{quote}
% \def\t{^^A
% \begin{tabular}{@{}>{\ttfamily}l@{ $\rightarrow$ }>{\ttfamily}l@{}}
%   atenddvi.sty & tex/latex/oberdiek/atenddvi.sty\\
%   atenddvi.pdf & doc/latex/oberdiek/atenddvi.pdf\\
%   atenddvi.dtx & source/latex/oberdiek/atenddvi.dtx\\
% \end{tabular}^^A
% }^^A
% \sbox0{\t}^^A
% \ifdim\wd0>\linewidth
%   \begingroup
%     \advance\linewidth by\leftmargin
%     \advance\linewidth by\rightmargin
%   \edef\x{\endgroup
%     \def\noexpand\lw{\the\linewidth}^^A
%   }\x
%   \def\lwbox{^^A
%     \leavevmode
%     \hbox to \linewidth{^^A
%       \kern-\leftmargin\relax
%       \hss
%       \usebox0
%       \hss
%       \kern-\rightmargin\relax
%     }^^A
%   }^^A
%   \ifdim\wd0>\lw
%     \sbox0{\small\t}^^A
%     \ifdim\wd0>\linewidth
%       \ifdim\wd0>\lw
%         \sbox0{\footnotesize\t}^^A
%         \ifdim\wd0>\linewidth
%           \ifdim\wd0>\lw
%             \sbox0{\scriptsize\t}^^A
%             \ifdim\wd0>\linewidth
%               \ifdim\wd0>\lw
%                 \sbox0{\tiny\t}^^A
%                 \ifdim\wd0>\linewidth
%                   \lwbox
%                 \else
%                   \usebox0
%                 \fi
%               \else
%                 \lwbox
%               \fi
%             \else
%               \usebox0
%             \fi
%           \else
%             \lwbox
%           \fi
%         \else
%           \usebox0
%         \fi
%       \else
%         \lwbox
%       \fi
%     \else
%       \usebox0
%     \fi
%   \else
%     \lwbox
%   \fi
% \else
%   \usebox0
% \fi
% \end{quote}
% If you have a \xfile{docstrip.cfg} that configures and enables \docstrip's
% TDS installing feature, then some files can already be in the right
% place, see the documentation of \docstrip.
%
% \subsection{Refresh file name databases}
%
% If your \TeX~distribution
% (\TeX\,Live, \mikTeX, \dots) relies on file name databases, you must refresh
% these. For example, \TeX\,Live\ users run \verb|texhash| or
% \verb|mktexlsr|.
%
% \subsection{Some details for the interested}
%
% \paragraph{Unpacking with \LaTeX.}
% The \xfile{.dtx} chooses its action depending on the format:
% \begin{description}
% \item[\plainTeX:] Run \docstrip\ and extract the files.
% \item[\LaTeX:] Generate the documentation.
% \end{description}
% If you insist on using \LaTeX\ for \docstrip\ (really,
% \docstrip\ does not need \LaTeX), then inform the autodetect routine
% about your intention:
% \begin{quote}
%   \verb|latex \let\install=y% \iffalse meta-comment
%
% File: atenddvi.dtx
% Version: 2016/05/16 v1.2
% Info: At end DVI hook
%
% Copyright (C)
%    2007 Heiko Oberdiek
%    2016-2019 Oberdiek Package Support Group
%    https://github.com/ho-tex/oberdiek/issues
%
% This work may be distributed and/or modified under the
% conditions of the LaTeX Project Public License, either
% version 1.3c of this license or (at your option) any later
% version. This version of this license is in
%    https://www.latex-project.org/lppl/lppl-1-3c.txt
% and the latest version of this license is in
%    https://www.latex-project.org/lppl.txt
% and version 1.3 or later is part of all distributions of
% LaTeX version 2005/12/01 or later.
%
% This work has the LPPL maintenance status "maintained".
%
% The Current Maintainers of this work are
% Heiko Oberdiek and the Oberdiek Package Support Group
% https://github.com/ho-tex/oberdiek/issues
%
% This work consists of the main source file atenddvi.dtx
% and the derived files
%    atenddvi.sty, atenddvi.pdf, atenddvi.ins, atenddvi.drv.
%
% Distribution:
%    CTAN:macros/latex/contrib/oberdiek/atenddvi.dtx
%    CTAN:macros/latex/contrib/oberdiek/atenddvi.pdf
%
% Unpacking:
%    (a) If atenddvi.ins is present:
%           tex atenddvi.ins
%    (b) Without atenddvi.ins:
%           tex atenddvi.dtx
%    (c) If you insist on using LaTeX
%           latex \let\install=y\input{atenddvi.dtx}
%        (quote the arguments according to the demands of your shell)
%
% Documentation:
%    (a) If atenddvi.drv is present:
%           latex atenddvi.drv
%    (b) Without atenddvi.drv:
%           latex atenddvi.dtx; ...
%    The class ltxdoc loads the configuration file ltxdoc.cfg
%    if available. Here you can specify further options, e.g.
%    use A4 as paper format:
%       \PassOptionsToClass{a4paper}{article}
%
%    Programm calls to get the documentation (example):
%       pdflatex atenddvi.dtx
%       makeindex -s gind.ist atenddvi.idx
%       pdflatex atenddvi.dtx
%       makeindex -s gind.ist atenddvi.idx
%       pdflatex atenddvi.dtx
%
% Installation:
%    TDS:tex/latex/oberdiek/atenddvi.sty
%    TDS:doc/latex/oberdiek/atenddvi.pdf
%    TDS:source/latex/oberdiek/atenddvi.dtx
%
%<*ignore>
\begingroup
  \catcode123=1 %
  \catcode125=2 %
  \def\x{LaTeX2e}%
\expandafter\endgroup
\ifcase 0\ifx\install y1\fi\expandafter
         \ifx\csname processbatchFile\endcsname\relax\else1\fi
         \ifx\fmtname\x\else 1\fi\relax
\else\csname fi\endcsname
%</ignore>
%<*install>
\input docstrip.tex
\Msg{************************************************************************}
\Msg{* Installation}
\Msg{* Package: atenddvi 2016/05/16 v1.2 At end DVI hook (HO)}
\Msg{************************************************************************}

\keepsilent
\askforoverwritefalse

\let\MetaPrefix\relax
\preamble

This is a generated file.

Project: atenddvi
Version: 2016/05/16 v1.2

Copyright (C)
   2007 Heiko Oberdiek
   2016-2019 Oberdiek Package Support Group

This work may be distributed and/or modified under the
conditions of the LaTeX Project Public License, either
version 1.3c of this license or (at your option) any later
version. This version of this license is in
   https://www.latex-project.org/lppl/lppl-1-3c.txt
and the latest version of this license is in
   https://www.latex-project.org/lppl.txt
and version 1.3 or later is part of all distributions of
LaTeX version 2005/12/01 or later.

This work has the LPPL maintenance status "maintained".

The Current Maintainers of this work are
Heiko Oberdiek and the Oberdiek Package Support Group
https://github.com/ho-tex/oberdiek/issues


This work consists of the main source file atenddvi.dtx
and the derived files
   atenddvi.sty, atenddvi.pdf, atenddvi.ins, atenddvi.drv.

\endpreamble
\let\MetaPrefix\DoubleperCent

\generate{%
  \file{atenddvi.ins}{\from{atenddvi.dtx}{install}}%
  \file{atenddvi.drv}{\from{atenddvi.dtx}{driver}}%
  \usedir{tex/latex/oberdiek}%
  \file{atenddvi.sty}{\from{atenddvi.dtx}{package}}%
  \nopreamble
  \nopostamble
%  \usedir{source/latex/oberdiek/catalogue}%
%  \file{atenddvi.xml}{\from{atenddvi.dtx}{catalogue}}%
}

\catcode32=13\relax% active space
\let =\space%
\Msg{************************************************************************}
\Msg{*}
\Msg{* To finish the installation you have to move the following}
\Msg{* file into a directory searched by TeX:}
\Msg{*}
\Msg{*     atenddvi.sty}
\Msg{*}
\Msg{* To produce the documentation run the file `atenddvi.drv'}
\Msg{* through LaTeX.}
\Msg{*}
\Msg{* Happy TeXing!}
\Msg{*}
\Msg{************************************************************************}

\endbatchfile
%</install>
%<*ignore>
\fi
%</ignore>
%<*driver>
\NeedsTeXFormat{LaTeX2e}
\ProvidesFile{atenddvi.drv}%
  [2016/05/16 v1.2 At end DVI hook (HO)]%
\documentclass{ltxdoc}
\usepackage{holtxdoc}[2011/11/22]
\begin{document}
  \DocInput{atenddvi.dtx}%
\end{document}
%</driver>
% \fi
%
%
% \CharacterTable
%  {Upper-case    \A\B\C\D\E\F\G\H\I\J\K\L\M\N\O\P\Q\R\S\T\U\V\W\X\Y\Z
%   Lower-case    \a\b\c\d\e\f\g\h\i\j\k\l\m\n\o\p\q\r\s\t\u\v\w\x\y\z
%   Digits        \0\1\2\3\4\5\6\7\8\9
%   Exclamation   \!     Double quote  \"     Hash (number) \#
%   Dollar        \$     Percent       \%     Ampersand     \&
%   Acute accent  \'     Left paren    \(     Right paren   \)
%   Asterisk      \*     Plus          \+     Comma         \,
%   Minus         \-     Point         \.     Solidus       \/
%   Colon         \:     Semicolon     \;     Less than     \<
%   Equals        \=     Greater than  \>     Question mark \?
%   Commercial at \@     Left bracket  \[     Backslash     \\
%   Right bracket \]     Circumflex    \^     Underscore    \_
%   Grave accent  \`     Left brace    \{     Vertical bar  \|
%   Right brace   \}     Tilde         \~}
%
% \GetFileInfo{atenddvi.drv}
%
% \title{The \xpackage{atenddvi} package}
% \date{2016/05/16 v1.2}
% \author{Heiko Oberdiek\thanks
% {Please report any issues at \url{https://github.com/ho-tex/oberdiek/issues}}}
%
% \maketitle
%
% \begin{abstract}
% \LaTeX\ offers \cs{AtBeginDvi}. This package \xpackage{atenddvi}
% provides the counterpart \cs{AtEndDvi}. The execution of its
% argument is delayed to the end of the document at the end of the
% last page. Thus \cs{special} and \cs{write} remain effective, because
% they are put into the last page. This is the main difference
% to \cs{AtEndDocument}.
% \end{abstract}
%
% \tableofcontents
%
% \section{Documentation}
%
% \begin{declcs}{AtEndDvi} \M{code}
% \end{declcs}
% Macro \cs{AtEndDvi} provides a hook mechanism to put \meta{code}
% at the end of the last output page. It is the logical counterpart
% to \cs{AtBeginDvi}. Despite the name the output type DVI, PDF or whatever
% does not matter.
%
% Unlike \cs{AtBeginDvi} the \meta{code} is not put in a box and
% therefore executed immediately. The hook for \cs{AtEndDvi} is based on
% a macro similar to \cs{AtBeginDocument} or \cs{AtEndDocument}. The
% execution of \meta{code} is delayed until the hook is executed on
% the last page.
%
% Commands such as \cs{special} or \cs{write} (not the \cs{immediate}
% variant) must go as nodes into the contents of a page to have the
% desired effect.
% When the hook for \cs{AtEndDocument} is executed, the last intended
% page may already be shipped out. Therefore \cs{special} or \cs{write}
% cannot be used in a reliable way without generating new page.
%
% This gap is closed by \cs{AtEndDvi} of this package \xpackage{atenddvi}.
% If the document is compiled the first time, the package remembers
% the last page in a reference. In the sceond run, it puts the hook
% on the page that has been detected in the previous run as last page.
% The package detectes if the number of pages has changed, and then
% generates a warning to rerun \LaTeX.
%
% \StopEventually{
% }
%
% \section{Implementation}
%
%    \begin{macrocode}
%<*package>
\NeedsTeXFormat{LaTeX2e}
\ProvidesPackage{atenddvi}%
  [2016/05/16 v1.2 At end DVI hook (HO)]%
%    \end{macrocode}
%
%    Load the required packages
%    \begin{macrocode}
\RequirePackage{zref-abspage,zref-lastpage}[2007/03/19]
\RequirePackage{atbegshi}
%    \end{macrocode}
%
%    \begin{macro}{\AtEndDvi@Hook}
%    Macro \cs{AtEndDvi@Hook} is the data storage macro
%    for the code that is executed later at end of the last page.
%    \begin{macrocode}
\let\AtEndDvi@Hook\@empty
%    \end{macrocode}
%    \end{macro}
%    \begin{macro}{\AtEndDvi}
%    Macro \cs{AtEndDvi} is called in the same way as
%    \cs{AtBeginDocument}. The argument is added to the hook macro.
%    \begin{macrocode}
\newcommand*{\AtEndDvi}{%
  \g@addto@macro\AtEndDvi@Hook
}
%    \end{macrocode}
%    \end{macro}
%
%    \begin{macro}{\AtEndDvi@AtBeginShipout}
%    \begin{macrocode}
\def\AtEndDvi@AtBeginShipout{%
  \begingroup
%    \end{macrocode}
%    The reference `LastPage' is marked used. If the reference
%    is not yet defined, then the user gets the warning because of
%    the undefined reference and the rerun warning at the end of
%    the compile run. However, we do not need a warning each page,
%    the first page is enough.
%    \begin{macrocode}
    \ifnum\value{abspage}=1 %
      \zref@refused{LastPage}%
    \fi
%    \end{macrocode}
%    The current absolute page number is compared with the
%    absolute page number of the reference `LastPage'.
%    \begin{macrocode}
    \ifnum\zref@extractdefault{LastPage}{abspage}{0}=\value{abspage}%
%    \end{macrocode}
%    \begin{macro}{\AtEndDvi@LastPage}
%    We found the right page and remember it in a macro.
%    \begin{macrocode}
      \xdef\AtEndDvi@LastPage{\number\value{abspage}}%
%    \end{macrocode}
%    \end{macro}
%    The hook of \cs{AtEndDvi} is now put on the last page
%    after the contents of the page.
%    \begin{macrocode}
      \global\setbox\AtBeginShipoutBox=\vbox{%
        \hbox{%
          \box\AtBeginShipoutBox
          \setbox\AtBeginShipoutBox=\hbox{%
            \begingroup
              \AtEndDvi@Hook
            \endgroup
          }%
          \wd\AtBeginShipoutBox=\z@
          \ht\AtBeginShipoutBox=\z@
          \dp\AtBeginShipoutBox=\z@
          \box\AtBeginShipoutBox
        }%
      }%
%    \end{macrocode}
%    We do not need the every page hook.
%    \begin{macrocode}
      \global\let\AtEndDvi@AtBeginShipout\@empty
%    \end{macrocode}
%    The hook is consumed, \cs{AtEndDvi} does not have an effect.
%    \begin{macrocode}
      \global\let\AtEndDvi\@gobble
%    \end{macrocode}
%    Make a protocol entry, which page is used by this package
%    as last page.
%    \begin{macrocode}
      \let\on@line\@empty
      \PackageInfo{atenddvi}{Last page = \AtEndDvi@LastPage}%
    \fi
  \endgroup
}
%    \end{macrocode}
%    \end{macro}
%
%    \begin{macro}{\AtEndDvi@AtBeginDocument}
%    In order to get as late as possible in the chain of the
%    every shipout hook, the call of \cs{AtBeginShipout} is delayed.
%    \begin{macrocode}
\def\AtEndDvi@AtBeginDocument{%
  \AtBeginShipout{\AtEndDvi@AtBeginShipout}%
%    \end{macrocode}
%    \begin{macro}{\AtEndDvi@Check}
%    After \cs{AtEndDocument} \LaTeX\ reads its \xfile{.aux} files
%    again. Code in \cs{AtEndDocument} could generate additional
%    pages. This is unlikely by code in the \xfile{.aux} file,
%    thus we use the \xfile{.aux} file to run macro
%    \cs{AtEndDvi@Check} for checking the last page.
%
%    During the first reading of the \xfile{.aux} file,
%    \cs{AtEndDvi@Check} is disabled, its real meaning
%    is assigned afterwards.
%    \begin{macrocode}
  \if@filesw
    \immediate\write\@mainaux{%
      \string\providecommand\string\AtEndDvi@Check{}%
    }%
    \immediate\write\@mainaux{%
      \string\AtEndDvi@Check
    }%
  \fi
  \let\AtEndDvi@Check\AtEndDvi@CheckImpl
}
%    \end{macrocode}
%    \end{macro}
%    \begin{macrocode}
\AtBeginDocument{\AtEndDvi@AtBeginDocument}
%    \end{macrocode}
%    \end{macro}
%
%    \begin{macro}{\AtEndDvi@CheckImpl}
%    First check is whether a last page was found at all.
%    Secondly the found last page is compared with the real last page.
%    \begin{macrocode}
\def\AtEndDvi@CheckImpl{%
  \@ifundefined{AtEndDvi@LastPage}{%
    \PackageWarningNoLine{atenddvi}{%
      Rerun LaTeX, last page not yet found%
    }%
  }{%
    \ifnum\AtEndDvi@LastPage=\value{abspage}%
    \else
      \PackageWarningNoLine{atenddvi}{%
        Rerun LaTeX, last page has changed%
      }%
    \fi
  }%
}
%    \end{macrocode}
%    \end{macro}
%
%    \begin{macrocode}
%</package>
%    \end{macrocode}
%
% \section{Installation}
%
% \subsection{Download}
%
% \paragraph{Package.} This package is available on
% CTAN\footnote{\CTANpkg{atenddvi}}:
% \begin{description}
% \item[\CTAN{macros/latex/contrib/oberdiek/atenddvi.dtx}] The source file.
% \item[\CTAN{macros/latex/contrib/oberdiek/atenddvi.pdf}] Documentation.
% \end{description}
%
%
% \paragraph{Bundle.} All the packages of the bundle `oberdiek'
% are also available in a TDS compliant ZIP archive. There
% the packages are already unpacked and the documentation files
% are generated. The files and directories obey the TDS standard.
% \begin{description}
% \item[\CTANinstall{install/macros/latex/contrib/oberdiek.tds.zip}]
% \end{description}
% \emph{TDS} refers to the standard ``A Directory Structure
% for \TeX\ Files'' (\CTAN{tds/tds.pdf}). Directories
% with \xfile{texmf} in their name are usually organized this way.
%
% \subsection{Bundle installation}
%
% \paragraph{Unpacking.} Unpack the \xfile{oberdiek.tds.zip} in the
% TDS tree (also known as \xfile{texmf} tree) of your choice.
% Example (linux):
% \begin{quote}
%   |unzip oberdiek.tds.zip -d ~/texmf|
% \end{quote}
%
% \subsection{Package installation}
%
% \paragraph{Unpacking.} The \xfile{.dtx} file is a self-extracting
% \docstrip\ archive. The files are extracted by running the
% \xfile{.dtx} through \plainTeX:
% \begin{quote}
%   \verb|tex atenddvi.dtx|
% \end{quote}
%
% \paragraph{TDS.} Now the different files must be moved into
% the different directories in your installation TDS tree
% (also known as \xfile{texmf} tree):
% \begin{quote}
% \def\t{^^A
% \begin{tabular}{@{}>{\ttfamily}l@{ $\rightarrow$ }>{\ttfamily}l@{}}
%   atenddvi.sty & tex/latex/oberdiek/atenddvi.sty\\
%   atenddvi.pdf & doc/latex/oberdiek/atenddvi.pdf\\
%   atenddvi.dtx & source/latex/oberdiek/atenddvi.dtx\\
% \end{tabular}^^A
% }^^A
% \sbox0{\t}^^A
% \ifdim\wd0>\linewidth
%   \begingroup
%     \advance\linewidth by\leftmargin
%     \advance\linewidth by\rightmargin
%   \edef\x{\endgroup
%     \def\noexpand\lw{\the\linewidth}^^A
%   }\x
%   \def\lwbox{^^A
%     \leavevmode
%     \hbox to \linewidth{^^A
%       \kern-\leftmargin\relax
%       \hss
%       \usebox0
%       \hss
%       \kern-\rightmargin\relax
%     }^^A
%   }^^A
%   \ifdim\wd0>\lw
%     \sbox0{\small\t}^^A
%     \ifdim\wd0>\linewidth
%       \ifdim\wd0>\lw
%         \sbox0{\footnotesize\t}^^A
%         \ifdim\wd0>\linewidth
%           \ifdim\wd0>\lw
%             \sbox0{\scriptsize\t}^^A
%             \ifdim\wd0>\linewidth
%               \ifdim\wd0>\lw
%                 \sbox0{\tiny\t}^^A
%                 \ifdim\wd0>\linewidth
%                   \lwbox
%                 \else
%                   \usebox0
%                 \fi
%               \else
%                 \lwbox
%               \fi
%             \else
%               \usebox0
%             \fi
%           \else
%             \lwbox
%           \fi
%         \else
%           \usebox0
%         \fi
%       \else
%         \lwbox
%       \fi
%     \else
%       \usebox0
%     \fi
%   \else
%     \lwbox
%   \fi
% \else
%   \usebox0
% \fi
% \end{quote}
% If you have a \xfile{docstrip.cfg} that configures and enables \docstrip's
% TDS installing feature, then some files can already be in the right
% place, see the documentation of \docstrip.
%
% \subsection{Refresh file name databases}
%
% If your \TeX~distribution
% (\TeX\,Live, \mikTeX, \dots) relies on file name databases, you must refresh
% these. For example, \TeX\,Live\ users run \verb|texhash| or
% \verb|mktexlsr|.
%
% \subsection{Some details for the interested}
%
% \paragraph{Unpacking with \LaTeX.}
% The \xfile{.dtx} chooses its action depending on the format:
% \begin{description}
% \item[\plainTeX:] Run \docstrip\ and extract the files.
% \item[\LaTeX:] Generate the documentation.
% \end{description}
% If you insist on using \LaTeX\ for \docstrip\ (really,
% \docstrip\ does not need \LaTeX), then inform the autodetect routine
% about your intention:
% \begin{quote}
%   \verb|latex \let\install=y\input{atenddvi.dtx}|
% \end{quote}
% Do not forget to quote the argument according to the demands
% of your shell.
%
% \paragraph{Generating the documentation.}
% You can use both the \xfile{.dtx} or the \xfile{.drv} to generate
% the documentation. The process can be configured by the
% configuration file \xfile{ltxdoc.cfg}. For instance, put this
% line into this file, if you want to have A4 as paper format:
% \begin{quote}
%   \verb|\PassOptionsToClass{a4paper}{article}|
% \end{quote}
% An example follows how to generate the
% documentation with pdf\LaTeX:
% \begin{quote}
%\begin{verbatim}
%pdflatex atenddvi.dtx
%makeindex -s gind.ist atenddvi.idx
%pdflatex atenddvi.dtx
%makeindex -s gind.ist atenddvi.idx
%pdflatex atenddvi.dtx
%\end{verbatim}
% \end{quote}
%
% \begin{History}
%   \begin{Version}{2007/03/20 v1.0}
%   \item
%     First version.
%   \end{Version}
%   \begin{Version}{2007/04/17 v1.1}
%   \item
%     Package \xpackage{atbegshi} replaces package \xpackage{everyshi}.
%   \end{Version}
%   \begin{Version}{2016/05/16 v1.2}
%   \item
%     Documentation updates.
%   \end{Version}
% \end{History}
%
% \PrintIndex
%
% \Finale
\endinput
|
% \end{quote}
% Do not forget to quote the argument according to the demands
% of your shell.
%
% \paragraph{Generating the documentation.}
% You can use both the \xfile{.dtx} or the \xfile{.drv} to generate
% the documentation. The process can be configured by the
% configuration file \xfile{ltxdoc.cfg}. For instance, put this
% line into this file, if you want to have A4 as paper format:
% \begin{quote}
%   \verb|\PassOptionsToClass{a4paper}{article}|
% \end{quote}
% An example follows how to generate the
% documentation with pdf\LaTeX:
% \begin{quote}
%\begin{verbatim}
%pdflatex atenddvi.dtx
%makeindex -s gind.ist atenddvi.idx
%pdflatex atenddvi.dtx
%makeindex -s gind.ist atenddvi.idx
%pdflatex atenddvi.dtx
%\end{verbatim}
% \end{quote}
%
% \begin{History}
%   \begin{Version}{2007/03/20 v1.0}
%   \item
%     First version.
%   \end{Version}
%   \begin{Version}{2007/04/17 v1.1}
%   \item
%     Package \xpackage{atbegshi} replaces package \xpackage{everyshi}.
%   \end{Version}
%   \begin{Version}{2016/05/16 v1.2}
%   \item
%     Documentation updates.
%   \end{Version}
% \end{History}
%
% \PrintIndex
%
% \Finale
\endinput
|
% \end{quote}
% Do not forget to quote the argument according to the demands
% of your shell.
%
% \paragraph{Generating the documentation.}
% You can use both the \xfile{.dtx} or the \xfile{.drv} to generate
% the documentation. The process can be configured by the
% configuration file \xfile{ltxdoc.cfg}. For instance, put this
% line into this file, if you want to have A4 as paper format:
% \begin{quote}
%   \verb|\PassOptionsToClass{a4paper}{article}|
% \end{quote}
% An example follows how to generate the
% documentation with pdf\LaTeX:
% \begin{quote}
%\begin{verbatim}
%pdflatex atenddvi.dtx
%makeindex -s gind.ist atenddvi.idx
%pdflatex atenddvi.dtx
%makeindex -s gind.ist atenddvi.idx
%pdflatex atenddvi.dtx
%\end{verbatim}
% \end{quote}
%
% \begin{History}
%   \begin{Version}{2007/03/20 v1.0}
%   \item
%     First version.
%   \end{Version}
%   \begin{Version}{2007/04/17 v1.1}
%   \item
%     Package \xpackage{atbegshi} replaces package \xpackage{everyshi}.
%   \end{Version}
%   \begin{Version}{2016/05/16 v1.2}
%   \item
%     Documentation updates.
%   \end{Version}
% \end{History}
%
% \PrintIndex
%
% \Finale
\endinput

%        (quote the arguments according to the demands of your shell)
%
% Documentation:
%    (a) If atenddvi.drv is present:
%           latex atenddvi.drv
%    (b) Without atenddvi.drv:
%           latex atenddvi.dtx; ...
%    The class ltxdoc loads the configuration file ltxdoc.cfg
%    if available. Here you can specify further options, e.g.
%    use A4 as paper format:
%       \PassOptionsToClass{a4paper}{article}
%
%    Programm calls to get the documentation (example):
%       pdflatex atenddvi.dtx
%       makeindex -s gind.ist atenddvi.idx
%       pdflatex atenddvi.dtx
%       makeindex -s gind.ist atenddvi.idx
%       pdflatex atenddvi.dtx
%
% Installation:
%    TDS:tex/latex/oberdiek/atenddvi.sty
%    TDS:doc/latex/oberdiek/atenddvi.pdf
%    TDS:source/latex/oberdiek/atenddvi.dtx
%
%<*ignore>
\begingroup
  \catcode123=1 %
  \catcode125=2 %
  \def\x{LaTeX2e}%
\expandafter\endgroup
\ifcase 0\ifx\install y1\fi\expandafter
         \ifx\csname processbatchFile\endcsname\relax\else1\fi
         \ifx\fmtname\x\else 1\fi\relax
\else\csname fi\endcsname
%</ignore>
%<*install>
\input docstrip.tex
\Msg{************************************************************************}
\Msg{* Installation}
\Msg{* Package: atenddvi 2016/05/16 v1.2 At end DVI hook (HO)}
\Msg{************************************************************************}

\keepsilent
\askforoverwritefalse

\let\MetaPrefix\relax
\preamble

This is a generated file.

Project: atenddvi
Version: 2016/05/16 v1.2

Copyright (C) 2007 by
   Heiko Oberdiek <heiko.oberdiek at googlemail.com>

This work may be distributed and/or modified under the
conditions of the LaTeX Project Public License, either
version 1.3c of this license or (at your option) any later
version. This version of this license is in
   http://www.latex-project.org/lppl/lppl-1-3c.txt
and the latest version of this license is in
   http://www.latex-project.org/lppl.txt
and version 1.3 or later is part of all distributions of
LaTeX version 2005/12/01 or later.

This work has the LPPL maintenance status "maintained".

This Current Maintainer of this work is Heiko Oberdiek.

This work consists of the main source file atenddvi.dtx
and the derived files
   atenddvi.sty, atenddvi.pdf, atenddvi.ins, atenddvi.drv.

\endpreamble
\let\MetaPrefix\DoubleperCent

\generate{%
  \file{atenddvi.ins}{\from{atenddvi.dtx}{install}}%
  \file{atenddvi.drv}{\from{atenddvi.dtx}{driver}}%
  \usedir{tex/latex/oberdiek}%
  \file{atenddvi.sty}{\from{atenddvi.dtx}{package}}%
  \nopreamble
  \nopostamble
%  \usedir{source/latex/oberdiek/catalogue}%
%  \file{atenddvi.xml}{\from{atenddvi.dtx}{catalogue}}%
}

\catcode32=13\relax% active space
\let =\space%
\Msg{************************************************************************}
\Msg{*}
\Msg{* To finish the installation you have to move the following}
\Msg{* file into a directory searched by TeX:}
\Msg{*}
\Msg{*     atenddvi.sty}
\Msg{*}
\Msg{* To produce the documentation run the file `atenddvi.drv'}
\Msg{* through LaTeX.}
\Msg{*}
\Msg{* Happy TeXing!}
\Msg{*}
\Msg{************************************************************************}

\endbatchfile
%</install>
%<*ignore>
\fi
%</ignore>
%<*driver>
\NeedsTeXFormat{LaTeX2e}
\ProvidesFile{atenddvi.drv}%
  [2016/05/16 v1.2 At end DVI hook (HO)]%
\documentclass{ltxdoc}
\usepackage{holtxdoc}[2011/11/22]
\begin{document}
  \DocInput{atenddvi.dtx}%
\end{document}
%</driver>
% \fi
%
%
% \CharacterTable
%  {Upper-case    \A\B\C\D\E\F\G\H\I\J\K\L\M\N\O\P\Q\R\S\T\U\V\W\X\Y\Z
%   Lower-case    \a\b\c\d\e\f\g\h\i\j\k\l\m\n\o\p\q\r\s\t\u\v\w\x\y\z
%   Digits        \0\1\2\3\4\5\6\7\8\9
%   Exclamation   \!     Double quote  \"     Hash (number) \#
%   Dollar        \$     Percent       \%     Ampersand     \&
%   Acute accent  \'     Left paren    \(     Right paren   \)
%   Asterisk      \*     Plus          \+     Comma         \,
%   Minus         \-     Point         \.     Solidus       \/
%   Colon         \:     Semicolon     \;     Less than     \<
%   Equals        \=     Greater than  \>     Question mark \?
%   Commercial at \@     Left bracket  \[     Backslash     \\
%   Right bracket \]     Circumflex    \^     Underscore    \_
%   Grave accent  \`     Left brace    \{     Vertical bar  \|
%   Right brace   \}     Tilde         \~}
%
% \GetFileInfo{atenddvi.drv}
%
% \title{The \xpackage{atenddvi} package}
% \date{2016/05/16 v1.2}
% \author{Heiko Oberdiek\thanks
% {Please report any issues at https://github.com/ho-tex/oberdiek/issues}\\
% \xemail{heiko.oberdiek at googlemail.com}}
%
% \maketitle
%
% \begin{abstract}
% \LaTeX\ offers \cs{AtBeginDvi}. This package \xpackage{atenddvi}
% provides the counterpart \cs{AtEndDvi}. The execution of its
% argument is delayed to the end of the document at the end of the
% last page. Thus \cs{special} and \cs{write} remain effective, because
% they are put into the last page. This is the main difference
% to \cs{AtEndDocument}.
% \end{abstract}
%
% \tableofcontents
%
% \section{Documentation}
%
% \begin{declcs}{AtEndDvi} \M{code}
% \end{declcs}
% Macro \cs{AtEndDvi} provides a hook mechanism to put \meta{code}
% at the end of the last output page. It is the logical counterpart
% to \cs{AtBeginDvi}. Despite the name the output type DVI, PDF or whatever
% does not matter.
%
% Unlike \cs{AtBeginDvi} the \meta{code} is not put in a box and
% therefore executed immediately. The hook for \cs{AtEndDvi} is based on
% a macro similar to \cs{AtBeginDocument} or \cs{AtEndDocument}. The
% execution of \meta{code} is delayed until the hook is executed on
% the last page.
%
% Commands such as \cs{special} or \cs{write} (not the \cs{immediate}
% variant) must go as nodes into the contents of a page to have the
% desired effect.
% When the hook for \cs{AtEndDocument} is executed, the last intended
% page may already be shipped out. Therefore \cs{special} or \cs{write}
% cannot be used in a reliable way without generating new page.
%
% This gap is closed by \cs{AtEndDvi} of this package \xpackage{atenddvi}.
% If the document is compiled the first time, the package remembers
% the last page in a reference. In the sceond run, it puts the hook
% on the page that has been detected in the previous run as last page.
% The package detectes if the number of pages has changed, and then
% generates a warning to rerun \LaTeX.
%
% \StopEventually{
% }
%
% \section{Implementation}
%
%    \begin{macrocode}
%<*package>
\NeedsTeXFormat{LaTeX2e}
\ProvidesPackage{atenddvi}%
  [2016/05/16 v1.2 At end DVI hook (HO)]%
%    \end{macrocode}
%
%    Load the required packages
%    \begin{macrocode}
\RequirePackage{zref-abspage,zref-lastpage}[2007/03/19]
\RequirePackage{atbegshi}
%    \end{macrocode}
%
%    \begin{macro}{\AtEndDvi@Hook}
%    Macro \cs{AtEndDvi@Hook} is the data storage macro
%    for the code that is executed later at end of the last page.
%    \begin{macrocode}
\let\AtEndDvi@Hook\@empty
%    \end{macrocode}
%    \end{macro}
%    \begin{macro}{\AtEndDvi}
%    Macro \cs{AtEndDvi} is called in the same way as
%    \cs{AtBeginDocument}. The argument is added to the hook macro.
%    \begin{macrocode}
\newcommand*{\AtEndDvi}{%
  \g@addto@macro\AtEndDvi@Hook
}
%    \end{macrocode}
%    \end{macro}
%
%    \begin{macro}{\AtEndDvi@AtBeginShipout}
%    \begin{macrocode}
\def\AtEndDvi@AtBeginShipout{%
  \begingroup
%    \end{macrocode}
%    The reference `LastPage' is marked used. If the reference
%    is not yet defined, then the user gets the warning because of
%    the undefined reference and the rerun warning at the end of
%    the compile run. However, we do not need a warning each page,
%    the first page is enough.
%    \begin{macrocode}
    \ifnum\value{abspage}=1 %
      \zref@refused{LastPage}%
    \fi
%    \end{macrocode}
%    The current absolute page number is compared with the
%    absolute page number of the reference `LastPage'.
%    \begin{macrocode}
    \ifnum\zref@extractdefault{LastPage}{abspage}{0}=\value{abspage}%
%    \end{macrocode}
%    \begin{macro}{\AtEndDvi@LastPage}
%    We found the right page and remember it in a macro.
%    \begin{macrocode}
      \xdef\AtEndDvi@LastPage{\number\value{abspage}}%
%    \end{macrocode}
%    \end{macro}
%    The hook of \cs{AtEndDvi} is now put on the last page
%    after the contents of the page.
%    \begin{macrocode}
      \global\setbox\AtBeginShipoutBox=\vbox{%
        \hbox{%
          \box\AtBeginShipoutBox
          \setbox\AtBeginShipoutBox=\hbox{%
            \begingroup
              \AtEndDvi@Hook
            \endgroup
          }%
          \wd\AtBeginShipoutBox=\z@
          \ht\AtBeginShipoutBox=\z@
          \dp\AtBeginShipoutBox=\z@
          \box\AtBeginShipoutBox
        }%
      }%
%    \end{macrocode}
%    We do not need the every page hook.
%    \begin{macrocode}
      \global\let\AtEndDvi@AtBeginShipout\@empty
%    \end{macrocode}
%    The hook is consumed, \cs{AtEndDvi} does not have an effect.
%    \begin{macrocode}
      \global\let\AtEndDvi\@gobble
%    \end{macrocode}
%    Make a protocol entry, which page is used by this package
%    as last page.
%    \begin{macrocode}
      \let\on@line\@empty
      \PackageInfo{atenddvi}{Last page = \AtEndDvi@LastPage}%
    \fi
  \endgroup
}
%    \end{macrocode}
%    \end{macro}
%
%    \begin{macro}{\AtEndDvi@AtBeginDocument}
%    In order to get as late as possible in the chain of the
%    every shipout hook, the call of \cs{AtBeginShipout} is delayed.
%    \begin{macrocode}
\def\AtEndDvi@AtBeginDocument{%
  \AtBeginShipout{\AtEndDvi@AtBeginShipout}%
%    \end{macrocode}
%    \begin{macro}{\AtEndDvi@Check}
%    After \cs{AtEndDocument} \LaTeX\ reads its \xfile{.aux} files
%    again. Code in \cs{AtEndDocument} could generate additional
%    pages. This is unlikely by code in the \xfile{.aux} file,
%    thus we use the \xfile{.aux} file to run macro
%    \cs{AtEndDvi@Check} for checking the last page.
%
%    During the first reading of the \xfile{.aux} file,
%    \cs{AtEndDvi@Check} is disabled, its real meaning
%    is assigned afterwards.
%    \begin{macrocode}
  \if@filesw
    \immediate\write\@mainaux{%
      \string\providecommand\string\AtEndDvi@Check{}%
    }%
    \immediate\write\@mainaux{%
      \string\AtEndDvi@Check
    }%
  \fi
  \let\AtEndDvi@Check\AtEndDvi@CheckImpl
}
%    \end{macrocode}
%    \end{macro}
%    \begin{macrocode}
\AtBeginDocument{\AtEndDvi@AtBeginDocument}
%    \end{macrocode}
%    \end{macro}
%
%    \begin{macro}{\AtEndDvi@CheckImpl}
%    First check is whether a last page was found at all.
%    Secondly the found last page is compared with the real last page.
%    \begin{macrocode}
\def\AtEndDvi@CheckImpl{%
  \@ifundefined{AtEndDvi@LastPage}{%
    \PackageWarningNoLine{atenddvi}{%
      Rerun LaTeX, last page not yet found%
    }%
  }{%
    \ifnum\AtEndDvi@LastPage=\value{abspage}%
    \else
      \PackageWarningNoLine{atenddvi}{%
        Rerun LaTeX, last page has changed%
      }%
    \fi
  }%
}
%    \end{macrocode}
%    \end{macro}
%
%    \begin{macrocode}
%</package>
%    \end{macrocode}
%
% \section{Installation}
%
% \subsection{Download}
%
% \paragraph{Package.} This package is available on
% CTAN\footnote{\url{http://ctan.org/pkg/atenddvi}}:
% \begin{description}
% \item[\CTAN{macros/latex/contrib/oberdiek/atenddvi.dtx}] The source file.
% \item[\CTAN{macros/latex/contrib/oberdiek/atenddvi.pdf}] Documentation.
% \end{description}
%
%
% \paragraph{Bundle.} All the packages of the bundle `oberdiek'
% are also available in a TDS compliant ZIP archive. There
% the packages are already unpacked and the documentation files
% are generated. The files and directories obey the TDS standard.
% \begin{description}
% \item[\CTANinstall{install/macros/latex/contrib/oberdiek.tds.zip}]
% \end{description}
% \emph{TDS} refers to the standard ``A Directory Structure
% for \TeX\ Files'' (\CTAN{tds/tds.pdf}). Directories
% with \xfile{texmf} in their name are usually organized this way.
%
% \subsection{Bundle installation}
%
% \paragraph{Unpacking.} Unpack the \xfile{oberdiek.tds.zip} in the
% TDS tree (also known as \xfile{texmf} tree) of your choice.
% Example (linux):
% \begin{quote}
%   |unzip oberdiek.tds.zip -d ~/texmf|
% \end{quote}
%
% \paragraph{Script installation.}
% Check the directory \xfile{TDS:scripts/oberdiek/} for
% scripts that need further installation steps.
% Package \xpackage{attachfile2} comes with the Perl script
% \xfile{pdfatfi.pl} that should be installed in such a way
% that it can be called as \texttt{pdfatfi}.
% Example (linux):
% \begin{quote}
%   |chmod +x scripts/oberdiek/pdfatfi.pl|\\
%   |cp scripts/oberdiek/pdfatfi.pl /usr/local/bin/|
% \end{quote}
%
% \subsection{Package installation}
%
% \paragraph{Unpacking.} The \xfile{.dtx} file is a self-extracting
% \docstrip\ archive. The files are extracted by running the
% \xfile{.dtx} through \plainTeX:
% \begin{quote}
%   \verb|tex atenddvi.dtx|
% \end{quote}
%
% \paragraph{TDS.} Now the different files must be moved into
% the different directories in your installation TDS tree
% (also known as \xfile{texmf} tree):
% \begin{quote}
% \def\t{^^A
% \begin{tabular}{@{}>{\ttfamily}l@{ $\rightarrow$ }>{\ttfamily}l@{}}
%   atenddvi.sty & tex/latex/oberdiek/atenddvi.sty\\
%   atenddvi.pdf & doc/latex/oberdiek/atenddvi.pdf\\
%   atenddvi.dtx & source/latex/oberdiek/atenddvi.dtx\\
% \end{tabular}^^A
% }^^A
% \sbox0{\t}^^A
% \ifdim\wd0>\linewidth
%   \begingroup
%     \advance\linewidth by\leftmargin
%     \advance\linewidth by\rightmargin
%   \edef\x{\endgroup
%     \def\noexpand\lw{\the\linewidth}^^A
%   }\x
%   \def\lwbox{^^A
%     \leavevmode
%     \hbox to \linewidth{^^A
%       \kern-\leftmargin\relax
%       \hss
%       \usebox0
%       \hss
%       \kern-\rightmargin\relax
%     }^^A
%   }^^A
%   \ifdim\wd0>\lw
%     \sbox0{\small\t}^^A
%     \ifdim\wd0>\linewidth
%       \ifdim\wd0>\lw
%         \sbox0{\footnotesize\t}^^A
%         \ifdim\wd0>\linewidth
%           \ifdim\wd0>\lw
%             \sbox0{\scriptsize\t}^^A
%             \ifdim\wd0>\linewidth
%               \ifdim\wd0>\lw
%                 \sbox0{\tiny\t}^^A
%                 \ifdim\wd0>\linewidth
%                   \lwbox
%                 \else
%                   \usebox0
%                 \fi
%               \else
%                 \lwbox
%               \fi
%             \else
%               \usebox0
%             \fi
%           \else
%             \lwbox
%           \fi
%         \else
%           \usebox0
%         \fi
%       \else
%         \lwbox
%       \fi
%     \else
%       \usebox0
%     \fi
%   \else
%     \lwbox
%   \fi
% \else
%   \usebox0
% \fi
% \end{quote}
% If you have a \xfile{docstrip.cfg} that configures and enables \docstrip's
% TDS installing feature, then some files can already be in the right
% place, see the documentation of \docstrip.
%
% \subsection{Refresh file name databases}
%
% If your \TeX~distribution
% (\teTeX, \mikTeX, \dots) relies on file name databases, you must refresh
% these. For example, \teTeX\ users run \verb|texhash| or
% \verb|mktexlsr|.
%
% \subsection{Some details for the interested}
%
% \paragraph{Attached source.}
%
% The PDF documentation on CTAN also includes the
% \xfile{.dtx} source file. It can be extracted by
% AcrobatReader 6 or higher. Another option is \textsf{pdftk},
% e.g. unpack the file into the current directory:
% \begin{quote}
%   \verb|pdftk atenddvi.pdf unpack_files output .|
% \end{quote}
%
% \paragraph{Unpacking with \LaTeX.}
% The \xfile{.dtx} chooses its action depending on the format:
% \begin{description}
% \item[\plainTeX:] Run \docstrip\ and extract the files.
% \item[\LaTeX:] Generate the documentation.
% \end{description}
% If you insist on using \LaTeX\ for \docstrip\ (really,
% \docstrip\ does not need \LaTeX), then inform the autodetect routine
% about your intention:
% \begin{quote}
%   \verb|latex \let\install=y% \iffalse meta-comment
%
% File: atenddvi.dtx
% Version: 2016/05/16 v1.2
% Info: At end DVI hook
%
% Copyright (C)
%    2007 Heiko Oberdiek
%    2016-2019 Oberdiek Package Support Group
%    https://github.com/ho-tex/oberdiek/issues
%
% This work may be distributed and/or modified under the
% conditions of the LaTeX Project Public License, either
% version 1.3c of this license or (at your option) any later
% version. This version of this license is in
%    https://www.latex-project.org/lppl/lppl-1-3c.txt
% and the latest version of this license is in
%    https://www.latex-project.org/lppl.txt
% and version 1.3 or later is part of all distributions of
% LaTeX version 2005/12/01 or later.
%
% This work has the LPPL maintenance status "maintained".
%
% The Current Maintainers of this work are
% Heiko Oberdiek and the Oberdiek Package Support Group
% https://github.com/ho-tex/oberdiek/issues
%
% This work consists of the main source file atenddvi.dtx
% and the derived files
%    atenddvi.sty, atenddvi.pdf, atenddvi.ins, atenddvi.drv.
%
% Distribution:
%    CTAN:macros/latex/contrib/oberdiek/atenddvi.dtx
%    CTAN:macros/latex/contrib/oberdiek/atenddvi.pdf
%
% Unpacking:
%    (a) If atenddvi.ins is present:
%           tex atenddvi.ins
%    (b) Without atenddvi.ins:
%           tex atenddvi.dtx
%    (c) If you insist on using LaTeX
%           latex \let\install=y% \iffalse meta-comment
%
% File: atenddvi.dtx
% Version: 2016/05/16 v1.2
% Info: At end DVI hook
%
% Copyright (C)
%    2007 Heiko Oberdiek
%    2016-2019 Oberdiek Package Support Group
%    https://github.com/ho-tex/oberdiek/issues
%
% This work may be distributed and/or modified under the
% conditions of the LaTeX Project Public License, either
% version 1.3c of this license or (at your option) any later
% version. This version of this license is in
%    https://www.latex-project.org/lppl/lppl-1-3c.txt
% and the latest version of this license is in
%    https://www.latex-project.org/lppl.txt
% and version 1.3 or later is part of all distributions of
% LaTeX version 2005/12/01 or later.
%
% This work has the LPPL maintenance status "maintained".
%
% The Current Maintainers of this work are
% Heiko Oberdiek and the Oberdiek Package Support Group
% https://github.com/ho-tex/oberdiek/issues
%
% This work consists of the main source file atenddvi.dtx
% and the derived files
%    atenddvi.sty, atenddvi.pdf, atenddvi.ins, atenddvi.drv.
%
% Distribution:
%    CTAN:macros/latex/contrib/oberdiek/atenddvi.dtx
%    CTAN:macros/latex/contrib/oberdiek/atenddvi.pdf
%
% Unpacking:
%    (a) If atenddvi.ins is present:
%           tex atenddvi.ins
%    (b) Without atenddvi.ins:
%           tex atenddvi.dtx
%    (c) If you insist on using LaTeX
%           latex \let\install=y% \iffalse meta-comment
%
% File: atenddvi.dtx
% Version: 2016/05/16 v1.2
% Info: At end DVI hook
%
% Copyright (C)
%    2007 Heiko Oberdiek
%    2016-2019 Oberdiek Package Support Group
%    https://github.com/ho-tex/oberdiek/issues
%
% This work may be distributed and/or modified under the
% conditions of the LaTeX Project Public License, either
% version 1.3c of this license or (at your option) any later
% version. This version of this license is in
%    https://www.latex-project.org/lppl/lppl-1-3c.txt
% and the latest version of this license is in
%    https://www.latex-project.org/lppl.txt
% and version 1.3 or later is part of all distributions of
% LaTeX version 2005/12/01 or later.
%
% This work has the LPPL maintenance status "maintained".
%
% The Current Maintainers of this work are
% Heiko Oberdiek and the Oberdiek Package Support Group
% https://github.com/ho-tex/oberdiek/issues
%
% This work consists of the main source file atenddvi.dtx
% and the derived files
%    atenddvi.sty, atenddvi.pdf, atenddvi.ins, atenddvi.drv.
%
% Distribution:
%    CTAN:macros/latex/contrib/oberdiek/atenddvi.dtx
%    CTAN:macros/latex/contrib/oberdiek/atenddvi.pdf
%
% Unpacking:
%    (a) If atenddvi.ins is present:
%           tex atenddvi.ins
%    (b) Without atenddvi.ins:
%           tex atenddvi.dtx
%    (c) If you insist on using LaTeX
%           latex \let\install=y\input{atenddvi.dtx}
%        (quote the arguments according to the demands of your shell)
%
% Documentation:
%    (a) If atenddvi.drv is present:
%           latex atenddvi.drv
%    (b) Without atenddvi.drv:
%           latex atenddvi.dtx; ...
%    The class ltxdoc loads the configuration file ltxdoc.cfg
%    if available. Here you can specify further options, e.g.
%    use A4 as paper format:
%       \PassOptionsToClass{a4paper}{article}
%
%    Programm calls to get the documentation (example):
%       pdflatex atenddvi.dtx
%       makeindex -s gind.ist atenddvi.idx
%       pdflatex atenddvi.dtx
%       makeindex -s gind.ist atenddvi.idx
%       pdflatex atenddvi.dtx
%
% Installation:
%    TDS:tex/latex/oberdiek/atenddvi.sty
%    TDS:doc/latex/oberdiek/atenddvi.pdf
%    TDS:source/latex/oberdiek/atenddvi.dtx
%
%<*ignore>
\begingroup
  \catcode123=1 %
  \catcode125=2 %
  \def\x{LaTeX2e}%
\expandafter\endgroup
\ifcase 0\ifx\install y1\fi\expandafter
         \ifx\csname processbatchFile\endcsname\relax\else1\fi
         \ifx\fmtname\x\else 1\fi\relax
\else\csname fi\endcsname
%</ignore>
%<*install>
\input docstrip.tex
\Msg{************************************************************************}
\Msg{* Installation}
\Msg{* Package: atenddvi 2016/05/16 v1.2 At end DVI hook (HO)}
\Msg{************************************************************************}

\keepsilent
\askforoverwritefalse

\let\MetaPrefix\relax
\preamble

This is a generated file.

Project: atenddvi
Version: 2016/05/16 v1.2

Copyright (C)
   2007 Heiko Oberdiek
   2016-2019 Oberdiek Package Support Group

This work may be distributed and/or modified under the
conditions of the LaTeX Project Public License, either
version 1.3c of this license or (at your option) any later
version. This version of this license is in
   https://www.latex-project.org/lppl/lppl-1-3c.txt
and the latest version of this license is in
   https://www.latex-project.org/lppl.txt
and version 1.3 or later is part of all distributions of
LaTeX version 2005/12/01 or later.

This work has the LPPL maintenance status "maintained".

The Current Maintainers of this work are
Heiko Oberdiek and the Oberdiek Package Support Group
https://github.com/ho-tex/oberdiek/issues


This work consists of the main source file atenddvi.dtx
and the derived files
   atenddvi.sty, atenddvi.pdf, atenddvi.ins, atenddvi.drv.

\endpreamble
\let\MetaPrefix\DoubleperCent

\generate{%
  \file{atenddvi.ins}{\from{atenddvi.dtx}{install}}%
  \file{atenddvi.drv}{\from{atenddvi.dtx}{driver}}%
  \usedir{tex/latex/oberdiek}%
  \file{atenddvi.sty}{\from{atenddvi.dtx}{package}}%
  \nopreamble
  \nopostamble
%  \usedir{source/latex/oberdiek/catalogue}%
%  \file{atenddvi.xml}{\from{atenddvi.dtx}{catalogue}}%
}

\catcode32=13\relax% active space
\let =\space%
\Msg{************************************************************************}
\Msg{*}
\Msg{* To finish the installation you have to move the following}
\Msg{* file into a directory searched by TeX:}
\Msg{*}
\Msg{*     atenddvi.sty}
\Msg{*}
\Msg{* To produce the documentation run the file `atenddvi.drv'}
\Msg{* through LaTeX.}
\Msg{*}
\Msg{* Happy TeXing!}
\Msg{*}
\Msg{************************************************************************}

\endbatchfile
%</install>
%<*ignore>
\fi
%</ignore>
%<*driver>
\NeedsTeXFormat{LaTeX2e}
\ProvidesFile{atenddvi.drv}%
  [2016/05/16 v1.2 At end DVI hook (HO)]%
\documentclass{ltxdoc}
\usepackage{holtxdoc}[2011/11/22]
\begin{document}
  \DocInput{atenddvi.dtx}%
\end{document}
%</driver>
% \fi
%
%
% \CharacterTable
%  {Upper-case    \A\B\C\D\E\F\G\H\I\J\K\L\M\N\O\P\Q\R\S\T\U\V\W\X\Y\Z
%   Lower-case    \a\b\c\d\e\f\g\h\i\j\k\l\m\n\o\p\q\r\s\t\u\v\w\x\y\z
%   Digits        \0\1\2\3\4\5\6\7\8\9
%   Exclamation   \!     Double quote  \"     Hash (number) \#
%   Dollar        \$     Percent       \%     Ampersand     \&
%   Acute accent  \'     Left paren    \(     Right paren   \)
%   Asterisk      \*     Plus          \+     Comma         \,
%   Minus         \-     Point         \.     Solidus       \/
%   Colon         \:     Semicolon     \;     Less than     \<
%   Equals        \=     Greater than  \>     Question mark \?
%   Commercial at \@     Left bracket  \[     Backslash     \\
%   Right bracket \]     Circumflex    \^     Underscore    \_
%   Grave accent  \`     Left brace    \{     Vertical bar  \|
%   Right brace   \}     Tilde         \~}
%
% \GetFileInfo{atenddvi.drv}
%
% \title{The \xpackage{atenddvi} package}
% \date{2016/05/16 v1.2}
% \author{Heiko Oberdiek\thanks
% {Please report any issues at \url{https://github.com/ho-tex/oberdiek/issues}}}
%
% \maketitle
%
% \begin{abstract}
% \LaTeX\ offers \cs{AtBeginDvi}. This package \xpackage{atenddvi}
% provides the counterpart \cs{AtEndDvi}. The execution of its
% argument is delayed to the end of the document at the end of the
% last page. Thus \cs{special} and \cs{write} remain effective, because
% they are put into the last page. This is the main difference
% to \cs{AtEndDocument}.
% \end{abstract}
%
% \tableofcontents
%
% \section{Documentation}
%
% \begin{declcs}{AtEndDvi} \M{code}
% \end{declcs}
% Macro \cs{AtEndDvi} provides a hook mechanism to put \meta{code}
% at the end of the last output page. It is the logical counterpart
% to \cs{AtBeginDvi}. Despite the name the output type DVI, PDF or whatever
% does not matter.
%
% Unlike \cs{AtBeginDvi} the \meta{code} is not put in a box and
% therefore executed immediately. The hook for \cs{AtEndDvi} is based on
% a macro similar to \cs{AtBeginDocument} or \cs{AtEndDocument}. The
% execution of \meta{code} is delayed until the hook is executed on
% the last page.
%
% Commands such as \cs{special} or \cs{write} (not the \cs{immediate}
% variant) must go as nodes into the contents of a page to have the
% desired effect.
% When the hook for \cs{AtEndDocument} is executed, the last intended
% page may already be shipped out. Therefore \cs{special} or \cs{write}
% cannot be used in a reliable way without generating new page.
%
% This gap is closed by \cs{AtEndDvi} of this package \xpackage{atenddvi}.
% If the document is compiled the first time, the package remembers
% the last page in a reference. In the sceond run, it puts the hook
% on the page that has been detected in the previous run as last page.
% The package detectes if the number of pages has changed, and then
% generates a warning to rerun \LaTeX.
%
% \StopEventually{
% }
%
% \section{Implementation}
%
%    \begin{macrocode}
%<*package>
\NeedsTeXFormat{LaTeX2e}
\ProvidesPackage{atenddvi}%
  [2016/05/16 v1.2 At end DVI hook (HO)]%
%    \end{macrocode}
%
%    Load the required packages
%    \begin{macrocode}
\RequirePackage{zref-abspage,zref-lastpage}[2007/03/19]
\RequirePackage{atbegshi}
%    \end{macrocode}
%
%    \begin{macro}{\AtEndDvi@Hook}
%    Macro \cs{AtEndDvi@Hook} is the data storage macro
%    for the code that is executed later at end of the last page.
%    \begin{macrocode}
\let\AtEndDvi@Hook\@empty
%    \end{macrocode}
%    \end{macro}
%    \begin{macro}{\AtEndDvi}
%    Macro \cs{AtEndDvi} is called in the same way as
%    \cs{AtBeginDocument}. The argument is added to the hook macro.
%    \begin{macrocode}
\newcommand*{\AtEndDvi}{%
  \g@addto@macro\AtEndDvi@Hook
}
%    \end{macrocode}
%    \end{macro}
%
%    \begin{macro}{\AtEndDvi@AtBeginShipout}
%    \begin{macrocode}
\def\AtEndDvi@AtBeginShipout{%
  \begingroup
%    \end{macrocode}
%    The reference `LastPage' is marked used. If the reference
%    is not yet defined, then the user gets the warning because of
%    the undefined reference and the rerun warning at the end of
%    the compile run. However, we do not need a warning each page,
%    the first page is enough.
%    \begin{macrocode}
    \ifnum\value{abspage}=1 %
      \zref@refused{LastPage}%
    \fi
%    \end{macrocode}
%    The current absolute page number is compared with the
%    absolute page number of the reference `LastPage'.
%    \begin{macrocode}
    \ifnum\zref@extractdefault{LastPage}{abspage}{0}=\value{abspage}%
%    \end{macrocode}
%    \begin{macro}{\AtEndDvi@LastPage}
%    We found the right page and remember it in a macro.
%    \begin{macrocode}
      \xdef\AtEndDvi@LastPage{\number\value{abspage}}%
%    \end{macrocode}
%    \end{macro}
%    The hook of \cs{AtEndDvi} is now put on the last page
%    after the contents of the page.
%    \begin{macrocode}
      \global\setbox\AtBeginShipoutBox=\vbox{%
        \hbox{%
          \box\AtBeginShipoutBox
          \setbox\AtBeginShipoutBox=\hbox{%
            \begingroup
              \AtEndDvi@Hook
            \endgroup
          }%
          \wd\AtBeginShipoutBox=\z@
          \ht\AtBeginShipoutBox=\z@
          \dp\AtBeginShipoutBox=\z@
          \box\AtBeginShipoutBox
        }%
      }%
%    \end{macrocode}
%    We do not need the every page hook.
%    \begin{macrocode}
      \global\let\AtEndDvi@AtBeginShipout\@empty
%    \end{macrocode}
%    The hook is consumed, \cs{AtEndDvi} does not have an effect.
%    \begin{macrocode}
      \global\let\AtEndDvi\@gobble
%    \end{macrocode}
%    Make a protocol entry, which page is used by this package
%    as last page.
%    \begin{macrocode}
      \let\on@line\@empty
      \PackageInfo{atenddvi}{Last page = \AtEndDvi@LastPage}%
    \fi
  \endgroup
}
%    \end{macrocode}
%    \end{macro}
%
%    \begin{macro}{\AtEndDvi@AtBeginDocument}
%    In order to get as late as possible in the chain of the
%    every shipout hook, the call of \cs{AtBeginShipout} is delayed.
%    \begin{macrocode}
\def\AtEndDvi@AtBeginDocument{%
  \AtBeginShipout{\AtEndDvi@AtBeginShipout}%
%    \end{macrocode}
%    \begin{macro}{\AtEndDvi@Check}
%    After \cs{AtEndDocument} \LaTeX\ reads its \xfile{.aux} files
%    again. Code in \cs{AtEndDocument} could generate additional
%    pages. This is unlikely by code in the \xfile{.aux} file,
%    thus we use the \xfile{.aux} file to run macro
%    \cs{AtEndDvi@Check} for checking the last page.
%
%    During the first reading of the \xfile{.aux} file,
%    \cs{AtEndDvi@Check} is disabled, its real meaning
%    is assigned afterwards.
%    \begin{macrocode}
  \if@filesw
    \immediate\write\@mainaux{%
      \string\providecommand\string\AtEndDvi@Check{}%
    }%
    \immediate\write\@mainaux{%
      \string\AtEndDvi@Check
    }%
  \fi
  \let\AtEndDvi@Check\AtEndDvi@CheckImpl
}
%    \end{macrocode}
%    \end{macro}
%    \begin{macrocode}
\AtBeginDocument{\AtEndDvi@AtBeginDocument}
%    \end{macrocode}
%    \end{macro}
%
%    \begin{macro}{\AtEndDvi@CheckImpl}
%    First check is whether a last page was found at all.
%    Secondly the found last page is compared with the real last page.
%    \begin{macrocode}
\def\AtEndDvi@CheckImpl{%
  \@ifundefined{AtEndDvi@LastPage}{%
    \PackageWarningNoLine{atenddvi}{%
      Rerun LaTeX, last page not yet found%
    }%
  }{%
    \ifnum\AtEndDvi@LastPage=\value{abspage}%
    \else
      \PackageWarningNoLine{atenddvi}{%
        Rerun LaTeX, last page has changed%
      }%
    \fi
  }%
}
%    \end{macrocode}
%    \end{macro}
%
%    \begin{macrocode}
%</package>
%    \end{macrocode}
%
% \section{Installation}
%
% \subsection{Download}
%
% \paragraph{Package.} This package is available on
% CTAN\footnote{\CTANpkg{atenddvi}}:
% \begin{description}
% \item[\CTAN{macros/latex/contrib/oberdiek/atenddvi.dtx}] The source file.
% \item[\CTAN{macros/latex/contrib/oberdiek/atenddvi.pdf}] Documentation.
% \end{description}
%
%
% \paragraph{Bundle.} All the packages of the bundle `oberdiek'
% are also available in a TDS compliant ZIP archive. There
% the packages are already unpacked and the documentation files
% are generated. The files and directories obey the TDS standard.
% \begin{description}
% \item[\CTANinstall{install/macros/latex/contrib/oberdiek.tds.zip}]
% \end{description}
% \emph{TDS} refers to the standard ``A Directory Structure
% for \TeX\ Files'' (\CTAN{tds/tds.pdf}). Directories
% with \xfile{texmf} in their name are usually organized this way.
%
% \subsection{Bundle installation}
%
% \paragraph{Unpacking.} Unpack the \xfile{oberdiek.tds.zip} in the
% TDS tree (also known as \xfile{texmf} tree) of your choice.
% Example (linux):
% \begin{quote}
%   |unzip oberdiek.tds.zip -d ~/texmf|
% \end{quote}
%
% \subsection{Package installation}
%
% \paragraph{Unpacking.} The \xfile{.dtx} file is a self-extracting
% \docstrip\ archive. The files are extracted by running the
% \xfile{.dtx} through \plainTeX:
% \begin{quote}
%   \verb|tex atenddvi.dtx|
% \end{quote}
%
% \paragraph{TDS.} Now the different files must be moved into
% the different directories in your installation TDS tree
% (also known as \xfile{texmf} tree):
% \begin{quote}
% \def\t{^^A
% \begin{tabular}{@{}>{\ttfamily}l@{ $\rightarrow$ }>{\ttfamily}l@{}}
%   atenddvi.sty & tex/latex/oberdiek/atenddvi.sty\\
%   atenddvi.pdf & doc/latex/oberdiek/atenddvi.pdf\\
%   atenddvi.dtx & source/latex/oberdiek/atenddvi.dtx\\
% \end{tabular}^^A
% }^^A
% \sbox0{\t}^^A
% \ifdim\wd0>\linewidth
%   \begingroup
%     \advance\linewidth by\leftmargin
%     \advance\linewidth by\rightmargin
%   \edef\x{\endgroup
%     \def\noexpand\lw{\the\linewidth}^^A
%   }\x
%   \def\lwbox{^^A
%     \leavevmode
%     \hbox to \linewidth{^^A
%       \kern-\leftmargin\relax
%       \hss
%       \usebox0
%       \hss
%       \kern-\rightmargin\relax
%     }^^A
%   }^^A
%   \ifdim\wd0>\lw
%     \sbox0{\small\t}^^A
%     \ifdim\wd0>\linewidth
%       \ifdim\wd0>\lw
%         \sbox0{\footnotesize\t}^^A
%         \ifdim\wd0>\linewidth
%           \ifdim\wd0>\lw
%             \sbox0{\scriptsize\t}^^A
%             \ifdim\wd0>\linewidth
%               \ifdim\wd0>\lw
%                 \sbox0{\tiny\t}^^A
%                 \ifdim\wd0>\linewidth
%                   \lwbox
%                 \else
%                   \usebox0
%                 \fi
%               \else
%                 \lwbox
%               \fi
%             \else
%               \usebox0
%             \fi
%           \else
%             \lwbox
%           \fi
%         \else
%           \usebox0
%         \fi
%       \else
%         \lwbox
%       \fi
%     \else
%       \usebox0
%     \fi
%   \else
%     \lwbox
%   \fi
% \else
%   \usebox0
% \fi
% \end{quote}
% If you have a \xfile{docstrip.cfg} that configures and enables \docstrip's
% TDS installing feature, then some files can already be in the right
% place, see the documentation of \docstrip.
%
% \subsection{Refresh file name databases}
%
% If your \TeX~distribution
% (\TeX\,Live, \mikTeX, \dots) relies on file name databases, you must refresh
% these. For example, \TeX\,Live\ users run \verb|texhash| or
% \verb|mktexlsr|.
%
% \subsection{Some details for the interested}
%
% \paragraph{Unpacking with \LaTeX.}
% The \xfile{.dtx} chooses its action depending on the format:
% \begin{description}
% \item[\plainTeX:] Run \docstrip\ and extract the files.
% \item[\LaTeX:] Generate the documentation.
% \end{description}
% If you insist on using \LaTeX\ for \docstrip\ (really,
% \docstrip\ does not need \LaTeX), then inform the autodetect routine
% about your intention:
% \begin{quote}
%   \verb|latex \let\install=y\input{atenddvi.dtx}|
% \end{quote}
% Do not forget to quote the argument according to the demands
% of your shell.
%
% \paragraph{Generating the documentation.}
% You can use both the \xfile{.dtx} or the \xfile{.drv} to generate
% the documentation. The process can be configured by the
% configuration file \xfile{ltxdoc.cfg}. For instance, put this
% line into this file, if you want to have A4 as paper format:
% \begin{quote}
%   \verb|\PassOptionsToClass{a4paper}{article}|
% \end{quote}
% An example follows how to generate the
% documentation with pdf\LaTeX:
% \begin{quote}
%\begin{verbatim}
%pdflatex atenddvi.dtx
%makeindex -s gind.ist atenddvi.idx
%pdflatex atenddvi.dtx
%makeindex -s gind.ist atenddvi.idx
%pdflatex atenddvi.dtx
%\end{verbatim}
% \end{quote}
%
% \begin{History}
%   \begin{Version}{2007/03/20 v1.0}
%   \item
%     First version.
%   \end{Version}
%   \begin{Version}{2007/04/17 v1.1}
%   \item
%     Package \xpackage{atbegshi} replaces package \xpackage{everyshi}.
%   \end{Version}
%   \begin{Version}{2016/05/16 v1.2}
%   \item
%     Documentation updates.
%   \end{Version}
% \end{History}
%
% \PrintIndex
%
% \Finale
\endinput

%        (quote the arguments according to the demands of your shell)
%
% Documentation:
%    (a) If atenddvi.drv is present:
%           latex atenddvi.drv
%    (b) Without atenddvi.drv:
%           latex atenddvi.dtx; ...
%    The class ltxdoc loads the configuration file ltxdoc.cfg
%    if available. Here you can specify further options, e.g.
%    use A4 as paper format:
%       \PassOptionsToClass{a4paper}{article}
%
%    Programm calls to get the documentation (example):
%       pdflatex atenddvi.dtx
%       makeindex -s gind.ist atenddvi.idx
%       pdflatex atenddvi.dtx
%       makeindex -s gind.ist atenddvi.idx
%       pdflatex atenddvi.dtx
%
% Installation:
%    TDS:tex/latex/oberdiek/atenddvi.sty
%    TDS:doc/latex/oberdiek/atenddvi.pdf
%    TDS:source/latex/oberdiek/atenddvi.dtx
%
%<*ignore>
\begingroup
  \catcode123=1 %
  \catcode125=2 %
  \def\x{LaTeX2e}%
\expandafter\endgroup
\ifcase 0\ifx\install y1\fi\expandafter
         \ifx\csname processbatchFile\endcsname\relax\else1\fi
         \ifx\fmtname\x\else 1\fi\relax
\else\csname fi\endcsname
%</ignore>
%<*install>
\input docstrip.tex
\Msg{************************************************************************}
\Msg{* Installation}
\Msg{* Package: atenddvi 2016/05/16 v1.2 At end DVI hook (HO)}
\Msg{************************************************************************}

\keepsilent
\askforoverwritefalse

\let\MetaPrefix\relax
\preamble

This is a generated file.

Project: atenddvi
Version: 2016/05/16 v1.2

Copyright (C)
   2007 Heiko Oberdiek
   2016-2019 Oberdiek Package Support Group

This work may be distributed and/or modified under the
conditions of the LaTeX Project Public License, either
version 1.3c of this license or (at your option) any later
version. This version of this license is in
   https://www.latex-project.org/lppl/lppl-1-3c.txt
and the latest version of this license is in
   https://www.latex-project.org/lppl.txt
and version 1.3 or later is part of all distributions of
LaTeX version 2005/12/01 or later.

This work has the LPPL maintenance status "maintained".

The Current Maintainers of this work are
Heiko Oberdiek and the Oberdiek Package Support Group
https://github.com/ho-tex/oberdiek/issues


This work consists of the main source file atenddvi.dtx
and the derived files
   atenddvi.sty, atenddvi.pdf, atenddvi.ins, atenddvi.drv.

\endpreamble
\let\MetaPrefix\DoubleperCent

\generate{%
  \file{atenddvi.ins}{\from{atenddvi.dtx}{install}}%
  \file{atenddvi.drv}{\from{atenddvi.dtx}{driver}}%
  \usedir{tex/latex/oberdiek}%
  \file{atenddvi.sty}{\from{atenddvi.dtx}{package}}%
  \nopreamble
  \nopostamble
%  \usedir{source/latex/oberdiek/catalogue}%
%  \file{atenddvi.xml}{\from{atenddvi.dtx}{catalogue}}%
}

\catcode32=13\relax% active space
\let =\space%
\Msg{************************************************************************}
\Msg{*}
\Msg{* To finish the installation you have to move the following}
\Msg{* file into a directory searched by TeX:}
\Msg{*}
\Msg{*     atenddvi.sty}
\Msg{*}
\Msg{* To produce the documentation run the file `atenddvi.drv'}
\Msg{* through LaTeX.}
\Msg{*}
\Msg{* Happy TeXing!}
\Msg{*}
\Msg{************************************************************************}

\endbatchfile
%</install>
%<*ignore>
\fi
%</ignore>
%<*driver>
\NeedsTeXFormat{LaTeX2e}
\ProvidesFile{atenddvi.drv}%
  [2016/05/16 v1.2 At end DVI hook (HO)]%
\documentclass{ltxdoc}
\usepackage{holtxdoc}[2011/11/22]
\begin{document}
  \DocInput{atenddvi.dtx}%
\end{document}
%</driver>
% \fi
%
%
% \CharacterTable
%  {Upper-case    \A\B\C\D\E\F\G\H\I\J\K\L\M\N\O\P\Q\R\S\T\U\V\W\X\Y\Z
%   Lower-case    \a\b\c\d\e\f\g\h\i\j\k\l\m\n\o\p\q\r\s\t\u\v\w\x\y\z
%   Digits        \0\1\2\3\4\5\6\7\8\9
%   Exclamation   \!     Double quote  \"     Hash (number) \#
%   Dollar        \$     Percent       \%     Ampersand     \&
%   Acute accent  \'     Left paren    \(     Right paren   \)
%   Asterisk      \*     Plus          \+     Comma         \,
%   Minus         \-     Point         \.     Solidus       \/
%   Colon         \:     Semicolon     \;     Less than     \<
%   Equals        \=     Greater than  \>     Question mark \?
%   Commercial at \@     Left bracket  \[     Backslash     \\
%   Right bracket \]     Circumflex    \^     Underscore    \_
%   Grave accent  \`     Left brace    \{     Vertical bar  \|
%   Right brace   \}     Tilde         \~}
%
% \GetFileInfo{atenddvi.drv}
%
% \title{The \xpackage{atenddvi} package}
% \date{2016/05/16 v1.2}
% \author{Heiko Oberdiek\thanks
% {Please report any issues at \url{https://github.com/ho-tex/oberdiek/issues}}}
%
% \maketitle
%
% \begin{abstract}
% \LaTeX\ offers \cs{AtBeginDvi}. This package \xpackage{atenddvi}
% provides the counterpart \cs{AtEndDvi}. The execution of its
% argument is delayed to the end of the document at the end of the
% last page. Thus \cs{special} and \cs{write} remain effective, because
% they are put into the last page. This is the main difference
% to \cs{AtEndDocument}.
% \end{abstract}
%
% \tableofcontents
%
% \section{Documentation}
%
% \begin{declcs}{AtEndDvi} \M{code}
% \end{declcs}
% Macro \cs{AtEndDvi} provides a hook mechanism to put \meta{code}
% at the end of the last output page. It is the logical counterpart
% to \cs{AtBeginDvi}. Despite the name the output type DVI, PDF or whatever
% does not matter.
%
% Unlike \cs{AtBeginDvi} the \meta{code} is not put in a box and
% therefore executed immediately. The hook for \cs{AtEndDvi} is based on
% a macro similar to \cs{AtBeginDocument} or \cs{AtEndDocument}. The
% execution of \meta{code} is delayed until the hook is executed on
% the last page.
%
% Commands such as \cs{special} or \cs{write} (not the \cs{immediate}
% variant) must go as nodes into the contents of a page to have the
% desired effect.
% When the hook for \cs{AtEndDocument} is executed, the last intended
% page may already be shipped out. Therefore \cs{special} or \cs{write}
% cannot be used in a reliable way without generating new page.
%
% This gap is closed by \cs{AtEndDvi} of this package \xpackage{atenddvi}.
% If the document is compiled the first time, the package remembers
% the last page in a reference. In the sceond run, it puts the hook
% on the page that has been detected in the previous run as last page.
% The package detectes if the number of pages has changed, and then
% generates a warning to rerun \LaTeX.
%
% \StopEventually{
% }
%
% \section{Implementation}
%
%    \begin{macrocode}
%<*package>
\NeedsTeXFormat{LaTeX2e}
\ProvidesPackage{atenddvi}%
  [2016/05/16 v1.2 At end DVI hook (HO)]%
%    \end{macrocode}
%
%    Load the required packages
%    \begin{macrocode}
\RequirePackage{zref-abspage,zref-lastpage}[2007/03/19]
\RequirePackage{atbegshi}
%    \end{macrocode}
%
%    \begin{macro}{\AtEndDvi@Hook}
%    Macro \cs{AtEndDvi@Hook} is the data storage macro
%    for the code that is executed later at end of the last page.
%    \begin{macrocode}
\let\AtEndDvi@Hook\@empty
%    \end{macrocode}
%    \end{macro}
%    \begin{macro}{\AtEndDvi}
%    Macro \cs{AtEndDvi} is called in the same way as
%    \cs{AtBeginDocument}. The argument is added to the hook macro.
%    \begin{macrocode}
\newcommand*{\AtEndDvi}{%
  \g@addto@macro\AtEndDvi@Hook
}
%    \end{macrocode}
%    \end{macro}
%
%    \begin{macro}{\AtEndDvi@AtBeginShipout}
%    \begin{macrocode}
\def\AtEndDvi@AtBeginShipout{%
  \begingroup
%    \end{macrocode}
%    The reference `LastPage' is marked used. If the reference
%    is not yet defined, then the user gets the warning because of
%    the undefined reference and the rerun warning at the end of
%    the compile run. However, we do not need a warning each page,
%    the first page is enough.
%    \begin{macrocode}
    \ifnum\value{abspage}=1 %
      \zref@refused{LastPage}%
    \fi
%    \end{macrocode}
%    The current absolute page number is compared with the
%    absolute page number of the reference `LastPage'.
%    \begin{macrocode}
    \ifnum\zref@extractdefault{LastPage}{abspage}{0}=\value{abspage}%
%    \end{macrocode}
%    \begin{macro}{\AtEndDvi@LastPage}
%    We found the right page and remember it in a macro.
%    \begin{macrocode}
      \xdef\AtEndDvi@LastPage{\number\value{abspage}}%
%    \end{macrocode}
%    \end{macro}
%    The hook of \cs{AtEndDvi} is now put on the last page
%    after the contents of the page.
%    \begin{macrocode}
      \global\setbox\AtBeginShipoutBox=\vbox{%
        \hbox{%
          \box\AtBeginShipoutBox
          \setbox\AtBeginShipoutBox=\hbox{%
            \begingroup
              \AtEndDvi@Hook
            \endgroup
          }%
          \wd\AtBeginShipoutBox=\z@
          \ht\AtBeginShipoutBox=\z@
          \dp\AtBeginShipoutBox=\z@
          \box\AtBeginShipoutBox
        }%
      }%
%    \end{macrocode}
%    We do not need the every page hook.
%    \begin{macrocode}
      \global\let\AtEndDvi@AtBeginShipout\@empty
%    \end{macrocode}
%    The hook is consumed, \cs{AtEndDvi} does not have an effect.
%    \begin{macrocode}
      \global\let\AtEndDvi\@gobble
%    \end{macrocode}
%    Make a protocol entry, which page is used by this package
%    as last page.
%    \begin{macrocode}
      \let\on@line\@empty
      \PackageInfo{atenddvi}{Last page = \AtEndDvi@LastPage}%
    \fi
  \endgroup
}
%    \end{macrocode}
%    \end{macro}
%
%    \begin{macro}{\AtEndDvi@AtBeginDocument}
%    In order to get as late as possible in the chain of the
%    every shipout hook, the call of \cs{AtBeginShipout} is delayed.
%    \begin{macrocode}
\def\AtEndDvi@AtBeginDocument{%
  \AtBeginShipout{\AtEndDvi@AtBeginShipout}%
%    \end{macrocode}
%    \begin{macro}{\AtEndDvi@Check}
%    After \cs{AtEndDocument} \LaTeX\ reads its \xfile{.aux} files
%    again. Code in \cs{AtEndDocument} could generate additional
%    pages. This is unlikely by code in the \xfile{.aux} file,
%    thus we use the \xfile{.aux} file to run macro
%    \cs{AtEndDvi@Check} for checking the last page.
%
%    During the first reading of the \xfile{.aux} file,
%    \cs{AtEndDvi@Check} is disabled, its real meaning
%    is assigned afterwards.
%    \begin{macrocode}
  \if@filesw
    \immediate\write\@mainaux{%
      \string\providecommand\string\AtEndDvi@Check{}%
    }%
    \immediate\write\@mainaux{%
      \string\AtEndDvi@Check
    }%
  \fi
  \let\AtEndDvi@Check\AtEndDvi@CheckImpl
}
%    \end{macrocode}
%    \end{macro}
%    \begin{macrocode}
\AtBeginDocument{\AtEndDvi@AtBeginDocument}
%    \end{macrocode}
%    \end{macro}
%
%    \begin{macro}{\AtEndDvi@CheckImpl}
%    First check is whether a last page was found at all.
%    Secondly the found last page is compared with the real last page.
%    \begin{macrocode}
\def\AtEndDvi@CheckImpl{%
  \@ifundefined{AtEndDvi@LastPage}{%
    \PackageWarningNoLine{atenddvi}{%
      Rerun LaTeX, last page not yet found%
    }%
  }{%
    \ifnum\AtEndDvi@LastPage=\value{abspage}%
    \else
      \PackageWarningNoLine{atenddvi}{%
        Rerun LaTeX, last page has changed%
      }%
    \fi
  }%
}
%    \end{macrocode}
%    \end{macro}
%
%    \begin{macrocode}
%</package>
%    \end{macrocode}
%
% \section{Installation}
%
% \subsection{Download}
%
% \paragraph{Package.} This package is available on
% CTAN\footnote{\CTANpkg{atenddvi}}:
% \begin{description}
% \item[\CTAN{macros/latex/contrib/oberdiek/atenddvi.dtx}] The source file.
% \item[\CTAN{macros/latex/contrib/oberdiek/atenddvi.pdf}] Documentation.
% \end{description}
%
%
% \paragraph{Bundle.} All the packages of the bundle `oberdiek'
% are also available in a TDS compliant ZIP archive. There
% the packages are already unpacked and the documentation files
% are generated. The files and directories obey the TDS standard.
% \begin{description}
% \item[\CTANinstall{install/macros/latex/contrib/oberdiek.tds.zip}]
% \end{description}
% \emph{TDS} refers to the standard ``A Directory Structure
% for \TeX\ Files'' (\CTAN{tds/tds.pdf}). Directories
% with \xfile{texmf} in their name are usually organized this way.
%
% \subsection{Bundle installation}
%
% \paragraph{Unpacking.} Unpack the \xfile{oberdiek.tds.zip} in the
% TDS tree (also known as \xfile{texmf} tree) of your choice.
% Example (linux):
% \begin{quote}
%   |unzip oberdiek.tds.zip -d ~/texmf|
% \end{quote}
%
% \subsection{Package installation}
%
% \paragraph{Unpacking.} The \xfile{.dtx} file is a self-extracting
% \docstrip\ archive. The files are extracted by running the
% \xfile{.dtx} through \plainTeX:
% \begin{quote}
%   \verb|tex atenddvi.dtx|
% \end{quote}
%
% \paragraph{TDS.} Now the different files must be moved into
% the different directories in your installation TDS tree
% (also known as \xfile{texmf} tree):
% \begin{quote}
% \def\t{^^A
% \begin{tabular}{@{}>{\ttfamily}l@{ $\rightarrow$ }>{\ttfamily}l@{}}
%   atenddvi.sty & tex/latex/oberdiek/atenddvi.sty\\
%   atenddvi.pdf & doc/latex/oberdiek/atenddvi.pdf\\
%   atenddvi.dtx & source/latex/oberdiek/atenddvi.dtx\\
% \end{tabular}^^A
% }^^A
% \sbox0{\t}^^A
% \ifdim\wd0>\linewidth
%   \begingroup
%     \advance\linewidth by\leftmargin
%     \advance\linewidth by\rightmargin
%   \edef\x{\endgroup
%     \def\noexpand\lw{\the\linewidth}^^A
%   }\x
%   \def\lwbox{^^A
%     \leavevmode
%     \hbox to \linewidth{^^A
%       \kern-\leftmargin\relax
%       \hss
%       \usebox0
%       \hss
%       \kern-\rightmargin\relax
%     }^^A
%   }^^A
%   \ifdim\wd0>\lw
%     \sbox0{\small\t}^^A
%     \ifdim\wd0>\linewidth
%       \ifdim\wd0>\lw
%         \sbox0{\footnotesize\t}^^A
%         \ifdim\wd0>\linewidth
%           \ifdim\wd0>\lw
%             \sbox0{\scriptsize\t}^^A
%             \ifdim\wd0>\linewidth
%               \ifdim\wd0>\lw
%                 \sbox0{\tiny\t}^^A
%                 \ifdim\wd0>\linewidth
%                   \lwbox
%                 \else
%                   \usebox0
%                 \fi
%               \else
%                 \lwbox
%               \fi
%             \else
%               \usebox0
%             \fi
%           \else
%             \lwbox
%           \fi
%         \else
%           \usebox0
%         \fi
%       \else
%         \lwbox
%       \fi
%     \else
%       \usebox0
%     \fi
%   \else
%     \lwbox
%   \fi
% \else
%   \usebox0
% \fi
% \end{quote}
% If you have a \xfile{docstrip.cfg} that configures and enables \docstrip's
% TDS installing feature, then some files can already be in the right
% place, see the documentation of \docstrip.
%
% \subsection{Refresh file name databases}
%
% If your \TeX~distribution
% (\TeX\,Live, \mikTeX, \dots) relies on file name databases, you must refresh
% these. For example, \TeX\,Live\ users run \verb|texhash| or
% \verb|mktexlsr|.
%
% \subsection{Some details for the interested}
%
% \paragraph{Unpacking with \LaTeX.}
% The \xfile{.dtx} chooses its action depending on the format:
% \begin{description}
% \item[\plainTeX:] Run \docstrip\ and extract the files.
% \item[\LaTeX:] Generate the documentation.
% \end{description}
% If you insist on using \LaTeX\ for \docstrip\ (really,
% \docstrip\ does not need \LaTeX), then inform the autodetect routine
% about your intention:
% \begin{quote}
%   \verb|latex \let\install=y% \iffalse meta-comment
%
% File: atenddvi.dtx
% Version: 2016/05/16 v1.2
% Info: At end DVI hook
%
% Copyright (C)
%    2007 Heiko Oberdiek
%    2016-2019 Oberdiek Package Support Group
%    https://github.com/ho-tex/oberdiek/issues
%
% This work may be distributed and/or modified under the
% conditions of the LaTeX Project Public License, either
% version 1.3c of this license or (at your option) any later
% version. This version of this license is in
%    https://www.latex-project.org/lppl/lppl-1-3c.txt
% and the latest version of this license is in
%    https://www.latex-project.org/lppl.txt
% and version 1.3 or later is part of all distributions of
% LaTeX version 2005/12/01 or later.
%
% This work has the LPPL maintenance status "maintained".
%
% The Current Maintainers of this work are
% Heiko Oberdiek and the Oberdiek Package Support Group
% https://github.com/ho-tex/oberdiek/issues
%
% This work consists of the main source file atenddvi.dtx
% and the derived files
%    atenddvi.sty, atenddvi.pdf, atenddvi.ins, atenddvi.drv.
%
% Distribution:
%    CTAN:macros/latex/contrib/oberdiek/atenddvi.dtx
%    CTAN:macros/latex/contrib/oberdiek/atenddvi.pdf
%
% Unpacking:
%    (a) If atenddvi.ins is present:
%           tex atenddvi.ins
%    (b) Without atenddvi.ins:
%           tex atenddvi.dtx
%    (c) If you insist on using LaTeX
%           latex \let\install=y\input{atenddvi.dtx}
%        (quote the arguments according to the demands of your shell)
%
% Documentation:
%    (a) If atenddvi.drv is present:
%           latex atenddvi.drv
%    (b) Without atenddvi.drv:
%           latex atenddvi.dtx; ...
%    The class ltxdoc loads the configuration file ltxdoc.cfg
%    if available. Here you can specify further options, e.g.
%    use A4 as paper format:
%       \PassOptionsToClass{a4paper}{article}
%
%    Programm calls to get the documentation (example):
%       pdflatex atenddvi.dtx
%       makeindex -s gind.ist atenddvi.idx
%       pdflatex atenddvi.dtx
%       makeindex -s gind.ist atenddvi.idx
%       pdflatex atenddvi.dtx
%
% Installation:
%    TDS:tex/latex/oberdiek/atenddvi.sty
%    TDS:doc/latex/oberdiek/atenddvi.pdf
%    TDS:source/latex/oberdiek/atenddvi.dtx
%
%<*ignore>
\begingroup
  \catcode123=1 %
  \catcode125=2 %
  \def\x{LaTeX2e}%
\expandafter\endgroup
\ifcase 0\ifx\install y1\fi\expandafter
         \ifx\csname processbatchFile\endcsname\relax\else1\fi
         \ifx\fmtname\x\else 1\fi\relax
\else\csname fi\endcsname
%</ignore>
%<*install>
\input docstrip.tex
\Msg{************************************************************************}
\Msg{* Installation}
\Msg{* Package: atenddvi 2016/05/16 v1.2 At end DVI hook (HO)}
\Msg{************************************************************************}

\keepsilent
\askforoverwritefalse

\let\MetaPrefix\relax
\preamble

This is a generated file.

Project: atenddvi
Version: 2016/05/16 v1.2

Copyright (C)
   2007 Heiko Oberdiek
   2016-2019 Oberdiek Package Support Group

This work may be distributed and/or modified under the
conditions of the LaTeX Project Public License, either
version 1.3c of this license or (at your option) any later
version. This version of this license is in
   https://www.latex-project.org/lppl/lppl-1-3c.txt
and the latest version of this license is in
   https://www.latex-project.org/lppl.txt
and version 1.3 or later is part of all distributions of
LaTeX version 2005/12/01 or later.

This work has the LPPL maintenance status "maintained".

The Current Maintainers of this work are
Heiko Oberdiek and the Oberdiek Package Support Group
https://github.com/ho-tex/oberdiek/issues


This work consists of the main source file atenddvi.dtx
and the derived files
   atenddvi.sty, atenddvi.pdf, atenddvi.ins, atenddvi.drv.

\endpreamble
\let\MetaPrefix\DoubleperCent

\generate{%
  \file{atenddvi.ins}{\from{atenddvi.dtx}{install}}%
  \file{atenddvi.drv}{\from{atenddvi.dtx}{driver}}%
  \usedir{tex/latex/oberdiek}%
  \file{atenddvi.sty}{\from{atenddvi.dtx}{package}}%
  \nopreamble
  \nopostamble
%  \usedir{source/latex/oberdiek/catalogue}%
%  \file{atenddvi.xml}{\from{atenddvi.dtx}{catalogue}}%
}

\catcode32=13\relax% active space
\let =\space%
\Msg{************************************************************************}
\Msg{*}
\Msg{* To finish the installation you have to move the following}
\Msg{* file into a directory searched by TeX:}
\Msg{*}
\Msg{*     atenddvi.sty}
\Msg{*}
\Msg{* To produce the documentation run the file `atenddvi.drv'}
\Msg{* through LaTeX.}
\Msg{*}
\Msg{* Happy TeXing!}
\Msg{*}
\Msg{************************************************************************}

\endbatchfile
%</install>
%<*ignore>
\fi
%</ignore>
%<*driver>
\NeedsTeXFormat{LaTeX2e}
\ProvidesFile{atenddvi.drv}%
  [2016/05/16 v1.2 At end DVI hook (HO)]%
\documentclass{ltxdoc}
\usepackage{holtxdoc}[2011/11/22]
\begin{document}
  \DocInput{atenddvi.dtx}%
\end{document}
%</driver>
% \fi
%
%
% \CharacterTable
%  {Upper-case    \A\B\C\D\E\F\G\H\I\J\K\L\M\N\O\P\Q\R\S\T\U\V\W\X\Y\Z
%   Lower-case    \a\b\c\d\e\f\g\h\i\j\k\l\m\n\o\p\q\r\s\t\u\v\w\x\y\z
%   Digits        \0\1\2\3\4\5\6\7\8\9
%   Exclamation   \!     Double quote  \"     Hash (number) \#
%   Dollar        \$     Percent       \%     Ampersand     \&
%   Acute accent  \'     Left paren    \(     Right paren   \)
%   Asterisk      \*     Plus          \+     Comma         \,
%   Minus         \-     Point         \.     Solidus       \/
%   Colon         \:     Semicolon     \;     Less than     \<
%   Equals        \=     Greater than  \>     Question mark \?
%   Commercial at \@     Left bracket  \[     Backslash     \\
%   Right bracket \]     Circumflex    \^     Underscore    \_
%   Grave accent  \`     Left brace    \{     Vertical bar  \|
%   Right brace   \}     Tilde         \~}
%
% \GetFileInfo{atenddvi.drv}
%
% \title{The \xpackage{atenddvi} package}
% \date{2016/05/16 v1.2}
% \author{Heiko Oberdiek\thanks
% {Please report any issues at \url{https://github.com/ho-tex/oberdiek/issues}}}
%
% \maketitle
%
% \begin{abstract}
% \LaTeX\ offers \cs{AtBeginDvi}. This package \xpackage{atenddvi}
% provides the counterpart \cs{AtEndDvi}. The execution of its
% argument is delayed to the end of the document at the end of the
% last page. Thus \cs{special} and \cs{write} remain effective, because
% they are put into the last page. This is the main difference
% to \cs{AtEndDocument}.
% \end{abstract}
%
% \tableofcontents
%
% \section{Documentation}
%
% \begin{declcs}{AtEndDvi} \M{code}
% \end{declcs}
% Macro \cs{AtEndDvi} provides a hook mechanism to put \meta{code}
% at the end of the last output page. It is the logical counterpart
% to \cs{AtBeginDvi}. Despite the name the output type DVI, PDF or whatever
% does not matter.
%
% Unlike \cs{AtBeginDvi} the \meta{code} is not put in a box and
% therefore executed immediately. The hook for \cs{AtEndDvi} is based on
% a macro similar to \cs{AtBeginDocument} or \cs{AtEndDocument}. The
% execution of \meta{code} is delayed until the hook is executed on
% the last page.
%
% Commands such as \cs{special} or \cs{write} (not the \cs{immediate}
% variant) must go as nodes into the contents of a page to have the
% desired effect.
% When the hook for \cs{AtEndDocument} is executed, the last intended
% page may already be shipped out. Therefore \cs{special} or \cs{write}
% cannot be used in a reliable way without generating new page.
%
% This gap is closed by \cs{AtEndDvi} of this package \xpackage{atenddvi}.
% If the document is compiled the first time, the package remembers
% the last page in a reference. In the sceond run, it puts the hook
% on the page that has been detected in the previous run as last page.
% The package detectes if the number of pages has changed, and then
% generates a warning to rerun \LaTeX.
%
% \StopEventually{
% }
%
% \section{Implementation}
%
%    \begin{macrocode}
%<*package>
\NeedsTeXFormat{LaTeX2e}
\ProvidesPackage{atenddvi}%
  [2016/05/16 v1.2 At end DVI hook (HO)]%
%    \end{macrocode}
%
%    Load the required packages
%    \begin{macrocode}
\RequirePackage{zref-abspage,zref-lastpage}[2007/03/19]
\RequirePackage{atbegshi}
%    \end{macrocode}
%
%    \begin{macro}{\AtEndDvi@Hook}
%    Macro \cs{AtEndDvi@Hook} is the data storage macro
%    for the code that is executed later at end of the last page.
%    \begin{macrocode}
\let\AtEndDvi@Hook\@empty
%    \end{macrocode}
%    \end{macro}
%    \begin{macro}{\AtEndDvi}
%    Macro \cs{AtEndDvi} is called in the same way as
%    \cs{AtBeginDocument}. The argument is added to the hook macro.
%    \begin{macrocode}
\newcommand*{\AtEndDvi}{%
  \g@addto@macro\AtEndDvi@Hook
}
%    \end{macrocode}
%    \end{macro}
%
%    \begin{macro}{\AtEndDvi@AtBeginShipout}
%    \begin{macrocode}
\def\AtEndDvi@AtBeginShipout{%
  \begingroup
%    \end{macrocode}
%    The reference `LastPage' is marked used. If the reference
%    is not yet defined, then the user gets the warning because of
%    the undefined reference and the rerun warning at the end of
%    the compile run. However, we do not need a warning each page,
%    the first page is enough.
%    \begin{macrocode}
    \ifnum\value{abspage}=1 %
      \zref@refused{LastPage}%
    \fi
%    \end{macrocode}
%    The current absolute page number is compared with the
%    absolute page number of the reference `LastPage'.
%    \begin{macrocode}
    \ifnum\zref@extractdefault{LastPage}{abspage}{0}=\value{abspage}%
%    \end{macrocode}
%    \begin{macro}{\AtEndDvi@LastPage}
%    We found the right page and remember it in a macro.
%    \begin{macrocode}
      \xdef\AtEndDvi@LastPage{\number\value{abspage}}%
%    \end{macrocode}
%    \end{macro}
%    The hook of \cs{AtEndDvi} is now put on the last page
%    after the contents of the page.
%    \begin{macrocode}
      \global\setbox\AtBeginShipoutBox=\vbox{%
        \hbox{%
          \box\AtBeginShipoutBox
          \setbox\AtBeginShipoutBox=\hbox{%
            \begingroup
              \AtEndDvi@Hook
            \endgroup
          }%
          \wd\AtBeginShipoutBox=\z@
          \ht\AtBeginShipoutBox=\z@
          \dp\AtBeginShipoutBox=\z@
          \box\AtBeginShipoutBox
        }%
      }%
%    \end{macrocode}
%    We do not need the every page hook.
%    \begin{macrocode}
      \global\let\AtEndDvi@AtBeginShipout\@empty
%    \end{macrocode}
%    The hook is consumed, \cs{AtEndDvi} does not have an effect.
%    \begin{macrocode}
      \global\let\AtEndDvi\@gobble
%    \end{macrocode}
%    Make a protocol entry, which page is used by this package
%    as last page.
%    \begin{macrocode}
      \let\on@line\@empty
      \PackageInfo{atenddvi}{Last page = \AtEndDvi@LastPage}%
    \fi
  \endgroup
}
%    \end{macrocode}
%    \end{macro}
%
%    \begin{macro}{\AtEndDvi@AtBeginDocument}
%    In order to get as late as possible in the chain of the
%    every shipout hook, the call of \cs{AtBeginShipout} is delayed.
%    \begin{macrocode}
\def\AtEndDvi@AtBeginDocument{%
  \AtBeginShipout{\AtEndDvi@AtBeginShipout}%
%    \end{macrocode}
%    \begin{macro}{\AtEndDvi@Check}
%    After \cs{AtEndDocument} \LaTeX\ reads its \xfile{.aux} files
%    again. Code in \cs{AtEndDocument} could generate additional
%    pages. This is unlikely by code in the \xfile{.aux} file,
%    thus we use the \xfile{.aux} file to run macro
%    \cs{AtEndDvi@Check} for checking the last page.
%
%    During the first reading of the \xfile{.aux} file,
%    \cs{AtEndDvi@Check} is disabled, its real meaning
%    is assigned afterwards.
%    \begin{macrocode}
  \if@filesw
    \immediate\write\@mainaux{%
      \string\providecommand\string\AtEndDvi@Check{}%
    }%
    \immediate\write\@mainaux{%
      \string\AtEndDvi@Check
    }%
  \fi
  \let\AtEndDvi@Check\AtEndDvi@CheckImpl
}
%    \end{macrocode}
%    \end{macro}
%    \begin{macrocode}
\AtBeginDocument{\AtEndDvi@AtBeginDocument}
%    \end{macrocode}
%    \end{macro}
%
%    \begin{macro}{\AtEndDvi@CheckImpl}
%    First check is whether a last page was found at all.
%    Secondly the found last page is compared with the real last page.
%    \begin{macrocode}
\def\AtEndDvi@CheckImpl{%
  \@ifundefined{AtEndDvi@LastPage}{%
    \PackageWarningNoLine{atenddvi}{%
      Rerun LaTeX, last page not yet found%
    }%
  }{%
    \ifnum\AtEndDvi@LastPage=\value{abspage}%
    \else
      \PackageWarningNoLine{atenddvi}{%
        Rerun LaTeX, last page has changed%
      }%
    \fi
  }%
}
%    \end{macrocode}
%    \end{macro}
%
%    \begin{macrocode}
%</package>
%    \end{macrocode}
%
% \section{Installation}
%
% \subsection{Download}
%
% \paragraph{Package.} This package is available on
% CTAN\footnote{\CTANpkg{atenddvi}}:
% \begin{description}
% \item[\CTAN{macros/latex/contrib/oberdiek/atenddvi.dtx}] The source file.
% \item[\CTAN{macros/latex/contrib/oberdiek/atenddvi.pdf}] Documentation.
% \end{description}
%
%
% \paragraph{Bundle.} All the packages of the bundle `oberdiek'
% are also available in a TDS compliant ZIP archive. There
% the packages are already unpacked and the documentation files
% are generated. The files and directories obey the TDS standard.
% \begin{description}
% \item[\CTANinstall{install/macros/latex/contrib/oberdiek.tds.zip}]
% \end{description}
% \emph{TDS} refers to the standard ``A Directory Structure
% for \TeX\ Files'' (\CTAN{tds/tds.pdf}). Directories
% with \xfile{texmf} in their name are usually organized this way.
%
% \subsection{Bundle installation}
%
% \paragraph{Unpacking.} Unpack the \xfile{oberdiek.tds.zip} in the
% TDS tree (also known as \xfile{texmf} tree) of your choice.
% Example (linux):
% \begin{quote}
%   |unzip oberdiek.tds.zip -d ~/texmf|
% \end{quote}
%
% \subsection{Package installation}
%
% \paragraph{Unpacking.} The \xfile{.dtx} file is a self-extracting
% \docstrip\ archive. The files are extracted by running the
% \xfile{.dtx} through \plainTeX:
% \begin{quote}
%   \verb|tex atenddvi.dtx|
% \end{quote}
%
% \paragraph{TDS.} Now the different files must be moved into
% the different directories in your installation TDS tree
% (also known as \xfile{texmf} tree):
% \begin{quote}
% \def\t{^^A
% \begin{tabular}{@{}>{\ttfamily}l@{ $\rightarrow$ }>{\ttfamily}l@{}}
%   atenddvi.sty & tex/latex/oberdiek/atenddvi.sty\\
%   atenddvi.pdf & doc/latex/oberdiek/atenddvi.pdf\\
%   atenddvi.dtx & source/latex/oberdiek/atenddvi.dtx\\
% \end{tabular}^^A
% }^^A
% \sbox0{\t}^^A
% \ifdim\wd0>\linewidth
%   \begingroup
%     \advance\linewidth by\leftmargin
%     \advance\linewidth by\rightmargin
%   \edef\x{\endgroup
%     \def\noexpand\lw{\the\linewidth}^^A
%   }\x
%   \def\lwbox{^^A
%     \leavevmode
%     \hbox to \linewidth{^^A
%       \kern-\leftmargin\relax
%       \hss
%       \usebox0
%       \hss
%       \kern-\rightmargin\relax
%     }^^A
%   }^^A
%   \ifdim\wd0>\lw
%     \sbox0{\small\t}^^A
%     \ifdim\wd0>\linewidth
%       \ifdim\wd0>\lw
%         \sbox0{\footnotesize\t}^^A
%         \ifdim\wd0>\linewidth
%           \ifdim\wd0>\lw
%             \sbox0{\scriptsize\t}^^A
%             \ifdim\wd0>\linewidth
%               \ifdim\wd0>\lw
%                 \sbox0{\tiny\t}^^A
%                 \ifdim\wd0>\linewidth
%                   \lwbox
%                 \else
%                   \usebox0
%                 \fi
%               \else
%                 \lwbox
%               \fi
%             \else
%               \usebox0
%             \fi
%           \else
%             \lwbox
%           \fi
%         \else
%           \usebox0
%         \fi
%       \else
%         \lwbox
%       \fi
%     \else
%       \usebox0
%     \fi
%   \else
%     \lwbox
%   \fi
% \else
%   \usebox0
% \fi
% \end{quote}
% If you have a \xfile{docstrip.cfg} that configures and enables \docstrip's
% TDS installing feature, then some files can already be in the right
% place, see the documentation of \docstrip.
%
% \subsection{Refresh file name databases}
%
% If your \TeX~distribution
% (\TeX\,Live, \mikTeX, \dots) relies on file name databases, you must refresh
% these. For example, \TeX\,Live\ users run \verb|texhash| or
% \verb|mktexlsr|.
%
% \subsection{Some details for the interested}
%
% \paragraph{Unpacking with \LaTeX.}
% The \xfile{.dtx} chooses its action depending on the format:
% \begin{description}
% \item[\plainTeX:] Run \docstrip\ and extract the files.
% \item[\LaTeX:] Generate the documentation.
% \end{description}
% If you insist on using \LaTeX\ for \docstrip\ (really,
% \docstrip\ does not need \LaTeX), then inform the autodetect routine
% about your intention:
% \begin{quote}
%   \verb|latex \let\install=y\input{atenddvi.dtx}|
% \end{quote}
% Do not forget to quote the argument according to the demands
% of your shell.
%
% \paragraph{Generating the documentation.}
% You can use both the \xfile{.dtx} or the \xfile{.drv} to generate
% the documentation. The process can be configured by the
% configuration file \xfile{ltxdoc.cfg}. For instance, put this
% line into this file, if you want to have A4 as paper format:
% \begin{quote}
%   \verb|\PassOptionsToClass{a4paper}{article}|
% \end{quote}
% An example follows how to generate the
% documentation with pdf\LaTeX:
% \begin{quote}
%\begin{verbatim}
%pdflatex atenddvi.dtx
%makeindex -s gind.ist atenddvi.idx
%pdflatex atenddvi.dtx
%makeindex -s gind.ist atenddvi.idx
%pdflatex atenddvi.dtx
%\end{verbatim}
% \end{quote}
%
% \begin{History}
%   \begin{Version}{2007/03/20 v1.0}
%   \item
%     First version.
%   \end{Version}
%   \begin{Version}{2007/04/17 v1.1}
%   \item
%     Package \xpackage{atbegshi} replaces package \xpackage{everyshi}.
%   \end{Version}
%   \begin{Version}{2016/05/16 v1.2}
%   \item
%     Documentation updates.
%   \end{Version}
% \end{History}
%
% \PrintIndex
%
% \Finale
\endinput
|
% \end{quote}
% Do not forget to quote the argument according to the demands
% of your shell.
%
% \paragraph{Generating the documentation.}
% You can use both the \xfile{.dtx} or the \xfile{.drv} to generate
% the documentation. The process can be configured by the
% configuration file \xfile{ltxdoc.cfg}. For instance, put this
% line into this file, if you want to have A4 as paper format:
% \begin{quote}
%   \verb|\PassOptionsToClass{a4paper}{article}|
% \end{quote}
% An example follows how to generate the
% documentation with pdf\LaTeX:
% \begin{quote}
%\begin{verbatim}
%pdflatex atenddvi.dtx
%makeindex -s gind.ist atenddvi.idx
%pdflatex atenddvi.dtx
%makeindex -s gind.ist atenddvi.idx
%pdflatex atenddvi.dtx
%\end{verbatim}
% \end{quote}
%
% \begin{History}
%   \begin{Version}{2007/03/20 v1.0}
%   \item
%     First version.
%   \end{Version}
%   \begin{Version}{2007/04/17 v1.1}
%   \item
%     Package \xpackage{atbegshi} replaces package \xpackage{everyshi}.
%   \end{Version}
%   \begin{Version}{2016/05/16 v1.2}
%   \item
%     Documentation updates.
%   \end{Version}
% \end{History}
%
% \PrintIndex
%
% \Finale
\endinput

%        (quote the arguments according to the demands of your shell)
%
% Documentation:
%    (a) If atenddvi.drv is present:
%           latex atenddvi.drv
%    (b) Without atenddvi.drv:
%           latex atenddvi.dtx; ...
%    The class ltxdoc loads the configuration file ltxdoc.cfg
%    if available. Here you can specify further options, e.g.
%    use A4 as paper format:
%       \PassOptionsToClass{a4paper}{article}
%
%    Programm calls to get the documentation (example):
%       pdflatex atenddvi.dtx
%       makeindex -s gind.ist atenddvi.idx
%       pdflatex atenddvi.dtx
%       makeindex -s gind.ist atenddvi.idx
%       pdflatex atenddvi.dtx
%
% Installation:
%    TDS:tex/latex/oberdiek/atenddvi.sty
%    TDS:doc/latex/oberdiek/atenddvi.pdf
%    TDS:source/latex/oberdiek/atenddvi.dtx
%
%<*ignore>
\begingroup
  \catcode123=1 %
  \catcode125=2 %
  \def\x{LaTeX2e}%
\expandafter\endgroup
\ifcase 0\ifx\install y1\fi\expandafter
         \ifx\csname processbatchFile\endcsname\relax\else1\fi
         \ifx\fmtname\x\else 1\fi\relax
\else\csname fi\endcsname
%</ignore>
%<*install>
\input docstrip.tex
\Msg{************************************************************************}
\Msg{* Installation}
\Msg{* Package: atenddvi 2016/05/16 v1.2 At end DVI hook (HO)}
\Msg{************************************************************************}

\keepsilent
\askforoverwritefalse

\let\MetaPrefix\relax
\preamble

This is a generated file.

Project: atenddvi
Version: 2016/05/16 v1.2

Copyright (C)
   2007 Heiko Oberdiek
   2016-2019 Oberdiek Package Support Group

This work may be distributed and/or modified under the
conditions of the LaTeX Project Public License, either
version 1.3c of this license or (at your option) any later
version. This version of this license is in
   https://www.latex-project.org/lppl/lppl-1-3c.txt
and the latest version of this license is in
   https://www.latex-project.org/lppl.txt
and version 1.3 or later is part of all distributions of
LaTeX version 2005/12/01 or later.

This work has the LPPL maintenance status "maintained".

The Current Maintainers of this work are
Heiko Oberdiek and the Oberdiek Package Support Group
https://github.com/ho-tex/oberdiek/issues


This work consists of the main source file atenddvi.dtx
and the derived files
   atenddvi.sty, atenddvi.pdf, atenddvi.ins, atenddvi.drv.

\endpreamble
\let\MetaPrefix\DoubleperCent

\generate{%
  \file{atenddvi.ins}{\from{atenddvi.dtx}{install}}%
  \file{atenddvi.drv}{\from{atenddvi.dtx}{driver}}%
  \usedir{tex/latex/oberdiek}%
  \file{atenddvi.sty}{\from{atenddvi.dtx}{package}}%
  \nopreamble
  \nopostamble
%  \usedir{source/latex/oberdiek/catalogue}%
%  \file{atenddvi.xml}{\from{atenddvi.dtx}{catalogue}}%
}

\catcode32=13\relax% active space
\let =\space%
\Msg{************************************************************************}
\Msg{*}
\Msg{* To finish the installation you have to move the following}
\Msg{* file into a directory searched by TeX:}
\Msg{*}
\Msg{*     atenddvi.sty}
\Msg{*}
\Msg{* To produce the documentation run the file `atenddvi.drv'}
\Msg{* through LaTeX.}
\Msg{*}
\Msg{* Happy TeXing!}
\Msg{*}
\Msg{************************************************************************}

\endbatchfile
%</install>
%<*ignore>
\fi
%</ignore>
%<*driver>
\NeedsTeXFormat{LaTeX2e}
\ProvidesFile{atenddvi.drv}%
  [2016/05/16 v1.2 At end DVI hook (HO)]%
\documentclass{ltxdoc}
\usepackage{holtxdoc}[2011/11/22]
\begin{document}
  \DocInput{atenddvi.dtx}%
\end{document}
%</driver>
% \fi
%
%
% \CharacterTable
%  {Upper-case    \A\B\C\D\E\F\G\H\I\J\K\L\M\N\O\P\Q\R\S\T\U\V\W\X\Y\Z
%   Lower-case    \a\b\c\d\e\f\g\h\i\j\k\l\m\n\o\p\q\r\s\t\u\v\w\x\y\z
%   Digits        \0\1\2\3\4\5\6\7\8\9
%   Exclamation   \!     Double quote  \"     Hash (number) \#
%   Dollar        \$     Percent       \%     Ampersand     \&
%   Acute accent  \'     Left paren    \(     Right paren   \)
%   Asterisk      \*     Plus          \+     Comma         \,
%   Minus         \-     Point         \.     Solidus       \/
%   Colon         \:     Semicolon     \;     Less than     \<
%   Equals        \=     Greater than  \>     Question mark \?
%   Commercial at \@     Left bracket  \[     Backslash     \\
%   Right bracket \]     Circumflex    \^     Underscore    \_
%   Grave accent  \`     Left brace    \{     Vertical bar  \|
%   Right brace   \}     Tilde         \~}
%
% \GetFileInfo{atenddvi.drv}
%
% \title{The \xpackage{atenddvi} package}
% \date{2016/05/16 v1.2}
% \author{Heiko Oberdiek\thanks
% {Please report any issues at \url{https://github.com/ho-tex/oberdiek/issues}}}
%
% \maketitle
%
% \begin{abstract}
% \LaTeX\ offers \cs{AtBeginDvi}. This package \xpackage{atenddvi}
% provides the counterpart \cs{AtEndDvi}. The execution of its
% argument is delayed to the end of the document at the end of the
% last page. Thus \cs{special} and \cs{write} remain effective, because
% they are put into the last page. This is the main difference
% to \cs{AtEndDocument}.
% \end{abstract}
%
% \tableofcontents
%
% \section{Documentation}
%
% \begin{declcs}{AtEndDvi} \M{code}
% \end{declcs}
% Macro \cs{AtEndDvi} provides a hook mechanism to put \meta{code}
% at the end of the last output page. It is the logical counterpart
% to \cs{AtBeginDvi}. Despite the name the output type DVI, PDF or whatever
% does not matter.
%
% Unlike \cs{AtBeginDvi} the \meta{code} is not put in a box and
% therefore executed immediately. The hook for \cs{AtEndDvi} is based on
% a macro similar to \cs{AtBeginDocument} or \cs{AtEndDocument}. The
% execution of \meta{code} is delayed until the hook is executed on
% the last page.
%
% Commands such as \cs{special} or \cs{write} (not the \cs{immediate}
% variant) must go as nodes into the contents of a page to have the
% desired effect.
% When the hook for \cs{AtEndDocument} is executed, the last intended
% page may already be shipped out. Therefore \cs{special} or \cs{write}
% cannot be used in a reliable way without generating new page.
%
% This gap is closed by \cs{AtEndDvi} of this package \xpackage{atenddvi}.
% If the document is compiled the first time, the package remembers
% the last page in a reference. In the sceond run, it puts the hook
% on the page that has been detected in the previous run as last page.
% The package detectes if the number of pages has changed, and then
% generates a warning to rerun \LaTeX.
%
% \StopEventually{
% }
%
% \section{Implementation}
%
%    \begin{macrocode}
%<*package>
\NeedsTeXFormat{LaTeX2e}
\ProvidesPackage{atenddvi}%
  [2016/05/16 v1.2 At end DVI hook (HO)]%
%    \end{macrocode}
%
%    Load the required packages
%    \begin{macrocode}
\RequirePackage{zref-abspage,zref-lastpage}[2007/03/19]
\RequirePackage{atbegshi}
%    \end{macrocode}
%
%    \begin{macro}{\AtEndDvi@Hook}
%    Macro \cs{AtEndDvi@Hook} is the data storage macro
%    for the code that is executed later at end of the last page.
%    \begin{macrocode}
\let\AtEndDvi@Hook\@empty
%    \end{macrocode}
%    \end{macro}
%    \begin{macro}{\AtEndDvi}
%    Macro \cs{AtEndDvi} is called in the same way as
%    \cs{AtBeginDocument}. The argument is added to the hook macro.
%    \begin{macrocode}
\newcommand*{\AtEndDvi}{%
  \g@addto@macro\AtEndDvi@Hook
}
%    \end{macrocode}
%    \end{macro}
%
%    \begin{macro}{\AtEndDvi@AtBeginShipout}
%    \begin{macrocode}
\def\AtEndDvi@AtBeginShipout{%
  \begingroup
%    \end{macrocode}
%    The reference `LastPage' is marked used. If the reference
%    is not yet defined, then the user gets the warning because of
%    the undefined reference and the rerun warning at the end of
%    the compile run. However, we do not need a warning each page,
%    the first page is enough.
%    \begin{macrocode}
    \ifnum\value{abspage}=1 %
      \zref@refused{LastPage}%
    \fi
%    \end{macrocode}
%    The current absolute page number is compared with the
%    absolute page number of the reference `LastPage'.
%    \begin{macrocode}
    \ifnum\zref@extractdefault{LastPage}{abspage}{0}=\value{abspage}%
%    \end{macrocode}
%    \begin{macro}{\AtEndDvi@LastPage}
%    We found the right page and remember it in a macro.
%    \begin{macrocode}
      \xdef\AtEndDvi@LastPage{\number\value{abspage}}%
%    \end{macrocode}
%    \end{macro}
%    The hook of \cs{AtEndDvi} is now put on the last page
%    after the contents of the page.
%    \begin{macrocode}
      \global\setbox\AtBeginShipoutBox=\vbox{%
        \hbox{%
          \box\AtBeginShipoutBox
          \setbox\AtBeginShipoutBox=\hbox{%
            \begingroup
              \AtEndDvi@Hook
            \endgroup
          }%
          \wd\AtBeginShipoutBox=\z@
          \ht\AtBeginShipoutBox=\z@
          \dp\AtBeginShipoutBox=\z@
          \box\AtBeginShipoutBox
        }%
      }%
%    \end{macrocode}
%    We do not need the every page hook.
%    \begin{macrocode}
      \global\let\AtEndDvi@AtBeginShipout\@empty
%    \end{macrocode}
%    The hook is consumed, \cs{AtEndDvi} does not have an effect.
%    \begin{macrocode}
      \global\let\AtEndDvi\@gobble
%    \end{macrocode}
%    Make a protocol entry, which page is used by this package
%    as last page.
%    \begin{macrocode}
      \let\on@line\@empty
      \PackageInfo{atenddvi}{Last page = \AtEndDvi@LastPage}%
    \fi
  \endgroup
}
%    \end{macrocode}
%    \end{macro}
%
%    \begin{macro}{\AtEndDvi@AtBeginDocument}
%    In order to get as late as possible in the chain of the
%    every shipout hook, the call of \cs{AtBeginShipout} is delayed.
%    \begin{macrocode}
\def\AtEndDvi@AtBeginDocument{%
  \AtBeginShipout{\AtEndDvi@AtBeginShipout}%
%    \end{macrocode}
%    \begin{macro}{\AtEndDvi@Check}
%    After \cs{AtEndDocument} \LaTeX\ reads its \xfile{.aux} files
%    again. Code in \cs{AtEndDocument} could generate additional
%    pages. This is unlikely by code in the \xfile{.aux} file,
%    thus we use the \xfile{.aux} file to run macro
%    \cs{AtEndDvi@Check} for checking the last page.
%
%    During the first reading of the \xfile{.aux} file,
%    \cs{AtEndDvi@Check} is disabled, its real meaning
%    is assigned afterwards.
%    \begin{macrocode}
  \if@filesw
    \immediate\write\@mainaux{%
      \string\providecommand\string\AtEndDvi@Check{}%
    }%
    \immediate\write\@mainaux{%
      \string\AtEndDvi@Check
    }%
  \fi
  \let\AtEndDvi@Check\AtEndDvi@CheckImpl
}
%    \end{macrocode}
%    \end{macro}
%    \begin{macrocode}
\AtBeginDocument{\AtEndDvi@AtBeginDocument}
%    \end{macrocode}
%    \end{macro}
%
%    \begin{macro}{\AtEndDvi@CheckImpl}
%    First check is whether a last page was found at all.
%    Secondly the found last page is compared with the real last page.
%    \begin{macrocode}
\def\AtEndDvi@CheckImpl{%
  \@ifundefined{AtEndDvi@LastPage}{%
    \PackageWarningNoLine{atenddvi}{%
      Rerun LaTeX, last page not yet found%
    }%
  }{%
    \ifnum\AtEndDvi@LastPage=\value{abspage}%
    \else
      \PackageWarningNoLine{atenddvi}{%
        Rerun LaTeX, last page has changed%
      }%
    \fi
  }%
}
%    \end{macrocode}
%    \end{macro}
%
%    \begin{macrocode}
%</package>
%    \end{macrocode}
%
% \section{Installation}
%
% \subsection{Download}
%
% \paragraph{Package.} This package is available on
% CTAN\footnote{\CTANpkg{atenddvi}}:
% \begin{description}
% \item[\CTAN{macros/latex/contrib/oberdiek/atenddvi.dtx}] The source file.
% \item[\CTAN{macros/latex/contrib/oberdiek/atenddvi.pdf}] Documentation.
% \end{description}
%
%
% \paragraph{Bundle.} All the packages of the bundle `oberdiek'
% are also available in a TDS compliant ZIP archive. There
% the packages are already unpacked and the documentation files
% are generated. The files and directories obey the TDS standard.
% \begin{description}
% \item[\CTANinstall{install/macros/latex/contrib/oberdiek.tds.zip}]
% \end{description}
% \emph{TDS} refers to the standard ``A Directory Structure
% for \TeX\ Files'' (\CTAN{tds/tds.pdf}). Directories
% with \xfile{texmf} in their name are usually organized this way.
%
% \subsection{Bundle installation}
%
% \paragraph{Unpacking.} Unpack the \xfile{oberdiek.tds.zip} in the
% TDS tree (also known as \xfile{texmf} tree) of your choice.
% Example (linux):
% \begin{quote}
%   |unzip oberdiek.tds.zip -d ~/texmf|
% \end{quote}
%
% \subsection{Package installation}
%
% \paragraph{Unpacking.} The \xfile{.dtx} file is a self-extracting
% \docstrip\ archive. The files are extracted by running the
% \xfile{.dtx} through \plainTeX:
% \begin{quote}
%   \verb|tex atenddvi.dtx|
% \end{quote}
%
% \paragraph{TDS.} Now the different files must be moved into
% the different directories in your installation TDS tree
% (also known as \xfile{texmf} tree):
% \begin{quote}
% \def\t{^^A
% \begin{tabular}{@{}>{\ttfamily}l@{ $\rightarrow$ }>{\ttfamily}l@{}}
%   atenddvi.sty & tex/latex/oberdiek/atenddvi.sty\\
%   atenddvi.pdf & doc/latex/oberdiek/atenddvi.pdf\\
%   atenddvi.dtx & source/latex/oberdiek/atenddvi.dtx\\
% \end{tabular}^^A
% }^^A
% \sbox0{\t}^^A
% \ifdim\wd0>\linewidth
%   \begingroup
%     \advance\linewidth by\leftmargin
%     \advance\linewidth by\rightmargin
%   \edef\x{\endgroup
%     \def\noexpand\lw{\the\linewidth}^^A
%   }\x
%   \def\lwbox{^^A
%     \leavevmode
%     \hbox to \linewidth{^^A
%       \kern-\leftmargin\relax
%       \hss
%       \usebox0
%       \hss
%       \kern-\rightmargin\relax
%     }^^A
%   }^^A
%   \ifdim\wd0>\lw
%     \sbox0{\small\t}^^A
%     \ifdim\wd0>\linewidth
%       \ifdim\wd0>\lw
%         \sbox0{\footnotesize\t}^^A
%         \ifdim\wd0>\linewidth
%           \ifdim\wd0>\lw
%             \sbox0{\scriptsize\t}^^A
%             \ifdim\wd0>\linewidth
%               \ifdim\wd0>\lw
%                 \sbox0{\tiny\t}^^A
%                 \ifdim\wd0>\linewidth
%                   \lwbox
%                 \else
%                   \usebox0
%                 \fi
%               \else
%                 \lwbox
%               \fi
%             \else
%               \usebox0
%             \fi
%           \else
%             \lwbox
%           \fi
%         \else
%           \usebox0
%         \fi
%       \else
%         \lwbox
%       \fi
%     \else
%       \usebox0
%     \fi
%   \else
%     \lwbox
%   \fi
% \else
%   \usebox0
% \fi
% \end{quote}
% If you have a \xfile{docstrip.cfg} that configures and enables \docstrip's
% TDS installing feature, then some files can already be in the right
% place, see the documentation of \docstrip.
%
% \subsection{Refresh file name databases}
%
% If your \TeX~distribution
% (\TeX\,Live, \mikTeX, \dots) relies on file name databases, you must refresh
% these. For example, \TeX\,Live\ users run \verb|texhash| or
% \verb|mktexlsr|.
%
% \subsection{Some details for the interested}
%
% \paragraph{Unpacking with \LaTeX.}
% The \xfile{.dtx} chooses its action depending on the format:
% \begin{description}
% \item[\plainTeX:] Run \docstrip\ and extract the files.
% \item[\LaTeX:] Generate the documentation.
% \end{description}
% If you insist on using \LaTeX\ for \docstrip\ (really,
% \docstrip\ does not need \LaTeX), then inform the autodetect routine
% about your intention:
% \begin{quote}
%   \verb|latex \let\install=y% \iffalse meta-comment
%
% File: atenddvi.dtx
% Version: 2016/05/16 v1.2
% Info: At end DVI hook
%
% Copyright (C)
%    2007 Heiko Oberdiek
%    2016-2019 Oberdiek Package Support Group
%    https://github.com/ho-tex/oberdiek/issues
%
% This work may be distributed and/or modified under the
% conditions of the LaTeX Project Public License, either
% version 1.3c of this license or (at your option) any later
% version. This version of this license is in
%    https://www.latex-project.org/lppl/lppl-1-3c.txt
% and the latest version of this license is in
%    https://www.latex-project.org/lppl.txt
% and version 1.3 or later is part of all distributions of
% LaTeX version 2005/12/01 or later.
%
% This work has the LPPL maintenance status "maintained".
%
% The Current Maintainers of this work are
% Heiko Oberdiek and the Oberdiek Package Support Group
% https://github.com/ho-tex/oberdiek/issues
%
% This work consists of the main source file atenddvi.dtx
% and the derived files
%    atenddvi.sty, atenddvi.pdf, atenddvi.ins, atenddvi.drv.
%
% Distribution:
%    CTAN:macros/latex/contrib/oberdiek/atenddvi.dtx
%    CTAN:macros/latex/contrib/oberdiek/atenddvi.pdf
%
% Unpacking:
%    (a) If atenddvi.ins is present:
%           tex atenddvi.ins
%    (b) Without atenddvi.ins:
%           tex atenddvi.dtx
%    (c) If you insist on using LaTeX
%           latex \let\install=y% \iffalse meta-comment
%
% File: atenddvi.dtx
% Version: 2016/05/16 v1.2
% Info: At end DVI hook
%
% Copyright (C)
%    2007 Heiko Oberdiek
%    2016-2019 Oberdiek Package Support Group
%    https://github.com/ho-tex/oberdiek/issues
%
% This work may be distributed and/or modified under the
% conditions of the LaTeX Project Public License, either
% version 1.3c of this license or (at your option) any later
% version. This version of this license is in
%    https://www.latex-project.org/lppl/lppl-1-3c.txt
% and the latest version of this license is in
%    https://www.latex-project.org/lppl.txt
% and version 1.3 or later is part of all distributions of
% LaTeX version 2005/12/01 or later.
%
% This work has the LPPL maintenance status "maintained".
%
% The Current Maintainers of this work are
% Heiko Oberdiek and the Oberdiek Package Support Group
% https://github.com/ho-tex/oberdiek/issues
%
% This work consists of the main source file atenddvi.dtx
% and the derived files
%    atenddvi.sty, atenddvi.pdf, atenddvi.ins, atenddvi.drv.
%
% Distribution:
%    CTAN:macros/latex/contrib/oberdiek/atenddvi.dtx
%    CTAN:macros/latex/contrib/oberdiek/atenddvi.pdf
%
% Unpacking:
%    (a) If atenddvi.ins is present:
%           tex atenddvi.ins
%    (b) Without atenddvi.ins:
%           tex atenddvi.dtx
%    (c) If you insist on using LaTeX
%           latex \let\install=y\input{atenddvi.dtx}
%        (quote the arguments according to the demands of your shell)
%
% Documentation:
%    (a) If atenddvi.drv is present:
%           latex atenddvi.drv
%    (b) Without atenddvi.drv:
%           latex atenddvi.dtx; ...
%    The class ltxdoc loads the configuration file ltxdoc.cfg
%    if available. Here you can specify further options, e.g.
%    use A4 as paper format:
%       \PassOptionsToClass{a4paper}{article}
%
%    Programm calls to get the documentation (example):
%       pdflatex atenddvi.dtx
%       makeindex -s gind.ist atenddvi.idx
%       pdflatex atenddvi.dtx
%       makeindex -s gind.ist atenddvi.idx
%       pdflatex atenddvi.dtx
%
% Installation:
%    TDS:tex/latex/oberdiek/atenddvi.sty
%    TDS:doc/latex/oberdiek/atenddvi.pdf
%    TDS:source/latex/oberdiek/atenddvi.dtx
%
%<*ignore>
\begingroup
  \catcode123=1 %
  \catcode125=2 %
  \def\x{LaTeX2e}%
\expandafter\endgroup
\ifcase 0\ifx\install y1\fi\expandafter
         \ifx\csname processbatchFile\endcsname\relax\else1\fi
         \ifx\fmtname\x\else 1\fi\relax
\else\csname fi\endcsname
%</ignore>
%<*install>
\input docstrip.tex
\Msg{************************************************************************}
\Msg{* Installation}
\Msg{* Package: atenddvi 2016/05/16 v1.2 At end DVI hook (HO)}
\Msg{************************************************************************}

\keepsilent
\askforoverwritefalse

\let\MetaPrefix\relax
\preamble

This is a generated file.

Project: atenddvi
Version: 2016/05/16 v1.2

Copyright (C)
   2007 Heiko Oberdiek
   2016-2019 Oberdiek Package Support Group

This work may be distributed and/or modified under the
conditions of the LaTeX Project Public License, either
version 1.3c of this license or (at your option) any later
version. This version of this license is in
   https://www.latex-project.org/lppl/lppl-1-3c.txt
and the latest version of this license is in
   https://www.latex-project.org/lppl.txt
and version 1.3 or later is part of all distributions of
LaTeX version 2005/12/01 or later.

This work has the LPPL maintenance status "maintained".

The Current Maintainers of this work are
Heiko Oberdiek and the Oberdiek Package Support Group
https://github.com/ho-tex/oberdiek/issues


This work consists of the main source file atenddvi.dtx
and the derived files
   atenddvi.sty, atenddvi.pdf, atenddvi.ins, atenddvi.drv.

\endpreamble
\let\MetaPrefix\DoubleperCent

\generate{%
  \file{atenddvi.ins}{\from{atenddvi.dtx}{install}}%
  \file{atenddvi.drv}{\from{atenddvi.dtx}{driver}}%
  \usedir{tex/latex/oberdiek}%
  \file{atenddvi.sty}{\from{atenddvi.dtx}{package}}%
  \nopreamble
  \nopostamble
%  \usedir{source/latex/oberdiek/catalogue}%
%  \file{atenddvi.xml}{\from{atenddvi.dtx}{catalogue}}%
}

\catcode32=13\relax% active space
\let =\space%
\Msg{************************************************************************}
\Msg{*}
\Msg{* To finish the installation you have to move the following}
\Msg{* file into a directory searched by TeX:}
\Msg{*}
\Msg{*     atenddvi.sty}
\Msg{*}
\Msg{* To produce the documentation run the file `atenddvi.drv'}
\Msg{* through LaTeX.}
\Msg{*}
\Msg{* Happy TeXing!}
\Msg{*}
\Msg{************************************************************************}

\endbatchfile
%</install>
%<*ignore>
\fi
%</ignore>
%<*driver>
\NeedsTeXFormat{LaTeX2e}
\ProvidesFile{atenddvi.drv}%
  [2016/05/16 v1.2 At end DVI hook (HO)]%
\documentclass{ltxdoc}
\usepackage{holtxdoc}[2011/11/22]
\begin{document}
  \DocInput{atenddvi.dtx}%
\end{document}
%</driver>
% \fi
%
%
% \CharacterTable
%  {Upper-case    \A\B\C\D\E\F\G\H\I\J\K\L\M\N\O\P\Q\R\S\T\U\V\W\X\Y\Z
%   Lower-case    \a\b\c\d\e\f\g\h\i\j\k\l\m\n\o\p\q\r\s\t\u\v\w\x\y\z
%   Digits        \0\1\2\3\4\5\6\7\8\9
%   Exclamation   \!     Double quote  \"     Hash (number) \#
%   Dollar        \$     Percent       \%     Ampersand     \&
%   Acute accent  \'     Left paren    \(     Right paren   \)
%   Asterisk      \*     Plus          \+     Comma         \,
%   Minus         \-     Point         \.     Solidus       \/
%   Colon         \:     Semicolon     \;     Less than     \<
%   Equals        \=     Greater than  \>     Question mark \?
%   Commercial at \@     Left bracket  \[     Backslash     \\
%   Right bracket \]     Circumflex    \^     Underscore    \_
%   Grave accent  \`     Left brace    \{     Vertical bar  \|
%   Right brace   \}     Tilde         \~}
%
% \GetFileInfo{atenddvi.drv}
%
% \title{The \xpackage{atenddvi} package}
% \date{2016/05/16 v1.2}
% \author{Heiko Oberdiek\thanks
% {Please report any issues at \url{https://github.com/ho-tex/oberdiek/issues}}}
%
% \maketitle
%
% \begin{abstract}
% \LaTeX\ offers \cs{AtBeginDvi}. This package \xpackage{atenddvi}
% provides the counterpart \cs{AtEndDvi}. The execution of its
% argument is delayed to the end of the document at the end of the
% last page. Thus \cs{special} and \cs{write} remain effective, because
% they are put into the last page. This is the main difference
% to \cs{AtEndDocument}.
% \end{abstract}
%
% \tableofcontents
%
% \section{Documentation}
%
% \begin{declcs}{AtEndDvi} \M{code}
% \end{declcs}
% Macro \cs{AtEndDvi} provides a hook mechanism to put \meta{code}
% at the end of the last output page. It is the logical counterpart
% to \cs{AtBeginDvi}. Despite the name the output type DVI, PDF or whatever
% does not matter.
%
% Unlike \cs{AtBeginDvi} the \meta{code} is not put in a box and
% therefore executed immediately. The hook for \cs{AtEndDvi} is based on
% a macro similar to \cs{AtBeginDocument} or \cs{AtEndDocument}. The
% execution of \meta{code} is delayed until the hook is executed on
% the last page.
%
% Commands such as \cs{special} or \cs{write} (not the \cs{immediate}
% variant) must go as nodes into the contents of a page to have the
% desired effect.
% When the hook for \cs{AtEndDocument} is executed, the last intended
% page may already be shipped out. Therefore \cs{special} or \cs{write}
% cannot be used in a reliable way without generating new page.
%
% This gap is closed by \cs{AtEndDvi} of this package \xpackage{atenddvi}.
% If the document is compiled the first time, the package remembers
% the last page in a reference. In the sceond run, it puts the hook
% on the page that has been detected in the previous run as last page.
% The package detectes if the number of pages has changed, and then
% generates a warning to rerun \LaTeX.
%
% \StopEventually{
% }
%
% \section{Implementation}
%
%    \begin{macrocode}
%<*package>
\NeedsTeXFormat{LaTeX2e}
\ProvidesPackage{atenddvi}%
  [2016/05/16 v1.2 At end DVI hook (HO)]%
%    \end{macrocode}
%
%    Load the required packages
%    \begin{macrocode}
\RequirePackage{zref-abspage,zref-lastpage}[2007/03/19]
\RequirePackage{atbegshi}
%    \end{macrocode}
%
%    \begin{macro}{\AtEndDvi@Hook}
%    Macro \cs{AtEndDvi@Hook} is the data storage macro
%    for the code that is executed later at end of the last page.
%    \begin{macrocode}
\let\AtEndDvi@Hook\@empty
%    \end{macrocode}
%    \end{macro}
%    \begin{macro}{\AtEndDvi}
%    Macro \cs{AtEndDvi} is called in the same way as
%    \cs{AtBeginDocument}. The argument is added to the hook macro.
%    \begin{macrocode}
\newcommand*{\AtEndDvi}{%
  \g@addto@macro\AtEndDvi@Hook
}
%    \end{macrocode}
%    \end{macro}
%
%    \begin{macro}{\AtEndDvi@AtBeginShipout}
%    \begin{macrocode}
\def\AtEndDvi@AtBeginShipout{%
  \begingroup
%    \end{macrocode}
%    The reference `LastPage' is marked used. If the reference
%    is not yet defined, then the user gets the warning because of
%    the undefined reference and the rerun warning at the end of
%    the compile run. However, we do not need a warning each page,
%    the first page is enough.
%    \begin{macrocode}
    \ifnum\value{abspage}=1 %
      \zref@refused{LastPage}%
    \fi
%    \end{macrocode}
%    The current absolute page number is compared with the
%    absolute page number of the reference `LastPage'.
%    \begin{macrocode}
    \ifnum\zref@extractdefault{LastPage}{abspage}{0}=\value{abspage}%
%    \end{macrocode}
%    \begin{macro}{\AtEndDvi@LastPage}
%    We found the right page and remember it in a macro.
%    \begin{macrocode}
      \xdef\AtEndDvi@LastPage{\number\value{abspage}}%
%    \end{macrocode}
%    \end{macro}
%    The hook of \cs{AtEndDvi} is now put on the last page
%    after the contents of the page.
%    \begin{macrocode}
      \global\setbox\AtBeginShipoutBox=\vbox{%
        \hbox{%
          \box\AtBeginShipoutBox
          \setbox\AtBeginShipoutBox=\hbox{%
            \begingroup
              \AtEndDvi@Hook
            \endgroup
          }%
          \wd\AtBeginShipoutBox=\z@
          \ht\AtBeginShipoutBox=\z@
          \dp\AtBeginShipoutBox=\z@
          \box\AtBeginShipoutBox
        }%
      }%
%    \end{macrocode}
%    We do not need the every page hook.
%    \begin{macrocode}
      \global\let\AtEndDvi@AtBeginShipout\@empty
%    \end{macrocode}
%    The hook is consumed, \cs{AtEndDvi} does not have an effect.
%    \begin{macrocode}
      \global\let\AtEndDvi\@gobble
%    \end{macrocode}
%    Make a protocol entry, which page is used by this package
%    as last page.
%    \begin{macrocode}
      \let\on@line\@empty
      \PackageInfo{atenddvi}{Last page = \AtEndDvi@LastPage}%
    \fi
  \endgroup
}
%    \end{macrocode}
%    \end{macro}
%
%    \begin{macro}{\AtEndDvi@AtBeginDocument}
%    In order to get as late as possible in the chain of the
%    every shipout hook, the call of \cs{AtBeginShipout} is delayed.
%    \begin{macrocode}
\def\AtEndDvi@AtBeginDocument{%
  \AtBeginShipout{\AtEndDvi@AtBeginShipout}%
%    \end{macrocode}
%    \begin{macro}{\AtEndDvi@Check}
%    After \cs{AtEndDocument} \LaTeX\ reads its \xfile{.aux} files
%    again. Code in \cs{AtEndDocument} could generate additional
%    pages. This is unlikely by code in the \xfile{.aux} file,
%    thus we use the \xfile{.aux} file to run macro
%    \cs{AtEndDvi@Check} for checking the last page.
%
%    During the first reading of the \xfile{.aux} file,
%    \cs{AtEndDvi@Check} is disabled, its real meaning
%    is assigned afterwards.
%    \begin{macrocode}
  \if@filesw
    \immediate\write\@mainaux{%
      \string\providecommand\string\AtEndDvi@Check{}%
    }%
    \immediate\write\@mainaux{%
      \string\AtEndDvi@Check
    }%
  \fi
  \let\AtEndDvi@Check\AtEndDvi@CheckImpl
}
%    \end{macrocode}
%    \end{macro}
%    \begin{macrocode}
\AtBeginDocument{\AtEndDvi@AtBeginDocument}
%    \end{macrocode}
%    \end{macro}
%
%    \begin{macro}{\AtEndDvi@CheckImpl}
%    First check is whether a last page was found at all.
%    Secondly the found last page is compared with the real last page.
%    \begin{macrocode}
\def\AtEndDvi@CheckImpl{%
  \@ifundefined{AtEndDvi@LastPage}{%
    \PackageWarningNoLine{atenddvi}{%
      Rerun LaTeX, last page not yet found%
    }%
  }{%
    \ifnum\AtEndDvi@LastPage=\value{abspage}%
    \else
      \PackageWarningNoLine{atenddvi}{%
        Rerun LaTeX, last page has changed%
      }%
    \fi
  }%
}
%    \end{macrocode}
%    \end{macro}
%
%    \begin{macrocode}
%</package>
%    \end{macrocode}
%
% \section{Installation}
%
% \subsection{Download}
%
% \paragraph{Package.} This package is available on
% CTAN\footnote{\CTANpkg{atenddvi}}:
% \begin{description}
% \item[\CTAN{macros/latex/contrib/oberdiek/atenddvi.dtx}] The source file.
% \item[\CTAN{macros/latex/contrib/oberdiek/atenddvi.pdf}] Documentation.
% \end{description}
%
%
% \paragraph{Bundle.} All the packages of the bundle `oberdiek'
% are also available in a TDS compliant ZIP archive. There
% the packages are already unpacked and the documentation files
% are generated. The files and directories obey the TDS standard.
% \begin{description}
% \item[\CTANinstall{install/macros/latex/contrib/oberdiek.tds.zip}]
% \end{description}
% \emph{TDS} refers to the standard ``A Directory Structure
% for \TeX\ Files'' (\CTAN{tds/tds.pdf}). Directories
% with \xfile{texmf} in their name are usually organized this way.
%
% \subsection{Bundle installation}
%
% \paragraph{Unpacking.} Unpack the \xfile{oberdiek.tds.zip} in the
% TDS tree (also known as \xfile{texmf} tree) of your choice.
% Example (linux):
% \begin{quote}
%   |unzip oberdiek.tds.zip -d ~/texmf|
% \end{quote}
%
% \subsection{Package installation}
%
% \paragraph{Unpacking.} The \xfile{.dtx} file is a self-extracting
% \docstrip\ archive. The files are extracted by running the
% \xfile{.dtx} through \plainTeX:
% \begin{quote}
%   \verb|tex atenddvi.dtx|
% \end{quote}
%
% \paragraph{TDS.} Now the different files must be moved into
% the different directories in your installation TDS tree
% (also known as \xfile{texmf} tree):
% \begin{quote}
% \def\t{^^A
% \begin{tabular}{@{}>{\ttfamily}l@{ $\rightarrow$ }>{\ttfamily}l@{}}
%   atenddvi.sty & tex/latex/oberdiek/atenddvi.sty\\
%   atenddvi.pdf & doc/latex/oberdiek/atenddvi.pdf\\
%   atenddvi.dtx & source/latex/oberdiek/atenddvi.dtx\\
% \end{tabular}^^A
% }^^A
% \sbox0{\t}^^A
% \ifdim\wd0>\linewidth
%   \begingroup
%     \advance\linewidth by\leftmargin
%     \advance\linewidth by\rightmargin
%   \edef\x{\endgroup
%     \def\noexpand\lw{\the\linewidth}^^A
%   }\x
%   \def\lwbox{^^A
%     \leavevmode
%     \hbox to \linewidth{^^A
%       \kern-\leftmargin\relax
%       \hss
%       \usebox0
%       \hss
%       \kern-\rightmargin\relax
%     }^^A
%   }^^A
%   \ifdim\wd0>\lw
%     \sbox0{\small\t}^^A
%     \ifdim\wd0>\linewidth
%       \ifdim\wd0>\lw
%         \sbox0{\footnotesize\t}^^A
%         \ifdim\wd0>\linewidth
%           \ifdim\wd0>\lw
%             \sbox0{\scriptsize\t}^^A
%             \ifdim\wd0>\linewidth
%               \ifdim\wd0>\lw
%                 \sbox0{\tiny\t}^^A
%                 \ifdim\wd0>\linewidth
%                   \lwbox
%                 \else
%                   \usebox0
%                 \fi
%               \else
%                 \lwbox
%               \fi
%             \else
%               \usebox0
%             \fi
%           \else
%             \lwbox
%           \fi
%         \else
%           \usebox0
%         \fi
%       \else
%         \lwbox
%       \fi
%     \else
%       \usebox0
%     \fi
%   \else
%     \lwbox
%   \fi
% \else
%   \usebox0
% \fi
% \end{quote}
% If you have a \xfile{docstrip.cfg} that configures and enables \docstrip's
% TDS installing feature, then some files can already be in the right
% place, see the documentation of \docstrip.
%
% \subsection{Refresh file name databases}
%
% If your \TeX~distribution
% (\TeX\,Live, \mikTeX, \dots) relies on file name databases, you must refresh
% these. For example, \TeX\,Live\ users run \verb|texhash| or
% \verb|mktexlsr|.
%
% \subsection{Some details for the interested}
%
% \paragraph{Unpacking with \LaTeX.}
% The \xfile{.dtx} chooses its action depending on the format:
% \begin{description}
% \item[\plainTeX:] Run \docstrip\ and extract the files.
% \item[\LaTeX:] Generate the documentation.
% \end{description}
% If you insist on using \LaTeX\ for \docstrip\ (really,
% \docstrip\ does not need \LaTeX), then inform the autodetect routine
% about your intention:
% \begin{quote}
%   \verb|latex \let\install=y\input{atenddvi.dtx}|
% \end{quote}
% Do not forget to quote the argument according to the demands
% of your shell.
%
% \paragraph{Generating the documentation.}
% You can use both the \xfile{.dtx} or the \xfile{.drv} to generate
% the documentation. The process can be configured by the
% configuration file \xfile{ltxdoc.cfg}. For instance, put this
% line into this file, if you want to have A4 as paper format:
% \begin{quote}
%   \verb|\PassOptionsToClass{a4paper}{article}|
% \end{quote}
% An example follows how to generate the
% documentation with pdf\LaTeX:
% \begin{quote}
%\begin{verbatim}
%pdflatex atenddvi.dtx
%makeindex -s gind.ist atenddvi.idx
%pdflatex atenddvi.dtx
%makeindex -s gind.ist atenddvi.idx
%pdflatex atenddvi.dtx
%\end{verbatim}
% \end{quote}
%
% \begin{History}
%   \begin{Version}{2007/03/20 v1.0}
%   \item
%     First version.
%   \end{Version}
%   \begin{Version}{2007/04/17 v1.1}
%   \item
%     Package \xpackage{atbegshi} replaces package \xpackage{everyshi}.
%   \end{Version}
%   \begin{Version}{2016/05/16 v1.2}
%   \item
%     Documentation updates.
%   \end{Version}
% \end{History}
%
% \PrintIndex
%
% \Finale
\endinput

%        (quote the arguments according to the demands of your shell)
%
% Documentation:
%    (a) If atenddvi.drv is present:
%           latex atenddvi.drv
%    (b) Without atenddvi.drv:
%           latex atenddvi.dtx; ...
%    The class ltxdoc loads the configuration file ltxdoc.cfg
%    if available. Here you can specify further options, e.g.
%    use A4 as paper format:
%       \PassOptionsToClass{a4paper}{article}
%
%    Programm calls to get the documentation (example):
%       pdflatex atenddvi.dtx
%       makeindex -s gind.ist atenddvi.idx
%       pdflatex atenddvi.dtx
%       makeindex -s gind.ist atenddvi.idx
%       pdflatex atenddvi.dtx
%
% Installation:
%    TDS:tex/latex/oberdiek/atenddvi.sty
%    TDS:doc/latex/oberdiek/atenddvi.pdf
%    TDS:source/latex/oberdiek/atenddvi.dtx
%
%<*ignore>
\begingroup
  \catcode123=1 %
  \catcode125=2 %
  \def\x{LaTeX2e}%
\expandafter\endgroup
\ifcase 0\ifx\install y1\fi\expandafter
         \ifx\csname processbatchFile\endcsname\relax\else1\fi
         \ifx\fmtname\x\else 1\fi\relax
\else\csname fi\endcsname
%</ignore>
%<*install>
\input docstrip.tex
\Msg{************************************************************************}
\Msg{* Installation}
\Msg{* Package: atenddvi 2016/05/16 v1.2 At end DVI hook (HO)}
\Msg{************************************************************************}

\keepsilent
\askforoverwritefalse

\let\MetaPrefix\relax
\preamble

This is a generated file.

Project: atenddvi
Version: 2016/05/16 v1.2

Copyright (C)
   2007 Heiko Oberdiek
   2016-2019 Oberdiek Package Support Group

This work may be distributed and/or modified under the
conditions of the LaTeX Project Public License, either
version 1.3c of this license or (at your option) any later
version. This version of this license is in
   https://www.latex-project.org/lppl/lppl-1-3c.txt
and the latest version of this license is in
   https://www.latex-project.org/lppl.txt
and version 1.3 or later is part of all distributions of
LaTeX version 2005/12/01 or later.

This work has the LPPL maintenance status "maintained".

The Current Maintainers of this work are
Heiko Oberdiek and the Oberdiek Package Support Group
https://github.com/ho-tex/oberdiek/issues


This work consists of the main source file atenddvi.dtx
and the derived files
   atenddvi.sty, atenddvi.pdf, atenddvi.ins, atenddvi.drv.

\endpreamble
\let\MetaPrefix\DoubleperCent

\generate{%
  \file{atenddvi.ins}{\from{atenddvi.dtx}{install}}%
  \file{atenddvi.drv}{\from{atenddvi.dtx}{driver}}%
  \usedir{tex/latex/oberdiek}%
  \file{atenddvi.sty}{\from{atenddvi.dtx}{package}}%
  \nopreamble
  \nopostamble
%  \usedir{source/latex/oberdiek/catalogue}%
%  \file{atenddvi.xml}{\from{atenddvi.dtx}{catalogue}}%
}

\catcode32=13\relax% active space
\let =\space%
\Msg{************************************************************************}
\Msg{*}
\Msg{* To finish the installation you have to move the following}
\Msg{* file into a directory searched by TeX:}
\Msg{*}
\Msg{*     atenddvi.sty}
\Msg{*}
\Msg{* To produce the documentation run the file `atenddvi.drv'}
\Msg{* through LaTeX.}
\Msg{*}
\Msg{* Happy TeXing!}
\Msg{*}
\Msg{************************************************************************}

\endbatchfile
%</install>
%<*ignore>
\fi
%</ignore>
%<*driver>
\NeedsTeXFormat{LaTeX2e}
\ProvidesFile{atenddvi.drv}%
  [2016/05/16 v1.2 At end DVI hook (HO)]%
\documentclass{ltxdoc}
\usepackage{holtxdoc}[2011/11/22]
\begin{document}
  \DocInput{atenddvi.dtx}%
\end{document}
%</driver>
% \fi
%
%
% \CharacterTable
%  {Upper-case    \A\B\C\D\E\F\G\H\I\J\K\L\M\N\O\P\Q\R\S\T\U\V\W\X\Y\Z
%   Lower-case    \a\b\c\d\e\f\g\h\i\j\k\l\m\n\o\p\q\r\s\t\u\v\w\x\y\z
%   Digits        \0\1\2\3\4\5\6\7\8\9
%   Exclamation   \!     Double quote  \"     Hash (number) \#
%   Dollar        \$     Percent       \%     Ampersand     \&
%   Acute accent  \'     Left paren    \(     Right paren   \)
%   Asterisk      \*     Plus          \+     Comma         \,
%   Minus         \-     Point         \.     Solidus       \/
%   Colon         \:     Semicolon     \;     Less than     \<
%   Equals        \=     Greater than  \>     Question mark \?
%   Commercial at \@     Left bracket  \[     Backslash     \\
%   Right bracket \]     Circumflex    \^     Underscore    \_
%   Grave accent  \`     Left brace    \{     Vertical bar  \|
%   Right brace   \}     Tilde         \~}
%
% \GetFileInfo{atenddvi.drv}
%
% \title{The \xpackage{atenddvi} package}
% \date{2016/05/16 v1.2}
% \author{Heiko Oberdiek\thanks
% {Please report any issues at \url{https://github.com/ho-tex/oberdiek/issues}}}
%
% \maketitle
%
% \begin{abstract}
% \LaTeX\ offers \cs{AtBeginDvi}. This package \xpackage{atenddvi}
% provides the counterpart \cs{AtEndDvi}. The execution of its
% argument is delayed to the end of the document at the end of the
% last page. Thus \cs{special} and \cs{write} remain effective, because
% they are put into the last page. This is the main difference
% to \cs{AtEndDocument}.
% \end{abstract}
%
% \tableofcontents
%
% \section{Documentation}
%
% \begin{declcs}{AtEndDvi} \M{code}
% \end{declcs}
% Macro \cs{AtEndDvi} provides a hook mechanism to put \meta{code}
% at the end of the last output page. It is the logical counterpart
% to \cs{AtBeginDvi}. Despite the name the output type DVI, PDF or whatever
% does not matter.
%
% Unlike \cs{AtBeginDvi} the \meta{code} is not put in a box and
% therefore executed immediately. The hook for \cs{AtEndDvi} is based on
% a macro similar to \cs{AtBeginDocument} or \cs{AtEndDocument}. The
% execution of \meta{code} is delayed until the hook is executed on
% the last page.
%
% Commands such as \cs{special} or \cs{write} (not the \cs{immediate}
% variant) must go as nodes into the contents of a page to have the
% desired effect.
% When the hook for \cs{AtEndDocument} is executed, the last intended
% page may already be shipped out. Therefore \cs{special} or \cs{write}
% cannot be used in a reliable way without generating new page.
%
% This gap is closed by \cs{AtEndDvi} of this package \xpackage{atenddvi}.
% If the document is compiled the first time, the package remembers
% the last page in a reference. In the sceond run, it puts the hook
% on the page that has been detected in the previous run as last page.
% The package detectes if the number of pages has changed, and then
% generates a warning to rerun \LaTeX.
%
% \StopEventually{
% }
%
% \section{Implementation}
%
%    \begin{macrocode}
%<*package>
\NeedsTeXFormat{LaTeX2e}
\ProvidesPackage{atenddvi}%
  [2016/05/16 v1.2 At end DVI hook (HO)]%
%    \end{macrocode}
%
%    Load the required packages
%    \begin{macrocode}
\RequirePackage{zref-abspage,zref-lastpage}[2007/03/19]
\RequirePackage{atbegshi}
%    \end{macrocode}
%
%    \begin{macro}{\AtEndDvi@Hook}
%    Macro \cs{AtEndDvi@Hook} is the data storage macro
%    for the code that is executed later at end of the last page.
%    \begin{macrocode}
\let\AtEndDvi@Hook\@empty
%    \end{macrocode}
%    \end{macro}
%    \begin{macro}{\AtEndDvi}
%    Macro \cs{AtEndDvi} is called in the same way as
%    \cs{AtBeginDocument}. The argument is added to the hook macro.
%    \begin{macrocode}
\newcommand*{\AtEndDvi}{%
  \g@addto@macro\AtEndDvi@Hook
}
%    \end{macrocode}
%    \end{macro}
%
%    \begin{macro}{\AtEndDvi@AtBeginShipout}
%    \begin{macrocode}
\def\AtEndDvi@AtBeginShipout{%
  \begingroup
%    \end{macrocode}
%    The reference `LastPage' is marked used. If the reference
%    is not yet defined, then the user gets the warning because of
%    the undefined reference and the rerun warning at the end of
%    the compile run. However, we do not need a warning each page,
%    the first page is enough.
%    \begin{macrocode}
    \ifnum\value{abspage}=1 %
      \zref@refused{LastPage}%
    \fi
%    \end{macrocode}
%    The current absolute page number is compared with the
%    absolute page number of the reference `LastPage'.
%    \begin{macrocode}
    \ifnum\zref@extractdefault{LastPage}{abspage}{0}=\value{abspage}%
%    \end{macrocode}
%    \begin{macro}{\AtEndDvi@LastPage}
%    We found the right page and remember it in a macro.
%    \begin{macrocode}
      \xdef\AtEndDvi@LastPage{\number\value{abspage}}%
%    \end{macrocode}
%    \end{macro}
%    The hook of \cs{AtEndDvi} is now put on the last page
%    after the contents of the page.
%    \begin{macrocode}
      \global\setbox\AtBeginShipoutBox=\vbox{%
        \hbox{%
          \box\AtBeginShipoutBox
          \setbox\AtBeginShipoutBox=\hbox{%
            \begingroup
              \AtEndDvi@Hook
            \endgroup
          }%
          \wd\AtBeginShipoutBox=\z@
          \ht\AtBeginShipoutBox=\z@
          \dp\AtBeginShipoutBox=\z@
          \box\AtBeginShipoutBox
        }%
      }%
%    \end{macrocode}
%    We do not need the every page hook.
%    \begin{macrocode}
      \global\let\AtEndDvi@AtBeginShipout\@empty
%    \end{macrocode}
%    The hook is consumed, \cs{AtEndDvi} does not have an effect.
%    \begin{macrocode}
      \global\let\AtEndDvi\@gobble
%    \end{macrocode}
%    Make a protocol entry, which page is used by this package
%    as last page.
%    \begin{macrocode}
      \let\on@line\@empty
      \PackageInfo{atenddvi}{Last page = \AtEndDvi@LastPage}%
    \fi
  \endgroup
}
%    \end{macrocode}
%    \end{macro}
%
%    \begin{macro}{\AtEndDvi@AtBeginDocument}
%    In order to get as late as possible in the chain of the
%    every shipout hook, the call of \cs{AtBeginShipout} is delayed.
%    \begin{macrocode}
\def\AtEndDvi@AtBeginDocument{%
  \AtBeginShipout{\AtEndDvi@AtBeginShipout}%
%    \end{macrocode}
%    \begin{macro}{\AtEndDvi@Check}
%    After \cs{AtEndDocument} \LaTeX\ reads its \xfile{.aux} files
%    again. Code in \cs{AtEndDocument} could generate additional
%    pages. This is unlikely by code in the \xfile{.aux} file,
%    thus we use the \xfile{.aux} file to run macro
%    \cs{AtEndDvi@Check} for checking the last page.
%
%    During the first reading of the \xfile{.aux} file,
%    \cs{AtEndDvi@Check} is disabled, its real meaning
%    is assigned afterwards.
%    \begin{macrocode}
  \if@filesw
    \immediate\write\@mainaux{%
      \string\providecommand\string\AtEndDvi@Check{}%
    }%
    \immediate\write\@mainaux{%
      \string\AtEndDvi@Check
    }%
  \fi
  \let\AtEndDvi@Check\AtEndDvi@CheckImpl
}
%    \end{macrocode}
%    \end{macro}
%    \begin{macrocode}
\AtBeginDocument{\AtEndDvi@AtBeginDocument}
%    \end{macrocode}
%    \end{macro}
%
%    \begin{macro}{\AtEndDvi@CheckImpl}
%    First check is whether a last page was found at all.
%    Secondly the found last page is compared with the real last page.
%    \begin{macrocode}
\def\AtEndDvi@CheckImpl{%
  \@ifundefined{AtEndDvi@LastPage}{%
    \PackageWarningNoLine{atenddvi}{%
      Rerun LaTeX, last page not yet found%
    }%
  }{%
    \ifnum\AtEndDvi@LastPage=\value{abspage}%
    \else
      \PackageWarningNoLine{atenddvi}{%
        Rerun LaTeX, last page has changed%
      }%
    \fi
  }%
}
%    \end{macrocode}
%    \end{macro}
%
%    \begin{macrocode}
%</package>
%    \end{macrocode}
%
% \section{Installation}
%
% \subsection{Download}
%
% \paragraph{Package.} This package is available on
% CTAN\footnote{\CTANpkg{atenddvi}}:
% \begin{description}
% \item[\CTAN{macros/latex/contrib/oberdiek/atenddvi.dtx}] The source file.
% \item[\CTAN{macros/latex/contrib/oberdiek/atenddvi.pdf}] Documentation.
% \end{description}
%
%
% \paragraph{Bundle.} All the packages of the bundle `oberdiek'
% are also available in a TDS compliant ZIP archive. There
% the packages are already unpacked and the documentation files
% are generated. The files and directories obey the TDS standard.
% \begin{description}
% \item[\CTANinstall{install/macros/latex/contrib/oberdiek.tds.zip}]
% \end{description}
% \emph{TDS} refers to the standard ``A Directory Structure
% for \TeX\ Files'' (\CTAN{tds/tds.pdf}). Directories
% with \xfile{texmf} in their name are usually organized this way.
%
% \subsection{Bundle installation}
%
% \paragraph{Unpacking.} Unpack the \xfile{oberdiek.tds.zip} in the
% TDS tree (also known as \xfile{texmf} tree) of your choice.
% Example (linux):
% \begin{quote}
%   |unzip oberdiek.tds.zip -d ~/texmf|
% \end{quote}
%
% \subsection{Package installation}
%
% \paragraph{Unpacking.} The \xfile{.dtx} file is a self-extracting
% \docstrip\ archive. The files are extracted by running the
% \xfile{.dtx} through \plainTeX:
% \begin{quote}
%   \verb|tex atenddvi.dtx|
% \end{quote}
%
% \paragraph{TDS.} Now the different files must be moved into
% the different directories in your installation TDS tree
% (also known as \xfile{texmf} tree):
% \begin{quote}
% \def\t{^^A
% \begin{tabular}{@{}>{\ttfamily}l@{ $\rightarrow$ }>{\ttfamily}l@{}}
%   atenddvi.sty & tex/latex/oberdiek/atenddvi.sty\\
%   atenddvi.pdf & doc/latex/oberdiek/atenddvi.pdf\\
%   atenddvi.dtx & source/latex/oberdiek/atenddvi.dtx\\
% \end{tabular}^^A
% }^^A
% \sbox0{\t}^^A
% \ifdim\wd0>\linewidth
%   \begingroup
%     \advance\linewidth by\leftmargin
%     \advance\linewidth by\rightmargin
%   \edef\x{\endgroup
%     \def\noexpand\lw{\the\linewidth}^^A
%   }\x
%   \def\lwbox{^^A
%     \leavevmode
%     \hbox to \linewidth{^^A
%       \kern-\leftmargin\relax
%       \hss
%       \usebox0
%       \hss
%       \kern-\rightmargin\relax
%     }^^A
%   }^^A
%   \ifdim\wd0>\lw
%     \sbox0{\small\t}^^A
%     \ifdim\wd0>\linewidth
%       \ifdim\wd0>\lw
%         \sbox0{\footnotesize\t}^^A
%         \ifdim\wd0>\linewidth
%           \ifdim\wd0>\lw
%             \sbox0{\scriptsize\t}^^A
%             \ifdim\wd0>\linewidth
%               \ifdim\wd0>\lw
%                 \sbox0{\tiny\t}^^A
%                 \ifdim\wd0>\linewidth
%                   \lwbox
%                 \else
%                   \usebox0
%                 \fi
%               \else
%                 \lwbox
%               \fi
%             \else
%               \usebox0
%             \fi
%           \else
%             \lwbox
%           \fi
%         \else
%           \usebox0
%         \fi
%       \else
%         \lwbox
%       \fi
%     \else
%       \usebox0
%     \fi
%   \else
%     \lwbox
%   \fi
% \else
%   \usebox0
% \fi
% \end{quote}
% If you have a \xfile{docstrip.cfg} that configures and enables \docstrip's
% TDS installing feature, then some files can already be in the right
% place, see the documentation of \docstrip.
%
% \subsection{Refresh file name databases}
%
% If your \TeX~distribution
% (\TeX\,Live, \mikTeX, \dots) relies on file name databases, you must refresh
% these. For example, \TeX\,Live\ users run \verb|texhash| or
% \verb|mktexlsr|.
%
% \subsection{Some details for the interested}
%
% \paragraph{Unpacking with \LaTeX.}
% The \xfile{.dtx} chooses its action depending on the format:
% \begin{description}
% \item[\plainTeX:] Run \docstrip\ and extract the files.
% \item[\LaTeX:] Generate the documentation.
% \end{description}
% If you insist on using \LaTeX\ for \docstrip\ (really,
% \docstrip\ does not need \LaTeX), then inform the autodetect routine
% about your intention:
% \begin{quote}
%   \verb|latex \let\install=y% \iffalse meta-comment
%
% File: atenddvi.dtx
% Version: 2016/05/16 v1.2
% Info: At end DVI hook
%
% Copyright (C)
%    2007 Heiko Oberdiek
%    2016-2019 Oberdiek Package Support Group
%    https://github.com/ho-tex/oberdiek/issues
%
% This work may be distributed and/or modified under the
% conditions of the LaTeX Project Public License, either
% version 1.3c of this license or (at your option) any later
% version. This version of this license is in
%    https://www.latex-project.org/lppl/lppl-1-3c.txt
% and the latest version of this license is in
%    https://www.latex-project.org/lppl.txt
% and version 1.3 or later is part of all distributions of
% LaTeX version 2005/12/01 or later.
%
% This work has the LPPL maintenance status "maintained".
%
% The Current Maintainers of this work are
% Heiko Oberdiek and the Oberdiek Package Support Group
% https://github.com/ho-tex/oberdiek/issues
%
% This work consists of the main source file atenddvi.dtx
% and the derived files
%    atenddvi.sty, atenddvi.pdf, atenddvi.ins, atenddvi.drv.
%
% Distribution:
%    CTAN:macros/latex/contrib/oberdiek/atenddvi.dtx
%    CTAN:macros/latex/contrib/oberdiek/atenddvi.pdf
%
% Unpacking:
%    (a) If atenddvi.ins is present:
%           tex atenddvi.ins
%    (b) Without atenddvi.ins:
%           tex atenddvi.dtx
%    (c) If you insist on using LaTeX
%           latex \let\install=y\input{atenddvi.dtx}
%        (quote the arguments according to the demands of your shell)
%
% Documentation:
%    (a) If atenddvi.drv is present:
%           latex atenddvi.drv
%    (b) Without atenddvi.drv:
%           latex atenddvi.dtx; ...
%    The class ltxdoc loads the configuration file ltxdoc.cfg
%    if available. Here you can specify further options, e.g.
%    use A4 as paper format:
%       \PassOptionsToClass{a4paper}{article}
%
%    Programm calls to get the documentation (example):
%       pdflatex atenddvi.dtx
%       makeindex -s gind.ist atenddvi.idx
%       pdflatex atenddvi.dtx
%       makeindex -s gind.ist atenddvi.idx
%       pdflatex atenddvi.dtx
%
% Installation:
%    TDS:tex/latex/oberdiek/atenddvi.sty
%    TDS:doc/latex/oberdiek/atenddvi.pdf
%    TDS:source/latex/oberdiek/atenddvi.dtx
%
%<*ignore>
\begingroup
  \catcode123=1 %
  \catcode125=2 %
  \def\x{LaTeX2e}%
\expandafter\endgroup
\ifcase 0\ifx\install y1\fi\expandafter
         \ifx\csname processbatchFile\endcsname\relax\else1\fi
         \ifx\fmtname\x\else 1\fi\relax
\else\csname fi\endcsname
%</ignore>
%<*install>
\input docstrip.tex
\Msg{************************************************************************}
\Msg{* Installation}
\Msg{* Package: atenddvi 2016/05/16 v1.2 At end DVI hook (HO)}
\Msg{************************************************************************}

\keepsilent
\askforoverwritefalse

\let\MetaPrefix\relax
\preamble

This is a generated file.

Project: atenddvi
Version: 2016/05/16 v1.2

Copyright (C)
   2007 Heiko Oberdiek
   2016-2019 Oberdiek Package Support Group

This work may be distributed and/or modified under the
conditions of the LaTeX Project Public License, either
version 1.3c of this license or (at your option) any later
version. This version of this license is in
   https://www.latex-project.org/lppl/lppl-1-3c.txt
and the latest version of this license is in
   https://www.latex-project.org/lppl.txt
and version 1.3 or later is part of all distributions of
LaTeX version 2005/12/01 or later.

This work has the LPPL maintenance status "maintained".

The Current Maintainers of this work are
Heiko Oberdiek and the Oberdiek Package Support Group
https://github.com/ho-tex/oberdiek/issues


This work consists of the main source file atenddvi.dtx
and the derived files
   atenddvi.sty, atenddvi.pdf, atenddvi.ins, atenddvi.drv.

\endpreamble
\let\MetaPrefix\DoubleperCent

\generate{%
  \file{atenddvi.ins}{\from{atenddvi.dtx}{install}}%
  \file{atenddvi.drv}{\from{atenddvi.dtx}{driver}}%
  \usedir{tex/latex/oberdiek}%
  \file{atenddvi.sty}{\from{atenddvi.dtx}{package}}%
  \nopreamble
  \nopostamble
%  \usedir{source/latex/oberdiek/catalogue}%
%  \file{atenddvi.xml}{\from{atenddvi.dtx}{catalogue}}%
}

\catcode32=13\relax% active space
\let =\space%
\Msg{************************************************************************}
\Msg{*}
\Msg{* To finish the installation you have to move the following}
\Msg{* file into a directory searched by TeX:}
\Msg{*}
\Msg{*     atenddvi.sty}
\Msg{*}
\Msg{* To produce the documentation run the file `atenddvi.drv'}
\Msg{* through LaTeX.}
\Msg{*}
\Msg{* Happy TeXing!}
\Msg{*}
\Msg{************************************************************************}

\endbatchfile
%</install>
%<*ignore>
\fi
%</ignore>
%<*driver>
\NeedsTeXFormat{LaTeX2e}
\ProvidesFile{atenddvi.drv}%
  [2016/05/16 v1.2 At end DVI hook (HO)]%
\documentclass{ltxdoc}
\usepackage{holtxdoc}[2011/11/22]
\begin{document}
  \DocInput{atenddvi.dtx}%
\end{document}
%</driver>
% \fi
%
%
% \CharacterTable
%  {Upper-case    \A\B\C\D\E\F\G\H\I\J\K\L\M\N\O\P\Q\R\S\T\U\V\W\X\Y\Z
%   Lower-case    \a\b\c\d\e\f\g\h\i\j\k\l\m\n\o\p\q\r\s\t\u\v\w\x\y\z
%   Digits        \0\1\2\3\4\5\6\7\8\9
%   Exclamation   \!     Double quote  \"     Hash (number) \#
%   Dollar        \$     Percent       \%     Ampersand     \&
%   Acute accent  \'     Left paren    \(     Right paren   \)
%   Asterisk      \*     Plus          \+     Comma         \,
%   Minus         \-     Point         \.     Solidus       \/
%   Colon         \:     Semicolon     \;     Less than     \<
%   Equals        \=     Greater than  \>     Question mark \?
%   Commercial at \@     Left bracket  \[     Backslash     \\
%   Right bracket \]     Circumflex    \^     Underscore    \_
%   Grave accent  \`     Left brace    \{     Vertical bar  \|
%   Right brace   \}     Tilde         \~}
%
% \GetFileInfo{atenddvi.drv}
%
% \title{The \xpackage{atenddvi} package}
% \date{2016/05/16 v1.2}
% \author{Heiko Oberdiek\thanks
% {Please report any issues at \url{https://github.com/ho-tex/oberdiek/issues}}}
%
% \maketitle
%
% \begin{abstract}
% \LaTeX\ offers \cs{AtBeginDvi}. This package \xpackage{atenddvi}
% provides the counterpart \cs{AtEndDvi}. The execution of its
% argument is delayed to the end of the document at the end of the
% last page. Thus \cs{special} and \cs{write} remain effective, because
% they are put into the last page. This is the main difference
% to \cs{AtEndDocument}.
% \end{abstract}
%
% \tableofcontents
%
% \section{Documentation}
%
% \begin{declcs}{AtEndDvi} \M{code}
% \end{declcs}
% Macro \cs{AtEndDvi} provides a hook mechanism to put \meta{code}
% at the end of the last output page. It is the logical counterpart
% to \cs{AtBeginDvi}. Despite the name the output type DVI, PDF or whatever
% does not matter.
%
% Unlike \cs{AtBeginDvi} the \meta{code} is not put in a box and
% therefore executed immediately. The hook for \cs{AtEndDvi} is based on
% a macro similar to \cs{AtBeginDocument} or \cs{AtEndDocument}. The
% execution of \meta{code} is delayed until the hook is executed on
% the last page.
%
% Commands such as \cs{special} or \cs{write} (not the \cs{immediate}
% variant) must go as nodes into the contents of a page to have the
% desired effect.
% When the hook for \cs{AtEndDocument} is executed, the last intended
% page may already be shipped out. Therefore \cs{special} or \cs{write}
% cannot be used in a reliable way without generating new page.
%
% This gap is closed by \cs{AtEndDvi} of this package \xpackage{atenddvi}.
% If the document is compiled the first time, the package remembers
% the last page in a reference. In the sceond run, it puts the hook
% on the page that has been detected in the previous run as last page.
% The package detectes if the number of pages has changed, and then
% generates a warning to rerun \LaTeX.
%
% \StopEventually{
% }
%
% \section{Implementation}
%
%    \begin{macrocode}
%<*package>
\NeedsTeXFormat{LaTeX2e}
\ProvidesPackage{atenddvi}%
  [2016/05/16 v1.2 At end DVI hook (HO)]%
%    \end{macrocode}
%
%    Load the required packages
%    \begin{macrocode}
\RequirePackage{zref-abspage,zref-lastpage}[2007/03/19]
\RequirePackage{atbegshi}
%    \end{macrocode}
%
%    \begin{macro}{\AtEndDvi@Hook}
%    Macro \cs{AtEndDvi@Hook} is the data storage macro
%    for the code that is executed later at end of the last page.
%    \begin{macrocode}
\let\AtEndDvi@Hook\@empty
%    \end{macrocode}
%    \end{macro}
%    \begin{macro}{\AtEndDvi}
%    Macro \cs{AtEndDvi} is called in the same way as
%    \cs{AtBeginDocument}. The argument is added to the hook macro.
%    \begin{macrocode}
\newcommand*{\AtEndDvi}{%
  \g@addto@macro\AtEndDvi@Hook
}
%    \end{macrocode}
%    \end{macro}
%
%    \begin{macro}{\AtEndDvi@AtBeginShipout}
%    \begin{macrocode}
\def\AtEndDvi@AtBeginShipout{%
  \begingroup
%    \end{macrocode}
%    The reference `LastPage' is marked used. If the reference
%    is not yet defined, then the user gets the warning because of
%    the undefined reference and the rerun warning at the end of
%    the compile run. However, we do not need a warning each page,
%    the first page is enough.
%    \begin{macrocode}
    \ifnum\value{abspage}=1 %
      \zref@refused{LastPage}%
    \fi
%    \end{macrocode}
%    The current absolute page number is compared with the
%    absolute page number of the reference `LastPage'.
%    \begin{macrocode}
    \ifnum\zref@extractdefault{LastPage}{abspage}{0}=\value{abspage}%
%    \end{macrocode}
%    \begin{macro}{\AtEndDvi@LastPage}
%    We found the right page and remember it in a macro.
%    \begin{macrocode}
      \xdef\AtEndDvi@LastPage{\number\value{abspage}}%
%    \end{macrocode}
%    \end{macro}
%    The hook of \cs{AtEndDvi} is now put on the last page
%    after the contents of the page.
%    \begin{macrocode}
      \global\setbox\AtBeginShipoutBox=\vbox{%
        \hbox{%
          \box\AtBeginShipoutBox
          \setbox\AtBeginShipoutBox=\hbox{%
            \begingroup
              \AtEndDvi@Hook
            \endgroup
          }%
          \wd\AtBeginShipoutBox=\z@
          \ht\AtBeginShipoutBox=\z@
          \dp\AtBeginShipoutBox=\z@
          \box\AtBeginShipoutBox
        }%
      }%
%    \end{macrocode}
%    We do not need the every page hook.
%    \begin{macrocode}
      \global\let\AtEndDvi@AtBeginShipout\@empty
%    \end{macrocode}
%    The hook is consumed, \cs{AtEndDvi} does not have an effect.
%    \begin{macrocode}
      \global\let\AtEndDvi\@gobble
%    \end{macrocode}
%    Make a protocol entry, which page is used by this package
%    as last page.
%    \begin{macrocode}
      \let\on@line\@empty
      \PackageInfo{atenddvi}{Last page = \AtEndDvi@LastPage}%
    \fi
  \endgroup
}
%    \end{macrocode}
%    \end{macro}
%
%    \begin{macro}{\AtEndDvi@AtBeginDocument}
%    In order to get as late as possible in the chain of the
%    every shipout hook, the call of \cs{AtBeginShipout} is delayed.
%    \begin{macrocode}
\def\AtEndDvi@AtBeginDocument{%
  \AtBeginShipout{\AtEndDvi@AtBeginShipout}%
%    \end{macrocode}
%    \begin{macro}{\AtEndDvi@Check}
%    After \cs{AtEndDocument} \LaTeX\ reads its \xfile{.aux} files
%    again. Code in \cs{AtEndDocument} could generate additional
%    pages. This is unlikely by code in the \xfile{.aux} file,
%    thus we use the \xfile{.aux} file to run macro
%    \cs{AtEndDvi@Check} for checking the last page.
%
%    During the first reading of the \xfile{.aux} file,
%    \cs{AtEndDvi@Check} is disabled, its real meaning
%    is assigned afterwards.
%    \begin{macrocode}
  \if@filesw
    \immediate\write\@mainaux{%
      \string\providecommand\string\AtEndDvi@Check{}%
    }%
    \immediate\write\@mainaux{%
      \string\AtEndDvi@Check
    }%
  \fi
  \let\AtEndDvi@Check\AtEndDvi@CheckImpl
}
%    \end{macrocode}
%    \end{macro}
%    \begin{macrocode}
\AtBeginDocument{\AtEndDvi@AtBeginDocument}
%    \end{macrocode}
%    \end{macro}
%
%    \begin{macro}{\AtEndDvi@CheckImpl}
%    First check is whether a last page was found at all.
%    Secondly the found last page is compared with the real last page.
%    \begin{macrocode}
\def\AtEndDvi@CheckImpl{%
  \@ifundefined{AtEndDvi@LastPage}{%
    \PackageWarningNoLine{atenddvi}{%
      Rerun LaTeX, last page not yet found%
    }%
  }{%
    \ifnum\AtEndDvi@LastPage=\value{abspage}%
    \else
      \PackageWarningNoLine{atenddvi}{%
        Rerun LaTeX, last page has changed%
      }%
    \fi
  }%
}
%    \end{macrocode}
%    \end{macro}
%
%    \begin{macrocode}
%</package>
%    \end{macrocode}
%
% \section{Installation}
%
% \subsection{Download}
%
% \paragraph{Package.} This package is available on
% CTAN\footnote{\CTANpkg{atenddvi}}:
% \begin{description}
% \item[\CTAN{macros/latex/contrib/oberdiek/atenddvi.dtx}] The source file.
% \item[\CTAN{macros/latex/contrib/oberdiek/atenddvi.pdf}] Documentation.
% \end{description}
%
%
% \paragraph{Bundle.} All the packages of the bundle `oberdiek'
% are also available in a TDS compliant ZIP archive. There
% the packages are already unpacked and the documentation files
% are generated. The files and directories obey the TDS standard.
% \begin{description}
% \item[\CTANinstall{install/macros/latex/contrib/oberdiek.tds.zip}]
% \end{description}
% \emph{TDS} refers to the standard ``A Directory Structure
% for \TeX\ Files'' (\CTAN{tds/tds.pdf}). Directories
% with \xfile{texmf} in their name are usually organized this way.
%
% \subsection{Bundle installation}
%
% \paragraph{Unpacking.} Unpack the \xfile{oberdiek.tds.zip} in the
% TDS tree (also known as \xfile{texmf} tree) of your choice.
% Example (linux):
% \begin{quote}
%   |unzip oberdiek.tds.zip -d ~/texmf|
% \end{quote}
%
% \subsection{Package installation}
%
% \paragraph{Unpacking.} The \xfile{.dtx} file is a self-extracting
% \docstrip\ archive. The files are extracted by running the
% \xfile{.dtx} through \plainTeX:
% \begin{quote}
%   \verb|tex atenddvi.dtx|
% \end{quote}
%
% \paragraph{TDS.} Now the different files must be moved into
% the different directories in your installation TDS tree
% (also known as \xfile{texmf} tree):
% \begin{quote}
% \def\t{^^A
% \begin{tabular}{@{}>{\ttfamily}l@{ $\rightarrow$ }>{\ttfamily}l@{}}
%   atenddvi.sty & tex/latex/oberdiek/atenddvi.sty\\
%   atenddvi.pdf & doc/latex/oberdiek/atenddvi.pdf\\
%   atenddvi.dtx & source/latex/oberdiek/atenddvi.dtx\\
% \end{tabular}^^A
% }^^A
% \sbox0{\t}^^A
% \ifdim\wd0>\linewidth
%   \begingroup
%     \advance\linewidth by\leftmargin
%     \advance\linewidth by\rightmargin
%   \edef\x{\endgroup
%     \def\noexpand\lw{\the\linewidth}^^A
%   }\x
%   \def\lwbox{^^A
%     \leavevmode
%     \hbox to \linewidth{^^A
%       \kern-\leftmargin\relax
%       \hss
%       \usebox0
%       \hss
%       \kern-\rightmargin\relax
%     }^^A
%   }^^A
%   \ifdim\wd0>\lw
%     \sbox0{\small\t}^^A
%     \ifdim\wd0>\linewidth
%       \ifdim\wd0>\lw
%         \sbox0{\footnotesize\t}^^A
%         \ifdim\wd0>\linewidth
%           \ifdim\wd0>\lw
%             \sbox0{\scriptsize\t}^^A
%             \ifdim\wd0>\linewidth
%               \ifdim\wd0>\lw
%                 \sbox0{\tiny\t}^^A
%                 \ifdim\wd0>\linewidth
%                   \lwbox
%                 \else
%                   \usebox0
%                 \fi
%               \else
%                 \lwbox
%               \fi
%             \else
%               \usebox0
%             \fi
%           \else
%             \lwbox
%           \fi
%         \else
%           \usebox0
%         \fi
%       \else
%         \lwbox
%       \fi
%     \else
%       \usebox0
%     \fi
%   \else
%     \lwbox
%   \fi
% \else
%   \usebox0
% \fi
% \end{quote}
% If you have a \xfile{docstrip.cfg} that configures and enables \docstrip's
% TDS installing feature, then some files can already be in the right
% place, see the documentation of \docstrip.
%
% \subsection{Refresh file name databases}
%
% If your \TeX~distribution
% (\TeX\,Live, \mikTeX, \dots) relies on file name databases, you must refresh
% these. For example, \TeX\,Live\ users run \verb|texhash| or
% \verb|mktexlsr|.
%
% \subsection{Some details for the interested}
%
% \paragraph{Unpacking with \LaTeX.}
% The \xfile{.dtx} chooses its action depending on the format:
% \begin{description}
% \item[\plainTeX:] Run \docstrip\ and extract the files.
% \item[\LaTeX:] Generate the documentation.
% \end{description}
% If you insist on using \LaTeX\ for \docstrip\ (really,
% \docstrip\ does not need \LaTeX), then inform the autodetect routine
% about your intention:
% \begin{quote}
%   \verb|latex \let\install=y\input{atenddvi.dtx}|
% \end{quote}
% Do not forget to quote the argument according to the demands
% of your shell.
%
% \paragraph{Generating the documentation.}
% You can use both the \xfile{.dtx} or the \xfile{.drv} to generate
% the documentation. The process can be configured by the
% configuration file \xfile{ltxdoc.cfg}. For instance, put this
% line into this file, if you want to have A4 as paper format:
% \begin{quote}
%   \verb|\PassOptionsToClass{a4paper}{article}|
% \end{quote}
% An example follows how to generate the
% documentation with pdf\LaTeX:
% \begin{quote}
%\begin{verbatim}
%pdflatex atenddvi.dtx
%makeindex -s gind.ist atenddvi.idx
%pdflatex atenddvi.dtx
%makeindex -s gind.ist atenddvi.idx
%pdflatex atenddvi.dtx
%\end{verbatim}
% \end{quote}
%
% \begin{History}
%   \begin{Version}{2007/03/20 v1.0}
%   \item
%     First version.
%   \end{Version}
%   \begin{Version}{2007/04/17 v1.1}
%   \item
%     Package \xpackage{atbegshi} replaces package \xpackage{everyshi}.
%   \end{Version}
%   \begin{Version}{2016/05/16 v1.2}
%   \item
%     Documentation updates.
%   \end{Version}
% \end{History}
%
% \PrintIndex
%
% \Finale
\endinput
|
% \end{quote}
% Do not forget to quote the argument according to the demands
% of your shell.
%
% \paragraph{Generating the documentation.}
% You can use both the \xfile{.dtx} or the \xfile{.drv} to generate
% the documentation. The process can be configured by the
% configuration file \xfile{ltxdoc.cfg}. For instance, put this
% line into this file, if you want to have A4 as paper format:
% \begin{quote}
%   \verb|\PassOptionsToClass{a4paper}{article}|
% \end{quote}
% An example follows how to generate the
% documentation with pdf\LaTeX:
% \begin{quote}
%\begin{verbatim}
%pdflatex atenddvi.dtx
%makeindex -s gind.ist atenddvi.idx
%pdflatex atenddvi.dtx
%makeindex -s gind.ist atenddvi.idx
%pdflatex atenddvi.dtx
%\end{verbatim}
% \end{quote}
%
% \begin{History}
%   \begin{Version}{2007/03/20 v1.0}
%   \item
%     First version.
%   \end{Version}
%   \begin{Version}{2007/04/17 v1.1}
%   \item
%     Package \xpackage{atbegshi} replaces package \xpackage{everyshi}.
%   \end{Version}
%   \begin{Version}{2016/05/16 v1.2}
%   \item
%     Documentation updates.
%   \end{Version}
% \end{History}
%
% \PrintIndex
%
% \Finale
\endinput
|
% \end{quote}
% Do not forget to quote the argument according to the demands
% of your shell.
%
% \paragraph{Generating the documentation.}
% You can use both the \xfile{.dtx} or the \xfile{.drv} to generate
% the documentation. The process can be configured by the
% configuration file \xfile{ltxdoc.cfg}. For instance, put this
% line into this file, if you want to have A4 as paper format:
% \begin{quote}
%   \verb|\PassOptionsToClass{a4paper}{article}|
% \end{quote}
% An example follows how to generate the
% documentation with pdf\LaTeX:
% \begin{quote}
%\begin{verbatim}
%pdflatex atenddvi.dtx
%makeindex -s gind.ist atenddvi.idx
%pdflatex atenddvi.dtx
%makeindex -s gind.ist atenddvi.idx
%pdflatex atenddvi.dtx
%\end{verbatim}
% \end{quote}
%
% \begin{History}
%   \begin{Version}{2007/03/20 v1.0}
%   \item
%     First version.
%   \end{Version}
%   \begin{Version}{2007/04/17 v1.1}
%   \item
%     Package \xpackage{atbegshi} replaces package \xpackage{everyshi}.
%   \end{Version}
%   \begin{Version}{2016/05/16 v1.2}
%   \item
%     Documentation updates.
%   \end{Version}
% \end{History}
%
% \PrintIndex
%
% \Finale
\endinput
|
% \end{quote}
% Do not forget to quote the argument according to the demands
% of your shell.
%
% \paragraph{Generating the documentation.}
% You can use both the \xfile{.dtx} or the \xfile{.drv} to generate
% the documentation. The process can be configured by the
% configuration file \xfile{ltxdoc.cfg}. For instance, put this
% line into this file, if you want to have A4 as paper format:
% \begin{quote}
%   \verb|\PassOptionsToClass{a4paper}{article}|
% \end{quote}
% An example follows how to generate the
% documentation with pdf\LaTeX:
% \begin{quote}
%\begin{verbatim}
%pdflatex atenddvi.dtx
%makeindex -s gind.ist atenddvi.idx
%pdflatex atenddvi.dtx
%makeindex -s gind.ist atenddvi.idx
%pdflatex atenddvi.dtx
%\end{verbatim}
% \end{quote}
%
% \section{Catalogue}
%
% The following XML file can be used as source for the
% \href{http://mirror.ctan.org/help/Catalogue/catalogue.html}{\TeX\ Catalogue}.
% The elements \texttt{caption} and \texttt{description} are imported
% from the original XML file from the Catalogue.
% The name of the XML file in the Catalogue is \xfile{atenddvi.xml}.
%    \begin{macrocode}
%<*catalogue>
<?xml version='1.0' encoding='us-ascii'?>
<!DOCTYPE entry SYSTEM 'catalogue.dtd'>
<entry datestamp='$Date$' modifier='$Author$' id='atenddvi'>
  <name>atenddvi</name>
  <caption>Provides the \AtEndDvi command.</caption>
  <authorref id='auth:oberdiek'/>
  <copyright owner='Heiko Oberdiek' year='2007'/>
  <license type='lppl1.3'/>
  <version number='1.2'/>
  <description>
    LaTeX offers <tt>\AtBeginDvi</tt>.  This package provides the
    counterpart <tt>\AtEndDvi</tt>. The execution of its argument is
    delayed to the end of the document at the end of the last page.
    At this point <tt>\special</tt> and <tt>\write</tt> remain
    effective, because they are put into the last page.  This is the
    main difference from the LaTeX command <tt>\AtEndDocument</tt>.
    <p/>
    The package is part of the <xref refid='oberdiek'>oberdiek</xref> bundle.
  </description>
  <documentation details='Package documentation'
      href='ctan:/macros/latex/contrib/oberdiek/atenddvi.pdf'/>
  <ctan file='true' path='/macros/latex/contrib/oberdiek/atenddvi.dtx'/>
  <miktex location='oberdiek'/>
  <texlive location='oberdiek'/>
  <install path='/macros/latex/contrib/oberdiek/oberdiek.tds.zip'/>
</entry>
%</catalogue>
%    \end{macrocode}
%
% \begin{History}
%   \begin{Version}{2007/03/20 v1.0}
%   \item
%     First version.
%   \end{Version}
%   \begin{Version}{2007/04/17 v1.1}
%   \item
%     Package \xpackage{atbegshi} replaces package \xpackage{everyshi}.
%   \end{Version}
%   \begin{Version}{2016/05/16 v1.2}
%   \item
%     Documentation updates.
%   \end{Version}
% \end{History}
%
% \PrintIndex
%
% \Finale
\endinput
