% \iffalse meta-comment
%
% File: resizegather.dtx
% Version: 2016/05/16 v1.3
% Info: Resize overly large equations
%
% Copyright (C) 2009, 2010 by
%    Heiko Oberdiek <heiko.oberdiek at googlemail.com>
%    2016
%    https://github.com/ho-tex/oberdiek/issues
%
% This work may be distributed and/or modified under the
% conditions of the LaTeX Project Public License, either
% version 1.3c of this license or (at your option) any later
% version. This version of this license is in
%    http://www.latex-project.org/lppl/lppl-1-3c.txt
% and the latest version of this license is in
%    http://www.latex-project.org/lppl.txt
% and version 1.3 or later is part of all distributions of
% LaTeX version 2005/12/01 or later.
%
% This work has the LPPL maintenance status "maintained".
%
% This Current Maintainer of this work is Heiko Oberdiek.
%
% This work consists of the main source file resizegather.dtx
% and the derived files
%    resizegather.sty, resizegather.pdf, resizegather.ins, resizegather.drv,
%    resizegather-test1.tex.
%
% Distribution:
%    CTAN:macros/latex/contrib/oberdiek/resizegather.dtx
%    CTAN:macros/latex/contrib/oberdiek/resizegather.pdf
%
% Unpacking:
%    (a) If resizegather.ins is present:
%           tex resizegather.ins
%    (b) Without resizegather.ins:
%           tex resizegather.dtx
%    (c) If you insist on using LaTeX
%           latex \let\install=y% \iffalse meta-comment
%
% File: resizegather.dtx
% Version: 2016/05/16 v1.3
% Info: Resize overly large equations
%
% Copyright (C) 2009, 2010 by
%    Heiko Oberdiek <heiko.oberdiek at googlemail.com>
%    2016
%    https://github.com/ho-tex/oberdiek/issues
%
% This work may be distributed and/or modified under the
% conditions of the LaTeX Project Public License, either
% version 1.3c of this license or (at your option) any later
% version. This version of this license is in
%    http://www.latex-project.org/lppl/lppl-1-3c.txt
% and the latest version of this license is in
%    http://www.latex-project.org/lppl.txt
% and version 1.3 or later is part of all distributions of
% LaTeX version 2005/12/01 or later.
%
% This work has the LPPL maintenance status "maintained".
%
% This Current Maintainer of this work is Heiko Oberdiek.
%
% This work consists of the main source file resizegather.dtx
% and the derived files
%    resizegather.sty, resizegather.pdf, resizegather.ins, resizegather.drv,
%    resizegather-test1.tex.
%
% Distribution:
%    CTAN:macros/latex/contrib/oberdiek/resizegather.dtx
%    CTAN:macros/latex/contrib/oberdiek/resizegather.pdf
%
% Unpacking:
%    (a) If resizegather.ins is present:
%           tex resizegather.ins
%    (b) Without resizegather.ins:
%           tex resizegather.dtx
%    (c) If you insist on using LaTeX
%           latex \let\install=y% \iffalse meta-comment
%
% File: resizegather.dtx
% Version: 2016/05/16 v1.3
% Info: Resize overly large equations
%
% Copyright (C) 2009, 2010 by
%    Heiko Oberdiek <heiko.oberdiek at googlemail.com>
%    2016
%    https://github.com/ho-tex/oberdiek/issues
%
% This work may be distributed and/or modified under the
% conditions of the LaTeX Project Public License, either
% version 1.3c of this license or (at your option) any later
% version. This version of this license is in
%    http://www.latex-project.org/lppl/lppl-1-3c.txt
% and the latest version of this license is in
%    http://www.latex-project.org/lppl.txt
% and version 1.3 or later is part of all distributions of
% LaTeX version 2005/12/01 or later.
%
% This work has the LPPL maintenance status "maintained".
%
% This Current Maintainer of this work is Heiko Oberdiek.
%
% This work consists of the main source file resizegather.dtx
% and the derived files
%    resizegather.sty, resizegather.pdf, resizegather.ins, resizegather.drv,
%    resizegather-test1.tex.
%
% Distribution:
%    CTAN:macros/latex/contrib/oberdiek/resizegather.dtx
%    CTAN:macros/latex/contrib/oberdiek/resizegather.pdf
%
% Unpacking:
%    (a) If resizegather.ins is present:
%           tex resizegather.ins
%    (b) Without resizegather.ins:
%           tex resizegather.dtx
%    (c) If you insist on using LaTeX
%           latex \let\install=y% \iffalse meta-comment
%
% File: resizegather.dtx
% Version: 2016/05/16 v1.3
% Info: Resize overly large equations
%
% Copyright (C) 2009, 2010 by
%    Heiko Oberdiek <heiko.oberdiek at googlemail.com>
%    2016
%    https://github.com/ho-tex/oberdiek/issues
%
% This work may be distributed and/or modified under the
% conditions of the LaTeX Project Public License, either
% version 1.3c of this license or (at your option) any later
% version. This version of this license is in
%    http://www.latex-project.org/lppl/lppl-1-3c.txt
% and the latest version of this license is in
%    http://www.latex-project.org/lppl.txt
% and version 1.3 or later is part of all distributions of
% LaTeX version 2005/12/01 or later.
%
% This work has the LPPL maintenance status "maintained".
%
% This Current Maintainer of this work is Heiko Oberdiek.
%
% This work consists of the main source file resizegather.dtx
% and the derived files
%    resizegather.sty, resizegather.pdf, resizegather.ins, resizegather.drv,
%    resizegather-test1.tex.
%
% Distribution:
%    CTAN:macros/latex/contrib/oberdiek/resizegather.dtx
%    CTAN:macros/latex/contrib/oberdiek/resizegather.pdf
%
% Unpacking:
%    (a) If resizegather.ins is present:
%           tex resizegather.ins
%    (b) Without resizegather.ins:
%           tex resizegather.dtx
%    (c) If you insist on using LaTeX
%           latex \let\install=y\input{resizegather.dtx}
%        (quote the arguments according to the demands of your shell)
%
% Documentation:
%    (a) If resizegather.drv is present:
%           latex resizegather.drv
%    (b) Without resizegather.drv:
%           latex resizegather.dtx; ...
%    The class ltxdoc loads the configuration file ltxdoc.cfg
%    if available. Here you can specify further options, e.g.
%    use A4 as paper format:
%       \PassOptionsToClass{a4paper}{article}
%
%    Programm calls to get the documentation (example):
%       pdflatex resizegather.dtx
%       makeindex -s gind.ist resizegather.idx
%       pdflatex resizegather.dtx
%       makeindex -s gind.ist resizegather.idx
%       pdflatex resizegather.dtx
%
% Installation:
%    TDS:tex/latex/oberdiek/resizegather.sty
%    TDS:doc/latex/oberdiek/resizegather.pdf
%    TDS:doc/latex/oberdiek/test/resizegather-test1.tex
%    TDS:source/latex/oberdiek/resizegather.dtx
%
%<*ignore>
\begingroup
  \catcode123=1 %
  \catcode125=2 %
  \def\x{LaTeX2e}%
\expandafter\endgroup
\ifcase 0\ifx\install y1\fi\expandafter
         \ifx\csname processbatchFile\endcsname\relax\else1\fi
         \ifx\fmtname\x\else 1\fi\relax
\else\csname fi\endcsname
%</ignore>
%<*install>
\input docstrip.tex
\Msg{************************************************************************}
\Msg{* Installation}
\Msg{* Package: resizegather 2016/05/16 v1.3 Resize overly large equations (HO)}
\Msg{************************************************************************}

\keepsilent
\askforoverwritefalse

\let\MetaPrefix\relax
\preamble

This is a generated file.

Project: resizegather
Version: 2016/05/16 v1.3

Copyright (C) 2009, 2010 by
   Heiko Oberdiek <heiko.oberdiek at googlemail.com>

This work may be distributed and/or modified under the
conditions of the LaTeX Project Public License, either
version 1.3c of this license or (at your option) any later
version. This version of this license is in
   http://www.latex-project.org/lppl/lppl-1-3c.txt
and the latest version of this license is in
   http://www.latex-project.org/lppl.txt
and version 1.3 or later is part of all distributions of
LaTeX version 2005/12/01 or later.

This work has the LPPL maintenance status "maintained".

This Current Maintainer of this work is Heiko Oberdiek.

This work consists of the main source file resizegather.dtx
and the derived files
   resizegather.sty, resizegather.pdf, resizegather.ins, resizegather.drv,
   resizegather-test1.tex.

\endpreamble
\let\MetaPrefix\DoubleperCent

\generate{%
  \file{resizegather.ins}{\from{resizegather.dtx}{install}}%
  \file{resizegather.drv}{\from{resizegather.dtx}{driver}}%
  \usedir{tex/latex/oberdiek}%
  \file{resizegather.sty}{\from{resizegather.dtx}{package}}%
%  \usedir{doc/latex/oberdiek/test}%
%  \file{resizegather-test1.tex}{\from{resizegather.dtx}{test1}}%
  \nopreamble
  \nopostamble
%  \usedir{source/latex/oberdiek/catalogue}%
%  \file{resizegather.xml}{\from{resizegather.dtx}{catalogue}}%
}

\catcode32=13\relax% active space
\let =\space%
\Msg{************************************************************************}
\Msg{*}
\Msg{* To finish the installation you have to move the following}
\Msg{* file into a directory searched by TeX:}
\Msg{*}
\Msg{*     resizegather.sty}
\Msg{*}
\Msg{* To produce the documentation run the file `resizegather.drv'}
\Msg{* through LaTeX.}
\Msg{*}
\Msg{* Happy TeXing!}
\Msg{*}
\Msg{************************************************************************}

\endbatchfile
%</install>
%<*ignore>
\fi
%</ignore>
%<*driver>
\NeedsTeXFormat{LaTeX2e}
\ProvidesFile{resizegather.drv}%
  [2016/05/16 v1.3 Resize overly large equations (HO)]%
\documentclass{ltxdoc}
\usepackage{holtxdoc}[2011/11/22]
\usepackage{ifluatex}
\ifluatex
\else
  \usepackage[T1]{fontenc}%
  \usepackage{textcomp}%
  \usepackage{lmodern}%
\fi
\begin{document}
  \DocInput{resizegather.dtx}%
\end{document}
%</driver>
% \fi
%
%
% \CharacterTable
%  {Upper-case    \A\B\C\D\E\F\G\H\I\J\K\L\M\N\O\P\Q\R\S\T\U\V\W\X\Y\Z
%   Lower-case    \a\b\c\d\e\f\g\h\i\j\k\l\m\n\o\p\q\r\s\t\u\v\w\x\y\z
%   Digits        \0\1\2\3\4\5\6\7\8\9
%   Exclamation   \!     Double quote  \"     Hash (number) \#
%   Dollar        \$     Percent       \%     Ampersand     \&
%   Acute accent  \'     Left paren    \(     Right paren   \)
%   Asterisk      \*     Plus          \+     Comma         \,
%   Minus         \-     Point         \.     Solidus       \/
%   Colon         \:     Semicolon     \;     Less than     \<
%   Equals        \=     Greater than  \>     Question mark \?
%   Commercial at \@     Left bracket  \[     Backslash     \\
%   Right bracket \]     Circumflex    \^     Underscore    \_
%   Grave accent  \`     Left brace    \{     Vertical bar  \|
%   Right brace   \}     Tilde         \~}
%
% \GetFileInfo{resizegather.drv}
%
% \title{The \xpackage{resizegather} package}
% \date{2016/05/16 v1.3}
% \author{Heiko Oberdiek\thanks
% {Please report any issues at https://github.com/ho-tex/oberdiek/issues}\\
% \xemail{heiko.oberdiek at googlemail.com}}
%
% \maketitle
%
% \begin{abstract}
% Equations that are too large are resized to fit the available
% space. The environment \textsf{gather} of package \xpackage{amsmath}
% is supported. Also the environments \textsf{equation} and
% \textsf{displaymath} are redefined using \textsf{gather}
% and its starred version.
% \end{abstract}
%
% \tableofcontents
%
% \makeatletter
% \def\env#1{^^A
%    \textsf{\@env#1*\@nil}^^A
% }%
% \def\@env#1*#2\@nil{^^A
%   #1^^A
%   \ifx\\#2\\^^A
%     \expandafter\@gobble
%   \else
%     \textasteriskcentered
%     \expandafter\@firstofone
%   \fi
%   {\@env#2\@nil}^^A
% }
% \makeatother
%
% \section{Documentation}
%
% Sometimes an equation is just a little to large to fit in the
% line. And breaking the equation across lines might be worse
% than downscaling the equation. This package implements this
% for the environments \env{gather} and \env{gather*} of
% package \xpackage{amsmath}. That package already measures
% the equations and simplifies the implementation of \xpackage{resizegather}
% that only needs to hook into \xpackage{amsmath}'s code to add
% the resizing feature.
%
% Resized equations are recorded in the \xext{log} file
% for small exceeds (default setting is smaller than five percent).
% Otherwise a warning is given.
%
% Also environments \env{equation} and \env{displaymath}
% are supported by redefining them using \env{gather}
% and \env{gather*}.
%
% \cs{[} and \cs{]} are not supported, because these macros
% are not in environment form that is required for
% \xpackage{amsmath}. The environment body is collected
% first to be able to process the body twice for measuring
% first.
%
% Also the environments using alignments are not supported.
% If a single equation line would be resized, the alignment
% would get lost. And resizing all equations of the alignment
% does not seem appropriate either.
%
% \subsection{Options}
%
% \begin{description}
% \item[\xoption{warningthreshold}:]
%   Print a warning if the original equation line exceeds
%   its available width by the given fraction.
%   Default is |0.05|: A warning is given if the equation
%   is too large by five percent.
%   Otherwise the exceed is recorded in the \xext{log} file
%   only.
% \end{description}
% The next options are boolean options. They are enabled
% by value |true| or if no value is given. They are switched
% off by value |false|.
% \begin{description}
% \item[\xoption{enable}:] The resize feature is active (default).
% \item[\xoption{disable}:] The complementary option for \xoption{enable},
%  added for convenience: |disable| (or |disable=true|) is the same
%  as |enable=false|.
% \item[\xoption{equations}:]
%   \LaTeX\ environments \textsf{equation} and \textsf{displaymath}
%   environments are redefined. These equations
%   are now using environment \env{gather} and
%   \env{gather*}. This is the default.
% \end{description}
% The following table shows additional options if you
% want to have finer control for the redefined
% environments:
% \begin{quote}
% \def\unchanged{\textit{unchanged}}
% \def\notprovided{\textit{not provided}}
% \begin{tabular}{l|ll}
% &\multicolumn{2}{c}{Environments}\\
% Option & \env{equation} & \env{displaymath}\\
% \hline
% \xoption{equations} & \env{gather} & \env{gather*}\\
% \xoption{equation} & \env{gather} & \unchanged\\
% \xoption{displaymath} & \unchanged & \env{gather*}\\
% \end{tabular}
% \end{quote}
% If such an option is switched off, the original meaning
% of the affected environments is restored.
%
% Options are evaluated in the following order:
% \begin{enumerate}
% \item
%  Configuration file \xfile{resizegather.cfg} using \cs{resizegathersetup}
%  if the file exists.
%  \item
%  Package options given for \cs{usepackage}.
%  \item
%  Later calls of \cs{resizegathersetup}.
% \end{enumerate}
% \begin{declcs}{resizegathersetup}\M{option list}
% \end{declcs}
% The options are key value options. Boolean options are enabled by
% default (without value) or by using the explicit value \texttt{true}.
% Value \texttt{false} disable the option.
%
% \subsection{Options for packages \xpackage{amsmath} or \xpackage{graphics}}
%
% The package loads the package \xpackage{amsmath} because is internally
% measures the equations first. Thus this package hooks into this code
% in order to resize the equations if they are too large.
% The resizing itself is done by \cs{resizebox} of package \xpackage{graphics}.
% If you need special options for these packages, just load them first or
% use global options when appropriate. Example:
% \begin{quote}
%\begin{verbatim}
%\usepackage[dvipdfm]{graphicx}% or graphics
%\usepackage[fleqn]{amsmath}
%\usepackage{resizegather}
%\end{verbatim}
%\end{quote}
%
% \StopEventually{
% }
%
% \section{Implementation}
%    \begin{macrocode}
%<*package>
%    \end{macrocode}
%    Reload check, especially if the package is not used with \LaTeX.
%    \begin{macrocode}
\begingroup\catcode61\catcode48\catcode32=10\relax%
  \catcode13=5 % ^^M
  \endlinechar=13 %
  \catcode35=6 % #
  \catcode39=12 % '
  \catcode44=12 % ,
  \catcode45=12 % -
  \catcode46=12 % .
  \catcode58=12 % :
  \catcode64=11 % @
  \catcode123=1 % {
  \catcode125=2 % }
  \expandafter\let\expandafter\x\csname ver@resizegather.sty\endcsname
  \ifx\x\relax % plain-TeX, first loading
  \else
    \def\empty{}%
    \ifx\x\empty % LaTeX, first loading,
      % variable is initialized, but \ProvidesPackage not yet seen
    \else
      \expandafter\ifx\csname PackageInfo\endcsname\relax
        \def\x#1#2{%
          \immediate\write-1{Package #1 Info: #2.}%
        }%
      \else
        \def\x#1#2{\PackageInfo{#1}{#2, stopped}}%
      \fi
      \x{resizegather}{The package is already loaded}%
      \aftergroup\endinput
    \fi
  \fi
\endgroup%
%    \end{macrocode}
%    Package identification:
%    \begin{macrocode}
\begingroup\catcode61\catcode48\catcode32=10\relax%
  \catcode13=5 % ^^M
  \endlinechar=13 %
  \catcode35=6 % #
  \catcode39=12 % '
  \catcode40=12 % (
  \catcode41=12 % )
  \catcode44=12 % ,
  \catcode45=12 % -
  \catcode46=12 % .
  \catcode47=12 % /
  \catcode58=12 % :
  \catcode64=11 % @
  \catcode91=12 % [
  \catcode93=12 % ]
  \catcode123=1 % {
  \catcode125=2 % }
  \expandafter\ifx\csname ProvidesPackage\endcsname\relax
    \def\x#1#2#3[#4]{\endgroup
      \immediate\write-1{Package: #3 #4}%
      \xdef#1{#4}%
    }%
  \else
    \def\x#1#2[#3]{\endgroup
      #2[{#3}]%
      \ifx#1\@undefined
        \xdef#1{#3}%
      \fi
      \ifx#1\relax
        \xdef#1{#3}%
      \fi
    }%
  \fi
\expandafter\x\csname ver@resizegather.sty\endcsname
\ProvidesPackage{resizegather}%
  [2016/05/16 v1.3 Resize overly large equations (HO)]%
%    \end{macrocode}
%
%    \begin{macrocode}
\begingroup\catcode61\catcode48\catcode32=10\relax%
  \catcode13=5 % ^^M
  \endlinechar=13 %
  \catcode123=1 % {
  \catcode125=2 % }
  \catcode64=11 % @
  \def\x{\endgroup
    \expandafter\edef\csname ResizeGather@AtEnd\endcsname{%
      \endlinechar=\the\endlinechar\relax
      \catcode13=\the\catcode13\relax
      \catcode32=\the\catcode32\relax
      \catcode35=\the\catcode35\relax
      \catcode61=\the\catcode61\relax
      \catcode64=\the\catcode64\relax
      \catcode123=\the\catcode123\relax
      \catcode125=\the\catcode125\relax
    }%
  }%
\x\catcode61\catcode48\catcode32=10\relax%
\catcode13=5 % ^^M
\endlinechar=13 %
\catcode35=6 % #
\catcode64=11 % @
\catcode123=1 % {
\catcode125=2 % }
\def\TMP@EnsureCode#1#2{%
  \edef\ResizeGather@AtEnd{%
    \ResizeGather@AtEnd
    \catcode#1=\the\catcode#1\relax
  }%
  \catcode#1=#2\relax
}
\TMP@EnsureCode{10}{12}% ^^J
\TMP@EnsureCode{33}{12}% !
\TMP@EnsureCode{36}{3}% $
\TMP@EnsureCode{38}{4}% &
\TMP@EnsureCode{39}{12}% '
\TMP@EnsureCode{40}{12}% (
\TMP@EnsureCode{41}{12}% )
\TMP@EnsureCode{42}{12}% *
\TMP@EnsureCode{43}{12}% +
\TMP@EnsureCode{44}{12}% ,
\TMP@EnsureCode{45}{12}% -
\TMP@EnsureCode{46}{12}% .
\TMP@EnsureCode{47}{12}% /
\TMP@EnsureCode{58}{12}% :
\TMP@EnsureCode{59}{12}% ;
\TMP@EnsureCode{60}{12}% <
\TMP@EnsureCode{62}{12}% >
\TMP@EnsureCode{63}{12}% ?
\TMP@EnsureCode{91}{12}% [
\TMP@EnsureCode{93}{12}% ]
\TMP@EnsureCode{94}{7}% ^ (superscript)
\TMP@EnsureCode{96}{12}% `
\edef\ResizeGather@AtEnd{\ResizeGather@AtEnd\noexpand\endinput}
%    \end{macrocode}
%
%    \begin{macrocode}
\RequirePackage{kvoptions}[2009/12/04]
\SetupKeyvalOptions{%
  family=resizegather,%
  prefix=ResizeGather@,%
}
%    \end{macrocode}
%    \begin{macrocode}
\@for\ResizeGather@option:=%
  centertags,%
  tbtags,%
  sumlimits,%
  nosumlimits,%
  intlimits,%
  nointlimits,%
  nonamelimits,%
  leqno,%
  reqno,%
  fleqn%
\do{%
  \edef\ResizeGather@temp{%
    \noexpand\DeclareVoidOption{\ResizeGather@option}{%
      \noexpand\PassOptionsToPackage{amsmath}{\ResizeGather@option}%
    }%
    \noexpand\AtEndOfPackage{%
      \noexpand\DisableKeyvalOption[%
        action=error,%
        package=resizegather,%
      ]{resizegather}{\ResizeGather@option}%
    }%
  }%
  \ResizeGather@temp
}
\@for\ResizeGather@option:=%
  draft,%
  final,%
  hiderotate,%
  hidescale,%
  hiresbb,%
  demo,%
  dvips,xdvi,dvipdf,dvipdfm,dvipdfmx,pdftex,dvipsone,%
  dviwindo,emtex,dviwin,pctexps,pctexwin,pctexhp,pctex32,%
  truetex,tcidvi,vtex,oztex,textures,xetex%
\do{%
  \edef\ResizeGather@temp{%
    \noexpand\DeclareVoidOption{\ResizeGather@option}{%
      \noexpand\PassOptionsToPackage{graphics}{\ResizeGather@option}%
    }%
    \noexpand\AtEndOfPackage{%
      \noexpand\DisableKeyvalOption[%
        action=error,%
        package=resizegather,%
      ]{resizegather}{\ResizeGather@option}%
    }%
  }%
  \ResizeGather@temp
}
%    \end{macrocode}
%
%    \begin{macrocode}
\DeclareBoolOption[true]{enable}
\DeclareComplementaryOption{disable}{enable}
\DeclareStringOption[.05]{warningthreshold}
\newif\ifResizeGather@NeedInit
\DeclareBoolOption[true]{equations}
\DeclareBoolOption[true]{equation}
\DeclareBoolOption[true]{displaymath}
\AddToKeyvalOption*{equations}{%
  \ResizeGather@NeedInittrue
  \ifResizeGather@equations
    \ResizeGather@equationtrue
    \ResizeGather@displaymathtrue
  \else
    \ResizeGather@equationfalse
    \ResizeGather@displaymathfalse
  \fi
}
\AddToKeyvalOption*{equation}{%
  \ResizeGather@NeedInittrue
}
\AddToKeyvalOption*{displaymath}{%
  \ResizeGather@NeedInittrue
}
%    \end{macrocode}
%
%    \begin{macro}{\resizegathersetup}
%    \begin{macrocode}
\newcommand*{\resizegathersetup}[1]{%
  \ResizeGather@NeedInitfalse
  \setkeys{resizegather}{#1}%
  \ifResizeGather@NeedInit
    \ResizeGather@init
  \fi
}
\let\ResizeGather@init\relax
%    \end{macrocode}
%    \end{macro}
%    \begin{macrocode}
\InputIfFileExists{resizegather.cfg}{}{}%
\ProcessKeyvalOptions*\relax
%    \end{macrocode}
%    \begin{macrocode}
\RequirePackage{amsmath}
\RequirePackage{graphics}
%    \end{macrocode}
%    \begin{macro}{\ResizeGather@redefine}
%    \begin{macrocode}
\def\ResizeGather@redefine#1#2#3#4#5{%
  \csname ifResizeGather@#1\endcsname
    \@ifundefined{ResizeGather@org@#2}{%
      \expandafter\let\csname ResizeGather@org@#2\expandafter\endcsname
                      \csname #2\endcsname
    }{}%
    \@ifundefined{ResizeGather@org@#3}{%
      \expandafter\let\csname ResizeGather@org@#3\expandafter\endcsname
                      \csname #3\endcsname
    }{}%
    \expandafter\edef\csname #2\endcsname{%
      \expandafter\noexpand\csname#4\endcsname
    }%
    \expandafter\edef\csname #3\endcsname{%
      \expandafter\noexpand\csname#5\endcsname
    }%
  \else
    \@ifundefined{ResizeGather@org@#2}{}{%
      \expandafter\let\csname #2\expandafter\endcsname
                      \csname ResizeGather@org@#2\endcsname
      \expandafter\let\csname #3\expandafter\endcsname
                      \csname ResizeGather@org@#3\endcsname
    }%
  \fi
}
%    \end{macrocode}
%    \end{macro}
%    \begin{macro}{\ResizeGather@init}
%    \begin{macrocode}
\def\ResizeGather@init{%
  \ResizeGather@redefine{equation}{equation}{endequation}%
    {gather}{endgather}%
  \ResizeGather@redefine{displaymath}{displaymath}{enddisplaymath}%
    {gather*}{endgather*}%
}
\ResizeGather@init
%    \end{macrocode}
%    \end{macro}
%
%    \begin{macro}{\ResizeGather@ResizeGather}
%    \begin{macrocode}
\def\ResizeGather@ResizeGather{%
  \ifResizeGather@enable
    \dimen@\displaywidth
    \if@fleqn
      \advance\dimen@-\@mathmargin
    \fi
    \ifdim\wdz@>\dimen@
      \begingroup
        \advance\dimen@ -\wdz@
        \dimen@ -\dimen@
        \ifdim\ResizeGather@warningthreshold\wdz@>\dimen@
          \expandafter\PackageInfo
        \else
          \expandafter\PackageWarning
        \fi
        {resizegather}{%
          Equation line \the\row@\space is too large %
          by \the\dimen@\MessageBreak
          in environment `\@currenvir'%
        }%
      \endgroup
      \setboxz@h to\dimen@{%
        \resizebox{\dimen@}{!}{\boxz@}%
        \hss
      }%
    \fi
  \fi
}
%    \end{macrocode}
%    \end{macro}
%    \begin{macro}{\calc@shift@gather}
%    \begin{macrocode}
\expandafter\def\expandafter\calc@shift@gather\expandafter{%
  \expandafter\ResizeGather@ResizeGather
  \calc@shift@gather
}
%    \end{macrocode}
%    \end{macro}
%    \begin{macro}{\ResizeGather@org@gmeasure@}
%    \begin{macrocode}
\let\ResizeGather@org@gmeasure@\gmeasure@
%    \end{macrocode}
%    \end{macro}
%    \begin{macro}{\gmeasure@}
%    \begin{macrocode}
\def\gmeasure@#1{%
  \ResizeGather@org@gmeasure@{#1}%
  \ifResizeGather@enable
    \ifdim\totwidth@>\displaywidth
      \totwidth@=\displaywidth
    \fi
  \fi
}
%    \end{macrocode}
%    \end{macro}
%
%    \begin{macrocode}
\ResizeGather@AtEnd%
%</package>
%    \end{macrocode}
%
% \section{Test}
%
% \subsection{Catcode checks for loading}
%
%    \begin{macrocode}
%<*test1>
%    \end{macrocode}
%    \begin{macrocode}
\catcode`\{=1 %
\catcode`\}=2 %
\catcode`\#=6 %
\catcode`\@=11 %
\expandafter\ifx\csname count@\endcsname\relax
  \countdef\count@=255 %
\fi
\expandafter\ifx\csname @gobble\endcsname\relax
  \long\def\@gobble#1{}%
\fi
\expandafter\ifx\csname @firstofone\endcsname\relax
  \long\def\@firstofone#1{#1}%
\fi
\expandafter\ifx\csname loop\endcsname\relax
  \expandafter\@firstofone
\else
  \expandafter\@gobble
\fi
{%
  \def\loop#1\repeat{%
    \def\body{#1}%
    \iterate
  }%
  \def\iterate{%
    \body
      \let\next\iterate
    \else
      \let\next\relax
    \fi
    \next
  }%
  \let\repeat=\fi
}%
\def\RestoreCatcodes{}
\count@=0 %
\loop
  \edef\RestoreCatcodes{%
    \RestoreCatcodes
    \catcode\the\count@=\the\catcode\count@\relax
  }%
\ifnum\count@<255 %
  \advance\count@ 1 %
\repeat

\def\RangeCatcodeInvalid#1#2{%
  \count@=#1\relax
  \loop
    \catcode\count@=15 %
  \ifnum\count@<#2\relax
    \advance\count@ 1 %
  \repeat
}
\def\RangeCatcodeCheck#1#2#3{%
  \count@=#1\relax
  \loop
    \ifnum#3=\catcode\count@
    \else
      \errmessage{%
        Character \the\count@\space
        with wrong catcode \the\catcode\count@\space
        instead of \number#3%
      }%
    \fi
  \ifnum\count@<#2\relax
    \advance\count@ 1 %
  \repeat
}
\def\space{ }
\expandafter\ifx\csname LoadCommand\endcsname\relax
  \def\LoadCommand{\input resizegather.sty\relax}%
\fi
\def\Test{%
  \RangeCatcodeInvalid{0}{47}%
  \RangeCatcodeInvalid{58}{64}%
  \RangeCatcodeInvalid{91}{96}%
  \RangeCatcodeInvalid{123}{255}%
  \catcode`\@=12 %
  \catcode`\\=0 %
  \catcode`\%=14 %
  \LoadCommand
  \RangeCatcodeCheck{0}{36}{15}%
  \RangeCatcodeCheck{37}{37}{14}%
  \RangeCatcodeCheck{38}{47}{15}%
  \RangeCatcodeCheck{48}{57}{12}%
  \RangeCatcodeCheck{58}{63}{15}%
  \RangeCatcodeCheck{64}{64}{12}%
  \RangeCatcodeCheck{65}{90}{11}%
  \RangeCatcodeCheck{91}{91}{15}%
  \RangeCatcodeCheck{92}{92}{0}%
  \RangeCatcodeCheck{93}{96}{15}%
  \RangeCatcodeCheck{97}{122}{11}%
  \RangeCatcodeCheck{123}{255}{15}%
  \RestoreCatcodes
}
\Test
\csname @@end\endcsname
\end
%    \end{macrocode}
%    \begin{macrocode}
%</test1>
%    \end{macrocode}
%
% \section{Installation}
%
% \subsection{Download}
%
% \paragraph{Package.} This package is available on
% CTAN\footnote{\url{https://ctan.org/pkg/resizegather}}:
% \begin{description}
% \item[\CTAN{macros/latex/contrib/oberdiek/resizegather.dtx}] The source file.
% \item[\CTAN{macros/latex/contrib/oberdiek/resizegather.pdf}] Documentation.
% \end{description}
%
%
% \paragraph{Bundle.} All the packages of the bundle `oberdiek'
% are also available in a TDS compliant ZIP archive. There
% the packages are already unpacked and the documentation files
% are generated. The files and directories obey the TDS standard.
% \begin{description}
% \item[\CTANinstall{install/macros/latex/contrib/oberdiek.tds.zip}]
% \end{description}
% \emph{TDS} refers to the standard ``A Directory Structure
% for \TeX\ Files'' (\CTAN{tds/tds.pdf}). Directories
% with \xfile{texmf} in their name are usually organized this way.
%
% \subsection{Bundle installation}
%
% \paragraph{Unpacking.} Unpack the \xfile{oberdiek.tds.zip} in the
% TDS tree (also known as \xfile{texmf} tree) of your choice.
% Example (linux):
% \begin{quote}
%   |unzip oberdiek.tds.zip -d ~/texmf|
% \end{quote}
%
% \paragraph{Script installation.}
% Check the directory \xfile{TDS:scripts/oberdiek/} for
% scripts that need further installation steps.
% Package \xpackage{attachfile2} comes with the Perl script
% \xfile{pdfatfi.pl} that should be installed in such a way
% that it can be called as \texttt{pdfatfi}.
% Example (linux):
% \begin{quote}
%   |chmod +x scripts/oberdiek/pdfatfi.pl|\\
%   |cp scripts/oberdiek/pdfatfi.pl /usr/local/bin/|
% \end{quote}
%
% \subsection{Package installation}
%
% \paragraph{Unpacking.} The \xfile{.dtx} file is a self-extracting
% \docstrip\ archive. The files are extracted by running the
% \xfile{.dtx} through \plainTeX:
% \begin{quote}
%   \verb|tex resizegather.dtx|
% \end{quote}
%
% \paragraph{TDS.} Now the different files must be moved into
% the different directories in your installation TDS tree
% (also known as \xfile{texmf} tree):
% \begin{quote}
% \def\t{^^A
% \begin{tabular}{@{}>{\ttfamily}l@{ $\rightarrow$ }>{\ttfamily}l@{}}
%   resizegather.sty & tex/latex/oberdiek/resizegather.sty\\
%   resizegather.pdf & doc/latex/oberdiek/resizegather.pdf\\
%   test/resizegather-test1.tex & doc/latex/oberdiek/test/resizegather-test1.tex\\
%   resizegather.dtx & source/latex/oberdiek/resizegather.dtx\\
% \end{tabular}^^A
% }^^A
% \sbox0{\t}^^A
% \ifdim\wd0>\linewidth
%   \begingroup
%     \advance\linewidth by\leftmargin
%     \advance\linewidth by\rightmargin
%   \edef\x{\endgroup
%     \def\noexpand\lw{\the\linewidth}^^A
%   }\x
%   \def\lwbox{^^A
%     \leavevmode
%     \hbox to \linewidth{^^A
%       \kern-\leftmargin\relax
%       \hss
%       \usebox0
%       \hss
%       \kern-\rightmargin\relax
%     }^^A
%   }^^A
%   \ifdim\wd0>\lw
%     \sbox0{\small\t}^^A
%     \ifdim\wd0>\linewidth
%       \ifdim\wd0>\lw
%         \sbox0{\footnotesize\t}^^A
%         \ifdim\wd0>\linewidth
%           \ifdim\wd0>\lw
%             \sbox0{\scriptsize\t}^^A
%             \ifdim\wd0>\linewidth
%               \ifdim\wd0>\lw
%                 \sbox0{\tiny\t}^^A
%                 \ifdim\wd0>\linewidth
%                   \lwbox
%                 \else
%                   \usebox0
%                 \fi
%               \else
%                 \lwbox
%               \fi
%             \else
%               \usebox0
%             \fi
%           \else
%             \lwbox
%           \fi
%         \else
%           \usebox0
%         \fi
%       \else
%         \lwbox
%       \fi
%     \else
%       \usebox0
%     \fi
%   \else
%     \lwbox
%   \fi
% \else
%   \usebox0
% \fi
% \end{quote}
% If you have a \xfile{docstrip.cfg} that configures and enables \docstrip's
% TDS installing feature, then some files can already be in the right
% place, see the documentation of \docstrip.
%
% \subsection{Refresh file name databases}
%
% If your \TeX~distribution
% (\teTeX, \mikTeX, \dots) relies on file name databases, you must refresh
% these. For example, \teTeX\ users run \verb|texhash| or
% \verb|mktexlsr|.
%
% \subsection{Some details for the interested}
%
% \paragraph{Attached source.}
%
% The PDF documentation on CTAN also includes the
% \xfile{.dtx} source file. It can be extracted by
% AcrobatReader 6 or higher. Another option is \textsf{pdftk},
% e.g. unpack the file into the current directory:
% \begin{quote}
%   \verb|pdftk resizegather.pdf unpack_files output .|
% \end{quote}
%
% \paragraph{Unpacking with \LaTeX.}
% The \xfile{.dtx} chooses its action depending on the format:
% \begin{description}
% \item[\plainTeX:] Run \docstrip\ and extract the files.
% \item[\LaTeX:] Generate the documentation.
% \end{description}
% If you insist on using \LaTeX\ for \docstrip\ (really,
% \docstrip\ does not need \LaTeX), then inform the autodetect routine
% about your intention:
% \begin{quote}
%   \verb|latex \let\install=y\input{resizegather.dtx}|
% \end{quote}
% Do not forget to quote the argument according to the demands
% of your shell.
%
% \paragraph{Generating the documentation.}
% You can use both the \xfile{.dtx} or the \xfile{.drv} to generate
% the documentation. The process can be configured by the
% configuration file \xfile{ltxdoc.cfg}. For instance, put this
% line into this file, if you want to have A4 as paper format:
% \begin{quote}
%   \verb|\PassOptionsToClass{a4paper}{article}|
% \end{quote}
% An example follows how to generate the
% documentation with pdf\LaTeX:
% \begin{quote}
%\begin{verbatim}
%pdflatex resizegather.dtx
%makeindex -s gind.ist resizegather.idx
%pdflatex resizegather.dtx
%makeindex -s gind.ist resizegather.idx
%pdflatex resizegather.dtx
%\end{verbatim}
% \end{quote}
%
% \section{Acknowledgement}
%
% \begin{description}
% \item[Dieter Jurzitza:]
% He wanted the resizing feature at the \TeX\ table
% in Karlsruhe of December 2009. Thus this package is a kind of
% Christmas present.
% \end{description}
%
% \begin{History}
%   \begin{Version}{2009/12/04 v1.0}
%   \item
%     The first version.
%   \end{Version}
%   \begin{Version}{2009/12/05 v1.1}
%   \item
%     Options \xoption{enable} and \xoption{disable} added.
%   \end{Version}
%   \begin{Version}{2010/03/01 v1.2}
%   \item
%     TDS location moved from `generic' to `latex'.
%   \end{Version}
%   \begin{Version}{2016/05/16 v1.3}
%   \item
%     Documentation updates.
%   \end{Version}
% \end{History}
%
% \PrintIndex
%
% \Finale
\endinput

%        (quote the arguments according to the demands of your shell)
%
% Documentation:
%    (a) If resizegather.drv is present:
%           latex resizegather.drv
%    (b) Without resizegather.drv:
%           latex resizegather.dtx; ...
%    The class ltxdoc loads the configuration file ltxdoc.cfg
%    if available. Here you can specify further options, e.g.
%    use A4 as paper format:
%       \PassOptionsToClass{a4paper}{article}
%
%    Programm calls to get the documentation (example):
%       pdflatex resizegather.dtx
%       makeindex -s gind.ist resizegather.idx
%       pdflatex resizegather.dtx
%       makeindex -s gind.ist resizegather.idx
%       pdflatex resizegather.dtx
%
% Installation:
%    TDS:tex/latex/oberdiek/resizegather.sty
%    TDS:doc/latex/oberdiek/resizegather.pdf
%    TDS:doc/latex/oberdiek/test/resizegather-test1.tex
%    TDS:source/latex/oberdiek/resizegather.dtx
%
%<*ignore>
\begingroup
  \catcode123=1 %
  \catcode125=2 %
  \def\x{LaTeX2e}%
\expandafter\endgroup
\ifcase 0\ifx\install y1\fi\expandafter
         \ifx\csname processbatchFile\endcsname\relax\else1\fi
         \ifx\fmtname\x\else 1\fi\relax
\else\csname fi\endcsname
%</ignore>
%<*install>
\input docstrip.tex
\Msg{************************************************************************}
\Msg{* Installation}
\Msg{* Package: resizegather 2016/05/16 v1.3 Resize overly large equations (HO)}
\Msg{************************************************************************}

\keepsilent
\askforoverwritefalse

\let\MetaPrefix\relax
\preamble

This is a generated file.

Project: resizegather
Version: 2016/05/16 v1.3

Copyright (C) 2009, 2010 by
   Heiko Oberdiek <heiko.oberdiek at googlemail.com>

This work may be distributed and/or modified under the
conditions of the LaTeX Project Public License, either
version 1.3c of this license or (at your option) any later
version. This version of this license is in
   http://www.latex-project.org/lppl/lppl-1-3c.txt
and the latest version of this license is in
   http://www.latex-project.org/lppl.txt
and version 1.3 or later is part of all distributions of
LaTeX version 2005/12/01 or later.

This work has the LPPL maintenance status "maintained".

This Current Maintainer of this work is Heiko Oberdiek.

This work consists of the main source file resizegather.dtx
and the derived files
   resizegather.sty, resizegather.pdf, resizegather.ins, resizegather.drv,
   resizegather-test1.tex.

\endpreamble
\let\MetaPrefix\DoubleperCent

\generate{%
  \file{resizegather.ins}{\from{resizegather.dtx}{install}}%
  \file{resizegather.drv}{\from{resizegather.dtx}{driver}}%
  \usedir{tex/latex/oberdiek}%
  \file{resizegather.sty}{\from{resizegather.dtx}{package}}%
%  \usedir{doc/latex/oberdiek/test}%
%  \file{resizegather-test1.tex}{\from{resizegather.dtx}{test1}}%
  \nopreamble
  \nopostamble
%  \usedir{source/latex/oberdiek/catalogue}%
%  \file{resizegather.xml}{\from{resizegather.dtx}{catalogue}}%
}

\catcode32=13\relax% active space
\let =\space%
\Msg{************************************************************************}
\Msg{*}
\Msg{* To finish the installation you have to move the following}
\Msg{* file into a directory searched by TeX:}
\Msg{*}
\Msg{*     resizegather.sty}
\Msg{*}
\Msg{* To produce the documentation run the file `resizegather.drv'}
\Msg{* through LaTeX.}
\Msg{*}
\Msg{* Happy TeXing!}
\Msg{*}
\Msg{************************************************************************}

\endbatchfile
%</install>
%<*ignore>
\fi
%</ignore>
%<*driver>
\NeedsTeXFormat{LaTeX2e}
\ProvidesFile{resizegather.drv}%
  [2016/05/16 v1.3 Resize overly large equations (HO)]%
\documentclass{ltxdoc}
\usepackage{holtxdoc}[2011/11/22]
\usepackage{ifluatex}
\ifluatex
\else
  \usepackage[T1]{fontenc}%
  \usepackage{textcomp}%
  \usepackage{lmodern}%
\fi
\begin{document}
  \DocInput{resizegather.dtx}%
\end{document}
%</driver>
% \fi
%
%
% \CharacterTable
%  {Upper-case    \A\B\C\D\E\F\G\H\I\J\K\L\M\N\O\P\Q\R\S\T\U\V\W\X\Y\Z
%   Lower-case    \a\b\c\d\e\f\g\h\i\j\k\l\m\n\o\p\q\r\s\t\u\v\w\x\y\z
%   Digits        \0\1\2\3\4\5\6\7\8\9
%   Exclamation   \!     Double quote  \"     Hash (number) \#
%   Dollar        \$     Percent       \%     Ampersand     \&
%   Acute accent  \'     Left paren    \(     Right paren   \)
%   Asterisk      \*     Plus          \+     Comma         \,
%   Minus         \-     Point         \.     Solidus       \/
%   Colon         \:     Semicolon     \;     Less than     \<
%   Equals        \=     Greater than  \>     Question mark \?
%   Commercial at \@     Left bracket  \[     Backslash     \\
%   Right bracket \]     Circumflex    \^     Underscore    \_
%   Grave accent  \`     Left brace    \{     Vertical bar  \|
%   Right brace   \}     Tilde         \~}
%
% \GetFileInfo{resizegather.drv}
%
% \title{The \xpackage{resizegather} package}
% \date{2016/05/16 v1.3}
% \author{Heiko Oberdiek\thanks
% {Please report any issues at https://github.com/ho-tex/oberdiek/issues}\\
% \xemail{heiko.oberdiek at googlemail.com}}
%
% \maketitle
%
% \begin{abstract}
% Equations that are too large are resized to fit the available
% space. The environment \textsf{gather} of package \xpackage{amsmath}
% is supported. Also the environments \textsf{equation} and
% \textsf{displaymath} are redefined using \textsf{gather}
% and its starred version.
% \end{abstract}
%
% \tableofcontents
%
% \makeatletter
% \def\env#1{^^A
%    \textsf{\@env#1*\@nil}^^A
% }%
% \def\@env#1*#2\@nil{^^A
%   #1^^A
%   \ifx\\#2\\^^A
%     \expandafter\@gobble
%   \else
%     \textasteriskcentered
%     \expandafter\@firstofone
%   \fi
%   {\@env#2\@nil}^^A
% }
% \makeatother
%
% \section{Documentation}
%
% Sometimes an equation is just a little to large to fit in the
% line. And breaking the equation across lines might be worse
% than downscaling the equation. This package implements this
% for the environments \env{gather} and \env{gather*} of
% package \xpackage{amsmath}. That package already measures
% the equations and simplifies the implementation of \xpackage{resizegather}
% that only needs to hook into \xpackage{amsmath}'s code to add
% the resizing feature.
%
% Resized equations are recorded in the \xext{log} file
% for small exceeds (default setting is smaller than five percent).
% Otherwise a warning is given.
%
% Also environments \env{equation} and \env{displaymath}
% are supported by redefining them using \env{gather}
% and \env{gather*}.
%
% \cs{[} and \cs{]} are not supported, because these macros
% are not in environment form that is required for
% \xpackage{amsmath}. The environment body is collected
% first to be able to process the body twice for measuring
% first.
%
% Also the environments using alignments are not supported.
% If a single equation line would be resized, the alignment
% would get lost. And resizing all equations of the alignment
% does not seem appropriate either.
%
% \subsection{Options}
%
% \begin{description}
% \item[\xoption{warningthreshold}:]
%   Print a warning if the original equation line exceeds
%   its available width by the given fraction.
%   Default is |0.05|: A warning is given if the equation
%   is too large by five percent.
%   Otherwise the exceed is recorded in the \xext{log} file
%   only.
% \end{description}
% The next options are boolean options. They are enabled
% by value |true| or if no value is given. They are switched
% off by value |false|.
% \begin{description}
% \item[\xoption{enable}:] The resize feature is active (default).
% \item[\xoption{disable}:] The complementary option for \xoption{enable},
%  added for convenience: |disable| (or |disable=true|) is the same
%  as |enable=false|.
% \item[\xoption{equations}:]
%   \LaTeX\ environments \textsf{equation} and \textsf{displaymath}
%   environments are redefined. These equations
%   are now using environment \env{gather} and
%   \env{gather*}. This is the default.
% \end{description}
% The following table shows additional options if you
% want to have finer control for the redefined
% environments:
% \begin{quote}
% \def\unchanged{\textit{unchanged}}
% \def\notprovided{\textit{not provided}}
% \begin{tabular}{l|ll}
% &\multicolumn{2}{c}{Environments}\\
% Option & \env{equation} & \env{displaymath}\\
% \hline
% \xoption{equations} & \env{gather} & \env{gather*}\\
% \xoption{equation} & \env{gather} & \unchanged\\
% \xoption{displaymath} & \unchanged & \env{gather*}\\
% \end{tabular}
% \end{quote}
% If such an option is switched off, the original meaning
% of the affected environments is restored.
%
% Options are evaluated in the following order:
% \begin{enumerate}
% \item
%  Configuration file \xfile{resizegather.cfg} using \cs{resizegathersetup}
%  if the file exists.
%  \item
%  Package options given for \cs{usepackage}.
%  \item
%  Later calls of \cs{resizegathersetup}.
% \end{enumerate}
% \begin{declcs}{resizegathersetup}\M{option list}
% \end{declcs}
% The options are key value options. Boolean options are enabled by
% default (without value) or by using the explicit value \texttt{true}.
% Value \texttt{false} disable the option.
%
% \subsection{Options for packages \xpackage{amsmath} or \xpackage{graphics}}
%
% The package loads the package \xpackage{amsmath} because is internally
% measures the equations first. Thus this package hooks into this code
% in order to resize the equations if they are too large.
% The resizing itself is done by \cs{resizebox} of package \xpackage{graphics}.
% If you need special options for these packages, just load them first or
% use global options when appropriate. Example:
% \begin{quote}
%\begin{verbatim}
%\usepackage[dvipdfm]{graphicx}% or graphics
%\usepackage[fleqn]{amsmath}
%\usepackage{resizegather}
%\end{verbatim}
%\end{quote}
%
% \StopEventually{
% }
%
% \section{Implementation}
%    \begin{macrocode}
%<*package>
%    \end{macrocode}
%    Reload check, especially if the package is not used with \LaTeX.
%    \begin{macrocode}
\begingroup\catcode61\catcode48\catcode32=10\relax%
  \catcode13=5 % ^^M
  \endlinechar=13 %
  \catcode35=6 % #
  \catcode39=12 % '
  \catcode44=12 % ,
  \catcode45=12 % -
  \catcode46=12 % .
  \catcode58=12 % :
  \catcode64=11 % @
  \catcode123=1 % {
  \catcode125=2 % }
  \expandafter\let\expandafter\x\csname ver@resizegather.sty\endcsname
  \ifx\x\relax % plain-TeX, first loading
  \else
    \def\empty{}%
    \ifx\x\empty % LaTeX, first loading,
      % variable is initialized, but \ProvidesPackage not yet seen
    \else
      \expandafter\ifx\csname PackageInfo\endcsname\relax
        \def\x#1#2{%
          \immediate\write-1{Package #1 Info: #2.}%
        }%
      \else
        \def\x#1#2{\PackageInfo{#1}{#2, stopped}}%
      \fi
      \x{resizegather}{The package is already loaded}%
      \aftergroup\endinput
    \fi
  \fi
\endgroup%
%    \end{macrocode}
%    Package identification:
%    \begin{macrocode}
\begingroup\catcode61\catcode48\catcode32=10\relax%
  \catcode13=5 % ^^M
  \endlinechar=13 %
  \catcode35=6 % #
  \catcode39=12 % '
  \catcode40=12 % (
  \catcode41=12 % )
  \catcode44=12 % ,
  \catcode45=12 % -
  \catcode46=12 % .
  \catcode47=12 % /
  \catcode58=12 % :
  \catcode64=11 % @
  \catcode91=12 % [
  \catcode93=12 % ]
  \catcode123=1 % {
  \catcode125=2 % }
  \expandafter\ifx\csname ProvidesPackage\endcsname\relax
    \def\x#1#2#3[#4]{\endgroup
      \immediate\write-1{Package: #3 #4}%
      \xdef#1{#4}%
    }%
  \else
    \def\x#1#2[#3]{\endgroup
      #2[{#3}]%
      \ifx#1\@undefined
        \xdef#1{#3}%
      \fi
      \ifx#1\relax
        \xdef#1{#3}%
      \fi
    }%
  \fi
\expandafter\x\csname ver@resizegather.sty\endcsname
\ProvidesPackage{resizegather}%
  [2016/05/16 v1.3 Resize overly large equations (HO)]%
%    \end{macrocode}
%
%    \begin{macrocode}
\begingroup\catcode61\catcode48\catcode32=10\relax%
  \catcode13=5 % ^^M
  \endlinechar=13 %
  \catcode123=1 % {
  \catcode125=2 % }
  \catcode64=11 % @
  \def\x{\endgroup
    \expandafter\edef\csname ResizeGather@AtEnd\endcsname{%
      \endlinechar=\the\endlinechar\relax
      \catcode13=\the\catcode13\relax
      \catcode32=\the\catcode32\relax
      \catcode35=\the\catcode35\relax
      \catcode61=\the\catcode61\relax
      \catcode64=\the\catcode64\relax
      \catcode123=\the\catcode123\relax
      \catcode125=\the\catcode125\relax
    }%
  }%
\x\catcode61\catcode48\catcode32=10\relax%
\catcode13=5 % ^^M
\endlinechar=13 %
\catcode35=6 % #
\catcode64=11 % @
\catcode123=1 % {
\catcode125=2 % }
\def\TMP@EnsureCode#1#2{%
  \edef\ResizeGather@AtEnd{%
    \ResizeGather@AtEnd
    \catcode#1=\the\catcode#1\relax
  }%
  \catcode#1=#2\relax
}
\TMP@EnsureCode{10}{12}% ^^J
\TMP@EnsureCode{33}{12}% !
\TMP@EnsureCode{36}{3}% $
\TMP@EnsureCode{38}{4}% &
\TMP@EnsureCode{39}{12}% '
\TMP@EnsureCode{40}{12}% (
\TMP@EnsureCode{41}{12}% )
\TMP@EnsureCode{42}{12}% *
\TMP@EnsureCode{43}{12}% +
\TMP@EnsureCode{44}{12}% ,
\TMP@EnsureCode{45}{12}% -
\TMP@EnsureCode{46}{12}% .
\TMP@EnsureCode{47}{12}% /
\TMP@EnsureCode{58}{12}% :
\TMP@EnsureCode{59}{12}% ;
\TMP@EnsureCode{60}{12}% <
\TMP@EnsureCode{62}{12}% >
\TMP@EnsureCode{63}{12}% ?
\TMP@EnsureCode{91}{12}% [
\TMP@EnsureCode{93}{12}% ]
\TMP@EnsureCode{94}{7}% ^ (superscript)
\TMP@EnsureCode{96}{12}% `
\edef\ResizeGather@AtEnd{\ResizeGather@AtEnd\noexpand\endinput}
%    \end{macrocode}
%
%    \begin{macrocode}
\RequirePackage{kvoptions}[2009/12/04]
\SetupKeyvalOptions{%
  family=resizegather,%
  prefix=ResizeGather@,%
}
%    \end{macrocode}
%    \begin{macrocode}
\@for\ResizeGather@option:=%
  centertags,%
  tbtags,%
  sumlimits,%
  nosumlimits,%
  intlimits,%
  nointlimits,%
  nonamelimits,%
  leqno,%
  reqno,%
  fleqn%
\do{%
  \edef\ResizeGather@temp{%
    \noexpand\DeclareVoidOption{\ResizeGather@option}{%
      \noexpand\PassOptionsToPackage{amsmath}{\ResizeGather@option}%
    }%
    \noexpand\AtEndOfPackage{%
      \noexpand\DisableKeyvalOption[%
        action=error,%
        package=resizegather,%
      ]{resizegather}{\ResizeGather@option}%
    }%
  }%
  \ResizeGather@temp
}
\@for\ResizeGather@option:=%
  draft,%
  final,%
  hiderotate,%
  hidescale,%
  hiresbb,%
  demo,%
  dvips,xdvi,dvipdf,dvipdfm,dvipdfmx,pdftex,dvipsone,%
  dviwindo,emtex,dviwin,pctexps,pctexwin,pctexhp,pctex32,%
  truetex,tcidvi,vtex,oztex,textures,xetex%
\do{%
  \edef\ResizeGather@temp{%
    \noexpand\DeclareVoidOption{\ResizeGather@option}{%
      \noexpand\PassOptionsToPackage{graphics}{\ResizeGather@option}%
    }%
    \noexpand\AtEndOfPackage{%
      \noexpand\DisableKeyvalOption[%
        action=error,%
        package=resizegather,%
      ]{resizegather}{\ResizeGather@option}%
    }%
  }%
  \ResizeGather@temp
}
%    \end{macrocode}
%
%    \begin{macrocode}
\DeclareBoolOption[true]{enable}
\DeclareComplementaryOption{disable}{enable}
\DeclareStringOption[.05]{warningthreshold}
\newif\ifResizeGather@NeedInit
\DeclareBoolOption[true]{equations}
\DeclareBoolOption[true]{equation}
\DeclareBoolOption[true]{displaymath}
\AddToKeyvalOption*{equations}{%
  \ResizeGather@NeedInittrue
  \ifResizeGather@equations
    \ResizeGather@equationtrue
    \ResizeGather@displaymathtrue
  \else
    \ResizeGather@equationfalse
    \ResizeGather@displaymathfalse
  \fi
}
\AddToKeyvalOption*{equation}{%
  \ResizeGather@NeedInittrue
}
\AddToKeyvalOption*{displaymath}{%
  \ResizeGather@NeedInittrue
}
%    \end{macrocode}
%
%    \begin{macro}{\resizegathersetup}
%    \begin{macrocode}
\newcommand*{\resizegathersetup}[1]{%
  \ResizeGather@NeedInitfalse
  \setkeys{resizegather}{#1}%
  \ifResizeGather@NeedInit
    \ResizeGather@init
  \fi
}
\let\ResizeGather@init\relax
%    \end{macrocode}
%    \end{macro}
%    \begin{macrocode}
\InputIfFileExists{resizegather.cfg}{}{}%
\ProcessKeyvalOptions*\relax
%    \end{macrocode}
%    \begin{macrocode}
\RequirePackage{amsmath}
\RequirePackage{graphics}
%    \end{macrocode}
%    \begin{macro}{\ResizeGather@redefine}
%    \begin{macrocode}
\def\ResizeGather@redefine#1#2#3#4#5{%
  \csname ifResizeGather@#1\endcsname
    \@ifundefined{ResizeGather@org@#2}{%
      \expandafter\let\csname ResizeGather@org@#2\expandafter\endcsname
                      \csname #2\endcsname
    }{}%
    \@ifundefined{ResizeGather@org@#3}{%
      \expandafter\let\csname ResizeGather@org@#3\expandafter\endcsname
                      \csname #3\endcsname
    }{}%
    \expandafter\edef\csname #2\endcsname{%
      \expandafter\noexpand\csname#4\endcsname
    }%
    \expandafter\edef\csname #3\endcsname{%
      \expandafter\noexpand\csname#5\endcsname
    }%
  \else
    \@ifundefined{ResizeGather@org@#2}{}{%
      \expandafter\let\csname #2\expandafter\endcsname
                      \csname ResizeGather@org@#2\endcsname
      \expandafter\let\csname #3\expandafter\endcsname
                      \csname ResizeGather@org@#3\endcsname
    }%
  \fi
}
%    \end{macrocode}
%    \end{macro}
%    \begin{macro}{\ResizeGather@init}
%    \begin{macrocode}
\def\ResizeGather@init{%
  \ResizeGather@redefine{equation}{equation}{endequation}%
    {gather}{endgather}%
  \ResizeGather@redefine{displaymath}{displaymath}{enddisplaymath}%
    {gather*}{endgather*}%
}
\ResizeGather@init
%    \end{macrocode}
%    \end{macro}
%
%    \begin{macro}{\ResizeGather@ResizeGather}
%    \begin{macrocode}
\def\ResizeGather@ResizeGather{%
  \ifResizeGather@enable
    \dimen@\displaywidth
    \if@fleqn
      \advance\dimen@-\@mathmargin
    \fi
    \ifdim\wdz@>\dimen@
      \begingroup
        \advance\dimen@ -\wdz@
        \dimen@ -\dimen@
        \ifdim\ResizeGather@warningthreshold\wdz@>\dimen@
          \expandafter\PackageInfo
        \else
          \expandafter\PackageWarning
        \fi
        {resizegather}{%
          Equation line \the\row@\space is too large %
          by \the\dimen@\MessageBreak
          in environment `\@currenvir'%
        }%
      \endgroup
      \setboxz@h to\dimen@{%
        \resizebox{\dimen@}{!}{\boxz@}%
        \hss
      }%
    \fi
  \fi
}
%    \end{macrocode}
%    \end{macro}
%    \begin{macro}{\calc@shift@gather}
%    \begin{macrocode}
\expandafter\def\expandafter\calc@shift@gather\expandafter{%
  \expandafter\ResizeGather@ResizeGather
  \calc@shift@gather
}
%    \end{macrocode}
%    \end{macro}
%    \begin{macro}{\ResizeGather@org@gmeasure@}
%    \begin{macrocode}
\let\ResizeGather@org@gmeasure@\gmeasure@
%    \end{macrocode}
%    \end{macro}
%    \begin{macro}{\gmeasure@}
%    \begin{macrocode}
\def\gmeasure@#1{%
  \ResizeGather@org@gmeasure@{#1}%
  \ifResizeGather@enable
    \ifdim\totwidth@>\displaywidth
      \totwidth@=\displaywidth
    \fi
  \fi
}
%    \end{macrocode}
%    \end{macro}
%
%    \begin{macrocode}
\ResizeGather@AtEnd%
%</package>
%    \end{macrocode}
%
% \section{Test}
%
% \subsection{Catcode checks for loading}
%
%    \begin{macrocode}
%<*test1>
%    \end{macrocode}
%    \begin{macrocode}
\catcode`\{=1 %
\catcode`\}=2 %
\catcode`\#=6 %
\catcode`\@=11 %
\expandafter\ifx\csname count@\endcsname\relax
  \countdef\count@=255 %
\fi
\expandafter\ifx\csname @gobble\endcsname\relax
  \long\def\@gobble#1{}%
\fi
\expandafter\ifx\csname @firstofone\endcsname\relax
  \long\def\@firstofone#1{#1}%
\fi
\expandafter\ifx\csname loop\endcsname\relax
  \expandafter\@firstofone
\else
  \expandafter\@gobble
\fi
{%
  \def\loop#1\repeat{%
    \def\body{#1}%
    \iterate
  }%
  \def\iterate{%
    \body
      \let\next\iterate
    \else
      \let\next\relax
    \fi
    \next
  }%
  \let\repeat=\fi
}%
\def\RestoreCatcodes{}
\count@=0 %
\loop
  \edef\RestoreCatcodes{%
    \RestoreCatcodes
    \catcode\the\count@=\the\catcode\count@\relax
  }%
\ifnum\count@<255 %
  \advance\count@ 1 %
\repeat

\def\RangeCatcodeInvalid#1#2{%
  \count@=#1\relax
  \loop
    \catcode\count@=15 %
  \ifnum\count@<#2\relax
    \advance\count@ 1 %
  \repeat
}
\def\RangeCatcodeCheck#1#2#3{%
  \count@=#1\relax
  \loop
    \ifnum#3=\catcode\count@
    \else
      \errmessage{%
        Character \the\count@\space
        with wrong catcode \the\catcode\count@\space
        instead of \number#3%
      }%
    \fi
  \ifnum\count@<#2\relax
    \advance\count@ 1 %
  \repeat
}
\def\space{ }
\expandafter\ifx\csname LoadCommand\endcsname\relax
  \def\LoadCommand{\input resizegather.sty\relax}%
\fi
\def\Test{%
  \RangeCatcodeInvalid{0}{47}%
  \RangeCatcodeInvalid{58}{64}%
  \RangeCatcodeInvalid{91}{96}%
  \RangeCatcodeInvalid{123}{255}%
  \catcode`\@=12 %
  \catcode`\\=0 %
  \catcode`\%=14 %
  \LoadCommand
  \RangeCatcodeCheck{0}{36}{15}%
  \RangeCatcodeCheck{37}{37}{14}%
  \RangeCatcodeCheck{38}{47}{15}%
  \RangeCatcodeCheck{48}{57}{12}%
  \RangeCatcodeCheck{58}{63}{15}%
  \RangeCatcodeCheck{64}{64}{12}%
  \RangeCatcodeCheck{65}{90}{11}%
  \RangeCatcodeCheck{91}{91}{15}%
  \RangeCatcodeCheck{92}{92}{0}%
  \RangeCatcodeCheck{93}{96}{15}%
  \RangeCatcodeCheck{97}{122}{11}%
  \RangeCatcodeCheck{123}{255}{15}%
  \RestoreCatcodes
}
\Test
\csname @@end\endcsname
\end
%    \end{macrocode}
%    \begin{macrocode}
%</test1>
%    \end{macrocode}
%
% \section{Installation}
%
% \subsection{Download}
%
% \paragraph{Package.} This package is available on
% CTAN\footnote{\url{https://ctan.org/pkg/resizegather}}:
% \begin{description}
% \item[\CTAN{macros/latex/contrib/oberdiek/resizegather.dtx}] The source file.
% \item[\CTAN{macros/latex/contrib/oberdiek/resizegather.pdf}] Documentation.
% \end{description}
%
%
% \paragraph{Bundle.} All the packages of the bundle `oberdiek'
% are also available in a TDS compliant ZIP archive. There
% the packages are already unpacked and the documentation files
% are generated. The files and directories obey the TDS standard.
% \begin{description}
% \item[\CTANinstall{install/macros/latex/contrib/oberdiek.tds.zip}]
% \end{description}
% \emph{TDS} refers to the standard ``A Directory Structure
% for \TeX\ Files'' (\CTAN{tds/tds.pdf}). Directories
% with \xfile{texmf} in their name are usually organized this way.
%
% \subsection{Bundle installation}
%
% \paragraph{Unpacking.} Unpack the \xfile{oberdiek.tds.zip} in the
% TDS tree (also known as \xfile{texmf} tree) of your choice.
% Example (linux):
% \begin{quote}
%   |unzip oberdiek.tds.zip -d ~/texmf|
% \end{quote}
%
% \paragraph{Script installation.}
% Check the directory \xfile{TDS:scripts/oberdiek/} for
% scripts that need further installation steps.
% Package \xpackage{attachfile2} comes with the Perl script
% \xfile{pdfatfi.pl} that should be installed in such a way
% that it can be called as \texttt{pdfatfi}.
% Example (linux):
% \begin{quote}
%   |chmod +x scripts/oberdiek/pdfatfi.pl|\\
%   |cp scripts/oberdiek/pdfatfi.pl /usr/local/bin/|
% \end{quote}
%
% \subsection{Package installation}
%
% \paragraph{Unpacking.} The \xfile{.dtx} file is a self-extracting
% \docstrip\ archive. The files are extracted by running the
% \xfile{.dtx} through \plainTeX:
% \begin{quote}
%   \verb|tex resizegather.dtx|
% \end{quote}
%
% \paragraph{TDS.} Now the different files must be moved into
% the different directories in your installation TDS tree
% (also known as \xfile{texmf} tree):
% \begin{quote}
% \def\t{^^A
% \begin{tabular}{@{}>{\ttfamily}l@{ $\rightarrow$ }>{\ttfamily}l@{}}
%   resizegather.sty & tex/latex/oberdiek/resizegather.sty\\
%   resizegather.pdf & doc/latex/oberdiek/resizegather.pdf\\
%   test/resizegather-test1.tex & doc/latex/oberdiek/test/resizegather-test1.tex\\
%   resizegather.dtx & source/latex/oberdiek/resizegather.dtx\\
% \end{tabular}^^A
% }^^A
% \sbox0{\t}^^A
% \ifdim\wd0>\linewidth
%   \begingroup
%     \advance\linewidth by\leftmargin
%     \advance\linewidth by\rightmargin
%   \edef\x{\endgroup
%     \def\noexpand\lw{\the\linewidth}^^A
%   }\x
%   \def\lwbox{^^A
%     \leavevmode
%     \hbox to \linewidth{^^A
%       \kern-\leftmargin\relax
%       \hss
%       \usebox0
%       \hss
%       \kern-\rightmargin\relax
%     }^^A
%   }^^A
%   \ifdim\wd0>\lw
%     \sbox0{\small\t}^^A
%     \ifdim\wd0>\linewidth
%       \ifdim\wd0>\lw
%         \sbox0{\footnotesize\t}^^A
%         \ifdim\wd0>\linewidth
%           \ifdim\wd0>\lw
%             \sbox0{\scriptsize\t}^^A
%             \ifdim\wd0>\linewidth
%               \ifdim\wd0>\lw
%                 \sbox0{\tiny\t}^^A
%                 \ifdim\wd0>\linewidth
%                   \lwbox
%                 \else
%                   \usebox0
%                 \fi
%               \else
%                 \lwbox
%               \fi
%             \else
%               \usebox0
%             \fi
%           \else
%             \lwbox
%           \fi
%         \else
%           \usebox0
%         \fi
%       \else
%         \lwbox
%       \fi
%     \else
%       \usebox0
%     \fi
%   \else
%     \lwbox
%   \fi
% \else
%   \usebox0
% \fi
% \end{quote}
% If you have a \xfile{docstrip.cfg} that configures and enables \docstrip's
% TDS installing feature, then some files can already be in the right
% place, see the documentation of \docstrip.
%
% \subsection{Refresh file name databases}
%
% If your \TeX~distribution
% (\teTeX, \mikTeX, \dots) relies on file name databases, you must refresh
% these. For example, \teTeX\ users run \verb|texhash| or
% \verb|mktexlsr|.
%
% \subsection{Some details for the interested}
%
% \paragraph{Attached source.}
%
% The PDF documentation on CTAN also includes the
% \xfile{.dtx} source file. It can be extracted by
% AcrobatReader 6 or higher. Another option is \textsf{pdftk},
% e.g. unpack the file into the current directory:
% \begin{quote}
%   \verb|pdftk resizegather.pdf unpack_files output .|
% \end{quote}
%
% \paragraph{Unpacking with \LaTeX.}
% The \xfile{.dtx} chooses its action depending on the format:
% \begin{description}
% \item[\plainTeX:] Run \docstrip\ and extract the files.
% \item[\LaTeX:] Generate the documentation.
% \end{description}
% If you insist on using \LaTeX\ for \docstrip\ (really,
% \docstrip\ does not need \LaTeX), then inform the autodetect routine
% about your intention:
% \begin{quote}
%   \verb|latex \let\install=y% \iffalse meta-comment
%
% File: resizegather.dtx
% Version: 2016/05/16 v1.3
% Info: Resize overly large equations
%
% Copyright (C) 2009, 2010 by
%    Heiko Oberdiek <heiko.oberdiek at googlemail.com>
%    2016
%    https://github.com/ho-tex/oberdiek/issues
%
% This work may be distributed and/or modified under the
% conditions of the LaTeX Project Public License, either
% version 1.3c of this license or (at your option) any later
% version. This version of this license is in
%    http://www.latex-project.org/lppl/lppl-1-3c.txt
% and the latest version of this license is in
%    http://www.latex-project.org/lppl.txt
% and version 1.3 or later is part of all distributions of
% LaTeX version 2005/12/01 or later.
%
% This work has the LPPL maintenance status "maintained".
%
% This Current Maintainer of this work is Heiko Oberdiek.
%
% This work consists of the main source file resizegather.dtx
% and the derived files
%    resizegather.sty, resizegather.pdf, resizegather.ins, resizegather.drv,
%    resizegather-test1.tex.
%
% Distribution:
%    CTAN:macros/latex/contrib/oberdiek/resizegather.dtx
%    CTAN:macros/latex/contrib/oberdiek/resizegather.pdf
%
% Unpacking:
%    (a) If resizegather.ins is present:
%           tex resizegather.ins
%    (b) Without resizegather.ins:
%           tex resizegather.dtx
%    (c) If you insist on using LaTeX
%           latex \let\install=y\input{resizegather.dtx}
%        (quote the arguments according to the demands of your shell)
%
% Documentation:
%    (a) If resizegather.drv is present:
%           latex resizegather.drv
%    (b) Without resizegather.drv:
%           latex resizegather.dtx; ...
%    The class ltxdoc loads the configuration file ltxdoc.cfg
%    if available. Here you can specify further options, e.g.
%    use A4 as paper format:
%       \PassOptionsToClass{a4paper}{article}
%
%    Programm calls to get the documentation (example):
%       pdflatex resizegather.dtx
%       makeindex -s gind.ist resizegather.idx
%       pdflatex resizegather.dtx
%       makeindex -s gind.ist resizegather.idx
%       pdflatex resizegather.dtx
%
% Installation:
%    TDS:tex/latex/oberdiek/resizegather.sty
%    TDS:doc/latex/oberdiek/resizegather.pdf
%    TDS:doc/latex/oberdiek/test/resizegather-test1.tex
%    TDS:source/latex/oberdiek/resizegather.dtx
%
%<*ignore>
\begingroup
  \catcode123=1 %
  \catcode125=2 %
  \def\x{LaTeX2e}%
\expandafter\endgroup
\ifcase 0\ifx\install y1\fi\expandafter
         \ifx\csname processbatchFile\endcsname\relax\else1\fi
         \ifx\fmtname\x\else 1\fi\relax
\else\csname fi\endcsname
%</ignore>
%<*install>
\input docstrip.tex
\Msg{************************************************************************}
\Msg{* Installation}
\Msg{* Package: resizegather 2016/05/16 v1.3 Resize overly large equations (HO)}
\Msg{************************************************************************}

\keepsilent
\askforoverwritefalse

\let\MetaPrefix\relax
\preamble

This is a generated file.

Project: resizegather
Version: 2016/05/16 v1.3

Copyright (C) 2009, 2010 by
   Heiko Oberdiek <heiko.oberdiek at googlemail.com>

This work may be distributed and/or modified under the
conditions of the LaTeX Project Public License, either
version 1.3c of this license or (at your option) any later
version. This version of this license is in
   http://www.latex-project.org/lppl/lppl-1-3c.txt
and the latest version of this license is in
   http://www.latex-project.org/lppl.txt
and version 1.3 or later is part of all distributions of
LaTeX version 2005/12/01 or later.

This work has the LPPL maintenance status "maintained".

This Current Maintainer of this work is Heiko Oberdiek.

This work consists of the main source file resizegather.dtx
and the derived files
   resizegather.sty, resizegather.pdf, resizegather.ins, resizegather.drv,
   resizegather-test1.tex.

\endpreamble
\let\MetaPrefix\DoubleperCent

\generate{%
  \file{resizegather.ins}{\from{resizegather.dtx}{install}}%
  \file{resizegather.drv}{\from{resizegather.dtx}{driver}}%
  \usedir{tex/latex/oberdiek}%
  \file{resizegather.sty}{\from{resizegather.dtx}{package}}%
%  \usedir{doc/latex/oberdiek/test}%
%  \file{resizegather-test1.tex}{\from{resizegather.dtx}{test1}}%
  \nopreamble
  \nopostamble
%  \usedir{source/latex/oberdiek/catalogue}%
%  \file{resizegather.xml}{\from{resizegather.dtx}{catalogue}}%
}

\catcode32=13\relax% active space
\let =\space%
\Msg{************************************************************************}
\Msg{*}
\Msg{* To finish the installation you have to move the following}
\Msg{* file into a directory searched by TeX:}
\Msg{*}
\Msg{*     resizegather.sty}
\Msg{*}
\Msg{* To produce the documentation run the file `resizegather.drv'}
\Msg{* through LaTeX.}
\Msg{*}
\Msg{* Happy TeXing!}
\Msg{*}
\Msg{************************************************************************}

\endbatchfile
%</install>
%<*ignore>
\fi
%</ignore>
%<*driver>
\NeedsTeXFormat{LaTeX2e}
\ProvidesFile{resizegather.drv}%
  [2016/05/16 v1.3 Resize overly large equations (HO)]%
\documentclass{ltxdoc}
\usepackage{holtxdoc}[2011/11/22]
\usepackage{ifluatex}
\ifluatex
\else
  \usepackage[T1]{fontenc}%
  \usepackage{textcomp}%
  \usepackage{lmodern}%
\fi
\begin{document}
  \DocInput{resizegather.dtx}%
\end{document}
%</driver>
% \fi
%
%
% \CharacterTable
%  {Upper-case    \A\B\C\D\E\F\G\H\I\J\K\L\M\N\O\P\Q\R\S\T\U\V\W\X\Y\Z
%   Lower-case    \a\b\c\d\e\f\g\h\i\j\k\l\m\n\o\p\q\r\s\t\u\v\w\x\y\z
%   Digits        \0\1\2\3\4\5\6\7\8\9
%   Exclamation   \!     Double quote  \"     Hash (number) \#
%   Dollar        \$     Percent       \%     Ampersand     \&
%   Acute accent  \'     Left paren    \(     Right paren   \)
%   Asterisk      \*     Plus          \+     Comma         \,
%   Minus         \-     Point         \.     Solidus       \/
%   Colon         \:     Semicolon     \;     Less than     \<
%   Equals        \=     Greater than  \>     Question mark \?
%   Commercial at \@     Left bracket  \[     Backslash     \\
%   Right bracket \]     Circumflex    \^     Underscore    \_
%   Grave accent  \`     Left brace    \{     Vertical bar  \|
%   Right brace   \}     Tilde         \~}
%
% \GetFileInfo{resizegather.drv}
%
% \title{The \xpackage{resizegather} package}
% \date{2016/05/16 v1.3}
% \author{Heiko Oberdiek\thanks
% {Please report any issues at https://github.com/ho-tex/oberdiek/issues}\\
% \xemail{heiko.oberdiek at googlemail.com}}
%
% \maketitle
%
% \begin{abstract}
% Equations that are too large are resized to fit the available
% space. The environment \textsf{gather} of package \xpackage{amsmath}
% is supported. Also the environments \textsf{equation} and
% \textsf{displaymath} are redefined using \textsf{gather}
% and its starred version.
% \end{abstract}
%
% \tableofcontents
%
% \makeatletter
% \def\env#1{^^A
%    \textsf{\@env#1*\@nil}^^A
% }%
% \def\@env#1*#2\@nil{^^A
%   #1^^A
%   \ifx\\#2\\^^A
%     \expandafter\@gobble
%   \else
%     \textasteriskcentered
%     \expandafter\@firstofone
%   \fi
%   {\@env#2\@nil}^^A
% }
% \makeatother
%
% \section{Documentation}
%
% Sometimes an equation is just a little to large to fit in the
% line. And breaking the equation across lines might be worse
% than downscaling the equation. This package implements this
% for the environments \env{gather} and \env{gather*} of
% package \xpackage{amsmath}. That package already measures
% the equations and simplifies the implementation of \xpackage{resizegather}
% that only needs to hook into \xpackage{amsmath}'s code to add
% the resizing feature.
%
% Resized equations are recorded in the \xext{log} file
% for small exceeds (default setting is smaller than five percent).
% Otherwise a warning is given.
%
% Also environments \env{equation} and \env{displaymath}
% are supported by redefining them using \env{gather}
% and \env{gather*}.
%
% \cs{[} and \cs{]} are not supported, because these macros
% are not in environment form that is required for
% \xpackage{amsmath}. The environment body is collected
% first to be able to process the body twice for measuring
% first.
%
% Also the environments using alignments are not supported.
% If a single equation line would be resized, the alignment
% would get lost. And resizing all equations of the alignment
% does not seem appropriate either.
%
% \subsection{Options}
%
% \begin{description}
% \item[\xoption{warningthreshold}:]
%   Print a warning if the original equation line exceeds
%   its available width by the given fraction.
%   Default is |0.05|: A warning is given if the equation
%   is too large by five percent.
%   Otherwise the exceed is recorded in the \xext{log} file
%   only.
% \end{description}
% The next options are boolean options. They are enabled
% by value |true| or if no value is given. They are switched
% off by value |false|.
% \begin{description}
% \item[\xoption{enable}:] The resize feature is active (default).
% \item[\xoption{disable}:] The complementary option for \xoption{enable},
%  added for convenience: |disable| (or |disable=true|) is the same
%  as |enable=false|.
% \item[\xoption{equations}:]
%   \LaTeX\ environments \textsf{equation} and \textsf{displaymath}
%   environments are redefined. These equations
%   are now using environment \env{gather} and
%   \env{gather*}. This is the default.
% \end{description}
% The following table shows additional options if you
% want to have finer control for the redefined
% environments:
% \begin{quote}
% \def\unchanged{\textit{unchanged}}
% \def\notprovided{\textit{not provided}}
% \begin{tabular}{l|ll}
% &\multicolumn{2}{c}{Environments}\\
% Option & \env{equation} & \env{displaymath}\\
% \hline
% \xoption{equations} & \env{gather} & \env{gather*}\\
% \xoption{equation} & \env{gather} & \unchanged\\
% \xoption{displaymath} & \unchanged & \env{gather*}\\
% \end{tabular}
% \end{quote}
% If such an option is switched off, the original meaning
% of the affected environments is restored.
%
% Options are evaluated in the following order:
% \begin{enumerate}
% \item
%  Configuration file \xfile{resizegather.cfg} using \cs{resizegathersetup}
%  if the file exists.
%  \item
%  Package options given for \cs{usepackage}.
%  \item
%  Later calls of \cs{resizegathersetup}.
% \end{enumerate}
% \begin{declcs}{resizegathersetup}\M{option list}
% \end{declcs}
% The options are key value options. Boolean options are enabled by
% default (without value) or by using the explicit value \texttt{true}.
% Value \texttt{false} disable the option.
%
% \subsection{Options for packages \xpackage{amsmath} or \xpackage{graphics}}
%
% The package loads the package \xpackage{amsmath} because is internally
% measures the equations first. Thus this package hooks into this code
% in order to resize the equations if they are too large.
% The resizing itself is done by \cs{resizebox} of package \xpackage{graphics}.
% If you need special options for these packages, just load them first or
% use global options when appropriate. Example:
% \begin{quote}
%\begin{verbatim}
%\usepackage[dvipdfm]{graphicx}% or graphics
%\usepackage[fleqn]{amsmath}
%\usepackage{resizegather}
%\end{verbatim}
%\end{quote}
%
% \StopEventually{
% }
%
% \section{Implementation}
%    \begin{macrocode}
%<*package>
%    \end{macrocode}
%    Reload check, especially if the package is not used with \LaTeX.
%    \begin{macrocode}
\begingroup\catcode61\catcode48\catcode32=10\relax%
  \catcode13=5 % ^^M
  \endlinechar=13 %
  \catcode35=6 % #
  \catcode39=12 % '
  \catcode44=12 % ,
  \catcode45=12 % -
  \catcode46=12 % .
  \catcode58=12 % :
  \catcode64=11 % @
  \catcode123=1 % {
  \catcode125=2 % }
  \expandafter\let\expandafter\x\csname ver@resizegather.sty\endcsname
  \ifx\x\relax % plain-TeX, first loading
  \else
    \def\empty{}%
    \ifx\x\empty % LaTeX, first loading,
      % variable is initialized, but \ProvidesPackage not yet seen
    \else
      \expandafter\ifx\csname PackageInfo\endcsname\relax
        \def\x#1#2{%
          \immediate\write-1{Package #1 Info: #2.}%
        }%
      \else
        \def\x#1#2{\PackageInfo{#1}{#2, stopped}}%
      \fi
      \x{resizegather}{The package is already loaded}%
      \aftergroup\endinput
    \fi
  \fi
\endgroup%
%    \end{macrocode}
%    Package identification:
%    \begin{macrocode}
\begingroup\catcode61\catcode48\catcode32=10\relax%
  \catcode13=5 % ^^M
  \endlinechar=13 %
  \catcode35=6 % #
  \catcode39=12 % '
  \catcode40=12 % (
  \catcode41=12 % )
  \catcode44=12 % ,
  \catcode45=12 % -
  \catcode46=12 % .
  \catcode47=12 % /
  \catcode58=12 % :
  \catcode64=11 % @
  \catcode91=12 % [
  \catcode93=12 % ]
  \catcode123=1 % {
  \catcode125=2 % }
  \expandafter\ifx\csname ProvidesPackage\endcsname\relax
    \def\x#1#2#3[#4]{\endgroup
      \immediate\write-1{Package: #3 #4}%
      \xdef#1{#4}%
    }%
  \else
    \def\x#1#2[#3]{\endgroup
      #2[{#3}]%
      \ifx#1\@undefined
        \xdef#1{#3}%
      \fi
      \ifx#1\relax
        \xdef#1{#3}%
      \fi
    }%
  \fi
\expandafter\x\csname ver@resizegather.sty\endcsname
\ProvidesPackage{resizegather}%
  [2016/05/16 v1.3 Resize overly large equations (HO)]%
%    \end{macrocode}
%
%    \begin{macrocode}
\begingroup\catcode61\catcode48\catcode32=10\relax%
  \catcode13=5 % ^^M
  \endlinechar=13 %
  \catcode123=1 % {
  \catcode125=2 % }
  \catcode64=11 % @
  \def\x{\endgroup
    \expandafter\edef\csname ResizeGather@AtEnd\endcsname{%
      \endlinechar=\the\endlinechar\relax
      \catcode13=\the\catcode13\relax
      \catcode32=\the\catcode32\relax
      \catcode35=\the\catcode35\relax
      \catcode61=\the\catcode61\relax
      \catcode64=\the\catcode64\relax
      \catcode123=\the\catcode123\relax
      \catcode125=\the\catcode125\relax
    }%
  }%
\x\catcode61\catcode48\catcode32=10\relax%
\catcode13=5 % ^^M
\endlinechar=13 %
\catcode35=6 % #
\catcode64=11 % @
\catcode123=1 % {
\catcode125=2 % }
\def\TMP@EnsureCode#1#2{%
  \edef\ResizeGather@AtEnd{%
    \ResizeGather@AtEnd
    \catcode#1=\the\catcode#1\relax
  }%
  \catcode#1=#2\relax
}
\TMP@EnsureCode{10}{12}% ^^J
\TMP@EnsureCode{33}{12}% !
\TMP@EnsureCode{36}{3}% $
\TMP@EnsureCode{38}{4}% &
\TMP@EnsureCode{39}{12}% '
\TMP@EnsureCode{40}{12}% (
\TMP@EnsureCode{41}{12}% )
\TMP@EnsureCode{42}{12}% *
\TMP@EnsureCode{43}{12}% +
\TMP@EnsureCode{44}{12}% ,
\TMP@EnsureCode{45}{12}% -
\TMP@EnsureCode{46}{12}% .
\TMP@EnsureCode{47}{12}% /
\TMP@EnsureCode{58}{12}% :
\TMP@EnsureCode{59}{12}% ;
\TMP@EnsureCode{60}{12}% <
\TMP@EnsureCode{62}{12}% >
\TMP@EnsureCode{63}{12}% ?
\TMP@EnsureCode{91}{12}% [
\TMP@EnsureCode{93}{12}% ]
\TMP@EnsureCode{94}{7}% ^ (superscript)
\TMP@EnsureCode{96}{12}% `
\edef\ResizeGather@AtEnd{\ResizeGather@AtEnd\noexpand\endinput}
%    \end{macrocode}
%
%    \begin{macrocode}
\RequirePackage{kvoptions}[2009/12/04]
\SetupKeyvalOptions{%
  family=resizegather,%
  prefix=ResizeGather@,%
}
%    \end{macrocode}
%    \begin{macrocode}
\@for\ResizeGather@option:=%
  centertags,%
  tbtags,%
  sumlimits,%
  nosumlimits,%
  intlimits,%
  nointlimits,%
  nonamelimits,%
  leqno,%
  reqno,%
  fleqn%
\do{%
  \edef\ResizeGather@temp{%
    \noexpand\DeclareVoidOption{\ResizeGather@option}{%
      \noexpand\PassOptionsToPackage{amsmath}{\ResizeGather@option}%
    }%
    \noexpand\AtEndOfPackage{%
      \noexpand\DisableKeyvalOption[%
        action=error,%
        package=resizegather,%
      ]{resizegather}{\ResizeGather@option}%
    }%
  }%
  \ResizeGather@temp
}
\@for\ResizeGather@option:=%
  draft,%
  final,%
  hiderotate,%
  hidescale,%
  hiresbb,%
  demo,%
  dvips,xdvi,dvipdf,dvipdfm,dvipdfmx,pdftex,dvipsone,%
  dviwindo,emtex,dviwin,pctexps,pctexwin,pctexhp,pctex32,%
  truetex,tcidvi,vtex,oztex,textures,xetex%
\do{%
  \edef\ResizeGather@temp{%
    \noexpand\DeclareVoidOption{\ResizeGather@option}{%
      \noexpand\PassOptionsToPackage{graphics}{\ResizeGather@option}%
    }%
    \noexpand\AtEndOfPackage{%
      \noexpand\DisableKeyvalOption[%
        action=error,%
        package=resizegather,%
      ]{resizegather}{\ResizeGather@option}%
    }%
  }%
  \ResizeGather@temp
}
%    \end{macrocode}
%
%    \begin{macrocode}
\DeclareBoolOption[true]{enable}
\DeclareComplementaryOption{disable}{enable}
\DeclareStringOption[.05]{warningthreshold}
\newif\ifResizeGather@NeedInit
\DeclareBoolOption[true]{equations}
\DeclareBoolOption[true]{equation}
\DeclareBoolOption[true]{displaymath}
\AddToKeyvalOption*{equations}{%
  \ResizeGather@NeedInittrue
  \ifResizeGather@equations
    \ResizeGather@equationtrue
    \ResizeGather@displaymathtrue
  \else
    \ResizeGather@equationfalse
    \ResizeGather@displaymathfalse
  \fi
}
\AddToKeyvalOption*{equation}{%
  \ResizeGather@NeedInittrue
}
\AddToKeyvalOption*{displaymath}{%
  \ResizeGather@NeedInittrue
}
%    \end{macrocode}
%
%    \begin{macro}{\resizegathersetup}
%    \begin{macrocode}
\newcommand*{\resizegathersetup}[1]{%
  \ResizeGather@NeedInitfalse
  \setkeys{resizegather}{#1}%
  \ifResizeGather@NeedInit
    \ResizeGather@init
  \fi
}
\let\ResizeGather@init\relax
%    \end{macrocode}
%    \end{macro}
%    \begin{macrocode}
\InputIfFileExists{resizegather.cfg}{}{}%
\ProcessKeyvalOptions*\relax
%    \end{macrocode}
%    \begin{macrocode}
\RequirePackage{amsmath}
\RequirePackage{graphics}
%    \end{macrocode}
%    \begin{macro}{\ResizeGather@redefine}
%    \begin{macrocode}
\def\ResizeGather@redefine#1#2#3#4#5{%
  \csname ifResizeGather@#1\endcsname
    \@ifundefined{ResizeGather@org@#2}{%
      \expandafter\let\csname ResizeGather@org@#2\expandafter\endcsname
                      \csname #2\endcsname
    }{}%
    \@ifundefined{ResizeGather@org@#3}{%
      \expandafter\let\csname ResizeGather@org@#3\expandafter\endcsname
                      \csname #3\endcsname
    }{}%
    \expandafter\edef\csname #2\endcsname{%
      \expandafter\noexpand\csname#4\endcsname
    }%
    \expandafter\edef\csname #3\endcsname{%
      \expandafter\noexpand\csname#5\endcsname
    }%
  \else
    \@ifundefined{ResizeGather@org@#2}{}{%
      \expandafter\let\csname #2\expandafter\endcsname
                      \csname ResizeGather@org@#2\endcsname
      \expandafter\let\csname #3\expandafter\endcsname
                      \csname ResizeGather@org@#3\endcsname
    }%
  \fi
}
%    \end{macrocode}
%    \end{macro}
%    \begin{macro}{\ResizeGather@init}
%    \begin{macrocode}
\def\ResizeGather@init{%
  \ResizeGather@redefine{equation}{equation}{endequation}%
    {gather}{endgather}%
  \ResizeGather@redefine{displaymath}{displaymath}{enddisplaymath}%
    {gather*}{endgather*}%
}
\ResizeGather@init
%    \end{macrocode}
%    \end{macro}
%
%    \begin{macro}{\ResizeGather@ResizeGather}
%    \begin{macrocode}
\def\ResizeGather@ResizeGather{%
  \ifResizeGather@enable
    \dimen@\displaywidth
    \if@fleqn
      \advance\dimen@-\@mathmargin
    \fi
    \ifdim\wdz@>\dimen@
      \begingroup
        \advance\dimen@ -\wdz@
        \dimen@ -\dimen@
        \ifdim\ResizeGather@warningthreshold\wdz@>\dimen@
          \expandafter\PackageInfo
        \else
          \expandafter\PackageWarning
        \fi
        {resizegather}{%
          Equation line \the\row@\space is too large %
          by \the\dimen@\MessageBreak
          in environment `\@currenvir'%
        }%
      \endgroup
      \setboxz@h to\dimen@{%
        \resizebox{\dimen@}{!}{\boxz@}%
        \hss
      }%
    \fi
  \fi
}
%    \end{macrocode}
%    \end{macro}
%    \begin{macro}{\calc@shift@gather}
%    \begin{macrocode}
\expandafter\def\expandafter\calc@shift@gather\expandafter{%
  \expandafter\ResizeGather@ResizeGather
  \calc@shift@gather
}
%    \end{macrocode}
%    \end{macro}
%    \begin{macro}{\ResizeGather@org@gmeasure@}
%    \begin{macrocode}
\let\ResizeGather@org@gmeasure@\gmeasure@
%    \end{macrocode}
%    \end{macro}
%    \begin{macro}{\gmeasure@}
%    \begin{macrocode}
\def\gmeasure@#1{%
  \ResizeGather@org@gmeasure@{#1}%
  \ifResizeGather@enable
    \ifdim\totwidth@>\displaywidth
      \totwidth@=\displaywidth
    \fi
  \fi
}
%    \end{macrocode}
%    \end{macro}
%
%    \begin{macrocode}
\ResizeGather@AtEnd%
%</package>
%    \end{macrocode}
%
% \section{Test}
%
% \subsection{Catcode checks for loading}
%
%    \begin{macrocode}
%<*test1>
%    \end{macrocode}
%    \begin{macrocode}
\catcode`\{=1 %
\catcode`\}=2 %
\catcode`\#=6 %
\catcode`\@=11 %
\expandafter\ifx\csname count@\endcsname\relax
  \countdef\count@=255 %
\fi
\expandafter\ifx\csname @gobble\endcsname\relax
  \long\def\@gobble#1{}%
\fi
\expandafter\ifx\csname @firstofone\endcsname\relax
  \long\def\@firstofone#1{#1}%
\fi
\expandafter\ifx\csname loop\endcsname\relax
  \expandafter\@firstofone
\else
  \expandafter\@gobble
\fi
{%
  \def\loop#1\repeat{%
    \def\body{#1}%
    \iterate
  }%
  \def\iterate{%
    \body
      \let\next\iterate
    \else
      \let\next\relax
    \fi
    \next
  }%
  \let\repeat=\fi
}%
\def\RestoreCatcodes{}
\count@=0 %
\loop
  \edef\RestoreCatcodes{%
    \RestoreCatcodes
    \catcode\the\count@=\the\catcode\count@\relax
  }%
\ifnum\count@<255 %
  \advance\count@ 1 %
\repeat

\def\RangeCatcodeInvalid#1#2{%
  \count@=#1\relax
  \loop
    \catcode\count@=15 %
  \ifnum\count@<#2\relax
    \advance\count@ 1 %
  \repeat
}
\def\RangeCatcodeCheck#1#2#3{%
  \count@=#1\relax
  \loop
    \ifnum#3=\catcode\count@
    \else
      \errmessage{%
        Character \the\count@\space
        with wrong catcode \the\catcode\count@\space
        instead of \number#3%
      }%
    \fi
  \ifnum\count@<#2\relax
    \advance\count@ 1 %
  \repeat
}
\def\space{ }
\expandafter\ifx\csname LoadCommand\endcsname\relax
  \def\LoadCommand{\input resizegather.sty\relax}%
\fi
\def\Test{%
  \RangeCatcodeInvalid{0}{47}%
  \RangeCatcodeInvalid{58}{64}%
  \RangeCatcodeInvalid{91}{96}%
  \RangeCatcodeInvalid{123}{255}%
  \catcode`\@=12 %
  \catcode`\\=0 %
  \catcode`\%=14 %
  \LoadCommand
  \RangeCatcodeCheck{0}{36}{15}%
  \RangeCatcodeCheck{37}{37}{14}%
  \RangeCatcodeCheck{38}{47}{15}%
  \RangeCatcodeCheck{48}{57}{12}%
  \RangeCatcodeCheck{58}{63}{15}%
  \RangeCatcodeCheck{64}{64}{12}%
  \RangeCatcodeCheck{65}{90}{11}%
  \RangeCatcodeCheck{91}{91}{15}%
  \RangeCatcodeCheck{92}{92}{0}%
  \RangeCatcodeCheck{93}{96}{15}%
  \RangeCatcodeCheck{97}{122}{11}%
  \RangeCatcodeCheck{123}{255}{15}%
  \RestoreCatcodes
}
\Test
\csname @@end\endcsname
\end
%    \end{macrocode}
%    \begin{macrocode}
%</test1>
%    \end{macrocode}
%
% \section{Installation}
%
% \subsection{Download}
%
% \paragraph{Package.} This package is available on
% CTAN\footnote{\url{https://ctan.org/pkg/resizegather}}:
% \begin{description}
% \item[\CTAN{macros/latex/contrib/oberdiek/resizegather.dtx}] The source file.
% \item[\CTAN{macros/latex/contrib/oberdiek/resizegather.pdf}] Documentation.
% \end{description}
%
%
% \paragraph{Bundle.} All the packages of the bundle `oberdiek'
% are also available in a TDS compliant ZIP archive. There
% the packages are already unpacked and the documentation files
% are generated. The files and directories obey the TDS standard.
% \begin{description}
% \item[\CTANinstall{install/macros/latex/contrib/oberdiek.tds.zip}]
% \end{description}
% \emph{TDS} refers to the standard ``A Directory Structure
% for \TeX\ Files'' (\CTAN{tds/tds.pdf}). Directories
% with \xfile{texmf} in their name are usually organized this way.
%
% \subsection{Bundle installation}
%
% \paragraph{Unpacking.} Unpack the \xfile{oberdiek.tds.zip} in the
% TDS tree (also known as \xfile{texmf} tree) of your choice.
% Example (linux):
% \begin{quote}
%   |unzip oberdiek.tds.zip -d ~/texmf|
% \end{quote}
%
% \paragraph{Script installation.}
% Check the directory \xfile{TDS:scripts/oberdiek/} for
% scripts that need further installation steps.
% Package \xpackage{attachfile2} comes with the Perl script
% \xfile{pdfatfi.pl} that should be installed in such a way
% that it can be called as \texttt{pdfatfi}.
% Example (linux):
% \begin{quote}
%   |chmod +x scripts/oberdiek/pdfatfi.pl|\\
%   |cp scripts/oberdiek/pdfatfi.pl /usr/local/bin/|
% \end{quote}
%
% \subsection{Package installation}
%
% \paragraph{Unpacking.} The \xfile{.dtx} file is a self-extracting
% \docstrip\ archive. The files are extracted by running the
% \xfile{.dtx} through \plainTeX:
% \begin{quote}
%   \verb|tex resizegather.dtx|
% \end{quote}
%
% \paragraph{TDS.} Now the different files must be moved into
% the different directories in your installation TDS tree
% (also known as \xfile{texmf} tree):
% \begin{quote}
% \def\t{^^A
% \begin{tabular}{@{}>{\ttfamily}l@{ $\rightarrow$ }>{\ttfamily}l@{}}
%   resizegather.sty & tex/latex/oberdiek/resizegather.sty\\
%   resizegather.pdf & doc/latex/oberdiek/resizegather.pdf\\
%   test/resizegather-test1.tex & doc/latex/oberdiek/test/resizegather-test1.tex\\
%   resizegather.dtx & source/latex/oberdiek/resizegather.dtx\\
% \end{tabular}^^A
% }^^A
% \sbox0{\t}^^A
% \ifdim\wd0>\linewidth
%   \begingroup
%     \advance\linewidth by\leftmargin
%     \advance\linewidth by\rightmargin
%   \edef\x{\endgroup
%     \def\noexpand\lw{\the\linewidth}^^A
%   }\x
%   \def\lwbox{^^A
%     \leavevmode
%     \hbox to \linewidth{^^A
%       \kern-\leftmargin\relax
%       \hss
%       \usebox0
%       \hss
%       \kern-\rightmargin\relax
%     }^^A
%   }^^A
%   \ifdim\wd0>\lw
%     \sbox0{\small\t}^^A
%     \ifdim\wd0>\linewidth
%       \ifdim\wd0>\lw
%         \sbox0{\footnotesize\t}^^A
%         \ifdim\wd0>\linewidth
%           \ifdim\wd0>\lw
%             \sbox0{\scriptsize\t}^^A
%             \ifdim\wd0>\linewidth
%               \ifdim\wd0>\lw
%                 \sbox0{\tiny\t}^^A
%                 \ifdim\wd0>\linewidth
%                   \lwbox
%                 \else
%                   \usebox0
%                 \fi
%               \else
%                 \lwbox
%               \fi
%             \else
%               \usebox0
%             \fi
%           \else
%             \lwbox
%           \fi
%         \else
%           \usebox0
%         \fi
%       \else
%         \lwbox
%       \fi
%     \else
%       \usebox0
%     \fi
%   \else
%     \lwbox
%   \fi
% \else
%   \usebox0
% \fi
% \end{quote}
% If you have a \xfile{docstrip.cfg} that configures and enables \docstrip's
% TDS installing feature, then some files can already be in the right
% place, see the documentation of \docstrip.
%
% \subsection{Refresh file name databases}
%
% If your \TeX~distribution
% (\teTeX, \mikTeX, \dots) relies on file name databases, you must refresh
% these. For example, \teTeX\ users run \verb|texhash| or
% \verb|mktexlsr|.
%
% \subsection{Some details for the interested}
%
% \paragraph{Attached source.}
%
% The PDF documentation on CTAN also includes the
% \xfile{.dtx} source file. It can be extracted by
% AcrobatReader 6 or higher. Another option is \textsf{pdftk},
% e.g. unpack the file into the current directory:
% \begin{quote}
%   \verb|pdftk resizegather.pdf unpack_files output .|
% \end{quote}
%
% \paragraph{Unpacking with \LaTeX.}
% The \xfile{.dtx} chooses its action depending on the format:
% \begin{description}
% \item[\plainTeX:] Run \docstrip\ and extract the files.
% \item[\LaTeX:] Generate the documentation.
% \end{description}
% If you insist on using \LaTeX\ for \docstrip\ (really,
% \docstrip\ does not need \LaTeX), then inform the autodetect routine
% about your intention:
% \begin{quote}
%   \verb|latex \let\install=y\input{resizegather.dtx}|
% \end{quote}
% Do not forget to quote the argument according to the demands
% of your shell.
%
% \paragraph{Generating the documentation.}
% You can use both the \xfile{.dtx} or the \xfile{.drv} to generate
% the documentation. The process can be configured by the
% configuration file \xfile{ltxdoc.cfg}. For instance, put this
% line into this file, if you want to have A4 as paper format:
% \begin{quote}
%   \verb|\PassOptionsToClass{a4paper}{article}|
% \end{quote}
% An example follows how to generate the
% documentation with pdf\LaTeX:
% \begin{quote}
%\begin{verbatim}
%pdflatex resizegather.dtx
%makeindex -s gind.ist resizegather.idx
%pdflatex resizegather.dtx
%makeindex -s gind.ist resizegather.idx
%pdflatex resizegather.dtx
%\end{verbatim}
% \end{quote}
%
% \section{Acknowledgement}
%
% \begin{description}
% \item[Dieter Jurzitza:]
% He wanted the resizing feature at the \TeX\ table
% in Karlsruhe of December 2009. Thus this package is a kind of
% Christmas present.
% \end{description}
%
% \begin{History}
%   \begin{Version}{2009/12/04 v1.0}
%   \item
%     The first version.
%   \end{Version}
%   \begin{Version}{2009/12/05 v1.1}
%   \item
%     Options \xoption{enable} and \xoption{disable} added.
%   \end{Version}
%   \begin{Version}{2010/03/01 v1.2}
%   \item
%     TDS location moved from `generic' to `latex'.
%   \end{Version}
%   \begin{Version}{2016/05/16 v1.3}
%   \item
%     Documentation updates.
%   \end{Version}
% \end{History}
%
% \PrintIndex
%
% \Finale
\endinput
|
% \end{quote}
% Do not forget to quote the argument according to the demands
% of your shell.
%
% \paragraph{Generating the documentation.}
% You can use both the \xfile{.dtx} or the \xfile{.drv} to generate
% the documentation. The process can be configured by the
% configuration file \xfile{ltxdoc.cfg}. For instance, put this
% line into this file, if you want to have A4 as paper format:
% \begin{quote}
%   \verb|\PassOptionsToClass{a4paper}{article}|
% \end{quote}
% An example follows how to generate the
% documentation with pdf\LaTeX:
% \begin{quote}
%\begin{verbatim}
%pdflatex resizegather.dtx
%makeindex -s gind.ist resizegather.idx
%pdflatex resizegather.dtx
%makeindex -s gind.ist resizegather.idx
%pdflatex resizegather.dtx
%\end{verbatim}
% \end{quote}
%
% \section{Acknowledgement}
%
% \begin{description}
% \item[Dieter Jurzitza:]
% He wanted the resizing feature at the \TeX\ table
% in Karlsruhe of December 2009. Thus this package is a kind of
% Christmas present.
% \end{description}
%
% \begin{History}
%   \begin{Version}{2009/12/04 v1.0}
%   \item
%     The first version.
%   \end{Version}
%   \begin{Version}{2009/12/05 v1.1}
%   \item
%     Options \xoption{enable} and \xoption{disable} added.
%   \end{Version}
%   \begin{Version}{2010/03/01 v1.2}
%   \item
%     TDS location moved from `generic' to `latex'.
%   \end{Version}
%   \begin{Version}{2016/05/16 v1.3}
%   \item
%     Documentation updates.
%   \end{Version}
% \end{History}
%
% \PrintIndex
%
% \Finale
\endinput

%        (quote the arguments according to the demands of your shell)
%
% Documentation:
%    (a) If resizegather.drv is present:
%           latex resizegather.drv
%    (b) Without resizegather.drv:
%           latex resizegather.dtx; ...
%    The class ltxdoc loads the configuration file ltxdoc.cfg
%    if available. Here you can specify further options, e.g.
%    use A4 as paper format:
%       \PassOptionsToClass{a4paper}{article}
%
%    Programm calls to get the documentation (example):
%       pdflatex resizegather.dtx
%       makeindex -s gind.ist resizegather.idx
%       pdflatex resizegather.dtx
%       makeindex -s gind.ist resizegather.idx
%       pdflatex resizegather.dtx
%
% Installation:
%    TDS:tex/latex/oberdiek/resizegather.sty
%    TDS:doc/latex/oberdiek/resizegather.pdf
%    TDS:doc/latex/oberdiek/test/resizegather-test1.tex
%    TDS:source/latex/oberdiek/resizegather.dtx
%
%<*ignore>
\begingroup
  \catcode123=1 %
  \catcode125=2 %
  \def\x{LaTeX2e}%
\expandafter\endgroup
\ifcase 0\ifx\install y1\fi\expandafter
         \ifx\csname processbatchFile\endcsname\relax\else1\fi
         \ifx\fmtname\x\else 1\fi\relax
\else\csname fi\endcsname
%</ignore>
%<*install>
\input docstrip.tex
\Msg{************************************************************************}
\Msg{* Installation}
\Msg{* Package: resizegather 2016/05/16 v1.3 Resize overly large equations (HO)}
\Msg{************************************************************************}

\keepsilent
\askforoverwritefalse

\let\MetaPrefix\relax
\preamble

This is a generated file.

Project: resizegather
Version: 2016/05/16 v1.3

Copyright (C) 2009, 2010 by
   Heiko Oberdiek <heiko.oberdiek at googlemail.com>

This work may be distributed and/or modified under the
conditions of the LaTeX Project Public License, either
version 1.3c of this license or (at your option) any later
version. This version of this license is in
   http://www.latex-project.org/lppl/lppl-1-3c.txt
and the latest version of this license is in
   http://www.latex-project.org/lppl.txt
and version 1.3 or later is part of all distributions of
LaTeX version 2005/12/01 or later.

This work has the LPPL maintenance status "maintained".

This Current Maintainer of this work is Heiko Oberdiek.

This work consists of the main source file resizegather.dtx
and the derived files
   resizegather.sty, resizegather.pdf, resizegather.ins, resizegather.drv,
   resizegather-test1.tex.

\endpreamble
\let\MetaPrefix\DoubleperCent

\generate{%
  \file{resizegather.ins}{\from{resizegather.dtx}{install}}%
  \file{resizegather.drv}{\from{resizegather.dtx}{driver}}%
  \usedir{tex/latex/oberdiek}%
  \file{resizegather.sty}{\from{resizegather.dtx}{package}}%
%  \usedir{doc/latex/oberdiek/test}%
%  \file{resizegather-test1.tex}{\from{resizegather.dtx}{test1}}%
  \nopreamble
  \nopostamble
%  \usedir{source/latex/oberdiek/catalogue}%
%  \file{resizegather.xml}{\from{resizegather.dtx}{catalogue}}%
}

\catcode32=13\relax% active space
\let =\space%
\Msg{************************************************************************}
\Msg{*}
\Msg{* To finish the installation you have to move the following}
\Msg{* file into a directory searched by TeX:}
\Msg{*}
\Msg{*     resizegather.sty}
\Msg{*}
\Msg{* To produce the documentation run the file `resizegather.drv'}
\Msg{* through LaTeX.}
\Msg{*}
\Msg{* Happy TeXing!}
\Msg{*}
\Msg{************************************************************************}

\endbatchfile
%</install>
%<*ignore>
\fi
%</ignore>
%<*driver>
\NeedsTeXFormat{LaTeX2e}
\ProvidesFile{resizegather.drv}%
  [2016/05/16 v1.3 Resize overly large equations (HO)]%
\documentclass{ltxdoc}
\usepackage{holtxdoc}[2011/11/22]
\usepackage{ifluatex}
\ifluatex
\else
  \usepackage[T1]{fontenc}%
  \usepackage{textcomp}%
  \usepackage{lmodern}%
\fi
\begin{document}
  \DocInput{resizegather.dtx}%
\end{document}
%</driver>
% \fi
%
%
% \CharacterTable
%  {Upper-case    \A\B\C\D\E\F\G\H\I\J\K\L\M\N\O\P\Q\R\S\T\U\V\W\X\Y\Z
%   Lower-case    \a\b\c\d\e\f\g\h\i\j\k\l\m\n\o\p\q\r\s\t\u\v\w\x\y\z
%   Digits        \0\1\2\3\4\5\6\7\8\9
%   Exclamation   \!     Double quote  \"     Hash (number) \#
%   Dollar        \$     Percent       \%     Ampersand     \&
%   Acute accent  \'     Left paren    \(     Right paren   \)
%   Asterisk      \*     Plus          \+     Comma         \,
%   Minus         \-     Point         \.     Solidus       \/
%   Colon         \:     Semicolon     \;     Less than     \<
%   Equals        \=     Greater than  \>     Question mark \?
%   Commercial at \@     Left bracket  \[     Backslash     \\
%   Right bracket \]     Circumflex    \^     Underscore    \_
%   Grave accent  \`     Left brace    \{     Vertical bar  \|
%   Right brace   \}     Tilde         \~}
%
% \GetFileInfo{resizegather.drv}
%
% \title{The \xpackage{resizegather} package}
% \date{2016/05/16 v1.3}
% \author{Heiko Oberdiek\thanks
% {Please report any issues at https://github.com/ho-tex/oberdiek/issues}\\
% \xemail{heiko.oberdiek at googlemail.com}}
%
% \maketitle
%
% \begin{abstract}
% Equations that are too large are resized to fit the available
% space. The environment \textsf{gather} of package \xpackage{amsmath}
% is supported. Also the environments \textsf{equation} and
% \textsf{displaymath} are redefined using \textsf{gather}
% and its starred version.
% \end{abstract}
%
% \tableofcontents
%
% \makeatletter
% \def\env#1{^^A
%    \textsf{\@env#1*\@nil}^^A
% }%
% \def\@env#1*#2\@nil{^^A
%   #1^^A
%   \ifx\\#2\\^^A
%     \expandafter\@gobble
%   \else
%     \textasteriskcentered
%     \expandafter\@firstofone
%   \fi
%   {\@env#2\@nil}^^A
% }
% \makeatother
%
% \section{Documentation}
%
% Sometimes an equation is just a little to large to fit in the
% line. And breaking the equation across lines might be worse
% than downscaling the equation. This package implements this
% for the environments \env{gather} and \env{gather*} of
% package \xpackage{amsmath}. That package already measures
% the equations and simplifies the implementation of \xpackage{resizegather}
% that only needs to hook into \xpackage{amsmath}'s code to add
% the resizing feature.
%
% Resized equations are recorded in the \xext{log} file
% for small exceeds (default setting is smaller than five percent).
% Otherwise a warning is given.
%
% Also environments \env{equation} and \env{displaymath}
% are supported by redefining them using \env{gather}
% and \env{gather*}.
%
% \cs{[} and \cs{]} are not supported, because these macros
% are not in environment form that is required for
% \xpackage{amsmath}. The environment body is collected
% first to be able to process the body twice for measuring
% first.
%
% Also the environments using alignments are not supported.
% If a single equation line would be resized, the alignment
% would get lost. And resizing all equations of the alignment
% does not seem appropriate either.
%
% \subsection{Options}
%
% \begin{description}
% \item[\xoption{warningthreshold}:]
%   Print a warning if the original equation line exceeds
%   its available width by the given fraction.
%   Default is |0.05|: A warning is given if the equation
%   is too large by five percent.
%   Otherwise the exceed is recorded in the \xext{log} file
%   only.
% \end{description}
% The next options are boolean options. They are enabled
% by value |true| or if no value is given. They are switched
% off by value |false|.
% \begin{description}
% \item[\xoption{enable}:] The resize feature is active (default).
% \item[\xoption{disable}:] The complementary option for \xoption{enable},
%  added for convenience: |disable| (or |disable=true|) is the same
%  as |enable=false|.
% \item[\xoption{equations}:]
%   \LaTeX\ environments \textsf{equation} and \textsf{displaymath}
%   environments are redefined. These equations
%   are now using environment \env{gather} and
%   \env{gather*}. This is the default.
% \end{description}
% The following table shows additional options if you
% want to have finer control for the redefined
% environments:
% \begin{quote}
% \def\unchanged{\textit{unchanged}}
% \def\notprovided{\textit{not provided}}
% \begin{tabular}{l|ll}
% &\multicolumn{2}{c}{Environments}\\
% Option & \env{equation} & \env{displaymath}\\
% \hline
% \xoption{equations} & \env{gather} & \env{gather*}\\
% \xoption{equation} & \env{gather} & \unchanged\\
% \xoption{displaymath} & \unchanged & \env{gather*}\\
% \end{tabular}
% \end{quote}
% If such an option is switched off, the original meaning
% of the affected environments is restored.
%
% Options are evaluated in the following order:
% \begin{enumerate}
% \item
%  Configuration file \xfile{resizegather.cfg} using \cs{resizegathersetup}
%  if the file exists.
%  \item
%  Package options given for \cs{usepackage}.
%  \item
%  Later calls of \cs{resizegathersetup}.
% \end{enumerate}
% \begin{declcs}{resizegathersetup}\M{option list}
% \end{declcs}
% The options are key value options. Boolean options are enabled by
% default (without value) or by using the explicit value \texttt{true}.
% Value \texttt{false} disable the option.
%
% \subsection{Options for packages \xpackage{amsmath} or \xpackage{graphics}}
%
% The package loads the package \xpackage{amsmath} because is internally
% measures the equations first. Thus this package hooks into this code
% in order to resize the equations if they are too large.
% The resizing itself is done by \cs{resizebox} of package \xpackage{graphics}.
% If you need special options for these packages, just load them first or
% use global options when appropriate. Example:
% \begin{quote}
%\begin{verbatim}
%\usepackage[dvipdfm]{graphicx}% or graphics
%\usepackage[fleqn]{amsmath}
%\usepackage{resizegather}
%\end{verbatim}
%\end{quote}
%
% \StopEventually{
% }
%
% \section{Implementation}
%    \begin{macrocode}
%<*package>
%    \end{macrocode}
%    Reload check, especially if the package is not used with \LaTeX.
%    \begin{macrocode}
\begingroup\catcode61\catcode48\catcode32=10\relax%
  \catcode13=5 % ^^M
  \endlinechar=13 %
  \catcode35=6 % #
  \catcode39=12 % '
  \catcode44=12 % ,
  \catcode45=12 % -
  \catcode46=12 % .
  \catcode58=12 % :
  \catcode64=11 % @
  \catcode123=1 % {
  \catcode125=2 % }
  \expandafter\let\expandafter\x\csname ver@resizegather.sty\endcsname
  \ifx\x\relax % plain-TeX, first loading
  \else
    \def\empty{}%
    \ifx\x\empty % LaTeX, first loading,
      % variable is initialized, but \ProvidesPackage not yet seen
    \else
      \expandafter\ifx\csname PackageInfo\endcsname\relax
        \def\x#1#2{%
          \immediate\write-1{Package #1 Info: #2.}%
        }%
      \else
        \def\x#1#2{\PackageInfo{#1}{#2, stopped}}%
      \fi
      \x{resizegather}{The package is already loaded}%
      \aftergroup\endinput
    \fi
  \fi
\endgroup%
%    \end{macrocode}
%    Package identification:
%    \begin{macrocode}
\begingroup\catcode61\catcode48\catcode32=10\relax%
  \catcode13=5 % ^^M
  \endlinechar=13 %
  \catcode35=6 % #
  \catcode39=12 % '
  \catcode40=12 % (
  \catcode41=12 % )
  \catcode44=12 % ,
  \catcode45=12 % -
  \catcode46=12 % .
  \catcode47=12 % /
  \catcode58=12 % :
  \catcode64=11 % @
  \catcode91=12 % [
  \catcode93=12 % ]
  \catcode123=1 % {
  \catcode125=2 % }
  \expandafter\ifx\csname ProvidesPackage\endcsname\relax
    \def\x#1#2#3[#4]{\endgroup
      \immediate\write-1{Package: #3 #4}%
      \xdef#1{#4}%
    }%
  \else
    \def\x#1#2[#3]{\endgroup
      #2[{#3}]%
      \ifx#1\@undefined
        \xdef#1{#3}%
      \fi
      \ifx#1\relax
        \xdef#1{#3}%
      \fi
    }%
  \fi
\expandafter\x\csname ver@resizegather.sty\endcsname
\ProvidesPackage{resizegather}%
  [2016/05/16 v1.3 Resize overly large equations (HO)]%
%    \end{macrocode}
%
%    \begin{macrocode}
\begingroup\catcode61\catcode48\catcode32=10\relax%
  \catcode13=5 % ^^M
  \endlinechar=13 %
  \catcode123=1 % {
  \catcode125=2 % }
  \catcode64=11 % @
  \def\x{\endgroup
    \expandafter\edef\csname ResizeGather@AtEnd\endcsname{%
      \endlinechar=\the\endlinechar\relax
      \catcode13=\the\catcode13\relax
      \catcode32=\the\catcode32\relax
      \catcode35=\the\catcode35\relax
      \catcode61=\the\catcode61\relax
      \catcode64=\the\catcode64\relax
      \catcode123=\the\catcode123\relax
      \catcode125=\the\catcode125\relax
    }%
  }%
\x\catcode61\catcode48\catcode32=10\relax%
\catcode13=5 % ^^M
\endlinechar=13 %
\catcode35=6 % #
\catcode64=11 % @
\catcode123=1 % {
\catcode125=2 % }
\def\TMP@EnsureCode#1#2{%
  \edef\ResizeGather@AtEnd{%
    \ResizeGather@AtEnd
    \catcode#1=\the\catcode#1\relax
  }%
  \catcode#1=#2\relax
}
\TMP@EnsureCode{10}{12}% ^^J
\TMP@EnsureCode{33}{12}% !
\TMP@EnsureCode{36}{3}% $
\TMP@EnsureCode{38}{4}% &
\TMP@EnsureCode{39}{12}% '
\TMP@EnsureCode{40}{12}% (
\TMP@EnsureCode{41}{12}% )
\TMP@EnsureCode{42}{12}% *
\TMP@EnsureCode{43}{12}% +
\TMP@EnsureCode{44}{12}% ,
\TMP@EnsureCode{45}{12}% -
\TMP@EnsureCode{46}{12}% .
\TMP@EnsureCode{47}{12}% /
\TMP@EnsureCode{58}{12}% :
\TMP@EnsureCode{59}{12}% ;
\TMP@EnsureCode{60}{12}% <
\TMP@EnsureCode{62}{12}% >
\TMP@EnsureCode{63}{12}% ?
\TMP@EnsureCode{91}{12}% [
\TMP@EnsureCode{93}{12}% ]
\TMP@EnsureCode{94}{7}% ^ (superscript)
\TMP@EnsureCode{96}{12}% `
\edef\ResizeGather@AtEnd{\ResizeGather@AtEnd\noexpand\endinput}
%    \end{macrocode}
%
%    \begin{macrocode}
\RequirePackage{kvoptions}[2009/12/04]
\SetupKeyvalOptions{%
  family=resizegather,%
  prefix=ResizeGather@,%
}
%    \end{macrocode}
%    \begin{macrocode}
\@for\ResizeGather@option:=%
  centertags,%
  tbtags,%
  sumlimits,%
  nosumlimits,%
  intlimits,%
  nointlimits,%
  nonamelimits,%
  leqno,%
  reqno,%
  fleqn%
\do{%
  \edef\ResizeGather@temp{%
    \noexpand\DeclareVoidOption{\ResizeGather@option}{%
      \noexpand\PassOptionsToPackage{amsmath}{\ResizeGather@option}%
    }%
    \noexpand\AtEndOfPackage{%
      \noexpand\DisableKeyvalOption[%
        action=error,%
        package=resizegather,%
      ]{resizegather}{\ResizeGather@option}%
    }%
  }%
  \ResizeGather@temp
}
\@for\ResizeGather@option:=%
  draft,%
  final,%
  hiderotate,%
  hidescale,%
  hiresbb,%
  demo,%
  dvips,xdvi,dvipdf,dvipdfm,dvipdfmx,pdftex,dvipsone,%
  dviwindo,emtex,dviwin,pctexps,pctexwin,pctexhp,pctex32,%
  truetex,tcidvi,vtex,oztex,textures,xetex%
\do{%
  \edef\ResizeGather@temp{%
    \noexpand\DeclareVoidOption{\ResizeGather@option}{%
      \noexpand\PassOptionsToPackage{graphics}{\ResizeGather@option}%
    }%
    \noexpand\AtEndOfPackage{%
      \noexpand\DisableKeyvalOption[%
        action=error,%
        package=resizegather,%
      ]{resizegather}{\ResizeGather@option}%
    }%
  }%
  \ResizeGather@temp
}
%    \end{macrocode}
%
%    \begin{macrocode}
\DeclareBoolOption[true]{enable}
\DeclareComplementaryOption{disable}{enable}
\DeclareStringOption[.05]{warningthreshold}
\newif\ifResizeGather@NeedInit
\DeclareBoolOption[true]{equations}
\DeclareBoolOption[true]{equation}
\DeclareBoolOption[true]{displaymath}
\AddToKeyvalOption*{equations}{%
  \ResizeGather@NeedInittrue
  \ifResizeGather@equations
    \ResizeGather@equationtrue
    \ResizeGather@displaymathtrue
  \else
    \ResizeGather@equationfalse
    \ResizeGather@displaymathfalse
  \fi
}
\AddToKeyvalOption*{equation}{%
  \ResizeGather@NeedInittrue
}
\AddToKeyvalOption*{displaymath}{%
  \ResizeGather@NeedInittrue
}
%    \end{macrocode}
%
%    \begin{macro}{\resizegathersetup}
%    \begin{macrocode}
\newcommand*{\resizegathersetup}[1]{%
  \ResizeGather@NeedInitfalse
  \setkeys{resizegather}{#1}%
  \ifResizeGather@NeedInit
    \ResizeGather@init
  \fi
}
\let\ResizeGather@init\relax
%    \end{macrocode}
%    \end{macro}
%    \begin{macrocode}
\InputIfFileExists{resizegather.cfg}{}{}%
\ProcessKeyvalOptions*\relax
%    \end{macrocode}
%    \begin{macrocode}
\RequirePackage{amsmath}
\RequirePackage{graphics}
%    \end{macrocode}
%    \begin{macro}{\ResizeGather@redefine}
%    \begin{macrocode}
\def\ResizeGather@redefine#1#2#3#4#5{%
  \csname ifResizeGather@#1\endcsname
    \@ifundefined{ResizeGather@org@#2}{%
      \expandafter\let\csname ResizeGather@org@#2\expandafter\endcsname
                      \csname #2\endcsname
    }{}%
    \@ifundefined{ResizeGather@org@#3}{%
      \expandafter\let\csname ResizeGather@org@#3\expandafter\endcsname
                      \csname #3\endcsname
    }{}%
    \expandafter\edef\csname #2\endcsname{%
      \expandafter\noexpand\csname#4\endcsname
    }%
    \expandafter\edef\csname #3\endcsname{%
      \expandafter\noexpand\csname#5\endcsname
    }%
  \else
    \@ifundefined{ResizeGather@org@#2}{}{%
      \expandafter\let\csname #2\expandafter\endcsname
                      \csname ResizeGather@org@#2\endcsname
      \expandafter\let\csname #3\expandafter\endcsname
                      \csname ResizeGather@org@#3\endcsname
    }%
  \fi
}
%    \end{macrocode}
%    \end{macro}
%    \begin{macro}{\ResizeGather@init}
%    \begin{macrocode}
\def\ResizeGather@init{%
  \ResizeGather@redefine{equation}{equation}{endequation}%
    {gather}{endgather}%
  \ResizeGather@redefine{displaymath}{displaymath}{enddisplaymath}%
    {gather*}{endgather*}%
}
\ResizeGather@init
%    \end{macrocode}
%    \end{macro}
%
%    \begin{macro}{\ResizeGather@ResizeGather}
%    \begin{macrocode}
\def\ResizeGather@ResizeGather{%
  \ifResizeGather@enable
    \dimen@\displaywidth
    \if@fleqn
      \advance\dimen@-\@mathmargin
    \fi
    \ifdim\wdz@>\dimen@
      \begingroup
        \advance\dimen@ -\wdz@
        \dimen@ -\dimen@
        \ifdim\ResizeGather@warningthreshold\wdz@>\dimen@
          \expandafter\PackageInfo
        \else
          \expandafter\PackageWarning
        \fi
        {resizegather}{%
          Equation line \the\row@\space is too large %
          by \the\dimen@\MessageBreak
          in environment `\@currenvir'%
        }%
      \endgroup
      \setboxz@h to\dimen@{%
        \resizebox{\dimen@}{!}{\boxz@}%
        \hss
      }%
    \fi
  \fi
}
%    \end{macrocode}
%    \end{macro}
%    \begin{macro}{\calc@shift@gather}
%    \begin{macrocode}
\expandafter\def\expandafter\calc@shift@gather\expandafter{%
  \expandafter\ResizeGather@ResizeGather
  \calc@shift@gather
}
%    \end{macrocode}
%    \end{macro}
%    \begin{macro}{\ResizeGather@org@gmeasure@}
%    \begin{macrocode}
\let\ResizeGather@org@gmeasure@\gmeasure@
%    \end{macrocode}
%    \end{macro}
%    \begin{macro}{\gmeasure@}
%    \begin{macrocode}
\def\gmeasure@#1{%
  \ResizeGather@org@gmeasure@{#1}%
  \ifResizeGather@enable
    \ifdim\totwidth@>\displaywidth
      \totwidth@=\displaywidth
    \fi
  \fi
}
%    \end{macrocode}
%    \end{macro}
%
%    \begin{macrocode}
\ResizeGather@AtEnd%
%</package>
%    \end{macrocode}
%
% \section{Test}
%
% \subsection{Catcode checks for loading}
%
%    \begin{macrocode}
%<*test1>
%    \end{macrocode}
%    \begin{macrocode}
\catcode`\{=1 %
\catcode`\}=2 %
\catcode`\#=6 %
\catcode`\@=11 %
\expandafter\ifx\csname count@\endcsname\relax
  \countdef\count@=255 %
\fi
\expandafter\ifx\csname @gobble\endcsname\relax
  \long\def\@gobble#1{}%
\fi
\expandafter\ifx\csname @firstofone\endcsname\relax
  \long\def\@firstofone#1{#1}%
\fi
\expandafter\ifx\csname loop\endcsname\relax
  \expandafter\@firstofone
\else
  \expandafter\@gobble
\fi
{%
  \def\loop#1\repeat{%
    \def\body{#1}%
    \iterate
  }%
  \def\iterate{%
    \body
      \let\next\iterate
    \else
      \let\next\relax
    \fi
    \next
  }%
  \let\repeat=\fi
}%
\def\RestoreCatcodes{}
\count@=0 %
\loop
  \edef\RestoreCatcodes{%
    \RestoreCatcodes
    \catcode\the\count@=\the\catcode\count@\relax
  }%
\ifnum\count@<255 %
  \advance\count@ 1 %
\repeat

\def\RangeCatcodeInvalid#1#2{%
  \count@=#1\relax
  \loop
    \catcode\count@=15 %
  \ifnum\count@<#2\relax
    \advance\count@ 1 %
  \repeat
}
\def\RangeCatcodeCheck#1#2#3{%
  \count@=#1\relax
  \loop
    \ifnum#3=\catcode\count@
    \else
      \errmessage{%
        Character \the\count@\space
        with wrong catcode \the\catcode\count@\space
        instead of \number#3%
      }%
    \fi
  \ifnum\count@<#2\relax
    \advance\count@ 1 %
  \repeat
}
\def\space{ }
\expandafter\ifx\csname LoadCommand\endcsname\relax
  \def\LoadCommand{\input resizegather.sty\relax}%
\fi
\def\Test{%
  \RangeCatcodeInvalid{0}{47}%
  \RangeCatcodeInvalid{58}{64}%
  \RangeCatcodeInvalid{91}{96}%
  \RangeCatcodeInvalid{123}{255}%
  \catcode`\@=12 %
  \catcode`\\=0 %
  \catcode`\%=14 %
  \LoadCommand
  \RangeCatcodeCheck{0}{36}{15}%
  \RangeCatcodeCheck{37}{37}{14}%
  \RangeCatcodeCheck{38}{47}{15}%
  \RangeCatcodeCheck{48}{57}{12}%
  \RangeCatcodeCheck{58}{63}{15}%
  \RangeCatcodeCheck{64}{64}{12}%
  \RangeCatcodeCheck{65}{90}{11}%
  \RangeCatcodeCheck{91}{91}{15}%
  \RangeCatcodeCheck{92}{92}{0}%
  \RangeCatcodeCheck{93}{96}{15}%
  \RangeCatcodeCheck{97}{122}{11}%
  \RangeCatcodeCheck{123}{255}{15}%
  \RestoreCatcodes
}
\Test
\csname @@end\endcsname
\end
%    \end{macrocode}
%    \begin{macrocode}
%</test1>
%    \end{macrocode}
%
% \section{Installation}
%
% \subsection{Download}
%
% \paragraph{Package.} This package is available on
% CTAN\footnote{\url{https://ctan.org/pkg/resizegather}}:
% \begin{description}
% \item[\CTAN{macros/latex/contrib/oberdiek/resizegather.dtx}] The source file.
% \item[\CTAN{macros/latex/contrib/oberdiek/resizegather.pdf}] Documentation.
% \end{description}
%
%
% \paragraph{Bundle.} All the packages of the bundle `oberdiek'
% are also available in a TDS compliant ZIP archive. There
% the packages are already unpacked and the documentation files
% are generated. The files and directories obey the TDS standard.
% \begin{description}
% \item[\CTANinstall{install/macros/latex/contrib/oberdiek.tds.zip}]
% \end{description}
% \emph{TDS} refers to the standard ``A Directory Structure
% for \TeX\ Files'' (\CTAN{tds/tds.pdf}). Directories
% with \xfile{texmf} in their name are usually organized this way.
%
% \subsection{Bundle installation}
%
% \paragraph{Unpacking.} Unpack the \xfile{oberdiek.tds.zip} in the
% TDS tree (also known as \xfile{texmf} tree) of your choice.
% Example (linux):
% \begin{quote}
%   |unzip oberdiek.tds.zip -d ~/texmf|
% \end{quote}
%
% \paragraph{Script installation.}
% Check the directory \xfile{TDS:scripts/oberdiek/} for
% scripts that need further installation steps.
% Package \xpackage{attachfile2} comes with the Perl script
% \xfile{pdfatfi.pl} that should be installed in such a way
% that it can be called as \texttt{pdfatfi}.
% Example (linux):
% \begin{quote}
%   |chmod +x scripts/oberdiek/pdfatfi.pl|\\
%   |cp scripts/oberdiek/pdfatfi.pl /usr/local/bin/|
% \end{quote}
%
% \subsection{Package installation}
%
% \paragraph{Unpacking.} The \xfile{.dtx} file is a self-extracting
% \docstrip\ archive. The files are extracted by running the
% \xfile{.dtx} through \plainTeX:
% \begin{quote}
%   \verb|tex resizegather.dtx|
% \end{quote}
%
% \paragraph{TDS.} Now the different files must be moved into
% the different directories in your installation TDS tree
% (also known as \xfile{texmf} tree):
% \begin{quote}
% \def\t{^^A
% \begin{tabular}{@{}>{\ttfamily}l@{ $\rightarrow$ }>{\ttfamily}l@{}}
%   resizegather.sty & tex/latex/oberdiek/resizegather.sty\\
%   resizegather.pdf & doc/latex/oberdiek/resizegather.pdf\\
%   test/resizegather-test1.tex & doc/latex/oberdiek/test/resizegather-test1.tex\\
%   resizegather.dtx & source/latex/oberdiek/resizegather.dtx\\
% \end{tabular}^^A
% }^^A
% \sbox0{\t}^^A
% \ifdim\wd0>\linewidth
%   \begingroup
%     \advance\linewidth by\leftmargin
%     \advance\linewidth by\rightmargin
%   \edef\x{\endgroup
%     \def\noexpand\lw{\the\linewidth}^^A
%   }\x
%   \def\lwbox{^^A
%     \leavevmode
%     \hbox to \linewidth{^^A
%       \kern-\leftmargin\relax
%       \hss
%       \usebox0
%       \hss
%       \kern-\rightmargin\relax
%     }^^A
%   }^^A
%   \ifdim\wd0>\lw
%     \sbox0{\small\t}^^A
%     \ifdim\wd0>\linewidth
%       \ifdim\wd0>\lw
%         \sbox0{\footnotesize\t}^^A
%         \ifdim\wd0>\linewidth
%           \ifdim\wd0>\lw
%             \sbox0{\scriptsize\t}^^A
%             \ifdim\wd0>\linewidth
%               \ifdim\wd0>\lw
%                 \sbox0{\tiny\t}^^A
%                 \ifdim\wd0>\linewidth
%                   \lwbox
%                 \else
%                   \usebox0
%                 \fi
%               \else
%                 \lwbox
%               \fi
%             \else
%               \usebox0
%             \fi
%           \else
%             \lwbox
%           \fi
%         \else
%           \usebox0
%         \fi
%       \else
%         \lwbox
%       \fi
%     \else
%       \usebox0
%     \fi
%   \else
%     \lwbox
%   \fi
% \else
%   \usebox0
% \fi
% \end{quote}
% If you have a \xfile{docstrip.cfg} that configures and enables \docstrip's
% TDS installing feature, then some files can already be in the right
% place, see the documentation of \docstrip.
%
% \subsection{Refresh file name databases}
%
% If your \TeX~distribution
% (\teTeX, \mikTeX, \dots) relies on file name databases, you must refresh
% these. For example, \teTeX\ users run \verb|texhash| or
% \verb|mktexlsr|.
%
% \subsection{Some details for the interested}
%
% \paragraph{Attached source.}
%
% The PDF documentation on CTAN also includes the
% \xfile{.dtx} source file. It can be extracted by
% AcrobatReader 6 or higher. Another option is \textsf{pdftk},
% e.g. unpack the file into the current directory:
% \begin{quote}
%   \verb|pdftk resizegather.pdf unpack_files output .|
% \end{quote}
%
% \paragraph{Unpacking with \LaTeX.}
% The \xfile{.dtx} chooses its action depending on the format:
% \begin{description}
% \item[\plainTeX:] Run \docstrip\ and extract the files.
% \item[\LaTeX:] Generate the documentation.
% \end{description}
% If you insist on using \LaTeX\ for \docstrip\ (really,
% \docstrip\ does not need \LaTeX), then inform the autodetect routine
% about your intention:
% \begin{quote}
%   \verb|latex \let\install=y% \iffalse meta-comment
%
% File: resizegather.dtx
% Version: 2016/05/16 v1.3
% Info: Resize overly large equations
%
% Copyright (C) 2009, 2010 by
%    Heiko Oberdiek <heiko.oberdiek at googlemail.com>
%    2016
%    https://github.com/ho-tex/oberdiek/issues
%
% This work may be distributed and/or modified under the
% conditions of the LaTeX Project Public License, either
% version 1.3c of this license or (at your option) any later
% version. This version of this license is in
%    http://www.latex-project.org/lppl/lppl-1-3c.txt
% and the latest version of this license is in
%    http://www.latex-project.org/lppl.txt
% and version 1.3 or later is part of all distributions of
% LaTeX version 2005/12/01 or later.
%
% This work has the LPPL maintenance status "maintained".
%
% This Current Maintainer of this work is Heiko Oberdiek.
%
% This work consists of the main source file resizegather.dtx
% and the derived files
%    resizegather.sty, resizegather.pdf, resizegather.ins, resizegather.drv,
%    resizegather-test1.tex.
%
% Distribution:
%    CTAN:macros/latex/contrib/oberdiek/resizegather.dtx
%    CTAN:macros/latex/contrib/oberdiek/resizegather.pdf
%
% Unpacking:
%    (a) If resizegather.ins is present:
%           tex resizegather.ins
%    (b) Without resizegather.ins:
%           tex resizegather.dtx
%    (c) If you insist on using LaTeX
%           latex \let\install=y% \iffalse meta-comment
%
% File: resizegather.dtx
% Version: 2016/05/16 v1.3
% Info: Resize overly large equations
%
% Copyright (C) 2009, 2010 by
%    Heiko Oberdiek <heiko.oberdiek at googlemail.com>
%    2016
%    https://github.com/ho-tex/oberdiek/issues
%
% This work may be distributed and/or modified under the
% conditions of the LaTeX Project Public License, either
% version 1.3c of this license or (at your option) any later
% version. This version of this license is in
%    http://www.latex-project.org/lppl/lppl-1-3c.txt
% and the latest version of this license is in
%    http://www.latex-project.org/lppl.txt
% and version 1.3 or later is part of all distributions of
% LaTeX version 2005/12/01 or later.
%
% This work has the LPPL maintenance status "maintained".
%
% This Current Maintainer of this work is Heiko Oberdiek.
%
% This work consists of the main source file resizegather.dtx
% and the derived files
%    resizegather.sty, resizegather.pdf, resizegather.ins, resizegather.drv,
%    resizegather-test1.tex.
%
% Distribution:
%    CTAN:macros/latex/contrib/oberdiek/resizegather.dtx
%    CTAN:macros/latex/contrib/oberdiek/resizegather.pdf
%
% Unpacking:
%    (a) If resizegather.ins is present:
%           tex resizegather.ins
%    (b) Without resizegather.ins:
%           tex resizegather.dtx
%    (c) If you insist on using LaTeX
%           latex \let\install=y\input{resizegather.dtx}
%        (quote the arguments according to the demands of your shell)
%
% Documentation:
%    (a) If resizegather.drv is present:
%           latex resizegather.drv
%    (b) Without resizegather.drv:
%           latex resizegather.dtx; ...
%    The class ltxdoc loads the configuration file ltxdoc.cfg
%    if available. Here you can specify further options, e.g.
%    use A4 as paper format:
%       \PassOptionsToClass{a4paper}{article}
%
%    Programm calls to get the documentation (example):
%       pdflatex resizegather.dtx
%       makeindex -s gind.ist resizegather.idx
%       pdflatex resizegather.dtx
%       makeindex -s gind.ist resizegather.idx
%       pdflatex resizegather.dtx
%
% Installation:
%    TDS:tex/latex/oberdiek/resizegather.sty
%    TDS:doc/latex/oberdiek/resizegather.pdf
%    TDS:doc/latex/oberdiek/test/resizegather-test1.tex
%    TDS:source/latex/oberdiek/resizegather.dtx
%
%<*ignore>
\begingroup
  \catcode123=1 %
  \catcode125=2 %
  \def\x{LaTeX2e}%
\expandafter\endgroup
\ifcase 0\ifx\install y1\fi\expandafter
         \ifx\csname processbatchFile\endcsname\relax\else1\fi
         \ifx\fmtname\x\else 1\fi\relax
\else\csname fi\endcsname
%</ignore>
%<*install>
\input docstrip.tex
\Msg{************************************************************************}
\Msg{* Installation}
\Msg{* Package: resizegather 2016/05/16 v1.3 Resize overly large equations (HO)}
\Msg{************************************************************************}

\keepsilent
\askforoverwritefalse

\let\MetaPrefix\relax
\preamble

This is a generated file.

Project: resizegather
Version: 2016/05/16 v1.3

Copyright (C) 2009, 2010 by
   Heiko Oberdiek <heiko.oberdiek at googlemail.com>

This work may be distributed and/or modified under the
conditions of the LaTeX Project Public License, either
version 1.3c of this license or (at your option) any later
version. This version of this license is in
   http://www.latex-project.org/lppl/lppl-1-3c.txt
and the latest version of this license is in
   http://www.latex-project.org/lppl.txt
and version 1.3 or later is part of all distributions of
LaTeX version 2005/12/01 or later.

This work has the LPPL maintenance status "maintained".

This Current Maintainer of this work is Heiko Oberdiek.

This work consists of the main source file resizegather.dtx
and the derived files
   resizegather.sty, resizegather.pdf, resizegather.ins, resizegather.drv,
   resizegather-test1.tex.

\endpreamble
\let\MetaPrefix\DoubleperCent

\generate{%
  \file{resizegather.ins}{\from{resizegather.dtx}{install}}%
  \file{resizegather.drv}{\from{resizegather.dtx}{driver}}%
  \usedir{tex/latex/oberdiek}%
  \file{resizegather.sty}{\from{resizegather.dtx}{package}}%
%  \usedir{doc/latex/oberdiek/test}%
%  \file{resizegather-test1.tex}{\from{resizegather.dtx}{test1}}%
  \nopreamble
  \nopostamble
%  \usedir{source/latex/oberdiek/catalogue}%
%  \file{resizegather.xml}{\from{resizegather.dtx}{catalogue}}%
}

\catcode32=13\relax% active space
\let =\space%
\Msg{************************************************************************}
\Msg{*}
\Msg{* To finish the installation you have to move the following}
\Msg{* file into a directory searched by TeX:}
\Msg{*}
\Msg{*     resizegather.sty}
\Msg{*}
\Msg{* To produce the documentation run the file `resizegather.drv'}
\Msg{* through LaTeX.}
\Msg{*}
\Msg{* Happy TeXing!}
\Msg{*}
\Msg{************************************************************************}

\endbatchfile
%</install>
%<*ignore>
\fi
%</ignore>
%<*driver>
\NeedsTeXFormat{LaTeX2e}
\ProvidesFile{resizegather.drv}%
  [2016/05/16 v1.3 Resize overly large equations (HO)]%
\documentclass{ltxdoc}
\usepackage{holtxdoc}[2011/11/22]
\usepackage{ifluatex}
\ifluatex
\else
  \usepackage[T1]{fontenc}%
  \usepackage{textcomp}%
  \usepackage{lmodern}%
\fi
\begin{document}
  \DocInput{resizegather.dtx}%
\end{document}
%</driver>
% \fi
%
%
% \CharacterTable
%  {Upper-case    \A\B\C\D\E\F\G\H\I\J\K\L\M\N\O\P\Q\R\S\T\U\V\W\X\Y\Z
%   Lower-case    \a\b\c\d\e\f\g\h\i\j\k\l\m\n\o\p\q\r\s\t\u\v\w\x\y\z
%   Digits        \0\1\2\3\4\5\6\7\8\9
%   Exclamation   \!     Double quote  \"     Hash (number) \#
%   Dollar        \$     Percent       \%     Ampersand     \&
%   Acute accent  \'     Left paren    \(     Right paren   \)
%   Asterisk      \*     Plus          \+     Comma         \,
%   Minus         \-     Point         \.     Solidus       \/
%   Colon         \:     Semicolon     \;     Less than     \<
%   Equals        \=     Greater than  \>     Question mark \?
%   Commercial at \@     Left bracket  \[     Backslash     \\
%   Right bracket \]     Circumflex    \^     Underscore    \_
%   Grave accent  \`     Left brace    \{     Vertical bar  \|
%   Right brace   \}     Tilde         \~}
%
% \GetFileInfo{resizegather.drv}
%
% \title{The \xpackage{resizegather} package}
% \date{2016/05/16 v1.3}
% \author{Heiko Oberdiek\thanks
% {Please report any issues at https://github.com/ho-tex/oberdiek/issues}\\
% \xemail{heiko.oberdiek at googlemail.com}}
%
% \maketitle
%
% \begin{abstract}
% Equations that are too large are resized to fit the available
% space. The environment \textsf{gather} of package \xpackage{amsmath}
% is supported. Also the environments \textsf{equation} and
% \textsf{displaymath} are redefined using \textsf{gather}
% and its starred version.
% \end{abstract}
%
% \tableofcontents
%
% \makeatletter
% \def\env#1{^^A
%    \textsf{\@env#1*\@nil}^^A
% }%
% \def\@env#1*#2\@nil{^^A
%   #1^^A
%   \ifx\\#2\\^^A
%     \expandafter\@gobble
%   \else
%     \textasteriskcentered
%     \expandafter\@firstofone
%   \fi
%   {\@env#2\@nil}^^A
% }
% \makeatother
%
% \section{Documentation}
%
% Sometimes an equation is just a little to large to fit in the
% line. And breaking the equation across lines might be worse
% than downscaling the equation. This package implements this
% for the environments \env{gather} and \env{gather*} of
% package \xpackage{amsmath}. That package already measures
% the equations and simplifies the implementation of \xpackage{resizegather}
% that only needs to hook into \xpackage{amsmath}'s code to add
% the resizing feature.
%
% Resized equations are recorded in the \xext{log} file
% for small exceeds (default setting is smaller than five percent).
% Otherwise a warning is given.
%
% Also environments \env{equation} and \env{displaymath}
% are supported by redefining them using \env{gather}
% and \env{gather*}.
%
% \cs{[} and \cs{]} are not supported, because these macros
% are not in environment form that is required for
% \xpackage{amsmath}. The environment body is collected
% first to be able to process the body twice for measuring
% first.
%
% Also the environments using alignments are not supported.
% If a single equation line would be resized, the alignment
% would get lost. And resizing all equations of the alignment
% does not seem appropriate either.
%
% \subsection{Options}
%
% \begin{description}
% \item[\xoption{warningthreshold}:]
%   Print a warning if the original equation line exceeds
%   its available width by the given fraction.
%   Default is |0.05|: A warning is given if the equation
%   is too large by five percent.
%   Otherwise the exceed is recorded in the \xext{log} file
%   only.
% \end{description}
% The next options are boolean options. They are enabled
% by value |true| or if no value is given. They are switched
% off by value |false|.
% \begin{description}
% \item[\xoption{enable}:] The resize feature is active (default).
% \item[\xoption{disable}:] The complementary option for \xoption{enable},
%  added for convenience: |disable| (or |disable=true|) is the same
%  as |enable=false|.
% \item[\xoption{equations}:]
%   \LaTeX\ environments \textsf{equation} and \textsf{displaymath}
%   environments are redefined. These equations
%   are now using environment \env{gather} and
%   \env{gather*}. This is the default.
% \end{description}
% The following table shows additional options if you
% want to have finer control for the redefined
% environments:
% \begin{quote}
% \def\unchanged{\textit{unchanged}}
% \def\notprovided{\textit{not provided}}
% \begin{tabular}{l|ll}
% &\multicolumn{2}{c}{Environments}\\
% Option & \env{equation} & \env{displaymath}\\
% \hline
% \xoption{equations} & \env{gather} & \env{gather*}\\
% \xoption{equation} & \env{gather} & \unchanged\\
% \xoption{displaymath} & \unchanged & \env{gather*}\\
% \end{tabular}
% \end{quote}
% If such an option is switched off, the original meaning
% of the affected environments is restored.
%
% Options are evaluated in the following order:
% \begin{enumerate}
% \item
%  Configuration file \xfile{resizegather.cfg} using \cs{resizegathersetup}
%  if the file exists.
%  \item
%  Package options given for \cs{usepackage}.
%  \item
%  Later calls of \cs{resizegathersetup}.
% \end{enumerate}
% \begin{declcs}{resizegathersetup}\M{option list}
% \end{declcs}
% The options are key value options. Boolean options are enabled by
% default (without value) or by using the explicit value \texttt{true}.
% Value \texttt{false} disable the option.
%
% \subsection{Options for packages \xpackage{amsmath} or \xpackage{graphics}}
%
% The package loads the package \xpackage{amsmath} because is internally
% measures the equations first. Thus this package hooks into this code
% in order to resize the equations if they are too large.
% The resizing itself is done by \cs{resizebox} of package \xpackage{graphics}.
% If you need special options for these packages, just load them first or
% use global options when appropriate. Example:
% \begin{quote}
%\begin{verbatim}
%\usepackage[dvipdfm]{graphicx}% or graphics
%\usepackage[fleqn]{amsmath}
%\usepackage{resizegather}
%\end{verbatim}
%\end{quote}
%
% \StopEventually{
% }
%
% \section{Implementation}
%    \begin{macrocode}
%<*package>
%    \end{macrocode}
%    Reload check, especially if the package is not used with \LaTeX.
%    \begin{macrocode}
\begingroup\catcode61\catcode48\catcode32=10\relax%
  \catcode13=5 % ^^M
  \endlinechar=13 %
  \catcode35=6 % #
  \catcode39=12 % '
  \catcode44=12 % ,
  \catcode45=12 % -
  \catcode46=12 % .
  \catcode58=12 % :
  \catcode64=11 % @
  \catcode123=1 % {
  \catcode125=2 % }
  \expandafter\let\expandafter\x\csname ver@resizegather.sty\endcsname
  \ifx\x\relax % plain-TeX, first loading
  \else
    \def\empty{}%
    \ifx\x\empty % LaTeX, first loading,
      % variable is initialized, but \ProvidesPackage not yet seen
    \else
      \expandafter\ifx\csname PackageInfo\endcsname\relax
        \def\x#1#2{%
          \immediate\write-1{Package #1 Info: #2.}%
        }%
      \else
        \def\x#1#2{\PackageInfo{#1}{#2, stopped}}%
      \fi
      \x{resizegather}{The package is already loaded}%
      \aftergroup\endinput
    \fi
  \fi
\endgroup%
%    \end{macrocode}
%    Package identification:
%    \begin{macrocode}
\begingroup\catcode61\catcode48\catcode32=10\relax%
  \catcode13=5 % ^^M
  \endlinechar=13 %
  \catcode35=6 % #
  \catcode39=12 % '
  \catcode40=12 % (
  \catcode41=12 % )
  \catcode44=12 % ,
  \catcode45=12 % -
  \catcode46=12 % .
  \catcode47=12 % /
  \catcode58=12 % :
  \catcode64=11 % @
  \catcode91=12 % [
  \catcode93=12 % ]
  \catcode123=1 % {
  \catcode125=2 % }
  \expandafter\ifx\csname ProvidesPackage\endcsname\relax
    \def\x#1#2#3[#4]{\endgroup
      \immediate\write-1{Package: #3 #4}%
      \xdef#1{#4}%
    }%
  \else
    \def\x#1#2[#3]{\endgroup
      #2[{#3}]%
      \ifx#1\@undefined
        \xdef#1{#3}%
      \fi
      \ifx#1\relax
        \xdef#1{#3}%
      \fi
    }%
  \fi
\expandafter\x\csname ver@resizegather.sty\endcsname
\ProvidesPackage{resizegather}%
  [2016/05/16 v1.3 Resize overly large equations (HO)]%
%    \end{macrocode}
%
%    \begin{macrocode}
\begingroup\catcode61\catcode48\catcode32=10\relax%
  \catcode13=5 % ^^M
  \endlinechar=13 %
  \catcode123=1 % {
  \catcode125=2 % }
  \catcode64=11 % @
  \def\x{\endgroup
    \expandafter\edef\csname ResizeGather@AtEnd\endcsname{%
      \endlinechar=\the\endlinechar\relax
      \catcode13=\the\catcode13\relax
      \catcode32=\the\catcode32\relax
      \catcode35=\the\catcode35\relax
      \catcode61=\the\catcode61\relax
      \catcode64=\the\catcode64\relax
      \catcode123=\the\catcode123\relax
      \catcode125=\the\catcode125\relax
    }%
  }%
\x\catcode61\catcode48\catcode32=10\relax%
\catcode13=5 % ^^M
\endlinechar=13 %
\catcode35=6 % #
\catcode64=11 % @
\catcode123=1 % {
\catcode125=2 % }
\def\TMP@EnsureCode#1#2{%
  \edef\ResizeGather@AtEnd{%
    \ResizeGather@AtEnd
    \catcode#1=\the\catcode#1\relax
  }%
  \catcode#1=#2\relax
}
\TMP@EnsureCode{10}{12}% ^^J
\TMP@EnsureCode{33}{12}% !
\TMP@EnsureCode{36}{3}% $
\TMP@EnsureCode{38}{4}% &
\TMP@EnsureCode{39}{12}% '
\TMP@EnsureCode{40}{12}% (
\TMP@EnsureCode{41}{12}% )
\TMP@EnsureCode{42}{12}% *
\TMP@EnsureCode{43}{12}% +
\TMP@EnsureCode{44}{12}% ,
\TMP@EnsureCode{45}{12}% -
\TMP@EnsureCode{46}{12}% .
\TMP@EnsureCode{47}{12}% /
\TMP@EnsureCode{58}{12}% :
\TMP@EnsureCode{59}{12}% ;
\TMP@EnsureCode{60}{12}% <
\TMP@EnsureCode{62}{12}% >
\TMP@EnsureCode{63}{12}% ?
\TMP@EnsureCode{91}{12}% [
\TMP@EnsureCode{93}{12}% ]
\TMP@EnsureCode{94}{7}% ^ (superscript)
\TMP@EnsureCode{96}{12}% `
\edef\ResizeGather@AtEnd{\ResizeGather@AtEnd\noexpand\endinput}
%    \end{macrocode}
%
%    \begin{macrocode}
\RequirePackage{kvoptions}[2009/12/04]
\SetupKeyvalOptions{%
  family=resizegather,%
  prefix=ResizeGather@,%
}
%    \end{macrocode}
%    \begin{macrocode}
\@for\ResizeGather@option:=%
  centertags,%
  tbtags,%
  sumlimits,%
  nosumlimits,%
  intlimits,%
  nointlimits,%
  nonamelimits,%
  leqno,%
  reqno,%
  fleqn%
\do{%
  \edef\ResizeGather@temp{%
    \noexpand\DeclareVoidOption{\ResizeGather@option}{%
      \noexpand\PassOptionsToPackage{amsmath}{\ResizeGather@option}%
    }%
    \noexpand\AtEndOfPackage{%
      \noexpand\DisableKeyvalOption[%
        action=error,%
        package=resizegather,%
      ]{resizegather}{\ResizeGather@option}%
    }%
  }%
  \ResizeGather@temp
}
\@for\ResizeGather@option:=%
  draft,%
  final,%
  hiderotate,%
  hidescale,%
  hiresbb,%
  demo,%
  dvips,xdvi,dvipdf,dvipdfm,dvipdfmx,pdftex,dvipsone,%
  dviwindo,emtex,dviwin,pctexps,pctexwin,pctexhp,pctex32,%
  truetex,tcidvi,vtex,oztex,textures,xetex%
\do{%
  \edef\ResizeGather@temp{%
    \noexpand\DeclareVoidOption{\ResizeGather@option}{%
      \noexpand\PassOptionsToPackage{graphics}{\ResizeGather@option}%
    }%
    \noexpand\AtEndOfPackage{%
      \noexpand\DisableKeyvalOption[%
        action=error,%
        package=resizegather,%
      ]{resizegather}{\ResizeGather@option}%
    }%
  }%
  \ResizeGather@temp
}
%    \end{macrocode}
%
%    \begin{macrocode}
\DeclareBoolOption[true]{enable}
\DeclareComplementaryOption{disable}{enable}
\DeclareStringOption[.05]{warningthreshold}
\newif\ifResizeGather@NeedInit
\DeclareBoolOption[true]{equations}
\DeclareBoolOption[true]{equation}
\DeclareBoolOption[true]{displaymath}
\AddToKeyvalOption*{equations}{%
  \ResizeGather@NeedInittrue
  \ifResizeGather@equations
    \ResizeGather@equationtrue
    \ResizeGather@displaymathtrue
  \else
    \ResizeGather@equationfalse
    \ResizeGather@displaymathfalse
  \fi
}
\AddToKeyvalOption*{equation}{%
  \ResizeGather@NeedInittrue
}
\AddToKeyvalOption*{displaymath}{%
  \ResizeGather@NeedInittrue
}
%    \end{macrocode}
%
%    \begin{macro}{\resizegathersetup}
%    \begin{macrocode}
\newcommand*{\resizegathersetup}[1]{%
  \ResizeGather@NeedInitfalse
  \setkeys{resizegather}{#1}%
  \ifResizeGather@NeedInit
    \ResizeGather@init
  \fi
}
\let\ResizeGather@init\relax
%    \end{macrocode}
%    \end{macro}
%    \begin{macrocode}
\InputIfFileExists{resizegather.cfg}{}{}%
\ProcessKeyvalOptions*\relax
%    \end{macrocode}
%    \begin{macrocode}
\RequirePackage{amsmath}
\RequirePackage{graphics}
%    \end{macrocode}
%    \begin{macro}{\ResizeGather@redefine}
%    \begin{macrocode}
\def\ResizeGather@redefine#1#2#3#4#5{%
  \csname ifResizeGather@#1\endcsname
    \@ifundefined{ResizeGather@org@#2}{%
      \expandafter\let\csname ResizeGather@org@#2\expandafter\endcsname
                      \csname #2\endcsname
    }{}%
    \@ifundefined{ResizeGather@org@#3}{%
      \expandafter\let\csname ResizeGather@org@#3\expandafter\endcsname
                      \csname #3\endcsname
    }{}%
    \expandafter\edef\csname #2\endcsname{%
      \expandafter\noexpand\csname#4\endcsname
    }%
    \expandafter\edef\csname #3\endcsname{%
      \expandafter\noexpand\csname#5\endcsname
    }%
  \else
    \@ifundefined{ResizeGather@org@#2}{}{%
      \expandafter\let\csname #2\expandafter\endcsname
                      \csname ResizeGather@org@#2\endcsname
      \expandafter\let\csname #3\expandafter\endcsname
                      \csname ResizeGather@org@#3\endcsname
    }%
  \fi
}
%    \end{macrocode}
%    \end{macro}
%    \begin{macro}{\ResizeGather@init}
%    \begin{macrocode}
\def\ResizeGather@init{%
  \ResizeGather@redefine{equation}{equation}{endequation}%
    {gather}{endgather}%
  \ResizeGather@redefine{displaymath}{displaymath}{enddisplaymath}%
    {gather*}{endgather*}%
}
\ResizeGather@init
%    \end{macrocode}
%    \end{macro}
%
%    \begin{macro}{\ResizeGather@ResizeGather}
%    \begin{macrocode}
\def\ResizeGather@ResizeGather{%
  \ifResizeGather@enable
    \dimen@\displaywidth
    \if@fleqn
      \advance\dimen@-\@mathmargin
    \fi
    \ifdim\wdz@>\dimen@
      \begingroup
        \advance\dimen@ -\wdz@
        \dimen@ -\dimen@
        \ifdim\ResizeGather@warningthreshold\wdz@>\dimen@
          \expandafter\PackageInfo
        \else
          \expandafter\PackageWarning
        \fi
        {resizegather}{%
          Equation line \the\row@\space is too large %
          by \the\dimen@\MessageBreak
          in environment `\@currenvir'%
        }%
      \endgroup
      \setboxz@h to\dimen@{%
        \resizebox{\dimen@}{!}{\boxz@}%
        \hss
      }%
    \fi
  \fi
}
%    \end{macrocode}
%    \end{macro}
%    \begin{macro}{\calc@shift@gather}
%    \begin{macrocode}
\expandafter\def\expandafter\calc@shift@gather\expandafter{%
  \expandafter\ResizeGather@ResizeGather
  \calc@shift@gather
}
%    \end{macrocode}
%    \end{macro}
%    \begin{macro}{\ResizeGather@org@gmeasure@}
%    \begin{macrocode}
\let\ResizeGather@org@gmeasure@\gmeasure@
%    \end{macrocode}
%    \end{macro}
%    \begin{macro}{\gmeasure@}
%    \begin{macrocode}
\def\gmeasure@#1{%
  \ResizeGather@org@gmeasure@{#1}%
  \ifResizeGather@enable
    \ifdim\totwidth@>\displaywidth
      \totwidth@=\displaywidth
    \fi
  \fi
}
%    \end{macrocode}
%    \end{macro}
%
%    \begin{macrocode}
\ResizeGather@AtEnd%
%</package>
%    \end{macrocode}
%
% \section{Test}
%
% \subsection{Catcode checks for loading}
%
%    \begin{macrocode}
%<*test1>
%    \end{macrocode}
%    \begin{macrocode}
\catcode`\{=1 %
\catcode`\}=2 %
\catcode`\#=6 %
\catcode`\@=11 %
\expandafter\ifx\csname count@\endcsname\relax
  \countdef\count@=255 %
\fi
\expandafter\ifx\csname @gobble\endcsname\relax
  \long\def\@gobble#1{}%
\fi
\expandafter\ifx\csname @firstofone\endcsname\relax
  \long\def\@firstofone#1{#1}%
\fi
\expandafter\ifx\csname loop\endcsname\relax
  \expandafter\@firstofone
\else
  \expandafter\@gobble
\fi
{%
  \def\loop#1\repeat{%
    \def\body{#1}%
    \iterate
  }%
  \def\iterate{%
    \body
      \let\next\iterate
    \else
      \let\next\relax
    \fi
    \next
  }%
  \let\repeat=\fi
}%
\def\RestoreCatcodes{}
\count@=0 %
\loop
  \edef\RestoreCatcodes{%
    \RestoreCatcodes
    \catcode\the\count@=\the\catcode\count@\relax
  }%
\ifnum\count@<255 %
  \advance\count@ 1 %
\repeat

\def\RangeCatcodeInvalid#1#2{%
  \count@=#1\relax
  \loop
    \catcode\count@=15 %
  \ifnum\count@<#2\relax
    \advance\count@ 1 %
  \repeat
}
\def\RangeCatcodeCheck#1#2#3{%
  \count@=#1\relax
  \loop
    \ifnum#3=\catcode\count@
    \else
      \errmessage{%
        Character \the\count@\space
        with wrong catcode \the\catcode\count@\space
        instead of \number#3%
      }%
    \fi
  \ifnum\count@<#2\relax
    \advance\count@ 1 %
  \repeat
}
\def\space{ }
\expandafter\ifx\csname LoadCommand\endcsname\relax
  \def\LoadCommand{\input resizegather.sty\relax}%
\fi
\def\Test{%
  \RangeCatcodeInvalid{0}{47}%
  \RangeCatcodeInvalid{58}{64}%
  \RangeCatcodeInvalid{91}{96}%
  \RangeCatcodeInvalid{123}{255}%
  \catcode`\@=12 %
  \catcode`\\=0 %
  \catcode`\%=14 %
  \LoadCommand
  \RangeCatcodeCheck{0}{36}{15}%
  \RangeCatcodeCheck{37}{37}{14}%
  \RangeCatcodeCheck{38}{47}{15}%
  \RangeCatcodeCheck{48}{57}{12}%
  \RangeCatcodeCheck{58}{63}{15}%
  \RangeCatcodeCheck{64}{64}{12}%
  \RangeCatcodeCheck{65}{90}{11}%
  \RangeCatcodeCheck{91}{91}{15}%
  \RangeCatcodeCheck{92}{92}{0}%
  \RangeCatcodeCheck{93}{96}{15}%
  \RangeCatcodeCheck{97}{122}{11}%
  \RangeCatcodeCheck{123}{255}{15}%
  \RestoreCatcodes
}
\Test
\csname @@end\endcsname
\end
%    \end{macrocode}
%    \begin{macrocode}
%</test1>
%    \end{macrocode}
%
% \section{Installation}
%
% \subsection{Download}
%
% \paragraph{Package.} This package is available on
% CTAN\footnote{\url{https://ctan.org/pkg/resizegather}}:
% \begin{description}
% \item[\CTAN{macros/latex/contrib/oberdiek/resizegather.dtx}] The source file.
% \item[\CTAN{macros/latex/contrib/oberdiek/resizegather.pdf}] Documentation.
% \end{description}
%
%
% \paragraph{Bundle.} All the packages of the bundle `oberdiek'
% are also available in a TDS compliant ZIP archive. There
% the packages are already unpacked and the documentation files
% are generated. The files and directories obey the TDS standard.
% \begin{description}
% \item[\CTANinstall{install/macros/latex/contrib/oberdiek.tds.zip}]
% \end{description}
% \emph{TDS} refers to the standard ``A Directory Structure
% for \TeX\ Files'' (\CTAN{tds/tds.pdf}). Directories
% with \xfile{texmf} in their name are usually organized this way.
%
% \subsection{Bundle installation}
%
% \paragraph{Unpacking.} Unpack the \xfile{oberdiek.tds.zip} in the
% TDS tree (also known as \xfile{texmf} tree) of your choice.
% Example (linux):
% \begin{quote}
%   |unzip oberdiek.tds.zip -d ~/texmf|
% \end{quote}
%
% \paragraph{Script installation.}
% Check the directory \xfile{TDS:scripts/oberdiek/} for
% scripts that need further installation steps.
% Package \xpackage{attachfile2} comes with the Perl script
% \xfile{pdfatfi.pl} that should be installed in such a way
% that it can be called as \texttt{pdfatfi}.
% Example (linux):
% \begin{quote}
%   |chmod +x scripts/oberdiek/pdfatfi.pl|\\
%   |cp scripts/oberdiek/pdfatfi.pl /usr/local/bin/|
% \end{quote}
%
% \subsection{Package installation}
%
% \paragraph{Unpacking.} The \xfile{.dtx} file is a self-extracting
% \docstrip\ archive. The files are extracted by running the
% \xfile{.dtx} through \plainTeX:
% \begin{quote}
%   \verb|tex resizegather.dtx|
% \end{quote}
%
% \paragraph{TDS.} Now the different files must be moved into
% the different directories in your installation TDS tree
% (also known as \xfile{texmf} tree):
% \begin{quote}
% \def\t{^^A
% \begin{tabular}{@{}>{\ttfamily}l@{ $\rightarrow$ }>{\ttfamily}l@{}}
%   resizegather.sty & tex/latex/oberdiek/resizegather.sty\\
%   resizegather.pdf & doc/latex/oberdiek/resizegather.pdf\\
%   test/resizegather-test1.tex & doc/latex/oberdiek/test/resizegather-test1.tex\\
%   resizegather.dtx & source/latex/oberdiek/resizegather.dtx\\
% \end{tabular}^^A
% }^^A
% \sbox0{\t}^^A
% \ifdim\wd0>\linewidth
%   \begingroup
%     \advance\linewidth by\leftmargin
%     \advance\linewidth by\rightmargin
%   \edef\x{\endgroup
%     \def\noexpand\lw{\the\linewidth}^^A
%   }\x
%   \def\lwbox{^^A
%     \leavevmode
%     \hbox to \linewidth{^^A
%       \kern-\leftmargin\relax
%       \hss
%       \usebox0
%       \hss
%       \kern-\rightmargin\relax
%     }^^A
%   }^^A
%   \ifdim\wd0>\lw
%     \sbox0{\small\t}^^A
%     \ifdim\wd0>\linewidth
%       \ifdim\wd0>\lw
%         \sbox0{\footnotesize\t}^^A
%         \ifdim\wd0>\linewidth
%           \ifdim\wd0>\lw
%             \sbox0{\scriptsize\t}^^A
%             \ifdim\wd0>\linewidth
%               \ifdim\wd0>\lw
%                 \sbox0{\tiny\t}^^A
%                 \ifdim\wd0>\linewidth
%                   \lwbox
%                 \else
%                   \usebox0
%                 \fi
%               \else
%                 \lwbox
%               \fi
%             \else
%               \usebox0
%             \fi
%           \else
%             \lwbox
%           \fi
%         \else
%           \usebox0
%         \fi
%       \else
%         \lwbox
%       \fi
%     \else
%       \usebox0
%     \fi
%   \else
%     \lwbox
%   \fi
% \else
%   \usebox0
% \fi
% \end{quote}
% If you have a \xfile{docstrip.cfg} that configures and enables \docstrip's
% TDS installing feature, then some files can already be in the right
% place, see the documentation of \docstrip.
%
% \subsection{Refresh file name databases}
%
% If your \TeX~distribution
% (\teTeX, \mikTeX, \dots) relies on file name databases, you must refresh
% these. For example, \teTeX\ users run \verb|texhash| or
% \verb|mktexlsr|.
%
% \subsection{Some details for the interested}
%
% \paragraph{Attached source.}
%
% The PDF documentation on CTAN also includes the
% \xfile{.dtx} source file. It can be extracted by
% AcrobatReader 6 or higher. Another option is \textsf{pdftk},
% e.g. unpack the file into the current directory:
% \begin{quote}
%   \verb|pdftk resizegather.pdf unpack_files output .|
% \end{quote}
%
% \paragraph{Unpacking with \LaTeX.}
% The \xfile{.dtx} chooses its action depending on the format:
% \begin{description}
% \item[\plainTeX:] Run \docstrip\ and extract the files.
% \item[\LaTeX:] Generate the documentation.
% \end{description}
% If you insist on using \LaTeX\ for \docstrip\ (really,
% \docstrip\ does not need \LaTeX), then inform the autodetect routine
% about your intention:
% \begin{quote}
%   \verb|latex \let\install=y\input{resizegather.dtx}|
% \end{quote}
% Do not forget to quote the argument according to the demands
% of your shell.
%
% \paragraph{Generating the documentation.}
% You can use both the \xfile{.dtx} or the \xfile{.drv} to generate
% the documentation. The process can be configured by the
% configuration file \xfile{ltxdoc.cfg}. For instance, put this
% line into this file, if you want to have A4 as paper format:
% \begin{quote}
%   \verb|\PassOptionsToClass{a4paper}{article}|
% \end{quote}
% An example follows how to generate the
% documentation with pdf\LaTeX:
% \begin{quote}
%\begin{verbatim}
%pdflatex resizegather.dtx
%makeindex -s gind.ist resizegather.idx
%pdflatex resizegather.dtx
%makeindex -s gind.ist resizegather.idx
%pdflatex resizegather.dtx
%\end{verbatim}
% \end{quote}
%
% \section{Acknowledgement}
%
% \begin{description}
% \item[Dieter Jurzitza:]
% He wanted the resizing feature at the \TeX\ table
% in Karlsruhe of December 2009. Thus this package is a kind of
% Christmas present.
% \end{description}
%
% \begin{History}
%   \begin{Version}{2009/12/04 v1.0}
%   \item
%     The first version.
%   \end{Version}
%   \begin{Version}{2009/12/05 v1.1}
%   \item
%     Options \xoption{enable} and \xoption{disable} added.
%   \end{Version}
%   \begin{Version}{2010/03/01 v1.2}
%   \item
%     TDS location moved from `generic' to `latex'.
%   \end{Version}
%   \begin{Version}{2016/05/16 v1.3}
%   \item
%     Documentation updates.
%   \end{Version}
% \end{History}
%
% \PrintIndex
%
% \Finale
\endinput

%        (quote the arguments according to the demands of your shell)
%
% Documentation:
%    (a) If resizegather.drv is present:
%           latex resizegather.drv
%    (b) Without resizegather.drv:
%           latex resizegather.dtx; ...
%    The class ltxdoc loads the configuration file ltxdoc.cfg
%    if available. Here you can specify further options, e.g.
%    use A4 as paper format:
%       \PassOptionsToClass{a4paper}{article}
%
%    Programm calls to get the documentation (example):
%       pdflatex resizegather.dtx
%       makeindex -s gind.ist resizegather.idx
%       pdflatex resizegather.dtx
%       makeindex -s gind.ist resizegather.idx
%       pdflatex resizegather.dtx
%
% Installation:
%    TDS:tex/latex/oberdiek/resizegather.sty
%    TDS:doc/latex/oberdiek/resizegather.pdf
%    TDS:doc/latex/oberdiek/test/resizegather-test1.tex
%    TDS:source/latex/oberdiek/resizegather.dtx
%
%<*ignore>
\begingroup
  \catcode123=1 %
  \catcode125=2 %
  \def\x{LaTeX2e}%
\expandafter\endgroup
\ifcase 0\ifx\install y1\fi\expandafter
         \ifx\csname processbatchFile\endcsname\relax\else1\fi
         \ifx\fmtname\x\else 1\fi\relax
\else\csname fi\endcsname
%</ignore>
%<*install>
\input docstrip.tex
\Msg{************************************************************************}
\Msg{* Installation}
\Msg{* Package: resizegather 2016/05/16 v1.3 Resize overly large equations (HO)}
\Msg{************************************************************************}

\keepsilent
\askforoverwritefalse

\let\MetaPrefix\relax
\preamble

This is a generated file.

Project: resizegather
Version: 2016/05/16 v1.3

Copyright (C) 2009, 2010 by
   Heiko Oberdiek <heiko.oberdiek at googlemail.com>

This work may be distributed and/or modified under the
conditions of the LaTeX Project Public License, either
version 1.3c of this license or (at your option) any later
version. This version of this license is in
   http://www.latex-project.org/lppl/lppl-1-3c.txt
and the latest version of this license is in
   http://www.latex-project.org/lppl.txt
and version 1.3 or later is part of all distributions of
LaTeX version 2005/12/01 or later.

This work has the LPPL maintenance status "maintained".

This Current Maintainer of this work is Heiko Oberdiek.

This work consists of the main source file resizegather.dtx
and the derived files
   resizegather.sty, resizegather.pdf, resizegather.ins, resizegather.drv,
   resizegather-test1.tex.

\endpreamble
\let\MetaPrefix\DoubleperCent

\generate{%
  \file{resizegather.ins}{\from{resizegather.dtx}{install}}%
  \file{resizegather.drv}{\from{resizegather.dtx}{driver}}%
  \usedir{tex/latex/oberdiek}%
  \file{resizegather.sty}{\from{resizegather.dtx}{package}}%
%  \usedir{doc/latex/oberdiek/test}%
%  \file{resizegather-test1.tex}{\from{resizegather.dtx}{test1}}%
  \nopreamble
  \nopostamble
%  \usedir{source/latex/oberdiek/catalogue}%
%  \file{resizegather.xml}{\from{resizegather.dtx}{catalogue}}%
}

\catcode32=13\relax% active space
\let =\space%
\Msg{************************************************************************}
\Msg{*}
\Msg{* To finish the installation you have to move the following}
\Msg{* file into a directory searched by TeX:}
\Msg{*}
\Msg{*     resizegather.sty}
\Msg{*}
\Msg{* To produce the documentation run the file `resizegather.drv'}
\Msg{* through LaTeX.}
\Msg{*}
\Msg{* Happy TeXing!}
\Msg{*}
\Msg{************************************************************************}

\endbatchfile
%</install>
%<*ignore>
\fi
%</ignore>
%<*driver>
\NeedsTeXFormat{LaTeX2e}
\ProvidesFile{resizegather.drv}%
  [2016/05/16 v1.3 Resize overly large equations (HO)]%
\documentclass{ltxdoc}
\usepackage{holtxdoc}[2011/11/22]
\usepackage{ifluatex}
\ifluatex
\else
  \usepackage[T1]{fontenc}%
  \usepackage{textcomp}%
  \usepackage{lmodern}%
\fi
\begin{document}
  \DocInput{resizegather.dtx}%
\end{document}
%</driver>
% \fi
%
%
% \CharacterTable
%  {Upper-case    \A\B\C\D\E\F\G\H\I\J\K\L\M\N\O\P\Q\R\S\T\U\V\W\X\Y\Z
%   Lower-case    \a\b\c\d\e\f\g\h\i\j\k\l\m\n\o\p\q\r\s\t\u\v\w\x\y\z
%   Digits        \0\1\2\3\4\5\6\7\8\9
%   Exclamation   \!     Double quote  \"     Hash (number) \#
%   Dollar        \$     Percent       \%     Ampersand     \&
%   Acute accent  \'     Left paren    \(     Right paren   \)
%   Asterisk      \*     Plus          \+     Comma         \,
%   Minus         \-     Point         \.     Solidus       \/
%   Colon         \:     Semicolon     \;     Less than     \<
%   Equals        \=     Greater than  \>     Question mark \?
%   Commercial at \@     Left bracket  \[     Backslash     \\
%   Right bracket \]     Circumflex    \^     Underscore    \_
%   Grave accent  \`     Left brace    \{     Vertical bar  \|
%   Right brace   \}     Tilde         \~}
%
% \GetFileInfo{resizegather.drv}
%
% \title{The \xpackage{resizegather} package}
% \date{2016/05/16 v1.3}
% \author{Heiko Oberdiek\thanks
% {Please report any issues at https://github.com/ho-tex/oberdiek/issues}\\
% \xemail{heiko.oberdiek at googlemail.com}}
%
% \maketitle
%
% \begin{abstract}
% Equations that are too large are resized to fit the available
% space. The environment \textsf{gather} of package \xpackage{amsmath}
% is supported. Also the environments \textsf{equation} and
% \textsf{displaymath} are redefined using \textsf{gather}
% and its starred version.
% \end{abstract}
%
% \tableofcontents
%
% \makeatletter
% \def\env#1{^^A
%    \textsf{\@env#1*\@nil}^^A
% }%
% \def\@env#1*#2\@nil{^^A
%   #1^^A
%   \ifx\\#2\\^^A
%     \expandafter\@gobble
%   \else
%     \textasteriskcentered
%     \expandafter\@firstofone
%   \fi
%   {\@env#2\@nil}^^A
% }
% \makeatother
%
% \section{Documentation}
%
% Sometimes an equation is just a little to large to fit in the
% line. And breaking the equation across lines might be worse
% than downscaling the equation. This package implements this
% for the environments \env{gather} and \env{gather*} of
% package \xpackage{amsmath}. That package already measures
% the equations and simplifies the implementation of \xpackage{resizegather}
% that only needs to hook into \xpackage{amsmath}'s code to add
% the resizing feature.
%
% Resized equations are recorded in the \xext{log} file
% for small exceeds (default setting is smaller than five percent).
% Otherwise a warning is given.
%
% Also environments \env{equation} and \env{displaymath}
% are supported by redefining them using \env{gather}
% and \env{gather*}.
%
% \cs{[} and \cs{]} are not supported, because these macros
% are not in environment form that is required for
% \xpackage{amsmath}. The environment body is collected
% first to be able to process the body twice for measuring
% first.
%
% Also the environments using alignments are not supported.
% If a single equation line would be resized, the alignment
% would get lost. And resizing all equations of the alignment
% does not seem appropriate either.
%
% \subsection{Options}
%
% \begin{description}
% \item[\xoption{warningthreshold}:]
%   Print a warning if the original equation line exceeds
%   its available width by the given fraction.
%   Default is |0.05|: A warning is given if the equation
%   is too large by five percent.
%   Otherwise the exceed is recorded in the \xext{log} file
%   only.
% \end{description}
% The next options are boolean options. They are enabled
% by value |true| or if no value is given. They are switched
% off by value |false|.
% \begin{description}
% \item[\xoption{enable}:] The resize feature is active (default).
% \item[\xoption{disable}:] The complementary option for \xoption{enable},
%  added for convenience: |disable| (or |disable=true|) is the same
%  as |enable=false|.
% \item[\xoption{equations}:]
%   \LaTeX\ environments \textsf{equation} and \textsf{displaymath}
%   environments are redefined. These equations
%   are now using environment \env{gather} and
%   \env{gather*}. This is the default.
% \end{description}
% The following table shows additional options if you
% want to have finer control for the redefined
% environments:
% \begin{quote}
% \def\unchanged{\textit{unchanged}}
% \def\notprovided{\textit{not provided}}
% \begin{tabular}{l|ll}
% &\multicolumn{2}{c}{Environments}\\
% Option & \env{equation} & \env{displaymath}\\
% \hline
% \xoption{equations} & \env{gather} & \env{gather*}\\
% \xoption{equation} & \env{gather} & \unchanged\\
% \xoption{displaymath} & \unchanged & \env{gather*}\\
% \end{tabular}
% \end{quote}
% If such an option is switched off, the original meaning
% of the affected environments is restored.
%
% Options are evaluated in the following order:
% \begin{enumerate}
% \item
%  Configuration file \xfile{resizegather.cfg} using \cs{resizegathersetup}
%  if the file exists.
%  \item
%  Package options given for \cs{usepackage}.
%  \item
%  Later calls of \cs{resizegathersetup}.
% \end{enumerate}
% \begin{declcs}{resizegathersetup}\M{option list}
% \end{declcs}
% The options are key value options. Boolean options are enabled by
% default (without value) or by using the explicit value \texttt{true}.
% Value \texttt{false} disable the option.
%
% \subsection{Options for packages \xpackage{amsmath} or \xpackage{graphics}}
%
% The package loads the package \xpackage{amsmath} because is internally
% measures the equations first. Thus this package hooks into this code
% in order to resize the equations if they are too large.
% The resizing itself is done by \cs{resizebox} of package \xpackage{graphics}.
% If you need special options for these packages, just load them first or
% use global options when appropriate. Example:
% \begin{quote}
%\begin{verbatim}
%\usepackage[dvipdfm]{graphicx}% or graphics
%\usepackage[fleqn]{amsmath}
%\usepackage{resizegather}
%\end{verbatim}
%\end{quote}
%
% \StopEventually{
% }
%
% \section{Implementation}
%    \begin{macrocode}
%<*package>
%    \end{macrocode}
%    Reload check, especially if the package is not used with \LaTeX.
%    \begin{macrocode}
\begingroup\catcode61\catcode48\catcode32=10\relax%
  \catcode13=5 % ^^M
  \endlinechar=13 %
  \catcode35=6 % #
  \catcode39=12 % '
  \catcode44=12 % ,
  \catcode45=12 % -
  \catcode46=12 % .
  \catcode58=12 % :
  \catcode64=11 % @
  \catcode123=1 % {
  \catcode125=2 % }
  \expandafter\let\expandafter\x\csname ver@resizegather.sty\endcsname
  \ifx\x\relax % plain-TeX, first loading
  \else
    \def\empty{}%
    \ifx\x\empty % LaTeX, first loading,
      % variable is initialized, but \ProvidesPackage not yet seen
    \else
      \expandafter\ifx\csname PackageInfo\endcsname\relax
        \def\x#1#2{%
          \immediate\write-1{Package #1 Info: #2.}%
        }%
      \else
        \def\x#1#2{\PackageInfo{#1}{#2, stopped}}%
      \fi
      \x{resizegather}{The package is already loaded}%
      \aftergroup\endinput
    \fi
  \fi
\endgroup%
%    \end{macrocode}
%    Package identification:
%    \begin{macrocode}
\begingroup\catcode61\catcode48\catcode32=10\relax%
  \catcode13=5 % ^^M
  \endlinechar=13 %
  \catcode35=6 % #
  \catcode39=12 % '
  \catcode40=12 % (
  \catcode41=12 % )
  \catcode44=12 % ,
  \catcode45=12 % -
  \catcode46=12 % .
  \catcode47=12 % /
  \catcode58=12 % :
  \catcode64=11 % @
  \catcode91=12 % [
  \catcode93=12 % ]
  \catcode123=1 % {
  \catcode125=2 % }
  \expandafter\ifx\csname ProvidesPackage\endcsname\relax
    \def\x#1#2#3[#4]{\endgroup
      \immediate\write-1{Package: #3 #4}%
      \xdef#1{#4}%
    }%
  \else
    \def\x#1#2[#3]{\endgroup
      #2[{#3}]%
      \ifx#1\@undefined
        \xdef#1{#3}%
      \fi
      \ifx#1\relax
        \xdef#1{#3}%
      \fi
    }%
  \fi
\expandafter\x\csname ver@resizegather.sty\endcsname
\ProvidesPackage{resizegather}%
  [2016/05/16 v1.3 Resize overly large equations (HO)]%
%    \end{macrocode}
%
%    \begin{macrocode}
\begingroup\catcode61\catcode48\catcode32=10\relax%
  \catcode13=5 % ^^M
  \endlinechar=13 %
  \catcode123=1 % {
  \catcode125=2 % }
  \catcode64=11 % @
  \def\x{\endgroup
    \expandafter\edef\csname ResizeGather@AtEnd\endcsname{%
      \endlinechar=\the\endlinechar\relax
      \catcode13=\the\catcode13\relax
      \catcode32=\the\catcode32\relax
      \catcode35=\the\catcode35\relax
      \catcode61=\the\catcode61\relax
      \catcode64=\the\catcode64\relax
      \catcode123=\the\catcode123\relax
      \catcode125=\the\catcode125\relax
    }%
  }%
\x\catcode61\catcode48\catcode32=10\relax%
\catcode13=5 % ^^M
\endlinechar=13 %
\catcode35=6 % #
\catcode64=11 % @
\catcode123=1 % {
\catcode125=2 % }
\def\TMP@EnsureCode#1#2{%
  \edef\ResizeGather@AtEnd{%
    \ResizeGather@AtEnd
    \catcode#1=\the\catcode#1\relax
  }%
  \catcode#1=#2\relax
}
\TMP@EnsureCode{10}{12}% ^^J
\TMP@EnsureCode{33}{12}% !
\TMP@EnsureCode{36}{3}% $
\TMP@EnsureCode{38}{4}% &
\TMP@EnsureCode{39}{12}% '
\TMP@EnsureCode{40}{12}% (
\TMP@EnsureCode{41}{12}% )
\TMP@EnsureCode{42}{12}% *
\TMP@EnsureCode{43}{12}% +
\TMP@EnsureCode{44}{12}% ,
\TMP@EnsureCode{45}{12}% -
\TMP@EnsureCode{46}{12}% .
\TMP@EnsureCode{47}{12}% /
\TMP@EnsureCode{58}{12}% :
\TMP@EnsureCode{59}{12}% ;
\TMP@EnsureCode{60}{12}% <
\TMP@EnsureCode{62}{12}% >
\TMP@EnsureCode{63}{12}% ?
\TMP@EnsureCode{91}{12}% [
\TMP@EnsureCode{93}{12}% ]
\TMP@EnsureCode{94}{7}% ^ (superscript)
\TMP@EnsureCode{96}{12}% `
\edef\ResizeGather@AtEnd{\ResizeGather@AtEnd\noexpand\endinput}
%    \end{macrocode}
%
%    \begin{macrocode}
\RequirePackage{kvoptions}[2009/12/04]
\SetupKeyvalOptions{%
  family=resizegather,%
  prefix=ResizeGather@,%
}
%    \end{macrocode}
%    \begin{macrocode}
\@for\ResizeGather@option:=%
  centertags,%
  tbtags,%
  sumlimits,%
  nosumlimits,%
  intlimits,%
  nointlimits,%
  nonamelimits,%
  leqno,%
  reqno,%
  fleqn%
\do{%
  \edef\ResizeGather@temp{%
    \noexpand\DeclareVoidOption{\ResizeGather@option}{%
      \noexpand\PassOptionsToPackage{amsmath}{\ResizeGather@option}%
    }%
    \noexpand\AtEndOfPackage{%
      \noexpand\DisableKeyvalOption[%
        action=error,%
        package=resizegather,%
      ]{resizegather}{\ResizeGather@option}%
    }%
  }%
  \ResizeGather@temp
}
\@for\ResizeGather@option:=%
  draft,%
  final,%
  hiderotate,%
  hidescale,%
  hiresbb,%
  demo,%
  dvips,xdvi,dvipdf,dvipdfm,dvipdfmx,pdftex,dvipsone,%
  dviwindo,emtex,dviwin,pctexps,pctexwin,pctexhp,pctex32,%
  truetex,tcidvi,vtex,oztex,textures,xetex%
\do{%
  \edef\ResizeGather@temp{%
    \noexpand\DeclareVoidOption{\ResizeGather@option}{%
      \noexpand\PassOptionsToPackage{graphics}{\ResizeGather@option}%
    }%
    \noexpand\AtEndOfPackage{%
      \noexpand\DisableKeyvalOption[%
        action=error,%
        package=resizegather,%
      ]{resizegather}{\ResizeGather@option}%
    }%
  }%
  \ResizeGather@temp
}
%    \end{macrocode}
%
%    \begin{macrocode}
\DeclareBoolOption[true]{enable}
\DeclareComplementaryOption{disable}{enable}
\DeclareStringOption[.05]{warningthreshold}
\newif\ifResizeGather@NeedInit
\DeclareBoolOption[true]{equations}
\DeclareBoolOption[true]{equation}
\DeclareBoolOption[true]{displaymath}
\AddToKeyvalOption*{equations}{%
  \ResizeGather@NeedInittrue
  \ifResizeGather@equations
    \ResizeGather@equationtrue
    \ResizeGather@displaymathtrue
  \else
    \ResizeGather@equationfalse
    \ResizeGather@displaymathfalse
  \fi
}
\AddToKeyvalOption*{equation}{%
  \ResizeGather@NeedInittrue
}
\AddToKeyvalOption*{displaymath}{%
  \ResizeGather@NeedInittrue
}
%    \end{macrocode}
%
%    \begin{macro}{\resizegathersetup}
%    \begin{macrocode}
\newcommand*{\resizegathersetup}[1]{%
  \ResizeGather@NeedInitfalse
  \setkeys{resizegather}{#1}%
  \ifResizeGather@NeedInit
    \ResizeGather@init
  \fi
}
\let\ResizeGather@init\relax
%    \end{macrocode}
%    \end{macro}
%    \begin{macrocode}
\InputIfFileExists{resizegather.cfg}{}{}%
\ProcessKeyvalOptions*\relax
%    \end{macrocode}
%    \begin{macrocode}
\RequirePackage{amsmath}
\RequirePackage{graphics}
%    \end{macrocode}
%    \begin{macro}{\ResizeGather@redefine}
%    \begin{macrocode}
\def\ResizeGather@redefine#1#2#3#4#5{%
  \csname ifResizeGather@#1\endcsname
    \@ifundefined{ResizeGather@org@#2}{%
      \expandafter\let\csname ResizeGather@org@#2\expandafter\endcsname
                      \csname #2\endcsname
    }{}%
    \@ifundefined{ResizeGather@org@#3}{%
      \expandafter\let\csname ResizeGather@org@#3\expandafter\endcsname
                      \csname #3\endcsname
    }{}%
    \expandafter\edef\csname #2\endcsname{%
      \expandafter\noexpand\csname#4\endcsname
    }%
    \expandafter\edef\csname #3\endcsname{%
      \expandafter\noexpand\csname#5\endcsname
    }%
  \else
    \@ifundefined{ResizeGather@org@#2}{}{%
      \expandafter\let\csname #2\expandafter\endcsname
                      \csname ResizeGather@org@#2\endcsname
      \expandafter\let\csname #3\expandafter\endcsname
                      \csname ResizeGather@org@#3\endcsname
    }%
  \fi
}
%    \end{macrocode}
%    \end{macro}
%    \begin{macro}{\ResizeGather@init}
%    \begin{macrocode}
\def\ResizeGather@init{%
  \ResizeGather@redefine{equation}{equation}{endequation}%
    {gather}{endgather}%
  \ResizeGather@redefine{displaymath}{displaymath}{enddisplaymath}%
    {gather*}{endgather*}%
}
\ResizeGather@init
%    \end{macrocode}
%    \end{macro}
%
%    \begin{macro}{\ResizeGather@ResizeGather}
%    \begin{macrocode}
\def\ResizeGather@ResizeGather{%
  \ifResizeGather@enable
    \dimen@\displaywidth
    \if@fleqn
      \advance\dimen@-\@mathmargin
    \fi
    \ifdim\wdz@>\dimen@
      \begingroup
        \advance\dimen@ -\wdz@
        \dimen@ -\dimen@
        \ifdim\ResizeGather@warningthreshold\wdz@>\dimen@
          \expandafter\PackageInfo
        \else
          \expandafter\PackageWarning
        \fi
        {resizegather}{%
          Equation line \the\row@\space is too large %
          by \the\dimen@\MessageBreak
          in environment `\@currenvir'%
        }%
      \endgroup
      \setboxz@h to\dimen@{%
        \resizebox{\dimen@}{!}{\boxz@}%
        \hss
      }%
    \fi
  \fi
}
%    \end{macrocode}
%    \end{macro}
%    \begin{macro}{\calc@shift@gather}
%    \begin{macrocode}
\expandafter\def\expandafter\calc@shift@gather\expandafter{%
  \expandafter\ResizeGather@ResizeGather
  \calc@shift@gather
}
%    \end{macrocode}
%    \end{macro}
%    \begin{macro}{\ResizeGather@org@gmeasure@}
%    \begin{macrocode}
\let\ResizeGather@org@gmeasure@\gmeasure@
%    \end{macrocode}
%    \end{macro}
%    \begin{macro}{\gmeasure@}
%    \begin{macrocode}
\def\gmeasure@#1{%
  \ResizeGather@org@gmeasure@{#1}%
  \ifResizeGather@enable
    \ifdim\totwidth@>\displaywidth
      \totwidth@=\displaywidth
    \fi
  \fi
}
%    \end{macrocode}
%    \end{macro}
%
%    \begin{macrocode}
\ResizeGather@AtEnd%
%</package>
%    \end{macrocode}
%
% \section{Test}
%
% \subsection{Catcode checks for loading}
%
%    \begin{macrocode}
%<*test1>
%    \end{macrocode}
%    \begin{macrocode}
\catcode`\{=1 %
\catcode`\}=2 %
\catcode`\#=6 %
\catcode`\@=11 %
\expandafter\ifx\csname count@\endcsname\relax
  \countdef\count@=255 %
\fi
\expandafter\ifx\csname @gobble\endcsname\relax
  \long\def\@gobble#1{}%
\fi
\expandafter\ifx\csname @firstofone\endcsname\relax
  \long\def\@firstofone#1{#1}%
\fi
\expandafter\ifx\csname loop\endcsname\relax
  \expandafter\@firstofone
\else
  \expandafter\@gobble
\fi
{%
  \def\loop#1\repeat{%
    \def\body{#1}%
    \iterate
  }%
  \def\iterate{%
    \body
      \let\next\iterate
    \else
      \let\next\relax
    \fi
    \next
  }%
  \let\repeat=\fi
}%
\def\RestoreCatcodes{}
\count@=0 %
\loop
  \edef\RestoreCatcodes{%
    \RestoreCatcodes
    \catcode\the\count@=\the\catcode\count@\relax
  }%
\ifnum\count@<255 %
  \advance\count@ 1 %
\repeat

\def\RangeCatcodeInvalid#1#2{%
  \count@=#1\relax
  \loop
    \catcode\count@=15 %
  \ifnum\count@<#2\relax
    \advance\count@ 1 %
  \repeat
}
\def\RangeCatcodeCheck#1#2#3{%
  \count@=#1\relax
  \loop
    \ifnum#3=\catcode\count@
    \else
      \errmessage{%
        Character \the\count@\space
        with wrong catcode \the\catcode\count@\space
        instead of \number#3%
      }%
    \fi
  \ifnum\count@<#2\relax
    \advance\count@ 1 %
  \repeat
}
\def\space{ }
\expandafter\ifx\csname LoadCommand\endcsname\relax
  \def\LoadCommand{\input resizegather.sty\relax}%
\fi
\def\Test{%
  \RangeCatcodeInvalid{0}{47}%
  \RangeCatcodeInvalid{58}{64}%
  \RangeCatcodeInvalid{91}{96}%
  \RangeCatcodeInvalid{123}{255}%
  \catcode`\@=12 %
  \catcode`\\=0 %
  \catcode`\%=14 %
  \LoadCommand
  \RangeCatcodeCheck{0}{36}{15}%
  \RangeCatcodeCheck{37}{37}{14}%
  \RangeCatcodeCheck{38}{47}{15}%
  \RangeCatcodeCheck{48}{57}{12}%
  \RangeCatcodeCheck{58}{63}{15}%
  \RangeCatcodeCheck{64}{64}{12}%
  \RangeCatcodeCheck{65}{90}{11}%
  \RangeCatcodeCheck{91}{91}{15}%
  \RangeCatcodeCheck{92}{92}{0}%
  \RangeCatcodeCheck{93}{96}{15}%
  \RangeCatcodeCheck{97}{122}{11}%
  \RangeCatcodeCheck{123}{255}{15}%
  \RestoreCatcodes
}
\Test
\csname @@end\endcsname
\end
%    \end{macrocode}
%    \begin{macrocode}
%</test1>
%    \end{macrocode}
%
% \section{Installation}
%
% \subsection{Download}
%
% \paragraph{Package.} This package is available on
% CTAN\footnote{\url{https://ctan.org/pkg/resizegather}}:
% \begin{description}
% \item[\CTAN{macros/latex/contrib/oberdiek/resizegather.dtx}] The source file.
% \item[\CTAN{macros/latex/contrib/oberdiek/resizegather.pdf}] Documentation.
% \end{description}
%
%
% \paragraph{Bundle.} All the packages of the bundle `oberdiek'
% are also available in a TDS compliant ZIP archive. There
% the packages are already unpacked and the documentation files
% are generated. The files and directories obey the TDS standard.
% \begin{description}
% \item[\CTANinstall{install/macros/latex/contrib/oberdiek.tds.zip}]
% \end{description}
% \emph{TDS} refers to the standard ``A Directory Structure
% for \TeX\ Files'' (\CTAN{tds/tds.pdf}). Directories
% with \xfile{texmf} in their name are usually organized this way.
%
% \subsection{Bundle installation}
%
% \paragraph{Unpacking.} Unpack the \xfile{oberdiek.tds.zip} in the
% TDS tree (also known as \xfile{texmf} tree) of your choice.
% Example (linux):
% \begin{quote}
%   |unzip oberdiek.tds.zip -d ~/texmf|
% \end{quote}
%
% \paragraph{Script installation.}
% Check the directory \xfile{TDS:scripts/oberdiek/} for
% scripts that need further installation steps.
% Package \xpackage{attachfile2} comes with the Perl script
% \xfile{pdfatfi.pl} that should be installed in such a way
% that it can be called as \texttt{pdfatfi}.
% Example (linux):
% \begin{quote}
%   |chmod +x scripts/oberdiek/pdfatfi.pl|\\
%   |cp scripts/oberdiek/pdfatfi.pl /usr/local/bin/|
% \end{quote}
%
% \subsection{Package installation}
%
% \paragraph{Unpacking.} The \xfile{.dtx} file is a self-extracting
% \docstrip\ archive. The files are extracted by running the
% \xfile{.dtx} through \plainTeX:
% \begin{quote}
%   \verb|tex resizegather.dtx|
% \end{quote}
%
% \paragraph{TDS.} Now the different files must be moved into
% the different directories in your installation TDS tree
% (also known as \xfile{texmf} tree):
% \begin{quote}
% \def\t{^^A
% \begin{tabular}{@{}>{\ttfamily}l@{ $\rightarrow$ }>{\ttfamily}l@{}}
%   resizegather.sty & tex/latex/oberdiek/resizegather.sty\\
%   resizegather.pdf & doc/latex/oberdiek/resizegather.pdf\\
%   test/resizegather-test1.tex & doc/latex/oberdiek/test/resizegather-test1.tex\\
%   resizegather.dtx & source/latex/oberdiek/resizegather.dtx\\
% \end{tabular}^^A
% }^^A
% \sbox0{\t}^^A
% \ifdim\wd0>\linewidth
%   \begingroup
%     \advance\linewidth by\leftmargin
%     \advance\linewidth by\rightmargin
%   \edef\x{\endgroup
%     \def\noexpand\lw{\the\linewidth}^^A
%   }\x
%   \def\lwbox{^^A
%     \leavevmode
%     \hbox to \linewidth{^^A
%       \kern-\leftmargin\relax
%       \hss
%       \usebox0
%       \hss
%       \kern-\rightmargin\relax
%     }^^A
%   }^^A
%   \ifdim\wd0>\lw
%     \sbox0{\small\t}^^A
%     \ifdim\wd0>\linewidth
%       \ifdim\wd0>\lw
%         \sbox0{\footnotesize\t}^^A
%         \ifdim\wd0>\linewidth
%           \ifdim\wd0>\lw
%             \sbox0{\scriptsize\t}^^A
%             \ifdim\wd0>\linewidth
%               \ifdim\wd0>\lw
%                 \sbox0{\tiny\t}^^A
%                 \ifdim\wd0>\linewidth
%                   \lwbox
%                 \else
%                   \usebox0
%                 \fi
%               \else
%                 \lwbox
%               \fi
%             \else
%               \usebox0
%             \fi
%           \else
%             \lwbox
%           \fi
%         \else
%           \usebox0
%         \fi
%       \else
%         \lwbox
%       \fi
%     \else
%       \usebox0
%     \fi
%   \else
%     \lwbox
%   \fi
% \else
%   \usebox0
% \fi
% \end{quote}
% If you have a \xfile{docstrip.cfg} that configures and enables \docstrip's
% TDS installing feature, then some files can already be in the right
% place, see the documentation of \docstrip.
%
% \subsection{Refresh file name databases}
%
% If your \TeX~distribution
% (\teTeX, \mikTeX, \dots) relies on file name databases, you must refresh
% these. For example, \teTeX\ users run \verb|texhash| or
% \verb|mktexlsr|.
%
% \subsection{Some details for the interested}
%
% \paragraph{Attached source.}
%
% The PDF documentation on CTAN also includes the
% \xfile{.dtx} source file. It can be extracted by
% AcrobatReader 6 or higher. Another option is \textsf{pdftk},
% e.g. unpack the file into the current directory:
% \begin{quote}
%   \verb|pdftk resizegather.pdf unpack_files output .|
% \end{quote}
%
% \paragraph{Unpacking with \LaTeX.}
% The \xfile{.dtx} chooses its action depending on the format:
% \begin{description}
% \item[\plainTeX:] Run \docstrip\ and extract the files.
% \item[\LaTeX:] Generate the documentation.
% \end{description}
% If you insist on using \LaTeX\ for \docstrip\ (really,
% \docstrip\ does not need \LaTeX), then inform the autodetect routine
% about your intention:
% \begin{quote}
%   \verb|latex \let\install=y% \iffalse meta-comment
%
% File: resizegather.dtx
% Version: 2016/05/16 v1.3
% Info: Resize overly large equations
%
% Copyright (C) 2009, 2010 by
%    Heiko Oberdiek <heiko.oberdiek at googlemail.com>
%    2016
%    https://github.com/ho-tex/oberdiek/issues
%
% This work may be distributed and/or modified under the
% conditions of the LaTeX Project Public License, either
% version 1.3c of this license or (at your option) any later
% version. This version of this license is in
%    http://www.latex-project.org/lppl/lppl-1-3c.txt
% and the latest version of this license is in
%    http://www.latex-project.org/lppl.txt
% and version 1.3 or later is part of all distributions of
% LaTeX version 2005/12/01 or later.
%
% This work has the LPPL maintenance status "maintained".
%
% This Current Maintainer of this work is Heiko Oberdiek.
%
% This work consists of the main source file resizegather.dtx
% and the derived files
%    resizegather.sty, resizegather.pdf, resizegather.ins, resizegather.drv,
%    resizegather-test1.tex.
%
% Distribution:
%    CTAN:macros/latex/contrib/oberdiek/resizegather.dtx
%    CTAN:macros/latex/contrib/oberdiek/resizegather.pdf
%
% Unpacking:
%    (a) If resizegather.ins is present:
%           tex resizegather.ins
%    (b) Without resizegather.ins:
%           tex resizegather.dtx
%    (c) If you insist on using LaTeX
%           latex \let\install=y\input{resizegather.dtx}
%        (quote the arguments according to the demands of your shell)
%
% Documentation:
%    (a) If resizegather.drv is present:
%           latex resizegather.drv
%    (b) Without resizegather.drv:
%           latex resizegather.dtx; ...
%    The class ltxdoc loads the configuration file ltxdoc.cfg
%    if available. Here you can specify further options, e.g.
%    use A4 as paper format:
%       \PassOptionsToClass{a4paper}{article}
%
%    Programm calls to get the documentation (example):
%       pdflatex resizegather.dtx
%       makeindex -s gind.ist resizegather.idx
%       pdflatex resizegather.dtx
%       makeindex -s gind.ist resizegather.idx
%       pdflatex resizegather.dtx
%
% Installation:
%    TDS:tex/latex/oberdiek/resizegather.sty
%    TDS:doc/latex/oberdiek/resizegather.pdf
%    TDS:doc/latex/oberdiek/test/resizegather-test1.tex
%    TDS:source/latex/oberdiek/resizegather.dtx
%
%<*ignore>
\begingroup
  \catcode123=1 %
  \catcode125=2 %
  \def\x{LaTeX2e}%
\expandafter\endgroup
\ifcase 0\ifx\install y1\fi\expandafter
         \ifx\csname processbatchFile\endcsname\relax\else1\fi
         \ifx\fmtname\x\else 1\fi\relax
\else\csname fi\endcsname
%</ignore>
%<*install>
\input docstrip.tex
\Msg{************************************************************************}
\Msg{* Installation}
\Msg{* Package: resizegather 2016/05/16 v1.3 Resize overly large equations (HO)}
\Msg{************************************************************************}

\keepsilent
\askforoverwritefalse

\let\MetaPrefix\relax
\preamble

This is a generated file.

Project: resizegather
Version: 2016/05/16 v1.3

Copyright (C) 2009, 2010 by
   Heiko Oberdiek <heiko.oberdiek at googlemail.com>

This work may be distributed and/or modified under the
conditions of the LaTeX Project Public License, either
version 1.3c of this license or (at your option) any later
version. This version of this license is in
   http://www.latex-project.org/lppl/lppl-1-3c.txt
and the latest version of this license is in
   http://www.latex-project.org/lppl.txt
and version 1.3 or later is part of all distributions of
LaTeX version 2005/12/01 or later.

This work has the LPPL maintenance status "maintained".

This Current Maintainer of this work is Heiko Oberdiek.

This work consists of the main source file resizegather.dtx
and the derived files
   resizegather.sty, resizegather.pdf, resizegather.ins, resizegather.drv,
   resizegather-test1.tex.

\endpreamble
\let\MetaPrefix\DoubleperCent

\generate{%
  \file{resizegather.ins}{\from{resizegather.dtx}{install}}%
  \file{resizegather.drv}{\from{resizegather.dtx}{driver}}%
  \usedir{tex/latex/oberdiek}%
  \file{resizegather.sty}{\from{resizegather.dtx}{package}}%
%  \usedir{doc/latex/oberdiek/test}%
%  \file{resizegather-test1.tex}{\from{resizegather.dtx}{test1}}%
  \nopreamble
  \nopostamble
%  \usedir{source/latex/oberdiek/catalogue}%
%  \file{resizegather.xml}{\from{resizegather.dtx}{catalogue}}%
}

\catcode32=13\relax% active space
\let =\space%
\Msg{************************************************************************}
\Msg{*}
\Msg{* To finish the installation you have to move the following}
\Msg{* file into a directory searched by TeX:}
\Msg{*}
\Msg{*     resizegather.sty}
\Msg{*}
\Msg{* To produce the documentation run the file `resizegather.drv'}
\Msg{* through LaTeX.}
\Msg{*}
\Msg{* Happy TeXing!}
\Msg{*}
\Msg{************************************************************************}

\endbatchfile
%</install>
%<*ignore>
\fi
%</ignore>
%<*driver>
\NeedsTeXFormat{LaTeX2e}
\ProvidesFile{resizegather.drv}%
  [2016/05/16 v1.3 Resize overly large equations (HO)]%
\documentclass{ltxdoc}
\usepackage{holtxdoc}[2011/11/22]
\usepackage{ifluatex}
\ifluatex
\else
  \usepackage[T1]{fontenc}%
  \usepackage{textcomp}%
  \usepackage{lmodern}%
\fi
\begin{document}
  \DocInput{resizegather.dtx}%
\end{document}
%</driver>
% \fi
%
%
% \CharacterTable
%  {Upper-case    \A\B\C\D\E\F\G\H\I\J\K\L\M\N\O\P\Q\R\S\T\U\V\W\X\Y\Z
%   Lower-case    \a\b\c\d\e\f\g\h\i\j\k\l\m\n\o\p\q\r\s\t\u\v\w\x\y\z
%   Digits        \0\1\2\3\4\5\6\7\8\9
%   Exclamation   \!     Double quote  \"     Hash (number) \#
%   Dollar        \$     Percent       \%     Ampersand     \&
%   Acute accent  \'     Left paren    \(     Right paren   \)
%   Asterisk      \*     Plus          \+     Comma         \,
%   Minus         \-     Point         \.     Solidus       \/
%   Colon         \:     Semicolon     \;     Less than     \<
%   Equals        \=     Greater than  \>     Question mark \?
%   Commercial at \@     Left bracket  \[     Backslash     \\
%   Right bracket \]     Circumflex    \^     Underscore    \_
%   Grave accent  \`     Left brace    \{     Vertical bar  \|
%   Right brace   \}     Tilde         \~}
%
% \GetFileInfo{resizegather.drv}
%
% \title{The \xpackage{resizegather} package}
% \date{2016/05/16 v1.3}
% \author{Heiko Oberdiek\thanks
% {Please report any issues at https://github.com/ho-tex/oberdiek/issues}\\
% \xemail{heiko.oberdiek at googlemail.com}}
%
% \maketitle
%
% \begin{abstract}
% Equations that are too large are resized to fit the available
% space. The environment \textsf{gather} of package \xpackage{amsmath}
% is supported. Also the environments \textsf{equation} and
% \textsf{displaymath} are redefined using \textsf{gather}
% and its starred version.
% \end{abstract}
%
% \tableofcontents
%
% \makeatletter
% \def\env#1{^^A
%    \textsf{\@env#1*\@nil}^^A
% }%
% \def\@env#1*#2\@nil{^^A
%   #1^^A
%   \ifx\\#2\\^^A
%     \expandafter\@gobble
%   \else
%     \textasteriskcentered
%     \expandafter\@firstofone
%   \fi
%   {\@env#2\@nil}^^A
% }
% \makeatother
%
% \section{Documentation}
%
% Sometimes an equation is just a little to large to fit in the
% line. And breaking the equation across lines might be worse
% than downscaling the equation. This package implements this
% for the environments \env{gather} and \env{gather*} of
% package \xpackage{amsmath}. That package already measures
% the equations and simplifies the implementation of \xpackage{resizegather}
% that only needs to hook into \xpackage{amsmath}'s code to add
% the resizing feature.
%
% Resized equations are recorded in the \xext{log} file
% for small exceeds (default setting is smaller than five percent).
% Otherwise a warning is given.
%
% Also environments \env{equation} and \env{displaymath}
% are supported by redefining them using \env{gather}
% and \env{gather*}.
%
% \cs{[} and \cs{]} are not supported, because these macros
% are not in environment form that is required for
% \xpackage{amsmath}. The environment body is collected
% first to be able to process the body twice for measuring
% first.
%
% Also the environments using alignments are not supported.
% If a single equation line would be resized, the alignment
% would get lost. And resizing all equations of the alignment
% does not seem appropriate either.
%
% \subsection{Options}
%
% \begin{description}
% \item[\xoption{warningthreshold}:]
%   Print a warning if the original equation line exceeds
%   its available width by the given fraction.
%   Default is |0.05|: A warning is given if the equation
%   is too large by five percent.
%   Otherwise the exceed is recorded in the \xext{log} file
%   only.
% \end{description}
% The next options are boolean options. They are enabled
% by value |true| or if no value is given. They are switched
% off by value |false|.
% \begin{description}
% \item[\xoption{enable}:] The resize feature is active (default).
% \item[\xoption{disable}:] The complementary option for \xoption{enable},
%  added for convenience: |disable| (or |disable=true|) is the same
%  as |enable=false|.
% \item[\xoption{equations}:]
%   \LaTeX\ environments \textsf{equation} and \textsf{displaymath}
%   environments are redefined. These equations
%   are now using environment \env{gather} and
%   \env{gather*}. This is the default.
% \end{description}
% The following table shows additional options if you
% want to have finer control for the redefined
% environments:
% \begin{quote}
% \def\unchanged{\textit{unchanged}}
% \def\notprovided{\textit{not provided}}
% \begin{tabular}{l|ll}
% &\multicolumn{2}{c}{Environments}\\
% Option & \env{equation} & \env{displaymath}\\
% \hline
% \xoption{equations} & \env{gather} & \env{gather*}\\
% \xoption{equation} & \env{gather} & \unchanged\\
% \xoption{displaymath} & \unchanged & \env{gather*}\\
% \end{tabular}
% \end{quote}
% If such an option is switched off, the original meaning
% of the affected environments is restored.
%
% Options are evaluated in the following order:
% \begin{enumerate}
% \item
%  Configuration file \xfile{resizegather.cfg} using \cs{resizegathersetup}
%  if the file exists.
%  \item
%  Package options given for \cs{usepackage}.
%  \item
%  Later calls of \cs{resizegathersetup}.
% \end{enumerate}
% \begin{declcs}{resizegathersetup}\M{option list}
% \end{declcs}
% The options are key value options. Boolean options are enabled by
% default (without value) or by using the explicit value \texttt{true}.
% Value \texttt{false} disable the option.
%
% \subsection{Options for packages \xpackage{amsmath} or \xpackage{graphics}}
%
% The package loads the package \xpackage{amsmath} because is internally
% measures the equations first. Thus this package hooks into this code
% in order to resize the equations if they are too large.
% The resizing itself is done by \cs{resizebox} of package \xpackage{graphics}.
% If you need special options for these packages, just load them first or
% use global options when appropriate. Example:
% \begin{quote}
%\begin{verbatim}
%\usepackage[dvipdfm]{graphicx}% or graphics
%\usepackage[fleqn]{amsmath}
%\usepackage{resizegather}
%\end{verbatim}
%\end{quote}
%
% \StopEventually{
% }
%
% \section{Implementation}
%    \begin{macrocode}
%<*package>
%    \end{macrocode}
%    Reload check, especially if the package is not used with \LaTeX.
%    \begin{macrocode}
\begingroup\catcode61\catcode48\catcode32=10\relax%
  \catcode13=5 % ^^M
  \endlinechar=13 %
  \catcode35=6 % #
  \catcode39=12 % '
  \catcode44=12 % ,
  \catcode45=12 % -
  \catcode46=12 % .
  \catcode58=12 % :
  \catcode64=11 % @
  \catcode123=1 % {
  \catcode125=2 % }
  \expandafter\let\expandafter\x\csname ver@resizegather.sty\endcsname
  \ifx\x\relax % plain-TeX, first loading
  \else
    \def\empty{}%
    \ifx\x\empty % LaTeX, first loading,
      % variable is initialized, but \ProvidesPackage not yet seen
    \else
      \expandafter\ifx\csname PackageInfo\endcsname\relax
        \def\x#1#2{%
          \immediate\write-1{Package #1 Info: #2.}%
        }%
      \else
        \def\x#1#2{\PackageInfo{#1}{#2, stopped}}%
      \fi
      \x{resizegather}{The package is already loaded}%
      \aftergroup\endinput
    \fi
  \fi
\endgroup%
%    \end{macrocode}
%    Package identification:
%    \begin{macrocode}
\begingroup\catcode61\catcode48\catcode32=10\relax%
  \catcode13=5 % ^^M
  \endlinechar=13 %
  \catcode35=6 % #
  \catcode39=12 % '
  \catcode40=12 % (
  \catcode41=12 % )
  \catcode44=12 % ,
  \catcode45=12 % -
  \catcode46=12 % .
  \catcode47=12 % /
  \catcode58=12 % :
  \catcode64=11 % @
  \catcode91=12 % [
  \catcode93=12 % ]
  \catcode123=1 % {
  \catcode125=2 % }
  \expandafter\ifx\csname ProvidesPackage\endcsname\relax
    \def\x#1#2#3[#4]{\endgroup
      \immediate\write-1{Package: #3 #4}%
      \xdef#1{#4}%
    }%
  \else
    \def\x#1#2[#3]{\endgroup
      #2[{#3}]%
      \ifx#1\@undefined
        \xdef#1{#3}%
      \fi
      \ifx#1\relax
        \xdef#1{#3}%
      \fi
    }%
  \fi
\expandafter\x\csname ver@resizegather.sty\endcsname
\ProvidesPackage{resizegather}%
  [2016/05/16 v1.3 Resize overly large equations (HO)]%
%    \end{macrocode}
%
%    \begin{macrocode}
\begingroup\catcode61\catcode48\catcode32=10\relax%
  \catcode13=5 % ^^M
  \endlinechar=13 %
  \catcode123=1 % {
  \catcode125=2 % }
  \catcode64=11 % @
  \def\x{\endgroup
    \expandafter\edef\csname ResizeGather@AtEnd\endcsname{%
      \endlinechar=\the\endlinechar\relax
      \catcode13=\the\catcode13\relax
      \catcode32=\the\catcode32\relax
      \catcode35=\the\catcode35\relax
      \catcode61=\the\catcode61\relax
      \catcode64=\the\catcode64\relax
      \catcode123=\the\catcode123\relax
      \catcode125=\the\catcode125\relax
    }%
  }%
\x\catcode61\catcode48\catcode32=10\relax%
\catcode13=5 % ^^M
\endlinechar=13 %
\catcode35=6 % #
\catcode64=11 % @
\catcode123=1 % {
\catcode125=2 % }
\def\TMP@EnsureCode#1#2{%
  \edef\ResizeGather@AtEnd{%
    \ResizeGather@AtEnd
    \catcode#1=\the\catcode#1\relax
  }%
  \catcode#1=#2\relax
}
\TMP@EnsureCode{10}{12}% ^^J
\TMP@EnsureCode{33}{12}% !
\TMP@EnsureCode{36}{3}% $
\TMP@EnsureCode{38}{4}% &
\TMP@EnsureCode{39}{12}% '
\TMP@EnsureCode{40}{12}% (
\TMP@EnsureCode{41}{12}% )
\TMP@EnsureCode{42}{12}% *
\TMP@EnsureCode{43}{12}% +
\TMP@EnsureCode{44}{12}% ,
\TMP@EnsureCode{45}{12}% -
\TMP@EnsureCode{46}{12}% .
\TMP@EnsureCode{47}{12}% /
\TMP@EnsureCode{58}{12}% :
\TMP@EnsureCode{59}{12}% ;
\TMP@EnsureCode{60}{12}% <
\TMP@EnsureCode{62}{12}% >
\TMP@EnsureCode{63}{12}% ?
\TMP@EnsureCode{91}{12}% [
\TMP@EnsureCode{93}{12}% ]
\TMP@EnsureCode{94}{7}% ^ (superscript)
\TMP@EnsureCode{96}{12}% `
\edef\ResizeGather@AtEnd{\ResizeGather@AtEnd\noexpand\endinput}
%    \end{macrocode}
%
%    \begin{macrocode}
\RequirePackage{kvoptions}[2009/12/04]
\SetupKeyvalOptions{%
  family=resizegather,%
  prefix=ResizeGather@,%
}
%    \end{macrocode}
%    \begin{macrocode}
\@for\ResizeGather@option:=%
  centertags,%
  tbtags,%
  sumlimits,%
  nosumlimits,%
  intlimits,%
  nointlimits,%
  nonamelimits,%
  leqno,%
  reqno,%
  fleqn%
\do{%
  \edef\ResizeGather@temp{%
    \noexpand\DeclareVoidOption{\ResizeGather@option}{%
      \noexpand\PassOptionsToPackage{amsmath}{\ResizeGather@option}%
    }%
    \noexpand\AtEndOfPackage{%
      \noexpand\DisableKeyvalOption[%
        action=error,%
        package=resizegather,%
      ]{resizegather}{\ResizeGather@option}%
    }%
  }%
  \ResizeGather@temp
}
\@for\ResizeGather@option:=%
  draft,%
  final,%
  hiderotate,%
  hidescale,%
  hiresbb,%
  demo,%
  dvips,xdvi,dvipdf,dvipdfm,dvipdfmx,pdftex,dvipsone,%
  dviwindo,emtex,dviwin,pctexps,pctexwin,pctexhp,pctex32,%
  truetex,tcidvi,vtex,oztex,textures,xetex%
\do{%
  \edef\ResizeGather@temp{%
    \noexpand\DeclareVoidOption{\ResizeGather@option}{%
      \noexpand\PassOptionsToPackage{graphics}{\ResizeGather@option}%
    }%
    \noexpand\AtEndOfPackage{%
      \noexpand\DisableKeyvalOption[%
        action=error,%
        package=resizegather,%
      ]{resizegather}{\ResizeGather@option}%
    }%
  }%
  \ResizeGather@temp
}
%    \end{macrocode}
%
%    \begin{macrocode}
\DeclareBoolOption[true]{enable}
\DeclareComplementaryOption{disable}{enable}
\DeclareStringOption[.05]{warningthreshold}
\newif\ifResizeGather@NeedInit
\DeclareBoolOption[true]{equations}
\DeclareBoolOption[true]{equation}
\DeclareBoolOption[true]{displaymath}
\AddToKeyvalOption*{equations}{%
  \ResizeGather@NeedInittrue
  \ifResizeGather@equations
    \ResizeGather@equationtrue
    \ResizeGather@displaymathtrue
  \else
    \ResizeGather@equationfalse
    \ResizeGather@displaymathfalse
  \fi
}
\AddToKeyvalOption*{equation}{%
  \ResizeGather@NeedInittrue
}
\AddToKeyvalOption*{displaymath}{%
  \ResizeGather@NeedInittrue
}
%    \end{macrocode}
%
%    \begin{macro}{\resizegathersetup}
%    \begin{macrocode}
\newcommand*{\resizegathersetup}[1]{%
  \ResizeGather@NeedInitfalse
  \setkeys{resizegather}{#1}%
  \ifResizeGather@NeedInit
    \ResizeGather@init
  \fi
}
\let\ResizeGather@init\relax
%    \end{macrocode}
%    \end{macro}
%    \begin{macrocode}
\InputIfFileExists{resizegather.cfg}{}{}%
\ProcessKeyvalOptions*\relax
%    \end{macrocode}
%    \begin{macrocode}
\RequirePackage{amsmath}
\RequirePackage{graphics}
%    \end{macrocode}
%    \begin{macro}{\ResizeGather@redefine}
%    \begin{macrocode}
\def\ResizeGather@redefine#1#2#3#4#5{%
  \csname ifResizeGather@#1\endcsname
    \@ifundefined{ResizeGather@org@#2}{%
      \expandafter\let\csname ResizeGather@org@#2\expandafter\endcsname
                      \csname #2\endcsname
    }{}%
    \@ifundefined{ResizeGather@org@#3}{%
      \expandafter\let\csname ResizeGather@org@#3\expandafter\endcsname
                      \csname #3\endcsname
    }{}%
    \expandafter\edef\csname #2\endcsname{%
      \expandafter\noexpand\csname#4\endcsname
    }%
    \expandafter\edef\csname #3\endcsname{%
      \expandafter\noexpand\csname#5\endcsname
    }%
  \else
    \@ifundefined{ResizeGather@org@#2}{}{%
      \expandafter\let\csname #2\expandafter\endcsname
                      \csname ResizeGather@org@#2\endcsname
      \expandafter\let\csname #3\expandafter\endcsname
                      \csname ResizeGather@org@#3\endcsname
    }%
  \fi
}
%    \end{macrocode}
%    \end{macro}
%    \begin{macro}{\ResizeGather@init}
%    \begin{macrocode}
\def\ResizeGather@init{%
  \ResizeGather@redefine{equation}{equation}{endequation}%
    {gather}{endgather}%
  \ResizeGather@redefine{displaymath}{displaymath}{enddisplaymath}%
    {gather*}{endgather*}%
}
\ResizeGather@init
%    \end{macrocode}
%    \end{macro}
%
%    \begin{macro}{\ResizeGather@ResizeGather}
%    \begin{macrocode}
\def\ResizeGather@ResizeGather{%
  \ifResizeGather@enable
    \dimen@\displaywidth
    \if@fleqn
      \advance\dimen@-\@mathmargin
    \fi
    \ifdim\wdz@>\dimen@
      \begingroup
        \advance\dimen@ -\wdz@
        \dimen@ -\dimen@
        \ifdim\ResizeGather@warningthreshold\wdz@>\dimen@
          \expandafter\PackageInfo
        \else
          \expandafter\PackageWarning
        \fi
        {resizegather}{%
          Equation line \the\row@\space is too large %
          by \the\dimen@\MessageBreak
          in environment `\@currenvir'%
        }%
      \endgroup
      \setboxz@h to\dimen@{%
        \resizebox{\dimen@}{!}{\boxz@}%
        \hss
      }%
    \fi
  \fi
}
%    \end{macrocode}
%    \end{macro}
%    \begin{macro}{\calc@shift@gather}
%    \begin{macrocode}
\expandafter\def\expandafter\calc@shift@gather\expandafter{%
  \expandafter\ResizeGather@ResizeGather
  \calc@shift@gather
}
%    \end{macrocode}
%    \end{macro}
%    \begin{macro}{\ResizeGather@org@gmeasure@}
%    \begin{macrocode}
\let\ResizeGather@org@gmeasure@\gmeasure@
%    \end{macrocode}
%    \end{macro}
%    \begin{macro}{\gmeasure@}
%    \begin{macrocode}
\def\gmeasure@#1{%
  \ResizeGather@org@gmeasure@{#1}%
  \ifResizeGather@enable
    \ifdim\totwidth@>\displaywidth
      \totwidth@=\displaywidth
    \fi
  \fi
}
%    \end{macrocode}
%    \end{macro}
%
%    \begin{macrocode}
\ResizeGather@AtEnd%
%</package>
%    \end{macrocode}
%
% \section{Test}
%
% \subsection{Catcode checks for loading}
%
%    \begin{macrocode}
%<*test1>
%    \end{macrocode}
%    \begin{macrocode}
\catcode`\{=1 %
\catcode`\}=2 %
\catcode`\#=6 %
\catcode`\@=11 %
\expandafter\ifx\csname count@\endcsname\relax
  \countdef\count@=255 %
\fi
\expandafter\ifx\csname @gobble\endcsname\relax
  \long\def\@gobble#1{}%
\fi
\expandafter\ifx\csname @firstofone\endcsname\relax
  \long\def\@firstofone#1{#1}%
\fi
\expandafter\ifx\csname loop\endcsname\relax
  \expandafter\@firstofone
\else
  \expandafter\@gobble
\fi
{%
  \def\loop#1\repeat{%
    \def\body{#1}%
    \iterate
  }%
  \def\iterate{%
    \body
      \let\next\iterate
    \else
      \let\next\relax
    \fi
    \next
  }%
  \let\repeat=\fi
}%
\def\RestoreCatcodes{}
\count@=0 %
\loop
  \edef\RestoreCatcodes{%
    \RestoreCatcodes
    \catcode\the\count@=\the\catcode\count@\relax
  }%
\ifnum\count@<255 %
  \advance\count@ 1 %
\repeat

\def\RangeCatcodeInvalid#1#2{%
  \count@=#1\relax
  \loop
    \catcode\count@=15 %
  \ifnum\count@<#2\relax
    \advance\count@ 1 %
  \repeat
}
\def\RangeCatcodeCheck#1#2#3{%
  \count@=#1\relax
  \loop
    \ifnum#3=\catcode\count@
    \else
      \errmessage{%
        Character \the\count@\space
        with wrong catcode \the\catcode\count@\space
        instead of \number#3%
      }%
    \fi
  \ifnum\count@<#2\relax
    \advance\count@ 1 %
  \repeat
}
\def\space{ }
\expandafter\ifx\csname LoadCommand\endcsname\relax
  \def\LoadCommand{\input resizegather.sty\relax}%
\fi
\def\Test{%
  \RangeCatcodeInvalid{0}{47}%
  \RangeCatcodeInvalid{58}{64}%
  \RangeCatcodeInvalid{91}{96}%
  \RangeCatcodeInvalid{123}{255}%
  \catcode`\@=12 %
  \catcode`\\=0 %
  \catcode`\%=14 %
  \LoadCommand
  \RangeCatcodeCheck{0}{36}{15}%
  \RangeCatcodeCheck{37}{37}{14}%
  \RangeCatcodeCheck{38}{47}{15}%
  \RangeCatcodeCheck{48}{57}{12}%
  \RangeCatcodeCheck{58}{63}{15}%
  \RangeCatcodeCheck{64}{64}{12}%
  \RangeCatcodeCheck{65}{90}{11}%
  \RangeCatcodeCheck{91}{91}{15}%
  \RangeCatcodeCheck{92}{92}{0}%
  \RangeCatcodeCheck{93}{96}{15}%
  \RangeCatcodeCheck{97}{122}{11}%
  \RangeCatcodeCheck{123}{255}{15}%
  \RestoreCatcodes
}
\Test
\csname @@end\endcsname
\end
%    \end{macrocode}
%    \begin{macrocode}
%</test1>
%    \end{macrocode}
%
% \section{Installation}
%
% \subsection{Download}
%
% \paragraph{Package.} This package is available on
% CTAN\footnote{\url{https://ctan.org/pkg/resizegather}}:
% \begin{description}
% \item[\CTAN{macros/latex/contrib/oberdiek/resizegather.dtx}] The source file.
% \item[\CTAN{macros/latex/contrib/oberdiek/resizegather.pdf}] Documentation.
% \end{description}
%
%
% \paragraph{Bundle.} All the packages of the bundle `oberdiek'
% are also available in a TDS compliant ZIP archive. There
% the packages are already unpacked and the documentation files
% are generated. The files and directories obey the TDS standard.
% \begin{description}
% \item[\CTANinstall{install/macros/latex/contrib/oberdiek.tds.zip}]
% \end{description}
% \emph{TDS} refers to the standard ``A Directory Structure
% for \TeX\ Files'' (\CTAN{tds/tds.pdf}). Directories
% with \xfile{texmf} in their name are usually organized this way.
%
% \subsection{Bundle installation}
%
% \paragraph{Unpacking.} Unpack the \xfile{oberdiek.tds.zip} in the
% TDS tree (also known as \xfile{texmf} tree) of your choice.
% Example (linux):
% \begin{quote}
%   |unzip oberdiek.tds.zip -d ~/texmf|
% \end{quote}
%
% \paragraph{Script installation.}
% Check the directory \xfile{TDS:scripts/oberdiek/} for
% scripts that need further installation steps.
% Package \xpackage{attachfile2} comes with the Perl script
% \xfile{pdfatfi.pl} that should be installed in such a way
% that it can be called as \texttt{pdfatfi}.
% Example (linux):
% \begin{quote}
%   |chmod +x scripts/oberdiek/pdfatfi.pl|\\
%   |cp scripts/oberdiek/pdfatfi.pl /usr/local/bin/|
% \end{quote}
%
% \subsection{Package installation}
%
% \paragraph{Unpacking.} The \xfile{.dtx} file is a self-extracting
% \docstrip\ archive. The files are extracted by running the
% \xfile{.dtx} through \plainTeX:
% \begin{quote}
%   \verb|tex resizegather.dtx|
% \end{quote}
%
% \paragraph{TDS.} Now the different files must be moved into
% the different directories in your installation TDS tree
% (also known as \xfile{texmf} tree):
% \begin{quote}
% \def\t{^^A
% \begin{tabular}{@{}>{\ttfamily}l@{ $\rightarrow$ }>{\ttfamily}l@{}}
%   resizegather.sty & tex/latex/oberdiek/resizegather.sty\\
%   resizegather.pdf & doc/latex/oberdiek/resizegather.pdf\\
%   test/resizegather-test1.tex & doc/latex/oberdiek/test/resizegather-test1.tex\\
%   resizegather.dtx & source/latex/oberdiek/resizegather.dtx\\
% \end{tabular}^^A
% }^^A
% \sbox0{\t}^^A
% \ifdim\wd0>\linewidth
%   \begingroup
%     \advance\linewidth by\leftmargin
%     \advance\linewidth by\rightmargin
%   \edef\x{\endgroup
%     \def\noexpand\lw{\the\linewidth}^^A
%   }\x
%   \def\lwbox{^^A
%     \leavevmode
%     \hbox to \linewidth{^^A
%       \kern-\leftmargin\relax
%       \hss
%       \usebox0
%       \hss
%       \kern-\rightmargin\relax
%     }^^A
%   }^^A
%   \ifdim\wd0>\lw
%     \sbox0{\small\t}^^A
%     \ifdim\wd0>\linewidth
%       \ifdim\wd0>\lw
%         \sbox0{\footnotesize\t}^^A
%         \ifdim\wd0>\linewidth
%           \ifdim\wd0>\lw
%             \sbox0{\scriptsize\t}^^A
%             \ifdim\wd0>\linewidth
%               \ifdim\wd0>\lw
%                 \sbox0{\tiny\t}^^A
%                 \ifdim\wd0>\linewidth
%                   \lwbox
%                 \else
%                   \usebox0
%                 \fi
%               \else
%                 \lwbox
%               \fi
%             \else
%               \usebox0
%             \fi
%           \else
%             \lwbox
%           \fi
%         \else
%           \usebox0
%         \fi
%       \else
%         \lwbox
%       \fi
%     \else
%       \usebox0
%     \fi
%   \else
%     \lwbox
%   \fi
% \else
%   \usebox0
% \fi
% \end{quote}
% If you have a \xfile{docstrip.cfg} that configures and enables \docstrip's
% TDS installing feature, then some files can already be in the right
% place, see the documentation of \docstrip.
%
% \subsection{Refresh file name databases}
%
% If your \TeX~distribution
% (\teTeX, \mikTeX, \dots) relies on file name databases, you must refresh
% these. For example, \teTeX\ users run \verb|texhash| or
% \verb|mktexlsr|.
%
% \subsection{Some details for the interested}
%
% \paragraph{Attached source.}
%
% The PDF documentation on CTAN also includes the
% \xfile{.dtx} source file. It can be extracted by
% AcrobatReader 6 or higher. Another option is \textsf{pdftk},
% e.g. unpack the file into the current directory:
% \begin{quote}
%   \verb|pdftk resizegather.pdf unpack_files output .|
% \end{quote}
%
% \paragraph{Unpacking with \LaTeX.}
% The \xfile{.dtx} chooses its action depending on the format:
% \begin{description}
% \item[\plainTeX:] Run \docstrip\ and extract the files.
% \item[\LaTeX:] Generate the documentation.
% \end{description}
% If you insist on using \LaTeX\ for \docstrip\ (really,
% \docstrip\ does not need \LaTeX), then inform the autodetect routine
% about your intention:
% \begin{quote}
%   \verb|latex \let\install=y\input{resizegather.dtx}|
% \end{quote}
% Do not forget to quote the argument according to the demands
% of your shell.
%
% \paragraph{Generating the documentation.}
% You can use both the \xfile{.dtx} or the \xfile{.drv} to generate
% the documentation. The process can be configured by the
% configuration file \xfile{ltxdoc.cfg}. For instance, put this
% line into this file, if you want to have A4 as paper format:
% \begin{quote}
%   \verb|\PassOptionsToClass{a4paper}{article}|
% \end{quote}
% An example follows how to generate the
% documentation with pdf\LaTeX:
% \begin{quote}
%\begin{verbatim}
%pdflatex resizegather.dtx
%makeindex -s gind.ist resizegather.idx
%pdflatex resizegather.dtx
%makeindex -s gind.ist resizegather.idx
%pdflatex resizegather.dtx
%\end{verbatim}
% \end{quote}
%
% \section{Acknowledgement}
%
% \begin{description}
% \item[Dieter Jurzitza:]
% He wanted the resizing feature at the \TeX\ table
% in Karlsruhe of December 2009. Thus this package is a kind of
% Christmas present.
% \end{description}
%
% \begin{History}
%   \begin{Version}{2009/12/04 v1.0}
%   \item
%     The first version.
%   \end{Version}
%   \begin{Version}{2009/12/05 v1.1}
%   \item
%     Options \xoption{enable} and \xoption{disable} added.
%   \end{Version}
%   \begin{Version}{2010/03/01 v1.2}
%   \item
%     TDS location moved from `generic' to `latex'.
%   \end{Version}
%   \begin{Version}{2016/05/16 v1.3}
%   \item
%     Documentation updates.
%   \end{Version}
% \end{History}
%
% \PrintIndex
%
% \Finale
\endinput
|
% \end{quote}
% Do not forget to quote the argument according to the demands
% of your shell.
%
% \paragraph{Generating the documentation.}
% You can use both the \xfile{.dtx} or the \xfile{.drv} to generate
% the documentation. The process can be configured by the
% configuration file \xfile{ltxdoc.cfg}. For instance, put this
% line into this file, if you want to have A4 as paper format:
% \begin{quote}
%   \verb|\PassOptionsToClass{a4paper}{article}|
% \end{quote}
% An example follows how to generate the
% documentation with pdf\LaTeX:
% \begin{quote}
%\begin{verbatim}
%pdflatex resizegather.dtx
%makeindex -s gind.ist resizegather.idx
%pdflatex resizegather.dtx
%makeindex -s gind.ist resizegather.idx
%pdflatex resizegather.dtx
%\end{verbatim}
% \end{quote}
%
% \section{Acknowledgement}
%
% \begin{description}
% \item[Dieter Jurzitza:]
% He wanted the resizing feature at the \TeX\ table
% in Karlsruhe of December 2009. Thus this package is a kind of
% Christmas present.
% \end{description}
%
% \begin{History}
%   \begin{Version}{2009/12/04 v1.0}
%   \item
%     The first version.
%   \end{Version}
%   \begin{Version}{2009/12/05 v1.1}
%   \item
%     Options \xoption{enable} and \xoption{disable} added.
%   \end{Version}
%   \begin{Version}{2010/03/01 v1.2}
%   \item
%     TDS location moved from `generic' to `latex'.
%   \end{Version}
%   \begin{Version}{2016/05/16 v1.3}
%   \item
%     Documentation updates.
%   \end{Version}
% \end{History}
%
% \PrintIndex
%
% \Finale
\endinput
|
% \end{quote}
% Do not forget to quote the argument according to the demands
% of your shell.
%
% \paragraph{Generating the documentation.}
% You can use both the \xfile{.dtx} or the \xfile{.drv} to generate
% the documentation. The process can be configured by the
% configuration file \xfile{ltxdoc.cfg}. For instance, put this
% line into this file, if you want to have A4 as paper format:
% \begin{quote}
%   \verb|\PassOptionsToClass{a4paper}{article}|
% \end{quote}
% An example follows how to generate the
% documentation with pdf\LaTeX:
% \begin{quote}
%\begin{verbatim}
%pdflatex resizegather.dtx
%makeindex -s gind.ist resizegather.idx
%pdflatex resizegather.dtx
%makeindex -s gind.ist resizegather.idx
%pdflatex resizegather.dtx
%\end{verbatim}
% \end{quote}
%
% \section{Acknowledgement}
%
% \begin{description}
% \item[Dieter Jurzitza:]
% He wanted the resizing feature at the \TeX\ table
% in Karlsruhe of December 2009. Thus this package is a kind of
% Christmas present.
% \end{description}
%
% \begin{History}
%   \begin{Version}{2009/12/04 v1.0}
%   \item
%     The first version.
%   \end{Version}
%   \begin{Version}{2009/12/05 v1.1}
%   \item
%     Options \xoption{enable} and \xoption{disable} added.
%   \end{Version}
%   \begin{Version}{2010/03/01 v1.2}
%   \item
%     TDS location moved from `generic' to `latex'.
%   \end{Version}
%   \begin{Version}{2016/05/16 v1.3}
%   \item
%     Documentation updates.
%   \end{Version}
% \end{History}
%
% \PrintIndex
%
% \Finale
\endinput

%        (quote the arguments according to the demands of your shell)
%
% Documentation:
%    (a) If resizegather.drv is present:
%           latex resizegather.drv
%    (b) Without resizegather.drv:
%           latex resizegather.dtx; ...
%    The class ltxdoc loads the configuration file ltxdoc.cfg
%    if available. Here you can specify further options, e.g.
%    use A4 as paper format:
%       \PassOptionsToClass{a4paper}{article}
%
%    Programm calls to get the documentation (example):
%       pdflatex resizegather.dtx
%       makeindex -s gind.ist resizegather.idx
%       pdflatex resizegather.dtx
%       makeindex -s gind.ist resizegather.idx
%       pdflatex resizegather.dtx
%
% Installation:
%    TDS:tex/latex/oberdiek/resizegather.sty
%    TDS:doc/latex/oberdiek/resizegather.pdf
%    TDS:doc/latex/oberdiek/test/resizegather-test1.tex
%    TDS:source/latex/oberdiek/resizegather.dtx
%
%<*ignore>
\begingroup
  \catcode123=1 %
  \catcode125=2 %
  \def\x{LaTeX2e}%
\expandafter\endgroup
\ifcase 0\ifx\install y1\fi\expandafter
         \ifx\csname processbatchFile\endcsname\relax\else1\fi
         \ifx\fmtname\x\else 1\fi\relax
\else\csname fi\endcsname
%</ignore>
%<*install>
\input docstrip.tex
\Msg{************************************************************************}
\Msg{* Installation}
\Msg{* Package: resizegather 2016/05/16 v1.3 Resize overly large equations (HO)}
\Msg{************************************************************************}

\keepsilent
\askforoverwritefalse

\let\MetaPrefix\relax
\preamble

This is a generated file.

Project: resizegather
Version: 2016/05/16 v1.3

Copyright (C) 2009, 2010 by
   Heiko Oberdiek <heiko.oberdiek at googlemail.com>

This work may be distributed and/or modified under the
conditions of the LaTeX Project Public License, either
version 1.3c of this license or (at your option) any later
version. This version of this license is in
   http://www.latex-project.org/lppl/lppl-1-3c.txt
and the latest version of this license is in
   http://www.latex-project.org/lppl.txt
and version 1.3 or later is part of all distributions of
LaTeX version 2005/12/01 or later.

This work has the LPPL maintenance status "maintained".

This Current Maintainer of this work is Heiko Oberdiek.

This work consists of the main source file resizegather.dtx
and the derived files
   resizegather.sty, resizegather.pdf, resizegather.ins, resizegather.drv,
   resizegather-test1.tex.

\endpreamble
\let\MetaPrefix\DoubleperCent

\generate{%
  \file{resizegather.ins}{\from{resizegather.dtx}{install}}%
  \file{resizegather.drv}{\from{resizegather.dtx}{driver}}%
  \usedir{tex/latex/oberdiek}%
  \file{resizegather.sty}{\from{resizegather.dtx}{package}}%
%  \usedir{doc/latex/oberdiek/test}%
%  \file{resizegather-test1.tex}{\from{resizegather.dtx}{test1}}%
  \nopreamble
  \nopostamble
%  \usedir{source/latex/oberdiek/catalogue}%
%  \file{resizegather.xml}{\from{resizegather.dtx}{catalogue}}%
}

\catcode32=13\relax% active space
\let =\space%
\Msg{************************************************************************}
\Msg{*}
\Msg{* To finish the installation you have to move the following}
\Msg{* file into a directory searched by TeX:}
\Msg{*}
\Msg{*     resizegather.sty}
\Msg{*}
\Msg{* To produce the documentation run the file `resizegather.drv'}
\Msg{* through LaTeX.}
\Msg{*}
\Msg{* Happy TeXing!}
\Msg{*}
\Msg{************************************************************************}

\endbatchfile
%</install>
%<*ignore>
\fi
%</ignore>
%<*driver>
\NeedsTeXFormat{LaTeX2e}
\ProvidesFile{resizegather.drv}%
  [2016/05/16 v1.3 Resize overly large equations (HO)]%
\documentclass{ltxdoc}
\usepackage{holtxdoc}[2011/11/22]
\usepackage{ifluatex}
\ifluatex
\else
  \usepackage[T1]{fontenc}%
  \usepackage{textcomp}%
  \usepackage{lmodern}%
\fi
\begin{document}
  \DocInput{resizegather.dtx}%
\end{document}
%</driver>
% \fi
%
%
% \CharacterTable
%  {Upper-case    \A\B\C\D\E\F\G\H\I\J\K\L\M\N\O\P\Q\R\S\T\U\V\W\X\Y\Z
%   Lower-case    \a\b\c\d\e\f\g\h\i\j\k\l\m\n\o\p\q\r\s\t\u\v\w\x\y\z
%   Digits        \0\1\2\3\4\5\6\7\8\9
%   Exclamation   \!     Double quote  \"     Hash (number) \#
%   Dollar        \$     Percent       \%     Ampersand     \&
%   Acute accent  \'     Left paren    \(     Right paren   \)
%   Asterisk      \*     Plus          \+     Comma         \,
%   Minus         \-     Point         \.     Solidus       \/
%   Colon         \:     Semicolon     \;     Less than     \<
%   Equals        \=     Greater than  \>     Question mark \?
%   Commercial at \@     Left bracket  \[     Backslash     \\
%   Right bracket \]     Circumflex    \^     Underscore    \_
%   Grave accent  \`     Left brace    \{     Vertical bar  \|
%   Right brace   \}     Tilde         \~}
%
% \GetFileInfo{resizegather.drv}
%
% \title{The \xpackage{resizegather} package}
% \date{2016/05/16 v1.3}
% \author{Heiko Oberdiek\thanks
% {Please report any issues at https://github.com/ho-tex/oberdiek/issues}\\
% \xemail{heiko.oberdiek at googlemail.com}}
%
% \maketitle
%
% \begin{abstract}
% Equations that are too large are resized to fit the available
% space. The environment \textsf{gather} of package \xpackage{amsmath}
% is supported. Also the environments \textsf{equation} and
% \textsf{displaymath} are redefined using \textsf{gather}
% and its starred version.
% \end{abstract}
%
% \tableofcontents
%
% \makeatletter
% \def\env#1{^^A
%    \textsf{\@env#1*\@nil}^^A
% }%
% \def\@env#1*#2\@nil{^^A
%   #1^^A
%   \ifx\\#2\\^^A
%     \expandafter\@gobble
%   \else
%     \textasteriskcentered
%     \expandafter\@firstofone
%   \fi
%   {\@env#2\@nil}^^A
% }
% \makeatother
%
% \section{Documentation}
%
% Sometimes an equation is just a little to large to fit in the
% line. And breaking the equation across lines might be worse
% than downscaling the equation. This package implements this
% for the environments \env{gather} and \env{gather*} of
% package \xpackage{amsmath}. That package already measures
% the equations and simplifies the implementation of \xpackage{resizegather}
% that only needs to hook into \xpackage{amsmath}'s code to add
% the resizing feature.
%
% Resized equations are recorded in the \xext{log} file
% for small exceeds (default setting is smaller than five percent).
% Otherwise a warning is given.
%
% Also environments \env{equation} and \env{displaymath}
% are supported by redefining them using \env{gather}
% and \env{gather*}.
%
% \cs{[} and \cs{]} are not supported, because these macros
% are not in environment form that is required for
% \xpackage{amsmath}. The environment body is collected
% first to be able to process the body twice for measuring
% first.
%
% Also the environments using alignments are not supported.
% If a single equation line would be resized, the alignment
% would get lost. And resizing all equations of the alignment
% does not seem appropriate either.
%
% \subsection{Options}
%
% \begin{description}
% \item[\xoption{warningthreshold}:]
%   Print a warning if the original equation line exceeds
%   its available width by the given fraction.
%   Default is |0.05|: A warning is given if the equation
%   is too large by five percent.
%   Otherwise the exceed is recorded in the \xext{log} file
%   only.
% \end{description}
% The next options are boolean options. They are enabled
% by value |true| or if no value is given. They are switched
% off by value |false|.
% \begin{description}
% \item[\xoption{enable}:] The resize feature is active (default).
% \item[\xoption{disable}:] The complementary option for \xoption{enable},
%  added for convenience: |disable| (or |disable=true|) is the same
%  as |enable=false|.
% \item[\xoption{equations}:]
%   \LaTeX\ environments \textsf{equation} and \textsf{displaymath}
%   environments are redefined. These equations
%   are now using environment \env{gather} and
%   \env{gather*}. This is the default.
% \end{description}
% The following table shows additional options if you
% want to have finer control for the redefined
% environments:
% \begin{quote}
% \def\unchanged{\textit{unchanged}}
% \def\notprovided{\textit{not provided}}
% \begin{tabular}{l|ll}
% &\multicolumn{2}{c}{Environments}\\
% Option & \env{equation} & \env{displaymath}\\
% \hline
% \xoption{equations} & \env{gather} & \env{gather*}\\
% \xoption{equation} & \env{gather} & \unchanged\\
% \xoption{displaymath} & \unchanged & \env{gather*}\\
% \end{tabular}
% \end{quote}
% If such an option is switched off, the original meaning
% of the affected environments is restored.
%
% Options are evaluated in the following order:
% \begin{enumerate}
% \item
%  Configuration file \xfile{resizegather.cfg} using \cs{resizegathersetup}
%  if the file exists.
%  \item
%  Package options given for \cs{usepackage}.
%  \item
%  Later calls of \cs{resizegathersetup}.
% \end{enumerate}
% \begin{declcs}{resizegathersetup}\M{option list}
% \end{declcs}
% The options are key value options. Boolean options are enabled by
% default (without value) or by using the explicit value \texttt{true}.
% Value \texttt{false} disable the option.
%
% \subsection{Options for packages \xpackage{amsmath} or \xpackage{graphics}}
%
% The package loads the package \xpackage{amsmath} because is internally
% measures the equations first. Thus this package hooks into this code
% in order to resize the equations if they are too large.
% The resizing itself is done by \cs{resizebox} of package \xpackage{graphics}.
% If you need special options for these packages, just load them first or
% use global options when appropriate. Example:
% \begin{quote}
%\begin{verbatim}
%\usepackage[dvipdfm]{graphicx}% or graphics
%\usepackage[fleqn]{amsmath}
%\usepackage{resizegather}
%\end{verbatim}
%\end{quote}
%
% \StopEventually{
% }
%
% \section{Implementation}
%    \begin{macrocode}
%<*package>
%    \end{macrocode}
%    Reload check, especially if the package is not used with \LaTeX.
%    \begin{macrocode}
\begingroup\catcode61\catcode48\catcode32=10\relax%
  \catcode13=5 % ^^M
  \endlinechar=13 %
  \catcode35=6 % #
  \catcode39=12 % '
  \catcode44=12 % ,
  \catcode45=12 % -
  \catcode46=12 % .
  \catcode58=12 % :
  \catcode64=11 % @
  \catcode123=1 % {
  \catcode125=2 % }
  \expandafter\let\expandafter\x\csname ver@resizegather.sty\endcsname
  \ifx\x\relax % plain-TeX, first loading
  \else
    \def\empty{}%
    \ifx\x\empty % LaTeX, first loading,
      % variable is initialized, but \ProvidesPackage not yet seen
    \else
      \expandafter\ifx\csname PackageInfo\endcsname\relax
        \def\x#1#2{%
          \immediate\write-1{Package #1 Info: #2.}%
        }%
      \else
        \def\x#1#2{\PackageInfo{#1}{#2, stopped}}%
      \fi
      \x{resizegather}{The package is already loaded}%
      \aftergroup\endinput
    \fi
  \fi
\endgroup%
%    \end{macrocode}
%    Package identification:
%    \begin{macrocode}
\begingroup\catcode61\catcode48\catcode32=10\relax%
  \catcode13=5 % ^^M
  \endlinechar=13 %
  \catcode35=6 % #
  \catcode39=12 % '
  \catcode40=12 % (
  \catcode41=12 % )
  \catcode44=12 % ,
  \catcode45=12 % -
  \catcode46=12 % .
  \catcode47=12 % /
  \catcode58=12 % :
  \catcode64=11 % @
  \catcode91=12 % [
  \catcode93=12 % ]
  \catcode123=1 % {
  \catcode125=2 % }
  \expandafter\ifx\csname ProvidesPackage\endcsname\relax
    \def\x#1#2#3[#4]{\endgroup
      \immediate\write-1{Package: #3 #4}%
      \xdef#1{#4}%
    }%
  \else
    \def\x#1#2[#3]{\endgroup
      #2[{#3}]%
      \ifx#1\@undefined
        \xdef#1{#3}%
      \fi
      \ifx#1\relax
        \xdef#1{#3}%
      \fi
    }%
  \fi
\expandafter\x\csname ver@resizegather.sty\endcsname
\ProvidesPackage{resizegather}%
  [2016/05/16 v1.3 Resize overly large equations (HO)]%
%    \end{macrocode}
%
%    \begin{macrocode}
\begingroup\catcode61\catcode48\catcode32=10\relax%
  \catcode13=5 % ^^M
  \endlinechar=13 %
  \catcode123=1 % {
  \catcode125=2 % }
  \catcode64=11 % @
  \def\x{\endgroup
    \expandafter\edef\csname ResizeGather@AtEnd\endcsname{%
      \endlinechar=\the\endlinechar\relax
      \catcode13=\the\catcode13\relax
      \catcode32=\the\catcode32\relax
      \catcode35=\the\catcode35\relax
      \catcode61=\the\catcode61\relax
      \catcode64=\the\catcode64\relax
      \catcode123=\the\catcode123\relax
      \catcode125=\the\catcode125\relax
    }%
  }%
\x\catcode61\catcode48\catcode32=10\relax%
\catcode13=5 % ^^M
\endlinechar=13 %
\catcode35=6 % #
\catcode64=11 % @
\catcode123=1 % {
\catcode125=2 % }
\def\TMP@EnsureCode#1#2{%
  \edef\ResizeGather@AtEnd{%
    \ResizeGather@AtEnd
    \catcode#1=\the\catcode#1\relax
  }%
  \catcode#1=#2\relax
}
\TMP@EnsureCode{10}{12}% ^^J
\TMP@EnsureCode{33}{12}% !
\TMP@EnsureCode{36}{3}% $
\TMP@EnsureCode{38}{4}% &
\TMP@EnsureCode{39}{12}% '
\TMP@EnsureCode{40}{12}% (
\TMP@EnsureCode{41}{12}% )
\TMP@EnsureCode{42}{12}% *
\TMP@EnsureCode{43}{12}% +
\TMP@EnsureCode{44}{12}% ,
\TMP@EnsureCode{45}{12}% -
\TMP@EnsureCode{46}{12}% .
\TMP@EnsureCode{47}{12}% /
\TMP@EnsureCode{58}{12}% :
\TMP@EnsureCode{59}{12}% ;
\TMP@EnsureCode{60}{12}% <
\TMP@EnsureCode{62}{12}% >
\TMP@EnsureCode{63}{12}% ?
\TMP@EnsureCode{91}{12}% [
\TMP@EnsureCode{93}{12}% ]
\TMP@EnsureCode{94}{7}% ^ (superscript)
\TMP@EnsureCode{96}{12}% `
\edef\ResizeGather@AtEnd{\ResizeGather@AtEnd\noexpand\endinput}
%    \end{macrocode}
%
%    \begin{macrocode}
\RequirePackage{kvoptions}[2009/12/04]
\SetupKeyvalOptions{%
  family=resizegather,%
  prefix=ResizeGather@,%
}
%    \end{macrocode}
%    \begin{macrocode}
\@for\ResizeGather@option:=%
  centertags,%
  tbtags,%
  sumlimits,%
  nosumlimits,%
  intlimits,%
  nointlimits,%
  nonamelimits,%
  leqno,%
  reqno,%
  fleqn%
\do{%
  \edef\ResizeGather@temp{%
    \noexpand\DeclareVoidOption{\ResizeGather@option}{%
      \noexpand\PassOptionsToPackage{amsmath}{\ResizeGather@option}%
    }%
    \noexpand\AtEndOfPackage{%
      \noexpand\DisableKeyvalOption[%
        action=error,%
        package=resizegather,%
      ]{resizegather}{\ResizeGather@option}%
    }%
  }%
  \ResizeGather@temp
}
\@for\ResizeGather@option:=%
  draft,%
  final,%
  hiderotate,%
  hidescale,%
  hiresbb,%
  demo,%
  dvips,xdvi,dvipdf,dvipdfm,dvipdfmx,pdftex,dvipsone,%
  dviwindo,emtex,dviwin,pctexps,pctexwin,pctexhp,pctex32,%
  truetex,tcidvi,vtex,oztex,textures,xetex%
\do{%
  \edef\ResizeGather@temp{%
    \noexpand\DeclareVoidOption{\ResizeGather@option}{%
      \noexpand\PassOptionsToPackage{graphics}{\ResizeGather@option}%
    }%
    \noexpand\AtEndOfPackage{%
      \noexpand\DisableKeyvalOption[%
        action=error,%
        package=resizegather,%
      ]{resizegather}{\ResizeGather@option}%
    }%
  }%
  \ResizeGather@temp
}
%    \end{macrocode}
%
%    \begin{macrocode}
\DeclareBoolOption[true]{enable}
\DeclareComplementaryOption{disable}{enable}
\DeclareStringOption[.05]{warningthreshold}
\newif\ifResizeGather@NeedInit
\DeclareBoolOption[true]{equations}
\DeclareBoolOption[true]{equation}
\DeclareBoolOption[true]{displaymath}
\AddToKeyvalOption*{equations}{%
  \ResizeGather@NeedInittrue
  \ifResizeGather@equations
    \ResizeGather@equationtrue
    \ResizeGather@displaymathtrue
  \else
    \ResizeGather@equationfalse
    \ResizeGather@displaymathfalse
  \fi
}
\AddToKeyvalOption*{equation}{%
  \ResizeGather@NeedInittrue
}
\AddToKeyvalOption*{displaymath}{%
  \ResizeGather@NeedInittrue
}
%    \end{macrocode}
%
%    \begin{macro}{\resizegathersetup}
%    \begin{macrocode}
\newcommand*{\resizegathersetup}[1]{%
  \ResizeGather@NeedInitfalse
  \setkeys{resizegather}{#1}%
  \ifResizeGather@NeedInit
    \ResizeGather@init
  \fi
}
\let\ResizeGather@init\relax
%    \end{macrocode}
%    \end{macro}
%    \begin{macrocode}
\InputIfFileExists{resizegather.cfg}{}{}%
\ProcessKeyvalOptions*\relax
%    \end{macrocode}
%    \begin{macrocode}
\RequirePackage{amsmath}
\RequirePackage{graphics}
%    \end{macrocode}
%    \begin{macro}{\ResizeGather@redefine}
%    \begin{macrocode}
\def\ResizeGather@redefine#1#2#3#4#5{%
  \csname ifResizeGather@#1\endcsname
    \@ifundefined{ResizeGather@org@#2}{%
      \expandafter\let\csname ResizeGather@org@#2\expandafter\endcsname
                      \csname #2\endcsname
    }{}%
    \@ifundefined{ResizeGather@org@#3}{%
      \expandafter\let\csname ResizeGather@org@#3\expandafter\endcsname
                      \csname #3\endcsname
    }{}%
    \expandafter\edef\csname #2\endcsname{%
      \expandafter\noexpand\csname#4\endcsname
    }%
    \expandafter\edef\csname #3\endcsname{%
      \expandafter\noexpand\csname#5\endcsname
    }%
  \else
    \@ifundefined{ResizeGather@org@#2}{}{%
      \expandafter\let\csname #2\expandafter\endcsname
                      \csname ResizeGather@org@#2\endcsname
      \expandafter\let\csname #3\expandafter\endcsname
                      \csname ResizeGather@org@#3\endcsname
    }%
  \fi
}
%    \end{macrocode}
%    \end{macro}
%    \begin{macro}{\ResizeGather@init}
%    \begin{macrocode}
\def\ResizeGather@init{%
  \ResizeGather@redefine{equation}{equation}{endequation}%
    {gather}{endgather}%
  \ResizeGather@redefine{displaymath}{displaymath}{enddisplaymath}%
    {gather*}{endgather*}%
}
\ResizeGather@init
%    \end{macrocode}
%    \end{macro}
%
%    \begin{macro}{\ResizeGather@ResizeGather}
%    \begin{macrocode}
\def\ResizeGather@ResizeGather{%
  \ifResizeGather@enable
    \dimen@\displaywidth
    \if@fleqn
      \advance\dimen@-\@mathmargin
    \fi
    \ifdim\wdz@>\dimen@
      \begingroup
        \advance\dimen@ -\wdz@
        \dimen@ -\dimen@
        \ifdim\ResizeGather@warningthreshold\wdz@>\dimen@
          \expandafter\PackageInfo
        \else
          \expandafter\PackageWarning
        \fi
        {resizegather}{%
          Equation line \the\row@\space is too large %
          by \the\dimen@\MessageBreak
          in environment `\@currenvir'%
        }%
      \endgroup
      \setboxz@h to\dimen@{%
        \resizebox{\dimen@}{!}{\boxz@}%
        \hss
      }%
    \fi
  \fi
}
%    \end{macrocode}
%    \end{macro}
%    \begin{macro}{\calc@shift@gather}
%    \begin{macrocode}
\expandafter\def\expandafter\calc@shift@gather\expandafter{%
  \expandafter\ResizeGather@ResizeGather
  \calc@shift@gather
}
%    \end{macrocode}
%    \end{macro}
%    \begin{macro}{\ResizeGather@org@gmeasure@}
%    \begin{macrocode}
\let\ResizeGather@org@gmeasure@\gmeasure@
%    \end{macrocode}
%    \end{macro}
%    \begin{macro}{\gmeasure@}
%    \begin{macrocode}
\def\gmeasure@#1{%
  \ResizeGather@org@gmeasure@{#1}%
  \ifResizeGather@enable
    \ifdim\totwidth@>\displaywidth
      \totwidth@=\displaywidth
    \fi
  \fi
}
%    \end{macrocode}
%    \end{macro}
%
%    \begin{macrocode}
\ResizeGather@AtEnd%
%</package>
%    \end{macrocode}
%
% \section{Test}
%
% \subsection{Catcode checks for loading}
%
%    \begin{macrocode}
%<*test1>
%    \end{macrocode}
%    \begin{macrocode}
\catcode`\{=1 %
\catcode`\}=2 %
\catcode`\#=6 %
\catcode`\@=11 %
\expandafter\ifx\csname count@\endcsname\relax
  \countdef\count@=255 %
\fi
\expandafter\ifx\csname @gobble\endcsname\relax
  \long\def\@gobble#1{}%
\fi
\expandafter\ifx\csname @firstofone\endcsname\relax
  \long\def\@firstofone#1{#1}%
\fi
\expandafter\ifx\csname loop\endcsname\relax
  \expandafter\@firstofone
\else
  \expandafter\@gobble
\fi
{%
  \def\loop#1\repeat{%
    \def\body{#1}%
    \iterate
  }%
  \def\iterate{%
    \body
      \let\next\iterate
    \else
      \let\next\relax
    \fi
    \next
  }%
  \let\repeat=\fi
}%
\def\RestoreCatcodes{}
\count@=0 %
\loop
  \edef\RestoreCatcodes{%
    \RestoreCatcodes
    \catcode\the\count@=\the\catcode\count@\relax
  }%
\ifnum\count@<255 %
  \advance\count@ 1 %
\repeat

\def\RangeCatcodeInvalid#1#2{%
  \count@=#1\relax
  \loop
    \catcode\count@=15 %
  \ifnum\count@<#2\relax
    \advance\count@ 1 %
  \repeat
}
\def\RangeCatcodeCheck#1#2#3{%
  \count@=#1\relax
  \loop
    \ifnum#3=\catcode\count@
    \else
      \errmessage{%
        Character \the\count@\space
        with wrong catcode \the\catcode\count@\space
        instead of \number#3%
      }%
    \fi
  \ifnum\count@<#2\relax
    \advance\count@ 1 %
  \repeat
}
\def\space{ }
\expandafter\ifx\csname LoadCommand\endcsname\relax
  \def\LoadCommand{\input resizegather.sty\relax}%
\fi
\def\Test{%
  \RangeCatcodeInvalid{0}{47}%
  \RangeCatcodeInvalid{58}{64}%
  \RangeCatcodeInvalid{91}{96}%
  \RangeCatcodeInvalid{123}{255}%
  \catcode`\@=12 %
  \catcode`\\=0 %
  \catcode`\%=14 %
  \LoadCommand
  \RangeCatcodeCheck{0}{36}{15}%
  \RangeCatcodeCheck{37}{37}{14}%
  \RangeCatcodeCheck{38}{47}{15}%
  \RangeCatcodeCheck{48}{57}{12}%
  \RangeCatcodeCheck{58}{63}{15}%
  \RangeCatcodeCheck{64}{64}{12}%
  \RangeCatcodeCheck{65}{90}{11}%
  \RangeCatcodeCheck{91}{91}{15}%
  \RangeCatcodeCheck{92}{92}{0}%
  \RangeCatcodeCheck{93}{96}{15}%
  \RangeCatcodeCheck{97}{122}{11}%
  \RangeCatcodeCheck{123}{255}{15}%
  \RestoreCatcodes
}
\Test
\csname @@end\endcsname
\end
%    \end{macrocode}
%    \begin{macrocode}
%</test1>
%    \end{macrocode}
%
% \section{Installation}
%
% \subsection{Download}
%
% \paragraph{Package.} This package is available on
% CTAN\footnote{\url{http://ctan.org/pkg/resizegather}}:
% \begin{description}
% \item[\CTAN{macros/latex/contrib/oberdiek/resizegather.dtx}] The source file.
% \item[\CTAN{macros/latex/contrib/oberdiek/resizegather.pdf}] Documentation.
% \end{description}
%
%
% \paragraph{Bundle.} All the packages of the bundle `oberdiek'
% are also available in a TDS compliant ZIP archive. There
% the packages are already unpacked and the documentation files
% are generated. The files and directories obey the TDS standard.
% \begin{description}
% \item[\CTAN{install/macros/latex/contrib/oberdiek.tds.zip}]
% \end{description}
% \emph{TDS} refers to the standard ``A Directory Structure
% for \TeX\ Files'' (\CTAN{tds/tds.pdf}). Directories
% with \xfile{texmf} in their name are usually organized this way.
%
% \subsection{Bundle installation}
%
% \paragraph{Unpacking.} Unpack the \xfile{oberdiek.tds.zip} in the
% TDS tree (also known as \xfile{texmf} tree) of your choice.
% Example (linux):
% \begin{quote}
%   |unzip oberdiek.tds.zip -d ~/texmf|
% \end{quote}
%
% \paragraph{Script installation.}
% Check the directory \xfile{TDS:scripts/oberdiek/} for
% scripts that need further installation steps.
% Package \xpackage{attachfile2} comes with the Perl script
% \xfile{pdfatfi.pl} that should be installed in such a way
% that it can be called as \texttt{pdfatfi}.
% Example (linux):
% \begin{quote}
%   |chmod +x scripts/oberdiek/pdfatfi.pl|\\
%   |cp scripts/oberdiek/pdfatfi.pl /usr/local/bin/|
% \end{quote}
%
% \subsection{Package installation}
%
% \paragraph{Unpacking.} The \xfile{.dtx} file is a self-extracting
% \docstrip\ archive. The files are extracted by running the
% \xfile{.dtx} through \plainTeX:
% \begin{quote}
%   \verb|tex resizegather.dtx|
% \end{quote}
%
% \paragraph{TDS.} Now the different files must be moved into
% the different directories in your installation TDS tree
% (also known as \xfile{texmf} tree):
% \begin{quote}
% \def\t{^^A
% \begin{tabular}{@{}>{\ttfamily}l@{ $\rightarrow$ }>{\ttfamily}l@{}}
%   resizegather.sty & tex/latex/oberdiek/resizegather.sty\\
%   resizegather.pdf & doc/latex/oberdiek/resizegather.pdf\\
%   test/resizegather-test1.tex & doc/latex/oberdiek/test/resizegather-test1.tex\\
%   resizegather.dtx & source/latex/oberdiek/resizegather.dtx\\
% \end{tabular}^^A
% }^^A
% \sbox0{\t}^^A
% \ifdim\wd0>\linewidth
%   \begingroup
%     \advance\linewidth by\leftmargin
%     \advance\linewidth by\rightmargin
%   \edef\x{\endgroup
%     \def\noexpand\lw{\the\linewidth}^^A
%   }\x
%   \def\lwbox{^^A
%     \leavevmode
%     \hbox to \linewidth{^^A
%       \kern-\leftmargin\relax
%       \hss
%       \usebox0
%       \hss
%       \kern-\rightmargin\relax
%     }^^A
%   }^^A
%   \ifdim\wd0>\lw
%     \sbox0{\small\t}^^A
%     \ifdim\wd0>\linewidth
%       \ifdim\wd0>\lw
%         \sbox0{\footnotesize\t}^^A
%         \ifdim\wd0>\linewidth
%           \ifdim\wd0>\lw
%             \sbox0{\scriptsize\t}^^A
%             \ifdim\wd0>\linewidth
%               \ifdim\wd0>\lw
%                 \sbox0{\tiny\t}^^A
%                 \ifdim\wd0>\linewidth
%                   \lwbox
%                 \else
%                   \usebox0
%                 \fi
%               \else
%                 \lwbox
%               \fi
%             \else
%               \usebox0
%             \fi
%           \else
%             \lwbox
%           \fi
%         \else
%           \usebox0
%         \fi
%       \else
%         \lwbox
%       \fi
%     \else
%       \usebox0
%     \fi
%   \else
%     \lwbox
%   \fi
% \else
%   \usebox0
% \fi
% \end{quote}
% If you have a \xfile{docstrip.cfg} that configures and enables \docstrip's
% TDS installing feature, then some files can already be in the right
% place, see the documentation of \docstrip.
%
% \subsection{Refresh file name databases}
%
% If your \TeX~distribution
% (\teTeX, \mikTeX, \dots) relies on file name databases, you must refresh
% these. For example, \teTeX\ users run \verb|texhash| or
% \verb|mktexlsr|.
%
% \subsection{Some details for the interested}
%
% \paragraph{Attached source.}
%
% The PDF documentation on CTAN also includes the
% \xfile{.dtx} source file. It can be extracted by
% AcrobatReader 6 or higher. Another option is \textsf{pdftk},
% e.g. unpack the file into the current directory:
% \begin{quote}
%   \verb|pdftk resizegather.pdf unpack_files output .|
% \end{quote}
%
% \paragraph{Unpacking with \LaTeX.}
% The \xfile{.dtx} chooses its action depending on the format:
% \begin{description}
% \item[\plainTeX:] Run \docstrip\ and extract the files.
% \item[\LaTeX:] Generate the documentation.
% \end{description}
% If you insist on using \LaTeX\ for \docstrip\ (really,
% \docstrip\ does not need \LaTeX), then inform the autodetect routine
% about your intention:
% \begin{quote}
%   \verb|latex \let\install=y% \iffalse meta-comment
%
% File: resizegather.dtx
% Version: 2016/05/16 v1.3
% Info: Resize overly large equations
%
% Copyright (C) 2009, 2010 by
%    Heiko Oberdiek <heiko.oberdiek at googlemail.com>
%    2016
%    https://github.com/ho-tex/oberdiek/issues
%
% This work may be distributed and/or modified under the
% conditions of the LaTeX Project Public License, either
% version 1.3c of this license or (at your option) any later
% version. This version of this license is in
%    http://www.latex-project.org/lppl/lppl-1-3c.txt
% and the latest version of this license is in
%    http://www.latex-project.org/lppl.txt
% and version 1.3 or later is part of all distributions of
% LaTeX version 2005/12/01 or later.
%
% This work has the LPPL maintenance status "maintained".
%
% This Current Maintainer of this work is Heiko Oberdiek.
%
% This work consists of the main source file resizegather.dtx
% and the derived files
%    resizegather.sty, resizegather.pdf, resizegather.ins, resizegather.drv,
%    resizegather-test1.tex.
%
% Distribution:
%    CTAN:macros/latex/contrib/oberdiek/resizegather.dtx
%    CTAN:macros/latex/contrib/oberdiek/resizegather.pdf
%
% Unpacking:
%    (a) If resizegather.ins is present:
%           tex resizegather.ins
%    (b) Without resizegather.ins:
%           tex resizegather.dtx
%    (c) If you insist on using LaTeX
%           latex \let\install=y% \iffalse meta-comment
%
% File: resizegather.dtx
% Version: 2016/05/16 v1.3
% Info: Resize overly large equations
%
% Copyright (C) 2009, 2010 by
%    Heiko Oberdiek <heiko.oberdiek at googlemail.com>
%    2016
%    https://github.com/ho-tex/oberdiek/issues
%
% This work may be distributed and/or modified under the
% conditions of the LaTeX Project Public License, either
% version 1.3c of this license or (at your option) any later
% version. This version of this license is in
%    http://www.latex-project.org/lppl/lppl-1-3c.txt
% and the latest version of this license is in
%    http://www.latex-project.org/lppl.txt
% and version 1.3 or later is part of all distributions of
% LaTeX version 2005/12/01 or later.
%
% This work has the LPPL maintenance status "maintained".
%
% This Current Maintainer of this work is Heiko Oberdiek.
%
% This work consists of the main source file resizegather.dtx
% and the derived files
%    resizegather.sty, resizegather.pdf, resizegather.ins, resizegather.drv,
%    resizegather-test1.tex.
%
% Distribution:
%    CTAN:macros/latex/contrib/oberdiek/resizegather.dtx
%    CTAN:macros/latex/contrib/oberdiek/resizegather.pdf
%
% Unpacking:
%    (a) If resizegather.ins is present:
%           tex resizegather.ins
%    (b) Without resizegather.ins:
%           tex resizegather.dtx
%    (c) If you insist on using LaTeX
%           latex \let\install=y% \iffalse meta-comment
%
% File: resizegather.dtx
% Version: 2016/05/16 v1.3
% Info: Resize overly large equations
%
% Copyright (C) 2009, 2010 by
%    Heiko Oberdiek <heiko.oberdiek at googlemail.com>
%    2016
%    https://github.com/ho-tex/oberdiek/issues
%
% This work may be distributed and/or modified under the
% conditions of the LaTeX Project Public License, either
% version 1.3c of this license or (at your option) any later
% version. This version of this license is in
%    http://www.latex-project.org/lppl/lppl-1-3c.txt
% and the latest version of this license is in
%    http://www.latex-project.org/lppl.txt
% and version 1.3 or later is part of all distributions of
% LaTeX version 2005/12/01 or later.
%
% This work has the LPPL maintenance status "maintained".
%
% This Current Maintainer of this work is Heiko Oberdiek.
%
% This work consists of the main source file resizegather.dtx
% and the derived files
%    resizegather.sty, resizegather.pdf, resizegather.ins, resizegather.drv,
%    resizegather-test1.tex.
%
% Distribution:
%    CTAN:macros/latex/contrib/oberdiek/resizegather.dtx
%    CTAN:macros/latex/contrib/oberdiek/resizegather.pdf
%
% Unpacking:
%    (a) If resizegather.ins is present:
%           tex resizegather.ins
%    (b) Without resizegather.ins:
%           tex resizegather.dtx
%    (c) If you insist on using LaTeX
%           latex \let\install=y\input{resizegather.dtx}
%        (quote the arguments according to the demands of your shell)
%
% Documentation:
%    (a) If resizegather.drv is present:
%           latex resizegather.drv
%    (b) Without resizegather.drv:
%           latex resizegather.dtx; ...
%    The class ltxdoc loads the configuration file ltxdoc.cfg
%    if available. Here you can specify further options, e.g.
%    use A4 as paper format:
%       \PassOptionsToClass{a4paper}{article}
%
%    Programm calls to get the documentation (example):
%       pdflatex resizegather.dtx
%       makeindex -s gind.ist resizegather.idx
%       pdflatex resizegather.dtx
%       makeindex -s gind.ist resizegather.idx
%       pdflatex resizegather.dtx
%
% Installation:
%    TDS:tex/latex/oberdiek/resizegather.sty
%    TDS:doc/latex/oberdiek/resizegather.pdf
%    TDS:doc/latex/oberdiek/test/resizegather-test1.tex
%    TDS:source/latex/oberdiek/resizegather.dtx
%
%<*ignore>
\begingroup
  \catcode123=1 %
  \catcode125=2 %
  \def\x{LaTeX2e}%
\expandafter\endgroup
\ifcase 0\ifx\install y1\fi\expandafter
         \ifx\csname processbatchFile\endcsname\relax\else1\fi
         \ifx\fmtname\x\else 1\fi\relax
\else\csname fi\endcsname
%</ignore>
%<*install>
\input docstrip.tex
\Msg{************************************************************************}
\Msg{* Installation}
\Msg{* Package: resizegather 2016/05/16 v1.3 Resize overly large equations (HO)}
\Msg{************************************************************************}

\keepsilent
\askforoverwritefalse

\let\MetaPrefix\relax
\preamble

This is a generated file.

Project: resizegather
Version: 2016/05/16 v1.3

Copyright (C) 2009, 2010 by
   Heiko Oberdiek <heiko.oberdiek at googlemail.com>

This work may be distributed and/or modified under the
conditions of the LaTeX Project Public License, either
version 1.3c of this license or (at your option) any later
version. This version of this license is in
   http://www.latex-project.org/lppl/lppl-1-3c.txt
and the latest version of this license is in
   http://www.latex-project.org/lppl.txt
and version 1.3 or later is part of all distributions of
LaTeX version 2005/12/01 or later.

This work has the LPPL maintenance status "maintained".

This Current Maintainer of this work is Heiko Oberdiek.

This work consists of the main source file resizegather.dtx
and the derived files
   resizegather.sty, resizegather.pdf, resizegather.ins, resizegather.drv,
   resizegather-test1.tex.

\endpreamble
\let\MetaPrefix\DoubleperCent

\generate{%
  \file{resizegather.ins}{\from{resizegather.dtx}{install}}%
  \file{resizegather.drv}{\from{resizegather.dtx}{driver}}%
  \usedir{tex/latex/oberdiek}%
  \file{resizegather.sty}{\from{resizegather.dtx}{package}}%
%  \usedir{doc/latex/oberdiek/test}%
%  \file{resizegather-test1.tex}{\from{resizegather.dtx}{test1}}%
  \nopreamble
  \nopostamble
%  \usedir{source/latex/oberdiek/catalogue}%
%  \file{resizegather.xml}{\from{resizegather.dtx}{catalogue}}%
}

\catcode32=13\relax% active space
\let =\space%
\Msg{************************************************************************}
\Msg{*}
\Msg{* To finish the installation you have to move the following}
\Msg{* file into a directory searched by TeX:}
\Msg{*}
\Msg{*     resizegather.sty}
\Msg{*}
\Msg{* To produce the documentation run the file `resizegather.drv'}
\Msg{* through LaTeX.}
\Msg{*}
\Msg{* Happy TeXing!}
\Msg{*}
\Msg{************************************************************************}

\endbatchfile
%</install>
%<*ignore>
\fi
%</ignore>
%<*driver>
\NeedsTeXFormat{LaTeX2e}
\ProvidesFile{resizegather.drv}%
  [2016/05/16 v1.3 Resize overly large equations (HO)]%
\documentclass{ltxdoc}
\usepackage{holtxdoc}[2011/11/22]
\usepackage{ifluatex}
\ifluatex
\else
  \usepackage[T1]{fontenc}%
  \usepackage{textcomp}%
  \usepackage{lmodern}%
\fi
\begin{document}
  \DocInput{resizegather.dtx}%
\end{document}
%</driver>
% \fi
%
%
% \CharacterTable
%  {Upper-case    \A\B\C\D\E\F\G\H\I\J\K\L\M\N\O\P\Q\R\S\T\U\V\W\X\Y\Z
%   Lower-case    \a\b\c\d\e\f\g\h\i\j\k\l\m\n\o\p\q\r\s\t\u\v\w\x\y\z
%   Digits        \0\1\2\3\4\5\6\7\8\9
%   Exclamation   \!     Double quote  \"     Hash (number) \#
%   Dollar        \$     Percent       \%     Ampersand     \&
%   Acute accent  \'     Left paren    \(     Right paren   \)
%   Asterisk      \*     Plus          \+     Comma         \,
%   Minus         \-     Point         \.     Solidus       \/
%   Colon         \:     Semicolon     \;     Less than     \<
%   Equals        \=     Greater than  \>     Question mark \?
%   Commercial at \@     Left bracket  \[     Backslash     \\
%   Right bracket \]     Circumflex    \^     Underscore    \_
%   Grave accent  \`     Left brace    \{     Vertical bar  \|
%   Right brace   \}     Tilde         \~}
%
% \GetFileInfo{resizegather.drv}
%
% \title{The \xpackage{resizegather} package}
% \date{2016/05/16 v1.3}
% \author{Heiko Oberdiek\thanks
% {Please report any issues at https://github.com/ho-tex/oberdiek/issues}\\
% \xemail{heiko.oberdiek at googlemail.com}}
%
% \maketitle
%
% \begin{abstract}
% Equations that are too large are resized to fit the available
% space. The environment \textsf{gather} of package \xpackage{amsmath}
% is supported. Also the environments \textsf{equation} and
% \textsf{displaymath} are redefined using \textsf{gather}
% and its starred version.
% \end{abstract}
%
% \tableofcontents
%
% \makeatletter
% \def\env#1{^^A
%    \textsf{\@env#1*\@nil}^^A
% }%
% \def\@env#1*#2\@nil{^^A
%   #1^^A
%   \ifx\\#2\\^^A
%     \expandafter\@gobble
%   \else
%     \textasteriskcentered
%     \expandafter\@firstofone
%   \fi
%   {\@env#2\@nil}^^A
% }
% \makeatother
%
% \section{Documentation}
%
% Sometimes an equation is just a little to large to fit in the
% line. And breaking the equation across lines might be worse
% than downscaling the equation. This package implements this
% for the environments \env{gather} and \env{gather*} of
% package \xpackage{amsmath}. That package already measures
% the equations and simplifies the implementation of \xpackage{resizegather}
% that only needs to hook into \xpackage{amsmath}'s code to add
% the resizing feature.
%
% Resized equations are recorded in the \xext{log} file
% for small exceeds (default setting is smaller than five percent).
% Otherwise a warning is given.
%
% Also environments \env{equation} and \env{displaymath}
% are supported by redefining them using \env{gather}
% and \env{gather*}.
%
% \cs{[} and \cs{]} are not supported, because these macros
% are not in environment form that is required for
% \xpackage{amsmath}. The environment body is collected
% first to be able to process the body twice for measuring
% first.
%
% Also the environments using alignments are not supported.
% If a single equation line would be resized, the alignment
% would get lost. And resizing all equations of the alignment
% does not seem appropriate either.
%
% \subsection{Options}
%
% \begin{description}
% \item[\xoption{warningthreshold}:]
%   Print a warning if the original equation line exceeds
%   its available width by the given fraction.
%   Default is |0.05|: A warning is given if the equation
%   is too large by five percent.
%   Otherwise the exceed is recorded in the \xext{log} file
%   only.
% \end{description}
% The next options are boolean options. They are enabled
% by value |true| or if no value is given. They are switched
% off by value |false|.
% \begin{description}
% \item[\xoption{enable}:] The resize feature is active (default).
% \item[\xoption{disable}:] The complementary option for \xoption{enable},
%  added for convenience: |disable| (or |disable=true|) is the same
%  as |enable=false|.
% \item[\xoption{equations}:]
%   \LaTeX\ environments \textsf{equation} and \textsf{displaymath}
%   environments are redefined. These equations
%   are now using environment \env{gather} and
%   \env{gather*}. This is the default.
% \end{description}
% The following table shows additional options if you
% want to have finer control for the redefined
% environments:
% \begin{quote}
% \def\unchanged{\textit{unchanged}}
% \def\notprovided{\textit{not provided}}
% \begin{tabular}{l|ll}
% &\multicolumn{2}{c}{Environments}\\
% Option & \env{equation} & \env{displaymath}\\
% \hline
% \xoption{equations} & \env{gather} & \env{gather*}\\
% \xoption{equation} & \env{gather} & \unchanged\\
% \xoption{displaymath} & \unchanged & \env{gather*}\\
% \end{tabular}
% \end{quote}
% If such an option is switched off, the original meaning
% of the affected environments is restored.
%
% Options are evaluated in the following order:
% \begin{enumerate}
% \item
%  Configuration file \xfile{resizegather.cfg} using \cs{resizegathersetup}
%  if the file exists.
%  \item
%  Package options given for \cs{usepackage}.
%  \item
%  Later calls of \cs{resizegathersetup}.
% \end{enumerate}
% \begin{declcs}{resizegathersetup}\M{option list}
% \end{declcs}
% The options are key value options. Boolean options are enabled by
% default (without value) or by using the explicit value \texttt{true}.
% Value \texttt{false} disable the option.
%
% \subsection{Options for packages \xpackage{amsmath} or \xpackage{graphics}}
%
% The package loads the package \xpackage{amsmath} because is internally
% measures the equations first. Thus this package hooks into this code
% in order to resize the equations if they are too large.
% The resizing itself is done by \cs{resizebox} of package \xpackage{graphics}.
% If you need special options for these packages, just load them first or
% use global options when appropriate. Example:
% \begin{quote}
%\begin{verbatim}
%\usepackage[dvipdfm]{graphicx}% or graphics
%\usepackage[fleqn]{amsmath}
%\usepackage{resizegather}
%\end{verbatim}
%\end{quote}
%
% \StopEventually{
% }
%
% \section{Implementation}
%    \begin{macrocode}
%<*package>
%    \end{macrocode}
%    Reload check, especially if the package is not used with \LaTeX.
%    \begin{macrocode}
\begingroup\catcode61\catcode48\catcode32=10\relax%
  \catcode13=5 % ^^M
  \endlinechar=13 %
  \catcode35=6 % #
  \catcode39=12 % '
  \catcode44=12 % ,
  \catcode45=12 % -
  \catcode46=12 % .
  \catcode58=12 % :
  \catcode64=11 % @
  \catcode123=1 % {
  \catcode125=2 % }
  \expandafter\let\expandafter\x\csname ver@resizegather.sty\endcsname
  \ifx\x\relax % plain-TeX, first loading
  \else
    \def\empty{}%
    \ifx\x\empty % LaTeX, first loading,
      % variable is initialized, but \ProvidesPackage not yet seen
    \else
      \expandafter\ifx\csname PackageInfo\endcsname\relax
        \def\x#1#2{%
          \immediate\write-1{Package #1 Info: #2.}%
        }%
      \else
        \def\x#1#2{\PackageInfo{#1}{#2, stopped}}%
      \fi
      \x{resizegather}{The package is already loaded}%
      \aftergroup\endinput
    \fi
  \fi
\endgroup%
%    \end{macrocode}
%    Package identification:
%    \begin{macrocode}
\begingroup\catcode61\catcode48\catcode32=10\relax%
  \catcode13=5 % ^^M
  \endlinechar=13 %
  \catcode35=6 % #
  \catcode39=12 % '
  \catcode40=12 % (
  \catcode41=12 % )
  \catcode44=12 % ,
  \catcode45=12 % -
  \catcode46=12 % .
  \catcode47=12 % /
  \catcode58=12 % :
  \catcode64=11 % @
  \catcode91=12 % [
  \catcode93=12 % ]
  \catcode123=1 % {
  \catcode125=2 % }
  \expandafter\ifx\csname ProvidesPackage\endcsname\relax
    \def\x#1#2#3[#4]{\endgroup
      \immediate\write-1{Package: #3 #4}%
      \xdef#1{#4}%
    }%
  \else
    \def\x#1#2[#3]{\endgroup
      #2[{#3}]%
      \ifx#1\@undefined
        \xdef#1{#3}%
      \fi
      \ifx#1\relax
        \xdef#1{#3}%
      \fi
    }%
  \fi
\expandafter\x\csname ver@resizegather.sty\endcsname
\ProvidesPackage{resizegather}%
  [2016/05/16 v1.3 Resize overly large equations (HO)]%
%    \end{macrocode}
%
%    \begin{macrocode}
\begingroup\catcode61\catcode48\catcode32=10\relax%
  \catcode13=5 % ^^M
  \endlinechar=13 %
  \catcode123=1 % {
  \catcode125=2 % }
  \catcode64=11 % @
  \def\x{\endgroup
    \expandafter\edef\csname ResizeGather@AtEnd\endcsname{%
      \endlinechar=\the\endlinechar\relax
      \catcode13=\the\catcode13\relax
      \catcode32=\the\catcode32\relax
      \catcode35=\the\catcode35\relax
      \catcode61=\the\catcode61\relax
      \catcode64=\the\catcode64\relax
      \catcode123=\the\catcode123\relax
      \catcode125=\the\catcode125\relax
    }%
  }%
\x\catcode61\catcode48\catcode32=10\relax%
\catcode13=5 % ^^M
\endlinechar=13 %
\catcode35=6 % #
\catcode64=11 % @
\catcode123=1 % {
\catcode125=2 % }
\def\TMP@EnsureCode#1#2{%
  \edef\ResizeGather@AtEnd{%
    \ResizeGather@AtEnd
    \catcode#1=\the\catcode#1\relax
  }%
  \catcode#1=#2\relax
}
\TMP@EnsureCode{10}{12}% ^^J
\TMP@EnsureCode{33}{12}% !
\TMP@EnsureCode{36}{3}% $
\TMP@EnsureCode{38}{4}% &
\TMP@EnsureCode{39}{12}% '
\TMP@EnsureCode{40}{12}% (
\TMP@EnsureCode{41}{12}% )
\TMP@EnsureCode{42}{12}% *
\TMP@EnsureCode{43}{12}% +
\TMP@EnsureCode{44}{12}% ,
\TMP@EnsureCode{45}{12}% -
\TMP@EnsureCode{46}{12}% .
\TMP@EnsureCode{47}{12}% /
\TMP@EnsureCode{58}{12}% :
\TMP@EnsureCode{59}{12}% ;
\TMP@EnsureCode{60}{12}% <
\TMP@EnsureCode{62}{12}% >
\TMP@EnsureCode{63}{12}% ?
\TMP@EnsureCode{91}{12}% [
\TMP@EnsureCode{93}{12}% ]
\TMP@EnsureCode{94}{7}% ^ (superscript)
\TMP@EnsureCode{96}{12}% `
\edef\ResizeGather@AtEnd{\ResizeGather@AtEnd\noexpand\endinput}
%    \end{macrocode}
%
%    \begin{macrocode}
\RequirePackage{kvoptions}[2009/12/04]
\SetupKeyvalOptions{%
  family=resizegather,%
  prefix=ResizeGather@,%
}
%    \end{macrocode}
%    \begin{macrocode}
\@for\ResizeGather@option:=%
  centertags,%
  tbtags,%
  sumlimits,%
  nosumlimits,%
  intlimits,%
  nointlimits,%
  nonamelimits,%
  leqno,%
  reqno,%
  fleqn%
\do{%
  \edef\ResizeGather@temp{%
    \noexpand\DeclareVoidOption{\ResizeGather@option}{%
      \noexpand\PassOptionsToPackage{amsmath}{\ResizeGather@option}%
    }%
    \noexpand\AtEndOfPackage{%
      \noexpand\DisableKeyvalOption[%
        action=error,%
        package=resizegather,%
      ]{resizegather}{\ResizeGather@option}%
    }%
  }%
  \ResizeGather@temp
}
\@for\ResizeGather@option:=%
  draft,%
  final,%
  hiderotate,%
  hidescale,%
  hiresbb,%
  demo,%
  dvips,xdvi,dvipdf,dvipdfm,dvipdfmx,pdftex,dvipsone,%
  dviwindo,emtex,dviwin,pctexps,pctexwin,pctexhp,pctex32,%
  truetex,tcidvi,vtex,oztex,textures,xetex%
\do{%
  \edef\ResizeGather@temp{%
    \noexpand\DeclareVoidOption{\ResizeGather@option}{%
      \noexpand\PassOptionsToPackage{graphics}{\ResizeGather@option}%
    }%
    \noexpand\AtEndOfPackage{%
      \noexpand\DisableKeyvalOption[%
        action=error,%
        package=resizegather,%
      ]{resizegather}{\ResizeGather@option}%
    }%
  }%
  \ResizeGather@temp
}
%    \end{macrocode}
%
%    \begin{macrocode}
\DeclareBoolOption[true]{enable}
\DeclareComplementaryOption{disable}{enable}
\DeclareStringOption[.05]{warningthreshold}
\newif\ifResizeGather@NeedInit
\DeclareBoolOption[true]{equations}
\DeclareBoolOption[true]{equation}
\DeclareBoolOption[true]{displaymath}
\AddToKeyvalOption*{equations}{%
  \ResizeGather@NeedInittrue
  \ifResizeGather@equations
    \ResizeGather@equationtrue
    \ResizeGather@displaymathtrue
  \else
    \ResizeGather@equationfalse
    \ResizeGather@displaymathfalse
  \fi
}
\AddToKeyvalOption*{equation}{%
  \ResizeGather@NeedInittrue
}
\AddToKeyvalOption*{displaymath}{%
  \ResizeGather@NeedInittrue
}
%    \end{macrocode}
%
%    \begin{macro}{\resizegathersetup}
%    \begin{macrocode}
\newcommand*{\resizegathersetup}[1]{%
  \ResizeGather@NeedInitfalse
  \setkeys{resizegather}{#1}%
  \ifResizeGather@NeedInit
    \ResizeGather@init
  \fi
}
\let\ResizeGather@init\relax
%    \end{macrocode}
%    \end{macro}
%    \begin{macrocode}
\InputIfFileExists{resizegather.cfg}{}{}%
\ProcessKeyvalOptions*\relax
%    \end{macrocode}
%    \begin{macrocode}
\RequirePackage{amsmath}
\RequirePackage{graphics}
%    \end{macrocode}
%    \begin{macro}{\ResizeGather@redefine}
%    \begin{macrocode}
\def\ResizeGather@redefine#1#2#3#4#5{%
  \csname ifResizeGather@#1\endcsname
    \@ifundefined{ResizeGather@org@#2}{%
      \expandafter\let\csname ResizeGather@org@#2\expandafter\endcsname
                      \csname #2\endcsname
    }{}%
    \@ifundefined{ResizeGather@org@#3}{%
      \expandafter\let\csname ResizeGather@org@#3\expandafter\endcsname
                      \csname #3\endcsname
    }{}%
    \expandafter\edef\csname #2\endcsname{%
      \expandafter\noexpand\csname#4\endcsname
    }%
    \expandafter\edef\csname #3\endcsname{%
      \expandafter\noexpand\csname#5\endcsname
    }%
  \else
    \@ifundefined{ResizeGather@org@#2}{}{%
      \expandafter\let\csname #2\expandafter\endcsname
                      \csname ResizeGather@org@#2\endcsname
      \expandafter\let\csname #3\expandafter\endcsname
                      \csname ResizeGather@org@#3\endcsname
    }%
  \fi
}
%    \end{macrocode}
%    \end{macro}
%    \begin{macro}{\ResizeGather@init}
%    \begin{macrocode}
\def\ResizeGather@init{%
  \ResizeGather@redefine{equation}{equation}{endequation}%
    {gather}{endgather}%
  \ResizeGather@redefine{displaymath}{displaymath}{enddisplaymath}%
    {gather*}{endgather*}%
}
\ResizeGather@init
%    \end{macrocode}
%    \end{macro}
%
%    \begin{macro}{\ResizeGather@ResizeGather}
%    \begin{macrocode}
\def\ResizeGather@ResizeGather{%
  \ifResizeGather@enable
    \dimen@\displaywidth
    \if@fleqn
      \advance\dimen@-\@mathmargin
    \fi
    \ifdim\wdz@>\dimen@
      \begingroup
        \advance\dimen@ -\wdz@
        \dimen@ -\dimen@
        \ifdim\ResizeGather@warningthreshold\wdz@>\dimen@
          \expandafter\PackageInfo
        \else
          \expandafter\PackageWarning
        \fi
        {resizegather}{%
          Equation line \the\row@\space is too large %
          by \the\dimen@\MessageBreak
          in environment `\@currenvir'%
        }%
      \endgroup
      \setboxz@h to\dimen@{%
        \resizebox{\dimen@}{!}{\boxz@}%
        \hss
      }%
    \fi
  \fi
}
%    \end{macrocode}
%    \end{macro}
%    \begin{macro}{\calc@shift@gather}
%    \begin{macrocode}
\expandafter\def\expandafter\calc@shift@gather\expandafter{%
  \expandafter\ResizeGather@ResizeGather
  \calc@shift@gather
}
%    \end{macrocode}
%    \end{macro}
%    \begin{macro}{\ResizeGather@org@gmeasure@}
%    \begin{macrocode}
\let\ResizeGather@org@gmeasure@\gmeasure@
%    \end{macrocode}
%    \end{macro}
%    \begin{macro}{\gmeasure@}
%    \begin{macrocode}
\def\gmeasure@#1{%
  \ResizeGather@org@gmeasure@{#1}%
  \ifResizeGather@enable
    \ifdim\totwidth@>\displaywidth
      \totwidth@=\displaywidth
    \fi
  \fi
}
%    \end{macrocode}
%    \end{macro}
%
%    \begin{macrocode}
\ResizeGather@AtEnd%
%</package>
%    \end{macrocode}
%
% \section{Test}
%
% \subsection{Catcode checks for loading}
%
%    \begin{macrocode}
%<*test1>
%    \end{macrocode}
%    \begin{macrocode}
\catcode`\{=1 %
\catcode`\}=2 %
\catcode`\#=6 %
\catcode`\@=11 %
\expandafter\ifx\csname count@\endcsname\relax
  \countdef\count@=255 %
\fi
\expandafter\ifx\csname @gobble\endcsname\relax
  \long\def\@gobble#1{}%
\fi
\expandafter\ifx\csname @firstofone\endcsname\relax
  \long\def\@firstofone#1{#1}%
\fi
\expandafter\ifx\csname loop\endcsname\relax
  \expandafter\@firstofone
\else
  \expandafter\@gobble
\fi
{%
  \def\loop#1\repeat{%
    \def\body{#1}%
    \iterate
  }%
  \def\iterate{%
    \body
      \let\next\iterate
    \else
      \let\next\relax
    \fi
    \next
  }%
  \let\repeat=\fi
}%
\def\RestoreCatcodes{}
\count@=0 %
\loop
  \edef\RestoreCatcodes{%
    \RestoreCatcodes
    \catcode\the\count@=\the\catcode\count@\relax
  }%
\ifnum\count@<255 %
  \advance\count@ 1 %
\repeat

\def\RangeCatcodeInvalid#1#2{%
  \count@=#1\relax
  \loop
    \catcode\count@=15 %
  \ifnum\count@<#2\relax
    \advance\count@ 1 %
  \repeat
}
\def\RangeCatcodeCheck#1#2#3{%
  \count@=#1\relax
  \loop
    \ifnum#3=\catcode\count@
    \else
      \errmessage{%
        Character \the\count@\space
        with wrong catcode \the\catcode\count@\space
        instead of \number#3%
      }%
    \fi
  \ifnum\count@<#2\relax
    \advance\count@ 1 %
  \repeat
}
\def\space{ }
\expandafter\ifx\csname LoadCommand\endcsname\relax
  \def\LoadCommand{\input resizegather.sty\relax}%
\fi
\def\Test{%
  \RangeCatcodeInvalid{0}{47}%
  \RangeCatcodeInvalid{58}{64}%
  \RangeCatcodeInvalid{91}{96}%
  \RangeCatcodeInvalid{123}{255}%
  \catcode`\@=12 %
  \catcode`\\=0 %
  \catcode`\%=14 %
  \LoadCommand
  \RangeCatcodeCheck{0}{36}{15}%
  \RangeCatcodeCheck{37}{37}{14}%
  \RangeCatcodeCheck{38}{47}{15}%
  \RangeCatcodeCheck{48}{57}{12}%
  \RangeCatcodeCheck{58}{63}{15}%
  \RangeCatcodeCheck{64}{64}{12}%
  \RangeCatcodeCheck{65}{90}{11}%
  \RangeCatcodeCheck{91}{91}{15}%
  \RangeCatcodeCheck{92}{92}{0}%
  \RangeCatcodeCheck{93}{96}{15}%
  \RangeCatcodeCheck{97}{122}{11}%
  \RangeCatcodeCheck{123}{255}{15}%
  \RestoreCatcodes
}
\Test
\csname @@end\endcsname
\end
%    \end{macrocode}
%    \begin{macrocode}
%</test1>
%    \end{macrocode}
%
% \section{Installation}
%
% \subsection{Download}
%
% \paragraph{Package.} This package is available on
% CTAN\footnote{\url{https://ctan.org/pkg/resizegather}}:
% \begin{description}
% \item[\CTAN{macros/latex/contrib/oberdiek/resizegather.dtx}] The source file.
% \item[\CTAN{macros/latex/contrib/oberdiek/resizegather.pdf}] Documentation.
% \end{description}
%
%
% \paragraph{Bundle.} All the packages of the bundle `oberdiek'
% are also available in a TDS compliant ZIP archive. There
% the packages are already unpacked and the documentation files
% are generated. The files and directories obey the TDS standard.
% \begin{description}
% \item[\CTANinstall{install/macros/latex/contrib/oberdiek.tds.zip}]
% \end{description}
% \emph{TDS} refers to the standard ``A Directory Structure
% for \TeX\ Files'' (\CTAN{tds/tds.pdf}). Directories
% with \xfile{texmf} in their name are usually organized this way.
%
% \subsection{Bundle installation}
%
% \paragraph{Unpacking.} Unpack the \xfile{oberdiek.tds.zip} in the
% TDS tree (also known as \xfile{texmf} tree) of your choice.
% Example (linux):
% \begin{quote}
%   |unzip oberdiek.tds.zip -d ~/texmf|
% \end{quote}
%
% \paragraph{Script installation.}
% Check the directory \xfile{TDS:scripts/oberdiek/} for
% scripts that need further installation steps.
% Package \xpackage{attachfile2} comes with the Perl script
% \xfile{pdfatfi.pl} that should be installed in such a way
% that it can be called as \texttt{pdfatfi}.
% Example (linux):
% \begin{quote}
%   |chmod +x scripts/oberdiek/pdfatfi.pl|\\
%   |cp scripts/oberdiek/pdfatfi.pl /usr/local/bin/|
% \end{quote}
%
% \subsection{Package installation}
%
% \paragraph{Unpacking.} The \xfile{.dtx} file is a self-extracting
% \docstrip\ archive. The files are extracted by running the
% \xfile{.dtx} through \plainTeX:
% \begin{quote}
%   \verb|tex resizegather.dtx|
% \end{quote}
%
% \paragraph{TDS.} Now the different files must be moved into
% the different directories in your installation TDS tree
% (also known as \xfile{texmf} tree):
% \begin{quote}
% \def\t{^^A
% \begin{tabular}{@{}>{\ttfamily}l@{ $\rightarrow$ }>{\ttfamily}l@{}}
%   resizegather.sty & tex/latex/oberdiek/resizegather.sty\\
%   resizegather.pdf & doc/latex/oberdiek/resizegather.pdf\\
%   test/resizegather-test1.tex & doc/latex/oberdiek/test/resizegather-test1.tex\\
%   resizegather.dtx & source/latex/oberdiek/resizegather.dtx\\
% \end{tabular}^^A
% }^^A
% \sbox0{\t}^^A
% \ifdim\wd0>\linewidth
%   \begingroup
%     \advance\linewidth by\leftmargin
%     \advance\linewidth by\rightmargin
%   \edef\x{\endgroup
%     \def\noexpand\lw{\the\linewidth}^^A
%   }\x
%   \def\lwbox{^^A
%     \leavevmode
%     \hbox to \linewidth{^^A
%       \kern-\leftmargin\relax
%       \hss
%       \usebox0
%       \hss
%       \kern-\rightmargin\relax
%     }^^A
%   }^^A
%   \ifdim\wd0>\lw
%     \sbox0{\small\t}^^A
%     \ifdim\wd0>\linewidth
%       \ifdim\wd0>\lw
%         \sbox0{\footnotesize\t}^^A
%         \ifdim\wd0>\linewidth
%           \ifdim\wd0>\lw
%             \sbox0{\scriptsize\t}^^A
%             \ifdim\wd0>\linewidth
%               \ifdim\wd0>\lw
%                 \sbox0{\tiny\t}^^A
%                 \ifdim\wd0>\linewidth
%                   \lwbox
%                 \else
%                   \usebox0
%                 \fi
%               \else
%                 \lwbox
%               \fi
%             \else
%               \usebox0
%             \fi
%           \else
%             \lwbox
%           \fi
%         \else
%           \usebox0
%         \fi
%       \else
%         \lwbox
%       \fi
%     \else
%       \usebox0
%     \fi
%   \else
%     \lwbox
%   \fi
% \else
%   \usebox0
% \fi
% \end{quote}
% If you have a \xfile{docstrip.cfg} that configures and enables \docstrip's
% TDS installing feature, then some files can already be in the right
% place, see the documentation of \docstrip.
%
% \subsection{Refresh file name databases}
%
% If your \TeX~distribution
% (\teTeX, \mikTeX, \dots) relies on file name databases, you must refresh
% these. For example, \teTeX\ users run \verb|texhash| or
% \verb|mktexlsr|.
%
% \subsection{Some details for the interested}
%
% \paragraph{Attached source.}
%
% The PDF documentation on CTAN also includes the
% \xfile{.dtx} source file. It can be extracted by
% AcrobatReader 6 or higher. Another option is \textsf{pdftk},
% e.g. unpack the file into the current directory:
% \begin{quote}
%   \verb|pdftk resizegather.pdf unpack_files output .|
% \end{quote}
%
% \paragraph{Unpacking with \LaTeX.}
% The \xfile{.dtx} chooses its action depending on the format:
% \begin{description}
% \item[\plainTeX:] Run \docstrip\ and extract the files.
% \item[\LaTeX:] Generate the documentation.
% \end{description}
% If you insist on using \LaTeX\ for \docstrip\ (really,
% \docstrip\ does not need \LaTeX), then inform the autodetect routine
% about your intention:
% \begin{quote}
%   \verb|latex \let\install=y\input{resizegather.dtx}|
% \end{quote}
% Do not forget to quote the argument according to the demands
% of your shell.
%
% \paragraph{Generating the documentation.}
% You can use both the \xfile{.dtx} or the \xfile{.drv} to generate
% the documentation. The process can be configured by the
% configuration file \xfile{ltxdoc.cfg}. For instance, put this
% line into this file, if you want to have A4 as paper format:
% \begin{quote}
%   \verb|\PassOptionsToClass{a4paper}{article}|
% \end{quote}
% An example follows how to generate the
% documentation with pdf\LaTeX:
% \begin{quote}
%\begin{verbatim}
%pdflatex resizegather.dtx
%makeindex -s gind.ist resizegather.idx
%pdflatex resizegather.dtx
%makeindex -s gind.ist resizegather.idx
%pdflatex resizegather.dtx
%\end{verbatim}
% \end{quote}
%
% \section{Acknowledgement}
%
% \begin{description}
% \item[Dieter Jurzitza:]
% He wanted the resizing feature at the \TeX\ table
% in Karlsruhe of December 2009. Thus this package is a kind of
% Christmas present.
% \end{description}
%
% \begin{History}
%   \begin{Version}{2009/12/04 v1.0}
%   \item
%     The first version.
%   \end{Version}
%   \begin{Version}{2009/12/05 v1.1}
%   \item
%     Options \xoption{enable} and \xoption{disable} added.
%   \end{Version}
%   \begin{Version}{2010/03/01 v1.2}
%   \item
%     TDS location moved from `generic' to `latex'.
%   \end{Version}
%   \begin{Version}{2016/05/16 v1.3}
%   \item
%     Documentation updates.
%   \end{Version}
% \end{History}
%
% \PrintIndex
%
% \Finale
\endinput

%        (quote the arguments according to the demands of your shell)
%
% Documentation:
%    (a) If resizegather.drv is present:
%           latex resizegather.drv
%    (b) Without resizegather.drv:
%           latex resizegather.dtx; ...
%    The class ltxdoc loads the configuration file ltxdoc.cfg
%    if available. Here you can specify further options, e.g.
%    use A4 as paper format:
%       \PassOptionsToClass{a4paper}{article}
%
%    Programm calls to get the documentation (example):
%       pdflatex resizegather.dtx
%       makeindex -s gind.ist resizegather.idx
%       pdflatex resizegather.dtx
%       makeindex -s gind.ist resizegather.idx
%       pdflatex resizegather.dtx
%
% Installation:
%    TDS:tex/latex/oberdiek/resizegather.sty
%    TDS:doc/latex/oberdiek/resizegather.pdf
%    TDS:doc/latex/oberdiek/test/resizegather-test1.tex
%    TDS:source/latex/oberdiek/resizegather.dtx
%
%<*ignore>
\begingroup
  \catcode123=1 %
  \catcode125=2 %
  \def\x{LaTeX2e}%
\expandafter\endgroup
\ifcase 0\ifx\install y1\fi\expandafter
         \ifx\csname processbatchFile\endcsname\relax\else1\fi
         \ifx\fmtname\x\else 1\fi\relax
\else\csname fi\endcsname
%</ignore>
%<*install>
\input docstrip.tex
\Msg{************************************************************************}
\Msg{* Installation}
\Msg{* Package: resizegather 2016/05/16 v1.3 Resize overly large equations (HO)}
\Msg{************************************************************************}

\keepsilent
\askforoverwritefalse

\let\MetaPrefix\relax
\preamble

This is a generated file.

Project: resizegather
Version: 2016/05/16 v1.3

Copyright (C) 2009, 2010 by
   Heiko Oberdiek <heiko.oberdiek at googlemail.com>

This work may be distributed and/or modified under the
conditions of the LaTeX Project Public License, either
version 1.3c of this license or (at your option) any later
version. This version of this license is in
   http://www.latex-project.org/lppl/lppl-1-3c.txt
and the latest version of this license is in
   http://www.latex-project.org/lppl.txt
and version 1.3 or later is part of all distributions of
LaTeX version 2005/12/01 or later.

This work has the LPPL maintenance status "maintained".

This Current Maintainer of this work is Heiko Oberdiek.

This work consists of the main source file resizegather.dtx
and the derived files
   resizegather.sty, resizegather.pdf, resizegather.ins, resizegather.drv,
   resizegather-test1.tex.

\endpreamble
\let\MetaPrefix\DoubleperCent

\generate{%
  \file{resizegather.ins}{\from{resizegather.dtx}{install}}%
  \file{resizegather.drv}{\from{resizegather.dtx}{driver}}%
  \usedir{tex/latex/oberdiek}%
  \file{resizegather.sty}{\from{resizegather.dtx}{package}}%
%  \usedir{doc/latex/oberdiek/test}%
%  \file{resizegather-test1.tex}{\from{resizegather.dtx}{test1}}%
  \nopreamble
  \nopostamble
%  \usedir{source/latex/oberdiek/catalogue}%
%  \file{resizegather.xml}{\from{resizegather.dtx}{catalogue}}%
}

\catcode32=13\relax% active space
\let =\space%
\Msg{************************************************************************}
\Msg{*}
\Msg{* To finish the installation you have to move the following}
\Msg{* file into a directory searched by TeX:}
\Msg{*}
\Msg{*     resizegather.sty}
\Msg{*}
\Msg{* To produce the documentation run the file `resizegather.drv'}
\Msg{* through LaTeX.}
\Msg{*}
\Msg{* Happy TeXing!}
\Msg{*}
\Msg{************************************************************************}

\endbatchfile
%</install>
%<*ignore>
\fi
%</ignore>
%<*driver>
\NeedsTeXFormat{LaTeX2e}
\ProvidesFile{resizegather.drv}%
  [2016/05/16 v1.3 Resize overly large equations (HO)]%
\documentclass{ltxdoc}
\usepackage{holtxdoc}[2011/11/22]
\usepackage{ifluatex}
\ifluatex
\else
  \usepackage[T1]{fontenc}%
  \usepackage{textcomp}%
  \usepackage{lmodern}%
\fi
\begin{document}
  \DocInput{resizegather.dtx}%
\end{document}
%</driver>
% \fi
%
%
% \CharacterTable
%  {Upper-case    \A\B\C\D\E\F\G\H\I\J\K\L\M\N\O\P\Q\R\S\T\U\V\W\X\Y\Z
%   Lower-case    \a\b\c\d\e\f\g\h\i\j\k\l\m\n\o\p\q\r\s\t\u\v\w\x\y\z
%   Digits        \0\1\2\3\4\5\6\7\8\9
%   Exclamation   \!     Double quote  \"     Hash (number) \#
%   Dollar        \$     Percent       \%     Ampersand     \&
%   Acute accent  \'     Left paren    \(     Right paren   \)
%   Asterisk      \*     Plus          \+     Comma         \,
%   Minus         \-     Point         \.     Solidus       \/
%   Colon         \:     Semicolon     \;     Less than     \<
%   Equals        \=     Greater than  \>     Question mark \?
%   Commercial at \@     Left bracket  \[     Backslash     \\
%   Right bracket \]     Circumflex    \^     Underscore    \_
%   Grave accent  \`     Left brace    \{     Vertical bar  \|
%   Right brace   \}     Tilde         \~}
%
% \GetFileInfo{resizegather.drv}
%
% \title{The \xpackage{resizegather} package}
% \date{2016/05/16 v1.3}
% \author{Heiko Oberdiek\thanks
% {Please report any issues at https://github.com/ho-tex/oberdiek/issues}\\
% \xemail{heiko.oberdiek at googlemail.com}}
%
% \maketitle
%
% \begin{abstract}
% Equations that are too large are resized to fit the available
% space. The environment \textsf{gather} of package \xpackage{amsmath}
% is supported. Also the environments \textsf{equation} and
% \textsf{displaymath} are redefined using \textsf{gather}
% and its starred version.
% \end{abstract}
%
% \tableofcontents
%
% \makeatletter
% \def\env#1{^^A
%    \textsf{\@env#1*\@nil}^^A
% }%
% \def\@env#1*#2\@nil{^^A
%   #1^^A
%   \ifx\\#2\\^^A
%     \expandafter\@gobble
%   \else
%     \textasteriskcentered
%     \expandafter\@firstofone
%   \fi
%   {\@env#2\@nil}^^A
% }
% \makeatother
%
% \section{Documentation}
%
% Sometimes an equation is just a little to large to fit in the
% line. And breaking the equation across lines might be worse
% than downscaling the equation. This package implements this
% for the environments \env{gather} and \env{gather*} of
% package \xpackage{amsmath}. That package already measures
% the equations and simplifies the implementation of \xpackage{resizegather}
% that only needs to hook into \xpackage{amsmath}'s code to add
% the resizing feature.
%
% Resized equations are recorded in the \xext{log} file
% for small exceeds (default setting is smaller than five percent).
% Otherwise a warning is given.
%
% Also environments \env{equation} and \env{displaymath}
% are supported by redefining them using \env{gather}
% and \env{gather*}.
%
% \cs{[} and \cs{]} are not supported, because these macros
% are not in environment form that is required for
% \xpackage{amsmath}. The environment body is collected
% first to be able to process the body twice for measuring
% first.
%
% Also the environments using alignments are not supported.
% If a single equation line would be resized, the alignment
% would get lost. And resizing all equations of the alignment
% does not seem appropriate either.
%
% \subsection{Options}
%
% \begin{description}
% \item[\xoption{warningthreshold}:]
%   Print a warning if the original equation line exceeds
%   its available width by the given fraction.
%   Default is |0.05|: A warning is given if the equation
%   is too large by five percent.
%   Otherwise the exceed is recorded in the \xext{log} file
%   only.
% \end{description}
% The next options are boolean options. They are enabled
% by value |true| or if no value is given. They are switched
% off by value |false|.
% \begin{description}
% \item[\xoption{enable}:] The resize feature is active (default).
% \item[\xoption{disable}:] The complementary option for \xoption{enable},
%  added for convenience: |disable| (or |disable=true|) is the same
%  as |enable=false|.
% \item[\xoption{equations}:]
%   \LaTeX\ environments \textsf{equation} and \textsf{displaymath}
%   environments are redefined. These equations
%   are now using environment \env{gather} and
%   \env{gather*}. This is the default.
% \end{description}
% The following table shows additional options if you
% want to have finer control for the redefined
% environments:
% \begin{quote}
% \def\unchanged{\textit{unchanged}}
% \def\notprovided{\textit{not provided}}
% \begin{tabular}{l|ll}
% &\multicolumn{2}{c}{Environments}\\
% Option & \env{equation} & \env{displaymath}\\
% \hline
% \xoption{equations} & \env{gather} & \env{gather*}\\
% \xoption{equation} & \env{gather} & \unchanged\\
% \xoption{displaymath} & \unchanged & \env{gather*}\\
% \end{tabular}
% \end{quote}
% If such an option is switched off, the original meaning
% of the affected environments is restored.
%
% Options are evaluated in the following order:
% \begin{enumerate}
% \item
%  Configuration file \xfile{resizegather.cfg} using \cs{resizegathersetup}
%  if the file exists.
%  \item
%  Package options given for \cs{usepackage}.
%  \item
%  Later calls of \cs{resizegathersetup}.
% \end{enumerate}
% \begin{declcs}{resizegathersetup}\M{option list}
% \end{declcs}
% The options are key value options. Boolean options are enabled by
% default (without value) or by using the explicit value \texttt{true}.
% Value \texttt{false} disable the option.
%
% \subsection{Options for packages \xpackage{amsmath} or \xpackage{graphics}}
%
% The package loads the package \xpackage{amsmath} because is internally
% measures the equations first. Thus this package hooks into this code
% in order to resize the equations if they are too large.
% The resizing itself is done by \cs{resizebox} of package \xpackage{graphics}.
% If you need special options for these packages, just load them first or
% use global options when appropriate. Example:
% \begin{quote}
%\begin{verbatim}
%\usepackage[dvipdfm]{graphicx}% or graphics
%\usepackage[fleqn]{amsmath}
%\usepackage{resizegather}
%\end{verbatim}
%\end{quote}
%
% \StopEventually{
% }
%
% \section{Implementation}
%    \begin{macrocode}
%<*package>
%    \end{macrocode}
%    Reload check, especially if the package is not used with \LaTeX.
%    \begin{macrocode}
\begingroup\catcode61\catcode48\catcode32=10\relax%
  \catcode13=5 % ^^M
  \endlinechar=13 %
  \catcode35=6 % #
  \catcode39=12 % '
  \catcode44=12 % ,
  \catcode45=12 % -
  \catcode46=12 % .
  \catcode58=12 % :
  \catcode64=11 % @
  \catcode123=1 % {
  \catcode125=2 % }
  \expandafter\let\expandafter\x\csname ver@resizegather.sty\endcsname
  \ifx\x\relax % plain-TeX, first loading
  \else
    \def\empty{}%
    \ifx\x\empty % LaTeX, first loading,
      % variable is initialized, but \ProvidesPackage not yet seen
    \else
      \expandafter\ifx\csname PackageInfo\endcsname\relax
        \def\x#1#2{%
          \immediate\write-1{Package #1 Info: #2.}%
        }%
      \else
        \def\x#1#2{\PackageInfo{#1}{#2, stopped}}%
      \fi
      \x{resizegather}{The package is already loaded}%
      \aftergroup\endinput
    \fi
  \fi
\endgroup%
%    \end{macrocode}
%    Package identification:
%    \begin{macrocode}
\begingroup\catcode61\catcode48\catcode32=10\relax%
  \catcode13=5 % ^^M
  \endlinechar=13 %
  \catcode35=6 % #
  \catcode39=12 % '
  \catcode40=12 % (
  \catcode41=12 % )
  \catcode44=12 % ,
  \catcode45=12 % -
  \catcode46=12 % .
  \catcode47=12 % /
  \catcode58=12 % :
  \catcode64=11 % @
  \catcode91=12 % [
  \catcode93=12 % ]
  \catcode123=1 % {
  \catcode125=2 % }
  \expandafter\ifx\csname ProvidesPackage\endcsname\relax
    \def\x#1#2#3[#4]{\endgroup
      \immediate\write-1{Package: #3 #4}%
      \xdef#1{#4}%
    }%
  \else
    \def\x#1#2[#3]{\endgroup
      #2[{#3}]%
      \ifx#1\@undefined
        \xdef#1{#3}%
      \fi
      \ifx#1\relax
        \xdef#1{#3}%
      \fi
    }%
  \fi
\expandafter\x\csname ver@resizegather.sty\endcsname
\ProvidesPackage{resizegather}%
  [2016/05/16 v1.3 Resize overly large equations (HO)]%
%    \end{macrocode}
%
%    \begin{macrocode}
\begingroup\catcode61\catcode48\catcode32=10\relax%
  \catcode13=5 % ^^M
  \endlinechar=13 %
  \catcode123=1 % {
  \catcode125=2 % }
  \catcode64=11 % @
  \def\x{\endgroup
    \expandafter\edef\csname ResizeGather@AtEnd\endcsname{%
      \endlinechar=\the\endlinechar\relax
      \catcode13=\the\catcode13\relax
      \catcode32=\the\catcode32\relax
      \catcode35=\the\catcode35\relax
      \catcode61=\the\catcode61\relax
      \catcode64=\the\catcode64\relax
      \catcode123=\the\catcode123\relax
      \catcode125=\the\catcode125\relax
    }%
  }%
\x\catcode61\catcode48\catcode32=10\relax%
\catcode13=5 % ^^M
\endlinechar=13 %
\catcode35=6 % #
\catcode64=11 % @
\catcode123=1 % {
\catcode125=2 % }
\def\TMP@EnsureCode#1#2{%
  \edef\ResizeGather@AtEnd{%
    \ResizeGather@AtEnd
    \catcode#1=\the\catcode#1\relax
  }%
  \catcode#1=#2\relax
}
\TMP@EnsureCode{10}{12}% ^^J
\TMP@EnsureCode{33}{12}% !
\TMP@EnsureCode{36}{3}% $
\TMP@EnsureCode{38}{4}% &
\TMP@EnsureCode{39}{12}% '
\TMP@EnsureCode{40}{12}% (
\TMP@EnsureCode{41}{12}% )
\TMP@EnsureCode{42}{12}% *
\TMP@EnsureCode{43}{12}% +
\TMP@EnsureCode{44}{12}% ,
\TMP@EnsureCode{45}{12}% -
\TMP@EnsureCode{46}{12}% .
\TMP@EnsureCode{47}{12}% /
\TMP@EnsureCode{58}{12}% :
\TMP@EnsureCode{59}{12}% ;
\TMP@EnsureCode{60}{12}% <
\TMP@EnsureCode{62}{12}% >
\TMP@EnsureCode{63}{12}% ?
\TMP@EnsureCode{91}{12}% [
\TMP@EnsureCode{93}{12}% ]
\TMP@EnsureCode{94}{7}% ^ (superscript)
\TMP@EnsureCode{96}{12}% `
\edef\ResizeGather@AtEnd{\ResizeGather@AtEnd\noexpand\endinput}
%    \end{macrocode}
%
%    \begin{macrocode}
\RequirePackage{kvoptions}[2009/12/04]
\SetupKeyvalOptions{%
  family=resizegather,%
  prefix=ResizeGather@,%
}
%    \end{macrocode}
%    \begin{macrocode}
\@for\ResizeGather@option:=%
  centertags,%
  tbtags,%
  sumlimits,%
  nosumlimits,%
  intlimits,%
  nointlimits,%
  nonamelimits,%
  leqno,%
  reqno,%
  fleqn%
\do{%
  \edef\ResizeGather@temp{%
    \noexpand\DeclareVoidOption{\ResizeGather@option}{%
      \noexpand\PassOptionsToPackage{amsmath}{\ResizeGather@option}%
    }%
    \noexpand\AtEndOfPackage{%
      \noexpand\DisableKeyvalOption[%
        action=error,%
        package=resizegather,%
      ]{resizegather}{\ResizeGather@option}%
    }%
  }%
  \ResizeGather@temp
}
\@for\ResizeGather@option:=%
  draft,%
  final,%
  hiderotate,%
  hidescale,%
  hiresbb,%
  demo,%
  dvips,xdvi,dvipdf,dvipdfm,dvipdfmx,pdftex,dvipsone,%
  dviwindo,emtex,dviwin,pctexps,pctexwin,pctexhp,pctex32,%
  truetex,tcidvi,vtex,oztex,textures,xetex%
\do{%
  \edef\ResizeGather@temp{%
    \noexpand\DeclareVoidOption{\ResizeGather@option}{%
      \noexpand\PassOptionsToPackage{graphics}{\ResizeGather@option}%
    }%
    \noexpand\AtEndOfPackage{%
      \noexpand\DisableKeyvalOption[%
        action=error,%
        package=resizegather,%
      ]{resizegather}{\ResizeGather@option}%
    }%
  }%
  \ResizeGather@temp
}
%    \end{macrocode}
%
%    \begin{macrocode}
\DeclareBoolOption[true]{enable}
\DeclareComplementaryOption{disable}{enable}
\DeclareStringOption[.05]{warningthreshold}
\newif\ifResizeGather@NeedInit
\DeclareBoolOption[true]{equations}
\DeclareBoolOption[true]{equation}
\DeclareBoolOption[true]{displaymath}
\AddToKeyvalOption*{equations}{%
  \ResizeGather@NeedInittrue
  \ifResizeGather@equations
    \ResizeGather@equationtrue
    \ResizeGather@displaymathtrue
  \else
    \ResizeGather@equationfalse
    \ResizeGather@displaymathfalse
  \fi
}
\AddToKeyvalOption*{equation}{%
  \ResizeGather@NeedInittrue
}
\AddToKeyvalOption*{displaymath}{%
  \ResizeGather@NeedInittrue
}
%    \end{macrocode}
%
%    \begin{macro}{\resizegathersetup}
%    \begin{macrocode}
\newcommand*{\resizegathersetup}[1]{%
  \ResizeGather@NeedInitfalse
  \setkeys{resizegather}{#1}%
  \ifResizeGather@NeedInit
    \ResizeGather@init
  \fi
}
\let\ResizeGather@init\relax
%    \end{macrocode}
%    \end{macro}
%    \begin{macrocode}
\InputIfFileExists{resizegather.cfg}{}{}%
\ProcessKeyvalOptions*\relax
%    \end{macrocode}
%    \begin{macrocode}
\RequirePackage{amsmath}
\RequirePackage{graphics}
%    \end{macrocode}
%    \begin{macro}{\ResizeGather@redefine}
%    \begin{macrocode}
\def\ResizeGather@redefine#1#2#3#4#5{%
  \csname ifResizeGather@#1\endcsname
    \@ifundefined{ResizeGather@org@#2}{%
      \expandafter\let\csname ResizeGather@org@#2\expandafter\endcsname
                      \csname #2\endcsname
    }{}%
    \@ifundefined{ResizeGather@org@#3}{%
      \expandafter\let\csname ResizeGather@org@#3\expandafter\endcsname
                      \csname #3\endcsname
    }{}%
    \expandafter\edef\csname #2\endcsname{%
      \expandafter\noexpand\csname#4\endcsname
    }%
    \expandafter\edef\csname #3\endcsname{%
      \expandafter\noexpand\csname#5\endcsname
    }%
  \else
    \@ifundefined{ResizeGather@org@#2}{}{%
      \expandafter\let\csname #2\expandafter\endcsname
                      \csname ResizeGather@org@#2\endcsname
      \expandafter\let\csname #3\expandafter\endcsname
                      \csname ResizeGather@org@#3\endcsname
    }%
  \fi
}
%    \end{macrocode}
%    \end{macro}
%    \begin{macro}{\ResizeGather@init}
%    \begin{macrocode}
\def\ResizeGather@init{%
  \ResizeGather@redefine{equation}{equation}{endequation}%
    {gather}{endgather}%
  \ResizeGather@redefine{displaymath}{displaymath}{enddisplaymath}%
    {gather*}{endgather*}%
}
\ResizeGather@init
%    \end{macrocode}
%    \end{macro}
%
%    \begin{macro}{\ResizeGather@ResizeGather}
%    \begin{macrocode}
\def\ResizeGather@ResizeGather{%
  \ifResizeGather@enable
    \dimen@\displaywidth
    \if@fleqn
      \advance\dimen@-\@mathmargin
    \fi
    \ifdim\wdz@>\dimen@
      \begingroup
        \advance\dimen@ -\wdz@
        \dimen@ -\dimen@
        \ifdim\ResizeGather@warningthreshold\wdz@>\dimen@
          \expandafter\PackageInfo
        \else
          \expandafter\PackageWarning
        \fi
        {resizegather}{%
          Equation line \the\row@\space is too large %
          by \the\dimen@\MessageBreak
          in environment `\@currenvir'%
        }%
      \endgroup
      \setboxz@h to\dimen@{%
        \resizebox{\dimen@}{!}{\boxz@}%
        \hss
      }%
    \fi
  \fi
}
%    \end{macrocode}
%    \end{macro}
%    \begin{macro}{\calc@shift@gather}
%    \begin{macrocode}
\expandafter\def\expandafter\calc@shift@gather\expandafter{%
  \expandafter\ResizeGather@ResizeGather
  \calc@shift@gather
}
%    \end{macrocode}
%    \end{macro}
%    \begin{macro}{\ResizeGather@org@gmeasure@}
%    \begin{macrocode}
\let\ResizeGather@org@gmeasure@\gmeasure@
%    \end{macrocode}
%    \end{macro}
%    \begin{macro}{\gmeasure@}
%    \begin{macrocode}
\def\gmeasure@#1{%
  \ResizeGather@org@gmeasure@{#1}%
  \ifResizeGather@enable
    \ifdim\totwidth@>\displaywidth
      \totwidth@=\displaywidth
    \fi
  \fi
}
%    \end{macrocode}
%    \end{macro}
%
%    \begin{macrocode}
\ResizeGather@AtEnd%
%</package>
%    \end{macrocode}
%
% \section{Test}
%
% \subsection{Catcode checks for loading}
%
%    \begin{macrocode}
%<*test1>
%    \end{macrocode}
%    \begin{macrocode}
\catcode`\{=1 %
\catcode`\}=2 %
\catcode`\#=6 %
\catcode`\@=11 %
\expandafter\ifx\csname count@\endcsname\relax
  \countdef\count@=255 %
\fi
\expandafter\ifx\csname @gobble\endcsname\relax
  \long\def\@gobble#1{}%
\fi
\expandafter\ifx\csname @firstofone\endcsname\relax
  \long\def\@firstofone#1{#1}%
\fi
\expandafter\ifx\csname loop\endcsname\relax
  \expandafter\@firstofone
\else
  \expandafter\@gobble
\fi
{%
  \def\loop#1\repeat{%
    \def\body{#1}%
    \iterate
  }%
  \def\iterate{%
    \body
      \let\next\iterate
    \else
      \let\next\relax
    \fi
    \next
  }%
  \let\repeat=\fi
}%
\def\RestoreCatcodes{}
\count@=0 %
\loop
  \edef\RestoreCatcodes{%
    \RestoreCatcodes
    \catcode\the\count@=\the\catcode\count@\relax
  }%
\ifnum\count@<255 %
  \advance\count@ 1 %
\repeat

\def\RangeCatcodeInvalid#1#2{%
  \count@=#1\relax
  \loop
    \catcode\count@=15 %
  \ifnum\count@<#2\relax
    \advance\count@ 1 %
  \repeat
}
\def\RangeCatcodeCheck#1#2#3{%
  \count@=#1\relax
  \loop
    \ifnum#3=\catcode\count@
    \else
      \errmessage{%
        Character \the\count@\space
        with wrong catcode \the\catcode\count@\space
        instead of \number#3%
      }%
    \fi
  \ifnum\count@<#2\relax
    \advance\count@ 1 %
  \repeat
}
\def\space{ }
\expandafter\ifx\csname LoadCommand\endcsname\relax
  \def\LoadCommand{\input resizegather.sty\relax}%
\fi
\def\Test{%
  \RangeCatcodeInvalid{0}{47}%
  \RangeCatcodeInvalid{58}{64}%
  \RangeCatcodeInvalid{91}{96}%
  \RangeCatcodeInvalid{123}{255}%
  \catcode`\@=12 %
  \catcode`\\=0 %
  \catcode`\%=14 %
  \LoadCommand
  \RangeCatcodeCheck{0}{36}{15}%
  \RangeCatcodeCheck{37}{37}{14}%
  \RangeCatcodeCheck{38}{47}{15}%
  \RangeCatcodeCheck{48}{57}{12}%
  \RangeCatcodeCheck{58}{63}{15}%
  \RangeCatcodeCheck{64}{64}{12}%
  \RangeCatcodeCheck{65}{90}{11}%
  \RangeCatcodeCheck{91}{91}{15}%
  \RangeCatcodeCheck{92}{92}{0}%
  \RangeCatcodeCheck{93}{96}{15}%
  \RangeCatcodeCheck{97}{122}{11}%
  \RangeCatcodeCheck{123}{255}{15}%
  \RestoreCatcodes
}
\Test
\csname @@end\endcsname
\end
%    \end{macrocode}
%    \begin{macrocode}
%</test1>
%    \end{macrocode}
%
% \section{Installation}
%
% \subsection{Download}
%
% \paragraph{Package.} This package is available on
% CTAN\footnote{\url{https://ctan.org/pkg/resizegather}}:
% \begin{description}
% \item[\CTAN{macros/latex/contrib/oberdiek/resizegather.dtx}] The source file.
% \item[\CTAN{macros/latex/contrib/oberdiek/resizegather.pdf}] Documentation.
% \end{description}
%
%
% \paragraph{Bundle.} All the packages of the bundle `oberdiek'
% are also available in a TDS compliant ZIP archive. There
% the packages are already unpacked and the documentation files
% are generated. The files and directories obey the TDS standard.
% \begin{description}
% \item[\CTANinstall{install/macros/latex/contrib/oberdiek.tds.zip}]
% \end{description}
% \emph{TDS} refers to the standard ``A Directory Structure
% for \TeX\ Files'' (\CTAN{tds/tds.pdf}). Directories
% with \xfile{texmf} in their name are usually organized this way.
%
% \subsection{Bundle installation}
%
% \paragraph{Unpacking.} Unpack the \xfile{oberdiek.tds.zip} in the
% TDS tree (also known as \xfile{texmf} tree) of your choice.
% Example (linux):
% \begin{quote}
%   |unzip oberdiek.tds.zip -d ~/texmf|
% \end{quote}
%
% \paragraph{Script installation.}
% Check the directory \xfile{TDS:scripts/oberdiek/} for
% scripts that need further installation steps.
% Package \xpackage{attachfile2} comes with the Perl script
% \xfile{pdfatfi.pl} that should be installed in such a way
% that it can be called as \texttt{pdfatfi}.
% Example (linux):
% \begin{quote}
%   |chmod +x scripts/oberdiek/pdfatfi.pl|\\
%   |cp scripts/oberdiek/pdfatfi.pl /usr/local/bin/|
% \end{quote}
%
% \subsection{Package installation}
%
% \paragraph{Unpacking.} The \xfile{.dtx} file is a self-extracting
% \docstrip\ archive. The files are extracted by running the
% \xfile{.dtx} through \plainTeX:
% \begin{quote}
%   \verb|tex resizegather.dtx|
% \end{quote}
%
% \paragraph{TDS.} Now the different files must be moved into
% the different directories in your installation TDS tree
% (also known as \xfile{texmf} tree):
% \begin{quote}
% \def\t{^^A
% \begin{tabular}{@{}>{\ttfamily}l@{ $\rightarrow$ }>{\ttfamily}l@{}}
%   resizegather.sty & tex/latex/oberdiek/resizegather.sty\\
%   resizegather.pdf & doc/latex/oberdiek/resizegather.pdf\\
%   test/resizegather-test1.tex & doc/latex/oberdiek/test/resizegather-test1.tex\\
%   resizegather.dtx & source/latex/oberdiek/resizegather.dtx\\
% \end{tabular}^^A
% }^^A
% \sbox0{\t}^^A
% \ifdim\wd0>\linewidth
%   \begingroup
%     \advance\linewidth by\leftmargin
%     \advance\linewidth by\rightmargin
%   \edef\x{\endgroup
%     \def\noexpand\lw{\the\linewidth}^^A
%   }\x
%   \def\lwbox{^^A
%     \leavevmode
%     \hbox to \linewidth{^^A
%       \kern-\leftmargin\relax
%       \hss
%       \usebox0
%       \hss
%       \kern-\rightmargin\relax
%     }^^A
%   }^^A
%   \ifdim\wd0>\lw
%     \sbox0{\small\t}^^A
%     \ifdim\wd0>\linewidth
%       \ifdim\wd0>\lw
%         \sbox0{\footnotesize\t}^^A
%         \ifdim\wd0>\linewidth
%           \ifdim\wd0>\lw
%             \sbox0{\scriptsize\t}^^A
%             \ifdim\wd0>\linewidth
%               \ifdim\wd0>\lw
%                 \sbox0{\tiny\t}^^A
%                 \ifdim\wd0>\linewidth
%                   \lwbox
%                 \else
%                   \usebox0
%                 \fi
%               \else
%                 \lwbox
%               \fi
%             \else
%               \usebox0
%             \fi
%           \else
%             \lwbox
%           \fi
%         \else
%           \usebox0
%         \fi
%       \else
%         \lwbox
%       \fi
%     \else
%       \usebox0
%     \fi
%   \else
%     \lwbox
%   \fi
% \else
%   \usebox0
% \fi
% \end{quote}
% If you have a \xfile{docstrip.cfg} that configures and enables \docstrip's
% TDS installing feature, then some files can already be in the right
% place, see the documentation of \docstrip.
%
% \subsection{Refresh file name databases}
%
% If your \TeX~distribution
% (\teTeX, \mikTeX, \dots) relies on file name databases, you must refresh
% these. For example, \teTeX\ users run \verb|texhash| or
% \verb|mktexlsr|.
%
% \subsection{Some details for the interested}
%
% \paragraph{Attached source.}
%
% The PDF documentation on CTAN also includes the
% \xfile{.dtx} source file. It can be extracted by
% AcrobatReader 6 or higher. Another option is \textsf{pdftk},
% e.g. unpack the file into the current directory:
% \begin{quote}
%   \verb|pdftk resizegather.pdf unpack_files output .|
% \end{quote}
%
% \paragraph{Unpacking with \LaTeX.}
% The \xfile{.dtx} chooses its action depending on the format:
% \begin{description}
% \item[\plainTeX:] Run \docstrip\ and extract the files.
% \item[\LaTeX:] Generate the documentation.
% \end{description}
% If you insist on using \LaTeX\ for \docstrip\ (really,
% \docstrip\ does not need \LaTeX), then inform the autodetect routine
% about your intention:
% \begin{quote}
%   \verb|latex \let\install=y% \iffalse meta-comment
%
% File: resizegather.dtx
% Version: 2016/05/16 v1.3
% Info: Resize overly large equations
%
% Copyright (C) 2009, 2010 by
%    Heiko Oberdiek <heiko.oberdiek at googlemail.com>
%    2016
%    https://github.com/ho-tex/oberdiek/issues
%
% This work may be distributed and/or modified under the
% conditions of the LaTeX Project Public License, either
% version 1.3c of this license or (at your option) any later
% version. This version of this license is in
%    http://www.latex-project.org/lppl/lppl-1-3c.txt
% and the latest version of this license is in
%    http://www.latex-project.org/lppl.txt
% and version 1.3 or later is part of all distributions of
% LaTeX version 2005/12/01 or later.
%
% This work has the LPPL maintenance status "maintained".
%
% This Current Maintainer of this work is Heiko Oberdiek.
%
% This work consists of the main source file resizegather.dtx
% and the derived files
%    resizegather.sty, resizegather.pdf, resizegather.ins, resizegather.drv,
%    resizegather-test1.tex.
%
% Distribution:
%    CTAN:macros/latex/contrib/oberdiek/resizegather.dtx
%    CTAN:macros/latex/contrib/oberdiek/resizegather.pdf
%
% Unpacking:
%    (a) If resizegather.ins is present:
%           tex resizegather.ins
%    (b) Without resizegather.ins:
%           tex resizegather.dtx
%    (c) If you insist on using LaTeX
%           latex \let\install=y\input{resizegather.dtx}
%        (quote the arguments according to the demands of your shell)
%
% Documentation:
%    (a) If resizegather.drv is present:
%           latex resizegather.drv
%    (b) Without resizegather.drv:
%           latex resizegather.dtx; ...
%    The class ltxdoc loads the configuration file ltxdoc.cfg
%    if available. Here you can specify further options, e.g.
%    use A4 as paper format:
%       \PassOptionsToClass{a4paper}{article}
%
%    Programm calls to get the documentation (example):
%       pdflatex resizegather.dtx
%       makeindex -s gind.ist resizegather.idx
%       pdflatex resizegather.dtx
%       makeindex -s gind.ist resizegather.idx
%       pdflatex resizegather.dtx
%
% Installation:
%    TDS:tex/latex/oberdiek/resizegather.sty
%    TDS:doc/latex/oberdiek/resizegather.pdf
%    TDS:doc/latex/oberdiek/test/resizegather-test1.tex
%    TDS:source/latex/oberdiek/resizegather.dtx
%
%<*ignore>
\begingroup
  \catcode123=1 %
  \catcode125=2 %
  \def\x{LaTeX2e}%
\expandafter\endgroup
\ifcase 0\ifx\install y1\fi\expandafter
         \ifx\csname processbatchFile\endcsname\relax\else1\fi
         \ifx\fmtname\x\else 1\fi\relax
\else\csname fi\endcsname
%</ignore>
%<*install>
\input docstrip.tex
\Msg{************************************************************************}
\Msg{* Installation}
\Msg{* Package: resizegather 2016/05/16 v1.3 Resize overly large equations (HO)}
\Msg{************************************************************************}

\keepsilent
\askforoverwritefalse

\let\MetaPrefix\relax
\preamble

This is a generated file.

Project: resizegather
Version: 2016/05/16 v1.3

Copyright (C) 2009, 2010 by
   Heiko Oberdiek <heiko.oberdiek at googlemail.com>

This work may be distributed and/or modified under the
conditions of the LaTeX Project Public License, either
version 1.3c of this license or (at your option) any later
version. This version of this license is in
   http://www.latex-project.org/lppl/lppl-1-3c.txt
and the latest version of this license is in
   http://www.latex-project.org/lppl.txt
and version 1.3 or later is part of all distributions of
LaTeX version 2005/12/01 or later.

This work has the LPPL maintenance status "maintained".

This Current Maintainer of this work is Heiko Oberdiek.

This work consists of the main source file resizegather.dtx
and the derived files
   resizegather.sty, resizegather.pdf, resizegather.ins, resizegather.drv,
   resizegather-test1.tex.

\endpreamble
\let\MetaPrefix\DoubleperCent

\generate{%
  \file{resizegather.ins}{\from{resizegather.dtx}{install}}%
  \file{resizegather.drv}{\from{resizegather.dtx}{driver}}%
  \usedir{tex/latex/oberdiek}%
  \file{resizegather.sty}{\from{resizegather.dtx}{package}}%
%  \usedir{doc/latex/oberdiek/test}%
%  \file{resizegather-test1.tex}{\from{resizegather.dtx}{test1}}%
  \nopreamble
  \nopostamble
%  \usedir{source/latex/oberdiek/catalogue}%
%  \file{resizegather.xml}{\from{resizegather.dtx}{catalogue}}%
}

\catcode32=13\relax% active space
\let =\space%
\Msg{************************************************************************}
\Msg{*}
\Msg{* To finish the installation you have to move the following}
\Msg{* file into a directory searched by TeX:}
\Msg{*}
\Msg{*     resizegather.sty}
\Msg{*}
\Msg{* To produce the documentation run the file `resizegather.drv'}
\Msg{* through LaTeX.}
\Msg{*}
\Msg{* Happy TeXing!}
\Msg{*}
\Msg{************************************************************************}

\endbatchfile
%</install>
%<*ignore>
\fi
%</ignore>
%<*driver>
\NeedsTeXFormat{LaTeX2e}
\ProvidesFile{resizegather.drv}%
  [2016/05/16 v1.3 Resize overly large equations (HO)]%
\documentclass{ltxdoc}
\usepackage{holtxdoc}[2011/11/22]
\usepackage{ifluatex}
\ifluatex
\else
  \usepackage[T1]{fontenc}%
  \usepackage{textcomp}%
  \usepackage{lmodern}%
\fi
\begin{document}
  \DocInput{resizegather.dtx}%
\end{document}
%</driver>
% \fi
%
%
% \CharacterTable
%  {Upper-case    \A\B\C\D\E\F\G\H\I\J\K\L\M\N\O\P\Q\R\S\T\U\V\W\X\Y\Z
%   Lower-case    \a\b\c\d\e\f\g\h\i\j\k\l\m\n\o\p\q\r\s\t\u\v\w\x\y\z
%   Digits        \0\1\2\3\4\5\6\7\8\9
%   Exclamation   \!     Double quote  \"     Hash (number) \#
%   Dollar        \$     Percent       \%     Ampersand     \&
%   Acute accent  \'     Left paren    \(     Right paren   \)
%   Asterisk      \*     Plus          \+     Comma         \,
%   Minus         \-     Point         \.     Solidus       \/
%   Colon         \:     Semicolon     \;     Less than     \<
%   Equals        \=     Greater than  \>     Question mark \?
%   Commercial at \@     Left bracket  \[     Backslash     \\
%   Right bracket \]     Circumflex    \^     Underscore    \_
%   Grave accent  \`     Left brace    \{     Vertical bar  \|
%   Right brace   \}     Tilde         \~}
%
% \GetFileInfo{resizegather.drv}
%
% \title{The \xpackage{resizegather} package}
% \date{2016/05/16 v1.3}
% \author{Heiko Oberdiek\thanks
% {Please report any issues at https://github.com/ho-tex/oberdiek/issues}\\
% \xemail{heiko.oberdiek at googlemail.com}}
%
% \maketitle
%
% \begin{abstract}
% Equations that are too large are resized to fit the available
% space. The environment \textsf{gather} of package \xpackage{amsmath}
% is supported. Also the environments \textsf{equation} and
% \textsf{displaymath} are redefined using \textsf{gather}
% and its starred version.
% \end{abstract}
%
% \tableofcontents
%
% \makeatletter
% \def\env#1{^^A
%    \textsf{\@env#1*\@nil}^^A
% }%
% \def\@env#1*#2\@nil{^^A
%   #1^^A
%   \ifx\\#2\\^^A
%     \expandafter\@gobble
%   \else
%     \textasteriskcentered
%     \expandafter\@firstofone
%   \fi
%   {\@env#2\@nil}^^A
% }
% \makeatother
%
% \section{Documentation}
%
% Sometimes an equation is just a little to large to fit in the
% line. And breaking the equation across lines might be worse
% than downscaling the equation. This package implements this
% for the environments \env{gather} and \env{gather*} of
% package \xpackage{amsmath}. That package already measures
% the equations and simplifies the implementation of \xpackage{resizegather}
% that only needs to hook into \xpackage{amsmath}'s code to add
% the resizing feature.
%
% Resized equations are recorded in the \xext{log} file
% for small exceeds (default setting is smaller than five percent).
% Otherwise a warning is given.
%
% Also environments \env{equation} and \env{displaymath}
% are supported by redefining them using \env{gather}
% and \env{gather*}.
%
% \cs{[} and \cs{]} are not supported, because these macros
% are not in environment form that is required for
% \xpackage{amsmath}. The environment body is collected
% first to be able to process the body twice for measuring
% first.
%
% Also the environments using alignments are not supported.
% If a single equation line would be resized, the alignment
% would get lost. And resizing all equations of the alignment
% does not seem appropriate either.
%
% \subsection{Options}
%
% \begin{description}
% \item[\xoption{warningthreshold}:]
%   Print a warning if the original equation line exceeds
%   its available width by the given fraction.
%   Default is |0.05|: A warning is given if the equation
%   is too large by five percent.
%   Otherwise the exceed is recorded in the \xext{log} file
%   only.
% \end{description}
% The next options are boolean options. They are enabled
% by value |true| or if no value is given. They are switched
% off by value |false|.
% \begin{description}
% \item[\xoption{enable}:] The resize feature is active (default).
% \item[\xoption{disable}:] The complementary option for \xoption{enable},
%  added for convenience: |disable| (or |disable=true|) is the same
%  as |enable=false|.
% \item[\xoption{equations}:]
%   \LaTeX\ environments \textsf{equation} and \textsf{displaymath}
%   environments are redefined. These equations
%   are now using environment \env{gather} and
%   \env{gather*}. This is the default.
% \end{description}
% The following table shows additional options if you
% want to have finer control for the redefined
% environments:
% \begin{quote}
% \def\unchanged{\textit{unchanged}}
% \def\notprovided{\textit{not provided}}
% \begin{tabular}{l|ll}
% &\multicolumn{2}{c}{Environments}\\
% Option & \env{equation} & \env{displaymath}\\
% \hline
% \xoption{equations} & \env{gather} & \env{gather*}\\
% \xoption{equation} & \env{gather} & \unchanged\\
% \xoption{displaymath} & \unchanged & \env{gather*}\\
% \end{tabular}
% \end{quote}
% If such an option is switched off, the original meaning
% of the affected environments is restored.
%
% Options are evaluated in the following order:
% \begin{enumerate}
% \item
%  Configuration file \xfile{resizegather.cfg} using \cs{resizegathersetup}
%  if the file exists.
%  \item
%  Package options given for \cs{usepackage}.
%  \item
%  Later calls of \cs{resizegathersetup}.
% \end{enumerate}
% \begin{declcs}{resizegathersetup}\M{option list}
% \end{declcs}
% The options are key value options. Boolean options are enabled by
% default (without value) or by using the explicit value \texttt{true}.
% Value \texttt{false} disable the option.
%
% \subsection{Options for packages \xpackage{amsmath} or \xpackage{graphics}}
%
% The package loads the package \xpackage{amsmath} because is internally
% measures the equations first. Thus this package hooks into this code
% in order to resize the equations if they are too large.
% The resizing itself is done by \cs{resizebox} of package \xpackage{graphics}.
% If you need special options for these packages, just load them first or
% use global options when appropriate. Example:
% \begin{quote}
%\begin{verbatim}
%\usepackage[dvipdfm]{graphicx}% or graphics
%\usepackage[fleqn]{amsmath}
%\usepackage{resizegather}
%\end{verbatim}
%\end{quote}
%
% \StopEventually{
% }
%
% \section{Implementation}
%    \begin{macrocode}
%<*package>
%    \end{macrocode}
%    Reload check, especially if the package is not used with \LaTeX.
%    \begin{macrocode}
\begingroup\catcode61\catcode48\catcode32=10\relax%
  \catcode13=5 % ^^M
  \endlinechar=13 %
  \catcode35=6 % #
  \catcode39=12 % '
  \catcode44=12 % ,
  \catcode45=12 % -
  \catcode46=12 % .
  \catcode58=12 % :
  \catcode64=11 % @
  \catcode123=1 % {
  \catcode125=2 % }
  \expandafter\let\expandafter\x\csname ver@resizegather.sty\endcsname
  \ifx\x\relax % plain-TeX, first loading
  \else
    \def\empty{}%
    \ifx\x\empty % LaTeX, first loading,
      % variable is initialized, but \ProvidesPackage not yet seen
    \else
      \expandafter\ifx\csname PackageInfo\endcsname\relax
        \def\x#1#2{%
          \immediate\write-1{Package #1 Info: #2.}%
        }%
      \else
        \def\x#1#2{\PackageInfo{#1}{#2, stopped}}%
      \fi
      \x{resizegather}{The package is already loaded}%
      \aftergroup\endinput
    \fi
  \fi
\endgroup%
%    \end{macrocode}
%    Package identification:
%    \begin{macrocode}
\begingroup\catcode61\catcode48\catcode32=10\relax%
  \catcode13=5 % ^^M
  \endlinechar=13 %
  \catcode35=6 % #
  \catcode39=12 % '
  \catcode40=12 % (
  \catcode41=12 % )
  \catcode44=12 % ,
  \catcode45=12 % -
  \catcode46=12 % .
  \catcode47=12 % /
  \catcode58=12 % :
  \catcode64=11 % @
  \catcode91=12 % [
  \catcode93=12 % ]
  \catcode123=1 % {
  \catcode125=2 % }
  \expandafter\ifx\csname ProvidesPackage\endcsname\relax
    \def\x#1#2#3[#4]{\endgroup
      \immediate\write-1{Package: #3 #4}%
      \xdef#1{#4}%
    }%
  \else
    \def\x#1#2[#3]{\endgroup
      #2[{#3}]%
      \ifx#1\@undefined
        \xdef#1{#3}%
      \fi
      \ifx#1\relax
        \xdef#1{#3}%
      \fi
    }%
  \fi
\expandafter\x\csname ver@resizegather.sty\endcsname
\ProvidesPackage{resizegather}%
  [2016/05/16 v1.3 Resize overly large equations (HO)]%
%    \end{macrocode}
%
%    \begin{macrocode}
\begingroup\catcode61\catcode48\catcode32=10\relax%
  \catcode13=5 % ^^M
  \endlinechar=13 %
  \catcode123=1 % {
  \catcode125=2 % }
  \catcode64=11 % @
  \def\x{\endgroup
    \expandafter\edef\csname ResizeGather@AtEnd\endcsname{%
      \endlinechar=\the\endlinechar\relax
      \catcode13=\the\catcode13\relax
      \catcode32=\the\catcode32\relax
      \catcode35=\the\catcode35\relax
      \catcode61=\the\catcode61\relax
      \catcode64=\the\catcode64\relax
      \catcode123=\the\catcode123\relax
      \catcode125=\the\catcode125\relax
    }%
  }%
\x\catcode61\catcode48\catcode32=10\relax%
\catcode13=5 % ^^M
\endlinechar=13 %
\catcode35=6 % #
\catcode64=11 % @
\catcode123=1 % {
\catcode125=2 % }
\def\TMP@EnsureCode#1#2{%
  \edef\ResizeGather@AtEnd{%
    \ResizeGather@AtEnd
    \catcode#1=\the\catcode#1\relax
  }%
  \catcode#1=#2\relax
}
\TMP@EnsureCode{10}{12}% ^^J
\TMP@EnsureCode{33}{12}% !
\TMP@EnsureCode{36}{3}% $
\TMP@EnsureCode{38}{4}% &
\TMP@EnsureCode{39}{12}% '
\TMP@EnsureCode{40}{12}% (
\TMP@EnsureCode{41}{12}% )
\TMP@EnsureCode{42}{12}% *
\TMP@EnsureCode{43}{12}% +
\TMP@EnsureCode{44}{12}% ,
\TMP@EnsureCode{45}{12}% -
\TMP@EnsureCode{46}{12}% .
\TMP@EnsureCode{47}{12}% /
\TMP@EnsureCode{58}{12}% :
\TMP@EnsureCode{59}{12}% ;
\TMP@EnsureCode{60}{12}% <
\TMP@EnsureCode{62}{12}% >
\TMP@EnsureCode{63}{12}% ?
\TMP@EnsureCode{91}{12}% [
\TMP@EnsureCode{93}{12}% ]
\TMP@EnsureCode{94}{7}% ^ (superscript)
\TMP@EnsureCode{96}{12}% `
\edef\ResizeGather@AtEnd{\ResizeGather@AtEnd\noexpand\endinput}
%    \end{macrocode}
%
%    \begin{macrocode}
\RequirePackage{kvoptions}[2009/12/04]
\SetupKeyvalOptions{%
  family=resizegather,%
  prefix=ResizeGather@,%
}
%    \end{macrocode}
%    \begin{macrocode}
\@for\ResizeGather@option:=%
  centertags,%
  tbtags,%
  sumlimits,%
  nosumlimits,%
  intlimits,%
  nointlimits,%
  nonamelimits,%
  leqno,%
  reqno,%
  fleqn%
\do{%
  \edef\ResizeGather@temp{%
    \noexpand\DeclareVoidOption{\ResizeGather@option}{%
      \noexpand\PassOptionsToPackage{amsmath}{\ResizeGather@option}%
    }%
    \noexpand\AtEndOfPackage{%
      \noexpand\DisableKeyvalOption[%
        action=error,%
        package=resizegather,%
      ]{resizegather}{\ResizeGather@option}%
    }%
  }%
  \ResizeGather@temp
}
\@for\ResizeGather@option:=%
  draft,%
  final,%
  hiderotate,%
  hidescale,%
  hiresbb,%
  demo,%
  dvips,xdvi,dvipdf,dvipdfm,dvipdfmx,pdftex,dvipsone,%
  dviwindo,emtex,dviwin,pctexps,pctexwin,pctexhp,pctex32,%
  truetex,tcidvi,vtex,oztex,textures,xetex%
\do{%
  \edef\ResizeGather@temp{%
    \noexpand\DeclareVoidOption{\ResizeGather@option}{%
      \noexpand\PassOptionsToPackage{graphics}{\ResizeGather@option}%
    }%
    \noexpand\AtEndOfPackage{%
      \noexpand\DisableKeyvalOption[%
        action=error,%
        package=resizegather,%
      ]{resizegather}{\ResizeGather@option}%
    }%
  }%
  \ResizeGather@temp
}
%    \end{macrocode}
%
%    \begin{macrocode}
\DeclareBoolOption[true]{enable}
\DeclareComplementaryOption{disable}{enable}
\DeclareStringOption[.05]{warningthreshold}
\newif\ifResizeGather@NeedInit
\DeclareBoolOption[true]{equations}
\DeclareBoolOption[true]{equation}
\DeclareBoolOption[true]{displaymath}
\AddToKeyvalOption*{equations}{%
  \ResizeGather@NeedInittrue
  \ifResizeGather@equations
    \ResizeGather@equationtrue
    \ResizeGather@displaymathtrue
  \else
    \ResizeGather@equationfalse
    \ResizeGather@displaymathfalse
  \fi
}
\AddToKeyvalOption*{equation}{%
  \ResizeGather@NeedInittrue
}
\AddToKeyvalOption*{displaymath}{%
  \ResizeGather@NeedInittrue
}
%    \end{macrocode}
%
%    \begin{macro}{\resizegathersetup}
%    \begin{macrocode}
\newcommand*{\resizegathersetup}[1]{%
  \ResizeGather@NeedInitfalse
  \setkeys{resizegather}{#1}%
  \ifResizeGather@NeedInit
    \ResizeGather@init
  \fi
}
\let\ResizeGather@init\relax
%    \end{macrocode}
%    \end{macro}
%    \begin{macrocode}
\InputIfFileExists{resizegather.cfg}{}{}%
\ProcessKeyvalOptions*\relax
%    \end{macrocode}
%    \begin{macrocode}
\RequirePackage{amsmath}
\RequirePackage{graphics}
%    \end{macrocode}
%    \begin{macro}{\ResizeGather@redefine}
%    \begin{macrocode}
\def\ResizeGather@redefine#1#2#3#4#5{%
  \csname ifResizeGather@#1\endcsname
    \@ifundefined{ResizeGather@org@#2}{%
      \expandafter\let\csname ResizeGather@org@#2\expandafter\endcsname
                      \csname #2\endcsname
    }{}%
    \@ifundefined{ResizeGather@org@#3}{%
      \expandafter\let\csname ResizeGather@org@#3\expandafter\endcsname
                      \csname #3\endcsname
    }{}%
    \expandafter\edef\csname #2\endcsname{%
      \expandafter\noexpand\csname#4\endcsname
    }%
    \expandafter\edef\csname #3\endcsname{%
      \expandafter\noexpand\csname#5\endcsname
    }%
  \else
    \@ifundefined{ResizeGather@org@#2}{}{%
      \expandafter\let\csname #2\expandafter\endcsname
                      \csname ResizeGather@org@#2\endcsname
      \expandafter\let\csname #3\expandafter\endcsname
                      \csname ResizeGather@org@#3\endcsname
    }%
  \fi
}
%    \end{macrocode}
%    \end{macro}
%    \begin{macro}{\ResizeGather@init}
%    \begin{macrocode}
\def\ResizeGather@init{%
  \ResizeGather@redefine{equation}{equation}{endequation}%
    {gather}{endgather}%
  \ResizeGather@redefine{displaymath}{displaymath}{enddisplaymath}%
    {gather*}{endgather*}%
}
\ResizeGather@init
%    \end{macrocode}
%    \end{macro}
%
%    \begin{macro}{\ResizeGather@ResizeGather}
%    \begin{macrocode}
\def\ResizeGather@ResizeGather{%
  \ifResizeGather@enable
    \dimen@\displaywidth
    \if@fleqn
      \advance\dimen@-\@mathmargin
    \fi
    \ifdim\wdz@>\dimen@
      \begingroup
        \advance\dimen@ -\wdz@
        \dimen@ -\dimen@
        \ifdim\ResizeGather@warningthreshold\wdz@>\dimen@
          \expandafter\PackageInfo
        \else
          \expandafter\PackageWarning
        \fi
        {resizegather}{%
          Equation line \the\row@\space is too large %
          by \the\dimen@\MessageBreak
          in environment `\@currenvir'%
        }%
      \endgroup
      \setboxz@h to\dimen@{%
        \resizebox{\dimen@}{!}{\boxz@}%
        \hss
      }%
    \fi
  \fi
}
%    \end{macrocode}
%    \end{macro}
%    \begin{macro}{\calc@shift@gather}
%    \begin{macrocode}
\expandafter\def\expandafter\calc@shift@gather\expandafter{%
  \expandafter\ResizeGather@ResizeGather
  \calc@shift@gather
}
%    \end{macrocode}
%    \end{macro}
%    \begin{macro}{\ResizeGather@org@gmeasure@}
%    \begin{macrocode}
\let\ResizeGather@org@gmeasure@\gmeasure@
%    \end{macrocode}
%    \end{macro}
%    \begin{macro}{\gmeasure@}
%    \begin{macrocode}
\def\gmeasure@#1{%
  \ResizeGather@org@gmeasure@{#1}%
  \ifResizeGather@enable
    \ifdim\totwidth@>\displaywidth
      \totwidth@=\displaywidth
    \fi
  \fi
}
%    \end{macrocode}
%    \end{macro}
%
%    \begin{macrocode}
\ResizeGather@AtEnd%
%</package>
%    \end{macrocode}
%
% \section{Test}
%
% \subsection{Catcode checks for loading}
%
%    \begin{macrocode}
%<*test1>
%    \end{macrocode}
%    \begin{macrocode}
\catcode`\{=1 %
\catcode`\}=2 %
\catcode`\#=6 %
\catcode`\@=11 %
\expandafter\ifx\csname count@\endcsname\relax
  \countdef\count@=255 %
\fi
\expandafter\ifx\csname @gobble\endcsname\relax
  \long\def\@gobble#1{}%
\fi
\expandafter\ifx\csname @firstofone\endcsname\relax
  \long\def\@firstofone#1{#1}%
\fi
\expandafter\ifx\csname loop\endcsname\relax
  \expandafter\@firstofone
\else
  \expandafter\@gobble
\fi
{%
  \def\loop#1\repeat{%
    \def\body{#1}%
    \iterate
  }%
  \def\iterate{%
    \body
      \let\next\iterate
    \else
      \let\next\relax
    \fi
    \next
  }%
  \let\repeat=\fi
}%
\def\RestoreCatcodes{}
\count@=0 %
\loop
  \edef\RestoreCatcodes{%
    \RestoreCatcodes
    \catcode\the\count@=\the\catcode\count@\relax
  }%
\ifnum\count@<255 %
  \advance\count@ 1 %
\repeat

\def\RangeCatcodeInvalid#1#2{%
  \count@=#1\relax
  \loop
    \catcode\count@=15 %
  \ifnum\count@<#2\relax
    \advance\count@ 1 %
  \repeat
}
\def\RangeCatcodeCheck#1#2#3{%
  \count@=#1\relax
  \loop
    \ifnum#3=\catcode\count@
    \else
      \errmessage{%
        Character \the\count@\space
        with wrong catcode \the\catcode\count@\space
        instead of \number#3%
      }%
    \fi
  \ifnum\count@<#2\relax
    \advance\count@ 1 %
  \repeat
}
\def\space{ }
\expandafter\ifx\csname LoadCommand\endcsname\relax
  \def\LoadCommand{\input resizegather.sty\relax}%
\fi
\def\Test{%
  \RangeCatcodeInvalid{0}{47}%
  \RangeCatcodeInvalid{58}{64}%
  \RangeCatcodeInvalid{91}{96}%
  \RangeCatcodeInvalid{123}{255}%
  \catcode`\@=12 %
  \catcode`\\=0 %
  \catcode`\%=14 %
  \LoadCommand
  \RangeCatcodeCheck{0}{36}{15}%
  \RangeCatcodeCheck{37}{37}{14}%
  \RangeCatcodeCheck{38}{47}{15}%
  \RangeCatcodeCheck{48}{57}{12}%
  \RangeCatcodeCheck{58}{63}{15}%
  \RangeCatcodeCheck{64}{64}{12}%
  \RangeCatcodeCheck{65}{90}{11}%
  \RangeCatcodeCheck{91}{91}{15}%
  \RangeCatcodeCheck{92}{92}{0}%
  \RangeCatcodeCheck{93}{96}{15}%
  \RangeCatcodeCheck{97}{122}{11}%
  \RangeCatcodeCheck{123}{255}{15}%
  \RestoreCatcodes
}
\Test
\csname @@end\endcsname
\end
%    \end{macrocode}
%    \begin{macrocode}
%</test1>
%    \end{macrocode}
%
% \section{Installation}
%
% \subsection{Download}
%
% \paragraph{Package.} This package is available on
% CTAN\footnote{\url{https://ctan.org/pkg/resizegather}}:
% \begin{description}
% \item[\CTAN{macros/latex/contrib/oberdiek/resizegather.dtx}] The source file.
% \item[\CTAN{macros/latex/contrib/oberdiek/resizegather.pdf}] Documentation.
% \end{description}
%
%
% \paragraph{Bundle.} All the packages of the bundle `oberdiek'
% are also available in a TDS compliant ZIP archive. There
% the packages are already unpacked and the documentation files
% are generated. The files and directories obey the TDS standard.
% \begin{description}
% \item[\CTANinstall{install/macros/latex/contrib/oberdiek.tds.zip}]
% \end{description}
% \emph{TDS} refers to the standard ``A Directory Structure
% for \TeX\ Files'' (\CTAN{tds/tds.pdf}). Directories
% with \xfile{texmf} in their name are usually organized this way.
%
% \subsection{Bundle installation}
%
% \paragraph{Unpacking.} Unpack the \xfile{oberdiek.tds.zip} in the
% TDS tree (also known as \xfile{texmf} tree) of your choice.
% Example (linux):
% \begin{quote}
%   |unzip oberdiek.tds.zip -d ~/texmf|
% \end{quote}
%
% \paragraph{Script installation.}
% Check the directory \xfile{TDS:scripts/oberdiek/} for
% scripts that need further installation steps.
% Package \xpackage{attachfile2} comes with the Perl script
% \xfile{pdfatfi.pl} that should be installed in such a way
% that it can be called as \texttt{pdfatfi}.
% Example (linux):
% \begin{quote}
%   |chmod +x scripts/oberdiek/pdfatfi.pl|\\
%   |cp scripts/oberdiek/pdfatfi.pl /usr/local/bin/|
% \end{quote}
%
% \subsection{Package installation}
%
% \paragraph{Unpacking.} The \xfile{.dtx} file is a self-extracting
% \docstrip\ archive. The files are extracted by running the
% \xfile{.dtx} through \plainTeX:
% \begin{quote}
%   \verb|tex resizegather.dtx|
% \end{quote}
%
% \paragraph{TDS.} Now the different files must be moved into
% the different directories in your installation TDS tree
% (also known as \xfile{texmf} tree):
% \begin{quote}
% \def\t{^^A
% \begin{tabular}{@{}>{\ttfamily}l@{ $\rightarrow$ }>{\ttfamily}l@{}}
%   resizegather.sty & tex/latex/oberdiek/resizegather.sty\\
%   resizegather.pdf & doc/latex/oberdiek/resizegather.pdf\\
%   test/resizegather-test1.tex & doc/latex/oberdiek/test/resizegather-test1.tex\\
%   resizegather.dtx & source/latex/oberdiek/resizegather.dtx\\
% \end{tabular}^^A
% }^^A
% \sbox0{\t}^^A
% \ifdim\wd0>\linewidth
%   \begingroup
%     \advance\linewidth by\leftmargin
%     \advance\linewidth by\rightmargin
%   \edef\x{\endgroup
%     \def\noexpand\lw{\the\linewidth}^^A
%   }\x
%   \def\lwbox{^^A
%     \leavevmode
%     \hbox to \linewidth{^^A
%       \kern-\leftmargin\relax
%       \hss
%       \usebox0
%       \hss
%       \kern-\rightmargin\relax
%     }^^A
%   }^^A
%   \ifdim\wd0>\lw
%     \sbox0{\small\t}^^A
%     \ifdim\wd0>\linewidth
%       \ifdim\wd0>\lw
%         \sbox0{\footnotesize\t}^^A
%         \ifdim\wd0>\linewidth
%           \ifdim\wd0>\lw
%             \sbox0{\scriptsize\t}^^A
%             \ifdim\wd0>\linewidth
%               \ifdim\wd0>\lw
%                 \sbox0{\tiny\t}^^A
%                 \ifdim\wd0>\linewidth
%                   \lwbox
%                 \else
%                   \usebox0
%                 \fi
%               \else
%                 \lwbox
%               \fi
%             \else
%               \usebox0
%             \fi
%           \else
%             \lwbox
%           \fi
%         \else
%           \usebox0
%         \fi
%       \else
%         \lwbox
%       \fi
%     \else
%       \usebox0
%     \fi
%   \else
%     \lwbox
%   \fi
% \else
%   \usebox0
% \fi
% \end{quote}
% If you have a \xfile{docstrip.cfg} that configures and enables \docstrip's
% TDS installing feature, then some files can already be in the right
% place, see the documentation of \docstrip.
%
% \subsection{Refresh file name databases}
%
% If your \TeX~distribution
% (\teTeX, \mikTeX, \dots) relies on file name databases, you must refresh
% these. For example, \teTeX\ users run \verb|texhash| or
% \verb|mktexlsr|.
%
% \subsection{Some details for the interested}
%
% \paragraph{Attached source.}
%
% The PDF documentation on CTAN also includes the
% \xfile{.dtx} source file. It can be extracted by
% AcrobatReader 6 or higher. Another option is \textsf{pdftk},
% e.g. unpack the file into the current directory:
% \begin{quote}
%   \verb|pdftk resizegather.pdf unpack_files output .|
% \end{quote}
%
% \paragraph{Unpacking with \LaTeX.}
% The \xfile{.dtx} chooses its action depending on the format:
% \begin{description}
% \item[\plainTeX:] Run \docstrip\ and extract the files.
% \item[\LaTeX:] Generate the documentation.
% \end{description}
% If you insist on using \LaTeX\ for \docstrip\ (really,
% \docstrip\ does not need \LaTeX), then inform the autodetect routine
% about your intention:
% \begin{quote}
%   \verb|latex \let\install=y\input{resizegather.dtx}|
% \end{quote}
% Do not forget to quote the argument according to the demands
% of your shell.
%
% \paragraph{Generating the documentation.}
% You can use both the \xfile{.dtx} or the \xfile{.drv} to generate
% the documentation. The process can be configured by the
% configuration file \xfile{ltxdoc.cfg}. For instance, put this
% line into this file, if you want to have A4 as paper format:
% \begin{quote}
%   \verb|\PassOptionsToClass{a4paper}{article}|
% \end{quote}
% An example follows how to generate the
% documentation with pdf\LaTeX:
% \begin{quote}
%\begin{verbatim}
%pdflatex resizegather.dtx
%makeindex -s gind.ist resizegather.idx
%pdflatex resizegather.dtx
%makeindex -s gind.ist resizegather.idx
%pdflatex resizegather.dtx
%\end{verbatim}
% \end{quote}
%
% \section{Acknowledgement}
%
% \begin{description}
% \item[Dieter Jurzitza:]
% He wanted the resizing feature at the \TeX\ table
% in Karlsruhe of December 2009. Thus this package is a kind of
% Christmas present.
% \end{description}
%
% \begin{History}
%   \begin{Version}{2009/12/04 v1.0}
%   \item
%     The first version.
%   \end{Version}
%   \begin{Version}{2009/12/05 v1.1}
%   \item
%     Options \xoption{enable} and \xoption{disable} added.
%   \end{Version}
%   \begin{Version}{2010/03/01 v1.2}
%   \item
%     TDS location moved from `generic' to `latex'.
%   \end{Version}
%   \begin{Version}{2016/05/16 v1.3}
%   \item
%     Documentation updates.
%   \end{Version}
% \end{History}
%
% \PrintIndex
%
% \Finale
\endinput
|
% \end{quote}
% Do not forget to quote the argument according to the demands
% of your shell.
%
% \paragraph{Generating the documentation.}
% You can use both the \xfile{.dtx} or the \xfile{.drv} to generate
% the documentation. The process can be configured by the
% configuration file \xfile{ltxdoc.cfg}. For instance, put this
% line into this file, if you want to have A4 as paper format:
% \begin{quote}
%   \verb|\PassOptionsToClass{a4paper}{article}|
% \end{quote}
% An example follows how to generate the
% documentation with pdf\LaTeX:
% \begin{quote}
%\begin{verbatim}
%pdflatex resizegather.dtx
%makeindex -s gind.ist resizegather.idx
%pdflatex resizegather.dtx
%makeindex -s gind.ist resizegather.idx
%pdflatex resizegather.dtx
%\end{verbatim}
% \end{quote}
%
% \section{Acknowledgement}
%
% \begin{description}
% \item[Dieter Jurzitza:]
% He wanted the resizing feature at the \TeX\ table
% in Karlsruhe of December 2009. Thus this package is a kind of
% Christmas present.
% \end{description}
%
% \begin{History}
%   \begin{Version}{2009/12/04 v1.0}
%   \item
%     The first version.
%   \end{Version}
%   \begin{Version}{2009/12/05 v1.1}
%   \item
%     Options \xoption{enable} and \xoption{disable} added.
%   \end{Version}
%   \begin{Version}{2010/03/01 v1.2}
%   \item
%     TDS location moved from `generic' to `latex'.
%   \end{Version}
%   \begin{Version}{2016/05/16 v1.3}
%   \item
%     Documentation updates.
%   \end{Version}
% \end{History}
%
% \PrintIndex
%
% \Finale
\endinput

%        (quote the arguments according to the demands of your shell)
%
% Documentation:
%    (a) If resizegather.drv is present:
%           latex resizegather.drv
%    (b) Without resizegather.drv:
%           latex resizegather.dtx; ...
%    The class ltxdoc loads the configuration file ltxdoc.cfg
%    if available. Here you can specify further options, e.g.
%    use A4 as paper format:
%       \PassOptionsToClass{a4paper}{article}
%
%    Programm calls to get the documentation (example):
%       pdflatex resizegather.dtx
%       makeindex -s gind.ist resizegather.idx
%       pdflatex resizegather.dtx
%       makeindex -s gind.ist resizegather.idx
%       pdflatex resizegather.dtx
%
% Installation:
%    TDS:tex/latex/oberdiek/resizegather.sty
%    TDS:doc/latex/oberdiek/resizegather.pdf
%    TDS:doc/latex/oberdiek/test/resizegather-test1.tex
%    TDS:source/latex/oberdiek/resizegather.dtx
%
%<*ignore>
\begingroup
  \catcode123=1 %
  \catcode125=2 %
  \def\x{LaTeX2e}%
\expandafter\endgroup
\ifcase 0\ifx\install y1\fi\expandafter
         \ifx\csname processbatchFile\endcsname\relax\else1\fi
         \ifx\fmtname\x\else 1\fi\relax
\else\csname fi\endcsname
%</ignore>
%<*install>
\input docstrip.tex
\Msg{************************************************************************}
\Msg{* Installation}
\Msg{* Package: resizegather 2016/05/16 v1.3 Resize overly large equations (HO)}
\Msg{************************************************************************}

\keepsilent
\askforoverwritefalse

\let\MetaPrefix\relax
\preamble

This is a generated file.

Project: resizegather
Version: 2016/05/16 v1.3

Copyright (C) 2009, 2010 by
   Heiko Oberdiek <heiko.oberdiek at googlemail.com>

This work may be distributed and/or modified under the
conditions of the LaTeX Project Public License, either
version 1.3c of this license or (at your option) any later
version. This version of this license is in
   http://www.latex-project.org/lppl/lppl-1-3c.txt
and the latest version of this license is in
   http://www.latex-project.org/lppl.txt
and version 1.3 or later is part of all distributions of
LaTeX version 2005/12/01 or later.

This work has the LPPL maintenance status "maintained".

This Current Maintainer of this work is Heiko Oberdiek.

This work consists of the main source file resizegather.dtx
and the derived files
   resizegather.sty, resizegather.pdf, resizegather.ins, resizegather.drv,
   resizegather-test1.tex.

\endpreamble
\let\MetaPrefix\DoubleperCent

\generate{%
  \file{resizegather.ins}{\from{resizegather.dtx}{install}}%
  \file{resizegather.drv}{\from{resizegather.dtx}{driver}}%
  \usedir{tex/latex/oberdiek}%
  \file{resizegather.sty}{\from{resizegather.dtx}{package}}%
%  \usedir{doc/latex/oberdiek/test}%
%  \file{resizegather-test1.tex}{\from{resizegather.dtx}{test1}}%
  \nopreamble
  \nopostamble
%  \usedir{source/latex/oberdiek/catalogue}%
%  \file{resizegather.xml}{\from{resizegather.dtx}{catalogue}}%
}

\catcode32=13\relax% active space
\let =\space%
\Msg{************************************************************************}
\Msg{*}
\Msg{* To finish the installation you have to move the following}
\Msg{* file into a directory searched by TeX:}
\Msg{*}
\Msg{*     resizegather.sty}
\Msg{*}
\Msg{* To produce the documentation run the file `resizegather.drv'}
\Msg{* through LaTeX.}
\Msg{*}
\Msg{* Happy TeXing!}
\Msg{*}
\Msg{************************************************************************}

\endbatchfile
%</install>
%<*ignore>
\fi
%</ignore>
%<*driver>
\NeedsTeXFormat{LaTeX2e}
\ProvidesFile{resizegather.drv}%
  [2016/05/16 v1.3 Resize overly large equations (HO)]%
\documentclass{ltxdoc}
\usepackage{holtxdoc}[2011/11/22]
\usepackage{ifluatex}
\ifluatex
\else
  \usepackage[T1]{fontenc}%
  \usepackage{textcomp}%
  \usepackage{lmodern}%
\fi
\begin{document}
  \DocInput{resizegather.dtx}%
\end{document}
%</driver>
% \fi
%
%
% \CharacterTable
%  {Upper-case    \A\B\C\D\E\F\G\H\I\J\K\L\M\N\O\P\Q\R\S\T\U\V\W\X\Y\Z
%   Lower-case    \a\b\c\d\e\f\g\h\i\j\k\l\m\n\o\p\q\r\s\t\u\v\w\x\y\z
%   Digits        \0\1\2\3\4\5\6\7\8\9
%   Exclamation   \!     Double quote  \"     Hash (number) \#
%   Dollar        \$     Percent       \%     Ampersand     \&
%   Acute accent  \'     Left paren    \(     Right paren   \)
%   Asterisk      \*     Plus          \+     Comma         \,
%   Minus         \-     Point         \.     Solidus       \/
%   Colon         \:     Semicolon     \;     Less than     \<
%   Equals        \=     Greater than  \>     Question mark \?
%   Commercial at \@     Left bracket  \[     Backslash     \\
%   Right bracket \]     Circumflex    \^     Underscore    \_
%   Grave accent  \`     Left brace    \{     Vertical bar  \|
%   Right brace   \}     Tilde         \~}
%
% \GetFileInfo{resizegather.drv}
%
% \title{The \xpackage{resizegather} package}
% \date{2016/05/16 v1.3}
% \author{Heiko Oberdiek\thanks
% {Please report any issues at https://github.com/ho-tex/oberdiek/issues}\\
% \xemail{heiko.oberdiek at googlemail.com}}
%
% \maketitle
%
% \begin{abstract}
% Equations that are too large are resized to fit the available
% space. The environment \textsf{gather} of package \xpackage{amsmath}
% is supported. Also the environments \textsf{equation} and
% \textsf{displaymath} are redefined using \textsf{gather}
% and its starred version.
% \end{abstract}
%
% \tableofcontents
%
% \makeatletter
% \def\env#1{^^A
%    \textsf{\@env#1*\@nil}^^A
% }%
% \def\@env#1*#2\@nil{^^A
%   #1^^A
%   \ifx\\#2\\^^A
%     \expandafter\@gobble
%   \else
%     \textasteriskcentered
%     \expandafter\@firstofone
%   \fi
%   {\@env#2\@nil}^^A
% }
% \makeatother
%
% \section{Documentation}
%
% Sometimes an equation is just a little to large to fit in the
% line. And breaking the equation across lines might be worse
% than downscaling the equation. This package implements this
% for the environments \env{gather} and \env{gather*} of
% package \xpackage{amsmath}. That package already measures
% the equations and simplifies the implementation of \xpackage{resizegather}
% that only needs to hook into \xpackage{amsmath}'s code to add
% the resizing feature.
%
% Resized equations are recorded in the \xext{log} file
% for small exceeds (default setting is smaller than five percent).
% Otherwise a warning is given.
%
% Also environments \env{equation} and \env{displaymath}
% are supported by redefining them using \env{gather}
% and \env{gather*}.
%
% \cs{[} and \cs{]} are not supported, because these macros
% are not in environment form that is required for
% \xpackage{amsmath}. The environment body is collected
% first to be able to process the body twice for measuring
% first.
%
% Also the environments using alignments are not supported.
% If a single equation line would be resized, the alignment
% would get lost. And resizing all equations of the alignment
% does not seem appropriate either.
%
% \subsection{Options}
%
% \begin{description}
% \item[\xoption{warningthreshold}:]
%   Print a warning if the original equation line exceeds
%   its available width by the given fraction.
%   Default is |0.05|: A warning is given if the equation
%   is too large by five percent.
%   Otherwise the exceed is recorded in the \xext{log} file
%   only.
% \end{description}
% The next options are boolean options. They are enabled
% by value |true| or if no value is given. They are switched
% off by value |false|.
% \begin{description}
% \item[\xoption{enable}:] The resize feature is active (default).
% \item[\xoption{disable}:] The complementary option for \xoption{enable},
%  added for convenience: |disable| (or |disable=true|) is the same
%  as |enable=false|.
% \item[\xoption{equations}:]
%   \LaTeX\ environments \textsf{equation} and \textsf{displaymath}
%   environments are redefined. These equations
%   are now using environment \env{gather} and
%   \env{gather*}. This is the default.
% \end{description}
% The following table shows additional options if you
% want to have finer control for the redefined
% environments:
% \begin{quote}
% \def\unchanged{\textit{unchanged}}
% \def\notprovided{\textit{not provided}}
% \begin{tabular}{l|ll}
% &\multicolumn{2}{c}{Environments}\\
% Option & \env{equation} & \env{displaymath}\\
% \hline
% \xoption{equations} & \env{gather} & \env{gather*}\\
% \xoption{equation} & \env{gather} & \unchanged\\
% \xoption{displaymath} & \unchanged & \env{gather*}\\
% \end{tabular}
% \end{quote}
% If such an option is switched off, the original meaning
% of the affected environments is restored.
%
% Options are evaluated in the following order:
% \begin{enumerate}
% \item
%  Configuration file \xfile{resizegather.cfg} using \cs{resizegathersetup}
%  if the file exists.
%  \item
%  Package options given for \cs{usepackage}.
%  \item
%  Later calls of \cs{resizegathersetup}.
% \end{enumerate}
% \begin{declcs}{resizegathersetup}\M{option list}
% \end{declcs}
% The options are key value options. Boolean options are enabled by
% default (without value) or by using the explicit value \texttt{true}.
% Value \texttt{false} disable the option.
%
% \subsection{Options for packages \xpackage{amsmath} or \xpackage{graphics}}
%
% The package loads the package \xpackage{amsmath} because is internally
% measures the equations first. Thus this package hooks into this code
% in order to resize the equations if they are too large.
% The resizing itself is done by \cs{resizebox} of package \xpackage{graphics}.
% If you need special options for these packages, just load them first or
% use global options when appropriate. Example:
% \begin{quote}
%\begin{verbatim}
%\usepackage[dvipdfm]{graphicx}% or graphics
%\usepackage[fleqn]{amsmath}
%\usepackage{resizegather}
%\end{verbatim}
%\end{quote}
%
% \StopEventually{
% }
%
% \section{Implementation}
%    \begin{macrocode}
%<*package>
%    \end{macrocode}
%    Reload check, especially if the package is not used with \LaTeX.
%    \begin{macrocode}
\begingroup\catcode61\catcode48\catcode32=10\relax%
  \catcode13=5 % ^^M
  \endlinechar=13 %
  \catcode35=6 % #
  \catcode39=12 % '
  \catcode44=12 % ,
  \catcode45=12 % -
  \catcode46=12 % .
  \catcode58=12 % :
  \catcode64=11 % @
  \catcode123=1 % {
  \catcode125=2 % }
  \expandafter\let\expandafter\x\csname ver@resizegather.sty\endcsname
  \ifx\x\relax % plain-TeX, first loading
  \else
    \def\empty{}%
    \ifx\x\empty % LaTeX, first loading,
      % variable is initialized, but \ProvidesPackage not yet seen
    \else
      \expandafter\ifx\csname PackageInfo\endcsname\relax
        \def\x#1#2{%
          \immediate\write-1{Package #1 Info: #2.}%
        }%
      \else
        \def\x#1#2{\PackageInfo{#1}{#2, stopped}}%
      \fi
      \x{resizegather}{The package is already loaded}%
      \aftergroup\endinput
    \fi
  \fi
\endgroup%
%    \end{macrocode}
%    Package identification:
%    \begin{macrocode}
\begingroup\catcode61\catcode48\catcode32=10\relax%
  \catcode13=5 % ^^M
  \endlinechar=13 %
  \catcode35=6 % #
  \catcode39=12 % '
  \catcode40=12 % (
  \catcode41=12 % )
  \catcode44=12 % ,
  \catcode45=12 % -
  \catcode46=12 % .
  \catcode47=12 % /
  \catcode58=12 % :
  \catcode64=11 % @
  \catcode91=12 % [
  \catcode93=12 % ]
  \catcode123=1 % {
  \catcode125=2 % }
  \expandafter\ifx\csname ProvidesPackage\endcsname\relax
    \def\x#1#2#3[#4]{\endgroup
      \immediate\write-1{Package: #3 #4}%
      \xdef#1{#4}%
    }%
  \else
    \def\x#1#2[#3]{\endgroup
      #2[{#3}]%
      \ifx#1\@undefined
        \xdef#1{#3}%
      \fi
      \ifx#1\relax
        \xdef#1{#3}%
      \fi
    }%
  \fi
\expandafter\x\csname ver@resizegather.sty\endcsname
\ProvidesPackage{resizegather}%
  [2016/05/16 v1.3 Resize overly large equations (HO)]%
%    \end{macrocode}
%
%    \begin{macrocode}
\begingroup\catcode61\catcode48\catcode32=10\relax%
  \catcode13=5 % ^^M
  \endlinechar=13 %
  \catcode123=1 % {
  \catcode125=2 % }
  \catcode64=11 % @
  \def\x{\endgroup
    \expandafter\edef\csname ResizeGather@AtEnd\endcsname{%
      \endlinechar=\the\endlinechar\relax
      \catcode13=\the\catcode13\relax
      \catcode32=\the\catcode32\relax
      \catcode35=\the\catcode35\relax
      \catcode61=\the\catcode61\relax
      \catcode64=\the\catcode64\relax
      \catcode123=\the\catcode123\relax
      \catcode125=\the\catcode125\relax
    }%
  }%
\x\catcode61\catcode48\catcode32=10\relax%
\catcode13=5 % ^^M
\endlinechar=13 %
\catcode35=6 % #
\catcode64=11 % @
\catcode123=1 % {
\catcode125=2 % }
\def\TMP@EnsureCode#1#2{%
  \edef\ResizeGather@AtEnd{%
    \ResizeGather@AtEnd
    \catcode#1=\the\catcode#1\relax
  }%
  \catcode#1=#2\relax
}
\TMP@EnsureCode{10}{12}% ^^J
\TMP@EnsureCode{33}{12}% !
\TMP@EnsureCode{36}{3}% $
\TMP@EnsureCode{38}{4}% &
\TMP@EnsureCode{39}{12}% '
\TMP@EnsureCode{40}{12}% (
\TMP@EnsureCode{41}{12}% )
\TMP@EnsureCode{42}{12}% *
\TMP@EnsureCode{43}{12}% +
\TMP@EnsureCode{44}{12}% ,
\TMP@EnsureCode{45}{12}% -
\TMP@EnsureCode{46}{12}% .
\TMP@EnsureCode{47}{12}% /
\TMP@EnsureCode{58}{12}% :
\TMP@EnsureCode{59}{12}% ;
\TMP@EnsureCode{60}{12}% <
\TMP@EnsureCode{62}{12}% >
\TMP@EnsureCode{63}{12}% ?
\TMP@EnsureCode{91}{12}% [
\TMP@EnsureCode{93}{12}% ]
\TMP@EnsureCode{94}{7}% ^ (superscript)
\TMP@EnsureCode{96}{12}% `
\edef\ResizeGather@AtEnd{\ResizeGather@AtEnd\noexpand\endinput}
%    \end{macrocode}
%
%    \begin{macrocode}
\RequirePackage{kvoptions}[2009/12/04]
\SetupKeyvalOptions{%
  family=resizegather,%
  prefix=ResizeGather@,%
}
%    \end{macrocode}
%    \begin{macrocode}
\@for\ResizeGather@option:=%
  centertags,%
  tbtags,%
  sumlimits,%
  nosumlimits,%
  intlimits,%
  nointlimits,%
  nonamelimits,%
  leqno,%
  reqno,%
  fleqn%
\do{%
  \edef\ResizeGather@temp{%
    \noexpand\DeclareVoidOption{\ResizeGather@option}{%
      \noexpand\PassOptionsToPackage{amsmath}{\ResizeGather@option}%
    }%
    \noexpand\AtEndOfPackage{%
      \noexpand\DisableKeyvalOption[%
        action=error,%
        package=resizegather,%
      ]{resizegather}{\ResizeGather@option}%
    }%
  }%
  \ResizeGather@temp
}
\@for\ResizeGather@option:=%
  draft,%
  final,%
  hiderotate,%
  hidescale,%
  hiresbb,%
  demo,%
  dvips,xdvi,dvipdf,dvipdfm,dvipdfmx,pdftex,dvipsone,%
  dviwindo,emtex,dviwin,pctexps,pctexwin,pctexhp,pctex32,%
  truetex,tcidvi,vtex,oztex,textures,xetex%
\do{%
  \edef\ResizeGather@temp{%
    \noexpand\DeclareVoidOption{\ResizeGather@option}{%
      \noexpand\PassOptionsToPackage{graphics}{\ResizeGather@option}%
    }%
    \noexpand\AtEndOfPackage{%
      \noexpand\DisableKeyvalOption[%
        action=error,%
        package=resizegather,%
      ]{resizegather}{\ResizeGather@option}%
    }%
  }%
  \ResizeGather@temp
}
%    \end{macrocode}
%
%    \begin{macrocode}
\DeclareBoolOption[true]{enable}
\DeclareComplementaryOption{disable}{enable}
\DeclareStringOption[.05]{warningthreshold}
\newif\ifResizeGather@NeedInit
\DeclareBoolOption[true]{equations}
\DeclareBoolOption[true]{equation}
\DeclareBoolOption[true]{displaymath}
\AddToKeyvalOption*{equations}{%
  \ResizeGather@NeedInittrue
  \ifResizeGather@equations
    \ResizeGather@equationtrue
    \ResizeGather@displaymathtrue
  \else
    \ResizeGather@equationfalse
    \ResizeGather@displaymathfalse
  \fi
}
\AddToKeyvalOption*{equation}{%
  \ResizeGather@NeedInittrue
}
\AddToKeyvalOption*{displaymath}{%
  \ResizeGather@NeedInittrue
}
%    \end{macrocode}
%
%    \begin{macro}{\resizegathersetup}
%    \begin{macrocode}
\newcommand*{\resizegathersetup}[1]{%
  \ResizeGather@NeedInitfalse
  \setkeys{resizegather}{#1}%
  \ifResizeGather@NeedInit
    \ResizeGather@init
  \fi
}
\let\ResizeGather@init\relax
%    \end{macrocode}
%    \end{macro}
%    \begin{macrocode}
\InputIfFileExists{resizegather.cfg}{}{}%
\ProcessKeyvalOptions*\relax
%    \end{macrocode}
%    \begin{macrocode}
\RequirePackage{amsmath}
\RequirePackage{graphics}
%    \end{macrocode}
%    \begin{macro}{\ResizeGather@redefine}
%    \begin{macrocode}
\def\ResizeGather@redefine#1#2#3#4#5{%
  \csname ifResizeGather@#1\endcsname
    \@ifundefined{ResizeGather@org@#2}{%
      \expandafter\let\csname ResizeGather@org@#2\expandafter\endcsname
                      \csname #2\endcsname
    }{}%
    \@ifundefined{ResizeGather@org@#3}{%
      \expandafter\let\csname ResizeGather@org@#3\expandafter\endcsname
                      \csname #3\endcsname
    }{}%
    \expandafter\edef\csname #2\endcsname{%
      \expandafter\noexpand\csname#4\endcsname
    }%
    \expandafter\edef\csname #3\endcsname{%
      \expandafter\noexpand\csname#5\endcsname
    }%
  \else
    \@ifundefined{ResizeGather@org@#2}{}{%
      \expandafter\let\csname #2\expandafter\endcsname
                      \csname ResizeGather@org@#2\endcsname
      \expandafter\let\csname #3\expandafter\endcsname
                      \csname ResizeGather@org@#3\endcsname
    }%
  \fi
}
%    \end{macrocode}
%    \end{macro}
%    \begin{macro}{\ResizeGather@init}
%    \begin{macrocode}
\def\ResizeGather@init{%
  \ResizeGather@redefine{equation}{equation}{endequation}%
    {gather}{endgather}%
  \ResizeGather@redefine{displaymath}{displaymath}{enddisplaymath}%
    {gather*}{endgather*}%
}
\ResizeGather@init
%    \end{macrocode}
%    \end{macro}
%
%    \begin{macro}{\ResizeGather@ResizeGather}
%    \begin{macrocode}
\def\ResizeGather@ResizeGather{%
  \ifResizeGather@enable
    \dimen@\displaywidth
    \if@fleqn
      \advance\dimen@-\@mathmargin
    \fi
    \ifdim\wdz@>\dimen@
      \begingroup
        \advance\dimen@ -\wdz@
        \dimen@ -\dimen@
        \ifdim\ResizeGather@warningthreshold\wdz@>\dimen@
          \expandafter\PackageInfo
        \else
          \expandafter\PackageWarning
        \fi
        {resizegather}{%
          Equation line \the\row@\space is too large %
          by \the\dimen@\MessageBreak
          in environment `\@currenvir'%
        }%
      \endgroup
      \setboxz@h to\dimen@{%
        \resizebox{\dimen@}{!}{\boxz@}%
        \hss
      }%
    \fi
  \fi
}
%    \end{macrocode}
%    \end{macro}
%    \begin{macro}{\calc@shift@gather}
%    \begin{macrocode}
\expandafter\def\expandafter\calc@shift@gather\expandafter{%
  \expandafter\ResizeGather@ResizeGather
  \calc@shift@gather
}
%    \end{macrocode}
%    \end{macro}
%    \begin{macro}{\ResizeGather@org@gmeasure@}
%    \begin{macrocode}
\let\ResizeGather@org@gmeasure@\gmeasure@
%    \end{macrocode}
%    \end{macro}
%    \begin{macro}{\gmeasure@}
%    \begin{macrocode}
\def\gmeasure@#1{%
  \ResizeGather@org@gmeasure@{#1}%
  \ifResizeGather@enable
    \ifdim\totwidth@>\displaywidth
      \totwidth@=\displaywidth
    \fi
  \fi
}
%    \end{macrocode}
%    \end{macro}
%
%    \begin{macrocode}
\ResizeGather@AtEnd%
%</package>
%    \end{macrocode}
%
% \section{Test}
%
% \subsection{Catcode checks for loading}
%
%    \begin{macrocode}
%<*test1>
%    \end{macrocode}
%    \begin{macrocode}
\catcode`\{=1 %
\catcode`\}=2 %
\catcode`\#=6 %
\catcode`\@=11 %
\expandafter\ifx\csname count@\endcsname\relax
  \countdef\count@=255 %
\fi
\expandafter\ifx\csname @gobble\endcsname\relax
  \long\def\@gobble#1{}%
\fi
\expandafter\ifx\csname @firstofone\endcsname\relax
  \long\def\@firstofone#1{#1}%
\fi
\expandafter\ifx\csname loop\endcsname\relax
  \expandafter\@firstofone
\else
  \expandafter\@gobble
\fi
{%
  \def\loop#1\repeat{%
    \def\body{#1}%
    \iterate
  }%
  \def\iterate{%
    \body
      \let\next\iterate
    \else
      \let\next\relax
    \fi
    \next
  }%
  \let\repeat=\fi
}%
\def\RestoreCatcodes{}
\count@=0 %
\loop
  \edef\RestoreCatcodes{%
    \RestoreCatcodes
    \catcode\the\count@=\the\catcode\count@\relax
  }%
\ifnum\count@<255 %
  \advance\count@ 1 %
\repeat

\def\RangeCatcodeInvalid#1#2{%
  \count@=#1\relax
  \loop
    \catcode\count@=15 %
  \ifnum\count@<#2\relax
    \advance\count@ 1 %
  \repeat
}
\def\RangeCatcodeCheck#1#2#3{%
  \count@=#1\relax
  \loop
    \ifnum#3=\catcode\count@
    \else
      \errmessage{%
        Character \the\count@\space
        with wrong catcode \the\catcode\count@\space
        instead of \number#3%
      }%
    \fi
  \ifnum\count@<#2\relax
    \advance\count@ 1 %
  \repeat
}
\def\space{ }
\expandafter\ifx\csname LoadCommand\endcsname\relax
  \def\LoadCommand{\input resizegather.sty\relax}%
\fi
\def\Test{%
  \RangeCatcodeInvalid{0}{47}%
  \RangeCatcodeInvalid{58}{64}%
  \RangeCatcodeInvalid{91}{96}%
  \RangeCatcodeInvalid{123}{255}%
  \catcode`\@=12 %
  \catcode`\\=0 %
  \catcode`\%=14 %
  \LoadCommand
  \RangeCatcodeCheck{0}{36}{15}%
  \RangeCatcodeCheck{37}{37}{14}%
  \RangeCatcodeCheck{38}{47}{15}%
  \RangeCatcodeCheck{48}{57}{12}%
  \RangeCatcodeCheck{58}{63}{15}%
  \RangeCatcodeCheck{64}{64}{12}%
  \RangeCatcodeCheck{65}{90}{11}%
  \RangeCatcodeCheck{91}{91}{15}%
  \RangeCatcodeCheck{92}{92}{0}%
  \RangeCatcodeCheck{93}{96}{15}%
  \RangeCatcodeCheck{97}{122}{11}%
  \RangeCatcodeCheck{123}{255}{15}%
  \RestoreCatcodes
}
\Test
\csname @@end\endcsname
\end
%    \end{macrocode}
%    \begin{macrocode}
%</test1>
%    \end{macrocode}
%
% \section{Installation}
%
% \subsection{Download}
%
% \paragraph{Package.} This package is available on
% CTAN\footnote{\url{https://ctan.org/pkg/resizegather}}:
% \begin{description}
% \item[\CTAN{macros/latex/contrib/oberdiek/resizegather.dtx}] The source file.
% \item[\CTAN{macros/latex/contrib/oberdiek/resizegather.pdf}] Documentation.
% \end{description}
%
%
% \paragraph{Bundle.} All the packages of the bundle `oberdiek'
% are also available in a TDS compliant ZIP archive. There
% the packages are already unpacked and the documentation files
% are generated. The files and directories obey the TDS standard.
% \begin{description}
% \item[\CTANinstall{install/macros/latex/contrib/oberdiek.tds.zip}]
% \end{description}
% \emph{TDS} refers to the standard ``A Directory Structure
% for \TeX\ Files'' (\CTAN{tds/tds.pdf}). Directories
% with \xfile{texmf} in their name are usually organized this way.
%
% \subsection{Bundle installation}
%
% \paragraph{Unpacking.} Unpack the \xfile{oberdiek.tds.zip} in the
% TDS tree (also known as \xfile{texmf} tree) of your choice.
% Example (linux):
% \begin{quote}
%   |unzip oberdiek.tds.zip -d ~/texmf|
% \end{quote}
%
% \paragraph{Script installation.}
% Check the directory \xfile{TDS:scripts/oberdiek/} for
% scripts that need further installation steps.
% Package \xpackage{attachfile2} comes with the Perl script
% \xfile{pdfatfi.pl} that should be installed in such a way
% that it can be called as \texttt{pdfatfi}.
% Example (linux):
% \begin{quote}
%   |chmod +x scripts/oberdiek/pdfatfi.pl|\\
%   |cp scripts/oberdiek/pdfatfi.pl /usr/local/bin/|
% \end{quote}
%
% \subsection{Package installation}
%
% \paragraph{Unpacking.} The \xfile{.dtx} file is a self-extracting
% \docstrip\ archive. The files are extracted by running the
% \xfile{.dtx} through \plainTeX:
% \begin{quote}
%   \verb|tex resizegather.dtx|
% \end{quote}
%
% \paragraph{TDS.} Now the different files must be moved into
% the different directories in your installation TDS tree
% (also known as \xfile{texmf} tree):
% \begin{quote}
% \def\t{^^A
% \begin{tabular}{@{}>{\ttfamily}l@{ $\rightarrow$ }>{\ttfamily}l@{}}
%   resizegather.sty & tex/latex/oberdiek/resizegather.sty\\
%   resizegather.pdf & doc/latex/oberdiek/resizegather.pdf\\
%   test/resizegather-test1.tex & doc/latex/oberdiek/test/resizegather-test1.tex\\
%   resizegather.dtx & source/latex/oberdiek/resizegather.dtx\\
% \end{tabular}^^A
% }^^A
% \sbox0{\t}^^A
% \ifdim\wd0>\linewidth
%   \begingroup
%     \advance\linewidth by\leftmargin
%     \advance\linewidth by\rightmargin
%   \edef\x{\endgroup
%     \def\noexpand\lw{\the\linewidth}^^A
%   }\x
%   \def\lwbox{^^A
%     \leavevmode
%     \hbox to \linewidth{^^A
%       \kern-\leftmargin\relax
%       \hss
%       \usebox0
%       \hss
%       \kern-\rightmargin\relax
%     }^^A
%   }^^A
%   \ifdim\wd0>\lw
%     \sbox0{\small\t}^^A
%     \ifdim\wd0>\linewidth
%       \ifdim\wd0>\lw
%         \sbox0{\footnotesize\t}^^A
%         \ifdim\wd0>\linewidth
%           \ifdim\wd0>\lw
%             \sbox0{\scriptsize\t}^^A
%             \ifdim\wd0>\linewidth
%               \ifdim\wd0>\lw
%                 \sbox0{\tiny\t}^^A
%                 \ifdim\wd0>\linewidth
%                   \lwbox
%                 \else
%                   \usebox0
%                 \fi
%               \else
%                 \lwbox
%               \fi
%             \else
%               \usebox0
%             \fi
%           \else
%             \lwbox
%           \fi
%         \else
%           \usebox0
%         \fi
%       \else
%         \lwbox
%       \fi
%     \else
%       \usebox0
%     \fi
%   \else
%     \lwbox
%   \fi
% \else
%   \usebox0
% \fi
% \end{quote}
% If you have a \xfile{docstrip.cfg} that configures and enables \docstrip's
% TDS installing feature, then some files can already be in the right
% place, see the documentation of \docstrip.
%
% \subsection{Refresh file name databases}
%
% If your \TeX~distribution
% (\teTeX, \mikTeX, \dots) relies on file name databases, you must refresh
% these. For example, \teTeX\ users run \verb|texhash| or
% \verb|mktexlsr|.
%
% \subsection{Some details for the interested}
%
% \paragraph{Attached source.}
%
% The PDF documentation on CTAN also includes the
% \xfile{.dtx} source file. It can be extracted by
% AcrobatReader 6 or higher. Another option is \textsf{pdftk},
% e.g. unpack the file into the current directory:
% \begin{quote}
%   \verb|pdftk resizegather.pdf unpack_files output .|
% \end{quote}
%
% \paragraph{Unpacking with \LaTeX.}
% The \xfile{.dtx} chooses its action depending on the format:
% \begin{description}
% \item[\plainTeX:] Run \docstrip\ and extract the files.
% \item[\LaTeX:] Generate the documentation.
% \end{description}
% If you insist on using \LaTeX\ for \docstrip\ (really,
% \docstrip\ does not need \LaTeX), then inform the autodetect routine
% about your intention:
% \begin{quote}
%   \verb|latex \let\install=y% \iffalse meta-comment
%
% File: resizegather.dtx
% Version: 2016/05/16 v1.3
% Info: Resize overly large equations
%
% Copyright (C) 2009, 2010 by
%    Heiko Oberdiek <heiko.oberdiek at googlemail.com>
%    2016
%    https://github.com/ho-tex/oberdiek/issues
%
% This work may be distributed and/or modified under the
% conditions of the LaTeX Project Public License, either
% version 1.3c of this license or (at your option) any later
% version. This version of this license is in
%    http://www.latex-project.org/lppl/lppl-1-3c.txt
% and the latest version of this license is in
%    http://www.latex-project.org/lppl.txt
% and version 1.3 or later is part of all distributions of
% LaTeX version 2005/12/01 or later.
%
% This work has the LPPL maintenance status "maintained".
%
% This Current Maintainer of this work is Heiko Oberdiek.
%
% This work consists of the main source file resizegather.dtx
% and the derived files
%    resizegather.sty, resizegather.pdf, resizegather.ins, resizegather.drv,
%    resizegather-test1.tex.
%
% Distribution:
%    CTAN:macros/latex/contrib/oberdiek/resizegather.dtx
%    CTAN:macros/latex/contrib/oberdiek/resizegather.pdf
%
% Unpacking:
%    (a) If resizegather.ins is present:
%           tex resizegather.ins
%    (b) Without resizegather.ins:
%           tex resizegather.dtx
%    (c) If you insist on using LaTeX
%           latex \let\install=y% \iffalse meta-comment
%
% File: resizegather.dtx
% Version: 2016/05/16 v1.3
% Info: Resize overly large equations
%
% Copyright (C) 2009, 2010 by
%    Heiko Oberdiek <heiko.oberdiek at googlemail.com>
%    2016
%    https://github.com/ho-tex/oberdiek/issues
%
% This work may be distributed and/or modified under the
% conditions of the LaTeX Project Public License, either
% version 1.3c of this license or (at your option) any later
% version. This version of this license is in
%    http://www.latex-project.org/lppl/lppl-1-3c.txt
% and the latest version of this license is in
%    http://www.latex-project.org/lppl.txt
% and version 1.3 or later is part of all distributions of
% LaTeX version 2005/12/01 or later.
%
% This work has the LPPL maintenance status "maintained".
%
% This Current Maintainer of this work is Heiko Oberdiek.
%
% This work consists of the main source file resizegather.dtx
% and the derived files
%    resizegather.sty, resizegather.pdf, resizegather.ins, resizegather.drv,
%    resizegather-test1.tex.
%
% Distribution:
%    CTAN:macros/latex/contrib/oberdiek/resizegather.dtx
%    CTAN:macros/latex/contrib/oberdiek/resizegather.pdf
%
% Unpacking:
%    (a) If resizegather.ins is present:
%           tex resizegather.ins
%    (b) Without resizegather.ins:
%           tex resizegather.dtx
%    (c) If you insist on using LaTeX
%           latex \let\install=y\input{resizegather.dtx}
%        (quote the arguments according to the demands of your shell)
%
% Documentation:
%    (a) If resizegather.drv is present:
%           latex resizegather.drv
%    (b) Without resizegather.drv:
%           latex resizegather.dtx; ...
%    The class ltxdoc loads the configuration file ltxdoc.cfg
%    if available. Here you can specify further options, e.g.
%    use A4 as paper format:
%       \PassOptionsToClass{a4paper}{article}
%
%    Programm calls to get the documentation (example):
%       pdflatex resizegather.dtx
%       makeindex -s gind.ist resizegather.idx
%       pdflatex resizegather.dtx
%       makeindex -s gind.ist resizegather.idx
%       pdflatex resizegather.dtx
%
% Installation:
%    TDS:tex/latex/oberdiek/resizegather.sty
%    TDS:doc/latex/oberdiek/resizegather.pdf
%    TDS:doc/latex/oberdiek/test/resizegather-test1.tex
%    TDS:source/latex/oberdiek/resizegather.dtx
%
%<*ignore>
\begingroup
  \catcode123=1 %
  \catcode125=2 %
  \def\x{LaTeX2e}%
\expandafter\endgroup
\ifcase 0\ifx\install y1\fi\expandafter
         \ifx\csname processbatchFile\endcsname\relax\else1\fi
         \ifx\fmtname\x\else 1\fi\relax
\else\csname fi\endcsname
%</ignore>
%<*install>
\input docstrip.tex
\Msg{************************************************************************}
\Msg{* Installation}
\Msg{* Package: resizegather 2016/05/16 v1.3 Resize overly large equations (HO)}
\Msg{************************************************************************}

\keepsilent
\askforoverwritefalse

\let\MetaPrefix\relax
\preamble

This is a generated file.

Project: resizegather
Version: 2016/05/16 v1.3

Copyright (C) 2009, 2010 by
   Heiko Oberdiek <heiko.oberdiek at googlemail.com>

This work may be distributed and/or modified under the
conditions of the LaTeX Project Public License, either
version 1.3c of this license or (at your option) any later
version. This version of this license is in
   http://www.latex-project.org/lppl/lppl-1-3c.txt
and the latest version of this license is in
   http://www.latex-project.org/lppl.txt
and version 1.3 or later is part of all distributions of
LaTeX version 2005/12/01 or later.

This work has the LPPL maintenance status "maintained".

This Current Maintainer of this work is Heiko Oberdiek.

This work consists of the main source file resizegather.dtx
and the derived files
   resizegather.sty, resizegather.pdf, resizegather.ins, resizegather.drv,
   resizegather-test1.tex.

\endpreamble
\let\MetaPrefix\DoubleperCent

\generate{%
  \file{resizegather.ins}{\from{resizegather.dtx}{install}}%
  \file{resizegather.drv}{\from{resizegather.dtx}{driver}}%
  \usedir{tex/latex/oberdiek}%
  \file{resizegather.sty}{\from{resizegather.dtx}{package}}%
%  \usedir{doc/latex/oberdiek/test}%
%  \file{resizegather-test1.tex}{\from{resizegather.dtx}{test1}}%
  \nopreamble
  \nopostamble
%  \usedir{source/latex/oberdiek/catalogue}%
%  \file{resizegather.xml}{\from{resizegather.dtx}{catalogue}}%
}

\catcode32=13\relax% active space
\let =\space%
\Msg{************************************************************************}
\Msg{*}
\Msg{* To finish the installation you have to move the following}
\Msg{* file into a directory searched by TeX:}
\Msg{*}
\Msg{*     resizegather.sty}
\Msg{*}
\Msg{* To produce the documentation run the file `resizegather.drv'}
\Msg{* through LaTeX.}
\Msg{*}
\Msg{* Happy TeXing!}
\Msg{*}
\Msg{************************************************************************}

\endbatchfile
%</install>
%<*ignore>
\fi
%</ignore>
%<*driver>
\NeedsTeXFormat{LaTeX2e}
\ProvidesFile{resizegather.drv}%
  [2016/05/16 v1.3 Resize overly large equations (HO)]%
\documentclass{ltxdoc}
\usepackage{holtxdoc}[2011/11/22]
\usepackage{ifluatex}
\ifluatex
\else
  \usepackage[T1]{fontenc}%
  \usepackage{textcomp}%
  \usepackage{lmodern}%
\fi
\begin{document}
  \DocInput{resizegather.dtx}%
\end{document}
%</driver>
% \fi
%
%
% \CharacterTable
%  {Upper-case    \A\B\C\D\E\F\G\H\I\J\K\L\M\N\O\P\Q\R\S\T\U\V\W\X\Y\Z
%   Lower-case    \a\b\c\d\e\f\g\h\i\j\k\l\m\n\o\p\q\r\s\t\u\v\w\x\y\z
%   Digits        \0\1\2\3\4\5\6\7\8\9
%   Exclamation   \!     Double quote  \"     Hash (number) \#
%   Dollar        \$     Percent       \%     Ampersand     \&
%   Acute accent  \'     Left paren    \(     Right paren   \)
%   Asterisk      \*     Plus          \+     Comma         \,
%   Minus         \-     Point         \.     Solidus       \/
%   Colon         \:     Semicolon     \;     Less than     \<
%   Equals        \=     Greater than  \>     Question mark \?
%   Commercial at \@     Left bracket  \[     Backslash     \\
%   Right bracket \]     Circumflex    \^     Underscore    \_
%   Grave accent  \`     Left brace    \{     Vertical bar  \|
%   Right brace   \}     Tilde         \~}
%
% \GetFileInfo{resizegather.drv}
%
% \title{The \xpackage{resizegather} package}
% \date{2016/05/16 v1.3}
% \author{Heiko Oberdiek\thanks
% {Please report any issues at https://github.com/ho-tex/oberdiek/issues}\\
% \xemail{heiko.oberdiek at googlemail.com}}
%
% \maketitle
%
% \begin{abstract}
% Equations that are too large are resized to fit the available
% space. The environment \textsf{gather} of package \xpackage{amsmath}
% is supported. Also the environments \textsf{equation} and
% \textsf{displaymath} are redefined using \textsf{gather}
% and its starred version.
% \end{abstract}
%
% \tableofcontents
%
% \makeatletter
% \def\env#1{^^A
%    \textsf{\@env#1*\@nil}^^A
% }%
% \def\@env#1*#2\@nil{^^A
%   #1^^A
%   \ifx\\#2\\^^A
%     \expandafter\@gobble
%   \else
%     \textasteriskcentered
%     \expandafter\@firstofone
%   \fi
%   {\@env#2\@nil}^^A
% }
% \makeatother
%
% \section{Documentation}
%
% Sometimes an equation is just a little to large to fit in the
% line. And breaking the equation across lines might be worse
% than downscaling the equation. This package implements this
% for the environments \env{gather} and \env{gather*} of
% package \xpackage{amsmath}. That package already measures
% the equations and simplifies the implementation of \xpackage{resizegather}
% that only needs to hook into \xpackage{amsmath}'s code to add
% the resizing feature.
%
% Resized equations are recorded in the \xext{log} file
% for small exceeds (default setting is smaller than five percent).
% Otherwise a warning is given.
%
% Also environments \env{equation} and \env{displaymath}
% are supported by redefining them using \env{gather}
% and \env{gather*}.
%
% \cs{[} and \cs{]} are not supported, because these macros
% are not in environment form that is required for
% \xpackage{amsmath}. The environment body is collected
% first to be able to process the body twice for measuring
% first.
%
% Also the environments using alignments are not supported.
% If a single equation line would be resized, the alignment
% would get lost. And resizing all equations of the alignment
% does not seem appropriate either.
%
% \subsection{Options}
%
% \begin{description}
% \item[\xoption{warningthreshold}:]
%   Print a warning if the original equation line exceeds
%   its available width by the given fraction.
%   Default is |0.05|: A warning is given if the equation
%   is too large by five percent.
%   Otherwise the exceed is recorded in the \xext{log} file
%   only.
% \end{description}
% The next options are boolean options. They are enabled
% by value |true| or if no value is given. They are switched
% off by value |false|.
% \begin{description}
% \item[\xoption{enable}:] The resize feature is active (default).
% \item[\xoption{disable}:] The complementary option for \xoption{enable},
%  added for convenience: |disable| (or |disable=true|) is the same
%  as |enable=false|.
% \item[\xoption{equations}:]
%   \LaTeX\ environments \textsf{equation} and \textsf{displaymath}
%   environments are redefined. These equations
%   are now using environment \env{gather} and
%   \env{gather*}. This is the default.
% \end{description}
% The following table shows additional options if you
% want to have finer control for the redefined
% environments:
% \begin{quote}
% \def\unchanged{\textit{unchanged}}
% \def\notprovided{\textit{not provided}}
% \begin{tabular}{l|ll}
% &\multicolumn{2}{c}{Environments}\\
% Option & \env{equation} & \env{displaymath}\\
% \hline
% \xoption{equations} & \env{gather} & \env{gather*}\\
% \xoption{equation} & \env{gather} & \unchanged\\
% \xoption{displaymath} & \unchanged & \env{gather*}\\
% \end{tabular}
% \end{quote}
% If such an option is switched off, the original meaning
% of the affected environments is restored.
%
% Options are evaluated in the following order:
% \begin{enumerate}
% \item
%  Configuration file \xfile{resizegather.cfg} using \cs{resizegathersetup}
%  if the file exists.
%  \item
%  Package options given for \cs{usepackage}.
%  \item
%  Later calls of \cs{resizegathersetup}.
% \end{enumerate}
% \begin{declcs}{resizegathersetup}\M{option list}
% \end{declcs}
% The options are key value options. Boolean options are enabled by
% default (without value) or by using the explicit value \texttt{true}.
% Value \texttt{false} disable the option.
%
% \subsection{Options for packages \xpackage{amsmath} or \xpackage{graphics}}
%
% The package loads the package \xpackage{amsmath} because is internally
% measures the equations first. Thus this package hooks into this code
% in order to resize the equations if they are too large.
% The resizing itself is done by \cs{resizebox} of package \xpackage{graphics}.
% If you need special options for these packages, just load them first or
% use global options when appropriate. Example:
% \begin{quote}
%\begin{verbatim}
%\usepackage[dvipdfm]{graphicx}% or graphics
%\usepackage[fleqn]{amsmath}
%\usepackage{resizegather}
%\end{verbatim}
%\end{quote}
%
% \StopEventually{
% }
%
% \section{Implementation}
%    \begin{macrocode}
%<*package>
%    \end{macrocode}
%    Reload check, especially if the package is not used with \LaTeX.
%    \begin{macrocode}
\begingroup\catcode61\catcode48\catcode32=10\relax%
  \catcode13=5 % ^^M
  \endlinechar=13 %
  \catcode35=6 % #
  \catcode39=12 % '
  \catcode44=12 % ,
  \catcode45=12 % -
  \catcode46=12 % .
  \catcode58=12 % :
  \catcode64=11 % @
  \catcode123=1 % {
  \catcode125=2 % }
  \expandafter\let\expandafter\x\csname ver@resizegather.sty\endcsname
  \ifx\x\relax % plain-TeX, first loading
  \else
    \def\empty{}%
    \ifx\x\empty % LaTeX, first loading,
      % variable is initialized, but \ProvidesPackage not yet seen
    \else
      \expandafter\ifx\csname PackageInfo\endcsname\relax
        \def\x#1#2{%
          \immediate\write-1{Package #1 Info: #2.}%
        }%
      \else
        \def\x#1#2{\PackageInfo{#1}{#2, stopped}}%
      \fi
      \x{resizegather}{The package is already loaded}%
      \aftergroup\endinput
    \fi
  \fi
\endgroup%
%    \end{macrocode}
%    Package identification:
%    \begin{macrocode}
\begingroup\catcode61\catcode48\catcode32=10\relax%
  \catcode13=5 % ^^M
  \endlinechar=13 %
  \catcode35=6 % #
  \catcode39=12 % '
  \catcode40=12 % (
  \catcode41=12 % )
  \catcode44=12 % ,
  \catcode45=12 % -
  \catcode46=12 % .
  \catcode47=12 % /
  \catcode58=12 % :
  \catcode64=11 % @
  \catcode91=12 % [
  \catcode93=12 % ]
  \catcode123=1 % {
  \catcode125=2 % }
  \expandafter\ifx\csname ProvidesPackage\endcsname\relax
    \def\x#1#2#3[#4]{\endgroup
      \immediate\write-1{Package: #3 #4}%
      \xdef#1{#4}%
    }%
  \else
    \def\x#1#2[#3]{\endgroup
      #2[{#3}]%
      \ifx#1\@undefined
        \xdef#1{#3}%
      \fi
      \ifx#1\relax
        \xdef#1{#3}%
      \fi
    }%
  \fi
\expandafter\x\csname ver@resizegather.sty\endcsname
\ProvidesPackage{resizegather}%
  [2016/05/16 v1.3 Resize overly large equations (HO)]%
%    \end{macrocode}
%
%    \begin{macrocode}
\begingroup\catcode61\catcode48\catcode32=10\relax%
  \catcode13=5 % ^^M
  \endlinechar=13 %
  \catcode123=1 % {
  \catcode125=2 % }
  \catcode64=11 % @
  \def\x{\endgroup
    \expandafter\edef\csname ResizeGather@AtEnd\endcsname{%
      \endlinechar=\the\endlinechar\relax
      \catcode13=\the\catcode13\relax
      \catcode32=\the\catcode32\relax
      \catcode35=\the\catcode35\relax
      \catcode61=\the\catcode61\relax
      \catcode64=\the\catcode64\relax
      \catcode123=\the\catcode123\relax
      \catcode125=\the\catcode125\relax
    }%
  }%
\x\catcode61\catcode48\catcode32=10\relax%
\catcode13=5 % ^^M
\endlinechar=13 %
\catcode35=6 % #
\catcode64=11 % @
\catcode123=1 % {
\catcode125=2 % }
\def\TMP@EnsureCode#1#2{%
  \edef\ResizeGather@AtEnd{%
    \ResizeGather@AtEnd
    \catcode#1=\the\catcode#1\relax
  }%
  \catcode#1=#2\relax
}
\TMP@EnsureCode{10}{12}% ^^J
\TMP@EnsureCode{33}{12}% !
\TMP@EnsureCode{36}{3}% $
\TMP@EnsureCode{38}{4}% &
\TMP@EnsureCode{39}{12}% '
\TMP@EnsureCode{40}{12}% (
\TMP@EnsureCode{41}{12}% )
\TMP@EnsureCode{42}{12}% *
\TMP@EnsureCode{43}{12}% +
\TMP@EnsureCode{44}{12}% ,
\TMP@EnsureCode{45}{12}% -
\TMP@EnsureCode{46}{12}% .
\TMP@EnsureCode{47}{12}% /
\TMP@EnsureCode{58}{12}% :
\TMP@EnsureCode{59}{12}% ;
\TMP@EnsureCode{60}{12}% <
\TMP@EnsureCode{62}{12}% >
\TMP@EnsureCode{63}{12}% ?
\TMP@EnsureCode{91}{12}% [
\TMP@EnsureCode{93}{12}% ]
\TMP@EnsureCode{94}{7}% ^ (superscript)
\TMP@EnsureCode{96}{12}% `
\edef\ResizeGather@AtEnd{\ResizeGather@AtEnd\noexpand\endinput}
%    \end{macrocode}
%
%    \begin{macrocode}
\RequirePackage{kvoptions}[2009/12/04]
\SetupKeyvalOptions{%
  family=resizegather,%
  prefix=ResizeGather@,%
}
%    \end{macrocode}
%    \begin{macrocode}
\@for\ResizeGather@option:=%
  centertags,%
  tbtags,%
  sumlimits,%
  nosumlimits,%
  intlimits,%
  nointlimits,%
  nonamelimits,%
  leqno,%
  reqno,%
  fleqn%
\do{%
  \edef\ResizeGather@temp{%
    \noexpand\DeclareVoidOption{\ResizeGather@option}{%
      \noexpand\PassOptionsToPackage{amsmath}{\ResizeGather@option}%
    }%
    \noexpand\AtEndOfPackage{%
      \noexpand\DisableKeyvalOption[%
        action=error,%
        package=resizegather,%
      ]{resizegather}{\ResizeGather@option}%
    }%
  }%
  \ResizeGather@temp
}
\@for\ResizeGather@option:=%
  draft,%
  final,%
  hiderotate,%
  hidescale,%
  hiresbb,%
  demo,%
  dvips,xdvi,dvipdf,dvipdfm,dvipdfmx,pdftex,dvipsone,%
  dviwindo,emtex,dviwin,pctexps,pctexwin,pctexhp,pctex32,%
  truetex,tcidvi,vtex,oztex,textures,xetex%
\do{%
  \edef\ResizeGather@temp{%
    \noexpand\DeclareVoidOption{\ResizeGather@option}{%
      \noexpand\PassOptionsToPackage{graphics}{\ResizeGather@option}%
    }%
    \noexpand\AtEndOfPackage{%
      \noexpand\DisableKeyvalOption[%
        action=error,%
        package=resizegather,%
      ]{resizegather}{\ResizeGather@option}%
    }%
  }%
  \ResizeGather@temp
}
%    \end{macrocode}
%
%    \begin{macrocode}
\DeclareBoolOption[true]{enable}
\DeclareComplementaryOption{disable}{enable}
\DeclareStringOption[.05]{warningthreshold}
\newif\ifResizeGather@NeedInit
\DeclareBoolOption[true]{equations}
\DeclareBoolOption[true]{equation}
\DeclareBoolOption[true]{displaymath}
\AddToKeyvalOption*{equations}{%
  \ResizeGather@NeedInittrue
  \ifResizeGather@equations
    \ResizeGather@equationtrue
    \ResizeGather@displaymathtrue
  \else
    \ResizeGather@equationfalse
    \ResizeGather@displaymathfalse
  \fi
}
\AddToKeyvalOption*{equation}{%
  \ResizeGather@NeedInittrue
}
\AddToKeyvalOption*{displaymath}{%
  \ResizeGather@NeedInittrue
}
%    \end{macrocode}
%
%    \begin{macro}{\resizegathersetup}
%    \begin{macrocode}
\newcommand*{\resizegathersetup}[1]{%
  \ResizeGather@NeedInitfalse
  \setkeys{resizegather}{#1}%
  \ifResizeGather@NeedInit
    \ResizeGather@init
  \fi
}
\let\ResizeGather@init\relax
%    \end{macrocode}
%    \end{macro}
%    \begin{macrocode}
\InputIfFileExists{resizegather.cfg}{}{}%
\ProcessKeyvalOptions*\relax
%    \end{macrocode}
%    \begin{macrocode}
\RequirePackage{amsmath}
\RequirePackage{graphics}
%    \end{macrocode}
%    \begin{macro}{\ResizeGather@redefine}
%    \begin{macrocode}
\def\ResizeGather@redefine#1#2#3#4#5{%
  \csname ifResizeGather@#1\endcsname
    \@ifundefined{ResizeGather@org@#2}{%
      \expandafter\let\csname ResizeGather@org@#2\expandafter\endcsname
                      \csname #2\endcsname
    }{}%
    \@ifundefined{ResizeGather@org@#3}{%
      \expandafter\let\csname ResizeGather@org@#3\expandafter\endcsname
                      \csname #3\endcsname
    }{}%
    \expandafter\edef\csname #2\endcsname{%
      \expandafter\noexpand\csname#4\endcsname
    }%
    \expandafter\edef\csname #3\endcsname{%
      \expandafter\noexpand\csname#5\endcsname
    }%
  \else
    \@ifundefined{ResizeGather@org@#2}{}{%
      \expandafter\let\csname #2\expandafter\endcsname
                      \csname ResizeGather@org@#2\endcsname
      \expandafter\let\csname #3\expandafter\endcsname
                      \csname ResizeGather@org@#3\endcsname
    }%
  \fi
}
%    \end{macrocode}
%    \end{macro}
%    \begin{macro}{\ResizeGather@init}
%    \begin{macrocode}
\def\ResizeGather@init{%
  \ResizeGather@redefine{equation}{equation}{endequation}%
    {gather}{endgather}%
  \ResizeGather@redefine{displaymath}{displaymath}{enddisplaymath}%
    {gather*}{endgather*}%
}
\ResizeGather@init
%    \end{macrocode}
%    \end{macro}
%
%    \begin{macro}{\ResizeGather@ResizeGather}
%    \begin{macrocode}
\def\ResizeGather@ResizeGather{%
  \ifResizeGather@enable
    \dimen@\displaywidth
    \if@fleqn
      \advance\dimen@-\@mathmargin
    \fi
    \ifdim\wdz@>\dimen@
      \begingroup
        \advance\dimen@ -\wdz@
        \dimen@ -\dimen@
        \ifdim\ResizeGather@warningthreshold\wdz@>\dimen@
          \expandafter\PackageInfo
        \else
          \expandafter\PackageWarning
        \fi
        {resizegather}{%
          Equation line \the\row@\space is too large %
          by \the\dimen@\MessageBreak
          in environment `\@currenvir'%
        }%
      \endgroup
      \setboxz@h to\dimen@{%
        \resizebox{\dimen@}{!}{\boxz@}%
        \hss
      }%
    \fi
  \fi
}
%    \end{macrocode}
%    \end{macro}
%    \begin{macro}{\calc@shift@gather}
%    \begin{macrocode}
\expandafter\def\expandafter\calc@shift@gather\expandafter{%
  \expandafter\ResizeGather@ResizeGather
  \calc@shift@gather
}
%    \end{macrocode}
%    \end{macro}
%    \begin{macro}{\ResizeGather@org@gmeasure@}
%    \begin{macrocode}
\let\ResizeGather@org@gmeasure@\gmeasure@
%    \end{macrocode}
%    \end{macro}
%    \begin{macro}{\gmeasure@}
%    \begin{macrocode}
\def\gmeasure@#1{%
  \ResizeGather@org@gmeasure@{#1}%
  \ifResizeGather@enable
    \ifdim\totwidth@>\displaywidth
      \totwidth@=\displaywidth
    \fi
  \fi
}
%    \end{macrocode}
%    \end{macro}
%
%    \begin{macrocode}
\ResizeGather@AtEnd%
%</package>
%    \end{macrocode}
%
% \section{Test}
%
% \subsection{Catcode checks for loading}
%
%    \begin{macrocode}
%<*test1>
%    \end{macrocode}
%    \begin{macrocode}
\catcode`\{=1 %
\catcode`\}=2 %
\catcode`\#=6 %
\catcode`\@=11 %
\expandafter\ifx\csname count@\endcsname\relax
  \countdef\count@=255 %
\fi
\expandafter\ifx\csname @gobble\endcsname\relax
  \long\def\@gobble#1{}%
\fi
\expandafter\ifx\csname @firstofone\endcsname\relax
  \long\def\@firstofone#1{#1}%
\fi
\expandafter\ifx\csname loop\endcsname\relax
  \expandafter\@firstofone
\else
  \expandafter\@gobble
\fi
{%
  \def\loop#1\repeat{%
    \def\body{#1}%
    \iterate
  }%
  \def\iterate{%
    \body
      \let\next\iterate
    \else
      \let\next\relax
    \fi
    \next
  }%
  \let\repeat=\fi
}%
\def\RestoreCatcodes{}
\count@=0 %
\loop
  \edef\RestoreCatcodes{%
    \RestoreCatcodes
    \catcode\the\count@=\the\catcode\count@\relax
  }%
\ifnum\count@<255 %
  \advance\count@ 1 %
\repeat

\def\RangeCatcodeInvalid#1#2{%
  \count@=#1\relax
  \loop
    \catcode\count@=15 %
  \ifnum\count@<#2\relax
    \advance\count@ 1 %
  \repeat
}
\def\RangeCatcodeCheck#1#2#3{%
  \count@=#1\relax
  \loop
    \ifnum#3=\catcode\count@
    \else
      \errmessage{%
        Character \the\count@\space
        with wrong catcode \the\catcode\count@\space
        instead of \number#3%
      }%
    \fi
  \ifnum\count@<#2\relax
    \advance\count@ 1 %
  \repeat
}
\def\space{ }
\expandafter\ifx\csname LoadCommand\endcsname\relax
  \def\LoadCommand{\input resizegather.sty\relax}%
\fi
\def\Test{%
  \RangeCatcodeInvalid{0}{47}%
  \RangeCatcodeInvalid{58}{64}%
  \RangeCatcodeInvalid{91}{96}%
  \RangeCatcodeInvalid{123}{255}%
  \catcode`\@=12 %
  \catcode`\\=0 %
  \catcode`\%=14 %
  \LoadCommand
  \RangeCatcodeCheck{0}{36}{15}%
  \RangeCatcodeCheck{37}{37}{14}%
  \RangeCatcodeCheck{38}{47}{15}%
  \RangeCatcodeCheck{48}{57}{12}%
  \RangeCatcodeCheck{58}{63}{15}%
  \RangeCatcodeCheck{64}{64}{12}%
  \RangeCatcodeCheck{65}{90}{11}%
  \RangeCatcodeCheck{91}{91}{15}%
  \RangeCatcodeCheck{92}{92}{0}%
  \RangeCatcodeCheck{93}{96}{15}%
  \RangeCatcodeCheck{97}{122}{11}%
  \RangeCatcodeCheck{123}{255}{15}%
  \RestoreCatcodes
}
\Test
\csname @@end\endcsname
\end
%    \end{macrocode}
%    \begin{macrocode}
%</test1>
%    \end{macrocode}
%
% \section{Installation}
%
% \subsection{Download}
%
% \paragraph{Package.} This package is available on
% CTAN\footnote{\url{https://ctan.org/pkg/resizegather}}:
% \begin{description}
% \item[\CTAN{macros/latex/contrib/oberdiek/resizegather.dtx}] The source file.
% \item[\CTAN{macros/latex/contrib/oberdiek/resizegather.pdf}] Documentation.
% \end{description}
%
%
% \paragraph{Bundle.} All the packages of the bundle `oberdiek'
% are also available in a TDS compliant ZIP archive. There
% the packages are already unpacked and the documentation files
% are generated. The files and directories obey the TDS standard.
% \begin{description}
% \item[\CTANinstall{install/macros/latex/contrib/oberdiek.tds.zip}]
% \end{description}
% \emph{TDS} refers to the standard ``A Directory Structure
% for \TeX\ Files'' (\CTAN{tds/tds.pdf}). Directories
% with \xfile{texmf} in their name are usually organized this way.
%
% \subsection{Bundle installation}
%
% \paragraph{Unpacking.} Unpack the \xfile{oberdiek.tds.zip} in the
% TDS tree (also known as \xfile{texmf} tree) of your choice.
% Example (linux):
% \begin{quote}
%   |unzip oberdiek.tds.zip -d ~/texmf|
% \end{quote}
%
% \paragraph{Script installation.}
% Check the directory \xfile{TDS:scripts/oberdiek/} for
% scripts that need further installation steps.
% Package \xpackage{attachfile2} comes with the Perl script
% \xfile{pdfatfi.pl} that should be installed in such a way
% that it can be called as \texttt{pdfatfi}.
% Example (linux):
% \begin{quote}
%   |chmod +x scripts/oberdiek/pdfatfi.pl|\\
%   |cp scripts/oberdiek/pdfatfi.pl /usr/local/bin/|
% \end{quote}
%
% \subsection{Package installation}
%
% \paragraph{Unpacking.} The \xfile{.dtx} file is a self-extracting
% \docstrip\ archive. The files are extracted by running the
% \xfile{.dtx} through \plainTeX:
% \begin{quote}
%   \verb|tex resizegather.dtx|
% \end{quote}
%
% \paragraph{TDS.} Now the different files must be moved into
% the different directories in your installation TDS tree
% (also known as \xfile{texmf} tree):
% \begin{quote}
% \def\t{^^A
% \begin{tabular}{@{}>{\ttfamily}l@{ $\rightarrow$ }>{\ttfamily}l@{}}
%   resizegather.sty & tex/latex/oberdiek/resizegather.sty\\
%   resizegather.pdf & doc/latex/oberdiek/resizegather.pdf\\
%   test/resizegather-test1.tex & doc/latex/oberdiek/test/resizegather-test1.tex\\
%   resizegather.dtx & source/latex/oberdiek/resizegather.dtx\\
% \end{tabular}^^A
% }^^A
% \sbox0{\t}^^A
% \ifdim\wd0>\linewidth
%   \begingroup
%     \advance\linewidth by\leftmargin
%     \advance\linewidth by\rightmargin
%   \edef\x{\endgroup
%     \def\noexpand\lw{\the\linewidth}^^A
%   }\x
%   \def\lwbox{^^A
%     \leavevmode
%     \hbox to \linewidth{^^A
%       \kern-\leftmargin\relax
%       \hss
%       \usebox0
%       \hss
%       \kern-\rightmargin\relax
%     }^^A
%   }^^A
%   \ifdim\wd0>\lw
%     \sbox0{\small\t}^^A
%     \ifdim\wd0>\linewidth
%       \ifdim\wd0>\lw
%         \sbox0{\footnotesize\t}^^A
%         \ifdim\wd0>\linewidth
%           \ifdim\wd0>\lw
%             \sbox0{\scriptsize\t}^^A
%             \ifdim\wd0>\linewidth
%               \ifdim\wd0>\lw
%                 \sbox0{\tiny\t}^^A
%                 \ifdim\wd0>\linewidth
%                   \lwbox
%                 \else
%                   \usebox0
%                 \fi
%               \else
%                 \lwbox
%               \fi
%             \else
%               \usebox0
%             \fi
%           \else
%             \lwbox
%           \fi
%         \else
%           \usebox0
%         \fi
%       \else
%         \lwbox
%       \fi
%     \else
%       \usebox0
%     \fi
%   \else
%     \lwbox
%   \fi
% \else
%   \usebox0
% \fi
% \end{quote}
% If you have a \xfile{docstrip.cfg} that configures and enables \docstrip's
% TDS installing feature, then some files can already be in the right
% place, see the documentation of \docstrip.
%
% \subsection{Refresh file name databases}
%
% If your \TeX~distribution
% (\teTeX, \mikTeX, \dots) relies on file name databases, you must refresh
% these. For example, \teTeX\ users run \verb|texhash| or
% \verb|mktexlsr|.
%
% \subsection{Some details for the interested}
%
% \paragraph{Attached source.}
%
% The PDF documentation on CTAN also includes the
% \xfile{.dtx} source file. It can be extracted by
% AcrobatReader 6 or higher. Another option is \textsf{pdftk},
% e.g. unpack the file into the current directory:
% \begin{quote}
%   \verb|pdftk resizegather.pdf unpack_files output .|
% \end{quote}
%
% \paragraph{Unpacking with \LaTeX.}
% The \xfile{.dtx} chooses its action depending on the format:
% \begin{description}
% \item[\plainTeX:] Run \docstrip\ and extract the files.
% \item[\LaTeX:] Generate the documentation.
% \end{description}
% If you insist on using \LaTeX\ for \docstrip\ (really,
% \docstrip\ does not need \LaTeX), then inform the autodetect routine
% about your intention:
% \begin{quote}
%   \verb|latex \let\install=y\input{resizegather.dtx}|
% \end{quote}
% Do not forget to quote the argument according to the demands
% of your shell.
%
% \paragraph{Generating the documentation.}
% You can use both the \xfile{.dtx} or the \xfile{.drv} to generate
% the documentation. The process can be configured by the
% configuration file \xfile{ltxdoc.cfg}. For instance, put this
% line into this file, if you want to have A4 as paper format:
% \begin{quote}
%   \verb|\PassOptionsToClass{a4paper}{article}|
% \end{quote}
% An example follows how to generate the
% documentation with pdf\LaTeX:
% \begin{quote}
%\begin{verbatim}
%pdflatex resizegather.dtx
%makeindex -s gind.ist resizegather.idx
%pdflatex resizegather.dtx
%makeindex -s gind.ist resizegather.idx
%pdflatex resizegather.dtx
%\end{verbatim}
% \end{quote}
%
% \section{Acknowledgement}
%
% \begin{description}
% \item[Dieter Jurzitza:]
% He wanted the resizing feature at the \TeX\ table
% in Karlsruhe of December 2009. Thus this package is a kind of
% Christmas present.
% \end{description}
%
% \begin{History}
%   \begin{Version}{2009/12/04 v1.0}
%   \item
%     The first version.
%   \end{Version}
%   \begin{Version}{2009/12/05 v1.1}
%   \item
%     Options \xoption{enable} and \xoption{disable} added.
%   \end{Version}
%   \begin{Version}{2010/03/01 v1.2}
%   \item
%     TDS location moved from `generic' to `latex'.
%   \end{Version}
%   \begin{Version}{2016/05/16 v1.3}
%   \item
%     Documentation updates.
%   \end{Version}
% \end{History}
%
% \PrintIndex
%
% \Finale
\endinput

%        (quote the arguments according to the demands of your shell)
%
% Documentation:
%    (a) If resizegather.drv is present:
%           latex resizegather.drv
%    (b) Without resizegather.drv:
%           latex resizegather.dtx; ...
%    The class ltxdoc loads the configuration file ltxdoc.cfg
%    if available. Here you can specify further options, e.g.
%    use A4 as paper format:
%       \PassOptionsToClass{a4paper}{article}
%
%    Programm calls to get the documentation (example):
%       pdflatex resizegather.dtx
%       makeindex -s gind.ist resizegather.idx
%       pdflatex resizegather.dtx
%       makeindex -s gind.ist resizegather.idx
%       pdflatex resizegather.dtx
%
% Installation:
%    TDS:tex/latex/oberdiek/resizegather.sty
%    TDS:doc/latex/oberdiek/resizegather.pdf
%    TDS:doc/latex/oberdiek/test/resizegather-test1.tex
%    TDS:source/latex/oberdiek/resizegather.dtx
%
%<*ignore>
\begingroup
  \catcode123=1 %
  \catcode125=2 %
  \def\x{LaTeX2e}%
\expandafter\endgroup
\ifcase 0\ifx\install y1\fi\expandafter
         \ifx\csname processbatchFile\endcsname\relax\else1\fi
         \ifx\fmtname\x\else 1\fi\relax
\else\csname fi\endcsname
%</ignore>
%<*install>
\input docstrip.tex
\Msg{************************************************************************}
\Msg{* Installation}
\Msg{* Package: resizegather 2016/05/16 v1.3 Resize overly large equations (HO)}
\Msg{************************************************************************}

\keepsilent
\askforoverwritefalse

\let\MetaPrefix\relax
\preamble

This is a generated file.

Project: resizegather
Version: 2016/05/16 v1.3

Copyright (C) 2009, 2010 by
   Heiko Oberdiek <heiko.oberdiek at googlemail.com>

This work may be distributed and/or modified under the
conditions of the LaTeX Project Public License, either
version 1.3c of this license or (at your option) any later
version. This version of this license is in
   http://www.latex-project.org/lppl/lppl-1-3c.txt
and the latest version of this license is in
   http://www.latex-project.org/lppl.txt
and version 1.3 or later is part of all distributions of
LaTeX version 2005/12/01 or later.

This work has the LPPL maintenance status "maintained".

This Current Maintainer of this work is Heiko Oberdiek.

This work consists of the main source file resizegather.dtx
and the derived files
   resizegather.sty, resizegather.pdf, resizegather.ins, resizegather.drv,
   resizegather-test1.tex.

\endpreamble
\let\MetaPrefix\DoubleperCent

\generate{%
  \file{resizegather.ins}{\from{resizegather.dtx}{install}}%
  \file{resizegather.drv}{\from{resizegather.dtx}{driver}}%
  \usedir{tex/latex/oberdiek}%
  \file{resizegather.sty}{\from{resizegather.dtx}{package}}%
%  \usedir{doc/latex/oberdiek/test}%
%  \file{resizegather-test1.tex}{\from{resizegather.dtx}{test1}}%
  \nopreamble
  \nopostamble
%  \usedir{source/latex/oberdiek/catalogue}%
%  \file{resizegather.xml}{\from{resizegather.dtx}{catalogue}}%
}

\catcode32=13\relax% active space
\let =\space%
\Msg{************************************************************************}
\Msg{*}
\Msg{* To finish the installation you have to move the following}
\Msg{* file into a directory searched by TeX:}
\Msg{*}
\Msg{*     resizegather.sty}
\Msg{*}
\Msg{* To produce the documentation run the file `resizegather.drv'}
\Msg{* through LaTeX.}
\Msg{*}
\Msg{* Happy TeXing!}
\Msg{*}
\Msg{************************************************************************}

\endbatchfile
%</install>
%<*ignore>
\fi
%</ignore>
%<*driver>
\NeedsTeXFormat{LaTeX2e}
\ProvidesFile{resizegather.drv}%
  [2016/05/16 v1.3 Resize overly large equations (HO)]%
\documentclass{ltxdoc}
\usepackage{holtxdoc}[2011/11/22]
\usepackage{ifluatex}
\ifluatex
\else
  \usepackage[T1]{fontenc}%
  \usepackage{textcomp}%
  \usepackage{lmodern}%
\fi
\begin{document}
  \DocInput{resizegather.dtx}%
\end{document}
%</driver>
% \fi
%
%
% \CharacterTable
%  {Upper-case    \A\B\C\D\E\F\G\H\I\J\K\L\M\N\O\P\Q\R\S\T\U\V\W\X\Y\Z
%   Lower-case    \a\b\c\d\e\f\g\h\i\j\k\l\m\n\o\p\q\r\s\t\u\v\w\x\y\z
%   Digits        \0\1\2\3\4\5\6\7\8\9
%   Exclamation   \!     Double quote  \"     Hash (number) \#
%   Dollar        \$     Percent       \%     Ampersand     \&
%   Acute accent  \'     Left paren    \(     Right paren   \)
%   Asterisk      \*     Plus          \+     Comma         \,
%   Minus         \-     Point         \.     Solidus       \/
%   Colon         \:     Semicolon     \;     Less than     \<
%   Equals        \=     Greater than  \>     Question mark \?
%   Commercial at \@     Left bracket  \[     Backslash     \\
%   Right bracket \]     Circumflex    \^     Underscore    \_
%   Grave accent  \`     Left brace    \{     Vertical bar  \|
%   Right brace   \}     Tilde         \~}
%
% \GetFileInfo{resizegather.drv}
%
% \title{The \xpackage{resizegather} package}
% \date{2016/05/16 v1.3}
% \author{Heiko Oberdiek\thanks
% {Please report any issues at https://github.com/ho-tex/oberdiek/issues}\\
% \xemail{heiko.oberdiek at googlemail.com}}
%
% \maketitle
%
% \begin{abstract}
% Equations that are too large are resized to fit the available
% space. The environment \textsf{gather} of package \xpackage{amsmath}
% is supported. Also the environments \textsf{equation} and
% \textsf{displaymath} are redefined using \textsf{gather}
% and its starred version.
% \end{abstract}
%
% \tableofcontents
%
% \makeatletter
% \def\env#1{^^A
%    \textsf{\@env#1*\@nil}^^A
% }%
% \def\@env#1*#2\@nil{^^A
%   #1^^A
%   \ifx\\#2\\^^A
%     \expandafter\@gobble
%   \else
%     \textasteriskcentered
%     \expandafter\@firstofone
%   \fi
%   {\@env#2\@nil}^^A
% }
% \makeatother
%
% \section{Documentation}
%
% Sometimes an equation is just a little to large to fit in the
% line. And breaking the equation across lines might be worse
% than downscaling the equation. This package implements this
% for the environments \env{gather} and \env{gather*} of
% package \xpackage{amsmath}. That package already measures
% the equations and simplifies the implementation of \xpackage{resizegather}
% that only needs to hook into \xpackage{amsmath}'s code to add
% the resizing feature.
%
% Resized equations are recorded in the \xext{log} file
% for small exceeds (default setting is smaller than five percent).
% Otherwise a warning is given.
%
% Also environments \env{equation} and \env{displaymath}
% are supported by redefining them using \env{gather}
% and \env{gather*}.
%
% \cs{[} and \cs{]} are not supported, because these macros
% are not in environment form that is required for
% \xpackage{amsmath}. The environment body is collected
% first to be able to process the body twice for measuring
% first.
%
% Also the environments using alignments are not supported.
% If a single equation line would be resized, the alignment
% would get lost. And resizing all equations of the alignment
% does not seem appropriate either.
%
% \subsection{Options}
%
% \begin{description}
% \item[\xoption{warningthreshold}:]
%   Print a warning if the original equation line exceeds
%   its available width by the given fraction.
%   Default is |0.05|: A warning is given if the equation
%   is too large by five percent.
%   Otherwise the exceed is recorded in the \xext{log} file
%   only.
% \end{description}
% The next options are boolean options. They are enabled
% by value |true| or if no value is given. They are switched
% off by value |false|.
% \begin{description}
% \item[\xoption{enable}:] The resize feature is active (default).
% \item[\xoption{disable}:] The complementary option for \xoption{enable},
%  added for convenience: |disable| (or |disable=true|) is the same
%  as |enable=false|.
% \item[\xoption{equations}:]
%   \LaTeX\ environments \textsf{equation} and \textsf{displaymath}
%   environments are redefined. These equations
%   are now using environment \env{gather} and
%   \env{gather*}. This is the default.
% \end{description}
% The following table shows additional options if you
% want to have finer control for the redefined
% environments:
% \begin{quote}
% \def\unchanged{\textit{unchanged}}
% \def\notprovided{\textit{not provided}}
% \begin{tabular}{l|ll}
% &\multicolumn{2}{c}{Environments}\\
% Option & \env{equation} & \env{displaymath}\\
% \hline
% \xoption{equations} & \env{gather} & \env{gather*}\\
% \xoption{equation} & \env{gather} & \unchanged\\
% \xoption{displaymath} & \unchanged & \env{gather*}\\
% \end{tabular}
% \end{quote}
% If such an option is switched off, the original meaning
% of the affected environments is restored.
%
% Options are evaluated in the following order:
% \begin{enumerate}
% \item
%  Configuration file \xfile{resizegather.cfg} using \cs{resizegathersetup}
%  if the file exists.
%  \item
%  Package options given for \cs{usepackage}.
%  \item
%  Later calls of \cs{resizegathersetup}.
% \end{enumerate}
% \begin{declcs}{resizegathersetup}\M{option list}
% \end{declcs}
% The options are key value options. Boolean options are enabled by
% default (without value) or by using the explicit value \texttt{true}.
% Value \texttt{false} disable the option.
%
% \subsection{Options for packages \xpackage{amsmath} or \xpackage{graphics}}
%
% The package loads the package \xpackage{amsmath} because is internally
% measures the equations first. Thus this package hooks into this code
% in order to resize the equations if they are too large.
% The resizing itself is done by \cs{resizebox} of package \xpackage{graphics}.
% If you need special options for these packages, just load them first or
% use global options when appropriate. Example:
% \begin{quote}
%\begin{verbatim}
%\usepackage[dvipdfm]{graphicx}% or graphics
%\usepackage[fleqn]{amsmath}
%\usepackage{resizegather}
%\end{verbatim}
%\end{quote}
%
% \StopEventually{
% }
%
% \section{Implementation}
%    \begin{macrocode}
%<*package>
%    \end{macrocode}
%    Reload check, especially if the package is not used with \LaTeX.
%    \begin{macrocode}
\begingroup\catcode61\catcode48\catcode32=10\relax%
  \catcode13=5 % ^^M
  \endlinechar=13 %
  \catcode35=6 % #
  \catcode39=12 % '
  \catcode44=12 % ,
  \catcode45=12 % -
  \catcode46=12 % .
  \catcode58=12 % :
  \catcode64=11 % @
  \catcode123=1 % {
  \catcode125=2 % }
  \expandafter\let\expandafter\x\csname ver@resizegather.sty\endcsname
  \ifx\x\relax % plain-TeX, first loading
  \else
    \def\empty{}%
    \ifx\x\empty % LaTeX, first loading,
      % variable is initialized, but \ProvidesPackage not yet seen
    \else
      \expandafter\ifx\csname PackageInfo\endcsname\relax
        \def\x#1#2{%
          \immediate\write-1{Package #1 Info: #2.}%
        }%
      \else
        \def\x#1#2{\PackageInfo{#1}{#2, stopped}}%
      \fi
      \x{resizegather}{The package is already loaded}%
      \aftergroup\endinput
    \fi
  \fi
\endgroup%
%    \end{macrocode}
%    Package identification:
%    \begin{macrocode}
\begingroup\catcode61\catcode48\catcode32=10\relax%
  \catcode13=5 % ^^M
  \endlinechar=13 %
  \catcode35=6 % #
  \catcode39=12 % '
  \catcode40=12 % (
  \catcode41=12 % )
  \catcode44=12 % ,
  \catcode45=12 % -
  \catcode46=12 % .
  \catcode47=12 % /
  \catcode58=12 % :
  \catcode64=11 % @
  \catcode91=12 % [
  \catcode93=12 % ]
  \catcode123=1 % {
  \catcode125=2 % }
  \expandafter\ifx\csname ProvidesPackage\endcsname\relax
    \def\x#1#2#3[#4]{\endgroup
      \immediate\write-1{Package: #3 #4}%
      \xdef#1{#4}%
    }%
  \else
    \def\x#1#2[#3]{\endgroup
      #2[{#3}]%
      \ifx#1\@undefined
        \xdef#1{#3}%
      \fi
      \ifx#1\relax
        \xdef#1{#3}%
      \fi
    }%
  \fi
\expandafter\x\csname ver@resizegather.sty\endcsname
\ProvidesPackage{resizegather}%
  [2016/05/16 v1.3 Resize overly large equations (HO)]%
%    \end{macrocode}
%
%    \begin{macrocode}
\begingroup\catcode61\catcode48\catcode32=10\relax%
  \catcode13=5 % ^^M
  \endlinechar=13 %
  \catcode123=1 % {
  \catcode125=2 % }
  \catcode64=11 % @
  \def\x{\endgroup
    \expandafter\edef\csname ResizeGather@AtEnd\endcsname{%
      \endlinechar=\the\endlinechar\relax
      \catcode13=\the\catcode13\relax
      \catcode32=\the\catcode32\relax
      \catcode35=\the\catcode35\relax
      \catcode61=\the\catcode61\relax
      \catcode64=\the\catcode64\relax
      \catcode123=\the\catcode123\relax
      \catcode125=\the\catcode125\relax
    }%
  }%
\x\catcode61\catcode48\catcode32=10\relax%
\catcode13=5 % ^^M
\endlinechar=13 %
\catcode35=6 % #
\catcode64=11 % @
\catcode123=1 % {
\catcode125=2 % }
\def\TMP@EnsureCode#1#2{%
  \edef\ResizeGather@AtEnd{%
    \ResizeGather@AtEnd
    \catcode#1=\the\catcode#1\relax
  }%
  \catcode#1=#2\relax
}
\TMP@EnsureCode{10}{12}% ^^J
\TMP@EnsureCode{33}{12}% !
\TMP@EnsureCode{36}{3}% $
\TMP@EnsureCode{38}{4}% &
\TMP@EnsureCode{39}{12}% '
\TMP@EnsureCode{40}{12}% (
\TMP@EnsureCode{41}{12}% )
\TMP@EnsureCode{42}{12}% *
\TMP@EnsureCode{43}{12}% +
\TMP@EnsureCode{44}{12}% ,
\TMP@EnsureCode{45}{12}% -
\TMP@EnsureCode{46}{12}% .
\TMP@EnsureCode{47}{12}% /
\TMP@EnsureCode{58}{12}% :
\TMP@EnsureCode{59}{12}% ;
\TMP@EnsureCode{60}{12}% <
\TMP@EnsureCode{62}{12}% >
\TMP@EnsureCode{63}{12}% ?
\TMP@EnsureCode{91}{12}% [
\TMP@EnsureCode{93}{12}% ]
\TMP@EnsureCode{94}{7}% ^ (superscript)
\TMP@EnsureCode{96}{12}% `
\edef\ResizeGather@AtEnd{\ResizeGather@AtEnd\noexpand\endinput}
%    \end{macrocode}
%
%    \begin{macrocode}
\RequirePackage{kvoptions}[2009/12/04]
\SetupKeyvalOptions{%
  family=resizegather,%
  prefix=ResizeGather@,%
}
%    \end{macrocode}
%    \begin{macrocode}
\@for\ResizeGather@option:=%
  centertags,%
  tbtags,%
  sumlimits,%
  nosumlimits,%
  intlimits,%
  nointlimits,%
  nonamelimits,%
  leqno,%
  reqno,%
  fleqn%
\do{%
  \edef\ResizeGather@temp{%
    \noexpand\DeclareVoidOption{\ResizeGather@option}{%
      \noexpand\PassOptionsToPackage{amsmath}{\ResizeGather@option}%
    }%
    \noexpand\AtEndOfPackage{%
      \noexpand\DisableKeyvalOption[%
        action=error,%
        package=resizegather,%
      ]{resizegather}{\ResizeGather@option}%
    }%
  }%
  \ResizeGather@temp
}
\@for\ResizeGather@option:=%
  draft,%
  final,%
  hiderotate,%
  hidescale,%
  hiresbb,%
  demo,%
  dvips,xdvi,dvipdf,dvipdfm,dvipdfmx,pdftex,dvipsone,%
  dviwindo,emtex,dviwin,pctexps,pctexwin,pctexhp,pctex32,%
  truetex,tcidvi,vtex,oztex,textures,xetex%
\do{%
  \edef\ResizeGather@temp{%
    \noexpand\DeclareVoidOption{\ResizeGather@option}{%
      \noexpand\PassOptionsToPackage{graphics}{\ResizeGather@option}%
    }%
    \noexpand\AtEndOfPackage{%
      \noexpand\DisableKeyvalOption[%
        action=error,%
        package=resizegather,%
      ]{resizegather}{\ResizeGather@option}%
    }%
  }%
  \ResizeGather@temp
}
%    \end{macrocode}
%
%    \begin{macrocode}
\DeclareBoolOption[true]{enable}
\DeclareComplementaryOption{disable}{enable}
\DeclareStringOption[.05]{warningthreshold}
\newif\ifResizeGather@NeedInit
\DeclareBoolOption[true]{equations}
\DeclareBoolOption[true]{equation}
\DeclareBoolOption[true]{displaymath}
\AddToKeyvalOption*{equations}{%
  \ResizeGather@NeedInittrue
  \ifResizeGather@equations
    \ResizeGather@equationtrue
    \ResizeGather@displaymathtrue
  \else
    \ResizeGather@equationfalse
    \ResizeGather@displaymathfalse
  \fi
}
\AddToKeyvalOption*{equation}{%
  \ResizeGather@NeedInittrue
}
\AddToKeyvalOption*{displaymath}{%
  \ResizeGather@NeedInittrue
}
%    \end{macrocode}
%
%    \begin{macro}{\resizegathersetup}
%    \begin{macrocode}
\newcommand*{\resizegathersetup}[1]{%
  \ResizeGather@NeedInitfalse
  \setkeys{resizegather}{#1}%
  \ifResizeGather@NeedInit
    \ResizeGather@init
  \fi
}
\let\ResizeGather@init\relax
%    \end{macrocode}
%    \end{macro}
%    \begin{macrocode}
\InputIfFileExists{resizegather.cfg}{}{}%
\ProcessKeyvalOptions*\relax
%    \end{macrocode}
%    \begin{macrocode}
\RequirePackage{amsmath}
\RequirePackage{graphics}
%    \end{macrocode}
%    \begin{macro}{\ResizeGather@redefine}
%    \begin{macrocode}
\def\ResizeGather@redefine#1#2#3#4#5{%
  \csname ifResizeGather@#1\endcsname
    \@ifundefined{ResizeGather@org@#2}{%
      \expandafter\let\csname ResizeGather@org@#2\expandafter\endcsname
                      \csname #2\endcsname
    }{}%
    \@ifundefined{ResizeGather@org@#3}{%
      \expandafter\let\csname ResizeGather@org@#3\expandafter\endcsname
                      \csname #3\endcsname
    }{}%
    \expandafter\edef\csname #2\endcsname{%
      \expandafter\noexpand\csname#4\endcsname
    }%
    \expandafter\edef\csname #3\endcsname{%
      \expandafter\noexpand\csname#5\endcsname
    }%
  \else
    \@ifundefined{ResizeGather@org@#2}{}{%
      \expandafter\let\csname #2\expandafter\endcsname
                      \csname ResizeGather@org@#2\endcsname
      \expandafter\let\csname #3\expandafter\endcsname
                      \csname ResizeGather@org@#3\endcsname
    }%
  \fi
}
%    \end{macrocode}
%    \end{macro}
%    \begin{macro}{\ResizeGather@init}
%    \begin{macrocode}
\def\ResizeGather@init{%
  \ResizeGather@redefine{equation}{equation}{endequation}%
    {gather}{endgather}%
  \ResizeGather@redefine{displaymath}{displaymath}{enddisplaymath}%
    {gather*}{endgather*}%
}
\ResizeGather@init
%    \end{macrocode}
%    \end{macro}
%
%    \begin{macro}{\ResizeGather@ResizeGather}
%    \begin{macrocode}
\def\ResizeGather@ResizeGather{%
  \ifResizeGather@enable
    \dimen@\displaywidth
    \if@fleqn
      \advance\dimen@-\@mathmargin
    \fi
    \ifdim\wdz@>\dimen@
      \begingroup
        \advance\dimen@ -\wdz@
        \dimen@ -\dimen@
        \ifdim\ResizeGather@warningthreshold\wdz@>\dimen@
          \expandafter\PackageInfo
        \else
          \expandafter\PackageWarning
        \fi
        {resizegather}{%
          Equation line \the\row@\space is too large %
          by \the\dimen@\MessageBreak
          in environment `\@currenvir'%
        }%
      \endgroup
      \setboxz@h to\dimen@{%
        \resizebox{\dimen@}{!}{\boxz@}%
        \hss
      }%
    \fi
  \fi
}
%    \end{macrocode}
%    \end{macro}
%    \begin{macro}{\calc@shift@gather}
%    \begin{macrocode}
\expandafter\def\expandafter\calc@shift@gather\expandafter{%
  \expandafter\ResizeGather@ResizeGather
  \calc@shift@gather
}
%    \end{macrocode}
%    \end{macro}
%    \begin{macro}{\ResizeGather@org@gmeasure@}
%    \begin{macrocode}
\let\ResizeGather@org@gmeasure@\gmeasure@
%    \end{macrocode}
%    \end{macro}
%    \begin{macro}{\gmeasure@}
%    \begin{macrocode}
\def\gmeasure@#1{%
  \ResizeGather@org@gmeasure@{#1}%
  \ifResizeGather@enable
    \ifdim\totwidth@>\displaywidth
      \totwidth@=\displaywidth
    \fi
  \fi
}
%    \end{macrocode}
%    \end{macro}
%
%    \begin{macrocode}
\ResizeGather@AtEnd%
%</package>
%    \end{macrocode}
%
% \section{Test}
%
% \subsection{Catcode checks for loading}
%
%    \begin{macrocode}
%<*test1>
%    \end{macrocode}
%    \begin{macrocode}
\catcode`\{=1 %
\catcode`\}=2 %
\catcode`\#=6 %
\catcode`\@=11 %
\expandafter\ifx\csname count@\endcsname\relax
  \countdef\count@=255 %
\fi
\expandafter\ifx\csname @gobble\endcsname\relax
  \long\def\@gobble#1{}%
\fi
\expandafter\ifx\csname @firstofone\endcsname\relax
  \long\def\@firstofone#1{#1}%
\fi
\expandafter\ifx\csname loop\endcsname\relax
  \expandafter\@firstofone
\else
  \expandafter\@gobble
\fi
{%
  \def\loop#1\repeat{%
    \def\body{#1}%
    \iterate
  }%
  \def\iterate{%
    \body
      \let\next\iterate
    \else
      \let\next\relax
    \fi
    \next
  }%
  \let\repeat=\fi
}%
\def\RestoreCatcodes{}
\count@=0 %
\loop
  \edef\RestoreCatcodes{%
    \RestoreCatcodes
    \catcode\the\count@=\the\catcode\count@\relax
  }%
\ifnum\count@<255 %
  \advance\count@ 1 %
\repeat

\def\RangeCatcodeInvalid#1#2{%
  \count@=#1\relax
  \loop
    \catcode\count@=15 %
  \ifnum\count@<#2\relax
    \advance\count@ 1 %
  \repeat
}
\def\RangeCatcodeCheck#1#2#3{%
  \count@=#1\relax
  \loop
    \ifnum#3=\catcode\count@
    \else
      \errmessage{%
        Character \the\count@\space
        with wrong catcode \the\catcode\count@\space
        instead of \number#3%
      }%
    \fi
  \ifnum\count@<#2\relax
    \advance\count@ 1 %
  \repeat
}
\def\space{ }
\expandafter\ifx\csname LoadCommand\endcsname\relax
  \def\LoadCommand{\input resizegather.sty\relax}%
\fi
\def\Test{%
  \RangeCatcodeInvalid{0}{47}%
  \RangeCatcodeInvalid{58}{64}%
  \RangeCatcodeInvalid{91}{96}%
  \RangeCatcodeInvalid{123}{255}%
  \catcode`\@=12 %
  \catcode`\\=0 %
  \catcode`\%=14 %
  \LoadCommand
  \RangeCatcodeCheck{0}{36}{15}%
  \RangeCatcodeCheck{37}{37}{14}%
  \RangeCatcodeCheck{38}{47}{15}%
  \RangeCatcodeCheck{48}{57}{12}%
  \RangeCatcodeCheck{58}{63}{15}%
  \RangeCatcodeCheck{64}{64}{12}%
  \RangeCatcodeCheck{65}{90}{11}%
  \RangeCatcodeCheck{91}{91}{15}%
  \RangeCatcodeCheck{92}{92}{0}%
  \RangeCatcodeCheck{93}{96}{15}%
  \RangeCatcodeCheck{97}{122}{11}%
  \RangeCatcodeCheck{123}{255}{15}%
  \RestoreCatcodes
}
\Test
\csname @@end\endcsname
\end
%    \end{macrocode}
%    \begin{macrocode}
%</test1>
%    \end{macrocode}
%
% \section{Installation}
%
% \subsection{Download}
%
% \paragraph{Package.} This package is available on
% CTAN\footnote{\url{https://ctan.org/pkg/resizegather}}:
% \begin{description}
% \item[\CTAN{macros/latex/contrib/oberdiek/resizegather.dtx}] The source file.
% \item[\CTAN{macros/latex/contrib/oberdiek/resizegather.pdf}] Documentation.
% \end{description}
%
%
% \paragraph{Bundle.} All the packages of the bundle `oberdiek'
% are also available in a TDS compliant ZIP archive. There
% the packages are already unpacked and the documentation files
% are generated. The files and directories obey the TDS standard.
% \begin{description}
% \item[\CTANinstall{install/macros/latex/contrib/oberdiek.tds.zip}]
% \end{description}
% \emph{TDS} refers to the standard ``A Directory Structure
% for \TeX\ Files'' (\CTAN{tds/tds.pdf}). Directories
% with \xfile{texmf} in their name are usually organized this way.
%
% \subsection{Bundle installation}
%
% \paragraph{Unpacking.} Unpack the \xfile{oberdiek.tds.zip} in the
% TDS tree (also known as \xfile{texmf} tree) of your choice.
% Example (linux):
% \begin{quote}
%   |unzip oberdiek.tds.zip -d ~/texmf|
% \end{quote}
%
% \paragraph{Script installation.}
% Check the directory \xfile{TDS:scripts/oberdiek/} for
% scripts that need further installation steps.
% Package \xpackage{attachfile2} comes with the Perl script
% \xfile{pdfatfi.pl} that should be installed in such a way
% that it can be called as \texttt{pdfatfi}.
% Example (linux):
% \begin{quote}
%   |chmod +x scripts/oberdiek/pdfatfi.pl|\\
%   |cp scripts/oberdiek/pdfatfi.pl /usr/local/bin/|
% \end{quote}
%
% \subsection{Package installation}
%
% \paragraph{Unpacking.} The \xfile{.dtx} file is a self-extracting
% \docstrip\ archive. The files are extracted by running the
% \xfile{.dtx} through \plainTeX:
% \begin{quote}
%   \verb|tex resizegather.dtx|
% \end{quote}
%
% \paragraph{TDS.} Now the different files must be moved into
% the different directories in your installation TDS tree
% (also known as \xfile{texmf} tree):
% \begin{quote}
% \def\t{^^A
% \begin{tabular}{@{}>{\ttfamily}l@{ $\rightarrow$ }>{\ttfamily}l@{}}
%   resizegather.sty & tex/latex/oberdiek/resizegather.sty\\
%   resizegather.pdf & doc/latex/oberdiek/resizegather.pdf\\
%   test/resizegather-test1.tex & doc/latex/oberdiek/test/resizegather-test1.tex\\
%   resizegather.dtx & source/latex/oberdiek/resizegather.dtx\\
% \end{tabular}^^A
% }^^A
% \sbox0{\t}^^A
% \ifdim\wd0>\linewidth
%   \begingroup
%     \advance\linewidth by\leftmargin
%     \advance\linewidth by\rightmargin
%   \edef\x{\endgroup
%     \def\noexpand\lw{\the\linewidth}^^A
%   }\x
%   \def\lwbox{^^A
%     \leavevmode
%     \hbox to \linewidth{^^A
%       \kern-\leftmargin\relax
%       \hss
%       \usebox0
%       \hss
%       \kern-\rightmargin\relax
%     }^^A
%   }^^A
%   \ifdim\wd0>\lw
%     \sbox0{\small\t}^^A
%     \ifdim\wd0>\linewidth
%       \ifdim\wd0>\lw
%         \sbox0{\footnotesize\t}^^A
%         \ifdim\wd0>\linewidth
%           \ifdim\wd0>\lw
%             \sbox0{\scriptsize\t}^^A
%             \ifdim\wd0>\linewidth
%               \ifdim\wd0>\lw
%                 \sbox0{\tiny\t}^^A
%                 \ifdim\wd0>\linewidth
%                   \lwbox
%                 \else
%                   \usebox0
%                 \fi
%               \else
%                 \lwbox
%               \fi
%             \else
%               \usebox0
%             \fi
%           \else
%             \lwbox
%           \fi
%         \else
%           \usebox0
%         \fi
%       \else
%         \lwbox
%       \fi
%     \else
%       \usebox0
%     \fi
%   \else
%     \lwbox
%   \fi
% \else
%   \usebox0
% \fi
% \end{quote}
% If you have a \xfile{docstrip.cfg} that configures and enables \docstrip's
% TDS installing feature, then some files can already be in the right
% place, see the documentation of \docstrip.
%
% \subsection{Refresh file name databases}
%
% If your \TeX~distribution
% (\teTeX, \mikTeX, \dots) relies on file name databases, you must refresh
% these. For example, \teTeX\ users run \verb|texhash| or
% \verb|mktexlsr|.
%
% \subsection{Some details for the interested}
%
% \paragraph{Attached source.}
%
% The PDF documentation on CTAN also includes the
% \xfile{.dtx} source file. It can be extracted by
% AcrobatReader 6 or higher. Another option is \textsf{pdftk},
% e.g. unpack the file into the current directory:
% \begin{quote}
%   \verb|pdftk resizegather.pdf unpack_files output .|
% \end{quote}
%
% \paragraph{Unpacking with \LaTeX.}
% The \xfile{.dtx} chooses its action depending on the format:
% \begin{description}
% \item[\plainTeX:] Run \docstrip\ and extract the files.
% \item[\LaTeX:] Generate the documentation.
% \end{description}
% If you insist on using \LaTeX\ for \docstrip\ (really,
% \docstrip\ does not need \LaTeX), then inform the autodetect routine
% about your intention:
% \begin{quote}
%   \verb|latex \let\install=y% \iffalse meta-comment
%
% File: resizegather.dtx
% Version: 2016/05/16 v1.3
% Info: Resize overly large equations
%
% Copyright (C) 2009, 2010 by
%    Heiko Oberdiek <heiko.oberdiek at googlemail.com>
%    2016
%    https://github.com/ho-tex/oberdiek/issues
%
% This work may be distributed and/or modified under the
% conditions of the LaTeX Project Public License, either
% version 1.3c of this license or (at your option) any later
% version. This version of this license is in
%    http://www.latex-project.org/lppl/lppl-1-3c.txt
% and the latest version of this license is in
%    http://www.latex-project.org/lppl.txt
% and version 1.3 or later is part of all distributions of
% LaTeX version 2005/12/01 or later.
%
% This work has the LPPL maintenance status "maintained".
%
% This Current Maintainer of this work is Heiko Oberdiek.
%
% This work consists of the main source file resizegather.dtx
% and the derived files
%    resizegather.sty, resizegather.pdf, resizegather.ins, resizegather.drv,
%    resizegather-test1.tex.
%
% Distribution:
%    CTAN:macros/latex/contrib/oberdiek/resizegather.dtx
%    CTAN:macros/latex/contrib/oberdiek/resizegather.pdf
%
% Unpacking:
%    (a) If resizegather.ins is present:
%           tex resizegather.ins
%    (b) Without resizegather.ins:
%           tex resizegather.dtx
%    (c) If you insist on using LaTeX
%           latex \let\install=y\input{resizegather.dtx}
%        (quote the arguments according to the demands of your shell)
%
% Documentation:
%    (a) If resizegather.drv is present:
%           latex resizegather.drv
%    (b) Without resizegather.drv:
%           latex resizegather.dtx; ...
%    The class ltxdoc loads the configuration file ltxdoc.cfg
%    if available. Here you can specify further options, e.g.
%    use A4 as paper format:
%       \PassOptionsToClass{a4paper}{article}
%
%    Programm calls to get the documentation (example):
%       pdflatex resizegather.dtx
%       makeindex -s gind.ist resizegather.idx
%       pdflatex resizegather.dtx
%       makeindex -s gind.ist resizegather.idx
%       pdflatex resizegather.dtx
%
% Installation:
%    TDS:tex/latex/oberdiek/resizegather.sty
%    TDS:doc/latex/oberdiek/resizegather.pdf
%    TDS:doc/latex/oberdiek/test/resizegather-test1.tex
%    TDS:source/latex/oberdiek/resizegather.dtx
%
%<*ignore>
\begingroup
  \catcode123=1 %
  \catcode125=2 %
  \def\x{LaTeX2e}%
\expandafter\endgroup
\ifcase 0\ifx\install y1\fi\expandafter
         \ifx\csname processbatchFile\endcsname\relax\else1\fi
         \ifx\fmtname\x\else 1\fi\relax
\else\csname fi\endcsname
%</ignore>
%<*install>
\input docstrip.tex
\Msg{************************************************************************}
\Msg{* Installation}
\Msg{* Package: resizegather 2016/05/16 v1.3 Resize overly large equations (HO)}
\Msg{************************************************************************}

\keepsilent
\askforoverwritefalse

\let\MetaPrefix\relax
\preamble

This is a generated file.

Project: resizegather
Version: 2016/05/16 v1.3

Copyright (C) 2009, 2010 by
   Heiko Oberdiek <heiko.oberdiek at googlemail.com>

This work may be distributed and/or modified under the
conditions of the LaTeX Project Public License, either
version 1.3c of this license or (at your option) any later
version. This version of this license is in
   http://www.latex-project.org/lppl/lppl-1-3c.txt
and the latest version of this license is in
   http://www.latex-project.org/lppl.txt
and version 1.3 or later is part of all distributions of
LaTeX version 2005/12/01 or later.

This work has the LPPL maintenance status "maintained".

This Current Maintainer of this work is Heiko Oberdiek.

This work consists of the main source file resizegather.dtx
and the derived files
   resizegather.sty, resizegather.pdf, resizegather.ins, resizegather.drv,
   resizegather-test1.tex.

\endpreamble
\let\MetaPrefix\DoubleperCent

\generate{%
  \file{resizegather.ins}{\from{resizegather.dtx}{install}}%
  \file{resizegather.drv}{\from{resizegather.dtx}{driver}}%
  \usedir{tex/latex/oberdiek}%
  \file{resizegather.sty}{\from{resizegather.dtx}{package}}%
%  \usedir{doc/latex/oberdiek/test}%
%  \file{resizegather-test1.tex}{\from{resizegather.dtx}{test1}}%
  \nopreamble
  \nopostamble
%  \usedir{source/latex/oberdiek/catalogue}%
%  \file{resizegather.xml}{\from{resizegather.dtx}{catalogue}}%
}

\catcode32=13\relax% active space
\let =\space%
\Msg{************************************************************************}
\Msg{*}
\Msg{* To finish the installation you have to move the following}
\Msg{* file into a directory searched by TeX:}
\Msg{*}
\Msg{*     resizegather.sty}
\Msg{*}
\Msg{* To produce the documentation run the file `resizegather.drv'}
\Msg{* through LaTeX.}
\Msg{*}
\Msg{* Happy TeXing!}
\Msg{*}
\Msg{************************************************************************}

\endbatchfile
%</install>
%<*ignore>
\fi
%</ignore>
%<*driver>
\NeedsTeXFormat{LaTeX2e}
\ProvidesFile{resizegather.drv}%
  [2016/05/16 v1.3 Resize overly large equations (HO)]%
\documentclass{ltxdoc}
\usepackage{holtxdoc}[2011/11/22]
\usepackage{ifluatex}
\ifluatex
\else
  \usepackage[T1]{fontenc}%
  \usepackage{textcomp}%
  \usepackage{lmodern}%
\fi
\begin{document}
  \DocInput{resizegather.dtx}%
\end{document}
%</driver>
% \fi
%
%
% \CharacterTable
%  {Upper-case    \A\B\C\D\E\F\G\H\I\J\K\L\M\N\O\P\Q\R\S\T\U\V\W\X\Y\Z
%   Lower-case    \a\b\c\d\e\f\g\h\i\j\k\l\m\n\o\p\q\r\s\t\u\v\w\x\y\z
%   Digits        \0\1\2\3\4\5\6\7\8\9
%   Exclamation   \!     Double quote  \"     Hash (number) \#
%   Dollar        \$     Percent       \%     Ampersand     \&
%   Acute accent  \'     Left paren    \(     Right paren   \)
%   Asterisk      \*     Plus          \+     Comma         \,
%   Minus         \-     Point         \.     Solidus       \/
%   Colon         \:     Semicolon     \;     Less than     \<
%   Equals        \=     Greater than  \>     Question mark \?
%   Commercial at \@     Left bracket  \[     Backslash     \\
%   Right bracket \]     Circumflex    \^     Underscore    \_
%   Grave accent  \`     Left brace    \{     Vertical bar  \|
%   Right brace   \}     Tilde         \~}
%
% \GetFileInfo{resizegather.drv}
%
% \title{The \xpackage{resizegather} package}
% \date{2016/05/16 v1.3}
% \author{Heiko Oberdiek\thanks
% {Please report any issues at https://github.com/ho-tex/oberdiek/issues}\\
% \xemail{heiko.oberdiek at googlemail.com}}
%
% \maketitle
%
% \begin{abstract}
% Equations that are too large are resized to fit the available
% space. The environment \textsf{gather} of package \xpackage{amsmath}
% is supported. Also the environments \textsf{equation} and
% \textsf{displaymath} are redefined using \textsf{gather}
% and its starred version.
% \end{abstract}
%
% \tableofcontents
%
% \makeatletter
% \def\env#1{^^A
%    \textsf{\@env#1*\@nil}^^A
% }%
% \def\@env#1*#2\@nil{^^A
%   #1^^A
%   \ifx\\#2\\^^A
%     \expandafter\@gobble
%   \else
%     \textasteriskcentered
%     \expandafter\@firstofone
%   \fi
%   {\@env#2\@nil}^^A
% }
% \makeatother
%
% \section{Documentation}
%
% Sometimes an equation is just a little to large to fit in the
% line. And breaking the equation across lines might be worse
% than downscaling the equation. This package implements this
% for the environments \env{gather} and \env{gather*} of
% package \xpackage{amsmath}. That package already measures
% the equations and simplifies the implementation of \xpackage{resizegather}
% that only needs to hook into \xpackage{amsmath}'s code to add
% the resizing feature.
%
% Resized equations are recorded in the \xext{log} file
% for small exceeds (default setting is smaller than five percent).
% Otherwise a warning is given.
%
% Also environments \env{equation} and \env{displaymath}
% are supported by redefining them using \env{gather}
% and \env{gather*}.
%
% \cs{[} and \cs{]} are not supported, because these macros
% are not in environment form that is required for
% \xpackage{amsmath}. The environment body is collected
% first to be able to process the body twice for measuring
% first.
%
% Also the environments using alignments are not supported.
% If a single equation line would be resized, the alignment
% would get lost. And resizing all equations of the alignment
% does not seem appropriate either.
%
% \subsection{Options}
%
% \begin{description}
% \item[\xoption{warningthreshold}:]
%   Print a warning if the original equation line exceeds
%   its available width by the given fraction.
%   Default is |0.05|: A warning is given if the equation
%   is too large by five percent.
%   Otherwise the exceed is recorded in the \xext{log} file
%   only.
% \end{description}
% The next options are boolean options. They are enabled
% by value |true| or if no value is given. They are switched
% off by value |false|.
% \begin{description}
% \item[\xoption{enable}:] The resize feature is active (default).
% \item[\xoption{disable}:] The complementary option for \xoption{enable},
%  added for convenience: |disable| (or |disable=true|) is the same
%  as |enable=false|.
% \item[\xoption{equations}:]
%   \LaTeX\ environments \textsf{equation} and \textsf{displaymath}
%   environments are redefined. These equations
%   are now using environment \env{gather} and
%   \env{gather*}. This is the default.
% \end{description}
% The following table shows additional options if you
% want to have finer control for the redefined
% environments:
% \begin{quote}
% \def\unchanged{\textit{unchanged}}
% \def\notprovided{\textit{not provided}}
% \begin{tabular}{l|ll}
% &\multicolumn{2}{c}{Environments}\\
% Option & \env{equation} & \env{displaymath}\\
% \hline
% \xoption{equations} & \env{gather} & \env{gather*}\\
% \xoption{equation} & \env{gather} & \unchanged\\
% \xoption{displaymath} & \unchanged & \env{gather*}\\
% \end{tabular}
% \end{quote}
% If such an option is switched off, the original meaning
% of the affected environments is restored.
%
% Options are evaluated in the following order:
% \begin{enumerate}
% \item
%  Configuration file \xfile{resizegather.cfg} using \cs{resizegathersetup}
%  if the file exists.
%  \item
%  Package options given for \cs{usepackage}.
%  \item
%  Later calls of \cs{resizegathersetup}.
% \end{enumerate}
% \begin{declcs}{resizegathersetup}\M{option list}
% \end{declcs}
% The options are key value options. Boolean options are enabled by
% default (without value) or by using the explicit value \texttt{true}.
% Value \texttt{false} disable the option.
%
% \subsection{Options for packages \xpackage{amsmath} or \xpackage{graphics}}
%
% The package loads the package \xpackage{amsmath} because is internally
% measures the equations first. Thus this package hooks into this code
% in order to resize the equations if they are too large.
% The resizing itself is done by \cs{resizebox} of package \xpackage{graphics}.
% If you need special options for these packages, just load them first or
% use global options when appropriate. Example:
% \begin{quote}
%\begin{verbatim}
%\usepackage[dvipdfm]{graphicx}% or graphics
%\usepackage[fleqn]{amsmath}
%\usepackage{resizegather}
%\end{verbatim}
%\end{quote}
%
% \StopEventually{
% }
%
% \section{Implementation}
%    \begin{macrocode}
%<*package>
%    \end{macrocode}
%    Reload check, especially if the package is not used with \LaTeX.
%    \begin{macrocode}
\begingroup\catcode61\catcode48\catcode32=10\relax%
  \catcode13=5 % ^^M
  \endlinechar=13 %
  \catcode35=6 % #
  \catcode39=12 % '
  \catcode44=12 % ,
  \catcode45=12 % -
  \catcode46=12 % .
  \catcode58=12 % :
  \catcode64=11 % @
  \catcode123=1 % {
  \catcode125=2 % }
  \expandafter\let\expandafter\x\csname ver@resizegather.sty\endcsname
  \ifx\x\relax % plain-TeX, first loading
  \else
    \def\empty{}%
    \ifx\x\empty % LaTeX, first loading,
      % variable is initialized, but \ProvidesPackage not yet seen
    \else
      \expandafter\ifx\csname PackageInfo\endcsname\relax
        \def\x#1#2{%
          \immediate\write-1{Package #1 Info: #2.}%
        }%
      \else
        \def\x#1#2{\PackageInfo{#1}{#2, stopped}}%
      \fi
      \x{resizegather}{The package is already loaded}%
      \aftergroup\endinput
    \fi
  \fi
\endgroup%
%    \end{macrocode}
%    Package identification:
%    \begin{macrocode}
\begingroup\catcode61\catcode48\catcode32=10\relax%
  \catcode13=5 % ^^M
  \endlinechar=13 %
  \catcode35=6 % #
  \catcode39=12 % '
  \catcode40=12 % (
  \catcode41=12 % )
  \catcode44=12 % ,
  \catcode45=12 % -
  \catcode46=12 % .
  \catcode47=12 % /
  \catcode58=12 % :
  \catcode64=11 % @
  \catcode91=12 % [
  \catcode93=12 % ]
  \catcode123=1 % {
  \catcode125=2 % }
  \expandafter\ifx\csname ProvidesPackage\endcsname\relax
    \def\x#1#2#3[#4]{\endgroup
      \immediate\write-1{Package: #3 #4}%
      \xdef#1{#4}%
    }%
  \else
    \def\x#1#2[#3]{\endgroup
      #2[{#3}]%
      \ifx#1\@undefined
        \xdef#1{#3}%
      \fi
      \ifx#1\relax
        \xdef#1{#3}%
      \fi
    }%
  \fi
\expandafter\x\csname ver@resizegather.sty\endcsname
\ProvidesPackage{resizegather}%
  [2016/05/16 v1.3 Resize overly large equations (HO)]%
%    \end{macrocode}
%
%    \begin{macrocode}
\begingroup\catcode61\catcode48\catcode32=10\relax%
  \catcode13=5 % ^^M
  \endlinechar=13 %
  \catcode123=1 % {
  \catcode125=2 % }
  \catcode64=11 % @
  \def\x{\endgroup
    \expandafter\edef\csname ResizeGather@AtEnd\endcsname{%
      \endlinechar=\the\endlinechar\relax
      \catcode13=\the\catcode13\relax
      \catcode32=\the\catcode32\relax
      \catcode35=\the\catcode35\relax
      \catcode61=\the\catcode61\relax
      \catcode64=\the\catcode64\relax
      \catcode123=\the\catcode123\relax
      \catcode125=\the\catcode125\relax
    }%
  }%
\x\catcode61\catcode48\catcode32=10\relax%
\catcode13=5 % ^^M
\endlinechar=13 %
\catcode35=6 % #
\catcode64=11 % @
\catcode123=1 % {
\catcode125=2 % }
\def\TMP@EnsureCode#1#2{%
  \edef\ResizeGather@AtEnd{%
    \ResizeGather@AtEnd
    \catcode#1=\the\catcode#1\relax
  }%
  \catcode#1=#2\relax
}
\TMP@EnsureCode{10}{12}% ^^J
\TMP@EnsureCode{33}{12}% !
\TMP@EnsureCode{36}{3}% $
\TMP@EnsureCode{38}{4}% &
\TMP@EnsureCode{39}{12}% '
\TMP@EnsureCode{40}{12}% (
\TMP@EnsureCode{41}{12}% )
\TMP@EnsureCode{42}{12}% *
\TMP@EnsureCode{43}{12}% +
\TMP@EnsureCode{44}{12}% ,
\TMP@EnsureCode{45}{12}% -
\TMP@EnsureCode{46}{12}% .
\TMP@EnsureCode{47}{12}% /
\TMP@EnsureCode{58}{12}% :
\TMP@EnsureCode{59}{12}% ;
\TMP@EnsureCode{60}{12}% <
\TMP@EnsureCode{62}{12}% >
\TMP@EnsureCode{63}{12}% ?
\TMP@EnsureCode{91}{12}% [
\TMP@EnsureCode{93}{12}% ]
\TMP@EnsureCode{94}{7}% ^ (superscript)
\TMP@EnsureCode{96}{12}% `
\edef\ResizeGather@AtEnd{\ResizeGather@AtEnd\noexpand\endinput}
%    \end{macrocode}
%
%    \begin{macrocode}
\RequirePackage{kvoptions}[2009/12/04]
\SetupKeyvalOptions{%
  family=resizegather,%
  prefix=ResizeGather@,%
}
%    \end{macrocode}
%    \begin{macrocode}
\@for\ResizeGather@option:=%
  centertags,%
  tbtags,%
  sumlimits,%
  nosumlimits,%
  intlimits,%
  nointlimits,%
  nonamelimits,%
  leqno,%
  reqno,%
  fleqn%
\do{%
  \edef\ResizeGather@temp{%
    \noexpand\DeclareVoidOption{\ResizeGather@option}{%
      \noexpand\PassOptionsToPackage{amsmath}{\ResizeGather@option}%
    }%
    \noexpand\AtEndOfPackage{%
      \noexpand\DisableKeyvalOption[%
        action=error,%
        package=resizegather,%
      ]{resizegather}{\ResizeGather@option}%
    }%
  }%
  \ResizeGather@temp
}
\@for\ResizeGather@option:=%
  draft,%
  final,%
  hiderotate,%
  hidescale,%
  hiresbb,%
  demo,%
  dvips,xdvi,dvipdf,dvipdfm,dvipdfmx,pdftex,dvipsone,%
  dviwindo,emtex,dviwin,pctexps,pctexwin,pctexhp,pctex32,%
  truetex,tcidvi,vtex,oztex,textures,xetex%
\do{%
  \edef\ResizeGather@temp{%
    \noexpand\DeclareVoidOption{\ResizeGather@option}{%
      \noexpand\PassOptionsToPackage{graphics}{\ResizeGather@option}%
    }%
    \noexpand\AtEndOfPackage{%
      \noexpand\DisableKeyvalOption[%
        action=error,%
        package=resizegather,%
      ]{resizegather}{\ResizeGather@option}%
    }%
  }%
  \ResizeGather@temp
}
%    \end{macrocode}
%
%    \begin{macrocode}
\DeclareBoolOption[true]{enable}
\DeclareComplementaryOption{disable}{enable}
\DeclareStringOption[.05]{warningthreshold}
\newif\ifResizeGather@NeedInit
\DeclareBoolOption[true]{equations}
\DeclareBoolOption[true]{equation}
\DeclareBoolOption[true]{displaymath}
\AddToKeyvalOption*{equations}{%
  \ResizeGather@NeedInittrue
  \ifResizeGather@equations
    \ResizeGather@equationtrue
    \ResizeGather@displaymathtrue
  \else
    \ResizeGather@equationfalse
    \ResizeGather@displaymathfalse
  \fi
}
\AddToKeyvalOption*{equation}{%
  \ResizeGather@NeedInittrue
}
\AddToKeyvalOption*{displaymath}{%
  \ResizeGather@NeedInittrue
}
%    \end{macrocode}
%
%    \begin{macro}{\resizegathersetup}
%    \begin{macrocode}
\newcommand*{\resizegathersetup}[1]{%
  \ResizeGather@NeedInitfalse
  \setkeys{resizegather}{#1}%
  \ifResizeGather@NeedInit
    \ResizeGather@init
  \fi
}
\let\ResizeGather@init\relax
%    \end{macrocode}
%    \end{macro}
%    \begin{macrocode}
\InputIfFileExists{resizegather.cfg}{}{}%
\ProcessKeyvalOptions*\relax
%    \end{macrocode}
%    \begin{macrocode}
\RequirePackage{amsmath}
\RequirePackage{graphics}
%    \end{macrocode}
%    \begin{macro}{\ResizeGather@redefine}
%    \begin{macrocode}
\def\ResizeGather@redefine#1#2#3#4#5{%
  \csname ifResizeGather@#1\endcsname
    \@ifundefined{ResizeGather@org@#2}{%
      \expandafter\let\csname ResizeGather@org@#2\expandafter\endcsname
                      \csname #2\endcsname
    }{}%
    \@ifundefined{ResizeGather@org@#3}{%
      \expandafter\let\csname ResizeGather@org@#3\expandafter\endcsname
                      \csname #3\endcsname
    }{}%
    \expandafter\edef\csname #2\endcsname{%
      \expandafter\noexpand\csname#4\endcsname
    }%
    \expandafter\edef\csname #3\endcsname{%
      \expandafter\noexpand\csname#5\endcsname
    }%
  \else
    \@ifundefined{ResizeGather@org@#2}{}{%
      \expandafter\let\csname #2\expandafter\endcsname
                      \csname ResizeGather@org@#2\endcsname
      \expandafter\let\csname #3\expandafter\endcsname
                      \csname ResizeGather@org@#3\endcsname
    }%
  \fi
}
%    \end{macrocode}
%    \end{macro}
%    \begin{macro}{\ResizeGather@init}
%    \begin{macrocode}
\def\ResizeGather@init{%
  \ResizeGather@redefine{equation}{equation}{endequation}%
    {gather}{endgather}%
  \ResizeGather@redefine{displaymath}{displaymath}{enddisplaymath}%
    {gather*}{endgather*}%
}
\ResizeGather@init
%    \end{macrocode}
%    \end{macro}
%
%    \begin{macro}{\ResizeGather@ResizeGather}
%    \begin{macrocode}
\def\ResizeGather@ResizeGather{%
  \ifResizeGather@enable
    \dimen@\displaywidth
    \if@fleqn
      \advance\dimen@-\@mathmargin
    \fi
    \ifdim\wdz@>\dimen@
      \begingroup
        \advance\dimen@ -\wdz@
        \dimen@ -\dimen@
        \ifdim\ResizeGather@warningthreshold\wdz@>\dimen@
          \expandafter\PackageInfo
        \else
          \expandafter\PackageWarning
        \fi
        {resizegather}{%
          Equation line \the\row@\space is too large %
          by \the\dimen@\MessageBreak
          in environment `\@currenvir'%
        }%
      \endgroup
      \setboxz@h to\dimen@{%
        \resizebox{\dimen@}{!}{\boxz@}%
        \hss
      }%
    \fi
  \fi
}
%    \end{macrocode}
%    \end{macro}
%    \begin{macro}{\calc@shift@gather}
%    \begin{macrocode}
\expandafter\def\expandafter\calc@shift@gather\expandafter{%
  \expandafter\ResizeGather@ResizeGather
  \calc@shift@gather
}
%    \end{macrocode}
%    \end{macro}
%    \begin{macro}{\ResizeGather@org@gmeasure@}
%    \begin{macrocode}
\let\ResizeGather@org@gmeasure@\gmeasure@
%    \end{macrocode}
%    \end{macro}
%    \begin{macro}{\gmeasure@}
%    \begin{macrocode}
\def\gmeasure@#1{%
  \ResizeGather@org@gmeasure@{#1}%
  \ifResizeGather@enable
    \ifdim\totwidth@>\displaywidth
      \totwidth@=\displaywidth
    \fi
  \fi
}
%    \end{macrocode}
%    \end{macro}
%
%    \begin{macrocode}
\ResizeGather@AtEnd%
%</package>
%    \end{macrocode}
%
% \section{Test}
%
% \subsection{Catcode checks for loading}
%
%    \begin{macrocode}
%<*test1>
%    \end{macrocode}
%    \begin{macrocode}
\catcode`\{=1 %
\catcode`\}=2 %
\catcode`\#=6 %
\catcode`\@=11 %
\expandafter\ifx\csname count@\endcsname\relax
  \countdef\count@=255 %
\fi
\expandafter\ifx\csname @gobble\endcsname\relax
  \long\def\@gobble#1{}%
\fi
\expandafter\ifx\csname @firstofone\endcsname\relax
  \long\def\@firstofone#1{#1}%
\fi
\expandafter\ifx\csname loop\endcsname\relax
  \expandafter\@firstofone
\else
  \expandafter\@gobble
\fi
{%
  \def\loop#1\repeat{%
    \def\body{#1}%
    \iterate
  }%
  \def\iterate{%
    \body
      \let\next\iterate
    \else
      \let\next\relax
    \fi
    \next
  }%
  \let\repeat=\fi
}%
\def\RestoreCatcodes{}
\count@=0 %
\loop
  \edef\RestoreCatcodes{%
    \RestoreCatcodes
    \catcode\the\count@=\the\catcode\count@\relax
  }%
\ifnum\count@<255 %
  \advance\count@ 1 %
\repeat

\def\RangeCatcodeInvalid#1#2{%
  \count@=#1\relax
  \loop
    \catcode\count@=15 %
  \ifnum\count@<#2\relax
    \advance\count@ 1 %
  \repeat
}
\def\RangeCatcodeCheck#1#2#3{%
  \count@=#1\relax
  \loop
    \ifnum#3=\catcode\count@
    \else
      \errmessage{%
        Character \the\count@\space
        with wrong catcode \the\catcode\count@\space
        instead of \number#3%
      }%
    \fi
  \ifnum\count@<#2\relax
    \advance\count@ 1 %
  \repeat
}
\def\space{ }
\expandafter\ifx\csname LoadCommand\endcsname\relax
  \def\LoadCommand{\input resizegather.sty\relax}%
\fi
\def\Test{%
  \RangeCatcodeInvalid{0}{47}%
  \RangeCatcodeInvalid{58}{64}%
  \RangeCatcodeInvalid{91}{96}%
  \RangeCatcodeInvalid{123}{255}%
  \catcode`\@=12 %
  \catcode`\\=0 %
  \catcode`\%=14 %
  \LoadCommand
  \RangeCatcodeCheck{0}{36}{15}%
  \RangeCatcodeCheck{37}{37}{14}%
  \RangeCatcodeCheck{38}{47}{15}%
  \RangeCatcodeCheck{48}{57}{12}%
  \RangeCatcodeCheck{58}{63}{15}%
  \RangeCatcodeCheck{64}{64}{12}%
  \RangeCatcodeCheck{65}{90}{11}%
  \RangeCatcodeCheck{91}{91}{15}%
  \RangeCatcodeCheck{92}{92}{0}%
  \RangeCatcodeCheck{93}{96}{15}%
  \RangeCatcodeCheck{97}{122}{11}%
  \RangeCatcodeCheck{123}{255}{15}%
  \RestoreCatcodes
}
\Test
\csname @@end\endcsname
\end
%    \end{macrocode}
%    \begin{macrocode}
%</test1>
%    \end{macrocode}
%
% \section{Installation}
%
% \subsection{Download}
%
% \paragraph{Package.} This package is available on
% CTAN\footnote{\url{https://ctan.org/pkg/resizegather}}:
% \begin{description}
% \item[\CTAN{macros/latex/contrib/oberdiek/resizegather.dtx}] The source file.
% \item[\CTAN{macros/latex/contrib/oberdiek/resizegather.pdf}] Documentation.
% \end{description}
%
%
% \paragraph{Bundle.} All the packages of the bundle `oberdiek'
% are also available in a TDS compliant ZIP archive. There
% the packages are already unpacked and the documentation files
% are generated. The files and directories obey the TDS standard.
% \begin{description}
% \item[\CTANinstall{install/macros/latex/contrib/oberdiek.tds.zip}]
% \end{description}
% \emph{TDS} refers to the standard ``A Directory Structure
% for \TeX\ Files'' (\CTAN{tds/tds.pdf}). Directories
% with \xfile{texmf} in their name are usually organized this way.
%
% \subsection{Bundle installation}
%
% \paragraph{Unpacking.} Unpack the \xfile{oberdiek.tds.zip} in the
% TDS tree (also known as \xfile{texmf} tree) of your choice.
% Example (linux):
% \begin{quote}
%   |unzip oberdiek.tds.zip -d ~/texmf|
% \end{quote}
%
% \paragraph{Script installation.}
% Check the directory \xfile{TDS:scripts/oberdiek/} for
% scripts that need further installation steps.
% Package \xpackage{attachfile2} comes with the Perl script
% \xfile{pdfatfi.pl} that should be installed in such a way
% that it can be called as \texttt{pdfatfi}.
% Example (linux):
% \begin{quote}
%   |chmod +x scripts/oberdiek/pdfatfi.pl|\\
%   |cp scripts/oberdiek/pdfatfi.pl /usr/local/bin/|
% \end{quote}
%
% \subsection{Package installation}
%
% \paragraph{Unpacking.} The \xfile{.dtx} file is a self-extracting
% \docstrip\ archive. The files are extracted by running the
% \xfile{.dtx} through \plainTeX:
% \begin{quote}
%   \verb|tex resizegather.dtx|
% \end{quote}
%
% \paragraph{TDS.} Now the different files must be moved into
% the different directories in your installation TDS tree
% (also known as \xfile{texmf} tree):
% \begin{quote}
% \def\t{^^A
% \begin{tabular}{@{}>{\ttfamily}l@{ $\rightarrow$ }>{\ttfamily}l@{}}
%   resizegather.sty & tex/latex/oberdiek/resizegather.sty\\
%   resizegather.pdf & doc/latex/oberdiek/resizegather.pdf\\
%   test/resizegather-test1.tex & doc/latex/oberdiek/test/resizegather-test1.tex\\
%   resizegather.dtx & source/latex/oberdiek/resizegather.dtx\\
% \end{tabular}^^A
% }^^A
% \sbox0{\t}^^A
% \ifdim\wd0>\linewidth
%   \begingroup
%     \advance\linewidth by\leftmargin
%     \advance\linewidth by\rightmargin
%   \edef\x{\endgroup
%     \def\noexpand\lw{\the\linewidth}^^A
%   }\x
%   \def\lwbox{^^A
%     \leavevmode
%     \hbox to \linewidth{^^A
%       \kern-\leftmargin\relax
%       \hss
%       \usebox0
%       \hss
%       \kern-\rightmargin\relax
%     }^^A
%   }^^A
%   \ifdim\wd0>\lw
%     \sbox0{\small\t}^^A
%     \ifdim\wd0>\linewidth
%       \ifdim\wd0>\lw
%         \sbox0{\footnotesize\t}^^A
%         \ifdim\wd0>\linewidth
%           \ifdim\wd0>\lw
%             \sbox0{\scriptsize\t}^^A
%             \ifdim\wd0>\linewidth
%               \ifdim\wd0>\lw
%                 \sbox0{\tiny\t}^^A
%                 \ifdim\wd0>\linewidth
%                   \lwbox
%                 \else
%                   \usebox0
%                 \fi
%               \else
%                 \lwbox
%               \fi
%             \else
%               \usebox0
%             \fi
%           \else
%             \lwbox
%           \fi
%         \else
%           \usebox0
%         \fi
%       \else
%         \lwbox
%       \fi
%     \else
%       \usebox0
%     \fi
%   \else
%     \lwbox
%   \fi
% \else
%   \usebox0
% \fi
% \end{quote}
% If you have a \xfile{docstrip.cfg} that configures and enables \docstrip's
% TDS installing feature, then some files can already be in the right
% place, see the documentation of \docstrip.
%
% \subsection{Refresh file name databases}
%
% If your \TeX~distribution
% (\teTeX, \mikTeX, \dots) relies on file name databases, you must refresh
% these. For example, \teTeX\ users run \verb|texhash| or
% \verb|mktexlsr|.
%
% \subsection{Some details for the interested}
%
% \paragraph{Attached source.}
%
% The PDF documentation on CTAN also includes the
% \xfile{.dtx} source file. It can be extracted by
% AcrobatReader 6 or higher. Another option is \textsf{pdftk},
% e.g. unpack the file into the current directory:
% \begin{quote}
%   \verb|pdftk resizegather.pdf unpack_files output .|
% \end{quote}
%
% \paragraph{Unpacking with \LaTeX.}
% The \xfile{.dtx} chooses its action depending on the format:
% \begin{description}
% \item[\plainTeX:] Run \docstrip\ and extract the files.
% \item[\LaTeX:] Generate the documentation.
% \end{description}
% If you insist on using \LaTeX\ for \docstrip\ (really,
% \docstrip\ does not need \LaTeX), then inform the autodetect routine
% about your intention:
% \begin{quote}
%   \verb|latex \let\install=y\input{resizegather.dtx}|
% \end{quote}
% Do not forget to quote the argument according to the demands
% of your shell.
%
% \paragraph{Generating the documentation.}
% You can use both the \xfile{.dtx} or the \xfile{.drv} to generate
% the documentation. The process can be configured by the
% configuration file \xfile{ltxdoc.cfg}. For instance, put this
% line into this file, if you want to have A4 as paper format:
% \begin{quote}
%   \verb|\PassOptionsToClass{a4paper}{article}|
% \end{quote}
% An example follows how to generate the
% documentation with pdf\LaTeX:
% \begin{quote}
%\begin{verbatim}
%pdflatex resizegather.dtx
%makeindex -s gind.ist resizegather.idx
%pdflatex resizegather.dtx
%makeindex -s gind.ist resizegather.idx
%pdflatex resizegather.dtx
%\end{verbatim}
% \end{quote}
%
% \section{Acknowledgement}
%
% \begin{description}
% \item[Dieter Jurzitza:]
% He wanted the resizing feature at the \TeX\ table
% in Karlsruhe of December 2009. Thus this package is a kind of
% Christmas present.
% \end{description}
%
% \begin{History}
%   \begin{Version}{2009/12/04 v1.0}
%   \item
%     The first version.
%   \end{Version}
%   \begin{Version}{2009/12/05 v1.1}
%   \item
%     Options \xoption{enable} and \xoption{disable} added.
%   \end{Version}
%   \begin{Version}{2010/03/01 v1.2}
%   \item
%     TDS location moved from `generic' to `latex'.
%   \end{Version}
%   \begin{Version}{2016/05/16 v1.3}
%   \item
%     Documentation updates.
%   \end{Version}
% \end{History}
%
% \PrintIndex
%
% \Finale
\endinput
|
% \end{quote}
% Do not forget to quote the argument according to the demands
% of your shell.
%
% \paragraph{Generating the documentation.}
% You can use both the \xfile{.dtx} or the \xfile{.drv} to generate
% the documentation. The process can be configured by the
% configuration file \xfile{ltxdoc.cfg}. For instance, put this
% line into this file, if you want to have A4 as paper format:
% \begin{quote}
%   \verb|\PassOptionsToClass{a4paper}{article}|
% \end{quote}
% An example follows how to generate the
% documentation with pdf\LaTeX:
% \begin{quote}
%\begin{verbatim}
%pdflatex resizegather.dtx
%makeindex -s gind.ist resizegather.idx
%pdflatex resizegather.dtx
%makeindex -s gind.ist resizegather.idx
%pdflatex resizegather.dtx
%\end{verbatim}
% \end{quote}
%
% \section{Acknowledgement}
%
% \begin{description}
% \item[Dieter Jurzitza:]
% He wanted the resizing feature at the \TeX\ table
% in Karlsruhe of December 2009. Thus this package is a kind of
% Christmas present.
% \end{description}
%
% \begin{History}
%   \begin{Version}{2009/12/04 v1.0}
%   \item
%     The first version.
%   \end{Version}
%   \begin{Version}{2009/12/05 v1.1}
%   \item
%     Options \xoption{enable} and \xoption{disable} added.
%   \end{Version}
%   \begin{Version}{2010/03/01 v1.2}
%   \item
%     TDS location moved from `generic' to `latex'.
%   \end{Version}
%   \begin{Version}{2016/05/16 v1.3}
%   \item
%     Documentation updates.
%   \end{Version}
% \end{History}
%
% \PrintIndex
%
% \Finale
\endinput
|
% \end{quote}
% Do not forget to quote the argument according to the demands
% of your shell.
%
% \paragraph{Generating the documentation.}
% You can use both the \xfile{.dtx} or the \xfile{.drv} to generate
% the documentation. The process can be configured by the
% configuration file \xfile{ltxdoc.cfg}. For instance, put this
% line into this file, if you want to have A4 as paper format:
% \begin{quote}
%   \verb|\PassOptionsToClass{a4paper}{article}|
% \end{quote}
% An example follows how to generate the
% documentation with pdf\LaTeX:
% \begin{quote}
%\begin{verbatim}
%pdflatex resizegather.dtx
%makeindex -s gind.ist resizegather.idx
%pdflatex resizegather.dtx
%makeindex -s gind.ist resizegather.idx
%pdflatex resizegather.dtx
%\end{verbatim}
% \end{quote}
%
% \section{Acknowledgement}
%
% \begin{description}
% \item[Dieter Jurzitza:]
% He wanted the resizing feature at the \TeX\ table
% in Karlsruhe of December 2009. Thus this package is a kind of
% Christmas present.
% \end{description}
%
% \begin{History}
%   \begin{Version}{2009/12/04 v1.0}
%   \item
%     The first version.
%   \end{Version}
%   \begin{Version}{2009/12/05 v1.1}
%   \item
%     Options \xoption{enable} and \xoption{disable} added.
%   \end{Version}
%   \begin{Version}{2010/03/01 v1.2}
%   \item
%     TDS location moved from `generic' to `latex'.
%   \end{Version}
%   \begin{Version}{2016/05/16 v1.3}
%   \item
%     Documentation updates.
%   \end{Version}
% \end{History}
%
% \PrintIndex
%
% \Finale
\endinput
|
% \end{quote}
% Do not forget to quote the argument according to the demands
% of your shell.
%
% \paragraph{Generating the documentation.}
% You can use both the \xfile{.dtx} or the \xfile{.drv} to generate
% the documentation. The process can be configured by the
% configuration file \xfile{ltxdoc.cfg}. For instance, put this
% line into this file, if you want to have A4 as paper format:
% \begin{quote}
%   \verb|\PassOptionsToClass{a4paper}{article}|
% \end{quote}
% An example follows how to generate the
% documentation with pdf\LaTeX:
% \begin{quote}
%\begin{verbatim}
%pdflatex resizegather.dtx
%makeindex -s gind.ist resizegather.idx
%pdflatex resizegather.dtx
%makeindex -s gind.ist resizegather.idx
%pdflatex resizegather.dtx
%\end{verbatim}
% \end{quote}
%
% \section{Acknowledgement}
%
% \begin{description}
% \item[Dieter Jurzitza:]
% He wanted the resizing feature at the \TeX\ table
% in Karlsruhe of December 2009. Thus this package is a kind of
% Christmas present.
% \end{description}
%
% \begin{History}
%   \begin{Version}{2009/12/04 v1.0}
%   \item
%     The first version.
%   \end{Version}
%   \begin{Version}{2009/12/05 v1.1}
%   \item
%     Options \xoption{enable} and \xoption{disable} added.
%   \end{Version}
%   \begin{Version}{2010/03/01 v1.2}
%   \item
%     TDS location moved from `generic' to `latex'.
%   \end{Version}
%   \begin{Version}{2016/05/16 v1.3}
%   \item
%     Documentation updates.
%   \end{Version}
% \end{History}
%
% \PrintIndex
%
% \Finale
\endinput
