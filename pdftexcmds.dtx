% \iffalse meta-comment
%
% File: pdftexcmds.dtx
% Version: 2019/07/25 v0.30
% Info: Utility functions of pdfTeX for LuaTeX
%
% Copyright (C) 2007, 2009-2011 by
%    Heiko Oberdiek <heiko.oberdiek at googlemail.com>
%
% This work may be distributed and/or modified under the
% conditions of the LaTeX Project Public License, either
% version 1.3c of this license or (at your option) any later
% version. This version of this license is in
%    https://www.latex-project.org/lppl/lppl-1-3c.txt
% and the latest version of this license is in
%    https://www.latex-project.org/lppl.txt
% and version 1.3 or later is part of all distributions of
% LaTeX version 2005/12/01 or later.
%
% This work has the LPPL maintenance status "maintained".
%
% The Current Maintainers of this work are
% Heiko Oberdiek and the Oberdiek Package Support Group
% https://github.com/ho-tex/oberdiek/issues
%
% The Base Interpreter refers to any `TeX-Format',
% because some files are installed in TDS:tex/generic//.
%
% This work consists of the main source file pdftexcmds.dtx
% and the derived files
%    pdftexcmds.sty, pdftexcmds.pdf, pdftexcmds.ins, pdftexcmds.drv,
%    pdftexcmds.bib, pdftexcmds-test1.tex, pdftexcmds-test2.tex,
%    pdftexcmds-test-shell.tex, pdftexcmds-test-escape.tex,
%    oberdiek.pdftexcmds.lua, pdftexcmds.lua.
%
% Distribution:
%    CTAN:macros/latex/contrib/oberdiek/pdftexcmds.dtx
%    CTAN:macros/latex/contrib/oberdiek/pdftexcmds.pdf
%
% Unpacking:
%    (a) If pdftexcmds.ins is present:
%           tex pdftexcmds.ins
%    (b) Without pdftexcmds.ins:
%           tex pdftexcmds.dtx
%    (c) If you insist on using LaTeX
%           latex \let\install=y% \iffalse meta-comment
%
% File: pdftexcmds.dtx
% Version: 2019/07/25 v0.30
% Info: Utility functions of pdfTeX for LuaTeX
%
% Copyright (C) 2007, 2009-2011 by
%    Heiko Oberdiek <heiko.oberdiek at googlemail.com>
%
% This work may be distributed and/or modified under the
% conditions of the LaTeX Project Public License, either
% version 1.3c of this license or (at your option) any later
% version. This version of this license is in
%    https://www.latex-project.org/lppl/lppl-1-3c.txt
% and the latest version of this license is in
%    https://www.latex-project.org/lppl.txt
% and version 1.3 or later is part of all distributions of
% LaTeX version 2005/12/01 or later.
%
% This work has the LPPL maintenance status "maintained".
%
% The Current Maintainers of this work are
% Heiko Oberdiek and the Oberdiek Package Support Group
% https://github.com/ho-tex/oberdiek/issues
%
% The Base Interpreter refers to any `TeX-Format',
% because some files are installed in TDS:tex/generic//.
%
% This work consists of the main source file pdftexcmds.dtx
% and the derived files
%    pdftexcmds.sty, pdftexcmds.pdf, pdftexcmds.ins, pdftexcmds.drv,
%    pdftexcmds.bib, pdftexcmds-test1.tex, pdftexcmds-test2.tex,
%    pdftexcmds-test-shell.tex, pdftexcmds-test-escape.tex,
%    oberdiek.pdftexcmds.lua, pdftexcmds.lua.
%
% Distribution:
%    CTAN:macros/latex/contrib/oberdiek/pdftexcmds.dtx
%    CTAN:macros/latex/contrib/oberdiek/pdftexcmds.pdf
%
% Unpacking:
%    (a) If pdftexcmds.ins is present:
%           tex pdftexcmds.ins
%    (b) Without pdftexcmds.ins:
%           tex pdftexcmds.dtx
%    (c) If you insist on using LaTeX
%           latex \let\install=y% \iffalse meta-comment
%
% File: pdftexcmds.dtx
% Version: 2019/07/25 v0.30
% Info: Utility functions of pdfTeX for LuaTeX
%
% Copyright (C) 2007, 2009-2011 by
%    Heiko Oberdiek <heiko.oberdiek at googlemail.com>
%
% This work may be distributed and/or modified under the
% conditions of the LaTeX Project Public License, either
% version 1.3c of this license or (at your option) any later
% version. This version of this license is in
%    https://www.latex-project.org/lppl/lppl-1-3c.txt
% and the latest version of this license is in
%    https://www.latex-project.org/lppl.txt
% and version 1.3 or later is part of all distributions of
% LaTeX version 2005/12/01 or later.
%
% This work has the LPPL maintenance status "maintained".
%
% The Current Maintainers of this work are
% Heiko Oberdiek and the Oberdiek Package Support Group
% https://github.com/ho-tex/oberdiek/issues
%
% The Base Interpreter refers to any `TeX-Format',
% because some files are installed in TDS:tex/generic//.
%
% This work consists of the main source file pdftexcmds.dtx
% and the derived files
%    pdftexcmds.sty, pdftexcmds.pdf, pdftexcmds.ins, pdftexcmds.drv,
%    pdftexcmds.bib, pdftexcmds-test1.tex, pdftexcmds-test2.tex,
%    pdftexcmds-test-shell.tex, pdftexcmds-test-escape.tex,
%    oberdiek.pdftexcmds.lua, pdftexcmds.lua.
%
% Distribution:
%    CTAN:macros/latex/contrib/oberdiek/pdftexcmds.dtx
%    CTAN:macros/latex/contrib/oberdiek/pdftexcmds.pdf
%
% Unpacking:
%    (a) If pdftexcmds.ins is present:
%           tex pdftexcmds.ins
%    (b) Without pdftexcmds.ins:
%           tex pdftexcmds.dtx
%    (c) If you insist on using LaTeX
%           latex \let\install=y% \iffalse meta-comment
%
% File: pdftexcmds.dtx
% Version: 2019/07/25 v0.30
% Info: Utility functions of pdfTeX for LuaTeX
%
% Copyright (C) 2007, 2009-2011 by
%    Heiko Oberdiek <heiko.oberdiek at googlemail.com>
%
% This work may be distributed and/or modified under the
% conditions of the LaTeX Project Public License, either
% version 1.3c of this license or (at your option) any later
% version. This version of this license is in
%    https://www.latex-project.org/lppl/lppl-1-3c.txt
% and the latest version of this license is in
%    https://www.latex-project.org/lppl.txt
% and version 1.3 or later is part of all distributions of
% LaTeX version 2005/12/01 or later.
%
% This work has the LPPL maintenance status "maintained".
%
% The Current Maintainers of this work are
% Heiko Oberdiek and the Oberdiek Package Support Group
% https://github.com/ho-tex/oberdiek/issues
%
% The Base Interpreter refers to any `TeX-Format',
% because some files are installed in TDS:tex/generic//.
%
% This work consists of the main source file pdftexcmds.dtx
% and the derived files
%    pdftexcmds.sty, pdftexcmds.pdf, pdftexcmds.ins, pdftexcmds.drv,
%    pdftexcmds.bib, pdftexcmds-test1.tex, pdftexcmds-test2.tex,
%    pdftexcmds-test-shell.tex, pdftexcmds-test-escape.tex,
%    oberdiek.pdftexcmds.lua, pdftexcmds.lua.
%
% Distribution:
%    CTAN:macros/latex/contrib/oberdiek/pdftexcmds.dtx
%    CTAN:macros/latex/contrib/oberdiek/pdftexcmds.pdf
%
% Unpacking:
%    (a) If pdftexcmds.ins is present:
%           tex pdftexcmds.ins
%    (b) Without pdftexcmds.ins:
%           tex pdftexcmds.dtx
%    (c) If you insist on using LaTeX
%           latex \let\install=y\input{pdftexcmds.dtx}
%        (quote the arguments according to the demands of your shell)
%
% Documentation:
%    (a) If pdftexcmds.drv is present:
%           latex pdftexcmds.drv
%    (b) Without pdftexcmds.drv:
%           latex pdftexcmds.dtx; ...
%    The class ltxdoc loads the configuration file ltxdoc.cfg
%    if available. Here you can specify further options, e.g.
%    use A4 as paper format:
%       \PassOptionsToClass{a4paper}{article}
%
%    Programm calls to get the documentation (example):
%       pdflatex pdftexcmds.dtx
%       bibtex pdftexcmds.aux
%       makeindex -s gind.ist pdftexcmds.idx
%       pdflatex pdftexcmds.dtx
%       makeindex -s gind.ist pdftexcmds.idx
%       pdflatex pdftexcmds.dtx
%
% Installation:
%    TDS:tex/generic/oberdiek/pdftexcmds.sty
%    TDS:scripts/oberdiek/oberdiek.pdftexcmds.lua
%    TDS:scripts/oberdiek/pdftexcmds.lua
%    TDS:doc/latex/oberdiek/pdftexcmds.pdf
%    TDS:doc/latex/oberdiek/test/pdftexcmds-test1.tex
%    TDS:doc/latex/oberdiek/test/pdftexcmds-test2.tex
%    TDS:doc/latex/oberdiek/test/pdftexcmds-test-shell.tex
%    TDS:doc/latex/oberdiek/test/pdftexcmds-test-escape.tex
%    TDS:source/latex/oberdiek/pdftexcmds.dtx
%
%<*ignore>
\begingroup
  \catcode123=1 %
  \catcode125=2 %
  \def\x{LaTeX2e}%
\expandafter\endgroup
\ifcase 0\ifx\install y1\fi\expandafter
         \ifx\csname processbatchFile\endcsname\relax\else1\fi
         \ifx\fmtname\x\else 1\fi\relax
\else\csname fi\endcsname
%</ignore>
%<*install>
\input docstrip.tex
\Msg{************************************************************************}
\Msg{* Installation}
\Msg{* Package: pdftexcmds 2019/07/25 v0.30 Utility functions of pdfTeX for LuaTeX (HO)}
\Msg{************************************************************************}

\keepsilent
\askforoverwritefalse

\let\MetaPrefix\relax
\preamble

This is a generated file.

Project: pdftexcmds
Version: 2019/07/25 v0.30

Copyright (C) 2007, 2009-2011 by
   Heiko Oberdiek <heiko.oberdiek at googlemail.com>

This work may be distributed and/or modified under the
conditions of the LaTeX Project Public License, either
version 1.3c of this license or (at your option) any later
version. This version of this license is in
   https://www.latex-project.org/lppl/lppl-1-3c.txt
and the latest version of this license is in
   https://www.latex-project.org/lppl.txt
and version 1.3 or later is part of all distributions of
LaTeX version 2005/12/01 or later.

This work has the LPPL maintenance status "maintained".

The Current Maintainers of this work are
Heiko Oberdiek and the Oberdiek Package Support Group
https://github.com/ho-tex/oberdiek/issues


The Base Interpreter refers to any `TeX-Format',
because some files are installed in TDS:tex/generic//.

This work consists of the main source file pdftexcmds.dtx
and the derived files
   pdftexcmds.sty, pdftexcmds.pdf, pdftexcmds.ins, pdftexcmds.drv,
   pdftexcmds.bib, pdftexcmds-test1.tex, pdftexcmds-test2.tex,
   pdftexcmds-test-shell.tex, pdftexcmds-test-escape.tex,
   oberdiek.pdftexcmds.lua, pdftexcmds.lua.

\endpreamble
\let\MetaPrefix\DoubleperCent

\generate{%
  \file{pdftexcmds.ins}{\from{pdftexcmds.dtx}{install}}%
  \file{pdftexcmds.drv}{\from{pdftexcmds.dtx}{driver}}%
  \nopreamble
  \nopostamble
  \file{pdftexcmds.bib}{\from{pdftexcmds.dtx}{bib}}%
  \usepreamble\defaultpreamble
  \usepostamble\defaultpostamble
  \usedir{tex/generic/oberdiek}%
  \file{pdftexcmds.sty}{\from{pdftexcmds.dtx}{package}}%
%  \usedir{doc/latex/oberdiek/test}%
%  \file{pdftexcmds-test1.tex}{\from{pdftexcmds.dtx}{test1}}%
%  \file{pdftexcmds-test2.tex}{\from{pdftexcmds.dtx}{test2}}%
%  \file{pdftexcmds-test-shell.tex}{\from{pdftexcmds.dtx}{test-shell}}%
%  \file{pdftexcmds-test-escape.tex}{\from{pdftexcmds.dtx}{test-escape}}%
  \nopreamble
  \nopostamble
%  \usedir{source/latex/oberdiek/catalogue}%
%  \file{pdftexcmds.xml}{\from{pdftexcmds.dtx}{catalogue}}%
}
\def\MetaPrefix{-- }
\def\defaultpostamble{%
  \MetaPrefix^^J%
  \MetaPrefix\space End of File `\outFileName'.%
}
\def\currentpostamble{\defaultpostamble}%
\generate{%
  \usedir{scripts/oberdiek}%
  \file{oberdiek.pdftexcmds.lua}{\from{pdftexcmds.dtx}{lua}}%
  \file{pdftexcmds.lua}{\from{pdftexcmds.dtx}{lua}}%
}

\catcode32=13\relax% active space
\let =\space%
\Msg{************************************************************************}
\Msg{*}
\Msg{* To finish the installation you have to move the following}
\Msg{* file into a directory searched by TeX:}
\Msg{*}
\Msg{*     pdftexcmds.sty}
\Msg{*}
\Msg{* And install the following script files:}
\Msg{*}
\Msg{*     oberdiek.pdftexcmds.lua, pdftexcmds.lua}
\Msg{*}
\Msg{* To produce the documentation run the file `pdftexcmds.drv'}
\Msg{* through LaTeX.}
\Msg{*}
\Msg{* Happy TeXing!}
\Msg{*}
\Msg{************************************************************************}

\endbatchfile
%</install>
%<*bib>
@online{AndyThomas:Analog,
  author={Thomas, Andy},
  title={Analog of {\texttt{\csname textbackslash\endcsname}pdfelapsedtime} for
      {\hologo{LuaTeX}} and {\hologo{XeTeX}}},
  url={http://tex.stackexchange.com/a/32531},
  urldate={2011-11-29},
}
%</bib>
%<*ignore>
\fi
%</ignore>
%<*driver>
\NeedsTeXFormat{LaTeX2e}
\ProvidesFile{pdftexcmds.drv}%
  [2019/07/25 v0.30 Utility functions of pdfTeX for LuaTeX (HO)]%
\documentclass{ltxdoc}
\usepackage{holtxdoc}[2011/11/22]
\usepackage{paralist}
\usepackage{csquotes}
\usepackage[
  backend=bibtex,
  bibencoding=ascii,
  alldates=iso8601,
]{biblatex}[2011/11/13]
\bibliography{oberdiek-source}
\bibliography{pdftexcmds}
\begin{document}
  \DocInput{pdftexcmds.dtx}%
\end{document}
%</driver>
% \fi
%
%
% \CharacterTable
%  {Upper-case    \A\B\C\D\E\F\G\H\I\J\K\L\M\N\O\P\Q\R\S\T\U\V\W\X\Y\Z
%   Lower-case    \a\b\c\d\e\f\g\h\i\j\k\l\m\n\o\p\q\r\s\t\u\v\w\x\y\z
%   Digits        \0\1\2\3\4\5\6\7\8\9
%   Exclamation   \!     Double quote  \"     Hash (number) \#
%   Dollar        \$     Percent       \%     Ampersand     \&
%   Acute accent  \'     Left paren    \(     Right paren   \)
%   Asterisk      \*     Plus          \+     Comma         \,
%   Minus         \-     Point         \.     Solidus       \/
%   Colon         \:     Semicolon     \;     Less than     \<
%   Equals        \=     Greater than  \>     Question mark \?
%   Commercial at \@     Left bracket  \[     Backslash     \\
%   Right bracket \]     Circumflex    \^     Underscore    \_
%   Grave accent  \`     Left brace    \{     Vertical bar  \|
%   Right brace   \}     Tilde         \~}
%
% \GetFileInfo{pdftexcmds.drv}
%
% \title{The \xpackage{pdftexcmds} package}
% \date{2019/07/25 v0.30}
% \author{Heiko Oberdiek\thanks
% {Please report any issues at \url{https://github.com/ho-tex/oberdiek/issues}}}
%
% \maketitle
%
% \begin{abstract}
% \hologo{LuaTeX} provides most of the commands of \hologo{pdfTeX} 1.40. However
% a number of utility functions are removed. This package tries to fill
% the gap and implements some of the missing primitive using Lua.
% \end{abstract}
%
% \tableofcontents
%
% \def\csi#1{\texttt{\textbackslash\textit{#1}}}
%
% \section{Documentation}
%
% Some primitives of \hologo{pdfTeX} \cite{pdftex-manual}
% are not defined by \hologo{LuaTeX} \cite{luatex-manual}.
% This package implements macro based solutions using Lua code
% for the following missing \hologo{pdfTeX} primitives;
% \begin{compactitem}
% \item \cs{pdfstrcmp}
% \item \cs{pdfunescapehex}
% \item \cs{pdfescapehex}
% \item \cs{pdfescapename}
% \item \cs{pdfescapestring}
% \item \cs{pdffilesize}
% \item \cs{pdffilemoddate}
% \item \cs{pdffiledump}
% \item \cs{pdfmdfivesum}
% \item \cs{pdfresettimer}
% \item \cs{pdfelapsedtime}
% \item |\immediate\write18|
% \end{compactitem}
% The original names of the primitives cannot be used:
% \begin{itemize}
% \item
% The syntax for their arguments cannot easily
% simulated by macros. The primitives using key words
% such as |file| (\cs{pdfmdfivesum}) or |offset| and |length|
% (\cs{pdffiledump}) and uses \meta{general text} for the other
% arguments. Using token registers assignments, \meta{general text} could
% be catched. However, the simulated primitives are expandable
% and register assignments would destroy this important property.
% (\meta{general text} allows something like |\expandafter\bgroup ...}|.)
% \item
% The original primitives can be expanded using one expansion step.
% The new macros need two expansion steps because of the additional
% macro expansion. Example:
% \begin{quote}
%   |\expandafter\foo\pdffilemoddate{file}|\\
%   vs.\\
%   |\expandafter\expandafter\expandafter|\\
%   |\foo\pdf@filemoddate{file}|
% \end{quote}
% \end{itemize}
%
% \hologo{LuaTeX} isn't stable yet and thus the status of this package is
% \emph{experimental}. Feedback is welcome.
%
% \subsection{General principles}
%
% \begin{description}
% \item[Naming convention:]
%   Usually this package defines a macro |\pdf@|\meta{cmd} if
%   \hologo{pdfTeX} provides |\pdf|\meta{cmd}.
% \item[Arguments:] The order of arguments in |\pdf@|\meta{cmd}
%   is the same as for the corresponding primitive of \hologo{pdfTeX}.
%   The arguments are ordinary undelimited \hologo{TeX} arguments,
%   no \meta{general text} and without additional keywords.
% \item[Expandibility:]
%   The macro |\pdf@|\meta{cmd} is expandable if the
%   corresponding \hologo{pdfTeX} primitive has this property.
%   Exact two expansion steps are necessary (first is the macro
%   expansion) except for \cs{pdf@primitive} and \cs{pdf@ifprimitive}.
%   The latter ones are not macros, but have the direct meaning of the
%   primitive.
% \item[Without \hologo{LuaTeX}:]
%   The macros |\pdf@|\meta{cmd} are mapped to the commands
%   of \hologo{pdfTeX} if they are available. Otherwise they are undefined.
% \item[Availability:]
%   The macros that the packages provides are undefined, if
%   the necessary primitives are not found and cannot be
%   implemented by Lua.
% \end{description}
%
% \subsection{Macros}
%
% \subsubsection[Strings]{Strings \cite[``7.15 Strings'']{pdftex-manual}}
%
% \begin{declcs}{pdf@strcmp} \M{stringA} \M{stringB}
% \end{declcs}
% Same as |\pdfstrcmp{|\meta{stringA}|}{|\meta{stringB}|}|.
%
% \begin{declcs}{pdf@unescapehex} \M{string}
% \end{declcs}
% Same as |\pdfunescapehex{|\meta{string}|}|.
% The argument is a byte string given in hexadecimal notation.
% The result are character tokens from 0 until 255 with
% catcode 12 and the space with catcode 10.
%
% \begin{declcs}{pdf@escapehex} \M{string}\\
%   \cs{pdf@escapestring} \M{string}\\
%   \cs{pdf@escapename} \M{string}
% \end{declcs}
% Same as the primitives of \hologo{pdfTeX}. However \hologo{pdfTeX} does not
% know about characters with codes 256 and larger. Thus the
% string is treated as byte string, characters with more than
% eight bits are ignored.
%
% \subsubsection[Files]{Files \cite[``7.18 Files'']{pdftex-manual}}
%
% \begin{declcs}{pdf@filesize} \M{filename}
% \end{declcs}
% Same as |\pdffilesize{|\meta{filename}|}|.
%
% \begin{declcs}{pdf@filemoddate} \M{filename}
% \end{declcs}
% Same as |\pdffilemoddate{|\meta{filename}|}|.
%
% \begin{declcs}{pdf@filedump} \M{offset} \M{length} \M{filename}
% \end{declcs}
% Same as |\pdffiledump offset| \meta{offset} |length| \meta{length}
% |{|\meta{filename}|}|. Both \meta{offset} and \meta{length} must
% not be empty, but must be a valid \hologo{TeX} number.
%
% \begin{declcs}{pdf@mdfivesum} \M{string}
% \end{declcs}
% Same as |\pdfmdfivesum{|\meta{string}|}|. Keyword |file| is supported
% by macro \cs{pdf@filemdfivesum}.
%
% \begin{declcs}{pdf@filemdfivesum} \M{filename}
% \end{declcs}
% Same as |\pdfmdfivesum file{|\meta{filename}|}|.
%
% \subsubsection[Timekeeping]{Timekeeping \cite[``7.17 Timekeeping'']{pdftex-manual}}
%
% The timekeeping macros are based on Andy Thomas' work \cite{AndyThomas:Analog}.
%
% \begin{declcs}{pdf@resettimer}
% \end{declcs}
% Same as \cs{pdfresettimer}, it resets the internal timer.
%
% \begin{declcs}{pdf@elapsedtime}
% \end{declcs}
% Same as \cs{pdfelapsedtime}. It behaves like a read-only integer.
% For printing purposes it can be prefixed by \cs{the} or \cs{number}.
% It measures the time in scaled seconds (seconds multiplied with 65536)
% since the latest call of \cs{pdf@resettimer} or start of
% program/package. The resolution, the shortest time interval that
% can be measured, depends on the program and system.
% \begin{itemize}
% \item \hologo{pdfTeX} with |gettimeofday|: $\ge$ 1/65536\,s
% \item \hologo{pdfTeX} with |ftime|: $\ge$ 1\,ms
% \item \hologo{pdfTeX} with |time|: $\ge$ 1\,s
% \item \hologo{LuaTeX}: $\ge$ 10\,ms\\
%  (|os.clock()| returns a float number with two decimal digits in
%  \hologo{LuaTeX} beta-0.70.1-2011061416 (rev 4277)).
% \end{itemize}
%
% \subsubsection[Miscellaneous]{Miscellaneous \cite[``7.21 Miscellaneous'']{pdftex-manual}}
%
% \begin{declcs}{pdf@draftmode}
% \end{declcs}
% If the \TeX\ compiler knows \cs{pdfdraftmode} or \cs{draftmode}
% (\hologo{pdfTeX},
% \hologo{LuaTeX}), then \cs{pdf@draftmode} returns, whether
% this mode is enabled. The result is an implicit number:
% one means the draft mode is available and enabled.
% If the value is zero, then the mode is not active or
% \cs{pdfdraftmode} is not available.
% An explicit number is yielded by \cs{number}\cs{pdf@draftmode}.
% The macro cannot
% be used to change the mode, see \cs{pdf@setdraftmode}.
%
% \begin{declcs}{pdf@ifdraftmode} \M{true} \M{false}
% \end{declcs}
% If \cs{pdfdraftmode} is available and enabled, \meta{true} is
% called, otherwise \meta{false} is executed.
%
% \begin{declcs}{pdf@setdraftmode} \M{value}
% \end{declcs}
% Macro \cs{pdf@setdraftmode} expects the number zero or one as
% \meta{value}. Zero deactivates the mode and one enables the draft mode.
% The macro does not have an effect, if the feature \cs{pdfdraftmode} is not
% available.
%
% \begin{declcs}{pdf@shellescape}
% \end{declcs}
% Same as |\pdfshellescape|. It is or expands to |1| if external
% commands can be executed and |0| otherwise. In \hologo{pdfTeX} external
% commands must be enabled first by command line option or
% configuration option. In \hologo{LuaTeX} option |--safer| disables
% the execution of external commands.
%
% In \hologo{LuaTeX} before 0.68.0 \cs{pdf@shellescape} is not
% available due to a bug in |os.execute()|. The argumentless form
% crashes in some circumstances with segmentation fault.
% (It is fixed in version 0.68.0 or revision 4167 of \hologo{LuaTeX}.
% and packported to some version of 0.67.0).
%
% Hints for usage:
% \begin{itemize}
% \item Before its use \cs{pdf@shellescape} should be tested,
% whether it is available. Example with package \xpackage{ltxcmds}
% (loaded by package \xpackage{pdftexcmds}):
%\begin{quote}
%\begin{verbatim}
%\ltx@IfUndefined{pdf@shellescape}{%
%  % \pdf@shellescape is undefined
%}{%
%  % \pdf@shellescape is available
%}
%\end{verbatim}
%\end{quote}
% Use \cs{ltx@ifundefined} in expandable contexts.
% \item \cs{pdf@shellescape} might be a numerical constant,
% expands to the primitive, or expands to a plain number.
% Therefore use it in contexts where these differences does not matter.
% \item Use in comparisons, e.g.:
%   \begin{quote}
%     |\ifnum\pdf@shellescape=0 ...|
%   \end{quote}
% \item Print the number: |\number\pdf@shellescape|
% \end{itemize}
%
% \begin{declcs}{pdf@system} \M{cmdline}
% \end{declcs}
% It is a wrapper for |\immediate\write18| in \hologo{pdfTeX} or
% |os.execute| in \hologo{LuaTeX}.
%
% In theory |os.execute|
% returns a status number. But its meaning is quite
% undefined. Are there some reliable properties?
% Does it make sense to provide an user interface to
% this status exit code?
%
% \begin{declcs}{pdf@primitive} \csi{cmd}
% \end{declcs}
% Same as \cs{pdfprimitive} in \hologo{pdfTeX} or \hologo{LuaTeX}.
% In \hologo{XeTeX} the
% primitive is called \cs{primitive}. Despite the current definition
% of the command \csi{cmd}, it's meaning as primitive is used.
%
% \begin{declcs}{pdf@ifprimitive} \csi{cmd}
% \end{declcs}
% Same as \cs{ifpdfprimitive} in \hologo{pdfTeX} or
% \hologo{LuaTeX}. \hologo{XeTeX} calls
% it \cs{ifprimitive}. It is a switch that checks if the command
% \csi{cmd} has it's primitive meaning.
%
% \subsubsection{Additional macro: \cs{pdf@isprimitive}}
%
% \begin{declcs}{pdf@isprimitive} \csi{cmd1} \csi{cmd2} \M{true} \M{false}
% \end{declcs}
% If \csi{cmd1} has the primitive meaning given by the primitive name
% of \csi{cmd2}, then the argument \meta{true} is executed, otherwise
% \meta{false}. The macro \cs{pdf@isprimitive} is expandable.
% Internally it checks the result of \cs{meaning} and is therefore
% available for all \hologo{TeX} variants, even the original \hologo{TeX}.
% Example with \hologo{LaTeX}:
%\begin{quote}
%\begin{verbatim}
%\makeatletter
%\pdf@isprimitive{@@input}{input}{%
%  \typeout{\string\@@input\space is original\string\input}%
%}{%
%  \typeout{Oops, \string\@@input\space is not the %
%           original\string\input}%
%}
%\end{verbatim}
%\end{quote}
%
% \subsubsection{Experimental}
%
% \begin{declcs}{pdf@unescapehexnative} \M{string}\\
%   \cs{pdf@escapehexnative} \M{string}\\
%   \cs{pdf@escapenamenative} \M{string}\\
%   \cs{pdf@mdfivesumnative} \M{string}
% \end{declcs}
% The variants without |native| in the macro name are supposed to
% be compatible with \hologo{pdfTeX}. However characters with more than
% eight bits are not supported and are ignored. If \hologo{LuaTeX} is
% running, then its UTF-8 coded strings are used. Thus the full
% unicode character range is supported. However the result
% differs from \hologo{pdfTeX} for characters with eight or more bits.
%
% \begin{declcs}{pdf@pipe} \M{cmdline}
% \end{declcs}
% It calls \meta{cmdline} and returns the output of the external
% program in the usual manner as byte string (catcode 12, space with
% catcode 10). The Lua documentation says, that the used |io.popen|
% may not be available on all platforms. Then macro \cs{pdf@pipe}
% is undefined.
%
% \StopEventually{
% }
%
% \section{Implementation}
%
%    \begin{macrocode}
%<*package>
%    \end{macrocode}
%
% \subsection{Reload check and package identification}
%    Reload check, especially if the package is not used with \LaTeX.
%    \begin{macrocode}
\begingroup\catcode61\catcode48\catcode32=10\relax%
  \catcode13=5 % ^^M
  \endlinechar=13 %
  \catcode35=6 % #
  \catcode39=12 % '
  \catcode44=12 % ,
  \catcode45=12 % -
  \catcode46=12 % .
  \catcode58=12 % :
  \catcode64=11 % @
  \catcode123=1 % {
  \catcode125=2 % }
  \expandafter\let\expandafter\x\csname ver@pdftexcmds.sty\endcsname
  \ifx\x\relax % plain-TeX, first loading
  \else
    \def\empty{}%
    \ifx\x\empty % LaTeX, first loading,
      % variable is initialized, but \ProvidesPackage not yet seen
    \else
      \expandafter\ifx\csname PackageInfo\endcsname\relax
        \def\x#1#2{%
          \immediate\write-1{Package #1 Info: #2.}%
        }%
      \else
        \def\x#1#2{\PackageInfo{#1}{#2, stopped}}%
      \fi
      \x{pdftexcmds}{The package is already loaded}%
      \aftergroup\endinput
    \fi
  \fi
\endgroup%
%    \end{macrocode}
%    Package identification:
%    \begin{macrocode}
\begingroup\catcode61\catcode48\catcode32=10\relax%
  \catcode13=5 % ^^M
  \endlinechar=13 %
  \catcode35=6 % #
  \catcode39=12 % '
  \catcode40=12 % (
  \catcode41=12 % )
  \catcode44=12 % ,
  \catcode45=12 % -
  \catcode46=12 % .
  \catcode47=12 % /
  \catcode58=12 % :
  \catcode64=11 % @
  \catcode91=12 % [
  \catcode93=12 % ]
  \catcode123=1 % {
  \catcode125=2 % }
  \expandafter\ifx\csname ProvidesPackage\endcsname\relax
    \def\x#1#2#3[#4]{\endgroup
      \immediate\write-1{Package: #3 #4}%
      \xdef#1{#4}%
    }%
  \else
    \def\x#1#2[#3]{\endgroup
      #2[{#3}]%
      \ifx#1\@undefined
        \xdef#1{#3}%
      \fi
      \ifx#1\relax
        \xdef#1{#3}%
      \fi
    }%
  \fi
\expandafter\x\csname ver@pdftexcmds.sty\endcsname
\ProvidesPackage{pdftexcmds}%
  [2019/07/25 v0.30 Utility functions of pdfTeX for LuaTeX (HO)]%
%    \end{macrocode}
%
% \subsection{Catcodes}
%
%    \begin{macrocode}
\begingroup\catcode61\catcode48\catcode32=10\relax%
  \catcode13=5 % ^^M
  \endlinechar=13 %
  \catcode123=1 % {
  \catcode125=2 % }
  \catcode64=11 % @
  \def\x{\endgroup
    \expandafter\edef\csname pdftexcmds@AtEnd\endcsname{%
      \endlinechar=\the\endlinechar\relax
      \catcode13=\the\catcode13\relax
      \catcode32=\the\catcode32\relax
      \catcode35=\the\catcode35\relax
      \catcode61=\the\catcode61\relax
      \catcode64=\the\catcode64\relax
      \catcode123=\the\catcode123\relax
      \catcode125=\the\catcode125\relax
    }%
  }%
\x\catcode61\catcode48\catcode32=10\relax%
\catcode13=5 % ^^M
\endlinechar=13 %
\catcode35=6 % #
\catcode64=11 % @
\catcode123=1 % {
\catcode125=2 % }
\def\TMP@EnsureCode#1#2{%
  \edef\pdftexcmds@AtEnd{%
    \pdftexcmds@AtEnd
    \catcode#1=\the\catcode#1\relax
  }%
  \catcode#1=#2\relax
}
\TMP@EnsureCode{0}{12}%
\TMP@EnsureCode{1}{12}%
\TMP@EnsureCode{2}{12}%
\TMP@EnsureCode{10}{12}% ^^J
\TMP@EnsureCode{33}{12}% !
\TMP@EnsureCode{34}{12}% "
\TMP@EnsureCode{38}{4}% &
\TMP@EnsureCode{39}{12}% '
\TMP@EnsureCode{40}{12}% (
\TMP@EnsureCode{41}{12}% )
\TMP@EnsureCode{42}{12}% *
\TMP@EnsureCode{43}{12}% +
\TMP@EnsureCode{44}{12}% ,
\TMP@EnsureCode{45}{12}% -
\TMP@EnsureCode{46}{12}% .
\TMP@EnsureCode{47}{12}% /
\TMP@EnsureCode{58}{12}% :
\TMP@EnsureCode{60}{12}% <
\TMP@EnsureCode{62}{12}% >
\TMP@EnsureCode{91}{12}% [
\TMP@EnsureCode{93}{12}% ]
\TMP@EnsureCode{94}{7}% ^ (superscript)
\TMP@EnsureCode{95}{12}% _ (other)
\TMP@EnsureCode{96}{12}% `
\TMP@EnsureCode{126}{12}% ~ (other)
\edef\pdftexcmds@AtEnd{%
  \pdftexcmds@AtEnd
  \escapechar=\number\escapechar\relax
  \noexpand\endinput
}
\escapechar=92 %
%    \end{macrocode}
%
% \subsection{Load packages}
%
%    \begin{macrocode}
\begingroup\expandafter\expandafter\expandafter\endgroup
\expandafter\ifx\csname RequirePackage\endcsname\relax
  \def\TMP@RequirePackage#1[#2]{%
    \begingroup\expandafter\expandafter\expandafter\endgroup
    \expandafter\ifx\csname ver@#1.sty\endcsname\relax
      \input #1.sty\relax
    \fi
  }%
  \TMP@RequirePackage{infwarerr}[2007/09/09]%
  \TMP@RequirePackage{ifluatex}[2010/03/01]%
  \TMP@RequirePackage{ltxcmds}[2010/12/02]%
  \TMP@RequirePackage{ifpdf}[2010/09/13]%
\else
  \RequirePackage{infwarerr}[2007/09/09]%
  \RequirePackage{ifluatex}[2010/03/01]%
  \RequirePackage{ltxcmds}[2010/12/02]%
  \RequirePackage{ifpdf}[2010/09/13]%
\fi
%    \end{macrocode}
%
% \subsection{Without \hologo{LuaTeX}}
%
%    \begin{macrocode}
\ifluatex
\else
  \@PackageInfoNoLine{pdftexcmds}{LuaTeX not detected}%
  \def\pdftexcmds@nopdftex{%
    \@PackageInfoNoLine{pdftexcmds}{pdfTeX >= 1.30 not detected}%
    \let\pdftexcmds@nopdftex\relax
  }%
  \def\pdftexcmds@temp#1{%
    \begingroup\expandafter\expandafter\expandafter\endgroup
    \expandafter\ifx\csname pdf#1\endcsname\relax
      \pdftexcmds@nopdftex
    \else
      \expandafter\def\csname pdf@#1\expandafter\endcsname
      \expandafter##\expandafter{%
        \csname pdf#1\endcsname
      }%
    \fi
  }%
  \pdftexcmds@temp{strcmp}%
  \pdftexcmds@temp{escapehex}%
  \let\pdf@escapehexnative\pdf@escapehex
  \pdftexcmds@temp{unescapehex}%
  \let\pdf@unescapehexnative\pdf@unescapehex
  \pdftexcmds@temp{escapestring}%
  \pdftexcmds@temp{escapename}%
  \pdftexcmds@temp{filesize}%
  \pdftexcmds@temp{filemoddate}%
  \begingroup\expandafter\expandafter\expandafter\endgroup
  \expandafter\ifx\csname pdfshellescape\endcsname\relax
    \pdftexcmds@nopdftex
    \ltx@IfUndefined{pdftexversion}{%
    }{%
      \ifnum\pdftexversion>120 % 1.21a supports \ifeof18
        \ifeof18 %
          \chardef\pdf@shellescape=0 %
        \else
          \chardef\pdf@shellescape=1 %
        \fi
      \fi
    }%
  \else
    \def\pdf@shellescape{%
      \pdfshellescape
    }%
  \fi
  \begingroup\expandafter\expandafter\expandafter\endgroup
  \expandafter\ifx\csname pdffiledump\endcsname\relax
    \pdftexcmds@nopdftex
  \else
    \def\pdf@filedump#1#2#3{%
      \pdffiledump offset#1 length#2{#3}%
    }%
  \fi
%    \end{macrocode}
%    \begin{macrocode}
  \begingroup\expandafter\expandafter\expandafter\endgroup
  \expandafter\ifx\csname pdfmdfivesum\endcsname\relax
    \begingroup\expandafter\expandafter\expandafter\endgroup
    \expandafter\ifx\csname mdfivesum\endcsname\relax
      \pdftexcmds@nopdftex
    \else
      \def\pdf@mdfivesum#{\mdfivesum}%
      \let\pdf@mdfivesumnative\pdf@mdfivesum
      \def\pdf@filemdfivesum#{\mdfivesum file}%
    \fi
  \else
    \def\pdf@mdfivesum#{\pdfmdfivesum}%
    \let\pdf@mdfivesumnative\pdf@mdfivesum
    \def\pdf@filemdfivesum#{\pdfmdfivesum file}%
  \fi
%    \end{macrocode}
%    \begin{macrocode}
  \def\pdf@system#{%
    \immediate\write18%
  }%
  \def\pdftexcmds@temp#1{%
    \begingroup\expandafter\expandafter\expandafter\endgroup
    \expandafter\ifx\csname pdf#1\endcsname\relax
      \pdftexcmds@nopdftex
    \else
      \expandafter\let\csname pdf@#1\expandafter\endcsname
      \csname pdf#1\endcsname
    \fi
  }%
  \pdftexcmds@temp{resettimer}%
  \pdftexcmds@temp{elapsedtime}%
\fi
%    \end{macrocode}
%
% \subsection{\cs{pdf@primitive}, \cs{pdf@ifprimitive}}
%
%    Since version 1.40.0 \hologo{pdfTeX} has \cs{pdfprimitive} and
%    \cs{ifpdfprimitive}. And \cs{pdfprimitive} was fixed in
%    version 1.40.4.
%
%    \hologo{XeTeX} provides them under the name \cs{primitive} and
%    \cs{ifprimitive}. \hologo{LuaTeX} knows both name variants,
%    but they have possibly to be enabled first (|tex.enableprimitives|).
%
%    Depending on the format TeX Live uses a prefix |luatex|.
%
%    Caution: \cs{let} must be used for the definition of
%    the macros, especially because of \cs{ifpdfprimitive}.
%
% \subsubsection{Using \hologo{LuaTeX}'s \texttt{tex.enableprimitives}}
%
%    \begin{macrocode}
\ifluatex
%    \end{macrocode}
%    \begin{macro}{\pdftexcmds@directlua}
%    \begin{macrocode}
  \ifnum\luatexversion<36 %
    \def\pdftexcmds@directlua{\directlua0 }%
  \else
    \let\pdftexcmds@directlua\directlua
  \fi
%    \end{macrocode}
%    \end{macro}
%
%    \begin{macrocode}
  \begingroup
    \newlinechar=10 %
    \endlinechar=\newlinechar
    \pdftexcmds@directlua{%
      if tex.enableprimitives then
        tex.enableprimitives(
          'pdf@',
          {'primitive', 'ifprimitive', 'pdfdraftmode','draftmode'}
        )
        tex.enableprimitives('', {'luaescapestring'})
      end
    }%
  \endgroup %
%    \end{macrocode}
%
%    \begin{macrocode}
\fi
%    \end{macrocode}
%
% \subsubsection{Trying various names to find the primitives}
%
%    \begin{macro}{\pdftexcmds@strip@prefix}
%    \begin{macrocode}
\def\pdftexcmds@strip@prefix#1>{}
%    \end{macrocode}
%    \end{macro}
%    \begin{macrocode}
\def\pdftexcmds@temp#1#2#3{%
  \begingroup\expandafter\expandafter\expandafter\endgroup
  \expandafter\ifx\csname pdf@#1\endcsname\relax
    \begingroup
      \def\x{#3}%
      \edef\x{\expandafter\pdftexcmds@strip@prefix\meaning\x}%
      \escapechar=-1 %
      \edef\y{\expandafter\meaning\csname#2\endcsname}%
    \expandafter\endgroup
    \ifx\x\y
      \expandafter\let\csname pdf@#1\expandafter\endcsname
      \csname #2\endcsname
    \fi
  \fi
}
%    \end{macrocode}
%
%    \begin{macro}{\pdf@primitive}
%    \begin{macrocode}
\pdftexcmds@temp{primitive}{pdfprimitive}{pdfprimitive}% pdfTeX, oldLuaTeX
\pdftexcmds@temp{primitive}{primitive}{primitive}% XeTeX, luatex
\pdftexcmds@temp{primitive}{luatexprimitive}{pdfprimitive}% oldLuaTeX
\pdftexcmds@temp{primitive}{luatexpdfprimitive}{pdfprimitive}% oldLuaTeX
%    \end{macrocode}
%    \end{macro}
%    \begin{macro}{\pdf@ifprimitive}
%    \begin{macrocode}
\pdftexcmds@temp{ifprimitive}{ifpdfprimitive}{ifpdfprimitive}% pdfTeX, oldLuaTeX
\pdftexcmds@temp{ifprimitive}{ifprimitive}{ifprimitive}% XeTeX, luatex
\pdftexcmds@temp{ifprimitive}{luatexifprimitive}{ifpdfprimitive}% oldLuaTeX
\pdftexcmds@temp{ifprimitive}{luatexifpdfprimitive}{ifpdfprimitive}% oldLuaTeX
%    \end{macrocode}
%    \end{macro}
%
%    Disable broken \cs{pdfprimitive}.
%    \begin{macrocode}
\ifluatex\else
\begingroup
  \expandafter\ifx\csname pdf@primitive\endcsname\relax
  \else
    \expandafter\ifx\csname pdftexversion\endcsname\relax
    \else
      \ifnum\pdftexversion=140 %
        \expandafter\ifx\csname pdftexrevision\endcsname\relax
        \else
          \ifnum\pdftexrevision<4 %
            \endgroup
            \let\pdf@primitive\@undefined
            \@PackageInfoNoLine{pdftexcmds}{%
              \string\pdf@primitive\space disabled, %
              because\MessageBreak
              \string\pdfprimitive\space is broken until pdfTeX 1.40.4%
            }%
            \begingroup
          \fi
        \fi
      \fi
    \fi
  \fi
\endgroup
\fi
%    \end{macrocode}
%
% \subsubsection{Result}
%
%    \begin{macrocode}
\begingroup
  \@PackageInfoNoLine{pdftexcmds}{%
    \string\pdf@primitive\space is %
    \expandafter\ifx\csname pdf@primitive\endcsname\relax not \fi
    available%
  }%
  \@PackageInfoNoLine{pdftexcmds}{%
    \string\pdf@ifprimitive\space is %
    \expandafter\ifx\csname pdf@ifprimitive\endcsname\relax not \fi
    available%
  }%
\endgroup
%    \end{macrocode}
%
% \subsection{\hologo{XeTeX}}
%
%    Look for primitives \cs{shellescape}, \cs{strcmp}.
%    \begin{macrocode}
\def\pdftexcmds@temp#1{%
  \begingroup\expandafter\expandafter\expandafter\endgroup
  \expandafter\ifx\csname pdf@#1\endcsname\relax
    \begingroup
      \escapechar=-1 %
      \edef\x{\expandafter\meaning\csname#1\endcsname}%
      \def\y{#1}%
      \def\z##1->{}%
      \edef\y{\expandafter\z\meaning\y}%
    \expandafter\endgroup
    \ifx\x\y
      \expandafter\def\csname pdf@#1\expandafter\endcsname
      \expandafter{%
        \csname#1\endcsname
      }%
    \fi
  \fi
}%
\pdftexcmds@temp{shellescape}%
\pdftexcmds@temp{strcmp}%
%    \end{macrocode}
%
% \subsection{\cs{pdf@isprimitive}}
%
%    \begin{macrocode}
\def\pdf@isprimitive{%
  \begingroup\expandafter\expandafter\expandafter\endgroup
  \expandafter\ifx\csname pdf@strcmp\endcsname\relax
    \long\def\pdf@isprimitive##1{%
      \expandafter\pdftexcmds@isprimitive\expandafter{\meaning##1}%
    }%
    \long\def\pdftexcmds@isprimitive##1##2{%
      \expandafter\pdftexcmds@@isprimitive\expandafter{\string##2}{##1}%
    }%
    \def\pdftexcmds@@isprimitive##1##2{%
      \ifnum0\pdftexcmds@equal##1\delimiter##2\delimiter=1 %
        \expandafter\ltx@firstoftwo
      \else
        \expandafter\ltx@secondoftwo
      \fi
    }%
    \def\pdftexcmds@equal##1##2\delimiter##3##4\delimiter{%
      \ifx##1##3%
        \ifx\relax##2##4\relax
          1%
        \else
          \ifx\relax##2\relax
          \else
            \ifx\relax##4\relax
            \else
              \pdftexcmds@equalcont{##2}{##4}%
            \fi
          \fi
        \fi
      \fi
    }%
    \def\pdftexcmds@equalcont##1{%
      \def\pdftexcmds@equalcont####1####2##1##1##1##1{%
        ##1##1##1##1%
        \pdftexcmds@equal####1\delimiter####2\delimiter
      }%
    }%
    \expandafter\pdftexcmds@equalcont\csname fi\endcsname
  \else
    \long\def\pdf@isprimitive##1##2{%
      \ifnum\pdf@strcmp{\meaning##1}{\string##2}=0 %
        \expandafter\ltx@firstoftwo
      \else
        \expandafter\ltx@secondoftwo
      \fi
    }%
  \fi
}
\ifluatex
\ifx\pdfdraftmode\@undefined
  \let\pdfdraftmode\draftmode
\fi
\else
  \pdf@isprimitive
\fi
%    \end{macrocode}
%
% \subsection{\cs{pdf@draftmode}}
%
%
%    \begin{macrocode}
\let\pdftexcmds@temp\ltx@zero %
\ltx@IfUndefined{pdfdraftmode}{%
  \@PackageInfoNoLine{pdftexcmds}{\ltx@backslashchar pdfdraftmode not found}%
}{%
  \ifpdf
    \let\pdftexcmds@temp\ltx@one
    \@PackageInfoNoLine{pdftexcmds}{\ltx@backslashchar pdfdraftmode found}%
  \else
    \@PackageInfoNoLine{pdftexcmds}{%
      \ltx@backslashchar pdfdraftmode is ignored in DVI mode%
    }%
  \fi
}
\ifcase\pdftexcmds@temp
%    \end{macrocode}
%    \begin{macro}{\pdf@draftmode}
%    \begin{macrocode}
  \let\pdf@draftmode\ltx@zero
%    \end{macrocode}
%    \end{macro}
%    \begin{macro}{\pdf@ifdraftmode}
%    \begin{macrocode}
  \let\pdf@ifdraftmode\ltx@secondoftwo
%    \end{macrocode}
%    \end{macro}
%    \begin{macro}{\pdftexcmds@setdraftmode}
%    \begin{macrocode}
  \def\pdftexcmds@setdraftmode#1{}%
%    \end{macrocode}
%    \end{macro}
%    \begin{macrocode}
\else
%    \end{macrocode}
%    \begin{macro}{\pdftexcmds@draftmode}
%    \begin{macrocode}
  \let\pdftexcmds@draftmode\pdfdraftmode
%    \end{macrocode}
%    \end{macro}
%    \begin{macro}{\pdf@ifdraftmode}
%    \begin{macrocode}
  \def\pdf@ifdraftmode{%
    \ifnum\pdftexcmds@draftmode=\ltx@one
      \expandafter\ltx@firstoftwo
    \else
      \expandafter\ltx@secondoftwo
    \fi
  }%
%    \end{macrocode}
%    \end{macro}
%    \begin{macro}{\pdf@draftmode}
%    \begin{macrocode}
  \def\pdf@draftmode{%
    \ifnum\pdftexcmds@draftmode=\ltx@one
      \expandafter\ltx@one
    \else
      \expandafter\ltx@zero
    \fi
  }%
%    \end{macrocode}
%    \end{macro}
%    \begin{macro}{\pdftexcmds@setdraftmode}
%    \begin{macrocode}
  \def\pdftexcmds@setdraftmode#1{%
    \pdftexcmds@draftmode=#1\relax
  }%
%    \end{macrocode}
%    \end{macro}
%    \begin{macrocode}
\fi
%    \end{macrocode}
%    \begin{macro}{\pdf@setdraftmode}
%    \begin{macrocode}
\def\pdf@setdraftmode#1{%
  \begingroup
    \count\ltx@cclv=#1\relax
  \edef\x{\endgroup
    \noexpand\pdftexcmds@@setdraftmode{\the\count\ltx@cclv}%
  }%
  \x
}
%    \end{macrocode}
%    \end{macro}
%    \begin{macro}{\pdftexcmds@@setdraftmode}
%    \begin{macrocode}
\def\pdftexcmds@@setdraftmode#1{%
  \ifcase#1 %
    \pdftexcmds@setdraftmode{#1}%
  \or
    \pdftexcmds@setdraftmode{#1}%
  \else
    \@PackageWarning{pdftexcmds}{%
      \string\pdf@setdraftmode: Ignoring\MessageBreak
      invalid value `#1'%
    }%
  \fi
}
%    \end{macrocode}
%    \end{macro}
%
% \subsection{Load Lua module}
%
%    \begin{macrocode}
\ifluatex
\else
  \expandafter\pdftexcmds@AtEnd
\fi%
%    \end{macrocode}
%
%    \begin{macrocode}
\ifnum\luatexversion<80
  \begingroup\expandafter\expandafter\expandafter\endgroup
  \expandafter\ifx\csname RequirePackage\endcsname\relax
    \def\TMP@RequirePackage#1[#2]{%
      \begingroup\expandafter\expandafter\expandafter\endgroup
      \expandafter\ifx\csname ver@#1.sty\endcsname\relax
        \input #1.sty\relax
      \fi
    }%
    \TMP@RequirePackage{luatex-loader}[2009/04/10]%
  \else
    \RequirePackage{luatex-loader}[2009/04/10]%
  \fi
\fi
\pdftexcmds@directlua{%
  require("pdftexcmds")%
}
\ifnum\luatexversion>37 %
  \ifnum0%
      \pdftexcmds@directlua{%
        if status.ini_version then %
          tex.write("1")%
        end%
      }>0 %
    \everyjob\expandafter{%
      \the\everyjob
      \pdftexcmds@directlua{%
        require("pdftexcmds")%
      }%
    }%
  \fi
\fi
\begingroup
  \def\x{2019/07/25 v0.30}%
  \ltx@onelevel@sanitize\x
  \edef\y{%
    \pdftexcmds@directlua{%
      if oberdiek.pdftexcmds.getversion then %
        oberdiek.pdftexcmds.getversion()%
      end%
    }%
  }%
  \ifx\x\y
  \else
    \@PackageError{pdftexcmds}{%
      Wrong version of lua module.\MessageBreak
      Package version: \x\MessageBreak
      Lua module: \y
    }\@ehc
  \fi
\endgroup
%    \end{macrocode}
%
% \subsection{Lua functions}
%
% \subsubsection{Helper macros}
%
%    \begin{macro}{\pdftexcmds@toks}
%    \begin{macrocode}
\begingroup\expandafter\expandafter\expandafter\endgroup
\expandafter\ifx\csname newtoks\endcsname\relax
  \toksdef\pdftexcmds@toks=0 %
\else
  \csname newtoks\endcsname\pdftexcmds@toks
\fi
%    \end{macrocode}
%    \end{macro}
%
%    \begin{macro}{\pdftexcmds@Patch}
%    \begin{macrocode}
\def\pdftexcmds@Patch{0}
\ifnum\luatexversion>40 %
  \ifnum\luatexversion<66 %
    \def\pdftexcmds@Patch{1}%
  \fi
\fi
%    \end{macrocode}
%    \end{macro}
%    \begin{macrocode}
\ifcase\pdftexcmds@Patch
  \catcode`\&=14 %
\else
  \catcode`\&=9 %
%    \end{macrocode}
%    \begin{macro}{\pdftexcmds@PatchDecode}
%    \begin{macrocode}
  \def\pdftexcmds@PatchDecode#1\@nil{%
    \pdftexcmds@DecodeA#1^^A^^A\@nil{}%
  }%
%    \end{macrocode}
%    \end{macro}
%    \begin{macro}{\pdftexcmds@DecodeA}
%    \begin{macrocode}
  \def\pdftexcmds@DecodeA#1^^A^^A#2\@nil#3{%
    \ifx\relax#2\relax
      \ltx@ReturnAfterElseFi{%
        \pdftexcmds@DecodeB#3#1^^A^^B\@nil{}%
      }%
    \else
      \ltx@ReturnAfterFi{%
        \pdftexcmds@DecodeA#2\@nil{#3#1^^@}%
      }%
    \fi
  }%
%    \end{macrocode}
%    \end{macro}
%    \begin{macro}{\pdftexcmds@DecodeB}
%    \begin{macrocode}
  \def\pdftexcmds@DecodeB#1^^A^^B#2\@nil#3{%
    \ifx\relax#2\relax%
      \ltx@ReturnAfterElseFi{%
        \ltx@zero
        #3#1%
      }%
    \else
      \ltx@ReturnAfterFi{%
        \pdftexcmds@DecodeB#2\@nil{#3#1^^A}%
      }%
    \fi
  }%
%    \end{macrocode}
%    \end{macro}
%    \begin{macrocode}
\fi
%    \end{macrocode}
%
%    \begin{macrocode}
\ifnum\luatexversion<36 %
\else
  \catcode`\0=9 %
\fi
%    \end{macrocode}
%
% \subsubsection[Strings]{Strings \cite[``7.15 Strings'']{pdftex-manual}}
%
%    \begin{macro}{\pdf@strcmp}
%    \begin{macrocode}
\long\def\pdf@strcmp#1#2{%
  \directlua0{%
    oberdiek.pdftexcmds.strcmp("\luaescapestring{#1}",%
        "\luaescapestring{#2}")%
  }%
}%
%    \end{macrocode}
%    \end{macro}
%    \begin{macrocode}
\pdf@isprimitive
%    \end{macrocode}
%    \begin{macro}{\pdf@escapehex}
%    \begin{macrocode}
\long\def\pdf@escapehex#1{%
  \directlua0{%
    oberdiek.pdftexcmds.escapehex("\luaescapestring{#1}", "byte")%
  }%
}%
%    \end{macrocode}
%    \end{macro}
%    \begin{macro}{\pdf@escapehexnative}
%    \begin{macrocode}
\long\def\pdf@escapehexnative#1{%
  \directlua0{%
    oberdiek.pdftexcmds.escapehex("\luaescapestring{#1}")%
  }%
}%
%    \end{macrocode}
%    \end{macro}
%    \begin{macro}{\pdf@unescapehex}
%    \begin{macrocode}
\def\pdf@unescapehex#1{%
& \romannumeral\expandafter\pdftexcmds@PatchDecode
  \the\expandafter\pdftexcmds@toks
  \directlua0{%
    oberdiek.pdftexcmds.toks="pdftexcmds@toks"%
    oberdiek.pdftexcmds.unescapehex("\luaescapestring{#1}", "byte", \pdftexcmds@Patch)%
  }%
& \@nil
}%
%    \end{macrocode}
%    \end{macro}
%    \begin{macro}{\pdf@unescapehexnative}
%    \begin{macrocode}
\def\pdf@unescapehexnative#1{%
& \romannumeral\expandafter\pdftexcmds@PatchDecode
  \the\expandafter\pdftexcmds@toks
  \directlua0{%
    oberdiek.pdftexcmds.toks="pdftexcmds@toks"%
    oberdiek.pdftexcmds.unescapehex("\luaescapestring{#1}", \pdftexcmds@Patch)%
  }%
& \@nil
}%
%    \end{macrocode}
%    \end{macro}
%    \begin{macro}{\pdf@escapestring}
%    \begin{macrocode}
\long\def\pdf@escapestring#1{%
  \directlua0{%
    oberdiek.pdftexcmds.escapestring("\luaescapestring{#1}", "byte")%
  }%
}
%    \end{macrocode}
%    \end{macro}
%    \begin{macro}{\pdf@escapename}
%    \begin{macrocode}
\long\def\pdf@escapename#1{%
  \directlua0{%
    oberdiek.pdftexcmds.escapename("\luaescapestring{#1}", "byte")%
  }%
}
%    \end{macrocode}
%    \end{macro}
%    \begin{macro}{\pdf@escapenamenative}
%    \begin{macrocode}
\long\def\pdf@escapenamenative#1{%
  \directlua0{%
    oberdiek.pdftexcmds.escapename("\luaescapestring{#1}")%
  }%
}
%    \end{macrocode}
%    \end{macro}
%
% \subsubsection[Files]{Files \cite[``7.18 Files'']{pdftex-manual}}
%
%    \begin{macro}{\pdf@filesize}
%    \begin{macrocode}
\def\pdf@filesize#1{%
  \directlua0{%
    oberdiek.pdftexcmds.filesize("\luaescapestring{#1}")%
  }%
}
%    \end{macrocode}
%    \end{macro}
%    \begin{macro}{\pdf@filemoddate}
%    \begin{macrocode}
\def\pdf@filemoddate#1{%
  \directlua0{%
    oberdiek.pdftexcmds.filemoddate("\luaescapestring{#1}")%
  }%
}
%    \end{macrocode}
%    \end{macro}
%    \begin{macro}{\pdf@filedump}
%    \begin{macrocode}
\def\pdf@filedump#1#2#3{%
  \directlua0{%
    oberdiek.pdftexcmds.filedump("\luaescapestring{\number#1}",%
        "\luaescapestring{\number#2}",%
        "\luaescapestring{#3}")%
  }%
}%
%    \end{macrocode}
%    \end{macro}
%    \begin{macro}{\pdf@mdfivesum}
%    \begin{macrocode}
\long\def\pdf@mdfivesum#1{%
  \directlua0{%
    oberdiek.pdftexcmds.mdfivesum("\luaescapestring{#1}", "byte")%
  }%
}%
%    \end{macrocode}
%    \end{macro}
%    \begin{macro}{\pdf@mdfivesumnative}
%    \begin{macrocode}
\long\def\pdf@mdfivesumnative#1{%
  \directlua0{%
    oberdiek.pdftexcmds.mdfivesum("\luaescapestring{#1}")%
  }%
}%
%    \end{macrocode}
%    \end{macro}
%    \begin{macro}{\pdf@filemdfivesum}
%    \begin{macrocode}
\def\pdf@filemdfivesum#1{%
  \directlua0{%
    oberdiek.pdftexcmds.filemdfivesum("\luaescapestring{#1}")%
  }%
}%
%    \end{macrocode}
%    \end{macro}
%
% \subsubsection[Timekeeping]{Timekeeping \cite[``7.17 Timekeeping'']{pdftex-manual}}
%
%    \begin{macro}{\protected}
%    \begin{macrocode}
\let\pdftexcmds@temp=Y%
\begingroup\expandafter\expandafter\expandafter\endgroup
\expandafter\ifx\csname protected\endcsname\relax
  \pdftexcmds@directlua0{%
    if tex.enableprimitives then %
      tex.enableprimitives('', {'protected'})%
    end%
  }%
\fi
\begingroup\expandafter\expandafter\expandafter\endgroup
\expandafter\ifx\csname protected\endcsname\relax
  \let\pdftexcmds@temp=N%
\fi
%    \end{macrocode}
%    \end{macro}
%    \begin{macro}{\numexpr}
%    \begin{macrocode}
\begingroup\expandafter\expandafter\expandafter\endgroup
\expandafter\ifx\csname numexpr\endcsname\relax
  \pdftexcmds@directlua0{%
    if tex.enableprimitives then %
      tex.enableprimitives('', {'numexpr'})%
    end%
  }%
\fi
\begingroup\expandafter\expandafter\expandafter\endgroup
\expandafter\ifx\csname numexpr\endcsname\relax
  \let\pdftexcmds@temp=N%
\fi
%    \end{macrocode}
%    \end{macro}
%
%    \begin{macrocode}
\ifx\pdftexcmds@temp N%
  \@PackageWarningNoLine{pdftexcmds}{%
    Definitions of \ltx@backslashchar pdf@resettimer and%
    \MessageBreak
    \ltx@backslashchar pdf@elapsedtime are skipped, because%
    \MessageBreak
    e-TeX's \ltx@backslashchar protected or %
    \ltx@backslashchar numexpr are missing%
  }%
\else
%    \end{macrocode}
%
%    \begin{macro}{\pdf@resettimer}
%    \begin{macrocode}
  \protected\def\pdf@resettimer{%
    \pdftexcmds@directlua0{%
      oberdiek.pdftexcmds.resettimer()%
    }%
  }%
%    \end{macrocode}
%    \end{macro}
%
%    \begin{macro}{\pdf@elapsedtime}
%    \begin{macrocode}
  \protected\def\pdf@elapsedtime{%
    \numexpr
      \pdftexcmds@directlua0{%
        oberdiek.pdftexcmds.elapsedtime()%
      }%
    \relax
  }%
%    \end{macrocode}
%    \end{macro}
%    \begin{macrocode}
\fi
%    \end{macrocode}
%
% \subsubsection{Shell escape}
%
%    \begin{macro}{\pdf@shellescape}
%
%    \begin{macrocode}
\ifnum\luatexversion<68 %
\else
  \protected\edef\pdf@shellescape{%
   \numexpr\directlua{tex.sprint(%
         \number\catcodetable@string,status.shell_escape)}\relax}
\fi
%    \end{macrocode}
%    \end{macro}
%
%    \begin{macro}{\pdf@system}
%    \begin{macrocode}
\def\pdf@system#1{%
  \directlua0{%
    oberdiek.pdftexcmds.system("\luaescapestring{#1}")%
  }%
}
%    \end{macrocode}
%    \end{macro}
%
%    \begin{macro}{\pdf@lastsystemstatus}
%    \begin{macrocode}
\def\pdf@lastsystemstatus{%
  \directlua0{%
    oberdiek.pdftexcmds.lastsystemstatus()%
  }%
}
%    \end{macrocode}
%    \end{macro}
%    \begin{macro}{\pdf@lastsystemexit}
%    \begin{macrocode}
\def\pdf@lastsystemexit{%
  \directlua0{%
    oberdiek.pdftexcmds.lastsystemexit()%
  }%
}
%    \end{macrocode}
%    \end{macro}
%
%    \begin{macrocode}
\catcode`\0=12 %
%    \end{macrocode}
%
%    \begin{macro}{\pdf@pipe}
%    Check availability of |io.popen| first.
%    \begin{macrocode}
\ifnum0%
    \pdftexcmds@directlua{%
      if io.popen then %
        tex.write("1")%
      end%
    }%
    =1 %
  \def\pdf@pipe#1{%
&   \romannumeral\expandafter\pdftexcmds@PatchDecode
    \the\expandafter\pdftexcmds@toks
    \pdftexcmds@directlua{%
      oberdiek.pdftexcmds.toks="pdftexcmds@toks"%
      oberdiek.pdftexcmds.pipe("\luaescapestring{#1}", \pdftexcmds@Patch)%
    }%
&   \@nil
  }%
\fi
%    \end{macrocode}
%    \end{macro}
%
%    \begin{macrocode}
\pdftexcmds@AtEnd%
%</package>
%    \end{macrocode}
%
% \subsection{Lua module}
%
%    \begin{macrocode}
%<*lua>
%    \end{macrocode}
%
%    \begin{macrocode}
oberdiek = oberdiek or {}
local pdftexcmds = oberdiek.pdftexcmds or {}
oberdiek.pdftexcmds = pdftexcmds
local systemexitstatus
function pdftexcmds.getversion()
  tex.write("2019/07/25 v0.30")
end
%    \end{macrocode}
%
% \subsubsection[Strings]{Strings \cite[``7.15 Strings'']{pdftex-manual}}
%
%    \begin{macrocode}
function pdftexcmds.strcmp(A, B)
  if A == B then
    tex.write("0")
  elseif A < B then
    tex.write("-1")
  else
    tex.write("1")
  end
end
local function utf8_to_byte(str)
  local i = 0
  local n = string.len(str)
  local t = {}
  while i < n do
    i = i + 1
    local a = string.byte(str, i)
    if a < 128 then
      table.insert(t, string.char(a))
    else
      if a >= 192 and i < n then
        i = i + 1
        local b = string.byte(str, i)
        if b < 128 or b >= 192 then
          i = i - 1
        elseif a == 194 then
          table.insert(t, string.char(b))
        elseif a == 195 then
          table.insert(t, string.char(b + 64))
        end
      end
    end
  end
  return table.concat(t)
end
function pdftexcmds.escapehex(str, mode)
  if mode == "byte" then
    str = utf8_to_byte(str)
  end
  tex.write((string.gsub(str, ".",
    function (ch)
      return string.format("%02X", string.byte(ch))
    end
  )))
end
%    \end{macrocode}
%    See procedure |unescapehex| in file \xfile{utils.c} of \hologo{pdfTeX}.
%    Caution: |tex.write| ignores leading spaces.
%    \begin{macrocode}
function pdftexcmds.unescapehex(str, mode, patch)
  local a = 0
  local first = true
  local result = {}
  for i = 1, string.len(str), 1 do
    local ch = string.byte(str, i)
    if ch >= 48 and ch <= 57 then
      ch = ch - 48
    elseif ch >= 65 and ch <= 70 then
      ch = ch - 55
    elseif ch >= 97 and ch <= 102 then
      ch = ch - 87
    else
      ch = nil
    end
    if ch then
      if first then
        a = ch * 16
        first = false
      else
        table.insert(result, a + ch)
        first = true
      end
    end
  end
  if not first then
    table.insert(result, a)
  end
  if patch == 1 then
    local temp = {}
    for i, a in ipairs(result) do
      if a == 0 then
        table.insert(temp, 1)
        table.insert(temp, 1)
      else
        if a == 1 then
          table.insert(temp, 1)
          table.insert(temp, 2)
        else
          table.insert(temp, a)
        end
      end
    end
    result = temp
  end
  if mode == "byte" then
    local utf8 = {}
    for i, a in ipairs(result) do
      if a < 128 then
        table.insert(utf8, a)
      else
        if a < 192 then
          table.insert(utf8, 194)
          a = a - 128
        else
          table.insert(utf8, 195)
          a = a - 192
        end
        table.insert(utf8, a + 128)
      end
    end
    result = utf8
  end
%    \end{macrocode}
%    this next line added for current luatex; this is the only
%    change in the file.  eroux, 28apr13. (v 0.21)
%    \begin{macrocode}
  local unpack = _G["unpack"] or table.unpack
  tex.settoks(pdftexcmds.toks, string.char(unpack(result)))
end
%    \end{macrocode}
%    See procedure |escapestring| in file \xfile{utils.c} of \hologo{pdfTeX}.
%    \begin{macrocode}
function pdftexcmds.escapestring(str, mode)
  if mode == "byte" then
    str = utf8_to_byte(str)
  end
  tex.write((string.gsub(str, ".",
    function (ch)
      local b = string.byte(ch)
      if b < 33 or b > 126 then
        return string.format("\\%.3o", b)
      end
      if b == 40 or b == 41 or b == 92 then
        return "\\" .. ch
      end
%    \end{macrocode}
%    Lua 5.1 returns the match in case of return value |nil|.
%    \begin{macrocode}
      return nil
    end
  )))
end
%    \end{macrocode}
%    See procedure |escapename| in file \xfile{utils.c} of \hologo{pdfTeX}.
%    \begin{macrocode}
function pdftexcmds.escapename(str, mode)
  if mode == "byte" then
    str = utf8_to_byte(str)
  end
  tex.write((string.gsub(str, ".",
    function (ch)
      local b = string.byte(ch)
      if b == 0 then
%    \end{macrocode}
%    In Lua 5.0 |nil| could be used for the empty string,
%    But |nil| returns the match in Lua 5.1, thus we use
%    the empty string explicitly.
%    \begin{macrocode}
        return ""
      end
      if b <= 32 or b >= 127
          or b == 35 or b == 37 or b == 40 or b == 41
          or b == 47 or b == 60 or b == 62 or b == 91
          or b == 93 or b == 123 or b == 125 then
        return string.format("#%.2X", b)
      else
%    \end{macrocode}
%    Lua 5.1 returns the match in case of return value |nil|.
%    \begin{macrocode}
        return nil
      end
    end
  )))
end
%    \end{macrocode}
%
% \subsubsection[Files]{Files \cite[``7.18 Files'']{pdftex-manual}}
%
%    \begin{macrocode}
function pdftexcmds.filesize(filename)
  local foundfile = kpse.find_file(filename, "tex", true)
  if foundfile then
    local size = lfs.attributes(foundfile, "size")
    if size then
      tex.write(size)
    end
  end
end
%    \end{macrocode}
%    See procedure |makepdftime| in file \xfile{utils.c} of \hologo{pdfTeX}.
%    \begin{macrocode}
function pdftexcmds.filemoddate(filename)
  local foundfile = kpse.find_file(filename, "tex", true)
  if foundfile then
    local date = lfs.attributes(foundfile, "modification")
    if date then
      local d = os.date("*t", date)
      if d.sec >= 60 then
        d.sec = 59
      end
      local u = os.date("!*t", date)
      local off = 60 * (d.hour - u.hour) + d.min - u.min
      if d.year ~= u.year then
        if d.year > u.year then
          off = off + 1440
        else
          off = off - 1440
        end
      elseif d.yday ~= u.yday then
        if d.yday > u.yday then
          off = off + 1440
        else
          off = off - 1440
        end
      end
      local timezone
      if off == 0 then
        timezone = "Z"
      else
        local hours = math.floor(off / 60)
        local mins = math.abs(off - hours * 60)
        timezone = string.format("%+03d'%02d'", hours, mins)
      end
      tex.write(string.format("D:%04d%02d%02d%02d%02d%02d%s",
          d.year, d.month, d.day, d.hour, d.min, d.sec, timezone))
    end
  end
end
function pdftexcmds.filedump(offset, length, filename)
  length = tonumber(length)
  if length and length > 0 then
    local foundfile = kpse.find_file(filename, "tex", true)
    if foundfile then
      offset = tonumber(offset)
      if not offset then
        offset = 0
      end
      local filehandle = io.open(foundfile, "rb")
      if filehandle then
        if offset > 0 then
          filehandle:seek("set", offset)
        end
        local dump = filehandle:read(length)
        pdftexcmds.escapehex(dump)
        filehandle:close()
      end
    end
  end
end
function pdftexcmds.mdfivesum(str, mode)
  if mode == "byte" then
    str = utf8_to_byte(str)
  end
  pdftexcmds.escapehex(md5.sum(str))
end
function pdftexcmds.filemdfivesum(filename)
  local foundfile = kpse.find_file(filename, "tex", true)
  if foundfile then
    local filehandle = io.open(foundfile, "rb")
    if filehandle then
      local contents = filehandle:read("*a")
      pdftexcmds.escapehex(md5.sum(contents))
      filehandle:close()
    end
  end
end
%    \end{macrocode}
%
% \subsubsection[Timekeeping]{Timekeeping \cite[``7.17 Timekeeping'']{pdftex-manual}}
%
%    The functions for timekeeping are based on
%    Andy Thomas' work \cite{AndyThomas:Analog}.
%    Changes:
%    \begin{itemize}
%    \item Overflow check is added.
%    \item |string.format| is used to avoid exponential number
%          representation for sure.
%    \item |tex.write| is used instead of |tex.print| to get
%          tokens with catcode 12 and without appended \cs{endlinechar}.
%    \end{itemize}
%    \begin{macrocode}
local basetime = 0
function pdftexcmds.resettimer()
  basetime = os.clock()
end
function pdftexcmds.elapsedtime()
  local val = (os.clock() - basetime) * 65536 + .5
  if val > 2147483647 then
    val = 2147483647
  end
  tex.write(string.format("%d", val))
end
%    \end{macrocode}
%
% \subsubsection[Miscellaneous]{Miscellaneous \cite[``7.21 Miscellaneous'']{pdftex-manual}}
%
%    \begin{macrocode}
function pdftexcmds.shellescape()
  if os.execute then
    if status
        and status.luatex_version
        and status.luatex_version >= 68 then
      tex.write(os.execute())
    else
      local result = os.execute()
      if result == 0 then
        tex.write("0")
      else
        if result == nil then
          tex.write("0")
        else
          tex.write("1")
        end
      end
    end
  else
    tex.write("0")
  end
end
function pdftexcmds.system(cmdline)
  systemexitstatus = nil
  texio.write_nl("log", "system(" .. cmdline .. ") ")
  if os.execute then
    texio.write("log", "executed.")
    systemexitstatus = os.execute(cmdline)
  else
    texio.write("log", "disabled.")
  end
end
function pdftexcmds.lastsystemstatus()
  local result = tonumber(systemexitstatus)
  if result then
    local x = math.floor(result / 256)
    tex.write(result - 256 * math.floor(result / 256))
  end
end
function pdftexcmds.lastsystemexit()
  local result = tonumber(systemexitstatus)
  if result then
    tex.write(math.floor(result / 256))
  end
end
function pdftexcmds.pipe(cmdline, patch)
  local result
  systemexitstatus = nil
  texio.write_nl("log", "pipe(" .. cmdline ..") ")
  if io.popen then
    texio.write("log", "executed.")
    local handle = io.popen(cmdline, "r")
    if handle then
      result = handle:read("*a")
      handle:close()
    end
  else
    texio.write("log", "disabled.")
  end
  if result then
    if patch == 1 then
      local temp = {}
      for i, a in ipairs(result) do
        if a == 0 then
          table.insert(temp, 1)
          table.insert(temp, 1)
        else
          if a == 1 then
            table.insert(temp, 1)
            table.insert(temp, 2)
          else
            table.insert(temp, a)
          end
        end
      end
      result = temp
    end
    tex.settoks(pdftexcmds.toks, result)
  else
    tex.settoks(pdftexcmds.toks, "")
  end
end
%    \end{macrocode}
%    \begin{macrocode}
%</lua>
%    \end{macrocode}
%
% \section{Test}
%
% \subsection{Catcode checks for loading}
%
%    \begin{macrocode}
%<*test1>
%    \end{macrocode}
%    \begin{macrocode}
\catcode`\{=1 %
\catcode`\}=2 %
\catcode`\#=6 %
\catcode`\@=11 %
\expandafter\ifx\csname count@\endcsname\relax
  \countdef\count@=255 %
\fi
\expandafter\ifx\csname @gobble\endcsname\relax
  \long\def\@gobble#1{}%
\fi
\expandafter\ifx\csname @firstofone\endcsname\relax
  \long\def\@firstofone#1{#1}%
\fi
\expandafter\ifx\csname loop\endcsname\relax
  \expandafter\@firstofone
\else
  \expandafter\@gobble
\fi
{%
  \def\loop#1\repeat{%
    \def\body{#1}%
    \iterate
  }%
  \def\iterate{%
    \body
      \let\next\iterate
    \else
      \let\next\relax
    \fi
    \next
  }%
  \let\repeat=\fi
}%
\def\RestoreCatcodes{}
\count@=0 %
\loop
  \edef\RestoreCatcodes{%
    \RestoreCatcodes
    \catcode\the\count@=\the\catcode\count@\relax
  }%
\ifnum\count@<255 %
  \advance\count@ 1 %
\repeat

\def\RangeCatcodeInvalid#1#2{%
  \count@=#1\relax
  \loop
    \catcode\count@=15 %
  \ifnum\count@<#2\relax
    \advance\count@ 1 %
  \repeat
}
\def\RangeCatcodeCheck#1#2#3{%
  \count@=#1\relax
  \loop
    \ifnum#3=\catcode\count@
    \else
      \errmessage{%
        Character \the\count@\space
        with wrong catcode \the\catcode\count@\space
        instead of \number#3%
      }%
    \fi
  \ifnum\count@<#2\relax
    \advance\count@ 1 %
  \repeat
}
\def\space{ }
\expandafter\ifx\csname LoadCommand\endcsname\relax
  \def\LoadCommand{\input pdftexcmds.sty\relax}%
\fi
\def\Test{%
  \RangeCatcodeInvalid{0}{47}%
  \RangeCatcodeInvalid{58}{64}%
  \RangeCatcodeInvalid{91}{96}%
  \RangeCatcodeInvalid{123}{255}%
  \catcode`\@=12 %
  \catcode`\\=0 %
  \catcode`\%=14 %
  \LoadCommand
  \RangeCatcodeCheck{0}{36}{15}%
  \RangeCatcodeCheck{37}{37}{14}%
  \RangeCatcodeCheck{38}{47}{15}%
  \RangeCatcodeCheck{48}{57}{12}%
  \RangeCatcodeCheck{58}{63}{15}%
  \RangeCatcodeCheck{64}{64}{12}%
  \RangeCatcodeCheck{65}{90}{11}%
  \RangeCatcodeCheck{91}{91}{15}%
  \RangeCatcodeCheck{92}{92}{0}%
  \RangeCatcodeCheck{93}{96}{15}%
  \RangeCatcodeCheck{97}{122}{11}%
  \RangeCatcodeCheck{123}{255}{15}%
  \RestoreCatcodes
}
\Test
\csname @@end\endcsname
\end
%    \end{macrocode}
%    \begin{macrocode}
%</test1>
%    \end{macrocode}
%
% \subsection{Test for \cs{pdf@isprimitive}}
%
%    \begin{macrocode}
%<*test2>
\catcode`\{=1 %
\catcode`\}=2 %
\catcode`\#=6 %
\catcode`\@=11 %
\input pdftexcmds.sty\relax
\def\msg#1{%
  \begingroup
    \escapechar=92 %
    \immediate\write16{#1}%
  \endgroup
}
\long\def\test#1#2#3#4{%
  \begingroup
    #4%
    \def\str{%
      Test \string\pdf@isprimitive
      {\string #1}{\string #2}{...}: %
    }%
    \pdf@isprimitive{#1}{#2}{%
      \ifx#3Y%
        \msg{\str true ==> OK.}%
      \else
        \errmessage{\str false ==> FAILED}%
      \fi
    }{%
      \ifx#3Y%
        \errmessage{\str true ==> FAILED}%
      \else
        \msg{\str false ==> OK.}%
      \fi
    }%
  \endgroup
}
\test\relax\relax Y{}
\test\foobar\relax Y{\let\foobar\relax}
\test\foobar\relax N{}
\test\hbox\hbox Y{}
\test\foobar@hbox\hbox Y{\let\foobar@hbox\hbox}
\test\if\if Y{}
\test\if\ifx N{}
\test\ifx\if N{}
\test\par\par Y{}
\test\hbox\par N{}
\test\par\hbox N{}
\csname @@end\endcsname\end
%</test2>
%    \end{macrocode}
%
% \subsection{Test for \cs{pdf@shellescape}}
%
%    \begin{macrocode}
%<*test-shell>
\catcode`\{=1 %
\catcode`\}=2 %
\catcode`\#=6 %
\catcode`\@=11 %
\input pdftexcmds.sty\relax
\def\msg#{\immediate\write16}
\def\MaybeEnd{}
\ifx\luatexversion\UnDeFiNeD
\else
  \ifnum\luatexversion<68 %
    \ifx\pdf@shellescape\@undefined
      \msg{SHELL=U}%
      \msg{OK (LuaTeX < 0.68)}%
    \else
      \msg{SHELL=defined}%
      \errmessage{Failed (LuaTeX < 0.68)}%
    \fi
    \def\MaybeEnd{\csname @@end\endcsname\end}%
  \fi
\fi
\MaybeEnd
\ifx\pdf@shellescape\@undefined
  \msg{SHELL=U}%
\else
  \msg{SHELL=\number\pdf@shellescape}%
\fi
\ifx\expected\@undefined
\else
  \ifx\expected\relax
    \msg{EXPECTED=U}%
    \ifx\pdf@shellescape\@undefined
      \msg{OK}%
    \else
      \errmessage{Failed}%
    \fi
  \else
    \msg{EXPECTED=\number\expected}%
    \ifnum\pdf@shellescape=\expected\relax
      \msg{OK}%
    \else
      \errmessage{Failed}%
    \fi
  \fi
\fi
\csname @@end\endcsname\end
%</test-shell>
%    \end{macrocode}
%
% \subsection{Test for escape functions}
%
%    \begin{macrocode}
%<*test-escape>
\catcode`\{=1 %
\catcode`\}=2 %
\catcode`\#=6 %
\catcode`\^=7 %
\catcode`\@=11 %
\errorcontextlines=1000 %
\input pdftexcmds.sty\relax
\def\msg#1{%
  \begingroup
    \escapechar=92 %
    \immediate\write16{#1}%
  \endgroup
}
%    \end{macrocode}
%    \begin{macrocode}
\begingroup
  \catcode`\@=11 %
  \countdef\count@=255 %
  \def\space{ }%
  \long\def\@whilenum#1\do #2{%
    \ifnum #1\relax
      #2\relax
      \@iwhilenum{#1\relax#2\relax}%
    \fi
  }%
  \long\def\@iwhilenum#1{%
    \ifnum #1%
      \expandafter\@iwhilenum
    \else
      \expandafter\ltx@gobble
    \fi
    {#1}%
  }%
  \gdef\AllBytes{}%
  \count@=0 %
  \catcode0=12 %
  \@whilenum\count@<256 \do{%
    \lccode0=\count@
    \ifnum\count@=32 %
      \xdef\AllBytes{\AllBytes\space}%
    \else
      \lowercase{%
        \xdef\AllBytes{\AllBytes^^@}%
      }%
    \fi
    \advance\count@ by 1 %
  }%
\endgroup
%    \end{macrocode}
%    \begin{macrocode}
\def\AllBytesHex{%
  000102030405060708090A0B0C0D0E0F%
  101112131415161718191A1B1C1D1E1F%
  202122232425262728292A2B2C2D2E2F%
  303132333435363738393A3B3C3D3E3F%
  404142434445464748494A4B4C4D4E4F%
  505152535455565758595A5B5C5D5E5F%
  606162636465666768696A6B6C6D6E6F%
  707172737475767778797A7B7C7D7E7F%
  808182838485868788898A8B8C8D8E8F%
  909192939495969798999A9B9C9D9E9F%
  A0A1A2A3A4A5A6A7A8A9AAABACADAEAF%
  B0B1B2B3B4B5B6B7B8B9BABBBCBDBEBF%
  C0C1C2C3C4C5C6C7C8C9CACBCCCDCECF%
  D0D1D2D3D4D5D6D7D8D9DADBDCDDDEDF%
  E0E1E2E3E4E5E6E7E8E9EAEBECEDEEEF%
  F0F1F2F3F4F5F6F7F8F9FAFBFCFDFEFF%
}
\ltx@onelevel@sanitize\AllBytesHex
\expandafter\lowercase\expandafter{%
  \expandafter\def\expandafter\AllBytesHexLC
      \expandafter{\AllBytesHex}%
}
\begingroup
  \catcode`\#=12 %
  \xdef\AllBytesName{%
    #01#02#03#04#05#06#07#08#09#0A#0B#0C#0D#0E#0F%
    #10#11#12#13#14#15#16#17#18#19#1A#1B#1C#1D#1E#1F%
    #20!"#23$#25&'#28#29*+,-.#2F%
    0123456789:;#3C=#3E?%
    @ABCDEFGHIJKLMNO%
    PQRSTUVWXYZ#5B\ltx@backslashchar#5D^_%
    `abcdefghijklmno%
    pqrstuvwxyz#7B|#7D\string~#7F%
    #80#81#82#83#84#85#86#87#88#89#8A#8B#8C#8D#8E#8F%
    #90#91#92#93#94#95#96#97#98#99#9A#9B#9C#9D#9E#9F%
    #A0#A1#A2#A3#A4#A5#A6#A7#A8#A9#AA#AB#AC#AD#AE#AF%
    #B0#B1#B2#B3#B4#B5#B6#B7#B8#B9#BA#BB#BC#BD#BE#BF%
    #C0#C1#C2#C3#C4#C5#C6#C7#C8#C9#CA#CB#CC#CD#CE#CF%
    #D0#D1#D2#D3#D4#D5#D6#D7#D8#D9#DA#DB#DC#DD#DE#DF%
    #E0#E1#E2#E3#E4#E5#E6#E7#E8#E9#EA#EB#EC#ED#EE#EF%
    #F0#F1#F2#F3#F4#F5#F6#F7#F8#F9#FA#FB#FC#FD#FE#FF%
  }%
\endgroup
\ltx@onelevel@sanitize\AllBytesName
\edef\AllBytesFromName{\expandafter\ltx@gobble\AllBytes}
\begingroup
  \def\|{|}%
  \edef\%{\ltx@percentchar}%
  \catcode`\|=0 %
  \catcode`\#=12 %
  \catcode`\~=12 %
  \catcode`\\=12 %
  |xdef|AllBytesString{%
    \000\001\002\003\004\005\006\007\010\011\012\013\014\015\016\017%
    \020\021\022\023\024\025\026\027\030\031\032\033\034\035\036\037%
    \040!"#$|%&'\(\)*+,-./%
    0123456789:;<=>?%
    @ABCDEFGHIJKLMNO%
    PQRSTUVWXYZ[\\]^_%
    `abcdefghijklmno%
    pqrstuvwxyz{||}~\177%
    \200\201\202\203\204\205\206\207\210\211\212\213\214\215\216\217%
    \220\221\222\223\224\225\226\227\230\231\232\233\234\235\236\237%
    \240\241\242\243\244\245\246\247\250\251\252\253\254\255\256\257%
    \260\261\262\263\264\265\266\267\270\271\272\273\274\275\276\277%
    \300\301\302\303\304\305\306\307\310\311\312\313\314\315\316\317%
    \320\321\322\323\324\325\326\327\330\331\332\333\334\335\336\337%
    \340\341\342\343\344\345\346\347\350\351\352\353\354\355\356\357%
    \360\361\362\363\364\365\366\367\370\371\372\373\374\375\376\377%
  }%
|endgroup
\ltx@onelevel@sanitize\AllBytesString
%    \end{macrocode}
%    \begin{macrocode}
\def\Test#1#2#3{%
  \begingroup
    \expandafter\expandafter\expandafter\def
    \expandafter\expandafter\expandafter\TestResult
    \expandafter\expandafter\expandafter{%
      #1{#2}%
    }%
    \ifx\TestResult#3%
    \else
      \newlinechar=10 %
      \msg{Expect:^^J#3}%
      \msg{Result:^^J\TestResult}%
      \errmessage{\string#2 -\string#1-> \string#3}%
    \fi
  \endgroup
}
\def\test#1#2#3{%
  \edef\TestFrom{#2}%
  \edef\TestExpect{#3}%
  \ltx@onelevel@sanitize\TestExpect
  \Test#1\TestFrom\TestExpect
}
\test\pdf@unescapehex{74657374}{test}
\begingroup
  \catcode0=12 %
  \catcode1=12 %
  \test\pdf@unescapehex{740074017400740174}{t^^@t^^At^^@t^^At}%
\endgroup
\Test\pdf@escapehex\AllBytes\AllBytesHex
\Test\pdf@unescapehex\AllBytesHex\AllBytes
\Test\pdf@escapename\AllBytes\AllBytesName
\Test\pdf@escapestring\AllBytes\AllBytesString
%    \end{macrocode}
%    \begin{macrocode}
\csname @@end\endcsname\end
%</test-escape>
%    \end{macrocode}
%
% \section{Installation}
%
% \subsection{Download}
%
% \paragraph{Package.} This package is available on
% CTAN\footnote{\CTANpkg{pdftexcmds}}:
% \begin{description}
% \item[\CTAN{macros/latex/contrib/oberdiek/pdftexcmds.dtx}] The source file.
% \item[\CTAN{macros/latex/contrib/oberdiek/pdftexcmds.pdf}] Documentation.
% \end{description}
%
%
% \paragraph{Bundle.} All the packages of the bundle `oberdiek'
% are also available in a TDS compliant ZIP archive. There
% the packages are already unpacked and the documentation files
% are generated. The files and directories obey the TDS standard.
% \begin{description}
% \item[\CTANinstall{install/macros/latex/contrib/oberdiek.tds.zip}]
% \end{description}
% \emph{TDS} refers to the standard ``A Directory Structure
% for \TeX\ Files'' (\CTAN{tds/tds.pdf}). Directories
% with \xfile{texmf} in their name are usually organized this way.
%
% \subsection{Bundle installation}
%
% \paragraph{Unpacking.} Unpack the \xfile{oberdiek.tds.zip} in the
% TDS tree (also known as \xfile{texmf} tree) of your choice.
% Example (linux):
% \begin{quote}
%   |unzip oberdiek.tds.zip -d ~/texmf|
% \end{quote}
%
% \paragraph{Script installation.}
% Check the directory \xfile{TDS:scripts/oberdiek/} for
% scripts that need further installation steps.
% Package \xpackage{attachfile2} comes with the Perl script
% \xfile{pdfatfi.pl} that should be installed in such a way
% that it can be called as \texttt{pdfatfi}.
% Example (linux):
% \begin{quote}
%   |chmod +x scripts/oberdiek/pdfatfi.pl|\\
%   |cp scripts/oberdiek/pdfatfi.pl /usr/local/bin/|
% \end{quote}
%
% \subsection{Package installation}
%
% \paragraph{Unpacking.} The \xfile{.dtx} file is a self-extracting
% \docstrip\ archive. The files are extracted by running the
% \xfile{.dtx} through \plainTeX:
% \begin{quote}
%   \verb|tex pdftexcmds.dtx|
% \end{quote}
%
% \paragraph{TDS.} Now the different files must be moved into
% the different directories in your installation TDS tree
% (also known as \xfile{texmf} tree):
% \begin{quote}
% \def\t{^^A
% \begin{tabular}{@{}>{\ttfamily}l@{ $\rightarrow$ }>{\ttfamily}l@{}}
%   pdftexcmds.sty & tex/generic/oberdiek/pdftexcmds.sty\\
%   oberdiek.pdftexcmds.lua & scripts/oberdiek/oberdiek.pdftexcmds.lua\\
%   pdftexcmds.lua & scripts/oberdiek/pdftexcmds.lua\\
%   pdftexcmds.pdf & doc/latex/oberdiek/pdftexcmds.pdf\\
%   test/pdftexcmds-test1.tex & doc/latex/oberdiek/test/pdftexcmds-test1.tex\\
%   test/pdftexcmds-test2.tex & doc/latex/oberdiek/test/pdftexcmds-test2.tex\\
%   test/pdftexcmds-test-shell.tex & doc/latex/oberdiek/test/pdftexcmds-test-shell.tex\\
%   test/pdftexcmds-test-escape.tex & doc/latex/oberdiek/test/pdftexcmds-test-escape.tex\\
%   pdftexcmds.dtx & source/latex/oberdiek/pdftexcmds.dtx\\
% \end{tabular}^^A
% }^^A
% \sbox0{\t}^^A
% \ifdim\wd0>\linewidth
%   \begingroup
%     \advance\linewidth by\leftmargin
%     \advance\linewidth by\rightmargin
%   \edef\x{\endgroup
%     \def\noexpand\lw{\the\linewidth}^^A
%   }\x
%   \def\lwbox{^^A
%     \leavevmode
%     \hbox to \linewidth{^^A
%       \kern-\leftmargin\relax
%       \hss
%       \usebox0
%       \hss
%       \kern-\rightmargin\relax
%     }^^A
%   }^^A
%   \ifdim\wd0>\lw
%     \sbox0{\small\t}^^A
%     \ifdim\wd0>\linewidth
%       \ifdim\wd0>\lw
%         \sbox0{\footnotesize\t}^^A
%         \ifdim\wd0>\linewidth
%           \ifdim\wd0>\lw
%             \sbox0{\scriptsize\t}^^A
%             \ifdim\wd0>\linewidth
%               \ifdim\wd0>\lw
%                 \sbox0{\tiny\t}^^A
%                 \ifdim\wd0>\linewidth
%                   \lwbox
%                 \else
%                   \usebox0
%                 \fi
%               \else
%                 \lwbox
%               \fi
%             \else
%               \usebox0
%             \fi
%           \else
%             \lwbox
%           \fi
%         \else
%           \usebox0
%         \fi
%       \else
%         \lwbox
%       \fi
%     \else
%       \usebox0
%     \fi
%   \else
%     \lwbox
%   \fi
% \else
%   \usebox0
% \fi
% \end{quote}
% If you have a \xfile{docstrip.cfg} that configures and enables \docstrip's
% TDS installing feature, then some files can already be in the right
% place, see the documentation of \docstrip.
%
% \subsection{Refresh file name databases}
%
% If your \TeX~distribution
% (\teTeX, \mikTeX, \dots) relies on file name databases, you must refresh
% these. For example, \teTeX\ users run \verb|texhash| or
% \verb|mktexlsr|.
%
% \subsection{Some details for the interested}
%
% \paragraph{Unpacking with \LaTeX.}
% The \xfile{.dtx} chooses its action depending on the format:
% \begin{description}
% \item[\plainTeX:] Run \docstrip\ and extract the files.
% \item[\LaTeX:] Generate the documentation.
% \end{description}
% If you insist on using \LaTeX\ for \docstrip\ (really,
% \docstrip\ does not need \LaTeX), then inform the autodetect routine
% about your intention:
% \begin{quote}
%   \verb|latex \let\install=y\input{pdftexcmds.dtx}|
% \end{quote}
% Do not forget to quote the argument according to the demands
% of your shell.
%
% \paragraph{Generating the documentation.}
% You can use both the \xfile{.dtx} or the \xfile{.drv} to generate
% the documentation. The process can be configured by the
% configuration file \xfile{ltxdoc.cfg}. For instance, put this
% line into this file, if you want to have A4 as paper format:
% \begin{quote}
%   \verb|\PassOptionsToClass{a4paper}{article}|
% \end{quote}
% An example follows how to generate the
% documentation with pdf\LaTeX:
% \begin{quote}
%\begin{verbatim}
%pdflatex pdftexcmds.dtx
%bibtex pdftexcmds.aux
%makeindex -s gind.ist pdftexcmds.idx
%pdflatex pdftexcmds.dtx
%makeindex -s gind.ist pdftexcmds.idx
%pdflatex pdftexcmds.dtx
%\end{verbatim}
% \end{quote}
%
% \printbibliography[
%   heading=bibnumbered,
% ]
%
% \begin{History}
%   \begin{Version}{2007/11/11 v0.1}
%   \item
%     First version.
%   \end{Version}
%   \begin{Version}{2007/11/12 v0.2}
%   \item
%     Short description fixed.
%   \end{Version}
%   \begin{Version}{2007/12/12 v0.3}
%   \item
%     Organization of Lua code as module.
%   \end{Version}
%   \begin{Version}{2009/04/10 v0.4}
%   \item
%     Adaptation for syntax change of \cs{directlua} in
%     \hologo{LuaTeX} 0.36.
%   \end{Version}
%   \begin{Version}{2009/09/22 v0.5}
%   \item
%     \cs{pdf@primitive}, \cs{pdf@ifprimitive} added.
%   \item
%     \hologo{XeTeX}'s variants are detected for
%     \cs{pdf@shellescape}, \cs{pdf@strcmp}, \cs{pdf@primitive},
%     \cs{pdf@ifprimitive}.
%   \end{Version}
%   \begin{Version}{2009/09/23 v0.6}
%   \item
%     Macro \cs{pdf@isprimitive} added.
%   \end{Version}
%   \begin{Version}{2009/12/12 v0.7}
%   \item
%     Short info shortened.
%   \end{Version}
%   \begin{Version}{2010/03/01 v0.8}
%   \item
%     Required date for package \xpackage{ifluatex} updated.
%   \end{Version}
%   \begin{Version}{2010/04/01 v0.9}
%   \item
%     Use \cs{ifeof18} for defining \cs{pdf@shellescape} between
%     \hologo{pdfTeX} 1.21a (inclusive) and 1.30.0 (exclusive).
%   \end{Version}
%   \begin{Version}{2010/11/04 v0.10}
%   \item
%     \cs{pdf@draftmode}, \cs{pdf@ifdraftmode} and
%     \cs{pdf@setdraftmode} added.
%   \end{Version}
%   \begin{Version}{2010/11/11 v0.11}
%   \item
%     Missing \cs{RequirePackage} for package \xpackage{ifpdf} added.
%   \end{Version}
%   \begin{Version}{2011/01/30 v0.12}
%   \item
%     Already loaded package files are not input in \hologo{plainTeX}.
%   \end{Version}
%   \begin{Version}{2011/03/04 v0.13}
%   \item
%     Improved Lua function \texttt{shellescape} that also
%     uses the result of \texttt{os.execute()} (thanks to Philipp Stephani).
%   \end{Version}
%   \begin{Version}{2011/04/10 v0.14}
%   \item
%     Version check of loaded module added.
%   \item
%     Patch for bug in \hologo{LuaTeX} between 0.40.6 and 0.65 that
%     is fixed in revision 4096.
%   \end{Version}
%   \begin{Version}{2011/04/16 v0.15}
%   \item
%     \hologo{LuaTeX}: \cs{pdf@shellescape} is only supported
%     for version 0.70.0 and higher due to a bug, \texttt{os.execute()}
%     crashes in some circumstances. Fixed in \hologo{LuaTeX}
%     beta-0.70.0, revision 4167.
%   \end{Version}
%   \begin{Version}{2011/04/22 v0.16}
%   \item
%     Previous fix was not working due to a wrong catcode of digit
%     zero (due to easily support the old \cs{directlua0}).
%     The version border is lowered to 0.68, because some
%     beta-0.67.0 seems also to work.
%   \end{Version}
%   \begin{Version}{2011/06/29 v0.17}
%   \item
%     Documentation addition to \cs{pdf@shellescape}.
%   \end{Version}
%   \begin{Version}{2011/07/01 v0.18}
%   \item
%     Add Lua module loading in \cs{everyjob} for \hologo{iniTeX}
%     (\hologo{LuaTeX} only).
%   \end{Version}
%   \begin{Version}{2011/07/28 v0.19}
%   \item
%     Missing space in an info message added (Martin M\"unch).
%   \end{Version}
%   \begin{Version}{2011/11/29 v0.20}
%   \item
%     \cs{pdf@resettimer} and \cs{pdf@elapsedtime} added
%     (thanks Andy Thomas).
%   \end{Version}
%   \begin{Version}{2016/05/10 v0.21}
%   \item
%      local unpack added
%     (thanks \'{E}lie Roux).
%   \end{Version}
%   \begin{Version}{2016/05/21 v0.22}
%   \item
%     adjust \cs{textbackslas}h usage in bib file for biber bug.
%   \end{Version}
%   \begin{Version}{2016/10/02 v0.23}
%   \item
%     add file.close to lua filehandles (github pull request).
%   \end{Version}
%   \begin{Version}{2017/01/29 v0.24}
%   \item
%     Avoid loading luatex-loader for current luatex. (Use
%     pdftexcmds.lua not oberdiek.pdftexcmds.lua to simplify file
%     search with standard require)
%   \end{Version}
%   \begin{Version}{2017/03/19 v0.25}
%   \item
%     New \cs{pdf@shellescape} for Lua\TeX, see github issue 20.
%   \end{Version}
%   \begin{Version}{2018/01/21 v0.26}
%   \item
%     use rb not r mode for file open github issue 34.
%   \end{Version}
%   \begin{Version}{2018/01/30 v0.27}
%   \item
%     \cs{pdf@mdfivesum} for \hologo{XeTeX}
%   \end{Version}
%   \begin{Version}{2018/09/07 v0.28}
%   \item
%     Fix catcode regime in luatex sprint for \cs{pdf@shellescape} GH issue 45
%   \end{Version}
%   \begin{Version}{2018/09/10 v0.29}
%   \item
%     Actually do the fix described above in the code, not just document it.
%   \end{Version}
%   \begin{Version}{2019/07/25 v0.30}
%   \item
%     remove uses of module function, see PR70
%   \end{Version}
% \end{History}
%
% \PrintIndex
%
% \Finale
\endinput

%        (quote the arguments according to the demands of your shell)
%
% Documentation:
%    (a) If pdftexcmds.drv is present:
%           latex pdftexcmds.drv
%    (b) Without pdftexcmds.drv:
%           latex pdftexcmds.dtx; ...
%    The class ltxdoc loads the configuration file ltxdoc.cfg
%    if available. Here you can specify further options, e.g.
%    use A4 as paper format:
%       \PassOptionsToClass{a4paper}{article}
%
%    Programm calls to get the documentation (example):
%       pdflatex pdftexcmds.dtx
%       bibtex pdftexcmds.aux
%       makeindex -s gind.ist pdftexcmds.idx
%       pdflatex pdftexcmds.dtx
%       makeindex -s gind.ist pdftexcmds.idx
%       pdflatex pdftexcmds.dtx
%
% Installation:
%    TDS:tex/generic/oberdiek/pdftexcmds.sty
%    TDS:scripts/oberdiek/oberdiek.pdftexcmds.lua
%    TDS:scripts/oberdiek/pdftexcmds.lua
%    TDS:doc/latex/oberdiek/pdftexcmds.pdf
%    TDS:doc/latex/oberdiek/test/pdftexcmds-test1.tex
%    TDS:doc/latex/oberdiek/test/pdftexcmds-test2.tex
%    TDS:doc/latex/oberdiek/test/pdftexcmds-test-shell.tex
%    TDS:doc/latex/oberdiek/test/pdftexcmds-test-escape.tex
%    TDS:source/latex/oberdiek/pdftexcmds.dtx
%
%<*ignore>
\begingroup
  \catcode123=1 %
  \catcode125=2 %
  \def\x{LaTeX2e}%
\expandafter\endgroup
\ifcase 0\ifx\install y1\fi\expandafter
         \ifx\csname processbatchFile\endcsname\relax\else1\fi
         \ifx\fmtname\x\else 1\fi\relax
\else\csname fi\endcsname
%</ignore>
%<*install>
\input docstrip.tex
\Msg{************************************************************************}
\Msg{* Installation}
\Msg{* Package: pdftexcmds 2019/07/25 v0.30 Utility functions of pdfTeX for LuaTeX (HO)}
\Msg{************************************************************************}

\keepsilent
\askforoverwritefalse

\let\MetaPrefix\relax
\preamble

This is a generated file.

Project: pdftexcmds
Version: 2019/07/25 v0.30

Copyright (C) 2007, 2009-2011 by
   Heiko Oberdiek <heiko.oberdiek at googlemail.com>

This work may be distributed and/or modified under the
conditions of the LaTeX Project Public License, either
version 1.3c of this license or (at your option) any later
version. This version of this license is in
   https://www.latex-project.org/lppl/lppl-1-3c.txt
and the latest version of this license is in
   https://www.latex-project.org/lppl.txt
and version 1.3 or later is part of all distributions of
LaTeX version 2005/12/01 or later.

This work has the LPPL maintenance status "maintained".

The Current Maintainers of this work are
Heiko Oberdiek and the Oberdiek Package Support Group
https://github.com/ho-tex/oberdiek/issues


The Base Interpreter refers to any `TeX-Format',
because some files are installed in TDS:tex/generic//.

This work consists of the main source file pdftexcmds.dtx
and the derived files
   pdftexcmds.sty, pdftexcmds.pdf, pdftexcmds.ins, pdftexcmds.drv,
   pdftexcmds.bib, pdftexcmds-test1.tex, pdftexcmds-test2.tex,
   pdftexcmds-test-shell.tex, pdftexcmds-test-escape.tex,
   oberdiek.pdftexcmds.lua, pdftexcmds.lua.

\endpreamble
\let\MetaPrefix\DoubleperCent

\generate{%
  \file{pdftexcmds.ins}{\from{pdftexcmds.dtx}{install}}%
  \file{pdftexcmds.drv}{\from{pdftexcmds.dtx}{driver}}%
  \nopreamble
  \nopostamble
  \file{pdftexcmds.bib}{\from{pdftexcmds.dtx}{bib}}%
  \usepreamble\defaultpreamble
  \usepostamble\defaultpostamble
  \usedir{tex/generic/oberdiek}%
  \file{pdftexcmds.sty}{\from{pdftexcmds.dtx}{package}}%
%  \usedir{doc/latex/oberdiek/test}%
%  \file{pdftexcmds-test1.tex}{\from{pdftexcmds.dtx}{test1}}%
%  \file{pdftexcmds-test2.tex}{\from{pdftexcmds.dtx}{test2}}%
%  \file{pdftexcmds-test-shell.tex}{\from{pdftexcmds.dtx}{test-shell}}%
%  \file{pdftexcmds-test-escape.tex}{\from{pdftexcmds.dtx}{test-escape}}%
  \nopreamble
  \nopostamble
%  \usedir{source/latex/oberdiek/catalogue}%
%  \file{pdftexcmds.xml}{\from{pdftexcmds.dtx}{catalogue}}%
}
\def\MetaPrefix{-- }
\def\defaultpostamble{%
  \MetaPrefix^^J%
  \MetaPrefix\space End of File `\outFileName'.%
}
\def\currentpostamble{\defaultpostamble}%
\generate{%
  \usedir{scripts/oberdiek}%
  \file{oberdiek.pdftexcmds.lua}{\from{pdftexcmds.dtx}{lua}}%
  \file{pdftexcmds.lua}{\from{pdftexcmds.dtx}{lua}}%
}

\catcode32=13\relax% active space
\let =\space%
\Msg{************************************************************************}
\Msg{*}
\Msg{* To finish the installation you have to move the following}
\Msg{* file into a directory searched by TeX:}
\Msg{*}
\Msg{*     pdftexcmds.sty}
\Msg{*}
\Msg{* And install the following script files:}
\Msg{*}
\Msg{*     oberdiek.pdftexcmds.lua, pdftexcmds.lua}
\Msg{*}
\Msg{* To produce the documentation run the file `pdftexcmds.drv'}
\Msg{* through LaTeX.}
\Msg{*}
\Msg{* Happy TeXing!}
\Msg{*}
\Msg{************************************************************************}

\endbatchfile
%</install>
%<*bib>
@online{AndyThomas:Analog,
  author={Thomas, Andy},
  title={Analog of {\texttt{\csname textbackslash\endcsname}pdfelapsedtime} for
      {\hologo{LuaTeX}} and {\hologo{XeTeX}}},
  url={http://tex.stackexchange.com/a/32531},
  urldate={2011-11-29},
}
%</bib>
%<*ignore>
\fi
%</ignore>
%<*driver>
\NeedsTeXFormat{LaTeX2e}
\ProvidesFile{pdftexcmds.drv}%
  [2019/07/25 v0.30 Utility functions of pdfTeX for LuaTeX (HO)]%
\documentclass{ltxdoc}
\usepackage{holtxdoc}[2011/11/22]
\usepackage{paralist}
\usepackage{csquotes}
\usepackage[
  backend=bibtex,
  bibencoding=ascii,
  alldates=iso8601,
]{biblatex}[2011/11/13]
\bibliography{oberdiek-source}
\bibliography{pdftexcmds}
\begin{document}
  \DocInput{pdftexcmds.dtx}%
\end{document}
%</driver>
% \fi
%
%
% \CharacterTable
%  {Upper-case    \A\B\C\D\E\F\G\H\I\J\K\L\M\N\O\P\Q\R\S\T\U\V\W\X\Y\Z
%   Lower-case    \a\b\c\d\e\f\g\h\i\j\k\l\m\n\o\p\q\r\s\t\u\v\w\x\y\z
%   Digits        \0\1\2\3\4\5\6\7\8\9
%   Exclamation   \!     Double quote  \"     Hash (number) \#
%   Dollar        \$     Percent       \%     Ampersand     \&
%   Acute accent  \'     Left paren    \(     Right paren   \)
%   Asterisk      \*     Plus          \+     Comma         \,
%   Minus         \-     Point         \.     Solidus       \/
%   Colon         \:     Semicolon     \;     Less than     \<
%   Equals        \=     Greater than  \>     Question mark \?
%   Commercial at \@     Left bracket  \[     Backslash     \\
%   Right bracket \]     Circumflex    \^     Underscore    \_
%   Grave accent  \`     Left brace    \{     Vertical bar  \|
%   Right brace   \}     Tilde         \~}
%
% \GetFileInfo{pdftexcmds.drv}
%
% \title{The \xpackage{pdftexcmds} package}
% \date{2019/07/25 v0.30}
% \author{Heiko Oberdiek\thanks
% {Please report any issues at \url{https://github.com/ho-tex/oberdiek/issues}}}
%
% \maketitle
%
% \begin{abstract}
% \hologo{LuaTeX} provides most of the commands of \hologo{pdfTeX} 1.40. However
% a number of utility functions are removed. This package tries to fill
% the gap and implements some of the missing primitive using Lua.
% \end{abstract}
%
% \tableofcontents
%
% \def\csi#1{\texttt{\textbackslash\textit{#1}}}
%
% \section{Documentation}
%
% Some primitives of \hologo{pdfTeX} \cite{pdftex-manual}
% are not defined by \hologo{LuaTeX} \cite{luatex-manual}.
% This package implements macro based solutions using Lua code
% for the following missing \hologo{pdfTeX} primitives;
% \begin{compactitem}
% \item \cs{pdfstrcmp}
% \item \cs{pdfunescapehex}
% \item \cs{pdfescapehex}
% \item \cs{pdfescapename}
% \item \cs{pdfescapestring}
% \item \cs{pdffilesize}
% \item \cs{pdffilemoddate}
% \item \cs{pdffiledump}
% \item \cs{pdfmdfivesum}
% \item \cs{pdfresettimer}
% \item \cs{pdfelapsedtime}
% \item |\immediate\write18|
% \end{compactitem}
% The original names of the primitives cannot be used:
% \begin{itemize}
% \item
% The syntax for their arguments cannot easily
% simulated by macros. The primitives using key words
% such as |file| (\cs{pdfmdfivesum}) or |offset| and |length|
% (\cs{pdffiledump}) and uses \meta{general text} for the other
% arguments. Using token registers assignments, \meta{general text} could
% be catched. However, the simulated primitives are expandable
% and register assignments would destroy this important property.
% (\meta{general text} allows something like |\expandafter\bgroup ...}|.)
% \item
% The original primitives can be expanded using one expansion step.
% The new macros need two expansion steps because of the additional
% macro expansion. Example:
% \begin{quote}
%   |\expandafter\foo\pdffilemoddate{file}|\\
%   vs.\\
%   |\expandafter\expandafter\expandafter|\\
%   |\foo\pdf@filemoddate{file}|
% \end{quote}
% \end{itemize}
%
% \hologo{LuaTeX} isn't stable yet and thus the status of this package is
% \emph{experimental}. Feedback is welcome.
%
% \subsection{General principles}
%
% \begin{description}
% \item[Naming convention:]
%   Usually this package defines a macro |\pdf@|\meta{cmd} if
%   \hologo{pdfTeX} provides |\pdf|\meta{cmd}.
% \item[Arguments:] The order of arguments in |\pdf@|\meta{cmd}
%   is the same as for the corresponding primitive of \hologo{pdfTeX}.
%   The arguments are ordinary undelimited \hologo{TeX} arguments,
%   no \meta{general text} and without additional keywords.
% \item[Expandibility:]
%   The macro |\pdf@|\meta{cmd} is expandable if the
%   corresponding \hologo{pdfTeX} primitive has this property.
%   Exact two expansion steps are necessary (first is the macro
%   expansion) except for \cs{pdf@primitive} and \cs{pdf@ifprimitive}.
%   The latter ones are not macros, but have the direct meaning of the
%   primitive.
% \item[Without \hologo{LuaTeX}:]
%   The macros |\pdf@|\meta{cmd} are mapped to the commands
%   of \hologo{pdfTeX} if they are available. Otherwise they are undefined.
% \item[Availability:]
%   The macros that the packages provides are undefined, if
%   the necessary primitives are not found and cannot be
%   implemented by Lua.
% \end{description}
%
% \subsection{Macros}
%
% \subsubsection[Strings]{Strings \cite[``7.15 Strings'']{pdftex-manual}}
%
% \begin{declcs}{pdf@strcmp} \M{stringA} \M{stringB}
% \end{declcs}
% Same as |\pdfstrcmp{|\meta{stringA}|}{|\meta{stringB}|}|.
%
% \begin{declcs}{pdf@unescapehex} \M{string}
% \end{declcs}
% Same as |\pdfunescapehex{|\meta{string}|}|.
% The argument is a byte string given in hexadecimal notation.
% The result are character tokens from 0 until 255 with
% catcode 12 and the space with catcode 10.
%
% \begin{declcs}{pdf@escapehex} \M{string}\\
%   \cs{pdf@escapestring} \M{string}\\
%   \cs{pdf@escapename} \M{string}
% \end{declcs}
% Same as the primitives of \hologo{pdfTeX}. However \hologo{pdfTeX} does not
% know about characters with codes 256 and larger. Thus the
% string is treated as byte string, characters with more than
% eight bits are ignored.
%
% \subsubsection[Files]{Files \cite[``7.18 Files'']{pdftex-manual}}
%
% \begin{declcs}{pdf@filesize} \M{filename}
% \end{declcs}
% Same as |\pdffilesize{|\meta{filename}|}|.
%
% \begin{declcs}{pdf@filemoddate} \M{filename}
% \end{declcs}
% Same as |\pdffilemoddate{|\meta{filename}|}|.
%
% \begin{declcs}{pdf@filedump} \M{offset} \M{length} \M{filename}
% \end{declcs}
% Same as |\pdffiledump offset| \meta{offset} |length| \meta{length}
% |{|\meta{filename}|}|. Both \meta{offset} and \meta{length} must
% not be empty, but must be a valid \hologo{TeX} number.
%
% \begin{declcs}{pdf@mdfivesum} \M{string}
% \end{declcs}
% Same as |\pdfmdfivesum{|\meta{string}|}|. Keyword |file| is supported
% by macro \cs{pdf@filemdfivesum}.
%
% \begin{declcs}{pdf@filemdfivesum} \M{filename}
% \end{declcs}
% Same as |\pdfmdfivesum file{|\meta{filename}|}|.
%
% \subsubsection[Timekeeping]{Timekeeping \cite[``7.17 Timekeeping'']{pdftex-manual}}
%
% The timekeeping macros are based on Andy Thomas' work \cite{AndyThomas:Analog}.
%
% \begin{declcs}{pdf@resettimer}
% \end{declcs}
% Same as \cs{pdfresettimer}, it resets the internal timer.
%
% \begin{declcs}{pdf@elapsedtime}
% \end{declcs}
% Same as \cs{pdfelapsedtime}. It behaves like a read-only integer.
% For printing purposes it can be prefixed by \cs{the} or \cs{number}.
% It measures the time in scaled seconds (seconds multiplied with 65536)
% since the latest call of \cs{pdf@resettimer} or start of
% program/package. The resolution, the shortest time interval that
% can be measured, depends on the program and system.
% \begin{itemize}
% \item \hologo{pdfTeX} with |gettimeofday|: $\ge$ 1/65536\,s
% \item \hologo{pdfTeX} with |ftime|: $\ge$ 1\,ms
% \item \hologo{pdfTeX} with |time|: $\ge$ 1\,s
% \item \hologo{LuaTeX}: $\ge$ 10\,ms\\
%  (|os.clock()| returns a float number with two decimal digits in
%  \hologo{LuaTeX} beta-0.70.1-2011061416 (rev 4277)).
% \end{itemize}
%
% \subsubsection[Miscellaneous]{Miscellaneous \cite[``7.21 Miscellaneous'']{pdftex-manual}}
%
% \begin{declcs}{pdf@draftmode}
% \end{declcs}
% If the \TeX\ compiler knows \cs{pdfdraftmode} or \cs{draftmode}
% (\hologo{pdfTeX},
% \hologo{LuaTeX}), then \cs{pdf@draftmode} returns, whether
% this mode is enabled. The result is an implicit number:
% one means the draft mode is available and enabled.
% If the value is zero, then the mode is not active or
% \cs{pdfdraftmode} is not available.
% An explicit number is yielded by \cs{number}\cs{pdf@draftmode}.
% The macro cannot
% be used to change the mode, see \cs{pdf@setdraftmode}.
%
% \begin{declcs}{pdf@ifdraftmode} \M{true} \M{false}
% \end{declcs}
% If \cs{pdfdraftmode} is available and enabled, \meta{true} is
% called, otherwise \meta{false} is executed.
%
% \begin{declcs}{pdf@setdraftmode} \M{value}
% \end{declcs}
% Macro \cs{pdf@setdraftmode} expects the number zero or one as
% \meta{value}. Zero deactivates the mode and one enables the draft mode.
% The macro does not have an effect, if the feature \cs{pdfdraftmode} is not
% available.
%
% \begin{declcs}{pdf@shellescape}
% \end{declcs}
% Same as |\pdfshellescape|. It is or expands to |1| if external
% commands can be executed and |0| otherwise. In \hologo{pdfTeX} external
% commands must be enabled first by command line option or
% configuration option. In \hologo{LuaTeX} option |--safer| disables
% the execution of external commands.
%
% In \hologo{LuaTeX} before 0.68.0 \cs{pdf@shellescape} is not
% available due to a bug in |os.execute()|. The argumentless form
% crashes in some circumstances with segmentation fault.
% (It is fixed in version 0.68.0 or revision 4167 of \hologo{LuaTeX}.
% and packported to some version of 0.67.0).
%
% Hints for usage:
% \begin{itemize}
% \item Before its use \cs{pdf@shellescape} should be tested,
% whether it is available. Example with package \xpackage{ltxcmds}
% (loaded by package \xpackage{pdftexcmds}):
%\begin{quote}
%\begin{verbatim}
%\ltx@IfUndefined{pdf@shellescape}{%
%  % \pdf@shellescape is undefined
%}{%
%  % \pdf@shellescape is available
%}
%\end{verbatim}
%\end{quote}
% Use \cs{ltx@ifundefined} in expandable contexts.
% \item \cs{pdf@shellescape} might be a numerical constant,
% expands to the primitive, or expands to a plain number.
% Therefore use it in contexts where these differences does not matter.
% \item Use in comparisons, e.g.:
%   \begin{quote}
%     |\ifnum\pdf@shellescape=0 ...|
%   \end{quote}
% \item Print the number: |\number\pdf@shellescape|
% \end{itemize}
%
% \begin{declcs}{pdf@system} \M{cmdline}
% \end{declcs}
% It is a wrapper for |\immediate\write18| in \hologo{pdfTeX} or
% |os.execute| in \hologo{LuaTeX}.
%
% In theory |os.execute|
% returns a status number. But its meaning is quite
% undefined. Are there some reliable properties?
% Does it make sense to provide an user interface to
% this status exit code?
%
% \begin{declcs}{pdf@primitive} \csi{cmd}
% \end{declcs}
% Same as \cs{pdfprimitive} in \hologo{pdfTeX} or \hologo{LuaTeX}.
% In \hologo{XeTeX} the
% primitive is called \cs{primitive}. Despite the current definition
% of the command \csi{cmd}, it's meaning as primitive is used.
%
% \begin{declcs}{pdf@ifprimitive} \csi{cmd}
% \end{declcs}
% Same as \cs{ifpdfprimitive} in \hologo{pdfTeX} or
% \hologo{LuaTeX}. \hologo{XeTeX} calls
% it \cs{ifprimitive}. It is a switch that checks if the command
% \csi{cmd} has it's primitive meaning.
%
% \subsubsection{Additional macro: \cs{pdf@isprimitive}}
%
% \begin{declcs}{pdf@isprimitive} \csi{cmd1} \csi{cmd2} \M{true} \M{false}
% \end{declcs}
% If \csi{cmd1} has the primitive meaning given by the primitive name
% of \csi{cmd2}, then the argument \meta{true} is executed, otherwise
% \meta{false}. The macro \cs{pdf@isprimitive} is expandable.
% Internally it checks the result of \cs{meaning} and is therefore
% available for all \hologo{TeX} variants, even the original \hologo{TeX}.
% Example with \hologo{LaTeX}:
%\begin{quote}
%\begin{verbatim}
%\makeatletter
%\pdf@isprimitive{@@input}{input}{%
%  \typeout{\string\@@input\space is original\string\input}%
%}{%
%  \typeout{Oops, \string\@@input\space is not the %
%           original\string\input}%
%}
%\end{verbatim}
%\end{quote}
%
% \subsubsection{Experimental}
%
% \begin{declcs}{pdf@unescapehexnative} \M{string}\\
%   \cs{pdf@escapehexnative} \M{string}\\
%   \cs{pdf@escapenamenative} \M{string}\\
%   \cs{pdf@mdfivesumnative} \M{string}
% \end{declcs}
% The variants without |native| in the macro name are supposed to
% be compatible with \hologo{pdfTeX}. However characters with more than
% eight bits are not supported and are ignored. If \hologo{LuaTeX} is
% running, then its UTF-8 coded strings are used. Thus the full
% unicode character range is supported. However the result
% differs from \hologo{pdfTeX} for characters with eight or more bits.
%
% \begin{declcs}{pdf@pipe} \M{cmdline}
% \end{declcs}
% It calls \meta{cmdline} and returns the output of the external
% program in the usual manner as byte string (catcode 12, space with
% catcode 10). The Lua documentation says, that the used |io.popen|
% may not be available on all platforms. Then macro \cs{pdf@pipe}
% is undefined.
%
% \StopEventually{
% }
%
% \section{Implementation}
%
%    \begin{macrocode}
%<*package>
%    \end{macrocode}
%
% \subsection{Reload check and package identification}
%    Reload check, especially if the package is not used with \LaTeX.
%    \begin{macrocode}
\begingroup\catcode61\catcode48\catcode32=10\relax%
  \catcode13=5 % ^^M
  \endlinechar=13 %
  \catcode35=6 % #
  \catcode39=12 % '
  \catcode44=12 % ,
  \catcode45=12 % -
  \catcode46=12 % .
  \catcode58=12 % :
  \catcode64=11 % @
  \catcode123=1 % {
  \catcode125=2 % }
  \expandafter\let\expandafter\x\csname ver@pdftexcmds.sty\endcsname
  \ifx\x\relax % plain-TeX, first loading
  \else
    \def\empty{}%
    \ifx\x\empty % LaTeX, first loading,
      % variable is initialized, but \ProvidesPackage not yet seen
    \else
      \expandafter\ifx\csname PackageInfo\endcsname\relax
        \def\x#1#2{%
          \immediate\write-1{Package #1 Info: #2.}%
        }%
      \else
        \def\x#1#2{\PackageInfo{#1}{#2, stopped}}%
      \fi
      \x{pdftexcmds}{The package is already loaded}%
      \aftergroup\endinput
    \fi
  \fi
\endgroup%
%    \end{macrocode}
%    Package identification:
%    \begin{macrocode}
\begingroup\catcode61\catcode48\catcode32=10\relax%
  \catcode13=5 % ^^M
  \endlinechar=13 %
  \catcode35=6 % #
  \catcode39=12 % '
  \catcode40=12 % (
  \catcode41=12 % )
  \catcode44=12 % ,
  \catcode45=12 % -
  \catcode46=12 % .
  \catcode47=12 % /
  \catcode58=12 % :
  \catcode64=11 % @
  \catcode91=12 % [
  \catcode93=12 % ]
  \catcode123=1 % {
  \catcode125=2 % }
  \expandafter\ifx\csname ProvidesPackage\endcsname\relax
    \def\x#1#2#3[#4]{\endgroup
      \immediate\write-1{Package: #3 #4}%
      \xdef#1{#4}%
    }%
  \else
    \def\x#1#2[#3]{\endgroup
      #2[{#3}]%
      \ifx#1\@undefined
        \xdef#1{#3}%
      \fi
      \ifx#1\relax
        \xdef#1{#3}%
      \fi
    }%
  \fi
\expandafter\x\csname ver@pdftexcmds.sty\endcsname
\ProvidesPackage{pdftexcmds}%
  [2019/07/25 v0.30 Utility functions of pdfTeX for LuaTeX (HO)]%
%    \end{macrocode}
%
% \subsection{Catcodes}
%
%    \begin{macrocode}
\begingroup\catcode61\catcode48\catcode32=10\relax%
  \catcode13=5 % ^^M
  \endlinechar=13 %
  \catcode123=1 % {
  \catcode125=2 % }
  \catcode64=11 % @
  \def\x{\endgroup
    \expandafter\edef\csname pdftexcmds@AtEnd\endcsname{%
      \endlinechar=\the\endlinechar\relax
      \catcode13=\the\catcode13\relax
      \catcode32=\the\catcode32\relax
      \catcode35=\the\catcode35\relax
      \catcode61=\the\catcode61\relax
      \catcode64=\the\catcode64\relax
      \catcode123=\the\catcode123\relax
      \catcode125=\the\catcode125\relax
    }%
  }%
\x\catcode61\catcode48\catcode32=10\relax%
\catcode13=5 % ^^M
\endlinechar=13 %
\catcode35=6 % #
\catcode64=11 % @
\catcode123=1 % {
\catcode125=2 % }
\def\TMP@EnsureCode#1#2{%
  \edef\pdftexcmds@AtEnd{%
    \pdftexcmds@AtEnd
    \catcode#1=\the\catcode#1\relax
  }%
  \catcode#1=#2\relax
}
\TMP@EnsureCode{0}{12}%
\TMP@EnsureCode{1}{12}%
\TMP@EnsureCode{2}{12}%
\TMP@EnsureCode{10}{12}% ^^J
\TMP@EnsureCode{33}{12}% !
\TMP@EnsureCode{34}{12}% "
\TMP@EnsureCode{38}{4}% &
\TMP@EnsureCode{39}{12}% '
\TMP@EnsureCode{40}{12}% (
\TMP@EnsureCode{41}{12}% )
\TMP@EnsureCode{42}{12}% *
\TMP@EnsureCode{43}{12}% +
\TMP@EnsureCode{44}{12}% ,
\TMP@EnsureCode{45}{12}% -
\TMP@EnsureCode{46}{12}% .
\TMP@EnsureCode{47}{12}% /
\TMP@EnsureCode{58}{12}% :
\TMP@EnsureCode{60}{12}% <
\TMP@EnsureCode{62}{12}% >
\TMP@EnsureCode{91}{12}% [
\TMP@EnsureCode{93}{12}% ]
\TMP@EnsureCode{94}{7}% ^ (superscript)
\TMP@EnsureCode{95}{12}% _ (other)
\TMP@EnsureCode{96}{12}% `
\TMP@EnsureCode{126}{12}% ~ (other)
\edef\pdftexcmds@AtEnd{%
  \pdftexcmds@AtEnd
  \escapechar=\number\escapechar\relax
  \noexpand\endinput
}
\escapechar=92 %
%    \end{macrocode}
%
% \subsection{Load packages}
%
%    \begin{macrocode}
\begingroup\expandafter\expandafter\expandafter\endgroup
\expandafter\ifx\csname RequirePackage\endcsname\relax
  \def\TMP@RequirePackage#1[#2]{%
    \begingroup\expandafter\expandafter\expandafter\endgroup
    \expandafter\ifx\csname ver@#1.sty\endcsname\relax
      \input #1.sty\relax
    \fi
  }%
  \TMP@RequirePackage{infwarerr}[2007/09/09]%
  \TMP@RequirePackage{ifluatex}[2010/03/01]%
  \TMP@RequirePackage{ltxcmds}[2010/12/02]%
  \TMP@RequirePackage{ifpdf}[2010/09/13]%
\else
  \RequirePackage{infwarerr}[2007/09/09]%
  \RequirePackage{ifluatex}[2010/03/01]%
  \RequirePackage{ltxcmds}[2010/12/02]%
  \RequirePackage{ifpdf}[2010/09/13]%
\fi
%    \end{macrocode}
%
% \subsection{Without \hologo{LuaTeX}}
%
%    \begin{macrocode}
\ifluatex
\else
  \@PackageInfoNoLine{pdftexcmds}{LuaTeX not detected}%
  \def\pdftexcmds@nopdftex{%
    \@PackageInfoNoLine{pdftexcmds}{pdfTeX >= 1.30 not detected}%
    \let\pdftexcmds@nopdftex\relax
  }%
  \def\pdftexcmds@temp#1{%
    \begingroup\expandafter\expandafter\expandafter\endgroup
    \expandafter\ifx\csname pdf#1\endcsname\relax
      \pdftexcmds@nopdftex
    \else
      \expandafter\def\csname pdf@#1\expandafter\endcsname
      \expandafter##\expandafter{%
        \csname pdf#1\endcsname
      }%
    \fi
  }%
  \pdftexcmds@temp{strcmp}%
  \pdftexcmds@temp{escapehex}%
  \let\pdf@escapehexnative\pdf@escapehex
  \pdftexcmds@temp{unescapehex}%
  \let\pdf@unescapehexnative\pdf@unescapehex
  \pdftexcmds@temp{escapestring}%
  \pdftexcmds@temp{escapename}%
  \pdftexcmds@temp{filesize}%
  \pdftexcmds@temp{filemoddate}%
  \begingroup\expandafter\expandafter\expandafter\endgroup
  \expandafter\ifx\csname pdfshellescape\endcsname\relax
    \pdftexcmds@nopdftex
    \ltx@IfUndefined{pdftexversion}{%
    }{%
      \ifnum\pdftexversion>120 % 1.21a supports \ifeof18
        \ifeof18 %
          \chardef\pdf@shellescape=0 %
        \else
          \chardef\pdf@shellescape=1 %
        \fi
      \fi
    }%
  \else
    \def\pdf@shellescape{%
      \pdfshellescape
    }%
  \fi
  \begingroup\expandafter\expandafter\expandafter\endgroup
  \expandafter\ifx\csname pdffiledump\endcsname\relax
    \pdftexcmds@nopdftex
  \else
    \def\pdf@filedump#1#2#3{%
      \pdffiledump offset#1 length#2{#3}%
    }%
  \fi
%    \end{macrocode}
%    \begin{macrocode}
  \begingroup\expandafter\expandafter\expandafter\endgroup
  \expandafter\ifx\csname pdfmdfivesum\endcsname\relax
    \begingroup\expandafter\expandafter\expandafter\endgroup
    \expandafter\ifx\csname mdfivesum\endcsname\relax
      \pdftexcmds@nopdftex
    \else
      \def\pdf@mdfivesum#{\mdfivesum}%
      \let\pdf@mdfivesumnative\pdf@mdfivesum
      \def\pdf@filemdfivesum#{\mdfivesum file}%
    \fi
  \else
    \def\pdf@mdfivesum#{\pdfmdfivesum}%
    \let\pdf@mdfivesumnative\pdf@mdfivesum
    \def\pdf@filemdfivesum#{\pdfmdfivesum file}%
  \fi
%    \end{macrocode}
%    \begin{macrocode}
  \def\pdf@system#{%
    \immediate\write18%
  }%
  \def\pdftexcmds@temp#1{%
    \begingroup\expandafter\expandafter\expandafter\endgroup
    \expandafter\ifx\csname pdf#1\endcsname\relax
      \pdftexcmds@nopdftex
    \else
      \expandafter\let\csname pdf@#1\expandafter\endcsname
      \csname pdf#1\endcsname
    \fi
  }%
  \pdftexcmds@temp{resettimer}%
  \pdftexcmds@temp{elapsedtime}%
\fi
%    \end{macrocode}
%
% \subsection{\cs{pdf@primitive}, \cs{pdf@ifprimitive}}
%
%    Since version 1.40.0 \hologo{pdfTeX} has \cs{pdfprimitive} and
%    \cs{ifpdfprimitive}. And \cs{pdfprimitive} was fixed in
%    version 1.40.4.
%
%    \hologo{XeTeX} provides them under the name \cs{primitive} and
%    \cs{ifprimitive}. \hologo{LuaTeX} knows both name variants,
%    but they have possibly to be enabled first (|tex.enableprimitives|).
%
%    Depending on the format TeX Live uses a prefix |luatex|.
%
%    Caution: \cs{let} must be used for the definition of
%    the macros, especially because of \cs{ifpdfprimitive}.
%
% \subsubsection{Using \hologo{LuaTeX}'s \texttt{tex.enableprimitives}}
%
%    \begin{macrocode}
\ifluatex
%    \end{macrocode}
%    \begin{macro}{\pdftexcmds@directlua}
%    \begin{macrocode}
  \ifnum\luatexversion<36 %
    \def\pdftexcmds@directlua{\directlua0 }%
  \else
    \let\pdftexcmds@directlua\directlua
  \fi
%    \end{macrocode}
%    \end{macro}
%
%    \begin{macrocode}
  \begingroup
    \newlinechar=10 %
    \endlinechar=\newlinechar
    \pdftexcmds@directlua{%
      if tex.enableprimitives then
        tex.enableprimitives(
          'pdf@',
          {'primitive', 'ifprimitive', 'pdfdraftmode','draftmode'}
        )
        tex.enableprimitives('', {'luaescapestring'})
      end
    }%
  \endgroup %
%    \end{macrocode}
%
%    \begin{macrocode}
\fi
%    \end{macrocode}
%
% \subsubsection{Trying various names to find the primitives}
%
%    \begin{macro}{\pdftexcmds@strip@prefix}
%    \begin{macrocode}
\def\pdftexcmds@strip@prefix#1>{}
%    \end{macrocode}
%    \end{macro}
%    \begin{macrocode}
\def\pdftexcmds@temp#1#2#3{%
  \begingroup\expandafter\expandafter\expandafter\endgroup
  \expandafter\ifx\csname pdf@#1\endcsname\relax
    \begingroup
      \def\x{#3}%
      \edef\x{\expandafter\pdftexcmds@strip@prefix\meaning\x}%
      \escapechar=-1 %
      \edef\y{\expandafter\meaning\csname#2\endcsname}%
    \expandafter\endgroup
    \ifx\x\y
      \expandafter\let\csname pdf@#1\expandafter\endcsname
      \csname #2\endcsname
    \fi
  \fi
}
%    \end{macrocode}
%
%    \begin{macro}{\pdf@primitive}
%    \begin{macrocode}
\pdftexcmds@temp{primitive}{pdfprimitive}{pdfprimitive}% pdfTeX, oldLuaTeX
\pdftexcmds@temp{primitive}{primitive}{primitive}% XeTeX, luatex
\pdftexcmds@temp{primitive}{luatexprimitive}{pdfprimitive}% oldLuaTeX
\pdftexcmds@temp{primitive}{luatexpdfprimitive}{pdfprimitive}% oldLuaTeX
%    \end{macrocode}
%    \end{macro}
%    \begin{macro}{\pdf@ifprimitive}
%    \begin{macrocode}
\pdftexcmds@temp{ifprimitive}{ifpdfprimitive}{ifpdfprimitive}% pdfTeX, oldLuaTeX
\pdftexcmds@temp{ifprimitive}{ifprimitive}{ifprimitive}% XeTeX, luatex
\pdftexcmds@temp{ifprimitive}{luatexifprimitive}{ifpdfprimitive}% oldLuaTeX
\pdftexcmds@temp{ifprimitive}{luatexifpdfprimitive}{ifpdfprimitive}% oldLuaTeX
%    \end{macrocode}
%    \end{macro}
%
%    Disable broken \cs{pdfprimitive}.
%    \begin{macrocode}
\ifluatex\else
\begingroup
  \expandafter\ifx\csname pdf@primitive\endcsname\relax
  \else
    \expandafter\ifx\csname pdftexversion\endcsname\relax
    \else
      \ifnum\pdftexversion=140 %
        \expandafter\ifx\csname pdftexrevision\endcsname\relax
        \else
          \ifnum\pdftexrevision<4 %
            \endgroup
            \let\pdf@primitive\@undefined
            \@PackageInfoNoLine{pdftexcmds}{%
              \string\pdf@primitive\space disabled, %
              because\MessageBreak
              \string\pdfprimitive\space is broken until pdfTeX 1.40.4%
            }%
            \begingroup
          \fi
        \fi
      \fi
    \fi
  \fi
\endgroup
\fi
%    \end{macrocode}
%
% \subsubsection{Result}
%
%    \begin{macrocode}
\begingroup
  \@PackageInfoNoLine{pdftexcmds}{%
    \string\pdf@primitive\space is %
    \expandafter\ifx\csname pdf@primitive\endcsname\relax not \fi
    available%
  }%
  \@PackageInfoNoLine{pdftexcmds}{%
    \string\pdf@ifprimitive\space is %
    \expandafter\ifx\csname pdf@ifprimitive\endcsname\relax not \fi
    available%
  }%
\endgroup
%    \end{macrocode}
%
% \subsection{\hologo{XeTeX}}
%
%    Look for primitives \cs{shellescape}, \cs{strcmp}.
%    \begin{macrocode}
\def\pdftexcmds@temp#1{%
  \begingroup\expandafter\expandafter\expandafter\endgroup
  \expandafter\ifx\csname pdf@#1\endcsname\relax
    \begingroup
      \escapechar=-1 %
      \edef\x{\expandafter\meaning\csname#1\endcsname}%
      \def\y{#1}%
      \def\z##1->{}%
      \edef\y{\expandafter\z\meaning\y}%
    \expandafter\endgroup
    \ifx\x\y
      \expandafter\def\csname pdf@#1\expandafter\endcsname
      \expandafter{%
        \csname#1\endcsname
      }%
    \fi
  \fi
}%
\pdftexcmds@temp{shellescape}%
\pdftexcmds@temp{strcmp}%
%    \end{macrocode}
%
% \subsection{\cs{pdf@isprimitive}}
%
%    \begin{macrocode}
\def\pdf@isprimitive{%
  \begingroup\expandafter\expandafter\expandafter\endgroup
  \expandafter\ifx\csname pdf@strcmp\endcsname\relax
    \long\def\pdf@isprimitive##1{%
      \expandafter\pdftexcmds@isprimitive\expandafter{\meaning##1}%
    }%
    \long\def\pdftexcmds@isprimitive##1##2{%
      \expandafter\pdftexcmds@@isprimitive\expandafter{\string##2}{##1}%
    }%
    \def\pdftexcmds@@isprimitive##1##2{%
      \ifnum0\pdftexcmds@equal##1\delimiter##2\delimiter=1 %
        \expandafter\ltx@firstoftwo
      \else
        \expandafter\ltx@secondoftwo
      \fi
    }%
    \def\pdftexcmds@equal##1##2\delimiter##3##4\delimiter{%
      \ifx##1##3%
        \ifx\relax##2##4\relax
          1%
        \else
          \ifx\relax##2\relax
          \else
            \ifx\relax##4\relax
            \else
              \pdftexcmds@equalcont{##2}{##4}%
            \fi
          \fi
        \fi
      \fi
    }%
    \def\pdftexcmds@equalcont##1{%
      \def\pdftexcmds@equalcont####1####2##1##1##1##1{%
        ##1##1##1##1%
        \pdftexcmds@equal####1\delimiter####2\delimiter
      }%
    }%
    \expandafter\pdftexcmds@equalcont\csname fi\endcsname
  \else
    \long\def\pdf@isprimitive##1##2{%
      \ifnum\pdf@strcmp{\meaning##1}{\string##2}=0 %
        \expandafter\ltx@firstoftwo
      \else
        \expandafter\ltx@secondoftwo
      \fi
    }%
  \fi
}
\ifluatex
\ifx\pdfdraftmode\@undefined
  \let\pdfdraftmode\draftmode
\fi
\else
  \pdf@isprimitive
\fi
%    \end{macrocode}
%
% \subsection{\cs{pdf@draftmode}}
%
%
%    \begin{macrocode}
\let\pdftexcmds@temp\ltx@zero %
\ltx@IfUndefined{pdfdraftmode}{%
  \@PackageInfoNoLine{pdftexcmds}{\ltx@backslashchar pdfdraftmode not found}%
}{%
  \ifpdf
    \let\pdftexcmds@temp\ltx@one
    \@PackageInfoNoLine{pdftexcmds}{\ltx@backslashchar pdfdraftmode found}%
  \else
    \@PackageInfoNoLine{pdftexcmds}{%
      \ltx@backslashchar pdfdraftmode is ignored in DVI mode%
    }%
  \fi
}
\ifcase\pdftexcmds@temp
%    \end{macrocode}
%    \begin{macro}{\pdf@draftmode}
%    \begin{macrocode}
  \let\pdf@draftmode\ltx@zero
%    \end{macrocode}
%    \end{macro}
%    \begin{macro}{\pdf@ifdraftmode}
%    \begin{macrocode}
  \let\pdf@ifdraftmode\ltx@secondoftwo
%    \end{macrocode}
%    \end{macro}
%    \begin{macro}{\pdftexcmds@setdraftmode}
%    \begin{macrocode}
  \def\pdftexcmds@setdraftmode#1{}%
%    \end{macrocode}
%    \end{macro}
%    \begin{macrocode}
\else
%    \end{macrocode}
%    \begin{macro}{\pdftexcmds@draftmode}
%    \begin{macrocode}
  \let\pdftexcmds@draftmode\pdfdraftmode
%    \end{macrocode}
%    \end{macro}
%    \begin{macro}{\pdf@ifdraftmode}
%    \begin{macrocode}
  \def\pdf@ifdraftmode{%
    \ifnum\pdftexcmds@draftmode=\ltx@one
      \expandafter\ltx@firstoftwo
    \else
      \expandafter\ltx@secondoftwo
    \fi
  }%
%    \end{macrocode}
%    \end{macro}
%    \begin{macro}{\pdf@draftmode}
%    \begin{macrocode}
  \def\pdf@draftmode{%
    \ifnum\pdftexcmds@draftmode=\ltx@one
      \expandafter\ltx@one
    \else
      \expandafter\ltx@zero
    \fi
  }%
%    \end{macrocode}
%    \end{macro}
%    \begin{macro}{\pdftexcmds@setdraftmode}
%    \begin{macrocode}
  \def\pdftexcmds@setdraftmode#1{%
    \pdftexcmds@draftmode=#1\relax
  }%
%    \end{macrocode}
%    \end{macro}
%    \begin{macrocode}
\fi
%    \end{macrocode}
%    \begin{macro}{\pdf@setdraftmode}
%    \begin{macrocode}
\def\pdf@setdraftmode#1{%
  \begingroup
    \count\ltx@cclv=#1\relax
  \edef\x{\endgroup
    \noexpand\pdftexcmds@@setdraftmode{\the\count\ltx@cclv}%
  }%
  \x
}
%    \end{macrocode}
%    \end{macro}
%    \begin{macro}{\pdftexcmds@@setdraftmode}
%    \begin{macrocode}
\def\pdftexcmds@@setdraftmode#1{%
  \ifcase#1 %
    \pdftexcmds@setdraftmode{#1}%
  \or
    \pdftexcmds@setdraftmode{#1}%
  \else
    \@PackageWarning{pdftexcmds}{%
      \string\pdf@setdraftmode: Ignoring\MessageBreak
      invalid value `#1'%
    }%
  \fi
}
%    \end{macrocode}
%    \end{macro}
%
% \subsection{Load Lua module}
%
%    \begin{macrocode}
\ifluatex
\else
  \expandafter\pdftexcmds@AtEnd
\fi%
%    \end{macrocode}
%
%    \begin{macrocode}
\ifnum\luatexversion<80
  \begingroup\expandafter\expandafter\expandafter\endgroup
  \expandafter\ifx\csname RequirePackage\endcsname\relax
    \def\TMP@RequirePackage#1[#2]{%
      \begingroup\expandafter\expandafter\expandafter\endgroup
      \expandafter\ifx\csname ver@#1.sty\endcsname\relax
        \input #1.sty\relax
      \fi
    }%
    \TMP@RequirePackage{luatex-loader}[2009/04/10]%
  \else
    \RequirePackage{luatex-loader}[2009/04/10]%
  \fi
\fi
\pdftexcmds@directlua{%
  require("pdftexcmds")%
}
\ifnum\luatexversion>37 %
  \ifnum0%
      \pdftexcmds@directlua{%
        if status.ini_version then %
          tex.write("1")%
        end%
      }>0 %
    \everyjob\expandafter{%
      \the\everyjob
      \pdftexcmds@directlua{%
        require("pdftexcmds")%
      }%
    }%
  \fi
\fi
\begingroup
  \def\x{2019/07/25 v0.30}%
  \ltx@onelevel@sanitize\x
  \edef\y{%
    \pdftexcmds@directlua{%
      if oberdiek.pdftexcmds.getversion then %
        oberdiek.pdftexcmds.getversion()%
      end%
    }%
  }%
  \ifx\x\y
  \else
    \@PackageError{pdftexcmds}{%
      Wrong version of lua module.\MessageBreak
      Package version: \x\MessageBreak
      Lua module: \y
    }\@ehc
  \fi
\endgroup
%    \end{macrocode}
%
% \subsection{Lua functions}
%
% \subsubsection{Helper macros}
%
%    \begin{macro}{\pdftexcmds@toks}
%    \begin{macrocode}
\begingroup\expandafter\expandafter\expandafter\endgroup
\expandafter\ifx\csname newtoks\endcsname\relax
  \toksdef\pdftexcmds@toks=0 %
\else
  \csname newtoks\endcsname\pdftexcmds@toks
\fi
%    \end{macrocode}
%    \end{macro}
%
%    \begin{macro}{\pdftexcmds@Patch}
%    \begin{macrocode}
\def\pdftexcmds@Patch{0}
\ifnum\luatexversion>40 %
  \ifnum\luatexversion<66 %
    \def\pdftexcmds@Patch{1}%
  \fi
\fi
%    \end{macrocode}
%    \end{macro}
%    \begin{macrocode}
\ifcase\pdftexcmds@Patch
  \catcode`\&=14 %
\else
  \catcode`\&=9 %
%    \end{macrocode}
%    \begin{macro}{\pdftexcmds@PatchDecode}
%    \begin{macrocode}
  \def\pdftexcmds@PatchDecode#1\@nil{%
    \pdftexcmds@DecodeA#1^^A^^A\@nil{}%
  }%
%    \end{macrocode}
%    \end{macro}
%    \begin{macro}{\pdftexcmds@DecodeA}
%    \begin{macrocode}
  \def\pdftexcmds@DecodeA#1^^A^^A#2\@nil#3{%
    \ifx\relax#2\relax
      \ltx@ReturnAfterElseFi{%
        \pdftexcmds@DecodeB#3#1^^A^^B\@nil{}%
      }%
    \else
      \ltx@ReturnAfterFi{%
        \pdftexcmds@DecodeA#2\@nil{#3#1^^@}%
      }%
    \fi
  }%
%    \end{macrocode}
%    \end{macro}
%    \begin{macro}{\pdftexcmds@DecodeB}
%    \begin{macrocode}
  \def\pdftexcmds@DecodeB#1^^A^^B#2\@nil#3{%
    \ifx\relax#2\relax%
      \ltx@ReturnAfterElseFi{%
        \ltx@zero
        #3#1%
      }%
    \else
      \ltx@ReturnAfterFi{%
        \pdftexcmds@DecodeB#2\@nil{#3#1^^A}%
      }%
    \fi
  }%
%    \end{macrocode}
%    \end{macro}
%    \begin{macrocode}
\fi
%    \end{macrocode}
%
%    \begin{macrocode}
\ifnum\luatexversion<36 %
\else
  \catcode`\0=9 %
\fi
%    \end{macrocode}
%
% \subsubsection[Strings]{Strings \cite[``7.15 Strings'']{pdftex-manual}}
%
%    \begin{macro}{\pdf@strcmp}
%    \begin{macrocode}
\long\def\pdf@strcmp#1#2{%
  \directlua0{%
    oberdiek.pdftexcmds.strcmp("\luaescapestring{#1}",%
        "\luaescapestring{#2}")%
  }%
}%
%    \end{macrocode}
%    \end{macro}
%    \begin{macrocode}
\pdf@isprimitive
%    \end{macrocode}
%    \begin{macro}{\pdf@escapehex}
%    \begin{macrocode}
\long\def\pdf@escapehex#1{%
  \directlua0{%
    oberdiek.pdftexcmds.escapehex("\luaescapestring{#1}", "byte")%
  }%
}%
%    \end{macrocode}
%    \end{macro}
%    \begin{macro}{\pdf@escapehexnative}
%    \begin{macrocode}
\long\def\pdf@escapehexnative#1{%
  \directlua0{%
    oberdiek.pdftexcmds.escapehex("\luaescapestring{#1}")%
  }%
}%
%    \end{macrocode}
%    \end{macro}
%    \begin{macro}{\pdf@unescapehex}
%    \begin{macrocode}
\def\pdf@unescapehex#1{%
& \romannumeral\expandafter\pdftexcmds@PatchDecode
  \the\expandafter\pdftexcmds@toks
  \directlua0{%
    oberdiek.pdftexcmds.toks="pdftexcmds@toks"%
    oberdiek.pdftexcmds.unescapehex("\luaescapestring{#1}", "byte", \pdftexcmds@Patch)%
  }%
& \@nil
}%
%    \end{macrocode}
%    \end{macro}
%    \begin{macro}{\pdf@unescapehexnative}
%    \begin{macrocode}
\def\pdf@unescapehexnative#1{%
& \romannumeral\expandafter\pdftexcmds@PatchDecode
  \the\expandafter\pdftexcmds@toks
  \directlua0{%
    oberdiek.pdftexcmds.toks="pdftexcmds@toks"%
    oberdiek.pdftexcmds.unescapehex("\luaescapestring{#1}", \pdftexcmds@Patch)%
  }%
& \@nil
}%
%    \end{macrocode}
%    \end{macro}
%    \begin{macro}{\pdf@escapestring}
%    \begin{macrocode}
\long\def\pdf@escapestring#1{%
  \directlua0{%
    oberdiek.pdftexcmds.escapestring("\luaescapestring{#1}", "byte")%
  }%
}
%    \end{macrocode}
%    \end{macro}
%    \begin{macro}{\pdf@escapename}
%    \begin{macrocode}
\long\def\pdf@escapename#1{%
  \directlua0{%
    oberdiek.pdftexcmds.escapename("\luaescapestring{#1}", "byte")%
  }%
}
%    \end{macrocode}
%    \end{macro}
%    \begin{macro}{\pdf@escapenamenative}
%    \begin{macrocode}
\long\def\pdf@escapenamenative#1{%
  \directlua0{%
    oberdiek.pdftexcmds.escapename("\luaescapestring{#1}")%
  }%
}
%    \end{macrocode}
%    \end{macro}
%
% \subsubsection[Files]{Files \cite[``7.18 Files'']{pdftex-manual}}
%
%    \begin{macro}{\pdf@filesize}
%    \begin{macrocode}
\def\pdf@filesize#1{%
  \directlua0{%
    oberdiek.pdftexcmds.filesize("\luaescapestring{#1}")%
  }%
}
%    \end{macrocode}
%    \end{macro}
%    \begin{macro}{\pdf@filemoddate}
%    \begin{macrocode}
\def\pdf@filemoddate#1{%
  \directlua0{%
    oberdiek.pdftexcmds.filemoddate("\luaescapestring{#1}")%
  }%
}
%    \end{macrocode}
%    \end{macro}
%    \begin{macro}{\pdf@filedump}
%    \begin{macrocode}
\def\pdf@filedump#1#2#3{%
  \directlua0{%
    oberdiek.pdftexcmds.filedump("\luaescapestring{\number#1}",%
        "\luaescapestring{\number#2}",%
        "\luaescapestring{#3}")%
  }%
}%
%    \end{macrocode}
%    \end{macro}
%    \begin{macro}{\pdf@mdfivesum}
%    \begin{macrocode}
\long\def\pdf@mdfivesum#1{%
  \directlua0{%
    oberdiek.pdftexcmds.mdfivesum("\luaescapestring{#1}", "byte")%
  }%
}%
%    \end{macrocode}
%    \end{macro}
%    \begin{macro}{\pdf@mdfivesumnative}
%    \begin{macrocode}
\long\def\pdf@mdfivesumnative#1{%
  \directlua0{%
    oberdiek.pdftexcmds.mdfivesum("\luaescapestring{#1}")%
  }%
}%
%    \end{macrocode}
%    \end{macro}
%    \begin{macro}{\pdf@filemdfivesum}
%    \begin{macrocode}
\def\pdf@filemdfivesum#1{%
  \directlua0{%
    oberdiek.pdftexcmds.filemdfivesum("\luaescapestring{#1}")%
  }%
}%
%    \end{macrocode}
%    \end{macro}
%
% \subsubsection[Timekeeping]{Timekeeping \cite[``7.17 Timekeeping'']{pdftex-manual}}
%
%    \begin{macro}{\protected}
%    \begin{macrocode}
\let\pdftexcmds@temp=Y%
\begingroup\expandafter\expandafter\expandafter\endgroup
\expandafter\ifx\csname protected\endcsname\relax
  \pdftexcmds@directlua0{%
    if tex.enableprimitives then %
      tex.enableprimitives('', {'protected'})%
    end%
  }%
\fi
\begingroup\expandafter\expandafter\expandafter\endgroup
\expandafter\ifx\csname protected\endcsname\relax
  \let\pdftexcmds@temp=N%
\fi
%    \end{macrocode}
%    \end{macro}
%    \begin{macro}{\numexpr}
%    \begin{macrocode}
\begingroup\expandafter\expandafter\expandafter\endgroup
\expandafter\ifx\csname numexpr\endcsname\relax
  \pdftexcmds@directlua0{%
    if tex.enableprimitives then %
      tex.enableprimitives('', {'numexpr'})%
    end%
  }%
\fi
\begingroup\expandafter\expandafter\expandafter\endgroup
\expandafter\ifx\csname numexpr\endcsname\relax
  \let\pdftexcmds@temp=N%
\fi
%    \end{macrocode}
%    \end{macro}
%
%    \begin{macrocode}
\ifx\pdftexcmds@temp N%
  \@PackageWarningNoLine{pdftexcmds}{%
    Definitions of \ltx@backslashchar pdf@resettimer and%
    \MessageBreak
    \ltx@backslashchar pdf@elapsedtime are skipped, because%
    \MessageBreak
    e-TeX's \ltx@backslashchar protected or %
    \ltx@backslashchar numexpr are missing%
  }%
\else
%    \end{macrocode}
%
%    \begin{macro}{\pdf@resettimer}
%    \begin{macrocode}
  \protected\def\pdf@resettimer{%
    \pdftexcmds@directlua0{%
      oberdiek.pdftexcmds.resettimer()%
    }%
  }%
%    \end{macrocode}
%    \end{macro}
%
%    \begin{macro}{\pdf@elapsedtime}
%    \begin{macrocode}
  \protected\def\pdf@elapsedtime{%
    \numexpr
      \pdftexcmds@directlua0{%
        oberdiek.pdftexcmds.elapsedtime()%
      }%
    \relax
  }%
%    \end{macrocode}
%    \end{macro}
%    \begin{macrocode}
\fi
%    \end{macrocode}
%
% \subsubsection{Shell escape}
%
%    \begin{macro}{\pdf@shellescape}
%
%    \begin{macrocode}
\ifnum\luatexversion<68 %
\else
  \protected\edef\pdf@shellescape{%
   \numexpr\directlua{tex.sprint(%
         \number\catcodetable@string,status.shell_escape)}\relax}
\fi
%    \end{macrocode}
%    \end{macro}
%
%    \begin{macro}{\pdf@system}
%    \begin{macrocode}
\def\pdf@system#1{%
  \directlua0{%
    oberdiek.pdftexcmds.system("\luaescapestring{#1}")%
  }%
}
%    \end{macrocode}
%    \end{macro}
%
%    \begin{macro}{\pdf@lastsystemstatus}
%    \begin{macrocode}
\def\pdf@lastsystemstatus{%
  \directlua0{%
    oberdiek.pdftexcmds.lastsystemstatus()%
  }%
}
%    \end{macrocode}
%    \end{macro}
%    \begin{macro}{\pdf@lastsystemexit}
%    \begin{macrocode}
\def\pdf@lastsystemexit{%
  \directlua0{%
    oberdiek.pdftexcmds.lastsystemexit()%
  }%
}
%    \end{macrocode}
%    \end{macro}
%
%    \begin{macrocode}
\catcode`\0=12 %
%    \end{macrocode}
%
%    \begin{macro}{\pdf@pipe}
%    Check availability of |io.popen| first.
%    \begin{macrocode}
\ifnum0%
    \pdftexcmds@directlua{%
      if io.popen then %
        tex.write("1")%
      end%
    }%
    =1 %
  \def\pdf@pipe#1{%
&   \romannumeral\expandafter\pdftexcmds@PatchDecode
    \the\expandafter\pdftexcmds@toks
    \pdftexcmds@directlua{%
      oberdiek.pdftexcmds.toks="pdftexcmds@toks"%
      oberdiek.pdftexcmds.pipe("\luaescapestring{#1}", \pdftexcmds@Patch)%
    }%
&   \@nil
  }%
\fi
%    \end{macrocode}
%    \end{macro}
%
%    \begin{macrocode}
\pdftexcmds@AtEnd%
%</package>
%    \end{macrocode}
%
% \subsection{Lua module}
%
%    \begin{macrocode}
%<*lua>
%    \end{macrocode}
%
%    \begin{macrocode}
oberdiek = oberdiek or {}
local pdftexcmds = oberdiek.pdftexcmds or {}
oberdiek.pdftexcmds = pdftexcmds
local systemexitstatus
function pdftexcmds.getversion()
  tex.write("2019/07/25 v0.30")
end
%    \end{macrocode}
%
% \subsubsection[Strings]{Strings \cite[``7.15 Strings'']{pdftex-manual}}
%
%    \begin{macrocode}
function pdftexcmds.strcmp(A, B)
  if A == B then
    tex.write("0")
  elseif A < B then
    tex.write("-1")
  else
    tex.write("1")
  end
end
local function utf8_to_byte(str)
  local i = 0
  local n = string.len(str)
  local t = {}
  while i < n do
    i = i + 1
    local a = string.byte(str, i)
    if a < 128 then
      table.insert(t, string.char(a))
    else
      if a >= 192 and i < n then
        i = i + 1
        local b = string.byte(str, i)
        if b < 128 or b >= 192 then
          i = i - 1
        elseif a == 194 then
          table.insert(t, string.char(b))
        elseif a == 195 then
          table.insert(t, string.char(b + 64))
        end
      end
    end
  end
  return table.concat(t)
end
function pdftexcmds.escapehex(str, mode)
  if mode == "byte" then
    str = utf8_to_byte(str)
  end
  tex.write((string.gsub(str, ".",
    function (ch)
      return string.format("%02X", string.byte(ch))
    end
  )))
end
%    \end{macrocode}
%    See procedure |unescapehex| in file \xfile{utils.c} of \hologo{pdfTeX}.
%    Caution: |tex.write| ignores leading spaces.
%    \begin{macrocode}
function pdftexcmds.unescapehex(str, mode, patch)
  local a = 0
  local first = true
  local result = {}
  for i = 1, string.len(str), 1 do
    local ch = string.byte(str, i)
    if ch >= 48 and ch <= 57 then
      ch = ch - 48
    elseif ch >= 65 and ch <= 70 then
      ch = ch - 55
    elseif ch >= 97 and ch <= 102 then
      ch = ch - 87
    else
      ch = nil
    end
    if ch then
      if first then
        a = ch * 16
        first = false
      else
        table.insert(result, a + ch)
        first = true
      end
    end
  end
  if not first then
    table.insert(result, a)
  end
  if patch == 1 then
    local temp = {}
    for i, a in ipairs(result) do
      if a == 0 then
        table.insert(temp, 1)
        table.insert(temp, 1)
      else
        if a == 1 then
          table.insert(temp, 1)
          table.insert(temp, 2)
        else
          table.insert(temp, a)
        end
      end
    end
    result = temp
  end
  if mode == "byte" then
    local utf8 = {}
    for i, a in ipairs(result) do
      if a < 128 then
        table.insert(utf8, a)
      else
        if a < 192 then
          table.insert(utf8, 194)
          a = a - 128
        else
          table.insert(utf8, 195)
          a = a - 192
        end
        table.insert(utf8, a + 128)
      end
    end
    result = utf8
  end
%    \end{macrocode}
%    this next line added for current luatex; this is the only
%    change in the file.  eroux, 28apr13. (v 0.21)
%    \begin{macrocode}
  local unpack = _G["unpack"] or table.unpack
  tex.settoks(pdftexcmds.toks, string.char(unpack(result)))
end
%    \end{macrocode}
%    See procedure |escapestring| in file \xfile{utils.c} of \hologo{pdfTeX}.
%    \begin{macrocode}
function pdftexcmds.escapestring(str, mode)
  if mode == "byte" then
    str = utf8_to_byte(str)
  end
  tex.write((string.gsub(str, ".",
    function (ch)
      local b = string.byte(ch)
      if b < 33 or b > 126 then
        return string.format("\\%.3o", b)
      end
      if b == 40 or b == 41 or b == 92 then
        return "\\" .. ch
      end
%    \end{macrocode}
%    Lua 5.1 returns the match in case of return value |nil|.
%    \begin{macrocode}
      return nil
    end
  )))
end
%    \end{macrocode}
%    See procedure |escapename| in file \xfile{utils.c} of \hologo{pdfTeX}.
%    \begin{macrocode}
function pdftexcmds.escapename(str, mode)
  if mode == "byte" then
    str = utf8_to_byte(str)
  end
  tex.write((string.gsub(str, ".",
    function (ch)
      local b = string.byte(ch)
      if b == 0 then
%    \end{macrocode}
%    In Lua 5.0 |nil| could be used for the empty string,
%    But |nil| returns the match in Lua 5.1, thus we use
%    the empty string explicitly.
%    \begin{macrocode}
        return ""
      end
      if b <= 32 or b >= 127
          or b == 35 or b == 37 or b == 40 or b == 41
          or b == 47 or b == 60 or b == 62 or b == 91
          or b == 93 or b == 123 or b == 125 then
        return string.format("#%.2X", b)
      else
%    \end{macrocode}
%    Lua 5.1 returns the match in case of return value |nil|.
%    \begin{macrocode}
        return nil
      end
    end
  )))
end
%    \end{macrocode}
%
% \subsubsection[Files]{Files \cite[``7.18 Files'']{pdftex-manual}}
%
%    \begin{macrocode}
function pdftexcmds.filesize(filename)
  local foundfile = kpse.find_file(filename, "tex", true)
  if foundfile then
    local size = lfs.attributes(foundfile, "size")
    if size then
      tex.write(size)
    end
  end
end
%    \end{macrocode}
%    See procedure |makepdftime| in file \xfile{utils.c} of \hologo{pdfTeX}.
%    \begin{macrocode}
function pdftexcmds.filemoddate(filename)
  local foundfile = kpse.find_file(filename, "tex", true)
  if foundfile then
    local date = lfs.attributes(foundfile, "modification")
    if date then
      local d = os.date("*t", date)
      if d.sec >= 60 then
        d.sec = 59
      end
      local u = os.date("!*t", date)
      local off = 60 * (d.hour - u.hour) + d.min - u.min
      if d.year ~= u.year then
        if d.year > u.year then
          off = off + 1440
        else
          off = off - 1440
        end
      elseif d.yday ~= u.yday then
        if d.yday > u.yday then
          off = off + 1440
        else
          off = off - 1440
        end
      end
      local timezone
      if off == 0 then
        timezone = "Z"
      else
        local hours = math.floor(off / 60)
        local mins = math.abs(off - hours * 60)
        timezone = string.format("%+03d'%02d'", hours, mins)
      end
      tex.write(string.format("D:%04d%02d%02d%02d%02d%02d%s",
          d.year, d.month, d.day, d.hour, d.min, d.sec, timezone))
    end
  end
end
function pdftexcmds.filedump(offset, length, filename)
  length = tonumber(length)
  if length and length > 0 then
    local foundfile = kpse.find_file(filename, "tex", true)
    if foundfile then
      offset = tonumber(offset)
      if not offset then
        offset = 0
      end
      local filehandle = io.open(foundfile, "rb")
      if filehandle then
        if offset > 0 then
          filehandle:seek("set", offset)
        end
        local dump = filehandle:read(length)
        pdftexcmds.escapehex(dump)
        filehandle:close()
      end
    end
  end
end
function pdftexcmds.mdfivesum(str, mode)
  if mode == "byte" then
    str = utf8_to_byte(str)
  end
  pdftexcmds.escapehex(md5.sum(str))
end
function pdftexcmds.filemdfivesum(filename)
  local foundfile = kpse.find_file(filename, "tex", true)
  if foundfile then
    local filehandle = io.open(foundfile, "rb")
    if filehandle then
      local contents = filehandle:read("*a")
      pdftexcmds.escapehex(md5.sum(contents))
      filehandle:close()
    end
  end
end
%    \end{macrocode}
%
% \subsubsection[Timekeeping]{Timekeeping \cite[``7.17 Timekeeping'']{pdftex-manual}}
%
%    The functions for timekeeping are based on
%    Andy Thomas' work \cite{AndyThomas:Analog}.
%    Changes:
%    \begin{itemize}
%    \item Overflow check is added.
%    \item |string.format| is used to avoid exponential number
%          representation for sure.
%    \item |tex.write| is used instead of |tex.print| to get
%          tokens with catcode 12 and without appended \cs{endlinechar}.
%    \end{itemize}
%    \begin{macrocode}
local basetime = 0
function pdftexcmds.resettimer()
  basetime = os.clock()
end
function pdftexcmds.elapsedtime()
  local val = (os.clock() - basetime) * 65536 + .5
  if val > 2147483647 then
    val = 2147483647
  end
  tex.write(string.format("%d", val))
end
%    \end{macrocode}
%
% \subsubsection[Miscellaneous]{Miscellaneous \cite[``7.21 Miscellaneous'']{pdftex-manual}}
%
%    \begin{macrocode}
function pdftexcmds.shellescape()
  if os.execute then
    if status
        and status.luatex_version
        and status.luatex_version >= 68 then
      tex.write(os.execute())
    else
      local result = os.execute()
      if result == 0 then
        tex.write("0")
      else
        if result == nil then
          tex.write("0")
        else
          tex.write("1")
        end
      end
    end
  else
    tex.write("0")
  end
end
function pdftexcmds.system(cmdline)
  systemexitstatus = nil
  texio.write_nl("log", "system(" .. cmdline .. ") ")
  if os.execute then
    texio.write("log", "executed.")
    systemexitstatus = os.execute(cmdline)
  else
    texio.write("log", "disabled.")
  end
end
function pdftexcmds.lastsystemstatus()
  local result = tonumber(systemexitstatus)
  if result then
    local x = math.floor(result / 256)
    tex.write(result - 256 * math.floor(result / 256))
  end
end
function pdftexcmds.lastsystemexit()
  local result = tonumber(systemexitstatus)
  if result then
    tex.write(math.floor(result / 256))
  end
end
function pdftexcmds.pipe(cmdline, patch)
  local result
  systemexitstatus = nil
  texio.write_nl("log", "pipe(" .. cmdline ..") ")
  if io.popen then
    texio.write("log", "executed.")
    local handle = io.popen(cmdline, "r")
    if handle then
      result = handle:read("*a")
      handle:close()
    end
  else
    texio.write("log", "disabled.")
  end
  if result then
    if patch == 1 then
      local temp = {}
      for i, a in ipairs(result) do
        if a == 0 then
          table.insert(temp, 1)
          table.insert(temp, 1)
        else
          if a == 1 then
            table.insert(temp, 1)
            table.insert(temp, 2)
          else
            table.insert(temp, a)
          end
        end
      end
      result = temp
    end
    tex.settoks(pdftexcmds.toks, result)
  else
    tex.settoks(pdftexcmds.toks, "")
  end
end
%    \end{macrocode}
%    \begin{macrocode}
%</lua>
%    \end{macrocode}
%
% \section{Test}
%
% \subsection{Catcode checks for loading}
%
%    \begin{macrocode}
%<*test1>
%    \end{macrocode}
%    \begin{macrocode}
\catcode`\{=1 %
\catcode`\}=2 %
\catcode`\#=6 %
\catcode`\@=11 %
\expandafter\ifx\csname count@\endcsname\relax
  \countdef\count@=255 %
\fi
\expandafter\ifx\csname @gobble\endcsname\relax
  \long\def\@gobble#1{}%
\fi
\expandafter\ifx\csname @firstofone\endcsname\relax
  \long\def\@firstofone#1{#1}%
\fi
\expandafter\ifx\csname loop\endcsname\relax
  \expandafter\@firstofone
\else
  \expandafter\@gobble
\fi
{%
  \def\loop#1\repeat{%
    \def\body{#1}%
    \iterate
  }%
  \def\iterate{%
    \body
      \let\next\iterate
    \else
      \let\next\relax
    \fi
    \next
  }%
  \let\repeat=\fi
}%
\def\RestoreCatcodes{}
\count@=0 %
\loop
  \edef\RestoreCatcodes{%
    \RestoreCatcodes
    \catcode\the\count@=\the\catcode\count@\relax
  }%
\ifnum\count@<255 %
  \advance\count@ 1 %
\repeat

\def\RangeCatcodeInvalid#1#2{%
  \count@=#1\relax
  \loop
    \catcode\count@=15 %
  \ifnum\count@<#2\relax
    \advance\count@ 1 %
  \repeat
}
\def\RangeCatcodeCheck#1#2#3{%
  \count@=#1\relax
  \loop
    \ifnum#3=\catcode\count@
    \else
      \errmessage{%
        Character \the\count@\space
        with wrong catcode \the\catcode\count@\space
        instead of \number#3%
      }%
    \fi
  \ifnum\count@<#2\relax
    \advance\count@ 1 %
  \repeat
}
\def\space{ }
\expandafter\ifx\csname LoadCommand\endcsname\relax
  \def\LoadCommand{\input pdftexcmds.sty\relax}%
\fi
\def\Test{%
  \RangeCatcodeInvalid{0}{47}%
  \RangeCatcodeInvalid{58}{64}%
  \RangeCatcodeInvalid{91}{96}%
  \RangeCatcodeInvalid{123}{255}%
  \catcode`\@=12 %
  \catcode`\\=0 %
  \catcode`\%=14 %
  \LoadCommand
  \RangeCatcodeCheck{0}{36}{15}%
  \RangeCatcodeCheck{37}{37}{14}%
  \RangeCatcodeCheck{38}{47}{15}%
  \RangeCatcodeCheck{48}{57}{12}%
  \RangeCatcodeCheck{58}{63}{15}%
  \RangeCatcodeCheck{64}{64}{12}%
  \RangeCatcodeCheck{65}{90}{11}%
  \RangeCatcodeCheck{91}{91}{15}%
  \RangeCatcodeCheck{92}{92}{0}%
  \RangeCatcodeCheck{93}{96}{15}%
  \RangeCatcodeCheck{97}{122}{11}%
  \RangeCatcodeCheck{123}{255}{15}%
  \RestoreCatcodes
}
\Test
\csname @@end\endcsname
\end
%    \end{macrocode}
%    \begin{macrocode}
%</test1>
%    \end{macrocode}
%
% \subsection{Test for \cs{pdf@isprimitive}}
%
%    \begin{macrocode}
%<*test2>
\catcode`\{=1 %
\catcode`\}=2 %
\catcode`\#=6 %
\catcode`\@=11 %
\input pdftexcmds.sty\relax
\def\msg#1{%
  \begingroup
    \escapechar=92 %
    \immediate\write16{#1}%
  \endgroup
}
\long\def\test#1#2#3#4{%
  \begingroup
    #4%
    \def\str{%
      Test \string\pdf@isprimitive
      {\string #1}{\string #2}{...}: %
    }%
    \pdf@isprimitive{#1}{#2}{%
      \ifx#3Y%
        \msg{\str true ==> OK.}%
      \else
        \errmessage{\str false ==> FAILED}%
      \fi
    }{%
      \ifx#3Y%
        \errmessage{\str true ==> FAILED}%
      \else
        \msg{\str false ==> OK.}%
      \fi
    }%
  \endgroup
}
\test\relax\relax Y{}
\test\foobar\relax Y{\let\foobar\relax}
\test\foobar\relax N{}
\test\hbox\hbox Y{}
\test\foobar@hbox\hbox Y{\let\foobar@hbox\hbox}
\test\if\if Y{}
\test\if\ifx N{}
\test\ifx\if N{}
\test\par\par Y{}
\test\hbox\par N{}
\test\par\hbox N{}
\csname @@end\endcsname\end
%</test2>
%    \end{macrocode}
%
% \subsection{Test for \cs{pdf@shellescape}}
%
%    \begin{macrocode}
%<*test-shell>
\catcode`\{=1 %
\catcode`\}=2 %
\catcode`\#=6 %
\catcode`\@=11 %
\input pdftexcmds.sty\relax
\def\msg#{\immediate\write16}
\def\MaybeEnd{}
\ifx\luatexversion\UnDeFiNeD
\else
  \ifnum\luatexversion<68 %
    \ifx\pdf@shellescape\@undefined
      \msg{SHELL=U}%
      \msg{OK (LuaTeX < 0.68)}%
    \else
      \msg{SHELL=defined}%
      \errmessage{Failed (LuaTeX < 0.68)}%
    \fi
    \def\MaybeEnd{\csname @@end\endcsname\end}%
  \fi
\fi
\MaybeEnd
\ifx\pdf@shellescape\@undefined
  \msg{SHELL=U}%
\else
  \msg{SHELL=\number\pdf@shellescape}%
\fi
\ifx\expected\@undefined
\else
  \ifx\expected\relax
    \msg{EXPECTED=U}%
    \ifx\pdf@shellescape\@undefined
      \msg{OK}%
    \else
      \errmessage{Failed}%
    \fi
  \else
    \msg{EXPECTED=\number\expected}%
    \ifnum\pdf@shellescape=\expected\relax
      \msg{OK}%
    \else
      \errmessage{Failed}%
    \fi
  \fi
\fi
\csname @@end\endcsname\end
%</test-shell>
%    \end{macrocode}
%
% \subsection{Test for escape functions}
%
%    \begin{macrocode}
%<*test-escape>
\catcode`\{=1 %
\catcode`\}=2 %
\catcode`\#=6 %
\catcode`\^=7 %
\catcode`\@=11 %
\errorcontextlines=1000 %
\input pdftexcmds.sty\relax
\def\msg#1{%
  \begingroup
    \escapechar=92 %
    \immediate\write16{#1}%
  \endgroup
}
%    \end{macrocode}
%    \begin{macrocode}
\begingroup
  \catcode`\@=11 %
  \countdef\count@=255 %
  \def\space{ }%
  \long\def\@whilenum#1\do #2{%
    \ifnum #1\relax
      #2\relax
      \@iwhilenum{#1\relax#2\relax}%
    \fi
  }%
  \long\def\@iwhilenum#1{%
    \ifnum #1%
      \expandafter\@iwhilenum
    \else
      \expandafter\ltx@gobble
    \fi
    {#1}%
  }%
  \gdef\AllBytes{}%
  \count@=0 %
  \catcode0=12 %
  \@whilenum\count@<256 \do{%
    \lccode0=\count@
    \ifnum\count@=32 %
      \xdef\AllBytes{\AllBytes\space}%
    \else
      \lowercase{%
        \xdef\AllBytes{\AllBytes^^@}%
      }%
    \fi
    \advance\count@ by 1 %
  }%
\endgroup
%    \end{macrocode}
%    \begin{macrocode}
\def\AllBytesHex{%
  000102030405060708090A0B0C0D0E0F%
  101112131415161718191A1B1C1D1E1F%
  202122232425262728292A2B2C2D2E2F%
  303132333435363738393A3B3C3D3E3F%
  404142434445464748494A4B4C4D4E4F%
  505152535455565758595A5B5C5D5E5F%
  606162636465666768696A6B6C6D6E6F%
  707172737475767778797A7B7C7D7E7F%
  808182838485868788898A8B8C8D8E8F%
  909192939495969798999A9B9C9D9E9F%
  A0A1A2A3A4A5A6A7A8A9AAABACADAEAF%
  B0B1B2B3B4B5B6B7B8B9BABBBCBDBEBF%
  C0C1C2C3C4C5C6C7C8C9CACBCCCDCECF%
  D0D1D2D3D4D5D6D7D8D9DADBDCDDDEDF%
  E0E1E2E3E4E5E6E7E8E9EAEBECEDEEEF%
  F0F1F2F3F4F5F6F7F8F9FAFBFCFDFEFF%
}
\ltx@onelevel@sanitize\AllBytesHex
\expandafter\lowercase\expandafter{%
  \expandafter\def\expandafter\AllBytesHexLC
      \expandafter{\AllBytesHex}%
}
\begingroup
  \catcode`\#=12 %
  \xdef\AllBytesName{%
    #01#02#03#04#05#06#07#08#09#0A#0B#0C#0D#0E#0F%
    #10#11#12#13#14#15#16#17#18#19#1A#1B#1C#1D#1E#1F%
    #20!"#23$#25&'#28#29*+,-.#2F%
    0123456789:;#3C=#3E?%
    @ABCDEFGHIJKLMNO%
    PQRSTUVWXYZ#5B\ltx@backslashchar#5D^_%
    `abcdefghijklmno%
    pqrstuvwxyz#7B|#7D\string~#7F%
    #80#81#82#83#84#85#86#87#88#89#8A#8B#8C#8D#8E#8F%
    #90#91#92#93#94#95#96#97#98#99#9A#9B#9C#9D#9E#9F%
    #A0#A1#A2#A3#A4#A5#A6#A7#A8#A9#AA#AB#AC#AD#AE#AF%
    #B0#B1#B2#B3#B4#B5#B6#B7#B8#B9#BA#BB#BC#BD#BE#BF%
    #C0#C1#C2#C3#C4#C5#C6#C7#C8#C9#CA#CB#CC#CD#CE#CF%
    #D0#D1#D2#D3#D4#D5#D6#D7#D8#D9#DA#DB#DC#DD#DE#DF%
    #E0#E1#E2#E3#E4#E5#E6#E7#E8#E9#EA#EB#EC#ED#EE#EF%
    #F0#F1#F2#F3#F4#F5#F6#F7#F8#F9#FA#FB#FC#FD#FE#FF%
  }%
\endgroup
\ltx@onelevel@sanitize\AllBytesName
\edef\AllBytesFromName{\expandafter\ltx@gobble\AllBytes}
\begingroup
  \def\|{|}%
  \edef\%{\ltx@percentchar}%
  \catcode`\|=0 %
  \catcode`\#=12 %
  \catcode`\~=12 %
  \catcode`\\=12 %
  |xdef|AllBytesString{%
    \000\001\002\003\004\005\006\007\010\011\012\013\014\015\016\017%
    \020\021\022\023\024\025\026\027\030\031\032\033\034\035\036\037%
    \040!"#$|%&'\(\)*+,-./%
    0123456789:;<=>?%
    @ABCDEFGHIJKLMNO%
    PQRSTUVWXYZ[\\]^_%
    `abcdefghijklmno%
    pqrstuvwxyz{||}~\177%
    \200\201\202\203\204\205\206\207\210\211\212\213\214\215\216\217%
    \220\221\222\223\224\225\226\227\230\231\232\233\234\235\236\237%
    \240\241\242\243\244\245\246\247\250\251\252\253\254\255\256\257%
    \260\261\262\263\264\265\266\267\270\271\272\273\274\275\276\277%
    \300\301\302\303\304\305\306\307\310\311\312\313\314\315\316\317%
    \320\321\322\323\324\325\326\327\330\331\332\333\334\335\336\337%
    \340\341\342\343\344\345\346\347\350\351\352\353\354\355\356\357%
    \360\361\362\363\364\365\366\367\370\371\372\373\374\375\376\377%
  }%
|endgroup
\ltx@onelevel@sanitize\AllBytesString
%    \end{macrocode}
%    \begin{macrocode}
\def\Test#1#2#3{%
  \begingroup
    \expandafter\expandafter\expandafter\def
    \expandafter\expandafter\expandafter\TestResult
    \expandafter\expandafter\expandafter{%
      #1{#2}%
    }%
    \ifx\TestResult#3%
    \else
      \newlinechar=10 %
      \msg{Expect:^^J#3}%
      \msg{Result:^^J\TestResult}%
      \errmessage{\string#2 -\string#1-> \string#3}%
    \fi
  \endgroup
}
\def\test#1#2#3{%
  \edef\TestFrom{#2}%
  \edef\TestExpect{#3}%
  \ltx@onelevel@sanitize\TestExpect
  \Test#1\TestFrom\TestExpect
}
\test\pdf@unescapehex{74657374}{test}
\begingroup
  \catcode0=12 %
  \catcode1=12 %
  \test\pdf@unescapehex{740074017400740174}{t^^@t^^At^^@t^^At}%
\endgroup
\Test\pdf@escapehex\AllBytes\AllBytesHex
\Test\pdf@unescapehex\AllBytesHex\AllBytes
\Test\pdf@escapename\AllBytes\AllBytesName
\Test\pdf@escapestring\AllBytes\AllBytesString
%    \end{macrocode}
%    \begin{macrocode}
\csname @@end\endcsname\end
%</test-escape>
%    \end{macrocode}
%
% \section{Installation}
%
% \subsection{Download}
%
% \paragraph{Package.} This package is available on
% CTAN\footnote{\CTANpkg{pdftexcmds}}:
% \begin{description}
% \item[\CTAN{macros/latex/contrib/oberdiek/pdftexcmds.dtx}] The source file.
% \item[\CTAN{macros/latex/contrib/oberdiek/pdftexcmds.pdf}] Documentation.
% \end{description}
%
%
% \paragraph{Bundle.} All the packages of the bundle `oberdiek'
% are also available in a TDS compliant ZIP archive. There
% the packages are already unpacked and the documentation files
% are generated. The files and directories obey the TDS standard.
% \begin{description}
% \item[\CTANinstall{install/macros/latex/contrib/oberdiek.tds.zip}]
% \end{description}
% \emph{TDS} refers to the standard ``A Directory Structure
% for \TeX\ Files'' (\CTAN{tds/tds.pdf}). Directories
% with \xfile{texmf} in their name are usually organized this way.
%
% \subsection{Bundle installation}
%
% \paragraph{Unpacking.} Unpack the \xfile{oberdiek.tds.zip} in the
% TDS tree (also known as \xfile{texmf} tree) of your choice.
% Example (linux):
% \begin{quote}
%   |unzip oberdiek.tds.zip -d ~/texmf|
% \end{quote}
%
% \paragraph{Script installation.}
% Check the directory \xfile{TDS:scripts/oberdiek/} for
% scripts that need further installation steps.
% Package \xpackage{attachfile2} comes with the Perl script
% \xfile{pdfatfi.pl} that should be installed in such a way
% that it can be called as \texttt{pdfatfi}.
% Example (linux):
% \begin{quote}
%   |chmod +x scripts/oberdiek/pdfatfi.pl|\\
%   |cp scripts/oberdiek/pdfatfi.pl /usr/local/bin/|
% \end{quote}
%
% \subsection{Package installation}
%
% \paragraph{Unpacking.} The \xfile{.dtx} file is a self-extracting
% \docstrip\ archive. The files are extracted by running the
% \xfile{.dtx} through \plainTeX:
% \begin{quote}
%   \verb|tex pdftexcmds.dtx|
% \end{quote}
%
% \paragraph{TDS.} Now the different files must be moved into
% the different directories in your installation TDS tree
% (also known as \xfile{texmf} tree):
% \begin{quote}
% \def\t{^^A
% \begin{tabular}{@{}>{\ttfamily}l@{ $\rightarrow$ }>{\ttfamily}l@{}}
%   pdftexcmds.sty & tex/generic/oberdiek/pdftexcmds.sty\\
%   oberdiek.pdftexcmds.lua & scripts/oberdiek/oberdiek.pdftexcmds.lua\\
%   pdftexcmds.lua & scripts/oberdiek/pdftexcmds.lua\\
%   pdftexcmds.pdf & doc/latex/oberdiek/pdftexcmds.pdf\\
%   test/pdftexcmds-test1.tex & doc/latex/oberdiek/test/pdftexcmds-test1.tex\\
%   test/pdftexcmds-test2.tex & doc/latex/oberdiek/test/pdftexcmds-test2.tex\\
%   test/pdftexcmds-test-shell.tex & doc/latex/oberdiek/test/pdftexcmds-test-shell.tex\\
%   test/pdftexcmds-test-escape.tex & doc/latex/oberdiek/test/pdftexcmds-test-escape.tex\\
%   pdftexcmds.dtx & source/latex/oberdiek/pdftexcmds.dtx\\
% \end{tabular}^^A
% }^^A
% \sbox0{\t}^^A
% \ifdim\wd0>\linewidth
%   \begingroup
%     \advance\linewidth by\leftmargin
%     \advance\linewidth by\rightmargin
%   \edef\x{\endgroup
%     \def\noexpand\lw{\the\linewidth}^^A
%   }\x
%   \def\lwbox{^^A
%     \leavevmode
%     \hbox to \linewidth{^^A
%       \kern-\leftmargin\relax
%       \hss
%       \usebox0
%       \hss
%       \kern-\rightmargin\relax
%     }^^A
%   }^^A
%   \ifdim\wd0>\lw
%     \sbox0{\small\t}^^A
%     \ifdim\wd0>\linewidth
%       \ifdim\wd0>\lw
%         \sbox0{\footnotesize\t}^^A
%         \ifdim\wd0>\linewidth
%           \ifdim\wd0>\lw
%             \sbox0{\scriptsize\t}^^A
%             \ifdim\wd0>\linewidth
%               \ifdim\wd0>\lw
%                 \sbox0{\tiny\t}^^A
%                 \ifdim\wd0>\linewidth
%                   \lwbox
%                 \else
%                   \usebox0
%                 \fi
%               \else
%                 \lwbox
%               \fi
%             \else
%               \usebox0
%             \fi
%           \else
%             \lwbox
%           \fi
%         \else
%           \usebox0
%         \fi
%       \else
%         \lwbox
%       \fi
%     \else
%       \usebox0
%     \fi
%   \else
%     \lwbox
%   \fi
% \else
%   \usebox0
% \fi
% \end{quote}
% If you have a \xfile{docstrip.cfg} that configures and enables \docstrip's
% TDS installing feature, then some files can already be in the right
% place, see the documentation of \docstrip.
%
% \subsection{Refresh file name databases}
%
% If your \TeX~distribution
% (\teTeX, \mikTeX, \dots) relies on file name databases, you must refresh
% these. For example, \teTeX\ users run \verb|texhash| or
% \verb|mktexlsr|.
%
% \subsection{Some details for the interested}
%
% \paragraph{Unpacking with \LaTeX.}
% The \xfile{.dtx} chooses its action depending on the format:
% \begin{description}
% \item[\plainTeX:] Run \docstrip\ and extract the files.
% \item[\LaTeX:] Generate the documentation.
% \end{description}
% If you insist on using \LaTeX\ for \docstrip\ (really,
% \docstrip\ does not need \LaTeX), then inform the autodetect routine
% about your intention:
% \begin{quote}
%   \verb|latex \let\install=y% \iffalse meta-comment
%
% File: pdftexcmds.dtx
% Version: 2019/07/25 v0.30
% Info: Utility functions of pdfTeX for LuaTeX
%
% Copyright (C) 2007, 2009-2011 by
%    Heiko Oberdiek <heiko.oberdiek at googlemail.com>
%
% This work may be distributed and/or modified under the
% conditions of the LaTeX Project Public License, either
% version 1.3c of this license or (at your option) any later
% version. This version of this license is in
%    https://www.latex-project.org/lppl/lppl-1-3c.txt
% and the latest version of this license is in
%    https://www.latex-project.org/lppl.txt
% and version 1.3 or later is part of all distributions of
% LaTeX version 2005/12/01 or later.
%
% This work has the LPPL maintenance status "maintained".
%
% The Current Maintainers of this work are
% Heiko Oberdiek and the Oberdiek Package Support Group
% https://github.com/ho-tex/oberdiek/issues
%
% The Base Interpreter refers to any `TeX-Format',
% because some files are installed in TDS:tex/generic//.
%
% This work consists of the main source file pdftexcmds.dtx
% and the derived files
%    pdftexcmds.sty, pdftexcmds.pdf, pdftexcmds.ins, pdftexcmds.drv,
%    pdftexcmds.bib, pdftexcmds-test1.tex, pdftexcmds-test2.tex,
%    pdftexcmds-test-shell.tex, pdftexcmds-test-escape.tex,
%    oberdiek.pdftexcmds.lua, pdftexcmds.lua.
%
% Distribution:
%    CTAN:macros/latex/contrib/oberdiek/pdftexcmds.dtx
%    CTAN:macros/latex/contrib/oberdiek/pdftexcmds.pdf
%
% Unpacking:
%    (a) If pdftexcmds.ins is present:
%           tex pdftexcmds.ins
%    (b) Without pdftexcmds.ins:
%           tex pdftexcmds.dtx
%    (c) If you insist on using LaTeX
%           latex \let\install=y\input{pdftexcmds.dtx}
%        (quote the arguments according to the demands of your shell)
%
% Documentation:
%    (a) If pdftexcmds.drv is present:
%           latex pdftexcmds.drv
%    (b) Without pdftexcmds.drv:
%           latex pdftexcmds.dtx; ...
%    The class ltxdoc loads the configuration file ltxdoc.cfg
%    if available. Here you can specify further options, e.g.
%    use A4 as paper format:
%       \PassOptionsToClass{a4paper}{article}
%
%    Programm calls to get the documentation (example):
%       pdflatex pdftexcmds.dtx
%       bibtex pdftexcmds.aux
%       makeindex -s gind.ist pdftexcmds.idx
%       pdflatex pdftexcmds.dtx
%       makeindex -s gind.ist pdftexcmds.idx
%       pdflatex pdftexcmds.dtx
%
% Installation:
%    TDS:tex/generic/oberdiek/pdftexcmds.sty
%    TDS:scripts/oberdiek/oberdiek.pdftexcmds.lua
%    TDS:scripts/oberdiek/pdftexcmds.lua
%    TDS:doc/latex/oberdiek/pdftexcmds.pdf
%    TDS:doc/latex/oberdiek/test/pdftexcmds-test1.tex
%    TDS:doc/latex/oberdiek/test/pdftexcmds-test2.tex
%    TDS:doc/latex/oberdiek/test/pdftexcmds-test-shell.tex
%    TDS:doc/latex/oberdiek/test/pdftexcmds-test-escape.tex
%    TDS:source/latex/oberdiek/pdftexcmds.dtx
%
%<*ignore>
\begingroup
  \catcode123=1 %
  \catcode125=2 %
  \def\x{LaTeX2e}%
\expandafter\endgroup
\ifcase 0\ifx\install y1\fi\expandafter
         \ifx\csname processbatchFile\endcsname\relax\else1\fi
         \ifx\fmtname\x\else 1\fi\relax
\else\csname fi\endcsname
%</ignore>
%<*install>
\input docstrip.tex
\Msg{************************************************************************}
\Msg{* Installation}
\Msg{* Package: pdftexcmds 2019/07/25 v0.30 Utility functions of pdfTeX for LuaTeX (HO)}
\Msg{************************************************************************}

\keepsilent
\askforoverwritefalse

\let\MetaPrefix\relax
\preamble

This is a generated file.

Project: pdftexcmds
Version: 2019/07/25 v0.30

Copyright (C) 2007, 2009-2011 by
   Heiko Oberdiek <heiko.oberdiek at googlemail.com>

This work may be distributed and/or modified under the
conditions of the LaTeX Project Public License, either
version 1.3c of this license or (at your option) any later
version. This version of this license is in
   https://www.latex-project.org/lppl/lppl-1-3c.txt
and the latest version of this license is in
   https://www.latex-project.org/lppl.txt
and version 1.3 or later is part of all distributions of
LaTeX version 2005/12/01 or later.

This work has the LPPL maintenance status "maintained".

The Current Maintainers of this work are
Heiko Oberdiek and the Oberdiek Package Support Group
https://github.com/ho-tex/oberdiek/issues


The Base Interpreter refers to any `TeX-Format',
because some files are installed in TDS:tex/generic//.

This work consists of the main source file pdftexcmds.dtx
and the derived files
   pdftexcmds.sty, pdftexcmds.pdf, pdftexcmds.ins, pdftexcmds.drv,
   pdftexcmds.bib, pdftexcmds-test1.tex, pdftexcmds-test2.tex,
   pdftexcmds-test-shell.tex, pdftexcmds-test-escape.tex,
   oberdiek.pdftexcmds.lua, pdftexcmds.lua.

\endpreamble
\let\MetaPrefix\DoubleperCent

\generate{%
  \file{pdftexcmds.ins}{\from{pdftexcmds.dtx}{install}}%
  \file{pdftexcmds.drv}{\from{pdftexcmds.dtx}{driver}}%
  \nopreamble
  \nopostamble
  \file{pdftexcmds.bib}{\from{pdftexcmds.dtx}{bib}}%
  \usepreamble\defaultpreamble
  \usepostamble\defaultpostamble
  \usedir{tex/generic/oberdiek}%
  \file{pdftexcmds.sty}{\from{pdftexcmds.dtx}{package}}%
%  \usedir{doc/latex/oberdiek/test}%
%  \file{pdftexcmds-test1.tex}{\from{pdftexcmds.dtx}{test1}}%
%  \file{pdftexcmds-test2.tex}{\from{pdftexcmds.dtx}{test2}}%
%  \file{pdftexcmds-test-shell.tex}{\from{pdftexcmds.dtx}{test-shell}}%
%  \file{pdftexcmds-test-escape.tex}{\from{pdftexcmds.dtx}{test-escape}}%
  \nopreamble
  \nopostamble
%  \usedir{source/latex/oberdiek/catalogue}%
%  \file{pdftexcmds.xml}{\from{pdftexcmds.dtx}{catalogue}}%
}
\def\MetaPrefix{-- }
\def\defaultpostamble{%
  \MetaPrefix^^J%
  \MetaPrefix\space End of File `\outFileName'.%
}
\def\currentpostamble{\defaultpostamble}%
\generate{%
  \usedir{scripts/oberdiek}%
  \file{oberdiek.pdftexcmds.lua}{\from{pdftexcmds.dtx}{lua}}%
  \file{pdftexcmds.lua}{\from{pdftexcmds.dtx}{lua}}%
}

\catcode32=13\relax% active space
\let =\space%
\Msg{************************************************************************}
\Msg{*}
\Msg{* To finish the installation you have to move the following}
\Msg{* file into a directory searched by TeX:}
\Msg{*}
\Msg{*     pdftexcmds.sty}
\Msg{*}
\Msg{* And install the following script files:}
\Msg{*}
\Msg{*     oberdiek.pdftexcmds.lua, pdftexcmds.lua}
\Msg{*}
\Msg{* To produce the documentation run the file `pdftexcmds.drv'}
\Msg{* through LaTeX.}
\Msg{*}
\Msg{* Happy TeXing!}
\Msg{*}
\Msg{************************************************************************}

\endbatchfile
%</install>
%<*bib>
@online{AndyThomas:Analog,
  author={Thomas, Andy},
  title={Analog of {\texttt{\csname textbackslash\endcsname}pdfelapsedtime} for
      {\hologo{LuaTeX}} and {\hologo{XeTeX}}},
  url={http://tex.stackexchange.com/a/32531},
  urldate={2011-11-29},
}
%</bib>
%<*ignore>
\fi
%</ignore>
%<*driver>
\NeedsTeXFormat{LaTeX2e}
\ProvidesFile{pdftexcmds.drv}%
  [2019/07/25 v0.30 Utility functions of pdfTeX for LuaTeX (HO)]%
\documentclass{ltxdoc}
\usepackage{holtxdoc}[2011/11/22]
\usepackage{paralist}
\usepackage{csquotes}
\usepackage[
  backend=bibtex,
  bibencoding=ascii,
  alldates=iso8601,
]{biblatex}[2011/11/13]
\bibliography{oberdiek-source}
\bibliography{pdftexcmds}
\begin{document}
  \DocInput{pdftexcmds.dtx}%
\end{document}
%</driver>
% \fi
%
%
% \CharacterTable
%  {Upper-case    \A\B\C\D\E\F\G\H\I\J\K\L\M\N\O\P\Q\R\S\T\U\V\W\X\Y\Z
%   Lower-case    \a\b\c\d\e\f\g\h\i\j\k\l\m\n\o\p\q\r\s\t\u\v\w\x\y\z
%   Digits        \0\1\2\3\4\5\6\7\8\9
%   Exclamation   \!     Double quote  \"     Hash (number) \#
%   Dollar        \$     Percent       \%     Ampersand     \&
%   Acute accent  \'     Left paren    \(     Right paren   \)
%   Asterisk      \*     Plus          \+     Comma         \,
%   Minus         \-     Point         \.     Solidus       \/
%   Colon         \:     Semicolon     \;     Less than     \<
%   Equals        \=     Greater than  \>     Question mark \?
%   Commercial at \@     Left bracket  \[     Backslash     \\
%   Right bracket \]     Circumflex    \^     Underscore    \_
%   Grave accent  \`     Left brace    \{     Vertical bar  \|
%   Right brace   \}     Tilde         \~}
%
% \GetFileInfo{pdftexcmds.drv}
%
% \title{The \xpackage{pdftexcmds} package}
% \date{2019/07/25 v0.30}
% \author{Heiko Oberdiek\thanks
% {Please report any issues at \url{https://github.com/ho-tex/oberdiek/issues}}}
%
% \maketitle
%
% \begin{abstract}
% \hologo{LuaTeX} provides most of the commands of \hologo{pdfTeX} 1.40. However
% a number of utility functions are removed. This package tries to fill
% the gap and implements some of the missing primitive using Lua.
% \end{abstract}
%
% \tableofcontents
%
% \def\csi#1{\texttt{\textbackslash\textit{#1}}}
%
% \section{Documentation}
%
% Some primitives of \hologo{pdfTeX} \cite{pdftex-manual}
% are not defined by \hologo{LuaTeX} \cite{luatex-manual}.
% This package implements macro based solutions using Lua code
% for the following missing \hologo{pdfTeX} primitives;
% \begin{compactitem}
% \item \cs{pdfstrcmp}
% \item \cs{pdfunescapehex}
% \item \cs{pdfescapehex}
% \item \cs{pdfescapename}
% \item \cs{pdfescapestring}
% \item \cs{pdffilesize}
% \item \cs{pdffilemoddate}
% \item \cs{pdffiledump}
% \item \cs{pdfmdfivesum}
% \item \cs{pdfresettimer}
% \item \cs{pdfelapsedtime}
% \item |\immediate\write18|
% \end{compactitem}
% The original names of the primitives cannot be used:
% \begin{itemize}
% \item
% The syntax for their arguments cannot easily
% simulated by macros. The primitives using key words
% such as |file| (\cs{pdfmdfivesum}) or |offset| and |length|
% (\cs{pdffiledump}) and uses \meta{general text} for the other
% arguments. Using token registers assignments, \meta{general text} could
% be catched. However, the simulated primitives are expandable
% and register assignments would destroy this important property.
% (\meta{general text} allows something like |\expandafter\bgroup ...}|.)
% \item
% The original primitives can be expanded using one expansion step.
% The new macros need two expansion steps because of the additional
% macro expansion. Example:
% \begin{quote}
%   |\expandafter\foo\pdffilemoddate{file}|\\
%   vs.\\
%   |\expandafter\expandafter\expandafter|\\
%   |\foo\pdf@filemoddate{file}|
% \end{quote}
% \end{itemize}
%
% \hologo{LuaTeX} isn't stable yet and thus the status of this package is
% \emph{experimental}. Feedback is welcome.
%
% \subsection{General principles}
%
% \begin{description}
% \item[Naming convention:]
%   Usually this package defines a macro |\pdf@|\meta{cmd} if
%   \hologo{pdfTeX} provides |\pdf|\meta{cmd}.
% \item[Arguments:] The order of arguments in |\pdf@|\meta{cmd}
%   is the same as for the corresponding primitive of \hologo{pdfTeX}.
%   The arguments are ordinary undelimited \hologo{TeX} arguments,
%   no \meta{general text} and without additional keywords.
% \item[Expandibility:]
%   The macro |\pdf@|\meta{cmd} is expandable if the
%   corresponding \hologo{pdfTeX} primitive has this property.
%   Exact two expansion steps are necessary (first is the macro
%   expansion) except for \cs{pdf@primitive} and \cs{pdf@ifprimitive}.
%   The latter ones are not macros, but have the direct meaning of the
%   primitive.
% \item[Without \hologo{LuaTeX}:]
%   The macros |\pdf@|\meta{cmd} are mapped to the commands
%   of \hologo{pdfTeX} if they are available. Otherwise they are undefined.
% \item[Availability:]
%   The macros that the packages provides are undefined, if
%   the necessary primitives are not found and cannot be
%   implemented by Lua.
% \end{description}
%
% \subsection{Macros}
%
% \subsubsection[Strings]{Strings \cite[``7.15 Strings'']{pdftex-manual}}
%
% \begin{declcs}{pdf@strcmp} \M{stringA} \M{stringB}
% \end{declcs}
% Same as |\pdfstrcmp{|\meta{stringA}|}{|\meta{stringB}|}|.
%
% \begin{declcs}{pdf@unescapehex} \M{string}
% \end{declcs}
% Same as |\pdfunescapehex{|\meta{string}|}|.
% The argument is a byte string given in hexadecimal notation.
% The result are character tokens from 0 until 255 with
% catcode 12 and the space with catcode 10.
%
% \begin{declcs}{pdf@escapehex} \M{string}\\
%   \cs{pdf@escapestring} \M{string}\\
%   \cs{pdf@escapename} \M{string}
% \end{declcs}
% Same as the primitives of \hologo{pdfTeX}. However \hologo{pdfTeX} does not
% know about characters with codes 256 and larger. Thus the
% string is treated as byte string, characters with more than
% eight bits are ignored.
%
% \subsubsection[Files]{Files \cite[``7.18 Files'']{pdftex-manual}}
%
% \begin{declcs}{pdf@filesize} \M{filename}
% \end{declcs}
% Same as |\pdffilesize{|\meta{filename}|}|.
%
% \begin{declcs}{pdf@filemoddate} \M{filename}
% \end{declcs}
% Same as |\pdffilemoddate{|\meta{filename}|}|.
%
% \begin{declcs}{pdf@filedump} \M{offset} \M{length} \M{filename}
% \end{declcs}
% Same as |\pdffiledump offset| \meta{offset} |length| \meta{length}
% |{|\meta{filename}|}|. Both \meta{offset} and \meta{length} must
% not be empty, but must be a valid \hologo{TeX} number.
%
% \begin{declcs}{pdf@mdfivesum} \M{string}
% \end{declcs}
% Same as |\pdfmdfivesum{|\meta{string}|}|. Keyword |file| is supported
% by macro \cs{pdf@filemdfivesum}.
%
% \begin{declcs}{pdf@filemdfivesum} \M{filename}
% \end{declcs}
% Same as |\pdfmdfivesum file{|\meta{filename}|}|.
%
% \subsubsection[Timekeeping]{Timekeeping \cite[``7.17 Timekeeping'']{pdftex-manual}}
%
% The timekeeping macros are based on Andy Thomas' work \cite{AndyThomas:Analog}.
%
% \begin{declcs}{pdf@resettimer}
% \end{declcs}
% Same as \cs{pdfresettimer}, it resets the internal timer.
%
% \begin{declcs}{pdf@elapsedtime}
% \end{declcs}
% Same as \cs{pdfelapsedtime}. It behaves like a read-only integer.
% For printing purposes it can be prefixed by \cs{the} or \cs{number}.
% It measures the time in scaled seconds (seconds multiplied with 65536)
% since the latest call of \cs{pdf@resettimer} or start of
% program/package. The resolution, the shortest time interval that
% can be measured, depends on the program and system.
% \begin{itemize}
% \item \hologo{pdfTeX} with |gettimeofday|: $\ge$ 1/65536\,s
% \item \hologo{pdfTeX} with |ftime|: $\ge$ 1\,ms
% \item \hologo{pdfTeX} with |time|: $\ge$ 1\,s
% \item \hologo{LuaTeX}: $\ge$ 10\,ms\\
%  (|os.clock()| returns a float number with two decimal digits in
%  \hologo{LuaTeX} beta-0.70.1-2011061416 (rev 4277)).
% \end{itemize}
%
% \subsubsection[Miscellaneous]{Miscellaneous \cite[``7.21 Miscellaneous'']{pdftex-manual}}
%
% \begin{declcs}{pdf@draftmode}
% \end{declcs}
% If the \TeX\ compiler knows \cs{pdfdraftmode} or \cs{draftmode}
% (\hologo{pdfTeX},
% \hologo{LuaTeX}), then \cs{pdf@draftmode} returns, whether
% this mode is enabled. The result is an implicit number:
% one means the draft mode is available and enabled.
% If the value is zero, then the mode is not active or
% \cs{pdfdraftmode} is not available.
% An explicit number is yielded by \cs{number}\cs{pdf@draftmode}.
% The macro cannot
% be used to change the mode, see \cs{pdf@setdraftmode}.
%
% \begin{declcs}{pdf@ifdraftmode} \M{true} \M{false}
% \end{declcs}
% If \cs{pdfdraftmode} is available and enabled, \meta{true} is
% called, otherwise \meta{false} is executed.
%
% \begin{declcs}{pdf@setdraftmode} \M{value}
% \end{declcs}
% Macro \cs{pdf@setdraftmode} expects the number zero or one as
% \meta{value}. Zero deactivates the mode and one enables the draft mode.
% The macro does not have an effect, if the feature \cs{pdfdraftmode} is not
% available.
%
% \begin{declcs}{pdf@shellescape}
% \end{declcs}
% Same as |\pdfshellescape|. It is or expands to |1| if external
% commands can be executed and |0| otherwise. In \hologo{pdfTeX} external
% commands must be enabled first by command line option or
% configuration option. In \hologo{LuaTeX} option |--safer| disables
% the execution of external commands.
%
% In \hologo{LuaTeX} before 0.68.0 \cs{pdf@shellescape} is not
% available due to a bug in |os.execute()|. The argumentless form
% crashes in some circumstances with segmentation fault.
% (It is fixed in version 0.68.0 or revision 4167 of \hologo{LuaTeX}.
% and packported to some version of 0.67.0).
%
% Hints for usage:
% \begin{itemize}
% \item Before its use \cs{pdf@shellescape} should be tested,
% whether it is available. Example with package \xpackage{ltxcmds}
% (loaded by package \xpackage{pdftexcmds}):
%\begin{quote}
%\begin{verbatim}
%\ltx@IfUndefined{pdf@shellescape}{%
%  % \pdf@shellescape is undefined
%}{%
%  % \pdf@shellescape is available
%}
%\end{verbatim}
%\end{quote}
% Use \cs{ltx@ifundefined} in expandable contexts.
% \item \cs{pdf@shellescape} might be a numerical constant,
% expands to the primitive, or expands to a plain number.
% Therefore use it in contexts where these differences does not matter.
% \item Use in comparisons, e.g.:
%   \begin{quote}
%     |\ifnum\pdf@shellescape=0 ...|
%   \end{quote}
% \item Print the number: |\number\pdf@shellescape|
% \end{itemize}
%
% \begin{declcs}{pdf@system} \M{cmdline}
% \end{declcs}
% It is a wrapper for |\immediate\write18| in \hologo{pdfTeX} or
% |os.execute| in \hologo{LuaTeX}.
%
% In theory |os.execute|
% returns a status number. But its meaning is quite
% undefined. Are there some reliable properties?
% Does it make sense to provide an user interface to
% this status exit code?
%
% \begin{declcs}{pdf@primitive} \csi{cmd}
% \end{declcs}
% Same as \cs{pdfprimitive} in \hologo{pdfTeX} or \hologo{LuaTeX}.
% In \hologo{XeTeX} the
% primitive is called \cs{primitive}. Despite the current definition
% of the command \csi{cmd}, it's meaning as primitive is used.
%
% \begin{declcs}{pdf@ifprimitive} \csi{cmd}
% \end{declcs}
% Same as \cs{ifpdfprimitive} in \hologo{pdfTeX} or
% \hologo{LuaTeX}. \hologo{XeTeX} calls
% it \cs{ifprimitive}. It is a switch that checks if the command
% \csi{cmd} has it's primitive meaning.
%
% \subsubsection{Additional macro: \cs{pdf@isprimitive}}
%
% \begin{declcs}{pdf@isprimitive} \csi{cmd1} \csi{cmd2} \M{true} \M{false}
% \end{declcs}
% If \csi{cmd1} has the primitive meaning given by the primitive name
% of \csi{cmd2}, then the argument \meta{true} is executed, otherwise
% \meta{false}. The macro \cs{pdf@isprimitive} is expandable.
% Internally it checks the result of \cs{meaning} and is therefore
% available for all \hologo{TeX} variants, even the original \hologo{TeX}.
% Example with \hologo{LaTeX}:
%\begin{quote}
%\begin{verbatim}
%\makeatletter
%\pdf@isprimitive{@@input}{input}{%
%  \typeout{\string\@@input\space is original\string\input}%
%}{%
%  \typeout{Oops, \string\@@input\space is not the %
%           original\string\input}%
%}
%\end{verbatim}
%\end{quote}
%
% \subsubsection{Experimental}
%
% \begin{declcs}{pdf@unescapehexnative} \M{string}\\
%   \cs{pdf@escapehexnative} \M{string}\\
%   \cs{pdf@escapenamenative} \M{string}\\
%   \cs{pdf@mdfivesumnative} \M{string}
% \end{declcs}
% The variants without |native| in the macro name are supposed to
% be compatible with \hologo{pdfTeX}. However characters with more than
% eight bits are not supported and are ignored. If \hologo{LuaTeX} is
% running, then its UTF-8 coded strings are used. Thus the full
% unicode character range is supported. However the result
% differs from \hologo{pdfTeX} for characters with eight or more bits.
%
% \begin{declcs}{pdf@pipe} \M{cmdline}
% \end{declcs}
% It calls \meta{cmdline} and returns the output of the external
% program in the usual manner as byte string (catcode 12, space with
% catcode 10). The Lua documentation says, that the used |io.popen|
% may not be available on all platforms. Then macro \cs{pdf@pipe}
% is undefined.
%
% \StopEventually{
% }
%
% \section{Implementation}
%
%    \begin{macrocode}
%<*package>
%    \end{macrocode}
%
% \subsection{Reload check and package identification}
%    Reload check, especially if the package is not used with \LaTeX.
%    \begin{macrocode}
\begingroup\catcode61\catcode48\catcode32=10\relax%
  \catcode13=5 % ^^M
  \endlinechar=13 %
  \catcode35=6 % #
  \catcode39=12 % '
  \catcode44=12 % ,
  \catcode45=12 % -
  \catcode46=12 % .
  \catcode58=12 % :
  \catcode64=11 % @
  \catcode123=1 % {
  \catcode125=2 % }
  \expandafter\let\expandafter\x\csname ver@pdftexcmds.sty\endcsname
  \ifx\x\relax % plain-TeX, first loading
  \else
    \def\empty{}%
    \ifx\x\empty % LaTeX, first loading,
      % variable is initialized, but \ProvidesPackage not yet seen
    \else
      \expandafter\ifx\csname PackageInfo\endcsname\relax
        \def\x#1#2{%
          \immediate\write-1{Package #1 Info: #2.}%
        }%
      \else
        \def\x#1#2{\PackageInfo{#1}{#2, stopped}}%
      \fi
      \x{pdftexcmds}{The package is already loaded}%
      \aftergroup\endinput
    \fi
  \fi
\endgroup%
%    \end{macrocode}
%    Package identification:
%    \begin{macrocode}
\begingroup\catcode61\catcode48\catcode32=10\relax%
  \catcode13=5 % ^^M
  \endlinechar=13 %
  \catcode35=6 % #
  \catcode39=12 % '
  \catcode40=12 % (
  \catcode41=12 % )
  \catcode44=12 % ,
  \catcode45=12 % -
  \catcode46=12 % .
  \catcode47=12 % /
  \catcode58=12 % :
  \catcode64=11 % @
  \catcode91=12 % [
  \catcode93=12 % ]
  \catcode123=1 % {
  \catcode125=2 % }
  \expandafter\ifx\csname ProvidesPackage\endcsname\relax
    \def\x#1#2#3[#4]{\endgroup
      \immediate\write-1{Package: #3 #4}%
      \xdef#1{#4}%
    }%
  \else
    \def\x#1#2[#3]{\endgroup
      #2[{#3}]%
      \ifx#1\@undefined
        \xdef#1{#3}%
      \fi
      \ifx#1\relax
        \xdef#1{#3}%
      \fi
    }%
  \fi
\expandafter\x\csname ver@pdftexcmds.sty\endcsname
\ProvidesPackage{pdftexcmds}%
  [2019/07/25 v0.30 Utility functions of pdfTeX for LuaTeX (HO)]%
%    \end{macrocode}
%
% \subsection{Catcodes}
%
%    \begin{macrocode}
\begingroup\catcode61\catcode48\catcode32=10\relax%
  \catcode13=5 % ^^M
  \endlinechar=13 %
  \catcode123=1 % {
  \catcode125=2 % }
  \catcode64=11 % @
  \def\x{\endgroup
    \expandafter\edef\csname pdftexcmds@AtEnd\endcsname{%
      \endlinechar=\the\endlinechar\relax
      \catcode13=\the\catcode13\relax
      \catcode32=\the\catcode32\relax
      \catcode35=\the\catcode35\relax
      \catcode61=\the\catcode61\relax
      \catcode64=\the\catcode64\relax
      \catcode123=\the\catcode123\relax
      \catcode125=\the\catcode125\relax
    }%
  }%
\x\catcode61\catcode48\catcode32=10\relax%
\catcode13=5 % ^^M
\endlinechar=13 %
\catcode35=6 % #
\catcode64=11 % @
\catcode123=1 % {
\catcode125=2 % }
\def\TMP@EnsureCode#1#2{%
  \edef\pdftexcmds@AtEnd{%
    \pdftexcmds@AtEnd
    \catcode#1=\the\catcode#1\relax
  }%
  \catcode#1=#2\relax
}
\TMP@EnsureCode{0}{12}%
\TMP@EnsureCode{1}{12}%
\TMP@EnsureCode{2}{12}%
\TMP@EnsureCode{10}{12}% ^^J
\TMP@EnsureCode{33}{12}% !
\TMP@EnsureCode{34}{12}% "
\TMP@EnsureCode{38}{4}% &
\TMP@EnsureCode{39}{12}% '
\TMP@EnsureCode{40}{12}% (
\TMP@EnsureCode{41}{12}% )
\TMP@EnsureCode{42}{12}% *
\TMP@EnsureCode{43}{12}% +
\TMP@EnsureCode{44}{12}% ,
\TMP@EnsureCode{45}{12}% -
\TMP@EnsureCode{46}{12}% .
\TMP@EnsureCode{47}{12}% /
\TMP@EnsureCode{58}{12}% :
\TMP@EnsureCode{60}{12}% <
\TMP@EnsureCode{62}{12}% >
\TMP@EnsureCode{91}{12}% [
\TMP@EnsureCode{93}{12}% ]
\TMP@EnsureCode{94}{7}% ^ (superscript)
\TMP@EnsureCode{95}{12}% _ (other)
\TMP@EnsureCode{96}{12}% `
\TMP@EnsureCode{126}{12}% ~ (other)
\edef\pdftexcmds@AtEnd{%
  \pdftexcmds@AtEnd
  \escapechar=\number\escapechar\relax
  \noexpand\endinput
}
\escapechar=92 %
%    \end{macrocode}
%
% \subsection{Load packages}
%
%    \begin{macrocode}
\begingroup\expandafter\expandafter\expandafter\endgroup
\expandafter\ifx\csname RequirePackage\endcsname\relax
  \def\TMP@RequirePackage#1[#2]{%
    \begingroup\expandafter\expandafter\expandafter\endgroup
    \expandafter\ifx\csname ver@#1.sty\endcsname\relax
      \input #1.sty\relax
    \fi
  }%
  \TMP@RequirePackage{infwarerr}[2007/09/09]%
  \TMP@RequirePackage{ifluatex}[2010/03/01]%
  \TMP@RequirePackage{ltxcmds}[2010/12/02]%
  \TMP@RequirePackage{ifpdf}[2010/09/13]%
\else
  \RequirePackage{infwarerr}[2007/09/09]%
  \RequirePackage{ifluatex}[2010/03/01]%
  \RequirePackage{ltxcmds}[2010/12/02]%
  \RequirePackage{ifpdf}[2010/09/13]%
\fi
%    \end{macrocode}
%
% \subsection{Without \hologo{LuaTeX}}
%
%    \begin{macrocode}
\ifluatex
\else
  \@PackageInfoNoLine{pdftexcmds}{LuaTeX not detected}%
  \def\pdftexcmds@nopdftex{%
    \@PackageInfoNoLine{pdftexcmds}{pdfTeX >= 1.30 not detected}%
    \let\pdftexcmds@nopdftex\relax
  }%
  \def\pdftexcmds@temp#1{%
    \begingroup\expandafter\expandafter\expandafter\endgroup
    \expandafter\ifx\csname pdf#1\endcsname\relax
      \pdftexcmds@nopdftex
    \else
      \expandafter\def\csname pdf@#1\expandafter\endcsname
      \expandafter##\expandafter{%
        \csname pdf#1\endcsname
      }%
    \fi
  }%
  \pdftexcmds@temp{strcmp}%
  \pdftexcmds@temp{escapehex}%
  \let\pdf@escapehexnative\pdf@escapehex
  \pdftexcmds@temp{unescapehex}%
  \let\pdf@unescapehexnative\pdf@unescapehex
  \pdftexcmds@temp{escapestring}%
  \pdftexcmds@temp{escapename}%
  \pdftexcmds@temp{filesize}%
  \pdftexcmds@temp{filemoddate}%
  \begingroup\expandafter\expandafter\expandafter\endgroup
  \expandafter\ifx\csname pdfshellescape\endcsname\relax
    \pdftexcmds@nopdftex
    \ltx@IfUndefined{pdftexversion}{%
    }{%
      \ifnum\pdftexversion>120 % 1.21a supports \ifeof18
        \ifeof18 %
          \chardef\pdf@shellescape=0 %
        \else
          \chardef\pdf@shellescape=1 %
        \fi
      \fi
    }%
  \else
    \def\pdf@shellescape{%
      \pdfshellescape
    }%
  \fi
  \begingroup\expandafter\expandafter\expandafter\endgroup
  \expandafter\ifx\csname pdffiledump\endcsname\relax
    \pdftexcmds@nopdftex
  \else
    \def\pdf@filedump#1#2#3{%
      \pdffiledump offset#1 length#2{#3}%
    }%
  \fi
%    \end{macrocode}
%    \begin{macrocode}
  \begingroup\expandafter\expandafter\expandafter\endgroup
  \expandafter\ifx\csname pdfmdfivesum\endcsname\relax
    \begingroup\expandafter\expandafter\expandafter\endgroup
    \expandafter\ifx\csname mdfivesum\endcsname\relax
      \pdftexcmds@nopdftex
    \else
      \def\pdf@mdfivesum#{\mdfivesum}%
      \let\pdf@mdfivesumnative\pdf@mdfivesum
      \def\pdf@filemdfivesum#{\mdfivesum file}%
    \fi
  \else
    \def\pdf@mdfivesum#{\pdfmdfivesum}%
    \let\pdf@mdfivesumnative\pdf@mdfivesum
    \def\pdf@filemdfivesum#{\pdfmdfivesum file}%
  \fi
%    \end{macrocode}
%    \begin{macrocode}
  \def\pdf@system#{%
    \immediate\write18%
  }%
  \def\pdftexcmds@temp#1{%
    \begingroup\expandafter\expandafter\expandafter\endgroup
    \expandafter\ifx\csname pdf#1\endcsname\relax
      \pdftexcmds@nopdftex
    \else
      \expandafter\let\csname pdf@#1\expandafter\endcsname
      \csname pdf#1\endcsname
    \fi
  }%
  \pdftexcmds@temp{resettimer}%
  \pdftexcmds@temp{elapsedtime}%
\fi
%    \end{macrocode}
%
% \subsection{\cs{pdf@primitive}, \cs{pdf@ifprimitive}}
%
%    Since version 1.40.0 \hologo{pdfTeX} has \cs{pdfprimitive} and
%    \cs{ifpdfprimitive}. And \cs{pdfprimitive} was fixed in
%    version 1.40.4.
%
%    \hologo{XeTeX} provides them under the name \cs{primitive} and
%    \cs{ifprimitive}. \hologo{LuaTeX} knows both name variants,
%    but they have possibly to be enabled first (|tex.enableprimitives|).
%
%    Depending on the format TeX Live uses a prefix |luatex|.
%
%    Caution: \cs{let} must be used for the definition of
%    the macros, especially because of \cs{ifpdfprimitive}.
%
% \subsubsection{Using \hologo{LuaTeX}'s \texttt{tex.enableprimitives}}
%
%    \begin{macrocode}
\ifluatex
%    \end{macrocode}
%    \begin{macro}{\pdftexcmds@directlua}
%    \begin{macrocode}
  \ifnum\luatexversion<36 %
    \def\pdftexcmds@directlua{\directlua0 }%
  \else
    \let\pdftexcmds@directlua\directlua
  \fi
%    \end{macrocode}
%    \end{macro}
%
%    \begin{macrocode}
  \begingroup
    \newlinechar=10 %
    \endlinechar=\newlinechar
    \pdftexcmds@directlua{%
      if tex.enableprimitives then
        tex.enableprimitives(
          'pdf@',
          {'primitive', 'ifprimitive', 'pdfdraftmode','draftmode'}
        )
        tex.enableprimitives('', {'luaescapestring'})
      end
    }%
  \endgroup %
%    \end{macrocode}
%
%    \begin{macrocode}
\fi
%    \end{macrocode}
%
% \subsubsection{Trying various names to find the primitives}
%
%    \begin{macro}{\pdftexcmds@strip@prefix}
%    \begin{macrocode}
\def\pdftexcmds@strip@prefix#1>{}
%    \end{macrocode}
%    \end{macro}
%    \begin{macrocode}
\def\pdftexcmds@temp#1#2#3{%
  \begingroup\expandafter\expandafter\expandafter\endgroup
  \expandafter\ifx\csname pdf@#1\endcsname\relax
    \begingroup
      \def\x{#3}%
      \edef\x{\expandafter\pdftexcmds@strip@prefix\meaning\x}%
      \escapechar=-1 %
      \edef\y{\expandafter\meaning\csname#2\endcsname}%
    \expandafter\endgroup
    \ifx\x\y
      \expandafter\let\csname pdf@#1\expandafter\endcsname
      \csname #2\endcsname
    \fi
  \fi
}
%    \end{macrocode}
%
%    \begin{macro}{\pdf@primitive}
%    \begin{macrocode}
\pdftexcmds@temp{primitive}{pdfprimitive}{pdfprimitive}% pdfTeX, oldLuaTeX
\pdftexcmds@temp{primitive}{primitive}{primitive}% XeTeX, luatex
\pdftexcmds@temp{primitive}{luatexprimitive}{pdfprimitive}% oldLuaTeX
\pdftexcmds@temp{primitive}{luatexpdfprimitive}{pdfprimitive}% oldLuaTeX
%    \end{macrocode}
%    \end{macro}
%    \begin{macro}{\pdf@ifprimitive}
%    \begin{macrocode}
\pdftexcmds@temp{ifprimitive}{ifpdfprimitive}{ifpdfprimitive}% pdfTeX, oldLuaTeX
\pdftexcmds@temp{ifprimitive}{ifprimitive}{ifprimitive}% XeTeX, luatex
\pdftexcmds@temp{ifprimitive}{luatexifprimitive}{ifpdfprimitive}% oldLuaTeX
\pdftexcmds@temp{ifprimitive}{luatexifpdfprimitive}{ifpdfprimitive}% oldLuaTeX
%    \end{macrocode}
%    \end{macro}
%
%    Disable broken \cs{pdfprimitive}.
%    \begin{macrocode}
\ifluatex\else
\begingroup
  \expandafter\ifx\csname pdf@primitive\endcsname\relax
  \else
    \expandafter\ifx\csname pdftexversion\endcsname\relax
    \else
      \ifnum\pdftexversion=140 %
        \expandafter\ifx\csname pdftexrevision\endcsname\relax
        \else
          \ifnum\pdftexrevision<4 %
            \endgroup
            \let\pdf@primitive\@undefined
            \@PackageInfoNoLine{pdftexcmds}{%
              \string\pdf@primitive\space disabled, %
              because\MessageBreak
              \string\pdfprimitive\space is broken until pdfTeX 1.40.4%
            }%
            \begingroup
          \fi
        \fi
      \fi
    \fi
  \fi
\endgroup
\fi
%    \end{macrocode}
%
% \subsubsection{Result}
%
%    \begin{macrocode}
\begingroup
  \@PackageInfoNoLine{pdftexcmds}{%
    \string\pdf@primitive\space is %
    \expandafter\ifx\csname pdf@primitive\endcsname\relax not \fi
    available%
  }%
  \@PackageInfoNoLine{pdftexcmds}{%
    \string\pdf@ifprimitive\space is %
    \expandafter\ifx\csname pdf@ifprimitive\endcsname\relax not \fi
    available%
  }%
\endgroup
%    \end{macrocode}
%
% \subsection{\hologo{XeTeX}}
%
%    Look for primitives \cs{shellescape}, \cs{strcmp}.
%    \begin{macrocode}
\def\pdftexcmds@temp#1{%
  \begingroup\expandafter\expandafter\expandafter\endgroup
  \expandafter\ifx\csname pdf@#1\endcsname\relax
    \begingroup
      \escapechar=-1 %
      \edef\x{\expandafter\meaning\csname#1\endcsname}%
      \def\y{#1}%
      \def\z##1->{}%
      \edef\y{\expandafter\z\meaning\y}%
    \expandafter\endgroup
    \ifx\x\y
      \expandafter\def\csname pdf@#1\expandafter\endcsname
      \expandafter{%
        \csname#1\endcsname
      }%
    \fi
  \fi
}%
\pdftexcmds@temp{shellescape}%
\pdftexcmds@temp{strcmp}%
%    \end{macrocode}
%
% \subsection{\cs{pdf@isprimitive}}
%
%    \begin{macrocode}
\def\pdf@isprimitive{%
  \begingroup\expandafter\expandafter\expandafter\endgroup
  \expandafter\ifx\csname pdf@strcmp\endcsname\relax
    \long\def\pdf@isprimitive##1{%
      \expandafter\pdftexcmds@isprimitive\expandafter{\meaning##1}%
    }%
    \long\def\pdftexcmds@isprimitive##1##2{%
      \expandafter\pdftexcmds@@isprimitive\expandafter{\string##2}{##1}%
    }%
    \def\pdftexcmds@@isprimitive##1##2{%
      \ifnum0\pdftexcmds@equal##1\delimiter##2\delimiter=1 %
        \expandafter\ltx@firstoftwo
      \else
        \expandafter\ltx@secondoftwo
      \fi
    }%
    \def\pdftexcmds@equal##1##2\delimiter##3##4\delimiter{%
      \ifx##1##3%
        \ifx\relax##2##4\relax
          1%
        \else
          \ifx\relax##2\relax
          \else
            \ifx\relax##4\relax
            \else
              \pdftexcmds@equalcont{##2}{##4}%
            \fi
          \fi
        \fi
      \fi
    }%
    \def\pdftexcmds@equalcont##1{%
      \def\pdftexcmds@equalcont####1####2##1##1##1##1{%
        ##1##1##1##1%
        \pdftexcmds@equal####1\delimiter####2\delimiter
      }%
    }%
    \expandafter\pdftexcmds@equalcont\csname fi\endcsname
  \else
    \long\def\pdf@isprimitive##1##2{%
      \ifnum\pdf@strcmp{\meaning##1}{\string##2}=0 %
        \expandafter\ltx@firstoftwo
      \else
        \expandafter\ltx@secondoftwo
      \fi
    }%
  \fi
}
\ifluatex
\ifx\pdfdraftmode\@undefined
  \let\pdfdraftmode\draftmode
\fi
\else
  \pdf@isprimitive
\fi
%    \end{macrocode}
%
% \subsection{\cs{pdf@draftmode}}
%
%
%    \begin{macrocode}
\let\pdftexcmds@temp\ltx@zero %
\ltx@IfUndefined{pdfdraftmode}{%
  \@PackageInfoNoLine{pdftexcmds}{\ltx@backslashchar pdfdraftmode not found}%
}{%
  \ifpdf
    \let\pdftexcmds@temp\ltx@one
    \@PackageInfoNoLine{pdftexcmds}{\ltx@backslashchar pdfdraftmode found}%
  \else
    \@PackageInfoNoLine{pdftexcmds}{%
      \ltx@backslashchar pdfdraftmode is ignored in DVI mode%
    }%
  \fi
}
\ifcase\pdftexcmds@temp
%    \end{macrocode}
%    \begin{macro}{\pdf@draftmode}
%    \begin{macrocode}
  \let\pdf@draftmode\ltx@zero
%    \end{macrocode}
%    \end{macro}
%    \begin{macro}{\pdf@ifdraftmode}
%    \begin{macrocode}
  \let\pdf@ifdraftmode\ltx@secondoftwo
%    \end{macrocode}
%    \end{macro}
%    \begin{macro}{\pdftexcmds@setdraftmode}
%    \begin{macrocode}
  \def\pdftexcmds@setdraftmode#1{}%
%    \end{macrocode}
%    \end{macro}
%    \begin{macrocode}
\else
%    \end{macrocode}
%    \begin{macro}{\pdftexcmds@draftmode}
%    \begin{macrocode}
  \let\pdftexcmds@draftmode\pdfdraftmode
%    \end{macrocode}
%    \end{macro}
%    \begin{macro}{\pdf@ifdraftmode}
%    \begin{macrocode}
  \def\pdf@ifdraftmode{%
    \ifnum\pdftexcmds@draftmode=\ltx@one
      \expandafter\ltx@firstoftwo
    \else
      \expandafter\ltx@secondoftwo
    \fi
  }%
%    \end{macrocode}
%    \end{macro}
%    \begin{macro}{\pdf@draftmode}
%    \begin{macrocode}
  \def\pdf@draftmode{%
    \ifnum\pdftexcmds@draftmode=\ltx@one
      \expandafter\ltx@one
    \else
      \expandafter\ltx@zero
    \fi
  }%
%    \end{macrocode}
%    \end{macro}
%    \begin{macro}{\pdftexcmds@setdraftmode}
%    \begin{macrocode}
  \def\pdftexcmds@setdraftmode#1{%
    \pdftexcmds@draftmode=#1\relax
  }%
%    \end{macrocode}
%    \end{macro}
%    \begin{macrocode}
\fi
%    \end{macrocode}
%    \begin{macro}{\pdf@setdraftmode}
%    \begin{macrocode}
\def\pdf@setdraftmode#1{%
  \begingroup
    \count\ltx@cclv=#1\relax
  \edef\x{\endgroup
    \noexpand\pdftexcmds@@setdraftmode{\the\count\ltx@cclv}%
  }%
  \x
}
%    \end{macrocode}
%    \end{macro}
%    \begin{macro}{\pdftexcmds@@setdraftmode}
%    \begin{macrocode}
\def\pdftexcmds@@setdraftmode#1{%
  \ifcase#1 %
    \pdftexcmds@setdraftmode{#1}%
  \or
    \pdftexcmds@setdraftmode{#1}%
  \else
    \@PackageWarning{pdftexcmds}{%
      \string\pdf@setdraftmode: Ignoring\MessageBreak
      invalid value `#1'%
    }%
  \fi
}
%    \end{macrocode}
%    \end{macro}
%
% \subsection{Load Lua module}
%
%    \begin{macrocode}
\ifluatex
\else
  \expandafter\pdftexcmds@AtEnd
\fi%
%    \end{macrocode}
%
%    \begin{macrocode}
\ifnum\luatexversion<80
  \begingroup\expandafter\expandafter\expandafter\endgroup
  \expandafter\ifx\csname RequirePackage\endcsname\relax
    \def\TMP@RequirePackage#1[#2]{%
      \begingroup\expandafter\expandafter\expandafter\endgroup
      \expandafter\ifx\csname ver@#1.sty\endcsname\relax
        \input #1.sty\relax
      \fi
    }%
    \TMP@RequirePackage{luatex-loader}[2009/04/10]%
  \else
    \RequirePackage{luatex-loader}[2009/04/10]%
  \fi
\fi
\pdftexcmds@directlua{%
  require("pdftexcmds")%
}
\ifnum\luatexversion>37 %
  \ifnum0%
      \pdftexcmds@directlua{%
        if status.ini_version then %
          tex.write("1")%
        end%
      }>0 %
    \everyjob\expandafter{%
      \the\everyjob
      \pdftexcmds@directlua{%
        require("pdftexcmds")%
      }%
    }%
  \fi
\fi
\begingroup
  \def\x{2019/07/25 v0.30}%
  \ltx@onelevel@sanitize\x
  \edef\y{%
    \pdftexcmds@directlua{%
      if oberdiek.pdftexcmds.getversion then %
        oberdiek.pdftexcmds.getversion()%
      end%
    }%
  }%
  \ifx\x\y
  \else
    \@PackageError{pdftexcmds}{%
      Wrong version of lua module.\MessageBreak
      Package version: \x\MessageBreak
      Lua module: \y
    }\@ehc
  \fi
\endgroup
%    \end{macrocode}
%
% \subsection{Lua functions}
%
% \subsubsection{Helper macros}
%
%    \begin{macro}{\pdftexcmds@toks}
%    \begin{macrocode}
\begingroup\expandafter\expandafter\expandafter\endgroup
\expandafter\ifx\csname newtoks\endcsname\relax
  \toksdef\pdftexcmds@toks=0 %
\else
  \csname newtoks\endcsname\pdftexcmds@toks
\fi
%    \end{macrocode}
%    \end{macro}
%
%    \begin{macro}{\pdftexcmds@Patch}
%    \begin{macrocode}
\def\pdftexcmds@Patch{0}
\ifnum\luatexversion>40 %
  \ifnum\luatexversion<66 %
    \def\pdftexcmds@Patch{1}%
  \fi
\fi
%    \end{macrocode}
%    \end{macro}
%    \begin{macrocode}
\ifcase\pdftexcmds@Patch
  \catcode`\&=14 %
\else
  \catcode`\&=9 %
%    \end{macrocode}
%    \begin{macro}{\pdftexcmds@PatchDecode}
%    \begin{macrocode}
  \def\pdftexcmds@PatchDecode#1\@nil{%
    \pdftexcmds@DecodeA#1^^A^^A\@nil{}%
  }%
%    \end{macrocode}
%    \end{macro}
%    \begin{macro}{\pdftexcmds@DecodeA}
%    \begin{macrocode}
  \def\pdftexcmds@DecodeA#1^^A^^A#2\@nil#3{%
    \ifx\relax#2\relax
      \ltx@ReturnAfterElseFi{%
        \pdftexcmds@DecodeB#3#1^^A^^B\@nil{}%
      }%
    \else
      \ltx@ReturnAfterFi{%
        \pdftexcmds@DecodeA#2\@nil{#3#1^^@}%
      }%
    \fi
  }%
%    \end{macrocode}
%    \end{macro}
%    \begin{macro}{\pdftexcmds@DecodeB}
%    \begin{macrocode}
  \def\pdftexcmds@DecodeB#1^^A^^B#2\@nil#3{%
    \ifx\relax#2\relax%
      \ltx@ReturnAfterElseFi{%
        \ltx@zero
        #3#1%
      }%
    \else
      \ltx@ReturnAfterFi{%
        \pdftexcmds@DecodeB#2\@nil{#3#1^^A}%
      }%
    \fi
  }%
%    \end{macrocode}
%    \end{macro}
%    \begin{macrocode}
\fi
%    \end{macrocode}
%
%    \begin{macrocode}
\ifnum\luatexversion<36 %
\else
  \catcode`\0=9 %
\fi
%    \end{macrocode}
%
% \subsubsection[Strings]{Strings \cite[``7.15 Strings'']{pdftex-manual}}
%
%    \begin{macro}{\pdf@strcmp}
%    \begin{macrocode}
\long\def\pdf@strcmp#1#2{%
  \directlua0{%
    oberdiek.pdftexcmds.strcmp("\luaescapestring{#1}",%
        "\luaescapestring{#2}")%
  }%
}%
%    \end{macrocode}
%    \end{macro}
%    \begin{macrocode}
\pdf@isprimitive
%    \end{macrocode}
%    \begin{macro}{\pdf@escapehex}
%    \begin{macrocode}
\long\def\pdf@escapehex#1{%
  \directlua0{%
    oberdiek.pdftexcmds.escapehex("\luaescapestring{#1}", "byte")%
  }%
}%
%    \end{macrocode}
%    \end{macro}
%    \begin{macro}{\pdf@escapehexnative}
%    \begin{macrocode}
\long\def\pdf@escapehexnative#1{%
  \directlua0{%
    oberdiek.pdftexcmds.escapehex("\luaescapestring{#1}")%
  }%
}%
%    \end{macrocode}
%    \end{macro}
%    \begin{macro}{\pdf@unescapehex}
%    \begin{macrocode}
\def\pdf@unescapehex#1{%
& \romannumeral\expandafter\pdftexcmds@PatchDecode
  \the\expandafter\pdftexcmds@toks
  \directlua0{%
    oberdiek.pdftexcmds.toks="pdftexcmds@toks"%
    oberdiek.pdftexcmds.unescapehex("\luaescapestring{#1}", "byte", \pdftexcmds@Patch)%
  }%
& \@nil
}%
%    \end{macrocode}
%    \end{macro}
%    \begin{macro}{\pdf@unescapehexnative}
%    \begin{macrocode}
\def\pdf@unescapehexnative#1{%
& \romannumeral\expandafter\pdftexcmds@PatchDecode
  \the\expandafter\pdftexcmds@toks
  \directlua0{%
    oberdiek.pdftexcmds.toks="pdftexcmds@toks"%
    oberdiek.pdftexcmds.unescapehex("\luaescapestring{#1}", \pdftexcmds@Patch)%
  }%
& \@nil
}%
%    \end{macrocode}
%    \end{macro}
%    \begin{macro}{\pdf@escapestring}
%    \begin{macrocode}
\long\def\pdf@escapestring#1{%
  \directlua0{%
    oberdiek.pdftexcmds.escapestring("\luaescapestring{#1}", "byte")%
  }%
}
%    \end{macrocode}
%    \end{macro}
%    \begin{macro}{\pdf@escapename}
%    \begin{macrocode}
\long\def\pdf@escapename#1{%
  \directlua0{%
    oberdiek.pdftexcmds.escapename("\luaescapestring{#1}", "byte")%
  }%
}
%    \end{macrocode}
%    \end{macro}
%    \begin{macro}{\pdf@escapenamenative}
%    \begin{macrocode}
\long\def\pdf@escapenamenative#1{%
  \directlua0{%
    oberdiek.pdftexcmds.escapename("\luaescapestring{#1}")%
  }%
}
%    \end{macrocode}
%    \end{macro}
%
% \subsubsection[Files]{Files \cite[``7.18 Files'']{pdftex-manual}}
%
%    \begin{macro}{\pdf@filesize}
%    \begin{macrocode}
\def\pdf@filesize#1{%
  \directlua0{%
    oberdiek.pdftexcmds.filesize("\luaescapestring{#1}")%
  }%
}
%    \end{macrocode}
%    \end{macro}
%    \begin{macro}{\pdf@filemoddate}
%    \begin{macrocode}
\def\pdf@filemoddate#1{%
  \directlua0{%
    oberdiek.pdftexcmds.filemoddate("\luaescapestring{#1}")%
  }%
}
%    \end{macrocode}
%    \end{macro}
%    \begin{macro}{\pdf@filedump}
%    \begin{macrocode}
\def\pdf@filedump#1#2#3{%
  \directlua0{%
    oberdiek.pdftexcmds.filedump("\luaescapestring{\number#1}",%
        "\luaescapestring{\number#2}",%
        "\luaescapestring{#3}")%
  }%
}%
%    \end{macrocode}
%    \end{macro}
%    \begin{macro}{\pdf@mdfivesum}
%    \begin{macrocode}
\long\def\pdf@mdfivesum#1{%
  \directlua0{%
    oberdiek.pdftexcmds.mdfivesum("\luaescapestring{#1}", "byte")%
  }%
}%
%    \end{macrocode}
%    \end{macro}
%    \begin{macro}{\pdf@mdfivesumnative}
%    \begin{macrocode}
\long\def\pdf@mdfivesumnative#1{%
  \directlua0{%
    oberdiek.pdftexcmds.mdfivesum("\luaescapestring{#1}")%
  }%
}%
%    \end{macrocode}
%    \end{macro}
%    \begin{macro}{\pdf@filemdfivesum}
%    \begin{macrocode}
\def\pdf@filemdfivesum#1{%
  \directlua0{%
    oberdiek.pdftexcmds.filemdfivesum("\luaescapestring{#1}")%
  }%
}%
%    \end{macrocode}
%    \end{macro}
%
% \subsubsection[Timekeeping]{Timekeeping \cite[``7.17 Timekeeping'']{pdftex-manual}}
%
%    \begin{macro}{\protected}
%    \begin{macrocode}
\let\pdftexcmds@temp=Y%
\begingroup\expandafter\expandafter\expandafter\endgroup
\expandafter\ifx\csname protected\endcsname\relax
  \pdftexcmds@directlua0{%
    if tex.enableprimitives then %
      tex.enableprimitives('', {'protected'})%
    end%
  }%
\fi
\begingroup\expandafter\expandafter\expandafter\endgroup
\expandafter\ifx\csname protected\endcsname\relax
  \let\pdftexcmds@temp=N%
\fi
%    \end{macrocode}
%    \end{macro}
%    \begin{macro}{\numexpr}
%    \begin{macrocode}
\begingroup\expandafter\expandafter\expandafter\endgroup
\expandafter\ifx\csname numexpr\endcsname\relax
  \pdftexcmds@directlua0{%
    if tex.enableprimitives then %
      tex.enableprimitives('', {'numexpr'})%
    end%
  }%
\fi
\begingroup\expandafter\expandafter\expandafter\endgroup
\expandafter\ifx\csname numexpr\endcsname\relax
  \let\pdftexcmds@temp=N%
\fi
%    \end{macrocode}
%    \end{macro}
%
%    \begin{macrocode}
\ifx\pdftexcmds@temp N%
  \@PackageWarningNoLine{pdftexcmds}{%
    Definitions of \ltx@backslashchar pdf@resettimer and%
    \MessageBreak
    \ltx@backslashchar pdf@elapsedtime are skipped, because%
    \MessageBreak
    e-TeX's \ltx@backslashchar protected or %
    \ltx@backslashchar numexpr are missing%
  }%
\else
%    \end{macrocode}
%
%    \begin{macro}{\pdf@resettimer}
%    \begin{macrocode}
  \protected\def\pdf@resettimer{%
    \pdftexcmds@directlua0{%
      oberdiek.pdftexcmds.resettimer()%
    }%
  }%
%    \end{macrocode}
%    \end{macro}
%
%    \begin{macro}{\pdf@elapsedtime}
%    \begin{macrocode}
  \protected\def\pdf@elapsedtime{%
    \numexpr
      \pdftexcmds@directlua0{%
        oberdiek.pdftexcmds.elapsedtime()%
      }%
    \relax
  }%
%    \end{macrocode}
%    \end{macro}
%    \begin{macrocode}
\fi
%    \end{macrocode}
%
% \subsubsection{Shell escape}
%
%    \begin{macro}{\pdf@shellescape}
%
%    \begin{macrocode}
\ifnum\luatexversion<68 %
\else
  \protected\edef\pdf@shellescape{%
   \numexpr\directlua{tex.sprint(%
         \number\catcodetable@string,status.shell_escape)}\relax}
\fi
%    \end{macrocode}
%    \end{macro}
%
%    \begin{macro}{\pdf@system}
%    \begin{macrocode}
\def\pdf@system#1{%
  \directlua0{%
    oberdiek.pdftexcmds.system("\luaescapestring{#1}")%
  }%
}
%    \end{macrocode}
%    \end{macro}
%
%    \begin{macro}{\pdf@lastsystemstatus}
%    \begin{macrocode}
\def\pdf@lastsystemstatus{%
  \directlua0{%
    oberdiek.pdftexcmds.lastsystemstatus()%
  }%
}
%    \end{macrocode}
%    \end{macro}
%    \begin{macro}{\pdf@lastsystemexit}
%    \begin{macrocode}
\def\pdf@lastsystemexit{%
  \directlua0{%
    oberdiek.pdftexcmds.lastsystemexit()%
  }%
}
%    \end{macrocode}
%    \end{macro}
%
%    \begin{macrocode}
\catcode`\0=12 %
%    \end{macrocode}
%
%    \begin{macro}{\pdf@pipe}
%    Check availability of |io.popen| first.
%    \begin{macrocode}
\ifnum0%
    \pdftexcmds@directlua{%
      if io.popen then %
        tex.write("1")%
      end%
    }%
    =1 %
  \def\pdf@pipe#1{%
&   \romannumeral\expandafter\pdftexcmds@PatchDecode
    \the\expandafter\pdftexcmds@toks
    \pdftexcmds@directlua{%
      oberdiek.pdftexcmds.toks="pdftexcmds@toks"%
      oberdiek.pdftexcmds.pipe("\luaescapestring{#1}", \pdftexcmds@Patch)%
    }%
&   \@nil
  }%
\fi
%    \end{macrocode}
%    \end{macro}
%
%    \begin{macrocode}
\pdftexcmds@AtEnd%
%</package>
%    \end{macrocode}
%
% \subsection{Lua module}
%
%    \begin{macrocode}
%<*lua>
%    \end{macrocode}
%
%    \begin{macrocode}
oberdiek = oberdiek or {}
local pdftexcmds = oberdiek.pdftexcmds or {}
oberdiek.pdftexcmds = pdftexcmds
local systemexitstatus
function pdftexcmds.getversion()
  tex.write("2019/07/25 v0.30")
end
%    \end{macrocode}
%
% \subsubsection[Strings]{Strings \cite[``7.15 Strings'']{pdftex-manual}}
%
%    \begin{macrocode}
function pdftexcmds.strcmp(A, B)
  if A == B then
    tex.write("0")
  elseif A < B then
    tex.write("-1")
  else
    tex.write("1")
  end
end
local function utf8_to_byte(str)
  local i = 0
  local n = string.len(str)
  local t = {}
  while i < n do
    i = i + 1
    local a = string.byte(str, i)
    if a < 128 then
      table.insert(t, string.char(a))
    else
      if a >= 192 and i < n then
        i = i + 1
        local b = string.byte(str, i)
        if b < 128 or b >= 192 then
          i = i - 1
        elseif a == 194 then
          table.insert(t, string.char(b))
        elseif a == 195 then
          table.insert(t, string.char(b + 64))
        end
      end
    end
  end
  return table.concat(t)
end
function pdftexcmds.escapehex(str, mode)
  if mode == "byte" then
    str = utf8_to_byte(str)
  end
  tex.write((string.gsub(str, ".",
    function (ch)
      return string.format("%02X", string.byte(ch))
    end
  )))
end
%    \end{macrocode}
%    See procedure |unescapehex| in file \xfile{utils.c} of \hologo{pdfTeX}.
%    Caution: |tex.write| ignores leading spaces.
%    \begin{macrocode}
function pdftexcmds.unescapehex(str, mode, patch)
  local a = 0
  local first = true
  local result = {}
  for i = 1, string.len(str), 1 do
    local ch = string.byte(str, i)
    if ch >= 48 and ch <= 57 then
      ch = ch - 48
    elseif ch >= 65 and ch <= 70 then
      ch = ch - 55
    elseif ch >= 97 and ch <= 102 then
      ch = ch - 87
    else
      ch = nil
    end
    if ch then
      if first then
        a = ch * 16
        first = false
      else
        table.insert(result, a + ch)
        first = true
      end
    end
  end
  if not first then
    table.insert(result, a)
  end
  if patch == 1 then
    local temp = {}
    for i, a in ipairs(result) do
      if a == 0 then
        table.insert(temp, 1)
        table.insert(temp, 1)
      else
        if a == 1 then
          table.insert(temp, 1)
          table.insert(temp, 2)
        else
          table.insert(temp, a)
        end
      end
    end
    result = temp
  end
  if mode == "byte" then
    local utf8 = {}
    for i, a in ipairs(result) do
      if a < 128 then
        table.insert(utf8, a)
      else
        if a < 192 then
          table.insert(utf8, 194)
          a = a - 128
        else
          table.insert(utf8, 195)
          a = a - 192
        end
        table.insert(utf8, a + 128)
      end
    end
    result = utf8
  end
%    \end{macrocode}
%    this next line added for current luatex; this is the only
%    change in the file.  eroux, 28apr13. (v 0.21)
%    \begin{macrocode}
  local unpack = _G["unpack"] or table.unpack
  tex.settoks(pdftexcmds.toks, string.char(unpack(result)))
end
%    \end{macrocode}
%    See procedure |escapestring| in file \xfile{utils.c} of \hologo{pdfTeX}.
%    \begin{macrocode}
function pdftexcmds.escapestring(str, mode)
  if mode == "byte" then
    str = utf8_to_byte(str)
  end
  tex.write((string.gsub(str, ".",
    function (ch)
      local b = string.byte(ch)
      if b < 33 or b > 126 then
        return string.format("\\%.3o", b)
      end
      if b == 40 or b == 41 or b == 92 then
        return "\\" .. ch
      end
%    \end{macrocode}
%    Lua 5.1 returns the match in case of return value |nil|.
%    \begin{macrocode}
      return nil
    end
  )))
end
%    \end{macrocode}
%    See procedure |escapename| in file \xfile{utils.c} of \hologo{pdfTeX}.
%    \begin{macrocode}
function pdftexcmds.escapename(str, mode)
  if mode == "byte" then
    str = utf8_to_byte(str)
  end
  tex.write((string.gsub(str, ".",
    function (ch)
      local b = string.byte(ch)
      if b == 0 then
%    \end{macrocode}
%    In Lua 5.0 |nil| could be used for the empty string,
%    But |nil| returns the match in Lua 5.1, thus we use
%    the empty string explicitly.
%    \begin{macrocode}
        return ""
      end
      if b <= 32 or b >= 127
          or b == 35 or b == 37 or b == 40 or b == 41
          or b == 47 or b == 60 or b == 62 or b == 91
          or b == 93 or b == 123 or b == 125 then
        return string.format("#%.2X", b)
      else
%    \end{macrocode}
%    Lua 5.1 returns the match in case of return value |nil|.
%    \begin{macrocode}
        return nil
      end
    end
  )))
end
%    \end{macrocode}
%
% \subsubsection[Files]{Files \cite[``7.18 Files'']{pdftex-manual}}
%
%    \begin{macrocode}
function pdftexcmds.filesize(filename)
  local foundfile = kpse.find_file(filename, "tex", true)
  if foundfile then
    local size = lfs.attributes(foundfile, "size")
    if size then
      tex.write(size)
    end
  end
end
%    \end{macrocode}
%    See procedure |makepdftime| in file \xfile{utils.c} of \hologo{pdfTeX}.
%    \begin{macrocode}
function pdftexcmds.filemoddate(filename)
  local foundfile = kpse.find_file(filename, "tex", true)
  if foundfile then
    local date = lfs.attributes(foundfile, "modification")
    if date then
      local d = os.date("*t", date)
      if d.sec >= 60 then
        d.sec = 59
      end
      local u = os.date("!*t", date)
      local off = 60 * (d.hour - u.hour) + d.min - u.min
      if d.year ~= u.year then
        if d.year > u.year then
          off = off + 1440
        else
          off = off - 1440
        end
      elseif d.yday ~= u.yday then
        if d.yday > u.yday then
          off = off + 1440
        else
          off = off - 1440
        end
      end
      local timezone
      if off == 0 then
        timezone = "Z"
      else
        local hours = math.floor(off / 60)
        local mins = math.abs(off - hours * 60)
        timezone = string.format("%+03d'%02d'", hours, mins)
      end
      tex.write(string.format("D:%04d%02d%02d%02d%02d%02d%s",
          d.year, d.month, d.day, d.hour, d.min, d.sec, timezone))
    end
  end
end
function pdftexcmds.filedump(offset, length, filename)
  length = tonumber(length)
  if length and length > 0 then
    local foundfile = kpse.find_file(filename, "tex", true)
    if foundfile then
      offset = tonumber(offset)
      if not offset then
        offset = 0
      end
      local filehandle = io.open(foundfile, "rb")
      if filehandle then
        if offset > 0 then
          filehandle:seek("set", offset)
        end
        local dump = filehandle:read(length)
        pdftexcmds.escapehex(dump)
        filehandle:close()
      end
    end
  end
end
function pdftexcmds.mdfivesum(str, mode)
  if mode == "byte" then
    str = utf8_to_byte(str)
  end
  pdftexcmds.escapehex(md5.sum(str))
end
function pdftexcmds.filemdfivesum(filename)
  local foundfile = kpse.find_file(filename, "tex", true)
  if foundfile then
    local filehandle = io.open(foundfile, "rb")
    if filehandle then
      local contents = filehandle:read("*a")
      pdftexcmds.escapehex(md5.sum(contents))
      filehandle:close()
    end
  end
end
%    \end{macrocode}
%
% \subsubsection[Timekeeping]{Timekeeping \cite[``7.17 Timekeeping'']{pdftex-manual}}
%
%    The functions for timekeeping are based on
%    Andy Thomas' work \cite{AndyThomas:Analog}.
%    Changes:
%    \begin{itemize}
%    \item Overflow check is added.
%    \item |string.format| is used to avoid exponential number
%          representation for sure.
%    \item |tex.write| is used instead of |tex.print| to get
%          tokens with catcode 12 and without appended \cs{endlinechar}.
%    \end{itemize}
%    \begin{macrocode}
local basetime = 0
function pdftexcmds.resettimer()
  basetime = os.clock()
end
function pdftexcmds.elapsedtime()
  local val = (os.clock() - basetime) * 65536 + .5
  if val > 2147483647 then
    val = 2147483647
  end
  tex.write(string.format("%d", val))
end
%    \end{macrocode}
%
% \subsubsection[Miscellaneous]{Miscellaneous \cite[``7.21 Miscellaneous'']{pdftex-manual}}
%
%    \begin{macrocode}
function pdftexcmds.shellescape()
  if os.execute then
    if status
        and status.luatex_version
        and status.luatex_version >= 68 then
      tex.write(os.execute())
    else
      local result = os.execute()
      if result == 0 then
        tex.write("0")
      else
        if result == nil then
          tex.write("0")
        else
          tex.write("1")
        end
      end
    end
  else
    tex.write("0")
  end
end
function pdftexcmds.system(cmdline)
  systemexitstatus = nil
  texio.write_nl("log", "system(" .. cmdline .. ") ")
  if os.execute then
    texio.write("log", "executed.")
    systemexitstatus = os.execute(cmdline)
  else
    texio.write("log", "disabled.")
  end
end
function pdftexcmds.lastsystemstatus()
  local result = tonumber(systemexitstatus)
  if result then
    local x = math.floor(result / 256)
    tex.write(result - 256 * math.floor(result / 256))
  end
end
function pdftexcmds.lastsystemexit()
  local result = tonumber(systemexitstatus)
  if result then
    tex.write(math.floor(result / 256))
  end
end
function pdftexcmds.pipe(cmdline, patch)
  local result
  systemexitstatus = nil
  texio.write_nl("log", "pipe(" .. cmdline ..") ")
  if io.popen then
    texio.write("log", "executed.")
    local handle = io.popen(cmdline, "r")
    if handle then
      result = handle:read("*a")
      handle:close()
    end
  else
    texio.write("log", "disabled.")
  end
  if result then
    if patch == 1 then
      local temp = {}
      for i, a in ipairs(result) do
        if a == 0 then
          table.insert(temp, 1)
          table.insert(temp, 1)
        else
          if a == 1 then
            table.insert(temp, 1)
            table.insert(temp, 2)
          else
            table.insert(temp, a)
          end
        end
      end
      result = temp
    end
    tex.settoks(pdftexcmds.toks, result)
  else
    tex.settoks(pdftexcmds.toks, "")
  end
end
%    \end{macrocode}
%    \begin{macrocode}
%</lua>
%    \end{macrocode}
%
% \section{Test}
%
% \subsection{Catcode checks for loading}
%
%    \begin{macrocode}
%<*test1>
%    \end{macrocode}
%    \begin{macrocode}
\catcode`\{=1 %
\catcode`\}=2 %
\catcode`\#=6 %
\catcode`\@=11 %
\expandafter\ifx\csname count@\endcsname\relax
  \countdef\count@=255 %
\fi
\expandafter\ifx\csname @gobble\endcsname\relax
  \long\def\@gobble#1{}%
\fi
\expandafter\ifx\csname @firstofone\endcsname\relax
  \long\def\@firstofone#1{#1}%
\fi
\expandafter\ifx\csname loop\endcsname\relax
  \expandafter\@firstofone
\else
  \expandafter\@gobble
\fi
{%
  \def\loop#1\repeat{%
    \def\body{#1}%
    \iterate
  }%
  \def\iterate{%
    \body
      \let\next\iterate
    \else
      \let\next\relax
    \fi
    \next
  }%
  \let\repeat=\fi
}%
\def\RestoreCatcodes{}
\count@=0 %
\loop
  \edef\RestoreCatcodes{%
    \RestoreCatcodes
    \catcode\the\count@=\the\catcode\count@\relax
  }%
\ifnum\count@<255 %
  \advance\count@ 1 %
\repeat

\def\RangeCatcodeInvalid#1#2{%
  \count@=#1\relax
  \loop
    \catcode\count@=15 %
  \ifnum\count@<#2\relax
    \advance\count@ 1 %
  \repeat
}
\def\RangeCatcodeCheck#1#2#3{%
  \count@=#1\relax
  \loop
    \ifnum#3=\catcode\count@
    \else
      \errmessage{%
        Character \the\count@\space
        with wrong catcode \the\catcode\count@\space
        instead of \number#3%
      }%
    \fi
  \ifnum\count@<#2\relax
    \advance\count@ 1 %
  \repeat
}
\def\space{ }
\expandafter\ifx\csname LoadCommand\endcsname\relax
  \def\LoadCommand{\input pdftexcmds.sty\relax}%
\fi
\def\Test{%
  \RangeCatcodeInvalid{0}{47}%
  \RangeCatcodeInvalid{58}{64}%
  \RangeCatcodeInvalid{91}{96}%
  \RangeCatcodeInvalid{123}{255}%
  \catcode`\@=12 %
  \catcode`\\=0 %
  \catcode`\%=14 %
  \LoadCommand
  \RangeCatcodeCheck{0}{36}{15}%
  \RangeCatcodeCheck{37}{37}{14}%
  \RangeCatcodeCheck{38}{47}{15}%
  \RangeCatcodeCheck{48}{57}{12}%
  \RangeCatcodeCheck{58}{63}{15}%
  \RangeCatcodeCheck{64}{64}{12}%
  \RangeCatcodeCheck{65}{90}{11}%
  \RangeCatcodeCheck{91}{91}{15}%
  \RangeCatcodeCheck{92}{92}{0}%
  \RangeCatcodeCheck{93}{96}{15}%
  \RangeCatcodeCheck{97}{122}{11}%
  \RangeCatcodeCheck{123}{255}{15}%
  \RestoreCatcodes
}
\Test
\csname @@end\endcsname
\end
%    \end{macrocode}
%    \begin{macrocode}
%</test1>
%    \end{macrocode}
%
% \subsection{Test for \cs{pdf@isprimitive}}
%
%    \begin{macrocode}
%<*test2>
\catcode`\{=1 %
\catcode`\}=2 %
\catcode`\#=6 %
\catcode`\@=11 %
\input pdftexcmds.sty\relax
\def\msg#1{%
  \begingroup
    \escapechar=92 %
    \immediate\write16{#1}%
  \endgroup
}
\long\def\test#1#2#3#4{%
  \begingroup
    #4%
    \def\str{%
      Test \string\pdf@isprimitive
      {\string #1}{\string #2}{...}: %
    }%
    \pdf@isprimitive{#1}{#2}{%
      \ifx#3Y%
        \msg{\str true ==> OK.}%
      \else
        \errmessage{\str false ==> FAILED}%
      \fi
    }{%
      \ifx#3Y%
        \errmessage{\str true ==> FAILED}%
      \else
        \msg{\str false ==> OK.}%
      \fi
    }%
  \endgroup
}
\test\relax\relax Y{}
\test\foobar\relax Y{\let\foobar\relax}
\test\foobar\relax N{}
\test\hbox\hbox Y{}
\test\foobar@hbox\hbox Y{\let\foobar@hbox\hbox}
\test\if\if Y{}
\test\if\ifx N{}
\test\ifx\if N{}
\test\par\par Y{}
\test\hbox\par N{}
\test\par\hbox N{}
\csname @@end\endcsname\end
%</test2>
%    \end{macrocode}
%
% \subsection{Test for \cs{pdf@shellescape}}
%
%    \begin{macrocode}
%<*test-shell>
\catcode`\{=1 %
\catcode`\}=2 %
\catcode`\#=6 %
\catcode`\@=11 %
\input pdftexcmds.sty\relax
\def\msg#{\immediate\write16}
\def\MaybeEnd{}
\ifx\luatexversion\UnDeFiNeD
\else
  \ifnum\luatexversion<68 %
    \ifx\pdf@shellescape\@undefined
      \msg{SHELL=U}%
      \msg{OK (LuaTeX < 0.68)}%
    \else
      \msg{SHELL=defined}%
      \errmessage{Failed (LuaTeX < 0.68)}%
    \fi
    \def\MaybeEnd{\csname @@end\endcsname\end}%
  \fi
\fi
\MaybeEnd
\ifx\pdf@shellescape\@undefined
  \msg{SHELL=U}%
\else
  \msg{SHELL=\number\pdf@shellescape}%
\fi
\ifx\expected\@undefined
\else
  \ifx\expected\relax
    \msg{EXPECTED=U}%
    \ifx\pdf@shellescape\@undefined
      \msg{OK}%
    \else
      \errmessage{Failed}%
    \fi
  \else
    \msg{EXPECTED=\number\expected}%
    \ifnum\pdf@shellescape=\expected\relax
      \msg{OK}%
    \else
      \errmessage{Failed}%
    \fi
  \fi
\fi
\csname @@end\endcsname\end
%</test-shell>
%    \end{macrocode}
%
% \subsection{Test for escape functions}
%
%    \begin{macrocode}
%<*test-escape>
\catcode`\{=1 %
\catcode`\}=2 %
\catcode`\#=6 %
\catcode`\^=7 %
\catcode`\@=11 %
\errorcontextlines=1000 %
\input pdftexcmds.sty\relax
\def\msg#1{%
  \begingroup
    \escapechar=92 %
    \immediate\write16{#1}%
  \endgroup
}
%    \end{macrocode}
%    \begin{macrocode}
\begingroup
  \catcode`\@=11 %
  \countdef\count@=255 %
  \def\space{ }%
  \long\def\@whilenum#1\do #2{%
    \ifnum #1\relax
      #2\relax
      \@iwhilenum{#1\relax#2\relax}%
    \fi
  }%
  \long\def\@iwhilenum#1{%
    \ifnum #1%
      \expandafter\@iwhilenum
    \else
      \expandafter\ltx@gobble
    \fi
    {#1}%
  }%
  \gdef\AllBytes{}%
  \count@=0 %
  \catcode0=12 %
  \@whilenum\count@<256 \do{%
    \lccode0=\count@
    \ifnum\count@=32 %
      \xdef\AllBytes{\AllBytes\space}%
    \else
      \lowercase{%
        \xdef\AllBytes{\AllBytes^^@}%
      }%
    \fi
    \advance\count@ by 1 %
  }%
\endgroup
%    \end{macrocode}
%    \begin{macrocode}
\def\AllBytesHex{%
  000102030405060708090A0B0C0D0E0F%
  101112131415161718191A1B1C1D1E1F%
  202122232425262728292A2B2C2D2E2F%
  303132333435363738393A3B3C3D3E3F%
  404142434445464748494A4B4C4D4E4F%
  505152535455565758595A5B5C5D5E5F%
  606162636465666768696A6B6C6D6E6F%
  707172737475767778797A7B7C7D7E7F%
  808182838485868788898A8B8C8D8E8F%
  909192939495969798999A9B9C9D9E9F%
  A0A1A2A3A4A5A6A7A8A9AAABACADAEAF%
  B0B1B2B3B4B5B6B7B8B9BABBBCBDBEBF%
  C0C1C2C3C4C5C6C7C8C9CACBCCCDCECF%
  D0D1D2D3D4D5D6D7D8D9DADBDCDDDEDF%
  E0E1E2E3E4E5E6E7E8E9EAEBECEDEEEF%
  F0F1F2F3F4F5F6F7F8F9FAFBFCFDFEFF%
}
\ltx@onelevel@sanitize\AllBytesHex
\expandafter\lowercase\expandafter{%
  \expandafter\def\expandafter\AllBytesHexLC
      \expandafter{\AllBytesHex}%
}
\begingroup
  \catcode`\#=12 %
  \xdef\AllBytesName{%
    #01#02#03#04#05#06#07#08#09#0A#0B#0C#0D#0E#0F%
    #10#11#12#13#14#15#16#17#18#19#1A#1B#1C#1D#1E#1F%
    #20!"#23$#25&'#28#29*+,-.#2F%
    0123456789:;#3C=#3E?%
    @ABCDEFGHIJKLMNO%
    PQRSTUVWXYZ#5B\ltx@backslashchar#5D^_%
    `abcdefghijklmno%
    pqrstuvwxyz#7B|#7D\string~#7F%
    #80#81#82#83#84#85#86#87#88#89#8A#8B#8C#8D#8E#8F%
    #90#91#92#93#94#95#96#97#98#99#9A#9B#9C#9D#9E#9F%
    #A0#A1#A2#A3#A4#A5#A6#A7#A8#A9#AA#AB#AC#AD#AE#AF%
    #B0#B1#B2#B3#B4#B5#B6#B7#B8#B9#BA#BB#BC#BD#BE#BF%
    #C0#C1#C2#C3#C4#C5#C6#C7#C8#C9#CA#CB#CC#CD#CE#CF%
    #D0#D1#D2#D3#D4#D5#D6#D7#D8#D9#DA#DB#DC#DD#DE#DF%
    #E0#E1#E2#E3#E4#E5#E6#E7#E8#E9#EA#EB#EC#ED#EE#EF%
    #F0#F1#F2#F3#F4#F5#F6#F7#F8#F9#FA#FB#FC#FD#FE#FF%
  }%
\endgroup
\ltx@onelevel@sanitize\AllBytesName
\edef\AllBytesFromName{\expandafter\ltx@gobble\AllBytes}
\begingroup
  \def\|{|}%
  \edef\%{\ltx@percentchar}%
  \catcode`\|=0 %
  \catcode`\#=12 %
  \catcode`\~=12 %
  \catcode`\\=12 %
  |xdef|AllBytesString{%
    \000\001\002\003\004\005\006\007\010\011\012\013\014\015\016\017%
    \020\021\022\023\024\025\026\027\030\031\032\033\034\035\036\037%
    \040!"#$|%&'\(\)*+,-./%
    0123456789:;<=>?%
    @ABCDEFGHIJKLMNO%
    PQRSTUVWXYZ[\\]^_%
    `abcdefghijklmno%
    pqrstuvwxyz{||}~\177%
    \200\201\202\203\204\205\206\207\210\211\212\213\214\215\216\217%
    \220\221\222\223\224\225\226\227\230\231\232\233\234\235\236\237%
    \240\241\242\243\244\245\246\247\250\251\252\253\254\255\256\257%
    \260\261\262\263\264\265\266\267\270\271\272\273\274\275\276\277%
    \300\301\302\303\304\305\306\307\310\311\312\313\314\315\316\317%
    \320\321\322\323\324\325\326\327\330\331\332\333\334\335\336\337%
    \340\341\342\343\344\345\346\347\350\351\352\353\354\355\356\357%
    \360\361\362\363\364\365\366\367\370\371\372\373\374\375\376\377%
  }%
|endgroup
\ltx@onelevel@sanitize\AllBytesString
%    \end{macrocode}
%    \begin{macrocode}
\def\Test#1#2#3{%
  \begingroup
    \expandafter\expandafter\expandafter\def
    \expandafter\expandafter\expandafter\TestResult
    \expandafter\expandafter\expandafter{%
      #1{#2}%
    }%
    \ifx\TestResult#3%
    \else
      \newlinechar=10 %
      \msg{Expect:^^J#3}%
      \msg{Result:^^J\TestResult}%
      \errmessage{\string#2 -\string#1-> \string#3}%
    \fi
  \endgroup
}
\def\test#1#2#3{%
  \edef\TestFrom{#2}%
  \edef\TestExpect{#3}%
  \ltx@onelevel@sanitize\TestExpect
  \Test#1\TestFrom\TestExpect
}
\test\pdf@unescapehex{74657374}{test}
\begingroup
  \catcode0=12 %
  \catcode1=12 %
  \test\pdf@unescapehex{740074017400740174}{t^^@t^^At^^@t^^At}%
\endgroup
\Test\pdf@escapehex\AllBytes\AllBytesHex
\Test\pdf@unescapehex\AllBytesHex\AllBytes
\Test\pdf@escapename\AllBytes\AllBytesName
\Test\pdf@escapestring\AllBytes\AllBytesString
%    \end{macrocode}
%    \begin{macrocode}
\csname @@end\endcsname\end
%</test-escape>
%    \end{macrocode}
%
% \section{Installation}
%
% \subsection{Download}
%
% \paragraph{Package.} This package is available on
% CTAN\footnote{\CTANpkg{pdftexcmds}}:
% \begin{description}
% \item[\CTAN{macros/latex/contrib/oberdiek/pdftexcmds.dtx}] The source file.
% \item[\CTAN{macros/latex/contrib/oberdiek/pdftexcmds.pdf}] Documentation.
% \end{description}
%
%
% \paragraph{Bundle.} All the packages of the bundle `oberdiek'
% are also available in a TDS compliant ZIP archive. There
% the packages are already unpacked and the documentation files
% are generated. The files and directories obey the TDS standard.
% \begin{description}
% \item[\CTANinstall{install/macros/latex/contrib/oberdiek.tds.zip}]
% \end{description}
% \emph{TDS} refers to the standard ``A Directory Structure
% for \TeX\ Files'' (\CTAN{tds/tds.pdf}). Directories
% with \xfile{texmf} in their name are usually organized this way.
%
% \subsection{Bundle installation}
%
% \paragraph{Unpacking.} Unpack the \xfile{oberdiek.tds.zip} in the
% TDS tree (also known as \xfile{texmf} tree) of your choice.
% Example (linux):
% \begin{quote}
%   |unzip oberdiek.tds.zip -d ~/texmf|
% \end{quote}
%
% \paragraph{Script installation.}
% Check the directory \xfile{TDS:scripts/oberdiek/} for
% scripts that need further installation steps.
% Package \xpackage{attachfile2} comes with the Perl script
% \xfile{pdfatfi.pl} that should be installed in such a way
% that it can be called as \texttt{pdfatfi}.
% Example (linux):
% \begin{quote}
%   |chmod +x scripts/oberdiek/pdfatfi.pl|\\
%   |cp scripts/oberdiek/pdfatfi.pl /usr/local/bin/|
% \end{quote}
%
% \subsection{Package installation}
%
% \paragraph{Unpacking.} The \xfile{.dtx} file is a self-extracting
% \docstrip\ archive. The files are extracted by running the
% \xfile{.dtx} through \plainTeX:
% \begin{quote}
%   \verb|tex pdftexcmds.dtx|
% \end{quote}
%
% \paragraph{TDS.} Now the different files must be moved into
% the different directories in your installation TDS tree
% (also known as \xfile{texmf} tree):
% \begin{quote}
% \def\t{^^A
% \begin{tabular}{@{}>{\ttfamily}l@{ $\rightarrow$ }>{\ttfamily}l@{}}
%   pdftexcmds.sty & tex/generic/oberdiek/pdftexcmds.sty\\
%   oberdiek.pdftexcmds.lua & scripts/oberdiek/oberdiek.pdftexcmds.lua\\
%   pdftexcmds.lua & scripts/oberdiek/pdftexcmds.lua\\
%   pdftexcmds.pdf & doc/latex/oberdiek/pdftexcmds.pdf\\
%   test/pdftexcmds-test1.tex & doc/latex/oberdiek/test/pdftexcmds-test1.tex\\
%   test/pdftexcmds-test2.tex & doc/latex/oberdiek/test/pdftexcmds-test2.tex\\
%   test/pdftexcmds-test-shell.tex & doc/latex/oberdiek/test/pdftexcmds-test-shell.tex\\
%   test/pdftexcmds-test-escape.tex & doc/latex/oberdiek/test/pdftexcmds-test-escape.tex\\
%   pdftexcmds.dtx & source/latex/oberdiek/pdftexcmds.dtx\\
% \end{tabular}^^A
% }^^A
% \sbox0{\t}^^A
% \ifdim\wd0>\linewidth
%   \begingroup
%     \advance\linewidth by\leftmargin
%     \advance\linewidth by\rightmargin
%   \edef\x{\endgroup
%     \def\noexpand\lw{\the\linewidth}^^A
%   }\x
%   \def\lwbox{^^A
%     \leavevmode
%     \hbox to \linewidth{^^A
%       \kern-\leftmargin\relax
%       \hss
%       \usebox0
%       \hss
%       \kern-\rightmargin\relax
%     }^^A
%   }^^A
%   \ifdim\wd0>\lw
%     \sbox0{\small\t}^^A
%     \ifdim\wd0>\linewidth
%       \ifdim\wd0>\lw
%         \sbox0{\footnotesize\t}^^A
%         \ifdim\wd0>\linewidth
%           \ifdim\wd0>\lw
%             \sbox0{\scriptsize\t}^^A
%             \ifdim\wd0>\linewidth
%               \ifdim\wd0>\lw
%                 \sbox0{\tiny\t}^^A
%                 \ifdim\wd0>\linewidth
%                   \lwbox
%                 \else
%                   \usebox0
%                 \fi
%               \else
%                 \lwbox
%               \fi
%             \else
%               \usebox0
%             \fi
%           \else
%             \lwbox
%           \fi
%         \else
%           \usebox0
%         \fi
%       \else
%         \lwbox
%       \fi
%     \else
%       \usebox0
%     \fi
%   \else
%     \lwbox
%   \fi
% \else
%   \usebox0
% \fi
% \end{quote}
% If you have a \xfile{docstrip.cfg} that configures and enables \docstrip's
% TDS installing feature, then some files can already be in the right
% place, see the documentation of \docstrip.
%
% \subsection{Refresh file name databases}
%
% If your \TeX~distribution
% (\teTeX, \mikTeX, \dots) relies on file name databases, you must refresh
% these. For example, \teTeX\ users run \verb|texhash| or
% \verb|mktexlsr|.
%
% \subsection{Some details for the interested}
%
% \paragraph{Unpacking with \LaTeX.}
% The \xfile{.dtx} chooses its action depending on the format:
% \begin{description}
% \item[\plainTeX:] Run \docstrip\ and extract the files.
% \item[\LaTeX:] Generate the documentation.
% \end{description}
% If you insist on using \LaTeX\ for \docstrip\ (really,
% \docstrip\ does not need \LaTeX), then inform the autodetect routine
% about your intention:
% \begin{quote}
%   \verb|latex \let\install=y\input{pdftexcmds.dtx}|
% \end{quote}
% Do not forget to quote the argument according to the demands
% of your shell.
%
% \paragraph{Generating the documentation.}
% You can use both the \xfile{.dtx} or the \xfile{.drv} to generate
% the documentation. The process can be configured by the
% configuration file \xfile{ltxdoc.cfg}. For instance, put this
% line into this file, if you want to have A4 as paper format:
% \begin{quote}
%   \verb|\PassOptionsToClass{a4paper}{article}|
% \end{quote}
% An example follows how to generate the
% documentation with pdf\LaTeX:
% \begin{quote}
%\begin{verbatim}
%pdflatex pdftexcmds.dtx
%bibtex pdftexcmds.aux
%makeindex -s gind.ist pdftexcmds.idx
%pdflatex pdftexcmds.dtx
%makeindex -s gind.ist pdftexcmds.idx
%pdflatex pdftexcmds.dtx
%\end{verbatim}
% \end{quote}
%
% \printbibliography[
%   heading=bibnumbered,
% ]
%
% \begin{History}
%   \begin{Version}{2007/11/11 v0.1}
%   \item
%     First version.
%   \end{Version}
%   \begin{Version}{2007/11/12 v0.2}
%   \item
%     Short description fixed.
%   \end{Version}
%   \begin{Version}{2007/12/12 v0.3}
%   \item
%     Organization of Lua code as module.
%   \end{Version}
%   \begin{Version}{2009/04/10 v0.4}
%   \item
%     Adaptation for syntax change of \cs{directlua} in
%     \hologo{LuaTeX} 0.36.
%   \end{Version}
%   \begin{Version}{2009/09/22 v0.5}
%   \item
%     \cs{pdf@primitive}, \cs{pdf@ifprimitive} added.
%   \item
%     \hologo{XeTeX}'s variants are detected for
%     \cs{pdf@shellescape}, \cs{pdf@strcmp}, \cs{pdf@primitive},
%     \cs{pdf@ifprimitive}.
%   \end{Version}
%   \begin{Version}{2009/09/23 v0.6}
%   \item
%     Macro \cs{pdf@isprimitive} added.
%   \end{Version}
%   \begin{Version}{2009/12/12 v0.7}
%   \item
%     Short info shortened.
%   \end{Version}
%   \begin{Version}{2010/03/01 v0.8}
%   \item
%     Required date for package \xpackage{ifluatex} updated.
%   \end{Version}
%   \begin{Version}{2010/04/01 v0.9}
%   \item
%     Use \cs{ifeof18} for defining \cs{pdf@shellescape} between
%     \hologo{pdfTeX} 1.21a (inclusive) and 1.30.0 (exclusive).
%   \end{Version}
%   \begin{Version}{2010/11/04 v0.10}
%   \item
%     \cs{pdf@draftmode}, \cs{pdf@ifdraftmode} and
%     \cs{pdf@setdraftmode} added.
%   \end{Version}
%   \begin{Version}{2010/11/11 v0.11}
%   \item
%     Missing \cs{RequirePackage} for package \xpackage{ifpdf} added.
%   \end{Version}
%   \begin{Version}{2011/01/30 v0.12}
%   \item
%     Already loaded package files are not input in \hologo{plainTeX}.
%   \end{Version}
%   \begin{Version}{2011/03/04 v0.13}
%   \item
%     Improved Lua function \texttt{shellescape} that also
%     uses the result of \texttt{os.execute()} (thanks to Philipp Stephani).
%   \end{Version}
%   \begin{Version}{2011/04/10 v0.14}
%   \item
%     Version check of loaded module added.
%   \item
%     Patch for bug in \hologo{LuaTeX} between 0.40.6 and 0.65 that
%     is fixed in revision 4096.
%   \end{Version}
%   \begin{Version}{2011/04/16 v0.15}
%   \item
%     \hologo{LuaTeX}: \cs{pdf@shellescape} is only supported
%     for version 0.70.0 and higher due to a bug, \texttt{os.execute()}
%     crashes in some circumstances. Fixed in \hologo{LuaTeX}
%     beta-0.70.0, revision 4167.
%   \end{Version}
%   \begin{Version}{2011/04/22 v0.16}
%   \item
%     Previous fix was not working due to a wrong catcode of digit
%     zero (due to easily support the old \cs{directlua0}).
%     The version border is lowered to 0.68, because some
%     beta-0.67.0 seems also to work.
%   \end{Version}
%   \begin{Version}{2011/06/29 v0.17}
%   \item
%     Documentation addition to \cs{pdf@shellescape}.
%   \end{Version}
%   \begin{Version}{2011/07/01 v0.18}
%   \item
%     Add Lua module loading in \cs{everyjob} for \hologo{iniTeX}
%     (\hologo{LuaTeX} only).
%   \end{Version}
%   \begin{Version}{2011/07/28 v0.19}
%   \item
%     Missing space in an info message added (Martin M\"unch).
%   \end{Version}
%   \begin{Version}{2011/11/29 v0.20}
%   \item
%     \cs{pdf@resettimer} and \cs{pdf@elapsedtime} added
%     (thanks Andy Thomas).
%   \end{Version}
%   \begin{Version}{2016/05/10 v0.21}
%   \item
%      local unpack added
%     (thanks \'{E}lie Roux).
%   \end{Version}
%   \begin{Version}{2016/05/21 v0.22}
%   \item
%     adjust \cs{textbackslas}h usage in bib file for biber bug.
%   \end{Version}
%   \begin{Version}{2016/10/02 v0.23}
%   \item
%     add file.close to lua filehandles (github pull request).
%   \end{Version}
%   \begin{Version}{2017/01/29 v0.24}
%   \item
%     Avoid loading luatex-loader for current luatex. (Use
%     pdftexcmds.lua not oberdiek.pdftexcmds.lua to simplify file
%     search with standard require)
%   \end{Version}
%   \begin{Version}{2017/03/19 v0.25}
%   \item
%     New \cs{pdf@shellescape} for Lua\TeX, see github issue 20.
%   \end{Version}
%   \begin{Version}{2018/01/21 v0.26}
%   \item
%     use rb not r mode for file open github issue 34.
%   \end{Version}
%   \begin{Version}{2018/01/30 v0.27}
%   \item
%     \cs{pdf@mdfivesum} for \hologo{XeTeX}
%   \end{Version}
%   \begin{Version}{2018/09/07 v0.28}
%   \item
%     Fix catcode regime in luatex sprint for \cs{pdf@shellescape} GH issue 45
%   \end{Version}
%   \begin{Version}{2018/09/10 v0.29}
%   \item
%     Actually do the fix described above in the code, not just document it.
%   \end{Version}
%   \begin{Version}{2019/07/25 v0.30}
%   \item
%     remove uses of module function, see PR70
%   \end{Version}
% \end{History}
%
% \PrintIndex
%
% \Finale
\endinput
|
% \end{quote}
% Do not forget to quote the argument according to the demands
% of your shell.
%
% \paragraph{Generating the documentation.}
% You can use both the \xfile{.dtx} or the \xfile{.drv} to generate
% the documentation. The process can be configured by the
% configuration file \xfile{ltxdoc.cfg}. For instance, put this
% line into this file, if you want to have A4 as paper format:
% \begin{quote}
%   \verb|\PassOptionsToClass{a4paper}{article}|
% \end{quote}
% An example follows how to generate the
% documentation with pdf\LaTeX:
% \begin{quote}
%\begin{verbatim}
%pdflatex pdftexcmds.dtx
%bibtex pdftexcmds.aux
%makeindex -s gind.ist pdftexcmds.idx
%pdflatex pdftexcmds.dtx
%makeindex -s gind.ist pdftexcmds.idx
%pdflatex pdftexcmds.dtx
%\end{verbatim}
% \end{quote}
%
% \printbibliography[
%   heading=bibnumbered,
% ]
%
% \begin{History}
%   \begin{Version}{2007/11/11 v0.1}
%   \item
%     First version.
%   \end{Version}
%   \begin{Version}{2007/11/12 v0.2}
%   \item
%     Short description fixed.
%   \end{Version}
%   \begin{Version}{2007/12/12 v0.3}
%   \item
%     Organization of Lua code as module.
%   \end{Version}
%   \begin{Version}{2009/04/10 v0.4}
%   \item
%     Adaptation for syntax change of \cs{directlua} in
%     \hologo{LuaTeX} 0.36.
%   \end{Version}
%   \begin{Version}{2009/09/22 v0.5}
%   \item
%     \cs{pdf@primitive}, \cs{pdf@ifprimitive} added.
%   \item
%     \hologo{XeTeX}'s variants are detected for
%     \cs{pdf@shellescape}, \cs{pdf@strcmp}, \cs{pdf@primitive},
%     \cs{pdf@ifprimitive}.
%   \end{Version}
%   \begin{Version}{2009/09/23 v0.6}
%   \item
%     Macro \cs{pdf@isprimitive} added.
%   \end{Version}
%   \begin{Version}{2009/12/12 v0.7}
%   \item
%     Short info shortened.
%   \end{Version}
%   \begin{Version}{2010/03/01 v0.8}
%   \item
%     Required date for package \xpackage{ifluatex} updated.
%   \end{Version}
%   \begin{Version}{2010/04/01 v0.9}
%   \item
%     Use \cs{ifeof18} for defining \cs{pdf@shellescape} between
%     \hologo{pdfTeX} 1.21a (inclusive) and 1.30.0 (exclusive).
%   \end{Version}
%   \begin{Version}{2010/11/04 v0.10}
%   \item
%     \cs{pdf@draftmode}, \cs{pdf@ifdraftmode} and
%     \cs{pdf@setdraftmode} added.
%   \end{Version}
%   \begin{Version}{2010/11/11 v0.11}
%   \item
%     Missing \cs{RequirePackage} for package \xpackage{ifpdf} added.
%   \end{Version}
%   \begin{Version}{2011/01/30 v0.12}
%   \item
%     Already loaded package files are not input in \hologo{plainTeX}.
%   \end{Version}
%   \begin{Version}{2011/03/04 v0.13}
%   \item
%     Improved Lua function \texttt{shellescape} that also
%     uses the result of \texttt{os.execute()} (thanks to Philipp Stephani).
%   \end{Version}
%   \begin{Version}{2011/04/10 v0.14}
%   \item
%     Version check of loaded module added.
%   \item
%     Patch for bug in \hologo{LuaTeX} between 0.40.6 and 0.65 that
%     is fixed in revision 4096.
%   \end{Version}
%   \begin{Version}{2011/04/16 v0.15}
%   \item
%     \hologo{LuaTeX}: \cs{pdf@shellescape} is only supported
%     for version 0.70.0 and higher due to a bug, \texttt{os.execute()}
%     crashes in some circumstances. Fixed in \hologo{LuaTeX}
%     beta-0.70.0, revision 4167.
%   \end{Version}
%   \begin{Version}{2011/04/22 v0.16}
%   \item
%     Previous fix was not working due to a wrong catcode of digit
%     zero (due to easily support the old \cs{directlua0}).
%     The version border is lowered to 0.68, because some
%     beta-0.67.0 seems also to work.
%   \end{Version}
%   \begin{Version}{2011/06/29 v0.17}
%   \item
%     Documentation addition to \cs{pdf@shellescape}.
%   \end{Version}
%   \begin{Version}{2011/07/01 v0.18}
%   \item
%     Add Lua module loading in \cs{everyjob} for \hologo{iniTeX}
%     (\hologo{LuaTeX} only).
%   \end{Version}
%   \begin{Version}{2011/07/28 v0.19}
%   \item
%     Missing space in an info message added (Martin M\"unch).
%   \end{Version}
%   \begin{Version}{2011/11/29 v0.20}
%   \item
%     \cs{pdf@resettimer} and \cs{pdf@elapsedtime} added
%     (thanks Andy Thomas).
%   \end{Version}
%   \begin{Version}{2016/05/10 v0.21}
%   \item
%      local unpack added
%     (thanks \'{E}lie Roux).
%   \end{Version}
%   \begin{Version}{2016/05/21 v0.22}
%   \item
%     adjust \cs{textbackslas}h usage in bib file for biber bug.
%   \end{Version}
%   \begin{Version}{2016/10/02 v0.23}
%   \item
%     add file.close to lua filehandles (github pull request).
%   \end{Version}
%   \begin{Version}{2017/01/29 v0.24}
%   \item
%     Avoid loading luatex-loader for current luatex. (Use
%     pdftexcmds.lua not oberdiek.pdftexcmds.lua to simplify file
%     search with standard require)
%   \end{Version}
%   \begin{Version}{2017/03/19 v0.25}
%   \item
%     New \cs{pdf@shellescape} for Lua\TeX, see github issue 20.
%   \end{Version}
%   \begin{Version}{2018/01/21 v0.26}
%   \item
%     use rb not r mode for file open github issue 34.
%   \end{Version}
%   \begin{Version}{2018/01/30 v0.27}
%   \item
%     \cs{pdf@mdfivesum} for \hologo{XeTeX}
%   \end{Version}
%   \begin{Version}{2018/09/07 v0.28}
%   \item
%     Fix catcode regime in luatex sprint for \cs{pdf@shellescape} GH issue 45
%   \end{Version}
%   \begin{Version}{2018/09/10 v0.29}
%   \item
%     Actually do the fix described above in the code, not just document it.
%   \end{Version}
%   \begin{Version}{2019/07/25 v0.30}
%   \item
%     remove uses of module function, see PR70
%   \end{Version}
% \end{History}
%
% \PrintIndex
%
% \Finale
\endinput

%        (quote the arguments according to the demands of your shell)
%
% Documentation:
%    (a) If pdftexcmds.drv is present:
%           latex pdftexcmds.drv
%    (b) Without pdftexcmds.drv:
%           latex pdftexcmds.dtx; ...
%    The class ltxdoc loads the configuration file ltxdoc.cfg
%    if available. Here you can specify further options, e.g.
%    use A4 as paper format:
%       \PassOptionsToClass{a4paper}{article}
%
%    Programm calls to get the documentation (example):
%       pdflatex pdftexcmds.dtx
%       bibtex pdftexcmds.aux
%       makeindex -s gind.ist pdftexcmds.idx
%       pdflatex pdftexcmds.dtx
%       makeindex -s gind.ist pdftexcmds.idx
%       pdflatex pdftexcmds.dtx
%
% Installation:
%    TDS:tex/generic/oberdiek/pdftexcmds.sty
%    TDS:scripts/oberdiek/oberdiek.pdftexcmds.lua
%    TDS:scripts/oberdiek/pdftexcmds.lua
%    TDS:doc/latex/oberdiek/pdftexcmds.pdf
%    TDS:doc/latex/oberdiek/test/pdftexcmds-test1.tex
%    TDS:doc/latex/oberdiek/test/pdftexcmds-test2.tex
%    TDS:doc/latex/oberdiek/test/pdftexcmds-test-shell.tex
%    TDS:doc/latex/oberdiek/test/pdftexcmds-test-escape.tex
%    TDS:source/latex/oberdiek/pdftexcmds.dtx
%
%<*ignore>
\begingroup
  \catcode123=1 %
  \catcode125=2 %
  \def\x{LaTeX2e}%
\expandafter\endgroup
\ifcase 0\ifx\install y1\fi\expandafter
         \ifx\csname processbatchFile\endcsname\relax\else1\fi
         \ifx\fmtname\x\else 1\fi\relax
\else\csname fi\endcsname
%</ignore>
%<*install>
\input docstrip.tex
\Msg{************************************************************************}
\Msg{* Installation}
\Msg{* Package: pdftexcmds 2019/07/25 v0.30 Utility functions of pdfTeX for LuaTeX (HO)}
\Msg{************************************************************************}

\keepsilent
\askforoverwritefalse

\let\MetaPrefix\relax
\preamble

This is a generated file.

Project: pdftexcmds
Version: 2019/07/25 v0.30

Copyright (C) 2007, 2009-2011 by
   Heiko Oberdiek <heiko.oberdiek at googlemail.com>

This work may be distributed and/or modified under the
conditions of the LaTeX Project Public License, either
version 1.3c of this license or (at your option) any later
version. This version of this license is in
   https://www.latex-project.org/lppl/lppl-1-3c.txt
and the latest version of this license is in
   https://www.latex-project.org/lppl.txt
and version 1.3 or later is part of all distributions of
LaTeX version 2005/12/01 or later.

This work has the LPPL maintenance status "maintained".

The Current Maintainers of this work are
Heiko Oberdiek and the Oberdiek Package Support Group
https://github.com/ho-tex/oberdiek/issues


The Base Interpreter refers to any `TeX-Format',
because some files are installed in TDS:tex/generic//.

This work consists of the main source file pdftexcmds.dtx
and the derived files
   pdftexcmds.sty, pdftexcmds.pdf, pdftexcmds.ins, pdftexcmds.drv,
   pdftexcmds.bib, pdftexcmds-test1.tex, pdftexcmds-test2.tex,
   pdftexcmds-test-shell.tex, pdftexcmds-test-escape.tex,
   oberdiek.pdftexcmds.lua, pdftexcmds.lua.

\endpreamble
\let\MetaPrefix\DoubleperCent

\generate{%
  \file{pdftexcmds.ins}{\from{pdftexcmds.dtx}{install}}%
  \file{pdftexcmds.drv}{\from{pdftexcmds.dtx}{driver}}%
  \nopreamble
  \nopostamble
  \file{pdftexcmds.bib}{\from{pdftexcmds.dtx}{bib}}%
  \usepreamble\defaultpreamble
  \usepostamble\defaultpostamble
  \usedir{tex/generic/oberdiek}%
  \file{pdftexcmds.sty}{\from{pdftexcmds.dtx}{package}}%
%  \usedir{doc/latex/oberdiek/test}%
%  \file{pdftexcmds-test1.tex}{\from{pdftexcmds.dtx}{test1}}%
%  \file{pdftexcmds-test2.tex}{\from{pdftexcmds.dtx}{test2}}%
%  \file{pdftexcmds-test-shell.tex}{\from{pdftexcmds.dtx}{test-shell}}%
%  \file{pdftexcmds-test-escape.tex}{\from{pdftexcmds.dtx}{test-escape}}%
  \nopreamble
  \nopostamble
%  \usedir{source/latex/oberdiek/catalogue}%
%  \file{pdftexcmds.xml}{\from{pdftexcmds.dtx}{catalogue}}%
}
\def\MetaPrefix{-- }
\def\defaultpostamble{%
  \MetaPrefix^^J%
  \MetaPrefix\space End of File `\outFileName'.%
}
\def\currentpostamble{\defaultpostamble}%
\generate{%
  \usedir{scripts/oberdiek}%
  \file{oberdiek.pdftexcmds.lua}{\from{pdftexcmds.dtx}{lua}}%
  \file{pdftexcmds.lua}{\from{pdftexcmds.dtx}{lua}}%
}

\catcode32=13\relax% active space
\let =\space%
\Msg{************************************************************************}
\Msg{*}
\Msg{* To finish the installation you have to move the following}
\Msg{* file into a directory searched by TeX:}
\Msg{*}
\Msg{*     pdftexcmds.sty}
\Msg{*}
\Msg{* And install the following script files:}
\Msg{*}
\Msg{*     oberdiek.pdftexcmds.lua, pdftexcmds.lua}
\Msg{*}
\Msg{* To produce the documentation run the file `pdftexcmds.drv'}
\Msg{* through LaTeX.}
\Msg{*}
\Msg{* Happy TeXing!}
\Msg{*}
\Msg{************************************************************************}

\endbatchfile
%</install>
%<*bib>
@online{AndyThomas:Analog,
  author={Thomas, Andy},
  title={Analog of {\texttt{\csname textbackslash\endcsname}pdfelapsedtime} for
      {\hologo{LuaTeX}} and {\hologo{XeTeX}}},
  url={http://tex.stackexchange.com/a/32531},
  urldate={2011-11-29},
}
%</bib>
%<*ignore>
\fi
%</ignore>
%<*driver>
\NeedsTeXFormat{LaTeX2e}
\ProvidesFile{pdftexcmds.drv}%
  [2019/07/25 v0.30 Utility functions of pdfTeX for LuaTeX (HO)]%
\documentclass{ltxdoc}
\usepackage{holtxdoc}[2011/11/22]
\usepackage{paralist}
\usepackage{csquotes}
\usepackage[
  backend=bibtex,
  bibencoding=ascii,
  alldates=iso8601,
]{biblatex}[2011/11/13]
\bibliography{oberdiek-source}
\bibliography{pdftexcmds}
\begin{document}
  \DocInput{pdftexcmds.dtx}%
\end{document}
%</driver>
% \fi
%
%
% \CharacterTable
%  {Upper-case    \A\B\C\D\E\F\G\H\I\J\K\L\M\N\O\P\Q\R\S\T\U\V\W\X\Y\Z
%   Lower-case    \a\b\c\d\e\f\g\h\i\j\k\l\m\n\o\p\q\r\s\t\u\v\w\x\y\z
%   Digits        \0\1\2\3\4\5\6\7\8\9
%   Exclamation   \!     Double quote  \"     Hash (number) \#
%   Dollar        \$     Percent       \%     Ampersand     \&
%   Acute accent  \'     Left paren    \(     Right paren   \)
%   Asterisk      \*     Plus          \+     Comma         \,
%   Minus         \-     Point         \.     Solidus       \/
%   Colon         \:     Semicolon     \;     Less than     \<
%   Equals        \=     Greater than  \>     Question mark \?
%   Commercial at \@     Left bracket  \[     Backslash     \\
%   Right bracket \]     Circumflex    \^     Underscore    \_
%   Grave accent  \`     Left brace    \{     Vertical bar  \|
%   Right brace   \}     Tilde         \~}
%
% \GetFileInfo{pdftexcmds.drv}
%
% \title{The \xpackage{pdftexcmds} package}
% \date{2019/07/25 v0.30}
% \author{Heiko Oberdiek\thanks
% {Please report any issues at \url{https://github.com/ho-tex/oberdiek/issues}}}
%
% \maketitle
%
% \begin{abstract}
% \hologo{LuaTeX} provides most of the commands of \hologo{pdfTeX} 1.40. However
% a number of utility functions are removed. This package tries to fill
% the gap and implements some of the missing primitive using Lua.
% \end{abstract}
%
% \tableofcontents
%
% \def\csi#1{\texttt{\textbackslash\textit{#1}}}
%
% \section{Documentation}
%
% Some primitives of \hologo{pdfTeX} \cite{pdftex-manual}
% are not defined by \hologo{LuaTeX} \cite{luatex-manual}.
% This package implements macro based solutions using Lua code
% for the following missing \hologo{pdfTeX} primitives;
% \begin{compactitem}
% \item \cs{pdfstrcmp}
% \item \cs{pdfunescapehex}
% \item \cs{pdfescapehex}
% \item \cs{pdfescapename}
% \item \cs{pdfescapestring}
% \item \cs{pdffilesize}
% \item \cs{pdffilemoddate}
% \item \cs{pdffiledump}
% \item \cs{pdfmdfivesum}
% \item \cs{pdfresettimer}
% \item \cs{pdfelapsedtime}
% \item |\immediate\write18|
% \end{compactitem}
% The original names of the primitives cannot be used:
% \begin{itemize}
% \item
% The syntax for their arguments cannot easily
% simulated by macros. The primitives using key words
% such as |file| (\cs{pdfmdfivesum}) or |offset| and |length|
% (\cs{pdffiledump}) and uses \meta{general text} for the other
% arguments. Using token registers assignments, \meta{general text} could
% be catched. However, the simulated primitives are expandable
% and register assignments would destroy this important property.
% (\meta{general text} allows something like |\expandafter\bgroup ...}|.)
% \item
% The original primitives can be expanded using one expansion step.
% The new macros need two expansion steps because of the additional
% macro expansion. Example:
% \begin{quote}
%   |\expandafter\foo\pdffilemoddate{file}|\\
%   vs.\\
%   |\expandafter\expandafter\expandafter|\\
%   |\foo\pdf@filemoddate{file}|
% \end{quote}
% \end{itemize}
%
% \hologo{LuaTeX} isn't stable yet and thus the status of this package is
% \emph{experimental}. Feedback is welcome.
%
% \subsection{General principles}
%
% \begin{description}
% \item[Naming convention:]
%   Usually this package defines a macro |\pdf@|\meta{cmd} if
%   \hologo{pdfTeX} provides |\pdf|\meta{cmd}.
% \item[Arguments:] The order of arguments in |\pdf@|\meta{cmd}
%   is the same as for the corresponding primitive of \hologo{pdfTeX}.
%   The arguments are ordinary undelimited \hologo{TeX} arguments,
%   no \meta{general text} and without additional keywords.
% \item[Expandibility:]
%   The macro |\pdf@|\meta{cmd} is expandable if the
%   corresponding \hologo{pdfTeX} primitive has this property.
%   Exact two expansion steps are necessary (first is the macro
%   expansion) except for \cs{pdf@primitive} and \cs{pdf@ifprimitive}.
%   The latter ones are not macros, but have the direct meaning of the
%   primitive.
% \item[Without \hologo{LuaTeX}:]
%   The macros |\pdf@|\meta{cmd} are mapped to the commands
%   of \hologo{pdfTeX} if they are available. Otherwise they are undefined.
% \item[Availability:]
%   The macros that the packages provides are undefined, if
%   the necessary primitives are not found and cannot be
%   implemented by Lua.
% \end{description}
%
% \subsection{Macros}
%
% \subsubsection[Strings]{Strings \cite[``7.15 Strings'']{pdftex-manual}}
%
% \begin{declcs}{pdf@strcmp} \M{stringA} \M{stringB}
% \end{declcs}
% Same as |\pdfstrcmp{|\meta{stringA}|}{|\meta{stringB}|}|.
%
% \begin{declcs}{pdf@unescapehex} \M{string}
% \end{declcs}
% Same as |\pdfunescapehex{|\meta{string}|}|.
% The argument is a byte string given in hexadecimal notation.
% The result are character tokens from 0 until 255 with
% catcode 12 and the space with catcode 10.
%
% \begin{declcs}{pdf@escapehex} \M{string}\\
%   \cs{pdf@escapestring} \M{string}\\
%   \cs{pdf@escapename} \M{string}
% \end{declcs}
% Same as the primitives of \hologo{pdfTeX}. However \hologo{pdfTeX} does not
% know about characters with codes 256 and larger. Thus the
% string is treated as byte string, characters with more than
% eight bits are ignored.
%
% \subsubsection[Files]{Files \cite[``7.18 Files'']{pdftex-manual}}
%
% \begin{declcs}{pdf@filesize} \M{filename}
% \end{declcs}
% Same as |\pdffilesize{|\meta{filename}|}|.
%
% \begin{declcs}{pdf@filemoddate} \M{filename}
% \end{declcs}
% Same as |\pdffilemoddate{|\meta{filename}|}|.
%
% \begin{declcs}{pdf@filedump} \M{offset} \M{length} \M{filename}
% \end{declcs}
% Same as |\pdffiledump offset| \meta{offset} |length| \meta{length}
% |{|\meta{filename}|}|. Both \meta{offset} and \meta{length} must
% not be empty, but must be a valid \hologo{TeX} number.
%
% \begin{declcs}{pdf@mdfivesum} \M{string}
% \end{declcs}
% Same as |\pdfmdfivesum{|\meta{string}|}|. Keyword |file| is supported
% by macro \cs{pdf@filemdfivesum}.
%
% \begin{declcs}{pdf@filemdfivesum} \M{filename}
% \end{declcs}
% Same as |\pdfmdfivesum file{|\meta{filename}|}|.
%
% \subsubsection[Timekeeping]{Timekeeping \cite[``7.17 Timekeeping'']{pdftex-manual}}
%
% The timekeeping macros are based on Andy Thomas' work \cite{AndyThomas:Analog}.
%
% \begin{declcs}{pdf@resettimer}
% \end{declcs}
% Same as \cs{pdfresettimer}, it resets the internal timer.
%
% \begin{declcs}{pdf@elapsedtime}
% \end{declcs}
% Same as \cs{pdfelapsedtime}. It behaves like a read-only integer.
% For printing purposes it can be prefixed by \cs{the} or \cs{number}.
% It measures the time in scaled seconds (seconds multiplied with 65536)
% since the latest call of \cs{pdf@resettimer} or start of
% program/package. The resolution, the shortest time interval that
% can be measured, depends on the program and system.
% \begin{itemize}
% \item \hologo{pdfTeX} with |gettimeofday|: $\ge$ 1/65536\,s
% \item \hologo{pdfTeX} with |ftime|: $\ge$ 1\,ms
% \item \hologo{pdfTeX} with |time|: $\ge$ 1\,s
% \item \hologo{LuaTeX}: $\ge$ 10\,ms\\
%  (|os.clock()| returns a float number with two decimal digits in
%  \hologo{LuaTeX} beta-0.70.1-2011061416 (rev 4277)).
% \end{itemize}
%
% \subsubsection[Miscellaneous]{Miscellaneous \cite[``7.21 Miscellaneous'']{pdftex-manual}}
%
% \begin{declcs}{pdf@draftmode}
% \end{declcs}
% If the \TeX\ compiler knows \cs{pdfdraftmode} or \cs{draftmode}
% (\hologo{pdfTeX},
% \hologo{LuaTeX}), then \cs{pdf@draftmode} returns, whether
% this mode is enabled. The result is an implicit number:
% one means the draft mode is available and enabled.
% If the value is zero, then the mode is not active or
% \cs{pdfdraftmode} is not available.
% An explicit number is yielded by \cs{number}\cs{pdf@draftmode}.
% The macro cannot
% be used to change the mode, see \cs{pdf@setdraftmode}.
%
% \begin{declcs}{pdf@ifdraftmode} \M{true} \M{false}
% \end{declcs}
% If \cs{pdfdraftmode} is available and enabled, \meta{true} is
% called, otherwise \meta{false} is executed.
%
% \begin{declcs}{pdf@setdraftmode} \M{value}
% \end{declcs}
% Macro \cs{pdf@setdraftmode} expects the number zero or one as
% \meta{value}. Zero deactivates the mode and one enables the draft mode.
% The macro does not have an effect, if the feature \cs{pdfdraftmode} is not
% available.
%
% \begin{declcs}{pdf@shellescape}
% \end{declcs}
% Same as |\pdfshellescape|. It is or expands to |1| if external
% commands can be executed and |0| otherwise. In \hologo{pdfTeX} external
% commands must be enabled first by command line option or
% configuration option. In \hologo{LuaTeX} option |--safer| disables
% the execution of external commands.
%
% In \hologo{LuaTeX} before 0.68.0 \cs{pdf@shellescape} is not
% available due to a bug in |os.execute()|. The argumentless form
% crashes in some circumstances with segmentation fault.
% (It is fixed in version 0.68.0 or revision 4167 of \hologo{LuaTeX}.
% and packported to some version of 0.67.0).
%
% Hints for usage:
% \begin{itemize}
% \item Before its use \cs{pdf@shellescape} should be tested,
% whether it is available. Example with package \xpackage{ltxcmds}
% (loaded by package \xpackage{pdftexcmds}):
%\begin{quote}
%\begin{verbatim}
%\ltx@IfUndefined{pdf@shellescape}{%
%  % \pdf@shellescape is undefined
%}{%
%  % \pdf@shellescape is available
%}
%\end{verbatim}
%\end{quote}
% Use \cs{ltx@ifundefined} in expandable contexts.
% \item \cs{pdf@shellescape} might be a numerical constant,
% expands to the primitive, or expands to a plain number.
% Therefore use it in contexts where these differences does not matter.
% \item Use in comparisons, e.g.:
%   \begin{quote}
%     |\ifnum\pdf@shellescape=0 ...|
%   \end{quote}
% \item Print the number: |\number\pdf@shellescape|
% \end{itemize}
%
% \begin{declcs}{pdf@system} \M{cmdline}
% \end{declcs}
% It is a wrapper for |\immediate\write18| in \hologo{pdfTeX} or
% |os.execute| in \hologo{LuaTeX}.
%
% In theory |os.execute|
% returns a status number. But its meaning is quite
% undefined. Are there some reliable properties?
% Does it make sense to provide an user interface to
% this status exit code?
%
% \begin{declcs}{pdf@primitive} \csi{cmd}
% \end{declcs}
% Same as \cs{pdfprimitive} in \hologo{pdfTeX} or \hologo{LuaTeX}.
% In \hologo{XeTeX} the
% primitive is called \cs{primitive}. Despite the current definition
% of the command \csi{cmd}, it's meaning as primitive is used.
%
% \begin{declcs}{pdf@ifprimitive} \csi{cmd}
% \end{declcs}
% Same as \cs{ifpdfprimitive} in \hologo{pdfTeX} or
% \hologo{LuaTeX}. \hologo{XeTeX} calls
% it \cs{ifprimitive}. It is a switch that checks if the command
% \csi{cmd} has it's primitive meaning.
%
% \subsubsection{Additional macro: \cs{pdf@isprimitive}}
%
% \begin{declcs}{pdf@isprimitive} \csi{cmd1} \csi{cmd2} \M{true} \M{false}
% \end{declcs}
% If \csi{cmd1} has the primitive meaning given by the primitive name
% of \csi{cmd2}, then the argument \meta{true} is executed, otherwise
% \meta{false}. The macro \cs{pdf@isprimitive} is expandable.
% Internally it checks the result of \cs{meaning} and is therefore
% available for all \hologo{TeX} variants, even the original \hologo{TeX}.
% Example with \hologo{LaTeX}:
%\begin{quote}
%\begin{verbatim}
%\makeatletter
%\pdf@isprimitive{@@input}{input}{%
%  \typeout{\string\@@input\space is original\string\input}%
%}{%
%  \typeout{Oops, \string\@@input\space is not the %
%           original\string\input}%
%}
%\end{verbatim}
%\end{quote}
%
% \subsubsection{Experimental}
%
% \begin{declcs}{pdf@unescapehexnative} \M{string}\\
%   \cs{pdf@escapehexnative} \M{string}\\
%   \cs{pdf@escapenamenative} \M{string}\\
%   \cs{pdf@mdfivesumnative} \M{string}
% \end{declcs}
% The variants without |native| in the macro name are supposed to
% be compatible with \hologo{pdfTeX}. However characters with more than
% eight bits are not supported and are ignored. If \hologo{LuaTeX} is
% running, then its UTF-8 coded strings are used. Thus the full
% unicode character range is supported. However the result
% differs from \hologo{pdfTeX} for characters with eight or more bits.
%
% \begin{declcs}{pdf@pipe} \M{cmdline}
% \end{declcs}
% It calls \meta{cmdline} and returns the output of the external
% program in the usual manner as byte string (catcode 12, space with
% catcode 10). The Lua documentation says, that the used |io.popen|
% may not be available on all platforms. Then macro \cs{pdf@pipe}
% is undefined.
%
% \StopEventually{
% }
%
% \section{Implementation}
%
%    \begin{macrocode}
%<*package>
%    \end{macrocode}
%
% \subsection{Reload check and package identification}
%    Reload check, especially if the package is not used with \LaTeX.
%    \begin{macrocode}
\begingroup\catcode61\catcode48\catcode32=10\relax%
  \catcode13=5 % ^^M
  \endlinechar=13 %
  \catcode35=6 % #
  \catcode39=12 % '
  \catcode44=12 % ,
  \catcode45=12 % -
  \catcode46=12 % .
  \catcode58=12 % :
  \catcode64=11 % @
  \catcode123=1 % {
  \catcode125=2 % }
  \expandafter\let\expandafter\x\csname ver@pdftexcmds.sty\endcsname
  \ifx\x\relax % plain-TeX, first loading
  \else
    \def\empty{}%
    \ifx\x\empty % LaTeX, first loading,
      % variable is initialized, but \ProvidesPackage not yet seen
    \else
      \expandafter\ifx\csname PackageInfo\endcsname\relax
        \def\x#1#2{%
          \immediate\write-1{Package #1 Info: #2.}%
        }%
      \else
        \def\x#1#2{\PackageInfo{#1}{#2, stopped}}%
      \fi
      \x{pdftexcmds}{The package is already loaded}%
      \aftergroup\endinput
    \fi
  \fi
\endgroup%
%    \end{macrocode}
%    Package identification:
%    \begin{macrocode}
\begingroup\catcode61\catcode48\catcode32=10\relax%
  \catcode13=5 % ^^M
  \endlinechar=13 %
  \catcode35=6 % #
  \catcode39=12 % '
  \catcode40=12 % (
  \catcode41=12 % )
  \catcode44=12 % ,
  \catcode45=12 % -
  \catcode46=12 % .
  \catcode47=12 % /
  \catcode58=12 % :
  \catcode64=11 % @
  \catcode91=12 % [
  \catcode93=12 % ]
  \catcode123=1 % {
  \catcode125=2 % }
  \expandafter\ifx\csname ProvidesPackage\endcsname\relax
    \def\x#1#2#3[#4]{\endgroup
      \immediate\write-1{Package: #3 #4}%
      \xdef#1{#4}%
    }%
  \else
    \def\x#1#2[#3]{\endgroup
      #2[{#3}]%
      \ifx#1\@undefined
        \xdef#1{#3}%
      \fi
      \ifx#1\relax
        \xdef#1{#3}%
      \fi
    }%
  \fi
\expandafter\x\csname ver@pdftexcmds.sty\endcsname
\ProvidesPackage{pdftexcmds}%
  [2019/07/25 v0.30 Utility functions of pdfTeX for LuaTeX (HO)]%
%    \end{macrocode}
%
% \subsection{Catcodes}
%
%    \begin{macrocode}
\begingroup\catcode61\catcode48\catcode32=10\relax%
  \catcode13=5 % ^^M
  \endlinechar=13 %
  \catcode123=1 % {
  \catcode125=2 % }
  \catcode64=11 % @
  \def\x{\endgroup
    \expandafter\edef\csname pdftexcmds@AtEnd\endcsname{%
      \endlinechar=\the\endlinechar\relax
      \catcode13=\the\catcode13\relax
      \catcode32=\the\catcode32\relax
      \catcode35=\the\catcode35\relax
      \catcode61=\the\catcode61\relax
      \catcode64=\the\catcode64\relax
      \catcode123=\the\catcode123\relax
      \catcode125=\the\catcode125\relax
    }%
  }%
\x\catcode61\catcode48\catcode32=10\relax%
\catcode13=5 % ^^M
\endlinechar=13 %
\catcode35=6 % #
\catcode64=11 % @
\catcode123=1 % {
\catcode125=2 % }
\def\TMP@EnsureCode#1#2{%
  \edef\pdftexcmds@AtEnd{%
    \pdftexcmds@AtEnd
    \catcode#1=\the\catcode#1\relax
  }%
  \catcode#1=#2\relax
}
\TMP@EnsureCode{0}{12}%
\TMP@EnsureCode{1}{12}%
\TMP@EnsureCode{2}{12}%
\TMP@EnsureCode{10}{12}% ^^J
\TMP@EnsureCode{33}{12}% !
\TMP@EnsureCode{34}{12}% "
\TMP@EnsureCode{38}{4}% &
\TMP@EnsureCode{39}{12}% '
\TMP@EnsureCode{40}{12}% (
\TMP@EnsureCode{41}{12}% )
\TMP@EnsureCode{42}{12}% *
\TMP@EnsureCode{43}{12}% +
\TMP@EnsureCode{44}{12}% ,
\TMP@EnsureCode{45}{12}% -
\TMP@EnsureCode{46}{12}% .
\TMP@EnsureCode{47}{12}% /
\TMP@EnsureCode{58}{12}% :
\TMP@EnsureCode{60}{12}% <
\TMP@EnsureCode{62}{12}% >
\TMP@EnsureCode{91}{12}% [
\TMP@EnsureCode{93}{12}% ]
\TMP@EnsureCode{94}{7}% ^ (superscript)
\TMP@EnsureCode{95}{12}% _ (other)
\TMP@EnsureCode{96}{12}% `
\TMP@EnsureCode{126}{12}% ~ (other)
\edef\pdftexcmds@AtEnd{%
  \pdftexcmds@AtEnd
  \escapechar=\number\escapechar\relax
  \noexpand\endinput
}
\escapechar=92 %
%    \end{macrocode}
%
% \subsection{Load packages}
%
%    \begin{macrocode}
\begingroup\expandafter\expandafter\expandafter\endgroup
\expandafter\ifx\csname RequirePackage\endcsname\relax
  \def\TMP@RequirePackage#1[#2]{%
    \begingroup\expandafter\expandafter\expandafter\endgroup
    \expandafter\ifx\csname ver@#1.sty\endcsname\relax
      \input #1.sty\relax
    \fi
  }%
  \TMP@RequirePackage{infwarerr}[2007/09/09]%
  \TMP@RequirePackage{ifluatex}[2010/03/01]%
  \TMP@RequirePackage{ltxcmds}[2010/12/02]%
  \TMP@RequirePackage{ifpdf}[2010/09/13]%
\else
  \RequirePackage{infwarerr}[2007/09/09]%
  \RequirePackage{ifluatex}[2010/03/01]%
  \RequirePackage{ltxcmds}[2010/12/02]%
  \RequirePackage{ifpdf}[2010/09/13]%
\fi
%    \end{macrocode}
%
% \subsection{Without \hologo{LuaTeX}}
%
%    \begin{macrocode}
\ifluatex
\else
  \@PackageInfoNoLine{pdftexcmds}{LuaTeX not detected}%
  \def\pdftexcmds@nopdftex{%
    \@PackageInfoNoLine{pdftexcmds}{pdfTeX >= 1.30 not detected}%
    \let\pdftexcmds@nopdftex\relax
  }%
  \def\pdftexcmds@temp#1{%
    \begingroup\expandafter\expandafter\expandafter\endgroup
    \expandafter\ifx\csname pdf#1\endcsname\relax
      \pdftexcmds@nopdftex
    \else
      \expandafter\def\csname pdf@#1\expandafter\endcsname
      \expandafter##\expandafter{%
        \csname pdf#1\endcsname
      }%
    \fi
  }%
  \pdftexcmds@temp{strcmp}%
  \pdftexcmds@temp{escapehex}%
  \let\pdf@escapehexnative\pdf@escapehex
  \pdftexcmds@temp{unescapehex}%
  \let\pdf@unescapehexnative\pdf@unescapehex
  \pdftexcmds@temp{escapestring}%
  \pdftexcmds@temp{escapename}%
  \pdftexcmds@temp{filesize}%
  \pdftexcmds@temp{filemoddate}%
  \begingroup\expandafter\expandafter\expandafter\endgroup
  \expandafter\ifx\csname pdfshellescape\endcsname\relax
    \pdftexcmds@nopdftex
    \ltx@IfUndefined{pdftexversion}{%
    }{%
      \ifnum\pdftexversion>120 % 1.21a supports \ifeof18
        \ifeof18 %
          \chardef\pdf@shellescape=0 %
        \else
          \chardef\pdf@shellescape=1 %
        \fi
      \fi
    }%
  \else
    \def\pdf@shellescape{%
      \pdfshellescape
    }%
  \fi
  \begingroup\expandafter\expandafter\expandafter\endgroup
  \expandafter\ifx\csname pdffiledump\endcsname\relax
    \pdftexcmds@nopdftex
  \else
    \def\pdf@filedump#1#2#3{%
      \pdffiledump offset#1 length#2{#3}%
    }%
  \fi
%    \end{macrocode}
%    \begin{macrocode}
  \begingroup\expandafter\expandafter\expandafter\endgroup
  \expandafter\ifx\csname pdfmdfivesum\endcsname\relax
    \begingroup\expandafter\expandafter\expandafter\endgroup
    \expandafter\ifx\csname mdfivesum\endcsname\relax
      \pdftexcmds@nopdftex
    \else
      \def\pdf@mdfivesum#{\mdfivesum}%
      \let\pdf@mdfivesumnative\pdf@mdfivesum
      \def\pdf@filemdfivesum#{\mdfivesum file}%
    \fi
  \else
    \def\pdf@mdfivesum#{\pdfmdfivesum}%
    \let\pdf@mdfivesumnative\pdf@mdfivesum
    \def\pdf@filemdfivesum#{\pdfmdfivesum file}%
  \fi
%    \end{macrocode}
%    \begin{macrocode}
  \def\pdf@system#{%
    \immediate\write18%
  }%
  \def\pdftexcmds@temp#1{%
    \begingroup\expandafter\expandafter\expandafter\endgroup
    \expandafter\ifx\csname pdf#1\endcsname\relax
      \pdftexcmds@nopdftex
    \else
      \expandafter\let\csname pdf@#1\expandafter\endcsname
      \csname pdf#1\endcsname
    \fi
  }%
  \pdftexcmds@temp{resettimer}%
  \pdftexcmds@temp{elapsedtime}%
\fi
%    \end{macrocode}
%
% \subsection{\cs{pdf@primitive}, \cs{pdf@ifprimitive}}
%
%    Since version 1.40.0 \hologo{pdfTeX} has \cs{pdfprimitive} and
%    \cs{ifpdfprimitive}. And \cs{pdfprimitive} was fixed in
%    version 1.40.4.
%
%    \hologo{XeTeX} provides them under the name \cs{primitive} and
%    \cs{ifprimitive}. \hologo{LuaTeX} knows both name variants,
%    but they have possibly to be enabled first (|tex.enableprimitives|).
%
%    Depending on the format TeX Live uses a prefix |luatex|.
%
%    Caution: \cs{let} must be used for the definition of
%    the macros, especially because of \cs{ifpdfprimitive}.
%
% \subsubsection{Using \hologo{LuaTeX}'s \texttt{tex.enableprimitives}}
%
%    \begin{macrocode}
\ifluatex
%    \end{macrocode}
%    \begin{macro}{\pdftexcmds@directlua}
%    \begin{macrocode}
  \ifnum\luatexversion<36 %
    \def\pdftexcmds@directlua{\directlua0 }%
  \else
    \let\pdftexcmds@directlua\directlua
  \fi
%    \end{macrocode}
%    \end{macro}
%
%    \begin{macrocode}
  \begingroup
    \newlinechar=10 %
    \endlinechar=\newlinechar
    \pdftexcmds@directlua{%
      if tex.enableprimitives then
        tex.enableprimitives(
          'pdf@',
          {'primitive', 'ifprimitive', 'pdfdraftmode','draftmode'}
        )
        tex.enableprimitives('', {'luaescapestring'})
      end
    }%
  \endgroup %
%    \end{macrocode}
%
%    \begin{macrocode}
\fi
%    \end{macrocode}
%
% \subsubsection{Trying various names to find the primitives}
%
%    \begin{macro}{\pdftexcmds@strip@prefix}
%    \begin{macrocode}
\def\pdftexcmds@strip@prefix#1>{}
%    \end{macrocode}
%    \end{macro}
%    \begin{macrocode}
\def\pdftexcmds@temp#1#2#3{%
  \begingroup\expandafter\expandafter\expandafter\endgroup
  \expandafter\ifx\csname pdf@#1\endcsname\relax
    \begingroup
      \def\x{#3}%
      \edef\x{\expandafter\pdftexcmds@strip@prefix\meaning\x}%
      \escapechar=-1 %
      \edef\y{\expandafter\meaning\csname#2\endcsname}%
    \expandafter\endgroup
    \ifx\x\y
      \expandafter\let\csname pdf@#1\expandafter\endcsname
      \csname #2\endcsname
    \fi
  \fi
}
%    \end{macrocode}
%
%    \begin{macro}{\pdf@primitive}
%    \begin{macrocode}
\pdftexcmds@temp{primitive}{pdfprimitive}{pdfprimitive}% pdfTeX, oldLuaTeX
\pdftexcmds@temp{primitive}{primitive}{primitive}% XeTeX, luatex
\pdftexcmds@temp{primitive}{luatexprimitive}{pdfprimitive}% oldLuaTeX
\pdftexcmds@temp{primitive}{luatexpdfprimitive}{pdfprimitive}% oldLuaTeX
%    \end{macrocode}
%    \end{macro}
%    \begin{macro}{\pdf@ifprimitive}
%    \begin{macrocode}
\pdftexcmds@temp{ifprimitive}{ifpdfprimitive}{ifpdfprimitive}% pdfTeX, oldLuaTeX
\pdftexcmds@temp{ifprimitive}{ifprimitive}{ifprimitive}% XeTeX, luatex
\pdftexcmds@temp{ifprimitive}{luatexifprimitive}{ifpdfprimitive}% oldLuaTeX
\pdftexcmds@temp{ifprimitive}{luatexifpdfprimitive}{ifpdfprimitive}% oldLuaTeX
%    \end{macrocode}
%    \end{macro}
%
%    Disable broken \cs{pdfprimitive}.
%    \begin{macrocode}
\ifluatex\else
\begingroup
  \expandafter\ifx\csname pdf@primitive\endcsname\relax
  \else
    \expandafter\ifx\csname pdftexversion\endcsname\relax
    \else
      \ifnum\pdftexversion=140 %
        \expandafter\ifx\csname pdftexrevision\endcsname\relax
        \else
          \ifnum\pdftexrevision<4 %
            \endgroup
            \let\pdf@primitive\@undefined
            \@PackageInfoNoLine{pdftexcmds}{%
              \string\pdf@primitive\space disabled, %
              because\MessageBreak
              \string\pdfprimitive\space is broken until pdfTeX 1.40.4%
            }%
            \begingroup
          \fi
        \fi
      \fi
    \fi
  \fi
\endgroup
\fi
%    \end{macrocode}
%
% \subsubsection{Result}
%
%    \begin{macrocode}
\begingroup
  \@PackageInfoNoLine{pdftexcmds}{%
    \string\pdf@primitive\space is %
    \expandafter\ifx\csname pdf@primitive\endcsname\relax not \fi
    available%
  }%
  \@PackageInfoNoLine{pdftexcmds}{%
    \string\pdf@ifprimitive\space is %
    \expandafter\ifx\csname pdf@ifprimitive\endcsname\relax not \fi
    available%
  }%
\endgroup
%    \end{macrocode}
%
% \subsection{\hologo{XeTeX}}
%
%    Look for primitives \cs{shellescape}, \cs{strcmp}.
%    \begin{macrocode}
\def\pdftexcmds@temp#1{%
  \begingroup\expandafter\expandafter\expandafter\endgroup
  \expandafter\ifx\csname pdf@#1\endcsname\relax
    \begingroup
      \escapechar=-1 %
      \edef\x{\expandafter\meaning\csname#1\endcsname}%
      \def\y{#1}%
      \def\z##1->{}%
      \edef\y{\expandafter\z\meaning\y}%
    \expandafter\endgroup
    \ifx\x\y
      \expandafter\def\csname pdf@#1\expandafter\endcsname
      \expandafter{%
        \csname#1\endcsname
      }%
    \fi
  \fi
}%
\pdftexcmds@temp{shellescape}%
\pdftexcmds@temp{strcmp}%
%    \end{macrocode}
%
% \subsection{\cs{pdf@isprimitive}}
%
%    \begin{macrocode}
\def\pdf@isprimitive{%
  \begingroup\expandafter\expandafter\expandafter\endgroup
  \expandafter\ifx\csname pdf@strcmp\endcsname\relax
    \long\def\pdf@isprimitive##1{%
      \expandafter\pdftexcmds@isprimitive\expandafter{\meaning##1}%
    }%
    \long\def\pdftexcmds@isprimitive##1##2{%
      \expandafter\pdftexcmds@@isprimitive\expandafter{\string##2}{##1}%
    }%
    \def\pdftexcmds@@isprimitive##1##2{%
      \ifnum0\pdftexcmds@equal##1\delimiter##2\delimiter=1 %
        \expandafter\ltx@firstoftwo
      \else
        \expandafter\ltx@secondoftwo
      \fi
    }%
    \def\pdftexcmds@equal##1##2\delimiter##3##4\delimiter{%
      \ifx##1##3%
        \ifx\relax##2##4\relax
          1%
        \else
          \ifx\relax##2\relax
          \else
            \ifx\relax##4\relax
            \else
              \pdftexcmds@equalcont{##2}{##4}%
            \fi
          \fi
        \fi
      \fi
    }%
    \def\pdftexcmds@equalcont##1{%
      \def\pdftexcmds@equalcont####1####2##1##1##1##1{%
        ##1##1##1##1%
        \pdftexcmds@equal####1\delimiter####2\delimiter
      }%
    }%
    \expandafter\pdftexcmds@equalcont\csname fi\endcsname
  \else
    \long\def\pdf@isprimitive##1##2{%
      \ifnum\pdf@strcmp{\meaning##1}{\string##2}=0 %
        \expandafter\ltx@firstoftwo
      \else
        \expandafter\ltx@secondoftwo
      \fi
    }%
  \fi
}
\ifluatex
\ifx\pdfdraftmode\@undefined
  \let\pdfdraftmode\draftmode
\fi
\else
  \pdf@isprimitive
\fi
%    \end{macrocode}
%
% \subsection{\cs{pdf@draftmode}}
%
%
%    \begin{macrocode}
\let\pdftexcmds@temp\ltx@zero %
\ltx@IfUndefined{pdfdraftmode}{%
  \@PackageInfoNoLine{pdftexcmds}{\ltx@backslashchar pdfdraftmode not found}%
}{%
  \ifpdf
    \let\pdftexcmds@temp\ltx@one
    \@PackageInfoNoLine{pdftexcmds}{\ltx@backslashchar pdfdraftmode found}%
  \else
    \@PackageInfoNoLine{pdftexcmds}{%
      \ltx@backslashchar pdfdraftmode is ignored in DVI mode%
    }%
  \fi
}
\ifcase\pdftexcmds@temp
%    \end{macrocode}
%    \begin{macro}{\pdf@draftmode}
%    \begin{macrocode}
  \let\pdf@draftmode\ltx@zero
%    \end{macrocode}
%    \end{macro}
%    \begin{macro}{\pdf@ifdraftmode}
%    \begin{macrocode}
  \let\pdf@ifdraftmode\ltx@secondoftwo
%    \end{macrocode}
%    \end{macro}
%    \begin{macro}{\pdftexcmds@setdraftmode}
%    \begin{macrocode}
  \def\pdftexcmds@setdraftmode#1{}%
%    \end{macrocode}
%    \end{macro}
%    \begin{macrocode}
\else
%    \end{macrocode}
%    \begin{macro}{\pdftexcmds@draftmode}
%    \begin{macrocode}
  \let\pdftexcmds@draftmode\pdfdraftmode
%    \end{macrocode}
%    \end{macro}
%    \begin{macro}{\pdf@ifdraftmode}
%    \begin{macrocode}
  \def\pdf@ifdraftmode{%
    \ifnum\pdftexcmds@draftmode=\ltx@one
      \expandafter\ltx@firstoftwo
    \else
      \expandafter\ltx@secondoftwo
    \fi
  }%
%    \end{macrocode}
%    \end{macro}
%    \begin{macro}{\pdf@draftmode}
%    \begin{macrocode}
  \def\pdf@draftmode{%
    \ifnum\pdftexcmds@draftmode=\ltx@one
      \expandafter\ltx@one
    \else
      \expandafter\ltx@zero
    \fi
  }%
%    \end{macrocode}
%    \end{macro}
%    \begin{macro}{\pdftexcmds@setdraftmode}
%    \begin{macrocode}
  \def\pdftexcmds@setdraftmode#1{%
    \pdftexcmds@draftmode=#1\relax
  }%
%    \end{macrocode}
%    \end{macro}
%    \begin{macrocode}
\fi
%    \end{macrocode}
%    \begin{macro}{\pdf@setdraftmode}
%    \begin{macrocode}
\def\pdf@setdraftmode#1{%
  \begingroup
    \count\ltx@cclv=#1\relax
  \edef\x{\endgroup
    \noexpand\pdftexcmds@@setdraftmode{\the\count\ltx@cclv}%
  }%
  \x
}
%    \end{macrocode}
%    \end{macro}
%    \begin{macro}{\pdftexcmds@@setdraftmode}
%    \begin{macrocode}
\def\pdftexcmds@@setdraftmode#1{%
  \ifcase#1 %
    \pdftexcmds@setdraftmode{#1}%
  \or
    \pdftexcmds@setdraftmode{#1}%
  \else
    \@PackageWarning{pdftexcmds}{%
      \string\pdf@setdraftmode: Ignoring\MessageBreak
      invalid value `#1'%
    }%
  \fi
}
%    \end{macrocode}
%    \end{macro}
%
% \subsection{Load Lua module}
%
%    \begin{macrocode}
\ifluatex
\else
  \expandafter\pdftexcmds@AtEnd
\fi%
%    \end{macrocode}
%
%    \begin{macrocode}
\ifnum\luatexversion<80
  \begingroup\expandafter\expandafter\expandafter\endgroup
  \expandafter\ifx\csname RequirePackage\endcsname\relax
    \def\TMP@RequirePackage#1[#2]{%
      \begingroup\expandafter\expandafter\expandafter\endgroup
      \expandafter\ifx\csname ver@#1.sty\endcsname\relax
        \input #1.sty\relax
      \fi
    }%
    \TMP@RequirePackage{luatex-loader}[2009/04/10]%
  \else
    \RequirePackage{luatex-loader}[2009/04/10]%
  \fi
\fi
\pdftexcmds@directlua{%
  require("pdftexcmds")%
}
\ifnum\luatexversion>37 %
  \ifnum0%
      \pdftexcmds@directlua{%
        if status.ini_version then %
          tex.write("1")%
        end%
      }>0 %
    \everyjob\expandafter{%
      \the\everyjob
      \pdftexcmds@directlua{%
        require("pdftexcmds")%
      }%
    }%
  \fi
\fi
\begingroup
  \def\x{2019/07/25 v0.30}%
  \ltx@onelevel@sanitize\x
  \edef\y{%
    \pdftexcmds@directlua{%
      if oberdiek.pdftexcmds.getversion then %
        oberdiek.pdftexcmds.getversion()%
      end%
    }%
  }%
  \ifx\x\y
  \else
    \@PackageError{pdftexcmds}{%
      Wrong version of lua module.\MessageBreak
      Package version: \x\MessageBreak
      Lua module: \y
    }\@ehc
  \fi
\endgroup
%    \end{macrocode}
%
% \subsection{Lua functions}
%
% \subsubsection{Helper macros}
%
%    \begin{macro}{\pdftexcmds@toks}
%    \begin{macrocode}
\begingroup\expandafter\expandafter\expandafter\endgroup
\expandafter\ifx\csname newtoks\endcsname\relax
  \toksdef\pdftexcmds@toks=0 %
\else
  \csname newtoks\endcsname\pdftexcmds@toks
\fi
%    \end{macrocode}
%    \end{macro}
%
%    \begin{macro}{\pdftexcmds@Patch}
%    \begin{macrocode}
\def\pdftexcmds@Patch{0}
\ifnum\luatexversion>40 %
  \ifnum\luatexversion<66 %
    \def\pdftexcmds@Patch{1}%
  \fi
\fi
%    \end{macrocode}
%    \end{macro}
%    \begin{macrocode}
\ifcase\pdftexcmds@Patch
  \catcode`\&=14 %
\else
  \catcode`\&=9 %
%    \end{macrocode}
%    \begin{macro}{\pdftexcmds@PatchDecode}
%    \begin{macrocode}
  \def\pdftexcmds@PatchDecode#1\@nil{%
    \pdftexcmds@DecodeA#1^^A^^A\@nil{}%
  }%
%    \end{macrocode}
%    \end{macro}
%    \begin{macro}{\pdftexcmds@DecodeA}
%    \begin{macrocode}
  \def\pdftexcmds@DecodeA#1^^A^^A#2\@nil#3{%
    \ifx\relax#2\relax
      \ltx@ReturnAfterElseFi{%
        \pdftexcmds@DecodeB#3#1^^A^^B\@nil{}%
      }%
    \else
      \ltx@ReturnAfterFi{%
        \pdftexcmds@DecodeA#2\@nil{#3#1^^@}%
      }%
    \fi
  }%
%    \end{macrocode}
%    \end{macro}
%    \begin{macro}{\pdftexcmds@DecodeB}
%    \begin{macrocode}
  \def\pdftexcmds@DecodeB#1^^A^^B#2\@nil#3{%
    \ifx\relax#2\relax%
      \ltx@ReturnAfterElseFi{%
        \ltx@zero
        #3#1%
      }%
    \else
      \ltx@ReturnAfterFi{%
        \pdftexcmds@DecodeB#2\@nil{#3#1^^A}%
      }%
    \fi
  }%
%    \end{macrocode}
%    \end{macro}
%    \begin{macrocode}
\fi
%    \end{macrocode}
%
%    \begin{macrocode}
\ifnum\luatexversion<36 %
\else
  \catcode`\0=9 %
\fi
%    \end{macrocode}
%
% \subsubsection[Strings]{Strings \cite[``7.15 Strings'']{pdftex-manual}}
%
%    \begin{macro}{\pdf@strcmp}
%    \begin{macrocode}
\long\def\pdf@strcmp#1#2{%
  \directlua0{%
    oberdiek.pdftexcmds.strcmp("\luaescapestring{#1}",%
        "\luaescapestring{#2}")%
  }%
}%
%    \end{macrocode}
%    \end{macro}
%    \begin{macrocode}
\pdf@isprimitive
%    \end{macrocode}
%    \begin{macro}{\pdf@escapehex}
%    \begin{macrocode}
\long\def\pdf@escapehex#1{%
  \directlua0{%
    oberdiek.pdftexcmds.escapehex("\luaescapestring{#1}", "byte")%
  }%
}%
%    \end{macrocode}
%    \end{macro}
%    \begin{macro}{\pdf@escapehexnative}
%    \begin{macrocode}
\long\def\pdf@escapehexnative#1{%
  \directlua0{%
    oberdiek.pdftexcmds.escapehex("\luaescapestring{#1}")%
  }%
}%
%    \end{macrocode}
%    \end{macro}
%    \begin{macro}{\pdf@unescapehex}
%    \begin{macrocode}
\def\pdf@unescapehex#1{%
& \romannumeral\expandafter\pdftexcmds@PatchDecode
  \the\expandafter\pdftexcmds@toks
  \directlua0{%
    oberdiek.pdftexcmds.toks="pdftexcmds@toks"%
    oberdiek.pdftexcmds.unescapehex("\luaescapestring{#1}", "byte", \pdftexcmds@Patch)%
  }%
& \@nil
}%
%    \end{macrocode}
%    \end{macro}
%    \begin{macro}{\pdf@unescapehexnative}
%    \begin{macrocode}
\def\pdf@unescapehexnative#1{%
& \romannumeral\expandafter\pdftexcmds@PatchDecode
  \the\expandafter\pdftexcmds@toks
  \directlua0{%
    oberdiek.pdftexcmds.toks="pdftexcmds@toks"%
    oberdiek.pdftexcmds.unescapehex("\luaescapestring{#1}", \pdftexcmds@Patch)%
  }%
& \@nil
}%
%    \end{macrocode}
%    \end{macro}
%    \begin{macro}{\pdf@escapestring}
%    \begin{macrocode}
\long\def\pdf@escapestring#1{%
  \directlua0{%
    oberdiek.pdftexcmds.escapestring("\luaescapestring{#1}", "byte")%
  }%
}
%    \end{macrocode}
%    \end{macro}
%    \begin{macro}{\pdf@escapename}
%    \begin{macrocode}
\long\def\pdf@escapename#1{%
  \directlua0{%
    oberdiek.pdftexcmds.escapename("\luaescapestring{#1}", "byte")%
  }%
}
%    \end{macrocode}
%    \end{macro}
%    \begin{macro}{\pdf@escapenamenative}
%    \begin{macrocode}
\long\def\pdf@escapenamenative#1{%
  \directlua0{%
    oberdiek.pdftexcmds.escapename("\luaescapestring{#1}")%
  }%
}
%    \end{macrocode}
%    \end{macro}
%
% \subsubsection[Files]{Files \cite[``7.18 Files'']{pdftex-manual}}
%
%    \begin{macro}{\pdf@filesize}
%    \begin{macrocode}
\def\pdf@filesize#1{%
  \directlua0{%
    oberdiek.pdftexcmds.filesize("\luaescapestring{#1}")%
  }%
}
%    \end{macrocode}
%    \end{macro}
%    \begin{macro}{\pdf@filemoddate}
%    \begin{macrocode}
\def\pdf@filemoddate#1{%
  \directlua0{%
    oberdiek.pdftexcmds.filemoddate("\luaescapestring{#1}")%
  }%
}
%    \end{macrocode}
%    \end{macro}
%    \begin{macro}{\pdf@filedump}
%    \begin{macrocode}
\def\pdf@filedump#1#2#3{%
  \directlua0{%
    oberdiek.pdftexcmds.filedump("\luaescapestring{\number#1}",%
        "\luaescapestring{\number#2}",%
        "\luaescapestring{#3}")%
  }%
}%
%    \end{macrocode}
%    \end{macro}
%    \begin{macro}{\pdf@mdfivesum}
%    \begin{macrocode}
\long\def\pdf@mdfivesum#1{%
  \directlua0{%
    oberdiek.pdftexcmds.mdfivesum("\luaescapestring{#1}", "byte")%
  }%
}%
%    \end{macrocode}
%    \end{macro}
%    \begin{macro}{\pdf@mdfivesumnative}
%    \begin{macrocode}
\long\def\pdf@mdfivesumnative#1{%
  \directlua0{%
    oberdiek.pdftexcmds.mdfivesum("\luaescapestring{#1}")%
  }%
}%
%    \end{macrocode}
%    \end{macro}
%    \begin{macro}{\pdf@filemdfivesum}
%    \begin{macrocode}
\def\pdf@filemdfivesum#1{%
  \directlua0{%
    oberdiek.pdftexcmds.filemdfivesum("\luaescapestring{#1}")%
  }%
}%
%    \end{macrocode}
%    \end{macro}
%
% \subsubsection[Timekeeping]{Timekeeping \cite[``7.17 Timekeeping'']{pdftex-manual}}
%
%    \begin{macro}{\protected}
%    \begin{macrocode}
\let\pdftexcmds@temp=Y%
\begingroup\expandafter\expandafter\expandafter\endgroup
\expandafter\ifx\csname protected\endcsname\relax
  \pdftexcmds@directlua0{%
    if tex.enableprimitives then %
      tex.enableprimitives('', {'protected'})%
    end%
  }%
\fi
\begingroup\expandafter\expandafter\expandafter\endgroup
\expandafter\ifx\csname protected\endcsname\relax
  \let\pdftexcmds@temp=N%
\fi
%    \end{macrocode}
%    \end{macro}
%    \begin{macro}{\numexpr}
%    \begin{macrocode}
\begingroup\expandafter\expandafter\expandafter\endgroup
\expandafter\ifx\csname numexpr\endcsname\relax
  \pdftexcmds@directlua0{%
    if tex.enableprimitives then %
      tex.enableprimitives('', {'numexpr'})%
    end%
  }%
\fi
\begingroup\expandafter\expandafter\expandafter\endgroup
\expandafter\ifx\csname numexpr\endcsname\relax
  \let\pdftexcmds@temp=N%
\fi
%    \end{macrocode}
%    \end{macro}
%
%    \begin{macrocode}
\ifx\pdftexcmds@temp N%
  \@PackageWarningNoLine{pdftexcmds}{%
    Definitions of \ltx@backslashchar pdf@resettimer and%
    \MessageBreak
    \ltx@backslashchar pdf@elapsedtime are skipped, because%
    \MessageBreak
    e-TeX's \ltx@backslashchar protected or %
    \ltx@backslashchar numexpr are missing%
  }%
\else
%    \end{macrocode}
%
%    \begin{macro}{\pdf@resettimer}
%    \begin{macrocode}
  \protected\def\pdf@resettimer{%
    \pdftexcmds@directlua0{%
      oberdiek.pdftexcmds.resettimer()%
    }%
  }%
%    \end{macrocode}
%    \end{macro}
%
%    \begin{macro}{\pdf@elapsedtime}
%    \begin{macrocode}
  \protected\def\pdf@elapsedtime{%
    \numexpr
      \pdftexcmds@directlua0{%
        oberdiek.pdftexcmds.elapsedtime()%
      }%
    \relax
  }%
%    \end{macrocode}
%    \end{macro}
%    \begin{macrocode}
\fi
%    \end{macrocode}
%
% \subsubsection{Shell escape}
%
%    \begin{macro}{\pdf@shellescape}
%
%    \begin{macrocode}
\ifnum\luatexversion<68 %
\else
  \protected\edef\pdf@shellescape{%
   \numexpr\directlua{tex.sprint(%
         \number\catcodetable@string,status.shell_escape)}\relax}
\fi
%    \end{macrocode}
%    \end{macro}
%
%    \begin{macro}{\pdf@system}
%    \begin{macrocode}
\def\pdf@system#1{%
  \directlua0{%
    oberdiek.pdftexcmds.system("\luaescapestring{#1}")%
  }%
}
%    \end{macrocode}
%    \end{macro}
%
%    \begin{macro}{\pdf@lastsystemstatus}
%    \begin{macrocode}
\def\pdf@lastsystemstatus{%
  \directlua0{%
    oberdiek.pdftexcmds.lastsystemstatus()%
  }%
}
%    \end{macrocode}
%    \end{macro}
%    \begin{macro}{\pdf@lastsystemexit}
%    \begin{macrocode}
\def\pdf@lastsystemexit{%
  \directlua0{%
    oberdiek.pdftexcmds.lastsystemexit()%
  }%
}
%    \end{macrocode}
%    \end{macro}
%
%    \begin{macrocode}
\catcode`\0=12 %
%    \end{macrocode}
%
%    \begin{macro}{\pdf@pipe}
%    Check availability of |io.popen| first.
%    \begin{macrocode}
\ifnum0%
    \pdftexcmds@directlua{%
      if io.popen then %
        tex.write("1")%
      end%
    }%
    =1 %
  \def\pdf@pipe#1{%
&   \romannumeral\expandafter\pdftexcmds@PatchDecode
    \the\expandafter\pdftexcmds@toks
    \pdftexcmds@directlua{%
      oberdiek.pdftexcmds.toks="pdftexcmds@toks"%
      oberdiek.pdftexcmds.pipe("\luaescapestring{#1}", \pdftexcmds@Patch)%
    }%
&   \@nil
  }%
\fi
%    \end{macrocode}
%    \end{macro}
%
%    \begin{macrocode}
\pdftexcmds@AtEnd%
%</package>
%    \end{macrocode}
%
% \subsection{Lua module}
%
%    \begin{macrocode}
%<*lua>
%    \end{macrocode}
%
%    \begin{macrocode}
oberdiek = oberdiek or {}
local pdftexcmds = oberdiek.pdftexcmds or {}
oberdiek.pdftexcmds = pdftexcmds
local systemexitstatus
function pdftexcmds.getversion()
  tex.write("2019/07/25 v0.30")
end
%    \end{macrocode}
%
% \subsubsection[Strings]{Strings \cite[``7.15 Strings'']{pdftex-manual}}
%
%    \begin{macrocode}
function pdftexcmds.strcmp(A, B)
  if A == B then
    tex.write("0")
  elseif A < B then
    tex.write("-1")
  else
    tex.write("1")
  end
end
local function utf8_to_byte(str)
  local i = 0
  local n = string.len(str)
  local t = {}
  while i < n do
    i = i + 1
    local a = string.byte(str, i)
    if a < 128 then
      table.insert(t, string.char(a))
    else
      if a >= 192 and i < n then
        i = i + 1
        local b = string.byte(str, i)
        if b < 128 or b >= 192 then
          i = i - 1
        elseif a == 194 then
          table.insert(t, string.char(b))
        elseif a == 195 then
          table.insert(t, string.char(b + 64))
        end
      end
    end
  end
  return table.concat(t)
end
function pdftexcmds.escapehex(str, mode)
  if mode == "byte" then
    str = utf8_to_byte(str)
  end
  tex.write((string.gsub(str, ".",
    function (ch)
      return string.format("%02X", string.byte(ch))
    end
  )))
end
%    \end{macrocode}
%    See procedure |unescapehex| in file \xfile{utils.c} of \hologo{pdfTeX}.
%    Caution: |tex.write| ignores leading spaces.
%    \begin{macrocode}
function pdftexcmds.unescapehex(str, mode, patch)
  local a = 0
  local first = true
  local result = {}
  for i = 1, string.len(str), 1 do
    local ch = string.byte(str, i)
    if ch >= 48 and ch <= 57 then
      ch = ch - 48
    elseif ch >= 65 and ch <= 70 then
      ch = ch - 55
    elseif ch >= 97 and ch <= 102 then
      ch = ch - 87
    else
      ch = nil
    end
    if ch then
      if first then
        a = ch * 16
        first = false
      else
        table.insert(result, a + ch)
        first = true
      end
    end
  end
  if not first then
    table.insert(result, a)
  end
  if patch == 1 then
    local temp = {}
    for i, a in ipairs(result) do
      if a == 0 then
        table.insert(temp, 1)
        table.insert(temp, 1)
      else
        if a == 1 then
          table.insert(temp, 1)
          table.insert(temp, 2)
        else
          table.insert(temp, a)
        end
      end
    end
    result = temp
  end
  if mode == "byte" then
    local utf8 = {}
    for i, a in ipairs(result) do
      if a < 128 then
        table.insert(utf8, a)
      else
        if a < 192 then
          table.insert(utf8, 194)
          a = a - 128
        else
          table.insert(utf8, 195)
          a = a - 192
        end
        table.insert(utf8, a + 128)
      end
    end
    result = utf8
  end
%    \end{macrocode}
%    this next line added for current luatex; this is the only
%    change in the file.  eroux, 28apr13. (v 0.21)
%    \begin{macrocode}
  local unpack = _G["unpack"] or table.unpack
  tex.settoks(pdftexcmds.toks, string.char(unpack(result)))
end
%    \end{macrocode}
%    See procedure |escapestring| in file \xfile{utils.c} of \hologo{pdfTeX}.
%    \begin{macrocode}
function pdftexcmds.escapestring(str, mode)
  if mode == "byte" then
    str = utf8_to_byte(str)
  end
  tex.write((string.gsub(str, ".",
    function (ch)
      local b = string.byte(ch)
      if b < 33 or b > 126 then
        return string.format("\\%.3o", b)
      end
      if b == 40 or b == 41 or b == 92 then
        return "\\" .. ch
      end
%    \end{macrocode}
%    Lua 5.1 returns the match in case of return value |nil|.
%    \begin{macrocode}
      return nil
    end
  )))
end
%    \end{macrocode}
%    See procedure |escapename| in file \xfile{utils.c} of \hologo{pdfTeX}.
%    \begin{macrocode}
function pdftexcmds.escapename(str, mode)
  if mode == "byte" then
    str = utf8_to_byte(str)
  end
  tex.write((string.gsub(str, ".",
    function (ch)
      local b = string.byte(ch)
      if b == 0 then
%    \end{macrocode}
%    In Lua 5.0 |nil| could be used for the empty string,
%    But |nil| returns the match in Lua 5.1, thus we use
%    the empty string explicitly.
%    \begin{macrocode}
        return ""
      end
      if b <= 32 or b >= 127
          or b == 35 or b == 37 or b == 40 or b == 41
          or b == 47 or b == 60 or b == 62 or b == 91
          or b == 93 or b == 123 or b == 125 then
        return string.format("#%.2X", b)
      else
%    \end{macrocode}
%    Lua 5.1 returns the match in case of return value |nil|.
%    \begin{macrocode}
        return nil
      end
    end
  )))
end
%    \end{macrocode}
%
% \subsubsection[Files]{Files \cite[``7.18 Files'']{pdftex-manual}}
%
%    \begin{macrocode}
function pdftexcmds.filesize(filename)
  local foundfile = kpse.find_file(filename, "tex", true)
  if foundfile then
    local size = lfs.attributes(foundfile, "size")
    if size then
      tex.write(size)
    end
  end
end
%    \end{macrocode}
%    See procedure |makepdftime| in file \xfile{utils.c} of \hologo{pdfTeX}.
%    \begin{macrocode}
function pdftexcmds.filemoddate(filename)
  local foundfile = kpse.find_file(filename, "tex", true)
  if foundfile then
    local date = lfs.attributes(foundfile, "modification")
    if date then
      local d = os.date("*t", date)
      if d.sec >= 60 then
        d.sec = 59
      end
      local u = os.date("!*t", date)
      local off = 60 * (d.hour - u.hour) + d.min - u.min
      if d.year ~= u.year then
        if d.year > u.year then
          off = off + 1440
        else
          off = off - 1440
        end
      elseif d.yday ~= u.yday then
        if d.yday > u.yday then
          off = off + 1440
        else
          off = off - 1440
        end
      end
      local timezone
      if off == 0 then
        timezone = "Z"
      else
        local hours = math.floor(off / 60)
        local mins = math.abs(off - hours * 60)
        timezone = string.format("%+03d'%02d'", hours, mins)
      end
      tex.write(string.format("D:%04d%02d%02d%02d%02d%02d%s",
          d.year, d.month, d.day, d.hour, d.min, d.sec, timezone))
    end
  end
end
function pdftexcmds.filedump(offset, length, filename)
  length = tonumber(length)
  if length and length > 0 then
    local foundfile = kpse.find_file(filename, "tex", true)
    if foundfile then
      offset = tonumber(offset)
      if not offset then
        offset = 0
      end
      local filehandle = io.open(foundfile, "rb")
      if filehandle then
        if offset > 0 then
          filehandle:seek("set", offset)
        end
        local dump = filehandle:read(length)
        pdftexcmds.escapehex(dump)
        filehandle:close()
      end
    end
  end
end
function pdftexcmds.mdfivesum(str, mode)
  if mode == "byte" then
    str = utf8_to_byte(str)
  end
  pdftexcmds.escapehex(md5.sum(str))
end
function pdftexcmds.filemdfivesum(filename)
  local foundfile = kpse.find_file(filename, "tex", true)
  if foundfile then
    local filehandle = io.open(foundfile, "rb")
    if filehandle then
      local contents = filehandle:read("*a")
      pdftexcmds.escapehex(md5.sum(contents))
      filehandle:close()
    end
  end
end
%    \end{macrocode}
%
% \subsubsection[Timekeeping]{Timekeeping \cite[``7.17 Timekeeping'']{pdftex-manual}}
%
%    The functions for timekeeping are based on
%    Andy Thomas' work \cite{AndyThomas:Analog}.
%    Changes:
%    \begin{itemize}
%    \item Overflow check is added.
%    \item |string.format| is used to avoid exponential number
%          representation for sure.
%    \item |tex.write| is used instead of |tex.print| to get
%          tokens with catcode 12 and without appended \cs{endlinechar}.
%    \end{itemize}
%    \begin{macrocode}
local basetime = 0
function pdftexcmds.resettimer()
  basetime = os.clock()
end
function pdftexcmds.elapsedtime()
  local val = (os.clock() - basetime) * 65536 + .5
  if val > 2147483647 then
    val = 2147483647
  end
  tex.write(string.format("%d", val))
end
%    \end{macrocode}
%
% \subsubsection[Miscellaneous]{Miscellaneous \cite[``7.21 Miscellaneous'']{pdftex-manual}}
%
%    \begin{macrocode}
function pdftexcmds.shellescape()
  if os.execute then
    if status
        and status.luatex_version
        and status.luatex_version >= 68 then
      tex.write(os.execute())
    else
      local result = os.execute()
      if result == 0 then
        tex.write("0")
      else
        if result == nil then
          tex.write("0")
        else
          tex.write("1")
        end
      end
    end
  else
    tex.write("0")
  end
end
function pdftexcmds.system(cmdline)
  systemexitstatus = nil
  texio.write_nl("log", "system(" .. cmdline .. ") ")
  if os.execute then
    texio.write("log", "executed.")
    systemexitstatus = os.execute(cmdline)
  else
    texio.write("log", "disabled.")
  end
end
function pdftexcmds.lastsystemstatus()
  local result = tonumber(systemexitstatus)
  if result then
    local x = math.floor(result / 256)
    tex.write(result - 256 * math.floor(result / 256))
  end
end
function pdftexcmds.lastsystemexit()
  local result = tonumber(systemexitstatus)
  if result then
    tex.write(math.floor(result / 256))
  end
end
function pdftexcmds.pipe(cmdline, patch)
  local result
  systemexitstatus = nil
  texio.write_nl("log", "pipe(" .. cmdline ..") ")
  if io.popen then
    texio.write("log", "executed.")
    local handle = io.popen(cmdline, "r")
    if handle then
      result = handle:read("*a")
      handle:close()
    end
  else
    texio.write("log", "disabled.")
  end
  if result then
    if patch == 1 then
      local temp = {}
      for i, a in ipairs(result) do
        if a == 0 then
          table.insert(temp, 1)
          table.insert(temp, 1)
        else
          if a == 1 then
            table.insert(temp, 1)
            table.insert(temp, 2)
          else
            table.insert(temp, a)
          end
        end
      end
      result = temp
    end
    tex.settoks(pdftexcmds.toks, result)
  else
    tex.settoks(pdftexcmds.toks, "")
  end
end
%    \end{macrocode}
%    \begin{macrocode}
%</lua>
%    \end{macrocode}
%
% \section{Test}
%
% \subsection{Catcode checks for loading}
%
%    \begin{macrocode}
%<*test1>
%    \end{macrocode}
%    \begin{macrocode}
\catcode`\{=1 %
\catcode`\}=2 %
\catcode`\#=6 %
\catcode`\@=11 %
\expandafter\ifx\csname count@\endcsname\relax
  \countdef\count@=255 %
\fi
\expandafter\ifx\csname @gobble\endcsname\relax
  \long\def\@gobble#1{}%
\fi
\expandafter\ifx\csname @firstofone\endcsname\relax
  \long\def\@firstofone#1{#1}%
\fi
\expandafter\ifx\csname loop\endcsname\relax
  \expandafter\@firstofone
\else
  \expandafter\@gobble
\fi
{%
  \def\loop#1\repeat{%
    \def\body{#1}%
    \iterate
  }%
  \def\iterate{%
    \body
      \let\next\iterate
    \else
      \let\next\relax
    \fi
    \next
  }%
  \let\repeat=\fi
}%
\def\RestoreCatcodes{}
\count@=0 %
\loop
  \edef\RestoreCatcodes{%
    \RestoreCatcodes
    \catcode\the\count@=\the\catcode\count@\relax
  }%
\ifnum\count@<255 %
  \advance\count@ 1 %
\repeat

\def\RangeCatcodeInvalid#1#2{%
  \count@=#1\relax
  \loop
    \catcode\count@=15 %
  \ifnum\count@<#2\relax
    \advance\count@ 1 %
  \repeat
}
\def\RangeCatcodeCheck#1#2#3{%
  \count@=#1\relax
  \loop
    \ifnum#3=\catcode\count@
    \else
      \errmessage{%
        Character \the\count@\space
        with wrong catcode \the\catcode\count@\space
        instead of \number#3%
      }%
    \fi
  \ifnum\count@<#2\relax
    \advance\count@ 1 %
  \repeat
}
\def\space{ }
\expandafter\ifx\csname LoadCommand\endcsname\relax
  \def\LoadCommand{\input pdftexcmds.sty\relax}%
\fi
\def\Test{%
  \RangeCatcodeInvalid{0}{47}%
  \RangeCatcodeInvalid{58}{64}%
  \RangeCatcodeInvalid{91}{96}%
  \RangeCatcodeInvalid{123}{255}%
  \catcode`\@=12 %
  \catcode`\\=0 %
  \catcode`\%=14 %
  \LoadCommand
  \RangeCatcodeCheck{0}{36}{15}%
  \RangeCatcodeCheck{37}{37}{14}%
  \RangeCatcodeCheck{38}{47}{15}%
  \RangeCatcodeCheck{48}{57}{12}%
  \RangeCatcodeCheck{58}{63}{15}%
  \RangeCatcodeCheck{64}{64}{12}%
  \RangeCatcodeCheck{65}{90}{11}%
  \RangeCatcodeCheck{91}{91}{15}%
  \RangeCatcodeCheck{92}{92}{0}%
  \RangeCatcodeCheck{93}{96}{15}%
  \RangeCatcodeCheck{97}{122}{11}%
  \RangeCatcodeCheck{123}{255}{15}%
  \RestoreCatcodes
}
\Test
\csname @@end\endcsname
\end
%    \end{macrocode}
%    \begin{macrocode}
%</test1>
%    \end{macrocode}
%
% \subsection{Test for \cs{pdf@isprimitive}}
%
%    \begin{macrocode}
%<*test2>
\catcode`\{=1 %
\catcode`\}=2 %
\catcode`\#=6 %
\catcode`\@=11 %
\input pdftexcmds.sty\relax
\def\msg#1{%
  \begingroup
    \escapechar=92 %
    \immediate\write16{#1}%
  \endgroup
}
\long\def\test#1#2#3#4{%
  \begingroup
    #4%
    \def\str{%
      Test \string\pdf@isprimitive
      {\string #1}{\string #2}{...}: %
    }%
    \pdf@isprimitive{#1}{#2}{%
      \ifx#3Y%
        \msg{\str true ==> OK.}%
      \else
        \errmessage{\str false ==> FAILED}%
      \fi
    }{%
      \ifx#3Y%
        \errmessage{\str true ==> FAILED}%
      \else
        \msg{\str false ==> OK.}%
      \fi
    }%
  \endgroup
}
\test\relax\relax Y{}
\test\foobar\relax Y{\let\foobar\relax}
\test\foobar\relax N{}
\test\hbox\hbox Y{}
\test\foobar@hbox\hbox Y{\let\foobar@hbox\hbox}
\test\if\if Y{}
\test\if\ifx N{}
\test\ifx\if N{}
\test\par\par Y{}
\test\hbox\par N{}
\test\par\hbox N{}
\csname @@end\endcsname\end
%</test2>
%    \end{macrocode}
%
% \subsection{Test for \cs{pdf@shellescape}}
%
%    \begin{macrocode}
%<*test-shell>
\catcode`\{=1 %
\catcode`\}=2 %
\catcode`\#=6 %
\catcode`\@=11 %
\input pdftexcmds.sty\relax
\def\msg#{\immediate\write16}
\def\MaybeEnd{}
\ifx\luatexversion\UnDeFiNeD
\else
  \ifnum\luatexversion<68 %
    \ifx\pdf@shellescape\@undefined
      \msg{SHELL=U}%
      \msg{OK (LuaTeX < 0.68)}%
    \else
      \msg{SHELL=defined}%
      \errmessage{Failed (LuaTeX < 0.68)}%
    \fi
    \def\MaybeEnd{\csname @@end\endcsname\end}%
  \fi
\fi
\MaybeEnd
\ifx\pdf@shellescape\@undefined
  \msg{SHELL=U}%
\else
  \msg{SHELL=\number\pdf@shellescape}%
\fi
\ifx\expected\@undefined
\else
  \ifx\expected\relax
    \msg{EXPECTED=U}%
    \ifx\pdf@shellescape\@undefined
      \msg{OK}%
    \else
      \errmessage{Failed}%
    \fi
  \else
    \msg{EXPECTED=\number\expected}%
    \ifnum\pdf@shellescape=\expected\relax
      \msg{OK}%
    \else
      \errmessage{Failed}%
    \fi
  \fi
\fi
\csname @@end\endcsname\end
%</test-shell>
%    \end{macrocode}
%
% \subsection{Test for escape functions}
%
%    \begin{macrocode}
%<*test-escape>
\catcode`\{=1 %
\catcode`\}=2 %
\catcode`\#=6 %
\catcode`\^=7 %
\catcode`\@=11 %
\errorcontextlines=1000 %
\input pdftexcmds.sty\relax
\def\msg#1{%
  \begingroup
    \escapechar=92 %
    \immediate\write16{#1}%
  \endgroup
}
%    \end{macrocode}
%    \begin{macrocode}
\begingroup
  \catcode`\@=11 %
  \countdef\count@=255 %
  \def\space{ }%
  \long\def\@whilenum#1\do #2{%
    \ifnum #1\relax
      #2\relax
      \@iwhilenum{#1\relax#2\relax}%
    \fi
  }%
  \long\def\@iwhilenum#1{%
    \ifnum #1%
      \expandafter\@iwhilenum
    \else
      \expandafter\ltx@gobble
    \fi
    {#1}%
  }%
  \gdef\AllBytes{}%
  \count@=0 %
  \catcode0=12 %
  \@whilenum\count@<256 \do{%
    \lccode0=\count@
    \ifnum\count@=32 %
      \xdef\AllBytes{\AllBytes\space}%
    \else
      \lowercase{%
        \xdef\AllBytes{\AllBytes^^@}%
      }%
    \fi
    \advance\count@ by 1 %
  }%
\endgroup
%    \end{macrocode}
%    \begin{macrocode}
\def\AllBytesHex{%
  000102030405060708090A0B0C0D0E0F%
  101112131415161718191A1B1C1D1E1F%
  202122232425262728292A2B2C2D2E2F%
  303132333435363738393A3B3C3D3E3F%
  404142434445464748494A4B4C4D4E4F%
  505152535455565758595A5B5C5D5E5F%
  606162636465666768696A6B6C6D6E6F%
  707172737475767778797A7B7C7D7E7F%
  808182838485868788898A8B8C8D8E8F%
  909192939495969798999A9B9C9D9E9F%
  A0A1A2A3A4A5A6A7A8A9AAABACADAEAF%
  B0B1B2B3B4B5B6B7B8B9BABBBCBDBEBF%
  C0C1C2C3C4C5C6C7C8C9CACBCCCDCECF%
  D0D1D2D3D4D5D6D7D8D9DADBDCDDDEDF%
  E0E1E2E3E4E5E6E7E8E9EAEBECEDEEEF%
  F0F1F2F3F4F5F6F7F8F9FAFBFCFDFEFF%
}
\ltx@onelevel@sanitize\AllBytesHex
\expandafter\lowercase\expandafter{%
  \expandafter\def\expandafter\AllBytesHexLC
      \expandafter{\AllBytesHex}%
}
\begingroup
  \catcode`\#=12 %
  \xdef\AllBytesName{%
    #01#02#03#04#05#06#07#08#09#0A#0B#0C#0D#0E#0F%
    #10#11#12#13#14#15#16#17#18#19#1A#1B#1C#1D#1E#1F%
    #20!"#23$#25&'#28#29*+,-.#2F%
    0123456789:;#3C=#3E?%
    @ABCDEFGHIJKLMNO%
    PQRSTUVWXYZ#5B\ltx@backslashchar#5D^_%
    `abcdefghijklmno%
    pqrstuvwxyz#7B|#7D\string~#7F%
    #80#81#82#83#84#85#86#87#88#89#8A#8B#8C#8D#8E#8F%
    #90#91#92#93#94#95#96#97#98#99#9A#9B#9C#9D#9E#9F%
    #A0#A1#A2#A3#A4#A5#A6#A7#A8#A9#AA#AB#AC#AD#AE#AF%
    #B0#B1#B2#B3#B4#B5#B6#B7#B8#B9#BA#BB#BC#BD#BE#BF%
    #C0#C1#C2#C3#C4#C5#C6#C7#C8#C9#CA#CB#CC#CD#CE#CF%
    #D0#D1#D2#D3#D4#D5#D6#D7#D8#D9#DA#DB#DC#DD#DE#DF%
    #E0#E1#E2#E3#E4#E5#E6#E7#E8#E9#EA#EB#EC#ED#EE#EF%
    #F0#F1#F2#F3#F4#F5#F6#F7#F8#F9#FA#FB#FC#FD#FE#FF%
  }%
\endgroup
\ltx@onelevel@sanitize\AllBytesName
\edef\AllBytesFromName{\expandafter\ltx@gobble\AllBytes}
\begingroup
  \def\|{|}%
  \edef\%{\ltx@percentchar}%
  \catcode`\|=0 %
  \catcode`\#=12 %
  \catcode`\~=12 %
  \catcode`\\=12 %
  |xdef|AllBytesString{%
    \000\001\002\003\004\005\006\007\010\011\012\013\014\015\016\017%
    \020\021\022\023\024\025\026\027\030\031\032\033\034\035\036\037%
    \040!"#$|%&'\(\)*+,-./%
    0123456789:;<=>?%
    @ABCDEFGHIJKLMNO%
    PQRSTUVWXYZ[\\]^_%
    `abcdefghijklmno%
    pqrstuvwxyz{||}~\177%
    \200\201\202\203\204\205\206\207\210\211\212\213\214\215\216\217%
    \220\221\222\223\224\225\226\227\230\231\232\233\234\235\236\237%
    \240\241\242\243\244\245\246\247\250\251\252\253\254\255\256\257%
    \260\261\262\263\264\265\266\267\270\271\272\273\274\275\276\277%
    \300\301\302\303\304\305\306\307\310\311\312\313\314\315\316\317%
    \320\321\322\323\324\325\326\327\330\331\332\333\334\335\336\337%
    \340\341\342\343\344\345\346\347\350\351\352\353\354\355\356\357%
    \360\361\362\363\364\365\366\367\370\371\372\373\374\375\376\377%
  }%
|endgroup
\ltx@onelevel@sanitize\AllBytesString
%    \end{macrocode}
%    \begin{macrocode}
\def\Test#1#2#3{%
  \begingroup
    \expandafter\expandafter\expandafter\def
    \expandafter\expandafter\expandafter\TestResult
    \expandafter\expandafter\expandafter{%
      #1{#2}%
    }%
    \ifx\TestResult#3%
    \else
      \newlinechar=10 %
      \msg{Expect:^^J#3}%
      \msg{Result:^^J\TestResult}%
      \errmessage{\string#2 -\string#1-> \string#3}%
    \fi
  \endgroup
}
\def\test#1#2#3{%
  \edef\TestFrom{#2}%
  \edef\TestExpect{#3}%
  \ltx@onelevel@sanitize\TestExpect
  \Test#1\TestFrom\TestExpect
}
\test\pdf@unescapehex{74657374}{test}
\begingroup
  \catcode0=12 %
  \catcode1=12 %
  \test\pdf@unescapehex{740074017400740174}{t^^@t^^At^^@t^^At}%
\endgroup
\Test\pdf@escapehex\AllBytes\AllBytesHex
\Test\pdf@unescapehex\AllBytesHex\AllBytes
\Test\pdf@escapename\AllBytes\AllBytesName
\Test\pdf@escapestring\AllBytes\AllBytesString
%    \end{macrocode}
%    \begin{macrocode}
\csname @@end\endcsname\end
%</test-escape>
%    \end{macrocode}
%
% \section{Installation}
%
% \subsection{Download}
%
% \paragraph{Package.} This package is available on
% CTAN\footnote{\CTANpkg{pdftexcmds}}:
% \begin{description}
% \item[\CTAN{macros/latex/contrib/oberdiek/pdftexcmds.dtx}] The source file.
% \item[\CTAN{macros/latex/contrib/oberdiek/pdftexcmds.pdf}] Documentation.
% \end{description}
%
%
% \paragraph{Bundle.} All the packages of the bundle `oberdiek'
% are also available in a TDS compliant ZIP archive. There
% the packages are already unpacked and the documentation files
% are generated. The files and directories obey the TDS standard.
% \begin{description}
% \item[\CTANinstall{install/macros/latex/contrib/oberdiek.tds.zip}]
% \end{description}
% \emph{TDS} refers to the standard ``A Directory Structure
% for \TeX\ Files'' (\CTAN{tds/tds.pdf}). Directories
% with \xfile{texmf} in their name are usually organized this way.
%
% \subsection{Bundle installation}
%
% \paragraph{Unpacking.} Unpack the \xfile{oberdiek.tds.zip} in the
% TDS tree (also known as \xfile{texmf} tree) of your choice.
% Example (linux):
% \begin{quote}
%   |unzip oberdiek.tds.zip -d ~/texmf|
% \end{quote}
%
% \paragraph{Script installation.}
% Check the directory \xfile{TDS:scripts/oberdiek/} for
% scripts that need further installation steps.
% Package \xpackage{attachfile2} comes with the Perl script
% \xfile{pdfatfi.pl} that should be installed in such a way
% that it can be called as \texttt{pdfatfi}.
% Example (linux):
% \begin{quote}
%   |chmod +x scripts/oberdiek/pdfatfi.pl|\\
%   |cp scripts/oberdiek/pdfatfi.pl /usr/local/bin/|
% \end{quote}
%
% \subsection{Package installation}
%
% \paragraph{Unpacking.} The \xfile{.dtx} file is a self-extracting
% \docstrip\ archive. The files are extracted by running the
% \xfile{.dtx} through \plainTeX:
% \begin{quote}
%   \verb|tex pdftexcmds.dtx|
% \end{quote}
%
% \paragraph{TDS.} Now the different files must be moved into
% the different directories in your installation TDS tree
% (also known as \xfile{texmf} tree):
% \begin{quote}
% \def\t{^^A
% \begin{tabular}{@{}>{\ttfamily}l@{ $\rightarrow$ }>{\ttfamily}l@{}}
%   pdftexcmds.sty & tex/generic/oberdiek/pdftexcmds.sty\\
%   oberdiek.pdftexcmds.lua & scripts/oberdiek/oberdiek.pdftexcmds.lua\\
%   pdftexcmds.lua & scripts/oberdiek/pdftexcmds.lua\\
%   pdftexcmds.pdf & doc/latex/oberdiek/pdftexcmds.pdf\\
%   test/pdftexcmds-test1.tex & doc/latex/oberdiek/test/pdftexcmds-test1.tex\\
%   test/pdftexcmds-test2.tex & doc/latex/oberdiek/test/pdftexcmds-test2.tex\\
%   test/pdftexcmds-test-shell.tex & doc/latex/oberdiek/test/pdftexcmds-test-shell.tex\\
%   test/pdftexcmds-test-escape.tex & doc/latex/oberdiek/test/pdftexcmds-test-escape.tex\\
%   pdftexcmds.dtx & source/latex/oberdiek/pdftexcmds.dtx\\
% \end{tabular}^^A
% }^^A
% \sbox0{\t}^^A
% \ifdim\wd0>\linewidth
%   \begingroup
%     \advance\linewidth by\leftmargin
%     \advance\linewidth by\rightmargin
%   \edef\x{\endgroup
%     \def\noexpand\lw{\the\linewidth}^^A
%   }\x
%   \def\lwbox{^^A
%     \leavevmode
%     \hbox to \linewidth{^^A
%       \kern-\leftmargin\relax
%       \hss
%       \usebox0
%       \hss
%       \kern-\rightmargin\relax
%     }^^A
%   }^^A
%   \ifdim\wd0>\lw
%     \sbox0{\small\t}^^A
%     \ifdim\wd0>\linewidth
%       \ifdim\wd0>\lw
%         \sbox0{\footnotesize\t}^^A
%         \ifdim\wd0>\linewidth
%           \ifdim\wd0>\lw
%             \sbox0{\scriptsize\t}^^A
%             \ifdim\wd0>\linewidth
%               \ifdim\wd0>\lw
%                 \sbox0{\tiny\t}^^A
%                 \ifdim\wd0>\linewidth
%                   \lwbox
%                 \else
%                   \usebox0
%                 \fi
%               \else
%                 \lwbox
%               \fi
%             \else
%               \usebox0
%             \fi
%           \else
%             \lwbox
%           \fi
%         \else
%           \usebox0
%         \fi
%       \else
%         \lwbox
%       \fi
%     \else
%       \usebox0
%     \fi
%   \else
%     \lwbox
%   \fi
% \else
%   \usebox0
% \fi
% \end{quote}
% If you have a \xfile{docstrip.cfg} that configures and enables \docstrip's
% TDS installing feature, then some files can already be in the right
% place, see the documentation of \docstrip.
%
% \subsection{Refresh file name databases}
%
% If your \TeX~distribution
% (\teTeX, \mikTeX, \dots) relies on file name databases, you must refresh
% these. For example, \teTeX\ users run \verb|texhash| or
% \verb|mktexlsr|.
%
% \subsection{Some details for the interested}
%
% \paragraph{Unpacking with \LaTeX.}
% The \xfile{.dtx} chooses its action depending on the format:
% \begin{description}
% \item[\plainTeX:] Run \docstrip\ and extract the files.
% \item[\LaTeX:] Generate the documentation.
% \end{description}
% If you insist on using \LaTeX\ for \docstrip\ (really,
% \docstrip\ does not need \LaTeX), then inform the autodetect routine
% about your intention:
% \begin{quote}
%   \verb|latex \let\install=y% \iffalse meta-comment
%
% File: pdftexcmds.dtx
% Version: 2019/07/25 v0.30
% Info: Utility functions of pdfTeX for LuaTeX
%
% Copyright (C) 2007, 2009-2011 by
%    Heiko Oberdiek <heiko.oberdiek at googlemail.com>
%
% This work may be distributed and/or modified under the
% conditions of the LaTeX Project Public License, either
% version 1.3c of this license or (at your option) any later
% version. This version of this license is in
%    https://www.latex-project.org/lppl/lppl-1-3c.txt
% and the latest version of this license is in
%    https://www.latex-project.org/lppl.txt
% and version 1.3 or later is part of all distributions of
% LaTeX version 2005/12/01 or later.
%
% This work has the LPPL maintenance status "maintained".
%
% The Current Maintainers of this work are
% Heiko Oberdiek and the Oberdiek Package Support Group
% https://github.com/ho-tex/oberdiek/issues
%
% The Base Interpreter refers to any `TeX-Format',
% because some files are installed in TDS:tex/generic//.
%
% This work consists of the main source file pdftexcmds.dtx
% and the derived files
%    pdftexcmds.sty, pdftexcmds.pdf, pdftexcmds.ins, pdftexcmds.drv,
%    pdftexcmds.bib, pdftexcmds-test1.tex, pdftexcmds-test2.tex,
%    pdftexcmds-test-shell.tex, pdftexcmds-test-escape.tex,
%    oberdiek.pdftexcmds.lua, pdftexcmds.lua.
%
% Distribution:
%    CTAN:macros/latex/contrib/oberdiek/pdftexcmds.dtx
%    CTAN:macros/latex/contrib/oberdiek/pdftexcmds.pdf
%
% Unpacking:
%    (a) If pdftexcmds.ins is present:
%           tex pdftexcmds.ins
%    (b) Without pdftexcmds.ins:
%           tex pdftexcmds.dtx
%    (c) If you insist on using LaTeX
%           latex \let\install=y% \iffalse meta-comment
%
% File: pdftexcmds.dtx
% Version: 2019/07/25 v0.30
% Info: Utility functions of pdfTeX for LuaTeX
%
% Copyright (C) 2007, 2009-2011 by
%    Heiko Oberdiek <heiko.oberdiek at googlemail.com>
%
% This work may be distributed and/or modified under the
% conditions of the LaTeX Project Public License, either
% version 1.3c of this license or (at your option) any later
% version. This version of this license is in
%    https://www.latex-project.org/lppl/lppl-1-3c.txt
% and the latest version of this license is in
%    https://www.latex-project.org/lppl.txt
% and version 1.3 or later is part of all distributions of
% LaTeX version 2005/12/01 or later.
%
% This work has the LPPL maintenance status "maintained".
%
% The Current Maintainers of this work are
% Heiko Oberdiek and the Oberdiek Package Support Group
% https://github.com/ho-tex/oberdiek/issues
%
% The Base Interpreter refers to any `TeX-Format',
% because some files are installed in TDS:tex/generic//.
%
% This work consists of the main source file pdftexcmds.dtx
% and the derived files
%    pdftexcmds.sty, pdftexcmds.pdf, pdftexcmds.ins, pdftexcmds.drv,
%    pdftexcmds.bib, pdftexcmds-test1.tex, pdftexcmds-test2.tex,
%    pdftexcmds-test-shell.tex, pdftexcmds-test-escape.tex,
%    oberdiek.pdftexcmds.lua, pdftexcmds.lua.
%
% Distribution:
%    CTAN:macros/latex/contrib/oberdiek/pdftexcmds.dtx
%    CTAN:macros/latex/contrib/oberdiek/pdftexcmds.pdf
%
% Unpacking:
%    (a) If pdftexcmds.ins is present:
%           tex pdftexcmds.ins
%    (b) Without pdftexcmds.ins:
%           tex pdftexcmds.dtx
%    (c) If you insist on using LaTeX
%           latex \let\install=y\input{pdftexcmds.dtx}
%        (quote the arguments according to the demands of your shell)
%
% Documentation:
%    (a) If pdftexcmds.drv is present:
%           latex pdftexcmds.drv
%    (b) Without pdftexcmds.drv:
%           latex pdftexcmds.dtx; ...
%    The class ltxdoc loads the configuration file ltxdoc.cfg
%    if available. Here you can specify further options, e.g.
%    use A4 as paper format:
%       \PassOptionsToClass{a4paper}{article}
%
%    Programm calls to get the documentation (example):
%       pdflatex pdftexcmds.dtx
%       bibtex pdftexcmds.aux
%       makeindex -s gind.ist pdftexcmds.idx
%       pdflatex pdftexcmds.dtx
%       makeindex -s gind.ist pdftexcmds.idx
%       pdflatex pdftexcmds.dtx
%
% Installation:
%    TDS:tex/generic/oberdiek/pdftexcmds.sty
%    TDS:scripts/oberdiek/oberdiek.pdftexcmds.lua
%    TDS:scripts/oberdiek/pdftexcmds.lua
%    TDS:doc/latex/oberdiek/pdftexcmds.pdf
%    TDS:doc/latex/oberdiek/test/pdftexcmds-test1.tex
%    TDS:doc/latex/oberdiek/test/pdftexcmds-test2.tex
%    TDS:doc/latex/oberdiek/test/pdftexcmds-test-shell.tex
%    TDS:doc/latex/oberdiek/test/pdftexcmds-test-escape.tex
%    TDS:source/latex/oberdiek/pdftexcmds.dtx
%
%<*ignore>
\begingroup
  \catcode123=1 %
  \catcode125=2 %
  \def\x{LaTeX2e}%
\expandafter\endgroup
\ifcase 0\ifx\install y1\fi\expandafter
         \ifx\csname processbatchFile\endcsname\relax\else1\fi
         \ifx\fmtname\x\else 1\fi\relax
\else\csname fi\endcsname
%</ignore>
%<*install>
\input docstrip.tex
\Msg{************************************************************************}
\Msg{* Installation}
\Msg{* Package: pdftexcmds 2019/07/25 v0.30 Utility functions of pdfTeX for LuaTeX (HO)}
\Msg{************************************************************************}

\keepsilent
\askforoverwritefalse

\let\MetaPrefix\relax
\preamble

This is a generated file.

Project: pdftexcmds
Version: 2019/07/25 v0.30

Copyright (C) 2007, 2009-2011 by
   Heiko Oberdiek <heiko.oberdiek at googlemail.com>

This work may be distributed and/or modified under the
conditions of the LaTeX Project Public License, either
version 1.3c of this license or (at your option) any later
version. This version of this license is in
   https://www.latex-project.org/lppl/lppl-1-3c.txt
and the latest version of this license is in
   https://www.latex-project.org/lppl.txt
and version 1.3 or later is part of all distributions of
LaTeX version 2005/12/01 or later.

This work has the LPPL maintenance status "maintained".

The Current Maintainers of this work are
Heiko Oberdiek and the Oberdiek Package Support Group
https://github.com/ho-tex/oberdiek/issues


The Base Interpreter refers to any `TeX-Format',
because some files are installed in TDS:tex/generic//.

This work consists of the main source file pdftexcmds.dtx
and the derived files
   pdftexcmds.sty, pdftexcmds.pdf, pdftexcmds.ins, pdftexcmds.drv,
   pdftexcmds.bib, pdftexcmds-test1.tex, pdftexcmds-test2.tex,
   pdftexcmds-test-shell.tex, pdftexcmds-test-escape.tex,
   oberdiek.pdftexcmds.lua, pdftexcmds.lua.

\endpreamble
\let\MetaPrefix\DoubleperCent

\generate{%
  \file{pdftexcmds.ins}{\from{pdftexcmds.dtx}{install}}%
  \file{pdftexcmds.drv}{\from{pdftexcmds.dtx}{driver}}%
  \nopreamble
  \nopostamble
  \file{pdftexcmds.bib}{\from{pdftexcmds.dtx}{bib}}%
  \usepreamble\defaultpreamble
  \usepostamble\defaultpostamble
  \usedir{tex/generic/oberdiek}%
  \file{pdftexcmds.sty}{\from{pdftexcmds.dtx}{package}}%
%  \usedir{doc/latex/oberdiek/test}%
%  \file{pdftexcmds-test1.tex}{\from{pdftexcmds.dtx}{test1}}%
%  \file{pdftexcmds-test2.tex}{\from{pdftexcmds.dtx}{test2}}%
%  \file{pdftexcmds-test-shell.tex}{\from{pdftexcmds.dtx}{test-shell}}%
%  \file{pdftexcmds-test-escape.tex}{\from{pdftexcmds.dtx}{test-escape}}%
  \nopreamble
  \nopostamble
%  \usedir{source/latex/oberdiek/catalogue}%
%  \file{pdftexcmds.xml}{\from{pdftexcmds.dtx}{catalogue}}%
}
\def\MetaPrefix{-- }
\def\defaultpostamble{%
  \MetaPrefix^^J%
  \MetaPrefix\space End of File `\outFileName'.%
}
\def\currentpostamble{\defaultpostamble}%
\generate{%
  \usedir{scripts/oberdiek}%
  \file{oberdiek.pdftexcmds.lua}{\from{pdftexcmds.dtx}{lua}}%
  \file{pdftexcmds.lua}{\from{pdftexcmds.dtx}{lua}}%
}

\catcode32=13\relax% active space
\let =\space%
\Msg{************************************************************************}
\Msg{*}
\Msg{* To finish the installation you have to move the following}
\Msg{* file into a directory searched by TeX:}
\Msg{*}
\Msg{*     pdftexcmds.sty}
\Msg{*}
\Msg{* And install the following script files:}
\Msg{*}
\Msg{*     oberdiek.pdftexcmds.lua, pdftexcmds.lua}
\Msg{*}
\Msg{* To produce the documentation run the file `pdftexcmds.drv'}
\Msg{* through LaTeX.}
\Msg{*}
\Msg{* Happy TeXing!}
\Msg{*}
\Msg{************************************************************************}

\endbatchfile
%</install>
%<*bib>
@online{AndyThomas:Analog,
  author={Thomas, Andy},
  title={Analog of {\texttt{\csname textbackslash\endcsname}pdfelapsedtime} for
      {\hologo{LuaTeX}} and {\hologo{XeTeX}}},
  url={http://tex.stackexchange.com/a/32531},
  urldate={2011-11-29},
}
%</bib>
%<*ignore>
\fi
%</ignore>
%<*driver>
\NeedsTeXFormat{LaTeX2e}
\ProvidesFile{pdftexcmds.drv}%
  [2019/07/25 v0.30 Utility functions of pdfTeX for LuaTeX (HO)]%
\documentclass{ltxdoc}
\usepackage{holtxdoc}[2011/11/22]
\usepackage{paralist}
\usepackage{csquotes}
\usepackage[
  backend=bibtex,
  bibencoding=ascii,
  alldates=iso8601,
]{biblatex}[2011/11/13]
\bibliography{oberdiek-source}
\bibliography{pdftexcmds}
\begin{document}
  \DocInput{pdftexcmds.dtx}%
\end{document}
%</driver>
% \fi
%
%
% \CharacterTable
%  {Upper-case    \A\B\C\D\E\F\G\H\I\J\K\L\M\N\O\P\Q\R\S\T\U\V\W\X\Y\Z
%   Lower-case    \a\b\c\d\e\f\g\h\i\j\k\l\m\n\o\p\q\r\s\t\u\v\w\x\y\z
%   Digits        \0\1\2\3\4\5\6\7\8\9
%   Exclamation   \!     Double quote  \"     Hash (number) \#
%   Dollar        \$     Percent       \%     Ampersand     \&
%   Acute accent  \'     Left paren    \(     Right paren   \)
%   Asterisk      \*     Plus          \+     Comma         \,
%   Minus         \-     Point         \.     Solidus       \/
%   Colon         \:     Semicolon     \;     Less than     \<
%   Equals        \=     Greater than  \>     Question mark \?
%   Commercial at \@     Left bracket  \[     Backslash     \\
%   Right bracket \]     Circumflex    \^     Underscore    \_
%   Grave accent  \`     Left brace    \{     Vertical bar  \|
%   Right brace   \}     Tilde         \~}
%
% \GetFileInfo{pdftexcmds.drv}
%
% \title{The \xpackage{pdftexcmds} package}
% \date{2019/07/25 v0.30}
% \author{Heiko Oberdiek\thanks
% {Please report any issues at \url{https://github.com/ho-tex/oberdiek/issues}}}
%
% \maketitle
%
% \begin{abstract}
% \hologo{LuaTeX} provides most of the commands of \hologo{pdfTeX} 1.40. However
% a number of utility functions are removed. This package tries to fill
% the gap and implements some of the missing primitive using Lua.
% \end{abstract}
%
% \tableofcontents
%
% \def\csi#1{\texttt{\textbackslash\textit{#1}}}
%
% \section{Documentation}
%
% Some primitives of \hologo{pdfTeX} \cite{pdftex-manual}
% are not defined by \hologo{LuaTeX} \cite{luatex-manual}.
% This package implements macro based solutions using Lua code
% for the following missing \hologo{pdfTeX} primitives;
% \begin{compactitem}
% \item \cs{pdfstrcmp}
% \item \cs{pdfunescapehex}
% \item \cs{pdfescapehex}
% \item \cs{pdfescapename}
% \item \cs{pdfescapestring}
% \item \cs{pdffilesize}
% \item \cs{pdffilemoddate}
% \item \cs{pdffiledump}
% \item \cs{pdfmdfivesum}
% \item \cs{pdfresettimer}
% \item \cs{pdfelapsedtime}
% \item |\immediate\write18|
% \end{compactitem}
% The original names of the primitives cannot be used:
% \begin{itemize}
% \item
% The syntax for their arguments cannot easily
% simulated by macros. The primitives using key words
% such as |file| (\cs{pdfmdfivesum}) or |offset| and |length|
% (\cs{pdffiledump}) and uses \meta{general text} for the other
% arguments. Using token registers assignments, \meta{general text} could
% be catched. However, the simulated primitives are expandable
% and register assignments would destroy this important property.
% (\meta{general text} allows something like |\expandafter\bgroup ...}|.)
% \item
% The original primitives can be expanded using one expansion step.
% The new macros need two expansion steps because of the additional
% macro expansion. Example:
% \begin{quote}
%   |\expandafter\foo\pdffilemoddate{file}|\\
%   vs.\\
%   |\expandafter\expandafter\expandafter|\\
%   |\foo\pdf@filemoddate{file}|
% \end{quote}
% \end{itemize}
%
% \hologo{LuaTeX} isn't stable yet and thus the status of this package is
% \emph{experimental}. Feedback is welcome.
%
% \subsection{General principles}
%
% \begin{description}
% \item[Naming convention:]
%   Usually this package defines a macro |\pdf@|\meta{cmd} if
%   \hologo{pdfTeX} provides |\pdf|\meta{cmd}.
% \item[Arguments:] The order of arguments in |\pdf@|\meta{cmd}
%   is the same as for the corresponding primitive of \hologo{pdfTeX}.
%   The arguments are ordinary undelimited \hologo{TeX} arguments,
%   no \meta{general text} and without additional keywords.
% \item[Expandibility:]
%   The macro |\pdf@|\meta{cmd} is expandable if the
%   corresponding \hologo{pdfTeX} primitive has this property.
%   Exact two expansion steps are necessary (first is the macro
%   expansion) except for \cs{pdf@primitive} and \cs{pdf@ifprimitive}.
%   The latter ones are not macros, but have the direct meaning of the
%   primitive.
% \item[Without \hologo{LuaTeX}:]
%   The macros |\pdf@|\meta{cmd} are mapped to the commands
%   of \hologo{pdfTeX} if they are available. Otherwise they are undefined.
% \item[Availability:]
%   The macros that the packages provides are undefined, if
%   the necessary primitives are not found and cannot be
%   implemented by Lua.
% \end{description}
%
% \subsection{Macros}
%
% \subsubsection[Strings]{Strings \cite[``7.15 Strings'']{pdftex-manual}}
%
% \begin{declcs}{pdf@strcmp} \M{stringA} \M{stringB}
% \end{declcs}
% Same as |\pdfstrcmp{|\meta{stringA}|}{|\meta{stringB}|}|.
%
% \begin{declcs}{pdf@unescapehex} \M{string}
% \end{declcs}
% Same as |\pdfunescapehex{|\meta{string}|}|.
% The argument is a byte string given in hexadecimal notation.
% The result are character tokens from 0 until 255 with
% catcode 12 and the space with catcode 10.
%
% \begin{declcs}{pdf@escapehex} \M{string}\\
%   \cs{pdf@escapestring} \M{string}\\
%   \cs{pdf@escapename} \M{string}
% \end{declcs}
% Same as the primitives of \hologo{pdfTeX}. However \hologo{pdfTeX} does not
% know about characters with codes 256 and larger. Thus the
% string is treated as byte string, characters with more than
% eight bits are ignored.
%
% \subsubsection[Files]{Files \cite[``7.18 Files'']{pdftex-manual}}
%
% \begin{declcs}{pdf@filesize} \M{filename}
% \end{declcs}
% Same as |\pdffilesize{|\meta{filename}|}|.
%
% \begin{declcs}{pdf@filemoddate} \M{filename}
% \end{declcs}
% Same as |\pdffilemoddate{|\meta{filename}|}|.
%
% \begin{declcs}{pdf@filedump} \M{offset} \M{length} \M{filename}
% \end{declcs}
% Same as |\pdffiledump offset| \meta{offset} |length| \meta{length}
% |{|\meta{filename}|}|. Both \meta{offset} and \meta{length} must
% not be empty, but must be a valid \hologo{TeX} number.
%
% \begin{declcs}{pdf@mdfivesum} \M{string}
% \end{declcs}
% Same as |\pdfmdfivesum{|\meta{string}|}|. Keyword |file| is supported
% by macro \cs{pdf@filemdfivesum}.
%
% \begin{declcs}{pdf@filemdfivesum} \M{filename}
% \end{declcs}
% Same as |\pdfmdfivesum file{|\meta{filename}|}|.
%
% \subsubsection[Timekeeping]{Timekeeping \cite[``7.17 Timekeeping'']{pdftex-manual}}
%
% The timekeeping macros are based on Andy Thomas' work \cite{AndyThomas:Analog}.
%
% \begin{declcs}{pdf@resettimer}
% \end{declcs}
% Same as \cs{pdfresettimer}, it resets the internal timer.
%
% \begin{declcs}{pdf@elapsedtime}
% \end{declcs}
% Same as \cs{pdfelapsedtime}. It behaves like a read-only integer.
% For printing purposes it can be prefixed by \cs{the} or \cs{number}.
% It measures the time in scaled seconds (seconds multiplied with 65536)
% since the latest call of \cs{pdf@resettimer} or start of
% program/package. The resolution, the shortest time interval that
% can be measured, depends on the program and system.
% \begin{itemize}
% \item \hologo{pdfTeX} with |gettimeofday|: $\ge$ 1/65536\,s
% \item \hologo{pdfTeX} with |ftime|: $\ge$ 1\,ms
% \item \hologo{pdfTeX} with |time|: $\ge$ 1\,s
% \item \hologo{LuaTeX}: $\ge$ 10\,ms\\
%  (|os.clock()| returns a float number with two decimal digits in
%  \hologo{LuaTeX} beta-0.70.1-2011061416 (rev 4277)).
% \end{itemize}
%
% \subsubsection[Miscellaneous]{Miscellaneous \cite[``7.21 Miscellaneous'']{pdftex-manual}}
%
% \begin{declcs}{pdf@draftmode}
% \end{declcs}
% If the \TeX\ compiler knows \cs{pdfdraftmode} or \cs{draftmode}
% (\hologo{pdfTeX},
% \hologo{LuaTeX}), then \cs{pdf@draftmode} returns, whether
% this mode is enabled. The result is an implicit number:
% one means the draft mode is available and enabled.
% If the value is zero, then the mode is not active or
% \cs{pdfdraftmode} is not available.
% An explicit number is yielded by \cs{number}\cs{pdf@draftmode}.
% The macro cannot
% be used to change the mode, see \cs{pdf@setdraftmode}.
%
% \begin{declcs}{pdf@ifdraftmode} \M{true} \M{false}
% \end{declcs}
% If \cs{pdfdraftmode} is available and enabled, \meta{true} is
% called, otherwise \meta{false} is executed.
%
% \begin{declcs}{pdf@setdraftmode} \M{value}
% \end{declcs}
% Macro \cs{pdf@setdraftmode} expects the number zero or one as
% \meta{value}. Zero deactivates the mode and one enables the draft mode.
% The macro does not have an effect, if the feature \cs{pdfdraftmode} is not
% available.
%
% \begin{declcs}{pdf@shellescape}
% \end{declcs}
% Same as |\pdfshellescape|. It is or expands to |1| if external
% commands can be executed and |0| otherwise. In \hologo{pdfTeX} external
% commands must be enabled first by command line option or
% configuration option. In \hologo{LuaTeX} option |--safer| disables
% the execution of external commands.
%
% In \hologo{LuaTeX} before 0.68.0 \cs{pdf@shellescape} is not
% available due to a bug in |os.execute()|. The argumentless form
% crashes in some circumstances with segmentation fault.
% (It is fixed in version 0.68.0 or revision 4167 of \hologo{LuaTeX}.
% and packported to some version of 0.67.0).
%
% Hints for usage:
% \begin{itemize}
% \item Before its use \cs{pdf@shellescape} should be tested,
% whether it is available. Example with package \xpackage{ltxcmds}
% (loaded by package \xpackage{pdftexcmds}):
%\begin{quote}
%\begin{verbatim}
%\ltx@IfUndefined{pdf@shellescape}{%
%  % \pdf@shellescape is undefined
%}{%
%  % \pdf@shellescape is available
%}
%\end{verbatim}
%\end{quote}
% Use \cs{ltx@ifundefined} in expandable contexts.
% \item \cs{pdf@shellescape} might be a numerical constant,
% expands to the primitive, or expands to a plain number.
% Therefore use it in contexts where these differences does not matter.
% \item Use in comparisons, e.g.:
%   \begin{quote}
%     |\ifnum\pdf@shellescape=0 ...|
%   \end{quote}
% \item Print the number: |\number\pdf@shellescape|
% \end{itemize}
%
% \begin{declcs}{pdf@system} \M{cmdline}
% \end{declcs}
% It is a wrapper for |\immediate\write18| in \hologo{pdfTeX} or
% |os.execute| in \hologo{LuaTeX}.
%
% In theory |os.execute|
% returns a status number. But its meaning is quite
% undefined. Are there some reliable properties?
% Does it make sense to provide an user interface to
% this status exit code?
%
% \begin{declcs}{pdf@primitive} \csi{cmd}
% \end{declcs}
% Same as \cs{pdfprimitive} in \hologo{pdfTeX} or \hologo{LuaTeX}.
% In \hologo{XeTeX} the
% primitive is called \cs{primitive}. Despite the current definition
% of the command \csi{cmd}, it's meaning as primitive is used.
%
% \begin{declcs}{pdf@ifprimitive} \csi{cmd}
% \end{declcs}
% Same as \cs{ifpdfprimitive} in \hologo{pdfTeX} or
% \hologo{LuaTeX}. \hologo{XeTeX} calls
% it \cs{ifprimitive}. It is a switch that checks if the command
% \csi{cmd} has it's primitive meaning.
%
% \subsubsection{Additional macro: \cs{pdf@isprimitive}}
%
% \begin{declcs}{pdf@isprimitive} \csi{cmd1} \csi{cmd2} \M{true} \M{false}
% \end{declcs}
% If \csi{cmd1} has the primitive meaning given by the primitive name
% of \csi{cmd2}, then the argument \meta{true} is executed, otherwise
% \meta{false}. The macro \cs{pdf@isprimitive} is expandable.
% Internally it checks the result of \cs{meaning} and is therefore
% available for all \hologo{TeX} variants, even the original \hologo{TeX}.
% Example with \hologo{LaTeX}:
%\begin{quote}
%\begin{verbatim}
%\makeatletter
%\pdf@isprimitive{@@input}{input}{%
%  \typeout{\string\@@input\space is original\string\input}%
%}{%
%  \typeout{Oops, \string\@@input\space is not the %
%           original\string\input}%
%}
%\end{verbatim}
%\end{quote}
%
% \subsubsection{Experimental}
%
% \begin{declcs}{pdf@unescapehexnative} \M{string}\\
%   \cs{pdf@escapehexnative} \M{string}\\
%   \cs{pdf@escapenamenative} \M{string}\\
%   \cs{pdf@mdfivesumnative} \M{string}
% \end{declcs}
% The variants without |native| in the macro name are supposed to
% be compatible with \hologo{pdfTeX}. However characters with more than
% eight bits are not supported and are ignored. If \hologo{LuaTeX} is
% running, then its UTF-8 coded strings are used. Thus the full
% unicode character range is supported. However the result
% differs from \hologo{pdfTeX} for characters with eight or more bits.
%
% \begin{declcs}{pdf@pipe} \M{cmdline}
% \end{declcs}
% It calls \meta{cmdline} and returns the output of the external
% program in the usual manner as byte string (catcode 12, space with
% catcode 10). The Lua documentation says, that the used |io.popen|
% may not be available on all platforms. Then macro \cs{pdf@pipe}
% is undefined.
%
% \StopEventually{
% }
%
% \section{Implementation}
%
%    \begin{macrocode}
%<*package>
%    \end{macrocode}
%
% \subsection{Reload check and package identification}
%    Reload check, especially if the package is not used with \LaTeX.
%    \begin{macrocode}
\begingroup\catcode61\catcode48\catcode32=10\relax%
  \catcode13=5 % ^^M
  \endlinechar=13 %
  \catcode35=6 % #
  \catcode39=12 % '
  \catcode44=12 % ,
  \catcode45=12 % -
  \catcode46=12 % .
  \catcode58=12 % :
  \catcode64=11 % @
  \catcode123=1 % {
  \catcode125=2 % }
  \expandafter\let\expandafter\x\csname ver@pdftexcmds.sty\endcsname
  \ifx\x\relax % plain-TeX, first loading
  \else
    \def\empty{}%
    \ifx\x\empty % LaTeX, first loading,
      % variable is initialized, but \ProvidesPackage not yet seen
    \else
      \expandafter\ifx\csname PackageInfo\endcsname\relax
        \def\x#1#2{%
          \immediate\write-1{Package #1 Info: #2.}%
        }%
      \else
        \def\x#1#2{\PackageInfo{#1}{#2, stopped}}%
      \fi
      \x{pdftexcmds}{The package is already loaded}%
      \aftergroup\endinput
    \fi
  \fi
\endgroup%
%    \end{macrocode}
%    Package identification:
%    \begin{macrocode}
\begingroup\catcode61\catcode48\catcode32=10\relax%
  \catcode13=5 % ^^M
  \endlinechar=13 %
  \catcode35=6 % #
  \catcode39=12 % '
  \catcode40=12 % (
  \catcode41=12 % )
  \catcode44=12 % ,
  \catcode45=12 % -
  \catcode46=12 % .
  \catcode47=12 % /
  \catcode58=12 % :
  \catcode64=11 % @
  \catcode91=12 % [
  \catcode93=12 % ]
  \catcode123=1 % {
  \catcode125=2 % }
  \expandafter\ifx\csname ProvidesPackage\endcsname\relax
    \def\x#1#2#3[#4]{\endgroup
      \immediate\write-1{Package: #3 #4}%
      \xdef#1{#4}%
    }%
  \else
    \def\x#1#2[#3]{\endgroup
      #2[{#3}]%
      \ifx#1\@undefined
        \xdef#1{#3}%
      \fi
      \ifx#1\relax
        \xdef#1{#3}%
      \fi
    }%
  \fi
\expandafter\x\csname ver@pdftexcmds.sty\endcsname
\ProvidesPackage{pdftexcmds}%
  [2019/07/25 v0.30 Utility functions of pdfTeX for LuaTeX (HO)]%
%    \end{macrocode}
%
% \subsection{Catcodes}
%
%    \begin{macrocode}
\begingroup\catcode61\catcode48\catcode32=10\relax%
  \catcode13=5 % ^^M
  \endlinechar=13 %
  \catcode123=1 % {
  \catcode125=2 % }
  \catcode64=11 % @
  \def\x{\endgroup
    \expandafter\edef\csname pdftexcmds@AtEnd\endcsname{%
      \endlinechar=\the\endlinechar\relax
      \catcode13=\the\catcode13\relax
      \catcode32=\the\catcode32\relax
      \catcode35=\the\catcode35\relax
      \catcode61=\the\catcode61\relax
      \catcode64=\the\catcode64\relax
      \catcode123=\the\catcode123\relax
      \catcode125=\the\catcode125\relax
    }%
  }%
\x\catcode61\catcode48\catcode32=10\relax%
\catcode13=5 % ^^M
\endlinechar=13 %
\catcode35=6 % #
\catcode64=11 % @
\catcode123=1 % {
\catcode125=2 % }
\def\TMP@EnsureCode#1#2{%
  \edef\pdftexcmds@AtEnd{%
    \pdftexcmds@AtEnd
    \catcode#1=\the\catcode#1\relax
  }%
  \catcode#1=#2\relax
}
\TMP@EnsureCode{0}{12}%
\TMP@EnsureCode{1}{12}%
\TMP@EnsureCode{2}{12}%
\TMP@EnsureCode{10}{12}% ^^J
\TMP@EnsureCode{33}{12}% !
\TMP@EnsureCode{34}{12}% "
\TMP@EnsureCode{38}{4}% &
\TMP@EnsureCode{39}{12}% '
\TMP@EnsureCode{40}{12}% (
\TMP@EnsureCode{41}{12}% )
\TMP@EnsureCode{42}{12}% *
\TMP@EnsureCode{43}{12}% +
\TMP@EnsureCode{44}{12}% ,
\TMP@EnsureCode{45}{12}% -
\TMP@EnsureCode{46}{12}% .
\TMP@EnsureCode{47}{12}% /
\TMP@EnsureCode{58}{12}% :
\TMP@EnsureCode{60}{12}% <
\TMP@EnsureCode{62}{12}% >
\TMP@EnsureCode{91}{12}% [
\TMP@EnsureCode{93}{12}% ]
\TMP@EnsureCode{94}{7}% ^ (superscript)
\TMP@EnsureCode{95}{12}% _ (other)
\TMP@EnsureCode{96}{12}% `
\TMP@EnsureCode{126}{12}% ~ (other)
\edef\pdftexcmds@AtEnd{%
  \pdftexcmds@AtEnd
  \escapechar=\number\escapechar\relax
  \noexpand\endinput
}
\escapechar=92 %
%    \end{macrocode}
%
% \subsection{Load packages}
%
%    \begin{macrocode}
\begingroup\expandafter\expandafter\expandafter\endgroup
\expandafter\ifx\csname RequirePackage\endcsname\relax
  \def\TMP@RequirePackage#1[#2]{%
    \begingroup\expandafter\expandafter\expandafter\endgroup
    \expandafter\ifx\csname ver@#1.sty\endcsname\relax
      \input #1.sty\relax
    \fi
  }%
  \TMP@RequirePackage{infwarerr}[2007/09/09]%
  \TMP@RequirePackage{ifluatex}[2010/03/01]%
  \TMP@RequirePackage{ltxcmds}[2010/12/02]%
  \TMP@RequirePackage{ifpdf}[2010/09/13]%
\else
  \RequirePackage{infwarerr}[2007/09/09]%
  \RequirePackage{ifluatex}[2010/03/01]%
  \RequirePackage{ltxcmds}[2010/12/02]%
  \RequirePackage{ifpdf}[2010/09/13]%
\fi
%    \end{macrocode}
%
% \subsection{Without \hologo{LuaTeX}}
%
%    \begin{macrocode}
\ifluatex
\else
  \@PackageInfoNoLine{pdftexcmds}{LuaTeX not detected}%
  \def\pdftexcmds@nopdftex{%
    \@PackageInfoNoLine{pdftexcmds}{pdfTeX >= 1.30 not detected}%
    \let\pdftexcmds@nopdftex\relax
  }%
  \def\pdftexcmds@temp#1{%
    \begingroup\expandafter\expandafter\expandafter\endgroup
    \expandafter\ifx\csname pdf#1\endcsname\relax
      \pdftexcmds@nopdftex
    \else
      \expandafter\def\csname pdf@#1\expandafter\endcsname
      \expandafter##\expandafter{%
        \csname pdf#1\endcsname
      }%
    \fi
  }%
  \pdftexcmds@temp{strcmp}%
  \pdftexcmds@temp{escapehex}%
  \let\pdf@escapehexnative\pdf@escapehex
  \pdftexcmds@temp{unescapehex}%
  \let\pdf@unescapehexnative\pdf@unescapehex
  \pdftexcmds@temp{escapestring}%
  \pdftexcmds@temp{escapename}%
  \pdftexcmds@temp{filesize}%
  \pdftexcmds@temp{filemoddate}%
  \begingroup\expandafter\expandafter\expandafter\endgroup
  \expandafter\ifx\csname pdfshellescape\endcsname\relax
    \pdftexcmds@nopdftex
    \ltx@IfUndefined{pdftexversion}{%
    }{%
      \ifnum\pdftexversion>120 % 1.21a supports \ifeof18
        \ifeof18 %
          \chardef\pdf@shellescape=0 %
        \else
          \chardef\pdf@shellescape=1 %
        \fi
      \fi
    }%
  \else
    \def\pdf@shellescape{%
      \pdfshellescape
    }%
  \fi
  \begingroup\expandafter\expandafter\expandafter\endgroup
  \expandafter\ifx\csname pdffiledump\endcsname\relax
    \pdftexcmds@nopdftex
  \else
    \def\pdf@filedump#1#2#3{%
      \pdffiledump offset#1 length#2{#3}%
    }%
  \fi
%    \end{macrocode}
%    \begin{macrocode}
  \begingroup\expandafter\expandafter\expandafter\endgroup
  \expandafter\ifx\csname pdfmdfivesum\endcsname\relax
    \begingroup\expandafter\expandafter\expandafter\endgroup
    \expandafter\ifx\csname mdfivesum\endcsname\relax
      \pdftexcmds@nopdftex
    \else
      \def\pdf@mdfivesum#{\mdfivesum}%
      \let\pdf@mdfivesumnative\pdf@mdfivesum
      \def\pdf@filemdfivesum#{\mdfivesum file}%
    \fi
  \else
    \def\pdf@mdfivesum#{\pdfmdfivesum}%
    \let\pdf@mdfivesumnative\pdf@mdfivesum
    \def\pdf@filemdfivesum#{\pdfmdfivesum file}%
  \fi
%    \end{macrocode}
%    \begin{macrocode}
  \def\pdf@system#{%
    \immediate\write18%
  }%
  \def\pdftexcmds@temp#1{%
    \begingroup\expandafter\expandafter\expandafter\endgroup
    \expandafter\ifx\csname pdf#1\endcsname\relax
      \pdftexcmds@nopdftex
    \else
      \expandafter\let\csname pdf@#1\expandafter\endcsname
      \csname pdf#1\endcsname
    \fi
  }%
  \pdftexcmds@temp{resettimer}%
  \pdftexcmds@temp{elapsedtime}%
\fi
%    \end{macrocode}
%
% \subsection{\cs{pdf@primitive}, \cs{pdf@ifprimitive}}
%
%    Since version 1.40.0 \hologo{pdfTeX} has \cs{pdfprimitive} and
%    \cs{ifpdfprimitive}. And \cs{pdfprimitive} was fixed in
%    version 1.40.4.
%
%    \hologo{XeTeX} provides them under the name \cs{primitive} and
%    \cs{ifprimitive}. \hologo{LuaTeX} knows both name variants,
%    but they have possibly to be enabled first (|tex.enableprimitives|).
%
%    Depending on the format TeX Live uses a prefix |luatex|.
%
%    Caution: \cs{let} must be used for the definition of
%    the macros, especially because of \cs{ifpdfprimitive}.
%
% \subsubsection{Using \hologo{LuaTeX}'s \texttt{tex.enableprimitives}}
%
%    \begin{macrocode}
\ifluatex
%    \end{macrocode}
%    \begin{macro}{\pdftexcmds@directlua}
%    \begin{macrocode}
  \ifnum\luatexversion<36 %
    \def\pdftexcmds@directlua{\directlua0 }%
  \else
    \let\pdftexcmds@directlua\directlua
  \fi
%    \end{macrocode}
%    \end{macro}
%
%    \begin{macrocode}
  \begingroup
    \newlinechar=10 %
    \endlinechar=\newlinechar
    \pdftexcmds@directlua{%
      if tex.enableprimitives then
        tex.enableprimitives(
          'pdf@',
          {'primitive', 'ifprimitive', 'pdfdraftmode','draftmode'}
        )
        tex.enableprimitives('', {'luaescapestring'})
      end
    }%
  \endgroup %
%    \end{macrocode}
%
%    \begin{macrocode}
\fi
%    \end{macrocode}
%
% \subsubsection{Trying various names to find the primitives}
%
%    \begin{macro}{\pdftexcmds@strip@prefix}
%    \begin{macrocode}
\def\pdftexcmds@strip@prefix#1>{}
%    \end{macrocode}
%    \end{macro}
%    \begin{macrocode}
\def\pdftexcmds@temp#1#2#3{%
  \begingroup\expandafter\expandafter\expandafter\endgroup
  \expandafter\ifx\csname pdf@#1\endcsname\relax
    \begingroup
      \def\x{#3}%
      \edef\x{\expandafter\pdftexcmds@strip@prefix\meaning\x}%
      \escapechar=-1 %
      \edef\y{\expandafter\meaning\csname#2\endcsname}%
    \expandafter\endgroup
    \ifx\x\y
      \expandafter\let\csname pdf@#1\expandafter\endcsname
      \csname #2\endcsname
    \fi
  \fi
}
%    \end{macrocode}
%
%    \begin{macro}{\pdf@primitive}
%    \begin{macrocode}
\pdftexcmds@temp{primitive}{pdfprimitive}{pdfprimitive}% pdfTeX, oldLuaTeX
\pdftexcmds@temp{primitive}{primitive}{primitive}% XeTeX, luatex
\pdftexcmds@temp{primitive}{luatexprimitive}{pdfprimitive}% oldLuaTeX
\pdftexcmds@temp{primitive}{luatexpdfprimitive}{pdfprimitive}% oldLuaTeX
%    \end{macrocode}
%    \end{macro}
%    \begin{macro}{\pdf@ifprimitive}
%    \begin{macrocode}
\pdftexcmds@temp{ifprimitive}{ifpdfprimitive}{ifpdfprimitive}% pdfTeX, oldLuaTeX
\pdftexcmds@temp{ifprimitive}{ifprimitive}{ifprimitive}% XeTeX, luatex
\pdftexcmds@temp{ifprimitive}{luatexifprimitive}{ifpdfprimitive}% oldLuaTeX
\pdftexcmds@temp{ifprimitive}{luatexifpdfprimitive}{ifpdfprimitive}% oldLuaTeX
%    \end{macrocode}
%    \end{macro}
%
%    Disable broken \cs{pdfprimitive}.
%    \begin{macrocode}
\ifluatex\else
\begingroup
  \expandafter\ifx\csname pdf@primitive\endcsname\relax
  \else
    \expandafter\ifx\csname pdftexversion\endcsname\relax
    \else
      \ifnum\pdftexversion=140 %
        \expandafter\ifx\csname pdftexrevision\endcsname\relax
        \else
          \ifnum\pdftexrevision<4 %
            \endgroup
            \let\pdf@primitive\@undefined
            \@PackageInfoNoLine{pdftexcmds}{%
              \string\pdf@primitive\space disabled, %
              because\MessageBreak
              \string\pdfprimitive\space is broken until pdfTeX 1.40.4%
            }%
            \begingroup
          \fi
        \fi
      \fi
    \fi
  \fi
\endgroup
\fi
%    \end{macrocode}
%
% \subsubsection{Result}
%
%    \begin{macrocode}
\begingroup
  \@PackageInfoNoLine{pdftexcmds}{%
    \string\pdf@primitive\space is %
    \expandafter\ifx\csname pdf@primitive\endcsname\relax not \fi
    available%
  }%
  \@PackageInfoNoLine{pdftexcmds}{%
    \string\pdf@ifprimitive\space is %
    \expandafter\ifx\csname pdf@ifprimitive\endcsname\relax not \fi
    available%
  }%
\endgroup
%    \end{macrocode}
%
% \subsection{\hologo{XeTeX}}
%
%    Look for primitives \cs{shellescape}, \cs{strcmp}.
%    \begin{macrocode}
\def\pdftexcmds@temp#1{%
  \begingroup\expandafter\expandafter\expandafter\endgroup
  \expandafter\ifx\csname pdf@#1\endcsname\relax
    \begingroup
      \escapechar=-1 %
      \edef\x{\expandafter\meaning\csname#1\endcsname}%
      \def\y{#1}%
      \def\z##1->{}%
      \edef\y{\expandafter\z\meaning\y}%
    \expandafter\endgroup
    \ifx\x\y
      \expandafter\def\csname pdf@#1\expandafter\endcsname
      \expandafter{%
        \csname#1\endcsname
      }%
    \fi
  \fi
}%
\pdftexcmds@temp{shellescape}%
\pdftexcmds@temp{strcmp}%
%    \end{macrocode}
%
% \subsection{\cs{pdf@isprimitive}}
%
%    \begin{macrocode}
\def\pdf@isprimitive{%
  \begingroup\expandafter\expandafter\expandafter\endgroup
  \expandafter\ifx\csname pdf@strcmp\endcsname\relax
    \long\def\pdf@isprimitive##1{%
      \expandafter\pdftexcmds@isprimitive\expandafter{\meaning##1}%
    }%
    \long\def\pdftexcmds@isprimitive##1##2{%
      \expandafter\pdftexcmds@@isprimitive\expandafter{\string##2}{##1}%
    }%
    \def\pdftexcmds@@isprimitive##1##2{%
      \ifnum0\pdftexcmds@equal##1\delimiter##2\delimiter=1 %
        \expandafter\ltx@firstoftwo
      \else
        \expandafter\ltx@secondoftwo
      \fi
    }%
    \def\pdftexcmds@equal##1##2\delimiter##3##4\delimiter{%
      \ifx##1##3%
        \ifx\relax##2##4\relax
          1%
        \else
          \ifx\relax##2\relax
          \else
            \ifx\relax##4\relax
            \else
              \pdftexcmds@equalcont{##2}{##4}%
            \fi
          \fi
        \fi
      \fi
    }%
    \def\pdftexcmds@equalcont##1{%
      \def\pdftexcmds@equalcont####1####2##1##1##1##1{%
        ##1##1##1##1%
        \pdftexcmds@equal####1\delimiter####2\delimiter
      }%
    }%
    \expandafter\pdftexcmds@equalcont\csname fi\endcsname
  \else
    \long\def\pdf@isprimitive##1##2{%
      \ifnum\pdf@strcmp{\meaning##1}{\string##2}=0 %
        \expandafter\ltx@firstoftwo
      \else
        \expandafter\ltx@secondoftwo
      \fi
    }%
  \fi
}
\ifluatex
\ifx\pdfdraftmode\@undefined
  \let\pdfdraftmode\draftmode
\fi
\else
  \pdf@isprimitive
\fi
%    \end{macrocode}
%
% \subsection{\cs{pdf@draftmode}}
%
%
%    \begin{macrocode}
\let\pdftexcmds@temp\ltx@zero %
\ltx@IfUndefined{pdfdraftmode}{%
  \@PackageInfoNoLine{pdftexcmds}{\ltx@backslashchar pdfdraftmode not found}%
}{%
  \ifpdf
    \let\pdftexcmds@temp\ltx@one
    \@PackageInfoNoLine{pdftexcmds}{\ltx@backslashchar pdfdraftmode found}%
  \else
    \@PackageInfoNoLine{pdftexcmds}{%
      \ltx@backslashchar pdfdraftmode is ignored in DVI mode%
    }%
  \fi
}
\ifcase\pdftexcmds@temp
%    \end{macrocode}
%    \begin{macro}{\pdf@draftmode}
%    \begin{macrocode}
  \let\pdf@draftmode\ltx@zero
%    \end{macrocode}
%    \end{macro}
%    \begin{macro}{\pdf@ifdraftmode}
%    \begin{macrocode}
  \let\pdf@ifdraftmode\ltx@secondoftwo
%    \end{macrocode}
%    \end{macro}
%    \begin{macro}{\pdftexcmds@setdraftmode}
%    \begin{macrocode}
  \def\pdftexcmds@setdraftmode#1{}%
%    \end{macrocode}
%    \end{macro}
%    \begin{macrocode}
\else
%    \end{macrocode}
%    \begin{macro}{\pdftexcmds@draftmode}
%    \begin{macrocode}
  \let\pdftexcmds@draftmode\pdfdraftmode
%    \end{macrocode}
%    \end{macro}
%    \begin{macro}{\pdf@ifdraftmode}
%    \begin{macrocode}
  \def\pdf@ifdraftmode{%
    \ifnum\pdftexcmds@draftmode=\ltx@one
      \expandafter\ltx@firstoftwo
    \else
      \expandafter\ltx@secondoftwo
    \fi
  }%
%    \end{macrocode}
%    \end{macro}
%    \begin{macro}{\pdf@draftmode}
%    \begin{macrocode}
  \def\pdf@draftmode{%
    \ifnum\pdftexcmds@draftmode=\ltx@one
      \expandafter\ltx@one
    \else
      \expandafter\ltx@zero
    \fi
  }%
%    \end{macrocode}
%    \end{macro}
%    \begin{macro}{\pdftexcmds@setdraftmode}
%    \begin{macrocode}
  \def\pdftexcmds@setdraftmode#1{%
    \pdftexcmds@draftmode=#1\relax
  }%
%    \end{macrocode}
%    \end{macro}
%    \begin{macrocode}
\fi
%    \end{macrocode}
%    \begin{macro}{\pdf@setdraftmode}
%    \begin{macrocode}
\def\pdf@setdraftmode#1{%
  \begingroup
    \count\ltx@cclv=#1\relax
  \edef\x{\endgroup
    \noexpand\pdftexcmds@@setdraftmode{\the\count\ltx@cclv}%
  }%
  \x
}
%    \end{macrocode}
%    \end{macro}
%    \begin{macro}{\pdftexcmds@@setdraftmode}
%    \begin{macrocode}
\def\pdftexcmds@@setdraftmode#1{%
  \ifcase#1 %
    \pdftexcmds@setdraftmode{#1}%
  \or
    \pdftexcmds@setdraftmode{#1}%
  \else
    \@PackageWarning{pdftexcmds}{%
      \string\pdf@setdraftmode: Ignoring\MessageBreak
      invalid value `#1'%
    }%
  \fi
}
%    \end{macrocode}
%    \end{macro}
%
% \subsection{Load Lua module}
%
%    \begin{macrocode}
\ifluatex
\else
  \expandafter\pdftexcmds@AtEnd
\fi%
%    \end{macrocode}
%
%    \begin{macrocode}
\ifnum\luatexversion<80
  \begingroup\expandafter\expandafter\expandafter\endgroup
  \expandafter\ifx\csname RequirePackage\endcsname\relax
    \def\TMP@RequirePackage#1[#2]{%
      \begingroup\expandafter\expandafter\expandafter\endgroup
      \expandafter\ifx\csname ver@#1.sty\endcsname\relax
        \input #1.sty\relax
      \fi
    }%
    \TMP@RequirePackage{luatex-loader}[2009/04/10]%
  \else
    \RequirePackage{luatex-loader}[2009/04/10]%
  \fi
\fi
\pdftexcmds@directlua{%
  require("pdftexcmds")%
}
\ifnum\luatexversion>37 %
  \ifnum0%
      \pdftexcmds@directlua{%
        if status.ini_version then %
          tex.write("1")%
        end%
      }>0 %
    \everyjob\expandafter{%
      \the\everyjob
      \pdftexcmds@directlua{%
        require("pdftexcmds")%
      }%
    }%
  \fi
\fi
\begingroup
  \def\x{2019/07/25 v0.30}%
  \ltx@onelevel@sanitize\x
  \edef\y{%
    \pdftexcmds@directlua{%
      if oberdiek.pdftexcmds.getversion then %
        oberdiek.pdftexcmds.getversion()%
      end%
    }%
  }%
  \ifx\x\y
  \else
    \@PackageError{pdftexcmds}{%
      Wrong version of lua module.\MessageBreak
      Package version: \x\MessageBreak
      Lua module: \y
    }\@ehc
  \fi
\endgroup
%    \end{macrocode}
%
% \subsection{Lua functions}
%
% \subsubsection{Helper macros}
%
%    \begin{macro}{\pdftexcmds@toks}
%    \begin{macrocode}
\begingroup\expandafter\expandafter\expandafter\endgroup
\expandafter\ifx\csname newtoks\endcsname\relax
  \toksdef\pdftexcmds@toks=0 %
\else
  \csname newtoks\endcsname\pdftexcmds@toks
\fi
%    \end{macrocode}
%    \end{macro}
%
%    \begin{macro}{\pdftexcmds@Patch}
%    \begin{macrocode}
\def\pdftexcmds@Patch{0}
\ifnum\luatexversion>40 %
  \ifnum\luatexversion<66 %
    \def\pdftexcmds@Patch{1}%
  \fi
\fi
%    \end{macrocode}
%    \end{macro}
%    \begin{macrocode}
\ifcase\pdftexcmds@Patch
  \catcode`\&=14 %
\else
  \catcode`\&=9 %
%    \end{macrocode}
%    \begin{macro}{\pdftexcmds@PatchDecode}
%    \begin{macrocode}
  \def\pdftexcmds@PatchDecode#1\@nil{%
    \pdftexcmds@DecodeA#1^^A^^A\@nil{}%
  }%
%    \end{macrocode}
%    \end{macro}
%    \begin{macro}{\pdftexcmds@DecodeA}
%    \begin{macrocode}
  \def\pdftexcmds@DecodeA#1^^A^^A#2\@nil#3{%
    \ifx\relax#2\relax
      \ltx@ReturnAfterElseFi{%
        \pdftexcmds@DecodeB#3#1^^A^^B\@nil{}%
      }%
    \else
      \ltx@ReturnAfterFi{%
        \pdftexcmds@DecodeA#2\@nil{#3#1^^@}%
      }%
    \fi
  }%
%    \end{macrocode}
%    \end{macro}
%    \begin{macro}{\pdftexcmds@DecodeB}
%    \begin{macrocode}
  \def\pdftexcmds@DecodeB#1^^A^^B#2\@nil#3{%
    \ifx\relax#2\relax%
      \ltx@ReturnAfterElseFi{%
        \ltx@zero
        #3#1%
      }%
    \else
      \ltx@ReturnAfterFi{%
        \pdftexcmds@DecodeB#2\@nil{#3#1^^A}%
      }%
    \fi
  }%
%    \end{macrocode}
%    \end{macro}
%    \begin{macrocode}
\fi
%    \end{macrocode}
%
%    \begin{macrocode}
\ifnum\luatexversion<36 %
\else
  \catcode`\0=9 %
\fi
%    \end{macrocode}
%
% \subsubsection[Strings]{Strings \cite[``7.15 Strings'']{pdftex-manual}}
%
%    \begin{macro}{\pdf@strcmp}
%    \begin{macrocode}
\long\def\pdf@strcmp#1#2{%
  \directlua0{%
    oberdiek.pdftexcmds.strcmp("\luaescapestring{#1}",%
        "\luaescapestring{#2}")%
  }%
}%
%    \end{macrocode}
%    \end{macro}
%    \begin{macrocode}
\pdf@isprimitive
%    \end{macrocode}
%    \begin{macro}{\pdf@escapehex}
%    \begin{macrocode}
\long\def\pdf@escapehex#1{%
  \directlua0{%
    oberdiek.pdftexcmds.escapehex("\luaescapestring{#1}", "byte")%
  }%
}%
%    \end{macrocode}
%    \end{macro}
%    \begin{macro}{\pdf@escapehexnative}
%    \begin{macrocode}
\long\def\pdf@escapehexnative#1{%
  \directlua0{%
    oberdiek.pdftexcmds.escapehex("\luaescapestring{#1}")%
  }%
}%
%    \end{macrocode}
%    \end{macro}
%    \begin{macro}{\pdf@unescapehex}
%    \begin{macrocode}
\def\pdf@unescapehex#1{%
& \romannumeral\expandafter\pdftexcmds@PatchDecode
  \the\expandafter\pdftexcmds@toks
  \directlua0{%
    oberdiek.pdftexcmds.toks="pdftexcmds@toks"%
    oberdiek.pdftexcmds.unescapehex("\luaescapestring{#1}", "byte", \pdftexcmds@Patch)%
  }%
& \@nil
}%
%    \end{macrocode}
%    \end{macro}
%    \begin{macro}{\pdf@unescapehexnative}
%    \begin{macrocode}
\def\pdf@unescapehexnative#1{%
& \romannumeral\expandafter\pdftexcmds@PatchDecode
  \the\expandafter\pdftexcmds@toks
  \directlua0{%
    oberdiek.pdftexcmds.toks="pdftexcmds@toks"%
    oberdiek.pdftexcmds.unescapehex("\luaescapestring{#1}", \pdftexcmds@Patch)%
  }%
& \@nil
}%
%    \end{macrocode}
%    \end{macro}
%    \begin{macro}{\pdf@escapestring}
%    \begin{macrocode}
\long\def\pdf@escapestring#1{%
  \directlua0{%
    oberdiek.pdftexcmds.escapestring("\luaescapestring{#1}", "byte")%
  }%
}
%    \end{macrocode}
%    \end{macro}
%    \begin{macro}{\pdf@escapename}
%    \begin{macrocode}
\long\def\pdf@escapename#1{%
  \directlua0{%
    oberdiek.pdftexcmds.escapename("\luaescapestring{#1}", "byte")%
  }%
}
%    \end{macrocode}
%    \end{macro}
%    \begin{macro}{\pdf@escapenamenative}
%    \begin{macrocode}
\long\def\pdf@escapenamenative#1{%
  \directlua0{%
    oberdiek.pdftexcmds.escapename("\luaescapestring{#1}")%
  }%
}
%    \end{macrocode}
%    \end{macro}
%
% \subsubsection[Files]{Files \cite[``7.18 Files'']{pdftex-manual}}
%
%    \begin{macro}{\pdf@filesize}
%    \begin{macrocode}
\def\pdf@filesize#1{%
  \directlua0{%
    oberdiek.pdftexcmds.filesize("\luaescapestring{#1}")%
  }%
}
%    \end{macrocode}
%    \end{macro}
%    \begin{macro}{\pdf@filemoddate}
%    \begin{macrocode}
\def\pdf@filemoddate#1{%
  \directlua0{%
    oberdiek.pdftexcmds.filemoddate("\luaescapestring{#1}")%
  }%
}
%    \end{macrocode}
%    \end{macro}
%    \begin{macro}{\pdf@filedump}
%    \begin{macrocode}
\def\pdf@filedump#1#2#3{%
  \directlua0{%
    oberdiek.pdftexcmds.filedump("\luaescapestring{\number#1}",%
        "\luaescapestring{\number#2}",%
        "\luaescapestring{#3}")%
  }%
}%
%    \end{macrocode}
%    \end{macro}
%    \begin{macro}{\pdf@mdfivesum}
%    \begin{macrocode}
\long\def\pdf@mdfivesum#1{%
  \directlua0{%
    oberdiek.pdftexcmds.mdfivesum("\luaescapestring{#1}", "byte")%
  }%
}%
%    \end{macrocode}
%    \end{macro}
%    \begin{macro}{\pdf@mdfivesumnative}
%    \begin{macrocode}
\long\def\pdf@mdfivesumnative#1{%
  \directlua0{%
    oberdiek.pdftexcmds.mdfivesum("\luaescapestring{#1}")%
  }%
}%
%    \end{macrocode}
%    \end{macro}
%    \begin{macro}{\pdf@filemdfivesum}
%    \begin{macrocode}
\def\pdf@filemdfivesum#1{%
  \directlua0{%
    oberdiek.pdftexcmds.filemdfivesum("\luaescapestring{#1}")%
  }%
}%
%    \end{macrocode}
%    \end{macro}
%
% \subsubsection[Timekeeping]{Timekeeping \cite[``7.17 Timekeeping'']{pdftex-manual}}
%
%    \begin{macro}{\protected}
%    \begin{macrocode}
\let\pdftexcmds@temp=Y%
\begingroup\expandafter\expandafter\expandafter\endgroup
\expandafter\ifx\csname protected\endcsname\relax
  \pdftexcmds@directlua0{%
    if tex.enableprimitives then %
      tex.enableprimitives('', {'protected'})%
    end%
  }%
\fi
\begingroup\expandafter\expandafter\expandafter\endgroup
\expandafter\ifx\csname protected\endcsname\relax
  \let\pdftexcmds@temp=N%
\fi
%    \end{macrocode}
%    \end{macro}
%    \begin{macro}{\numexpr}
%    \begin{macrocode}
\begingroup\expandafter\expandafter\expandafter\endgroup
\expandafter\ifx\csname numexpr\endcsname\relax
  \pdftexcmds@directlua0{%
    if tex.enableprimitives then %
      tex.enableprimitives('', {'numexpr'})%
    end%
  }%
\fi
\begingroup\expandafter\expandafter\expandafter\endgroup
\expandafter\ifx\csname numexpr\endcsname\relax
  \let\pdftexcmds@temp=N%
\fi
%    \end{macrocode}
%    \end{macro}
%
%    \begin{macrocode}
\ifx\pdftexcmds@temp N%
  \@PackageWarningNoLine{pdftexcmds}{%
    Definitions of \ltx@backslashchar pdf@resettimer and%
    \MessageBreak
    \ltx@backslashchar pdf@elapsedtime are skipped, because%
    \MessageBreak
    e-TeX's \ltx@backslashchar protected or %
    \ltx@backslashchar numexpr are missing%
  }%
\else
%    \end{macrocode}
%
%    \begin{macro}{\pdf@resettimer}
%    \begin{macrocode}
  \protected\def\pdf@resettimer{%
    \pdftexcmds@directlua0{%
      oberdiek.pdftexcmds.resettimer()%
    }%
  }%
%    \end{macrocode}
%    \end{macro}
%
%    \begin{macro}{\pdf@elapsedtime}
%    \begin{macrocode}
  \protected\def\pdf@elapsedtime{%
    \numexpr
      \pdftexcmds@directlua0{%
        oberdiek.pdftexcmds.elapsedtime()%
      }%
    \relax
  }%
%    \end{macrocode}
%    \end{macro}
%    \begin{macrocode}
\fi
%    \end{macrocode}
%
% \subsubsection{Shell escape}
%
%    \begin{macro}{\pdf@shellescape}
%
%    \begin{macrocode}
\ifnum\luatexversion<68 %
\else
  \protected\edef\pdf@shellescape{%
   \numexpr\directlua{tex.sprint(%
         \number\catcodetable@string,status.shell_escape)}\relax}
\fi
%    \end{macrocode}
%    \end{macro}
%
%    \begin{macro}{\pdf@system}
%    \begin{macrocode}
\def\pdf@system#1{%
  \directlua0{%
    oberdiek.pdftexcmds.system("\luaescapestring{#1}")%
  }%
}
%    \end{macrocode}
%    \end{macro}
%
%    \begin{macro}{\pdf@lastsystemstatus}
%    \begin{macrocode}
\def\pdf@lastsystemstatus{%
  \directlua0{%
    oberdiek.pdftexcmds.lastsystemstatus()%
  }%
}
%    \end{macrocode}
%    \end{macro}
%    \begin{macro}{\pdf@lastsystemexit}
%    \begin{macrocode}
\def\pdf@lastsystemexit{%
  \directlua0{%
    oberdiek.pdftexcmds.lastsystemexit()%
  }%
}
%    \end{macrocode}
%    \end{macro}
%
%    \begin{macrocode}
\catcode`\0=12 %
%    \end{macrocode}
%
%    \begin{macro}{\pdf@pipe}
%    Check availability of |io.popen| first.
%    \begin{macrocode}
\ifnum0%
    \pdftexcmds@directlua{%
      if io.popen then %
        tex.write("1")%
      end%
    }%
    =1 %
  \def\pdf@pipe#1{%
&   \romannumeral\expandafter\pdftexcmds@PatchDecode
    \the\expandafter\pdftexcmds@toks
    \pdftexcmds@directlua{%
      oberdiek.pdftexcmds.toks="pdftexcmds@toks"%
      oberdiek.pdftexcmds.pipe("\luaescapestring{#1}", \pdftexcmds@Patch)%
    }%
&   \@nil
  }%
\fi
%    \end{macrocode}
%    \end{macro}
%
%    \begin{macrocode}
\pdftexcmds@AtEnd%
%</package>
%    \end{macrocode}
%
% \subsection{Lua module}
%
%    \begin{macrocode}
%<*lua>
%    \end{macrocode}
%
%    \begin{macrocode}
oberdiek = oberdiek or {}
local pdftexcmds = oberdiek.pdftexcmds or {}
oberdiek.pdftexcmds = pdftexcmds
local systemexitstatus
function pdftexcmds.getversion()
  tex.write("2019/07/25 v0.30")
end
%    \end{macrocode}
%
% \subsubsection[Strings]{Strings \cite[``7.15 Strings'']{pdftex-manual}}
%
%    \begin{macrocode}
function pdftexcmds.strcmp(A, B)
  if A == B then
    tex.write("0")
  elseif A < B then
    tex.write("-1")
  else
    tex.write("1")
  end
end
local function utf8_to_byte(str)
  local i = 0
  local n = string.len(str)
  local t = {}
  while i < n do
    i = i + 1
    local a = string.byte(str, i)
    if a < 128 then
      table.insert(t, string.char(a))
    else
      if a >= 192 and i < n then
        i = i + 1
        local b = string.byte(str, i)
        if b < 128 or b >= 192 then
          i = i - 1
        elseif a == 194 then
          table.insert(t, string.char(b))
        elseif a == 195 then
          table.insert(t, string.char(b + 64))
        end
      end
    end
  end
  return table.concat(t)
end
function pdftexcmds.escapehex(str, mode)
  if mode == "byte" then
    str = utf8_to_byte(str)
  end
  tex.write((string.gsub(str, ".",
    function (ch)
      return string.format("%02X", string.byte(ch))
    end
  )))
end
%    \end{macrocode}
%    See procedure |unescapehex| in file \xfile{utils.c} of \hologo{pdfTeX}.
%    Caution: |tex.write| ignores leading spaces.
%    \begin{macrocode}
function pdftexcmds.unescapehex(str, mode, patch)
  local a = 0
  local first = true
  local result = {}
  for i = 1, string.len(str), 1 do
    local ch = string.byte(str, i)
    if ch >= 48 and ch <= 57 then
      ch = ch - 48
    elseif ch >= 65 and ch <= 70 then
      ch = ch - 55
    elseif ch >= 97 and ch <= 102 then
      ch = ch - 87
    else
      ch = nil
    end
    if ch then
      if first then
        a = ch * 16
        first = false
      else
        table.insert(result, a + ch)
        first = true
      end
    end
  end
  if not first then
    table.insert(result, a)
  end
  if patch == 1 then
    local temp = {}
    for i, a in ipairs(result) do
      if a == 0 then
        table.insert(temp, 1)
        table.insert(temp, 1)
      else
        if a == 1 then
          table.insert(temp, 1)
          table.insert(temp, 2)
        else
          table.insert(temp, a)
        end
      end
    end
    result = temp
  end
  if mode == "byte" then
    local utf8 = {}
    for i, a in ipairs(result) do
      if a < 128 then
        table.insert(utf8, a)
      else
        if a < 192 then
          table.insert(utf8, 194)
          a = a - 128
        else
          table.insert(utf8, 195)
          a = a - 192
        end
        table.insert(utf8, a + 128)
      end
    end
    result = utf8
  end
%    \end{macrocode}
%    this next line added for current luatex; this is the only
%    change in the file.  eroux, 28apr13. (v 0.21)
%    \begin{macrocode}
  local unpack = _G["unpack"] or table.unpack
  tex.settoks(pdftexcmds.toks, string.char(unpack(result)))
end
%    \end{macrocode}
%    See procedure |escapestring| in file \xfile{utils.c} of \hologo{pdfTeX}.
%    \begin{macrocode}
function pdftexcmds.escapestring(str, mode)
  if mode == "byte" then
    str = utf8_to_byte(str)
  end
  tex.write((string.gsub(str, ".",
    function (ch)
      local b = string.byte(ch)
      if b < 33 or b > 126 then
        return string.format("\\%.3o", b)
      end
      if b == 40 or b == 41 or b == 92 then
        return "\\" .. ch
      end
%    \end{macrocode}
%    Lua 5.1 returns the match in case of return value |nil|.
%    \begin{macrocode}
      return nil
    end
  )))
end
%    \end{macrocode}
%    See procedure |escapename| in file \xfile{utils.c} of \hologo{pdfTeX}.
%    \begin{macrocode}
function pdftexcmds.escapename(str, mode)
  if mode == "byte" then
    str = utf8_to_byte(str)
  end
  tex.write((string.gsub(str, ".",
    function (ch)
      local b = string.byte(ch)
      if b == 0 then
%    \end{macrocode}
%    In Lua 5.0 |nil| could be used for the empty string,
%    But |nil| returns the match in Lua 5.1, thus we use
%    the empty string explicitly.
%    \begin{macrocode}
        return ""
      end
      if b <= 32 or b >= 127
          or b == 35 or b == 37 or b == 40 or b == 41
          or b == 47 or b == 60 or b == 62 or b == 91
          or b == 93 or b == 123 or b == 125 then
        return string.format("#%.2X", b)
      else
%    \end{macrocode}
%    Lua 5.1 returns the match in case of return value |nil|.
%    \begin{macrocode}
        return nil
      end
    end
  )))
end
%    \end{macrocode}
%
% \subsubsection[Files]{Files \cite[``7.18 Files'']{pdftex-manual}}
%
%    \begin{macrocode}
function pdftexcmds.filesize(filename)
  local foundfile = kpse.find_file(filename, "tex", true)
  if foundfile then
    local size = lfs.attributes(foundfile, "size")
    if size then
      tex.write(size)
    end
  end
end
%    \end{macrocode}
%    See procedure |makepdftime| in file \xfile{utils.c} of \hologo{pdfTeX}.
%    \begin{macrocode}
function pdftexcmds.filemoddate(filename)
  local foundfile = kpse.find_file(filename, "tex", true)
  if foundfile then
    local date = lfs.attributes(foundfile, "modification")
    if date then
      local d = os.date("*t", date)
      if d.sec >= 60 then
        d.sec = 59
      end
      local u = os.date("!*t", date)
      local off = 60 * (d.hour - u.hour) + d.min - u.min
      if d.year ~= u.year then
        if d.year > u.year then
          off = off + 1440
        else
          off = off - 1440
        end
      elseif d.yday ~= u.yday then
        if d.yday > u.yday then
          off = off + 1440
        else
          off = off - 1440
        end
      end
      local timezone
      if off == 0 then
        timezone = "Z"
      else
        local hours = math.floor(off / 60)
        local mins = math.abs(off - hours * 60)
        timezone = string.format("%+03d'%02d'", hours, mins)
      end
      tex.write(string.format("D:%04d%02d%02d%02d%02d%02d%s",
          d.year, d.month, d.day, d.hour, d.min, d.sec, timezone))
    end
  end
end
function pdftexcmds.filedump(offset, length, filename)
  length = tonumber(length)
  if length and length > 0 then
    local foundfile = kpse.find_file(filename, "tex", true)
    if foundfile then
      offset = tonumber(offset)
      if not offset then
        offset = 0
      end
      local filehandle = io.open(foundfile, "rb")
      if filehandle then
        if offset > 0 then
          filehandle:seek("set", offset)
        end
        local dump = filehandle:read(length)
        pdftexcmds.escapehex(dump)
        filehandle:close()
      end
    end
  end
end
function pdftexcmds.mdfivesum(str, mode)
  if mode == "byte" then
    str = utf8_to_byte(str)
  end
  pdftexcmds.escapehex(md5.sum(str))
end
function pdftexcmds.filemdfivesum(filename)
  local foundfile = kpse.find_file(filename, "tex", true)
  if foundfile then
    local filehandle = io.open(foundfile, "rb")
    if filehandle then
      local contents = filehandle:read("*a")
      pdftexcmds.escapehex(md5.sum(contents))
      filehandle:close()
    end
  end
end
%    \end{macrocode}
%
% \subsubsection[Timekeeping]{Timekeeping \cite[``7.17 Timekeeping'']{pdftex-manual}}
%
%    The functions for timekeeping are based on
%    Andy Thomas' work \cite{AndyThomas:Analog}.
%    Changes:
%    \begin{itemize}
%    \item Overflow check is added.
%    \item |string.format| is used to avoid exponential number
%          representation for sure.
%    \item |tex.write| is used instead of |tex.print| to get
%          tokens with catcode 12 and without appended \cs{endlinechar}.
%    \end{itemize}
%    \begin{macrocode}
local basetime = 0
function pdftexcmds.resettimer()
  basetime = os.clock()
end
function pdftexcmds.elapsedtime()
  local val = (os.clock() - basetime) * 65536 + .5
  if val > 2147483647 then
    val = 2147483647
  end
  tex.write(string.format("%d", val))
end
%    \end{macrocode}
%
% \subsubsection[Miscellaneous]{Miscellaneous \cite[``7.21 Miscellaneous'']{pdftex-manual}}
%
%    \begin{macrocode}
function pdftexcmds.shellescape()
  if os.execute then
    if status
        and status.luatex_version
        and status.luatex_version >= 68 then
      tex.write(os.execute())
    else
      local result = os.execute()
      if result == 0 then
        tex.write("0")
      else
        if result == nil then
          tex.write("0")
        else
          tex.write("1")
        end
      end
    end
  else
    tex.write("0")
  end
end
function pdftexcmds.system(cmdline)
  systemexitstatus = nil
  texio.write_nl("log", "system(" .. cmdline .. ") ")
  if os.execute then
    texio.write("log", "executed.")
    systemexitstatus = os.execute(cmdline)
  else
    texio.write("log", "disabled.")
  end
end
function pdftexcmds.lastsystemstatus()
  local result = tonumber(systemexitstatus)
  if result then
    local x = math.floor(result / 256)
    tex.write(result - 256 * math.floor(result / 256))
  end
end
function pdftexcmds.lastsystemexit()
  local result = tonumber(systemexitstatus)
  if result then
    tex.write(math.floor(result / 256))
  end
end
function pdftexcmds.pipe(cmdline, patch)
  local result
  systemexitstatus = nil
  texio.write_nl("log", "pipe(" .. cmdline ..") ")
  if io.popen then
    texio.write("log", "executed.")
    local handle = io.popen(cmdline, "r")
    if handle then
      result = handle:read("*a")
      handle:close()
    end
  else
    texio.write("log", "disabled.")
  end
  if result then
    if patch == 1 then
      local temp = {}
      for i, a in ipairs(result) do
        if a == 0 then
          table.insert(temp, 1)
          table.insert(temp, 1)
        else
          if a == 1 then
            table.insert(temp, 1)
            table.insert(temp, 2)
          else
            table.insert(temp, a)
          end
        end
      end
      result = temp
    end
    tex.settoks(pdftexcmds.toks, result)
  else
    tex.settoks(pdftexcmds.toks, "")
  end
end
%    \end{macrocode}
%    \begin{macrocode}
%</lua>
%    \end{macrocode}
%
% \section{Test}
%
% \subsection{Catcode checks for loading}
%
%    \begin{macrocode}
%<*test1>
%    \end{macrocode}
%    \begin{macrocode}
\catcode`\{=1 %
\catcode`\}=2 %
\catcode`\#=6 %
\catcode`\@=11 %
\expandafter\ifx\csname count@\endcsname\relax
  \countdef\count@=255 %
\fi
\expandafter\ifx\csname @gobble\endcsname\relax
  \long\def\@gobble#1{}%
\fi
\expandafter\ifx\csname @firstofone\endcsname\relax
  \long\def\@firstofone#1{#1}%
\fi
\expandafter\ifx\csname loop\endcsname\relax
  \expandafter\@firstofone
\else
  \expandafter\@gobble
\fi
{%
  \def\loop#1\repeat{%
    \def\body{#1}%
    \iterate
  }%
  \def\iterate{%
    \body
      \let\next\iterate
    \else
      \let\next\relax
    \fi
    \next
  }%
  \let\repeat=\fi
}%
\def\RestoreCatcodes{}
\count@=0 %
\loop
  \edef\RestoreCatcodes{%
    \RestoreCatcodes
    \catcode\the\count@=\the\catcode\count@\relax
  }%
\ifnum\count@<255 %
  \advance\count@ 1 %
\repeat

\def\RangeCatcodeInvalid#1#2{%
  \count@=#1\relax
  \loop
    \catcode\count@=15 %
  \ifnum\count@<#2\relax
    \advance\count@ 1 %
  \repeat
}
\def\RangeCatcodeCheck#1#2#3{%
  \count@=#1\relax
  \loop
    \ifnum#3=\catcode\count@
    \else
      \errmessage{%
        Character \the\count@\space
        with wrong catcode \the\catcode\count@\space
        instead of \number#3%
      }%
    \fi
  \ifnum\count@<#2\relax
    \advance\count@ 1 %
  \repeat
}
\def\space{ }
\expandafter\ifx\csname LoadCommand\endcsname\relax
  \def\LoadCommand{\input pdftexcmds.sty\relax}%
\fi
\def\Test{%
  \RangeCatcodeInvalid{0}{47}%
  \RangeCatcodeInvalid{58}{64}%
  \RangeCatcodeInvalid{91}{96}%
  \RangeCatcodeInvalid{123}{255}%
  \catcode`\@=12 %
  \catcode`\\=0 %
  \catcode`\%=14 %
  \LoadCommand
  \RangeCatcodeCheck{0}{36}{15}%
  \RangeCatcodeCheck{37}{37}{14}%
  \RangeCatcodeCheck{38}{47}{15}%
  \RangeCatcodeCheck{48}{57}{12}%
  \RangeCatcodeCheck{58}{63}{15}%
  \RangeCatcodeCheck{64}{64}{12}%
  \RangeCatcodeCheck{65}{90}{11}%
  \RangeCatcodeCheck{91}{91}{15}%
  \RangeCatcodeCheck{92}{92}{0}%
  \RangeCatcodeCheck{93}{96}{15}%
  \RangeCatcodeCheck{97}{122}{11}%
  \RangeCatcodeCheck{123}{255}{15}%
  \RestoreCatcodes
}
\Test
\csname @@end\endcsname
\end
%    \end{macrocode}
%    \begin{macrocode}
%</test1>
%    \end{macrocode}
%
% \subsection{Test for \cs{pdf@isprimitive}}
%
%    \begin{macrocode}
%<*test2>
\catcode`\{=1 %
\catcode`\}=2 %
\catcode`\#=6 %
\catcode`\@=11 %
\input pdftexcmds.sty\relax
\def\msg#1{%
  \begingroup
    \escapechar=92 %
    \immediate\write16{#1}%
  \endgroup
}
\long\def\test#1#2#3#4{%
  \begingroup
    #4%
    \def\str{%
      Test \string\pdf@isprimitive
      {\string #1}{\string #2}{...}: %
    }%
    \pdf@isprimitive{#1}{#2}{%
      \ifx#3Y%
        \msg{\str true ==> OK.}%
      \else
        \errmessage{\str false ==> FAILED}%
      \fi
    }{%
      \ifx#3Y%
        \errmessage{\str true ==> FAILED}%
      \else
        \msg{\str false ==> OK.}%
      \fi
    }%
  \endgroup
}
\test\relax\relax Y{}
\test\foobar\relax Y{\let\foobar\relax}
\test\foobar\relax N{}
\test\hbox\hbox Y{}
\test\foobar@hbox\hbox Y{\let\foobar@hbox\hbox}
\test\if\if Y{}
\test\if\ifx N{}
\test\ifx\if N{}
\test\par\par Y{}
\test\hbox\par N{}
\test\par\hbox N{}
\csname @@end\endcsname\end
%</test2>
%    \end{macrocode}
%
% \subsection{Test for \cs{pdf@shellescape}}
%
%    \begin{macrocode}
%<*test-shell>
\catcode`\{=1 %
\catcode`\}=2 %
\catcode`\#=6 %
\catcode`\@=11 %
\input pdftexcmds.sty\relax
\def\msg#{\immediate\write16}
\def\MaybeEnd{}
\ifx\luatexversion\UnDeFiNeD
\else
  \ifnum\luatexversion<68 %
    \ifx\pdf@shellescape\@undefined
      \msg{SHELL=U}%
      \msg{OK (LuaTeX < 0.68)}%
    \else
      \msg{SHELL=defined}%
      \errmessage{Failed (LuaTeX < 0.68)}%
    \fi
    \def\MaybeEnd{\csname @@end\endcsname\end}%
  \fi
\fi
\MaybeEnd
\ifx\pdf@shellescape\@undefined
  \msg{SHELL=U}%
\else
  \msg{SHELL=\number\pdf@shellescape}%
\fi
\ifx\expected\@undefined
\else
  \ifx\expected\relax
    \msg{EXPECTED=U}%
    \ifx\pdf@shellescape\@undefined
      \msg{OK}%
    \else
      \errmessage{Failed}%
    \fi
  \else
    \msg{EXPECTED=\number\expected}%
    \ifnum\pdf@shellescape=\expected\relax
      \msg{OK}%
    \else
      \errmessage{Failed}%
    \fi
  \fi
\fi
\csname @@end\endcsname\end
%</test-shell>
%    \end{macrocode}
%
% \subsection{Test for escape functions}
%
%    \begin{macrocode}
%<*test-escape>
\catcode`\{=1 %
\catcode`\}=2 %
\catcode`\#=6 %
\catcode`\^=7 %
\catcode`\@=11 %
\errorcontextlines=1000 %
\input pdftexcmds.sty\relax
\def\msg#1{%
  \begingroup
    \escapechar=92 %
    \immediate\write16{#1}%
  \endgroup
}
%    \end{macrocode}
%    \begin{macrocode}
\begingroup
  \catcode`\@=11 %
  \countdef\count@=255 %
  \def\space{ }%
  \long\def\@whilenum#1\do #2{%
    \ifnum #1\relax
      #2\relax
      \@iwhilenum{#1\relax#2\relax}%
    \fi
  }%
  \long\def\@iwhilenum#1{%
    \ifnum #1%
      \expandafter\@iwhilenum
    \else
      \expandafter\ltx@gobble
    \fi
    {#1}%
  }%
  \gdef\AllBytes{}%
  \count@=0 %
  \catcode0=12 %
  \@whilenum\count@<256 \do{%
    \lccode0=\count@
    \ifnum\count@=32 %
      \xdef\AllBytes{\AllBytes\space}%
    \else
      \lowercase{%
        \xdef\AllBytes{\AllBytes^^@}%
      }%
    \fi
    \advance\count@ by 1 %
  }%
\endgroup
%    \end{macrocode}
%    \begin{macrocode}
\def\AllBytesHex{%
  000102030405060708090A0B0C0D0E0F%
  101112131415161718191A1B1C1D1E1F%
  202122232425262728292A2B2C2D2E2F%
  303132333435363738393A3B3C3D3E3F%
  404142434445464748494A4B4C4D4E4F%
  505152535455565758595A5B5C5D5E5F%
  606162636465666768696A6B6C6D6E6F%
  707172737475767778797A7B7C7D7E7F%
  808182838485868788898A8B8C8D8E8F%
  909192939495969798999A9B9C9D9E9F%
  A0A1A2A3A4A5A6A7A8A9AAABACADAEAF%
  B0B1B2B3B4B5B6B7B8B9BABBBCBDBEBF%
  C0C1C2C3C4C5C6C7C8C9CACBCCCDCECF%
  D0D1D2D3D4D5D6D7D8D9DADBDCDDDEDF%
  E0E1E2E3E4E5E6E7E8E9EAEBECEDEEEF%
  F0F1F2F3F4F5F6F7F8F9FAFBFCFDFEFF%
}
\ltx@onelevel@sanitize\AllBytesHex
\expandafter\lowercase\expandafter{%
  \expandafter\def\expandafter\AllBytesHexLC
      \expandafter{\AllBytesHex}%
}
\begingroup
  \catcode`\#=12 %
  \xdef\AllBytesName{%
    #01#02#03#04#05#06#07#08#09#0A#0B#0C#0D#0E#0F%
    #10#11#12#13#14#15#16#17#18#19#1A#1B#1C#1D#1E#1F%
    #20!"#23$#25&'#28#29*+,-.#2F%
    0123456789:;#3C=#3E?%
    @ABCDEFGHIJKLMNO%
    PQRSTUVWXYZ#5B\ltx@backslashchar#5D^_%
    `abcdefghijklmno%
    pqrstuvwxyz#7B|#7D\string~#7F%
    #80#81#82#83#84#85#86#87#88#89#8A#8B#8C#8D#8E#8F%
    #90#91#92#93#94#95#96#97#98#99#9A#9B#9C#9D#9E#9F%
    #A0#A1#A2#A3#A4#A5#A6#A7#A8#A9#AA#AB#AC#AD#AE#AF%
    #B0#B1#B2#B3#B4#B5#B6#B7#B8#B9#BA#BB#BC#BD#BE#BF%
    #C0#C1#C2#C3#C4#C5#C6#C7#C8#C9#CA#CB#CC#CD#CE#CF%
    #D0#D1#D2#D3#D4#D5#D6#D7#D8#D9#DA#DB#DC#DD#DE#DF%
    #E0#E1#E2#E3#E4#E5#E6#E7#E8#E9#EA#EB#EC#ED#EE#EF%
    #F0#F1#F2#F3#F4#F5#F6#F7#F8#F9#FA#FB#FC#FD#FE#FF%
  }%
\endgroup
\ltx@onelevel@sanitize\AllBytesName
\edef\AllBytesFromName{\expandafter\ltx@gobble\AllBytes}
\begingroup
  \def\|{|}%
  \edef\%{\ltx@percentchar}%
  \catcode`\|=0 %
  \catcode`\#=12 %
  \catcode`\~=12 %
  \catcode`\\=12 %
  |xdef|AllBytesString{%
    \000\001\002\003\004\005\006\007\010\011\012\013\014\015\016\017%
    \020\021\022\023\024\025\026\027\030\031\032\033\034\035\036\037%
    \040!"#$|%&'\(\)*+,-./%
    0123456789:;<=>?%
    @ABCDEFGHIJKLMNO%
    PQRSTUVWXYZ[\\]^_%
    `abcdefghijklmno%
    pqrstuvwxyz{||}~\177%
    \200\201\202\203\204\205\206\207\210\211\212\213\214\215\216\217%
    \220\221\222\223\224\225\226\227\230\231\232\233\234\235\236\237%
    \240\241\242\243\244\245\246\247\250\251\252\253\254\255\256\257%
    \260\261\262\263\264\265\266\267\270\271\272\273\274\275\276\277%
    \300\301\302\303\304\305\306\307\310\311\312\313\314\315\316\317%
    \320\321\322\323\324\325\326\327\330\331\332\333\334\335\336\337%
    \340\341\342\343\344\345\346\347\350\351\352\353\354\355\356\357%
    \360\361\362\363\364\365\366\367\370\371\372\373\374\375\376\377%
  }%
|endgroup
\ltx@onelevel@sanitize\AllBytesString
%    \end{macrocode}
%    \begin{macrocode}
\def\Test#1#2#3{%
  \begingroup
    \expandafter\expandafter\expandafter\def
    \expandafter\expandafter\expandafter\TestResult
    \expandafter\expandafter\expandafter{%
      #1{#2}%
    }%
    \ifx\TestResult#3%
    \else
      \newlinechar=10 %
      \msg{Expect:^^J#3}%
      \msg{Result:^^J\TestResult}%
      \errmessage{\string#2 -\string#1-> \string#3}%
    \fi
  \endgroup
}
\def\test#1#2#3{%
  \edef\TestFrom{#2}%
  \edef\TestExpect{#3}%
  \ltx@onelevel@sanitize\TestExpect
  \Test#1\TestFrom\TestExpect
}
\test\pdf@unescapehex{74657374}{test}
\begingroup
  \catcode0=12 %
  \catcode1=12 %
  \test\pdf@unescapehex{740074017400740174}{t^^@t^^At^^@t^^At}%
\endgroup
\Test\pdf@escapehex\AllBytes\AllBytesHex
\Test\pdf@unescapehex\AllBytesHex\AllBytes
\Test\pdf@escapename\AllBytes\AllBytesName
\Test\pdf@escapestring\AllBytes\AllBytesString
%    \end{macrocode}
%    \begin{macrocode}
\csname @@end\endcsname\end
%</test-escape>
%    \end{macrocode}
%
% \section{Installation}
%
% \subsection{Download}
%
% \paragraph{Package.} This package is available on
% CTAN\footnote{\CTANpkg{pdftexcmds}}:
% \begin{description}
% \item[\CTAN{macros/latex/contrib/oberdiek/pdftexcmds.dtx}] The source file.
% \item[\CTAN{macros/latex/contrib/oberdiek/pdftexcmds.pdf}] Documentation.
% \end{description}
%
%
% \paragraph{Bundle.} All the packages of the bundle `oberdiek'
% are also available in a TDS compliant ZIP archive. There
% the packages are already unpacked and the documentation files
% are generated. The files and directories obey the TDS standard.
% \begin{description}
% \item[\CTANinstall{install/macros/latex/contrib/oberdiek.tds.zip}]
% \end{description}
% \emph{TDS} refers to the standard ``A Directory Structure
% for \TeX\ Files'' (\CTAN{tds/tds.pdf}). Directories
% with \xfile{texmf} in their name are usually organized this way.
%
% \subsection{Bundle installation}
%
% \paragraph{Unpacking.} Unpack the \xfile{oberdiek.tds.zip} in the
% TDS tree (also known as \xfile{texmf} tree) of your choice.
% Example (linux):
% \begin{quote}
%   |unzip oberdiek.tds.zip -d ~/texmf|
% \end{quote}
%
% \paragraph{Script installation.}
% Check the directory \xfile{TDS:scripts/oberdiek/} for
% scripts that need further installation steps.
% Package \xpackage{attachfile2} comes with the Perl script
% \xfile{pdfatfi.pl} that should be installed in such a way
% that it can be called as \texttt{pdfatfi}.
% Example (linux):
% \begin{quote}
%   |chmod +x scripts/oberdiek/pdfatfi.pl|\\
%   |cp scripts/oberdiek/pdfatfi.pl /usr/local/bin/|
% \end{quote}
%
% \subsection{Package installation}
%
% \paragraph{Unpacking.} The \xfile{.dtx} file is a self-extracting
% \docstrip\ archive. The files are extracted by running the
% \xfile{.dtx} through \plainTeX:
% \begin{quote}
%   \verb|tex pdftexcmds.dtx|
% \end{quote}
%
% \paragraph{TDS.} Now the different files must be moved into
% the different directories in your installation TDS tree
% (also known as \xfile{texmf} tree):
% \begin{quote}
% \def\t{^^A
% \begin{tabular}{@{}>{\ttfamily}l@{ $\rightarrow$ }>{\ttfamily}l@{}}
%   pdftexcmds.sty & tex/generic/oberdiek/pdftexcmds.sty\\
%   oberdiek.pdftexcmds.lua & scripts/oberdiek/oberdiek.pdftexcmds.lua\\
%   pdftexcmds.lua & scripts/oberdiek/pdftexcmds.lua\\
%   pdftexcmds.pdf & doc/latex/oberdiek/pdftexcmds.pdf\\
%   test/pdftexcmds-test1.tex & doc/latex/oberdiek/test/pdftexcmds-test1.tex\\
%   test/pdftexcmds-test2.tex & doc/latex/oberdiek/test/pdftexcmds-test2.tex\\
%   test/pdftexcmds-test-shell.tex & doc/latex/oberdiek/test/pdftexcmds-test-shell.tex\\
%   test/pdftexcmds-test-escape.tex & doc/latex/oberdiek/test/pdftexcmds-test-escape.tex\\
%   pdftexcmds.dtx & source/latex/oberdiek/pdftexcmds.dtx\\
% \end{tabular}^^A
% }^^A
% \sbox0{\t}^^A
% \ifdim\wd0>\linewidth
%   \begingroup
%     \advance\linewidth by\leftmargin
%     \advance\linewidth by\rightmargin
%   \edef\x{\endgroup
%     \def\noexpand\lw{\the\linewidth}^^A
%   }\x
%   \def\lwbox{^^A
%     \leavevmode
%     \hbox to \linewidth{^^A
%       \kern-\leftmargin\relax
%       \hss
%       \usebox0
%       \hss
%       \kern-\rightmargin\relax
%     }^^A
%   }^^A
%   \ifdim\wd0>\lw
%     \sbox0{\small\t}^^A
%     \ifdim\wd0>\linewidth
%       \ifdim\wd0>\lw
%         \sbox0{\footnotesize\t}^^A
%         \ifdim\wd0>\linewidth
%           \ifdim\wd0>\lw
%             \sbox0{\scriptsize\t}^^A
%             \ifdim\wd0>\linewidth
%               \ifdim\wd0>\lw
%                 \sbox0{\tiny\t}^^A
%                 \ifdim\wd0>\linewidth
%                   \lwbox
%                 \else
%                   \usebox0
%                 \fi
%               \else
%                 \lwbox
%               \fi
%             \else
%               \usebox0
%             \fi
%           \else
%             \lwbox
%           \fi
%         \else
%           \usebox0
%         \fi
%       \else
%         \lwbox
%       \fi
%     \else
%       \usebox0
%     \fi
%   \else
%     \lwbox
%   \fi
% \else
%   \usebox0
% \fi
% \end{quote}
% If you have a \xfile{docstrip.cfg} that configures and enables \docstrip's
% TDS installing feature, then some files can already be in the right
% place, see the documentation of \docstrip.
%
% \subsection{Refresh file name databases}
%
% If your \TeX~distribution
% (\teTeX, \mikTeX, \dots) relies on file name databases, you must refresh
% these. For example, \teTeX\ users run \verb|texhash| or
% \verb|mktexlsr|.
%
% \subsection{Some details for the interested}
%
% \paragraph{Unpacking with \LaTeX.}
% The \xfile{.dtx} chooses its action depending on the format:
% \begin{description}
% \item[\plainTeX:] Run \docstrip\ and extract the files.
% \item[\LaTeX:] Generate the documentation.
% \end{description}
% If you insist on using \LaTeX\ for \docstrip\ (really,
% \docstrip\ does not need \LaTeX), then inform the autodetect routine
% about your intention:
% \begin{quote}
%   \verb|latex \let\install=y\input{pdftexcmds.dtx}|
% \end{quote}
% Do not forget to quote the argument according to the demands
% of your shell.
%
% \paragraph{Generating the documentation.}
% You can use both the \xfile{.dtx} or the \xfile{.drv} to generate
% the documentation. The process can be configured by the
% configuration file \xfile{ltxdoc.cfg}. For instance, put this
% line into this file, if you want to have A4 as paper format:
% \begin{quote}
%   \verb|\PassOptionsToClass{a4paper}{article}|
% \end{quote}
% An example follows how to generate the
% documentation with pdf\LaTeX:
% \begin{quote}
%\begin{verbatim}
%pdflatex pdftexcmds.dtx
%bibtex pdftexcmds.aux
%makeindex -s gind.ist pdftexcmds.idx
%pdflatex pdftexcmds.dtx
%makeindex -s gind.ist pdftexcmds.idx
%pdflatex pdftexcmds.dtx
%\end{verbatim}
% \end{quote}
%
% \printbibliography[
%   heading=bibnumbered,
% ]
%
% \begin{History}
%   \begin{Version}{2007/11/11 v0.1}
%   \item
%     First version.
%   \end{Version}
%   \begin{Version}{2007/11/12 v0.2}
%   \item
%     Short description fixed.
%   \end{Version}
%   \begin{Version}{2007/12/12 v0.3}
%   \item
%     Organization of Lua code as module.
%   \end{Version}
%   \begin{Version}{2009/04/10 v0.4}
%   \item
%     Adaptation for syntax change of \cs{directlua} in
%     \hologo{LuaTeX} 0.36.
%   \end{Version}
%   \begin{Version}{2009/09/22 v0.5}
%   \item
%     \cs{pdf@primitive}, \cs{pdf@ifprimitive} added.
%   \item
%     \hologo{XeTeX}'s variants are detected for
%     \cs{pdf@shellescape}, \cs{pdf@strcmp}, \cs{pdf@primitive},
%     \cs{pdf@ifprimitive}.
%   \end{Version}
%   \begin{Version}{2009/09/23 v0.6}
%   \item
%     Macro \cs{pdf@isprimitive} added.
%   \end{Version}
%   \begin{Version}{2009/12/12 v0.7}
%   \item
%     Short info shortened.
%   \end{Version}
%   \begin{Version}{2010/03/01 v0.8}
%   \item
%     Required date for package \xpackage{ifluatex} updated.
%   \end{Version}
%   \begin{Version}{2010/04/01 v0.9}
%   \item
%     Use \cs{ifeof18} for defining \cs{pdf@shellescape} between
%     \hologo{pdfTeX} 1.21a (inclusive) and 1.30.0 (exclusive).
%   \end{Version}
%   \begin{Version}{2010/11/04 v0.10}
%   \item
%     \cs{pdf@draftmode}, \cs{pdf@ifdraftmode} and
%     \cs{pdf@setdraftmode} added.
%   \end{Version}
%   \begin{Version}{2010/11/11 v0.11}
%   \item
%     Missing \cs{RequirePackage} for package \xpackage{ifpdf} added.
%   \end{Version}
%   \begin{Version}{2011/01/30 v0.12}
%   \item
%     Already loaded package files are not input in \hologo{plainTeX}.
%   \end{Version}
%   \begin{Version}{2011/03/04 v0.13}
%   \item
%     Improved Lua function \texttt{shellescape} that also
%     uses the result of \texttt{os.execute()} (thanks to Philipp Stephani).
%   \end{Version}
%   \begin{Version}{2011/04/10 v0.14}
%   \item
%     Version check of loaded module added.
%   \item
%     Patch for bug in \hologo{LuaTeX} between 0.40.6 and 0.65 that
%     is fixed in revision 4096.
%   \end{Version}
%   \begin{Version}{2011/04/16 v0.15}
%   \item
%     \hologo{LuaTeX}: \cs{pdf@shellescape} is only supported
%     for version 0.70.0 and higher due to a bug, \texttt{os.execute()}
%     crashes in some circumstances. Fixed in \hologo{LuaTeX}
%     beta-0.70.0, revision 4167.
%   \end{Version}
%   \begin{Version}{2011/04/22 v0.16}
%   \item
%     Previous fix was not working due to a wrong catcode of digit
%     zero (due to easily support the old \cs{directlua0}).
%     The version border is lowered to 0.68, because some
%     beta-0.67.0 seems also to work.
%   \end{Version}
%   \begin{Version}{2011/06/29 v0.17}
%   \item
%     Documentation addition to \cs{pdf@shellescape}.
%   \end{Version}
%   \begin{Version}{2011/07/01 v0.18}
%   \item
%     Add Lua module loading in \cs{everyjob} for \hologo{iniTeX}
%     (\hologo{LuaTeX} only).
%   \end{Version}
%   \begin{Version}{2011/07/28 v0.19}
%   \item
%     Missing space in an info message added (Martin M\"unch).
%   \end{Version}
%   \begin{Version}{2011/11/29 v0.20}
%   \item
%     \cs{pdf@resettimer} and \cs{pdf@elapsedtime} added
%     (thanks Andy Thomas).
%   \end{Version}
%   \begin{Version}{2016/05/10 v0.21}
%   \item
%      local unpack added
%     (thanks \'{E}lie Roux).
%   \end{Version}
%   \begin{Version}{2016/05/21 v0.22}
%   \item
%     adjust \cs{textbackslas}h usage in bib file for biber bug.
%   \end{Version}
%   \begin{Version}{2016/10/02 v0.23}
%   \item
%     add file.close to lua filehandles (github pull request).
%   \end{Version}
%   \begin{Version}{2017/01/29 v0.24}
%   \item
%     Avoid loading luatex-loader for current luatex. (Use
%     pdftexcmds.lua not oberdiek.pdftexcmds.lua to simplify file
%     search with standard require)
%   \end{Version}
%   \begin{Version}{2017/03/19 v0.25}
%   \item
%     New \cs{pdf@shellescape} for Lua\TeX, see github issue 20.
%   \end{Version}
%   \begin{Version}{2018/01/21 v0.26}
%   \item
%     use rb not r mode for file open github issue 34.
%   \end{Version}
%   \begin{Version}{2018/01/30 v0.27}
%   \item
%     \cs{pdf@mdfivesum} for \hologo{XeTeX}
%   \end{Version}
%   \begin{Version}{2018/09/07 v0.28}
%   \item
%     Fix catcode regime in luatex sprint for \cs{pdf@shellescape} GH issue 45
%   \end{Version}
%   \begin{Version}{2018/09/10 v0.29}
%   \item
%     Actually do the fix described above in the code, not just document it.
%   \end{Version}
%   \begin{Version}{2019/07/25 v0.30}
%   \item
%     remove uses of module function, see PR70
%   \end{Version}
% \end{History}
%
% \PrintIndex
%
% \Finale
\endinput

%        (quote the arguments according to the demands of your shell)
%
% Documentation:
%    (a) If pdftexcmds.drv is present:
%           latex pdftexcmds.drv
%    (b) Without pdftexcmds.drv:
%           latex pdftexcmds.dtx; ...
%    The class ltxdoc loads the configuration file ltxdoc.cfg
%    if available. Here you can specify further options, e.g.
%    use A4 as paper format:
%       \PassOptionsToClass{a4paper}{article}
%
%    Programm calls to get the documentation (example):
%       pdflatex pdftexcmds.dtx
%       bibtex pdftexcmds.aux
%       makeindex -s gind.ist pdftexcmds.idx
%       pdflatex pdftexcmds.dtx
%       makeindex -s gind.ist pdftexcmds.idx
%       pdflatex pdftexcmds.dtx
%
% Installation:
%    TDS:tex/generic/oberdiek/pdftexcmds.sty
%    TDS:scripts/oberdiek/oberdiek.pdftexcmds.lua
%    TDS:scripts/oberdiek/pdftexcmds.lua
%    TDS:doc/latex/oberdiek/pdftexcmds.pdf
%    TDS:doc/latex/oberdiek/test/pdftexcmds-test1.tex
%    TDS:doc/latex/oberdiek/test/pdftexcmds-test2.tex
%    TDS:doc/latex/oberdiek/test/pdftexcmds-test-shell.tex
%    TDS:doc/latex/oberdiek/test/pdftexcmds-test-escape.tex
%    TDS:source/latex/oberdiek/pdftexcmds.dtx
%
%<*ignore>
\begingroup
  \catcode123=1 %
  \catcode125=2 %
  \def\x{LaTeX2e}%
\expandafter\endgroup
\ifcase 0\ifx\install y1\fi\expandafter
         \ifx\csname processbatchFile\endcsname\relax\else1\fi
         \ifx\fmtname\x\else 1\fi\relax
\else\csname fi\endcsname
%</ignore>
%<*install>
\input docstrip.tex
\Msg{************************************************************************}
\Msg{* Installation}
\Msg{* Package: pdftexcmds 2019/07/25 v0.30 Utility functions of pdfTeX for LuaTeX (HO)}
\Msg{************************************************************************}

\keepsilent
\askforoverwritefalse

\let\MetaPrefix\relax
\preamble

This is a generated file.

Project: pdftexcmds
Version: 2019/07/25 v0.30

Copyright (C) 2007, 2009-2011 by
   Heiko Oberdiek <heiko.oberdiek at googlemail.com>

This work may be distributed and/or modified under the
conditions of the LaTeX Project Public License, either
version 1.3c of this license or (at your option) any later
version. This version of this license is in
   https://www.latex-project.org/lppl/lppl-1-3c.txt
and the latest version of this license is in
   https://www.latex-project.org/lppl.txt
and version 1.3 or later is part of all distributions of
LaTeX version 2005/12/01 or later.

This work has the LPPL maintenance status "maintained".

The Current Maintainers of this work are
Heiko Oberdiek and the Oberdiek Package Support Group
https://github.com/ho-tex/oberdiek/issues


The Base Interpreter refers to any `TeX-Format',
because some files are installed in TDS:tex/generic//.

This work consists of the main source file pdftexcmds.dtx
and the derived files
   pdftexcmds.sty, pdftexcmds.pdf, pdftexcmds.ins, pdftexcmds.drv,
   pdftexcmds.bib, pdftexcmds-test1.tex, pdftexcmds-test2.tex,
   pdftexcmds-test-shell.tex, pdftexcmds-test-escape.tex,
   oberdiek.pdftexcmds.lua, pdftexcmds.lua.

\endpreamble
\let\MetaPrefix\DoubleperCent

\generate{%
  \file{pdftexcmds.ins}{\from{pdftexcmds.dtx}{install}}%
  \file{pdftexcmds.drv}{\from{pdftexcmds.dtx}{driver}}%
  \nopreamble
  \nopostamble
  \file{pdftexcmds.bib}{\from{pdftexcmds.dtx}{bib}}%
  \usepreamble\defaultpreamble
  \usepostamble\defaultpostamble
  \usedir{tex/generic/oberdiek}%
  \file{pdftexcmds.sty}{\from{pdftexcmds.dtx}{package}}%
%  \usedir{doc/latex/oberdiek/test}%
%  \file{pdftexcmds-test1.tex}{\from{pdftexcmds.dtx}{test1}}%
%  \file{pdftexcmds-test2.tex}{\from{pdftexcmds.dtx}{test2}}%
%  \file{pdftexcmds-test-shell.tex}{\from{pdftexcmds.dtx}{test-shell}}%
%  \file{pdftexcmds-test-escape.tex}{\from{pdftexcmds.dtx}{test-escape}}%
  \nopreamble
  \nopostamble
%  \usedir{source/latex/oberdiek/catalogue}%
%  \file{pdftexcmds.xml}{\from{pdftexcmds.dtx}{catalogue}}%
}
\def\MetaPrefix{-- }
\def\defaultpostamble{%
  \MetaPrefix^^J%
  \MetaPrefix\space End of File `\outFileName'.%
}
\def\currentpostamble{\defaultpostamble}%
\generate{%
  \usedir{scripts/oberdiek}%
  \file{oberdiek.pdftexcmds.lua}{\from{pdftexcmds.dtx}{lua}}%
  \file{pdftexcmds.lua}{\from{pdftexcmds.dtx}{lua}}%
}

\catcode32=13\relax% active space
\let =\space%
\Msg{************************************************************************}
\Msg{*}
\Msg{* To finish the installation you have to move the following}
\Msg{* file into a directory searched by TeX:}
\Msg{*}
\Msg{*     pdftexcmds.sty}
\Msg{*}
\Msg{* And install the following script files:}
\Msg{*}
\Msg{*     oberdiek.pdftexcmds.lua, pdftexcmds.lua}
\Msg{*}
\Msg{* To produce the documentation run the file `pdftexcmds.drv'}
\Msg{* through LaTeX.}
\Msg{*}
\Msg{* Happy TeXing!}
\Msg{*}
\Msg{************************************************************************}

\endbatchfile
%</install>
%<*bib>
@online{AndyThomas:Analog,
  author={Thomas, Andy},
  title={Analog of {\texttt{\csname textbackslash\endcsname}pdfelapsedtime} for
      {\hologo{LuaTeX}} and {\hologo{XeTeX}}},
  url={http://tex.stackexchange.com/a/32531},
  urldate={2011-11-29},
}
%</bib>
%<*ignore>
\fi
%</ignore>
%<*driver>
\NeedsTeXFormat{LaTeX2e}
\ProvidesFile{pdftexcmds.drv}%
  [2019/07/25 v0.30 Utility functions of pdfTeX for LuaTeX (HO)]%
\documentclass{ltxdoc}
\usepackage{holtxdoc}[2011/11/22]
\usepackage{paralist}
\usepackage{csquotes}
\usepackage[
  backend=bibtex,
  bibencoding=ascii,
  alldates=iso8601,
]{biblatex}[2011/11/13]
\bibliography{oberdiek-source}
\bibliography{pdftexcmds}
\begin{document}
  \DocInput{pdftexcmds.dtx}%
\end{document}
%</driver>
% \fi
%
%
% \CharacterTable
%  {Upper-case    \A\B\C\D\E\F\G\H\I\J\K\L\M\N\O\P\Q\R\S\T\U\V\W\X\Y\Z
%   Lower-case    \a\b\c\d\e\f\g\h\i\j\k\l\m\n\o\p\q\r\s\t\u\v\w\x\y\z
%   Digits        \0\1\2\3\4\5\6\7\8\9
%   Exclamation   \!     Double quote  \"     Hash (number) \#
%   Dollar        \$     Percent       \%     Ampersand     \&
%   Acute accent  \'     Left paren    \(     Right paren   \)
%   Asterisk      \*     Plus          \+     Comma         \,
%   Minus         \-     Point         \.     Solidus       \/
%   Colon         \:     Semicolon     \;     Less than     \<
%   Equals        \=     Greater than  \>     Question mark \?
%   Commercial at \@     Left bracket  \[     Backslash     \\
%   Right bracket \]     Circumflex    \^     Underscore    \_
%   Grave accent  \`     Left brace    \{     Vertical bar  \|
%   Right brace   \}     Tilde         \~}
%
% \GetFileInfo{pdftexcmds.drv}
%
% \title{The \xpackage{pdftexcmds} package}
% \date{2019/07/25 v0.30}
% \author{Heiko Oberdiek\thanks
% {Please report any issues at \url{https://github.com/ho-tex/oberdiek/issues}}}
%
% \maketitle
%
% \begin{abstract}
% \hologo{LuaTeX} provides most of the commands of \hologo{pdfTeX} 1.40. However
% a number of utility functions are removed. This package tries to fill
% the gap and implements some of the missing primitive using Lua.
% \end{abstract}
%
% \tableofcontents
%
% \def\csi#1{\texttt{\textbackslash\textit{#1}}}
%
% \section{Documentation}
%
% Some primitives of \hologo{pdfTeX} \cite{pdftex-manual}
% are not defined by \hologo{LuaTeX} \cite{luatex-manual}.
% This package implements macro based solutions using Lua code
% for the following missing \hologo{pdfTeX} primitives;
% \begin{compactitem}
% \item \cs{pdfstrcmp}
% \item \cs{pdfunescapehex}
% \item \cs{pdfescapehex}
% \item \cs{pdfescapename}
% \item \cs{pdfescapestring}
% \item \cs{pdffilesize}
% \item \cs{pdffilemoddate}
% \item \cs{pdffiledump}
% \item \cs{pdfmdfivesum}
% \item \cs{pdfresettimer}
% \item \cs{pdfelapsedtime}
% \item |\immediate\write18|
% \end{compactitem}
% The original names of the primitives cannot be used:
% \begin{itemize}
% \item
% The syntax for their arguments cannot easily
% simulated by macros. The primitives using key words
% such as |file| (\cs{pdfmdfivesum}) or |offset| and |length|
% (\cs{pdffiledump}) and uses \meta{general text} for the other
% arguments. Using token registers assignments, \meta{general text} could
% be catched. However, the simulated primitives are expandable
% and register assignments would destroy this important property.
% (\meta{general text} allows something like |\expandafter\bgroup ...}|.)
% \item
% The original primitives can be expanded using one expansion step.
% The new macros need two expansion steps because of the additional
% macro expansion. Example:
% \begin{quote}
%   |\expandafter\foo\pdffilemoddate{file}|\\
%   vs.\\
%   |\expandafter\expandafter\expandafter|\\
%   |\foo\pdf@filemoddate{file}|
% \end{quote}
% \end{itemize}
%
% \hologo{LuaTeX} isn't stable yet and thus the status of this package is
% \emph{experimental}. Feedback is welcome.
%
% \subsection{General principles}
%
% \begin{description}
% \item[Naming convention:]
%   Usually this package defines a macro |\pdf@|\meta{cmd} if
%   \hologo{pdfTeX} provides |\pdf|\meta{cmd}.
% \item[Arguments:] The order of arguments in |\pdf@|\meta{cmd}
%   is the same as for the corresponding primitive of \hologo{pdfTeX}.
%   The arguments are ordinary undelimited \hologo{TeX} arguments,
%   no \meta{general text} and without additional keywords.
% \item[Expandibility:]
%   The macro |\pdf@|\meta{cmd} is expandable if the
%   corresponding \hologo{pdfTeX} primitive has this property.
%   Exact two expansion steps are necessary (first is the macro
%   expansion) except for \cs{pdf@primitive} and \cs{pdf@ifprimitive}.
%   The latter ones are not macros, but have the direct meaning of the
%   primitive.
% \item[Without \hologo{LuaTeX}:]
%   The macros |\pdf@|\meta{cmd} are mapped to the commands
%   of \hologo{pdfTeX} if they are available. Otherwise they are undefined.
% \item[Availability:]
%   The macros that the packages provides are undefined, if
%   the necessary primitives are not found and cannot be
%   implemented by Lua.
% \end{description}
%
% \subsection{Macros}
%
% \subsubsection[Strings]{Strings \cite[``7.15 Strings'']{pdftex-manual}}
%
% \begin{declcs}{pdf@strcmp} \M{stringA} \M{stringB}
% \end{declcs}
% Same as |\pdfstrcmp{|\meta{stringA}|}{|\meta{stringB}|}|.
%
% \begin{declcs}{pdf@unescapehex} \M{string}
% \end{declcs}
% Same as |\pdfunescapehex{|\meta{string}|}|.
% The argument is a byte string given in hexadecimal notation.
% The result are character tokens from 0 until 255 with
% catcode 12 and the space with catcode 10.
%
% \begin{declcs}{pdf@escapehex} \M{string}\\
%   \cs{pdf@escapestring} \M{string}\\
%   \cs{pdf@escapename} \M{string}
% \end{declcs}
% Same as the primitives of \hologo{pdfTeX}. However \hologo{pdfTeX} does not
% know about characters with codes 256 and larger. Thus the
% string is treated as byte string, characters with more than
% eight bits are ignored.
%
% \subsubsection[Files]{Files \cite[``7.18 Files'']{pdftex-manual}}
%
% \begin{declcs}{pdf@filesize} \M{filename}
% \end{declcs}
% Same as |\pdffilesize{|\meta{filename}|}|.
%
% \begin{declcs}{pdf@filemoddate} \M{filename}
% \end{declcs}
% Same as |\pdffilemoddate{|\meta{filename}|}|.
%
% \begin{declcs}{pdf@filedump} \M{offset} \M{length} \M{filename}
% \end{declcs}
% Same as |\pdffiledump offset| \meta{offset} |length| \meta{length}
% |{|\meta{filename}|}|. Both \meta{offset} and \meta{length} must
% not be empty, but must be a valid \hologo{TeX} number.
%
% \begin{declcs}{pdf@mdfivesum} \M{string}
% \end{declcs}
% Same as |\pdfmdfivesum{|\meta{string}|}|. Keyword |file| is supported
% by macro \cs{pdf@filemdfivesum}.
%
% \begin{declcs}{pdf@filemdfivesum} \M{filename}
% \end{declcs}
% Same as |\pdfmdfivesum file{|\meta{filename}|}|.
%
% \subsubsection[Timekeeping]{Timekeeping \cite[``7.17 Timekeeping'']{pdftex-manual}}
%
% The timekeeping macros are based on Andy Thomas' work \cite{AndyThomas:Analog}.
%
% \begin{declcs}{pdf@resettimer}
% \end{declcs}
% Same as \cs{pdfresettimer}, it resets the internal timer.
%
% \begin{declcs}{pdf@elapsedtime}
% \end{declcs}
% Same as \cs{pdfelapsedtime}. It behaves like a read-only integer.
% For printing purposes it can be prefixed by \cs{the} or \cs{number}.
% It measures the time in scaled seconds (seconds multiplied with 65536)
% since the latest call of \cs{pdf@resettimer} or start of
% program/package. The resolution, the shortest time interval that
% can be measured, depends on the program and system.
% \begin{itemize}
% \item \hologo{pdfTeX} with |gettimeofday|: $\ge$ 1/65536\,s
% \item \hologo{pdfTeX} with |ftime|: $\ge$ 1\,ms
% \item \hologo{pdfTeX} with |time|: $\ge$ 1\,s
% \item \hologo{LuaTeX}: $\ge$ 10\,ms\\
%  (|os.clock()| returns a float number with two decimal digits in
%  \hologo{LuaTeX} beta-0.70.1-2011061416 (rev 4277)).
% \end{itemize}
%
% \subsubsection[Miscellaneous]{Miscellaneous \cite[``7.21 Miscellaneous'']{pdftex-manual}}
%
% \begin{declcs}{pdf@draftmode}
% \end{declcs}
% If the \TeX\ compiler knows \cs{pdfdraftmode} or \cs{draftmode}
% (\hologo{pdfTeX},
% \hologo{LuaTeX}), then \cs{pdf@draftmode} returns, whether
% this mode is enabled. The result is an implicit number:
% one means the draft mode is available and enabled.
% If the value is zero, then the mode is not active or
% \cs{pdfdraftmode} is not available.
% An explicit number is yielded by \cs{number}\cs{pdf@draftmode}.
% The macro cannot
% be used to change the mode, see \cs{pdf@setdraftmode}.
%
% \begin{declcs}{pdf@ifdraftmode} \M{true} \M{false}
% \end{declcs}
% If \cs{pdfdraftmode} is available and enabled, \meta{true} is
% called, otherwise \meta{false} is executed.
%
% \begin{declcs}{pdf@setdraftmode} \M{value}
% \end{declcs}
% Macro \cs{pdf@setdraftmode} expects the number zero or one as
% \meta{value}. Zero deactivates the mode and one enables the draft mode.
% The macro does not have an effect, if the feature \cs{pdfdraftmode} is not
% available.
%
% \begin{declcs}{pdf@shellescape}
% \end{declcs}
% Same as |\pdfshellescape|. It is or expands to |1| if external
% commands can be executed and |0| otherwise. In \hologo{pdfTeX} external
% commands must be enabled first by command line option or
% configuration option. In \hologo{LuaTeX} option |--safer| disables
% the execution of external commands.
%
% In \hologo{LuaTeX} before 0.68.0 \cs{pdf@shellescape} is not
% available due to a bug in |os.execute()|. The argumentless form
% crashes in some circumstances with segmentation fault.
% (It is fixed in version 0.68.0 or revision 4167 of \hologo{LuaTeX}.
% and packported to some version of 0.67.0).
%
% Hints for usage:
% \begin{itemize}
% \item Before its use \cs{pdf@shellescape} should be tested,
% whether it is available. Example with package \xpackage{ltxcmds}
% (loaded by package \xpackage{pdftexcmds}):
%\begin{quote}
%\begin{verbatim}
%\ltx@IfUndefined{pdf@shellescape}{%
%  % \pdf@shellescape is undefined
%}{%
%  % \pdf@shellescape is available
%}
%\end{verbatim}
%\end{quote}
% Use \cs{ltx@ifundefined} in expandable contexts.
% \item \cs{pdf@shellescape} might be a numerical constant,
% expands to the primitive, or expands to a plain number.
% Therefore use it in contexts where these differences does not matter.
% \item Use in comparisons, e.g.:
%   \begin{quote}
%     |\ifnum\pdf@shellescape=0 ...|
%   \end{quote}
% \item Print the number: |\number\pdf@shellescape|
% \end{itemize}
%
% \begin{declcs}{pdf@system} \M{cmdline}
% \end{declcs}
% It is a wrapper for |\immediate\write18| in \hologo{pdfTeX} or
% |os.execute| in \hologo{LuaTeX}.
%
% In theory |os.execute|
% returns a status number. But its meaning is quite
% undefined. Are there some reliable properties?
% Does it make sense to provide an user interface to
% this status exit code?
%
% \begin{declcs}{pdf@primitive} \csi{cmd}
% \end{declcs}
% Same as \cs{pdfprimitive} in \hologo{pdfTeX} or \hologo{LuaTeX}.
% In \hologo{XeTeX} the
% primitive is called \cs{primitive}. Despite the current definition
% of the command \csi{cmd}, it's meaning as primitive is used.
%
% \begin{declcs}{pdf@ifprimitive} \csi{cmd}
% \end{declcs}
% Same as \cs{ifpdfprimitive} in \hologo{pdfTeX} or
% \hologo{LuaTeX}. \hologo{XeTeX} calls
% it \cs{ifprimitive}. It is a switch that checks if the command
% \csi{cmd} has it's primitive meaning.
%
% \subsubsection{Additional macro: \cs{pdf@isprimitive}}
%
% \begin{declcs}{pdf@isprimitive} \csi{cmd1} \csi{cmd2} \M{true} \M{false}
% \end{declcs}
% If \csi{cmd1} has the primitive meaning given by the primitive name
% of \csi{cmd2}, then the argument \meta{true} is executed, otherwise
% \meta{false}. The macro \cs{pdf@isprimitive} is expandable.
% Internally it checks the result of \cs{meaning} and is therefore
% available for all \hologo{TeX} variants, even the original \hologo{TeX}.
% Example with \hologo{LaTeX}:
%\begin{quote}
%\begin{verbatim}
%\makeatletter
%\pdf@isprimitive{@@input}{input}{%
%  \typeout{\string\@@input\space is original\string\input}%
%}{%
%  \typeout{Oops, \string\@@input\space is not the %
%           original\string\input}%
%}
%\end{verbatim}
%\end{quote}
%
% \subsubsection{Experimental}
%
% \begin{declcs}{pdf@unescapehexnative} \M{string}\\
%   \cs{pdf@escapehexnative} \M{string}\\
%   \cs{pdf@escapenamenative} \M{string}\\
%   \cs{pdf@mdfivesumnative} \M{string}
% \end{declcs}
% The variants without |native| in the macro name are supposed to
% be compatible with \hologo{pdfTeX}. However characters with more than
% eight bits are not supported and are ignored. If \hologo{LuaTeX} is
% running, then its UTF-8 coded strings are used. Thus the full
% unicode character range is supported. However the result
% differs from \hologo{pdfTeX} for characters with eight or more bits.
%
% \begin{declcs}{pdf@pipe} \M{cmdline}
% \end{declcs}
% It calls \meta{cmdline} and returns the output of the external
% program in the usual manner as byte string (catcode 12, space with
% catcode 10). The Lua documentation says, that the used |io.popen|
% may not be available on all platforms. Then macro \cs{pdf@pipe}
% is undefined.
%
% \StopEventually{
% }
%
% \section{Implementation}
%
%    \begin{macrocode}
%<*package>
%    \end{macrocode}
%
% \subsection{Reload check and package identification}
%    Reload check, especially if the package is not used with \LaTeX.
%    \begin{macrocode}
\begingroup\catcode61\catcode48\catcode32=10\relax%
  \catcode13=5 % ^^M
  \endlinechar=13 %
  \catcode35=6 % #
  \catcode39=12 % '
  \catcode44=12 % ,
  \catcode45=12 % -
  \catcode46=12 % .
  \catcode58=12 % :
  \catcode64=11 % @
  \catcode123=1 % {
  \catcode125=2 % }
  \expandafter\let\expandafter\x\csname ver@pdftexcmds.sty\endcsname
  \ifx\x\relax % plain-TeX, first loading
  \else
    \def\empty{}%
    \ifx\x\empty % LaTeX, first loading,
      % variable is initialized, but \ProvidesPackage not yet seen
    \else
      \expandafter\ifx\csname PackageInfo\endcsname\relax
        \def\x#1#2{%
          \immediate\write-1{Package #1 Info: #2.}%
        }%
      \else
        \def\x#1#2{\PackageInfo{#1}{#2, stopped}}%
      \fi
      \x{pdftexcmds}{The package is already loaded}%
      \aftergroup\endinput
    \fi
  \fi
\endgroup%
%    \end{macrocode}
%    Package identification:
%    \begin{macrocode}
\begingroup\catcode61\catcode48\catcode32=10\relax%
  \catcode13=5 % ^^M
  \endlinechar=13 %
  \catcode35=6 % #
  \catcode39=12 % '
  \catcode40=12 % (
  \catcode41=12 % )
  \catcode44=12 % ,
  \catcode45=12 % -
  \catcode46=12 % .
  \catcode47=12 % /
  \catcode58=12 % :
  \catcode64=11 % @
  \catcode91=12 % [
  \catcode93=12 % ]
  \catcode123=1 % {
  \catcode125=2 % }
  \expandafter\ifx\csname ProvidesPackage\endcsname\relax
    \def\x#1#2#3[#4]{\endgroup
      \immediate\write-1{Package: #3 #4}%
      \xdef#1{#4}%
    }%
  \else
    \def\x#1#2[#3]{\endgroup
      #2[{#3}]%
      \ifx#1\@undefined
        \xdef#1{#3}%
      \fi
      \ifx#1\relax
        \xdef#1{#3}%
      \fi
    }%
  \fi
\expandafter\x\csname ver@pdftexcmds.sty\endcsname
\ProvidesPackage{pdftexcmds}%
  [2019/07/25 v0.30 Utility functions of pdfTeX for LuaTeX (HO)]%
%    \end{macrocode}
%
% \subsection{Catcodes}
%
%    \begin{macrocode}
\begingroup\catcode61\catcode48\catcode32=10\relax%
  \catcode13=5 % ^^M
  \endlinechar=13 %
  \catcode123=1 % {
  \catcode125=2 % }
  \catcode64=11 % @
  \def\x{\endgroup
    \expandafter\edef\csname pdftexcmds@AtEnd\endcsname{%
      \endlinechar=\the\endlinechar\relax
      \catcode13=\the\catcode13\relax
      \catcode32=\the\catcode32\relax
      \catcode35=\the\catcode35\relax
      \catcode61=\the\catcode61\relax
      \catcode64=\the\catcode64\relax
      \catcode123=\the\catcode123\relax
      \catcode125=\the\catcode125\relax
    }%
  }%
\x\catcode61\catcode48\catcode32=10\relax%
\catcode13=5 % ^^M
\endlinechar=13 %
\catcode35=6 % #
\catcode64=11 % @
\catcode123=1 % {
\catcode125=2 % }
\def\TMP@EnsureCode#1#2{%
  \edef\pdftexcmds@AtEnd{%
    \pdftexcmds@AtEnd
    \catcode#1=\the\catcode#1\relax
  }%
  \catcode#1=#2\relax
}
\TMP@EnsureCode{0}{12}%
\TMP@EnsureCode{1}{12}%
\TMP@EnsureCode{2}{12}%
\TMP@EnsureCode{10}{12}% ^^J
\TMP@EnsureCode{33}{12}% !
\TMP@EnsureCode{34}{12}% "
\TMP@EnsureCode{38}{4}% &
\TMP@EnsureCode{39}{12}% '
\TMP@EnsureCode{40}{12}% (
\TMP@EnsureCode{41}{12}% )
\TMP@EnsureCode{42}{12}% *
\TMP@EnsureCode{43}{12}% +
\TMP@EnsureCode{44}{12}% ,
\TMP@EnsureCode{45}{12}% -
\TMP@EnsureCode{46}{12}% .
\TMP@EnsureCode{47}{12}% /
\TMP@EnsureCode{58}{12}% :
\TMP@EnsureCode{60}{12}% <
\TMP@EnsureCode{62}{12}% >
\TMP@EnsureCode{91}{12}% [
\TMP@EnsureCode{93}{12}% ]
\TMP@EnsureCode{94}{7}% ^ (superscript)
\TMP@EnsureCode{95}{12}% _ (other)
\TMP@EnsureCode{96}{12}% `
\TMP@EnsureCode{126}{12}% ~ (other)
\edef\pdftexcmds@AtEnd{%
  \pdftexcmds@AtEnd
  \escapechar=\number\escapechar\relax
  \noexpand\endinput
}
\escapechar=92 %
%    \end{macrocode}
%
% \subsection{Load packages}
%
%    \begin{macrocode}
\begingroup\expandafter\expandafter\expandafter\endgroup
\expandafter\ifx\csname RequirePackage\endcsname\relax
  \def\TMP@RequirePackage#1[#2]{%
    \begingroup\expandafter\expandafter\expandafter\endgroup
    \expandafter\ifx\csname ver@#1.sty\endcsname\relax
      \input #1.sty\relax
    \fi
  }%
  \TMP@RequirePackage{infwarerr}[2007/09/09]%
  \TMP@RequirePackage{ifluatex}[2010/03/01]%
  \TMP@RequirePackage{ltxcmds}[2010/12/02]%
  \TMP@RequirePackage{ifpdf}[2010/09/13]%
\else
  \RequirePackage{infwarerr}[2007/09/09]%
  \RequirePackage{ifluatex}[2010/03/01]%
  \RequirePackage{ltxcmds}[2010/12/02]%
  \RequirePackage{ifpdf}[2010/09/13]%
\fi
%    \end{macrocode}
%
% \subsection{Without \hologo{LuaTeX}}
%
%    \begin{macrocode}
\ifluatex
\else
  \@PackageInfoNoLine{pdftexcmds}{LuaTeX not detected}%
  \def\pdftexcmds@nopdftex{%
    \@PackageInfoNoLine{pdftexcmds}{pdfTeX >= 1.30 not detected}%
    \let\pdftexcmds@nopdftex\relax
  }%
  \def\pdftexcmds@temp#1{%
    \begingroup\expandafter\expandafter\expandafter\endgroup
    \expandafter\ifx\csname pdf#1\endcsname\relax
      \pdftexcmds@nopdftex
    \else
      \expandafter\def\csname pdf@#1\expandafter\endcsname
      \expandafter##\expandafter{%
        \csname pdf#1\endcsname
      }%
    \fi
  }%
  \pdftexcmds@temp{strcmp}%
  \pdftexcmds@temp{escapehex}%
  \let\pdf@escapehexnative\pdf@escapehex
  \pdftexcmds@temp{unescapehex}%
  \let\pdf@unescapehexnative\pdf@unescapehex
  \pdftexcmds@temp{escapestring}%
  \pdftexcmds@temp{escapename}%
  \pdftexcmds@temp{filesize}%
  \pdftexcmds@temp{filemoddate}%
  \begingroup\expandafter\expandafter\expandafter\endgroup
  \expandafter\ifx\csname pdfshellescape\endcsname\relax
    \pdftexcmds@nopdftex
    \ltx@IfUndefined{pdftexversion}{%
    }{%
      \ifnum\pdftexversion>120 % 1.21a supports \ifeof18
        \ifeof18 %
          \chardef\pdf@shellescape=0 %
        \else
          \chardef\pdf@shellescape=1 %
        \fi
      \fi
    }%
  \else
    \def\pdf@shellescape{%
      \pdfshellescape
    }%
  \fi
  \begingroup\expandafter\expandafter\expandafter\endgroup
  \expandafter\ifx\csname pdffiledump\endcsname\relax
    \pdftexcmds@nopdftex
  \else
    \def\pdf@filedump#1#2#3{%
      \pdffiledump offset#1 length#2{#3}%
    }%
  \fi
%    \end{macrocode}
%    \begin{macrocode}
  \begingroup\expandafter\expandafter\expandafter\endgroup
  \expandafter\ifx\csname pdfmdfivesum\endcsname\relax
    \begingroup\expandafter\expandafter\expandafter\endgroup
    \expandafter\ifx\csname mdfivesum\endcsname\relax
      \pdftexcmds@nopdftex
    \else
      \def\pdf@mdfivesum#{\mdfivesum}%
      \let\pdf@mdfivesumnative\pdf@mdfivesum
      \def\pdf@filemdfivesum#{\mdfivesum file}%
    \fi
  \else
    \def\pdf@mdfivesum#{\pdfmdfivesum}%
    \let\pdf@mdfivesumnative\pdf@mdfivesum
    \def\pdf@filemdfivesum#{\pdfmdfivesum file}%
  \fi
%    \end{macrocode}
%    \begin{macrocode}
  \def\pdf@system#{%
    \immediate\write18%
  }%
  \def\pdftexcmds@temp#1{%
    \begingroup\expandafter\expandafter\expandafter\endgroup
    \expandafter\ifx\csname pdf#1\endcsname\relax
      \pdftexcmds@nopdftex
    \else
      \expandafter\let\csname pdf@#1\expandafter\endcsname
      \csname pdf#1\endcsname
    \fi
  }%
  \pdftexcmds@temp{resettimer}%
  \pdftexcmds@temp{elapsedtime}%
\fi
%    \end{macrocode}
%
% \subsection{\cs{pdf@primitive}, \cs{pdf@ifprimitive}}
%
%    Since version 1.40.0 \hologo{pdfTeX} has \cs{pdfprimitive} and
%    \cs{ifpdfprimitive}. And \cs{pdfprimitive} was fixed in
%    version 1.40.4.
%
%    \hologo{XeTeX} provides them under the name \cs{primitive} and
%    \cs{ifprimitive}. \hologo{LuaTeX} knows both name variants,
%    but they have possibly to be enabled first (|tex.enableprimitives|).
%
%    Depending on the format TeX Live uses a prefix |luatex|.
%
%    Caution: \cs{let} must be used for the definition of
%    the macros, especially because of \cs{ifpdfprimitive}.
%
% \subsubsection{Using \hologo{LuaTeX}'s \texttt{tex.enableprimitives}}
%
%    \begin{macrocode}
\ifluatex
%    \end{macrocode}
%    \begin{macro}{\pdftexcmds@directlua}
%    \begin{macrocode}
  \ifnum\luatexversion<36 %
    \def\pdftexcmds@directlua{\directlua0 }%
  \else
    \let\pdftexcmds@directlua\directlua
  \fi
%    \end{macrocode}
%    \end{macro}
%
%    \begin{macrocode}
  \begingroup
    \newlinechar=10 %
    \endlinechar=\newlinechar
    \pdftexcmds@directlua{%
      if tex.enableprimitives then
        tex.enableprimitives(
          'pdf@',
          {'primitive', 'ifprimitive', 'pdfdraftmode','draftmode'}
        )
        tex.enableprimitives('', {'luaescapestring'})
      end
    }%
  \endgroup %
%    \end{macrocode}
%
%    \begin{macrocode}
\fi
%    \end{macrocode}
%
% \subsubsection{Trying various names to find the primitives}
%
%    \begin{macro}{\pdftexcmds@strip@prefix}
%    \begin{macrocode}
\def\pdftexcmds@strip@prefix#1>{}
%    \end{macrocode}
%    \end{macro}
%    \begin{macrocode}
\def\pdftexcmds@temp#1#2#3{%
  \begingroup\expandafter\expandafter\expandafter\endgroup
  \expandafter\ifx\csname pdf@#1\endcsname\relax
    \begingroup
      \def\x{#3}%
      \edef\x{\expandafter\pdftexcmds@strip@prefix\meaning\x}%
      \escapechar=-1 %
      \edef\y{\expandafter\meaning\csname#2\endcsname}%
    \expandafter\endgroup
    \ifx\x\y
      \expandafter\let\csname pdf@#1\expandafter\endcsname
      \csname #2\endcsname
    \fi
  \fi
}
%    \end{macrocode}
%
%    \begin{macro}{\pdf@primitive}
%    \begin{macrocode}
\pdftexcmds@temp{primitive}{pdfprimitive}{pdfprimitive}% pdfTeX, oldLuaTeX
\pdftexcmds@temp{primitive}{primitive}{primitive}% XeTeX, luatex
\pdftexcmds@temp{primitive}{luatexprimitive}{pdfprimitive}% oldLuaTeX
\pdftexcmds@temp{primitive}{luatexpdfprimitive}{pdfprimitive}% oldLuaTeX
%    \end{macrocode}
%    \end{macro}
%    \begin{macro}{\pdf@ifprimitive}
%    \begin{macrocode}
\pdftexcmds@temp{ifprimitive}{ifpdfprimitive}{ifpdfprimitive}% pdfTeX, oldLuaTeX
\pdftexcmds@temp{ifprimitive}{ifprimitive}{ifprimitive}% XeTeX, luatex
\pdftexcmds@temp{ifprimitive}{luatexifprimitive}{ifpdfprimitive}% oldLuaTeX
\pdftexcmds@temp{ifprimitive}{luatexifpdfprimitive}{ifpdfprimitive}% oldLuaTeX
%    \end{macrocode}
%    \end{macro}
%
%    Disable broken \cs{pdfprimitive}.
%    \begin{macrocode}
\ifluatex\else
\begingroup
  \expandafter\ifx\csname pdf@primitive\endcsname\relax
  \else
    \expandafter\ifx\csname pdftexversion\endcsname\relax
    \else
      \ifnum\pdftexversion=140 %
        \expandafter\ifx\csname pdftexrevision\endcsname\relax
        \else
          \ifnum\pdftexrevision<4 %
            \endgroup
            \let\pdf@primitive\@undefined
            \@PackageInfoNoLine{pdftexcmds}{%
              \string\pdf@primitive\space disabled, %
              because\MessageBreak
              \string\pdfprimitive\space is broken until pdfTeX 1.40.4%
            }%
            \begingroup
          \fi
        \fi
      \fi
    \fi
  \fi
\endgroup
\fi
%    \end{macrocode}
%
% \subsubsection{Result}
%
%    \begin{macrocode}
\begingroup
  \@PackageInfoNoLine{pdftexcmds}{%
    \string\pdf@primitive\space is %
    \expandafter\ifx\csname pdf@primitive\endcsname\relax not \fi
    available%
  }%
  \@PackageInfoNoLine{pdftexcmds}{%
    \string\pdf@ifprimitive\space is %
    \expandafter\ifx\csname pdf@ifprimitive\endcsname\relax not \fi
    available%
  }%
\endgroup
%    \end{macrocode}
%
% \subsection{\hologo{XeTeX}}
%
%    Look for primitives \cs{shellescape}, \cs{strcmp}.
%    \begin{macrocode}
\def\pdftexcmds@temp#1{%
  \begingroup\expandafter\expandafter\expandafter\endgroup
  \expandafter\ifx\csname pdf@#1\endcsname\relax
    \begingroup
      \escapechar=-1 %
      \edef\x{\expandafter\meaning\csname#1\endcsname}%
      \def\y{#1}%
      \def\z##1->{}%
      \edef\y{\expandafter\z\meaning\y}%
    \expandafter\endgroup
    \ifx\x\y
      \expandafter\def\csname pdf@#1\expandafter\endcsname
      \expandafter{%
        \csname#1\endcsname
      }%
    \fi
  \fi
}%
\pdftexcmds@temp{shellescape}%
\pdftexcmds@temp{strcmp}%
%    \end{macrocode}
%
% \subsection{\cs{pdf@isprimitive}}
%
%    \begin{macrocode}
\def\pdf@isprimitive{%
  \begingroup\expandafter\expandafter\expandafter\endgroup
  \expandafter\ifx\csname pdf@strcmp\endcsname\relax
    \long\def\pdf@isprimitive##1{%
      \expandafter\pdftexcmds@isprimitive\expandafter{\meaning##1}%
    }%
    \long\def\pdftexcmds@isprimitive##1##2{%
      \expandafter\pdftexcmds@@isprimitive\expandafter{\string##2}{##1}%
    }%
    \def\pdftexcmds@@isprimitive##1##2{%
      \ifnum0\pdftexcmds@equal##1\delimiter##2\delimiter=1 %
        \expandafter\ltx@firstoftwo
      \else
        \expandafter\ltx@secondoftwo
      \fi
    }%
    \def\pdftexcmds@equal##1##2\delimiter##3##4\delimiter{%
      \ifx##1##3%
        \ifx\relax##2##4\relax
          1%
        \else
          \ifx\relax##2\relax
          \else
            \ifx\relax##4\relax
            \else
              \pdftexcmds@equalcont{##2}{##4}%
            \fi
          \fi
        \fi
      \fi
    }%
    \def\pdftexcmds@equalcont##1{%
      \def\pdftexcmds@equalcont####1####2##1##1##1##1{%
        ##1##1##1##1%
        \pdftexcmds@equal####1\delimiter####2\delimiter
      }%
    }%
    \expandafter\pdftexcmds@equalcont\csname fi\endcsname
  \else
    \long\def\pdf@isprimitive##1##2{%
      \ifnum\pdf@strcmp{\meaning##1}{\string##2}=0 %
        \expandafter\ltx@firstoftwo
      \else
        \expandafter\ltx@secondoftwo
      \fi
    }%
  \fi
}
\ifluatex
\ifx\pdfdraftmode\@undefined
  \let\pdfdraftmode\draftmode
\fi
\else
  \pdf@isprimitive
\fi
%    \end{macrocode}
%
% \subsection{\cs{pdf@draftmode}}
%
%
%    \begin{macrocode}
\let\pdftexcmds@temp\ltx@zero %
\ltx@IfUndefined{pdfdraftmode}{%
  \@PackageInfoNoLine{pdftexcmds}{\ltx@backslashchar pdfdraftmode not found}%
}{%
  \ifpdf
    \let\pdftexcmds@temp\ltx@one
    \@PackageInfoNoLine{pdftexcmds}{\ltx@backslashchar pdfdraftmode found}%
  \else
    \@PackageInfoNoLine{pdftexcmds}{%
      \ltx@backslashchar pdfdraftmode is ignored in DVI mode%
    }%
  \fi
}
\ifcase\pdftexcmds@temp
%    \end{macrocode}
%    \begin{macro}{\pdf@draftmode}
%    \begin{macrocode}
  \let\pdf@draftmode\ltx@zero
%    \end{macrocode}
%    \end{macro}
%    \begin{macro}{\pdf@ifdraftmode}
%    \begin{macrocode}
  \let\pdf@ifdraftmode\ltx@secondoftwo
%    \end{macrocode}
%    \end{macro}
%    \begin{macro}{\pdftexcmds@setdraftmode}
%    \begin{macrocode}
  \def\pdftexcmds@setdraftmode#1{}%
%    \end{macrocode}
%    \end{macro}
%    \begin{macrocode}
\else
%    \end{macrocode}
%    \begin{macro}{\pdftexcmds@draftmode}
%    \begin{macrocode}
  \let\pdftexcmds@draftmode\pdfdraftmode
%    \end{macrocode}
%    \end{macro}
%    \begin{macro}{\pdf@ifdraftmode}
%    \begin{macrocode}
  \def\pdf@ifdraftmode{%
    \ifnum\pdftexcmds@draftmode=\ltx@one
      \expandafter\ltx@firstoftwo
    \else
      \expandafter\ltx@secondoftwo
    \fi
  }%
%    \end{macrocode}
%    \end{macro}
%    \begin{macro}{\pdf@draftmode}
%    \begin{macrocode}
  \def\pdf@draftmode{%
    \ifnum\pdftexcmds@draftmode=\ltx@one
      \expandafter\ltx@one
    \else
      \expandafter\ltx@zero
    \fi
  }%
%    \end{macrocode}
%    \end{macro}
%    \begin{macro}{\pdftexcmds@setdraftmode}
%    \begin{macrocode}
  \def\pdftexcmds@setdraftmode#1{%
    \pdftexcmds@draftmode=#1\relax
  }%
%    \end{macrocode}
%    \end{macro}
%    \begin{macrocode}
\fi
%    \end{macrocode}
%    \begin{macro}{\pdf@setdraftmode}
%    \begin{macrocode}
\def\pdf@setdraftmode#1{%
  \begingroup
    \count\ltx@cclv=#1\relax
  \edef\x{\endgroup
    \noexpand\pdftexcmds@@setdraftmode{\the\count\ltx@cclv}%
  }%
  \x
}
%    \end{macrocode}
%    \end{macro}
%    \begin{macro}{\pdftexcmds@@setdraftmode}
%    \begin{macrocode}
\def\pdftexcmds@@setdraftmode#1{%
  \ifcase#1 %
    \pdftexcmds@setdraftmode{#1}%
  \or
    \pdftexcmds@setdraftmode{#1}%
  \else
    \@PackageWarning{pdftexcmds}{%
      \string\pdf@setdraftmode: Ignoring\MessageBreak
      invalid value `#1'%
    }%
  \fi
}
%    \end{macrocode}
%    \end{macro}
%
% \subsection{Load Lua module}
%
%    \begin{macrocode}
\ifluatex
\else
  \expandafter\pdftexcmds@AtEnd
\fi%
%    \end{macrocode}
%
%    \begin{macrocode}
\ifnum\luatexversion<80
  \begingroup\expandafter\expandafter\expandafter\endgroup
  \expandafter\ifx\csname RequirePackage\endcsname\relax
    \def\TMP@RequirePackage#1[#2]{%
      \begingroup\expandafter\expandafter\expandafter\endgroup
      \expandafter\ifx\csname ver@#1.sty\endcsname\relax
        \input #1.sty\relax
      \fi
    }%
    \TMP@RequirePackage{luatex-loader}[2009/04/10]%
  \else
    \RequirePackage{luatex-loader}[2009/04/10]%
  \fi
\fi
\pdftexcmds@directlua{%
  require("pdftexcmds")%
}
\ifnum\luatexversion>37 %
  \ifnum0%
      \pdftexcmds@directlua{%
        if status.ini_version then %
          tex.write("1")%
        end%
      }>0 %
    \everyjob\expandafter{%
      \the\everyjob
      \pdftexcmds@directlua{%
        require("pdftexcmds")%
      }%
    }%
  \fi
\fi
\begingroup
  \def\x{2019/07/25 v0.30}%
  \ltx@onelevel@sanitize\x
  \edef\y{%
    \pdftexcmds@directlua{%
      if oberdiek.pdftexcmds.getversion then %
        oberdiek.pdftexcmds.getversion()%
      end%
    }%
  }%
  \ifx\x\y
  \else
    \@PackageError{pdftexcmds}{%
      Wrong version of lua module.\MessageBreak
      Package version: \x\MessageBreak
      Lua module: \y
    }\@ehc
  \fi
\endgroup
%    \end{macrocode}
%
% \subsection{Lua functions}
%
% \subsubsection{Helper macros}
%
%    \begin{macro}{\pdftexcmds@toks}
%    \begin{macrocode}
\begingroup\expandafter\expandafter\expandafter\endgroup
\expandafter\ifx\csname newtoks\endcsname\relax
  \toksdef\pdftexcmds@toks=0 %
\else
  \csname newtoks\endcsname\pdftexcmds@toks
\fi
%    \end{macrocode}
%    \end{macro}
%
%    \begin{macro}{\pdftexcmds@Patch}
%    \begin{macrocode}
\def\pdftexcmds@Patch{0}
\ifnum\luatexversion>40 %
  \ifnum\luatexversion<66 %
    \def\pdftexcmds@Patch{1}%
  \fi
\fi
%    \end{macrocode}
%    \end{macro}
%    \begin{macrocode}
\ifcase\pdftexcmds@Patch
  \catcode`\&=14 %
\else
  \catcode`\&=9 %
%    \end{macrocode}
%    \begin{macro}{\pdftexcmds@PatchDecode}
%    \begin{macrocode}
  \def\pdftexcmds@PatchDecode#1\@nil{%
    \pdftexcmds@DecodeA#1^^A^^A\@nil{}%
  }%
%    \end{macrocode}
%    \end{macro}
%    \begin{macro}{\pdftexcmds@DecodeA}
%    \begin{macrocode}
  \def\pdftexcmds@DecodeA#1^^A^^A#2\@nil#3{%
    \ifx\relax#2\relax
      \ltx@ReturnAfterElseFi{%
        \pdftexcmds@DecodeB#3#1^^A^^B\@nil{}%
      }%
    \else
      \ltx@ReturnAfterFi{%
        \pdftexcmds@DecodeA#2\@nil{#3#1^^@}%
      }%
    \fi
  }%
%    \end{macrocode}
%    \end{macro}
%    \begin{macro}{\pdftexcmds@DecodeB}
%    \begin{macrocode}
  \def\pdftexcmds@DecodeB#1^^A^^B#2\@nil#3{%
    \ifx\relax#2\relax%
      \ltx@ReturnAfterElseFi{%
        \ltx@zero
        #3#1%
      }%
    \else
      \ltx@ReturnAfterFi{%
        \pdftexcmds@DecodeB#2\@nil{#3#1^^A}%
      }%
    \fi
  }%
%    \end{macrocode}
%    \end{macro}
%    \begin{macrocode}
\fi
%    \end{macrocode}
%
%    \begin{macrocode}
\ifnum\luatexversion<36 %
\else
  \catcode`\0=9 %
\fi
%    \end{macrocode}
%
% \subsubsection[Strings]{Strings \cite[``7.15 Strings'']{pdftex-manual}}
%
%    \begin{macro}{\pdf@strcmp}
%    \begin{macrocode}
\long\def\pdf@strcmp#1#2{%
  \directlua0{%
    oberdiek.pdftexcmds.strcmp("\luaescapestring{#1}",%
        "\luaescapestring{#2}")%
  }%
}%
%    \end{macrocode}
%    \end{macro}
%    \begin{macrocode}
\pdf@isprimitive
%    \end{macrocode}
%    \begin{macro}{\pdf@escapehex}
%    \begin{macrocode}
\long\def\pdf@escapehex#1{%
  \directlua0{%
    oberdiek.pdftexcmds.escapehex("\luaescapestring{#1}", "byte")%
  }%
}%
%    \end{macrocode}
%    \end{macro}
%    \begin{macro}{\pdf@escapehexnative}
%    \begin{macrocode}
\long\def\pdf@escapehexnative#1{%
  \directlua0{%
    oberdiek.pdftexcmds.escapehex("\luaescapestring{#1}")%
  }%
}%
%    \end{macrocode}
%    \end{macro}
%    \begin{macro}{\pdf@unescapehex}
%    \begin{macrocode}
\def\pdf@unescapehex#1{%
& \romannumeral\expandafter\pdftexcmds@PatchDecode
  \the\expandafter\pdftexcmds@toks
  \directlua0{%
    oberdiek.pdftexcmds.toks="pdftexcmds@toks"%
    oberdiek.pdftexcmds.unescapehex("\luaescapestring{#1}", "byte", \pdftexcmds@Patch)%
  }%
& \@nil
}%
%    \end{macrocode}
%    \end{macro}
%    \begin{macro}{\pdf@unescapehexnative}
%    \begin{macrocode}
\def\pdf@unescapehexnative#1{%
& \romannumeral\expandafter\pdftexcmds@PatchDecode
  \the\expandafter\pdftexcmds@toks
  \directlua0{%
    oberdiek.pdftexcmds.toks="pdftexcmds@toks"%
    oberdiek.pdftexcmds.unescapehex("\luaescapestring{#1}", \pdftexcmds@Patch)%
  }%
& \@nil
}%
%    \end{macrocode}
%    \end{macro}
%    \begin{macro}{\pdf@escapestring}
%    \begin{macrocode}
\long\def\pdf@escapestring#1{%
  \directlua0{%
    oberdiek.pdftexcmds.escapestring("\luaescapestring{#1}", "byte")%
  }%
}
%    \end{macrocode}
%    \end{macro}
%    \begin{macro}{\pdf@escapename}
%    \begin{macrocode}
\long\def\pdf@escapename#1{%
  \directlua0{%
    oberdiek.pdftexcmds.escapename("\luaescapestring{#1}", "byte")%
  }%
}
%    \end{macrocode}
%    \end{macro}
%    \begin{macro}{\pdf@escapenamenative}
%    \begin{macrocode}
\long\def\pdf@escapenamenative#1{%
  \directlua0{%
    oberdiek.pdftexcmds.escapename("\luaescapestring{#1}")%
  }%
}
%    \end{macrocode}
%    \end{macro}
%
% \subsubsection[Files]{Files \cite[``7.18 Files'']{pdftex-manual}}
%
%    \begin{macro}{\pdf@filesize}
%    \begin{macrocode}
\def\pdf@filesize#1{%
  \directlua0{%
    oberdiek.pdftexcmds.filesize("\luaescapestring{#1}")%
  }%
}
%    \end{macrocode}
%    \end{macro}
%    \begin{macro}{\pdf@filemoddate}
%    \begin{macrocode}
\def\pdf@filemoddate#1{%
  \directlua0{%
    oberdiek.pdftexcmds.filemoddate("\luaescapestring{#1}")%
  }%
}
%    \end{macrocode}
%    \end{macro}
%    \begin{macro}{\pdf@filedump}
%    \begin{macrocode}
\def\pdf@filedump#1#2#3{%
  \directlua0{%
    oberdiek.pdftexcmds.filedump("\luaescapestring{\number#1}",%
        "\luaescapestring{\number#2}",%
        "\luaescapestring{#3}")%
  }%
}%
%    \end{macrocode}
%    \end{macro}
%    \begin{macro}{\pdf@mdfivesum}
%    \begin{macrocode}
\long\def\pdf@mdfivesum#1{%
  \directlua0{%
    oberdiek.pdftexcmds.mdfivesum("\luaescapestring{#1}", "byte")%
  }%
}%
%    \end{macrocode}
%    \end{macro}
%    \begin{macro}{\pdf@mdfivesumnative}
%    \begin{macrocode}
\long\def\pdf@mdfivesumnative#1{%
  \directlua0{%
    oberdiek.pdftexcmds.mdfivesum("\luaescapestring{#1}")%
  }%
}%
%    \end{macrocode}
%    \end{macro}
%    \begin{macro}{\pdf@filemdfivesum}
%    \begin{macrocode}
\def\pdf@filemdfivesum#1{%
  \directlua0{%
    oberdiek.pdftexcmds.filemdfivesum("\luaescapestring{#1}")%
  }%
}%
%    \end{macrocode}
%    \end{macro}
%
% \subsubsection[Timekeeping]{Timekeeping \cite[``7.17 Timekeeping'']{pdftex-manual}}
%
%    \begin{macro}{\protected}
%    \begin{macrocode}
\let\pdftexcmds@temp=Y%
\begingroup\expandafter\expandafter\expandafter\endgroup
\expandafter\ifx\csname protected\endcsname\relax
  \pdftexcmds@directlua0{%
    if tex.enableprimitives then %
      tex.enableprimitives('', {'protected'})%
    end%
  }%
\fi
\begingroup\expandafter\expandafter\expandafter\endgroup
\expandafter\ifx\csname protected\endcsname\relax
  \let\pdftexcmds@temp=N%
\fi
%    \end{macrocode}
%    \end{macro}
%    \begin{macro}{\numexpr}
%    \begin{macrocode}
\begingroup\expandafter\expandafter\expandafter\endgroup
\expandafter\ifx\csname numexpr\endcsname\relax
  \pdftexcmds@directlua0{%
    if tex.enableprimitives then %
      tex.enableprimitives('', {'numexpr'})%
    end%
  }%
\fi
\begingroup\expandafter\expandafter\expandafter\endgroup
\expandafter\ifx\csname numexpr\endcsname\relax
  \let\pdftexcmds@temp=N%
\fi
%    \end{macrocode}
%    \end{macro}
%
%    \begin{macrocode}
\ifx\pdftexcmds@temp N%
  \@PackageWarningNoLine{pdftexcmds}{%
    Definitions of \ltx@backslashchar pdf@resettimer and%
    \MessageBreak
    \ltx@backslashchar pdf@elapsedtime are skipped, because%
    \MessageBreak
    e-TeX's \ltx@backslashchar protected or %
    \ltx@backslashchar numexpr are missing%
  }%
\else
%    \end{macrocode}
%
%    \begin{macro}{\pdf@resettimer}
%    \begin{macrocode}
  \protected\def\pdf@resettimer{%
    \pdftexcmds@directlua0{%
      oberdiek.pdftexcmds.resettimer()%
    }%
  }%
%    \end{macrocode}
%    \end{macro}
%
%    \begin{macro}{\pdf@elapsedtime}
%    \begin{macrocode}
  \protected\def\pdf@elapsedtime{%
    \numexpr
      \pdftexcmds@directlua0{%
        oberdiek.pdftexcmds.elapsedtime()%
      }%
    \relax
  }%
%    \end{macrocode}
%    \end{macro}
%    \begin{macrocode}
\fi
%    \end{macrocode}
%
% \subsubsection{Shell escape}
%
%    \begin{macro}{\pdf@shellescape}
%
%    \begin{macrocode}
\ifnum\luatexversion<68 %
\else
  \protected\edef\pdf@shellescape{%
   \numexpr\directlua{tex.sprint(%
         \number\catcodetable@string,status.shell_escape)}\relax}
\fi
%    \end{macrocode}
%    \end{macro}
%
%    \begin{macro}{\pdf@system}
%    \begin{macrocode}
\def\pdf@system#1{%
  \directlua0{%
    oberdiek.pdftexcmds.system("\luaescapestring{#1}")%
  }%
}
%    \end{macrocode}
%    \end{macro}
%
%    \begin{macro}{\pdf@lastsystemstatus}
%    \begin{macrocode}
\def\pdf@lastsystemstatus{%
  \directlua0{%
    oberdiek.pdftexcmds.lastsystemstatus()%
  }%
}
%    \end{macrocode}
%    \end{macro}
%    \begin{macro}{\pdf@lastsystemexit}
%    \begin{macrocode}
\def\pdf@lastsystemexit{%
  \directlua0{%
    oberdiek.pdftexcmds.lastsystemexit()%
  }%
}
%    \end{macrocode}
%    \end{macro}
%
%    \begin{macrocode}
\catcode`\0=12 %
%    \end{macrocode}
%
%    \begin{macro}{\pdf@pipe}
%    Check availability of |io.popen| first.
%    \begin{macrocode}
\ifnum0%
    \pdftexcmds@directlua{%
      if io.popen then %
        tex.write("1")%
      end%
    }%
    =1 %
  \def\pdf@pipe#1{%
&   \romannumeral\expandafter\pdftexcmds@PatchDecode
    \the\expandafter\pdftexcmds@toks
    \pdftexcmds@directlua{%
      oberdiek.pdftexcmds.toks="pdftexcmds@toks"%
      oberdiek.pdftexcmds.pipe("\luaescapestring{#1}", \pdftexcmds@Patch)%
    }%
&   \@nil
  }%
\fi
%    \end{macrocode}
%    \end{macro}
%
%    \begin{macrocode}
\pdftexcmds@AtEnd%
%</package>
%    \end{macrocode}
%
% \subsection{Lua module}
%
%    \begin{macrocode}
%<*lua>
%    \end{macrocode}
%
%    \begin{macrocode}
oberdiek = oberdiek or {}
local pdftexcmds = oberdiek.pdftexcmds or {}
oberdiek.pdftexcmds = pdftexcmds
local systemexitstatus
function pdftexcmds.getversion()
  tex.write("2019/07/25 v0.30")
end
%    \end{macrocode}
%
% \subsubsection[Strings]{Strings \cite[``7.15 Strings'']{pdftex-manual}}
%
%    \begin{macrocode}
function pdftexcmds.strcmp(A, B)
  if A == B then
    tex.write("0")
  elseif A < B then
    tex.write("-1")
  else
    tex.write("1")
  end
end
local function utf8_to_byte(str)
  local i = 0
  local n = string.len(str)
  local t = {}
  while i < n do
    i = i + 1
    local a = string.byte(str, i)
    if a < 128 then
      table.insert(t, string.char(a))
    else
      if a >= 192 and i < n then
        i = i + 1
        local b = string.byte(str, i)
        if b < 128 or b >= 192 then
          i = i - 1
        elseif a == 194 then
          table.insert(t, string.char(b))
        elseif a == 195 then
          table.insert(t, string.char(b + 64))
        end
      end
    end
  end
  return table.concat(t)
end
function pdftexcmds.escapehex(str, mode)
  if mode == "byte" then
    str = utf8_to_byte(str)
  end
  tex.write((string.gsub(str, ".",
    function (ch)
      return string.format("%02X", string.byte(ch))
    end
  )))
end
%    \end{macrocode}
%    See procedure |unescapehex| in file \xfile{utils.c} of \hologo{pdfTeX}.
%    Caution: |tex.write| ignores leading spaces.
%    \begin{macrocode}
function pdftexcmds.unescapehex(str, mode, patch)
  local a = 0
  local first = true
  local result = {}
  for i = 1, string.len(str), 1 do
    local ch = string.byte(str, i)
    if ch >= 48 and ch <= 57 then
      ch = ch - 48
    elseif ch >= 65 and ch <= 70 then
      ch = ch - 55
    elseif ch >= 97 and ch <= 102 then
      ch = ch - 87
    else
      ch = nil
    end
    if ch then
      if first then
        a = ch * 16
        first = false
      else
        table.insert(result, a + ch)
        first = true
      end
    end
  end
  if not first then
    table.insert(result, a)
  end
  if patch == 1 then
    local temp = {}
    for i, a in ipairs(result) do
      if a == 0 then
        table.insert(temp, 1)
        table.insert(temp, 1)
      else
        if a == 1 then
          table.insert(temp, 1)
          table.insert(temp, 2)
        else
          table.insert(temp, a)
        end
      end
    end
    result = temp
  end
  if mode == "byte" then
    local utf8 = {}
    for i, a in ipairs(result) do
      if a < 128 then
        table.insert(utf8, a)
      else
        if a < 192 then
          table.insert(utf8, 194)
          a = a - 128
        else
          table.insert(utf8, 195)
          a = a - 192
        end
        table.insert(utf8, a + 128)
      end
    end
    result = utf8
  end
%    \end{macrocode}
%    this next line added for current luatex; this is the only
%    change in the file.  eroux, 28apr13. (v 0.21)
%    \begin{macrocode}
  local unpack = _G["unpack"] or table.unpack
  tex.settoks(pdftexcmds.toks, string.char(unpack(result)))
end
%    \end{macrocode}
%    See procedure |escapestring| in file \xfile{utils.c} of \hologo{pdfTeX}.
%    \begin{macrocode}
function pdftexcmds.escapestring(str, mode)
  if mode == "byte" then
    str = utf8_to_byte(str)
  end
  tex.write((string.gsub(str, ".",
    function (ch)
      local b = string.byte(ch)
      if b < 33 or b > 126 then
        return string.format("\\%.3o", b)
      end
      if b == 40 or b == 41 or b == 92 then
        return "\\" .. ch
      end
%    \end{macrocode}
%    Lua 5.1 returns the match in case of return value |nil|.
%    \begin{macrocode}
      return nil
    end
  )))
end
%    \end{macrocode}
%    See procedure |escapename| in file \xfile{utils.c} of \hologo{pdfTeX}.
%    \begin{macrocode}
function pdftexcmds.escapename(str, mode)
  if mode == "byte" then
    str = utf8_to_byte(str)
  end
  tex.write((string.gsub(str, ".",
    function (ch)
      local b = string.byte(ch)
      if b == 0 then
%    \end{macrocode}
%    In Lua 5.0 |nil| could be used for the empty string,
%    But |nil| returns the match in Lua 5.1, thus we use
%    the empty string explicitly.
%    \begin{macrocode}
        return ""
      end
      if b <= 32 or b >= 127
          or b == 35 or b == 37 or b == 40 or b == 41
          or b == 47 or b == 60 or b == 62 or b == 91
          or b == 93 or b == 123 or b == 125 then
        return string.format("#%.2X", b)
      else
%    \end{macrocode}
%    Lua 5.1 returns the match in case of return value |nil|.
%    \begin{macrocode}
        return nil
      end
    end
  )))
end
%    \end{macrocode}
%
% \subsubsection[Files]{Files \cite[``7.18 Files'']{pdftex-manual}}
%
%    \begin{macrocode}
function pdftexcmds.filesize(filename)
  local foundfile = kpse.find_file(filename, "tex", true)
  if foundfile then
    local size = lfs.attributes(foundfile, "size")
    if size then
      tex.write(size)
    end
  end
end
%    \end{macrocode}
%    See procedure |makepdftime| in file \xfile{utils.c} of \hologo{pdfTeX}.
%    \begin{macrocode}
function pdftexcmds.filemoddate(filename)
  local foundfile = kpse.find_file(filename, "tex", true)
  if foundfile then
    local date = lfs.attributes(foundfile, "modification")
    if date then
      local d = os.date("*t", date)
      if d.sec >= 60 then
        d.sec = 59
      end
      local u = os.date("!*t", date)
      local off = 60 * (d.hour - u.hour) + d.min - u.min
      if d.year ~= u.year then
        if d.year > u.year then
          off = off + 1440
        else
          off = off - 1440
        end
      elseif d.yday ~= u.yday then
        if d.yday > u.yday then
          off = off + 1440
        else
          off = off - 1440
        end
      end
      local timezone
      if off == 0 then
        timezone = "Z"
      else
        local hours = math.floor(off / 60)
        local mins = math.abs(off - hours * 60)
        timezone = string.format("%+03d'%02d'", hours, mins)
      end
      tex.write(string.format("D:%04d%02d%02d%02d%02d%02d%s",
          d.year, d.month, d.day, d.hour, d.min, d.sec, timezone))
    end
  end
end
function pdftexcmds.filedump(offset, length, filename)
  length = tonumber(length)
  if length and length > 0 then
    local foundfile = kpse.find_file(filename, "tex", true)
    if foundfile then
      offset = tonumber(offset)
      if not offset then
        offset = 0
      end
      local filehandle = io.open(foundfile, "rb")
      if filehandle then
        if offset > 0 then
          filehandle:seek("set", offset)
        end
        local dump = filehandle:read(length)
        pdftexcmds.escapehex(dump)
        filehandle:close()
      end
    end
  end
end
function pdftexcmds.mdfivesum(str, mode)
  if mode == "byte" then
    str = utf8_to_byte(str)
  end
  pdftexcmds.escapehex(md5.sum(str))
end
function pdftexcmds.filemdfivesum(filename)
  local foundfile = kpse.find_file(filename, "tex", true)
  if foundfile then
    local filehandle = io.open(foundfile, "rb")
    if filehandle then
      local contents = filehandle:read("*a")
      pdftexcmds.escapehex(md5.sum(contents))
      filehandle:close()
    end
  end
end
%    \end{macrocode}
%
% \subsubsection[Timekeeping]{Timekeeping \cite[``7.17 Timekeeping'']{pdftex-manual}}
%
%    The functions for timekeeping are based on
%    Andy Thomas' work \cite{AndyThomas:Analog}.
%    Changes:
%    \begin{itemize}
%    \item Overflow check is added.
%    \item |string.format| is used to avoid exponential number
%          representation for sure.
%    \item |tex.write| is used instead of |tex.print| to get
%          tokens with catcode 12 and without appended \cs{endlinechar}.
%    \end{itemize}
%    \begin{macrocode}
local basetime = 0
function pdftexcmds.resettimer()
  basetime = os.clock()
end
function pdftexcmds.elapsedtime()
  local val = (os.clock() - basetime) * 65536 + .5
  if val > 2147483647 then
    val = 2147483647
  end
  tex.write(string.format("%d", val))
end
%    \end{macrocode}
%
% \subsubsection[Miscellaneous]{Miscellaneous \cite[``7.21 Miscellaneous'']{pdftex-manual}}
%
%    \begin{macrocode}
function pdftexcmds.shellescape()
  if os.execute then
    if status
        and status.luatex_version
        and status.luatex_version >= 68 then
      tex.write(os.execute())
    else
      local result = os.execute()
      if result == 0 then
        tex.write("0")
      else
        if result == nil then
          tex.write("0")
        else
          tex.write("1")
        end
      end
    end
  else
    tex.write("0")
  end
end
function pdftexcmds.system(cmdline)
  systemexitstatus = nil
  texio.write_nl("log", "system(" .. cmdline .. ") ")
  if os.execute then
    texio.write("log", "executed.")
    systemexitstatus = os.execute(cmdline)
  else
    texio.write("log", "disabled.")
  end
end
function pdftexcmds.lastsystemstatus()
  local result = tonumber(systemexitstatus)
  if result then
    local x = math.floor(result / 256)
    tex.write(result - 256 * math.floor(result / 256))
  end
end
function pdftexcmds.lastsystemexit()
  local result = tonumber(systemexitstatus)
  if result then
    tex.write(math.floor(result / 256))
  end
end
function pdftexcmds.pipe(cmdline, patch)
  local result
  systemexitstatus = nil
  texio.write_nl("log", "pipe(" .. cmdline ..") ")
  if io.popen then
    texio.write("log", "executed.")
    local handle = io.popen(cmdline, "r")
    if handle then
      result = handle:read("*a")
      handle:close()
    end
  else
    texio.write("log", "disabled.")
  end
  if result then
    if patch == 1 then
      local temp = {}
      for i, a in ipairs(result) do
        if a == 0 then
          table.insert(temp, 1)
          table.insert(temp, 1)
        else
          if a == 1 then
            table.insert(temp, 1)
            table.insert(temp, 2)
          else
            table.insert(temp, a)
          end
        end
      end
      result = temp
    end
    tex.settoks(pdftexcmds.toks, result)
  else
    tex.settoks(pdftexcmds.toks, "")
  end
end
%    \end{macrocode}
%    \begin{macrocode}
%</lua>
%    \end{macrocode}
%
% \section{Test}
%
% \subsection{Catcode checks for loading}
%
%    \begin{macrocode}
%<*test1>
%    \end{macrocode}
%    \begin{macrocode}
\catcode`\{=1 %
\catcode`\}=2 %
\catcode`\#=6 %
\catcode`\@=11 %
\expandafter\ifx\csname count@\endcsname\relax
  \countdef\count@=255 %
\fi
\expandafter\ifx\csname @gobble\endcsname\relax
  \long\def\@gobble#1{}%
\fi
\expandafter\ifx\csname @firstofone\endcsname\relax
  \long\def\@firstofone#1{#1}%
\fi
\expandafter\ifx\csname loop\endcsname\relax
  \expandafter\@firstofone
\else
  \expandafter\@gobble
\fi
{%
  \def\loop#1\repeat{%
    \def\body{#1}%
    \iterate
  }%
  \def\iterate{%
    \body
      \let\next\iterate
    \else
      \let\next\relax
    \fi
    \next
  }%
  \let\repeat=\fi
}%
\def\RestoreCatcodes{}
\count@=0 %
\loop
  \edef\RestoreCatcodes{%
    \RestoreCatcodes
    \catcode\the\count@=\the\catcode\count@\relax
  }%
\ifnum\count@<255 %
  \advance\count@ 1 %
\repeat

\def\RangeCatcodeInvalid#1#2{%
  \count@=#1\relax
  \loop
    \catcode\count@=15 %
  \ifnum\count@<#2\relax
    \advance\count@ 1 %
  \repeat
}
\def\RangeCatcodeCheck#1#2#3{%
  \count@=#1\relax
  \loop
    \ifnum#3=\catcode\count@
    \else
      \errmessage{%
        Character \the\count@\space
        with wrong catcode \the\catcode\count@\space
        instead of \number#3%
      }%
    \fi
  \ifnum\count@<#2\relax
    \advance\count@ 1 %
  \repeat
}
\def\space{ }
\expandafter\ifx\csname LoadCommand\endcsname\relax
  \def\LoadCommand{\input pdftexcmds.sty\relax}%
\fi
\def\Test{%
  \RangeCatcodeInvalid{0}{47}%
  \RangeCatcodeInvalid{58}{64}%
  \RangeCatcodeInvalid{91}{96}%
  \RangeCatcodeInvalid{123}{255}%
  \catcode`\@=12 %
  \catcode`\\=0 %
  \catcode`\%=14 %
  \LoadCommand
  \RangeCatcodeCheck{0}{36}{15}%
  \RangeCatcodeCheck{37}{37}{14}%
  \RangeCatcodeCheck{38}{47}{15}%
  \RangeCatcodeCheck{48}{57}{12}%
  \RangeCatcodeCheck{58}{63}{15}%
  \RangeCatcodeCheck{64}{64}{12}%
  \RangeCatcodeCheck{65}{90}{11}%
  \RangeCatcodeCheck{91}{91}{15}%
  \RangeCatcodeCheck{92}{92}{0}%
  \RangeCatcodeCheck{93}{96}{15}%
  \RangeCatcodeCheck{97}{122}{11}%
  \RangeCatcodeCheck{123}{255}{15}%
  \RestoreCatcodes
}
\Test
\csname @@end\endcsname
\end
%    \end{macrocode}
%    \begin{macrocode}
%</test1>
%    \end{macrocode}
%
% \subsection{Test for \cs{pdf@isprimitive}}
%
%    \begin{macrocode}
%<*test2>
\catcode`\{=1 %
\catcode`\}=2 %
\catcode`\#=6 %
\catcode`\@=11 %
\input pdftexcmds.sty\relax
\def\msg#1{%
  \begingroup
    \escapechar=92 %
    \immediate\write16{#1}%
  \endgroup
}
\long\def\test#1#2#3#4{%
  \begingroup
    #4%
    \def\str{%
      Test \string\pdf@isprimitive
      {\string #1}{\string #2}{...}: %
    }%
    \pdf@isprimitive{#1}{#2}{%
      \ifx#3Y%
        \msg{\str true ==> OK.}%
      \else
        \errmessage{\str false ==> FAILED}%
      \fi
    }{%
      \ifx#3Y%
        \errmessage{\str true ==> FAILED}%
      \else
        \msg{\str false ==> OK.}%
      \fi
    }%
  \endgroup
}
\test\relax\relax Y{}
\test\foobar\relax Y{\let\foobar\relax}
\test\foobar\relax N{}
\test\hbox\hbox Y{}
\test\foobar@hbox\hbox Y{\let\foobar@hbox\hbox}
\test\if\if Y{}
\test\if\ifx N{}
\test\ifx\if N{}
\test\par\par Y{}
\test\hbox\par N{}
\test\par\hbox N{}
\csname @@end\endcsname\end
%</test2>
%    \end{macrocode}
%
% \subsection{Test for \cs{pdf@shellescape}}
%
%    \begin{macrocode}
%<*test-shell>
\catcode`\{=1 %
\catcode`\}=2 %
\catcode`\#=6 %
\catcode`\@=11 %
\input pdftexcmds.sty\relax
\def\msg#{\immediate\write16}
\def\MaybeEnd{}
\ifx\luatexversion\UnDeFiNeD
\else
  \ifnum\luatexversion<68 %
    \ifx\pdf@shellescape\@undefined
      \msg{SHELL=U}%
      \msg{OK (LuaTeX < 0.68)}%
    \else
      \msg{SHELL=defined}%
      \errmessage{Failed (LuaTeX < 0.68)}%
    \fi
    \def\MaybeEnd{\csname @@end\endcsname\end}%
  \fi
\fi
\MaybeEnd
\ifx\pdf@shellescape\@undefined
  \msg{SHELL=U}%
\else
  \msg{SHELL=\number\pdf@shellescape}%
\fi
\ifx\expected\@undefined
\else
  \ifx\expected\relax
    \msg{EXPECTED=U}%
    \ifx\pdf@shellescape\@undefined
      \msg{OK}%
    \else
      \errmessage{Failed}%
    \fi
  \else
    \msg{EXPECTED=\number\expected}%
    \ifnum\pdf@shellescape=\expected\relax
      \msg{OK}%
    \else
      \errmessage{Failed}%
    \fi
  \fi
\fi
\csname @@end\endcsname\end
%</test-shell>
%    \end{macrocode}
%
% \subsection{Test for escape functions}
%
%    \begin{macrocode}
%<*test-escape>
\catcode`\{=1 %
\catcode`\}=2 %
\catcode`\#=6 %
\catcode`\^=7 %
\catcode`\@=11 %
\errorcontextlines=1000 %
\input pdftexcmds.sty\relax
\def\msg#1{%
  \begingroup
    \escapechar=92 %
    \immediate\write16{#1}%
  \endgroup
}
%    \end{macrocode}
%    \begin{macrocode}
\begingroup
  \catcode`\@=11 %
  \countdef\count@=255 %
  \def\space{ }%
  \long\def\@whilenum#1\do #2{%
    \ifnum #1\relax
      #2\relax
      \@iwhilenum{#1\relax#2\relax}%
    \fi
  }%
  \long\def\@iwhilenum#1{%
    \ifnum #1%
      \expandafter\@iwhilenum
    \else
      \expandafter\ltx@gobble
    \fi
    {#1}%
  }%
  \gdef\AllBytes{}%
  \count@=0 %
  \catcode0=12 %
  \@whilenum\count@<256 \do{%
    \lccode0=\count@
    \ifnum\count@=32 %
      \xdef\AllBytes{\AllBytes\space}%
    \else
      \lowercase{%
        \xdef\AllBytes{\AllBytes^^@}%
      }%
    \fi
    \advance\count@ by 1 %
  }%
\endgroup
%    \end{macrocode}
%    \begin{macrocode}
\def\AllBytesHex{%
  000102030405060708090A0B0C0D0E0F%
  101112131415161718191A1B1C1D1E1F%
  202122232425262728292A2B2C2D2E2F%
  303132333435363738393A3B3C3D3E3F%
  404142434445464748494A4B4C4D4E4F%
  505152535455565758595A5B5C5D5E5F%
  606162636465666768696A6B6C6D6E6F%
  707172737475767778797A7B7C7D7E7F%
  808182838485868788898A8B8C8D8E8F%
  909192939495969798999A9B9C9D9E9F%
  A0A1A2A3A4A5A6A7A8A9AAABACADAEAF%
  B0B1B2B3B4B5B6B7B8B9BABBBCBDBEBF%
  C0C1C2C3C4C5C6C7C8C9CACBCCCDCECF%
  D0D1D2D3D4D5D6D7D8D9DADBDCDDDEDF%
  E0E1E2E3E4E5E6E7E8E9EAEBECEDEEEF%
  F0F1F2F3F4F5F6F7F8F9FAFBFCFDFEFF%
}
\ltx@onelevel@sanitize\AllBytesHex
\expandafter\lowercase\expandafter{%
  \expandafter\def\expandafter\AllBytesHexLC
      \expandafter{\AllBytesHex}%
}
\begingroup
  \catcode`\#=12 %
  \xdef\AllBytesName{%
    #01#02#03#04#05#06#07#08#09#0A#0B#0C#0D#0E#0F%
    #10#11#12#13#14#15#16#17#18#19#1A#1B#1C#1D#1E#1F%
    #20!"#23$#25&'#28#29*+,-.#2F%
    0123456789:;#3C=#3E?%
    @ABCDEFGHIJKLMNO%
    PQRSTUVWXYZ#5B\ltx@backslashchar#5D^_%
    `abcdefghijklmno%
    pqrstuvwxyz#7B|#7D\string~#7F%
    #80#81#82#83#84#85#86#87#88#89#8A#8B#8C#8D#8E#8F%
    #90#91#92#93#94#95#96#97#98#99#9A#9B#9C#9D#9E#9F%
    #A0#A1#A2#A3#A4#A5#A6#A7#A8#A9#AA#AB#AC#AD#AE#AF%
    #B0#B1#B2#B3#B4#B5#B6#B7#B8#B9#BA#BB#BC#BD#BE#BF%
    #C0#C1#C2#C3#C4#C5#C6#C7#C8#C9#CA#CB#CC#CD#CE#CF%
    #D0#D1#D2#D3#D4#D5#D6#D7#D8#D9#DA#DB#DC#DD#DE#DF%
    #E0#E1#E2#E3#E4#E5#E6#E7#E8#E9#EA#EB#EC#ED#EE#EF%
    #F0#F1#F2#F3#F4#F5#F6#F7#F8#F9#FA#FB#FC#FD#FE#FF%
  }%
\endgroup
\ltx@onelevel@sanitize\AllBytesName
\edef\AllBytesFromName{\expandafter\ltx@gobble\AllBytes}
\begingroup
  \def\|{|}%
  \edef\%{\ltx@percentchar}%
  \catcode`\|=0 %
  \catcode`\#=12 %
  \catcode`\~=12 %
  \catcode`\\=12 %
  |xdef|AllBytesString{%
    \000\001\002\003\004\005\006\007\010\011\012\013\014\015\016\017%
    \020\021\022\023\024\025\026\027\030\031\032\033\034\035\036\037%
    \040!"#$|%&'\(\)*+,-./%
    0123456789:;<=>?%
    @ABCDEFGHIJKLMNO%
    PQRSTUVWXYZ[\\]^_%
    `abcdefghijklmno%
    pqrstuvwxyz{||}~\177%
    \200\201\202\203\204\205\206\207\210\211\212\213\214\215\216\217%
    \220\221\222\223\224\225\226\227\230\231\232\233\234\235\236\237%
    \240\241\242\243\244\245\246\247\250\251\252\253\254\255\256\257%
    \260\261\262\263\264\265\266\267\270\271\272\273\274\275\276\277%
    \300\301\302\303\304\305\306\307\310\311\312\313\314\315\316\317%
    \320\321\322\323\324\325\326\327\330\331\332\333\334\335\336\337%
    \340\341\342\343\344\345\346\347\350\351\352\353\354\355\356\357%
    \360\361\362\363\364\365\366\367\370\371\372\373\374\375\376\377%
  }%
|endgroup
\ltx@onelevel@sanitize\AllBytesString
%    \end{macrocode}
%    \begin{macrocode}
\def\Test#1#2#3{%
  \begingroup
    \expandafter\expandafter\expandafter\def
    \expandafter\expandafter\expandafter\TestResult
    \expandafter\expandafter\expandafter{%
      #1{#2}%
    }%
    \ifx\TestResult#3%
    \else
      \newlinechar=10 %
      \msg{Expect:^^J#3}%
      \msg{Result:^^J\TestResult}%
      \errmessage{\string#2 -\string#1-> \string#3}%
    \fi
  \endgroup
}
\def\test#1#2#3{%
  \edef\TestFrom{#2}%
  \edef\TestExpect{#3}%
  \ltx@onelevel@sanitize\TestExpect
  \Test#1\TestFrom\TestExpect
}
\test\pdf@unescapehex{74657374}{test}
\begingroup
  \catcode0=12 %
  \catcode1=12 %
  \test\pdf@unescapehex{740074017400740174}{t^^@t^^At^^@t^^At}%
\endgroup
\Test\pdf@escapehex\AllBytes\AllBytesHex
\Test\pdf@unescapehex\AllBytesHex\AllBytes
\Test\pdf@escapename\AllBytes\AllBytesName
\Test\pdf@escapestring\AllBytes\AllBytesString
%    \end{macrocode}
%    \begin{macrocode}
\csname @@end\endcsname\end
%</test-escape>
%    \end{macrocode}
%
% \section{Installation}
%
% \subsection{Download}
%
% \paragraph{Package.} This package is available on
% CTAN\footnote{\CTANpkg{pdftexcmds}}:
% \begin{description}
% \item[\CTAN{macros/latex/contrib/oberdiek/pdftexcmds.dtx}] The source file.
% \item[\CTAN{macros/latex/contrib/oberdiek/pdftexcmds.pdf}] Documentation.
% \end{description}
%
%
% \paragraph{Bundle.} All the packages of the bundle `oberdiek'
% are also available in a TDS compliant ZIP archive. There
% the packages are already unpacked and the documentation files
% are generated. The files and directories obey the TDS standard.
% \begin{description}
% \item[\CTANinstall{install/macros/latex/contrib/oberdiek.tds.zip}]
% \end{description}
% \emph{TDS} refers to the standard ``A Directory Structure
% for \TeX\ Files'' (\CTAN{tds/tds.pdf}). Directories
% with \xfile{texmf} in their name are usually organized this way.
%
% \subsection{Bundle installation}
%
% \paragraph{Unpacking.} Unpack the \xfile{oberdiek.tds.zip} in the
% TDS tree (also known as \xfile{texmf} tree) of your choice.
% Example (linux):
% \begin{quote}
%   |unzip oberdiek.tds.zip -d ~/texmf|
% \end{quote}
%
% \paragraph{Script installation.}
% Check the directory \xfile{TDS:scripts/oberdiek/} for
% scripts that need further installation steps.
% Package \xpackage{attachfile2} comes with the Perl script
% \xfile{pdfatfi.pl} that should be installed in such a way
% that it can be called as \texttt{pdfatfi}.
% Example (linux):
% \begin{quote}
%   |chmod +x scripts/oberdiek/pdfatfi.pl|\\
%   |cp scripts/oberdiek/pdfatfi.pl /usr/local/bin/|
% \end{quote}
%
% \subsection{Package installation}
%
% \paragraph{Unpacking.} The \xfile{.dtx} file is a self-extracting
% \docstrip\ archive. The files are extracted by running the
% \xfile{.dtx} through \plainTeX:
% \begin{quote}
%   \verb|tex pdftexcmds.dtx|
% \end{quote}
%
% \paragraph{TDS.} Now the different files must be moved into
% the different directories in your installation TDS tree
% (also known as \xfile{texmf} tree):
% \begin{quote}
% \def\t{^^A
% \begin{tabular}{@{}>{\ttfamily}l@{ $\rightarrow$ }>{\ttfamily}l@{}}
%   pdftexcmds.sty & tex/generic/oberdiek/pdftexcmds.sty\\
%   oberdiek.pdftexcmds.lua & scripts/oberdiek/oberdiek.pdftexcmds.lua\\
%   pdftexcmds.lua & scripts/oberdiek/pdftexcmds.lua\\
%   pdftexcmds.pdf & doc/latex/oberdiek/pdftexcmds.pdf\\
%   test/pdftexcmds-test1.tex & doc/latex/oberdiek/test/pdftexcmds-test1.tex\\
%   test/pdftexcmds-test2.tex & doc/latex/oberdiek/test/pdftexcmds-test2.tex\\
%   test/pdftexcmds-test-shell.tex & doc/latex/oberdiek/test/pdftexcmds-test-shell.tex\\
%   test/pdftexcmds-test-escape.tex & doc/latex/oberdiek/test/pdftexcmds-test-escape.tex\\
%   pdftexcmds.dtx & source/latex/oberdiek/pdftexcmds.dtx\\
% \end{tabular}^^A
% }^^A
% \sbox0{\t}^^A
% \ifdim\wd0>\linewidth
%   \begingroup
%     \advance\linewidth by\leftmargin
%     \advance\linewidth by\rightmargin
%   \edef\x{\endgroup
%     \def\noexpand\lw{\the\linewidth}^^A
%   }\x
%   \def\lwbox{^^A
%     \leavevmode
%     \hbox to \linewidth{^^A
%       \kern-\leftmargin\relax
%       \hss
%       \usebox0
%       \hss
%       \kern-\rightmargin\relax
%     }^^A
%   }^^A
%   \ifdim\wd0>\lw
%     \sbox0{\small\t}^^A
%     \ifdim\wd0>\linewidth
%       \ifdim\wd0>\lw
%         \sbox0{\footnotesize\t}^^A
%         \ifdim\wd0>\linewidth
%           \ifdim\wd0>\lw
%             \sbox0{\scriptsize\t}^^A
%             \ifdim\wd0>\linewidth
%               \ifdim\wd0>\lw
%                 \sbox0{\tiny\t}^^A
%                 \ifdim\wd0>\linewidth
%                   \lwbox
%                 \else
%                   \usebox0
%                 \fi
%               \else
%                 \lwbox
%               \fi
%             \else
%               \usebox0
%             \fi
%           \else
%             \lwbox
%           \fi
%         \else
%           \usebox0
%         \fi
%       \else
%         \lwbox
%       \fi
%     \else
%       \usebox0
%     \fi
%   \else
%     \lwbox
%   \fi
% \else
%   \usebox0
% \fi
% \end{quote}
% If you have a \xfile{docstrip.cfg} that configures and enables \docstrip's
% TDS installing feature, then some files can already be in the right
% place, see the documentation of \docstrip.
%
% \subsection{Refresh file name databases}
%
% If your \TeX~distribution
% (\teTeX, \mikTeX, \dots) relies on file name databases, you must refresh
% these. For example, \teTeX\ users run \verb|texhash| or
% \verb|mktexlsr|.
%
% \subsection{Some details for the interested}
%
% \paragraph{Unpacking with \LaTeX.}
% The \xfile{.dtx} chooses its action depending on the format:
% \begin{description}
% \item[\plainTeX:] Run \docstrip\ and extract the files.
% \item[\LaTeX:] Generate the documentation.
% \end{description}
% If you insist on using \LaTeX\ for \docstrip\ (really,
% \docstrip\ does not need \LaTeX), then inform the autodetect routine
% about your intention:
% \begin{quote}
%   \verb|latex \let\install=y% \iffalse meta-comment
%
% File: pdftexcmds.dtx
% Version: 2019/07/25 v0.30
% Info: Utility functions of pdfTeX for LuaTeX
%
% Copyright (C) 2007, 2009-2011 by
%    Heiko Oberdiek <heiko.oberdiek at googlemail.com>
%
% This work may be distributed and/or modified under the
% conditions of the LaTeX Project Public License, either
% version 1.3c of this license or (at your option) any later
% version. This version of this license is in
%    https://www.latex-project.org/lppl/lppl-1-3c.txt
% and the latest version of this license is in
%    https://www.latex-project.org/lppl.txt
% and version 1.3 or later is part of all distributions of
% LaTeX version 2005/12/01 or later.
%
% This work has the LPPL maintenance status "maintained".
%
% The Current Maintainers of this work are
% Heiko Oberdiek and the Oberdiek Package Support Group
% https://github.com/ho-tex/oberdiek/issues
%
% The Base Interpreter refers to any `TeX-Format',
% because some files are installed in TDS:tex/generic//.
%
% This work consists of the main source file pdftexcmds.dtx
% and the derived files
%    pdftexcmds.sty, pdftexcmds.pdf, pdftexcmds.ins, pdftexcmds.drv,
%    pdftexcmds.bib, pdftexcmds-test1.tex, pdftexcmds-test2.tex,
%    pdftexcmds-test-shell.tex, pdftexcmds-test-escape.tex,
%    oberdiek.pdftexcmds.lua, pdftexcmds.lua.
%
% Distribution:
%    CTAN:macros/latex/contrib/oberdiek/pdftexcmds.dtx
%    CTAN:macros/latex/contrib/oberdiek/pdftexcmds.pdf
%
% Unpacking:
%    (a) If pdftexcmds.ins is present:
%           tex pdftexcmds.ins
%    (b) Without pdftexcmds.ins:
%           tex pdftexcmds.dtx
%    (c) If you insist on using LaTeX
%           latex \let\install=y\input{pdftexcmds.dtx}
%        (quote the arguments according to the demands of your shell)
%
% Documentation:
%    (a) If pdftexcmds.drv is present:
%           latex pdftexcmds.drv
%    (b) Without pdftexcmds.drv:
%           latex pdftexcmds.dtx; ...
%    The class ltxdoc loads the configuration file ltxdoc.cfg
%    if available. Here you can specify further options, e.g.
%    use A4 as paper format:
%       \PassOptionsToClass{a4paper}{article}
%
%    Programm calls to get the documentation (example):
%       pdflatex pdftexcmds.dtx
%       bibtex pdftexcmds.aux
%       makeindex -s gind.ist pdftexcmds.idx
%       pdflatex pdftexcmds.dtx
%       makeindex -s gind.ist pdftexcmds.idx
%       pdflatex pdftexcmds.dtx
%
% Installation:
%    TDS:tex/generic/oberdiek/pdftexcmds.sty
%    TDS:scripts/oberdiek/oberdiek.pdftexcmds.lua
%    TDS:scripts/oberdiek/pdftexcmds.lua
%    TDS:doc/latex/oberdiek/pdftexcmds.pdf
%    TDS:doc/latex/oberdiek/test/pdftexcmds-test1.tex
%    TDS:doc/latex/oberdiek/test/pdftexcmds-test2.tex
%    TDS:doc/latex/oberdiek/test/pdftexcmds-test-shell.tex
%    TDS:doc/latex/oberdiek/test/pdftexcmds-test-escape.tex
%    TDS:source/latex/oberdiek/pdftexcmds.dtx
%
%<*ignore>
\begingroup
  \catcode123=1 %
  \catcode125=2 %
  \def\x{LaTeX2e}%
\expandafter\endgroup
\ifcase 0\ifx\install y1\fi\expandafter
         \ifx\csname processbatchFile\endcsname\relax\else1\fi
         \ifx\fmtname\x\else 1\fi\relax
\else\csname fi\endcsname
%</ignore>
%<*install>
\input docstrip.tex
\Msg{************************************************************************}
\Msg{* Installation}
\Msg{* Package: pdftexcmds 2019/07/25 v0.30 Utility functions of pdfTeX for LuaTeX (HO)}
\Msg{************************************************************************}

\keepsilent
\askforoverwritefalse

\let\MetaPrefix\relax
\preamble

This is a generated file.

Project: pdftexcmds
Version: 2019/07/25 v0.30

Copyright (C) 2007, 2009-2011 by
   Heiko Oberdiek <heiko.oberdiek at googlemail.com>

This work may be distributed and/or modified under the
conditions of the LaTeX Project Public License, either
version 1.3c of this license or (at your option) any later
version. This version of this license is in
   https://www.latex-project.org/lppl/lppl-1-3c.txt
and the latest version of this license is in
   https://www.latex-project.org/lppl.txt
and version 1.3 or later is part of all distributions of
LaTeX version 2005/12/01 or later.

This work has the LPPL maintenance status "maintained".

The Current Maintainers of this work are
Heiko Oberdiek and the Oberdiek Package Support Group
https://github.com/ho-tex/oberdiek/issues


The Base Interpreter refers to any `TeX-Format',
because some files are installed in TDS:tex/generic//.

This work consists of the main source file pdftexcmds.dtx
and the derived files
   pdftexcmds.sty, pdftexcmds.pdf, pdftexcmds.ins, pdftexcmds.drv,
   pdftexcmds.bib, pdftexcmds-test1.tex, pdftexcmds-test2.tex,
   pdftexcmds-test-shell.tex, pdftexcmds-test-escape.tex,
   oberdiek.pdftexcmds.lua, pdftexcmds.lua.

\endpreamble
\let\MetaPrefix\DoubleperCent

\generate{%
  \file{pdftexcmds.ins}{\from{pdftexcmds.dtx}{install}}%
  \file{pdftexcmds.drv}{\from{pdftexcmds.dtx}{driver}}%
  \nopreamble
  \nopostamble
  \file{pdftexcmds.bib}{\from{pdftexcmds.dtx}{bib}}%
  \usepreamble\defaultpreamble
  \usepostamble\defaultpostamble
  \usedir{tex/generic/oberdiek}%
  \file{pdftexcmds.sty}{\from{pdftexcmds.dtx}{package}}%
%  \usedir{doc/latex/oberdiek/test}%
%  \file{pdftexcmds-test1.tex}{\from{pdftexcmds.dtx}{test1}}%
%  \file{pdftexcmds-test2.tex}{\from{pdftexcmds.dtx}{test2}}%
%  \file{pdftexcmds-test-shell.tex}{\from{pdftexcmds.dtx}{test-shell}}%
%  \file{pdftexcmds-test-escape.tex}{\from{pdftexcmds.dtx}{test-escape}}%
  \nopreamble
  \nopostamble
%  \usedir{source/latex/oberdiek/catalogue}%
%  \file{pdftexcmds.xml}{\from{pdftexcmds.dtx}{catalogue}}%
}
\def\MetaPrefix{-- }
\def\defaultpostamble{%
  \MetaPrefix^^J%
  \MetaPrefix\space End of File `\outFileName'.%
}
\def\currentpostamble{\defaultpostamble}%
\generate{%
  \usedir{scripts/oberdiek}%
  \file{oberdiek.pdftexcmds.lua}{\from{pdftexcmds.dtx}{lua}}%
  \file{pdftexcmds.lua}{\from{pdftexcmds.dtx}{lua}}%
}

\catcode32=13\relax% active space
\let =\space%
\Msg{************************************************************************}
\Msg{*}
\Msg{* To finish the installation you have to move the following}
\Msg{* file into a directory searched by TeX:}
\Msg{*}
\Msg{*     pdftexcmds.sty}
\Msg{*}
\Msg{* And install the following script files:}
\Msg{*}
\Msg{*     oberdiek.pdftexcmds.lua, pdftexcmds.lua}
\Msg{*}
\Msg{* To produce the documentation run the file `pdftexcmds.drv'}
\Msg{* through LaTeX.}
\Msg{*}
\Msg{* Happy TeXing!}
\Msg{*}
\Msg{************************************************************************}

\endbatchfile
%</install>
%<*bib>
@online{AndyThomas:Analog,
  author={Thomas, Andy},
  title={Analog of {\texttt{\csname textbackslash\endcsname}pdfelapsedtime} for
      {\hologo{LuaTeX}} and {\hologo{XeTeX}}},
  url={http://tex.stackexchange.com/a/32531},
  urldate={2011-11-29},
}
%</bib>
%<*ignore>
\fi
%</ignore>
%<*driver>
\NeedsTeXFormat{LaTeX2e}
\ProvidesFile{pdftexcmds.drv}%
  [2019/07/25 v0.30 Utility functions of pdfTeX for LuaTeX (HO)]%
\documentclass{ltxdoc}
\usepackage{holtxdoc}[2011/11/22]
\usepackage{paralist}
\usepackage{csquotes}
\usepackage[
  backend=bibtex,
  bibencoding=ascii,
  alldates=iso8601,
]{biblatex}[2011/11/13]
\bibliography{oberdiek-source}
\bibliography{pdftexcmds}
\begin{document}
  \DocInput{pdftexcmds.dtx}%
\end{document}
%</driver>
% \fi
%
%
% \CharacterTable
%  {Upper-case    \A\B\C\D\E\F\G\H\I\J\K\L\M\N\O\P\Q\R\S\T\U\V\W\X\Y\Z
%   Lower-case    \a\b\c\d\e\f\g\h\i\j\k\l\m\n\o\p\q\r\s\t\u\v\w\x\y\z
%   Digits        \0\1\2\3\4\5\6\7\8\9
%   Exclamation   \!     Double quote  \"     Hash (number) \#
%   Dollar        \$     Percent       \%     Ampersand     \&
%   Acute accent  \'     Left paren    \(     Right paren   \)
%   Asterisk      \*     Plus          \+     Comma         \,
%   Minus         \-     Point         \.     Solidus       \/
%   Colon         \:     Semicolon     \;     Less than     \<
%   Equals        \=     Greater than  \>     Question mark \?
%   Commercial at \@     Left bracket  \[     Backslash     \\
%   Right bracket \]     Circumflex    \^     Underscore    \_
%   Grave accent  \`     Left brace    \{     Vertical bar  \|
%   Right brace   \}     Tilde         \~}
%
% \GetFileInfo{pdftexcmds.drv}
%
% \title{The \xpackage{pdftexcmds} package}
% \date{2019/07/25 v0.30}
% \author{Heiko Oberdiek\thanks
% {Please report any issues at \url{https://github.com/ho-tex/oberdiek/issues}}}
%
% \maketitle
%
% \begin{abstract}
% \hologo{LuaTeX} provides most of the commands of \hologo{pdfTeX} 1.40. However
% a number of utility functions are removed. This package tries to fill
% the gap and implements some of the missing primitive using Lua.
% \end{abstract}
%
% \tableofcontents
%
% \def\csi#1{\texttt{\textbackslash\textit{#1}}}
%
% \section{Documentation}
%
% Some primitives of \hologo{pdfTeX} \cite{pdftex-manual}
% are not defined by \hologo{LuaTeX} \cite{luatex-manual}.
% This package implements macro based solutions using Lua code
% for the following missing \hologo{pdfTeX} primitives;
% \begin{compactitem}
% \item \cs{pdfstrcmp}
% \item \cs{pdfunescapehex}
% \item \cs{pdfescapehex}
% \item \cs{pdfescapename}
% \item \cs{pdfescapestring}
% \item \cs{pdffilesize}
% \item \cs{pdffilemoddate}
% \item \cs{pdffiledump}
% \item \cs{pdfmdfivesum}
% \item \cs{pdfresettimer}
% \item \cs{pdfelapsedtime}
% \item |\immediate\write18|
% \end{compactitem}
% The original names of the primitives cannot be used:
% \begin{itemize}
% \item
% The syntax for their arguments cannot easily
% simulated by macros. The primitives using key words
% such as |file| (\cs{pdfmdfivesum}) or |offset| and |length|
% (\cs{pdffiledump}) and uses \meta{general text} for the other
% arguments. Using token registers assignments, \meta{general text} could
% be catched. However, the simulated primitives are expandable
% and register assignments would destroy this important property.
% (\meta{general text} allows something like |\expandafter\bgroup ...}|.)
% \item
% The original primitives can be expanded using one expansion step.
% The new macros need two expansion steps because of the additional
% macro expansion. Example:
% \begin{quote}
%   |\expandafter\foo\pdffilemoddate{file}|\\
%   vs.\\
%   |\expandafter\expandafter\expandafter|\\
%   |\foo\pdf@filemoddate{file}|
% \end{quote}
% \end{itemize}
%
% \hologo{LuaTeX} isn't stable yet and thus the status of this package is
% \emph{experimental}. Feedback is welcome.
%
% \subsection{General principles}
%
% \begin{description}
% \item[Naming convention:]
%   Usually this package defines a macro |\pdf@|\meta{cmd} if
%   \hologo{pdfTeX} provides |\pdf|\meta{cmd}.
% \item[Arguments:] The order of arguments in |\pdf@|\meta{cmd}
%   is the same as for the corresponding primitive of \hologo{pdfTeX}.
%   The arguments are ordinary undelimited \hologo{TeX} arguments,
%   no \meta{general text} and without additional keywords.
% \item[Expandibility:]
%   The macro |\pdf@|\meta{cmd} is expandable if the
%   corresponding \hologo{pdfTeX} primitive has this property.
%   Exact two expansion steps are necessary (first is the macro
%   expansion) except for \cs{pdf@primitive} and \cs{pdf@ifprimitive}.
%   The latter ones are not macros, but have the direct meaning of the
%   primitive.
% \item[Without \hologo{LuaTeX}:]
%   The macros |\pdf@|\meta{cmd} are mapped to the commands
%   of \hologo{pdfTeX} if they are available. Otherwise they are undefined.
% \item[Availability:]
%   The macros that the packages provides are undefined, if
%   the necessary primitives are not found and cannot be
%   implemented by Lua.
% \end{description}
%
% \subsection{Macros}
%
% \subsubsection[Strings]{Strings \cite[``7.15 Strings'']{pdftex-manual}}
%
% \begin{declcs}{pdf@strcmp} \M{stringA} \M{stringB}
% \end{declcs}
% Same as |\pdfstrcmp{|\meta{stringA}|}{|\meta{stringB}|}|.
%
% \begin{declcs}{pdf@unescapehex} \M{string}
% \end{declcs}
% Same as |\pdfunescapehex{|\meta{string}|}|.
% The argument is a byte string given in hexadecimal notation.
% The result are character tokens from 0 until 255 with
% catcode 12 and the space with catcode 10.
%
% \begin{declcs}{pdf@escapehex} \M{string}\\
%   \cs{pdf@escapestring} \M{string}\\
%   \cs{pdf@escapename} \M{string}
% \end{declcs}
% Same as the primitives of \hologo{pdfTeX}. However \hologo{pdfTeX} does not
% know about characters with codes 256 and larger. Thus the
% string is treated as byte string, characters with more than
% eight bits are ignored.
%
% \subsubsection[Files]{Files \cite[``7.18 Files'']{pdftex-manual}}
%
% \begin{declcs}{pdf@filesize} \M{filename}
% \end{declcs}
% Same as |\pdffilesize{|\meta{filename}|}|.
%
% \begin{declcs}{pdf@filemoddate} \M{filename}
% \end{declcs}
% Same as |\pdffilemoddate{|\meta{filename}|}|.
%
% \begin{declcs}{pdf@filedump} \M{offset} \M{length} \M{filename}
% \end{declcs}
% Same as |\pdffiledump offset| \meta{offset} |length| \meta{length}
% |{|\meta{filename}|}|. Both \meta{offset} and \meta{length} must
% not be empty, but must be a valid \hologo{TeX} number.
%
% \begin{declcs}{pdf@mdfivesum} \M{string}
% \end{declcs}
% Same as |\pdfmdfivesum{|\meta{string}|}|. Keyword |file| is supported
% by macro \cs{pdf@filemdfivesum}.
%
% \begin{declcs}{pdf@filemdfivesum} \M{filename}
% \end{declcs}
% Same as |\pdfmdfivesum file{|\meta{filename}|}|.
%
% \subsubsection[Timekeeping]{Timekeeping \cite[``7.17 Timekeeping'']{pdftex-manual}}
%
% The timekeeping macros are based on Andy Thomas' work \cite{AndyThomas:Analog}.
%
% \begin{declcs}{pdf@resettimer}
% \end{declcs}
% Same as \cs{pdfresettimer}, it resets the internal timer.
%
% \begin{declcs}{pdf@elapsedtime}
% \end{declcs}
% Same as \cs{pdfelapsedtime}. It behaves like a read-only integer.
% For printing purposes it can be prefixed by \cs{the} or \cs{number}.
% It measures the time in scaled seconds (seconds multiplied with 65536)
% since the latest call of \cs{pdf@resettimer} or start of
% program/package. The resolution, the shortest time interval that
% can be measured, depends on the program and system.
% \begin{itemize}
% \item \hologo{pdfTeX} with |gettimeofday|: $\ge$ 1/65536\,s
% \item \hologo{pdfTeX} with |ftime|: $\ge$ 1\,ms
% \item \hologo{pdfTeX} with |time|: $\ge$ 1\,s
% \item \hologo{LuaTeX}: $\ge$ 10\,ms\\
%  (|os.clock()| returns a float number with two decimal digits in
%  \hologo{LuaTeX} beta-0.70.1-2011061416 (rev 4277)).
% \end{itemize}
%
% \subsubsection[Miscellaneous]{Miscellaneous \cite[``7.21 Miscellaneous'']{pdftex-manual}}
%
% \begin{declcs}{pdf@draftmode}
% \end{declcs}
% If the \TeX\ compiler knows \cs{pdfdraftmode} or \cs{draftmode}
% (\hologo{pdfTeX},
% \hologo{LuaTeX}), then \cs{pdf@draftmode} returns, whether
% this mode is enabled. The result is an implicit number:
% one means the draft mode is available and enabled.
% If the value is zero, then the mode is not active or
% \cs{pdfdraftmode} is not available.
% An explicit number is yielded by \cs{number}\cs{pdf@draftmode}.
% The macro cannot
% be used to change the mode, see \cs{pdf@setdraftmode}.
%
% \begin{declcs}{pdf@ifdraftmode} \M{true} \M{false}
% \end{declcs}
% If \cs{pdfdraftmode} is available and enabled, \meta{true} is
% called, otherwise \meta{false} is executed.
%
% \begin{declcs}{pdf@setdraftmode} \M{value}
% \end{declcs}
% Macro \cs{pdf@setdraftmode} expects the number zero or one as
% \meta{value}. Zero deactivates the mode and one enables the draft mode.
% The macro does not have an effect, if the feature \cs{pdfdraftmode} is not
% available.
%
% \begin{declcs}{pdf@shellescape}
% \end{declcs}
% Same as |\pdfshellescape|. It is or expands to |1| if external
% commands can be executed and |0| otherwise. In \hologo{pdfTeX} external
% commands must be enabled first by command line option or
% configuration option. In \hologo{LuaTeX} option |--safer| disables
% the execution of external commands.
%
% In \hologo{LuaTeX} before 0.68.0 \cs{pdf@shellescape} is not
% available due to a bug in |os.execute()|. The argumentless form
% crashes in some circumstances with segmentation fault.
% (It is fixed in version 0.68.0 or revision 4167 of \hologo{LuaTeX}.
% and packported to some version of 0.67.0).
%
% Hints for usage:
% \begin{itemize}
% \item Before its use \cs{pdf@shellescape} should be tested,
% whether it is available. Example with package \xpackage{ltxcmds}
% (loaded by package \xpackage{pdftexcmds}):
%\begin{quote}
%\begin{verbatim}
%\ltx@IfUndefined{pdf@shellescape}{%
%  % \pdf@shellescape is undefined
%}{%
%  % \pdf@shellescape is available
%}
%\end{verbatim}
%\end{quote}
% Use \cs{ltx@ifundefined} in expandable contexts.
% \item \cs{pdf@shellescape} might be a numerical constant,
% expands to the primitive, or expands to a plain number.
% Therefore use it in contexts where these differences does not matter.
% \item Use in comparisons, e.g.:
%   \begin{quote}
%     |\ifnum\pdf@shellescape=0 ...|
%   \end{quote}
% \item Print the number: |\number\pdf@shellescape|
% \end{itemize}
%
% \begin{declcs}{pdf@system} \M{cmdline}
% \end{declcs}
% It is a wrapper for |\immediate\write18| in \hologo{pdfTeX} or
% |os.execute| in \hologo{LuaTeX}.
%
% In theory |os.execute|
% returns a status number. But its meaning is quite
% undefined. Are there some reliable properties?
% Does it make sense to provide an user interface to
% this status exit code?
%
% \begin{declcs}{pdf@primitive} \csi{cmd}
% \end{declcs}
% Same as \cs{pdfprimitive} in \hologo{pdfTeX} or \hologo{LuaTeX}.
% In \hologo{XeTeX} the
% primitive is called \cs{primitive}. Despite the current definition
% of the command \csi{cmd}, it's meaning as primitive is used.
%
% \begin{declcs}{pdf@ifprimitive} \csi{cmd}
% \end{declcs}
% Same as \cs{ifpdfprimitive} in \hologo{pdfTeX} or
% \hologo{LuaTeX}. \hologo{XeTeX} calls
% it \cs{ifprimitive}. It is a switch that checks if the command
% \csi{cmd} has it's primitive meaning.
%
% \subsubsection{Additional macro: \cs{pdf@isprimitive}}
%
% \begin{declcs}{pdf@isprimitive} \csi{cmd1} \csi{cmd2} \M{true} \M{false}
% \end{declcs}
% If \csi{cmd1} has the primitive meaning given by the primitive name
% of \csi{cmd2}, then the argument \meta{true} is executed, otherwise
% \meta{false}. The macro \cs{pdf@isprimitive} is expandable.
% Internally it checks the result of \cs{meaning} and is therefore
% available for all \hologo{TeX} variants, even the original \hologo{TeX}.
% Example with \hologo{LaTeX}:
%\begin{quote}
%\begin{verbatim}
%\makeatletter
%\pdf@isprimitive{@@input}{input}{%
%  \typeout{\string\@@input\space is original\string\input}%
%}{%
%  \typeout{Oops, \string\@@input\space is not the %
%           original\string\input}%
%}
%\end{verbatim}
%\end{quote}
%
% \subsubsection{Experimental}
%
% \begin{declcs}{pdf@unescapehexnative} \M{string}\\
%   \cs{pdf@escapehexnative} \M{string}\\
%   \cs{pdf@escapenamenative} \M{string}\\
%   \cs{pdf@mdfivesumnative} \M{string}
% \end{declcs}
% The variants without |native| in the macro name are supposed to
% be compatible with \hologo{pdfTeX}. However characters with more than
% eight bits are not supported and are ignored. If \hologo{LuaTeX} is
% running, then its UTF-8 coded strings are used. Thus the full
% unicode character range is supported. However the result
% differs from \hologo{pdfTeX} for characters with eight or more bits.
%
% \begin{declcs}{pdf@pipe} \M{cmdline}
% \end{declcs}
% It calls \meta{cmdline} and returns the output of the external
% program in the usual manner as byte string (catcode 12, space with
% catcode 10). The Lua documentation says, that the used |io.popen|
% may not be available on all platforms. Then macro \cs{pdf@pipe}
% is undefined.
%
% \StopEventually{
% }
%
% \section{Implementation}
%
%    \begin{macrocode}
%<*package>
%    \end{macrocode}
%
% \subsection{Reload check and package identification}
%    Reload check, especially if the package is not used with \LaTeX.
%    \begin{macrocode}
\begingroup\catcode61\catcode48\catcode32=10\relax%
  \catcode13=5 % ^^M
  \endlinechar=13 %
  \catcode35=6 % #
  \catcode39=12 % '
  \catcode44=12 % ,
  \catcode45=12 % -
  \catcode46=12 % .
  \catcode58=12 % :
  \catcode64=11 % @
  \catcode123=1 % {
  \catcode125=2 % }
  \expandafter\let\expandafter\x\csname ver@pdftexcmds.sty\endcsname
  \ifx\x\relax % plain-TeX, first loading
  \else
    \def\empty{}%
    \ifx\x\empty % LaTeX, first loading,
      % variable is initialized, but \ProvidesPackage not yet seen
    \else
      \expandafter\ifx\csname PackageInfo\endcsname\relax
        \def\x#1#2{%
          \immediate\write-1{Package #1 Info: #2.}%
        }%
      \else
        \def\x#1#2{\PackageInfo{#1}{#2, stopped}}%
      \fi
      \x{pdftexcmds}{The package is already loaded}%
      \aftergroup\endinput
    \fi
  \fi
\endgroup%
%    \end{macrocode}
%    Package identification:
%    \begin{macrocode}
\begingroup\catcode61\catcode48\catcode32=10\relax%
  \catcode13=5 % ^^M
  \endlinechar=13 %
  \catcode35=6 % #
  \catcode39=12 % '
  \catcode40=12 % (
  \catcode41=12 % )
  \catcode44=12 % ,
  \catcode45=12 % -
  \catcode46=12 % .
  \catcode47=12 % /
  \catcode58=12 % :
  \catcode64=11 % @
  \catcode91=12 % [
  \catcode93=12 % ]
  \catcode123=1 % {
  \catcode125=2 % }
  \expandafter\ifx\csname ProvidesPackage\endcsname\relax
    \def\x#1#2#3[#4]{\endgroup
      \immediate\write-1{Package: #3 #4}%
      \xdef#1{#4}%
    }%
  \else
    \def\x#1#2[#3]{\endgroup
      #2[{#3}]%
      \ifx#1\@undefined
        \xdef#1{#3}%
      \fi
      \ifx#1\relax
        \xdef#1{#3}%
      \fi
    }%
  \fi
\expandafter\x\csname ver@pdftexcmds.sty\endcsname
\ProvidesPackage{pdftexcmds}%
  [2019/07/25 v0.30 Utility functions of pdfTeX for LuaTeX (HO)]%
%    \end{macrocode}
%
% \subsection{Catcodes}
%
%    \begin{macrocode}
\begingroup\catcode61\catcode48\catcode32=10\relax%
  \catcode13=5 % ^^M
  \endlinechar=13 %
  \catcode123=1 % {
  \catcode125=2 % }
  \catcode64=11 % @
  \def\x{\endgroup
    \expandafter\edef\csname pdftexcmds@AtEnd\endcsname{%
      \endlinechar=\the\endlinechar\relax
      \catcode13=\the\catcode13\relax
      \catcode32=\the\catcode32\relax
      \catcode35=\the\catcode35\relax
      \catcode61=\the\catcode61\relax
      \catcode64=\the\catcode64\relax
      \catcode123=\the\catcode123\relax
      \catcode125=\the\catcode125\relax
    }%
  }%
\x\catcode61\catcode48\catcode32=10\relax%
\catcode13=5 % ^^M
\endlinechar=13 %
\catcode35=6 % #
\catcode64=11 % @
\catcode123=1 % {
\catcode125=2 % }
\def\TMP@EnsureCode#1#2{%
  \edef\pdftexcmds@AtEnd{%
    \pdftexcmds@AtEnd
    \catcode#1=\the\catcode#1\relax
  }%
  \catcode#1=#2\relax
}
\TMP@EnsureCode{0}{12}%
\TMP@EnsureCode{1}{12}%
\TMP@EnsureCode{2}{12}%
\TMP@EnsureCode{10}{12}% ^^J
\TMP@EnsureCode{33}{12}% !
\TMP@EnsureCode{34}{12}% "
\TMP@EnsureCode{38}{4}% &
\TMP@EnsureCode{39}{12}% '
\TMP@EnsureCode{40}{12}% (
\TMP@EnsureCode{41}{12}% )
\TMP@EnsureCode{42}{12}% *
\TMP@EnsureCode{43}{12}% +
\TMP@EnsureCode{44}{12}% ,
\TMP@EnsureCode{45}{12}% -
\TMP@EnsureCode{46}{12}% .
\TMP@EnsureCode{47}{12}% /
\TMP@EnsureCode{58}{12}% :
\TMP@EnsureCode{60}{12}% <
\TMP@EnsureCode{62}{12}% >
\TMP@EnsureCode{91}{12}% [
\TMP@EnsureCode{93}{12}% ]
\TMP@EnsureCode{94}{7}% ^ (superscript)
\TMP@EnsureCode{95}{12}% _ (other)
\TMP@EnsureCode{96}{12}% `
\TMP@EnsureCode{126}{12}% ~ (other)
\edef\pdftexcmds@AtEnd{%
  \pdftexcmds@AtEnd
  \escapechar=\number\escapechar\relax
  \noexpand\endinput
}
\escapechar=92 %
%    \end{macrocode}
%
% \subsection{Load packages}
%
%    \begin{macrocode}
\begingroup\expandafter\expandafter\expandafter\endgroup
\expandafter\ifx\csname RequirePackage\endcsname\relax
  \def\TMP@RequirePackage#1[#2]{%
    \begingroup\expandafter\expandafter\expandafter\endgroup
    \expandafter\ifx\csname ver@#1.sty\endcsname\relax
      \input #1.sty\relax
    \fi
  }%
  \TMP@RequirePackage{infwarerr}[2007/09/09]%
  \TMP@RequirePackage{ifluatex}[2010/03/01]%
  \TMP@RequirePackage{ltxcmds}[2010/12/02]%
  \TMP@RequirePackage{ifpdf}[2010/09/13]%
\else
  \RequirePackage{infwarerr}[2007/09/09]%
  \RequirePackage{ifluatex}[2010/03/01]%
  \RequirePackage{ltxcmds}[2010/12/02]%
  \RequirePackage{ifpdf}[2010/09/13]%
\fi
%    \end{macrocode}
%
% \subsection{Without \hologo{LuaTeX}}
%
%    \begin{macrocode}
\ifluatex
\else
  \@PackageInfoNoLine{pdftexcmds}{LuaTeX not detected}%
  \def\pdftexcmds@nopdftex{%
    \@PackageInfoNoLine{pdftexcmds}{pdfTeX >= 1.30 not detected}%
    \let\pdftexcmds@nopdftex\relax
  }%
  \def\pdftexcmds@temp#1{%
    \begingroup\expandafter\expandafter\expandafter\endgroup
    \expandafter\ifx\csname pdf#1\endcsname\relax
      \pdftexcmds@nopdftex
    \else
      \expandafter\def\csname pdf@#1\expandafter\endcsname
      \expandafter##\expandafter{%
        \csname pdf#1\endcsname
      }%
    \fi
  }%
  \pdftexcmds@temp{strcmp}%
  \pdftexcmds@temp{escapehex}%
  \let\pdf@escapehexnative\pdf@escapehex
  \pdftexcmds@temp{unescapehex}%
  \let\pdf@unescapehexnative\pdf@unescapehex
  \pdftexcmds@temp{escapestring}%
  \pdftexcmds@temp{escapename}%
  \pdftexcmds@temp{filesize}%
  \pdftexcmds@temp{filemoddate}%
  \begingroup\expandafter\expandafter\expandafter\endgroup
  \expandafter\ifx\csname pdfshellescape\endcsname\relax
    \pdftexcmds@nopdftex
    \ltx@IfUndefined{pdftexversion}{%
    }{%
      \ifnum\pdftexversion>120 % 1.21a supports \ifeof18
        \ifeof18 %
          \chardef\pdf@shellescape=0 %
        \else
          \chardef\pdf@shellescape=1 %
        \fi
      \fi
    }%
  \else
    \def\pdf@shellescape{%
      \pdfshellescape
    }%
  \fi
  \begingroup\expandafter\expandafter\expandafter\endgroup
  \expandafter\ifx\csname pdffiledump\endcsname\relax
    \pdftexcmds@nopdftex
  \else
    \def\pdf@filedump#1#2#3{%
      \pdffiledump offset#1 length#2{#3}%
    }%
  \fi
%    \end{macrocode}
%    \begin{macrocode}
  \begingroup\expandafter\expandafter\expandafter\endgroup
  \expandafter\ifx\csname pdfmdfivesum\endcsname\relax
    \begingroup\expandafter\expandafter\expandafter\endgroup
    \expandafter\ifx\csname mdfivesum\endcsname\relax
      \pdftexcmds@nopdftex
    \else
      \def\pdf@mdfivesum#{\mdfivesum}%
      \let\pdf@mdfivesumnative\pdf@mdfivesum
      \def\pdf@filemdfivesum#{\mdfivesum file}%
    \fi
  \else
    \def\pdf@mdfivesum#{\pdfmdfivesum}%
    \let\pdf@mdfivesumnative\pdf@mdfivesum
    \def\pdf@filemdfivesum#{\pdfmdfivesum file}%
  \fi
%    \end{macrocode}
%    \begin{macrocode}
  \def\pdf@system#{%
    \immediate\write18%
  }%
  \def\pdftexcmds@temp#1{%
    \begingroup\expandafter\expandafter\expandafter\endgroup
    \expandafter\ifx\csname pdf#1\endcsname\relax
      \pdftexcmds@nopdftex
    \else
      \expandafter\let\csname pdf@#1\expandafter\endcsname
      \csname pdf#1\endcsname
    \fi
  }%
  \pdftexcmds@temp{resettimer}%
  \pdftexcmds@temp{elapsedtime}%
\fi
%    \end{macrocode}
%
% \subsection{\cs{pdf@primitive}, \cs{pdf@ifprimitive}}
%
%    Since version 1.40.0 \hologo{pdfTeX} has \cs{pdfprimitive} and
%    \cs{ifpdfprimitive}. And \cs{pdfprimitive} was fixed in
%    version 1.40.4.
%
%    \hologo{XeTeX} provides them under the name \cs{primitive} and
%    \cs{ifprimitive}. \hologo{LuaTeX} knows both name variants,
%    but they have possibly to be enabled first (|tex.enableprimitives|).
%
%    Depending on the format TeX Live uses a prefix |luatex|.
%
%    Caution: \cs{let} must be used for the definition of
%    the macros, especially because of \cs{ifpdfprimitive}.
%
% \subsubsection{Using \hologo{LuaTeX}'s \texttt{tex.enableprimitives}}
%
%    \begin{macrocode}
\ifluatex
%    \end{macrocode}
%    \begin{macro}{\pdftexcmds@directlua}
%    \begin{macrocode}
  \ifnum\luatexversion<36 %
    \def\pdftexcmds@directlua{\directlua0 }%
  \else
    \let\pdftexcmds@directlua\directlua
  \fi
%    \end{macrocode}
%    \end{macro}
%
%    \begin{macrocode}
  \begingroup
    \newlinechar=10 %
    \endlinechar=\newlinechar
    \pdftexcmds@directlua{%
      if tex.enableprimitives then
        tex.enableprimitives(
          'pdf@',
          {'primitive', 'ifprimitive', 'pdfdraftmode','draftmode'}
        )
        tex.enableprimitives('', {'luaescapestring'})
      end
    }%
  \endgroup %
%    \end{macrocode}
%
%    \begin{macrocode}
\fi
%    \end{macrocode}
%
% \subsubsection{Trying various names to find the primitives}
%
%    \begin{macro}{\pdftexcmds@strip@prefix}
%    \begin{macrocode}
\def\pdftexcmds@strip@prefix#1>{}
%    \end{macrocode}
%    \end{macro}
%    \begin{macrocode}
\def\pdftexcmds@temp#1#2#3{%
  \begingroup\expandafter\expandafter\expandafter\endgroup
  \expandafter\ifx\csname pdf@#1\endcsname\relax
    \begingroup
      \def\x{#3}%
      \edef\x{\expandafter\pdftexcmds@strip@prefix\meaning\x}%
      \escapechar=-1 %
      \edef\y{\expandafter\meaning\csname#2\endcsname}%
    \expandafter\endgroup
    \ifx\x\y
      \expandafter\let\csname pdf@#1\expandafter\endcsname
      \csname #2\endcsname
    \fi
  \fi
}
%    \end{macrocode}
%
%    \begin{macro}{\pdf@primitive}
%    \begin{macrocode}
\pdftexcmds@temp{primitive}{pdfprimitive}{pdfprimitive}% pdfTeX, oldLuaTeX
\pdftexcmds@temp{primitive}{primitive}{primitive}% XeTeX, luatex
\pdftexcmds@temp{primitive}{luatexprimitive}{pdfprimitive}% oldLuaTeX
\pdftexcmds@temp{primitive}{luatexpdfprimitive}{pdfprimitive}% oldLuaTeX
%    \end{macrocode}
%    \end{macro}
%    \begin{macro}{\pdf@ifprimitive}
%    \begin{macrocode}
\pdftexcmds@temp{ifprimitive}{ifpdfprimitive}{ifpdfprimitive}% pdfTeX, oldLuaTeX
\pdftexcmds@temp{ifprimitive}{ifprimitive}{ifprimitive}% XeTeX, luatex
\pdftexcmds@temp{ifprimitive}{luatexifprimitive}{ifpdfprimitive}% oldLuaTeX
\pdftexcmds@temp{ifprimitive}{luatexifpdfprimitive}{ifpdfprimitive}% oldLuaTeX
%    \end{macrocode}
%    \end{macro}
%
%    Disable broken \cs{pdfprimitive}.
%    \begin{macrocode}
\ifluatex\else
\begingroup
  \expandafter\ifx\csname pdf@primitive\endcsname\relax
  \else
    \expandafter\ifx\csname pdftexversion\endcsname\relax
    \else
      \ifnum\pdftexversion=140 %
        \expandafter\ifx\csname pdftexrevision\endcsname\relax
        \else
          \ifnum\pdftexrevision<4 %
            \endgroup
            \let\pdf@primitive\@undefined
            \@PackageInfoNoLine{pdftexcmds}{%
              \string\pdf@primitive\space disabled, %
              because\MessageBreak
              \string\pdfprimitive\space is broken until pdfTeX 1.40.4%
            }%
            \begingroup
          \fi
        \fi
      \fi
    \fi
  \fi
\endgroup
\fi
%    \end{macrocode}
%
% \subsubsection{Result}
%
%    \begin{macrocode}
\begingroup
  \@PackageInfoNoLine{pdftexcmds}{%
    \string\pdf@primitive\space is %
    \expandafter\ifx\csname pdf@primitive\endcsname\relax not \fi
    available%
  }%
  \@PackageInfoNoLine{pdftexcmds}{%
    \string\pdf@ifprimitive\space is %
    \expandafter\ifx\csname pdf@ifprimitive\endcsname\relax not \fi
    available%
  }%
\endgroup
%    \end{macrocode}
%
% \subsection{\hologo{XeTeX}}
%
%    Look for primitives \cs{shellescape}, \cs{strcmp}.
%    \begin{macrocode}
\def\pdftexcmds@temp#1{%
  \begingroup\expandafter\expandafter\expandafter\endgroup
  \expandafter\ifx\csname pdf@#1\endcsname\relax
    \begingroup
      \escapechar=-1 %
      \edef\x{\expandafter\meaning\csname#1\endcsname}%
      \def\y{#1}%
      \def\z##1->{}%
      \edef\y{\expandafter\z\meaning\y}%
    \expandafter\endgroup
    \ifx\x\y
      \expandafter\def\csname pdf@#1\expandafter\endcsname
      \expandafter{%
        \csname#1\endcsname
      }%
    \fi
  \fi
}%
\pdftexcmds@temp{shellescape}%
\pdftexcmds@temp{strcmp}%
%    \end{macrocode}
%
% \subsection{\cs{pdf@isprimitive}}
%
%    \begin{macrocode}
\def\pdf@isprimitive{%
  \begingroup\expandafter\expandafter\expandafter\endgroup
  \expandafter\ifx\csname pdf@strcmp\endcsname\relax
    \long\def\pdf@isprimitive##1{%
      \expandafter\pdftexcmds@isprimitive\expandafter{\meaning##1}%
    }%
    \long\def\pdftexcmds@isprimitive##1##2{%
      \expandafter\pdftexcmds@@isprimitive\expandafter{\string##2}{##1}%
    }%
    \def\pdftexcmds@@isprimitive##1##2{%
      \ifnum0\pdftexcmds@equal##1\delimiter##2\delimiter=1 %
        \expandafter\ltx@firstoftwo
      \else
        \expandafter\ltx@secondoftwo
      \fi
    }%
    \def\pdftexcmds@equal##1##2\delimiter##3##4\delimiter{%
      \ifx##1##3%
        \ifx\relax##2##4\relax
          1%
        \else
          \ifx\relax##2\relax
          \else
            \ifx\relax##4\relax
            \else
              \pdftexcmds@equalcont{##2}{##4}%
            \fi
          \fi
        \fi
      \fi
    }%
    \def\pdftexcmds@equalcont##1{%
      \def\pdftexcmds@equalcont####1####2##1##1##1##1{%
        ##1##1##1##1%
        \pdftexcmds@equal####1\delimiter####2\delimiter
      }%
    }%
    \expandafter\pdftexcmds@equalcont\csname fi\endcsname
  \else
    \long\def\pdf@isprimitive##1##2{%
      \ifnum\pdf@strcmp{\meaning##1}{\string##2}=0 %
        \expandafter\ltx@firstoftwo
      \else
        \expandafter\ltx@secondoftwo
      \fi
    }%
  \fi
}
\ifluatex
\ifx\pdfdraftmode\@undefined
  \let\pdfdraftmode\draftmode
\fi
\else
  \pdf@isprimitive
\fi
%    \end{macrocode}
%
% \subsection{\cs{pdf@draftmode}}
%
%
%    \begin{macrocode}
\let\pdftexcmds@temp\ltx@zero %
\ltx@IfUndefined{pdfdraftmode}{%
  \@PackageInfoNoLine{pdftexcmds}{\ltx@backslashchar pdfdraftmode not found}%
}{%
  \ifpdf
    \let\pdftexcmds@temp\ltx@one
    \@PackageInfoNoLine{pdftexcmds}{\ltx@backslashchar pdfdraftmode found}%
  \else
    \@PackageInfoNoLine{pdftexcmds}{%
      \ltx@backslashchar pdfdraftmode is ignored in DVI mode%
    }%
  \fi
}
\ifcase\pdftexcmds@temp
%    \end{macrocode}
%    \begin{macro}{\pdf@draftmode}
%    \begin{macrocode}
  \let\pdf@draftmode\ltx@zero
%    \end{macrocode}
%    \end{macro}
%    \begin{macro}{\pdf@ifdraftmode}
%    \begin{macrocode}
  \let\pdf@ifdraftmode\ltx@secondoftwo
%    \end{macrocode}
%    \end{macro}
%    \begin{macro}{\pdftexcmds@setdraftmode}
%    \begin{macrocode}
  \def\pdftexcmds@setdraftmode#1{}%
%    \end{macrocode}
%    \end{macro}
%    \begin{macrocode}
\else
%    \end{macrocode}
%    \begin{macro}{\pdftexcmds@draftmode}
%    \begin{macrocode}
  \let\pdftexcmds@draftmode\pdfdraftmode
%    \end{macrocode}
%    \end{macro}
%    \begin{macro}{\pdf@ifdraftmode}
%    \begin{macrocode}
  \def\pdf@ifdraftmode{%
    \ifnum\pdftexcmds@draftmode=\ltx@one
      \expandafter\ltx@firstoftwo
    \else
      \expandafter\ltx@secondoftwo
    \fi
  }%
%    \end{macrocode}
%    \end{macro}
%    \begin{macro}{\pdf@draftmode}
%    \begin{macrocode}
  \def\pdf@draftmode{%
    \ifnum\pdftexcmds@draftmode=\ltx@one
      \expandafter\ltx@one
    \else
      \expandafter\ltx@zero
    \fi
  }%
%    \end{macrocode}
%    \end{macro}
%    \begin{macro}{\pdftexcmds@setdraftmode}
%    \begin{macrocode}
  \def\pdftexcmds@setdraftmode#1{%
    \pdftexcmds@draftmode=#1\relax
  }%
%    \end{macrocode}
%    \end{macro}
%    \begin{macrocode}
\fi
%    \end{macrocode}
%    \begin{macro}{\pdf@setdraftmode}
%    \begin{macrocode}
\def\pdf@setdraftmode#1{%
  \begingroup
    \count\ltx@cclv=#1\relax
  \edef\x{\endgroup
    \noexpand\pdftexcmds@@setdraftmode{\the\count\ltx@cclv}%
  }%
  \x
}
%    \end{macrocode}
%    \end{macro}
%    \begin{macro}{\pdftexcmds@@setdraftmode}
%    \begin{macrocode}
\def\pdftexcmds@@setdraftmode#1{%
  \ifcase#1 %
    \pdftexcmds@setdraftmode{#1}%
  \or
    \pdftexcmds@setdraftmode{#1}%
  \else
    \@PackageWarning{pdftexcmds}{%
      \string\pdf@setdraftmode: Ignoring\MessageBreak
      invalid value `#1'%
    }%
  \fi
}
%    \end{macrocode}
%    \end{macro}
%
% \subsection{Load Lua module}
%
%    \begin{macrocode}
\ifluatex
\else
  \expandafter\pdftexcmds@AtEnd
\fi%
%    \end{macrocode}
%
%    \begin{macrocode}
\ifnum\luatexversion<80
  \begingroup\expandafter\expandafter\expandafter\endgroup
  \expandafter\ifx\csname RequirePackage\endcsname\relax
    \def\TMP@RequirePackage#1[#2]{%
      \begingroup\expandafter\expandafter\expandafter\endgroup
      \expandafter\ifx\csname ver@#1.sty\endcsname\relax
        \input #1.sty\relax
      \fi
    }%
    \TMP@RequirePackage{luatex-loader}[2009/04/10]%
  \else
    \RequirePackage{luatex-loader}[2009/04/10]%
  \fi
\fi
\pdftexcmds@directlua{%
  require("pdftexcmds")%
}
\ifnum\luatexversion>37 %
  \ifnum0%
      \pdftexcmds@directlua{%
        if status.ini_version then %
          tex.write("1")%
        end%
      }>0 %
    \everyjob\expandafter{%
      \the\everyjob
      \pdftexcmds@directlua{%
        require("pdftexcmds")%
      }%
    }%
  \fi
\fi
\begingroup
  \def\x{2019/07/25 v0.30}%
  \ltx@onelevel@sanitize\x
  \edef\y{%
    \pdftexcmds@directlua{%
      if oberdiek.pdftexcmds.getversion then %
        oberdiek.pdftexcmds.getversion()%
      end%
    }%
  }%
  \ifx\x\y
  \else
    \@PackageError{pdftexcmds}{%
      Wrong version of lua module.\MessageBreak
      Package version: \x\MessageBreak
      Lua module: \y
    }\@ehc
  \fi
\endgroup
%    \end{macrocode}
%
% \subsection{Lua functions}
%
% \subsubsection{Helper macros}
%
%    \begin{macro}{\pdftexcmds@toks}
%    \begin{macrocode}
\begingroup\expandafter\expandafter\expandafter\endgroup
\expandafter\ifx\csname newtoks\endcsname\relax
  \toksdef\pdftexcmds@toks=0 %
\else
  \csname newtoks\endcsname\pdftexcmds@toks
\fi
%    \end{macrocode}
%    \end{macro}
%
%    \begin{macro}{\pdftexcmds@Patch}
%    \begin{macrocode}
\def\pdftexcmds@Patch{0}
\ifnum\luatexversion>40 %
  \ifnum\luatexversion<66 %
    \def\pdftexcmds@Patch{1}%
  \fi
\fi
%    \end{macrocode}
%    \end{macro}
%    \begin{macrocode}
\ifcase\pdftexcmds@Patch
  \catcode`\&=14 %
\else
  \catcode`\&=9 %
%    \end{macrocode}
%    \begin{macro}{\pdftexcmds@PatchDecode}
%    \begin{macrocode}
  \def\pdftexcmds@PatchDecode#1\@nil{%
    \pdftexcmds@DecodeA#1^^A^^A\@nil{}%
  }%
%    \end{macrocode}
%    \end{macro}
%    \begin{macro}{\pdftexcmds@DecodeA}
%    \begin{macrocode}
  \def\pdftexcmds@DecodeA#1^^A^^A#2\@nil#3{%
    \ifx\relax#2\relax
      \ltx@ReturnAfterElseFi{%
        \pdftexcmds@DecodeB#3#1^^A^^B\@nil{}%
      }%
    \else
      \ltx@ReturnAfterFi{%
        \pdftexcmds@DecodeA#2\@nil{#3#1^^@}%
      }%
    \fi
  }%
%    \end{macrocode}
%    \end{macro}
%    \begin{macro}{\pdftexcmds@DecodeB}
%    \begin{macrocode}
  \def\pdftexcmds@DecodeB#1^^A^^B#2\@nil#3{%
    \ifx\relax#2\relax%
      \ltx@ReturnAfterElseFi{%
        \ltx@zero
        #3#1%
      }%
    \else
      \ltx@ReturnAfterFi{%
        \pdftexcmds@DecodeB#2\@nil{#3#1^^A}%
      }%
    \fi
  }%
%    \end{macrocode}
%    \end{macro}
%    \begin{macrocode}
\fi
%    \end{macrocode}
%
%    \begin{macrocode}
\ifnum\luatexversion<36 %
\else
  \catcode`\0=9 %
\fi
%    \end{macrocode}
%
% \subsubsection[Strings]{Strings \cite[``7.15 Strings'']{pdftex-manual}}
%
%    \begin{macro}{\pdf@strcmp}
%    \begin{macrocode}
\long\def\pdf@strcmp#1#2{%
  \directlua0{%
    oberdiek.pdftexcmds.strcmp("\luaescapestring{#1}",%
        "\luaescapestring{#2}")%
  }%
}%
%    \end{macrocode}
%    \end{macro}
%    \begin{macrocode}
\pdf@isprimitive
%    \end{macrocode}
%    \begin{macro}{\pdf@escapehex}
%    \begin{macrocode}
\long\def\pdf@escapehex#1{%
  \directlua0{%
    oberdiek.pdftexcmds.escapehex("\luaescapestring{#1}", "byte")%
  }%
}%
%    \end{macrocode}
%    \end{macro}
%    \begin{macro}{\pdf@escapehexnative}
%    \begin{macrocode}
\long\def\pdf@escapehexnative#1{%
  \directlua0{%
    oberdiek.pdftexcmds.escapehex("\luaescapestring{#1}")%
  }%
}%
%    \end{macrocode}
%    \end{macro}
%    \begin{macro}{\pdf@unescapehex}
%    \begin{macrocode}
\def\pdf@unescapehex#1{%
& \romannumeral\expandafter\pdftexcmds@PatchDecode
  \the\expandafter\pdftexcmds@toks
  \directlua0{%
    oberdiek.pdftexcmds.toks="pdftexcmds@toks"%
    oberdiek.pdftexcmds.unescapehex("\luaescapestring{#1}", "byte", \pdftexcmds@Patch)%
  }%
& \@nil
}%
%    \end{macrocode}
%    \end{macro}
%    \begin{macro}{\pdf@unescapehexnative}
%    \begin{macrocode}
\def\pdf@unescapehexnative#1{%
& \romannumeral\expandafter\pdftexcmds@PatchDecode
  \the\expandafter\pdftexcmds@toks
  \directlua0{%
    oberdiek.pdftexcmds.toks="pdftexcmds@toks"%
    oberdiek.pdftexcmds.unescapehex("\luaescapestring{#1}", \pdftexcmds@Patch)%
  }%
& \@nil
}%
%    \end{macrocode}
%    \end{macro}
%    \begin{macro}{\pdf@escapestring}
%    \begin{macrocode}
\long\def\pdf@escapestring#1{%
  \directlua0{%
    oberdiek.pdftexcmds.escapestring("\luaescapestring{#1}", "byte")%
  }%
}
%    \end{macrocode}
%    \end{macro}
%    \begin{macro}{\pdf@escapename}
%    \begin{macrocode}
\long\def\pdf@escapename#1{%
  \directlua0{%
    oberdiek.pdftexcmds.escapename("\luaescapestring{#1}", "byte")%
  }%
}
%    \end{macrocode}
%    \end{macro}
%    \begin{macro}{\pdf@escapenamenative}
%    \begin{macrocode}
\long\def\pdf@escapenamenative#1{%
  \directlua0{%
    oberdiek.pdftexcmds.escapename("\luaescapestring{#1}")%
  }%
}
%    \end{macrocode}
%    \end{macro}
%
% \subsubsection[Files]{Files \cite[``7.18 Files'']{pdftex-manual}}
%
%    \begin{macro}{\pdf@filesize}
%    \begin{macrocode}
\def\pdf@filesize#1{%
  \directlua0{%
    oberdiek.pdftexcmds.filesize("\luaescapestring{#1}")%
  }%
}
%    \end{macrocode}
%    \end{macro}
%    \begin{macro}{\pdf@filemoddate}
%    \begin{macrocode}
\def\pdf@filemoddate#1{%
  \directlua0{%
    oberdiek.pdftexcmds.filemoddate("\luaescapestring{#1}")%
  }%
}
%    \end{macrocode}
%    \end{macro}
%    \begin{macro}{\pdf@filedump}
%    \begin{macrocode}
\def\pdf@filedump#1#2#3{%
  \directlua0{%
    oberdiek.pdftexcmds.filedump("\luaescapestring{\number#1}",%
        "\luaescapestring{\number#2}",%
        "\luaescapestring{#3}")%
  }%
}%
%    \end{macrocode}
%    \end{macro}
%    \begin{macro}{\pdf@mdfivesum}
%    \begin{macrocode}
\long\def\pdf@mdfivesum#1{%
  \directlua0{%
    oberdiek.pdftexcmds.mdfivesum("\luaescapestring{#1}", "byte")%
  }%
}%
%    \end{macrocode}
%    \end{macro}
%    \begin{macro}{\pdf@mdfivesumnative}
%    \begin{macrocode}
\long\def\pdf@mdfivesumnative#1{%
  \directlua0{%
    oberdiek.pdftexcmds.mdfivesum("\luaescapestring{#1}")%
  }%
}%
%    \end{macrocode}
%    \end{macro}
%    \begin{macro}{\pdf@filemdfivesum}
%    \begin{macrocode}
\def\pdf@filemdfivesum#1{%
  \directlua0{%
    oberdiek.pdftexcmds.filemdfivesum("\luaescapestring{#1}")%
  }%
}%
%    \end{macrocode}
%    \end{macro}
%
% \subsubsection[Timekeeping]{Timekeeping \cite[``7.17 Timekeeping'']{pdftex-manual}}
%
%    \begin{macro}{\protected}
%    \begin{macrocode}
\let\pdftexcmds@temp=Y%
\begingroup\expandafter\expandafter\expandafter\endgroup
\expandafter\ifx\csname protected\endcsname\relax
  \pdftexcmds@directlua0{%
    if tex.enableprimitives then %
      tex.enableprimitives('', {'protected'})%
    end%
  }%
\fi
\begingroup\expandafter\expandafter\expandafter\endgroup
\expandafter\ifx\csname protected\endcsname\relax
  \let\pdftexcmds@temp=N%
\fi
%    \end{macrocode}
%    \end{macro}
%    \begin{macro}{\numexpr}
%    \begin{macrocode}
\begingroup\expandafter\expandafter\expandafter\endgroup
\expandafter\ifx\csname numexpr\endcsname\relax
  \pdftexcmds@directlua0{%
    if tex.enableprimitives then %
      tex.enableprimitives('', {'numexpr'})%
    end%
  }%
\fi
\begingroup\expandafter\expandafter\expandafter\endgroup
\expandafter\ifx\csname numexpr\endcsname\relax
  \let\pdftexcmds@temp=N%
\fi
%    \end{macrocode}
%    \end{macro}
%
%    \begin{macrocode}
\ifx\pdftexcmds@temp N%
  \@PackageWarningNoLine{pdftexcmds}{%
    Definitions of \ltx@backslashchar pdf@resettimer and%
    \MessageBreak
    \ltx@backslashchar pdf@elapsedtime are skipped, because%
    \MessageBreak
    e-TeX's \ltx@backslashchar protected or %
    \ltx@backslashchar numexpr are missing%
  }%
\else
%    \end{macrocode}
%
%    \begin{macro}{\pdf@resettimer}
%    \begin{macrocode}
  \protected\def\pdf@resettimer{%
    \pdftexcmds@directlua0{%
      oberdiek.pdftexcmds.resettimer()%
    }%
  }%
%    \end{macrocode}
%    \end{macro}
%
%    \begin{macro}{\pdf@elapsedtime}
%    \begin{macrocode}
  \protected\def\pdf@elapsedtime{%
    \numexpr
      \pdftexcmds@directlua0{%
        oberdiek.pdftexcmds.elapsedtime()%
      }%
    \relax
  }%
%    \end{macrocode}
%    \end{macro}
%    \begin{macrocode}
\fi
%    \end{macrocode}
%
% \subsubsection{Shell escape}
%
%    \begin{macro}{\pdf@shellescape}
%
%    \begin{macrocode}
\ifnum\luatexversion<68 %
\else
  \protected\edef\pdf@shellescape{%
   \numexpr\directlua{tex.sprint(%
         \number\catcodetable@string,status.shell_escape)}\relax}
\fi
%    \end{macrocode}
%    \end{macro}
%
%    \begin{macro}{\pdf@system}
%    \begin{macrocode}
\def\pdf@system#1{%
  \directlua0{%
    oberdiek.pdftexcmds.system("\luaescapestring{#1}")%
  }%
}
%    \end{macrocode}
%    \end{macro}
%
%    \begin{macro}{\pdf@lastsystemstatus}
%    \begin{macrocode}
\def\pdf@lastsystemstatus{%
  \directlua0{%
    oberdiek.pdftexcmds.lastsystemstatus()%
  }%
}
%    \end{macrocode}
%    \end{macro}
%    \begin{macro}{\pdf@lastsystemexit}
%    \begin{macrocode}
\def\pdf@lastsystemexit{%
  \directlua0{%
    oberdiek.pdftexcmds.lastsystemexit()%
  }%
}
%    \end{macrocode}
%    \end{macro}
%
%    \begin{macrocode}
\catcode`\0=12 %
%    \end{macrocode}
%
%    \begin{macro}{\pdf@pipe}
%    Check availability of |io.popen| first.
%    \begin{macrocode}
\ifnum0%
    \pdftexcmds@directlua{%
      if io.popen then %
        tex.write("1")%
      end%
    }%
    =1 %
  \def\pdf@pipe#1{%
&   \romannumeral\expandafter\pdftexcmds@PatchDecode
    \the\expandafter\pdftexcmds@toks
    \pdftexcmds@directlua{%
      oberdiek.pdftexcmds.toks="pdftexcmds@toks"%
      oberdiek.pdftexcmds.pipe("\luaescapestring{#1}", \pdftexcmds@Patch)%
    }%
&   \@nil
  }%
\fi
%    \end{macrocode}
%    \end{macro}
%
%    \begin{macrocode}
\pdftexcmds@AtEnd%
%</package>
%    \end{macrocode}
%
% \subsection{Lua module}
%
%    \begin{macrocode}
%<*lua>
%    \end{macrocode}
%
%    \begin{macrocode}
oberdiek = oberdiek or {}
local pdftexcmds = oberdiek.pdftexcmds or {}
oberdiek.pdftexcmds = pdftexcmds
local systemexitstatus
function pdftexcmds.getversion()
  tex.write("2019/07/25 v0.30")
end
%    \end{macrocode}
%
% \subsubsection[Strings]{Strings \cite[``7.15 Strings'']{pdftex-manual}}
%
%    \begin{macrocode}
function pdftexcmds.strcmp(A, B)
  if A == B then
    tex.write("0")
  elseif A < B then
    tex.write("-1")
  else
    tex.write("1")
  end
end
local function utf8_to_byte(str)
  local i = 0
  local n = string.len(str)
  local t = {}
  while i < n do
    i = i + 1
    local a = string.byte(str, i)
    if a < 128 then
      table.insert(t, string.char(a))
    else
      if a >= 192 and i < n then
        i = i + 1
        local b = string.byte(str, i)
        if b < 128 or b >= 192 then
          i = i - 1
        elseif a == 194 then
          table.insert(t, string.char(b))
        elseif a == 195 then
          table.insert(t, string.char(b + 64))
        end
      end
    end
  end
  return table.concat(t)
end
function pdftexcmds.escapehex(str, mode)
  if mode == "byte" then
    str = utf8_to_byte(str)
  end
  tex.write((string.gsub(str, ".",
    function (ch)
      return string.format("%02X", string.byte(ch))
    end
  )))
end
%    \end{macrocode}
%    See procedure |unescapehex| in file \xfile{utils.c} of \hologo{pdfTeX}.
%    Caution: |tex.write| ignores leading spaces.
%    \begin{macrocode}
function pdftexcmds.unescapehex(str, mode, patch)
  local a = 0
  local first = true
  local result = {}
  for i = 1, string.len(str), 1 do
    local ch = string.byte(str, i)
    if ch >= 48 and ch <= 57 then
      ch = ch - 48
    elseif ch >= 65 and ch <= 70 then
      ch = ch - 55
    elseif ch >= 97 and ch <= 102 then
      ch = ch - 87
    else
      ch = nil
    end
    if ch then
      if first then
        a = ch * 16
        first = false
      else
        table.insert(result, a + ch)
        first = true
      end
    end
  end
  if not first then
    table.insert(result, a)
  end
  if patch == 1 then
    local temp = {}
    for i, a in ipairs(result) do
      if a == 0 then
        table.insert(temp, 1)
        table.insert(temp, 1)
      else
        if a == 1 then
          table.insert(temp, 1)
          table.insert(temp, 2)
        else
          table.insert(temp, a)
        end
      end
    end
    result = temp
  end
  if mode == "byte" then
    local utf8 = {}
    for i, a in ipairs(result) do
      if a < 128 then
        table.insert(utf8, a)
      else
        if a < 192 then
          table.insert(utf8, 194)
          a = a - 128
        else
          table.insert(utf8, 195)
          a = a - 192
        end
        table.insert(utf8, a + 128)
      end
    end
    result = utf8
  end
%    \end{macrocode}
%    this next line added for current luatex; this is the only
%    change in the file.  eroux, 28apr13. (v 0.21)
%    \begin{macrocode}
  local unpack = _G["unpack"] or table.unpack
  tex.settoks(pdftexcmds.toks, string.char(unpack(result)))
end
%    \end{macrocode}
%    See procedure |escapestring| in file \xfile{utils.c} of \hologo{pdfTeX}.
%    \begin{macrocode}
function pdftexcmds.escapestring(str, mode)
  if mode == "byte" then
    str = utf8_to_byte(str)
  end
  tex.write((string.gsub(str, ".",
    function (ch)
      local b = string.byte(ch)
      if b < 33 or b > 126 then
        return string.format("\\%.3o", b)
      end
      if b == 40 or b == 41 or b == 92 then
        return "\\" .. ch
      end
%    \end{macrocode}
%    Lua 5.1 returns the match in case of return value |nil|.
%    \begin{macrocode}
      return nil
    end
  )))
end
%    \end{macrocode}
%    See procedure |escapename| in file \xfile{utils.c} of \hologo{pdfTeX}.
%    \begin{macrocode}
function pdftexcmds.escapename(str, mode)
  if mode == "byte" then
    str = utf8_to_byte(str)
  end
  tex.write((string.gsub(str, ".",
    function (ch)
      local b = string.byte(ch)
      if b == 0 then
%    \end{macrocode}
%    In Lua 5.0 |nil| could be used for the empty string,
%    But |nil| returns the match in Lua 5.1, thus we use
%    the empty string explicitly.
%    \begin{macrocode}
        return ""
      end
      if b <= 32 or b >= 127
          or b == 35 or b == 37 or b == 40 or b == 41
          or b == 47 or b == 60 or b == 62 or b == 91
          or b == 93 or b == 123 or b == 125 then
        return string.format("#%.2X", b)
      else
%    \end{macrocode}
%    Lua 5.1 returns the match in case of return value |nil|.
%    \begin{macrocode}
        return nil
      end
    end
  )))
end
%    \end{macrocode}
%
% \subsubsection[Files]{Files \cite[``7.18 Files'']{pdftex-manual}}
%
%    \begin{macrocode}
function pdftexcmds.filesize(filename)
  local foundfile = kpse.find_file(filename, "tex", true)
  if foundfile then
    local size = lfs.attributes(foundfile, "size")
    if size then
      tex.write(size)
    end
  end
end
%    \end{macrocode}
%    See procedure |makepdftime| in file \xfile{utils.c} of \hologo{pdfTeX}.
%    \begin{macrocode}
function pdftexcmds.filemoddate(filename)
  local foundfile = kpse.find_file(filename, "tex", true)
  if foundfile then
    local date = lfs.attributes(foundfile, "modification")
    if date then
      local d = os.date("*t", date)
      if d.sec >= 60 then
        d.sec = 59
      end
      local u = os.date("!*t", date)
      local off = 60 * (d.hour - u.hour) + d.min - u.min
      if d.year ~= u.year then
        if d.year > u.year then
          off = off + 1440
        else
          off = off - 1440
        end
      elseif d.yday ~= u.yday then
        if d.yday > u.yday then
          off = off + 1440
        else
          off = off - 1440
        end
      end
      local timezone
      if off == 0 then
        timezone = "Z"
      else
        local hours = math.floor(off / 60)
        local mins = math.abs(off - hours * 60)
        timezone = string.format("%+03d'%02d'", hours, mins)
      end
      tex.write(string.format("D:%04d%02d%02d%02d%02d%02d%s",
          d.year, d.month, d.day, d.hour, d.min, d.sec, timezone))
    end
  end
end
function pdftexcmds.filedump(offset, length, filename)
  length = tonumber(length)
  if length and length > 0 then
    local foundfile = kpse.find_file(filename, "tex", true)
    if foundfile then
      offset = tonumber(offset)
      if not offset then
        offset = 0
      end
      local filehandle = io.open(foundfile, "rb")
      if filehandle then
        if offset > 0 then
          filehandle:seek("set", offset)
        end
        local dump = filehandle:read(length)
        pdftexcmds.escapehex(dump)
        filehandle:close()
      end
    end
  end
end
function pdftexcmds.mdfivesum(str, mode)
  if mode == "byte" then
    str = utf8_to_byte(str)
  end
  pdftexcmds.escapehex(md5.sum(str))
end
function pdftexcmds.filemdfivesum(filename)
  local foundfile = kpse.find_file(filename, "tex", true)
  if foundfile then
    local filehandle = io.open(foundfile, "rb")
    if filehandle then
      local contents = filehandle:read("*a")
      pdftexcmds.escapehex(md5.sum(contents))
      filehandle:close()
    end
  end
end
%    \end{macrocode}
%
% \subsubsection[Timekeeping]{Timekeeping \cite[``7.17 Timekeeping'']{pdftex-manual}}
%
%    The functions for timekeeping are based on
%    Andy Thomas' work \cite{AndyThomas:Analog}.
%    Changes:
%    \begin{itemize}
%    \item Overflow check is added.
%    \item |string.format| is used to avoid exponential number
%          representation for sure.
%    \item |tex.write| is used instead of |tex.print| to get
%          tokens with catcode 12 and without appended \cs{endlinechar}.
%    \end{itemize}
%    \begin{macrocode}
local basetime = 0
function pdftexcmds.resettimer()
  basetime = os.clock()
end
function pdftexcmds.elapsedtime()
  local val = (os.clock() - basetime) * 65536 + .5
  if val > 2147483647 then
    val = 2147483647
  end
  tex.write(string.format("%d", val))
end
%    \end{macrocode}
%
% \subsubsection[Miscellaneous]{Miscellaneous \cite[``7.21 Miscellaneous'']{pdftex-manual}}
%
%    \begin{macrocode}
function pdftexcmds.shellescape()
  if os.execute then
    if status
        and status.luatex_version
        and status.luatex_version >= 68 then
      tex.write(os.execute())
    else
      local result = os.execute()
      if result == 0 then
        tex.write("0")
      else
        if result == nil then
          tex.write("0")
        else
          tex.write("1")
        end
      end
    end
  else
    tex.write("0")
  end
end
function pdftexcmds.system(cmdline)
  systemexitstatus = nil
  texio.write_nl("log", "system(" .. cmdline .. ") ")
  if os.execute then
    texio.write("log", "executed.")
    systemexitstatus = os.execute(cmdline)
  else
    texio.write("log", "disabled.")
  end
end
function pdftexcmds.lastsystemstatus()
  local result = tonumber(systemexitstatus)
  if result then
    local x = math.floor(result / 256)
    tex.write(result - 256 * math.floor(result / 256))
  end
end
function pdftexcmds.lastsystemexit()
  local result = tonumber(systemexitstatus)
  if result then
    tex.write(math.floor(result / 256))
  end
end
function pdftexcmds.pipe(cmdline, patch)
  local result
  systemexitstatus = nil
  texio.write_nl("log", "pipe(" .. cmdline ..") ")
  if io.popen then
    texio.write("log", "executed.")
    local handle = io.popen(cmdline, "r")
    if handle then
      result = handle:read("*a")
      handle:close()
    end
  else
    texio.write("log", "disabled.")
  end
  if result then
    if patch == 1 then
      local temp = {}
      for i, a in ipairs(result) do
        if a == 0 then
          table.insert(temp, 1)
          table.insert(temp, 1)
        else
          if a == 1 then
            table.insert(temp, 1)
            table.insert(temp, 2)
          else
            table.insert(temp, a)
          end
        end
      end
      result = temp
    end
    tex.settoks(pdftexcmds.toks, result)
  else
    tex.settoks(pdftexcmds.toks, "")
  end
end
%    \end{macrocode}
%    \begin{macrocode}
%</lua>
%    \end{macrocode}
%
% \section{Test}
%
% \subsection{Catcode checks for loading}
%
%    \begin{macrocode}
%<*test1>
%    \end{macrocode}
%    \begin{macrocode}
\catcode`\{=1 %
\catcode`\}=2 %
\catcode`\#=6 %
\catcode`\@=11 %
\expandafter\ifx\csname count@\endcsname\relax
  \countdef\count@=255 %
\fi
\expandafter\ifx\csname @gobble\endcsname\relax
  \long\def\@gobble#1{}%
\fi
\expandafter\ifx\csname @firstofone\endcsname\relax
  \long\def\@firstofone#1{#1}%
\fi
\expandafter\ifx\csname loop\endcsname\relax
  \expandafter\@firstofone
\else
  \expandafter\@gobble
\fi
{%
  \def\loop#1\repeat{%
    \def\body{#1}%
    \iterate
  }%
  \def\iterate{%
    \body
      \let\next\iterate
    \else
      \let\next\relax
    \fi
    \next
  }%
  \let\repeat=\fi
}%
\def\RestoreCatcodes{}
\count@=0 %
\loop
  \edef\RestoreCatcodes{%
    \RestoreCatcodes
    \catcode\the\count@=\the\catcode\count@\relax
  }%
\ifnum\count@<255 %
  \advance\count@ 1 %
\repeat

\def\RangeCatcodeInvalid#1#2{%
  \count@=#1\relax
  \loop
    \catcode\count@=15 %
  \ifnum\count@<#2\relax
    \advance\count@ 1 %
  \repeat
}
\def\RangeCatcodeCheck#1#2#3{%
  \count@=#1\relax
  \loop
    \ifnum#3=\catcode\count@
    \else
      \errmessage{%
        Character \the\count@\space
        with wrong catcode \the\catcode\count@\space
        instead of \number#3%
      }%
    \fi
  \ifnum\count@<#2\relax
    \advance\count@ 1 %
  \repeat
}
\def\space{ }
\expandafter\ifx\csname LoadCommand\endcsname\relax
  \def\LoadCommand{\input pdftexcmds.sty\relax}%
\fi
\def\Test{%
  \RangeCatcodeInvalid{0}{47}%
  \RangeCatcodeInvalid{58}{64}%
  \RangeCatcodeInvalid{91}{96}%
  \RangeCatcodeInvalid{123}{255}%
  \catcode`\@=12 %
  \catcode`\\=0 %
  \catcode`\%=14 %
  \LoadCommand
  \RangeCatcodeCheck{0}{36}{15}%
  \RangeCatcodeCheck{37}{37}{14}%
  \RangeCatcodeCheck{38}{47}{15}%
  \RangeCatcodeCheck{48}{57}{12}%
  \RangeCatcodeCheck{58}{63}{15}%
  \RangeCatcodeCheck{64}{64}{12}%
  \RangeCatcodeCheck{65}{90}{11}%
  \RangeCatcodeCheck{91}{91}{15}%
  \RangeCatcodeCheck{92}{92}{0}%
  \RangeCatcodeCheck{93}{96}{15}%
  \RangeCatcodeCheck{97}{122}{11}%
  \RangeCatcodeCheck{123}{255}{15}%
  \RestoreCatcodes
}
\Test
\csname @@end\endcsname
\end
%    \end{macrocode}
%    \begin{macrocode}
%</test1>
%    \end{macrocode}
%
% \subsection{Test for \cs{pdf@isprimitive}}
%
%    \begin{macrocode}
%<*test2>
\catcode`\{=1 %
\catcode`\}=2 %
\catcode`\#=6 %
\catcode`\@=11 %
\input pdftexcmds.sty\relax
\def\msg#1{%
  \begingroup
    \escapechar=92 %
    \immediate\write16{#1}%
  \endgroup
}
\long\def\test#1#2#3#4{%
  \begingroup
    #4%
    \def\str{%
      Test \string\pdf@isprimitive
      {\string #1}{\string #2}{...}: %
    }%
    \pdf@isprimitive{#1}{#2}{%
      \ifx#3Y%
        \msg{\str true ==> OK.}%
      \else
        \errmessage{\str false ==> FAILED}%
      \fi
    }{%
      \ifx#3Y%
        \errmessage{\str true ==> FAILED}%
      \else
        \msg{\str false ==> OK.}%
      \fi
    }%
  \endgroup
}
\test\relax\relax Y{}
\test\foobar\relax Y{\let\foobar\relax}
\test\foobar\relax N{}
\test\hbox\hbox Y{}
\test\foobar@hbox\hbox Y{\let\foobar@hbox\hbox}
\test\if\if Y{}
\test\if\ifx N{}
\test\ifx\if N{}
\test\par\par Y{}
\test\hbox\par N{}
\test\par\hbox N{}
\csname @@end\endcsname\end
%</test2>
%    \end{macrocode}
%
% \subsection{Test for \cs{pdf@shellescape}}
%
%    \begin{macrocode}
%<*test-shell>
\catcode`\{=1 %
\catcode`\}=2 %
\catcode`\#=6 %
\catcode`\@=11 %
\input pdftexcmds.sty\relax
\def\msg#{\immediate\write16}
\def\MaybeEnd{}
\ifx\luatexversion\UnDeFiNeD
\else
  \ifnum\luatexversion<68 %
    \ifx\pdf@shellescape\@undefined
      \msg{SHELL=U}%
      \msg{OK (LuaTeX < 0.68)}%
    \else
      \msg{SHELL=defined}%
      \errmessage{Failed (LuaTeX < 0.68)}%
    \fi
    \def\MaybeEnd{\csname @@end\endcsname\end}%
  \fi
\fi
\MaybeEnd
\ifx\pdf@shellescape\@undefined
  \msg{SHELL=U}%
\else
  \msg{SHELL=\number\pdf@shellescape}%
\fi
\ifx\expected\@undefined
\else
  \ifx\expected\relax
    \msg{EXPECTED=U}%
    \ifx\pdf@shellescape\@undefined
      \msg{OK}%
    \else
      \errmessage{Failed}%
    \fi
  \else
    \msg{EXPECTED=\number\expected}%
    \ifnum\pdf@shellescape=\expected\relax
      \msg{OK}%
    \else
      \errmessage{Failed}%
    \fi
  \fi
\fi
\csname @@end\endcsname\end
%</test-shell>
%    \end{macrocode}
%
% \subsection{Test for escape functions}
%
%    \begin{macrocode}
%<*test-escape>
\catcode`\{=1 %
\catcode`\}=2 %
\catcode`\#=6 %
\catcode`\^=7 %
\catcode`\@=11 %
\errorcontextlines=1000 %
\input pdftexcmds.sty\relax
\def\msg#1{%
  \begingroup
    \escapechar=92 %
    \immediate\write16{#1}%
  \endgroup
}
%    \end{macrocode}
%    \begin{macrocode}
\begingroup
  \catcode`\@=11 %
  \countdef\count@=255 %
  \def\space{ }%
  \long\def\@whilenum#1\do #2{%
    \ifnum #1\relax
      #2\relax
      \@iwhilenum{#1\relax#2\relax}%
    \fi
  }%
  \long\def\@iwhilenum#1{%
    \ifnum #1%
      \expandafter\@iwhilenum
    \else
      \expandafter\ltx@gobble
    \fi
    {#1}%
  }%
  \gdef\AllBytes{}%
  \count@=0 %
  \catcode0=12 %
  \@whilenum\count@<256 \do{%
    \lccode0=\count@
    \ifnum\count@=32 %
      \xdef\AllBytes{\AllBytes\space}%
    \else
      \lowercase{%
        \xdef\AllBytes{\AllBytes^^@}%
      }%
    \fi
    \advance\count@ by 1 %
  }%
\endgroup
%    \end{macrocode}
%    \begin{macrocode}
\def\AllBytesHex{%
  000102030405060708090A0B0C0D0E0F%
  101112131415161718191A1B1C1D1E1F%
  202122232425262728292A2B2C2D2E2F%
  303132333435363738393A3B3C3D3E3F%
  404142434445464748494A4B4C4D4E4F%
  505152535455565758595A5B5C5D5E5F%
  606162636465666768696A6B6C6D6E6F%
  707172737475767778797A7B7C7D7E7F%
  808182838485868788898A8B8C8D8E8F%
  909192939495969798999A9B9C9D9E9F%
  A0A1A2A3A4A5A6A7A8A9AAABACADAEAF%
  B0B1B2B3B4B5B6B7B8B9BABBBCBDBEBF%
  C0C1C2C3C4C5C6C7C8C9CACBCCCDCECF%
  D0D1D2D3D4D5D6D7D8D9DADBDCDDDEDF%
  E0E1E2E3E4E5E6E7E8E9EAEBECEDEEEF%
  F0F1F2F3F4F5F6F7F8F9FAFBFCFDFEFF%
}
\ltx@onelevel@sanitize\AllBytesHex
\expandafter\lowercase\expandafter{%
  \expandafter\def\expandafter\AllBytesHexLC
      \expandafter{\AllBytesHex}%
}
\begingroup
  \catcode`\#=12 %
  \xdef\AllBytesName{%
    #01#02#03#04#05#06#07#08#09#0A#0B#0C#0D#0E#0F%
    #10#11#12#13#14#15#16#17#18#19#1A#1B#1C#1D#1E#1F%
    #20!"#23$#25&'#28#29*+,-.#2F%
    0123456789:;#3C=#3E?%
    @ABCDEFGHIJKLMNO%
    PQRSTUVWXYZ#5B\ltx@backslashchar#5D^_%
    `abcdefghijklmno%
    pqrstuvwxyz#7B|#7D\string~#7F%
    #80#81#82#83#84#85#86#87#88#89#8A#8B#8C#8D#8E#8F%
    #90#91#92#93#94#95#96#97#98#99#9A#9B#9C#9D#9E#9F%
    #A0#A1#A2#A3#A4#A5#A6#A7#A8#A9#AA#AB#AC#AD#AE#AF%
    #B0#B1#B2#B3#B4#B5#B6#B7#B8#B9#BA#BB#BC#BD#BE#BF%
    #C0#C1#C2#C3#C4#C5#C6#C7#C8#C9#CA#CB#CC#CD#CE#CF%
    #D0#D1#D2#D3#D4#D5#D6#D7#D8#D9#DA#DB#DC#DD#DE#DF%
    #E0#E1#E2#E3#E4#E5#E6#E7#E8#E9#EA#EB#EC#ED#EE#EF%
    #F0#F1#F2#F3#F4#F5#F6#F7#F8#F9#FA#FB#FC#FD#FE#FF%
  }%
\endgroup
\ltx@onelevel@sanitize\AllBytesName
\edef\AllBytesFromName{\expandafter\ltx@gobble\AllBytes}
\begingroup
  \def\|{|}%
  \edef\%{\ltx@percentchar}%
  \catcode`\|=0 %
  \catcode`\#=12 %
  \catcode`\~=12 %
  \catcode`\\=12 %
  |xdef|AllBytesString{%
    \000\001\002\003\004\005\006\007\010\011\012\013\014\015\016\017%
    \020\021\022\023\024\025\026\027\030\031\032\033\034\035\036\037%
    \040!"#$|%&'\(\)*+,-./%
    0123456789:;<=>?%
    @ABCDEFGHIJKLMNO%
    PQRSTUVWXYZ[\\]^_%
    `abcdefghijklmno%
    pqrstuvwxyz{||}~\177%
    \200\201\202\203\204\205\206\207\210\211\212\213\214\215\216\217%
    \220\221\222\223\224\225\226\227\230\231\232\233\234\235\236\237%
    \240\241\242\243\244\245\246\247\250\251\252\253\254\255\256\257%
    \260\261\262\263\264\265\266\267\270\271\272\273\274\275\276\277%
    \300\301\302\303\304\305\306\307\310\311\312\313\314\315\316\317%
    \320\321\322\323\324\325\326\327\330\331\332\333\334\335\336\337%
    \340\341\342\343\344\345\346\347\350\351\352\353\354\355\356\357%
    \360\361\362\363\364\365\366\367\370\371\372\373\374\375\376\377%
  }%
|endgroup
\ltx@onelevel@sanitize\AllBytesString
%    \end{macrocode}
%    \begin{macrocode}
\def\Test#1#2#3{%
  \begingroup
    \expandafter\expandafter\expandafter\def
    \expandafter\expandafter\expandafter\TestResult
    \expandafter\expandafter\expandafter{%
      #1{#2}%
    }%
    \ifx\TestResult#3%
    \else
      \newlinechar=10 %
      \msg{Expect:^^J#3}%
      \msg{Result:^^J\TestResult}%
      \errmessage{\string#2 -\string#1-> \string#3}%
    \fi
  \endgroup
}
\def\test#1#2#3{%
  \edef\TestFrom{#2}%
  \edef\TestExpect{#3}%
  \ltx@onelevel@sanitize\TestExpect
  \Test#1\TestFrom\TestExpect
}
\test\pdf@unescapehex{74657374}{test}
\begingroup
  \catcode0=12 %
  \catcode1=12 %
  \test\pdf@unescapehex{740074017400740174}{t^^@t^^At^^@t^^At}%
\endgroup
\Test\pdf@escapehex\AllBytes\AllBytesHex
\Test\pdf@unescapehex\AllBytesHex\AllBytes
\Test\pdf@escapename\AllBytes\AllBytesName
\Test\pdf@escapestring\AllBytes\AllBytesString
%    \end{macrocode}
%    \begin{macrocode}
\csname @@end\endcsname\end
%</test-escape>
%    \end{macrocode}
%
% \section{Installation}
%
% \subsection{Download}
%
% \paragraph{Package.} This package is available on
% CTAN\footnote{\CTANpkg{pdftexcmds}}:
% \begin{description}
% \item[\CTAN{macros/latex/contrib/oberdiek/pdftexcmds.dtx}] The source file.
% \item[\CTAN{macros/latex/contrib/oberdiek/pdftexcmds.pdf}] Documentation.
% \end{description}
%
%
% \paragraph{Bundle.} All the packages of the bundle `oberdiek'
% are also available in a TDS compliant ZIP archive. There
% the packages are already unpacked and the documentation files
% are generated. The files and directories obey the TDS standard.
% \begin{description}
% \item[\CTANinstall{install/macros/latex/contrib/oberdiek.tds.zip}]
% \end{description}
% \emph{TDS} refers to the standard ``A Directory Structure
% for \TeX\ Files'' (\CTAN{tds/tds.pdf}). Directories
% with \xfile{texmf} in their name are usually organized this way.
%
% \subsection{Bundle installation}
%
% \paragraph{Unpacking.} Unpack the \xfile{oberdiek.tds.zip} in the
% TDS tree (also known as \xfile{texmf} tree) of your choice.
% Example (linux):
% \begin{quote}
%   |unzip oberdiek.tds.zip -d ~/texmf|
% \end{quote}
%
% \paragraph{Script installation.}
% Check the directory \xfile{TDS:scripts/oberdiek/} for
% scripts that need further installation steps.
% Package \xpackage{attachfile2} comes with the Perl script
% \xfile{pdfatfi.pl} that should be installed in such a way
% that it can be called as \texttt{pdfatfi}.
% Example (linux):
% \begin{quote}
%   |chmod +x scripts/oberdiek/pdfatfi.pl|\\
%   |cp scripts/oberdiek/pdfatfi.pl /usr/local/bin/|
% \end{quote}
%
% \subsection{Package installation}
%
% \paragraph{Unpacking.} The \xfile{.dtx} file is a self-extracting
% \docstrip\ archive. The files are extracted by running the
% \xfile{.dtx} through \plainTeX:
% \begin{quote}
%   \verb|tex pdftexcmds.dtx|
% \end{quote}
%
% \paragraph{TDS.} Now the different files must be moved into
% the different directories in your installation TDS tree
% (also known as \xfile{texmf} tree):
% \begin{quote}
% \def\t{^^A
% \begin{tabular}{@{}>{\ttfamily}l@{ $\rightarrow$ }>{\ttfamily}l@{}}
%   pdftexcmds.sty & tex/generic/oberdiek/pdftexcmds.sty\\
%   oberdiek.pdftexcmds.lua & scripts/oberdiek/oberdiek.pdftexcmds.lua\\
%   pdftexcmds.lua & scripts/oberdiek/pdftexcmds.lua\\
%   pdftexcmds.pdf & doc/latex/oberdiek/pdftexcmds.pdf\\
%   test/pdftexcmds-test1.tex & doc/latex/oberdiek/test/pdftexcmds-test1.tex\\
%   test/pdftexcmds-test2.tex & doc/latex/oberdiek/test/pdftexcmds-test2.tex\\
%   test/pdftexcmds-test-shell.tex & doc/latex/oberdiek/test/pdftexcmds-test-shell.tex\\
%   test/pdftexcmds-test-escape.tex & doc/latex/oberdiek/test/pdftexcmds-test-escape.tex\\
%   pdftexcmds.dtx & source/latex/oberdiek/pdftexcmds.dtx\\
% \end{tabular}^^A
% }^^A
% \sbox0{\t}^^A
% \ifdim\wd0>\linewidth
%   \begingroup
%     \advance\linewidth by\leftmargin
%     \advance\linewidth by\rightmargin
%   \edef\x{\endgroup
%     \def\noexpand\lw{\the\linewidth}^^A
%   }\x
%   \def\lwbox{^^A
%     \leavevmode
%     \hbox to \linewidth{^^A
%       \kern-\leftmargin\relax
%       \hss
%       \usebox0
%       \hss
%       \kern-\rightmargin\relax
%     }^^A
%   }^^A
%   \ifdim\wd0>\lw
%     \sbox0{\small\t}^^A
%     \ifdim\wd0>\linewidth
%       \ifdim\wd0>\lw
%         \sbox0{\footnotesize\t}^^A
%         \ifdim\wd0>\linewidth
%           \ifdim\wd0>\lw
%             \sbox0{\scriptsize\t}^^A
%             \ifdim\wd0>\linewidth
%               \ifdim\wd0>\lw
%                 \sbox0{\tiny\t}^^A
%                 \ifdim\wd0>\linewidth
%                   \lwbox
%                 \else
%                   \usebox0
%                 \fi
%               \else
%                 \lwbox
%               \fi
%             \else
%               \usebox0
%             \fi
%           \else
%             \lwbox
%           \fi
%         \else
%           \usebox0
%         \fi
%       \else
%         \lwbox
%       \fi
%     \else
%       \usebox0
%     \fi
%   \else
%     \lwbox
%   \fi
% \else
%   \usebox0
% \fi
% \end{quote}
% If you have a \xfile{docstrip.cfg} that configures and enables \docstrip's
% TDS installing feature, then some files can already be in the right
% place, see the documentation of \docstrip.
%
% \subsection{Refresh file name databases}
%
% If your \TeX~distribution
% (\teTeX, \mikTeX, \dots) relies on file name databases, you must refresh
% these. For example, \teTeX\ users run \verb|texhash| or
% \verb|mktexlsr|.
%
% \subsection{Some details for the interested}
%
% \paragraph{Unpacking with \LaTeX.}
% The \xfile{.dtx} chooses its action depending on the format:
% \begin{description}
% \item[\plainTeX:] Run \docstrip\ and extract the files.
% \item[\LaTeX:] Generate the documentation.
% \end{description}
% If you insist on using \LaTeX\ for \docstrip\ (really,
% \docstrip\ does not need \LaTeX), then inform the autodetect routine
% about your intention:
% \begin{quote}
%   \verb|latex \let\install=y\input{pdftexcmds.dtx}|
% \end{quote}
% Do not forget to quote the argument according to the demands
% of your shell.
%
% \paragraph{Generating the documentation.}
% You can use both the \xfile{.dtx} or the \xfile{.drv} to generate
% the documentation. The process can be configured by the
% configuration file \xfile{ltxdoc.cfg}. For instance, put this
% line into this file, if you want to have A4 as paper format:
% \begin{quote}
%   \verb|\PassOptionsToClass{a4paper}{article}|
% \end{quote}
% An example follows how to generate the
% documentation with pdf\LaTeX:
% \begin{quote}
%\begin{verbatim}
%pdflatex pdftexcmds.dtx
%bibtex pdftexcmds.aux
%makeindex -s gind.ist pdftexcmds.idx
%pdflatex pdftexcmds.dtx
%makeindex -s gind.ist pdftexcmds.idx
%pdflatex pdftexcmds.dtx
%\end{verbatim}
% \end{quote}
%
% \printbibliography[
%   heading=bibnumbered,
% ]
%
% \begin{History}
%   \begin{Version}{2007/11/11 v0.1}
%   \item
%     First version.
%   \end{Version}
%   \begin{Version}{2007/11/12 v0.2}
%   \item
%     Short description fixed.
%   \end{Version}
%   \begin{Version}{2007/12/12 v0.3}
%   \item
%     Organization of Lua code as module.
%   \end{Version}
%   \begin{Version}{2009/04/10 v0.4}
%   \item
%     Adaptation for syntax change of \cs{directlua} in
%     \hologo{LuaTeX} 0.36.
%   \end{Version}
%   \begin{Version}{2009/09/22 v0.5}
%   \item
%     \cs{pdf@primitive}, \cs{pdf@ifprimitive} added.
%   \item
%     \hologo{XeTeX}'s variants are detected for
%     \cs{pdf@shellescape}, \cs{pdf@strcmp}, \cs{pdf@primitive},
%     \cs{pdf@ifprimitive}.
%   \end{Version}
%   \begin{Version}{2009/09/23 v0.6}
%   \item
%     Macro \cs{pdf@isprimitive} added.
%   \end{Version}
%   \begin{Version}{2009/12/12 v0.7}
%   \item
%     Short info shortened.
%   \end{Version}
%   \begin{Version}{2010/03/01 v0.8}
%   \item
%     Required date for package \xpackage{ifluatex} updated.
%   \end{Version}
%   \begin{Version}{2010/04/01 v0.9}
%   \item
%     Use \cs{ifeof18} for defining \cs{pdf@shellescape} between
%     \hologo{pdfTeX} 1.21a (inclusive) and 1.30.0 (exclusive).
%   \end{Version}
%   \begin{Version}{2010/11/04 v0.10}
%   \item
%     \cs{pdf@draftmode}, \cs{pdf@ifdraftmode} and
%     \cs{pdf@setdraftmode} added.
%   \end{Version}
%   \begin{Version}{2010/11/11 v0.11}
%   \item
%     Missing \cs{RequirePackage} for package \xpackage{ifpdf} added.
%   \end{Version}
%   \begin{Version}{2011/01/30 v0.12}
%   \item
%     Already loaded package files are not input in \hologo{plainTeX}.
%   \end{Version}
%   \begin{Version}{2011/03/04 v0.13}
%   \item
%     Improved Lua function \texttt{shellescape} that also
%     uses the result of \texttt{os.execute()} (thanks to Philipp Stephani).
%   \end{Version}
%   \begin{Version}{2011/04/10 v0.14}
%   \item
%     Version check of loaded module added.
%   \item
%     Patch for bug in \hologo{LuaTeX} between 0.40.6 and 0.65 that
%     is fixed in revision 4096.
%   \end{Version}
%   \begin{Version}{2011/04/16 v0.15}
%   \item
%     \hologo{LuaTeX}: \cs{pdf@shellescape} is only supported
%     for version 0.70.0 and higher due to a bug, \texttt{os.execute()}
%     crashes in some circumstances. Fixed in \hologo{LuaTeX}
%     beta-0.70.0, revision 4167.
%   \end{Version}
%   \begin{Version}{2011/04/22 v0.16}
%   \item
%     Previous fix was not working due to a wrong catcode of digit
%     zero (due to easily support the old \cs{directlua0}).
%     The version border is lowered to 0.68, because some
%     beta-0.67.0 seems also to work.
%   \end{Version}
%   \begin{Version}{2011/06/29 v0.17}
%   \item
%     Documentation addition to \cs{pdf@shellescape}.
%   \end{Version}
%   \begin{Version}{2011/07/01 v0.18}
%   \item
%     Add Lua module loading in \cs{everyjob} for \hologo{iniTeX}
%     (\hologo{LuaTeX} only).
%   \end{Version}
%   \begin{Version}{2011/07/28 v0.19}
%   \item
%     Missing space in an info message added (Martin M\"unch).
%   \end{Version}
%   \begin{Version}{2011/11/29 v0.20}
%   \item
%     \cs{pdf@resettimer} and \cs{pdf@elapsedtime} added
%     (thanks Andy Thomas).
%   \end{Version}
%   \begin{Version}{2016/05/10 v0.21}
%   \item
%      local unpack added
%     (thanks \'{E}lie Roux).
%   \end{Version}
%   \begin{Version}{2016/05/21 v0.22}
%   \item
%     adjust \cs{textbackslas}h usage in bib file for biber bug.
%   \end{Version}
%   \begin{Version}{2016/10/02 v0.23}
%   \item
%     add file.close to lua filehandles (github pull request).
%   \end{Version}
%   \begin{Version}{2017/01/29 v0.24}
%   \item
%     Avoid loading luatex-loader for current luatex. (Use
%     pdftexcmds.lua not oberdiek.pdftexcmds.lua to simplify file
%     search with standard require)
%   \end{Version}
%   \begin{Version}{2017/03/19 v0.25}
%   \item
%     New \cs{pdf@shellescape} for Lua\TeX, see github issue 20.
%   \end{Version}
%   \begin{Version}{2018/01/21 v0.26}
%   \item
%     use rb not r mode for file open github issue 34.
%   \end{Version}
%   \begin{Version}{2018/01/30 v0.27}
%   \item
%     \cs{pdf@mdfivesum} for \hologo{XeTeX}
%   \end{Version}
%   \begin{Version}{2018/09/07 v0.28}
%   \item
%     Fix catcode regime in luatex sprint for \cs{pdf@shellescape} GH issue 45
%   \end{Version}
%   \begin{Version}{2018/09/10 v0.29}
%   \item
%     Actually do the fix described above in the code, not just document it.
%   \end{Version}
%   \begin{Version}{2019/07/25 v0.30}
%   \item
%     remove uses of module function, see PR70
%   \end{Version}
% \end{History}
%
% \PrintIndex
%
% \Finale
\endinput
|
% \end{quote}
% Do not forget to quote the argument according to the demands
% of your shell.
%
% \paragraph{Generating the documentation.}
% You can use both the \xfile{.dtx} or the \xfile{.drv} to generate
% the documentation. The process can be configured by the
% configuration file \xfile{ltxdoc.cfg}. For instance, put this
% line into this file, if you want to have A4 as paper format:
% \begin{quote}
%   \verb|\PassOptionsToClass{a4paper}{article}|
% \end{quote}
% An example follows how to generate the
% documentation with pdf\LaTeX:
% \begin{quote}
%\begin{verbatim}
%pdflatex pdftexcmds.dtx
%bibtex pdftexcmds.aux
%makeindex -s gind.ist pdftexcmds.idx
%pdflatex pdftexcmds.dtx
%makeindex -s gind.ist pdftexcmds.idx
%pdflatex pdftexcmds.dtx
%\end{verbatim}
% \end{quote}
%
% \printbibliography[
%   heading=bibnumbered,
% ]
%
% \begin{History}
%   \begin{Version}{2007/11/11 v0.1}
%   \item
%     First version.
%   \end{Version}
%   \begin{Version}{2007/11/12 v0.2}
%   \item
%     Short description fixed.
%   \end{Version}
%   \begin{Version}{2007/12/12 v0.3}
%   \item
%     Organization of Lua code as module.
%   \end{Version}
%   \begin{Version}{2009/04/10 v0.4}
%   \item
%     Adaptation for syntax change of \cs{directlua} in
%     \hologo{LuaTeX} 0.36.
%   \end{Version}
%   \begin{Version}{2009/09/22 v0.5}
%   \item
%     \cs{pdf@primitive}, \cs{pdf@ifprimitive} added.
%   \item
%     \hologo{XeTeX}'s variants are detected for
%     \cs{pdf@shellescape}, \cs{pdf@strcmp}, \cs{pdf@primitive},
%     \cs{pdf@ifprimitive}.
%   \end{Version}
%   \begin{Version}{2009/09/23 v0.6}
%   \item
%     Macro \cs{pdf@isprimitive} added.
%   \end{Version}
%   \begin{Version}{2009/12/12 v0.7}
%   \item
%     Short info shortened.
%   \end{Version}
%   \begin{Version}{2010/03/01 v0.8}
%   \item
%     Required date for package \xpackage{ifluatex} updated.
%   \end{Version}
%   \begin{Version}{2010/04/01 v0.9}
%   \item
%     Use \cs{ifeof18} for defining \cs{pdf@shellescape} between
%     \hologo{pdfTeX} 1.21a (inclusive) and 1.30.0 (exclusive).
%   \end{Version}
%   \begin{Version}{2010/11/04 v0.10}
%   \item
%     \cs{pdf@draftmode}, \cs{pdf@ifdraftmode} and
%     \cs{pdf@setdraftmode} added.
%   \end{Version}
%   \begin{Version}{2010/11/11 v0.11}
%   \item
%     Missing \cs{RequirePackage} for package \xpackage{ifpdf} added.
%   \end{Version}
%   \begin{Version}{2011/01/30 v0.12}
%   \item
%     Already loaded package files are not input in \hologo{plainTeX}.
%   \end{Version}
%   \begin{Version}{2011/03/04 v0.13}
%   \item
%     Improved Lua function \texttt{shellescape} that also
%     uses the result of \texttt{os.execute()} (thanks to Philipp Stephani).
%   \end{Version}
%   \begin{Version}{2011/04/10 v0.14}
%   \item
%     Version check of loaded module added.
%   \item
%     Patch for bug in \hologo{LuaTeX} between 0.40.6 and 0.65 that
%     is fixed in revision 4096.
%   \end{Version}
%   \begin{Version}{2011/04/16 v0.15}
%   \item
%     \hologo{LuaTeX}: \cs{pdf@shellescape} is only supported
%     for version 0.70.0 and higher due to a bug, \texttt{os.execute()}
%     crashes in some circumstances. Fixed in \hologo{LuaTeX}
%     beta-0.70.0, revision 4167.
%   \end{Version}
%   \begin{Version}{2011/04/22 v0.16}
%   \item
%     Previous fix was not working due to a wrong catcode of digit
%     zero (due to easily support the old \cs{directlua0}).
%     The version border is lowered to 0.68, because some
%     beta-0.67.0 seems also to work.
%   \end{Version}
%   \begin{Version}{2011/06/29 v0.17}
%   \item
%     Documentation addition to \cs{pdf@shellescape}.
%   \end{Version}
%   \begin{Version}{2011/07/01 v0.18}
%   \item
%     Add Lua module loading in \cs{everyjob} for \hologo{iniTeX}
%     (\hologo{LuaTeX} only).
%   \end{Version}
%   \begin{Version}{2011/07/28 v0.19}
%   \item
%     Missing space in an info message added (Martin M\"unch).
%   \end{Version}
%   \begin{Version}{2011/11/29 v0.20}
%   \item
%     \cs{pdf@resettimer} and \cs{pdf@elapsedtime} added
%     (thanks Andy Thomas).
%   \end{Version}
%   \begin{Version}{2016/05/10 v0.21}
%   \item
%      local unpack added
%     (thanks \'{E}lie Roux).
%   \end{Version}
%   \begin{Version}{2016/05/21 v0.22}
%   \item
%     adjust \cs{textbackslas}h usage in bib file for biber bug.
%   \end{Version}
%   \begin{Version}{2016/10/02 v0.23}
%   \item
%     add file.close to lua filehandles (github pull request).
%   \end{Version}
%   \begin{Version}{2017/01/29 v0.24}
%   \item
%     Avoid loading luatex-loader for current luatex. (Use
%     pdftexcmds.lua not oberdiek.pdftexcmds.lua to simplify file
%     search with standard require)
%   \end{Version}
%   \begin{Version}{2017/03/19 v0.25}
%   \item
%     New \cs{pdf@shellescape} for Lua\TeX, see github issue 20.
%   \end{Version}
%   \begin{Version}{2018/01/21 v0.26}
%   \item
%     use rb not r mode for file open github issue 34.
%   \end{Version}
%   \begin{Version}{2018/01/30 v0.27}
%   \item
%     \cs{pdf@mdfivesum} for \hologo{XeTeX}
%   \end{Version}
%   \begin{Version}{2018/09/07 v0.28}
%   \item
%     Fix catcode regime in luatex sprint for \cs{pdf@shellescape} GH issue 45
%   \end{Version}
%   \begin{Version}{2018/09/10 v0.29}
%   \item
%     Actually do the fix described above in the code, not just document it.
%   \end{Version}
%   \begin{Version}{2019/07/25 v0.30}
%   \item
%     remove uses of module function, see PR70
%   \end{Version}
% \end{History}
%
% \PrintIndex
%
% \Finale
\endinput
|
% \end{quote}
% Do not forget to quote the argument according to the demands
% of your shell.
%
% \paragraph{Generating the documentation.}
% You can use both the \xfile{.dtx} or the \xfile{.drv} to generate
% the documentation. The process can be configured by the
% configuration file \xfile{ltxdoc.cfg}. For instance, put this
% line into this file, if you want to have A4 as paper format:
% \begin{quote}
%   \verb|\PassOptionsToClass{a4paper}{article}|
% \end{quote}
% An example follows how to generate the
% documentation with pdf\LaTeX:
% \begin{quote}
%\begin{verbatim}
%pdflatex pdftexcmds.dtx
%bibtex pdftexcmds.aux
%makeindex -s gind.ist pdftexcmds.idx
%pdflatex pdftexcmds.dtx
%makeindex -s gind.ist pdftexcmds.idx
%pdflatex pdftexcmds.dtx
%\end{verbatim}
% \end{quote}
%
% \printbibliography[
%   heading=bibnumbered,
% ]
%
% \begin{History}
%   \begin{Version}{2007/11/11 v0.1}
%   \item
%     First version.
%   \end{Version}
%   \begin{Version}{2007/11/12 v0.2}
%   \item
%     Short description fixed.
%   \end{Version}
%   \begin{Version}{2007/12/12 v0.3}
%   \item
%     Organization of Lua code as module.
%   \end{Version}
%   \begin{Version}{2009/04/10 v0.4}
%   \item
%     Adaptation for syntax change of \cs{directlua} in
%     \hologo{LuaTeX} 0.36.
%   \end{Version}
%   \begin{Version}{2009/09/22 v0.5}
%   \item
%     \cs{pdf@primitive}, \cs{pdf@ifprimitive} added.
%   \item
%     \hologo{XeTeX}'s variants are detected for
%     \cs{pdf@shellescape}, \cs{pdf@strcmp}, \cs{pdf@primitive},
%     \cs{pdf@ifprimitive}.
%   \end{Version}
%   \begin{Version}{2009/09/23 v0.6}
%   \item
%     Macro \cs{pdf@isprimitive} added.
%   \end{Version}
%   \begin{Version}{2009/12/12 v0.7}
%   \item
%     Short info shortened.
%   \end{Version}
%   \begin{Version}{2010/03/01 v0.8}
%   \item
%     Required date for package \xpackage{ifluatex} updated.
%   \end{Version}
%   \begin{Version}{2010/04/01 v0.9}
%   \item
%     Use \cs{ifeof18} for defining \cs{pdf@shellescape} between
%     \hologo{pdfTeX} 1.21a (inclusive) and 1.30.0 (exclusive).
%   \end{Version}
%   \begin{Version}{2010/11/04 v0.10}
%   \item
%     \cs{pdf@draftmode}, \cs{pdf@ifdraftmode} and
%     \cs{pdf@setdraftmode} added.
%   \end{Version}
%   \begin{Version}{2010/11/11 v0.11}
%   \item
%     Missing \cs{RequirePackage} for package \xpackage{ifpdf} added.
%   \end{Version}
%   \begin{Version}{2011/01/30 v0.12}
%   \item
%     Already loaded package files are not input in \hologo{plainTeX}.
%   \end{Version}
%   \begin{Version}{2011/03/04 v0.13}
%   \item
%     Improved Lua function \texttt{shellescape} that also
%     uses the result of \texttt{os.execute()} (thanks to Philipp Stephani).
%   \end{Version}
%   \begin{Version}{2011/04/10 v0.14}
%   \item
%     Version check of loaded module added.
%   \item
%     Patch for bug in \hologo{LuaTeX} between 0.40.6 and 0.65 that
%     is fixed in revision 4096.
%   \end{Version}
%   \begin{Version}{2011/04/16 v0.15}
%   \item
%     \hologo{LuaTeX}: \cs{pdf@shellescape} is only supported
%     for version 0.70.0 and higher due to a bug, \texttt{os.execute()}
%     crashes in some circumstances. Fixed in \hologo{LuaTeX}
%     beta-0.70.0, revision 4167.
%   \end{Version}
%   \begin{Version}{2011/04/22 v0.16}
%   \item
%     Previous fix was not working due to a wrong catcode of digit
%     zero (due to easily support the old \cs{directlua0}).
%     The version border is lowered to 0.68, because some
%     beta-0.67.0 seems also to work.
%   \end{Version}
%   \begin{Version}{2011/06/29 v0.17}
%   \item
%     Documentation addition to \cs{pdf@shellescape}.
%   \end{Version}
%   \begin{Version}{2011/07/01 v0.18}
%   \item
%     Add Lua module loading in \cs{everyjob} for \hologo{iniTeX}
%     (\hologo{LuaTeX} only).
%   \end{Version}
%   \begin{Version}{2011/07/28 v0.19}
%   \item
%     Missing space in an info message added (Martin M\"unch).
%   \end{Version}
%   \begin{Version}{2011/11/29 v0.20}
%   \item
%     \cs{pdf@resettimer} and \cs{pdf@elapsedtime} added
%     (thanks Andy Thomas).
%   \end{Version}
%   \begin{Version}{2016/05/10 v0.21}
%   \item
%      local unpack added
%     (thanks \'{E}lie Roux).
%   \end{Version}
%   \begin{Version}{2016/05/21 v0.22}
%   \item
%     adjust \cs{textbackslas}h usage in bib file for biber bug.
%   \end{Version}
%   \begin{Version}{2016/10/02 v0.23}
%   \item
%     add file.close to lua filehandles (github pull request).
%   \end{Version}
%   \begin{Version}{2017/01/29 v0.24}
%   \item
%     Avoid loading luatex-loader for current luatex. (Use
%     pdftexcmds.lua not oberdiek.pdftexcmds.lua to simplify file
%     search with standard require)
%   \end{Version}
%   \begin{Version}{2017/03/19 v0.25}
%   \item
%     New \cs{pdf@shellescape} for Lua\TeX, see github issue 20.
%   \end{Version}
%   \begin{Version}{2018/01/21 v0.26}
%   \item
%     use rb not r mode for file open github issue 34.
%   \end{Version}
%   \begin{Version}{2018/01/30 v0.27}
%   \item
%     \cs{pdf@mdfivesum} for \hologo{XeTeX}
%   \end{Version}
%   \begin{Version}{2018/09/07 v0.28}
%   \item
%     Fix catcode regime in luatex sprint for \cs{pdf@shellescape} GH issue 45
%   \end{Version}
%   \begin{Version}{2018/09/10 v0.29}
%   \item
%     Actually do the fix described above in the code, not just document it.
%   \end{Version}
%   \begin{Version}{2019/07/25 v0.30}
%   \item
%     remove uses of module function, see PR70
%   \end{Version}
% \end{History}
%
% \PrintIndex
%
% \Finale
\endinput

%        (quote the arguments according to the demands of your shell)
%
% Documentation:
%    (a) If pdftexcmds.drv is present:
%           latex pdftexcmds.drv
%    (b) Without pdftexcmds.drv:
%           latex pdftexcmds.dtx; ...
%    The class ltxdoc loads the configuration file ltxdoc.cfg
%    if available. Here you can specify further options, e.g.
%    use A4 as paper format:
%       \PassOptionsToClass{a4paper}{article}
%
%    Programm calls to get the documentation (example):
%       pdflatex pdftexcmds.dtx
%       bibtex pdftexcmds.aux
%       makeindex -s gind.ist pdftexcmds.idx
%       pdflatex pdftexcmds.dtx
%       makeindex -s gind.ist pdftexcmds.idx
%       pdflatex pdftexcmds.dtx
%
% Installation:
%    TDS:tex/generic/oberdiek/pdftexcmds.sty
%    TDS:scripts/oberdiek/oberdiek.pdftexcmds.lua
%    TDS:scripts/oberdiek/pdftexcmds.lua
%    TDS:doc/latex/oberdiek/pdftexcmds.pdf
%    TDS:doc/latex/oberdiek/test/pdftexcmds-test1.tex
%    TDS:doc/latex/oberdiek/test/pdftexcmds-test2.tex
%    TDS:doc/latex/oberdiek/test/pdftexcmds-test-shell.tex
%    TDS:doc/latex/oberdiek/test/pdftexcmds-test-escape.tex
%    TDS:source/latex/oberdiek/pdftexcmds.dtx
%
%<*ignore>
\begingroup
  \catcode123=1 %
  \catcode125=2 %
  \def\x{LaTeX2e}%
\expandafter\endgroup
\ifcase 0\ifx\install y1\fi\expandafter
         \ifx\csname processbatchFile\endcsname\relax\else1\fi
         \ifx\fmtname\x\else 1\fi\relax
\else\csname fi\endcsname
%</ignore>
%<*install>
\input docstrip.tex
\Msg{************************************************************************}
\Msg{* Installation}
\Msg{* Package: pdftexcmds 2019/07/25 v0.30 Utility functions of pdfTeX for LuaTeX (HO)}
\Msg{************************************************************************}

\keepsilent
\askforoverwritefalse

\let\MetaPrefix\relax
\preamble

This is a generated file.

Project: pdftexcmds
Version: 2019/07/25 v0.30

Copyright (C) 2007, 2009-2011 by
   Heiko Oberdiek <heiko.oberdiek at googlemail.com>

This work may be distributed and/or modified under the
conditions of the LaTeX Project Public License, either
version 1.3c of this license or (at your option) any later
version. This version of this license is in
   https://www.latex-project.org/lppl/lppl-1-3c.txt
and the latest version of this license is in
   https://www.latex-project.org/lppl.txt
and version 1.3 or later is part of all distributions of
LaTeX version 2005/12/01 or later.

This work has the LPPL maintenance status "maintained".

The Current Maintainers of this work are
Heiko Oberdiek and the Oberdiek Package Support Group
https://github.com/ho-tex/oberdiek/issues


The Base Interpreter refers to any `TeX-Format',
because some files are installed in TDS:tex/generic//.

This work consists of the main source file pdftexcmds.dtx
and the derived files
   pdftexcmds.sty, pdftexcmds.pdf, pdftexcmds.ins, pdftexcmds.drv,
   pdftexcmds.bib, pdftexcmds-test1.tex, pdftexcmds-test2.tex,
   pdftexcmds-test-shell.tex, pdftexcmds-test-escape.tex,
   oberdiek.pdftexcmds.lua, pdftexcmds.lua.

\endpreamble
\let\MetaPrefix\DoubleperCent

\generate{%
  \file{pdftexcmds.ins}{\from{pdftexcmds.dtx}{install}}%
  \file{pdftexcmds.drv}{\from{pdftexcmds.dtx}{driver}}%
  \nopreamble
  \nopostamble
  \file{pdftexcmds.bib}{\from{pdftexcmds.dtx}{bib}}%
  \usepreamble\defaultpreamble
  \usepostamble\defaultpostamble
  \usedir{tex/generic/oberdiek}%
  \file{pdftexcmds.sty}{\from{pdftexcmds.dtx}{package}}%
%  \usedir{doc/latex/oberdiek/test}%
%  \file{pdftexcmds-test1.tex}{\from{pdftexcmds.dtx}{test1}}%
%  \file{pdftexcmds-test2.tex}{\from{pdftexcmds.dtx}{test2}}%
%  \file{pdftexcmds-test-shell.tex}{\from{pdftexcmds.dtx}{test-shell}}%
%  \file{pdftexcmds-test-escape.tex}{\from{pdftexcmds.dtx}{test-escape}}%
  \nopreamble
  \nopostamble
%  \usedir{source/latex/oberdiek/catalogue}%
%  \file{pdftexcmds.xml}{\from{pdftexcmds.dtx}{catalogue}}%
}
\def\MetaPrefix{-- }
\def\defaultpostamble{%
  \MetaPrefix^^J%
  \MetaPrefix\space End of File `\outFileName'.%
}
\def\currentpostamble{\defaultpostamble}%
\generate{%
  \usedir{scripts/oberdiek}%
  \file{oberdiek.pdftexcmds.lua}{\from{pdftexcmds.dtx}{lua}}%
  \file{pdftexcmds.lua}{\from{pdftexcmds.dtx}{lua}}%
}

\catcode32=13\relax% active space
\let =\space%
\Msg{************************************************************************}
\Msg{*}
\Msg{* To finish the installation you have to move the following}
\Msg{* file into a directory searched by TeX:}
\Msg{*}
\Msg{*     pdftexcmds.sty}
\Msg{*}
\Msg{* And install the following script files:}
\Msg{*}
\Msg{*     oberdiek.pdftexcmds.lua, pdftexcmds.lua}
\Msg{*}
\Msg{* To produce the documentation run the file `pdftexcmds.drv'}
\Msg{* through LaTeX.}
\Msg{*}
\Msg{* Happy TeXing!}
\Msg{*}
\Msg{************************************************************************}

\endbatchfile
%</install>
%<*bib>
@online{AndyThomas:Analog,
  author={Thomas, Andy},
  title={Analog of {\texttt{\csname textbackslash\endcsname}pdfelapsedtime} for
      {\hologo{LuaTeX}} and {\hologo{XeTeX}}},
  url={http://tex.stackexchange.com/a/32531},
  urldate={2011-11-29},
}
%</bib>
%<*ignore>
\fi
%</ignore>
%<*driver>
\NeedsTeXFormat{LaTeX2e}
\ProvidesFile{pdftexcmds.drv}%
  [2019/07/25 v0.30 Utility functions of pdfTeX for LuaTeX (HO)]%
\documentclass{ltxdoc}
\usepackage{holtxdoc}[2011/11/22]
\usepackage{paralist}
\usepackage{csquotes}
\usepackage[
  backend=bibtex,
  bibencoding=ascii,
  alldates=iso8601,
]{biblatex}[2011/11/13]
\bibliography{oberdiek-source}
\bibliography{pdftexcmds}
\begin{document}
  \DocInput{pdftexcmds.dtx}%
\end{document}
%</driver>
% \fi
%
%
% \CharacterTable
%  {Upper-case    \A\B\C\D\E\F\G\H\I\J\K\L\M\N\O\P\Q\R\S\T\U\V\W\X\Y\Z
%   Lower-case    \a\b\c\d\e\f\g\h\i\j\k\l\m\n\o\p\q\r\s\t\u\v\w\x\y\z
%   Digits        \0\1\2\3\4\5\6\7\8\9
%   Exclamation   \!     Double quote  \"     Hash (number) \#
%   Dollar        \$     Percent       \%     Ampersand     \&
%   Acute accent  \'     Left paren    \(     Right paren   \)
%   Asterisk      \*     Plus          \+     Comma         \,
%   Minus         \-     Point         \.     Solidus       \/
%   Colon         \:     Semicolon     \;     Less than     \<
%   Equals        \=     Greater than  \>     Question mark \?
%   Commercial at \@     Left bracket  \[     Backslash     \\
%   Right bracket \]     Circumflex    \^     Underscore    \_
%   Grave accent  \`     Left brace    \{     Vertical bar  \|
%   Right brace   \}     Tilde         \~}
%
% \GetFileInfo{pdftexcmds.drv}
%
% \title{The \xpackage{pdftexcmds} package}
% \date{2019/07/25 v0.30}
% \author{Heiko Oberdiek\thanks
% {Please report any issues at \url{https://github.com/ho-tex/oberdiek/issues}}}
%
% \maketitle
%
% \begin{abstract}
% \hologo{LuaTeX} provides most of the commands of \hologo{pdfTeX} 1.40. However
% a number of utility functions are removed. This package tries to fill
% the gap and implements some of the missing primitive using Lua.
% \end{abstract}
%
% \tableofcontents
%
% \def\csi#1{\texttt{\textbackslash\textit{#1}}}
%
% \section{Documentation}
%
% Some primitives of \hologo{pdfTeX} \cite{pdftex-manual}
% are not defined by \hologo{LuaTeX} \cite{luatex-manual}.
% This package implements macro based solutions using Lua code
% for the following missing \hologo{pdfTeX} primitives;
% \begin{compactitem}
% \item \cs{pdfstrcmp}
% \item \cs{pdfunescapehex}
% \item \cs{pdfescapehex}
% \item \cs{pdfescapename}
% \item \cs{pdfescapestring}
% \item \cs{pdffilesize}
% \item \cs{pdffilemoddate}
% \item \cs{pdffiledump}
% \item \cs{pdfmdfivesum}
% \item \cs{pdfresettimer}
% \item \cs{pdfelapsedtime}
% \item |\immediate\write18|
% \end{compactitem}
% The original names of the primitives cannot be used:
% \begin{itemize}
% \item
% The syntax for their arguments cannot easily
% simulated by macros. The primitives using key words
% such as |file| (\cs{pdfmdfivesum}) or |offset| and |length|
% (\cs{pdffiledump}) and uses \meta{general text} for the other
% arguments. Using token registers assignments, \meta{general text} could
% be catched. However, the simulated primitives are expandable
% and register assignments would destroy this important property.
% (\meta{general text} allows something like |\expandafter\bgroup ...}|.)
% \item
% The original primitives can be expanded using one expansion step.
% The new macros need two expansion steps because of the additional
% macro expansion. Example:
% \begin{quote}
%   |\expandafter\foo\pdffilemoddate{file}|\\
%   vs.\\
%   |\expandafter\expandafter\expandafter|\\
%   |\foo\pdf@filemoddate{file}|
% \end{quote}
% \end{itemize}
%
% \hologo{LuaTeX} isn't stable yet and thus the status of this package is
% \emph{experimental}. Feedback is welcome.
%
% \subsection{General principles}
%
% \begin{description}
% \item[Naming convention:]
%   Usually this package defines a macro |\pdf@|\meta{cmd} if
%   \hologo{pdfTeX} provides |\pdf|\meta{cmd}.
% \item[Arguments:] The order of arguments in |\pdf@|\meta{cmd}
%   is the same as for the corresponding primitive of \hologo{pdfTeX}.
%   The arguments are ordinary undelimited \hologo{TeX} arguments,
%   no \meta{general text} and without additional keywords.
% \item[Expandibility:]
%   The macro |\pdf@|\meta{cmd} is expandable if the
%   corresponding \hologo{pdfTeX} primitive has this property.
%   Exact two expansion steps are necessary (first is the macro
%   expansion) except for \cs{pdf@primitive} and \cs{pdf@ifprimitive}.
%   The latter ones are not macros, but have the direct meaning of the
%   primitive.
% \item[Without \hologo{LuaTeX}:]
%   The macros |\pdf@|\meta{cmd} are mapped to the commands
%   of \hologo{pdfTeX} if they are available. Otherwise they are undefined.
% \item[Availability:]
%   The macros that the packages provides are undefined, if
%   the necessary primitives are not found and cannot be
%   implemented by Lua.
% \end{description}
%
% \subsection{Macros}
%
% \subsubsection[Strings]{Strings \cite[``7.15 Strings'']{pdftex-manual}}
%
% \begin{declcs}{pdf@strcmp} \M{stringA} \M{stringB}
% \end{declcs}
% Same as |\pdfstrcmp{|\meta{stringA}|}{|\meta{stringB}|}|.
%
% \begin{declcs}{pdf@unescapehex} \M{string}
% \end{declcs}
% Same as |\pdfunescapehex{|\meta{string}|}|.
% The argument is a byte string given in hexadecimal notation.
% The result are character tokens from 0 until 255 with
% catcode 12 and the space with catcode 10.
%
% \begin{declcs}{pdf@escapehex} \M{string}\\
%   \cs{pdf@escapestring} \M{string}\\
%   \cs{pdf@escapename} \M{string}
% \end{declcs}
% Same as the primitives of \hologo{pdfTeX}. However \hologo{pdfTeX} does not
% know about characters with codes 256 and larger. Thus the
% string is treated as byte string, characters with more than
% eight bits are ignored.
%
% \subsubsection[Files]{Files \cite[``7.18 Files'']{pdftex-manual}}
%
% \begin{declcs}{pdf@filesize} \M{filename}
% \end{declcs}
% Same as |\pdffilesize{|\meta{filename}|}|.
%
% \begin{declcs}{pdf@filemoddate} \M{filename}
% \end{declcs}
% Same as |\pdffilemoddate{|\meta{filename}|}|.
%
% \begin{declcs}{pdf@filedump} \M{offset} \M{length} \M{filename}
% \end{declcs}
% Same as |\pdffiledump offset| \meta{offset} |length| \meta{length}
% |{|\meta{filename}|}|. Both \meta{offset} and \meta{length} must
% not be empty, but must be a valid \hologo{TeX} number.
%
% \begin{declcs}{pdf@mdfivesum} \M{string}
% \end{declcs}
% Same as |\pdfmdfivesum{|\meta{string}|}|. Keyword |file| is supported
% by macro \cs{pdf@filemdfivesum}.
%
% \begin{declcs}{pdf@filemdfivesum} \M{filename}
% \end{declcs}
% Same as |\pdfmdfivesum file{|\meta{filename}|}|.
%
% \subsubsection[Timekeeping]{Timekeeping \cite[``7.17 Timekeeping'']{pdftex-manual}}
%
% The timekeeping macros are based on Andy Thomas' work \cite{AndyThomas:Analog}.
%
% \begin{declcs}{pdf@resettimer}
% \end{declcs}
% Same as \cs{pdfresettimer}, it resets the internal timer.
%
% \begin{declcs}{pdf@elapsedtime}
% \end{declcs}
% Same as \cs{pdfelapsedtime}. It behaves like a read-only integer.
% For printing purposes it can be prefixed by \cs{the} or \cs{number}.
% It measures the time in scaled seconds (seconds multiplied with 65536)
% since the latest call of \cs{pdf@resettimer} or start of
% program/package. The resolution, the shortest time interval that
% can be measured, depends on the program and system.
% \begin{itemize}
% \item \hologo{pdfTeX} with |gettimeofday|: $\ge$ 1/65536\,s
% \item \hologo{pdfTeX} with |ftime|: $\ge$ 1\,ms
% \item \hologo{pdfTeX} with |time|: $\ge$ 1\,s
% \item \hologo{LuaTeX}: $\ge$ 10\,ms\\
%  (|os.clock()| returns a float number with two decimal digits in
%  \hologo{LuaTeX} beta-0.70.1-2011061416 (rev 4277)).
% \end{itemize}
%
% \subsubsection[Miscellaneous]{Miscellaneous \cite[``7.21 Miscellaneous'']{pdftex-manual}}
%
% \begin{declcs}{pdf@draftmode}
% \end{declcs}
% If the \TeX\ compiler knows \cs{pdfdraftmode} or \cs{draftmode}
% (\hologo{pdfTeX},
% \hologo{LuaTeX}), then \cs{pdf@draftmode} returns, whether
% this mode is enabled. The result is an implicit number:
% one means the draft mode is available and enabled.
% If the value is zero, then the mode is not active or
% \cs{pdfdraftmode} is not available.
% An explicit number is yielded by \cs{number}\cs{pdf@draftmode}.
% The macro cannot
% be used to change the mode, see \cs{pdf@setdraftmode}.
%
% \begin{declcs}{pdf@ifdraftmode} \M{true} \M{false}
% \end{declcs}
% If \cs{pdfdraftmode} is available and enabled, \meta{true} is
% called, otherwise \meta{false} is executed.
%
% \begin{declcs}{pdf@setdraftmode} \M{value}
% \end{declcs}
% Macro \cs{pdf@setdraftmode} expects the number zero or one as
% \meta{value}. Zero deactivates the mode and one enables the draft mode.
% The macro does not have an effect, if the feature \cs{pdfdraftmode} is not
% available.
%
% \begin{declcs}{pdf@shellescape}
% \end{declcs}
% Same as |\pdfshellescape|. It is or expands to |1| if external
% commands can be executed and |0| otherwise. In \hologo{pdfTeX} external
% commands must be enabled first by command line option or
% configuration option. In \hologo{LuaTeX} option |--safer| disables
% the execution of external commands.
%
% In \hologo{LuaTeX} before 0.68.0 \cs{pdf@shellescape} is not
% available due to a bug in |os.execute()|. The argumentless form
% crashes in some circumstances with segmentation fault.
% (It is fixed in version 0.68.0 or revision 4167 of \hologo{LuaTeX}.
% and packported to some version of 0.67.0).
%
% Hints for usage:
% \begin{itemize}
% \item Before its use \cs{pdf@shellescape} should be tested,
% whether it is available. Example with package \xpackage{ltxcmds}
% (loaded by package \xpackage{pdftexcmds}):
%\begin{quote}
%\begin{verbatim}
%\ltx@IfUndefined{pdf@shellescape}{%
%  % \pdf@shellescape is undefined
%}{%
%  % \pdf@shellescape is available
%}
%\end{verbatim}
%\end{quote}
% Use \cs{ltx@ifundefined} in expandable contexts.
% \item \cs{pdf@shellescape} might be a numerical constant,
% expands to the primitive, or expands to a plain number.
% Therefore use it in contexts where these differences does not matter.
% \item Use in comparisons, e.g.:
%   \begin{quote}
%     |\ifnum\pdf@shellescape=0 ...|
%   \end{quote}
% \item Print the number: |\number\pdf@shellescape|
% \end{itemize}
%
% \begin{declcs}{pdf@system} \M{cmdline}
% \end{declcs}
% It is a wrapper for |\immediate\write18| in \hologo{pdfTeX} or
% |os.execute| in \hologo{LuaTeX}.
%
% In theory |os.execute|
% returns a status number. But its meaning is quite
% undefined. Are there some reliable properties?
% Does it make sense to provide an user interface to
% this status exit code?
%
% \begin{declcs}{pdf@primitive} \csi{cmd}
% \end{declcs}
% Same as \cs{pdfprimitive} in \hologo{pdfTeX} or \hologo{LuaTeX}.
% In \hologo{XeTeX} the
% primitive is called \cs{primitive}. Despite the current definition
% of the command \csi{cmd}, it's meaning as primitive is used.
%
% \begin{declcs}{pdf@ifprimitive} \csi{cmd}
% \end{declcs}
% Same as \cs{ifpdfprimitive} in \hologo{pdfTeX} or
% \hologo{LuaTeX}. \hologo{XeTeX} calls
% it \cs{ifprimitive}. It is a switch that checks if the command
% \csi{cmd} has it's primitive meaning.
%
% \subsubsection{Additional macro: \cs{pdf@isprimitive}}
%
% \begin{declcs}{pdf@isprimitive} \csi{cmd1} \csi{cmd2} \M{true} \M{false}
% \end{declcs}
% If \csi{cmd1} has the primitive meaning given by the primitive name
% of \csi{cmd2}, then the argument \meta{true} is executed, otherwise
% \meta{false}. The macro \cs{pdf@isprimitive} is expandable.
% Internally it checks the result of \cs{meaning} and is therefore
% available for all \hologo{TeX} variants, even the original \hologo{TeX}.
% Example with \hologo{LaTeX}:
%\begin{quote}
%\begin{verbatim}
%\makeatletter
%\pdf@isprimitive{@@input}{input}{%
%  \typeout{\string\@@input\space is original\string\input}%
%}{%
%  \typeout{Oops, \string\@@input\space is not the %
%           original\string\input}%
%}
%\end{verbatim}
%\end{quote}
%
% \subsubsection{Experimental}
%
% \begin{declcs}{pdf@unescapehexnative} \M{string}\\
%   \cs{pdf@escapehexnative} \M{string}\\
%   \cs{pdf@escapenamenative} \M{string}\\
%   \cs{pdf@mdfivesumnative} \M{string}
% \end{declcs}
% The variants without |native| in the macro name are supposed to
% be compatible with \hologo{pdfTeX}. However characters with more than
% eight bits are not supported and are ignored. If \hologo{LuaTeX} is
% running, then its UTF-8 coded strings are used. Thus the full
% unicode character range is supported. However the result
% differs from \hologo{pdfTeX} for characters with eight or more bits.
%
% \begin{declcs}{pdf@pipe} \M{cmdline}
% \end{declcs}
% It calls \meta{cmdline} and returns the output of the external
% program in the usual manner as byte string (catcode 12, space with
% catcode 10). The Lua documentation says, that the used |io.popen|
% may not be available on all platforms. Then macro \cs{pdf@pipe}
% is undefined.
%
% \StopEventually{
% }
%
% \section{Implementation}
%
%    \begin{macrocode}
%<*package>
%    \end{macrocode}
%
% \subsection{Reload check and package identification}
%    Reload check, especially if the package is not used with \LaTeX.
%    \begin{macrocode}
\begingroup\catcode61\catcode48\catcode32=10\relax%
  \catcode13=5 % ^^M
  \endlinechar=13 %
  \catcode35=6 % #
  \catcode39=12 % '
  \catcode44=12 % ,
  \catcode45=12 % -
  \catcode46=12 % .
  \catcode58=12 % :
  \catcode64=11 % @
  \catcode123=1 % {
  \catcode125=2 % }
  \expandafter\let\expandafter\x\csname ver@pdftexcmds.sty\endcsname
  \ifx\x\relax % plain-TeX, first loading
  \else
    \def\empty{}%
    \ifx\x\empty % LaTeX, first loading,
      % variable is initialized, but \ProvidesPackage not yet seen
    \else
      \expandafter\ifx\csname PackageInfo\endcsname\relax
        \def\x#1#2{%
          \immediate\write-1{Package #1 Info: #2.}%
        }%
      \else
        \def\x#1#2{\PackageInfo{#1}{#2, stopped}}%
      \fi
      \x{pdftexcmds}{The package is already loaded}%
      \aftergroup\endinput
    \fi
  \fi
\endgroup%
%    \end{macrocode}
%    Package identification:
%    \begin{macrocode}
\begingroup\catcode61\catcode48\catcode32=10\relax%
  \catcode13=5 % ^^M
  \endlinechar=13 %
  \catcode35=6 % #
  \catcode39=12 % '
  \catcode40=12 % (
  \catcode41=12 % )
  \catcode44=12 % ,
  \catcode45=12 % -
  \catcode46=12 % .
  \catcode47=12 % /
  \catcode58=12 % :
  \catcode64=11 % @
  \catcode91=12 % [
  \catcode93=12 % ]
  \catcode123=1 % {
  \catcode125=2 % }
  \expandafter\ifx\csname ProvidesPackage\endcsname\relax
    \def\x#1#2#3[#4]{\endgroup
      \immediate\write-1{Package: #3 #4}%
      \xdef#1{#4}%
    }%
  \else
    \def\x#1#2[#3]{\endgroup
      #2[{#3}]%
      \ifx#1\@undefined
        \xdef#1{#3}%
      \fi
      \ifx#1\relax
        \xdef#1{#3}%
      \fi
    }%
  \fi
\expandafter\x\csname ver@pdftexcmds.sty\endcsname
\ProvidesPackage{pdftexcmds}%
  [2019/07/25 v0.30 Utility functions of pdfTeX for LuaTeX (HO)]%
%    \end{macrocode}
%
% \subsection{Catcodes}
%
%    \begin{macrocode}
\begingroup\catcode61\catcode48\catcode32=10\relax%
  \catcode13=5 % ^^M
  \endlinechar=13 %
  \catcode123=1 % {
  \catcode125=2 % }
  \catcode64=11 % @
  \def\x{\endgroup
    \expandafter\edef\csname pdftexcmds@AtEnd\endcsname{%
      \endlinechar=\the\endlinechar\relax
      \catcode13=\the\catcode13\relax
      \catcode32=\the\catcode32\relax
      \catcode35=\the\catcode35\relax
      \catcode61=\the\catcode61\relax
      \catcode64=\the\catcode64\relax
      \catcode123=\the\catcode123\relax
      \catcode125=\the\catcode125\relax
    }%
  }%
\x\catcode61\catcode48\catcode32=10\relax%
\catcode13=5 % ^^M
\endlinechar=13 %
\catcode35=6 % #
\catcode64=11 % @
\catcode123=1 % {
\catcode125=2 % }
\def\TMP@EnsureCode#1#2{%
  \edef\pdftexcmds@AtEnd{%
    \pdftexcmds@AtEnd
    \catcode#1=\the\catcode#1\relax
  }%
  \catcode#1=#2\relax
}
\TMP@EnsureCode{0}{12}%
\TMP@EnsureCode{1}{12}%
\TMP@EnsureCode{2}{12}%
\TMP@EnsureCode{10}{12}% ^^J
\TMP@EnsureCode{33}{12}% !
\TMP@EnsureCode{34}{12}% "
\TMP@EnsureCode{38}{4}% &
\TMP@EnsureCode{39}{12}% '
\TMP@EnsureCode{40}{12}% (
\TMP@EnsureCode{41}{12}% )
\TMP@EnsureCode{42}{12}% *
\TMP@EnsureCode{43}{12}% +
\TMP@EnsureCode{44}{12}% ,
\TMP@EnsureCode{45}{12}% -
\TMP@EnsureCode{46}{12}% .
\TMP@EnsureCode{47}{12}% /
\TMP@EnsureCode{58}{12}% :
\TMP@EnsureCode{60}{12}% <
\TMP@EnsureCode{62}{12}% >
\TMP@EnsureCode{91}{12}% [
\TMP@EnsureCode{93}{12}% ]
\TMP@EnsureCode{94}{7}% ^ (superscript)
\TMP@EnsureCode{95}{12}% _ (other)
\TMP@EnsureCode{96}{12}% `
\TMP@EnsureCode{126}{12}% ~ (other)
\edef\pdftexcmds@AtEnd{%
  \pdftexcmds@AtEnd
  \escapechar=\number\escapechar\relax
  \noexpand\endinput
}
\escapechar=92 %
%    \end{macrocode}
%
% \subsection{Load packages}
%
%    \begin{macrocode}
\begingroup\expandafter\expandafter\expandafter\endgroup
\expandafter\ifx\csname RequirePackage\endcsname\relax
  \def\TMP@RequirePackage#1[#2]{%
    \begingroup\expandafter\expandafter\expandafter\endgroup
    \expandafter\ifx\csname ver@#1.sty\endcsname\relax
      \input #1.sty\relax
    \fi
  }%
  \TMP@RequirePackage{infwarerr}[2007/09/09]%
  \TMP@RequirePackage{ifluatex}[2010/03/01]%
  \TMP@RequirePackage{ltxcmds}[2010/12/02]%
  \TMP@RequirePackage{ifpdf}[2010/09/13]%
\else
  \RequirePackage{infwarerr}[2007/09/09]%
  \RequirePackage{ifluatex}[2010/03/01]%
  \RequirePackage{ltxcmds}[2010/12/02]%
  \RequirePackage{ifpdf}[2010/09/13]%
\fi
%    \end{macrocode}
%
% \subsection{Without \hologo{LuaTeX}}
%
%    \begin{macrocode}
\ifluatex
\else
  \@PackageInfoNoLine{pdftexcmds}{LuaTeX not detected}%
  \def\pdftexcmds@nopdftex{%
    \@PackageInfoNoLine{pdftexcmds}{pdfTeX >= 1.30 not detected}%
    \let\pdftexcmds@nopdftex\relax
  }%
  \def\pdftexcmds@temp#1{%
    \begingroup\expandafter\expandafter\expandafter\endgroup
    \expandafter\ifx\csname pdf#1\endcsname\relax
      \pdftexcmds@nopdftex
    \else
      \expandafter\def\csname pdf@#1\expandafter\endcsname
      \expandafter##\expandafter{%
        \csname pdf#1\endcsname
      }%
    \fi
  }%
  \pdftexcmds@temp{strcmp}%
  \pdftexcmds@temp{escapehex}%
  \let\pdf@escapehexnative\pdf@escapehex
  \pdftexcmds@temp{unescapehex}%
  \let\pdf@unescapehexnative\pdf@unescapehex
  \pdftexcmds@temp{escapestring}%
  \pdftexcmds@temp{escapename}%
  \pdftexcmds@temp{filesize}%
  \pdftexcmds@temp{filemoddate}%
  \begingroup\expandafter\expandafter\expandafter\endgroup
  \expandafter\ifx\csname pdfshellescape\endcsname\relax
    \pdftexcmds@nopdftex
    \ltx@IfUndefined{pdftexversion}{%
    }{%
      \ifnum\pdftexversion>120 % 1.21a supports \ifeof18
        \ifeof18 %
          \chardef\pdf@shellescape=0 %
        \else
          \chardef\pdf@shellescape=1 %
        \fi
      \fi
    }%
  \else
    \def\pdf@shellescape{%
      \pdfshellescape
    }%
  \fi
  \begingroup\expandafter\expandafter\expandafter\endgroup
  \expandafter\ifx\csname pdffiledump\endcsname\relax
    \pdftexcmds@nopdftex
  \else
    \def\pdf@filedump#1#2#3{%
      \pdffiledump offset#1 length#2{#3}%
    }%
  \fi
%    \end{macrocode}
%    \begin{macrocode}
  \begingroup\expandafter\expandafter\expandafter\endgroup
  \expandafter\ifx\csname pdfmdfivesum\endcsname\relax
    \begingroup\expandafter\expandafter\expandafter\endgroup
    \expandafter\ifx\csname mdfivesum\endcsname\relax
      \pdftexcmds@nopdftex
    \else
      \def\pdf@mdfivesum#{\mdfivesum}%
      \let\pdf@mdfivesumnative\pdf@mdfivesum
      \def\pdf@filemdfivesum#{\mdfivesum file}%
    \fi
  \else
    \def\pdf@mdfivesum#{\pdfmdfivesum}%
    \let\pdf@mdfivesumnative\pdf@mdfivesum
    \def\pdf@filemdfivesum#{\pdfmdfivesum file}%
  \fi
%    \end{macrocode}
%    \begin{macrocode}
  \def\pdf@system#{%
    \immediate\write18%
  }%
  \def\pdftexcmds@temp#1{%
    \begingroup\expandafter\expandafter\expandafter\endgroup
    \expandafter\ifx\csname pdf#1\endcsname\relax
      \pdftexcmds@nopdftex
    \else
      \expandafter\let\csname pdf@#1\expandafter\endcsname
      \csname pdf#1\endcsname
    \fi
  }%
  \pdftexcmds@temp{resettimer}%
  \pdftexcmds@temp{elapsedtime}%
\fi
%    \end{macrocode}
%
% \subsection{\cs{pdf@primitive}, \cs{pdf@ifprimitive}}
%
%    Since version 1.40.0 \hologo{pdfTeX} has \cs{pdfprimitive} and
%    \cs{ifpdfprimitive}. And \cs{pdfprimitive} was fixed in
%    version 1.40.4.
%
%    \hologo{XeTeX} provides them under the name \cs{primitive} and
%    \cs{ifprimitive}. \hologo{LuaTeX} knows both name variants,
%    but they have possibly to be enabled first (|tex.enableprimitives|).
%
%    Depending on the format TeX Live uses a prefix |luatex|.
%
%    Caution: \cs{let} must be used for the definition of
%    the macros, especially because of \cs{ifpdfprimitive}.
%
% \subsubsection{Using \hologo{LuaTeX}'s \texttt{tex.enableprimitives}}
%
%    \begin{macrocode}
\ifluatex
%    \end{macrocode}
%    \begin{macro}{\pdftexcmds@directlua}
%    \begin{macrocode}
  \ifnum\luatexversion<36 %
    \def\pdftexcmds@directlua{\directlua0 }%
  \else
    \let\pdftexcmds@directlua\directlua
  \fi
%    \end{macrocode}
%    \end{macro}
%
%    \begin{macrocode}
  \begingroup
    \newlinechar=10 %
    \endlinechar=\newlinechar
    \pdftexcmds@directlua{%
      if tex.enableprimitives then
        tex.enableprimitives(
          'pdf@',
          {'primitive', 'ifprimitive', 'pdfdraftmode','draftmode'}
        )
        tex.enableprimitives('', {'luaescapestring'})
      end
    }%
  \endgroup %
%    \end{macrocode}
%
%    \begin{macrocode}
\fi
%    \end{macrocode}
%
% \subsubsection{Trying various names to find the primitives}
%
%    \begin{macro}{\pdftexcmds@strip@prefix}
%    \begin{macrocode}
\def\pdftexcmds@strip@prefix#1>{}
%    \end{macrocode}
%    \end{macro}
%    \begin{macrocode}
\def\pdftexcmds@temp#1#2#3{%
  \begingroup\expandafter\expandafter\expandafter\endgroup
  \expandafter\ifx\csname pdf@#1\endcsname\relax
    \begingroup
      \def\x{#3}%
      \edef\x{\expandafter\pdftexcmds@strip@prefix\meaning\x}%
      \escapechar=-1 %
      \edef\y{\expandafter\meaning\csname#2\endcsname}%
    \expandafter\endgroup
    \ifx\x\y
      \expandafter\let\csname pdf@#1\expandafter\endcsname
      \csname #2\endcsname
    \fi
  \fi
}
%    \end{macrocode}
%
%    \begin{macro}{\pdf@primitive}
%    \begin{macrocode}
\pdftexcmds@temp{primitive}{pdfprimitive}{pdfprimitive}% pdfTeX, oldLuaTeX
\pdftexcmds@temp{primitive}{primitive}{primitive}% XeTeX, luatex
\pdftexcmds@temp{primitive}{luatexprimitive}{pdfprimitive}% oldLuaTeX
\pdftexcmds@temp{primitive}{luatexpdfprimitive}{pdfprimitive}% oldLuaTeX
%    \end{macrocode}
%    \end{macro}
%    \begin{macro}{\pdf@ifprimitive}
%    \begin{macrocode}
\pdftexcmds@temp{ifprimitive}{ifpdfprimitive}{ifpdfprimitive}% pdfTeX, oldLuaTeX
\pdftexcmds@temp{ifprimitive}{ifprimitive}{ifprimitive}% XeTeX, luatex
\pdftexcmds@temp{ifprimitive}{luatexifprimitive}{ifpdfprimitive}% oldLuaTeX
\pdftexcmds@temp{ifprimitive}{luatexifpdfprimitive}{ifpdfprimitive}% oldLuaTeX
%    \end{macrocode}
%    \end{macro}
%
%    Disable broken \cs{pdfprimitive}.
%    \begin{macrocode}
\ifluatex\else
\begingroup
  \expandafter\ifx\csname pdf@primitive\endcsname\relax
  \else
    \expandafter\ifx\csname pdftexversion\endcsname\relax
    \else
      \ifnum\pdftexversion=140 %
        \expandafter\ifx\csname pdftexrevision\endcsname\relax
        \else
          \ifnum\pdftexrevision<4 %
            \endgroup
            \let\pdf@primitive\@undefined
            \@PackageInfoNoLine{pdftexcmds}{%
              \string\pdf@primitive\space disabled, %
              because\MessageBreak
              \string\pdfprimitive\space is broken until pdfTeX 1.40.4%
            }%
            \begingroup
          \fi
        \fi
      \fi
    \fi
  \fi
\endgroup
\fi
%    \end{macrocode}
%
% \subsubsection{Result}
%
%    \begin{macrocode}
\begingroup
  \@PackageInfoNoLine{pdftexcmds}{%
    \string\pdf@primitive\space is %
    \expandafter\ifx\csname pdf@primitive\endcsname\relax not \fi
    available%
  }%
  \@PackageInfoNoLine{pdftexcmds}{%
    \string\pdf@ifprimitive\space is %
    \expandafter\ifx\csname pdf@ifprimitive\endcsname\relax not \fi
    available%
  }%
\endgroup
%    \end{macrocode}
%
% \subsection{\hologo{XeTeX}}
%
%    Look for primitives \cs{shellescape}, \cs{strcmp}.
%    \begin{macrocode}
\def\pdftexcmds@temp#1{%
  \begingroup\expandafter\expandafter\expandafter\endgroup
  \expandafter\ifx\csname pdf@#1\endcsname\relax
    \begingroup
      \escapechar=-1 %
      \edef\x{\expandafter\meaning\csname#1\endcsname}%
      \def\y{#1}%
      \def\z##1->{}%
      \edef\y{\expandafter\z\meaning\y}%
    \expandafter\endgroup
    \ifx\x\y
      \expandafter\def\csname pdf@#1\expandafter\endcsname
      \expandafter{%
        \csname#1\endcsname
      }%
    \fi
  \fi
}%
\pdftexcmds@temp{shellescape}%
\pdftexcmds@temp{strcmp}%
%    \end{macrocode}
%
% \subsection{\cs{pdf@isprimitive}}
%
%    \begin{macrocode}
\def\pdf@isprimitive{%
  \begingroup\expandafter\expandafter\expandafter\endgroup
  \expandafter\ifx\csname pdf@strcmp\endcsname\relax
    \long\def\pdf@isprimitive##1{%
      \expandafter\pdftexcmds@isprimitive\expandafter{\meaning##1}%
    }%
    \long\def\pdftexcmds@isprimitive##1##2{%
      \expandafter\pdftexcmds@@isprimitive\expandafter{\string##2}{##1}%
    }%
    \def\pdftexcmds@@isprimitive##1##2{%
      \ifnum0\pdftexcmds@equal##1\delimiter##2\delimiter=1 %
        \expandafter\ltx@firstoftwo
      \else
        \expandafter\ltx@secondoftwo
      \fi
    }%
    \def\pdftexcmds@equal##1##2\delimiter##3##4\delimiter{%
      \ifx##1##3%
        \ifx\relax##2##4\relax
          1%
        \else
          \ifx\relax##2\relax
          \else
            \ifx\relax##4\relax
            \else
              \pdftexcmds@equalcont{##2}{##4}%
            \fi
          \fi
        \fi
      \fi
    }%
    \def\pdftexcmds@equalcont##1{%
      \def\pdftexcmds@equalcont####1####2##1##1##1##1{%
        ##1##1##1##1%
        \pdftexcmds@equal####1\delimiter####2\delimiter
      }%
    }%
    \expandafter\pdftexcmds@equalcont\csname fi\endcsname
  \else
    \long\def\pdf@isprimitive##1##2{%
      \ifnum\pdf@strcmp{\meaning##1}{\string##2}=0 %
        \expandafter\ltx@firstoftwo
      \else
        \expandafter\ltx@secondoftwo
      \fi
    }%
  \fi
}
\ifluatex
\ifx\pdfdraftmode\@undefined
  \let\pdfdraftmode\draftmode
\fi
\else
  \pdf@isprimitive
\fi
%    \end{macrocode}
%
% \subsection{\cs{pdf@draftmode}}
%
%
%    \begin{macrocode}
\let\pdftexcmds@temp\ltx@zero %
\ltx@IfUndefined{pdfdraftmode}{%
  \@PackageInfoNoLine{pdftexcmds}{\ltx@backslashchar pdfdraftmode not found}%
}{%
  \ifpdf
    \let\pdftexcmds@temp\ltx@one
    \@PackageInfoNoLine{pdftexcmds}{\ltx@backslashchar pdfdraftmode found}%
  \else
    \@PackageInfoNoLine{pdftexcmds}{%
      \ltx@backslashchar pdfdraftmode is ignored in DVI mode%
    }%
  \fi
}
\ifcase\pdftexcmds@temp
%    \end{macrocode}
%    \begin{macro}{\pdf@draftmode}
%    \begin{macrocode}
  \let\pdf@draftmode\ltx@zero
%    \end{macrocode}
%    \end{macro}
%    \begin{macro}{\pdf@ifdraftmode}
%    \begin{macrocode}
  \let\pdf@ifdraftmode\ltx@secondoftwo
%    \end{macrocode}
%    \end{macro}
%    \begin{macro}{\pdftexcmds@setdraftmode}
%    \begin{macrocode}
  \def\pdftexcmds@setdraftmode#1{}%
%    \end{macrocode}
%    \end{macro}
%    \begin{macrocode}
\else
%    \end{macrocode}
%    \begin{macro}{\pdftexcmds@draftmode}
%    \begin{macrocode}
  \let\pdftexcmds@draftmode\pdfdraftmode
%    \end{macrocode}
%    \end{macro}
%    \begin{macro}{\pdf@ifdraftmode}
%    \begin{macrocode}
  \def\pdf@ifdraftmode{%
    \ifnum\pdftexcmds@draftmode=\ltx@one
      \expandafter\ltx@firstoftwo
    \else
      \expandafter\ltx@secondoftwo
    \fi
  }%
%    \end{macrocode}
%    \end{macro}
%    \begin{macro}{\pdf@draftmode}
%    \begin{macrocode}
  \def\pdf@draftmode{%
    \ifnum\pdftexcmds@draftmode=\ltx@one
      \expandafter\ltx@one
    \else
      \expandafter\ltx@zero
    \fi
  }%
%    \end{macrocode}
%    \end{macro}
%    \begin{macro}{\pdftexcmds@setdraftmode}
%    \begin{macrocode}
  \def\pdftexcmds@setdraftmode#1{%
    \pdftexcmds@draftmode=#1\relax
  }%
%    \end{macrocode}
%    \end{macro}
%    \begin{macrocode}
\fi
%    \end{macrocode}
%    \begin{macro}{\pdf@setdraftmode}
%    \begin{macrocode}
\def\pdf@setdraftmode#1{%
  \begingroup
    \count\ltx@cclv=#1\relax
  \edef\x{\endgroup
    \noexpand\pdftexcmds@@setdraftmode{\the\count\ltx@cclv}%
  }%
  \x
}
%    \end{macrocode}
%    \end{macro}
%    \begin{macro}{\pdftexcmds@@setdraftmode}
%    \begin{macrocode}
\def\pdftexcmds@@setdraftmode#1{%
  \ifcase#1 %
    \pdftexcmds@setdraftmode{#1}%
  \or
    \pdftexcmds@setdraftmode{#1}%
  \else
    \@PackageWarning{pdftexcmds}{%
      \string\pdf@setdraftmode: Ignoring\MessageBreak
      invalid value `#1'%
    }%
  \fi
}
%    \end{macrocode}
%    \end{macro}
%
% \subsection{Load Lua module}
%
%    \begin{macrocode}
\ifluatex
\else
  \expandafter\pdftexcmds@AtEnd
\fi%
%    \end{macrocode}
%
%    \begin{macrocode}
\ifnum\luatexversion<80
  \begingroup\expandafter\expandafter\expandafter\endgroup
  \expandafter\ifx\csname RequirePackage\endcsname\relax
    \def\TMP@RequirePackage#1[#2]{%
      \begingroup\expandafter\expandafter\expandafter\endgroup
      \expandafter\ifx\csname ver@#1.sty\endcsname\relax
        \input #1.sty\relax
      \fi
    }%
    \TMP@RequirePackage{luatex-loader}[2009/04/10]%
  \else
    \RequirePackage{luatex-loader}[2009/04/10]%
  \fi
\fi
\pdftexcmds@directlua{%
  require("pdftexcmds")%
}
\ifnum\luatexversion>37 %
  \ifnum0%
      \pdftexcmds@directlua{%
        if status.ini_version then %
          tex.write("1")%
        end%
      }>0 %
    \everyjob\expandafter{%
      \the\everyjob
      \pdftexcmds@directlua{%
        require("pdftexcmds")%
      }%
    }%
  \fi
\fi
\begingroup
  \def\x{2019/07/25 v0.30}%
  \ltx@onelevel@sanitize\x
  \edef\y{%
    \pdftexcmds@directlua{%
      if oberdiek.pdftexcmds.getversion then %
        oberdiek.pdftexcmds.getversion()%
      end%
    }%
  }%
  \ifx\x\y
  \else
    \@PackageError{pdftexcmds}{%
      Wrong version of lua module.\MessageBreak
      Package version: \x\MessageBreak
      Lua module: \y
    }\@ehc
  \fi
\endgroup
%    \end{macrocode}
%
% \subsection{Lua functions}
%
% \subsubsection{Helper macros}
%
%    \begin{macro}{\pdftexcmds@toks}
%    \begin{macrocode}
\begingroup\expandafter\expandafter\expandafter\endgroup
\expandafter\ifx\csname newtoks\endcsname\relax
  \toksdef\pdftexcmds@toks=0 %
\else
  \csname newtoks\endcsname\pdftexcmds@toks
\fi
%    \end{macrocode}
%    \end{macro}
%
%    \begin{macro}{\pdftexcmds@Patch}
%    \begin{macrocode}
\def\pdftexcmds@Patch{0}
\ifnum\luatexversion>40 %
  \ifnum\luatexversion<66 %
    \def\pdftexcmds@Patch{1}%
  \fi
\fi
%    \end{macrocode}
%    \end{macro}
%    \begin{macrocode}
\ifcase\pdftexcmds@Patch
  \catcode`\&=14 %
\else
  \catcode`\&=9 %
%    \end{macrocode}
%    \begin{macro}{\pdftexcmds@PatchDecode}
%    \begin{macrocode}
  \def\pdftexcmds@PatchDecode#1\@nil{%
    \pdftexcmds@DecodeA#1^^A^^A\@nil{}%
  }%
%    \end{macrocode}
%    \end{macro}
%    \begin{macro}{\pdftexcmds@DecodeA}
%    \begin{macrocode}
  \def\pdftexcmds@DecodeA#1^^A^^A#2\@nil#3{%
    \ifx\relax#2\relax
      \ltx@ReturnAfterElseFi{%
        \pdftexcmds@DecodeB#3#1^^A^^B\@nil{}%
      }%
    \else
      \ltx@ReturnAfterFi{%
        \pdftexcmds@DecodeA#2\@nil{#3#1^^@}%
      }%
    \fi
  }%
%    \end{macrocode}
%    \end{macro}
%    \begin{macro}{\pdftexcmds@DecodeB}
%    \begin{macrocode}
  \def\pdftexcmds@DecodeB#1^^A^^B#2\@nil#3{%
    \ifx\relax#2\relax%
      \ltx@ReturnAfterElseFi{%
        \ltx@zero
        #3#1%
      }%
    \else
      \ltx@ReturnAfterFi{%
        \pdftexcmds@DecodeB#2\@nil{#3#1^^A}%
      }%
    \fi
  }%
%    \end{macrocode}
%    \end{macro}
%    \begin{macrocode}
\fi
%    \end{macrocode}
%
%    \begin{macrocode}
\ifnum\luatexversion<36 %
\else
  \catcode`\0=9 %
\fi
%    \end{macrocode}
%
% \subsubsection[Strings]{Strings \cite[``7.15 Strings'']{pdftex-manual}}
%
%    \begin{macro}{\pdf@strcmp}
%    \begin{macrocode}
\long\def\pdf@strcmp#1#2{%
  \directlua0{%
    oberdiek.pdftexcmds.strcmp("\luaescapestring{#1}",%
        "\luaescapestring{#2}")%
  }%
}%
%    \end{macrocode}
%    \end{macro}
%    \begin{macrocode}
\pdf@isprimitive
%    \end{macrocode}
%    \begin{macro}{\pdf@escapehex}
%    \begin{macrocode}
\long\def\pdf@escapehex#1{%
  \directlua0{%
    oberdiek.pdftexcmds.escapehex("\luaescapestring{#1}", "byte")%
  }%
}%
%    \end{macrocode}
%    \end{macro}
%    \begin{macro}{\pdf@escapehexnative}
%    \begin{macrocode}
\long\def\pdf@escapehexnative#1{%
  \directlua0{%
    oberdiek.pdftexcmds.escapehex("\luaescapestring{#1}")%
  }%
}%
%    \end{macrocode}
%    \end{macro}
%    \begin{macro}{\pdf@unescapehex}
%    \begin{macrocode}
\def\pdf@unescapehex#1{%
& \romannumeral\expandafter\pdftexcmds@PatchDecode
  \the\expandafter\pdftexcmds@toks
  \directlua0{%
    oberdiek.pdftexcmds.toks="pdftexcmds@toks"%
    oberdiek.pdftexcmds.unescapehex("\luaescapestring{#1}", "byte", \pdftexcmds@Patch)%
  }%
& \@nil
}%
%    \end{macrocode}
%    \end{macro}
%    \begin{macro}{\pdf@unescapehexnative}
%    \begin{macrocode}
\def\pdf@unescapehexnative#1{%
& \romannumeral\expandafter\pdftexcmds@PatchDecode
  \the\expandafter\pdftexcmds@toks
  \directlua0{%
    oberdiek.pdftexcmds.toks="pdftexcmds@toks"%
    oberdiek.pdftexcmds.unescapehex("\luaescapestring{#1}", \pdftexcmds@Patch)%
  }%
& \@nil
}%
%    \end{macrocode}
%    \end{macro}
%    \begin{macro}{\pdf@escapestring}
%    \begin{macrocode}
\long\def\pdf@escapestring#1{%
  \directlua0{%
    oberdiek.pdftexcmds.escapestring("\luaescapestring{#1}", "byte")%
  }%
}
%    \end{macrocode}
%    \end{macro}
%    \begin{macro}{\pdf@escapename}
%    \begin{macrocode}
\long\def\pdf@escapename#1{%
  \directlua0{%
    oberdiek.pdftexcmds.escapename("\luaescapestring{#1}", "byte")%
  }%
}
%    \end{macrocode}
%    \end{macro}
%    \begin{macro}{\pdf@escapenamenative}
%    \begin{macrocode}
\long\def\pdf@escapenamenative#1{%
  \directlua0{%
    oberdiek.pdftexcmds.escapename("\luaescapestring{#1}")%
  }%
}
%    \end{macrocode}
%    \end{macro}
%
% \subsubsection[Files]{Files \cite[``7.18 Files'']{pdftex-manual}}
%
%    \begin{macro}{\pdf@filesize}
%    \begin{macrocode}
\def\pdf@filesize#1{%
  \directlua0{%
    oberdiek.pdftexcmds.filesize("\luaescapestring{#1}")%
  }%
}
%    \end{macrocode}
%    \end{macro}
%    \begin{macro}{\pdf@filemoddate}
%    \begin{macrocode}
\def\pdf@filemoddate#1{%
  \directlua0{%
    oberdiek.pdftexcmds.filemoddate("\luaescapestring{#1}")%
  }%
}
%    \end{macrocode}
%    \end{macro}
%    \begin{macro}{\pdf@filedump}
%    \begin{macrocode}
\def\pdf@filedump#1#2#3{%
  \directlua0{%
    oberdiek.pdftexcmds.filedump("\luaescapestring{\number#1}",%
        "\luaescapestring{\number#2}",%
        "\luaescapestring{#3}")%
  }%
}%
%    \end{macrocode}
%    \end{macro}
%    \begin{macro}{\pdf@mdfivesum}
%    \begin{macrocode}
\long\def\pdf@mdfivesum#1{%
  \directlua0{%
    oberdiek.pdftexcmds.mdfivesum("\luaescapestring{#1}", "byte")%
  }%
}%
%    \end{macrocode}
%    \end{macro}
%    \begin{macro}{\pdf@mdfivesumnative}
%    \begin{macrocode}
\long\def\pdf@mdfivesumnative#1{%
  \directlua0{%
    oberdiek.pdftexcmds.mdfivesum("\luaescapestring{#1}")%
  }%
}%
%    \end{macrocode}
%    \end{macro}
%    \begin{macro}{\pdf@filemdfivesum}
%    \begin{macrocode}
\def\pdf@filemdfivesum#1{%
  \directlua0{%
    oberdiek.pdftexcmds.filemdfivesum("\luaescapestring{#1}")%
  }%
}%
%    \end{macrocode}
%    \end{macro}
%
% \subsubsection[Timekeeping]{Timekeeping \cite[``7.17 Timekeeping'']{pdftex-manual}}
%
%    \begin{macro}{\protected}
%    \begin{macrocode}
\let\pdftexcmds@temp=Y%
\begingroup\expandafter\expandafter\expandafter\endgroup
\expandafter\ifx\csname protected\endcsname\relax
  \pdftexcmds@directlua0{%
    if tex.enableprimitives then %
      tex.enableprimitives('', {'protected'})%
    end%
  }%
\fi
\begingroup\expandafter\expandafter\expandafter\endgroup
\expandafter\ifx\csname protected\endcsname\relax
  \let\pdftexcmds@temp=N%
\fi
%    \end{macrocode}
%    \end{macro}
%    \begin{macro}{\numexpr}
%    \begin{macrocode}
\begingroup\expandafter\expandafter\expandafter\endgroup
\expandafter\ifx\csname numexpr\endcsname\relax
  \pdftexcmds@directlua0{%
    if tex.enableprimitives then %
      tex.enableprimitives('', {'numexpr'})%
    end%
  }%
\fi
\begingroup\expandafter\expandafter\expandafter\endgroup
\expandafter\ifx\csname numexpr\endcsname\relax
  \let\pdftexcmds@temp=N%
\fi
%    \end{macrocode}
%    \end{macro}
%
%    \begin{macrocode}
\ifx\pdftexcmds@temp N%
  \@PackageWarningNoLine{pdftexcmds}{%
    Definitions of \ltx@backslashchar pdf@resettimer and%
    \MessageBreak
    \ltx@backslashchar pdf@elapsedtime are skipped, because%
    \MessageBreak
    e-TeX's \ltx@backslashchar protected or %
    \ltx@backslashchar numexpr are missing%
  }%
\else
%    \end{macrocode}
%
%    \begin{macro}{\pdf@resettimer}
%    \begin{macrocode}
  \protected\def\pdf@resettimer{%
    \pdftexcmds@directlua0{%
      oberdiek.pdftexcmds.resettimer()%
    }%
  }%
%    \end{macrocode}
%    \end{macro}
%
%    \begin{macro}{\pdf@elapsedtime}
%    \begin{macrocode}
  \protected\def\pdf@elapsedtime{%
    \numexpr
      \pdftexcmds@directlua0{%
        oberdiek.pdftexcmds.elapsedtime()%
      }%
    \relax
  }%
%    \end{macrocode}
%    \end{macro}
%    \begin{macrocode}
\fi
%    \end{macrocode}
%
% \subsubsection{Shell escape}
%
%    \begin{macro}{\pdf@shellescape}
%
%    \begin{macrocode}
\ifnum\luatexversion<68 %
\else
  \protected\edef\pdf@shellescape{%
   \numexpr\directlua{tex.sprint(%
         \number\catcodetable@string,status.shell_escape)}\relax}
\fi
%    \end{macrocode}
%    \end{macro}
%
%    \begin{macro}{\pdf@system}
%    \begin{macrocode}
\def\pdf@system#1{%
  \directlua0{%
    oberdiek.pdftexcmds.system("\luaescapestring{#1}")%
  }%
}
%    \end{macrocode}
%    \end{macro}
%
%    \begin{macro}{\pdf@lastsystemstatus}
%    \begin{macrocode}
\def\pdf@lastsystemstatus{%
  \directlua0{%
    oberdiek.pdftexcmds.lastsystemstatus()%
  }%
}
%    \end{macrocode}
%    \end{macro}
%    \begin{macro}{\pdf@lastsystemexit}
%    \begin{macrocode}
\def\pdf@lastsystemexit{%
  \directlua0{%
    oberdiek.pdftexcmds.lastsystemexit()%
  }%
}
%    \end{macrocode}
%    \end{macro}
%
%    \begin{macrocode}
\catcode`\0=12 %
%    \end{macrocode}
%
%    \begin{macro}{\pdf@pipe}
%    Check availability of |io.popen| first.
%    \begin{macrocode}
\ifnum0%
    \pdftexcmds@directlua{%
      if io.popen then %
        tex.write("1")%
      end%
    }%
    =1 %
  \def\pdf@pipe#1{%
&   \romannumeral\expandafter\pdftexcmds@PatchDecode
    \the\expandafter\pdftexcmds@toks
    \pdftexcmds@directlua{%
      oberdiek.pdftexcmds.toks="pdftexcmds@toks"%
      oberdiek.pdftexcmds.pipe("\luaescapestring{#1}", \pdftexcmds@Patch)%
    }%
&   \@nil
  }%
\fi
%    \end{macrocode}
%    \end{macro}
%
%    \begin{macrocode}
\pdftexcmds@AtEnd%
%</package>
%    \end{macrocode}
%
% \subsection{Lua module}
%
%    \begin{macrocode}
%<*lua>
%    \end{macrocode}
%
%    \begin{macrocode}
oberdiek = oberdiek or {}
local pdftexcmds = oberdiek.pdftexcmds or {}
oberdiek.pdftexcmds = pdftexcmds
local systemexitstatus
function pdftexcmds.getversion()
  tex.write("2019/07/25 v0.30")
end
%    \end{macrocode}
%
% \subsubsection[Strings]{Strings \cite[``7.15 Strings'']{pdftex-manual}}
%
%    \begin{macrocode}
function pdftexcmds.strcmp(A, B)
  if A == B then
    tex.write("0")
  elseif A < B then
    tex.write("-1")
  else
    tex.write("1")
  end
end
local function utf8_to_byte(str)
  local i = 0
  local n = string.len(str)
  local t = {}
  while i < n do
    i = i + 1
    local a = string.byte(str, i)
    if a < 128 then
      table.insert(t, string.char(a))
    else
      if a >= 192 and i < n then
        i = i + 1
        local b = string.byte(str, i)
        if b < 128 or b >= 192 then
          i = i - 1
        elseif a == 194 then
          table.insert(t, string.char(b))
        elseif a == 195 then
          table.insert(t, string.char(b + 64))
        end
      end
    end
  end
  return table.concat(t)
end
function pdftexcmds.escapehex(str, mode)
  if mode == "byte" then
    str = utf8_to_byte(str)
  end
  tex.write((string.gsub(str, ".",
    function (ch)
      return string.format("%02X", string.byte(ch))
    end
  )))
end
%    \end{macrocode}
%    See procedure |unescapehex| in file \xfile{utils.c} of \hologo{pdfTeX}.
%    Caution: |tex.write| ignores leading spaces.
%    \begin{macrocode}
function pdftexcmds.unescapehex(str, mode, patch)
  local a = 0
  local first = true
  local result = {}
  for i = 1, string.len(str), 1 do
    local ch = string.byte(str, i)
    if ch >= 48 and ch <= 57 then
      ch = ch - 48
    elseif ch >= 65 and ch <= 70 then
      ch = ch - 55
    elseif ch >= 97 and ch <= 102 then
      ch = ch - 87
    else
      ch = nil
    end
    if ch then
      if first then
        a = ch * 16
        first = false
      else
        table.insert(result, a + ch)
        first = true
      end
    end
  end
  if not first then
    table.insert(result, a)
  end
  if patch == 1 then
    local temp = {}
    for i, a in ipairs(result) do
      if a == 0 then
        table.insert(temp, 1)
        table.insert(temp, 1)
      else
        if a == 1 then
          table.insert(temp, 1)
          table.insert(temp, 2)
        else
          table.insert(temp, a)
        end
      end
    end
    result = temp
  end
  if mode == "byte" then
    local utf8 = {}
    for i, a in ipairs(result) do
      if a < 128 then
        table.insert(utf8, a)
      else
        if a < 192 then
          table.insert(utf8, 194)
          a = a - 128
        else
          table.insert(utf8, 195)
          a = a - 192
        end
        table.insert(utf8, a + 128)
      end
    end
    result = utf8
  end
%    \end{macrocode}
%    this next line added for current luatex; this is the only
%    change in the file.  eroux, 28apr13. (v 0.21)
%    \begin{macrocode}
  local unpack = _G["unpack"] or table.unpack
  tex.settoks(pdftexcmds.toks, string.char(unpack(result)))
end
%    \end{macrocode}
%    See procedure |escapestring| in file \xfile{utils.c} of \hologo{pdfTeX}.
%    \begin{macrocode}
function pdftexcmds.escapestring(str, mode)
  if mode == "byte" then
    str = utf8_to_byte(str)
  end
  tex.write((string.gsub(str, ".",
    function (ch)
      local b = string.byte(ch)
      if b < 33 or b > 126 then
        return string.format("\\%.3o", b)
      end
      if b == 40 or b == 41 or b == 92 then
        return "\\" .. ch
      end
%    \end{macrocode}
%    Lua 5.1 returns the match in case of return value |nil|.
%    \begin{macrocode}
      return nil
    end
  )))
end
%    \end{macrocode}
%    See procedure |escapename| in file \xfile{utils.c} of \hologo{pdfTeX}.
%    \begin{macrocode}
function pdftexcmds.escapename(str, mode)
  if mode == "byte" then
    str = utf8_to_byte(str)
  end
  tex.write((string.gsub(str, ".",
    function (ch)
      local b = string.byte(ch)
      if b == 0 then
%    \end{macrocode}
%    In Lua 5.0 |nil| could be used for the empty string,
%    But |nil| returns the match in Lua 5.1, thus we use
%    the empty string explicitly.
%    \begin{macrocode}
        return ""
      end
      if b <= 32 or b >= 127
          or b == 35 or b == 37 or b == 40 or b == 41
          or b == 47 or b == 60 or b == 62 or b == 91
          or b == 93 or b == 123 or b == 125 then
        return string.format("#%.2X", b)
      else
%    \end{macrocode}
%    Lua 5.1 returns the match in case of return value |nil|.
%    \begin{macrocode}
        return nil
      end
    end
  )))
end
%    \end{macrocode}
%
% \subsubsection[Files]{Files \cite[``7.18 Files'']{pdftex-manual}}
%
%    \begin{macrocode}
function pdftexcmds.filesize(filename)
  local foundfile = kpse.find_file(filename, "tex", true)
  if foundfile then
    local size = lfs.attributes(foundfile, "size")
    if size then
      tex.write(size)
    end
  end
end
%    \end{macrocode}
%    See procedure |makepdftime| in file \xfile{utils.c} of \hologo{pdfTeX}.
%    \begin{macrocode}
function pdftexcmds.filemoddate(filename)
  local foundfile = kpse.find_file(filename, "tex", true)
  if foundfile then
    local date = lfs.attributes(foundfile, "modification")
    if date then
      local d = os.date("*t", date)
      if d.sec >= 60 then
        d.sec = 59
      end
      local u = os.date("!*t", date)
      local off = 60 * (d.hour - u.hour) + d.min - u.min
      if d.year ~= u.year then
        if d.year > u.year then
          off = off + 1440
        else
          off = off - 1440
        end
      elseif d.yday ~= u.yday then
        if d.yday > u.yday then
          off = off + 1440
        else
          off = off - 1440
        end
      end
      local timezone
      if off == 0 then
        timezone = "Z"
      else
        local hours = math.floor(off / 60)
        local mins = math.abs(off - hours * 60)
        timezone = string.format("%+03d'%02d'", hours, mins)
      end
      tex.write(string.format("D:%04d%02d%02d%02d%02d%02d%s",
          d.year, d.month, d.day, d.hour, d.min, d.sec, timezone))
    end
  end
end
function pdftexcmds.filedump(offset, length, filename)
  length = tonumber(length)
  if length and length > 0 then
    local foundfile = kpse.find_file(filename, "tex", true)
    if foundfile then
      offset = tonumber(offset)
      if not offset then
        offset = 0
      end
      local filehandle = io.open(foundfile, "rb")
      if filehandle then
        if offset > 0 then
          filehandle:seek("set", offset)
        end
        local dump = filehandle:read(length)
        pdftexcmds.escapehex(dump)
        filehandle:close()
      end
    end
  end
end
function pdftexcmds.mdfivesum(str, mode)
  if mode == "byte" then
    str = utf8_to_byte(str)
  end
  pdftexcmds.escapehex(md5.sum(str))
end
function pdftexcmds.filemdfivesum(filename)
  local foundfile = kpse.find_file(filename, "tex", true)
  if foundfile then
    local filehandle = io.open(foundfile, "rb")
    if filehandle then
      local contents = filehandle:read("*a")
      pdftexcmds.escapehex(md5.sum(contents))
      filehandle:close()
    end
  end
end
%    \end{macrocode}
%
% \subsubsection[Timekeeping]{Timekeeping \cite[``7.17 Timekeeping'']{pdftex-manual}}
%
%    The functions for timekeeping are based on
%    Andy Thomas' work \cite{AndyThomas:Analog}.
%    Changes:
%    \begin{itemize}
%    \item Overflow check is added.
%    \item |string.format| is used to avoid exponential number
%          representation for sure.
%    \item |tex.write| is used instead of |tex.print| to get
%          tokens with catcode 12 and without appended \cs{endlinechar}.
%    \end{itemize}
%    \begin{macrocode}
local basetime = 0
function pdftexcmds.resettimer()
  basetime = os.clock()
end
function pdftexcmds.elapsedtime()
  local val = (os.clock() - basetime) * 65536 + .5
  if val > 2147483647 then
    val = 2147483647
  end
  tex.write(string.format("%d", val))
end
%    \end{macrocode}
%
% \subsubsection[Miscellaneous]{Miscellaneous \cite[``7.21 Miscellaneous'']{pdftex-manual}}
%
%    \begin{macrocode}
function pdftexcmds.shellescape()
  if os.execute then
    if status
        and status.luatex_version
        and status.luatex_version >= 68 then
      tex.write(os.execute())
    else
      local result = os.execute()
      if result == 0 then
        tex.write("0")
      else
        if result == nil then
          tex.write("0")
        else
          tex.write("1")
        end
      end
    end
  else
    tex.write("0")
  end
end
function pdftexcmds.system(cmdline)
  systemexitstatus = nil
  texio.write_nl("log", "system(" .. cmdline .. ") ")
  if os.execute then
    texio.write("log", "executed.")
    systemexitstatus = os.execute(cmdline)
  else
    texio.write("log", "disabled.")
  end
end
function pdftexcmds.lastsystemstatus()
  local result = tonumber(systemexitstatus)
  if result then
    local x = math.floor(result / 256)
    tex.write(result - 256 * math.floor(result / 256))
  end
end
function pdftexcmds.lastsystemexit()
  local result = tonumber(systemexitstatus)
  if result then
    tex.write(math.floor(result / 256))
  end
end
function pdftexcmds.pipe(cmdline, patch)
  local result
  systemexitstatus = nil
  texio.write_nl("log", "pipe(" .. cmdline ..") ")
  if io.popen then
    texio.write("log", "executed.")
    local handle = io.popen(cmdline, "r")
    if handle then
      result = handle:read("*a")
      handle:close()
    end
  else
    texio.write("log", "disabled.")
  end
  if result then
    if patch == 1 then
      local temp = {}
      for i, a in ipairs(result) do
        if a == 0 then
          table.insert(temp, 1)
          table.insert(temp, 1)
        else
          if a == 1 then
            table.insert(temp, 1)
            table.insert(temp, 2)
          else
            table.insert(temp, a)
          end
        end
      end
      result = temp
    end
    tex.settoks(pdftexcmds.toks, result)
  else
    tex.settoks(pdftexcmds.toks, "")
  end
end
%    \end{macrocode}
%    \begin{macrocode}
%</lua>
%    \end{macrocode}
%
% \section{Test}
%
% \subsection{Catcode checks for loading}
%
%    \begin{macrocode}
%<*test1>
%    \end{macrocode}
%    \begin{macrocode}
\catcode`\{=1 %
\catcode`\}=2 %
\catcode`\#=6 %
\catcode`\@=11 %
\expandafter\ifx\csname count@\endcsname\relax
  \countdef\count@=255 %
\fi
\expandafter\ifx\csname @gobble\endcsname\relax
  \long\def\@gobble#1{}%
\fi
\expandafter\ifx\csname @firstofone\endcsname\relax
  \long\def\@firstofone#1{#1}%
\fi
\expandafter\ifx\csname loop\endcsname\relax
  \expandafter\@firstofone
\else
  \expandafter\@gobble
\fi
{%
  \def\loop#1\repeat{%
    \def\body{#1}%
    \iterate
  }%
  \def\iterate{%
    \body
      \let\next\iterate
    \else
      \let\next\relax
    \fi
    \next
  }%
  \let\repeat=\fi
}%
\def\RestoreCatcodes{}
\count@=0 %
\loop
  \edef\RestoreCatcodes{%
    \RestoreCatcodes
    \catcode\the\count@=\the\catcode\count@\relax
  }%
\ifnum\count@<255 %
  \advance\count@ 1 %
\repeat

\def\RangeCatcodeInvalid#1#2{%
  \count@=#1\relax
  \loop
    \catcode\count@=15 %
  \ifnum\count@<#2\relax
    \advance\count@ 1 %
  \repeat
}
\def\RangeCatcodeCheck#1#2#3{%
  \count@=#1\relax
  \loop
    \ifnum#3=\catcode\count@
    \else
      \errmessage{%
        Character \the\count@\space
        with wrong catcode \the\catcode\count@\space
        instead of \number#3%
      }%
    \fi
  \ifnum\count@<#2\relax
    \advance\count@ 1 %
  \repeat
}
\def\space{ }
\expandafter\ifx\csname LoadCommand\endcsname\relax
  \def\LoadCommand{\input pdftexcmds.sty\relax}%
\fi
\def\Test{%
  \RangeCatcodeInvalid{0}{47}%
  \RangeCatcodeInvalid{58}{64}%
  \RangeCatcodeInvalid{91}{96}%
  \RangeCatcodeInvalid{123}{255}%
  \catcode`\@=12 %
  \catcode`\\=0 %
  \catcode`\%=14 %
  \LoadCommand
  \RangeCatcodeCheck{0}{36}{15}%
  \RangeCatcodeCheck{37}{37}{14}%
  \RangeCatcodeCheck{38}{47}{15}%
  \RangeCatcodeCheck{48}{57}{12}%
  \RangeCatcodeCheck{58}{63}{15}%
  \RangeCatcodeCheck{64}{64}{12}%
  \RangeCatcodeCheck{65}{90}{11}%
  \RangeCatcodeCheck{91}{91}{15}%
  \RangeCatcodeCheck{92}{92}{0}%
  \RangeCatcodeCheck{93}{96}{15}%
  \RangeCatcodeCheck{97}{122}{11}%
  \RangeCatcodeCheck{123}{255}{15}%
  \RestoreCatcodes
}
\Test
\csname @@end\endcsname
\end
%    \end{macrocode}
%    \begin{macrocode}
%</test1>
%    \end{macrocode}
%
% \subsection{Test for \cs{pdf@isprimitive}}
%
%    \begin{macrocode}
%<*test2>
\catcode`\{=1 %
\catcode`\}=2 %
\catcode`\#=6 %
\catcode`\@=11 %
\input pdftexcmds.sty\relax
\def\msg#1{%
  \begingroup
    \escapechar=92 %
    \immediate\write16{#1}%
  \endgroup
}
\long\def\test#1#2#3#4{%
  \begingroup
    #4%
    \def\str{%
      Test \string\pdf@isprimitive
      {\string #1}{\string #2}{...}: %
    }%
    \pdf@isprimitive{#1}{#2}{%
      \ifx#3Y%
        \msg{\str true ==> OK.}%
      \else
        \errmessage{\str false ==> FAILED}%
      \fi
    }{%
      \ifx#3Y%
        \errmessage{\str true ==> FAILED}%
      \else
        \msg{\str false ==> OK.}%
      \fi
    }%
  \endgroup
}
\test\relax\relax Y{}
\test\foobar\relax Y{\let\foobar\relax}
\test\foobar\relax N{}
\test\hbox\hbox Y{}
\test\foobar@hbox\hbox Y{\let\foobar@hbox\hbox}
\test\if\if Y{}
\test\if\ifx N{}
\test\ifx\if N{}
\test\par\par Y{}
\test\hbox\par N{}
\test\par\hbox N{}
\csname @@end\endcsname\end
%</test2>
%    \end{macrocode}
%
% \subsection{Test for \cs{pdf@shellescape}}
%
%    \begin{macrocode}
%<*test-shell>
\catcode`\{=1 %
\catcode`\}=2 %
\catcode`\#=6 %
\catcode`\@=11 %
\input pdftexcmds.sty\relax
\def\msg#{\immediate\write16}
\def\MaybeEnd{}
\ifx\luatexversion\UnDeFiNeD
\else
  \ifnum\luatexversion<68 %
    \ifx\pdf@shellescape\@undefined
      \msg{SHELL=U}%
      \msg{OK (LuaTeX < 0.68)}%
    \else
      \msg{SHELL=defined}%
      \errmessage{Failed (LuaTeX < 0.68)}%
    \fi
    \def\MaybeEnd{\csname @@end\endcsname\end}%
  \fi
\fi
\MaybeEnd
\ifx\pdf@shellescape\@undefined
  \msg{SHELL=U}%
\else
  \msg{SHELL=\number\pdf@shellescape}%
\fi
\ifx\expected\@undefined
\else
  \ifx\expected\relax
    \msg{EXPECTED=U}%
    \ifx\pdf@shellescape\@undefined
      \msg{OK}%
    \else
      \errmessage{Failed}%
    \fi
  \else
    \msg{EXPECTED=\number\expected}%
    \ifnum\pdf@shellescape=\expected\relax
      \msg{OK}%
    \else
      \errmessage{Failed}%
    \fi
  \fi
\fi
\csname @@end\endcsname\end
%</test-shell>
%    \end{macrocode}
%
% \subsection{Test for escape functions}
%
%    \begin{macrocode}
%<*test-escape>
\catcode`\{=1 %
\catcode`\}=2 %
\catcode`\#=6 %
\catcode`\^=7 %
\catcode`\@=11 %
\errorcontextlines=1000 %
\input pdftexcmds.sty\relax
\def\msg#1{%
  \begingroup
    \escapechar=92 %
    \immediate\write16{#1}%
  \endgroup
}
%    \end{macrocode}
%    \begin{macrocode}
\begingroup
  \catcode`\@=11 %
  \countdef\count@=255 %
  \def\space{ }%
  \long\def\@whilenum#1\do #2{%
    \ifnum #1\relax
      #2\relax
      \@iwhilenum{#1\relax#2\relax}%
    \fi
  }%
  \long\def\@iwhilenum#1{%
    \ifnum #1%
      \expandafter\@iwhilenum
    \else
      \expandafter\ltx@gobble
    \fi
    {#1}%
  }%
  \gdef\AllBytes{}%
  \count@=0 %
  \catcode0=12 %
  \@whilenum\count@<256 \do{%
    \lccode0=\count@
    \ifnum\count@=32 %
      \xdef\AllBytes{\AllBytes\space}%
    \else
      \lowercase{%
        \xdef\AllBytes{\AllBytes^^@}%
      }%
    \fi
    \advance\count@ by 1 %
  }%
\endgroup
%    \end{macrocode}
%    \begin{macrocode}
\def\AllBytesHex{%
  000102030405060708090A0B0C0D0E0F%
  101112131415161718191A1B1C1D1E1F%
  202122232425262728292A2B2C2D2E2F%
  303132333435363738393A3B3C3D3E3F%
  404142434445464748494A4B4C4D4E4F%
  505152535455565758595A5B5C5D5E5F%
  606162636465666768696A6B6C6D6E6F%
  707172737475767778797A7B7C7D7E7F%
  808182838485868788898A8B8C8D8E8F%
  909192939495969798999A9B9C9D9E9F%
  A0A1A2A3A4A5A6A7A8A9AAABACADAEAF%
  B0B1B2B3B4B5B6B7B8B9BABBBCBDBEBF%
  C0C1C2C3C4C5C6C7C8C9CACBCCCDCECF%
  D0D1D2D3D4D5D6D7D8D9DADBDCDDDEDF%
  E0E1E2E3E4E5E6E7E8E9EAEBECEDEEEF%
  F0F1F2F3F4F5F6F7F8F9FAFBFCFDFEFF%
}
\ltx@onelevel@sanitize\AllBytesHex
\expandafter\lowercase\expandafter{%
  \expandafter\def\expandafter\AllBytesHexLC
      \expandafter{\AllBytesHex}%
}
\begingroup
  \catcode`\#=12 %
  \xdef\AllBytesName{%
    #01#02#03#04#05#06#07#08#09#0A#0B#0C#0D#0E#0F%
    #10#11#12#13#14#15#16#17#18#19#1A#1B#1C#1D#1E#1F%
    #20!"#23$#25&'#28#29*+,-.#2F%
    0123456789:;#3C=#3E?%
    @ABCDEFGHIJKLMNO%
    PQRSTUVWXYZ#5B\ltx@backslashchar#5D^_%
    `abcdefghijklmno%
    pqrstuvwxyz#7B|#7D\string~#7F%
    #80#81#82#83#84#85#86#87#88#89#8A#8B#8C#8D#8E#8F%
    #90#91#92#93#94#95#96#97#98#99#9A#9B#9C#9D#9E#9F%
    #A0#A1#A2#A3#A4#A5#A6#A7#A8#A9#AA#AB#AC#AD#AE#AF%
    #B0#B1#B2#B3#B4#B5#B6#B7#B8#B9#BA#BB#BC#BD#BE#BF%
    #C0#C1#C2#C3#C4#C5#C6#C7#C8#C9#CA#CB#CC#CD#CE#CF%
    #D0#D1#D2#D3#D4#D5#D6#D7#D8#D9#DA#DB#DC#DD#DE#DF%
    #E0#E1#E2#E3#E4#E5#E6#E7#E8#E9#EA#EB#EC#ED#EE#EF%
    #F0#F1#F2#F3#F4#F5#F6#F7#F8#F9#FA#FB#FC#FD#FE#FF%
  }%
\endgroup
\ltx@onelevel@sanitize\AllBytesName
\edef\AllBytesFromName{\expandafter\ltx@gobble\AllBytes}
\begingroup
  \def\|{|}%
  \edef\%{\ltx@percentchar}%
  \catcode`\|=0 %
  \catcode`\#=12 %
  \catcode`\~=12 %
  \catcode`\\=12 %
  |xdef|AllBytesString{%
    \000\001\002\003\004\005\006\007\010\011\012\013\014\015\016\017%
    \020\021\022\023\024\025\026\027\030\031\032\033\034\035\036\037%
    \040!"#$|%&'\(\)*+,-./%
    0123456789:;<=>?%
    @ABCDEFGHIJKLMNO%
    PQRSTUVWXYZ[\\]^_%
    `abcdefghijklmno%
    pqrstuvwxyz{||}~\177%
    \200\201\202\203\204\205\206\207\210\211\212\213\214\215\216\217%
    \220\221\222\223\224\225\226\227\230\231\232\233\234\235\236\237%
    \240\241\242\243\244\245\246\247\250\251\252\253\254\255\256\257%
    \260\261\262\263\264\265\266\267\270\271\272\273\274\275\276\277%
    \300\301\302\303\304\305\306\307\310\311\312\313\314\315\316\317%
    \320\321\322\323\324\325\326\327\330\331\332\333\334\335\336\337%
    \340\341\342\343\344\345\346\347\350\351\352\353\354\355\356\357%
    \360\361\362\363\364\365\366\367\370\371\372\373\374\375\376\377%
  }%
|endgroup
\ltx@onelevel@sanitize\AllBytesString
%    \end{macrocode}
%    \begin{macrocode}
\def\Test#1#2#3{%
  \begingroup
    \expandafter\expandafter\expandafter\def
    \expandafter\expandafter\expandafter\TestResult
    \expandafter\expandafter\expandafter{%
      #1{#2}%
    }%
    \ifx\TestResult#3%
    \else
      \newlinechar=10 %
      \msg{Expect:^^J#3}%
      \msg{Result:^^J\TestResult}%
      \errmessage{\string#2 -\string#1-> \string#3}%
    \fi
  \endgroup
}
\def\test#1#2#3{%
  \edef\TestFrom{#2}%
  \edef\TestExpect{#3}%
  \ltx@onelevel@sanitize\TestExpect
  \Test#1\TestFrom\TestExpect
}
\test\pdf@unescapehex{74657374}{test}
\begingroup
  \catcode0=12 %
  \catcode1=12 %
  \test\pdf@unescapehex{740074017400740174}{t^^@t^^At^^@t^^At}%
\endgroup
\Test\pdf@escapehex\AllBytes\AllBytesHex
\Test\pdf@unescapehex\AllBytesHex\AllBytes
\Test\pdf@escapename\AllBytes\AllBytesName
\Test\pdf@escapestring\AllBytes\AllBytesString
%    \end{macrocode}
%    \begin{macrocode}
\csname @@end\endcsname\end
%</test-escape>
%    \end{macrocode}
%
% \section{Installation}
%
% \subsection{Download}
%
% \paragraph{Package.} This package is available on
% CTAN\footnote{\CTANpkg{pdftexcmds}}:
% \begin{description}
% \item[\CTAN{macros/latex/contrib/oberdiek/pdftexcmds.dtx}] The source file.
% \item[\CTAN{macros/latex/contrib/oberdiek/pdftexcmds.pdf}] Documentation.
% \end{description}
%
%
% \paragraph{Bundle.} All the packages of the bundle `oberdiek'
% are also available in a TDS compliant ZIP archive. There
% the packages are already unpacked and the documentation files
% are generated. The files and directories obey the TDS standard.
% \begin{description}
% \item[\CTANinstall{install/macros/latex/contrib/oberdiek.tds.zip}]
% \end{description}
% \emph{TDS} refers to the standard ``A Directory Structure
% for \TeX\ Files'' (\CTAN{tds/tds.pdf}). Directories
% with \xfile{texmf} in their name are usually organized this way.
%
% \subsection{Bundle installation}
%
% \paragraph{Unpacking.} Unpack the \xfile{oberdiek.tds.zip} in the
% TDS tree (also known as \xfile{texmf} tree) of your choice.
% Example (linux):
% \begin{quote}
%   |unzip oberdiek.tds.zip -d ~/texmf|
% \end{quote}
%
% \paragraph{Script installation.}
% Check the directory \xfile{TDS:scripts/oberdiek/} for
% scripts that need further installation steps.
% Package \xpackage{attachfile2} comes with the Perl script
% \xfile{pdfatfi.pl} that should be installed in such a way
% that it can be called as \texttt{pdfatfi}.
% Example (linux):
% \begin{quote}
%   |chmod +x scripts/oberdiek/pdfatfi.pl|\\
%   |cp scripts/oberdiek/pdfatfi.pl /usr/local/bin/|
% \end{quote}
%
% \subsection{Package installation}
%
% \paragraph{Unpacking.} The \xfile{.dtx} file is a self-extracting
% \docstrip\ archive. The files are extracted by running the
% \xfile{.dtx} through \plainTeX:
% \begin{quote}
%   \verb|tex pdftexcmds.dtx|
% \end{quote}
%
% \paragraph{TDS.} Now the different files must be moved into
% the different directories in your installation TDS tree
% (also known as \xfile{texmf} tree):
% \begin{quote}
% \def\t{^^A
% \begin{tabular}{@{}>{\ttfamily}l@{ $\rightarrow$ }>{\ttfamily}l@{}}
%   pdftexcmds.sty & tex/generic/oberdiek/pdftexcmds.sty\\
%   oberdiek.pdftexcmds.lua & scripts/oberdiek/oberdiek.pdftexcmds.lua\\
%   pdftexcmds.lua & scripts/oberdiek/pdftexcmds.lua\\
%   pdftexcmds.pdf & doc/latex/oberdiek/pdftexcmds.pdf\\
%   test/pdftexcmds-test1.tex & doc/latex/oberdiek/test/pdftexcmds-test1.tex\\
%   test/pdftexcmds-test2.tex & doc/latex/oberdiek/test/pdftexcmds-test2.tex\\
%   test/pdftexcmds-test-shell.tex & doc/latex/oberdiek/test/pdftexcmds-test-shell.tex\\
%   test/pdftexcmds-test-escape.tex & doc/latex/oberdiek/test/pdftexcmds-test-escape.tex\\
%   pdftexcmds.dtx & source/latex/oberdiek/pdftexcmds.dtx\\
% \end{tabular}^^A
% }^^A
% \sbox0{\t}^^A
% \ifdim\wd0>\linewidth
%   \begingroup
%     \advance\linewidth by\leftmargin
%     \advance\linewidth by\rightmargin
%   \edef\x{\endgroup
%     \def\noexpand\lw{\the\linewidth}^^A
%   }\x
%   \def\lwbox{^^A
%     \leavevmode
%     \hbox to \linewidth{^^A
%       \kern-\leftmargin\relax
%       \hss
%       \usebox0
%       \hss
%       \kern-\rightmargin\relax
%     }^^A
%   }^^A
%   \ifdim\wd0>\lw
%     \sbox0{\small\t}^^A
%     \ifdim\wd0>\linewidth
%       \ifdim\wd0>\lw
%         \sbox0{\footnotesize\t}^^A
%         \ifdim\wd0>\linewidth
%           \ifdim\wd0>\lw
%             \sbox0{\scriptsize\t}^^A
%             \ifdim\wd0>\linewidth
%               \ifdim\wd0>\lw
%                 \sbox0{\tiny\t}^^A
%                 \ifdim\wd0>\linewidth
%                   \lwbox
%                 \else
%                   \usebox0
%                 \fi
%               \else
%                 \lwbox
%               \fi
%             \else
%               \usebox0
%             \fi
%           \else
%             \lwbox
%           \fi
%         \else
%           \usebox0
%         \fi
%       \else
%         \lwbox
%       \fi
%     \else
%       \usebox0
%     \fi
%   \else
%     \lwbox
%   \fi
% \else
%   \usebox0
% \fi
% \end{quote}
% If you have a \xfile{docstrip.cfg} that configures and enables \docstrip's
% TDS installing feature, then some files can already be in the right
% place, see the documentation of \docstrip.
%
% \subsection{Refresh file name databases}
%
% If your \TeX~distribution
% (\teTeX, \mikTeX, \dots) relies on file name databases, you must refresh
% these. For example, \teTeX\ users run \verb|texhash| or
% \verb|mktexlsr|.
%
% \subsection{Some details for the interested}
%
% \paragraph{Attached source.}
%
% The PDF documentation on CTAN also includes the
% \xfile{.dtx} source file. It can be extracted by
% AcrobatReader 6 or higher. Another option is \textsf{pdftk},
% e.g. unpack the file into the current directory:
% \begin{quote}
%   \verb|pdftk pdftexcmds.pdf unpack_files output .|
% \end{quote}
%
% \paragraph{Unpacking with \LaTeX.}
% The \xfile{.dtx} chooses its action depending on the format:
% \begin{description}
% \item[\plainTeX:] Run \docstrip\ and extract the files.
% \item[\LaTeX:] Generate the documentation.
% \end{description}
% If you insist on using \LaTeX\ for \docstrip\ (really,
% \docstrip\ does not need \LaTeX), then inform the autodetect routine
% about your intention:
% \begin{quote}
%   \verb|latex \let\install=y% \iffalse meta-comment
%
% File: pdftexcmds.dtx
% Version: 2019/07/25 v0.30
% Info: Utility functions of pdfTeX for LuaTeX
%
% Copyright (C) 2007, 2009-2011 by
%    Heiko Oberdiek <heiko.oberdiek at googlemail.com>
%
% This work may be distributed and/or modified under the
% conditions of the LaTeX Project Public License, either
% version 1.3c of this license or (at your option) any later
% version. This version of this license is in
%    https://www.latex-project.org/lppl/lppl-1-3c.txt
% and the latest version of this license is in
%    https://www.latex-project.org/lppl.txt
% and version 1.3 or later is part of all distributions of
% LaTeX version 2005/12/01 or later.
%
% This work has the LPPL maintenance status "maintained".
%
% The Current Maintainers of this work are
% Heiko Oberdiek and the Oberdiek Package Support Group
% https://github.com/ho-tex/oberdiek/issues
%
% The Base Interpreter refers to any `TeX-Format',
% because some files are installed in TDS:tex/generic//.
%
% This work consists of the main source file pdftexcmds.dtx
% and the derived files
%    pdftexcmds.sty, pdftexcmds.pdf, pdftexcmds.ins, pdftexcmds.drv,
%    pdftexcmds.bib, pdftexcmds-test1.tex, pdftexcmds-test2.tex,
%    pdftexcmds-test-shell.tex, pdftexcmds-test-escape.tex,
%    oberdiek.pdftexcmds.lua, pdftexcmds.lua.
%
% Distribution:
%    CTAN:macros/latex/contrib/oberdiek/pdftexcmds.dtx
%    CTAN:macros/latex/contrib/oberdiek/pdftexcmds.pdf
%
% Unpacking:
%    (a) If pdftexcmds.ins is present:
%           tex pdftexcmds.ins
%    (b) Without pdftexcmds.ins:
%           tex pdftexcmds.dtx
%    (c) If you insist on using LaTeX
%           latex \let\install=y% \iffalse meta-comment
%
% File: pdftexcmds.dtx
% Version: 2019/07/25 v0.30
% Info: Utility functions of pdfTeX for LuaTeX
%
% Copyright (C) 2007, 2009-2011 by
%    Heiko Oberdiek <heiko.oberdiek at googlemail.com>
%
% This work may be distributed and/or modified under the
% conditions of the LaTeX Project Public License, either
% version 1.3c of this license or (at your option) any later
% version. This version of this license is in
%    https://www.latex-project.org/lppl/lppl-1-3c.txt
% and the latest version of this license is in
%    https://www.latex-project.org/lppl.txt
% and version 1.3 or later is part of all distributions of
% LaTeX version 2005/12/01 or later.
%
% This work has the LPPL maintenance status "maintained".
%
% The Current Maintainers of this work are
% Heiko Oberdiek and the Oberdiek Package Support Group
% https://github.com/ho-tex/oberdiek/issues
%
% The Base Interpreter refers to any `TeX-Format',
% because some files are installed in TDS:tex/generic//.
%
% This work consists of the main source file pdftexcmds.dtx
% and the derived files
%    pdftexcmds.sty, pdftexcmds.pdf, pdftexcmds.ins, pdftexcmds.drv,
%    pdftexcmds.bib, pdftexcmds-test1.tex, pdftexcmds-test2.tex,
%    pdftexcmds-test-shell.tex, pdftexcmds-test-escape.tex,
%    oberdiek.pdftexcmds.lua, pdftexcmds.lua.
%
% Distribution:
%    CTAN:macros/latex/contrib/oberdiek/pdftexcmds.dtx
%    CTAN:macros/latex/contrib/oberdiek/pdftexcmds.pdf
%
% Unpacking:
%    (a) If pdftexcmds.ins is present:
%           tex pdftexcmds.ins
%    (b) Without pdftexcmds.ins:
%           tex pdftexcmds.dtx
%    (c) If you insist on using LaTeX
%           latex \let\install=y% \iffalse meta-comment
%
% File: pdftexcmds.dtx
% Version: 2019/07/25 v0.30
% Info: Utility functions of pdfTeX for LuaTeX
%
% Copyright (C) 2007, 2009-2011 by
%    Heiko Oberdiek <heiko.oberdiek at googlemail.com>
%
% This work may be distributed and/or modified under the
% conditions of the LaTeX Project Public License, either
% version 1.3c of this license or (at your option) any later
% version. This version of this license is in
%    https://www.latex-project.org/lppl/lppl-1-3c.txt
% and the latest version of this license is in
%    https://www.latex-project.org/lppl.txt
% and version 1.3 or later is part of all distributions of
% LaTeX version 2005/12/01 or later.
%
% This work has the LPPL maintenance status "maintained".
%
% The Current Maintainers of this work are
% Heiko Oberdiek and the Oberdiek Package Support Group
% https://github.com/ho-tex/oberdiek/issues
%
% The Base Interpreter refers to any `TeX-Format',
% because some files are installed in TDS:tex/generic//.
%
% This work consists of the main source file pdftexcmds.dtx
% and the derived files
%    pdftexcmds.sty, pdftexcmds.pdf, pdftexcmds.ins, pdftexcmds.drv,
%    pdftexcmds.bib, pdftexcmds-test1.tex, pdftexcmds-test2.tex,
%    pdftexcmds-test-shell.tex, pdftexcmds-test-escape.tex,
%    oberdiek.pdftexcmds.lua, pdftexcmds.lua.
%
% Distribution:
%    CTAN:macros/latex/contrib/oberdiek/pdftexcmds.dtx
%    CTAN:macros/latex/contrib/oberdiek/pdftexcmds.pdf
%
% Unpacking:
%    (a) If pdftexcmds.ins is present:
%           tex pdftexcmds.ins
%    (b) Without pdftexcmds.ins:
%           tex pdftexcmds.dtx
%    (c) If you insist on using LaTeX
%           latex \let\install=y\input{pdftexcmds.dtx}
%        (quote the arguments according to the demands of your shell)
%
% Documentation:
%    (a) If pdftexcmds.drv is present:
%           latex pdftexcmds.drv
%    (b) Without pdftexcmds.drv:
%           latex pdftexcmds.dtx; ...
%    The class ltxdoc loads the configuration file ltxdoc.cfg
%    if available. Here you can specify further options, e.g.
%    use A4 as paper format:
%       \PassOptionsToClass{a4paper}{article}
%
%    Programm calls to get the documentation (example):
%       pdflatex pdftexcmds.dtx
%       bibtex pdftexcmds.aux
%       makeindex -s gind.ist pdftexcmds.idx
%       pdflatex pdftexcmds.dtx
%       makeindex -s gind.ist pdftexcmds.idx
%       pdflatex pdftexcmds.dtx
%
% Installation:
%    TDS:tex/generic/oberdiek/pdftexcmds.sty
%    TDS:scripts/oberdiek/oberdiek.pdftexcmds.lua
%    TDS:scripts/oberdiek/pdftexcmds.lua
%    TDS:doc/latex/oberdiek/pdftexcmds.pdf
%    TDS:doc/latex/oberdiek/test/pdftexcmds-test1.tex
%    TDS:doc/latex/oberdiek/test/pdftexcmds-test2.tex
%    TDS:doc/latex/oberdiek/test/pdftexcmds-test-shell.tex
%    TDS:doc/latex/oberdiek/test/pdftexcmds-test-escape.tex
%    TDS:source/latex/oberdiek/pdftexcmds.dtx
%
%<*ignore>
\begingroup
  \catcode123=1 %
  \catcode125=2 %
  \def\x{LaTeX2e}%
\expandafter\endgroup
\ifcase 0\ifx\install y1\fi\expandafter
         \ifx\csname processbatchFile\endcsname\relax\else1\fi
         \ifx\fmtname\x\else 1\fi\relax
\else\csname fi\endcsname
%</ignore>
%<*install>
\input docstrip.tex
\Msg{************************************************************************}
\Msg{* Installation}
\Msg{* Package: pdftexcmds 2019/07/25 v0.30 Utility functions of pdfTeX for LuaTeX (HO)}
\Msg{************************************************************************}

\keepsilent
\askforoverwritefalse

\let\MetaPrefix\relax
\preamble

This is a generated file.

Project: pdftexcmds
Version: 2019/07/25 v0.30

Copyright (C) 2007, 2009-2011 by
   Heiko Oberdiek <heiko.oberdiek at googlemail.com>

This work may be distributed and/or modified under the
conditions of the LaTeX Project Public License, either
version 1.3c of this license or (at your option) any later
version. This version of this license is in
   https://www.latex-project.org/lppl/lppl-1-3c.txt
and the latest version of this license is in
   https://www.latex-project.org/lppl.txt
and version 1.3 or later is part of all distributions of
LaTeX version 2005/12/01 or later.

This work has the LPPL maintenance status "maintained".

The Current Maintainers of this work are
Heiko Oberdiek and the Oberdiek Package Support Group
https://github.com/ho-tex/oberdiek/issues


The Base Interpreter refers to any `TeX-Format',
because some files are installed in TDS:tex/generic//.

This work consists of the main source file pdftexcmds.dtx
and the derived files
   pdftexcmds.sty, pdftexcmds.pdf, pdftexcmds.ins, pdftexcmds.drv,
   pdftexcmds.bib, pdftexcmds-test1.tex, pdftexcmds-test2.tex,
   pdftexcmds-test-shell.tex, pdftexcmds-test-escape.tex,
   oberdiek.pdftexcmds.lua, pdftexcmds.lua.

\endpreamble
\let\MetaPrefix\DoubleperCent

\generate{%
  \file{pdftexcmds.ins}{\from{pdftexcmds.dtx}{install}}%
  \file{pdftexcmds.drv}{\from{pdftexcmds.dtx}{driver}}%
  \nopreamble
  \nopostamble
  \file{pdftexcmds.bib}{\from{pdftexcmds.dtx}{bib}}%
  \usepreamble\defaultpreamble
  \usepostamble\defaultpostamble
  \usedir{tex/generic/oberdiek}%
  \file{pdftexcmds.sty}{\from{pdftexcmds.dtx}{package}}%
%  \usedir{doc/latex/oberdiek/test}%
%  \file{pdftexcmds-test1.tex}{\from{pdftexcmds.dtx}{test1}}%
%  \file{pdftexcmds-test2.tex}{\from{pdftexcmds.dtx}{test2}}%
%  \file{pdftexcmds-test-shell.tex}{\from{pdftexcmds.dtx}{test-shell}}%
%  \file{pdftexcmds-test-escape.tex}{\from{pdftexcmds.dtx}{test-escape}}%
  \nopreamble
  \nopostamble
%  \usedir{source/latex/oberdiek/catalogue}%
%  \file{pdftexcmds.xml}{\from{pdftexcmds.dtx}{catalogue}}%
}
\def\MetaPrefix{-- }
\def\defaultpostamble{%
  \MetaPrefix^^J%
  \MetaPrefix\space End of File `\outFileName'.%
}
\def\currentpostamble{\defaultpostamble}%
\generate{%
  \usedir{scripts/oberdiek}%
  \file{oberdiek.pdftexcmds.lua}{\from{pdftexcmds.dtx}{lua}}%
  \file{pdftexcmds.lua}{\from{pdftexcmds.dtx}{lua}}%
}

\catcode32=13\relax% active space
\let =\space%
\Msg{************************************************************************}
\Msg{*}
\Msg{* To finish the installation you have to move the following}
\Msg{* file into a directory searched by TeX:}
\Msg{*}
\Msg{*     pdftexcmds.sty}
\Msg{*}
\Msg{* And install the following script files:}
\Msg{*}
\Msg{*     oberdiek.pdftexcmds.lua, pdftexcmds.lua}
\Msg{*}
\Msg{* To produce the documentation run the file `pdftexcmds.drv'}
\Msg{* through LaTeX.}
\Msg{*}
\Msg{* Happy TeXing!}
\Msg{*}
\Msg{************************************************************************}

\endbatchfile
%</install>
%<*bib>
@online{AndyThomas:Analog,
  author={Thomas, Andy},
  title={Analog of {\texttt{\csname textbackslash\endcsname}pdfelapsedtime} for
      {\hologo{LuaTeX}} and {\hologo{XeTeX}}},
  url={http://tex.stackexchange.com/a/32531},
  urldate={2011-11-29},
}
%</bib>
%<*ignore>
\fi
%</ignore>
%<*driver>
\NeedsTeXFormat{LaTeX2e}
\ProvidesFile{pdftexcmds.drv}%
  [2019/07/25 v0.30 Utility functions of pdfTeX for LuaTeX (HO)]%
\documentclass{ltxdoc}
\usepackage{holtxdoc}[2011/11/22]
\usepackage{paralist}
\usepackage{csquotes}
\usepackage[
  backend=bibtex,
  bibencoding=ascii,
  alldates=iso8601,
]{biblatex}[2011/11/13]
\bibliography{oberdiek-source}
\bibliography{pdftexcmds}
\begin{document}
  \DocInput{pdftexcmds.dtx}%
\end{document}
%</driver>
% \fi
%
%
% \CharacterTable
%  {Upper-case    \A\B\C\D\E\F\G\H\I\J\K\L\M\N\O\P\Q\R\S\T\U\V\W\X\Y\Z
%   Lower-case    \a\b\c\d\e\f\g\h\i\j\k\l\m\n\o\p\q\r\s\t\u\v\w\x\y\z
%   Digits        \0\1\2\3\4\5\6\7\8\9
%   Exclamation   \!     Double quote  \"     Hash (number) \#
%   Dollar        \$     Percent       \%     Ampersand     \&
%   Acute accent  \'     Left paren    \(     Right paren   \)
%   Asterisk      \*     Plus          \+     Comma         \,
%   Minus         \-     Point         \.     Solidus       \/
%   Colon         \:     Semicolon     \;     Less than     \<
%   Equals        \=     Greater than  \>     Question mark \?
%   Commercial at \@     Left bracket  \[     Backslash     \\
%   Right bracket \]     Circumflex    \^     Underscore    \_
%   Grave accent  \`     Left brace    \{     Vertical bar  \|
%   Right brace   \}     Tilde         \~}
%
% \GetFileInfo{pdftexcmds.drv}
%
% \title{The \xpackage{pdftexcmds} package}
% \date{2019/07/25 v0.30}
% \author{Heiko Oberdiek\thanks
% {Please report any issues at \url{https://github.com/ho-tex/oberdiek/issues}}}
%
% \maketitle
%
% \begin{abstract}
% \hologo{LuaTeX} provides most of the commands of \hologo{pdfTeX} 1.40. However
% a number of utility functions are removed. This package tries to fill
% the gap and implements some of the missing primitive using Lua.
% \end{abstract}
%
% \tableofcontents
%
% \def\csi#1{\texttt{\textbackslash\textit{#1}}}
%
% \section{Documentation}
%
% Some primitives of \hologo{pdfTeX} \cite{pdftex-manual}
% are not defined by \hologo{LuaTeX} \cite{luatex-manual}.
% This package implements macro based solutions using Lua code
% for the following missing \hologo{pdfTeX} primitives;
% \begin{compactitem}
% \item \cs{pdfstrcmp}
% \item \cs{pdfunescapehex}
% \item \cs{pdfescapehex}
% \item \cs{pdfescapename}
% \item \cs{pdfescapestring}
% \item \cs{pdffilesize}
% \item \cs{pdffilemoddate}
% \item \cs{pdffiledump}
% \item \cs{pdfmdfivesum}
% \item \cs{pdfresettimer}
% \item \cs{pdfelapsedtime}
% \item |\immediate\write18|
% \end{compactitem}
% The original names of the primitives cannot be used:
% \begin{itemize}
% \item
% The syntax for their arguments cannot easily
% simulated by macros. The primitives using key words
% such as |file| (\cs{pdfmdfivesum}) or |offset| and |length|
% (\cs{pdffiledump}) and uses \meta{general text} for the other
% arguments. Using token registers assignments, \meta{general text} could
% be catched. However, the simulated primitives are expandable
% and register assignments would destroy this important property.
% (\meta{general text} allows something like |\expandafter\bgroup ...}|.)
% \item
% The original primitives can be expanded using one expansion step.
% The new macros need two expansion steps because of the additional
% macro expansion. Example:
% \begin{quote}
%   |\expandafter\foo\pdffilemoddate{file}|\\
%   vs.\\
%   |\expandafter\expandafter\expandafter|\\
%   |\foo\pdf@filemoddate{file}|
% \end{quote}
% \end{itemize}
%
% \hologo{LuaTeX} isn't stable yet and thus the status of this package is
% \emph{experimental}. Feedback is welcome.
%
% \subsection{General principles}
%
% \begin{description}
% \item[Naming convention:]
%   Usually this package defines a macro |\pdf@|\meta{cmd} if
%   \hologo{pdfTeX} provides |\pdf|\meta{cmd}.
% \item[Arguments:] The order of arguments in |\pdf@|\meta{cmd}
%   is the same as for the corresponding primitive of \hologo{pdfTeX}.
%   The arguments are ordinary undelimited \hologo{TeX} arguments,
%   no \meta{general text} and without additional keywords.
% \item[Expandibility:]
%   The macro |\pdf@|\meta{cmd} is expandable if the
%   corresponding \hologo{pdfTeX} primitive has this property.
%   Exact two expansion steps are necessary (first is the macro
%   expansion) except for \cs{pdf@primitive} and \cs{pdf@ifprimitive}.
%   The latter ones are not macros, but have the direct meaning of the
%   primitive.
% \item[Without \hologo{LuaTeX}:]
%   The macros |\pdf@|\meta{cmd} are mapped to the commands
%   of \hologo{pdfTeX} if they are available. Otherwise they are undefined.
% \item[Availability:]
%   The macros that the packages provides are undefined, if
%   the necessary primitives are not found and cannot be
%   implemented by Lua.
% \end{description}
%
% \subsection{Macros}
%
% \subsubsection[Strings]{Strings \cite[``7.15 Strings'']{pdftex-manual}}
%
% \begin{declcs}{pdf@strcmp} \M{stringA} \M{stringB}
% \end{declcs}
% Same as |\pdfstrcmp{|\meta{stringA}|}{|\meta{stringB}|}|.
%
% \begin{declcs}{pdf@unescapehex} \M{string}
% \end{declcs}
% Same as |\pdfunescapehex{|\meta{string}|}|.
% The argument is a byte string given in hexadecimal notation.
% The result are character tokens from 0 until 255 with
% catcode 12 and the space with catcode 10.
%
% \begin{declcs}{pdf@escapehex} \M{string}\\
%   \cs{pdf@escapestring} \M{string}\\
%   \cs{pdf@escapename} \M{string}
% \end{declcs}
% Same as the primitives of \hologo{pdfTeX}. However \hologo{pdfTeX} does not
% know about characters with codes 256 and larger. Thus the
% string is treated as byte string, characters with more than
% eight bits are ignored.
%
% \subsubsection[Files]{Files \cite[``7.18 Files'']{pdftex-manual}}
%
% \begin{declcs}{pdf@filesize} \M{filename}
% \end{declcs}
% Same as |\pdffilesize{|\meta{filename}|}|.
%
% \begin{declcs}{pdf@filemoddate} \M{filename}
% \end{declcs}
% Same as |\pdffilemoddate{|\meta{filename}|}|.
%
% \begin{declcs}{pdf@filedump} \M{offset} \M{length} \M{filename}
% \end{declcs}
% Same as |\pdffiledump offset| \meta{offset} |length| \meta{length}
% |{|\meta{filename}|}|. Both \meta{offset} and \meta{length} must
% not be empty, but must be a valid \hologo{TeX} number.
%
% \begin{declcs}{pdf@mdfivesum} \M{string}
% \end{declcs}
% Same as |\pdfmdfivesum{|\meta{string}|}|. Keyword |file| is supported
% by macro \cs{pdf@filemdfivesum}.
%
% \begin{declcs}{pdf@filemdfivesum} \M{filename}
% \end{declcs}
% Same as |\pdfmdfivesum file{|\meta{filename}|}|.
%
% \subsubsection[Timekeeping]{Timekeeping \cite[``7.17 Timekeeping'']{pdftex-manual}}
%
% The timekeeping macros are based on Andy Thomas' work \cite{AndyThomas:Analog}.
%
% \begin{declcs}{pdf@resettimer}
% \end{declcs}
% Same as \cs{pdfresettimer}, it resets the internal timer.
%
% \begin{declcs}{pdf@elapsedtime}
% \end{declcs}
% Same as \cs{pdfelapsedtime}. It behaves like a read-only integer.
% For printing purposes it can be prefixed by \cs{the} or \cs{number}.
% It measures the time in scaled seconds (seconds multiplied with 65536)
% since the latest call of \cs{pdf@resettimer} or start of
% program/package. The resolution, the shortest time interval that
% can be measured, depends on the program and system.
% \begin{itemize}
% \item \hologo{pdfTeX} with |gettimeofday|: $\ge$ 1/65536\,s
% \item \hologo{pdfTeX} with |ftime|: $\ge$ 1\,ms
% \item \hologo{pdfTeX} with |time|: $\ge$ 1\,s
% \item \hologo{LuaTeX}: $\ge$ 10\,ms\\
%  (|os.clock()| returns a float number with two decimal digits in
%  \hologo{LuaTeX} beta-0.70.1-2011061416 (rev 4277)).
% \end{itemize}
%
% \subsubsection[Miscellaneous]{Miscellaneous \cite[``7.21 Miscellaneous'']{pdftex-manual}}
%
% \begin{declcs}{pdf@draftmode}
% \end{declcs}
% If the \TeX\ compiler knows \cs{pdfdraftmode} or \cs{draftmode}
% (\hologo{pdfTeX},
% \hologo{LuaTeX}), then \cs{pdf@draftmode} returns, whether
% this mode is enabled. The result is an implicit number:
% one means the draft mode is available and enabled.
% If the value is zero, then the mode is not active or
% \cs{pdfdraftmode} is not available.
% An explicit number is yielded by \cs{number}\cs{pdf@draftmode}.
% The macro cannot
% be used to change the mode, see \cs{pdf@setdraftmode}.
%
% \begin{declcs}{pdf@ifdraftmode} \M{true} \M{false}
% \end{declcs}
% If \cs{pdfdraftmode} is available and enabled, \meta{true} is
% called, otherwise \meta{false} is executed.
%
% \begin{declcs}{pdf@setdraftmode} \M{value}
% \end{declcs}
% Macro \cs{pdf@setdraftmode} expects the number zero or one as
% \meta{value}. Zero deactivates the mode and one enables the draft mode.
% The macro does not have an effect, if the feature \cs{pdfdraftmode} is not
% available.
%
% \begin{declcs}{pdf@shellescape}
% \end{declcs}
% Same as |\pdfshellescape|. It is or expands to |1| if external
% commands can be executed and |0| otherwise. In \hologo{pdfTeX} external
% commands must be enabled first by command line option or
% configuration option. In \hologo{LuaTeX} option |--safer| disables
% the execution of external commands.
%
% In \hologo{LuaTeX} before 0.68.0 \cs{pdf@shellescape} is not
% available due to a bug in |os.execute()|. The argumentless form
% crashes in some circumstances with segmentation fault.
% (It is fixed in version 0.68.0 or revision 4167 of \hologo{LuaTeX}.
% and packported to some version of 0.67.0).
%
% Hints for usage:
% \begin{itemize}
% \item Before its use \cs{pdf@shellescape} should be tested,
% whether it is available. Example with package \xpackage{ltxcmds}
% (loaded by package \xpackage{pdftexcmds}):
%\begin{quote}
%\begin{verbatim}
%\ltx@IfUndefined{pdf@shellescape}{%
%  % \pdf@shellescape is undefined
%}{%
%  % \pdf@shellescape is available
%}
%\end{verbatim}
%\end{quote}
% Use \cs{ltx@ifundefined} in expandable contexts.
% \item \cs{pdf@shellescape} might be a numerical constant,
% expands to the primitive, or expands to a plain number.
% Therefore use it in contexts where these differences does not matter.
% \item Use in comparisons, e.g.:
%   \begin{quote}
%     |\ifnum\pdf@shellescape=0 ...|
%   \end{quote}
% \item Print the number: |\number\pdf@shellescape|
% \end{itemize}
%
% \begin{declcs}{pdf@system} \M{cmdline}
% \end{declcs}
% It is a wrapper for |\immediate\write18| in \hologo{pdfTeX} or
% |os.execute| in \hologo{LuaTeX}.
%
% In theory |os.execute|
% returns a status number. But its meaning is quite
% undefined. Are there some reliable properties?
% Does it make sense to provide an user interface to
% this status exit code?
%
% \begin{declcs}{pdf@primitive} \csi{cmd}
% \end{declcs}
% Same as \cs{pdfprimitive} in \hologo{pdfTeX} or \hologo{LuaTeX}.
% In \hologo{XeTeX} the
% primitive is called \cs{primitive}. Despite the current definition
% of the command \csi{cmd}, it's meaning as primitive is used.
%
% \begin{declcs}{pdf@ifprimitive} \csi{cmd}
% \end{declcs}
% Same as \cs{ifpdfprimitive} in \hologo{pdfTeX} or
% \hologo{LuaTeX}. \hologo{XeTeX} calls
% it \cs{ifprimitive}. It is a switch that checks if the command
% \csi{cmd} has it's primitive meaning.
%
% \subsubsection{Additional macro: \cs{pdf@isprimitive}}
%
% \begin{declcs}{pdf@isprimitive} \csi{cmd1} \csi{cmd2} \M{true} \M{false}
% \end{declcs}
% If \csi{cmd1} has the primitive meaning given by the primitive name
% of \csi{cmd2}, then the argument \meta{true} is executed, otherwise
% \meta{false}. The macro \cs{pdf@isprimitive} is expandable.
% Internally it checks the result of \cs{meaning} and is therefore
% available for all \hologo{TeX} variants, even the original \hologo{TeX}.
% Example with \hologo{LaTeX}:
%\begin{quote}
%\begin{verbatim}
%\makeatletter
%\pdf@isprimitive{@@input}{input}{%
%  \typeout{\string\@@input\space is original\string\input}%
%}{%
%  \typeout{Oops, \string\@@input\space is not the %
%           original\string\input}%
%}
%\end{verbatim}
%\end{quote}
%
% \subsubsection{Experimental}
%
% \begin{declcs}{pdf@unescapehexnative} \M{string}\\
%   \cs{pdf@escapehexnative} \M{string}\\
%   \cs{pdf@escapenamenative} \M{string}\\
%   \cs{pdf@mdfivesumnative} \M{string}
% \end{declcs}
% The variants without |native| in the macro name are supposed to
% be compatible with \hologo{pdfTeX}. However characters with more than
% eight bits are not supported and are ignored. If \hologo{LuaTeX} is
% running, then its UTF-8 coded strings are used. Thus the full
% unicode character range is supported. However the result
% differs from \hologo{pdfTeX} for characters with eight or more bits.
%
% \begin{declcs}{pdf@pipe} \M{cmdline}
% \end{declcs}
% It calls \meta{cmdline} and returns the output of the external
% program in the usual manner as byte string (catcode 12, space with
% catcode 10). The Lua documentation says, that the used |io.popen|
% may not be available on all platforms. Then macro \cs{pdf@pipe}
% is undefined.
%
% \StopEventually{
% }
%
% \section{Implementation}
%
%    \begin{macrocode}
%<*package>
%    \end{macrocode}
%
% \subsection{Reload check and package identification}
%    Reload check, especially if the package is not used with \LaTeX.
%    \begin{macrocode}
\begingroup\catcode61\catcode48\catcode32=10\relax%
  \catcode13=5 % ^^M
  \endlinechar=13 %
  \catcode35=6 % #
  \catcode39=12 % '
  \catcode44=12 % ,
  \catcode45=12 % -
  \catcode46=12 % .
  \catcode58=12 % :
  \catcode64=11 % @
  \catcode123=1 % {
  \catcode125=2 % }
  \expandafter\let\expandafter\x\csname ver@pdftexcmds.sty\endcsname
  \ifx\x\relax % plain-TeX, first loading
  \else
    \def\empty{}%
    \ifx\x\empty % LaTeX, first loading,
      % variable is initialized, but \ProvidesPackage not yet seen
    \else
      \expandafter\ifx\csname PackageInfo\endcsname\relax
        \def\x#1#2{%
          \immediate\write-1{Package #1 Info: #2.}%
        }%
      \else
        \def\x#1#2{\PackageInfo{#1}{#2, stopped}}%
      \fi
      \x{pdftexcmds}{The package is already loaded}%
      \aftergroup\endinput
    \fi
  \fi
\endgroup%
%    \end{macrocode}
%    Package identification:
%    \begin{macrocode}
\begingroup\catcode61\catcode48\catcode32=10\relax%
  \catcode13=5 % ^^M
  \endlinechar=13 %
  \catcode35=6 % #
  \catcode39=12 % '
  \catcode40=12 % (
  \catcode41=12 % )
  \catcode44=12 % ,
  \catcode45=12 % -
  \catcode46=12 % .
  \catcode47=12 % /
  \catcode58=12 % :
  \catcode64=11 % @
  \catcode91=12 % [
  \catcode93=12 % ]
  \catcode123=1 % {
  \catcode125=2 % }
  \expandafter\ifx\csname ProvidesPackage\endcsname\relax
    \def\x#1#2#3[#4]{\endgroup
      \immediate\write-1{Package: #3 #4}%
      \xdef#1{#4}%
    }%
  \else
    \def\x#1#2[#3]{\endgroup
      #2[{#3}]%
      \ifx#1\@undefined
        \xdef#1{#3}%
      \fi
      \ifx#1\relax
        \xdef#1{#3}%
      \fi
    }%
  \fi
\expandafter\x\csname ver@pdftexcmds.sty\endcsname
\ProvidesPackage{pdftexcmds}%
  [2019/07/25 v0.30 Utility functions of pdfTeX for LuaTeX (HO)]%
%    \end{macrocode}
%
% \subsection{Catcodes}
%
%    \begin{macrocode}
\begingroup\catcode61\catcode48\catcode32=10\relax%
  \catcode13=5 % ^^M
  \endlinechar=13 %
  \catcode123=1 % {
  \catcode125=2 % }
  \catcode64=11 % @
  \def\x{\endgroup
    \expandafter\edef\csname pdftexcmds@AtEnd\endcsname{%
      \endlinechar=\the\endlinechar\relax
      \catcode13=\the\catcode13\relax
      \catcode32=\the\catcode32\relax
      \catcode35=\the\catcode35\relax
      \catcode61=\the\catcode61\relax
      \catcode64=\the\catcode64\relax
      \catcode123=\the\catcode123\relax
      \catcode125=\the\catcode125\relax
    }%
  }%
\x\catcode61\catcode48\catcode32=10\relax%
\catcode13=5 % ^^M
\endlinechar=13 %
\catcode35=6 % #
\catcode64=11 % @
\catcode123=1 % {
\catcode125=2 % }
\def\TMP@EnsureCode#1#2{%
  \edef\pdftexcmds@AtEnd{%
    \pdftexcmds@AtEnd
    \catcode#1=\the\catcode#1\relax
  }%
  \catcode#1=#2\relax
}
\TMP@EnsureCode{0}{12}%
\TMP@EnsureCode{1}{12}%
\TMP@EnsureCode{2}{12}%
\TMP@EnsureCode{10}{12}% ^^J
\TMP@EnsureCode{33}{12}% !
\TMP@EnsureCode{34}{12}% "
\TMP@EnsureCode{38}{4}% &
\TMP@EnsureCode{39}{12}% '
\TMP@EnsureCode{40}{12}% (
\TMP@EnsureCode{41}{12}% )
\TMP@EnsureCode{42}{12}% *
\TMP@EnsureCode{43}{12}% +
\TMP@EnsureCode{44}{12}% ,
\TMP@EnsureCode{45}{12}% -
\TMP@EnsureCode{46}{12}% .
\TMP@EnsureCode{47}{12}% /
\TMP@EnsureCode{58}{12}% :
\TMP@EnsureCode{60}{12}% <
\TMP@EnsureCode{62}{12}% >
\TMP@EnsureCode{91}{12}% [
\TMP@EnsureCode{93}{12}% ]
\TMP@EnsureCode{94}{7}% ^ (superscript)
\TMP@EnsureCode{95}{12}% _ (other)
\TMP@EnsureCode{96}{12}% `
\TMP@EnsureCode{126}{12}% ~ (other)
\edef\pdftexcmds@AtEnd{%
  \pdftexcmds@AtEnd
  \escapechar=\number\escapechar\relax
  \noexpand\endinput
}
\escapechar=92 %
%    \end{macrocode}
%
% \subsection{Load packages}
%
%    \begin{macrocode}
\begingroup\expandafter\expandafter\expandafter\endgroup
\expandafter\ifx\csname RequirePackage\endcsname\relax
  \def\TMP@RequirePackage#1[#2]{%
    \begingroup\expandafter\expandafter\expandafter\endgroup
    \expandafter\ifx\csname ver@#1.sty\endcsname\relax
      \input #1.sty\relax
    \fi
  }%
  \TMP@RequirePackage{infwarerr}[2007/09/09]%
  \TMP@RequirePackage{ifluatex}[2010/03/01]%
  \TMP@RequirePackage{ltxcmds}[2010/12/02]%
  \TMP@RequirePackage{ifpdf}[2010/09/13]%
\else
  \RequirePackage{infwarerr}[2007/09/09]%
  \RequirePackage{ifluatex}[2010/03/01]%
  \RequirePackage{ltxcmds}[2010/12/02]%
  \RequirePackage{ifpdf}[2010/09/13]%
\fi
%    \end{macrocode}
%
% \subsection{Without \hologo{LuaTeX}}
%
%    \begin{macrocode}
\ifluatex
\else
  \@PackageInfoNoLine{pdftexcmds}{LuaTeX not detected}%
  \def\pdftexcmds@nopdftex{%
    \@PackageInfoNoLine{pdftexcmds}{pdfTeX >= 1.30 not detected}%
    \let\pdftexcmds@nopdftex\relax
  }%
  \def\pdftexcmds@temp#1{%
    \begingroup\expandafter\expandafter\expandafter\endgroup
    \expandafter\ifx\csname pdf#1\endcsname\relax
      \pdftexcmds@nopdftex
    \else
      \expandafter\def\csname pdf@#1\expandafter\endcsname
      \expandafter##\expandafter{%
        \csname pdf#1\endcsname
      }%
    \fi
  }%
  \pdftexcmds@temp{strcmp}%
  \pdftexcmds@temp{escapehex}%
  \let\pdf@escapehexnative\pdf@escapehex
  \pdftexcmds@temp{unescapehex}%
  \let\pdf@unescapehexnative\pdf@unescapehex
  \pdftexcmds@temp{escapestring}%
  \pdftexcmds@temp{escapename}%
  \pdftexcmds@temp{filesize}%
  \pdftexcmds@temp{filemoddate}%
  \begingroup\expandafter\expandafter\expandafter\endgroup
  \expandafter\ifx\csname pdfshellescape\endcsname\relax
    \pdftexcmds@nopdftex
    \ltx@IfUndefined{pdftexversion}{%
    }{%
      \ifnum\pdftexversion>120 % 1.21a supports \ifeof18
        \ifeof18 %
          \chardef\pdf@shellescape=0 %
        \else
          \chardef\pdf@shellescape=1 %
        \fi
      \fi
    }%
  \else
    \def\pdf@shellescape{%
      \pdfshellescape
    }%
  \fi
  \begingroup\expandafter\expandafter\expandafter\endgroup
  \expandafter\ifx\csname pdffiledump\endcsname\relax
    \pdftexcmds@nopdftex
  \else
    \def\pdf@filedump#1#2#3{%
      \pdffiledump offset#1 length#2{#3}%
    }%
  \fi
%    \end{macrocode}
%    \begin{macrocode}
  \begingroup\expandafter\expandafter\expandafter\endgroup
  \expandafter\ifx\csname pdfmdfivesum\endcsname\relax
    \begingroup\expandafter\expandafter\expandafter\endgroup
    \expandafter\ifx\csname mdfivesum\endcsname\relax
      \pdftexcmds@nopdftex
    \else
      \def\pdf@mdfivesum#{\mdfivesum}%
      \let\pdf@mdfivesumnative\pdf@mdfivesum
      \def\pdf@filemdfivesum#{\mdfivesum file}%
    \fi
  \else
    \def\pdf@mdfivesum#{\pdfmdfivesum}%
    \let\pdf@mdfivesumnative\pdf@mdfivesum
    \def\pdf@filemdfivesum#{\pdfmdfivesum file}%
  \fi
%    \end{macrocode}
%    \begin{macrocode}
  \def\pdf@system#{%
    \immediate\write18%
  }%
  \def\pdftexcmds@temp#1{%
    \begingroup\expandafter\expandafter\expandafter\endgroup
    \expandafter\ifx\csname pdf#1\endcsname\relax
      \pdftexcmds@nopdftex
    \else
      \expandafter\let\csname pdf@#1\expandafter\endcsname
      \csname pdf#1\endcsname
    \fi
  }%
  \pdftexcmds@temp{resettimer}%
  \pdftexcmds@temp{elapsedtime}%
\fi
%    \end{macrocode}
%
% \subsection{\cs{pdf@primitive}, \cs{pdf@ifprimitive}}
%
%    Since version 1.40.0 \hologo{pdfTeX} has \cs{pdfprimitive} and
%    \cs{ifpdfprimitive}. And \cs{pdfprimitive} was fixed in
%    version 1.40.4.
%
%    \hologo{XeTeX} provides them under the name \cs{primitive} and
%    \cs{ifprimitive}. \hologo{LuaTeX} knows both name variants,
%    but they have possibly to be enabled first (|tex.enableprimitives|).
%
%    Depending on the format TeX Live uses a prefix |luatex|.
%
%    Caution: \cs{let} must be used for the definition of
%    the macros, especially because of \cs{ifpdfprimitive}.
%
% \subsubsection{Using \hologo{LuaTeX}'s \texttt{tex.enableprimitives}}
%
%    \begin{macrocode}
\ifluatex
%    \end{macrocode}
%    \begin{macro}{\pdftexcmds@directlua}
%    \begin{macrocode}
  \ifnum\luatexversion<36 %
    \def\pdftexcmds@directlua{\directlua0 }%
  \else
    \let\pdftexcmds@directlua\directlua
  \fi
%    \end{macrocode}
%    \end{macro}
%
%    \begin{macrocode}
  \begingroup
    \newlinechar=10 %
    \endlinechar=\newlinechar
    \pdftexcmds@directlua{%
      if tex.enableprimitives then
        tex.enableprimitives(
          'pdf@',
          {'primitive', 'ifprimitive', 'pdfdraftmode','draftmode'}
        )
        tex.enableprimitives('', {'luaescapestring'})
      end
    }%
  \endgroup %
%    \end{macrocode}
%
%    \begin{macrocode}
\fi
%    \end{macrocode}
%
% \subsubsection{Trying various names to find the primitives}
%
%    \begin{macro}{\pdftexcmds@strip@prefix}
%    \begin{macrocode}
\def\pdftexcmds@strip@prefix#1>{}
%    \end{macrocode}
%    \end{macro}
%    \begin{macrocode}
\def\pdftexcmds@temp#1#2#3{%
  \begingroup\expandafter\expandafter\expandafter\endgroup
  \expandafter\ifx\csname pdf@#1\endcsname\relax
    \begingroup
      \def\x{#3}%
      \edef\x{\expandafter\pdftexcmds@strip@prefix\meaning\x}%
      \escapechar=-1 %
      \edef\y{\expandafter\meaning\csname#2\endcsname}%
    \expandafter\endgroup
    \ifx\x\y
      \expandafter\let\csname pdf@#1\expandafter\endcsname
      \csname #2\endcsname
    \fi
  \fi
}
%    \end{macrocode}
%
%    \begin{macro}{\pdf@primitive}
%    \begin{macrocode}
\pdftexcmds@temp{primitive}{pdfprimitive}{pdfprimitive}% pdfTeX, oldLuaTeX
\pdftexcmds@temp{primitive}{primitive}{primitive}% XeTeX, luatex
\pdftexcmds@temp{primitive}{luatexprimitive}{pdfprimitive}% oldLuaTeX
\pdftexcmds@temp{primitive}{luatexpdfprimitive}{pdfprimitive}% oldLuaTeX
%    \end{macrocode}
%    \end{macro}
%    \begin{macro}{\pdf@ifprimitive}
%    \begin{macrocode}
\pdftexcmds@temp{ifprimitive}{ifpdfprimitive}{ifpdfprimitive}% pdfTeX, oldLuaTeX
\pdftexcmds@temp{ifprimitive}{ifprimitive}{ifprimitive}% XeTeX, luatex
\pdftexcmds@temp{ifprimitive}{luatexifprimitive}{ifpdfprimitive}% oldLuaTeX
\pdftexcmds@temp{ifprimitive}{luatexifpdfprimitive}{ifpdfprimitive}% oldLuaTeX
%    \end{macrocode}
%    \end{macro}
%
%    Disable broken \cs{pdfprimitive}.
%    \begin{macrocode}
\ifluatex\else
\begingroup
  \expandafter\ifx\csname pdf@primitive\endcsname\relax
  \else
    \expandafter\ifx\csname pdftexversion\endcsname\relax
    \else
      \ifnum\pdftexversion=140 %
        \expandafter\ifx\csname pdftexrevision\endcsname\relax
        \else
          \ifnum\pdftexrevision<4 %
            \endgroup
            \let\pdf@primitive\@undefined
            \@PackageInfoNoLine{pdftexcmds}{%
              \string\pdf@primitive\space disabled, %
              because\MessageBreak
              \string\pdfprimitive\space is broken until pdfTeX 1.40.4%
            }%
            \begingroup
          \fi
        \fi
      \fi
    \fi
  \fi
\endgroup
\fi
%    \end{macrocode}
%
% \subsubsection{Result}
%
%    \begin{macrocode}
\begingroup
  \@PackageInfoNoLine{pdftexcmds}{%
    \string\pdf@primitive\space is %
    \expandafter\ifx\csname pdf@primitive\endcsname\relax not \fi
    available%
  }%
  \@PackageInfoNoLine{pdftexcmds}{%
    \string\pdf@ifprimitive\space is %
    \expandafter\ifx\csname pdf@ifprimitive\endcsname\relax not \fi
    available%
  }%
\endgroup
%    \end{macrocode}
%
% \subsection{\hologo{XeTeX}}
%
%    Look for primitives \cs{shellescape}, \cs{strcmp}.
%    \begin{macrocode}
\def\pdftexcmds@temp#1{%
  \begingroup\expandafter\expandafter\expandafter\endgroup
  \expandafter\ifx\csname pdf@#1\endcsname\relax
    \begingroup
      \escapechar=-1 %
      \edef\x{\expandafter\meaning\csname#1\endcsname}%
      \def\y{#1}%
      \def\z##1->{}%
      \edef\y{\expandafter\z\meaning\y}%
    \expandafter\endgroup
    \ifx\x\y
      \expandafter\def\csname pdf@#1\expandafter\endcsname
      \expandafter{%
        \csname#1\endcsname
      }%
    \fi
  \fi
}%
\pdftexcmds@temp{shellescape}%
\pdftexcmds@temp{strcmp}%
%    \end{macrocode}
%
% \subsection{\cs{pdf@isprimitive}}
%
%    \begin{macrocode}
\def\pdf@isprimitive{%
  \begingroup\expandafter\expandafter\expandafter\endgroup
  \expandafter\ifx\csname pdf@strcmp\endcsname\relax
    \long\def\pdf@isprimitive##1{%
      \expandafter\pdftexcmds@isprimitive\expandafter{\meaning##1}%
    }%
    \long\def\pdftexcmds@isprimitive##1##2{%
      \expandafter\pdftexcmds@@isprimitive\expandafter{\string##2}{##1}%
    }%
    \def\pdftexcmds@@isprimitive##1##2{%
      \ifnum0\pdftexcmds@equal##1\delimiter##2\delimiter=1 %
        \expandafter\ltx@firstoftwo
      \else
        \expandafter\ltx@secondoftwo
      \fi
    }%
    \def\pdftexcmds@equal##1##2\delimiter##3##4\delimiter{%
      \ifx##1##3%
        \ifx\relax##2##4\relax
          1%
        \else
          \ifx\relax##2\relax
          \else
            \ifx\relax##4\relax
            \else
              \pdftexcmds@equalcont{##2}{##4}%
            \fi
          \fi
        \fi
      \fi
    }%
    \def\pdftexcmds@equalcont##1{%
      \def\pdftexcmds@equalcont####1####2##1##1##1##1{%
        ##1##1##1##1%
        \pdftexcmds@equal####1\delimiter####2\delimiter
      }%
    }%
    \expandafter\pdftexcmds@equalcont\csname fi\endcsname
  \else
    \long\def\pdf@isprimitive##1##2{%
      \ifnum\pdf@strcmp{\meaning##1}{\string##2}=0 %
        \expandafter\ltx@firstoftwo
      \else
        \expandafter\ltx@secondoftwo
      \fi
    }%
  \fi
}
\ifluatex
\ifx\pdfdraftmode\@undefined
  \let\pdfdraftmode\draftmode
\fi
\else
  \pdf@isprimitive
\fi
%    \end{macrocode}
%
% \subsection{\cs{pdf@draftmode}}
%
%
%    \begin{macrocode}
\let\pdftexcmds@temp\ltx@zero %
\ltx@IfUndefined{pdfdraftmode}{%
  \@PackageInfoNoLine{pdftexcmds}{\ltx@backslashchar pdfdraftmode not found}%
}{%
  \ifpdf
    \let\pdftexcmds@temp\ltx@one
    \@PackageInfoNoLine{pdftexcmds}{\ltx@backslashchar pdfdraftmode found}%
  \else
    \@PackageInfoNoLine{pdftexcmds}{%
      \ltx@backslashchar pdfdraftmode is ignored in DVI mode%
    }%
  \fi
}
\ifcase\pdftexcmds@temp
%    \end{macrocode}
%    \begin{macro}{\pdf@draftmode}
%    \begin{macrocode}
  \let\pdf@draftmode\ltx@zero
%    \end{macrocode}
%    \end{macro}
%    \begin{macro}{\pdf@ifdraftmode}
%    \begin{macrocode}
  \let\pdf@ifdraftmode\ltx@secondoftwo
%    \end{macrocode}
%    \end{macro}
%    \begin{macro}{\pdftexcmds@setdraftmode}
%    \begin{macrocode}
  \def\pdftexcmds@setdraftmode#1{}%
%    \end{macrocode}
%    \end{macro}
%    \begin{macrocode}
\else
%    \end{macrocode}
%    \begin{macro}{\pdftexcmds@draftmode}
%    \begin{macrocode}
  \let\pdftexcmds@draftmode\pdfdraftmode
%    \end{macrocode}
%    \end{macro}
%    \begin{macro}{\pdf@ifdraftmode}
%    \begin{macrocode}
  \def\pdf@ifdraftmode{%
    \ifnum\pdftexcmds@draftmode=\ltx@one
      \expandafter\ltx@firstoftwo
    \else
      \expandafter\ltx@secondoftwo
    \fi
  }%
%    \end{macrocode}
%    \end{macro}
%    \begin{macro}{\pdf@draftmode}
%    \begin{macrocode}
  \def\pdf@draftmode{%
    \ifnum\pdftexcmds@draftmode=\ltx@one
      \expandafter\ltx@one
    \else
      \expandafter\ltx@zero
    \fi
  }%
%    \end{macrocode}
%    \end{macro}
%    \begin{macro}{\pdftexcmds@setdraftmode}
%    \begin{macrocode}
  \def\pdftexcmds@setdraftmode#1{%
    \pdftexcmds@draftmode=#1\relax
  }%
%    \end{macrocode}
%    \end{macro}
%    \begin{macrocode}
\fi
%    \end{macrocode}
%    \begin{macro}{\pdf@setdraftmode}
%    \begin{macrocode}
\def\pdf@setdraftmode#1{%
  \begingroup
    \count\ltx@cclv=#1\relax
  \edef\x{\endgroup
    \noexpand\pdftexcmds@@setdraftmode{\the\count\ltx@cclv}%
  }%
  \x
}
%    \end{macrocode}
%    \end{macro}
%    \begin{macro}{\pdftexcmds@@setdraftmode}
%    \begin{macrocode}
\def\pdftexcmds@@setdraftmode#1{%
  \ifcase#1 %
    \pdftexcmds@setdraftmode{#1}%
  \or
    \pdftexcmds@setdraftmode{#1}%
  \else
    \@PackageWarning{pdftexcmds}{%
      \string\pdf@setdraftmode: Ignoring\MessageBreak
      invalid value `#1'%
    }%
  \fi
}
%    \end{macrocode}
%    \end{macro}
%
% \subsection{Load Lua module}
%
%    \begin{macrocode}
\ifluatex
\else
  \expandafter\pdftexcmds@AtEnd
\fi%
%    \end{macrocode}
%
%    \begin{macrocode}
\ifnum\luatexversion<80
  \begingroup\expandafter\expandafter\expandafter\endgroup
  \expandafter\ifx\csname RequirePackage\endcsname\relax
    \def\TMP@RequirePackage#1[#2]{%
      \begingroup\expandafter\expandafter\expandafter\endgroup
      \expandafter\ifx\csname ver@#1.sty\endcsname\relax
        \input #1.sty\relax
      \fi
    }%
    \TMP@RequirePackage{luatex-loader}[2009/04/10]%
  \else
    \RequirePackage{luatex-loader}[2009/04/10]%
  \fi
\fi
\pdftexcmds@directlua{%
  require("pdftexcmds")%
}
\ifnum\luatexversion>37 %
  \ifnum0%
      \pdftexcmds@directlua{%
        if status.ini_version then %
          tex.write("1")%
        end%
      }>0 %
    \everyjob\expandafter{%
      \the\everyjob
      \pdftexcmds@directlua{%
        require("pdftexcmds")%
      }%
    }%
  \fi
\fi
\begingroup
  \def\x{2019/07/25 v0.30}%
  \ltx@onelevel@sanitize\x
  \edef\y{%
    \pdftexcmds@directlua{%
      if oberdiek.pdftexcmds.getversion then %
        oberdiek.pdftexcmds.getversion()%
      end%
    }%
  }%
  \ifx\x\y
  \else
    \@PackageError{pdftexcmds}{%
      Wrong version of lua module.\MessageBreak
      Package version: \x\MessageBreak
      Lua module: \y
    }\@ehc
  \fi
\endgroup
%    \end{macrocode}
%
% \subsection{Lua functions}
%
% \subsubsection{Helper macros}
%
%    \begin{macro}{\pdftexcmds@toks}
%    \begin{macrocode}
\begingroup\expandafter\expandafter\expandafter\endgroup
\expandafter\ifx\csname newtoks\endcsname\relax
  \toksdef\pdftexcmds@toks=0 %
\else
  \csname newtoks\endcsname\pdftexcmds@toks
\fi
%    \end{macrocode}
%    \end{macro}
%
%    \begin{macro}{\pdftexcmds@Patch}
%    \begin{macrocode}
\def\pdftexcmds@Patch{0}
\ifnum\luatexversion>40 %
  \ifnum\luatexversion<66 %
    \def\pdftexcmds@Patch{1}%
  \fi
\fi
%    \end{macrocode}
%    \end{macro}
%    \begin{macrocode}
\ifcase\pdftexcmds@Patch
  \catcode`\&=14 %
\else
  \catcode`\&=9 %
%    \end{macrocode}
%    \begin{macro}{\pdftexcmds@PatchDecode}
%    \begin{macrocode}
  \def\pdftexcmds@PatchDecode#1\@nil{%
    \pdftexcmds@DecodeA#1^^A^^A\@nil{}%
  }%
%    \end{macrocode}
%    \end{macro}
%    \begin{macro}{\pdftexcmds@DecodeA}
%    \begin{macrocode}
  \def\pdftexcmds@DecodeA#1^^A^^A#2\@nil#3{%
    \ifx\relax#2\relax
      \ltx@ReturnAfterElseFi{%
        \pdftexcmds@DecodeB#3#1^^A^^B\@nil{}%
      }%
    \else
      \ltx@ReturnAfterFi{%
        \pdftexcmds@DecodeA#2\@nil{#3#1^^@}%
      }%
    \fi
  }%
%    \end{macrocode}
%    \end{macro}
%    \begin{macro}{\pdftexcmds@DecodeB}
%    \begin{macrocode}
  \def\pdftexcmds@DecodeB#1^^A^^B#2\@nil#3{%
    \ifx\relax#2\relax%
      \ltx@ReturnAfterElseFi{%
        \ltx@zero
        #3#1%
      }%
    \else
      \ltx@ReturnAfterFi{%
        \pdftexcmds@DecodeB#2\@nil{#3#1^^A}%
      }%
    \fi
  }%
%    \end{macrocode}
%    \end{macro}
%    \begin{macrocode}
\fi
%    \end{macrocode}
%
%    \begin{macrocode}
\ifnum\luatexversion<36 %
\else
  \catcode`\0=9 %
\fi
%    \end{macrocode}
%
% \subsubsection[Strings]{Strings \cite[``7.15 Strings'']{pdftex-manual}}
%
%    \begin{macro}{\pdf@strcmp}
%    \begin{macrocode}
\long\def\pdf@strcmp#1#2{%
  \directlua0{%
    oberdiek.pdftexcmds.strcmp("\luaescapestring{#1}",%
        "\luaescapestring{#2}")%
  }%
}%
%    \end{macrocode}
%    \end{macro}
%    \begin{macrocode}
\pdf@isprimitive
%    \end{macrocode}
%    \begin{macro}{\pdf@escapehex}
%    \begin{macrocode}
\long\def\pdf@escapehex#1{%
  \directlua0{%
    oberdiek.pdftexcmds.escapehex("\luaescapestring{#1}", "byte")%
  }%
}%
%    \end{macrocode}
%    \end{macro}
%    \begin{macro}{\pdf@escapehexnative}
%    \begin{macrocode}
\long\def\pdf@escapehexnative#1{%
  \directlua0{%
    oberdiek.pdftexcmds.escapehex("\luaescapestring{#1}")%
  }%
}%
%    \end{macrocode}
%    \end{macro}
%    \begin{macro}{\pdf@unescapehex}
%    \begin{macrocode}
\def\pdf@unescapehex#1{%
& \romannumeral\expandafter\pdftexcmds@PatchDecode
  \the\expandafter\pdftexcmds@toks
  \directlua0{%
    oberdiek.pdftexcmds.toks="pdftexcmds@toks"%
    oberdiek.pdftexcmds.unescapehex("\luaescapestring{#1}", "byte", \pdftexcmds@Patch)%
  }%
& \@nil
}%
%    \end{macrocode}
%    \end{macro}
%    \begin{macro}{\pdf@unescapehexnative}
%    \begin{macrocode}
\def\pdf@unescapehexnative#1{%
& \romannumeral\expandafter\pdftexcmds@PatchDecode
  \the\expandafter\pdftexcmds@toks
  \directlua0{%
    oberdiek.pdftexcmds.toks="pdftexcmds@toks"%
    oberdiek.pdftexcmds.unescapehex("\luaescapestring{#1}", \pdftexcmds@Patch)%
  }%
& \@nil
}%
%    \end{macrocode}
%    \end{macro}
%    \begin{macro}{\pdf@escapestring}
%    \begin{macrocode}
\long\def\pdf@escapestring#1{%
  \directlua0{%
    oberdiek.pdftexcmds.escapestring("\luaescapestring{#1}", "byte")%
  }%
}
%    \end{macrocode}
%    \end{macro}
%    \begin{macro}{\pdf@escapename}
%    \begin{macrocode}
\long\def\pdf@escapename#1{%
  \directlua0{%
    oberdiek.pdftexcmds.escapename("\luaescapestring{#1}", "byte")%
  }%
}
%    \end{macrocode}
%    \end{macro}
%    \begin{macro}{\pdf@escapenamenative}
%    \begin{macrocode}
\long\def\pdf@escapenamenative#1{%
  \directlua0{%
    oberdiek.pdftexcmds.escapename("\luaescapestring{#1}")%
  }%
}
%    \end{macrocode}
%    \end{macro}
%
% \subsubsection[Files]{Files \cite[``7.18 Files'']{pdftex-manual}}
%
%    \begin{macro}{\pdf@filesize}
%    \begin{macrocode}
\def\pdf@filesize#1{%
  \directlua0{%
    oberdiek.pdftexcmds.filesize("\luaescapestring{#1}")%
  }%
}
%    \end{macrocode}
%    \end{macro}
%    \begin{macro}{\pdf@filemoddate}
%    \begin{macrocode}
\def\pdf@filemoddate#1{%
  \directlua0{%
    oberdiek.pdftexcmds.filemoddate("\luaescapestring{#1}")%
  }%
}
%    \end{macrocode}
%    \end{macro}
%    \begin{macro}{\pdf@filedump}
%    \begin{macrocode}
\def\pdf@filedump#1#2#3{%
  \directlua0{%
    oberdiek.pdftexcmds.filedump("\luaescapestring{\number#1}",%
        "\luaescapestring{\number#2}",%
        "\luaescapestring{#3}")%
  }%
}%
%    \end{macrocode}
%    \end{macro}
%    \begin{macro}{\pdf@mdfivesum}
%    \begin{macrocode}
\long\def\pdf@mdfivesum#1{%
  \directlua0{%
    oberdiek.pdftexcmds.mdfivesum("\luaescapestring{#1}", "byte")%
  }%
}%
%    \end{macrocode}
%    \end{macro}
%    \begin{macro}{\pdf@mdfivesumnative}
%    \begin{macrocode}
\long\def\pdf@mdfivesumnative#1{%
  \directlua0{%
    oberdiek.pdftexcmds.mdfivesum("\luaescapestring{#1}")%
  }%
}%
%    \end{macrocode}
%    \end{macro}
%    \begin{macro}{\pdf@filemdfivesum}
%    \begin{macrocode}
\def\pdf@filemdfivesum#1{%
  \directlua0{%
    oberdiek.pdftexcmds.filemdfivesum("\luaescapestring{#1}")%
  }%
}%
%    \end{macrocode}
%    \end{macro}
%
% \subsubsection[Timekeeping]{Timekeeping \cite[``7.17 Timekeeping'']{pdftex-manual}}
%
%    \begin{macro}{\protected}
%    \begin{macrocode}
\let\pdftexcmds@temp=Y%
\begingroup\expandafter\expandafter\expandafter\endgroup
\expandafter\ifx\csname protected\endcsname\relax
  \pdftexcmds@directlua0{%
    if tex.enableprimitives then %
      tex.enableprimitives('', {'protected'})%
    end%
  }%
\fi
\begingroup\expandafter\expandafter\expandafter\endgroup
\expandafter\ifx\csname protected\endcsname\relax
  \let\pdftexcmds@temp=N%
\fi
%    \end{macrocode}
%    \end{macro}
%    \begin{macro}{\numexpr}
%    \begin{macrocode}
\begingroup\expandafter\expandafter\expandafter\endgroup
\expandafter\ifx\csname numexpr\endcsname\relax
  \pdftexcmds@directlua0{%
    if tex.enableprimitives then %
      tex.enableprimitives('', {'numexpr'})%
    end%
  }%
\fi
\begingroup\expandafter\expandafter\expandafter\endgroup
\expandafter\ifx\csname numexpr\endcsname\relax
  \let\pdftexcmds@temp=N%
\fi
%    \end{macrocode}
%    \end{macro}
%
%    \begin{macrocode}
\ifx\pdftexcmds@temp N%
  \@PackageWarningNoLine{pdftexcmds}{%
    Definitions of \ltx@backslashchar pdf@resettimer and%
    \MessageBreak
    \ltx@backslashchar pdf@elapsedtime are skipped, because%
    \MessageBreak
    e-TeX's \ltx@backslashchar protected or %
    \ltx@backslashchar numexpr are missing%
  }%
\else
%    \end{macrocode}
%
%    \begin{macro}{\pdf@resettimer}
%    \begin{macrocode}
  \protected\def\pdf@resettimer{%
    \pdftexcmds@directlua0{%
      oberdiek.pdftexcmds.resettimer()%
    }%
  }%
%    \end{macrocode}
%    \end{macro}
%
%    \begin{macro}{\pdf@elapsedtime}
%    \begin{macrocode}
  \protected\def\pdf@elapsedtime{%
    \numexpr
      \pdftexcmds@directlua0{%
        oberdiek.pdftexcmds.elapsedtime()%
      }%
    \relax
  }%
%    \end{macrocode}
%    \end{macro}
%    \begin{macrocode}
\fi
%    \end{macrocode}
%
% \subsubsection{Shell escape}
%
%    \begin{macro}{\pdf@shellescape}
%
%    \begin{macrocode}
\ifnum\luatexversion<68 %
\else
  \protected\edef\pdf@shellescape{%
   \numexpr\directlua{tex.sprint(%
         \number\catcodetable@string,status.shell_escape)}\relax}
\fi
%    \end{macrocode}
%    \end{macro}
%
%    \begin{macro}{\pdf@system}
%    \begin{macrocode}
\def\pdf@system#1{%
  \directlua0{%
    oberdiek.pdftexcmds.system("\luaescapestring{#1}")%
  }%
}
%    \end{macrocode}
%    \end{macro}
%
%    \begin{macro}{\pdf@lastsystemstatus}
%    \begin{macrocode}
\def\pdf@lastsystemstatus{%
  \directlua0{%
    oberdiek.pdftexcmds.lastsystemstatus()%
  }%
}
%    \end{macrocode}
%    \end{macro}
%    \begin{macro}{\pdf@lastsystemexit}
%    \begin{macrocode}
\def\pdf@lastsystemexit{%
  \directlua0{%
    oberdiek.pdftexcmds.lastsystemexit()%
  }%
}
%    \end{macrocode}
%    \end{macro}
%
%    \begin{macrocode}
\catcode`\0=12 %
%    \end{macrocode}
%
%    \begin{macro}{\pdf@pipe}
%    Check availability of |io.popen| first.
%    \begin{macrocode}
\ifnum0%
    \pdftexcmds@directlua{%
      if io.popen then %
        tex.write("1")%
      end%
    }%
    =1 %
  \def\pdf@pipe#1{%
&   \romannumeral\expandafter\pdftexcmds@PatchDecode
    \the\expandafter\pdftexcmds@toks
    \pdftexcmds@directlua{%
      oberdiek.pdftexcmds.toks="pdftexcmds@toks"%
      oberdiek.pdftexcmds.pipe("\luaescapestring{#1}", \pdftexcmds@Patch)%
    }%
&   \@nil
  }%
\fi
%    \end{macrocode}
%    \end{macro}
%
%    \begin{macrocode}
\pdftexcmds@AtEnd%
%</package>
%    \end{macrocode}
%
% \subsection{Lua module}
%
%    \begin{macrocode}
%<*lua>
%    \end{macrocode}
%
%    \begin{macrocode}
oberdiek = oberdiek or {}
local pdftexcmds = oberdiek.pdftexcmds or {}
oberdiek.pdftexcmds = pdftexcmds
local systemexitstatus
function pdftexcmds.getversion()
  tex.write("2019/07/25 v0.30")
end
%    \end{macrocode}
%
% \subsubsection[Strings]{Strings \cite[``7.15 Strings'']{pdftex-manual}}
%
%    \begin{macrocode}
function pdftexcmds.strcmp(A, B)
  if A == B then
    tex.write("0")
  elseif A < B then
    tex.write("-1")
  else
    tex.write("1")
  end
end
local function utf8_to_byte(str)
  local i = 0
  local n = string.len(str)
  local t = {}
  while i < n do
    i = i + 1
    local a = string.byte(str, i)
    if a < 128 then
      table.insert(t, string.char(a))
    else
      if a >= 192 and i < n then
        i = i + 1
        local b = string.byte(str, i)
        if b < 128 or b >= 192 then
          i = i - 1
        elseif a == 194 then
          table.insert(t, string.char(b))
        elseif a == 195 then
          table.insert(t, string.char(b + 64))
        end
      end
    end
  end
  return table.concat(t)
end
function pdftexcmds.escapehex(str, mode)
  if mode == "byte" then
    str = utf8_to_byte(str)
  end
  tex.write((string.gsub(str, ".",
    function (ch)
      return string.format("%02X", string.byte(ch))
    end
  )))
end
%    \end{macrocode}
%    See procedure |unescapehex| in file \xfile{utils.c} of \hologo{pdfTeX}.
%    Caution: |tex.write| ignores leading spaces.
%    \begin{macrocode}
function pdftexcmds.unescapehex(str, mode, patch)
  local a = 0
  local first = true
  local result = {}
  for i = 1, string.len(str), 1 do
    local ch = string.byte(str, i)
    if ch >= 48 and ch <= 57 then
      ch = ch - 48
    elseif ch >= 65 and ch <= 70 then
      ch = ch - 55
    elseif ch >= 97 and ch <= 102 then
      ch = ch - 87
    else
      ch = nil
    end
    if ch then
      if first then
        a = ch * 16
        first = false
      else
        table.insert(result, a + ch)
        first = true
      end
    end
  end
  if not first then
    table.insert(result, a)
  end
  if patch == 1 then
    local temp = {}
    for i, a in ipairs(result) do
      if a == 0 then
        table.insert(temp, 1)
        table.insert(temp, 1)
      else
        if a == 1 then
          table.insert(temp, 1)
          table.insert(temp, 2)
        else
          table.insert(temp, a)
        end
      end
    end
    result = temp
  end
  if mode == "byte" then
    local utf8 = {}
    for i, a in ipairs(result) do
      if a < 128 then
        table.insert(utf8, a)
      else
        if a < 192 then
          table.insert(utf8, 194)
          a = a - 128
        else
          table.insert(utf8, 195)
          a = a - 192
        end
        table.insert(utf8, a + 128)
      end
    end
    result = utf8
  end
%    \end{macrocode}
%    this next line added for current luatex; this is the only
%    change in the file.  eroux, 28apr13. (v 0.21)
%    \begin{macrocode}
  local unpack = _G["unpack"] or table.unpack
  tex.settoks(pdftexcmds.toks, string.char(unpack(result)))
end
%    \end{macrocode}
%    See procedure |escapestring| in file \xfile{utils.c} of \hologo{pdfTeX}.
%    \begin{macrocode}
function pdftexcmds.escapestring(str, mode)
  if mode == "byte" then
    str = utf8_to_byte(str)
  end
  tex.write((string.gsub(str, ".",
    function (ch)
      local b = string.byte(ch)
      if b < 33 or b > 126 then
        return string.format("\\%.3o", b)
      end
      if b == 40 or b == 41 or b == 92 then
        return "\\" .. ch
      end
%    \end{macrocode}
%    Lua 5.1 returns the match in case of return value |nil|.
%    \begin{macrocode}
      return nil
    end
  )))
end
%    \end{macrocode}
%    See procedure |escapename| in file \xfile{utils.c} of \hologo{pdfTeX}.
%    \begin{macrocode}
function pdftexcmds.escapename(str, mode)
  if mode == "byte" then
    str = utf8_to_byte(str)
  end
  tex.write((string.gsub(str, ".",
    function (ch)
      local b = string.byte(ch)
      if b == 0 then
%    \end{macrocode}
%    In Lua 5.0 |nil| could be used for the empty string,
%    But |nil| returns the match in Lua 5.1, thus we use
%    the empty string explicitly.
%    \begin{macrocode}
        return ""
      end
      if b <= 32 or b >= 127
          or b == 35 or b == 37 or b == 40 or b == 41
          or b == 47 or b == 60 or b == 62 or b == 91
          or b == 93 or b == 123 or b == 125 then
        return string.format("#%.2X", b)
      else
%    \end{macrocode}
%    Lua 5.1 returns the match in case of return value |nil|.
%    \begin{macrocode}
        return nil
      end
    end
  )))
end
%    \end{macrocode}
%
% \subsubsection[Files]{Files \cite[``7.18 Files'']{pdftex-manual}}
%
%    \begin{macrocode}
function pdftexcmds.filesize(filename)
  local foundfile = kpse.find_file(filename, "tex", true)
  if foundfile then
    local size = lfs.attributes(foundfile, "size")
    if size then
      tex.write(size)
    end
  end
end
%    \end{macrocode}
%    See procedure |makepdftime| in file \xfile{utils.c} of \hologo{pdfTeX}.
%    \begin{macrocode}
function pdftexcmds.filemoddate(filename)
  local foundfile = kpse.find_file(filename, "tex", true)
  if foundfile then
    local date = lfs.attributes(foundfile, "modification")
    if date then
      local d = os.date("*t", date)
      if d.sec >= 60 then
        d.sec = 59
      end
      local u = os.date("!*t", date)
      local off = 60 * (d.hour - u.hour) + d.min - u.min
      if d.year ~= u.year then
        if d.year > u.year then
          off = off + 1440
        else
          off = off - 1440
        end
      elseif d.yday ~= u.yday then
        if d.yday > u.yday then
          off = off + 1440
        else
          off = off - 1440
        end
      end
      local timezone
      if off == 0 then
        timezone = "Z"
      else
        local hours = math.floor(off / 60)
        local mins = math.abs(off - hours * 60)
        timezone = string.format("%+03d'%02d'", hours, mins)
      end
      tex.write(string.format("D:%04d%02d%02d%02d%02d%02d%s",
          d.year, d.month, d.day, d.hour, d.min, d.sec, timezone))
    end
  end
end
function pdftexcmds.filedump(offset, length, filename)
  length = tonumber(length)
  if length and length > 0 then
    local foundfile = kpse.find_file(filename, "tex", true)
    if foundfile then
      offset = tonumber(offset)
      if not offset then
        offset = 0
      end
      local filehandle = io.open(foundfile, "rb")
      if filehandle then
        if offset > 0 then
          filehandle:seek("set", offset)
        end
        local dump = filehandle:read(length)
        pdftexcmds.escapehex(dump)
        filehandle:close()
      end
    end
  end
end
function pdftexcmds.mdfivesum(str, mode)
  if mode == "byte" then
    str = utf8_to_byte(str)
  end
  pdftexcmds.escapehex(md5.sum(str))
end
function pdftexcmds.filemdfivesum(filename)
  local foundfile = kpse.find_file(filename, "tex", true)
  if foundfile then
    local filehandle = io.open(foundfile, "rb")
    if filehandle then
      local contents = filehandle:read("*a")
      pdftexcmds.escapehex(md5.sum(contents))
      filehandle:close()
    end
  end
end
%    \end{macrocode}
%
% \subsubsection[Timekeeping]{Timekeeping \cite[``7.17 Timekeeping'']{pdftex-manual}}
%
%    The functions for timekeeping are based on
%    Andy Thomas' work \cite{AndyThomas:Analog}.
%    Changes:
%    \begin{itemize}
%    \item Overflow check is added.
%    \item |string.format| is used to avoid exponential number
%          representation for sure.
%    \item |tex.write| is used instead of |tex.print| to get
%          tokens with catcode 12 and without appended \cs{endlinechar}.
%    \end{itemize}
%    \begin{macrocode}
local basetime = 0
function pdftexcmds.resettimer()
  basetime = os.clock()
end
function pdftexcmds.elapsedtime()
  local val = (os.clock() - basetime) * 65536 + .5
  if val > 2147483647 then
    val = 2147483647
  end
  tex.write(string.format("%d", val))
end
%    \end{macrocode}
%
% \subsubsection[Miscellaneous]{Miscellaneous \cite[``7.21 Miscellaneous'']{pdftex-manual}}
%
%    \begin{macrocode}
function pdftexcmds.shellescape()
  if os.execute then
    if status
        and status.luatex_version
        and status.luatex_version >= 68 then
      tex.write(os.execute())
    else
      local result = os.execute()
      if result == 0 then
        tex.write("0")
      else
        if result == nil then
          tex.write("0")
        else
          tex.write("1")
        end
      end
    end
  else
    tex.write("0")
  end
end
function pdftexcmds.system(cmdline)
  systemexitstatus = nil
  texio.write_nl("log", "system(" .. cmdline .. ") ")
  if os.execute then
    texio.write("log", "executed.")
    systemexitstatus = os.execute(cmdline)
  else
    texio.write("log", "disabled.")
  end
end
function pdftexcmds.lastsystemstatus()
  local result = tonumber(systemexitstatus)
  if result then
    local x = math.floor(result / 256)
    tex.write(result - 256 * math.floor(result / 256))
  end
end
function pdftexcmds.lastsystemexit()
  local result = tonumber(systemexitstatus)
  if result then
    tex.write(math.floor(result / 256))
  end
end
function pdftexcmds.pipe(cmdline, patch)
  local result
  systemexitstatus = nil
  texio.write_nl("log", "pipe(" .. cmdline ..") ")
  if io.popen then
    texio.write("log", "executed.")
    local handle = io.popen(cmdline, "r")
    if handle then
      result = handle:read("*a")
      handle:close()
    end
  else
    texio.write("log", "disabled.")
  end
  if result then
    if patch == 1 then
      local temp = {}
      for i, a in ipairs(result) do
        if a == 0 then
          table.insert(temp, 1)
          table.insert(temp, 1)
        else
          if a == 1 then
            table.insert(temp, 1)
            table.insert(temp, 2)
          else
            table.insert(temp, a)
          end
        end
      end
      result = temp
    end
    tex.settoks(pdftexcmds.toks, result)
  else
    tex.settoks(pdftexcmds.toks, "")
  end
end
%    \end{macrocode}
%    \begin{macrocode}
%</lua>
%    \end{macrocode}
%
% \section{Test}
%
% \subsection{Catcode checks for loading}
%
%    \begin{macrocode}
%<*test1>
%    \end{macrocode}
%    \begin{macrocode}
\catcode`\{=1 %
\catcode`\}=2 %
\catcode`\#=6 %
\catcode`\@=11 %
\expandafter\ifx\csname count@\endcsname\relax
  \countdef\count@=255 %
\fi
\expandafter\ifx\csname @gobble\endcsname\relax
  \long\def\@gobble#1{}%
\fi
\expandafter\ifx\csname @firstofone\endcsname\relax
  \long\def\@firstofone#1{#1}%
\fi
\expandafter\ifx\csname loop\endcsname\relax
  \expandafter\@firstofone
\else
  \expandafter\@gobble
\fi
{%
  \def\loop#1\repeat{%
    \def\body{#1}%
    \iterate
  }%
  \def\iterate{%
    \body
      \let\next\iterate
    \else
      \let\next\relax
    \fi
    \next
  }%
  \let\repeat=\fi
}%
\def\RestoreCatcodes{}
\count@=0 %
\loop
  \edef\RestoreCatcodes{%
    \RestoreCatcodes
    \catcode\the\count@=\the\catcode\count@\relax
  }%
\ifnum\count@<255 %
  \advance\count@ 1 %
\repeat

\def\RangeCatcodeInvalid#1#2{%
  \count@=#1\relax
  \loop
    \catcode\count@=15 %
  \ifnum\count@<#2\relax
    \advance\count@ 1 %
  \repeat
}
\def\RangeCatcodeCheck#1#2#3{%
  \count@=#1\relax
  \loop
    \ifnum#3=\catcode\count@
    \else
      \errmessage{%
        Character \the\count@\space
        with wrong catcode \the\catcode\count@\space
        instead of \number#3%
      }%
    \fi
  \ifnum\count@<#2\relax
    \advance\count@ 1 %
  \repeat
}
\def\space{ }
\expandafter\ifx\csname LoadCommand\endcsname\relax
  \def\LoadCommand{\input pdftexcmds.sty\relax}%
\fi
\def\Test{%
  \RangeCatcodeInvalid{0}{47}%
  \RangeCatcodeInvalid{58}{64}%
  \RangeCatcodeInvalid{91}{96}%
  \RangeCatcodeInvalid{123}{255}%
  \catcode`\@=12 %
  \catcode`\\=0 %
  \catcode`\%=14 %
  \LoadCommand
  \RangeCatcodeCheck{0}{36}{15}%
  \RangeCatcodeCheck{37}{37}{14}%
  \RangeCatcodeCheck{38}{47}{15}%
  \RangeCatcodeCheck{48}{57}{12}%
  \RangeCatcodeCheck{58}{63}{15}%
  \RangeCatcodeCheck{64}{64}{12}%
  \RangeCatcodeCheck{65}{90}{11}%
  \RangeCatcodeCheck{91}{91}{15}%
  \RangeCatcodeCheck{92}{92}{0}%
  \RangeCatcodeCheck{93}{96}{15}%
  \RangeCatcodeCheck{97}{122}{11}%
  \RangeCatcodeCheck{123}{255}{15}%
  \RestoreCatcodes
}
\Test
\csname @@end\endcsname
\end
%    \end{macrocode}
%    \begin{macrocode}
%</test1>
%    \end{macrocode}
%
% \subsection{Test for \cs{pdf@isprimitive}}
%
%    \begin{macrocode}
%<*test2>
\catcode`\{=1 %
\catcode`\}=2 %
\catcode`\#=6 %
\catcode`\@=11 %
\input pdftexcmds.sty\relax
\def\msg#1{%
  \begingroup
    \escapechar=92 %
    \immediate\write16{#1}%
  \endgroup
}
\long\def\test#1#2#3#4{%
  \begingroup
    #4%
    \def\str{%
      Test \string\pdf@isprimitive
      {\string #1}{\string #2}{...}: %
    }%
    \pdf@isprimitive{#1}{#2}{%
      \ifx#3Y%
        \msg{\str true ==> OK.}%
      \else
        \errmessage{\str false ==> FAILED}%
      \fi
    }{%
      \ifx#3Y%
        \errmessage{\str true ==> FAILED}%
      \else
        \msg{\str false ==> OK.}%
      \fi
    }%
  \endgroup
}
\test\relax\relax Y{}
\test\foobar\relax Y{\let\foobar\relax}
\test\foobar\relax N{}
\test\hbox\hbox Y{}
\test\foobar@hbox\hbox Y{\let\foobar@hbox\hbox}
\test\if\if Y{}
\test\if\ifx N{}
\test\ifx\if N{}
\test\par\par Y{}
\test\hbox\par N{}
\test\par\hbox N{}
\csname @@end\endcsname\end
%</test2>
%    \end{macrocode}
%
% \subsection{Test for \cs{pdf@shellescape}}
%
%    \begin{macrocode}
%<*test-shell>
\catcode`\{=1 %
\catcode`\}=2 %
\catcode`\#=6 %
\catcode`\@=11 %
\input pdftexcmds.sty\relax
\def\msg#{\immediate\write16}
\def\MaybeEnd{}
\ifx\luatexversion\UnDeFiNeD
\else
  \ifnum\luatexversion<68 %
    \ifx\pdf@shellescape\@undefined
      \msg{SHELL=U}%
      \msg{OK (LuaTeX < 0.68)}%
    \else
      \msg{SHELL=defined}%
      \errmessage{Failed (LuaTeX < 0.68)}%
    \fi
    \def\MaybeEnd{\csname @@end\endcsname\end}%
  \fi
\fi
\MaybeEnd
\ifx\pdf@shellescape\@undefined
  \msg{SHELL=U}%
\else
  \msg{SHELL=\number\pdf@shellescape}%
\fi
\ifx\expected\@undefined
\else
  \ifx\expected\relax
    \msg{EXPECTED=U}%
    \ifx\pdf@shellescape\@undefined
      \msg{OK}%
    \else
      \errmessage{Failed}%
    \fi
  \else
    \msg{EXPECTED=\number\expected}%
    \ifnum\pdf@shellescape=\expected\relax
      \msg{OK}%
    \else
      \errmessage{Failed}%
    \fi
  \fi
\fi
\csname @@end\endcsname\end
%</test-shell>
%    \end{macrocode}
%
% \subsection{Test for escape functions}
%
%    \begin{macrocode}
%<*test-escape>
\catcode`\{=1 %
\catcode`\}=2 %
\catcode`\#=6 %
\catcode`\^=7 %
\catcode`\@=11 %
\errorcontextlines=1000 %
\input pdftexcmds.sty\relax
\def\msg#1{%
  \begingroup
    \escapechar=92 %
    \immediate\write16{#1}%
  \endgroup
}
%    \end{macrocode}
%    \begin{macrocode}
\begingroup
  \catcode`\@=11 %
  \countdef\count@=255 %
  \def\space{ }%
  \long\def\@whilenum#1\do #2{%
    \ifnum #1\relax
      #2\relax
      \@iwhilenum{#1\relax#2\relax}%
    \fi
  }%
  \long\def\@iwhilenum#1{%
    \ifnum #1%
      \expandafter\@iwhilenum
    \else
      \expandafter\ltx@gobble
    \fi
    {#1}%
  }%
  \gdef\AllBytes{}%
  \count@=0 %
  \catcode0=12 %
  \@whilenum\count@<256 \do{%
    \lccode0=\count@
    \ifnum\count@=32 %
      \xdef\AllBytes{\AllBytes\space}%
    \else
      \lowercase{%
        \xdef\AllBytes{\AllBytes^^@}%
      }%
    \fi
    \advance\count@ by 1 %
  }%
\endgroup
%    \end{macrocode}
%    \begin{macrocode}
\def\AllBytesHex{%
  000102030405060708090A0B0C0D0E0F%
  101112131415161718191A1B1C1D1E1F%
  202122232425262728292A2B2C2D2E2F%
  303132333435363738393A3B3C3D3E3F%
  404142434445464748494A4B4C4D4E4F%
  505152535455565758595A5B5C5D5E5F%
  606162636465666768696A6B6C6D6E6F%
  707172737475767778797A7B7C7D7E7F%
  808182838485868788898A8B8C8D8E8F%
  909192939495969798999A9B9C9D9E9F%
  A0A1A2A3A4A5A6A7A8A9AAABACADAEAF%
  B0B1B2B3B4B5B6B7B8B9BABBBCBDBEBF%
  C0C1C2C3C4C5C6C7C8C9CACBCCCDCECF%
  D0D1D2D3D4D5D6D7D8D9DADBDCDDDEDF%
  E0E1E2E3E4E5E6E7E8E9EAEBECEDEEEF%
  F0F1F2F3F4F5F6F7F8F9FAFBFCFDFEFF%
}
\ltx@onelevel@sanitize\AllBytesHex
\expandafter\lowercase\expandafter{%
  \expandafter\def\expandafter\AllBytesHexLC
      \expandafter{\AllBytesHex}%
}
\begingroup
  \catcode`\#=12 %
  \xdef\AllBytesName{%
    #01#02#03#04#05#06#07#08#09#0A#0B#0C#0D#0E#0F%
    #10#11#12#13#14#15#16#17#18#19#1A#1B#1C#1D#1E#1F%
    #20!"#23$#25&'#28#29*+,-.#2F%
    0123456789:;#3C=#3E?%
    @ABCDEFGHIJKLMNO%
    PQRSTUVWXYZ#5B\ltx@backslashchar#5D^_%
    `abcdefghijklmno%
    pqrstuvwxyz#7B|#7D\string~#7F%
    #80#81#82#83#84#85#86#87#88#89#8A#8B#8C#8D#8E#8F%
    #90#91#92#93#94#95#96#97#98#99#9A#9B#9C#9D#9E#9F%
    #A0#A1#A2#A3#A4#A5#A6#A7#A8#A9#AA#AB#AC#AD#AE#AF%
    #B0#B1#B2#B3#B4#B5#B6#B7#B8#B9#BA#BB#BC#BD#BE#BF%
    #C0#C1#C2#C3#C4#C5#C6#C7#C8#C9#CA#CB#CC#CD#CE#CF%
    #D0#D1#D2#D3#D4#D5#D6#D7#D8#D9#DA#DB#DC#DD#DE#DF%
    #E0#E1#E2#E3#E4#E5#E6#E7#E8#E9#EA#EB#EC#ED#EE#EF%
    #F0#F1#F2#F3#F4#F5#F6#F7#F8#F9#FA#FB#FC#FD#FE#FF%
  }%
\endgroup
\ltx@onelevel@sanitize\AllBytesName
\edef\AllBytesFromName{\expandafter\ltx@gobble\AllBytes}
\begingroup
  \def\|{|}%
  \edef\%{\ltx@percentchar}%
  \catcode`\|=0 %
  \catcode`\#=12 %
  \catcode`\~=12 %
  \catcode`\\=12 %
  |xdef|AllBytesString{%
    \000\001\002\003\004\005\006\007\010\011\012\013\014\015\016\017%
    \020\021\022\023\024\025\026\027\030\031\032\033\034\035\036\037%
    \040!"#$|%&'\(\)*+,-./%
    0123456789:;<=>?%
    @ABCDEFGHIJKLMNO%
    PQRSTUVWXYZ[\\]^_%
    `abcdefghijklmno%
    pqrstuvwxyz{||}~\177%
    \200\201\202\203\204\205\206\207\210\211\212\213\214\215\216\217%
    \220\221\222\223\224\225\226\227\230\231\232\233\234\235\236\237%
    \240\241\242\243\244\245\246\247\250\251\252\253\254\255\256\257%
    \260\261\262\263\264\265\266\267\270\271\272\273\274\275\276\277%
    \300\301\302\303\304\305\306\307\310\311\312\313\314\315\316\317%
    \320\321\322\323\324\325\326\327\330\331\332\333\334\335\336\337%
    \340\341\342\343\344\345\346\347\350\351\352\353\354\355\356\357%
    \360\361\362\363\364\365\366\367\370\371\372\373\374\375\376\377%
  }%
|endgroup
\ltx@onelevel@sanitize\AllBytesString
%    \end{macrocode}
%    \begin{macrocode}
\def\Test#1#2#3{%
  \begingroup
    \expandafter\expandafter\expandafter\def
    \expandafter\expandafter\expandafter\TestResult
    \expandafter\expandafter\expandafter{%
      #1{#2}%
    }%
    \ifx\TestResult#3%
    \else
      \newlinechar=10 %
      \msg{Expect:^^J#3}%
      \msg{Result:^^J\TestResult}%
      \errmessage{\string#2 -\string#1-> \string#3}%
    \fi
  \endgroup
}
\def\test#1#2#3{%
  \edef\TestFrom{#2}%
  \edef\TestExpect{#3}%
  \ltx@onelevel@sanitize\TestExpect
  \Test#1\TestFrom\TestExpect
}
\test\pdf@unescapehex{74657374}{test}
\begingroup
  \catcode0=12 %
  \catcode1=12 %
  \test\pdf@unescapehex{740074017400740174}{t^^@t^^At^^@t^^At}%
\endgroup
\Test\pdf@escapehex\AllBytes\AllBytesHex
\Test\pdf@unescapehex\AllBytesHex\AllBytes
\Test\pdf@escapename\AllBytes\AllBytesName
\Test\pdf@escapestring\AllBytes\AllBytesString
%    \end{macrocode}
%    \begin{macrocode}
\csname @@end\endcsname\end
%</test-escape>
%    \end{macrocode}
%
% \section{Installation}
%
% \subsection{Download}
%
% \paragraph{Package.} This package is available on
% CTAN\footnote{\CTANpkg{pdftexcmds}}:
% \begin{description}
% \item[\CTAN{macros/latex/contrib/oberdiek/pdftexcmds.dtx}] The source file.
% \item[\CTAN{macros/latex/contrib/oberdiek/pdftexcmds.pdf}] Documentation.
% \end{description}
%
%
% \paragraph{Bundle.} All the packages of the bundle `oberdiek'
% are also available in a TDS compliant ZIP archive. There
% the packages are already unpacked and the documentation files
% are generated. The files and directories obey the TDS standard.
% \begin{description}
% \item[\CTANinstall{install/macros/latex/contrib/oberdiek.tds.zip}]
% \end{description}
% \emph{TDS} refers to the standard ``A Directory Structure
% for \TeX\ Files'' (\CTAN{tds/tds.pdf}). Directories
% with \xfile{texmf} in their name are usually organized this way.
%
% \subsection{Bundle installation}
%
% \paragraph{Unpacking.} Unpack the \xfile{oberdiek.tds.zip} in the
% TDS tree (also known as \xfile{texmf} tree) of your choice.
% Example (linux):
% \begin{quote}
%   |unzip oberdiek.tds.zip -d ~/texmf|
% \end{quote}
%
% \paragraph{Script installation.}
% Check the directory \xfile{TDS:scripts/oberdiek/} for
% scripts that need further installation steps.
% Package \xpackage{attachfile2} comes with the Perl script
% \xfile{pdfatfi.pl} that should be installed in such a way
% that it can be called as \texttt{pdfatfi}.
% Example (linux):
% \begin{quote}
%   |chmod +x scripts/oberdiek/pdfatfi.pl|\\
%   |cp scripts/oberdiek/pdfatfi.pl /usr/local/bin/|
% \end{quote}
%
% \subsection{Package installation}
%
% \paragraph{Unpacking.} The \xfile{.dtx} file is a self-extracting
% \docstrip\ archive. The files are extracted by running the
% \xfile{.dtx} through \plainTeX:
% \begin{quote}
%   \verb|tex pdftexcmds.dtx|
% \end{quote}
%
% \paragraph{TDS.} Now the different files must be moved into
% the different directories in your installation TDS tree
% (also known as \xfile{texmf} tree):
% \begin{quote}
% \def\t{^^A
% \begin{tabular}{@{}>{\ttfamily}l@{ $\rightarrow$ }>{\ttfamily}l@{}}
%   pdftexcmds.sty & tex/generic/oberdiek/pdftexcmds.sty\\
%   oberdiek.pdftexcmds.lua & scripts/oberdiek/oberdiek.pdftexcmds.lua\\
%   pdftexcmds.lua & scripts/oberdiek/pdftexcmds.lua\\
%   pdftexcmds.pdf & doc/latex/oberdiek/pdftexcmds.pdf\\
%   test/pdftexcmds-test1.tex & doc/latex/oberdiek/test/pdftexcmds-test1.tex\\
%   test/pdftexcmds-test2.tex & doc/latex/oberdiek/test/pdftexcmds-test2.tex\\
%   test/pdftexcmds-test-shell.tex & doc/latex/oberdiek/test/pdftexcmds-test-shell.tex\\
%   test/pdftexcmds-test-escape.tex & doc/latex/oberdiek/test/pdftexcmds-test-escape.tex\\
%   pdftexcmds.dtx & source/latex/oberdiek/pdftexcmds.dtx\\
% \end{tabular}^^A
% }^^A
% \sbox0{\t}^^A
% \ifdim\wd0>\linewidth
%   \begingroup
%     \advance\linewidth by\leftmargin
%     \advance\linewidth by\rightmargin
%   \edef\x{\endgroup
%     \def\noexpand\lw{\the\linewidth}^^A
%   }\x
%   \def\lwbox{^^A
%     \leavevmode
%     \hbox to \linewidth{^^A
%       \kern-\leftmargin\relax
%       \hss
%       \usebox0
%       \hss
%       \kern-\rightmargin\relax
%     }^^A
%   }^^A
%   \ifdim\wd0>\lw
%     \sbox0{\small\t}^^A
%     \ifdim\wd0>\linewidth
%       \ifdim\wd0>\lw
%         \sbox0{\footnotesize\t}^^A
%         \ifdim\wd0>\linewidth
%           \ifdim\wd0>\lw
%             \sbox0{\scriptsize\t}^^A
%             \ifdim\wd0>\linewidth
%               \ifdim\wd0>\lw
%                 \sbox0{\tiny\t}^^A
%                 \ifdim\wd0>\linewidth
%                   \lwbox
%                 \else
%                   \usebox0
%                 \fi
%               \else
%                 \lwbox
%               \fi
%             \else
%               \usebox0
%             \fi
%           \else
%             \lwbox
%           \fi
%         \else
%           \usebox0
%         \fi
%       \else
%         \lwbox
%       \fi
%     \else
%       \usebox0
%     \fi
%   \else
%     \lwbox
%   \fi
% \else
%   \usebox0
% \fi
% \end{quote}
% If you have a \xfile{docstrip.cfg} that configures and enables \docstrip's
% TDS installing feature, then some files can already be in the right
% place, see the documentation of \docstrip.
%
% \subsection{Refresh file name databases}
%
% If your \TeX~distribution
% (\teTeX, \mikTeX, \dots) relies on file name databases, you must refresh
% these. For example, \teTeX\ users run \verb|texhash| or
% \verb|mktexlsr|.
%
% \subsection{Some details for the interested}
%
% \paragraph{Unpacking with \LaTeX.}
% The \xfile{.dtx} chooses its action depending on the format:
% \begin{description}
% \item[\plainTeX:] Run \docstrip\ and extract the files.
% \item[\LaTeX:] Generate the documentation.
% \end{description}
% If you insist on using \LaTeX\ for \docstrip\ (really,
% \docstrip\ does not need \LaTeX), then inform the autodetect routine
% about your intention:
% \begin{quote}
%   \verb|latex \let\install=y\input{pdftexcmds.dtx}|
% \end{quote}
% Do not forget to quote the argument according to the demands
% of your shell.
%
% \paragraph{Generating the documentation.}
% You can use both the \xfile{.dtx} or the \xfile{.drv} to generate
% the documentation. The process can be configured by the
% configuration file \xfile{ltxdoc.cfg}. For instance, put this
% line into this file, if you want to have A4 as paper format:
% \begin{quote}
%   \verb|\PassOptionsToClass{a4paper}{article}|
% \end{quote}
% An example follows how to generate the
% documentation with pdf\LaTeX:
% \begin{quote}
%\begin{verbatim}
%pdflatex pdftexcmds.dtx
%bibtex pdftexcmds.aux
%makeindex -s gind.ist pdftexcmds.idx
%pdflatex pdftexcmds.dtx
%makeindex -s gind.ist pdftexcmds.idx
%pdflatex pdftexcmds.dtx
%\end{verbatim}
% \end{quote}
%
% \printbibliography[
%   heading=bibnumbered,
% ]
%
% \begin{History}
%   \begin{Version}{2007/11/11 v0.1}
%   \item
%     First version.
%   \end{Version}
%   \begin{Version}{2007/11/12 v0.2}
%   \item
%     Short description fixed.
%   \end{Version}
%   \begin{Version}{2007/12/12 v0.3}
%   \item
%     Organization of Lua code as module.
%   \end{Version}
%   \begin{Version}{2009/04/10 v0.4}
%   \item
%     Adaptation for syntax change of \cs{directlua} in
%     \hologo{LuaTeX} 0.36.
%   \end{Version}
%   \begin{Version}{2009/09/22 v0.5}
%   \item
%     \cs{pdf@primitive}, \cs{pdf@ifprimitive} added.
%   \item
%     \hologo{XeTeX}'s variants are detected for
%     \cs{pdf@shellescape}, \cs{pdf@strcmp}, \cs{pdf@primitive},
%     \cs{pdf@ifprimitive}.
%   \end{Version}
%   \begin{Version}{2009/09/23 v0.6}
%   \item
%     Macro \cs{pdf@isprimitive} added.
%   \end{Version}
%   \begin{Version}{2009/12/12 v0.7}
%   \item
%     Short info shortened.
%   \end{Version}
%   \begin{Version}{2010/03/01 v0.8}
%   \item
%     Required date for package \xpackage{ifluatex} updated.
%   \end{Version}
%   \begin{Version}{2010/04/01 v0.9}
%   \item
%     Use \cs{ifeof18} for defining \cs{pdf@shellescape} between
%     \hologo{pdfTeX} 1.21a (inclusive) and 1.30.0 (exclusive).
%   \end{Version}
%   \begin{Version}{2010/11/04 v0.10}
%   \item
%     \cs{pdf@draftmode}, \cs{pdf@ifdraftmode} and
%     \cs{pdf@setdraftmode} added.
%   \end{Version}
%   \begin{Version}{2010/11/11 v0.11}
%   \item
%     Missing \cs{RequirePackage} for package \xpackage{ifpdf} added.
%   \end{Version}
%   \begin{Version}{2011/01/30 v0.12}
%   \item
%     Already loaded package files are not input in \hologo{plainTeX}.
%   \end{Version}
%   \begin{Version}{2011/03/04 v0.13}
%   \item
%     Improved Lua function \texttt{shellescape} that also
%     uses the result of \texttt{os.execute()} (thanks to Philipp Stephani).
%   \end{Version}
%   \begin{Version}{2011/04/10 v0.14}
%   \item
%     Version check of loaded module added.
%   \item
%     Patch for bug in \hologo{LuaTeX} between 0.40.6 and 0.65 that
%     is fixed in revision 4096.
%   \end{Version}
%   \begin{Version}{2011/04/16 v0.15}
%   \item
%     \hologo{LuaTeX}: \cs{pdf@shellescape} is only supported
%     for version 0.70.0 and higher due to a bug, \texttt{os.execute()}
%     crashes in some circumstances. Fixed in \hologo{LuaTeX}
%     beta-0.70.0, revision 4167.
%   \end{Version}
%   \begin{Version}{2011/04/22 v0.16}
%   \item
%     Previous fix was not working due to a wrong catcode of digit
%     zero (due to easily support the old \cs{directlua0}).
%     The version border is lowered to 0.68, because some
%     beta-0.67.0 seems also to work.
%   \end{Version}
%   \begin{Version}{2011/06/29 v0.17}
%   \item
%     Documentation addition to \cs{pdf@shellescape}.
%   \end{Version}
%   \begin{Version}{2011/07/01 v0.18}
%   \item
%     Add Lua module loading in \cs{everyjob} for \hologo{iniTeX}
%     (\hologo{LuaTeX} only).
%   \end{Version}
%   \begin{Version}{2011/07/28 v0.19}
%   \item
%     Missing space in an info message added (Martin M\"unch).
%   \end{Version}
%   \begin{Version}{2011/11/29 v0.20}
%   \item
%     \cs{pdf@resettimer} and \cs{pdf@elapsedtime} added
%     (thanks Andy Thomas).
%   \end{Version}
%   \begin{Version}{2016/05/10 v0.21}
%   \item
%      local unpack added
%     (thanks \'{E}lie Roux).
%   \end{Version}
%   \begin{Version}{2016/05/21 v0.22}
%   \item
%     adjust \cs{textbackslas}h usage in bib file for biber bug.
%   \end{Version}
%   \begin{Version}{2016/10/02 v0.23}
%   \item
%     add file.close to lua filehandles (github pull request).
%   \end{Version}
%   \begin{Version}{2017/01/29 v0.24}
%   \item
%     Avoid loading luatex-loader for current luatex. (Use
%     pdftexcmds.lua not oberdiek.pdftexcmds.lua to simplify file
%     search with standard require)
%   \end{Version}
%   \begin{Version}{2017/03/19 v0.25}
%   \item
%     New \cs{pdf@shellescape} for Lua\TeX, see github issue 20.
%   \end{Version}
%   \begin{Version}{2018/01/21 v0.26}
%   \item
%     use rb not r mode for file open github issue 34.
%   \end{Version}
%   \begin{Version}{2018/01/30 v0.27}
%   \item
%     \cs{pdf@mdfivesum} for \hologo{XeTeX}
%   \end{Version}
%   \begin{Version}{2018/09/07 v0.28}
%   \item
%     Fix catcode regime in luatex sprint for \cs{pdf@shellescape} GH issue 45
%   \end{Version}
%   \begin{Version}{2018/09/10 v0.29}
%   \item
%     Actually do the fix described above in the code, not just document it.
%   \end{Version}
%   \begin{Version}{2019/07/25 v0.30}
%   \item
%     remove uses of module function, see PR70
%   \end{Version}
% \end{History}
%
% \PrintIndex
%
% \Finale
\endinput

%        (quote the arguments according to the demands of your shell)
%
% Documentation:
%    (a) If pdftexcmds.drv is present:
%           latex pdftexcmds.drv
%    (b) Without pdftexcmds.drv:
%           latex pdftexcmds.dtx; ...
%    The class ltxdoc loads the configuration file ltxdoc.cfg
%    if available. Here you can specify further options, e.g.
%    use A4 as paper format:
%       \PassOptionsToClass{a4paper}{article}
%
%    Programm calls to get the documentation (example):
%       pdflatex pdftexcmds.dtx
%       bibtex pdftexcmds.aux
%       makeindex -s gind.ist pdftexcmds.idx
%       pdflatex pdftexcmds.dtx
%       makeindex -s gind.ist pdftexcmds.idx
%       pdflatex pdftexcmds.dtx
%
% Installation:
%    TDS:tex/generic/oberdiek/pdftexcmds.sty
%    TDS:scripts/oberdiek/oberdiek.pdftexcmds.lua
%    TDS:scripts/oberdiek/pdftexcmds.lua
%    TDS:doc/latex/oberdiek/pdftexcmds.pdf
%    TDS:doc/latex/oberdiek/test/pdftexcmds-test1.tex
%    TDS:doc/latex/oberdiek/test/pdftexcmds-test2.tex
%    TDS:doc/latex/oberdiek/test/pdftexcmds-test-shell.tex
%    TDS:doc/latex/oberdiek/test/pdftexcmds-test-escape.tex
%    TDS:source/latex/oberdiek/pdftexcmds.dtx
%
%<*ignore>
\begingroup
  \catcode123=1 %
  \catcode125=2 %
  \def\x{LaTeX2e}%
\expandafter\endgroup
\ifcase 0\ifx\install y1\fi\expandafter
         \ifx\csname processbatchFile\endcsname\relax\else1\fi
         \ifx\fmtname\x\else 1\fi\relax
\else\csname fi\endcsname
%</ignore>
%<*install>
\input docstrip.tex
\Msg{************************************************************************}
\Msg{* Installation}
\Msg{* Package: pdftexcmds 2019/07/25 v0.30 Utility functions of pdfTeX for LuaTeX (HO)}
\Msg{************************************************************************}

\keepsilent
\askforoverwritefalse

\let\MetaPrefix\relax
\preamble

This is a generated file.

Project: pdftexcmds
Version: 2019/07/25 v0.30

Copyright (C) 2007, 2009-2011 by
   Heiko Oberdiek <heiko.oberdiek at googlemail.com>

This work may be distributed and/or modified under the
conditions of the LaTeX Project Public License, either
version 1.3c of this license or (at your option) any later
version. This version of this license is in
   https://www.latex-project.org/lppl/lppl-1-3c.txt
and the latest version of this license is in
   https://www.latex-project.org/lppl.txt
and version 1.3 or later is part of all distributions of
LaTeX version 2005/12/01 or later.

This work has the LPPL maintenance status "maintained".

The Current Maintainers of this work are
Heiko Oberdiek and the Oberdiek Package Support Group
https://github.com/ho-tex/oberdiek/issues


The Base Interpreter refers to any `TeX-Format',
because some files are installed in TDS:tex/generic//.

This work consists of the main source file pdftexcmds.dtx
and the derived files
   pdftexcmds.sty, pdftexcmds.pdf, pdftexcmds.ins, pdftexcmds.drv,
   pdftexcmds.bib, pdftexcmds-test1.tex, pdftexcmds-test2.tex,
   pdftexcmds-test-shell.tex, pdftexcmds-test-escape.tex,
   oberdiek.pdftexcmds.lua, pdftexcmds.lua.

\endpreamble
\let\MetaPrefix\DoubleperCent

\generate{%
  \file{pdftexcmds.ins}{\from{pdftexcmds.dtx}{install}}%
  \file{pdftexcmds.drv}{\from{pdftexcmds.dtx}{driver}}%
  \nopreamble
  \nopostamble
  \file{pdftexcmds.bib}{\from{pdftexcmds.dtx}{bib}}%
  \usepreamble\defaultpreamble
  \usepostamble\defaultpostamble
  \usedir{tex/generic/oberdiek}%
  \file{pdftexcmds.sty}{\from{pdftexcmds.dtx}{package}}%
%  \usedir{doc/latex/oberdiek/test}%
%  \file{pdftexcmds-test1.tex}{\from{pdftexcmds.dtx}{test1}}%
%  \file{pdftexcmds-test2.tex}{\from{pdftexcmds.dtx}{test2}}%
%  \file{pdftexcmds-test-shell.tex}{\from{pdftexcmds.dtx}{test-shell}}%
%  \file{pdftexcmds-test-escape.tex}{\from{pdftexcmds.dtx}{test-escape}}%
  \nopreamble
  \nopostamble
%  \usedir{source/latex/oberdiek/catalogue}%
%  \file{pdftexcmds.xml}{\from{pdftexcmds.dtx}{catalogue}}%
}
\def\MetaPrefix{-- }
\def\defaultpostamble{%
  \MetaPrefix^^J%
  \MetaPrefix\space End of File `\outFileName'.%
}
\def\currentpostamble{\defaultpostamble}%
\generate{%
  \usedir{scripts/oberdiek}%
  \file{oberdiek.pdftexcmds.lua}{\from{pdftexcmds.dtx}{lua}}%
  \file{pdftexcmds.lua}{\from{pdftexcmds.dtx}{lua}}%
}

\catcode32=13\relax% active space
\let =\space%
\Msg{************************************************************************}
\Msg{*}
\Msg{* To finish the installation you have to move the following}
\Msg{* file into a directory searched by TeX:}
\Msg{*}
\Msg{*     pdftexcmds.sty}
\Msg{*}
\Msg{* And install the following script files:}
\Msg{*}
\Msg{*     oberdiek.pdftexcmds.lua, pdftexcmds.lua}
\Msg{*}
\Msg{* To produce the documentation run the file `pdftexcmds.drv'}
\Msg{* through LaTeX.}
\Msg{*}
\Msg{* Happy TeXing!}
\Msg{*}
\Msg{************************************************************************}

\endbatchfile
%</install>
%<*bib>
@online{AndyThomas:Analog,
  author={Thomas, Andy},
  title={Analog of {\texttt{\csname textbackslash\endcsname}pdfelapsedtime} for
      {\hologo{LuaTeX}} and {\hologo{XeTeX}}},
  url={http://tex.stackexchange.com/a/32531},
  urldate={2011-11-29},
}
%</bib>
%<*ignore>
\fi
%</ignore>
%<*driver>
\NeedsTeXFormat{LaTeX2e}
\ProvidesFile{pdftexcmds.drv}%
  [2019/07/25 v0.30 Utility functions of pdfTeX for LuaTeX (HO)]%
\documentclass{ltxdoc}
\usepackage{holtxdoc}[2011/11/22]
\usepackage{paralist}
\usepackage{csquotes}
\usepackage[
  backend=bibtex,
  bibencoding=ascii,
  alldates=iso8601,
]{biblatex}[2011/11/13]
\bibliography{oberdiek-source}
\bibliography{pdftexcmds}
\begin{document}
  \DocInput{pdftexcmds.dtx}%
\end{document}
%</driver>
% \fi
%
%
% \CharacterTable
%  {Upper-case    \A\B\C\D\E\F\G\H\I\J\K\L\M\N\O\P\Q\R\S\T\U\V\W\X\Y\Z
%   Lower-case    \a\b\c\d\e\f\g\h\i\j\k\l\m\n\o\p\q\r\s\t\u\v\w\x\y\z
%   Digits        \0\1\2\3\4\5\6\7\8\9
%   Exclamation   \!     Double quote  \"     Hash (number) \#
%   Dollar        \$     Percent       \%     Ampersand     \&
%   Acute accent  \'     Left paren    \(     Right paren   \)
%   Asterisk      \*     Plus          \+     Comma         \,
%   Minus         \-     Point         \.     Solidus       \/
%   Colon         \:     Semicolon     \;     Less than     \<
%   Equals        \=     Greater than  \>     Question mark \?
%   Commercial at \@     Left bracket  \[     Backslash     \\
%   Right bracket \]     Circumflex    \^     Underscore    \_
%   Grave accent  \`     Left brace    \{     Vertical bar  \|
%   Right brace   \}     Tilde         \~}
%
% \GetFileInfo{pdftexcmds.drv}
%
% \title{The \xpackage{pdftexcmds} package}
% \date{2019/07/25 v0.30}
% \author{Heiko Oberdiek\thanks
% {Please report any issues at \url{https://github.com/ho-tex/oberdiek/issues}}}
%
% \maketitle
%
% \begin{abstract}
% \hologo{LuaTeX} provides most of the commands of \hologo{pdfTeX} 1.40. However
% a number of utility functions are removed. This package tries to fill
% the gap and implements some of the missing primitive using Lua.
% \end{abstract}
%
% \tableofcontents
%
% \def\csi#1{\texttt{\textbackslash\textit{#1}}}
%
% \section{Documentation}
%
% Some primitives of \hologo{pdfTeX} \cite{pdftex-manual}
% are not defined by \hologo{LuaTeX} \cite{luatex-manual}.
% This package implements macro based solutions using Lua code
% for the following missing \hologo{pdfTeX} primitives;
% \begin{compactitem}
% \item \cs{pdfstrcmp}
% \item \cs{pdfunescapehex}
% \item \cs{pdfescapehex}
% \item \cs{pdfescapename}
% \item \cs{pdfescapestring}
% \item \cs{pdffilesize}
% \item \cs{pdffilemoddate}
% \item \cs{pdffiledump}
% \item \cs{pdfmdfivesum}
% \item \cs{pdfresettimer}
% \item \cs{pdfelapsedtime}
% \item |\immediate\write18|
% \end{compactitem}
% The original names of the primitives cannot be used:
% \begin{itemize}
% \item
% The syntax for their arguments cannot easily
% simulated by macros. The primitives using key words
% such as |file| (\cs{pdfmdfivesum}) or |offset| and |length|
% (\cs{pdffiledump}) and uses \meta{general text} for the other
% arguments. Using token registers assignments, \meta{general text} could
% be catched. However, the simulated primitives are expandable
% and register assignments would destroy this important property.
% (\meta{general text} allows something like |\expandafter\bgroup ...}|.)
% \item
% The original primitives can be expanded using one expansion step.
% The new macros need two expansion steps because of the additional
% macro expansion. Example:
% \begin{quote}
%   |\expandafter\foo\pdffilemoddate{file}|\\
%   vs.\\
%   |\expandafter\expandafter\expandafter|\\
%   |\foo\pdf@filemoddate{file}|
% \end{quote}
% \end{itemize}
%
% \hologo{LuaTeX} isn't stable yet and thus the status of this package is
% \emph{experimental}. Feedback is welcome.
%
% \subsection{General principles}
%
% \begin{description}
% \item[Naming convention:]
%   Usually this package defines a macro |\pdf@|\meta{cmd} if
%   \hologo{pdfTeX} provides |\pdf|\meta{cmd}.
% \item[Arguments:] The order of arguments in |\pdf@|\meta{cmd}
%   is the same as for the corresponding primitive of \hologo{pdfTeX}.
%   The arguments are ordinary undelimited \hologo{TeX} arguments,
%   no \meta{general text} and without additional keywords.
% \item[Expandibility:]
%   The macro |\pdf@|\meta{cmd} is expandable if the
%   corresponding \hologo{pdfTeX} primitive has this property.
%   Exact two expansion steps are necessary (first is the macro
%   expansion) except for \cs{pdf@primitive} and \cs{pdf@ifprimitive}.
%   The latter ones are not macros, but have the direct meaning of the
%   primitive.
% \item[Without \hologo{LuaTeX}:]
%   The macros |\pdf@|\meta{cmd} are mapped to the commands
%   of \hologo{pdfTeX} if they are available. Otherwise they are undefined.
% \item[Availability:]
%   The macros that the packages provides are undefined, if
%   the necessary primitives are not found and cannot be
%   implemented by Lua.
% \end{description}
%
% \subsection{Macros}
%
% \subsubsection[Strings]{Strings \cite[``7.15 Strings'']{pdftex-manual}}
%
% \begin{declcs}{pdf@strcmp} \M{stringA} \M{stringB}
% \end{declcs}
% Same as |\pdfstrcmp{|\meta{stringA}|}{|\meta{stringB}|}|.
%
% \begin{declcs}{pdf@unescapehex} \M{string}
% \end{declcs}
% Same as |\pdfunescapehex{|\meta{string}|}|.
% The argument is a byte string given in hexadecimal notation.
% The result are character tokens from 0 until 255 with
% catcode 12 and the space with catcode 10.
%
% \begin{declcs}{pdf@escapehex} \M{string}\\
%   \cs{pdf@escapestring} \M{string}\\
%   \cs{pdf@escapename} \M{string}
% \end{declcs}
% Same as the primitives of \hologo{pdfTeX}. However \hologo{pdfTeX} does not
% know about characters with codes 256 and larger. Thus the
% string is treated as byte string, characters with more than
% eight bits are ignored.
%
% \subsubsection[Files]{Files \cite[``7.18 Files'']{pdftex-manual}}
%
% \begin{declcs}{pdf@filesize} \M{filename}
% \end{declcs}
% Same as |\pdffilesize{|\meta{filename}|}|.
%
% \begin{declcs}{pdf@filemoddate} \M{filename}
% \end{declcs}
% Same as |\pdffilemoddate{|\meta{filename}|}|.
%
% \begin{declcs}{pdf@filedump} \M{offset} \M{length} \M{filename}
% \end{declcs}
% Same as |\pdffiledump offset| \meta{offset} |length| \meta{length}
% |{|\meta{filename}|}|. Both \meta{offset} and \meta{length} must
% not be empty, but must be a valid \hologo{TeX} number.
%
% \begin{declcs}{pdf@mdfivesum} \M{string}
% \end{declcs}
% Same as |\pdfmdfivesum{|\meta{string}|}|. Keyword |file| is supported
% by macro \cs{pdf@filemdfivesum}.
%
% \begin{declcs}{pdf@filemdfivesum} \M{filename}
% \end{declcs}
% Same as |\pdfmdfivesum file{|\meta{filename}|}|.
%
% \subsubsection[Timekeeping]{Timekeeping \cite[``7.17 Timekeeping'']{pdftex-manual}}
%
% The timekeeping macros are based on Andy Thomas' work \cite{AndyThomas:Analog}.
%
% \begin{declcs}{pdf@resettimer}
% \end{declcs}
% Same as \cs{pdfresettimer}, it resets the internal timer.
%
% \begin{declcs}{pdf@elapsedtime}
% \end{declcs}
% Same as \cs{pdfelapsedtime}. It behaves like a read-only integer.
% For printing purposes it can be prefixed by \cs{the} or \cs{number}.
% It measures the time in scaled seconds (seconds multiplied with 65536)
% since the latest call of \cs{pdf@resettimer} or start of
% program/package. The resolution, the shortest time interval that
% can be measured, depends on the program and system.
% \begin{itemize}
% \item \hologo{pdfTeX} with |gettimeofday|: $\ge$ 1/65536\,s
% \item \hologo{pdfTeX} with |ftime|: $\ge$ 1\,ms
% \item \hologo{pdfTeX} with |time|: $\ge$ 1\,s
% \item \hologo{LuaTeX}: $\ge$ 10\,ms\\
%  (|os.clock()| returns a float number with two decimal digits in
%  \hologo{LuaTeX} beta-0.70.1-2011061416 (rev 4277)).
% \end{itemize}
%
% \subsubsection[Miscellaneous]{Miscellaneous \cite[``7.21 Miscellaneous'']{pdftex-manual}}
%
% \begin{declcs}{pdf@draftmode}
% \end{declcs}
% If the \TeX\ compiler knows \cs{pdfdraftmode} or \cs{draftmode}
% (\hologo{pdfTeX},
% \hologo{LuaTeX}), then \cs{pdf@draftmode} returns, whether
% this mode is enabled. The result is an implicit number:
% one means the draft mode is available and enabled.
% If the value is zero, then the mode is not active or
% \cs{pdfdraftmode} is not available.
% An explicit number is yielded by \cs{number}\cs{pdf@draftmode}.
% The macro cannot
% be used to change the mode, see \cs{pdf@setdraftmode}.
%
% \begin{declcs}{pdf@ifdraftmode} \M{true} \M{false}
% \end{declcs}
% If \cs{pdfdraftmode} is available and enabled, \meta{true} is
% called, otherwise \meta{false} is executed.
%
% \begin{declcs}{pdf@setdraftmode} \M{value}
% \end{declcs}
% Macro \cs{pdf@setdraftmode} expects the number zero or one as
% \meta{value}. Zero deactivates the mode and one enables the draft mode.
% The macro does not have an effect, if the feature \cs{pdfdraftmode} is not
% available.
%
% \begin{declcs}{pdf@shellescape}
% \end{declcs}
% Same as |\pdfshellescape|. It is or expands to |1| if external
% commands can be executed and |0| otherwise. In \hologo{pdfTeX} external
% commands must be enabled first by command line option or
% configuration option. In \hologo{LuaTeX} option |--safer| disables
% the execution of external commands.
%
% In \hologo{LuaTeX} before 0.68.0 \cs{pdf@shellescape} is not
% available due to a bug in |os.execute()|. The argumentless form
% crashes in some circumstances with segmentation fault.
% (It is fixed in version 0.68.0 or revision 4167 of \hologo{LuaTeX}.
% and packported to some version of 0.67.0).
%
% Hints for usage:
% \begin{itemize}
% \item Before its use \cs{pdf@shellescape} should be tested,
% whether it is available. Example with package \xpackage{ltxcmds}
% (loaded by package \xpackage{pdftexcmds}):
%\begin{quote}
%\begin{verbatim}
%\ltx@IfUndefined{pdf@shellescape}{%
%  % \pdf@shellescape is undefined
%}{%
%  % \pdf@shellescape is available
%}
%\end{verbatim}
%\end{quote}
% Use \cs{ltx@ifundefined} in expandable contexts.
% \item \cs{pdf@shellescape} might be a numerical constant,
% expands to the primitive, or expands to a plain number.
% Therefore use it in contexts where these differences does not matter.
% \item Use in comparisons, e.g.:
%   \begin{quote}
%     |\ifnum\pdf@shellescape=0 ...|
%   \end{quote}
% \item Print the number: |\number\pdf@shellescape|
% \end{itemize}
%
% \begin{declcs}{pdf@system} \M{cmdline}
% \end{declcs}
% It is a wrapper for |\immediate\write18| in \hologo{pdfTeX} or
% |os.execute| in \hologo{LuaTeX}.
%
% In theory |os.execute|
% returns a status number. But its meaning is quite
% undefined. Are there some reliable properties?
% Does it make sense to provide an user interface to
% this status exit code?
%
% \begin{declcs}{pdf@primitive} \csi{cmd}
% \end{declcs}
% Same as \cs{pdfprimitive} in \hologo{pdfTeX} or \hologo{LuaTeX}.
% In \hologo{XeTeX} the
% primitive is called \cs{primitive}. Despite the current definition
% of the command \csi{cmd}, it's meaning as primitive is used.
%
% \begin{declcs}{pdf@ifprimitive} \csi{cmd}
% \end{declcs}
% Same as \cs{ifpdfprimitive} in \hologo{pdfTeX} or
% \hologo{LuaTeX}. \hologo{XeTeX} calls
% it \cs{ifprimitive}. It is a switch that checks if the command
% \csi{cmd} has it's primitive meaning.
%
% \subsubsection{Additional macro: \cs{pdf@isprimitive}}
%
% \begin{declcs}{pdf@isprimitive} \csi{cmd1} \csi{cmd2} \M{true} \M{false}
% \end{declcs}
% If \csi{cmd1} has the primitive meaning given by the primitive name
% of \csi{cmd2}, then the argument \meta{true} is executed, otherwise
% \meta{false}. The macro \cs{pdf@isprimitive} is expandable.
% Internally it checks the result of \cs{meaning} and is therefore
% available for all \hologo{TeX} variants, even the original \hologo{TeX}.
% Example with \hologo{LaTeX}:
%\begin{quote}
%\begin{verbatim}
%\makeatletter
%\pdf@isprimitive{@@input}{input}{%
%  \typeout{\string\@@input\space is original\string\input}%
%}{%
%  \typeout{Oops, \string\@@input\space is not the %
%           original\string\input}%
%}
%\end{verbatim}
%\end{quote}
%
% \subsubsection{Experimental}
%
% \begin{declcs}{pdf@unescapehexnative} \M{string}\\
%   \cs{pdf@escapehexnative} \M{string}\\
%   \cs{pdf@escapenamenative} \M{string}\\
%   \cs{pdf@mdfivesumnative} \M{string}
% \end{declcs}
% The variants without |native| in the macro name are supposed to
% be compatible with \hologo{pdfTeX}. However characters with more than
% eight bits are not supported and are ignored. If \hologo{LuaTeX} is
% running, then its UTF-8 coded strings are used. Thus the full
% unicode character range is supported. However the result
% differs from \hologo{pdfTeX} for characters with eight or more bits.
%
% \begin{declcs}{pdf@pipe} \M{cmdline}
% \end{declcs}
% It calls \meta{cmdline} and returns the output of the external
% program in the usual manner as byte string (catcode 12, space with
% catcode 10). The Lua documentation says, that the used |io.popen|
% may not be available on all platforms. Then macro \cs{pdf@pipe}
% is undefined.
%
% \StopEventually{
% }
%
% \section{Implementation}
%
%    \begin{macrocode}
%<*package>
%    \end{macrocode}
%
% \subsection{Reload check and package identification}
%    Reload check, especially if the package is not used with \LaTeX.
%    \begin{macrocode}
\begingroup\catcode61\catcode48\catcode32=10\relax%
  \catcode13=5 % ^^M
  \endlinechar=13 %
  \catcode35=6 % #
  \catcode39=12 % '
  \catcode44=12 % ,
  \catcode45=12 % -
  \catcode46=12 % .
  \catcode58=12 % :
  \catcode64=11 % @
  \catcode123=1 % {
  \catcode125=2 % }
  \expandafter\let\expandafter\x\csname ver@pdftexcmds.sty\endcsname
  \ifx\x\relax % plain-TeX, first loading
  \else
    \def\empty{}%
    \ifx\x\empty % LaTeX, first loading,
      % variable is initialized, but \ProvidesPackage not yet seen
    \else
      \expandafter\ifx\csname PackageInfo\endcsname\relax
        \def\x#1#2{%
          \immediate\write-1{Package #1 Info: #2.}%
        }%
      \else
        \def\x#1#2{\PackageInfo{#1}{#2, stopped}}%
      \fi
      \x{pdftexcmds}{The package is already loaded}%
      \aftergroup\endinput
    \fi
  \fi
\endgroup%
%    \end{macrocode}
%    Package identification:
%    \begin{macrocode}
\begingroup\catcode61\catcode48\catcode32=10\relax%
  \catcode13=5 % ^^M
  \endlinechar=13 %
  \catcode35=6 % #
  \catcode39=12 % '
  \catcode40=12 % (
  \catcode41=12 % )
  \catcode44=12 % ,
  \catcode45=12 % -
  \catcode46=12 % .
  \catcode47=12 % /
  \catcode58=12 % :
  \catcode64=11 % @
  \catcode91=12 % [
  \catcode93=12 % ]
  \catcode123=1 % {
  \catcode125=2 % }
  \expandafter\ifx\csname ProvidesPackage\endcsname\relax
    \def\x#1#2#3[#4]{\endgroup
      \immediate\write-1{Package: #3 #4}%
      \xdef#1{#4}%
    }%
  \else
    \def\x#1#2[#3]{\endgroup
      #2[{#3}]%
      \ifx#1\@undefined
        \xdef#1{#3}%
      \fi
      \ifx#1\relax
        \xdef#1{#3}%
      \fi
    }%
  \fi
\expandafter\x\csname ver@pdftexcmds.sty\endcsname
\ProvidesPackage{pdftexcmds}%
  [2019/07/25 v0.30 Utility functions of pdfTeX for LuaTeX (HO)]%
%    \end{macrocode}
%
% \subsection{Catcodes}
%
%    \begin{macrocode}
\begingroup\catcode61\catcode48\catcode32=10\relax%
  \catcode13=5 % ^^M
  \endlinechar=13 %
  \catcode123=1 % {
  \catcode125=2 % }
  \catcode64=11 % @
  \def\x{\endgroup
    \expandafter\edef\csname pdftexcmds@AtEnd\endcsname{%
      \endlinechar=\the\endlinechar\relax
      \catcode13=\the\catcode13\relax
      \catcode32=\the\catcode32\relax
      \catcode35=\the\catcode35\relax
      \catcode61=\the\catcode61\relax
      \catcode64=\the\catcode64\relax
      \catcode123=\the\catcode123\relax
      \catcode125=\the\catcode125\relax
    }%
  }%
\x\catcode61\catcode48\catcode32=10\relax%
\catcode13=5 % ^^M
\endlinechar=13 %
\catcode35=6 % #
\catcode64=11 % @
\catcode123=1 % {
\catcode125=2 % }
\def\TMP@EnsureCode#1#2{%
  \edef\pdftexcmds@AtEnd{%
    \pdftexcmds@AtEnd
    \catcode#1=\the\catcode#1\relax
  }%
  \catcode#1=#2\relax
}
\TMP@EnsureCode{0}{12}%
\TMP@EnsureCode{1}{12}%
\TMP@EnsureCode{2}{12}%
\TMP@EnsureCode{10}{12}% ^^J
\TMP@EnsureCode{33}{12}% !
\TMP@EnsureCode{34}{12}% "
\TMP@EnsureCode{38}{4}% &
\TMP@EnsureCode{39}{12}% '
\TMP@EnsureCode{40}{12}% (
\TMP@EnsureCode{41}{12}% )
\TMP@EnsureCode{42}{12}% *
\TMP@EnsureCode{43}{12}% +
\TMP@EnsureCode{44}{12}% ,
\TMP@EnsureCode{45}{12}% -
\TMP@EnsureCode{46}{12}% .
\TMP@EnsureCode{47}{12}% /
\TMP@EnsureCode{58}{12}% :
\TMP@EnsureCode{60}{12}% <
\TMP@EnsureCode{62}{12}% >
\TMP@EnsureCode{91}{12}% [
\TMP@EnsureCode{93}{12}% ]
\TMP@EnsureCode{94}{7}% ^ (superscript)
\TMP@EnsureCode{95}{12}% _ (other)
\TMP@EnsureCode{96}{12}% `
\TMP@EnsureCode{126}{12}% ~ (other)
\edef\pdftexcmds@AtEnd{%
  \pdftexcmds@AtEnd
  \escapechar=\number\escapechar\relax
  \noexpand\endinput
}
\escapechar=92 %
%    \end{macrocode}
%
% \subsection{Load packages}
%
%    \begin{macrocode}
\begingroup\expandafter\expandafter\expandafter\endgroup
\expandafter\ifx\csname RequirePackage\endcsname\relax
  \def\TMP@RequirePackage#1[#2]{%
    \begingroup\expandafter\expandafter\expandafter\endgroup
    \expandafter\ifx\csname ver@#1.sty\endcsname\relax
      \input #1.sty\relax
    \fi
  }%
  \TMP@RequirePackage{infwarerr}[2007/09/09]%
  \TMP@RequirePackage{ifluatex}[2010/03/01]%
  \TMP@RequirePackage{ltxcmds}[2010/12/02]%
  \TMP@RequirePackage{ifpdf}[2010/09/13]%
\else
  \RequirePackage{infwarerr}[2007/09/09]%
  \RequirePackage{ifluatex}[2010/03/01]%
  \RequirePackage{ltxcmds}[2010/12/02]%
  \RequirePackage{ifpdf}[2010/09/13]%
\fi
%    \end{macrocode}
%
% \subsection{Without \hologo{LuaTeX}}
%
%    \begin{macrocode}
\ifluatex
\else
  \@PackageInfoNoLine{pdftexcmds}{LuaTeX not detected}%
  \def\pdftexcmds@nopdftex{%
    \@PackageInfoNoLine{pdftexcmds}{pdfTeX >= 1.30 not detected}%
    \let\pdftexcmds@nopdftex\relax
  }%
  \def\pdftexcmds@temp#1{%
    \begingroup\expandafter\expandafter\expandafter\endgroup
    \expandafter\ifx\csname pdf#1\endcsname\relax
      \pdftexcmds@nopdftex
    \else
      \expandafter\def\csname pdf@#1\expandafter\endcsname
      \expandafter##\expandafter{%
        \csname pdf#1\endcsname
      }%
    \fi
  }%
  \pdftexcmds@temp{strcmp}%
  \pdftexcmds@temp{escapehex}%
  \let\pdf@escapehexnative\pdf@escapehex
  \pdftexcmds@temp{unescapehex}%
  \let\pdf@unescapehexnative\pdf@unescapehex
  \pdftexcmds@temp{escapestring}%
  \pdftexcmds@temp{escapename}%
  \pdftexcmds@temp{filesize}%
  \pdftexcmds@temp{filemoddate}%
  \begingroup\expandafter\expandafter\expandafter\endgroup
  \expandafter\ifx\csname pdfshellescape\endcsname\relax
    \pdftexcmds@nopdftex
    \ltx@IfUndefined{pdftexversion}{%
    }{%
      \ifnum\pdftexversion>120 % 1.21a supports \ifeof18
        \ifeof18 %
          \chardef\pdf@shellescape=0 %
        \else
          \chardef\pdf@shellescape=1 %
        \fi
      \fi
    }%
  \else
    \def\pdf@shellescape{%
      \pdfshellescape
    }%
  \fi
  \begingroup\expandafter\expandafter\expandafter\endgroup
  \expandafter\ifx\csname pdffiledump\endcsname\relax
    \pdftexcmds@nopdftex
  \else
    \def\pdf@filedump#1#2#3{%
      \pdffiledump offset#1 length#2{#3}%
    }%
  \fi
%    \end{macrocode}
%    \begin{macrocode}
  \begingroup\expandafter\expandafter\expandafter\endgroup
  \expandafter\ifx\csname pdfmdfivesum\endcsname\relax
    \begingroup\expandafter\expandafter\expandafter\endgroup
    \expandafter\ifx\csname mdfivesum\endcsname\relax
      \pdftexcmds@nopdftex
    \else
      \def\pdf@mdfivesum#{\mdfivesum}%
      \let\pdf@mdfivesumnative\pdf@mdfivesum
      \def\pdf@filemdfivesum#{\mdfivesum file}%
    \fi
  \else
    \def\pdf@mdfivesum#{\pdfmdfivesum}%
    \let\pdf@mdfivesumnative\pdf@mdfivesum
    \def\pdf@filemdfivesum#{\pdfmdfivesum file}%
  \fi
%    \end{macrocode}
%    \begin{macrocode}
  \def\pdf@system#{%
    \immediate\write18%
  }%
  \def\pdftexcmds@temp#1{%
    \begingroup\expandafter\expandafter\expandafter\endgroup
    \expandafter\ifx\csname pdf#1\endcsname\relax
      \pdftexcmds@nopdftex
    \else
      \expandafter\let\csname pdf@#1\expandafter\endcsname
      \csname pdf#1\endcsname
    \fi
  }%
  \pdftexcmds@temp{resettimer}%
  \pdftexcmds@temp{elapsedtime}%
\fi
%    \end{macrocode}
%
% \subsection{\cs{pdf@primitive}, \cs{pdf@ifprimitive}}
%
%    Since version 1.40.0 \hologo{pdfTeX} has \cs{pdfprimitive} and
%    \cs{ifpdfprimitive}. And \cs{pdfprimitive} was fixed in
%    version 1.40.4.
%
%    \hologo{XeTeX} provides them under the name \cs{primitive} and
%    \cs{ifprimitive}. \hologo{LuaTeX} knows both name variants,
%    but they have possibly to be enabled first (|tex.enableprimitives|).
%
%    Depending on the format TeX Live uses a prefix |luatex|.
%
%    Caution: \cs{let} must be used for the definition of
%    the macros, especially because of \cs{ifpdfprimitive}.
%
% \subsubsection{Using \hologo{LuaTeX}'s \texttt{tex.enableprimitives}}
%
%    \begin{macrocode}
\ifluatex
%    \end{macrocode}
%    \begin{macro}{\pdftexcmds@directlua}
%    \begin{macrocode}
  \ifnum\luatexversion<36 %
    \def\pdftexcmds@directlua{\directlua0 }%
  \else
    \let\pdftexcmds@directlua\directlua
  \fi
%    \end{macrocode}
%    \end{macro}
%
%    \begin{macrocode}
  \begingroup
    \newlinechar=10 %
    \endlinechar=\newlinechar
    \pdftexcmds@directlua{%
      if tex.enableprimitives then
        tex.enableprimitives(
          'pdf@',
          {'primitive', 'ifprimitive', 'pdfdraftmode','draftmode'}
        )
        tex.enableprimitives('', {'luaescapestring'})
      end
    }%
  \endgroup %
%    \end{macrocode}
%
%    \begin{macrocode}
\fi
%    \end{macrocode}
%
% \subsubsection{Trying various names to find the primitives}
%
%    \begin{macro}{\pdftexcmds@strip@prefix}
%    \begin{macrocode}
\def\pdftexcmds@strip@prefix#1>{}
%    \end{macrocode}
%    \end{macro}
%    \begin{macrocode}
\def\pdftexcmds@temp#1#2#3{%
  \begingroup\expandafter\expandafter\expandafter\endgroup
  \expandafter\ifx\csname pdf@#1\endcsname\relax
    \begingroup
      \def\x{#3}%
      \edef\x{\expandafter\pdftexcmds@strip@prefix\meaning\x}%
      \escapechar=-1 %
      \edef\y{\expandafter\meaning\csname#2\endcsname}%
    \expandafter\endgroup
    \ifx\x\y
      \expandafter\let\csname pdf@#1\expandafter\endcsname
      \csname #2\endcsname
    \fi
  \fi
}
%    \end{macrocode}
%
%    \begin{macro}{\pdf@primitive}
%    \begin{macrocode}
\pdftexcmds@temp{primitive}{pdfprimitive}{pdfprimitive}% pdfTeX, oldLuaTeX
\pdftexcmds@temp{primitive}{primitive}{primitive}% XeTeX, luatex
\pdftexcmds@temp{primitive}{luatexprimitive}{pdfprimitive}% oldLuaTeX
\pdftexcmds@temp{primitive}{luatexpdfprimitive}{pdfprimitive}% oldLuaTeX
%    \end{macrocode}
%    \end{macro}
%    \begin{macro}{\pdf@ifprimitive}
%    \begin{macrocode}
\pdftexcmds@temp{ifprimitive}{ifpdfprimitive}{ifpdfprimitive}% pdfTeX, oldLuaTeX
\pdftexcmds@temp{ifprimitive}{ifprimitive}{ifprimitive}% XeTeX, luatex
\pdftexcmds@temp{ifprimitive}{luatexifprimitive}{ifpdfprimitive}% oldLuaTeX
\pdftexcmds@temp{ifprimitive}{luatexifpdfprimitive}{ifpdfprimitive}% oldLuaTeX
%    \end{macrocode}
%    \end{macro}
%
%    Disable broken \cs{pdfprimitive}.
%    \begin{macrocode}
\ifluatex\else
\begingroup
  \expandafter\ifx\csname pdf@primitive\endcsname\relax
  \else
    \expandafter\ifx\csname pdftexversion\endcsname\relax
    \else
      \ifnum\pdftexversion=140 %
        \expandafter\ifx\csname pdftexrevision\endcsname\relax
        \else
          \ifnum\pdftexrevision<4 %
            \endgroup
            \let\pdf@primitive\@undefined
            \@PackageInfoNoLine{pdftexcmds}{%
              \string\pdf@primitive\space disabled, %
              because\MessageBreak
              \string\pdfprimitive\space is broken until pdfTeX 1.40.4%
            }%
            \begingroup
          \fi
        \fi
      \fi
    \fi
  \fi
\endgroup
\fi
%    \end{macrocode}
%
% \subsubsection{Result}
%
%    \begin{macrocode}
\begingroup
  \@PackageInfoNoLine{pdftexcmds}{%
    \string\pdf@primitive\space is %
    \expandafter\ifx\csname pdf@primitive\endcsname\relax not \fi
    available%
  }%
  \@PackageInfoNoLine{pdftexcmds}{%
    \string\pdf@ifprimitive\space is %
    \expandafter\ifx\csname pdf@ifprimitive\endcsname\relax not \fi
    available%
  }%
\endgroup
%    \end{macrocode}
%
% \subsection{\hologo{XeTeX}}
%
%    Look for primitives \cs{shellescape}, \cs{strcmp}.
%    \begin{macrocode}
\def\pdftexcmds@temp#1{%
  \begingroup\expandafter\expandafter\expandafter\endgroup
  \expandafter\ifx\csname pdf@#1\endcsname\relax
    \begingroup
      \escapechar=-1 %
      \edef\x{\expandafter\meaning\csname#1\endcsname}%
      \def\y{#1}%
      \def\z##1->{}%
      \edef\y{\expandafter\z\meaning\y}%
    \expandafter\endgroup
    \ifx\x\y
      \expandafter\def\csname pdf@#1\expandafter\endcsname
      \expandafter{%
        \csname#1\endcsname
      }%
    \fi
  \fi
}%
\pdftexcmds@temp{shellescape}%
\pdftexcmds@temp{strcmp}%
%    \end{macrocode}
%
% \subsection{\cs{pdf@isprimitive}}
%
%    \begin{macrocode}
\def\pdf@isprimitive{%
  \begingroup\expandafter\expandafter\expandafter\endgroup
  \expandafter\ifx\csname pdf@strcmp\endcsname\relax
    \long\def\pdf@isprimitive##1{%
      \expandafter\pdftexcmds@isprimitive\expandafter{\meaning##1}%
    }%
    \long\def\pdftexcmds@isprimitive##1##2{%
      \expandafter\pdftexcmds@@isprimitive\expandafter{\string##2}{##1}%
    }%
    \def\pdftexcmds@@isprimitive##1##2{%
      \ifnum0\pdftexcmds@equal##1\delimiter##2\delimiter=1 %
        \expandafter\ltx@firstoftwo
      \else
        \expandafter\ltx@secondoftwo
      \fi
    }%
    \def\pdftexcmds@equal##1##2\delimiter##3##4\delimiter{%
      \ifx##1##3%
        \ifx\relax##2##4\relax
          1%
        \else
          \ifx\relax##2\relax
          \else
            \ifx\relax##4\relax
            \else
              \pdftexcmds@equalcont{##2}{##4}%
            \fi
          \fi
        \fi
      \fi
    }%
    \def\pdftexcmds@equalcont##1{%
      \def\pdftexcmds@equalcont####1####2##1##1##1##1{%
        ##1##1##1##1%
        \pdftexcmds@equal####1\delimiter####2\delimiter
      }%
    }%
    \expandafter\pdftexcmds@equalcont\csname fi\endcsname
  \else
    \long\def\pdf@isprimitive##1##2{%
      \ifnum\pdf@strcmp{\meaning##1}{\string##2}=0 %
        \expandafter\ltx@firstoftwo
      \else
        \expandafter\ltx@secondoftwo
      \fi
    }%
  \fi
}
\ifluatex
\ifx\pdfdraftmode\@undefined
  \let\pdfdraftmode\draftmode
\fi
\else
  \pdf@isprimitive
\fi
%    \end{macrocode}
%
% \subsection{\cs{pdf@draftmode}}
%
%
%    \begin{macrocode}
\let\pdftexcmds@temp\ltx@zero %
\ltx@IfUndefined{pdfdraftmode}{%
  \@PackageInfoNoLine{pdftexcmds}{\ltx@backslashchar pdfdraftmode not found}%
}{%
  \ifpdf
    \let\pdftexcmds@temp\ltx@one
    \@PackageInfoNoLine{pdftexcmds}{\ltx@backslashchar pdfdraftmode found}%
  \else
    \@PackageInfoNoLine{pdftexcmds}{%
      \ltx@backslashchar pdfdraftmode is ignored in DVI mode%
    }%
  \fi
}
\ifcase\pdftexcmds@temp
%    \end{macrocode}
%    \begin{macro}{\pdf@draftmode}
%    \begin{macrocode}
  \let\pdf@draftmode\ltx@zero
%    \end{macrocode}
%    \end{macro}
%    \begin{macro}{\pdf@ifdraftmode}
%    \begin{macrocode}
  \let\pdf@ifdraftmode\ltx@secondoftwo
%    \end{macrocode}
%    \end{macro}
%    \begin{macro}{\pdftexcmds@setdraftmode}
%    \begin{macrocode}
  \def\pdftexcmds@setdraftmode#1{}%
%    \end{macrocode}
%    \end{macro}
%    \begin{macrocode}
\else
%    \end{macrocode}
%    \begin{macro}{\pdftexcmds@draftmode}
%    \begin{macrocode}
  \let\pdftexcmds@draftmode\pdfdraftmode
%    \end{macrocode}
%    \end{macro}
%    \begin{macro}{\pdf@ifdraftmode}
%    \begin{macrocode}
  \def\pdf@ifdraftmode{%
    \ifnum\pdftexcmds@draftmode=\ltx@one
      \expandafter\ltx@firstoftwo
    \else
      \expandafter\ltx@secondoftwo
    \fi
  }%
%    \end{macrocode}
%    \end{macro}
%    \begin{macro}{\pdf@draftmode}
%    \begin{macrocode}
  \def\pdf@draftmode{%
    \ifnum\pdftexcmds@draftmode=\ltx@one
      \expandafter\ltx@one
    \else
      \expandafter\ltx@zero
    \fi
  }%
%    \end{macrocode}
%    \end{macro}
%    \begin{macro}{\pdftexcmds@setdraftmode}
%    \begin{macrocode}
  \def\pdftexcmds@setdraftmode#1{%
    \pdftexcmds@draftmode=#1\relax
  }%
%    \end{macrocode}
%    \end{macro}
%    \begin{macrocode}
\fi
%    \end{macrocode}
%    \begin{macro}{\pdf@setdraftmode}
%    \begin{macrocode}
\def\pdf@setdraftmode#1{%
  \begingroup
    \count\ltx@cclv=#1\relax
  \edef\x{\endgroup
    \noexpand\pdftexcmds@@setdraftmode{\the\count\ltx@cclv}%
  }%
  \x
}
%    \end{macrocode}
%    \end{macro}
%    \begin{macro}{\pdftexcmds@@setdraftmode}
%    \begin{macrocode}
\def\pdftexcmds@@setdraftmode#1{%
  \ifcase#1 %
    \pdftexcmds@setdraftmode{#1}%
  \or
    \pdftexcmds@setdraftmode{#1}%
  \else
    \@PackageWarning{pdftexcmds}{%
      \string\pdf@setdraftmode: Ignoring\MessageBreak
      invalid value `#1'%
    }%
  \fi
}
%    \end{macrocode}
%    \end{macro}
%
% \subsection{Load Lua module}
%
%    \begin{macrocode}
\ifluatex
\else
  \expandafter\pdftexcmds@AtEnd
\fi%
%    \end{macrocode}
%
%    \begin{macrocode}
\ifnum\luatexversion<80
  \begingroup\expandafter\expandafter\expandafter\endgroup
  \expandafter\ifx\csname RequirePackage\endcsname\relax
    \def\TMP@RequirePackage#1[#2]{%
      \begingroup\expandafter\expandafter\expandafter\endgroup
      \expandafter\ifx\csname ver@#1.sty\endcsname\relax
        \input #1.sty\relax
      \fi
    }%
    \TMP@RequirePackage{luatex-loader}[2009/04/10]%
  \else
    \RequirePackage{luatex-loader}[2009/04/10]%
  \fi
\fi
\pdftexcmds@directlua{%
  require("pdftexcmds")%
}
\ifnum\luatexversion>37 %
  \ifnum0%
      \pdftexcmds@directlua{%
        if status.ini_version then %
          tex.write("1")%
        end%
      }>0 %
    \everyjob\expandafter{%
      \the\everyjob
      \pdftexcmds@directlua{%
        require("pdftexcmds")%
      }%
    }%
  \fi
\fi
\begingroup
  \def\x{2019/07/25 v0.30}%
  \ltx@onelevel@sanitize\x
  \edef\y{%
    \pdftexcmds@directlua{%
      if oberdiek.pdftexcmds.getversion then %
        oberdiek.pdftexcmds.getversion()%
      end%
    }%
  }%
  \ifx\x\y
  \else
    \@PackageError{pdftexcmds}{%
      Wrong version of lua module.\MessageBreak
      Package version: \x\MessageBreak
      Lua module: \y
    }\@ehc
  \fi
\endgroup
%    \end{macrocode}
%
% \subsection{Lua functions}
%
% \subsubsection{Helper macros}
%
%    \begin{macro}{\pdftexcmds@toks}
%    \begin{macrocode}
\begingroup\expandafter\expandafter\expandafter\endgroup
\expandafter\ifx\csname newtoks\endcsname\relax
  \toksdef\pdftexcmds@toks=0 %
\else
  \csname newtoks\endcsname\pdftexcmds@toks
\fi
%    \end{macrocode}
%    \end{macro}
%
%    \begin{macro}{\pdftexcmds@Patch}
%    \begin{macrocode}
\def\pdftexcmds@Patch{0}
\ifnum\luatexversion>40 %
  \ifnum\luatexversion<66 %
    \def\pdftexcmds@Patch{1}%
  \fi
\fi
%    \end{macrocode}
%    \end{macro}
%    \begin{macrocode}
\ifcase\pdftexcmds@Patch
  \catcode`\&=14 %
\else
  \catcode`\&=9 %
%    \end{macrocode}
%    \begin{macro}{\pdftexcmds@PatchDecode}
%    \begin{macrocode}
  \def\pdftexcmds@PatchDecode#1\@nil{%
    \pdftexcmds@DecodeA#1^^A^^A\@nil{}%
  }%
%    \end{macrocode}
%    \end{macro}
%    \begin{macro}{\pdftexcmds@DecodeA}
%    \begin{macrocode}
  \def\pdftexcmds@DecodeA#1^^A^^A#2\@nil#3{%
    \ifx\relax#2\relax
      \ltx@ReturnAfterElseFi{%
        \pdftexcmds@DecodeB#3#1^^A^^B\@nil{}%
      }%
    \else
      \ltx@ReturnAfterFi{%
        \pdftexcmds@DecodeA#2\@nil{#3#1^^@}%
      }%
    \fi
  }%
%    \end{macrocode}
%    \end{macro}
%    \begin{macro}{\pdftexcmds@DecodeB}
%    \begin{macrocode}
  \def\pdftexcmds@DecodeB#1^^A^^B#2\@nil#3{%
    \ifx\relax#2\relax%
      \ltx@ReturnAfterElseFi{%
        \ltx@zero
        #3#1%
      }%
    \else
      \ltx@ReturnAfterFi{%
        \pdftexcmds@DecodeB#2\@nil{#3#1^^A}%
      }%
    \fi
  }%
%    \end{macrocode}
%    \end{macro}
%    \begin{macrocode}
\fi
%    \end{macrocode}
%
%    \begin{macrocode}
\ifnum\luatexversion<36 %
\else
  \catcode`\0=9 %
\fi
%    \end{macrocode}
%
% \subsubsection[Strings]{Strings \cite[``7.15 Strings'']{pdftex-manual}}
%
%    \begin{macro}{\pdf@strcmp}
%    \begin{macrocode}
\long\def\pdf@strcmp#1#2{%
  \directlua0{%
    oberdiek.pdftexcmds.strcmp("\luaescapestring{#1}",%
        "\luaescapestring{#2}")%
  }%
}%
%    \end{macrocode}
%    \end{macro}
%    \begin{macrocode}
\pdf@isprimitive
%    \end{macrocode}
%    \begin{macro}{\pdf@escapehex}
%    \begin{macrocode}
\long\def\pdf@escapehex#1{%
  \directlua0{%
    oberdiek.pdftexcmds.escapehex("\luaescapestring{#1}", "byte")%
  }%
}%
%    \end{macrocode}
%    \end{macro}
%    \begin{macro}{\pdf@escapehexnative}
%    \begin{macrocode}
\long\def\pdf@escapehexnative#1{%
  \directlua0{%
    oberdiek.pdftexcmds.escapehex("\luaescapestring{#1}")%
  }%
}%
%    \end{macrocode}
%    \end{macro}
%    \begin{macro}{\pdf@unescapehex}
%    \begin{macrocode}
\def\pdf@unescapehex#1{%
& \romannumeral\expandafter\pdftexcmds@PatchDecode
  \the\expandafter\pdftexcmds@toks
  \directlua0{%
    oberdiek.pdftexcmds.toks="pdftexcmds@toks"%
    oberdiek.pdftexcmds.unescapehex("\luaescapestring{#1}", "byte", \pdftexcmds@Patch)%
  }%
& \@nil
}%
%    \end{macrocode}
%    \end{macro}
%    \begin{macro}{\pdf@unescapehexnative}
%    \begin{macrocode}
\def\pdf@unescapehexnative#1{%
& \romannumeral\expandafter\pdftexcmds@PatchDecode
  \the\expandafter\pdftexcmds@toks
  \directlua0{%
    oberdiek.pdftexcmds.toks="pdftexcmds@toks"%
    oberdiek.pdftexcmds.unescapehex("\luaescapestring{#1}", \pdftexcmds@Patch)%
  }%
& \@nil
}%
%    \end{macrocode}
%    \end{macro}
%    \begin{macro}{\pdf@escapestring}
%    \begin{macrocode}
\long\def\pdf@escapestring#1{%
  \directlua0{%
    oberdiek.pdftexcmds.escapestring("\luaescapestring{#1}", "byte")%
  }%
}
%    \end{macrocode}
%    \end{macro}
%    \begin{macro}{\pdf@escapename}
%    \begin{macrocode}
\long\def\pdf@escapename#1{%
  \directlua0{%
    oberdiek.pdftexcmds.escapename("\luaescapestring{#1}", "byte")%
  }%
}
%    \end{macrocode}
%    \end{macro}
%    \begin{macro}{\pdf@escapenamenative}
%    \begin{macrocode}
\long\def\pdf@escapenamenative#1{%
  \directlua0{%
    oberdiek.pdftexcmds.escapename("\luaescapestring{#1}")%
  }%
}
%    \end{macrocode}
%    \end{macro}
%
% \subsubsection[Files]{Files \cite[``7.18 Files'']{pdftex-manual}}
%
%    \begin{macro}{\pdf@filesize}
%    \begin{macrocode}
\def\pdf@filesize#1{%
  \directlua0{%
    oberdiek.pdftexcmds.filesize("\luaescapestring{#1}")%
  }%
}
%    \end{macrocode}
%    \end{macro}
%    \begin{macro}{\pdf@filemoddate}
%    \begin{macrocode}
\def\pdf@filemoddate#1{%
  \directlua0{%
    oberdiek.pdftexcmds.filemoddate("\luaescapestring{#1}")%
  }%
}
%    \end{macrocode}
%    \end{macro}
%    \begin{macro}{\pdf@filedump}
%    \begin{macrocode}
\def\pdf@filedump#1#2#3{%
  \directlua0{%
    oberdiek.pdftexcmds.filedump("\luaescapestring{\number#1}",%
        "\luaescapestring{\number#2}",%
        "\luaescapestring{#3}")%
  }%
}%
%    \end{macrocode}
%    \end{macro}
%    \begin{macro}{\pdf@mdfivesum}
%    \begin{macrocode}
\long\def\pdf@mdfivesum#1{%
  \directlua0{%
    oberdiek.pdftexcmds.mdfivesum("\luaescapestring{#1}", "byte")%
  }%
}%
%    \end{macrocode}
%    \end{macro}
%    \begin{macro}{\pdf@mdfivesumnative}
%    \begin{macrocode}
\long\def\pdf@mdfivesumnative#1{%
  \directlua0{%
    oberdiek.pdftexcmds.mdfivesum("\luaescapestring{#1}")%
  }%
}%
%    \end{macrocode}
%    \end{macro}
%    \begin{macro}{\pdf@filemdfivesum}
%    \begin{macrocode}
\def\pdf@filemdfivesum#1{%
  \directlua0{%
    oberdiek.pdftexcmds.filemdfivesum("\luaescapestring{#1}")%
  }%
}%
%    \end{macrocode}
%    \end{macro}
%
% \subsubsection[Timekeeping]{Timekeeping \cite[``7.17 Timekeeping'']{pdftex-manual}}
%
%    \begin{macro}{\protected}
%    \begin{macrocode}
\let\pdftexcmds@temp=Y%
\begingroup\expandafter\expandafter\expandafter\endgroup
\expandafter\ifx\csname protected\endcsname\relax
  \pdftexcmds@directlua0{%
    if tex.enableprimitives then %
      tex.enableprimitives('', {'protected'})%
    end%
  }%
\fi
\begingroup\expandafter\expandafter\expandafter\endgroup
\expandafter\ifx\csname protected\endcsname\relax
  \let\pdftexcmds@temp=N%
\fi
%    \end{macrocode}
%    \end{macro}
%    \begin{macro}{\numexpr}
%    \begin{macrocode}
\begingroup\expandafter\expandafter\expandafter\endgroup
\expandafter\ifx\csname numexpr\endcsname\relax
  \pdftexcmds@directlua0{%
    if tex.enableprimitives then %
      tex.enableprimitives('', {'numexpr'})%
    end%
  }%
\fi
\begingroup\expandafter\expandafter\expandafter\endgroup
\expandafter\ifx\csname numexpr\endcsname\relax
  \let\pdftexcmds@temp=N%
\fi
%    \end{macrocode}
%    \end{macro}
%
%    \begin{macrocode}
\ifx\pdftexcmds@temp N%
  \@PackageWarningNoLine{pdftexcmds}{%
    Definitions of \ltx@backslashchar pdf@resettimer and%
    \MessageBreak
    \ltx@backslashchar pdf@elapsedtime are skipped, because%
    \MessageBreak
    e-TeX's \ltx@backslashchar protected or %
    \ltx@backslashchar numexpr are missing%
  }%
\else
%    \end{macrocode}
%
%    \begin{macro}{\pdf@resettimer}
%    \begin{macrocode}
  \protected\def\pdf@resettimer{%
    \pdftexcmds@directlua0{%
      oberdiek.pdftexcmds.resettimer()%
    }%
  }%
%    \end{macrocode}
%    \end{macro}
%
%    \begin{macro}{\pdf@elapsedtime}
%    \begin{macrocode}
  \protected\def\pdf@elapsedtime{%
    \numexpr
      \pdftexcmds@directlua0{%
        oberdiek.pdftexcmds.elapsedtime()%
      }%
    \relax
  }%
%    \end{macrocode}
%    \end{macro}
%    \begin{macrocode}
\fi
%    \end{macrocode}
%
% \subsubsection{Shell escape}
%
%    \begin{macro}{\pdf@shellescape}
%
%    \begin{macrocode}
\ifnum\luatexversion<68 %
\else
  \protected\edef\pdf@shellescape{%
   \numexpr\directlua{tex.sprint(%
         \number\catcodetable@string,status.shell_escape)}\relax}
\fi
%    \end{macrocode}
%    \end{macro}
%
%    \begin{macro}{\pdf@system}
%    \begin{macrocode}
\def\pdf@system#1{%
  \directlua0{%
    oberdiek.pdftexcmds.system("\luaescapestring{#1}")%
  }%
}
%    \end{macrocode}
%    \end{macro}
%
%    \begin{macro}{\pdf@lastsystemstatus}
%    \begin{macrocode}
\def\pdf@lastsystemstatus{%
  \directlua0{%
    oberdiek.pdftexcmds.lastsystemstatus()%
  }%
}
%    \end{macrocode}
%    \end{macro}
%    \begin{macro}{\pdf@lastsystemexit}
%    \begin{macrocode}
\def\pdf@lastsystemexit{%
  \directlua0{%
    oberdiek.pdftexcmds.lastsystemexit()%
  }%
}
%    \end{macrocode}
%    \end{macro}
%
%    \begin{macrocode}
\catcode`\0=12 %
%    \end{macrocode}
%
%    \begin{macro}{\pdf@pipe}
%    Check availability of |io.popen| first.
%    \begin{macrocode}
\ifnum0%
    \pdftexcmds@directlua{%
      if io.popen then %
        tex.write("1")%
      end%
    }%
    =1 %
  \def\pdf@pipe#1{%
&   \romannumeral\expandafter\pdftexcmds@PatchDecode
    \the\expandafter\pdftexcmds@toks
    \pdftexcmds@directlua{%
      oberdiek.pdftexcmds.toks="pdftexcmds@toks"%
      oberdiek.pdftexcmds.pipe("\luaescapestring{#1}", \pdftexcmds@Patch)%
    }%
&   \@nil
  }%
\fi
%    \end{macrocode}
%    \end{macro}
%
%    \begin{macrocode}
\pdftexcmds@AtEnd%
%</package>
%    \end{macrocode}
%
% \subsection{Lua module}
%
%    \begin{macrocode}
%<*lua>
%    \end{macrocode}
%
%    \begin{macrocode}
oberdiek = oberdiek or {}
local pdftexcmds = oberdiek.pdftexcmds or {}
oberdiek.pdftexcmds = pdftexcmds
local systemexitstatus
function pdftexcmds.getversion()
  tex.write("2019/07/25 v0.30")
end
%    \end{macrocode}
%
% \subsubsection[Strings]{Strings \cite[``7.15 Strings'']{pdftex-manual}}
%
%    \begin{macrocode}
function pdftexcmds.strcmp(A, B)
  if A == B then
    tex.write("0")
  elseif A < B then
    tex.write("-1")
  else
    tex.write("1")
  end
end
local function utf8_to_byte(str)
  local i = 0
  local n = string.len(str)
  local t = {}
  while i < n do
    i = i + 1
    local a = string.byte(str, i)
    if a < 128 then
      table.insert(t, string.char(a))
    else
      if a >= 192 and i < n then
        i = i + 1
        local b = string.byte(str, i)
        if b < 128 or b >= 192 then
          i = i - 1
        elseif a == 194 then
          table.insert(t, string.char(b))
        elseif a == 195 then
          table.insert(t, string.char(b + 64))
        end
      end
    end
  end
  return table.concat(t)
end
function pdftexcmds.escapehex(str, mode)
  if mode == "byte" then
    str = utf8_to_byte(str)
  end
  tex.write((string.gsub(str, ".",
    function (ch)
      return string.format("%02X", string.byte(ch))
    end
  )))
end
%    \end{macrocode}
%    See procedure |unescapehex| in file \xfile{utils.c} of \hologo{pdfTeX}.
%    Caution: |tex.write| ignores leading spaces.
%    \begin{macrocode}
function pdftexcmds.unescapehex(str, mode, patch)
  local a = 0
  local first = true
  local result = {}
  for i = 1, string.len(str), 1 do
    local ch = string.byte(str, i)
    if ch >= 48 and ch <= 57 then
      ch = ch - 48
    elseif ch >= 65 and ch <= 70 then
      ch = ch - 55
    elseif ch >= 97 and ch <= 102 then
      ch = ch - 87
    else
      ch = nil
    end
    if ch then
      if first then
        a = ch * 16
        first = false
      else
        table.insert(result, a + ch)
        first = true
      end
    end
  end
  if not first then
    table.insert(result, a)
  end
  if patch == 1 then
    local temp = {}
    for i, a in ipairs(result) do
      if a == 0 then
        table.insert(temp, 1)
        table.insert(temp, 1)
      else
        if a == 1 then
          table.insert(temp, 1)
          table.insert(temp, 2)
        else
          table.insert(temp, a)
        end
      end
    end
    result = temp
  end
  if mode == "byte" then
    local utf8 = {}
    for i, a in ipairs(result) do
      if a < 128 then
        table.insert(utf8, a)
      else
        if a < 192 then
          table.insert(utf8, 194)
          a = a - 128
        else
          table.insert(utf8, 195)
          a = a - 192
        end
        table.insert(utf8, a + 128)
      end
    end
    result = utf8
  end
%    \end{macrocode}
%    this next line added for current luatex; this is the only
%    change in the file.  eroux, 28apr13. (v 0.21)
%    \begin{macrocode}
  local unpack = _G["unpack"] or table.unpack
  tex.settoks(pdftexcmds.toks, string.char(unpack(result)))
end
%    \end{macrocode}
%    See procedure |escapestring| in file \xfile{utils.c} of \hologo{pdfTeX}.
%    \begin{macrocode}
function pdftexcmds.escapestring(str, mode)
  if mode == "byte" then
    str = utf8_to_byte(str)
  end
  tex.write((string.gsub(str, ".",
    function (ch)
      local b = string.byte(ch)
      if b < 33 or b > 126 then
        return string.format("\\%.3o", b)
      end
      if b == 40 or b == 41 or b == 92 then
        return "\\" .. ch
      end
%    \end{macrocode}
%    Lua 5.1 returns the match in case of return value |nil|.
%    \begin{macrocode}
      return nil
    end
  )))
end
%    \end{macrocode}
%    See procedure |escapename| in file \xfile{utils.c} of \hologo{pdfTeX}.
%    \begin{macrocode}
function pdftexcmds.escapename(str, mode)
  if mode == "byte" then
    str = utf8_to_byte(str)
  end
  tex.write((string.gsub(str, ".",
    function (ch)
      local b = string.byte(ch)
      if b == 0 then
%    \end{macrocode}
%    In Lua 5.0 |nil| could be used for the empty string,
%    But |nil| returns the match in Lua 5.1, thus we use
%    the empty string explicitly.
%    \begin{macrocode}
        return ""
      end
      if b <= 32 or b >= 127
          or b == 35 or b == 37 or b == 40 or b == 41
          or b == 47 or b == 60 or b == 62 or b == 91
          or b == 93 or b == 123 or b == 125 then
        return string.format("#%.2X", b)
      else
%    \end{macrocode}
%    Lua 5.1 returns the match in case of return value |nil|.
%    \begin{macrocode}
        return nil
      end
    end
  )))
end
%    \end{macrocode}
%
% \subsubsection[Files]{Files \cite[``7.18 Files'']{pdftex-manual}}
%
%    \begin{macrocode}
function pdftexcmds.filesize(filename)
  local foundfile = kpse.find_file(filename, "tex", true)
  if foundfile then
    local size = lfs.attributes(foundfile, "size")
    if size then
      tex.write(size)
    end
  end
end
%    \end{macrocode}
%    See procedure |makepdftime| in file \xfile{utils.c} of \hologo{pdfTeX}.
%    \begin{macrocode}
function pdftexcmds.filemoddate(filename)
  local foundfile = kpse.find_file(filename, "tex", true)
  if foundfile then
    local date = lfs.attributes(foundfile, "modification")
    if date then
      local d = os.date("*t", date)
      if d.sec >= 60 then
        d.sec = 59
      end
      local u = os.date("!*t", date)
      local off = 60 * (d.hour - u.hour) + d.min - u.min
      if d.year ~= u.year then
        if d.year > u.year then
          off = off + 1440
        else
          off = off - 1440
        end
      elseif d.yday ~= u.yday then
        if d.yday > u.yday then
          off = off + 1440
        else
          off = off - 1440
        end
      end
      local timezone
      if off == 0 then
        timezone = "Z"
      else
        local hours = math.floor(off / 60)
        local mins = math.abs(off - hours * 60)
        timezone = string.format("%+03d'%02d'", hours, mins)
      end
      tex.write(string.format("D:%04d%02d%02d%02d%02d%02d%s",
          d.year, d.month, d.day, d.hour, d.min, d.sec, timezone))
    end
  end
end
function pdftexcmds.filedump(offset, length, filename)
  length = tonumber(length)
  if length and length > 0 then
    local foundfile = kpse.find_file(filename, "tex", true)
    if foundfile then
      offset = tonumber(offset)
      if not offset then
        offset = 0
      end
      local filehandle = io.open(foundfile, "rb")
      if filehandle then
        if offset > 0 then
          filehandle:seek("set", offset)
        end
        local dump = filehandle:read(length)
        pdftexcmds.escapehex(dump)
        filehandle:close()
      end
    end
  end
end
function pdftexcmds.mdfivesum(str, mode)
  if mode == "byte" then
    str = utf8_to_byte(str)
  end
  pdftexcmds.escapehex(md5.sum(str))
end
function pdftexcmds.filemdfivesum(filename)
  local foundfile = kpse.find_file(filename, "tex", true)
  if foundfile then
    local filehandle = io.open(foundfile, "rb")
    if filehandle then
      local contents = filehandle:read("*a")
      pdftexcmds.escapehex(md5.sum(contents))
      filehandle:close()
    end
  end
end
%    \end{macrocode}
%
% \subsubsection[Timekeeping]{Timekeeping \cite[``7.17 Timekeeping'']{pdftex-manual}}
%
%    The functions for timekeeping are based on
%    Andy Thomas' work \cite{AndyThomas:Analog}.
%    Changes:
%    \begin{itemize}
%    \item Overflow check is added.
%    \item |string.format| is used to avoid exponential number
%          representation for sure.
%    \item |tex.write| is used instead of |tex.print| to get
%          tokens with catcode 12 and without appended \cs{endlinechar}.
%    \end{itemize}
%    \begin{macrocode}
local basetime = 0
function pdftexcmds.resettimer()
  basetime = os.clock()
end
function pdftexcmds.elapsedtime()
  local val = (os.clock() - basetime) * 65536 + .5
  if val > 2147483647 then
    val = 2147483647
  end
  tex.write(string.format("%d", val))
end
%    \end{macrocode}
%
% \subsubsection[Miscellaneous]{Miscellaneous \cite[``7.21 Miscellaneous'']{pdftex-manual}}
%
%    \begin{macrocode}
function pdftexcmds.shellescape()
  if os.execute then
    if status
        and status.luatex_version
        and status.luatex_version >= 68 then
      tex.write(os.execute())
    else
      local result = os.execute()
      if result == 0 then
        tex.write("0")
      else
        if result == nil then
          tex.write("0")
        else
          tex.write("1")
        end
      end
    end
  else
    tex.write("0")
  end
end
function pdftexcmds.system(cmdline)
  systemexitstatus = nil
  texio.write_nl("log", "system(" .. cmdline .. ") ")
  if os.execute then
    texio.write("log", "executed.")
    systemexitstatus = os.execute(cmdline)
  else
    texio.write("log", "disabled.")
  end
end
function pdftexcmds.lastsystemstatus()
  local result = tonumber(systemexitstatus)
  if result then
    local x = math.floor(result / 256)
    tex.write(result - 256 * math.floor(result / 256))
  end
end
function pdftexcmds.lastsystemexit()
  local result = tonumber(systemexitstatus)
  if result then
    tex.write(math.floor(result / 256))
  end
end
function pdftexcmds.pipe(cmdline, patch)
  local result
  systemexitstatus = nil
  texio.write_nl("log", "pipe(" .. cmdline ..") ")
  if io.popen then
    texio.write("log", "executed.")
    local handle = io.popen(cmdline, "r")
    if handle then
      result = handle:read("*a")
      handle:close()
    end
  else
    texio.write("log", "disabled.")
  end
  if result then
    if patch == 1 then
      local temp = {}
      for i, a in ipairs(result) do
        if a == 0 then
          table.insert(temp, 1)
          table.insert(temp, 1)
        else
          if a == 1 then
            table.insert(temp, 1)
            table.insert(temp, 2)
          else
            table.insert(temp, a)
          end
        end
      end
      result = temp
    end
    tex.settoks(pdftexcmds.toks, result)
  else
    tex.settoks(pdftexcmds.toks, "")
  end
end
%    \end{macrocode}
%    \begin{macrocode}
%</lua>
%    \end{macrocode}
%
% \section{Test}
%
% \subsection{Catcode checks for loading}
%
%    \begin{macrocode}
%<*test1>
%    \end{macrocode}
%    \begin{macrocode}
\catcode`\{=1 %
\catcode`\}=2 %
\catcode`\#=6 %
\catcode`\@=11 %
\expandafter\ifx\csname count@\endcsname\relax
  \countdef\count@=255 %
\fi
\expandafter\ifx\csname @gobble\endcsname\relax
  \long\def\@gobble#1{}%
\fi
\expandafter\ifx\csname @firstofone\endcsname\relax
  \long\def\@firstofone#1{#1}%
\fi
\expandafter\ifx\csname loop\endcsname\relax
  \expandafter\@firstofone
\else
  \expandafter\@gobble
\fi
{%
  \def\loop#1\repeat{%
    \def\body{#1}%
    \iterate
  }%
  \def\iterate{%
    \body
      \let\next\iterate
    \else
      \let\next\relax
    \fi
    \next
  }%
  \let\repeat=\fi
}%
\def\RestoreCatcodes{}
\count@=0 %
\loop
  \edef\RestoreCatcodes{%
    \RestoreCatcodes
    \catcode\the\count@=\the\catcode\count@\relax
  }%
\ifnum\count@<255 %
  \advance\count@ 1 %
\repeat

\def\RangeCatcodeInvalid#1#2{%
  \count@=#1\relax
  \loop
    \catcode\count@=15 %
  \ifnum\count@<#2\relax
    \advance\count@ 1 %
  \repeat
}
\def\RangeCatcodeCheck#1#2#3{%
  \count@=#1\relax
  \loop
    \ifnum#3=\catcode\count@
    \else
      \errmessage{%
        Character \the\count@\space
        with wrong catcode \the\catcode\count@\space
        instead of \number#3%
      }%
    \fi
  \ifnum\count@<#2\relax
    \advance\count@ 1 %
  \repeat
}
\def\space{ }
\expandafter\ifx\csname LoadCommand\endcsname\relax
  \def\LoadCommand{\input pdftexcmds.sty\relax}%
\fi
\def\Test{%
  \RangeCatcodeInvalid{0}{47}%
  \RangeCatcodeInvalid{58}{64}%
  \RangeCatcodeInvalid{91}{96}%
  \RangeCatcodeInvalid{123}{255}%
  \catcode`\@=12 %
  \catcode`\\=0 %
  \catcode`\%=14 %
  \LoadCommand
  \RangeCatcodeCheck{0}{36}{15}%
  \RangeCatcodeCheck{37}{37}{14}%
  \RangeCatcodeCheck{38}{47}{15}%
  \RangeCatcodeCheck{48}{57}{12}%
  \RangeCatcodeCheck{58}{63}{15}%
  \RangeCatcodeCheck{64}{64}{12}%
  \RangeCatcodeCheck{65}{90}{11}%
  \RangeCatcodeCheck{91}{91}{15}%
  \RangeCatcodeCheck{92}{92}{0}%
  \RangeCatcodeCheck{93}{96}{15}%
  \RangeCatcodeCheck{97}{122}{11}%
  \RangeCatcodeCheck{123}{255}{15}%
  \RestoreCatcodes
}
\Test
\csname @@end\endcsname
\end
%    \end{macrocode}
%    \begin{macrocode}
%</test1>
%    \end{macrocode}
%
% \subsection{Test for \cs{pdf@isprimitive}}
%
%    \begin{macrocode}
%<*test2>
\catcode`\{=1 %
\catcode`\}=2 %
\catcode`\#=6 %
\catcode`\@=11 %
\input pdftexcmds.sty\relax
\def\msg#1{%
  \begingroup
    \escapechar=92 %
    \immediate\write16{#1}%
  \endgroup
}
\long\def\test#1#2#3#4{%
  \begingroup
    #4%
    \def\str{%
      Test \string\pdf@isprimitive
      {\string #1}{\string #2}{...}: %
    }%
    \pdf@isprimitive{#1}{#2}{%
      \ifx#3Y%
        \msg{\str true ==> OK.}%
      \else
        \errmessage{\str false ==> FAILED}%
      \fi
    }{%
      \ifx#3Y%
        \errmessage{\str true ==> FAILED}%
      \else
        \msg{\str false ==> OK.}%
      \fi
    }%
  \endgroup
}
\test\relax\relax Y{}
\test\foobar\relax Y{\let\foobar\relax}
\test\foobar\relax N{}
\test\hbox\hbox Y{}
\test\foobar@hbox\hbox Y{\let\foobar@hbox\hbox}
\test\if\if Y{}
\test\if\ifx N{}
\test\ifx\if N{}
\test\par\par Y{}
\test\hbox\par N{}
\test\par\hbox N{}
\csname @@end\endcsname\end
%</test2>
%    \end{macrocode}
%
% \subsection{Test for \cs{pdf@shellescape}}
%
%    \begin{macrocode}
%<*test-shell>
\catcode`\{=1 %
\catcode`\}=2 %
\catcode`\#=6 %
\catcode`\@=11 %
\input pdftexcmds.sty\relax
\def\msg#{\immediate\write16}
\def\MaybeEnd{}
\ifx\luatexversion\UnDeFiNeD
\else
  \ifnum\luatexversion<68 %
    \ifx\pdf@shellescape\@undefined
      \msg{SHELL=U}%
      \msg{OK (LuaTeX < 0.68)}%
    \else
      \msg{SHELL=defined}%
      \errmessage{Failed (LuaTeX < 0.68)}%
    \fi
    \def\MaybeEnd{\csname @@end\endcsname\end}%
  \fi
\fi
\MaybeEnd
\ifx\pdf@shellescape\@undefined
  \msg{SHELL=U}%
\else
  \msg{SHELL=\number\pdf@shellescape}%
\fi
\ifx\expected\@undefined
\else
  \ifx\expected\relax
    \msg{EXPECTED=U}%
    \ifx\pdf@shellescape\@undefined
      \msg{OK}%
    \else
      \errmessage{Failed}%
    \fi
  \else
    \msg{EXPECTED=\number\expected}%
    \ifnum\pdf@shellescape=\expected\relax
      \msg{OK}%
    \else
      \errmessage{Failed}%
    \fi
  \fi
\fi
\csname @@end\endcsname\end
%</test-shell>
%    \end{macrocode}
%
% \subsection{Test for escape functions}
%
%    \begin{macrocode}
%<*test-escape>
\catcode`\{=1 %
\catcode`\}=2 %
\catcode`\#=6 %
\catcode`\^=7 %
\catcode`\@=11 %
\errorcontextlines=1000 %
\input pdftexcmds.sty\relax
\def\msg#1{%
  \begingroup
    \escapechar=92 %
    \immediate\write16{#1}%
  \endgroup
}
%    \end{macrocode}
%    \begin{macrocode}
\begingroup
  \catcode`\@=11 %
  \countdef\count@=255 %
  \def\space{ }%
  \long\def\@whilenum#1\do #2{%
    \ifnum #1\relax
      #2\relax
      \@iwhilenum{#1\relax#2\relax}%
    \fi
  }%
  \long\def\@iwhilenum#1{%
    \ifnum #1%
      \expandafter\@iwhilenum
    \else
      \expandafter\ltx@gobble
    \fi
    {#1}%
  }%
  \gdef\AllBytes{}%
  \count@=0 %
  \catcode0=12 %
  \@whilenum\count@<256 \do{%
    \lccode0=\count@
    \ifnum\count@=32 %
      \xdef\AllBytes{\AllBytes\space}%
    \else
      \lowercase{%
        \xdef\AllBytes{\AllBytes^^@}%
      }%
    \fi
    \advance\count@ by 1 %
  }%
\endgroup
%    \end{macrocode}
%    \begin{macrocode}
\def\AllBytesHex{%
  000102030405060708090A0B0C0D0E0F%
  101112131415161718191A1B1C1D1E1F%
  202122232425262728292A2B2C2D2E2F%
  303132333435363738393A3B3C3D3E3F%
  404142434445464748494A4B4C4D4E4F%
  505152535455565758595A5B5C5D5E5F%
  606162636465666768696A6B6C6D6E6F%
  707172737475767778797A7B7C7D7E7F%
  808182838485868788898A8B8C8D8E8F%
  909192939495969798999A9B9C9D9E9F%
  A0A1A2A3A4A5A6A7A8A9AAABACADAEAF%
  B0B1B2B3B4B5B6B7B8B9BABBBCBDBEBF%
  C0C1C2C3C4C5C6C7C8C9CACBCCCDCECF%
  D0D1D2D3D4D5D6D7D8D9DADBDCDDDEDF%
  E0E1E2E3E4E5E6E7E8E9EAEBECEDEEEF%
  F0F1F2F3F4F5F6F7F8F9FAFBFCFDFEFF%
}
\ltx@onelevel@sanitize\AllBytesHex
\expandafter\lowercase\expandafter{%
  \expandafter\def\expandafter\AllBytesHexLC
      \expandafter{\AllBytesHex}%
}
\begingroup
  \catcode`\#=12 %
  \xdef\AllBytesName{%
    #01#02#03#04#05#06#07#08#09#0A#0B#0C#0D#0E#0F%
    #10#11#12#13#14#15#16#17#18#19#1A#1B#1C#1D#1E#1F%
    #20!"#23$#25&'#28#29*+,-.#2F%
    0123456789:;#3C=#3E?%
    @ABCDEFGHIJKLMNO%
    PQRSTUVWXYZ#5B\ltx@backslashchar#5D^_%
    `abcdefghijklmno%
    pqrstuvwxyz#7B|#7D\string~#7F%
    #80#81#82#83#84#85#86#87#88#89#8A#8B#8C#8D#8E#8F%
    #90#91#92#93#94#95#96#97#98#99#9A#9B#9C#9D#9E#9F%
    #A0#A1#A2#A3#A4#A5#A6#A7#A8#A9#AA#AB#AC#AD#AE#AF%
    #B0#B1#B2#B3#B4#B5#B6#B7#B8#B9#BA#BB#BC#BD#BE#BF%
    #C0#C1#C2#C3#C4#C5#C6#C7#C8#C9#CA#CB#CC#CD#CE#CF%
    #D0#D1#D2#D3#D4#D5#D6#D7#D8#D9#DA#DB#DC#DD#DE#DF%
    #E0#E1#E2#E3#E4#E5#E6#E7#E8#E9#EA#EB#EC#ED#EE#EF%
    #F0#F1#F2#F3#F4#F5#F6#F7#F8#F9#FA#FB#FC#FD#FE#FF%
  }%
\endgroup
\ltx@onelevel@sanitize\AllBytesName
\edef\AllBytesFromName{\expandafter\ltx@gobble\AllBytes}
\begingroup
  \def\|{|}%
  \edef\%{\ltx@percentchar}%
  \catcode`\|=0 %
  \catcode`\#=12 %
  \catcode`\~=12 %
  \catcode`\\=12 %
  |xdef|AllBytesString{%
    \000\001\002\003\004\005\006\007\010\011\012\013\014\015\016\017%
    \020\021\022\023\024\025\026\027\030\031\032\033\034\035\036\037%
    \040!"#$|%&'\(\)*+,-./%
    0123456789:;<=>?%
    @ABCDEFGHIJKLMNO%
    PQRSTUVWXYZ[\\]^_%
    `abcdefghijklmno%
    pqrstuvwxyz{||}~\177%
    \200\201\202\203\204\205\206\207\210\211\212\213\214\215\216\217%
    \220\221\222\223\224\225\226\227\230\231\232\233\234\235\236\237%
    \240\241\242\243\244\245\246\247\250\251\252\253\254\255\256\257%
    \260\261\262\263\264\265\266\267\270\271\272\273\274\275\276\277%
    \300\301\302\303\304\305\306\307\310\311\312\313\314\315\316\317%
    \320\321\322\323\324\325\326\327\330\331\332\333\334\335\336\337%
    \340\341\342\343\344\345\346\347\350\351\352\353\354\355\356\357%
    \360\361\362\363\364\365\366\367\370\371\372\373\374\375\376\377%
  }%
|endgroup
\ltx@onelevel@sanitize\AllBytesString
%    \end{macrocode}
%    \begin{macrocode}
\def\Test#1#2#3{%
  \begingroup
    \expandafter\expandafter\expandafter\def
    \expandafter\expandafter\expandafter\TestResult
    \expandafter\expandafter\expandafter{%
      #1{#2}%
    }%
    \ifx\TestResult#3%
    \else
      \newlinechar=10 %
      \msg{Expect:^^J#3}%
      \msg{Result:^^J\TestResult}%
      \errmessage{\string#2 -\string#1-> \string#3}%
    \fi
  \endgroup
}
\def\test#1#2#3{%
  \edef\TestFrom{#2}%
  \edef\TestExpect{#3}%
  \ltx@onelevel@sanitize\TestExpect
  \Test#1\TestFrom\TestExpect
}
\test\pdf@unescapehex{74657374}{test}
\begingroup
  \catcode0=12 %
  \catcode1=12 %
  \test\pdf@unescapehex{740074017400740174}{t^^@t^^At^^@t^^At}%
\endgroup
\Test\pdf@escapehex\AllBytes\AllBytesHex
\Test\pdf@unescapehex\AllBytesHex\AllBytes
\Test\pdf@escapename\AllBytes\AllBytesName
\Test\pdf@escapestring\AllBytes\AllBytesString
%    \end{macrocode}
%    \begin{macrocode}
\csname @@end\endcsname\end
%</test-escape>
%    \end{macrocode}
%
% \section{Installation}
%
% \subsection{Download}
%
% \paragraph{Package.} This package is available on
% CTAN\footnote{\CTANpkg{pdftexcmds}}:
% \begin{description}
% \item[\CTAN{macros/latex/contrib/oberdiek/pdftexcmds.dtx}] The source file.
% \item[\CTAN{macros/latex/contrib/oberdiek/pdftexcmds.pdf}] Documentation.
% \end{description}
%
%
% \paragraph{Bundle.} All the packages of the bundle `oberdiek'
% are also available in a TDS compliant ZIP archive. There
% the packages are already unpacked and the documentation files
% are generated. The files and directories obey the TDS standard.
% \begin{description}
% \item[\CTANinstall{install/macros/latex/contrib/oberdiek.tds.zip}]
% \end{description}
% \emph{TDS} refers to the standard ``A Directory Structure
% for \TeX\ Files'' (\CTAN{tds/tds.pdf}). Directories
% with \xfile{texmf} in their name are usually organized this way.
%
% \subsection{Bundle installation}
%
% \paragraph{Unpacking.} Unpack the \xfile{oberdiek.tds.zip} in the
% TDS tree (also known as \xfile{texmf} tree) of your choice.
% Example (linux):
% \begin{quote}
%   |unzip oberdiek.tds.zip -d ~/texmf|
% \end{quote}
%
% \paragraph{Script installation.}
% Check the directory \xfile{TDS:scripts/oberdiek/} for
% scripts that need further installation steps.
% Package \xpackage{attachfile2} comes with the Perl script
% \xfile{pdfatfi.pl} that should be installed in such a way
% that it can be called as \texttt{pdfatfi}.
% Example (linux):
% \begin{quote}
%   |chmod +x scripts/oberdiek/pdfatfi.pl|\\
%   |cp scripts/oberdiek/pdfatfi.pl /usr/local/bin/|
% \end{quote}
%
% \subsection{Package installation}
%
% \paragraph{Unpacking.} The \xfile{.dtx} file is a self-extracting
% \docstrip\ archive. The files are extracted by running the
% \xfile{.dtx} through \plainTeX:
% \begin{quote}
%   \verb|tex pdftexcmds.dtx|
% \end{quote}
%
% \paragraph{TDS.} Now the different files must be moved into
% the different directories in your installation TDS tree
% (also known as \xfile{texmf} tree):
% \begin{quote}
% \def\t{^^A
% \begin{tabular}{@{}>{\ttfamily}l@{ $\rightarrow$ }>{\ttfamily}l@{}}
%   pdftexcmds.sty & tex/generic/oberdiek/pdftexcmds.sty\\
%   oberdiek.pdftexcmds.lua & scripts/oberdiek/oberdiek.pdftexcmds.lua\\
%   pdftexcmds.lua & scripts/oberdiek/pdftexcmds.lua\\
%   pdftexcmds.pdf & doc/latex/oberdiek/pdftexcmds.pdf\\
%   test/pdftexcmds-test1.tex & doc/latex/oberdiek/test/pdftexcmds-test1.tex\\
%   test/pdftexcmds-test2.tex & doc/latex/oberdiek/test/pdftexcmds-test2.tex\\
%   test/pdftexcmds-test-shell.tex & doc/latex/oberdiek/test/pdftexcmds-test-shell.tex\\
%   test/pdftexcmds-test-escape.tex & doc/latex/oberdiek/test/pdftexcmds-test-escape.tex\\
%   pdftexcmds.dtx & source/latex/oberdiek/pdftexcmds.dtx\\
% \end{tabular}^^A
% }^^A
% \sbox0{\t}^^A
% \ifdim\wd0>\linewidth
%   \begingroup
%     \advance\linewidth by\leftmargin
%     \advance\linewidth by\rightmargin
%   \edef\x{\endgroup
%     \def\noexpand\lw{\the\linewidth}^^A
%   }\x
%   \def\lwbox{^^A
%     \leavevmode
%     \hbox to \linewidth{^^A
%       \kern-\leftmargin\relax
%       \hss
%       \usebox0
%       \hss
%       \kern-\rightmargin\relax
%     }^^A
%   }^^A
%   \ifdim\wd0>\lw
%     \sbox0{\small\t}^^A
%     \ifdim\wd0>\linewidth
%       \ifdim\wd0>\lw
%         \sbox0{\footnotesize\t}^^A
%         \ifdim\wd0>\linewidth
%           \ifdim\wd0>\lw
%             \sbox0{\scriptsize\t}^^A
%             \ifdim\wd0>\linewidth
%               \ifdim\wd0>\lw
%                 \sbox0{\tiny\t}^^A
%                 \ifdim\wd0>\linewidth
%                   \lwbox
%                 \else
%                   \usebox0
%                 \fi
%               \else
%                 \lwbox
%               \fi
%             \else
%               \usebox0
%             \fi
%           \else
%             \lwbox
%           \fi
%         \else
%           \usebox0
%         \fi
%       \else
%         \lwbox
%       \fi
%     \else
%       \usebox0
%     \fi
%   \else
%     \lwbox
%   \fi
% \else
%   \usebox0
% \fi
% \end{quote}
% If you have a \xfile{docstrip.cfg} that configures and enables \docstrip's
% TDS installing feature, then some files can already be in the right
% place, see the documentation of \docstrip.
%
% \subsection{Refresh file name databases}
%
% If your \TeX~distribution
% (\teTeX, \mikTeX, \dots) relies on file name databases, you must refresh
% these. For example, \teTeX\ users run \verb|texhash| or
% \verb|mktexlsr|.
%
% \subsection{Some details for the interested}
%
% \paragraph{Unpacking with \LaTeX.}
% The \xfile{.dtx} chooses its action depending on the format:
% \begin{description}
% \item[\plainTeX:] Run \docstrip\ and extract the files.
% \item[\LaTeX:] Generate the documentation.
% \end{description}
% If you insist on using \LaTeX\ for \docstrip\ (really,
% \docstrip\ does not need \LaTeX), then inform the autodetect routine
% about your intention:
% \begin{quote}
%   \verb|latex \let\install=y% \iffalse meta-comment
%
% File: pdftexcmds.dtx
% Version: 2019/07/25 v0.30
% Info: Utility functions of pdfTeX for LuaTeX
%
% Copyright (C) 2007, 2009-2011 by
%    Heiko Oberdiek <heiko.oberdiek at googlemail.com>
%
% This work may be distributed and/or modified under the
% conditions of the LaTeX Project Public License, either
% version 1.3c of this license or (at your option) any later
% version. This version of this license is in
%    https://www.latex-project.org/lppl/lppl-1-3c.txt
% and the latest version of this license is in
%    https://www.latex-project.org/lppl.txt
% and version 1.3 or later is part of all distributions of
% LaTeX version 2005/12/01 or later.
%
% This work has the LPPL maintenance status "maintained".
%
% The Current Maintainers of this work are
% Heiko Oberdiek and the Oberdiek Package Support Group
% https://github.com/ho-tex/oberdiek/issues
%
% The Base Interpreter refers to any `TeX-Format',
% because some files are installed in TDS:tex/generic//.
%
% This work consists of the main source file pdftexcmds.dtx
% and the derived files
%    pdftexcmds.sty, pdftexcmds.pdf, pdftexcmds.ins, pdftexcmds.drv,
%    pdftexcmds.bib, pdftexcmds-test1.tex, pdftexcmds-test2.tex,
%    pdftexcmds-test-shell.tex, pdftexcmds-test-escape.tex,
%    oberdiek.pdftexcmds.lua, pdftexcmds.lua.
%
% Distribution:
%    CTAN:macros/latex/contrib/oberdiek/pdftexcmds.dtx
%    CTAN:macros/latex/contrib/oberdiek/pdftexcmds.pdf
%
% Unpacking:
%    (a) If pdftexcmds.ins is present:
%           tex pdftexcmds.ins
%    (b) Without pdftexcmds.ins:
%           tex pdftexcmds.dtx
%    (c) If you insist on using LaTeX
%           latex \let\install=y\input{pdftexcmds.dtx}
%        (quote the arguments according to the demands of your shell)
%
% Documentation:
%    (a) If pdftexcmds.drv is present:
%           latex pdftexcmds.drv
%    (b) Without pdftexcmds.drv:
%           latex pdftexcmds.dtx; ...
%    The class ltxdoc loads the configuration file ltxdoc.cfg
%    if available. Here you can specify further options, e.g.
%    use A4 as paper format:
%       \PassOptionsToClass{a4paper}{article}
%
%    Programm calls to get the documentation (example):
%       pdflatex pdftexcmds.dtx
%       bibtex pdftexcmds.aux
%       makeindex -s gind.ist pdftexcmds.idx
%       pdflatex pdftexcmds.dtx
%       makeindex -s gind.ist pdftexcmds.idx
%       pdflatex pdftexcmds.dtx
%
% Installation:
%    TDS:tex/generic/oberdiek/pdftexcmds.sty
%    TDS:scripts/oberdiek/oberdiek.pdftexcmds.lua
%    TDS:scripts/oberdiek/pdftexcmds.lua
%    TDS:doc/latex/oberdiek/pdftexcmds.pdf
%    TDS:doc/latex/oberdiek/test/pdftexcmds-test1.tex
%    TDS:doc/latex/oberdiek/test/pdftexcmds-test2.tex
%    TDS:doc/latex/oberdiek/test/pdftexcmds-test-shell.tex
%    TDS:doc/latex/oberdiek/test/pdftexcmds-test-escape.tex
%    TDS:source/latex/oberdiek/pdftexcmds.dtx
%
%<*ignore>
\begingroup
  \catcode123=1 %
  \catcode125=2 %
  \def\x{LaTeX2e}%
\expandafter\endgroup
\ifcase 0\ifx\install y1\fi\expandafter
         \ifx\csname processbatchFile\endcsname\relax\else1\fi
         \ifx\fmtname\x\else 1\fi\relax
\else\csname fi\endcsname
%</ignore>
%<*install>
\input docstrip.tex
\Msg{************************************************************************}
\Msg{* Installation}
\Msg{* Package: pdftexcmds 2019/07/25 v0.30 Utility functions of pdfTeX for LuaTeX (HO)}
\Msg{************************************************************************}

\keepsilent
\askforoverwritefalse

\let\MetaPrefix\relax
\preamble

This is a generated file.

Project: pdftexcmds
Version: 2019/07/25 v0.30

Copyright (C) 2007, 2009-2011 by
   Heiko Oberdiek <heiko.oberdiek at googlemail.com>

This work may be distributed and/or modified under the
conditions of the LaTeX Project Public License, either
version 1.3c of this license or (at your option) any later
version. This version of this license is in
   https://www.latex-project.org/lppl/lppl-1-3c.txt
and the latest version of this license is in
   https://www.latex-project.org/lppl.txt
and version 1.3 or later is part of all distributions of
LaTeX version 2005/12/01 or later.

This work has the LPPL maintenance status "maintained".

The Current Maintainers of this work are
Heiko Oberdiek and the Oberdiek Package Support Group
https://github.com/ho-tex/oberdiek/issues


The Base Interpreter refers to any `TeX-Format',
because some files are installed in TDS:tex/generic//.

This work consists of the main source file pdftexcmds.dtx
and the derived files
   pdftexcmds.sty, pdftexcmds.pdf, pdftexcmds.ins, pdftexcmds.drv,
   pdftexcmds.bib, pdftexcmds-test1.tex, pdftexcmds-test2.tex,
   pdftexcmds-test-shell.tex, pdftexcmds-test-escape.tex,
   oberdiek.pdftexcmds.lua, pdftexcmds.lua.

\endpreamble
\let\MetaPrefix\DoubleperCent

\generate{%
  \file{pdftexcmds.ins}{\from{pdftexcmds.dtx}{install}}%
  \file{pdftexcmds.drv}{\from{pdftexcmds.dtx}{driver}}%
  \nopreamble
  \nopostamble
  \file{pdftexcmds.bib}{\from{pdftexcmds.dtx}{bib}}%
  \usepreamble\defaultpreamble
  \usepostamble\defaultpostamble
  \usedir{tex/generic/oberdiek}%
  \file{pdftexcmds.sty}{\from{pdftexcmds.dtx}{package}}%
%  \usedir{doc/latex/oberdiek/test}%
%  \file{pdftexcmds-test1.tex}{\from{pdftexcmds.dtx}{test1}}%
%  \file{pdftexcmds-test2.tex}{\from{pdftexcmds.dtx}{test2}}%
%  \file{pdftexcmds-test-shell.tex}{\from{pdftexcmds.dtx}{test-shell}}%
%  \file{pdftexcmds-test-escape.tex}{\from{pdftexcmds.dtx}{test-escape}}%
  \nopreamble
  \nopostamble
%  \usedir{source/latex/oberdiek/catalogue}%
%  \file{pdftexcmds.xml}{\from{pdftexcmds.dtx}{catalogue}}%
}
\def\MetaPrefix{-- }
\def\defaultpostamble{%
  \MetaPrefix^^J%
  \MetaPrefix\space End of File `\outFileName'.%
}
\def\currentpostamble{\defaultpostamble}%
\generate{%
  \usedir{scripts/oberdiek}%
  \file{oberdiek.pdftexcmds.lua}{\from{pdftexcmds.dtx}{lua}}%
  \file{pdftexcmds.lua}{\from{pdftexcmds.dtx}{lua}}%
}

\catcode32=13\relax% active space
\let =\space%
\Msg{************************************************************************}
\Msg{*}
\Msg{* To finish the installation you have to move the following}
\Msg{* file into a directory searched by TeX:}
\Msg{*}
\Msg{*     pdftexcmds.sty}
\Msg{*}
\Msg{* And install the following script files:}
\Msg{*}
\Msg{*     oberdiek.pdftexcmds.lua, pdftexcmds.lua}
\Msg{*}
\Msg{* To produce the documentation run the file `pdftexcmds.drv'}
\Msg{* through LaTeX.}
\Msg{*}
\Msg{* Happy TeXing!}
\Msg{*}
\Msg{************************************************************************}

\endbatchfile
%</install>
%<*bib>
@online{AndyThomas:Analog,
  author={Thomas, Andy},
  title={Analog of {\texttt{\csname textbackslash\endcsname}pdfelapsedtime} for
      {\hologo{LuaTeX}} and {\hologo{XeTeX}}},
  url={http://tex.stackexchange.com/a/32531},
  urldate={2011-11-29},
}
%</bib>
%<*ignore>
\fi
%</ignore>
%<*driver>
\NeedsTeXFormat{LaTeX2e}
\ProvidesFile{pdftexcmds.drv}%
  [2019/07/25 v0.30 Utility functions of pdfTeX for LuaTeX (HO)]%
\documentclass{ltxdoc}
\usepackage{holtxdoc}[2011/11/22]
\usepackage{paralist}
\usepackage{csquotes}
\usepackage[
  backend=bibtex,
  bibencoding=ascii,
  alldates=iso8601,
]{biblatex}[2011/11/13]
\bibliography{oberdiek-source}
\bibliography{pdftexcmds}
\begin{document}
  \DocInput{pdftexcmds.dtx}%
\end{document}
%</driver>
% \fi
%
%
% \CharacterTable
%  {Upper-case    \A\B\C\D\E\F\G\H\I\J\K\L\M\N\O\P\Q\R\S\T\U\V\W\X\Y\Z
%   Lower-case    \a\b\c\d\e\f\g\h\i\j\k\l\m\n\o\p\q\r\s\t\u\v\w\x\y\z
%   Digits        \0\1\2\3\4\5\6\7\8\9
%   Exclamation   \!     Double quote  \"     Hash (number) \#
%   Dollar        \$     Percent       \%     Ampersand     \&
%   Acute accent  \'     Left paren    \(     Right paren   \)
%   Asterisk      \*     Plus          \+     Comma         \,
%   Minus         \-     Point         \.     Solidus       \/
%   Colon         \:     Semicolon     \;     Less than     \<
%   Equals        \=     Greater than  \>     Question mark \?
%   Commercial at \@     Left bracket  \[     Backslash     \\
%   Right bracket \]     Circumflex    \^     Underscore    \_
%   Grave accent  \`     Left brace    \{     Vertical bar  \|
%   Right brace   \}     Tilde         \~}
%
% \GetFileInfo{pdftexcmds.drv}
%
% \title{The \xpackage{pdftexcmds} package}
% \date{2019/07/25 v0.30}
% \author{Heiko Oberdiek\thanks
% {Please report any issues at \url{https://github.com/ho-tex/oberdiek/issues}}}
%
% \maketitle
%
% \begin{abstract}
% \hologo{LuaTeX} provides most of the commands of \hologo{pdfTeX} 1.40. However
% a number of utility functions are removed. This package tries to fill
% the gap and implements some of the missing primitive using Lua.
% \end{abstract}
%
% \tableofcontents
%
% \def\csi#1{\texttt{\textbackslash\textit{#1}}}
%
% \section{Documentation}
%
% Some primitives of \hologo{pdfTeX} \cite{pdftex-manual}
% are not defined by \hologo{LuaTeX} \cite{luatex-manual}.
% This package implements macro based solutions using Lua code
% for the following missing \hologo{pdfTeX} primitives;
% \begin{compactitem}
% \item \cs{pdfstrcmp}
% \item \cs{pdfunescapehex}
% \item \cs{pdfescapehex}
% \item \cs{pdfescapename}
% \item \cs{pdfescapestring}
% \item \cs{pdffilesize}
% \item \cs{pdffilemoddate}
% \item \cs{pdffiledump}
% \item \cs{pdfmdfivesum}
% \item \cs{pdfresettimer}
% \item \cs{pdfelapsedtime}
% \item |\immediate\write18|
% \end{compactitem}
% The original names of the primitives cannot be used:
% \begin{itemize}
% \item
% The syntax for their arguments cannot easily
% simulated by macros. The primitives using key words
% such as |file| (\cs{pdfmdfivesum}) or |offset| and |length|
% (\cs{pdffiledump}) and uses \meta{general text} for the other
% arguments. Using token registers assignments, \meta{general text} could
% be catched. However, the simulated primitives are expandable
% and register assignments would destroy this important property.
% (\meta{general text} allows something like |\expandafter\bgroup ...}|.)
% \item
% The original primitives can be expanded using one expansion step.
% The new macros need two expansion steps because of the additional
% macro expansion. Example:
% \begin{quote}
%   |\expandafter\foo\pdffilemoddate{file}|\\
%   vs.\\
%   |\expandafter\expandafter\expandafter|\\
%   |\foo\pdf@filemoddate{file}|
% \end{quote}
% \end{itemize}
%
% \hologo{LuaTeX} isn't stable yet and thus the status of this package is
% \emph{experimental}. Feedback is welcome.
%
% \subsection{General principles}
%
% \begin{description}
% \item[Naming convention:]
%   Usually this package defines a macro |\pdf@|\meta{cmd} if
%   \hologo{pdfTeX} provides |\pdf|\meta{cmd}.
% \item[Arguments:] The order of arguments in |\pdf@|\meta{cmd}
%   is the same as for the corresponding primitive of \hologo{pdfTeX}.
%   The arguments are ordinary undelimited \hologo{TeX} arguments,
%   no \meta{general text} and without additional keywords.
% \item[Expandibility:]
%   The macro |\pdf@|\meta{cmd} is expandable if the
%   corresponding \hologo{pdfTeX} primitive has this property.
%   Exact two expansion steps are necessary (first is the macro
%   expansion) except for \cs{pdf@primitive} and \cs{pdf@ifprimitive}.
%   The latter ones are not macros, but have the direct meaning of the
%   primitive.
% \item[Without \hologo{LuaTeX}:]
%   The macros |\pdf@|\meta{cmd} are mapped to the commands
%   of \hologo{pdfTeX} if they are available. Otherwise they are undefined.
% \item[Availability:]
%   The macros that the packages provides are undefined, if
%   the necessary primitives are not found and cannot be
%   implemented by Lua.
% \end{description}
%
% \subsection{Macros}
%
% \subsubsection[Strings]{Strings \cite[``7.15 Strings'']{pdftex-manual}}
%
% \begin{declcs}{pdf@strcmp} \M{stringA} \M{stringB}
% \end{declcs}
% Same as |\pdfstrcmp{|\meta{stringA}|}{|\meta{stringB}|}|.
%
% \begin{declcs}{pdf@unescapehex} \M{string}
% \end{declcs}
% Same as |\pdfunescapehex{|\meta{string}|}|.
% The argument is a byte string given in hexadecimal notation.
% The result are character tokens from 0 until 255 with
% catcode 12 and the space with catcode 10.
%
% \begin{declcs}{pdf@escapehex} \M{string}\\
%   \cs{pdf@escapestring} \M{string}\\
%   \cs{pdf@escapename} \M{string}
% \end{declcs}
% Same as the primitives of \hologo{pdfTeX}. However \hologo{pdfTeX} does not
% know about characters with codes 256 and larger. Thus the
% string is treated as byte string, characters with more than
% eight bits are ignored.
%
% \subsubsection[Files]{Files \cite[``7.18 Files'']{pdftex-manual}}
%
% \begin{declcs}{pdf@filesize} \M{filename}
% \end{declcs}
% Same as |\pdffilesize{|\meta{filename}|}|.
%
% \begin{declcs}{pdf@filemoddate} \M{filename}
% \end{declcs}
% Same as |\pdffilemoddate{|\meta{filename}|}|.
%
% \begin{declcs}{pdf@filedump} \M{offset} \M{length} \M{filename}
% \end{declcs}
% Same as |\pdffiledump offset| \meta{offset} |length| \meta{length}
% |{|\meta{filename}|}|. Both \meta{offset} and \meta{length} must
% not be empty, but must be a valid \hologo{TeX} number.
%
% \begin{declcs}{pdf@mdfivesum} \M{string}
% \end{declcs}
% Same as |\pdfmdfivesum{|\meta{string}|}|. Keyword |file| is supported
% by macro \cs{pdf@filemdfivesum}.
%
% \begin{declcs}{pdf@filemdfivesum} \M{filename}
% \end{declcs}
% Same as |\pdfmdfivesum file{|\meta{filename}|}|.
%
% \subsubsection[Timekeeping]{Timekeeping \cite[``7.17 Timekeeping'']{pdftex-manual}}
%
% The timekeeping macros are based on Andy Thomas' work \cite{AndyThomas:Analog}.
%
% \begin{declcs}{pdf@resettimer}
% \end{declcs}
% Same as \cs{pdfresettimer}, it resets the internal timer.
%
% \begin{declcs}{pdf@elapsedtime}
% \end{declcs}
% Same as \cs{pdfelapsedtime}. It behaves like a read-only integer.
% For printing purposes it can be prefixed by \cs{the} or \cs{number}.
% It measures the time in scaled seconds (seconds multiplied with 65536)
% since the latest call of \cs{pdf@resettimer} or start of
% program/package. The resolution, the shortest time interval that
% can be measured, depends on the program and system.
% \begin{itemize}
% \item \hologo{pdfTeX} with |gettimeofday|: $\ge$ 1/65536\,s
% \item \hologo{pdfTeX} with |ftime|: $\ge$ 1\,ms
% \item \hologo{pdfTeX} with |time|: $\ge$ 1\,s
% \item \hologo{LuaTeX}: $\ge$ 10\,ms\\
%  (|os.clock()| returns a float number with two decimal digits in
%  \hologo{LuaTeX} beta-0.70.1-2011061416 (rev 4277)).
% \end{itemize}
%
% \subsubsection[Miscellaneous]{Miscellaneous \cite[``7.21 Miscellaneous'']{pdftex-manual}}
%
% \begin{declcs}{pdf@draftmode}
% \end{declcs}
% If the \TeX\ compiler knows \cs{pdfdraftmode} or \cs{draftmode}
% (\hologo{pdfTeX},
% \hologo{LuaTeX}), then \cs{pdf@draftmode} returns, whether
% this mode is enabled. The result is an implicit number:
% one means the draft mode is available and enabled.
% If the value is zero, then the mode is not active or
% \cs{pdfdraftmode} is not available.
% An explicit number is yielded by \cs{number}\cs{pdf@draftmode}.
% The macro cannot
% be used to change the mode, see \cs{pdf@setdraftmode}.
%
% \begin{declcs}{pdf@ifdraftmode} \M{true} \M{false}
% \end{declcs}
% If \cs{pdfdraftmode} is available and enabled, \meta{true} is
% called, otherwise \meta{false} is executed.
%
% \begin{declcs}{pdf@setdraftmode} \M{value}
% \end{declcs}
% Macro \cs{pdf@setdraftmode} expects the number zero or one as
% \meta{value}. Zero deactivates the mode and one enables the draft mode.
% The macro does not have an effect, if the feature \cs{pdfdraftmode} is not
% available.
%
% \begin{declcs}{pdf@shellescape}
% \end{declcs}
% Same as |\pdfshellescape|. It is or expands to |1| if external
% commands can be executed and |0| otherwise. In \hologo{pdfTeX} external
% commands must be enabled first by command line option or
% configuration option. In \hologo{LuaTeX} option |--safer| disables
% the execution of external commands.
%
% In \hologo{LuaTeX} before 0.68.0 \cs{pdf@shellescape} is not
% available due to a bug in |os.execute()|. The argumentless form
% crashes in some circumstances with segmentation fault.
% (It is fixed in version 0.68.0 or revision 4167 of \hologo{LuaTeX}.
% and packported to some version of 0.67.0).
%
% Hints for usage:
% \begin{itemize}
% \item Before its use \cs{pdf@shellescape} should be tested,
% whether it is available. Example with package \xpackage{ltxcmds}
% (loaded by package \xpackage{pdftexcmds}):
%\begin{quote}
%\begin{verbatim}
%\ltx@IfUndefined{pdf@shellescape}{%
%  % \pdf@shellescape is undefined
%}{%
%  % \pdf@shellescape is available
%}
%\end{verbatim}
%\end{quote}
% Use \cs{ltx@ifundefined} in expandable contexts.
% \item \cs{pdf@shellescape} might be a numerical constant,
% expands to the primitive, or expands to a plain number.
% Therefore use it in contexts where these differences does not matter.
% \item Use in comparisons, e.g.:
%   \begin{quote}
%     |\ifnum\pdf@shellescape=0 ...|
%   \end{quote}
% \item Print the number: |\number\pdf@shellescape|
% \end{itemize}
%
% \begin{declcs}{pdf@system} \M{cmdline}
% \end{declcs}
% It is a wrapper for |\immediate\write18| in \hologo{pdfTeX} or
% |os.execute| in \hologo{LuaTeX}.
%
% In theory |os.execute|
% returns a status number. But its meaning is quite
% undefined. Are there some reliable properties?
% Does it make sense to provide an user interface to
% this status exit code?
%
% \begin{declcs}{pdf@primitive} \csi{cmd}
% \end{declcs}
% Same as \cs{pdfprimitive} in \hologo{pdfTeX} or \hologo{LuaTeX}.
% In \hologo{XeTeX} the
% primitive is called \cs{primitive}. Despite the current definition
% of the command \csi{cmd}, it's meaning as primitive is used.
%
% \begin{declcs}{pdf@ifprimitive} \csi{cmd}
% \end{declcs}
% Same as \cs{ifpdfprimitive} in \hologo{pdfTeX} or
% \hologo{LuaTeX}. \hologo{XeTeX} calls
% it \cs{ifprimitive}. It is a switch that checks if the command
% \csi{cmd} has it's primitive meaning.
%
% \subsubsection{Additional macro: \cs{pdf@isprimitive}}
%
% \begin{declcs}{pdf@isprimitive} \csi{cmd1} \csi{cmd2} \M{true} \M{false}
% \end{declcs}
% If \csi{cmd1} has the primitive meaning given by the primitive name
% of \csi{cmd2}, then the argument \meta{true} is executed, otherwise
% \meta{false}. The macro \cs{pdf@isprimitive} is expandable.
% Internally it checks the result of \cs{meaning} and is therefore
% available for all \hologo{TeX} variants, even the original \hologo{TeX}.
% Example with \hologo{LaTeX}:
%\begin{quote}
%\begin{verbatim}
%\makeatletter
%\pdf@isprimitive{@@input}{input}{%
%  \typeout{\string\@@input\space is original\string\input}%
%}{%
%  \typeout{Oops, \string\@@input\space is not the %
%           original\string\input}%
%}
%\end{verbatim}
%\end{quote}
%
% \subsubsection{Experimental}
%
% \begin{declcs}{pdf@unescapehexnative} \M{string}\\
%   \cs{pdf@escapehexnative} \M{string}\\
%   \cs{pdf@escapenamenative} \M{string}\\
%   \cs{pdf@mdfivesumnative} \M{string}
% \end{declcs}
% The variants without |native| in the macro name are supposed to
% be compatible with \hologo{pdfTeX}. However characters with more than
% eight bits are not supported and are ignored. If \hologo{LuaTeX} is
% running, then its UTF-8 coded strings are used. Thus the full
% unicode character range is supported. However the result
% differs from \hologo{pdfTeX} for characters with eight or more bits.
%
% \begin{declcs}{pdf@pipe} \M{cmdline}
% \end{declcs}
% It calls \meta{cmdline} and returns the output of the external
% program in the usual manner as byte string (catcode 12, space with
% catcode 10). The Lua documentation says, that the used |io.popen|
% may not be available on all platforms. Then macro \cs{pdf@pipe}
% is undefined.
%
% \StopEventually{
% }
%
% \section{Implementation}
%
%    \begin{macrocode}
%<*package>
%    \end{macrocode}
%
% \subsection{Reload check and package identification}
%    Reload check, especially if the package is not used with \LaTeX.
%    \begin{macrocode}
\begingroup\catcode61\catcode48\catcode32=10\relax%
  \catcode13=5 % ^^M
  \endlinechar=13 %
  \catcode35=6 % #
  \catcode39=12 % '
  \catcode44=12 % ,
  \catcode45=12 % -
  \catcode46=12 % .
  \catcode58=12 % :
  \catcode64=11 % @
  \catcode123=1 % {
  \catcode125=2 % }
  \expandafter\let\expandafter\x\csname ver@pdftexcmds.sty\endcsname
  \ifx\x\relax % plain-TeX, first loading
  \else
    \def\empty{}%
    \ifx\x\empty % LaTeX, first loading,
      % variable is initialized, but \ProvidesPackage not yet seen
    \else
      \expandafter\ifx\csname PackageInfo\endcsname\relax
        \def\x#1#2{%
          \immediate\write-1{Package #1 Info: #2.}%
        }%
      \else
        \def\x#1#2{\PackageInfo{#1}{#2, stopped}}%
      \fi
      \x{pdftexcmds}{The package is already loaded}%
      \aftergroup\endinput
    \fi
  \fi
\endgroup%
%    \end{macrocode}
%    Package identification:
%    \begin{macrocode}
\begingroup\catcode61\catcode48\catcode32=10\relax%
  \catcode13=5 % ^^M
  \endlinechar=13 %
  \catcode35=6 % #
  \catcode39=12 % '
  \catcode40=12 % (
  \catcode41=12 % )
  \catcode44=12 % ,
  \catcode45=12 % -
  \catcode46=12 % .
  \catcode47=12 % /
  \catcode58=12 % :
  \catcode64=11 % @
  \catcode91=12 % [
  \catcode93=12 % ]
  \catcode123=1 % {
  \catcode125=2 % }
  \expandafter\ifx\csname ProvidesPackage\endcsname\relax
    \def\x#1#2#3[#4]{\endgroup
      \immediate\write-1{Package: #3 #4}%
      \xdef#1{#4}%
    }%
  \else
    \def\x#1#2[#3]{\endgroup
      #2[{#3}]%
      \ifx#1\@undefined
        \xdef#1{#3}%
      \fi
      \ifx#1\relax
        \xdef#1{#3}%
      \fi
    }%
  \fi
\expandafter\x\csname ver@pdftexcmds.sty\endcsname
\ProvidesPackage{pdftexcmds}%
  [2019/07/25 v0.30 Utility functions of pdfTeX for LuaTeX (HO)]%
%    \end{macrocode}
%
% \subsection{Catcodes}
%
%    \begin{macrocode}
\begingroup\catcode61\catcode48\catcode32=10\relax%
  \catcode13=5 % ^^M
  \endlinechar=13 %
  \catcode123=1 % {
  \catcode125=2 % }
  \catcode64=11 % @
  \def\x{\endgroup
    \expandafter\edef\csname pdftexcmds@AtEnd\endcsname{%
      \endlinechar=\the\endlinechar\relax
      \catcode13=\the\catcode13\relax
      \catcode32=\the\catcode32\relax
      \catcode35=\the\catcode35\relax
      \catcode61=\the\catcode61\relax
      \catcode64=\the\catcode64\relax
      \catcode123=\the\catcode123\relax
      \catcode125=\the\catcode125\relax
    }%
  }%
\x\catcode61\catcode48\catcode32=10\relax%
\catcode13=5 % ^^M
\endlinechar=13 %
\catcode35=6 % #
\catcode64=11 % @
\catcode123=1 % {
\catcode125=2 % }
\def\TMP@EnsureCode#1#2{%
  \edef\pdftexcmds@AtEnd{%
    \pdftexcmds@AtEnd
    \catcode#1=\the\catcode#1\relax
  }%
  \catcode#1=#2\relax
}
\TMP@EnsureCode{0}{12}%
\TMP@EnsureCode{1}{12}%
\TMP@EnsureCode{2}{12}%
\TMP@EnsureCode{10}{12}% ^^J
\TMP@EnsureCode{33}{12}% !
\TMP@EnsureCode{34}{12}% "
\TMP@EnsureCode{38}{4}% &
\TMP@EnsureCode{39}{12}% '
\TMP@EnsureCode{40}{12}% (
\TMP@EnsureCode{41}{12}% )
\TMP@EnsureCode{42}{12}% *
\TMP@EnsureCode{43}{12}% +
\TMP@EnsureCode{44}{12}% ,
\TMP@EnsureCode{45}{12}% -
\TMP@EnsureCode{46}{12}% .
\TMP@EnsureCode{47}{12}% /
\TMP@EnsureCode{58}{12}% :
\TMP@EnsureCode{60}{12}% <
\TMP@EnsureCode{62}{12}% >
\TMP@EnsureCode{91}{12}% [
\TMP@EnsureCode{93}{12}% ]
\TMP@EnsureCode{94}{7}% ^ (superscript)
\TMP@EnsureCode{95}{12}% _ (other)
\TMP@EnsureCode{96}{12}% `
\TMP@EnsureCode{126}{12}% ~ (other)
\edef\pdftexcmds@AtEnd{%
  \pdftexcmds@AtEnd
  \escapechar=\number\escapechar\relax
  \noexpand\endinput
}
\escapechar=92 %
%    \end{macrocode}
%
% \subsection{Load packages}
%
%    \begin{macrocode}
\begingroup\expandafter\expandafter\expandafter\endgroup
\expandafter\ifx\csname RequirePackage\endcsname\relax
  \def\TMP@RequirePackage#1[#2]{%
    \begingroup\expandafter\expandafter\expandafter\endgroup
    \expandafter\ifx\csname ver@#1.sty\endcsname\relax
      \input #1.sty\relax
    \fi
  }%
  \TMP@RequirePackage{infwarerr}[2007/09/09]%
  \TMP@RequirePackage{ifluatex}[2010/03/01]%
  \TMP@RequirePackage{ltxcmds}[2010/12/02]%
  \TMP@RequirePackage{ifpdf}[2010/09/13]%
\else
  \RequirePackage{infwarerr}[2007/09/09]%
  \RequirePackage{ifluatex}[2010/03/01]%
  \RequirePackage{ltxcmds}[2010/12/02]%
  \RequirePackage{ifpdf}[2010/09/13]%
\fi
%    \end{macrocode}
%
% \subsection{Without \hologo{LuaTeX}}
%
%    \begin{macrocode}
\ifluatex
\else
  \@PackageInfoNoLine{pdftexcmds}{LuaTeX not detected}%
  \def\pdftexcmds@nopdftex{%
    \@PackageInfoNoLine{pdftexcmds}{pdfTeX >= 1.30 not detected}%
    \let\pdftexcmds@nopdftex\relax
  }%
  \def\pdftexcmds@temp#1{%
    \begingroup\expandafter\expandafter\expandafter\endgroup
    \expandafter\ifx\csname pdf#1\endcsname\relax
      \pdftexcmds@nopdftex
    \else
      \expandafter\def\csname pdf@#1\expandafter\endcsname
      \expandafter##\expandafter{%
        \csname pdf#1\endcsname
      }%
    \fi
  }%
  \pdftexcmds@temp{strcmp}%
  \pdftexcmds@temp{escapehex}%
  \let\pdf@escapehexnative\pdf@escapehex
  \pdftexcmds@temp{unescapehex}%
  \let\pdf@unescapehexnative\pdf@unescapehex
  \pdftexcmds@temp{escapestring}%
  \pdftexcmds@temp{escapename}%
  \pdftexcmds@temp{filesize}%
  \pdftexcmds@temp{filemoddate}%
  \begingroup\expandafter\expandafter\expandafter\endgroup
  \expandafter\ifx\csname pdfshellescape\endcsname\relax
    \pdftexcmds@nopdftex
    \ltx@IfUndefined{pdftexversion}{%
    }{%
      \ifnum\pdftexversion>120 % 1.21a supports \ifeof18
        \ifeof18 %
          \chardef\pdf@shellescape=0 %
        \else
          \chardef\pdf@shellescape=1 %
        \fi
      \fi
    }%
  \else
    \def\pdf@shellescape{%
      \pdfshellescape
    }%
  \fi
  \begingroup\expandafter\expandafter\expandafter\endgroup
  \expandafter\ifx\csname pdffiledump\endcsname\relax
    \pdftexcmds@nopdftex
  \else
    \def\pdf@filedump#1#2#3{%
      \pdffiledump offset#1 length#2{#3}%
    }%
  \fi
%    \end{macrocode}
%    \begin{macrocode}
  \begingroup\expandafter\expandafter\expandafter\endgroup
  \expandafter\ifx\csname pdfmdfivesum\endcsname\relax
    \begingroup\expandafter\expandafter\expandafter\endgroup
    \expandafter\ifx\csname mdfivesum\endcsname\relax
      \pdftexcmds@nopdftex
    \else
      \def\pdf@mdfivesum#{\mdfivesum}%
      \let\pdf@mdfivesumnative\pdf@mdfivesum
      \def\pdf@filemdfivesum#{\mdfivesum file}%
    \fi
  \else
    \def\pdf@mdfivesum#{\pdfmdfivesum}%
    \let\pdf@mdfivesumnative\pdf@mdfivesum
    \def\pdf@filemdfivesum#{\pdfmdfivesum file}%
  \fi
%    \end{macrocode}
%    \begin{macrocode}
  \def\pdf@system#{%
    \immediate\write18%
  }%
  \def\pdftexcmds@temp#1{%
    \begingroup\expandafter\expandafter\expandafter\endgroup
    \expandafter\ifx\csname pdf#1\endcsname\relax
      \pdftexcmds@nopdftex
    \else
      \expandafter\let\csname pdf@#1\expandafter\endcsname
      \csname pdf#1\endcsname
    \fi
  }%
  \pdftexcmds@temp{resettimer}%
  \pdftexcmds@temp{elapsedtime}%
\fi
%    \end{macrocode}
%
% \subsection{\cs{pdf@primitive}, \cs{pdf@ifprimitive}}
%
%    Since version 1.40.0 \hologo{pdfTeX} has \cs{pdfprimitive} and
%    \cs{ifpdfprimitive}. And \cs{pdfprimitive} was fixed in
%    version 1.40.4.
%
%    \hologo{XeTeX} provides them under the name \cs{primitive} and
%    \cs{ifprimitive}. \hologo{LuaTeX} knows both name variants,
%    but they have possibly to be enabled first (|tex.enableprimitives|).
%
%    Depending on the format TeX Live uses a prefix |luatex|.
%
%    Caution: \cs{let} must be used for the definition of
%    the macros, especially because of \cs{ifpdfprimitive}.
%
% \subsubsection{Using \hologo{LuaTeX}'s \texttt{tex.enableprimitives}}
%
%    \begin{macrocode}
\ifluatex
%    \end{macrocode}
%    \begin{macro}{\pdftexcmds@directlua}
%    \begin{macrocode}
  \ifnum\luatexversion<36 %
    \def\pdftexcmds@directlua{\directlua0 }%
  \else
    \let\pdftexcmds@directlua\directlua
  \fi
%    \end{macrocode}
%    \end{macro}
%
%    \begin{macrocode}
  \begingroup
    \newlinechar=10 %
    \endlinechar=\newlinechar
    \pdftexcmds@directlua{%
      if tex.enableprimitives then
        tex.enableprimitives(
          'pdf@',
          {'primitive', 'ifprimitive', 'pdfdraftmode','draftmode'}
        )
        tex.enableprimitives('', {'luaescapestring'})
      end
    }%
  \endgroup %
%    \end{macrocode}
%
%    \begin{macrocode}
\fi
%    \end{macrocode}
%
% \subsubsection{Trying various names to find the primitives}
%
%    \begin{macro}{\pdftexcmds@strip@prefix}
%    \begin{macrocode}
\def\pdftexcmds@strip@prefix#1>{}
%    \end{macrocode}
%    \end{macro}
%    \begin{macrocode}
\def\pdftexcmds@temp#1#2#3{%
  \begingroup\expandafter\expandafter\expandafter\endgroup
  \expandafter\ifx\csname pdf@#1\endcsname\relax
    \begingroup
      \def\x{#3}%
      \edef\x{\expandafter\pdftexcmds@strip@prefix\meaning\x}%
      \escapechar=-1 %
      \edef\y{\expandafter\meaning\csname#2\endcsname}%
    \expandafter\endgroup
    \ifx\x\y
      \expandafter\let\csname pdf@#1\expandafter\endcsname
      \csname #2\endcsname
    \fi
  \fi
}
%    \end{macrocode}
%
%    \begin{macro}{\pdf@primitive}
%    \begin{macrocode}
\pdftexcmds@temp{primitive}{pdfprimitive}{pdfprimitive}% pdfTeX, oldLuaTeX
\pdftexcmds@temp{primitive}{primitive}{primitive}% XeTeX, luatex
\pdftexcmds@temp{primitive}{luatexprimitive}{pdfprimitive}% oldLuaTeX
\pdftexcmds@temp{primitive}{luatexpdfprimitive}{pdfprimitive}% oldLuaTeX
%    \end{macrocode}
%    \end{macro}
%    \begin{macro}{\pdf@ifprimitive}
%    \begin{macrocode}
\pdftexcmds@temp{ifprimitive}{ifpdfprimitive}{ifpdfprimitive}% pdfTeX, oldLuaTeX
\pdftexcmds@temp{ifprimitive}{ifprimitive}{ifprimitive}% XeTeX, luatex
\pdftexcmds@temp{ifprimitive}{luatexifprimitive}{ifpdfprimitive}% oldLuaTeX
\pdftexcmds@temp{ifprimitive}{luatexifpdfprimitive}{ifpdfprimitive}% oldLuaTeX
%    \end{macrocode}
%    \end{macro}
%
%    Disable broken \cs{pdfprimitive}.
%    \begin{macrocode}
\ifluatex\else
\begingroup
  \expandafter\ifx\csname pdf@primitive\endcsname\relax
  \else
    \expandafter\ifx\csname pdftexversion\endcsname\relax
    \else
      \ifnum\pdftexversion=140 %
        \expandafter\ifx\csname pdftexrevision\endcsname\relax
        \else
          \ifnum\pdftexrevision<4 %
            \endgroup
            \let\pdf@primitive\@undefined
            \@PackageInfoNoLine{pdftexcmds}{%
              \string\pdf@primitive\space disabled, %
              because\MessageBreak
              \string\pdfprimitive\space is broken until pdfTeX 1.40.4%
            }%
            \begingroup
          \fi
        \fi
      \fi
    \fi
  \fi
\endgroup
\fi
%    \end{macrocode}
%
% \subsubsection{Result}
%
%    \begin{macrocode}
\begingroup
  \@PackageInfoNoLine{pdftexcmds}{%
    \string\pdf@primitive\space is %
    \expandafter\ifx\csname pdf@primitive\endcsname\relax not \fi
    available%
  }%
  \@PackageInfoNoLine{pdftexcmds}{%
    \string\pdf@ifprimitive\space is %
    \expandafter\ifx\csname pdf@ifprimitive\endcsname\relax not \fi
    available%
  }%
\endgroup
%    \end{macrocode}
%
% \subsection{\hologo{XeTeX}}
%
%    Look for primitives \cs{shellescape}, \cs{strcmp}.
%    \begin{macrocode}
\def\pdftexcmds@temp#1{%
  \begingroup\expandafter\expandafter\expandafter\endgroup
  \expandafter\ifx\csname pdf@#1\endcsname\relax
    \begingroup
      \escapechar=-1 %
      \edef\x{\expandafter\meaning\csname#1\endcsname}%
      \def\y{#1}%
      \def\z##1->{}%
      \edef\y{\expandafter\z\meaning\y}%
    \expandafter\endgroup
    \ifx\x\y
      \expandafter\def\csname pdf@#1\expandafter\endcsname
      \expandafter{%
        \csname#1\endcsname
      }%
    \fi
  \fi
}%
\pdftexcmds@temp{shellescape}%
\pdftexcmds@temp{strcmp}%
%    \end{macrocode}
%
% \subsection{\cs{pdf@isprimitive}}
%
%    \begin{macrocode}
\def\pdf@isprimitive{%
  \begingroup\expandafter\expandafter\expandafter\endgroup
  \expandafter\ifx\csname pdf@strcmp\endcsname\relax
    \long\def\pdf@isprimitive##1{%
      \expandafter\pdftexcmds@isprimitive\expandafter{\meaning##1}%
    }%
    \long\def\pdftexcmds@isprimitive##1##2{%
      \expandafter\pdftexcmds@@isprimitive\expandafter{\string##2}{##1}%
    }%
    \def\pdftexcmds@@isprimitive##1##2{%
      \ifnum0\pdftexcmds@equal##1\delimiter##2\delimiter=1 %
        \expandafter\ltx@firstoftwo
      \else
        \expandafter\ltx@secondoftwo
      \fi
    }%
    \def\pdftexcmds@equal##1##2\delimiter##3##4\delimiter{%
      \ifx##1##3%
        \ifx\relax##2##4\relax
          1%
        \else
          \ifx\relax##2\relax
          \else
            \ifx\relax##4\relax
            \else
              \pdftexcmds@equalcont{##2}{##4}%
            \fi
          \fi
        \fi
      \fi
    }%
    \def\pdftexcmds@equalcont##1{%
      \def\pdftexcmds@equalcont####1####2##1##1##1##1{%
        ##1##1##1##1%
        \pdftexcmds@equal####1\delimiter####2\delimiter
      }%
    }%
    \expandafter\pdftexcmds@equalcont\csname fi\endcsname
  \else
    \long\def\pdf@isprimitive##1##2{%
      \ifnum\pdf@strcmp{\meaning##1}{\string##2}=0 %
        \expandafter\ltx@firstoftwo
      \else
        \expandafter\ltx@secondoftwo
      \fi
    }%
  \fi
}
\ifluatex
\ifx\pdfdraftmode\@undefined
  \let\pdfdraftmode\draftmode
\fi
\else
  \pdf@isprimitive
\fi
%    \end{macrocode}
%
% \subsection{\cs{pdf@draftmode}}
%
%
%    \begin{macrocode}
\let\pdftexcmds@temp\ltx@zero %
\ltx@IfUndefined{pdfdraftmode}{%
  \@PackageInfoNoLine{pdftexcmds}{\ltx@backslashchar pdfdraftmode not found}%
}{%
  \ifpdf
    \let\pdftexcmds@temp\ltx@one
    \@PackageInfoNoLine{pdftexcmds}{\ltx@backslashchar pdfdraftmode found}%
  \else
    \@PackageInfoNoLine{pdftexcmds}{%
      \ltx@backslashchar pdfdraftmode is ignored in DVI mode%
    }%
  \fi
}
\ifcase\pdftexcmds@temp
%    \end{macrocode}
%    \begin{macro}{\pdf@draftmode}
%    \begin{macrocode}
  \let\pdf@draftmode\ltx@zero
%    \end{macrocode}
%    \end{macro}
%    \begin{macro}{\pdf@ifdraftmode}
%    \begin{macrocode}
  \let\pdf@ifdraftmode\ltx@secondoftwo
%    \end{macrocode}
%    \end{macro}
%    \begin{macro}{\pdftexcmds@setdraftmode}
%    \begin{macrocode}
  \def\pdftexcmds@setdraftmode#1{}%
%    \end{macrocode}
%    \end{macro}
%    \begin{macrocode}
\else
%    \end{macrocode}
%    \begin{macro}{\pdftexcmds@draftmode}
%    \begin{macrocode}
  \let\pdftexcmds@draftmode\pdfdraftmode
%    \end{macrocode}
%    \end{macro}
%    \begin{macro}{\pdf@ifdraftmode}
%    \begin{macrocode}
  \def\pdf@ifdraftmode{%
    \ifnum\pdftexcmds@draftmode=\ltx@one
      \expandafter\ltx@firstoftwo
    \else
      \expandafter\ltx@secondoftwo
    \fi
  }%
%    \end{macrocode}
%    \end{macro}
%    \begin{macro}{\pdf@draftmode}
%    \begin{macrocode}
  \def\pdf@draftmode{%
    \ifnum\pdftexcmds@draftmode=\ltx@one
      \expandafter\ltx@one
    \else
      \expandafter\ltx@zero
    \fi
  }%
%    \end{macrocode}
%    \end{macro}
%    \begin{macro}{\pdftexcmds@setdraftmode}
%    \begin{macrocode}
  \def\pdftexcmds@setdraftmode#1{%
    \pdftexcmds@draftmode=#1\relax
  }%
%    \end{macrocode}
%    \end{macro}
%    \begin{macrocode}
\fi
%    \end{macrocode}
%    \begin{macro}{\pdf@setdraftmode}
%    \begin{macrocode}
\def\pdf@setdraftmode#1{%
  \begingroup
    \count\ltx@cclv=#1\relax
  \edef\x{\endgroup
    \noexpand\pdftexcmds@@setdraftmode{\the\count\ltx@cclv}%
  }%
  \x
}
%    \end{macrocode}
%    \end{macro}
%    \begin{macro}{\pdftexcmds@@setdraftmode}
%    \begin{macrocode}
\def\pdftexcmds@@setdraftmode#1{%
  \ifcase#1 %
    \pdftexcmds@setdraftmode{#1}%
  \or
    \pdftexcmds@setdraftmode{#1}%
  \else
    \@PackageWarning{pdftexcmds}{%
      \string\pdf@setdraftmode: Ignoring\MessageBreak
      invalid value `#1'%
    }%
  \fi
}
%    \end{macrocode}
%    \end{macro}
%
% \subsection{Load Lua module}
%
%    \begin{macrocode}
\ifluatex
\else
  \expandafter\pdftexcmds@AtEnd
\fi%
%    \end{macrocode}
%
%    \begin{macrocode}
\ifnum\luatexversion<80
  \begingroup\expandafter\expandafter\expandafter\endgroup
  \expandafter\ifx\csname RequirePackage\endcsname\relax
    \def\TMP@RequirePackage#1[#2]{%
      \begingroup\expandafter\expandafter\expandafter\endgroup
      \expandafter\ifx\csname ver@#1.sty\endcsname\relax
        \input #1.sty\relax
      \fi
    }%
    \TMP@RequirePackage{luatex-loader}[2009/04/10]%
  \else
    \RequirePackage{luatex-loader}[2009/04/10]%
  \fi
\fi
\pdftexcmds@directlua{%
  require("pdftexcmds")%
}
\ifnum\luatexversion>37 %
  \ifnum0%
      \pdftexcmds@directlua{%
        if status.ini_version then %
          tex.write("1")%
        end%
      }>0 %
    \everyjob\expandafter{%
      \the\everyjob
      \pdftexcmds@directlua{%
        require("pdftexcmds")%
      }%
    }%
  \fi
\fi
\begingroup
  \def\x{2019/07/25 v0.30}%
  \ltx@onelevel@sanitize\x
  \edef\y{%
    \pdftexcmds@directlua{%
      if oberdiek.pdftexcmds.getversion then %
        oberdiek.pdftexcmds.getversion()%
      end%
    }%
  }%
  \ifx\x\y
  \else
    \@PackageError{pdftexcmds}{%
      Wrong version of lua module.\MessageBreak
      Package version: \x\MessageBreak
      Lua module: \y
    }\@ehc
  \fi
\endgroup
%    \end{macrocode}
%
% \subsection{Lua functions}
%
% \subsubsection{Helper macros}
%
%    \begin{macro}{\pdftexcmds@toks}
%    \begin{macrocode}
\begingroup\expandafter\expandafter\expandafter\endgroup
\expandafter\ifx\csname newtoks\endcsname\relax
  \toksdef\pdftexcmds@toks=0 %
\else
  \csname newtoks\endcsname\pdftexcmds@toks
\fi
%    \end{macrocode}
%    \end{macro}
%
%    \begin{macro}{\pdftexcmds@Patch}
%    \begin{macrocode}
\def\pdftexcmds@Patch{0}
\ifnum\luatexversion>40 %
  \ifnum\luatexversion<66 %
    \def\pdftexcmds@Patch{1}%
  \fi
\fi
%    \end{macrocode}
%    \end{macro}
%    \begin{macrocode}
\ifcase\pdftexcmds@Patch
  \catcode`\&=14 %
\else
  \catcode`\&=9 %
%    \end{macrocode}
%    \begin{macro}{\pdftexcmds@PatchDecode}
%    \begin{macrocode}
  \def\pdftexcmds@PatchDecode#1\@nil{%
    \pdftexcmds@DecodeA#1^^A^^A\@nil{}%
  }%
%    \end{macrocode}
%    \end{macro}
%    \begin{macro}{\pdftexcmds@DecodeA}
%    \begin{macrocode}
  \def\pdftexcmds@DecodeA#1^^A^^A#2\@nil#3{%
    \ifx\relax#2\relax
      \ltx@ReturnAfterElseFi{%
        \pdftexcmds@DecodeB#3#1^^A^^B\@nil{}%
      }%
    \else
      \ltx@ReturnAfterFi{%
        \pdftexcmds@DecodeA#2\@nil{#3#1^^@}%
      }%
    \fi
  }%
%    \end{macrocode}
%    \end{macro}
%    \begin{macro}{\pdftexcmds@DecodeB}
%    \begin{macrocode}
  \def\pdftexcmds@DecodeB#1^^A^^B#2\@nil#3{%
    \ifx\relax#2\relax%
      \ltx@ReturnAfterElseFi{%
        \ltx@zero
        #3#1%
      }%
    \else
      \ltx@ReturnAfterFi{%
        \pdftexcmds@DecodeB#2\@nil{#3#1^^A}%
      }%
    \fi
  }%
%    \end{macrocode}
%    \end{macro}
%    \begin{macrocode}
\fi
%    \end{macrocode}
%
%    \begin{macrocode}
\ifnum\luatexversion<36 %
\else
  \catcode`\0=9 %
\fi
%    \end{macrocode}
%
% \subsubsection[Strings]{Strings \cite[``7.15 Strings'']{pdftex-manual}}
%
%    \begin{macro}{\pdf@strcmp}
%    \begin{macrocode}
\long\def\pdf@strcmp#1#2{%
  \directlua0{%
    oberdiek.pdftexcmds.strcmp("\luaescapestring{#1}",%
        "\luaescapestring{#2}")%
  }%
}%
%    \end{macrocode}
%    \end{macro}
%    \begin{macrocode}
\pdf@isprimitive
%    \end{macrocode}
%    \begin{macro}{\pdf@escapehex}
%    \begin{macrocode}
\long\def\pdf@escapehex#1{%
  \directlua0{%
    oberdiek.pdftexcmds.escapehex("\luaescapestring{#1}", "byte")%
  }%
}%
%    \end{macrocode}
%    \end{macro}
%    \begin{macro}{\pdf@escapehexnative}
%    \begin{macrocode}
\long\def\pdf@escapehexnative#1{%
  \directlua0{%
    oberdiek.pdftexcmds.escapehex("\luaescapestring{#1}")%
  }%
}%
%    \end{macrocode}
%    \end{macro}
%    \begin{macro}{\pdf@unescapehex}
%    \begin{macrocode}
\def\pdf@unescapehex#1{%
& \romannumeral\expandafter\pdftexcmds@PatchDecode
  \the\expandafter\pdftexcmds@toks
  \directlua0{%
    oberdiek.pdftexcmds.toks="pdftexcmds@toks"%
    oberdiek.pdftexcmds.unescapehex("\luaescapestring{#1}", "byte", \pdftexcmds@Patch)%
  }%
& \@nil
}%
%    \end{macrocode}
%    \end{macro}
%    \begin{macro}{\pdf@unescapehexnative}
%    \begin{macrocode}
\def\pdf@unescapehexnative#1{%
& \romannumeral\expandafter\pdftexcmds@PatchDecode
  \the\expandafter\pdftexcmds@toks
  \directlua0{%
    oberdiek.pdftexcmds.toks="pdftexcmds@toks"%
    oberdiek.pdftexcmds.unescapehex("\luaescapestring{#1}", \pdftexcmds@Patch)%
  }%
& \@nil
}%
%    \end{macrocode}
%    \end{macro}
%    \begin{macro}{\pdf@escapestring}
%    \begin{macrocode}
\long\def\pdf@escapestring#1{%
  \directlua0{%
    oberdiek.pdftexcmds.escapestring("\luaescapestring{#1}", "byte")%
  }%
}
%    \end{macrocode}
%    \end{macro}
%    \begin{macro}{\pdf@escapename}
%    \begin{macrocode}
\long\def\pdf@escapename#1{%
  \directlua0{%
    oberdiek.pdftexcmds.escapename("\luaescapestring{#1}", "byte")%
  }%
}
%    \end{macrocode}
%    \end{macro}
%    \begin{macro}{\pdf@escapenamenative}
%    \begin{macrocode}
\long\def\pdf@escapenamenative#1{%
  \directlua0{%
    oberdiek.pdftexcmds.escapename("\luaescapestring{#1}")%
  }%
}
%    \end{macrocode}
%    \end{macro}
%
% \subsubsection[Files]{Files \cite[``7.18 Files'']{pdftex-manual}}
%
%    \begin{macro}{\pdf@filesize}
%    \begin{macrocode}
\def\pdf@filesize#1{%
  \directlua0{%
    oberdiek.pdftexcmds.filesize("\luaescapestring{#1}")%
  }%
}
%    \end{macrocode}
%    \end{macro}
%    \begin{macro}{\pdf@filemoddate}
%    \begin{macrocode}
\def\pdf@filemoddate#1{%
  \directlua0{%
    oberdiek.pdftexcmds.filemoddate("\luaescapestring{#1}")%
  }%
}
%    \end{macrocode}
%    \end{macro}
%    \begin{macro}{\pdf@filedump}
%    \begin{macrocode}
\def\pdf@filedump#1#2#3{%
  \directlua0{%
    oberdiek.pdftexcmds.filedump("\luaescapestring{\number#1}",%
        "\luaescapestring{\number#2}",%
        "\luaescapestring{#3}")%
  }%
}%
%    \end{macrocode}
%    \end{macro}
%    \begin{macro}{\pdf@mdfivesum}
%    \begin{macrocode}
\long\def\pdf@mdfivesum#1{%
  \directlua0{%
    oberdiek.pdftexcmds.mdfivesum("\luaescapestring{#1}", "byte")%
  }%
}%
%    \end{macrocode}
%    \end{macro}
%    \begin{macro}{\pdf@mdfivesumnative}
%    \begin{macrocode}
\long\def\pdf@mdfivesumnative#1{%
  \directlua0{%
    oberdiek.pdftexcmds.mdfivesum("\luaescapestring{#1}")%
  }%
}%
%    \end{macrocode}
%    \end{macro}
%    \begin{macro}{\pdf@filemdfivesum}
%    \begin{macrocode}
\def\pdf@filemdfivesum#1{%
  \directlua0{%
    oberdiek.pdftexcmds.filemdfivesum("\luaescapestring{#1}")%
  }%
}%
%    \end{macrocode}
%    \end{macro}
%
% \subsubsection[Timekeeping]{Timekeeping \cite[``7.17 Timekeeping'']{pdftex-manual}}
%
%    \begin{macro}{\protected}
%    \begin{macrocode}
\let\pdftexcmds@temp=Y%
\begingroup\expandafter\expandafter\expandafter\endgroup
\expandafter\ifx\csname protected\endcsname\relax
  \pdftexcmds@directlua0{%
    if tex.enableprimitives then %
      tex.enableprimitives('', {'protected'})%
    end%
  }%
\fi
\begingroup\expandafter\expandafter\expandafter\endgroup
\expandafter\ifx\csname protected\endcsname\relax
  \let\pdftexcmds@temp=N%
\fi
%    \end{macrocode}
%    \end{macro}
%    \begin{macro}{\numexpr}
%    \begin{macrocode}
\begingroup\expandafter\expandafter\expandafter\endgroup
\expandafter\ifx\csname numexpr\endcsname\relax
  \pdftexcmds@directlua0{%
    if tex.enableprimitives then %
      tex.enableprimitives('', {'numexpr'})%
    end%
  }%
\fi
\begingroup\expandafter\expandafter\expandafter\endgroup
\expandafter\ifx\csname numexpr\endcsname\relax
  \let\pdftexcmds@temp=N%
\fi
%    \end{macrocode}
%    \end{macro}
%
%    \begin{macrocode}
\ifx\pdftexcmds@temp N%
  \@PackageWarningNoLine{pdftexcmds}{%
    Definitions of \ltx@backslashchar pdf@resettimer and%
    \MessageBreak
    \ltx@backslashchar pdf@elapsedtime are skipped, because%
    \MessageBreak
    e-TeX's \ltx@backslashchar protected or %
    \ltx@backslashchar numexpr are missing%
  }%
\else
%    \end{macrocode}
%
%    \begin{macro}{\pdf@resettimer}
%    \begin{macrocode}
  \protected\def\pdf@resettimer{%
    \pdftexcmds@directlua0{%
      oberdiek.pdftexcmds.resettimer()%
    }%
  }%
%    \end{macrocode}
%    \end{macro}
%
%    \begin{macro}{\pdf@elapsedtime}
%    \begin{macrocode}
  \protected\def\pdf@elapsedtime{%
    \numexpr
      \pdftexcmds@directlua0{%
        oberdiek.pdftexcmds.elapsedtime()%
      }%
    \relax
  }%
%    \end{macrocode}
%    \end{macro}
%    \begin{macrocode}
\fi
%    \end{macrocode}
%
% \subsubsection{Shell escape}
%
%    \begin{macro}{\pdf@shellescape}
%
%    \begin{macrocode}
\ifnum\luatexversion<68 %
\else
  \protected\edef\pdf@shellescape{%
   \numexpr\directlua{tex.sprint(%
         \number\catcodetable@string,status.shell_escape)}\relax}
\fi
%    \end{macrocode}
%    \end{macro}
%
%    \begin{macro}{\pdf@system}
%    \begin{macrocode}
\def\pdf@system#1{%
  \directlua0{%
    oberdiek.pdftexcmds.system("\luaescapestring{#1}")%
  }%
}
%    \end{macrocode}
%    \end{macro}
%
%    \begin{macro}{\pdf@lastsystemstatus}
%    \begin{macrocode}
\def\pdf@lastsystemstatus{%
  \directlua0{%
    oberdiek.pdftexcmds.lastsystemstatus()%
  }%
}
%    \end{macrocode}
%    \end{macro}
%    \begin{macro}{\pdf@lastsystemexit}
%    \begin{macrocode}
\def\pdf@lastsystemexit{%
  \directlua0{%
    oberdiek.pdftexcmds.lastsystemexit()%
  }%
}
%    \end{macrocode}
%    \end{macro}
%
%    \begin{macrocode}
\catcode`\0=12 %
%    \end{macrocode}
%
%    \begin{macro}{\pdf@pipe}
%    Check availability of |io.popen| first.
%    \begin{macrocode}
\ifnum0%
    \pdftexcmds@directlua{%
      if io.popen then %
        tex.write("1")%
      end%
    }%
    =1 %
  \def\pdf@pipe#1{%
&   \romannumeral\expandafter\pdftexcmds@PatchDecode
    \the\expandafter\pdftexcmds@toks
    \pdftexcmds@directlua{%
      oberdiek.pdftexcmds.toks="pdftexcmds@toks"%
      oberdiek.pdftexcmds.pipe("\luaescapestring{#1}", \pdftexcmds@Patch)%
    }%
&   \@nil
  }%
\fi
%    \end{macrocode}
%    \end{macro}
%
%    \begin{macrocode}
\pdftexcmds@AtEnd%
%</package>
%    \end{macrocode}
%
% \subsection{Lua module}
%
%    \begin{macrocode}
%<*lua>
%    \end{macrocode}
%
%    \begin{macrocode}
oberdiek = oberdiek or {}
local pdftexcmds = oberdiek.pdftexcmds or {}
oberdiek.pdftexcmds = pdftexcmds
local systemexitstatus
function pdftexcmds.getversion()
  tex.write("2019/07/25 v0.30")
end
%    \end{macrocode}
%
% \subsubsection[Strings]{Strings \cite[``7.15 Strings'']{pdftex-manual}}
%
%    \begin{macrocode}
function pdftexcmds.strcmp(A, B)
  if A == B then
    tex.write("0")
  elseif A < B then
    tex.write("-1")
  else
    tex.write("1")
  end
end
local function utf8_to_byte(str)
  local i = 0
  local n = string.len(str)
  local t = {}
  while i < n do
    i = i + 1
    local a = string.byte(str, i)
    if a < 128 then
      table.insert(t, string.char(a))
    else
      if a >= 192 and i < n then
        i = i + 1
        local b = string.byte(str, i)
        if b < 128 or b >= 192 then
          i = i - 1
        elseif a == 194 then
          table.insert(t, string.char(b))
        elseif a == 195 then
          table.insert(t, string.char(b + 64))
        end
      end
    end
  end
  return table.concat(t)
end
function pdftexcmds.escapehex(str, mode)
  if mode == "byte" then
    str = utf8_to_byte(str)
  end
  tex.write((string.gsub(str, ".",
    function (ch)
      return string.format("%02X", string.byte(ch))
    end
  )))
end
%    \end{macrocode}
%    See procedure |unescapehex| in file \xfile{utils.c} of \hologo{pdfTeX}.
%    Caution: |tex.write| ignores leading spaces.
%    \begin{macrocode}
function pdftexcmds.unescapehex(str, mode, patch)
  local a = 0
  local first = true
  local result = {}
  for i = 1, string.len(str), 1 do
    local ch = string.byte(str, i)
    if ch >= 48 and ch <= 57 then
      ch = ch - 48
    elseif ch >= 65 and ch <= 70 then
      ch = ch - 55
    elseif ch >= 97 and ch <= 102 then
      ch = ch - 87
    else
      ch = nil
    end
    if ch then
      if first then
        a = ch * 16
        first = false
      else
        table.insert(result, a + ch)
        first = true
      end
    end
  end
  if not first then
    table.insert(result, a)
  end
  if patch == 1 then
    local temp = {}
    for i, a in ipairs(result) do
      if a == 0 then
        table.insert(temp, 1)
        table.insert(temp, 1)
      else
        if a == 1 then
          table.insert(temp, 1)
          table.insert(temp, 2)
        else
          table.insert(temp, a)
        end
      end
    end
    result = temp
  end
  if mode == "byte" then
    local utf8 = {}
    for i, a in ipairs(result) do
      if a < 128 then
        table.insert(utf8, a)
      else
        if a < 192 then
          table.insert(utf8, 194)
          a = a - 128
        else
          table.insert(utf8, 195)
          a = a - 192
        end
        table.insert(utf8, a + 128)
      end
    end
    result = utf8
  end
%    \end{macrocode}
%    this next line added for current luatex; this is the only
%    change in the file.  eroux, 28apr13. (v 0.21)
%    \begin{macrocode}
  local unpack = _G["unpack"] or table.unpack
  tex.settoks(pdftexcmds.toks, string.char(unpack(result)))
end
%    \end{macrocode}
%    See procedure |escapestring| in file \xfile{utils.c} of \hologo{pdfTeX}.
%    \begin{macrocode}
function pdftexcmds.escapestring(str, mode)
  if mode == "byte" then
    str = utf8_to_byte(str)
  end
  tex.write((string.gsub(str, ".",
    function (ch)
      local b = string.byte(ch)
      if b < 33 or b > 126 then
        return string.format("\\%.3o", b)
      end
      if b == 40 or b == 41 or b == 92 then
        return "\\" .. ch
      end
%    \end{macrocode}
%    Lua 5.1 returns the match in case of return value |nil|.
%    \begin{macrocode}
      return nil
    end
  )))
end
%    \end{macrocode}
%    See procedure |escapename| in file \xfile{utils.c} of \hologo{pdfTeX}.
%    \begin{macrocode}
function pdftexcmds.escapename(str, mode)
  if mode == "byte" then
    str = utf8_to_byte(str)
  end
  tex.write((string.gsub(str, ".",
    function (ch)
      local b = string.byte(ch)
      if b == 0 then
%    \end{macrocode}
%    In Lua 5.0 |nil| could be used for the empty string,
%    But |nil| returns the match in Lua 5.1, thus we use
%    the empty string explicitly.
%    \begin{macrocode}
        return ""
      end
      if b <= 32 or b >= 127
          or b == 35 or b == 37 or b == 40 or b == 41
          or b == 47 or b == 60 or b == 62 or b == 91
          or b == 93 or b == 123 or b == 125 then
        return string.format("#%.2X", b)
      else
%    \end{macrocode}
%    Lua 5.1 returns the match in case of return value |nil|.
%    \begin{macrocode}
        return nil
      end
    end
  )))
end
%    \end{macrocode}
%
% \subsubsection[Files]{Files \cite[``7.18 Files'']{pdftex-manual}}
%
%    \begin{macrocode}
function pdftexcmds.filesize(filename)
  local foundfile = kpse.find_file(filename, "tex", true)
  if foundfile then
    local size = lfs.attributes(foundfile, "size")
    if size then
      tex.write(size)
    end
  end
end
%    \end{macrocode}
%    See procedure |makepdftime| in file \xfile{utils.c} of \hologo{pdfTeX}.
%    \begin{macrocode}
function pdftexcmds.filemoddate(filename)
  local foundfile = kpse.find_file(filename, "tex", true)
  if foundfile then
    local date = lfs.attributes(foundfile, "modification")
    if date then
      local d = os.date("*t", date)
      if d.sec >= 60 then
        d.sec = 59
      end
      local u = os.date("!*t", date)
      local off = 60 * (d.hour - u.hour) + d.min - u.min
      if d.year ~= u.year then
        if d.year > u.year then
          off = off + 1440
        else
          off = off - 1440
        end
      elseif d.yday ~= u.yday then
        if d.yday > u.yday then
          off = off + 1440
        else
          off = off - 1440
        end
      end
      local timezone
      if off == 0 then
        timezone = "Z"
      else
        local hours = math.floor(off / 60)
        local mins = math.abs(off - hours * 60)
        timezone = string.format("%+03d'%02d'", hours, mins)
      end
      tex.write(string.format("D:%04d%02d%02d%02d%02d%02d%s",
          d.year, d.month, d.day, d.hour, d.min, d.sec, timezone))
    end
  end
end
function pdftexcmds.filedump(offset, length, filename)
  length = tonumber(length)
  if length and length > 0 then
    local foundfile = kpse.find_file(filename, "tex", true)
    if foundfile then
      offset = tonumber(offset)
      if not offset then
        offset = 0
      end
      local filehandle = io.open(foundfile, "rb")
      if filehandle then
        if offset > 0 then
          filehandle:seek("set", offset)
        end
        local dump = filehandle:read(length)
        pdftexcmds.escapehex(dump)
        filehandle:close()
      end
    end
  end
end
function pdftexcmds.mdfivesum(str, mode)
  if mode == "byte" then
    str = utf8_to_byte(str)
  end
  pdftexcmds.escapehex(md5.sum(str))
end
function pdftexcmds.filemdfivesum(filename)
  local foundfile = kpse.find_file(filename, "tex", true)
  if foundfile then
    local filehandle = io.open(foundfile, "rb")
    if filehandle then
      local contents = filehandle:read("*a")
      pdftexcmds.escapehex(md5.sum(contents))
      filehandle:close()
    end
  end
end
%    \end{macrocode}
%
% \subsubsection[Timekeeping]{Timekeeping \cite[``7.17 Timekeeping'']{pdftex-manual}}
%
%    The functions for timekeeping are based on
%    Andy Thomas' work \cite{AndyThomas:Analog}.
%    Changes:
%    \begin{itemize}
%    \item Overflow check is added.
%    \item |string.format| is used to avoid exponential number
%          representation for sure.
%    \item |tex.write| is used instead of |tex.print| to get
%          tokens with catcode 12 and without appended \cs{endlinechar}.
%    \end{itemize}
%    \begin{macrocode}
local basetime = 0
function pdftexcmds.resettimer()
  basetime = os.clock()
end
function pdftexcmds.elapsedtime()
  local val = (os.clock() - basetime) * 65536 + .5
  if val > 2147483647 then
    val = 2147483647
  end
  tex.write(string.format("%d", val))
end
%    \end{macrocode}
%
% \subsubsection[Miscellaneous]{Miscellaneous \cite[``7.21 Miscellaneous'']{pdftex-manual}}
%
%    \begin{macrocode}
function pdftexcmds.shellescape()
  if os.execute then
    if status
        and status.luatex_version
        and status.luatex_version >= 68 then
      tex.write(os.execute())
    else
      local result = os.execute()
      if result == 0 then
        tex.write("0")
      else
        if result == nil then
          tex.write("0")
        else
          tex.write("1")
        end
      end
    end
  else
    tex.write("0")
  end
end
function pdftexcmds.system(cmdline)
  systemexitstatus = nil
  texio.write_nl("log", "system(" .. cmdline .. ") ")
  if os.execute then
    texio.write("log", "executed.")
    systemexitstatus = os.execute(cmdline)
  else
    texio.write("log", "disabled.")
  end
end
function pdftexcmds.lastsystemstatus()
  local result = tonumber(systemexitstatus)
  if result then
    local x = math.floor(result / 256)
    tex.write(result - 256 * math.floor(result / 256))
  end
end
function pdftexcmds.lastsystemexit()
  local result = tonumber(systemexitstatus)
  if result then
    tex.write(math.floor(result / 256))
  end
end
function pdftexcmds.pipe(cmdline, patch)
  local result
  systemexitstatus = nil
  texio.write_nl("log", "pipe(" .. cmdline ..") ")
  if io.popen then
    texio.write("log", "executed.")
    local handle = io.popen(cmdline, "r")
    if handle then
      result = handle:read("*a")
      handle:close()
    end
  else
    texio.write("log", "disabled.")
  end
  if result then
    if patch == 1 then
      local temp = {}
      for i, a in ipairs(result) do
        if a == 0 then
          table.insert(temp, 1)
          table.insert(temp, 1)
        else
          if a == 1 then
            table.insert(temp, 1)
            table.insert(temp, 2)
          else
            table.insert(temp, a)
          end
        end
      end
      result = temp
    end
    tex.settoks(pdftexcmds.toks, result)
  else
    tex.settoks(pdftexcmds.toks, "")
  end
end
%    \end{macrocode}
%    \begin{macrocode}
%</lua>
%    \end{macrocode}
%
% \section{Test}
%
% \subsection{Catcode checks for loading}
%
%    \begin{macrocode}
%<*test1>
%    \end{macrocode}
%    \begin{macrocode}
\catcode`\{=1 %
\catcode`\}=2 %
\catcode`\#=6 %
\catcode`\@=11 %
\expandafter\ifx\csname count@\endcsname\relax
  \countdef\count@=255 %
\fi
\expandafter\ifx\csname @gobble\endcsname\relax
  \long\def\@gobble#1{}%
\fi
\expandafter\ifx\csname @firstofone\endcsname\relax
  \long\def\@firstofone#1{#1}%
\fi
\expandafter\ifx\csname loop\endcsname\relax
  \expandafter\@firstofone
\else
  \expandafter\@gobble
\fi
{%
  \def\loop#1\repeat{%
    \def\body{#1}%
    \iterate
  }%
  \def\iterate{%
    \body
      \let\next\iterate
    \else
      \let\next\relax
    \fi
    \next
  }%
  \let\repeat=\fi
}%
\def\RestoreCatcodes{}
\count@=0 %
\loop
  \edef\RestoreCatcodes{%
    \RestoreCatcodes
    \catcode\the\count@=\the\catcode\count@\relax
  }%
\ifnum\count@<255 %
  \advance\count@ 1 %
\repeat

\def\RangeCatcodeInvalid#1#2{%
  \count@=#1\relax
  \loop
    \catcode\count@=15 %
  \ifnum\count@<#2\relax
    \advance\count@ 1 %
  \repeat
}
\def\RangeCatcodeCheck#1#2#3{%
  \count@=#1\relax
  \loop
    \ifnum#3=\catcode\count@
    \else
      \errmessage{%
        Character \the\count@\space
        with wrong catcode \the\catcode\count@\space
        instead of \number#3%
      }%
    \fi
  \ifnum\count@<#2\relax
    \advance\count@ 1 %
  \repeat
}
\def\space{ }
\expandafter\ifx\csname LoadCommand\endcsname\relax
  \def\LoadCommand{\input pdftexcmds.sty\relax}%
\fi
\def\Test{%
  \RangeCatcodeInvalid{0}{47}%
  \RangeCatcodeInvalid{58}{64}%
  \RangeCatcodeInvalid{91}{96}%
  \RangeCatcodeInvalid{123}{255}%
  \catcode`\@=12 %
  \catcode`\\=0 %
  \catcode`\%=14 %
  \LoadCommand
  \RangeCatcodeCheck{0}{36}{15}%
  \RangeCatcodeCheck{37}{37}{14}%
  \RangeCatcodeCheck{38}{47}{15}%
  \RangeCatcodeCheck{48}{57}{12}%
  \RangeCatcodeCheck{58}{63}{15}%
  \RangeCatcodeCheck{64}{64}{12}%
  \RangeCatcodeCheck{65}{90}{11}%
  \RangeCatcodeCheck{91}{91}{15}%
  \RangeCatcodeCheck{92}{92}{0}%
  \RangeCatcodeCheck{93}{96}{15}%
  \RangeCatcodeCheck{97}{122}{11}%
  \RangeCatcodeCheck{123}{255}{15}%
  \RestoreCatcodes
}
\Test
\csname @@end\endcsname
\end
%    \end{macrocode}
%    \begin{macrocode}
%</test1>
%    \end{macrocode}
%
% \subsection{Test for \cs{pdf@isprimitive}}
%
%    \begin{macrocode}
%<*test2>
\catcode`\{=1 %
\catcode`\}=2 %
\catcode`\#=6 %
\catcode`\@=11 %
\input pdftexcmds.sty\relax
\def\msg#1{%
  \begingroup
    \escapechar=92 %
    \immediate\write16{#1}%
  \endgroup
}
\long\def\test#1#2#3#4{%
  \begingroup
    #4%
    \def\str{%
      Test \string\pdf@isprimitive
      {\string #1}{\string #2}{...}: %
    }%
    \pdf@isprimitive{#1}{#2}{%
      \ifx#3Y%
        \msg{\str true ==> OK.}%
      \else
        \errmessage{\str false ==> FAILED}%
      \fi
    }{%
      \ifx#3Y%
        \errmessage{\str true ==> FAILED}%
      \else
        \msg{\str false ==> OK.}%
      \fi
    }%
  \endgroup
}
\test\relax\relax Y{}
\test\foobar\relax Y{\let\foobar\relax}
\test\foobar\relax N{}
\test\hbox\hbox Y{}
\test\foobar@hbox\hbox Y{\let\foobar@hbox\hbox}
\test\if\if Y{}
\test\if\ifx N{}
\test\ifx\if N{}
\test\par\par Y{}
\test\hbox\par N{}
\test\par\hbox N{}
\csname @@end\endcsname\end
%</test2>
%    \end{macrocode}
%
% \subsection{Test for \cs{pdf@shellescape}}
%
%    \begin{macrocode}
%<*test-shell>
\catcode`\{=1 %
\catcode`\}=2 %
\catcode`\#=6 %
\catcode`\@=11 %
\input pdftexcmds.sty\relax
\def\msg#{\immediate\write16}
\def\MaybeEnd{}
\ifx\luatexversion\UnDeFiNeD
\else
  \ifnum\luatexversion<68 %
    \ifx\pdf@shellescape\@undefined
      \msg{SHELL=U}%
      \msg{OK (LuaTeX < 0.68)}%
    \else
      \msg{SHELL=defined}%
      \errmessage{Failed (LuaTeX < 0.68)}%
    \fi
    \def\MaybeEnd{\csname @@end\endcsname\end}%
  \fi
\fi
\MaybeEnd
\ifx\pdf@shellescape\@undefined
  \msg{SHELL=U}%
\else
  \msg{SHELL=\number\pdf@shellescape}%
\fi
\ifx\expected\@undefined
\else
  \ifx\expected\relax
    \msg{EXPECTED=U}%
    \ifx\pdf@shellescape\@undefined
      \msg{OK}%
    \else
      \errmessage{Failed}%
    \fi
  \else
    \msg{EXPECTED=\number\expected}%
    \ifnum\pdf@shellescape=\expected\relax
      \msg{OK}%
    \else
      \errmessage{Failed}%
    \fi
  \fi
\fi
\csname @@end\endcsname\end
%</test-shell>
%    \end{macrocode}
%
% \subsection{Test for escape functions}
%
%    \begin{macrocode}
%<*test-escape>
\catcode`\{=1 %
\catcode`\}=2 %
\catcode`\#=6 %
\catcode`\^=7 %
\catcode`\@=11 %
\errorcontextlines=1000 %
\input pdftexcmds.sty\relax
\def\msg#1{%
  \begingroup
    \escapechar=92 %
    \immediate\write16{#1}%
  \endgroup
}
%    \end{macrocode}
%    \begin{macrocode}
\begingroup
  \catcode`\@=11 %
  \countdef\count@=255 %
  \def\space{ }%
  \long\def\@whilenum#1\do #2{%
    \ifnum #1\relax
      #2\relax
      \@iwhilenum{#1\relax#2\relax}%
    \fi
  }%
  \long\def\@iwhilenum#1{%
    \ifnum #1%
      \expandafter\@iwhilenum
    \else
      \expandafter\ltx@gobble
    \fi
    {#1}%
  }%
  \gdef\AllBytes{}%
  \count@=0 %
  \catcode0=12 %
  \@whilenum\count@<256 \do{%
    \lccode0=\count@
    \ifnum\count@=32 %
      \xdef\AllBytes{\AllBytes\space}%
    \else
      \lowercase{%
        \xdef\AllBytes{\AllBytes^^@}%
      }%
    \fi
    \advance\count@ by 1 %
  }%
\endgroup
%    \end{macrocode}
%    \begin{macrocode}
\def\AllBytesHex{%
  000102030405060708090A0B0C0D0E0F%
  101112131415161718191A1B1C1D1E1F%
  202122232425262728292A2B2C2D2E2F%
  303132333435363738393A3B3C3D3E3F%
  404142434445464748494A4B4C4D4E4F%
  505152535455565758595A5B5C5D5E5F%
  606162636465666768696A6B6C6D6E6F%
  707172737475767778797A7B7C7D7E7F%
  808182838485868788898A8B8C8D8E8F%
  909192939495969798999A9B9C9D9E9F%
  A0A1A2A3A4A5A6A7A8A9AAABACADAEAF%
  B0B1B2B3B4B5B6B7B8B9BABBBCBDBEBF%
  C0C1C2C3C4C5C6C7C8C9CACBCCCDCECF%
  D0D1D2D3D4D5D6D7D8D9DADBDCDDDEDF%
  E0E1E2E3E4E5E6E7E8E9EAEBECEDEEEF%
  F0F1F2F3F4F5F6F7F8F9FAFBFCFDFEFF%
}
\ltx@onelevel@sanitize\AllBytesHex
\expandafter\lowercase\expandafter{%
  \expandafter\def\expandafter\AllBytesHexLC
      \expandafter{\AllBytesHex}%
}
\begingroup
  \catcode`\#=12 %
  \xdef\AllBytesName{%
    #01#02#03#04#05#06#07#08#09#0A#0B#0C#0D#0E#0F%
    #10#11#12#13#14#15#16#17#18#19#1A#1B#1C#1D#1E#1F%
    #20!"#23$#25&'#28#29*+,-.#2F%
    0123456789:;#3C=#3E?%
    @ABCDEFGHIJKLMNO%
    PQRSTUVWXYZ#5B\ltx@backslashchar#5D^_%
    `abcdefghijklmno%
    pqrstuvwxyz#7B|#7D\string~#7F%
    #80#81#82#83#84#85#86#87#88#89#8A#8B#8C#8D#8E#8F%
    #90#91#92#93#94#95#96#97#98#99#9A#9B#9C#9D#9E#9F%
    #A0#A1#A2#A3#A4#A5#A6#A7#A8#A9#AA#AB#AC#AD#AE#AF%
    #B0#B1#B2#B3#B4#B5#B6#B7#B8#B9#BA#BB#BC#BD#BE#BF%
    #C0#C1#C2#C3#C4#C5#C6#C7#C8#C9#CA#CB#CC#CD#CE#CF%
    #D0#D1#D2#D3#D4#D5#D6#D7#D8#D9#DA#DB#DC#DD#DE#DF%
    #E0#E1#E2#E3#E4#E5#E6#E7#E8#E9#EA#EB#EC#ED#EE#EF%
    #F0#F1#F2#F3#F4#F5#F6#F7#F8#F9#FA#FB#FC#FD#FE#FF%
  }%
\endgroup
\ltx@onelevel@sanitize\AllBytesName
\edef\AllBytesFromName{\expandafter\ltx@gobble\AllBytes}
\begingroup
  \def\|{|}%
  \edef\%{\ltx@percentchar}%
  \catcode`\|=0 %
  \catcode`\#=12 %
  \catcode`\~=12 %
  \catcode`\\=12 %
  |xdef|AllBytesString{%
    \000\001\002\003\004\005\006\007\010\011\012\013\014\015\016\017%
    \020\021\022\023\024\025\026\027\030\031\032\033\034\035\036\037%
    \040!"#$|%&'\(\)*+,-./%
    0123456789:;<=>?%
    @ABCDEFGHIJKLMNO%
    PQRSTUVWXYZ[\\]^_%
    `abcdefghijklmno%
    pqrstuvwxyz{||}~\177%
    \200\201\202\203\204\205\206\207\210\211\212\213\214\215\216\217%
    \220\221\222\223\224\225\226\227\230\231\232\233\234\235\236\237%
    \240\241\242\243\244\245\246\247\250\251\252\253\254\255\256\257%
    \260\261\262\263\264\265\266\267\270\271\272\273\274\275\276\277%
    \300\301\302\303\304\305\306\307\310\311\312\313\314\315\316\317%
    \320\321\322\323\324\325\326\327\330\331\332\333\334\335\336\337%
    \340\341\342\343\344\345\346\347\350\351\352\353\354\355\356\357%
    \360\361\362\363\364\365\366\367\370\371\372\373\374\375\376\377%
  }%
|endgroup
\ltx@onelevel@sanitize\AllBytesString
%    \end{macrocode}
%    \begin{macrocode}
\def\Test#1#2#3{%
  \begingroup
    \expandafter\expandafter\expandafter\def
    \expandafter\expandafter\expandafter\TestResult
    \expandafter\expandafter\expandafter{%
      #1{#2}%
    }%
    \ifx\TestResult#3%
    \else
      \newlinechar=10 %
      \msg{Expect:^^J#3}%
      \msg{Result:^^J\TestResult}%
      \errmessage{\string#2 -\string#1-> \string#3}%
    \fi
  \endgroup
}
\def\test#1#2#3{%
  \edef\TestFrom{#2}%
  \edef\TestExpect{#3}%
  \ltx@onelevel@sanitize\TestExpect
  \Test#1\TestFrom\TestExpect
}
\test\pdf@unescapehex{74657374}{test}
\begingroup
  \catcode0=12 %
  \catcode1=12 %
  \test\pdf@unescapehex{740074017400740174}{t^^@t^^At^^@t^^At}%
\endgroup
\Test\pdf@escapehex\AllBytes\AllBytesHex
\Test\pdf@unescapehex\AllBytesHex\AllBytes
\Test\pdf@escapename\AllBytes\AllBytesName
\Test\pdf@escapestring\AllBytes\AllBytesString
%    \end{macrocode}
%    \begin{macrocode}
\csname @@end\endcsname\end
%</test-escape>
%    \end{macrocode}
%
% \section{Installation}
%
% \subsection{Download}
%
% \paragraph{Package.} This package is available on
% CTAN\footnote{\CTANpkg{pdftexcmds}}:
% \begin{description}
% \item[\CTAN{macros/latex/contrib/oberdiek/pdftexcmds.dtx}] The source file.
% \item[\CTAN{macros/latex/contrib/oberdiek/pdftexcmds.pdf}] Documentation.
% \end{description}
%
%
% \paragraph{Bundle.} All the packages of the bundle `oberdiek'
% are also available in a TDS compliant ZIP archive. There
% the packages are already unpacked and the documentation files
% are generated. The files and directories obey the TDS standard.
% \begin{description}
% \item[\CTANinstall{install/macros/latex/contrib/oberdiek.tds.zip}]
% \end{description}
% \emph{TDS} refers to the standard ``A Directory Structure
% for \TeX\ Files'' (\CTAN{tds/tds.pdf}). Directories
% with \xfile{texmf} in their name are usually organized this way.
%
% \subsection{Bundle installation}
%
% \paragraph{Unpacking.} Unpack the \xfile{oberdiek.tds.zip} in the
% TDS tree (also known as \xfile{texmf} tree) of your choice.
% Example (linux):
% \begin{quote}
%   |unzip oberdiek.tds.zip -d ~/texmf|
% \end{quote}
%
% \paragraph{Script installation.}
% Check the directory \xfile{TDS:scripts/oberdiek/} for
% scripts that need further installation steps.
% Package \xpackage{attachfile2} comes with the Perl script
% \xfile{pdfatfi.pl} that should be installed in such a way
% that it can be called as \texttt{pdfatfi}.
% Example (linux):
% \begin{quote}
%   |chmod +x scripts/oberdiek/pdfatfi.pl|\\
%   |cp scripts/oberdiek/pdfatfi.pl /usr/local/bin/|
% \end{quote}
%
% \subsection{Package installation}
%
% \paragraph{Unpacking.} The \xfile{.dtx} file is a self-extracting
% \docstrip\ archive. The files are extracted by running the
% \xfile{.dtx} through \plainTeX:
% \begin{quote}
%   \verb|tex pdftexcmds.dtx|
% \end{quote}
%
% \paragraph{TDS.} Now the different files must be moved into
% the different directories in your installation TDS tree
% (also known as \xfile{texmf} tree):
% \begin{quote}
% \def\t{^^A
% \begin{tabular}{@{}>{\ttfamily}l@{ $\rightarrow$ }>{\ttfamily}l@{}}
%   pdftexcmds.sty & tex/generic/oberdiek/pdftexcmds.sty\\
%   oberdiek.pdftexcmds.lua & scripts/oberdiek/oberdiek.pdftexcmds.lua\\
%   pdftexcmds.lua & scripts/oberdiek/pdftexcmds.lua\\
%   pdftexcmds.pdf & doc/latex/oberdiek/pdftexcmds.pdf\\
%   test/pdftexcmds-test1.tex & doc/latex/oberdiek/test/pdftexcmds-test1.tex\\
%   test/pdftexcmds-test2.tex & doc/latex/oberdiek/test/pdftexcmds-test2.tex\\
%   test/pdftexcmds-test-shell.tex & doc/latex/oberdiek/test/pdftexcmds-test-shell.tex\\
%   test/pdftexcmds-test-escape.tex & doc/latex/oberdiek/test/pdftexcmds-test-escape.tex\\
%   pdftexcmds.dtx & source/latex/oberdiek/pdftexcmds.dtx\\
% \end{tabular}^^A
% }^^A
% \sbox0{\t}^^A
% \ifdim\wd0>\linewidth
%   \begingroup
%     \advance\linewidth by\leftmargin
%     \advance\linewidth by\rightmargin
%   \edef\x{\endgroup
%     \def\noexpand\lw{\the\linewidth}^^A
%   }\x
%   \def\lwbox{^^A
%     \leavevmode
%     \hbox to \linewidth{^^A
%       \kern-\leftmargin\relax
%       \hss
%       \usebox0
%       \hss
%       \kern-\rightmargin\relax
%     }^^A
%   }^^A
%   \ifdim\wd0>\lw
%     \sbox0{\small\t}^^A
%     \ifdim\wd0>\linewidth
%       \ifdim\wd0>\lw
%         \sbox0{\footnotesize\t}^^A
%         \ifdim\wd0>\linewidth
%           \ifdim\wd0>\lw
%             \sbox0{\scriptsize\t}^^A
%             \ifdim\wd0>\linewidth
%               \ifdim\wd0>\lw
%                 \sbox0{\tiny\t}^^A
%                 \ifdim\wd0>\linewidth
%                   \lwbox
%                 \else
%                   \usebox0
%                 \fi
%               \else
%                 \lwbox
%               \fi
%             \else
%               \usebox0
%             \fi
%           \else
%             \lwbox
%           \fi
%         \else
%           \usebox0
%         \fi
%       \else
%         \lwbox
%       \fi
%     \else
%       \usebox0
%     \fi
%   \else
%     \lwbox
%   \fi
% \else
%   \usebox0
% \fi
% \end{quote}
% If you have a \xfile{docstrip.cfg} that configures and enables \docstrip's
% TDS installing feature, then some files can already be in the right
% place, see the documentation of \docstrip.
%
% \subsection{Refresh file name databases}
%
% If your \TeX~distribution
% (\teTeX, \mikTeX, \dots) relies on file name databases, you must refresh
% these. For example, \teTeX\ users run \verb|texhash| or
% \verb|mktexlsr|.
%
% \subsection{Some details for the interested}
%
% \paragraph{Unpacking with \LaTeX.}
% The \xfile{.dtx} chooses its action depending on the format:
% \begin{description}
% \item[\plainTeX:] Run \docstrip\ and extract the files.
% \item[\LaTeX:] Generate the documentation.
% \end{description}
% If you insist on using \LaTeX\ for \docstrip\ (really,
% \docstrip\ does not need \LaTeX), then inform the autodetect routine
% about your intention:
% \begin{quote}
%   \verb|latex \let\install=y\input{pdftexcmds.dtx}|
% \end{quote}
% Do not forget to quote the argument according to the demands
% of your shell.
%
% \paragraph{Generating the documentation.}
% You can use both the \xfile{.dtx} or the \xfile{.drv} to generate
% the documentation. The process can be configured by the
% configuration file \xfile{ltxdoc.cfg}. For instance, put this
% line into this file, if you want to have A4 as paper format:
% \begin{quote}
%   \verb|\PassOptionsToClass{a4paper}{article}|
% \end{quote}
% An example follows how to generate the
% documentation with pdf\LaTeX:
% \begin{quote}
%\begin{verbatim}
%pdflatex pdftexcmds.dtx
%bibtex pdftexcmds.aux
%makeindex -s gind.ist pdftexcmds.idx
%pdflatex pdftexcmds.dtx
%makeindex -s gind.ist pdftexcmds.idx
%pdflatex pdftexcmds.dtx
%\end{verbatim}
% \end{quote}
%
% \printbibliography[
%   heading=bibnumbered,
% ]
%
% \begin{History}
%   \begin{Version}{2007/11/11 v0.1}
%   \item
%     First version.
%   \end{Version}
%   \begin{Version}{2007/11/12 v0.2}
%   \item
%     Short description fixed.
%   \end{Version}
%   \begin{Version}{2007/12/12 v0.3}
%   \item
%     Organization of Lua code as module.
%   \end{Version}
%   \begin{Version}{2009/04/10 v0.4}
%   \item
%     Adaptation for syntax change of \cs{directlua} in
%     \hologo{LuaTeX} 0.36.
%   \end{Version}
%   \begin{Version}{2009/09/22 v0.5}
%   \item
%     \cs{pdf@primitive}, \cs{pdf@ifprimitive} added.
%   \item
%     \hologo{XeTeX}'s variants are detected for
%     \cs{pdf@shellescape}, \cs{pdf@strcmp}, \cs{pdf@primitive},
%     \cs{pdf@ifprimitive}.
%   \end{Version}
%   \begin{Version}{2009/09/23 v0.6}
%   \item
%     Macro \cs{pdf@isprimitive} added.
%   \end{Version}
%   \begin{Version}{2009/12/12 v0.7}
%   \item
%     Short info shortened.
%   \end{Version}
%   \begin{Version}{2010/03/01 v0.8}
%   \item
%     Required date for package \xpackage{ifluatex} updated.
%   \end{Version}
%   \begin{Version}{2010/04/01 v0.9}
%   \item
%     Use \cs{ifeof18} for defining \cs{pdf@shellescape} between
%     \hologo{pdfTeX} 1.21a (inclusive) and 1.30.0 (exclusive).
%   \end{Version}
%   \begin{Version}{2010/11/04 v0.10}
%   \item
%     \cs{pdf@draftmode}, \cs{pdf@ifdraftmode} and
%     \cs{pdf@setdraftmode} added.
%   \end{Version}
%   \begin{Version}{2010/11/11 v0.11}
%   \item
%     Missing \cs{RequirePackage} for package \xpackage{ifpdf} added.
%   \end{Version}
%   \begin{Version}{2011/01/30 v0.12}
%   \item
%     Already loaded package files are not input in \hologo{plainTeX}.
%   \end{Version}
%   \begin{Version}{2011/03/04 v0.13}
%   \item
%     Improved Lua function \texttt{shellescape} that also
%     uses the result of \texttt{os.execute()} (thanks to Philipp Stephani).
%   \end{Version}
%   \begin{Version}{2011/04/10 v0.14}
%   \item
%     Version check of loaded module added.
%   \item
%     Patch for bug in \hologo{LuaTeX} between 0.40.6 and 0.65 that
%     is fixed in revision 4096.
%   \end{Version}
%   \begin{Version}{2011/04/16 v0.15}
%   \item
%     \hologo{LuaTeX}: \cs{pdf@shellescape} is only supported
%     for version 0.70.0 and higher due to a bug, \texttt{os.execute()}
%     crashes in some circumstances. Fixed in \hologo{LuaTeX}
%     beta-0.70.0, revision 4167.
%   \end{Version}
%   \begin{Version}{2011/04/22 v0.16}
%   \item
%     Previous fix was not working due to a wrong catcode of digit
%     zero (due to easily support the old \cs{directlua0}).
%     The version border is lowered to 0.68, because some
%     beta-0.67.0 seems also to work.
%   \end{Version}
%   \begin{Version}{2011/06/29 v0.17}
%   \item
%     Documentation addition to \cs{pdf@shellescape}.
%   \end{Version}
%   \begin{Version}{2011/07/01 v0.18}
%   \item
%     Add Lua module loading in \cs{everyjob} for \hologo{iniTeX}
%     (\hologo{LuaTeX} only).
%   \end{Version}
%   \begin{Version}{2011/07/28 v0.19}
%   \item
%     Missing space in an info message added (Martin M\"unch).
%   \end{Version}
%   \begin{Version}{2011/11/29 v0.20}
%   \item
%     \cs{pdf@resettimer} and \cs{pdf@elapsedtime} added
%     (thanks Andy Thomas).
%   \end{Version}
%   \begin{Version}{2016/05/10 v0.21}
%   \item
%      local unpack added
%     (thanks \'{E}lie Roux).
%   \end{Version}
%   \begin{Version}{2016/05/21 v0.22}
%   \item
%     adjust \cs{textbackslas}h usage in bib file for biber bug.
%   \end{Version}
%   \begin{Version}{2016/10/02 v0.23}
%   \item
%     add file.close to lua filehandles (github pull request).
%   \end{Version}
%   \begin{Version}{2017/01/29 v0.24}
%   \item
%     Avoid loading luatex-loader for current luatex. (Use
%     pdftexcmds.lua not oberdiek.pdftexcmds.lua to simplify file
%     search with standard require)
%   \end{Version}
%   \begin{Version}{2017/03/19 v0.25}
%   \item
%     New \cs{pdf@shellescape} for Lua\TeX, see github issue 20.
%   \end{Version}
%   \begin{Version}{2018/01/21 v0.26}
%   \item
%     use rb not r mode for file open github issue 34.
%   \end{Version}
%   \begin{Version}{2018/01/30 v0.27}
%   \item
%     \cs{pdf@mdfivesum} for \hologo{XeTeX}
%   \end{Version}
%   \begin{Version}{2018/09/07 v0.28}
%   \item
%     Fix catcode regime in luatex sprint for \cs{pdf@shellescape} GH issue 45
%   \end{Version}
%   \begin{Version}{2018/09/10 v0.29}
%   \item
%     Actually do the fix described above in the code, not just document it.
%   \end{Version}
%   \begin{Version}{2019/07/25 v0.30}
%   \item
%     remove uses of module function, see PR70
%   \end{Version}
% \end{History}
%
% \PrintIndex
%
% \Finale
\endinput
|
% \end{quote}
% Do not forget to quote the argument according to the demands
% of your shell.
%
% \paragraph{Generating the documentation.}
% You can use both the \xfile{.dtx} or the \xfile{.drv} to generate
% the documentation. The process can be configured by the
% configuration file \xfile{ltxdoc.cfg}. For instance, put this
% line into this file, if you want to have A4 as paper format:
% \begin{quote}
%   \verb|\PassOptionsToClass{a4paper}{article}|
% \end{quote}
% An example follows how to generate the
% documentation with pdf\LaTeX:
% \begin{quote}
%\begin{verbatim}
%pdflatex pdftexcmds.dtx
%bibtex pdftexcmds.aux
%makeindex -s gind.ist pdftexcmds.idx
%pdflatex pdftexcmds.dtx
%makeindex -s gind.ist pdftexcmds.idx
%pdflatex pdftexcmds.dtx
%\end{verbatim}
% \end{quote}
%
% \printbibliography[
%   heading=bibnumbered,
% ]
%
% \begin{History}
%   \begin{Version}{2007/11/11 v0.1}
%   \item
%     First version.
%   \end{Version}
%   \begin{Version}{2007/11/12 v0.2}
%   \item
%     Short description fixed.
%   \end{Version}
%   \begin{Version}{2007/12/12 v0.3}
%   \item
%     Organization of Lua code as module.
%   \end{Version}
%   \begin{Version}{2009/04/10 v0.4}
%   \item
%     Adaptation for syntax change of \cs{directlua} in
%     \hologo{LuaTeX} 0.36.
%   \end{Version}
%   \begin{Version}{2009/09/22 v0.5}
%   \item
%     \cs{pdf@primitive}, \cs{pdf@ifprimitive} added.
%   \item
%     \hologo{XeTeX}'s variants are detected for
%     \cs{pdf@shellescape}, \cs{pdf@strcmp}, \cs{pdf@primitive},
%     \cs{pdf@ifprimitive}.
%   \end{Version}
%   \begin{Version}{2009/09/23 v0.6}
%   \item
%     Macro \cs{pdf@isprimitive} added.
%   \end{Version}
%   \begin{Version}{2009/12/12 v0.7}
%   \item
%     Short info shortened.
%   \end{Version}
%   \begin{Version}{2010/03/01 v0.8}
%   \item
%     Required date for package \xpackage{ifluatex} updated.
%   \end{Version}
%   \begin{Version}{2010/04/01 v0.9}
%   \item
%     Use \cs{ifeof18} for defining \cs{pdf@shellescape} between
%     \hologo{pdfTeX} 1.21a (inclusive) and 1.30.0 (exclusive).
%   \end{Version}
%   \begin{Version}{2010/11/04 v0.10}
%   \item
%     \cs{pdf@draftmode}, \cs{pdf@ifdraftmode} and
%     \cs{pdf@setdraftmode} added.
%   \end{Version}
%   \begin{Version}{2010/11/11 v0.11}
%   \item
%     Missing \cs{RequirePackage} for package \xpackage{ifpdf} added.
%   \end{Version}
%   \begin{Version}{2011/01/30 v0.12}
%   \item
%     Already loaded package files are not input in \hologo{plainTeX}.
%   \end{Version}
%   \begin{Version}{2011/03/04 v0.13}
%   \item
%     Improved Lua function \texttt{shellescape} that also
%     uses the result of \texttt{os.execute()} (thanks to Philipp Stephani).
%   \end{Version}
%   \begin{Version}{2011/04/10 v0.14}
%   \item
%     Version check of loaded module added.
%   \item
%     Patch for bug in \hologo{LuaTeX} between 0.40.6 and 0.65 that
%     is fixed in revision 4096.
%   \end{Version}
%   \begin{Version}{2011/04/16 v0.15}
%   \item
%     \hologo{LuaTeX}: \cs{pdf@shellescape} is only supported
%     for version 0.70.0 and higher due to a bug, \texttt{os.execute()}
%     crashes in some circumstances. Fixed in \hologo{LuaTeX}
%     beta-0.70.0, revision 4167.
%   \end{Version}
%   \begin{Version}{2011/04/22 v0.16}
%   \item
%     Previous fix was not working due to a wrong catcode of digit
%     zero (due to easily support the old \cs{directlua0}).
%     The version border is lowered to 0.68, because some
%     beta-0.67.0 seems also to work.
%   \end{Version}
%   \begin{Version}{2011/06/29 v0.17}
%   \item
%     Documentation addition to \cs{pdf@shellescape}.
%   \end{Version}
%   \begin{Version}{2011/07/01 v0.18}
%   \item
%     Add Lua module loading in \cs{everyjob} for \hologo{iniTeX}
%     (\hologo{LuaTeX} only).
%   \end{Version}
%   \begin{Version}{2011/07/28 v0.19}
%   \item
%     Missing space in an info message added (Martin M\"unch).
%   \end{Version}
%   \begin{Version}{2011/11/29 v0.20}
%   \item
%     \cs{pdf@resettimer} and \cs{pdf@elapsedtime} added
%     (thanks Andy Thomas).
%   \end{Version}
%   \begin{Version}{2016/05/10 v0.21}
%   \item
%      local unpack added
%     (thanks \'{E}lie Roux).
%   \end{Version}
%   \begin{Version}{2016/05/21 v0.22}
%   \item
%     adjust \cs{textbackslas}h usage in bib file for biber bug.
%   \end{Version}
%   \begin{Version}{2016/10/02 v0.23}
%   \item
%     add file.close to lua filehandles (github pull request).
%   \end{Version}
%   \begin{Version}{2017/01/29 v0.24}
%   \item
%     Avoid loading luatex-loader for current luatex. (Use
%     pdftexcmds.lua not oberdiek.pdftexcmds.lua to simplify file
%     search with standard require)
%   \end{Version}
%   \begin{Version}{2017/03/19 v0.25}
%   \item
%     New \cs{pdf@shellescape} for Lua\TeX, see github issue 20.
%   \end{Version}
%   \begin{Version}{2018/01/21 v0.26}
%   \item
%     use rb not r mode for file open github issue 34.
%   \end{Version}
%   \begin{Version}{2018/01/30 v0.27}
%   \item
%     \cs{pdf@mdfivesum} for \hologo{XeTeX}
%   \end{Version}
%   \begin{Version}{2018/09/07 v0.28}
%   \item
%     Fix catcode regime in luatex sprint for \cs{pdf@shellescape} GH issue 45
%   \end{Version}
%   \begin{Version}{2018/09/10 v0.29}
%   \item
%     Actually do the fix described above in the code, not just document it.
%   \end{Version}
%   \begin{Version}{2019/07/25 v0.30}
%   \item
%     remove uses of module function, see PR70
%   \end{Version}
% \end{History}
%
% \PrintIndex
%
% \Finale
\endinput

%        (quote the arguments according to the demands of your shell)
%
% Documentation:
%    (a) If pdftexcmds.drv is present:
%           latex pdftexcmds.drv
%    (b) Without pdftexcmds.drv:
%           latex pdftexcmds.dtx; ...
%    The class ltxdoc loads the configuration file ltxdoc.cfg
%    if available. Here you can specify further options, e.g.
%    use A4 as paper format:
%       \PassOptionsToClass{a4paper}{article}
%
%    Programm calls to get the documentation (example):
%       pdflatex pdftexcmds.dtx
%       bibtex pdftexcmds.aux
%       makeindex -s gind.ist pdftexcmds.idx
%       pdflatex pdftexcmds.dtx
%       makeindex -s gind.ist pdftexcmds.idx
%       pdflatex pdftexcmds.dtx
%
% Installation:
%    TDS:tex/generic/oberdiek/pdftexcmds.sty
%    TDS:scripts/oberdiek/oberdiek.pdftexcmds.lua
%    TDS:scripts/oberdiek/pdftexcmds.lua
%    TDS:doc/latex/oberdiek/pdftexcmds.pdf
%    TDS:doc/latex/oberdiek/test/pdftexcmds-test1.tex
%    TDS:doc/latex/oberdiek/test/pdftexcmds-test2.tex
%    TDS:doc/latex/oberdiek/test/pdftexcmds-test-shell.tex
%    TDS:doc/latex/oberdiek/test/pdftexcmds-test-escape.tex
%    TDS:source/latex/oberdiek/pdftexcmds.dtx
%
%<*ignore>
\begingroup
  \catcode123=1 %
  \catcode125=2 %
  \def\x{LaTeX2e}%
\expandafter\endgroup
\ifcase 0\ifx\install y1\fi\expandafter
         \ifx\csname processbatchFile\endcsname\relax\else1\fi
         \ifx\fmtname\x\else 1\fi\relax
\else\csname fi\endcsname
%</ignore>
%<*install>
\input docstrip.tex
\Msg{************************************************************************}
\Msg{* Installation}
\Msg{* Package: pdftexcmds 2019/07/25 v0.30 Utility functions of pdfTeX for LuaTeX (HO)}
\Msg{************************************************************************}

\keepsilent
\askforoverwritefalse

\let\MetaPrefix\relax
\preamble

This is a generated file.

Project: pdftexcmds
Version: 2019/07/25 v0.30

Copyright (C) 2007, 2009-2011 by
   Heiko Oberdiek <heiko.oberdiek at googlemail.com>

This work may be distributed and/or modified under the
conditions of the LaTeX Project Public License, either
version 1.3c of this license or (at your option) any later
version. This version of this license is in
   https://www.latex-project.org/lppl/lppl-1-3c.txt
and the latest version of this license is in
   https://www.latex-project.org/lppl.txt
and version 1.3 or later is part of all distributions of
LaTeX version 2005/12/01 or later.

This work has the LPPL maintenance status "maintained".

The Current Maintainers of this work are
Heiko Oberdiek and the Oberdiek Package Support Group
https://github.com/ho-tex/oberdiek/issues


The Base Interpreter refers to any `TeX-Format',
because some files are installed in TDS:tex/generic//.

This work consists of the main source file pdftexcmds.dtx
and the derived files
   pdftexcmds.sty, pdftexcmds.pdf, pdftexcmds.ins, pdftexcmds.drv,
   pdftexcmds.bib, pdftexcmds-test1.tex, pdftexcmds-test2.tex,
   pdftexcmds-test-shell.tex, pdftexcmds-test-escape.tex,
   oberdiek.pdftexcmds.lua, pdftexcmds.lua.

\endpreamble
\let\MetaPrefix\DoubleperCent

\generate{%
  \file{pdftexcmds.ins}{\from{pdftexcmds.dtx}{install}}%
  \file{pdftexcmds.drv}{\from{pdftexcmds.dtx}{driver}}%
  \nopreamble
  \nopostamble
  \file{pdftexcmds.bib}{\from{pdftexcmds.dtx}{bib}}%
  \usepreamble\defaultpreamble
  \usepostamble\defaultpostamble
  \usedir{tex/generic/oberdiek}%
  \file{pdftexcmds.sty}{\from{pdftexcmds.dtx}{package}}%
%  \usedir{doc/latex/oberdiek/test}%
%  \file{pdftexcmds-test1.tex}{\from{pdftexcmds.dtx}{test1}}%
%  \file{pdftexcmds-test2.tex}{\from{pdftexcmds.dtx}{test2}}%
%  \file{pdftexcmds-test-shell.tex}{\from{pdftexcmds.dtx}{test-shell}}%
%  \file{pdftexcmds-test-escape.tex}{\from{pdftexcmds.dtx}{test-escape}}%
  \nopreamble
  \nopostamble
%  \usedir{source/latex/oberdiek/catalogue}%
%  \file{pdftexcmds.xml}{\from{pdftexcmds.dtx}{catalogue}}%
}
\def\MetaPrefix{-- }
\def\defaultpostamble{%
  \MetaPrefix^^J%
  \MetaPrefix\space End of File `\outFileName'.%
}
\def\currentpostamble{\defaultpostamble}%
\generate{%
  \usedir{scripts/oberdiek}%
  \file{oberdiek.pdftexcmds.lua}{\from{pdftexcmds.dtx}{lua}}%
  \file{pdftexcmds.lua}{\from{pdftexcmds.dtx}{lua}}%
}

\catcode32=13\relax% active space
\let =\space%
\Msg{************************************************************************}
\Msg{*}
\Msg{* To finish the installation you have to move the following}
\Msg{* file into a directory searched by TeX:}
\Msg{*}
\Msg{*     pdftexcmds.sty}
\Msg{*}
\Msg{* And install the following script files:}
\Msg{*}
\Msg{*     oberdiek.pdftexcmds.lua, pdftexcmds.lua}
\Msg{*}
\Msg{* To produce the documentation run the file `pdftexcmds.drv'}
\Msg{* through LaTeX.}
\Msg{*}
\Msg{* Happy TeXing!}
\Msg{*}
\Msg{************************************************************************}

\endbatchfile
%</install>
%<*bib>
@online{AndyThomas:Analog,
  author={Thomas, Andy},
  title={Analog of {\texttt{\csname textbackslash\endcsname}pdfelapsedtime} for
      {\hologo{LuaTeX}} and {\hologo{XeTeX}}},
  url={http://tex.stackexchange.com/a/32531},
  urldate={2011-11-29},
}
%</bib>
%<*ignore>
\fi
%</ignore>
%<*driver>
\NeedsTeXFormat{LaTeX2e}
\ProvidesFile{pdftexcmds.drv}%
  [2019/07/25 v0.30 Utility functions of pdfTeX for LuaTeX (HO)]%
\documentclass{ltxdoc}
\usepackage{holtxdoc}[2011/11/22]
\usepackage{paralist}
\usepackage{csquotes}
\usepackage[
  backend=bibtex,
  bibencoding=ascii,
  alldates=iso8601,
]{biblatex}[2011/11/13]
\bibliography{oberdiek-source}
\bibliography{pdftexcmds}
\begin{document}
  \DocInput{pdftexcmds.dtx}%
\end{document}
%</driver>
% \fi
%
%
% \CharacterTable
%  {Upper-case    \A\B\C\D\E\F\G\H\I\J\K\L\M\N\O\P\Q\R\S\T\U\V\W\X\Y\Z
%   Lower-case    \a\b\c\d\e\f\g\h\i\j\k\l\m\n\o\p\q\r\s\t\u\v\w\x\y\z
%   Digits        \0\1\2\3\4\5\6\7\8\9
%   Exclamation   \!     Double quote  \"     Hash (number) \#
%   Dollar        \$     Percent       \%     Ampersand     \&
%   Acute accent  \'     Left paren    \(     Right paren   \)
%   Asterisk      \*     Plus          \+     Comma         \,
%   Minus         \-     Point         \.     Solidus       \/
%   Colon         \:     Semicolon     \;     Less than     \<
%   Equals        \=     Greater than  \>     Question mark \?
%   Commercial at \@     Left bracket  \[     Backslash     \\
%   Right bracket \]     Circumflex    \^     Underscore    \_
%   Grave accent  \`     Left brace    \{     Vertical bar  \|
%   Right brace   \}     Tilde         \~}
%
% \GetFileInfo{pdftexcmds.drv}
%
% \title{The \xpackage{pdftexcmds} package}
% \date{2019/07/25 v0.30}
% \author{Heiko Oberdiek\thanks
% {Please report any issues at \url{https://github.com/ho-tex/oberdiek/issues}}}
%
% \maketitle
%
% \begin{abstract}
% \hologo{LuaTeX} provides most of the commands of \hologo{pdfTeX} 1.40. However
% a number of utility functions are removed. This package tries to fill
% the gap and implements some of the missing primitive using Lua.
% \end{abstract}
%
% \tableofcontents
%
% \def\csi#1{\texttt{\textbackslash\textit{#1}}}
%
% \section{Documentation}
%
% Some primitives of \hologo{pdfTeX} \cite{pdftex-manual}
% are not defined by \hologo{LuaTeX} \cite{luatex-manual}.
% This package implements macro based solutions using Lua code
% for the following missing \hologo{pdfTeX} primitives;
% \begin{compactitem}
% \item \cs{pdfstrcmp}
% \item \cs{pdfunescapehex}
% \item \cs{pdfescapehex}
% \item \cs{pdfescapename}
% \item \cs{pdfescapestring}
% \item \cs{pdffilesize}
% \item \cs{pdffilemoddate}
% \item \cs{pdffiledump}
% \item \cs{pdfmdfivesum}
% \item \cs{pdfresettimer}
% \item \cs{pdfelapsedtime}
% \item |\immediate\write18|
% \end{compactitem}
% The original names of the primitives cannot be used:
% \begin{itemize}
% \item
% The syntax for their arguments cannot easily
% simulated by macros. The primitives using key words
% such as |file| (\cs{pdfmdfivesum}) or |offset| and |length|
% (\cs{pdffiledump}) and uses \meta{general text} for the other
% arguments. Using token registers assignments, \meta{general text} could
% be catched. However, the simulated primitives are expandable
% and register assignments would destroy this important property.
% (\meta{general text} allows something like |\expandafter\bgroup ...}|.)
% \item
% The original primitives can be expanded using one expansion step.
% The new macros need two expansion steps because of the additional
% macro expansion. Example:
% \begin{quote}
%   |\expandafter\foo\pdffilemoddate{file}|\\
%   vs.\\
%   |\expandafter\expandafter\expandafter|\\
%   |\foo\pdf@filemoddate{file}|
% \end{quote}
% \end{itemize}
%
% \hologo{LuaTeX} isn't stable yet and thus the status of this package is
% \emph{experimental}. Feedback is welcome.
%
% \subsection{General principles}
%
% \begin{description}
% \item[Naming convention:]
%   Usually this package defines a macro |\pdf@|\meta{cmd} if
%   \hologo{pdfTeX} provides |\pdf|\meta{cmd}.
% \item[Arguments:] The order of arguments in |\pdf@|\meta{cmd}
%   is the same as for the corresponding primitive of \hologo{pdfTeX}.
%   The arguments are ordinary undelimited \hologo{TeX} arguments,
%   no \meta{general text} and without additional keywords.
% \item[Expandibility:]
%   The macro |\pdf@|\meta{cmd} is expandable if the
%   corresponding \hologo{pdfTeX} primitive has this property.
%   Exact two expansion steps are necessary (first is the macro
%   expansion) except for \cs{pdf@primitive} and \cs{pdf@ifprimitive}.
%   The latter ones are not macros, but have the direct meaning of the
%   primitive.
% \item[Without \hologo{LuaTeX}:]
%   The macros |\pdf@|\meta{cmd} are mapped to the commands
%   of \hologo{pdfTeX} if they are available. Otherwise they are undefined.
% \item[Availability:]
%   The macros that the packages provides are undefined, if
%   the necessary primitives are not found and cannot be
%   implemented by Lua.
% \end{description}
%
% \subsection{Macros}
%
% \subsubsection[Strings]{Strings \cite[``7.15 Strings'']{pdftex-manual}}
%
% \begin{declcs}{pdf@strcmp} \M{stringA} \M{stringB}
% \end{declcs}
% Same as |\pdfstrcmp{|\meta{stringA}|}{|\meta{stringB}|}|.
%
% \begin{declcs}{pdf@unescapehex} \M{string}
% \end{declcs}
% Same as |\pdfunescapehex{|\meta{string}|}|.
% The argument is a byte string given in hexadecimal notation.
% The result are character tokens from 0 until 255 with
% catcode 12 and the space with catcode 10.
%
% \begin{declcs}{pdf@escapehex} \M{string}\\
%   \cs{pdf@escapestring} \M{string}\\
%   \cs{pdf@escapename} \M{string}
% \end{declcs}
% Same as the primitives of \hologo{pdfTeX}. However \hologo{pdfTeX} does not
% know about characters with codes 256 and larger. Thus the
% string is treated as byte string, characters with more than
% eight bits are ignored.
%
% \subsubsection[Files]{Files \cite[``7.18 Files'']{pdftex-manual}}
%
% \begin{declcs}{pdf@filesize} \M{filename}
% \end{declcs}
% Same as |\pdffilesize{|\meta{filename}|}|.
%
% \begin{declcs}{pdf@filemoddate} \M{filename}
% \end{declcs}
% Same as |\pdffilemoddate{|\meta{filename}|}|.
%
% \begin{declcs}{pdf@filedump} \M{offset} \M{length} \M{filename}
% \end{declcs}
% Same as |\pdffiledump offset| \meta{offset} |length| \meta{length}
% |{|\meta{filename}|}|. Both \meta{offset} and \meta{length} must
% not be empty, but must be a valid \hologo{TeX} number.
%
% \begin{declcs}{pdf@mdfivesum} \M{string}
% \end{declcs}
% Same as |\pdfmdfivesum{|\meta{string}|}|. Keyword |file| is supported
% by macro \cs{pdf@filemdfivesum}.
%
% \begin{declcs}{pdf@filemdfivesum} \M{filename}
% \end{declcs}
% Same as |\pdfmdfivesum file{|\meta{filename}|}|.
%
% \subsubsection[Timekeeping]{Timekeeping \cite[``7.17 Timekeeping'']{pdftex-manual}}
%
% The timekeeping macros are based on Andy Thomas' work \cite{AndyThomas:Analog}.
%
% \begin{declcs}{pdf@resettimer}
% \end{declcs}
% Same as \cs{pdfresettimer}, it resets the internal timer.
%
% \begin{declcs}{pdf@elapsedtime}
% \end{declcs}
% Same as \cs{pdfelapsedtime}. It behaves like a read-only integer.
% For printing purposes it can be prefixed by \cs{the} or \cs{number}.
% It measures the time in scaled seconds (seconds multiplied with 65536)
% since the latest call of \cs{pdf@resettimer} or start of
% program/package. The resolution, the shortest time interval that
% can be measured, depends on the program and system.
% \begin{itemize}
% \item \hologo{pdfTeX} with |gettimeofday|: $\ge$ 1/65536\,s
% \item \hologo{pdfTeX} with |ftime|: $\ge$ 1\,ms
% \item \hologo{pdfTeX} with |time|: $\ge$ 1\,s
% \item \hologo{LuaTeX}: $\ge$ 10\,ms\\
%  (|os.clock()| returns a float number with two decimal digits in
%  \hologo{LuaTeX} beta-0.70.1-2011061416 (rev 4277)).
% \end{itemize}
%
% \subsubsection[Miscellaneous]{Miscellaneous \cite[``7.21 Miscellaneous'']{pdftex-manual}}
%
% \begin{declcs}{pdf@draftmode}
% \end{declcs}
% If the \TeX\ compiler knows \cs{pdfdraftmode} or \cs{draftmode}
% (\hologo{pdfTeX},
% \hologo{LuaTeX}), then \cs{pdf@draftmode} returns, whether
% this mode is enabled. The result is an implicit number:
% one means the draft mode is available and enabled.
% If the value is zero, then the mode is not active or
% \cs{pdfdraftmode} is not available.
% An explicit number is yielded by \cs{number}\cs{pdf@draftmode}.
% The macro cannot
% be used to change the mode, see \cs{pdf@setdraftmode}.
%
% \begin{declcs}{pdf@ifdraftmode} \M{true} \M{false}
% \end{declcs}
% If \cs{pdfdraftmode} is available and enabled, \meta{true} is
% called, otherwise \meta{false} is executed.
%
% \begin{declcs}{pdf@setdraftmode} \M{value}
% \end{declcs}
% Macro \cs{pdf@setdraftmode} expects the number zero or one as
% \meta{value}. Zero deactivates the mode and one enables the draft mode.
% The macro does not have an effect, if the feature \cs{pdfdraftmode} is not
% available.
%
% \begin{declcs}{pdf@shellescape}
% \end{declcs}
% Same as |\pdfshellescape|. It is or expands to |1| if external
% commands can be executed and |0| otherwise. In \hologo{pdfTeX} external
% commands must be enabled first by command line option or
% configuration option. In \hologo{LuaTeX} option |--safer| disables
% the execution of external commands.
%
% In \hologo{LuaTeX} before 0.68.0 \cs{pdf@shellescape} is not
% available due to a bug in |os.execute()|. The argumentless form
% crashes in some circumstances with segmentation fault.
% (It is fixed in version 0.68.0 or revision 4167 of \hologo{LuaTeX}.
% and packported to some version of 0.67.0).
%
% Hints for usage:
% \begin{itemize}
% \item Before its use \cs{pdf@shellescape} should be tested,
% whether it is available. Example with package \xpackage{ltxcmds}
% (loaded by package \xpackage{pdftexcmds}):
%\begin{quote}
%\begin{verbatim}
%\ltx@IfUndefined{pdf@shellescape}{%
%  % \pdf@shellescape is undefined
%}{%
%  % \pdf@shellescape is available
%}
%\end{verbatim}
%\end{quote}
% Use \cs{ltx@ifundefined} in expandable contexts.
% \item \cs{pdf@shellescape} might be a numerical constant,
% expands to the primitive, or expands to a plain number.
% Therefore use it in contexts where these differences does not matter.
% \item Use in comparisons, e.g.:
%   \begin{quote}
%     |\ifnum\pdf@shellescape=0 ...|
%   \end{quote}
% \item Print the number: |\number\pdf@shellescape|
% \end{itemize}
%
% \begin{declcs}{pdf@system} \M{cmdline}
% \end{declcs}
% It is a wrapper for |\immediate\write18| in \hologo{pdfTeX} or
% |os.execute| in \hologo{LuaTeX}.
%
% In theory |os.execute|
% returns a status number. But its meaning is quite
% undefined. Are there some reliable properties?
% Does it make sense to provide an user interface to
% this status exit code?
%
% \begin{declcs}{pdf@primitive} \csi{cmd}
% \end{declcs}
% Same as \cs{pdfprimitive} in \hologo{pdfTeX} or \hologo{LuaTeX}.
% In \hologo{XeTeX} the
% primitive is called \cs{primitive}. Despite the current definition
% of the command \csi{cmd}, it's meaning as primitive is used.
%
% \begin{declcs}{pdf@ifprimitive} \csi{cmd}
% \end{declcs}
% Same as \cs{ifpdfprimitive} in \hologo{pdfTeX} or
% \hologo{LuaTeX}. \hologo{XeTeX} calls
% it \cs{ifprimitive}. It is a switch that checks if the command
% \csi{cmd} has it's primitive meaning.
%
% \subsubsection{Additional macro: \cs{pdf@isprimitive}}
%
% \begin{declcs}{pdf@isprimitive} \csi{cmd1} \csi{cmd2} \M{true} \M{false}
% \end{declcs}
% If \csi{cmd1} has the primitive meaning given by the primitive name
% of \csi{cmd2}, then the argument \meta{true} is executed, otherwise
% \meta{false}. The macro \cs{pdf@isprimitive} is expandable.
% Internally it checks the result of \cs{meaning} and is therefore
% available for all \hologo{TeX} variants, even the original \hologo{TeX}.
% Example with \hologo{LaTeX}:
%\begin{quote}
%\begin{verbatim}
%\makeatletter
%\pdf@isprimitive{@@input}{input}{%
%  \typeout{\string\@@input\space is original\string\input}%
%}{%
%  \typeout{Oops, \string\@@input\space is not the %
%           original\string\input}%
%}
%\end{verbatim}
%\end{quote}
%
% \subsubsection{Experimental}
%
% \begin{declcs}{pdf@unescapehexnative} \M{string}\\
%   \cs{pdf@escapehexnative} \M{string}\\
%   \cs{pdf@escapenamenative} \M{string}\\
%   \cs{pdf@mdfivesumnative} \M{string}
% \end{declcs}
% The variants without |native| in the macro name are supposed to
% be compatible with \hologo{pdfTeX}. However characters with more than
% eight bits are not supported and are ignored. If \hologo{LuaTeX} is
% running, then its UTF-8 coded strings are used. Thus the full
% unicode character range is supported. However the result
% differs from \hologo{pdfTeX} for characters with eight or more bits.
%
% \begin{declcs}{pdf@pipe} \M{cmdline}
% \end{declcs}
% It calls \meta{cmdline} and returns the output of the external
% program in the usual manner as byte string (catcode 12, space with
% catcode 10). The Lua documentation says, that the used |io.popen|
% may not be available on all platforms. Then macro \cs{pdf@pipe}
% is undefined.
%
% \StopEventually{
% }
%
% \section{Implementation}
%
%    \begin{macrocode}
%<*package>
%    \end{macrocode}
%
% \subsection{Reload check and package identification}
%    Reload check, especially if the package is not used with \LaTeX.
%    \begin{macrocode}
\begingroup\catcode61\catcode48\catcode32=10\relax%
  \catcode13=5 % ^^M
  \endlinechar=13 %
  \catcode35=6 % #
  \catcode39=12 % '
  \catcode44=12 % ,
  \catcode45=12 % -
  \catcode46=12 % .
  \catcode58=12 % :
  \catcode64=11 % @
  \catcode123=1 % {
  \catcode125=2 % }
  \expandafter\let\expandafter\x\csname ver@pdftexcmds.sty\endcsname
  \ifx\x\relax % plain-TeX, first loading
  \else
    \def\empty{}%
    \ifx\x\empty % LaTeX, first loading,
      % variable is initialized, but \ProvidesPackage not yet seen
    \else
      \expandafter\ifx\csname PackageInfo\endcsname\relax
        \def\x#1#2{%
          \immediate\write-1{Package #1 Info: #2.}%
        }%
      \else
        \def\x#1#2{\PackageInfo{#1}{#2, stopped}}%
      \fi
      \x{pdftexcmds}{The package is already loaded}%
      \aftergroup\endinput
    \fi
  \fi
\endgroup%
%    \end{macrocode}
%    Package identification:
%    \begin{macrocode}
\begingroup\catcode61\catcode48\catcode32=10\relax%
  \catcode13=5 % ^^M
  \endlinechar=13 %
  \catcode35=6 % #
  \catcode39=12 % '
  \catcode40=12 % (
  \catcode41=12 % )
  \catcode44=12 % ,
  \catcode45=12 % -
  \catcode46=12 % .
  \catcode47=12 % /
  \catcode58=12 % :
  \catcode64=11 % @
  \catcode91=12 % [
  \catcode93=12 % ]
  \catcode123=1 % {
  \catcode125=2 % }
  \expandafter\ifx\csname ProvidesPackage\endcsname\relax
    \def\x#1#2#3[#4]{\endgroup
      \immediate\write-1{Package: #3 #4}%
      \xdef#1{#4}%
    }%
  \else
    \def\x#1#2[#3]{\endgroup
      #2[{#3}]%
      \ifx#1\@undefined
        \xdef#1{#3}%
      \fi
      \ifx#1\relax
        \xdef#1{#3}%
      \fi
    }%
  \fi
\expandafter\x\csname ver@pdftexcmds.sty\endcsname
\ProvidesPackage{pdftexcmds}%
  [2019/07/25 v0.30 Utility functions of pdfTeX for LuaTeX (HO)]%
%    \end{macrocode}
%
% \subsection{Catcodes}
%
%    \begin{macrocode}
\begingroup\catcode61\catcode48\catcode32=10\relax%
  \catcode13=5 % ^^M
  \endlinechar=13 %
  \catcode123=1 % {
  \catcode125=2 % }
  \catcode64=11 % @
  \def\x{\endgroup
    \expandafter\edef\csname pdftexcmds@AtEnd\endcsname{%
      \endlinechar=\the\endlinechar\relax
      \catcode13=\the\catcode13\relax
      \catcode32=\the\catcode32\relax
      \catcode35=\the\catcode35\relax
      \catcode61=\the\catcode61\relax
      \catcode64=\the\catcode64\relax
      \catcode123=\the\catcode123\relax
      \catcode125=\the\catcode125\relax
    }%
  }%
\x\catcode61\catcode48\catcode32=10\relax%
\catcode13=5 % ^^M
\endlinechar=13 %
\catcode35=6 % #
\catcode64=11 % @
\catcode123=1 % {
\catcode125=2 % }
\def\TMP@EnsureCode#1#2{%
  \edef\pdftexcmds@AtEnd{%
    \pdftexcmds@AtEnd
    \catcode#1=\the\catcode#1\relax
  }%
  \catcode#1=#2\relax
}
\TMP@EnsureCode{0}{12}%
\TMP@EnsureCode{1}{12}%
\TMP@EnsureCode{2}{12}%
\TMP@EnsureCode{10}{12}% ^^J
\TMP@EnsureCode{33}{12}% !
\TMP@EnsureCode{34}{12}% "
\TMP@EnsureCode{38}{4}% &
\TMP@EnsureCode{39}{12}% '
\TMP@EnsureCode{40}{12}% (
\TMP@EnsureCode{41}{12}% )
\TMP@EnsureCode{42}{12}% *
\TMP@EnsureCode{43}{12}% +
\TMP@EnsureCode{44}{12}% ,
\TMP@EnsureCode{45}{12}% -
\TMP@EnsureCode{46}{12}% .
\TMP@EnsureCode{47}{12}% /
\TMP@EnsureCode{58}{12}% :
\TMP@EnsureCode{60}{12}% <
\TMP@EnsureCode{62}{12}% >
\TMP@EnsureCode{91}{12}% [
\TMP@EnsureCode{93}{12}% ]
\TMP@EnsureCode{94}{7}% ^ (superscript)
\TMP@EnsureCode{95}{12}% _ (other)
\TMP@EnsureCode{96}{12}% `
\TMP@EnsureCode{126}{12}% ~ (other)
\edef\pdftexcmds@AtEnd{%
  \pdftexcmds@AtEnd
  \escapechar=\number\escapechar\relax
  \noexpand\endinput
}
\escapechar=92 %
%    \end{macrocode}
%
% \subsection{Load packages}
%
%    \begin{macrocode}
\begingroup\expandafter\expandafter\expandafter\endgroup
\expandafter\ifx\csname RequirePackage\endcsname\relax
  \def\TMP@RequirePackage#1[#2]{%
    \begingroup\expandafter\expandafter\expandafter\endgroup
    \expandafter\ifx\csname ver@#1.sty\endcsname\relax
      \input #1.sty\relax
    \fi
  }%
  \TMP@RequirePackage{infwarerr}[2007/09/09]%
  \TMP@RequirePackage{ifluatex}[2010/03/01]%
  \TMP@RequirePackage{ltxcmds}[2010/12/02]%
  \TMP@RequirePackage{ifpdf}[2010/09/13]%
\else
  \RequirePackage{infwarerr}[2007/09/09]%
  \RequirePackage{ifluatex}[2010/03/01]%
  \RequirePackage{ltxcmds}[2010/12/02]%
  \RequirePackage{ifpdf}[2010/09/13]%
\fi
%    \end{macrocode}
%
% \subsection{Without \hologo{LuaTeX}}
%
%    \begin{macrocode}
\ifluatex
\else
  \@PackageInfoNoLine{pdftexcmds}{LuaTeX not detected}%
  \def\pdftexcmds@nopdftex{%
    \@PackageInfoNoLine{pdftexcmds}{pdfTeX >= 1.30 not detected}%
    \let\pdftexcmds@nopdftex\relax
  }%
  \def\pdftexcmds@temp#1{%
    \begingroup\expandafter\expandafter\expandafter\endgroup
    \expandafter\ifx\csname pdf#1\endcsname\relax
      \pdftexcmds@nopdftex
    \else
      \expandafter\def\csname pdf@#1\expandafter\endcsname
      \expandafter##\expandafter{%
        \csname pdf#1\endcsname
      }%
    \fi
  }%
  \pdftexcmds@temp{strcmp}%
  \pdftexcmds@temp{escapehex}%
  \let\pdf@escapehexnative\pdf@escapehex
  \pdftexcmds@temp{unescapehex}%
  \let\pdf@unescapehexnative\pdf@unescapehex
  \pdftexcmds@temp{escapestring}%
  \pdftexcmds@temp{escapename}%
  \pdftexcmds@temp{filesize}%
  \pdftexcmds@temp{filemoddate}%
  \begingroup\expandafter\expandafter\expandafter\endgroup
  \expandafter\ifx\csname pdfshellescape\endcsname\relax
    \pdftexcmds@nopdftex
    \ltx@IfUndefined{pdftexversion}{%
    }{%
      \ifnum\pdftexversion>120 % 1.21a supports \ifeof18
        \ifeof18 %
          \chardef\pdf@shellescape=0 %
        \else
          \chardef\pdf@shellescape=1 %
        \fi
      \fi
    }%
  \else
    \def\pdf@shellescape{%
      \pdfshellescape
    }%
  \fi
  \begingroup\expandafter\expandafter\expandafter\endgroup
  \expandafter\ifx\csname pdffiledump\endcsname\relax
    \pdftexcmds@nopdftex
  \else
    \def\pdf@filedump#1#2#3{%
      \pdffiledump offset#1 length#2{#3}%
    }%
  \fi
%    \end{macrocode}
%    \begin{macrocode}
  \begingroup\expandafter\expandafter\expandafter\endgroup
  \expandafter\ifx\csname pdfmdfivesum\endcsname\relax
    \begingroup\expandafter\expandafter\expandafter\endgroup
    \expandafter\ifx\csname mdfivesum\endcsname\relax
      \pdftexcmds@nopdftex
    \else
      \def\pdf@mdfivesum#{\mdfivesum}%
      \let\pdf@mdfivesumnative\pdf@mdfivesum
      \def\pdf@filemdfivesum#{\mdfivesum file}%
    \fi
  \else
    \def\pdf@mdfivesum#{\pdfmdfivesum}%
    \let\pdf@mdfivesumnative\pdf@mdfivesum
    \def\pdf@filemdfivesum#{\pdfmdfivesum file}%
  \fi
%    \end{macrocode}
%    \begin{macrocode}
  \def\pdf@system#{%
    \immediate\write18%
  }%
  \def\pdftexcmds@temp#1{%
    \begingroup\expandafter\expandafter\expandafter\endgroup
    \expandafter\ifx\csname pdf#1\endcsname\relax
      \pdftexcmds@nopdftex
    \else
      \expandafter\let\csname pdf@#1\expandafter\endcsname
      \csname pdf#1\endcsname
    \fi
  }%
  \pdftexcmds@temp{resettimer}%
  \pdftexcmds@temp{elapsedtime}%
\fi
%    \end{macrocode}
%
% \subsection{\cs{pdf@primitive}, \cs{pdf@ifprimitive}}
%
%    Since version 1.40.0 \hologo{pdfTeX} has \cs{pdfprimitive} and
%    \cs{ifpdfprimitive}. And \cs{pdfprimitive} was fixed in
%    version 1.40.4.
%
%    \hologo{XeTeX} provides them under the name \cs{primitive} and
%    \cs{ifprimitive}. \hologo{LuaTeX} knows both name variants,
%    but they have possibly to be enabled first (|tex.enableprimitives|).
%
%    Depending on the format TeX Live uses a prefix |luatex|.
%
%    Caution: \cs{let} must be used for the definition of
%    the macros, especially because of \cs{ifpdfprimitive}.
%
% \subsubsection{Using \hologo{LuaTeX}'s \texttt{tex.enableprimitives}}
%
%    \begin{macrocode}
\ifluatex
%    \end{macrocode}
%    \begin{macro}{\pdftexcmds@directlua}
%    \begin{macrocode}
  \ifnum\luatexversion<36 %
    \def\pdftexcmds@directlua{\directlua0 }%
  \else
    \let\pdftexcmds@directlua\directlua
  \fi
%    \end{macrocode}
%    \end{macro}
%
%    \begin{macrocode}
  \begingroup
    \newlinechar=10 %
    \endlinechar=\newlinechar
    \pdftexcmds@directlua{%
      if tex.enableprimitives then
        tex.enableprimitives(
          'pdf@',
          {'primitive', 'ifprimitive', 'pdfdraftmode','draftmode'}
        )
        tex.enableprimitives('', {'luaescapestring'})
      end
    }%
  \endgroup %
%    \end{macrocode}
%
%    \begin{macrocode}
\fi
%    \end{macrocode}
%
% \subsubsection{Trying various names to find the primitives}
%
%    \begin{macro}{\pdftexcmds@strip@prefix}
%    \begin{macrocode}
\def\pdftexcmds@strip@prefix#1>{}
%    \end{macrocode}
%    \end{macro}
%    \begin{macrocode}
\def\pdftexcmds@temp#1#2#3{%
  \begingroup\expandafter\expandafter\expandafter\endgroup
  \expandafter\ifx\csname pdf@#1\endcsname\relax
    \begingroup
      \def\x{#3}%
      \edef\x{\expandafter\pdftexcmds@strip@prefix\meaning\x}%
      \escapechar=-1 %
      \edef\y{\expandafter\meaning\csname#2\endcsname}%
    \expandafter\endgroup
    \ifx\x\y
      \expandafter\let\csname pdf@#1\expandafter\endcsname
      \csname #2\endcsname
    \fi
  \fi
}
%    \end{macrocode}
%
%    \begin{macro}{\pdf@primitive}
%    \begin{macrocode}
\pdftexcmds@temp{primitive}{pdfprimitive}{pdfprimitive}% pdfTeX, oldLuaTeX
\pdftexcmds@temp{primitive}{primitive}{primitive}% XeTeX, luatex
\pdftexcmds@temp{primitive}{luatexprimitive}{pdfprimitive}% oldLuaTeX
\pdftexcmds@temp{primitive}{luatexpdfprimitive}{pdfprimitive}% oldLuaTeX
%    \end{macrocode}
%    \end{macro}
%    \begin{macro}{\pdf@ifprimitive}
%    \begin{macrocode}
\pdftexcmds@temp{ifprimitive}{ifpdfprimitive}{ifpdfprimitive}% pdfTeX, oldLuaTeX
\pdftexcmds@temp{ifprimitive}{ifprimitive}{ifprimitive}% XeTeX, luatex
\pdftexcmds@temp{ifprimitive}{luatexifprimitive}{ifpdfprimitive}% oldLuaTeX
\pdftexcmds@temp{ifprimitive}{luatexifpdfprimitive}{ifpdfprimitive}% oldLuaTeX
%    \end{macrocode}
%    \end{macro}
%
%    Disable broken \cs{pdfprimitive}.
%    \begin{macrocode}
\ifluatex\else
\begingroup
  \expandafter\ifx\csname pdf@primitive\endcsname\relax
  \else
    \expandafter\ifx\csname pdftexversion\endcsname\relax
    \else
      \ifnum\pdftexversion=140 %
        \expandafter\ifx\csname pdftexrevision\endcsname\relax
        \else
          \ifnum\pdftexrevision<4 %
            \endgroup
            \let\pdf@primitive\@undefined
            \@PackageInfoNoLine{pdftexcmds}{%
              \string\pdf@primitive\space disabled, %
              because\MessageBreak
              \string\pdfprimitive\space is broken until pdfTeX 1.40.4%
            }%
            \begingroup
          \fi
        \fi
      \fi
    \fi
  \fi
\endgroup
\fi
%    \end{macrocode}
%
% \subsubsection{Result}
%
%    \begin{macrocode}
\begingroup
  \@PackageInfoNoLine{pdftexcmds}{%
    \string\pdf@primitive\space is %
    \expandafter\ifx\csname pdf@primitive\endcsname\relax not \fi
    available%
  }%
  \@PackageInfoNoLine{pdftexcmds}{%
    \string\pdf@ifprimitive\space is %
    \expandafter\ifx\csname pdf@ifprimitive\endcsname\relax not \fi
    available%
  }%
\endgroup
%    \end{macrocode}
%
% \subsection{\hologo{XeTeX}}
%
%    Look for primitives \cs{shellescape}, \cs{strcmp}.
%    \begin{macrocode}
\def\pdftexcmds@temp#1{%
  \begingroup\expandafter\expandafter\expandafter\endgroup
  \expandafter\ifx\csname pdf@#1\endcsname\relax
    \begingroup
      \escapechar=-1 %
      \edef\x{\expandafter\meaning\csname#1\endcsname}%
      \def\y{#1}%
      \def\z##1->{}%
      \edef\y{\expandafter\z\meaning\y}%
    \expandafter\endgroup
    \ifx\x\y
      \expandafter\def\csname pdf@#1\expandafter\endcsname
      \expandafter{%
        \csname#1\endcsname
      }%
    \fi
  \fi
}%
\pdftexcmds@temp{shellescape}%
\pdftexcmds@temp{strcmp}%
%    \end{macrocode}
%
% \subsection{\cs{pdf@isprimitive}}
%
%    \begin{macrocode}
\def\pdf@isprimitive{%
  \begingroup\expandafter\expandafter\expandafter\endgroup
  \expandafter\ifx\csname pdf@strcmp\endcsname\relax
    \long\def\pdf@isprimitive##1{%
      \expandafter\pdftexcmds@isprimitive\expandafter{\meaning##1}%
    }%
    \long\def\pdftexcmds@isprimitive##1##2{%
      \expandafter\pdftexcmds@@isprimitive\expandafter{\string##2}{##1}%
    }%
    \def\pdftexcmds@@isprimitive##1##2{%
      \ifnum0\pdftexcmds@equal##1\delimiter##2\delimiter=1 %
        \expandafter\ltx@firstoftwo
      \else
        \expandafter\ltx@secondoftwo
      \fi
    }%
    \def\pdftexcmds@equal##1##2\delimiter##3##4\delimiter{%
      \ifx##1##3%
        \ifx\relax##2##4\relax
          1%
        \else
          \ifx\relax##2\relax
          \else
            \ifx\relax##4\relax
            \else
              \pdftexcmds@equalcont{##2}{##4}%
            \fi
          \fi
        \fi
      \fi
    }%
    \def\pdftexcmds@equalcont##1{%
      \def\pdftexcmds@equalcont####1####2##1##1##1##1{%
        ##1##1##1##1%
        \pdftexcmds@equal####1\delimiter####2\delimiter
      }%
    }%
    \expandafter\pdftexcmds@equalcont\csname fi\endcsname
  \else
    \long\def\pdf@isprimitive##1##2{%
      \ifnum\pdf@strcmp{\meaning##1}{\string##2}=0 %
        \expandafter\ltx@firstoftwo
      \else
        \expandafter\ltx@secondoftwo
      \fi
    }%
  \fi
}
\ifluatex
\ifx\pdfdraftmode\@undefined
  \let\pdfdraftmode\draftmode
\fi
\else
  \pdf@isprimitive
\fi
%    \end{macrocode}
%
% \subsection{\cs{pdf@draftmode}}
%
%
%    \begin{macrocode}
\let\pdftexcmds@temp\ltx@zero %
\ltx@IfUndefined{pdfdraftmode}{%
  \@PackageInfoNoLine{pdftexcmds}{\ltx@backslashchar pdfdraftmode not found}%
}{%
  \ifpdf
    \let\pdftexcmds@temp\ltx@one
    \@PackageInfoNoLine{pdftexcmds}{\ltx@backslashchar pdfdraftmode found}%
  \else
    \@PackageInfoNoLine{pdftexcmds}{%
      \ltx@backslashchar pdfdraftmode is ignored in DVI mode%
    }%
  \fi
}
\ifcase\pdftexcmds@temp
%    \end{macrocode}
%    \begin{macro}{\pdf@draftmode}
%    \begin{macrocode}
  \let\pdf@draftmode\ltx@zero
%    \end{macrocode}
%    \end{macro}
%    \begin{macro}{\pdf@ifdraftmode}
%    \begin{macrocode}
  \let\pdf@ifdraftmode\ltx@secondoftwo
%    \end{macrocode}
%    \end{macro}
%    \begin{macro}{\pdftexcmds@setdraftmode}
%    \begin{macrocode}
  \def\pdftexcmds@setdraftmode#1{}%
%    \end{macrocode}
%    \end{macro}
%    \begin{macrocode}
\else
%    \end{macrocode}
%    \begin{macro}{\pdftexcmds@draftmode}
%    \begin{macrocode}
  \let\pdftexcmds@draftmode\pdfdraftmode
%    \end{macrocode}
%    \end{macro}
%    \begin{macro}{\pdf@ifdraftmode}
%    \begin{macrocode}
  \def\pdf@ifdraftmode{%
    \ifnum\pdftexcmds@draftmode=\ltx@one
      \expandafter\ltx@firstoftwo
    \else
      \expandafter\ltx@secondoftwo
    \fi
  }%
%    \end{macrocode}
%    \end{macro}
%    \begin{macro}{\pdf@draftmode}
%    \begin{macrocode}
  \def\pdf@draftmode{%
    \ifnum\pdftexcmds@draftmode=\ltx@one
      \expandafter\ltx@one
    \else
      \expandafter\ltx@zero
    \fi
  }%
%    \end{macrocode}
%    \end{macro}
%    \begin{macro}{\pdftexcmds@setdraftmode}
%    \begin{macrocode}
  \def\pdftexcmds@setdraftmode#1{%
    \pdftexcmds@draftmode=#1\relax
  }%
%    \end{macrocode}
%    \end{macro}
%    \begin{macrocode}
\fi
%    \end{macrocode}
%    \begin{macro}{\pdf@setdraftmode}
%    \begin{macrocode}
\def\pdf@setdraftmode#1{%
  \begingroup
    \count\ltx@cclv=#1\relax
  \edef\x{\endgroup
    \noexpand\pdftexcmds@@setdraftmode{\the\count\ltx@cclv}%
  }%
  \x
}
%    \end{macrocode}
%    \end{macro}
%    \begin{macro}{\pdftexcmds@@setdraftmode}
%    \begin{macrocode}
\def\pdftexcmds@@setdraftmode#1{%
  \ifcase#1 %
    \pdftexcmds@setdraftmode{#1}%
  \or
    \pdftexcmds@setdraftmode{#1}%
  \else
    \@PackageWarning{pdftexcmds}{%
      \string\pdf@setdraftmode: Ignoring\MessageBreak
      invalid value `#1'%
    }%
  \fi
}
%    \end{macrocode}
%    \end{macro}
%
% \subsection{Load Lua module}
%
%    \begin{macrocode}
\ifluatex
\else
  \expandafter\pdftexcmds@AtEnd
\fi%
%    \end{macrocode}
%
%    \begin{macrocode}
\ifnum\luatexversion<80
  \begingroup\expandafter\expandafter\expandafter\endgroup
  \expandafter\ifx\csname RequirePackage\endcsname\relax
    \def\TMP@RequirePackage#1[#2]{%
      \begingroup\expandafter\expandafter\expandafter\endgroup
      \expandafter\ifx\csname ver@#1.sty\endcsname\relax
        \input #1.sty\relax
      \fi
    }%
    \TMP@RequirePackage{luatex-loader}[2009/04/10]%
  \else
    \RequirePackage{luatex-loader}[2009/04/10]%
  \fi
\fi
\pdftexcmds@directlua{%
  require("pdftexcmds")%
}
\ifnum\luatexversion>37 %
  \ifnum0%
      \pdftexcmds@directlua{%
        if status.ini_version then %
          tex.write("1")%
        end%
      }>0 %
    \everyjob\expandafter{%
      \the\everyjob
      \pdftexcmds@directlua{%
        require("pdftexcmds")%
      }%
    }%
  \fi
\fi
\begingroup
  \def\x{2019/07/25 v0.30}%
  \ltx@onelevel@sanitize\x
  \edef\y{%
    \pdftexcmds@directlua{%
      if oberdiek.pdftexcmds.getversion then %
        oberdiek.pdftexcmds.getversion()%
      end%
    }%
  }%
  \ifx\x\y
  \else
    \@PackageError{pdftexcmds}{%
      Wrong version of lua module.\MessageBreak
      Package version: \x\MessageBreak
      Lua module: \y
    }\@ehc
  \fi
\endgroup
%    \end{macrocode}
%
% \subsection{Lua functions}
%
% \subsubsection{Helper macros}
%
%    \begin{macro}{\pdftexcmds@toks}
%    \begin{macrocode}
\begingroup\expandafter\expandafter\expandafter\endgroup
\expandafter\ifx\csname newtoks\endcsname\relax
  \toksdef\pdftexcmds@toks=0 %
\else
  \csname newtoks\endcsname\pdftexcmds@toks
\fi
%    \end{macrocode}
%    \end{macro}
%
%    \begin{macro}{\pdftexcmds@Patch}
%    \begin{macrocode}
\def\pdftexcmds@Patch{0}
\ifnum\luatexversion>40 %
  \ifnum\luatexversion<66 %
    \def\pdftexcmds@Patch{1}%
  \fi
\fi
%    \end{macrocode}
%    \end{macro}
%    \begin{macrocode}
\ifcase\pdftexcmds@Patch
  \catcode`\&=14 %
\else
  \catcode`\&=9 %
%    \end{macrocode}
%    \begin{macro}{\pdftexcmds@PatchDecode}
%    \begin{macrocode}
  \def\pdftexcmds@PatchDecode#1\@nil{%
    \pdftexcmds@DecodeA#1^^A^^A\@nil{}%
  }%
%    \end{macrocode}
%    \end{macro}
%    \begin{macro}{\pdftexcmds@DecodeA}
%    \begin{macrocode}
  \def\pdftexcmds@DecodeA#1^^A^^A#2\@nil#3{%
    \ifx\relax#2\relax
      \ltx@ReturnAfterElseFi{%
        \pdftexcmds@DecodeB#3#1^^A^^B\@nil{}%
      }%
    \else
      \ltx@ReturnAfterFi{%
        \pdftexcmds@DecodeA#2\@nil{#3#1^^@}%
      }%
    \fi
  }%
%    \end{macrocode}
%    \end{macro}
%    \begin{macro}{\pdftexcmds@DecodeB}
%    \begin{macrocode}
  \def\pdftexcmds@DecodeB#1^^A^^B#2\@nil#3{%
    \ifx\relax#2\relax%
      \ltx@ReturnAfterElseFi{%
        \ltx@zero
        #3#1%
      }%
    \else
      \ltx@ReturnAfterFi{%
        \pdftexcmds@DecodeB#2\@nil{#3#1^^A}%
      }%
    \fi
  }%
%    \end{macrocode}
%    \end{macro}
%    \begin{macrocode}
\fi
%    \end{macrocode}
%
%    \begin{macrocode}
\ifnum\luatexversion<36 %
\else
  \catcode`\0=9 %
\fi
%    \end{macrocode}
%
% \subsubsection[Strings]{Strings \cite[``7.15 Strings'']{pdftex-manual}}
%
%    \begin{macro}{\pdf@strcmp}
%    \begin{macrocode}
\long\def\pdf@strcmp#1#2{%
  \directlua0{%
    oberdiek.pdftexcmds.strcmp("\luaescapestring{#1}",%
        "\luaescapestring{#2}")%
  }%
}%
%    \end{macrocode}
%    \end{macro}
%    \begin{macrocode}
\pdf@isprimitive
%    \end{macrocode}
%    \begin{macro}{\pdf@escapehex}
%    \begin{macrocode}
\long\def\pdf@escapehex#1{%
  \directlua0{%
    oberdiek.pdftexcmds.escapehex("\luaescapestring{#1}", "byte")%
  }%
}%
%    \end{macrocode}
%    \end{macro}
%    \begin{macro}{\pdf@escapehexnative}
%    \begin{macrocode}
\long\def\pdf@escapehexnative#1{%
  \directlua0{%
    oberdiek.pdftexcmds.escapehex("\luaescapestring{#1}")%
  }%
}%
%    \end{macrocode}
%    \end{macro}
%    \begin{macro}{\pdf@unescapehex}
%    \begin{macrocode}
\def\pdf@unescapehex#1{%
& \romannumeral\expandafter\pdftexcmds@PatchDecode
  \the\expandafter\pdftexcmds@toks
  \directlua0{%
    oberdiek.pdftexcmds.toks="pdftexcmds@toks"%
    oberdiek.pdftexcmds.unescapehex("\luaescapestring{#1}", "byte", \pdftexcmds@Patch)%
  }%
& \@nil
}%
%    \end{macrocode}
%    \end{macro}
%    \begin{macro}{\pdf@unescapehexnative}
%    \begin{macrocode}
\def\pdf@unescapehexnative#1{%
& \romannumeral\expandafter\pdftexcmds@PatchDecode
  \the\expandafter\pdftexcmds@toks
  \directlua0{%
    oberdiek.pdftexcmds.toks="pdftexcmds@toks"%
    oberdiek.pdftexcmds.unescapehex("\luaescapestring{#1}", \pdftexcmds@Patch)%
  }%
& \@nil
}%
%    \end{macrocode}
%    \end{macro}
%    \begin{macro}{\pdf@escapestring}
%    \begin{macrocode}
\long\def\pdf@escapestring#1{%
  \directlua0{%
    oberdiek.pdftexcmds.escapestring("\luaescapestring{#1}", "byte")%
  }%
}
%    \end{macrocode}
%    \end{macro}
%    \begin{macro}{\pdf@escapename}
%    \begin{macrocode}
\long\def\pdf@escapename#1{%
  \directlua0{%
    oberdiek.pdftexcmds.escapename("\luaescapestring{#1}", "byte")%
  }%
}
%    \end{macrocode}
%    \end{macro}
%    \begin{macro}{\pdf@escapenamenative}
%    \begin{macrocode}
\long\def\pdf@escapenamenative#1{%
  \directlua0{%
    oberdiek.pdftexcmds.escapename("\luaescapestring{#1}")%
  }%
}
%    \end{macrocode}
%    \end{macro}
%
% \subsubsection[Files]{Files \cite[``7.18 Files'']{pdftex-manual}}
%
%    \begin{macro}{\pdf@filesize}
%    \begin{macrocode}
\def\pdf@filesize#1{%
  \directlua0{%
    oberdiek.pdftexcmds.filesize("\luaescapestring{#1}")%
  }%
}
%    \end{macrocode}
%    \end{macro}
%    \begin{macro}{\pdf@filemoddate}
%    \begin{macrocode}
\def\pdf@filemoddate#1{%
  \directlua0{%
    oberdiek.pdftexcmds.filemoddate("\luaescapestring{#1}")%
  }%
}
%    \end{macrocode}
%    \end{macro}
%    \begin{macro}{\pdf@filedump}
%    \begin{macrocode}
\def\pdf@filedump#1#2#3{%
  \directlua0{%
    oberdiek.pdftexcmds.filedump("\luaescapestring{\number#1}",%
        "\luaescapestring{\number#2}",%
        "\luaescapestring{#3}")%
  }%
}%
%    \end{macrocode}
%    \end{macro}
%    \begin{macro}{\pdf@mdfivesum}
%    \begin{macrocode}
\long\def\pdf@mdfivesum#1{%
  \directlua0{%
    oberdiek.pdftexcmds.mdfivesum("\luaescapestring{#1}", "byte")%
  }%
}%
%    \end{macrocode}
%    \end{macro}
%    \begin{macro}{\pdf@mdfivesumnative}
%    \begin{macrocode}
\long\def\pdf@mdfivesumnative#1{%
  \directlua0{%
    oberdiek.pdftexcmds.mdfivesum("\luaescapestring{#1}")%
  }%
}%
%    \end{macrocode}
%    \end{macro}
%    \begin{macro}{\pdf@filemdfivesum}
%    \begin{macrocode}
\def\pdf@filemdfivesum#1{%
  \directlua0{%
    oberdiek.pdftexcmds.filemdfivesum("\luaescapestring{#1}")%
  }%
}%
%    \end{macrocode}
%    \end{macro}
%
% \subsubsection[Timekeeping]{Timekeeping \cite[``7.17 Timekeeping'']{pdftex-manual}}
%
%    \begin{macro}{\protected}
%    \begin{macrocode}
\let\pdftexcmds@temp=Y%
\begingroup\expandafter\expandafter\expandafter\endgroup
\expandafter\ifx\csname protected\endcsname\relax
  \pdftexcmds@directlua0{%
    if tex.enableprimitives then %
      tex.enableprimitives('', {'protected'})%
    end%
  }%
\fi
\begingroup\expandafter\expandafter\expandafter\endgroup
\expandafter\ifx\csname protected\endcsname\relax
  \let\pdftexcmds@temp=N%
\fi
%    \end{macrocode}
%    \end{macro}
%    \begin{macro}{\numexpr}
%    \begin{macrocode}
\begingroup\expandafter\expandafter\expandafter\endgroup
\expandafter\ifx\csname numexpr\endcsname\relax
  \pdftexcmds@directlua0{%
    if tex.enableprimitives then %
      tex.enableprimitives('', {'numexpr'})%
    end%
  }%
\fi
\begingroup\expandafter\expandafter\expandafter\endgroup
\expandafter\ifx\csname numexpr\endcsname\relax
  \let\pdftexcmds@temp=N%
\fi
%    \end{macrocode}
%    \end{macro}
%
%    \begin{macrocode}
\ifx\pdftexcmds@temp N%
  \@PackageWarningNoLine{pdftexcmds}{%
    Definitions of \ltx@backslashchar pdf@resettimer and%
    \MessageBreak
    \ltx@backslashchar pdf@elapsedtime are skipped, because%
    \MessageBreak
    e-TeX's \ltx@backslashchar protected or %
    \ltx@backslashchar numexpr are missing%
  }%
\else
%    \end{macrocode}
%
%    \begin{macro}{\pdf@resettimer}
%    \begin{macrocode}
  \protected\def\pdf@resettimer{%
    \pdftexcmds@directlua0{%
      oberdiek.pdftexcmds.resettimer()%
    }%
  }%
%    \end{macrocode}
%    \end{macro}
%
%    \begin{macro}{\pdf@elapsedtime}
%    \begin{macrocode}
  \protected\def\pdf@elapsedtime{%
    \numexpr
      \pdftexcmds@directlua0{%
        oberdiek.pdftexcmds.elapsedtime()%
      }%
    \relax
  }%
%    \end{macrocode}
%    \end{macro}
%    \begin{macrocode}
\fi
%    \end{macrocode}
%
% \subsubsection{Shell escape}
%
%    \begin{macro}{\pdf@shellescape}
%
%    \begin{macrocode}
\ifnum\luatexversion<68 %
\else
  \protected\edef\pdf@shellescape{%
   \numexpr\directlua{tex.sprint(%
         \number\catcodetable@string,status.shell_escape)}\relax}
\fi
%    \end{macrocode}
%    \end{macro}
%
%    \begin{macro}{\pdf@system}
%    \begin{macrocode}
\def\pdf@system#1{%
  \directlua0{%
    oberdiek.pdftexcmds.system("\luaescapestring{#1}")%
  }%
}
%    \end{macrocode}
%    \end{macro}
%
%    \begin{macro}{\pdf@lastsystemstatus}
%    \begin{macrocode}
\def\pdf@lastsystemstatus{%
  \directlua0{%
    oberdiek.pdftexcmds.lastsystemstatus()%
  }%
}
%    \end{macrocode}
%    \end{macro}
%    \begin{macro}{\pdf@lastsystemexit}
%    \begin{macrocode}
\def\pdf@lastsystemexit{%
  \directlua0{%
    oberdiek.pdftexcmds.lastsystemexit()%
  }%
}
%    \end{macrocode}
%    \end{macro}
%
%    \begin{macrocode}
\catcode`\0=12 %
%    \end{macrocode}
%
%    \begin{macro}{\pdf@pipe}
%    Check availability of |io.popen| first.
%    \begin{macrocode}
\ifnum0%
    \pdftexcmds@directlua{%
      if io.popen then %
        tex.write("1")%
      end%
    }%
    =1 %
  \def\pdf@pipe#1{%
&   \romannumeral\expandafter\pdftexcmds@PatchDecode
    \the\expandafter\pdftexcmds@toks
    \pdftexcmds@directlua{%
      oberdiek.pdftexcmds.toks="pdftexcmds@toks"%
      oberdiek.pdftexcmds.pipe("\luaescapestring{#1}", \pdftexcmds@Patch)%
    }%
&   \@nil
  }%
\fi
%    \end{macrocode}
%    \end{macro}
%
%    \begin{macrocode}
\pdftexcmds@AtEnd%
%</package>
%    \end{macrocode}
%
% \subsection{Lua module}
%
%    \begin{macrocode}
%<*lua>
%    \end{macrocode}
%
%    \begin{macrocode}
oberdiek = oberdiek or {}
local pdftexcmds = oberdiek.pdftexcmds or {}
oberdiek.pdftexcmds = pdftexcmds
local systemexitstatus
function pdftexcmds.getversion()
  tex.write("2019/07/25 v0.30")
end
%    \end{macrocode}
%
% \subsubsection[Strings]{Strings \cite[``7.15 Strings'']{pdftex-manual}}
%
%    \begin{macrocode}
function pdftexcmds.strcmp(A, B)
  if A == B then
    tex.write("0")
  elseif A < B then
    tex.write("-1")
  else
    tex.write("1")
  end
end
local function utf8_to_byte(str)
  local i = 0
  local n = string.len(str)
  local t = {}
  while i < n do
    i = i + 1
    local a = string.byte(str, i)
    if a < 128 then
      table.insert(t, string.char(a))
    else
      if a >= 192 and i < n then
        i = i + 1
        local b = string.byte(str, i)
        if b < 128 or b >= 192 then
          i = i - 1
        elseif a == 194 then
          table.insert(t, string.char(b))
        elseif a == 195 then
          table.insert(t, string.char(b + 64))
        end
      end
    end
  end
  return table.concat(t)
end
function pdftexcmds.escapehex(str, mode)
  if mode == "byte" then
    str = utf8_to_byte(str)
  end
  tex.write((string.gsub(str, ".",
    function (ch)
      return string.format("%02X", string.byte(ch))
    end
  )))
end
%    \end{macrocode}
%    See procedure |unescapehex| in file \xfile{utils.c} of \hologo{pdfTeX}.
%    Caution: |tex.write| ignores leading spaces.
%    \begin{macrocode}
function pdftexcmds.unescapehex(str, mode, patch)
  local a = 0
  local first = true
  local result = {}
  for i = 1, string.len(str), 1 do
    local ch = string.byte(str, i)
    if ch >= 48 and ch <= 57 then
      ch = ch - 48
    elseif ch >= 65 and ch <= 70 then
      ch = ch - 55
    elseif ch >= 97 and ch <= 102 then
      ch = ch - 87
    else
      ch = nil
    end
    if ch then
      if first then
        a = ch * 16
        first = false
      else
        table.insert(result, a + ch)
        first = true
      end
    end
  end
  if not first then
    table.insert(result, a)
  end
  if patch == 1 then
    local temp = {}
    for i, a in ipairs(result) do
      if a == 0 then
        table.insert(temp, 1)
        table.insert(temp, 1)
      else
        if a == 1 then
          table.insert(temp, 1)
          table.insert(temp, 2)
        else
          table.insert(temp, a)
        end
      end
    end
    result = temp
  end
  if mode == "byte" then
    local utf8 = {}
    for i, a in ipairs(result) do
      if a < 128 then
        table.insert(utf8, a)
      else
        if a < 192 then
          table.insert(utf8, 194)
          a = a - 128
        else
          table.insert(utf8, 195)
          a = a - 192
        end
        table.insert(utf8, a + 128)
      end
    end
    result = utf8
  end
%    \end{macrocode}
%    this next line added for current luatex; this is the only
%    change in the file.  eroux, 28apr13. (v 0.21)
%    \begin{macrocode}
  local unpack = _G["unpack"] or table.unpack
  tex.settoks(pdftexcmds.toks, string.char(unpack(result)))
end
%    \end{macrocode}
%    See procedure |escapestring| in file \xfile{utils.c} of \hologo{pdfTeX}.
%    \begin{macrocode}
function pdftexcmds.escapestring(str, mode)
  if mode == "byte" then
    str = utf8_to_byte(str)
  end
  tex.write((string.gsub(str, ".",
    function (ch)
      local b = string.byte(ch)
      if b < 33 or b > 126 then
        return string.format("\\%.3o", b)
      end
      if b == 40 or b == 41 or b == 92 then
        return "\\" .. ch
      end
%    \end{macrocode}
%    Lua 5.1 returns the match in case of return value |nil|.
%    \begin{macrocode}
      return nil
    end
  )))
end
%    \end{macrocode}
%    See procedure |escapename| in file \xfile{utils.c} of \hologo{pdfTeX}.
%    \begin{macrocode}
function pdftexcmds.escapename(str, mode)
  if mode == "byte" then
    str = utf8_to_byte(str)
  end
  tex.write((string.gsub(str, ".",
    function (ch)
      local b = string.byte(ch)
      if b == 0 then
%    \end{macrocode}
%    In Lua 5.0 |nil| could be used for the empty string,
%    But |nil| returns the match in Lua 5.1, thus we use
%    the empty string explicitly.
%    \begin{macrocode}
        return ""
      end
      if b <= 32 or b >= 127
          or b == 35 or b == 37 or b == 40 or b == 41
          or b == 47 or b == 60 or b == 62 or b == 91
          or b == 93 or b == 123 or b == 125 then
        return string.format("#%.2X", b)
      else
%    \end{macrocode}
%    Lua 5.1 returns the match in case of return value |nil|.
%    \begin{macrocode}
        return nil
      end
    end
  )))
end
%    \end{macrocode}
%
% \subsubsection[Files]{Files \cite[``7.18 Files'']{pdftex-manual}}
%
%    \begin{macrocode}
function pdftexcmds.filesize(filename)
  local foundfile = kpse.find_file(filename, "tex", true)
  if foundfile then
    local size = lfs.attributes(foundfile, "size")
    if size then
      tex.write(size)
    end
  end
end
%    \end{macrocode}
%    See procedure |makepdftime| in file \xfile{utils.c} of \hologo{pdfTeX}.
%    \begin{macrocode}
function pdftexcmds.filemoddate(filename)
  local foundfile = kpse.find_file(filename, "tex", true)
  if foundfile then
    local date = lfs.attributes(foundfile, "modification")
    if date then
      local d = os.date("*t", date)
      if d.sec >= 60 then
        d.sec = 59
      end
      local u = os.date("!*t", date)
      local off = 60 * (d.hour - u.hour) + d.min - u.min
      if d.year ~= u.year then
        if d.year > u.year then
          off = off + 1440
        else
          off = off - 1440
        end
      elseif d.yday ~= u.yday then
        if d.yday > u.yday then
          off = off + 1440
        else
          off = off - 1440
        end
      end
      local timezone
      if off == 0 then
        timezone = "Z"
      else
        local hours = math.floor(off / 60)
        local mins = math.abs(off - hours * 60)
        timezone = string.format("%+03d'%02d'", hours, mins)
      end
      tex.write(string.format("D:%04d%02d%02d%02d%02d%02d%s",
          d.year, d.month, d.day, d.hour, d.min, d.sec, timezone))
    end
  end
end
function pdftexcmds.filedump(offset, length, filename)
  length = tonumber(length)
  if length and length > 0 then
    local foundfile = kpse.find_file(filename, "tex", true)
    if foundfile then
      offset = tonumber(offset)
      if not offset then
        offset = 0
      end
      local filehandle = io.open(foundfile, "rb")
      if filehandle then
        if offset > 0 then
          filehandle:seek("set", offset)
        end
        local dump = filehandle:read(length)
        pdftexcmds.escapehex(dump)
        filehandle:close()
      end
    end
  end
end
function pdftexcmds.mdfivesum(str, mode)
  if mode == "byte" then
    str = utf8_to_byte(str)
  end
  pdftexcmds.escapehex(md5.sum(str))
end
function pdftexcmds.filemdfivesum(filename)
  local foundfile = kpse.find_file(filename, "tex", true)
  if foundfile then
    local filehandle = io.open(foundfile, "rb")
    if filehandle then
      local contents = filehandle:read("*a")
      pdftexcmds.escapehex(md5.sum(contents))
      filehandle:close()
    end
  end
end
%    \end{macrocode}
%
% \subsubsection[Timekeeping]{Timekeeping \cite[``7.17 Timekeeping'']{pdftex-manual}}
%
%    The functions for timekeeping are based on
%    Andy Thomas' work \cite{AndyThomas:Analog}.
%    Changes:
%    \begin{itemize}
%    \item Overflow check is added.
%    \item |string.format| is used to avoid exponential number
%          representation for sure.
%    \item |tex.write| is used instead of |tex.print| to get
%          tokens with catcode 12 and without appended \cs{endlinechar}.
%    \end{itemize}
%    \begin{macrocode}
local basetime = 0
function pdftexcmds.resettimer()
  basetime = os.clock()
end
function pdftexcmds.elapsedtime()
  local val = (os.clock() - basetime) * 65536 + .5
  if val > 2147483647 then
    val = 2147483647
  end
  tex.write(string.format("%d", val))
end
%    \end{macrocode}
%
% \subsubsection[Miscellaneous]{Miscellaneous \cite[``7.21 Miscellaneous'']{pdftex-manual}}
%
%    \begin{macrocode}
function pdftexcmds.shellescape()
  if os.execute then
    if status
        and status.luatex_version
        and status.luatex_version >= 68 then
      tex.write(os.execute())
    else
      local result = os.execute()
      if result == 0 then
        tex.write("0")
      else
        if result == nil then
          tex.write("0")
        else
          tex.write("1")
        end
      end
    end
  else
    tex.write("0")
  end
end
function pdftexcmds.system(cmdline)
  systemexitstatus = nil
  texio.write_nl("log", "system(" .. cmdline .. ") ")
  if os.execute then
    texio.write("log", "executed.")
    systemexitstatus = os.execute(cmdline)
  else
    texio.write("log", "disabled.")
  end
end
function pdftexcmds.lastsystemstatus()
  local result = tonumber(systemexitstatus)
  if result then
    local x = math.floor(result / 256)
    tex.write(result - 256 * math.floor(result / 256))
  end
end
function pdftexcmds.lastsystemexit()
  local result = tonumber(systemexitstatus)
  if result then
    tex.write(math.floor(result / 256))
  end
end
function pdftexcmds.pipe(cmdline, patch)
  local result
  systemexitstatus = nil
  texio.write_nl("log", "pipe(" .. cmdline ..") ")
  if io.popen then
    texio.write("log", "executed.")
    local handle = io.popen(cmdline, "r")
    if handle then
      result = handle:read("*a")
      handle:close()
    end
  else
    texio.write("log", "disabled.")
  end
  if result then
    if patch == 1 then
      local temp = {}
      for i, a in ipairs(result) do
        if a == 0 then
          table.insert(temp, 1)
          table.insert(temp, 1)
        else
          if a == 1 then
            table.insert(temp, 1)
            table.insert(temp, 2)
          else
            table.insert(temp, a)
          end
        end
      end
      result = temp
    end
    tex.settoks(pdftexcmds.toks, result)
  else
    tex.settoks(pdftexcmds.toks, "")
  end
end
%    \end{macrocode}
%    \begin{macrocode}
%</lua>
%    \end{macrocode}
%
% \section{Test}
%
% \subsection{Catcode checks for loading}
%
%    \begin{macrocode}
%<*test1>
%    \end{macrocode}
%    \begin{macrocode}
\catcode`\{=1 %
\catcode`\}=2 %
\catcode`\#=6 %
\catcode`\@=11 %
\expandafter\ifx\csname count@\endcsname\relax
  \countdef\count@=255 %
\fi
\expandafter\ifx\csname @gobble\endcsname\relax
  \long\def\@gobble#1{}%
\fi
\expandafter\ifx\csname @firstofone\endcsname\relax
  \long\def\@firstofone#1{#1}%
\fi
\expandafter\ifx\csname loop\endcsname\relax
  \expandafter\@firstofone
\else
  \expandafter\@gobble
\fi
{%
  \def\loop#1\repeat{%
    \def\body{#1}%
    \iterate
  }%
  \def\iterate{%
    \body
      \let\next\iterate
    \else
      \let\next\relax
    \fi
    \next
  }%
  \let\repeat=\fi
}%
\def\RestoreCatcodes{}
\count@=0 %
\loop
  \edef\RestoreCatcodes{%
    \RestoreCatcodes
    \catcode\the\count@=\the\catcode\count@\relax
  }%
\ifnum\count@<255 %
  \advance\count@ 1 %
\repeat

\def\RangeCatcodeInvalid#1#2{%
  \count@=#1\relax
  \loop
    \catcode\count@=15 %
  \ifnum\count@<#2\relax
    \advance\count@ 1 %
  \repeat
}
\def\RangeCatcodeCheck#1#2#3{%
  \count@=#1\relax
  \loop
    \ifnum#3=\catcode\count@
    \else
      \errmessage{%
        Character \the\count@\space
        with wrong catcode \the\catcode\count@\space
        instead of \number#3%
      }%
    \fi
  \ifnum\count@<#2\relax
    \advance\count@ 1 %
  \repeat
}
\def\space{ }
\expandafter\ifx\csname LoadCommand\endcsname\relax
  \def\LoadCommand{\input pdftexcmds.sty\relax}%
\fi
\def\Test{%
  \RangeCatcodeInvalid{0}{47}%
  \RangeCatcodeInvalid{58}{64}%
  \RangeCatcodeInvalid{91}{96}%
  \RangeCatcodeInvalid{123}{255}%
  \catcode`\@=12 %
  \catcode`\\=0 %
  \catcode`\%=14 %
  \LoadCommand
  \RangeCatcodeCheck{0}{36}{15}%
  \RangeCatcodeCheck{37}{37}{14}%
  \RangeCatcodeCheck{38}{47}{15}%
  \RangeCatcodeCheck{48}{57}{12}%
  \RangeCatcodeCheck{58}{63}{15}%
  \RangeCatcodeCheck{64}{64}{12}%
  \RangeCatcodeCheck{65}{90}{11}%
  \RangeCatcodeCheck{91}{91}{15}%
  \RangeCatcodeCheck{92}{92}{0}%
  \RangeCatcodeCheck{93}{96}{15}%
  \RangeCatcodeCheck{97}{122}{11}%
  \RangeCatcodeCheck{123}{255}{15}%
  \RestoreCatcodes
}
\Test
\csname @@end\endcsname
\end
%    \end{macrocode}
%    \begin{macrocode}
%</test1>
%    \end{macrocode}
%
% \subsection{Test for \cs{pdf@isprimitive}}
%
%    \begin{macrocode}
%<*test2>
\catcode`\{=1 %
\catcode`\}=2 %
\catcode`\#=6 %
\catcode`\@=11 %
\input pdftexcmds.sty\relax
\def\msg#1{%
  \begingroup
    \escapechar=92 %
    \immediate\write16{#1}%
  \endgroup
}
\long\def\test#1#2#3#4{%
  \begingroup
    #4%
    \def\str{%
      Test \string\pdf@isprimitive
      {\string #1}{\string #2}{...}: %
    }%
    \pdf@isprimitive{#1}{#2}{%
      \ifx#3Y%
        \msg{\str true ==> OK.}%
      \else
        \errmessage{\str false ==> FAILED}%
      \fi
    }{%
      \ifx#3Y%
        \errmessage{\str true ==> FAILED}%
      \else
        \msg{\str false ==> OK.}%
      \fi
    }%
  \endgroup
}
\test\relax\relax Y{}
\test\foobar\relax Y{\let\foobar\relax}
\test\foobar\relax N{}
\test\hbox\hbox Y{}
\test\foobar@hbox\hbox Y{\let\foobar@hbox\hbox}
\test\if\if Y{}
\test\if\ifx N{}
\test\ifx\if N{}
\test\par\par Y{}
\test\hbox\par N{}
\test\par\hbox N{}
\csname @@end\endcsname\end
%</test2>
%    \end{macrocode}
%
% \subsection{Test for \cs{pdf@shellescape}}
%
%    \begin{macrocode}
%<*test-shell>
\catcode`\{=1 %
\catcode`\}=2 %
\catcode`\#=6 %
\catcode`\@=11 %
\input pdftexcmds.sty\relax
\def\msg#{\immediate\write16}
\def\MaybeEnd{}
\ifx\luatexversion\UnDeFiNeD
\else
  \ifnum\luatexversion<68 %
    \ifx\pdf@shellescape\@undefined
      \msg{SHELL=U}%
      \msg{OK (LuaTeX < 0.68)}%
    \else
      \msg{SHELL=defined}%
      \errmessage{Failed (LuaTeX < 0.68)}%
    \fi
    \def\MaybeEnd{\csname @@end\endcsname\end}%
  \fi
\fi
\MaybeEnd
\ifx\pdf@shellescape\@undefined
  \msg{SHELL=U}%
\else
  \msg{SHELL=\number\pdf@shellescape}%
\fi
\ifx\expected\@undefined
\else
  \ifx\expected\relax
    \msg{EXPECTED=U}%
    \ifx\pdf@shellescape\@undefined
      \msg{OK}%
    \else
      \errmessage{Failed}%
    \fi
  \else
    \msg{EXPECTED=\number\expected}%
    \ifnum\pdf@shellescape=\expected\relax
      \msg{OK}%
    \else
      \errmessage{Failed}%
    \fi
  \fi
\fi
\csname @@end\endcsname\end
%</test-shell>
%    \end{macrocode}
%
% \subsection{Test for escape functions}
%
%    \begin{macrocode}
%<*test-escape>
\catcode`\{=1 %
\catcode`\}=2 %
\catcode`\#=6 %
\catcode`\^=7 %
\catcode`\@=11 %
\errorcontextlines=1000 %
\input pdftexcmds.sty\relax
\def\msg#1{%
  \begingroup
    \escapechar=92 %
    \immediate\write16{#1}%
  \endgroup
}
%    \end{macrocode}
%    \begin{macrocode}
\begingroup
  \catcode`\@=11 %
  \countdef\count@=255 %
  \def\space{ }%
  \long\def\@whilenum#1\do #2{%
    \ifnum #1\relax
      #2\relax
      \@iwhilenum{#1\relax#2\relax}%
    \fi
  }%
  \long\def\@iwhilenum#1{%
    \ifnum #1%
      \expandafter\@iwhilenum
    \else
      \expandafter\ltx@gobble
    \fi
    {#1}%
  }%
  \gdef\AllBytes{}%
  \count@=0 %
  \catcode0=12 %
  \@whilenum\count@<256 \do{%
    \lccode0=\count@
    \ifnum\count@=32 %
      \xdef\AllBytes{\AllBytes\space}%
    \else
      \lowercase{%
        \xdef\AllBytes{\AllBytes^^@}%
      }%
    \fi
    \advance\count@ by 1 %
  }%
\endgroup
%    \end{macrocode}
%    \begin{macrocode}
\def\AllBytesHex{%
  000102030405060708090A0B0C0D0E0F%
  101112131415161718191A1B1C1D1E1F%
  202122232425262728292A2B2C2D2E2F%
  303132333435363738393A3B3C3D3E3F%
  404142434445464748494A4B4C4D4E4F%
  505152535455565758595A5B5C5D5E5F%
  606162636465666768696A6B6C6D6E6F%
  707172737475767778797A7B7C7D7E7F%
  808182838485868788898A8B8C8D8E8F%
  909192939495969798999A9B9C9D9E9F%
  A0A1A2A3A4A5A6A7A8A9AAABACADAEAF%
  B0B1B2B3B4B5B6B7B8B9BABBBCBDBEBF%
  C0C1C2C3C4C5C6C7C8C9CACBCCCDCECF%
  D0D1D2D3D4D5D6D7D8D9DADBDCDDDEDF%
  E0E1E2E3E4E5E6E7E8E9EAEBECEDEEEF%
  F0F1F2F3F4F5F6F7F8F9FAFBFCFDFEFF%
}
\ltx@onelevel@sanitize\AllBytesHex
\expandafter\lowercase\expandafter{%
  \expandafter\def\expandafter\AllBytesHexLC
      \expandafter{\AllBytesHex}%
}
\begingroup
  \catcode`\#=12 %
  \xdef\AllBytesName{%
    #01#02#03#04#05#06#07#08#09#0A#0B#0C#0D#0E#0F%
    #10#11#12#13#14#15#16#17#18#19#1A#1B#1C#1D#1E#1F%
    #20!"#23$#25&'#28#29*+,-.#2F%
    0123456789:;#3C=#3E?%
    @ABCDEFGHIJKLMNO%
    PQRSTUVWXYZ#5B\ltx@backslashchar#5D^_%
    `abcdefghijklmno%
    pqrstuvwxyz#7B|#7D\string~#7F%
    #80#81#82#83#84#85#86#87#88#89#8A#8B#8C#8D#8E#8F%
    #90#91#92#93#94#95#96#97#98#99#9A#9B#9C#9D#9E#9F%
    #A0#A1#A2#A3#A4#A5#A6#A7#A8#A9#AA#AB#AC#AD#AE#AF%
    #B0#B1#B2#B3#B4#B5#B6#B7#B8#B9#BA#BB#BC#BD#BE#BF%
    #C0#C1#C2#C3#C4#C5#C6#C7#C8#C9#CA#CB#CC#CD#CE#CF%
    #D0#D1#D2#D3#D4#D5#D6#D7#D8#D9#DA#DB#DC#DD#DE#DF%
    #E0#E1#E2#E3#E4#E5#E6#E7#E8#E9#EA#EB#EC#ED#EE#EF%
    #F0#F1#F2#F3#F4#F5#F6#F7#F8#F9#FA#FB#FC#FD#FE#FF%
  }%
\endgroup
\ltx@onelevel@sanitize\AllBytesName
\edef\AllBytesFromName{\expandafter\ltx@gobble\AllBytes}
\begingroup
  \def\|{|}%
  \edef\%{\ltx@percentchar}%
  \catcode`\|=0 %
  \catcode`\#=12 %
  \catcode`\~=12 %
  \catcode`\\=12 %
  |xdef|AllBytesString{%
    \000\001\002\003\004\005\006\007\010\011\012\013\014\015\016\017%
    \020\021\022\023\024\025\026\027\030\031\032\033\034\035\036\037%
    \040!"#$|%&'\(\)*+,-./%
    0123456789:;<=>?%
    @ABCDEFGHIJKLMNO%
    PQRSTUVWXYZ[\\]^_%
    `abcdefghijklmno%
    pqrstuvwxyz{||}~\177%
    \200\201\202\203\204\205\206\207\210\211\212\213\214\215\216\217%
    \220\221\222\223\224\225\226\227\230\231\232\233\234\235\236\237%
    \240\241\242\243\244\245\246\247\250\251\252\253\254\255\256\257%
    \260\261\262\263\264\265\266\267\270\271\272\273\274\275\276\277%
    \300\301\302\303\304\305\306\307\310\311\312\313\314\315\316\317%
    \320\321\322\323\324\325\326\327\330\331\332\333\334\335\336\337%
    \340\341\342\343\344\345\346\347\350\351\352\353\354\355\356\357%
    \360\361\362\363\364\365\366\367\370\371\372\373\374\375\376\377%
  }%
|endgroup
\ltx@onelevel@sanitize\AllBytesString
%    \end{macrocode}
%    \begin{macrocode}
\def\Test#1#2#3{%
  \begingroup
    \expandafter\expandafter\expandafter\def
    \expandafter\expandafter\expandafter\TestResult
    \expandafter\expandafter\expandafter{%
      #1{#2}%
    }%
    \ifx\TestResult#3%
    \else
      \newlinechar=10 %
      \msg{Expect:^^J#3}%
      \msg{Result:^^J\TestResult}%
      \errmessage{\string#2 -\string#1-> \string#3}%
    \fi
  \endgroup
}
\def\test#1#2#3{%
  \edef\TestFrom{#2}%
  \edef\TestExpect{#3}%
  \ltx@onelevel@sanitize\TestExpect
  \Test#1\TestFrom\TestExpect
}
\test\pdf@unescapehex{74657374}{test}
\begingroup
  \catcode0=12 %
  \catcode1=12 %
  \test\pdf@unescapehex{740074017400740174}{t^^@t^^At^^@t^^At}%
\endgroup
\Test\pdf@escapehex\AllBytes\AllBytesHex
\Test\pdf@unescapehex\AllBytesHex\AllBytes
\Test\pdf@escapename\AllBytes\AllBytesName
\Test\pdf@escapestring\AllBytes\AllBytesString
%    \end{macrocode}
%    \begin{macrocode}
\csname @@end\endcsname\end
%</test-escape>
%    \end{macrocode}
%
% \section{Installation}
%
% \subsection{Download}
%
% \paragraph{Package.} This package is available on
% CTAN\footnote{\CTANpkg{pdftexcmds}}:
% \begin{description}
% \item[\CTAN{macros/latex/contrib/oberdiek/pdftexcmds.dtx}] The source file.
% \item[\CTAN{macros/latex/contrib/oberdiek/pdftexcmds.pdf}] Documentation.
% \end{description}
%
%
% \paragraph{Bundle.} All the packages of the bundle `oberdiek'
% are also available in a TDS compliant ZIP archive. There
% the packages are already unpacked and the documentation files
% are generated. The files and directories obey the TDS standard.
% \begin{description}
% \item[\CTANinstall{install/macros/latex/contrib/oberdiek.tds.zip}]
% \end{description}
% \emph{TDS} refers to the standard ``A Directory Structure
% for \TeX\ Files'' (\CTAN{tds/tds.pdf}). Directories
% with \xfile{texmf} in their name are usually organized this way.
%
% \subsection{Bundle installation}
%
% \paragraph{Unpacking.} Unpack the \xfile{oberdiek.tds.zip} in the
% TDS tree (also known as \xfile{texmf} tree) of your choice.
% Example (linux):
% \begin{quote}
%   |unzip oberdiek.tds.zip -d ~/texmf|
% \end{quote}
%
% \paragraph{Script installation.}
% Check the directory \xfile{TDS:scripts/oberdiek/} for
% scripts that need further installation steps.
% Package \xpackage{attachfile2} comes with the Perl script
% \xfile{pdfatfi.pl} that should be installed in such a way
% that it can be called as \texttt{pdfatfi}.
% Example (linux):
% \begin{quote}
%   |chmod +x scripts/oberdiek/pdfatfi.pl|\\
%   |cp scripts/oberdiek/pdfatfi.pl /usr/local/bin/|
% \end{quote}
%
% \subsection{Package installation}
%
% \paragraph{Unpacking.} The \xfile{.dtx} file is a self-extracting
% \docstrip\ archive. The files are extracted by running the
% \xfile{.dtx} through \plainTeX:
% \begin{quote}
%   \verb|tex pdftexcmds.dtx|
% \end{quote}
%
% \paragraph{TDS.} Now the different files must be moved into
% the different directories in your installation TDS tree
% (also known as \xfile{texmf} tree):
% \begin{quote}
% \def\t{^^A
% \begin{tabular}{@{}>{\ttfamily}l@{ $\rightarrow$ }>{\ttfamily}l@{}}
%   pdftexcmds.sty & tex/generic/oberdiek/pdftexcmds.sty\\
%   oberdiek.pdftexcmds.lua & scripts/oberdiek/oberdiek.pdftexcmds.lua\\
%   pdftexcmds.lua & scripts/oberdiek/pdftexcmds.lua\\
%   pdftexcmds.pdf & doc/latex/oberdiek/pdftexcmds.pdf\\
%   test/pdftexcmds-test1.tex & doc/latex/oberdiek/test/pdftexcmds-test1.tex\\
%   test/pdftexcmds-test2.tex & doc/latex/oberdiek/test/pdftexcmds-test2.tex\\
%   test/pdftexcmds-test-shell.tex & doc/latex/oberdiek/test/pdftexcmds-test-shell.tex\\
%   test/pdftexcmds-test-escape.tex & doc/latex/oberdiek/test/pdftexcmds-test-escape.tex\\
%   pdftexcmds.dtx & source/latex/oberdiek/pdftexcmds.dtx\\
% \end{tabular}^^A
% }^^A
% \sbox0{\t}^^A
% \ifdim\wd0>\linewidth
%   \begingroup
%     \advance\linewidth by\leftmargin
%     \advance\linewidth by\rightmargin
%   \edef\x{\endgroup
%     \def\noexpand\lw{\the\linewidth}^^A
%   }\x
%   \def\lwbox{^^A
%     \leavevmode
%     \hbox to \linewidth{^^A
%       \kern-\leftmargin\relax
%       \hss
%       \usebox0
%       \hss
%       \kern-\rightmargin\relax
%     }^^A
%   }^^A
%   \ifdim\wd0>\lw
%     \sbox0{\small\t}^^A
%     \ifdim\wd0>\linewidth
%       \ifdim\wd0>\lw
%         \sbox0{\footnotesize\t}^^A
%         \ifdim\wd0>\linewidth
%           \ifdim\wd0>\lw
%             \sbox0{\scriptsize\t}^^A
%             \ifdim\wd0>\linewidth
%               \ifdim\wd0>\lw
%                 \sbox0{\tiny\t}^^A
%                 \ifdim\wd0>\linewidth
%                   \lwbox
%                 \else
%                   \usebox0
%                 \fi
%               \else
%                 \lwbox
%               \fi
%             \else
%               \usebox0
%             \fi
%           \else
%             \lwbox
%           \fi
%         \else
%           \usebox0
%         \fi
%       \else
%         \lwbox
%       \fi
%     \else
%       \usebox0
%     \fi
%   \else
%     \lwbox
%   \fi
% \else
%   \usebox0
% \fi
% \end{quote}
% If you have a \xfile{docstrip.cfg} that configures and enables \docstrip's
% TDS installing feature, then some files can already be in the right
% place, see the documentation of \docstrip.
%
% \subsection{Refresh file name databases}
%
% If your \TeX~distribution
% (\teTeX, \mikTeX, \dots) relies on file name databases, you must refresh
% these. For example, \teTeX\ users run \verb|texhash| or
% \verb|mktexlsr|.
%
% \subsection{Some details for the interested}
%
% \paragraph{Unpacking with \LaTeX.}
% The \xfile{.dtx} chooses its action depending on the format:
% \begin{description}
% \item[\plainTeX:] Run \docstrip\ and extract the files.
% \item[\LaTeX:] Generate the documentation.
% \end{description}
% If you insist on using \LaTeX\ for \docstrip\ (really,
% \docstrip\ does not need \LaTeX), then inform the autodetect routine
% about your intention:
% \begin{quote}
%   \verb|latex \let\install=y% \iffalse meta-comment
%
% File: pdftexcmds.dtx
% Version: 2019/07/25 v0.30
% Info: Utility functions of pdfTeX for LuaTeX
%
% Copyright (C) 2007, 2009-2011 by
%    Heiko Oberdiek <heiko.oberdiek at googlemail.com>
%
% This work may be distributed and/or modified under the
% conditions of the LaTeX Project Public License, either
% version 1.3c of this license or (at your option) any later
% version. This version of this license is in
%    https://www.latex-project.org/lppl/lppl-1-3c.txt
% and the latest version of this license is in
%    https://www.latex-project.org/lppl.txt
% and version 1.3 or later is part of all distributions of
% LaTeX version 2005/12/01 or later.
%
% This work has the LPPL maintenance status "maintained".
%
% The Current Maintainers of this work are
% Heiko Oberdiek and the Oberdiek Package Support Group
% https://github.com/ho-tex/oberdiek/issues
%
% The Base Interpreter refers to any `TeX-Format',
% because some files are installed in TDS:tex/generic//.
%
% This work consists of the main source file pdftexcmds.dtx
% and the derived files
%    pdftexcmds.sty, pdftexcmds.pdf, pdftexcmds.ins, pdftexcmds.drv,
%    pdftexcmds.bib, pdftexcmds-test1.tex, pdftexcmds-test2.tex,
%    pdftexcmds-test-shell.tex, pdftexcmds-test-escape.tex,
%    oberdiek.pdftexcmds.lua, pdftexcmds.lua.
%
% Distribution:
%    CTAN:macros/latex/contrib/oberdiek/pdftexcmds.dtx
%    CTAN:macros/latex/contrib/oberdiek/pdftexcmds.pdf
%
% Unpacking:
%    (a) If pdftexcmds.ins is present:
%           tex pdftexcmds.ins
%    (b) Without pdftexcmds.ins:
%           tex pdftexcmds.dtx
%    (c) If you insist on using LaTeX
%           latex \let\install=y% \iffalse meta-comment
%
% File: pdftexcmds.dtx
% Version: 2019/07/25 v0.30
% Info: Utility functions of pdfTeX for LuaTeX
%
% Copyright (C) 2007, 2009-2011 by
%    Heiko Oberdiek <heiko.oberdiek at googlemail.com>
%
% This work may be distributed and/or modified under the
% conditions of the LaTeX Project Public License, either
% version 1.3c of this license or (at your option) any later
% version. This version of this license is in
%    https://www.latex-project.org/lppl/lppl-1-3c.txt
% and the latest version of this license is in
%    https://www.latex-project.org/lppl.txt
% and version 1.3 or later is part of all distributions of
% LaTeX version 2005/12/01 or later.
%
% This work has the LPPL maintenance status "maintained".
%
% The Current Maintainers of this work are
% Heiko Oberdiek and the Oberdiek Package Support Group
% https://github.com/ho-tex/oberdiek/issues
%
% The Base Interpreter refers to any `TeX-Format',
% because some files are installed in TDS:tex/generic//.
%
% This work consists of the main source file pdftexcmds.dtx
% and the derived files
%    pdftexcmds.sty, pdftexcmds.pdf, pdftexcmds.ins, pdftexcmds.drv,
%    pdftexcmds.bib, pdftexcmds-test1.tex, pdftexcmds-test2.tex,
%    pdftexcmds-test-shell.tex, pdftexcmds-test-escape.tex,
%    oberdiek.pdftexcmds.lua, pdftexcmds.lua.
%
% Distribution:
%    CTAN:macros/latex/contrib/oberdiek/pdftexcmds.dtx
%    CTAN:macros/latex/contrib/oberdiek/pdftexcmds.pdf
%
% Unpacking:
%    (a) If pdftexcmds.ins is present:
%           tex pdftexcmds.ins
%    (b) Without pdftexcmds.ins:
%           tex pdftexcmds.dtx
%    (c) If you insist on using LaTeX
%           latex \let\install=y\input{pdftexcmds.dtx}
%        (quote the arguments according to the demands of your shell)
%
% Documentation:
%    (a) If pdftexcmds.drv is present:
%           latex pdftexcmds.drv
%    (b) Without pdftexcmds.drv:
%           latex pdftexcmds.dtx; ...
%    The class ltxdoc loads the configuration file ltxdoc.cfg
%    if available. Here you can specify further options, e.g.
%    use A4 as paper format:
%       \PassOptionsToClass{a4paper}{article}
%
%    Programm calls to get the documentation (example):
%       pdflatex pdftexcmds.dtx
%       bibtex pdftexcmds.aux
%       makeindex -s gind.ist pdftexcmds.idx
%       pdflatex pdftexcmds.dtx
%       makeindex -s gind.ist pdftexcmds.idx
%       pdflatex pdftexcmds.dtx
%
% Installation:
%    TDS:tex/generic/oberdiek/pdftexcmds.sty
%    TDS:scripts/oberdiek/oberdiek.pdftexcmds.lua
%    TDS:scripts/oberdiek/pdftexcmds.lua
%    TDS:doc/latex/oberdiek/pdftexcmds.pdf
%    TDS:doc/latex/oberdiek/test/pdftexcmds-test1.tex
%    TDS:doc/latex/oberdiek/test/pdftexcmds-test2.tex
%    TDS:doc/latex/oberdiek/test/pdftexcmds-test-shell.tex
%    TDS:doc/latex/oberdiek/test/pdftexcmds-test-escape.tex
%    TDS:source/latex/oberdiek/pdftexcmds.dtx
%
%<*ignore>
\begingroup
  \catcode123=1 %
  \catcode125=2 %
  \def\x{LaTeX2e}%
\expandafter\endgroup
\ifcase 0\ifx\install y1\fi\expandafter
         \ifx\csname processbatchFile\endcsname\relax\else1\fi
         \ifx\fmtname\x\else 1\fi\relax
\else\csname fi\endcsname
%</ignore>
%<*install>
\input docstrip.tex
\Msg{************************************************************************}
\Msg{* Installation}
\Msg{* Package: pdftexcmds 2019/07/25 v0.30 Utility functions of pdfTeX for LuaTeX (HO)}
\Msg{************************************************************************}

\keepsilent
\askforoverwritefalse

\let\MetaPrefix\relax
\preamble

This is a generated file.

Project: pdftexcmds
Version: 2019/07/25 v0.30

Copyright (C) 2007, 2009-2011 by
   Heiko Oberdiek <heiko.oberdiek at googlemail.com>

This work may be distributed and/or modified under the
conditions of the LaTeX Project Public License, either
version 1.3c of this license or (at your option) any later
version. This version of this license is in
   https://www.latex-project.org/lppl/lppl-1-3c.txt
and the latest version of this license is in
   https://www.latex-project.org/lppl.txt
and version 1.3 or later is part of all distributions of
LaTeX version 2005/12/01 or later.

This work has the LPPL maintenance status "maintained".

The Current Maintainers of this work are
Heiko Oberdiek and the Oberdiek Package Support Group
https://github.com/ho-tex/oberdiek/issues


The Base Interpreter refers to any `TeX-Format',
because some files are installed in TDS:tex/generic//.

This work consists of the main source file pdftexcmds.dtx
and the derived files
   pdftexcmds.sty, pdftexcmds.pdf, pdftexcmds.ins, pdftexcmds.drv,
   pdftexcmds.bib, pdftexcmds-test1.tex, pdftexcmds-test2.tex,
   pdftexcmds-test-shell.tex, pdftexcmds-test-escape.tex,
   oberdiek.pdftexcmds.lua, pdftexcmds.lua.

\endpreamble
\let\MetaPrefix\DoubleperCent

\generate{%
  \file{pdftexcmds.ins}{\from{pdftexcmds.dtx}{install}}%
  \file{pdftexcmds.drv}{\from{pdftexcmds.dtx}{driver}}%
  \nopreamble
  \nopostamble
  \file{pdftexcmds.bib}{\from{pdftexcmds.dtx}{bib}}%
  \usepreamble\defaultpreamble
  \usepostamble\defaultpostamble
  \usedir{tex/generic/oberdiek}%
  \file{pdftexcmds.sty}{\from{pdftexcmds.dtx}{package}}%
%  \usedir{doc/latex/oberdiek/test}%
%  \file{pdftexcmds-test1.tex}{\from{pdftexcmds.dtx}{test1}}%
%  \file{pdftexcmds-test2.tex}{\from{pdftexcmds.dtx}{test2}}%
%  \file{pdftexcmds-test-shell.tex}{\from{pdftexcmds.dtx}{test-shell}}%
%  \file{pdftexcmds-test-escape.tex}{\from{pdftexcmds.dtx}{test-escape}}%
  \nopreamble
  \nopostamble
%  \usedir{source/latex/oberdiek/catalogue}%
%  \file{pdftexcmds.xml}{\from{pdftexcmds.dtx}{catalogue}}%
}
\def\MetaPrefix{-- }
\def\defaultpostamble{%
  \MetaPrefix^^J%
  \MetaPrefix\space End of File `\outFileName'.%
}
\def\currentpostamble{\defaultpostamble}%
\generate{%
  \usedir{scripts/oberdiek}%
  \file{oberdiek.pdftexcmds.lua}{\from{pdftexcmds.dtx}{lua}}%
  \file{pdftexcmds.lua}{\from{pdftexcmds.dtx}{lua}}%
}

\catcode32=13\relax% active space
\let =\space%
\Msg{************************************************************************}
\Msg{*}
\Msg{* To finish the installation you have to move the following}
\Msg{* file into a directory searched by TeX:}
\Msg{*}
\Msg{*     pdftexcmds.sty}
\Msg{*}
\Msg{* And install the following script files:}
\Msg{*}
\Msg{*     oberdiek.pdftexcmds.lua, pdftexcmds.lua}
\Msg{*}
\Msg{* To produce the documentation run the file `pdftexcmds.drv'}
\Msg{* through LaTeX.}
\Msg{*}
\Msg{* Happy TeXing!}
\Msg{*}
\Msg{************************************************************************}

\endbatchfile
%</install>
%<*bib>
@online{AndyThomas:Analog,
  author={Thomas, Andy},
  title={Analog of {\texttt{\csname textbackslash\endcsname}pdfelapsedtime} for
      {\hologo{LuaTeX}} and {\hologo{XeTeX}}},
  url={http://tex.stackexchange.com/a/32531},
  urldate={2011-11-29},
}
%</bib>
%<*ignore>
\fi
%</ignore>
%<*driver>
\NeedsTeXFormat{LaTeX2e}
\ProvidesFile{pdftexcmds.drv}%
  [2019/07/25 v0.30 Utility functions of pdfTeX for LuaTeX (HO)]%
\documentclass{ltxdoc}
\usepackage{holtxdoc}[2011/11/22]
\usepackage{paralist}
\usepackage{csquotes}
\usepackage[
  backend=bibtex,
  bibencoding=ascii,
  alldates=iso8601,
]{biblatex}[2011/11/13]
\bibliography{oberdiek-source}
\bibliography{pdftexcmds}
\begin{document}
  \DocInput{pdftexcmds.dtx}%
\end{document}
%</driver>
% \fi
%
%
% \CharacterTable
%  {Upper-case    \A\B\C\D\E\F\G\H\I\J\K\L\M\N\O\P\Q\R\S\T\U\V\W\X\Y\Z
%   Lower-case    \a\b\c\d\e\f\g\h\i\j\k\l\m\n\o\p\q\r\s\t\u\v\w\x\y\z
%   Digits        \0\1\2\3\4\5\6\7\8\9
%   Exclamation   \!     Double quote  \"     Hash (number) \#
%   Dollar        \$     Percent       \%     Ampersand     \&
%   Acute accent  \'     Left paren    \(     Right paren   \)
%   Asterisk      \*     Plus          \+     Comma         \,
%   Minus         \-     Point         \.     Solidus       \/
%   Colon         \:     Semicolon     \;     Less than     \<
%   Equals        \=     Greater than  \>     Question mark \?
%   Commercial at \@     Left bracket  \[     Backslash     \\
%   Right bracket \]     Circumflex    \^     Underscore    \_
%   Grave accent  \`     Left brace    \{     Vertical bar  \|
%   Right brace   \}     Tilde         \~}
%
% \GetFileInfo{pdftexcmds.drv}
%
% \title{The \xpackage{pdftexcmds} package}
% \date{2019/07/25 v0.30}
% \author{Heiko Oberdiek\thanks
% {Please report any issues at \url{https://github.com/ho-tex/oberdiek/issues}}}
%
% \maketitle
%
% \begin{abstract}
% \hologo{LuaTeX} provides most of the commands of \hologo{pdfTeX} 1.40. However
% a number of utility functions are removed. This package tries to fill
% the gap and implements some of the missing primitive using Lua.
% \end{abstract}
%
% \tableofcontents
%
% \def\csi#1{\texttt{\textbackslash\textit{#1}}}
%
% \section{Documentation}
%
% Some primitives of \hologo{pdfTeX} \cite{pdftex-manual}
% are not defined by \hologo{LuaTeX} \cite{luatex-manual}.
% This package implements macro based solutions using Lua code
% for the following missing \hologo{pdfTeX} primitives;
% \begin{compactitem}
% \item \cs{pdfstrcmp}
% \item \cs{pdfunescapehex}
% \item \cs{pdfescapehex}
% \item \cs{pdfescapename}
% \item \cs{pdfescapestring}
% \item \cs{pdffilesize}
% \item \cs{pdffilemoddate}
% \item \cs{pdffiledump}
% \item \cs{pdfmdfivesum}
% \item \cs{pdfresettimer}
% \item \cs{pdfelapsedtime}
% \item |\immediate\write18|
% \end{compactitem}
% The original names of the primitives cannot be used:
% \begin{itemize}
% \item
% The syntax for their arguments cannot easily
% simulated by macros. The primitives using key words
% such as |file| (\cs{pdfmdfivesum}) or |offset| and |length|
% (\cs{pdffiledump}) and uses \meta{general text} for the other
% arguments. Using token registers assignments, \meta{general text} could
% be catched. However, the simulated primitives are expandable
% and register assignments would destroy this important property.
% (\meta{general text} allows something like |\expandafter\bgroup ...}|.)
% \item
% The original primitives can be expanded using one expansion step.
% The new macros need two expansion steps because of the additional
% macro expansion. Example:
% \begin{quote}
%   |\expandafter\foo\pdffilemoddate{file}|\\
%   vs.\\
%   |\expandafter\expandafter\expandafter|\\
%   |\foo\pdf@filemoddate{file}|
% \end{quote}
% \end{itemize}
%
% \hologo{LuaTeX} isn't stable yet and thus the status of this package is
% \emph{experimental}. Feedback is welcome.
%
% \subsection{General principles}
%
% \begin{description}
% \item[Naming convention:]
%   Usually this package defines a macro |\pdf@|\meta{cmd} if
%   \hologo{pdfTeX} provides |\pdf|\meta{cmd}.
% \item[Arguments:] The order of arguments in |\pdf@|\meta{cmd}
%   is the same as for the corresponding primitive of \hologo{pdfTeX}.
%   The arguments are ordinary undelimited \hologo{TeX} arguments,
%   no \meta{general text} and without additional keywords.
% \item[Expandibility:]
%   The macro |\pdf@|\meta{cmd} is expandable if the
%   corresponding \hologo{pdfTeX} primitive has this property.
%   Exact two expansion steps are necessary (first is the macro
%   expansion) except for \cs{pdf@primitive} and \cs{pdf@ifprimitive}.
%   The latter ones are not macros, but have the direct meaning of the
%   primitive.
% \item[Without \hologo{LuaTeX}:]
%   The macros |\pdf@|\meta{cmd} are mapped to the commands
%   of \hologo{pdfTeX} if they are available. Otherwise they are undefined.
% \item[Availability:]
%   The macros that the packages provides are undefined, if
%   the necessary primitives are not found and cannot be
%   implemented by Lua.
% \end{description}
%
% \subsection{Macros}
%
% \subsubsection[Strings]{Strings \cite[``7.15 Strings'']{pdftex-manual}}
%
% \begin{declcs}{pdf@strcmp} \M{stringA} \M{stringB}
% \end{declcs}
% Same as |\pdfstrcmp{|\meta{stringA}|}{|\meta{stringB}|}|.
%
% \begin{declcs}{pdf@unescapehex} \M{string}
% \end{declcs}
% Same as |\pdfunescapehex{|\meta{string}|}|.
% The argument is a byte string given in hexadecimal notation.
% The result are character tokens from 0 until 255 with
% catcode 12 and the space with catcode 10.
%
% \begin{declcs}{pdf@escapehex} \M{string}\\
%   \cs{pdf@escapestring} \M{string}\\
%   \cs{pdf@escapename} \M{string}
% \end{declcs}
% Same as the primitives of \hologo{pdfTeX}. However \hologo{pdfTeX} does not
% know about characters with codes 256 and larger. Thus the
% string is treated as byte string, characters with more than
% eight bits are ignored.
%
% \subsubsection[Files]{Files \cite[``7.18 Files'']{pdftex-manual}}
%
% \begin{declcs}{pdf@filesize} \M{filename}
% \end{declcs}
% Same as |\pdffilesize{|\meta{filename}|}|.
%
% \begin{declcs}{pdf@filemoddate} \M{filename}
% \end{declcs}
% Same as |\pdffilemoddate{|\meta{filename}|}|.
%
% \begin{declcs}{pdf@filedump} \M{offset} \M{length} \M{filename}
% \end{declcs}
% Same as |\pdffiledump offset| \meta{offset} |length| \meta{length}
% |{|\meta{filename}|}|. Both \meta{offset} and \meta{length} must
% not be empty, but must be a valid \hologo{TeX} number.
%
% \begin{declcs}{pdf@mdfivesum} \M{string}
% \end{declcs}
% Same as |\pdfmdfivesum{|\meta{string}|}|. Keyword |file| is supported
% by macro \cs{pdf@filemdfivesum}.
%
% \begin{declcs}{pdf@filemdfivesum} \M{filename}
% \end{declcs}
% Same as |\pdfmdfivesum file{|\meta{filename}|}|.
%
% \subsubsection[Timekeeping]{Timekeeping \cite[``7.17 Timekeeping'']{pdftex-manual}}
%
% The timekeeping macros are based on Andy Thomas' work \cite{AndyThomas:Analog}.
%
% \begin{declcs}{pdf@resettimer}
% \end{declcs}
% Same as \cs{pdfresettimer}, it resets the internal timer.
%
% \begin{declcs}{pdf@elapsedtime}
% \end{declcs}
% Same as \cs{pdfelapsedtime}. It behaves like a read-only integer.
% For printing purposes it can be prefixed by \cs{the} or \cs{number}.
% It measures the time in scaled seconds (seconds multiplied with 65536)
% since the latest call of \cs{pdf@resettimer} or start of
% program/package. The resolution, the shortest time interval that
% can be measured, depends on the program and system.
% \begin{itemize}
% \item \hologo{pdfTeX} with |gettimeofday|: $\ge$ 1/65536\,s
% \item \hologo{pdfTeX} with |ftime|: $\ge$ 1\,ms
% \item \hologo{pdfTeX} with |time|: $\ge$ 1\,s
% \item \hologo{LuaTeX}: $\ge$ 10\,ms\\
%  (|os.clock()| returns a float number with two decimal digits in
%  \hologo{LuaTeX} beta-0.70.1-2011061416 (rev 4277)).
% \end{itemize}
%
% \subsubsection[Miscellaneous]{Miscellaneous \cite[``7.21 Miscellaneous'']{pdftex-manual}}
%
% \begin{declcs}{pdf@draftmode}
% \end{declcs}
% If the \TeX\ compiler knows \cs{pdfdraftmode} or \cs{draftmode}
% (\hologo{pdfTeX},
% \hologo{LuaTeX}), then \cs{pdf@draftmode} returns, whether
% this mode is enabled. The result is an implicit number:
% one means the draft mode is available and enabled.
% If the value is zero, then the mode is not active or
% \cs{pdfdraftmode} is not available.
% An explicit number is yielded by \cs{number}\cs{pdf@draftmode}.
% The macro cannot
% be used to change the mode, see \cs{pdf@setdraftmode}.
%
% \begin{declcs}{pdf@ifdraftmode} \M{true} \M{false}
% \end{declcs}
% If \cs{pdfdraftmode} is available and enabled, \meta{true} is
% called, otherwise \meta{false} is executed.
%
% \begin{declcs}{pdf@setdraftmode} \M{value}
% \end{declcs}
% Macro \cs{pdf@setdraftmode} expects the number zero or one as
% \meta{value}. Zero deactivates the mode and one enables the draft mode.
% The macro does not have an effect, if the feature \cs{pdfdraftmode} is not
% available.
%
% \begin{declcs}{pdf@shellescape}
% \end{declcs}
% Same as |\pdfshellescape|. It is or expands to |1| if external
% commands can be executed and |0| otherwise. In \hologo{pdfTeX} external
% commands must be enabled first by command line option or
% configuration option. In \hologo{LuaTeX} option |--safer| disables
% the execution of external commands.
%
% In \hologo{LuaTeX} before 0.68.0 \cs{pdf@shellescape} is not
% available due to a bug in |os.execute()|. The argumentless form
% crashes in some circumstances with segmentation fault.
% (It is fixed in version 0.68.0 or revision 4167 of \hologo{LuaTeX}.
% and packported to some version of 0.67.0).
%
% Hints for usage:
% \begin{itemize}
% \item Before its use \cs{pdf@shellescape} should be tested,
% whether it is available. Example with package \xpackage{ltxcmds}
% (loaded by package \xpackage{pdftexcmds}):
%\begin{quote}
%\begin{verbatim}
%\ltx@IfUndefined{pdf@shellescape}{%
%  % \pdf@shellescape is undefined
%}{%
%  % \pdf@shellescape is available
%}
%\end{verbatim}
%\end{quote}
% Use \cs{ltx@ifundefined} in expandable contexts.
% \item \cs{pdf@shellescape} might be a numerical constant,
% expands to the primitive, or expands to a plain number.
% Therefore use it in contexts where these differences does not matter.
% \item Use in comparisons, e.g.:
%   \begin{quote}
%     |\ifnum\pdf@shellescape=0 ...|
%   \end{quote}
% \item Print the number: |\number\pdf@shellescape|
% \end{itemize}
%
% \begin{declcs}{pdf@system} \M{cmdline}
% \end{declcs}
% It is a wrapper for |\immediate\write18| in \hologo{pdfTeX} or
% |os.execute| in \hologo{LuaTeX}.
%
% In theory |os.execute|
% returns a status number. But its meaning is quite
% undefined. Are there some reliable properties?
% Does it make sense to provide an user interface to
% this status exit code?
%
% \begin{declcs}{pdf@primitive} \csi{cmd}
% \end{declcs}
% Same as \cs{pdfprimitive} in \hologo{pdfTeX} or \hologo{LuaTeX}.
% In \hologo{XeTeX} the
% primitive is called \cs{primitive}. Despite the current definition
% of the command \csi{cmd}, it's meaning as primitive is used.
%
% \begin{declcs}{pdf@ifprimitive} \csi{cmd}
% \end{declcs}
% Same as \cs{ifpdfprimitive} in \hologo{pdfTeX} or
% \hologo{LuaTeX}. \hologo{XeTeX} calls
% it \cs{ifprimitive}. It is a switch that checks if the command
% \csi{cmd} has it's primitive meaning.
%
% \subsubsection{Additional macro: \cs{pdf@isprimitive}}
%
% \begin{declcs}{pdf@isprimitive} \csi{cmd1} \csi{cmd2} \M{true} \M{false}
% \end{declcs}
% If \csi{cmd1} has the primitive meaning given by the primitive name
% of \csi{cmd2}, then the argument \meta{true} is executed, otherwise
% \meta{false}. The macro \cs{pdf@isprimitive} is expandable.
% Internally it checks the result of \cs{meaning} and is therefore
% available for all \hologo{TeX} variants, even the original \hologo{TeX}.
% Example with \hologo{LaTeX}:
%\begin{quote}
%\begin{verbatim}
%\makeatletter
%\pdf@isprimitive{@@input}{input}{%
%  \typeout{\string\@@input\space is original\string\input}%
%}{%
%  \typeout{Oops, \string\@@input\space is not the %
%           original\string\input}%
%}
%\end{verbatim}
%\end{quote}
%
% \subsubsection{Experimental}
%
% \begin{declcs}{pdf@unescapehexnative} \M{string}\\
%   \cs{pdf@escapehexnative} \M{string}\\
%   \cs{pdf@escapenamenative} \M{string}\\
%   \cs{pdf@mdfivesumnative} \M{string}
% \end{declcs}
% The variants without |native| in the macro name are supposed to
% be compatible with \hologo{pdfTeX}. However characters with more than
% eight bits are not supported and are ignored. If \hologo{LuaTeX} is
% running, then its UTF-8 coded strings are used. Thus the full
% unicode character range is supported. However the result
% differs from \hologo{pdfTeX} for characters with eight or more bits.
%
% \begin{declcs}{pdf@pipe} \M{cmdline}
% \end{declcs}
% It calls \meta{cmdline} and returns the output of the external
% program in the usual manner as byte string (catcode 12, space with
% catcode 10). The Lua documentation says, that the used |io.popen|
% may not be available on all platforms. Then macro \cs{pdf@pipe}
% is undefined.
%
% \StopEventually{
% }
%
% \section{Implementation}
%
%    \begin{macrocode}
%<*package>
%    \end{macrocode}
%
% \subsection{Reload check and package identification}
%    Reload check, especially if the package is not used with \LaTeX.
%    \begin{macrocode}
\begingroup\catcode61\catcode48\catcode32=10\relax%
  \catcode13=5 % ^^M
  \endlinechar=13 %
  \catcode35=6 % #
  \catcode39=12 % '
  \catcode44=12 % ,
  \catcode45=12 % -
  \catcode46=12 % .
  \catcode58=12 % :
  \catcode64=11 % @
  \catcode123=1 % {
  \catcode125=2 % }
  \expandafter\let\expandafter\x\csname ver@pdftexcmds.sty\endcsname
  \ifx\x\relax % plain-TeX, first loading
  \else
    \def\empty{}%
    \ifx\x\empty % LaTeX, first loading,
      % variable is initialized, but \ProvidesPackage not yet seen
    \else
      \expandafter\ifx\csname PackageInfo\endcsname\relax
        \def\x#1#2{%
          \immediate\write-1{Package #1 Info: #2.}%
        }%
      \else
        \def\x#1#2{\PackageInfo{#1}{#2, stopped}}%
      \fi
      \x{pdftexcmds}{The package is already loaded}%
      \aftergroup\endinput
    \fi
  \fi
\endgroup%
%    \end{macrocode}
%    Package identification:
%    \begin{macrocode}
\begingroup\catcode61\catcode48\catcode32=10\relax%
  \catcode13=5 % ^^M
  \endlinechar=13 %
  \catcode35=6 % #
  \catcode39=12 % '
  \catcode40=12 % (
  \catcode41=12 % )
  \catcode44=12 % ,
  \catcode45=12 % -
  \catcode46=12 % .
  \catcode47=12 % /
  \catcode58=12 % :
  \catcode64=11 % @
  \catcode91=12 % [
  \catcode93=12 % ]
  \catcode123=1 % {
  \catcode125=2 % }
  \expandafter\ifx\csname ProvidesPackage\endcsname\relax
    \def\x#1#2#3[#4]{\endgroup
      \immediate\write-1{Package: #3 #4}%
      \xdef#1{#4}%
    }%
  \else
    \def\x#1#2[#3]{\endgroup
      #2[{#3}]%
      \ifx#1\@undefined
        \xdef#1{#3}%
      \fi
      \ifx#1\relax
        \xdef#1{#3}%
      \fi
    }%
  \fi
\expandafter\x\csname ver@pdftexcmds.sty\endcsname
\ProvidesPackage{pdftexcmds}%
  [2019/07/25 v0.30 Utility functions of pdfTeX for LuaTeX (HO)]%
%    \end{macrocode}
%
% \subsection{Catcodes}
%
%    \begin{macrocode}
\begingroup\catcode61\catcode48\catcode32=10\relax%
  \catcode13=5 % ^^M
  \endlinechar=13 %
  \catcode123=1 % {
  \catcode125=2 % }
  \catcode64=11 % @
  \def\x{\endgroup
    \expandafter\edef\csname pdftexcmds@AtEnd\endcsname{%
      \endlinechar=\the\endlinechar\relax
      \catcode13=\the\catcode13\relax
      \catcode32=\the\catcode32\relax
      \catcode35=\the\catcode35\relax
      \catcode61=\the\catcode61\relax
      \catcode64=\the\catcode64\relax
      \catcode123=\the\catcode123\relax
      \catcode125=\the\catcode125\relax
    }%
  }%
\x\catcode61\catcode48\catcode32=10\relax%
\catcode13=5 % ^^M
\endlinechar=13 %
\catcode35=6 % #
\catcode64=11 % @
\catcode123=1 % {
\catcode125=2 % }
\def\TMP@EnsureCode#1#2{%
  \edef\pdftexcmds@AtEnd{%
    \pdftexcmds@AtEnd
    \catcode#1=\the\catcode#1\relax
  }%
  \catcode#1=#2\relax
}
\TMP@EnsureCode{0}{12}%
\TMP@EnsureCode{1}{12}%
\TMP@EnsureCode{2}{12}%
\TMP@EnsureCode{10}{12}% ^^J
\TMP@EnsureCode{33}{12}% !
\TMP@EnsureCode{34}{12}% "
\TMP@EnsureCode{38}{4}% &
\TMP@EnsureCode{39}{12}% '
\TMP@EnsureCode{40}{12}% (
\TMP@EnsureCode{41}{12}% )
\TMP@EnsureCode{42}{12}% *
\TMP@EnsureCode{43}{12}% +
\TMP@EnsureCode{44}{12}% ,
\TMP@EnsureCode{45}{12}% -
\TMP@EnsureCode{46}{12}% .
\TMP@EnsureCode{47}{12}% /
\TMP@EnsureCode{58}{12}% :
\TMP@EnsureCode{60}{12}% <
\TMP@EnsureCode{62}{12}% >
\TMP@EnsureCode{91}{12}% [
\TMP@EnsureCode{93}{12}% ]
\TMP@EnsureCode{94}{7}% ^ (superscript)
\TMP@EnsureCode{95}{12}% _ (other)
\TMP@EnsureCode{96}{12}% `
\TMP@EnsureCode{126}{12}% ~ (other)
\edef\pdftexcmds@AtEnd{%
  \pdftexcmds@AtEnd
  \escapechar=\number\escapechar\relax
  \noexpand\endinput
}
\escapechar=92 %
%    \end{macrocode}
%
% \subsection{Load packages}
%
%    \begin{macrocode}
\begingroup\expandafter\expandafter\expandafter\endgroup
\expandafter\ifx\csname RequirePackage\endcsname\relax
  \def\TMP@RequirePackage#1[#2]{%
    \begingroup\expandafter\expandafter\expandafter\endgroup
    \expandafter\ifx\csname ver@#1.sty\endcsname\relax
      \input #1.sty\relax
    \fi
  }%
  \TMP@RequirePackage{infwarerr}[2007/09/09]%
  \TMP@RequirePackage{ifluatex}[2010/03/01]%
  \TMP@RequirePackage{ltxcmds}[2010/12/02]%
  \TMP@RequirePackage{ifpdf}[2010/09/13]%
\else
  \RequirePackage{infwarerr}[2007/09/09]%
  \RequirePackage{ifluatex}[2010/03/01]%
  \RequirePackage{ltxcmds}[2010/12/02]%
  \RequirePackage{ifpdf}[2010/09/13]%
\fi
%    \end{macrocode}
%
% \subsection{Without \hologo{LuaTeX}}
%
%    \begin{macrocode}
\ifluatex
\else
  \@PackageInfoNoLine{pdftexcmds}{LuaTeX not detected}%
  \def\pdftexcmds@nopdftex{%
    \@PackageInfoNoLine{pdftexcmds}{pdfTeX >= 1.30 not detected}%
    \let\pdftexcmds@nopdftex\relax
  }%
  \def\pdftexcmds@temp#1{%
    \begingroup\expandafter\expandafter\expandafter\endgroup
    \expandafter\ifx\csname pdf#1\endcsname\relax
      \pdftexcmds@nopdftex
    \else
      \expandafter\def\csname pdf@#1\expandafter\endcsname
      \expandafter##\expandafter{%
        \csname pdf#1\endcsname
      }%
    \fi
  }%
  \pdftexcmds@temp{strcmp}%
  \pdftexcmds@temp{escapehex}%
  \let\pdf@escapehexnative\pdf@escapehex
  \pdftexcmds@temp{unescapehex}%
  \let\pdf@unescapehexnative\pdf@unescapehex
  \pdftexcmds@temp{escapestring}%
  \pdftexcmds@temp{escapename}%
  \pdftexcmds@temp{filesize}%
  \pdftexcmds@temp{filemoddate}%
  \begingroup\expandafter\expandafter\expandafter\endgroup
  \expandafter\ifx\csname pdfshellescape\endcsname\relax
    \pdftexcmds@nopdftex
    \ltx@IfUndefined{pdftexversion}{%
    }{%
      \ifnum\pdftexversion>120 % 1.21a supports \ifeof18
        \ifeof18 %
          \chardef\pdf@shellescape=0 %
        \else
          \chardef\pdf@shellescape=1 %
        \fi
      \fi
    }%
  \else
    \def\pdf@shellescape{%
      \pdfshellescape
    }%
  \fi
  \begingroup\expandafter\expandafter\expandafter\endgroup
  \expandafter\ifx\csname pdffiledump\endcsname\relax
    \pdftexcmds@nopdftex
  \else
    \def\pdf@filedump#1#2#3{%
      \pdffiledump offset#1 length#2{#3}%
    }%
  \fi
%    \end{macrocode}
%    \begin{macrocode}
  \begingroup\expandafter\expandafter\expandafter\endgroup
  \expandafter\ifx\csname pdfmdfivesum\endcsname\relax
    \begingroup\expandafter\expandafter\expandafter\endgroup
    \expandafter\ifx\csname mdfivesum\endcsname\relax
      \pdftexcmds@nopdftex
    \else
      \def\pdf@mdfivesum#{\mdfivesum}%
      \let\pdf@mdfivesumnative\pdf@mdfivesum
      \def\pdf@filemdfivesum#{\mdfivesum file}%
    \fi
  \else
    \def\pdf@mdfivesum#{\pdfmdfivesum}%
    \let\pdf@mdfivesumnative\pdf@mdfivesum
    \def\pdf@filemdfivesum#{\pdfmdfivesum file}%
  \fi
%    \end{macrocode}
%    \begin{macrocode}
  \def\pdf@system#{%
    \immediate\write18%
  }%
  \def\pdftexcmds@temp#1{%
    \begingroup\expandafter\expandafter\expandafter\endgroup
    \expandafter\ifx\csname pdf#1\endcsname\relax
      \pdftexcmds@nopdftex
    \else
      \expandafter\let\csname pdf@#1\expandafter\endcsname
      \csname pdf#1\endcsname
    \fi
  }%
  \pdftexcmds@temp{resettimer}%
  \pdftexcmds@temp{elapsedtime}%
\fi
%    \end{macrocode}
%
% \subsection{\cs{pdf@primitive}, \cs{pdf@ifprimitive}}
%
%    Since version 1.40.0 \hologo{pdfTeX} has \cs{pdfprimitive} and
%    \cs{ifpdfprimitive}. And \cs{pdfprimitive} was fixed in
%    version 1.40.4.
%
%    \hologo{XeTeX} provides them under the name \cs{primitive} and
%    \cs{ifprimitive}. \hologo{LuaTeX} knows both name variants,
%    but they have possibly to be enabled first (|tex.enableprimitives|).
%
%    Depending on the format TeX Live uses a prefix |luatex|.
%
%    Caution: \cs{let} must be used for the definition of
%    the macros, especially because of \cs{ifpdfprimitive}.
%
% \subsubsection{Using \hologo{LuaTeX}'s \texttt{tex.enableprimitives}}
%
%    \begin{macrocode}
\ifluatex
%    \end{macrocode}
%    \begin{macro}{\pdftexcmds@directlua}
%    \begin{macrocode}
  \ifnum\luatexversion<36 %
    \def\pdftexcmds@directlua{\directlua0 }%
  \else
    \let\pdftexcmds@directlua\directlua
  \fi
%    \end{macrocode}
%    \end{macro}
%
%    \begin{macrocode}
  \begingroup
    \newlinechar=10 %
    \endlinechar=\newlinechar
    \pdftexcmds@directlua{%
      if tex.enableprimitives then
        tex.enableprimitives(
          'pdf@',
          {'primitive', 'ifprimitive', 'pdfdraftmode','draftmode'}
        )
        tex.enableprimitives('', {'luaescapestring'})
      end
    }%
  \endgroup %
%    \end{macrocode}
%
%    \begin{macrocode}
\fi
%    \end{macrocode}
%
% \subsubsection{Trying various names to find the primitives}
%
%    \begin{macro}{\pdftexcmds@strip@prefix}
%    \begin{macrocode}
\def\pdftexcmds@strip@prefix#1>{}
%    \end{macrocode}
%    \end{macro}
%    \begin{macrocode}
\def\pdftexcmds@temp#1#2#3{%
  \begingroup\expandafter\expandafter\expandafter\endgroup
  \expandafter\ifx\csname pdf@#1\endcsname\relax
    \begingroup
      \def\x{#3}%
      \edef\x{\expandafter\pdftexcmds@strip@prefix\meaning\x}%
      \escapechar=-1 %
      \edef\y{\expandafter\meaning\csname#2\endcsname}%
    \expandafter\endgroup
    \ifx\x\y
      \expandafter\let\csname pdf@#1\expandafter\endcsname
      \csname #2\endcsname
    \fi
  \fi
}
%    \end{macrocode}
%
%    \begin{macro}{\pdf@primitive}
%    \begin{macrocode}
\pdftexcmds@temp{primitive}{pdfprimitive}{pdfprimitive}% pdfTeX, oldLuaTeX
\pdftexcmds@temp{primitive}{primitive}{primitive}% XeTeX, luatex
\pdftexcmds@temp{primitive}{luatexprimitive}{pdfprimitive}% oldLuaTeX
\pdftexcmds@temp{primitive}{luatexpdfprimitive}{pdfprimitive}% oldLuaTeX
%    \end{macrocode}
%    \end{macro}
%    \begin{macro}{\pdf@ifprimitive}
%    \begin{macrocode}
\pdftexcmds@temp{ifprimitive}{ifpdfprimitive}{ifpdfprimitive}% pdfTeX, oldLuaTeX
\pdftexcmds@temp{ifprimitive}{ifprimitive}{ifprimitive}% XeTeX, luatex
\pdftexcmds@temp{ifprimitive}{luatexifprimitive}{ifpdfprimitive}% oldLuaTeX
\pdftexcmds@temp{ifprimitive}{luatexifpdfprimitive}{ifpdfprimitive}% oldLuaTeX
%    \end{macrocode}
%    \end{macro}
%
%    Disable broken \cs{pdfprimitive}.
%    \begin{macrocode}
\ifluatex\else
\begingroup
  \expandafter\ifx\csname pdf@primitive\endcsname\relax
  \else
    \expandafter\ifx\csname pdftexversion\endcsname\relax
    \else
      \ifnum\pdftexversion=140 %
        \expandafter\ifx\csname pdftexrevision\endcsname\relax
        \else
          \ifnum\pdftexrevision<4 %
            \endgroup
            \let\pdf@primitive\@undefined
            \@PackageInfoNoLine{pdftexcmds}{%
              \string\pdf@primitive\space disabled, %
              because\MessageBreak
              \string\pdfprimitive\space is broken until pdfTeX 1.40.4%
            }%
            \begingroup
          \fi
        \fi
      \fi
    \fi
  \fi
\endgroup
\fi
%    \end{macrocode}
%
% \subsubsection{Result}
%
%    \begin{macrocode}
\begingroup
  \@PackageInfoNoLine{pdftexcmds}{%
    \string\pdf@primitive\space is %
    \expandafter\ifx\csname pdf@primitive\endcsname\relax not \fi
    available%
  }%
  \@PackageInfoNoLine{pdftexcmds}{%
    \string\pdf@ifprimitive\space is %
    \expandafter\ifx\csname pdf@ifprimitive\endcsname\relax not \fi
    available%
  }%
\endgroup
%    \end{macrocode}
%
% \subsection{\hologo{XeTeX}}
%
%    Look for primitives \cs{shellescape}, \cs{strcmp}.
%    \begin{macrocode}
\def\pdftexcmds@temp#1{%
  \begingroup\expandafter\expandafter\expandafter\endgroup
  \expandafter\ifx\csname pdf@#1\endcsname\relax
    \begingroup
      \escapechar=-1 %
      \edef\x{\expandafter\meaning\csname#1\endcsname}%
      \def\y{#1}%
      \def\z##1->{}%
      \edef\y{\expandafter\z\meaning\y}%
    \expandafter\endgroup
    \ifx\x\y
      \expandafter\def\csname pdf@#1\expandafter\endcsname
      \expandafter{%
        \csname#1\endcsname
      }%
    \fi
  \fi
}%
\pdftexcmds@temp{shellescape}%
\pdftexcmds@temp{strcmp}%
%    \end{macrocode}
%
% \subsection{\cs{pdf@isprimitive}}
%
%    \begin{macrocode}
\def\pdf@isprimitive{%
  \begingroup\expandafter\expandafter\expandafter\endgroup
  \expandafter\ifx\csname pdf@strcmp\endcsname\relax
    \long\def\pdf@isprimitive##1{%
      \expandafter\pdftexcmds@isprimitive\expandafter{\meaning##1}%
    }%
    \long\def\pdftexcmds@isprimitive##1##2{%
      \expandafter\pdftexcmds@@isprimitive\expandafter{\string##2}{##1}%
    }%
    \def\pdftexcmds@@isprimitive##1##2{%
      \ifnum0\pdftexcmds@equal##1\delimiter##2\delimiter=1 %
        \expandafter\ltx@firstoftwo
      \else
        \expandafter\ltx@secondoftwo
      \fi
    }%
    \def\pdftexcmds@equal##1##2\delimiter##3##4\delimiter{%
      \ifx##1##3%
        \ifx\relax##2##4\relax
          1%
        \else
          \ifx\relax##2\relax
          \else
            \ifx\relax##4\relax
            \else
              \pdftexcmds@equalcont{##2}{##4}%
            \fi
          \fi
        \fi
      \fi
    }%
    \def\pdftexcmds@equalcont##1{%
      \def\pdftexcmds@equalcont####1####2##1##1##1##1{%
        ##1##1##1##1%
        \pdftexcmds@equal####1\delimiter####2\delimiter
      }%
    }%
    \expandafter\pdftexcmds@equalcont\csname fi\endcsname
  \else
    \long\def\pdf@isprimitive##1##2{%
      \ifnum\pdf@strcmp{\meaning##1}{\string##2}=0 %
        \expandafter\ltx@firstoftwo
      \else
        \expandafter\ltx@secondoftwo
      \fi
    }%
  \fi
}
\ifluatex
\ifx\pdfdraftmode\@undefined
  \let\pdfdraftmode\draftmode
\fi
\else
  \pdf@isprimitive
\fi
%    \end{macrocode}
%
% \subsection{\cs{pdf@draftmode}}
%
%
%    \begin{macrocode}
\let\pdftexcmds@temp\ltx@zero %
\ltx@IfUndefined{pdfdraftmode}{%
  \@PackageInfoNoLine{pdftexcmds}{\ltx@backslashchar pdfdraftmode not found}%
}{%
  \ifpdf
    \let\pdftexcmds@temp\ltx@one
    \@PackageInfoNoLine{pdftexcmds}{\ltx@backslashchar pdfdraftmode found}%
  \else
    \@PackageInfoNoLine{pdftexcmds}{%
      \ltx@backslashchar pdfdraftmode is ignored in DVI mode%
    }%
  \fi
}
\ifcase\pdftexcmds@temp
%    \end{macrocode}
%    \begin{macro}{\pdf@draftmode}
%    \begin{macrocode}
  \let\pdf@draftmode\ltx@zero
%    \end{macrocode}
%    \end{macro}
%    \begin{macro}{\pdf@ifdraftmode}
%    \begin{macrocode}
  \let\pdf@ifdraftmode\ltx@secondoftwo
%    \end{macrocode}
%    \end{macro}
%    \begin{macro}{\pdftexcmds@setdraftmode}
%    \begin{macrocode}
  \def\pdftexcmds@setdraftmode#1{}%
%    \end{macrocode}
%    \end{macro}
%    \begin{macrocode}
\else
%    \end{macrocode}
%    \begin{macro}{\pdftexcmds@draftmode}
%    \begin{macrocode}
  \let\pdftexcmds@draftmode\pdfdraftmode
%    \end{macrocode}
%    \end{macro}
%    \begin{macro}{\pdf@ifdraftmode}
%    \begin{macrocode}
  \def\pdf@ifdraftmode{%
    \ifnum\pdftexcmds@draftmode=\ltx@one
      \expandafter\ltx@firstoftwo
    \else
      \expandafter\ltx@secondoftwo
    \fi
  }%
%    \end{macrocode}
%    \end{macro}
%    \begin{macro}{\pdf@draftmode}
%    \begin{macrocode}
  \def\pdf@draftmode{%
    \ifnum\pdftexcmds@draftmode=\ltx@one
      \expandafter\ltx@one
    \else
      \expandafter\ltx@zero
    \fi
  }%
%    \end{macrocode}
%    \end{macro}
%    \begin{macro}{\pdftexcmds@setdraftmode}
%    \begin{macrocode}
  \def\pdftexcmds@setdraftmode#1{%
    \pdftexcmds@draftmode=#1\relax
  }%
%    \end{macrocode}
%    \end{macro}
%    \begin{macrocode}
\fi
%    \end{macrocode}
%    \begin{macro}{\pdf@setdraftmode}
%    \begin{macrocode}
\def\pdf@setdraftmode#1{%
  \begingroup
    \count\ltx@cclv=#1\relax
  \edef\x{\endgroup
    \noexpand\pdftexcmds@@setdraftmode{\the\count\ltx@cclv}%
  }%
  \x
}
%    \end{macrocode}
%    \end{macro}
%    \begin{macro}{\pdftexcmds@@setdraftmode}
%    \begin{macrocode}
\def\pdftexcmds@@setdraftmode#1{%
  \ifcase#1 %
    \pdftexcmds@setdraftmode{#1}%
  \or
    \pdftexcmds@setdraftmode{#1}%
  \else
    \@PackageWarning{pdftexcmds}{%
      \string\pdf@setdraftmode: Ignoring\MessageBreak
      invalid value `#1'%
    }%
  \fi
}
%    \end{macrocode}
%    \end{macro}
%
% \subsection{Load Lua module}
%
%    \begin{macrocode}
\ifluatex
\else
  \expandafter\pdftexcmds@AtEnd
\fi%
%    \end{macrocode}
%
%    \begin{macrocode}
\ifnum\luatexversion<80
  \begingroup\expandafter\expandafter\expandafter\endgroup
  \expandafter\ifx\csname RequirePackage\endcsname\relax
    \def\TMP@RequirePackage#1[#2]{%
      \begingroup\expandafter\expandafter\expandafter\endgroup
      \expandafter\ifx\csname ver@#1.sty\endcsname\relax
        \input #1.sty\relax
      \fi
    }%
    \TMP@RequirePackage{luatex-loader}[2009/04/10]%
  \else
    \RequirePackage{luatex-loader}[2009/04/10]%
  \fi
\fi
\pdftexcmds@directlua{%
  require("pdftexcmds")%
}
\ifnum\luatexversion>37 %
  \ifnum0%
      \pdftexcmds@directlua{%
        if status.ini_version then %
          tex.write("1")%
        end%
      }>0 %
    \everyjob\expandafter{%
      \the\everyjob
      \pdftexcmds@directlua{%
        require("pdftexcmds")%
      }%
    }%
  \fi
\fi
\begingroup
  \def\x{2019/07/25 v0.30}%
  \ltx@onelevel@sanitize\x
  \edef\y{%
    \pdftexcmds@directlua{%
      if oberdiek.pdftexcmds.getversion then %
        oberdiek.pdftexcmds.getversion()%
      end%
    }%
  }%
  \ifx\x\y
  \else
    \@PackageError{pdftexcmds}{%
      Wrong version of lua module.\MessageBreak
      Package version: \x\MessageBreak
      Lua module: \y
    }\@ehc
  \fi
\endgroup
%    \end{macrocode}
%
% \subsection{Lua functions}
%
% \subsubsection{Helper macros}
%
%    \begin{macro}{\pdftexcmds@toks}
%    \begin{macrocode}
\begingroup\expandafter\expandafter\expandafter\endgroup
\expandafter\ifx\csname newtoks\endcsname\relax
  \toksdef\pdftexcmds@toks=0 %
\else
  \csname newtoks\endcsname\pdftexcmds@toks
\fi
%    \end{macrocode}
%    \end{macro}
%
%    \begin{macro}{\pdftexcmds@Patch}
%    \begin{macrocode}
\def\pdftexcmds@Patch{0}
\ifnum\luatexversion>40 %
  \ifnum\luatexversion<66 %
    \def\pdftexcmds@Patch{1}%
  \fi
\fi
%    \end{macrocode}
%    \end{macro}
%    \begin{macrocode}
\ifcase\pdftexcmds@Patch
  \catcode`\&=14 %
\else
  \catcode`\&=9 %
%    \end{macrocode}
%    \begin{macro}{\pdftexcmds@PatchDecode}
%    \begin{macrocode}
  \def\pdftexcmds@PatchDecode#1\@nil{%
    \pdftexcmds@DecodeA#1^^A^^A\@nil{}%
  }%
%    \end{macrocode}
%    \end{macro}
%    \begin{macro}{\pdftexcmds@DecodeA}
%    \begin{macrocode}
  \def\pdftexcmds@DecodeA#1^^A^^A#2\@nil#3{%
    \ifx\relax#2\relax
      \ltx@ReturnAfterElseFi{%
        \pdftexcmds@DecodeB#3#1^^A^^B\@nil{}%
      }%
    \else
      \ltx@ReturnAfterFi{%
        \pdftexcmds@DecodeA#2\@nil{#3#1^^@}%
      }%
    \fi
  }%
%    \end{macrocode}
%    \end{macro}
%    \begin{macro}{\pdftexcmds@DecodeB}
%    \begin{macrocode}
  \def\pdftexcmds@DecodeB#1^^A^^B#2\@nil#3{%
    \ifx\relax#2\relax%
      \ltx@ReturnAfterElseFi{%
        \ltx@zero
        #3#1%
      }%
    \else
      \ltx@ReturnAfterFi{%
        \pdftexcmds@DecodeB#2\@nil{#3#1^^A}%
      }%
    \fi
  }%
%    \end{macrocode}
%    \end{macro}
%    \begin{macrocode}
\fi
%    \end{macrocode}
%
%    \begin{macrocode}
\ifnum\luatexversion<36 %
\else
  \catcode`\0=9 %
\fi
%    \end{macrocode}
%
% \subsubsection[Strings]{Strings \cite[``7.15 Strings'']{pdftex-manual}}
%
%    \begin{macro}{\pdf@strcmp}
%    \begin{macrocode}
\long\def\pdf@strcmp#1#2{%
  \directlua0{%
    oberdiek.pdftexcmds.strcmp("\luaescapestring{#1}",%
        "\luaescapestring{#2}")%
  }%
}%
%    \end{macrocode}
%    \end{macro}
%    \begin{macrocode}
\pdf@isprimitive
%    \end{macrocode}
%    \begin{macro}{\pdf@escapehex}
%    \begin{macrocode}
\long\def\pdf@escapehex#1{%
  \directlua0{%
    oberdiek.pdftexcmds.escapehex("\luaescapestring{#1}", "byte")%
  }%
}%
%    \end{macrocode}
%    \end{macro}
%    \begin{macro}{\pdf@escapehexnative}
%    \begin{macrocode}
\long\def\pdf@escapehexnative#1{%
  \directlua0{%
    oberdiek.pdftexcmds.escapehex("\luaescapestring{#1}")%
  }%
}%
%    \end{macrocode}
%    \end{macro}
%    \begin{macro}{\pdf@unescapehex}
%    \begin{macrocode}
\def\pdf@unescapehex#1{%
& \romannumeral\expandafter\pdftexcmds@PatchDecode
  \the\expandafter\pdftexcmds@toks
  \directlua0{%
    oberdiek.pdftexcmds.toks="pdftexcmds@toks"%
    oberdiek.pdftexcmds.unescapehex("\luaescapestring{#1}", "byte", \pdftexcmds@Patch)%
  }%
& \@nil
}%
%    \end{macrocode}
%    \end{macro}
%    \begin{macro}{\pdf@unescapehexnative}
%    \begin{macrocode}
\def\pdf@unescapehexnative#1{%
& \romannumeral\expandafter\pdftexcmds@PatchDecode
  \the\expandafter\pdftexcmds@toks
  \directlua0{%
    oberdiek.pdftexcmds.toks="pdftexcmds@toks"%
    oberdiek.pdftexcmds.unescapehex("\luaescapestring{#1}", \pdftexcmds@Patch)%
  }%
& \@nil
}%
%    \end{macrocode}
%    \end{macro}
%    \begin{macro}{\pdf@escapestring}
%    \begin{macrocode}
\long\def\pdf@escapestring#1{%
  \directlua0{%
    oberdiek.pdftexcmds.escapestring("\luaescapestring{#1}", "byte")%
  }%
}
%    \end{macrocode}
%    \end{macro}
%    \begin{macro}{\pdf@escapename}
%    \begin{macrocode}
\long\def\pdf@escapename#1{%
  \directlua0{%
    oberdiek.pdftexcmds.escapename("\luaescapestring{#1}", "byte")%
  }%
}
%    \end{macrocode}
%    \end{macro}
%    \begin{macro}{\pdf@escapenamenative}
%    \begin{macrocode}
\long\def\pdf@escapenamenative#1{%
  \directlua0{%
    oberdiek.pdftexcmds.escapename("\luaescapestring{#1}")%
  }%
}
%    \end{macrocode}
%    \end{macro}
%
% \subsubsection[Files]{Files \cite[``7.18 Files'']{pdftex-manual}}
%
%    \begin{macro}{\pdf@filesize}
%    \begin{macrocode}
\def\pdf@filesize#1{%
  \directlua0{%
    oberdiek.pdftexcmds.filesize("\luaescapestring{#1}")%
  }%
}
%    \end{macrocode}
%    \end{macro}
%    \begin{macro}{\pdf@filemoddate}
%    \begin{macrocode}
\def\pdf@filemoddate#1{%
  \directlua0{%
    oberdiek.pdftexcmds.filemoddate("\luaescapestring{#1}")%
  }%
}
%    \end{macrocode}
%    \end{macro}
%    \begin{macro}{\pdf@filedump}
%    \begin{macrocode}
\def\pdf@filedump#1#2#3{%
  \directlua0{%
    oberdiek.pdftexcmds.filedump("\luaescapestring{\number#1}",%
        "\luaescapestring{\number#2}",%
        "\luaescapestring{#3}")%
  }%
}%
%    \end{macrocode}
%    \end{macro}
%    \begin{macro}{\pdf@mdfivesum}
%    \begin{macrocode}
\long\def\pdf@mdfivesum#1{%
  \directlua0{%
    oberdiek.pdftexcmds.mdfivesum("\luaescapestring{#1}", "byte")%
  }%
}%
%    \end{macrocode}
%    \end{macro}
%    \begin{macro}{\pdf@mdfivesumnative}
%    \begin{macrocode}
\long\def\pdf@mdfivesumnative#1{%
  \directlua0{%
    oberdiek.pdftexcmds.mdfivesum("\luaescapestring{#1}")%
  }%
}%
%    \end{macrocode}
%    \end{macro}
%    \begin{macro}{\pdf@filemdfivesum}
%    \begin{macrocode}
\def\pdf@filemdfivesum#1{%
  \directlua0{%
    oberdiek.pdftexcmds.filemdfivesum("\luaescapestring{#1}")%
  }%
}%
%    \end{macrocode}
%    \end{macro}
%
% \subsubsection[Timekeeping]{Timekeeping \cite[``7.17 Timekeeping'']{pdftex-manual}}
%
%    \begin{macro}{\protected}
%    \begin{macrocode}
\let\pdftexcmds@temp=Y%
\begingroup\expandafter\expandafter\expandafter\endgroup
\expandafter\ifx\csname protected\endcsname\relax
  \pdftexcmds@directlua0{%
    if tex.enableprimitives then %
      tex.enableprimitives('', {'protected'})%
    end%
  }%
\fi
\begingroup\expandafter\expandafter\expandafter\endgroup
\expandafter\ifx\csname protected\endcsname\relax
  \let\pdftexcmds@temp=N%
\fi
%    \end{macrocode}
%    \end{macro}
%    \begin{macro}{\numexpr}
%    \begin{macrocode}
\begingroup\expandafter\expandafter\expandafter\endgroup
\expandafter\ifx\csname numexpr\endcsname\relax
  \pdftexcmds@directlua0{%
    if tex.enableprimitives then %
      tex.enableprimitives('', {'numexpr'})%
    end%
  }%
\fi
\begingroup\expandafter\expandafter\expandafter\endgroup
\expandafter\ifx\csname numexpr\endcsname\relax
  \let\pdftexcmds@temp=N%
\fi
%    \end{macrocode}
%    \end{macro}
%
%    \begin{macrocode}
\ifx\pdftexcmds@temp N%
  \@PackageWarningNoLine{pdftexcmds}{%
    Definitions of \ltx@backslashchar pdf@resettimer and%
    \MessageBreak
    \ltx@backslashchar pdf@elapsedtime are skipped, because%
    \MessageBreak
    e-TeX's \ltx@backslashchar protected or %
    \ltx@backslashchar numexpr are missing%
  }%
\else
%    \end{macrocode}
%
%    \begin{macro}{\pdf@resettimer}
%    \begin{macrocode}
  \protected\def\pdf@resettimer{%
    \pdftexcmds@directlua0{%
      oberdiek.pdftexcmds.resettimer()%
    }%
  }%
%    \end{macrocode}
%    \end{macro}
%
%    \begin{macro}{\pdf@elapsedtime}
%    \begin{macrocode}
  \protected\def\pdf@elapsedtime{%
    \numexpr
      \pdftexcmds@directlua0{%
        oberdiek.pdftexcmds.elapsedtime()%
      }%
    \relax
  }%
%    \end{macrocode}
%    \end{macro}
%    \begin{macrocode}
\fi
%    \end{macrocode}
%
% \subsubsection{Shell escape}
%
%    \begin{macro}{\pdf@shellescape}
%
%    \begin{macrocode}
\ifnum\luatexversion<68 %
\else
  \protected\edef\pdf@shellescape{%
   \numexpr\directlua{tex.sprint(%
         \number\catcodetable@string,status.shell_escape)}\relax}
\fi
%    \end{macrocode}
%    \end{macro}
%
%    \begin{macro}{\pdf@system}
%    \begin{macrocode}
\def\pdf@system#1{%
  \directlua0{%
    oberdiek.pdftexcmds.system("\luaescapestring{#1}")%
  }%
}
%    \end{macrocode}
%    \end{macro}
%
%    \begin{macro}{\pdf@lastsystemstatus}
%    \begin{macrocode}
\def\pdf@lastsystemstatus{%
  \directlua0{%
    oberdiek.pdftexcmds.lastsystemstatus()%
  }%
}
%    \end{macrocode}
%    \end{macro}
%    \begin{macro}{\pdf@lastsystemexit}
%    \begin{macrocode}
\def\pdf@lastsystemexit{%
  \directlua0{%
    oberdiek.pdftexcmds.lastsystemexit()%
  }%
}
%    \end{macrocode}
%    \end{macro}
%
%    \begin{macrocode}
\catcode`\0=12 %
%    \end{macrocode}
%
%    \begin{macro}{\pdf@pipe}
%    Check availability of |io.popen| first.
%    \begin{macrocode}
\ifnum0%
    \pdftexcmds@directlua{%
      if io.popen then %
        tex.write("1")%
      end%
    }%
    =1 %
  \def\pdf@pipe#1{%
&   \romannumeral\expandafter\pdftexcmds@PatchDecode
    \the\expandafter\pdftexcmds@toks
    \pdftexcmds@directlua{%
      oberdiek.pdftexcmds.toks="pdftexcmds@toks"%
      oberdiek.pdftexcmds.pipe("\luaescapestring{#1}", \pdftexcmds@Patch)%
    }%
&   \@nil
  }%
\fi
%    \end{macrocode}
%    \end{macro}
%
%    \begin{macrocode}
\pdftexcmds@AtEnd%
%</package>
%    \end{macrocode}
%
% \subsection{Lua module}
%
%    \begin{macrocode}
%<*lua>
%    \end{macrocode}
%
%    \begin{macrocode}
oberdiek = oberdiek or {}
local pdftexcmds = oberdiek.pdftexcmds or {}
oberdiek.pdftexcmds = pdftexcmds
local systemexitstatus
function pdftexcmds.getversion()
  tex.write("2019/07/25 v0.30")
end
%    \end{macrocode}
%
% \subsubsection[Strings]{Strings \cite[``7.15 Strings'']{pdftex-manual}}
%
%    \begin{macrocode}
function pdftexcmds.strcmp(A, B)
  if A == B then
    tex.write("0")
  elseif A < B then
    tex.write("-1")
  else
    tex.write("1")
  end
end
local function utf8_to_byte(str)
  local i = 0
  local n = string.len(str)
  local t = {}
  while i < n do
    i = i + 1
    local a = string.byte(str, i)
    if a < 128 then
      table.insert(t, string.char(a))
    else
      if a >= 192 and i < n then
        i = i + 1
        local b = string.byte(str, i)
        if b < 128 or b >= 192 then
          i = i - 1
        elseif a == 194 then
          table.insert(t, string.char(b))
        elseif a == 195 then
          table.insert(t, string.char(b + 64))
        end
      end
    end
  end
  return table.concat(t)
end
function pdftexcmds.escapehex(str, mode)
  if mode == "byte" then
    str = utf8_to_byte(str)
  end
  tex.write((string.gsub(str, ".",
    function (ch)
      return string.format("%02X", string.byte(ch))
    end
  )))
end
%    \end{macrocode}
%    See procedure |unescapehex| in file \xfile{utils.c} of \hologo{pdfTeX}.
%    Caution: |tex.write| ignores leading spaces.
%    \begin{macrocode}
function pdftexcmds.unescapehex(str, mode, patch)
  local a = 0
  local first = true
  local result = {}
  for i = 1, string.len(str), 1 do
    local ch = string.byte(str, i)
    if ch >= 48 and ch <= 57 then
      ch = ch - 48
    elseif ch >= 65 and ch <= 70 then
      ch = ch - 55
    elseif ch >= 97 and ch <= 102 then
      ch = ch - 87
    else
      ch = nil
    end
    if ch then
      if first then
        a = ch * 16
        first = false
      else
        table.insert(result, a + ch)
        first = true
      end
    end
  end
  if not first then
    table.insert(result, a)
  end
  if patch == 1 then
    local temp = {}
    for i, a in ipairs(result) do
      if a == 0 then
        table.insert(temp, 1)
        table.insert(temp, 1)
      else
        if a == 1 then
          table.insert(temp, 1)
          table.insert(temp, 2)
        else
          table.insert(temp, a)
        end
      end
    end
    result = temp
  end
  if mode == "byte" then
    local utf8 = {}
    for i, a in ipairs(result) do
      if a < 128 then
        table.insert(utf8, a)
      else
        if a < 192 then
          table.insert(utf8, 194)
          a = a - 128
        else
          table.insert(utf8, 195)
          a = a - 192
        end
        table.insert(utf8, a + 128)
      end
    end
    result = utf8
  end
%    \end{macrocode}
%    this next line added for current luatex; this is the only
%    change in the file.  eroux, 28apr13. (v 0.21)
%    \begin{macrocode}
  local unpack = _G["unpack"] or table.unpack
  tex.settoks(pdftexcmds.toks, string.char(unpack(result)))
end
%    \end{macrocode}
%    See procedure |escapestring| in file \xfile{utils.c} of \hologo{pdfTeX}.
%    \begin{macrocode}
function pdftexcmds.escapestring(str, mode)
  if mode == "byte" then
    str = utf8_to_byte(str)
  end
  tex.write((string.gsub(str, ".",
    function (ch)
      local b = string.byte(ch)
      if b < 33 or b > 126 then
        return string.format("\\%.3o", b)
      end
      if b == 40 or b == 41 or b == 92 then
        return "\\" .. ch
      end
%    \end{macrocode}
%    Lua 5.1 returns the match in case of return value |nil|.
%    \begin{macrocode}
      return nil
    end
  )))
end
%    \end{macrocode}
%    See procedure |escapename| in file \xfile{utils.c} of \hologo{pdfTeX}.
%    \begin{macrocode}
function pdftexcmds.escapename(str, mode)
  if mode == "byte" then
    str = utf8_to_byte(str)
  end
  tex.write((string.gsub(str, ".",
    function (ch)
      local b = string.byte(ch)
      if b == 0 then
%    \end{macrocode}
%    In Lua 5.0 |nil| could be used for the empty string,
%    But |nil| returns the match in Lua 5.1, thus we use
%    the empty string explicitly.
%    \begin{macrocode}
        return ""
      end
      if b <= 32 or b >= 127
          or b == 35 or b == 37 or b == 40 or b == 41
          or b == 47 or b == 60 or b == 62 or b == 91
          or b == 93 or b == 123 or b == 125 then
        return string.format("#%.2X", b)
      else
%    \end{macrocode}
%    Lua 5.1 returns the match in case of return value |nil|.
%    \begin{macrocode}
        return nil
      end
    end
  )))
end
%    \end{macrocode}
%
% \subsubsection[Files]{Files \cite[``7.18 Files'']{pdftex-manual}}
%
%    \begin{macrocode}
function pdftexcmds.filesize(filename)
  local foundfile = kpse.find_file(filename, "tex", true)
  if foundfile then
    local size = lfs.attributes(foundfile, "size")
    if size then
      tex.write(size)
    end
  end
end
%    \end{macrocode}
%    See procedure |makepdftime| in file \xfile{utils.c} of \hologo{pdfTeX}.
%    \begin{macrocode}
function pdftexcmds.filemoddate(filename)
  local foundfile = kpse.find_file(filename, "tex", true)
  if foundfile then
    local date = lfs.attributes(foundfile, "modification")
    if date then
      local d = os.date("*t", date)
      if d.sec >= 60 then
        d.sec = 59
      end
      local u = os.date("!*t", date)
      local off = 60 * (d.hour - u.hour) + d.min - u.min
      if d.year ~= u.year then
        if d.year > u.year then
          off = off + 1440
        else
          off = off - 1440
        end
      elseif d.yday ~= u.yday then
        if d.yday > u.yday then
          off = off + 1440
        else
          off = off - 1440
        end
      end
      local timezone
      if off == 0 then
        timezone = "Z"
      else
        local hours = math.floor(off / 60)
        local mins = math.abs(off - hours * 60)
        timezone = string.format("%+03d'%02d'", hours, mins)
      end
      tex.write(string.format("D:%04d%02d%02d%02d%02d%02d%s",
          d.year, d.month, d.day, d.hour, d.min, d.sec, timezone))
    end
  end
end
function pdftexcmds.filedump(offset, length, filename)
  length = tonumber(length)
  if length and length > 0 then
    local foundfile = kpse.find_file(filename, "tex", true)
    if foundfile then
      offset = tonumber(offset)
      if not offset then
        offset = 0
      end
      local filehandle = io.open(foundfile, "rb")
      if filehandle then
        if offset > 0 then
          filehandle:seek("set", offset)
        end
        local dump = filehandle:read(length)
        pdftexcmds.escapehex(dump)
        filehandle:close()
      end
    end
  end
end
function pdftexcmds.mdfivesum(str, mode)
  if mode == "byte" then
    str = utf8_to_byte(str)
  end
  pdftexcmds.escapehex(md5.sum(str))
end
function pdftexcmds.filemdfivesum(filename)
  local foundfile = kpse.find_file(filename, "tex", true)
  if foundfile then
    local filehandle = io.open(foundfile, "rb")
    if filehandle then
      local contents = filehandle:read("*a")
      pdftexcmds.escapehex(md5.sum(contents))
      filehandle:close()
    end
  end
end
%    \end{macrocode}
%
% \subsubsection[Timekeeping]{Timekeeping \cite[``7.17 Timekeeping'']{pdftex-manual}}
%
%    The functions for timekeeping are based on
%    Andy Thomas' work \cite{AndyThomas:Analog}.
%    Changes:
%    \begin{itemize}
%    \item Overflow check is added.
%    \item |string.format| is used to avoid exponential number
%          representation for sure.
%    \item |tex.write| is used instead of |tex.print| to get
%          tokens with catcode 12 and without appended \cs{endlinechar}.
%    \end{itemize}
%    \begin{macrocode}
local basetime = 0
function pdftexcmds.resettimer()
  basetime = os.clock()
end
function pdftexcmds.elapsedtime()
  local val = (os.clock() - basetime) * 65536 + .5
  if val > 2147483647 then
    val = 2147483647
  end
  tex.write(string.format("%d", val))
end
%    \end{macrocode}
%
% \subsubsection[Miscellaneous]{Miscellaneous \cite[``7.21 Miscellaneous'']{pdftex-manual}}
%
%    \begin{macrocode}
function pdftexcmds.shellescape()
  if os.execute then
    if status
        and status.luatex_version
        and status.luatex_version >= 68 then
      tex.write(os.execute())
    else
      local result = os.execute()
      if result == 0 then
        tex.write("0")
      else
        if result == nil then
          tex.write("0")
        else
          tex.write("1")
        end
      end
    end
  else
    tex.write("0")
  end
end
function pdftexcmds.system(cmdline)
  systemexitstatus = nil
  texio.write_nl("log", "system(" .. cmdline .. ") ")
  if os.execute then
    texio.write("log", "executed.")
    systemexitstatus = os.execute(cmdline)
  else
    texio.write("log", "disabled.")
  end
end
function pdftexcmds.lastsystemstatus()
  local result = tonumber(systemexitstatus)
  if result then
    local x = math.floor(result / 256)
    tex.write(result - 256 * math.floor(result / 256))
  end
end
function pdftexcmds.lastsystemexit()
  local result = tonumber(systemexitstatus)
  if result then
    tex.write(math.floor(result / 256))
  end
end
function pdftexcmds.pipe(cmdline, patch)
  local result
  systemexitstatus = nil
  texio.write_nl("log", "pipe(" .. cmdline ..") ")
  if io.popen then
    texio.write("log", "executed.")
    local handle = io.popen(cmdline, "r")
    if handle then
      result = handle:read("*a")
      handle:close()
    end
  else
    texio.write("log", "disabled.")
  end
  if result then
    if patch == 1 then
      local temp = {}
      for i, a in ipairs(result) do
        if a == 0 then
          table.insert(temp, 1)
          table.insert(temp, 1)
        else
          if a == 1 then
            table.insert(temp, 1)
            table.insert(temp, 2)
          else
            table.insert(temp, a)
          end
        end
      end
      result = temp
    end
    tex.settoks(pdftexcmds.toks, result)
  else
    tex.settoks(pdftexcmds.toks, "")
  end
end
%    \end{macrocode}
%    \begin{macrocode}
%</lua>
%    \end{macrocode}
%
% \section{Test}
%
% \subsection{Catcode checks for loading}
%
%    \begin{macrocode}
%<*test1>
%    \end{macrocode}
%    \begin{macrocode}
\catcode`\{=1 %
\catcode`\}=2 %
\catcode`\#=6 %
\catcode`\@=11 %
\expandafter\ifx\csname count@\endcsname\relax
  \countdef\count@=255 %
\fi
\expandafter\ifx\csname @gobble\endcsname\relax
  \long\def\@gobble#1{}%
\fi
\expandafter\ifx\csname @firstofone\endcsname\relax
  \long\def\@firstofone#1{#1}%
\fi
\expandafter\ifx\csname loop\endcsname\relax
  \expandafter\@firstofone
\else
  \expandafter\@gobble
\fi
{%
  \def\loop#1\repeat{%
    \def\body{#1}%
    \iterate
  }%
  \def\iterate{%
    \body
      \let\next\iterate
    \else
      \let\next\relax
    \fi
    \next
  }%
  \let\repeat=\fi
}%
\def\RestoreCatcodes{}
\count@=0 %
\loop
  \edef\RestoreCatcodes{%
    \RestoreCatcodes
    \catcode\the\count@=\the\catcode\count@\relax
  }%
\ifnum\count@<255 %
  \advance\count@ 1 %
\repeat

\def\RangeCatcodeInvalid#1#2{%
  \count@=#1\relax
  \loop
    \catcode\count@=15 %
  \ifnum\count@<#2\relax
    \advance\count@ 1 %
  \repeat
}
\def\RangeCatcodeCheck#1#2#3{%
  \count@=#1\relax
  \loop
    \ifnum#3=\catcode\count@
    \else
      \errmessage{%
        Character \the\count@\space
        with wrong catcode \the\catcode\count@\space
        instead of \number#3%
      }%
    \fi
  \ifnum\count@<#2\relax
    \advance\count@ 1 %
  \repeat
}
\def\space{ }
\expandafter\ifx\csname LoadCommand\endcsname\relax
  \def\LoadCommand{\input pdftexcmds.sty\relax}%
\fi
\def\Test{%
  \RangeCatcodeInvalid{0}{47}%
  \RangeCatcodeInvalid{58}{64}%
  \RangeCatcodeInvalid{91}{96}%
  \RangeCatcodeInvalid{123}{255}%
  \catcode`\@=12 %
  \catcode`\\=0 %
  \catcode`\%=14 %
  \LoadCommand
  \RangeCatcodeCheck{0}{36}{15}%
  \RangeCatcodeCheck{37}{37}{14}%
  \RangeCatcodeCheck{38}{47}{15}%
  \RangeCatcodeCheck{48}{57}{12}%
  \RangeCatcodeCheck{58}{63}{15}%
  \RangeCatcodeCheck{64}{64}{12}%
  \RangeCatcodeCheck{65}{90}{11}%
  \RangeCatcodeCheck{91}{91}{15}%
  \RangeCatcodeCheck{92}{92}{0}%
  \RangeCatcodeCheck{93}{96}{15}%
  \RangeCatcodeCheck{97}{122}{11}%
  \RangeCatcodeCheck{123}{255}{15}%
  \RestoreCatcodes
}
\Test
\csname @@end\endcsname
\end
%    \end{macrocode}
%    \begin{macrocode}
%</test1>
%    \end{macrocode}
%
% \subsection{Test for \cs{pdf@isprimitive}}
%
%    \begin{macrocode}
%<*test2>
\catcode`\{=1 %
\catcode`\}=2 %
\catcode`\#=6 %
\catcode`\@=11 %
\input pdftexcmds.sty\relax
\def\msg#1{%
  \begingroup
    \escapechar=92 %
    \immediate\write16{#1}%
  \endgroup
}
\long\def\test#1#2#3#4{%
  \begingroup
    #4%
    \def\str{%
      Test \string\pdf@isprimitive
      {\string #1}{\string #2}{...}: %
    }%
    \pdf@isprimitive{#1}{#2}{%
      \ifx#3Y%
        \msg{\str true ==> OK.}%
      \else
        \errmessage{\str false ==> FAILED}%
      \fi
    }{%
      \ifx#3Y%
        \errmessage{\str true ==> FAILED}%
      \else
        \msg{\str false ==> OK.}%
      \fi
    }%
  \endgroup
}
\test\relax\relax Y{}
\test\foobar\relax Y{\let\foobar\relax}
\test\foobar\relax N{}
\test\hbox\hbox Y{}
\test\foobar@hbox\hbox Y{\let\foobar@hbox\hbox}
\test\if\if Y{}
\test\if\ifx N{}
\test\ifx\if N{}
\test\par\par Y{}
\test\hbox\par N{}
\test\par\hbox N{}
\csname @@end\endcsname\end
%</test2>
%    \end{macrocode}
%
% \subsection{Test for \cs{pdf@shellescape}}
%
%    \begin{macrocode}
%<*test-shell>
\catcode`\{=1 %
\catcode`\}=2 %
\catcode`\#=6 %
\catcode`\@=11 %
\input pdftexcmds.sty\relax
\def\msg#{\immediate\write16}
\def\MaybeEnd{}
\ifx\luatexversion\UnDeFiNeD
\else
  \ifnum\luatexversion<68 %
    \ifx\pdf@shellescape\@undefined
      \msg{SHELL=U}%
      \msg{OK (LuaTeX < 0.68)}%
    \else
      \msg{SHELL=defined}%
      \errmessage{Failed (LuaTeX < 0.68)}%
    \fi
    \def\MaybeEnd{\csname @@end\endcsname\end}%
  \fi
\fi
\MaybeEnd
\ifx\pdf@shellescape\@undefined
  \msg{SHELL=U}%
\else
  \msg{SHELL=\number\pdf@shellescape}%
\fi
\ifx\expected\@undefined
\else
  \ifx\expected\relax
    \msg{EXPECTED=U}%
    \ifx\pdf@shellescape\@undefined
      \msg{OK}%
    \else
      \errmessage{Failed}%
    \fi
  \else
    \msg{EXPECTED=\number\expected}%
    \ifnum\pdf@shellescape=\expected\relax
      \msg{OK}%
    \else
      \errmessage{Failed}%
    \fi
  \fi
\fi
\csname @@end\endcsname\end
%</test-shell>
%    \end{macrocode}
%
% \subsection{Test for escape functions}
%
%    \begin{macrocode}
%<*test-escape>
\catcode`\{=1 %
\catcode`\}=2 %
\catcode`\#=6 %
\catcode`\^=7 %
\catcode`\@=11 %
\errorcontextlines=1000 %
\input pdftexcmds.sty\relax
\def\msg#1{%
  \begingroup
    \escapechar=92 %
    \immediate\write16{#1}%
  \endgroup
}
%    \end{macrocode}
%    \begin{macrocode}
\begingroup
  \catcode`\@=11 %
  \countdef\count@=255 %
  \def\space{ }%
  \long\def\@whilenum#1\do #2{%
    \ifnum #1\relax
      #2\relax
      \@iwhilenum{#1\relax#2\relax}%
    \fi
  }%
  \long\def\@iwhilenum#1{%
    \ifnum #1%
      \expandafter\@iwhilenum
    \else
      \expandafter\ltx@gobble
    \fi
    {#1}%
  }%
  \gdef\AllBytes{}%
  \count@=0 %
  \catcode0=12 %
  \@whilenum\count@<256 \do{%
    \lccode0=\count@
    \ifnum\count@=32 %
      \xdef\AllBytes{\AllBytes\space}%
    \else
      \lowercase{%
        \xdef\AllBytes{\AllBytes^^@}%
      }%
    \fi
    \advance\count@ by 1 %
  }%
\endgroup
%    \end{macrocode}
%    \begin{macrocode}
\def\AllBytesHex{%
  000102030405060708090A0B0C0D0E0F%
  101112131415161718191A1B1C1D1E1F%
  202122232425262728292A2B2C2D2E2F%
  303132333435363738393A3B3C3D3E3F%
  404142434445464748494A4B4C4D4E4F%
  505152535455565758595A5B5C5D5E5F%
  606162636465666768696A6B6C6D6E6F%
  707172737475767778797A7B7C7D7E7F%
  808182838485868788898A8B8C8D8E8F%
  909192939495969798999A9B9C9D9E9F%
  A0A1A2A3A4A5A6A7A8A9AAABACADAEAF%
  B0B1B2B3B4B5B6B7B8B9BABBBCBDBEBF%
  C0C1C2C3C4C5C6C7C8C9CACBCCCDCECF%
  D0D1D2D3D4D5D6D7D8D9DADBDCDDDEDF%
  E0E1E2E3E4E5E6E7E8E9EAEBECEDEEEF%
  F0F1F2F3F4F5F6F7F8F9FAFBFCFDFEFF%
}
\ltx@onelevel@sanitize\AllBytesHex
\expandafter\lowercase\expandafter{%
  \expandafter\def\expandafter\AllBytesHexLC
      \expandafter{\AllBytesHex}%
}
\begingroup
  \catcode`\#=12 %
  \xdef\AllBytesName{%
    #01#02#03#04#05#06#07#08#09#0A#0B#0C#0D#0E#0F%
    #10#11#12#13#14#15#16#17#18#19#1A#1B#1C#1D#1E#1F%
    #20!"#23$#25&'#28#29*+,-.#2F%
    0123456789:;#3C=#3E?%
    @ABCDEFGHIJKLMNO%
    PQRSTUVWXYZ#5B\ltx@backslashchar#5D^_%
    `abcdefghijklmno%
    pqrstuvwxyz#7B|#7D\string~#7F%
    #80#81#82#83#84#85#86#87#88#89#8A#8B#8C#8D#8E#8F%
    #90#91#92#93#94#95#96#97#98#99#9A#9B#9C#9D#9E#9F%
    #A0#A1#A2#A3#A4#A5#A6#A7#A8#A9#AA#AB#AC#AD#AE#AF%
    #B0#B1#B2#B3#B4#B5#B6#B7#B8#B9#BA#BB#BC#BD#BE#BF%
    #C0#C1#C2#C3#C4#C5#C6#C7#C8#C9#CA#CB#CC#CD#CE#CF%
    #D0#D1#D2#D3#D4#D5#D6#D7#D8#D9#DA#DB#DC#DD#DE#DF%
    #E0#E1#E2#E3#E4#E5#E6#E7#E8#E9#EA#EB#EC#ED#EE#EF%
    #F0#F1#F2#F3#F4#F5#F6#F7#F8#F9#FA#FB#FC#FD#FE#FF%
  }%
\endgroup
\ltx@onelevel@sanitize\AllBytesName
\edef\AllBytesFromName{\expandafter\ltx@gobble\AllBytes}
\begingroup
  \def\|{|}%
  \edef\%{\ltx@percentchar}%
  \catcode`\|=0 %
  \catcode`\#=12 %
  \catcode`\~=12 %
  \catcode`\\=12 %
  |xdef|AllBytesString{%
    \000\001\002\003\004\005\006\007\010\011\012\013\014\015\016\017%
    \020\021\022\023\024\025\026\027\030\031\032\033\034\035\036\037%
    \040!"#$|%&'\(\)*+,-./%
    0123456789:;<=>?%
    @ABCDEFGHIJKLMNO%
    PQRSTUVWXYZ[\\]^_%
    `abcdefghijklmno%
    pqrstuvwxyz{||}~\177%
    \200\201\202\203\204\205\206\207\210\211\212\213\214\215\216\217%
    \220\221\222\223\224\225\226\227\230\231\232\233\234\235\236\237%
    \240\241\242\243\244\245\246\247\250\251\252\253\254\255\256\257%
    \260\261\262\263\264\265\266\267\270\271\272\273\274\275\276\277%
    \300\301\302\303\304\305\306\307\310\311\312\313\314\315\316\317%
    \320\321\322\323\324\325\326\327\330\331\332\333\334\335\336\337%
    \340\341\342\343\344\345\346\347\350\351\352\353\354\355\356\357%
    \360\361\362\363\364\365\366\367\370\371\372\373\374\375\376\377%
  }%
|endgroup
\ltx@onelevel@sanitize\AllBytesString
%    \end{macrocode}
%    \begin{macrocode}
\def\Test#1#2#3{%
  \begingroup
    \expandafter\expandafter\expandafter\def
    \expandafter\expandafter\expandafter\TestResult
    \expandafter\expandafter\expandafter{%
      #1{#2}%
    }%
    \ifx\TestResult#3%
    \else
      \newlinechar=10 %
      \msg{Expect:^^J#3}%
      \msg{Result:^^J\TestResult}%
      \errmessage{\string#2 -\string#1-> \string#3}%
    \fi
  \endgroup
}
\def\test#1#2#3{%
  \edef\TestFrom{#2}%
  \edef\TestExpect{#3}%
  \ltx@onelevel@sanitize\TestExpect
  \Test#1\TestFrom\TestExpect
}
\test\pdf@unescapehex{74657374}{test}
\begingroup
  \catcode0=12 %
  \catcode1=12 %
  \test\pdf@unescapehex{740074017400740174}{t^^@t^^At^^@t^^At}%
\endgroup
\Test\pdf@escapehex\AllBytes\AllBytesHex
\Test\pdf@unescapehex\AllBytesHex\AllBytes
\Test\pdf@escapename\AllBytes\AllBytesName
\Test\pdf@escapestring\AllBytes\AllBytesString
%    \end{macrocode}
%    \begin{macrocode}
\csname @@end\endcsname\end
%</test-escape>
%    \end{macrocode}
%
% \section{Installation}
%
% \subsection{Download}
%
% \paragraph{Package.} This package is available on
% CTAN\footnote{\CTANpkg{pdftexcmds}}:
% \begin{description}
% \item[\CTAN{macros/latex/contrib/oberdiek/pdftexcmds.dtx}] The source file.
% \item[\CTAN{macros/latex/contrib/oberdiek/pdftexcmds.pdf}] Documentation.
% \end{description}
%
%
% \paragraph{Bundle.} All the packages of the bundle `oberdiek'
% are also available in a TDS compliant ZIP archive. There
% the packages are already unpacked and the documentation files
% are generated. The files and directories obey the TDS standard.
% \begin{description}
% \item[\CTANinstall{install/macros/latex/contrib/oberdiek.tds.zip}]
% \end{description}
% \emph{TDS} refers to the standard ``A Directory Structure
% for \TeX\ Files'' (\CTAN{tds/tds.pdf}). Directories
% with \xfile{texmf} in their name are usually organized this way.
%
% \subsection{Bundle installation}
%
% \paragraph{Unpacking.} Unpack the \xfile{oberdiek.tds.zip} in the
% TDS tree (also known as \xfile{texmf} tree) of your choice.
% Example (linux):
% \begin{quote}
%   |unzip oberdiek.tds.zip -d ~/texmf|
% \end{quote}
%
% \paragraph{Script installation.}
% Check the directory \xfile{TDS:scripts/oberdiek/} for
% scripts that need further installation steps.
% Package \xpackage{attachfile2} comes with the Perl script
% \xfile{pdfatfi.pl} that should be installed in such a way
% that it can be called as \texttt{pdfatfi}.
% Example (linux):
% \begin{quote}
%   |chmod +x scripts/oberdiek/pdfatfi.pl|\\
%   |cp scripts/oberdiek/pdfatfi.pl /usr/local/bin/|
% \end{quote}
%
% \subsection{Package installation}
%
% \paragraph{Unpacking.} The \xfile{.dtx} file is a self-extracting
% \docstrip\ archive. The files are extracted by running the
% \xfile{.dtx} through \plainTeX:
% \begin{quote}
%   \verb|tex pdftexcmds.dtx|
% \end{quote}
%
% \paragraph{TDS.} Now the different files must be moved into
% the different directories in your installation TDS tree
% (also known as \xfile{texmf} tree):
% \begin{quote}
% \def\t{^^A
% \begin{tabular}{@{}>{\ttfamily}l@{ $\rightarrow$ }>{\ttfamily}l@{}}
%   pdftexcmds.sty & tex/generic/oberdiek/pdftexcmds.sty\\
%   oberdiek.pdftexcmds.lua & scripts/oberdiek/oberdiek.pdftexcmds.lua\\
%   pdftexcmds.lua & scripts/oberdiek/pdftexcmds.lua\\
%   pdftexcmds.pdf & doc/latex/oberdiek/pdftexcmds.pdf\\
%   test/pdftexcmds-test1.tex & doc/latex/oberdiek/test/pdftexcmds-test1.tex\\
%   test/pdftexcmds-test2.tex & doc/latex/oberdiek/test/pdftexcmds-test2.tex\\
%   test/pdftexcmds-test-shell.tex & doc/latex/oberdiek/test/pdftexcmds-test-shell.tex\\
%   test/pdftexcmds-test-escape.tex & doc/latex/oberdiek/test/pdftexcmds-test-escape.tex\\
%   pdftexcmds.dtx & source/latex/oberdiek/pdftexcmds.dtx\\
% \end{tabular}^^A
% }^^A
% \sbox0{\t}^^A
% \ifdim\wd0>\linewidth
%   \begingroup
%     \advance\linewidth by\leftmargin
%     \advance\linewidth by\rightmargin
%   \edef\x{\endgroup
%     \def\noexpand\lw{\the\linewidth}^^A
%   }\x
%   \def\lwbox{^^A
%     \leavevmode
%     \hbox to \linewidth{^^A
%       \kern-\leftmargin\relax
%       \hss
%       \usebox0
%       \hss
%       \kern-\rightmargin\relax
%     }^^A
%   }^^A
%   \ifdim\wd0>\lw
%     \sbox0{\small\t}^^A
%     \ifdim\wd0>\linewidth
%       \ifdim\wd0>\lw
%         \sbox0{\footnotesize\t}^^A
%         \ifdim\wd0>\linewidth
%           \ifdim\wd0>\lw
%             \sbox0{\scriptsize\t}^^A
%             \ifdim\wd0>\linewidth
%               \ifdim\wd0>\lw
%                 \sbox0{\tiny\t}^^A
%                 \ifdim\wd0>\linewidth
%                   \lwbox
%                 \else
%                   \usebox0
%                 \fi
%               \else
%                 \lwbox
%               \fi
%             \else
%               \usebox0
%             \fi
%           \else
%             \lwbox
%           \fi
%         \else
%           \usebox0
%         \fi
%       \else
%         \lwbox
%       \fi
%     \else
%       \usebox0
%     \fi
%   \else
%     \lwbox
%   \fi
% \else
%   \usebox0
% \fi
% \end{quote}
% If you have a \xfile{docstrip.cfg} that configures and enables \docstrip's
% TDS installing feature, then some files can already be in the right
% place, see the documentation of \docstrip.
%
% \subsection{Refresh file name databases}
%
% If your \TeX~distribution
% (\teTeX, \mikTeX, \dots) relies on file name databases, you must refresh
% these. For example, \teTeX\ users run \verb|texhash| or
% \verb|mktexlsr|.
%
% \subsection{Some details for the interested}
%
% \paragraph{Unpacking with \LaTeX.}
% The \xfile{.dtx} chooses its action depending on the format:
% \begin{description}
% \item[\plainTeX:] Run \docstrip\ and extract the files.
% \item[\LaTeX:] Generate the documentation.
% \end{description}
% If you insist on using \LaTeX\ for \docstrip\ (really,
% \docstrip\ does not need \LaTeX), then inform the autodetect routine
% about your intention:
% \begin{quote}
%   \verb|latex \let\install=y\input{pdftexcmds.dtx}|
% \end{quote}
% Do not forget to quote the argument according to the demands
% of your shell.
%
% \paragraph{Generating the documentation.}
% You can use both the \xfile{.dtx} or the \xfile{.drv} to generate
% the documentation. The process can be configured by the
% configuration file \xfile{ltxdoc.cfg}. For instance, put this
% line into this file, if you want to have A4 as paper format:
% \begin{quote}
%   \verb|\PassOptionsToClass{a4paper}{article}|
% \end{quote}
% An example follows how to generate the
% documentation with pdf\LaTeX:
% \begin{quote}
%\begin{verbatim}
%pdflatex pdftexcmds.dtx
%bibtex pdftexcmds.aux
%makeindex -s gind.ist pdftexcmds.idx
%pdflatex pdftexcmds.dtx
%makeindex -s gind.ist pdftexcmds.idx
%pdflatex pdftexcmds.dtx
%\end{verbatim}
% \end{quote}
%
% \printbibliography[
%   heading=bibnumbered,
% ]
%
% \begin{History}
%   \begin{Version}{2007/11/11 v0.1}
%   \item
%     First version.
%   \end{Version}
%   \begin{Version}{2007/11/12 v0.2}
%   \item
%     Short description fixed.
%   \end{Version}
%   \begin{Version}{2007/12/12 v0.3}
%   \item
%     Organization of Lua code as module.
%   \end{Version}
%   \begin{Version}{2009/04/10 v0.4}
%   \item
%     Adaptation for syntax change of \cs{directlua} in
%     \hologo{LuaTeX} 0.36.
%   \end{Version}
%   \begin{Version}{2009/09/22 v0.5}
%   \item
%     \cs{pdf@primitive}, \cs{pdf@ifprimitive} added.
%   \item
%     \hologo{XeTeX}'s variants are detected for
%     \cs{pdf@shellescape}, \cs{pdf@strcmp}, \cs{pdf@primitive},
%     \cs{pdf@ifprimitive}.
%   \end{Version}
%   \begin{Version}{2009/09/23 v0.6}
%   \item
%     Macro \cs{pdf@isprimitive} added.
%   \end{Version}
%   \begin{Version}{2009/12/12 v0.7}
%   \item
%     Short info shortened.
%   \end{Version}
%   \begin{Version}{2010/03/01 v0.8}
%   \item
%     Required date for package \xpackage{ifluatex} updated.
%   \end{Version}
%   \begin{Version}{2010/04/01 v0.9}
%   \item
%     Use \cs{ifeof18} for defining \cs{pdf@shellescape} between
%     \hologo{pdfTeX} 1.21a (inclusive) and 1.30.0 (exclusive).
%   \end{Version}
%   \begin{Version}{2010/11/04 v0.10}
%   \item
%     \cs{pdf@draftmode}, \cs{pdf@ifdraftmode} and
%     \cs{pdf@setdraftmode} added.
%   \end{Version}
%   \begin{Version}{2010/11/11 v0.11}
%   \item
%     Missing \cs{RequirePackage} for package \xpackage{ifpdf} added.
%   \end{Version}
%   \begin{Version}{2011/01/30 v0.12}
%   \item
%     Already loaded package files are not input in \hologo{plainTeX}.
%   \end{Version}
%   \begin{Version}{2011/03/04 v0.13}
%   \item
%     Improved Lua function \texttt{shellescape} that also
%     uses the result of \texttt{os.execute()} (thanks to Philipp Stephani).
%   \end{Version}
%   \begin{Version}{2011/04/10 v0.14}
%   \item
%     Version check of loaded module added.
%   \item
%     Patch for bug in \hologo{LuaTeX} between 0.40.6 and 0.65 that
%     is fixed in revision 4096.
%   \end{Version}
%   \begin{Version}{2011/04/16 v0.15}
%   \item
%     \hologo{LuaTeX}: \cs{pdf@shellescape} is only supported
%     for version 0.70.0 and higher due to a bug, \texttt{os.execute()}
%     crashes in some circumstances. Fixed in \hologo{LuaTeX}
%     beta-0.70.0, revision 4167.
%   \end{Version}
%   \begin{Version}{2011/04/22 v0.16}
%   \item
%     Previous fix was not working due to a wrong catcode of digit
%     zero (due to easily support the old \cs{directlua0}).
%     The version border is lowered to 0.68, because some
%     beta-0.67.0 seems also to work.
%   \end{Version}
%   \begin{Version}{2011/06/29 v0.17}
%   \item
%     Documentation addition to \cs{pdf@shellescape}.
%   \end{Version}
%   \begin{Version}{2011/07/01 v0.18}
%   \item
%     Add Lua module loading in \cs{everyjob} for \hologo{iniTeX}
%     (\hologo{LuaTeX} only).
%   \end{Version}
%   \begin{Version}{2011/07/28 v0.19}
%   \item
%     Missing space in an info message added (Martin M\"unch).
%   \end{Version}
%   \begin{Version}{2011/11/29 v0.20}
%   \item
%     \cs{pdf@resettimer} and \cs{pdf@elapsedtime} added
%     (thanks Andy Thomas).
%   \end{Version}
%   \begin{Version}{2016/05/10 v0.21}
%   \item
%      local unpack added
%     (thanks \'{E}lie Roux).
%   \end{Version}
%   \begin{Version}{2016/05/21 v0.22}
%   \item
%     adjust \cs{textbackslas}h usage in bib file for biber bug.
%   \end{Version}
%   \begin{Version}{2016/10/02 v0.23}
%   \item
%     add file.close to lua filehandles (github pull request).
%   \end{Version}
%   \begin{Version}{2017/01/29 v0.24}
%   \item
%     Avoid loading luatex-loader for current luatex. (Use
%     pdftexcmds.lua not oberdiek.pdftexcmds.lua to simplify file
%     search with standard require)
%   \end{Version}
%   \begin{Version}{2017/03/19 v0.25}
%   \item
%     New \cs{pdf@shellescape} for Lua\TeX, see github issue 20.
%   \end{Version}
%   \begin{Version}{2018/01/21 v0.26}
%   \item
%     use rb not r mode for file open github issue 34.
%   \end{Version}
%   \begin{Version}{2018/01/30 v0.27}
%   \item
%     \cs{pdf@mdfivesum} for \hologo{XeTeX}
%   \end{Version}
%   \begin{Version}{2018/09/07 v0.28}
%   \item
%     Fix catcode regime in luatex sprint for \cs{pdf@shellescape} GH issue 45
%   \end{Version}
%   \begin{Version}{2018/09/10 v0.29}
%   \item
%     Actually do the fix described above in the code, not just document it.
%   \end{Version}
%   \begin{Version}{2019/07/25 v0.30}
%   \item
%     remove uses of module function, see PR70
%   \end{Version}
% \end{History}
%
% \PrintIndex
%
% \Finale
\endinput

%        (quote the arguments according to the demands of your shell)
%
% Documentation:
%    (a) If pdftexcmds.drv is present:
%           latex pdftexcmds.drv
%    (b) Without pdftexcmds.drv:
%           latex pdftexcmds.dtx; ...
%    The class ltxdoc loads the configuration file ltxdoc.cfg
%    if available. Here you can specify further options, e.g.
%    use A4 as paper format:
%       \PassOptionsToClass{a4paper}{article}
%
%    Programm calls to get the documentation (example):
%       pdflatex pdftexcmds.dtx
%       bibtex pdftexcmds.aux
%       makeindex -s gind.ist pdftexcmds.idx
%       pdflatex pdftexcmds.dtx
%       makeindex -s gind.ist pdftexcmds.idx
%       pdflatex pdftexcmds.dtx
%
% Installation:
%    TDS:tex/generic/oberdiek/pdftexcmds.sty
%    TDS:scripts/oberdiek/oberdiek.pdftexcmds.lua
%    TDS:scripts/oberdiek/pdftexcmds.lua
%    TDS:doc/latex/oberdiek/pdftexcmds.pdf
%    TDS:doc/latex/oberdiek/test/pdftexcmds-test1.tex
%    TDS:doc/latex/oberdiek/test/pdftexcmds-test2.tex
%    TDS:doc/latex/oberdiek/test/pdftexcmds-test-shell.tex
%    TDS:doc/latex/oberdiek/test/pdftexcmds-test-escape.tex
%    TDS:source/latex/oberdiek/pdftexcmds.dtx
%
%<*ignore>
\begingroup
  \catcode123=1 %
  \catcode125=2 %
  \def\x{LaTeX2e}%
\expandafter\endgroup
\ifcase 0\ifx\install y1\fi\expandafter
         \ifx\csname processbatchFile\endcsname\relax\else1\fi
         \ifx\fmtname\x\else 1\fi\relax
\else\csname fi\endcsname
%</ignore>
%<*install>
\input docstrip.tex
\Msg{************************************************************************}
\Msg{* Installation}
\Msg{* Package: pdftexcmds 2019/07/25 v0.30 Utility functions of pdfTeX for LuaTeX (HO)}
\Msg{************************************************************************}

\keepsilent
\askforoverwritefalse

\let\MetaPrefix\relax
\preamble

This is a generated file.

Project: pdftexcmds
Version: 2019/07/25 v0.30

Copyright (C) 2007, 2009-2011 by
   Heiko Oberdiek <heiko.oberdiek at googlemail.com>

This work may be distributed and/or modified under the
conditions of the LaTeX Project Public License, either
version 1.3c of this license or (at your option) any later
version. This version of this license is in
   https://www.latex-project.org/lppl/lppl-1-3c.txt
and the latest version of this license is in
   https://www.latex-project.org/lppl.txt
and version 1.3 or later is part of all distributions of
LaTeX version 2005/12/01 or later.

This work has the LPPL maintenance status "maintained".

The Current Maintainers of this work are
Heiko Oberdiek and the Oberdiek Package Support Group
https://github.com/ho-tex/oberdiek/issues


The Base Interpreter refers to any `TeX-Format',
because some files are installed in TDS:tex/generic//.

This work consists of the main source file pdftexcmds.dtx
and the derived files
   pdftexcmds.sty, pdftexcmds.pdf, pdftexcmds.ins, pdftexcmds.drv,
   pdftexcmds.bib, pdftexcmds-test1.tex, pdftexcmds-test2.tex,
   pdftexcmds-test-shell.tex, pdftexcmds-test-escape.tex,
   oberdiek.pdftexcmds.lua, pdftexcmds.lua.

\endpreamble
\let\MetaPrefix\DoubleperCent

\generate{%
  \file{pdftexcmds.ins}{\from{pdftexcmds.dtx}{install}}%
  \file{pdftexcmds.drv}{\from{pdftexcmds.dtx}{driver}}%
  \nopreamble
  \nopostamble
  \file{pdftexcmds.bib}{\from{pdftexcmds.dtx}{bib}}%
  \usepreamble\defaultpreamble
  \usepostamble\defaultpostamble
  \usedir{tex/generic/oberdiek}%
  \file{pdftexcmds.sty}{\from{pdftexcmds.dtx}{package}}%
%  \usedir{doc/latex/oberdiek/test}%
%  \file{pdftexcmds-test1.tex}{\from{pdftexcmds.dtx}{test1}}%
%  \file{pdftexcmds-test2.tex}{\from{pdftexcmds.dtx}{test2}}%
%  \file{pdftexcmds-test-shell.tex}{\from{pdftexcmds.dtx}{test-shell}}%
%  \file{pdftexcmds-test-escape.tex}{\from{pdftexcmds.dtx}{test-escape}}%
  \nopreamble
  \nopostamble
%  \usedir{source/latex/oberdiek/catalogue}%
%  \file{pdftexcmds.xml}{\from{pdftexcmds.dtx}{catalogue}}%
}
\def\MetaPrefix{-- }
\def\defaultpostamble{%
  \MetaPrefix^^J%
  \MetaPrefix\space End of File `\outFileName'.%
}
\def\currentpostamble{\defaultpostamble}%
\generate{%
  \usedir{scripts/oberdiek}%
  \file{oberdiek.pdftexcmds.lua}{\from{pdftexcmds.dtx}{lua}}%
  \file{pdftexcmds.lua}{\from{pdftexcmds.dtx}{lua}}%
}

\catcode32=13\relax% active space
\let =\space%
\Msg{************************************************************************}
\Msg{*}
\Msg{* To finish the installation you have to move the following}
\Msg{* file into a directory searched by TeX:}
\Msg{*}
\Msg{*     pdftexcmds.sty}
\Msg{*}
\Msg{* And install the following script files:}
\Msg{*}
\Msg{*     oberdiek.pdftexcmds.lua, pdftexcmds.lua}
\Msg{*}
\Msg{* To produce the documentation run the file `pdftexcmds.drv'}
\Msg{* through LaTeX.}
\Msg{*}
\Msg{* Happy TeXing!}
\Msg{*}
\Msg{************************************************************************}

\endbatchfile
%</install>
%<*bib>
@online{AndyThomas:Analog,
  author={Thomas, Andy},
  title={Analog of {\texttt{\csname textbackslash\endcsname}pdfelapsedtime} for
      {\hologo{LuaTeX}} and {\hologo{XeTeX}}},
  url={http://tex.stackexchange.com/a/32531},
  urldate={2011-11-29},
}
%</bib>
%<*ignore>
\fi
%</ignore>
%<*driver>
\NeedsTeXFormat{LaTeX2e}
\ProvidesFile{pdftexcmds.drv}%
  [2019/07/25 v0.30 Utility functions of pdfTeX for LuaTeX (HO)]%
\documentclass{ltxdoc}
\usepackage{holtxdoc}[2011/11/22]
\usepackage{paralist}
\usepackage{csquotes}
\usepackage[
  backend=bibtex,
  bibencoding=ascii,
  alldates=iso8601,
]{biblatex}[2011/11/13]
\bibliography{oberdiek-source}
\bibliography{pdftexcmds}
\begin{document}
  \DocInput{pdftexcmds.dtx}%
\end{document}
%</driver>
% \fi
%
%
% \CharacterTable
%  {Upper-case    \A\B\C\D\E\F\G\H\I\J\K\L\M\N\O\P\Q\R\S\T\U\V\W\X\Y\Z
%   Lower-case    \a\b\c\d\e\f\g\h\i\j\k\l\m\n\o\p\q\r\s\t\u\v\w\x\y\z
%   Digits        \0\1\2\3\4\5\6\7\8\9
%   Exclamation   \!     Double quote  \"     Hash (number) \#
%   Dollar        \$     Percent       \%     Ampersand     \&
%   Acute accent  \'     Left paren    \(     Right paren   \)
%   Asterisk      \*     Plus          \+     Comma         \,
%   Minus         \-     Point         \.     Solidus       \/
%   Colon         \:     Semicolon     \;     Less than     \<
%   Equals        \=     Greater than  \>     Question mark \?
%   Commercial at \@     Left bracket  \[     Backslash     \\
%   Right bracket \]     Circumflex    \^     Underscore    \_
%   Grave accent  \`     Left brace    \{     Vertical bar  \|
%   Right brace   \}     Tilde         \~}
%
% \GetFileInfo{pdftexcmds.drv}
%
% \title{The \xpackage{pdftexcmds} package}
% \date{2019/07/25 v0.30}
% \author{Heiko Oberdiek\thanks
% {Please report any issues at \url{https://github.com/ho-tex/oberdiek/issues}}}
%
% \maketitle
%
% \begin{abstract}
% \hologo{LuaTeX} provides most of the commands of \hologo{pdfTeX} 1.40. However
% a number of utility functions are removed. This package tries to fill
% the gap and implements some of the missing primitive using Lua.
% \end{abstract}
%
% \tableofcontents
%
% \def\csi#1{\texttt{\textbackslash\textit{#1}}}
%
% \section{Documentation}
%
% Some primitives of \hologo{pdfTeX} \cite{pdftex-manual}
% are not defined by \hologo{LuaTeX} \cite{luatex-manual}.
% This package implements macro based solutions using Lua code
% for the following missing \hologo{pdfTeX} primitives;
% \begin{compactitem}
% \item \cs{pdfstrcmp}
% \item \cs{pdfunescapehex}
% \item \cs{pdfescapehex}
% \item \cs{pdfescapename}
% \item \cs{pdfescapestring}
% \item \cs{pdffilesize}
% \item \cs{pdffilemoddate}
% \item \cs{pdffiledump}
% \item \cs{pdfmdfivesum}
% \item \cs{pdfresettimer}
% \item \cs{pdfelapsedtime}
% \item |\immediate\write18|
% \end{compactitem}
% The original names of the primitives cannot be used:
% \begin{itemize}
% \item
% The syntax for their arguments cannot easily
% simulated by macros. The primitives using key words
% such as |file| (\cs{pdfmdfivesum}) or |offset| and |length|
% (\cs{pdffiledump}) and uses \meta{general text} for the other
% arguments. Using token registers assignments, \meta{general text} could
% be catched. However, the simulated primitives are expandable
% and register assignments would destroy this important property.
% (\meta{general text} allows something like |\expandafter\bgroup ...}|.)
% \item
% The original primitives can be expanded using one expansion step.
% The new macros need two expansion steps because of the additional
% macro expansion. Example:
% \begin{quote}
%   |\expandafter\foo\pdffilemoddate{file}|\\
%   vs.\\
%   |\expandafter\expandafter\expandafter|\\
%   |\foo\pdf@filemoddate{file}|
% \end{quote}
% \end{itemize}
%
% \hologo{LuaTeX} isn't stable yet and thus the status of this package is
% \emph{experimental}. Feedback is welcome.
%
% \subsection{General principles}
%
% \begin{description}
% \item[Naming convention:]
%   Usually this package defines a macro |\pdf@|\meta{cmd} if
%   \hologo{pdfTeX} provides |\pdf|\meta{cmd}.
% \item[Arguments:] The order of arguments in |\pdf@|\meta{cmd}
%   is the same as for the corresponding primitive of \hologo{pdfTeX}.
%   The arguments are ordinary undelimited \hologo{TeX} arguments,
%   no \meta{general text} and without additional keywords.
% \item[Expandibility:]
%   The macro |\pdf@|\meta{cmd} is expandable if the
%   corresponding \hologo{pdfTeX} primitive has this property.
%   Exact two expansion steps are necessary (first is the macro
%   expansion) except for \cs{pdf@primitive} and \cs{pdf@ifprimitive}.
%   The latter ones are not macros, but have the direct meaning of the
%   primitive.
% \item[Without \hologo{LuaTeX}:]
%   The macros |\pdf@|\meta{cmd} are mapped to the commands
%   of \hologo{pdfTeX} if they are available. Otherwise they are undefined.
% \item[Availability:]
%   The macros that the packages provides are undefined, if
%   the necessary primitives are not found and cannot be
%   implemented by Lua.
% \end{description}
%
% \subsection{Macros}
%
% \subsubsection[Strings]{Strings \cite[``7.15 Strings'']{pdftex-manual}}
%
% \begin{declcs}{pdf@strcmp} \M{stringA} \M{stringB}
% \end{declcs}
% Same as |\pdfstrcmp{|\meta{stringA}|}{|\meta{stringB}|}|.
%
% \begin{declcs}{pdf@unescapehex} \M{string}
% \end{declcs}
% Same as |\pdfunescapehex{|\meta{string}|}|.
% The argument is a byte string given in hexadecimal notation.
% The result are character tokens from 0 until 255 with
% catcode 12 and the space with catcode 10.
%
% \begin{declcs}{pdf@escapehex} \M{string}\\
%   \cs{pdf@escapestring} \M{string}\\
%   \cs{pdf@escapename} \M{string}
% \end{declcs}
% Same as the primitives of \hologo{pdfTeX}. However \hologo{pdfTeX} does not
% know about characters with codes 256 and larger. Thus the
% string is treated as byte string, characters with more than
% eight bits are ignored.
%
% \subsubsection[Files]{Files \cite[``7.18 Files'']{pdftex-manual}}
%
% \begin{declcs}{pdf@filesize} \M{filename}
% \end{declcs}
% Same as |\pdffilesize{|\meta{filename}|}|.
%
% \begin{declcs}{pdf@filemoddate} \M{filename}
% \end{declcs}
% Same as |\pdffilemoddate{|\meta{filename}|}|.
%
% \begin{declcs}{pdf@filedump} \M{offset} \M{length} \M{filename}
% \end{declcs}
% Same as |\pdffiledump offset| \meta{offset} |length| \meta{length}
% |{|\meta{filename}|}|. Both \meta{offset} and \meta{length} must
% not be empty, but must be a valid \hologo{TeX} number.
%
% \begin{declcs}{pdf@mdfivesum} \M{string}
% \end{declcs}
% Same as |\pdfmdfivesum{|\meta{string}|}|. Keyword |file| is supported
% by macro \cs{pdf@filemdfivesum}.
%
% \begin{declcs}{pdf@filemdfivesum} \M{filename}
% \end{declcs}
% Same as |\pdfmdfivesum file{|\meta{filename}|}|.
%
% \subsubsection[Timekeeping]{Timekeeping \cite[``7.17 Timekeeping'']{pdftex-manual}}
%
% The timekeeping macros are based on Andy Thomas' work \cite{AndyThomas:Analog}.
%
% \begin{declcs}{pdf@resettimer}
% \end{declcs}
% Same as \cs{pdfresettimer}, it resets the internal timer.
%
% \begin{declcs}{pdf@elapsedtime}
% \end{declcs}
% Same as \cs{pdfelapsedtime}. It behaves like a read-only integer.
% For printing purposes it can be prefixed by \cs{the} or \cs{number}.
% It measures the time in scaled seconds (seconds multiplied with 65536)
% since the latest call of \cs{pdf@resettimer} or start of
% program/package. The resolution, the shortest time interval that
% can be measured, depends on the program and system.
% \begin{itemize}
% \item \hologo{pdfTeX} with |gettimeofday|: $\ge$ 1/65536\,s
% \item \hologo{pdfTeX} with |ftime|: $\ge$ 1\,ms
% \item \hologo{pdfTeX} with |time|: $\ge$ 1\,s
% \item \hologo{LuaTeX}: $\ge$ 10\,ms\\
%  (|os.clock()| returns a float number with two decimal digits in
%  \hologo{LuaTeX} beta-0.70.1-2011061416 (rev 4277)).
% \end{itemize}
%
% \subsubsection[Miscellaneous]{Miscellaneous \cite[``7.21 Miscellaneous'']{pdftex-manual}}
%
% \begin{declcs}{pdf@draftmode}
% \end{declcs}
% If the \TeX\ compiler knows \cs{pdfdraftmode} or \cs{draftmode}
% (\hologo{pdfTeX},
% \hologo{LuaTeX}), then \cs{pdf@draftmode} returns, whether
% this mode is enabled. The result is an implicit number:
% one means the draft mode is available and enabled.
% If the value is zero, then the mode is not active or
% \cs{pdfdraftmode} is not available.
% An explicit number is yielded by \cs{number}\cs{pdf@draftmode}.
% The macro cannot
% be used to change the mode, see \cs{pdf@setdraftmode}.
%
% \begin{declcs}{pdf@ifdraftmode} \M{true} \M{false}
% \end{declcs}
% If \cs{pdfdraftmode} is available and enabled, \meta{true} is
% called, otherwise \meta{false} is executed.
%
% \begin{declcs}{pdf@setdraftmode} \M{value}
% \end{declcs}
% Macro \cs{pdf@setdraftmode} expects the number zero or one as
% \meta{value}. Zero deactivates the mode and one enables the draft mode.
% The macro does not have an effect, if the feature \cs{pdfdraftmode} is not
% available.
%
% \begin{declcs}{pdf@shellescape}
% \end{declcs}
% Same as |\pdfshellescape|. It is or expands to |1| if external
% commands can be executed and |0| otherwise. In \hologo{pdfTeX} external
% commands must be enabled first by command line option or
% configuration option. In \hologo{LuaTeX} option |--safer| disables
% the execution of external commands.
%
% In \hologo{LuaTeX} before 0.68.0 \cs{pdf@shellescape} is not
% available due to a bug in |os.execute()|. The argumentless form
% crashes in some circumstances with segmentation fault.
% (It is fixed in version 0.68.0 or revision 4167 of \hologo{LuaTeX}.
% and packported to some version of 0.67.0).
%
% Hints for usage:
% \begin{itemize}
% \item Before its use \cs{pdf@shellescape} should be tested,
% whether it is available. Example with package \xpackage{ltxcmds}
% (loaded by package \xpackage{pdftexcmds}):
%\begin{quote}
%\begin{verbatim}
%\ltx@IfUndefined{pdf@shellescape}{%
%  % \pdf@shellescape is undefined
%}{%
%  % \pdf@shellescape is available
%}
%\end{verbatim}
%\end{quote}
% Use \cs{ltx@ifundefined} in expandable contexts.
% \item \cs{pdf@shellescape} might be a numerical constant,
% expands to the primitive, or expands to a plain number.
% Therefore use it in contexts where these differences does not matter.
% \item Use in comparisons, e.g.:
%   \begin{quote}
%     |\ifnum\pdf@shellescape=0 ...|
%   \end{quote}
% \item Print the number: |\number\pdf@shellescape|
% \end{itemize}
%
% \begin{declcs}{pdf@system} \M{cmdline}
% \end{declcs}
% It is a wrapper for |\immediate\write18| in \hologo{pdfTeX} or
% |os.execute| in \hologo{LuaTeX}.
%
% In theory |os.execute|
% returns a status number. But its meaning is quite
% undefined. Are there some reliable properties?
% Does it make sense to provide an user interface to
% this status exit code?
%
% \begin{declcs}{pdf@primitive} \csi{cmd}
% \end{declcs}
% Same as \cs{pdfprimitive} in \hologo{pdfTeX} or \hologo{LuaTeX}.
% In \hologo{XeTeX} the
% primitive is called \cs{primitive}. Despite the current definition
% of the command \csi{cmd}, it's meaning as primitive is used.
%
% \begin{declcs}{pdf@ifprimitive} \csi{cmd}
% \end{declcs}
% Same as \cs{ifpdfprimitive} in \hologo{pdfTeX} or
% \hologo{LuaTeX}. \hologo{XeTeX} calls
% it \cs{ifprimitive}. It is a switch that checks if the command
% \csi{cmd} has it's primitive meaning.
%
% \subsubsection{Additional macro: \cs{pdf@isprimitive}}
%
% \begin{declcs}{pdf@isprimitive} \csi{cmd1} \csi{cmd2} \M{true} \M{false}
% \end{declcs}
% If \csi{cmd1} has the primitive meaning given by the primitive name
% of \csi{cmd2}, then the argument \meta{true} is executed, otherwise
% \meta{false}. The macro \cs{pdf@isprimitive} is expandable.
% Internally it checks the result of \cs{meaning} and is therefore
% available for all \hologo{TeX} variants, even the original \hologo{TeX}.
% Example with \hologo{LaTeX}:
%\begin{quote}
%\begin{verbatim}
%\makeatletter
%\pdf@isprimitive{@@input}{input}{%
%  \typeout{\string\@@input\space is original\string\input}%
%}{%
%  \typeout{Oops, \string\@@input\space is not the %
%           original\string\input}%
%}
%\end{verbatim}
%\end{quote}
%
% \subsubsection{Experimental}
%
% \begin{declcs}{pdf@unescapehexnative} \M{string}\\
%   \cs{pdf@escapehexnative} \M{string}\\
%   \cs{pdf@escapenamenative} \M{string}\\
%   \cs{pdf@mdfivesumnative} \M{string}
% \end{declcs}
% The variants without |native| in the macro name are supposed to
% be compatible with \hologo{pdfTeX}. However characters with more than
% eight bits are not supported and are ignored. If \hologo{LuaTeX} is
% running, then its UTF-8 coded strings are used. Thus the full
% unicode character range is supported. However the result
% differs from \hologo{pdfTeX} for characters with eight or more bits.
%
% \begin{declcs}{pdf@pipe} \M{cmdline}
% \end{declcs}
% It calls \meta{cmdline} and returns the output of the external
% program in the usual manner as byte string (catcode 12, space with
% catcode 10). The Lua documentation says, that the used |io.popen|
% may not be available on all platforms. Then macro \cs{pdf@pipe}
% is undefined.
%
% \StopEventually{
% }
%
% \section{Implementation}
%
%    \begin{macrocode}
%<*package>
%    \end{macrocode}
%
% \subsection{Reload check and package identification}
%    Reload check, especially if the package is not used with \LaTeX.
%    \begin{macrocode}
\begingroup\catcode61\catcode48\catcode32=10\relax%
  \catcode13=5 % ^^M
  \endlinechar=13 %
  \catcode35=6 % #
  \catcode39=12 % '
  \catcode44=12 % ,
  \catcode45=12 % -
  \catcode46=12 % .
  \catcode58=12 % :
  \catcode64=11 % @
  \catcode123=1 % {
  \catcode125=2 % }
  \expandafter\let\expandafter\x\csname ver@pdftexcmds.sty\endcsname
  \ifx\x\relax % plain-TeX, first loading
  \else
    \def\empty{}%
    \ifx\x\empty % LaTeX, first loading,
      % variable is initialized, but \ProvidesPackage not yet seen
    \else
      \expandafter\ifx\csname PackageInfo\endcsname\relax
        \def\x#1#2{%
          \immediate\write-1{Package #1 Info: #2.}%
        }%
      \else
        \def\x#1#2{\PackageInfo{#1}{#2, stopped}}%
      \fi
      \x{pdftexcmds}{The package is already loaded}%
      \aftergroup\endinput
    \fi
  \fi
\endgroup%
%    \end{macrocode}
%    Package identification:
%    \begin{macrocode}
\begingroup\catcode61\catcode48\catcode32=10\relax%
  \catcode13=5 % ^^M
  \endlinechar=13 %
  \catcode35=6 % #
  \catcode39=12 % '
  \catcode40=12 % (
  \catcode41=12 % )
  \catcode44=12 % ,
  \catcode45=12 % -
  \catcode46=12 % .
  \catcode47=12 % /
  \catcode58=12 % :
  \catcode64=11 % @
  \catcode91=12 % [
  \catcode93=12 % ]
  \catcode123=1 % {
  \catcode125=2 % }
  \expandafter\ifx\csname ProvidesPackage\endcsname\relax
    \def\x#1#2#3[#4]{\endgroup
      \immediate\write-1{Package: #3 #4}%
      \xdef#1{#4}%
    }%
  \else
    \def\x#1#2[#3]{\endgroup
      #2[{#3}]%
      \ifx#1\@undefined
        \xdef#1{#3}%
      \fi
      \ifx#1\relax
        \xdef#1{#3}%
      \fi
    }%
  \fi
\expandafter\x\csname ver@pdftexcmds.sty\endcsname
\ProvidesPackage{pdftexcmds}%
  [2019/07/25 v0.30 Utility functions of pdfTeX for LuaTeX (HO)]%
%    \end{macrocode}
%
% \subsection{Catcodes}
%
%    \begin{macrocode}
\begingroup\catcode61\catcode48\catcode32=10\relax%
  \catcode13=5 % ^^M
  \endlinechar=13 %
  \catcode123=1 % {
  \catcode125=2 % }
  \catcode64=11 % @
  \def\x{\endgroup
    \expandafter\edef\csname pdftexcmds@AtEnd\endcsname{%
      \endlinechar=\the\endlinechar\relax
      \catcode13=\the\catcode13\relax
      \catcode32=\the\catcode32\relax
      \catcode35=\the\catcode35\relax
      \catcode61=\the\catcode61\relax
      \catcode64=\the\catcode64\relax
      \catcode123=\the\catcode123\relax
      \catcode125=\the\catcode125\relax
    }%
  }%
\x\catcode61\catcode48\catcode32=10\relax%
\catcode13=5 % ^^M
\endlinechar=13 %
\catcode35=6 % #
\catcode64=11 % @
\catcode123=1 % {
\catcode125=2 % }
\def\TMP@EnsureCode#1#2{%
  \edef\pdftexcmds@AtEnd{%
    \pdftexcmds@AtEnd
    \catcode#1=\the\catcode#1\relax
  }%
  \catcode#1=#2\relax
}
\TMP@EnsureCode{0}{12}%
\TMP@EnsureCode{1}{12}%
\TMP@EnsureCode{2}{12}%
\TMP@EnsureCode{10}{12}% ^^J
\TMP@EnsureCode{33}{12}% !
\TMP@EnsureCode{34}{12}% "
\TMP@EnsureCode{38}{4}% &
\TMP@EnsureCode{39}{12}% '
\TMP@EnsureCode{40}{12}% (
\TMP@EnsureCode{41}{12}% )
\TMP@EnsureCode{42}{12}% *
\TMP@EnsureCode{43}{12}% +
\TMP@EnsureCode{44}{12}% ,
\TMP@EnsureCode{45}{12}% -
\TMP@EnsureCode{46}{12}% .
\TMP@EnsureCode{47}{12}% /
\TMP@EnsureCode{58}{12}% :
\TMP@EnsureCode{60}{12}% <
\TMP@EnsureCode{62}{12}% >
\TMP@EnsureCode{91}{12}% [
\TMP@EnsureCode{93}{12}% ]
\TMP@EnsureCode{94}{7}% ^ (superscript)
\TMP@EnsureCode{95}{12}% _ (other)
\TMP@EnsureCode{96}{12}% `
\TMP@EnsureCode{126}{12}% ~ (other)
\edef\pdftexcmds@AtEnd{%
  \pdftexcmds@AtEnd
  \escapechar=\number\escapechar\relax
  \noexpand\endinput
}
\escapechar=92 %
%    \end{macrocode}
%
% \subsection{Load packages}
%
%    \begin{macrocode}
\begingroup\expandafter\expandafter\expandafter\endgroup
\expandafter\ifx\csname RequirePackage\endcsname\relax
  \def\TMP@RequirePackage#1[#2]{%
    \begingroup\expandafter\expandafter\expandafter\endgroup
    \expandafter\ifx\csname ver@#1.sty\endcsname\relax
      \input #1.sty\relax
    \fi
  }%
  \TMP@RequirePackage{infwarerr}[2007/09/09]%
  \TMP@RequirePackage{ifluatex}[2010/03/01]%
  \TMP@RequirePackage{ltxcmds}[2010/12/02]%
  \TMP@RequirePackage{ifpdf}[2010/09/13]%
\else
  \RequirePackage{infwarerr}[2007/09/09]%
  \RequirePackage{ifluatex}[2010/03/01]%
  \RequirePackage{ltxcmds}[2010/12/02]%
  \RequirePackage{ifpdf}[2010/09/13]%
\fi
%    \end{macrocode}
%
% \subsection{Without \hologo{LuaTeX}}
%
%    \begin{macrocode}
\ifluatex
\else
  \@PackageInfoNoLine{pdftexcmds}{LuaTeX not detected}%
  \def\pdftexcmds@nopdftex{%
    \@PackageInfoNoLine{pdftexcmds}{pdfTeX >= 1.30 not detected}%
    \let\pdftexcmds@nopdftex\relax
  }%
  \def\pdftexcmds@temp#1{%
    \begingroup\expandafter\expandafter\expandafter\endgroup
    \expandafter\ifx\csname pdf#1\endcsname\relax
      \pdftexcmds@nopdftex
    \else
      \expandafter\def\csname pdf@#1\expandafter\endcsname
      \expandafter##\expandafter{%
        \csname pdf#1\endcsname
      }%
    \fi
  }%
  \pdftexcmds@temp{strcmp}%
  \pdftexcmds@temp{escapehex}%
  \let\pdf@escapehexnative\pdf@escapehex
  \pdftexcmds@temp{unescapehex}%
  \let\pdf@unescapehexnative\pdf@unescapehex
  \pdftexcmds@temp{escapestring}%
  \pdftexcmds@temp{escapename}%
  \pdftexcmds@temp{filesize}%
  \pdftexcmds@temp{filemoddate}%
  \begingroup\expandafter\expandafter\expandafter\endgroup
  \expandafter\ifx\csname pdfshellescape\endcsname\relax
    \pdftexcmds@nopdftex
    \ltx@IfUndefined{pdftexversion}{%
    }{%
      \ifnum\pdftexversion>120 % 1.21a supports \ifeof18
        \ifeof18 %
          \chardef\pdf@shellescape=0 %
        \else
          \chardef\pdf@shellescape=1 %
        \fi
      \fi
    }%
  \else
    \def\pdf@shellescape{%
      \pdfshellescape
    }%
  \fi
  \begingroup\expandafter\expandafter\expandafter\endgroup
  \expandafter\ifx\csname pdffiledump\endcsname\relax
    \pdftexcmds@nopdftex
  \else
    \def\pdf@filedump#1#2#3{%
      \pdffiledump offset#1 length#2{#3}%
    }%
  \fi
%    \end{macrocode}
%    \begin{macrocode}
  \begingroup\expandafter\expandafter\expandafter\endgroup
  \expandafter\ifx\csname pdfmdfivesum\endcsname\relax
    \begingroup\expandafter\expandafter\expandafter\endgroup
    \expandafter\ifx\csname mdfivesum\endcsname\relax
      \pdftexcmds@nopdftex
    \else
      \def\pdf@mdfivesum#{\mdfivesum}%
      \let\pdf@mdfivesumnative\pdf@mdfivesum
      \def\pdf@filemdfivesum#{\mdfivesum file}%
    \fi
  \else
    \def\pdf@mdfivesum#{\pdfmdfivesum}%
    \let\pdf@mdfivesumnative\pdf@mdfivesum
    \def\pdf@filemdfivesum#{\pdfmdfivesum file}%
  \fi
%    \end{macrocode}
%    \begin{macrocode}
  \def\pdf@system#{%
    \immediate\write18%
  }%
  \def\pdftexcmds@temp#1{%
    \begingroup\expandafter\expandafter\expandafter\endgroup
    \expandafter\ifx\csname pdf#1\endcsname\relax
      \pdftexcmds@nopdftex
    \else
      \expandafter\let\csname pdf@#1\expandafter\endcsname
      \csname pdf#1\endcsname
    \fi
  }%
  \pdftexcmds@temp{resettimer}%
  \pdftexcmds@temp{elapsedtime}%
\fi
%    \end{macrocode}
%
% \subsection{\cs{pdf@primitive}, \cs{pdf@ifprimitive}}
%
%    Since version 1.40.0 \hologo{pdfTeX} has \cs{pdfprimitive} and
%    \cs{ifpdfprimitive}. And \cs{pdfprimitive} was fixed in
%    version 1.40.4.
%
%    \hologo{XeTeX} provides them under the name \cs{primitive} and
%    \cs{ifprimitive}. \hologo{LuaTeX} knows both name variants,
%    but they have possibly to be enabled first (|tex.enableprimitives|).
%
%    Depending on the format TeX Live uses a prefix |luatex|.
%
%    Caution: \cs{let} must be used for the definition of
%    the macros, especially because of \cs{ifpdfprimitive}.
%
% \subsubsection{Using \hologo{LuaTeX}'s \texttt{tex.enableprimitives}}
%
%    \begin{macrocode}
\ifluatex
%    \end{macrocode}
%    \begin{macro}{\pdftexcmds@directlua}
%    \begin{macrocode}
  \ifnum\luatexversion<36 %
    \def\pdftexcmds@directlua{\directlua0 }%
  \else
    \let\pdftexcmds@directlua\directlua
  \fi
%    \end{macrocode}
%    \end{macro}
%
%    \begin{macrocode}
  \begingroup
    \newlinechar=10 %
    \endlinechar=\newlinechar
    \pdftexcmds@directlua{%
      if tex.enableprimitives then
        tex.enableprimitives(
          'pdf@',
          {'primitive', 'ifprimitive', 'pdfdraftmode','draftmode'}
        )
        tex.enableprimitives('', {'luaescapestring'})
      end
    }%
  \endgroup %
%    \end{macrocode}
%
%    \begin{macrocode}
\fi
%    \end{macrocode}
%
% \subsubsection{Trying various names to find the primitives}
%
%    \begin{macro}{\pdftexcmds@strip@prefix}
%    \begin{macrocode}
\def\pdftexcmds@strip@prefix#1>{}
%    \end{macrocode}
%    \end{macro}
%    \begin{macrocode}
\def\pdftexcmds@temp#1#2#3{%
  \begingroup\expandafter\expandafter\expandafter\endgroup
  \expandafter\ifx\csname pdf@#1\endcsname\relax
    \begingroup
      \def\x{#3}%
      \edef\x{\expandafter\pdftexcmds@strip@prefix\meaning\x}%
      \escapechar=-1 %
      \edef\y{\expandafter\meaning\csname#2\endcsname}%
    \expandafter\endgroup
    \ifx\x\y
      \expandafter\let\csname pdf@#1\expandafter\endcsname
      \csname #2\endcsname
    \fi
  \fi
}
%    \end{macrocode}
%
%    \begin{macro}{\pdf@primitive}
%    \begin{macrocode}
\pdftexcmds@temp{primitive}{pdfprimitive}{pdfprimitive}% pdfTeX, oldLuaTeX
\pdftexcmds@temp{primitive}{primitive}{primitive}% XeTeX, luatex
\pdftexcmds@temp{primitive}{luatexprimitive}{pdfprimitive}% oldLuaTeX
\pdftexcmds@temp{primitive}{luatexpdfprimitive}{pdfprimitive}% oldLuaTeX
%    \end{macrocode}
%    \end{macro}
%    \begin{macro}{\pdf@ifprimitive}
%    \begin{macrocode}
\pdftexcmds@temp{ifprimitive}{ifpdfprimitive}{ifpdfprimitive}% pdfTeX, oldLuaTeX
\pdftexcmds@temp{ifprimitive}{ifprimitive}{ifprimitive}% XeTeX, luatex
\pdftexcmds@temp{ifprimitive}{luatexifprimitive}{ifpdfprimitive}% oldLuaTeX
\pdftexcmds@temp{ifprimitive}{luatexifpdfprimitive}{ifpdfprimitive}% oldLuaTeX
%    \end{macrocode}
%    \end{macro}
%
%    Disable broken \cs{pdfprimitive}.
%    \begin{macrocode}
\ifluatex\else
\begingroup
  \expandafter\ifx\csname pdf@primitive\endcsname\relax
  \else
    \expandafter\ifx\csname pdftexversion\endcsname\relax
    \else
      \ifnum\pdftexversion=140 %
        \expandafter\ifx\csname pdftexrevision\endcsname\relax
        \else
          \ifnum\pdftexrevision<4 %
            \endgroup
            \let\pdf@primitive\@undefined
            \@PackageInfoNoLine{pdftexcmds}{%
              \string\pdf@primitive\space disabled, %
              because\MessageBreak
              \string\pdfprimitive\space is broken until pdfTeX 1.40.4%
            }%
            \begingroup
          \fi
        \fi
      \fi
    \fi
  \fi
\endgroup
\fi
%    \end{macrocode}
%
% \subsubsection{Result}
%
%    \begin{macrocode}
\begingroup
  \@PackageInfoNoLine{pdftexcmds}{%
    \string\pdf@primitive\space is %
    \expandafter\ifx\csname pdf@primitive\endcsname\relax not \fi
    available%
  }%
  \@PackageInfoNoLine{pdftexcmds}{%
    \string\pdf@ifprimitive\space is %
    \expandafter\ifx\csname pdf@ifprimitive\endcsname\relax not \fi
    available%
  }%
\endgroup
%    \end{macrocode}
%
% \subsection{\hologo{XeTeX}}
%
%    Look for primitives \cs{shellescape}, \cs{strcmp}.
%    \begin{macrocode}
\def\pdftexcmds@temp#1{%
  \begingroup\expandafter\expandafter\expandafter\endgroup
  \expandafter\ifx\csname pdf@#1\endcsname\relax
    \begingroup
      \escapechar=-1 %
      \edef\x{\expandafter\meaning\csname#1\endcsname}%
      \def\y{#1}%
      \def\z##1->{}%
      \edef\y{\expandafter\z\meaning\y}%
    \expandafter\endgroup
    \ifx\x\y
      \expandafter\def\csname pdf@#1\expandafter\endcsname
      \expandafter{%
        \csname#1\endcsname
      }%
    \fi
  \fi
}%
\pdftexcmds@temp{shellescape}%
\pdftexcmds@temp{strcmp}%
%    \end{macrocode}
%
% \subsection{\cs{pdf@isprimitive}}
%
%    \begin{macrocode}
\def\pdf@isprimitive{%
  \begingroup\expandafter\expandafter\expandafter\endgroup
  \expandafter\ifx\csname pdf@strcmp\endcsname\relax
    \long\def\pdf@isprimitive##1{%
      \expandafter\pdftexcmds@isprimitive\expandafter{\meaning##1}%
    }%
    \long\def\pdftexcmds@isprimitive##1##2{%
      \expandafter\pdftexcmds@@isprimitive\expandafter{\string##2}{##1}%
    }%
    \def\pdftexcmds@@isprimitive##1##2{%
      \ifnum0\pdftexcmds@equal##1\delimiter##2\delimiter=1 %
        \expandafter\ltx@firstoftwo
      \else
        \expandafter\ltx@secondoftwo
      \fi
    }%
    \def\pdftexcmds@equal##1##2\delimiter##3##4\delimiter{%
      \ifx##1##3%
        \ifx\relax##2##4\relax
          1%
        \else
          \ifx\relax##2\relax
          \else
            \ifx\relax##4\relax
            \else
              \pdftexcmds@equalcont{##2}{##4}%
            \fi
          \fi
        \fi
      \fi
    }%
    \def\pdftexcmds@equalcont##1{%
      \def\pdftexcmds@equalcont####1####2##1##1##1##1{%
        ##1##1##1##1%
        \pdftexcmds@equal####1\delimiter####2\delimiter
      }%
    }%
    \expandafter\pdftexcmds@equalcont\csname fi\endcsname
  \else
    \long\def\pdf@isprimitive##1##2{%
      \ifnum\pdf@strcmp{\meaning##1}{\string##2}=0 %
        \expandafter\ltx@firstoftwo
      \else
        \expandafter\ltx@secondoftwo
      \fi
    }%
  \fi
}
\ifluatex
\ifx\pdfdraftmode\@undefined
  \let\pdfdraftmode\draftmode
\fi
\else
  \pdf@isprimitive
\fi
%    \end{macrocode}
%
% \subsection{\cs{pdf@draftmode}}
%
%
%    \begin{macrocode}
\let\pdftexcmds@temp\ltx@zero %
\ltx@IfUndefined{pdfdraftmode}{%
  \@PackageInfoNoLine{pdftexcmds}{\ltx@backslashchar pdfdraftmode not found}%
}{%
  \ifpdf
    \let\pdftexcmds@temp\ltx@one
    \@PackageInfoNoLine{pdftexcmds}{\ltx@backslashchar pdfdraftmode found}%
  \else
    \@PackageInfoNoLine{pdftexcmds}{%
      \ltx@backslashchar pdfdraftmode is ignored in DVI mode%
    }%
  \fi
}
\ifcase\pdftexcmds@temp
%    \end{macrocode}
%    \begin{macro}{\pdf@draftmode}
%    \begin{macrocode}
  \let\pdf@draftmode\ltx@zero
%    \end{macrocode}
%    \end{macro}
%    \begin{macro}{\pdf@ifdraftmode}
%    \begin{macrocode}
  \let\pdf@ifdraftmode\ltx@secondoftwo
%    \end{macrocode}
%    \end{macro}
%    \begin{macro}{\pdftexcmds@setdraftmode}
%    \begin{macrocode}
  \def\pdftexcmds@setdraftmode#1{}%
%    \end{macrocode}
%    \end{macro}
%    \begin{macrocode}
\else
%    \end{macrocode}
%    \begin{macro}{\pdftexcmds@draftmode}
%    \begin{macrocode}
  \let\pdftexcmds@draftmode\pdfdraftmode
%    \end{macrocode}
%    \end{macro}
%    \begin{macro}{\pdf@ifdraftmode}
%    \begin{macrocode}
  \def\pdf@ifdraftmode{%
    \ifnum\pdftexcmds@draftmode=\ltx@one
      \expandafter\ltx@firstoftwo
    \else
      \expandafter\ltx@secondoftwo
    \fi
  }%
%    \end{macrocode}
%    \end{macro}
%    \begin{macro}{\pdf@draftmode}
%    \begin{macrocode}
  \def\pdf@draftmode{%
    \ifnum\pdftexcmds@draftmode=\ltx@one
      \expandafter\ltx@one
    \else
      \expandafter\ltx@zero
    \fi
  }%
%    \end{macrocode}
%    \end{macro}
%    \begin{macro}{\pdftexcmds@setdraftmode}
%    \begin{macrocode}
  \def\pdftexcmds@setdraftmode#1{%
    \pdftexcmds@draftmode=#1\relax
  }%
%    \end{macrocode}
%    \end{macro}
%    \begin{macrocode}
\fi
%    \end{macrocode}
%    \begin{macro}{\pdf@setdraftmode}
%    \begin{macrocode}
\def\pdf@setdraftmode#1{%
  \begingroup
    \count\ltx@cclv=#1\relax
  \edef\x{\endgroup
    \noexpand\pdftexcmds@@setdraftmode{\the\count\ltx@cclv}%
  }%
  \x
}
%    \end{macrocode}
%    \end{macro}
%    \begin{macro}{\pdftexcmds@@setdraftmode}
%    \begin{macrocode}
\def\pdftexcmds@@setdraftmode#1{%
  \ifcase#1 %
    \pdftexcmds@setdraftmode{#1}%
  \or
    \pdftexcmds@setdraftmode{#1}%
  \else
    \@PackageWarning{pdftexcmds}{%
      \string\pdf@setdraftmode: Ignoring\MessageBreak
      invalid value `#1'%
    }%
  \fi
}
%    \end{macrocode}
%    \end{macro}
%
% \subsection{Load Lua module}
%
%    \begin{macrocode}
\ifluatex
\else
  \expandafter\pdftexcmds@AtEnd
\fi%
%    \end{macrocode}
%
%    \begin{macrocode}
\ifnum\luatexversion<80
  \begingroup\expandafter\expandafter\expandafter\endgroup
  \expandafter\ifx\csname RequirePackage\endcsname\relax
    \def\TMP@RequirePackage#1[#2]{%
      \begingroup\expandafter\expandafter\expandafter\endgroup
      \expandafter\ifx\csname ver@#1.sty\endcsname\relax
        \input #1.sty\relax
      \fi
    }%
    \TMP@RequirePackage{luatex-loader}[2009/04/10]%
  \else
    \RequirePackage{luatex-loader}[2009/04/10]%
  \fi
\fi
\pdftexcmds@directlua{%
  require("pdftexcmds")%
}
\ifnum\luatexversion>37 %
  \ifnum0%
      \pdftexcmds@directlua{%
        if status.ini_version then %
          tex.write("1")%
        end%
      }>0 %
    \everyjob\expandafter{%
      \the\everyjob
      \pdftexcmds@directlua{%
        require("pdftexcmds")%
      }%
    }%
  \fi
\fi
\begingroup
  \def\x{2019/07/25 v0.30}%
  \ltx@onelevel@sanitize\x
  \edef\y{%
    \pdftexcmds@directlua{%
      if oberdiek.pdftexcmds.getversion then %
        oberdiek.pdftexcmds.getversion()%
      end%
    }%
  }%
  \ifx\x\y
  \else
    \@PackageError{pdftexcmds}{%
      Wrong version of lua module.\MessageBreak
      Package version: \x\MessageBreak
      Lua module: \y
    }\@ehc
  \fi
\endgroup
%    \end{macrocode}
%
% \subsection{Lua functions}
%
% \subsubsection{Helper macros}
%
%    \begin{macro}{\pdftexcmds@toks}
%    \begin{macrocode}
\begingroup\expandafter\expandafter\expandafter\endgroup
\expandafter\ifx\csname newtoks\endcsname\relax
  \toksdef\pdftexcmds@toks=0 %
\else
  \csname newtoks\endcsname\pdftexcmds@toks
\fi
%    \end{macrocode}
%    \end{macro}
%
%    \begin{macro}{\pdftexcmds@Patch}
%    \begin{macrocode}
\def\pdftexcmds@Patch{0}
\ifnum\luatexversion>40 %
  \ifnum\luatexversion<66 %
    \def\pdftexcmds@Patch{1}%
  \fi
\fi
%    \end{macrocode}
%    \end{macro}
%    \begin{macrocode}
\ifcase\pdftexcmds@Patch
  \catcode`\&=14 %
\else
  \catcode`\&=9 %
%    \end{macrocode}
%    \begin{macro}{\pdftexcmds@PatchDecode}
%    \begin{macrocode}
  \def\pdftexcmds@PatchDecode#1\@nil{%
    \pdftexcmds@DecodeA#1^^A^^A\@nil{}%
  }%
%    \end{macrocode}
%    \end{macro}
%    \begin{macro}{\pdftexcmds@DecodeA}
%    \begin{macrocode}
  \def\pdftexcmds@DecodeA#1^^A^^A#2\@nil#3{%
    \ifx\relax#2\relax
      \ltx@ReturnAfterElseFi{%
        \pdftexcmds@DecodeB#3#1^^A^^B\@nil{}%
      }%
    \else
      \ltx@ReturnAfterFi{%
        \pdftexcmds@DecodeA#2\@nil{#3#1^^@}%
      }%
    \fi
  }%
%    \end{macrocode}
%    \end{macro}
%    \begin{macro}{\pdftexcmds@DecodeB}
%    \begin{macrocode}
  \def\pdftexcmds@DecodeB#1^^A^^B#2\@nil#3{%
    \ifx\relax#2\relax%
      \ltx@ReturnAfterElseFi{%
        \ltx@zero
        #3#1%
      }%
    \else
      \ltx@ReturnAfterFi{%
        \pdftexcmds@DecodeB#2\@nil{#3#1^^A}%
      }%
    \fi
  }%
%    \end{macrocode}
%    \end{macro}
%    \begin{macrocode}
\fi
%    \end{macrocode}
%
%    \begin{macrocode}
\ifnum\luatexversion<36 %
\else
  \catcode`\0=9 %
\fi
%    \end{macrocode}
%
% \subsubsection[Strings]{Strings \cite[``7.15 Strings'']{pdftex-manual}}
%
%    \begin{macro}{\pdf@strcmp}
%    \begin{macrocode}
\long\def\pdf@strcmp#1#2{%
  \directlua0{%
    oberdiek.pdftexcmds.strcmp("\luaescapestring{#1}",%
        "\luaescapestring{#2}")%
  }%
}%
%    \end{macrocode}
%    \end{macro}
%    \begin{macrocode}
\pdf@isprimitive
%    \end{macrocode}
%    \begin{macro}{\pdf@escapehex}
%    \begin{macrocode}
\long\def\pdf@escapehex#1{%
  \directlua0{%
    oberdiek.pdftexcmds.escapehex("\luaescapestring{#1}", "byte")%
  }%
}%
%    \end{macrocode}
%    \end{macro}
%    \begin{macro}{\pdf@escapehexnative}
%    \begin{macrocode}
\long\def\pdf@escapehexnative#1{%
  \directlua0{%
    oberdiek.pdftexcmds.escapehex("\luaescapestring{#1}")%
  }%
}%
%    \end{macrocode}
%    \end{macro}
%    \begin{macro}{\pdf@unescapehex}
%    \begin{macrocode}
\def\pdf@unescapehex#1{%
& \romannumeral\expandafter\pdftexcmds@PatchDecode
  \the\expandafter\pdftexcmds@toks
  \directlua0{%
    oberdiek.pdftexcmds.toks="pdftexcmds@toks"%
    oberdiek.pdftexcmds.unescapehex("\luaescapestring{#1}", "byte", \pdftexcmds@Patch)%
  }%
& \@nil
}%
%    \end{macrocode}
%    \end{macro}
%    \begin{macro}{\pdf@unescapehexnative}
%    \begin{macrocode}
\def\pdf@unescapehexnative#1{%
& \romannumeral\expandafter\pdftexcmds@PatchDecode
  \the\expandafter\pdftexcmds@toks
  \directlua0{%
    oberdiek.pdftexcmds.toks="pdftexcmds@toks"%
    oberdiek.pdftexcmds.unescapehex("\luaescapestring{#1}", \pdftexcmds@Patch)%
  }%
& \@nil
}%
%    \end{macrocode}
%    \end{macro}
%    \begin{macro}{\pdf@escapestring}
%    \begin{macrocode}
\long\def\pdf@escapestring#1{%
  \directlua0{%
    oberdiek.pdftexcmds.escapestring("\luaescapestring{#1}", "byte")%
  }%
}
%    \end{macrocode}
%    \end{macro}
%    \begin{macro}{\pdf@escapename}
%    \begin{macrocode}
\long\def\pdf@escapename#1{%
  \directlua0{%
    oberdiek.pdftexcmds.escapename("\luaescapestring{#1}", "byte")%
  }%
}
%    \end{macrocode}
%    \end{macro}
%    \begin{macro}{\pdf@escapenamenative}
%    \begin{macrocode}
\long\def\pdf@escapenamenative#1{%
  \directlua0{%
    oberdiek.pdftexcmds.escapename("\luaescapestring{#1}")%
  }%
}
%    \end{macrocode}
%    \end{macro}
%
% \subsubsection[Files]{Files \cite[``7.18 Files'']{pdftex-manual}}
%
%    \begin{macro}{\pdf@filesize}
%    \begin{macrocode}
\def\pdf@filesize#1{%
  \directlua0{%
    oberdiek.pdftexcmds.filesize("\luaescapestring{#1}")%
  }%
}
%    \end{macrocode}
%    \end{macro}
%    \begin{macro}{\pdf@filemoddate}
%    \begin{macrocode}
\def\pdf@filemoddate#1{%
  \directlua0{%
    oberdiek.pdftexcmds.filemoddate("\luaescapestring{#1}")%
  }%
}
%    \end{macrocode}
%    \end{macro}
%    \begin{macro}{\pdf@filedump}
%    \begin{macrocode}
\def\pdf@filedump#1#2#3{%
  \directlua0{%
    oberdiek.pdftexcmds.filedump("\luaescapestring{\number#1}",%
        "\luaescapestring{\number#2}",%
        "\luaescapestring{#3}")%
  }%
}%
%    \end{macrocode}
%    \end{macro}
%    \begin{macro}{\pdf@mdfivesum}
%    \begin{macrocode}
\long\def\pdf@mdfivesum#1{%
  \directlua0{%
    oberdiek.pdftexcmds.mdfivesum("\luaescapestring{#1}", "byte")%
  }%
}%
%    \end{macrocode}
%    \end{macro}
%    \begin{macro}{\pdf@mdfivesumnative}
%    \begin{macrocode}
\long\def\pdf@mdfivesumnative#1{%
  \directlua0{%
    oberdiek.pdftexcmds.mdfivesum("\luaescapestring{#1}")%
  }%
}%
%    \end{macrocode}
%    \end{macro}
%    \begin{macro}{\pdf@filemdfivesum}
%    \begin{macrocode}
\def\pdf@filemdfivesum#1{%
  \directlua0{%
    oberdiek.pdftexcmds.filemdfivesum("\luaescapestring{#1}")%
  }%
}%
%    \end{macrocode}
%    \end{macro}
%
% \subsubsection[Timekeeping]{Timekeeping \cite[``7.17 Timekeeping'']{pdftex-manual}}
%
%    \begin{macro}{\protected}
%    \begin{macrocode}
\let\pdftexcmds@temp=Y%
\begingroup\expandafter\expandafter\expandafter\endgroup
\expandafter\ifx\csname protected\endcsname\relax
  \pdftexcmds@directlua0{%
    if tex.enableprimitives then %
      tex.enableprimitives('', {'protected'})%
    end%
  }%
\fi
\begingroup\expandafter\expandafter\expandafter\endgroup
\expandafter\ifx\csname protected\endcsname\relax
  \let\pdftexcmds@temp=N%
\fi
%    \end{macrocode}
%    \end{macro}
%    \begin{macro}{\numexpr}
%    \begin{macrocode}
\begingroup\expandafter\expandafter\expandafter\endgroup
\expandafter\ifx\csname numexpr\endcsname\relax
  \pdftexcmds@directlua0{%
    if tex.enableprimitives then %
      tex.enableprimitives('', {'numexpr'})%
    end%
  }%
\fi
\begingroup\expandafter\expandafter\expandafter\endgroup
\expandafter\ifx\csname numexpr\endcsname\relax
  \let\pdftexcmds@temp=N%
\fi
%    \end{macrocode}
%    \end{macro}
%
%    \begin{macrocode}
\ifx\pdftexcmds@temp N%
  \@PackageWarningNoLine{pdftexcmds}{%
    Definitions of \ltx@backslashchar pdf@resettimer and%
    \MessageBreak
    \ltx@backslashchar pdf@elapsedtime are skipped, because%
    \MessageBreak
    e-TeX's \ltx@backslashchar protected or %
    \ltx@backslashchar numexpr are missing%
  }%
\else
%    \end{macrocode}
%
%    \begin{macro}{\pdf@resettimer}
%    \begin{macrocode}
  \protected\def\pdf@resettimer{%
    \pdftexcmds@directlua0{%
      oberdiek.pdftexcmds.resettimer()%
    }%
  }%
%    \end{macrocode}
%    \end{macro}
%
%    \begin{macro}{\pdf@elapsedtime}
%    \begin{macrocode}
  \protected\def\pdf@elapsedtime{%
    \numexpr
      \pdftexcmds@directlua0{%
        oberdiek.pdftexcmds.elapsedtime()%
      }%
    \relax
  }%
%    \end{macrocode}
%    \end{macro}
%    \begin{macrocode}
\fi
%    \end{macrocode}
%
% \subsubsection{Shell escape}
%
%    \begin{macro}{\pdf@shellescape}
%
%    \begin{macrocode}
\ifnum\luatexversion<68 %
\else
  \protected\edef\pdf@shellescape{%
   \numexpr\directlua{tex.sprint(%
         \number\catcodetable@string,status.shell_escape)}\relax}
\fi
%    \end{macrocode}
%    \end{macro}
%
%    \begin{macro}{\pdf@system}
%    \begin{macrocode}
\def\pdf@system#1{%
  \directlua0{%
    oberdiek.pdftexcmds.system("\luaescapestring{#1}")%
  }%
}
%    \end{macrocode}
%    \end{macro}
%
%    \begin{macro}{\pdf@lastsystemstatus}
%    \begin{macrocode}
\def\pdf@lastsystemstatus{%
  \directlua0{%
    oberdiek.pdftexcmds.lastsystemstatus()%
  }%
}
%    \end{macrocode}
%    \end{macro}
%    \begin{macro}{\pdf@lastsystemexit}
%    \begin{macrocode}
\def\pdf@lastsystemexit{%
  \directlua0{%
    oberdiek.pdftexcmds.lastsystemexit()%
  }%
}
%    \end{macrocode}
%    \end{macro}
%
%    \begin{macrocode}
\catcode`\0=12 %
%    \end{macrocode}
%
%    \begin{macro}{\pdf@pipe}
%    Check availability of |io.popen| first.
%    \begin{macrocode}
\ifnum0%
    \pdftexcmds@directlua{%
      if io.popen then %
        tex.write("1")%
      end%
    }%
    =1 %
  \def\pdf@pipe#1{%
&   \romannumeral\expandafter\pdftexcmds@PatchDecode
    \the\expandafter\pdftexcmds@toks
    \pdftexcmds@directlua{%
      oberdiek.pdftexcmds.toks="pdftexcmds@toks"%
      oberdiek.pdftexcmds.pipe("\luaescapestring{#1}", \pdftexcmds@Patch)%
    }%
&   \@nil
  }%
\fi
%    \end{macrocode}
%    \end{macro}
%
%    \begin{macrocode}
\pdftexcmds@AtEnd%
%</package>
%    \end{macrocode}
%
% \subsection{Lua module}
%
%    \begin{macrocode}
%<*lua>
%    \end{macrocode}
%
%    \begin{macrocode}
oberdiek = oberdiek or {}
local pdftexcmds = oberdiek.pdftexcmds or {}
oberdiek.pdftexcmds = pdftexcmds
local systemexitstatus
function pdftexcmds.getversion()
  tex.write("2019/07/25 v0.30")
end
%    \end{macrocode}
%
% \subsubsection[Strings]{Strings \cite[``7.15 Strings'']{pdftex-manual}}
%
%    \begin{macrocode}
function pdftexcmds.strcmp(A, B)
  if A == B then
    tex.write("0")
  elseif A < B then
    tex.write("-1")
  else
    tex.write("1")
  end
end
local function utf8_to_byte(str)
  local i = 0
  local n = string.len(str)
  local t = {}
  while i < n do
    i = i + 1
    local a = string.byte(str, i)
    if a < 128 then
      table.insert(t, string.char(a))
    else
      if a >= 192 and i < n then
        i = i + 1
        local b = string.byte(str, i)
        if b < 128 or b >= 192 then
          i = i - 1
        elseif a == 194 then
          table.insert(t, string.char(b))
        elseif a == 195 then
          table.insert(t, string.char(b + 64))
        end
      end
    end
  end
  return table.concat(t)
end
function pdftexcmds.escapehex(str, mode)
  if mode == "byte" then
    str = utf8_to_byte(str)
  end
  tex.write((string.gsub(str, ".",
    function (ch)
      return string.format("%02X", string.byte(ch))
    end
  )))
end
%    \end{macrocode}
%    See procedure |unescapehex| in file \xfile{utils.c} of \hologo{pdfTeX}.
%    Caution: |tex.write| ignores leading spaces.
%    \begin{macrocode}
function pdftexcmds.unescapehex(str, mode, patch)
  local a = 0
  local first = true
  local result = {}
  for i = 1, string.len(str), 1 do
    local ch = string.byte(str, i)
    if ch >= 48 and ch <= 57 then
      ch = ch - 48
    elseif ch >= 65 and ch <= 70 then
      ch = ch - 55
    elseif ch >= 97 and ch <= 102 then
      ch = ch - 87
    else
      ch = nil
    end
    if ch then
      if first then
        a = ch * 16
        first = false
      else
        table.insert(result, a + ch)
        first = true
      end
    end
  end
  if not first then
    table.insert(result, a)
  end
  if patch == 1 then
    local temp = {}
    for i, a in ipairs(result) do
      if a == 0 then
        table.insert(temp, 1)
        table.insert(temp, 1)
      else
        if a == 1 then
          table.insert(temp, 1)
          table.insert(temp, 2)
        else
          table.insert(temp, a)
        end
      end
    end
    result = temp
  end
  if mode == "byte" then
    local utf8 = {}
    for i, a in ipairs(result) do
      if a < 128 then
        table.insert(utf8, a)
      else
        if a < 192 then
          table.insert(utf8, 194)
          a = a - 128
        else
          table.insert(utf8, 195)
          a = a - 192
        end
        table.insert(utf8, a + 128)
      end
    end
    result = utf8
  end
%    \end{macrocode}
%    this next line added for current luatex; this is the only
%    change in the file.  eroux, 28apr13. (v 0.21)
%    \begin{macrocode}
  local unpack = _G["unpack"] or table.unpack
  tex.settoks(pdftexcmds.toks, string.char(unpack(result)))
end
%    \end{macrocode}
%    See procedure |escapestring| in file \xfile{utils.c} of \hologo{pdfTeX}.
%    \begin{macrocode}
function pdftexcmds.escapestring(str, mode)
  if mode == "byte" then
    str = utf8_to_byte(str)
  end
  tex.write((string.gsub(str, ".",
    function (ch)
      local b = string.byte(ch)
      if b < 33 or b > 126 then
        return string.format("\\%.3o", b)
      end
      if b == 40 or b == 41 or b == 92 then
        return "\\" .. ch
      end
%    \end{macrocode}
%    Lua 5.1 returns the match in case of return value |nil|.
%    \begin{macrocode}
      return nil
    end
  )))
end
%    \end{macrocode}
%    See procedure |escapename| in file \xfile{utils.c} of \hologo{pdfTeX}.
%    \begin{macrocode}
function pdftexcmds.escapename(str, mode)
  if mode == "byte" then
    str = utf8_to_byte(str)
  end
  tex.write((string.gsub(str, ".",
    function (ch)
      local b = string.byte(ch)
      if b == 0 then
%    \end{macrocode}
%    In Lua 5.0 |nil| could be used for the empty string,
%    But |nil| returns the match in Lua 5.1, thus we use
%    the empty string explicitly.
%    \begin{macrocode}
        return ""
      end
      if b <= 32 or b >= 127
          or b == 35 or b == 37 or b == 40 or b == 41
          or b == 47 or b == 60 or b == 62 or b == 91
          or b == 93 or b == 123 or b == 125 then
        return string.format("#%.2X", b)
      else
%    \end{macrocode}
%    Lua 5.1 returns the match in case of return value |nil|.
%    \begin{macrocode}
        return nil
      end
    end
  )))
end
%    \end{macrocode}
%
% \subsubsection[Files]{Files \cite[``7.18 Files'']{pdftex-manual}}
%
%    \begin{macrocode}
function pdftexcmds.filesize(filename)
  local foundfile = kpse.find_file(filename, "tex", true)
  if foundfile then
    local size = lfs.attributes(foundfile, "size")
    if size then
      tex.write(size)
    end
  end
end
%    \end{macrocode}
%    See procedure |makepdftime| in file \xfile{utils.c} of \hologo{pdfTeX}.
%    \begin{macrocode}
function pdftexcmds.filemoddate(filename)
  local foundfile = kpse.find_file(filename, "tex", true)
  if foundfile then
    local date = lfs.attributes(foundfile, "modification")
    if date then
      local d = os.date("*t", date)
      if d.sec >= 60 then
        d.sec = 59
      end
      local u = os.date("!*t", date)
      local off = 60 * (d.hour - u.hour) + d.min - u.min
      if d.year ~= u.year then
        if d.year > u.year then
          off = off + 1440
        else
          off = off - 1440
        end
      elseif d.yday ~= u.yday then
        if d.yday > u.yday then
          off = off + 1440
        else
          off = off - 1440
        end
      end
      local timezone
      if off == 0 then
        timezone = "Z"
      else
        local hours = math.floor(off / 60)
        local mins = math.abs(off - hours * 60)
        timezone = string.format("%+03d'%02d'", hours, mins)
      end
      tex.write(string.format("D:%04d%02d%02d%02d%02d%02d%s",
          d.year, d.month, d.day, d.hour, d.min, d.sec, timezone))
    end
  end
end
function pdftexcmds.filedump(offset, length, filename)
  length = tonumber(length)
  if length and length > 0 then
    local foundfile = kpse.find_file(filename, "tex", true)
    if foundfile then
      offset = tonumber(offset)
      if not offset then
        offset = 0
      end
      local filehandle = io.open(foundfile, "rb")
      if filehandle then
        if offset > 0 then
          filehandle:seek("set", offset)
        end
        local dump = filehandle:read(length)
        pdftexcmds.escapehex(dump)
        filehandle:close()
      end
    end
  end
end
function pdftexcmds.mdfivesum(str, mode)
  if mode == "byte" then
    str = utf8_to_byte(str)
  end
  pdftexcmds.escapehex(md5.sum(str))
end
function pdftexcmds.filemdfivesum(filename)
  local foundfile = kpse.find_file(filename, "tex", true)
  if foundfile then
    local filehandle = io.open(foundfile, "rb")
    if filehandle then
      local contents = filehandle:read("*a")
      pdftexcmds.escapehex(md5.sum(contents))
      filehandle:close()
    end
  end
end
%    \end{macrocode}
%
% \subsubsection[Timekeeping]{Timekeeping \cite[``7.17 Timekeeping'']{pdftex-manual}}
%
%    The functions for timekeeping are based on
%    Andy Thomas' work \cite{AndyThomas:Analog}.
%    Changes:
%    \begin{itemize}
%    \item Overflow check is added.
%    \item |string.format| is used to avoid exponential number
%          representation for sure.
%    \item |tex.write| is used instead of |tex.print| to get
%          tokens with catcode 12 and without appended \cs{endlinechar}.
%    \end{itemize}
%    \begin{macrocode}
local basetime = 0
function pdftexcmds.resettimer()
  basetime = os.clock()
end
function pdftexcmds.elapsedtime()
  local val = (os.clock() - basetime) * 65536 + .5
  if val > 2147483647 then
    val = 2147483647
  end
  tex.write(string.format("%d", val))
end
%    \end{macrocode}
%
% \subsubsection[Miscellaneous]{Miscellaneous \cite[``7.21 Miscellaneous'']{pdftex-manual}}
%
%    \begin{macrocode}
function pdftexcmds.shellescape()
  if os.execute then
    if status
        and status.luatex_version
        and status.luatex_version >= 68 then
      tex.write(os.execute())
    else
      local result = os.execute()
      if result == 0 then
        tex.write("0")
      else
        if result == nil then
          tex.write("0")
        else
          tex.write("1")
        end
      end
    end
  else
    tex.write("0")
  end
end
function pdftexcmds.system(cmdline)
  systemexitstatus = nil
  texio.write_nl("log", "system(" .. cmdline .. ") ")
  if os.execute then
    texio.write("log", "executed.")
    systemexitstatus = os.execute(cmdline)
  else
    texio.write("log", "disabled.")
  end
end
function pdftexcmds.lastsystemstatus()
  local result = tonumber(systemexitstatus)
  if result then
    local x = math.floor(result / 256)
    tex.write(result - 256 * math.floor(result / 256))
  end
end
function pdftexcmds.lastsystemexit()
  local result = tonumber(systemexitstatus)
  if result then
    tex.write(math.floor(result / 256))
  end
end
function pdftexcmds.pipe(cmdline, patch)
  local result
  systemexitstatus = nil
  texio.write_nl("log", "pipe(" .. cmdline ..") ")
  if io.popen then
    texio.write("log", "executed.")
    local handle = io.popen(cmdline, "r")
    if handle then
      result = handle:read("*a")
      handle:close()
    end
  else
    texio.write("log", "disabled.")
  end
  if result then
    if patch == 1 then
      local temp = {}
      for i, a in ipairs(result) do
        if a == 0 then
          table.insert(temp, 1)
          table.insert(temp, 1)
        else
          if a == 1 then
            table.insert(temp, 1)
            table.insert(temp, 2)
          else
            table.insert(temp, a)
          end
        end
      end
      result = temp
    end
    tex.settoks(pdftexcmds.toks, result)
  else
    tex.settoks(pdftexcmds.toks, "")
  end
end
%    \end{macrocode}
%    \begin{macrocode}
%</lua>
%    \end{macrocode}
%
% \section{Test}
%
% \subsection{Catcode checks for loading}
%
%    \begin{macrocode}
%<*test1>
%    \end{macrocode}
%    \begin{macrocode}
\catcode`\{=1 %
\catcode`\}=2 %
\catcode`\#=6 %
\catcode`\@=11 %
\expandafter\ifx\csname count@\endcsname\relax
  \countdef\count@=255 %
\fi
\expandafter\ifx\csname @gobble\endcsname\relax
  \long\def\@gobble#1{}%
\fi
\expandafter\ifx\csname @firstofone\endcsname\relax
  \long\def\@firstofone#1{#1}%
\fi
\expandafter\ifx\csname loop\endcsname\relax
  \expandafter\@firstofone
\else
  \expandafter\@gobble
\fi
{%
  \def\loop#1\repeat{%
    \def\body{#1}%
    \iterate
  }%
  \def\iterate{%
    \body
      \let\next\iterate
    \else
      \let\next\relax
    \fi
    \next
  }%
  \let\repeat=\fi
}%
\def\RestoreCatcodes{}
\count@=0 %
\loop
  \edef\RestoreCatcodes{%
    \RestoreCatcodes
    \catcode\the\count@=\the\catcode\count@\relax
  }%
\ifnum\count@<255 %
  \advance\count@ 1 %
\repeat

\def\RangeCatcodeInvalid#1#2{%
  \count@=#1\relax
  \loop
    \catcode\count@=15 %
  \ifnum\count@<#2\relax
    \advance\count@ 1 %
  \repeat
}
\def\RangeCatcodeCheck#1#2#3{%
  \count@=#1\relax
  \loop
    \ifnum#3=\catcode\count@
    \else
      \errmessage{%
        Character \the\count@\space
        with wrong catcode \the\catcode\count@\space
        instead of \number#3%
      }%
    \fi
  \ifnum\count@<#2\relax
    \advance\count@ 1 %
  \repeat
}
\def\space{ }
\expandafter\ifx\csname LoadCommand\endcsname\relax
  \def\LoadCommand{\input pdftexcmds.sty\relax}%
\fi
\def\Test{%
  \RangeCatcodeInvalid{0}{47}%
  \RangeCatcodeInvalid{58}{64}%
  \RangeCatcodeInvalid{91}{96}%
  \RangeCatcodeInvalid{123}{255}%
  \catcode`\@=12 %
  \catcode`\\=0 %
  \catcode`\%=14 %
  \LoadCommand
  \RangeCatcodeCheck{0}{36}{15}%
  \RangeCatcodeCheck{37}{37}{14}%
  \RangeCatcodeCheck{38}{47}{15}%
  \RangeCatcodeCheck{48}{57}{12}%
  \RangeCatcodeCheck{58}{63}{15}%
  \RangeCatcodeCheck{64}{64}{12}%
  \RangeCatcodeCheck{65}{90}{11}%
  \RangeCatcodeCheck{91}{91}{15}%
  \RangeCatcodeCheck{92}{92}{0}%
  \RangeCatcodeCheck{93}{96}{15}%
  \RangeCatcodeCheck{97}{122}{11}%
  \RangeCatcodeCheck{123}{255}{15}%
  \RestoreCatcodes
}
\Test
\csname @@end\endcsname
\end
%    \end{macrocode}
%    \begin{macrocode}
%</test1>
%    \end{macrocode}
%
% \subsection{Test for \cs{pdf@isprimitive}}
%
%    \begin{macrocode}
%<*test2>
\catcode`\{=1 %
\catcode`\}=2 %
\catcode`\#=6 %
\catcode`\@=11 %
\input pdftexcmds.sty\relax
\def\msg#1{%
  \begingroup
    \escapechar=92 %
    \immediate\write16{#1}%
  \endgroup
}
\long\def\test#1#2#3#4{%
  \begingroup
    #4%
    \def\str{%
      Test \string\pdf@isprimitive
      {\string #1}{\string #2}{...}: %
    }%
    \pdf@isprimitive{#1}{#2}{%
      \ifx#3Y%
        \msg{\str true ==> OK.}%
      \else
        \errmessage{\str false ==> FAILED}%
      \fi
    }{%
      \ifx#3Y%
        \errmessage{\str true ==> FAILED}%
      \else
        \msg{\str false ==> OK.}%
      \fi
    }%
  \endgroup
}
\test\relax\relax Y{}
\test\foobar\relax Y{\let\foobar\relax}
\test\foobar\relax N{}
\test\hbox\hbox Y{}
\test\foobar@hbox\hbox Y{\let\foobar@hbox\hbox}
\test\if\if Y{}
\test\if\ifx N{}
\test\ifx\if N{}
\test\par\par Y{}
\test\hbox\par N{}
\test\par\hbox N{}
\csname @@end\endcsname\end
%</test2>
%    \end{macrocode}
%
% \subsection{Test for \cs{pdf@shellescape}}
%
%    \begin{macrocode}
%<*test-shell>
\catcode`\{=1 %
\catcode`\}=2 %
\catcode`\#=6 %
\catcode`\@=11 %
\input pdftexcmds.sty\relax
\def\msg#{\immediate\write16}
\def\MaybeEnd{}
\ifx\luatexversion\UnDeFiNeD
\else
  \ifnum\luatexversion<68 %
    \ifx\pdf@shellescape\@undefined
      \msg{SHELL=U}%
      \msg{OK (LuaTeX < 0.68)}%
    \else
      \msg{SHELL=defined}%
      \errmessage{Failed (LuaTeX < 0.68)}%
    \fi
    \def\MaybeEnd{\csname @@end\endcsname\end}%
  \fi
\fi
\MaybeEnd
\ifx\pdf@shellescape\@undefined
  \msg{SHELL=U}%
\else
  \msg{SHELL=\number\pdf@shellescape}%
\fi
\ifx\expected\@undefined
\else
  \ifx\expected\relax
    \msg{EXPECTED=U}%
    \ifx\pdf@shellescape\@undefined
      \msg{OK}%
    \else
      \errmessage{Failed}%
    \fi
  \else
    \msg{EXPECTED=\number\expected}%
    \ifnum\pdf@shellescape=\expected\relax
      \msg{OK}%
    \else
      \errmessage{Failed}%
    \fi
  \fi
\fi
\csname @@end\endcsname\end
%</test-shell>
%    \end{macrocode}
%
% \subsection{Test for escape functions}
%
%    \begin{macrocode}
%<*test-escape>
\catcode`\{=1 %
\catcode`\}=2 %
\catcode`\#=6 %
\catcode`\^=7 %
\catcode`\@=11 %
\errorcontextlines=1000 %
\input pdftexcmds.sty\relax
\def\msg#1{%
  \begingroup
    \escapechar=92 %
    \immediate\write16{#1}%
  \endgroup
}
%    \end{macrocode}
%    \begin{macrocode}
\begingroup
  \catcode`\@=11 %
  \countdef\count@=255 %
  \def\space{ }%
  \long\def\@whilenum#1\do #2{%
    \ifnum #1\relax
      #2\relax
      \@iwhilenum{#1\relax#2\relax}%
    \fi
  }%
  \long\def\@iwhilenum#1{%
    \ifnum #1%
      \expandafter\@iwhilenum
    \else
      \expandafter\ltx@gobble
    \fi
    {#1}%
  }%
  \gdef\AllBytes{}%
  \count@=0 %
  \catcode0=12 %
  \@whilenum\count@<256 \do{%
    \lccode0=\count@
    \ifnum\count@=32 %
      \xdef\AllBytes{\AllBytes\space}%
    \else
      \lowercase{%
        \xdef\AllBytes{\AllBytes^^@}%
      }%
    \fi
    \advance\count@ by 1 %
  }%
\endgroup
%    \end{macrocode}
%    \begin{macrocode}
\def\AllBytesHex{%
  000102030405060708090A0B0C0D0E0F%
  101112131415161718191A1B1C1D1E1F%
  202122232425262728292A2B2C2D2E2F%
  303132333435363738393A3B3C3D3E3F%
  404142434445464748494A4B4C4D4E4F%
  505152535455565758595A5B5C5D5E5F%
  606162636465666768696A6B6C6D6E6F%
  707172737475767778797A7B7C7D7E7F%
  808182838485868788898A8B8C8D8E8F%
  909192939495969798999A9B9C9D9E9F%
  A0A1A2A3A4A5A6A7A8A9AAABACADAEAF%
  B0B1B2B3B4B5B6B7B8B9BABBBCBDBEBF%
  C0C1C2C3C4C5C6C7C8C9CACBCCCDCECF%
  D0D1D2D3D4D5D6D7D8D9DADBDCDDDEDF%
  E0E1E2E3E4E5E6E7E8E9EAEBECEDEEEF%
  F0F1F2F3F4F5F6F7F8F9FAFBFCFDFEFF%
}
\ltx@onelevel@sanitize\AllBytesHex
\expandafter\lowercase\expandafter{%
  \expandafter\def\expandafter\AllBytesHexLC
      \expandafter{\AllBytesHex}%
}
\begingroup
  \catcode`\#=12 %
  \xdef\AllBytesName{%
    #01#02#03#04#05#06#07#08#09#0A#0B#0C#0D#0E#0F%
    #10#11#12#13#14#15#16#17#18#19#1A#1B#1C#1D#1E#1F%
    #20!"#23$#25&'#28#29*+,-.#2F%
    0123456789:;#3C=#3E?%
    @ABCDEFGHIJKLMNO%
    PQRSTUVWXYZ#5B\ltx@backslashchar#5D^_%
    `abcdefghijklmno%
    pqrstuvwxyz#7B|#7D\string~#7F%
    #80#81#82#83#84#85#86#87#88#89#8A#8B#8C#8D#8E#8F%
    #90#91#92#93#94#95#96#97#98#99#9A#9B#9C#9D#9E#9F%
    #A0#A1#A2#A3#A4#A5#A6#A7#A8#A9#AA#AB#AC#AD#AE#AF%
    #B0#B1#B2#B3#B4#B5#B6#B7#B8#B9#BA#BB#BC#BD#BE#BF%
    #C0#C1#C2#C3#C4#C5#C6#C7#C8#C9#CA#CB#CC#CD#CE#CF%
    #D0#D1#D2#D3#D4#D5#D6#D7#D8#D9#DA#DB#DC#DD#DE#DF%
    #E0#E1#E2#E3#E4#E5#E6#E7#E8#E9#EA#EB#EC#ED#EE#EF%
    #F0#F1#F2#F3#F4#F5#F6#F7#F8#F9#FA#FB#FC#FD#FE#FF%
  }%
\endgroup
\ltx@onelevel@sanitize\AllBytesName
\edef\AllBytesFromName{\expandafter\ltx@gobble\AllBytes}
\begingroup
  \def\|{|}%
  \edef\%{\ltx@percentchar}%
  \catcode`\|=0 %
  \catcode`\#=12 %
  \catcode`\~=12 %
  \catcode`\\=12 %
  |xdef|AllBytesString{%
    \000\001\002\003\004\005\006\007\010\011\012\013\014\015\016\017%
    \020\021\022\023\024\025\026\027\030\031\032\033\034\035\036\037%
    \040!"#$|%&'\(\)*+,-./%
    0123456789:;<=>?%
    @ABCDEFGHIJKLMNO%
    PQRSTUVWXYZ[\\]^_%
    `abcdefghijklmno%
    pqrstuvwxyz{||}~\177%
    \200\201\202\203\204\205\206\207\210\211\212\213\214\215\216\217%
    \220\221\222\223\224\225\226\227\230\231\232\233\234\235\236\237%
    \240\241\242\243\244\245\246\247\250\251\252\253\254\255\256\257%
    \260\261\262\263\264\265\266\267\270\271\272\273\274\275\276\277%
    \300\301\302\303\304\305\306\307\310\311\312\313\314\315\316\317%
    \320\321\322\323\324\325\326\327\330\331\332\333\334\335\336\337%
    \340\341\342\343\344\345\346\347\350\351\352\353\354\355\356\357%
    \360\361\362\363\364\365\366\367\370\371\372\373\374\375\376\377%
  }%
|endgroup
\ltx@onelevel@sanitize\AllBytesString
%    \end{macrocode}
%    \begin{macrocode}
\def\Test#1#2#3{%
  \begingroup
    \expandafter\expandafter\expandafter\def
    \expandafter\expandafter\expandafter\TestResult
    \expandafter\expandafter\expandafter{%
      #1{#2}%
    }%
    \ifx\TestResult#3%
    \else
      \newlinechar=10 %
      \msg{Expect:^^J#3}%
      \msg{Result:^^J\TestResult}%
      \errmessage{\string#2 -\string#1-> \string#3}%
    \fi
  \endgroup
}
\def\test#1#2#3{%
  \edef\TestFrom{#2}%
  \edef\TestExpect{#3}%
  \ltx@onelevel@sanitize\TestExpect
  \Test#1\TestFrom\TestExpect
}
\test\pdf@unescapehex{74657374}{test}
\begingroup
  \catcode0=12 %
  \catcode1=12 %
  \test\pdf@unescapehex{740074017400740174}{t^^@t^^At^^@t^^At}%
\endgroup
\Test\pdf@escapehex\AllBytes\AllBytesHex
\Test\pdf@unescapehex\AllBytesHex\AllBytes
\Test\pdf@escapename\AllBytes\AllBytesName
\Test\pdf@escapestring\AllBytes\AllBytesString
%    \end{macrocode}
%    \begin{macrocode}
\csname @@end\endcsname\end
%</test-escape>
%    \end{macrocode}
%
% \section{Installation}
%
% \subsection{Download}
%
% \paragraph{Package.} This package is available on
% CTAN\footnote{\CTANpkg{pdftexcmds}}:
% \begin{description}
% \item[\CTAN{macros/latex/contrib/oberdiek/pdftexcmds.dtx}] The source file.
% \item[\CTAN{macros/latex/contrib/oberdiek/pdftexcmds.pdf}] Documentation.
% \end{description}
%
%
% \paragraph{Bundle.} All the packages of the bundle `oberdiek'
% are also available in a TDS compliant ZIP archive. There
% the packages are already unpacked and the documentation files
% are generated. The files and directories obey the TDS standard.
% \begin{description}
% \item[\CTANinstall{install/macros/latex/contrib/oberdiek.tds.zip}]
% \end{description}
% \emph{TDS} refers to the standard ``A Directory Structure
% for \TeX\ Files'' (\CTAN{tds/tds.pdf}). Directories
% with \xfile{texmf} in their name are usually organized this way.
%
% \subsection{Bundle installation}
%
% \paragraph{Unpacking.} Unpack the \xfile{oberdiek.tds.zip} in the
% TDS tree (also known as \xfile{texmf} tree) of your choice.
% Example (linux):
% \begin{quote}
%   |unzip oberdiek.tds.zip -d ~/texmf|
% \end{quote}
%
% \paragraph{Script installation.}
% Check the directory \xfile{TDS:scripts/oberdiek/} for
% scripts that need further installation steps.
% Package \xpackage{attachfile2} comes with the Perl script
% \xfile{pdfatfi.pl} that should be installed in such a way
% that it can be called as \texttt{pdfatfi}.
% Example (linux):
% \begin{quote}
%   |chmod +x scripts/oberdiek/pdfatfi.pl|\\
%   |cp scripts/oberdiek/pdfatfi.pl /usr/local/bin/|
% \end{quote}
%
% \subsection{Package installation}
%
% \paragraph{Unpacking.} The \xfile{.dtx} file is a self-extracting
% \docstrip\ archive. The files are extracted by running the
% \xfile{.dtx} through \plainTeX:
% \begin{quote}
%   \verb|tex pdftexcmds.dtx|
% \end{quote}
%
% \paragraph{TDS.} Now the different files must be moved into
% the different directories in your installation TDS tree
% (also known as \xfile{texmf} tree):
% \begin{quote}
% \def\t{^^A
% \begin{tabular}{@{}>{\ttfamily}l@{ $\rightarrow$ }>{\ttfamily}l@{}}
%   pdftexcmds.sty & tex/generic/oberdiek/pdftexcmds.sty\\
%   oberdiek.pdftexcmds.lua & scripts/oberdiek/oberdiek.pdftexcmds.lua\\
%   pdftexcmds.lua & scripts/oberdiek/pdftexcmds.lua\\
%   pdftexcmds.pdf & doc/latex/oberdiek/pdftexcmds.pdf\\
%   test/pdftexcmds-test1.tex & doc/latex/oberdiek/test/pdftexcmds-test1.tex\\
%   test/pdftexcmds-test2.tex & doc/latex/oberdiek/test/pdftexcmds-test2.tex\\
%   test/pdftexcmds-test-shell.tex & doc/latex/oberdiek/test/pdftexcmds-test-shell.tex\\
%   test/pdftexcmds-test-escape.tex & doc/latex/oberdiek/test/pdftexcmds-test-escape.tex\\
%   pdftexcmds.dtx & source/latex/oberdiek/pdftexcmds.dtx\\
% \end{tabular}^^A
% }^^A
% \sbox0{\t}^^A
% \ifdim\wd0>\linewidth
%   \begingroup
%     \advance\linewidth by\leftmargin
%     \advance\linewidth by\rightmargin
%   \edef\x{\endgroup
%     \def\noexpand\lw{\the\linewidth}^^A
%   }\x
%   \def\lwbox{^^A
%     \leavevmode
%     \hbox to \linewidth{^^A
%       \kern-\leftmargin\relax
%       \hss
%       \usebox0
%       \hss
%       \kern-\rightmargin\relax
%     }^^A
%   }^^A
%   \ifdim\wd0>\lw
%     \sbox0{\small\t}^^A
%     \ifdim\wd0>\linewidth
%       \ifdim\wd0>\lw
%         \sbox0{\footnotesize\t}^^A
%         \ifdim\wd0>\linewidth
%           \ifdim\wd0>\lw
%             \sbox0{\scriptsize\t}^^A
%             \ifdim\wd0>\linewidth
%               \ifdim\wd0>\lw
%                 \sbox0{\tiny\t}^^A
%                 \ifdim\wd0>\linewidth
%                   \lwbox
%                 \else
%                   \usebox0
%                 \fi
%               \else
%                 \lwbox
%               \fi
%             \else
%               \usebox0
%             \fi
%           \else
%             \lwbox
%           \fi
%         \else
%           \usebox0
%         \fi
%       \else
%         \lwbox
%       \fi
%     \else
%       \usebox0
%     \fi
%   \else
%     \lwbox
%   \fi
% \else
%   \usebox0
% \fi
% \end{quote}
% If you have a \xfile{docstrip.cfg} that configures and enables \docstrip's
% TDS installing feature, then some files can already be in the right
% place, see the documentation of \docstrip.
%
% \subsection{Refresh file name databases}
%
% If your \TeX~distribution
% (\teTeX, \mikTeX, \dots) relies on file name databases, you must refresh
% these. For example, \teTeX\ users run \verb|texhash| or
% \verb|mktexlsr|.
%
% \subsection{Some details for the interested}
%
% \paragraph{Unpacking with \LaTeX.}
% The \xfile{.dtx} chooses its action depending on the format:
% \begin{description}
% \item[\plainTeX:] Run \docstrip\ and extract the files.
% \item[\LaTeX:] Generate the documentation.
% \end{description}
% If you insist on using \LaTeX\ for \docstrip\ (really,
% \docstrip\ does not need \LaTeX), then inform the autodetect routine
% about your intention:
% \begin{quote}
%   \verb|latex \let\install=y% \iffalse meta-comment
%
% File: pdftexcmds.dtx
% Version: 2019/07/25 v0.30
% Info: Utility functions of pdfTeX for LuaTeX
%
% Copyright (C) 2007, 2009-2011 by
%    Heiko Oberdiek <heiko.oberdiek at googlemail.com>
%
% This work may be distributed and/or modified under the
% conditions of the LaTeX Project Public License, either
% version 1.3c of this license or (at your option) any later
% version. This version of this license is in
%    https://www.latex-project.org/lppl/lppl-1-3c.txt
% and the latest version of this license is in
%    https://www.latex-project.org/lppl.txt
% and version 1.3 or later is part of all distributions of
% LaTeX version 2005/12/01 or later.
%
% This work has the LPPL maintenance status "maintained".
%
% The Current Maintainers of this work are
% Heiko Oberdiek and the Oberdiek Package Support Group
% https://github.com/ho-tex/oberdiek/issues
%
% The Base Interpreter refers to any `TeX-Format',
% because some files are installed in TDS:tex/generic//.
%
% This work consists of the main source file pdftexcmds.dtx
% and the derived files
%    pdftexcmds.sty, pdftexcmds.pdf, pdftexcmds.ins, pdftexcmds.drv,
%    pdftexcmds.bib, pdftexcmds-test1.tex, pdftexcmds-test2.tex,
%    pdftexcmds-test-shell.tex, pdftexcmds-test-escape.tex,
%    oberdiek.pdftexcmds.lua, pdftexcmds.lua.
%
% Distribution:
%    CTAN:macros/latex/contrib/oberdiek/pdftexcmds.dtx
%    CTAN:macros/latex/contrib/oberdiek/pdftexcmds.pdf
%
% Unpacking:
%    (a) If pdftexcmds.ins is present:
%           tex pdftexcmds.ins
%    (b) Without pdftexcmds.ins:
%           tex pdftexcmds.dtx
%    (c) If you insist on using LaTeX
%           latex \let\install=y\input{pdftexcmds.dtx}
%        (quote the arguments according to the demands of your shell)
%
% Documentation:
%    (a) If pdftexcmds.drv is present:
%           latex pdftexcmds.drv
%    (b) Without pdftexcmds.drv:
%           latex pdftexcmds.dtx; ...
%    The class ltxdoc loads the configuration file ltxdoc.cfg
%    if available. Here you can specify further options, e.g.
%    use A4 as paper format:
%       \PassOptionsToClass{a4paper}{article}
%
%    Programm calls to get the documentation (example):
%       pdflatex pdftexcmds.dtx
%       bibtex pdftexcmds.aux
%       makeindex -s gind.ist pdftexcmds.idx
%       pdflatex pdftexcmds.dtx
%       makeindex -s gind.ist pdftexcmds.idx
%       pdflatex pdftexcmds.dtx
%
% Installation:
%    TDS:tex/generic/oberdiek/pdftexcmds.sty
%    TDS:scripts/oberdiek/oberdiek.pdftexcmds.lua
%    TDS:scripts/oberdiek/pdftexcmds.lua
%    TDS:doc/latex/oberdiek/pdftexcmds.pdf
%    TDS:doc/latex/oberdiek/test/pdftexcmds-test1.tex
%    TDS:doc/latex/oberdiek/test/pdftexcmds-test2.tex
%    TDS:doc/latex/oberdiek/test/pdftexcmds-test-shell.tex
%    TDS:doc/latex/oberdiek/test/pdftexcmds-test-escape.tex
%    TDS:source/latex/oberdiek/pdftexcmds.dtx
%
%<*ignore>
\begingroup
  \catcode123=1 %
  \catcode125=2 %
  \def\x{LaTeX2e}%
\expandafter\endgroup
\ifcase 0\ifx\install y1\fi\expandafter
         \ifx\csname processbatchFile\endcsname\relax\else1\fi
         \ifx\fmtname\x\else 1\fi\relax
\else\csname fi\endcsname
%</ignore>
%<*install>
\input docstrip.tex
\Msg{************************************************************************}
\Msg{* Installation}
\Msg{* Package: pdftexcmds 2019/07/25 v0.30 Utility functions of pdfTeX for LuaTeX (HO)}
\Msg{************************************************************************}

\keepsilent
\askforoverwritefalse

\let\MetaPrefix\relax
\preamble

This is a generated file.

Project: pdftexcmds
Version: 2019/07/25 v0.30

Copyright (C) 2007, 2009-2011 by
   Heiko Oberdiek <heiko.oberdiek at googlemail.com>

This work may be distributed and/or modified under the
conditions of the LaTeX Project Public License, either
version 1.3c of this license or (at your option) any later
version. This version of this license is in
   https://www.latex-project.org/lppl/lppl-1-3c.txt
and the latest version of this license is in
   https://www.latex-project.org/lppl.txt
and version 1.3 or later is part of all distributions of
LaTeX version 2005/12/01 or later.

This work has the LPPL maintenance status "maintained".

The Current Maintainers of this work are
Heiko Oberdiek and the Oberdiek Package Support Group
https://github.com/ho-tex/oberdiek/issues


The Base Interpreter refers to any `TeX-Format',
because some files are installed in TDS:tex/generic//.

This work consists of the main source file pdftexcmds.dtx
and the derived files
   pdftexcmds.sty, pdftexcmds.pdf, pdftexcmds.ins, pdftexcmds.drv,
   pdftexcmds.bib, pdftexcmds-test1.tex, pdftexcmds-test2.tex,
   pdftexcmds-test-shell.tex, pdftexcmds-test-escape.tex,
   oberdiek.pdftexcmds.lua, pdftexcmds.lua.

\endpreamble
\let\MetaPrefix\DoubleperCent

\generate{%
  \file{pdftexcmds.ins}{\from{pdftexcmds.dtx}{install}}%
  \file{pdftexcmds.drv}{\from{pdftexcmds.dtx}{driver}}%
  \nopreamble
  \nopostamble
  \file{pdftexcmds.bib}{\from{pdftexcmds.dtx}{bib}}%
  \usepreamble\defaultpreamble
  \usepostamble\defaultpostamble
  \usedir{tex/generic/oberdiek}%
  \file{pdftexcmds.sty}{\from{pdftexcmds.dtx}{package}}%
%  \usedir{doc/latex/oberdiek/test}%
%  \file{pdftexcmds-test1.tex}{\from{pdftexcmds.dtx}{test1}}%
%  \file{pdftexcmds-test2.tex}{\from{pdftexcmds.dtx}{test2}}%
%  \file{pdftexcmds-test-shell.tex}{\from{pdftexcmds.dtx}{test-shell}}%
%  \file{pdftexcmds-test-escape.tex}{\from{pdftexcmds.dtx}{test-escape}}%
  \nopreamble
  \nopostamble
%  \usedir{source/latex/oberdiek/catalogue}%
%  \file{pdftexcmds.xml}{\from{pdftexcmds.dtx}{catalogue}}%
}
\def\MetaPrefix{-- }
\def\defaultpostamble{%
  \MetaPrefix^^J%
  \MetaPrefix\space End of File `\outFileName'.%
}
\def\currentpostamble{\defaultpostamble}%
\generate{%
  \usedir{scripts/oberdiek}%
  \file{oberdiek.pdftexcmds.lua}{\from{pdftexcmds.dtx}{lua}}%
  \file{pdftexcmds.lua}{\from{pdftexcmds.dtx}{lua}}%
}

\catcode32=13\relax% active space
\let =\space%
\Msg{************************************************************************}
\Msg{*}
\Msg{* To finish the installation you have to move the following}
\Msg{* file into a directory searched by TeX:}
\Msg{*}
\Msg{*     pdftexcmds.sty}
\Msg{*}
\Msg{* And install the following script files:}
\Msg{*}
\Msg{*     oberdiek.pdftexcmds.lua, pdftexcmds.lua}
\Msg{*}
\Msg{* To produce the documentation run the file `pdftexcmds.drv'}
\Msg{* through LaTeX.}
\Msg{*}
\Msg{* Happy TeXing!}
\Msg{*}
\Msg{************************************************************************}

\endbatchfile
%</install>
%<*bib>
@online{AndyThomas:Analog,
  author={Thomas, Andy},
  title={Analog of {\texttt{\csname textbackslash\endcsname}pdfelapsedtime} for
      {\hologo{LuaTeX}} and {\hologo{XeTeX}}},
  url={http://tex.stackexchange.com/a/32531},
  urldate={2011-11-29},
}
%</bib>
%<*ignore>
\fi
%</ignore>
%<*driver>
\NeedsTeXFormat{LaTeX2e}
\ProvidesFile{pdftexcmds.drv}%
  [2019/07/25 v0.30 Utility functions of pdfTeX for LuaTeX (HO)]%
\documentclass{ltxdoc}
\usepackage{holtxdoc}[2011/11/22]
\usepackage{paralist}
\usepackage{csquotes}
\usepackage[
  backend=bibtex,
  bibencoding=ascii,
  alldates=iso8601,
]{biblatex}[2011/11/13]
\bibliography{oberdiek-source}
\bibliography{pdftexcmds}
\begin{document}
  \DocInput{pdftexcmds.dtx}%
\end{document}
%</driver>
% \fi
%
%
% \CharacterTable
%  {Upper-case    \A\B\C\D\E\F\G\H\I\J\K\L\M\N\O\P\Q\R\S\T\U\V\W\X\Y\Z
%   Lower-case    \a\b\c\d\e\f\g\h\i\j\k\l\m\n\o\p\q\r\s\t\u\v\w\x\y\z
%   Digits        \0\1\2\3\4\5\6\7\8\9
%   Exclamation   \!     Double quote  \"     Hash (number) \#
%   Dollar        \$     Percent       \%     Ampersand     \&
%   Acute accent  \'     Left paren    \(     Right paren   \)
%   Asterisk      \*     Plus          \+     Comma         \,
%   Minus         \-     Point         \.     Solidus       \/
%   Colon         \:     Semicolon     \;     Less than     \<
%   Equals        \=     Greater than  \>     Question mark \?
%   Commercial at \@     Left bracket  \[     Backslash     \\
%   Right bracket \]     Circumflex    \^     Underscore    \_
%   Grave accent  \`     Left brace    \{     Vertical bar  \|
%   Right brace   \}     Tilde         \~}
%
% \GetFileInfo{pdftexcmds.drv}
%
% \title{The \xpackage{pdftexcmds} package}
% \date{2019/07/25 v0.30}
% \author{Heiko Oberdiek\thanks
% {Please report any issues at \url{https://github.com/ho-tex/oberdiek/issues}}}
%
% \maketitle
%
% \begin{abstract}
% \hologo{LuaTeX} provides most of the commands of \hologo{pdfTeX} 1.40. However
% a number of utility functions are removed. This package tries to fill
% the gap and implements some of the missing primitive using Lua.
% \end{abstract}
%
% \tableofcontents
%
% \def\csi#1{\texttt{\textbackslash\textit{#1}}}
%
% \section{Documentation}
%
% Some primitives of \hologo{pdfTeX} \cite{pdftex-manual}
% are not defined by \hologo{LuaTeX} \cite{luatex-manual}.
% This package implements macro based solutions using Lua code
% for the following missing \hologo{pdfTeX} primitives;
% \begin{compactitem}
% \item \cs{pdfstrcmp}
% \item \cs{pdfunescapehex}
% \item \cs{pdfescapehex}
% \item \cs{pdfescapename}
% \item \cs{pdfescapestring}
% \item \cs{pdffilesize}
% \item \cs{pdffilemoddate}
% \item \cs{pdffiledump}
% \item \cs{pdfmdfivesum}
% \item \cs{pdfresettimer}
% \item \cs{pdfelapsedtime}
% \item |\immediate\write18|
% \end{compactitem}
% The original names of the primitives cannot be used:
% \begin{itemize}
% \item
% The syntax for their arguments cannot easily
% simulated by macros. The primitives using key words
% such as |file| (\cs{pdfmdfivesum}) or |offset| and |length|
% (\cs{pdffiledump}) and uses \meta{general text} for the other
% arguments. Using token registers assignments, \meta{general text} could
% be catched. However, the simulated primitives are expandable
% and register assignments would destroy this important property.
% (\meta{general text} allows something like |\expandafter\bgroup ...}|.)
% \item
% The original primitives can be expanded using one expansion step.
% The new macros need two expansion steps because of the additional
% macro expansion. Example:
% \begin{quote}
%   |\expandafter\foo\pdffilemoddate{file}|\\
%   vs.\\
%   |\expandafter\expandafter\expandafter|\\
%   |\foo\pdf@filemoddate{file}|
% \end{quote}
% \end{itemize}
%
% \hologo{LuaTeX} isn't stable yet and thus the status of this package is
% \emph{experimental}. Feedback is welcome.
%
% \subsection{General principles}
%
% \begin{description}
% \item[Naming convention:]
%   Usually this package defines a macro |\pdf@|\meta{cmd} if
%   \hologo{pdfTeX} provides |\pdf|\meta{cmd}.
% \item[Arguments:] The order of arguments in |\pdf@|\meta{cmd}
%   is the same as for the corresponding primitive of \hologo{pdfTeX}.
%   The arguments are ordinary undelimited \hologo{TeX} arguments,
%   no \meta{general text} and without additional keywords.
% \item[Expandibility:]
%   The macro |\pdf@|\meta{cmd} is expandable if the
%   corresponding \hologo{pdfTeX} primitive has this property.
%   Exact two expansion steps are necessary (first is the macro
%   expansion) except for \cs{pdf@primitive} and \cs{pdf@ifprimitive}.
%   The latter ones are not macros, but have the direct meaning of the
%   primitive.
% \item[Without \hologo{LuaTeX}:]
%   The macros |\pdf@|\meta{cmd} are mapped to the commands
%   of \hologo{pdfTeX} if they are available. Otherwise they are undefined.
% \item[Availability:]
%   The macros that the packages provides are undefined, if
%   the necessary primitives are not found and cannot be
%   implemented by Lua.
% \end{description}
%
% \subsection{Macros}
%
% \subsubsection[Strings]{Strings \cite[``7.15 Strings'']{pdftex-manual}}
%
% \begin{declcs}{pdf@strcmp} \M{stringA} \M{stringB}
% \end{declcs}
% Same as |\pdfstrcmp{|\meta{stringA}|}{|\meta{stringB}|}|.
%
% \begin{declcs}{pdf@unescapehex} \M{string}
% \end{declcs}
% Same as |\pdfunescapehex{|\meta{string}|}|.
% The argument is a byte string given in hexadecimal notation.
% The result are character tokens from 0 until 255 with
% catcode 12 and the space with catcode 10.
%
% \begin{declcs}{pdf@escapehex} \M{string}\\
%   \cs{pdf@escapestring} \M{string}\\
%   \cs{pdf@escapename} \M{string}
% \end{declcs}
% Same as the primitives of \hologo{pdfTeX}. However \hologo{pdfTeX} does not
% know about characters with codes 256 and larger. Thus the
% string is treated as byte string, characters with more than
% eight bits are ignored.
%
% \subsubsection[Files]{Files \cite[``7.18 Files'']{pdftex-manual}}
%
% \begin{declcs}{pdf@filesize} \M{filename}
% \end{declcs}
% Same as |\pdffilesize{|\meta{filename}|}|.
%
% \begin{declcs}{pdf@filemoddate} \M{filename}
% \end{declcs}
% Same as |\pdffilemoddate{|\meta{filename}|}|.
%
% \begin{declcs}{pdf@filedump} \M{offset} \M{length} \M{filename}
% \end{declcs}
% Same as |\pdffiledump offset| \meta{offset} |length| \meta{length}
% |{|\meta{filename}|}|. Both \meta{offset} and \meta{length} must
% not be empty, but must be a valid \hologo{TeX} number.
%
% \begin{declcs}{pdf@mdfivesum} \M{string}
% \end{declcs}
% Same as |\pdfmdfivesum{|\meta{string}|}|. Keyword |file| is supported
% by macro \cs{pdf@filemdfivesum}.
%
% \begin{declcs}{pdf@filemdfivesum} \M{filename}
% \end{declcs}
% Same as |\pdfmdfivesum file{|\meta{filename}|}|.
%
% \subsubsection[Timekeeping]{Timekeeping \cite[``7.17 Timekeeping'']{pdftex-manual}}
%
% The timekeeping macros are based on Andy Thomas' work \cite{AndyThomas:Analog}.
%
% \begin{declcs}{pdf@resettimer}
% \end{declcs}
% Same as \cs{pdfresettimer}, it resets the internal timer.
%
% \begin{declcs}{pdf@elapsedtime}
% \end{declcs}
% Same as \cs{pdfelapsedtime}. It behaves like a read-only integer.
% For printing purposes it can be prefixed by \cs{the} or \cs{number}.
% It measures the time in scaled seconds (seconds multiplied with 65536)
% since the latest call of \cs{pdf@resettimer} or start of
% program/package. The resolution, the shortest time interval that
% can be measured, depends on the program and system.
% \begin{itemize}
% \item \hologo{pdfTeX} with |gettimeofday|: $\ge$ 1/65536\,s
% \item \hologo{pdfTeX} with |ftime|: $\ge$ 1\,ms
% \item \hologo{pdfTeX} with |time|: $\ge$ 1\,s
% \item \hologo{LuaTeX}: $\ge$ 10\,ms\\
%  (|os.clock()| returns a float number with two decimal digits in
%  \hologo{LuaTeX} beta-0.70.1-2011061416 (rev 4277)).
% \end{itemize}
%
% \subsubsection[Miscellaneous]{Miscellaneous \cite[``7.21 Miscellaneous'']{pdftex-manual}}
%
% \begin{declcs}{pdf@draftmode}
% \end{declcs}
% If the \TeX\ compiler knows \cs{pdfdraftmode} or \cs{draftmode}
% (\hologo{pdfTeX},
% \hologo{LuaTeX}), then \cs{pdf@draftmode} returns, whether
% this mode is enabled. The result is an implicit number:
% one means the draft mode is available and enabled.
% If the value is zero, then the mode is not active or
% \cs{pdfdraftmode} is not available.
% An explicit number is yielded by \cs{number}\cs{pdf@draftmode}.
% The macro cannot
% be used to change the mode, see \cs{pdf@setdraftmode}.
%
% \begin{declcs}{pdf@ifdraftmode} \M{true} \M{false}
% \end{declcs}
% If \cs{pdfdraftmode} is available and enabled, \meta{true} is
% called, otherwise \meta{false} is executed.
%
% \begin{declcs}{pdf@setdraftmode} \M{value}
% \end{declcs}
% Macro \cs{pdf@setdraftmode} expects the number zero or one as
% \meta{value}. Zero deactivates the mode and one enables the draft mode.
% The macro does not have an effect, if the feature \cs{pdfdraftmode} is not
% available.
%
% \begin{declcs}{pdf@shellescape}
% \end{declcs}
% Same as |\pdfshellescape|. It is or expands to |1| if external
% commands can be executed and |0| otherwise. In \hologo{pdfTeX} external
% commands must be enabled first by command line option or
% configuration option. In \hologo{LuaTeX} option |--safer| disables
% the execution of external commands.
%
% In \hologo{LuaTeX} before 0.68.0 \cs{pdf@shellescape} is not
% available due to a bug in |os.execute()|. The argumentless form
% crashes in some circumstances with segmentation fault.
% (It is fixed in version 0.68.0 or revision 4167 of \hologo{LuaTeX}.
% and packported to some version of 0.67.0).
%
% Hints for usage:
% \begin{itemize}
% \item Before its use \cs{pdf@shellescape} should be tested,
% whether it is available. Example with package \xpackage{ltxcmds}
% (loaded by package \xpackage{pdftexcmds}):
%\begin{quote}
%\begin{verbatim}
%\ltx@IfUndefined{pdf@shellescape}{%
%  % \pdf@shellescape is undefined
%}{%
%  % \pdf@shellescape is available
%}
%\end{verbatim}
%\end{quote}
% Use \cs{ltx@ifundefined} in expandable contexts.
% \item \cs{pdf@shellescape} might be a numerical constant,
% expands to the primitive, or expands to a plain number.
% Therefore use it in contexts where these differences does not matter.
% \item Use in comparisons, e.g.:
%   \begin{quote}
%     |\ifnum\pdf@shellescape=0 ...|
%   \end{quote}
% \item Print the number: |\number\pdf@shellescape|
% \end{itemize}
%
% \begin{declcs}{pdf@system} \M{cmdline}
% \end{declcs}
% It is a wrapper for |\immediate\write18| in \hologo{pdfTeX} or
% |os.execute| in \hologo{LuaTeX}.
%
% In theory |os.execute|
% returns a status number. But its meaning is quite
% undefined. Are there some reliable properties?
% Does it make sense to provide an user interface to
% this status exit code?
%
% \begin{declcs}{pdf@primitive} \csi{cmd}
% \end{declcs}
% Same as \cs{pdfprimitive} in \hologo{pdfTeX} or \hologo{LuaTeX}.
% In \hologo{XeTeX} the
% primitive is called \cs{primitive}. Despite the current definition
% of the command \csi{cmd}, it's meaning as primitive is used.
%
% \begin{declcs}{pdf@ifprimitive} \csi{cmd}
% \end{declcs}
% Same as \cs{ifpdfprimitive} in \hologo{pdfTeX} or
% \hologo{LuaTeX}. \hologo{XeTeX} calls
% it \cs{ifprimitive}. It is a switch that checks if the command
% \csi{cmd} has it's primitive meaning.
%
% \subsubsection{Additional macro: \cs{pdf@isprimitive}}
%
% \begin{declcs}{pdf@isprimitive} \csi{cmd1} \csi{cmd2} \M{true} \M{false}
% \end{declcs}
% If \csi{cmd1} has the primitive meaning given by the primitive name
% of \csi{cmd2}, then the argument \meta{true} is executed, otherwise
% \meta{false}. The macro \cs{pdf@isprimitive} is expandable.
% Internally it checks the result of \cs{meaning} and is therefore
% available for all \hologo{TeX} variants, even the original \hologo{TeX}.
% Example with \hologo{LaTeX}:
%\begin{quote}
%\begin{verbatim}
%\makeatletter
%\pdf@isprimitive{@@input}{input}{%
%  \typeout{\string\@@input\space is original\string\input}%
%}{%
%  \typeout{Oops, \string\@@input\space is not the %
%           original\string\input}%
%}
%\end{verbatim}
%\end{quote}
%
% \subsubsection{Experimental}
%
% \begin{declcs}{pdf@unescapehexnative} \M{string}\\
%   \cs{pdf@escapehexnative} \M{string}\\
%   \cs{pdf@escapenamenative} \M{string}\\
%   \cs{pdf@mdfivesumnative} \M{string}
% \end{declcs}
% The variants without |native| in the macro name are supposed to
% be compatible with \hologo{pdfTeX}. However characters with more than
% eight bits are not supported and are ignored. If \hologo{LuaTeX} is
% running, then its UTF-8 coded strings are used. Thus the full
% unicode character range is supported. However the result
% differs from \hologo{pdfTeX} for characters with eight or more bits.
%
% \begin{declcs}{pdf@pipe} \M{cmdline}
% \end{declcs}
% It calls \meta{cmdline} and returns the output of the external
% program in the usual manner as byte string (catcode 12, space with
% catcode 10). The Lua documentation says, that the used |io.popen|
% may not be available on all platforms. Then macro \cs{pdf@pipe}
% is undefined.
%
% \StopEventually{
% }
%
% \section{Implementation}
%
%    \begin{macrocode}
%<*package>
%    \end{macrocode}
%
% \subsection{Reload check and package identification}
%    Reload check, especially if the package is not used with \LaTeX.
%    \begin{macrocode}
\begingroup\catcode61\catcode48\catcode32=10\relax%
  \catcode13=5 % ^^M
  \endlinechar=13 %
  \catcode35=6 % #
  \catcode39=12 % '
  \catcode44=12 % ,
  \catcode45=12 % -
  \catcode46=12 % .
  \catcode58=12 % :
  \catcode64=11 % @
  \catcode123=1 % {
  \catcode125=2 % }
  \expandafter\let\expandafter\x\csname ver@pdftexcmds.sty\endcsname
  \ifx\x\relax % plain-TeX, first loading
  \else
    \def\empty{}%
    \ifx\x\empty % LaTeX, first loading,
      % variable is initialized, but \ProvidesPackage not yet seen
    \else
      \expandafter\ifx\csname PackageInfo\endcsname\relax
        \def\x#1#2{%
          \immediate\write-1{Package #1 Info: #2.}%
        }%
      \else
        \def\x#1#2{\PackageInfo{#1}{#2, stopped}}%
      \fi
      \x{pdftexcmds}{The package is already loaded}%
      \aftergroup\endinput
    \fi
  \fi
\endgroup%
%    \end{macrocode}
%    Package identification:
%    \begin{macrocode}
\begingroup\catcode61\catcode48\catcode32=10\relax%
  \catcode13=5 % ^^M
  \endlinechar=13 %
  \catcode35=6 % #
  \catcode39=12 % '
  \catcode40=12 % (
  \catcode41=12 % )
  \catcode44=12 % ,
  \catcode45=12 % -
  \catcode46=12 % .
  \catcode47=12 % /
  \catcode58=12 % :
  \catcode64=11 % @
  \catcode91=12 % [
  \catcode93=12 % ]
  \catcode123=1 % {
  \catcode125=2 % }
  \expandafter\ifx\csname ProvidesPackage\endcsname\relax
    \def\x#1#2#3[#4]{\endgroup
      \immediate\write-1{Package: #3 #4}%
      \xdef#1{#4}%
    }%
  \else
    \def\x#1#2[#3]{\endgroup
      #2[{#3}]%
      \ifx#1\@undefined
        \xdef#1{#3}%
      \fi
      \ifx#1\relax
        \xdef#1{#3}%
      \fi
    }%
  \fi
\expandafter\x\csname ver@pdftexcmds.sty\endcsname
\ProvidesPackage{pdftexcmds}%
  [2019/07/25 v0.30 Utility functions of pdfTeX for LuaTeX (HO)]%
%    \end{macrocode}
%
% \subsection{Catcodes}
%
%    \begin{macrocode}
\begingroup\catcode61\catcode48\catcode32=10\relax%
  \catcode13=5 % ^^M
  \endlinechar=13 %
  \catcode123=1 % {
  \catcode125=2 % }
  \catcode64=11 % @
  \def\x{\endgroup
    \expandafter\edef\csname pdftexcmds@AtEnd\endcsname{%
      \endlinechar=\the\endlinechar\relax
      \catcode13=\the\catcode13\relax
      \catcode32=\the\catcode32\relax
      \catcode35=\the\catcode35\relax
      \catcode61=\the\catcode61\relax
      \catcode64=\the\catcode64\relax
      \catcode123=\the\catcode123\relax
      \catcode125=\the\catcode125\relax
    }%
  }%
\x\catcode61\catcode48\catcode32=10\relax%
\catcode13=5 % ^^M
\endlinechar=13 %
\catcode35=6 % #
\catcode64=11 % @
\catcode123=1 % {
\catcode125=2 % }
\def\TMP@EnsureCode#1#2{%
  \edef\pdftexcmds@AtEnd{%
    \pdftexcmds@AtEnd
    \catcode#1=\the\catcode#1\relax
  }%
  \catcode#1=#2\relax
}
\TMP@EnsureCode{0}{12}%
\TMP@EnsureCode{1}{12}%
\TMP@EnsureCode{2}{12}%
\TMP@EnsureCode{10}{12}% ^^J
\TMP@EnsureCode{33}{12}% !
\TMP@EnsureCode{34}{12}% "
\TMP@EnsureCode{38}{4}% &
\TMP@EnsureCode{39}{12}% '
\TMP@EnsureCode{40}{12}% (
\TMP@EnsureCode{41}{12}% )
\TMP@EnsureCode{42}{12}% *
\TMP@EnsureCode{43}{12}% +
\TMP@EnsureCode{44}{12}% ,
\TMP@EnsureCode{45}{12}% -
\TMP@EnsureCode{46}{12}% .
\TMP@EnsureCode{47}{12}% /
\TMP@EnsureCode{58}{12}% :
\TMP@EnsureCode{60}{12}% <
\TMP@EnsureCode{62}{12}% >
\TMP@EnsureCode{91}{12}% [
\TMP@EnsureCode{93}{12}% ]
\TMP@EnsureCode{94}{7}% ^ (superscript)
\TMP@EnsureCode{95}{12}% _ (other)
\TMP@EnsureCode{96}{12}% `
\TMP@EnsureCode{126}{12}% ~ (other)
\edef\pdftexcmds@AtEnd{%
  \pdftexcmds@AtEnd
  \escapechar=\number\escapechar\relax
  \noexpand\endinput
}
\escapechar=92 %
%    \end{macrocode}
%
% \subsection{Load packages}
%
%    \begin{macrocode}
\begingroup\expandafter\expandafter\expandafter\endgroup
\expandafter\ifx\csname RequirePackage\endcsname\relax
  \def\TMP@RequirePackage#1[#2]{%
    \begingroup\expandafter\expandafter\expandafter\endgroup
    \expandafter\ifx\csname ver@#1.sty\endcsname\relax
      \input #1.sty\relax
    \fi
  }%
  \TMP@RequirePackage{infwarerr}[2007/09/09]%
  \TMP@RequirePackage{ifluatex}[2010/03/01]%
  \TMP@RequirePackage{ltxcmds}[2010/12/02]%
  \TMP@RequirePackage{ifpdf}[2010/09/13]%
\else
  \RequirePackage{infwarerr}[2007/09/09]%
  \RequirePackage{ifluatex}[2010/03/01]%
  \RequirePackage{ltxcmds}[2010/12/02]%
  \RequirePackage{ifpdf}[2010/09/13]%
\fi
%    \end{macrocode}
%
% \subsection{Without \hologo{LuaTeX}}
%
%    \begin{macrocode}
\ifluatex
\else
  \@PackageInfoNoLine{pdftexcmds}{LuaTeX not detected}%
  \def\pdftexcmds@nopdftex{%
    \@PackageInfoNoLine{pdftexcmds}{pdfTeX >= 1.30 not detected}%
    \let\pdftexcmds@nopdftex\relax
  }%
  \def\pdftexcmds@temp#1{%
    \begingroup\expandafter\expandafter\expandafter\endgroup
    \expandafter\ifx\csname pdf#1\endcsname\relax
      \pdftexcmds@nopdftex
    \else
      \expandafter\def\csname pdf@#1\expandafter\endcsname
      \expandafter##\expandafter{%
        \csname pdf#1\endcsname
      }%
    \fi
  }%
  \pdftexcmds@temp{strcmp}%
  \pdftexcmds@temp{escapehex}%
  \let\pdf@escapehexnative\pdf@escapehex
  \pdftexcmds@temp{unescapehex}%
  \let\pdf@unescapehexnative\pdf@unescapehex
  \pdftexcmds@temp{escapestring}%
  \pdftexcmds@temp{escapename}%
  \pdftexcmds@temp{filesize}%
  \pdftexcmds@temp{filemoddate}%
  \begingroup\expandafter\expandafter\expandafter\endgroup
  \expandafter\ifx\csname pdfshellescape\endcsname\relax
    \pdftexcmds@nopdftex
    \ltx@IfUndefined{pdftexversion}{%
    }{%
      \ifnum\pdftexversion>120 % 1.21a supports \ifeof18
        \ifeof18 %
          \chardef\pdf@shellescape=0 %
        \else
          \chardef\pdf@shellescape=1 %
        \fi
      \fi
    }%
  \else
    \def\pdf@shellescape{%
      \pdfshellescape
    }%
  \fi
  \begingroup\expandafter\expandafter\expandafter\endgroup
  \expandafter\ifx\csname pdffiledump\endcsname\relax
    \pdftexcmds@nopdftex
  \else
    \def\pdf@filedump#1#2#3{%
      \pdffiledump offset#1 length#2{#3}%
    }%
  \fi
%    \end{macrocode}
%    \begin{macrocode}
  \begingroup\expandafter\expandafter\expandafter\endgroup
  \expandafter\ifx\csname pdfmdfivesum\endcsname\relax
    \begingroup\expandafter\expandafter\expandafter\endgroup
    \expandafter\ifx\csname mdfivesum\endcsname\relax
      \pdftexcmds@nopdftex
    \else
      \def\pdf@mdfivesum#{\mdfivesum}%
      \let\pdf@mdfivesumnative\pdf@mdfivesum
      \def\pdf@filemdfivesum#{\mdfivesum file}%
    \fi
  \else
    \def\pdf@mdfivesum#{\pdfmdfivesum}%
    \let\pdf@mdfivesumnative\pdf@mdfivesum
    \def\pdf@filemdfivesum#{\pdfmdfivesum file}%
  \fi
%    \end{macrocode}
%    \begin{macrocode}
  \def\pdf@system#{%
    \immediate\write18%
  }%
  \def\pdftexcmds@temp#1{%
    \begingroup\expandafter\expandafter\expandafter\endgroup
    \expandafter\ifx\csname pdf#1\endcsname\relax
      \pdftexcmds@nopdftex
    \else
      \expandafter\let\csname pdf@#1\expandafter\endcsname
      \csname pdf#1\endcsname
    \fi
  }%
  \pdftexcmds@temp{resettimer}%
  \pdftexcmds@temp{elapsedtime}%
\fi
%    \end{macrocode}
%
% \subsection{\cs{pdf@primitive}, \cs{pdf@ifprimitive}}
%
%    Since version 1.40.0 \hologo{pdfTeX} has \cs{pdfprimitive} and
%    \cs{ifpdfprimitive}. And \cs{pdfprimitive} was fixed in
%    version 1.40.4.
%
%    \hologo{XeTeX} provides them under the name \cs{primitive} and
%    \cs{ifprimitive}. \hologo{LuaTeX} knows both name variants,
%    but they have possibly to be enabled first (|tex.enableprimitives|).
%
%    Depending on the format TeX Live uses a prefix |luatex|.
%
%    Caution: \cs{let} must be used for the definition of
%    the macros, especially because of \cs{ifpdfprimitive}.
%
% \subsubsection{Using \hologo{LuaTeX}'s \texttt{tex.enableprimitives}}
%
%    \begin{macrocode}
\ifluatex
%    \end{macrocode}
%    \begin{macro}{\pdftexcmds@directlua}
%    \begin{macrocode}
  \ifnum\luatexversion<36 %
    \def\pdftexcmds@directlua{\directlua0 }%
  \else
    \let\pdftexcmds@directlua\directlua
  \fi
%    \end{macrocode}
%    \end{macro}
%
%    \begin{macrocode}
  \begingroup
    \newlinechar=10 %
    \endlinechar=\newlinechar
    \pdftexcmds@directlua{%
      if tex.enableprimitives then
        tex.enableprimitives(
          'pdf@',
          {'primitive', 'ifprimitive', 'pdfdraftmode','draftmode'}
        )
        tex.enableprimitives('', {'luaescapestring'})
      end
    }%
  \endgroup %
%    \end{macrocode}
%
%    \begin{macrocode}
\fi
%    \end{macrocode}
%
% \subsubsection{Trying various names to find the primitives}
%
%    \begin{macro}{\pdftexcmds@strip@prefix}
%    \begin{macrocode}
\def\pdftexcmds@strip@prefix#1>{}
%    \end{macrocode}
%    \end{macro}
%    \begin{macrocode}
\def\pdftexcmds@temp#1#2#3{%
  \begingroup\expandafter\expandafter\expandafter\endgroup
  \expandafter\ifx\csname pdf@#1\endcsname\relax
    \begingroup
      \def\x{#3}%
      \edef\x{\expandafter\pdftexcmds@strip@prefix\meaning\x}%
      \escapechar=-1 %
      \edef\y{\expandafter\meaning\csname#2\endcsname}%
    \expandafter\endgroup
    \ifx\x\y
      \expandafter\let\csname pdf@#1\expandafter\endcsname
      \csname #2\endcsname
    \fi
  \fi
}
%    \end{macrocode}
%
%    \begin{macro}{\pdf@primitive}
%    \begin{macrocode}
\pdftexcmds@temp{primitive}{pdfprimitive}{pdfprimitive}% pdfTeX, oldLuaTeX
\pdftexcmds@temp{primitive}{primitive}{primitive}% XeTeX, luatex
\pdftexcmds@temp{primitive}{luatexprimitive}{pdfprimitive}% oldLuaTeX
\pdftexcmds@temp{primitive}{luatexpdfprimitive}{pdfprimitive}% oldLuaTeX
%    \end{macrocode}
%    \end{macro}
%    \begin{macro}{\pdf@ifprimitive}
%    \begin{macrocode}
\pdftexcmds@temp{ifprimitive}{ifpdfprimitive}{ifpdfprimitive}% pdfTeX, oldLuaTeX
\pdftexcmds@temp{ifprimitive}{ifprimitive}{ifprimitive}% XeTeX, luatex
\pdftexcmds@temp{ifprimitive}{luatexifprimitive}{ifpdfprimitive}% oldLuaTeX
\pdftexcmds@temp{ifprimitive}{luatexifpdfprimitive}{ifpdfprimitive}% oldLuaTeX
%    \end{macrocode}
%    \end{macro}
%
%    Disable broken \cs{pdfprimitive}.
%    \begin{macrocode}
\ifluatex\else
\begingroup
  \expandafter\ifx\csname pdf@primitive\endcsname\relax
  \else
    \expandafter\ifx\csname pdftexversion\endcsname\relax
    \else
      \ifnum\pdftexversion=140 %
        \expandafter\ifx\csname pdftexrevision\endcsname\relax
        \else
          \ifnum\pdftexrevision<4 %
            \endgroup
            \let\pdf@primitive\@undefined
            \@PackageInfoNoLine{pdftexcmds}{%
              \string\pdf@primitive\space disabled, %
              because\MessageBreak
              \string\pdfprimitive\space is broken until pdfTeX 1.40.4%
            }%
            \begingroup
          \fi
        \fi
      \fi
    \fi
  \fi
\endgroup
\fi
%    \end{macrocode}
%
% \subsubsection{Result}
%
%    \begin{macrocode}
\begingroup
  \@PackageInfoNoLine{pdftexcmds}{%
    \string\pdf@primitive\space is %
    \expandafter\ifx\csname pdf@primitive\endcsname\relax not \fi
    available%
  }%
  \@PackageInfoNoLine{pdftexcmds}{%
    \string\pdf@ifprimitive\space is %
    \expandafter\ifx\csname pdf@ifprimitive\endcsname\relax not \fi
    available%
  }%
\endgroup
%    \end{macrocode}
%
% \subsection{\hologo{XeTeX}}
%
%    Look for primitives \cs{shellescape}, \cs{strcmp}.
%    \begin{macrocode}
\def\pdftexcmds@temp#1{%
  \begingroup\expandafter\expandafter\expandafter\endgroup
  \expandafter\ifx\csname pdf@#1\endcsname\relax
    \begingroup
      \escapechar=-1 %
      \edef\x{\expandafter\meaning\csname#1\endcsname}%
      \def\y{#1}%
      \def\z##1->{}%
      \edef\y{\expandafter\z\meaning\y}%
    \expandafter\endgroup
    \ifx\x\y
      \expandafter\def\csname pdf@#1\expandafter\endcsname
      \expandafter{%
        \csname#1\endcsname
      }%
    \fi
  \fi
}%
\pdftexcmds@temp{shellescape}%
\pdftexcmds@temp{strcmp}%
%    \end{macrocode}
%
% \subsection{\cs{pdf@isprimitive}}
%
%    \begin{macrocode}
\def\pdf@isprimitive{%
  \begingroup\expandafter\expandafter\expandafter\endgroup
  \expandafter\ifx\csname pdf@strcmp\endcsname\relax
    \long\def\pdf@isprimitive##1{%
      \expandafter\pdftexcmds@isprimitive\expandafter{\meaning##1}%
    }%
    \long\def\pdftexcmds@isprimitive##1##2{%
      \expandafter\pdftexcmds@@isprimitive\expandafter{\string##2}{##1}%
    }%
    \def\pdftexcmds@@isprimitive##1##2{%
      \ifnum0\pdftexcmds@equal##1\delimiter##2\delimiter=1 %
        \expandafter\ltx@firstoftwo
      \else
        \expandafter\ltx@secondoftwo
      \fi
    }%
    \def\pdftexcmds@equal##1##2\delimiter##3##4\delimiter{%
      \ifx##1##3%
        \ifx\relax##2##4\relax
          1%
        \else
          \ifx\relax##2\relax
          \else
            \ifx\relax##4\relax
            \else
              \pdftexcmds@equalcont{##2}{##4}%
            \fi
          \fi
        \fi
      \fi
    }%
    \def\pdftexcmds@equalcont##1{%
      \def\pdftexcmds@equalcont####1####2##1##1##1##1{%
        ##1##1##1##1%
        \pdftexcmds@equal####1\delimiter####2\delimiter
      }%
    }%
    \expandafter\pdftexcmds@equalcont\csname fi\endcsname
  \else
    \long\def\pdf@isprimitive##1##2{%
      \ifnum\pdf@strcmp{\meaning##1}{\string##2}=0 %
        \expandafter\ltx@firstoftwo
      \else
        \expandafter\ltx@secondoftwo
      \fi
    }%
  \fi
}
\ifluatex
\ifx\pdfdraftmode\@undefined
  \let\pdfdraftmode\draftmode
\fi
\else
  \pdf@isprimitive
\fi
%    \end{macrocode}
%
% \subsection{\cs{pdf@draftmode}}
%
%
%    \begin{macrocode}
\let\pdftexcmds@temp\ltx@zero %
\ltx@IfUndefined{pdfdraftmode}{%
  \@PackageInfoNoLine{pdftexcmds}{\ltx@backslashchar pdfdraftmode not found}%
}{%
  \ifpdf
    \let\pdftexcmds@temp\ltx@one
    \@PackageInfoNoLine{pdftexcmds}{\ltx@backslashchar pdfdraftmode found}%
  \else
    \@PackageInfoNoLine{pdftexcmds}{%
      \ltx@backslashchar pdfdraftmode is ignored in DVI mode%
    }%
  \fi
}
\ifcase\pdftexcmds@temp
%    \end{macrocode}
%    \begin{macro}{\pdf@draftmode}
%    \begin{macrocode}
  \let\pdf@draftmode\ltx@zero
%    \end{macrocode}
%    \end{macro}
%    \begin{macro}{\pdf@ifdraftmode}
%    \begin{macrocode}
  \let\pdf@ifdraftmode\ltx@secondoftwo
%    \end{macrocode}
%    \end{macro}
%    \begin{macro}{\pdftexcmds@setdraftmode}
%    \begin{macrocode}
  \def\pdftexcmds@setdraftmode#1{}%
%    \end{macrocode}
%    \end{macro}
%    \begin{macrocode}
\else
%    \end{macrocode}
%    \begin{macro}{\pdftexcmds@draftmode}
%    \begin{macrocode}
  \let\pdftexcmds@draftmode\pdfdraftmode
%    \end{macrocode}
%    \end{macro}
%    \begin{macro}{\pdf@ifdraftmode}
%    \begin{macrocode}
  \def\pdf@ifdraftmode{%
    \ifnum\pdftexcmds@draftmode=\ltx@one
      \expandafter\ltx@firstoftwo
    \else
      \expandafter\ltx@secondoftwo
    \fi
  }%
%    \end{macrocode}
%    \end{macro}
%    \begin{macro}{\pdf@draftmode}
%    \begin{macrocode}
  \def\pdf@draftmode{%
    \ifnum\pdftexcmds@draftmode=\ltx@one
      \expandafter\ltx@one
    \else
      \expandafter\ltx@zero
    \fi
  }%
%    \end{macrocode}
%    \end{macro}
%    \begin{macro}{\pdftexcmds@setdraftmode}
%    \begin{macrocode}
  \def\pdftexcmds@setdraftmode#1{%
    \pdftexcmds@draftmode=#1\relax
  }%
%    \end{macrocode}
%    \end{macro}
%    \begin{macrocode}
\fi
%    \end{macrocode}
%    \begin{macro}{\pdf@setdraftmode}
%    \begin{macrocode}
\def\pdf@setdraftmode#1{%
  \begingroup
    \count\ltx@cclv=#1\relax
  \edef\x{\endgroup
    \noexpand\pdftexcmds@@setdraftmode{\the\count\ltx@cclv}%
  }%
  \x
}
%    \end{macrocode}
%    \end{macro}
%    \begin{macro}{\pdftexcmds@@setdraftmode}
%    \begin{macrocode}
\def\pdftexcmds@@setdraftmode#1{%
  \ifcase#1 %
    \pdftexcmds@setdraftmode{#1}%
  \or
    \pdftexcmds@setdraftmode{#1}%
  \else
    \@PackageWarning{pdftexcmds}{%
      \string\pdf@setdraftmode: Ignoring\MessageBreak
      invalid value `#1'%
    }%
  \fi
}
%    \end{macrocode}
%    \end{macro}
%
% \subsection{Load Lua module}
%
%    \begin{macrocode}
\ifluatex
\else
  \expandafter\pdftexcmds@AtEnd
\fi%
%    \end{macrocode}
%
%    \begin{macrocode}
\ifnum\luatexversion<80
  \begingroup\expandafter\expandafter\expandafter\endgroup
  \expandafter\ifx\csname RequirePackage\endcsname\relax
    \def\TMP@RequirePackage#1[#2]{%
      \begingroup\expandafter\expandafter\expandafter\endgroup
      \expandafter\ifx\csname ver@#1.sty\endcsname\relax
        \input #1.sty\relax
      \fi
    }%
    \TMP@RequirePackage{luatex-loader}[2009/04/10]%
  \else
    \RequirePackage{luatex-loader}[2009/04/10]%
  \fi
\fi
\pdftexcmds@directlua{%
  require("pdftexcmds")%
}
\ifnum\luatexversion>37 %
  \ifnum0%
      \pdftexcmds@directlua{%
        if status.ini_version then %
          tex.write("1")%
        end%
      }>0 %
    \everyjob\expandafter{%
      \the\everyjob
      \pdftexcmds@directlua{%
        require("pdftexcmds")%
      }%
    }%
  \fi
\fi
\begingroup
  \def\x{2019/07/25 v0.30}%
  \ltx@onelevel@sanitize\x
  \edef\y{%
    \pdftexcmds@directlua{%
      if oberdiek.pdftexcmds.getversion then %
        oberdiek.pdftexcmds.getversion()%
      end%
    }%
  }%
  \ifx\x\y
  \else
    \@PackageError{pdftexcmds}{%
      Wrong version of lua module.\MessageBreak
      Package version: \x\MessageBreak
      Lua module: \y
    }\@ehc
  \fi
\endgroup
%    \end{macrocode}
%
% \subsection{Lua functions}
%
% \subsubsection{Helper macros}
%
%    \begin{macro}{\pdftexcmds@toks}
%    \begin{macrocode}
\begingroup\expandafter\expandafter\expandafter\endgroup
\expandafter\ifx\csname newtoks\endcsname\relax
  \toksdef\pdftexcmds@toks=0 %
\else
  \csname newtoks\endcsname\pdftexcmds@toks
\fi
%    \end{macrocode}
%    \end{macro}
%
%    \begin{macro}{\pdftexcmds@Patch}
%    \begin{macrocode}
\def\pdftexcmds@Patch{0}
\ifnum\luatexversion>40 %
  \ifnum\luatexversion<66 %
    \def\pdftexcmds@Patch{1}%
  \fi
\fi
%    \end{macrocode}
%    \end{macro}
%    \begin{macrocode}
\ifcase\pdftexcmds@Patch
  \catcode`\&=14 %
\else
  \catcode`\&=9 %
%    \end{macrocode}
%    \begin{macro}{\pdftexcmds@PatchDecode}
%    \begin{macrocode}
  \def\pdftexcmds@PatchDecode#1\@nil{%
    \pdftexcmds@DecodeA#1^^A^^A\@nil{}%
  }%
%    \end{macrocode}
%    \end{macro}
%    \begin{macro}{\pdftexcmds@DecodeA}
%    \begin{macrocode}
  \def\pdftexcmds@DecodeA#1^^A^^A#2\@nil#3{%
    \ifx\relax#2\relax
      \ltx@ReturnAfterElseFi{%
        \pdftexcmds@DecodeB#3#1^^A^^B\@nil{}%
      }%
    \else
      \ltx@ReturnAfterFi{%
        \pdftexcmds@DecodeA#2\@nil{#3#1^^@}%
      }%
    \fi
  }%
%    \end{macrocode}
%    \end{macro}
%    \begin{macro}{\pdftexcmds@DecodeB}
%    \begin{macrocode}
  \def\pdftexcmds@DecodeB#1^^A^^B#2\@nil#3{%
    \ifx\relax#2\relax%
      \ltx@ReturnAfterElseFi{%
        \ltx@zero
        #3#1%
      }%
    \else
      \ltx@ReturnAfterFi{%
        \pdftexcmds@DecodeB#2\@nil{#3#1^^A}%
      }%
    \fi
  }%
%    \end{macrocode}
%    \end{macro}
%    \begin{macrocode}
\fi
%    \end{macrocode}
%
%    \begin{macrocode}
\ifnum\luatexversion<36 %
\else
  \catcode`\0=9 %
\fi
%    \end{macrocode}
%
% \subsubsection[Strings]{Strings \cite[``7.15 Strings'']{pdftex-manual}}
%
%    \begin{macro}{\pdf@strcmp}
%    \begin{macrocode}
\long\def\pdf@strcmp#1#2{%
  \directlua0{%
    oberdiek.pdftexcmds.strcmp("\luaescapestring{#1}",%
        "\luaescapestring{#2}")%
  }%
}%
%    \end{macrocode}
%    \end{macro}
%    \begin{macrocode}
\pdf@isprimitive
%    \end{macrocode}
%    \begin{macro}{\pdf@escapehex}
%    \begin{macrocode}
\long\def\pdf@escapehex#1{%
  \directlua0{%
    oberdiek.pdftexcmds.escapehex("\luaescapestring{#1}", "byte")%
  }%
}%
%    \end{macrocode}
%    \end{macro}
%    \begin{macro}{\pdf@escapehexnative}
%    \begin{macrocode}
\long\def\pdf@escapehexnative#1{%
  \directlua0{%
    oberdiek.pdftexcmds.escapehex("\luaescapestring{#1}")%
  }%
}%
%    \end{macrocode}
%    \end{macro}
%    \begin{macro}{\pdf@unescapehex}
%    \begin{macrocode}
\def\pdf@unescapehex#1{%
& \romannumeral\expandafter\pdftexcmds@PatchDecode
  \the\expandafter\pdftexcmds@toks
  \directlua0{%
    oberdiek.pdftexcmds.toks="pdftexcmds@toks"%
    oberdiek.pdftexcmds.unescapehex("\luaescapestring{#1}", "byte", \pdftexcmds@Patch)%
  }%
& \@nil
}%
%    \end{macrocode}
%    \end{macro}
%    \begin{macro}{\pdf@unescapehexnative}
%    \begin{macrocode}
\def\pdf@unescapehexnative#1{%
& \romannumeral\expandafter\pdftexcmds@PatchDecode
  \the\expandafter\pdftexcmds@toks
  \directlua0{%
    oberdiek.pdftexcmds.toks="pdftexcmds@toks"%
    oberdiek.pdftexcmds.unescapehex("\luaescapestring{#1}", \pdftexcmds@Patch)%
  }%
& \@nil
}%
%    \end{macrocode}
%    \end{macro}
%    \begin{macro}{\pdf@escapestring}
%    \begin{macrocode}
\long\def\pdf@escapestring#1{%
  \directlua0{%
    oberdiek.pdftexcmds.escapestring("\luaescapestring{#1}", "byte")%
  }%
}
%    \end{macrocode}
%    \end{macro}
%    \begin{macro}{\pdf@escapename}
%    \begin{macrocode}
\long\def\pdf@escapename#1{%
  \directlua0{%
    oberdiek.pdftexcmds.escapename("\luaescapestring{#1}", "byte")%
  }%
}
%    \end{macrocode}
%    \end{macro}
%    \begin{macro}{\pdf@escapenamenative}
%    \begin{macrocode}
\long\def\pdf@escapenamenative#1{%
  \directlua0{%
    oberdiek.pdftexcmds.escapename("\luaescapestring{#1}")%
  }%
}
%    \end{macrocode}
%    \end{macro}
%
% \subsubsection[Files]{Files \cite[``7.18 Files'']{pdftex-manual}}
%
%    \begin{macro}{\pdf@filesize}
%    \begin{macrocode}
\def\pdf@filesize#1{%
  \directlua0{%
    oberdiek.pdftexcmds.filesize("\luaescapestring{#1}")%
  }%
}
%    \end{macrocode}
%    \end{macro}
%    \begin{macro}{\pdf@filemoddate}
%    \begin{macrocode}
\def\pdf@filemoddate#1{%
  \directlua0{%
    oberdiek.pdftexcmds.filemoddate("\luaescapestring{#1}")%
  }%
}
%    \end{macrocode}
%    \end{macro}
%    \begin{macro}{\pdf@filedump}
%    \begin{macrocode}
\def\pdf@filedump#1#2#3{%
  \directlua0{%
    oberdiek.pdftexcmds.filedump("\luaescapestring{\number#1}",%
        "\luaescapestring{\number#2}",%
        "\luaescapestring{#3}")%
  }%
}%
%    \end{macrocode}
%    \end{macro}
%    \begin{macro}{\pdf@mdfivesum}
%    \begin{macrocode}
\long\def\pdf@mdfivesum#1{%
  \directlua0{%
    oberdiek.pdftexcmds.mdfivesum("\luaescapestring{#1}", "byte")%
  }%
}%
%    \end{macrocode}
%    \end{macro}
%    \begin{macro}{\pdf@mdfivesumnative}
%    \begin{macrocode}
\long\def\pdf@mdfivesumnative#1{%
  \directlua0{%
    oberdiek.pdftexcmds.mdfivesum("\luaescapestring{#1}")%
  }%
}%
%    \end{macrocode}
%    \end{macro}
%    \begin{macro}{\pdf@filemdfivesum}
%    \begin{macrocode}
\def\pdf@filemdfivesum#1{%
  \directlua0{%
    oberdiek.pdftexcmds.filemdfivesum("\luaescapestring{#1}")%
  }%
}%
%    \end{macrocode}
%    \end{macro}
%
% \subsubsection[Timekeeping]{Timekeeping \cite[``7.17 Timekeeping'']{pdftex-manual}}
%
%    \begin{macro}{\protected}
%    \begin{macrocode}
\let\pdftexcmds@temp=Y%
\begingroup\expandafter\expandafter\expandafter\endgroup
\expandafter\ifx\csname protected\endcsname\relax
  \pdftexcmds@directlua0{%
    if tex.enableprimitives then %
      tex.enableprimitives('', {'protected'})%
    end%
  }%
\fi
\begingroup\expandafter\expandafter\expandafter\endgroup
\expandafter\ifx\csname protected\endcsname\relax
  \let\pdftexcmds@temp=N%
\fi
%    \end{macrocode}
%    \end{macro}
%    \begin{macro}{\numexpr}
%    \begin{macrocode}
\begingroup\expandafter\expandafter\expandafter\endgroup
\expandafter\ifx\csname numexpr\endcsname\relax
  \pdftexcmds@directlua0{%
    if tex.enableprimitives then %
      tex.enableprimitives('', {'numexpr'})%
    end%
  }%
\fi
\begingroup\expandafter\expandafter\expandafter\endgroup
\expandafter\ifx\csname numexpr\endcsname\relax
  \let\pdftexcmds@temp=N%
\fi
%    \end{macrocode}
%    \end{macro}
%
%    \begin{macrocode}
\ifx\pdftexcmds@temp N%
  \@PackageWarningNoLine{pdftexcmds}{%
    Definitions of \ltx@backslashchar pdf@resettimer and%
    \MessageBreak
    \ltx@backslashchar pdf@elapsedtime are skipped, because%
    \MessageBreak
    e-TeX's \ltx@backslashchar protected or %
    \ltx@backslashchar numexpr are missing%
  }%
\else
%    \end{macrocode}
%
%    \begin{macro}{\pdf@resettimer}
%    \begin{macrocode}
  \protected\def\pdf@resettimer{%
    \pdftexcmds@directlua0{%
      oberdiek.pdftexcmds.resettimer()%
    }%
  }%
%    \end{macrocode}
%    \end{macro}
%
%    \begin{macro}{\pdf@elapsedtime}
%    \begin{macrocode}
  \protected\def\pdf@elapsedtime{%
    \numexpr
      \pdftexcmds@directlua0{%
        oberdiek.pdftexcmds.elapsedtime()%
      }%
    \relax
  }%
%    \end{macrocode}
%    \end{macro}
%    \begin{macrocode}
\fi
%    \end{macrocode}
%
% \subsubsection{Shell escape}
%
%    \begin{macro}{\pdf@shellescape}
%
%    \begin{macrocode}
\ifnum\luatexversion<68 %
\else
  \protected\edef\pdf@shellescape{%
   \numexpr\directlua{tex.sprint(%
         \number\catcodetable@string,status.shell_escape)}\relax}
\fi
%    \end{macrocode}
%    \end{macro}
%
%    \begin{macro}{\pdf@system}
%    \begin{macrocode}
\def\pdf@system#1{%
  \directlua0{%
    oberdiek.pdftexcmds.system("\luaescapestring{#1}")%
  }%
}
%    \end{macrocode}
%    \end{macro}
%
%    \begin{macro}{\pdf@lastsystemstatus}
%    \begin{macrocode}
\def\pdf@lastsystemstatus{%
  \directlua0{%
    oberdiek.pdftexcmds.lastsystemstatus()%
  }%
}
%    \end{macrocode}
%    \end{macro}
%    \begin{macro}{\pdf@lastsystemexit}
%    \begin{macrocode}
\def\pdf@lastsystemexit{%
  \directlua0{%
    oberdiek.pdftexcmds.lastsystemexit()%
  }%
}
%    \end{macrocode}
%    \end{macro}
%
%    \begin{macrocode}
\catcode`\0=12 %
%    \end{macrocode}
%
%    \begin{macro}{\pdf@pipe}
%    Check availability of |io.popen| first.
%    \begin{macrocode}
\ifnum0%
    \pdftexcmds@directlua{%
      if io.popen then %
        tex.write("1")%
      end%
    }%
    =1 %
  \def\pdf@pipe#1{%
&   \romannumeral\expandafter\pdftexcmds@PatchDecode
    \the\expandafter\pdftexcmds@toks
    \pdftexcmds@directlua{%
      oberdiek.pdftexcmds.toks="pdftexcmds@toks"%
      oberdiek.pdftexcmds.pipe("\luaescapestring{#1}", \pdftexcmds@Patch)%
    }%
&   \@nil
  }%
\fi
%    \end{macrocode}
%    \end{macro}
%
%    \begin{macrocode}
\pdftexcmds@AtEnd%
%</package>
%    \end{macrocode}
%
% \subsection{Lua module}
%
%    \begin{macrocode}
%<*lua>
%    \end{macrocode}
%
%    \begin{macrocode}
oberdiek = oberdiek or {}
local pdftexcmds = oberdiek.pdftexcmds or {}
oberdiek.pdftexcmds = pdftexcmds
local systemexitstatus
function pdftexcmds.getversion()
  tex.write("2019/07/25 v0.30")
end
%    \end{macrocode}
%
% \subsubsection[Strings]{Strings \cite[``7.15 Strings'']{pdftex-manual}}
%
%    \begin{macrocode}
function pdftexcmds.strcmp(A, B)
  if A == B then
    tex.write("0")
  elseif A < B then
    tex.write("-1")
  else
    tex.write("1")
  end
end
local function utf8_to_byte(str)
  local i = 0
  local n = string.len(str)
  local t = {}
  while i < n do
    i = i + 1
    local a = string.byte(str, i)
    if a < 128 then
      table.insert(t, string.char(a))
    else
      if a >= 192 and i < n then
        i = i + 1
        local b = string.byte(str, i)
        if b < 128 or b >= 192 then
          i = i - 1
        elseif a == 194 then
          table.insert(t, string.char(b))
        elseif a == 195 then
          table.insert(t, string.char(b + 64))
        end
      end
    end
  end
  return table.concat(t)
end
function pdftexcmds.escapehex(str, mode)
  if mode == "byte" then
    str = utf8_to_byte(str)
  end
  tex.write((string.gsub(str, ".",
    function (ch)
      return string.format("%02X", string.byte(ch))
    end
  )))
end
%    \end{macrocode}
%    See procedure |unescapehex| in file \xfile{utils.c} of \hologo{pdfTeX}.
%    Caution: |tex.write| ignores leading spaces.
%    \begin{macrocode}
function pdftexcmds.unescapehex(str, mode, patch)
  local a = 0
  local first = true
  local result = {}
  for i = 1, string.len(str), 1 do
    local ch = string.byte(str, i)
    if ch >= 48 and ch <= 57 then
      ch = ch - 48
    elseif ch >= 65 and ch <= 70 then
      ch = ch - 55
    elseif ch >= 97 and ch <= 102 then
      ch = ch - 87
    else
      ch = nil
    end
    if ch then
      if first then
        a = ch * 16
        first = false
      else
        table.insert(result, a + ch)
        first = true
      end
    end
  end
  if not first then
    table.insert(result, a)
  end
  if patch == 1 then
    local temp = {}
    for i, a in ipairs(result) do
      if a == 0 then
        table.insert(temp, 1)
        table.insert(temp, 1)
      else
        if a == 1 then
          table.insert(temp, 1)
          table.insert(temp, 2)
        else
          table.insert(temp, a)
        end
      end
    end
    result = temp
  end
  if mode == "byte" then
    local utf8 = {}
    for i, a in ipairs(result) do
      if a < 128 then
        table.insert(utf8, a)
      else
        if a < 192 then
          table.insert(utf8, 194)
          a = a - 128
        else
          table.insert(utf8, 195)
          a = a - 192
        end
        table.insert(utf8, a + 128)
      end
    end
    result = utf8
  end
%    \end{macrocode}
%    this next line added for current luatex; this is the only
%    change in the file.  eroux, 28apr13. (v 0.21)
%    \begin{macrocode}
  local unpack = _G["unpack"] or table.unpack
  tex.settoks(pdftexcmds.toks, string.char(unpack(result)))
end
%    \end{macrocode}
%    See procedure |escapestring| in file \xfile{utils.c} of \hologo{pdfTeX}.
%    \begin{macrocode}
function pdftexcmds.escapestring(str, mode)
  if mode == "byte" then
    str = utf8_to_byte(str)
  end
  tex.write((string.gsub(str, ".",
    function (ch)
      local b = string.byte(ch)
      if b < 33 or b > 126 then
        return string.format("\\%.3o", b)
      end
      if b == 40 or b == 41 or b == 92 then
        return "\\" .. ch
      end
%    \end{macrocode}
%    Lua 5.1 returns the match in case of return value |nil|.
%    \begin{macrocode}
      return nil
    end
  )))
end
%    \end{macrocode}
%    See procedure |escapename| in file \xfile{utils.c} of \hologo{pdfTeX}.
%    \begin{macrocode}
function pdftexcmds.escapename(str, mode)
  if mode == "byte" then
    str = utf8_to_byte(str)
  end
  tex.write((string.gsub(str, ".",
    function (ch)
      local b = string.byte(ch)
      if b == 0 then
%    \end{macrocode}
%    In Lua 5.0 |nil| could be used for the empty string,
%    But |nil| returns the match in Lua 5.1, thus we use
%    the empty string explicitly.
%    \begin{macrocode}
        return ""
      end
      if b <= 32 or b >= 127
          or b == 35 or b == 37 or b == 40 or b == 41
          or b == 47 or b == 60 or b == 62 or b == 91
          or b == 93 or b == 123 or b == 125 then
        return string.format("#%.2X", b)
      else
%    \end{macrocode}
%    Lua 5.1 returns the match in case of return value |nil|.
%    \begin{macrocode}
        return nil
      end
    end
  )))
end
%    \end{macrocode}
%
% \subsubsection[Files]{Files \cite[``7.18 Files'']{pdftex-manual}}
%
%    \begin{macrocode}
function pdftexcmds.filesize(filename)
  local foundfile = kpse.find_file(filename, "tex", true)
  if foundfile then
    local size = lfs.attributes(foundfile, "size")
    if size then
      tex.write(size)
    end
  end
end
%    \end{macrocode}
%    See procedure |makepdftime| in file \xfile{utils.c} of \hologo{pdfTeX}.
%    \begin{macrocode}
function pdftexcmds.filemoddate(filename)
  local foundfile = kpse.find_file(filename, "tex", true)
  if foundfile then
    local date = lfs.attributes(foundfile, "modification")
    if date then
      local d = os.date("*t", date)
      if d.sec >= 60 then
        d.sec = 59
      end
      local u = os.date("!*t", date)
      local off = 60 * (d.hour - u.hour) + d.min - u.min
      if d.year ~= u.year then
        if d.year > u.year then
          off = off + 1440
        else
          off = off - 1440
        end
      elseif d.yday ~= u.yday then
        if d.yday > u.yday then
          off = off + 1440
        else
          off = off - 1440
        end
      end
      local timezone
      if off == 0 then
        timezone = "Z"
      else
        local hours = math.floor(off / 60)
        local mins = math.abs(off - hours * 60)
        timezone = string.format("%+03d'%02d'", hours, mins)
      end
      tex.write(string.format("D:%04d%02d%02d%02d%02d%02d%s",
          d.year, d.month, d.day, d.hour, d.min, d.sec, timezone))
    end
  end
end
function pdftexcmds.filedump(offset, length, filename)
  length = tonumber(length)
  if length and length > 0 then
    local foundfile = kpse.find_file(filename, "tex", true)
    if foundfile then
      offset = tonumber(offset)
      if not offset then
        offset = 0
      end
      local filehandle = io.open(foundfile, "rb")
      if filehandle then
        if offset > 0 then
          filehandle:seek("set", offset)
        end
        local dump = filehandle:read(length)
        pdftexcmds.escapehex(dump)
        filehandle:close()
      end
    end
  end
end
function pdftexcmds.mdfivesum(str, mode)
  if mode == "byte" then
    str = utf8_to_byte(str)
  end
  pdftexcmds.escapehex(md5.sum(str))
end
function pdftexcmds.filemdfivesum(filename)
  local foundfile = kpse.find_file(filename, "tex", true)
  if foundfile then
    local filehandle = io.open(foundfile, "rb")
    if filehandle then
      local contents = filehandle:read("*a")
      pdftexcmds.escapehex(md5.sum(contents))
      filehandle:close()
    end
  end
end
%    \end{macrocode}
%
% \subsubsection[Timekeeping]{Timekeeping \cite[``7.17 Timekeeping'']{pdftex-manual}}
%
%    The functions for timekeeping are based on
%    Andy Thomas' work \cite{AndyThomas:Analog}.
%    Changes:
%    \begin{itemize}
%    \item Overflow check is added.
%    \item |string.format| is used to avoid exponential number
%          representation for sure.
%    \item |tex.write| is used instead of |tex.print| to get
%          tokens with catcode 12 and without appended \cs{endlinechar}.
%    \end{itemize}
%    \begin{macrocode}
local basetime = 0
function pdftexcmds.resettimer()
  basetime = os.clock()
end
function pdftexcmds.elapsedtime()
  local val = (os.clock() - basetime) * 65536 + .5
  if val > 2147483647 then
    val = 2147483647
  end
  tex.write(string.format("%d", val))
end
%    \end{macrocode}
%
% \subsubsection[Miscellaneous]{Miscellaneous \cite[``7.21 Miscellaneous'']{pdftex-manual}}
%
%    \begin{macrocode}
function pdftexcmds.shellescape()
  if os.execute then
    if status
        and status.luatex_version
        and status.luatex_version >= 68 then
      tex.write(os.execute())
    else
      local result = os.execute()
      if result == 0 then
        tex.write("0")
      else
        if result == nil then
          tex.write("0")
        else
          tex.write("1")
        end
      end
    end
  else
    tex.write("0")
  end
end
function pdftexcmds.system(cmdline)
  systemexitstatus = nil
  texio.write_nl("log", "system(" .. cmdline .. ") ")
  if os.execute then
    texio.write("log", "executed.")
    systemexitstatus = os.execute(cmdline)
  else
    texio.write("log", "disabled.")
  end
end
function pdftexcmds.lastsystemstatus()
  local result = tonumber(systemexitstatus)
  if result then
    local x = math.floor(result / 256)
    tex.write(result - 256 * math.floor(result / 256))
  end
end
function pdftexcmds.lastsystemexit()
  local result = tonumber(systemexitstatus)
  if result then
    tex.write(math.floor(result / 256))
  end
end
function pdftexcmds.pipe(cmdline, patch)
  local result
  systemexitstatus = nil
  texio.write_nl("log", "pipe(" .. cmdline ..") ")
  if io.popen then
    texio.write("log", "executed.")
    local handle = io.popen(cmdline, "r")
    if handle then
      result = handle:read("*a")
      handle:close()
    end
  else
    texio.write("log", "disabled.")
  end
  if result then
    if patch == 1 then
      local temp = {}
      for i, a in ipairs(result) do
        if a == 0 then
          table.insert(temp, 1)
          table.insert(temp, 1)
        else
          if a == 1 then
            table.insert(temp, 1)
            table.insert(temp, 2)
          else
            table.insert(temp, a)
          end
        end
      end
      result = temp
    end
    tex.settoks(pdftexcmds.toks, result)
  else
    tex.settoks(pdftexcmds.toks, "")
  end
end
%    \end{macrocode}
%    \begin{macrocode}
%</lua>
%    \end{macrocode}
%
% \section{Test}
%
% \subsection{Catcode checks for loading}
%
%    \begin{macrocode}
%<*test1>
%    \end{macrocode}
%    \begin{macrocode}
\catcode`\{=1 %
\catcode`\}=2 %
\catcode`\#=6 %
\catcode`\@=11 %
\expandafter\ifx\csname count@\endcsname\relax
  \countdef\count@=255 %
\fi
\expandafter\ifx\csname @gobble\endcsname\relax
  \long\def\@gobble#1{}%
\fi
\expandafter\ifx\csname @firstofone\endcsname\relax
  \long\def\@firstofone#1{#1}%
\fi
\expandafter\ifx\csname loop\endcsname\relax
  \expandafter\@firstofone
\else
  \expandafter\@gobble
\fi
{%
  \def\loop#1\repeat{%
    \def\body{#1}%
    \iterate
  }%
  \def\iterate{%
    \body
      \let\next\iterate
    \else
      \let\next\relax
    \fi
    \next
  }%
  \let\repeat=\fi
}%
\def\RestoreCatcodes{}
\count@=0 %
\loop
  \edef\RestoreCatcodes{%
    \RestoreCatcodes
    \catcode\the\count@=\the\catcode\count@\relax
  }%
\ifnum\count@<255 %
  \advance\count@ 1 %
\repeat

\def\RangeCatcodeInvalid#1#2{%
  \count@=#1\relax
  \loop
    \catcode\count@=15 %
  \ifnum\count@<#2\relax
    \advance\count@ 1 %
  \repeat
}
\def\RangeCatcodeCheck#1#2#3{%
  \count@=#1\relax
  \loop
    \ifnum#3=\catcode\count@
    \else
      \errmessage{%
        Character \the\count@\space
        with wrong catcode \the\catcode\count@\space
        instead of \number#3%
      }%
    \fi
  \ifnum\count@<#2\relax
    \advance\count@ 1 %
  \repeat
}
\def\space{ }
\expandafter\ifx\csname LoadCommand\endcsname\relax
  \def\LoadCommand{\input pdftexcmds.sty\relax}%
\fi
\def\Test{%
  \RangeCatcodeInvalid{0}{47}%
  \RangeCatcodeInvalid{58}{64}%
  \RangeCatcodeInvalid{91}{96}%
  \RangeCatcodeInvalid{123}{255}%
  \catcode`\@=12 %
  \catcode`\\=0 %
  \catcode`\%=14 %
  \LoadCommand
  \RangeCatcodeCheck{0}{36}{15}%
  \RangeCatcodeCheck{37}{37}{14}%
  \RangeCatcodeCheck{38}{47}{15}%
  \RangeCatcodeCheck{48}{57}{12}%
  \RangeCatcodeCheck{58}{63}{15}%
  \RangeCatcodeCheck{64}{64}{12}%
  \RangeCatcodeCheck{65}{90}{11}%
  \RangeCatcodeCheck{91}{91}{15}%
  \RangeCatcodeCheck{92}{92}{0}%
  \RangeCatcodeCheck{93}{96}{15}%
  \RangeCatcodeCheck{97}{122}{11}%
  \RangeCatcodeCheck{123}{255}{15}%
  \RestoreCatcodes
}
\Test
\csname @@end\endcsname
\end
%    \end{macrocode}
%    \begin{macrocode}
%</test1>
%    \end{macrocode}
%
% \subsection{Test for \cs{pdf@isprimitive}}
%
%    \begin{macrocode}
%<*test2>
\catcode`\{=1 %
\catcode`\}=2 %
\catcode`\#=6 %
\catcode`\@=11 %
\input pdftexcmds.sty\relax
\def\msg#1{%
  \begingroup
    \escapechar=92 %
    \immediate\write16{#1}%
  \endgroup
}
\long\def\test#1#2#3#4{%
  \begingroup
    #4%
    \def\str{%
      Test \string\pdf@isprimitive
      {\string #1}{\string #2}{...}: %
    }%
    \pdf@isprimitive{#1}{#2}{%
      \ifx#3Y%
        \msg{\str true ==> OK.}%
      \else
        \errmessage{\str false ==> FAILED}%
      \fi
    }{%
      \ifx#3Y%
        \errmessage{\str true ==> FAILED}%
      \else
        \msg{\str false ==> OK.}%
      \fi
    }%
  \endgroup
}
\test\relax\relax Y{}
\test\foobar\relax Y{\let\foobar\relax}
\test\foobar\relax N{}
\test\hbox\hbox Y{}
\test\foobar@hbox\hbox Y{\let\foobar@hbox\hbox}
\test\if\if Y{}
\test\if\ifx N{}
\test\ifx\if N{}
\test\par\par Y{}
\test\hbox\par N{}
\test\par\hbox N{}
\csname @@end\endcsname\end
%</test2>
%    \end{macrocode}
%
% \subsection{Test for \cs{pdf@shellescape}}
%
%    \begin{macrocode}
%<*test-shell>
\catcode`\{=1 %
\catcode`\}=2 %
\catcode`\#=6 %
\catcode`\@=11 %
\input pdftexcmds.sty\relax
\def\msg#{\immediate\write16}
\def\MaybeEnd{}
\ifx\luatexversion\UnDeFiNeD
\else
  \ifnum\luatexversion<68 %
    \ifx\pdf@shellescape\@undefined
      \msg{SHELL=U}%
      \msg{OK (LuaTeX < 0.68)}%
    \else
      \msg{SHELL=defined}%
      \errmessage{Failed (LuaTeX < 0.68)}%
    \fi
    \def\MaybeEnd{\csname @@end\endcsname\end}%
  \fi
\fi
\MaybeEnd
\ifx\pdf@shellescape\@undefined
  \msg{SHELL=U}%
\else
  \msg{SHELL=\number\pdf@shellescape}%
\fi
\ifx\expected\@undefined
\else
  \ifx\expected\relax
    \msg{EXPECTED=U}%
    \ifx\pdf@shellescape\@undefined
      \msg{OK}%
    \else
      \errmessage{Failed}%
    \fi
  \else
    \msg{EXPECTED=\number\expected}%
    \ifnum\pdf@shellescape=\expected\relax
      \msg{OK}%
    \else
      \errmessage{Failed}%
    \fi
  \fi
\fi
\csname @@end\endcsname\end
%</test-shell>
%    \end{macrocode}
%
% \subsection{Test for escape functions}
%
%    \begin{macrocode}
%<*test-escape>
\catcode`\{=1 %
\catcode`\}=2 %
\catcode`\#=6 %
\catcode`\^=7 %
\catcode`\@=11 %
\errorcontextlines=1000 %
\input pdftexcmds.sty\relax
\def\msg#1{%
  \begingroup
    \escapechar=92 %
    \immediate\write16{#1}%
  \endgroup
}
%    \end{macrocode}
%    \begin{macrocode}
\begingroup
  \catcode`\@=11 %
  \countdef\count@=255 %
  \def\space{ }%
  \long\def\@whilenum#1\do #2{%
    \ifnum #1\relax
      #2\relax
      \@iwhilenum{#1\relax#2\relax}%
    \fi
  }%
  \long\def\@iwhilenum#1{%
    \ifnum #1%
      \expandafter\@iwhilenum
    \else
      \expandafter\ltx@gobble
    \fi
    {#1}%
  }%
  \gdef\AllBytes{}%
  \count@=0 %
  \catcode0=12 %
  \@whilenum\count@<256 \do{%
    \lccode0=\count@
    \ifnum\count@=32 %
      \xdef\AllBytes{\AllBytes\space}%
    \else
      \lowercase{%
        \xdef\AllBytes{\AllBytes^^@}%
      }%
    \fi
    \advance\count@ by 1 %
  }%
\endgroup
%    \end{macrocode}
%    \begin{macrocode}
\def\AllBytesHex{%
  000102030405060708090A0B0C0D0E0F%
  101112131415161718191A1B1C1D1E1F%
  202122232425262728292A2B2C2D2E2F%
  303132333435363738393A3B3C3D3E3F%
  404142434445464748494A4B4C4D4E4F%
  505152535455565758595A5B5C5D5E5F%
  606162636465666768696A6B6C6D6E6F%
  707172737475767778797A7B7C7D7E7F%
  808182838485868788898A8B8C8D8E8F%
  909192939495969798999A9B9C9D9E9F%
  A0A1A2A3A4A5A6A7A8A9AAABACADAEAF%
  B0B1B2B3B4B5B6B7B8B9BABBBCBDBEBF%
  C0C1C2C3C4C5C6C7C8C9CACBCCCDCECF%
  D0D1D2D3D4D5D6D7D8D9DADBDCDDDEDF%
  E0E1E2E3E4E5E6E7E8E9EAEBECEDEEEF%
  F0F1F2F3F4F5F6F7F8F9FAFBFCFDFEFF%
}
\ltx@onelevel@sanitize\AllBytesHex
\expandafter\lowercase\expandafter{%
  \expandafter\def\expandafter\AllBytesHexLC
      \expandafter{\AllBytesHex}%
}
\begingroup
  \catcode`\#=12 %
  \xdef\AllBytesName{%
    #01#02#03#04#05#06#07#08#09#0A#0B#0C#0D#0E#0F%
    #10#11#12#13#14#15#16#17#18#19#1A#1B#1C#1D#1E#1F%
    #20!"#23$#25&'#28#29*+,-.#2F%
    0123456789:;#3C=#3E?%
    @ABCDEFGHIJKLMNO%
    PQRSTUVWXYZ#5B\ltx@backslashchar#5D^_%
    `abcdefghijklmno%
    pqrstuvwxyz#7B|#7D\string~#7F%
    #80#81#82#83#84#85#86#87#88#89#8A#8B#8C#8D#8E#8F%
    #90#91#92#93#94#95#96#97#98#99#9A#9B#9C#9D#9E#9F%
    #A0#A1#A2#A3#A4#A5#A6#A7#A8#A9#AA#AB#AC#AD#AE#AF%
    #B0#B1#B2#B3#B4#B5#B6#B7#B8#B9#BA#BB#BC#BD#BE#BF%
    #C0#C1#C2#C3#C4#C5#C6#C7#C8#C9#CA#CB#CC#CD#CE#CF%
    #D0#D1#D2#D3#D4#D5#D6#D7#D8#D9#DA#DB#DC#DD#DE#DF%
    #E0#E1#E2#E3#E4#E5#E6#E7#E8#E9#EA#EB#EC#ED#EE#EF%
    #F0#F1#F2#F3#F4#F5#F6#F7#F8#F9#FA#FB#FC#FD#FE#FF%
  }%
\endgroup
\ltx@onelevel@sanitize\AllBytesName
\edef\AllBytesFromName{\expandafter\ltx@gobble\AllBytes}
\begingroup
  \def\|{|}%
  \edef\%{\ltx@percentchar}%
  \catcode`\|=0 %
  \catcode`\#=12 %
  \catcode`\~=12 %
  \catcode`\\=12 %
  |xdef|AllBytesString{%
    \000\001\002\003\004\005\006\007\010\011\012\013\014\015\016\017%
    \020\021\022\023\024\025\026\027\030\031\032\033\034\035\036\037%
    \040!"#$|%&'\(\)*+,-./%
    0123456789:;<=>?%
    @ABCDEFGHIJKLMNO%
    PQRSTUVWXYZ[\\]^_%
    `abcdefghijklmno%
    pqrstuvwxyz{||}~\177%
    \200\201\202\203\204\205\206\207\210\211\212\213\214\215\216\217%
    \220\221\222\223\224\225\226\227\230\231\232\233\234\235\236\237%
    \240\241\242\243\244\245\246\247\250\251\252\253\254\255\256\257%
    \260\261\262\263\264\265\266\267\270\271\272\273\274\275\276\277%
    \300\301\302\303\304\305\306\307\310\311\312\313\314\315\316\317%
    \320\321\322\323\324\325\326\327\330\331\332\333\334\335\336\337%
    \340\341\342\343\344\345\346\347\350\351\352\353\354\355\356\357%
    \360\361\362\363\364\365\366\367\370\371\372\373\374\375\376\377%
  }%
|endgroup
\ltx@onelevel@sanitize\AllBytesString
%    \end{macrocode}
%    \begin{macrocode}
\def\Test#1#2#3{%
  \begingroup
    \expandafter\expandafter\expandafter\def
    \expandafter\expandafter\expandafter\TestResult
    \expandafter\expandafter\expandafter{%
      #1{#2}%
    }%
    \ifx\TestResult#3%
    \else
      \newlinechar=10 %
      \msg{Expect:^^J#3}%
      \msg{Result:^^J\TestResult}%
      \errmessage{\string#2 -\string#1-> \string#3}%
    \fi
  \endgroup
}
\def\test#1#2#3{%
  \edef\TestFrom{#2}%
  \edef\TestExpect{#3}%
  \ltx@onelevel@sanitize\TestExpect
  \Test#1\TestFrom\TestExpect
}
\test\pdf@unescapehex{74657374}{test}
\begingroup
  \catcode0=12 %
  \catcode1=12 %
  \test\pdf@unescapehex{740074017400740174}{t^^@t^^At^^@t^^At}%
\endgroup
\Test\pdf@escapehex\AllBytes\AllBytesHex
\Test\pdf@unescapehex\AllBytesHex\AllBytes
\Test\pdf@escapename\AllBytes\AllBytesName
\Test\pdf@escapestring\AllBytes\AllBytesString
%    \end{macrocode}
%    \begin{macrocode}
\csname @@end\endcsname\end
%</test-escape>
%    \end{macrocode}
%
% \section{Installation}
%
% \subsection{Download}
%
% \paragraph{Package.} This package is available on
% CTAN\footnote{\CTANpkg{pdftexcmds}}:
% \begin{description}
% \item[\CTAN{macros/latex/contrib/oberdiek/pdftexcmds.dtx}] The source file.
% \item[\CTAN{macros/latex/contrib/oberdiek/pdftexcmds.pdf}] Documentation.
% \end{description}
%
%
% \paragraph{Bundle.} All the packages of the bundle `oberdiek'
% are also available in a TDS compliant ZIP archive. There
% the packages are already unpacked and the documentation files
% are generated. The files and directories obey the TDS standard.
% \begin{description}
% \item[\CTANinstall{install/macros/latex/contrib/oberdiek.tds.zip}]
% \end{description}
% \emph{TDS} refers to the standard ``A Directory Structure
% for \TeX\ Files'' (\CTAN{tds/tds.pdf}). Directories
% with \xfile{texmf} in their name are usually organized this way.
%
% \subsection{Bundle installation}
%
% \paragraph{Unpacking.} Unpack the \xfile{oberdiek.tds.zip} in the
% TDS tree (also known as \xfile{texmf} tree) of your choice.
% Example (linux):
% \begin{quote}
%   |unzip oberdiek.tds.zip -d ~/texmf|
% \end{quote}
%
% \paragraph{Script installation.}
% Check the directory \xfile{TDS:scripts/oberdiek/} for
% scripts that need further installation steps.
% Package \xpackage{attachfile2} comes with the Perl script
% \xfile{pdfatfi.pl} that should be installed in such a way
% that it can be called as \texttt{pdfatfi}.
% Example (linux):
% \begin{quote}
%   |chmod +x scripts/oberdiek/pdfatfi.pl|\\
%   |cp scripts/oberdiek/pdfatfi.pl /usr/local/bin/|
% \end{quote}
%
% \subsection{Package installation}
%
% \paragraph{Unpacking.} The \xfile{.dtx} file is a self-extracting
% \docstrip\ archive. The files are extracted by running the
% \xfile{.dtx} through \plainTeX:
% \begin{quote}
%   \verb|tex pdftexcmds.dtx|
% \end{quote}
%
% \paragraph{TDS.} Now the different files must be moved into
% the different directories in your installation TDS tree
% (also known as \xfile{texmf} tree):
% \begin{quote}
% \def\t{^^A
% \begin{tabular}{@{}>{\ttfamily}l@{ $\rightarrow$ }>{\ttfamily}l@{}}
%   pdftexcmds.sty & tex/generic/oberdiek/pdftexcmds.sty\\
%   oberdiek.pdftexcmds.lua & scripts/oberdiek/oberdiek.pdftexcmds.lua\\
%   pdftexcmds.lua & scripts/oberdiek/pdftexcmds.lua\\
%   pdftexcmds.pdf & doc/latex/oberdiek/pdftexcmds.pdf\\
%   test/pdftexcmds-test1.tex & doc/latex/oberdiek/test/pdftexcmds-test1.tex\\
%   test/pdftexcmds-test2.tex & doc/latex/oberdiek/test/pdftexcmds-test2.tex\\
%   test/pdftexcmds-test-shell.tex & doc/latex/oberdiek/test/pdftexcmds-test-shell.tex\\
%   test/pdftexcmds-test-escape.tex & doc/latex/oberdiek/test/pdftexcmds-test-escape.tex\\
%   pdftexcmds.dtx & source/latex/oberdiek/pdftexcmds.dtx\\
% \end{tabular}^^A
% }^^A
% \sbox0{\t}^^A
% \ifdim\wd0>\linewidth
%   \begingroup
%     \advance\linewidth by\leftmargin
%     \advance\linewidth by\rightmargin
%   \edef\x{\endgroup
%     \def\noexpand\lw{\the\linewidth}^^A
%   }\x
%   \def\lwbox{^^A
%     \leavevmode
%     \hbox to \linewidth{^^A
%       \kern-\leftmargin\relax
%       \hss
%       \usebox0
%       \hss
%       \kern-\rightmargin\relax
%     }^^A
%   }^^A
%   \ifdim\wd0>\lw
%     \sbox0{\small\t}^^A
%     \ifdim\wd0>\linewidth
%       \ifdim\wd0>\lw
%         \sbox0{\footnotesize\t}^^A
%         \ifdim\wd0>\linewidth
%           \ifdim\wd0>\lw
%             \sbox0{\scriptsize\t}^^A
%             \ifdim\wd0>\linewidth
%               \ifdim\wd0>\lw
%                 \sbox0{\tiny\t}^^A
%                 \ifdim\wd0>\linewidth
%                   \lwbox
%                 \else
%                   \usebox0
%                 \fi
%               \else
%                 \lwbox
%               \fi
%             \else
%               \usebox0
%             \fi
%           \else
%             \lwbox
%           \fi
%         \else
%           \usebox0
%         \fi
%       \else
%         \lwbox
%       \fi
%     \else
%       \usebox0
%     \fi
%   \else
%     \lwbox
%   \fi
% \else
%   \usebox0
% \fi
% \end{quote}
% If you have a \xfile{docstrip.cfg} that configures and enables \docstrip's
% TDS installing feature, then some files can already be in the right
% place, see the documentation of \docstrip.
%
% \subsection{Refresh file name databases}
%
% If your \TeX~distribution
% (\teTeX, \mikTeX, \dots) relies on file name databases, you must refresh
% these. For example, \teTeX\ users run \verb|texhash| or
% \verb|mktexlsr|.
%
% \subsection{Some details for the interested}
%
% \paragraph{Unpacking with \LaTeX.}
% The \xfile{.dtx} chooses its action depending on the format:
% \begin{description}
% \item[\plainTeX:] Run \docstrip\ and extract the files.
% \item[\LaTeX:] Generate the documentation.
% \end{description}
% If you insist on using \LaTeX\ for \docstrip\ (really,
% \docstrip\ does not need \LaTeX), then inform the autodetect routine
% about your intention:
% \begin{quote}
%   \verb|latex \let\install=y\input{pdftexcmds.dtx}|
% \end{quote}
% Do not forget to quote the argument according to the demands
% of your shell.
%
% \paragraph{Generating the documentation.}
% You can use both the \xfile{.dtx} or the \xfile{.drv} to generate
% the documentation. The process can be configured by the
% configuration file \xfile{ltxdoc.cfg}. For instance, put this
% line into this file, if you want to have A4 as paper format:
% \begin{quote}
%   \verb|\PassOptionsToClass{a4paper}{article}|
% \end{quote}
% An example follows how to generate the
% documentation with pdf\LaTeX:
% \begin{quote}
%\begin{verbatim}
%pdflatex pdftexcmds.dtx
%bibtex pdftexcmds.aux
%makeindex -s gind.ist pdftexcmds.idx
%pdflatex pdftexcmds.dtx
%makeindex -s gind.ist pdftexcmds.idx
%pdflatex pdftexcmds.dtx
%\end{verbatim}
% \end{quote}
%
% \printbibliography[
%   heading=bibnumbered,
% ]
%
% \begin{History}
%   \begin{Version}{2007/11/11 v0.1}
%   \item
%     First version.
%   \end{Version}
%   \begin{Version}{2007/11/12 v0.2}
%   \item
%     Short description fixed.
%   \end{Version}
%   \begin{Version}{2007/12/12 v0.3}
%   \item
%     Organization of Lua code as module.
%   \end{Version}
%   \begin{Version}{2009/04/10 v0.4}
%   \item
%     Adaptation for syntax change of \cs{directlua} in
%     \hologo{LuaTeX} 0.36.
%   \end{Version}
%   \begin{Version}{2009/09/22 v0.5}
%   \item
%     \cs{pdf@primitive}, \cs{pdf@ifprimitive} added.
%   \item
%     \hologo{XeTeX}'s variants are detected for
%     \cs{pdf@shellescape}, \cs{pdf@strcmp}, \cs{pdf@primitive},
%     \cs{pdf@ifprimitive}.
%   \end{Version}
%   \begin{Version}{2009/09/23 v0.6}
%   \item
%     Macro \cs{pdf@isprimitive} added.
%   \end{Version}
%   \begin{Version}{2009/12/12 v0.7}
%   \item
%     Short info shortened.
%   \end{Version}
%   \begin{Version}{2010/03/01 v0.8}
%   \item
%     Required date for package \xpackage{ifluatex} updated.
%   \end{Version}
%   \begin{Version}{2010/04/01 v0.9}
%   \item
%     Use \cs{ifeof18} for defining \cs{pdf@shellescape} between
%     \hologo{pdfTeX} 1.21a (inclusive) and 1.30.0 (exclusive).
%   \end{Version}
%   \begin{Version}{2010/11/04 v0.10}
%   \item
%     \cs{pdf@draftmode}, \cs{pdf@ifdraftmode} and
%     \cs{pdf@setdraftmode} added.
%   \end{Version}
%   \begin{Version}{2010/11/11 v0.11}
%   \item
%     Missing \cs{RequirePackage} for package \xpackage{ifpdf} added.
%   \end{Version}
%   \begin{Version}{2011/01/30 v0.12}
%   \item
%     Already loaded package files are not input in \hologo{plainTeX}.
%   \end{Version}
%   \begin{Version}{2011/03/04 v0.13}
%   \item
%     Improved Lua function \texttt{shellescape} that also
%     uses the result of \texttt{os.execute()} (thanks to Philipp Stephani).
%   \end{Version}
%   \begin{Version}{2011/04/10 v0.14}
%   \item
%     Version check of loaded module added.
%   \item
%     Patch for bug in \hologo{LuaTeX} between 0.40.6 and 0.65 that
%     is fixed in revision 4096.
%   \end{Version}
%   \begin{Version}{2011/04/16 v0.15}
%   \item
%     \hologo{LuaTeX}: \cs{pdf@shellescape} is only supported
%     for version 0.70.0 and higher due to a bug, \texttt{os.execute()}
%     crashes in some circumstances. Fixed in \hologo{LuaTeX}
%     beta-0.70.0, revision 4167.
%   \end{Version}
%   \begin{Version}{2011/04/22 v0.16}
%   \item
%     Previous fix was not working due to a wrong catcode of digit
%     zero (due to easily support the old \cs{directlua0}).
%     The version border is lowered to 0.68, because some
%     beta-0.67.0 seems also to work.
%   \end{Version}
%   \begin{Version}{2011/06/29 v0.17}
%   \item
%     Documentation addition to \cs{pdf@shellescape}.
%   \end{Version}
%   \begin{Version}{2011/07/01 v0.18}
%   \item
%     Add Lua module loading in \cs{everyjob} for \hologo{iniTeX}
%     (\hologo{LuaTeX} only).
%   \end{Version}
%   \begin{Version}{2011/07/28 v0.19}
%   \item
%     Missing space in an info message added (Martin M\"unch).
%   \end{Version}
%   \begin{Version}{2011/11/29 v0.20}
%   \item
%     \cs{pdf@resettimer} and \cs{pdf@elapsedtime} added
%     (thanks Andy Thomas).
%   \end{Version}
%   \begin{Version}{2016/05/10 v0.21}
%   \item
%      local unpack added
%     (thanks \'{E}lie Roux).
%   \end{Version}
%   \begin{Version}{2016/05/21 v0.22}
%   \item
%     adjust \cs{textbackslas}h usage in bib file for biber bug.
%   \end{Version}
%   \begin{Version}{2016/10/02 v0.23}
%   \item
%     add file.close to lua filehandles (github pull request).
%   \end{Version}
%   \begin{Version}{2017/01/29 v0.24}
%   \item
%     Avoid loading luatex-loader for current luatex. (Use
%     pdftexcmds.lua not oberdiek.pdftexcmds.lua to simplify file
%     search with standard require)
%   \end{Version}
%   \begin{Version}{2017/03/19 v0.25}
%   \item
%     New \cs{pdf@shellescape} for Lua\TeX, see github issue 20.
%   \end{Version}
%   \begin{Version}{2018/01/21 v0.26}
%   \item
%     use rb not r mode for file open github issue 34.
%   \end{Version}
%   \begin{Version}{2018/01/30 v0.27}
%   \item
%     \cs{pdf@mdfivesum} for \hologo{XeTeX}
%   \end{Version}
%   \begin{Version}{2018/09/07 v0.28}
%   \item
%     Fix catcode regime in luatex sprint for \cs{pdf@shellescape} GH issue 45
%   \end{Version}
%   \begin{Version}{2018/09/10 v0.29}
%   \item
%     Actually do the fix described above in the code, not just document it.
%   \end{Version}
%   \begin{Version}{2019/07/25 v0.30}
%   \item
%     remove uses of module function, see PR70
%   \end{Version}
% \end{History}
%
% \PrintIndex
%
% \Finale
\endinput
|
% \end{quote}
% Do not forget to quote the argument according to the demands
% of your shell.
%
% \paragraph{Generating the documentation.}
% You can use both the \xfile{.dtx} or the \xfile{.drv} to generate
% the documentation. The process can be configured by the
% configuration file \xfile{ltxdoc.cfg}. For instance, put this
% line into this file, if you want to have A4 as paper format:
% \begin{quote}
%   \verb|\PassOptionsToClass{a4paper}{article}|
% \end{quote}
% An example follows how to generate the
% documentation with pdf\LaTeX:
% \begin{quote}
%\begin{verbatim}
%pdflatex pdftexcmds.dtx
%bibtex pdftexcmds.aux
%makeindex -s gind.ist pdftexcmds.idx
%pdflatex pdftexcmds.dtx
%makeindex -s gind.ist pdftexcmds.idx
%pdflatex pdftexcmds.dtx
%\end{verbatim}
% \end{quote}
%
% \printbibliography[
%   heading=bibnumbered,
% ]
%
% \begin{History}
%   \begin{Version}{2007/11/11 v0.1}
%   \item
%     First version.
%   \end{Version}
%   \begin{Version}{2007/11/12 v0.2}
%   \item
%     Short description fixed.
%   \end{Version}
%   \begin{Version}{2007/12/12 v0.3}
%   \item
%     Organization of Lua code as module.
%   \end{Version}
%   \begin{Version}{2009/04/10 v0.4}
%   \item
%     Adaptation for syntax change of \cs{directlua} in
%     \hologo{LuaTeX} 0.36.
%   \end{Version}
%   \begin{Version}{2009/09/22 v0.5}
%   \item
%     \cs{pdf@primitive}, \cs{pdf@ifprimitive} added.
%   \item
%     \hologo{XeTeX}'s variants are detected for
%     \cs{pdf@shellescape}, \cs{pdf@strcmp}, \cs{pdf@primitive},
%     \cs{pdf@ifprimitive}.
%   \end{Version}
%   \begin{Version}{2009/09/23 v0.6}
%   \item
%     Macro \cs{pdf@isprimitive} added.
%   \end{Version}
%   \begin{Version}{2009/12/12 v0.7}
%   \item
%     Short info shortened.
%   \end{Version}
%   \begin{Version}{2010/03/01 v0.8}
%   \item
%     Required date for package \xpackage{ifluatex} updated.
%   \end{Version}
%   \begin{Version}{2010/04/01 v0.9}
%   \item
%     Use \cs{ifeof18} for defining \cs{pdf@shellescape} between
%     \hologo{pdfTeX} 1.21a (inclusive) and 1.30.0 (exclusive).
%   \end{Version}
%   \begin{Version}{2010/11/04 v0.10}
%   \item
%     \cs{pdf@draftmode}, \cs{pdf@ifdraftmode} and
%     \cs{pdf@setdraftmode} added.
%   \end{Version}
%   \begin{Version}{2010/11/11 v0.11}
%   \item
%     Missing \cs{RequirePackage} for package \xpackage{ifpdf} added.
%   \end{Version}
%   \begin{Version}{2011/01/30 v0.12}
%   \item
%     Already loaded package files are not input in \hologo{plainTeX}.
%   \end{Version}
%   \begin{Version}{2011/03/04 v0.13}
%   \item
%     Improved Lua function \texttt{shellescape} that also
%     uses the result of \texttt{os.execute()} (thanks to Philipp Stephani).
%   \end{Version}
%   \begin{Version}{2011/04/10 v0.14}
%   \item
%     Version check of loaded module added.
%   \item
%     Patch for bug in \hologo{LuaTeX} between 0.40.6 and 0.65 that
%     is fixed in revision 4096.
%   \end{Version}
%   \begin{Version}{2011/04/16 v0.15}
%   \item
%     \hologo{LuaTeX}: \cs{pdf@shellescape} is only supported
%     for version 0.70.0 and higher due to a bug, \texttt{os.execute()}
%     crashes in some circumstances. Fixed in \hologo{LuaTeX}
%     beta-0.70.0, revision 4167.
%   \end{Version}
%   \begin{Version}{2011/04/22 v0.16}
%   \item
%     Previous fix was not working due to a wrong catcode of digit
%     zero (due to easily support the old \cs{directlua0}).
%     The version border is lowered to 0.68, because some
%     beta-0.67.0 seems also to work.
%   \end{Version}
%   \begin{Version}{2011/06/29 v0.17}
%   \item
%     Documentation addition to \cs{pdf@shellescape}.
%   \end{Version}
%   \begin{Version}{2011/07/01 v0.18}
%   \item
%     Add Lua module loading in \cs{everyjob} for \hologo{iniTeX}
%     (\hologo{LuaTeX} only).
%   \end{Version}
%   \begin{Version}{2011/07/28 v0.19}
%   \item
%     Missing space in an info message added (Martin M\"unch).
%   \end{Version}
%   \begin{Version}{2011/11/29 v0.20}
%   \item
%     \cs{pdf@resettimer} and \cs{pdf@elapsedtime} added
%     (thanks Andy Thomas).
%   \end{Version}
%   \begin{Version}{2016/05/10 v0.21}
%   \item
%      local unpack added
%     (thanks \'{E}lie Roux).
%   \end{Version}
%   \begin{Version}{2016/05/21 v0.22}
%   \item
%     adjust \cs{textbackslas}h usage in bib file for biber bug.
%   \end{Version}
%   \begin{Version}{2016/10/02 v0.23}
%   \item
%     add file.close to lua filehandles (github pull request).
%   \end{Version}
%   \begin{Version}{2017/01/29 v0.24}
%   \item
%     Avoid loading luatex-loader for current luatex. (Use
%     pdftexcmds.lua not oberdiek.pdftexcmds.lua to simplify file
%     search with standard require)
%   \end{Version}
%   \begin{Version}{2017/03/19 v0.25}
%   \item
%     New \cs{pdf@shellescape} for Lua\TeX, see github issue 20.
%   \end{Version}
%   \begin{Version}{2018/01/21 v0.26}
%   \item
%     use rb not r mode for file open github issue 34.
%   \end{Version}
%   \begin{Version}{2018/01/30 v0.27}
%   \item
%     \cs{pdf@mdfivesum} for \hologo{XeTeX}
%   \end{Version}
%   \begin{Version}{2018/09/07 v0.28}
%   \item
%     Fix catcode regime in luatex sprint for \cs{pdf@shellescape} GH issue 45
%   \end{Version}
%   \begin{Version}{2018/09/10 v0.29}
%   \item
%     Actually do the fix described above in the code, not just document it.
%   \end{Version}
%   \begin{Version}{2019/07/25 v0.30}
%   \item
%     remove uses of module function, see PR70
%   \end{Version}
% \end{History}
%
% \PrintIndex
%
% \Finale
\endinput
|
% \end{quote}
% Do not forget to quote the argument according to the demands
% of your shell.
%
% \paragraph{Generating the documentation.}
% You can use both the \xfile{.dtx} or the \xfile{.drv} to generate
% the documentation. The process can be configured by the
% configuration file \xfile{ltxdoc.cfg}. For instance, put this
% line into this file, if you want to have A4 as paper format:
% \begin{quote}
%   \verb|\PassOptionsToClass{a4paper}{article}|
% \end{quote}
% An example follows how to generate the
% documentation with pdf\LaTeX:
% \begin{quote}
%\begin{verbatim}
%pdflatex pdftexcmds.dtx
%bibtex pdftexcmds.aux
%makeindex -s gind.ist pdftexcmds.idx
%pdflatex pdftexcmds.dtx
%makeindex -s gind.ist pdftexcmds.idx
%pdflatex pdftexcmds.dtx
%\end{verbatim}
% \end{quote}
%
% \printbibliography[
%   heading=bibnumbered,
% ]
%
% \begin{History}
%   \begin{Version}{2007/11/11 v0.1}
%   \item
%     First version.
%   \end{Version}
%   \begin{Version}{2007/11/12 v0.2}
%   \item
%     Short description fixed.
%   \end{Version}
%   \begin{Version}{2007/12/12 v0.3}
%   \item
%     Organization of Lua code as module.
%   \end{Version}
%   \begin{Version}{2009/04/10 v0.4}
%   \item
%     Adaptation for syntax change of \cs{directlua} in
%     \hologo{LuaTeX} 0.36.
%   \end{Version}
%   \begin{Version}{2009/09/22 v0.5}
%   \item
%     \cs{pdf@primitive}, \cs{pdf@ifprimitive} added.
%   \item
%     \hologo{XeTeX}'s variants are detected for
%     \cs{pdf@shellescape}, \cs{pdf@strcmp}, \cs{pdf@primitive},
%     \cs{pdf@ifprimitive}.
%   \end{Version}
%   \begin{Version}{2009/09/23 v0.6}
%   \item
%     Macro \cs{pdf@isprimitive} added.
%   \end{Version}
%   \begin{Version}{2009/12/12 v0.7}
%   \item
%     Short info shortened.
%   \end{Version}
%   \begin{Version}{2010/03/01 v0.8}
%   \item
%     Required date for package \xpackage{ifluatex} updated.
%   \end{Version}
%   \begin{Version}{2010/04/01 v0.9}
%   \item
%     Use \cs{ifeof18} for defining \cs{pdf@shellescape} between
%     \hologo{pdfTeX} 1.21a (inclusive) and 1.30.0 (exclusive).
%   \end{Version}
%   \begin{Version}{2010/11/04 v0.10}
%   \item
%     \cs{pdf@draftmode}, \cs{pdf@ifdraftmode} and
%     \cs{pdf@setdraftmode} added.
%   \end{Version}
%   \begin{Version}{2010/11/11 v0.11}
%   \item
%     Missing \cs{RequirePackage} for package \xpackage{ifpdf} added.
%   \end{Version}
%   \begin{Version}{2011/01/30 v0.12}
%   \item
%     Already loaded package files are not input in \hologo{plainTeX}.
%   \end{Version}
%   \begin{Version}{2011/03/04 v0.13}
%   \item
%     Improved Lua function \texttt{shellescape} that also
%     uses the result of \texttt{os.execute()} (thanks to Philipp Stephani).
%   \end{Version}
%   \begin{Version}{2011/04/10 v0.14}
%   \item
%     Version check of loaded module added.
%   \item
%     Patch for bug in \hologo{LuaTeX} between 0.40.6 and 0.65 that
%     is fixed in revision 4096.
%   \end{Version}
%   \begin{Version}{2011/04/16 v0.15}
%   \item
%     \hologo{LuaTeX}: \cs{pdf@shellescape} is only supported
%     for version 0.70.0 and higher due to a bug, \texttt{os.execute()}
%     crashes in some circumstances. Fixed in \hologo{LuaTeX}
%     beta-0.70.0, revision 4167.
%   \end{Version}
%   \begin{Version}{2011/04/22 v0.16}
%   \item
%     Previous fix was not working due to a wrong catcode of digit
%     zero (due to easily support the old \cs{directlua0}).
%     The version border is lowered to 0.68, because some
%     beta-0.67.0 seems also to work.
%   \end{Version}
%   \begin{Version}{2011/06/29 v0.17}
%   \item
%     Documentation addition to \cs{pdf@shellescape}.
%   \end{Version}
%   \begin{Version}{2011/07/01 v0.18}
%   \item
%     Add Lua module loading in \cs{everyjob} for \hologo{iniTeX}
%     (\hologo{LuaTeX} only).
%   \end{Version}
%   \begin{Version}{2011/07/28 v0.19}
%   \item
%     Missing space in an info message added (Martin M\"unch).
%   \end{Version}
%   \begin{Version}{2011/11/29 v0.20}
%   \item
%     \cs{pdf@resettimer} and \cs{pdf@elapsedtime} added
%     (thanks Andy Thomas).
%   \end{Version}
%   \begin{Version}{2016/05/10 v0.21}
%   \item
%      local unpack added
%     (thanks \'{E}lie Roux).
%   \end{Version}
%   \begin{Version}{2016/05/21 v0.22}
%   \item
%     adjust \cs{textbackslas}h usage in bib file for biber bug.
%   \end{Version}
%   \begin{Version}{2016/10/02 v0.23}
%   \item
%     add file.close to lua filehandles (github pull request).
%   \end{Version}
%   \begin{Version}{2017/01/29 v0.24}
%   \item
%     Avoid loading luatex-loader for current luatex. (Use
%     pdftexcmds.lua not oberdiek.pdftexcmds.lua to simplify file
%     search with standard require)
%   \end{Version}
%   \begin{Version}{2017/03/19 v0.25}
%   \item
%     New \cs{pdf@shellescape} for Lua\TeX, see github issue 20.
%   \end{Version}
%   \begin{Version}{2018/01/21 v0.26}
%   \item
%     use rb not r mode for file open github issue 34.
%   \end{Version}
%   \begin{Version}{2018/01/30 v0.27}
%   \item
%     \cs{pdf@mdfivesum} for \hologo{XeTeX}
%   \end{Version}
%   \begin{Version}{2018/09/07 v0.28}
%   \item
%     Fix catcode regime in luatex sprint for \cs{pdf@shellescape} GH issue 45
%   \end{Version}
%   \begin{Version}{2018/09/10 v0.29}
%   \item
%     Actually do the fix described above in the code, not just document it.
%   \end{Version}
%   \begin{Version}{2019/07/25 v0.30}
%   \item
%     remove uses of module function, see PR70
%   \end{Version}
% \end{History}
%
% \PrintIndex
%
% \Finale
\endinput
|
% \end{quote}
% Do not forget to quote the argument according to the demands
% of your shell.
%
% \paragraph{Generating the documentation.}
% You can use both the \xfile{.dtx} or the \xfile{.drv} to generate
% the documentation. The process can be configured by the
% configuration file \xfile{ltxdoc.cfg}. For instance, put this
% line into this file, if you want to have A4 as paper format:
% \begin{quote}
%   \verb|\PassOptionsToClass{a4paper}{article}|
% \end{quote}
% An example follows how to generate the
% documentation with pdf\LaTeX:
% \begin{quote}
%\begin{verbatim}
%pdflatex pdftexcmds.dtx
%bibtex pdftexcmds.aux
%makeindex -s gind.ist pdftexcmds.idx
%pdflatex pdftexcmds.dtx
%makeindex -s gind.ist pdftexcmds.idx
%pdflatex pdftexcmds.dtx
%\end{verbatim}
% \end{quote}
%
% \printbibliography[
%   heading=bibnumbered,
% ]
%
% \begin{History}
%   \begin{Version}{2007/11/11 v0.1}
%   \item
%     First version.
%   \end{Version}
%   \begin{Version}{2007/11/12 v0.2}
%   \item
%     Short description fixed.
%   \end{Version}
%   \begin{Version}{2007/12/12 v0.3}
%   \item
%     Organization of Lua code as module.
%   \end{Version}
%   \begin{Version}{2009/04/10 v0.4}
%   \item
%     Adaptation for syntax change of \cs{directlua} in
%     \hologo{LuaTeX} 0.36.
%   \end{Version}
%   \begin{Version}{2009/09/22 v0.5}
%   \item
%     \cs{pdf@primitive}, \cs{pdf@ifprimitive} added.
%   \item
%     \hologo{XeTeX}'s variants are detected for
%     \cs{pdf@shellescape}, \cs{pdf@strcmp}, \cs{pdf@primitive},
%     \cs{pdf@ifprimitive}.
%   \end{Version}
%   \begin{Version}{2009/09/23 v0.6}
%   \item
%     Macro \cs{pdf@isprimitive} added.
%   \end{Version}
%   \begin{Version}{2009/12/12 v0.7}
%   \item
%     Short info shortened.
%   \end{Version}
%   \begin{Version}{2010/03/01 v0.8}
%   \item
%     Required date for package \xpackage{ifluatex} updated.
%   \end{Version}
%   \begin{Version}{2010/04/01 v0.9}
%   \item
%     Use \cs{ifeof18} for defining \cs{pdf@shellescape} between
%     \hologo{pdfTeX} 1.21a (inclusive) and 1.30.0 (exclusive).
%   \end{Version}
%   \begin{Version}{2010/11/04 v0.10}
%   \item
%     \cs{pdf@draftmode}, \cs{pdf@ifdraftmode} and
%     \cs{pdf@setdraftmode} added.
%   \end{Version}
%   \begin{Version}{2010/11/11 v0.11}
%   \item
%     Missing \cs{RequirePackage} for package \xpackage{ifpdf} added.
%   \end{Version}
%   \begin{Version}{2011/01/30 v0.12}
%   \item
%     Already loaded package files are not input in \hologo{plainTeX}.
%   \end{Version}
%   \begin{Version}{2011/03/04 v0.13}
%   \item
%     Improved Lua function \texttt{shellescape} that also
%     uses the result of \texttt{os.execute()} (thanks to Philipp Stephani).
%   \end{Version}
%   \begin{Version}{2011/04/10 v0.14}
%   \item
%     Version check of loaded module added.
%   \item
%     Patch for bug in \hologo{LuaTeX} between 0.40.6 and 0.65 that
%     is fixed in revision 4096.
%   \end{Version}
%   \begin{Version}{2011/04/16 v0.15}
%   \item
%     \hologo{LuaTeX}: \cs{pdf@shellescape} is only supported
%     for version 0.70.0 and higher due to a bug, \texttt{os.execute()}
%     crashes in some circumstances. Fixed in \hologo{LuaTeX}
%     beta-0.70.0, revision 4167.
%   \end{Version}
%   \begin{Version}{2011/04/22 v0.16}
%   \item
%     Previous fix was not working due to a wrong catcode of digit
%     zero (due to easily support the old \cs{directlua0}).
%     The version border is lowered to 0.68, because some
%     beta-0.67.0 seems also to work.
%   \end{Version}
%   \begin{Version}{2011/06/29 v0.17}
%   \item
%     Documentation addition to \cs{pdf@shellescape}.
%   \end{Version}
%   \begin{Version}{2011/07/01 v0.18}
%   \item
%     Add Lua module loading in \cs{everyjob} for \hologo{iniTeX}
%     (\hologo{LuaTeX} only).
%   \end{Version}
%   \begin{Version}{2011/07/28 v0.19}
%   \item
%     Missing space in an info message added (Martin M\"unch).
%   \end{Version}
%   \begin{Version}{2011/11/29 v0.20}
%   \item
%     \cs{pdf@resettimer} and \cs{pdf@elapsedtime} added
%     (thanks Andy Thomas).
%   \end{Version}
%   \begin{Version}{2016/05/10 v0.21}
%   \item
%      local unpack added
%     (thanks \'{E}lie Roux).
%   \end{Version}
%   \begin{Version}{2016/05/21 v0.22}
%   \item
%     adjust \cs{textbackslas}h usage in bib file for biber bug.
%   \end{Version}
%   \begin{Version}{2016/10/02 v0.23}
%   \item
%     add file.close to lua filehandles (github pull request).
%   \end{Version}
%   \begin{Version}{2017/01/29 v0.24}
%   \item
%     Avoid loading luatex-loader for current luatex. (Use
%     pdftexcmds.lua not oberdiek.pdftexcmds.lua to simplify file
%     search with standard require)
%   \end{Version}
%   \begin{Version}{2017/03/19 v0.25}
%   \item
%     New \cs{pdf@shellescape} for Lua\TeX, see github issue 20.
%   \end{Version}
%   \begin{Version}{2018/01/21 v0.26}
%   \item
%     use rb not r mode for file open github issue 34.
%   \end{Version}
%   \begin{Version}{2018/01/30 v0.27}
%   \item
%     \cs{pdf@mdfivesum} for \hologo{XeTeX}
%   \end{Version}
%   \begin{Version}{2018/09/07 v0.28}
%   \item
%     Fix catcode regime in luatex sprint for \cs{pdf@shellescape} GH issue 45
%   \end{Version}
%   \begin{Version}{2018/09/10 v0.29}
%   \item
%     Actually do the fix described above in the code, not just document it.
%   \end{Version}
%   \begin{Version}{2019/07/25 v0.30}
%   \item
%     remove uses of module function, see PR70
%   \end{Version}
% \end{History}
%
% \PrintIndex
%
% \Finale
\endinput
