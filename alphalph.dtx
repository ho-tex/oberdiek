% \iffalse meta-comment
%
% File: alphalph.dtx
% Version: 2016/05/16 v2.5
% Info: Convert numbers to letters
%
% Copyright (C) 1999, 2006-2008, 2010, 2011 by
%    Heiko Oberdiek <heiko.oberdiek at googlemail.com>
%    2016
%    https://github.com/ho-tex/oberdiek/issues
%
% This work may be distributed and/or modified under the
% conditions of the LaTeX Project Public License, either
% version 1.3c of this license or (at your option) any later
% version. This version of this license is in
%    https://www.latex-project.org/lppl/lppl-1-3c.txt
% and the latest version of this license is in
%    https://www.latex-project.org/lppl.txt
% and version 1.3 or later is part of all distributions of
% LaTeX version 2005/12/01 or later.
%
% This work has the LPPL maintenance status "maintained".
%
% The Current Maintainers of this work are
% Heiko Oberdiek and the Oberdiek Package Support Group
% https://github.com/ho-tex/oberdiek/issues
%
% The Base Interpreter refers to any `TeX-Format',
% because some files are installed in TDS:tex/generic//.
%
% This work consists of the main source file alphalph.dtx
% and the derived files
%    alphalph.sty, alphalph.pdf, alphalph.ins, alphalph.drv,
%    alphalph-test1.tex, alphalph-test2.tex, alphalph-test3.tex.
%
% Distribution:
%    CTAN:macros/latex/contrib/oberdiek/alphalph.dtx
%    CTAN:macros/latex/contrib/oberdiek/alphalph.pdf
%
% Unpacking:
%    (a) If alphalph.ins is present:
%           tex alphalph.ins
%    (b) Without alphalph.ins:
%           tex alphalph.dtx
%    (c) If you insist on using LaTeX
%           latex \let\install=y% \iffalse meta-comment
%
% File: alphalph.dtx
% Version: 2016/05/16 v2.5
% Info: Convert numbers to letters
%
% Copyright (C) 1999, 2006-2008, 2010, 2011 by
%    Heiko Oberdiek <heiko.oberdiek at googlemail.com>
%    2016
%    https://github.com/ho-tex/oberdiek/issues
%
% This work may be distributed and/or modified under the
% conditions of the LaTeX Project Public License, either
% version 1.3c of this license or (at your option) any later
% version. This version of this license is in
%    http://www.latex-project.org/lppl/lppl-1-3c.txt
% and the latest version of this license is in
%    http://www.latex-project.org/lppl.txt
% and version 1.3 or later is part of all distributions of
% LaTeX version 2005/12/01 or later.
%
% This work has the LPPL maintenance status "maintained".
%
% This Current Maintainer of this work is Heiko Oberdiek.
%
% The Base Interpreter refers to any `TeX-Format',
% because some files are installed in TDS:tex/generic//.
%
% This work consists of the main source file alphalph.dtx
% and the derived files
%    alphalph.sty, alphalph.pdf, alphalph.ins, alphalph.drv,
%    alphalph-test1.tex, alphalph-test2.tex, alphalph-test3.tex.
%
% Distribution:
%    CTAN:macros/latex/contrib/oberdiek/alphalph.dtx
%    CTAN:macros/latex/contrib/oberdiek/alphalph.pdf
%
% Unpacking:
%    (a) If alphalph.ins is present:
%           tex alphalph.ins
%    (b) Without alphalph.ins:
%           tex alphalph.dtx
%    (c) If you insist on using LaTeX
%           latex \let\install=y% \iffalse meta-comment
%
% File: alphalph.dtx
% Version: 2016/05/16 v2.5
% Info: Convert numbers to letters
%
% Copyright (C) 1999, 2006-2008, 2010, 2011 by
%    Heiko Oberdiek <heiko.oberdiek at googlemail.com>
%    2016
%    https://github.com/ho-tex/oberdiek/issues
%
% This work may be distributed and/or modified under the
% conditions of the LaTeX Project Public License, either
% version 1.3c of this license or (at your option) any later
% version. This version of this license is in
%    http://www.latex-project.org/lppl/lppl-1-3c.txt
% and the latest version of this license is in
%    http://www.latex-project.org/lppl.txt
% and version 1.3 or later is part of all distributions of
% LaTeX version 2005/12/01 or later.
%
% This work has the LPPL maintenance status "maintained".
%
% This Current Maintainer of this work is Heiko Oberdiek.
%
% The Base Interpreter refers to any `TeX-Format',
% because some files are installed in TDS:tex/generic//.
%
% This work consists of the main source file alphalph.dtx
% and the derived files
%    alphalph.sty, alphalph.pdf, alphalph.ins, alphalph.drv,
%    alphalph-test1.tex, alphalph-test2.tex, alphalph-test3.tex.
%
% Distribution:
%    CTAN:macros/latex/contrib/oberdiek/alphalph.dtx
%    CTAN:macros/latex/contrib/oberdiek/alphalph.pdf
%
% Unpacking:
%    (a) If alphalph.ins is present:
%           tex alphalph.ins
%    (b) Without alphalph.ins:
%           tex alphalph.dtx
%    (c) If you insist on using LaTeX
%           latex \let\install=y% \iffalse meta-comment
%
% File: alphalph.dtx
% Version: 2016/05/16 v2.5
% Info: Convert numbers to letters
%
% Copyright (C) 1999, 2006-2008, 2010, 2011 by
%    Heiko Oberdiek <heiko.oberdiek at googlemail.com>
%    2016
%    https://github.com/ho-tex/oberdiek/issues
%
% This work may be distributed and/or modified under the
% conditions of the LaTeX Project Public License, either
% version 1.3c of this license or (at your option) any later
% version. This version of this license is in
%    http://www.latex-project.org/lppl/lppl-1-3c.txt
% and the latest version of this license is in
%    http://www.latex-project.org/lppl.txt
% and version 1.3 or later is part of all distributions of
% LaTeX version 2005/12/01 or later.
%
% This work has the LPPL maintenance status "maintained".
%
% This Current Maintainer of this work is Heiko Oberdiek.
%
% The Base Interpreter refers to any `TeX-Format',
% because some files are installed in TDS:tex/generic//.
%
% This work consists of the main source file alphalph.dtx
% and the derived files
%    alphalph.sty, alphalph.pdf, alphalph.ins, alphalph.drv,
%    alphalph-test1.tex, alphalph-test2.tex, alphalph-test3.tex.
%
% Distribution:
%    CTAN:macros/latex/contrib/oberdiek/alphalph.dtx
%    CTAN:macros/latex/contrib/oberdiek/alphalph.pdf
%
% Unpacking:
%    (a) If alphalph.ins is present:
%           tex alphalph.ins
%    (b) Without alphalph.ins:
%           tex alphalph.dtx
%    (c) If you insist on using LaTeX
%           latex \let\install=y\input{alphalph.dtx}
%        (quote the arguments according to the demands of your shell)
%
% Documentation:
%    (a) If alphalph.drv is present:
%           latex alphalph.drv
%    (b) Without alphalph.drv:
%           latex alphalph.dtx; ...
%    The class ltxdoc loads the configuration file ltxdoc.cfg
%    if available. Here you can specify further options, e.g.
%    use A4 as paper format:
%       \PassOptionsToClass{a4paper}{article}
%
%    Programm calls to get the documentation (example):
%       pdflatex alphalph.dtx
%       makeindex -s gind.ist alphalph.idx
%       pdflatex alphalph.dtx
%       makeindex -s gind.ist alphalph.idx
%       pdflatex alphalph.dtx
%
% Installation:
%    TDS:tex/generic/oberdiek/alphalph.sty
%    TDS:doc/latex/oberdiek/alphalph.pdf
%    TDS:doc/latex/oberdiek/test/alphalph-test1.tex
%    TDS:doc/latex/oberdiek/test/alphalph-test2.tex
%    TDS:doc/latex/oberdiek/test/alphalph-test3.tex
%    TDS:source/latex/oberdiek/alphalph.dtx
%
%<*ignore>
\begingroup
  \catcode123=1 %
  \catcode125=2 %
  \def\x{LaTeX2e}%
\expandafter\endgroup
\ifcase 0\ifx\install y1\fi\expandafter
         \ifx\csname processbatchFile\endcsname\relax\else1\fi
         \ifx\fmtname\x\else 1\fi\relax
\else\csname fi\endcsname
%</ignore>
%<*install>
\input docstrip.tex
\Msg{************************************************************************}
\Msg{* Installation}
\Msg{* Package: alphalph 2016/05/16 v2.5 Convert numbers to letters (HO)}
\Msg{************************************************************************}

\keepsilent
\askforoverwritefalse

\let\MetaPrefix\relax
\preamble

This is a generated file.

Project: alphalph
Version: 2016/05/16 v2.5

Copyright (C) 1999, 2006-2008, 2010, 2011 by
   Heiko Oberdiek <heiko.oberdiek at googlemail.com>

This work may be distributed and/or modified under the
conditions of the LaTeX Project Public License, either
version 1.3c of this license or (at your option) any later
version. This version of this license is in
   http://www.latex-project.org/lppl/lppl-1-3c.txt
and the latest version of this license is in
   http://www.latex-project.org/lppl.txt
and version 1.3 or later is part of all distributions of
LaTeX version 2005/12/01 or later.

This work has the LPPL maintenance status "maintained".

This Current Maintainer of this work is Heiko Oberdiek.

The Base Interpreter refers to any `TeX-Format',
because some files are installed in TDS:tex/generic//.

This work consists of the main source file alphalph.dtx
and the derived files
   alphalph.sty, alphalph.pdf, alphalph.ins, alphalph.drv,
   alphalph-test1.tex, alphalph-test2.tex, alphalph-test3.tex.

\endpreamble
\let\MetaPrefix\DoubleperCent

\generate{%
  \file{alphalph.ins}{\from{alphalph.dtx}{install}}%
  \file{alphalph.drv}{\from{alphalph.dtx}{driver}}%
  \usedir{tex/generic/oberdiek}%
  \file{alphalph.sty}{\from{alphalph.dtx}{package}}%
%  \usedir{doc/latex/oberdiek/test}%
%  \file{alphalph-test1.tex}{\from{alphalph.dtx}{test1}}%
%  \file{alphalph-test2.tex}{\from{alphalph.dtx}{test2,etex}}%
%  \file{alphalph-test3.tex}{\from{alphalph.dtx}{test2,noetex}}%
  \nopreamble
  \nopostamble
%  \usedir{source/latex/oberdiek/catalogue}%
%  \file{alphalph.xml}{\from{alphalph.dtx}{catalogue}}%
}

\catcode32=13\relax% active space
\let =\space%
\Msg{************************************************************************}
\Msg{*}
\Msg{* To finish the installation you have to move the following}
\Msg{* file into a directory searched by TeX:}
\Msg{*}
\Msg{*     alphalph.sty}
\Msg{*}
\Msg{* To produce the documentation run the file `alphalph.drv'}
\Msg{* through LaTeX.}
\Msg{*}
\Msg{* Happy TeXing!}
\Msg{*}
\Msg{************************************************************************}

\endbatchfile
%</install>
%<*ignore>
\fi
%</ignore>
%<*driver>
\NeedsTeXFormat{LaTeX2e}
\ProvidesFile{alphalph.drv}%
  [2016/05/16 v2.5 Convert numbers to letters (HO)]%
\documentclass{ltxdoc}
\usepackage{alphalph}[2016/05/16]
\makeatletter
\newalphalph{\fnsymbolwrap}[wrap]{\@fnsymbol}{}
\newcommand*{\fnsymbolsingle}[1]{%
  \ensuremath{%
    \ifcase#1%
    \or *%
    \or \dagger
    \or \ddagger
    \or \mathsection
    \or \mathparagraph
    \else
      \@ctrerr
    \fi
  }%
}
\makeatother
\newalphalph{\fnsymbolmult}[mult]{\fnsymbolsingle}{}
\usepackage{amsmath}
\DeclareMathOperator{\opDiv}{div}
\DeclareMathOperator{\opMod}{mod}
\newcommand*{\Def}{%
  \ensuremath{%
    \mathrel{\mathop{:}}=%
  }%
}
\usepackage{holtxdoc}[2011/11/22]
\begin{document}
  \DocInput{alphalph.dtx}%
\end{document}
%</driver>
% \fi
%
%
% \CharacterTable
%  {Upper-case    \A\B\C\D\E\F\G\H\I\J\K\L\M\N\O\P\Q\R\S\T\U\V\W\X\Y\Z
%   Lower-case    \a\b\c\d\e\f\g\h\i\j\k\l\m\n\o\p\q\r\s\t\u\v\w\x\y\z
%   Digits        \0\1\2\3\4\5\6\7\8\9
%   Exclamation   \!     Double quote  \"     Hash (number) \#
%   Dollar        \$     Percent       \%     Ampersand     \&
%   Acute accent  \'     Left paren    \(     Right paren   \)
%   Asterisk      \*     Plus          \+     Comma         \,
%   Minus         \-     Point         \.     Solidus       \/
%   Colon         \:     Semicolon     \;     Less than     \<
%   Equals        \=     Greater than  \>     Question mark \?
%   Commercial at \@     Left bracket  \[     Backslash     \\
%   Right bracket \]     Circumflex    \^     Underscore    \_
%   Grave accent  \`     Left brace    \{     Vertical bar  \|
%   Right brace   \}     Tilde         \~}
%
% \GetFileInfo{alphalph.drv}
%
% \title{The \xpackage{alphalph} package}
% \date{2016/05/16 v2.5}
% \author{Heiko Oberdiek\thanks
% {Please report any issues at \url{https://github.com/ho-tex/oberdiek/issues}}\\
% \xemail{heiko.oberdiek at googlemail.com}}
%
% \maketitle
%
% \begin{abstract}
% The package provides methods to represent numbers with a limited
% set of symbols. Both \hologo{LaTeX} and \hologo{plainTeX} are supported.
% \end{abstract}
%
% \tableofcontents
%
% \section{Documentation}
%
% \subsection{Introduction}
% \hologo{LaTeX} counters can be represented in different ways
% by using presentation commands:
% \begin{quote}
%   \cs{arabic}, \cs{roman}, \cs{Roman},\\
%   \cs{alph}, \cs{Alph}, \cs{fnsymbol}
% \end{quote}
% The ranges of supported counter values are more or less
% restricted. Only \cs{arabic} can be used with any counter
% value \hologo{TeX} supports.
% \begin{quote}
% \catcode`\|=12 %
% \begin{tabular}{@{}l|l|l|l@{}}
% Presentation & Supported & Ignored & Error message\\
% command      & domain    & values  & ``Counter too large''\\
% \hline
% \cs{arabic}
%   & \ttfamily -MAX..MAX & &\\
% \cs{roman}, \cs{Roman}
%   & \ttfamily 1..MAX & \ttfamily -MAX..0 &\\
% \cs{alph}, \cs{Alph}
%   & \ttfamily 1..26 & 0 & \ttfamily -MAX..-1, 27..MAX\\
% \cs{fnsymbol}
%   & \ttfamily 1..9 & 0 & \ttfamily -MAX..-1, 10..MAX\\
% \hline
% \end{tabular}\\[1ex]
% \texttt{MAX} = \texttt{2147483647}
% \end{quote}
% Ordinal numbers are often used in documents: numbering
% of chapters, sections, figures, footnotes and so on.
% The layouter chooses \cs{Alph} for chapter numbers
% and \cs{fnsymbol} for footnotes. But what can be done
% if there are more than 26 chapters or more than 10 footnotes?
% This package \xpackage{alphalph} allows to define new
% presentation commands. They rely on a existing command
% and define presentations for values greater the limits.
% Three different methods are provided by the package.
% In the following use cases they are presentated.
%
% \subsection{Use cases}
%
% \subsubsection{Number system based on symbols}
%
% Asume you are writing a book and your lecturer demands
% that chapter numbers must be letters. But you have already
% 30 chapters and you have only 26 letters?
%
% In the decimal system the situation would be clear.
% If you run out of digits, you are using more digits
% to represent a number. This method can be also be used
% for letters. After chapter 26 with |Z| we us |AA|, |AB|, |AC|, and |AD|
% for the remaining chapters.
%
% Happily this package already defines this presentation command:
%\begin{quote}
%\begin{verbatim}
%\usepackage{alphalph}
%\renewcommand*{\thechapter}{%
%  \AlphAlph{\value{chapter}}%
%}
%\end{verbatim}
%\end{quote}
% \cs{AlphAlph} generates:
%   \AlphAlph{1}, \AlphAlph{2}, \AlphAlph{3}, \dots,
%   \AlphAlph{26}, \AlphAlph{27}, \AlphAlph{28}, \dots
%
% The other presentation command is \cs{alphalph} for lowercase letters.
%
% \subsubsection{Wrap symbols around}
%
% Nine footnote symbols are quite a few.
% Too soon the symbols are consumed
% and \hologo{LaTeX} complains with the error ``Counter too large''.
% However, it could be acceptable to start again with the symbols
% from the beginning, especially if there are less than nine symbols
% on a page.
% This could be achieved by a counter reset.
% But finding the right place can be difficult or needs manual
% actions. Also a
% unique counter value can be desirable (e.g. for generating
% unique anchor/link names).
% Package \xpackage{alphalph} allows you to define
% a macro that implements a ``wrap around'', but letting
% the value of the counter untouched:
%\begin{quote}
%\begin{verbatim}
%\usepackage{alphalph}
%\makeatletter
%\newalphalph{\fnsymbolwrap}[wrap]{\@fnsymbol}{}
%\makeatother
%\renewcommand*{\thefootnote}{%
%  \fnsymbolwrap{\value{footnote}}%
%}
%\end{verbatim}
%\end{quote}
% \cs{fnsymbolwrap} generates:
% \fnsymbolwrap{1}~(1), \fnsymbolwrap{2}~(2), \fnsymbolwrap{3}~(3), \dots,
% \fnsymbolwrap{9}~(9), \fnsymbolwrap{10}~(10), \fnsymbolwrap{11}~11, \dots
%
% \subsubsection{Multiple symbols}
%
% \Hologo{LaTeX}'s standard set of footnote symbols contains doubled
% symbols at the higher positions. Could this principle
% be generalized?
% Yes, but first we need a clean footnote symbol list without
% doubled entries, example:
%\begin{quote}
%\begin{verbatim}
%\usepackage{alphalph}
%\makeatletter
%\newcommand*{\fnsymbolsingle}[1]{%
%  \ensuremath{%
%    \ifcase#1%
%    \or *%
%    \or \dagger
%    \or \ddagger
%    \or \mathsection
%    \or \mathparagraph
%    \else
%      \@ctrerr
%    \fi
%  }%
%}
%\makeatother
%\newalphalph{\fnsymbolmult}[mult]{\fnsymbolsingle}{}
%\renewcommand*{\thefootnote}{%
%  \fnsymbolmult{\value{footnote}}%
%}
%\end{verbatim}
%\end{quote}
% The own definition of \cs{fnsymbolsingle} has the
% advantage that this list can easily modified.
% Otherwise you can use \cs{@fnsymbol} directly,
% because it uses the same first five symbols.
%\begin{quote}
%\begin{verbatim}
%\usepackage{alphalph}
%\makeatletter
%\newalphalph{\fnsymbolmult}[mult]{\@fnsymbol}{5}
%\makeatother
%\renewcommand*{\thefootnote}{%
%  \fnsymbolmult{\value{footnote}}%
%}
%\end{verbatim}
%\end{quote}
% \cs{fnsymbolmult} generates:
% \fnsymbolmult{1}~(1), \fnsymbolmult{2}~(2), \fnsymbolmult{3}~(3),
% \fnsymbolmult{4}~(4), \fnsymbolmult{5}~(5), \fnsymbolmult{6}~(6),
% \dots, \fnsymbolmult{16}~16, \fnsymbolmult{17}~17, \dots
%
% \bigskip
% The same method can also be used for the chapter problem
% in the first discussed use case:
%\begin{quote}
%\begin{verbatim}
%\usepackage{alphalph}
%\makeatletter
%\newalphalph{\AlphMult}[mult]{\@Alph}{26}
%\makeatother
%\renewcommand*{\chapter}{%
%  \AlphMult{\value{chapter}}%
%}
%\end{verbatim}
%\end{quote}
% \cs{AlphMult} then generates AA, BB, CC, and DD for chapters 27--30.
%
% \subsection{Glossary}
%
% \begin{description}
% \item[Counter presentation command] is a macro that
%   expects a \hologo{LaTeX} counter name as argument.
%   Numbers cannot be used. Examples:
%   \cs{arabic}, \cs{alph}, \cs{fnsymbol}.
% \item[Number presentation command] is a macro that
%   expects a number as argument. A number is anything
%   that \hologo{TeX} accepts as number including \cs{value}.
%   Examples: \cs{alphalph}, \cs{AlphAlph}, \cs{alphalph@alph}
%
%   However, \cs{alph} or \cs{fnsymbol} are not number
%   presentation commands because they expect a counter
%   name as argument. Happily \hologo{LaTeX} counter presentation
%   commands internally uses number presentation commands
%   with the same name, but prefixed by `@'. Thus
%   \cs{@alph}, \cs{@fnsymbol} are number presentation commands.
% \item[Symbols provider] is a command that can be used
%   to get a list of symbols. For example, \cs{@Alph}
%   provides the 26 uppercase letters from `A' to `Z'.
%   Basically a symbol provider is a number presentation
%   command, usually with a limited range.
% \item[Number of symbols] is the number of the
%   last symbol slot of a symbol provider. Thus \cs{@Alph}
%   generates 26 symbols, \cs{@fnsymbol} provides 9 symbols.
% \end{description}
%
% \subsection{Package usage}
%
% The package \Package{alphalph} can be used with both \hologo{plainTeX}
% and \hologo{LaTeX}:
% \begin{description}
% \item[\hologo{plainTeX}:] |\input alphalph.sty|
% \item[\hologo{LaTeX2e}:]   |\usepackage{alphalph}|\\
%      There aren't any options.
% \end{description}
%
% \subsection{User commands}
%
% \begin{declcs}{AlphAlph} \M{number}\\
%   \cs{alphalph} \M{number}
% \end{declcs}
% Both macros are number presentation commands that
% expects a number as argument. \hologo{LaTeX} counters
% are used with \cs{value}.
%
% The macros represents a number by letters.
% First single letters |A..Z| are used, then
% two letters |AA..ZZ|, three letters |AAA...ZZZ|, \dots
% follow.
%
% Macro \cs{AlphAlph} uses uppercase letters,
% \cs{alphalph} generates the lowercase variant.
% \begin{quote}
%   \catcode`\|=12
%   \def\entry#1{
%     \texttt{#1}
%     & \texttt{\AlphAlph{#1}}
%     & \texttt{\alphalph{#1}}
%   }
%   \begin{tabular}{@{}l|l|l@{}}
%     \meta{number} & \verb|\AlphAlph{|\meta{number}\verb|}|
%                   & \verb|\alphalph{|\meta{number}\verb|}|\\
%     \hline
%     \entry{1}\\
%     \entry{2}\\
%     \entry{26}\\
%     \entry{27}\\
%     \entry{30}\\
%     \entry{2000}\\
%     \entry{3752127}\\
%     \entry{10786572}\\
%     \entry{2147483647}\\
%   \end{tabular}
% \end{quote}
%
% \begin{declcs}{newalphalph}
%   \M{cmd}
%   |[|\meta{method}|]|
%   \M{symbols provider}
%   \M{number of symbols}
% \end{declcs}
% Macro \cs{newalphalph} defines \meta{cmd} as new
% number presentation command. Like \cs{newcommand} an error
% is thrown, if macro \meta{cmd} already exists.
%
% The \meta{method} is one of |alph|, |wrap|, or |mult|.
% The default is |alph|.
%
% As symbol provider a number presentation command
% can be used, e.g. \cs{@fnsymbol}, \cs{@Alph}, or
% \cs{alphalph@alph}.
%
% The last argument is the number of symbols.
% If the argument is empty, then \cs{newalphalph} tries
% to find this number itself. \hologo{LaTeX}'s number presentation
% commands throw an error message, if the number is too large.
% This error message is put in a macro \cs{@ctrerr}.
% Thus \cs{newalphalph} calls the symbol provider and tests
% a number by typesetting it in a temporary box. The error
% macro \cs{@ctrerr} is catched, it proofs that the number
% is not supported. Also if the width of the result is zero
% the number is considered as unavailable.
%
% The empty argument is useful for potentially variable lists.
% However if the end cannot be detected, then the number of
% symbols must be given. This is also a lot faster. Therefore don't
% let the argument empty without reason.
%
% \subsection{Programmer commands}
%
% \begin{declcs}{alphalph@Alph} \M{number}\\
%   \cs{alphalph@alph} \M{number}
% \end{declcs}
% They are basically the same as \cs{@Alph} and \cs{@alph}.
% Some languages of package \xpackage{babel} redefine
% \hologo{LaTeX}'s macros to include some font setup that
% breaks expandibility. Therefore \cs{AlphAlph} and \cs{alphalph}
% are based on \cs{alphalph@Alph} and \cs{alphalph@alph} to get
% the letters. The behaviour of these symbol providers for numbers
% outside the range |1..26| is undefined.
%
% \subsection{Design principles}
%
% \subsubsection{Number presentation commands}
%
% All number presentation commands that this package defines
% (including \cs{alphalph} and \cs{AlphAlph}) have the following
% properties:
% \begin{itemize}
% \item They are fully expandable. This means that they can safely
%       \begin{itemize}
%       \item be written to a file,
%       \item used in moving arguments (\hologo{LaTeX}: they are \emph{robust}),
%       \item used in a \cmd{\csname}-\cmd{\endcsname} pair.
%       \end{itemize}
% \item If the argument is zero or negative, the commands expand
%       to nothing like \cmd{\romannumeral}.
% \item The argument is a \hologo{TeX} number. Anything that would be
%       accepted by \cmd{\number} is a valid argument:
%       \begin{itemize}
%       \item explicite constants,
%       \item macros that expand to a number,
%       \item count registers, \hologo{LaTeX} counter can used
%             via \cmd{\value}, e.\,g.:\\
%             |\alphalph{\value{page}}|
%       \item \dots
%       \end{itemize}
% \item \hologo{eTeX}'s numeric expressions are supported, if \hologo{eTeX} is
%       available. Then \cs{numexpr} is applied to the argument.
%       Package \cs{calc}'s expressions are not supported.
%       That would violate the expandibility.
% \end{itemize}
%
% \subsubsection{General usability}
%
% \begin{description}
% \item[\hologo{TeX} format:]
%   The package does not depend on \hologo{LaTeX}, it can also be used
%   by \hologo{plainTeX}, for example.
% \item[\hologo{eTeX}:] \eTeX is supported, the macros are shorter and
%   faster. But \eTeX's extensions are not requirements.
%   Without \hologo{eTeX}, just the implementation changes. The properties
%   remain unchanged.
% \end{description}
%
% \StopEventually{
% }
%
% \section{Implementation}
% \subsection{Begin\texorpdfstring{ of package}{}}
%    \begin{macrocode}
%<*package>
%    \end{macrocode}
%    Reload check, especially if the package is not used with \LaTeX.
%    \begin{macrocode}
\begingroup\catcode61\catcode48\catcode32=10\relax%
  \catcode13=5 % ^^M
  \endlinechar=13 %
  \catcode35=6 % #
  \catcode39=12 % '
  \catcode44=12 % ,
  \catcode45=12 % -
  \catcode46=12 % .
  \catcode58=12 % :
  \catcode64=11 % @
  \catcode123=1 % {
  \catcode125=2 % }
  \expandafter\let\expandafter\x\csname ver@alphalph.sty\endcsname
  \ifx\x\relax % plain-TeX, first loading
  \else
    \def\empty{}%
    \ifx\x\empty % LaTeX, first loading,
      % variable is initialized, but \ProvidesPackage not yet seen
    \else
      \expandafter\ifx\csname PackageInfo\endcsname\relax
        \def\x#1#2{%
          \immediate\write-1{Package #1 Info: #2.}%
        }%
      \else
        \def\x#1#2{\PackageInfo{#1}{#2, stopped}}%
      \fi
      \x{alphalph}{The package is already loaded}%
      \aftergroup\endinput
    \fi
  \fi
\endgroup%
%    \end{macrocode}
%    Package identification:
%    \begin{macrocode}
\begingroup\catcode61\catcode48\catcode32=10\relax%
  \catcode13=5 % ^^M
  \endlinechar=13 %
  \catcode35=6 % #
  \catcode39=12 % '
  \catcode40=12 % (
  \catcode41=12 % )
  \catcode44=12 % ,
  \catcode45=12 % -
  \catcode46=12 % .
  \catcode47=12 % /
  \catcode58=12 % :
  \catcode64=11 % @
  \catcode91=12 % [
  \catcode93=12 % ]
  \catcode123=1 % {
  \catcode125=2 % }
  \expandafter\ifx\csname ProvidesPackage\endcsname\relax
    \def\x#1#2#3[#4]{\endgroup
      \immediate\write-1{Package: #3 #4}%
      \xdef#1{#4}%
    }%
  \else
    \def\x#1#2[#3]{\endgroup
      #2[{#3}]%
      \ifx#1\@undefined
        \xdef#1{#3}%
      \fi
      \ifx#1\relax
        \xdef#1{#3}%
      \fi
    }%
  \fi
\expandafter\x\csname ver@alphalph.sty\endcsname
\ProvidesPackage{alphalph}%
  [2016/05/16 v2.5 Convert numbers to letters (HO)]%
%    \end{macrocode}
%
% \subsection{Catcodes}
%
%    \begin{macrocode}
\begingroup\catcode61\catcode48\catcode32=10\relax%
  \catcode13=5 % ^^M
  \endlinechar=13 %
  \catcode123=1 % {
  \catcode125=2 % }
  \catcode64=11 % @
  \def\x{\endgroup
    \expandafter\edef\csname AlPh@AtEnd\endcsname{%
      \endlinechar=\the\endlinechar\relax
      \catcode13=\the\catcode13\relax
      \catcode32=\the\catcode32\relax
      \catcode35=\the\catcode35\relax
      \catcode61=\the\catcode61\relax
      \catcode64=\the\catcode64\relax
      \catcode123=\the\catcode123\relax
      \catcode125=\the\catcode125\relax
    }%
  }%
\x\catcode61\catcode48\catcode32=10\relax%
\catcode13=5 % ^^M
\endlinechar=13 %
\catcode35=6 % #
\catcode64=11 % @
\catcode123=1 % {
\catcode125=2 % }
\def\TMP@EnsureCode#1#2{%
  \edef\AlPh@AtEnd{%
    \AlPh@AtEnd
    \catcode#1=\the\catcode#1\relax
  }%
  \catcode#1=#2\relax
}
\TMP@EnsureCode{33}{12}% !
\TMP@EnsureCode{39}{12}% '
\TMP@EnsureCode{40}{12}% (
\TMP@EnsureCode{41}{12}% )
\TMP@EnsureCode{43}{12}% +
\TMP@EnsureCode{44}{12}% ,
\TMP@EnsureCode{46}{12}% .
\TMP@EnsureCode{47}{12}% /
\TMP@EnsureCode{59}{12}% ;
\TMP@EnsureCode{60}{12}% <
\TMP@EnsureCode{62}{12}% >
\TMP@EnsureCode{91}{12}% [
\TMP@EnsureCode{93}{12}% ]
\TMP@EnsureCode{96}{12}% `
\TMP@EnsureCode{124}{12}% |
\edef\AlPh@AtEnd{\AlPh@AtEnd\noexpand\endinput}
%    \end{macrocode}
%
% \subsection{Package loading}
%
%    \begin{macrocode}
\begingroup\expandafter\expandafter\expandafter\endgroup
\expandafter\ifx\csname RequirePackage\endcsname\relax
  \input infwarerr.sty\relax
  \input intcalc.sty\relax
\else
  \RequirePackage{infwarerr}[2007/09/09]%
  \RequirePackage{intcalc}[2007/09/09]%
\fi
%    \end{macrocode}
%
% \subsection{\hologo{eTeX} detection}
%
%    \begin{macrocode}
\begingroup\expandafter\expandafter\expandafter\endgroup
\expandafter\ifx\csname numexpr\endcsname\relax
  \catcode124=9 % '!': ignore
  \catcode43=14 % '+': comment
\else
  \catcode124=14 % '!': comment
  \catcode43=9   % '+': ignore
\fi
%    \end{macrocode}
%
% \subsection{Help macros}
%
%    \begin{macro}{\AlPh@Error}
%    \begin{macrocode}
\def\AlPh@Error#1{%
  \begingroup
    \escapechar=92 % backslash
    \@PackageError{alphalph}{#1}\@ehc
  \endgroup
}
%    \end{macrocode}
%    \end{macro}
%
%    \begin{macro}{\AlPh@IfDefinable}
%    \begin{macrocode}
\begingroup\expandafter\expandafter\expandafter\endgroup
\expandafter\ifx\csname @ifdefinable\endcsname\relax
  \def\AlPh@IfDefinable#1#2{%
    \ifcase\ifx#1\@undefined\else\ifx#1\relax\else1\fi\fi0 %
      #2%
    \else
      \AlPh@Error{%
        Command \string#1 already defined%
      }%
    \fi
  }%
%    \end{macrocode}
%    \end{macro}
%    \begin{macrocode}
\else
%    \end{macrocode}
%    \begin{macro}{\AlPh@IfDefinable}
%    \begin{macrocode}
  \let\AlPh@IfDefinable\@ifdefinable
%    \end{macrocode}
%    \end{macro}
%    \begin{macrocode}
\fi
%    \end{macrocode}
%
%    \begin{macro}{\@ReturnAfterElseFi}
%    \begin{macro}{\@ReturnAfterFi}
%    The following commands moves the `then' and `else' part respectively
%    behind the |\if|-construct. This prevents a too deep |\if|-nesting
%    and so a \hologo{TeX} capacity error because of a limited input stack size.
%    I use this trick in several packages, so I don't prefix these internal
%    commands in order not to have the same macros with different names.
%    (It saves memory.)
%    \begin{macrocode}
\long\def\@ReturnAfterElseFi#1\else#2\fi{\fi#1}
\long\def\@ReturnAfterFi#1\fi{\fi#1}
%    \end{macrocode}
%    \end{macro}
%    \end{macro}
%
%    \begin{macro}{\@gobblefour}
%    \Hologo{LaTeX} defines commands for eating arguments.
%    Define \cs{@gobblefour} if it is not defined (\hologo{plainTeX}).
%    \begin{macrocode}
\expandafter\ifx\csname @gobblefour\endcsname\relax
  \long\def\@gobblefour#1#2#3#4{}%
\fi
%    \end{macrocode}
%    \end{macro}
%
%    \begin{macro}{AlPh@IfOptArg}
%    \begin{macrocode}
\begingroup\expandafter\expandafter\expandafter\endgroup
\expandafter\ifx\csname kernel@ifnextchar\endcsname\relax
  \begingroup\expandafter\expandafter\expandafter\endgroup
  \expandafter\ifx\csname @ifnextchar\endcsname\relax
    \def\AlPh@IfOptArg#1#2{%
      \def\AlPh@TempA{#1}%
      \def\AlPh@TempB{#2}%
      \futurelet\AlPh@Token\AlPh@IfOptArgNext
    }%
    \let\AlPh@BracketLeft=[%]
    \def\AlPh@IfOptArgNext{%
      \ifx\AlPh@Token\AlPh@BracketLeft
        \expandafter\AlPh@TempA
      \else
        \expandafter\AlPh@TempB
      \fi
    }%
  \else
    \def\AlPh@IfOptArg{\@ifnextchar[}%]
  \fi
\else
  \def\AlPh@IfOptArg{\kernel@ifnextchar[}%]
\fi
%    \end{macrocode}
%    \end{macro}
%
% \subsection{Symbol provider}
%
% \subsubsection{Alphabet}
%
%    The output of \cs{alphalph} and \cs{AlphAlph} should be
%    usable as part of command names (see \cs{@namedef}, \cs{csname}, \dots).
%    Unhappily some languages of package \xpackage{babel}
%    redefine \hologo{LaTeX}'s \cs{@alph} and \cs{@Alph} in a manner that
%    they cannot be used in expandable context any more.
%    Therefore package \xpackage{alphalph} provides its own
%    commands.
%
%    \begin{macro}{\alphalph@Alph}
%    \begin{macro}{\alphalph@alph}
%    The two commands \cmd{\AlPh@Alph} and \cmd{\AlPh@alph} convert a
%    number into a letter (uppercase and lowercase respectivly).
%    The character |@| is used as an error symbol,
%    if the number isn't in the range of 1 until 26.
%    Here we need no space after the number |#1|, because the error
%    symbol |@| for the zero case stops scanning the number.
%    This error symbol should not appear anywhere (except for bugs).
%    \begin{macrocode}
\def\alphalph@Alph#1{%
  \ifcase#1%
    @%
  \or A\or B\or C\or D\or E\or F\or G\or H\or I\or J\or K\or L\or M%
  \or N\or O\or P\or Q\or R\or S\or T\or U\or V\or W\or X\or Y\or Z%
  \else
    \AlPh@ctrerr
    @%
  \fi
}
\def\alphalph@alph#1{%
  \ifcase#1%
    @%
  \or a\or b\or c\or d\or e\or f\or g\or h\or i\or j\or k\or l\or m%
  \or n\or o\or p\or q\or r\or s\or t\or u\or v\or w\or x\or y\or z%
  \else
    \AlPh@ctrerr
    @%
  \fi
}
%    \end{macrocode}
%    \end{macro}
%    \end{macro}
%    \begin{macro}{\AlPh@ctrerr}
%    Macro \cs{AlPh@ctrerr} is used as hook for the algorithm
%    to get the available number of symbols.
%    \begin{macrocode}
\def\AlPh@ctrerr{}
%    \end{macrocode}
%    \end{macro}
%
% \subsection{Finding number of symbols}
%
%    \begin{macro}{\AlPh@GetNumberOfSymbols}
%    \noindent
%    |#1|: symbols provider
%    \begin{macrocode}
\def\AlPh@GetNumberOfSymbols#1{%
  \AlPh@TestNumber1!{#1}%
  \ifAlPh@Unavailable
    \def\AlPh@Number{0}%
    \AlPh@Error{No symbols found}%
  \else
    \def\AlPh@Number{1}%
    \AlPh@ExpSearch2!{#1}%
  \fi
}
%    \end{macrocode}
%    \end{macro}
%    \begin{macro}{\ifAlPh@Unavailable}
%    \begin{macrocode}
\let\ifAlPh@Unavailable\iffalse
\def\AlPh@Unavailabletrue{%
  \global\let\ifAlPh@Unavailable\iftrue
}
\def\AlPh@Unavailablefalse{%
  \global\let\ifAlPh@Unavailable\iffalse
}
%    \end{macrocode}
%    \end{macro}
%    \begin{macro}{\AlPh@TestNumber}
%    |#1|: number to be tested\\
%    |#2|: symbols provider
%    \begin{macrocode}
\def\AlPh@TestNumber#1!#2{%
  \AlPh@Unavailablefalse
  \begingroup
    \setbox0=\hbox{%
      \begingroup % color
        \let\@ctrerr\AlPh@Unavailabletrue
        \let\AlPh@ctrerr\AlPh@Unavailabletrue
        #2{#1}%
      \endgroup
    }%
    \ifdim\wd0=0pt %
      \AlPh@Unavailabletrue
    \fi
  \endgroup
}
%    \end{macrocode}
%    \end{macro}
%    \begin{macro}{\AlPh@ExpSearch}
%    |#1|: number to be tested\\
%    |#2|: symbols provider
%    \begin{macrocode}
\def\AlPh@ExpSearch#1!#2{%
  \let\AlPh@Next\relax
  \AlPh@TestNumber#1!{#2}%
  \ifAlPh@Unavailable
    \expandafter\AlPh@BinSearch\AlPh@Number!#1!{#2}%
  \else
    \def\AlPh@Number{#1}%
    \ifnum#1>1073741823 %
      \AlPh@TestNumber2147483647!{#2}%
      \ifAlPh@Unavailable
        \AlPh@BinSearch#1!2147483647!{#2}%
      \else
        \def\AlPh@Number{0}%
        \AlPh@Error{%
          Maximal symbol number not found%
        }%
      \fi
    \else
      \def\AlPh@Next{%
        \expandafter\AlPh@ExpSearch\number\intcalcShl{#1}!{#2}%
      }%
    \fi
  \fi
  \AlPh@Next
}
%    \end{macrocode}
%    \end{macro}
%    \begin{macro}{\AlPh@BinSearch}
%    |#1|: available number\\
%    |#2|: unavailable number, |#2| $>$ |#1|\\
%    |#3|: symbols provider
%    \begin{macrocode}
\def\AlPh@BinSearch#1!#2!#3{%
  \expandafter\AlPh@ProcessBinSearch
  \number\intcalcShr{\intcalcAdd{#1}{#2}}!%
  #1!#2!{#3}%
}
%    \end{macrocode}
%    \end{macro}
%    \begin{macro}{\AlPh@ProcessBinSearch}
%    |#1|: number to be tested, |#2| $\leq$ |#1| $\leq$ |#3|\\
%    |#2|: available number\\
%    |#3|: unavailable number\\
%    |#4|: symbols provider
%    \begin{macrocode}
\def\AlPh@ProcessBinSearch#1!#2!#3!#4{%
  \let\AlPh@Next\relax
  \ifnum#1>#2 %
    \ifnum#1<#3 %
      \AlPh@TestNumber#1!{#4}%
      \ifAlPh@Unavailable
        \def\AlPh@Next{%
          \AlPh@BinSearch#2!#1!{#4}%
        }%
      \else
        \def\AlPh@Next{%
          \AlPh@BinSearch#1!#3!{#4}%
        }%
      \fi
    \else
      \def\AlPh@Number{#2}%
    \fi
  \else
    \def\AlPh@Number{#2}%
  \fi
  \AlPh@Next
}
%    \end{macrocode}
%    \end{macro}
%
% \subsection{Methods}
%
%    The names of method macros start with \cs{AlPh@Method}.
%    These macros do the main job in converting a number to
%    its representation.
%    A method command is called with three arguments.
%    The first argument is the number of symbols. The
%    second argument is the basic macro for converting
%    a number with limited number range. The last parameter
%    is the number that needs converting.
%
% \subsubsection{Common methods}
%
%    \begin{macro}{\AlPh@CheckPositive}
%    \noindent
%    |#1|: number to be checked
%    |#2|: continuation macro\\
%    |#3|: number of symbols (hidden here)\\
%    |#4|: symbol provider (hidden here)
%    \begin{macrocode}
\def\AlPh@CheckPositive#1!#2{%
  \ifnum#1<1 %
    \expandafter\@gobblefour
  \fi
  #2{#1}%
}
%    \end{macrocode}
%    \end{macro}
%
% \subsubsection{Method `alph'}
%
%    \begin{macro}{\AlPh@Method@alph}
%    \noindent
%    |#1|: number of symbols\\
%    |#2|: symbols provider\\
%    |#3|: number to be converted
%    \begin{macrocode}
\def\AlPh@Method@alph#1#2#3{%
  \expandafter\AlPh@CheckPositive
|   \number#3!%
+   \the\numexpr#3!%
    \AlPh@ProcessAlph
    {#1}{#2}%
}
%    \end{macrocode}
%    \end{macro}
%    \begin{macro}{\AlPh@ProcessAlph}
%    |#1|: current number\\
%    |#2|: number of symbols\\
%    |#3|: symbols provider
%    \begin{macrocode}
\def\AlPh@ProcessAlph#1#2#3{%
  \ifnum#1>#2 %
    \@ReturnAfterElseFi{%
      \expandafter\AlPh@StepAlph\number
        \intcalcInc{%
          \intcalcMod{\intcalcDec{#1}}{#2}%
        }%
      \expandafter!\number
        \intcalcDiv{\intcalcDec{#1}}{#2}%
      !{#2}{#3}%
    }%
  \else
    \@ReturnAfterFi{%
      #3{#1}%
    }%
  \fi
}
%    \end{macrocode}
%    \end{macro}
%    \begin{macro}{\AlPh@StepAlph}
%    |#1|: current last digit\\
%    |#2|: new current number\\
%    |#3|: number of symbols\\
%    |#4|: symbols provider
%    \begin{macrocode}
\def\AlPh@StepAlph#1!#2!#3#4{%
  \AlPh@ProcessAlph{#2}{#3}{#4}%
  #4{#1}%
}
%    \end{macrocode}
%    \end{macro}
%
% \subsubsection{Method `wrap'}
%
%    \begin{macro}{\AlPh@Method@wrap}
%    \noindent
%    |#1|: number of symbols\\
%    |#2|: symbols provider\\
%    |#3|: number to be converted
%    \begin{macrocode}
\def\AlPh@Method@wrap#1#2#3{%
  \expandafter\AlPh@CheckPositive
|   \number#3!%
+   \the\numexpr#3!%
    \AlPh@ProcessWrap
    {#1}{#2}%
}
%    \end{macrocode}
%    \end{macro}
%
%    \begin{macro}{\AlPh@ProcessWrap}
%    |#1|: number to be converted\\
%    |#2|: number of symbols\\
%    |#3|: symbols provider
%    \begin{macrocode}
\def\AlPh@ProcessWrap#1#2#3{%
  \ifnum#1>#2 %
    \@ReturnAfterElseFi{%
      \expandafter\AlPh@StepWrap\number
        \intcalcInc{\intcalcMod{\intcalcDec{#1}}{#2}}%
      !{#3}%
    }%
  \else
    \@ReturnAfterFi{%
      #3{#1}%
    }%
  \fi
}
%    \end{macrocode}
%    \end{macro}
%    \begin{macro}{\AlPh@StepWrap}
%    |#1|: final number\\
%    |#2|: symbols provider
%    \begin{macrocode}
\def\AlPh@StepWrap#1!#2{%
  #2{#1}%
}
%    \end{macrocode}
%    \end{macro}
%
% \subsubsection{Method `mult'}
%
%    After the number of symbols is exhausted,
%    repetitions of the symbol are used.
%    \begin{gather*}
%      x \Def \text{number to be converted}\\
%      n \Def \text{number of symbols}\\
%      r \Def \text{repetition length}\\
%      s \Def \text{symbol slot}\\
%      r = ((x - 1) \div n) + 1\\
%      s = ((x - 1) \mod n) + 1
%    \end{gather*}
%    \begin{macro}{\AlPh@Method@mult}
%    \noindent
%    |#1|: number of symbols\\
%    |#2|: symbols provider\\
%    |#3|: number to be converted
%    \begin{macrocode}
\def\AlPh@Method@mult#1#2#3{%
  \expandafter\AlPh@CheckPositive
|   \number#3!%
+   \the\numexpr#3!%
    \AlPh@ProcessMult
    {#1}{#2}%
}
%    \end{macrocode}
%    \end{macro}
%
%    \begin{macro}{\AlPh@ProcessMult}
%    |#1|: number to be converted\\
%    |#2|: number of symbols\\
%    |#3|: symbols provider
%    \begin{macrocode}
\def\AlPh@ProcessMult#1#2#3{%
  \ifnum#1>#2 %
    \@ReturnAfterElseFi{%
      \expandafter\AlPh@StepMult\romannumeral
        \intcalcInc{\intcalcDiv{\intcalcDec{#1}}{#2}}%
        000%
      \expandafter!\number
        \intcalcInc{\intcalcMod{\intcalcDec{#1}}{#2}}%
      !{#3}%
    }%
  \else
    \@ReturnAfterFi{%
      #3{#1}%
    }%
  \fi
}
%    \end{macrocode}
%    \end{macro}
%    \begin{macro}{\AlPh@StepMult}
%    |#1#2|: repetitions coded as list of character `m'\\
%    |#3|: symbol slot\\
%    |#4|: symbols provider
%    \begin{macrocode}
\def\AlPh@StepMult#1#2!#3!#4{%
  \ifx\\#2\\%
  \else
    \@ReturnAfterFi{%
      \AlPh@StepMult#2!#3!{#4}%
    }%
  \fi
  #4{#3}%
}
%    \end{macrocode}
%    \end{macro}
%
% \subsection{User interface}
%
%    \begin{macro}{\newalphalph}
%    Macro \cs{newalphalph} had three arguments in versions below 2.0.
%    For the new method argument we use an optional argument an first
%    position.\\
%    |#1|: cmd\\
%    |[#2]|: method name: |alph| (default), |wrap|, |mult|\\ % hash-ok
%    |#3|: symbols provider\\
%    |#4|: number of symbols
%    \begin{macrocode}
\AlPh@IfDefinable\newalphalph{%
  \def\newalphalph#1{%
    \AlPh@IfOptArg{%
      \AlPh@newalphalph{#1}%
    }{%
      \AlPh@newalphalph{#1}[alph]%
    }%
  }%
}
%    \end{macrocode}
%    \end{macro}
%    \begin{macro}{\AlPh@newalphalph}
%    |#1|: cmd
%    |#2|: method name\\
%    |#3|: symbols provider\\
%    |#4|: number of symbols
%    \begin{macrocode}
\def\AlPh@newalphalph#1[#2]#3#4{%
  \begingroup\expandafter\expandafter\expandafter\endgroup
  \expandafter\ifx\csname AlPh@Method@#2\endcsname\relax
    \AlPh@Error{%
      Unknown method %
|     `#2'%
+     `\detokenize{#2}'%
    }%
  \else
    \ifx\\#4\\%
      \AlPh@GetNumberOfSymbols{#3}%
      \ifcase\AlPh@Number
      \else
        \begingroup
          \escapechar=92 % backslash
          \@PackageInfo{alphalph}{%
            Number of symbols for \string#1 is \AlPh@Number
          }%
        \endgroup
        \expandafter\AlPh@NewAlphAlph
        \csname AlPh@Method@#2\expandafter\endcsname
        \AlPh@Number!{#1}{#3}%
      \fi
    \else
      \expandafter\AlPh@NewAlphAlph
      \csname AlPh@Method@#2\expandafter\endcsname
|     \number#4!%
+     \the\numexpr#4!%
      {#1}{#3}%
    \fi
  \fi
}%
%    \end{macrocode}
%    \end{macro}
%    \begin{macro}{\AlPh@NewAlphAlph}
%    |#1|: method macro\\
%    |#2|: number of symbols\\
%    |#3|: cmd\\
%    |#4|: symbols provider
%    \begin{macrocode}
\def\AlPh@NewAlphAlph#1#2!#3#4{%
  \AlPh@IfDefinable#3{%
    \ifnum#2>0 %
      \def#3{#1{#2}{#4}}%
    \else
      \AlPh@Error{%
          Definition of \string#3 failed,\MessageBreak
          because number of symbols (#2) is not positive%
        }%
    \fi
  }%
}
%    \end{macrocode}
%    \end{macro}
%
%    \begin{macro}{\AlphAlph}
%    \begin{macrocode}
\newalphalph\AlphAlph\alphalph@Alph{26}
%    \end{macrocode}
%    \end{macro}
%    \begin{macro}{\alphalph}
%    \begin{macrocode}
\newalphalph\alphalph\alphalph@alph{26}
%    \end{macrocode}
%    \end{macro}
%
%    \begin{macrocode}
\AlPh@AtEnd%
%</package>
%    \end{macrocode}
%
% \section{Test}
%
% \subsection{Catcode checks for loading}
%
%    \begin{macrocode}
%<*test1>
%    \end{macrocode}
%    \begin{macrocode}
\catcode`\{=1 %
\catcode`\}=2 %
\catcode`\#=6 %
\catcode`\@=11 %
\expandafter\ifx\csname count@\endcsname\relax
  \countdef\count@=255 %
\fi
\expandafter\ifx\csname @gobble\endcsname\relax
  \long\def\@gobble#1{}%
\fi
\expandafter\ifx\csname @firstofone\endcsname\relax
  \long\def\@firstofone#1{#1}%
\fi
\expandafter\ifx\csname loop\endcsname\relax
  \expandafter\@firstofone
\else
  \expandafter\@gobble
\fi
{%
  \def\loop#1\repeat{%
    \def\body{#1}%
    \iterate
  }%
  \def\iterate{%
    \body
      \let\next\iterate
    \else
      \let\next\relax
    \fi
    \next
  }%
  \let\repeat=\fi
}%
\def\RestoreCatcodes{}
\count@=0 %
\loop
  \edef\RestoreCatcodes{%
    \RestoreCatcodes
    \catcode\the\count@=\the\catcode\count@\relax
  }%
\ifnum\count@<255 %
  \advance\count@ 1 %
\repeat

\def\RangeCatcodeInvalid#1#2{%
  \count@=#1\relax
  \loop
    \catcode\count@=15 %
  \ifnum\count@<#2\relax
    \advance\count@ 1 %
  \repeat
}
\def\RangeCatcodeCheck#1#2#3{%
  \count@=#1\relax
  \loop
    \ifnum#3=\catcode\count@
    \else
      \errmessage{%
        Character \the\count@\space
        with wrong catcode \the\catcode\count@\space
        instead of \number#3%
      }%
    \fi
  \ifnum\count@<#2\relax
    \advance\count@ 1 %
  \repeat
}
\def\space{ }
\expandafter\ifx\csname LoadCommand\endcsname\relax
  \def\LoadCommand{\input alphalph.sty\relax}%
\fi
\def\Test{%
  \RangeCatcodeInvalid{0}{47}%
  \RangeCatcodeInvalid{58}{64}%
  \RangeCatcodeInvalid{91}{96}%
  \RangeCatcodeInvalid{123}{255}%
  \catcode`\@=12 %
  \catcode`\\=0 %
  \catcode`\%=14 %
  \LoadCommand
  \RangeCatcodeCheck{0}{36}{15}%
  \RangeCatcodeCheck{37}{37}{14}%
  \RangeCatcodeCheck{38}{47}{15}%
  \RangeCatcodeCheck{48}{57}{12}%
  \RangeCatcodeCheck{58}{63}{15}%
  \RangeCatcodeCheck{64}{64}{12}%
  \RangeCatcodeCheck{65}{90}{11}%
  \RangeCatcodeCheck{91}{91}{15}%
  \RangeCatcodeCheck{92}{92}{0}%
  \RangeCatcodeCheck{93}{96}{15}%
  \RangeCatcodeCheck{97}{122}{11}%
  \RangeCatcodeCheck{123}{255}{15}%
  \RestoreCatcodes
}
\Test
\csname @@end\endcsname
\end
%    \end{macrocode}
%    \begin{macrocode}
%</test1>
%    \end{macrocode}
%
% \section{Macro tests}
%
%    \begin{macrocode}
%<*test2>
\NeedsTeXFormat{LaTeX2e}
\nofiles
\documentclass{article}
%<*noetex>
\makeatletter
\let\saved@numexpr\numexpr
\newcommand*{\DisableNumexpr}{%
  \let\numexpr\@undefined
}
\newcommand*{\RestoreNumexpr}{%
  \let\numexpr\saved@numexpr
}
\DisableNumexpr
%</noetex>
\usepackage{alphalph}[2016/05/16]
%<noetex>\RestoreNumexpr
\usepackage{qstest}
\IncludeTests{*}
\LogTests{log}{*}{*}

\newcommand*{\TestCmd}[3]{%
  \setbox0=\hbox{%
%<noetex>    \DisableNumexpr
    \edef\TestString{#1{#2}}%
    \expandafter\Expect\expandafter{\TestString}{#3}%
    \edef\TestString{#1{#2} }%
    \expandafter\Expect\expandafter{\TestString}{#3 }%
  }%
  \Expect*{\the\wd0}{0.0pt}%
}

\makeatletter
\newalphalph\LaTeXAlphAlph\@Alph{26}
\newalphalph\LaTeXalphalph\@alph{26}
\newalphalph\AlphWrap[wrap]\alphalph@Alph{26}
\newalphalph\alphwrap[wrap]\alphalph@alph{26}
\newalphalph\LaTeXAlphWrap[wrap]\@Alph{26}
\newalphalph\LaTeXalphwrap[wrap]\@alph{26}
\def\LastSymbol#1{%
  \ifx\\#1\\%
  \else
    \@LastSymbol#1\@nil
  \fi
}
\def\@LastSymbol#1#2\@nil{%
  \ifx\\#2\\%
    #1%
  \else
    \@LastSymbol#2\@nil
  \fi
}
\makeatother
\newcommand*{\TestAlph}[2]{%
  \uppercase{\TestCallCmd\AlphAlph{#2}}{#1}%
  \lowercase{\TestCallCmd\alphalph{#2}}{#1}%
  \uppercase{\TestCallCmd\LaTeXAlphAlph{#2}}{#1}%
  \lowercase{\TestCallCmd\LaTeXalphalph{#2}}{#1}%
  \edef\WrapString{\LastSymbol{#2}}%
  \expandafter\TestAlphWrap\expandafter{\WrapString}{#1}%
}
\newcommand*{\TestAlphWrap}[2]{%
  \uppercase{\TestCallCmd\AlphWrap{#1}}{#2}%
  \lowercase{\TestCallCmd\alphwrap{#1}}{#2}%
  \uppercase{\TestCallCmd\LaTeXAlphWrap{#1}}{#2}%
  \lowercase{\TestCallCmd\LaTeXalphwrap{#1}}{#2}%
}
\newcommand*{\TestCallCmd}[3]{%
  \TestCmd#1{#3}{#2}%
}
\begin{qstest}{AlphSymbols}{alphalph, AlphAlph, symbols}
  \TestAlph{1}{a}%
  \TestAlph{2}{b}%
  \TestAlph{3}{c}%
  \TestAlph{4}{d}%
  \TestAlph{5}{e}%
  \TestAlph{6}{f}%
  \TestAlph{7}{g}%
  \TestAlph{8}{h}%
  \TestAlph{9}{i}%
  \TestAlph{10}{j}%
  \TestAlph{11}{k}%
  \TestAlph{12}{l}%
  \TestAlph{13}{m}%
  \TestAlph{14}{n}%
  \TestAlph{15}{o}%
  \TestAlph{16}{p}%
  \TestAlph{17}{q}%
  \TestAlph{18}{r}%
  \TestAlph{19}{s}%
  \TestAlph{20}{t}%
  \TestAlph{21}{u}%
  \TestAlph{22}{v}%
  \TestAlph{23}{w}%
  \TestAlph{24}{x}%
  \TestAlph{25}{y}%
  \TestAlph{26}{z}%
\end{qstest}
\begin{qstest}{AlphRange}{alphalph, range}
  \TestAlph{0}{}%
  \TestAlph{-1}{}%
  \TestAlph{-2147483647}{}%
  \TestAlph{27}{aa}%
  \TestAlph{28}{ab}%
  \TestAlph{52}{az}%
  \TestAlph{53}{ba}%
  \TestAlph{78}{bz}%
  \TestAlph{79}{ca}%
  \TestAlph{702}{zz}%
  \TestAlph{703}{aaa}%
  \TestAlph{2147483647}{fxshrxw}%
\end{qstest}

\makeatletter
\newcommand*{\myvocals}[1]{%
  \ifcase#1X\or A\or E\or I\or O\or U\else Y\fi
}
\makeatother
\newalphalph\vocalsvocals\myvocals{5}
\newcommand*{\TestVocals}{%
  \TestCmd\vocalsvocals
}
\begin{qstest}{vocals}{vocals}
  \TestVocals{0}{}%
  \TestVocals{1}{A}%
  \TestVocals{2}{E}%
  \TestVocals{3}{I}%
  \TestVocals{4}{O}%
  \TestVocals{5}{U}%
  \TestVocals{6}{AA}%
  \TestVocals{7}{AE}%
  \TestVocals{8}{AI}%
  \TestVocals{9}{AO}%
  \TestVocals{10}{AU}%
  \TestVocals{11}{EA}%
  \TestVocals{24}{OO}%
  \TestVocals{25}{OU}%
  \TestVocals{26}{UA}%
  \TestVocals{29}{UO}%
  \TestVocals{30}{UU}%
  \TestVocals{31}{AAA}%
  \TestVocals{155}{UUU}%
  \TestVocals{156}{AAAA}%
  \TestVocals{2147483647}{AIIOOEEIOIIUOE}%
\end{qstest}

\makeatletter
\newalphalph\AlphMult[mult]{\alphalph@Alph}{26}
\newalphalph\alphmult[mult]{\alphalph@alph}{26}
\newalphalph\LaTeXAlphMult[mult]{\@Alph}{26}
\newalphalph\LaTeXalphmult[mult]{\@alph}{26}
\makeatother
\newcommand*{\TestMult}[2]{%
  \uppercase{\TestCallCmd\AlphMult{#2}}{#1}%
  \lowercase{\TestCallCmd\alphmult{#2}}{#1}%
  \uppercase{\TestCallCmd\LaTeXAlphMult{#2}}{#1}%
  \lowercase{\TestCallCmd\LaTeXalphmult{#2}}{#1}%
}
\begin{qstest}{mult}{mult}
  \TestMult{0}{}%
  \TestMult{-1}{}%
  \TestMult{-2147483647}{}%
  \TestMult{1}{a}%
  \TestMult{2}{b}%
  \TestMult{26}{z}%
  \TestMult{27}{aa}%
  \TestMult{28}{bb}%
  \TestMult{52}{zz}%
  \TestMult{53}{aaa}%
  \TestMult{54}{bbb}%
  \TestMult{259}{yyyyyyyyyy}%
  \TestMult{260}{zzzzzzzzzz}%
  \TestMult{261}{aaaaaaaaaaa}%
  \TestMult{262}{bbbbbbbbbbb}%
\end{qstest}

\def\myvocalsB#1{%
  \ifcase#1\or A\or E\or I\or O\or U\fi
}
\begin{qstest}{symbolnum}{symbolnum}
  \makeatletter
  \def\Test#1#2{%
    \let\TestCmd\relax
    \newalphalph\TestCmd{#1}{}%
    \Expect*{\AlPh@Number}{#2}%
  }%
  \Test\@alph{26}%
  \Test\@Alph{26}%
  \Test\@fnsymbol{9}%
  \Test\myvocalsB{5}%
  \Test\alphalph@alph{26}%
  \Test\alphalph@Alph{26}%
\end{qstest}

\begin{qstest}{list}{list}
  \makeatletter
  \def\catch#1\relax{%
    \def\FoundList{\catch#1}%
  }%
  \def\Test[#1]#2#3#4{%
    \let\testcmd\relax
    \newalphalph\testcmd[{#1}]{\catch}{#2}%
    \testcmd{#3}|\relax
    \expandafter\Expect\expandafter{\FoundList}{#4|}%
    %
    \let\SavedCatch\catch
    \def\catch{\noexpand\catch\noexpand\foo}%
    \edef\Result{#4|}%
    \@onelevel@sanitize\Result
    \let\catch\SavedCatch
    \let\testcmd\relax
    \newalphalph\testcmd[{#1}]{\catch\foo}{#2}%
    \testcmd{#3}|\relax
    \@onelevel@sanitize\FoundList
    \Expect*{\FoundList}*{\Result}%
  }%
  \Test[alph]{26}{3}{\catch{3}}%
  \Test[alph]{26}{12}{\catch{12}}%
  \Test[alph]{26}{27}{\catch{1}\catch{1}}%
  \Test[alph]{26}{78}{\catch{2}\catch{26}}%
  \Test[wrap]{26}{7}{\catch{7}}%
  \Test[wrap]{26}{14}{\catch{14}}%
  \Test[wrap]{26}{80}{\catch{2}}%
  \Test[wrap]{26}{700}{\catch{24}}%
  \Test[mult]{26}{4}{\catch{4}}%
  \Test[mult]{26}{17}{\catch{17}}%
  \Test[mult]{26}{54}{\catch{2}\catch{2}\catch{2}}%
\end{qstest}

\begin{document}
\end{document}
%</test2>
%    \end{macrocode}
%
% \section{Installation}
%
% \subsection{Download}
%
% \paragraph{Package.} This package is available on
% CTAN\footnote{\CTANpkg{alphalph}}:
% \begin{description}
% \item[\CTAN{macros/latex/contrib/oberdiek/alphalph.dtx}] The source file.
% \item[\CTAN{macros/latex/contrib/oberdiek/alphalph.pdf}] Documentation.
% \end{description}
%
%
% \paragraph{Bundle.} All the packages of the bundle `oberdiek'
% are also available in a TDS compliant ZIP archive. There
% the packages are already unpacked and the documentation files
% are generated. The files and directories obey the TDS standard.
% \begin{description}
% \item[\CTANinstall{install/macros/latex/contrib/oberdiek.tds.zip}]
% \end{description}
% \emph{TDS} refers to the standard ``A Directory Structure
% for \TeX\ Files'' (\CTAN{tds/tds.pdf}). Directories
% with \xfile{texmf} in their name are usually organized this way.
%
% \subsection{Bundle installation}
%
% \paragraph{Unpacking.} Unpack the \xfile{oberdiek.tds.zip} in the
% TDS tree (also known as \xfile{texmf} tree) of your choice.
% Example (linux):
% \begin{quote}
%   |unzip oberdiek.tds.zip -d ~/texmf|
% \end{quote}
%
% \paragraph{Script installation.}
% Check the directory \xfile{TDS:scripts/oberdiek/} for
% scripts that need further installation steps.
% Package \xpackage{attachfile2} comes with the Perl script
% \xfile{pdfatfi.pl} that should be installed in such a way
% that it can be called as \texttt{pdfatfi}.
% Example (linux):
% \begin{quote}
%   |chmod +x scripts/oberdiek/pdfatfi.pl|\\
%   |cp scripts/oberdiek/pdfatfi.pl /usr/local/bin/|
% \end{quote}
%
% \subsection{Package installation}
%
% \paragraph{Unpacking.} The \xfile{.dtx} file is a self-extracting
% \docstrip\ archive. The files are extracted by running the
% \xfile{.dtx} through \plainTeX:
% \begin{quote}
%   \verb|tex alphalph.dtx|
% \end{quote}
%
% \paragraph{TDS.} Now the different files must be moved into
% the different directories in your installation TDS tree
% (also known as \xfile{texmf} tree):
% \begin{quote}
% \def\t{^^A
% \begin{tabular}{@{}>{\ttfamily}l@{ $\rightarrow$ }>{\ttfamily}l@{}}
%   alphalph.sty & tex/generic/oberdiek/alphalph.sty\\
%   alphalph.pdf & doc/latex/oberdiek/alphalph.pdf\\
%   test/alphalph-test1.tex & doc/latex/oberdiek/test/alphalph-test1.tex\\
%   test/alphalph-test2.tex & doc/latex/oberdiek/test/alphalph-test2.tex\\
%   test/alphalph-test3.tex & doc/latex/oberdiek/test/alphalph-test3.tex\\
%   alphalph.dtx & source/latex/oberdiek/alphalph.dtx\\
% \end{tabular}^^A
% }^^A
% \sbox0{\t}^^A
% \ifdim\wd0>\linewidth
%   \begingroup
%     \advance\linewidth by\leftmargin
%     \advance\linewidth by\rightmargin
%   \edef\x{\endgroup
%     \def\noexpand\lw{\the\linewidth}^^A
%   }\x
%   \def\lwbox{^^A
%     \leavevmode
%     \hbox to \linewidth{^^A
%       \kern-\leftmargin\relax
%       \hss
%       \usebox0
%       \hss
%       \kern-\rightmargin\relax
%     }^^A
%   }^^A
%   \ifdim\wd0>\lw
%     \sbox0{\small\t}^^A
%     \ifdim\wd0>\linewidth
%       \ifdim\wd0>\lw
%         \sbox0{\footnotesize\t}^^A
%         \ifdim\wd0>\linewidth
%           \ifdim\wd0>\lw
%             \sbox0{\scriptsize\t}^^A
%             \ifdim\wd0>\linewidth
%               \ifdim\wd0>\lw
%                 \sbox0{\tiny\t}^^A
%                 \ifdim\wd0>\linewidth
%                   \lwbox
%                 \else
%                   \usebox0
%                 \fi
%               \else
%                 \lwbox
%               \fi
%             \else
%               \usebox0
%             \fi
%           \else
%             \lwbox
%           \fi
%         \else
%           \usebox0
%         \fi
%       \else
%         \lwbox
%       \fi
%     \else
%       \usebox0
%     \fi
%   \else
%     \lwbox
%   \fi
% \else
%   \usebox0
% \fi
% \end{quote}
% If you have a \xfile{docstrip.cfg} that configures and enables \docstrip's
% TDS installing feature, then some files can already be in the right
% place, see the documentation of \docstrip.
%
% \subsection{Refresh file name databases}
%
% If your \TeX~distribution
% (\teTeX, \mikTeX, \dots) relies on file name databases, you must refresh
% these. For example, \teTeX\ users run \verb|texhash| or
% \verb|mktexlsr|.
%
% \subsection{Some details for the interested}
%
% \paragraph{Attached source.}
%
% The PDF documentation on CTAN also includes the
% \xfile{.dtx} source file. It can be extracted by
% AcrobatReader 6 or higher. Another option is \textsf{pdftk},
% e.g. unpack the file into the current directory:
% \begin{quote}
%   \verb|pdftk alphalph.pdf unpack_files output .|
% \end{quote}
%
% \paragraph{Unpacking with \LaTeX.}
% The \xfile{.dtx} chooses its action depending on the format:
% \begin{description}
% \item[\plainTeX:] Run \docstrip\ and extract the files.
% \item[\LaTeX:] Generate the documentation.
% \end{description}
% If you insist on using \LaTeX\ for \docstrip\ (really,
% \docstrip\ does not need \LaTeX), then inform the autodetect routine
% about your intention:
% \begin{quote}
%   \verb|latex \let\install=y\input{alphalph.dtx}|
% \end{quote}
% Do not forget to quote the argument according to the demands
% of your shell.
%
% \paragraph{Generating the documentation.}
% You can use both the \xfile{.dtx} or the \xfile{.drv} to generate
% the documentation. The process can be configured by the
% configuration file \xfile{ltxdoc.cfg}. For instance, put this
% line into this file, if you want to have A4 as paper format:
% \begin{quote}
%   \verb|\PassOptionsToClass{a4paper}{article}|
% \end{quote}
% An example follows how to generate the
% documentation with pdf\LaTeX:
% \begin{quote}
%\begin{verbatim}
%pdflatex alphalph.dtx
%makeindex -s gind.ist alphalph.idx
%pdflatex alphalph.dtx
%makeindex -s gind.ist alphalph.idx
%pdflatex alphalph.dtx
%\end{verbatim}
% \end{quote}
%
% \begin{History}
%   \begin{Version}{1999/03/19 v0.1}
%   \item
%     The first version was built as a response to a
%     \URL{\link{question}}^^A
%     {https://groups.google.com/group/comp.text.tex/msg/17a74cd721641038}^^A
%     of \NameEmail{Will Douglas}{william.douglas@wolfson.ox.ac.uk}
%     and the
%     \URL{\link{request}}^^A
%     {https://groups.google.com/group/comp.text.tex/msg/8f9768825640315f}^^A
%     of \NameEmail{Donald Arsenau}{asnd@reg.triumf.ca},
%     published in the newsgroup
%     \href{news:comp.text.tex}{comp.text.tex}:
%     \URL{``\link{Re: alph counters \textgreater\ 26}''}^^A
%     {https://groups.google.com/group/comp.text.tex/msg/cec563eef8bf65d0}
%   \item
%     Copyright: LPPL (\CTAN{macros/latex/base/lppl.txt})
%   \end{Version}
%   \begin{Version}{1999/04/12 v1.0}
%   \item
%     Documentation added in dtx format.
%   \item
%     \hologo{eTeX} support added.
%   \end{Version}
%   \begin{Version}{1999/04/13 v1.1}
%   \item
%     Minor documentation change.
%   \item
%     First CTAN release.
%   \end{Version}
%   \begin{Version}{1999/06/26 v1.2}
%   \item
%     First generic code about \cmd{\ProvidesPackage} improved.
%   \item
%     Documentation: Installation part revised.
%   \end{Version}
%   \begin{Version}{2006/02/20 v1.3}
%   \item
%     Reload check (for \hologo{plainTeX})
%   \item
%     New DTX framework.
%   \item
%     LPPL 1.3
%   \end{Version}
%   \begin{Version}{2006/05/30 v1.4}
%   \item
%     \cs{newalphalph} added.
%   \end{Version}
%   \begin{Version}{2007/04/11 v1.5}
%   \item
%     Line ends sanitized.
%   \end{Version}
%   \begin{Version}{2007/09/09 v2.0}
%   \item
%     New implementation that uses package \cs{intcalc}.
%     This removes the dependency on \hologo{eTeX}.
%   \item
%     \cs{newalphalph} is extended to support new methods `wrap' and 'multi'.
%   \item
%     Documentation rewritten.
%   \end{Version}
%   \begin{Version}{2008/08/11 v2.1}
%   \item
%     Code is not changed.
%   \item
%     URLs updated from \texttt{www.dejanews.com}
%     to \texttt{groups.google.com}.
%   \end{Version}
%   \begin{Version}{2010/03/01 v2.2}
%   \item
%     Compatibility with \hologo{iniTeX}.
%   \end{Version}
%   \begin{Version}{2010/04/18 v2.3}
%   \item
%     Documentation fixes (Martin M\"unch).
%   \end{Version}
%   \begin{Version}{2011/05/13 v2.4}
%   \item
%     Documentation fixes (Jim Diamond) and using package
%     \xpackage{hologo} for the documentation.
%   \item
%     Catalogue file added.
%   \end{Version}
%   \begin{Version}{2016/05/16 v2.5}
%   \item
%     Documentation updates.
%   \end{Version}
% \end{History}
%
% \PrintIndex
%
% \Finale
\endinput

%        (quote the arguments according to the demands of your shell)
%
% Documentation:
%    (a) If alphalph.drv is present:
%           latex alphalph.drv
%    (b) Without alphalph.drv:
%           latex alphalph.dtx; ...
%    The class ltxdoc loads the configuration file ltxdoc.cfg
%    if available. Here you can specify further options, e.g.
%    use A4 as paper format:
%       \PassOptionsToClass{a4paper}{article}
%
%    Programm calls to get the documentation (example):
%       pdflatex alphalph.dtx
%       makeindex -s gind.ist alphalph.idx
%       pdflatex alphalph.dtx
%       makeindex -s gind.ist alphalph.idx
%       pdflatex alphalph.dtx
%
% Installation:
%    TDS:tex/generic/oberdiek/alphalph.sty
%    TDS:doc/latex/oberdiek/alphalph.pdf
%    TDS:doc/latex/oberdiek/test/alphalph-test1.tex
%    TDS:doc/latex/oberdiek/test/alphalph-test2.tex
%    TDS:doc/latex/oberdiek/test/alphalph-test3.tex
%    TDS:source/latex/oberdiek/alphalph.dtx
%
%<*ignore>
\begingroup
  \catcode123=1 %
  \catcode125=2 %
  \def\x{LaTeX2e}%
\expandafter\endgroup
\ifcase 0\ifx\install y1\fi\expandafter
         \ifx\csname processbatchFile\endcsname\relax\else1\fi
         \ifx\fmtname\x\else 1\fi\relax
\else\csname fi\endcsname
%</ignore>
%<*install>
\input docstrip.tex
\Msg{************************************************************************}
\Msg{* Installation}
\Msg{* Package: alphalph 2016/05/16 v2.5 Convert numbers to letters (HO)}
\Msg{************************************************************************}

\keepsilent
\askforoverwritefalse

\let\MetaPrefix\relax
\preamble

This is a generated file.

Project: alphalph
Version: 2016/05/16 v2.5

Copyright (C) 1999, 2006-2008, 2010, 2011 by
   Heiko Oberdiek <heiko.oberdiek at googlemail.com>

This work may be distributed and/or modified under the
conditions of the LaTeX Project Public License, either
version 1.3c of this license or (at your option) any later
version. This version of this license is in
   http://www.latex-project.org/lppl/lppl-1-3c.txt
and the latest version of this license is in
   http://www.latex-project.org/lppl.txt
and version 1.3 or later is part of all distributions of
LaTeX version 2005/12/01 or later.

This work has the LPPL maintenance status "maintained".

This Current Maintainer of this work is Heiko Oberdiek.

The Base Interpreter refers to any `TeX-Format',
because some files are installed in TDS:tex/generic//.

This work consists of the main source file alphalph.dtx
and the derived files
   alphalph.sty, alphalph.pdf, alphalph.ins, alphalph.drv,
   alphalph-test1.tex, alphalph-test2.tex, alphalph-test3.tex.

\endpreamble
\let\MetaPrefix\DoubleperCent

\generate{%
  \file{alphalph.ins}{\from{alphalph.dtx}{install}}%
  \file{alphalph.drv}{\from{alphalph.dtx}{driver}}%
  \usedir{tex/generic/oberdiek}%
  \file{alphalph.sty}{\from{alphalph.dtx}{package}}%
%  \usedir{doc/latex/oberdiek/test}%
%  \file{alphalph-test1.tex}{\from{alphalph.dtx}{test1}}%
%  \file{alphalph-test2.tex}{\from{alphalph.dtx}{test2,etex}}%
%  \file{alphalph-test3.tex}{\from{alphalph.dtx}{test2,noetex}}%
  \nopreamble
  \nopostamble
%  \usedir{source/latex/oberdiek/catalogue}%
%  \file{alphalph.xml}{\from{alphalph.dtx}{catalogue}}%
}

\catcode32=13\relax% active space
\let =\space%
\Msg{************************************************************************}
\Msg{*}
\Msg{* To finish the installation you have to move the following}
\Msg{* file into a directory searched by TeX:}
\Msg{*}
\Msg{*     alphalph.sty}
\Msg{*}
\Msg{* To produce the documentation run the file `alphalph.drv'}
\Msg{* through LaTeX.}
\Msg{*}
\Msg{* Happy TeXing!}
\Msg{*}
\Msg{************************************************************************}

\endbatchfile
%</install>
%<*ignore>
\fi
%</ignore>
%<*driver>
\NeedsTeXFormat{LaTeX2e}
\ProvidesFile{alphalph.drv}%
  [2016/05/16 v2.5 Convert numbers to letters (HO)]%
\documentclass{ltxdoc}
\usepackage{alphalph}[2016/05/16]
\makeatletter
\newalphalph{\fnsymbolwrap}[wrap]{\@fnsymbol}{}
\newcommand*{\fnsymbolsingle}[1]{%
  \ensuremath{%
    \ifcase#1%
    \or *%
    \or \dagger
    \or \ddagger
    \or \mathsection
    \or \mathparagraph
    \else
      \@ctrerr
    \fi
  }%
}
\makeatother
\newalphalph{\fnsymbolmult}[mult]{\fnsymbolsingle}{}
\usepackage{amsmath}
\DeclareMathOperator{\opDiv}{div}
\DeclareMathOperator{\opMod}{mod}
\newcommand*{\Def}{%
  \ensuremath{%
    \mathrel{\mathop{:}}=%
  }%
}
\usepackage{holtxdoc}[2011/11/22]
\begin{document}
  \DocInput{alphalph.dtx}%
\end{document}
%</driver>
% \fi
%
%
% \CharacterTable
%  {Upper-case    \A\B\C\D\E\F\G\H\I\J\K\L\M\N\O\P\Q\R\S\T\U\V\W\X\Y\Z
%   Lower-case    \a\b\c\d\e\f\g\h\i\j\k\l\m\n\o\p\q\r\s\t\u\v\w\x\y\z
%   Digits        \0\1\2\3\4\5\6\7\8\9
%   Exclamation   \!     Double quote  \"     Hash (number) \#
%   Dollar        \$     Percent       \%     Ampersand     \&
%   Acute accent  \'     Left paren    \(     Right paren   \)
%   Asterisk      \*     Plus          \+     Comma         \,
%   Minus         \-     Point         \.     Solidus       \/
%   Colon         \:     Semicolon     \;     Less than     \<
%   Equals        \=     Greater than  \>     Question mark \?
%   Commercial at \@     Left bracket  \[     Backslash     \\
%   Right bracket \]     Circumflex    \^     Underscore    \_
%   Grave accent  \`     Left brace    \{     Vertical bar  \|
%   Right brace   \}     Tilde         \~}
%
% \GetFileInfo{alphalph.drv}
%
% \title{The \xpackage{alphalph} package}
% \date{2016/05/16 v2.5}
% \author{Heiko Oberdiek\thanks
% {Please report any issues at \url{https://github.com/ho-tex/oberdiek/issues}}\\
% \xemail{heiko.oberdiek at googlemail.com}}
%
% \maketitle
%
% \begin{abstract}
% The package provides methods to represent numbers with a limited
% set of symbols. Both \hologo{LaTeX} and \hologo{plainTeX} are supported.
% \end{abstract}
%
% \tableofcontents
%
% \section{Documentation}
%
% \subsection{Introduction}
% \hologo{LaTeX} counters can be represented in different ways
% by using presentation commands:
% \begin{quote}
%   \cs{arabic}, \cs{roman}, \cs{Roman},\\
%   \cs{alph}, \cs{Alph}, \cs{fnsymbol}
% \end{quote}
% The ranges of supported counter values are more or less
% restricted. Only \cs{arabic} can be used with any counter
% value \hologo{TeX} supports.
% \begin{quote}
% \catcode`\|=12 %
% \begin{tabular}{@{}l|l|l|l@{}}
% Presentation & Supported & Ignored & Error message\\
% command      & domain    & values  & ``Counter too large''\\
% \hline
% \cs{arabic}
%   & \ttfamily -MAX..MAX & &\\
% \cs{roman}, \cs{Roman}
%   & \ttfamily 1..MAX & \ttfamily -MAX..0 &\\
% \cs{alph}, \cs{Alph}
%   & \ttfamily 1..26 & 0 & \ttfamily -MAX..-1, 27..MAX\\
% \cs{fnsymbol}
%   & \ttfamily 1..9 & 0 & \ttfamily -MAX..-1, 10..MAX\\
% \hline
% \end{tabular}\\[1ex]
% \texttt{MAX} = \texttt{2147483647}
% \end{quote}
% Ordinal numbers are often used in documents: numbering
% of chapters, sections, figures, footnotes and so on.
% The layouter chooses \cs{Alph} for chapter numbers
% and \cs{fnsymbol} for footnotes. But what can be done
% if there are more than 26 chapters or more than 10 footnotes?
% This package \xpackage{alphalph} allows to define new
% presentation commands. They rely on a existing command
% and define presentations for values greater the limits.
% Three different methods are provided by the package.
% In the following use cases they are presentated.
%
% \subsection{Use cases}
%
% \subsubsection{Number system based on symbols}
%
% Asume you are writing a book and your lecturer demands
% that chapter numbers must be letters. But you have already
% 30 chapters and you have only 26 letters?
%
% In the decimal system the situation would be clear.
% If you run out of digits, you are using more digits
% to represent a number. This method can be also be used
% for letters. After chapter 26 with |Z| we us |AA|, |AB|, |AC|, and |AD|
% for the remaining chapters.
%
% Happily this package already defines this presentation command:
%\begin{quote}
%\begin{verbatim}
%\usepackage{alphalph}
%\renewcommand*{\thechapter}{%
%  \AlphAlph{\value{chapter}}%
%}
%\end{verbatim}
%\end{quote}
% \cs{AlphAlph} generates:
%   \AlphAlph{1}, \AlphAlph{2}, \AlphAlph{3}, \dots,
%   \AlphAlph{26}, \AlphAlph{27}, \AlphAlph{28}, \dots
%
% The other presentation command is \cs{alphalph} for lowercase letters.
%
% \subsubsection{Wrap symbols around}
%
% Nine footnote symbols are quite a few.
% Too soon the symbols are consumed
% and \hologo{LaTeX} complains with the error ``Counter too large''.
% However, it could be acceptable to start again with the symbols
% from the beginning, especially if there are less than nine symbols
% on a page.
% This could be achieved by a counter reset.
% But finding the right place can be difficult or needs manual
% actions. Also a
% unique counter value can be desirable (e.g. for generating
% unique anchor/link names).
% Package \xpackage{alphalph} allows you to define
% a macro that implements a ``wrap around'', but letting
% the value of the counter untouched:
%\begin{quote}
%\begin{verbatim}
%\usepackage{alphalph}
%\makeatletter
%\newalphalph{\fnsymbolwrap}[wrap]{\@fnsymbol}{}
%\makeatother
%\renewcommand*{\thefootnote}{%
%  \fnsymbolwrap{\value{footnote}}%
%}
%\end{verbatim}
%\end{quote}
% \cs{fnsymbolwrap} generates:
% \fnsymbolwrap{1}~(1), \fnsymbolwrap{2}~(2), \fnsymbolwrap{3}~(3), \dots,
% \fnsymbolwrap{9}~(9), \fnsymbolwrap{10}~(10), \fnsymbolwrap{11}~11, \dots
%
% \subsubsection{Multiple symbols}
%
% \Hologo{LaTeX}'s standard set of footnote symbols contains doubled
% symbols at the higher positions. Could this principle
% be generalized?
% Yes, but first we need a clean footnote symbol list without
% doubled entries, example:
%\begin{quote}
%\begin{verbatim}
%\usepackage{alphalph}
%\makeatletter
%\newcommand*{\fnsymbolsingle}[1]{%
%  \ensuremath{%
%    \ifcase#1%
%    \or *%
%    \or \dagger
%    \or \ddagger
%    \or \mathsection
%    \or \mathparagraph
%    \else
%      \@ctrerr
%    \fi
%  }%
%}
%\makeatother
%\newalphalph{\fnsymbolmult}[mult]{\fnsymbolsingle}{}
%\renewcommand*{\thefootnote}{%
%  \fnsymbolmult{\value{footnote}}%
%}
%\end{verbatim}
%\end{quote}
% The own definition of \cs{fnsymbolsingle} has the
% advantage that this list can easily modified.
% Otherwise you can use \cs{@fnsymbol} directly,
% because it uses the same first five symbols.
%\begin{quote}
%\begin{verbatim}
%\usepackage{alphalph}
%\makeatletter
%\newalphalph{\fnsymbolmult}[mult]{\@fnsymbol}{5}
%\makeatother
%\renewcommand*{\thefootnote}{%
%  \fnsymbolmult{\value{footnote}}%
%}
%\end{verbatim}
%\end{quote}
% \cs{fnsymbolmult} generates:
% \fnsymbolmult{1}~(1), \fnsymbolmult{2}~(2), \fnsymbolmult{3}~(3),
% \fnsymbolmult{4}~(4), \fnsymbolmult{5}~(5), \fnsymbolmult{6}~(6),
% \dots, \fnsymbolmult{16}~16, \fnsymbolmult{17}~17, \dots
%
% \bigskip
% The same method can also be used for the chapter problem
% in the first discussed use case:
%\begin{quote}
%\begin{verbatim}
%\usepackage{alphalph}
%\makeatletter
%\newalphalph{\AlphMult}[mult]{\@Alph}{26}
%\makeatother
%\renewcommand*{\chapter}{%
%  \AlphMult{\value{chapter}}%
%}
%\end{verbatim}
%\end{quote}
% \cs{AlphMult} then generates AA, BB, CC, and DD for chapters 27--30.
%
% \subsection{Glossary}
%
% \begin{description}
% \item[Counter presentation command] is a macro that
%   expects a \hologo{LaTeX} counter name as argument.
%   Numbers cannot be used. Examples:
%   \cs{arabic}, \cs{alph}, \cs{fnsymbol}.
% \item[Number presentation command] is a macro that
%   expects a number as argument. A number is anything
%   that \hologo{TeX} accepts as number including \cs{value}.
%   Examples: \cs{alphalph}, \cs{AlphAlph}, \cs{alphalph@alph}
%
%   However, \cs{alph} or \cs{fnsymbol} are not number
%   presentation commands because they expect a counter
%   name as argument. Happily \hologo{LaTeX} counter presentation
%   commands internally uses number presentation commands
%   with the same name, but prefixed by `@'. Thus
%   \cs{@alph}, \cs{@fnsymbol} are number presentation commands.
% \item[Symbols provider] is a command that can be used
%   to get a list of symbols. For example, \cs{@Alph}
%   provides the 26 uppercase letters from `A' to `Z'.
%   Basically a symbol provider is a number presentation
%   command, usually with a limited range.
% \item[Number of symbols] is the number of the
%   last symbol slot of a symbol provider. Thus \cs{@Alph}
%   generates 26 symbols, \cs{@fnsymbol} provides 9 symbols.
% \end{description}
%
% \subsection{Package usage}
%
% The package \Package{alphalph} can be used with both \hologo{plainTeX}
% and \hologo{LaTeX}:
% \begin{description}
% \item[\hologo{plainTeX}:] |\input alphalph.sty|
% \item[\hologo{LaTeX2e}:]   |\usepackage{alphalph}|\\
%      There aren't any options.
% \end{description}
%
% \subsection{User commands}
%
% \begin{declcs}{AlphAlph} \M{number}\\
%   \cs{alphalph} \M{number}
% \end{declcs}
% Both macros are number presentation commands that
% expects a number as argument. \hologo{LaTeX} counters
% are used with \cs{value}.
%
% The macros represents a number by letters.
% First single letters |A..Z| are used, then
% two letters |AA..ZZ|, three letters |AAA...ZZZ|, \dots
% follow.
%
% Macro \cs{AlphAlph} uses uppercase letters,
% \cs{alphalph} generates the lowercase variant.
% \begin{quote}
%   \catcode`\|=12
%   \def\entry#1{
%     \texttt{#1}
%     & \texttt{\AlphAlph{#1}}
%     & \texttt{\alphalph{#1}}
%   }
%   \begin{tabular}{@{}l|l|l@{}}
%     \meta{number} & \verb|\AlphAlph{|\meta{number}\verb|}|
%                   & \verb|\alphalph{|\meta{number}\verb|}|\\
%     \hline
%     \entry{1}\\
%     \entry{2}\\
%     \entry{26}\\
%     \entry{27}\\
%     \entry{30}\\
%     \entry{2000}\\
%     \entry{3752127}\\
%     \entry{10786572}\\
%     \entry{2147483647}\\
%   \end{tabular}
% \end{quote}
%
% \begin{declcs}{newalphalph}
%   \M{cmd}
%   |[|\meta{method}|]|
%   \M{symbols provider}
%   \M{number of symbols}
% \end{declcs}
% Macro \cs{newalphalph} defines \meta{cmd} as new
% number presentation command. Like \cs{newcommand} an error
% is thrown, if macro \meta{cmd} already exists.
%
% The \meta{method} is one of |alph|, |wrap|, or |mult|.
% The default is |alph|.
%
% As symbol provider a number presentation command
% can be used, e.g. \cs{@fnsymbol}, \cs{@Alph}, or
% \cs{alphalph@alph}.
%
% The last argument is the number of symbols.
% If the argument is empty, then \cs{newalphalph} tries
% to find this number itself. \hologo{LaTeX}'s number presentation
% commands throw an error message, if the number is too large.
% This error message is put in a macro \cs{@ctrerr}.
% Thus \cs{newalphalph} calls the symbol provider and tests
% a number by typesetting it in a temporary box. The error
% macro \cs{@ctrerr} is catched, it proofs that the number
% is not supported. Also if the width of the result is zero
% the number is considered as unavailable.
%
% The empty argument is useful for potentially variable lists.
% However if the end cannot be detected, then the number of
% symbols must be given. This is also a lot faster. Therefore don't
% let the argument empty without reason.
%
% \subsection{Programmer commands}
%
% \begin{declcs}{alphalph@Alph} \M{number}\\
%   \cs{alphalph@alph} \M{number}
% \end{declcs}
% They are basically the same as \cs{@Alph} and \cs{@alph}.
% Some languages of package \xpackage{babel} redefine
% \hologo{LaTeX}'s macros to include some font setup that
% breaks expandibility. Therefore \cs{AlphAlph} and \cs{alphalph}
% are based on \cs{alphalph@Alph} and \cs{alphalph@alph} to get
% the letters. The behaviour of these symbol providers for numbers
% outside the range |1..26| is undefined.
%
% \subsection{Design principles}
%
% \subsubsection{Number presentation commands}
%
% All number presentation commands that this package defines
% (including \cs{alphalph} and \cs{AlphAlph}) have the following
% properties:
% \begin{itemize}
% \item They are fully expandable. This means that they can safely
%       \begin{itemize}
%       \item be written to a file,
%       \item used in moving arguments (\hologo{LaTeX}: they are \emph{robust}),
%       \item used in a \cmd{\csname}-\cmd{\endcsname} pair.
%       \end{itemize}
% \item If the argument is zero or negative, the commands expand
%       to nothing like \cmd{\romannumeral}.
% \item The argument is a \hologo{TeX} number. Anything that would be
%       accepted by \cmd{\number} is a valid argument:
%       \begin{itemize}
%       \item explicite constants,
%       \item macros that expand to a number,
%       \item count registers, \hologo{LaTeX} counter can used
%             via \cmd{\value}, e.\,g.:\\
%             |\alphalph{\value{page}}|
%       \item \dots
%       \end{itemize}
% \item \hologo{eTeX}'s numeric expressions are supported, if \hologo{eTeX} is
%       available. Then \cs{numexpr} is applied to the argument.
%       Package \cs{calc}'s expressions are not supported.
%       That would violate the expandibility.
% \end{itemize}
%
% \subsubsection{General usability}
%
% \begin{description}
% \item[\hologo{TeX} format:]
%   The package does not depend on \hologo{LaTeX}, it can also be used
%   by \hologo{plainTeX}, for example.
% \item[\hologo{eTeX}:] \eTeX is supported, the macros are shorter and
%   faster. But \eTeX's extensions are not requirements.
%   Without \hologo{eTeX}, just the implementation changes. The properties
%   remain unchanged.
% \end{description}
%
% \StopEventually{
% }
%
% \section{Implementation}
% \subsection{Begin\texorpdfstring{ of package}{}}
%    \begin{macrocode}
%<*package>
%    \end{macrocode}
%    Reload check, especially if the package is not used with \LaTeX.
%    \begin{macrocode}
\begingroup\catcode61\catcode48\catcode32=10\relax%
  \catcode13=5 % ^^M
  \endlinechar=13 %
  \catcode35=6 % #
  \catcode39=12 % '
  \catcode44=12 % ,
  \catcode45=12 % -
  \catcode46=12 % .
  \catcode58=12 % :
  \catcode64=11 % @
  \catcode123=1 % {
  \catcode125=2 % }
  \expandafter\let\expandafter\x\csname ver@alphalph.sty\endcsname
  \ifx\x\relax % plain-TeX, first loading
  \else
    \def\empty{}%
    \ifx\x\empty % LaTeX, first loading,
      % variable is initialized, but \ProvidesPackage not yet seen
    \else
      \expandafter\ifx\csname PackageInfo\endcsname\relax
        \def\x#1#2{%
          \immediate\write-1{Package #1 Info: #2.}%
        }%
      \else
        \def\x#1#2{\PackageInfo{#1}{#2, stopped}}%
      \fi
      \x{alphalph}{The package is already loaded}%
      \aftergroup\endinput
    \fi
  \fi
\endgroup%
%    \end{macrocode}
%    Package identification:
%    \begin{macrocode}
\begingroup\catcode61\catcode48\catcode32=10\relax%
  \catcode13=5 % ^^M
  \endlinechar=13 %
  \catcode35=6 % #
  \catcode39=12 % '
  \catcode40=12 % (
  \catcode41=12 % )
  \catcode44=12 % ,
  \catcode45=12 % -
  \catcode46=12 % .
  \catcode47=12 % /
  \catcode58=12 % :
  \catcode64=11 % @
  \catcode91=12 % [
  \catcode93=12 % ]
  \catcode123=1 % {
  \catcode125=2 % }
  \expandafter\ifx\csname ProvidesPackage\endcsname\relax
    \def\x#1#2#3[#4]{\endgroup
      \immediate\write-1{Package: #3 #4}%
      \xdef#1{#4}%
    }%
  \else
    \def\x#1#2[#3]{\endgroup
      #2[{#3}]%
      \ifx#1\@undefined
        \xdef#1{#3}%
      \fi
      \ifx#1\relax
        \xdef#1{#3}%
      \fi
    }%
  \fi
\expandafter\x\csname ver@alphalph.sty\endcsname
\ProvidesPackage{alphalph}%
  [2016/05/16 v2.5 Convert numbers to letters (HO)]%
%    \end{macrocode}
%
% \subsection{Catcodes}
%
%    \begin{macrocode}
\begingroup\catcode61\catcode48\catcode32=10\relax%
  \catcode13=5 % ^^M
  \endlinechar=13 %
  \catcode123=1 % {
  \catcode125=2 % }
  \catcode64=11 % @
  \def\x{\endgroup
    \expandafter\edef\csname AlPh@AtEnd\endcsname{%
      \endlinechar=\the\endlinechar\relax
      \catcode13=\the\catcode13\relax
      \catcode32=\the\catcode32\relax
      \catcode35=\the\catcode35\relax
      \catcode61=\the\catcode61\relax
      \catcode64=\the\catcode64\relax
      \catcode123=\the\catcode123\relax
      \catcode125=\the\catcode125\relax
    }%
  }%
\x\catcode61\catcode48\catcode32=10\relax%
\catcode13=5 % ^^M
\endlinechar=13 %
\catcode35=6 % #
\catcode64=11 % @
\catcode123=1 % {
\catcode125=2 % }
\def\TMP@EnsureCode#1#2{%
  \edef\AlPh@AtEnd{%
    \AlPh@AtEnd
    \catcode#1=\the\catcode#1\relax
  }%
  \catcode#1=#2\relax
}
\TMP@EnsureCode{33}{12}% !
\TMP@EnsureCode{39}{12}% '
\TMP@EnsureCode{40}{12}% (
\TMP@EnsureCode{41}{12}% )
\TMP@EnsureCode{43}{12}% +
\TMP@EnsureCode{44}{12}% ,
\TMP@EnsureCode{46}{12}% .
\TMP@EnsureCode{47}{12}% /
\TMP@EnsureCode{59}{12}% ;
\TMP@EnsureCode{60}{12}% <
\TMP@EnsureCode{62}{12}% >
\TMP@EnsureCode{91}{12}% [
\TMP@EnsureCode{93}{12}% ]
\TMP@EnsureCode{96}{12}% `
\TMP@EnsureCode{124}{12}% |
\edef\AlPh@AtEnd{\AlPh@AtEnd\noexpand\endinput}
%    \end{macrocode}
%
% \subsection{Package loading}
%
%    \begin{macrocode}
\begingroup\expandafter\expandafter\expandafter\endgroup
\expandafter\ifx\csname RequirePackage\endcsname\relax
  \input infwarerr.sty\relax
  \input intcalc.sty\relax
\else
  \RequirePackage{infwarerr}[2007/09/09]%
  \RequirePackage{intcalc}[2007/09/09]%
\fi
%    \end{macrocode}
%
% \subsection{\hologo{eTeX} detection}
%
%    \begin{macrocode}
\begingroup\expandafter\expandafter\expandafter\endgroup
\expandafter\ifx\csname numexpr\endcsname\relax
  \catcode124=9 % '!': ignore
  \catcode43=14 % '+': comment
\else
  \catcode124=14 % '!': comment
  \catcode43=9   % '+': ignore
\fi
%    \end{macrocode}
%
% \subsection{Help macros}
%
%    \begin{macro}{\AlPh@Error}
%    \begin{macrocode}
\def\AlPh@Error#1{%
  \begingroup
    \escapechar=92 % backslash
    \@PackageError{alphalph}{#1}\@ehc
  \endgroup
}
%    \end{macrocode}
%    \end{macro}
%
%    \begin{macro}{\AlPh@IfDefinable}
%    \begin{macrocode}
\begingroup\expandafter\expandafter\expandafter\endgroup
\expandafter\ifx\csname @ifdefinable\endcsname\relax
  \def\AlPh@IfDefinable#1#2{%
    \ifcase\ifx#1\@undefined\else\ifx#1\relax\else1\fi\fi0 %
      #2%
    \else
      \AlPh@Error{%
        Command \string#1 already defined%
      }%
    \fi
  }%
%    \end{macrocode}
%    \end{macro}
%    \begin{macrocode}
\else
%    \end{macrocode}
%    \begin{macro}{\AlPh@IfDefinable}
%    \begin{macrocode}
  \let\AlPh@IfDefinable\@ifdefinable
%    \end{macrocode}
%    \end{macro}
%    \begin{macrocode}
\fi
%    \end{macrocode}
%
%    \begin{macro}{\@ReturnAfterElseFi}
%    \begin{macro}{\@ReturnAfterFi}
%    The following commands moves the `then' and `else' part respectively
%    behind the |\if|-construct. This prevents a too deep |\if|-nesting
%    and so a \hologo{TeX} capacity error because of a limited input stack size.
%    I use this trick in several packages, so I don't prefix these internal
%    commands in order not to have the same macros with different names.
%    (It saves memory.)
%    \begin{macrocode}
\long\def\@ReturnAfterElseFi#1\else#2\fi{\fi#1}
\long\def\@ReturnAfterFi#1\fi{\fi#1}
%    \end{macrocode}
%    \end{macro}
%    \end{macro}
%
%    \begin{macro}{\@gobblefour}
%    \Hologo{LaTeX} defines commands for eating arguments.
%    Define \cs{@gobblefour} if it is not defined (\hologo{plainTeX}).
%    \begin{macrocode}
\expandafter\ifx\csname @gobblefour\endcsname\relax
  \long\def\@gobblefour#1#2#3#4{}%
\fi
%    \end{macrocode}
%    \end{macro}
%
%    \begin{macro}{AlPh@IfOptArg}
%    \begin{macrocode}
\begingroup\expandafter\expandafter\expandafter\endgroup
\expandafter\ifx\csname kernel@ifnextchar\endcsname\relax
  \begingroup\expandafter\expandafter\expandafter\endgroup
  \expandafter\ifx\csname @ifnextchar\endcsname\relax
    \def\AlPh@IfOptArg#1#2{%
      \def\AlPh@TempA{#1}%
      \def\AlPh@TempB{#2}%
      \futurelet\AlPh@Token\AlPh@IfOptArgNext
    }%
    \let\AlPh@BracketLeft=[%]
    \def\AlPh@IfOptArgNext{%
      \ifx\AlPh@Token\AlPh@BracketLeft
        \expandafter\AlPh@TempA
      \else
        \expandafter\AlPh@TempB
      \fi
    }%
  \else
    \def\AlPh@IfOptArg{\@ifnextchar[}%]
  \fi
\else
  \def\AlPh@IfOptArg{\kernel@ifnextchar[}%]
\fi
%    \end{macrocode}
%    \end{macro}
%
% \subsection{Symbol provider}
%
% \subsubsection{Alphabet}
%
%    The output of \cs{alphalph} and \cs{AlphAlph} should be
%    usable as part of command names (see \cs{@namedef}, \cs{csname}, \dots).
%    Unhappily some languages of package \xpackage{babel}
%    redefine \hologo{LaTeX}'s \cs{@alph} and \cs{@Alph} in a manner that
%    they cannot be used in expandable context any more.
%    Therefore package \xpackage{alphalph} provides its own
%    commands.
%
%    \begin{macro}{\alphalph@Alph}
%    \begin{macro}{\alphalph@alph}
%    The two commands \cmd{\AlPh@Alph} and \cmd{\AlPh@alph} convert a
%    number into a letter (uppercase and lowercase respectivly).
%    The character |@| is used as an error symbol,
%    if the number isn't in the range of 1 until 26.
%    Here we need no space after the number |#1|, because the error
%    symbol |@| for the zero case stops scanning the number.
%    This error symbol should not appear anywhere (except for bugs).
%    \begin{macrocode}
\def\alphalph@Alph#1{%
  \ifcase#1%
    @%
  \or A\or B\or C\or D\or E\or F\or G\or H\or I\or J\or K\or L\or M%
  \or N\or O\or P\or Q\or R\or S\or T\or U\or V\or W\or X\or Y\or Z%
  \else
    \AlPh@ctrerr
    @%
  \fi
}
\def\alphalph@alph#1{%
  \ifcase#1%
    @%
  \or a\or b\or c\or d\or e\or f\or g\or h\or i\or j\or k\or l\or m%
  \or n\or o\or p\or q\or r\or s\or t\or u\or v\or w\or x\or y\or z%
  \else
    \AlPh@ctrerr
    @%
  \fi
}
%    \end{macrocode}
%    \end{macro}
%    \end{macro}
%    \begin{macro}{\AlPh@ctrerr}
%    Macro \cs{AlPh@ctrerr} is used as hook for the algorithm
%    to get the available number of symbols.
%    \begin{macrocode}
\def\AlPh@ctrerr{}
%    \end{macrocode}
%    \end{macro}
%
% \subsection{Finding number of symbols}
%
%    \begin{macro}{\AlPh@GetNumberOfSymbols}
%    \noindent
%    |#1|: symbols provider
%    \begin{macrocode}
\def\AlPh@GetNumberOfSymbols#1{%
  \AlPh@TestNumber1!{#1}%
  \ifAlPh@Unavailable
    \def\AlPh@Number{0}%
    \AlPh@Error{No symbols found}%
  \else
    \def\AlPh@Number{1}%
    \AlPh@ExpSearch2!{#1}%
  \fi
}
%    \end{macrocode}
%    \end{macro}
%    \begin{macro}{\ifAlPh@Unavailable}
%    \begin{macrocode}
\let\ifAlPh@Unavailable\iffalse
\def\AlPh@Unavailabletrue{%
  \global\let\ifAlPh@Unavailable\iftrue
}
\def\AlPh@Unavailablefalse{%
  \global\let\ifAlPh@Unavailable\iffalse
}
%    \end{macrocode}
%    \end{macro}
%    \begin{macro}{\AlPh@TestNumber}
%    |#1|: number to be tested\\
%    |#2|: symbols provider
%    \begin{macrocode}
\def\AlPh@TestNumber#1!#2{%
  \AlPh@Unavailablefalse
  \begingroup
    \setbox0=\hbox{%
      \begingroup % color
        \let\@ctrerr\AlPh@Unavailabletrue
        \let\AlPh@ctrerr\AlPh@Unavailabletrue
        #2{#1}%
      \endgroup
    }%
    \ifdim\wd0=0pt %
      \AlPh@Unavailabletrue
    \fi
  \endgroup
}
%    \end{macrocode}
%    \end{macro}
%    \begin{macro}{\AlPh@ExpSearch}
%    |#1|: number to be tested\\
%    |#2|: symbols provider
%    \begin{macrocode}
\def\AlPh@ExpSearch#1!#2{%
  \let\AlPh@Next\relax
  \AlPh@TestNumber#1!{#2}%
  \ifAlPh@Unavailable
    \expandafter\AlPh@BinSearch\AlPh@Number!#1!{#2}%
  \else
    \def\AlPh@Number{#1}%
    \ifnum#1>1073741823 %
      \AlPh@TestNumber2147483647!{#2}%
      \ifAlPh@Unavailable
        \AlPh@BinSearch#1!2147483647!{#2}%
      \else
        \def\AlPh@Number{0}%
        \AlPh@Error{%
          Maximal symbol number not found%
        }%
      \fi
    \else
      \def\AlPh@Next{%
        \expandafter\AlPh@ExpSearch\number\intcalcShl{#1}!{#2}%
      }%
    \fi
  \fi
  \AlPh@Next
}
%    \end{macrocode}
%    \end{macro}
%    \begin{macro}{\AlPh@BinSearch}
%    |#1|: available number\\
%    |#2|: unavailable number, |#2| $>$ |#1|\\
%    |#3|: symbols provider
%    \begin{macrocode}
\def\AlPh@BinSearch#1!#2!#3{%
  \expandafter\AlPh@ProcessBinSearch
  \number\intcalcShr{\intcalcAdd{#1}{#2}}!%
  #1!#2!{#3}%
}
%    \end{macrocode}
%    \end{macro}
%    \begin{macro}{\AlPh@ProcessBinSearch}
%    |#1|: number to be tested, |#2| $\leq$ |#1| $\leq$ |#3|\\
%    |#2|: available number\\
%    |#3|: unavailable number\\
%    |#4|: symbols provider
%    \begin{macrocode}
\def\AlPh@ProcessBinSearch#1!#2!#3!#4{%
  \let\AlPh@Next\relax
  \ifnum#1>#2 %
    \ifnum#1<#3 %
      \AlPh@TestNumber#1!{#4}%
      \ifAlPh@Unavailable
        \def\AlPh@Next{%
          \AlPh@BinSearch#2!#1!{#4}%
        }%
      \else
        \def\AlPh@Next{%
          \AlPh@BinSearch#1!#3!{#4}%
        }%
      \fi
    \else
      \def\AlPh@Number{#2}%
    \fi
  \else
    \def\AlPh@Number{#2}%
  \fi
  \AlPh@Next
}
%    \end{macrocode}
%    \end{macro}
%
% \subsection{Methods}
%
%    The names of method macros start with \cs{AlPh@Method}.
%    These macros do the main job in converting a number to
%    its representation.
%    A method command is called with three arguments.
%    The first argument is the number of symbols. The
%    second argument is the basic macro for converting
%    a number with limited number range. The last parameter
%    is the number that needs converting.
%
% \subsubsection{Common methods}
%
%    \begin{macro}{\AlPh@CheckPositive}
%    \noindent
%    |#1|: number to be checked
%    |#2|: continuation macro\\
%    |#3|: number of symbols (hidden here)\\
%    |#4|: symbol provider (hidden here)
%    \begin{macrocode}
\def\AlPh@CheckPositive#1!#2{%
  \ifnum#1<1 %
    \expandafter\@gobblefour
  \fi
  #2{#1}%
}
%    \end{macrocode}
%    \end{macro}
%
% \subsubsection{Method `alph'}
%
%    \begin{macro}{\AlPh@Method@alph}
%    \noindent
%    |#1|: number of symbols\\
%    |#2|: symbols provider\\
%    |#3|: number to be converted
%    \begin{macrocode}
\def\AlPh@Method@alph#1#2#3{%
  \expandafter\AlPh@CheckPositive
|   \number#3!%
+   \the\numexpr#3!%
    \AlPh@ProcessAlph
    {#1}{#2}%
}
%    \end{macrocode}
%    \end{macro}
%    \begin{macro}{\AlPh@ProcessAlph}
%    |#1|: current number\\
%    |#2|: number of symbols\\
%    |#3|: symbols provider
%    \begin{macrocode}
\def\AlPh@ProcessAlph#1#2#3{%
  \ifnum#1>#2 %
    \@ReturnAfterElseFi{%
      \expandafter\AlPh@StepAlph\number
        \intcalcInc{%
          \intcalcMod{\intcalcDec{#1}}{#2}%
        }%
      \expandafter!\number
        \intcalcDiv{\intcalcDec{#1}}{#2}%
      !{#2}{#3}%
    }%
  \else
    \@ReturnAfterFi{%
      #3{#1}%
    }%
  \fi
}
%    \end{macrocode}
%    \end{macro}
%    \begin{macro}{\AlPh@StepAlph}
%    |#1|: current last digit\\
%    |#2|: new current number\\
%    |#3|: number of symbols\\
%    |#4|: symbols provider
%    \begin{macrocode}
\def\AlPh@StepAlph#1!#2!#3#4{%
  \AlPh@ProcessAlph{#2}{#3}{#4}%
  #4{#1}%
}
%    \end{macrocode}
%    \end{macro}
%
% \subsubsection{Method `wrap'}
%
%    \begin{macro}{\AlPh@Method@wrap}
%    \noindent
%    |#1|: number of symbols\\
%    |#2|: symbols provider\\
%    |#3|: number to be converted
%    \begin{macrocode}
\def\AlPh@Method@wrap#1#2#3{%
  \expandafter\AlPh@CheckPositive
|   \number#3!%
+   \the\numexpr#3!%
    \AlPh@ProcessWrap
    {#1}{#2}%
}
%    \end{macrocode}
%    \end{macro}
%
%    \begin{macro}{\AlPh@ProcessWrap}
%    |#1|: number to be converted\\
%    |#2|: number of symbols\\
%    |#3|: symbols provider
%    \begin{macrocode}
\def\AlPh@ProcessWrap#1#2#3{%
  \ifnum#1>#2 %
    \@ReturnAfterElseFi{%
      \expandafter\AlPh@StepWrap\number
        \intcalcInc{\intcalcMod{\intcalcDec{#1}}{#2}}%
      !{#3}%
    }%
  \else
    \@ReturnAfterFi{%
      #3{#1}%
    }%
  \fi
}
%    \end{macrocode}
%    \end{macro}
%    \begin{macro}{\AlPh@StepWrap}
%    |#1|: final number\\
%    |#2|: symbols provider
%    \begin{macrocode}
\def\AlPh@StepWrap#1!#2{%
  #2{#1}%
}
%    \end{macrocode}
%    \end{macro}
%
% \subsubsection{Method `mult'}
%
%    After the number of symbols is exhausted,
%    repetitions of the symbol are used.
%    \begin{gather*}
%      x \Def \text{number to be converted}\\
%      n \Def \text{number of symbols}\\
%      r \Def \text{repetition length}\\
%      s \Def \text{symbol slot}\\
%      r = ((x - 1) \div n) + 1\\
%      s = ((x - 1) \mod n) + 1
%    \end{gather*}
%    \begin{macro}{\AlPh@Method@mult}
%    \noindent
%    |#1|: number of symbols\\
%    |#2|: symbols provider\\
%    |#3|: number to be converted
%    \begin{macrocode}
\def\AlPh@Method@mult#1#2#3{%
  \expandafter\AlPh@CheckPositive
|   \number#3!%
+   \the\numexpr#3!%
    \AlPh@ProcessMult
    {#1}{#2}%
}
%    \end{macrocode}
%    \end{macro}
%
%    \begin{macro}{\AlPh@ProcessMult}
%    |#1|: number to be converted\\
%    |#2|: number of symbols\\
%    |#3|: symbols provider
%    \begin{macrocode}
\def\AlPh@ProcessMult#1#2#3{%
  \ifnum#1>#2 %
    \@ReturnAfterElseFi{%
      \expandafter\AlPh@StepMult\romannumeral
        \intcalcInc{\intcalcDiv{\intcalcDec{#1}}{#2}}%
        000%
      \expandafter!\number
        \intcalcInc{\intcalcMod{\intcalcDec{#1}}{#2}}%
      !{#3}%
    }%
  \else
    \@ReturnAfterFi{%
      #3{#1}%
    }%
  \fi
}
%    \end{macrocode}
%    \end{macro}
%    \begin{macro}{\AlPh@StepMult}
%    |#1#2|: repetitions coded as list of character `m'\\
%    |#3|: symbol slot\\
%    |#4|: symbols provider
%    \begin{macrocode}
\def\AlPh@StepMult#1#2!#3!#4{%
  \ifx\\#2\\%
  \else
    \@ReturnAfterFi{%
      \AlPh@StepMult#2!#3!{#4}%
    }%
  \fi
  #4{#3}%
}
%    \end{macrocode}
%    \end{macro}
%
% \subsection{User interface}
%
%    \begin{macro}{\newalphalph}
%    Macro \cs{newalphalph} had three arguments in versions below 2.0.
%    For the new method argument we use an optional argument an first
%    position.\\
%    |#1|: cmd\\
%    |[#2]|: method name: |alph| (default), |wrap|, |mult|\\ % hash-ok
%    |#3|: symbols provider\\
%    |#4|: number of symbols
%    \begin{macrocode}
\AlPh@IfDefinable\newalphalph{%
  \def\newalphalph#1{%
    \AlPh@IfOptArg{%
      \AlPh@newalphalph{#1}%
    }{%
      \AlPh@newalphalph{#1}[alph]%
    }%
  }%
}
%    \end{macrocode}
%    \end{macro}
%    \begin{macro}{\AlPh@newalphalph}
%    |#1|: cmd
%    |#2|: method name\\
%    |#3|: symbols provider\\
%    |#4|: number of symbols
%    \begin{macrocode}
\def\AlPh@newalphalph#1[#2]#3#4{%
  \begingroup\expandafter\expandafter\expandafter\endgroup
  \expandafter\ifx\csname AlPh@Method@#2\endcsname\relax
    \AlPh@Error{%
      Unknown method %
|     `#2'%
+     `\detokenize{#2}'%
    }%
  \else
    \ifx\\#4\\%
      \AlPh@GetNumberOfSymbols{#3}%
      \ifcase\AlPh@Number
      \else
        \begingroup
          \escapechar=92 % backslash
          \@PackageInfo{alphalph}{%
            Number of symbols for \string#1 is \AlPh@Number
          }%
        \endgroup
        \expandafter\AlPh@NewAlphAlph
        \csname AlPh@Method@#2\expandafter\endcsname
        \AlPh@Number!{#1}{#3}%
      \fi
    \else
      \expandafter\AlPh@NewAlphAlph
      \csname AlPh@Method@#2\expandafter\endcsname
|     \number#4!%
+     \the\numexpr#4!%
      {#1}{#3}%
    \fi
  \fi
}%
%    \end{macrocode}
%    \end{macro}
%    \begin{macro}{\AlPh@NewAlphAlph}
%    |#1|: method macro\\
%    |#2|: number of symbols\\
%    |#3|: cmd\\
%    |#4|: symbols provider
%    \begin{macrocode}
\def\AlPh@NewAlphAlph#1#2!#3#4{%
  \AlPh@IfDefinable#3{%
    \ifnum#2>0 %
      \def#3{#1{#2}{#4}}%
    \else
      \AlPh@Error{%
          Definition of \string#3 failed,\MessageBreak
          because number of symbols (#2) is not positive%
        }%
    \fi
  }%
}
%    \end{macrocode}
%    \end{macro}
%
%    \begin{macro}{\AlphAlph}
%    \begin{macrocode}
\newalphalph\AlphAlph\alphalph@Alph{26}
%    \end{macrocode}
%    \end{macro}
%    \begin{macro}{\alphalph}
%    \begin{macrocode}
\newalphalph\alphalph\alphalph@alph{26}
%    \end{macrocode}
%    \end{macro}
%
%    \begin{macrocode}
\AlPh@AtEnd%
%</package>
%    \end{macrocode}
%
% \section{Test}
%
% \subsection{Catcode checks for loading}
%
%    \begin{macrocode}
%<*test1>
%    \end{macrocode}
%    \begin{macrocode}
\catcode`\{=1 %
\catcode`\}=2 %
\catcode`\#=6 %
\catcode`\@=11 %
\expandafter\ifx\csname count@\endcsname\relax
  \countdef\count@=255 %
\fi
\expandafter\ifx\csname @gobble\endcsname\relax
  \long\def\@gobble#1{}%
\fi
\expandafter\ifx\csname @firstofone\endcsname\relax
  \long\def\@firstofone#1{#1}%
\fi
\expandafter\ifx\csname loop\endcsname\relax
  \expandafter\@firstofone
\else
  \expandafter\@gobble
\fi
{%
  \def\loop#1\repeat{%
    \def\body{#1}%
    \iterate
  }%
  \def\iterate{%
    \body
      \let\next\iterate
    \else
      \let\next\relax
    \fi
    \next
  }%
  \let\repeat=\fi
}%
\def\RestoreCatcodes{}
\count@=0 %
\loop
  \edef\RestoreCatcodes{%
    \RestoreCatcodes
    \catcode\the\count@=\the\catcode\count@\relax
  }%
\ifnum\count@<255 %
  \advance\count@ 1 %
\repeat

\def\RangeCatcodeInvalid#1#2{%
  \count@=#1\relax
  \loop
    \catcode\count@=15 %
  \ifnum\count@<#2\relax
    \advance\count@ 1 %
  \repeat
}
\def\RangeCatcodeCheck#1#2#3{%
  \count@=#1\relax
  \loop
    \ifnum#3=\catcode\count@
    \else
      \errmessage{%
        Character \the\count@\space
        with wrong catcode \the\catcode\count@\space
        instead of \number#3%
      }%
    \fi
  \ifnum\count@<#2\relax
    \advance\count@ 1 %
  \repeat
}
\def\space{ }
\expandafter\ifx\csname LoadCommand\endcsname\relax
  \def\LoadCommand{\input alphalph.sty\relax}%
\fi
\def\Test{%
  \RangeCatcodeInvalid{0}{47}%
  \RangeCatcodeInvalid{58}{64}%
  \RangeCatcodeInvalid{91}{96}%
  \RangeCatcodeInvalid{123}{255}%
  \catcode`\@=12 %
  \catcode`\\=0 %
  \catcode`\%=14 %
  \LoadCommand
  \RangeCatcodeCheck{0}{36}{15}%
  \RangeCatcodeCheck{37}{37}{14}%
  \RangeCatcodeCheck{38}{47}{15}%
  \RangeCatcodeCheck{48}{57}{12}%
  \RangeCatcodeCheck{58}{63}{15}%
  \RangeCatcodeCheck{64}{64}{12}%
  \RangeCatcodeCheck{65}{90}{11}%
  \RangeCatcodeCheck{91}{91}{15}%
  \RangeCatcodeCheck{92}{92}{0}%
  \RangeCatcodeCheck{93}{96}{15}%
  \RangeCatcodeCheck{97}{122}{11}%
  \RangeCatcodeCheck{123}{255}{15}%
  \RestoreCatcodes
}
\Test
\csname @@end\endcsname
\end
%    \end{macrocode}
%    \begin{macrocode}
%</test1>
%    \end{macrocode}
%
% \section{Macro tests}
%
%    \begin{macrocode}
%<*test2>
\NeedsTeXFormat{LaTeX2e}
\nofiles
\documentclass{article}
%<*noetex>
\makeatletter
\let\saved@numexpr\numexpr
\newcommand*{\DisableNumexpr}{%
  \let\numexpr\@undefined
}
\newcommand*{\RestoreNumexpr}{%
  \let\numexpr\saved@numexpr
}
\DisableNumexpr
%</noetex>
\usepackage{alphalph}[2016/05/16]
%<noetex>\RestoreNumexpr
\usepackage{qstest}
\IncludeTests{*}
\LogTests{log}{*}{*}

\newcommand*{\TestCmd}[3]{%
  \setbox0=\hbox{%
%<noetex>    \DisableNumexpr
    \edef\TestString{#1{#2}}%
    \expandafter\Expect\expandafter{\TestString}{#3}%
    \edef\TestString{#1{#2} }%
    \expandafter\Expect\expandafter{\TestString}{#3 }%
  }%
  \Expect*{\the\wd0}{0.0pt}%
}

\makeatletter
\newalphalph\LaTeXAlphAlph\@Alph{26}
\newalphalph\LaTeXalphalph\@alph{26}
\newalphalph\AlphWrap[wrap]\alphalph@Alph{26}
\newalphalph\alphwrap[wrap]\alphalph@alph{26}
\newalphalph\LaTeXAlphWrap[wrap]\@Alph{26}
\newalphalph\LaTeXalphwrap[wrap]\@alph{26}
\def\LastSymbol#1{%
  \ifx\\#1\\%
  \else
    \@LastSymbol#1\@nil
  \fi
}
\def\@LastSymbol#1#2\@nil{%
  \ifx\\#2\\%
    #1%
  \else
    \@LastSymbol#2\@nil
  \fi
}
\makeatother
\newcommand*{\TestAlph}[2]{%
  \uppercase{\TestCallCmd\AlphAlph{#2}}{#1}%
  \lowercase{\TestCallCmd\alphalph{#2}}{#1}%
  \uppercase{\TestCallCmd\LaTeXAlphAlph{#2}}{#1}%
  \lowercase{\TestCallCmd\LaTeXalphalph{#2}}{#1}%
  \edef\WrapString{\LastSymbol{#2}}%
  \expandafter\TestAlphWrap\expandafter{\WrapString}{#1}%
}
\newcommand*{\TestAlphWrap}[2]{%
  \uppercase{\TestCallCmd\AlphWrap{#1}}{#2}%
  \lowercase{\TestCallCmd\alphwrap{#1}}{#2}%
  \uppercase{\TestCallCmd\LaTeXAlphWrap{#1}}{#2}%
  \lowercase{\TestCallCmd\LaTeXalphwrap{#1}}{#2}%
}
\newcommand*{\TestCallCmd}[3]{%
  \TestCmd#1{#3}{#2}%
}
\begin{qstest}{AlphSymbols}{alphalph, AlphAlph, symbols}
  \TestAlph{1}{a}%
  \TestAlph{2}{b}%
  \TestAlph{3}{c}%
  \TestAlph{4}{d}%
  \TestAlph{5}{e}%
  \TestAlph{6}{f}%
  \TestAlph{7}{g}%
  \TestAlph{8}{h}%
  \TestAlph{9}{i}%
  \TestAlph{10}{j}%
  \TestAlph{11}{k}%
  \TestAlph{12}{l}%
  \TestAlph{13}{m}%
  \TestAlph{14}{n}%
  \TestAlph{15}{o}%
  \TestAlph{16}{p}%
  \TestAlph{17}{q}%
  \TestAlph{18}{r}%
  \TestAlph{19}{s}%
  \TestAlph{20}{t}%
  \TestAlph{21}{u}%
  \TestAlph{22}{v}%
  \TestAlph{23}{w}%
  \TestAlph{24}{x}%
  \TestAlph{25}{y}%
  \TestAlph{26}{z}%
\end{qstest}
\begin{qstest}{AlphRange}{alphalph, range}
  \TestAlph{0}{}%
  \TestAlph{-1}{}%
  \TestAlph{-2147483647}{}%
  \TestAlph{27}{aa}%
  \TestAlph{28}{ab}%
  \TestAlph{52}{az}%
  \TestAlph{53}{ba}%
  \TestAlph{78}{bz}%
  \TestAlph{79}{ca}%
  \TestAlph{702}{zz}%
  \TestAlph{703}{aaa}%
  \TestAlph{2147483647}{fxshrxw}%
\end{qstest}

\makeatletter
\newcommand*{\myvocals}[1]{%
  \ifcase#1X\or A\or E\or I\or O\or U\else Y\fi
}
\makeatother
\newalphalph\vocalsvocals\myvocals{5}
\newcommand*{\TestVocals}{%
  \TestCmd\vocalsvocals
}
\begin{qstest}{vocals}{vocals}
  \TestVocals{0}{}%
  \TestVocals{1}{A}%
  \TestVocals{2}{E}%
  \TestVocals{3}{I}%
  \TestVocals{4}{O}%
  \TestVocals{5}{U}%
  \TestVocals{6}{AA}%
  \TestVocals{7}{AE}%
  \TestVocals{8}{AI}%
  \TestVocals{9}{AO}%
  \TestVocals{10}{AU}%
  \TestVocals{11}{EA}%
  \TestVocals{24}{OO}%
  \TestVocals{25}{OU}%
  \TestVocals{26}{UA}%
  \TestVocals{29}{UO}%
  \TestVocals{30}{UU}%
  \TestVocals{31}{AAA}%
  \TestVocals{155}{UUU}%
  \TestVocals{156}{AAAA}%
  \TestVocals{2147483647}{AIIOOEEIOIIUOE}%
\end{qstest}

\makeatletter
\newalphalph\AlphMult[mult]{\alphalph@Alph}{26}
\newalphalph\alphmult[mult]{\alphalph@alph}{26}
\newalphalph\LaTeXAlphMult[mult]{\@Alph}{26}
\newalphalph\LaTeXalphmult[mult]{\@alph}{26}
\makeatother
\newcommand*{\TestMult}[2]{%
  \uppercase{\TestCallCmd\AlphMult{#2}}{#1}%
  \lowercase{\TestCallCmd\alphmult{#2}}{#1}%
  \uppercase{\TestCallCmd\LaTeXAlphMult{#2}}{#1}%
  \lowercase{\TestCallCmd\LaTeXalphmult{#2}}{#1}%
}
\begin{qstest}{mult}{mult}
  \TestMult{0}{}%
  \TestMult{-1}{}%
  \TestMult{-2147483647}{}%
  \TestMult{1}{a}%
  \TestMult{2}{b}%
  \TestMult{26}{z}%
  \TestMult{27}{aa}%
  \TestMult{28}{bb}%
  \TestMult{52}{zz}%
  \TestMult{53}{aaa}%
  \TestMult{54}{bbb}%
  \TestMult{259}{yyyyyyyyyy}%
  \TestMult{260}{zzzzzzzzzz}%
  \TestMult{261}{aaaaaaaaaaa}%
  \TestMult{262}{bbbbbbbbbbb}%
\end{qstest}

\def\myvocalsB#1{%
  \ifcase#1\or A\or E\or I\or O\or U\fi
}
\begin{qstest}{symbolnum}{symbolnum}
  \makeatletter
  \def\Test#1#2{%
    \let\TestCmd\relax
    \newalphalph\TestCmd{#1}{}%
    \Expect*{\AlPh@Number}{#2}%
  }%
  \Test\@alph{26}%
  \Test\@Alph{26}%
  \Test\@fnsymbol{9}%
  \Test\myvocalsB{5}%
  \Test\alphalph@alph{26}%
  \Test\alphalph@Alph{26}%
\end{qstest}

\begin{qstest}{list}{list}
  \makeatletter
  \def\catch#1\relax{%
    \def\FoundList{\catch#1}%
  }%
  \def\Test[#1]#2#3#4{%
    \let\testcmd\relax
    \newalphalph\testcmd[{#1}]{\catch}{#2}%
    \testcmd{#3}|\relax
    \expandafter\Expect\expandafter{\FoundList}{#4|}%
    %
    \let\SavedCatch\catch
    \def\catch{\noexpand\catch\noexpand\foo}%
    \edef\Result{#4|}%
    \@onelevel@sanitize\Result
    \let\catch\SavedCatch
    \let\testcmd\relax
    \newalphalph\testcmd[{#1}]{\catch\foo}{#2}%
    \testcmd{#3}|\relax
    \@onelevel@sanitize\FoundList
    \Expect*{\FoundList}*{\Result}%
  }%
  \Test[alph]{26}{3}{\catch{3}}%
  \Test[alph]{26}{12}{\catch{12}}%
  \Test[alph]{26}{27}{\catch{1}\catch{1}}%
  \Test[alph]{26}{78}{\catch{2}\catch{26}}%
  \Test[wrap]{26}{7}{\catch{7}}%
  \Test[wrap]{26}{14}{\catch{14}}%
  \Test[wrap]{26}{80}{\catch{2}}%
  \Test[wrap]{26}{700}{\catch{24}}%
  \Test[mult]{26}{4}{\catch{4}}%
  \Test[mult]{26}{17}{\catch{17}}%
  \Test[mult]{26}{54}{\catch{2}\catch{2}\catch{2}}%
\end{qstest}

\begin{document}
\end{document}
%</test2>
%    \end{macrocode}
%
% \section{Installation}
%
% \subsection{Download}
%
% \paragraph{Package.} This package is available on
% CTAN\footnote{\CTANpkg{alphalph}}:
% \begin{description}
% \item[\CTAN{macros/latex/contrib/oberdiek/alphalph.dtx}] The source file.
% \item[\CTAN{macros/latex/contrib/oberdiek/alphalph.pdf}] Documentation.
% \end{description}
%
%
% \paragraph{Bundle.} All the packages of the bundle `oberdiek'
% are also available in a TDS compliant ZIP archive. There
% the packages are already unpacked and the documentation files
% are generated. The files and directories obey the TDS standard.
% \begin{description}
% \item[\CTANinstall{install/macros/latex/contrib/oberdiek.tds.zip}]
% \end{description}
% \emph{TDS} refers to the standard ``A Directory Structure
% for \TeX\ Files'' (\CTAN{tds/tds.pdf}). Directories
% with \xfile{texmf} in their name are usually organized this way.
%
% \subsection{Bundle installation}
%
% \paragraph{Unpacking.} Unpack the \xfile{oberdiek.tds.zip} in the
% TDS tree (also known as \xfile{texmf} tree) of your choice.
% Example (linux):
% \begin{quote}
%   |unzip oberdiek.tds.zip -d ~/texmf|
% \end{quote}
%
% \paragraph{Script installation.}
% Check the directory \xfile{TDS:scripts/oberdiek/} for
% scripts that need further installation steps.
% Package \xpackage{attachfile2} comes with the Perl script
% \xfile{pdfatfi.pl} that should be installed in such a way
% that it can be called as \texttt{pdfatfi}.
% Example (linux):
% \begin{quote}
%   |chmod +x scripts/oberdiek/pdfatfi.pl|\\
%   |cp scripts/oberdiek/pdfatfi.pl /usr/local/bin/|
% \end{quote}
%
% \subsection{Package installation}
%
% \paragraph{Unpacking.} The \xfile{.dtx} file is a self-extracting
% \docstrip\ archive. The files are extracted by running the
% \xfile{.dtx} through \plainTeX:
% \begin{quote}
%   \verb|tex alphalph.dtx|
% \end{quote}
%
% \paragraph{TDS.} Now the different files must be moved into
% the different directories in your installation TDS tree
% (also known as \xfile{texmf} tree):
% \begin{quote}
% \def\t{^^A
% \begin{tabular}{@{}>{\ttfamily}l@{ $\rightarrow$ }>{\ttfamily}l@{}}
%   alphalph.sty & tex/generic/oberdiek/alphalph.sty\\
%   alphalph.pdf & doc/latex/oberdiek/alphalph.pdf\\
%   test/alphalph-test1.tex & doc/latex/oberdiek/test/alphalph-test1.tex\\
%   test/alphalph-test2.tex & doc/latex/oberdiek/test/alphalph-test2.tex\\
%   test/alphalph-test3.tex & doc/latex/oberdiek/test/alphalph-test3.tex\\
%   alphalph.dtx & source/latex/oberdiek/alphalph.dtx\\
% \end{tabular}^^A
% }^^A
% \sbox0{\t}^^A
% \ifdim\wd0>\linewidth
%   \begingroup
%     \advance\linewidth by\leftmargin
%     \advance\linewidth by\rightmargin
%   \edef\x{\endgroup
%     \def\noexpand\lw{\the\linewidth}^^A
%   }\x
%   \def\lwbox{^^A
%     \leavevmode
%     \hbox to \linewidth{^^A
%       \kern-\leftmargin\relax
%       \hss
%       \usebox0
%       \hss
%       \kern-\rightmargin\relax
%     }^^A
%   }^^A
%   \ifdim\wd0>\lw
%     \sbox0{\small\t}^^A
%     \ifdim\wd0>\linewidth
%       \ifdim\wd0>\lw
%         \sbox0{\footnotesize\t}^^A
%         \ifdim\wd0>\linewidth
%           \ifdim\wd0>\lw
%             \sbox0{\scriptsize\t}^^A
%             \ifdim\wd0>\linewidth
%               \ifdim\wd0>\lw
%                 \sbox0{\tiny\t}^^A
%                 \ifdim\wd0>\linewidth
%                   \lwbox
%                 \else
%                   \usebox0
%                 \fi
%               \else
%                 \lwbox
%               \fi
%             \else
%               \usebox0
%             \fi
%           \else
%             \lwbox
%           \fi
%         \else
%           \usebox0
%         \fi
%       \else
%         \lwbox
%       \fi
%     \else
%       \usebox0
%     \fi
%   \else
%     \lwbox
%   \fi
% \else
%   \usebox0
% \fi
% \end{quote}
% If you have a \xfile{docstrip.cfg} that configures and enables \docstrip's
% TDS installing feature, then some files can already be in the right
% place, see the documentation of \docstrip.
%
% \subsection{Refresh file name databases}
%
% If your \TeX~distribution
% (\teTeX, \mikTeX, \dots) relies on file name databases, you must refresh
% these. For example, \teTeX\ users run \verb|texhash| or
% \verb|mktexlsr|.
%
% \subsection{Some details for the interested}
%
% \paragraph{Attached source.}
%
% The PDF documentation on CTAN also includes the
% \xfile{.dtx} source file. It can be extracted by
% AcrobatReader 6 or higher. Another option is \textsf{pdftk},
% e.g. unpack the file into the current directory:
% \begin{quote}
%   \verb|pdftk alphalph.pdf unpack_files output .|
% \end{quote}
%
% \paragraph{Unpacking with \LaTeX.}
% The \xfile{.dtx} chooses its action depending on the format:
% \begin{description}
% \item[\plainTeX:] Run \docstrip\ and extract the files.
% \item[\LaTeX:] Generate the documentation.
% \end{description}
% If you insist on using \LaTeX\ for \docstrip\ (really,
% \docstrip\ does not need \LaTeX), then inform the autodetect routine
% about your intention:
% \begin{quote}
%   \verb|latex \let\install=y% \iffalse meta-comment
%
% File: alphalph.dtx
% Version: 2016/05/16 v2.5
% Info: Convert numbers to letters
%
% Copyright (C) 1999, 2006-2008, 2010, 2011 by
%    Heiko Oberdiek <heiko.oberdiek at googlemail.com>
%    2016
%    https://github.com/ho-tex/oberdiek/issues
%
% This work may be distributed and/or modified under the
% conditions of the LaTeX Project Public License, either
% version 1.3c of this license or (at your option) any later
% version. This version of this license is in
%    http://www.latex-project.org/lppl/lppl-1-3c.txt
% and the latest version of this license is in
%    http://www.latex-project.org/lppl.txt
% and version 1.3 or later is part of all distributions of
% LaTeX version 2005/12/01 or later.
%
% This work has the LPPL maintenance status "maintained".
%
% This Current Maintainer of this work is Heiko Oberdiek.
%
% The Base Interpreter refers to any `TeX-Format',
% because some files are installed in TDS:tex/generic//.
%
% This work consists of the main source file alphalph.dtx
% and the derived files
%    alphalph.sty, alphalph.pdf, alphalph.ins, alphalph.drv,
%    alphalph-test1.tex, alphalph-test2.tex, alphalph-test3.tex.
%
% Distribution:
%    CTAN:macros/latex/contrib/oberdiek/alphalph.dtx
%    CTAN:macros/latex/contrib/oberdiek/alphalph.pdf
%
% Unpacking:
%    (a) If alphalph.ins is present:
%           tex alphalph.ins
%    (b) Without alphalph.ins:
%           tex alphalph.dtx
%    (c) If you insist on using LaTeX
%           latex \let\install=y\input{alphalph.dtx}
%        (quote the arguments according to the demands of your shell)
%
% Documentation:
%    (a) If alphalph.drv is present:
%           latex alphalph.drv
%    (b) Without alphalph.drv:
%           latex alphalph.dtx; ...
%    The class ltxdoc loads the configuration file ltxdoc.cfg
%    if available. Here you can specify further options, e.g.
%    use A4 as paper format:
%       \PassOptionsToClass{a4paper}{article}
%
%    Programm calls to get the documentation (example):
%       pdflatex alphalph.dtx
%       makeindex -s gind.ist alphalph.idx
%       pdflatex alphalph.dtx
%       makeindex -s gind.ist alphalph.idx
%       pdflatex alphalph.dtx
%
% Installation:
%    TDS:tex/generic/oberdiek/alphalph.sty
%    TDS:doc/latex/oberdiek/alphalph.pdf
%    TDS:doc/latex/oberdiek/test/alphalph-test1.tex
%    TDS:doc/latex/oberdiek/test/alphalph-test2.tex
%    TDS:doc/latex/oberdiek/test/alphalph-test3.tex
%    TDS:source/latex/oberdiek/alphalph.dtx
%
%<*ignore>
\begingroup
  \catcode123=1 %
  \catcode125=2 %
  \def\x{LaTeX2e}%
\expandafter\endgroup
\ifcase 0\ifx\install y1\fi\expandafter
         \ifx\csname processbatchFile\endcsname\relax\else1\fi
         \ifx\fmtname\x\else 1\fi\relax
\else\csname fi\endcsname
%</ignore>
%<*install>
\input docstrip.tex
\Msg{************************************************************************}
\Msg{* Installation}
\Msg{* Package: alphalph 2016/05/16 v2.5 Convert numbers to letters (HO)}
\Msg{************************************************************************}

\keepsilent
\askforoverwritefalse

\let\MetaPrefix\relax
\preamble

This is a generated file.

Project: alphalph
Version: 2016/05/16 v2.5

Copyright (C) 1999, 2006-2008, 2010, 2011 by
   Heiko Oberdiek <heiko.oberdiek at googlemail.com>

This work may be distributed and/or modified under the
conditions of the LaTeX Project Public License, either
version 1.3c of this license or (at your option) any later
version. This version of this license is in
   http://www.latex-project.org/lppl/lppl-1-3c.txt
and the latest version of this license is in
   http://www.latex-project.org/lppl.txt
and version 1.3 or later is part of all distributions of
LaTeX version 2005/12/01 or later.

This work has the LPPL maintenance status "maintained".

This Current Maintainer of this work is Heiko Oberdiek.

The Base Interpreter refers to any `TeX-Format',
because some files are installed in TDS:tex/generic//.

This work consists of the main source file alphalph.dtx
and the derived files
   alphalph.sty, alphalph.pdf, alphalph.ins, alphalph.drv,
   alphalph-test1.tex, alphalph-test2.tex, alphalph-test3.tex.

\endpreamble
\let\MetaPrefix\DoubleperCent

\generate{%
  \file{alphalph.ins}{\from{alphalph.dtx}{install}}%
  \file{alphalph.drv}{\from{alphalph.dtx}{driver}}%
  \usedir{tex/generic/oberdiek}%
  \file{alphalph.sty}{\from{alphalph.dtx}{package}}%
%  \usedir{doc/latex/oberdiek/test}%
%  \file{alphalph-test1.tex}{\from{alphalph.dtx}{test1}}%
%  \file{alphalph-test2.tex}{\from{alphalph.dtx}{test2,etex}}%
%  \file{alphalph-test3.tex}{\from{alphalph.dtx}{test2,noetex}}%
  \nopreamble
  \nopostamble
%  \usedir{source/latex/oberdiek/catalogue}%
%  \file{alphalph.xml}{\from{alphalph.dtx}{catalogue}}%
}

\catcode32=13\relax% active space
\let =\space%
\Msg{************************************************************************}
\Msg{*}
\Msg{* To finish the installation you have to move the following}
\Msg{* file into a directory searched by TeX:}
\Msg{*}
\Msg{*     alphalph.sty}
\Msg{*}
\Msg{* To produce the documentation run the file `alphalph.drv'}
\Msg{* through LaTeX.}
\Msg{*}
\Msg{* Happy TeXing!}
\Msg{*}
\Msg{************************************************************************}

\endbatchfile
%</install>
%<*ignore>
\fi
%</ignore>
%<*driver>
\NeedsTeXFormat{LaTeX2e}
\ProvidesFile{alphalph.drv}%
  [2016/05/16 v2.5 Convert numbers to letters (HO)]%
\documentclass{ltxdoc}
\usepackage{alphalph}[2016/05/16]
\makeatletter
\newalphalph{\fnsymbolwrap}[wrap]{\@fnsymbol}{}
\newcommand*{\fnsymbolsingle}[1]{%
  \ensuremath{%
    \ifcase#1%
    \or *%
    \or \dagger
    \or \ddagger
    \or \mathsection
    \or \mathparagraph
    \else
      \@ctrerr
    \fi
  }%
}
\makeatother
\newalphalph{\fnsymbolmult}[mult]{\fnsymbolsingle}{}
\usepackage{amsmath}
\DeclareMathOperator{\opDiv}{div}
\DeclareMathOperator{\opMod}{mod}
\newcommand*{\Def}{%
  \ensuremath{%
    \mathrel{\mathop{:}}=%
  }%
}
\usepackage{holtxdoc}[2011/11/22]
\begin{document}
  \DocInput{alphalph.dtx}%
\end{document}
%</driver>
% \fi
%
%
% \CharacterTable
%  {Upper-case    \A\B\C\D\E\F\G\H\I\J\K\L\M\N\O\P\Q\R\S\T\U\V\W\X\Y\Z
%   Lower-case    \a\b\c\d\e\f\g\h\i\j\k\l\m\n\o\p\q\r\s\t\u\v\w\x\y\z
%   Digits        \0\1\2\3\4\5\6\7\8\9
%   Exclamation   \!     Double quote  \"     Hash (number) \#
%   Dollar        \$     Percent       \%     Ampersand     \&
%   Acute accent  \'     Left paren    \(     Right paren   \)
%   Asterisk      \*     Plus          \+     Comma         \,
%   Minus         \-     Point         \.     Solidus       \/
%   Colon         \:     Semicolon     \;     Less than     \<
%   Equals        \=     Greater than  \>     Question mark \?
%   Commercial at \@     Left bracket  \[     Backslash     \\
%   Right bracket \]     Circumflex    \^     Underscore    \_
%   Grave accent  \`     Left brace    \{     Vertical bar  \|
%   Right brace   \}     Tilde         \~}
%
% \GetFileInfo{alphalph.drv}
%
% \title{The \xpackage{alphalph} package}
% \date{2016/05/16 v2.5}
% \author{Heiko Oberdiek\thanks
% {Please report any issues at \url{https://github.com/ho-tex/oberdiek/issues}}\\
% \xemail{heiko.oberdiek at googlemail.com}}
%
% \maketitle
%
% \begin{abstract}
% The package provides methods to represent numbers with a limited
% set of symbols. Both \hologo{LaTeX} and \hologo{plainTeX} are supported.
% \end{abstract}
%
% \tableofcontents
%
% \section{Documentation}
%
% \subsection{Introduction}
% \hologo{LaTeX} counters can be represented in different ways
% by using presentation commands:
% \begin{quote}
%   \cs{arabic}, \cs{roman}, \cs{Roman},\\
%   \cs{alph}, \cs{Alph}, \cs{fnsymbol}
% \end{quote}
% The ranges of supported counter values are more or less
% restricted. Only \cs{arabic} can be used with any counter
% value \hologo{TeX} supports.
% \begin{quote}
% \catcode`\|=12 %
% \begin{tabular}{@{}l|l|l|l@{}}
% Presentation & Supported & Ignored & Error message\\
% command      & domain    & values  & ``Counter too large''\\
% \hline
% \cs{arabic}
%   & \ttfamily -MAX..MAX & &\\
% \cs{roman}, \cs{Roman}
%   & \ttfamily 1..MAX & \ttfamily -MAX..0 &\\
% \cs{alph}, \cs{Alph}
%   & \ttfamily 1..26 & 0 & \ttfamily -MAX..-1, 27..MAX\\
% \cs{fnsymbol}
%   & \ttfamily 1..9 & 0 & \ttfamily -MAX..-1, 10..MAX\\
% \hline
% \end{tabular}\\[1ex]
% \texttt{MAX} = \texttt{2147483647}
% \end{quote}
% Ordinal numbers are often used in documents: numbering
% of chapters, sections, figures, footnotes and so on.
% The layouter chooses \cs{Alph} for chapter numbers
% and \cs{fnsymbol} for footnotes. But what can be done
% if there are more than 26 chapters or more than 10 footnotes?
% This package \xpackage{alphalph} allows to define new
% presentation commands. They rely on a existing command
% and define presentations for values greater the limits.
% Three different methods are provided by the package.
% In the following use cases they are presentated.
%
% \subsection{Use cases}
%
% \subsubsection{Number system based on symbols}
%
% Asume you are writing a book and your lecturer demands
% that chapter numbers must be letters. But you have already
% 30 chapters and you have only 26 letters?
%
% In the decimal system the situation would be clear.
% If you run out of digits, you are using more digits
% to represent a number. This method can be also be used
% for letters. After chapter 26 with |Z| we us |AA|, |AB|, |AC|, and |AD|
% for the remaining chapters.
%
% Happily this package already defines this presentation command:
%\begin{quote}
%\begin{verbatim}
%\usepackage{alphalph}
%\renewcommand*{\thechapter}{%
%  \AlphAlph{\value{chapter}}%
%}
%\end{verbatim}
%\end{quote}
% \cs{AlphAlph} generates:
%   \AlphAlph{1}, \AlphAlph{2}, \AlphAlph{3}, \dots,
%   \AlphAlph{26}, \AlphAlph{27}, \AlphAlph{28}, \dots
%
% The other presentation command is \cs{alphalph} for lowercase letters.
%
% \subsubsection{Wrap symbols around}
%
% Nine footnote symbols are quite a few.
% Too soon the symbols are consumed
% and \hologo{LaTeX} complains with the error ``Counter too large''.
% However, it could be acceptable to start again with the symbols
% from the beginning, especially if there are less than nine symbols
% on a page.
% This could be achieved by a counter reset.
% But finding the right place can be difficult or needs manual
% actions. Also a
% unique counter value can be desirable (e.g. for generating
% unique anchor/link names).
% Package \xpackage{alphalph} allows you to define
% a macro that implements a ``wrap around'', but letting
% the value of the counter untouched:
%\begin{quote}
%\begin{verbatim}
%\usepackage{alphalph}
%\makeatletter
%\newalphalph{\fnsymbolwrap}[wrap]{\@fnsymbol}{}
%\makeatother
%\renewcommand*{\thefootnote}{%
%  \fnsymbolwrap{\value{footnote}}%
%}
%\end{verbatim}
%\end{quote}
% \cs{fnsymbolwrap} generates:
% \fnsymbolwrap{1}~(1), \fnsymbolwrap{2}~(2), \fnsymbolwrap{3}~(3), \dots,
% \fnsymbolwrap{9}~(9), \fnsymbolwrap{10}~(10), \fnsymbolwrap{11}~11, \dots
%
% \subsubsection{Multiple symbols}
%
% \Hologo{LaTeX}'s standard set of footnote symbols contains doubled
% symbols at the higher positions. Could this principle
% be generalized?
% Yes, but first we need a clean footnote symbol list without
% doubled entries, example:
%\begin{quote}
%\begin{verbatim}
%\usepackage{alphalph}
%\makeatletter
%\newcommand*{\fnsymbolsingle}[1]{%
%  \ensuremath{%
%    \ifcase#1%
%    \or *%
%    \or \dagger
%    \or \ddagger
%    \or \mathsection
%    \or \mathparagraph
%    \else
%      \@ctrerr
%    \fi
%  }%
%}
%\makeatother
%\newalphalph{\fnsymbolmult}[mult]{\fnsymbolsingle}{}
%\renewcommand*{\thefootnote}{%
%  \fnsymbolmult{\value{footnote}}%
%}
%\end{verbatim}
%\end{quote}
% The own definition of \cs{fnsymbolsingle} has the
% advantage that this list can easily modified.
% Otherwise you can use \cs{@fnsymbol} directly,
% because it uses the same first five symbols.
%\begin{quote}
%\begin{verbatim}
%\usepackage{alphalph}
%\makeatletter
%\newalphalph{\fnsymbolmult}[mult]{\@fnsymbol}{5}
%\makeatother
%\renewcommand*{\thefootnote}{%
%  \fnsymbolmult{\value{footnote}}%
%}
%\end{verbatim}
%\end{quote}
% \cs{fnsymbolmult} generates:
% \fnsymbolmult{1}~(1), \fnsymbolmult{2}~(2), \fnsymbolmult{3}~(3),
% \fnsymbolmult{4}~(4), \fnsymbolmult{5}~(5), \fnsymbolmult{6}~(6),
% \dots, \fnsymbolmult{16}~16, \fnsymbolmult{17}~17, \dots
%
% \bigskip
% The same method can also be used for the chapter problem
% in the first discussed use case:
%\begin{quote}
%\begin{verbatim}
%\usepackage{alphalph}
%\makeatletter
%\newalphalph{\AlphMult}[mult]{\@Alph}{26}
%\makeatother
%\renewcommand*{\chapter}{%
%  \AlphMult{\value{chapter}}%
%}
%\end{verbatim}
%\end{quote}
% \cs{AlphMult} then generates AA, BB, CC, and DD for chapters 27--30.
%
% \subsection{Glossary}
%
% \begin{description}
% \item[Counter presentation command] is a macro that
%   expects a \hologo{LaTeX} counter name as argument.
%   Numbers cannot be used. Examples:
%   \cs{arabic}, \cs{alph}, \cs{fnsymbol}.
% \item[Number presentation command] is a macro that
%   expects a number as argument. A number is anything
%   that \hologo{TeX} accepts as number including \cs{value}.
%   Examples: \cs{alphalph}, \cs{AlphAlph}, \cs{alphalph@alph}
%
%   However, \cs{alph} or \cs{fnsymbol} are not number
%   presentation commands because they expect a counter
%   name as argument. Happily \hologo{LaTeX} counter presentation
%   commands internally uses number presentation commands
%   with the same name, but prefixed by `@'. Thus
%   \cs{@alph}, \cs{@fnsymbol} are number presentation commands.
% \item[Symbols provider] is a command that can be used
%   to get a list of symbols. For example, \cs{@Alph}
%   provides the 26 uppercase letters from `A' to `Z'.
%   Basically a symbol provider is a number presentation
%   command, usually with a limited range.
% \item[Number of symbols] is the number of the
%   last symbol slot of a symbol provider. Thus \cs{@Alph}
%   generates 26 symbols, \cs{@fnsymbol} provides 9 symbols.
% \end{description}
%
% \subsection{Package usage}
%
% The package \Package{alphalph} can be used with both \hologo{plainTeX}
% and \hologo{LaTeX}:
% \begin{description}
% \item[\hologo{plainTeX}:] |\input alphalph.sty|
% \item[\hologo{LaTeX2e}:]   |\usepackage{alphalph}|\\
%      There aren't any options.
% \end{description}
%
% \subsection{User commands}
%
% \begin{declcs}{AlphAlph} \M{number}\\
%   \cs{alphalph} \M{number}
% \end{declcs}
% Both macros are number presentation commands that
% expects a number as argument. \hologo{LaTeX} counters
% are used with \cs{value}.
%
% The macros represents a number by letters.
% First single letters |A..Z| are used, then
% two letters |AA..ZZ|, three letters |AAA...ZZZ|, \dots
% follow.
%
% Macro \cs{AlphAlph} uses uppercase letters,
% \cs{alphalph} generates the lowercase variant.
% \begin{quote}
%   \catcode`\|=12
%   \def\entry#1{
%     \texttt{#1}
%     & \texttt{\AlphAlph{#1}}
%     & \texttt{\alphalph{#1}}
%   }
%   \begin{tabular}{@{}l|l|l@{}}
%     \meta{number} & \verb|\AlphAlph{|\meta{number}\verb|}|
%                   & \verb|\alphalph{|\meta{number}\verb|}|\\
%     \hline
%     \entry{1}\\
%     \entry{2}\\
%     \entry{26}\\
%     \entry{27}\\
%     \entry{30}\\
%     \entry{2000}\\
%     \entry{3752127}\\
%     \entry{10786572}\\
%     \entry{2147483647}\\
%   \end{tabular}
% \end{quote}
%
% \begin{declcs}{newalphalph}
%   \M{cmd}
%   |[|\meta{method}|]|
%   \M{symbols provider}
%   \M{number of symbols}
% \end{declcs}
% Macro \cs{newalphalph} defines \meta{cmd} as new
% number presentation command. Like \cs{newcommand} an error
% is thrown, if macro \meta{cmd} already exists.
%
% The \meta{method} is one of |alph|, |wrap|, or |mult|.
% The default is |alph|.
%
% As symbol provider a number presentation command
% can be used, e.g. \cs{@fnsymbol}, \cs{@Alph}, or
% \cs{alphalph@alph}.
%
% The last argument is the number of symbols.
% If the argument is empty, then \cs{newalphalph} tries
% to find this number itself. \hologo{LaTeX}'s number presentation
% commands throw an error message, if the number is too large.
% This error message is put in a macro \cs{@ctrerr}.
% Thus \cs{newalphalph} calls the symbol provider and tests
% a number by typesetting it in a temporary box. The error
% macro \cs{@ctrerr} is catched, it proofs that the number
% is not supported. Also if the width of the result is zero
% the number is considered as unavailable.
%
% The empty argument is useful for potentially variable lists.
% However if the end cannot be detected, then the number of
% symbols must be given. This is also a lot faster. Therefore don't
% let the argument empty without reason.
%
% \subsection{Programmer commands}
%
% \begin{declcs}{alphalph@Alph} \M{number}\\
%   \cs{alphalph@alph} \M{number}
% \end{declcs}
% They are basically the same as \cs{@Alph} and \cs{@alph}.
% Some languages of package \xpackage{babel} redefine
% \hologo{LaTeX}'s macros to include some font setup that
% breaks expandibility. Therefore \cs{AlphAlph} and \cs{alphalph}
% are based on \cs{alphalph@Alph} and \cs{alphalph@alph} to get
% the letters. The behaviour of these symbol providers for numbers
% outside the range |1..26| is undefined.
%
% \subsection{Design principles}
%
% \subsubsection{Number presentation commands}
%
% All number presentation commands that this package defines
% (including \cs{alphalph} and \cs{AlphAlph}) have the following
% properties:
% \begin{itemize}
% \item They are fully expandable. This means that they can safely
%       \begin{itemize}
%       \item be written to a file,
%       \item used in moving arguments (\hologo{LaTeX}: they are \emph{robust}),
%       \item used in a \cmd{\csname}-\cmd{\endcsname} pair.
%       \end{itemize}
% \item If the argument is zero or negative, the commands expand
%       to nothing like \cmd{\romannumeral}.
% \item The argument is a \hologo{TeX} number. Anything that would be
%       accepted by \cmd{\number} is a valid argument:
%       \begin{itemize}
%       \item explicite constants,
%       \item macros that expand to a number,
%       \item count registers, \hologo{LaTeX} counter can used
%             via \cmd{\value}, e.\,g.:\\
%             |\alphalph{\value{page}}|
%       \item \dots
%       \end{itemize}
% \item \hologo{eTeX}'s numeric expressions are supported, if \hologo{eTeX} is
%       available. Then \cs{numexpr} is applied to the argument.
%       Package \cs{calc}'s expressions are not supported.
%       That would violate the expandibility.
% \end{itemize}
%
% \subsubsection{General usability}
%
% \begin{description}
% \item[\hologo{TeX} format:]
%   The package does not depend on \hologo{LaTeX}, it can also be used
%   by \hologo{plainTeX}, for example.
% \item[\hologo{eTeX}:] \eTeX is supported, the macros are shorter and
%   faster. But \eTeX's extensions are not requirements.
%   Without \hologo{eTeX}, just the implementation changes. The properties
%   remain unchanged.
% \end{description}
%
% \StopEventually{
% }
%
% \section{Implementation}
% \subsection{Begin\texorpdfstring{ of package}{}}
%    \begin{macrocode}
%<*package>
%    \end{macrocode}
%    Reload check, especially if the package is not used with \LaTeX.
%    \begin{macrocode}
\begingroup\catcode61\catcode48\catcode32=10\relax%
  \catcode13=5 % ^^M
  \endlinechar=13 %
  \catcode35=6 % #
  \catcode39=12 % '
  \catcode44=12 % ,
  \catcode45=12 % -
  \catcode46=12 % .
  \catcode58=12 % :
  \catcode64=11 % @
  \catcode123=1 % {
  \catcode125=2 % }
  \expandafter\let\expandafter\x\csname ver@alphalph.sty\endcsname
  \ifx\x\relax % plain-TeX, first loading
  \else
    \def\empty{}%
    \ifx\x\empty % LaTeX, first loading,
      % variable is initialized, but \ProvidesPackage not yet seen
    \else
      \expandafter\ifx\csname PackageInfo\endcsname\relax
        \def\x#1#2{%
          \immediate\write-1{Package #1 Info: #2.}%
        }%
      \else
        \def\x#1#2{\PackageInfo{#1}{#2, stopped}}%
      \fi
      \x{alphalph}{The package is already loaded}%
      \aftergroup\endinput
    \fi
  \fi
\endgroup%
%    \end{macrocode}
%    Package identification:
%    \begin{macrocode}
\begingroup\catcode61\catcode48\catcode32=10\relax%
  \catcode13=5 % ^^M
  \endlinechar=13 %
  \catcode35=6 % #
  \catcode39=12 % '
  \catcode40=12 % (
  \catcode41=12 % )
  \catcode44=12 % ,
  \catcode45=12 % -
  \catcode46=12 % .
  \catcode47=12 % /
  \catcode58=12 % :
  \catcode64=11 % @
  \catcode91=12 % [
  \catcode93=12 % ]
  \catcode123=1 % {
  \catcode125=2 % }
  \expandafter\ifx\csname ProvidesPackage\endcsname\relax
    \def\x#1#2#3[#4]{\endgroup
      \immediate\write-1{Package: #3 #4}%
      \xdef#1{#4}%
    }%
  \else
    \def\x#1#2[#3]{\endgroup
      #2[{#3}]%
      \ifx#1\@undefined
        \xdef#1{#3}%
      \fi
      \ifx#1\relax
        \xdef#1{#3}%
      \fi
    }%
  \fi
\expandafter\x\csname ver@alphalph.sty\endcsname
\ProvidesPackage{alphalph}%
  [2016/05/16 v2.5 Convert numbers to letters (HO)]%
%    \end{macrocode}
%
% \subsection{Catcodes}
%
%    \begin{macrocode}
\begingroup\catcode61\catcode48\catcode32=10\relax%
  \catcode13=5 % ^^M
  \endlinechar=13 %
  \catcode123=1 % {
  \catcode125=2 % }
  \catcode64=11 % @
  \def\x{\endgroup
    \expandafter\edef\csname AlPh@AtEnd\endcsname{%
      \endlinechar=\the\endlinechar\relax
      \catcode13=\the\catcode13\relax
      \catcode32=\the\catcode32\relax
      \catcode35=\the\catcode35\relax
      \catcode61=\the\catcode61\relax
      \catcode64=\the\catcode64\relax
      \catcode123=\the\catcode123\relax
      \catcode125=\the\catcode125\relax
    }%
  }%
\x\catcode61\catcode48\catcode32=10\relax%
\catcode13=5 % ^^M
\endlinechar=13 %
\catcode35=6 % #
\catcode64=11 % @
\catcode123=1 % {
\catcode125=2 % }
\def\TMP@EnsureCode#1#2{%
  \edef\AlPh@AtEnd{%
    \AlPh@AtEnd
    \catcode#1=\the\catcode#1\relax
  }%
  \catcode#1=#2\relax
}
\TMP@EnsureCode{33}{12}% !
\TMP@EnsureCode{39}{12}% '
\TMP@EnsureCode{40}{12}% (
\TMP@EnsureCode{41}{12}% )
\TMP@EnsureCode{43}{12}% +
\TMP@EnsureCode{44}{12}% ,
\TMP@EnsureCode{46}{12}% .
\TMP@EnsureCode{47}{12}% /
\TMP@EnsureCode{59}{12}% ;
\TMP@EnsureCode{60}{12}% <
\TMP@EnsureCode{62}{12}% >
\TMP@EnsureCode{91}{12}% [
\TMP@EnsureCode{93}{12}% ]
\TMP@EnsureCode{96}{12}% `
\TMP@EnsureCode{124}{12}% |
\edef\AlPh@AtEnd{\AlPh@AtEnd\noexpand\endinput}
%    \end{macrocode}
%
% \subsection{Package loading}
%
%    \begin{macrocode}
\begingroup\expandafter\expandafter\expandafter\endgroup
\expandafter\ifx\csname RequirePackage\endcsname\relax
  \input infwarerr.sty\relax
  \input intcalc.sty\relax
\else
  \RequirePackage{infwarerr}[2007/09/09]%
  \RequirePackage{intcalc}[2007/09/09]%
\fi
%    \end{macrocode}
%
% \subsection{\hologo{eTeX} detection}
%
%    \begin{macrocode}
\begingroup\expandafter\expandafter\expandafter\endgroup
\expandafter\ifx\csname numexpr\endcsname\relax
  \catcode124=9 % '!': ignore
  \catcode43=14 % '+': comment
\else
  \catcode124=14 % '!': comment
  \catcode43=9   % '+': ignore
\fi
%    \end{macrocode}
%
% \subsection{Help macros}
%
%    \begin{macro}{\AlPh@Error}
%    \begin{macrocode}
\def\AlPh@Error#1{%
  \begingroup
    \escapechar=92 % backslash
    \@PackageError{alphalph}{#1}\@ehc
  \endgroup
}
%    \end{macrocode}
%    \end{macro}
%
%    \begin{macro}{\AlPh@IfDefinable}
%    \begin{macrocode}
\begingroup\expandafter\expandafter\expandafter\endgroup
\expandafter\ifx\csname @ifdefinable\endcsname\relax
  \def\AlPh@IfDefinable#1#2{%
    \ifcase\ifx#1\@undefined\else\ifx#1\relax\else1\fi\fi0 %
      #2%
    \else
      \AlPh@Error{%
        Command \string#1 already defined%
      }%
    \fi
  }%
%    \end{macrocode}
%    \end{macro}
%    \begin{macrocode}
\else
%    \end{macrocode}
%    \begin{macro}{\AlPh@IfDefinable}
%    \begin{macrocode}
  \let\AlPh@IfDefinable\@ifdefinable
%    \end{macrocode}
%    \end{macro}
%    \begin{macrocode}
\fi
%    \end{macrocode}
%
%    \begin{macro}{\@ReturnAfterElseFi}
%    \begin{macro}{\@ReturnAfterFi}
%    The following commands moves the `then' and `else' part respectively
%    behind the |\if|-construct. This prevents a too deep |\if|-nesting
%    and so a \hologo{TeX} capacity error because of a limited input stack size.
%    I use this trick in several packages, so I don't prefix these internal
%    commands in order not to have the same macros with different names.
%    (It saves memory.)
%    \begin{macrocode}
\long\def\@ReturnAfterElseFi#1\else#2\fi{\fi#1}
\long\def\@ReturnAfterFi#1\fi{\fi#1}
%    \end{macrocode}
%    \end{macro}
%    \end{macro}
%
%    \begin{macro}{\@gobblefour}
%    \Hologo{LaTeX} defines commands for eating arguments.
%    Define \cs{@gobblefour} if it is not defined (\hologo{plainTeX}).
%    \begin{macrocode}
\expandafter\ifx\csname @gobblefour\endcsname\relax
  \long\def\@gobblefour#1#2#3#4{}%
\fi
%    \end{macrocode}
%    \end{macro}
%
%    \begin{macro}{AlPh@IfOptArg}
%    \begin{macrocode}
\begingroup\expandafter\expandafter\expandafter\endgroup
\expandafter\ifx\csname kernel@ifnextchar\endcsname\relax
  \begingroup\expandafter\expandafter\expandafter\endgroup
  \expandafter\ifx\csname @ifnextchar\endcsname\relax
    \def\AlPh@IfOptArg#1#2{%
      \def\AlPh@TempA{#1}%
      \def\AlPh@TempB{#2}%
      \futurelet\AlPh@Token\AlPh@IfOptArgNext
    }%
    \let\AlPh@BracketLeft=[%]
    \def\AlPh@IfOptArgNext{%
      \ifx\AlPh@Token\AlPh@BracketLeft
        \expandafter\AlPh@TempA
      \else
        \expandafter\AlPh@TempB
      \fi
    }%
  \else
    \def\AlPh@IfOptArg{\@ifnextchar[}%]
  \fi
\else
  \def\AlPh@IfOptArg{\kernel@ifnextchar[}%]
\fi
%    \end{macrocode}
%    \end{macro}
%
% \subsection{Symbol provider}
%
% \subsubsection{Alphabet}
%
%    The output of \cs{alphalph} and \cs{AlphAlph} should be
%    usable as part of command names (see \cs{@namedef}, \cs{csname}, \dots).
%    Unhappily some languages of package \xpackage{babel}
%    redefine \hologo{LaTeX}'s \cs{@alph} and \cs{@Alph} in a manner that
%    they cannot be used in expandable context any more.
%    Therefore package \xpackage{alphalph} provides its own
%    commands.
%
%    \begin{macro}{\alphalph@Alph}
%    \begin{macro}{\alphalph@alph}
%    The two commands \cmd{\AlPh@Alph} and \cmd{\AlPh@alph} convert a
%    number into a letter (uppercase and lowercase respectivly).
%    The character |@| is used as an error symbol,
%    if the number isn't in the range of 1 until 26.
%    Here we need no space after the number |#1|, because the error
%    symbol |@| for the zero case stops scanning the number.
%    This error symbol should not appear anywhere (except for bugs).
%    \begin{macrocode}
\def\alphalph@Alph#1{%
  \ifcase#1%
    @%
  \or A\or B\or C\or D\or E\or F\or G\or H\or I\or J\or K\or L\or M%
  \or N\or O\or P\or Q\or R\or S\or T\or U\or V\or W\or X\or Y\or Z%
  \else
    \AlPh@ctrerr
    @%
  \fi
}
\def\alphalph@alph#1{%
  \ifcase#1%
    @%
  \or a\or b\or c\or d\or e\or f\or g\or h\or i\or j\or k\or l\or m%
  \or n\or o\or p\or q\or r\or s\or t\or u\or v\or w\or x\or y\or z%
  \else
    \AlPh@ctrerr
    @%
  \fi
}
%    \end{macrocode}
%    \end{macro}
%    \end{macro}
%    \begin{macro}{\AlPh@ctrerr}
%    Macro \cs{AlPh@ctrerr} is used as hook for the algorithm
%    to get the available number of symbols.
%    \begin{macrocode}
\def\AlPh@ctrerr{}
%    \end{macrocode}
%    \end{macro}
%
% \subsection{Finding number of symbols}
%
%    \begin{macro}{\AlPh@GetNumberOfSymbols}
%    \noindent
%    |#1|: symbols provider
%    \begin{macrocode}
\def\AlPh@GetNumberOfSymbols#1{%
  \AlPh@TestNumber1!{#1}%
  \ifAlPh@Unavailable
    \def\AlPh@Number{0}%
    \AlPh@Error{No symbols found}%
  \else
    \def\AlPh@Number{1}%
    \AlPh@ExpSearch2!{#1}%
  \fi
}
%    \end{macrocode}
%    \end{macro}
%    \begin{macro}{\ifAlPh@Unavailable}
%    \begin{macrocode}
\let\ifAlPh@Unavailable\iffalse
\def\AlPh@Unavailabletrue{%
  \global\let\ifAlPh@Unavailable\iftrue
}
\def\AlPh@Unavailablefalse{%
  \global\let\ifAlPh@Unavailable\iffalse
}
%    \end{macrocode}
%    \end{macro}
%    \begin{macro}{\AlPh@TestNumber}
%    |#1|: number to be tested\\
%    |#2|: symbols provider
%    \begin{macrocode}
\def\AlPh@TestNumber#1!#2{%
  \AlPh@Unavailablefalse
  \begingroup
    \setbox0=\hbox{%
      \begingroup % color
        \let\@ctrerr\AlPh@Unavailabletrue
        \let\AlPh@ctrerr\AlPh@Unavailabletrue
        #2{#1}%
      \endgroup
    }%
    \ifdim\wd0=0pt %
      \AlPh@Unavailabletrue
    \fi
  \endgroup
}
%    \end{macrocode}
%    \end{macro}
%    \begin{macro}{\AlPh@ExpSearch}
%    |#1|: number to be tested\\
%    |#2|: symbols provider
%    \begin{macrocode}
\def\AlPh@ExpSearch#1!#2{%
  \let\AlPh@Next\relax
  \AlPh@TestNumber#1!{#2}%
  \ifAlPh@Unavailable
    \expandafter\AlPh@BinSearch\AlPh@Number!#1!{#2}%
  \else
    \def\AlPh@Number{#1}%
    \ifnum#1>1073741823 %
      \AlPh@TestNumber2147483647!{#2}%
      \ifAlPh@Unavailable
        \AlPh@BinSearch#1!2147483647!{#2}%
      \else
        \def\AlPh@Number{0}%
        \AlPh@Error{%
          Maximal symbol number not found%
        }%
      \fi
    \else
      \def\AlPh@Next{%
        \expandafter\AlPh@ExpSearch\number\intcalcShl{#1}!{#2}%
      }%
    \fi
  \fi
  \AlPh@Next
}
%    \end{macrocode}
%    \end{macro}
%    \begin{macro}{\AlPh@BinSearch}
%    |#1|: available number\\
%    |#2|: unavailable number, |#2| $>$ |#1|\\
%    |#3|: symbols provider
%    \begin{macrocode}
\def\AlPh@BinSearch#1!#2!#3{%
  \expandafter\AlPh@ProcessBinSearch
  \number\intcalcShr{\intcalcAdd{#1}{#2}}!%
  #1!#2!{#3}%
}
%    \end{macrocode}
%    \end{macro}
%    \begin{macro}{\AlPh@ProcessBinSearch}
%    |#1|: number to be tested, |#2| $\leq$ |#1| $\leq$ |#3|\\
%    |#2|: available number\\
%    |#3|: unavailable number\\
%    |#4|: symbols provider
%    \begin{macrocode}
\def\AlPh@ProcessBinSearch#1!#2!#3!#4{%
  \let\AlPh@Next\relax
  \ifnum#1>#2 %
    \ifnum#1<#3 %
      \AlPh@TestNumber#1!{#4}%
      \ifAlPh@Unavailable
        \def\AlPh@Next{%
          \AlPh@BinSearch#2!#1!{#4}%
        }%
      \else
        \def\AlPh@Next{%
          \AlPh@BinSearch#1!#3!{#4}%
        }%
      \fi
    \else
      \def\AlPh@Number{#2}%
    \fi
  \else
    \def\AlPh@Number{#2}%
  \fi
  \AlPh@Next
}
%    \end{macrocode}
%    \end{macro}
%
% \subsection{Methods}
%
%    The names of method macros start with \cs{AlPh@Method}.
%    These macros do the main job in converting a number to
%    its representation.
%    A method command is called with three arguments.
%    The first argument is the number of symbols. The
%    second argument is the basic macro for converting
%    a number with limited number range. The last parameter
%    is the number that needs converting.
%
% \subsubsection{Common methods}
%
%    \begin{macro}{\AlPh@CheckPositive}
%    \noindent
%    |#1|: number to be checked
%    |#2|: continuation macro\\
%    |#3|: number of symbols (hidden here)\\
%    |#4|: symbol provider (hidden here)
%    \begin{macrocode}
\def\AlPh@CheckPositive#1!#2{%
  \ifnum#1<1 %
    \expandafter\@gobblefour
  \fi
  #2{#1}%
}
%    \end{macrocode}
%    \end{macro}
%
% \subsubsection{Method `alph'}
%
%    \begin{macro}{\AlPh@Method@alph}
%    \noindent
%    |#1|: number of symbols\\
%    |#2|: symbols provider\\
%    |#3|: number to be converted
%    \begin{macrocode}
\def\AlPh@Method@alph#1#2#3{%
  \expandafter\AlPh@CheckPositive
|   \number#3!%
+   \the\numexpr#3!%
    \AlPh@ProcessAlph
    {#1}{#2}%
}
%    \end{macrocode}
%    \end{macro}
%    \begin{macro}{\AlPh@ProcessAlph}
%    |#1|: current number\\
%    |#2|: number of symbols\\
%    |#3|: symbols provider
%    \begin{macrocode}
\def\AlPh@ProcessAlph#1#2#3{%
  \ifnum#1>#2 %
    \@ReturnAfterElseFi{%
      \expandafter\AlPh@StepAlph\number
        \intcalcInc{%
          \intcalcMod{\intcalcDec{#1}}{#2}%
        }%
      \expandafter!\number
        \intcalcDiv{\intcalcDec{#1}}{#2}%
      !{#2}{#3}%
    }%
  \else
    \@ReturnAfterFi{%
      #3{#1}%
    }%
  \fi
}
%    \end{macrocode}
%    \end{macro}
%    \begin{macro}{\AlPh@StepAlph}
%    |#1|: current last digit\\
%    |#2|: new current number\\
%    |#3|: number of symbols\\
%    |#4|: symbols provider
%    \begin{macrocode}
\def\AlPh@StepAlph#1!#2!#3#4{%
  \AlPh@ProcessAlph{#2}{#3}{#4}%
  #4{#1}%
}
%    \end{macrocode}
%    \end{macro}
%
% \subsubsection{Method `wrap'}
%
%    \begin{macro}{\AlPh@Method@wrap}
%    \noindent
%    |#1|: number of symbols\\
%    |#2|: symbols provider\\
%    |#3|: number to be converted
%    \begin{macrocode}
\def\AlPh@Method@wrap#1#2#3{%
  \expandafter\AlPh@CheckPositive
|   \number#3!%
+   \the\numexpr#3!%
    \AlPh@ProcessWrap
    {#1}{#2}%
}
%    \end{macrocode}
%    \end{macro}
%
%    \begin{macro}{\AlPh@ProcessWrap}
%    |#1|: number to be converted\\
%    |#2|: number of symbols\\
%    |#3|: symbols provider
%    \begin{macrocode}
\def\AlPh@ProcessWrap#1#2#3{%
  \ifnum#1>#2 %
    \@ReturnAfterElseFi{%
      \expandafter\AlPh@StepWrap\number
        \intcalcInc{\intcalcMod{\intcalcDec{#1}}{#2}}%
      !{#3}%
    }%
  \else
    \@ReturnAfterFi{%
      #3{#1}%
    }%
  \fi
}
%    \end{macrocode}
%    \end{macro}
%    \begin{macro}{\AlPh@StepWrap}
%    |#1|: final number\\
%    |#2|: symbols provider
%    \begin{macrocode}
\def\AlPh@StepWrap#1!#2{%
  #2{#1}%
}
%    \end{macrocode}
%    \end{macro}
%
% \subsubsection{Method `mult'}
%
%    After the number of symbols is exhausted,
%    repetitions of the symbol are used.
%    \begin{gather*}
%      x \Def \text{number to be converted}\\
%      n \Def \text{number of symbols}\\
%      r \Def \text{repetition length}\\
%      s \Def \text{symbol slot}\\
%      r = ((x - 1) \div n) + 1\\
%      s = ((x - 1) \mod n) + 1
%    \end{gather*}
%    \begin{macro}{\AlPh@Method@mult}
%    \noindent
%    |#1|: number of symbols\\
%    |#2|: symbols provider\\
%    |#3|: number to be converted
%    \begin{macrocode}
\def\AlPh@Method@mult#1#2#3{%
  \expandafter\AlPh@CheckPositive
|   \number#3!%
+   \the\numexpr#3!%
    \AlPh@ProcessMult
    {#1}{#2}%
}
%    \end{macrocode}
%    \end{macro}
%
%    \begin{macro}{\AlPh@ProcessMult}
%    |#1|: number to be converted\\
%    |#2|: number of symbols\\
%    |#3|: symbols provider
%    \begin{macrocode}
\def\AlPh@ProcessMult#1#2#3{%
  \ifnum#1>#2 %
    \@ReturnAfterElseFi{%
      \expandafter\AlPh@StepMult\romannumeral
        \intcalcInc{\intcalcDiv{\intcalcDec{#1}}{#2}}%
        000%
      \expandafter!\number
        \intcalcInc{\intcalcMod{\intcalcDec{#1}}{#2}}%
      !{#3}%
    }%
  \else
    \@ReturnAfterFi{%
      #3{#1}%
    }%
  \fi
}
%    \end{macrocode}
%    \end{macro}
%    \begin{macro}{\AlPh@StepMult}
%    |#1#2|: repetitions coded as list of character `m'\\
%    |#3|: symbol slot\\
%    |#4|: symbols provider
%    \begin{macrocode}
\def\AlPh@StepMult#1#2!#3!#4{%
  \ifx\\#2\\%
  \else
    \@ReturnAfterFi{%
      \AlPh@StepMult#2!#3!{#4}%
    }%
  \fi
  #4{#3}%
}
%    \end{macrocode}
%    \end{macro}
%
% \subsection{User interface}
%
%    \begin{macro}{\newalphalph}
%    Macro \cs{newalphalph} had three arguments in versions below 2.0.
%    For the new method argument we use an optional argument an first
%    position.\\
%    |#1|: cmd\\
%    |[#2]|: method name: |alph| (default), |wrap|, |mult|\\ % hash-ok
%    |#3|: symbols provider\\
%    |#4|: number of symbols
%    \begin{macrocode}
\AlPh@IfDefinable\newalphalph{%
  \def\newalphalph#1{%
    \AlPh@IfOptArg{%
      \AlPh@newalphalph{#1}%
    }{%
      \AlPh@newalphalph{#1}[alph]%
    }%
  }%
}
%    \end{macrocode}
%    \end{macro}
%    \begin{macro}{\AlPh@newalphalph}
%    |#1|: cmd
%    |#2|: method name\\
%    |#3|: symbols provider\\
%    |#4|: number of symbols
%    \begin{macrocode}
\def\AlPh@newalphalph#1[#2]#3#4{%
  \begingroup\expandafter\expandafter\expandafter\endgroup
  \expandafter\ifx\csname AlPh@Method@#2\endcsname\relax
    \AlPh@Error{%
      Unknown method %
|     `#2'%
+     `\detokenize{#2}'%
    }%
  \else
    \ifx\\#4\\%
      \AlPh@GetNumberOfSymbols{#3}%
      \ifcase\AlPh@Number
      \else
        \begingroup
          \escapechar=92 % backslash
          \@PackageInfo{alphalph}{%
            Number of symbols for \string#1 is \AlPh@Number
          }%
        \endgroup
        \expandafter\AlPh@NewAlphAlph
        \csname AlPh@Method@#2\expandafter\endcsname
        \AlPh@Number!{#1}{#3}%
      \fi
    \else
      \expandafter\AlPh@NewAlphAlph
      \csname AlPh@Method@#2\expandafter\endcsname
|     \number#4!%
+     \the\numexpr#4!%
      {#1}{#3}%
    \fi
  \fi
}%
%    \end{macrocode}
%    \end{macro}
%    \begin{macro}{\AlPh@NewAlphAlph}
%    |#1|: method macro\\
%    |#2|: number of symbols\\
%    |#3|: cmd\\
%    |#4|: symbols provider
%    \begin{macrocode}
\def\AlPh@NewAlphAlph#1#2!#3#4{%
  \AlPh@IfDefinable#3{%
    \ifnum#2>0 %
      \def#3{#1{#2}{#4}}%
    \else
      \AlPh@Error{%
          Definition of \string#3 failed,\MessageBreak
          because number of symbols (#2) is not positive%
        }%
    \fi
  }%
}
%    \end{macrocode}
%    \end{macro}
%
%    \begin{macro}{\AlphAlph}
%    \begin{macrocode}
\newalphalph\AlphAlph\alphalph@Alph{26}
%    \end{macrocode}
%    \end{macro}
%    \begin{macro}{\alphalph}
%    \begin{macrocode}
\newalphalph\alphalph\alphalph@alph{26}
%    \end{macrocode}
%    \end{macro}
%
%    \begin{macrocode}
\AlPh@AtEnd%
%</package>
%    \end{macrocode}
%
% \section{Test}
%
% \subsection{Catcode checks for loading}
%
%    \begin{macrocode}
%<*test1>
%    \end{macrocode}
%    \begin{macrocode}
\catcode`\{=1 %
\catcode`\}=2 %
\catcode`\#=6 %
\catcode`\@=11 %
\expandafter\ifx\csname count@\endcsname\relax
  \countdef\count@=255 %
\fi
\expandafter\ifx\csname @gobble\endcsname\relax
  \long\def\@gobble#1{}%
\fi
\expandafter\ifx\csname @firstofone\endcsname\relax
  \long\def\@firstofone#1{#1}%
\fi
\expandafter\ifx\csname loop\endcsname\relax
  \expandafter\@firstofone
\else
  \expandafter\@gobble
\fi
{%
  \def\loop#1\repeat{%
    \def\body{#1}%
    \iterate
  }%
  \def\iterate{%
    \body
      \let\next\iterate
    \else
      \let\next\relax
    \fi
    \next
  }%
  \let\repeat=\fi
}%
\def\RestoreCatcodes{}
\count@=0 %
\loop
  \edef\RestoreCatcodes{%
    \RestoreCatcodes
    \catcode\the\count@=\the\catcode\count@\relax
  }%
\ifnum\count@<255 %
  \advance\count@ 1 %
\repeat

\def\RangeCatcodeInvalid#1#2{%
  \count@=#1\relax
  \loop
    \catcode\count@=15 %
  \ifnum\count@<#2\relax
    \advance\count@ 1 %
  \repeat
}
\def\RangeCatcodeCheck#1#2#3{%
  \count@=#1\relax
  \loop
    \ifnum#3=\catcode\count@
    \else
      \errmessage{%
        Character \the\count@\space
        with wrong catcode \the\catcode\count@\space
        instead of \number#3%
      }%
    \fi
  \ifnum\count@<#2\relax
    \advance\count@ 1 %
  \repeat
}
\def\space{ }
\expandafter\ifx\csname LoadCommand\endcsname\relax
  \def\LoadCommand{\input alphalph.sty\relax}%
\fi
\def\Test{%
  \RangeCatcodeInvalid{0}{47}%
  \RangeCatcodeInvalid{58}{64}%
  \RangeCatcodeInvalid{91}{96}%
  \RangeCatcodeInvalid{123}{255}%
  \catcode`\@=12 %
  \catcode`\\=0 %
  \catcode`\%=14 %
  \LoadCommand
  \RangeCatcodeCheck{0}{36}{15}%
  \RangeCatcodeCheck{37}{37}{14}%
  \RangeCatcodeCheck{38}{47}{15}%
  \RangeCatcodeCheck{48}{57}{12}%
  \RangeCatcodeCheck{58}{63}{15}%
  \RangeCatcodeCheck{64}{64}{12}%
  \RangeCatcodeCheck{65}{90}{11}%
  \RangeCatcodeCheck{91}{91}{15}%
  \RangeCatcodeCheck{92}{92}{0}%
  \RangeCatcodeCheck{93}{96}{15}%
  \RangeCatcodeCheck{97}{122}{11}%
  \RangeCatcodeCheck{123}{255}{15}%
  \RestoreCatcodes
}
\Test
\csname @@end\endcsname
\end
%    \end{macrocode}
%    \begin{macrocode}
%</test1>
%    \end{macrocode}
%
% \section{Macro tests}
%
%    \begin{macrocode}
%<*test2>
\NeedsTeXFormat{LaTeX2e}
\nofiles
\documentclass{article}
%<*noetex>
\makeatletter
\let\saved@numexpr\numexpr
\newcommand*{\DisableNumexpr}{%
  \let\numexpr\@undefined
}
\newcommand*{\RestoreNumexpr}{%
  \let\numexpr\saved@numexpr
}
\DisableNumexpr
%</noetex>
\usepackage{alphalph}[2016/05/16]
%<noetex>\RestoreNumexpr
\usepackage{qstest}
\IncludeTests{*}
\LogTests{log}{*}{*}

\newcommand*{\TestCmd}[3]{%
  \setbox0=\hbox{%
%<noetex>    \DisableNumexpr
    \edef\TestString{#1{#2}}%
    \expandafter\Expect\expandafter{\TestString}{#3}%
    \edef\TestString{#1{#2} }%
    \expandafter\Expect\expandafter{\TestString}{#3 }%
  }%
  \Expect*{\the\wd0}{0.0pt}%
}

\makeatletter
\newalphalph\LaTeXAlphAlph\@Alph{26}
\newalphalph\LaTeXalphalph\@alph{26}
\newalphalph\AlphWrap[wrap]\alphalph@Alph{26}
\newalphalph\alphwrap[wrap]\alphalph@alph{26}
\newalphalph\LaTeXAlphWrap[wrap]\@Alph{26}
\newalphalph\LaTeXalphwrap[wrap]\@alph{26}
\def\LastSymbol#1{%
  \ifx\\#1\\%
  \else
    \@LastSymbol#1\@nil
  \fi
}
\def\@LastSymbol#1#2\@nil{%
  \ifx\\#2\\%
    #1%
  \else
    \@LastSymbol#2\@nil
  \fi
}
\makeatother
\newcommand*{\TestAlph}[2]{%
  \uppercase{\TestCallCmd\AlphAlph{#2}}{#1}%
  \lowercase{\TestCallCmd\alphalph{#2}}{#1}%
  \uppercase{\TestCallCmd\LaTeXAlphAlph{#2}}{#1}%
  \lowercase{\TestCallCmd\LaTeXalphalph{#2}}{#1}%
  \edef\WrapString{\LastSymbol{#2}}%
  \expandafter\TestAlphWrap\expandafter{\WrapString}{#1}%
}
\newcommand*{\TestAlphWrap}[2]{%
  \uppercase{\TestCallCmd\AlphWrap{#1}}{#2}%
  \lowercase{\TestCallCmd\alphwrap{#1}}{#2}%
  \uppercase{\TestCallCmd\LaTeXAlphWrap{#1}}{#2}%
  \lowercase{\TestCallCmd\LaTeXalphwrap{#1}}{#2}%
}
\newcommand*{\TestCallCmd}[3]{%
  \TestCmd#1{#3}{#2}%
}
\begin{qstest}{AlphSymbols}{alphalph, AlphAlph, symbols}
  \TestAlph{1}{a}%
  \TestAlph{2}{b}%
  \TestAlph{3}{c}%
  \TestAlph{4}{d}%
  \TestAlph{5}{e}%
  \TestAlph{6}{f}%
  \TestAlph{7}{g}%
  \TestAlph{8}{h}%
  \TestAlph{9}{i}%
  \TestAlph{10}{j}%
  \TestAlph{11}{k}%
  \TestAlph{12}{l}%
  \TestAlph{13}{m}%
  \TestAlph{14}{n}%
  \TestAlph{15}{o}%
  \TestAlph{16}{p}%
  \TestAlph{17}{q}%
  \TestAlph{18}{r}%
  \TestAlph{19}{s}%
  \TestAlph{20}{t}%
  \TestAlph{21}{u}%
  \TestAlph{22}{v}%
  \TestAlph{23}{w}%
  \TestAlph{24}{x}%
  \TestAlph{25}{y}%
  \TestAlph{26}{z}%
\end{qstest}
\begin{qstest}{AlphRange}{alphalph, range}
  \TestAlph{0}{}%
  \TestAlph{-1}{}%
  \TestAlph{-2147483647}{}%
  \TestAlph{27}{aa}%
  \TestAlph{28}{ab}%
  \TestAlph{52}{az}%
  \TestAlph{53}{ba}%
  \TestAlph{78}{bz}%
  \TestAlph{79}{ca}%
  \TestAlph{702}{zz}%
  \TestAlph{703}{aaa}%
  \TestAlph{2147483647}{fxshrxw}%
\end{qstest}

\makeatletter
\newcommand*{\myvocals}[1]{%
  \ifcase#1X\or A\or E\or I\or O\or U\else Y\fi
}
\makeatother
\newalphalph\vocalsvocals\myvocals{5}
\newcommand*{\TestVocals}{%
  \TestCmd\vocalsvocals
}
\begin{qstest}{vocals}{vocals}
  \TestVocals{0}{}%
  \TestVocals{1}{A}%
  \TestVocals{2}{E}%
  \TestVocals{3}{I}%
  \TestVocals{4}{O}%
  \TestVocals{5}{U}%
  \TestVocals{6}{AA}%
  \TestVocals{7}{AE}%
  \TestVocals{8}{AI}%
  \TestVocals{9}{AO}%
  \TestVocals{10}{AU}%
  \TestVocals{11}{EA}%
  \TestVocals{24}{OO}%
  \TestVocals{25}{OU}%
  \TestVocals{26}{UA}%
  \TestVocals{29}{UO}%
  \TestVocals{30}{UU}%
  \TestVocals{31}{AAA}%
  \TestVocals{155}{UUU}%
  \TestVocals{156}{AAAA}%
  \TestVocals{2147483647}{AIIOOEEIOIIUOE}%
\end{qstest}

\makeatletter
\newalphalph\AlphMult[mult]{\alphalph@Alph}{26}
\newalphalph\alphmult[mult]{\alphalph@alph}{26}
\newalphalph\LaTeXAlphMult[mult]{\@Alph}{26}
\newalphalph\LaTeXalphmult[mult]{\@alph}{26}
\makeatother
\newcommand*{\TestMult}[2]{%
  \uppercase{\TestCallCmd\AlphMult{#2}}{#1}%
  \lowercase{\TestCallCmd\alphmult{#2}}{#1}%
  \uppercase{\TestCallCmd\LaTeXAlphMult{#2}}{#1}%
  \lowercase{\TestCallCmd\LaTeXalphmult{#2}}{#1}%
}
\begin{qstest}{mult}{mult}
  \TestMult{0}{}%
  \TestMult{-1}{}%
  \TestMult{-2147483647}{}%
  \TestMult{1}{a}%
  \TestMult{2}{b}%
  \TestMult{26}{z}%
  \TestMult{27}{aa}%
  \TestMult{28}{bb}%
  \TestMult{52}{zz}%
  \TestMult{53}{aaa}%
  \TestMult{54}{bbb}%
  \TestMult{259}{yyyyyyyyyy}%
  \TestMult{260}{zzzzzzzzzz}%
  \TestMult{261}{aaaaaaaaaaa}%
  \TestMult{262}{bbbbbbbbbbb}%
\end{qstest}

\def\myvocalsB#1{%
  \ifcase#1\or A\or E\or I\or O\or U\fi
}
\begin{qstest}{symbolnum}{symbolnum}
  \makeatletter
  \def\Test#1#2{%
    \let\TestCmd\relax
    \newalphalph\TestCmd{#1}{}%
    \Expect*{\AlPh@Number}{#2}%
  }%
  \Test\@alph{26}%
  \Test\@Alph{26}%
  \Test\@fnsymbol{9}%
  \Test\myvocalsB{5}%
  \Test\alphalph@alph{26}%
  \Test\alphalph@Alph{26}%
\end{qstest}

\begin{qstest}{list}{list}
  \makeatletter
  \def\catch#1\relax{%
    \def\FoundList{\catch#1}%
  }%
  \def\Test[#1]#2#3#4{%
    \let\testcmd\relax
    \newalphalph\testcmd[{#1}]{\catch}{#2}%
    \testcmd{#3}|\relax
    \expandafter\Expect\expandafter{\FoundList}{#4|}%
    %
    \let\SavedCatch\catch
    \def\catch{\noexpand\catch\noexpand\foo}%
    \edef\Result{#4|}%
    \@onelevel@sanitize\Result
    \let\catch\SavedCatch
    \let\testcmd\relax
    \newalphalph\testcmd[{#1}]{\catch\foo}{#2}%
    \testcmd{#3}|\relax
    \@onelevel@sanitize\FoundList
    \Expect*{\FoundList}*{\Result}%
  }%
  \Test[alph]{26}{3}{\catch{3}}%
  \Test[alph]{26}{12}{\catch{12}}%
  \Test[alph]{26}{27}{\catch{1}\catch{1}}%
  \Test[alph]{26}{78}{\catch{2}\catch{26}}%
  \Test[wrap]{26}{7}{\catch{7}}%
  \Test[wrap]{26}{14}{\catch{14}}%
  \Test[wrap]{26}{80}{\catch{2}}%
  \Test[wrap]{26}{700}{\catch{24}}%
  \Test[mult]{26}{4}{\catch{4}}%
  \Test[mult]{26}{17}{\catch{17}}%
  \Test[mult]{26}{54}{\catch{2}\catch{2}\catch{2}}%
\end{qstest}

\begin{document}
\end{document}
%</test2>
%    \end{macrocode}
%
% \section{Installation}
%
% \subsection{Download}
%
% \paragraph{Package.} This package is available on
% CTAN\footnote{\CTANpkg{alphalph}}:
% \begin{description}
% \item[\CTAN{macros/latex/contrib/oberdiek/alphalph.dtx}] The source file.
% \item[\CTAN{macros/latex/contrib/oberdiek/alphalph.pdf}] Documentation.
% \end{description}
%
%
% \paragraph{Bundle.} All the packages of the bundle `oberdiek'
% are also available in a TDS compliant ZIP archive. There
% the packages are already unpacked and the documentation files
% are generated. The files and directories obey the TDS standard.
% \begin{description}
% \item[\CTANinstall{install/macros/latex/contrib/oberdiek.tds.zip}]
% \end{description}
% \emph{TDS} refers to the standard ``A Directory Structure
% for \TeX\ Files'' (\CTAN{tds/tds.pdf}). Directories
% with \xfile{texmf} in their name are usually organized this way.
%
% \subsection{Bundle installation}
%
% \paragraph{Unpacking.} Unpack the \xfile{oberdiek.tds.zip} in the
% TDS tree (also known as \xfile{texmf} tree) of your choice.
% Example (linux):
% \begin{quote}
%   |unzip oberdiek.tds.zip -d ~/texmf|
% \end{quote}
%
% \paragraph{Script installation.}
% Check the directory \xfile{TDS:scripts/oberdiek/} for
% scripts that need further installation steps.
% Package \xpackage{attachfile2} comes with the Perl script
% \xfile{pdfatfi.pl} that should be installed in such a way
% that it can be called as \texttt{pdfatfi}.
% Example (linux):
% \begin{quote}
%   |chmod +x scripts/oberdiek/pdfatfi.pl|\\
%   |cp scripts/oberdiek/pdfatfi.pl /usr/local/bin/|
% \end{quote}
%
% \subsection{Package installation}
%
% \paragraph{Unpacking.} The \xfile{.dtx} file is a self-extracting
% \docstrip\ archive. The files are extracted by running the
% \xfile{.dtx} through \plainTeX:
% \begin{quote}
%   \verb|tex alphalph.dtx|
% \end{quote}
%
% \paragraph{TDS.} Now the different files must be moved into
% the different directories in your installation TDS tree
% (also known as \xfile{texmf} tree):
% \begin{quote}
% \def\t{^^A
% \begin{tabular}{@{}>{\ttfamily}l@{ $\rightarrow$ }>{\ttfamily}l@{}}
%   alphalph.sty & tex/generic/oberdiek/alphalph.sty\\
%   alphalph.pdf & doc/latex/oberdiek/alphalph.pdf\\
%   test/alphalph-test1.tex & doc/latex/oberdiek/test/alphalph-test1.tex\\
%   test/alphalph-test2.tex & doc/latex/oberdiek/test/alphalph-test2.tex\\
%   test/alphalph-test3.tex & doc/latex/oberdiek/test/alphalph-test3.tex\\
%   alphalph.dtx & source/latex/oberdiek/alphalph.dtx\\
% \end{tabular}^^A
% }^^A
% \sbox0{\t}^^A
% \ifdim\wd0>\linewidth
%   \begingroup
%     \advance\linewidth by\leftmargin
%     \advance\linewidth by\rightmargin
%   \edef\x{\endgroup
%     \def\noexpand\lw{\the\linewidth}^^A
%   }\x
%   \def\lwbox{^^A
%     \leavevmode
%     \hbox to \linewidth{^^A
%       \kern-\leftmargin\relax
%       \hss
%       \usebox0
%       \hss
%       \kern-\rightmargin\relax
%     }^^A
%   }^^A
%   \ifdim\wd0>\lw
%     \sbox0{\small\t}^^A
%     \ifdim\wd0>\linewidth
%       \ifdim\wd0>\lw
%         \sbox0{\footnotesize\t}^^A
%         \ifdim\wd0>\linewidth
%           \ifdim\wd0>\lw
%             \sbox0{\scriptsize\t}^^A
%             \ifdim\wd0>\linewidth
%               \ifdim\wd0>\lw
%                 \sbox0{\tiny\t}^^A
%                 \ifdim\wd0>\linewidth
%                   \lwbox
%                 \else
%                   \usebox0
%                 \fi
%               \else
%                 \lwbox
%               \fi
%             \else
%               \usebox0
%             \fi
%           \else
%             \lwbox
%           \fi
%         \else
%           \usebox0
%         \fi
%       \else
%         \lwbox
%       \fi
%     \else
%       \usebox0
%     \fi
%   \else
%     \lwbox
%   \fi
% \else
%   \usebox0
% \fi
% \end{quote}
% If you have a \xfile{docstrip.cfg} that configures and enables \docstrip's
% TDS installing feature, then some files can already be in the right
% place, see the documentation of \docstrip.
%
% \subsection{Refresh file name databases}
%
% If your \TeX~distribution
% (\teTeX, \mikTeX, \dots) relies on file name databases, you must refresh
% these. For example, \teTeX\ users run \verb|texhash| or
% \verb|mktexlsr|.
%
% \subsection{Some details for the interested}
%
% \paragraph{Attached source.}
%
% The PDF documentation on CTAN also includes the
% \xfile{.dtx} source file. It can be extracted by
% AcrobatReader 6 or higher. Another option is \textsf{pdftk},
% e.g. unpack the file into the current directory:
% \begin{quote}
%   \verb|pdftk alphalph.pdf unpack_files output .|
% \end{quote}
%
% \paragraph{Unpacking with \LaTeX.}
% The \xfile{.dtx} chooses its action depending on the format:
% \begin{description}
% \item[\plainTeX:] Run \docstrip\ and extract the files.
% \item[\LaTeX:] Generate the documentation.
% \end{description}
% If you insist on using \LaTeX\ for \docstrip\ (really,
% \docstrip\ does not need \LaTeX), then inform the autodetect routine
% about your intention:
% \begin{quote}
%   \verb|latex \let\install=y\input{alphalph.dtx}|
% \end{quote}
% Do not forget to quote the argument according to the demands
% of your shell.
%
% \paragraph{Generating the documentation.}
% You can use both the \xfile{.dtx} or the \xfile{.drv} to generate
% the documentation. The process can be configured by the
% configuration file \xfile{ltxdoc.cfg}. For instance, put this
% line into this file, if you want to have A4 as paper format:
% \begin{quote}
%   \verb|\PassOptionsToClass{a4paper}{article}|
% \end{quote}
% An example follows how to generate the
% documentation with pdf\LaTeX:
% \begin{quote}
%\begin{verbatim}
%pdflatex alphalph.dtx
%makeindex -s gind.ist alphalph.idx
%pdflatex alphalph.dtx
%makeindex -s gind.ist alphalph.idx
%pdflatex alphalph.dtx
%\end{verbatim}
% \end{quote}
%
% \begin{History}
%   \begin{Version}{1999/03/19 v0.1}
%   \item
%     The first version was built as a response to a
%     \URL{\link{question}}^^A
%     {https://groups.google.com/group/comp.text.tex/msg/17a74cd721641038}^^A
%     of \NameEmail{Will Douglas}{william.douglas@wolfson.ox.ac.uk}
%     and the
%     \URL{\link{request}}^^A
%     {https://groups.google.com/group/comp.text.tex/msg/8f9768825640315f}^^A
%     of \NameEmail{Donald Arsenau}{asnd@reg.triumf.ca},
%     published in the newsgroup
%     \href{news:comp.text.tex}{comp.text.tex}:
%     \URL{``\link{Re: alph counters \textgreater\ 26}''}^^A
%     {https://groups.google.com/group/comp.text.tex/msg/cec563eef8bf65d0}
%   \item
%     Copyright: LPPL (\CTAN{macros/latex/base/lppl.txt})
%   \end{Version}
%   \begin{Version}{1999/04/12 v1.0}
%   \item
%     Documentation added in dtx format.
%   \item
%     \hologo{eTeX} support added.
%   \end{Version}
%   \begin{Version}{1999/04/13 v1.1}
%   \item
%     Minor documentation change.
%   \item
%     First CTAN release.
%   \end{Version}
%   \begin{Version}{1999/06/26 v1.2}
%   \item
%     First generic code about \cmd{\ProvidesPackage} improved.
%   \item
%     Documentation: Installation part revised.
%   \end{Version}
%   \begin{Version}{2006/02/20 v1.3}
%   \item
%     Reload check (for \hologo{plainTeX})
%   \item
%     New DTX framework.
%   \item
%     LPPL 1.3
%   \end{Version}
%   \begin{Version}{2006/05/30 v1.4}
%   \item
%     \cs{newalphalph} added.
%   \end{Version}
%   \begin{Version}{2007/04/11 v1.5}
%   \item
%     Line ends sanitized.
%   \end{Version}
%   \begin{Version}{2007/09/09 v2.0}
%   \item
%     New implementation that uses package \cs{intcalc}.
%     This removes the dependency on \hologo{eTeX}.
%   \item
%     \cs{newalphalph} is extended to support new methods `wrap' and 'multi'.
%   \item
%     Documentation rewritten.
%   \end{Version}
%   \begin{Version}{2008/08/11 v2.1}
%   \item
%     Code is not changed.
%   \item
%     URLs updated from \texttt{www.dejanews.com}
%     to \texttt{groups.google.com}.
%   \end{Version}
%   \begin{Version}{2010/03/01 v2.2}
%   \item
%     Compatibility with \hologo{iniTeX}.
%   \end{Version}
%   \begin{Version}{2010/04/18 v2.3}
%   \item
%     Documentation fixes (Martin M\"unch).
%   \end{Version}
%   \begin{Version}{2011/05/13 v2.4}
%   \item
%     Documentation fixes (Jim Diamond) and using package
%     \xpackage{hologo} for the documentation.
%   \item
%     Catalogue file added.
%   \end{Version}
%   \begin{Version}{2016/05/16 v2.5}
%   \item
%     Documentation updates.
%   \end{Version}
% \end{History}
%
% \PrintIndex
%
% \Finale
\endinput
|
% \end{quote}
% Do not forget to quote the argument according to the demands
% of your shell.
%
% \paragraph{Generating the documentation.}
% You can use both the \xfile{.dtx} or the \xfile{.drv} to generate
% the documentation. The process can be configured by the
% configuration file \xfile{ltxdoc.cfg}. For instance, put this
% line into this file, if you want to have A4 as paper format:
% \begin{quote}
%   \verb|\PassOptionsToClass{a4paper}{article}|
% \end{quote}
% An example follows how to generate the
% documentation with pdf\LaTeX:
% \begin{quote}
%\begin{verbatim}
%pdflatex alphalph.dtx
%makeindex -s gind.ist alphalph.idx
%pdflatex alphalph.dtx
%makeindex -s gind.ist alphalph.idx
%pdflatex alphalph.dtx
%\end{verbatim}
% \end{quote}
%
% \begin{History}
%   \begin{Version}{1999/03/19 v0.1}
%   \item
%     The first version was built as a response to a
%     \URL{\link{question}}^^A
%     {https://groups.google.com/group/comp.text.tex/msg/17a74cd721641038}^^A
%     of \NameEmail{Will Douglas}{william.douglas@wolfson.ox.ac.uk}
%     and the
%     \URL{\link{request}}^^A
%     {https://groups.google.com/group/comp.text.tex/msg/8f9768825640315f}^^A
%     of \NameEmail{Donald Arsenau}{asnd@reg.triumf.ca},
%     published in the newsgroup
%     \href{news:comp.text.tex}{comp.text.tex}:
%     \URL{``\link{Re: alph counters \textgreater\ 26}''}^^A
%     {https://groups.google.com/group/comp.text.tex/msg/cec563eef8bf65d0}
%   \item
%     Copyright: LPPL (\CTAN{macros/latex/base/lppl.txt})
%   \end{Version}
%   \begin{Version}{1999/04/12 v1.0}
%   \item
%     Documentation added in dtx format.
%   \item
%     \hologo{eTeX} support added.
%   \end{Version}
%   \begin{Version}{1999/04/13 v1.1}
%   \item
%     Minor documentation change.
%   \item
%     First CTAN release.
%   \end{Version}
%   \begin{Version}{1999/06/26 v1.2}
%   \item
%     First generic code about \cmd{\ProvidesPackage} improved.
%   \item
%     Documentation: Installation part revised.
%   \end{Version}
%   \begin{Version}{2006/02/20 v1.3}
%   \item
%     Reload check (for \hologo{plainTeX})
%   \item
%     New DTX framework.
%   \item
%     LPPL 1.3
%   \end{Version}
%   \begin{Version}{2006/05/30 v1.4}
%   \item
%     \cs{newalphalph} added.
%   \end{Version}
%   \begin{Version}{2007/04/11 v1.5}
%   \item
%     Line ends sanitized.
%   \end{Version}
%   \begin{Version}{2007/09/09 v2.0}
%   \item
%     New implementation that uses package \cs{intcalc}.
%     This removes the dependency on \hologo{eTeX}.
%   \item
%     \cs{newalphalph} is extended to support new methods `wrap' and 'multi'.
%   \item
%     Documentation rewritten.
%   \end{Version}
%   \begin{Version}{2008/08/11 v2.1}
%   \item
%     Code is not changed.
%   \item
%     URLs updated from \texttt{www.dejanews.com}
%     to \texttt{groups.google.com}.
%   \end{Version}
%   \begin{Version}{2010/03/01 v2.2}
%   \item
%     Compatibility with \hologo{iniTeX}.
%   \end{Version}
%   \begin{Version}{2010/04/18 v2.3}
%   \item
%     Documentation fixes (Martin M\"unch).
%   \end{Version}
%   \begin{Version}{2011/05/13 v2.4}
%   \item
%     Documentation fixes (Jim Diamond) and using package
%     \xpackage{hologo} for the documentation.
%   \item
%     Catalogue file added.
%   \end{Version}
%   \begin{Version}{2016/05/16 v2.5}
%   \item
%     Documentation updates.
%   \end{Version}
% \end{History}
%
% \PrintIndex
%
% \Finale
\endinput

%        (quote the arguments according to the demands of your shell)
%
% Documentation:
%    (a) If alphalph.drv is present:
%           latex alphalph.drv
%    (b) Without alphalph.drv:
%           latex alphalph.dtx; ...
%    The class ltxdoc loads the configuration file ltxdoc.cfg
%    if available. Here you can specify further options, e.g.
%    use A4 as paper format:
%       \PassOptionsToClass{a4paper}{article}
%
%    Programm calls to get the documentation (example):
%       pdflatex alphalph.dtx
%       makeindex -s gind.ist alphalph.idx
%       pdflatex alphalph.dtx
%       makeindex -s gind.ist alphalph.idx
%       pdflatex alphalph.dtx
%
% Installation:
%    TDS:tex/generic/oberdiek/alphalph.sty
%    TDS:doc/latex/oberdiek/alphalph.pdf
%    TDS:doc/latex/oberdiek/test/alphalph-test1.tex
%    TDS:doc/latex/oberdiek/test/alphalph-test2.tex
%    TDS:doc/latex/oberdiek/test/alphalph-test3.tex
%    TDS:source/latex/oberdiek/alphalph.dtx
%
%<*ignore>
\begingroup
  \catcode123=1 %
  \catcode125=2 %
  \def\x{LaTeX2e}%
\expandafter\endgroup
\ifcase 0\ifx\install y1\fi\expandafter
         \ifx\csname processbatchFile\endcsname\relax\else1\fi
         \ifx\fmtname\x\else 1\fi\relax
\else\csname fi\endcsname
%</ignore>
%<*install>
\input docstrip.tex
\Msg{************************************************************************}
\Msg{* Installation}
\Msg{* Package: alphalph 2016/05/16 v2.5 Convert numbers to letters (HO)}
\Msg{************************************************************************}

\keepsilent
\askforoverwritefalse

\let\MetaPrefix\relax
\preamble

This is a generated file.

Project: alphalph
Version: 2016/05/16 v2.5

Copyright (C) 1999, 2006-2008, 2010, 2011 by
   Heiko Oberdiek <heiko.oberdiek at googlemail.com>

This work may be distributed and/or modified under the
conditions of the LaTeX Project Public License, either
version 1.3c of this license or (at your option) any later
version. This version of this license is in
   http://www.latex-project.org/lppl/lppl-1-3c.txt
and the latest version of this license is in
   http://www.latex-project.org/lppl.txt
and version 1.3 or later is part of all distributions of
LaTeX version 2005/12/01 or later.

This work has the LPPL maintenance status "maintained".

This Current Maintainer of this work is Heiko Oberdiek.

The Base Interpreter refers to any `TeX-Format',
because some files are installed in TDS:tex/generic//.

This work consists of the main source file alphalph.dtx
and the derived files
   alphalph.sty, alphalph.pdf, alphalph.ins, alphalph.drv,
   alphalph-test1.tex, alphalph-test2.tex, alphalph-test3.tex.

\endpreamble
\let\MetaPrefix\DoubleperCent

\generate{%
  \file{alphalph.ins}{\from{alphalph.dtx}{install}}%
  \file{alphalph.drv}{\from{alphalph.dtx}{driver}}%
  \usedir{tex/generic/oberdiek}%
  \file{alphalph.sty}{\from{alphalph.dtx}{package}}%
%  \usedir{doc/latex/oberdiek/test}%
%  \file{alphalph-test1.tex}{\from{alphalph.dtx}{test1}}%
%  \file{alphalph-test2.tex}{\from{alphalph.dtx}{test2,etex}}%
%  \file{alphalph-test3.tex}{\from{alphalph.dtx}{test2,noetex}}%
  \nopreamble
  \nopostamble
%  \usedir{source/latex/oberdiek/catalogue}%
%  \file{alphalph.xml}{\from{alphalph.dtx}{catalogue}}%
}

\catcode32=13\relax% active space
\let =\space%
\Msg{************************************************************************}
\Msg{*}
\Msg{* To finish the installation you have to move the following}
\Msg{* file into a directory searched by TeX:}
\Msg{*}
\Msg{*     alphalph.sty}
\Msg{*}
\Msg{* To produce the documentation run the file `alphalph.drv'}
\Msg{* through LaTeX.}
\Msg{*}
\Msg{* Happy TeXing!}
\Msg{*}
\Msg{************************************************************************}

\endbatchfile
%</install>
%<*ignore>
\fi
%</ignore>
%<*driver>
\NeedsTeXFormat{LaTeX2e}
\ProvidesFile{alphalph.drv}%
  [2016/05/16 v2.5 Convert numbers to letters (HO)]%
\documentclass{ltxdoc}
\usepackage{alphalph}[2016/05/16]
\makeatletter
\newalphalph{\fnsymbolwrap}[wrap]{\@fnsymbol}{}
\newcommand*{\fnsymbolsingle}[1]{%
  \ensuremath{%
    \ifcase#1%
    \or *%
    \or \dagger
    \or \ddagger
    \or \mathsection
    \or \mathparagraph
    \else
      \@ctrerr
    \fi
  }%
}
\makeatother
\newalphalph{\fnsymbolmult}[mult]{\fnsymbolsingle}{}
\usepackage{amsmath}
\DeclareMathOperator{\opDiv}{div}
\DeclareMathOperator{\opMod}{mod}
\newcommand*{\Def}{%
  \ensuremath{%
    \mathrel{\mathop{:}}=%
  }%
}
\usepackage{holtxdoc}[2011/11/22]
\begin{document}
  \DocInput{alphalph.dtx}%
\end{document}
%</driver>
% \fi
%
%
% \CharacterTable
%  {Upper-case    \A\B\C\D\E\F\G\H\I\J\K\L\M\N\O\P\Q\R\S\T\U\V\W\X\Y\Z
%   Lower-case    \a\b\c\d\e\f\g\h\i\j\k\l\m\n\o\p\q\r\s\t\u\v\w\x\y\z
%   Digits        \0\1\2\3\4\5\6\7\8\9
%   Exclamation   \!     Double quote  \"     Hash (number) \#
%   Dollar        \$     Percent       \%     Ampersand     \&
%   Acute accent  \'     Left paren    \(     Right paren   \)
%   Asterisk      \*     Plus          \+     Comma         \,
%   Minus         \-     Point         \.     Solidus       \/
%   Colon         \:     Semicolon     \;     Less than     \<
%   Equals        \=     Greater than  \>     Question mark \?
%   Commercial at \@     Left bracket  \[     Backslash     \\
%   Right bracket \]     Circumflex    \^     Underscore    \_
%   Grave accent  \`     Left brace    \{     Vertical bar  \|
%   Right brace   \}     Tilde         \~}
%
% \GetFileInfo{alphalph.drv}
%
% \title{The \xpackage{alphalph} package}
% \date{2016/05/16 v2.5}
% \author{Heiko Oberdiek\thanks
% {Please report any issues at \url{https://github.com/ho-tex/oberdiek/issues}}\\
% \xemail{heiko.oberdiek at googlemail.com}}
%
% \maketitle
%
% \begin{abstract}
% The package provides methods to represent numbers with a limited
% set of symbols. Both \hologo{LaTeX} and \hologo{plainTeX} are supported.
% \end{abstract}
%
% \tableofcontents
%
% \section{Documentation}
%
% \subsection{Introduction}
% \hologo{LaTeX} counters can be represented in different ways
% by using presentation commands:
% \begin{quote}
%   \cs{arabic}, \cs{roman}, \cs{Roman},\\
%   \cs{alph}, \cs{Alph}, \cs{fnsymbol}
% \end{quote}
% The ranges of supported counter values are more or less
% restricted. Only \cs{arabic} can be used with any counter
% value \hologo{TeX} supports.
% \begin{quote}
% \catcode`\|=12 %
% \begin{tabular}{@{}l|l|l|l@{}}
% Presentation & Supported & Ignored & Error message\\
% command      & domain    & values  & ``Counter too large''\\
% \hline
% \cs{arabic}
%   & \ttfamily -MAX..MAX & &\\
% \cs{roman}, \cs{Roman}
%   & \ttfamily 1..MAX & \ttfamily -MAX..0 &\\
% \cs{alph}, \cs{Alph}
%   & \ttfamily 1..26 & 0 & \ttfamily -MAX..-1, 27..MAX\\
% \cs{fnsymbol}
%   & \ttfamily 1..9 & 0 & \ttfamily -MAX..-1, 10..MAX\\
% \hline
% \end{tabular}\\[1ex]
% \texttt{MAX} = \texttt{2147483647}
% \end{quote}
% Ordinal numbers are often used in documents: numbering
% of chapters, sections, figures, footnotes and so on.
% The layouter chooses \cs{Alph} for chapter numbers
% and \cs{fnsymbol} for footnotes. But what can be done
% if there are more than 26 chapters or more than 10 footnotes?
% This package \xpackage{alphalph} allows to define new
% presentation commands. They rely on a existing command
% and define presentations for values greater the limits.
% Three different methods are provided by the package.
% In the following use cases they are presentated.
%
% \subsection{Use cases}
%
% \subsubsection{Number system based on symbols}
%
% Asume you are writing a book and your lecturer demands
% that chapter numbers must be letters. But you have already
% 30 chapters and you have only 26 letters?
%
% In the decimal system the situation would be clear.
% If you run out of digits, you are using more digits
% to represent a number. This method can be also be used
% for letters. After chapter 26 with |Z| we us |AA|, |AB|, |AC|, and |AD|
% for the remaining chapters.
%
% Happily this package already defines this presentation command:
%\begin{quote}
%\begin{verbatim}
%\usepackage{alphalph}
%\renewcommand*{\thechapter}{%
%  \AlphAlph{\value{chapter}}%
%}
%\end{verbatim}
%\end{quote}
% \cs{AlphAlph} generates:
%   \AlphAlph{1}, \AlphAlph{2}, \AlphAlph{3}, \dots,
%   \AlphAlph{26}, \AlphAlph{27}, \AlphAlph{28}, \dots
%
% The other presentation command is \cs{alphalph} for lowercase letters.
%
% \subsubsection{Wrap symbols around}
%
% Nine footnote symbols are quite a few.
% Too soon the symbols are consumed
% and \hologo{LaTeX} complains with the error ``Counter too large''.
% However, it could be acceptable to start again with the symbols
% from the beginning, especially if there are less than nine symbols
% on a page.
% This could be achieved by a counter reset.
% But finding the right place can be difficult or needs manual
% actions. Also a
% unique counter value can be desirable (e.g. for generating
% unique anchor/link names).
% Package \xpackage{alphalph} allows you to define
% a macro that implements a ``wrap around'', but letting
% the value of the counter untouched:
%\begin{quote}
%\begin{verbatim}
%\usepackage{alphalph}
%\makeatletter
%\newalphalph{\fnsymbolwrap}[wrap]{\@fnsymbol}{}
%\makeatother
%\renewcommand*{\thefootnote}{%
%  \fnsymbolwrap{\value{footnote}}%
%}
%\end{verbatim}
%\end{quote}
% \cs{fnsymbolwrap} generates:
% \fnsymbolwrap{1}~(1), \fnsymbolwrap{2}~(2), \fnsymbolwrap{3}~(3), \dots,
% \fnsymbolwrap{9}~(9), \fnsymbolwrap{10}~(10), \fnsymbolwrap{11}~11, \dots
%
% \subsubsection{Multiple symbols}
%
% \Hologo{LaTeX}'s standard set of footnote symbols contains doubled
% symbols at the higher positions. Could this principle
% be generalized?
% Yes, but first we need a clean footnote symbol list without
% doubled entries, example:
%\begin{quote}
%\begin{verbatim}
%\usepackage{alphalph}
%\makeatletter
%\newcommand*{\fnsymbolsingle}[1]{%
%  \ensuremath{%
%    \ifcase#1%
%    \or *%
%    \or \dagger
%    \or \ddagger
%    \or \mathsection
%    \or \mathparagraph
%    \else
%      \@ctrerr
%    \fi
%  }%
%}
%\makeatother
%\newalphalph{\fnsymbolmult}[mult]{\fnsymbolsingle}{}
%\renewcommand*{\thefootnote}{%
%  \fnsymbolmult{\value{footnote}}%
%}
%\end{verbatim}
%\end{quote}
% The own definition of \cs{fnsymbolsingle} has the
% advantage that this list can easily modified.
% Otherwise you can use \cs{@fnsymbol} directly,
% because it uses the same first five symbols.
%\begin{quote}
%\begin{verbatim}
%\usepackage{alphalph}
%\makeatletter
%\newalphalph{\fnsymbolmult}[mult]{\@fnsymbol}{5}
%\makeatother
%\renewcommand*{\thefootnote}{%
%  \fnsymbolmult{\value{footnote}}%
%}
%\end{verbatim}
%\end{quote}
% \cs{fnsymbolmult} generates:
% \fnsymbolmult{1}~(1), \fnsymbolmult{2}~(2), \fnsymbolmult{3}~(3),
% \fnsymbolmult{4}~(4), \fnsymbolmult{5}~(5), \fnsymbolmult{6}~(6),
% \dots, \fnsymbolmult{16}~16, \fnsymbolmult{17}~17, \dots
%
% \bigskip
% The same method can also be used for the chapter problem
% in the first discussed use case:
%\begin{quote}
%\begin{verbatim}
%\usepackage{alphalph}
%\makeatletter
%\newalphalph{\AlphMult}[mult]{\@Alph}{26}
%\makeatother
%\renewcommand*{\chapter}{%
%  \AlphMult{\value{chapter}}%
%}
%\end{verbatim}
%\end{quote}
% \cs{AlphMult} then generates AA, BB, CC, and DD for chapters 27--30.
%
% \subsection{Glossary}
%
% \begin{description}
% \item[Counter presentation command] is a macro that
%   expects a \hologo{LaTeX} counter name as argument.
%   Numbers cannot be used. Examples:
%   \cs{arabic}, \cs{alph}, \cs{fnsymbol}.
% \item[Number presentation command] is a macro that
%   expects a number as argument. A number is anything
%   that \hologo{TeX} accepts as number including \cs{value}.
%   Examples: \cs{alphalph}, \cs{AlphAlph}, \cs{alphalph@alph}
%
%   However, \cs{alph} or \cs{fnsymbol} are not number
%   presentation commands because they expect a counter
%   name as argument. Happily \hologo{LaTeX} counter presentation
%   commands internally uses number presentation commands
%   with the same name, but prefixed by `@'. Thus
%   \cs{@alph}, \cs{@fnsymbol} are number presentation commands.
% \item[Symbols provider] is a command that can be used
%   to get a list of symbols. For example, \cs{@Alph}
%   provides the 26 uppercase letters from `A' to `Z'.
%   Basically a symbol provider is a number presentation
%   command, usually with a limited range.
% \item[Number of symbols] is the number of the
%   last symbol slot of a symbol provider. Thus \cs{@Alph}
%   generates 26 symbols, \cs{@fnsymbol} provides 9 symbols.
% \end{description}
%
% \subsection{Package usage}
%
% The package \Package{alphalph} can be used with both \hologo{plainTeX}
% and \hologo{LaTeX}:
% \begin{description}
% \item[\hologo{plainTeX}:] |\input alphalph.sty|
% \item[\hologo{LaTeX2e}:]   |\usepackage{alphalph}|\\
%      There aren't any options.
% \end{description}
%
% \subsection{User commands}
%
% \begin{declcs}{AlphAlph} \M{number}\\
%   \cs{alphalph} \M{number}
% \end{declcs}
% Both macros are number presentation commands that
% expects a number as argument. \hologo{LaTeX} counters
% are used with \cs{value}.
%
% The macros represents a number by letters.
% First single letters |A..Z| are used, then
% two letters |AA..ZZ|, three letters |AAA...ZZZ|, \dots
% follow.
%
% Macro \cs{AlphAlph} uses uppercase letters,
% \cs{alphalph} generates the lowercase variant.
% \begin{quote}
%   \catcode`\|=12
%   \def\entry#1{
%     \texttt{#1}
%     & \texttt{\AlphAlph{#1}}
%     & \texttt{\alphalph{#1}}
%   }
%   \begin{tabular}{@{}l|l|l@{}}
%     \meta{number} & \verb|\AlphAlph{|\meta{number}\verb|}|
%                   & \verb|\alphalph{|\meta{number}\verb|}|\\
%     \hline
%     \entry{1}\\
%     \entry{2}\\
%     \entry{26}\\
%     \entry{27}\\
%     \entry{30}\\
%     \entry{2000}\\
%     \entry{3752127}\\
%     \entry{10786572}\\
%     \entry{2147483647}\\
%   \end{tabular}
% \end{quote}
%
% \begin{declcs}{newalphalph}
%   \M{cmd}
%   |[|\meta{method}|]|
%   \M{symbols provider}
%   \M{number of symbols}
% \end{declcs}
% Macro \cs{newalphalph} defines \meta{cmd} as new
% number presentation command. Like \cs{newcommand} an error
% is thrown, if macro \meta{cmd} already exists.
%
% The \meta{method} is one of |alph|, |wrap|, or |mult|.
% The default is |alph|.
%
% As symbol provider a number presentation command
% can be used, e.g. \cs{@fnsymbol}, \cs{@Alph}, or
% \cs{alphalph@alph}.
%
% The last argument is the number of symbols.
% If the argument is empty, then \cs{newalphalph} tries
% to find this number itself. \hologo{LaTeX}'s number presentation
% commands throw an error message, if the number is too large.
% This error message is put in a macro \cs{@ctrerr}.
% Thus \cs{newalphalph} calls the symbol provider and tests
% a number by typesetting it in a temporary box. The error
% macro \cs{@ctrerr} is catched, it proofs that the number
% is not supported. Also if the width of the result is zero
% the number is considered as unavailable.
%
% The empty argument is useful for potentially variable lists.
% However if the end cannot be detected, then the number of
% symbols must be given. This is also a lot faster. Therefore don't
% let the argument empty without reason.
%
% \subsection{Programmer commands}
%
% \begin{declcs}{alphalph@Alph} \M{number}\\
%   \cs{alphalph@alph} \M{number}
% \end{declcs}
% They are basically the same as \cs{@Alph} and \cs{@alph}.
% Some languages of package \xpackage{babel} redefine
% \hologo{LaTeX}'s macros to include some font setup that
% breaks expandibility. Therefore \cs{AlphAlph} and \cs{alphalph}
% are based on \cs{alphalph@Alph} and \cs{alphalph@alph} to get
% the letters. The behaviour of these symbol providers for numbers
% outside the range |1..26| is undefined.
%
% \subsection{Design principles}
%
% \subsubsection{Number presentation commands}
%
% All number presentation commands that this package defines
% (including \cs{alphalph} and \cs{AlphAlph}) have the following
% properties:
% \begin{itemize}
% \item They are fully expandable. This means that they can safely
%       \begin{itemize}
%       \item be written to a file,
%       \item used in moving arguments (\hologo{LaTeX}: they are \emph{robust}),
%       \item used in a \cmd{\csname}-\cmd{\endcsname} pair.
%       \end{itemize}
% \item If the argument is zero or negative, the commands expand
%       to nothing like \cmd{\romannumeral}.
% \item The argument is a \hologo{TeX} number. Anything that would be
%       accepted by \cmd{\number} is a valid argument:
%       \begin{itemize}
%       \item explicite constants,
%       \item macros that expand to a number,
%       \item count registers, \hologo{LaTeX} counter can used
%             via \cmd{\value}, e.\,g.:\\
%             |\alphalph{\value{page}}|
%       \item \dots
%       \end{itemize}
% \item \hologo{eTeX}'s numeric expressions are supported, if \hologo{eTeX} is
%       available. Then \cs{numexpr} is applied to the argument.
%       Package \cs{calc}'s expressions are not supported.
%       That would violate the expandibility.
% \end{itemize}
%
% \subsubsection{General usability}
%
% \begin{description}
% \item[\hologo{TeX} format:]
%   The package does not depend on \hologo{LaTeX}, it can also be used
%   by \hologo{plainTeX}, for example.
% \item[\hologo{eTeX}:] \eTeX is supported, the macros are shorter and
%   faster. But \eTeX's extensions are not requirements.
%   Without \hologo{eTeX}, just the implementation changes. The properties
%   remain unchanged.
% \end{description}
%
% \StopEventually{
% }
%
% \section{Implementation}
% \subsection{Begin\texorpdfstring{ of package}{}}
%    \begin{macrocode}
%<*package>
%    \end{macrocode}
%    Reload check, especially if the package is not used with \LaTeX.
%    \begin{macrocode}
\begingroup\catcode61\catcode48\catcode32=10\relax%
  \catcode13=5 % ^^M
  \endlinechar=13 %
  \catcode35=6 % #
  \catcode39=12 % '
  \catcode44=12 % ,
  \catcode45=12 % -
  \catcode46=12 % .
  \catcode58=12 % :
  \catcode64=11 % @
  \catcode123=1 % {
  \catcode125=2 % }
  \expandafter\let\expandafter\x\csname ver@alphalph.sty\endcsname
  \ifx\x\relax % plain-TeX, first loading
  \else
    \def\empty{}%
    \ifx\x\empty % LaTeX, first loading,
      % variable is initialized, but \ProvidesPackage not yet seen
    \else
      \expandafter\ifx\csname PackageInfo\endcsname\relax
        \def\x#1#2{%
          \immediate\write-1{Package #1 Info: #2.}%
        }%
      \else
        \def\x#1#2{\PackageInfo{#1}{#2, stopped}}%
      \fi
      \x{alphalph}{The package is already loaded}%
      \aftergroup\endinput
    \fi
  \fi
\endgroup%
%    \end{macrocode}
%    Package identification:
%    \begin{macrocode}
\begingroup\catcode61\catcode48\catcode32=10\relax%
  \catcode13=5 % ^^M
  \endlinechar=13 %
  \catcode35=6 % #
  \catcode39=12 % '
  \catcode40=12 % (
  \catcode41=12 % )
  \catcode44=12 % ,
  \catcode45=12 % -
  \catcode46=12 % .
  \catcode47=12 % /
  \catcode58=12 % :
  \catcode64=11 % @
  \catcode91=12 % [
  \catcode93=12 % ]
  \catcode123=1 % {
  \catcode125=2 % }
  \expandafter\ifx\csname ProvidesPackage\endcsname\relax
    \def\x#1#2#3[#4]{\endgroup
      \immediate\write-1{Package: #3 #4}%
      \xdef#1{#4}%
    }%
  \else
    \def\x#1#2[#3]{\endgroup
      #2[{#3}]%
      \ifx#1\@undefined
        \xdef#1{#3}%
      \fi
      \ifx#1\relax
        \xdef#1{#3}%
      \fi
    }%
  \fi
\expandafter\x\csname ver@alphalph.sty\endcsname
\ProvidesPackage{alphalph}%
  [2016/05/16 v2.5 Convert numbers to letters (HO)]%
%    \end{macrocode}
%
% \subsection{Catcodes}
%
%    \begin{macrocode}
\begingroup\catcode61\catcode48\catcode32=10\relax%
  \catcode13=5 % ^^M
  \endlinechar=13 %
  \catcode123=1 % {
  \catcode125=2 % }
  \catcode64=11 % @
  \def\x{\endgroup
    \expandafter\edef\csname AlPh@AtEnd\endcsname{%
      \endlinechar=\the\endlinechar\relax
      \catcode13=\the\catcode13\relax
      \catcode32=\the\catcode32\relax
      \catcode35=\the\catcode35\relax
      \catcode61=\the\catcode61\relax
      \catcode64=\the\catcode64\relax
      \catcode123=\the\catcode123\relax
      \catcode125=\the\catcode125\relax
    }%
  }%
\x\catcode61\catcode48\catcode32=10\relax%
\catcode13=5 % ^^M
\endlinechar=13 %
\catcode35=6 % #
\catcode64=11 % @
\catcode123=1 % {
\catcode125=2 % }
\def\TMP@EnsureCode#1#2{%
  \edef\AlPh@AtEnd{%
    \AlPh@AtEnd
    \catcode#1=\the\catcode#1\relax
  }%
  \catcode#1=#2\relax
}
\TMP@EnsureCode{33}{12}% !
\TMP@EnsureCode{39}{12}% '
\TMP@EnsureCode{40}{12}% (
\TMP@EnsureCode{41}{12}% )
\TMP@EnsureCode{43}{12}% +
\TMP@EnsureCode{44}{12}% ,
\TMP@EnsureCode{46}{12}% .
\TMP@EnsureCode{47}{12}% /
\TMP@EnsureCode{59}{12}% ;
\TMP@EnsureCode{60}{12}% <
\TMP@EnsureCode{62}{12}% >
\TMP@EnsureCode{91}{12}% [
\TMP@EnsureCode{93}{12}% ]
\TMP@EnsureCode{96}{12}% `
\TMP@EnsureCode{124}{12}% |
\edef\AlPh@AtEnd{\AlPh@AtEnd\noexpand\endinput}
%    \end{macrocode}
%
% \subsection{Package loading}
%
%    \begin{macrocode}
\begingroup\expandafter\expandafter\expandafter\endgroup
\expandafter\ifx\csname RequirePackage\endcsname\relax
  \input infwarerr.sty\relax
  \input intcalc.sty\relax
\else
  \RequirePackage{infwarerr}[2007/09/09]%
  \RequirePackage{intcalc}[2007/09/09]%
\fi
%    \end{macrocode}
%
% \subsection{\hologo{eTeX} detection}
%
%    \begin{macrocode}
\begingroup\expandafter\expandafter\expandafter\endgroup
\expandafter\ifx\csname numexpr\endcsname\relax
  \catcode124=9 % '!': ignore
  \catcode43=14 % '+': comment
\else
  \catcode124=14 % '!': comment
  \catcode43=9   % '+': ignore
\fi
%    \end{macrocode}
%
% \subsection{Help macros}
%
%    \begin{macro}{\AlPh@Error}
%    \begin{macrocode}
\def\AlPh@Error#1{%
  \begingroup
    \escapechar=92 % backslash
    \@PackageError{alphalph}{#1}\@ehc
  \endgroup
}
%    \end{macrocode}
%    \end{macro}
%
%    \begin{macro}{\AlPh@IfDefinable}
%    \begin{macrocode}
\begingroup\expandafter\expandafter\expandafter\endgroup
\expandafter\ifx\csname @ifdefinable\endcsname\relax
  \def\AlPh@IfDefinable#1#2{%
    \ifcase\ifx#1\@undefined\else\ifx#1\relax\else1\fi\fi0 %
      #2%
    \else
      \AlPh@Error{%
        Command \string#1 already defined%
      }%
    \fi
  }%
%    \end{macrocode}
%    \end{macro}
%    \begin{macrocode}
\else
%    \end{macrocode}
%    \begin{macro}{\AlPh@IfDefinable}
%    \begin{macrocode}
  \let\AlPh@IfDefinable\@ifdefinable
%    \end{macrocode}
%    \end{macro}
%    \begin{macrocode}
\fi
%    \end{macrocode}
%
%    \begin{macro}{\@ReturnAfterElseFi}
%    \begin{macro}{\@ReturnAfterFi}
%    The following commands moves the `then' and `else' part respectively
%    behind the |\if|-construct. This prevents a too deep |\if|-nesting
%    and so a \hologo{TeX} capacity error because of a limited input stack size.
%    I use this trick in several packages, so I don't prefix these internal
%    commands in order not to have the same macros with different names.
%    (It saves memory.)
%    \begin{macrocode}
\long\def\@ReturnAfterElseFi#1\else#2\fi{\fi#1}
\long\def\@ReturnAfterFi#1\fi{\fi#1}
%    \end{macrocode}
%    \end{macro}
%    \end{macro}
%
%    \begin{macro}{\@gobblefour}
%    \Hologo{LaTeX} defines commands for eating arguments.
%    Define \cs{@gobblefour} if it is not defined (\hologo{plainTeX}).
%    \begin{macrocode}
\expandafter\ifx\csname @gobblefour\endcsname\relax
  \long\def\@gobblefour#1#2#3#4{}%
\fi
%    \end{macrocode}
%    \end{macro}
%
%    \begin{macro}{AlPh@IfOptArg}
%    \begin{macrocode}
\begingroup\expandafter\expandafter\expandafter\endgroup
\expandafter\ifx\csname kernel@ifnextchar\endcsname\relax
  \begingroup\expandafter\expandafter\expandafter\endgroup
  \expandafter\ifx\csname @ifnextchar\endcsname\relax
    \def\AlPh@IfOptArg#1#2{%
      \def\AlPh@TempA{#1}%
      \def\AlPh@TempB{#2}%
      \futurelet\AlPh@Token\AlPh@IfOptArgNext
    }%
    \let\AlPh@BracketLeft=[%]
    \def\AlPh@IfOptArgNext{%
      \ifx\AlPh@Token\AlPh@BracketLeft
        \expandafter\AlPh@TempA
      \else
        \expandafter\AlPh@TempB
      \fi
    }%
  \else
    \def\AlPh@IfOptArg{\@ifnextchar[}%]
  \fi
\else
  \def\AlPh@IfOptArg{\kernel@ifnextchar[}%]
\fi
%    \end{macrocode}
%    \end{macro}
%
% \subsection{Symbol provider}
%
% \subsubsection{Alphabet}
%
%    The output of \cs{alphalph} and \cs{AlphAlph} should be
%    usable as part of command names (see \cs{@namedef}, \cs{csname}, \dots).
%    Unhappily some languages of package \xpackage{babel}
%    redefine \hologo{LaTeX}'s \cs{@alph} and \cs{@Alph} in a manner that
%    they cannot be used in expandable context any more.
%    Therefore package \xpackage{alphalph} provides its own
%    commands.
%
%    \begin{macro}{\alphalph@Alph}
%    \begin{macro}{\alphalph@alph}
%    The two commands \cmd{\AlPh@Alph} and \cmd{\AlPh@alph} convert a
%    number into a letter (uppercase and lowercase respectivly).
%    The character |@| is used as an error symbol,
%    if the number isn't in the range of 1 until 26.
%    Here we need no space after the number |#1|, because the error
%    symbol |@| for the zero case stops scanning the number.
%    This error symbol should not appear anywhere (except for bugs).
%    \begin{macrocode}
\def\alphalph@Alph#1{%
  \ifcase#1%
    @%
  \or A\or B\or C\or D\or E\or F\or G\or H\or I\or J\or K\or L\or M%
  \or N\or O\or P\or Q\or R\or S\or T\or U\or V\or W\or X\or Y\or Z%
  \else
    \AlPh@ctrerr
    @%
  \fi
}
\def\alphalph@alph#1{%
  \ifcase#1%
    @%
  \or a\or b\or c\or d\or e\or f\or g\or h\or i\or j\or k\or l\or m%
  \or n\or o\or p\or q\or r\or s\or t\or u\or v\or w\or x\or y\or z%
  \else
    \AlPh@ctrerr
    @%
  \fi
}
%    \end{macrocode}
%    \end{macro}
%    \end{macro}
%    \begin{macro}{\AlPh@ctrerr}
%    Macro \cs{AlPh@ctrerr} is used as hook for the algorithm
%    to get the available number of symbols.
%    \begin{macrocode}
\def\AlPh@ctrerr{}
%    \end{macrocode}
%    \end{macro}
%
% \subsection{Finding number of symbols}
%
%    \begin{macro}{\AlPh@GetNumberOfSymbols}
%    \noindent
%    |#1|: symbols provider
%    \begin{macrocode}
\def\AlPh@GetNumberOfSymbols#1{%
  \AlPh@TestNumber1!{#1}%
  \ifAlPh@Unavailable
    \def\AlPh@Number{0}%
    \AlPh@Error{No symbols found}%
  \else
    \def\AlPh@Number{1}%
    \AlPh@ExpSearch2!{#1}%
  \fi
}
%    \end{macrocode}
%    \end{macro}
%    \begin{macro}{\ifAlPh@Unavailable}
%    \begin{macrocode}
\let\ifAlPh@Unavailable\iffalse
\def\AlPh@Unavailabletrue{%
  \global\let\ifAlPh@Unavailable\iftrue
}
\def\AlPh@Unavailablefalse{%
  \global\let\ifAlPh@Unavailable\iffalse
}
%    \end{macrocode}
%    \end{macro}
%    \begin{macro}{\AlPh@TestNumber}
%    |#1|: number to be tested\\
%    |#2|: symbols provider
%    \begin{macrocode}
\def\AlPh@TestNumber#1!#2{%
  \AlPh@Unavailablefalse
  \begingroup
    \setbox0=\hbox{%
      \begingroup % color
        \let\@ctrerr\AlPh@Unavailabletrue
        \let\AlPh@ctrerr\AlPh@Unavailabletrue
        #2{#1}%
      \endgroup
    }%
    \ifdim\wd0=0pt %
      \AlPh@Unavailabletrue
    \fi
  \endgroup
}
%    \end{macrocode}
%    \end{macro}
%    \begin{macro}{\AlPh@ExpSearch}
%    |#1|: number to be tested\\
%    |#2|: symbols provider
%    \begin{macrocode}
\def\AlPh@ExpSearch#1!#2{%
  \let\AlPh@Next\relax
  \AlPh@TestNumber#1!{#2}%
  \ifAlPh@Unavailable
    \expandafter\AlPh@BinSearch\AlPh@Number!#1!{#2}%
  \else
    \def\AlPh@Number{#1}%
    \ifnum#1>1073741823 %
      \AlPh@TestNumber2147483647!{#2}%
      \ifAlPh@Unavailable
        \AlPh@BinSearch#1!2147483647!{#2}%
      \else
        \def\AlPh@Number{0}%
        \AlPh@Error{%
          Maximal symbol number not found%
        }%
      \fi
    \else
      \def\AlPh@Next{%
        \expandafter\AlPh@ExpSearch\number\intcalcShl{#1}!{#2}%
      }%
    \fi
  \fi
  \AlPh@Next
}
%    \end{macrocode}
%    \end{macro}
%    \begin{macro}{\AlPh@BinSearch}
%    |#1|: available number\\
%    |#2|: unavailable number, |#2| $>$ |#1|\\
%    |#3|: symbols provider
%    \begin{macrocode}
\def\AlPh@BinSearch#1!#2!#3{%
  \expandafter\AlPh@ProcessBinSearch
  \number\intcalcShr{\intcalcAdd{#1}{#2}}!%
  #1!#2!{#3}%
}
%    \end{macrocode}
%    \end{macro}
%    \begin{macro}{\AlPh@ProcessBinSearch}
%    |#1|: number to be tested, |#2| $\leq$ |#1| $\leq$ |#3|\\
%    |#2|: available number\\
%    |#3|: unavailable number\\
%    |#4|: symbols provider
%    \begin{macrocode}
\def\AlPh@ProcessBinSearch#1!#2!#3!#4{%
  \let\AlPh@Next\relax
  \ifnum#1>#2 %
    \ifnum#1<#3 %
      \AlPh@TestNumber#1!{#4}%
      \ifAlPh@Unavailable
        \def\AlPh@Next{%
          \AlPh@BinSearch#2!#1!{#4}%
        }%
      \else
        \def\AlPh@Next{%
          \AlPh@BinSearch#1!#3!{#4}%
        }%
      \fi
    \else
      \def\AlPh@Number{#2}%
    \fi
  \else
    \def\AlPh@Number{#2}%
  \fi
  \AlPh@Next
}
%    \end{macrocode}
%    \end{macro}
%
% \subsection{Methods}
%
%    The names of method macros start with \cs{AlPh@Method}.
%    These macros do the main job in converting a number to
%    its representation.
%    A method command is called with three arguments.
%    The first argument is the number of symbols. The
%    second argument is the basic macro for converting
%    a number with limited number range. The last parameter
%    is the number that needs converting.
%
% \subsubsection{Common methods}
%
%    \begin{macro}{\AlPh@CheckPositive}
%    \noindent
%    |#1|: number to be checked
%    |#2|: continuation macro\\
%    |#3|: number of symbols (hidden here)\\
%    |#4|: symbol provider (hidden here)
%    \begin{macrocode}
\def\AlPh@CheckPositive#1!#2{%
  \ifnum#1<1 %
    \expandafter\@gobblefour
  \fi
  #2{#1}%
}
%    \end{macrocode}
%    \end{macro}
%
% \subsubsection{Method `alph'}
%
%    \begin{macro}{\AlPh@Method@alph}
%    \noindent
%    |#1|: number of symbols\\
%    |#2|: symbols provider\\
%    |#3|: number to be converted
%    \begin{macrocode}
\def\AlPh@Method@alph#1#2#3{%
  \expandafter\AlPh@CheckPositive
|   \number#3!%
+   \the\numexpr#3!%
    \AlPh@ProcessAlph
    {#1}{#2}%
}
%    \end{macrocode}
%    \end{macro}
%    \begin{macro}{\AlPh@ProcessAlph}
%    |#1|: current number\\
%    |#2|: number of symbols\\
%    |#3|: symbols provider
%    \begin{macrocode}
\def\AlPh@ProcessAlph#1#2#3{%
  \ifnum#1>#2 %
    \@ReturnAfterElseFi{%
      \expandafter\AlPh@StepAlph\number
        \intcalcInc{%
          \intcalcMod{\intcalcDec{#1}}{#2}%
        }%
      \expandafter!\number
        \intcalcDiv{\intcalcDec{#1}}{#2}%
      !{#2}{#3}%
    }%
  \else
    \@ReturnAfterFi{%
      #3{#1}%
    }%
  \fi
}
%    \end{macrocode}
%    \end{macro}
%    \begin{macro}{\AlPh@StepAlph}
%    |#1|: current last digit\\
%    |#2|: new current number\\
%    |#3|: number of symbols\\
%    |#4|: symbols provider
%    \begin{macrocode}
\def\AlPh@StepAlph#1!#2!#3#4{%
  \AlPh@ProcessAlph{#2}{#3}{#4}%
  #4{#1}%
}
%    \end{macrocode}
%    \end{macro}
%
% \subsubsection{Method `wrap'}
%
%    \begin{macro}{\AlPh@Method@wrap}
%    \noindent
%    |#1|: number of symbols\\
%    |#2|: symbols provider\\
%    |#3|: number to be converted
%    \begin{macrocode}
\def\AlPh@Method@wrap#1#2#3{%
  \expandafter\AlPh@CheckPositive
|   \number#3!%
+   \the\numexpr#3!%
    \AlPh@ProcessWrap
    {#1}{#2}%
}
%    \end{macrocode}
%    \end{macro}
%
%    \begin{macro}{\AlPh@ProcessWrap}
%    |#1|: number to be converted\\
%    |#2|: number of symbols\\
%    |#3|: symbols provider
%    \begin{macrocode}
\def\AlPh@ProcessWrap#1#2#3{%
  \ifnum#1>#2 %
    \@ReturnAfterElseFi{%
      \expandafter\AlPh@StepWrap\number
        \intcalcInc{\intcalcMod{\intcalcDec{#1}}{#2}}%
      !{#3}%
    }%
  \else
    \@ReturnAfterFi{%
      #3{#1}%
    }%
  \fi
}
%    \end{macrocode}
%    \end{macro}
%    \begin{macro}{\AlPh@StepWrap}
%    |#1|: final number\\
%    |#2|: symbols provider
%    \begin{macrocode}
\def\AlPh@StepWrap#1!#2{%
  #2{#1}%
}
%    \end{macrocode}
%    \end{macro}
%
% \subsubsection{Method `mult'}
%
%    After the number of symbols is exhausted,
%    repetitions of the symbol are used.
%    \begin{gather*}
%      x \Def \text{number to be converted}\\
%      n \Def \text{number of symbols}\\
%      r \Def \text{repetition length}\\
%      s \Def \text{symbol slot}\\
%      r = ((x - 1) \div n) + 1\\
%      s = ((x - 1) \mod n) + 1
%    \end{gather*}
%    \begin{macro}{\AlPh@Method@mult}
%    \noindent
%    |#1|: number of symbols\\
%    |#2|: symbols provider\\
%    |#3|: number to be converted
%    \begin{macrocode}
\def\AlPh@Method@mult#1#2#3{%
  \expandafter\AlPh@CheckPositive
|   \number#3!%
+   \the\numexpr#3!%
    \AlPh@ProcessMult
    {#1}{#2}%
}
%    \end{macrocode}
%    \end{macro}
%
%    \begin{macro}{\AlPh@ProcessMult}
%    |#1|: number to be converted\\
%    |#2|: number of symbols\\
%    |#3|: symbols provider
%    \begin{macrocode}
\def\AlPh@ProcessMult#1#2#3{%
  \ifnum#1>#2 %
    \@ReturnAfterElseFi{%
      \expandafter\AlPh@StepMult\romannumeral
        \intcalcInc{\intcalcDiv{\intcalcDec{#1}}{#2}}%
        000%
      \expandafter!\number
        \intcalcInc{\intcalcMod{\intcalcDec{#1}}{#2}}%
      !{#3}%
    }%
  \else
    \@ReturnAfterFi{%
      #3{#1}%
    }%
  \fi
}
%    \end{macrocode}
%    \end{macro}
%    \begin{macro}{\AlPh@StepMult}
%    |#1#2|: repetitions coded as list of character `m'\\
%    |#3|: symbol slot\\
%    |#4|: symbols provider
%    \begin{macrocode}
\def\AlPh@StepMult#1#2!#3!#4{%
  \ifx\\#2\\%
  \else
    \@ReturnAfterFi{%
      \AlPh@StepMult#2!#3!{#4}%
    }%
  \fi
  #4{#3}%
}
%    \end{macrocode}
%    \end{macro}
%
% \subsection{User interface}
%
%    \begin{macro}{\newalphalph}
%    Macro \cs{newalphalph} had three arguments in versions below 2.0.
%    For the new method argument we use an optional argument an first
%    position.\\
%    |#1|: cmd\\
%    |[#2]|: method name: |alph| (default), |wrap|, |mult|\\ % hash-ok
%    |#3|: symbols provider\\
%    |#4|: number of symbols
%    \begin{macrocode}
\AlPh@IfDefinable\newalphalph{%
  \def\newalphalph#1{%
    \AlPh@IfOptArg{%
      \AlPh@newalphalph{#1}%
    }{%
      \AlPh@newalphalph{#1}[alph]%
    }%
  }%
}
%    \end{macrocode}
%    \end{macro}
%    \begin{macro}{\AlPh@newalphalph}
%    |#1|: cmd
%    |#2|: method name\\
%    |#3|: symbols provider\\
%    |#4|: number of symbols
%    \begin{macrocode}
\def\AlPh@newalphalph#1[#2]#3#4{%
  \begingroup\expandafter\expandafter\expandafter\endgroup
  \expandafter\ifx\csname AlPh@Method@#2\endcsname\relax
    \AlPh@Error{%
      Unknown method %
|     `#2'%
+     `\detokenize{#2}'%
    }%
  \else
    \ifx\\#4\\%
      \AlPh@GetNumberOfSymbols{#3}%
      \ifcase\AlPh@Number
      \else
        \begingroup
          \escapechar=92 % backslash
          \@PackageInfo{alphalph}{%
            Number of symbols for \string#1 is \AlPh@Number
          }%
        \endgroup
        \expandafter\AlPh@NewAlphAlph
        \csname AlPh@Method@#2\expandafter\endcsname
        \AlPh@Number!{#1}{#3}%
      \fi
    \else
      \expandafter\AlPh@NewAlphAlph
      \csname AlPh@Method@#2\expandafter\endcsname
|     \number#4!%
+     \the\numexpr#4!%
      {#1}{#3}%
    \fi
  \fi
}%
%    \end{macrocode}
%    \end{macro}
%    \begin{macro}{\AlPh@NewAlphAlph}
%    |#1|: method macro\\
%    |#2|: number of symbols\\
%    |#3|: cmd\\
%    |#4|: symbols provider
%    \begin{macrocode}
\def\AlPh@NewAlphAlph#1#2!#3#4{%
  \AlPh@IfDefinable#3{%
    \ifnum#2>0 %
      \def#3{#1{#2}{#4}}%
    \else
      \AlPh@Error{%
          Definition of \string#3 failed,\MessageBreak
          because number of symbols (#2) is not positive%
        }%
    \fi
  }%
}
%    \end{macrocode}
%    \end{macro}
%
%    \begin{macro}{\AlphAlph}
%    \begin{macrocode}
\newalphalph\AlphAlph\alphalph@Alph{26}
%    \end{macrocode}
%    \end{macro}
%    \begin{macro}{\alphalph}
%    \begin{macrocode}
\newalphalph\alphalph\alphalph@alph{26}
%    \end{macrocode}
%    \end{macro}
%
%    \begin{macrocode}
\AlPh@AtEnd%
%</package>
%    \end{macrocode}
%
% \section{Test}
%
% \subsection{Catcode checks for loading}
%
%    \begin{macrocode}
%<*test1>
%    \end{macrocode}
%    \begin{macrocode}
\catcode`\{=1 %
\catcode`\}=2 %
\catcode`\#=6 %
\catcode`\@=11 %
\expandafter\ifx\csname count@\endcsname\relax
  \countdef\count@=255 %
\fi
\expandafter\ifx\csname @gobble\endcsname\relax
  \long\def\@gobble#1{}%
\fi
\expandafter\ifx\csname @firstofone\endcsname\relax
  \long\def\@firstofone#1{#1}%
\fi
\expandafter\ifx\csname loop\endcsname\relax
  \expandafter\@firstofone
\else
  \expandafter\@gobble
\fi
{%
  \def\loop#1\repeat{%
    \def\body{#1}%
    \iterate
  }%
  \def\iterate{%
    \body
      \let\next\iterate
    \else
      \let\next\relax
    \fi
    \next
  }%
  \let\repeat=\fi
}%
\def\RestoreCatcodes{}
\count@=0 %
\loop
  \edef\RestoreCatcodes{%
    \RestoreCatcodes
    \catcode\the\count@=\the\catcode\count@\relax
  }%
\ifnum\count@<255 %
  \advance\count@ 1 %
\repeat

\def\RangeCatcodeInvalid#1#2{%
  \count@=#1\relax
  \loop
    \catcode\count@=15 %
  \ifnum\count@<#2\relax
    \advance\count@ 1 %
  \repeat
}
\def\RangeCatcodeCheck#1#2#3{%
  \count@=#1\relax
  \loop
    \ifnum#3=\catcode\count@
    \else
      \errmessage{%
        Character \the\count@\space
        with wrong catcode \the\catcode\count@\space
        instead of \number#3%
      }%
    \fi
  \ifnum\count@<#2\relax
    \advance\count@ 1 %
  \repeat
}
\def\space{ }
\expandafter\ifx\csname LoadCommand\endcsname\relax
  \def\LoadCommand{\input alphalph.sty\relax}%
\fi
\def\Test{%
  \RangeCatcodeInvalid{0}{47}%
  \RangeCatcodeInvalid{58}{64}%
  \RangeCatcodeInvalid{91}{96}%
  \RangeCatcodeInvalid{123}{255}%
  \catcode`\@=12 %
  \catcode`\\=0 %
  \catcode`\%=14 %
  \LoadCommand
  \RangeCatcodeCheck{0}{36}{15}%
  \RangeCatcodeCheck{37}{37}{14}%
  \RangeCatcodeCheck{38}{47}{15}%
  \RangeCatcodeCheck{48}{57}{12}%
  \RangeCatcodeCheck{58}{63}{15}%
  \RangeCatcodeCheck{64}{64}{12}%
  \RangeCatcodeCheck{65}{90}{11}%
  \RangeCatcodeCheck{91}{91}{15}%
  \RangeCatcodeCheck{92}{92}{0}%
  \RangeCatcodeCheck{93}{96}{15}%
  \RangeCatcodeCheck{97}{122}{11}%
  \RangeCatcodeCheck{123}{255}{15}%
  \RestoreCatcodes
}
\Test
\csname @@end\endcsname
\end
%    \end{macrocode}
%    \begin{macrocode}
%</test1>
%    \end{macrocode}
%
% \section{Macro tests}
%
%    \begin{macrocode}
%<*test2>
\NeedsTeXFormat{LaTeX2e}
\nofiles
\documentclass{article}
%<*noetex>
\makeatletter
\let\saved@numexpr\numexpr
\newcommand*{\DisableNumexpr}{%
  \let\numexpr\@undefined
}
\newcommand*{\RestoreNumexpr}{%
  \let\numexpr\saved@numexpr
}
\DisableNumexpr
%</noetex>
\usepackage{alphalph}[2016/05/16]
%<noetex>\RestoreNumexpr
\usepackage{qstest}
\IncludeTests{*}
\LogTests{log}{*}{*}

\newcommand*{\TestCmd}[3]{%
  \setbox0=\hbox{%
%<noetex>    \DisableNumexpr
    \edef\TestString{#1{#2}}%
    \expandafter\Expect\expandafter{\TestString}{#3}%
    \edef\TestString{#1{#2} }%
    \expandafter\Expect\expandafter{\TestString}{#3 }%
  }%
  \Expect*{\the\wd0}{0.0pt}%
}

\makeatletter
\newalphalph\LaTeXAlphAlph\@Alph{26}
\newalphalph\LaTeXalphalph\@alph{26}
\newalphalph\AlphWrap[wrap]\alphalph@Alph{26}
\newalphalph\alphwrap[wrap]\alphalph@alph{26}
\newalphalph\LaTeXAlphWrap[wrap]\@Alph{26}
\newalphalph\LaTeXalphwrap[wrap]\@alph{26}
\def\LastSymbol#1{%
  \ifx\\#1\\%
  \else
    \@LastSymbol#1\@nil
  \fi
}
\def\@LastSymbol#1#2\@nil{%
  \ifx\\#2\\%
    #1%
  \else
    \@LastSymbol#2\@nil
  \fi
}
\makeatother
\newcommand*{\TestAlph}[2]{%
  \uppercase{\TestCallCmd\AlphAlph{#2}}{#1}%
  \lowercase{\TestCallCmd\alphalph{#2}}{#1}%
  \uppercase{\TestCallCmd\LaTeXAlphAlph{#2}}{#1}%
  \lowercase{\TestCallCmd\LaTeXalphalph{#2}}{#1}%
  \edef\WrapString{\LastSymbol{#2}}%
  \expandafter\TestAlphWrap\expandafter{\WrapString}{#1}%
}
\newcommand*{\TestAlphWrap}[2]{%
  \uppercase{\TestCallCmd\AlphWrap{#1}}{#2}%
  \lowercase{\TestCallCmd\alphwrap{#1}}{#2}%
  \uppercase{\TestCallCmd\LaTeXAlphWrap{#1}}{#2}%
  \lowercase{\TestCallCmd\LaTeXalphwrap{#1}}{#2}%
}
\newcommand*{\TestCallCmd}[3]{%
  \TestCmd#1{#3}{#2}%
}
\begin{qstest}{AlphSymbols}{alphalph, AlphAlph, symbols}
  \TestAlph{1}{a}%
  \TestAlph{2}{b}%
  \TestAlph{3}{c}%
  \TestAlph{4}{d}%
  \TestAlph{5}{e}%
  \TestAlph{6}{f}%
  \TestAlph{7}{g}%
  \TestAlph{8}{h}%
  \TestAlph{9}{i}%
  \TestAlph{10}{j}%
  \TestAlph{11}{k}%
  \TestAlph{12}{l}%
  \TestAlph{13}{m}%
  \TestAlph{14}{n}%
  \TestAlph{15}{o}%
  \TestAlph{16}{p}%
  \TestAlph{17}{q}%
  \TestAlph{18}{r}%
  \TestAlph{19}{s}%
  \TestAlph{20}{t}%
  \TestAlph{21}{u}%
  \TestAlph{22}{v}%
  \TestAlph{23}{w}%
  \TestAlph{24}{x}%
  \TestAlph{25}{y}%
  \TestAlph{26}{z}%
\end{qstest}
\begin{qstest}{AlphRange}{alphalph, range}
  \TestAlph{0}{}%
  \TestAlph{-1}{}%
  \TestAlph{-2147483647}{}%
  \TestAlph{27}{aa}%
  \TestAlph{28}{ab}%
  \TestAlph{52}{az}%
  \TestAlph{53}{ba}%
  \TestAlph{78}{bz}%
  \TestAlph{79}{ca}%
  \TestAlph{702}{zz}%
  \TestAlph{703}{aaa}%
  \TestAlph{2147483647}{fxshrxw}%
\end{qstest}

\makeatletter
\newcommand*{\myvocals}[1]{%
  \ifcase#1X\or A\or E\or I\or O\or U\else Y\fi
}
\makeatother
\newalphalph\vocalsvocals\myvocals{5}
\newcommand*{\TestVocals}{%
  \TestCmd\vocalsvocals
}
\begin{qstest}{vocals}{vocals}
  \TestVocals{0}{}%
  \TestVocals{1}{A}%
  \TestVocals{2}{E}%
  \TestVocals{3}{I}%
  \TestVocals{4}{O}%
  \TestVocals{5}{U}%
  \TestVocals{6}{AA}%
  \TestVocals{7}{AE}%
  \TestVocals{8}{AI}%
  \TestVocals{9}{AO}%
  \TestVocals{10}{AU}%
  \TestVocals{11}{EA}%
  \TestVocals{24}{OO}%
  \TestVocals{25}{OU}%
  \TestVocals{26}{UA}%
  \TestVocals{29}{UO}%
  \TestVocals{30}{UU}%
  \TestVocals{31}{AAA}%
  \TestVocals{155}{UUU}%
  \TestVocals{156}{AAAA}%
  \TestVocals{2147483647}{AIIOOEEIOIIUOE}%
\end{qstest}

\makeatletter
\newalphalph\AlphMult[mult]{\alphalph@Alph}{26}
\newalphalph\alphmult[mult]{\alphalph@alph}{26}
\newalphalph\LaTeXAlphMult[mult]{\@Alph}{26}
\newalphalph\LaTeXalphmult[mult]{\@alph}{26}
\makeatother
\newcommand*{\TestMult}[2]{%
  \uppercase{\TestCallCmd\AlphMult{#2}}{#1}%
  \lowercase{\TestCallCmd\alphmult{#2}}{#1}%
  \uppercase{\TestCallCmd\LaTeXAlphMult{#2}}{#1}%
  \lowercase{\TestCallCmd\LaTeXalphmult{#2}}{#1}%
}
\begin{qstest}{mult}{mult}
  \TestMult{0}{}%
  \TestMult{-1}{}%
  \TestMult{-2147483647}{}%
  \TestMult{1}{a}%
  \TestMult{2}{b}%
  \TestMult{26}{z}%
  \TestMult{27}{aa}%
  \TestMult{28}{bb}%
  \TestMult{52}{zz}%
  \TestMult{53}{aaa}%
  \TestMult{54}{bbb}%
  \TestMult{259}{yyyyyyyyyy}%
  \TestMult{260}{zzzzzzzzzz}%
  \TestMult{261}{aaaaaaaaaaa}%
  \TestMult{262}{bbbbbbbbbbb}%
\end{qstest}

\def\myvocalsB#1{%
  \ifcase#1\or A\or E\or I\or O\or U\fi
}
\begin{qstest}{symbolnum}{symbolnum}
  \makeatletter
  \def\Test#1#2{%
    \let\TestCmd\relax
    \newalphalph\TestCmd{#1}{}%
    \Expect*{\AlPh@Number}{#2}%
  }%
  \Test\@alph{26}%
  \Test\@Alph{26}%
  \Test\@fnsymbol{9}%
  \Test\myvocalsB{5}%
  \Test\alphalph@alph{26}%
  \Test\alphalph@Alph{26}%
\end{qstest}

\begin{qstest}{list}{list}
  \makeatletter
  \def\catch#1\relax{%
    \def\FoundList{\catch#1}%
  }%
  \def\Test[#1]#2#3#4{%
    \let\testcmd\relax
    \newalphalph\testcmd[{#1}]{\catch}{#2}%
    \testcmd{#3}|\relax
    \expandafter\Expect\expandafter{\FoundList}{#4|}%
    %
    \let\SavedCatch\catch
    \def\catch{\noexpand\catch\noexpand\foo}%
    \edef\Result{#4|}%
    \@onelevel@sanitize\Result
    \let\catch\SavedCatch
    \let\testcmd\relax
    \newalphalph\testcmd[{#1}]{\catch\foo}{#2}%
    \testcmd{#3}|\relax
    \@onelevel@sanitize\FoundList
    \Expect*{\FoundList}*{\Result}%
  }%
  \Test[alph]{26}{3}{\catch{3}}%
  \Test[alph]{26}{12}{\catch{12}}%
  \Test[alph]{26}{27}{\catch{1}\catch{1}}%
  \Test[alph]{26}{78}{\catch{2}\catch{26}}%
  \Test[wrap]{26}{7}{\catch{7}}%
  \Test[wrap]{26}{14}{\catch{14}}%
  \Test[wrap]{26}{80}{\catch{2}}%
  \Test[wrap]{26}{700}{\catch{24}}%
  \Test[mult]{26}{4}{\catch{4}}%
  \Test[mult]{26}{17}{\catch{17}}%
  \Test[mult]{26}{54}{\catch{2}\catch{2}\catch{2}}%
\end{qstest}

\begin{document}
\end{document}
%</test2>
%    \end{macrocode}
%
% \section{Installation}
%
% \subsection{Download}
%
% \paragraph{Package.} This package is available on
% CTAN\footnote{\CTANpkg{alphalph}}:
% \begin{description}
% \item[\CTAN{macros/latex/contrib/oberdiek/alphalph.dtx}] The source file.
% \item[\CTAN{macros/latex/contrib/oberdiek/alphalph.pdf}] Documentation.
% \end{description}
%
%
% \paragraph{Bundle.} All the packages of the bundle `oberdiek'
% are also available in a TDS compliant ZIP archive. There
% the packages are already unpacked and the documentation files
% are generated. The files and directories obey the TDS standard.
% \begin{description}
% \item[\CTANinstall{install/macros/latex/contrib/oberdiek.tds.zip}]
% \end{description}
% \emph{TDS} refers to the standard ``A Directory Structure
% for \TeX\ Files'' (\CTAN{tds/tds.pdf}). Directories
% with \xfile{texmf} in their name are usually organized this way.
%
% \subsection{Bundle installation}
%
% \paragraph{Unpacking.} Unpack the \xfile{oberdiek.tds.zip} in the
% TDS tree (also known as \xfile{texmf} tree) of your choice.
% Example (linux):
% \begin{quote}
%   |unzip oberdiek.tds.zip -d ~/texmf|
% \end{quote}
%
% \paragraph{Script installation.}
% Check the directory \xfile{TDS:scripts/oberdiek/} for
% scripts that need further installation steps.
% Package \xpackage{attachfile2} comes with the Perl script
% \xfile{pdfatfi.pl} that should be installed in such a way
% that it can be called as \texttt{pdfatfi}.
% Example (linux):
% \begin{quote}
%   |chmod +x scripts/oberdiek/pdfatfi.pl|\\
%   |cp scripts/oberdiek/pdfatfi.pl /usr/local/bin/|
% \end{quote}
%
% \subsection{Package installation}
%
% \paragraph{Unpacking.} The \xfile{.dtx} file is a self-extracting
% \docstrip\ archive. The files are extracted by running the
% \xfile{.dtx} through \plainTeX:
% \begin{quote}
%   \verb|tex alphalph.dtx|
% \end{quote}
%
% \paragraph{TDS.} Now the different files must be moved into
% the different directories in your installation TDS tree
% (also known as \xfile{texmf} tree):
% \begin{quote}
% \def\t{^^A
% \begin{tabular}{@{}>{\ttfamily}l@{ $\rightarrow$ }>{\ttfamily}l@{}}
%   alphalph.sty & tex/generic/oberdiek/alphalph.sty\\
%   alphalph.pdf & doc/latex/oberdiek/alphalph.pdf\\
%   test/alphalph-test1.tex & doc/latex/oberdiek/test/alphalph-test1.tex\\
%   test/alphalph-test2.tex & doc/latex/oberdiek/test/alphalph-test2.tex\\
%   test/alphalph-test3.tex & doc/latex/oberdiek/test/alphalph-test3.tex\\
%   alphalph.dtx & source/latex/oberdiek/alphalph.dtx\\
% \end{tabular}^^A
% }^^A
% \sbox0{\t}^^A
% \ifdim\wd0>\linewidth
%   \begingroup
%     \advance\linewidth by\leftmargin
%     \advance\linewidth by\rightmargin
%   \edef\x{\endgroup
%     \def\noexpand\lw{\the\linewidth}^^A
%   }\x
%   \def\lwbox{^^A
%     \leavevmode
%     \hbox to \linewidth{^^A
%       \kern-\leftmargin\relax
%       \hss
%       \usebox0
%       \hss
%       \kern-\rightmargin\relax
%     }^^A
%   }^^A
%   \ifdim\wd0>\lw
%     \sbox0{\small\t}^^A
%     \ifdim\wd0>\linewidth
%       \ifdim\wd0>\lw
%         \sbox0{\footnotesize\t}^^A
%         \ifdim\wd0>\linewidth
%           \ifdim\wd0>\lw
%             \sbox0{\scriptsize\t}^^A
%             \ifdim\wd0>\linewidth
%               \ifdim\wd0>\lw
%                 \sbox0{\tiny\t}^^A
%                 \ifdim\wd0>\linewidth
%                   \lwbox
%                 \else
%                   \usebox0
%                 \fi
%               \else
%                 \lwbox
%               \fi
%             \else
%               \usebox0
%             \fi
%           \else
%             \lwbox
%           \fi
%         \else
%           \usebox0
%         \fi
%       \else
%         \lwbox
%       \fi
%     \else
%       \usebox0
%     \fi
%   \else
%     \lwbox
%   \fi
% \else
%   \usebox0
% \fi
% \end{quote}
% If you have a \xfile{docstrip.cfg} that configures and enables \docstrip's
% TDS installing feature, then some files can already be in the right
% place, see the documentation of \docstrip.
%
% \subsection{Refresh file name databases}
%
% If your \TeX~distribution
% (\teTeX, \mikTeX, \dots) relies on file name databases, you must refresh
% these. For example, \teTeX\ users run \verb|texhash| or
% \verb|mktexlsr|.
%
% \subsection{Some details for the interested}
%
% \paragraph{Attached source.}
%
% The PDF documentation on CTAN also includes the
% \xfile{.dtx} source file. It can be extracted by
% AcrobatReader 6 or higher. Another option is \textsf{pdftk},
% e.g. unpack the file into the current directory:
% \begin{quote}
%   \verb|pdftk alphalph.pdf unpack_files output .|
% \end{quote}
%
% \paragraph{Unpacking with \LaTeX.}
% The \xfile{.dtx} chooses its action depending on the format:
% \begin{description}
% \item[\plainTeX:] Run \docstrip\ and extract the files.
% \item[\LaTeX:] Generate the documentation.
% \end{description}
% If you insist on using \LaTeX\ for \docstrip\ (really,
% \docstrip\ does not need \LaTeX), then inform the autodetect routine
% about your intention:
% \begin{quote}
%   \verb|latex \let\install=y% \iffalse meta-comment
%
% File: alphalph.dtx
% Version: 2016/05/16 v2.5
% Info: Convert numbers to letters
%
% Copyright (C) 1999, 2006-2008, 2010, 2011 by
%    Heiko Oberdiek <heiko.oberdiek at googlemail.com>
%    2016
%    https://github.com/ho-tex/oberdiek/issues
%
% This work may be distributed and/or modified under the
% conditions of the LaTeX Project Public License, either
% version 1.3c of this license or (at your option) any later
% version. This version of this license is in
%    http://www.latex-project.org/lppl/lppl-1-3c.txt
% and the latest version of this license is in
%    http://www.latex-project.org/lppl.txt
% and version 1.3 or later is part of all distributions of
% LaTeX version 2005/12/01 or later.
%
% This work has the LPPL maintenance status "maintained".
%
% This Current Maintainer of this work is Heiko Oberdiek.
%
% The Base Interpreter refers to any `TeX-Format',
% because some files are installed in TDS:tex/generic//.
%
% This work consists of the main source file alphalph.dtx
% and the derived files
%    alphalph.sty, alphalph.pdf, alphalph.ins, alphalph.drv,
%    alphalph-test1.tex, alphalph-test2.tex, alphalph-test3.tex.
%
% Distribution:
%    CTAN:macros/latex/contrib/oberdiek/alphalph.dtx
%    CTAN:macros/latex/contrib/oberdiek/alphalph.pdf
%
% Unpacking:
%    (a) If alphalph.ins is present:
%           tex alphalph.ins
%    (b) Without alphalph.ins:
%           tex alphalph.dtx
%    (c) If you insist on using LaTeX
%           latex \let\install=y% \iffalse meta-comment
%
% File: alphalph.dtx
% Version: 2016/05/16 v2.5
% Info: Convert numbers to letters
%
% Copyright (C) 1999, 2006-2008, 2010, 2011 by
%    Heiko Oberdiek <heiko.oberdiek at googlemail.com>
%    2016
%    https://github.com/ho-tex/oberdiek/issues
%
% This work may be distributed and/or modified under the
% conditions of the LaTeX Project Public License, either
% version 1.3c of this license or (at your option) any later
% version. This version of this license is in
%    http://www.latex-project.org/lppl/lppl-1-3c.txt
% and the latest version of this license is in
%    http://www.latex-project.org/lppl.txt
% and version 1.3 or later is part of all distributions of
% LaTeX version 2005/12/01 or later.
%
% This work has the LPPL maintenance status "maintained".
%
% This Current Maintainer of this work is Heiko Oberdiek.
%
% The Base Interpreter refers to any `TeX-Format',
% because some files are installed in TDS:tex/generic//.
%
% This work consists of the main source file alphalph.dtx
% and the derived files
%    alphalph.sty, alphalph.pdf, alphalph.ins, alphalph.drv,
%    alphalph-test1.tex, alphalph-test2.tex, alphalph-test3.tex.
%
% Distribution:
%    CTAN:macros/latex/contrib/oberdiek/alphalph.dtx
%    CTAN:macros/latex/contrib/oberdiek/alphalph.pdf
%
% Unpacking:
%    (a) If alphalph.ins is present:
%           tex alphalph.ins
%    (b) Without alphalph.ins:
%           tex alphalph.dtx
%    (c) If you insist on using LaTeX
%           latex \let\install=y\input{alphalph.dtx}
%        (quote the arguments according to the demands of your shell)
%
% Documentation:
%    (a) If alphalph.drv is present:
%           latex alphalph.drv
%    (b) Without alphalph.drv:
%           latex alphalph.dtx; ...
%    The class ltxdoc loads the configuration file ltxdoc.cfg
%    if available. Here you can specify further options, e.g.
%    use A4 as paper format:
%       \PassOptionsToClass{a4paper}{article}
%
%    Programm calls to get the documentation (example):
%       pdflatex alphalph.dtx
%       makeindex -s gind.ist alphalph.idx
%       pdflatex alphalph.dtx
%       makeindex -s gind.ist alphalph.idx
%       pdflatex alphalph.dtx
%
% Installation:
%    TDS:tex/generic/oberdiek/alphalph.sty
%    TDS:doc/latex/oberdiek/alphalph.pdf
%    TDS:doc/latex/oberdiek/test/alphalph-test1.tex
%    TDS:doc/latex/oberdiek/test/alphalph-test2.tex
%    TDS:doc/latex/oberdiek/test/alphalph-test3.tex
%    TDS:source/latex/oberdiek/alphalph.dtx
%
%<*ignore>
\begingroup
  \catcode123=1 %
  \catcode125=2 %
  \def\x{LaTeX2e}%
\expandafter\endgroup
\ifcase 0\ifx\install y1\fi\expandafter
         \ifx\csname processbatchFile\endcsname\relax\else1\fi
         \ifx\fmtname\x\else 1\fi\relax
\else\csname fi\endcsname
%</ignore>
%<*install>
\input docstrip.tex
\Msg{************************************************************************}
\Msg{* Installation}
\Msg{* Package: alphalph 2016/05/16 v2.5 Convert numbers to letters (HO)}
\Msg{************************************************************************}

\keepsilent
\askforoverwritefalse

\let\MetaPrefix\relax
\preamble

This is a generated file.

Project: alphalph
Version: 2016/05/16 v2.5

Copyright (C) 1999, 2006-2008, 2010, 2011 by
   Heiko Oberdiek <heiko.oberdiek at googlemail.com>

This work may be distributed and/or modified under the
conditions of the LaTeX Project Public License, either
version 1.3c of this license or (at your option) any later
version. This version of this license is in
   http://www.latex-project.org/lppl/lppl-1-3c.txt
and the latest version of this license is in
   http://www.latex-project.org/lppl.txt
and version 1.3 or later is part of all distributions of
LaTeX version 2005/12/01 or later.

This work has the LPPL maintenance status "maintained".

This Current Maintainer of this work is Heiko Oberdiek.

The Base Interpreter refers to any `TeX-Format',
because some files are installed in TDS:tex/generic//.

This work consists of the main source file alphalph.dtx
and the derived files
   alphalph.sty, alphalph.pdf, alphalph.ins, alphalph.drv,
   alphalph-test1.tex, alphalph-test2.tex, alphalph-test3.tex.

\endpreamble
\let\MetaPrefix\DoubleperCent

\generate{%
  \file{alphalph.ins}{\from{alphalph.dtx}{install}}%
  \file{alphalph.drv}{\from{alphalph.dtx}{driver}}%
  \usedir{tex/generic/oberdiek}%
  \file{alphalph.sty}{\from{alphalph.dtx}{package}}%
%  \usedir{doc/latex/oberdiek/test}%
%  \file{alphalph-test1.tex}{\from{alphalph.dtx}{test1}}%
%  \file{alphalph-test2.tex}{\from{alphalph.dtx}{test2,etex}}%
%  \file{alphalph-test3.tex}{\from{alphalph.dtx}{test2,noetex}}%
  \nopreamble
  \nopostamble
%  \usedir{source/latex/oberdiek/catalogue}%
%  \file{alphalph.xml}{\from{alphalph.dtx}{catalogue}}%
}

\catcode32=13\relax% active space
\let =\space%
\Msg{************************************************************************}
\Msg{*}
\Msg{* To finish the installation you have to move the following}
\Msg{* file into a directory searched by TeX:}
\Msg{*}
\Msg{*     alphalph.sty}
\Msg{*}
\Msg{* To produce the documentation run the file `alphalph.drv'}
\Msg{* through LaTeX.}
\Msg{*}
\Msg{* Happy TeXing!}
\Msg{*}
\Msg{************************************************************************}

\endbatchfile
%</install>
%<*ignore>
\fi
%</ignore>
%<*driver>
\NeedsTeXFormat{LaTeX2e}
\ProvidesFile{alphalph.drv}%
  [2016/05/16 v2.5 Convert numbers to letters (HO)]%
\documentclass{ltxdoc}
\usepackage{alphalph}[2016/05/16]
\makeatletter
\newalphalph{\fnsymbolwrap}[wrap]{\@fnsymbol}{}
\newcommand*{\fnsymbolsingle}[1]{%
  \ensuremath{%
    \ifcase#1%
    \or *%
    \or \dagger
    \or \ddagger
    \or \mathsection
    \or \mathparagraph
    \else
      \@ctrerr
    \fi
  }%
}
\makeatother
\newalphalph{\fnsymbolmult}[mult]{\fnsymbolsingle}{}
\usepackage{amsmath}
\DeclareMathOperator{\opDiv}{div}
\DeclareMathOperator{\opMod}{mod}
\newcommand*{\Def}{%
  \ensuremath{%
    \mathrel{\mathop{:}}=%
  }%
}
\usepackage{holtxdoc}[2011/11/22]
\begin{document}
  \DocInput{alphalph.dtx}%
\end{document}
%</driver>
% \fi
%
%
% \CharacterTable
%  {Upper-case    \A\B\C\D\E\F\G\H\I\J\K\L\M\N\O\P\Q\R\S\T\U\V\W\X\Y\Z
%   Lower-case    \a\b\c\d\e\f\g\h\i\j\k\l\m\n\o\p\q\r\s\t\u\v\w\x\y\z
%   Digits        \0\1\2\3\4\5\6\7\8\9
%   Exclamation   \!     Double quote  \"     Hash (number) \#
%   Dollar        \$     Percent       \%     Ampersand     \&
%   Acute accent  \'     Left paren    \(     Right paren   \)
%   Asterisk      \*     Plus          \+     Comma         \,
%   Minus         \-     Point         \.     Solidus       \/
%   Colon         \:     Semicolon     \;     Less than     \<
%   Equals        \=     Greater than  \>     Question mark \?
%   Commercial at \@     Left bracket  \[     Backslash     \\
%   Right bracket \]     Circumflex    \^     Underscore    \_
%   Grave accent  \`     Left brace    \{     Vertical bar  \|
%   Right brace   \}     Tilde         \~}
%
% \GetFileInfo{alphalph.drv}
%
% \title{The \xpackage{alphalph} package}
% \date{2016/05/16 v2.5}
% \author{Heiko Oberdiek\thanks
% {Please report any issues at \url{https://github.com/ho-tex/oberdiek/issues}}\\
% \xemail{heiko.oberdiek at googlemail.com}}
%
% \maketitle
%
% \begin{abstract}
% The package provides methods to represent numbers with a limited
% set of symbols. Both \hologo{LaTeX} and \hologo{plainTeX} are supported.
% \end{abstract}
%
% \tableofcontents
%
% \section{Documentation}
%
% \subsection{Introduction}
% \hologo{LaTeX} counters can be represented in different ways
% by using presentation commands:
% \begin{quote}
%   \cs{arabic}, \cs{roman}, \cs{Roman},\\
%   \cs{alph}, \cs{Alph}, \cs{fnsymbol}
% \end{quote}
% The ranges of supported counter values are more or less
% restricted. Only \cs{arabic} can be used with any counter
% value \hologo{TeX} supports.
% \begin{quote}
% \catcode`\|=12 %
% \begin{tabular}{@{}l|l|l|l@{}}
% Presentation & Supported & Ignored & Error message\\
% command      & domain    & values  & ``Counter too large''\\
% \hline
% \cs{arabic}
%   & \ttfamily -MAX..MAX & &\\
% \cs{roman}, \cs{Roman}
%   & \ttfamily 1..MAX & \ttfamily -MAX..0 &\\
% \cs{alph}, \cs{Alph}
%   & \ttfamily 1..26 & 0 & \ttfamily -MAX..-1, 27..MAX\\
% \cs{fnsymbol}
%   & \ttfamily 1..9 & 0 & \ttfamily -MAX..-1, 10..MAX\\
% \hline
% \end{tabular}\\[1ex]
% \texttt{MAX} = \texttt{2147483647}
% \end{quote}
% Ordinal numbers are often used in documents: numbering
% of chapters, sections, figures, footnotes and so on.
% The layouter chooses \cs{Alph} for chapter numbers
% and \cs{fnsymbol} for footnotes. But what can be done
% if there are more than 26 chapters or more than 10 footnotes?
% This package \xpackage{alphalph} allows to define new
% presentation commands. They rely on a existing command
% and define presentations for values greater the limits.
% Three different methods are provided by the package.
% In the following use cases they are presentated.
%
% \subsection{Use cases}
%
% \subsubsection{Number system based on symbols}
%
% Asume you are writing a book and your lecturer demands
% that chapter numbers must be letters. But you have already
% 30 chapters and you have only 26 letters?
%
% In the decimal system the situation would be clear.
% If you run out of digits, you are using more digits
% to represent a number. This method can be also be used
% for letters. After chapter 26 with |Z| we us |AA|, |AB|, |AC|, and |AD|
% for the remaining chapters.
%
% Happily this package already defines this presentation command:
%\begin{quote}
%\begin{verbatim}
%\usepackage{alphalph}
%\renewcommand*{\thechapter}{%
%  \AlphAlph{\value{chapter}}%
%}
%\end{verbatim}
%\end{quote}
% \cs{AlphAlph} generates:
%   \AlphAlph{1}, \AlphAlph{2}, \AlphAlph{3}, \dots,
%   \AlphAlph{26}, \AlphAlph{27}, \AlphAlph{28}, \dots
%
% The other presentation command is \cs{alphalph} for lowercase letters.
%
% \subsubsection{Wrap symbols around}
%
% Nine footnote symbols are quite a few.
% Too soon the symbols are consumed
% and \hologo{LaTeX} complains with the error ``Counter too large''.
% However, it could be acceptable to start again with the symbols
% from the beginning, especially if there are less than nine symbols
% on a page.
% This could be achieved by a counter reset.
% But finding the right place can be difficult or needs manual
% actions. Also a
% unique counter value can be desirable (e.g. for generating
% unique anchor/link names).
% Package \xpackage{alphalph} allows you to define
% a macro that implements a ``wrap around'', but letting
% the value of the counter untouched:
%\begin{quote}
%\begin{verbatim}
%\usepackage{alphalph}
%\makeatletter
%\newalphalph{\fnsymbolwrap}[wrap]{\@fnsymbol}{}
%\makeatother
%\renewcommand*{\thefootnote}{%
%  \fnsymbolwrap{\value{footnote}}%
%}
%\end{verbatim}
%\end{quote}
% \cs{fnsymbolwrap} generates:
% \fnsymbolwrap{1}~(1), \fnsymbolwrap{2}~(2), \fnsymbolwrap{3}~(3), \dots,
% \fnsymbolwrap{9}~(9), \fnsymbolwrap{10}~(10), \fnsymbolwrap{11}~11, \dots
%
% \subsubsection{Multiple symbols}
%
% \Hologo{LaTeX}'s standard set of footnote symbols contains doubled
% symbols at the higher positions. Could this principle
% be generalized?
% Yes, but first we need a clean footnote symbol list without
% doubled entries, example:
%\begin{quote}
%\begin{verbatim}
%\usepackage{alphalph}
%\makeatletter
%\newcommand*{\fnsymbolsingle}[1]{%
%  \ensuremath{%
%    \ifcase#1%
%    \or *%
%    \or \dagger
%    \or \ddagger
%    \or \mathsection
%    \or \mathparagraph
%    \else
%      \@ctrerr
%    \fi
%  }%
%}
%\makeatother
%\newalphalph{\fnsymbolmult}[mult]{\fnsymbolsingle}{}
%\renewcommand*{\thefootnote}{%
%  \fnsymbolmult{\value{footnote}}%
%}
%\end{verbatim}
%\end{quote}
% The own definition of \cs{fnsymbolsingle} has the
% advantage that this list can easily modified.
% Otherwise you can use \cs{@fnsymbol} directly,
% because it uses the same first five symbols.
%\begin{quote}
%\begin{verbatim}
%\usepackage{alphalph}
%\makeatletter
%\newalphalph{\fnsymbolmult}[mult]{\@fnsymbol}{5}
%\makeatother
%\renewcommand*{\thefootnote}{%
%  \fnsymbolmult{\value{footnote}}%
%}
%\end{verbatim}
%\end{quote}
% \cs{fnsymbolmult} generates:
% \fnsymbolmult{1}~(1), \fnsymbolmult{2}~(2), \fnsymbolmult{3}~(3),
% \fnsymbolmult{4}~(4), \fnsymbolmult{5}~(5), \fnsymbolmult{6}~(6),
% \dots, \fnsymbolmult{16}~16, \fnsymbolmult{17}~17, \dots
%
% \bigskip
% The same method can also be used for the chapter problem
% in the first discussed use case:
%\begin{quote}
%\begin{verbatim}
%\usepackage{alphalph}
%\makeatletter
%\newalphalph{\AlphMult}[mult]{\@Alph}{26}
%\makeatother
%\renewcommand*{\chapter}{%
%  \AlphMult{\value{chapter}}%
%}
%\end{verbatim}
%\end{quote}
% \cs{AlphMult} then generates AA, BB, CC, and DD for chapters 27--30.
%
% \subsection{Glossary}
%
% \begin{description}
% \item[Counter presentation command] is a macro that
%   expects a \hologo{LaTeX} counter name as argument.
%   Numbers cannot be used. Examples:
%   \cs{arabic}, \cs{alph}, \cs{fnsymbol}.
% \item[Number presentation command] is a macro that
%   expects a number as argument. A number is anything
%   that \hologo{TeX} accepts as number including \cs{value}.
%   Examples: \cs{alphalph}, \cs{AlphAlph}, \cs{alphalph@alph}
%
%   However, \cs{alph} or \cs{fnsymbol} are not number
%   presentation commands because they expect a counter
%   name as argument. Happily \hologo{LaTeX} counter presentation
%   commands internally uses number presentation commands
%   with the same name, but prefixed by `@'. Thus
%   \cs{@alph}, \cs{@fnsymbol} are number presentation commands.
% \item[Symbols provider] is a command that can be used
%   to get a list of symbols. For example, \cs{@Alph}
%   provides the 26 uppercase letters from `A' to `Z'.
%   Basically a symbol provider is a number presentation
%   command, usually with a limited range.
% \item[Number of symbols] is the number of the
%   last symbol slot of a symbol provider. Thus \cs{@Alph}
%   generates 26 symbols, \cs{@fnsymbol} provides 9 symbols.
% \end{description}
%
% \subsection{Package usage}
%
% The package \Package{alphalph} can be used with both \hologo{plainTeX}
% and \hologo{LaTeX}:
% \begin{description}
% \item[\hologo{plainTeX}:] |\input alphalph.sty|
% \item[\hologo{LaTeX2e}:]   |\usepackage{alphalph}|\\
%      There aren't any options.
% \end{description}
%
% \subsection{User commands}
%
% \begin{declcs}{AlphAlph} \M{number}\\
%   \cs{alphalph} \M{number}
% \end{declcs}
% Both macros are number presentation commands that
% expects a number as argument. \hologo{LaTeX} counters
% are used with \cs{value}.
%
% The macros represents a number by letters.
% First single letters |A..Z| are used, then
% two letters |AA..ZZ|, three letters |AAA...ZZZ|, \dots
% follow.
%
% Macro \cs{AlphAlph} uses uppercase letters,
% \cs{alphalph} generates the lowercase variant.
% \begin{quote}
%   \catcode`\|=12
%   \def\entry#1{
%     \texttt{#1}
%     & \texttt{\AlphAlph{#1}}
%     & \texttt{\alphalph{#1}}
%   }
%   \begin{tabular}{@{}l|l|l@{}}
%     \meta{number} & \verb|\AlphAlph{|\meta{number}\verb|}|
%                   & \verb|\alphalph{|\meta{number}\verb|}|\\
%     \hline
%     \entry{1}\\
%     \entry{2}\\
%     \entry{26}\\
%     \entry{27}\\
%     \entry{30}\\
%     \entry{2000}\\
%     \entry{3752127}\\
%     \entry{10786572}\\
%     \entry{2147483647}\\
%   \end{tabular}
% \end{quote}
%
% \begin{declcs}{newalphalph}
%   \M{cmd}
%   |[|\meta{method}|]|
%   \M{symbols provider}
%   \M{number of symbols}
% \end{declcs}
% Macro \cs{newalphalph} defines \meta{cmd} as new
% number presentation command. Like \cs{newcommand} an error
% is thrown, if macro \meta{cmd} already exists.
%
% The \meta{method} is one of |alph|, |wrap|, or |mult|.
% The default is |alph|.
%
% As symbol provider a number presentation command
% can be used, e.g. \cs{@fnsymbol}, \cs{@Alph}, or
% \cs{alphalph@alph}.
%
% The last argument is the number of symbols.
% If the argument is empty, then \cs{newalphalph} tries
% to find this number itself. \hologo{LaTeX}'s number presentation
% commands throw an error message, if the number is too large.
% This error message is put in a macro \cs{@ctrerr}.
% Thus \cs{newalphalph} calls the symbol provider and tests
% a number by typesetting it in a temporary box. The error
% macro \cs{@ctrerr} is catched, it proofs that the number
% is not supported. Also if the width of the result is zero
% the number is considered as unavailable.
%
% The empty argument is useful for potentially variable lists.
% However if the end cannot be detected, then the number of
% symbols must be given. This is also a lot faster. Therefore don't
% let the argument empty without reason.
%
% \subsection{Programmer commands}
%
% \begin{declcs}{alphalph@Alph} \M{number}\\
%   \cs{alphalph@alph} \M{number}
% \end{declcs}
% They are basically the same as \cs{@Alph} and \cs{@alph}.
% Some languages of package \xpackage{babel} redefine
% \hologo{LaTeX}'s macros to include some font setup that
% breaks expandibility. Therefore \cs{AlphAlph} and \cs{alphalph}
% are based on \cs{alphalph@Alph} and \cs{alphalph@alph} to get
% the letters. The behaviour of these symbol providers for numbers
% outside the range |1..26| is undefined.
%
% \subsection{Design principles}
%
% \subsubsection{Number presentation commands}
%
% All number presentation commands that this package defines
% (including \cs{alphalph} and \cs{AlphAlph}) have the following
% properties:
% \begin{itemize}
% \item They are fully expandable. This means that they can safely
%       \begin{itemize}
%       \item be written to a file,
%       \item used in moving arguments (\hologo{LaTeX}: they are \emph{robust}),
%       \item used in a \cmd{\csname}-\cmd{\endcsname} pair.
%       \end{itemize}
% \item If the argument is zero or negative, the commands expand
%       to nothing like \cmd{\romannumeral}.
% \item The argument is a \hologo{TeX} number. Anything that would be
%       accepted by \cmd{\number} is a valid argument:
%       \begin{itemize}
%       \item explicite constants,
%       \item macros that expand to a number,
%       \item count registers, \hologo{LaTeX} counter can used
%             via \cmd{\value}, e.\,g.:\\
%             |\alphalph{\value{page}}|
%       \item \dots
%       \end{itemize}
% \item \hologo{eTeX}'s numeric expressions are supported, if \hologo{eTeX} is
%       available. Then \cs{numexpr} is applied to the argument.
%       Package \cs{calc}'s expressions are not supported.
%       That would violate the expandibility.
% \end{itemize}
%
% \subsubsection{General usability}
%
% \begin{description}
% \item[\hologo{TeX} format:]
%   The package does not depend on \hologo{LaTeX}, it can also be used
%   by \hologo{plainTeX}, for example.
% \item[\hologo{eTeX}:] \eTeX is supported, the macros are shorter and
%   faster. But \eTeX's extensions are not requirements.
%   Without \hologo{eTeX}, just the implementation changes. The properties
%   remain unchanged.
% \end{description}
%
% \StopEventually{
% }
%
% \section{Implementation}
% \subsection{Begin\texorpdfstring{ of package}{}}
%    \begin{macrocode}
%<*package>
%    \end{macrocode}
%    Reload check, especially if the package is not used with \LaTeX.
%    \begin{macrocode}
\begingroup\catcode61\catcode48\catcode32=10\relax%
  \catcode13=5 % ^^M
  \endlinechar=13 %
  \catcode35=6 % #
  \catcode39=12 % '
  \catcode44=12 % ,
  \catcode45=12 % -
  \catcode46=12 % .
  \catcode58=12 % :
  \catcode64=11 % @
  \catcode123=1 % {
  \catcode125=2 % }
  \expandafter\let\expandafter\x\csname ver@alphalph.sty\endcsname
  \ifx\x\relax % plain-TeX, first loading
  \else
    \def\empty{}%
    \ifx\x\empty % LaTeX, first loading,
      % variable is initialized, but \ProvidesPackage not yet seen
    \else
      \expandafter\ifx\csname PackageInfo\endcsname\relax
        \def\x#1#2{%
          \immediate\write-1{Package #1 Info: #2.}%
        }%
      \else
        \def\x#1#2{\PackageInfo{#1}{#2, stopped}}%
      \fi
      \x{alphalph}{The package is already loaded}%
      \aftergroup\endinput
    \fi
  \fi
\endgroup%
%    \end{macrocode}
%    Package identification:
%    \begin{macrocode}
\begingroup\catcode61\catcode48\catcode32=10\relax%
  \catcode13=5 % ^^M
  \endlinechar=13 %
  \catcode35=6 % #
  \catcode39=12 % '
  \catcode40=12 % (
  \catcode41=12 % )
  \catcode44=12 % ,
  \catcode45=12 % -
  \catcode46=12 % .
  \catcode47=12 % /
  \catcode58=12 % :
  \catcode64=11 % @
  \catcode91=12 % [
  \catcode93=12 % ]
  \catcode123=1 % {
  \catcode125=2 % }
  \expandafter\ifx\csname ProvidesPackage\endcsname\relax
    \def\x#1#2#3[#4]{\endgroup
      \immediate\write-1{Package: #3 #4}%
      \xdef#1{#4}%
    }%
  \else
    \def\x#1#2[#3]{\endgroup
      #2[{#3}]%
      \ifx#1\@undefined
        \xdef#1{#3}%
      \fi
      \ifx#1\relax
        \xdef#1{#3}%
      \fi
    }%
  \fi
\expandafter\x\csname ver@alphalph.sty\endcsname
\ProvidesPackage{alphalph}%
  [2016/05/16 v2.5 Convert numbers to letters (HO)]%
%    \end{macrocode}
%
% \subsection{Catcodes}
%
%    \begin{macrocode}
\begingroup\catcode61\catcode48\catcode32=10\relax%
  \catcode13=5 % ^^M
  \endlinechar=13 %
  \catcode123=1 % {
  \catcode125=2 % }
  \catcode64=11 % @
  \def\x{\endgroup
    \expandafter\edef\csname AlPh@AtEnd\endcsname{%
      \endlinechar=\the\endlinechar\relax
      \catcode13=\the\catcode13\relax
      \catcode32=\the\catcode32\relax
      \catcode35=\the\catcode35\relax
      \catcode61=\the\catcode61\relax
      \catcode64=\the\catcode64\relax
      \catcode123=\the\catcode123\relax
      \catcode125=\the\catcode125\relax
    }%
  }%
\x\catcode61\catcode48\catcode32=10\relax%
\catcode13=5 % ^^M
\endlinechar=13 %
\catcode35=6 % #
\catcode64=11 % @
\catcode123=1 % {
\catcode125=2 % }
\def\TMP@EnsureCode#1#2{%
  \edef\AlPh@AtEnd{%
    \AlPh@AtEnd
    \catcode#1=\the\catcode#1\relax
  }%
  \catcode#1=#2\relax
}
\TMP@EnsureCode{33}{12}% !
\TMP@EnsureCode{39}{12}% '
\TMP@EnsureCode{40}{12}% (
\TMP@EnsureCode{41}{12}% )
\TMP@EnsureCode{43}{12}% +
\TMP@EnsureCode{44}{12}% ,
\TMP@EnsureCode{46}{12}% .
\TMP@EnsureCode{47}{12}% /
\TMP@EnsureCode{59}{12}% ;
\TMP@EnsureCode{60}{12}% <
\TMP@EnsureCode{62}{12}% >
\TMP@EnsureCode{91}{12}% [
\TMP@EnsureCode{93}{12}% ]
\TMP@EnsureCode{96}{12}% `
\TMP@EnsureCode{124}{12}% |
\edef\AlPh@AtEnd{\AlPh@AtEnd\noexpand\endinput}
%    \end{macrocode}
%
% \subsection{Package loading}
%
%    \begin{macrocode}
\begingroup\expandafter\expandafter\expandafter\endgroup
\expandafter\ifx\csname RequirePackage\endcsname\relax
  \input infwarerr.sty\relax
  \input intcalc.sty\relax
\else
  \RequirePackage{infwarerr}[2007/09/09]%
  \RequirePackage{intcalc}[2007/09/09]%
\fi
%    \end{macrocode}
%
% \subsection{\hologo{eTeX} detection}
%
%    \begin{macrocode}
\begingroup\expandafter\expandafter\expandafter\endgroup
\expandafter\ifx\csname numexpr\endcsname\relax
  \catcode124=9 % '!': ignore
  \catcode43=14 % '+': comment
\else
  \catcode124=14 % '!': comment
  \catcode43=9   % '+': ignore
\fi
%    \end{macrocode}
%
% \subsection{Help macros}
%
%    \begin{macro}{\AlPh@Error}
%    \begin{macrocode}
\def\AlPh@Error#1{%
  \begingroup
    \escapechar=92 % backslash
    \@PackageError{alphalph}{#1}\@ehc
  \endgroup
}
%    \end{macrocode}
%    \end{macro}
%
%    \begin{macro}{\AlPh@IfDefinable}
%    \begin{macrocode}
\begingroup\expandafter\expandafter\expandafter\endgroup
\expandafter\ifx\csname @ifdefinable\endcsname\relax
  \def\AlPh@IfDefinable#1#2{%
    \ifcase\ifx#1\@undefined\else\ifx#1\relax\else1\fi\fi0 %
      #2%
    \else
      \AlPh@Error{%
        Command \string#1 already defined%
      }%
    \fi
  }%
%    \end{macrocode}
%    \end{macro}
%    \begin{macrocode}
\else
%    \end{macrocode}
%    \begin{macro}{\AlPh@IfDefinable}
%    \begin{macrocode}
  \let\AlPh@IfDefinable\@ifdefinable
%    \end{macrocode}
%    \end{macro}
%    \begin{macrocode}
\fi
%    \end{macrocode}
%
%    \begin{macro}{\@ReturnAfterElseFi}
%    \begin{macro}{\@ReturnAfterFi}
%    The following commands moves the `then' and `else' part respectively
%    behind the |\if|-construct. This prevents a too deep |\if|-nesting
%    and so a \hologo{TeX} capacity error because of a limited input stack size.
%    I use this trick in several packages, so I don't prefix these internal
%    commands in order not to have the same macros with different names.
%    (It saves memory.)
%    \begin{macrocode}
\long\def\@ReturnAfterElseFi#1\else#2\fi{\fi#1}
\long\def\@ReturnAfterFi#1\fi{\fi#1}
%    \end{macrocode}
%    \end{macro}
%    \end{macro}
%
%    \begin{macro}{\@gobblefour}
%    \Hologo{LaTeX} defines commands for eating arguments.
%    Define \cs{@gobblefour} if it is not defined (\hologo{plainTeX}).
%    \begin{macrocode}
\expandafter\ifx\csname @gobblefour\endcsname\relax
  \long\def\@gobblefour#1#2#3#4{}%
\fi
%    \end{macrocode}
%    \end{macro}
%
%    \begin{macro}{AlPh@IfOptArg}
%    \begin{macrocode}
\begingroup\expandafter\expandafter\expandafter\endgroup
\expandafter\ifx\csname kernel@ifnextchar\endcsname\relax
  \begingroup\expandafter\expandafter\expandafter\endgroup
  \expandafter\ifx\csname @ifnextchar\endcsname\relax
    \def\AlPh@IfOptArg#1#2{%
      \def\AlPh@TempA{#1}%
      \def\AlPh@TempB{#2}%
      \futurelet\AlPh@Token\AlPh@IfOptArgNext
    }%
    \let\AlPh@BracketLeft=[%]
    \def\AlPh@IfOptArgNext{%
      \ifx\AlPh@Token\AlPh@BracketLeft
        \expandafter\AlPh@TempA
      \else
        \expandafter\AlPh@TempB
      \fi
    }%
  \else
    \def\AlPh@IfOptArg{\@ifnextchar[}%]
  \fi
\else
  \def\AlPh@IfOptArg{\kernel@ifnextchar[}%]
\fi
%    \end{macrocode}
%    \end{macro}
%
% \subsection{Symbol provider}
%
% \subsubsection{Alphabet}
%
%    The output of \cs{alphalph} and \cs{AlphAlph} should be
%    usable as part of command names (see \cs{@namedef}, \cs{csname}, \dots).
%    Unhappily some languages of package \xpackage{babel}
%    redefine \hologo{LaTeX}'s \cs{@alph} and \cs{@Alph} in a manner that
%    they cannot be used in expandable context any more.
%    Therefore package \xpackage{alphalph} provides its own
%    commands.
%
%    \begin{macro}{\alphalph@Alph}
%    \begin{macro}{\alphalph@alph}
%    The two commands \cmd{\AlPh@Alph} and \cmd{\AlPh@alph} convert a
%    number into a letter (uppercase and lowercase respectivly).
%    The character |@| is used as an error symbol,
%    if the number isn't in the range of 1 until 26.
%    Here we need no space after the number |#1|, because the error
%    symbol |@| for the zero case stops scanning the number.
%    This error symbol should not appear anywhere (except for bugs).
%    \begin{macrocode}
\def\alphalph@Alph#1{%
  \ifcase#1%
    @%
  \or A\or B\or C\or D\or E\or F\or G\or H\or I\or J\or K\or L\or M%
  \or N\or O\or P\or Q\or R\or S\or T\or U\or V\or W\or X\or Y\or Z%
  \else
    \AlPh@ctrerr
    @%
  \fi
}
\def\alphalph@alph#1{%
  \ifcase#1%
    @%
  \or a\or b\or c\or d\or e\or f\or g\or h\or i\or j\or k\or l\or m%
  \or n\or o\or p\or q\or r\or s\or t\or u\or v\or w\or x\or y\or z%
  \else
    \AlPh@ctrerr
    @%
  \fi
}
%    \end{macrocode}
%    \end{macro}
%    \end{macro}
%    \begin{macro}{\AlPh@ctrerr}
%    Macro \cs{AlPh@ctrerr} is used as hook for the algorithm
%    to get the available number of symbols.
%    \begin{macrocode}
\def\AlPh@ctrerr{}
%    \end{macrocode}
%    \end{macro}
%
% \subsection{Finding number of symbols}
%
%    \begin{macro}{\AlPh@GetNumberOfSymbols}
%    \noindent
%    |#1|: symbols provider
%    \begin{macrocode}
\def\AlPh@GetNumberOfSymbols#1{%
  \AlPh@TestNumber1!{#1}%
  \ifAlPh@Unavailable
    \def\AlPh@Number{0}%
    \AlPh@Error{No symbols found}%
  \else
    \def\AlPh@Number{1}%
    \AlPh@ExpSearch2!{#1}%
  \fi
}
%    \end{macrocode}
%    \end{macro}
%    \begin{macro}{\ifAlPh@Unavailable}
%    \begin{macrocode}
\let\ifAlPh@Unavailable\iffalse
\def\AlPh@Unavailabletrue{%
  \global\let\ifAlPh@Unavailable\iftrue
}
\def\AlPh@Unavailablefalse{%
  \global\let\ifAlPh@Unavailable\iffalse
}
%    \end{macrocode}
%    \end{macro}
%    \begin{macro}{\AlPh@TestNumber}
%    |#1|: number to be tested\\
%    |#2|: symbols provider
%    \begin{macrocode}
\def\AlPh@TestNumber#1!#2{%
  \AlPh@Unavailablefalse
  \begingroup
    \setbox0=\hbox{%
      \begingroup % color
        \let\@ctrerr\AlPh@Unavailabletrue
        \let\AlPh@ctrerr\AlPh@Unavailabletrue
        #2{#1}%
      \endgroup
    }%
    \ifdim\wd0=0pt %
      \AlPh@Unavailabletrue
    \fi
  \endgroup
}
%    \end{macrocode}
%    \end{macro}
%    \begin{macro}{\AlPh@ExpSearch}
%    |#1|: number to be tested\\
%    |#2|: symbols provider
%    \begin{macrocode}
\def\AlPh@ExpSearch#1!#2{%
  \let\AlPh@Next\relax
  \AlPh@TestNumber#1!{#2}%
  \ifAlPh@Unavailable
    \expandafter\AlPh@BinSearch\AlPh@Number!#1!{#2}%
  \else
    \def\AlPh@Number{#1}%
    \ifnum#1>1073741823 %
      \AlPh@TestNumber2147483647!{#2}%
      \ifAlPh@Unavailable
        \AlPh@BinSearch#1!2147483647!{#2}%
      \else
        \def\AlPh@Number{0}%
        \AlPh@Error{%
          Maximal symbol number not found%
        }%
      \fi
    \else
      \def\AlPh@Next{%
        \expandafter\AlPh@ExpSearch\number\intcalcShl{#1}!{#2}%
      }%
    \fi
  \fi
  \AlPh@Next
}
%    \end{macrocode}
%    \end{macro}
%    \begin{macro}{\AlPh@BinSearch}
%    |#1|: available number\\
%    |#2|: unavailable number, |#2| $>$ |#1|\\
%    |#3|: symbols provider
%    \begin{macrocode}
\def\AlPh@BinSearch#1!#2!#3{%
  \expandafter\AlPh@ProcessBinSearch
  \number\intcalcShr{\intcalcAdd{#1}{#2}}!%
  #1!#2!{#3}%
}
%    \end{macrocode}
%    \end{macro}
%    \begin{macro}{\AlPh@ProcessBinSearch}
%    |#1|: number to be tested, |#2| $\leq$ |#1| $\leq$ |#3|\\
%    |#2|: available number\\
%    |#3|: unavailable number\\
%    |#4|: symbols provider
%    \begin{macrocode}
\def\AlPh@ProcessBinSearch#1!#2!#3!#4{%
  \let\AlPh@Next\relax
  \ifnum#1>#2 %
    \ifnum#1<#3 %
      \AlPh@TestNumber#1!{#4}%
      \ifAlPh@Unavailable
        \def\AlPh@Next{%
          \AlPh@BinSearch#2!#1!{#4}%
        }%
      \else
        \def\AlPh@Next{%
          \AlPh@BinSearch#1!#3!{#4}%
        }%
      \fi
    \else
      \def\AlPh@Number{#2}%
    \fi
  \else
    \def\AlPh@Number{#2}%
  \fi
  \AlPh@Next
}
%    \end{macrocode}
%    \end{macro}
%
% \subsection{Methods}
%
%    The names of method macros start with \cs{AlPh@Method}.
%    These macros do the main job in converting a number to
%    its representation.
%    A method command is called with three arguments.
%    The first argument is the number of symbols. The
%    second argument is the basic macro for converting
%    a number with limited number range. The last parameter
%    is the number that needs converting.
%
% \subsubsection{Common methods}
%
%    \begin{macro}{\AlPh@CheckPositive}
%    \noindent
%    |#1|: number to be checked
%    |#2|: continuation macro\\
%    |#3|: number of symbols (hidden here)\\
%    |#4|: symbol provider (hidden here)
%    \begin{macrocode}
\def\AlPh@CheckPositive#1!#2{%
  \ifnum#1<1 %
    \expandafter\@gobblefour
  \fi
  #2{#1}%
}
%    \end{macrocode}
%    \end{macro}
%
% \subsubsection{Method `alph'}
%
%    \begin{macro}{\AlPh@Method@alph}
%    \noindent
%    |#1|: number of symbols\\
%    |#2|: symbols provider\\
%    |#3|: number to be converted
%    \begin{macrocode}
\def\AlPh@Method@alph#1#2#3{%
  \expandafter\AlPh@CheckPositive
|   \number#3!%
+   \the\numexpr#3!%
    \AlPh@ProcessAlph
    {#1}{#2}%
}
%    \end{macrocode}
%    \end{macro}
%    \begin{macro}{\AlPh@ProcessAlph}
%    |#1|: current number\\
%    |#2|: number of symbols\\
%    |#3|: symbols provider
%    \begin{macrocode}
\def\AlPh@ProcessAlph#1#2#3{%
  \ifnum#1>#2 %
    \@ReturnAfterElseFi{%
      \expandafter\AlPh@StepAlph\number
        \intcalcInc{%
          \intcalcMod{\intcalcDec{#1}}{#2}%
        }%
      \expandafter!\number
        \intcalcDiv{\intcalcDec{#1}}{#2}%
      !{#2}{#3}%
    }%
  \else
    \@ReturnAfterFi{%
      #3{#1}%
    }%
  \fi
}
%    \end{macrocode}
%    \end{macro}
%    \begin{macro}{\AlPh@StepAlph}
%    |#1|: current last digit\\
%    |#2|: new current number\\
%    |#3|: number of symbols\\
%    |#4|: symbols provider
%    \begin{macrocode}
\def\AlPh@StepAlph#1!#2!#3#4{%
  \AlPh@ProcessAlph{#2}{#3}{#4}%
  #4{#1}%
}
%    \end{macrocode}
%    \end{macro}
%
% \subsubsection{Method `wrap'}
%
%    \begin{macro}{\AlPh@Method@wrap}
%    \noindent
%    |#1|: number of symbols\\
%    |#2|: symbols provider\\
%    |#3|: number to be converted
%    \begin{macrocode}
\def\AlPh@Method@wrap#1#2#3{%
  \expandafter\AlPh@CheckPositive
|   \number#3!%
+   \the\numexpr#3!%
    \AlPh@ProcessWrap
    {#1}{#2}%
}
%    \end{macrocode}
%    \end{macro}
%
%    \begin{macro}{\AlPh@ProcessWrap}
%    |#1|: number to be converted\\
%    |#2|: number of symbols\\
%    |#3|: symbols provider
%    \begin{macrocode}
\def\AlPh@ProcessWrap#1#2#3{%
  \ifnum#1>#2 %
    \@ReturnAfterElseFi{%
      \expandafter\AlPh@StepWrap\number
        \intcalcInc{\intcalcMod{\intcalcDec{#1}}{#2}}%
      !{#3}%
    }%
  \else
    \@ReturnAfterFi{%
      #3{#1}%
    }%
  \fi
}
%    \end{macrocode}
%    \end{macro}
%    \begin{macro}{\AlPh@StepWrap}
%    |#1|: final number\\
%    |#2|: symbols provider
%    \begin{macrocode}
\def\AlPh@StepWrap#1!#2{%
  #2{#1}%
}
%    \end{macrocode}
%    \end{macro}
%
% \subsubsection{Method `mult'}
%
%    After the number of symbols is exhausted,
%    repetitions of the symbol are used.
%    \begin{gather*}
%      x \Def \text{number to be converted}\\
%      n \Def \text{number of symbols}\\
%      r \Def \text{repetition length}\\
%      s \Def \text{symbol slot}\\
%      r = ((x - 1) \div n) + 1\\
%      s = ((x - 1) \mod n) + 1
%    \end{gather*}
%    \begin{macro}{\AlPh@Method@mult}
%    \noindent
%    |#1|: number of symbols\\
%    |#2|: symbols provider\\
%    |#3|: number to be converted
%    \begin{macrocode}
\def\AlPh@Method@mult#1#2#3{%
  \expandafter\AlPh@CheckPositive
|   \number#3!%
+   \the\numexpr#3!%
    \AlPh@ProcessMult
    {#1}{#2}%
}
%    \end{macrocode}
%    \end{macro}
%
%    \begin{macro}{\AlPh@ProcessMult}
%    |#1|: number to be converted\\
%    |#2|: number of symbols\\
%    |#3|: symbols provider
%    \begin{macrocode}
\def\AlPh@ProcessMult#1#2#3{%
  \ifnum#1>#2 %
    \@ReturnAfterElseFi{%
      \expandafter\AlPh@StepMult\romannumeral
        \intcalcInc{\intcalcDiv{\intcalcDec{#1}}{#2}}%
        000%
      \expandafter!\number
        \intcalcInc{\intcalcMod{\intcalcDec{#1}}{#2}}%
      !{#3}%
    }%
  \else
    \@ReturnAfterFi{%
      #3{#1}%
    }%
  \fi
}
%    \end{macrocode}
%    \end{macro}
%    \begin{macro}{\AlPh@StepMult}
%    |#1#2|: repetitions coded as list of character `m'\\
%    |#3|: symbol slot\\
%    |#4|: symbols provider
%    \begin{macrocode}
\def\AlPh@StepMult#1#2!#3!#4{%
  \ifx\\#2\\%
  \else
    \@ReturnAfterFi{%
      \AlPh@StepMult#2!#3!{#4}%
    }%
  \fi
  #4{#3}%
}
%    \end{macrocode}
%    \end{macro}
%
% \subsection{User interface}
%
%    \begin{macro}{\newalphalph}
%    Macro \cs{newalphalph} had three arguments in versions below 2.0.
%    For the new method argument we use an optional argument an first
%    position.\\
%    |#1|: cmd\\
%    |[#2]|: method name: |alph| (default), |wrap|, |mult|\\ % hash-ok
%    |#3|: symbols provider\\
%    |#4|: number of symbols
%    \begin{macrocode}
\AlPh@IfDefinable\newalphalph{%
  \def\newalphalph#1{%
    \AlPh@IfOptArg{%
      \AlPh@newalphalph{#1}%
    }{%
      \AlPh@newalphalph{#1}[alph]%
    }%
  }%
}
%    \end{macrocode}
%    \end{macro}
%    \begin{macro}{\AlPh@newalphalph}
%    |#1|: cmd
%    |#2|: method name\\
%    |#3|: symbols provider\\
%    |#4|: number of symbols
%    \begin{macrocode}
\def\AlPh@newalphalph#1[#2]#3#4{%
  \begingroup\expandafter\expandafter\expandafter\endgroup
  \expandafter\ifx\csname AlPh@Method@#2\endcsname\relax
    \AlPh@Error{%
      Unknown method %
|     `#2'%
+     `\detokenize{#2}'%
    }%
  \else
    \ifx\\#4\\%
      \AlPh@GetNumberOfSymbols{#3}%
      \ifcase\AlPh@Number
      \else
        \begingroup
          \escapechar=92 % backslash
          \@PackageInfo{alphalph}{%
            Number of symbols for \string#1 is \AlPh@Number
          }%
        \endgroup
        \expandafter\AlPh@NewAlphAlph
        \csname AlPh@Method@#2\expandafter\endcsname
        \AlPh@Number!{#1}{#3}%
      \fi
    \else
      \expandafter\AlPh@NewAlphAlph
      \csname AlPh@Method@#2\expandafter\endcsname
|     \number#4!%
+     \the\numexpr#4!%
      {#1}{#3}%
    \fi
  \fi
}%
%    \end{macrocode}
%    \end{macro}
%    \begin{macro}{\AlPh@NewAlphAlph}
%    |#1|: method macro\\
%    |#2|: number of symbols\\
%    |#3|: cmd\\
%    |#4|: symbols provider
%    \begin{macrocode}
\def\AlPh@NewAlphAlph#1#2!#3#4{%
  \AlPh@IfDefinable#3{%
    \ifnum#2>0 %
      \def#3{#1{#2}{#4}}%
    \else
      \AlPh@Error{%
          Definition of \string#3 failed,\MessageBreak
          because number of symbols (#2) is not positive%
        }%
    \fi
  }%
}
%    \end{macrocode}
%    \end{macro}
%
%    \begin{macro}{\AlphAlph}
%    \begin{macrocode}
\newalphalph\AlphAlph\alphalph@Alph{26}
%    \end{macrocode}
%    \end{macro}
%    \begin{macro}{\alphalph}
%    \begin{macrocode}
\newalphalph\alphalph\alphalph@alph{26}
%    \end{macrocode}
%    \end{macro}
%
%    \begin{macrocode}
\AlPh@AtEnd%
%</package>
%    \end{macrocode}
%
% \section{Test}
%
% \subsection{Catcode checks for loading}
%
%    \begin{macrocode}
%<*test1>
%    \end{macrocode}
%    \begin{macrocode}
\catcode`\{=1 %
\catcode`\}=2 %
\catcode`\#=6 %
\catcode`\@=11 %
\expandafter\ifx\csname count@\endcsname\relax
  \countdef\count@=255 %
\fi
\expandafter\ifx\csname @gobble\endcsname\relax
  \long\def\@gobble#1{}%
\fi
\expandafter\ifx\csname @firstofone\endcsname\relax
  \long\def\@firstofone#1{#1}%
\fi
\expandafter\ifx\csname loop\endcsname\relax
  \expandafter\@firstofone
\else
  \expandafter\@gobble
\fi
{%
  \def\loop#1\repeat{%
    \def\body{#1}%
    \iterate
  }%
  \def\iterate{%
    \body
      \let\next\iterate
    \else
      \let\next\relax
    \fi
    \next
  }%
  \let\repeat=\fi
}%
\def\RestoreCatcodes{}
\count@=0 %
\loop
  \edef\RestoreCatcodes{%
    \RestoreCatcodes
    \catcode\the\count@=\the\catcode\count@\relax
  }%
\ifnum\count@<255 %
  \advance\count@ 1 %
\repeat

\def\RangeCatcodeInvalid#1#2{%
  \count@=#1\relax
  \loop
    \catcode\count@=15 %
  \ifnum\count@<#2\relax
    \advance\count@ 1 %
  \repeat
}
\def\RangeCatcodeCheck#1#2#3{%
  \count@=#1\relax
  \loop
    \ifnum#3=\catcode\count@
    \else
      \errmessage{%
        Character \the\count@\space
        with wrong catcode \the\catcode\count@\space
        instead of \number#3%
      }%
    \fi
  \ifnum\count@<#2\relax
    \advance\count@ 1 %
  \repeat
}
\def\space{ }
\expandafter\ifx\csname LoadCommand\endcsname\relax
  \def\LoadCommand{\input alphalph.sty\relax}%
\fi
\def\Test{%
  \RangeCatcodeInvalid{0}{47}%
  \RangeCatcodeInvalid{58}{64}%
  \RangeCatcodeInvalid{91}{96}%
  \RangeCatcodeInvalid{123}{255}%
  \catcode`\@=12 %
  \catcode`\\=0 %
  \catcode`\%=14 %
  \LoadCommand
  \RangeCatcodeCheck{0}{36}{15}%
  \RangeCatcodeCheck{37}{37}{14}%
  \RangeCatcodeCheck{38}{47}{15}%
  \RangeCatcodeCheck{48}{57}{12}%
  \RangeCatcodeCheck{58}{63}{15}%
  \RangeCatcodeCheck{64}{64}{12}%
  \RangeCatcodeCheck{65}{90}{11}%
  \RangeCatcodeCheck{91}{91}{15}%
  \RangeCatcodeCheck{92}{92}{0}%
  \RangeCatcodeCheck{93}{96}{15}%
  \RangeCatcodeCheck{97}{122}{11}%
  \RangeCatcodeCheck{123}{255}{15}%
  \RestoreCatcodes
}
\Test
\csname @@end\endcsname
\end
%    \end{macrocode}
%    \begin{macrocode}
%</test1>
%    \end{macrocode}
%
% \section{Macro tests}
%
%    \begin{macrocode}
%<*test2>
\NeedsTeXFormat{LaTeX2e}
\nofiles
\documentclass{article}
%<*noetex>
\makeatletter
\let\saved@numexpr\numexpr
\newcommand*{\DisableNumexpr}{%
  \let\numexpr\@undefined
}
\newcommand*{\RestoreNumexpr}{%
  \let\numexpr\saved@numexpr
}
\DisableNumexpr
%</noetex>
\usepackage{alphalph}[2016/05/16]
%<noetex>\RestoreNumexpr
\usepackage{qstest}
\IncludeTests{*}
\LogTests{log}{*}{*}

\newcommand*{\TestCmd}[3]{%
  \setbox0=\hbox{%
%<noetex>    \DisableNumexpr
    \edef\TestString{#1{#2}}%
    \expandafter\Expect\expandafter{\TestString}{#3}%
    \edef\TestString{#1{#2} }%
    \expandafter\Expect\expandafter{\TestString}{#3 }%
  }%
  \Expect*{\the\wd0}{0.0pt}%
}

\makeatletter
\newalphalph\LaTeXAlphAlph\@Alph{26}
\newalphalph\LaTeXalphalph\@alph{26}
\newalphalph\AlphWrap[wrap]\alphalph@Alph{26}
\newalphalph\alphwrap[wrap]\alphalph@alph{26}
\newalphalph\LaTeXAlphWrap[wrap]\@Alph{26}
\newalphalph\LaTeXalphwrap[wrap]\@alph{26}
\def\LastSymbol#1{%
  \ifx\\#1\\%
  \else
    \@LastSymbol#1\@nil
  \fi
}
\def\@LastSymbol#1#2\@nil{%
  \ifx\\#2\\%
    #1%
  \else
    \@LastSymbol#2\@nil
  \fi
}
\makeatother
\newcommand*{\TestAlph}[2]{%
  \uppercase{\TestCallCmd\AlphAlph{#2}}{#1}%
  \lowercase{\TestCallCmd\alphalph{#2}}{#1}%
  \uppercase{\TestCallCmd\LaTeXAlphAlph{#2}}{#1}%
  \lowercase{\TestCallCmd\LaTeXalphalph{#2}}{#1}%
  \edef\WrapString{\LastSymbol{#2}}%
  \expandafter\TestAlphWrap\expandafter{\WrapString}{#1}%
}
\newcommand*{\TestAlphWrap}[2]{%
  \uppercase{\TestCallCmd\AlphWrap{#1}}{#2}%
  \lowercase{\TestCallCmd\alphwrap{#1}}{#2}%
  \uppercase{\TestCallCmd\LaTeXAlphWrap{#1}}{#2}%
  \lowercase{\TestCallCmd\LaTeXalphwrap{#1}}{#2}%
}
\newcommand*{\TestCallCmd}[3]{%
  \TestCmd#1{#3}{#2}%
}
\begin{qstest}{AlphSymbols}{alphalph, AlphAlph, symbols}
  \TestAlph{1}{a}%
  \TestAlph{2}{b}%
  \TestAlph{3}{c}%
  \TestAlph{4}{d}%
  \TestAlph{5}{e}%
  \TestAlph{6}{f}%
  \TestAlph{7}{g}%
  \TestAlph{8}{h}%
  \TestAlph{9}{i}%
  \TestAlph{10}{j}%
  \TestAlph{11}{k}%
  \TestAlph{12}{l}%
  \TestAlph{13}{m}%
  \TestAlph{14}{n}%
  \TestAlph{15}{o}%
  \TestAlph{16}{p}%
  \TestAlph{17}{q}%
  \TestAlph{18}{r}%
  \TestAlph{19}{s}%
  \TestAlph{20}{t}%
  \TestAlph{21}{u}%
  \TestAlph{22}{v}%
  \TestAlph{23}{w}%
  \TestAlph{24}{x}%
  \TestAlph{25}{y}%
  \TestAlph{26}{z}%
\end{qstest}
\begin{qstest}{AlphRange}{alphalph, range}
  \TestAlph{0}{}%
  \TestAlph{-1}{}%
  \TestAlph{-2147483647}{}%
  \TestAlph{27}{aa}%
  \TestAlph{28}{ab}%
  \TestAlph{52}{az}%
  \TestAlph{53}{ba}%
  \TestAlph{78}{bz}%
  \TestAlph{79}{ca}%
  \TestAlph{702}{zz}%
  \TestAlph{703}{aaa}%
  \TestAlph{2147483647}{fxshrxw}%
\end{qstest}

\makeatletter
\newcommand*{\myvocals}[1]{%
  \ifcase#1X\or A\or E\or I\or O\or U\else Y\fi
}
\makeatother
\newalphalph\vocalsvocals\myvocals{5}
\newcommand*{\TestVocals}{%
  \TestCmd\vocalsvocals
}
\begin{qstest}{vocals}{vocals}
  \TestVocals{0}{}%
  \TestVocals{1}{A}%
  \TestVocals{2}{E}%
  \TestVocals{3}{I}%
  \TestVocals{4}{O}%
  \TestVocals{5}{U}%
  \TestVocals{6}{AA}%
  \TestVocals{7}{AE}%
  \TestVocals{8}{AI}%
  \TestVocals{9}{AO}%
  \TestVocals{10}{AU}%
  \TestVocals{11}{EA}%
  \TestVocals{24}{OO}%
  \TestVocals{25}{OU}%
  \TestVocals{26}{UA}%
  \TestVocals{29}{UO}%
  \TestVocals{30}{UU}%
  \TestVocals{31}{AAA}%
  \TestVocals{155}{UUU}%
  \TestVocals{156}{AAAA}%
  \TestVocals{2147483647}{AIIOOEEIOIIUOE}%
\end{qstest}

\makeatletter
\newalphalph\AlphMult[mult]{\alphalph@Alph}{26}
\newalphalph\alphmult[mult]{\alphalph@alph}{26}
\newalphalph\LaTeXAlphMult[mult]{\@Alph}{26}
\newalphalph\LaTeXalphmult[mult]{\@alph}{26}
\makeatother
\newcommand*{\TestMult}[2]{%
  \uppercase{\TestCallCmd\AlphMult{#2}}{#1}%
  \lowercase{\TestCallCmd\alphmult{#2}}{#1}%
  \uppercase{\TestCallCmd\LaTeXAlphMult{#2}}{#1}%
  \lowercase{\TestCallCmd\LaTeXalphmult{#2}}{#1}%
}
\begin{qstest}{mult}{mult}
  \TestMult{0}{}%
  \TestMult{-1}{}%
  \TestMult{-2147483647}{}%
  \TestMult{1}{a}%
  \TestMult{2}{b}%
  \TestMult{26}{z}%
  \TestMult{27}{aa}%
  \TestMult{28}{bb}%
  \TestMult{52}{zz}%
  \TestMult{53}{aaa}%
  \TestMult{54}{bbb}%
  \TestMult{259}{yyyyyyyyyy}%
  \TestMult{260}{zzzzzzzzzz}%
  \TestMult{261}{aaaaaaaaaaa}%
  \TestMult{262}{bbbbbbbbbbb}%
\end{qstest}

\def\myvocalsB#1{%
  \ifcase#1\or A\or E\or I\or O\or U\fi
}
\begin{qstest}{symbolnum}{symbolnum}
  \makeatletter
  \def\Test#1#2{%
    \let\TestCmd\relax
    \newalphalph\TestCmd{#1}{}%
    \Expect*{\AlPh@Number}{#2}%
  }%
  \Test\@alph{26}%
  \Test\@Alph{26}%
  \Test\@fnsymbol{9}%
  \Test\myvocalsB{5}%
  \Test\alphalph@alph{26}%
  \Test\alphalph@Alph{26}%
\end{qstest}

\begin{qstest}{list}{list}
  \makeatletter
  \def\catch#1\relax{%
    \def\FoundList{\catch#1}%
  }%
  \def\Test[#1]#2#3#4{%
    \let\testcmd\relax
    \newalphalph\testcmd[{#1}]{\catch}{#2}%
    \testcmd{#3}|\relax
    \expandafter\Expect\expandafter{\FoundList}{#4|}%
    %
    \let\SavedCatch\catch
    \def\catch{\noexpand\catch\noexpand\foo}%
    \edef\Result{#4|}%
    \@onelevel@sanitize\Result
    \let\catch\SavedCatch
    \let\testcmd\relax
    \newalphalph\testcmd[{#1}]{\catch\foo}{#2}%
    \testcmd{#3}|\relax
    \@onelevel@sanitize\FoundList
    \Expect*{\FoundList}*{\Result}%
  }%
  \Test[alph]{26}{3}{\catch{3}}%
  \Test[alph]{26}{12}{\catch{12}}%
  \Test[alph]{26}{27}{\catch{1}\catch{1}}%
  \Test[alph]{26}{78}{\catch{2}\catch{26}}%
  \Test[wrap]{26}{7}{\catch{7}}%
  \Test[wrap]{26}{14}{\catch{14}}%
  \Test[wrap]{26}{80}{\catch{2}}%
  \Test[wrap]{26}{700}{\catch{24}}%
  \Test[mult]{26}{4}{\catch{4}}%
  \Test[mult]{26}{17}{\catch{17}}%
  \Test[mult]{26}{54}{\catch{2}\catch{2}\catch{2}}%
\end{qstest}

\begin{document}
\end{document}
%</test2>
%    \end{macrocode}
%
% \section{Installation}
%
% \subsection{Download}
%
% \paragraph{Package.} This package is available on
% CTAN\footnote{\CTANpkg{alphalph}}:
% \begin{description}
% \item[\CTAN{macros/latex/contrib/oberdiek/alphalph.dtx}] The source file.
% \item[\CTAN{macros/latex/contrib/oberdiek/alphalph.pdf}] Documentation.
% \end{description}
%
%
% \paragraph{Bundle.} All the packages of the bundle `oberdiek'
% are also available in a TDS compliant ZIP archive. There
% the packages are already unpacked and the documentation files
% are generated. The files and directories obey the TDS standard.
% \begin{description}
% \item[\CTANinstall{install/macros/latex/contrib/oberdiek.tds.zip}]
% \end{description}
% \emph{TDS} refers to the standard ``A Directory Structure
% for \TeX\ Files'' (\CTAN{tds/tds.pdf}). Directories
% with \xfile{texmf} in their name are usually organized this way.
%
% \subsection{Bundle installation}
%
% \paragraph{Unpacking.} Unpack the \xfile{oberdiek.tds.zip} in the
% TDS tree (also known as \xfile{texmf} tree) of your choice.
% Example (linux):
% \begin{quote}
%   |unzip oberdiek.tds.zip -d ~/texmf|
% \end{quote}
%
% \paragraph{Script installation.}
% Check the directory \xfile{TDS:scripts/oberdiek/} for
% scripts that need further installation steps.
% Package \xpackage{attachfile2} comes with the Perl script
% \xfile{pdfatfi.pl} that should be installed in such a way
% that it can be called as \texttt{pdfatfi}.
% Example (linux):
% \begin{quote}
%   |chmod +x scripts/oberdiek/pdfatfi.pl|\\
%   |cp scripts/oberdiek/pdfatfi.pl /usr/local/bin/|
% \end{quote}
%
% \subsection{Package installation}
%
% \paragraph{Unpacking.} The \xfile{.dtx} file is a self-extracting
% \docstrip\ archive. The files are extracted by running the
% \xfile{.dtx} through \plainTeX:
% \begin{quote}
%   \verb|tex alphalph.dtx|
% \end{quote}
%
% \paragraph{TDS.} Now the different files must be moved into
% the different directories in your installation TDS tree
% (also known as \xfile{texmf} tree):
% \begin{quote}
% \def\t{^^A
% \begin{tabular}{@{}>{\ttfamily}l@{ $\rightarrow$ }>{\ttfamily}l@{}}
%   alphalph.sty & tex/generic/oberdiek/alphalph.sty\\
%   alphalph.pdf & doc/latex/oberdiek/alphalph.pdf\\
%   test/alphalph-test1.tex & doc/latex/oberdiek/test/alphalph-test1.tex\\
%   test/alphalph-test2.tex & doc/latex/oberdiek/test/alphalph-test2.tex\\
%   test/alphalph-test3.tex & doc/latex/oberdiek/test/alphalph-test3.tex\\
%   alphalph.dtx & source/latex/oberdiek/alphalph.dtx\\
% \end{tabular}^^A
% }^^A
% \sbox0{\t}^^A
% \ifdim\wd0>\linewidth
%   \begingroup
%     \advance\linewidth by\leftmargin
%     \advance\linewidth by\rightmargin
%   \edef\x{\endgroup
%     \def\noexpand\lw{\the\linewidth}^^A
%   }\x
%   \def\lwbox{^^A
%     \leavevmode
%     \hbox to \linewidth{^^A
%       \kern-\leftmargin\relax
%       \hss
%       \usebox0
%       \hss
%       \kern-\rightmargin\relax
%     }^^A
%   }^^A
%   \ifdim\wd0>\lw
%     \sbox0{\small\t}^^A
%     \ifdim\wd0>\linewidth
%       \ifdim\wd0>\lw
%         \sbox0{\footnotesize\t}^^A
%         \ifdim\wd0>\linewidth
%           \ifdim\wd0>\lw
%             \sbox0{\scriptsize\t}^^A
%             \ifdim\wd0>\linewidth
%               \ifdim\wd0>\lw
%                 \sbox0{\tiny\t}^^A
%                 \ifdim\wd0>\linewidth
%                   \lwbox
%                 \else
%                   \usebox0
%                 \fi
%               \else
%                 \lwbox
%               \fi
%             \else
%               \usebox0
%             \fi
%           \else
%             \lwbox
%           \fi
%         \else
%           \usebox0
%         \fi
%       \else
%         \lwbox
%       \fi
%     \else
%       \usebox0
%     \fi
%   \else
%     \lwbox
%   \fi
% \else
%   \usebox0
% \fi
% \end{quote}
% If you have a \xfile{docstrip.cfg} that configures and enables \docstrip's
% TDS installing feature, then some files can already be in the right
% place, see the documentation of \docstrip.
%
% \subsection{Refresh file name databases}
%
% If your \TeX~distribution
% (\teTeX, \mikTeX, \dots) relies on file name databases, you must refresh
% these. For example, \teTeX\ users run \verb|texhash| or
% \verb|mktexlsr|.
%
% \subsection{Some details for the interested}
%
% \paragraph{Attached source.}
%
% The PDF documentation on CTAN also includes the
% \xfile{.dtx} source file. It can be extracted by
% AcrobatReader 6 or higher. Another option is \textsf{pdftk},
% e.g. unpack the file into the current directory:
% \begin{quote}
%   \verb|pdftk alphalph.pdf unpack_files output .|
% \end{quote}
%
% \paragraph{Unpacking with \LaTeX.}
% The \xfile{.dtx} chooses its action depending on the format:
% \begin{description}
% \item[\plainTeX:] Run \docstrip\ and extract the files.
% \item[\LaTeX:] Generate the documentation.
% \end{description}
% If you insist on using \LaTeX\ for \docstrip\ (really,
% \docstrip\ does not need \LaTeX), then inform the autodetect routine
% about your intention:
% \begin{quote}
%   \verb|latex \let\install=y\input{alphalph.dtx}|
% \end{quote}
% Do not forget to quote the argument according to the demands
% of your shell.
%
% \paragraph{Generating the documentation.}
% You can use both the \xfile{.dtx} or the \xfile{.drv} to generate
% the documentation. The process can be configured by the
% configuration file \xfile{ltxdoc.cfg}. For instance, put this
% line into this file, if you want to have A4 as paper format:
% \begin{quote}
%   \verb|\PassOptionsToClass{a4paper}{article}|
% \end{quote}
% An example follows how to generate the
% documentation with pdf\LaTeX:
% \begin{quote}
%\begin{verbatim}
%pdflatex alphalph.dtx
%makeindex -s gind.ist alphalph.idx
%pdflatex alphalph.dtx
%makeindex -s gind.ist alphalph.idx
%pdflatex alphalph.dtx
%\end{verbatim}
% \end{quote}
%
% \begin{History}
%   \begin{Version}{1999/03/19 v0.1}
%   \item
%     The first version was built as a response to a
%     \URL{\link{question}}^^A
%     {https://groups.google.com/group/comp.text.tex/msg/17a74cd721641038}^^A
%     of \NameEmail{Will Douglas}{william.douglas@wolfson.ox.ac.uk}
%     and the
%     \URL{\link{request}}^^A
%     {https://groups.google.com/group/comp.text.tex/msg/8f9768825640315f}^^A
%     of \NameEmail{Donald Arsenau}{asnd@reg.triumf.ca},
%     published in the newsgroup
%     \href{news:comp.text.tex}{comp.text.tex}:
%     \URL{``\link{Re: alph counters \textgreater\ 26}''}^^A
%     {https://groups.google.com/group/comp.text.tex/msg/cec563eef8bf65d0}
%   \item
%     Copyright: LPPL (\CTAN{macros/latex/base/lppl.txt})
%   \end{Version}
%   \begin{Version}{1999/04/12 v1.0}
%   \item
%     Documentation added in dtx format.
%   \item
%     \hologo{eTeX} support added.
%   \end{Version}
%   \begin{Version}{1999/04/13 v1.1}
%   \item
%     Minor documentation change.
%   \item
%     First CTAN release.
%   \end{Version}
%   \begin{Version}{1999/06/26 v1.2}
%   \item
%     First generic code about \cmd{\ProvidesPackage} improved.
%   \item
%     Documentation: Installation part revised.
%   \end{Version}
%   \begin{Version}{2006/02/20 v1.3}
%   \item
%     Reload check (for \hologo{plainTeX})
%   \item
%     New DTX framework.
%   \item
%     LPPL 1.3
%   \end{Version}
%   \begin{Version}{2006/05/30 v1.4}
%   \item
%     \cs{newalphalph} added.
%   \end{Version}
%   \begin{Version}{2007/04/11 v1.5}
%   \item
%     Line ends sanitized.
%   \end{Version}
%   \begin{Version}{2007/09/09 v2.0}
%   \item
%     New implementation that uses package \cs{intcalc}.
%     This removes the dependency on \hologo{eTeX}.
%   \item
%     \cs{newalphalph} is extended to support new methods `wrap' and 'multi'.
%   \item
%     Documentation rewritten.
%   \end{Version}
%   \begin{Version}{2008/08/11 v2.1}
%   \item
%     Code is not changed.
%   \item
%     URLs updated from \texttt{www.dejanews.com}
%     to \texttt{groups.google.com}.
%   \end{Version}
%   \begin{Version}{2010/03/01 v2.2}
%   \item
%     Compatibility with \hologo{iniTeX}.
%   \end{Version}
%   \begin{Version}{2010/04/18 v2.3}
%   \item
%     Documentation fixes (Martin M\"unch).
%   \end{Version}
%   \begin{Version}{2011/05/13 v2.4}
%   \item
%     Documentation fixes (Jim Diamond) and using package
%     \xpackage{hologo} for the documentation.
%   \item
%     Catalogue file added.
%   \end{Version}
%   \begin{Version}{2016/05/16 v2.5}
%   \item
%     Documentation updates.
%   \end{Version}
% \end{History}
%
% \PrintIndex
%
% \Finale
\endinput

%        (quote the arguments according to the demands of your shell)
%
% Documentation:
%    (a) If alphalph.drv is present:
%           latex alphalph.drv
%    (b) Without alphalph.drv:
%           latex alphalph.dtx; ...
%    The class ltxdoc loads the configuration file ltxdoc.cfg
%    if available. Here you can specify further options, e.g.
%    use A4 as paper format:
%       \PassOptionsToClass{a4paper}{article}
%
%    Programm calls to get the documentation (example):
%       pdflatex alphalph.dtx
%       makeindex -s gind.ist alphalph.idx
%       pdflatex alphalph.dtx
%       makeindex -s gind.ist alphalph.idx
%       pdflatex alphalph.dtx
%
% Installation:
%    TDS:tex/generic/oberdiek/alphalph.sty
%    TDS:doc/latex/oberdiek/alphalph.pdf
%    TDS:doc/latex/oberdiek/test/alphalph-test1.tex
%    TDS:doc/latex/oberdiek/test/alphalph-test2.tex
%    TDS:doc/latex/oberdiek/test/alphalph-test3.tex
%    TDS:source/latex/oberdiek/alphalph.dtx
%
%<*ignore>
\begingroup
  \catcode123=1 %
  \catcode125=2 %
  \def\x{LaTeX2e}%
\expandafter\endgroup
\ifcase 0\ifx\install y1\fi\expandafter
         \ifx\csname processbatchFile\endcsname\relax\else1\fi
         \ifx\fmtname\x\else 1\fi\relax
\else\csname fi\endcsname
%</ignore>
%<*install>
\input docstrip.tex
\Msg{************************************************************************}
\Msg{* Installation}
\Msg{* Package: alphalph 2016/05/16 v2.5 Convert numbers to letters (HO)}
\Msg{************************************************************************}

\keepsilent
\askforoverwritefalse

\let\MetaPrefix\relax
\preamble

This is a generated file.

Project: alphalph
Version: 2016/05/16 v2.5

Copyright (C) 1999, 2006-2008, 2010, 2011 by
   Heiko Oberdiek <heiko.oberdiek at googlemail.com>

This work may be distributed and/or modified under the
conditions of the LaTeX Project Public License, either
version 1.3c of this license or (at your option) any later
version. This version of this license is in
   http://www.latex-project.org/lppl/lppl-1-3c.txt
and the latest version of this license is in
   http://www.latex-project.org/lppl.txt
and version 1.3 or later is part of all distributions of
LaTeX version 2005/12/01 or later.

This work has the LPPL maintenance status "maintained".

This Current Maintainer of this work is Heiko Oberdiek.

The Base Interpreter refers to any `TeX-Format',
because some files are installed in TDS:tex/generic//.

This work consists of the main source file alphalph.dtx
and the derived files
   alphalph.sty, alphalph.pdf, alphalph.ins, alphalph.drv,
   alphalph-test1.tex, alphalph-test2.tex, alphalph-test3.tex.

\endpreamble
\let\MetaPrefix\DoubleperCent

\generate{%
  \file{alphalph.ins}{\from{alphalph.dtx}{install}}%
  \file{alphalph.drv}{\from{alphalph.dtx}{driver}}%
  \usedir{tex/generic/oberdiek}%
  \file{alphalph.sty}{\from{alphalph.dtx}{package}}%
%  \usedir{doc/latex/oberdiek/test}%
%  \file{alphalph-test1.tex}{\from{alphalph.dtx}{test1}}%
%  \file{alphalph-test2.tex}{\from{alphalph.dtx}{test2,etex}}%
%  \file{alphalph-test3.tex}{\from{alphalph.dtx}{test2,noetex}}%
  \nopreamble
  \nopostamble
%  \usedir{source/latex/oberdiek/catalogue}%
%  \file{alphalph.xml}{\from{alphalph.dtx}{catalogue}}%
}

\catcode32=13\relax% active space
\let =\space%
\Msg{************************************************************************}
\Msg{*}
\Msg{* To finish the installation you have to move the following}
\Msg{* file into a directory searched by TeX:}
\Msg{*}
\Msg{*     alphalph.sty}
\Msg{*}
\Msg{* To produce the documentation run the file `alphalph.drv'}
\Msg{* through LaTeX.}
\Msg{*}
\Msg{* Happy TeXing!}
\Msg{*}
\Msg{************************************************************************}

\endbatchfile
%</install>
%<*ignore>
\fi
%</ignore>
%<*driver>
\NeedsTeXFormat{LaTeX2e}
\ProvidesFile{alphalph.drv}%
  [2016/05/16 v2.5 Convert numbers to letters (HO)]%
\documentclass{ltxdoc}
\usepackage{alphalph}[2016/05/16]
\makeatletter
\newalphalph{\fnsymbolwrap}[wrap]{\@fnsymbol}{}
\newcommand*{\fnsymbolsingle}[1]{%
  \ensuremath{%
    \ifcase#1%
    \or *%
    \or \dagger
    \or \ddagger
    \or \mathsection
    \or \mathparagraph
    \else
      \@ctrerr
    \fi
  }%
}
\makeatother
\newalphalph{\fnsymbolmult}[mult]{\fnsymbolsingle}{}
\usepackage{amsmath}
\DeclareMathOperator{\opDiv}{div}
\DeclareMathOperator{\opMod}{mod}
\newcommand*{\Def}{%
  \ensuremath{%
    \mathrel{\mathop{:}}=%
  }%
}
\usepackage{holtxdoc}[2011/11/22]
\begin{document}
  \DocInput{alphalph.dtx}%
\end{document}
%</driver>
% \fi
%
%
% \CharacterTable
%  {Upper-case    \A\B\C\D\E\F\G\H\I\J\K\L\M\N\O\P\Q\R\S\T\U\V\W\X\Y\Z
%   Lower-case    \a\b\c\d\e\f\g\h\i\j\k\l\m\n\o\p\q\r\s\t\u\v\w\x\y\z
%   Digits        \0\1\2\3\4\5\6\7\8\9
%   Exclamation   \!     Double quote  \"     Hash (number) \#
%   Dollar        \$     Percent       \%     Ampersand     \&
%   Acute accent  \'     Left paren    \(     Right paren   \)
%   Asterisk      \*     Plus          \+     Comma         \,
%   Minus         \-     Point         \.     Solidus       \/
%   Colon         \:     Semicolon     \;     Less than     \<
%   Equals        \=     Greater than  \>     Question mark \?
%   Commercial at \@     Left bracket  \[     Backslash     \\
%   Right bracket \]     Circumflex    \^     Underscore    \_
%   Grave accent  \`     Left brace    \{     Vertical bar  \|
%   Right brace   \}     Tilde         \~}
%
% \GetFileInfo{alphalph.drv}
%
% \title{The \xpackage{alphalph} package}
% \date{2016/05/16 v2.5}
% \author{Heiko Oberdiek\thanks
% {Please report any issues at \url{https://github.com/ho-tex/oberdiek/issues}}\\
% \xemail{heiko.oberdiek at googlemail.com}}
%
% \maketitle
%
% \begin{abstract}
% The package provides methods to represent numbers with a limited
% set of symbols. Both \hologo{LaTeX} and \hologo{plainTeX} are supported.
% \end{abstract}
%
% \tableofcontents
%
% \section{Documentation}
%
% \subsection{Introduction}
% \hologo{LaTeX} counters can be represented in different ways
% by using presentation commands:
% \begin{quote}
%   \cs{arabic}, \cs{roman}, \cs{Roman},\\
%   \cs{alph}, \cs{Alph}, \cs{fnsymbol}
% \end{quote}
% The ranges of supported counter values are more or less
% restricted. Only \cs{arabic} can be used with any counter
% value \hologo{TeX} supports.
% \begin{quote}
% \catcode`\|=12 %
% \begin{tabular}{@{}l|l|l|l@{}}
% Presentation & Supported & Ignored & Error message\\
% command      & domain    & values  & ``Counter too large''\\
% \hline
% \cs{arabic}
%   & \ttfamily -MAX..MAX & &\\
% \cs{roman}, \cs{Roman}
%   & \ttfamily 1..MAX & \ttfamily -MAX..0 &\\
% \cs{alph}, \cs{Alph}
%   & \ttfamily 1..26 & 0 & \ttfamily -MAX..-1, 27..MAX\\
% \cs{fnsymbol}
%   & \ttfamily 1..9 & 0 & \ttfamily -MAX..-1, 10..MAX\\
% \hline
% \end{tabular}\\[1ex]
% \texttt{MAX} = \texttt{2147483647}
% \end{quote}
% Ordinal numbers are often used in documents: numbering
% of chapters, sections, figures, footnotes and so on.
% The layouter chooses \cs{Alph} for chapter numbers
% and \cs{fnsymbol} for footnotes. But what can be done
% if there are more than 26 chapters or more than 10 footnotes?
% This package \xpackage{alphalph} allows to define new
% presentation commands. They rely on a existing command
% and define presentations for values greater the limits.
% Three different methods are provided by the package.
% In the following use cases they are presentated.
%
% \subsection{Use cases}
%
% \subsubsection{Number system based on symbols}
%
% Asume you are writing a book and your lecturer demands
% that chapter numbers must be letters. But you have already
% 30 chapters and you have only 26 letters?
%
% In the decimal system the situation would be clear.
% If you run out of digits, you are using more digits
% to represent a number. This method can be also be used
% for letters. After chapter 26 with |Z| we us |AA|, |AB|, |AC|, and |AD|
% for the remaining chapters.
%
% Happily this package already defines this presentation command:
%\begin{quote}
%\begin{verbatim}
%\usepackage{alphalph}
%\renewcommand*{\thechapter}{%
%  \AlphAlph{\value{chapter}}%
%}
%\end{verbatim}
%\end{quote}
% \cs{AlphAlph} generates:
%   \AlphAlph{1}, \AlphAlph{2}, \AlphAlph{3}, \dots,
%   \AlphAlph{26}, \AlphAlph{27}, \AlphAlph{28}, \dots
%
% The other presentation command is \cs{alphalph} for lowercase letters.
%
% \subsubsection{Wrap symbols around}
%
% Nine footnote symbols are quite a few.
% Too soon the symbols are consumed
% and \hologo{LaTeX} complains with the error ``Counter too large''.
% However, it could be acceptable to start again with the symbols
% from the beginning, especially if there are less than nine symbols
% on a page.
% This could be achieved by a counter reset.
% But finding the right place can be difficult or needs manual
% actions. Also a
% unique counter value can be desirable (e.g. for generating
% unique anchor/link names).
% Package \xpackage{alphalph} allows you to define
% a macro that implements a ``wrap around'', but letting
% the value of the counter untouched:
%\begin{quote}
%\begin{verbatim}
%\usepackage{alphalph}
%\makeatletter
%\newalphalph{\fnsymbolwrap}[wrap]{\@fnsymbol}{}
%\makeatother
%\renewcommand*{\thefootnote}{%
%  \fnsymbolwrap{\value{footnote}}%
%}
%\end{verbatim}
%\end{quote}
% \cs{fnsymbolwrap} generates:
% \fnsymbolwrap{1}~(1), \fnsymbolwrap{2}~(2), \fnsymbolwrap{3}~(3), \dots,
% \fnsymbolwrap{9}~(9), \fnsymbolwrap{10}~(10), \fnsymbolwrap{11}~11, \dots
%
% \subsubsection{Multiple symbols}
%
% \Hologo{LaTeX}'s standard set of footnote symbols contains doubled
% symbols at the higher positions. Could this principle
% be generalized?
% Yes, but first we need a clean footnote symbol list without
% doubled entries, example:
%\begin{quote}
%\begin{verbatim}
%\usepackage{alphalph}
%\makeatletter
%\newcommand*{\fnsymbolsingle}[1]{%
%  \ensuremath{%
%    \ifcase#1%
%    \or *%
%    \or \dagger
%    \or \ddagger
%    \or \mathsection
%    \or \mathparagraph
%    \else
%      \@ctrerr
%    \fi
%  }%
%}
%\makeatother
%\newalphalph{\fnsymbolmult}[mult]{\fnsymbolsingle}{}
%\renewcommand*{\thefootnote}{%
%  \fnsymbolmult{\value{footnote}}%
%}
%\end{verbatim}
%\end{quote}
% The own definition of \cs{fnsymbolsingle} has the
% advantage that this list can easily modified.
% Otherwise you can use \cs{@fnsymbol} directly,
% because it uses the same first five symbols.
%\begin{quote}
%\begin{verbatim}
%\usepackage{alphalph}
%\makeatletter
%\newalphalph{\fnsymbolmult}[mult]{\@fnsymbol}{5}
%\makeatother
%\renewcommand*{\thefootnote}{%
%  \fnsymbolmult{\value{footnote}}%
%}
%\end{verbatim}
%\end{quote}
% \cs{fnsymbolmult} generates:
% \fnsymbolmult{1}~(1), \fnsymbolmult{2}~(2), \fnsymbolmult{3}~(3),
% \fnsymbolmult{4}~(4), \fnsymbolmult{5}~(5), \fnsymbolmult{6}~(6),
% \dots, \fnsymbolmult{16}~16, \fnsymbolmult{17}~17, \dots
%
% \bigskip
% The same method can also be used for the chapter problem
% in the first discussed use case:
%\begin{quote}
%\begin{verbatim}
%\usepackage{alphalph}
%\makeatletter
%\newalphalph{\AlphMult}[mult]{\@Alph}{26}
%\makeatother
%\renewcommand*{\chapter}{%
%  \AlphMult{\value{chapter}}%
%}
%\end{verbatim}
%\end{quote}
% \cs{AlphMult} then generates AA, BB, CC, and DD for chapters 27--30.
%
% \subsection{Glossary}
%
% \begin{description}
% \item[Counter presentation command] is a macro that
%   expects a \hologo{LaTeX} counter name as argument.
%   Numbers cannot be used. Examples:
%   \cs{arabic}, \cs{alph}, \cs{fnsymbol}.
% \item[Number presentation command] is a macro that
%   expects a number as argument. A number is anything
%   that \hologo{TeX} accepts as number including \cs{value}.
%   Examples: \cs{alphalph}, \cs{AlphAlph}, \cs{alphalph@alph}
%
%   However, \cs{alph} or \cs{fnsymbol} are not number
%   presentation commands because they expect a counter
%   name as argument. Happily \hologo{LaTeX} counter presentation
%   commands internally uses number presentation commands
%   with the same name, but prefixed by `@'. Thus
%   \cs{@alph}, \cs{@fnsymbol} are number presentation commands.
% \item[Symbols provider] is a command that can be used
%   to get a list of symbols. For example, \cs{@Alph}
%   provides the 26 uppercase letters from `A' to `Z'.
%   Basically a symbol provider is a number presentation
%   command, usually with a limited range.
% \item[Number of symbols] is the number of the
%   last symbol slot of a symbol provider. Thus \cs{@Alph}
%   generates 26 symbols, \cs{@fnsymbol} provides 9 symbols.
% \end{description}
%
% \subsection{Package usage}
%
% The package \Package{alphalph} can be used with both \hologo{plainTeX}
% and \hologo{LaTeX}:
% \begin{description}
% \item[\hologo{plainTeX}:] |\input alphalph.sty|
% \item[\hologo{LaTeX2e}:]   |\usepackage{alphalph}|\\
%      There aren't any options.
% \end{description}
%
% \subsection{User commands}
%
% \begin{declcs}{AlphAlph} \M{number}\\
%   \cs{alphalph} \M{number}
% \end{declcs}
% Both macros are number presentation commands that
% expects a number as argument. \hologo{LaTeX} counters
% are used with \cs{value}.
%
% The macros represents a number by letters.
% First single letters |A..Z| are used, then
% two letters |AA..ZZ|, three letters |AAA...ZZZ|, \dots
% follow.
%
% Macro \cs{AlphAlph} uses uppercase letters,
% \cs{alphalph} generates the lowercase variant.
% \begin{quote}
%   \catcode`\|=12
%   \def\entry#1{
%     \texttt{#1}
%     & \texttt{\AlphAlph{#1}}
%     & \texttt{\alphalph{#1}}
%   }
%   \begin{tabular}{@{}l|l|l@{}}
%     \meta{number} & \verb|\AlphAlph{|\meta{number}\verb|}|
%                   & \verb|\alphalph{|\meta{number}\verb|}|\\
%     \hline
%     \entry{1}\\
%     \entry{2}\\
%     \entry{26}\\
%     \entry{27}\\
%     \entry{30}\\
%     \entry{2000}\\
%     \entry{3752127}\\
%     \entry{10786572}\\
%     \entry{2147483647}\\
%   \end{tabular}
% \end{quote}
%
% \begin{declcs}{newalphalph}
%   \M{cmd}
%   |[|\meta{method}|]|
%   \M{symbols provider}
%   \M{number of symbols}
% \end{declcs}
% Macro \cs{newalphalph} defines \meta{cmd} as new
% number presentation command. Like \cs{newcommand} an error
% is thrown, if macro \meta{cmd} already exists.
%
% The \meta{method} is one of |alph|, |wrap|, or |mult|.
% The default is |alph|.
%
% As symbol provider a number presentation command
% can be used, e.g. \cs{@fnsymbol}, \cs{@Alph}, or
% \cs{alphalph@alph}.
%
% The last argument is the number of symbols.
% If the argument is empty, then \cs{newalphalph} tries
% to find this number itself. \hologo{LaTeX}'s number presentation
% commands throw an error message, if the number is too large.
% This error message is put in a macro \cs{@ctrerr}.
% Thus \cs{newalphalph} calls the symbol provider and tests
% a number by typesetting it in a temporary box. The error
% macro \cs{@ctrerr} is catched, it proofs that the number
% is not supported. Also if the width of the result is zero
% the number is considered as unavailable.
%
% The empty argument is useful for potentially variable lists.
% However if the end cannot be detected, then the number of
% symbols must be given. This is also a lot faster. Therefore don't
% let the argument empty without reason.
%
% \subsection{Programmer commands}
%
% \begin{declcs}{alphalph@Alph} \M{number}\\
%   \cs{alphalph@alph} \M{number}
% \end{declcs}
% They are basically the same as \cs{@Alph} and \cs{@alph}.
% Some languages of package \xpackage{babel} redefine
% \hologo{LaTeX}'s macros to include some font setup that
% breaks expandibility. Therefore \cs{AlphAlph} and \cs{alphalph}
% are based on \cs{alphalph@Alph} and \cs{alphalph@alph} to get
% the letters. The behaviour of these symbol providers for numbers
% outside the range |1..26| is undefined.
%
% \subsection{Design principles}
%
% \subsubsection{Number presentation commands}
%
% All number presentation commands that this package defines
% (including \cs{alphalph} and \cs{AlphAlph}) have the following
% properties:
% \begin{itemize}
% \item They are fully expandable. This means that they can safely
%       \begin{itemize}
%       \item be written to a file,
%       \item used in moving arguments (\hologo{LaTeX}: they are \emph{robust}),
%       \item used in a \cmd{\csname}-\cmd{\endcsname} pair.
%       \end{itemize}
% \item If the argument is zero or negative, the commands expand
%       to nothing like \cmd{\romannumeral}.
% \item The argument is a \hologo{TeX} number. Anything that would be
%       accepted by \cmd{\number} is a valid argument:
%       \begin{itemize}
%       \item explicite constants,
%       \item macros that expand to a number,
%       \item count registers, \hologo{LaTeX} counter can used
%             via \cmd{\value}, e.\,g.:\\
%             |\alphalph{\value{page}}|
%       \item \dots
%       \end{itemize}
% \item \hologo{eTeX}'s numeric expressions are supported, if \hologo{eTeX} is
%       available. Then \cs{numexpr} is applied to the argument.
%       Package \cs{calc}'s expressions are not supported.
%       That would violate the expandibility.
% \end{itemize}
%
% \subsubsection{General usability}
%
% \begin{description}
% \item[\hologo{TeX} format:]
%   The package does not depend on \hologo{LaTeX}, it can also be used
%   by \hologo{plainTeX}, for example.
% \item[\hologo{eTeX}:] \eTeX is supported, the macros are shorter and
%   faster. But \eTeX's extensions are not requirements.
%   Without \hologo{eTeX}, just the implementation changes. The properties
%   remain unchanged.
% \end{description}
%
% \StopEventually{
% }
%
% \section{Implementation}
% \subsection{Begin\texorpdfstring{ of package}{}}
%    \begin{macrocode}
%<*package>
%    \end{macrocode}
%    Reload check, especially if the package is not used with \LaTeX.
%    \begin{macrocode}
\begingroup\catcode61\catcode48\catcode32=10\relax%
  \catcode13=5 % ^^M
  \endlinechar=13 %
  \catcode35=6 % #
  \catcode39=12 % '
  \catcode44=12 % ,
  \catcode45=12 % -
  \catcode46=12 % .
  \catcode58=12 % :
  \catcode64=11 % @
  \catcode123=1 % {
  \catcode125=2 % }
  \expandafter\let\expandafter\x\csname ver@alphalph.sty\endcsname
  \ifx\x\relax % plain-TeX, first loading
  \else
    \def\empty{}%
    \ifx\x\empty % LaTeX, first loading,
      % variable is initialized, but \ProvidesPackage not yet seen
    \else
      \expandafter\ifx\csname PackageInfo\endcsname\relax
        \def\x#1#2{%
          \immediate\write-1{Package #1 Info: #2.}%
        }%
      \else
        \def\x#1#2{\PackageInfo{#1}{#2, stopped}}%
      \fi
      \x{alphalph}{The package is already loaded}%
      \aftergroup\endinput
    \fi
  \fi
\endgroup%
%    \end{macrocode}
%    Package identification:
%    \begin{macrocode}
\begingroup\catcode61\catcode48\catcode32=10\relax%
  \catcode13=5 % ^^M
  \endlinechar=13 %
  \catcode35=6 % #
  \catcode39=12 % '
  \catcode40=12 % (
  \catcode41=12 % )
  \catcode44=12 % ,
  \catcode45=12 % -
  \catcode46=12 % .
  \catcode47=12 % /
  \catcode58=12 % :
  \catcode64=11 % @
  \catcode91=12 % [
  \catcode93=12 % ]
  \catcode123=1 % {
  \catcode125=2 % }
  \expandafter\ifx\csname ProvidesPackage\endcsname\relax
    \def\x#1#2#3[#4]{\endgroup
      \immediate\write-1{Package: #3 #4}%
      \xdef#1{#4}%
    }%
  \else
    \def\x#1#2[#3]{\endgroup
      #2[{#3}]%
      \ifx#1\@undefined
        \xdef#1{#3}%
      \fi
      \ifx#1\relax
        \xdef#1{#3}%
      \fi
    }%
  \fi
\expandafter\x\csname ver@alphalph.sty\endcsname
\ProvidesPackage{alphalph}%
  [2016/05/16 v2.5 Convert numbers to letters (HO)]%
%    \end{macrocode}
%
% \subsection{Catcodes}
%
%    \begin{macrocode}
\begingroup\catcode61\catcode48\catcode32=10\relax%
  \catcode13=5 % ^^M
  \endlinechar=13 %
  \catcode123=1 % {
  \catcode125=2 % }
  \catcode64=11 % @
  \def\x{\endgroup
    \expandafter\edef\csname AlPh@AtEnd\endcsname{%
      \endlinechar=\the\endlinechar\relax
      \catcode13=\the\catcode13\relax
      \catcode32=\the\catcode32\relax
      \catcode35=\the\catcode35\relax
      \catcode61=\the\catcode61\relax
      \catcode64=\the\catcode64\relax
      \catcode123=\the\catcode123\relax
      \catcode125=\the\catcode125\relax
    }%
  }%
\x\catcode61\catcode48\catcode32=10\relax%
\catcode13=5 % ^^M
\endlinechar=13 %
\catcode35=6 % #
\catcode64=11 % @
\catcode123=1 % {
\catcode125=2 % }
\def\TMP@EnsureCode#1#2{%
  \edef\AlPh@AtEnd{%
    \AlPh@AtEnd
    \catcode#1=\the\catcode#1\relax
  }%
  \catcode#1=#2\relax
}
\TMP@EnsureCode{33}{12}% !
\TMP@EnsureCode{39}{12}% '
\TMP@EnsureCode{40}{12}% (
\TMP@EnsureCode{41}{12}% )
\TMP@EnsureCode{43}{12}% +
\TMP@EnsureCode{44}{12}% ,
\TMP@EnsureCode{46}{12}% .
\TMP@EnsureCode{47}{12}% /
\TMP@EnsureCode{59}{12}% ;
\TMP@EnsureCode{60}{12}% <
\TMP@EnsureCode{62}{12}% >
\TMP@EnsureCode{91}{12}% [
\TMP@EnsureCode{93}{12}% ]
\TMP@EnsureCode{96}{12}% `
\TMP@EnsureCode{124}{12}% |
\edef\AlPh@AtEnd{\AlPh@AtEnd\noexpand\endinput}
%    \end{macrocode}
%
% \subsection{Package loading}
%
%    \begin{macrocode}
\begingroup\expandafter\expandafter\expandafter\endgroup
\expandafter\ifx\csname RequirePackage\endcsname\relax
  \input infwarerr.sty\relax
  \input intcalc.sty\relax
\else
  \RequirePackage{infwarerr}[2007/09/09]%
  \RequirePackage{intcalc}[2007/09/09]%
\fi
%    \end{macrocode}
%
% \subsection{\hologo{eTeX} detection}
%
%    \begin{macrocode}
\begingroup\expandafter\expandafter\expandafter\endgroup
\expandafter\ifx\csname numexpr\endcsname\relax
  \catcode124=9 % '!': ignore
  \catcode43=14 % '+': comment
\else
  \catcode124=14 % '!': comment
  \catcode43=9   % '+': ignore
\fi
%    \end{macrocode}
%
% \subsection{Help macros}
%
%    \begin{macro}{\AlPh@Error}
%    \begin{macrocode}
\def\AlPh@Error#1{%
  \begingroup
    \escapechar=92 % backslash
    \@PackageError{alphalph}{#1}\@ehc
  \endgroup
}
%    \end{macrocode}
%    \end{macro}
%
%    \begin{macro}{\AlPh@IfDefinable}
%    \begin{macrocode}
\begingroup\expandafter\expandafter\expandafter\endgroup
\expandafter\ifx\csname @ifdefinable\endcsname\relax
  \def\AlPh@IfDefinable#1#2{%
    \ifcase\ifx#1\@undefined\else\ifx#1\relax\else1\fi\fi0 %
      #2%
    \else
      \AlPh@Error{%
        Command \string#1 already defined%
      }%
    \fi
  }%
%    \end{macrocode}
%    \end{macro}
%    \begin{macrocode}
\else
%    \end{macrocode}
%    \begin{macro}{\AlPh@IfDefinable}
%    \begin{macrocode}
  \let\AlPh@IfDefinable\@ifdefinable
%    \end{macrocode}
%    \end{macro}
%    \begin{macrocode}
\fi
%    \end{macrocode}
%
%    \begin{macro}{\@ReturnAfterElseFi}
%    \begin{macro}{\@ReturnAfterFi}
%    The following commands moves the `then' and `else' part respectively
%    behind the |\if|-construct. This prevents a too deep |\if|-nesting
%    and so a \hologo{TeX} capacity error because of a limited input stack size.
%    I use this trick in several packages, so I don't prefix these internal
%    commands in order not to have the same macros with different names.
%    (It saves memory.)
%    \begin{macrocode}
\long\def\@ReturnAfterElseFi#1\else#2\fi{\fi#1}
\long\def\@ReturnAfterFi#1\fi{\fi#1}
%    \end{macrocode}
%    \end{macro}
%    \end{macro}
%
%    \begin{macro}{\@gobblefour}
%    \Hologo{LaTeX} defines commands for eating arguments.
%    Define \cs{@gobblefour} if it is not defined (\hologo{plainTeX}).
%    \begin{macrocode}
\expandafter\ifx\csname @gobblefour\endcsname\relax
  \long\def\@gobblefour#1#2#3#4{}%
\fi
%    \end{macrocode}
%    \end{macro}
%
%    \begin{macro}{AlPh@IfOptArg}
%    \begin{macrocode}
\begingroup\expandafter\expandafter\expandafter\endgroup
\expandafter\ifx\csname kernel@ifnextchar\endcsname\relax
  \begingroup\expandafter\expandafter\expandafter\endgroup
  \expandafter\ifx\csname @ifnextchar\endcsname\relax
    \def\AlPh@IfOptArg#1#2{%
      \def\AlPh@TempA{#1}%
      \def\AlPh@TempB{#2}%
      \futurelet\AlPh@Token\AlPh@IfOptArgNext
    }%
    \let\AlPh@BracketLeft=[%]
    \def\AlPh@IfOptArgNext{%
      \ifx\AlPh@Token\AlPh@BracketLeft
        \expandafter\AlPh@TempA
      \else
        \expandafter\AlPh@TempB
      \fi
    }%
  \else
    \def\AlPh@IfOptArg{\@ifnextchar[}%]
  \fi
\else
  \def\AlPh@IfOptArg{\kernel@ifnextchar[}%]
\fi
%    \end{macrocode}
%    \end{macro}
%
% \subsection{Symbol provider}
%
% \subsubsection{Alphabet}
%
%    The output of \cs{alphalph} and \cs{AlphAlph} should be
%    usable as part of command names (see \cs{@namedef}, \cs{csname}, \dots).
%    Unhappily some languages of package \xpackage{babel}
%    redefine \hologo{LaTeX}'s \cs{@alph} and \cs{@Alph} in a manner that
%    they cannot be used in expandable context any more.
%    Therefore package \xpackage{alphalph} provides its own
%    commands.
%
%    \begin{macro}{\alphalph@Alph}
%    \begin{macro}{\alphalph@alph}
%    The two commands \cmd{\AlPh@Alph} and \cmd{\AlPh@alph} convert a
%    number into a letter (uppercase and lowercase respectivly).
%    The character |@| is used as an error symbol,
%    if the number isn't in the range of 1 until 26.
%    Here we need no space after the number |#1|, because the error
%    symbol |@| for the zero case stops scanning the number.
%    This error symbol should not appear anywhere (except for bugs).
%    \begin{macrocode}
\def\alphalph@Alph#1{%
  \ifcase#1%
    @%
  \or A\or B\or C\or D\or E\or F\or G\or H\or I\or J\or K\or L\or M%
  \or N\or O\or P\or Q\or R\or S\or T\or U\or V\or W\or X\or Y\or Z%
  \else
    \AlPh@ctrerr
    @%
  \fi
}
\def\alphalph@alph#1{%
  \ifcase#1%
    @%
  \or a\or b\or c\or d\or e\or f\or g\or h\or i\or j\or k\or l\or m%
  \or n\or o\or p\or q\or r\or s\or t\or u\or v\or w\or x\or y\or z%
  \else
    \AlPh@ctrerr
    @%
  \fi
}
%    \end{macrocode}
%    \end{macro}
%    \end{macro}
%    \begin{macro}{\AlPh@ctrerr}
%    Macro \cs{AlPh@ctrerr} is used as hook for the algorithm
%    to get the available number of symbols.
%    \begin{macrocode}
\def\AlPh@ctrerr{}
%    \end{macrocode}
%    \end{macro}
%
% \subsection{Finding number of symbols}
%
%    \begin{macro}{\AlPh@GetNumberOfSymbols}
%    \noindent
%    |#1|: symbols provider
%    \begin{macrocode}
\def\AlPh@GetNumberOfSymbols#1{%
  \AlPh@TestNumber1!{#1}%
  \ifAlPh@Unavailable
    \def\AlPh@Number{0}%
    \AlPh@Error{No symbols found}%
  \else
    \def\AlPh@Number{1}%
    \AlPh@ExpSearch2!{#1}%
  \fi
}
%    \end{macrocode}
%    \end{macro}
%    \begin{macro}{\ifAlPh@Unavailable}
%    \begin{macrocode}
\let\ifAlPh@Unavailable\iffalse
\def\AlPh@Unavailabletrue{%
  \global\let\ifAlPh@Unavailable\iftrue
}
\def\AlPh@Unavailablefalse{%
  \global\let\ifAlPh@Unavailable\iffalse
}
%    \end{macrocode}
%    \end{macro}
%    \begin{macro}{\AlPh@TestNumber}
%    |#1|: number to be tested\\
%    |#2|: symbols provider
%    \begin{macrocode}
\def\AlPh@TestNumber#1!#2{%
  \AlPh@Unavailablefalse
  \begingroup
    \setbox0=\hbox{%
      \begingroup % color
        \let\@ctrerr\AlPh@Unavailabletrue
        \let\AlPh@ctrerr\AlPh@Unavailabletrue
        #2{#1}%
      \endgroup
    }%
    \ifdim\wd0=0pt %
      \AlPh@Unavailabletrue
    \fi
  \endgroup
}
%    \end{macrocode}
%    \end{macro}
%    \begin{macro}{\AlPh@ExpSearch}
%    |#1|: number to be tested\\
%    |#2|: symbols provider
%    \begin{macrocode}
\def\AlPh@ExpSearch#1!#2{%
  \let\AlPh@Next\relax
  \AlPh@TestNumber#1!{#2}%
  \ifAlPh@Unavailable
    \expandafter\AlPh@BinSearch\AlPh@Number!#1!{#2}%
  \else
    \def\AlPh@Number{#1}%
    \ifnum#1>1073741823 %
      \AlPh@TestNumber2147483647!{#2}%
      \ifAlPh@Unavailable
        \AlPh@BinSearch#1!2147483647!{#2}%
      \else
        \def\AlPh@Number{0}%
        \AlPh@Error{%
          Maximal symbol number not found%
        }%
      \fi
    \else
      \def\AlPh@Next{%
        \expandafter\AlPh@ExpSearch\number\intcalcShl{#1}!{#2}%
      }%
    \fi
  \fi
  \AlPh@Next
}
%    \end{macrocode}
%    \end{macro}
%    \begin{macro}{\AlPh@BinSearch}
%    |#1|: available number\\
%    |#2|: unavailable number, |#2| $>$ |#1|\\
%    |#3|: symbols provider
%    \begin{macrocode}
\def\AlPh@BinSearch#1!#2!#3{%
  \expandafter\AlPh@ProcessBinSearch
  \number\intcalcShr{\intcalcAdd{#1}{#2}}!%
  #1!#2!{#3}%
}
%    \end{macrocode}
%    \end{macro}
%    \begin{macro}{\AlPh@ProcessBinSearch}
%    |#1|: number to be tested, |#2| $\leq$ |#1| $\leq$ |#3|\\
%    |#2|: available number\\
%    |#3|: unavailable number\\
%    |#4|: symbols provider
%    \begin{macrocode}
\def\AlPh@ProcessBinSearch#1!#2!#3!#4{%
  \let\AlPh@Next\relax
  \ifnum#1>#2 %
    \ifnum#1<#3 %
      \AlPh@TestNumber#1!{#4}%
      \ifAlPh@Unavailable
        \def\AlPh@Next{%
          \AlPh@BinSearch#2!#1!{#4}%
        }%
      \else
        \def\AlPh@Next{%
          \AlPh@BinSearch#1!#3!{#4}%
        }%
      \fi
    \else
      \def\AlPh@Number{#2}%
    \fi
  \else
    \def\AlPh@Number{#2}%
  \fi
  \AlPh@Next
}
%    \end{macrocode}
%    \end{macro}
%
% \subsection{Methods}
%
%    The names of method macros start with \cs{AlPh@Method}.
%    These macros do the main job in converting a number to
%    its representation.
%    A method command is called with three arguments.
%    The first argument is the number of symbols. The
%    second argument is the basic macro for converting
%    a number with limited number range. The last parameter
%    is the number that needs converting.
%
% \subsubsection{Common methods}
%
%    \begin{macro}{\AlPh@CheckPositive}
%    \noindent
%    |#1|: number to be checked
%    |#2|: continuation macro\\
%    |#3|: number of symbols (hidden here)\\
%    |#4|: symbol provider (hidden here)
%    \begin{macrocode}
\def\AlPh@CheckPositive#1!#2{%
  \ifnum#1<1 %
    \expandafter\@gobblefour
  \fi
  #2{#1}%
}
%    \end{macrocode}
%    \end{macro}
%
% \subsubsection{Method `alph'}
%
%    \begin{macro}{\AlPh@Method@alph}
%    \noindent
%    |#1|: number of symbols\\
%    |#2|: symbols provider\\
%    |#3|: number to be converted
%    \begin{macrocode}
\def\AlPh@Method@alph#1#2#3{%
  \expandafter\AlPh@CheckPositive
|   \number#3!%
+   \the\numexpr#3!%
    \AlPh@ProcessAlph
    {#1}{#2}%
}
%    \end{macrocode}
%    \end{macro}
%    \begin{macro}{\AlPh@ProcessAlph}
%    |#1|: current number\\
%    |#2|: number of symbols\\
%    |#3|: symbols provider
%    \begin{macrocode}
\def\AlPh@ProcessAlph#1#2#3{%
  \ifnum#1>#2 %
    \@ReturnAfterElseFi{%
      \expandafter\AlPh@StepAlph\number
        \intcalcInc{%
          \intcalcMod{\intcalcDec{#1}}{#2}%
        }%
      \expandafter!\number
        \intcalcDiv{\intcalcDec{#1}}{#2}%
      !{#2}{#3}%
    }%
  \else
    \@ReturnAfterFi{%
      #3{#1}%
    }%
  \fi
}
%    \end{macrocode}
%    \end{macro}
%    \begin{macro}{\AlPh@StepAlph}
%    |#1|: current last digit\\
%    |#2|: new current number\\
%    |#3|: number of symbols\\
%    |#4|: symbols provider
%    \begin{macrocode}
\def\AlPh@StepAlph#1!#2!#3#4{%
  \AlPh@ProcessAlph{#2}{#3}{#4}%
  #4{#1}%
}
%    \end{macrocode}
%    \end{macro}
%
% \subsubsection{Method `wrap'}
%
%    \begin{macro}{\AlPh@Method@wrap}
%    \noindent
%    |#1|: number of symbols\\
%    |#2|: symbols provider\\
%    |#3|: number to be converted
%    \begin{macrocode}
\def\AlPh@Method@wrap#1#2#3{%
  \expandafter\AlPh@CheckPositive
|   \number#3!%
+   \the\numexpr#3!%
    \AlPh@ProcessWrap
    {#1}{#2}%
}
%    \end{macrocode}
%    \end{macro}
%
%    \begin{macro}{\AlPh@ProcessWrap}
%    |#1|: number to be converted\\
%    |#2|: number of symbols\\
%    |#3|: symbols provider
%    \begin{macrocode}
\def\AlPh@ProcessWrap#1#2#3{%
  \ifnum#1>#2 %
    \@ReturnAfterElseFi{%
      \expandafter\AlPh@StepWrap\number
        \intcalcInc{\intcalcMod{\intcalcDec{#1}}{#2}}%
      !{#3}%
    }%
  \else
    \@ReturnAfterFi{%
      #3{#1}%
    }%
  \fi
}
%    \end{macrocode}
%    \end{macro}
%    \begin{macro}{\AlPh@StepWrap}
%    |#1|: final number\\
%    |#2|: symbols provider
%    \begin{macrocode}
\def\AlPh@StepWrap#1!#2{%
  #2{#1}%
}
%    \end{macrocode}
%    \end{macro}
%
% \subsubsection{Method `mult'}
%
%    After the number of symbols is exhausted,
%    repetitions of the symbol are used.
%    \begin{gather*}
%      x \Def \text{number to be converted}\\
%      n \Def \text{number of symbols}\\
%      r \Def \text{repetition length}\\
%      s \Def \text{symbol slot}\\
%      r = ((x - 1) \div n) + 1\\
%      s = ((x - 1) \mod n) + 1
%    \end{gather*}
%    \begin{macro}{\AlPh@Method@mult}
%    \noindent
%    |#1|: number of symbols\\
%    |#2|: symbols provider\\
%    |#3|: number to be converted
%    \begin{macrocode}
\def\AlPh@Method@mult#1#2#3{%
  \expandafter\AlPh@CheckPositive
|   \number#3!%
+   \the\numexpr#3!%
    \AlPh@ProcessMult
    {#1}{#2}%
}
%    \end{macrocode}
%    \end{macro}
%
%    \begin{macro}{\AlPh@ProcessMult}
%    |#1|: number to be converted\\
%    |#2|: number of symbols\\
%    |#3|: symbols provider
%    \begin{macrocode}
\def\AlPh@ProcessMult#1#2#3{%
  \ifnum#1>#2 %
    \@ReturnAfterElseFi{%
      \expandafter\AlPh@StepMult\romannumeral
        \intcalcInc{\intcalcDiv{\intcalcDec{#1}}{#2}}%
        000%
      \expandafter!\number
        \intcalcInc{\intcalcMod{\intcalcDec{#1}}{#2}}%
      !{#3}%
    }%
  \else
    \@ReturnAfterFi{%
      #3{#1}%
    }%
  \fi
}
%    \end{macrocode}
%    \end{macro}
%    \begin{macro}{\AlPh@StepMult}
%    |#1#2|: repetitions coded as list of character `m'\\
%    |#3|: symbol slot\\
%    |#4|: symbols provider
%    \begin{macrocode}
\def\AlPh@StepMult#1#2!#3!#4{%
  \ifx\\#2\\%
  \else
    \@ReturnAfterFi{%
      \AlPh@StepMult#2!#3!{#4}%
    }%
  \fi
  #4{#3}%
}
%    \end{macrocode}
%    \end{macro}
%
% \subsection{User interface}
%
%    \begin{macro}{\newalphalph}
%    Macro \cs{newalphalph} had three arguments in versions below 2.0.
%    For the new method argument we use an optional argument an first
%    position.\\
%    |#1|: cmd\\
%    |[#2]|: method name: |alph| (default), |wrap|, |mult|\\ % hash-ok
%    |#3|: symbols provider\\
%    |#4|: number of symbols
%    \begin{macrocode}
\AlPh@IfDefinable\newalphalph{%
  \def\newalphalph#1{%
    \AlPh@IfOptArg{%
      \AlPh@newalphalph{#1}%
    }{%
      \AlPh@newalphalph{#1}[alph]%
    }%
  }%
}
%    \end{macrocode}
%    \end{macro}
%    \begin{macro}{\AlPh@newalphalph}
%    |#1|: cmd
%    |#2|: method name\\
%    |#3|: symbols provider\\
%    |#4|: number of symbols
%    \begin{macrocode}
\def\AlPh@newalphalph#1[#2]#3#4{%
  \begingroup\expandafter\expandafter\expandafter\endgroup
  \expandafter\ifx\csname AlPh@Method@#2\endcsname\relax
    \AlPh@Error{%
      Unknown method %
|     `#2'%
+     `\detokenize{#2}'%
    }%
  \else
    \ifx\\#4\\%
      \AlPh@GetNumberOfSymbols{#3}%
      \ifcase\AlPh@Number
      \else
        \begingroup
          \escapechar=92 % backslash
          \@PackageInfo{alphalph}{%
            Number of symbols for \string#1 is \AlPh@Number
          }%
        \endgroup
        \expandafter\AlPh@NewAlphAlph
        \csname AlPh@Method@#2\expandafter\endcsname
        \AlPh@Number!{#1}{#3}%
      \fi
    \else
      \expandafter\AlPh@NewAlphAlph
      \csname AlPh@Method@#2\expandafter\endcsname
|     \number#4!%
+     \the\numexpr#4!%
      {#1}{#3}%
    \fi
  \fi
}%
%    \end{macrocode}
%    \end{macro}
%    \begin{macro}{\AlPh@NewAlphAlph}
%    |#1|: method macro\\
%    |#2|: number of symbols\\
%    |#3|: cmd\\
%    |#4|: symbols provider
%    \begin{macrocode}
\def\AlPh@NewAlphAlph#1#2!#3#4{%
  \AlPh@IfDefinable#3{%
    \ifnum#2>0 %
      \def#3{#1{#2}{#4}}%
    \else
      \AlPh@Error{%
          Definition of \string#3 failed,\MessageBreak
          because number of symbols (#2) is not positive%
        }%
    \fi
  }%
}
%    \end{macrocode}
%    \end{macro}
%
%    \begin{macro}{\AlphAlph}
%    \begin{macrocode}
\newalphalph\AlphAlph\alphalph@Alph{26}
%    \end{macrocode}
%    \end{macro}
%    \begin{macro}{\alphalph}
%    \begin{macrocode}
\newalphalph\alphalph\alphalph@alph{26}
%    \end{macrocode}
%    \end{macro}
%
%    \begin{macrocode}
\AlPh@AtEnd%
%</package>
%    \end{macrocode}
%
% \section{Test}
%
% \subsection{Catcode checks for loading}
%
%    \begin{macrocode}
%<*test1>
%    \end{macrocode}
%    \begin{macrocode}
\catcode`\{=1 %
\catcode`\}=2 %
\catcode`\#=6 %
\catcode`\@=11 %
\expandafter\ifx\csname count@\endcsname\relax
  \countdef\count@=255 %
\fi
\expandafter\ifx\csname @gobble\endcsname\relax
  \long\def\@gobble#1{}%
\fi
\expandafter\ifx\csname @firstofone\endcsname\relax
  \long\def\@firstofone#1{#1}%
\fi
\expandafter\ifx\csname loop\endcsname\relax
  \expandafter\@firstofone
\else
  \expandafter\@gobble
\fi
{%
  \def\loop#1\repeat{%
    \def\body{#1}%
    \iterate
  }%
  \def\iterate{%
    \body
      \let\next\iterate
    \else
      \let\next\relax
    \fi
    \next
  }%
  \let\repeat=\fi
}%
\def\RestoreCatcodes{}
\count@=0 %
\loop
  \edef\RestoreCatcodes{%
    \RestoreCatcodes
    \catcode\the\count@=\the\catcode\count@\relax
  }%
\ifnum\count@<255 %
  \advance\count@ 1 %
\repeat

\def\RangeCatcodeInvalid#1#2{%
  \count@=#1\relax
  \loop
    \catcode\count@=15 %
  \ifnum\count@<#2\relax
    \advance\count@ 1 %
  \repeat
}
\def\RangeCatcodeCheck#1#2#3{%
  \count@=#1\relax
  \loop
    \ifnum#3=\catcode\count@
    \else
      \errmessage{%
        Character \the\count@\space
        with wrong catcode \the\catcode\count@\space
        instead of \number#3%
      }%
    \fi
  \ifnum\count@<#2\relax
    \advance\count@ 1 %
  \repeat
}
\def\space{ }
\expandafter\ifx\csname LoadCommand\endcsname\relax
  \def\LoadCommand{\input alphalph.sty\relax}%
\fi
\def\Test{%
  \RangeCatcodeInvalid{0}{47}%
  \RangeCatcodeInvalid{58}{64}%
  \RangeCatcodeInvalid{91}{96}%
  \RangeCatcodeInvalid{123}{255}%
  \catcode`\@=12 %
  \catcode`\\=0 %
  \catcode`\%=14 %
  \LoadCommand
  \RangeCatcodeCheck{0}{36}{15}%
  \RangeCatcodeCheck{37}{37}{14}%
  \RangeCatcodeCheck{38}{47}{15}%
  \RangeCatcodeCheck{48}{57}{12}%
  \RangeCatcodeCheck{58}{63}{15}%
  \RangeCatcodeCheck{64}{64}{12}%
  \RangeCatcodeCheck{65}{90}{11}%
  \RangeCatcodeCheck{91}{91}{15}%
  \RangeCatcodeCheck{92}{92}{0}%
  \RangeCatcodeCheck{93}{96}{15}%
  \RangeCatcodeCheck{97}{122}{11}%
  \RangeCatcodeCheck{123}{255}{15}%
  \RestoreCatcodes
}
\Test
\csname @@end\endcsname
\end
%    \end{macrocode}
%    \begin{macrocode}
%</test1>
%    \end{macrocode}
%
% \section{Macro tests}
%
%    \begin{macrocode}
%<*test2>
\NeedsTeXFormat{LaTeX2e}
\nofiles
\documentclass{article}
%<*noetex>
\makeatletter
\let\saved@numexpr\numexpr
\newcommand*{\DisableNumexpr}{%
  \let\numexpr\@undefined
}
\newcommand*{\RestoreNumexpr}{%
  \let\numexpr\saved@numexpr
}
\DisableNumexpr
%</noetex>
\usepackage{alphalph}[2016/05/16]
%<noetex>\RestoreNumexpr
\usepackage{qstest}
\IncludeTests{*}
\LogTests{log}{*}{*}

\newcommand*{\TestCmd}[3]{%
  \setbox0=\hbox{%
%<noetex>    \DisableNumexpr
    \edef\TestString{#1{#2}}%
    \expandafter\Expect\expandafter{\TestString}{#3}%
    \edef\TestString{#1{#2} }%
    \expandafter\Expect\expandafter{\TestString}{#3 }%
  }%
  \Expect*{\the\wd0}{0.0pt}%
}

\makeatletter
\newalphalph\LaTeXAlphAlph\@Alph{26}
\newalphalph\LaTeXalphalph\@alph{26}
\newalphalph\AlphWrap[wrap]\alphalph@Alph{26}
\newalphalph\alphwrap[wrap]\alphalph@alph{26}
\newalphalph\LaTeXAlphWrap[wrap]\@Alph{26}
\newalphalph\LaTeXalphwrap[wrap]\@alph{26}
\def\LastSymbol#1{%
  \ifx\\#1\\%
  \else
    \@LastSymbol#1\@nil
  \fi
}
\def\@LastSymbol#1#2\@nil{%
  \ifx\\#2\\%
    #1%
  \else
    \@LastSymbol#2\@nil
  \fi
}
\makeatother
\newcommand*{\TestAlph}[2]{%
  \uppercase{\TestCallCmd\AlphAlph{#2}}{#1}%
  \lowercase{\TestCallCmd\alphalph{#2}}{#1}%
  \uppercase{\TestCallCmd\LaTeXAlphAlph{#2}}{#1}%
  \lowercase{\TestCallCmd\LaTeXalphalph{#2}}{#1}%
  \edef\WrapString{\LastSymbol{#2}}%
  \expandafter\TestAlphWrap\expandafter{\WrapString}{#1}%
}
\newcommand*{\TestAlphWrap}[2]{%
  \uppercase{\TestCallCmd\AlphWrap{#1}}{#2}%
  \lowercase{\TestCallCmd\alphwrap{#1}}{#2}%
  \uppercase{\TestCallCmd\LaTeXAlphWrap{#1}}{#2}%
  \lowercase{\TestCallCmd\LaTeXalphwrap{#1}}{#2}%
}
\newcommand*{\TestCallCmd}[3]{%
  \TestCmd#1{#3}{#2}%
}
\begin{qstest}{AlphSymbols}{alphalph, AlphAlph, symbols}
  \TestAlph{1}{a}%
  \TestAlph{2}{b}%
  \TestAlph{3}{c}%
  \TestAlph{4}{d}%
  \TestAlph{5}{e}%
  \TestAlph{6}{f}%
  \TestAlph{7}{g}%
  \TestAlph{8}{h}%
  \TestAlph{9}{i}%
  \TestAlph{10}{j}%
  \TestAlph{11}{k}%
  \TestAlph{12}{l}%
  \TestAlph{13}{m}%
  \TestAlph{14}{n}%
  \TestAlph{15}{o}%
  \TestAlph{16}{p}%
  \TestAlph{17}{q}%
  \TestAlph{18}{r}%
  \TestAlph{19}{s}%
  \TestAlph{20}{t}%
  \TestAlph{21}{u}%
  \TestAlph{22}{v}%
  \TestAlph{23}{w}%
  \TestAlph{24}{x}%
  \TestAlph{25}{y}%
  \TestAlph{26}{z}%
\end{qstest}
\begin{qstest}{AlphRange}{alphalph, range}
  \TestAlph{0}{}%
  \TestAlph{-1}{}%
  \TestAlph{-2147483647}{}%
  \TestAlph{27}{aa}%
  \TestAlph{28}{ab}%
  \TestAlph{52}{az}%
  \TestAlph{53}{ba}%
  \TestAlph{78}{bz}%
  \TestAlph{79}{ca}%
  \TestAlph{702}{zz}%
  \TestAlph{703}{aaa}%
  \TestAlph{2147483647}{fxshrxw}%
\end{qstest}

\makeatletter
\newcommand*{\myvocals}[1]{%
  \ifcase#1X\or A\or E\or I\or O\or U\else Y\fi
}
\makeatother
\newalphalph\vocalsvocals\myvocals{5}
\newcommand*{\TestVocals}{%
  \TestCmd\vocalsvocals
}
\begin{qstest}{vocals}{vocals}
  \TestVocals{0}{}%
  \TestVocals{1}{A}%
  \TestVocals{2}{E}%
  \TestVocals{3}{I}%
  \TestVocals{4}{O}%
  \TestVocals{5}{U}%
  \TestVocals{6}{AA}%
  \TestVocals{7}{AE}%
  \TestVocals{8}{AI}%
  \TestVocals{9}{AO}%
  \TestVocals{10}{AU}%
  \TestVocals{11}{EA}%
  \TestVocals{24}{OO}%
  \TestVocals{25}{OU}%
  \TestVocals{26}{UA}%
  \TestVocals{29}{UO}%
  \TestVocals{30}{UU}%
  \TestVocals{31}{AAA}%
  \TestVocals{155}{UUU}%
  \TestVocals{156}{AAAA}%
  \TestVocals{2147483647}{AIIOOEEIOIIUOE}%
\end{qstest}

\makeatletter
\newalphalph\AlphMult[mult]{\alphalph@Alph}{26}
\newalphalph\alphmult[mult]{\alphalph@alph}{26}
\newalphalph\LaTeXAlphMult[mult]{\@Alph}{26}
\newalphalph\LaTeXalphmult[mult]{\@alph}{26}
\makeatother
\newcommand*{\TestMult}[2]{%
  \uppercase{\TestCallCmd\AlphMult{#2}}{#1}%
  \lowercase{\TestCallCmd\alphmult{#2}}{#1}%
  \uppercase{\TestCallCmd\LaTeXAlphMult{#2}}{#1}%
  \lowercase{\TestCallCmd\LaTeXalphmult{#2}}{#1}%
}
\begin{qstest}{mult}{mult}
  \TestMult{0}{}%
  \TestMult{-1}{}%
  \TestMult{-2147483647}{}%
  \TestMult{1}{a}%
  \TestMult{2}{b}%
  \TestMult{26}{z}%
  \TestMult{27}{aa}%
  \TestMult{28}{bb}%
  \TestMult{52}{zz}%
  \TestMult{53}{aaa}%
  \TestMult{54}{bbb}%
  \TestMult{259}{yyyyyyyyyy}%
  \TestMult{260}{zzzzzzzzzz}%
  \TestMult{261}{aaaaaaaaaaa}%
  \TestMult{262}{bbbbbbbbbbb}%
\end{qstest}

\def\myvocalsB#1{%
  \ifcase#1\or A\or E\or I\or O\or U\fi
}
\begin{qstest}{symbolnum}{symbolnum}
  \makeatletter
  \def\Test#1#2{%
    \let\TestCmd\relax
    \newalphalph\TestCmd{#1}{}%
    \Expect*{\AlPh@Number}{#2}%
  }%
  \Test\@alph{26}%
  \Test\@Alph{26}%
  \Test\@fnsymbol{9}%
  \Test\myvocalsB{5}%
  \Test\alphalph@alph{26}%
  \Test\alphalph@Alph{26}%
\end{qstest}

\begin{qstest}{list}{list}
  \makeatletter
  \def\catch#1\relax{%
    \def\FoundList{\catch#1}%
  }%
  \def\Test[#1]#2#3#4{%
    \let\testcmd\relax
    \newalphalph\testcmd[{#1}]{\catch}{#2}%
    \testcmd{#3}|\relax
    \expandafter\Expect\expandafter{\FoundList}{#4|}%
    %
    \let\SavedCatch\catch
    \def\catch{\noexpand\catch\noexpand\foo}%
    \edef\Result{#4|}%
    \@onelevel@sanitize\Result
    \let\catch\SavedCatch
    \let\testcmd\relax
    \newalphalph\testcmd[{#1}]{\catch\foo}{#2}%
    \testcmd{#3}|\relax
    \@onelevel@sanitize\FoundList
    \Expect*{\FoundList}*{\Result}%
  }%
  \Test[alph]{26}{3}{\catch{3}}%
  \Test[alph]{26}{12}{\catch{12}}%
  \Test[alph]{26}{27}{\catch{1}\catch{1}}%
  \Test[alph]{26}{78}{\catch{2}\catch{26}}%
  \Test[wrap]{26}{7}{\catch{7}}%
  \Test[wrap]{26}{14}{\catch{14}}%
  \Test[wrap]{26}{80}{\catch{2}}%
  \Test[wrap]{26}{700}{\catch{24}}%
  \Test[mult]{26}{4}{\catch{4}}%
  \Test[mult]{26}{17}{\catch{17}}%
  \Test[mult]{26}{54}{\catch{2}\catch{2}\catch{2}}%
\end{qstest}

\begin{document}
\end{document}
%</test2>
%    \end{macrocode}
%
% \section{Installation}
%
% \subsection{Download}
%
% \paragraph{Package.} This package is available on
% CTAN\footnote{\CTANpkg{alphalph}}:
% \begin{description}
% \item[\CTAN{macros/latex/contrib/oberdiek/alphalph.dtx}] The source file.
% \item[\CTAN{macros/latex/contrib/oberdiek/alphalph.pdf}] Documentation.
% \end{description}
%
%
% \paragraph{Bundle.} All the packages of the bundle `oberdiek'
% are also available in a TDS compliant ZIP archive. There
% the packages are already unpacked and the documentation files
% are generated. The files and directories obey the TDS standard.
% \begin{description}
% \item[\CTANinstall{install/macros/latex/contrib/oberdiek.tds.zip}]
% \end{description}
% \emph{TDS} refers to the standard ``A Directory Structure
% for \TeX\ Files'' (\CTAN{tds/tds.pdf}). Directories
% with \xfile{texmf} in their name are usually organized this way.
%
% \subsection{Bundle installation}
%
% \paragraph{Unpacking.} Unpack the \xfile{oberdiek.tds.zip} in the
% TDS tree (also known as \xfile{texmf} tree) of your choice.
% Example (linux):
% \begin{quote}
%   |unzip oberdiek.tds.zip -d ~/texmf|
% \end{quote}
%
% \paragraph{Script installation.}
% Check the directory \xfile{TDS:scripts/oberdiek/} for
% scripts that need further installation steps.
% Package \xpackage{attachfile2} comes with the Perl script
% \xfile{pdfatfi.pl} that should be installed in such a way
% that it can be called as \texttt{pdfatfi}.
% Example (linux):
% \begin{quote}
%   |chmod +x scripts/oberdiek/pdfatfi.pl|\\
%   |cp scripts/oberdiek/pdfatfi.pl /usr/local/bin/|
% \end{quote}
%
% \subsection{Package installation}
%
% \paragraph{Unpacking.} The \xfile{.dtx} file is a self-extracting
% \docstrip\ archive. The files are extracted by running the
% \xfile{.dtx} through \plainTeX:
% \begin{quote}
%   \verb|tex alphalph.dtx|
% \end{quote}
%
% \paragraph{TDS.} Now the different files must be moved into
% the different directories in your installation TDS tree
% (also known as \xfile{texmf} tree):
% \begin{quote}
% \def\t{^^A
% \begin{tabular}{@{}>{\ttfamily}l@{ $\rightarrow$ }>{\ttfamily}l@{}}
%   alphalph.sty & tex/generic/oberdiek/alphalph.sty\\
%   alphalph.pdf & doc/latex/oberdiek/alphalph.pdf\\
%   test/alphalph-test1.tex & doc/latex/oberdiek/test/alphalph-test1.tex\\
%   test/alphalph-test2.tex & doc/latex/oberdiek/test/alphalph-test2.tex\\
%   test/alphalph-test3.tex & doc/latex/oberdiek/test/alphalph-test3.tex\\
%   alphalph.dtx & source/latex/oberdiek/alphalph.dtx\\
% \end{tabular}^^A
% }^^A
% \sbox0{\t}^^A
% \ifdim\wd0>\linewidth
%   \begingroup
%     \advance\linewidth by\leftmargin
%     \advance\linewidth by\rightmargin
%   \edef\x{\endgroup
%     \def\noexpand\lw{\the\linewidth}^^A
%   }\x
%   \def\lwbox{^^A
%     \leavevmode
%     \hbox to \linewidth{^^A
%       \kern-\leftmargin\relax
%       \hss
%       \usebox0
%       \hss
%       \kern-\rightmargin\relax
%     }^^A
%   }^^A
%   \ifdim\wd0>\lw
%     \sbox0{\small\t}^^A
%     \ifdim\wd0>\linewidth
%       \ifdim\wd0>\lw
%         \sbox0{\footnotesize\t}^^A
%         \ifdim\wd0>\linewidth
%           \ifdim\wd0>\lw
%             \sbox0{\scriptsize\t}^^A
%             \ifdim\wd0>\linewidth
%               \ifdim\wd0>\lw
%                 \sbox0{\tiny\t}^^A
%                 \ifdim\wd0>\linewidth
%                   \lwbox
%                 \else
%                   \usebox0
%                 \fi
%               \else
%                 \lwbox
%               \fi
%             \else
%               \usebox0
%             \fi
%           \else
%             \lwbox
%           \fi
%         \else
%           \usebox0
%         \fi
%       \else
%         \lwbox
%       \fi
%     \else
%       \usebox0
%     \fi
%   \else
%     \lwbox
%   \fi
% \else
%   \usebox0
% \fi
% \end{quote}
% If you have a \xfile{docstrip.cfg} that configures and enables \docstrip's
% TDS installing feature, then some files can already be in the right
% place, see the documentation of \docstrip.
%
% \subsection{Refresh file name databases}
%
% If your \TeX~distribution
% (\teTeX, \mikTeX, \dots) relies on file name databases, you must refresh
% these. For example, \teTeX\ users run \verb|texhash| or
% \verb|mktexlsr|.
%
% \subsection{Some details for the interested}
%
% \paragraph{Attached source.}
%
% The PDF documentation on CTAN also includes the
% \xfile{.dtx} source file. It can be extracted by
% AcrobatReader 6 or higher. Another option is \textsf{pdftk},
% e.g. unpack the file into the current directory:
% \begin{quote}
%   \verb|pdftk alphalph.pdf unpack_files output .|
% \end{quote}
%
% \paragraph{Unpacking with \LaTeX.}
% The \xfile{.dtx} chooses its action depending on the format:
% \begin{description}
% \item[\plainTeX:] Run \docstrip\ and extract the files.
% \item[\LaTeX:] Generate the documentation.
% \end{description}
% If you insist on using \LaTeX\ for \docstrip\ (really,
% \docstrip\ does not need \LaTeX), then inform the autodetect routine
% about your intention:
% \begin{quote}
%   \verb|latex \let\install=y% \iffalse meta-comment
%
% File: alphalph.dtx
% Version: 2016/05/16 v2.5
% Info: Convert numbers to letters
%
% Copyright (C) 1999, 2006-2008, 2010, 2011 by
%    Heiko Oberdiek <heiko.oberdiek at googlemail.com>
%    2016
%    https://github.com/ho-tex/oberdiek/issues
%
% This work may be distributed and/or modified under the
% conditions of the LaTeX Project Public License, either
% version 1.3c of this license or (at your option) any later
% version. This version of this license is in
%    http://www.latex-project.org/lppl/lppl-1-3c.txt
% and the latest version of this license is in
%    http://www.latex-project.org/lppl.txt
% and version 1.3 or later is part of all distributions of
% LaTeX version 2005/12/01 or later.
%
% This work has the LPPL maintenance status "maintained".
%
% This Current Maintainer of this work is Heiko Oberdiek.
%
% The Base Interpreter refers to any `TeX-Format',
% because some files are installed in TDS:tex/generic//.
%
% This work consists of the main source file alphalph.dtx
% and the derived files
%    alphalph.sty, alphalph.pdf, alphalph.ins, alphalph.drv,
%    alphalph-test1.tex, alphalph-test2.tex, alphalph-test3.tex.
%
% Distribution:
%    CTAN:macros/latex/contrib/oberdiek/alphalph.dtx
%    CTAN:macros/latex/contrib/oberdiek/alphalph.pdf
%
% Unpacking:
%    (a) If alphalph.ins is present:
%           tex alphalph.ins
%    (b) Without alphalph.ins:
%           tex alphalph.dtx
%    (c) If you insist on using LaTeX
%           latex \let\install=y\input{alphalph.dtx}
%        (quote the arguments according to the demands of your shell)
%
% Documentation:
%    (a) If alphalph.drv is present:
%           latex alphalph.drv
%    (b) Without alphalph.drv:
%           latex alphalph.dtx; ...
%    The class ltxdoc loads the configuration file ltxdoc.cfg
%    if available. Here you can specify further options, e.g.
%    use A4 as paper format:
%       \PassOptionsToClass{a4paper}{article}
%
%    Programm calls to get the documentation (example):
%       pdflatex alphalph.dtx
%       makeindex -s gind.ist alphalph.idx
%       pdflatex alphalph.dtx
%       makeindex -s gind.ist alphalph.idx
%       pdflatex alphalph.dtx
%
% Installation:
%    TDS:tex/generic/oberdiek/alphalph.sty
%    TDS:doc/latex/oberdiek/alphalph.pdf
%    TDS:doc/latex/oberdiek/test/alphalph-test1.tex
%    TDS:doc/latex/oberdiek/test/alphalph-test2.tex
%    TDS:doc/latex/oberdiek/test/alphalph-test3.tex
%    TDS:source/latex/oberdiek/alphalph.dtx
%
%<*ignore>
\begingroup
  \catcode123=1 %
  \catcode125=2 %
  \def\x{LaTeX2e}%
\expandafter\endgroup
\ifcase 0\ifx\install y1\fi\expandafter
         \ifx\csname processbatchFile\endcsname\relax\else1\fi
         \ifx\fmtname\x\else 1\fi\relax
\else\csname fi\endcsname
%</ignore>
%<*install>
\input docstrip.tex
\Msg{************************************************************************}
\Msg{* Installation}
\Msg{* Package: alphalph 2016/05/16 v2.5 Convert numbers to letters (HO)}
\Msg{************************************************************************}

\keepsilent
\askforoverwritefalse

\let\MetaPrefix\relax
\preamble

This is a generated file.

Project: alphalph
Version: 2016/05/16 v2.5

Copyright (C) 1999, 2006-2008, 2010, 2011 by
   Heiko Oberdiek <heiko.oberdiek at googlemail.com>

This work may be distributed and/or modified under the
conditions of the LaTeX Project Public License, either
version 1.3c of this license or (at your option) any later
version. This version of this license is in
   http://www.latex-project.org/lppl/lppl-1-3c.txt
and the latest version of this license is in
   http://www.latex-project.org/lppl.txt
and version 1.3 or later is part of all distributions of
LaTeX version 2005/12/01 or later.

This work has the LPPL maintenance status "maintained".

This Current Maintainer of this work is Heiko Oberdiek.

The Base Interpreter refers to any `TeX-Format',
because some files are installed in TDS:tex/generic//.

This work consists of the main source file alphalph.dtx
and the derived files
   alphalph.sty, alphalph.pdf, alphalph.ins, alphalph.drv,
   alphalph-test1.tex, alphalph-test2.tex, alphalph-test3.tex.

\endpreamble
\let\MetaPrefix\DoubleperCent

\generate{%
  \file{alphalph.ins}{\from{alphalph.dtx}{install}}%
  \file{alphalph.drv}{\from{alphalph.dtx}{driver}}%
  \usedir{tex/generic/oberdiek}%
  \file{alphalph.sty}{\from{alphalph.dtx}{package}}%
%  \usedir{doc/latex/oberdiek/test}%
%  \file{alphalph-test1.tex}{\from{alphalph.dtx}{test1}}%
%  \file{alphalph-test2.tex}{\from{alphalph.dtx}{test2,etex}}%
%  \file{alphalph-test3.tex}{\from{alphalph.dtx}{test2,noetex}}%
  \nopreamble
  \nopostamble
%  \usedir{source/latex/oberdiek/catalogue}%
%  \file{alphalph.xml}{\from{alphalph.dtx}{catalogue}}%
}

\catcode32=13\relax% active space
\let =\space%
\Msg{************************************************************************}
\Msg{*}
\Msg{* To finish the installation you have to move the following}
\Msg{* file into a directory searched by TeX:}
\Msg{*}
\Msg{*     alphalph.sty}
\Msg{*}
\Msg{* To produce the documentation run the file `alphalph.drv'}
\Msg{* through LaTeX.}
\Msg{*}
\Msg{* Happy TeXing!}
\Msg{*}
\Msg{************************************************************************}

\endbatchfile
%</install>
%<*ignore>
\fi
%</ignore>
%<*driver>
\NeedsTeXFormat{LaTeX2e}
\ProvidesFile{alphalph.drv}%
  [2016/05/16 v2.5 Convert numbers to letters (HO)]%
\documentclass{ltxdoc}
\usepackage{alphalph}[2016/05/16]
\makeatletter
\newalphalph{\fnsymbolwrap}[wrap]{\@fnsymbol}{}
\newcommand*{\fnsymbolsingle}[1]{%
  \ensuremath{%
    \ifcase#1%
    \or *%
    \or \dagger
    \or \ddagger
    \or \mathsection
    \or \mathparagraph
    \else
      \@ctrerr
    \fi
  }%
}
\makeatother
\newalphalph{\fnsymbolmult}[mult]{\fnsymbolsingle}{}
\usepackage{amsmath}
\DeclareMathOperator{\opDiv}{div}
\DeclareMathOperator{\opMod}{mod}
\newcommand*{\Def}{%
  \ensuremath{%
    \mathrel{\mathop{:}}=%
  }%
}
\usepackage{holtxdoc}[2011/11/22]
\begin{document}
  \DocInput{alphalph.dtx}%
\end{document}
%</driver>
% \fi
%
%
% \CharacterTable
%  {Upper-case    \A\B\C\D\E\F\G\H\I\J\K\L\M\N\O\P\Q\R\S\T\U\V\W\X\Y\Z
%   Lower-case    \a\b\c\d\e\f\g\h\i\j\k\l\m\n\o\p\q\r\s\t\u\v\w\x\y\z
%   Digits        \0\1\2\3\4\5\6\7\8\9
%   Exclamation   \!     Double quote  \"     Hash (number) \#
%   Dollar        \$     Percent       \%     Ampersand     \&
%   Acute accent  \'     Left paren    \(     Right paren   \)
%   Asterisk      \*     Plus          \+     Comma         \,
%   Minus         \-     Point         \.     Solidus       \/
%   Colon         \:     Semicolon     \;     Less than     \<
%   Equals        \=     Greater than  \>     Question mark \?
%   Commercial at \@     Left bracket  \[     Backslash     \\
%   Right bracket \]     Circumflex    \^     Underscore    \_
%   Grave accent  \`     Left brace    \{     Vertical bar  \|
%   Right brace   \}     Tilde         \~}
%
% \GetFileInfo{alphalph.drv}
%
% \title{The \xpackage{alphalph} package}
% \date{2016/05/16 v2.5}
% \author{Heiko Oberdiek\thanks
% {Please report any issues at \url{https://github.com/ho-tex/oberdiek/issues}}\\
% \xemail{heiko.oberdiek at googlemail.com}}
%
% \maketitle
%
% \begin{abstract}
% The package provides methods to represent numbers with a limited
% set of symbols. Both \hologo{LaTeX} and \hologo{plainTeX} are supported.
% \end{abstract}
%
% \tableofcontents
%
% \section{Documentation}
%
% \subsection{Introduction}
% \hologo{LaTeX} counters can be represented in different ways
% by using presentation commands:
% \begin{quote}
%   \cs{arabic}, \cs{roman}, \cs{Roman},\\
%   \cs{alph}, \cs{Alph}, \cs{fnsymbol}
% \end{quote}
% The ranges of supported counter values are more or less
% restricted. Only \cs{arabic} can be used with any counter
% value \hologo{TeX} supports.
% \begin{quote}
% \catcode`\|=12 %
% \begin{tabular}{@{}l|l|l|l@{}}
% Presentation & Supported & Ignored & Error message\\
% command      & domain    & values  & ``Counter too large''\\
% \hline
% \cs{arabic}
%   & \ttfamily -MAX..MAX & &\\
% \cs{roman}, \cs{Roman}
%   & \ttfamily 1..MAX & \ttfamily -MAX..0 &\\
% \cs{alph}, \cs{Alph}
%   & \ttfamily 1..26 & 0 & \ttfamily -MAX..-1, 27..MAX\\
% \cs{fnsymbol}
%   & \ttfamily 1..9 & 0 & \ttfamily -MAX..-1, 10..MAX\\
% \hline
% \end{tabular}\\[1ex]
% \texttt{MAX} = \texttt{2147483647}
% \end{quote}
% Ordinal numbers are often used in documents: numbering
% of chapters, sections, figures, footnotes and so on.
% The layouter chooses \cs{Alph} for chapter numbers
% and \cs{fnsymbol} for footnotes. But what can be done
% if there are more than 26 chapters or more than 10 footnotes?
% This package \xpackage{alphalph} allows to define new
% presentation commands. They rely on a existing command
% and define presentations for values greater the limits.
% Three different methods are provided by the package.
% In the following use cases they are presentated.
%
% \subsection{Use cases}
%
% \subsubsection{Number system based on symbols}
%
% Asume you are writing a book and your lecturer demands
% that chapter numbers must be letters. But you have already
% 30 chapters and you have only 26 letters?
%
% In the decimal system the situation would be clear.
% If you run out of digits, you are using more digits
% to represent a number. This method can be also be used
% for letters. After chapter 26 with |Z| we us |AA|, |AB|, |AC|, and |AD|
% for the remaining chapters.
%
% Happily this package already defines this presentation command:
%\begin{quote}
%\begin{verbatim}
%\usepackage{alphalph}
%\renewcommand*{\thechapter}{%
%  \AlphAlph{\value{chapter}}%
%}
%\end{verbatim}
%\end{quote}
% \cs{AlphAlph} generates:
%   \AlphAlph{1}, \AlphAlph{2}, \AlphAlph{3}, \dots,
%   \AlphAlph{26}, \AlphAlph{27}, \AlphAlph{28}, \dots
%
% The other presentation command is \cs{alphalph} for lowercase letters.
%
% \subsubsection{Wrap symbols around}
%
% Nine footnote symbols are quite a few.
% Too soon the symbols are consumed
% and \hologo{LaTeX} complains with the error ``Counter too large''.
% However, it could be acceptable to start again with the symbols
% from the beginning, especially if there are less than nine symbols
% on a page.
% This could be achieved by a counter reset.
% But finding the right place can be difficult or needs manual
% actions. Also a
% unique counter value can be desirable (e.g. for generating
% unique anchor/link names).
% Package \xpackage{alphalph} allows you to define
% a macro that implements a ``wrap around'', but letting
% the value of the counter untouched:
%\begin{quote}
%\begin{verbatim}
%\usepackage{alphalph}
%\makeatletter
%\newalphalph{\fnsymbolwrap}[wrap]{\@fnsymbol}{}
%\makeatother
%\renewcommand*{\thefootnote}{%
%  \fnsymbolwrap{\value{footnote}}%
%}
%\end{verbatim}
%\end{quote}
% \cs{fnsymbolwrap} generates:
% \fnsymbolwrap{1}~(1), \fnsymbolwrap{2}~(2), \fnsymbolwrap{3}~(3), \dots,
% \fnsymbolwrap{9}~(9), \fnsymbolwrap{10}~(10), \fnsymbolwrap{11}~11, \dots
%
% \subsubsection{Multiple symbols}
%
% \Hologo{LaTeX}'s standard set of footnote symbols contains doubled
% symbols at the higher positions. Could this principle
% be generalized?
% Yes, but first we need a clean footnote symbol list without
% doubled entries, example:
%\begin{quote}
%\begin{verbatim}
%\usepackage{alphalph}
%\makeatletter
%\newcommand*{\fnsymbolsingle}[1]{%
%  \ensuremath{%
%    \ifcase#1%
%    \or *%
%    \or \dagger
%    \or \ddagger
%    \or \mathsection
%    \or \mathparagraph
%    \else
%      \@ctrerr
%    \fi
%  }%
%}
%\makeatother
%\newalphalph{\fnsymbolmult}[mult]{\fnsymbolsingle}{}
%\renewcommand*{\thefootnote}{%
%  \fnsymbolmult{\value{footnote}}%
%}
%\end{verbatim}
%\end{quote}
% The own definition of \cs{fnsymbolsingle} has the
% advantage that this list can easily modified.
% Otherwise you can use \cs{@fnsymbol} directly,
% because it uses the same first five symbols.
%\begin{quote}
%\begin{verbatim}
%\usepackage{alphalph}
%\makeatletter
%\newalphalph{\fnsymbolmult}[mult]{\@fnsymbol}{5}
%\makeatother
%\renewcommand*{\thefootnote}{%
%  \fnsymbolmult{\value{footnote}}%
%}
%\end{verbatim}
%\end{quote}
% \cs{fnsymbolmult} generates:
% \fnsymbolmult{1}~(1), \fnsymbolmult{2}~(2), \fnsymbolmult{3}~(3),
% \fnsymbolmult{4}~(4), \fnsymbolmult{5}~(5), \fnsymbolmult{6}~(6),
% \dots, \fnsymbolmult{16}~16, \fnsymbolmult{17}~17, \dots
%
% \bigskip
% The same method can also be used for the chapter problem
% in the first discussed use case:
%\begin{quote}
%\begin{verbatim}
%\usepackage{alphalph}
%\makeatletter
%\newalphalph{\AlphMult}[mult]{\@Alph}{26}
%\makeatother
%\renewcommand*{\chapter}{%
%  \AlphMult{\value{chapter}}%
%}
%\end{verbatim}
%\end{quote}
% \cs{AlphMult} then generates AA, BB, CC, and DD for chapters 27--30.
%
% \subsection{Glossary}
%
% \begin{description}
% \item[Counter presentation command] is a macro that
%   expects a \hologo{LaTeX} counter name as argument.
%   Numbers cannot be used. Examples:
%   \cs{arabic}, \cs{alph}, \cs{fnsymbol}.
% \item[Number presentation command] is a macro that
%   expects a number as argument. A number is anything
%   that \hologo{TeX} accepts as number including \cs{value}.
%   Examples: \cs{alphalph}, \cs{AlphAlph}, \cs{alphalph@alph}
%
%   However, \cs{alph} or \cs{fnsymbol} are not number
%   presentation commands because they expect a counter
%   name as argument. Happily \hologo{LaTeX} counter presentation
%   commands internally uses number presentation commands
%   with the same name, but prefixed by `@'. Thus
%   \cs{@alph}, \cs{@fnsymbol} are number presentation commands.
% \item[Symbols provider] is a command that can be used
%   to get a list of symbols. For example, \cs{@Alph}
%   provides the 26 uppercase letters from `A' to `Z'.
%   Basically a symbol provider is a number presentation
%   command, usually with a limited range.
% \item[Number of symbols] is the number of the
%   last symbol slot of a symbol provider. Thus \cs{@Alph}
%   generates 26 symbols, \cs{@fnsymbol} provides 9 symbols.
% \end{description}
%
% \subsection{Package usage}
%
% The package \Package{alphalph} can be used with both \hologo{plainTeX}
% and \hologo{LaTeX}:
% \begin{description}
% \item[\hologo{plainTeX}:] |\input alphalph.sty|
% \item[\hologo{LaTeX2e}:]   |\usepackage{alphalph}|\\
%      There aren't any options.
% \end{description}
%
% \subsection{User commands}
%
% \begin{declcs}{AlphAlph} \M{number}\\
%   \cs{alphalph} \M{number}
% \end{declcs}
% Both macros are number presentation commands that
% expects a number as argument. \hologo{LaTeX} counters
% are used with \cs{value}.
%
% The macros represents a number by letters.
% First single letters |A..Z| are used, then
% two letters |AA..ZZ|, three letters |AAA...ZZZ|, \dots
% follow.
%
% Macro \cs{AlphAlph} uses uppercase letters,
% \cs{alphalph} generates the lowercase variant.
% \begin{quote}
%   \catcode`\|=12
%   \def\entry#1{
%     \texttt{#1}
%     & \texttt{\AlphAlph{#1}}
%     & \texttt{\alphalph{#1}}
%   }
%   \begin{tabular}{@{}l|l|l@{}}
%     \meta{number} & \verb|\AlphAlph{|\meta{number}\verb|}|
%                   & \verb|\alphalph{|\meta{number}\verb|}|\\
%     \hline
%     \entry{1}\\
%     \entry{2}\\
%     \entry{26}\\
%     \entry{27}\\
%     \entry{30}\\
%     \entry{2000}\\
%     \entry{3752127}\\
%     \entry{10786572}\\
%     \entry{2147483647}\\
%   \end{tabular}
% \end{quote}
%
% \begin{declcs}{newalphalph}
%   \M{cmd}
%   |[|\meta{method}|]|
%   \M{symbols provider}
%   \M{number of symbols}
% \end{declcs}
% Macro \cs{newalphalph} defines \meta{cmd} as new
% number presentation command. Like \cs{newcommand} an error
% is thrown, if macro \meta{cmd} already exists.
%
% The \meta{method} is one of |alph|, |wrap|, or |mult|.
% The default is |alph|.
%
% As symbol provider a number presentation command
% can be used, e.g. \cs{@fnsymbol}, \cs{@Alph}, or
% \cs{alphalph@alph}.
%
% The last argument is the number of symbols.
% If the argument is empty, then \cs{newalphalph} tries
% to find this number itself. \hologo{LaTeX}'s number presentation
% commands throw an error message, if the number is too large.
% This error message is put in a macro \cs{@ctrerr}.
% Thus \cs{newalphalph} calls the symbol provider and tests
% a number by typesetting it in a temporary box. The error
% macro \cs{@ctrerr} is catched, it proofs that the number
% is not supported. Also if the width of the result is zero
% the number is considered as unavailable.
%
% The empty argument is useful for potentially variable lists.
% However if the end cannot be detected, then the number of
% symbols must be given. This is also a lot faster. Therefore don't
% let the argument empty without reason.
%
% \subsection{Programmer commands}
%
% \begin{declcs}{alphalph@Alph} \M{number}\\
%   \cs{alphalph@alph} \M{number}
% \end{declcs}
% They are basically the same as \cs{@Alph} and \cs{@alph}.
% Some languages of package \xpackage{babel} redefine
% \hologo{LaTeX}'s macros to include some font setup that
% breaks expandibility. Therefore \cs{AlphAlph} and \cs{alphalph}
% are based on \cs{alphalph@Alph} and \cs{alphalph@alph} to get
% the letters. The behaviour of these symbol providers for numbers
% outside the range |1..26| is undefined.
%
% \subsection{Design principles}
%
% \subsubsection{Number presentation commands}
%
% All number presentation commands that this package defines
% (including \cs{alphalph} and \cs{AlphAlph}) have the following
% properties:
% \begin{itemize}
% \item They are fully expandable. This means that they can safely
%       \begin{itemize}
%       \item be written to a file,
%       \item used in moving arguments (\hologo{LaTeX}: they are \emph{robust}),
%       \item used in a \cmd{\csname}-\cmd{\endcsname} pair.
%       \end{itemize}
% \item If the argument is zero or negative, the commands expand
%       to nothing like \cmd{\romannumeral}.
% \item The argument is a \hologo{TeX} number. Anything that would be
%       accepted by \cmd{\number} is a valid argument:
%       \begin{itemize}
%       \item explicite constants,
%       \item macros that expand to a number,
%       \item count registers, \hologo{LaTeX} counter can used
%             via \cmd{\value}, e.\,g.:\\
%             |\alphalph{\value{page}}|
%       \item \dots
%       \end{itemize}
% \item \hologo{eTeX}'s numeric expressions are supported, if \hologo{eTeX} is
%       available. Then \cs{numexpr} is applied to the argument.
%       Package \cs{calc}'s expressions are not supported.
%       That would violate the expandibility.
% \end{itemize}
%
% \subsubsection{General usability}
%
% \begin{description}
% \item[\hologo{TeX} format:]
%   The package does not depend on \hologo{LaTeX}, it can also be used
%   by \hologo{plainTeX}, for example.
% \item[\hologo{eTeX}:] \eTeX is supported, the macros are shorter and
%   faster. But \eTeX's extensions are not requirements.
%   Without \hologo{eTeX}, just the implementation changes. The properties
%   remain unchanged.
% \end{description}
%
% \StopEventually{
% }
%
% \section{Implementation}
% \subsection{Begin\texorpdfstring{ of package}{}}
%    \begin{macrocode}
%<*package>
%    \end{macrocode}
%    Reload check, especially if the package is not used with \LaTeX.
%    \begin{macrocode}
\begingroup\catcode61\catcode48\catcode32=10\relax%
  \catcode13=5 % ^^M
  \endlinechar=13 %
  \catcode35=6 % #
  \catcode39=12 % '
  \catcode44=12 % ,
  \catcode45=12 % -
  \catcode46=12 % .
  \catcode58=12 % :
  \catcode64=11 % @
  \catcode123=1 % {
  \catcode125=2 % }
  \expandafter\let\expandafter\x\csname ver@alphalph.sty\endcsname
  \ifx\x\relax % plain-TeX, first loading
  \else
    \def\empty{}%
    \ifx\x\empty % LaTeX, first loading,
      % variable is initialized, but \ProvidesPackage not yet seen
    \else
      \expandafter\ifx\csname PackageInfo\endcsname\relax
        \def\x#1#2{%
          \immediate\write-1{Package #1 Info: #2.}%
        }%
      \else
        \def\x#1#2{\PackageInfo{#1}{#2, stopped}}%
      \fi
      \x{alphalph}{The package is already loaded}%
      \aftergroup\endinput
    \fi
  \fi
\endgroup%
%    \end{macrocode}
%    Package identification:
%    \begin{macrocode}
\begingroup\catcode61\catcode48\catcode32=10\relax%
  \catcode13=5 % ^^M
  \endlinechar=13 %
  \catcode35=6 % #
  \catcode39=12 % '
  \catcode40=12 % (
  \catcode41=12 % )
  \catcode44=12 % ,
  \catcode45=12 % -
  \catcode46=12 % .
  \catcode47=12 % /
  \catcode58=12 % :
  \catcode64=11 % @
  \catcode91=12 % [
  \catcode93=12 % ]
  \catcode123=1 % {
  \catcode125=2 % }
  \expandafter\ifx\csname ProvidesPackage\endcsname\relax
    \def\x#1#2#3[#4]{\endgroup
      \immediate\write-1{Package: #3 #4}%
      \xdef#1{#4}%
    }%
  \else
    \def\x#1#2[#3]{\endgroup
      #2[{#3}]%
      \ifx#1\@undefined
        \xdef#1{#3}%
      \fi
      \ifx#1\relax
        \xdef#1{#3}%
      \fi
    }%
  \fi
\expandafter\x\csname ver@alphalph.sty\endcsname
\ProvidesPackage{alphalph}%
  [2016/05/16 v2.5 Convert numbers to letters (HO)]%
%    \end{macrocode}
%
% \subsection{Catcodes}
%
%    \begin{macrocode}
\begingroup\catcode61\catcode48\catcode32=10\relax%
  \catcode13=5 % ^^M
  \endlinechar=13 %
  \catcode123=1 % {
  \catcode125=2 % }
  \catcode64=11 % @
  \def\x{\endgroup
    \expandafter\edef\csname AlPh@AtEnd\endcsname{%
      \endlinechar=\the\endlinechar\relax
      \catcode13=\the\catcode13\relax
      \catcode32=\the\catcode32\relax
      \catcode35=\the\catcode35\relax
      \catcode61=\the\catcode61\relax
      \catcode64=\the\catcode64\relax
      \catcode123=\the\catcode123\relax
      \catcode125=\the\catcode125\relax
    }%
  }%
\x\catcode61\catcode48\catcode32=10\relax%
\catcode13=5 % ^^M
\endlinechar=13 %
\catcode35=6 % #
\catcode64=11 % @
\catcode123=1 % {
\catcode125=2 % }
\def\TMP@EnsureCode#1#2{%
  \edef\AlPh@AtEnd{%
    \AlPh@AtEnd
    \catcode#1=\the\catcode#1\relax
  }%
  \catcode#1=#2\relax
}
\TMP@EnsureCode{33}{12}% !
\TMP@EnsureCode{39}{12}% '
\TMP@EnsureCode{40}{12}% (
\TMP@EnsureCode{41}{12}% )
\TMP@EnsureCode{43}{12}% +
\TMP@EnsureCode{44}{12}% ,
\TMP@EnsureCode{46}{12}% .
\TMP@EnsureCode{47}{12}% /
\TMP@EnsureCode{59}{12}% ;
\TMP@EnsureCode{60}{12}% <
\TMP@EnsureCode{62}{12}% >
\TMP@EnsureCode{91}{12}% [
\TMP@EnsureCode{93}{12}% ]
\TMP@EnsureCode{96}{12}% `
\TMP@EnsureCode{124}{12}% |
\edef\AlPh@AtEnd{\AlPh@AtEnd\noexpand\endinput}
%    \end{macrocode}
%
% \subsection{Package loading}
%
%    \begin{macrocode}
\begingroup\expandafter\expandafter\expandafter\endgroup
\expandafter\ifx\csname RequirePackage\endcsname\relax
  \input infwarerr.sty\relax
  \input intcalc.sty\relax
\else
  \RequirePackage{infwarerr}[2007/09/09]%
  \RequirePackage{intcalc}[2007/09/09]%
\fi
%    \end{macrocode}
%
% \subsection{\hologo{eTeX} detection}
%
%    \begin{macrocode}
\begingroup\expandafter\expandafter\expandafter\endgroup
\expandafter\ifx\csname numexpr\endcsname\relax
  \catcode124=9 % '!': ignore
  \catcode43=14 % '+': comment
\else
  \catcode124=14 % '!': comment
  \catcode43=9   % '+': ignore
\fi
%    \end{macrocode}
%
% \subsection{Help macros}
%
%    \begin{macro}{\AlPh@Error}
%    \begin{macrocode}
\def\AlPh@Error#1{%
  \begingroup
    \escapechar=92 % backslash
    \@PackageError{alphalph}{#1}\@ehc
  \endgroup
}
%    \end{macrocode}
%    \end{macro}
%
%    \begin{macro}{\AlPh@IfDefinable}
%    \begin{macrocode}
\begingroup\expandafter\expandafter\expandafter\endgroup
\expandafter\ifx\csname @ifdefinable\endcsname\relax
  \def\AlPh@IfDefinable#1#2{%
    \ifcase\ifx#1\@undefined\else\ifx#1\relax\else1\fi\fi0 %
      #2%
    \else
      \AlPh@Error{%
        Command \string#1 already defined%
      }%
    \fi
  }%
%    \end{macrocode}
%    \end{macro}
%    \begin{macrocode}
\else
%    \end{macrocode}
%    \begin{macro}{\AlPh@IfDefinable}
%    \begin{macrocode}
  \let\AlPh@IfDefinable\@ifdefinable
%    \end{macrocode}
%    \end{macro}
%    \begin{macrocode}
\fi
%    \end{macrocode}
%
%    \begin{macro}{\@ReturnAfterElseFi}
%    \begin{macro}{\@ReturnAfterFi}
%    The following commands moves the `then' and `else' part respectively
%    behind the |\if|-construct. This prevents a too deep |\if|-nesting
%    and so a \hologo{TeX} capacity error because of a limited input stack size.
%    I use this trick in several packages, so I don't prefix these internal
%    commands in order not to have the same macros with different names.
%    (It saves memory.)
%    \begin{macrocode}
\long\def\@ReturnAfterElseFi#1\else#2\fi{\fi#1}
\long\def\@ReturnAfterFi#1\fi{\fi#1}
%    \end{macrocode}
%    \end{macro}
%    \end{macro}
%
%    \begin{macro}{\@gobblefour}
%    \Hologo{LaTeX} defines commands for eating arguments.
%    Define \cs{@gobblefour} if it is not defined (\hologo{plainTeX}).
%    \begin{macrocode}
\expandafter\ifx\csname @gobblefour\endcsname\relax
  \long\def\@gobblefour#1#2#3#4{}%
\fi
%    \end{macrocode}
%    \end{macro}
%
%    \begin{macro}{AlPh@IfOptArg}
%    \begin{macrocode}
\begingroup\expandafter\expandafter\expandafter\endgroup
\expandafter\ifx\csname kernel@ifnextchar\endcsname\relax
  \begingroup\expandafter\expandafter\expandafter\endgroup
  \expandafter\ifx\csname @ifnextchar\endcsname\relax
    \def\AlPh@IfOptArg#1#2{%
      \def\AlPh@TempA{#1}%
      \def\AlPh@TempB{#2}%
      \futurelet\AlPh@Token\AlPh@IfOptArgNext
    }%
    \let\AlPh@BracketLeft=[%]
    \def\AlPh@IfOptArgNext{%
      \ifx\AlPh@Token\AlPh@BracketLeft
        \expandafter\AlPh@TempA
      \else
        \expandafter\AlPh@TempB
      \fi
    }%
  \else
    \def\AlPh@IfOptArg{\@ifnextchar[}%]
  \fi
\else
  \def\AlPh@IfOptArg{\kernel@ifnextchar[}%]
\fi
%    \end{macrocode}
%    \end{macro}
%
% \subsection{Symbol provider}
%
% \subsubsection{Alphabet}
%
%    The output of \cs{alphalph} and \cs{AlphAlph} should be
%    usable as part of command names (see \cs{@namedef}, \cs{csname}, \dots).
%    Unhappily some languages of package \xpackage{babel}
%    redefine \hologo{LaTeX}'s \cs{@alph} and \cs{@Alph} in a manner that
%    they cannot be used in expandable context any more.
%    Therefore package \xpackage{alphalph} provides its own
%    commands.
%
%    \begin{macro}{\alphalph@Alph}
%    \begin{macro}{\alphalph@alph}
%    The two commands \cmd{\AlPh@Alph} and \cmd{\AlPh@alph} convert a
%    number into a letter (uppercase and lowercase respectivly).
%    The character |@| is used as an error symbol,
%    if the number isn't in the range of 1 until 26.
%    Here we need no space after the number |#1|, because the error
%    symbol |@| for the zero case stops scanning the number.
%    This error symbol should not appear anywhere (except for bugs).
%    \begin{macrocode}
\def\alphalph@Alph#1{%
  \ifcase#1%
    @%
  \or A\or B\or C\or D\or E\or F\or G\or H\or I\or J\or K\or L\or M%
  \or N\or O\or P\or Q\or R\or S\or T\or U\or V\or W\or X\or Y\or Z%
  \else
    \AlPh@ctrerr
    @%
  \fi
}
\def\alphalph@alph#1{%
  \ifcase#1%
    @%
  \or a\or b\or c\or d\or e\or f\or g\or h\or i\or j\or k\or l\or m%
  \or n\or o\or p\or q\or r\or s\or t\or u\or v\or w\or x\or y\or z%
  \else
    \AlPh@ctrerr
    @%
  \fi
}
%    \end{macrocode}
%    \end{macro}
%    \end{macro}
%    \begin{macro}{\AlPh@ctrerr}
%    Macro \cs{AlPh@ctrerr} is used as hook for the algorithm
%    to get the available number of symbols.
%    \begin{macrocode}
\def\AlPh@ctrerr{}
%    \end{macrocode}
%    \end{macro}
%
% \subsection{Finding number of symbols}
%
%    \begin{macro}{\AlPh@GetNumberOfSymbols}
%    \noindent
%    |#1|: symbols provider
%    \begin{macrocode}
\def\AlPh@GetNumberOfSymbols#1{%
  \AlPh@TestNumber1!{#1}%
  \ifAlPh@Unavailable
    \def\AlPh@Number{0}%
    \AlPh@Error{No symbols found}%
  \else
    \def\AlPh@Number{1}%
    \AlPh@ExpSearch2!{#1}%
  \fi
}
%    \end{macrocode}
%    \end{macro}
%    \begin{macro}{\ifAlPh@Unavailable}
%    \begin{macrocode}
\let\ifAlPh@Unavailable\iffalse
\def\AlPh@Unavailabletrue{%
  \global\let\ifAlPh@Unavailable\iftrue
}
\def\AlPh@Unavailablefalse{%
  \global\let\ifAlPh@Unavailable\iffalse
}
%    \end{macrocode}
%    \end{macro}
%    \begin{macro}{\AlPh@TestNumber}
%    |#1|: number to be tested\\
%    |#2|: symbols provider
%    \begin{macrocode}
\def\AlPh@TestNumber#1!#2{%
  \AlPh@Unavailablefalse
  \begingroup
    \setbox0=\hbox{%
      \begingroup % color
        \let\@ctrerr\AlPh@Unavailabletrue
        \let\AlPh@ctrerr\AlPh@Unavailabletrue
        #2{#1}%
      \endgroup
    }%
    \ifdim\wd0=0pt %
      \AlPh@Unavailabletrue
    \fi
  \endgroup
}
%    \end{macrocode}
%    \end{macro}
%    \begin{macro}{\AlPh@ExpSearch}
%    |#1|: number to be tested\\
%    |#2|: symbols provider
%    \begin{macrocode}
\def\AlPh@ExpSearch#1!#2{%
  \let\AlPh@Next\relax
  \AlPh@TestNumber#1!{#2}%
  \ifAlPh@Unavailable
    \expandafter\AlPh@BinSearch\AlPh@Number!#1!{#2}%
  \else
    \def\AlPh@Number{#1}%
    \ifnum#1>1073741823 %
      \AlPh@TestNumber2147483647!{#2}%
      \ifAlPh@Unavailable
        \AlPh@BinSearch#1!2147483647!{#2}%
      \else
        \def\AlPh@Number{0}%
        \AlPh@Error{%
          Maximal symbol number not found%
        }%
      \fi
    \else
      \def\AlPh@Next{%
        \expandafter\AlPh@ExpSearch\number\intcalcShl{#1}!{#2}%
      }%
    \fi
  \fi
  \AlPh@Next
}
%    \end{macrocode}
%    \end{macro}
%    \begin{macro}{\AlPh@BinSearch}
%    |#1|: available number\\
%    |#2|: unavailable number, |#2| $>$ |#1|\\
%    |#3|: symbols provider
%    \begin{macrocode}
\def\AlPh@BinSearch#1!#2!#3{%
  \expandafter\AlPh@ProcessBinSearch
  \number\intcalcShr{\intcalcAdd{#1}{#2}}!%
  #1!#2!{#3}%
}
%    \end{macrocode}
%    \end{macro}
%    \begin{macro}{\AlPh@ProcessBinSearch}
%    |#1|: number to be tested, |#2| $\leq$ |#1| $\leq$ |#3|\\
%    |#2|: available number\\
%    |#3|: unavailable number\\
%    |#4|: symbols provider
%    \begin{macrocode}
\def\AlPh@ProcessBinSearch#1!#2!#3!#4{%
  \let\AlPh@Next\relax
  \ifnum#1>#2 %
    \ifnum#1<#3 %
      \AlPh@TestNumber#1!{#4}%
      \ifAlPh@Unavailable
        \def\AlPh@Next{%
          \AlPh@BinSearch#2!#1!{#4}%
        }%
      \else
        \def\AlPh@Next{%
          \AlPh@BinSearch#1!#3!{#4}%
        }%
      \fi
    \else
      \def\AlPh@Number{#2}%
    \fi
  \else
    \def\AlPh@Number{#2}%
  \fi
  \AlPh@Next
}
%    \end{macrocode}
%    \end{macro}
%
% \subsection{Methods}
%
%    The names of method macros start with \cs{AlPh@Method}.
%    These macros do the main job in converting a number to
%    its representation.
%    A method command is called with three arguments.
%    The first argument is the number of symbols. The
%    second argument is the basic macro for converting
%    a number with limited number range. The last parameter
%    is the number that needs converting.
%
% \subsubsection{Common methods}
%
%    \begin{macro}{\AlPh@CheckPositive}
%    \noindent
%    |#1|: number to be checked
%    |#2|: continuation macro\\
%    |#3|: number of symbols (hidden here)\\
%    |#4|: symbol provider (hidden here)
%    \begin{macrocode}
\def\AlPh@CheckPositive#1!#2{%
  \ifnum#1<1 %
    \expandafter\@gobblefour
  \fi
  #2{#1}%
}
%    \end{macrocode}
%    \end{macro}
%
% \subsubsection{Method `alph'}
%
%    \begin{macro}{\AlPh@Method@alph}
%    \noindent
%    |#1|: number of symbols\\
%    |#2|: symbols provider\\
%    |#3|: number to be converted
%    \begin{macrocode}
\def\AlPh@Method@alph#1#2#3{%
  \expandafter\AlPh@CheckPositive
|   \number#3!%
+   \the\numexpr#3!%
    \AlPh@ProcessAlph
    {#1}{#2}%
}
%    \end{macrocode}
%    \end{macro}
%    \begin{macro}{\AlPh@ProcessAlph}
%    |#1|: current number\\
%    |#2|: number of symbols\\
%    |#3|: symbols provider
%    \begin{macrocode}
\def\AlPh@ProcessAlph#1#2#3{%
  \ifnum#1>#2 %
    \@ReturnAfterElseFi{%
      \expandafter\AlPh@StepAlph\number
        \intcalcInc{%
          \intcalcMod{\intcalcDec{#1}}{#2}%
        }%
      \expandafter!\number
        \intcalcDiv{\intcalcDec{#1}}{#2}%
      !{#2}{#3}%
    }%
  \else
    \@ReturnAfterFi{%
      #3{#1}%
    }%
  \fi
}
%    \end{macrocode}
%    \end{macro}
%    \begin{macro}{\AlPh@StepAlph}
%    |#1|: current last digit\\
%    |#2|: new current number\\
%    |#3|: number of symbols\\
%    |#4|: symbols provider
%    \begin{macrocode}
\def\AlPh@StepAlph#1!#2!#3#4{%
  \AlPh@ProcessAlph{#2}{#3}{#4}%
  #4{#1}%
}
%    \end{macrocode}
%    \end{macro}
%
% \subsubsection{Method `wrap'}
%
%    \begin{macro}{\AlPh@Method@wrap}
%    \noindent
%    |#1|: number of symbols\\
%    |#2|: symbols provider\\
%    |#3|: number to be converted
%    \begin{macrocode}
\def\AlPh@Method@wrap#1#2#3{%
  \expandafter\AlPh@CheckPositive
|   \number#3!%
+   \the\numexpr#3!%
    \AlPh@ProcessWrap
    {#1}{#2}%
}
%    \end{macrocode}
%    \end{macro}
%
%    \begin{macro}{\AlPh@ProcessWrap}
%    |#1|: number to be converted\\
%    |#2|: number of symbols\\
%    |#3|: symbols provider
%    \begin{macrocode}
\def\AlPh@ProcessWrap#1#2#3{%
  \ifnum#1>#2 %
    \@ReturnAfterElseFi{%
      \expandafter\AlPh@StepWrap\number
        \intcalcInc{\intcalcMod{\intcalcDec{#1}}{#2}}%
      !{#3}%
    }%
  \else
    \@ReturnAfterFi{%
      #3{#1}%
    }%
  \fi
}
%    \end{macrocode}
%    \end{macro}
%    \begin{macro}{\AlPh@StepWrap}
%    |#1|: final number\\
%    |#2|: symbols provider
%    \begin{macrocode}
\def\AlPh@StepWrap#1!#2{%
  #2{#1}%
}
%    \end{macrocode}
%    \end{macro}
%
% \subsubsection{Method `mult'}
%
%    After the number of symbols is exhausted,
%    repetitions of the symbol are used.
%    \begin{gather*}
%      x \Def \text{number to be converted}\\
%      n \Def \text{number of symbols}\\
%      r \Def \text{repetition length}\\
%      s \Def \text{symbol slot}\\
%      r = ((x - 1) \div n) + 1\\
%      s = ((x - 1) \mod n) + 1
%    \end{gather*}
%    \begin{macro}{\AlPh@Method@mult}
%    \noindent
%    |#1|: number of symbols\\
%    |#2|: symbols provider\\
%    |#3|: number to be converted
%    \begin{macrocode}
\def\AlPh@Method@mult#1#2#3{%
  \expandafter\AlPh@CheckPositive
|   \number#3!%
+   \the\numexpr#3!%
    \AlPh@ProcessMult
    {#1}{#2}%
}
%    \end{macrocode}
%    \end{macro}
%
%    \begin{macro}{\AlPh@ProcessMult}
%    |#1|: number to be converted\\
%    |#2|: number of symbols\\
%    |#3|: symbols provider
%    \begin{macrocode}
\def\AlPh@ProcessMult#1#2#3{%
  \ifnum#1>#2 %
    \@ReturnAfterElseFi{%
      \expandafter\AlPh@StepMult\romannumeral
        \intcalcInc{\intcalcDiv{\intcalcDec{#1}}{#2}}%
        000%
      \expandafter!\number
        \intcalcInc{\intcalcMod{\intcalcDec{#1}}{#2}}%
      !{#3}%
    }%
  \else
    \@ReturnAfterFi{%
      #3{#1}%
    }%
  \fi
}
%    \end{macrocode}
%    \end{macro}
%    \begin{macro}{\AlPh@StepMult}
%    |#1#2|: repetitions coded as list of character `m'\\
%    |#3|: symbol slot\\
%    |#4|: symbols provider
%    \begin{macrocode}
\def\AlPh@StepMult#1#2!#3!#4{%
  \ifx\\#2\\%
  \else
    \@ReturnAfterFi{%
      \AlPh@StepMult#2!#3!{#4}%
    }%
  \fi
  #4{#3}%
}
%    \end{macrocode}
%    \end{macro}
%
% \subsection{User interface}
%
%    \begin{macro}{\newalphalph}
%    Macro \cs{newalphalph} had three arguments in versions below 2.0.
%    For the new method argument we use an optional argument an first
%    position.\\
%    |#1|: cmd\\
%    |[#2]|: method name: |alph| (default), |wrap|, |mult|\\ % hash-ok
%    |#3|: symbols provider\\
%    |#4|: number of symbols
%    \begin{macrocode}
\AlPh@IfDefinable\newalphalph{%
  \def\newalphalph#1{%
    \AlPh@IfOptArg{%
      \AlPh@newalphalph{#1}%
    }{%
      \AlPh@newalphalph{#1}[alph]%
    }%
  }%
}
%    \end{macrocode}
%    \end{macro}
%    \begin{macro}{\AlPh@newalphalph}
%    |#1|: cmd
%    |#2|: method name\\
%    |#3|: symbols provider\\
%    |#4|: number of symbols
%    \begin{macrocode}
\def\AlPh@newalphalph#1[#2]#3#4{%
  \begingroup\expandafter\expandafter\expandafter\endgroup
  \expandafter\ifx\csname AlPh@Method@#2\endcsname\relax
    \AlPh@Error{%
      Unknown method %
|     `#2'%
+     `\detokenize{#2}'%
    }%
  \else
    \ifx\\#4\\%
      \AlPh@GetNumberOfSymbols{#3}%
      \ifcase\AlPh@Number
      \else
        \begingroup
          \escapechar=92 % backslash
          \@PackageInfo{alphalph}{%
            Number of symbols for \string#1 is \AlPh@Number
          }%
        \endgroup
        \expandafter\AlPh@NewAlphAlph
        \csname AlPh@Method@#2\expandafter\endcsname
        \AlPh@Number!{#1}{#3}%
      \fi
    \else
      \expandafter\AlPh@NewAlphAlph
      \csname AlPh@Method@#2\expandafter\endcsname
|     \number#4!%
+     \the\numexpr#4!%
      {#1}{#3}%
    \fi
  \fi
}%
%    \end{macrocode}
%    \end{macro}
%    \begin{macro}{\AlPh@NewAlphAlph}
%    |#1|: method macro\\
%    |#2|: number of symbols\\
%    |#3|: cmd\\
%    |#4|: symbols provider
%    \begin{macrocode}
\def\AlPh@NewAlphAlph#1#2!#3#4{%
  \AlPh@IfDefinable#3{%
    \ifnum#2>0 %
      \def#3{#1{#2}{#4}}%
    \else
      \AlPh@Error{%
          Definition of \string#3 failed,\MessageBreak
          because number of symbols (#2) is not positive%
        }%
    \fi
  }%
}
%    \end{macrocode}
%    \end{macro}
%
%    \begin{macro}{\AlphAlph}
%    \begin{macrocode}
\newalphalph\AlphAlph\alphalph@Alph{26}
%    \end{macrocode}
%    \end{macro}
%    \begin{macro}{\alphalph}
%    \begin{macrocode}
\newalphalph\alphalph\alphalph@alph{26}
%    \end{macrocode}
%    \end{macro}
%
%    \begin{macrocode}
\AlPh@AtEnd%
%</package>
%    \end{macrocode}
%
% \section{Test}
%
% \subsection{Catcode checks for loading}
%
%    \begin{macrocode}
%<*test1>
%    \end{macrocode}
%    \begin{macrocode}
\catcode`\{=1 %
\catcode`\}=2 %
\catcode`\#=6 %
\catcode`\@=11 %
\expandafter\ifx\csname count@\endcsname\relax
  \countdef\count@=255 %
\fi
\expandafter\ifx\csname @gobble\endcsname\relax
  \long\def\@gobble#1{}%
\fi
\expandafter\ifx\csname @firstofone\endcsname\relax
  \long\def\@firstofone#1{#1}%
\fi
\expandafter\ifx\csname loop\endcsname\relax
  \expandafter\@firstofone
\else
  \expandafter\@gobble
\fi
{%
  \def\loop#1\repeat{%
    \def\body{#1}%
    \iterate
  }%
  \def\iterate{%
    \body
      \let\next\iterate
    \else
      \let\next\relax
    \fi
    \next
  }%
  \let\repeat=\fi
}%
\def\RestoreCatcodes{}
\count@=0 %
\loop
  \edef\RestoreCatcodes{%
    \RestoreCatcodes
    \catcode\the\count@=\the\catcode\count@\relax
  }%
\ifnum\count@<255 %
  \advance\count@ 1 %
\repeat

\def\RangeCatcodeInvalid#1#2{%
  \count@=#1\relax
  \loop
    \catcode\count@=15 %
  \ifnum\count@<#2\relax
    \advance\count@ 1 %
  \repeat
}
\def\RangeCatcodeCheck#1#2#3{%
  \count@=#1\relax
  \loop
    \ifnum#3=\catcode\count@
    \else
      \errmessage{%
        Character \the\count@\space
        with wrong catcode \the\catcode\count@\space
        instead of \number#3%
      }%
    \fi
  \ifnum\count@<#2\relax
    \advance\count@ 1 %
  \repeat
}
\def\space{ }
\expandafter\ifx\csname LoadCommand\endcsname\relax
  \def\LoadCommand{\input alphalph.sty\relax}%
\fi
\def\Test{%
  \RangeCatcodeInvalid{0}{47}%
  \RangeCatcodeInvalid{58}{64}%
  \RangeCatcodeInvalid{91}{96}%
  \RangeCatcodeInvalid{123}{255}%
  \catcode`\@=12 %
  \catcode`\\=0 %
  \catcode`\%=14 %
  \LoadCommand
  \RangeCatcodeCheck{0}{36}{15}%
  \RangeCatcodeCheck{37}{37}{14}%
  \RangeCatcodeCheck{38}{47}{15}%
  \RangeCatcodeCheck{48}{57}{12}%
  \RangeCatcodeCheck{58}{63}{15}%
  \RangeCatcodeCheck{64}{64}{12}%
  \RangeCatcodeCheck{65}{90}{11}%
  \RangeCatcodeCheck{91}{91}{15}%
  \RangeCatcodeCheck{92}{92}{0}%
  \RangeCatcodeCheck{93}{96}{15}%
  \RangeCatcodeCheck{97}{122}{11}%
  \RangeCatcodeCheck{123}{255}{15}%
  \RestoreCatcodes
}
\Test
\csname @@end\endcsname
\end
%    \end{macrocode}
%    \begin{macrocode}
%</test1>
%    \end{macrocode}
%
% \section{Macro tests}
%
%    \begin{macrocode}
%<*test2>
\NeedsTeXFormat{LaTeX2e}
\nofiles
\documentclass{article}
%<*noetex>
\makeatletter
\let\saved@numexpr\numexpr
\newcommand*{\DisableNumexpr}{%
  \let\numexpr\@undefined
}
\newcommand*{\RestoreNumexpr}{%
  \let\numexpr\saved@numexpr
}
\DisableNumexpr
%</noetex>
\usepackage{alphalph}[2016/05/16]
%<noetex>\RestoreNumexpr
\usepackage{qstest}
\IncludeTests{*}
\LogTests{log}{*}{*}

\newcommand*{\TestCmd}[3]{%
  \setbox0=\hbox{%
%<noetex>    \DisableNumexpr
    \edef\TestString{#1{#2}}%
    \expandafter\Expect\expandafter{\TestString}{#3}%
    \edef\TestString{#1{#2} }%
    \expandafter\Expect\expandafter{\TestString}{#3 }%
  }%
  \Expect*{\the\wd0}{0.0pt}%
}

\makeatletter
\newalphalph\LaTeXAlphAlph\@Alph{26}
\newalphalph\LaTeXalphalph\@alph{26}
\newalphalph\AlphWrap[wrap]\alphalph@Alph{26}
\newalphalph\alphwrap[wrap]\alphalph@alph{26}
\newalphalph\LaTeXAlphWrap[wrap]\@Alph{26}
\newalphalph\LaTeXalphwrap[wrap]\@alph{26}
\def\LastSymbol#1{%
  \ifx\\#1\\%
  \else
    \@LastSymbol#1\@nil
  \fi
}
\def\@LastSymbol#1#2\@nil{%
  \ifx\\#2\\%
    #1%
  \else
    \@LastSymbol#2\@nil
  \fi
}
\makeatother
\newcommand*{\TestAlph}[2]{%
  \uppercase{\TestCallCmd\AlphAlph{#2}}{#1}%
  \lowercase{\TestCallCmd\alphalph{#2}}{#1}%
  \uppercase{\TestCallCmd\LaTeXAlphAlph{#2}}{#1}%
  \lowercase{\TestCallCmd\LaTeXalphalph{#2}}{#1}%
  \edef\WrapString{\LastSymbol{#2}}%
  \expandafter\TestAlphWrap\expandafter{\WrapString}{#1}%
}
\newcommand*{\TestAlphWrap}[2]{%
  \uppercase{\TestCallCmd\AlphWrap{#1}}{#2}%
  \lowercase{\TestCallCmd\alphwrap{#1}}{#2}%
  \uppercase{\TestCallCmd\LaTeXAlphWrap{#1}}{#2}%
  \lowercase{\TestCallCmd\LaTeXalphwrap{#1}}{#2}%
}
\newcommand*{\TestCallCmd}[3]{%
  \TestCmd#1{#3}{#2}%
}
\begin{qstest}{AlphSymbols}{alphalph, AlphAlph, symbols}
  \TestAlph{1}{a}%
  \TestAlph{2}{b}%
  \TestAlph{3}{c}%
  \TestAlph{4}{d}%
  \TestAlph{5}{e}%
  \TestAlph{6}{f}%
  \TestAlph{7}{g}%
  \TestAlph{8}{h}%
  \TestAlph{9}{i}%
  \TestAlph{10}{j}%
  \TestAlph{11}{k}%
  \TestAlph{12}{l}%
  \TestAlph{13}{m}%
  \TestAlph{14}{n}%
  \TestAlph{15}{o}%
  \TestAlph{16}{p}%
  \TestAlph{17}{q}%
  \TestAlph{18}{r}%
  \TestAlph{19}{s}%
  \TestAlph{20}{t}%
  \TestAlph{21}{u}%
  \TestAlph{22}{v}%
  \TestAlph{23}{w}%
  \TestAlph{24}{x}%
  \TestAlph{25}{y}%
  \TestAlph{26}{z}%
\end{qstest}
\begin{qstest}{AlphRange}{alphalph, range}
  \TestAlph{0}{}%
  \TestAlph{-1}{}%
  \TestAlph{-2147483647}{}%
  \TestAlph{27}{aa}%
  \TestAlph{28}{ab}%
  \TestAlph{52}{az}%
  \TestAlph{53}{ba}%
  \TestAlph{78}{bz}%
  \TestAlph{79}{ca}%
  \TestAlph{702}{zz}%
  \TestAlph{703}{aaa}%
  \TestAlph{2147483647}{fxshrxw}%
\end{qstest}

\makeatletter
\newcommand*{\myvocals}[1]{%
  \ifcase#1X\or A\or E\or I\or O\or U\else Y\fi
}
\makeatother
\newalphalph\vocalsvocals\myvocals{5}
\newcommand*{\TestVocals}{%
  \TestCmd\vocalsvocals
}
\begin{qstest}{vocals}{vocals}
  \TestVocals{0}{}%
  \TestVocals{1}{A}%
  \TestVocals{2}{E}%
  \TestVocals{3}{I}%
  \TestVocals{4}{O}%
  \TestVocals{5}{U}%
  \TestVocals{6}{AA}%
  \TestVocals{7}{AE}%
  \TestVocals{8}{AI}%
  \TestVocals{9}{AO}%
  \TestVocals{10}{AU}%
  \TestVocals{11}{EA}%
  \TestVocals{24}{OO}%
  \TestVocals{25}{OU}%
  \TestVocals{26}{UA}%
  \TestVocals{29}{UO}%
  \TestVocals{30}{UU}%
  \TestVocals{31}{AAA}%
  \TestVocals{155}{UUU}%
  \TestVocals{156}{AAAA}%
  \TestVocals{2147483647}{AIIOOEEIOIIUOE}%
\end{qstest}

\makeatletter
\newalphalph\AlphMult[mult]{\alphalph@Alph}{26}
\newalphalph\alphmult[mult]{\alphalph@alph}{26}
\newalphalph\LaTeXAlphMult[mult]{\@Alph}{26}
\newalphalph\LaTeXalphmult[mult]{\@alph}{26}
\makeatother
\newcommand*{\TestMult}[2]{%
  \uppercase{\TestCallCmd\AlphMult{#2}}{#1}%
  \lowercase{\TestCallCmd\alphmult{#2}}{#1}%
  \uppercase{\TestCallCmd\LaTeXAlphMult{#2}}{#1}%
  \lowercase{\TestCallCmd\LaTeXalphmult{#2}}{#1}%
}
\begin{qstest}{mult}{mult}
  \TestMult{0}{}%
  \TestMult{-1}{}%
  \TestMult{-2147483647}{}%
  \TestMult{1}{a}%
  \TestMult{2}{b}%
  \TestMult{26}{z}%
  \TestMult{27}{aa}%
  \TestMult{28}{bb}%
  \TestMult{52}{zz}%
  \TestMult{53}{aaa}%
  \TestMult{54}{bbb}%
  \TestMult{259}{yyyyyyyyyy}%
  \TestMult{260}{zzzzzzzzzz}%
  \TestMult{261}{aaaaaaaaaaa}%
  \TestMult{262}{bbbbbbbbbbb}%
\end{qstest}

\def\myvocalsB#1{%
  \ifcase#1\or A\or E\or I\or O\or U\fi
}
\begin{qstest}{symbolnum}{symbolnum}
  \makeatletter
  \def\Test#1#2{%
    \let\TestCmd\relax
    \newalphalph\TestCmd{#1}{}%
    \Expect*{\AlPh@Number}{#2}%
  }%
  \Test\@alph{26}%
  \Test\@Alph{26}%
  \Test\@fnsymbol{9}%
  \Test\myvocalsB{5}%
  \Test\alphalph@alph{26}%
  \Test\alphalph@Alph{26}%
\end{qstest}

\begin{qstest}{list}{list}
  \makeatletter
  \def\catch#1\relax{%
    \def\FoundList{\catch#1}%
  }%
  \def\Test[#1]#2#3#4{%
    \let\testcmd\relax
    \newalphalph\testcmd[{#1}]{\catch}{#2}%
    \testcmd{#3}|\relax
    \expandafter\Expect\expandafter{\FoundList}{#4|}%
    %
    \let\SavedCatch\catch
    \def\catch{\noexpand\catch\noexpand\foo}%
    \edef\Result{#4|}%
    \@onelevel@sanitize\Result
    \let\catch\SavedCatch
    \let\testcmd\relax
    \newalphalph\testcmd[{#1}]{\catch\foo}{#2}%
    \testcmd{#3}|\relax
    \@onelevel@sanitize\FoundList
    \Expect*{\FoundList}*{\Result}%
  }%
  \Test[alph]{26}{3}{\catch{3}}%
  \Test[alph]{26}{12}{\catch{12}}%
  \Test[alph]{26}{27}{\catch{1}\catch{1}}%
  \Test[alph]{26}{78}{\catch{2}\catch{26}}%
  \Test[wrap]{26}{7}{\catch{7}}%
  \Test[wrap]{26}{14}{\catch{14}}%
  \Test[wrap]{26}{80}{\catch{2}}%
  \Test[wrap]{26}{700}{\catch{24}}%
  \Test[mult]{26}{4}{\catch{4}}%
  \Test[mult]{26}{17}{\catch{17}}%
  \Test[mult]{26}{54}{\catch{2}\catch{2}\catch{2}}%
\end{qstest}

\begin{document}
\end{document}
%</test2>
%    \end{macrocode}
%
% \section{Installation}
%
% \subsection{Download}
%
% \paragraph{Package.} This package is available on
% CTAN\footnote{\CTANpkg{alphalph}}:
% \begin{description}
% \item[\CTAN{macros/latex/contrib/oberdiek/alphalph.dtx}] The source file.
% \item[\CTAN{macros/latex/contrib/oberdiek/alphalph.pdf}] Documentation.
% \end{description}
%
%
% \paragraph{Bundle.} All the packages of the bundle `oberdiek'
% are also available in a TDS compliant ZIP archive. There
% the packages are already unpacked and the documentation files
% are generated. The files and directories obey the TDS standard.
% \begin{description}
% \item[\CTANinstall{install/macros/latex/contrib/oberdiek.tds.zip}]
% \end{description}
% \emph{TDS} refers to the standard ``A Directory Structure
% for \TeX\ Files'' (\CTAN{tds/tds.pdf}). Directories
% with \xfile{texmf} in their name are usually organized this way.
%
% \subsection{Bundle installation}
%
% \paragraph{Unpacking.} Unpack the \xfile{oberdiek.tds.zip} in the
% TDS tree (also known as \xfile{texmf} tree) of your choice.
% Example (linux):
% \begin{quote}
%   |unzip oberdiek.tds.zip -d ~/texmf|
% \end{quote}
%
% \paragraph{Script installation.}
% Check the directory \xfile{TDS:scripts/oberdiek/} for
% scripts that need further installation steps.
% Package \xpackage{attachfile2} comes with the Perl script
% \xfile{pdfatfi.pl} that should be installed in such a way
% that it can be called as \texttt{pdfatfi}.
% Example (linux):
% \begin{quote}
%   |chmod +x scripts/oberdiek/pdfatfi.pl|\\
%   |cp scripts/oberdiek/pdfatfi.pl /usr/local/bin/|
% \end{quote}
%
% \subsection{Package installation}
%
% \paragraph{Unpacking.} The \xfile{.dtx} file is a self-extracting
% \docstrip\ archive. The files are extracted by running the
% \xfile{.dtx} through \plainTeX:
% \begin{quote}
%   \verb|tex alphalph.dtx|
% \end{quote}
%
% \paragraph{TDS.} Now the different files must be moved into
% the different directories in your installation TDS tree
% (also known as \xfile{texmf} tree):
% \begin{quote}
% \def\t{^^A
% \begin{tabular}{@{}>{\ttfamily}l@{ $\rightarrow$ }>{\ttfamily}l@{}}
%   alphalph.sty & tex/generic/oberdiek/alphalph.sty\\
%   alphalph.pdf & doc/latex/oberdiek/alphalph.pdf\\
%   test/alphalph-test1.tex & doc/latex/oberdiek/test/alphalph-test1.tex\\
%   test/alphalph-test2.tex & doc/latex/oberdiek/test/alphalph-test2.tex\\
%   test/alphalph-test3.tex & doc/latex/oberdiek/test/alphalph-test3.tex\\
%   alphalph.dtx & source/latex/oberdiek/alphalph.dtx\\
% \end{tabular}^^A
% }^^A
% \sbox0{\t}^^A
% \ifdim\wd0>\linewidth
%   \begingroup
%     \advance\linewidth by\leftmargin
%     \advance\linewidth by\rightmargin
%   \edef\x{\endgroup
%     \def\noexpand\lw{\the\linewidth}^^A
%   }\x
%   \def\lwbox{^^A
%     \leavevmode
%     \hbox to \linewidth{^^A
%       \kern-\leftmargin\relax
%       \hss
%       \usebox0
%       \hss
%       \kern-\rightmargin\relax
%     }^^A
%   }^^A
%   \ifdim\wd0>\lw
%     \sbox0{\small\t}^^A
%     \ifdim\wd0>\linewidth
%       \ifdim\wd0>\lw
%         \sbox0{\footnotesize\t}^^A
%         \ifdim\wd0>\linewidth
%           \ifdim\wd0>\lw
%             \sbox0{\scriptsize\t}^^A
%             \ifdim\wd0>\linewidth
%               \ifdim\wd0>\lw
%                 \sbox0{\tiny\t}^^A
%                 \ifdim\wd0>\linewidth
%                   \lwbox
%                 \else
%                   \usebox0
%                 \fi
%               \else
%                 \lwbox
%               \fi
%             \else
%               \usebox0
%             \fi
%           \else
%             \lwbox
%           \fi
%         \else
%           \usebox0
%         \fi
%       \else
%         \lwbox
%       \fi
%     \else
%       \usebox0
%     \fi
%   \else
%     \lwbox
%   \fi
% \else
%   \usebox0
% \fi
% \end{quote}
% If you have a \xfile{docstrip.cfg} that configures and enables \docstrip's
% TDS installing feature, then some files can already be in the right
% place, see the documentation of \docstrip.
%
% \subsection{Refresh file name databases}
%
% If your \TeX~distribution
% (\teTeX, \mikTeX, \dots) relies on file name databases, you must refresh
% these. For example, \teTeX\ users run \verb|texhash| or
% \verb|mktexlsr|.
%
% \subsection{Some details for the interested}
%
% \paragraph{Attached source.}
%
% The PDF documentation on CTAN also includes the
% \xfile{.dtx} source file. It can be extracted by
% AcrobatReader 6 or higher. Another option is \textsf{pdftk},
% e.g. unpack the file into the current directory:
% \begin{quote}
%   \verb|pdftk alphalph.pdf unpack_files output .|
% \end{quote}
%
% \paragraph{Unpacking with \LaTeX.}
% The \xfile{.dtx} chooses its action depending on the format:
% \begin{description}
% \item[\plainTeX:] Run \docstrip\ and extract the files.
% \item[\LaTeX:] Generate the documentation.
% \end{description}
% If you insist on using \LaTeX\ for \docstrip\ (really,
% \docstrip\ does not need \LaTeX), then inform the autodetect routine
% about your intention:
% \begin{quote}
%   \verb|latex \let\install=y\input{alphalph.dtx}|
% \end{quote}
% Do not forget to quote the argument according to the demands
% of your shell.
%
% \paragraph{Generating the documentation.}
% You can use both the \xfile{.dtx} or the \xfile{.drv} to generate
% the documentation. The process can be configured by the
% configuration file \xfile{ltxdoc.cfg}. For instance, put this
% line into this file, if you want to have A4 as paper format:
% \begin{quote}
%   \verb|\PassOptionsToClass{a4paper}{article}|
% \end{quote}
% An example follows how to generate the
% documentation with pdf\LaTeX:
% \begin{quote}
%\begin{verbatim}
%pdflatex alphalph.dtx
%makeindex -s gind.ist alphalph.idx
%pdflatex alphalph.dtx
%makeindex -s gind.ist alphalph.idx
%pdflatex alphalph.dtx
%\end{verbatim}
% \end{quote}
%
% \begin{History}
%   \begin{Version}{1999/03/19 v0.1}
%   \item
%     The first version was built as a response to a
%     \URL{\link{question}}^^A
%     {https://groups.google.com/group/comp.text.tex/msg/17a74cd721641038}^^A
%     of \NameEmail{Will Douglas}{william.douglas@wolfson.ox.ac.uk}
%     and the
%     \URL{\link{request}}^^A
%     {https://groups.google.com/group/comp.text.tex/msg/8f9768825640315f}^^A
%     of \NameEmail{Donald Arsenau}{asnd@reg.triumf.ca},
%     published in the newsgroup
%     \href{news:comp.text.tex}{comp.text.tex}:
%     \URL{``\link{Re: alph counters \textgreater\ 26}''}^^A
%     {https://groups.google.com/group/comp.text.tex/msg/cec563eef8bf65d0}
%   \item
%     Copyright: LPPL (\CTAN{macros/latex/base/lppl.txt})
%   \end{Version}
%   \begin{Version}{1999/04/12 v1.0}
%   \item
%     Documentation added in dtx format.
%   \item
%     \hologo{eTeX} support added.
%   \end{Version}
%   \begin{Version}{1999/04/13 v1.1}
%   \item
%     Minor documentation change.
%   \item
%     First CTAN release.
%   \end{Version}
%   \begin{Version}{1999/06/26 v1.2}
%   \item
%     First generic code about \cmd{\ProvidesPackage} improved.
%   \item
%     Documentation: Installation part revised.
%   \end{Version}
%   \begin{Version}{2006/02/20 v1.3}
%   \item
%     Reload check (for \hologo{plainTeX})
%   \item
%     New DTX framework.
%   \item
%     LPPL 1.3
%   \end{Version}
%   \begin{Version}{2006/05/30 v1.4}
%   \item
%     \cs{newalphalph} added.
%   \end{Version}
%   \begin{Version}{2007/04/11 v1.5}
%   \item
%     Line ends sanitized.
%   \end{Version}
%   \begin{Version}{2007/09/09 v2.0}
%   \item
%     New implementation that uses package \cs{intcalc}.
%     This removes the dependency on \hologo{eTeX}.
%   \item
%     \cs{newalphalph} is extended to support new methods `wrap' and 'multi'.
%   \item
%     Documentation rewritten.
%   \end{Version}
%   \begin{Version}{2008/08/11 v2.1}
%   \item
%     Code is not changed.
%   \item
%     URLs updated from \texttt{www.dejanews.com}
%     to \texttt{groups.google.com}.
%   \end{Version}
%   \begin{Version}{2010/03/01 v2.2}
%   \item
%     Compatibility with \hologo{iniTeX}.
%   \end{Version}
%   \begin{Version}{2010/04/18 v2.3}
%   \item
%     Documentation fixes (Martin M\"unch).
%   \end{Version}
%   \begin{Version}{2011/05/13 v2.4}
%   \item
%     Documentation fixes (Jim Diamond) and using package
%     \xpackage{hologo} for the documentation.
%   \item
%     Catalogue file added.
%   \end{Version}
%   \begin{Version}{2016/05/16 v2.5}
%   \item
%     Documentation updates.
%   \end{Version}
% \end{History}
%
% \PrintIndex
%
% \Finale
\endinput
|
% \end{quote}
% Do not forget to quote the argument according to the demands
% of your shell.
%
% \paragraph{Generating the documentation.}
% You can use both the \xfile{.dtx} or the \xfile{.drv} to generate
% the documentation. The process can be configured by the
% configuration file \xfile{ltxdoc.cfg}. For instance, put this
% line into this file, if you want to have A4 as paper format:
% \begin{quote}
%   \verb|\PassOptionsToClass{a4paper}{article}|
% \end{quote}
% An example follows how to generate the
% documentation with pdf\LaTeX:
% \begin{quote}
%\begin{verbatim}
%pdflatex alphalph.dtx
%makeindex -s gind.ist alphalph.idx
%pdflatex alphalph.dtx
%makeindex -s gind.ist alphalph.idx
%pdflatex alphalph.dtx
%\end{verbatim}
% \end{quote}
%
% \begin{History}
%   \begin{Version}{1999/03/19 v0.1}
%   \item
%     The first version was built as a response to a
%     \URL{\link{question}}^^A
%     {https://groups.google.com/group/comp.text.tex/msg/17a74cd721641038}^^A
%     of \NameEmail{Will Douglas}{william.douglas@wolfson.ox.ac.uk}
%     and the
%     \URL{\link{request}}^^A
%     {https://groups.google.com/group/comp.text.tex/msg/8f9768825640315f}^^A
%     of \NameEmail{Donald Arsenau}{asnd@reg.triumf.ca},
%     published in the newsgroup
%     \href{news:comp.text.tex}{comp.text.tex}:
%     \URL{``\link{Re: alph counters \textgreater\ 26}''}^^A
%     {https://groups.google.com/group/comp.text.tex/msg/cec563eef8bf65d0}
%   \item
%     Copyright: LPPL (\CTAN{macros/latex/base/lppl.txt})
%   \end{Version}
%   \begin{Version}{1999/04/12 v1.0}
%   \item
%     Documentation added in dtx format.
%   \item
%     \hologo{eTeX} support added.
%   \end{Version}
%   \begin{Version}{1999/04/13 v1.1}
%   \item
%     Minor documentation change.
%   \item
%     First CTAN release.
%   \end{Version}
%   \begin{Version}{1999/06/26 v1.2}
%   \item
%     First generic code about \cmd{\ProvidesPackage} improved.
%   \item
%     Documentation: Installation part revised.
%   \end{Version}
%   \begin{Version}{2006/02/20 v1.3}
%   \item
%     Reload check (for \hologo{plainTeX})
%   \item
%     New DTX framework.
%   \item
%     LPPL 1.3
%   \end{Version}
%   \begin{Version}{2006/05/30 v1.4}
%   \item
%     \cs{newalphalph} added.
%   \end{Version}
%   \begin{Version}{2007/04/11 v1.5}
%   \item
%     Line ends sanitized.
%   \end{Version}
%   \begin{Version}{2007/09/09 v2.0}
%   \item
%     New implementation that uses package \cs{intcalc}.
%     This removes the dependency on \hologo{eTeX}.
%   \item
%     \cs{newalphalph} is extended to support new methods `wrap' and 'multi'.
%   \item
%     Documentation rewritten.
%   \end{Version}
%   \begin{Version}{2008/08/11 v2.1}
%   \item
%     Code is not changed.
%   \item
%     URLs updated from \texttt{www.dejanews.com}
%     to \texttt{groups.google.com}.
%   \end{Version}
%   \begin{Version}{2010/03/01 v2.2}
%   \item
%     Compatibility with \hologo{iniTeX}.
%   \end{Version}
%   \begin{Version}{2010/04/18 v2.3}
%   \item
%     Documentation fixes (Martin M\"unch).
%   \end{Version}
%   \begin{Version}{2011/05/13 v2.4}
%   \item
%     Documentation fixes (Jim Diamond) and using package
%     \xpackage{hologo} for the documentation.
%   \item
%     Catalogue file added.
%   \end{Version}
%   \begin{Version}{2016/05/16 v2.5}
%   \item
%     Documentation updates.
%   \end{Version}
% \end{History}
%
% \PrintIndex
%
% \Finale
\endinput
|
% \end{quote}
% Do not forget to quote the argument according to the demands
% of your shell.
%
% \paragraph{Generating the documentation.}
% You can use both the \xfile{.dtx} or the \xfile{.drv} to generate
% the documentation. The process can be configured by the
% configuration file \xfile{ltxdoc.cfg}. For instance, put this
% line into this file, if you want to have A4 as paper format:
% \begin{quote}
%   \verb|\PassOptionsToClass{a4paper}{article}|
% \end{quote}
% An example follows how to generate the
% documentation with pdf\LaTeX:
% \begin{quote}
%\begin{verbatim}
%pdflatex alphalph.dtx
%makeindex -s gind.ist alphalph.idx
%pdflatex alphalph.dtx
%makeindex -s gind.ist alphalph.idx
%pdflatex alphalph.dtx
%\end{verbatim}
% \end{quote}
%
% \begin{History}
%   \begin{Version}{1999/03/19 v0.1}
%   \item
%     The first version was built as a response to a
%     \URL{\link{question}}^^A
%     {https://groups.google.com/group/comp.text.tex/msg/17a74cd721641038}^^A
%     of \NameEmail{Will Douglas}{william.douglas@wolfson.ox.ac.uk}
%     and the
%     \URL{\link{request}}^^A
%     {https://groups.google.com/group/comp.text.tex/msg/8f9768825640315f}^^A
%     of \NameEmail{Donald Arsenau}{asnd@reg.triumf.ca},
%     published in the newsgroup
%     \href{news:comp.text.tex}{comp.text.tex}:
%     \URL{``\link{Re: alph counters \textgreater\ 26}''}^^A
%     {https://groups.google.com/group/comp.text.tex/msg/cec563eef8bf65d0}
%   \item
%     Copyright: LPPL (\CTAN{macros/latex/base/lppl.txt})
%   \end{Version}
%   \begin{Version}{1999/04/12 v1.0}
%   \item
%     Documentation added in dtx format.
%   \item
%     \hologo{eTeX} support added.
%   \end{Version}
%   \begin{Version}{1999/04/13 v1.1}
%   \item
%     Minor documentation change.
%   \item
%     First CTAN release.
%   \end{Version}
%   \begin{Version}{1999/06/26 v1.2}
%   \item
%     First generic code about \cmd{\ProvidesPackage} improved.
%   \item
%     Documentation: Installation part revised.
%   \end{Version}
%   \begin{Version}{2006/02/20 v1.3}
%   \item
%     Reload check (for \hologo{plainTeX})
%   \item
%     New DTX framework.
%   \item
%     LPPL 1.3
%   \end{Version}
%   \begin{Version}{2006/05/30 v1.4}
%   \item
%     \cs{newalphalph} added.
%   \end{Version}
%   \begin{Version}{2007/04/11 v1.5}
%   \item
%     Line ends sanitized.
%   \end{Version}
%   \begin{Version}{2007/09/09 v2.0}
%   \item
%     New implementation that uses package \cs{intcalc}.
%     This removes the dependency on \hologo{eTeX}.
%   \item
%     \cs{newalphalph} is extended to support new methods `wrap' and 'multi'.
%   \item
%     Documentation rewritten.
%   \end{Version}
%   \begin{Version}{2008/08/11 v2.1}
%   \item
%     Code is not changed.
%   \item
%     URLs updated from \texttt{www.dejanews.com}
%     to \texttt{groups.google.com}.
%   \end{Version}
%   \begin{Version}{2010/03/01 v2.2}
%   \item
%     Compatibility with \hologo{iniTeX}.
%   \end{Version}
%   \begin{Version}{2010/04/18 v2.3}
%   \item
%     Documentation fixes (Martin M\"unch).
%   \end{Version}
%   \begin{Version}{2011/05/13 v2.4}
%   \item
%     Documentation fixes (Jim Diamond) and using package
%     \xpackage{hologo} for the documentation.
%   \item
%     Catalogue file added.
%   \end{Version}
%   \begin{Version}{2016/05/16 v2.5}
%   \item
%     Documentation updates.
%   \end{Version}
% \end{History}
%
% \PrintIndex
%
% \Finale
\endinput

%        (quote the arguments according to the demands of your shell)
%
% Documentation:
%    (a) If alphalph.drv is present:
%           latex alphalph.drv
%    (b) Without alphalph.drv:
%           latex alphalph.dtx; ...
%    The class ltxdoc loads the configuration file ltxdoc.cfg
%    if available. Here you can specify further options, e.g.
%    use A4 as paper format:
%       \PassOptionsToClass{a4paper}{article}
%
%    Programm calls to get the documentation (example):
%       pdflatex alphalph.dtx
%       makeindex -s gind.ist alphalph.idx
%       pdflatex alphalph.dtx
%       makeindex -s gind.ist alphalph.idx
%       pdflatex alphalph.dtx
%
% Installation:
%    TDS:tex/generic/oberdiek/alphalph.sty
%    TDS:doc/latex/oberdiek/alphalph.pdf
%    TDS:doc/latex/oberdiek/test/alphalph-test1.tex
%    TDS:doc/latex/oberdiek/test/alphalph-test2.tex
%    TDS:doc/latex/oberdiek/test/alphalph-test3.tex
%    TDS:source/latex/oberdiek/alphalph.dtx
%
%<*ignore>
\begingroup
  \catcode123=1 %
  \catcode125=2 %
  \def\x{LaTeX2e}%
\expandafter\endgroup
\ifcase 0\ifx\install y1\fi\expandafter
         \ifx\csname processbatchFile\endcsname\relax\else1\fi
         \ifx\fmtname\x\else 1\fi\relax
\else\csname fi\endcsname
%</ignore>
%<*install>
\input docstrip.tex
\Msg{************************************************************************}
\Msg{* Installation}
\Msg{* Package: alphalph 2016/05/16 v2.5 Convert numbers to letters (HO)}
\Msg{************************************************************************}

\keepsilent
\askforoverwritefalse

\let\MetaPrefix\relax
\preamble

This is a generated file.

Project: alphalph
Version: 2016/05/16 v2.5

Copyright (C) 1999, 2006-2008, 2010, 2011 by
   Heiko Oberdiek <heiko.oberdiek at googlemail.com>

This work may be distributed and/or modified under the
conditions of the LaTeX Project Public License, either
version 1.3c of this license or (at your option) any later
version. This version of this license is in
   https://www.latex-project.org/lppl/lppl-1-3c.txt
and the latest version of this license is in
   https://www.latex-project.org/lppl.txt
and version 1.3 or later is part of all distributions of
LaTeX version 2005/12/01 or later.

This work has the LPPL maintenance status "maintained".

The Current Maintainers of this work are
Heiko Oberdiek and the Oberdiek Package Support Group
https://github.com/ho-tex/oberdiek/issues


The Base Interpreter refers to any `TeX-Format',
because some files are installed in TDS:tex/generic//.

This work consists of the main source file alphalph.dtx
and the derived files
   alphalph.sty, alphalph.pdf, alphalph.ins, alphalph.drv,
   alphalph-test1.tex, alphalph-test2.tex, alphalph-test3.tex.

\endpreamble
\let\MetaPrefix\DoubleperCent

\generate{%
  \file{alphalph.ins}{\from{alphalph.dtx}{install}}%
  \file{alphalph.drv}{\from{alphalph.dtx}{driver}}%
  \usedir{tex/generic/oberdiek}%
  \file{alphalph.sty}{\from{alphalph.dtx}{package}}%
%  \usedir{doc/latex/oberdiek/test}%
%  \file{alphalph-test1.tex}{\from{alphalph.dtx}{test1}}%
%  \file{alphalph-test2.tex}{\from{alphalph.dtx}{test2,etex}}%
%  \file{alphalph-test3.tex}{\from{alphalph.dtx}{test2,noetex}}%
  \nopreamble
  \nopostamble
%  \usedir{source/latex/oberdiek/catalogue}%
%  \file{alphalph.xml}{\from{alphalph.dtx}{catalogue}}%
}

\catcode32=13\relax% active space
\let =\space%
\Msg{************************************************************************}
\Msg{*}
\Msg{* To finish the installation you have to move the following}
\Msg{* file into a directory searched by TeX:}
\Msg{*}
\Msg{*     alphalph.sty}
\Msg{*}
\Msg{* To produce the documentation run the file `alphalph.drv'}
\Msg{* through LaTeX.}
\Msg{*}
\Msg{* Happy TeXing!}
\Msg{*}
\Msg{************************************************************************}

\endbatchfile
%</install>
%<*ignore>
\fi
%</ignore>
%<*driver>
\NeedsTeXFormat{LaTeX2e}
\ProvidesFile{alphalph.drv}%
  [2016/05/16 v2.5 Convert numbers to letters (HO)]%
\documentclass{ltxdoc}
\usepackage{alphalph}[2016/05/16]
\makeatletter
\newalphalph{\fnsymbolwrap}[wrap]{\@fnsymbol}{}
\newcommand*{\fnsymbolsingle}[1]{%
  \ensuremath{%
    \ifcase#1%
    \or *%
    \or \dagger
    \or \ddagger
    \or \mathsection
    \or \mathparagraph
    \else
      \@ctrerr
    \fi
  }%
}
\makeatother
\newalphalph{\fnsymbolmult}[mult]{\fnsymbolsingle}{}
\usepackage{amsmath}
\DeclareMathOperator{\opDiv}{div}
\DeclareMathOperator{\opMod}{mod}
\newcommand*{\Def}{%
  \ensuremath{%
    \mathrel{\mathop{:}}=%
  }%
}
\usepackage{holtxdoc}[2011/11/22]
\begin{document}
  \DocInput{alphalph.dtx}%
\end{document}
%</driver>
% \fi
%
%
% \CharacterTable
%  {Upper-case    \A\B\C\D\E\F\G\H\I\J\K\L\M\N\O\P\Q\R\S\T\U\V\W\X\Y\Z
%   Lower-case    \a\b\c\d\e\f\g\h\i\j\k\l\m\n\o\p\q\r\s\t\u\v\w\x\y\z
%   Digits        \0\1\2\3\4\5\6\7\8\9
%   Exclamation   \!     Double quote  \"     Hash (number) \#
%   Dollar        \$     Percent       \%     Ampersand     \&
%   Acute accent  \'     Left paren    \(     Right paren   \)
%   Asterisk      \*     Plus          \+     Comma         \,
%   Minus         \-     Point         \.     Solidus       \/
%   Colon         \:     Semicolon     \;     Less than     \<
%   Equals        \=     Greater than  \>     Question mark \?
%   Commercial at \@     Left bracket  \[     Backslash     \\
%   Right bracket \]     Circumflex    \^     Underscore    \_
%   Grave accent  \`     Left brace    \{     Vertical bar  \|
%   Right brace   \}     Tilde         \~}
%
% \GetFileInfo{alphalph.drv}
%
% \title{The \xpackage{alphalph} package}
% \date{2016/05/16 v2.5}
% \author{Heiko Oberdiek\thanks
% {Please report any issues at \url{https://github.com/ho-tex/oberdiek/issues}}}
%
% \maketitle
%
% \begin{abstract}
% The package provides methods to represent numbers with a limited
% set of symbols. Both \hologo{LaTeX} and \hologo{plainTeX} are supported.
% \end{abstract}
%
% \tableofcontents
%
% \section{Documentation}
%
% \subsection{Introduction}
% \hologo{LaTeX} counters can be represented in different ways
% by using presentation commands:
% \begin{quote}
%   \cs{arabic}, \cs{roman}, \cs{Roman},\\
%   \cs{alph}, \cs{Alph}, \cs{fnsymbol}
% \end{quote}
% The ranges of supported counter values are more or less
% restricted. Only \cs{arabic} can be used with any counter
% value \hologo{TeX} supports.
% \begin{quote}
% \catcode`\|=12 %
% \begin{tabular}{@{}l|l|l|l@{}}
% Presentation & Supported & Ignored & Error message\\
% command      & domain    & values  & ``Counter too large''\\
% \hline
% \cs{arabic}
%   & \ttfamily -MAX..MAX & &\\
% \cs{roman}, \cs{Roman}
%   & \ttfamily 1..MAX & \ttfamily -MAX..0 &\\
% \cs{alph}, \cs{Alph}
%   & \ttfamily 1..26 & 0 & \ttfamily -MAX..-1, 27..MAX\\
% \cs{fnsymbol}
%   & \ttfamily 1..9 & 0 & \ttfamily -MAX..-1, 10..MAX\\
% \hline
% \end{tabular}\\[1ex]
% \texttt{MAX} = \texttt{2147483647}
% \end{quote}
% Ordinal numbers are often used in documents: numbering
% of chapters, sections, figures, footnotes and so on.
% The layouter chooses \cs{Alph} for chapter numbers
% and \cs{fnsymbol} for footnotes. But what can be done
% if there are more than 26 chapters or more than 10 footnotes?
% This package \xpackage{alphalph} allows to define new
% presentation commands. They rely on a existing command
% and define presentations for values greater the limits.
% Three different methods are provided by the package.
% In the following use cases they are presentated.
%
% \subsection{Use cases}
%
% \subsubsection{Number system based on symbols}
%
% Asume you are writing a book and your lecturer demands
% that chapter numbers must be letters. But you have already
% 30 chapters and you have only 26 letters?
%
% In the decimal system the situation would be clear.
% If you run out of digits, you are using more digits
% to represent a number. This method can be also be used
% for letters. After chapter 26 with |Z| we us |AA|, |AB|, |AC|, and |AD|
% for the remaining chapters.
%
% Happily this package already defines this presentation command:
%\begin{quote}
%\begin{verbatim}
%\usepackage{alphalph}
%\renewcommand*{\thechapter}{%
%  \AlphAlph{\value{chapter}}%
%}
%\end{verbatim}
%\end{quote}
% \cs{AlphAlph} generates:
%   \AlphAlph{1}, \AlphAlph{2}, \AlphAlph{3}, \dots,
%   \AlphAlph{26}, \AlphAlph{27}, \AlphAlph{28}, \dots
%
% The other presentation command is \cs{alphalph} for lowercase letters.
%
% \subsubsection{Wrap symbols around}
%
% Nine footnote symbols are quite a few.
% Too soon the symbols are consumed
% and \hologo{LaTeX} complains with the error ``Counter too large''.
% However, it could be acceptable to start again with the symbols
% from the beginning, especially if there are less than nine symbols
% on a page.
% This could be achieved by a counter reset.
% But finding the right place can be difficult or needs manual
% actions. Also a
% unique counter value can be desirable (e.g. for generating
% unique anchor/link names).
% Package \xpackage{alphalph} allows you to define
% a macro that implements a ``wrap around'', but letting
% the value of the counter untouched:
%\begin{quote}
%\begin{verbatim}
%\usepackage{alphalph}
%\makeatletter
%\newalphalph{\fnsymbolwrap}[wrap]{\@fnsymbol}{}
%\makeatother
%\renewcommand*{\thefootnote}{%
%  \fnsymbolwrap{\value{footnote}}%
%}
%\end{verbatim}
%\end{quote}
% \cs{fnsymbolwrap} generates:
% \fnsymbolwrap{1}~(1), \fnsymbolwrap{2}~(2), \fnsymbolwrap{3}~(3), \dots,
% \fnsymbolwrap{9}~(9), \fnsymbolwrap{10}~(10), \fnsymbolwrap{11}~11, \dots
%
% \subsubsection{Multiple symbols}
%
% \Hologo{LaTeX}'s standard set of footnote symbols contains doubled
% symbols at the higher positions. Could this principle
% be generalized?
% Yes, but first we need a clean footnote symbol list without
% doubled entries, example:
%\begin{quote}
%\begin{verbatim}
%\usepackage{alphalph}
%\makeatletter
%\newcommand*{\fnsymbolsingle}[1]{%
%  \ensuremath{%
%    \ifcase#1%
%    \or *%
%    \or \dagger
%    \or \ddagger
%    \or \mathsection
%    \or \mathparagraph
%    \else
%      \@ctrerr
%    \fi
%  }%
%}
%\makeatother
%\newalphalph{\fnsymbolmult}[mult]{\fnsymbolsingle}{}
%\renewcommand*{\thefootnote}{%
%  \fnsymbolmult{\value{footnote}}%
%}
%\end{verbatim}
%\end{quote}
% The own definition of \cs{fnsymbolsingle} has the
% advantage that this list can easily modified.
% Otherwise you can use \cs{@fnsymbol} directly,
% because it uses the same first five symbols.
%\begin{quote}
%\begin{verbatim}
%\usepackage{alphalph}
%\makeatletter
%\newalphalph{\fnsymbolmult}[mult]{\@fnsymbol}{5}
%\makeatother
%\renewcommand*{\thefootnote}{%
%  \fnsymbolmult{\value{footnote}}%
%}
%\end{verbatim}
%\end{quote}
% \cs{fnsymbolmult} generates:
% \fnsymbolmult{1}~(1), \fnsymbolmult{2}~(2), \fnsymbolmult{3}~(3),
% \fnsymbolmult{4}~(4), \fnsymbolmult{5}~(5), \fnsymbolmult{6}~(6),
% \dots, \fnsymbolmult{16}~16, \fnsymbolmult{17}~17, \dots
%
% \bigskip
% The same method can also be used for the chapter problem
% in the first discussed use case:
%\begin{quote}
%\begin{verbatim}
%\usepackage{alphalph}
%\makeatletter
%\newalphalph{\AlphMult}[mult]{\@Alph}{26}
%\makeatother
%\renewcommand*{\chapter}{%
%  \AlphMult{\value{chapter}}%
%}
%\end{verbatim}
%\end{quote}
% \cs{AlphMult} then generates AA, BB, CC, and DD for chapters 27--30.
%
% \subsection{Glossary}
%
% \begin{description}
% \item[Counter presentation command] is a macro that
%   expects a \hologo{LaTeX} counter name as argument.
%   Numbers cannot be used. Examples:
%   \cs{arabic}, \cs{alph}, \cs{fnsymbol}.
% \item[Number presentation command] is a macro that
%   expects a number as argument. A number is anything
%   that \hologo{TeX} accepts as number including \cs{value}.
%   Examples: \cs{alphalph}, \cs{AlphAlph}, \cs{alphalph@alph}
%
%   However, \cs{alph} or \cs{fnsymbol} are not number
%   presentation commands because they expect a counter
%   name as argument. Happily \hologo{LaTeX} counter presentation
%   commands internally uses number presentation commands
%   with the same name, but prefixed by `@'. Thus
%   \cs{@alph}, \cs{@fnsymbol} are number presentation commands.
% \item[Symbols provider] is a command that can be used
%   to get a list of symbols. For example, \cs{@Alph}
%   provides the 26 uppercase letters from `A' to `Z'.
%   Basically a symbol provider is a number presentation
%   command, usually with a limited range.
% \item[Number of symbols] is the number of the
%   last symbol slot of a symbol provider. Thus \cs{@Alph}
%   generates 26 symbols, \cs{@fnsymbol} provides 9 symbols.
% \end{description}
%
% \subsection{Package usage}
%
% The package \Package{alphalph} can be used with both \hologo{plainTeX}
% and \hologo{LaTeX}:
% \begin{description}
% \item[\hologo{plainTeX}:] |\input alphalph.sty|
% \item[\hologo{LaTeX2e}:]   |\usepackage{alphalph}|\\
%      There aren't any options.
% \end{description}
%
% \subsection{User commands}
%
% \begin{declcs}{AlphAlph} \M{number}\\
%   \cs{alphalph} \M{number}
% \end{declcs}
% Both macros are number presentation commands that
% expects a number as argument. \hologo{LaTeX} counters
% are used with \cs{value}.
%
% The macros represents a number by letters.
% First single letters |A..Z| are used, then
% two letters |AA..ZZ|, three letters |AAA...ZZZ|, \dots
% follow.
%
% Macro \cs{AlphAlph} uses uppercase letters,
% \cs{alphalph} generates the lowercase variant.
% \begin{quote}
%   \catcode`\|=12
%   \def\entry#1{
%     \texttt{#1}
%     & \texttt{\AlphAlph{#1}}
%     & \texttt{\alphalph{#1}}
%   }
%   \begin{tabular}{@{}l|l|l@{}}
%     \meta{number} & \verb|\AlphAlph{|\meta{number}\verb|}|
%                   & \verb|\alphalph{|\meta{number}\verb|}|\\
%     \hline
%     \entry{1}\\
%     \entry{2}\\
%     \entry{26}\\
%     \entry{27}\\
%     \entry{30}\\
%     \entry{2000}\\
%     \entry{3752127}\\
%     \entry{10786572}\\
%     \entry{2147483647}\\
%   \end{tabular}
% \end{quote}
%
% \begin{declcs}{newalphalph}
%   \M{cmd}
%   |[|\meta{method}|]|
%   \M{symbols provider}
%   \M{number of symbols}
% \end{declcs}
% Macro \cs{newalphalph} defines \meta{cmd} as new
% number presentation command. Like \cs{newcommand} an error
% is thrown, if macro \meta{cmd} already exists.
%
% The \meta{method} is one of |alph|, |wrap|, or |mult|.
% The default is |alph|.
%
% As symbol provider a number presentation command
% can be used, e.g. \cs{@fnsymbol}, \cs{@Alph}, or
% \cs{alphalph@alph}.
%
% The last argument is the number of symbols.
% If the argument is empty, then \cs{newalphalph} tries
% to find this number itself. \hologo{LaTeX}'s number presentation
% commands throw an error message, if the number is too large.
% This error message is put in a macro \cs{@ctrerr}.
% Thus \cs{newalphalph} calls the symbol provider and tests
% a number by typesetting it in a temporary box. The error
% macro \cs{@ctrerr} is catched, it proofs that the number
% is not supported. Also if the width of the result is zero
% the number is considered as unavailable.
%
% The empty argument is useful for potentially variable lists.
% However if the end cannot be detected, then the number of
% symbols must be given. This is also a lot faster. Therefore don't
% let the argument empty without reason.
%
% \subsection{Programmer commands}
%
% \begin{declcs}{alphalph@Alph} \M{number}\\
%   \cs{alphalph@alph} \M{number}
% \end{declcs}
% They are basically the same as \cs{@Alph} and \cs{@alph}.
% Some languages of package \xpackage{babel} redefine
% \hologo{LaTeX}'s macros to include some font setup that
% breaks expandibility. Therefore \cs{AlphAlph} and \cs{alphalph}
% are based on \cs{alphalph@Alph} and \cs{alphalph@alph} to get
% the letters. The behaviour of these symbol providers for numbers
% outside the range |1..26| is undefined.
%
% \subsection{Design principles}
%
% \subsubsection{Number presentation commands}
%
% All number presentation commands that this package defines
% (including \cs{alphalph} and \cs{AlphAlph}) have the following
% properties:
% \begin{itemize}
% \item They are fully expandable. This means that they can safely
%       \begin{itemize}
%       \item be written to a file,
%       \item used in moving arguments (\hologo{LaTeX}: they are \emph{robust}),
%       \item used in a \cmd{\csname}-\cmd{\endcsname} pair.
%       \end{itemize}
% \item If the argument is zero or negative, the commands expand
%       to nothing like \cmd{\romannumeral}.
% \item The argument is a \hologo{TeX} number. Anything that would be
%       accepted by \cmd{\number} is a valid argument:
%       \begin{itemize}
%       \item explicite constants,
%       \item macros that expand to a number,
%       \item count registers, \hologo{LaTeX} counter can used
%             via \cmd{\value}, e.\,g.:\\
%             |\alphalph{\value{page}}|
%       \item \dots
%       \end{itemize}
% \item \hologo{eTeX}'s numeric expressions are supported, if \hologo{eTeX} is
%       available. Then \cs{numexpr} is applied to the argument.
%       Package \cs{calc}'s expressions are not supported.
%       That would violate the expandibility.
% \end{itemize}
%
% \subsubsection{General usability}
%
% \begin{description}
% \item[\hologo{TeX} format:]
%   The package does not depend on \hologo{LaTeX}, it can also be used
%   by \hologo{plainTeX}, for example.
% \item[\hologo{eTeX}:] \eTeX is supported, the macros are shorter and
%   faster. But \eTeX's extensions are not requirements.
%   Without \hologo{eTeX}, just the implementation changes. The properties
%   remain unchanged.
% \end{description}
%
% \StopEventually{
% }
%
% \section{Implementation}
% \subsection{Begin\texorpdfstring{ of package}{}}
%    \begin{macrocode}
%<*package>
%    \end{macrocode}
%    Reload check, especially if the package is not used with \LaTeX.
%    \begin{macrocode}
\begingroup\catcode61\catcode48\catcode32=10\relax%
  \catcode13=5 % ^^M
  \endlinechar=13 %
  \catcode35=6 % #
  \catcode39=12 % '
  \catcode44=12 % ,
  \catcode45=12 % -
  \catcode46=12 % .
  \catcode58=12 % :
  \catcode64=11 % @
  \catcode123=1 % {
  \catcode125=2 % }
  \expandafter\let\expandafter\x\csname ver@alphalph.sty\endcsname
  \ifx\x\relax % plain-TeX, first loading
  \else
    \def\empty{}%
    \ifx\x\empty % LaTeX, first loading,
      % variable is initialized, but \ProvidesPackage not yet seen
    \else
      \expandafter\ifx\csname PackageInfo\endcsname\relax
        \def\x#1#2{%
          \immediate\write-1{Package #1 Info: #2.}%
        }%
      \else
        \def\x#1#2{\PackageInfo{#1}{#2, stopped}}%
      \fi
      \x{alphalph}{The package is already loaded}%
      \aftergroup\endinput
    \fi
  \fi
\endgroup%
%    \end{macrocode}
%    Package identification:
%    \begin{macrocode}
\begingroup\catcode61\catcode48\catcode32=10\relax%
  \catcode13=5 % ^^M
  \endlinechar=13 %
  \catcode35=6 % #
  \catcode39=12 % '
  \catcode40=12 % (
  \catcode41=12 % )
  \catcode44=12 % ,
  \catcode45=12 % -
  \catcode46=12 % .
  \catcode47=12 % /
  \catcode58=12 % :
  \catcode64=11 % @
  \catcode91=12 % [
  \catcode93=12 % ]
  \catcode123=1 % {
  \catcode125=2 % }
  \expandafter\ifx\csname ProvidesPackage\endcsname\relax
    \def\x#1#2#3[#4]{\endgroup
      \immediate\write-1{Package: #3 #4}%
      \xdef#1{#4}%
    }%
  \else
    \def\x#1#2[#3]{\endgroup
      #2[{#3}]%
      \ifx#1\@undefined
        \xdef#1{#3}%
      \fi
      \ifx#1\relax
        \xdef#1{#3}%
      \fi
    }%
  \fi
\expandafter\x\csname ver@alphalph.sty\endcsname
\ProvidesPackage{alphalph}%
  [2016/05/16 v2.5 Convert numbers to letters (HO)]%
%    \end{macrocode}
%
% \subsection{Catcodes}
%
%    \begin{macrocode}
\begingroup\catcode61\catcode48\catcode32=10\relax%
  \catcode13=5 % ^^M
  \endlinechar=13 %
  \catcode123=1 % {
  \catcode125=2 % }
  \catcode64=11 % @
  \def\x{\endgroup
    \expandafter\edef\csname AlPh@AtEnd\endcsname{%
      \endlinechar=\the\endlinechar\relax
      \catcode13=\the\catcode13\relax
      \catcode32=\the\catcode32\relax
      \catcode35=\the\catcode35\relax
      \catcode61=\the\catcode61\relax
      \catcode64=\the\catcode64\relax
      \catcode123=\the\catcode123\relax
      \catcode125=\the\catcode125\relax
    }%
  }%
\x\catcode61\catcode48\catcode32=10\relax%
\catcode13=5 % ^^M
\endlinechar=13 %
\catcode35=6 % #
\catcode64=11 % @
\catcode123=1 % {
\catcode125=2 % }
\def\TMP@EnsureCode#1#2{%
  \edef\AlPh@AtEnd{%
    \AlPh@AtEnd
    \catcode#1=\the\catcode#1\relax
  }%
  \catcode#1=#2\relax
}
\TMP@EnsureCode{33}{12}% !
\TMP@EnsureCode{39}{12}% '
\TMP@EnsureCode{40}{12}% (
\TMP@EnsureCode{41}{12}% )
\TMP@EnsureCode{43}{12}% +
\TMP@EnsureCode{44}{12}% ,
\TMP@EnsureCode{46}{12}% .
\TMP@EnsureCode{47}{12}% /
\TMP@EnsureCode{59}{12}% ;
\TMP@EnsureCode{60}{12}% <
\TMP@EnsureCode{62}{12}% >
\TMP@EnsureCode{91}{12}% [
\TMP@EnsureCode{93}{12}% ]
\TMP@EnsureCode{96}{12}% `
\TMP@EnsureCode{124}{12}% |
\edef\AlPh@AtEnd{\AlPh@AtEnd\noexpand\endinput}
%    \end{macrocode}
%
% \subsection{Package loading}
%
%    \begin{macrocode}
\begingroup\expandafter\expandafter\expandafter\endgroup
\expandafter\ifx\csname RequirePackage\endcsname\relax
  \input infwarerr.sty\relax
  \input intcalc.sty\relax
\else
  \RequirePackage{infwarerr}[2007/09/09]%
  \RequirePackage{intcalc}[2007/09/09]%
\fi
%    \end{macrocode}
%
% \subsection{\hologo{eTeX} detection}
%
%    \begin{macrocode}
\begingroup\expandafter\expandafter\expandafter\endgroup
\expandafter\ifx\csname numexpr\endcsname\relax
  \catcode124=9 % '!': ignore
  \catcode43=14 % '+': comment
\else
  \catcode124=14 % '!': comment
  \catcode43=9   % '+': ignore
\fi
%    \end{macrocode}
%
% \subsection{Help macros}
%
%    \begin{macro}{\AlPh@Error}
%    \begin{macrocode}
\def\AlPh@Error#1{%
  \begingroup
    \escapechar=92 % backslash
    \@PackageError{alphalph}{#1}\@ehc
  \endgroup
}
%    \end{macrocode}
%    \end{macro}
%
%    \begin{macro}{\AlPh@IfDefinable}
%    \begin{macrocode}
\begingroup\expandafter\expandafter\expandafter\endgroup
\expandafter\ifx\csname @ifdefinable\endcsname\relax
  \def\AlPh@IfDefinable#1#2{%
    \ifcase\ifx#1\@undefined\else\ifx#1\relax\else1\fi\fi0 %
      #2%
    \else
      \AlPh@Error{%
        Command \string#1 already defined%
      }%
    \fi
  }%
%    \end{macrocode}
%    \end{macro}
%    \begin{macrocode}
\else
%    \end{macrocode}
%    \begin{macro}{\AlPh@IfDefinable}
%    \begin{macrocode}
  \let\AlPh@IfDefinable\@ifdefinable
%    \end{macrocode}
%    \end{macro}
%    \begin{macrocode}
\fi
%    \end{macrocode}
%
%    \begin{macro}{\@ReturnAfterElseFi}
%    \begin{macro}{\@ReturnAfterFi}
%    The following commands moves the `then' and `else' part respectively
%    behind the |\if|-construct. This prevents a too deep |\if|-nesting
%    and so a \hologo{TeX} capacity error because of a limited input stack size.
%    I use this trick in several packages, so I don't prefix these internal
%    commands in order not to have the same macros with different names.
%    (It saves memory.)
%    \begin{macrocode}
\long\def\@ReturnAfterElseFi#1\else#2\fi{\fi#1}
\long\def\@ReturnAfterFi#1\fi{\fi#1}
%    \end{macrocode}
%    \end{macro}
%    \end{macro}
%
%    \begin{macro}{\@gobblefour}
%    \Hologo{LaTeX} defines commands for eating arguments.
%    Define \cs{@gobblefour} if it is not defined (\hologo{plainTeX}).
%    \begin{macrocode}
\expandafter\ifx\csname @gobblefour\endcsname\relax
  \long\def\@gobblefour#1#2#3#4{}%
\fi
%    \end{macrocode}
%    \end{macro}
%
%    \begin{macro}{AlPh@IfOptArg}
%    \begin{macrocode}
\begingroup\expandafter\expandafter\expandafter\endgroup
\expandafter\ifx\csname kernel@ifnextchar\endcsname\relax
  \begingroup\expandafter\expandafter\expandafter\endgroup
  \expandafter\ifx\csname @ifnextchar\endcsname\relax
    \def\AlPh@IfOptArg#1#2{%
      \def\AlPh@TempA{#1}%
      \def\AlPh@TempB{#2}%
      \futurelet\AlPh@Token\AlPh@IfOptArgNext
    }%
    \let\AlPh@BracketLeft=[%]
    \def\AlPh@IfOptArgNext{%
      \ifx\AlPh@Token\AlPh@BracketLeft
        \expandafter\AlPh@TempA
      \else
        \expandafter\AlPh@TempB
      \fi
    }%
  \else
    \def\AlPh@IfOptArg{\@ifnextchar[}%]
  \fi
\else
  \def\AlPh@IfOptArg{\kernel@ifnextchar[}%]
\fi
%    \end{macrocode}
%    \end{macro}
%
% \subsection{Symbol provider}
%
% \subsubsection{Alphabet}
%
%    The output of \cs{alphalph} and \cs{AlphAlph} should be
%    usable as part of command names (see \cs{@namedef}, \cs{csname}, \dots).
%    Unhappily some languages of package \xpackage{babel}
%    redefine \hologo{LaTeX}'s \cs{@alph} and \cs{@Alph} in a manner that
%    they cannot be used in expandable context any more.
%    Therefore package \xpackage{alphalph} provides its own
%    commands.
%
%    \begin{macro}{\alphalph@Alph}
%    \begin{macro}{\alphalph@alph}
%    The two commands \cmd{\AlPh@Alph} and \cmd{\AlPh@alph} convert a
%    number into a letter (uppercase and lowercase respectivly).
%    The character |@| is used as an error symbol,
%    if the number isn't in the range of 1 until 26.
%    Here we need no space after the number |#1|, because the error
%    symbol |@| for the zero case stops scanning the number.
%    This error symbol should not appear anywhere (except for bugs).
%    \begin{macrocode}
\def\alphalph@Alph#1{%
  \ifcase#1%
    @%
  \or A\or B\or C\or D\or E\or F\or G\or H\or I\or J\or K\or L\or M%
  \or N\or O\or P\or Q\or R\or S\or T\or U\or V\or W\or X\or Y\or Z%
  \else
    \AlPh@ctrerr
    @%
  \fi
}
\def\alphalph@alph#1{%
  \ifcase#1%
    @%
  \or a\or b\or c\or d\or e\or f\or g\or h\or i\or j\or k\or l\or m%
  \or n\or o\or p\or q\or r\or s\or t\or u\or v\or w\or x\or y\or z%
  \else
    \AlPh@ctrerr
    @%
  \fi
}
%    \end{macrocode}
%    \end{macro}
%    \end{macro}
%    \begin{macro}{\AlPh@ctrerr}
%    Macro \cs{AlPh@ctrerr} is used as hook for the algorithm
%    to get the available number of symbols.
%    \begin{macrocode}
\def\AlPh@ctrerr{}
%    \end{macrocode}
%    \end{macro}
%
% \subsection{Finding number of symbols}
%
%    \begin{macro}{\AlPh@GetNumberOfSymbols}
%    \noindent
%    |#1|: symbols provider
%    \begin{macrocode}
\def\AlPh@GetNumberOfSymbols#1{%
  \AlPh@TestNumber1!{#1}%
  \ifAlPh@Unavailable
    \def\AlPh@Number{0}%
    \AlPh@Error{No symbols found}%
  \else
    \def\AlPh@Number{1}%
    \AlPh@ExpSearch2!{#1}%
  \fi
}
%    \end{macrocode}
%    \end{macro}
%    \begin{macro}{\ifAlPh@Unavailable}
%    \begin{macrocode}
\let\ifAlPh@Unavailable\iffalse
\def\AlPh@Unavailabletrue{%
  \global\let\ifAlPh@Unavailable\iftrue
}
\def\AlPh@Unavailablefalse{%
  \global\let\ifAlPh@Unavailable\iffalse
}
%    \end{macrocode}
%    \end{macro}
%    \begin{macro}{\AlPh@TestNumber}
%    |#1|: number to be tested\\
%    |#2|: symbols provider
%    \begin{macrocode}
\def\AlPh@TestNumber#1!#2{%
  \AlPh@Unavailablefalse
  \begingroup
    \setbox0=\hbox{%
      \begingroup % color
        \let\@ctrerr\AlPh@Unavailabletrue
        \let\AlPh@ctrerr\AlPh@Unavailabletrue
        #2{#1}%
      \endgroup
    }%
    \ifdim\wd0=0pt %
      \AlPh@Unavailabletrue
    \fi
  \endgroup
}
%    \end{macrocode}
%    \end{macro}
%    \begin{macro}{\AlPh@ExpSearch}
%    |#1|: number to be tested\\
%    |#2|: symbols provider
%    \begin{macrocode}
\def\AlPh@ExpSearch#1!#2{%
  \let\AlPh@Next\relax
  \AlPh@TestNumber#1!{#2}%
  \ifAlPh@Unavailable
    \expandafter\AlPh@BinSearch\AlPh@Number!#1!{#2}%
  \else
    \def\AlPh@Number{#1}%
    \ifnum#1>1073741823 %
      \AlPh@TestNumber2147483647!{#2}%
      \ifAlPh@Unavailable
        \AlPh@BinSearch#1!2147483647!{#2}%
      \else
        \def\AlPh@Number{0}%
        \AlPh@Error{%
          Maximal symbol number not found%
        }%
      \fi
    \else
      \def\AlPh@Next{%
        \expandafter\AlPh@ExpSearch\number\intcalcShl{#1}!{#2}%
      }%
    \fi
  \fi
  \AlPh@Next
}
%    \end{macrocode}
%    \end{macro}
%    \begin{macro}{\AlPh@BinSearch}
%    |#1|: available number\\
%    |#2|: unavailable number, |#2| $>$ |#1|\\
%    |#3|: symbols provider
%    \begin{macrocode}
\def\AlPh@BinSearch#1!#2!#3{%
  \expandafter\AlPh@ProcessBinSearch
  \number\intcalcShr{\intcalcAdd{#1}{#2}}!%
  #1!#2!{#3}%
}
%    \end{macrocode}
%    \end{macro}
%    \begin{macro}{\AlPh@ProcessBinSearch}
%    |#1|: number to be tested, |#2| $\leq$ |#1| $\leq$ |#3|\\
%    |#2|: available number\\
%    |#3|: unavailable number\\
%    |#4|: symbols provider
%    \begin{macrocode}
\def\AlPh@ProcessBinSearch#1!#2!#3!#4{%
  \let\AlPh@Next\relax
  \ifnum#1>#2 %
    \ifnum#1<#3 %
      \AlPh@TestNumber#1!{#4}%
      \ifAlPh@Unavailable
        \def\AlPh@Next{%
          \AlPh@BinSearch#2!#1!{#4}%
        }%
      \else
        \def\AlPh@Next{%
          \AlPh@BinSearch#1!#3!{#4}%
        }%
      \fi
    \else
      \def\AlPh@Number{#2}%
    \fi
  \else
    \def\AlPh@Number{#2}%
  \fi
  \AlPh@Next
}
%    \end{macrocode}
%    \end{macro}
%
% \subsection{Methods}
%
%    The names of method macros start with \cs{AlPh@Method}.
%    These macros do the main job in converting a number to
%    its representation.
%    A method command is called with three arguments.
%    The first argument is the number of symbols. The
%    second argument is the basic macro for converting
%    a number with limited number range. The last parameter
%    is the number that needs converting.
%
% \subsubsection{Common methods}
%
%    \begin{macro}{\AlPh@CheckPositive}
%    \noindent
%    |#1|: number to be checked
%    |#2|: continuation macro\\
%    |#3|: number of symbols (hidden here)\\
%    |#4|: symbol provider (hidden here)
%    \begin{macrocode}
\def\AlPh@CheckPositive#1!#2{%
  \ifnum#1<1 %
    \expandafter\@gobblefour
  \fi
  #2{#1}%
}
%    \end{macrocode}
%    \end{macro}
%
% \subsubsection{Method `alph'}
%
%    \begin{macro}{\AlPh@Method@alph}
%    \noindent
%    |#1|: number of symbols\\
%    |#2|: symbols provider\\
%    |#3|: number to be converted
%    \begin{macrocode}
\def\AlPh@Method@alph#1#2#3{%
  \expandafter\AlPh@CheckPositive
|   \number#3!%
+   \the\numexpr#3!%
    \AlPh@ProcessAlph
    {#1}{#2}%
}
%    \end{macrocode}
%    \end{macro}
%    \begin{macro}{\AlPh@ProcessAlph}
%    |#1|: current number\\
%    |#2|: number of symbols\\
%    |#3|: symbols provider
%    \begin{macrocode}
\def\AlPh@ProcessAlph#1#2#3{%
  \ifnum#1>#2 %
    \@ReturnAfterElseFi{%
      \expandafter\AlPh@StepAlph\number
        \intcalcInc{%
          \intcalcMod{\intcalcDec{#1}}{#2}%
        }%
      \expandafter!\number
        \intcalcDiv{\intcalcDec{#1}}{#2}%
      !{#2}{#3}%
    }%
  \else
    \@ReturnAfterFi{%
      #3{#1}%
    }%
  \fi
}
%    \end{macrocode}
%    \end{macro}
%    \begin{macro}{\AlPh@StepAlph}
%    |#1|: current last digit\\
%    |#2|: new current number\\
%    |#3|: number of symbols\\
%    |#4|: symbols provider
%    \begin{macrocode}
\def\AlPh@StepAlph#1!#2!#3#4{%
  \AlPh@ProcessAlph{#2}{#3}{#4}%
  #4{#1}%
}
%    \end{macrocode}
%    \end{macro}
%
% \subsubsection{Method `wrap'}
%
%    \begin{macro}{\AlPh@Method@wrap}
%    \noindent
%    |#1|: number of symbols\\
%    |#2|: symbols provider\\
%    |#3|: number to be converted
%    \begin{macrocode}
\def\AlPh@Method@wrap#1#2#3{%
  \expandafter\AlPh@CheckPositive
|   \number#3!%
+   \the\numexpr#3!%
    \AlPh@ProcessWrap
    {#1}{#2}%
}
%    \end{macrocode}
%    \end{macro}
%
%    \begin{macro}{\AlPh@ProcessWrap}
%    |#1|: number to be converted\\
%    |#2|: number of symbols\\
%    |#3|: symbols provider
%    \begin{macrocode}
\def\AlPh@ProcessWrap#1#2#3{%
  \ifnum#1>#2 %
    \@ReturnAfterElseFi{%
      \expandafter\AlPh@StepWrap\number
        \intcalcInc{\intcalcMod{\intcalcDec{#1}}{#2}}%
      !{#3}%
    }%
  \else
    \@ReturnAfterFi{%
      #3{#1}%
    }%
  \fi
}
%    \end{macrocode}
%    \end{macro}
%    \begin{macro}{\AlPh@StepWrap}
%    |#1|: final number\\
%    |#2|: symbols provider
%    \begin{macrocode}
\def\AlPh@StepWrap#1!#2{%
  #2{#1}%
}
%    \end{macrocode}
%    \end{macro}
%
% \subsubsection{Method `mult'}
%
%    After the number of symbols is exhausted,
%    repetitions of the symbol are used.
%    \begin{gather*}
%      x \Def \text{number to be converted}\\
%      n \Def \text{number of symbols}\\
%      r \Def \text{repetition length}\\
%      s \Def \text{symbol slot}\\
%      r = ((x - 1) \div n) + 1\\
%      s = ((x - 1) \mod n) + 1
%    \end{gather*}
%    \begin{macro}{\AlPh@Method@mult}
%    \noindent
%    |#1|: number of symbols\\
%    |#2|: symbols provider\\
%    |#3|: number to be converted
%    \begin{macrocode}
\def\AlPh@Method@mult#1#2#3{%
  \expandafter\AlPh@CheckPositive
|   \number#3!%
+   \the\numexpr#3!%
    \AlPh@ProcessMult
    {#1}{#2}%
}
%    \end{macrocode}
%    \end{macro}
%
%    \begin{macro}{\AlPh@ProcessMult}
%    |#1|: number to be converted\\
%    |#2|: number of symbols\\
%    |#3|: symbols provider
%    \begin{macrocode}
\def\AlPh@ProcessMult#1#2#3{%
  \ifnum#1>#2 %
    \@ReturnAfterElseFi{%
      \expandafter\AlPh@StepMult\romannumeral
        \intcalcInc{\intcalcDiv{\intcalcDec{#1}}{#2}}%
        000%
      \expandafter!\number
        \intcalcInc{\intcalcMod{\intcalcDec{#1}}{#2}}%
      !{#3}%
    }%
  \else
    \@ReturnAfterFi{%
      #3{#1}%
    }%
  \fi
}
%    \end{macrocode}
%    \end{macro}
%    \begin{macro}{\AlPh@StepMult}
%    |#1#2|: repetitions coded as list of character `m'\\
%    |#3|: symbol slot\\
%    |#4|: symbols provider
%    \begin{macrocode}
\def\AlPh@StepMult#1#2!#3!#4{%
  \ifx\\#2\\%
  \else
    \@ReturnAfterFi{%
      \AlPh@StepMult#2!#3!{#4}%
    }%
  \fi
  #4{#3}%
}
%    \end{macrocode}
%    \end{macro}
%
% \subsection{User interface}
%
%    \begin{macro}{\newalphalph}
%    Macro \cs{newalphalph} had three arguments in versions below 2.0.
%    For the new method argument we use an optional argument an first
%    position.\\
%    |#1|: cmd\\
%    |[#2]|: method name: |alph| (default), |wrap|, |mult|\\ % hash-ok
%    |#3|: symbols provider\\
%    |#4|: number of symbols
%    \begin{macrocode}
\AlPh@IfDefinable\newalphalph{%
  \def\newalphalph#1{%
    \AlPh@IfOptArg{%
      \AlPh@newalphalph{#1}%
    }{%
      \AlPh@newalphalph{#1}[alph]%
    }%
  }%
}
%    \end{macrocode}
%    \end{macro}
%    \begin{macro}{\AlPh@newalphalph}
%    |#1|: cmd
%    |#2|: method name\\
%    |#3|: symbols provider\\
%    |#4|: number of symbols
%    \begin{macrocode}
\def\AlPh@newalphalph#1[#2]#3#4{%
  \begingroup\expandafter\expandafter\expandafter\endgroup
  \expandafter\ifx\csname AlPh@Method@#2\endcsname\relax
    \AlPh@Error{%
      Unknown method %
|     `#2'%
+     `\detokenize{#2}'%
    }%
  \else
    \ifx\\#4\\%
      \AlPh@GetNumberOfSymbols{#3}%
      \ifcase\AlPh@Number
      \else
        \begingroup
          \escapechar=92 % backslash
          \@PackageInfo{alphalph}{%
            Number of symbols for \string#1 is \AlPh@Number
          }%
        \endgroup
        \expandafter\AlPh@NewAlphAlph
        \csname AlPh@Method@#2\expandafter\endcsname
        \AlPh@Number!{#1}{#3}%
      \fi
    \else
      \expandafter\AlPh@NewAlphAlph
      \csname AlPh@Method@#2\expandafter\endcsname
|     \number#4!%
+     \the\numexpr#4!%
      {#1}{#3}%
    \fi
  \fi
}%
%    \end{macrocode}
%    \end{macro}
%    \begin{macro}{\AlPh@NewAlphAlph}
%    |#1|: method macro\\
%    |#2|: number of symbols\\
%    |#3|: cmd\\
%    |#4|: symbols provider
%    \begin{macrocode}
\def\AlPh@NewAlphAlph#1#2!#3#4{%
  \AlPh@IfDefinable#3{%
    \ifnum#2>0 %
      \def#3{#1{#2}{#4}}%
    \else
      \AlPh@Error{%
          Definition of \string#3 failed,\MessageBreak
          because number of symbols (#2) is not positive%
        }%
    \fi
  }%
}
%    \end{macrocode}
%    \end{macro}
%
%    \begin{macro}{\AlphAlph}
%    \begin{macrocode}
\newalphalph\AlphAlph\alphalph@Alph{26}
%    \end{macrocode}
%    \end{macro}
%    \begin{macro}{\alphalph}
%    \begin{macrocode}
\newalphalph\alphalph\alphalph@alph{26}
%    \end{macrocode}
%    \end{macro}
%
%    \begin{macrocode}
\AlPh@AtEnd%
%</package>
%    \end{macrocode}
%
% \section{Test}
%
% \subsection{Catcode checks for loading}
%
%    \begin{macrocode}
%<*test1>
%    \end{macrocode}
%    \begin{macrocode}
\catcode`\{=1 %
\catcode`\}=2 %
\catcode`\#=6 %
\catcode`\@=11 %
\expandafter\ifx\csname count@\endcsname\relax
  \countdef\count@=255 %
\fi
\expandafter\ifx\csname @gobble\endcsname\relax
  \long\def\@gobble#1{}%
\fi
\expandafter\ifx\csname @firstofone\endcsname\relax
  \long\def\@firstofone#1{#1}%
\fi
\expandafter\ifx\csname loop\endcsname\relax
  \expandafter\@firstofone
\else
  \expandafter\@gobble
\fi
{%
  \def\loop#1\repeat{%
    \def\body{#1}%
    \iterate
  }%
  \def\iterate{%
    \body
      \let\next\iterate
    \else
      \let\next\relax
    \fi
    \next
  }%
  \let\repeat=\fi
}%
\def\RestoreCatcodes{}
\count@=0 %
\loop
  \edef\RestoreCatcodes{%
    \RestoreCatcodes
    \catcode\the\count@=\the\catcode\count@\relax
  }%
\ifnum\count@<255 %
  \advance\count@ 1 %
\repeat

\def\RangeCatcodeInvalid#1#2{%
  \count@=#1\relax
  \loop
    \catcode\count@=15 %
  \ifnum\count@<#2\relax
    \advance\count@ 1 %
  \repeat
}
\def\RangeCatcodeCheck#1#2#3{%
  \count@=#1\relax
  \loop
    \ifnum#3=\catcode\count@
    \else
      \errmessage{%
        Character \the\count@\space
        with wrong catcode \the\catcode\count@\space
        instead of \number#3%
      }%
    \fi
  \ifnum\count@<#2\relax
    \advance\count@ 1 %
  \repeat
}
\def\space{ }
\expandafter\ifx\csname LoadCommand\endcsname\relax
  \def\LoadCommand{\input alphalph.sty\relax}%
\fi
\def\Test{%
  \RangeCatcodeInvalid{0}{47}%
  \RangeCatcodeInvalid{58}{64}%
  \RangeCatcodeInvalid{91}{96}%
  \RangeCatcodeInvalid{123}{255}%
  \catcode`\@=12 %
  \catcode`\\=0 %
  \catcode`\%=14 %
  \LoadCommand
  \RangeCatcodeCheck{0}{36}{15}%
  \RangeCatcodeCheck{37}{37}{14}%
  \RangeCatcodeCheck{38}{47}{15}%
  \RangeCatcodeCheck{48}{57}{12}%
  \RangeCatcodeCheck{58}{63}{15}%
  \RangeCatcodeCheck{64}{64}{12}%
  \RangeCatcodeCheck{65}{90}{11}%
  \RangeCatcodeCheck{91}{91}{15}%
  \RangeCatcodeCheck{92}{92}{0}%
  \RangeCatcodeCheck{93}{96}{15}%
  \RangeCatcodeCheck{97}{122}{11}%
  \RangeCatcodeCheck{123}{255}{15}%
  \RestoreCatcodes
}
\Test
\csname @@end\endcsname
\end
%    \end{macrocode}
%    \begin{macrocode}
%</test1>
%    \end{macrocode}
%
% \section{Macro tests}
%
%    \begin{macrocode}
%<*test2>
\NeedsTeXFormat{LaTeX2e}
\nofiles
\documentclass{article}
%<*noetex>
\makeatletter
\let\saved@numexpr\numexpr
\newcommand*{\DisableNumexpr}{%
  \let\numexpr\@undefined
}
\newcommand*{\RestoreNumexpr}{%
  \let\numexpr\saved@numexpr
}
\DisableNumexpr
%</noetex>
\usepackage{alphalph}[2016/05/16]
%<noetex>\RestoreNumexpr
\usepackage{qstest}
\IncludeTests{*}
\LogTests{log}{*}{*}

\newcommand*{\TestCmd}[3]{%
  \setbox0=\hbox{%
%<noetex>    \DisableNumexpr
    \edef\TestString{#1{#2}}%
    \expandafter\Expect\expandafter{\TestString}{#3}%
    \edef\TestString{#1{#2} }%
    \expandafter\Expect\expandafter{\TestString}{#3 }%
  }%
  \Expect*{\the\wd0}{0.0pt}%
}

\makeatletter
\newalphalph\LaTeXAlphAlph\@Alph{26}
\newalphalph\LaTeXalphalph\@alph{26}
\newalphalph\AlphWrap[wrap]\alphalph@Alph{26}
\newalphalph\alphwrap[wrap]\alphalph@alph{26}
\newalphalph\LaTeXAlphWrap[wrap]\@Alph{26}
\newalphalph\LaTeXalphwrap[wrap]\@alph{26}
\def\LastSymbol#1{%
  \ifx\\#1\\%
  \else
    \@LastSymbol#1\@nil
  \fi
}
\def\@LastSymbol#1#2\@nil{%
  \ifx\\#2\\%
    #1%
  \else
    \@LastSymbol#2\@nil
  \fi
}
\makeatother
\newcommand*{\TestAlph}[2]{%
  \uppercase{\TestCallCmd\AlphAlph{#2}}{#1}%
  \lowercase{\TestCallCmd\alphalph{#2}}{#1}%
  \uppercase{\TestCallCmd\LaTeXAlphAlph{#2}}{#1}%
  \lowercase{\TestCallCmd\LaTeXalphalph{#2}}{#1}%
  \edef\WrapString{\LastSymbol{#2}}%
  \expandafter\TestAlphWrap\expandafter{\WrapString}{#1}%
}
\newcommand*{\TestAlphWrap}[2]{%
  \uppercase{\TestCallCmd\AlphWrap{#1}}{#2}%
  \lowercase{\TestCallCmd\alphwrap{#1}}{#2}%
  \uppercase{\TestCallCmd\LaTeXAlphWrap{#1}}{#2}%
  \lowercase{\TestCallCmd\LaTeXalphwrap{#1}}{#2}%
}
\newcommand*{\TestCallCmd}[3]{%
  \TestCmd#1{#3}{#2}%
}
\begin{qstest}{AlphSymbols}{alphalph, AlphAlph, symbols}
  \TestAlph{1}{a}%
  \TestAlph{2}{b}%
  \TestAlph{3}{c}%
  \TestAlph{4}{d}%
  \TestAlph{5}{e}%
  \TestAlph{6}{f}%
  \TestAlph{7}{g}%
  \TestAlph{8}{h}%
  \TestAlph{9}{i}%
  \TestAlph{10}{j}%
  \TestAlph{11}{k}%
  \TestAlph{12}{l}%
  \TestAlph{13}{m}%
  \TestAlph{14}{n}%
  \TestAlph{15}{o}%
  \TestAlph{16}{p}%
  \TestAlph{17}{q}%
  \TestAlph{18}{r}%
  \TestAlph{19}{s}%
  \TestAlph{20}{t}%
  \TestAlph{21}{u}%
  \TestAlph{22}{v}%
  \TestAlph{23}{w}%
  \TestAlph{24}{x}%
  \TestAlph{25}{y}%
  \TestAlph{26}{z}%
\end{qstest}
\begin{qstest}{AlphRange}{alphalph, range}
  \TestAlph{0}{}%
  \TestAlph{-1}{}%
  \TestAlph{-2147483647}{}%
  \TestAlph{27}{aa}%
  \TestAlph{28}{ab}%
  \TestAlph{52}{az}%
  \TestAlph{53}{ba}%
  \TestAlph{78}{bz}%
  \TestAlph{79}{ca}%
  \TestAlph{702}{zz}%
  \TestAlph{703}{aaa}%
  \TestAlph{2147483647}{fxshrxw}%
\end{qstest}

\makeatletter
\newcommand*{\myvocals}[1]{%
  \ifcase#1X\or A\or E\or I\or O\or U\else Y\fi
}
\makeatother
\newalphalph\vocalsvocals\myvocals{5}
\newcommand*{\TestVocals}{%
  \TestCmd\vocalsvocals
}
\begin{qstest}{vocals}{vocals}
  \TestVocals{0}{}%
  \TestVocals{1}{A}%
  \TestVocals{2}{E}%
  \TestVocals{3}{I}%
  \TestVocals{4}{O}%
  \TestVocals{5}{U}%
  \TestVocals{6}{AA}%
  \TestVocals{7}{AE}%
  \TestVocals{8}{AI}%
  \TestVocals{9}{AO}%
  \TestVocals{10}{AU}%
  \TestVocals{11}{EA}%
  \TestVocals{24}{OO}%
  \TestVocals{25}{OU}%
  \TestVocals{26}{UA}%
  \TestVocals{29}{UO}%
  \TestVocals{30}{UU}%
  \TestVocals{31}{AAA}%
  \TestVocals{155}{UUU}%
  \TestVocals{156}{AAAA}%
  \TestVocals{2147483647}{AIIOOEEIOIIUOE}%
\end{qstest}

\makeatletter
\newalphalph\AlphMult[mult]{\alphalph@Alph}{26}
\newalphalph\alphmult[mult]{\alphalph@alph}{26}
\newalphalph\LaTeXAlphMult[mult]{\@Alph}{26}
\newalphalph\LaTeXalphmult[mult]{\@alph}{26}
\makeatother
\newcommand*{\TestMult}[2]{%
  \uppercase{\TestCallCmd\AlphMult{#2}}{#1}%
  \lowercase{\TestCallCmd\alphmult{#2}}{#1}%
  \uppercase{\TestCallCmd\LaTeXAlphMult{#2}}{#1}%
  \lowercase{\TestCallCmd\LaTeXalphmult{#2}}{#1}%
}
\begin{qstest}{mult}{mult}
  \TestMult{0}{}%
  \TestMult{-1}{}%
  \TestMult{-2147483647}{}%
  \TestMult{1}{a}%
  \TestMult{2}{b}%
  \TestMult{26}{z}%
  \TestMult{27}{aa}%
  \TestMult{28}{bb}%
  \TestMult{52}{zz}%
  \TestMult{53}{aaa}%
  \TestMult{54}{bbb}%
  \TestMult{259}{yyyyyyyyyy}%
  \TestMult{260}{zzzzzzzzzz}%
  \TestMult{261}{aaaaaaaaaaa}%
  \TestMult{262}{bbbbbbbbbbb}%
\end{qstest}

\def\myvocalsB#1{%
  \ifcase#1\or A\or E\or I\or O\or U\fi
}
\begin{qstest}{symbolnum}{symbolnum}
  \makeatletter
  \def\Test#1#2{%
    \let\TestCmd\relax
    \newalphalph\TestCmd{#1}{}%
    \Expect*{\AlPh@Number}{#2}%
  }%
  \Test\@alph{26}%
  \Test\@Alph{26}%
  \Test\@fnsymbol{9}%
  \Test\myvocalsB{5}%
  \Test\alphalph@alph{26}%
  \Test\alphalph@Alph{26}%
\end{qstest}

\begin{qstest}{list}{list}
  \makeatletter
  \def\catch#1\relax{%
    \def\FoundList{\catch#1}%
  }%
  \def\Test[#1]#2#3#4{%
    \let\testcmd\relax
    \newalphalph\testcmd[{#1}]{\catch}{#2}%
    \testcmd{#3}|\relax
    \expandafter\Expect\expandafter{\FoundList}{#4|}%
    %
    \let\SavedCatch\catch
    \def\catch{\noexpand\catch\noexpand\foo}%
    \edef\Result{#4|}%
    \@onelevel@sanitize\Result
    \let\catch\SavedCatch
    \let\testcmd\relax
    \newalphalph\testcmd[{#1}]{\catch\foo}{#2}%
    \testcmd{#3}|\relax
    \@onelevel@sanitize\FoundList
    \Expect*{\FoundList}*{\Result}%
  }%
  \Test[alph]{26}{3}{\catch{3}}%
  \Test[alph]{26}{12}{\catch{12}}%
  \Test[alph]{26}{27}{\catch{1}\catch{1}}%
  \Test[alph]{26}{78}{\catch{2}\catch{26}}%
  \Test[wrap]{26}{7}{\catch{7}}%
  \Test[wrap]{26}{14}{\catch{14}}%
  \Test[wrap]{26}{80}{\catch{2}}%
  \Test[wrap]{26}{700}{\catch{24}}%
  \Test[mult]{26}{4}{\catch{4}}%
  \Test[mult]{26}{17}{\catch{17}}%
  \Test[mult]{26}{54}{\catch{2}\catch{2}\catch{2}}%
\end{qstest}

\begin{document}
\end{document}
%</test2>
%    \end{macrocode}
%
% \section{Installation}
%
% \subsection{Download}
%
% \paragraph{Package.} This package is available on
% CTAN\footnote{\CTANpkg{alphalph}}:
% \begin{description}
% \item[\CTAN{macros/latex/contrib/oberdiek/alphalph.dtx}] The source file.
% \item[\CTAN{macros/latex/contrib/oberdiek/alphalph.pdf}] Documentation.
% \end{description}
%
%
% \paragraph{Bundle.} All the packages of the bundle `oberdiek'
% are also available in a TDS compliant ZIP archive. There
% the packages are already unpacked and the documentation files
% are generated. The files and directories obey the TDS standard.
% \begin{description}
% \item[\CTANinstall{install/macros/latex/contrib/oberdiek.tds.zip}]
% \end{description}
% \emph{TDS} refers to the standard ``A Directory Structure
% for \TeX\ Files'' (\CTAN{tds/tds.pdf}). Directories
% with \xfile{texmf} in their name are usually organized this way.
%
% \subsection{Bundle installation}
%
% \paragraph{Unpacking.} Unpack the \xfile{oberdiek.tds.zip} in the
% TDS tree (also known as \xfile{texmf} tree) of your choice.
% Example (linux):
% \begin{quote}
%   |unzip oberdiek.tds.zip -d ~/texmf|
% \end{quote}
%
% \paragraph{Script installation.}
% Check the directory \xfile{TDS:scripts/oberdiek/} for
% scripts that need further installation steps.

%
% \subsection{Package installation}
%
% \paragraph{Unpacking.} The \xfile{.dtx} file is a self-extracting
% \docstrip\ archive. The files are extracted by running the
% \xfile{.dtx} through \plainTeX:
% \begin{quote}
%   \verb|tex alphalph.dtx|
% \end{quote}
%
% \paragraph{TDS.} Now the different files must be moved into
% the different directories in your installation TDS tree
% (also known as \xfile{texmf} tree):
% \begin{quote}
% \def\t{^^A
% \begin{tabular}{@{}>{\ttfamily}l@{ $\rightarrow$ }>{\ttfamily}l@{}}
%   alphalph.sty & tex/generic/oberdiek/alphalph.sty\\
%   alphalph.pdf & doc/latex/oberdiek/alphalph.pdf\\
%   test/alphalph-test1.tex & doc/latex/oberdiek/test/alphalph-test1.tex\\
%   test/alphalph-test2.tex & doc/latex/oberdiek/test/alphalph-test2.tex\\
%   test/alphalph-test3.tex & doc/latex/oberdiek/test/alphalph-test3.tex\\
%   alphalph.dtx & source/latex/oberdiek/alphalph.dtx\\
% \end{tabular}^^A
% }^^A
% \sbox0{\t}^^A
% \ifdim\wd0>\linewidth
%   \begingroup
%     \advance\linewidth by\leftmargin
%     \advance\linewidth by\rightmargin
%   \edef\x{\endgroup
%     \def\noexpand\lw{\the\linewidth}^^A
%   }\x
%   \def\lwbox{^^A
%     \leavevmode
%     \hbox to \linewidth{^^A
%       \kern-\leftmargin\relax
%       \hss
%       \usebox0
%       \hss
%       \kern-\rightmargin\relax
%     }^^A
%   }^^A
%   \ifdim\wd0>\lw
%     \sbox0{\small\t}^^A
%     \ifdim\wd0>\linewidth
%       \ifdim\wd0>\lw
%         \sbox0{\footnotesize\t}^^A
%         \ifdim\wd0>\linewidth
%           \ifdim\wd0>\lw
%             \sbox0{\scriptsize\t}^^A
%             \ifdim\wd0>\linewidth
%               \ifdim\wd0>\lw
%                 \sbox0{\tiny\t}^^A
%                 \ifdim\wd0>\linewidth
%                   \lwbox
%                 \else
%                   \usebox0
%                 \fi
%               \else
%                 \lwbox
%               \fi
%             \else
%               \usebox0
%             \fi
%           \else
%             \lwbox
%           \fi
%         \else
%           \usebox0
%         \fi
%       \else
%         \lwbox
%       \fi
%     \else
%       \usebox0
%     \fi
%   \else
%     \lwbox
%   \fi
% \else
%   \usebox0
% \fi
% \end{quote}
% If you have a \xfile{docstrip.cfg} that configures and enables \docstrip's
% TDS installing feature, then some files can already be in the right
% place, see the documentation of \docstrip.
%
% \subsection{Refresh file name databases}
%
% If your \TeX~distribution
% (\teTeX, \mikTeX, \dots) relies on file name databases, you must refresh
% these. For example, \teTeX\ users run \verb|texhash| or
% \verb|mktexlsr|.
%
% \subsection{Some details for the interested}
%
% \paragraph{Unpacking with \LaTeX.}
% The \xfile{.dtx} chooses its action depending on the format:
% \begin{description}
% \item[\plainTeX:] Run \docstrip\ and extract the files.
% \item[\LaTeX:] Generate the documentation.
% \end{description}
% If you insist on using \LaTeX\ for \docstrip\ (really,
% \docstrip\ does not need \LaTeX), then inform the autodetect routine
% about your intention:
% \begin{quote}
%   \verb|latex \let\install=y% \iffalse meta-comment
%
% File: alphalph.dtx
% Version: 2016/05/16 v2.5
% Info: Convert numbers to letters
%
% Copyright (C) 1999, 2006-2008, 2010, 2011 by
%    Heiko Oberdiek <heiko.oberdiek at googlemail.com>
%    2016
%    https://github.com/ho-tex/oberdiek/issues
%
% This work may be distributed and/or modified under the
% conditions of the LaTeX Project Public License, either
% version 1.3c of this license or (at your option) any later
% version. This version of this license is in
%    http://www.latex-project.org/lppl/lppl-1-3c.txt
% and the latest version of this license is in
%    http://www.latex-project.org/lppl.txt
% and version 1.3 or later is part of all distributions of
% LaTeX version 2005/12/01 or later.
%
% This work has the LPPL maintenance status "maintained".
%
% This Current Maintainer of this work is Heiko Oberdiek.
%
% The Base Interpreter refers to any `TeX-Format',
% because some files are installed in TDS:tex/generic//.
%
% This work consists of the main source file alphalph.dtx
% and the derived files
%    alphalph.sty, alphalph.pdf, alphalph.ins, alphalph.drv,
%    alphalph-test1.tex, alphalph-test2.tex, alphalph-test3.tex.
%
% Distribution:
%    CTAN:macros/latex/contrib/oberdiek/alphalph.dtx
%    CTAN:macros/latex/contrib/oberdiek/alphalph.pdf
%
% Unpacking:
%    (a) If alphalph.ins is present:
%           tex alphalph.ins
%    (b) Without alphalph.ins:
%           tex alphalph.dtx
%    (c) If you insist on using LaTeX
%           latex \let\install=y% \iffalse meta-comment
%
% File: alphalph.dtx
% Version: 2016/05/16 v2.5
% Info: Convert numbers to letters
%
% Copyright (C) 1999, 2006-2008, 2010, 2011 by
%    Heiko Oberdiek <heiko.oberdiek at googlemail.com>
%    2016
%    https://github.com/ho-tex/oberdiek/issues
%
% This work may be distributed and/or modified under the
% conditions of the LaTeX Project Public License, either
% version 1.3c of this license or (at your option) any later
% version. This version of this license is in
%    http://www.latex-project.org/lppl/lppl-1-3c.txt
% and the latest version of this license is in
%    http://www.latex-project.org/lppl.txt
% and version 1.3 or later is part of all distributions of
% LaTeX version 2005/12/01 or later.
%
% This work has the LPPL maintenance status "maintained".
%
% This Current Maintainer of this work is Heiko Oberdiek.
%
% The Base Interpreter refers to any `TeX-Format',
% because some files are installed in TDS:tex/generic//.
%
% This work consists of the main source file alphalph.dtx
% and the derived files
%    alphalph.sty, alphalph.pdf, alphalph.ins, alphalph.drv,
%    alphalph-test1.tex, alphalph-test2.tex, alphalph-test3.tex.
%
% Distribution:
%    CTAN:macros/latex/contrib/oberdiek/alphalph.dtx
%    CTAN:macros/latex/contrib/oberdiek/alphalph.pdf
%
% Unpacking:
%    (a) If alphalph.ins is present:
%           tex alphalph.ins
%    (b) Without alphalph.ins:
%           tex alphalph.dtx
%    (c) If you insist on using LaTeX
%           latex \let\install=y% \iffalse meta-comment
%
% File: alphalph.dtx
% Version: 2016/05/16 v2.5
% Info: Convert numbers to letters
%
% Copyright (C) 1999, 2006-2008, 2010, 2011 by
%    Heiko Oberdiek <heiko.oberdiek at googlemail.com>
%    2016
%    https://github.com/ho-tex/oberdiek/issues
%
% This work may be distributed and/or modified under the
% conditions of the LaTeX Project Public License, either
% version 1.3c of this license or (at your option) any later
% version. This version of this license is in
%    http://www.latex-project.org/lppl/lppl-1-3c.txt
% and the latest version of this license is in
%    http://www.latex-project.org/lppl.txt
% and version 1.3 or later is part of all distributions of
% LaTeX version 2005/12/01 or later.
%
% This work has the LPPL maintenance status "maintained".
%
% This Current Maintainer of this work is Heiko Oberdiek.
%
% The Base Interpreter refers to any `TeX-Format',
% because some files are installed in TDS:tex/generic//.
%
% This work consists of the main source file alphalph.dtx
% and the derived files
%    alphalph.sty, alphalph.pdf, alphalph.ins, alphalph.drv,
%    alphalph-test1.tex, alphalph-test2.tex, alphalph-test3.tex.
%
% Distribution:
%    CTAN:macros/latex/contrib/oberdiek/alphalph.dtx
%    CTAN:macros/latex/contrib/oberdiek/alphalph.pdf
%
% Unpacking:
%    (a) If alphalph.ins is present:
%           tex alphalph.ins
%    (b) Without alphalph.ins:
%           tex alphalph.dtx
%    (c) If you insist on using LaTeX
%           latex \let\install=y\input{alphalph.dtx}
%        (quote the arguments according to the demands of your shell)
%
% Documentation:
%    (a) If alphalph.drv is present:
%           latex alphalph.drv
%    (b) Without alphalph.drv:
%           latex alphalph.dtx; ...
%    The class ltxdoc loads the configuration file ltxdoc.cfg
%    if available. Here you can specify further options, e.g.
%    use A4 as paper format:
%       \PassOptionsToClass{a4paper}{article}
%
%    Programm calls to get the documentation (example):
%       pdflatex alphalph.dtx
%       makeindex -s gind.ist alphalph.idx
%       pdflatex alphalph.dtx
%       makeindex -s gind.ist alphalph.idx
%       pdflatex alphalph.dtx
%
% Installation:
%    TDS:tex/generic/oberdiek/alphalph.sty
%    TDS:doc/latex/oberdiek/alphalph.pdf
%    TDS:doc/latex/oberdiek/test/alphalph-test1.tex
%    TDS:doc/latex/oberdiek/test/alphalph-test2.tex
%    TDS:doc/latex/oberdiek/test/alphalph-test3.tex
%    TDS:source/latex/oberdiek/alphalph.dtx
%
%<*ignore>
\begingroup
  \catcode123=1 %
  \catcode125=2 %
  \def\x{LaTeX2e}%
\expandafter\endgroup
\ifcase 0\ifx\install y1\fi\expandafter
         \ifx\csname processbatchFile\endcsname\relax\else1\fi
         \ifx\fmtname\x\else 1\fi\relax
\else\csname fi\endcsname
%</ignore>
%<*install>
\input docstrip.tex
\Msg{************************************************************************}
\Msg{* Installation}
\Msg{* Package: alphalph 2016/05/16 v2.5 Convert numbers to letters (HO)}
\Msg{************************************************************************}

\keepsilent
\askforoverwritefalse

\let\MetaPrefix\relax
\preamble

This is a generated file.

Project: alphalph
Version: 2016/05/16 v2.5

Copyright (C) 1999, 2006-2008, 2010, 2011 by
   Heiko Oberdiek <heiko.oberdiek at googlemail.com>

This work may be distributed and/or modified under the
conditions of the LaTeX Project Public License, either
version 1.3c of this license or (at your option) any later
version. This version of this license is in
   http://www.latex-project.org/lppl/lppl-1-3c.txt
and the latest version of this license is in
   http://www.latex-project.org/lppl.txt
and version 1.3 or later is part of all distributions of
LaTeX version 2005/12/01 or later.

This work has the LPPL maintenance status "maintained".

This Current Maintainer of this work is Heiko Oberdiek.

The Base Interpreter refers to any `TeX-Format',
because some files are installed in TDS:tex/generic//.

This work consists of the main source file alphalph.dtx
and the derived files
   alphalph.sty, alphalph.pdf, alphalph.ins, alphalph.drv,
   alphalph-test1.tex, alphalph-test2.tex, alphalph-test3.tex.

\endpreamble
\let\MetaPrefix\DoubleperCent

\generate{%
  \file{alphalph.ins}{\from{alphalph.dtx}{install}}%
  \file{alphalph.drv}{\from{alphalph.dtx}{driver}}%
  \usedir{tex/generic/oberdiek}%
  \file{alphalph.sty}{\from{alphalph.dtx}{package}}%
%  \usedir{doc/latex/oberdiek/test}%
%  \file{alphalph-test1.tex}{\from{alphalph.dtx}{test1}}%
%  \file{alphalph-test2.tex}{\from{alphalph.dtx}{test2,etex}}%
%  \file{alphalph-test3.tex}{\from{alphalph.dtx}{test2,noetex}}%
  \nopreamble
  \nopostamble
%  \usedir{source/latex/oberdiek/catalogue}%
%  \file{alphalph.xml}{\from{alphalph.dtx}{catalogue}}%
}

\catcode32=13\relax% active space
\let =\space%
\Msg{************************************************************************}
\Msg{*}
\Msg{* To finish the installation you have to move the following}
\Msg{* file into a directory searched by TeX:}
\Msg{*}
\Msg{*     alphalph.sty}
\Msg{*}
\Msg{* To produce the documentation run the file `alphalph.drv'}
\Msg{* through LaTeX.}
\Msg{*}
\Msg{* Happy TeXing!}
\Msg{*}
\Msg{************************************************************************}

\endbatchfile
%</install>
%<*ignore>
\fi
%</ignore>
%<*driver>
\NeedsTeXFormat{LaTeX2e}
\ProvidesFile{alphalph.drv}%
  [2016/05/16 v2.5 Convert numbers to letters (HO)]%
\documentclass{ltxdoc}
\usepackage{alphalph}[2016/05/16]
\makeatletter
\newalphalph{\fnsymbolwrap}[wrap]{\@fnsymbol}{}
\newcommand*{\fnsymbolsingle}[1]{%
  \ensuremath{%
    \ifcase#1%
    \or *%
    \or \dagger
    \or \ddagger
    \or \mathsection
    \or \mathparagraph
    \else
      \@ctrerr
    \fi
  }%
}
\makeatother
\newalphalph{\fnsymbolmult}[mult]{\fnsymbolsingle}{}
\usepackage{amsmath}
\DeclareMathOperator{\opDiv}{div}
\DeclareMathOperator{\opMod}{mod}
\newcommand*{\Def}{%
  \ensuremath{%
    \mathrel{\mathop{:}}=%
  }%
}
\usepackage{holtxdoc}[2011/11/22]
\begin{document}
  \DocInput{alphalph.dtx}%
\end{document}
%</driver>
% \fi
%
%
% \CharacterTable
%  {Upper-case    \A\B\C\D\E\F\G\H\I\J\K\L\M\N\O\P\Q\R\S\T\U\V\W\X\Y\Z
%   Lower-case    \a\b\c\d\e\f\g\h\i\j\k\l\m\n\o\p\q\r\s\t\u\v\w\x\y\z
%   Digits        \0\1\2\3\4\5\6\7\8\9
%   Exclamation   \!     Double quote  \"     Hash (number) \#
%   Dollar        \$     Percent       \%     Ampersand     \&
%   Acute accent  \'     Left paren    \(     Right paren   \)
%   Asterisk      \*     Plus          \+     Comma         \,
%   Minus         \-     Point         \.     Solidus       \/
%   Colon         \:     Semicolon     \;     Less than     \<
%   Equals        \=     Greater than  \>     Question mark \?
%   Commercial at \@     Left bracket  \[     Backslash     \\
%   Right bracket \]     Circumflex    \^     Underscore    \_
%   Grave accent  \`     Left brace    \{     Vertical bar  \|
%   Right brace   \}     Tilde         \~}
%
% \GetFileInfo{alphalph.drv}
%
% \title{The \xpackage{alphalph} package}
% \date{2016/05/16 v2.5}
% \author{Heiko Oberdiek\thanks
% {Please report any issues at \url{https://github.com/ho-tex/oberdiek/issues}}\\
% \xemail{heiko.oberdiek at googlemail.com}}
%
% \maketitle
%
% \begin{abstract}
% The package provides methods to represent numbers with a limited
% set of symbols. Both \hologo{LaTeX} and \hologo{plainTeX} are supported.
% \end{abstract}
%
% \tableofcontents
%
% \section{Documentation}
%
% \subsection{Introduction}
% \hologo{LaTeX} counters can be represented in different ways
% by using presentation commands:
% \begin{quote}
%   \cs{arabic}, \cs{roman}, \cs{Roman},\\
%   \cs{alph}, \cs{Alph}, \cs{fnsymbol}
% \end{quote}
% The ranges of supported counter values are more or less
% restricted. Only \cs{arabic} can be used with any counter
% value \hologo{TeX} supports.
% \begin{quote}
% \catcode`\|=12 %
% \begin{tabular}{@{}l|l|l|l@{}}
% Presentation & Supported & Ignored & Error message\\
% command      & domain    & values  & ``Counter too large''\\
% \hline
% \cs{arabic}
%   & \ttfamily -MAX..MAX & &\\
% \cs{roman}, \cs{Roman}
%   & \ttfamily 1..MAX & \ttfamily -MAX..0 &\\
% \cs{alph}, \cs{Alph}
%   & \ttfamily 1..26 & 0 & \ttfamily -MAX..-1, 27..MAX\\
% \cs{fnsymbol}
%   & \ttfamily 1..9 & 0 & \ttfamily -MAX..-1, 10..MAX\\
% \hline
% \end{tabular}\\[1ex]
% \texttt{MAX} = \texttt{2147483647}
% \end{quote}
% Ordinal numbers are often used in documents: numbering
% of chapters, sections, figures, footnotes and so on.
% The layouter chooses \cs{Alph} for chapter numbers
% and \cs{fnsymbol} for footnotes. But what can be done
% if there are more than 26 chapters or more than 10 footnotes?
% This package \xpackage{alphalph} allows to define new
% presentation commands. They rely on a existing command
% and define presentations for values greater the limits.
% Three different methods are provided by the package.
% In the following use cases they are presentated.
%
% \subsection{Use cases}
%
% \subsubsection{Number system based on symbols}
%
% Asume you are writing a book and your lecturer demands
% that chapter numbers must be letters. But you have already
% 30 chapters and you have only 26 letters?
%
% In the decimal system the situation would be clear.
% If you run out of digits, you are using more digits
% to represent a number. This method can be also be used
% for letters. After chapter 26 with |Z| we us |AA|, |AB|, |AC|, and |AD|
% for the remaining chapters.
%
% Happily this package already defines this presentation command:
%\begin{quote}
%\begin{verbatim}
%\usepackage{alphalph}
%\renewcommand*{\thechapter}{%
%  \AlphAlph{\value{chapter}}%
%}
%\end{verbatim}
%\end{quote}
% \cs{AlphAlph} generates:
%   \AlphAlph{1}, \AlphAlph{2}, \AlphAlph{3}, \dots,
%   \AlphAlph{26}, \AlphAlph{27}, \AlphAlph{28}, \dots
%
% The other presentation command is \cs{alphalph} for lowercase letters.
%
% \subsubsection{Wrap symbols around}
%
% Nine footnote symbols are quite a few.
% Too soon the symbols are consumed
% and \hologo{LaTeX} complains with the error ``Counter too large''.
% However, it could be acceptable to start again with the symbols
% from the beginning, especially if there are less than nine symbols
% on a page.
% This could be achieved by a counter reset.
% But finding the right place can be difficult or needs manual
% actions. Also a
% unique counter value can be desirable (e.g. for generating
% unique anchor/link names).
% Package \xpackage{alphalph} allows you to define
% a macro that implements a ``wrap around'', but letting
% the value of the counter untouched:
%\begin{quote}
%\begin{verbatim}
%\usepackage{alphalph}
%\makeatletter
%\newalphalph{\fnsymbolwrap}[wrap]{\@fnsymbol}{}
%\makeatother
%\renewcommand*{\thefootnote}{%
%  \fnsymbolwrap{\value{footnote}}%
%}
%\end{verbatim}
%\end{quote}
% \cs{fnsymbolwrap} generates:
% \fnsymbolwrap{1}~(1), \fnsymbolwrap{2}~(2), \fnsymbolwrap{3}~(3), \dots,
% \fnsymbolwrap{9}~(9), \fnsymbolwrap{10}~(10), \fnsymbolwrap{11}~11, \dots
%
% \subsubsection{Multiple symbols}
%
% \Hologo{LaTeX}'s standard set of footnote symbols contains doubled
% symbols at the higher positions. Could this principle
% be generalized?
% Yes, but first we need a clean footnote symbol list without
% doubled entries, example:
%\begin{quote}
%\begin{verbatim}
%\usepackage{alphalph}
%\makeatletter
%\newcommand*{\fnsymbolsingle}[1]{%
%  \ensuremath{%
%    \ifcase#1%
%    \or *%
%    \or \dagger
%    \or \ddagger
%    \or \mathsection
%    \or \mathparagraph
%    \else
%      \@ctrerr
%    \fi
%  }%
%}
%\makeatother
%\newalphalph{\fnsymbolmult}[mult]{\fnsymbolsingle}{}
%\renewcommand*{\thefootnote}{%
%  \fnsymbolmult{\value{footnote}}%
%}
%\end{verbatim}
%\end{quote}
% The own definition of \cs{fnsymbolsingle} has the
% advantage that this list can easily modified.
% Otherwise you can use \cs{@fnsymbol} directly,
% because it uses the same first five symbols.
%\begin{quote}
%\begin{verbatim}
%\usepackage{alphalph}
%\makeatletter
%\newalphalph{\fnsymbolmult}[mult]{\@fnsymbol}{5}
%\makeatother
%\renewcommand*{\thefootnote}{%
%  \fnsymbolmult{\value{footnote}}%
%}
%\end{verbatim}
%\end{quote}
% \cs{fnsymbolmult} generates:
% \fnsymbolmult{1}~(1), \fnsymbolmult{2}~(2), \fnsymbolmult{3}~(3),
% \fnsymbolmult{4}~(4), \fnsymbolmult{5}~(5), \fnsymbolmult{6}~(6),
% \dots, \fnsymbolmult{16}~16, \fnsymbolmult{17}~17, \dots
%
% \bigskip
% The same method can also be used for the chapter problem
% in the first discussed use case:
%\begin{quote}
%\begin{verbatim}
%\usepackage{alphalph}
%\makeatletter
%\newalphalph{\AlphMult}[mult]{\@Alph}{26}
%\makeatother
%\renewcommand*{\chapter}{%
%  \AlphMult{\value{chapter}}%
%}
%\end{verbatim}
%\end{quote}
% \cs{AlphMult} then generates AA, BB, CC, and DD for chapters 27--30.
%
% \subsection{Glossary}
%
% \begin{description}
% \item[Counter presentation command] is a macro that
%   expects a \hologo{LaTeX} counter name as argument.
%   Numbers cannot be used. Examples:
%   \cs{arabic}, \cs{alph}, \cs{fnsymbol}.
% \item[Number presentation command] is a macro that
%   expects a number as argument. A number is anything
%   that \hologo{TeX} accepts as number including \cs{value}.
%   Examples: \cs{alphalph}, \cs{AlphAlph}, \cs{alphalph@alph}
%
%   However, \cs{alph} or \cs{fnsymbol} are not number
%   presentation commands because they expect a counter
%   name as argument. Happily \hologo{LaTeX} counter presentation
%   commands internally uses number presentation commands
%   with the same name, but prefixed by `@'. Thus
%   \cs{@alph}, \cs{@fnsymbol} are number presentation commands.
% \item[Symbols provider] is a command that can be used
%   to get a list of symbols. For example, \cs{@Alph}
%   provides the 26 uppercase letters from `A' to `Z'.
%   Basically a symbol provider is a number presentation
%   command, usually with a limited range.
% \item[Number of symbols] is the number of the
%   last symbol slot of a symbol provider. Thus \cs{@Alph}
%   generates 26 symbols, \cs{@fnsymbol} provides 9 symbols.
% \end{description}
%
% \subsection{Package usage}
%
% The package \Package{alphalph} can be used with both \hologo{plainTeX}
% and \hologo{LaTeX}:
% \begin{description}
% \item[\hologo{plainTeX}:] |\input alphalph.sty|
% \item[\hologo{LaTeX2e}:]   |\usepackage{alphalph}|\\
%      There aren't any options.
% \end{description}
%
% \subsection{User commands}
%
% \begin{declcs}{AlphAlph} \M{number}\\
%   \cs{alphalph} \M{number}
% \end{declcs}
% Both macros are number presentation commands that
% expects a number as argument. \hologo{LaTeX} counters
% are used with \cs{value}.
%
% The macros represents a number by letters.
% First single letters |A..Z| are used, then
% two letters |AA..ZZ|, three letters |AAA...ZZZ|, \dots
% follow.
%
% Macro \cs{AlphAlph} uses uppercase letters,
% \cs{alphalph} generates the lowercase variant.
% \begin{quote}
%   \catcode`\|=12
%   \def\entry#1{
%     \texttt{#1}
%     & \texttt{\AlphAlph{#1}}
%     & \texttt{\alphalph{#1}}
%   }
%   \begin{tabular}{@{}l|l|l@{}}
%     \meta{number} & \verb|\AlphAlph{|\meta{number}\verb|}|
%                   & \verb|\alphalph{|\meta{number}\verb|}|\\
%     \hline
%     \entry{1}\\
%     \entry{2}\\
%     \entry{26}\\
%     \entry{27}\\
%     \entry{30}\\
%     \entry{2000}\\
%     \entry{3752127}\\
%     \entry{10786572}\\
%     \entry{2147483647}\\
%   \end{tabular}
% \end{quote}
%
% \begin{declcs}{newalphalph}
%   \M{cmd}
%   |[|\meta{method}|]|
%   \M{symbols provider}
%   \M{number of symbols}
% \end{declcs}
% Macro \cs{newalphalph} defines \meta{cmd} as new
% number presentation command. Like \cs{newcommand} an error
% is thrown, if macro \meta{cmd} already exists.
%
% The \meta{method} is one of |alph|, |wrap|, or |mult|.
% The default is |alph|.
%
% As symbol provider a number presentation command
% can be used, e.g. \cs{@fnsymbol}, \cs{@Alph}, or
% \cs{alphalph@alph}.
%
% The last argument is the number of symbols.
% If the argument is empty, then \cs{newalphalph} tries
% to find this number itself. \hologo{LaTeX}'s number presentation
% commands throw an error message, if the number is too large.
% This error message is put in a macro \cs{@ctrerr}.
% Thus \cs{newalphalph} calls the symbol provider and tests
% a number by typesetting it in a temporary box. The error
% macro \cs{@ctrerr} is catched, it proofs that the number
% is not supported. Also if the width of the result is zero
% the number is considered as unavailable.
%
% The empty argument is useful for potentially variable lists.
% However if the end cannot be detected, then the number of
% symbols must be given. This is also a lot faster. Therefore don't
% let the argument empty without reason.
%
% \subsection{Programmer commands}
%
% \begin{declcs}{alphalph@Alph} \M{number}\\
%   \cs{alphalph@alph} \M{number}
% \end{declcs}
% They are basically the same as \cs{@Alph} and \cs{@alph}.
% Some languages of package \xpackage{babel} redefine
% \hologo{LaTeX}'s macros to include some font setup that
% breaks expandibility. Therefore \cs{AlphAlph} and \cs{alphalph}
% are based on \cs{alphalph@Alph} and \cs{alphalph@alph} to get
% the letters. The behaviour of these symbol providers for numbers
% outside the range |1..26| is undefined.
%
% \subsection{Design principles}
%
% \subsubsection{Number presentation commands}
%
% All number presentation commands that this package defines
% (including \cs{alphalph} and \cs{AlphAlph}) have the following
% properties:
% \begin{itemize}
% \item They are fully expandable. This means that they can safely
%       \begin{itemize}
%       \item be written to a file,
%       \item used in moving arguments (\hologo{LaTeX}: they are \emph{robust}),
%       \item used in a \cmd{\csname}-\cmd{\endcsname} pair.
%       \end{itemize}
% \item If the argument is zero or negative, the commands expand
%       to nothing like \cmd{\romannumeral}.
% \item The argument is a \hologo{TeX} number. Anything that would be
%       accepted by \cmd{\number} is a valid argument:
%       \begin{itemize}
%       \item explicite constants,
%       \item macros that expand to a number,
%       \item count registers, \hologo{LaTeX} counter can used
%             via \cmd{\value}, e.\,g.:\\
%             |\alphalph{\value{page}}|
%       \item \dots
%       \end{itemize}
% \item \hologo{eTeX}'s numeric expressions are supported, if \hologo{eTeX} is
%       available. Then \cs{numexpr} is applied to the argument.
%       Package \cs{calc}'s expressions are not supported.
%       That would violate the expandibility.
% \end{itemize}
%
% \subsubsection{General usability}
%
% \begin{description}
% \item[\hologo{TeX} format:]
%   The package does not depend on \hologo{LaTeX}, it can also be used
%   by \hologo{plainTeX}, for example.
% \item[\hologo{eTeX}:] \eTeX is supported, the macros are shorter and
%   faster. But \eTeX's extensions are not requirements.
%   Without \hologo{eTeX}, just the implementation changes. The properties
%   remain unchanged.
% \end{description}
%
% \StopEventually{
% }
%
% \section{Implementation}
% \subsection{Begin\texorpdfstring{ of package}{}}
%    \begin{macrocode}
%<*package>
%    \end{macrocode}
%    Reload check, especially if the package is not used with \LaTeX.
%    \begin{macrocode}
\begingroup\catcode61\catcode48\catcode32=10\relax%
  \catcode13=5 % ^^M
  \endlinechar=13 %
  \catcode35=6 % #
  \catcode39=12 % '
  \catcode44=12 % ,
  \catcode45=12 % -
  \catcode46=12 % .
  \catcode58=12 % :
  \catcode64=11 % @
  \catcode123=1 % {
  \catcode125=2 % }
  \expandafter\let\expandafter\x\csname ver@alphalph.sty\endcsname
  \ifx\x\relax % plain-TeX, first loading
  \else
    \def\empty{}%
    \ifx\x\empty % LaTeX, first loading,
      % variable is initialized, but \ProvidesPackage not yet seen
    \else
      \expandafter\ifx\csname PackageInfo\endcsname\relax
        \def\x#1#2{%
          \immediate\write-1{Package #1 Info: #2.}%
        }%
      \else
        \def\x#1#2{\PackageInfo{#1}{#2, stopped}}%
      \fi
      \x{alphalph}{The package is already loaded}%
      \aftergroup\endinput
    \fi
  \fi
\endgroup%
%    \end{macrocode}
%    Package identification:
%    \begin{macrocode}
\begingroup\catcode61\catcode48\catcode32=10\relax%
  \catcode13=5 % ^^M
  \endlinechar=13 %
  \catcode35=6 % #
  \catcode39=12 % '
  \catcode40=12 % (
  \catcode41=12 % )
  \catcode44=12 % ,
  \catcode45=12 % -
  \catcode46=12 % .
  \catcode47=12 % /
  \catcode58=12 % :
  \catcode64=11 % @
  \catcode91=12 % [
  \catcode93=12 % ]
  \catcode123=1 % {
  \catcode125=2 % }
  \expandafter\ifx\csname ProvidesPackage\endcsname\relax
    \def\x#1#2#3[#4]{\endgroup
      \immediate\write-1{Package: #3 #4}%
      \xdef#1{#4}%
    }%
  \else
    \def\x#1#2[#3]{\endgroup
      #2[{#3}]%
      \ifx#1\@undefined
        \xdef#1{#3}%
      \fi
      \ifx#1\relax
        \xdef#1{#3}%
      \fi
    }%
  \fi
\expandafter\x\csname ver@alphalph.sty\endcsname
\ProvidesPackage{alphalph}%
  [2016/05/16 v2.5 Convert numbers to letters (HO)]%
%    \end{macrocode}
%
% \subsection{Catcodes}
%
%    \begin{macrocode}
\begingroup\catcode61\catcode48\catcode32=10\relax%
  \catcode13=5 % ^^M
  \endlinechar=13 %
  \catcode123=1 % {
  \catcode125=2 % }
  \catcode64=11 % @
  \def\x{\endgroup
    \expandafter\edef\csname AlPh@AtEnd\endcsname{%
      \endlinechar=\the\endlinechar\relax
      \catcode13=\the\catcode13\relax
      \catcode32=\the\catcode32\relax
      \catcode35=\the\catcode35\relax
      \catcode61=\the\catcode61\relax
      \catcode64=\the\catcode64\relax
      \catcode123=\the\catcode123\relax
      \catcode125=\the\catcode125\relax
    }%
  }%
\x\catcode61\catcode48\catcode32=10\relax%
\catcode13=5 % ^^M
\endlinechar=13 %
\catcode35=6 % #
\catcode64=11 % @
\catcode123=1 % {
\catcode125=2 % }
\def\TMP@EnsureCode#1#2{%
  \edef\AlPh@AtEnd{%
    \AlPh@AtEnd
    \catcode#1=\the\catcode#1\relax
  }%
  \catcode#1=#2\relax
}
\TMP@EnsureCode{33}{12}% !
\TMP@EnsureCode{39}{12}% '
\TMP@EnsureCode{40}{12}% (
\TMP@EnsureCode{41}{12}% )
\TMP@EnsureCode{43}{12}% +
\TMP@EnsureCode{44}{12}% ,
\TMP@EnsureCode{46}{12}% .
\TMP@EnsureCode{47}{12}% /
\TMP@EnsureCode{59}{12}% ;
\TMP@EnsureCode{60}{12}% <
\TMP@EnsureCode{62}{12}% >
\TMP@EnsureCode{91}{12}% [
\TMP@EnsureCode{93}{12}% ]
\TMP@EnsureCode{96}{12}% `
\TMP@EnsureCode{124}{12}% |
\edef\AlPh@AtEnd{\AlPh@AtEnd\noexpand\endinput}
%    \end{macrocode}
%
% \subsection{Package loading}
%
%    \begin{macrocode}
\begingroup\expandafter\expandafter\expandafter\endgroup
\expandafter\ifx\csname RequirePackage\endcsname\relax
  \input infwarerr.sty\relax
  \input intcalc.sty\relax
\else
  \RequirePackage{infwarerr}[2007/09/09]%
  \RequirePackage{intcalc}[2007/09/09]%
\fi
%    \end{macrocode}
%
% \subsection{\hologo{eTeX} detection}
%
%    \begin{macrocode}
\begingroup\expandafter\expandafter\expandafter\endgroup
\expandafter\ifx\csname numexpr\endcsname\relax
  \catcode124=9 % '!': ignore
  \catcode43=14 % '+': comment
\else
  \catcode124=14 % '!': comment
  \catcode43=9   % '+': ignore
\fi
%    \end{macrocode}
%
% \subsection{Help macros}
%
%    \begin{macro}{\AlPh@Error}
%    \begin{macrocode}
\def\AlPh@Error#1{%
  \begingroup
    \escapechar=92 % backslash
    \@PackageError{alphalph}{#1}\@ehc
  \endgroup
}
%    \end{macrocode}
%    \end{macro}
%
%    \begin{macro}{\AlPh@IfDefinable}
%    \begin{macrocode}
\begingroup\expandafter\expandafter\expandafter\endgroup
\expandafter\ifx\csname @ifdefinable\endcsname\relax
  \def\AlPh@IfDefinable#1#2{%
    \ifcase\ifx#1\@undefined\else\ifx#1\relax\else1\fi\fi0 %
      #2%
    \else
      \AlPh@Error{%
        Command \string#1 already defined%
      }%
    \fi
  }%
%    \end{macrocode}
%    \end{macro}
%    \begin{macrocode}
\else
%    \end{macrocode}
%    \begin{macro}{\AlPh@IfDefinable}
%    \begin{macrocode}
  \let\AlPh@IfDefinable\@ifdefinable
%    \end{macrocode}
%    \end{macro}
%    \begin{macrocode}
\fi
%    \end{macrocode}
%
%    \begin{macro}{\@ReturnAfterElseFi}
%    \begin{macro}{\@ReturnAfterFi}
%    The following commands moves the `then' and `else' part respectively
%    behind the |\if|-construct. This prevents a too deep |\if|-nesting
%    and so a \hologo{TeX} capacity error because of a limited input stack size.
%    I use this trick in several packages, so I don't prefix these internal
%    commands in order not to have the same macros with different names.
%    (It saves memory.)
%    \begin{macrocode}
\long\def\@ReturnAfterElseFi#1\else#2\fi{\fi#1}
\long\def\@ReturnAfterFi#1\fi{\fi#1}
%    \end{macrocode}
%    \end{macro}
%    \end{macro}
%
%    \begin{macro}{\@gobblefour}
%    \Hologo{LaTeX} defines commands for eating arguments.
%    Define \cs{@gobblefour} if it is not defined (\hologo{plainTeX}).
%    \begin{macrocode}
\expandafter\ifx\csname @gobblefour\endcsname\relax
  \long\def\@gobblefour#1#2#3#4{}%
\fi
%    \end{macrocode}
%    \end{macro}
%
%    \begin{macro}{AlPh@IfOptArg}
%    \begin{macrocode}
\begingroup\expandafter\expandafter\expandafter\endgroup
\expandafter\ifx\csname kernel@ifnextchar\endcsname\relax
  \begingroup\expandafter\expandafter\expandafter\endgroup
  \expandafter\ifx\csname @ifnextchar\endcsname\relax
    \def\AlPh@IfOptArg#1#2{%
      \def\AlPh@TempA{#1}%
      \def\AlPh@TempB{#2}%
      \futurelet\AlPh@Token\AlPh@IfOptArgNext
    }%
    \let\AlPh@BracketLeft=[%]
    \def\AlPh@IfOptArgNext{%
      \ifx\AlPh@Token\AlPh@BracketLeft
        \expandafter\AlPh@TempA
      \else
        \expandafter\AlPh@TempB
      \fi
    }%
  \else
    \def\AlPh@IfOptArg{\@ifnextchar[}%]
  \fi
\else
  \def\AlPh@IfOptArg{\kernel@ifnextchar[}%]
\fi
%    \end{macrocode}
%    \end{macro}
%
% \subsection{Symbol provider}
%
% \subsubsection{Alphabet}
%
%    The output of \cs{alphalph} and \cs{AlphAlph} should be
%    usable as part of command names (see \cs{@namedef}, \cs{csname}, \dots).
%    Unhappily some languages of package \xpackage{babel}
%    redefine \hologo{LaTeX}'s \cs{@alph} and \cs{@Alph} in a manner that
%    they cannot be used in expandable context any more.
%    Therefore package \xpackage{alphalph} provides its own
%    commands.
%
%    \begin{macro}{\alphalph@Alph}
%    \begin{macro}{\alphalph@alph}
%    The two commands \cmd{\AlPh@Alph} and \cmd{\AlPh@alph} convert a
%    number into a letter (uppercase and lowercase respectivly).
%    The character |@| is used as an error symbol,
%    if the number isn't in the range of 1 until 26.
%    Here we need no space after the number |#1|, because the error
%    symbol |@| for the zero case stops scanning the number.
%    This error symbol should not appear anywhere (except for bugs).
%    \begin{macrocode}
\def\alphalph@Alph#1{%
  \ifcase#1%
    @%
  \or A\or B\or C\or D\or E\or F\or G\or H\or I\or J\or K\or L\or M%
  \or N\or O\or P\or Q\or R\or S\or T\or U\or V\or W\or X\or Y\or Z%
  \else
    \AlPh@ctrerr
    @%
  \fi
}
\def\alphalph@alph#1{%
  \ifcase#1%
    @%
  \or a\or b\or c\or d\or e\or f\or g\or h\or i\or j\or k\or l\or m%
  \or n\or o\or p\or q\or r\or s\or t\or u\or v\or w\or x\or y\or z%
  \else
    \AlPh@ctrerr
    @%
  \fi
}
%    \end{macrocode}
%    \end{macro}
%    \end{macro}
%    \begin{macro}{\AlPh@ctrerr}
%    Macro \cs{AlPh@ctrerr} is used as hook for the algorithm
%    to get the available number of symbols.
%    \begin{macrocode}
\def\AlPh@ctrerr{}
%    \end{macrocode}
%    \end{macro}
%
% \subsection{Finding number of symbols}
%
%    \begin{macro}{\AlPh@GetNumberOfSymbols}
%    \noindent
%    |#1|: symbols provider
%    \begin{macrocode}
\def\AlPh@GetNumberOfSymbols#1{%
  \AlPh@TestNumber1!{#1}%
  \ifAlPh@Unavailable
    \def\AlPh@Number{0}%
    \AlPh@Error{No symbols found}%
  \else
    \def\AlPh@Number{1}%
    \AlPh@ExpSearch2!{#1}%
  \fi
}
%    \end{macrocode}
%    \end{macro}
%    \begin{macro}{\ifAlPh@Unavailable}
%    \begin{macrocode}
\let\ifAlPh@Unavailable\iffalse
\def\AlPh@Unavailabletrue{%
  \global\let\ifAlPh@Unavailable\iftrue
}
\def\AlPh@Unavailablefalse{%
  \global\let\ifAlPh@Unavailable\iffalse
}
%    \end{macrocode}
%    \end{macro}
%    \begin{macro}{\AlPh@TestNumber}
%    |#1|: number to be tested\\
%    |#2|: symbols provider
%    \begin{macrocode}
\def\AlPh@TestNumber#1!#2{%
  \AlPh@Unavailablefalse
  \begingroup
    \setbox0=\hbox{%
      \begingroup % color
        \let\@ctrerr\AlPh@Unavailabletrue
        \let\AlPh@ctrerr\AlPh@Unavailabletrue
        #2{#1}%
      \endgroup
    }%
    \ifdim\wd0=0pt %
      \AlPh@Unavailabletrue
    \fi
  \endgroup
}
%    \end{macrocode}
%    \end{macro}
%    \begin{macro}{\AlPh@ExpSearch}
%    |#1|: number to be tested\\
%    |#2|: symbols provider
%    \begin{macrocode}
\def\AlPh@ExpSearch#1!#2{%
  \let\AlPh@Next\relax
  \AlPh@TestNumber#1!{#2}%
  \ifAlPh@Unavailable
    \expandafter\AlPh@BinSearch\AlPh@Number!#1!{#2}%
  \else
    \def\AlPh@Number{#1}%
    \ifnum#1>1073741823 %
      \AlPh@TestNumber2147483647!{#2}%
      \ifAlPh@Unavailable
        \AlPh@BinSearch#1!2147483647!{#2}%
      \else
        \def\AlPh@Number{0}%
        \AlPh@Error{%
          Maximal symbol number not found%
        }%
      \fi
    \else
      \def\AlPh@Next{%
        \expandafter\AlPh@ExpSearch\number\intcalcShl{#1}!{#2}%
      }%
    \fi
  \fi
  \AlPh@Next
}
%    \end{macrocode}
%    \end{macro}
%    \begin{macro}{\AlPh@BinSearch}
%    |#1|: available number\\
%    |#2|: unavailable number, |#2| $>$ |#1|\\
%    |#3|: symbols provider
%    \begin{macrocode}
\def\AlPh@BinSearch#1!#2!#3{%
  \expandafter\AlPh@ProcessBinSearch
  \number\intcalcShr{\intcalcAdd{#1}{#2}}!%
  #1!#2!{#3}%
}
%    \end{macrocode}
%    \end{macro}
%    \begin{macro}{\AlPh@ProcessBinSearch}
%    |#1|: number to be tested, |#2| $\leq$ |#1| $\leq$ |#3|\\
%    |#2|: available number\\
%    |#3|: unavailable number\\
%    |#4|: symbols provider
%    \begin{macrocode}
\def\AlPh@ProcessBinSearch#1!#2!#3!#4{%
  \let\AlPh@Next\relax
  \ifnum#1>#2 %
    \ifnum#1<#3 %
      \AlPh@TestNumber#1!{#4}%
      \ifAlPh@Unavailable
        \def\AlPh@Next{%
          \AlPh@BinSearch#2!#1!{#4}%
        }%
      \else
        \def\AlPh@Next{%
          \AlPh@BinSearch#1!#3!{#4}%
        }%
      \fi
    \else
      \def\AlPh@Number{#2}%
    \fi
  \else
    \def\AlPh@Number{#2}%
  \fi
  \AlPh@Next
}
%    \end{macrocode}
%    \end{macro}
%
% \subsection{Methods}
%
%    The names of method macros start with \cs{AlPh@Method}.
%    These macros do the main job in converting a number to
%    its representation.
%    A method command is called with three arguments.
%    The first argument is the number of symbols. The
%    second argument is the basic macro for converting
%    a number with limited number range. The last parameter
%    is the number that needs converting.
%
% \subsubsection{Common methods}
%
%    \begin{macro}{\AlPh@CheckPositive}
%    \noindent
%    |#1|: number to be checked
%    |#2|: continuation macro\\
%    |#3|: number of symbols (hidden here)\\
%    |#4|: symbol provider (hidden here)
%    \begin{macrocode}
\def\AlPh@CheckPositive#1!#2{%
  \ifnum#1<1 %
    \expandafter\@gobblefour
  \fi
  #2{#1}%
}
%    \end{macrocode}
%    \end{macro}
%
% \subsubsection{Method `alph'}
%
%    \begin{macro}{\AlPh@Method@alph}
%    \noindent
%    |#1|: number of symbols\\
%    |#2|: symbols provider\\
%    |#3|: number to be converted
%    \begin{macrocode}
\def\AlPh@Method@alph#1#2#3{%
  \expandafter\AlPh@CheckPositive
|   \number#3!%
+   \the\numexpr#3!%
    \AlPh@ProcessAlph
    {#1}{#2}%
}
%    \end{macrocode}
%    \end{macro}
%    \begin{macro}{\AlPh@ProcessAlph}
%    |#1|: current number\\
%    |#2|: number of symbols\\
%    |#3|: symbols provider
%    \begin{macrocode}
\def\AlPh@ProcessAlph#1#2#3{%
  \ifnum#1>#2 %
    \@ReturnAfterElseFi{%
      \expandafter\AlPh@StepAlph\number
        \intcalcInc{%
          \intcalcMod{\intcalcDec{#1}}{#2}%
        }%
      \expandafter!\number
        \intcalcDiv{\intcalcDec{#1}}{#2}%
      !{#2}{#3}%
    }%
  \else
    \@ReturnAfterFi{%
      #3{#1}%
    }%
  \fi
}
%    \end{macrocode}
%    \end{macro}
%    \begin{macro}{\AlPh@StepAlph}
%    |#1|: current last digit\\
%    |#2|: new current number\\
%    |#3|: number of symbols\\
%    |#4|: symbols provider
%    \begin{macrocode}
\def\AlPh@StepAlph#1!#2!#3#4{%
  \AlPh@ProcessAlph{#2}{#3}{#4}%
  #4{#1}%
}
%    \end{macrocode}
%    \end{macro}
%
% \subsubsection{Method `wrap'}
%
%    \begin{macro}{\AlPh@Method@wrap}
%    \noindent
%    |#1|: number of symbols\\
%    |#2|: symbols provider\\
%    |#3|: number to be converted
%    \begin{macrocode}
\def\AlPh@Method@wrap#1#2#3{%
  \expandafter\AlPh@CheckPositive
|   \number#3!%
+   \the\numexpr#3!%
    \AlPh@ProcessWrap
    {#1}{#2}%
}
%    \end{macrocode}
%    \end{macro}
%
%    \begin{macro}{\AlPh@ProcessWrap}
%    |#1|: number to be converted\\
%    |#2|: number of symbols\\
%    |#3|: symbols provider
%    \begin{macrocode}
\def\AlPh@ProcessWrap#1#2#3{%
  \ifnum#1>#2 %
    \@ReturnAfterElseFi{%
      \expandafter\AlPh@StepWrap\number
        \intcalcInc{\intcalcMod{\intcalcDec{#1}}{#2}}%
      !{#3}%
    }%
  \else
    \@ReturnAfterFi{%
      #3{#1}%
    }%
  \fi
}
%    \end{macrocode}
%    \end{macro}
%    \begin{macro}{\AlPh@StepWrap}
%    |#1|: final number\\
%    |#2|: symbols provider
%    \begin{macrocode}
\def\AlPh@StepWrap#1!#2{%
  #2{#1}%
}
%    \end{macrocode}
%    \end{macro}
%
% \subsubsection{Method `mult'}
%
%    After the number of symbols is exhausted,
%    repetitions of the symbol are used.
%    \begin{gather*}
%      x \Def \text{number to be converted}\\
%      n \Def \text{number of symbols}\\
%      r \Def \text{repetition length}\\
%      s \Def \text{symbol slot}\\
%      r = ((x - 1) \div n) + 1\\
%      s = ((x - 1) \mod n) + 1
%    \end{gather*}
%    \begin{macro}{\AlPh@Method@mult}
%    \noindent
%    |#1|: number of symbols\\
%    |#2|: symbols provider\\
%    |#3|: number to be converted
%    \begin{macrocode}
\def\AlPh@Method@mult#1#2#3{%
  \expandafter\AlPh@CheckPositive
|   \number#3!%
+   \the\numexpr#3!%
    \AlPh@ProcessMult
    {#1}{#2}%
}
%    \end{macrocode}
%    \end{macro}
%
%    \begin{macro}{\AlPh@ProcessMult}
%    |#1|: number to be converted\\
%    |#2|: number of symbols\\
%    |#3|: symbols provider
%    \begin{macrocode}
\def\AlPh@ProcessMult#1#2#3{%
  \ifnum#1>#2 %
    \@ReturnAfterElseFi{%
      \expandafter\AlPh@StepMult\romannumeral
        \intcalcInc{\intcalcDiv{\intcalcDec{#1}}{#2}}%
        000%
      \expandafter!\number
        \intcalcInc{\intcalcMod{\intcalcDec{#1}}{#2}}%
      !{#3}%
    }%
  \else
    \@ReturnAfterFi{%
      #3{#1}%
    }%
  \fi
}
%    \end{macrocode}
%    \end{macro}
%    \begin{macro}{\AlPh@StepMult}
%    |#1#2|: repetitions coded as list of character `m'\\
%    |#3|: symbol slot\\
%    |#4|: symbols provider
%    \begin{macrocode}
\def\AlPh@StepMult#1#2!#3!#4{%
  \ifx\\#2\\%
  \else
    \@ReturnAfterFi{%
      \AlPh@StepMult#2!#3!{#4}%
    }%
  \fi
  #4{#3}%
}
%    \end{macrocode}
%    \end{macro}
%
% \subsection{User interface}
%
%    \begin{macro}{\newalphalph}
%    Macro \cs{newalphalph} had three arguments in versions below 2.0.
%    For the new method argument we use an optional argument an first
%    position.\\
%    |#1|: cmd\\
%    |[#2]|: method name: |alph| (default), |wrap|, |mult|\\ % hash-ok
%    |#3|: symbols provider\\
%    |#4|: number of symbols
%    \begin{macrocode}
\AlPh@IfDefinable\newalphalph{%
  \def\newalphalph#1{%
    \AlPh@IfOptArg{%
      \AlPh@newalphalph{#1}%
    }{%
      \AlPh@newalphalph{#1}[alph]%
    }%
  }%
}
%    \end{macrocode}
%    \end{macro}
%    \begin{macro}{\AlPh@newalphalph}
%    |#1|: cmd
%    |#2|: method name\\
%    |#3|: symbols provider\\
%    |#4|: number of symbols
%    \begin{macrocode}
\def\AlPh@newalphalph#1[#2]#3#4{%
  \begingroup\expandafter\expandafter\expandafter\endgroup
  \expandafter\ifx\csname AlPh@Method@#2\endcsname\relax
    \AlPh@Error{%
      Unknown method %
|     `#2'%
+     `\detokenize{#2}'%
    }%
  \else
    \ifx\\#4\\%
      \AlPh@GetNumberOfSymbols{#3}%
      \ifcase\AlPh@Number
      \else
        \begingroup
          \escapechar=92 % backslash
          \@PackageInfo{alphalph}{%
            Number of symbols for \string#1 is \AlPh@Number
          }%
        \endgroup
        \expandafter\AlPh@NewAlphAlph
        \csname AlPh@Method@#2\expandafter\endcsname
        \AlPh@Number!{#1}{#3}%
      \fi
    \else
      \expandafter\AlPh@NewAlphAlph
      \csname AlPh@Method@#2\expandafter\endcsname
|     \number#4!%
+     \the\numexpr#4!%
      {#1}{#3}%
    \fi
  \fi
}%
%    \end{macrocode}
%    \end{macro}
%    \begin{macro}{\AlPh@NewAlphAlph}
%    |#1|: method macro\\
%    |#2|: number of symbols\\
%    |#3|: cmd\\
%    |#4|: symbols provider
%    \begin{macrocode}
\def\AlPh@NewAlphAlph#1#2!#3#4{%
  \AlPh@IfDefinable#3{%
    \ifnum#2>0 %
      \def#3{#1{#2}{#4}}%
    \else
      \AlPh@Error{%
          Definition of \string#3 failed,\MessageBreak
          because number of symbols (#2) is not positive%
        }%
    \fi
  }%
}
%    \end{macrocode}
%    \end{macro}
%
%    \begin{macro}{\AlphAlph}
%    \begin{macrocode}
\newalphalph\AlphAlph\alphalph@Alph{26}
%    \end{macrocode}
%    \end{macro}
%    \begin{macro}{\alphalph}
%    \begin{macrocode}
\newalphalph\alphalph\alphalph@alph{26}
%    \end{macrocode}
%    \end{macro}
%
%    \begin{macrocode}
\AlPh@AtEnd%
%</package>
%    \end{macrocode}
%
% \section{Test}
%
% \subsection{Catcode checks for loading}
%
%    \begin{macrocode}
%<*test1>
%    \end{macrocode}
%    \begin{macrocode}
\catcode`\{=1 %
\catcode`\}=2 %
\catcode`\#=6 %
\catcode`\@=11 %
\expandafter\ifx\csname count@\endcsname\relax
  \countdef\count@=255 %
\fi
\expandafter\ifx\csname @gobble\endcsname\relax
  \long\def\@gobble#1{}%
\fi
\expandafter\ifx\csname @firstofone\endcsname\relax
  \long\def\@firstofone#1{#1}%
\fi
\expandafter\ifx\csname loop\endcsname\relax
  \expandafter\@firstofone
\else
  \expandafter\@gobble
\fi
{%
  \def\loop#1\repeat{%
    \def\body{#1}%
    \iterate
  }%
  \def\iterate{%
    \body
      \let\next\iterate
    \else
      \let\next\relax
    \fi
    \next
  }%
  \let\repeat=\fi
}%
\def\RestoreCatcodes{}
\count@=0 %
\loop
  \edef\RestoreCatcodes{%
    \RestoreCatcodes
    \catcode\the\count@=\the\catcode\count@\relax
  }%
\ifnum\count@<255 %
  \advance\count@ 1 %
\repeat

\def\RangeCatcodeInvalid#1#2{%
  \count@=#1\relax
  \loop
    \catcode\count@=15 %
  \ifnum\count@<#2\relax
    \advance\count@ 1 %
  \repeat
}
\def\RangeCatcodeCheck#1#2#3{%
  \count@=#1\relax
  \loop
    \ifnum#3=\catcode\count@
    \else
      \errmessage{%
        Character \the\count@\space
        with wrong catcode \the\catcode\count@\space
        instead of \number#3%
      }%
    \fi
  \ifnum\count@<#2\relax
    \advance\count@ 1 %
  \repeat
}
\def\space{ }
\expandafter\ifx\csname LoadCommand\endcsname\relax
  \def\LoadCommand{\input alphalph.sty\relax}%
\fi
\def\Test{%
  \RangeCatcodeInvalid{0}{47}%
  \RangeCatcodeInvalid{58}{64}%
  \RangeCatcodeInvalid{91}{96}%
  \RangeCatcodeInvalid{123}{255}%
  \catcode`\@=12 %
  \catcode`\\=0 %
  \catcode`\%=14 %
  \LoadCommand
  \RangeCatcodeCheck{0}{36}{15}%
  \RangeCatcodeCheck{37}{37}{14}%
  \RangeCatcodeCheck{38}{47}{15}%
  \RangeCatcodeCheck{48}{57}{12}%
  \RangeCatcodeCheck{58}{63}{15}%
  \RangeCatcodeCheck{64}{64}{12}%
  \RangeCatcodeCheck{65}{90}{11}%
  \RangeCatcodeCheck{91}{91}{15}%
  \RangeCatcodeCheck{92}{92}{0}%
  \RangeCatcodeCheck{93}{96}{15}%
  \RangeCatcodeCheck{97}{122}{11}%
  \RangeCatcodeCheck{123}{255}{15}%
  \RestoreCatcodes
}
\Test
\csname @@end\endcsname
\end
%    \end{macrocode}
%    \begin{macrocode}
%</test1>
%    \end{macrocode}
%
% \section{Macro tests}
%
%    \begin{macrocode}
%<*test2>
\NeedsTeXFormat{LaTeX2e}
\nofiles
\documentclass{article}
%<*noetex>
\makeatletter
\let\saved@numexpr\numexpr
\newcommand*{\DisableNumexpr}{%
  \let\numexpr\@undefined
}
\newcommand*{\RestoreNumexpr}{%
  \let\numexpr\saved@numexpr
}
\DisableNumexpr
%</noetex>
\usepackage{alphalph}[2016/05/16]
%<noetex>\RestoreNumexpr
\usepackage{qstest}
\IncludeTests{*}
\LogTests{log}{*}{*}

\newcommand*{\TestCmd}[3]{%
  \setbox0=\hbox{%
%<noetex>    \DisableNumexpr
    \edef\TestString{#1{#2}}%
    \expandafter\Expect\expandafter{\TestString}{#3}%
    \edef\TestString{#1{#2} }%
    \expandafter\Expect\expandafter{\TestString}{#3 }%
  }%
  \Expect*{\the\wd0}{0.0pt}%
}

\makeatletter
\newalphalph\LaTeXAlphAlph\@Alph{26}
\newalphalph\LaTeXalphalph\@alph{26}
\newalphalph\AlphWrap[wrap]\alphalph@Alph{26}
\newalphalph\alphwrap[wrap]\alphalph@alph{26}
\newalphalph\LaTeXAlphWrap[wrap]\@Alph{26}
\newalphalph\LaTeXalphwrap[wrap]\@alph{26}
\def\LastSymbol#1{%
  \ifx\\#1\\%
  \else
    \@LastSymbol#1\@nil
  \fi
}
\def\@LastSymbol#1#2\@nil{%
  \ifx\\#2\\%
    #1%
  \else
    \@LastSymbol#2\@nil
  \fi
}
\makeatother
\newcommand*{\TestAlph}[2]{%
  \uppercase{\TestCallCmd\AlphAlph{#2}}{#1}%
  \lowercase{\TestCallCmd\alphalph{#2}}{#1}%
  \uppercase{\TestCallCmd\LaTeXAlphAlph{#2}}{#1}%
  \lowercase{\TestCallCmd\LaTeXalphalph{#2}}{#1}%
  \edef\WrapString{\LastSymbol{#2}}%
  \expandafter\TestAlphWrap\expandafter{\WrapString}{#1}%
}
\newcommand*{\TestAlphWrap}[2]{%
  \uppercase{\TestCallCmd\AlphWrap{#1}}{#2}%
  \lowercase{\TestCallCmd\alphwrap{#1}}{#2}%
  \uppercase{\TestCallCmd\LaTeXAlphWrap{#1}}{#2}%
  \lowercase{\TestCallCmd\LaTeXalphwrap{#1}}{#2}%
}
\newcommand*{\TestCallCmd}[3]{%
  \TestCmd#1{#3}{#2}%
}
\begin{qstest}{AlphSymbols}{alphalph, AlphAlph, symbols}
  \TestAlph{1}{a}%
  \TestAlph{2}{b}%
  \TestAlph{3}{c}%
  \TestAlph{4}{d}%
  \TestAlph{5}{e}%
  \TestAlph{6}{f}%
  \TestAlph{7}{g}%
  \TestAlph{8}{h}%
  \TestAlph{9}{i}%
  \TestAlph{10}{j}%
  \TestAlph{11}{k}%
  \TestAlph{12}{l}%
  \TestAlph{13}{m}%
  \TestAlph{14}{n}%
  \TestAlph{15}{o}%
  \TestAlph{16}{p}%
  \TestAlph{17}{q}%
  \TestAlph{18}{r}%
  \TestAlph{19}{s}%
  \TestAlph{20}{t}%
  \TestAlph{21}{u}%
  \TestAlph{22}{v}%
  \TestAlph{23}{w}%
  \TestAlph{24}{x}%
  \TestAlph{25}{y}%
  \TestAlph{26}{z}%
\end{qstest}
\begin{qstest}{AlphRange}{alphalph, range}
  \TestAlph{0}{}%
  \TestAlph{-1}{}%
  \TestAlph{-2147483647}{}%
  \TestAlph{27}{aa}%
  \TestAlph{28}{ab}%
  \TestAlph{52}{az}%
  \TestAlph{53}{ba}%
  \TestAlph{78}{bz}%
  \TestAlph{79}{ca}%
  \TestAlph{702}{zz}%
  \TestAlph{703}{aaa}%
  \TestAlph{2147483647}{fxshrxw}%
\end{qstest}

\makeatletter
\newcommand*{\myvocals}[1]{%
  \ifcase#1X\or A\or E\or I\or O\or U\else Y\fi
}
\makeatother
\newalphalph\vocalsvocals\myvocals{5}
\newcommand*{\TestVocals}{%
  \TestCmd\vocalsvocals
}
\begin{qstest}{vocals}{vocals}
  \TestVocals{0}{}%
  \TestVocals{1}{A}%
  \TestVocals{2}{E}%
  \TestVocals{3}{I}%
  \TestVocals{4}{O}%
  \TestVocals{5}{U}%
  \TestVocals{6}{AA}%
  \TestVocals{7}{AE}%
  \TestVocals{8}{AI}%
  \TestVocals{9}{AO}%
  \TestVocals{10}{AU}%
  \TestVocals{11}{EA}%
  \TestVocals{24}{OO}%
  \TestVocals{25}{OU}%
  \TestVocals{26}{UA}%
  \TestVocals{29}{UO}%
  \TestVocals{30}{UU}%
  \TestVocals{31}{AAA}%
  \TestVocals{155}{UUU}%
  \TestVocals{156}{AAAA}%
  \TestVocals{2147483647}{AIIOOEEIOIIUOE}%
\end{qstest}

\makeatletter
\newalphalph\AlphMult[mult]{\alphalph@Alph}{26}
\newalphalph\alphmult[mult]{\alphalph@alph}{26}
\newalphalph\LaTeXAlphMult[mult]{\@Alph}{26}
\newalphalph\LaTeXalphmult[mult]{\@alph}{26}
\makeatother
\newcommand*{\TestMult}[2]{%
  \uppercase{\TestCallCmd\AlphMult{#2}}{#1}%
  \lowercase{\TestCallCmd\alphmult{#2}}{#1}%
  \uppercase{\TestCallCmd\LaTeXAlphMult{#2}}{#1}%
  \lowercase{\TestCallCmd\LaTeXalphmult{#2}}{#1}%
}
\begin{qstest}{mult}{mult}
  \TestMult{0}{}%
  \TestMult{-1}{}%
  \TestMult{-2147483647}{}%
  \TestMult{1}{a}%
  \TestMult{2}{b}%
  \TestMult{26}{z}%
  \TestMult{27}{aa}%
  \TestMult{28}{bb}%
  \TestMult{52}{zz}%
  \TestMult{53}{aaa}%
  \TestMult{54}{bbb}%
  \TestMult{259}{yyyyyyyyyy}%
  \TestMult{260}{zzzzzzzzzz}%
  \TestMult{261}{aaaaaaaaaaa}%
  \TestMult{262}{bbbbbbbbbbb}%
\end{qstest}

\def\myvocalsB#1{%
  \ifcase#1\or A\or E\or I\or O\or U\fi
}
\begin{qstest}{symbolnum}{symbolnum}
  \makeatletter
  \def\Test#1#2{%
    \let\TestCmd\relax
    \newalphalph\TestCmd{#1}{}%
    \Expect*{\AlPh@Number}{#2}%
  }%
  \Test\@alph{26}%
  \Test\@Alph{26}%
  \Test\@fnsymbol{9}%
  \Test\myvocalsB{5}%
  \Test\alphalph@alph{26}%
  \Test\alphalph@Alph{26}%
\end{qstest}

\begin{qstest}{list}{list}
  \makeatletter
  \def\catch#1\relax{%
    \def\FoundList{\catch#1}%
  }%
  \def\Test[#1]#2#3#4{%
    \let\testcmd\relax
    \newalphalph\testcmd[{#1}]{\catch}{#2}%
    \testcmd{#3}|\relax
    \expandafter\Expect\expandafter{\FoundList}{#4|}%
    %
    \let\SavedCatch\catch
    \def\catch{\noexpand\catch\noexpand\foo}%
    \edef\Result{#4|}%
    \@onelevel@sanitize\Result
    \let\catch\SavedCatch
    \let\testcmd\relax
    \newalphalph\testcmd[{#1}]{\catch\foo}{#2}%
    \testcmd{#3}|\relax
    \@onelevel@sanitize\FoundList
    \Expect*{\FoundList}*{\Result}%
  }%
  \Test[alph]{26}{3}{\catch{3}}%
  \Test[alph]{26}{12}{\catch{12}}%
  \Test[alph]{26}{27}{\catch{1}\catch{1}}%
  \Test[alph]{26}{78}{\catch{2}\catch{26}}%
  \Test[wrap]{26}{7}{\catch{7}}%
  \Test[wrap]{26}{14}{\catch{14}}%
  \Test[wrap]{26}{80}{\catch{2}}%
  \Test[wrap]{26}{700}{\catch{24}}%
  \Test[mult]{26}{4}{\catch{4}}%
  \Test[mult]{26}{17}{\catch{17}}%
  \Test[mult]{26}{54}{\catch{2}\catch{2}\catch{2}}%
\end{qstest}

\begin{document}
\end{document}
%</test2>
%    \end{macrocode}
%
% \section{Installation}
%
% \subsection{Download}
%
% \paragraph{Package.} This package is available on
% CTAN\footnote{\CTANpkg{alphalph}}:
% \begin{description}
% \item[\CTAN{macros/latex/contrib/oberdiek/alphalph.dtx}] The source file.
% \item[\CTAN{macros/latex/contrib/oberdiek/alphalph.pdf}] Documentation.
% \end{description}
%
%
% \paragraph{Bundle.} All the packages of the bundle `oberdiek'
% are also available in a TDS compliant ZIP archive. There
% the packages are already unpacked and the documentation files
% are generated. The files and directories obey the TDS standard.
% \begin{description}
% \item[\CTANinstall{install/macros/latex/contrib/oberdiek.tds.zip}]
% \end{description}
% \emph{TDS} refers to the standard ``A Directory Structure
% for \TeX\ Files'' (\CTAN{tds/tds.pdf}). Directories
% with \xfile{texmf} in their name are usually organized this way.
%
% \subsection{Bundle installation}
%
% \paragraph{Unpacking.} Unpack the \xfile{oberdiek.tds.zip} in the
% TDS tree (also known as \xfile{texmf} tree) of your choice.
% Example (linux):
% \begin{quote}
%   |unzip oberdiek.tds.zip -d ~/texmf|
% \end{quote}
%
% \paragraph{Script installation.}
% Check the directory \xfile{TDS:scripts/oberdiek/} for
% scripts that need further installation steps.
% Package \xpackage{attachfile2} comes with the Perl script
% \xfile{pdfatfi.pl} that should be installed in such a way
% that it can be called as \texttt{pdfatfi}.
% Example (linux):
% \begin{quote}
%   |chmod +x scripts/oberdiek/pdfatfi.pl|\\
%   |cp scripts/oberdiek/pdfatfi.pl /usr/local/bin/|
% \end{quote}
%
% \subsection{Package installation}
%
% \paragraph{Unpacking.} The \xfile{.dtx} file is a self-extracting
% \docstrip\ archive. The files are extracted by running the
% \xfile{.dtx} through \plainTeX:
% \begin{quote}
%   \verb|tex alphalph.dtx|
% \end{quote}
%
% \paragraph{TDS.} Now the different files must be moved into
% the different directories in your installation TDS tree
% (also known as \xfile{texmf} tree):
% \begin{quote}
% \def\t{^^A
% \begin{tabular}{@{}>{\ttfamily}l@{ $\rightarrow$ }>{\ttfamily}l@{}}
%   alphalph.sty & tex/generic/oberdiek/alphalph.sty\\
%   alphalph.pdf & doc/latex/oberdiek/alphalph.pdf\\
%   test/alphalph-test1.tex & doc/latex/oberdiek/test/alphalph-test1.tex\\
%   test/alphalph-test2.tex & doc/latex/oberdiek/test/alphalph-test2.tex\\
%   test/alphalph-test3.tex & doc/latex/oberdiek/test/alphalph-test3.tex\\
%   alphalph.dtx & source/latex/oberdiek/alphalph.dtx\\
% \end{tabular}^^A
% }^^A
% \sbox0{\t}^^A
% \ifdim\wd0>\linewidth
%   \begingroup
%     \advance\linewidth by\leftmargin
%     \advance\linewidth by\rightmargin
%   \edef\x{\endgroup
%     \def\noexpand\lw{\the\linewidth}^^A
%   }\x
%   \def\lwbox{^^A
%     \leavevmode
%     \hbox to \linewidth{^^A
%       \kern-\leftmargin\relax
%       \hss
%       \usebox0
%       \hss
%       \kern-\rightmargin\relax
%     }^^A
%   }^^A
%   \ifdim\wd0>\lw
%     \sbox0{\small\t}^^A
%     \ifdim\wd0>\linewidth
%       \ifdim\wd0>\lw
%         \sbox0{\footnotesize\t}^^A
%         \ifdim\wd0>\linewidth
%           \ifdim\wd0>\lw
%             \sbox0{\scriptsize\t}^^A
%             \ifdim\wd0>\linewidth
%               \ifdim\wd0>\lw
%                 \sbox0{\tiny\t}^^A
%                 \ifdim\wd0>\linewidth
%                   \lwbox
%                 \else
%                   \usebox0
%                 \fi
%               \else
%                 \lwbox
%               \fi
%             \else
%               \usebox0
%             \fi
%           \else
%             \lwbox
%           \fi
%         \else
%           \usebox0
%         \fi
%       \else
%         \lwbox
%       \fi
%     \else
%       \usebox0
%     \fi
%   \else
%     \lwbox
%   \fi
% \else
%   \usebox0
% \fi
% \end{quote}
% If you have a \xfile{docstrip.cfg} that configures and enables \docstrip's
% TDS installing feature, then some files can already be in the right
% place, see the documentation of \docstrip.
%
% \subsection{Refresh file name databases}
%
% If your \TeX~distribution
% (\teTeX, \mikTeX, \dots) relies on file name databases, you must refresh
% these. For example, \teTeX\ users run \verb|texhash| or
% \verb|mktexlsr|.
%
% \subsection{Some details for the interested}
%
% \paragraph{Attached source.}
%
% The PDF documentation on CTAN also includes the
% \xfile{.dtx} source file. It can be extracted by
% AcrobatReader 6 or higher. Another option is \textsf{pdftk},
% e.g. unpack the file into the current directory:
% \begin{quote}
%   \verb|pdftk alphalph.pdf unpack_files output .|
% \end{quote}
%
% \paragraph{Unpacking with \LaTeX.}
% The \xfile{.dtx} chooses its action depending on the format:
% \begin{description}
% \item[\plainTeX:] Run \docstrip\ and extract the files.
% \item[\LaTeX:] Generate the documentation.
% \end{description}
% If you insist on using \LaTeX\ for \docstrip\ (really,
% \docstrip\ does not need \LaTeX), then inform the autodetect routine
% about your intention:
% \begin{quote}
%   \verb|latex \let\install=y\input{alphalph.dtx}|
% \end{quote}
% Do not forget to quote the argument according to the demands
% of your shell.
%
% \paragraph{Generating the documentation.}
% You can use both the \xfile{.dtx} or the \xfile{.drv} to generate
% the documentation. The process can be configured by the
% configuration file \xfile{ltxdoc.cfg}. For instance, put this
% line into this file, if you want to have A4 as paper format:
% \begin{quote}
%   \verb|\PassOptionsToClass{a4paper}{article}|
% \end{quote}
% An example follows how to generate the
% documentation with pdf\LaTeX:
% \begin{quote}
%\begin{verbatim}
%pdflatex alphalph.dtx
%makeindex -s gind.ist alphalph.idx
%pdflatex alphalph.dtx
%makeindex -s gind.ist alphalph.idx
%pdflatex alphalph.dtx
%\end{verbatim}
% \end{quote}
%
% \begin{History}
%   \begin{Version}{1999/03/19 v0.1}
%   \item
%     The first version was built as a response to a
%     \URL{\link{question}}^^A
%     {https://groups.google.com/group/comp.text.tex/msg/17a74cd721641038}^^A
%     of \NameEmail{Will Douglas}{william.douglas@wolfson.ox.ac.uk}
%     and the
%     \URL{\link{request}}^^A
%     {https://groups.google.com/group/comp.text.tex/msg/8f9768825640315f}^^A
%     of \NameEmail{Donald Arsenau}{asnd@reg.triumf.ca},
%     published in the newsgroup
%     \href{news:comp.text.tex}{comp.text.tex}:
%     \URL{``\link{Re: alph counters \textgreater\ 26}''}^^A
%     {https://groups.google.com/group/comp.text.tex/msg/cec563eef8bf65d0}
%   \item
%     Copyright: LPPL (\CTAN{macros/latex/base/lppl.txt})
%   \end{Version}
%   \begin{Version}{1999/04/12 v1.0}
%   \item
%     Documentation added in dtx format.
%   \item
%     \hologo{eTeX} support added.
%   \end{Version}
%   \begin{Version}{1999/04/13 v1.1}
%   \item
%     Minor documentation change.
%   \item
%     First CTAN release.
%   \end{Version}
%   \begin{Version}{1999/06/26 v1.2}
%   \item
%     First generic code about \cmd{\ProvidesPackage} improved.
%   \item
%     Documentation: Installation part revised.
%   \end{Version}
%   \begin{Version}{2006/02/20 v1.3}
%   \item
%     Reload check (for \hologo{plainTeX})
%   \item
%     New DTX framework.
%   \item
%     LPPL 1.3
%   \end{Version}
%   \begin{Version}{2006/05/30 v1.4}
%   \item
%     \cs{newalphalph} added.
%   \end{Version}
%   \begin{Version}{2007/04/11 v1.5}
%   \item
%     Line ends sanitized.
%   \end{Version}
%   \begin{Version}{2007/09/09 v2.0}
%   \item
%     New implementation that uses package \cs{intcalc}.
%     This removes the dependency on \hologo{eTeX}.
%   \item
%     \cs{newalphalph} is extended to support new methods `wrap' and 'multi'.
%   \item
%     Documentation rewritten.
%   \end{Version}
%   \begin{Version}{2008/08/11 v2.1}
%   \item
%     Code is not changed.
%   \item
%     URLs updated from \texttt{www.dejanews.com}
%     to \texttt{groups.google.com}.
%   \end{Version}
%   \begin{Version}{2010/03/01 v2.2}
%   \item
%     Compatibility with \hologo{iniTeX}.
%   \end{Version}
%   \begin{Version}{2010/04/18 v2.3}
%   \item
%     Documentation fixes (Martin M\"unch).
%   \end{Version}
%   \begin{Version}{2011/05/13 v2.4}
%   \item
%     Documentation fixes (Jim Diamond) and using package
%     \xpackage{hologo} for the documentation.
%   \item
%     Catalogue file added.
%   \end{Version}
%   \begin{Version}{2016/05/16 v2.5}
%   \item
%     Documentation updates.
%   \end{Version}
% \end{History}
%
% \PrintIndex
%
% \Finale
\endinput

%        (quote the arguments according to the demands of your shell)
%
% Documentation:
%    (a) If alphalph.drv is present:
%           latex alphalph.drv
%    (b) Without alphalph.drv:
%           latex alphalph.dtx; ...
%    The class ltxdoc loads the configuration file ltxdoc.cfg
%    if available. Here you can specify further options, e.g.
%    use A4 as paper format:
%       \PassOptionsToClass{a4paper}{article}
%
%    Programm calls to get the documentation (example):
%       pdflatex alphalph.dtx
%       makeindex -s gind.ist alphalph.idx
%       pdflatex alphalph.dtx
%       makeindex -s gind.ist alphalph.idx
%       pdflatex alphalph.dtx
%
% Installation:
%    TDS:tex/generic/oberdiek/alphalph.sty
%    TDS:doc/latex/oberdiek/alphalph.pdf
%    TDS:doc/latex/oberdiek/test/alphalph-test1.tex
%    TDS:doc/latex/oberdiek/test/alphalph-test2.tex
%    TDS:doc/latex/oberdiek/test/alphalph-test3.tex
%    TDS:source/latex/oberdiek/alphalph.dtx
%
%<*ignore>
\begingroup
  \catcode123=1 %
  \catcode125=2 %
  \def\x{LaTeX2e}%
\expandafter\endgroup
\ifcase 0\ifx\install y1\fi\expandafter
         \ifx\csname processbatchFile\endcsname\relax\else1\fi
         \ifx\fmtname\x\else 1\fi\relax
\else\csname fi\endcsname
%</ignore>
%<*install>
\input docstrip.tex
\Msg{************************************************************************}
\Msg{* Installation}
\Msg{* Package: alphalph 2016/05/16 v2.5 Convert numbers to letters (HO)}
\Msg{************************************************************************}

\keepsilent
\askforoverwritefalse

\let\MetaPrefix\relax
\preamble

This is a generated file.

Project: alphalph
Version: 2016/05/16 v2.5

Copyright (C) 1999, 2006-2008, 2010, 2011 by
   Heiko Oberdiek <heiko.oberdiek at googlemail.com>

This work may be distributed and/or modified under the
conditions of the LaTeX Project Public License, either
version 1.3c of this license or (at your option) any later
version. This version of this license is in
   http://www.latex-project.org/lppl/lppl-1-3c.txt
and the latest version of this license is in
   http://www.latex-project.org/lppl.txt
and version 1.3 or later is part of all distributions of
LaTeX version 2005/12/01 or later.

This work has the LPPL maintenance status "maintained".

This Current Maintainer of this work is Heiko Oberdiek.

The Base Interpreter refers to any `TeX-Format',
because some files are installed in TDS:tex/generic//.

This work consists of the main source file alphalph.dtx
and the derived files
   alphalph.sty, alphalph.pdf, alphalph.ins, alphalph.drv,
   alphalph-test1.tex, alphalph-test2.tex, alphalph-test3.tex.

\endpreamble
\let\MetaPrefix\DoubleperCent

\generate{%
  \file{alphalph.ins}{\from{alphalph.dtx}{install}}%
  \file{alphalph.drv}{\from{alphalph.dtx}{driver}}%
  \usedir{tex/generic/oberdiek}%
  \file{alphalph.sty}{\from{alphalph.dtx}{package}}%
%  \usedir{doc/latex/oberdiek/test}%
%  \file{alphalph-test1.tex}{\from{alphalph.dtx}{test1}}%
%  \file{alphalph-test2.tex}{\from{alphalph.dtx}{test2,etex}}%
%  \file{alphalph-test3.tex}{\from{alphalph.dtx}{test2,noetex}}%
  \nopreamble
  \nopostamble
%  \usedir{source/latex/oberdiek/catalogue}%
%  \file{alphalph.xml}{\from{alphalph.dtx}{catalogue}}%
}

\catcode32=13\relax% active space
\let =\space%
\Msg{************************************************************************}
\Msg{*}
\Msg{* To finish the installation you have to move the following}
\Msg{* file into a directory searched by TeX:}
\Msg{*}
\Msg{*     alphalph.sty}
\Msg{*}
\Msg{* To produce the documentation run the file `alphalph.drv'}
\Msg{* through LaTeX.}
\Msg{*}
\Msg{* Happy TeXing!}
\Msg{*}
\Msg{************************************************************************}

\endbatchfile
%</install>
%<*ignore>
\fi
%</ignore>
%<*driver>
\NeedsTeXFormat{LaTeX2e}
\ProvidesFile{alphalph.drv}%
  [2016/05/16 v2.5 Convert numbers to letters (HO)]%
\documentclass{ltxdoc}
\usepackage{alphalph}[2016/05/16]
\makeatletter
\newalphalph{\fnsymbolwrap}[wrap]{\@fnsymbol}{}
\newcommand*{\fnsymbolsingle}[1]{%
  \ensuremath{%
    \ifcase#1%
    \or *%
    \or \dagger
    \or \ddagger
    \or \mathsection
    \or \mathparagraph
    \else
      \@ctrerr
    \fi
  }%
}
\makeatother
\newalphalph{\fnsymbolmult}[mult]{\fnsymbolsingle}{}
\usepackage{amsmath}
\DeclareMathOperator{\opDiv}{div}
\DeclareMathOperator{\opMod}{mod}
\newcommand*{\Def}{%
  \ensuremath{%
    \mathrel{\mathop{:}}=%
  }%
}
\usepackage{holtxdoc}[2011/11/22]
\begin{document}
  \DocInput{alphalph.dtx}%
\end{document}
%</driver>
% \fi
%
%
% \CharacterTable
%  {Upper-case    \A\B\C\D\E\F\G\H\I\J\K\L\M\N\O\P\Q\R\S\T\U\V\W\X\Y\Z
%   Lower-case    \a\b\c\d\e\f\g\h\i\j\k\l\m\n\o\p\q\r\s\t\u\v\w\x\y\z
%   Digits        \0\1\2\3\4\5\6\7\8\9
%   Exclamation   \!     Double quote  \"     Hash (number) \#
%   Dollar        \$     Percent       \%     Ampersand     \&
%   Acute accent  \'     Left paren    \(     Right paren   \)
%   Asterisk      \*     Plus          \+     Comma         \,
%   Minus         \-     Point         \.     Solidus       \/
%   Colon         \:     Semicolon     \;     Less than     \<
%   Equals        \=     Greater than  \>     Question mark \?
%   Commercial at \@     Left bracket  \[     Backslash     \\
%   Right bracket \]     Circumflex    \^     Underscore    \_
%   Grave accent  \`     Left brace    \{     Vertical bar  \|
%   Right brace   \}     Tilde         \~}
%
% \GetFileInfo{alphalph.drv}
%
% \title{The \xpackage{alphalph} package}
% \date{2016/05/16 v2.5}
% \author{Heiko Oberdiek\thanks
% {Please report any issues at \url{https://github.com/ho-tex/oberdiek/issues}}\\
% \xemail{heiko.oberdiek at googlemail.com}}
%
% \maketitle
%
% \begin{abstract}
% The package provides methods to represent numbers with a limited
% set of symbols. Both \hologo{LaTeX} and \hologo{plainTeX} are supported.
% \end{abstract}
%
% \tableofcontents
%
% \section{Documentation}
%
% \subsection{Introduction}
% \hologo{LaTeX} counters can be represented in different ways
% by using presentation commands:
% \begin{quote}
%   \cs{arabic}, \cs{roman}, \cs{Roman},\\
%   \cs{alph}, \cs{Alph}, \cs{fnsymbol}
% \end{quote}
% The ranges of supported counter values are more or less
% restricted. Only \cs{arabic} can be used with any counter
% value \hologo{TeX} supports.
% \begin{quote}
% \catcode`\|=12 %
% \begin{tabular}{@{}l|l|l|l@{}}
% Presentation & Supported & Ignored & Error message\\
% command      & domain    & values  & ``Counter too large''\\
% \hline
% \cs{arabic}
%   & \ttfamily -MAX..MAX & &\\
% \cs{roman}, \cs{Roman}
%   & \ttfamily 1..MAX & \ttfamily -MAX..0 &\\
% \cs{alph}, \cs{Alph}
%   & \ttfamily 1..26 & 0 & \ttfamily -MAX..-1, 27..MAX\\
% \cs{fnsymbol}
%   & \ttfamily 1..9 & 0 & \ttfamily -MAX..-1, 10..MAX\\
% \hline
% \end{tabular}\\[1ex]
% \texttt{MAX} = \texttt{2147483647}
% \end{quote}
% Ordinal numbers are often used in documents: numbering
% of chapters, sections, figures, footnotes and so on.
% The layouter chooses \cs{Alph} for chapter numbers
% and \cs{fnsymbol} for footnotes. But what can be done
% if there are more than 26 chapters or more than 10 footnotes?
% This package \xpackage{alphalph} allows to define new
% presentation commands. They rely on a existing command
% and define presentations for values greater the limits.
% Three different methods are provided by the package.
% In the following use cases they are presentated.
%
% \subsection{Use cases}
%
% \subsubsection{Number system based on symbols}
%
% Asume you are writing a book and your lecturer demands
% that chapter numbers must be letters. But you have already
% 30 chapters and you have only 26 letters?
%
% In the decimal system the situation would be clear.
% If you run out of digits, you are using more digits
% to represent a number. This method can be also be used
% for letters. After chapter 26 with |Z| we us |AA|, |AB|, |AC|, and |AD|
% for the remaining chapters.
%
% Happily this package already defines this presentation command:
%\begin{quote}
%\begin{verbatim}
%\usepackage{alphalph}
%\renewcommand*{\thechapter}{%
%  \AlphAlph{\value{chapter}}%
%}
%\end{verbatim}
%\end{quote}
% \cs{AlphAlph} generates:
%   \AlphAlph{1}, \AlphAlph{2}, \AlphAlph{3}, \dots,
%   \AlphAlph{26}, \AlphAlph{27}, \AlphAlph{28}, \dots
%
% The other presentation command is \cs{alphalph} for lowercase letters.
%
% \subsubsection{Wrap symbols around}
%
% Nine footnote symbols are quite a few.
% Too soon the symbols are consumed
% and \hologo{LaTeX} complains with the error ``Counter too large''.
% However, it could be acceptable to start again with the symbols
% from the beginning, especially if there are less than nine symbols
% on a page.
% This could be achieved by a counter reset.
% But finding the right place can be difficult or needs manual
% actions. Also a
% unique counter value can be desirable (e.g. for generating
% unique anchor/link names).
% Package \xpackage{alphalph} allows you to define
% a macro that implements a ``wrap around'', but letting
% the value of the counter untouched:
%\begin{quote}
%\begin{verbatim}
%\usepackage{alphalph}
%\makeatletter
%\newalphalph{\fnsymbolwrap}[wrap]{\@fnsymbol}{}
%\makeatother
%\renewcommand*{\thefootnote}{%
%  \fnsymbolwrap{\value{footnote}}%
%}
%\end{verbatim}
%\end{quote}
% \cs{fnsymbolwrap} generates:
% \fnsymbolwrap{1}~(1), \fnsymbolwrap{2}~(2), \fnsymbolwrap{3}~(3), \dots,
% \fnsymbolwrap{9}~(9), \fnsymbolwrap{10}~(10), \fnsymbolwrap{11}~11, \dots
%
% \subsubsection{Multiple symbols}
%
% \Hologo{LaTeX}'s standard set of footnote symbols contains doubled
% symbols at the higher positions. Could this principle
% be generalized?
% Yes, but first we need a clean footnote symbol list without
% doubled entries, example:
%\begin{quote}
%\begin{verbatim}
%\usepackage{alphalph}
%\makeatletter
%\newcommand*{\fnsymbolsingle}[1]{%
%  \ensuremath{%
%    \ifcase#1%
%    \or *%
%    \or \dagger
%    \or \ddagger
%    \or \mathsection
%    \or \mathparagraph
%    \else
%      \@ctrerr
%    \fi
%  }%
%}
%\makeatother
%\newalphalph{\fnsymbolmult}[mult]{\fnsymbolsingle}{}
%\renewcommand*{\thefootnote}{%
%  \fnsymbolmult{\value{footnote}}%
%}
%\end{verbatim}
%\end{quote}
% The own definition of \cs{fnsymbolsingle} has the
% advantage that this list can easily modified.
% Otherwise you can use \cs{@fnsymbol} directly,
% because it uses the same first five symbols.
%\begin{quote}
%\begin{verbatim}
%\usepackage{alphalph}
%\makeatletter
%\newalphalph{\fnsymbolmult}[mult]{\@fnsymbol}{5}
%\makeatother
%\renewcommand*{\thefootnote}{%
%  \fnsymbolmult{\value{footnote}}%
%}
%\end{verbatim}
%\end{quote}
% \cs{fnsymbolmult} generates:
% \fnsymbolmult{1}~(1), \fnsymbolmult{2}~(2), \fnsymbolmult{3}~(3),
% \fnsymbolmult{4}~(4), \fnsymbolmult{5}~(5), \fnsymbolmult{6}~(6),
% \dots, \fnsymbolmult{16}~16, \fnsymbolmult{17}~17, \dots
%
% \bigskip
% The same method can also be used for the chapter problem
% in the first discussed use case:
%\begin{quote}
%\begin{verbatim}
%\usepackage{alphalph}
%\makeatletter
%\newalphalph{\AlphMult}[mult]{\@Alph}{26}
%\makeatother
%\renewcommand*{\chapter}{%
%  \AlphMult{\value{chapter}}%
%}
%\end{verbatim}
%\end{quote}
% \cs{AlphMult} then generates AA, BB, CC, and DD for chapters 27--30.
%
% \subsection{Glossary}
%
% \begin{description}
% \item[Counter presentation command] is a macro that
%   expects a \hologo{LaTeX} counter name as argument.
%   Numbers cannot be used. Examples:
%   \cs{arabic}, \cs{alph}, \cs{fnsymbol}.
% \item[Number presentation command] is a macro that
%   expects a number as argument. A number is anything
%   that \hologo{TeX} accepts as number including \cs{value}.
%   Examples: \cs{alphalph}, \cs{AlphAlph}, \cs{alphalph@alph}
%
%   However, \cs{alph} or \cs{fnsymbol} are not number
%   presentation commands because they expect a counter
%   name as argument. Happily \hologo{LaTeX} counter presentation
%   commands internally uses number presentation commands
%   with the same name, but prefixed by `@'. Thus
%   \cs{@alph}, \cs{@fnsymbol} are number presentation commands.
% \item[Symbols provider] is a command that can be used
%   to get a list of symbols. For example, \cs{@Alph}
%   provides the 26 uppercase letters from `A' to `Z'.
%   Basically a symbol provider is a number presentation
%   command, usually with a limited range.
% \item[Number of symbols] is the number of the
%   last symbol slot of a symbol provider. Thus \cs{@Alph}
%   generates 26 symbols, \cs{@fnsymbol} provides 9 symbols.
% \end{description}
%
% \subsection{Package usage}
%
% The package \Package{alphalph} can be used with both \hologo{plainTeX}
% and \hologo{LaTeX}:
% \begin{description}
% \item[\hologo{plainTeX}:] |\input alphalph.sty|
% \item[\hologo{LaTeX2e}:]   |\usepackage{alphalph}|\\
%      There aren't any options.
% \end{description}
%
% \subsection{User commands}
%
% \begin{declcs}{AlphAlph} \M{number}\\
%   \cs{alphalph} \M{number}
% \end{declcs}
% Both macros are number presentation commands that
% expects a number as argument. \hologo{LaTeX} counters
% are used with \cs{value}.
%
% The macros represents a number by letters.
% First single letters |A..Z| are used, then
% two letters |AA..ZZ|, three letters |AAA...ZZZ|, \dots
% follow.
%
% Macro \cs{AlphAlph} uses uppercase letters,
% \cs{alphalph} generates the lowercase variant.
% \begin{quote}
%   \catcode`\|=12
%   \def\entry#1{
%     \texttt{#1}
%     & \texttt{\AlphAlph{#1}}
%     & \texttt{\alphalph{#1}}
%   }
%   \begin{tabular}{@{}l|l|l@{}}
%     \meta{number} & \verb|\AlphAlph{|\meta{number}\verb|}|
%                   & \verb|\alphalph{|\meta{number}\verb|}|\\
%     \hline
%     \entry{1}\\
%     \entry{2}\\
%     \entry{26}\\
%     \entry{27}\\
%     \entry{30}\\
%     \entry{2000}\\
%     \entry{3752127}\\
%     \entry{10786572}\\
%     \entry{2147483647}\\
%   \end{tabular}
% \end{quote}
%
% \begin{declcs}{newalphalph}
%   \M{cmd}
%   |[|\meta{method}|]|
%   \M{symbols provider}
%   \M{number of symbols}
% \end{declcs}
% Macro \cs{newalphalph} defines \meta{cmd} as new
% number presentation command. Like \cs{newcommand} an error
% is thrown, if macro \meta{cmd} already exists.
%
% The \meta{method} is one of |alph|, |wrap|, or |mult|.
% The default is |alph|.
%
% As symbol provider a number presentation command
% can be used, e.g. \cs{@fnsymbol}, \cs{@Alph}, or
% \cs{alphalph@alph}.
%
% The last argument is the number of symbols.
% If the argument is empty, then \cs{newalphalph} tries
% to find this number itself. \hologo{LaTeX}'s number presentation
% commands throw an error message, if the number is too large.
% This error message is put in a macro \cs{@ctrerr}.
% Thus \cs{newalphalph} calls the symbol provider and tests
% a number by typesetting it in a temporary box. The error
% macro \cs{@ctrerr} is catched, it proofs that the number
% is not supported. Also if the width of the result is zero
% the number is considered as unavailable.
%
% The empty argument is useful for potentially variable lists.
% However if the end cannot be detected, then the number of
% symbols must be given. This is also a lot faster. Therefore don't
% let the argument empty without reason.
%
% \subsection{Programmer commands}
%
% \begin{declcs}{alphalph@Alph} \M{number}\\
%   \cs{alphalph@alph} \M{number}
% \end{declcs}
% They are basically the same as \cs{@Alph} and \cs{@alph}.
% Some languages of package \xpackage{babel} redefine
% \hologo{LaTeX}'s macros to include some font setup that
% breaks expandibility. Therefore \cs{AlphAlph} and \cs{alphalph}
% are based on \cs{alphalph@Alph} and \cs{alphalph@alph} to get
% the letters. The behaviour of these symbol providers for numbers
% outside the range |1..26| is undefined.
%
% \subsection{Design principles}
%
% \subsubsection{Number presentation commands}
%
% All number presentation commands that this package defines
% (including \cs{alphalph} and \cs{AlphAlph}) have the following
% properties:
% \begin{itemize}
% \item They are fully expandable. This means that they can safely
%       \begin{itemize}
%       \item be written to a file,
%       \item used in moving arguments (\hologo{LaTeX}: they are \emph{robust}),
%       \item used in a \cmd{\csname}-\cmd{\endcsname} pair.
%       \end{itemize}
% \item If the argument is zero or negative, the commands expand
%       to nothing like \cmd{\romannumeral}.
% \item The argument is a \hologo{TeX} number. Anything that would be
%       accepted by \cmd{\number} is a valid argument:
%       \begin{itemize}
%       \item explicite constants,
%       \item macros that expand to a number,
%       \item count registers, \hologo{LaTeX} counter can used
%             via \cmd{\value}, e.\,g.:\\
%             |\alphalph{\value{page}}|
%       \item \dots
%       \end{itemize}
% \item \hologo{eTeX}'s numeric expressions are supported, if \hologo{eTeX} is
%       available. Then \cs{numexpr} is applied to the argument.
%       Package \cs{calc}'s expressions are not supported.
%       That would violate the expandibility.
% \end{itemize}
%
% \subsubsection{General usability}
%
% \begin{description}
% \item[\hologo{TeX} format:]
%   The package does not depend on \hologo{LaTeX}, it can also be used
%   by \hologo{plainTeX}, for example.
% \item[\hologo{eTeX}:] \eTeX is supported, the macros are shorter and
%   faster. But \eTeX's extensions are not requirements.
%   Without \hologo{eTeX}, just the implementation changes. The properties
%   remain unchanged.
% \end{description}
%
% \StopEventually{
% }
%
% \section{Implementation}
% \subsection{Begin\texorpdfstring{ of package}{}}
%    \begin{macrocode}
%<*package>
%    \end{macrocode}
%    Reload check, especially if the package is not used with \LaTeX.
%    \begin{macrocode}
\begingroup\catcode61\catcode48\catcode32=10\relax%
  \catcode13=5 % ^^M
  \endlinechar=13 %
  \catcode35=6 % #
  \catcode39=12 % '
  \catcode44=12 % ,
  \catcode45=12 % -
  \catcode46=12 % .
  \catcode58=12 % :
  \catcode64=11 % @
  \catcode123=1 % {
  \catcode125=2 % }
  \expandafter\let\expandafter\x\csname ver@alphalph.sty\endcsname
  \ifx\x\relax % plain-TeX, first loading
  \else
    \def\empty{}%
    \ifx\x\empty % LaTeX, first loading,
      % variable is initialized, but \ProvidesPackage not yet seen
    \else
      \expandafter\ifx\csname PackageInfo\endcsname\relax
        \def\x#1#2{%
          \immediate\write-1{Package #1 Info: #2.}%
        }%
      \else
        \def\x#1#2{\PackageInfo{#1}{#2, stopped}}%
      \fi
      \x{alphalph}{The package is already loaded}%
      \aftergroup\endinput
    \fi
  \fi
\endgroup%
%    \end{macrocode}
%    Package identification:
%    \begin{macrocode}
\begingroup\catcode61\catcode48\catcode32=10\relax%
  \catcode13=5 % ^^M
  \endlinechar=13 %
  \catcode35=6 % #
  \catcode39=12 % '
  \catcode40=12 % (
  \catcode41=12 % )
  \catcode44=12 % ,
  \catcode45=12 % -
  \catcode46=12 % .
  \catcode47=12 % /
  \catcode58=12 % :
  \catcode64=11 % @
  \catcode91=12 % [
  \catcode93=12 % ]
  \catcode123=1 % {
  \catcode125=2 % }
  \expandafter\ifx\csname ProvidesPackage\endcsname\relax
    \def\x#1#2#3[#4]{\endgroup
      \immediate\write-1{Package: #3 #4}%
      \xdef#1{#4}%
    }%
  \else
    \def\x#1#2[#3]{\endgroup
      #2[{#3}]%
      \ifx#1\@undefined
        \xdef#1{#3}%
      \fi
      \ifx#1\relax
        \xdef#1{#3}%
      \fi
    }%
  \fi
\expandafter\x\csname ver@alphalph.sty\endcsname
\ProvidesPackage{alphalph}%
  [2016/05/16 v2.5 Convert numbers to letters (HO)]%
%    \end{macrocode}
%
% \subsection{Catcodes}
%
%    \begin{macrocode}
\begingroup\catcode61\catcode48\catcode32=10\relax%
  \catcode13=5 % ^^M
  \endlinechar=13 %
  \catcode123=1 % {
  \catcode125=2 % }
  \catcode64=11 % @
  \def\x{\endgroup
    \expandafter\edef\csname AlPh@AtEnd\endcsname{%
      \endlinechar=\the\endlinechar\relax
      \catcode13=\the\catcode13\relax
      \catcode32=\the\catcode32\relax
      \catcode35=\the\catcode35\relax
      \catcode61=\the\catcode61\relax
      \catcode64=\the\catcode64\relax
      \catcode123=\the\catcode123\relax
      \catcode125=\the\catcode125\relax
    }%
  }%
\x\catcode61\catcode48\catcode32=10\relax%
\catcode13=5 % ^^M
\endlinechar=13 %
\catcode35=6 % #
\catcode64=11 % @
\catcode123=1 % {
\catcode125=2 % }
\def\TMP@EnsureCode#1#2{%
  \edef\AlPh@AtEnd{%
    \AlPh@AtEnd
    \catcode#1=\the\catcode#1\relax
  }%
  \catcode#1=#2\relax
}
\TMP@EnsureCode{33}{12}% !
\TMP@EnsureCode{39}{12}% '
\TMP@EnsureCode{40}{12}% (
\TMP@EnsureCode{41}{12}% )
\TMP@EnsureCode{43}{12}% +
\TMP@EnsureCode{44}{12}% ,
\TMP@EnsureCode{46}{12}% .
\TMP@EnsureCode{47}{12}% /
\TMP@EnsureCode{59}{12}% ;
\TMP@EnsureCode{60}{12}% <
\TMP@EnsureCode{62}{12}% >
\TMP@EnsureCode{91}{12}% [
\TMP@EnsureCode{93}{12}% ]
\TMP@EnsureCode{96}{12}% `
\TMP@EnsureCode{124}{12}% |
\edef\AlPh@AtEnd{\AlPh@AtEnd\noexpand\endinput}
%    \end{macrocode}
%
% \subsection{Package loading}
%
%    \begin{macrocode}
\begingroup\expandafter\expandafter\expandafter\endgroup
\expandafter\ifx\csname RequirePackage\endcsname\relax
  \input infwarerr.sty\relax
  \input intcalc.sty\relax
\else
  \RequirePackage{infwarerr}[2007/09/09]%
  \RequirePackage{intcalc}[2007/09/09]%
\fi
%    \end{macrocode}
%
% \subsection{\hologo{eTeX} detection}
%
%    \begin{macrocode}
\begingroup\expandafter\expandafter\expandafter\endgroup
\expandafter\ifx\csname numexpr\endcsname\relax
  \catcode124=9 % '!': ignore
  \catcode43=14 % '+': comment
\else
  \catcode124=14 % '!': comment
  \catcode43=9   % '+': ignore
\fi
%    \end{macrocode}
%
% \subsection{Help macros}
%
%    \begin{macro}{\AlPh@Error}
%    \begin{macrocode}
\def\AlPh@Error#1{%
  \begingroup
    \escapechar=92 % backslash
    \@PackageError{alphalph}{#1}\@ehc
  \endgroup
}
%    \end{macrocode}
%    \end{macro}
%
%    \begin{macro}{\AlPh@IfDefinable}
%    \begin{macrocode}
\begingroup\expandafter\expandafter\expandafter\endgroup
\expandafter\ifx\csname @ifdefinable\endcsname\relax
  \def\AlPh@IfDefinable#1#2{%
    \ifcase\ifx#1\@undefined\else\ifx#1\relax\else1\fi\fi0 %
      #2%
    \else
      \AlPh@Error{%
        Command \string#1 already defined%
      }%
    \fi
  }%
%    \end{macrocode}
%    \end{macro}
%    \begin{macrocode}
\else
%    \end{macrocode}
%    \begin{macro}{\AlPh@IfDefinable}
%    \begin{macrocode}
  \let\AlPh@IfDefinable\@ifdefinable
%    \end{macrocode}
%    \end{macro}
%    \begin{macrocode}
\fi
%    \end{macrocode}
%
%    \begin{macro}{\@ReturnAfterElseFi}
%    \begin{macro}{\@ReturnAfterFi}
%    The following commands moves the `then' and `else' part respectively
%    behind the |\if|-construct. This prevents a too deep |\if|-nesting
%    and so a \hologo{TeX} capacity error because of a limited input stack size.
%    I use this trick in several packages, so I don't prefix these internal
%    commands in order not to have the same macros with different names.
%    (It saves memory.)
%    \begin{macrocode}
\long\def\@ReturnAfterElseFi#1\else#2\fi{\fi#1}
\long\def\@ReturnAfterFi#1\fi{\fi#1}
%    \end{macrocode}
%    \end{macro}
%    \end{macro}
%
%    \begin{macro}{\@gobblefour}
%    \Hologo{LaTeX} defines commands for eating arguments.
%    Define \cs{@gobblefour} if it is not defined (\hologo{plainTeX}).
%    \begin{macrocode}
\expandafter\ifx\csname @gobblefour\endcsname\relax
  \long\def\@gobblefour#1#2#3#4{}%
\fi
%    \end{macrocode}
%    \end{macro}
%
%    \begin{macro}{AlPh@IfOptArg}
%    \begin{macrocode}
\begingroup\expandafter\expandafter\expandafter\endgroup
\expandafter\ifx\csname kernel@ifnextchar\endcsname\relax
  \begingroup\expandafter\expandafter\expandafter\endgroup
  \expandafter\ifx\csname @ifnextchar\endcsname\relax
    \def\AlPh@IfOptArg#1#2{%
      \def\AlPh@TempA{#1}%
      \def\AlPh@TempB{#2}%
      \futurelet\AlPh@Token\AlPh@IfOptArgNext
    }%
    \let\AlPh@BracketLeft=[%]
    \def\AlPh@IfOptArgNext{%
      \ifx\AlPh@Token\AlPh@BracketLeft
        \expandafter\AlPh@TempA
      \else
        \expandafter\AlPh@TempB
      \fi
    }%
  \else
    \def\AlPh@IfOptArg{\@ifnextchar[}%]
  \fi
\else
  \def\AlPh@IfOptArg{\kernel@ifnextchar[}%]
\fi
%    \end{macrocode}
%    \end{macro}
%
% \subsection{Symbol provider}
%
% \subsubsection{Alphabet}
%
%    The output of \cs{alphalph} and \cs{AlphAlph} should be
%    usable as part of command names (see \cs{@namedef}, \cs{csname}, \dots).
%    Unhappily some languages of package \xpackage{babel}
%    redefine \hologo{LaTeX}'s \cs{@alph} and \cs{@Alph} in a manner that
%    they cannot be used in expandable context any more.
%    Therefore package \xpackage{alphalph} provides its own
%    commands.
%
%    \begin{macro}{\alphalph@Alph}
%    \begin{macro}{\alphalph@alph}
%    The two commands \cmd{\AlPh@Alph} and \cmd{\AlPh@alph} convert a
%    number into a letter (uppercase and lowercase respectivly).
%    The character |@| is used as an error symbol,
%    if the number isn't in the range of 1 until 26.
%    Here we need no space after the number |#1|, because the error
%    symbol |@| for the zero case stops scanning the number.
%    This error symbol should not appear anywhere (except for bugs).
%    \begin{macrocode}
\def\alphalph@Alph#1{%
  \ifcase#1%
    @%
  \or A\or B\or C\or D\or E\or F\or G\or H\or I\or J\or K\or L\or M%
  \or N\or O\or P\or Q\or R\or S\or T\or U\or V\or W\or X\or Y\or Z%
  \else
    \AlPh@ctrerr
    @%
  \fi
}
\def\alphalph@alph#1{%
  \ifcase#1%
    @%
  \or a\or b\or c\or d\or e\or f\or g\or h\or i\or j\or k\or l\or m%
  \or n\or o\or p\or q\or r\or s\or t\or u\or v\or w\or x\or y\or z%
  \else
    \AlPh@ctrerr
    @%
  \fi
}
%    \end{macrocode}
%    \end{macro}
%    \end{macro}
%    \begin{macro}{\AlPh@ctrerr}
%    Macro \cs{AlPh@ctrerr} is used as hook for the algorithm
%    to get the available number of symbols.
%    \begin{macrocode}
\def\AlPh@ctrerr{}
%    \end{macrocode}
%    \end{macro}
%
% \subsection{Finding number of symbols}
%
%    \begin{macro}{\AlPh@GetNumberOfSymbols}
%    \noindent
%    |#1|: symbols provider
%    \begin{macrocode}
\def\AlPh@GetNumberOfSymbols#1{%
  \AlPh@TestNumber1!{#1}%
  \ifAlPh@Unavailable
    \def\AlPh@Number{0}%
    \AlPh@Error{No symbols found}%
  \else
    \def\AlPh@Number{1}%
    \AlPh@ExpSearch2!{#1}%
  \fi
}
%    \end{macrocode}
%    \end{macro}
%    \begin{macro}{\ifAlPh@Unavailable}
%    \begin{macrocode}
\let\ifAlPh@Unavailable\iffalse
\def\AlPh@Unavailabletrue{%
  \global\let\ifAlPh@Unavailable\iftrue
}
\def\AlPh@Unavailablefalse{%
  \global\let\ifAlPh@Unavailable\iffalse
}
%    \end{macrocode}
%    \end{macro}
%    \begin{macro}{\AlPh@TestNumber}
%    |#1|: number to be tested\\
%    |#2|: symbols provider
%    \begin{macrocode}
\def\AlPh@TestNumber#1!#2{%
  \AlPh@Unavailablefalse
  \begingroup
    \setbox0=\hbox{%
      \begingroup % color
        \let\@ctrerr\AlPh@Unavailabletrue
        \let\AlPh@ctrerr\AlPh@Unavailabletrue
        #2{#1}%
      \endgroup
    }%
    \ifdim\wd0=0pt %
      \AlPh@Unavailabletrue
    \fi
  \endgroup
}
%    \end{macrocode}
%    \end{macro}
%    \begin{macro}{\AlPh@ExpSearch}
%    |#1|: number to be tested\\
%    |#2|: symbols provider
%    \begin{macrocode}
\def\AlPh@ExpSearch#1!#2{%
  \let\AlPh@Next\relax
  \AlPh@TestNumber#1!{#2}%
  \ifAlPh@Unavailable
    \expandafter\AlPh@BinSearch\AlPh@Number!#1!{#2}%
  \else
    \def\AlPh@Number{#1}%
    \ifnum#1>1073741823 %
      \AlPh@TestNumber2147483647!{#2}%
      \ifAlPh@Unavailable
        \AlPh@BinSearch#1!2147483647!{#2}%
      \else
        \def\AlPh@Number{0}%
        \AlPh@Error{%
          Maximal symbol number not found%
        }%
      \fi
    \else
      \def\AlPh@Next{%
        \expandafter\AlPh@ExpSearch\number\intcalcShl{#1}!{#2}%
      }%
    \fi
  \fi
  \AlPh@Next
}
%    \end{macrocode}
%    \end{macro}
%    \begin{macro}{\AlPh@BinSearch}
%    |#1|: available number\\
%    |#2|: unavailable number, |#2| $>$ |#1|\\
%    |#3|: symbols provider
%    \begin{macrocode}
\def\AlPh@BinSearch#1!#2!#3{%
  \expandafter\AlPh@ProcessBinSearch
  \number\intcalcShr{\intcalcAdd{#1}{#2}}!%
  #1!#2!{#3}%
}
%    \end{macrocode}
%    \end{macro}
%    \begin{macro}{\AlPh@ProcessBinSearch}
%    |#1|: number to be tested, |#2| $\leq$ |#1| $\leq$ |#3|\\
%    |#2|: available number\\
%    |#3|: unavailable number\\
%    |#4|: symbols provider
%    \begin{macrocode}
\def\AlPh@ProcessBinSearch#1!#2!#3!#4{%
  \let\AlPh@Next\relax
  \ifnum#1>#2 %
    \ifnum#1<#3 %
      \AlPh@TestNumber#1!{#4}%
      \ifAlPh@Unavailable
        \def\AlPh@Next{%
          \AlPh@BinSearch#2!#1!{#4}%
        }%
      \else
        \def\AlPh@Next{%
          \AlPh@BinSearch#1!#3!{#4}%
        }%
      \fi
    \else
      \def\AlPh@Number{#2}%
    \fi
  \else
    \def\AlPh@Number{#2}%
  \fi
  \AlPh@Next
}
%    \end{macrocode}
%    \end{macro}
%
% \subsection{Methods}
%
%    The names of method macros start with \cs{AlPh@Method}.
%    These macros do the main job in converting a number to
%    its representation.
%    A method command is called with three arguments.
%    The first argument is the number of symbols. The
%    second argument is the basic macro for converting
%    a number with limited number range. The last parameter
%    is the number that needs converting.
%
% \subsubsection{Common methods}
%
%    \begin{macro}{\AlPh@CheckPositive}
%    \noindent
%    |#1|: number to be checked
%    |#2|: continuation macro\\
%    |#3|: number of symbols (hidden here)\\
%    |#4|: symbol provider (hidden here)
%    \begin{macrocode}
\def\AlPh@CheckPositive#1!#2{%
  \ifnum#1<1 %
    \expandafter\@gobblefour
  \fi
  #2{#1}%
}
%    \end{macrocode}
%    \end{macro}
%
% \subsubsection{Method `alph'}
%
%    \begin{macro}{\AlPh@Method@alph}
%    \noindent
%    |#1|: number of symbols\\
%    |#2|: symbols provider\\
%    |#3|: number to be converted
%    \begin{macrocode}
\def\AlPh@Method@alph#1#2#3{%
  \expandafter\AlPh@CheckPositive
|   \number#3!%
+   \the\numexpr#3!%
    \AlPh@ProcessAlph
    {#1}{#2}%
}
%    \end{macrocode}
%    \end{macro}
%    \begin{macro}{\AlPh@ProcessAlph}
%    |#1|: current number\\
%    |#2|: number of symbols\\
%    |#3|: symbols provider
%    \begin{macrocode}
\def\AlPh@ProcessAlph#1#2#3{%
  \ifnum#1>#2 %
    \@ReturnAfterElseFi{%
      \expandafter\AlPh@StepAlph\number
        \intcalcInc{%
          \intcalcMod{\intcalcDec{#1}}{#2}%
        }%
      \expandafter!\number
        \intcalcDiv{\intcalcDec{#1}}{#2}%
      !{#2}{#3}%
    }%
  \else
    \@ReturnAfterFi{%
      #3{#1}%
    }%
  \fi
}
%    \end{macrocode}
%    \end{macro}
%    \begin{macro}{\AlPh@StepAlph}
%    |#1|: current last digit\\
%    |#2|: new current number\\
%    |#3|: number of symbols\\
%    |#4|: symbols provider
%    \begin{macrocode}
\def\AlPh@StepAlph#1!#2!#3#4{%
  \AlPh@ProcessAlph{#2}{#3}{#4}%
  #4{#1}%
}
%    \end{macrocode}
%    \end{macro}
%
% \subsubsection{Method `wrap'}
%
%    \begin{macro}{\AlPh@Method@wrap}
%    \noindent
%    |#1|: number of symbols\\
%    |#2|: symbols provider\\
%    |#3|: number to be converted
%    \begin{macrocode}
\def\AlPh@Method@wrap#1#2#3{%
  \expandafter\AlPh@CheckPositive
|   \number#3!%
+   \the\numexpr#3!%
    \AlPh@ProcessWrap
    {#1}{#2}%
}
%    \end{macrocode}
%    \end{macro}
%
%    \begin{macro}{\AlPh@ProcessWrap}
%    |#1|: number to be converted\\
%    |#2|: number of symbols\\
%    |#3|: symbols provider
%    \begin{macrocode}
\def\AlPh@ProcessWrap#1#2#3{%
  \ifnum#1>#2 %
    \@ReturnAfterElseFi{%
      \expandafter\AlPh@StepWrap\number
        \intcalcInc{\intcalcMod{\intcalcDec{#1}}{#2}}%
      !{#3}%
    }%
  \else
    \@ReturnAfterFi{%
      #3{#1}%
    }%
  \fi
}
%    \end{macrocode}
%    \end{macro}
%    \begin{macro}{\AlPh@StepWrap}
%    |#1|: final number\\
%    |#2|: symbols provider
%    \begin{macrocode}
\def\AlPh@StepWrap#1!#2{%
  #2{#1}%
}
%    \end{macrocode}
%    \end{macro}
%
% \subsubsection{Method `mult'}
%
%    After the number of symbols is exhausted,
%    repetitions of the symbol are used.
%    \begin{gather*}
%      x \Def \text{number to be converted}\\
%      n \Def \text{number of symbols}\\
%      r \Def \text{repetition length}\\
%      s \Def \text{symbol slot}\\
%      r = ((x - 1) \div n) + 1\\
%      s = ((x - 1) \mod n) + 1
%    \end{gather*}
%    \begin{macro}{\AlPh@Method@mult}
%    \noindent
%    |#1|: number of symbols\\
%    |#2|: symbols provider\\
%    |#3|: number to be converted
%    \begin{macrocode}
\def\AlPh@Method@mult#1#2#3{%
  \expandafter\AlPh@CheckPositive
|   \number#3!%
+   \the\numexpr#3!%
    \AlPh@ProcessMult
    {#1}{#2}%
}
%    \end{macrocode}
%    \end{macro}
%
%    \begin{macro}{\AlPh@ProcessMult}
%    |#1|: number to be converted\\
%    |#2|: number of symbols\\
%    |#3|: symbols provider
%    \begin{macrocode}
\def\AlPh@ProcessMult#1#2#3{%
  \ifnum#1>#2 %
    \@ReturnAfterElseFi{%
      \expandafter\AlPh@StepMult\romannumeral
        \intcalcInc{\intcalcDiv{\intcalcDec{#1}}{#2}}%
        000%
      \expandafter!\number
        \intcalcInc{\intcalcMod{\intcalcDec{#1}}{#2}}%
      !{#3}%
    }%
  \else
    \@ReturnAfterFi{%
      #3{#1}%
    }%
  \fi
}
%    \end{macrocode}
%    \end{macro}
%    \begin{macro}{\AlPh@StepMult}
%    |#1#2|: repetitions coded as list of character `m'\\
%    |#3|: symbol slot\\
%    |#4|: symbols provider
%    \begin{macrocode}
\def\AlPh@StepMult#1#2!#3!#4{%
  \ifx\\#2\\%
  \else
    \@ReturnAfterFi{%
      \AlPh@StepMult#2!#3!{#4}%
    }%
  \fi
  #4{#3}%
}
%    \end{macrocode}
%    \end{macro}
%
% \subsection{User interface}
%
%    \begin{macro}{\newalphalph}
%    Macro \cs{newalphalph} had three arguments in versions below 2.0.
%    For the new method argument we use an optional argument an first
%    position.\\
%    |#1|: cmd\\
%    |[#2]|: method name: |alph| (default), |wrap|, |mult|\\ % hash-ok
%    |#3|: symbols provider\\
%    |#4|: number of symbols
%    \begin{macrocode}
\AlPh@IfDefinable\newalphalph{%
  \def\newalphalph#1{%
    \AlPh@IfOptArg{%
      \AlPh@newalphalph{#1}%
    }{%
      \AlPh@newalphalph{#1}[alph]%
    }%
  }%
}
%    \end{macrocode}
%    \end{macro}
%    \begin{macro}{\AlPh@newalphalph}
%    |#1|: cmd
%    |#2|: method name\\
%    |#3|: symbols provider\\
%    |#4|: number of symbols
%    \begin{macrocode}
\def\AlPh@newalphalph#1[#2]#3#4{%
  \begingroup\expandafter\expandafter\expandafter\endgroup
  \expandafter\ifx\csname AlPh@Method@#2\endcsname\relax
    \AlPh@Error{%
      Unknown method %
|     `#2'%
+     `\detokenize{#2}'%
    }%
  \else
    \ifx\\#4\\%
      \AlPh@GetNumberOfSymbols{#3}%
      \ifcase\AlPh@Number
      \else
        \begingroup
          \escapechar=92 % backslash
          \@PackageInfo{alphalph}{%
            Number of symbols for \string#1 is \AlPh@Number
          }%
        \endgroup
        \expandafter\AlPh@NewAlphAlph
        \csname AlPh@Method@#2\expandafter\endcsname
        \AlPh@Number!{#1}{#3}%
      \fi
    \else
      \expandafter\AlPh@NewAlphAlph
      \csname AlPh@Method@#2\expandafter\endcsname
|     \number#4!%
+     \the\numexpr#4!%
      {#1}{#3}%
    \fi
  \fi
}%
%    \end{macrocode}
%    \end{macro}
%    \begin{macro}{\AlPh@NewAlphAlph}
%    |#1|: method macro\\
%    |#2|: number of symbols\\
%    |#3|: cmd\\
%    |#4|: symbols provider
%    \begin{macrocode}
\def\AlPh@NewAlphAlph#1#2!#3#4{%
  \AlPh@IfDefinable#3{%
    \ifnum#2>0 %
      \def#3{#1{#2}{#4}}%
    \else
      \AlPh@Error{%
          Definition of \string#3 failed,\MessageBreak
          because number of symbols (#2) is not positive%
        }%
    \fi
  }%
}
%    \end{macrocode}
%    \end{macro}
%
%    \begin{macro}{\AlphAlph}
%    \begin{macrocode}
\newalphalph\AlphAlph\alphalph@Alph{26}
%    \end{macrocode}
%    \end{macro}
%    \begin{macro}{\alphalph}
%    \begin{macrocode}
\newalphalph\alphalph\alphalph@alph{26}
%    \end{macrocode}
%    \end{macro}
%
%    \begin{macrocode}
\AlPh@AtEnd%
%</package>
%    \end{macrocode}
%
% \section{Test}
%
% \subsection{Catcode checks for loading}
%
%    \begin{macrocode}
%<*test1>
%    \end{macrocode}
%    \begin{macrocode}
\catcode`\{=1 %
\catcode`\}=2 %
\catcode`\#=6 %
\catcode`\@=11 %
\expandafter\ifx\csname count@\endcsname\relax
  \countdef\count@=255 %
\fi
\expandafter\ifx\csname @gobble\endcsname\relax
  \long\def\@gobble#1{}%
\fi
\expandafter\ifx\csname @firstofone\endcsname\relax
  \long\def\@firstofone#1{#1}%
\fi
\expandafter\ifx\csname loop\endcsname\relax
  \expandafter\@firstofone
\else
  \expandafter\@gobble
\fi
{%
  \def\loop#1\repeat{%
    \def\body{#1}%
    \iterate
  }%
  \def\iterate{%
    \body
      \let\next\iterate
    \else
      \let\next\relax
    \fi
    \next
  }%
  \let\repeat=\fi
}%
\def\RestoreCatcodes{}
\count@=0 %
\loop
  \edef\RestoreCatcodes{%
    \RestoreCatcodes
    \catcode\the\count@=\the\catcode\count@\relax
  }%
\ifnum\count@<255 %
  \advance\count@ 1 %
\repeat

\def\RangeCatcodeInvalid#1#2{%
  \count@=#1\relax
  \loop
    \catcode\count@=15 %
  \ifnum\count@<#2\relax
    \advance\count@ 1 %
  \repeat
}
\def\RangeCatcodeCheck#1#2#3{%
  \count@=#1\relax
  \loop
    \ifnum#3=\catcode\count@
    \else
      \errmessage{%
        Character \the\count@\space
        with wrong catcode \the\catcode\count@\space
        instead of \number#3%
      }%
    \fi
  \ifnum\count@<#2\relax
    \advance\count@ 1 %
  \repeat
}
\def\space{ }
\expandafter\ifx\csname LoadCommand\endcsname\relax
  \def\LoadCommand{\input alphalph.sty\relax}%
\fi
\def\Test{%
  \RangeCatcodeInvalid{0}{47}%
  \RangeCatcodeInvalid{58}{64}%
  \RangeCatcodeInvalid{91}{96}%
  \RangeCatcodeInvalid{123}{255}%
  \catcode`\@=12 %
  \catcode`\\=0 %
  \catcode`\%=14 %
  \LoadCommand
  \RangeCatcodeCheck{0}{36}{15}%
  \RangeCatcodeCheck{37}{37}{14}%
  \RangeCatcodeCheck{38}{47}{15}%
  \RangeCatcodeCheck{48}{57}{12}%
  \RangeCatcodeCheck{58}{63}{15}%
  \RangeCatcodeCheck{64}{64}{12}%
  \RangeCatcodeCheck{65}{90}{11}%
  \RangeCatcodeCheck{91}{91}{15}%
  \RangeCatcodeCheck{92}{92}{0}%
  \RangeCatcodeCheck{93}{96}{15}%
  \RangeCatcodeCheck{97}{122}{11}%
  \RangeCatcodeCheck{123}{255}{15}%
  \RestoreCatcodes
}
\Test
\csname @@end\endcsname
\end
%    \end{macrocode}
%    \begin{macrocode}
%</test1>
%    \end{macrocode}
%
% \section{Macro tests}
%
%    \begin{macrocode}
%<*test2>
\NeedsTeXFormat{LaTeX2e}
\nofiles
\documentclass{article}
%<*noetex>
\makeatletter
\let\saved@numexpr\numexpr
\newcommand*{\DisableNumexpr}{%
  \let\numexpr\@undefined
}
\newcommand*{\RestoreNumexpr}{%
  \let\numexpr\saved@numexpr
}
\DisableNumexpr
%</noetex>
\usepackage{alphalph}[2016/05/16]
%<noetex>\RestoreNumexpr
\usepackage{qstest}
\IncludeTests{*}
\LogTests{log}{*}{*}

\newcommand*{\TestCmd}[3]{%
  \setbox0=\hbox{%
%<noetex>    \DisableNumexpr
    \edef\TestString{#1{#2}}%
    \expandafter\Expect\expandafter{\TestString}{#3}%
    \edef\TestString{#1{#2} }%
    \expandafter\Expect\expandafter{\TestString}{#3 }%
  }%
  \Expect*{\the\wd0}{0.0pt}%
}

\makeatletter
\newalphalph\LaTeXAlphAlph\@Alph{26}
\newalphalph\LaTeXalphalph\@alph{26}
\newalphalph\AlphWrap[wrap]\alphalph@Alph{26}
\newalphalph\alphwrap[wrap]\alphalph@alph{26}
\newalphalph\LaTeXAlphWrap[wrap]\@Alph{26}
\newalphalph\LaTeXalphwrap[wrap]\@alph{26}
\def\LastSymbol#1{%
  \ifx\\#1\\%
  \else
    \@LastSymbol#1\@nil
  \fi
}
\def\@LastSymbol#1#2\@nil{%
  \ifx\\#2\\%
    #1%
  \else
    \@LastSymbol#2\@nil
  \fi
}
\makeatother
\newcommand*{\TestAlph}[2]{%
  \uppercase{\TestCallCmd\AlphAlph{#2}}{#1}%
  \lowercase{\TestCallCmd\alphalph{#2}}{#1}%
  \uppercase{\TestCallCmd\LaTeXAlphAlph{#2}}{#1}%
  \lowercase{\TestCallCmd\LaTeXalphalph{#2}}{#1}%
  \edef\WrapString{\LastSymbol{#2}}%
  \expandafter\TestAlphWrap\expandafter{\WrapString}{#1}%
}
\newcommand*{\TestAlphWrap}[2]{%
  \uppercase{\TestCallCmd\AlphWrap{#1}}{#2}%
  \lowercase{\TestCallCmd\alphwrap{#1}}{#2}%
  \uppercase{\TestCallCmd\LaTeXAlphWrap{#1}}{#2}%
  \lowercase{\TestCallCmd\LaTeXalphwrap{#1}}{#2}%
}
\newcommand*{\TestCallCmd}[3]{%
  \TestCmd#1{#3}{#2}%
}
\begin{qstest}{AlphSymbols}{alphalph, AlphAlph, symbols}
  \TestAlph{1}{a}%
  \TestAlph{2}{b}%
  \TestAlph{3}{c}%
  \TestAlph{4}{d}%
  \TestAlph{5}{e}%
  \TestAlph{6}{f}%
  \TestAlph{7}{g}%
  \TestAlph{8}{h}%
  \TestAlph{9}{i}%
  \TestAlph{10}{j}%
  \TestAlph{11}{k}%
  \TestAlph{12}{l}%
  \TestAlph{13}{m}%
  \TestAlph{14}{n}%
  \TestAlph{15}{o}%
  \TestAlph{16}{p}%
  \TestAlph{17}{q}%
  \TestAlph{18}{r}%
  \TestAlph{19}{s}%
  \TestAlph{20}{t}%
  \TestAlph{21}{u}%
  \TestAlph{22}{v}%
  \TestAlph{23}{w}%
  \TestAlph{24}{x}%
  \TestAlph{25}{y}%
  \TestAlph{26}{z}%
\end{qstest}
\begin{qstest}{AlphRange}{alphalph, range}
  \TestAlph{0}{}%
  \TestAlph{-1}{}%
  \TestAlph{-2147483647}{}%
  \TestAlph{27}{aa}%
  \TestAlph{28}{ab}%
  \TestAlph{52}{az}%
  \TestAlph{53}{ba}%
  \TestAlph{78}{bz}%
  \TestAlph{79}{ca}%
  \TestAlph{702}{zz}%
  \TestAlph{703}{aaa}%
  \TestAlph{2147483647}{fxshrxw}%
\end{qstest}

\makeatletter
\newcommand*{\myvocals}[1]{%
  \ifcase#1X\or A\or E\or I\or O\or U\else Y\fi
}
\makeatother
\newalphalph\vocalsvocals\myvocals{5}
\newcommand*{\TestVocals}{%
  \TestCmd\vocalsvocals
}
\begin{qstest}{vocals}{vocals}
  \TestVocals{0}{}%
  \TestVocals{1}{A}%
  \TestVocals{2}{E}%
  \TestVocals{3}{I}%
  \TestVocals{4}{O}%
  \TestVocals{5}{U}%
  \TestVocals{6}{AA}%
  \TestVocals{7}{AE}%
  \TestVocals{8}{AI}%
  \TestVocals{9}{AO}%
  \TestVocals{10}{AU}%
  \TestVocals{11}{EA}%
  \TestVocals{24}{OO}%
  \TestVocals{25}{OU}%
  \TestVocals{26}{UA}%
  \TestVocals{29}{UO}%
  \TestVocals{30}{UU}%
  \TestVocals{31}{AAA}%
  \TestVocals{155}{UUU}%
  \TestVocals{156}{AAAA}%
  \TestVocals{2147483647}{AIIOOEEIOIIUOE}%
\end{qstest}

\makeatletter
\newalphalph\AlphMult[mult]{\alphalph@Alph}{26}
\newalphalph\alphmult[mult]{\alphalph@alph}{26}
\newalphalph\LaTeXAlphMult[mult]{\@Alph}{26}
\newalphalph\LaTeXalphmult[mult]{\@alph}{26}
\makeatother
\newcommand*{\TestMult}[2]{%
  \uppercase{\TestCallCmd\AlphMult{#2}}{#1}%
  \lowercase{\TestCallCmd\alphmult{#2}}{#1}%
  \uppercase{\TestCallCmd\LaTeXAlphMult{#2}}{#1}%
  \lowercase{\TestCallCmd\LaTeXalphmult{#2}}{#1}%
}
\begin{qstest}{mult}{mult}
  \TestMult{0}{}%
  \TestMult{-1}{}%
  \TestMult{-2147483647}{}%
  \TestMult{1}{a}%
  \TestMult{2}{b}%
  \TestMult{26}{z}%
  \TestMult{27}{aa}%
  \TestMult{28}{bb}%
  \TestMult{52}{zz}%
  \TestMult{53}{aaa}%
  \TestMult{54}{bbb}%
  \TestMult{259}{yyyyyyyyyy}%
  \TestMult{260}{zzzzzzzzzz}%
  \TestMult{261}{aaaaaaaaaaa}%
  \TestMult{262}{bbbbbbbbbbb}%
\end{qstest}

\def\myvocalsB#1{%
  \ifcase#1\or A\or E\or I\or O\or U\fi
}
\begin{qstest}{symbolnum}{symbolnum}
  \makeatletter
  \def\Test#1#2{%
    \let\TestCmd\relax
    \newalphalph\TestCmd{#1}{}%
    \Expect*{\AlPh@Number}{#2}%
  }%
  \Test\@alph{26}%
  \Test\@Alph{26}%
  \Test\@fnsymbol{9}%
  \Test\myvocalsB{5}%
  \Test\alphalph@alph{26}%
  \Test\alphalph@Alph{26}%
\end{qstest}

\begin{qstest}{list}{list}
  \makeatletter
  \def\catch#1\relax{%
    \def\FoundList{\catch#1}%
  }%
  \def\Test[#1]#2#3#4{%
    \let\testcmd\relax
    \newalphalph\testcmd[{#1}]{\catch}{#2}%
    \testcmd{#3}|\relax
    \expandafter\Expect\expandafter{\FoundList}{#4|}%
    %
    \let\SavedCatch\catch
    \def\catch{\noexpand\catch\noexpand\foo}%
    \edef\Result{#4|}%
    \@onelevel@sanitize\Result
    \let\catch\SavedCatch
    \let\testcmd\relax
    \newalphalph\testcmd[{#1}]{\catch\foo}{#2}%
    \testcmd{#3}|\relax
    \@onelevel@sanitize\FoundList
    \Expect*{\FoundList}*{\Result}%
  }%
  \Test[alph]{26}{3}{\catch{3}}%
  \Test[alph]{26}{12}{\catch{12}}%
  \Test[alph]{26}{27}{\catch{1}\catch{1}}%
  \Test[alph]{26}{78}{\catch{2}\catch{26}}%
  \Test[wrap]{26}{7}{\catch{7}}%
  \Test[wrap]{26}{14}{\catch{14}}%
  \Test[wrap]{26}{80}{\catch{2}}%
  \Test[wrap]{26}{700}{\catch{24}}%
  \Test[mult]{26}{4}{\catch{4}}%
  \Test[mult]{26}{17}{\catch{17}}%
  \Test[mult]{26}{54}{\catch{2}\catch{2}\catch{2}}%
\end{qstest}

\begin{document}
\end{document}
%</test2>
%    \end{macrocode}
%
% \section{Installation}
%
% \subsection{Download}
%
% \paragraph{Package.} This package is available on
% CTAN\footnote{\CTANpkg{alphalph}}:
% \begin{description}
% \item[\CTAN{macros/latex/contrib/oberdiek/alphalph.dtx}] The source file.
% \item[\CTAN{macros/latex/contrib/oberdiek/alphalph.pdf}] Documentation.
% \end{description}
%
%
% \paragraph{Bundle.} All the packages of the bundle `oberdiek'
% are also available in a TDS compliant ZIP archive. There
% the packages are already unpacked and the documentation files
% are generated. The files and directories obey the TDS standard.
% \begin{description}
% \item[\CTANinstall{install/macros/latex/contrib/oberdiek.tds.zip}]
% \end{description}
% \emph{TDS} refers to the standard ``A Directory Structure
% for \TeX\ Files'' (\CTAN{tds/tds.pdf}). Directories
% with \xfile{texmf} in their name are usually organized this way.
%
% \subsection{Bundle installation}
%
% \paragraph{Unpacking.} Unpack the \xfile{oberdiek.tds.zip} in the
% TDS tree (also known as \xfile{texmf} tree) of your choice.
% Example (linux):
% \begin{quote}
%   |unzip oberdiek.tds.zip -d ~/texmf|
% \end{quote}
%
% \paragraph{Script installation.}
% Check the directory \xfile{TDS:scripts/oberdiek/} for
% scripts that need further installation steps.
% Package \xpackage{attachfile2} comes with the Perl script
% \xfile{pdfatfi.pl} that should be installed in such a way
% that it can be called as \texttt{pdfatfi}.
% Example (linux):
% \begin{quote}
%   |chmod +x scripts/oberdiek/pdfatfi.pl|\\
%   |cp scripts/oberdiek/pdfatfi.pl /usr/local/bin/|
% \end{quote}
%
% \subsection{Package installation}
%
% \paragraph{Unpacking.} The \xfile{.dtx} file is a self-extracting
% \docstrip\ archive. The files are extracted by running the
% \xfile{.dtx} through \plainTeX:
% \begin{quote}
%   \verb|tex alphalph.dtx|
% \end{quote}
%
% \paragraph{TDS.} Now the different files must be moved into
% the different directories in your installation TDS tree
% (also known as \xfile{texmf} tree):
% \begin{quote}
% \def\t{^^A
% \begin{tabular}{@{}>{\ttfamily}l@{ $\rightarrow$ }>{\ttfamily}l@{}}
%   alphalph.sty & tex/generic/oberdiek/alphalph.sty\\
%   alphalph.pdf & doc/latex/oberdiek/alphalph.pdf\\
%   test/alphalph-test1.tex & doc/latex/oberdiek/test/alphalph-test1.tex\\
%   test/alphalph-test2.tex & doc/latex/oberdiek/test/alphalph-test2.tex\\
%   test/alphalph-test3.tex & doc/latex/oberdiek/test/alphalph-test3.tex\\
%   alphalph.dtx & source/latex/oberdiek/alphalph.dtx\\
% \end{tabular}^^A
% }^^A
% \sbox0{\t}^^A
% \ifdim\wd0>\linewidth
%   \begingroup
%     \advance\linewidth by\leftmargin
%     \advance\linewidth by\rightmargin
%   \edef\x{\endgroup
%     \def\noexpand\lw{\the\linewidth}^^A
%   }\x
%   \def\lwbox{^^A
%     \leavevmode
%     \hbox to \linewidth{^^A
%       \kern-\leftmargin\relax
%       \hss
%       \usebox0
%       \hss
%       \kern-\rightmargin\relax
%     }^^A
%   }^^A
%   \ifdim\wd0>\lw
%     \sbox0{\small\t}^^A
%     \ifdim\wd0>\linewidth
%       \ifdim\wd0>\lw
%         \sbox0{\footnotesize\t}^^A
%         \ifdim\wd0>\linewidth
%           \ifdim\wd0>\lw
%             \sbox0{\scriptsize\t}^^A
%             \ifdim\wd0>\linewidth
%               \ifdim\wd0>\lw
%                 \sbox0{\tiny\t}^^A
%                 \ifdim\wd0>\linewidth
%                   \lwbox
%                 \else
%                   \usebox0
%                 \fi
%               \else
%                 \lwbox
%               \fi
%             \else
%               \usebox0
%             \fi
%           \else
%             \lwbox
%           \fi
%         \else
%           \usebox0
%         \fi
%       \else
%         \lwbox
%       \fi
%     \else
%       \usebox0
%     \fi
%   \else
%     \lwbox
%   \fi
% \else
%   \usebox0
% \fi
% \end{quote}
% If you have a \xfile{docstrip.cfg} that configures and enables \docstrip's
% TDS installing feature, then some files can already be in the right
% place, see the documentation of \docstrip.
%
% \subsection{Refresh file name databases}
%
% If your \TeX~distribution
% (\teTeX, \mikTeX, \dots) relies on file name databases, you must refresh
% these. For example, \teTeX\ users run \verb|texhash| or
% \verb|mktexlsr|.
%
% \subsection{Some details for the interested}
%
% \paragraph{Attached source.}
%
% The PDF documentation on CTAN also includes the
% \xfile{.dtx} source file. It can be extracted by
% AcrobatReader 6 or higher. Another option is \textsf{pdftk},
% e.g. unpack the file into the current directory:
% \begin{quote}
%   \verb|pdftk alphalph.pdf unpack_files output .|
% \end{quote}
%
% \paragraph{Unpacking with \LaTeX.}
% The \xfile{.dtx} chooses its action depending on the format:
% \begin{description}
% \item[\plainTeX:] Run \docstrip\ and extract the files.
% \item[\LaTeX:] Generate the documentation.
% \end{description}
% If you insist on using \LaTeX\ for \docstrip\ (really,
% \docstrip\ does not need \LaTeX), then inform the autodetect routine
% about your intention:
% \begin{quote}
%   \verb|latex \let\install=y% \iffalse meta-comment
%
% File: alphalph.dtx
% Version: 2016/05/16 v2.5
% Info: Convert numbers to letters
%
% Copyright (C) 1999, 2006-2008, 2010, 2011 by
%    Heiko Oberdiek <heiko.oberdiek at googlemail.com>
%    2016
%    https://github.com/ho-tex/oberdiek/issues
%
% This work may be distributed and/or modified under the
% conditions of the LaTeX Project Public License, either
% version 1.3c of this license or (at your option) any later
% version. This version of this license is in
%    http://www.latex-project.org/lppl/lppl-1-3c.txt
% and the latest version of this license is in
%    http://www.latex-project.org/lppl.txt
% and version 1.3 or later is part of all distributions of
% LaTeX version 2005/12/01 or later.
%
% This work has the LPPL maintenance status "maintained".
%
% This Current Maintainer of this work is Heiko Oberdiek.
%
% The Base Interpreter refers to any `TeX-Format',
% because some files are installed in TDS:tex/generic//.
%
% This work consists of the main source file alphalph.dtx
% and the derived files
%    alphalph.sty, alphalph.pdf, alphalph.ins, alphalph.drv,
%    alphalph-test1.tex, alphalph-test2.tex, alphalph-test3.tex.
%
% Distribution:
%    CTAN:macros/latex/contrib/oberdiek/alphalph.dtx
%    CTAN:macros/latex/contrib/oberdiek/alphalph.pdf
%
% Unpacking:
%    (a) If alphalph.ins is present:
%           tex alphalph.ins
%    (b) Without alphalph.ins:
%           tex alphalph.dtx
%    (c) If you insist on using LaTeX
%           latex \let\install=y\input{alphalph.dtx}
%        (quote the arguments according to the demands of your shell)
%
% Documentation:
%    (a) If alphalph.drv is present:
%           latex alphalph.drv
%    (b) Without alphalph.drv:
%           latex alphalph.dtx; ...
%    The class ltxdoc loads the configuration file ltxdoc.cfg
%    if available. Here you can specify further options, e.g.
%    use A4 as paper format:
%       \PassOptionsToClass{a4paper}{article}
%
%    Programm calls to get the documentation (example):
%       pdflatex alphalph.dtx
%       makeindex -s gind.ist alphalph.idx
%       pdflatex alphalph.dtx
%       makeindex -s gind.ist alphalph.idx
%       pdflatex alphalph.dtx
%
% Installation:
%    TDS:tex/generic/oberdiek/alphalph.sty
%    TDS:doc/latex/oberdiek/alphalph.pdf
%    TDS:doc/latex/oberdiek/test/alphalph-test1.tex
%    TDS:doc/latex/oberdiek/test/alphalph-test2.tex
%    TDS:doc/latex/oberdiek/test/alphalph-test3.tex
%    TDS:source/latex/oberdiek/alphalph.dtx
%
%<*ignore>
\begingroup
  \catcode123=1 %
  \catcode125=2 %
  \def\x{LaTeX2e}%
\expandafter\endgroup
\ifcase 0\ifx\install y1\fi\expandafter
         \ifx\csname processbatchFile\endcsname\relax\else1\fi
         \ifx\fmtname\x\else 1\fi\relax
\else\csname fi\endcsname
%</ignore>
%<*install>
\input docstrip.tex
\Msg{************************************************************************}
\Msg{* Installation}
\Msg{* Package: alphalph 2016/05/16 v2.5 Convert numbers to letters (HO)}
\Msg{************************************************************************}

\keepsilent
\askforoverwritefalse

\let\MetaPrefix\relax
\preamble

This is a generated file.

Project: alphalph
Version: 2016/05/16 v2.5

Copyright (C) 1999, 2006-2008, 2010, 2011 by
   Heiko Oberdiek <heiko.oberdiek at googlemail.com>

This work may be distributed and/or modified under the
conditions of the LaTeX Project Public License, either
version 1.3c of this license or (at your option) any later
version. This version of this license is in
   http://www.latex-project.org/lppl/lppl-1-3c.txt
and the latest version of this license is in
   http://www.latex-project.org/lppl.txt
and version 1.3 or later is part of all distributions of
LaTeX version 2005/12/01 or later.

This work has the LPPL maintenance status "maintained".

This Current Maintainer of this work is Heiko Oberdiek.

The Base Interpreter refers to any `TeX-Format',
because some files are installed in TDS:tex/generic//.

This work consists of the main source file alphalph.dtx
and the derived files
   alphalph.sty, alphalph.pdf, alphalph.ins, alphalph.drv,
   alphalph-test1.tex, alphalph-test2.tex, alphalph-test3.tex.

\endpreamble
\let\MetaPrefix\DoubleperCent

\generate{%
  \file{alphalph.ins}{\from{alphalph.dtx}{install}}%
  \file{alphalph.drv}{\from{alphalph.dtx}{driver}}%
  \usedir{tex/generic/oberdiek}%
  \file{alphalph.sty}{\from{alphalph.dtx}{package}}%
%  \usedir{doc/latex/oberdiek/test}%
%  \file{alphalph-test1.tex}{\from{alphalph.dtx}{test1}}%
%  \file{alphalph-test2.tex}{\from{alphalph.dtx}{test2,etex}}%
%  \file{alphalph-test3.tex}{\from{alphalph.dtx}{test2,noetex}}%
  \nopreamble
  \nopostamble
%  \usedir{source/latex/oberdiek/catalogue}%
%  \file{alphalph.xml}{\from{alphalph.dtx}{catalogue}}%
}

\catcode32=13\relax% active space
\let =\space%
\Msg{************************************************************************}
\Msg{*}
\Msg{* To finish the installation you have to move the following}
\Msg{* file into a directory searched by TeX:}
\Msg{*}
\Msg{*     alphalph.sty}
\Msg{*}
\Msg{* To produce the documentation run the file `alphalph.drv'}
\Msg{* through LaTeX.}
\Msg{*}
\Msg{* Happy TeXing!}
\Msg{*}
\Msg{************************************************************************}

\endbatchfile
%</install>
%<*ignore>
\fi
%</ignore>
%<*driver>
\NeedsTeXFormat{LaTeX2e}
\ProvidesFile{alphalph.drv}%
  [2016/05/16 v2.5 Convert numbers to letters (HO)]%
\documentclass{ltxdoc}
\usepackage{alphalph}[2016/05/16]
\makeatletter
\newalphalph{\fnsymbolwrap}[wrap]{\@fnsymbol}{}
\newcommand*{\fnsymbolsingle}[1]{%
  \ensuremath{%
    \ifcase#1%
    \or *%
    \or \dagger
    \or \ddagger
    \or \mathsection
    \or \mathparagraph
    \else
      \@ctrerr
    \fi
  }%
}
\makeatother
\newalphalph{\fnsymbolmult}[mult]{\fnsymbolsingle}{}
\usepackage{amsmath}
\DeclareMathOperator{\opDiv}{div}
\DeclareMathOperator{\opMod}{mod}
\newcommand*{\Def}{%
  \ensuremath{%
    \mathrel{\mathop{:}}=%
  }%
}
\usepackage{holtxdoc}[2011/11/22]
\begin{document}
  \DocInput{alphalph.dtx}%
\end{document}
%</driver>
% \fi
%
%
% \CharacterTable
%  {Upper-case    \A\B\C\D\E\F\G\H\I\J\K\L\M\N\O\P\Q\R\S\T\U\V\W\X\Y\Z
%   Lower-case    \a\b\c\d\e\f\g\h\i\j\k\l\m\n\o\p\q\r\s\t\u\v\w\x\y\z
%   Digits        \0\1\2\3\4\5\6\7\8\9
%   Exclamation   \!     Double quote  \"     Hash (number) \#
%   Dollar        \$     Percent       \%     Ampersand     \&
%   Acute accent  \'     Left paren    \(     Right paren   \)
%   Asterisk      \*     Plus          \+     Comma         \,
%   Minus         \-     Point         \.     Solidus       \/
%   Colon         \:     Semicolon     \;     Less than     \<
%   Equals        \=     Greater than  \>     Question mark \?
%   Commercial at \@     Left bracket  \[     Backslash     \\
%   Right bracket \]     Circumflex    \^     Underscore    \_
%   Grave accent  \`     Left brace    \{     Vertical bar  \|
%   Right brace   \}     Tilde         \~}
%
% \GetFileInfo{alphalph.drv}
%
% \title{The \xpackage{alphalph} package}
% \date{2016/05/16 v2.5}
% \author{Heiko Oberdiek\thanks
% {Please report any issues at \url{https://github.com/ho-tex/oberdiek/issues}}\\
% \xemail{heiko.oberdiek at googlemail.com}}
%
% \maketitle
%
% \begin{abstract}
% The package provides methods to represent numbers with a limited
% set of symbols. Both \hologo{LaTeX} and \hologo{plainTeX} are supported.
% \end{abstract}
%
% \tableofcontents
%
% \section{Documentation}
%
% \subsection{Introduction}
% \hologo{LaTeX} counters can be represented in different ways
% by using presentation commands:
% \begin{quote}
%   \cs{arabic}, \cs{roman}, \cs{Roman},\\
%   \cs{alph}, \cs{Alph}, \cs{fnsymbol}
% \end{quote}
% The ranges of supported counter values are more or less
% restricted. Only \cs{arabic} can be used with any counter
% value \hologo{TeX} supports.
% \begin{quote}
% \catcode`\|=12 %
% \begin{tabular}{@{}l|l|l|l@{}}
% Presentation & Supported & Ignored & Error message\\
% command      & domain    & values  & ``Counter too large''\\
% \hline
% \cs{arabic}
%   & \ttfamily -MAX..MAX & &\\
% \cs{roman}, \cs{Roman}
%   & \ttfamily 1..MAX & \ttfamily -MAX..0 &\\
% \cs{alph}, \cs{Alph}
%   & \ttfamily 1..26 & 0 & \ttfamily -MAX..-1, 27..MAX\\
% \cs{fnsymbol}
%   & \ttfamily 1..9 & 0 & \ttfamily -MAX..-1, 10..MAX\\
% \hline
% \end{tabular}\\[1ex]
% \texttt{MAX} = \texttt{2147483647}
% \end{quote}
% Ordinal numbers are often used in documents: numbering
% of chapters, sections, figures, footnotes and so on.
% The layouter chooses \cs{Alph} for chapter numbers
% and \cs{fnsymbol} for footnotes. But what can be done
% if there are more than 26 chapters or more than 10 footnotes?
% This package \xpackage{alphalph} allows to define new
% presentation commands. They rely on a existing command
% and define presentations for values greater the limits.
% Three different methods are provided by the package.
% In the following use cases they are presentated.
%
% \subsection{Use cases}
%
% \subsubsection{Number system based on symbols}
%
% Asume you are writing a book and your lecturer demands
% that chapter numbers must be letters. But you have already
% 30 chapters and you have only 26 letters?
%
% In the decimal system the situation would be clear.
% If you run out of digits, you are using more digits
% to represent a number. This method can be also be used
% for letters. After chapter 26 with |Z| we us |AA|, |AB|, |AC|, and |AD|
% for the remaining chapters.
%
% Happily this package already defines this presentation command:
%\begin{quote}
%\begin{verbatim}
%\usepackage{alphalph}
%\renewcommand*{\thechapter}{%
%  \AlphAlph{\value{chapter}}%
%}
%\end{verbatim}
%\end{quote}
% \cs{AlphAlph} generates:
%   \AlphAlph{1}, \AlphAlph{2}, \AlphAlph{3}, \dots,
%   \AlphAlph{26}, \AlphAlph{27}, \AlphAlph{28}, \dots
%
% The other presentation command is \cs{alphalph} for lowercase letters.
%
% \subsubsection{Wrap symbols around}
%
% Nine footnote symbols are quite a few.
% Too soon the symbols are consumed
% and \hologo{LaTeX} complains with the error ``Counter too large''.
% However, it could be acceptable to start again with the symbols
% from the beginning, especially if there are less than nine symbols
% on a page.
% This could be achieved by a counter reset.
% But finding the right place can be difficult or needs manual
% actions. Also a
% unique counter value can be desirable (e.g. for generating
% unique anchor/link names).
% Package \xpackage{alphalph} allows you to define
% a macro that implements a ``wrap around'', but letting
% the value of the counter untouched:
%\begin{quote}
%\begin{verbatim}
%\usepackage{alphalph}
%\makeatletter
%\newalphalph{\fnsymbolwrap}[wrap]{\@fnsymbol}{}
%\makeatother
%\renewcommand*{\thefootnote}{%
%  \fnsymbolwrap{\value{footnote}}%
%}
%\end{verbatim}
%\end{quote}
% \cs{fnsymbolwrap} generates:
% \fnsymbolwrap{1}~(1), \fnsymbolwrap{2}~(2), \fnsymbolwrap{3}~(3), \dots,
% \fnsymbolwrap{9}~(9), \fnsymbolwrap{10}~(10), \fnsymbolwrap{11}~11, \dots
%
% \subsubsection{Multiple symbols}
%
% \Hologo{LaTeX}'s standard set of footnote symbols contains doubled
% symbols at the higher positions. Could this principle
% be generalized?
% Yes, but first we need a clean footnote symbol list without
% doubled entries, example:
%\begin{quote}
%\begin{verbatim}
%\usepackage{alphalph}
%\makeatletter
%\newcommand*{\fnsymbolsingle}[1]{%
%  \ensuremath{%
%    \ifcase#1%
%    \or *%
%    \or \dagger
%    \or \ddagger
%    \or \mathsection
%    \or \mathparagraph
%    \else
%      \@ctrerr
%    \fi
%  }%
%}
%\makeatother
%\newalphalph{\fnsymbolmult}[mult]{\fnsymbolsingle}{}
%\renewcommand*{\thefootnote}{%
%  \fnsymbolmult{\value{footnote}}%
%}
%\end{verbatim}
%\end{quote}
% The own definition of \cs{fnsymbolsingle} has the
% advantage that this list can easily modified.
% Otherwise you can use \cs{@fnsymbol} directly,
% because it uses the same first five symbols.
%\begin{quote}
%\begin{verbatim}
%\usepackage{alphalph}
%\makeatletter
%\newalphalph{\fnsymbolmult}[mult]{\@fnsymbol}{5}
%\makeatother
%\renewcommand*{\thefootnote}{%
%  \fnsymbolmult{\value{footnote}}%
%}
%\end{verbatim}
%\end{quote}
% \cs{fnsymbolmult} generates:
% \fnsymbolmult{1}~(1), \fnsymbolmult{2}~(2), \fnsymbolmult{3}~(3),
% \fnsymbolmult{4}~(4), \fnsymbolmult{5}~(5), \fnsymbolmult{6}~(6),
% \dots, \fnsymbolmult{16}~16, \fnsymbolmult{17}~17, \dots
%
% \bigskip
% The same method can also be used for the chapter problem
% in the first discussed use case:
%\begin{quote}
%\begin{verbatim}
%\usepackage{alphalph}
%\makeatletter
%\newalphalph{\AlphMult}[mult]{\@Alph}{26}
%\makeatother
%\renewcommand*{\chapter}{%
%  \AlphMult{\value{chapter}}%
%}
%\end{verbatim}
%\end{quote}
% \cs{AlphMult} then generates AA, BB, CC, and DD for chapters 27--30.
%
% \subsection{Glossary}
%
% \begin{description}
% \item[Counter presentation command] is a macro that
%   expects a \hologo{LaTeX} counter name as argument.
%   Numbers cannot be used. Examples:
%   \cs{arabic}, \cs{alph}, \cs{fnsymbol}.
% \item[Number presentation command] is a macro that
%   expects a number as argument. A number is anything
%   that \hologo{TeX} accepts as number including \cs{value}.
%   Examples: \cs{alphalph}, \cs{AlphAlph}, \cs{alphalph@alph}
%
%   However, \cs{alph} or \cs{fnsymbol} are not number
%   presentation commands because they expect a counter
%   name as argument. Happily \hologo{LaTeX} counter presentation
%   commands internally uses number presentation commands
%   with the same name, but prefixed by `@'. Thus
%   \cs{@alph}, \cs{@fnsymbol} are number presentation commands.
% \item[Symbols provider] is a command that can be used
%   to get a list of symbols. For example, \cs{@Alph}
%   provides the 26 uppercase letters from `A' to `Z'.
%   Basically a symbol provider is a number presentation
%   command, usually with a limited range.
% \item[Number of symbols] is the number of the
%   last symbol slot of a symbol provider. Thus \cs{@Alph}
%   generates 26 symbols, \cs{@fnsymbol} provides 9 symbols.
% \end{description}
%
% \subsection{Package usage}
%
% The package \Package{alphalph} can be used with both \hologo{plainTeX}
% and \hologo{LaTeX}:
% \begin{description}
% \item[\hologo{plainTeX}:] |\input alphalph.sty|
% \item[\hologo{LaTeX2e}:]   |\usepackage{alphalph}|\\
%      There aren't any options.
% \end{description}
%
% \subsection{User commands}
%
% \begin{declcs}{AlphAlph} \M{number}\\
%   \cs{alphalph} \M{number}
% \end{declcs}
% Both macros are number presentation commands that
% expects a number as argument. \hologo{LaTeX} counters
% are used with \cs{value}.
%
% The macros represents a number by letters.
% First single letters |A..Z| are used, then
% two letters |AA..ZZ|, three letters |AAA...ZZZ|, \dots
% follow.
%
% Macro \cs{AlphAlph} uses uppercase letters,
% \cs{alphalph} generates the lowercase variant.
% \begin{quote}
%   \catcode`\|=12
%   \def\entry#1{
%     \texttt{#1}
%     & \texttt{\AlphAlph{#1}}
%     & \texttt{\alphalph{#1}}
%   }
%   \begin{tabular}{@{}l|l|l@{}}
%     \meta{number} & \verb|\AlphAlph{|\meta{number}\verb|}|
%                   & \verb|\alphalph{|\meta{number}\verb|}|\\
%     \hline
%     \entry{1}\\
%     \entry{2}\\
%     \entry{26}\\
%     \entry{27}\\
%     \entry{30}\\
%     \entry{2000}\\
%     \entry{3752127}\\
%     \entry{10786572}\\
%     \entry{2147483647}\\
%   \end{tabular}
% \end{quote}
%
% \begin{declcs}{newalphalph}
%   \M{cmd}
%   |[|\meta{method}|]|
%   \M{symbols provider}
%   \M{number of symbols}
% \end{declcs}
% Macro \cs{newalphalph} defines \meta{cmd} as new
% number presentation command. Like \cs{newcommand} an error
% is thrown, if macro \meta{cmd} already exists.
%
% The \meta{method} is one of |alph|, |wrap|, or |mult|.
% The default is |alph|.
%
% As symbol provider a number presentation command
% can be used, e.g. \cs{@fnsymbol}, \cs{@Alph}, or
% \cs{alphalph@alph}.
%
% The last argument is the number of symbols.
% If the argument is empty, then \cs{newalphalph} tries
% to find this number itself. \hologo{LaTeX}'s number presentation
% commands throw an error message, if the number is too large.
% This error message is put in a macro \cs{@ctrerr}.
% Thus \cs{newalphalph} calls the symbol provider and tests
% a number by typesetting it in a temporary box. The error
% macro \cs{@ctrerr} is catched, it proofs that the number
% is not supported. Also if the width of the result is zero
% the number is considered as unavailable.
%
% The empty argument is useful for potentially variable lists.
% However if the end cannot be detected, then the number of
% symbols must be given. This is also a lot faster. Therefore don't
% let the argument empty without reason.
%
% \subsection{Programmer commands}
%
% \begin{declcs}{alphalph@Alph} \M{number}\\
%   \cs{alphalph@alph} \M{number}
% \end{declcs}
% They are basically the same as \cs{@Alph} and \cs{@alph}.
% Some languages of package \xpackage{babel} redefine
% \hologo{LaTeX}'s macros to include some font setup that
% breaks expandibility. Therefore \cs{AlphAlph} and \cs{alphalph}
% are based on \cs{alphalph@Alph} and \cs{alphalph@alph} to get
% the letters. The behaviour of these symbol providers for numbers
% outside the range |1..26| is undefined.
%
% \subsection{Design principles}
%
% \subsubsection{Number presentation commands}
%
% All number presentation commands that this package defines
% (including \cs{alphalph} and \cs{AlphAlph}) have the following
% properties:
% \begin{itemize}
% \item They are fully expandable. This means that they can safely
%       \begin{itemize}
%       \item be written to a file,
%       \item used in moving arguments (\hologo{LaTeX}: they are \emph{robust}),
%       \item used in a \cmd{\csname}-\cmd{\endcsname} pair.
%       \end{itemize}
% \item If the argument is zero or negative, the commands expand
%       to nothing like \cmd{\romannumeral}.
% \item The argument is a \hologo{TeX} number. Anything that would be
%       accepted by \cmd{\number} is a valid argument:
%       \begin{itemize}
%       \item explicite constants,
%       \item macros that expand to a number,
%       \item count registers, \hologo{LaTeX} counter can used
%             via \cmd{\value}, e.\,g.:\\
%             |\alphalph{\value{page}}|
%       \item \dots
%       \end{itemize}
% \item \hologo{eTeX}'s numeric expressions are supported, if \hologo{eTeX} is
%       available. Then \cs{numexpr} is applied to the argument.
%       Package \cs{calc}'s expressions are not supported.
%       That would violate the expandibility.
% \end{itemize}
%
% \subsubsection{General usability}
%
% \begin{description}
% \item[\hologo{TeX} format:]
%   The package does not depend on \hologo{LaTeX}, it can also be used
%   by \hologo{plainTeX}, for example.
% \item[\hologo{eTeX}:] \eTeX is supported, the macros are shorter and
%   faster. But \eTeX's extensions are not requirements.
%   Without \hologo{eTeX}, just the implementation changes. The properties
%   remain unchanged.
% \end{description}
%
% \StopEventually{
% }
%
% \section{Implementation}
% \subsection{Begin\texorpdfstring{ of package}{}}
%    \begin{macrocode}
%<*package>
%    \end{macrocode}
%    Reload check, especially if the package is not used with \LaTeX.
%    \begin{macrocode}
\begingroup\catcode61\catcode48\catcode32=10\relax%
  \catcode13=5 % ^^M
  \endlinechar=13 %
  \catcode35=6 % #
  \catcode39=12 % '
  \catcode44=12 % ,
  \catcode45=12 % -
  \catcode46=12 % .
  \catcode58=12 % :
  \catcode64=11 % @
  \catcode123=1 % {
  \catcode125=2 % }
  \expandafter\let\expandafter\x\csname ver@alphalph.sty\endcsname
  \ifx\x\relax % plain-TeX, first loading
  \else
    \def\empty{}%
    \ifx\x\empty % LaTeX, first loading,
      % variable is initialized, but \ProvidesPackage not yet seen
    \else
      \expandafter\ifx\csname PackageInfo\endcsname\relax
        \def\x#1#2{%
          \immediate\write-1{Package #1 Info: #2.}%
        }%
      \else
        \def\x#1#2{\PackageInfo{#1}{#2, stopped}}%
      \fi
      \x{alphalph}{The package is already loaded}%
      \aftergroup\endinput
    \fi
  \fi
\endgroup%
%    \end{macrocode}
%    Package identification:
%    \begin{macrocode}
\begingroup\catcode61\catcode48\catcode32=10\relax%
  \catcode13=5 % ^^M
  \endlinechar=13 %
  \catcode35=6 % #
  \catcode39=12 % '
  \catcode40=12 % (
  \catcode41=12 % )
  \catcode44=12 % ,
  \catcode45=12 % -
  \catcode46=12 % .
  \catcode47=12 % /
  \catcode58=12 % :
  \catcode64=11 % @
  \catcode91=12 % [
  \catcode93=12 % ]
  \catcode123=1 % {
  \catcode125=2 % }
  \expandafter\ifx\csname ProvidesPackage\endcsname\relax
    \def\x#1#2#3[#4]{\endgroup
      \immediate\write-1{Package: #3 #4}%
      \xdef#1{#4}%
    }%
  \else
    \def\x#1#2[#3]{\endgroup
      #2[{#3}]%
      \ifx#1\@undefined
        \xdef#1{#3}%
      \fi
      \ifx#1\relax
        \xdef#1{#3}%
      \fi
    }%
  \fi
\expandafter\x\csname ver@alphalph.sty\endcsname
\ProvidesPackage{alphalph}%
  [2016/05/16 v2.5 Convert numbers to letters (HO)]%
%    \end{macrocode}
%
% \subsection{Catcodes}
%
%    \begin{macrocode}
\begingroup\catcode61\catcode48\catcode32=10\relax%
  \catcode13=5 % ^^M
  \endlinechar=13 %
  \catcode123=1 % {
  \catcode125=2 % }
  \catcode64=11 % @
  \def\x{\endgroup
    \expandafter\edef\csname AlPh@AtEnd\endcsname{%
      \endlinechar=\the\endlinechar\relax
      \catcode13=\the\catcode13\relax
      \catcode32=\the\catcode32\relax
      \catcode35=\the\catcode35\relax
      \catcode61=\the\catcode61\relax
      \catcode64=\the\catcode64\relax
      \catcode123=\the\catcode123\relax
      \catcode125=\the\catcode125\relax
    }%
  }%
\x\catcode61\catcode48\catcode32=10\relax%
\catcode13=5 % ^^M
\endlinechar=13 %
\catcode35=6 % #
\catcode64=11 % @
\catcode123=1 % {
\catcode125=2 % }
\def\TMP@EnsureCode#1#2{%
  \edef\AlPh@AtEnd{%
    \AlPh@AtEnd
    \catcode#1=\the\catcode#1\relax
  }%
  \catcode#1=#2\relax
}
\TMP@EnsureCode{33}{12}% !
\TMP@EnsureCode{39}{12}% '
\TMP@EnsureCode{40}{12}% (
\TMP@EnsureCode{41}{12}% )
\TMP@EnsureCode{43}{12}% +
\TMP@EnsureCode{44}{12}% ,
\TMP@EnsureCode{46}{12}% .
\TMP@EnsureCode{47}{12}% /
\TMP@EnsureCode{59}{12}% ;
\TMP@EnsureCode{60}{12}% <
\TMP@EnsureCode{62}{12}% >
\TMP@EnsureCode{91}{12}% [
\TMP@EnsureCode{93}{12}% ]
\TMP@EnsureCode{96}{12}% `
\TMP@EnsureCode{124}{12}% |
\edef\AlPh@AtEnd{\AlPh@AtEnd\noexpand\endinput}
%    \end{macrocode}
%
% \subsection{Package loading}
%
%    \begin{macrocode}
\begingroup\expandafter\expandafter\expandafter\endgroup
\expandafter\ifx\csname RequirePackage\endcsname\relax
  \input infwarerr.sty\relax
  \input intcalc.sty\relax
\else
  \RequirePackage{infwarerr}[2007/09/09]%
  \RequirePackage{intcalc}[2007/09/09]%
\fi
%    \end{macrocode}
%
% \subsection{\hologo{eTeX} detection}
%
%    \begin{macrocode}
\begingroup\expandafter\expandafter\expandafter\endgroup
\expandafter\ifx\csname numexpr\endcsname\relax
  \catcode124=9 % '!': ignore
  \catcode43=14 % '+': comment
\else
  \catcode124=14 % '!': comment
  \catcode43=9   % '+': ignore
\fi
%    \end{macrocode}
%
% \subsection{Help macros}
%
%    \begin{macro}{\AlPh@Error}
%    \begin{macrocode}
\def\AlPh@Error#1{%
  \begingroup
    \escapechar=92 % backslash
    \@PackageError{alphalph}{#1}\@ehc
  \endgroup
}
%    \end{macrocode}
%    \end{macro}
%
%    \begin{macro}{\AlPh@IfDefinable}
%    \begin{macrocode}
\begingroup\expandafter\expandafter\expandafter\endgroup
\expandafter\ifx\csname @ifdefinable\endcsname\relax
  \def\AlPh@IfDefinable#1#2{%
    \ifcase\ifx#1\@undefined\else\ifx#1\relax\else1\fi\fi0 %
      #2%
    \else
      \AlPh@Error{%
        Command \string#1 already defined%
      }%
    \fi
  }%
%    \end{macrocode}
%    \end{macro}
%    \begin{macrocode}
\else
%    \end{macrocode}
%    \begin{macro}{\AlPh@IfDefinable}
%    \begin{macrocode}
  \let\AlPh@IfDefinable\@ifdefinable
%    \end{macrocode}
%    \end{macro}
%    \begin{macrocode}
\fi
%    \end{macrocode}
%
%    \begin{macro}{\@ReturnAfterElseFi}
%    \begin{macro}{\@ReturnAfterFi}
%    The following commands moves the `then' and `else' part respectively
%    behind the |\if|-construct. This prevents a too deep |\if|-nesting
%    and so a \hologo{TeX} capacity error because of a limited input stack size.
%    I use this trick in several packages, so I don't prefix these internal
%    commands in order not to have the same macros with different names.
%    (It saves memory.)
%    \begin{macrocode}
\long\def\@ReturnAfterElseFi#1\else#2\fi{\fi#1}
\long\def\@ReturnAfterFi#1\fi{\fi#1}
%    \end{macrocode}
%    \end{macro}
%    \end{macro}
%
%    \begin{macro}{\@gobblefour}
%    \Hologo{LaTeX} defines commands for eating arguments.
%    Define \cs{@gobblefour} if it is not defined (\hologo{plainTeX}).
%    \begin{macrocode}
\expandafter\ifx\csname @gobblefour\endcsname\relax
  \long\def\@gobblefour#1#2#3#4{}%
\fi
%    \end{macrocode}
%    \end{macro}
%
%    \begin{macro}{AlPh@IfOptArg}
%    \begin{macrocode}
\begingroup\expandafter\expandafter\expandafter\endgroup
\expandafter\ifx\csname kernel@ifnextchar\endcsname\relax
  \begingroup\expandafter\expandafter\expandafter\endgroup
  \expandafter\ifx\csname @ifnextchar\endcsname\relax
    \def\AlPh@IfOptArg#1#2{%
      \def\AlPh@TempA{#1}%
      \def\AlPh@TempB{#2}%
      \futurelet\AlPh@Token\AlPh@IfOptArgNext
    }%
    \let\AlPh@BracketLeft=[%]
    \def\AlPh@IfOptArgNext{%
      \ifx\AlPh@Token\AlPh@BracketLeft
        \expandafter\AlPh@TempA
      \else
        \expandafter\AlPh@TempB
      \fi
    }%
  \else
    \def\AlPh@IfOptArg{\@ifnextchar[}%]
  \fi
\else
  \def\AlPh@IfOptArg{\kernel@ifnextchar[}%]
\fi
%    \end{macrocode}
%    \end{macro}
%
% \subsection{Symbol provider}
%
% \subsubsection{Alphabet}
%
%    The output of \cs{alphalph} and \cs{AlphAlph} should be
%    usable as part of command names (see \cs{@namedef}, \cs{csname}, \dots).
%    Unhappily some languages of package \xpackage{babel}
%    redefine \hologo{LaTeX}'s \cs{@alph} and \cs{@Alph} in a manner that
%    they cannot be used in expandable context any more.
%    Therefore package \xpackage{alphalph} provides its own
%    commands.
%
%    \begin{macro}{\alphalph@Alph}
%    \begin{macro}{\alphalph@alph}
%    The two commands \cmd{\AlPh@Alph} and \cmd{\AlPh@alph} convert a
%    number into a letter (uppercase and lowercase respectivly).
%    The character |@| is used as an error symbol,
%    if the number isn't in the range of 1 until 26.
%    Here we need no space after the number |#1|, because the error
%    symbol |@| for the zero case stops scanning the number.
%    This error symbol should not appear anywhere (except for bugs).
%    \begin{macrocode}
\def\alphalph@Alph#1{%
  \ifcase#1%
    @%
  \or A\or B\or C\or D\or E\or F\or G\or H\or I\or J\or K\or L\or M%
  \or N\or O\or P\or Q\or R\or S\or T\or U\or V\or W\or X\or Y\or Z%
  \else
    \AlPh@ctrerr
    @%
  \fi
}
\def\alphalph@alph#1{%
  \ifcase#1%
    @%
  \or a\or b\or c\or d\or e\or f\or g\or h\or i\or j\or k\or l\or m%
  \or n\or o\or p\or q\or r\or s\or t\or u\or v\or w\or x\or y\or z%
  \else
    \AlPh@ctrerr
    @%
  \fi
}
%    \end{macrocode}
%    \end{macro}
%    \end{macro}
%    \begin{macro}{\AlPh@ctrerr}
%    Macro \cs{AlPh@ctrerr} is used as hook for the algorithm
%    to get the available number of symbols.
%    \begin{macrocode}
\def\AlPh@ctrerr{}
%    \end{macrocode}
%    \end{macro}
%
% \subsection{Finding number of symbols}
%
%    \begin{macro}{\AlPh@GetNumberOfSymbols}
%    \noindent
%    |#1|: symbols provider
%    \begin{macrocode}
\def\AlPh@GetNumberOfSymbols#1{%
  \AlPh@TestNumber1!{#1}%
  \ifAlPh@Unavailable
    \def\AlPh@Number{0}%
    \AlPh@Error{No symbols found}%
  \else
    \def\AlPh@Number{1}%
    \AlPh@ExpSearch2!{#1}%
  \fi
}
%    \end{macrocode}
%    \end{macro}
%    \begin{macro}{\ifAlPh@Unavailable}
%    \begin{macrocode}
\let\ifAlPh@Unavailable\iffalse
\def\AlPh@Unavailabletrue{%
  \global\let\ifAlPh@Unavailable\iftrue
}
\def\AlPh@Unavailablefalse{%
  \global\let\ifAlPh@Unavailable\iffalse
}
%    \end{macrocode}
%    \end{macro}
%    \begin{macro}{\AlPh@TestNumber}
%    |#1|: number to be tested\\
%    |#2|: symbols provider
%    \begin{macrocode}
\def\AlPh@TestNumber#1!#2{%
  \AlPh@Unavailablefalse
  \begingroup
    \setbox0=\hbox{%
      \begingroup % color
        \let\@ctrerr\AlPh@Unavailabletrue
        \let\AlPh@ctrerr\AlPh@Unavailabletrue
        #2{#1}%
      \endgroup
    }%
    \ifdim\wd0=0pt %
      \AlPh@Unavailabletrue
    \fi
  \endgroup
}
%    \end{macrocode}
%    \end{macro}
%    \begin{macro}{\AlPh@ExpSearch}
%    |#1|: number to be tested\\
%    |#2|: symbols provider
%    \begin{macrocode}
\def\AlPh@ExpSearch#1!#2{%
  \let\AlPh@Next\relax
  \AlPh@TestNumber#1!{#2}%
  \ifAlPh@Unavailable
    \expandafter\AlPh@BinSearch\AlPh@Number!#1!{#2}%
  \else
    \def\AlPh@Number{#1}%
    \ifnum#1>1073741823 %
      \AlPh@TestNumber2147483647!{#2}%
      \ifAlPh@Unavailable
        \AlPh@BinSearch#1!2147483647!{#2}%
      \else
        \def\AlPh@Number{0}%
        \AlPh@Error{%
          Maximal symbol number not found%
        }%
      \fi
    \else
      \def\AlPh@Next{%
        \expandafter\AlPh@ExpSearch\number\intcalcShl{#1}!{#2}%
      }%
    \fi
  \fi
  \AlPh@Next
}
%    \end{macrocode}
%    \end{macro}
%    \begin{macro}{\AlPh@BinSearch}
%    |#1|: available number\\
%    |#2|: unavailable number, |#2| $>$ |#1|\\
%    |#3|: symbols provider
%    \begin{macrocode}
\def\AlPh@BinSearch#1!#2!#3{%
  \expandafter\AlPh@ProcessBinSearch
  \number\intcalcShr{\intcalcAdd{#1}{#2}}!%
  #1!#2!{#3}%
}
%    \end{macrocode}
%    \end{macro}
%    \begin{macro}{\AlPh@ProcessBinSearch}
%    |#1|: number to be tested, |#2| $\leq$ |#1| $\leq$ |#3|\\
%    |#2|: available number\\
%    |#3|: unavailable number\\
%    |#4|: symbols provider
%    \begin{macrocode}
\def\AlPh@ProcessBinSearch#1!#2!#3!#4{%
  \let\AlPh@Next\relax
  \ifnum#1>#2 %
    \ifnum#1<#3 %
      \AlPh@TestNumber#1!{#4}%
      \ifAlPh@Unavailable
        \def\AlPh@Next{%
          \AlPh@BinSearch#2!#1!{#4}%
        }%
      \else
        \def\AlPh@Next{%
          \AlPh@BinSearch#1!#3!{#4}%
        }%
      \fi
    \else
      \def\AlPh@Number{#2}%
    \fi
  \else
    \def\AlPh@Number{#2}%
  \fi
  \AlPh@Next
}
%    \end{macrocode}
%    \end{macro}
%
% \subsection{Methods}
%
%    The names of method macros start with \cs{AlPh@Method}.
%    These macros do the main job in converting a number to
%    its representation.
%    A method command is called with three arguments.
%    The first argument is the number of symbols. The
%    second argument is the basic macro for converting
%    a number with limited number range. The last parameter
%    is the number that needs converting.
%
% \subsubsection{Common methods}
%
%    \begin{macro}{\AlPh@CheckPositive}
%    \noindent
%    |#1|: number to be checked
%    |#2|: continuation macro\\
%    |#3|: number of symbols (hidden here)\\
%    |#4|: symbol provider (hidden here)
%    \begin{macrocode}
\def\AlPh@CheckPositive#1!#2{%
  \ifnum#1<1 %
    \expandafter\@gobblefour
  \fi
  #2{#1}%
}
%    \end{macrocode}
%    \end{macro}
%
% \subsubsection{Method `alph'}
%
%    \begin{macro}{\AlPh@Method@alph}
%    \noindent
%    |#1|: number of symbols\\
%    |#2|: symbols provider\\
%    |#3|: number to be converted
%    \begin{macrocode}
\def\AlPh@Method@alph#1#2#3{%
  \expandafter\AlPh@CheckPositive
|   \number#3!%
+   \the\numexpr#3!%
    \AlPh@ProcessAlph
    {#1}{#2}%
}
%    \end{macrocode}
%    \end{macro}
%    \begin{macro}{\AlPh@ProcessAlph}
%    |#1|: current number\\
%    |#2|: number of symbols\\
%    |#3|: symbols provider
%    \begin{macrocode}
\def\AlPh@ProcessAlph#1#2#3{%
  \ifnum#1>#2 %
    \@ReturnAfterElseFi{%
      \expandafter\AlPh@StepAlph\number
        \intcalcInc{%
          \intcalcMod{\intcalcDec{#1}}{#2}%
        }%
      \expandafter!\number
        \intcalcDiv{\intcalcDec{#1}}{#2}%
      !{#2}{#3}%
    }%
  \else
    \@ReturnAfterFi{%
      #3{#1}%
    }%
  \fi
}
%    \end{macrocode}
%    \end{macro}
%    \begin{macro}{\AlPh@StepAlph}
%    |#1|: current last digit\\
%    |#2|: new current number\\
%    |#3|: number of symbols\\
%    |#4|: symbols provider
%    \begin{macrocode}
\def\AlPh@StepAlph#1!#2!#3#4{%
  \AlPh@ProcessAlph{#2}{#3}{#4}%
  #4{#1}%
}
%    \end{macrocode}
%    \end{macro}
%
% \subsubsection{Method `wrap'}
%
%    \begin{macro}{\AlPh@Method@wrap}
%    \noindent
%    |#1|: number of symbols\\
%    |#2|: symbols provider\\
%    |#3|: number to be converted
%    \begin{macrocode}
\def\AlPh@Method@wrap#1#2#3{%
  \expandafter\AlPh@CheckPositive
|   \number#3!%
+   \the\numexpr#3!%
    \AlPh@ProcessWrap
    {#1}{#2}%
}
%    \end{macrocode}
%    \end{macro}
%
%    \begin{macro}{\AlPh@ProcessWrap}
%    |#1|: number to be converted\\
%    |#2|: number of symbols\\
%    |#3|: symbols provider
%    \begin{macrocode}
\def\AlPh@ProcessWrap#1#2#3{%
  \ifnum#1>#2 %
    \@ReturnAfterElseFi{%
      \expandafter\AlPh@StepWrap\number
        \intcalcInc{\intcalcMod{\intcalcDec{#1}}{#2}}%
      !{#3}%
    }%
  \else
    \@ReturnAfterFi{%
      #3{#1}%
    }%
  \fi
}
%    \end{macrocode}
%    \end{macro}
%    \begin{macro}{\AlPh@StepWrap}
%    |#1|: final number\\
%    |#2|: symbols provider
%    \begin{macrocode}
\def\AlPh@StepWrap#1!#2{%
  #2{#1}%
}
%    \end{macrocode}
%    \end{macro}
%
% \subsubsection{Method `mult'}
%
%    After the number of symbols is exhausted,
%    repetitions of the symbol are used.
%    \begin{gather*}
%      x \Def \text{number to be converted}\\
%      n \Def \text{number of symbols}\\
%      r \Def \text{repetition length}\\
%      s \Def \text{symbol slot}\\
%      r = ((x - 1) \div n) + 1\\
%      s = ((x - 1) \mod n) + 1
%    \end{gather*}
%    \begin{macro}{\AlPh@Method@mult}
%    \noindent
%    |#1|: number of symbols\\
%    |#2|: symbols provider\\
%    |#3|: number to be converted
%    \begin{macrocode}
\def\AlPh@Method@mult#1#2#3{%
  \expandafter\AlPh@CheckPositive
|   \number#3!%
+   \the\numexpr#3!%
    \AlPh@ProcessMult
    {#1}{#2}%
}
%    \end{macrocode}
%    \end{macro}
%
%    \begin{macro}{\AlPh@ProcessMult}
%    |#1|: number to be converted\\
%    |#2|: number of symbols\\
%    |#3|: symbols provider
%    \begin{macrocode}
\def\AlPh@ProcessMult#1#2#3{%
  \ifnum#1>#2 %
    \@ReturnAfterElseFi{%
      \expandafter\AlPh@StepMult\romannumeral
        \intcalcInc{\intcalcDiv{\intcalcDec{#1}}{#2}}%
        000%
      \expandafter!\number
        \intcalcInc{\intcalcMod{\intcalcDec{#1}}{#2}}%
      !{#3}%
    }%
  \else
    \@ReturnAfterFi{%
      #3{#1}%
    }%
  \fi
}
%    \end{macrocode}
%    \end{macro}
%    \begin{macro}{\AlPh@StepMult}
%    |#1#2|: repetitions coded as list of character `m'\\
%    |#3|: symbol slot\\
%    |#4|: symbols provider
%    \begin{macrocode}
\def\AlPh@StepMult#1#2!#3!#4{%
  \ifx\\#2\\%
  \else
    \@ReturnAfterFi{%
      \AlPh@StepMult#2!#3!{#4}%
    }%
  \fi
  #4{#3}%
}
%    \end{macrocode}
%    \end{macro}
%
% \subsection{User interface}
%
%    \begin{macro}{\newalphalph}
%    Macro \cs{newalphalph} had three arguments in versions below 2.0.
%    For the new method argument we use an optional argument an first
%    position.\\
%    |#1|: cmd\\
%    |[#2]|: method name: |alph| (default), |wrap|, |mult|\\ % hash-ok
%    |#3|: symbols provider\\
%    |#4|: number of symbols
%    \begin{macrocode}
\AlPh@IfDefinable\newalphalph{%
  \def\newalphalph#1{%
    \AlPh@IfOptArg{%
      \AlPh@newalphalph{#1}%
    }{%
      \AlPh@newalphalph{#1}[alph]%
    }%
  }%
}
%    \end{macrocode}
%    \end{macro}
%    \begin{macro}{\AlPh@newalphalph}
%    |#1|: cmd
%    |#2|: method name\\
%    |#3|: symbols provider\\
%    |#4|: number of symbols
%    \begin{macrocode}
\def\AlPh@newalphalph#1[#2]#3#4{%
  \begingroup\expandafter\expandafter\expandafter\endgroup
  \expandafter\ifx\csname AlPh@Method@#2\endcsname\relax
    \AlPh@Error{%
      Unknown method %
|     `#2'%
+     `\detokenize{#2}'%
    }%
  \else
    \ifx\\#4\\%
      \AlPh@GetNumberOfSymbols{#3}%
      \ifcase\AlPh@Number
      \else
        \begingroup
          \escapechar=92 % backslash
          \@PackageInfo{alphalph}{%
            Number of symbols for \string#1 is \AlPh@Number
          }%
        \endgroup
        \expandafter\AlPh@NewAlphAlph
        \csname AlPh@Method@#2\expandafter\endcsname
        \AlPh@Number!{#1}{#3}%
      \fi
    \else
      \expandafter\AlPh@NewAlphAlph
      \csname AlPh@Method@#2\expandafter\endcsname
|     \number#4!%
+     \the\numexpr#4!%
      {#1}{#3}%
    \fi
  \fi
}%
%    \end{macrocode}
%    \end{macro}
%    \begin{macro}{\AlPh@NewAlphAlph}
%    |#1|: method macro\\
%    |#2|: number of symbols\\
%    |#3|: cmd\\
%    |#4|: symbols provider
%    \begin{macrocode}
\def\AlPh@NewAlphAlph#1#2!#3#4{%
  \AlPh@IfDefinable#3{%
    \ifnum#2>0 %
      \def#3{#1{#2}{#4}}%
    \else
      \AlPh@Error{%
          Definition of \string#3 failed,\MessageBreak
          because number of symbols (#2) is not positive%
        }%
    \fi
  }%
}
%    \end{macrocode}
%    \end{macro}
%
%    \begin{macro}{\AlphAlph}
%    \begin{macrocode}
\newalphalph\AlphAlph\alphalph@Alph{26}
%    \end{macrocode}
%    \end{macro}
%    \begin{macro}{\alphalph}
%    \begin{macrocode}
\newalphalph\alphalph\alphalph@alph{26}
%    \end{macrocode}
%    \end{macro}
%
%    \begin{macrocode}
\AlPh@AtEnd%
%</package>
%    \end{macrocode}
%
% \section{Test}
%
% \subsection{Catcode checks for loading}
%
%    \begin{macrocode}
%<*test1>
%    \end{macrocode}
%    \begin{macrocode}
\catcode`\{=1 %
\catcode`\}=2 %
\catcode`\#=6 %
\catcode`\@=11 %
\expandafter\ifx\csname count@\endcsname\relax
  \countdef\count@=255 %
\fi
\expandafter\ifx\csname @gobble\endcsname\relax
  \long\def\@gobble#1{}%
\fi
\expandafter\ifx\csname @firstofone\endcsname\relax
  \long\def\@firstofone#1{#1}%
\fi
\expandafter\ifx\csname loop\endcsname\relax
  \expandafter\@firstofone
\else
  \expandafter\@gobble
\fi
{%
  \def\loop#1\repeat{%
    \def\body{#1}%
    \iterate
  }%
  \def\iterate{%
    \body
      \let\next\iterate
    \else
      \let\next\relax
    \fi
    \next
  }%
  \let\repeat=\fi
}%
\def\RestoreCatcodes{}
\count@=0 %
\loop
  \edef\RestoreCatcodes{%
    \RestoreCatcodes
    \catcode\the\count@=\the\catcode\count@\relax
  }%
\ifnum\count@<255 %
  \advance\count@ 1 %
\repeat

\def\RangeCatcodeInvalid#1#2{%
  \count@=#1\relax
  \loop
    \catcode\count@=15 %
  \ifnum\count@<#2\relax
    \advance\count@ 1 %
  \repeat
}
\def\RangeCatcodeCheck#1#2#3{%
  \count@=#1\relax
  \loop
    \ifnum#3=\catcode\count@
    \else
      \errmessage{%
        Character \the\count@\space
        with wrong catcode \the\catcode\count@\space
        instead of \number#3%
      }%
    \fi
  \ifnum\count@<#2\relax
    \advance\count@ 1 %
  \repeat
}
\def\space{ }
\expandafter\ifx\csname LoadCommand\endcsname\relax
  \def\LoadCommand{\input alphalph.sty\relax}%
\fi
\def\Test{%
  \RangeCatcodeInvalid{0}{47}%
  \RangeCatcodeInvalid{58}{64}%
  \RangeCatcodeInvalid{91}{96}%
  \RangeCatcodeInvalid{123}{255}%
  \catcode`\@=12 %
  \catcode`\\=0 %
  \catcode`\%=14 %
  \LoadCommand
  \RangeCatcodeCheck{0}{36}{15}%
  \RangeCatcodeCheck{37}{37}{14}%
  \RangeCatcodeCheck{38}{47}{15}%
  \RangeCatcodeCheck{48}{57}{12}%
  \RangeCatcodeCheck{58}{63}{15}%
  \RangeCatcodeCheck{64}{64}{12}%
  \RangeCatcodeCheck{65}{90}{11}%
  \RangeCatcodeCheck{91}{91}{15}%
  \RangeCatcodeCheck{92}{92}{0}%
  \RangeCatcodeCheck{93}{96}{15}%
  \RangeCatcodeCheck{97}{122}{11}%
  \RangeCatcodeCheck{123}{255}{15}%
  \RestoreCatcodes
}
\Test
\csname @@end\endcsname
\end
%    \end{macrocode}
%    \begin{macrocode}
%</test1>
%    \end{macrocode}
%
% \section{Macro tests}
%
%    \begin{macrocode}
%<*test2>
\NeedsTeXFormat{LaTeX2e}
\nofiles
\documentclass{article}
%<*noetex>
\makeatletter
\let\saved@numexpr\numexpr
\newcommand*{\DisableNumexpr}{%
  \let\numexpr\@undefined
}
\newcommand*{\RestoreNumexpr}{%
  \let\numexpr\saved@numexpr
}
\DisableNumexpr
%</noetex>
\usepackage{alphalph}[2016/05/16]
%<noetex>\RestoreNumexpr
\usepackage{qstest}
\IncludeTests{*}
\LogTests{log}{*}{*}

\newcommand*{\TestCmd}[3]{%
  \setbox0=\hbox{%
%<noetex>    \DisableNumexpr
    \edef\TestString{#1{#2}}%
    \expandafter\Expect\expandafter{\TestString}{#3}%
    \edef\TestString{#1{#2} }%
    \expandafter\Expect\expandafter{\TestString}{#3 }%
  }%
  \Expect*{\the\wd0}{0.0pt}%
}

\makeatletter
\newalphalph\LaTeXAlphAlph\@Alph{26}
\newalphalph\LaTeXalphalph\@alph{26}
\newalphalph\AlphWrap[wrap]\alphalph@Alph{26}
\newalphalph\alphwrap[wrap]\alphalph@alph{26}
\newalphalph\LaTeXAlphWrap[wrap]\@Alph{26}
\newalphalph\LaTeXalphwrap[wrap]\@alph{26}
\def\LastSymbol#1{%
  \ifx\\#1\\%
  \else
    \@LastSymbol#1\@nil
  \fi
}
\def\@LastSymbol#1#2\@nil{%
  \ifx\\#2\\%
    #1%
  \else
    \@LastSymbol#2\@nil
  \fi
}
\makeatother
\newcommand*{\TestAlph}[2]{%
  \uppercase{\TestCallCmd\AlphAlph{#2}}{#1}%
  \lowercase{\TestCallCmd\alphalph{#2}}{#1}%
  \uppercase{\TestCallCmd\LaTeXAlphAlph{#2}}{#1}%
  \lowercase{\TestCallCmd\LaTeXalphalph{#2}}{#1}%
  \edef\WrapString{\LastSymbol{#2}}%
  \expandafter\TestAlphWrap\expandafter{\WrapString}{#1}%
}
\newcommand*{\TestAlphWrap}[2]{%
  \uppercase{\TestCallCmd\AlphWrap{#1}}{#2}%
  \lowercase{\TestCallCmd\alphwrap{#1}}{#2}%
  \uppercase{\TestCallCmd\LaTeXAlphWrap{#1}}{#2}%
  \lowercase{\TestCallCmd\LaTeXalphwrap{#1}}{#2}%
}
\newcommand*{\TestCallCmd}[3]{%
  \TestCmd#1{#3}{#2}%
}
\begin{qstest}{AlphSymbols}{alphalph, AlphAlph, symbols}
  \TestAlph{1}{a}%
  \TestAlph{2}{b}%
  \TestAlph{3}{c}%
  \TestAlph{4}{d}%
  \TestAlph{5}{e}%
  \TestAlph{6}{f}%
  \TestAlph{7}{g}%
  \TestAlph{8}{h}%
  \TestAlph{9}{i}%
  \TestAlph{10}{j}%
  \TestAlph{11}{k}%
  \TestAlph{12}{l}%
  \TestAlph{13}{m}%
  \TestAlph{14}{n}%
  \TestAlph{15}{o}%
  \TestAlph{16}{p}%
  \TestAlph{17}{q}%
  \TestAlph{18}{r}%
  \TestAlph{19}{s}%
  \TestAlph{20}{t}%
  \TestAlph{21}{u}%
  \TestAlph{22}{v}%
  \TestAlph{23}{w}%
  \TestAlph{24}{x}%
  \TestAlph{25}{y}%
  \TestAlph{26}{z}%
\end{qstest}
\begin{qstest}{AlphRange}{alphalph, range}
  \TestAlph{0}{}%
  \TestAlph{-1}{}%
  \TestAlph{-2147483647}{}%
  \TestAlph{27}{aa}%
  \TestAlph{28}{ab}%
  \TestAlph{52}{az}%
  \TestAlph{53}{ba}%
  \TestAlph{78}{bz}%
  \TestAlph{79}{ca}%
  \TestAlph{702}{zz}%
  \TestAlph{703}{aaa}%
  \TestAlph{2147483647}{fxshrxw}%
\end{qstest}

\makeatletter
\newcommand*{\myvocals}[1]{%
  \ifcase#1X\or A\or E\or I\or O\or U\else Y\fi
}
\makeatother
\newalphalph\vocalsvocals\myvocals{5}
\newcommand*{\TestVocals}{%
  \TestCmd\vocalsvocals
}
\begin{qstest}{vocals}{vocals}
  \TestVocals{0}{}%
  \TestVocals{1}{A}%
  \TestVocals{2}{E}%
  \TestVocals{3}{I}%
  \TestVocals{4}{O}%
  \TestVocals{5}{U}%
  \TestVocals{6}{AA}%
  \TestVocals{7}{AE}%
  \TestVocals{8}{AI}%
  \TestVocals{9}{AO}%
  \TestVocals{10}{AU}%
  \TestVocals{11}{EA}%
  \TestVocals{24}{OO}%
  \TestVocals{25}{OU}%
  \TestVocals{26}{UA}%
  \TestVocals{29}{UO}%
  \TestVocals{30}{UU}%
  \TestVocals{31}{AAA}%
  \TestVocals{155}{UUU}%
  \TestVocals{156}{AAAA}%
  \TestVocals{2147483647}{AIIOOEEIOIIUOE}%
\end{qstest}

\makeatletter
\newalphalph\AlphMult[mult]{\alphalph@Alph}{26}
\newalphalph\alphmult[mult]{\alphalph@alph}{26}
\newalphalph\LaTeXAlphMult[mult]{\@Alph}{26}
\newalphalph\LaTeXalphmult[mult]{\@alph}{26}
\makeatother
\newcommand*{\TestMult}[2]{%
  \uppercase{\TestCallCmd\AlphMult{#2}}{#1}%
  \lowercase{\TestCallCmd\alphmult{#2}}{#1}%
  \uppercase{\TestCallCmd\LaTeXAlphMult{#2}}{#1}%
  \lowercase{\TestCallCmd\LaTeXalphmult{#2}}{#1}%
}
\begin{qstest}{mult}{mult}
  \TestMult{0}{}%
  \TestMult{-1}{}%
  \TestMult{-2147483647}{}%
  \TestMult{1}{a}%
  \TestMult{2}{b}%
  \TestMult{26}{z}%
  \TestMult{27}{aa}%
  \TestMult{28}{bb}%
  \TestMult{52}{zz}%
  \TestMult{53}{aaa}%
  \TestMult{54}{bbb}%
  \TestMult{259}{yyyyyyyyyy}%
  \TestMult{260}{zzzzzzzzzz}%
  \TestMult{261}{aaaaaaaaaaa}%
  \TestMult{262}{bbbbbbbbbbb}%
\end{qstest}

\def\myvocalsB#1{%
  \ifcase#1\or A\or E\or I\or O\or U\fi
}
\begin{qstest}{symbolnum}{symbolnum}
  \makeatletter
  \def\Test#1#2{%
    \let\TestCmd\relax
    \newalphalph\TestCmd{#1}{}%
    \Expect*{\AlPh@Number}{#2}%
  }%
  \Test\@alph{26}%
  \Test\@Alph{26}%
  \Test\@fnsymbol{9}%
  \Test\myvocalsB{5}%
  \Test\alphalph@alph{26}%
  \Test\alphalph@Alph{26}%
\end{qstest}

\begin{qstest}{list}{list}
  \makeatletter
  \def\catch#1\relax{%
    \def\FoundList{\catch#1}%
  }%
  \def\Test[#1]#2#3#4{%
    \let\testcmd\relax
    \newalphalph\testcmd[{#1}]{\catch}{#2}%
    \testcmd{#3}|\relax
    \expandafter\Expect\expandafter{\FoundList}{#4|}%
    %
    \let\SavedCatch\catch
    \def\catch{\noexpand\catch\noexpand\foo}%
    \edef\Result{#4|}%
    \@onelevel@sanitize\Result
    \let\catch\SavedCatch
    \let\testcmd\relax
    \newalphalph\testcmd[{#1}]{\catch\foo}{#2}%
    \testcmd{#3}|\relax
    \@onelevel@sanitize\FoundList
    \Expect*{\FoundList}*{\Result}%
  }%
  \Test[alph]{26}{3}{\catch{3}}%
  \Test[alph]{26}{12}{\catch{12}}%
  \Test[alph]{26}{27}{\catch{1}\catch{1}}%
  \Test[alph]{26}{78}{\catch{2}\catch{26}}%
  \Test[wrap]{26}{7}{\catch{7}}%
  \Test[wrap]{26}{14}{\catch{14}}%
  \Test[wrap]{26}{80}{\catch{2}}%
  \Test[wrap]{26}{700}{\catch{24}}%
  \Test[mult]{26}{4}{\catch{4}}%
  \Test[mult]{26}{17}{\catch{17}}%
  \Test[mult]{26}{54}{\catch{2}\catch{2}\catch{2}}%
\end{qstest}

\begin{document}
\end{document}
%</test2>
%    \end{macrocode}
%
% \section{Installation}
%
% \subsection{Download}
%
% \paragraph{Package.} This package is available on
% CTAN\footnote{\CTANpkg{alphalph}}:
% \begin{description}
% \item[\CTAN{macros/latex/contrib/oberdiek/alphalph.dtx}] The source file.
% \item[\CTAN{macros/latex/contrib/oberdiek/alphalph.pdf}] Documentation.
% \end{description}
%
%
% \paragraph{Bundle.} All the packages of the bundle `oberdiek'
% are also available in a TDS compliant ZIP archive. There
% the packages are already unpacked and the documentation files
% are generated. The files and directories obey the TDS standard.
% \begin{description}
% \item[\CTANinstall{install/macros/latex/contrib/oberdiek.tds.zip}]
% \end{description}
% \emph{TDS} refers to the standard ``A Directory Structure
% for \TeX\ Files'' (\CTAN{tds/tds.pdf}). Directories
% with \xfile{texmf} in their name are usually organized this way.
%
% \subsection{Bundle installation}
%
% \paragraph{Unpacking.} Unpack the \xfile{oberdiek.tds.zip} in the
% TDS tree (also known as \xfile{texmf} tree) of your choice.
% Example (linux):
% \begin{quote}
%   |unzip oberdiek.tds.zip -d ~/texmf|
% \end{quote}
%
% \paragraph{Script installation.}
% Check the directory \xfile{TDS:scripts/oberdiek/} for
% scripts that need further installation steps.
% Package \xpackage{attachfile2} comes with the Perl script
% \xfile{pdfatfi.pl} that should be installed in such a way
% that it can be called as \texttt{pdfatfi}.
% Example (linux):
% \begin{quote}
%   |chmod +x scripts/oberdiek/pdfatfi.pl|\\
%   |cp scripts/oberdiek/pdfatfi.pl /usr/local/bin/|
% \end{quote}
%
% \subsection{Package installation}
%
% \paragraph{Unpacking.} The \xfile{.dtx} file is a self-extracting
% \docstrip\ archive. The files are extracted by running the
% \xfile{.dtx} through \plainTeX:
% \begin{quote}
%   \verb|tex alphalph.dtx|
% \end{quote}
%
% \paragraph{TDS.} Now the different files must be moved into
% the different directories in your installation TDS tree
% (also known as \xfile{texmf} tree):
% \begin{quote}
% \def\t{^^A
% \begin{tabular}{@{}>{\ttfamily}l@{ $\rightarrow$ }>{\ttfamily}l@{}}
%   alphalph.sty & tex/generic/oberdiek/alphalph.sty\\
%   alphalph.pdf & doc/latex/oberdiek/alphalph.pdf\\
%   test/alphalph-test1.tex & doc/latex/oberdiek/test/alphalph-test1.tex\\
%   test/alphalph-test2.tex & doc/latex/oberdiek/test/alphalph-test2.tex\\
%   test/alphalph-test3.tex & doc/latex/oberdiek/test/alphalph-test3.tex\\
%   alphalph.dtx & source/latex/oberdiek/alphalph.dtx\\
% \end{tabular}^^A
% }^^A
% \sbox0{\t}^^A
% \ifdim\wd0>\linewidth
%   \begingroup
%     \advance\linewidth by\leftmargin
%     \advance\linewidth by\rightmargin
%   \edef\x{\endgroup
%     \def\noexpand\lw{\the\linewidth}^^A
%   }\x
%   \def\lwbox{^^A
%     \leavevmode
%     \hbox to \linewidth{^^A
%       \kern-\leftmargin\relax
%       \hss
%       \usebox0
%       \hss
%       \kern-\rightmargin\relax
%     }^^A
%   }^^A
%   \ifdim\wd0>\lw
%     \sbox0{\small\t}^^A
%     \ifdim\wd0>\linewidth
%       \ifdim\wd0>\lw
%         \sbox0{\footnotesize\t}^^A
%         \ifdim\wd0>\linewidth
%           \ifdim\wd0>\lw
%             \sbox0{\scriptsize\t}^^A
%             \ifdim\wd0>\linewidth
%               \ifdim\wd0>\lw
%                 \sbox0{\tiny\t}^^A
%                 \ifdim\wd0>\linewidth
%                   \lwbox
%                 \else
%                   \usebox0
%                 \fi
%               \else
%                 \lwbox
%               \fi
%             \else
%               \usebox0
%             \fi
%           \else
%             \lwbox
%           \fi
%         \else
%           \usebox0
%         \fi
%       \else
%         \lwbox
%       \fi
%     \else
%       \usebox0
%     \fi
%   \else
%     \lwbox
%   \fi
% \else
%   \usebox0
% \fi
% \end{quote}
% If you have a \xfile{docstrip.cfg} that configures and enables \docstrip's
% TDS installing feature, then some files can already be in the right
% place, see the documentation of \docstrip.
%
% \subsection{Refresh file name databases}
%
% If your \TeX~distribution
% (\teTeX, \mikTeX, \dots) relies on file name databases, you must refresh
% these. For example, \teTeX\ users run \verb|texhash| or
% \verb|mktexlsr|.
%
% \subsection{Some details for the interested}
%
% \paragraph{Attached source.}
%
% The PDF documentation on CTAN also includes the
% \xfile{.dtx} source file. It can be extracted by
% AcrobatReader 6 or higher. Another option is \textsf{pdftk},
% e.g. unpack the file into the current directory:
% \begin{quote}
%   \verb|pdftk alphalph.pdf unpack_files output .|
% \end{quote}
%
% \paragraph{Unpacking with \LaTeX.}
% The \xfile{.dtx} chooses its action depending on the format:
% \begin{description}
% \item[\plainTeX:] Run \docstrip\ and extract the files.
% \item[\LaTeX:] Generate the documentation.
% \end{description}
% If you insist on using \LaTeX\ for \docstrip\ (really,
% \docstrip\ does not need \LaTeX), then inform the autodetect routine
% about your intention:
% \begin{quote}
%   \verb|latex \let\install=y\input{alphalph.dtx}|
% \end{quote}
% Do not forget to quote the argument according to the demands
% of your shell.
%
% \paragraph{Generating the documentation.}
% You can use both the \xfile{.dtx} or the \xfile{.drv} to generate
% the documentation. The process can be configured by the
% configuration file \xfile{ltxdoc.cfg}. For instance, put this
% line into this file, if you want to have A4 as paper format:
% \begin{quote}
%   \verb|\PassOptionsToClass{a4paper}{article}|
% \end{quote}
% An example follows how to generate the
% documentation with pdf\LaTeX:
% \begin{quote}
%\begin{verbatim}
%pdflatex alphalph.dtx
%makeindex -s gind.ist alphalph.idx
%pdflatex alphalph.dtx
%makeindex -s gind.ist alphalph.idx
%pdflatex alphalph.dtx
%\end{verbatim}
% \end{quote}
%
% \begin{History}
%   \begin{Version}{1999/03/19 v0.1}
%   \item
%     The first version was built as a response to a
%     \URL{\link{question}}^^A
%     {https://groups.google.com/group/comp.text.tex/msg/17a74cd721641038}^^A
%     of \NameEmail{Will Douglas}{william.douglas@wolfson.ox.ac.uk}
%     and the
%     \URL{\link{request}}^^A
%     {https://groups.google.com/group/comp.text.tex/msg/8f9768825640315f}^^A
%     of \NameEmail{Donald Arsenau}{asnd@reg.triumf.ca},
%     published in the newsgroup
%     \href{news:comp.text.tex}{comp.text.tex}:
%     \URL{``\link{Re: alph counters \textgreater\ 26}''}^^A
%     {https://groups.google.com/group/comp.text.tex/msg/cec563eef8bf65d0}
%   \item
%     Copyright: LPPL (\CTAN{macros/latex/base/lppl.txt})
%   \end{Version}
%   \begin{Version}{1999/04/12 v1.0}
%   \item
%     Documentation added in dtx format.
%   \item
%     \hologo{eTeX} support added.
%   \end{Version}
%   \begin{Version}{1999/04/13 v1.1}
%   \item
%     Minor documentation change.
%   \item
%     First CTAN release.
%   \end{Version}
%   \begin{Version}{1999/06/26 v1.2}
%   \item
%     First generic code about \cmd{\ProvidesPackage} improved.
%   \item
%     Documentation: Installation part revised.
%   \end{Version}
%   \begin{Version}{2006/02/20 v1.3}
%   \item
%     Reload check (for \hologo{plainTeX})
%   \item
%     New DTX framework.
%   \item
%     LPPL 1.3
%   \end{Version}
%   \begin{Version}{2006/05/30 v1.4}
%   \item
%     \cs{newalphalph} added.
%   \end{Version}
%   \begin{Version}{2007/04/11 v1.5}
%   \item
%     Line ends sanitized.
%   \end{Version}
%   \begin{Version}{2007/09/09 v2.0}
%   \item
%     New implementation that uses package \cs{intcalc}.
%     This removes the dependency on \hologo{eTeX}.
%   \item
%     \cs{newalphalph} is extended to support new methods `wrap' and 'multi'.
%   \item
%     Documentation rewritten.
%   \end{Version}
%   \begin{Version}{2008/08/11 v2.1}
%   \item
%     Code is not changed.
%   \item
%     URLs updated from \texttt{www.dejanews.com}
%     to \texttt{groups.google.com}.
%   \end{Version}
%   \begin{Version}{2010/03/01 v2.2}
%   \item
%     Compatibility with \hologo{iniTeX}.
%   \end{Version}
%   \begin{Version}{2010/04/18 v2.3}
%   \item
%     Documentation fixes (Martin M\"unch).
%   \end{Version}
%   \begin{Version}{2011/05/13 v2.4}
%   \item
%     Documentation fixes (Jim Diamond) and using package
%     \xpackage{hologo} for the documentation.
%   \item
%     Catalogue file added.
%   \end{Version}
%   \begin{Version}{2016/05/16 v2.5}
%   \item
%     Documentation updates.
%   \end{Version}
% \end{History}
%
% \PrintIndex
%
% \Finale
\endinput
|
% \end{quote}
% Do not forget to quote the argument according to the demands
% of your shell.
%
% \paragraph{Generating the documentation.}
% You can use both the \xfile{.dtx} or the \xfile{.drv} to generate
% the documentation. The process can be configured by the
% configuration file \xfile{ltxdoc.cfg}. For instance, put this
% line into this file, if you want to have A4 as paper format:
% \begin{quote}
%   \verb|\PassOptionsToClass{a4paper}{article}|
% \end{quote}
% An example follows how to generate the
% documentation with pdf\LaTeX:
% \begin{quote}
%\begin{verbatim}
%pdflatex alphalph.dtx
%makeindex -s gind.ist alphalph.idx
%pdflatex alphalph.dtx
%makeindex -s gind.ist alphalph.idx
%pdflatex alphalph.dtx
%\end{verbatim}
% \end{quote}
%
% \begin{History}
%   \begin{Version}{1999/03/19 v0.1}
%   \item
%     The first version was built as a response to a
%     \URL{\link{question}}^^A
%     {https://groups.google.com/group/comp.text.tex/msg/17a74cd721641038}^^A
%     of \NameEmail{Will Douglas}{william.douglas@wolfson.ox.ac.uk}
%     and the
%     \URL{\link{request}}^^A
%     {https://groups.google.com/group/comp.text.tex/msg/8f9768825640315f}^^A
%     of \NameEmail{Donald Arsenau}{asnd@reg.triumf.ca},
%     published in the newsgroup
%     \href{news:comp.text.tex}{comp.text.tex}:
%     \URL{``\link{Re: alph counters \textgreater\ 26}''}^^A
%     {https://groups.google.com/group/comp.text.tex/msg/cec563eef8bf65d0}
%   \item
%     Copyright: LPPL (\CTAN{macros/latex/base/lppl.txt})
%   \end{Version}
%   \begin{Version}{1999/04/12 v1.0}
%   \item
%     Documentation added in dtx format.
%   \item
%     \hologo{eTeX} support added.
%   \end{Version}
%   \begin{Version}{1999/04/13 v1.1}
%   \item
%     Minor documentation change.
%   \item
%     First CTAN release.
%   \end{Version}
%   \begin{Version}{1999/06/26 v1.2}
%   \item
%     First generic code about \cmd{\ProvidesPackage} improved.
%   \item
%     Documentation: Installation part revised.
%   \end{Version}
%   \begin{Version}{2006/02/20 v1.3}
%   \item
%     Reload check (for \hologo{plainTeX})
%   \item
%     New DTX framework.
%   \item
%     LPPL 1.3
%   \end{Version}
%   \begin{Version}{2006/05/30 v1.4}
%   \item
%     \cs{newalphalph} added.
%   \end{Version}
%   \begin{Version}{2007/04/11 v1.5}
%   \item
%     Line ends sanitized.
%   \end{Version}
%   \begin{Version}{2007/09/09 v2.0}
%   \item
%     New implementation that uses package \cs{intcalc}.
%     This removes the dependency on \hologo{eTeX}.
%   \item
%     \cs{newalphalph} is extended to support new methods `wrap' and 'multi'.
%   \item
%     Documentation rewritten.
%   \end{Version}
%   \begin{Version}{2008/08/11 v2.1}
%   \item
%     Code is not changed.
%   \item
%     URLs updated from \texttt{www.dejanews.com}
%     to \texttt{groups.google.com}.
%   \end{Version}
%   \begin{Version}{2010/03/01 v2.2}
%   \item
%     Compatibility with \hologo{iniTeX}.
%   \end{Version}
%   \begin{Version}{2010/04/18 v2.3}
%   \item
%     Documentation fixes (Martin M\"unch).
%   \end{Version}
%   \begin{Version}{2011/05/13 v2.4}
%   \item
%     Documentation fixes (Jim Diamond) and using package
%     \xpackage{hologo} for the documentation.
%   \item
%     Catalogue file added.
%   \end{Version}
%   \begin{Version}{2016/05/16 v2.5}
%   \item
%     Documentation updates.
%   \end{Version}
% \end{History}
%
% \PrintIndex
%
% \Finale
\endinput

%        (quote the arguments according to the demands of your shell)
%
% Documentation:
%    (a) If alphalph.drv is present:
%           latex alphalph.drv
%    (b) Without alphalph.drv:
%           latex alphalph.dtx; ...
%    The class ltxdoc loads the configuration file ltxdoc.cfg
%    if available. Here you can specify further options, e.g.
%    use A4 as paper format:
%       \PassOptionsToClass{a4paper}{article}
%
%    Programm calls to get the documentation (example):
%       pdflatex alphalph.dtx
%       makeindex -s gind.ist alphalph.idx
%       pdflatex alphalph.dtx
%       makeindex -s gind.ist alphalph.idx
%       pdflatex alphalph.dtx
%
% Installation:
%    TDS:tex/generic/oberdiek/alphalph.sty
%    TDS:doc/latex/oberdiek/alphalph.pdf
%    TDS:doc/latex/oberdiek/test/alphalph-test1.tex
%    TDS:doc/latex/oberdiek/test/alphalph-test2.tex
%    TDS:doc/latex/oberdiek/test/alphalph-test3.tex
%    TDS:source/latex/oberdiek/alphalph.dtx
%
%<*ignore>
\begingroup
  \catcode123=1 %
  \catcode125=2 %
  \def\x{LaTeX2e}%
\expandafter\endgroup
\ifcase 0\ifx\install y1\fi\expandafter
         \ifx\csname processbatchFile\endcsname\relax\else1\fi
         \ifx\fmtname\x\else 1\fi\relax
\else\csname fi\endcsname
%</ignore>
%<*install>
\input docstrip.tex
\Msg{************************************************************************}
\Msg{* Installation}
\Msg{* Package: alphalph 2016/05/16 v2.5 Convert numbers to letters (HO)}
\Msg{************************************************************************}

\keepsilent
\askforoverwritefalse

\let\MetaPrefix\relax
\preamble

This is a generated file.

Project: alphalph
Version: 2016/05/16 v2.5

Copyright (C) 1999, 2006-2008, 2010, 2011 by
   Heiko Oberdiek <heiko.oberdiek at googlemail.com>

This work may be distributed and/or modified under the
conditions of the LaTeX Project Public License, either
version 1.3c of this license or (at your option) any later
version. This version of this license is in
   http://www.latex-project.org/lppl/lppl-1-3c.txt
and the latest version of this license is in
   http://www.latex-project.org/lppl.txt
and version 1.3 or later is part of all distributions of
LaTeX version 2005/12/01 or later.

This work has the LPPL maintenance status "maintained".

This Current Maintainer of this work is Heiko Oberdiek.

The Base Interpreter refers to any `TeX-Format',
because some files are installed in TDS:tex/generic//.

This work consists of the main source file alphalph.dtx
and the derived files
   alphalph.sty, alphalph.pdf, alphalph.ins, alphalph.drv,
   alphalph-test1.tex, alphalph-test2.tex, alphalph-test3.tex.

\endpreamble
\let\MetaPrefix\DoubleperCent

\generate{%
  \file{alphalph.ins}{\from{alphalph.dtx}{install}}%
  \file{alphalph.drv}{\from{alphalph.dtx}{driver}}%
  \usedir{tex/generic/oberdiek}%
  \file{alphalph.sty}{\from{alphalph.dtx}{package}}%
%  \usedir{doc/latex/oberdiek/test}%
%  \file{alphalph-test1.tex}{\from{alphalph.dtx}{test1}}%
%  \file{alphalph-test2.tex}{\from{alphalph.dtx}{test2,etex}}%
%  \file{alphalph-test3.tex}{\from{alphalph.dtx}{test2,noetex}}%
  \nopreamble
  \nopostamble
%  \usedir{source/latex/oberdiek/catalogue}%
%  \file{alphalph.xml}{\from{alphalph.dtx}{catalogue}}%
}

\catcode32=13\relax% active space
\let =\space%
\Msg{************************************************************************}
\Msg{*}
\Msg{* To finish the installation you have to move the following}
\Msg{* file into a directory searched by TeX:}
\Msg{*}
\Msg{*     alphalph.sty}
\Msg{*}
\Msg{* To produce the documentation run the file `alphalph.drv'}
\Msg{* through LaTeX.}
\Msg{*}
\Msg{* Happy TeXing!}
\Msg{*}
\Msg{************************************************************************}

\endbatchfile
%</install>
%<*ignore>
\fi
%</ignore>
%<*driver>
\NeedsTeXFormat{LaTeX2e}
\ProvidesFile{alphalph.drv}%
  [2016/05/16 v2.5 Convert numbers to letters (HO)]%
\documentclass{ltxdoc}
\usepackage{alphalph}[2016/05/16]
\makeatletter
\newalphalph{\fnsymbolwrap}[wrap]{\@fnsymbol}{}
\newcommand*{\fnsymbolsingle}[1]{%
  \ensuremath{%
    \ifcase#1%
    \or *%
    \or \dagger
    \or \ddagger
    \or \mathsection
    \or \mathparagraph
    \else
      \@ctrerr
    \fi
  }%
}
\makeatother
\newalphalph{\fnsymbolmult}[mult]{\fnsymbolsingle}{}
\usepackage{amsmath}
\DeclareMathOperator{\opDiv}{div}
\DeclareMathOperator{\opMod}{mod}
\newcommand*{\Def}{%
  \ensuremath{%
    \mathrel{\mathop{:}}=%
  }%
}
\usepackage{holtxdoc}[2011/11/22]
\begin{document}
  \DocInput{alphalph.dtx}%
\end{document}
%</driver>
% \fi
%
%
% \CharacterTable
%  {Upper-case    \A\B\C\D\E\F\G\H\I\J\K\L\M\N\O\P\Q\R\S\T\U\V\W\X\Y\Z
%   Lower-case    \a\b\c\d\e\f\g\h\i\j\k\l\m\n\o\p\q\r\s\t\u\v\w\x\y\z
%   Digits        \0\1\2\3\4\5\6\7\8\9
%   Exclamation   \!     Double quote  \"     Hash (number) \#
%   Dollar        \$     Percent       \%     Ampersand     \&
%   Acute accent  \'     Left paren    \(     Right paren   \)
%   Asterisk      \*     Plus          \+     Comma         \,
%   Minus         \-     Point         \.     Solidus       \/
%   Colon         \:     Semicolon     \;     Less than     \<
%   Equals        \=     Greater than  \>     Question mark \?
%   Commercial at \@     Left bracket  \[     Backslash     \\
%   Right bracket \]     Circumflex    \^     Underscore    \_
%   Grave accent  \`     Left brace    \{     Vertical bar  \|
%   Right brace   \}     Tilde         \~}
%
% \GetFileInfo{alphalph.drv}
%
% \title{The \xpackage{alphalph} package}
% \date{2016/05/16 v2.5}
% \author{Heiko Oberdiek\thanks
% {Please report any issues at \url{https://github.com/ho-tex/oberdiek/issues}}\\
% \xemail{heiko.oberdiek at googlemail.com}}
%
% \maketitle
%
% \begin{abstract}
% The package provides methods to represent numbers with a limited
% set of symbols. Both \hologo{LaTeX} and \hologo{plainTeX} are supported.
% \end{abstract}
%
% \tableofcontents
%
% \section{Documentation}
%
% \subsection{Introduction}
% \hologo{LaTeX} counters can be represented in different ways
% by using presentation commands:
% \begin{quote}
%   \cs{arabic}, \cs{roman}, \cs{Roman},\\
%   \cs{alph}, \cs{Alph}, \cs{fnsymbol}
% \end{quote}
% The ranges of supported counter values are more or less
% restricted. Only \cs{arabic} can be used with any counter
% value \hologo{TeX} supports.
% \begin{quote}
% \catcode`\|=12 %
% \begin{tabular}{@{}l|l|l|l@{}}
% Presentation & Supported & Ignored & Error message\\
% command      & domain    & values  & ``Counter too large''\\
% \hline
% \cs{arabic}
%   & \ttfamily -MAX..MAX & &\\
% \cs{roman}, \cs{Roman}
%   & \ttfamily 1..MAX & \ttfamily -MAX..0 &\\
% \cs{alph}, \cs{Alph}
%   & \ttfamily 1..26 & 0 & \ttfamily -MAX..-1, 27..MAX\\
% \cs{fnsymbol}
%   & \ttfamily 1..9 & 0 & \ttfamily -MAX..-1, 10..MAX\\
% \hline
% \end{tabular}\\[1ex]
% \texttt{MAX} = \texttt{2147483647}
% \end{quote}
% Ordinal numbers are often used in documents: numbering
% of chapters, sections, figures, footnotes and so on.
% The layouter chooses \cs{Alph} for chapter numbers
% and \cs{fnsymbol} for footnotes. But what can be done
% if there are more than 26 chapters or more than 10 footnotes?
% This package \xpackage{alphalph} allows to define new
% presentation commands. They rely on a existing command
% and define presentations for values greater the limits.
% Three different methods are provided by the package.
% In the following use cases they are presentated.
%
% \subsection{Use cases}
%
% \subsubsection{Number system based on symbols}
%
% Asume you are writing a book and your lecturer demands
% that chapter numbers must be letters. But you have already
% 30 chapters and you have only 26 letters?
%
% In the decimal system the situation would be clear.
% If you run out of digits, you are using more digits
% to represent a number. This method can be also be used
% for letters. After chapter 26 with |Z| we us |AA|, |AB|, |AC|, and |AD|
% for the remaining chapters.
%
% Happily this package already defines this presentation command:
%\begin{quote}
%\begin{verbatim}
%\usepackage{alphalph}
%\renewcommand*{\thechapter}{%
%  \AlphAlph{\value{chapter}}%
%}
%\end{verbatim}
%\end{quote}
% \cs{AlphAlph} generates:
%   \AlphAlph{1}, \AlphAlph{2}, \AlphAlph{3}, \dots,
%   \AlphAlph{26}, \AlphAlph{27}, \AlphAlph{28}, \dots
%
% The other presentation command is \cs{alphalph} for lowercase letters.
%
% \subsubsection{Wrap symbols around}
%
% Nine footnote symbols are quite a few.
% Too soon the symbols are consumed
% and \hologo{LaTeX} complains with the error ``Counter too large''.
% However, it could be acceptable to start again with the symbols
% from the beginning, especially if there are less than nine symbols
% on a page.
% This could be achieved by a counter reset.
% But finding the right place can be difficult or needs manual
% actions. Also a
% unique counter value can be desirable (e.g. for generating
% unique anchor/link names).
% Package \xpackage{alphalph} allows you to define
% a macro that implements a ``wrap around'', but letting
% the value of the counter untouched:
%\begin{quote}
%\begin{verbatim}
%\usepackage{alphalph}
%\makeatletter
%\newalphalph{\fnsymbolwrap}[wrap]{\@fnsymbol}{}
%\makeatother
%\renewcommand*{\thefootnote}{%
%  \fnsymbolwrap{\value{footnote}}%
%}
%\end{verbatim}
%\end{quote}
% \cs{fnsymbolwrap} generates:
% \fnsymbolwrap{1}~(1), \fnsymbolwrap{2}~(2), \fnsymbolwrap{3}~(3), \dots,
% \fnsymbolwrap{9}~(9), \fnsymbolwrap{10}~(10), \fnsymbolwrap{11}~11, \dots
%
% \subsubsection{Multiple symbols}
%
% \Hologo{LaTeX}'s standard set of footnote symbols contains doubled
% symbols at the higher positions. Could this principle
% be generalized?
% Yes, but first we need a clean footnote symbol list without
% doubled entries, example:
%\begin{quote}
%\begin{verbatim}
%\usepackage{alphalph}
%\makeatletter
%\newcommand*{\fnsymbolsingle}[1]{%
%  \ensuremath{%
%    \ifcase#1%
%    \or *%
%    \or \dagger
%    \or \ddagger
%    \or \mathsection
%    \or \mathparagraph
%    \else
%      \@ctrerr
%    \fi
%  }%
%}
%\makeatother
%\newalphalph{\fnsymbolmult}[mult]{\fnsymbolsingle}{}
%\renewcommand*{\thefootnote}{%
%  \fnsymbolmult{\value{footnote}}%
%}
%\end{verbatim}
%\end{quote}
% The own definition of \cs{fnsymbolsingle} has the
% advantage that this list can easily modified.
% Otherwise you can use \cs{@fnsymbol} directly,
% because it uses the same first five symbols.
%\begin{quote}
%\begin{verbatim}
%\usepackage{alphalph}
%\makeatletter
%\newalphalph{\fnsymbolmult}[mult]{\@fnsymbol}{5}
%\makeatother
%\renewcommand*{\thefootnote}{%
%  \fnsymbolmult{\value{footnote}}%
%}
%\end{verbatim}
%\end{quote}
% \cs{fnsymbolmult} generates:
% \fnsymbolmult{1}~(1), \fnsymbolmult{2}~(2), \fnsymbolmult{3}~(3),
% \fnsymbolmult{4}~(4), \fnsymbolmult{5}~(5), \fnsymbolmult{6}~(6),
% \dots, \fnsymbolmult{16}~16, \fnsymbolmult{17}~17, \dots
%
% \bigskip
% The same method can also be used for the chapter problem
% in the first discussed use case:
%\begin{quote}
%\begin{verbatim}
%\usepackage{alphalph}
%\makeatletter
%\newalphalph{\AlphMult}[mult]{\@Alph}{26}
%\makeatother
%\renewcommand*{\chapter}{%
%  \AlphMult{\value{chapter}}%
%}
%\end{verbatim}
%\end{quote}
% \cs{AlphMult} then generates AA, BB, CC, and DD for chapters 27--30.
%
% \subsection{Glossary}
%
% \begin{description}
% \item[Counter presentation command] is a macro that
%   expects a \hologo{LaTeX} counter name as argument.
%   Numbers cannot be used. Examples:
%   \cs{arabic}, \cs{alph}, \cs{fnsymbol}.
% \item[Number presentation command] is a macro that
%   expects a number as argument. A number is anything
%   that \hologo{TeX} accepts as number including \cs{value}.
%   Examples: \cs{alphalph}, \cs{AlphAlph}, \cs{alphalph@alph}
%
%   However, \cs{alph} or \cs{fnsymbol} are not number
%   presentation commands because they expect a counter
%   name as argument. Happily \hologo{LaTeX} counter presentation
%   commands internally uses number presentation commands
%   with the same name, but prefixed by `@'. Thus
%   \cs{@alph}, \cs{@fnsymbol} are number presentation commands.
% \item[Symbols provider] is a command that can be used
%   to get a list of symbols. For example, \cs{@Alph}
%   provides the 26 uppercase letters from `A' to `Z'.
%   Basically a symbol provider is a number presentation
%   command, usually with a limited range.
% \item[Number of symbols] is the number of the
%   last symbol slot of a symbol provider. Thus \cs{@Alph}
%   generates 26 symbols, \cs{@fnsymbol} provides 9 symbols.
% \end{description}
%
% \subsection{Package usage}
%
% The package \Package{alphalph} can be used with both \hologo{plainTeX}
% and \hologo{LaTeX}:
% \begin{description}
% \item[\hologo{plainTeX}:] |\input alphalph.sty|
% \item[\hologo{LaTeX2e}:]   |\usepackage{alphalph}|\\
%      There aren't any options.
% \end{description}
%
% \subsection{User commands}
%
% \begin{declcs}{AlphAlph} \M{number}\\
%   \cs{alphalph} \M{number}
% \end{declcs}
% Both macros are number presentation commands that
% expects a number as argument. \hologo{LaTeX} counters
% are used with \cs{value}.
%
% The macros represents a number by letters.
% First single letters |A..Z| are used, then
% two letters |AA..ZZ|, three letters |AAA...ZZZ|, \dots
% follow.
%
% Macro \cs{AlphAlph} uses uppercase letters,
% \cs{alphalph} generates the lowercase variant.
% \begin{quote}
%   \catcode`\|=12
%   \def\entry#1{
%     \texttt{#1}
%     & \texttt{\AlphAlph{#1}}
%     & \texttt{\alphalph{#1}}
%   }
%   \begin{tabular}{@{}l|l|l@{}}
%     \meta{number} & \verb|\AlphAlph{|\meta{number}\verb|}|
%                   & \verb|\alphalph{|\meta{number}\verb|}|\\
%     \hline
%     \entry{1}\\
%     \entry{2}\\
%     \entry{26}\\
%     \entry{27}\\
%     \entry{30}\\
%     \entry{2000}\\
%     \entry{3752127}\\
%     \entry{10786572}\\
%     \entry{2147483647}\\
%   \end{tabular}
% \end{quote}
%
% \begin{declcs}{newalphalph}
%   \M{cmd}
%   |[|\meta{method}|]|
%   \M{symbols provider}
%   \M{number of symbols}
% \end{declcs}
% Macro \cs{newalphalph} defines \meta{cmd} as new
% number presentation command. Like \cs{newcommand} an error
% is thrown, if macro \meta{cmd} already exists.
%
% The \meta{method} is one of |alph|, |wrap|, or |mult|.
% The default is |alph|.
%
% As symbol provider a number presentation command
% can be used, e.g. \cs{@fnsymbol}, \cs{@Alph}, or
% \cs{alphalph@alph}.
%
% The last argument is the number of symbols.
% If the argument is empty, then \cs{newalphalph} tries
% to find this number itself. \hologo{LaTeX}'s number presentation
% commands throw an error message, if the number is too large.
% This error message is put in a macro \cs{@ctrerr}.
% Thus \cs{newalphalph} calls the symbol provider and tests
% a number by typesetting it in a temporary box. The error
% macro \cs{@ctrerr} is catched, it proofs that the number
% is not supported. Also if the width of the result is zero
% the number is considered as unavailable.
%
% The empty argument is useful for potentially variable lists.
% However if the end cannot be detected, then the number of
% symbols must be given. This is also a lot faster. Therefore don't
% let the argument empty without reason.
%
% \subsection{Programmer commands}
%
% \begin{declcs}{alphalph@Alph} \M{number}\\
%   \cs{alphalph@alph} \M{number}
% \end{declcs}
% They are basically the same as \cs{@Alph} and \cs{@alph}.
% Some languages of package \xpackage{babel} redefine
% \hologo{LaTeX}'s macros to include some font setup that
% breaks expandibility. Therefore \cs{AlphAlph} and \cs{alphalph}
% are based on \cs{alphalph@Alph} and \cs{alphalph@alph} to get
% the letters. The behaviour of these symbol providers for numbers
% outside the range |1..26| is undefined.
%
% \subsection{Design principles}
%
% \subsubsection{Number presentation commands}
%
% All number presentation commands that this package defines
% (including \cs{alphalph} and \cs{AlphAlph}) have the following
% properties:
% \begin{itemize}
% \item They are fully expandable. This means that they can safely
%       \begin{itemize}
%       \item be written to a file,
%       \item used in moving arguments (\hologo{LaTeX}: they are \emph{robust}),
%       \item used in a \cmd{\csname}-\cmd{\endcsname} pair.
%       \end{itemize}
% \item If the argument is zero or negative, the commands expand
%       to nothing like \cmd{\romannumeral}.
% \item The argument is a \hologo{TeX} number. Anything that would be
%       accepted by \cmd{\number} is a valid argument:
%       \begin{itemize}
%       \item explicite constants,
%       \item macros that expand to a number,
%       \item count registers, \hologo{LaTeX} counter can used
%             via \cmd{\value}, e.\,g.:\\
%             |\alphalph{\value{page}}|
%       \item \dots
%       \end{itemize}
% \item \hologo{eTeX}'s numeric expressions are supported, if \hologo{eTeX} is
%       available. Then \cs{numexpr} is applied to the argument.
%       Package \cs{calc}'s expressions are not supported.
%       That would violate the expandibility.
% \end{itemize}
%
% \subsubsection{General usability}
%
% \begin{description}
% \item[\hologo{TeX} format:]
%   The package does not depend on \hologo{LaTeX}, it can also be used
%   by \hologo{plainTeX}, for example.
% \item[\hologo{eTeX}:] \eTeX is supported, the macros are shorter and
%   faster. But \eTeX's extensions are not requirements.
%   Without \hologo{eTeX}, just the implementation changes. The properties
%   remain unchanged.
% \end{description}
%
% \StopEventually{
% }
%
% \section{Implementation}
% \subsection{Begin\texorpdfstring{ of package}{}}
%    \begin{macrocode}
%<*package>
%    \end{macrocode}
%    Reload check, especially if the package is not used with \LaTeX.
%    \begin{macrocode}
\begingroup\catcode61\catcode48\catcode32=10\relax%
  \catcode13=5 % ^^M
  \endlinechar=13 %
  \catcode35=6 % #
  \catcode39=12 % '
  \catcode44=12 % ,
  \catcode45=12 % -
  \catcode46=12 % .
  \catcode58=12 % :
  \catcode64=11 % @
  \catcode123=1 % {
  \catcode125=2 % }
  \expandafter\let\expandafter\x\csname ver@alphalph.sty\endcsname
  \ifx\x\relax % plain-TeX, first loading
  \else
    \def\empty{}%
    \ifx\x\empty % LaTeX, first loading,
      % variable is initialized, but \ProvidesPackage not yet seen
    \else
      \expandafter\ifx\csname PackageInfo\endcsname\relax
        \def\x#1#2{%
          \immediate\write-1{Package #1 Info: #2.}%
        }%
      \else
        \def\x#1#2{\PackageInfo{#1}{#2, stopped}}%
      \fi
      \x{alphalph}{The package is already loaded}%
      \aftergroup\endinput
    \fi
  \fi
\endgroup%
%    \end{macrocode}
%    Package identification:
%    \begin{macrocode}
\begingroup\catcode61\catcode48\catcode32=10\relax%
  \catcode13=5 % ^^M
  \endlinechar=13 %
  \catcode35=6 % #
  \catcode39=12 % '
  \catcode40=12 % (
  \catcode41=12 % )
  \catcode44=12 % ,
  \catcode45=12 % -
  \catcode46=12 % .
  \catcode47=12 % /
  \catcode58=12 % :
  \catcode64=11 % @
  \catcode91=12 % [
  \catcode93=12 % ]
  \catcode123=1 % {
  \catcode125=2 % }
  \expandafter\ifx\csname ProvidesPackage\endcsname\relax
    \def\x#1#2#3[#4]{\endgroup
      \immediate\write-1{Package: #3 #4}%
      \xdef#1{#4}%
    }%
  \else
    \def\x#1#2[#3]{\endgroup
      #2[{#3}]%
      \ifx#1\@undefined
        \xdef#1{#3}%
      \fi
      \ifx#1\relax
        \xdef#1{#3}%
      \fi
    }%
  \fi
\expandafter\x\csname ver@alphalph.sty\endcsname
\ProvidesPackage{alphalph}%
  [2016/05/16 v2.5 Convert numbers to letters (HO)]%
%    \end{macrocode}
%
% \subsection{Catcodes}
%
%    \begin{macrocode}
\begingroup\catcode61\catcode48\catcode32=10\relax%
  \catcode13=5 % ^^M
  \endlinechar=13 %
  \catcode123=1 % {
  \catcode125=2 % }
  \catcode64=11 % @
  \def\x{\endgroup
    \expandafter\edef\csname AlPh@AtEnd\endcsname{%
      \endlinechar=\the\endlinechar\relax
      \catcode13=\the\catcode13\relax
      \catcode32=\the\catcode32\relax
      \catcode35=\the\catcode35\relax
      \catcode61=\the\catcode61\relax
      \catcode64=\the\catcode64\relax
      \catcode123=\the\catcode123\relax
      \catcode125=\the\catcode125\relax
    }%
  }%
\x\catcode61\catcode48\catcode32=10\relax%
\catcode13=5 % ^^M
\endlinechar=13 %
\catcode35=6 % #
\catcode64=11 % @
\catcode123=1 % {
\catcode125=2 % }
\def\TMP@EnsureCode#1#2{%
  \edef\AlPh@AtEnd{%
    \AlPh@AtEnd
    \catcode#1=\the\catcode#1\relax
  }%
  \catcode#1=#2\relax
}
\TMP@EnsureCode{33}{12}% !
\TMP@EnsureCode{39}{12}% '
\TMP@EnsureCode{40}{12}% (
\TMP@EnsureCode{41}{12}% )
\TMP@EnsureCode{43}{12}% +
\TMP@EnsureCode{44}{12}% ,
\TMP@EnsureCode{46}{12}% .
\TMP@EnsureCode{47}{12}% /
\TMP@EnsureCode{59}{12}% ;
\TMP@EnsureCode{60}{12}% <
\TMP@EnsureCode{62}{12}% >
\TMP@EnsureCode{91}{12}% [
\TMP@EnsureCode{93}{12}% ]
\TMP@EnsureCode{96}{12}% `
\TMP@EnsureCode{124}{12}% |
\edef\AlPh@AtEnd{\AlPh@AtEnd\noexpand\endinput}
%    \end{macrocode}
%
% \subsection{Package loading}
%
%    \begin{macrocode}
\begingroup\expandafter\expandafter\expandafter\endgroup
\expandafter\ifx\csname RequirePackage\endcsname\relax
  \input infwarerr.sty\relax
  \input intcalc.sty\relax
\else
  \RequirePackage{infwarerr}[2007/09/09]%
  \RequirePackage{intcalc}[2007/09/09]%
\fi
%    \end{macrocode}
%
% \subsection{\hologo{eTeX} detection}
%
%    \begin{macrocode}
\begingroup\expandafter\expandafter\expandafter\endgroup
\expandafter\ifx\csname numexpr\endcsname\relax
  \catcode124=9 % '!': ignore
  \catcode43=14 % '+': comment
\else
  \catcode124=14 % '!': comment
  \catcode43=9   % '+': ignore
\fi
%    \end{macrocode}
%
% \subsection{Help macros}
%
%    \begin{macro}{\AlPh@Error}
%    \begin{macrocode}
\def\AlPh@Error#1{%
  \begingroup
    \escapechar=92 % backslash
    \@PackageError{alphalph}{#1}\@ehc
  \endgroup
}
%    \end{macrocode}
%    \end{macro}
%
%    \begin{macro}{\AlPh@IfDefinable}
%    \begin{macrocode}
\begingroup\expandafter\expandafter\expandafter\endgroup
\expandafter\ifx\csname @ifdefinable\endcsname\relax
  \def\AlPh@IfDefinable#1#2{%
    \ifcase\ifx#1\@undefined\else\ifx#1\relax\else1\fi\fi0 %
      #2%
    \else
      \AlPh@Error{%
        Command \string#1 already defined%
      }%
    \fi
  }%
%    \end{macrocode}
%    \end{macro}
%    \begin{macrocode}
\else
%    \end{macrocode}
%    \begin{macro}{\AlPh@IfDefinable}
%    \begin{macrocode}
  \let\AlPh@IfDefinable\@ifdefinable
%    \end{macrocode}
%    \end{macro}
%    \begin{macrocode}
\fi
%    \end{macrocode}
%
%    \begin{macro}{\@ReturnAfterElseFi}
%    \begin{macro}{\@ReturnAfterFi}
%    The following commands moves the `then' and `else' part respectively
%    behind the |\if|-construct. This prevents a too deep |\if|-nesting
%    and so a \hologo{TeX} capacity error because of a limited input stack size.
%    I use this trick in several packages, so I don't prefix these internal
%    commands in order not to have the same macros with different names.
%    (It saves memory.)
%    \begin{macrocode}
\long\def\@ReturnAfterElseFi#1\else#2\fi{\fi#1}
\long\def\@ReturnAfterFi#1\fi{\fi#1}
%    \end{macrocode}
%    \end{macro}
%    \end{macro}
%
%    \begin{macro}{\@gobblefour}
%    \Hologo{LaTeX} defines commands for eating arguments.
%    Define \cs{@gobblefour} if it is not defined (\hologo{plainTeX}).
%    \begin{macrocode}
\expandafter\ifx\csname @gobblefour\endcsname\relax
  \long\def\@gobblefour#1#2#3#4{}%
\fi
%    \end{macrocode}
%    \end{macro}
%
%    \begin{macro}{AlPh@IfOptArg}
%    \begin{macrocode}
\begingroup\expandafter\expandafter\expandafter\endgroup
\expandafter\ifx\csname kernel@ifnextchar\endcsname\relax
  \begingroup\expandafter\expandafter\expandafter\endgroup
  \expandafter\ifx\csname @ifnextchar\endcsname\relax
    \def\AlPh@IfOptArg#1#2{%
      \def\AlPh@TempA{#1}%
      \def\AlPh@TempB{#2}%
      \futurelet\AlPh@Token\AlPh@IfOptArgNext
    }%
    \let\AlPh@BracketLeft=[%]
    \def\AlPh@IfOptArgNext{%
      \ifx\AlPh@Token\AlPh@BracketLeft
        \expandafter\AlPh@TempA
      \else
        \expandafter\AlPh@TempB
      \fi
    }%
  \else
    \def\AlPh@IfOptArg{\@ifnextchar[}%]
  \fi
\else
  \def\AlPh@IfOptArg{\kernel@ifnextchar[}%]
\fi
%    \end{macrocode}
%    \end{macro}
%
% \subsection{Symbol provider}
%
% \subsubsection{Alphabet}
%
%    The output of \cs{alphalph} and \cs{AlphAlph} should be
%    usable as part of command names (see \cs{@namedef}, \cs{csname}, \dots).
%    Unhappily some languages of package \xpackage{babel}
%    redefine \hologo{LaTeX}'s \cs{@alph} and \cs{@Alph} in a manner that
%    they cannot be used in expandable context any more.
%    Therefore package \xpackage{alphalph} provides its own
%    commands.
%
%    \begin{macro}{\alphalph@Alph}
%    \begin{macro}{\alphalph@alph}
%    The two commands \cmd{\AlPh@Alph} and \cmd{\AlPh@alph} convert a
%    number into a letter (uppercase and lowercase respectivly).
%    The character |@| is used as an error symbol,
%    if the number isn't in the range of 1 until 26.
%    Here we need no space after the number |#1|, because the error
%    symbol |@| for the zero case stops scanning the number.
%    This error symbol should not appear anywhere (except for bugs).
%    \begin{macrocode}
\def\alphalph@Alph#1{%
  \ifcase#1%
    @%
  \or A\or B\or C\or D\or E\or F\or G\or H\or I\or J\or K\or L\or M%
  \or N\or O\or P\or Q\or R\or S\or T\or U\or V\or W\or X\or Y\or Z%
  \else
    \AlPh@ctrerr
    @%
  \fi
}
\def\alphalph@alph#1{%
  \ifcase#1%
    @%
  \or a\or b\or c\or d\or e\or f\or g\or h\or i\or j\or k\or l\or m%
  \or n\or o\or p\or q\or r\or s\or t\or u\or v\or w\or x\or y\or z%
  \else
    \AlPh@ctrerr
    @%
  \fi
}
%    \end{macrocode}
%    \end{macro}
%    \end{macro}
%    \begin{macro}{\AlPh@ctrerr}
%    Macro \cs{AlPh@ctrerr} is used as hook for the algorithm
%    to get the available number of symbols.
%    \begin{macrocode}
\def\AlPh@ctrerr{}
%    \end{macrocode}
%    \end{macro}
%
% \subsection{Finding number of symbols}
%
%    \begin{macro}{\AlPh@GetNumberOfSymbols}
%    \noindent
%    |#1|: symbols provider
%    \begin{macrocode}
\def\AlPh@GetNumberOfSymbols#1{%
  \AlPh@TestNumber1!{#1}%
  \ifAlPh@Unavailable
    \def\AlPh@Number{0}%
    \AlPh@Error{No symbols found}%
  \else
    \def\AlPh@Number{1}%
    \AlPh@ExpSearch2!{#1}%
  \fi
}
%    \end{macrocode}
%    \end{macro}
%    \begin{macro}{\ifAlPh@Unavailable}
%    \begin{macrocode}
\let\ifAlPh@Unavailable\iffalse
\def\AlPh@Unavailabletrue{%
  \global\let\ifAlPh@Unavailable\iftrue
}
\def\AlPh@Unavailablefalse{%
  \global\let\ifAlPh@Unavailable\iffalse
}
%    \end{macrocode}
%    \end{macro}
%    \begin{macro}{\AlPh@TestNumber}
%    |#1|: number to be tested\\
%    |#2|: symbols provider
%    \begin{macrocode}
\def\AlPh@TestNumber#1!#2{%
  \AlPh@Unavailablefalse
  \begingroup
    \setbox0=\hbox{%
      \begingroup % color
        \let\@ctrerr\AlPh@Unavailabletrue
        \let\AlPh@ctrerr\AlPh@Unavailabletrue
        #2{#1}%
      \endgroup
    }%
    \ifdim\wd0=0pt %
      \AlPh@Unavailabletrue
    \fi
  \endgroup
}
%    \end{macrocode}
%    \end{macro}
%    \begin{macro}{\AlPh@ExpSearch}
%    |#1|: number to be tested\\
%    |#2|: symbols provider
%    \begin{macrocode}
\def\AlPh@ExpSearch#1!#2{%
  \let\AlPh@Next\relax
  \AlPh@TestNumber#1!{#2}%
  \ifAlPh@Unavailable
    \expandafter\AlPh@BinSearch\AlPh@Number!#1!{#2}%
  \else
    \def\AlPh@Number{#1}%
    \ifnum#1>1073741823 %
      \AlPh@TestNumber2147483647!{#2}%
      \ifAlPh@Unavailable
        \AlPh@BinSearch#1!2147483647!{#2}%
      \else
        \def\AlPh@Number{0}%
        \AlPh@Error{%
          Maximal symbol number not found%
        }%
      \fi
    \else
      \def\AlPh@Next{%
        \expandafter\AlPh@ExpSearch\number\intcalcShl{#1}!{#2}%
      }%
    \fi
  \fi
  \AlPh@Next
}
%    \end{macrocode}
%    \end{macro}
%    \begin{macro}{\AlPh@BinSearch}
%    |#1|: available number\\
%    |#2|: unavailable number, |#2| $>$ |#1|\\
%    |#3|: symbols provider
%    \begin{macrocode}
\def\AlPh@BinSearch#1!#2!#3{%
  \expandafter\AlPh@ProcessBinSearch
  \number\intcalcShr{\intcalcAdd{#1}{#2}}!%
  #1!#2!{#3}%
}
%    \end{macrocode}
%    \end{macro}
%    \begin{macro}{\AlPh@ProcessBinSearch}
%    |#1|: number to be tested, |#2| $\leq$ |#1| $\leq$ |#3|\\
%    |#2|: available number\\
%    |#3|: unavailable number\\
%    |#4|: symbols provider
%    \begin{macrocode}
\def\AlPh@ProcessBinSearch#1!#2!#3!#4{%
  \let\AlPh@Next\relax
  \ifnum#1>#2 %
    \ifnum#1<#3 %
      \AlPh@TestNumber#1!{#4}%
      \ifAlPh@Unavailable
        \def\AlPh@Next{%
          \AlPh@BinSearch#2!#1!{#4}%
        }%
      \else
        \def\AlPh@Next{%
          \AlPh@BinSearch#1!#3!{#4}%
        }%
      \fi
    \else
      \def\AlPh@Number{#2}%
    \fi
  \else
    \def\AlPh@Number{#2}%
  \fi
  \AlPh@Next
}
%    \end{macrocode}
%    \end{macro}
%
% \subsection{Methods}
%
%    The names of method macros start with \cs{AlPh@Method}.
%    These macros do the main job in converting a number to
%    its representation.
%    A method command is called with three arguments.
%    The first argument is the number of symbols. The
%    second argument is the basic macro for converting
%    a number with limited number range. The last parameter
%    is the number that needs converting.
%
% \subsubsection{Common methods}
%
%    \begin{macro}{\AlPh@CheckPositive}
%    \noindent
%    |#1|: number to be checked
%    |#2|: continuation macro\\
%    |#3|: number of symbols (hidden here)\\
%    |#4|: symbol provider (hidden here)
%    \begin{macrocode}
\def\AlPh@CheckPositive#1!#2{%
  \ifnum#1<1 %
    \expandafter\@gobblefour
  \fi
  #2{#1}%
}
%    \end{macrocode}
%    \end{macro}
%
% \subsubsection{Method `alph'}
%
%    \begin{macro}{\AlPh@Method@alph}
%    \noindent
%    |#1|: number of symbols\\
%    |#2|: symbols provider\\
%    |#3|: number to be converted
%    \begin{macrocode}
\def\AlPh@Method@alph#1#2#3{%
  \expandafter\AlPh@CheckPositive
|   \number#3!%
+   \the\numexpr#3!%
    \AlPh@ProcessAlph
    {#1}{#2}%
}
%    \end{macrocode}
%    \end{macro}
%    \begin{macro}{\AlPh@ProcessAlph}
%    |#1|: current number\\
%    |#2|: number of symbols\\
%    |#3|: symbols provider
%    \begin{macrocode}
\def\AlPh@ProcessAlph#1#2#3{%
  \ifnum#1>#2 %
    \@ReturnAfterElseFi{%
      \expandafter\AlPh@StepAlph\number
        \intcalcInc{%
          \intcalcMod{\intcalcDec{#1}}{#2}%
        }%
      \expandafter!\number
        \intcalcDiv{\intcalcDec{#1}}{#2}%
      !{#2}{#3}%
    }%
  \else
    \@ReturnAfterFi{%
      #3{#1}%
    }%
  \fi
}
%    \end{macrocode}
%    \end{macro}
%    \begin{macro}{\AlPh@StepAlph}
%    |#1|: current last digit\\
%    |#2|: new current number\\
%    |#3|: number of symbols\\
%    |#4|: symbols provider
%    \begin{macrocode}
\def\AlPh@StepAlph#1!#2!#3#4{%
  \AlPh@ProcessAlph{#2}{#3}{#4}%
  #4{#1}%
}
%    \end{macrocode}
%    \end{macro}
%
% \subsubsection{Method `wrap'}
%
%    \begin{macro}{\AlPh@Method@wrap}
%    \noindent
%    |#1|: number of symbols\\
%    |#2|: symbols provider\\
%    |#3|: number to be converted
%    \begin{macrocode}
\def\AlPh@Method@wrap#1#2#3{%
  \expandafter\AlPh@CheckPositive
|   \number#3!%
+   \the\numexpr#3!%
    \AlPh@ProcessWrap
    {#1}{#2}%
}
%    \end{macrocode}
%    \end{macro}
%
%    \begin{macro}{\AlPh@ProcessWrap}
%    |#1|: number to be converted\\
%    |#2|: number of symbols\\
%    |#3|: symbols provider
%    \begin{macrocode}
\def\AlPh@ProcessWrap#1#2#3{%
  \ifnum#1>#2 %
    \@ReturnAfterElseFi{%
      \expandafter\AlPh@StepWrap\number
        \intcalcInc{\intcalcMod{\intcalcDec{#1}}{#2}}%
      !{#3}%
    }%
  \else
    \@ReturnAfterFi{%
      #3{#1}%
    }%
  \fi
}
%    \end{macrocode}
%    \end{macro}
%    \begin{macro}{\AlPh@StepWrap}
%    |#1|: final number\\
%    |#2|: symbols provider
%    \begin{macrocode}
\def\AlPh@StepWrap#1!#2{%
  #2{#1}%
}
%    \end{macrocode}
%    \end{macro}
%
% \subsubsection{Method `mult'}
%
%    After the number of symbols is exhausted,
%    repetitions of the symbol are used.
%    \begin{gather*}
%      x \Def \text{number to be converted}\\
%      n \Def \text{number of symbols}\\
%      r \Def \text{repetition length}\\
%      s \Def \text{symbol slot}\\
%      r = ((x - 1) \div n) + 1\\
%      s = ((x - 1) \mod n) + 1
%    \end{gather*}
%    \begin{macro}{\AlPh@Method@mult}
%    \noindent
%    |#1|: number of symbols\\
%    |#2|: symbols provider\\
%    |#3|: number to be converted
%    \begin{macrocode}
\def\AlPh@Method@mult#1#2#3{%
  \expandafter\AlPh@CheckPositive
|   \number#3!%
+   \the\numexpr#3!%
    \AlPh@ProcessMult
    {#1}{#2}%
}
%    \end{macrocode}
%    \end{macro}
%
%    \begin{macro}{\AlPh@ProcessMult}
%    |#1|: number to be converted\\
%    |#2|: number of symbols\\
%    |#3|: symbols provider
%    \begin{macrocode}
\def\AlPh@ProcessMult#1#2#3{%
  \ifnum#1>#2 %
    \@ReturnAfterElseFi{%
      \expandafter\AlPh@StepMult\romannumeral
        \intcalcInc{\intcalcDiv{\intcalcDec{#1}}{#2}}%
        000%
      \expandafter!\number
        \intcalcInc{\intcalcMod{\intcalcDec{#1}}{#2}}%
      !{#3}%
    }%
  \else
    \@ReturnAfterFi{%
      #3{#1}%
    }%
  \fi
}
%    \end{macrocode}
%    \end{macro}
%    \begin{macro}{\AlPh@StepMult}
%    |#1#2|: repetitions coded as list of character `m'\\
%    |#3|: symbol slot\\
%    |#4|: symbols provider
%    \begin{macrocode}
\def\AlPh@StepMult#1#2!#3!#4{%
  \ifx\\#2\\%
  \else
    \@ReturnAfterFi{%
      \AlPh@StepMult#2!#3!{#4}%
    }%
  \fi
  #4{#3}%
}
%    \end{macrocode}
%    \end{macro}
%
% \subsection{User interface}
%
%    \begin{macro}{\newalphalph}
%    Macro \cs{newalphalph} had three arguments in versions below 2.0.
%    For the new method argument we use an optional argument an first
%    position.\\
%    |#1|: cmd\\
%    |[#2]|: method name: |alph| (default), |wrap|, |mult|\\ % hash-ok
%    |#3|: symbols provider\\
%    |#4|: number of symbols
%    \begin{macrocode}
\AlPh@IfDefinable\newalphalph{%
  \def\newalphalph#1{%
    \AlPh@IfOptArg{%
      \AlPh@newalphalph{#1}%
    }{%
      \AlPh@newalphalph{#1}[alph]%
    }%
  }%
}
%    \end{macrocode}
%    \end{macro}
%    \begin{macro}{\AlPh@newalphalph}
%    |#1|: cmd
%    |#2|: method name\\
%    |#3|: symbols provider\\
%    |#4|: number of symbols
%    \begin{macrocode}
\def\AlPh@newalphalph#1[#2]#3#4{%
  \begingroup\expandafter\expandafter\expandafter\endgroup
  \expandafter\ifx\csname AlPh@Method@#2\endcsname\relax
    \AlPh@Error{%
      Unknown method %
|     `#2'%
+     `\detokenize{#2}'%
    }%
  \else
    \ifx\\#4\\%
      \AlPh@GetNumberOfSymbols{#3}%
      \ifcase\AlPh@Number
      \else
        \begingroup
          \escapechar=92 % backslash
          \@PackageInfo{alphalph}{%
            Number of symbols for \string#1 is \AlPh@Number
          }%
        \endgroup
        \expandafter\AlPh@NewAlphAlph
        \csname AlPh@Method@#2\expandafter\endcsname
        \AlPh@Number!{#1}{#3}%
      \fi
    \else
      \expandafter\AlPh@NewAlphAlph
      \csname AlPh@Method@#2\expandafter\endcsname
|     \number#4!%
+     \the\numexpr#4!%
      {#1}{#3}%
    \fi
  \fi
}%
%    \end{macrocode}
%    \end{macro}
%    \begin{macro}{\AlPh@NewAlphAlph}
%    |#1|: method macro\\
%    |#2|: number of symbols\\
%    |#3|: cmd\\
%    |#4|: symbols provider
%    \begin{macrocode}
\def\AlPh@NewAlphAlph#1#2!#3#4{%
  \AlPh@IfDefinable#3{%
    \ifnum#2>0 %
      \def#3{#1{#2}{#4}}%
    \else
      \AlPh@Error{%
          Definition of \string#3 failed,\MessageBreak
          because number of symbols (#2) is not positive%
        }%
    \fi
  }%
}
%    \end{macrocode}
%    \end{macro}
%
%    \begin{macro}{\AlphAlph}
%    \begin{macrocode}
\newalphalph\AlphAlph\alphalph@Alph{26}
%    \end{macrocode}
%    \end{macro}
%    \begin{macro}{\alphalph}
%    \begin{macrocode}
\newalphalph\alphalph\alphalph@alph{26}
%    \end{macrocode}
%    \end{macro}
%
%    \begin{macrocode}
\AlPh@AtEnd%
%</package>
%    \end{macrocode}
%
% \section{Test}
%
% \subsection{Catcode checks for loading}
%
%    \begin{macrocode}
%<*test1>
%    \end{macrocode}
%    \begin{macrocode}
\catcode`\{=1 %
\catcode`\}=2 %
\catcode`\#=6 %
\catcode`\@=11 %
\expandafter\ifx\csname count@\endcsname\relax
  \countdef\count@=255 %
\fi
\expandafter\ifx\csname @gobble\endcsname\relax
  \long\def\@gobble#1{}%
\fi
\expandafter\ifx\csname @firstofone\endcsname\relax
  \long\def\@firstofone#1{#1}%
\fi
\expandafter\ifx\csname loop\endcsname\relax
  \expandafter\@firstofone
\else
  \expandafter\@gobble
\fi
{%
  \def\loop#1\repeat{%
    \def\body{#1}%
    \iterate
  }%
  \def\iterate{%
    \body
      \let\next\iterate
    \else
      \let\next\relax
    \fi
    \next
  }%
  \let\repeat=\fi
}%
\def\RestoreCatcodes{}
\count@=0 %
\loop
  \edef\RestoreCatcodes{%
    \RestoreCatcodes
    \catcode\the\count@=\the\catcode\count@\relax
  }%
\ifnum\count@<255 %
  \advance\count@ 1 %
\repeat

\def\RangeCatcodeInvalid#1#2{%
  \count@=#1\relax
  \loop
    \catcode\count@=15 %
  \ifnum\count@<#2\relax
    \advance\count@ 1 %
  \repeat
}
\def\RangeCatcodeCheck#1#2#3{%
  \count@=#1\relax
  \loop
    \ifnum#3=\catcode\count@
    \else
      \errmessage{%
        Character \the\count@\space
        with wrong catcode \the\catcode\count@\space
        instead of \number#3%
      }%
    \fi
  \ifnum\count@<#2\relax
    \advance\count@ 1 %
  \repeat
}
\def\space{ }
\expandafter\ifx\csname LoadCommand\endcsname\relax
  \def\LoadCommand{\input alphalph.sty\relax}%
\fi
\def\Test{%
  \RangeCatcodeInvalid{0}{47}%
  \RangeCatcodeInvalid{58}{64}%
  \RangeCatcodeInvalid{91}{96}%
  \RangeCatcodeInvalid{123}{255}%
  \catcode`\@=12 %
  \catcode`\\=0 %
  \catcode`\%=14 %
  \LoadCommand
  \RangeCatcodeCheck{0}{36}{15}%
  \RangeCatcodeCheck{37}{37}{14}%
  \RangeCatcodeCheck{38}{47}{15}%
  \RangeCatcodeCheck{48}{57}{12}%
  \RangeCatcodeCheck{58}{63}{15}%
  \RangeCatcodeCheck{64}{64}{12}%
  \RangeCatcodeCheck{65}{90}{11}%
  \RangeCatcodeCheck{91}{91}{15}%
  \RangeCatcodeCheck{92}{92}{0}%
  \RangeCatcodeCheck{93}{96}{15}%
  \RangeCatcodeCheck{97}{122}{11}%
  \RangeCatcodeCheck{123}{255}{15}%
  \RestoreCatcodes
}
\Test
\csname @@end\endcsname
\end
%    \end{macrocode}
%    \begin{macrocode}
%</test1>
%    \end{macrocode}
%
% \section{Macro tests}
%
%    \begin{macrocode}
%<*test2>
\NeedsTeXFormat{LaTeX2e}
\nofiles
\documentclass{article}
%<*noetex>
\makeatletter
\let\saved@numexpr\numexpr
\newcommand*{\DisableNumexpr}{%
  \let\numexpr\@undefined
}
\newcommand*{\RestoreNumexpr}{%
  \let\numexpr\saved@numexpr
}
\DisableNumexpr
%</noetex>
\usepackage{alphalph}[2016/05/16]
%<noetex>\RestoreNumexpr
\usepackage{qstest}
\IncludeTests{*}
\LogTests{log}{*}{*}

\newcommand*{\TestCmd}[3]{%
  \setbox0=\hbox{%
%<noetex>    \DisableNumexpr
    \edef\TestString{#1{#2}}%
    \expandafter\Expect\expandafter{\TestString}{#3}%
    \edef\TestString{#1{#2} }%
    \expandafter\Expect\expandafter{\TestString}{#3 }%
  }%
  \Expect*{\the\wd0}{0.0pt}%
}

\makeatletter
\newalphalph\LaTeXAlphAlph\@Alph{26}
\newalphalph\LaTeXalphalph\@alph{26}
\newalphalph\AlphWrap[wrap]\alphalph@Alph{26}
\newalphalph\alphwrap[wrap]\alphalph@alph{26}
\newalphalph\LaTeXAlphWrap[wrap]\@Alph{26}
\newalphalph\LaTeXalphwrap[wrap]\@alph{26}
\def\LastSymbol#1{%
  \ifx\\#1\\%
  \else
    \@LastSymbol#1\@nil
  \fi
}
\def\@LastSymbol#1#2\@nil{%
  \ifx\\#2\\%
    #1%
  \else
    \@LastSymbol#2\@nil
  \fi
}
\makeatother
\newcommand*{\TestAlph}[2]{%
  \uppercase{\TestCallCmd\AlphAlph{#2}}{#1}%
  \lowercase{\TestCallCmd\alphalph{#2}}{#1}%
  \uppercase{\TestCallCmd\LaTeXAlphAlph{#2}}{#1}%
  \lowercase{\TestCallCmd\LaTeXalphalph{#2}}{#1}%
  \edef\WrapString{\LastSymbol{#2}}%
  \expandafter\TestAlphWrap\expandafter{\WrapString}{#1}%
}
\newcommand*{\TestAlphWrap}[2]{%
  \uppercase{\TestCallCmd\AlphWrap{#1}}{#2}%
  \lowercase{\TestCallCmd\alphwrap{#1}}{#2}%
  \uppercase{\TestCallCmd\LaTeXAlphWrap{#1}}{#2}%
  \lowercase{\TestCallCmd\LaTeXalphwrap{#1}}{#2}%
}
\newcommand*{\TestCallCmd}[3]{%
  \TestCmd#1{#3}{#2}%
}
\begin{qstest}{AlphSymbols}{alphalph, AlphAlph, symbols}
  \TestAlph{1}{a}%
  \TestAlph{2}{b}%
  \TestAlph{3}{c}%
  \TestAlph{4}{d}%
  \TestAlph{5}{e}%
  \TestAlph{6}{f}%
  \TestAlph{7}{g}%
  \TestAlph{8}{h}%
  \TestAlph{9}{i}%
  \TestAlph{10}{j}%
  \TestAlph{11}{k}%
  \TestAlph{12}{l}%
  \TestAlph{13}{m}%
  \TestAlph{14}{n}%
  \TestAlph{15}{o}%
  \TestAlph{16}{p}%
  \TestAlph{17}{q}%
  \TestAlph{18}{r}%
  \TestAlph{19}{s}%
  \TestAlph{20}{t}%
  \TestAlph{21}{u}%
  \TestAlph{22}{v}%
  \TestAlph{23}{w}%
  \TestAlph{24}{x}%
  \TestAlph{25}{y}%
  \TestAlph{26}{z}%
\end{qstest}
\begin{qstest}{AlphRange}{alphalph, range}
  \TestAlph{0}{}%
  \TestAlph{-1}{}%
  \TestAlph{-2147483647}{}%
  \TestAlph{27}{aa}%
  \TestAlph{28}{ab}%
  \TestAlph{52}{az}%
  \TestAlph{53}{ba}%
  \TestAlph{78}{bz}%
  \TestAlph{79}{ca}%
  \TestAlph{702}{zz}%
  \TestAlph{703}{aaa}%
  \TestAlph{2147483647}{fxshrxw}%
\end{qstest}

\makeatletter
\newcommand*{\myvocals}[1]{%
  \ifcase#1X\or A\or E\or I\or O\or U\else Y\fi
}
\makeatother
\newalphalph\vocalsvocals\myvocals{5}
\newcommand*{\TestVocals}{%
  \TestCmd\vocalsvocals
}
\begin{qstest}{vocals}{vocals}
  \TestVocals{0}{}%
  \TestVocals{1}{A}%
  \TestVocals{2}{E}%
  \TestVocals{3}{I}%
  \TestVocals{4}{O}%
  \TestVocals{5}{U}%
  \TestVocals{6}{AA}%
  \TestVocals{7}{AE}%
  \TestVocals{8}{AI}%
  \TestVocals{9}{AO}%
  \TestVocals{10}{AU}%
  \TestVocals{11}{EA}%
  \TestVocals{24}{OO}%
  \TestVocals{25}{OU}%
  \TestVocals{26}{UA}%
  \TestVocals{29}{UO}%
  \TestVocals{30}{UU}%
  \TestVocals{31}{AAA}%
  \TestVocals{155}{UUU}%
  \TestVocals{156}{AAAA}%
  \TestVocals{2147483647}{AIIOOEEIOIIUOE}%
\end{qstest}

\makeatletter
\newalphalph\AlphMult[mult]{\alphalph@Alph}{26}
\newalphalph\alphmult[mult]{\alphalph@alph}{26}
\newalphalph\LaTeXAlphMult[mult]{\@Alph}{26}
\newalphalph\LaTeXalphmult[mult]{\@alph}{26}
\makeatother
\newcommand*{\TestMult}[2]{%
  \uppercase{\TestCallCmd\AlphMult{#2}}{#1}%
  \lowercase{\TestCallCmd\alphmult{#2}}{#1}%
  \uppercase{\TestCallCmd\LaTeXAlphMult{#2}}{#1}%
  \lowercase{\TestCallCmd\LaTeXalphmult{#2}}{#1}%
}
\begin{qstest}{mult}{mult}
  \TestMult{0}{}%
  \TestMult{-1}{}%
  \TestMult{-2147483647}{}%
  \TestMult{1}{a}%
  \TestMult{2}{b}%
  \TestMult{26}{z}%
  \TestMult{27}{aa}%
  \TestMult{28}{bb}%
  \TestMult{52}{zz}%
  \TestMult{53}{aaa}%
  \TestMult{54}{bbb}%
  \TestMult{259}{yyyyyyyyyy}%
  \TestMult{260}{zzzzzzzzzz}%
  \TestMult{261}{aaaaaaaaaaa}%
  \TestMult{262}{bbbbbbbbbbb}%
\end{qstest}

\def\myvocalsB#1{%
  \ifcase#1\or A\or E\or I\or O\or U\fi
}
\begin{qstest}{symbolnum}{symbolnum}
  \makeatletter
  \def\Test#1#2{%
    \let\TestCmd\relax
    \newalphalph\TestCmd{#1}{}%
    \Expect*{\AlPh@Number}{#2}%
  }%
  \Test\@alph{26}%
  \Test\@Alph{26}%
  \Test\@fnsymbol{9}%
  \Test\myvocalsB{5}%
  \Test\alphalph@alph{26}%
  \Test\alphalph@Alph{26}%
\end{qstest}

\begin{qstest}{list}{list}
  \makeatletter
  \def\catch#1\relax{%
    \def\FoundList{\catch#1}%
  }%
  \def\Test[#1]#2#3#4{%
    \let\testcmd\relax
    \newalphalph\testcmd[{#1}]{\catch}{#2}%
    \testcmd{#3}|\relax
    \expandafter\Expect\expandafter{\FoundList}{#4|}%
    %
    \let\SavedCatch\catch
    \def\catch{\noexpand\catch\noexpand\foo}%
    \edef\Result{#4|}%
    \@onelevel@sanitize\Result
    \let\catch\SavedCatch
    \let\testcmd\relax
    \newalphalph\testcmd[{#1}]{\catch\foo}{#2}%
    \testcmd{#3}|\relax
    \@onelevel@sanitize\FoundList
    \Expect*{\FoundList}*{\Result}%
  }%
  \Test[alph]{26}{3}{\catch{3}}%
  \Test[alph]{26}{12}{\catch{12}}%
  \Test[alph]{26}{27}{\catch{1}\catch{1}}%
  \Test[alph]{26}{78}{\catch{2}\catch{26}}%
  \Test[wrap]{26}{7}{\catch{7}}%
  \Test[wrap]{26}{14}{\catch{14}}%
  \Test[wrap]{26}{80}{\catch{2}}%
  \Test[wrap]{26}{700}{\catch{24}}%
  \Test[mult]{26}{4}{\catch{4}}%
  \Test[mult]{26}{17}{\catch{17}}%
  \Test[mult]{26}{54}{\catch{2}\catch{2}\catch{2}}%
\end{qstest}

\begin{document}
\end{document}
%</test2>
%    \end{macrocode}
%
% \section{Installation}
%
% \subsection{Download}
%
% \paragraph{Package.} This package is available on
% CTAN\footnote{\CTANpkg{alphalph}}:
% \begin{description}
% \item[\CTAN{macros/latex/contrib/oberdiek/alphalph.dtx}] The source file.
% \item[\CTAN{macros/latex/contrib/oberdiek/alphalph.pdf}] Documentation.
% \end{description}
%
%
% \paragraph{Bundle.} All the packages of the bundle `oberdiek'
% are also available in a TDS compliant ZIP archive. There
% the packages are already unpacked and the documentation files
% are generated. The files and directories obey the TDS standard.
% \begin{description}
% \item[\CTANinstall{install/macros/latex/contrib/oberdiek.tds.zip}]
% \end{description}
% \emph{TDS} refers to the standard ``A Directory Structure
% for \TeX\ Files'' (\CTAN{tds/tds.pdf}). Directories
% with \xfile{texmf} in their name are usually organized this way.
%
% \subsection{Bundle installation}
%
% \paragraph{Unpacking.} Unpack the \xfile{oberdiek.tds.zip} in the
% TDS tree (also known as \xfile{texmf} tree) of your choice.
% Example (linux):
% \begin{quote}
%   |unzip oberdiek.tds.zip -d ~/texmf|
% \end{quote}
%
% \paragraph{Script installation.}
% Check the directory \xfile{TDS:scripts/oberdiek/} for
% scripts that need further installation steps.
% Package \xpackage{attachfile2} comes with the Perl script
% \xfile{pdfatfi.pl} that should be installed in such a way
% that it can be called as \texttt{pdfatfi}.
% Example (linux):
% \begin{quote}
%   |chmod +x scripts/oberdiek/pdfatfi.pl|\\
%   |cp scripts/oberdiek/pdfatfi.pl /usr/local/bin/|
% \end{quote}
%
% \subsection{Package installation}
%
% \paragraph{Unpacking.} The \xfile{.dtx} file is a self-extracting
% \docstrip\ archive. The files are extracted by running the
% \xfile{.dtx} through \plainTeX:
% \begin{quote}
%   \verb|tex alphalph.dtx|
% \end{quote}
%
% \paragraph{TDS.} Now the different files must be moved into
% the different directories in your installation TDS tree
% (also known as \xfile{texmf} tree):
% \begin{quote}
% \def\t{^^A
% \begin{tabular}{@{}>{\ttfamily}l@{ $\rightarrow$ }>{\ttfamily}l@{}}
%   alphalph.sty & tex/generic/oberdiek/alphalph.sty\\
%   alphalph.pdf & doc/latex/oberdiek/alphalph.pdf\\
%   test/alphalph-test1.tex & doc/latex/oberdiek/test/alphalph-test1.tex\\
%   test/alphalph-test2.tex & doc/latex/oberdiek/test/alphalph-test2.tex\\
%   test/alphalph-test3.tex & doc/latex/oberdiek/test/alphalph-test3.tex\\
%   alphalph.dtx & source/latex/oberdiek/alphalph.dtx\\
% \end{tabular}^^A
% }^^A
% \sbox0{\t}^^A
% \ifdim\wd0>\linewidth
%   \begingroup
%     \advance\linewidth by\leftmargin
%     \advance\linewidth by\rightmargin
%   \edef\x{\endgroup
%     \def\noexpand\lw{\the\linewidth}^^A
%   }\x
%   \def\lwbox{^^A
%     \leavevmode
%     \hbox to \linewidth{^^A
%       \kern-\leftmargin\relax
%       \hss
%       \usebox0
%       \hss
%       \kern-\rightmargin\relax
%     }^^A
%   }^^A
%   \ifdim\wd0>\lw
%     \sbox0{\small\t}^^A
%     \ifdim\wd0>\linewidth
%       \ifdim\wd0>\lw
%         \sbox0{\footnotesize\t}^^A
%         \ifdim\wd0>\linewidth
%           \ifdim\wd0>\lw
%             \sbox0{\scriptsize\t}^^A
%             \ifdim\wd0>\linewidth
%               \ifdim\wd0>\lw
%                 \sbox0{\tiny\t}^^A
%                 \ifdim\wd0>\linewidth
%                   \lwbox
%                 \else
%                   \usebox0
%                 \fi
%               \else
%                 \lwbox
%               \fi
%             \else
%               \usebox0
%             \fi
%           \else
%             \lwbox
%           \fi
%         \else
%           \usebox0
%         \fi
%       \else
%         \lwbox
%       \fi
%     \else
%       \usebox0
%     \fi
%   \else
%     \lwbox
%   \fi
% \else
%   \usebox0
% \fi
% \end{quote}
% If you have a \xfile{docstrip.cfg} that configures and enables \docstrip's
% TDS installing feature, then some files can already be in the right
% place, see the documentation of \docstrip.
%
% \subsection{Refresh file name databases}
%
% If your \TeX~distribution
% (\teTeX, \mikTeX, \dots) relies on file name databases, you must refresh
% these. For example, \teTeX\ users run \verb|texhash| or
% \verb|mktexlsr|.
%
% \subsection{Some details for the interested}
%
% \paragraph{Attached source.}
%
% The PDF documentation on CTAN also includes the
% \xfile{.dtx} source file. It can be extracted by
% AcrobatReader 6 or higher. Another option is \textsf{pdftk},
% e.g. unpack the file into the current directory:
% \begin{quote}
%   \verb|pdftk alphalph.pdf unpack_files output .|
% \end{quote}
%
% \paragraph{Unpacking with \LaTeX.}
% The \xfile{.dtx} chooses its action depending on the format:
% \begin{description}
% \item[\plainTeX:] Run \docstrip\ and extract the files.
% \item[\LaTeX:] Generate the documentation.
% \end{description}
% If you insist on using \LaTeX\ for \docstrip\ (really,
% \docstrip\ does not need \LaTeX), then inform the autodetect routine
% about your intention:
% \begin{quote}
%   \verb|latex \let\install=y% \iffalse meta-comment
%
% File: alphalph.dtx
% Version: 2016/05/16 v2.5
% Info: Convert numbers to letters
%
% Copyright (C) 1999, 2006-2008, 2010, 2011 by
%    Heiko Oberdiek <heiko.oberdiek at googlemail.com>
%    2016
%    https://github.com/ho-tex/oberdiek/issues
%
% This work may be distributed and/or modified under the
% conditions of the LaTeX Project Public License, either
% version 1.3c of this license or (at your option) any later
% version. This version of this license is in
%    http://www.latex-project.org/lppl/lppl-1-3c.txt
% and the latest version of this license is in
%    http://www.latex-project.org/lppl.txt
% and version 1.3 or later is part of all distributions of
% LaTeX version 2005/12/01 or later.
%
% This work has the LPPL maintenance status "maintained".
%
% This Current Maintainer of this work is Heiko Oberdiek.
%
% The Base Interpreter refers to any `TeX-Format',
% because some files are installed in TDS:tex/generic//.
%
% This work consists of the main source file alphalph.dtx
% and the derived files
%    alphalph.sty, alphalph.pdf, alphalph.ins, alphalph.drv,
%    alphalph-test1.tex, alphalph-test2.tex, alphalph-test3.tex.
%
% Distribution:
%    CTAN:macros/latex/contrib/oberdiek/alphalph.dtx
%    CTAN:macros/latex/contrib/oberdiek/alphalph.pdf
%
% Unpacking:
%    (a) If alphalph.ins is present:
%           tex alphalph.ins
%    (b) Without alphalph.ins:
%           tex alphalph.dtx
%    (c) If you insist on using LaTeX
%           latex \let\install=y% \iffalse meta-comment
%
% File: alphalph.dtx
% Version: 2016/05/16 v2.5
% Info: Convert numbers to letters
%
% Copyright (C) 1999, 2006-2008, 2010, 2011 by
%    Heiko Oberdiek <heiko.oberdiek at googlemail.com>
%    2016
%    https://github.com/ho-tex/oberdiek/issues
%
% This work may be distributed and/or modified under the
% conditions of the LaTeX Project Public License, either
% version 1.3c of this license or (at your option) any later
% version. This version of this license is in
%    http://www.latex-project.org/lppl/lppl-1-3c.txt
% and the latest version of this license is in
%    http://www.latex-project.org/lppl.txt
% and version 1.3 or later is part of all distributions of
% LaTeX version 2005/12/01 or later.
%
% This work has the LPPL maintenance status "maintained".
%
% This Current Maintainer of this work is Heiko Oberdiek.
%
% The Base Interpreter refers to any `TeX-Format',
% because some files are installed in TDS:tex/generic//.
%
% This work consists of the main source file alphalph.dtx
% and the derived files
%    alphalph.sty, alphalph.pdf, alphalph.ins, alphalph.drv,
%    alphalph-test1.tex, alphalph-test2.tex, alphalph-test3.tex.
%
% Distribution:
%    CTAN:macros/latex/contrib/oberdiek/alphalph.dtx
%    CTAN:macros/latex/contrib/oberdiek/alphalph.pdf
%
% Unpacking:
%    (a) If alphalph.ins is present:
%           tex alphalph.ins
%    (b) Without alphalph.ins:
%           tex alphalph.dtx
%    (c) If you insist on using LaTeX
%           latex \let\install=y\input{alphalph.dtx}
%        (quote the arguments according to the demands of your shell)
%
% Documentation:
%    (a) If alphalph.drv is present:
%           latex alphalph.drv
%    (b) Without alphalph.drv:
%           latex alphalph.dtx; ...
%    The class ltxdoc loads the configuration file ltxdoc.cfg
%    if available. Here you can specify further options, e.g.
%    use A4 as paper format:
%       \PassOptionsToClass{a4paper}{article}
%
%    Programm calls to get the documentation (example):
%       pdflatex alphalph.dtx
%       makeindex -s gind.ist alphalph.idx
%       pdflatex alphalph.dtx
%       makeindex -s gind.ist alphalph.idx
%       pdflatex alphalph.dtx
%
% Installation:
%    TDS:tex/generic/oberdiek/alphalph.sty
%    TDS:doc/latex/oberdiek/alphalph.pdf
%    TDS:doc/latex/oberdiek/test/alphalph-test1.tex
%    TDS:doc/latex/oberdiek/test/alphalph-test2.tex
%    TDS:doc/latex/oberdiek/test/alphalph-test3.tex
%    TDS:source/latex/oberdiek/alphalph.dtx
%
%<*ignore>
\begingroup
  \catcode123=1 %
  \catcode125=2 %
  \def\x{LaTeX2e}%
\expandafter\endgroup
\ifcase 0\ifx\install y1\fi\expandafter
         \ifx\csname processbatchFile\endcsname\relax\else1\fi
         \ifx\fmtname\x\else 1\fi\relax
\else\csname fi\endcsname
%</ignore>
%<*install>
\input docstrip.tex
\Msg{************************************************************************}
\Msg{* Installation}
\Msg{* Package: alphalph 2016/05/16 v2.5 Convert numbers to letters (HO)}
\Msg{************************************************************************}

\keepsilent
\askforoverwritefalse

\let\MetaPrefix\relax
\preamble

This is a generated file.

Project: alphalph
Version: 2016/05/16 v2.5

Copyright (C) 1999, 2006-2008, 2010, 2011 by
   Heiko Oberdiek <heiko.oberdiek at googlemail.com>

This work may be distributed and/or modified under the
conditions of the LaTeX Project Public License, either
version 1.3c of this license or (at your option) any later
version. This version of this license is in
   http://www.latex-project.org/lppl/lppl-1-3c.txt
and the latest version of this license is in
   http://www.latex-project.org/lppl.txt
and version 1.3 or later is part of all distributions of
LaTeX version 2005/12/01 or later.

This work has the LPPL maintenance status "maintained".

This Current Maintainer of this work is Heiko Oberdiek.

The Base Interpreter refers to any `TeX-Format',
because some files are installed in TDS:tex/generic//.

This work consists of the main source file alphalph.dtx
and the derived files
   alphalph.sty, alphalph.pdf, alphalph.ins, alphalph.drv,
   alphalph-test1.tex, alphalph-test2.tex, alphalph-test3.tex.

\endpreamble
\let\MetaPrefix\DoubleperCent

\generate{%
  \file{alphalph.ins}{\from{alphalph.dtx}{install}}%
  \file{alphalph.drv}{\from{alphalph.dtx}{driver}}%
  \usedir{tex/generic/oberdiek}%
  \file{alphalph.sty}{\from{alphalph.dtx}{package}}%
%  \usedir{doc/latex/oberdiek/test}%
%  \file{alphalph-test1.tex}{\from{alphalph.dtx}{test1}}%
%  \file{alphalph-test2.tex}{\from{alphalph.dtx}{test2,etex}}%
%  \file{alphalph-test3.tex}{\from{alphalph.dtx}{test2,noetex}}%
  \nopreamble
  \nopostamble
%  \usedir{source/latex/oberdiek/catalogue}%
%  \file{alphalph.xml}{\from{alphalph.dtx}{catalogue}}%
}

\catcode32=13\relax% active space
\let =\space%
\Msg{************************************************************************}
\Msg{*}
\Msg{* To finish the installation you have to move the following}
\Msg{* file into a directory searched by TeX:}
\Msg{*}
\Msg{*     alphalph.sty}
\Msg{*}
\Msg{* To produce the documentation run the file `alphalph.drv'}
\Msg{* through LaTeX.}
\Msg{*}
\Msg{* Happy TeXing!}
\Msg{*}
\Msg{************************************************************************}

\endbatchfile
%</install>
%<*ignore>
\fi
%</ignore>
%<*driver>
\NeedsTeXFormat{LaTeX2e}
\ProvidesFile{alphalph.drv}%
  [2016/05/16 v2.5 Convert numbers to letters (HO)]%
\documentclass{ltxdoc}
\usepackage{alphalph}[2016/05/16]
\makeatletter
\newalphalph{\fnsymbolwrap}[wrap]{\@fnsymbol}{}
\newcommand*{\fnsymbolsingle}[1]{%
  \ensuremath{%
    \ifcase#1%
    \or *%
    \or \dagger
    \or \ddagger
    \or \mathsection
    \or \mathparagraph
    \else
      \@ctrerr
    \fi
  }%
}
\makeatother
\newalphalph{\fnsymbolmult}[mult]{\fnsymbolsingle}{}
\usepackage{amsmath}
\DeclareMathOperator{\opDiv}{div}
\DeclareMathOperator{\opMod}{mod}
\newcommand*{\Def}{%
  \ensuremath{%
    \mathrel{\mathop{:}}=%
  }%
}
\usepackage{holtxdoc}[2011/11/22]
\begin{document}
  \DocInput{alphalph.dtx}%
\end{document}
%</driver>
% \fi
%
%
% \CharacterTable
%  {Upper-case    \A\B\C\D\E\F\G\H\I\J\K\L\M\N\O\P\Q\R\S\T\U\V\W\X\Y\Z
%   Lower-case    \a\b\c\d\e\f\g\h\i\j\k\l\m\n\o\p\q\r\s\t\u\v\w\x\y\z
%   Digits        \0\1\2\3\4\5\6\7\8\9
%   Exclamation   \!     Double quote  \"     Hash (number) \#
%   Dollar        \$     Percent       \%     Ampersand     \&
%   Acute accent  \'     Left paren    \(     Right paren   \)
%   Asterisk      \*     Plus          \+     Comma         \,
%   Minus         \-     Point         \.     Solidus       \/
%   Colon         \:     Semicolon     \;     Less than     \<
%   Equals        \=     Greater than  \>     Question mark \?
%   Commercial at \@     Left bracket  \[     Backslash     \\
%   Right bracket \]     Circumflex    \^     Underscore    \_
%   Grave accent  \`     Left brace    \{     Vertical bar  \|
%   Right brace   \}     Tilde         \~}
%
% \GetFileInfo{alphalph.drv}
%
% \title{The \xpackage{alphalph} package}
% \date{2016/05/16 v2.5}
% \author{Heiko Oberdiek\thanks
% {Please report any issues at \url{https://github.com/ho-tex/oberdiek/issues}}\\
% \xemail{heiko.oberdiek at googlemail.com}}
%
% \maketitle
%
% \begin{abstract}
% The package provides methods to represent numbers with a limited
% set of symbols. Both \hologo{LaTeX} and \hologo{plainTeX} are supported.
% \end{abstract}
%
% \tableofcontents
%
% \section{Documentation}
%
% \subsection{Introduction}
% \hologo{LaTeX} counters can be represented in different ways
% by using presentation commands:
% \begin{quote}
%   \cs{arabic}, \cs{roman}, \cs{Roman},\\
%   \cs{alph}, \cs{Alph}, \cs{fnsymbol}
% \end{quote}
% The ranges of supported counter values are more or less
% restricted. Only \cs{arabic} can be used with any counter
% value \hologo{TeX} supports.
% \begin{quote}
% \catcode`\|=12 %
% \begin{tabular}{@{}l|l|l|l@{}}
% Presentation & Supported & Ignored & Error message\\
% command      & domain    & values  & ``Counter too large''\\
% \hline
% \cs{arabic}
%   & \ttfamily -MAX..MAX & &\\
% \cs{roman}, \cs{Roman}
%   & \ttfamily 1..MAX & \ttfamily -MAX..0 &\\
% \cs{alph}, \cs{Alph}
%   & \ttfamily 1..26 & 0 & \ttfamily -MAX..-1, 27..MAX\\
% \cs{fnsymbol}
%   & \ttfamily 1..9 & 0 & \ttfamily -MAX..-1, 10..MAX\\
% \hline
% \end{tabular}\\[1ex]
% \texttt{MAX} = \texttt{2147483647}
% \end{quote}
% Ordinal numbers are often used in documents: numbering
% of chapters, sections, figures, footnotes and so on.
% The layouter chooses \cs{Alph} for chapter numbers
% and \cs{fnsymbol} for footnotes. But what can be done
% if there are more than 26 chapters or more than 10 footnotes?
% This package \xpackage{alphalph} allows to define new
% presentation commands. They rely on a existing command
% and define presentations for values greater the limits.
% Three different methods are provided by the package.
% In the following use cases they are presentated.
%
% \subsection{Use cases}
%
% \subsubsection{Number system based on symbols}
%
% Asume you are writing a book and your lecturer demands
% that chapter numbers must be letters. But you have already
% 30 chapters and you have only 26 letters?
%
% In the decimal system the situation would be clear.
% If you run out of digits, you are using more digits
% to represent a number. This method can be also be used
% for letters. After chapter 26 with |Z| we us |AA|, |AB|, |AC|, and |AD|
% for the remaining chapters.
%
% Happily this package already defines this presentation command:
%\begin{quote}
%\begin{verbatim}
%\usepackage{alphalph}
%\renewcommand*{\thechapter}{%
%  \AlphAlph{\value{chapter}}%
%}
%\end{verbatim}
%\end{quote}
% \cs{AlphAlph} generates:
%   \AlphAlph{1}, \AlphAlph{2}, \AlphAlph{3}, \dots,
%   \AlphAlph{26}, \AlphAlph{27}, \AlphAlph{28}, \dots
%
% The other presentation command is \cs{alphalph} for lowercase letters.
%
% \subsubsection{Wrap symbols around}
%
% Nine footnote symbols are quite a few.
% Too soon the symbols are consumed
% and \hologo{LaTeX} complains with the error ``Counter too large''.
% However, it could be acceptable to start again with the symbols
% from the beginning, especially if there are less than nine symbols
% on a page.
% This could be achieved by a counter reset.
% But finding the right place can be difficult or needs manual
% actions. Also a
% unique counter value can be desirable (e.g. for generating
% unique anchor/link names).
% Package \xpackage{alphalph} allows you to define
% a macro that implements a ``wrap around'', but letting
% the value of the counter untouched:
%\begin{quote}
%\begin{verbatim}
%\usepackage{alphalph}
%\makeatletter
%\newalphalph{\fnsymbolwrap}[wrap]{\@fnsymbol}{}
%\makeatother
%\renewcommand*{\thefootnote}{%
%  \fnsymbolwrap{\value{footnote}}%
%}
%\end{verbatim}
%\end{quote}
% \cs{fnsymbolwrap} generates:
% \fnsymbolwrap{1}~(1), \fnsymbolwrap{2}~(2), \fnsymbolwrap{3}~(3), \dots,
% \fnsymbolwrap{9}~(9), \fnsymbolwrap{10}~(10), \fnsymbolwrap{11}~11, \dots
%
% \subsubsection{Multiple symbols}
%
% \Hologo{LaTeX}'s standard set of footnote symbols contains doubled
% symbols at the higher positions. Could this principle
% be generalized?
% Yes, but first we need a clean footnote symbol list without
% doubled entries, example:
%\begin{quote}
%\begin{verbatim}
%\usepackage{alphalph}
%\makeatletter
%\newcommand*{\fnsymbolsingle}[1]{%
%  \ensuremath{%
%    \ifcase#1%
%    \or *%
%    \or \dagger
%    \or \ddagger
%    \or \mathsection
%    \or \mathparagraph
%    \else
%      \@ctrerr
%    \fi
%  }%
%}
%\makeatother
%\newalphalph{\fnsymbolmult}[mult]{\fnsymbolsingle}{}
%\renewcommand*{\thefootnote}{%
%  \fnsymbolmult{\value{footnote}}%
%}
%\end{verbatim}
%\end{quote}
% The own definition of \cs{fnsymbolsingle} has the
% advantage that this list can easily modified.
% Otherwise you can use \cs{@fnsymbol} directly,
% because it uses the same first five symbols.
%\begin{quote}
%\begin{verbatim}
%\usepackage{alphalph}
%\makeatletter
%\newalphalph{\fnsymbolmult}[mult]{\@fnsymbol}{5}
%\makeatother
%\renewcommand*{\thefootnote}{%
%  \fnsymbolmult{\value{footnote}}%
%}
%\end{verbatim}
%\end{quote}
% \cs{fnsymbolmult} generates:
% \fnsymbolmult{1}~(1), \fnsymbolmult{2}~(2), \fnsymbolmult{3}~(3),
% \fnsymbolmult{4}~(4), \fnsymbolmult{5}~(5), \fnsymbolmult{6}~(6),
% \dots, \fnsymbolmult{16}~16, \fnsymbolmult{17}~17, \dots
%
% \bigskip
% The same method can also be used for the chapter problem
% in the first discussed use case:
%\begin{quote}
%\begin{verbatim}
%\usepackage{alphalph}
%\makeatletter
%\newalphalph{\AlphMult}[mult]{\@Alph}{26}
%\makeatother
%\renewcommand*{\chapter}{%
%  \AlphMult{\value{chapter}}%
%}
%\end{verbatim}
%\end{quote}
% \cs{AlphMult} then generates AA, BB, CC, and DD for chapters 27--30.
%
% \subsection{Glossary}
%
% \begin{description}
% \item[Counter presentation command] is a macro that
%   expects a \hologo{LaTeX} counter name as argument.
%   Numbers cannot be used. Examples:
%   \cs{arabic}, \cs{alph}, \cs{fnsymbol}.
% \item[Number presentation command] is a macro that
%   expects a number as argument. A number is anything
%   that \hologo{TeX} accepts as number including \cs{value}.
%   Examples: \cs{alphalph}, \cs{AlphAlph}, \cs{alphalph@alph}
%
%   However, \cs{alph} or \cs{fnsymbol} are not number
%   presentation commands because they expect a counter
%   name as argument. Happily \hologo{LaTeX} counter presentation
%   commands internally uses number presentation commands
%   with the same name, but prefixed by `@'. Thus
%   \cs{@alph}, \cs{@fnsymbol} are number presentation commands.
% \item[Symbols provider] is a command that can be used
%   to get a list of symbols. For example, \cs{@Alph}
%   provides the 26 uppercase letters from `A' to `Z'.
%   Basically a symbol provider is a number presentation
%   command, usually with a limited range.
% \item[Number of symbols] is the number of the
%   last symbol slot of a symbol provider. Thus \cs{@Alph}
%   generates 26 symbols, \cs{@fnsymbol} provides 9 symbols.
% \end{description}
%
% \subsection{Package usage}
%
% The package \Package{alphalph} can be used with both \hologo{plainTeX}
% and \hologo{LaTeX}:
% \begin{description}
% \item[\hologo{plainTeX}:] |\input alphalph.sty|
% \item[\hologo{LaTeX2e}:]   |\usepackage{alphalph}|\\
%      There aren't any options.
% \end{description}
%
% \subsection{User commands}
%
% \begin{declcs}{AlphAlph} \M{number}\\
%   \cs{alphalph} \M{number}
% \end{declcs}
% Both macros are number presentation commands that
% expects a number as argument. \hologo{LaTeX} counters
% are used with \cs{value}.
%
% The macros represents a number by letters.
% First single letters |A..Z| are used, then
% two letters |AA..ZZ|, three letters |AAA...ZZZ|, \dots
% follow.
%
% Macro \cs{AlphAlph} uses uppercase letters,
% \cs{alphalph} generates the lowercase variant.
% \begin{quote}
%   \catcode`\|=12
%   \def\entry#1{
%     \texttt{#1}
%     & \texttt{\AlphAlph{#1}}
%     & \texttt{\alphalph{#1}}
%   }
%   \begin{tabular}{@{}l|l|l@{}}
%     \meta{number} & \verb|\AlphAlph{|\meta{number}\verb|}|
%                   & \verb|\alphalph{|\meta{number}\verb|}|\\
%     \hline
%     \entry{1}\\
%     \entry{2}\\
%     \entry{26}\\
%     \entry{27}\\
%     \entry{30}\\
%     \entry{2000}\\
%     \entry{3752127}\\
%     \entry{10786572}\\
%     \entry{2147483647}\\
%   \end{tabular}
% \end{quote}
%
% \begin{declcs}{newalphalph}
%   \M{cmd}
%   |[|\meta{method}|]|
%   \M{symbols provider}
%   \M{number of symbols}
% \end{declcs}
% Macro \cs{newalphalph} defines \meta{cmd} as new
% number presentation command. Like \cs{newcommand} an error
% is thrown, if macro \meta{cmd} already exists.
%
% The \meta{method} is one of |alph|, |wrap|, or |mult|.
% The default is |alph|.
%
% As symbol provider a number presentation command
% can be used, e.g. \cs{@fnsymbol}, \cs{@Alph}, or
% \cs{alphalph@alph}.
%
% The last argument is the number of symbols.
% If the argument is empty, then \cs{newalphalph} tries
% to find this number itself. \hologo{LaTeX}'s number presentation
% commands throw an error message, if the number is too large.
% This error message is put in a macro \cs{@ctrerr}.
% Thus \cs{newalphalph} calls the symbol provider and tests
% a number by typesetting it in a temporary box. The error
% macro \cs{@ctrerr} is catched, it proofs that the number
% is not supported. Also if the width of the result is zero
% the number is considered as unavailable.
%
% The empty argument is useful for potentially variable lists.
% However if the end cannot be detected, then the number of
% symbols must be given. This is also a lot faster. Therefore don't
% let the argument empty without reason.
%
% \subsection{Programmer commands}
%
% \begin{declcs}{alphalph@Alph} \M{number}\\
%   \cs{alphalph@alph} \M{number}
% \end{declcs}
% They are basically the same as \cs{@Alph} and \cs{@alph}.
% Some languages of package \xpackage{babel} redefine
% \hologo{LaTeX}'s macros to include some font setup that
% breaks expandibility. Therefore \cs{AlphAlph} and \cs{alphalph}
% are based on \cs{alphalph@Alph} and \cs{alphalph@alph} to get
% the letters. The behaviour of these symbol providers for numbers
% outside the range |1..26| is undefined.
%
% \subsection{Design principles}
%
% \subsubsection{Number presentation commands}
%
% All number presentation commands that this package defines
% (including \cs{alphalph} and \cs{AlphAlph}) have the following
% properties:
% \begin{itemize}
% \item They are fully expandable. This means that they can safely
%       \begin{itemize}
%       \item be written to a file,
%       \item used in moving arguments (\hologo{LaTeX}: they are \emph{robust}),
%       \item used in a \cmd{\csname}-\cmd{\endcsname} pair.
%       \end{itemize}
% \item If the argument is zero or negative, the commands expand
%       to nothing like \cmd{\romannumeral}.
% \item The argument is a \hologo{TeX} number. Anything that would be
%       accepted by \cmd{\number} is a valid argument:
%       \begin{itemize}
%       \item explicite constants,
%       \item macros that expand to a number,
%       \item count registers, \hologo{LaTeX} counter can used
%             via \cmd{\value}, e.\,g.:\\
%             |\alphalph{\value{page}}|
%       \item \dots
%       \end{itemize}
% \item \hologo{eTeX}'s numeric expressions are supported, if \hologo{eTeX} is
%       available. Then \cs{numexpr} is applied to the argument.
%       Package \cs{calc}'s expressions are not supported.
%       That would violate the expandibility.
% \end{itemize}
%
% \subsubsection{General usability}
%
% \begin{description}
% \item[\hologo{TeX} format:]
%   The package does not depend on \hologo{LaTeX}, it can also be used
%   by \hologo{plainTeX}, for example.
% \item[\hologo{eTeX}:] \eTeX is supported, the macros are shorter and
%   faster. But \eTeX's extensions are not requirements.
%   Without \hologo{eTeX}, just the implementation changes. The properties
%   remain unchanged.
% \end{description}
%
% \StopEventually{
% }
%
% \section{Implementation}
% \subsection{Begin\texorpdfstring{ of package}{}}
%    \begin{macrocode}
%<*package>
%    \end{macrocode}
%    Reload check, especially if the package is not used with \LaTeX.
%    \begin{macrocode}
\begingroup\catcode61\catcode48\catcode32=10\relax%
  \catcode13=5 % ^^M
  \endlinechar=13 %
  \catcode35=6 % #
  \catcode39=12 % '
  \catcode44=12 % ,
  \catcode45=12 % -
  \catcode46=12 % .
  \catcode58=12 % :
  \catcode64=11 % @
  \catcode123=1 % {
  \catcode125=2 % }
  \expandafter\let\expandafter\x\csname ver@alphalph.sty\endcsname
  \ifx\x\relax % plain-TeX, first loading
  \else
    \def\empty{}%
    \ifx\x\empty % LaTeX, first loading,
      % variable is initialized, but \ProvidesPackage not yet seen
    \else
      \expandafter\ifx\csname PackageInfo\endcsname\relax
        \def\x#1#2{%
          \immediate\write-1{Package #1 Info: #2.}%
        }%
      \else
        \def\x#1#2{\PackageInfo{#1}{#2, stopped}}%
      \fi
      \x{alphalph}{The package is already loaded}%
      \aftergroup\endinput
    \fi
  \fi
\endgroup%
%    \end{macrocode}
%    Package identification:
%    \begin{macrocode}
\begingroup\catcode61\catcode48\catcode32=10\relax%
  \catcode13=5 % ^^M
  \endlinechar=13 %
  \catcode35=6 % #
  \catcode39=12 % '
  \catcode40=12 % (
  \catcode41=12 % )
  \catcode44=12 % ,
  \catcode45=12 % -
  \catcode46=12 % .
  \catcode47=12 % /
  \catcode58=12 % :
  \catcode64=11 % @
  \catcode91=12 % [
  \catcode93=12 % ]
  \catcode123=1 % {
  \catcode125=2 % }
  \expandafter\ifx\csname ProvidesPackage\endcsname\relax
    \def\x#1#2#3[#4]{\endgroup
      \immediate\write-1{Package: #3 #4}%
      \xdef#1{#4}%
    }%
  \else
    \def\x#1#2[#3]{\endgroup
      #2[{#3}]%
      \ifx#1\@undefined
        \xdef#1{#3}%
      \fi
      \ifx#1\relax
        \xdef#1{#3}%
      \fi
    }%
  \fi
\expandafter\x\csname ver@alphalph.sty\endcsname
\ProvidesPackage{alphalph}%
  [2016/05/16 v2.5 Convert numbers to letters (HO)]%
%    \end{macrocode}
%
% \subsection{Catcodes}
%
%    \begin{macrocode}
\begingroup\catcode61\catcode48\catcode32=10\relax%
  \catcode13=5 % ^^M
  \endlinechar=13 %
  \catcode123=1 % {
  \catcode125=2 % }
  \catcode64=11 % @
  \def\x{\endgroup
    \expandafter\edef\csname AlPh@AtEnd\endcsname{%
      \endlinechar=\the\endlinechar\relax
      \catcode13=\the\catcode13\relax
      \catcode32=\the\catcode32\relax
      \catcode35=\the\catcode35\relax
      \catcode61=\the\catcode61\relax
      \catcode64=\the\catcode64\relax
      \catcode123=\the\catcode123\relax
      \catcode125=\the\catcode125\relax
    }%
  }%
\x\catcode61\catcode48\catcode32=10\relax%
\catcode13=5 % ^^M
\endlinechar=13 %
\catcode35=6 % #
\catcode64=11 % @
\catcode123=1 % {
\catcode125=2 % }
\def\TMP@EnsureCode#1#2{%
  \edef\AlPh@AtEnd{%
    \AlPh@AtEnd
    \catcode#1=\the\catcode#1\relax
  }%
  \catcode#1=#2\relax
}
\TMP@EnsureCode{33}{12}% !
\TMP@EnsureCode{39}{12}% '
\TMP@EnsureCode{40}{12}% (
\TMP@EnsureCode{41}{12}% )
\TMP@EnsureCode{43}{12}% +
\TMP@EnsureCode{44}{12}% ,
\TMP@EnsureCode{46}{12}% .
\TMP@EnsureCode{47}{12}% /
\TMP@EnsureCode{59}{12}% ;
\TMP@EnsureCode{60}{12}% <
\TMP@EnsureCode{62}{12}% >
\TMP@EnsureCode{91}{12}% [
\TMP@EnsureCode{93}{12}% ]
\TMP@EnsureCode{96}{12}% `
\TMP@EnsureCode{124}{12}% |
\edef\AlPh@AtEnd{\AlPh@AtEnd\noexpand\endinput}
%    \end{macrocode}
%
% \subsection{Package loading}
%
%    \begin{macrocode}
\begingroup\expandafter\expandafter\expandafter\endgroup
\expandafter\ifx\csname RequirePackage\endcsname\relax
  \input infwarerr.sty\relax
  \input intcalc.sty\relax
\else
  \RequirePackage{infwarerr}[2007/09/09]%
  \RequirePackage{intcalc}[2007/09/09]%
\fi
%    \end{macrocode}
%
% \subsection{\hologo{eTeX} detection}
%
%    \begin{macrocode}
\begingroup\expandafter\expandafter\expandafter\endgroup
\expandafter\ifx\csname numexpr\endcsname\relax
  \catcode124=9 % '!': ignore
  \catcode43=14 % '+': comment
\else
  \catcode124=14 % '!': comment
  \catcode43=9   % '+': ignore
\fi
%    \end{macrocode}
%
% \subsection{Help macros}
%
%    \begin{macro}{\AlPh@Error}
%    \begin{macrocode}
\def\AlPh@Error#1{%
  \begingroup
    \escapechar=92 % backslash
    \@PackageError{alphalph}{#1}\@ehc
  \endgroup
}
%    \end{macrocode}
%    \end{macro}
%
%    \begin{macro}{\AlPh@IfDefinable}
%    \begin{macrocode}
\begingroup\expandafter\expandafter\expandafter\endgroup
\expandafter\ifx\csname @ifdefinable\endcsname\relax
  \def\AlPh@IfDefinable#1#2{%
    \ifcase\ifx#1\@undefined\else\ifx#1\relax\else1\fi\fi0 %
      #2%
    \else
      \AlPh@Error{%
        Command \string#1 already defined%
      }%
    \fi
  }%
%    \end{macrocode}
%    \end{macro}
%    \begin{macrocode}
\else
%    \end{macrocode}
%    \begin{macro}{\AlPh@IfDefinable}
%    \begin{macrocode}
  \let\AlPh@IfDefinable\@ifdefinable
%    \end{macrocode}
%    \end{macro}
%    \begin{macrocode}
\fi
%    \end{macrocode}
%
%    \begin{macro}{\@ReturnAfterElseFi}
%    \begin{macro}{\@ReturnAfterFi}
%    The following commands moves the `then' and `else' part respectively
%    behind the |\if|-construct. This prevents a too deep |\if|-nesting
%    and so a \hologo{TeX} capacity error because of a limited input stack size.
%    I use this trick in several packages, so I don't prefix these internal
%    commands in order not to have the same macros with different names.
%    (It saves memory.)
%    \begin{macrocode}
\long\def\@ReturnAfterElseFi#1\else#2\fi{\fi#1}
\long\def\@ReturnAfterFi#1\fi{\fi#1}
%    \end{macrocode}
%    \end{macro}
%    \end{macro}
%
%    \begin{macro}{\@gobblefour}
%    \Hologo{LaTeX} defines commands for eating arguments.
%    Define \cs{@gobblefour} if it is not defined (\hologo{plainTeX}).
%    \begin{macrocode}
\expandafter\ifx\csname @gobblefour\endcsname\relax
  \long\def\@gobblefour#1#2#3#4{}%
\fi
%    \end{macrocode}
%    \end{macro}
%
%    \begin{macro}{AlPh@IfOptArg}
%    \begin{macrocode}
\begingroup\expandafter\expandafter\expandafter\endgroup
\expandafter\ifx\csname kernel@ifnextchar\endcsname\relax
  \begingroup\expandafter\expandafter\expandafter\endgroup
  \expandafter\ifx\csname @ifnextchar\endcsname\relax
    \def\AlPh@IfOptArg#1#2{%
      \def\AlPh@TempA{#1}%
      \def\AlPh@TempB{#2}%
      \futurelet\AlPh@Token\AlPh@IfOptArgNext
    }%
    \let\AlPh@BracketLeft=[%]
    \def\AlPh@IfOptArgNext{%
      \ifx\AlPh@Token\AlPh@BracketLeft
        \expandafter\AlPh@TempA
      \else
        \expandafter\AlPh@TempB
      \fi
    }%
  \else
    \def\AlPh@IfOptArg{\@ifnextchar[}%]
  \fi
\else
  \def\AlPh@IfOptArg{\kernel@ifnextchar[}%]
\fi
%    \end{macrocode}
%    \end{macro}
%
% \subsection{Symbol provider}
%
% \subsubsection{Alphabet}
%
%    The output of \cs{alphalph} and \cs{AlphAlph} should be
%    usable as part of command names (see \cs{@namedef}, \cs{csname}, \dots).
%    Unhappily some languages of package \xpackage{babel}
%    redefine \hologo{LaTeX}'s \cs{@alph} and \cs{@Alph} in a manner that
%    they cannot be used in expandable context any more.
%    Therefore package \xpackage{alphalph} provides its own
%    commands.
%
%    \begin{macro}{\alphalph@Alph}
%    \begin{macro}{\alphalph@alph}
%    The two commands \cmd{\AlPh@Alph} and \cmd{\AlPh@alph} convert a
%    number into a letter (uppercase and lowercase respectivly).
%    The character |@| is used as an error symbol,
%    if the number isn't in the range of 1 until 26.
%    Here we need no space after the number |#1|, because the error
%    symbol |@| for the zero case stops scanning the number.
%    This error symbol should not appear anywhere (except for bugs).
%    \begin{macrocode}
\def\alphalph@Alph#1{%
  \ifcase#1%
    @%
  \or A\or B\or C\or D\or E\or F\or G\or H\or I\or J\or K\or L\or M%
  \or N\or O\or P\or Q\or R\or S\or T\or U\or V\or W\or X\or Y\or Z%
  \else
    \AlPh@ctrerr
    @%
  \fi
}
\def\alphalph@alph#1{%
  \ifcase#1%
    @%
  \or a\or b\or c\or d\or e\or f\or g\or h\or i\or j\or k\or l\or m%
  \or n\or o\or p\or q\or r\or s\or t\or u\or v\or w\or x\or y\or z%
  \else
    \AlPh@ctrerr
    @%
  \fi
}
%    \end{macrocode}
%    \end{macro}
%    \end{macro}
%    \begin{macro}{\AlPh@ctrerr}
%    Macro \cs{AlPh@ctrerr} is used as hook for the algorithm
%    to get the available number of symbols.
%    \begin{macrocode}
\def\AlPh@ctrerr{}
%    \end{macrocode}
%    \end{macro}
%
% \subsection{Finding number of symbols}
%
%    \begin{macro}{\AlPh@GetNumberOfSymbols}
%    \noindent
%    |#1|: symbols provider
%    \begin{macrocode}
\def\AlPh@GetNumberOfSymbols#1{%
  \AlPh@TestNumber1!{#1}%
  \ifAlPh@Unavailable
    \def\AlPh@Number{0}%
    \AlPh@Error{No symbols found}%
  \else
    \def\AlPh@Number{1}%
    \AlPh@ExpSearch2!{#1}%
  \fi
}
%    \end{macrocode}
%    \end{macro}
%    \begin{macro}{\ifAlPh@Unavailable}
%    \begin{macrocode}
\let\ifAlPh@Unavailable\iffalse
\def\AlPh@Unavailabletrue{%
  \global\let\ifAlPh@Unavailable\iftrue
}
\def\AlPh@Unavailablefalse{%
  \global\let\ifAlPh@Unavailable\iffalse
}
%    \end{macrocode}
%    \end{macro}
%    \begin{macro}{\AlPh@TestNumber}
%    |#1|: number to be tested\\
%    |#2|: symbols provider
%    \begin{macrocode}
\def\AlPh@TestNumber#1!#2{%
  \AlPh@Unavailablefalse
  \begingroup
    \setbox0=\hbox{%
      \begingroup % color
        \let\@ctrerr\AlPh@Unavailabletrue
        \let\AlPh@ctrerr\AlPh@Unavailabletrue
        #2{#1}%
      \endgroup
    }%
    \ifdim\wd0=0pt %
      \AlPh@Unavailabletrue
    \fi
  \endgroup
}
%    \end{macrocode}
%    \end{macro}
%    \begin{macro}{\AlPh@ExpSearch}
%    |#1|: number to be tested\\
%    |#2|: symbols provider
%    \begin{macrocode}
\def\AlPh@ExpSearch#1!#2{%
  \let\AlPh@Next\relax
  \AlPh@TestNumber#1!{#2}%
  \ifAlPh@Unavailable
    \expandafter\AlPh@BinSearch\AlPh@Number!#1!{#2}%
  \else
    \def\AlPh@Number{#1}%
    \ifnum#1>1073741823 %
      \AlPh@TestNumber2147483647!{#2}%
      \ifAlPh@Unavailable
        \AlPh@BinSearch#1!2147483647!{#2}%
      \else
        \def\AlPh@Number{0}%
        \AlPh@Error{%
          Maximal symbol number not found%
        }%
      \fi
    \else
      \def\AlPh@Next{%
        \expandafter\AlPh@ExpSearch\number\intcalcShl{#1}!{#2}%
      }%
    \fi
  \fi
  \AlPh@Next
}
%    \end{macrocode}
%    \end{macro}
%    \begin{macro}{\AlPh@BinSearch}
%    |#1|: available number\\
%    |#2|: unavailable number, |#2| $>$ |#1|\\
%    |#3|: symbols provider
%    \begin{macrocode}
\def\AlPh@BinSearch#1!#2!#3{%
  \expandafter\AlPh@ProcessBinSearch
  \number\intcalcShr{\intcalcAdd{#1}{#2}}!%
  #1!#2!{#3}%
}
%    \end{macrocode}
%    \end{macro}
%    \begin{macro}{\AlPh@ProcessBinSearch}
%    |#1|: number to be tested, |#2| $\leq$ |#1| $\leq$ |#3|\\
%    |#2|: available number\\
%    |#3|: unavailable number\\
%    |#4|: symbols provider
%    \begin{macrocode}
\def\AlPh@ProcessBinSearch#1!#2!#3!#4{%
  \let\AlPh@Next\relax
  \ifnum#1>#2 %
    \ifnum#1<#3 %
      \AlPh@TestNumber#1!{#4}%
      \ifAlPh@Unavailable
        \def\AlPh@Next{%
          \AlPh@BinSearch#2!#1!{#4}%
        }%
      \else
        \def\AlPh@Next{%
          \AlPh@BinSearch#1!#3!{#4}%
        }%
      \fi
    \else
      \def\AlPh@Number{#2}%
    \fi
  \else
    \def\AlPh@Number{#2}%
  \fi
  \AlPh@Next
}
%    \end{macrocode}
%    \end{macro}
%
% \subsection{Methods}
%
%    The names of method macros start with \cs{AlPh@Method}.
%    These macros do the main job in converting a number to
%    its representation.
%    A method command is called with three arguments.
%    The first argument is the number of symbols. The
%    second argument is the basic macro for converting
%    a number with limited number range. The last parameter
%    is the number that needs converting.
%
% \subsubsection{Common methods}
%
%    \begin{macro}{\AlPh@CheckPositive}
%    \noindent
%    |#1|: number to be checked
%    |#2|: continuation macro\\
%    |#3|: number of symbols (hidden here)\\
%    |#4|: symbol provider (hidden here)
%    \begin{macrocode}
\def\AlPh@CheckPositive#1!#2{%
  \ifnum#1<1 %
    \expandafter\@gobblefour
  \fi
  #2{#1}%
}
%    \end{macrocode}
%    \end{macro}
%
% \subsubsection{Method `alph'}
%
%    \begin{macro}{\AlPh@Method@alph}
%    \noindent
%    |#1|: number of symbols\\
%    |#2|: symbols provider\\
%    |#3|: number to be converted
%    \begin{macrocode}
\def\AlPh@Method@alph#1#2#3{%
  \expandafter\AlPh@CheckPositive
|   \number#3!%
+   \the\numexpr#3!%
    \AlPh@ProcessAlph
    {#1}{#2}%
}
%    \end{macrocode}
%    \end{macro}
%    \begin{macro}{\AlPh@ProcessAlph}
%    |#1|: current number\\
%    |#2|: number of symbols\\
%    |#3|: symbols provider
%    \begin{macrocode}
\def\AlPh@ProcessAlph#1#2#3{%
  \ifnum#1>#2 %
    \@ReturnAfterElseFi{%
      \expandafter\AlPh@StepAlph\number
        \intcalcInc{%
          \intcalcMod{\intcalcDec{#1}}{#2}%
        }%
      \expandafter!\number
        \intcalcDiv{\intcalcDec{#1}}{#2}%
      !{#2}{#3}%
    }%
  \else
    \@ReturnAfterFi{%
      #3{#1}%
    }%
  \fi
}
%    \end{macrocode}
%    \end{macro}
%    \begin{macro}{\AlPh@StepAlph}
%    |#1|: current last digit\\
%    |#2|: new current number\\
%    |#3|: number of symbols\\
%    |#4|: symbols provider
%    \begin{macrocode}
\def\AlPh@StepAlph#1!#2!#3#4{%
  \AlPh@ProcessAlph{#2}{#3}{#4}%
  #4{#1}%
}
%    \end{macrocode}
%    \end{macro}
%
% \subsubsection{Method `wrap'}
%
%    \begin{macro}{\AlPh@Method@wrap}
%    \noindent
%    |#1|: number of symbols\\
%    |#2|: symbols provider\\
%    |#3|: number to be converted
%    \begin{macrocode}
\def\AlPh@Method@wrap#1#2#3{%
  \expandafter\AlPh@CheckPositive
|   \number#3!%
+   \the\numexpr#3!%
    \AlPh@ProcessWrap
    {#1}{#2}%
}
%    \end{macrocode}
%    \end{macro}
%
%    \begin{macro}{\AlPh@ProcessWrap}
%    |#1|: number to be converted\\
%    |#2|: number of symbols\\
%    |#3|: symbols provider
%    \begin{macrocode}
\def\AlPh@ProcessWrap#1#2#3{%
  \ifnum#1>#2 %
    \@ReturnAfterElseFi{%
      \expandafter\AlPh@StepWrap\number
        \intcalcInc{\intcalcMod{\intcalcDec{#1}}{#2}}%
      !{#3}%
    }%
  \else
    \@ReturnAfterFi{%
      #3{#1}%
    }%
  \fi
}
%    \end{macrocode}
%    \end{macro}
%    \begin{macro}{\AlPh@StepWrap}
%    |#1|: final number\\
%    |#2|: symbols provider
%    \begin{macrocode}
\def\AlPh@StepWrap#1!#2{%
  #2{#1}%
}
%    \end{macrocode}
%    \end{macro}
%
% \subsubsection{Method `mult'}
%
%    After the number of symbols is exhausted,
%    repetitions of the symbol are used.
%    \begin{gather*}
%      x \Def \text{number to be converted}\\
%      n \Def \text{number of symbols}\\
%      r \Def \text{repetition length}\\
%      s \Def \text{symbol slot}\\
%      r = ((x - 1) \div n) + 1\\
%      s = ((x - 1) \mod n) + 1
%    \end{gather*}
%    \begin{macro}{\AlPh@Method@mult}
%    \noindent
%    |#1|: number of symbols\\
%    |#2|: symbols provider\\
%    |#3|: number to be converted
%    \begin{macrocode}
\def\AlPh@Method@mult#1#2#3{%
  \expandafter\AlPh@CheckPositive
|   \number#3!%
+   \the\numexpr#3!%
    \AlPh@ProcessMult
    {#1}{#2}%
}
%    \end{macrocode}
%    \end{macro}
%
%    \begin{macro}{\AlPh@ProcessMult}
%    |#1|: number to be converted\\
%    |#2|: number of symbols\\
%    |#3|: symbols provider
%    \begin{macrocode}
\def\AlPh@ProcessMult#1#2#3{%
  \ifnum#1>#2 %
    \@ReturnAfterElseFi{%
      \expandafter\AlPh@StepMult\romannumeral
        \intcalcInc{\intcalcDiv{\intcalcDec{#1}}{#2}}%
        000%
      \expandafter!\number
        \intcalcInc{\intcalcMod{\intcalcDec{#1}}{#2}}%
      !{#3}%
    }%
  \else
    \@ReturnAfterFi{%
      #3{#1}%
    }%
  \fi
}
%    \end{macrocode}
%    \end{macro}
%    \begin{macro}{\AlPh@StepMult}
%    |#1#2|: repetitions coded as list of character `m'\\
%    |#3|: symbol slot\\
%    |#4|: symbols provider
%    \begin{macrocode}
\def\AlPh@StepMult#1#2!#3!#4{%
  \ifx\\#2\\%
  \else
    \@ReturnAfterFi{%
      \AlPh@StepMult#2!#3!{#4}%
    }%
  \fi
  #4{#3}%
}
%    \end{macrocode}
%    \end{macro}
%
% \subsection{User interface}
%
%    \begin{macro}{\newalphalph}
%    Macro \cs{newalphalph} had three arguments in versions below 2.0.
%    For the new method argument we use an optional argument an first
%    position.\\
%    |#1|: cmd\\
%    |[#2]|: method name: |alph| (default), |wrap|, |mult|\\ % hash-ok
%    |#3|: symbols provider\\
%    |#4|: number of symbols
%    \begin{macrocode}
\AlPh@IfDefinable\newalphalph{%
  \def\newalphalph#1{%
    \AlPh@IfOptArg{%
      \AlPh@newalphalph{#1}%
    }{%
      \AlPh@newalphalph{#1}[alph]%
    }%
  }%
}
%    \end{macrocode}
%    \end{macro}
%    \begin{macro}{\AlPh@newalphalph}
%    |#1|: cmd
%    |#2|: method name\\
%    |#3|: symbols provider\\
%    |#4|: number of symbols
%    \begin{macrocode}
\def\AlPh@newalphalph#1[#2]#3#4{%
  \begingroup\expandafter\expandafter\expandafter\endgroup
  \expandafter\ifx\csname AlPh@Method@#2\endcsname\relax
    \AlPh@Error{%
      Unknown method %
|     `#2'%
+     `\detokenize{#2}'%
    }%
  \else
    \ifx\\#4\\%
      \AlPh@GetNumberOfSymbols{#3}%
      \ifcase\AlPh@Number
      \else
        \begingroup
          \escapechar=92 % backslash
          \@PackageInfo{alphalph}{%
            Number of symbols for \string#1 is \AlPh@Number
          }%
        \endgroup
        \expandafter\AlPh@NewAlphAlph
        \csname AlPh@Method@#2\expandafter\endcsname
        \AlPh@Number!{#1}{#3}%
      \fi
    \else
      \expandafter\AlPh@NewAlphAlph
      \csname AlPh@Method@#2\expandafter\endcsname
|     \number#4!%
+     \the\numexpr#4!%
      {#1}{#3}%
    \fi
  \fi
}%
%    \end{macrocode}
%    \end{macro}
%    \begin{macro}{\AlPh@NewAlphAlph}
%    |#1|: method macro\\
%    |#2|: number of symbols\\
%    |#3|: cmd\\
%    |#4|: symbols provider
%    \begin{macrocode}
\def\AlPh@NewAlphAlph#1#2!#3#4{%
  \AlPh@IfDefinable#3{%
    \ifnum#2>0 %
      \def#3{#1{#2}{#4}}%
    \else
      \AlPh@Error{%
          Definition of \string#3 failed,\MessageBreak
          because number of symbols (#2) is not positive%
        }%
    \fi
  }%
}
%    \end{macrocode}
%    \end{macro}
%
%    \begin{macro}{\AlphAlph}
%    \begin{macrocode}
\newalphalph\AlphAlph\alphalph@Alph{26}
%    \end{macrocode}
%    \end{macro}
%    \begin{macro}{\alphalph}
%    \begin{macrocode}
\newalphalph\alphalph\alphalph@alph{26}
%    \end{macrocode}
%    \end{macro}
%
%    \begin{macrocode}
\AlPh@AtEnd%
%</package>
%    \end{macrocode}
%
% \section{Test}
%
% \subsection{Catcode checks for loading}
%
%    \begin{macrocode}
%<*test1>
%    \end{macrocode}
%    \begin{macrocode}
\catcode`\{=1 %
\catcode`\}=2 %
\catcode`\#=6 %
\catcode`\@=11 %
\expandafter\ifx\csname count@\endcsname\relax
  \countdef\count@=255 %
\fi
\expandafter\ifx\csname @gobble\endcsname\relax
  \long\def\@gobble#1{}%
\fi
\expandafter\ifx\csname @firstofone\endcsname\relax
  \long\def\@firstofone#1{#1}%
\fi
\expandafter\ifx\csname loop\endcsname\relax
  \expandafter\@firstofone
\else
  \expandafter\@gobble
\fi
{%
  \def\loop#1\repeat{%
    \def\body{#1}%
    \iterate
  }%
  \def\iterate{%
    \body
      \let\next\iterate
    \else
      \let\next\relax
    \fi
    \next
  }%
  \let\repeat=\fi
}%
\def\RestoreCatcodes{}
\count@=0 %
\loop
  \edef\RestoreCatcodes{%
    \RestoreCatcodes
    \catcode\the\count@=\the\catcode\count@\relax
  }%
\ifnum\count@<255 %
  \advance\count@ 1 %
\repeat

\def\RangeCatcodeInvalid#1#2{%
  \count@=#1\relax
  \loop
    \catcode\count@=15 %
  \ifnum\count@<#2\relax
    \advance\count@ 1 %
  \repeat
}
\def\RangeCatcodeCheck#1#2#3{%
  \count@=#1\relax
  \loop
    \ifnum#3=\catcode\count@
    \else
      \errmessage{%
        Character \the\count@\space
        with wrong catcode \the\catcode\count@\space
        instead of \number#3%
      }%
    \fi
  \ifnum\count@<#2\relax
    \advance\count@ 1 %
  \repeat
}
\def\space{ }
\expandafter\ifx\csname LoadCommand\endcsname\relax
  \def\LoadCommand{\input alphalph.sty\relax}%
\fi
\def\Test{%
  \RangeCatcodeInvalid{0}{47}%
  \RangeCatcodeInvalid{58}{64}%
  \RangeCatcodeInvalid{91}{96}%
  \RangeCatcodeInvalid{123}{255}%
  \catcode`\@=12 %
  \catcode`\\=0 %
  \catcode`\%=14 %
  \LoadCommand
  \RangeCatcodeCheck{0}{36}{15}%
  \RangeCatcodeCheck{37}{37}{14}%
  \RangeCatcodeCheck{38}{47}{15}%
  \RangeCatcodeCheck{48}{57}{12}%
  \RangeCatcodeCheck{58}{63}{15}%
  \RangeCatcodeCheck{64}{64}{12}%
  \RangeCatcodeCheck{65}{90}{11}%
  \RangeCatcodeCheck{91}{91}{15}%
  \RangeCatcodeCheck{92}{92}{0}%
  \RangeCatcodeCheck{93}{96}{15}%
  \RangeCatcodeCheck{97}{122}{11}%
  \RangeCatcodeCheck{123}{255}{15}%
  \RestoreCatcodes
}
\Test
\csname @@end\endcsname
\end
%    \end{macrocode}
%    \begin{macrocode}
%</test1>
%    \end{macrocode}
%
% \section{Macro tests}
%
%    \begin{macrocode}
%<*test2>
\NeedsTeXFormat{LaTeX2e}
\nofiles
\documentclass{article}
%<*noetex>
\makeatletter
\let\saved@numexpr\numexpr
\newcommand*{\DisableNumexpr}{%
  \let\numexpr\@undefined
}
\newcommand*{\RestoreNumexpr}{%
  \let\numexpr\saved@numexpr
}
\DisableNumexpr
%</noetex>
\usepackage{alphalph}[2016/05/16]
%<noetex>\RestoreNumexpr
\usepackage{qstest}
\IncludeTests{*}
\LogTests{log}{*}{*}

\newcommand*{\TestCmd}[3]{%
  \setbox0=\hbox{%
%<noetex>    \DisableNumexpr
    \edef\TestString{#1{#2}}%
    \expandafter\Expect\expandafter{\TestString}{#3}%
    \edef\TestString{#1{#2} }%
    \expandafter\Expect\expandafter{\TestString}{#3 }%
  }%
  \Expect*{\the\wd0}{0.0pt}%
}

\makeatletter
\newalphalph\LaTeXAlphAlph\@Alph{26}
\newalphalph\LaTeXalphalph\@alph{26}
\newalphalph\AlphWrap[wrap]\alphalph@Alph{26}
\newalphalph\alphwrap[wrap]\alphalph@alph{26}
\newalphalph\LaTeXAlphWrap[wrap]\@Alph{26}
\newalphalph\LaTeXalphwrap[wrap]\@alph{26}
\def\LastSymbol#1{%
  \ifx\\#1\\%
  \else
    \@LastSymbol#1\@nil
  \fi
}
\def\@LastSymbol#1#2\@nil{%
  \ifx\\#2\\%
    #1%
  \else
    \@LastSymbol#2\@nil
  \fi
}
\makeatother
\newcommand*{\TestAlph}[2]{%
  \uppercase{\TestCallCmd\AlphAlph{#2}}{#1}%
  \lowercase{\TestCallCmd\alphalph{#2}}{#1}%
  \uppercase{\TestCallCmd\LaTeXAlphAlph{#2}}{#1}%
  \lowercase{\TestCallCmd\LaTeXalphalph{#2}}{#1}%
  \edef\WrapString{\LastSymbol{#2}}%
  \expandafter\TestAlphWrap\expandafter{\WrapString}{#1}%
}
\newcommand*{\TestAlphWrap}[2]{%
  \uppercase{\TestCallCmd\AlphWrap{#1}}{#2}%
  \lowercase{\TestCallCmd\alphwrap{#1}}{#2}%
  \uppercase{\TestCallCmd\LaTeXAlphWrap{#1}}{#2}%
  \lowercase{\TestCallCmd\LaTeXalphwrap{#1}}{#2}%
}
\newcommand*{\TestCallCmd}[3]{%
  \TestCmd#1{#3}{#2}%
}
\begin{qstest}{AlphSymbols}{alphalph, AlphAlph, symbols}
  \TestAlph{1}{a}%
  \TestAlph{2}{b}%
  \TestAlph{3}{c}%
  \TestAlph{4}{d}%
  \TestAlph{5}{e}%
  \TestAlph{6}{f}%
  \TestAlph{7}{g}%
  \TestAlph{8}{h}%
  \TestAlph{9}{i}%
  \TestAlph{10}{j}%
  \TestAlph{11}{k}%
  \TestAlph{12}{l}%
  \TestAlph{13}{m}%
  \TestAlph{14}{n}%
  \TestAlph{15}{o}%
  \TestAlph{16}{p}%
  \TestAlph{17}{q}%
  \TestAlph{18}{r}%
  \TestAlph{19}{s}%
  \TestAlph{20}{t}%
  \TestAlph{21}{u}%
  \TestAlph{22}{v}%
  \TestAlph{23}{w}%
  \TestAlph{24}{x}%
  \TestAlph{25}{y}%
  \TestAlph{26}{z}%
\end{qstest}
\begin{qstest}{AlphRange}{alphalph, range}
  \TestAlph{0}{}%
  \TestAlph{-1}{}%
  \TestAlph{-2147483647}{}%
  \TestAlph{27}{aa}%
  \TestAlph{28}{ab}%
  \TestAlph{52}{az}%
  \TestAlph{53}{ba}%
  \TestAlph{78}{bz}%
  \TestAlph{79}{ca}%
  \TestAlph{702}{zz}%
  \TestAlph{703}{aaa}%
  \TestAlph{2147483647}{fxshrxw}%
\end{qstest}

\makeatletter
\newcommand*{\myvocals}[1]{%
  \ifcase#1X\or A\or E\or I\or O\or U\else Y\fi
}
\makeatother
\newalphalph\vocalsvocals\myvocals{5}
\newcommand*{\TestVocals}{%
  \TestCmd\vocalsvocals
}
\begin{qstest}{vocals}{vocals}
  \TestVocals{0}{}%
  \TestVocals{1}{A}%
  \TestVocals{2}{E}%
  \TestVocals{3}{I}%
  \TestVocals{4}{O}%
  \TestVocals{5}{U}%
  \TestVocals{6}{AA}%
  \TestVocals{7}{AE}%
  \TestVocals{8}{AI}%
  \TestVocals{9}{AO}%
  \TestVocals{10}{AU}%
  \TestVocals{11}{EA}%
  \TestVocals{24}{OO}%
  \TestVocals{25}{OU}%
  \TestVocals{26}{UA}%
  \TestVocals{29}{UO}%
  \TestVocals{30}{UU}%
  \TestVocals{31}{AAA}%
  \TestVocals{155}{UUU}%
  \TestVocals{156}{AAAA}%
  \TestVocals{2147483647}{AIIOOEEIOIIUOE}%
\end{qstest}

\makeatletter
\newalphalph\AlphMult[mult]{\alphalph@Alph}{26}
\newalphalph\alphmult[mult]{\alphalph@alph}{26}
\newalphalph\LaTeXAlphMult[mult]{\@Alph}{26}
\newalphalph\LaTeXalphmult[mult]{\@alph}{26}
\makeatother
\newcommand*{\TestMult}[2]{%
  \uppercase{\TestCallCmd\AlphMult{#2}}{#1}%
  \lowercase{\TestCallCmd\alphmult{#2}}{#1}%
  \uppercase{\TestCallCmd\LaTeXAlphMult{#2}}{#1}%
  \lowercase{\TestCallCmd\LaTeXalphmult{#2}}{#1}%
}
\begin{qstest}{mult}{mult}
  \TestMult{0}{}%
  \TestMult{-1}{}%
  \TestMult{-2147483647}{}%
  \TestMult{1}{a}%
  \TestMult{2}{b}%
  \TestMult{26}{z}%
  \TestMult{27}{aa}%
  \TestMult{28}{bb}%
  \TestMult{52}{zz}%
  \TestMult{53}{aaa}%
  \TestMult{54}{bbb}%
  \TestMult{259}{yyyyyyyyyy}%
  \TestMult{260}{zzzzzzzzzz}%
  \TestMult{261}{aaaaaaaaaaa}%
  \TestMult{262}{bbbbbbbbbbb}%
\end{qstest}

\def\myvocalsB#1{%
  \ifcase#1\or A\or E\or I\or O\or U\fi
}
\begin{qstest}{symbolnum}{symbolnum}
  \makeatletter
  \def\Test#1#2{%
    \let\TestCmd\relax
    \newalphalph\TestCmd{#1}{}%
    \Expect*{\AlPh@Number}{#2}%
  }%
  \Test\@alph{26}%
  \Test\@Alph{26}%
  \Test\@fnsymbol{9}%
  \Test\myvocalsB{5}%
  \Test\alphalph@alph{26}%
  \Test\alphalph@Alph{26}%
\end{qstest}

\begin{qstest}{list}{list}
  \makeatletter
  \def\catch#1\relax{%
    \def\FoundList{\catch#1}%
  }%
  \def\Test[#1]#2#3#4{%
    \let\testcmd\relax
    \newalphalph\testcmd[{#1}]{\catch}{#2}%
    \testcmd{#3}|\relax
    \expandafter\Expect\expandafter{\FoundList}{#4|}%
    %
    \let\SavedCatch\catch
    \def\catch{\noexpand\catch\noexpand\foo}%
    \edef\Result{#4|}%
    \@onelevel@sanitize\Result
    \let\catch\SavedCatch
    \let\testcmd\relax
    \newalphalph\testcmd[{#1}]{\catch\foo}{#2}%
    \testcmd{#3}|\relax
    \@onelevel@sanitize\FoundList
    \Expect*{\FoundList}*{\Result}%
  }%
  \Test[alph]{26}{3}{\catch{3}}%
  \Test[alph]{26}{12}{\catch{12}}%
  \Test[alph]{26}{27}{\catch{1}\catch{1}}%
  \Test[alph]{26}{78}{\catch{2}\catch{26}}%
  \Test[wrap]{26}{7}{\catch{7}}%
  \Test[wrap]{26}{14}{\catch{14}}%
  \Test[wrap]{26}{80}{\catch{2}}%
  \Test[wrap]{26}{700}{\catch{24}}%
  \Test[mult]{26}{4}{\catch{4}}%
  \Test[mult]{26}{17}{\catch{17}}%
  \Test[mult]{26}{54}{\catch{2}\catch{2}\catch{2}}%
\end{qstest}

\begin{document}
\end{document}
%</test2>
%    \end{macrocode}
%
% \section{Installation}
%
% \subsection{Download}
%
% \paragraph{Package.} This package is available on
% CTAN\footnote{\CTANpkg{alphalph}}:
% \begin{description}
% \item[\CTAN{macros/latex/contrib/oberdiek/alphalph.dtx}] The source file.
% \item[\CTAN{macros/latex/contrib/oberdiek/alphalph.pdf}] Documentation.
% \end{description}
%
%
% \paragraph{Bundle.} All the packages of the bundle `oberdiek'
% are also available in a TDS compliant ZIP archive. There
% the packages are already unpacked and the documentation files
% are generated. The files and directories obey the TDS standard.
% \begin{description}
% \item[\CTANinstall{install/macros/latex/contrib/oberdiek.tds.zip}]
% \end{description}
% \emph{TDS} refers to the standard ``A Directory Structure
% for \TeX\ Files'' (\CTAN{tds/tds.pdf}). Directories
% with \xfile{texmf} in their name are usually organized this way.
%
% \subsection{Bundle installation}
%
% \paragraph{Unpacking.} Unpack the \xfile{oberdiek.tds.zip} in the
% TDS tree (also known as \xfile{texmf} tree) of your choice.
% Example (linux):
% \begin{quote}
%   |unzip oberdiek.tds.zip -d ~/texmf|
% \end{quote}
%
% \paragraph{Script installation.}
% Check the directory \xfile{TDS:scripts/oberdiek/} for
% scripts that need further installation steps.
% Package \xpackage{attachfile2} comes with the Perl script
% \xfile{pdfatfi.pl} that should be installed in such a way
% that it can be called as \texttt{pdfatfi}.
% Example (linux):
% \begin{quote}
%   |chmod +x scripts/oberdiek/pdfatfi.pl|\\
%   |cp scripts/oberdiek/pdfatfi.pl /usr/local/bin/|
% \end{quote}
%
% \subsection{Package installation}
%
% \paragraph{Unpacking.} The \xfile{.dtx} file is a self-extracting
% \docstrip\ archive. The files are extracted by running the
% \xfile{.dtx} through \plainTeX:
% \begin{quote}
%   \verb|tex alphalph.dtx|
% \end{quote}
%
% \paragraph{TDS.} Now the different files must be moved into
% the different directories in your installation TDS tree
% (also known as \xfile{texmf} tree):
% \begin{quote}
% \def\t{^^A
% \begin{tabular}{@{}>{\ttfamily}l@{ $\rightarrow$ }>{\ttfamily}l@{}}
%   alphalph.sty & tex/generic/oberdiek/alphalph.sty\\
%   alphalph.pdf & doc/latex/oberdiek/alphalph.pdf\\
%   test/alphalph-test1.tex & doc/latex/oberdiek/test/alphalph-test1.tex\\
%   test/alphalph-test2.tex & doc/latex/oberdiek/test/alphalph-test2.tex\\
%   test/alphalph-test3.tex & doc/latex/oberdiek/test/alphalph-test3.tex\\
%   alphalph.dtx & source/latex/oberdiek/alphalph.dtx\\
% \end{tabular}^^A
% }^^A
% \sbox0{\t}^^A
% \ifdim\wd0>\linewidth
%   \begingroup
%     \advance\linewidth by\leftmargin
%     \advance\linewidth by\rightmargin
%   \edef\x{\endgroup
%     \def\noexpand\lw{\the\linewidth}^^A
%   }\x
%   \def\lwbox{^^A
%     \leavevmode
%     \hbox to \linewidth{^^A
%       \kern-\leftmargin\relax
%       \hss
%       \usebox0
%       \hss
%       \kern-\rightmargin\relax
%     }^^A
%   }^^A
%   \ifdim\wd0>\lw
%     \sbox0{\small\t}^^A
%     \ifdim\wd0>\linewidth
%       \ifdim\wd0>\lw
%         \sbox0{\footnotesize\t}^^A
%         \ifdim\wd0>\linewidth
%           \ifdim\wd0>\lw
%             \sbox0{\scriptsize\t}^^A
%             \ifdim\wd0>\linewidth
%               \ifdim\wd0>\lw
%                 \sbox0{\tiny\t}^^A
%                 \ifdim\wd0>\linewidth
%                   \lwbox
%                 \else
%                   \usebox0
%                 \fi
%               \else
%                 \lwbox
%               \fi
%             \else
%               \usebox0
%             \fi
%           \else
%             \lwbox
%           \fi
%         \else
%           \usebox0
%         \fi
%       \else
%         \lwbox
%       \fi
%     \else
%       \usebox0
%     \fi
%   \else
%     \lwbox
%   \fi
% \else
%   \usebox0
% \fi
% \end{quote}
% If you have a \xfile{docstrip.cfg} that configures and enables \docstrip's
% TDS installing feature, then some files can already be in the right
% place, see the documentation of \docstrip.
%
% \subsection{Refresh file name databases}
%
% If your \TeX~distribution
% (\teTeX, \mikTeX, \dots) relies on file name databases, you must refresh
% these. For example, \teTeX\ users run \verb|texhash| or
% \verb|mktexlsr|.
%
% \subsection{Some details for the interested}
%
% \paragraph{Attached source.}
%
% The PDF documentation on CTAN also includes the
% \xfile{.dtx} source file. It can be extracted by
% AcrobatReader 6 or higher. Another option is \textsf{pdftk},
% e.g. unpack the file into the current directory:
% \begin{quote}
%   \verb|pdftk alphalph.pdf unpack_files output .|
% \end{quote}
%
% \paragraph{Unpacking with \LaTeX.}
% The \xfile{.dtx} chooses its action depending on the format:
% \begin{description}
% \item[\plainTeX:] Run \docstrip\ and extract the files.
% \item[\LaTeX:] Generate the documentation.
% \end{description}
% If you insist on using \LaTeX\ for \docstrip\ (really,
% \docstrip\ does not need \LaTeX), then inform the autodetect routine
% about your intention:
% \begin{quote}
%   \verb|latex \let\install=y\input{alphalph.dtx}|
% \end{quote}
% Do not forget to quote the argument according to the demands
% of your shell.
%
% \paragraph{Generating the documentation.}
% You can use both the \xfile{.dtx} or the \xfile{.drv} to generate
% the documentation. The process can be configured by the
% configuration file \xfile{ltxdoc.cfg}. For instance, put this
% line into this file, if you want to have A4 as paper format:
% \begin{quote}
%   \verb|\PassOptionsToClass{a4paper}{article}|
% \end{quote}
% An example follows how to generate the
% documentation with pdf\LaTeX:
% \begin{quote}
%\begin{verbatim}
%pdflatex alphalph.dtx
%makeindex -s gind.ist alphalph.idx
%pdflatex alphalph.dtx
%makeindex -s gind.ist alphalph.idx
%pdflatex alphalph.dtx
%\end{verbatim}
% \end{quote}
%
% \begin{History}
%   \begin{Version}{1999/03/19 v0.1}
%   \item
%     The first version was built as a response to a
%     \URL{\link{question}}^^A
%     {https://groups.google.com/group/comp.text.tex/msg/17a74cd721641038}^^A
%     of \NameEmail{Will Douglas}{william.douglas@wolfson.ox.ac.uk}
%     and the
%     \URL{\link{request}}^^A
%     {https://groups.google.com/group/comp.text.tex/msg/8f9768825640315f}^^A
%     of \NameEmail{Donald Arsenau}{asnd@reg.triumf.ca},
%     published in the newsgroup
%     \href{news:comp.text.tex}{comp.text.tex}:
%     \URL{``\link{Re: alph counters \textgreater\ 26}''}^^A
%     {https://groups.google.com/group/comp.text.tex/msg/cec563eef8bf65d0}
%   \item
%     Copyright: LPPL (\CTAN{macros/latex/base/lppl.txt})
%   \end{Version}
%   \begin{Version}{1999/04/12 v1.0}
%   \item
%     Documentation added in dtx format.
%   \item
%     \hologo{eTeX} support added.
%   \end{Version}
%   \begin{Version}{1999/04/13 v1.1}
%   \item
%     Minor documentation change.
%   \item
%     First CTAN release.
%   \end{Version}
%   \begin{Version}{1999/06/26 v1.2}
%   \item
%     First generic code about \cmd{\ProvidesPackage} improved.
%   \item
%     Documentation: Installation part revised.
%   \end{Version}
%   \begin{Version}{2006/02/20 v1.3}
%   \item
%     Reload check (for \hologo{plainTeX})
%   \item
%     New DTX framework.
%   \item
%     LPPL 1.3
%   \end{Version}
%   \begin{Version}{2006/05/30 v1.4}
%   \item
%     \cs{newalphalph} added.
%   \end{Version}
%   \begin{Version}{2007/04/11 v1.5}
%   \item
%     Line ends sanitized.
%   \end{Version}
%   \begin{Version}{2007/09/09 v2.0}
%   \item
%     New implementation that uses package \cs{intcalc}.
%     This removes the dependency on \hologo{eTeX}.
%   \item
%     \cs{newalphalph} is extended to support new methods `wrap' and 'multi'.
%   \item
%     Documentation rewritten.
%   \end{Version}
%   \begin{Version}{2008/08/11 v2.1}
%   \item
%     Code is not changed.
%   \item
%     URLs updated from \texttt{www.dejanews.com}
%     to \texttt{groups.google.com}.
%   \end{Version}
%   \begin{Version}{2010/03/01 v2.2}
%   \item
%     Compatibility with \hologo{iniTeX}.
%   \end{Version}
%   \begin{Version}{2010/04/18 v2.3}
%   \item
%     Documentation fixes (Martin M\"unch).
%   \end{Version}
%   \begin{Version}{2011/05/13 v2.4}
%   \item
%     Documentation fixes (Jim Diamond) and using package
%     \xpackage{hologo} for the documentation.
%   \item
%     Catalogue file added.
%   \end{Version}
%   \begin{Version}{2016/05/16 v2.5}
%   \item
%     Documentation updates.
%   \end{Version}
% \end{History}
%
% \PrintIndex
%
% \Finale
\endinput

%        (quote the arguments according to the demands of your shell)
%
% Documentation:
%    (a) If alphalph.drv is present:
%           latex alphalph.drv
%    (b) Without alphalph.drv:
%           latex alphalph.dtx; ...
%    The class ltxdoc loads the configuration file ltxdoc.cfg
%    if available. Here you can specify further options, e.g.
%    use A4 as paper format:
%       \PassOptionsToClass{a4paper}{article}
%
%    Programm calls to get the documentation (example):
%       pdflatex alphalph.dtx
%       makeindex -s gind.ist alphalph.idx
%       pdflatex alphalph.dtx
%       makeindex -s gind.ist alphalph.idx
%       pdflatex alphalph.dtx
%
% Installation:
%    TDS:tex/generic/oberdiek/alphalph.sty
%    TDS:doc/latex/oberdiek/alphalph.pdf
%    TDS:doc/latex/oberdiek/test/alphalph-test1.tex
%    TDS:doc/latex/oberdiek/test/alphalph-test2.tex
%    TDS:doc/latex/oberdiek/test/alphalph-test3.tex
%    TDS:source/latex/oberdiek/alphalph.dtx
%
%<*ignore>
\begingroup
  \catcode123=1 %
  \catcode125=2 %
  \def\x{LaTeX2e}%
\expandafter\endgroup
\ifcase 0\ifx\install y1\fi\expandafter
         \ifx\csname processbatchFile\endcsname\relax\else1\fi
         \ifx\fmtname\x\else 1\fi\relax
\else\csname fi\endcsname
%</ignore>
%<*install>
\input docstrip.tex
\Msg{************************************************************************}
\Msg{* Installation}
\Msg{* Package: alphalph 2016/05/16 v2.5 Convert numbers to letters (HO)}
\Msg{************************************************************************}

\keepsilent
\askforoverwritefalse

\let\MetaPrefix\relax
\preamble

This is a generated file.

Project: alphalph
Version: 2016/05/16 v2.5

Copyright (C) 1999, 2006-2008, 2010, 2011 by
   Heiko Oberdiek <heiko.oberdiek at googlemail.com>

This work may be distributed and/or modified under the
conditions of the LaTeX Project Public License, either
version 1.3c of this license or (at your option) any later
version. This version of this license is in
   http://www.latex-project.org/lppl/lppl-1-3c.txt
and the latest version of this license is in
   http://www.latex-project.org/lppl.txt
and version 1.3 or later is part of all distributions of
LaTeX version 2005/12/01 or later.

This work has the LPPL maintenance status "maintained".

This Current Maintainer of this work is Heiko Oberdiek.

The Base Interpreter refers to any `TeX-Format',
because some files are installed in TDS:tex/generic//.

This work consists of the main source file alphalph.dtx
and the derived files
   alphalph.sty, alphalph.pdf, alphalph.ins, alphalph.drv,
   alphalph-test1.tex, alphalph-test2.tex, alphalph-test3.tex.

\endpreamble
\let\MetaPrefix\DoubleperCent

\generate{%
  \file{alphalph.ins}{\from{alphalph.dtx}{install}}%
  \file{alphalph.drv}{\from{alphalph.dtx}{driver}}%
  \usedir{tex/generic/oberdiek}%
  \file{alphalph.sty}{\from{alphalph.dtx}{package}}%
%  \usedir{doc/latex/oberdiek/test}%
%  \file{alphalph-test1.tex}{\from{alphalph.dtx}{test1}}%
%  \file{alphalph-test2.tex}{\from{alphalph.dtx}{test2,etex}}%
%  \file{alphalph-test3.tex}{\from{alphalph.dtx}{test2,noetex}}%
  \nopreamble
  \nopostamble
%  \usedir{source/latex/oberdiek/catalogue}%
%  \file{alphalph.xml}{\from{alphalph.dtx}{catalogue}}%
}

\catcode32=13\relax% active space
\let =\space%
\Msg{************************************************************************}
\Msg{*}
\Msg{* To finish the installation you have to move the following}
\Msg{* file into a directory searched by TeX:}
\Msg{*}
\Msg{*     alphalph.sty}
\Msg{*}
\Msg{* To produce the documentation run the file `alphalph.drv'}
\Msg{* through LaTeX.}
\Msg{*}
\Msg{* Happy TeXing!}
\Msg{*}
\Msg{************************************************************************}

\endbatchfile
%</install>
%<*ignore>
\fi
%</ignore>
%<*driver>
\NeedsTeXFormat{LaTeX2e}
\ProvidesFile{alphalph.drv}%
  [2016/05/16 v2.5 Convert numbers to letters (HO)]%
\documentclass{ltxdoc}
\usepackage{alphalph}[2016/05/16]
\makeatletter
\newalphalph{\fnsymbolwrap}[wrap]{\@fnsymbol}{}
\newcommand*{\fnsymbolsingle}[1]{%
  \ensuremath{%
    \ifcase#1%
    \or *%
    \or \dagger
    \or \ddagger
    \or \mathsection
    \or \mathparagraph
    \else
      \@ctrerr
    \fi
  }%
}
\makeatother
\newalphalph{\fnsymbolmult}[mult]{\fnsymbolsingle}{}
\usepackage{amsmath}
\DeclareMathOperator{\opDiv}{div}
\DeclareMathOperator{\opMod}{mod}
\newcommand*{\Def}{%
  \ensuremath{%
    \mathrel{\mathop{:}}=%
  }%
}
\usepackage{holtxdoc}[2011/11/22]
\begin{document}
  \DocInput{alphalph.dtx}%
\end{document}
%</driver>
% \fi
%
%
% \CharacterTable
%  {Upper-case    \A\B\C\D\E\F\G\H\I\J\K\L\M\N\O\P\Q\R\S\T\U\V\W\X\Y\Z
%   Lower-case    \a\b\c\d\e\f\g\h\i\j\k\l\m\n\o\p\q\r\s\t\u\v\w\x\y\z
%   Digits        \0\1\2\3\4\5\6\7\8\9
%   Exclamation   \!     Double quote  \"     Hash (number) \#
%   Dollar        \$     Percent       \%     Ampersand     \&
%   Acute accent  \'     Left paren    \(     Right paren   \)
%   Asterisk      \*     Plus          \+     Comma         \,
%   Minus         \-     Point         \.     Solidus       \/
%   Colon         \:     Semicolon     \;     Less than     \<
%   Equals        \=     Greater than  \>     Question mark \?
%   Commercial at \@     Left bracket  \[     Backslash     \\
%   Right bracket \]     Circumflex    \^     Underscore    \_
%   Grave accent  \`     Left brace    \{     Vertical bar  \|
%   Right brace   \}     Tilde         \~}
%
% \GetFileInfo{alphalph.drv}
%
% \title{The \xpackage{alphalph} package}
% \date{2016/05/16 v2.5}
% \author{Heiko Oberdiek\thanks
% {Please report any issues at \url{https://github.com/ho-tex/oberdiek/issues}}\\
% \xemail{heiko.oberdiek at googlemail.com}}
%
% \maketitle
%
% \begin{abstract}
% The package provides methods to represent numbers with a limited
% set of symbols. Both \hologo{LaTeX} and \hologo{plainTeX} are supported.
% \end{abstract}
%
% \tableofcontents
%
% \section{Documentation}
%
% \subsection{Introduction}
% \hologo{LaTeX} counters can be represented in different ways
% by using presentation commands:
% \begin{quote}
%   \cs{arabic}, \cs{roman}, \cs{Roman},\\
%   \cs{alph}, \cs{Alph}, \cs{fnsymbol}
% \end{quote}
% The ranges of supported counter values are more or less
% restricted. Only \cs{arabic} can be used with any counter
% value \hologo{TeX} supports.
% \begin{quote}
% \catcode`\|=12 %
% \begin{tabular}{@{}l|l|l|l@{}}
% Presentation & Supported & Ignored & Error message\\
% command      & domain    & values  & ``Counter too large''\\
% \hline
% \cs{arabic}
%   & \ttfamily -MAX..MAX & &\\
% \cs{roman}, \cs{Roman}
%   & \ttfamily 1..MAX & \ttfamily -MAX..0 &\\
% \cs{alph}, \cs{Alph}
%   & \ttfamily 1..26 & 0 & \ttfamily -MAX..-1, 27..MAX\\
% \cs{fnsymbol}
%   & \ttfamily 1..9 & 0 & \ttfamily -MAX..-1, 10..MAX\\
% \hline
% \end{tabular}\\[1ex]
% \texttt{MAX} = \texttt{2147483647}
% \end{quote}
% Ordinal numbers are often used in documents: numbering
% of chapters, sections, figures, footnotes and so on.
% The layouter chooses \cs{Alph} for chapter numbers
% and \cs{fnsymbol} for footnotes. But what can be done
% if there are more than 26 chapters or more than 10 footnotes?
% This package \xpackage{alphalph} allows to define new
% presentation commands. They rely on a existing command
% and define presentations for values greater the limits.
% Three different methods are provided by the package.
% In the following use cases they are presentated.
%
% \subsection{Use cases}
%
% \subsubsection{Number system based on symbols}
%
% Asume you are writing a book and your lecturer demands
% that chapter numbers must be letters. But you have already
% 30 chapters and you have only 26 letters?
%
% In the decimal system the situation would be clear.
% If you run out of digits, you are using more digits
% to represent a number. This method can be also be used
% for letters. After chapter 26 with |Z| we us |AA|, |AB|, |AC|, and |AD|
% for the remaining chapters.
%
% Happily this package already defines this presentation command:
%\begin{quote}
%\begin{verbatim}
%\usepackage{alphalph}
%\renewcommand*{\thechapter}{%
%  \AlphAlph{\value{chapter}}%
%}
%\end{verbatim}
%\end{quote}
% \cs{AlphAlph} generates:
%   \AlphAlph{1}, \AlphAlph{2}, \AlphAlph{3}, \dots,
%   \AlphAlph{26}, \AlphAlph{27}, \AlphAlph{28}, \dots
%
% The other presentation command is \cs{alphalph} for lowercase letters.
%
% \subsubsection{Wrap symbols around}
%
% Nine footnote symbols are quite a few.
% Too soon the symbols are consumed
% and \hologo{LaTeX} complains with the error ``Counter too large''.
% However, it could be acceptable to start again with the symbols
% from the beginning, especially if there are less than nine symbols
% on a page.
% This could be achieved by a counter reset.
% But finding the right place can be difficult or needs manual
% actions. Also a
% unique counter value can be desirable (e.g. for generating
% unique anchor/link names).
% Package \xpackage{alphalph} allows you to define
% a macro that implements a ``wrap around'', but letting
% the value of the counter untouched:
%\begin{quote}
%\begin{verbatim}
%\usepackage{alphalph}
%\makeatletter
%\newalphalph{\fnsymbolwrap}[wrap]{\@fnsymbol}{}
%\makeatother
%\renewcommand*{\thefootnote}{%
%  \fnsymbolwrap{\value{footnote}}%
%}
%\end{verbatim}
%\end{quote}
% \cs{fnsymbolwrap} generates:
% \fnsymbolwrap{1}~(1), \fnsymbolwrap{2}~(2), \fnsymbolwrap{3}~(3), \dots,
% \fnsymbolwrap{9}~(9), \fnsymbolwrap{10}~(10), \fnsymbolwrap{11}~11, \dots
%
% \subsubsection{Multiple symbols}
%
% \Hologo{LaTeX}'s standard set of footnote symbols contains doubled
% symbols at the higher positions. Could this principle
% be generalized?
% Yes, but first we need a clean footnote symbol list without
% doubled entries, example:
%\begin{quote}
%\begin{verbatim}
%\usepackage{alphalph}
%\makeatletter
%\newcommand*{\fnsymbolsingle}[1]{%
%  \ensuremath{%
%    \ifcase#1%
%    \or *%
%    \or \dagger
%    \or \ddagger
%    \or \mathsection
%    \or \mathparagraph
%    \else
%      \@ctrerr
%    \fi
%  }%
%}
%\makeatother
%\newalphalph{\fnsymbolmult}[mult]{\fnsymbolsingle}{}
%\renewcommand*{\thefootnote}{%
%  \fnsymbolmult{\value{footnote}}%
%}
%\end{verbatim}
%\end{quote}
% The own definition of \cs{fnsymbolsingle} has the
% advantage that this list can easily modified.
% Otherwise you can use \cs{@fnsymbol} directly,
% because it uses the same first five symbols.
%\begin{quote}
%\begin{verbatim}
%\usepackage{alphalph}
%\makeatletter
%\newalphalph{\fnsymbolmult}[mult]{\@fnsymbol}{5}
%\makeatother
%\renewcommand*{\thefootnote}{%
%  \fnsymbolmult{\value{footnote}}%
%}
%\end{verbatim}
%\end{quote}
% \cs{fnsymbolmult} generates:
% \fnsymbolmult{1}~(1), \fnsymbolmult{2}~(2), \fnsymbolmult{3}~(3),
% \fnsymbolmult{4}~(4), \fnsymbolmult{5}~(5), \fnsymbolmult{6}~(6),
% \dots, \fnsymbolmult{16}~16, \fnsymbolmult{17}~17, \dots
%
% \bigskip
% The same method can also be used for the chapter problem
% in the first discussed use case:
%\begin{quote}
%\begin{verbatim}
%\usepackage{alphalph}
%\makeatletter
%\newalphalph{\AlphMult}[mult]{\@Alph}{26}
%\makeatother
%\renewcommand*{\chapter}{%
%  \AlphMult{\value{chapter}}%
%}
%\end{verbatim}
%\end{quote}
% \cs{AlphMult} then generates AA, BB, CC, and DD for chapters 27--30.
%
% \subsection{Glossary}
%
% \begin{description}
% \item[Counter presentation command] is a macro that
%   expects a \hologo{LaTeX} counter name as argument.
%   Numbers cannot be used. Examples:
%   \cs{arabic}, \cs{alph}, \cs{fnsymbol}.
% \item[Number presentation command] is a macro that
%   expects a number as argument. A number is anything
%   that \hologo{TeX} accepts as number including \cs{value}.
%   Examples: \cs{alphalph}, \cs{AlphAlph}, \cs{alphalph@alph}
%
%   However, \cs{alph} or \cs{fnsymbol} are not number
%   presentation commands because they expect a counter
%   name as argument. Happily \hologo{LaTeX} counter presentation
%   commands internally uses number presentation commands
%   with the same name, but prefixed by `@'. Thus
%   \cs{@alph}, \cs{@fnsymbol} are number presentation commands.
% \item[Symbols provider] is a command that can be used
%   to get a list of symbols. For example, \cs{@Alph}
%   provides the 26 uppercase letters from `A' to `Z'.
%   Basically a symbol provider is a number presentation
%   command, usually with a limited range.
% \item[Number of symbols] is the number of the
%   last symbol slot of a symbol provider. Thus \cs{@Alph}
%   generates 26 symbols, \cs{@fnsymbol} provides 9 symbols.
% \end{description}
%
% \subsection{Package usage}
%
% The package \Package{alphalph} can be used with both \hologo{plainTeX}
% and \hologo{LaTeX}:
% \begin{description}
% \item[\hologo{plainTeX}:] |\input alphalph.sty|
% \item[\hologo{LaTeX2e}:]   |\usepackage{alphalph}|\\
%      There aren't any options.
% \end{description}
%
% \subsection{User commands}
%
% \begin{declcs}{AlphAlph} \M{number}\\
%   \cs{alphalph} \M{number}
% \end{declcs}
% Both macros are number presentation commands that
% expects a number as argument. \hologo{LaTeX} counters
% are used with \cs{value}.
%
% The macros represents a number by letters.
% First single letters |A..Z| are used, then
% two letters |AA..ZZ|, three letters |AAA...ZZZ|, \dots
% follow.
%
% Macro \cs{AlphAlph} uses uppercase letters,
% \cs{alphalph} generates the lowercase variant.
% \begin{quote}
%   \catcode`\|=12
%   \def\entry#1{
%     \texttt{#1}
%     & \texttt{\AlphAlph{#1}}
%     & \texttt{\alphalph{#1}}
%   }
%   \begin{tabular}{@{}l|l|l@{}}
%     \meta{number} & \verb|\AlphAlph{|\meta{number}\verb|}|
%                   & \verb|\alphalph{|\meta{number}\verb|}|\\
%     \hline
%     \entry{1}\\
%     \entry{2}\\
%     \entry{26}\\
%     \entry{27}\\
%     \entry{30}\\
%     \entry{2000}\\
%     \entry{3752127}\\
%     \entry{10786572}\\
%     \entry{2147483647}\\
%   \end{tabular}
% \end{quote}
%
% \begin{declcs}{newalphalph}
%   \M{cmd}
%   |[|\meta{method}|]|
%   \M{symbols provider}
%   \M{number of symbols}
% \end{declcs}
% Macro \cs{newalphalph} defines \meta{cmd} as new
% number presentation command. Like \cs{newcommand} an error
% is thrown, if macro \meta{cmd} already exists.
%
% The \meta{method} is one of |alph|, |wrap|, or |mult|.
% The default is |alph|.
%
% As symbol provider a number presentation command
% can be used, e.g. \cs{@fnsymbol}, \cs{@Alph}, or
% \cs{alphalph@alph}.
%
% The last argument is the number of symbols.
% If the argument is empty, then \cs{newalphalph} tries
% to find this number itself. \hologo{LaTeX}'s number presentation
% commands throw an error message, if the number is too large.
% This error message is put in a macro \cs{@ctrerr}.
% Thus \cs{newalphalph} calls the symbol provider and tests
% a number by typesetting it in a temporary box. The error
% macro \cs{@ctrerr} is catched, it proofs that the number
% is not supported. Also if the width of the result is zero
% the number is considered as unavailable.
%
% The empty argument is useful for potentially variable lists.
% However if the end cannot be detected, then the number of
% symbols must be given. This is also a lot faster. Therefore don't
% let the argument empty without reason.
%
% \subsection{Programmer commands}
%
% \begin{declcs}{alphalph@Alph} \M{number}\\
%   \cs{alphalph@alph} \M{number}
% \end{declcs}
% They are basically the same as \cs{@Alph} and \cs{@alph}.
% Some languages of package \xpackage{babel} redefine
% \hologo{LaTeX}'s macros to include some font setup that
% breaks expandibility. Therefore \cs{AlphAlph} and \cs{alphalph}
% are based on \cs{alphalph@Alph} and \cs{alphalph@alph} to get
% the letters. The behaviour of these symbol providers for numbers
% outside the range |1..26| is undefined.
%
% \subsection{Design principles}
%
% \subsubsection{Number presentation commands}
%
% All number presentation commands that this package defines
% (including \cs{alphalph} and \cs{AlphAlph}) have the following
% properties:
% \begin{itemize}
% \item They are fully expandable. This means that they can safely
%       \begin{itemize}
%       \item be written to a file,
%       \item used in moving arguments (\hologo{LaTeX}: they are \emph{robust}),
%       \item used in a \cmd{\csname}-\cmd{\endcsname} pair.
%       \end{itemize}
% \item If the argument is zero or negative, the commands expand
%       to nothing like \cmd{\romannumeral}.
% \item The argument is a \hologo{TeX} number. Anything that would be
%       accepted by \cmd{\number} is a valid argument:
%       \begin{itemize}
%       \item explicite constants,
%       \item macros that expand to a number,
%       \item count registers, \hologo{LaTeX} counter can used
%             via \cmd{\value}, e.\,g.:\\
%             |\alphalph{\value{page}}|
%       \item \dots
%       \end{itemize}
% \item \hologo{eTeX}'s numeric expressions are supported, if \hologo{eTeX} is
%       available. Then \cs{numexpr} is applied to the argument.
%       Package \cs{calc}'s expressions are not supported.
%       That would violate the expandibility.
% \end{itemize}
%
% \subsubsection{General usability}
%
% \begin{description}
% \item[\hologo{TeX} format:]
%   The package does not depend on \hologo{LaTeX}, it can also be used
%   by \hologo{plainTeX}, for example.
% \item[\hologo{eTeX}:] \eTeX is supported, the macros are shorter and
%   faster. But \eTeX's extensions are not requirements.
%   Without \hologo{eTeX}, just the implementation changes. The properties
%   remain unchanged.
% \end{description}
%
% \StopEventually{
% }
%
% \section{Implementation}
% \subsection{Begin\texorpdfstring{ of package}{}}
%    \begin{macrocode}
%<*package>
%    \end{macrocode}
%    Reload check, especially if the package is not used with \LaTeX.
%    \begin{macrocode}
\begingroup\catcode61\catcode48\catcode32=10\relax%
  \catcode13=5 % ^^M
  \endlinechar=13 %
  \catcode35=6 % #
  \catcode39=12 % '
  \catcode44=12 % ,
  \catcode45=12 % -
  \catcode46=12 % .
  \catcode58=12 % :
  \catcode64=11 % @
  \catcode123=1 % {
  \catcode125=2 % }
  \expandafter\let\expandafter\x\csname ver@alphalph.sty\endcsname
  \ifx\x\relax % plain-TeX, first loading
  \else
    \def\empty{}%
    \ifx\x\empty % LaTeX, first loading,
      % variable is initialized, but \ProvidesPackage not yet seen
    \else
      \expandafter\ifx\csname PackageInfo\endcsname\relax
        \def\x#1#2{%
          \immediate\write-1{Package #1 Info: #2.}%
        }%
      \else
        \def\x#1#2{\PackageInfo{#1}{#2, stopped}}%
      \fi
      \x{alphalph}{The package is already loaded}%
      \aftergroup\endinput
    \fi
  \fi
\endgroup%
%    \end{macrocode}
%    Package identification:
%    \begin{macrocode}
\begingroup\catcode61\catcode48\catcode32=10\relax%
  \catcode13=5 % ^^M
  \endlinechar=13 %
  \catcode35=6 % #
  \catcode39=12 % '
  \catcode40=12 % (
  \catcode41=12 % )
  \catcode44=12 % ,
  \catcode45=12 % -
  \catcode46=12 % .
  \catcode47=12 % /
  \catcode58=12 % :
  \catcode64=11 % @
  \catcode91=12 % [
  \catcode93=12 % ]
  \catcode123=1 % {
  \catcode125=2 % }
  \expandafter\ifx\csname ProvidesPackage\endcsname\relax
    \def\x#1#2#3[#4]{\endgroup
      \immediate\write-1{Package: #3 #4}%
      \xdef#1{#4}%
    }%
  \else
    \def\x#1#2[#3]{\endgroup
      #2[{#3}]%
      \ifx#1\@undefined
        \xdef#1{#3}%
      \fi
      \ifx#1\relax
        \xdef#1{#3}%
      \fi
    }%
  \fi
\expandafter\x\csname ver@alphalph.sty\endcsname
\ProvidesPackage{alphalph}%
  [2016/05/16 v2.5 Convert numbers to letters (HO)]%
%    \end{macrocode}
%
% \subsection{Catcodes}
%
%    \begin{macrocode}
\begingroup\catcode61\catcode48\catcode32=10\relax%
  \catcode13=5 % ^^M
  \endlinechar=13 %
  \catcode123=1 % {
  \catcode125=2 % }
  \catcode64=11 % @
  \def\x{\endgroup
    \expandafter\edef\csname AlPh@AtEnd\endcsname{%
      \endlinechar=\the\endlinechar\relax
      \catcode13=\the\catcode13\relax
      \catcode32=\the\catcode32\relax
      \catcode35=\the\catcode35\relax
      \catcode61=\the\catcode61\relax
      \catcode64=\the\catcode64\relax
      \catcode123=\the\catcode123\relax
      \catcode125=\the\catcode125\relax
    }%
  }%
\x\catcode61\catcode48\catcode32=10\relax%
\catcode13=5 % ^^M
\endlinechar=13 %
\catcode35=6 % #
\catcode64=11 % @
\catcode123=1 % {
\catcode125=2 % }
\def\TMP@EnsureCode#1#2{%
  \edef\AlPh@AtEnd{%
    \AlPh@AtEnd
    \catcode#1=\the\catcode#1\relax
  }%
  \catcode#1=#2\relax
}
\TMP@EnsureCode{33}{12}% !
\TMP@EnsureCode{39}{12}% '
\TMP@EnsureCode{40}{12}% (
\TMP@EnsureCode{41}{12}% )
\TMP@EnsureCode{43}{12}% +
\TMP@EnsureCode{44}{12}% ,
\TMP@EnsureCode{46}{12}% .
\TMP@EnsureCode{47}{12}% /
\TMP@EnsureCode{59}{12}% ;
\TMP@EnsureCode{60}{12}% <
\TMP@EnsureCode{62}{12}% >
\TMP@EnsureCode{91}{12}% [
\TMP@EnsureCode{93}{12}% ]
\TMP@EnsureCode{96}{12}% `
\TMP@EnsureCode{124}{12}% |
\edef\AlPh@AtEnd{\AlPh@AtEnd\noexpand\endinput}
%    \end{macrocode}
%
% \subsection{Package loading}
%
%    \begin{macrocode}
\begingroup\expandafter\expandafter\expandafter\endgroup
\expandafter\ifx\csname RequirePackage\endcsname\relax
  \input infwarerr.sty\relax
  \input intcalc.sty\relax
\else
  \RequirePackage{infwarerr}[2007/09/09]%
  \RequirePackage{intcalc}[2007/09/09]%
\fi
%    \end{macrocode}
%
% \subsection{\hologo{eTeX} detection}
%
%    \begin{macrocode}
\begingroup\expandafter\expandafter\expandafter\endgroup
\expandafter\ifx\csname numexpr\endcsname\relax
  \catcode124=9 % '!': ignore
  \catcode43=14 % '+': comment
\else
  \catcode124=14 % '!': comment
  \catcode43=9   % '+': ignore
\fi
%    \end{macrocode}
%
% \subsection{Help macros}
%
%    \begin{macro}{\AlPh@Error}
%    \begin{macrocode}
\def\AlPh@Error#1{%
  \begingroup
    \escapechar=92 % backslash
    \@PackageError{alphalph}{#1}\@ehc
  \endgroup
}
%    \end{macrocode}
%    \end{macro}
%
%    \begin{macro}{\AlPh@IfDefinable}
%    \begin{macrocode}
\begingroup\expandafter\expandafter\expandafter\endgroup
\expandafter\ifx\csname @ifdefinable\endcsname\relax
  \def\AlPh@IfDefinable#1#2{%
    \ifcase\ifx#1\@undefined\else\ifx#1\relax\else1\fi\fi0 %
      #2%
    \else
      \AlPh@Error{%
        Command \string#1 already defined%
      }%
    \fi
  }%
%    \end{macrocode}
%    \end{macro}
%    \begin{macrocode}
\else
%    \end{macrocode}
%    \begin{macro}{\AlPh@IfDefinable}
%    \begin{macrocode}
  \let\AlPh@IfDefinable\@ifdefinable
%    \end{macrocode}
%    \end{macro}
%    \begin{macrocode}
\fi
%    \end{macrocode}
%
%    \begin{macro}{\@ReturnAfterElseFi}
%    \begin{macro}{\@ReturnAfterFi}
%    The following commands moves the `then' and `else' part respectively
%    behind the |\if|-construct. This prevents a too deep |\if|-nesting
%    and so a \hologo{TeX} capacity error because of a limited input stack size.
%    I use this trick in several packages, so I don't prefix these internal
%    commands in order not to have the same macros with different names.
%    (It saves memory.)
%    \begin{macrocode}
\long\def\@ReturnAfterElseFi#1\else#2\fi{\fi#1}
\long\def\@ReturnAfterFi#1\fi{\fi#1}
%    \end{macrocode}
%    \end{macro}
%    \end{macro}
%
%    \begin{macro}{\@gobblefour}
%    \Hologo{LaTeX} defines commands for eating arguments.
%    Define \cs{@gobblefour} if it is not defined (\hologo{plainTeX}).
%    \begin{macrocode}
\expandafter\ifx\csname @gobblefour\endcsname\relax
  \long\def\@gobblefour#1#2#3#4{}%
\fi
%    \end{macrocode}
%    \end{macro}
%
%    \begin{macro}{AlPh@IfOptArg}
%    \begin{macrocode}
\begingroup\expandafter\expandafter\expandafter\endgroup
\expandafter\ifx\csname kernel@ifnextchar\endcsname\relax
  \begingroup\expandafter\expandafter\expandafter\endgroup
  \expandafter\ifx\csname @ifnextchar\endcsname\relax
    \def\AlPh@IfOptArg#1#2{%
      \def\AlPh@TempA{#1}%
      \def\AlPh@TempB{#2}%
      \futurelet\AlPh@Token\AlPh@IfOptArgNext
    }%
    \let\AlPh@BracketLeft=[%]
    \def\AlPh@IfOptArgNext{%
      \ifx\AlPh@Token\AlPh@BracketLeft
        \expandafter\AlPh@TempA
      \else
        \expandafter\AlPh@TempB
      \fi
    }%
  \else
    \def\AlPh@IfOptArg{\@ifnextchar[}%]
  \fi
\else
  \def\AlPh@IfOptArg{\kernel@ifnextchar[}%]
\fi
%    \end{macrocode}
%    \end{macro}
%
% \subsection{Symbol provider}
%
% \subsubsection{Alphabet}
%
%    The output of \cs{alphalph} and \cs{AlphAlph} should be
%    usable as part of command names (see \cs{@namedef}, \cs{csname}, \dots).
%    Unhappily some languages of package \xpackage{babel}
%    redefine \hologo{LaTeX}'s \cs{@alph} and \cs{@Alph} in a manner that
%    they cannot be used in expandable context any more.
%    Therefore package \xpackage{alphalph} provides its own
%    commands.
%
%    \begin{macro}{\alphalph@Alph}
%    \begin{macro}{\alphalph@alph}
%    The two commands \cmd{\AlPh@Alph} and \cmd{\AlPh@alph} convert a
%    number into a letter (uppercase and lowercase respectivly).
%    The character |@| is used as an error symbol,
%    if the number isn't in the range of 1 until 26.
%    Here we need no space after the number |#1|, because the error
%    symbol |@| for the zero case stops scanning the number.
%    This error symbol should not appear anywhere (except for bugs).
%    \begin{macrocode}
\def\alphalph@Alph#1{%
  \ifcase#1%
    @%
  \or A\or B\or C\or D\or E\or F\or G\or H\or I\or J\or K\or L\or M%
  \or N\or O\or P\or Q\or R\or S\or T\or U\or V\or W\or X\or Y\or Z%
  \else
    \AlPh@ctrerr
    @%
  \fi
}
\def\alphalph@alph#1{%
  \ifcase#1%
    @%
  \or a\or b\or c\or d\or e\or f\or g\or h\or i\or j\or k\or l\or m%
  \or n\or o\or p\or q\or r\or s\or t\or u\or v\or w\or x\or y\or z%
  \else
    \AlPh@ctrerr
    @%
  \fi
}
%    \end{macrocode}
%    \end{macro}
%    \end{macro}
%    \begin{macro}{\AlPh@ctrerr}
%    Macro \cs{AlPh@ctrerr} is used as hook for the algorithm
%    to get the available number of symbols.
%    \begin{macrocode}
\def\AlPh@ctrerr{}
%    \end{macrocode}
%    \end{macro}
%
% \subsection{Finding number of symbols}
%
%    \begin{macro}{\AlPh@GetNumberOfSymbols}
%    \noindent
%    |#1|: symbols provider
%    \begin{macrocode}
\def\AlPh@GetNumberOfSymbols#1{%
  \AlPh@TestNumber1!{#1}%
  \ifAlPh@Unavailable
    \def\AlPh@Number{0}%
    \AlPh@Error{No symbols found}%
  \else
    \def\AlPh@Number{1}%
    \AlPh@ExpSearch2!{#1}%
  \fi
}
%    \end{macrocode}
%    \end{macro}
%    \begin{macro}{\ifAlPh@Unavailable}
%    \begin{macrocode}
\let\ifAlPh@Unavailable\iffalse
\def\AlPh@Unavailabletrue{%
  \global\let\ifAlPh@Unavailable\iftrue
}
\def\AlPh@Unavailablefalse{%
  \global\let\ifAlPh@Unavailable\iffalse
}
%    \end{macrocode}
%    \end{macro}
%    \begin{macro}{\AlPh@TestNumber}
%    |#1|: number to be tested\\
%    |#2|: symbols provider
%    \begin{macrocode}
\def\AlPh@TestNumber#1!#2{%
  \AlPh@Unavailablefalse
  \begingroup
    \setbox0=\hbox{%
      \begingroup % color
        \let\@ctrerr\AlPh@Unavailabletrue
        \let\AlPh@ctrerr\AlPh@Unavailabletrue
        #2{#1}%
      \endgroup
    }%
    \ifdim\wd0=0pt %
      \AlPh@Unavailabletrue
    \fi
  \endgroup
}
%    \end{macrocode}
%    \end{macro}
%    \begin{macro}{\AlPh@ExpSearch}
%    |#1|: number to be tested\\
%    |#2|: symbols provider
%    \begin{macrocode}
\def\AlPh@ExpSearch#1!#2{%
  \let\AlPh@Next\relax
  \AlPh@TestNumber#1!{#2}%
  \ifAlPh@Unavailable
    \expandafter\AlPh@BinSearch\AlPh@Number!#1!{#2}%
  \else
    \def\AlPh@Number{#1}%
    \ifnum#1>1073741823 %
      \AlPh@TestNumber2147483647!{#2}%
      \ifAlPh@Unavailable
        \AlPh@BinSearch#1!2147483647!{#2}%
      \else
        \def\AlPh@Number{0}%
        \AlPh@Error{%
          Maximal symbol number not found%
        }%
      \fi
    \else
      \def\AlPh@Next{%
        \expandafter\AlPh@ExpSearch\number\intcalcShl{#1}!{#2}%
      }%
    \fi
  \fi
  \AlPh@Next
}
%    \end{macrocode}
%    \end{macro}
%    \begin{macro}{\AlPh@BinSearch}
%    |#1|: available number\\
%    |#2|: unavailable number, |#2| $>$ |#1|\\
%    |#3|: symbols provider
%    \begin{macrocode}
\def\AlPh@BinSearch#1!#2!#3{%
  \expandafter\AlPh@ProcessBinSearch
  \number\intcalcShr{\intcalcAdd{#1}{#2}}!%
  #1!#2!{#3}%
}
%    \end{macrocode}
%    \end{macro}
%    \begin{macro}{\AlPh@ProcessBinSearch}
%    |#1|: number to be tested, |#2| $\leq$ |#1| $\leq$ |#3|\\
%    |#2|: available number\\
%    |#3|: unavailable number\\
%    |#4|: symbols provider
%    \begin{macrocode}
\def\AlPh@ProcessBinSearch#1!#2!#3!#4{%
  \let\AlPh@Next\relax
  \ifnum#1>#2 %
    \ifnum#1<#3 %
      \AlPh@TestNumber#1!{#4}%
      \ifAlPh@Unavailable
        \def\AlPh@Next{%
          \AlPh@BinSearch#2!#1!{#4}%
        }%
      \else
        \def\AlPh@Next{%
          \AlPh@BinSearch#1!#3!{#4}%
        }%
      \fi
    \else
      \def\AlPh@Number{#2}%
    \fi
  \else
    \def\AlPh@Number{#2}%
  \fi
  \AlPh@Next
}
%    \end{macrocode}
%    \end{macro}
%
% \subsection{Methods}
%
%    The names of method macros start with \cs{AlPh@Method}.
%    These macros do the main job in converting a number to
%    its representation.
%    A method command is called with three arguments.
%    The first argument is the number of symbols. The
%    second argument is the basic macro for converting
%    a number with limited number range. The last parameter
%    is the number that needs converting.
%
% \subsubsection{Common methods}
%
%    \begin{macro}{\AlPh@CheckPositive}
%    \noindent
%    |#1|: number to be checked
%    |#2|: continuation macro\\
%    |#3|: number of symbols (hidden here)\\
%    |#4|: symbol provider (hidden here)
%    \begin{macrocode}
\def\AlPh@CheckPositive#1!#2{%
  \ifnum#1<1 %
    \expandafter\@gobblefour
  \fi
  #2{#1}%
}
%    \end{macrocode}
%    \end{macro}
%
% \subsubsection{Method `alph'}
%
%    \begin{macro}{\AlPh@Method@alph}
%    \noindent
%    |#1|: number of symbols\\
%    |#2|: symbols provider\\
%    |#3|: number to be converted
%    \begin{macrocode}
\def\AlPh@Method@alph#1#2#3{%
  \expandafter\AlPh@CheckPositive
|   \number#3!%
+   \the\numexpr#3!%
    \AlPh@ProcessAlph
    {#1}{#2}%
}
%    \end{macrocode}
%    \end{macro}
%    \begin{macro}{\AlPh@ProcessAlph}
%    |#1|: current number\\
%    |#2|: number of symbols\\
%    |#3|: symbols provider
%    \begin{macrocode}
\def\AlPh@ProcessAlph#1#2#3{%
  \ifnum#1>#2 %
    \@ReturnAfterElseFi{%
      \expandafter\AlPh@StepAlph\number
        \intcalcInc{%
          \intcalcMod{\intcalcDec{#1}}{#2}%
        }%
      \expandafter!\number
        \intcalcDiv{\intcalcDec{#1}}{#2}%
      !{#2}{#3}%
    }%
  \else
    \@ReturnAfterFi{%
      #3{#1}%
    }%
  \fi
}
%    \end{macrocode}
%    \end{macro}
%    \begin{macro}{\AlPh@StepAlph}
%    |#1|: current last digit\\
%    |#2|: new current number\\
%    |#3|: number of symbols\\
%    |#4|: symbols provider
%    \begin{macrocode}
\def\AlPh@StepAlph#1!#2!#3#4{%
  \AlPh@ProcessAlph{#2}{#3}{#4}%
  #4{#1}%
}
%    \end{macrocode}
%    \end{macro}
%
% \subsubsection{Method `wrap'}
%
%    \begin{macro}{\AlPh@Method@wrap}
%    \noindent
%    |#1|: number of symbols\\
%    |#2|: symbols provider\\
%    |#3|: number to be converted
%    \begin{macrocode}
\def\AlPh@Method@wrap#1#2#3{%
  \expandafter\AlPh@CheckPositive
|   \number#3!%
+   \the\numexpr#3!%
    \AlPh@ProcessWrap
    {#1}{#2}%
}
%    \end{macrocode}
%    \end{macro}
%
%    \begin{macro}{\AlPh@ProcessWrap}
%    |#1|: number to be converted\\
%    |#2|: number of symbols\\
%    |#3|: symbols provider
%    \begin{macrocode}
\def\AlPh@ProcessWrap#1#2#3{%
  \ifnum#1>#2 %
    \@ReturnAfterElseFi{%
      \expandafter\AlPh@StepWrap\number
        \intcalcInc{\intcalcMod{\intcalcDec{#1}}{#2}}%
      !{#3}%
    }%
  \else
    \@ReturnAfterFi{%
      #3{#1}%
    }%
  \fi
}
%    \end{macrocode}
%    \end{macro}
%    \begin{macro}{\AlPh@StepWrap}
%    |#1|: final number\\
%    |#2|: symbols provider
%    \begin{macrocode}
\def\AlPh@StepWrap#1!#2{%
  #2{#1}%
}
%    \end{macrocode}
%    \end{macro}
%
% \subsubsection{Method `mult'}
%
%    After the number of symbols is exhausted,
%    repetitions of the symbol are used.
%    \begin{gather*}
%      x \Def \text{number to be converted}\\
%      n \Def \text{number of symbols}\\
%      r \Def \text{repetition length}\\
%      s \Def \text{symbol slot}\\
%      r = ((x - 1) \div n) + 1\\
%      s = ((x - 1) \mod n) + 1
%    \end{gather*}
%    \begin{macro}{\AlPh@Method@mult}
%    \noindent
%    |#1|: number of symbols\\
%    |#2|: symbols provider\\
%    |#3|: number to be converted
%    \begin{macrocode}
\def\AlPh@Method@mult#1#2#3{%
  \expandafter\AlPh@CheckPositive
|   \number#3!%
+   \the\numexpr#3!%
    \AlPh@ProcessMult
    {#1}{#2}%
}
%    \end{macrocode}
%    \end{macro}
%
%    \begin{macro}{\AlPh@ProcessMult}
%    |#1|: number to be converted\\
%    |#2|: number of symbols\\
%    |#3|: symbols provider
%    \begin{macrocode}
\def\AlPh@ProcessMult#1#2#3{%
  \ifnum#1>#2 %
    \@ReturnAfterElseFi{%
      \expandafter\AlPh@StepMult\romannumeral
        \intcalcInc{\intcalcDiv{\intcalcDec{#1}}{#2}}%
        000%
      \expandafter!\number
        \intcalcInc{\intcalcMod{\intcalcDec{#1}}{#2}}%
      !{#3}%
    }%
  \else
    \@ReturnAfterFi{%
      #3{#1}%
    }%
  \fi
}
%    \end{macrocode}
%    \end{macro}
%    \begin{macro}{\AlPh@StepMult}
%    |#1#2|: repetitions coded as list of character `m'\\
%    |#3|: symbol slot\\
%    |#4|: symbols provider
%    \begin{macrocode}
\def\AlPh@StepMult#1#2!#3!#4{%
  \ifx\\#2\\%
  \else
    \@ReturnAfterFi{%
      \AlPh@StepMult#2!#3!{#4}%
    }%
  \fi
  #4{#3}%
}
%    \end{macrocode}
%    \end{macro}
%
% \subsection{User interface}
%
%    \begin{macro}{\newalphalph}
%    Macro \cs{newalphalph} had three arguments in versions below 2.0.
%    For the new method argument we use an optional argument an first
%    position.\\
%    |#1|: cmd\\
%    |[#2]|: method name: |alph| (default), |wrap|, |mult|\\ % hash-ok
%    |#3|: symbols provider\\
%    |#4|: number of symbols
%    \begin{macrocode}
\AlPh@IfDefinable\newalphalph{%
  \def\newalphalph#1{%
    \AlPh@IfOptArg{%
      \AlPh@newalphalph{#1}%
    }{%
      \AlPh@newalphalph{#1}[alph]%
    }%
  }%
}
%    \end{macrocode}
%    \end{macro}
%    \begin{macro}{\AlPh@newalphalph}
%    |#1|: cmd
%    |#2|: method name\\
%    |#3|: symbols provider\\
%    |#4|: number of symbols
%    \begin{macrocode}
\def\AlPh@newalphalph#1[#2]#3#4{%
  \begingroup\expandafter\expandafter\expandafter\endgroup
  \expandafter\ifx\csname AlPh@Method@#2\endcsname\relax
    \AlPh@Error{%
      Unknown method %
|     `#2'%
+     `\detokenize{#2}'%
    }%
  \else
    \ifx\\#4\\%
      \AlPh@GetNumberOfSymbols{#3}%
      \ifcase\AlPh@Number
      \else
        \begingroup
          \escapechar=92 % backslash
          \@PackageInfo{alphalph}{%
            Number of symbols for \string#1 is \AlPh@Number
          }%
        \endgroup
        \expandafter\AlPh@NewAlphAlph
        \csname AlPh@Method@#2\expandafter\endcsname
        \AlPh@Number!{#1}{#3}%
      \fi
    \else
      \expandafter\AlPh@NewAlphAlph
      \csname AlPh@Method@#2\expandafter\endcsname
|     \number#4!%
+     \the\numexpr#4!%
      {#1}{#3}%
    \fi
  \fi
}%
%    \end{macrocode}
%    \end{macro}
%    \begin{macro}{\AlPh@NewAlphAlph}
%    |#1|: method macro\\
%    |#2|: number of symbols\\
%    |#3|: cmd\\
%    |#4|: symbols provider
%    \begin{macrocode}
\def\AlPh@NewAlphAlph#1#2!#3#4{%
  \AlPh@IfDefinable#3{%
    \ifnum#2>0 %
      \def#3{#1{#2}{#4}}%
    \else
      \AlPh@Error{%
          Definition of \string#3 failed,\MessageBreak
          because number of symbols (#2) is not positive%
        }%
    \fi
  }%
}
%    \end{macrocode}
%    \end{macro}
%
%    \begin{macro}{\AlphAlph}
%    \begin{macrocode}
\newalphalph\AlphAlph\alphalph@Alph{26}
%    \end{macrocode}
%    \end{macro}
%    \begin{macro}{\alphalph}
%    \begin{macrocode}
\newalphalph\alphalph\alphalph@alph{26}
%    \end{macrocode}
%    \end{macro}
%
%    \begin{macrocode}
\AlPh@AtEnd%
%</package>
%    \end{macrocode}
%
% \section{Test}
%
% \subsection{Catcode checks for loading}
%
%    \begin{macrocode}
%<*test1>
%    \end{macrocode}
%    \begin{macrocode}
\catcode`\{=1 %
\catcode`\}=2 %
\catcode`\#=6 %
\catcode`\@=11 %
\expandafter\ifx\csname count@\endcsname\relax
  \countdef\count@=255 %
\fi
\expandafter\ifx\csname @gobble\endcsname\relax
  \long\def\@gobble#1{}%
\fi
\expandafter\ifx\csname @firstofone\endcsname\relax
  \long\def\@firstofone#1{#1}%
\fi
\expandafter\ifx\csname loop\endcsname\relax
  \expandafter\@firstofone
\else
  \expandafter\@gobble
\fi
{%
  \def\loop#1\repeat{%
    \def\body{#1}%
    \iterate
  }%
  \def\iterate{%
    \body
      \let\next\iterate
    \else
      \let\next\relax
    \fi
    \next
  }%
  \let\repeat=\fi
}%
\def\RestoreCatcodes{}
\count@=0 %
\loop
  \edef\RestoreCatcodes{%
    \RestoreCatcodes
    \catcode\the\count@=\the\catcode\count@\relax
  }%
\ifnum\count@<255 %
  \advance\count@ 1 %
\repeat

\def\RangeCatcodeInvalid#1#2{%
  \count@=#1\relax
  \loop
    \catcode\count@=15 %
  \ifnum\count@<#2\relax
    \advance\count@ 1 %
  \repeat
}
\def\RangeCatcodeCheck#1#2#3{%
  \count@=#1\relax
  \loop
    \ifnum#3=\catcode\count@
    \else
      \errmessage{%
        Character \the\count@\space
        with wrong catcode \the\catcode\count@\space
        instead of \number#3%
      }%
    \fi
  \ifnum\count@<#2\relax
    \advance\count@ 1 %
  \repeat
}
\def\space{ }
\expandafter\ifx\csname LoadCommand\endcsname\relax
  \def\LoadCommand{\input alphalph.sty\relax}%
\fi
\def\Test{%
  \RangeCatcodeInvalid{0}{47}%
  \RangeCatcodeInvalid{58}{64}%
  \RangeCatcodeInvalid{91}{96}%
  \RangeCatcodeInvalid{123}{255}%
  \catcode`\@=12 %
  \catcode`\\=0 %
  \catcode`\%=14 %
  \LoadCommand
  \RangeCatcodeCheck{0}{36}{15}%
  \RangeCatcodeCheck{37}{37}{14}%
  \RangeCatcodeCheck{38}{47}{15}%
  \RangeCatcodeCheck{48}{57}{12}%
  \RangeCatcodeCheck{58}{63}{15}%
  \RangeCatcodeCheck{64}{64}{12}%
  \RangeCatcodeCheck{65}{90}{11}%
  \RangeCatcodeCheck{91}{91}{15}%
  \RangeCatcodeCheck{92}{92}{0}%
  \RangeCatcodeCheck{93}{96}{15}%
  \RangeCatcodeCheck{97}{122}{11}%
  \RangeCatcodeCheck{123}{255}{15}%
  \RestoreCatcodes
}
\Test
\csname @@end\endcsname
\end
%    \end{macrocode}
%    \begin{macrocode}
%</test1>
%    \end{macrocode}
%
% \section{Macro tests}
%
%    \begin{macrocode}
%<*test2>
\NeedsTeXFormat{LaTeX2e}
\nofiles
\documentclass{article}
%<*noetex>
\makeatletter
\let\saved@numexpr\numexpr
\newcommand*{\DisableNumexpr}{%
  \let\numexpr\@undefined
}
\newcommand*{\RestoreNumexpr}{%
  \let\numexpr\saved@numexpr
}
\DisableNumexpr
%</noetex>
\usepackage{alphalph}[2016/05/16]
%<noetex>\RestoreNumexpr
\usepackage{qstest}
\IncludeTests{*}
\LogTests{log}{*}{*}

\newcommand*{\TestCmd}[3]{%
  \setbox0=\hbox{%
%<noetex>    \DisableNumexpr
    \edef\TestString{#1{#2}}%
    \expandafter\Expect\expandafter{\TestString}{#3}%
    \edef\TestString{#1{#2} }%
    \expandafter\Expect\expandafter{\TestString}{#3 }%
  }%
  \Expect*{\the\wd0}{0.0pt}%
}

\makeatletter
\newalphalph\LaTeXAlphAlph\@Alph{26}
\newalphalph\LaTeXalphalph\@alph{26}
\newalphalph\AlphWrap[wrap]\alphalph@Alph{26}
\newalphalph\alphwrap[wrap]\alphalph@alph{26}
\newalphalph\LaTeXAlphWrap[wrap]\@Alph{26}
\newalphalph\LaTeXalphwrap[wrap]\@alph{26}
\def\LastSymbol#1{%
  \ifx\\#1\\%
  \else
    \@LastSymbol#1\@nil
  \fi
}
\def\@LastSymbol#1#2\@nil{%
  \ifx\\#2\\%
    #1%
  \else
    \@LastSymbol#2\@nil
  \fi
}
\makeatother
\newcommand*{\TestAlph}[2]{%
  \uppercase{\TestCallCmd\AlphAlph{#2}}{#1}%
  \lowercase{\TestCallCmd\alphalph{#2}}{#1}%
  \uppercase{\TestCallCmd\LaTeXAlphAlph{#2}}{#1}%
  \lowercase{\TestCallCmd\LaTeXalphalph{#2}}{#1}%
  \edef\WrapString{\LastSymbol{#2}}%
  \expandafter\TestAlphWrap\expandafter{\WrapString}{#1}%
}
\newcommand*{\TestAlphWrap}[2]{%
  \uppercase{\TestCallCmd\AlphWrap{#1}}{#2}%
  \lowercase{\TestCallCmd\alphwrap{#1}}{#2}%
  \uppercase{\TestCallCmd\LaTeXAlphWrap{#1}}{#2}%
  \lowercase{\TestCallCmd\LaTeXalphwrap{#1}}{#2}%
}
\newcommand*{\TestCallCmd}[3]{%
  \TestCmd#1{#3}{#2}%
}
\begin{qstest}{AlphSymbols}{alphalph, AlphAlph, symbols}
  \TestAlph{1}{a}%
  \TestAlph{2}{b}%
  \TestAlph{3}{c}%
  \TestAlph{4}{d}%
  \TestAlph{5}{e}%
  \TestAlph{6}{f}%
  \TestAlph{7}{g}%
  \TestAlph{8}{h}%
  \TestAlph{9}{i}%
  \TestAlph{10}{j}%
  \TestAlph{11}{k}%
  \TestAlph{12}{l}%
  \TestAlph{13}{m}%
  \TestAlph{14}{n}%
  \TestAlph{15}{o}%
  \TestAlph{16}{p}%
  \TestAlph{17}{q}%
  \TestAlph{18}{r}%
  \TestAlph{19}{s}%
  \TestAlph{20}{t}%
  \TestAlph{21}{u}%
  \TestAlph{22}{v}%
  \TestAlph{23}{w}%
  \TestAlph{24}{x}%
  \TestAlph{25}{y}%
  \TestAlph{26}{z}%
\end{qstest}
\begin{qstest}{AlphRange}{alphalph, range}
  \TestAlph{0}{}%
  \TestAlph{-1}{}%
  \TestAlph{-2147483647}{}%
  \TestAlph{27}{aa}%
  \TestAlph{28}{ab}%
  \TestAlph{52}{az}%
  \TestAlph{53}{ba}%
  \TestAlph{78}{bz}%
  \TestAlph{79}{ca}%
  \TestAlph{702}{zz}%
  \TestAlph{703}{aaa}%
  \TestAlph{2147483647}{fxshrxw}%
\end{qstest}

\makeatletter
\newcommand*{\myvocals}[1]{%
  \ifcase#1X\or A\or E\or I\or O\or U\else Y\fi
}
\makeatother
\newalphalph\vocalsvocals\myvocals{5}
\newcommand*{\TestVocals}{%
  \TestCmd\vocalsvocals
}
\begin{qstest}{vocals}{vocals}
  \TestVocals{0}{}%
  \TestVocals{1}{A}%
  \TestVocals{2}{E}%
  \TestVocals{3}{I}%
  \TestVocals{4}{O}%
  \TestVocals{5}{U}%
  \TestVocals{6}{AA}%
  \TestVocals{7}{AE}%
  \TestVocals{8}{AI}%
  \TestVocals{9}{AO}%
  \TestVocals{10}{AU}%
  \TestVocals{11}{EA}%
  \TestVocals{24}{OO}%
  \TestVocals{25}{OU}%
  \TestVocals{26}{UA}%
  \TestVocals{29}{UO}%
  \TestVocals{30}{UU}%
  \TestVocals{31}{AAA}%
  \TestVocals{155}{UUU}%
  \TestVocals{156}{AAAA}%
  \TestVocals{2147483647}{AIIOOEEIOIIUOE}%
\end{qstest}

\makeatletter
\newalphalph\AlphMult[mult]{\alphalph@Alph}{26}
\newalphalph\alphmult[mult]{\alphalph@alph}{26}
\newalphalph\LaTeXAlphMult[mult]{\@Alph}{26}
\newalphalph\LaTeXalphmult[mult]{\@alph}{26}
\makeatother
\newcommand*{\TestMult}[2]{%
  \uppercase{\TestCallCmd\AlphMult{#2}}{#1}%
  \lowercase{\TestCallCmd\alphmult{#2}}{#1}%
  \uppercase{\TestCallCmd\LaTeXAlphMult{#2}}{#1}%
  \lowercase{\TestCallCmd\LaTeXalphmult{#2}}{#1}%
}
\begin{qstest}{mult}{mult}
  \TestMult{0}{}%
  \TestMult{-1}{}%
  \TestMult{-2147483647}{}%
  \TestMult{1}{a}%
  \TestMult{2}{b}%
  \TestMult{26}{z}%
  \TestMult{27}{aa}%
  \TestMult{28}{bb}%
  \TestMult{52}{zz}%
  \TestMult{53}{aaa}%
  \TestMult{54}{bbb}%
  \TestMult{259}{yyyyyyyyyy}%
  \TestMult{260}{zzzzzzzzzz}%
  \TestMult{261}{aaaaaaaaaaa}%
  \TestMult{262}{bbbbbbbbbbb}%
\end{qstest}

\def\myvocalsB#1{%
  \ifcase#1\or A\or E\or I\or O\or U\fi
}
\begin{qstest}{symbolnum}{symbolnum}
  \makeatletter
  \def\Test#1#2{%
    \let\TestCmd\relax
    \newalphalph\TestCmd{#1}{}%
    \Expect*{\AlPh@Number}{#2}%
  }%
  \Test\@alph{26}%
  \Test\@Alph{26}%
  \Test\@fnsymbol{9}%
  \Test\myvocalsB{5}%
  \Test\alphalph@alph{26}%
  \Test\alphalph@Alph{26}%
\end{qstest}

\begin{qstest}{list}{list}
  \makeatletter
  \def\catch#1\relax{%
    \def\FoundList{\catch#1}%
  }%
  \def\Test[#1]#2#3#4{%
    \let\testcmd\relax
    \newalphalph\testcmd[{#1}]{\catch}{#2}%
    \testcmd{#3}|\relax
    \expandafter\Expect\expandafter{\FoundList}{#4|}%
    %
    \let\SavedCatch\catch
    \def\catch{\noexpand\catch\noexpand\foo}%
    \edef\Result{#4|}%
    \@onelevel@sanitize\Result
    \let\catch\SavedCatch
    \let\testcmd\relax
    \newalphalph\testcmd[{#1}]{\catch\foo}{#2}%
    \testcmd{#3}|\relax
    \@onelevel@sanitize\FoundList
    \Expect*{\FoundList}*{\Result}%
  }%
  \Test[alph]{26}{3}{\catch{3}}%
  \Test[alph]{26}{12}{\catch{12}}%
  \Test[alph]{26}{27}{\catch{1}\catch{1}}%
  \Test[alph]{26}{78}{\catch{2}\catch{26}}%
  \Test[wrap]{26}{7}{\catch{7}}%
  \Test[wrap]{26}{14}{\catch{14}}%
  \Test[wrap]{26}{80}{\catch{2}}%
  \Test[wrap]{26}{700}{\catch{24}}%
  \Test[mult]{26}{4}{\catch{4}}%
  \Test[mult]{26}{17}{\catch{17}}%
  \Test[mult]{26}{54}{\catch{2}\catch{2}\catch{2}}%
\end{qstest}

\begin{document}
\end{document}
%</test2>
%    \end{macrocode}
%
% \section{Installation}
%
% \subsection{Download}
%
% \paragraph{Package.} This package is available on
% CTAN\footnote{\CTANpkg{alphalph}}:
% \begin{description}
% \item[\CTAN{macros/latex/contrib/oberdiek/alphalph.dtx}] The source file.
% \item[\CTAN{macros/latex/contrib/oberdiek/alphalph.pdf}] Documentation.
% \end{description}
%
%
% \paragraph{Bundle.} All the packages of the bundle `oberdiek'
% are also available in a TDS compliant ZIP archive. There
% the packages are already unpacked and the documentation files
% are generated. The files and directories obey the TDS standard.
% \begin{description}
% \item[\CTANinstall{install/macros/latex/contrib/oberdiek.tds.zip}]
% \end{description}
% \emph{TDS} refers to the standard ``A Directory Structure
% for \TeX\ Files'' (\CTAN{tds/tds.pdf}). Directories
% with \xfile{texmf} in their name are usually organized this way.
%
% \subsection{Bundle installation}
%
% \paragraph{Unpacking.} Unpack the \xfile{oberdiek.tds.zip} in the
% TDS tree (also known as \xfile{texmf} tree) of your choice.
% Example (linux):
% \begin{quote}
%   |unzip oberdiek.tds.zip -d ~/texmf|
% \end{quote}
%
% \paragraph{Script installation.}
% Check the directory \xfile{TDS:scripts/oberdiek/} for
% scripts that need further installation steps.
% Package \xpackage{attachfile2} comes with the Perl script
% \xfile{pdfatfi.pl} that should be installed in such a way
% that it can be called as \texttt{pdfatfi}.
% Example (linux):
% \begin{quote}
%   |chmod +x scripts/oberdiek/pdfatfi.pl|\\
%   |cp scripts/oberdiek/pdfatfi.pl /usr/local/bin/|
% \end{quote}
%
% \subsection{Package installation}
%
% \paragraph{Unpacking.} The \xfile{.dtx} file is a self-extracting
% \docstrip\ archive. The files are extracted by running the
% \xfile{.dtx} through \plainTeX:
% \begin{quote}
%   \verb|tex alphalph.dtx|
% \end{quote}
%
% \paragraph{TDS.} Now the different files must be moved into
% the different directories in your installation TDS tree
% (also known as \xfile{texmf} tree):
% \begin{quote}
% \def\t{^^A
% \begin{tabular}{@{}>{\ttfamily}l@{ $\rightarrow$ }>{\ttfamily}l@{}}
%   alphalph.sty & tex/generic/oberdiek/alphalph.sty\\
%   alphalph.pdf & doc/latex/oberdiek/alphalph.pdf\\
%   test/alphalph-test1.tex & doc/latex/oberdiek/test/alphalph-test1.tex\\
%   test/alphalph-test2.tex & doc/latex/oberdiek/test/alphalph-test2.tex\\
%   test/alphalph-test3.tex & doc/latex/oberdiek/test/alphalph-test3.tex\\
%   alphalph.dtx & source/latex/oberdiek/alphalph.dtx\\
% \end{tabular}^^A
% }^^A
% \sbox0{\t}^^A
% \ifdim\wd0>\linewidth
%   \begingroup
%     \advance\linewidth by\leftmargin
%     \advance\linewidth by\rightmargin
%   \edef\x{\endgroup
%     \def\noexpand\lw{\the\linewidth}^^A
%   }\x
%   \def\lwbox{^^A
%     \leavevmode
%     \hbox to \linewidth{^^A
%       \kern-\leftmargin\relax
%       \hss
%       \usebox0
%       \hss
%       \kern-\rightmargin\relax
%     }^^A
%   }^^A
%   \ifdim\wd0>\lw
%     \sbox0{\small\t}^^A
%     \ifdim\wd0>\linewidth
%       \ifdim\wd0>\lw
%         \sbox0{\footnotesize\t}^^A
%         \ifdim\wd0>\linewidth
%           \ifdim\wd0>\lw
%             \sbox0{\scriptsize\t}^^A
%             \ifdim\wd0>\linewidth
%               \ifdim\wd0>\lw
%                 \sbox0{\tiny\t}^^A
%                 \ifdim\wd0>\linewidth
%                   \lwbox
%                 \else
%                   \usebox0
%                 \fi
%               \else
%                 \lwbox
%               \fi
%             \else
%               \usebox0
%             \fi
%           \else
%             \lwbox
%           \fi
%         \else
%           \usebox0
%         \fi
%       \else
%         \lwbox
%       \fi
%     \else
%       \usebox0
%     \fi
%   \else
%     \lwbox
%   \fi
% \else
%   \usebox0
% \fi
% \end{quote}
% If you have a \xfile{docstrip.cfg} that configures and enables \docstrip's
% TDS installing feature, then some files can already be in the right
% place, see the documentation of \docstrip.
%
% \subsection{Refresh file name databases}
%
% If your \TeX~distribution
% (\teTeX, \mikTeX, \dots) relies on file name databases, you must refresh
% these. For example, \teTeX\ users run \verb|texhash| or
% \verb|mktexlsr|.
%
% \subsection{Some details for the interested}
%
% \paragraph{Attached source.}
%
% The PDF documentation on CTAN also includes the
% \xfile{.dtx} source file. It can be extracted by
% AcrobatReader 6 or higher. Another option is \textsf{pdftk},
% e.g. unpack the file into the current directory:
% \begin{quote}
%   \verb|pdftk alphalph.pdf unpack_files output .|
% \end{quote}
%
% \paragraph{Unpacking with \LaTeX.}
% The \xfile{.dtx} chooses its action depending on the format:
% \begin{description}
% \item[\plainTeX:] Run \docstrip\ and extract the files.
% \item[\LaTeX:] Generate the documentation.
% \end{description}
% If you insist on using \LaTeX\ for \docstrip\ (really,
% \docstrip\ does not need \LaTeX), then inform the autodetect routine
% about your intention:
% \begin{quote}
%   \verb|latex \let\install=y% \iffalse meta-comment
%
% File: alphalph.dtx
% Version: 2016/05/16 v2.5
% Info: Convert numbers to letters
%
% Copyright (C) 1999, 2006-2008, 2010, 2011 by
%    Heiko Oberdiek <heiko.oberdiek at googlemail.com>
%    2016
%    https://github.com/ho-tex/oberdiek/issues
%
% This work may be distributed and/or modified under the
% conditions of the LaTeX Project Public License, either
% version 1.3c of this license or (at your option) any later
% version. This version of this license is in
%    http://www.latex-project.org/lppl/lppl-1-3c.txt
% and the latest version of this license is in
%    http://www.latex-project.org/lppl.txt
% and version 1.3 or later is part of all distributions of
% LaTeX version 2005/12/01 or later.
%
% This work has the LPPL maintenance status "maintained".
%
% This Current Maintainer of this work is Heiko Oberdiek.
%
% The Base Interpreter refers to any `TeX-Format',
% because some files are installed in TDS:tex/generic//.
%
% This work consists of the main source file alphalph.dtx
% and the derived files
%    alphalph.sty, alphalph.pdf, alphalph.ins, alphalph.drv,
%    alphalph-test1.tex, alphalph-test2.tex, alphalph-test3.tex.
%
% Distribution:
%    CTAN:macros/latex/contrib/oberdiek/alphalph.dtx
%    CTAN:macros/latex/contrib/oberdiek/alphalph.pdf
%
% Unpacking:
%    (a) If alphalph.ins is present:
%           tex alphalph.ins
%    (b) Without alphalph.ins:
%           tex alphalph.dtx
%    (c) If you insist on using LaTeX
%           latex \let\install=y\input{alphalph.dtx}
%        (quote the arguments according to the demands of your shell)
%
% Documentation:
%    (a) If alphalph.drv is present:
%           latex alphalph.drv
%    (b) Without alphalph.drv:
%           latex alphalph.dtx; ...
%    The class ltxdoc loads the configuration file ltxdoc.cfg
%    if available. Here you can specify further options, e.g.
%    use A4 as paper format:
%       \PassOptionsToClass{a4paper}{article}
%
%    Programm calls to get the documentation (example):
%       pdflatex alphalph.dtx
%       makeindex -s gind.ist alphalph.idx
%       pdflatex alphalph.dtx
%       makeindex -s gind.ist alphalph.idx
%       pdflatex alphalph.dtx
%
% Installation:
%    TDS:tex/generic/oberdiek/alphalph.sty
%    TDS:doc/latex/oberdiek/alphalph.pdf
%    TDS:doc/latex/oberdiek/test/alphalph-test1.tex
%    TDS:doc/latex/oberdiek/test/alphalph-test2.tex
%    TDS:doc/latex/oberdiek/test/alphalph-test3.tex
%    TDS:source/latex/oberdiek/alphalph.dtx
%
%<*ignore>
\begingroup
  \catcode123=1 %
  \catcode125=2 %
  \def\x{LaTeX2e}%
\expandafter\endgroup
\ifcase 0\ifx\install y1\fi\expandafter
         \ifx\csname processbatchFile\endcsname\relax\else1\fi
         \ifx\fmtname\x\else 1\fi\relax
\else\csname fi\endcsname
%</ignore>
%<*install>
\input docstrip.tex
\Msg{************************************************************************}
\Msg{* Installation}
\Msg{* Package: alphalph 2016/05/16 v2.5 Convert numbers to letters (HO)}
\Msg{************************************************************************}

\keepsilent
\askforoverwritefalse

\let\MetaPrefix\relax
\preamble

This is a generated file.

Project: alphalph
Version: 2016/05/16 v2.5

Copyright (C) 1999, 2006-2008, 2010, 2011 by
   Heiko Oberdiek <heiko.oberdiek at googlemail.com>

This work may be distributed and/or modified under the
conditions of the LaTeX Project Public License, either
version 1.3c of this license or (at your option) any later
version. This version of this license is in
   http://www.latex-project.org/lppl/lppl-1-3c.txt
and the latest version of this license is in
   http://www.latex-project.org/lppl.txt
and version 1.3 or later is part of all distributions of
LaTeX version 2005/12/01 or later.

This work has the LPPL maintenance status "maintained".

This Current Maintainer of this work is Heiko Oberdiek.

The Base Interpreter refers to any `TeX-Format',
because some files are installed in TDS:tex/generic//.

This work consists of the main source file alphalph.dtx
and the derived files
   alphalph.sty, alphalph.pdf, alphalph.ins, alphalph.drv,
   alphalph-test1.tex, alphalph-test2.tex, alphalph-test3.tex.

\endpreamble
\let\MetaPrefix\DoubleperCent

\generate{%
  \file{alphalph.ins}{\from{alphalph.dtx}{install}}%
  \file{alphalph.drv}{\from{alphalph.dtx}{driver}}%
  \usedir{tex/generic/oberdiek}%
  \file{alphalph.sty}{\from{alphalph.dtx}{package}}%
%  \usedir{doc/latex/oberdiek/test}%
%  \file{alphalph-test1.tex}{\from{alphalph.dtx}{test1}}%
%  \file{alphalph-test2.tex}{\from{alphalph.dtx}{test2,etex}}%
%  \file{alphalph-test3.tex}{\from{alphalph.dtx}{test2,noetex}}%
  \nopreamble
  \nopostamble
%  \usedir{source/latex/oberdiek/catalogue}%
%  \file{alphalph.xml}{\from{alphalph.dtx}{catalogue}}%
}

\catcode32=13\relax% active space
\let =\space%
\Msg{************************************************************************}
\Msg{*}
\Msg{* To finish the installation you have to move the following}
\Msg{* file into a directory searched by TeX:}
\Msg{*}
\Msg{*     alphalph.sty}
\Msg{*}
\Msg{* To produce the documentation run the file `alphalph.drv'}
\Msg{* through LaTeX.}
\Msg{*}
\Msg{* Happy TeXing!}
\Msg{*}
\Msg{************************************************************************}

\endbatchfile
%</install>
%<*ignore>
\fi
%</ignore>
%<*driver>
\NeedsTeXFormat{LaTeX2e}
\ProvidesFile{alphalph.drv}%
  [2016/05/16 v2.5 Convert numbers to letters (HO)]%
\documentclass{ltxdoc}
\usepackage{alphalph}[2016/05/16]
\makeatletter
\newalphalph{\fnsymbolwrap}[wrap]{\@fnsymbol}{}
\newcommand*{\fnsymbolsingle}[1]{%
  \ensuremath{%
    \ifcase#1%
    \or *%
    \or \dagger
    \or \ddagger
    \or \mathsection
    \or \mathparagraph
    \else
      \@ctrerr
    \fi
  }%
}
\makeatother
\newalphalph{\fnsymbolmult}[mult]{\fnsymbolsingle}{}
\usepackage{amsmath}
\DeclareMathOperator{\opDiv}{div}
\DeclareMathOperator{\opMod}{mod}
\newcommand*{\Def}{%
  \ensuremath{%
    \mathrel{\mathop{:}}=%
  }%
}
\usepackage{holtxdoc}[2011/11/22]
\begin{document}
  \DocInput{alphalph.dtx}%
\end{document}
%</driver>
% \fi
%
%
% \CharacterTable
%  {Upper-case    \A\B\C\D\E\F\G\H\I\J\K\L\M\N\O\P\Q\R\S\T\U\V\W\X\Y\Z
%   Lower-case    \a\b\c\d\e\f\g\h\i\j\k\l\m\n\o\p\q\r\s\t\u\v\w\x\y\z
%   Digits        \0\1\2\3\4\5\6\7\8\9
%   Exclamation   \!     Double quote  \"     Hash (number) \#
%   Dollar        \$     Percent       \%     Ampersand     \&
%   Acute accent  \'     Left paren    \(     Right paren   \)
%   Asterisk      \*     Plus          \+     Comma         \,
%   Minus         \-     Point         \.     Solidus       \/
%   Colon         \:     Semicolon     \;     Less than     \<
%   Equals        \=     Greater than  \>     Question mark \?
%   Commercial at \@     Left bracket  \[     Backslash     \\
%   Right bracket \]     Circumflex    \^     Underscore    \_
%   Grave accent  \`     Left brace    \{     Vertical bar  \|
%   Right brace   \}     Tilde         \~}
%
% \GetFileInfo{alphalph.drv}
%
% \title{The \xpackage{alphalph} package}
% \date{2016/05/16 v2.5}
% \author{Heiko Oberdiek\thanks
% {Please report any issues at \url{https://github.com/ho-tex/oberdiek/issues}}\\
% \xemail{heiko.oberdiek at googlemail.com}}
%
% \maketitle
%
% \begin{abstract}
% The package provides methods to represent numbers with a limited
% set of symbols. Both \hologo{LaTeX} and \hologo{plainTeX} are supported.
% \end{abstract}
%
% \tableofcontents
%
% \section{Documentation}
%
% \subsection{Introduction}
% \hologo{LaTeX} counters can be represented in different ways
% by using presentation commands:
% \begin{quote}
%   \cs{arabic}, \cs{roman}, \cs{Roman},\\
%   \cs{alph}, \cs{Alph}, \cs{fnsymbol}
% \end{quote}
% The ranges of supported counter values are more or less
% restricted. Only \cs{arabic} can be used with any counter
% value \hologo{TeX} supports.
% \begin{quote}
% \catcode`\|=12 %
% \begin{tabular}{@{}l|l|l|l@{}}
% Presentation & Supported & Ignored & Error message\\
% command      & domain    & values  & ``Counter too large''\\
% \hline
% \cs{arabic}
%   & \ttfamily -MAX..MAX & &\\
% \cs{roman}, \cs{Roman}
%   & \ttfamily 1..MAX & \ttfamily -MAX..0 &\\
% \cs{alph}, \cs{Alph}
%   & \ttfamily 1..26 & 0 & \ttfamily -MAX..-1, 27..MAX\\
% \cs{fnsymbol}
%   & \ttfamily 1..9 & 0 & \ttfamily -MAX..-1, 10..MAX\\
% \hline
% \end{tabular}\\[1ex]
% \texttt{MAX} = \texttt{2147483647}
% \end{quote}
% Ordinal numbers are often used in documents: numbering
% of chapters, sections, figures, footnotes and so on.
% The layouter chooses \cs{Alph} for chapter numbers
% and \cs{fnsymbol} for footnotes. But what can be done
% if there are more than 26 chapters or more than 10 footnotes?
% This package \xpackage{alphalph} allows to define new
% presentation commands. They rely on a existing command
% and define presentations for values greater the limits.
% Three different methods are provided by the package.
% In the following use cases they are presentated.
%
% \subsection{Use cases}
%
% \subsubsection{Number system based on symbols}
%
% Asume you are writing a book and your lecturer demands
% that chapter numbers must be letters. But you have already
% 30 chapters and you have only 26 letters?
%
% In the decimal system the situation would be clear.
% If you run out of digits, you are using more digits
% to represent a number. This method can be also be used
% for letters. After chapter 26 with |Z| we us |AA|, |AB|, |AC|, and |AD|
% for the remaining chapters.
%
% Happily this package already defines this presentation command:
%\begin{quote}
%\begin{verbatim}
%\usepackage{alphalph}
%\renewcommand*{\thechapter}{%
%  \AlphAlph{\value{chapter}}%
%}
%\end{verbatim}
%\end{quote}
% \cs{AlphAlph} generates:
%   \AlphAlph{1}, \AlphAlph{2}, \AlphAlph{3}, \dots,
%   \AlphAlph{26}, \AlphAlph{27}, \AlphAlph{28}, \dots
%
% The other presentation command is \cs{alphalph} for lowercase letters.
%
% \subsubsection{Wrap symbols around}
%
% Nine footnote symbols are quite a few.
% Too soon the symbols are consumed
% and \hologo{LaTeX} complains with the error ``Counter too large''.
% However, it could be acceptable to start again with the symbols
% from the beginning, especially if there are less than nine symbols
% on a page.
% This could be achieved by a counter reset.
% But finding the right place can be difficult or needs manual
% actions. Also a
% unique counter value can be desirable (e.g. for generating
% unique anchor/link names).
% Package \xpackage{alphalph} allows you to define
% a macro that implements a ``wrap around'', but letting
% the value of the counter untouched:
%\begin{quote}
%\begin{verbatim}
%\usepackage{alphalph}
%\makeatletter
%\newalphalph{\fnsymbolwrap}[wrap]{\@fnsymbol}{}
%\makeatother
%\renewcommand*{\thefootnote}{%
%  \fnsymbolwrap{\value{footnote}}%
%}
%\end{verbatim}
%\end{quote}
% \cs{fnsymbolwrap} generates:
% \fnsymbolwrap{1}~(1), \fnsymbolwrap{2}~(2), \fnsymbolwrap{3}~(3), \dots,
% \fnsymbolwrap{9}~(9), \fnsymbolwrap{10}~(10), \fnsymbolwrap{11}~11, \dots
%
% \subsubsection{Multiple symbols}
%
% \Hologo{LaTeX}'s standard set of footnote symbols contains doubled
% symbols at the higher positions. Could this principle
% be generalized?
% Yes, but first we need a clean footnote symbol list without
% doubled entries, example:
%\begin{quote}
%\begin{verbatim}
%\usepackage{alphalph}
%\makeatletter
%\newcommand*{\fnsymbolsingle}[1]{%
%  \ensuremath{%
%    \ifcase#1%
%    \or *%
%    \or \dagger
%    \or \ddagger
%    \or \mathsection
%    \or \mathparagraph
%    \else
%      \@ctrerr
%    \fi
%  }%
%}
%\makeatother
%\newalphalph{\fnsymbolmult}[mult]{\fnsymbolsingle}{}
%\renewcommand*{\thefootnote}{%
%  \fnsymbolmult{\value{footnote}}%
%}
%\end{verbatim}
%\end{quote}
% The own definition of \cs{fnsymbolsingle} has the
% advantage that this list can easily modified.
% Otherwise you can use \cs{@fnsymbol} directly,
% because it uses the same first five symbols.
%\begin{quote}
%\begin{verbatim}
%\usepackage{alphalph}
%\makeatletter
%\newalphalph{\fnsymbolmult}[mult]{\@fnsymbol}{5}
%\makeatother
%\renewcommand*{\thefootnote}{%
%  \fnsymbolmult{\value{footnote}}%
%}
%\end{verbatim}
%\end{quote}
% \cs{fnsymbolmult} generates:
% \fnsymbolmult{1}~(1), \fnsymbolmult{2}~(2), \fnsymbolmult{3}~(3),
% \fnsymbolmult{4}~(4), \fnsymbolmult{5}~(5), \fnsymbolmult{6}~(6),
% \dots, \fnsymbolmult{16}~16, \fnsymbolmult{17}~17, \dots
%
% \bigskip
% The same method can also be used for the chapter problem
% in the first discussed use case:
%\begin{quote}
%\begin{verbatim}
%\usepackage{alphalph}
%\makeatletter
%\newalphalph{\AlphMult}[mult]{\@Alph}{26}
%\makeatother
%\renewcommand*{\chapter}{%
%  \AlphMult{\value{chapter}}%
%}
%\end{verbatim}
%\end{quote}
% \cs{AlphMult} then generates AA, BB, CC, and DD for chapters 27--30.
%
% \subsection{Glossary}
%
% \begin{description}
% \item[Counter presentation command] is a macro that
%   expects a \hologo{LaTeX} counter name as argument.
%   Numbers cannot be used. Examples:
%   \cs{arabic}, \cs{alph}, \cs{fnsymbol}.
% \item[Number presentation command] is a macro that
%   expects a number as argument. A number is anything
%   that \hologo{TeX} accepts as number including \cs{value}.
%   Examples: \cs{alphalph}, \cs{AlphAlph}, \cs{alphalph@alph}
%
%   However, \cs{alph} or \cs{fnsymbol} are not number
%   presentation commands because they expect a counter
%   name as argument. Happily \hologo{LaTeX} counter presentation
%   commands internally uses number presentation commands
%   with the same name, but prefixed by `@'. Thus
%   \cs{@alph}, \cs{@fnsymbol} are number presentation commands.
% \item[Symbols provider] is a command that can be used
%   to get a list of symbols. For example, \cs{@Alph}
%   provides the 26 uppercase letters from `A' to `Z'.
%   Basically a symbol provider is a number presentation
%   command, usually with a limited range.
% \item[Number of symbols] is the number of the
%   last symbol slot of a symbol provider. Thus \cs{@Alph}
%   generates 26 symbols, \cs{@fnsymbol} provides 9 symbols.
% \end{description}
%
% \subsection{Package usage}
%
% The package \Package{alphalph} can be used with both \hologo{plainTeX}
% and \hologo{LaTeX}:
% \begin{description}
% \item[\hologo{plainTeX}:] |\input alphalph.sty|
% \item[\hologo{LaTeX2e}:]   |\usepackage{alphalph}|\\
%      There aren't any options.
% \end{description}
%
% \subsection{User commands}
%
% \begin{declcs}{AlphAlph} \M{number}\\
%   \cs{alphalph} \M{number}
% \end{declcs}
% Both macros are number presentation commands that
% expects a number as argument. \hologo{LaTeX} counters
% are used with \cs{value}.
%
% The macros represents a number by letters.
% First single letters |A..Z| are used, then
% two letters |AA..ZZ|, three letters |AAA...ZZZ|, \dots
% follow.
%
% Macro \cs{AlphAlph} uses uppercase letters,
% \cs{alphalph} generates the lowercase variant.
% \begin{quote}
%   \catcode`\|=12
%   \def\entry#1{
%     \texttt{#1}
%     & \texttt{\AlphAlph{#1}}
%     & \texttt{\alphalph{#1}}
%   }
%   \begin{tabular}{@{}l|l|l@{}}
%     \meta{number} & \verb|\AlphAlph{|\meta{number}\verb|}|
%                   & \verb|\alphalph{|\meta{number}\verb|}|\\
%     \hline
%     \entry{1}\\
%     \entry{2}\\
%     \entry{26}\\
%     \entry{27}\\
%     \entry{30}\\
%     \entry{2000}\\
%     \entry{3752127}\\
%     \entry{10786572}\\
%     \entry{2147483647}\\
%   \end{tabular}
% \end{quote}
%
% \begin{declcs}{newalphalph}
%   \M{cmd}
%   |[|\meta{method}|]|
%   \M{symbols provider}
%   \M{number of symbols}
% \end{declcs}
% Macro \cs{newalphalph} defines \meta{cmd} as new
% number presentation command. Like \cs{newcommand} an error
% is thrown, if macro \meta{cmd} already exists.
%
% The \meta{method} is one of |alph|, |wrap|, or |mult|.
% The default is |alph|.
%
% As symbol provider a number presentation command
% can be used, e.g. \cs{@fnsymbol}, \cs{@Alph}, or
% \cs{alphalph@alph}.
%
% The last argument is the number of symbols.
% If the argument is empty, then \cs{newalphalph} tries
% to find this number itself. \hologo{LaTeX}'s number presentation
% commands throw an error message, if the number is too large.
% This error message is put in a macro \cs{@ctrerr}.
% Thus \cs{newalphalph} calls the symbol provider and tests
% a number by typesetting it in a temporary box. The error
% macro \cs{@ctrerr} is catched, it proofs that the number
% is not supported. Also if the width of the result is zero
% the number is considered as unavailable.
%
% The empty argument is useful for potentially variable lists.
% However if the end cannot be detected, then the number of
% symbols must be given. This is also a lot faster. Therefore don't
% let the argument empty without reason.
%
% \subsection{Programmer commands}
%
% \begin{declcs}{alphalph@Alph} \M{number}\\
%   \cs{alphalph@alph} \M{number}
% \end{declcs}
% They are basically the same as \cs{@Alph} and \cs{@alph}.
% Some languages of package \xpackage{babel} redefine
% \hologo{LaTeX}'s macros to include some font setup that
% breaks expandibility. Therefore \cs{AlphAlph} and \cs{alphalph}
% are based on \cs{alphalph@Alph} and \cs{alphalph@alph} to get
% the letters. The behaviour of these symbol providers for numbers
% outside the range |1..26| is undefined.
%
% \subsection{Design principles}
%
% \subsubsection{Number presentation commands}
%
% All number presentation commands that this package defines
% (including \cs{alphalph} and \cs{AlphAlph}) have the following
% properties:
% \begin{itemize}
% \item They are fully expandable. This means that they can safely
%       \begin{itemize}
%       \item be written to a file,
%       \item used in moving arguments (\hologo{LaTeX}: they are \emph{robust}),
%       \item used in a \cmd{\csname}-\cmd{\endcsname} pair.
%       \end{itemize}
% \item If the argument is zero or negative, the commands expand
%       to nothing like \cmd{\romannumeral}.
% \item The argument is a \hologo{TeX} number. Anything that would be
%       accepted by \cmd{\number} is a valid argument:
%       \begin{itemize}
%       \item explicite constants,
%       \item macros that expand to a number,
%       \item count registers, \hologo{LaTeX} counter can used
%             via \cmd{\value}, e.\,g.:\\
%             |\alphalph{\value{page}}|
%       \item \dots
%       \end{itemize}
% \item \hologo{eTeX}'s numeric expressions are supported, if \hologo{eTeX} is
%       available. Then \cs{numexpr} is applied to the argument.
%       Package \cs{calc}'s expressions are not supported.
%       That would violate the expandibility.
% \end{itemize}
%
% \subsubsection{General usability}
%
% \begin{description}
% \item[\hologo{TeX} format:]
%   The package does not depend on \hologo{LaTeX}, it can also be used
%   by \hologo{plainTeX}, for example.
% \item[\hologo{eTeX}:] \eTeX is supported, the macros are shorter and
%   faster. But \eTeX's extensions are not requirements.
%   Without \hologo{eTeX}, just the implementation changes. The properties
%   remain unchanged.
% \end{description}
%
% \StopEventually{
% }
%
% \section{Implementation}
% \subsection{Begin\texorpdfstring{ of package}{}}
%    \begin{macrocode}
%<*package>
%    \end{macrocode}
%    Reload check, especially if the package is not used with \LaTeX.
%    \begin{macrocode}
\begingroup\catcode61\catcode48\catcode32=10\relax%
  \catcode13=5 % ^^M
  \endlinechar=13 %
  \catcode35=6 % #
  \catcode39=12 % '
  \catcode44=12 % ,
  \catcode45=12 % -
  \catcode46=12 % .
  \catcode58=12 % :
  \catcode64=11 % @
  \catcode123=1 % {
  \catcode125=2 % }
  \expandafter\let\expandafter\x\csname ver@alphalph.sty\endcsname
  \ifx\x\relax % plain-TeX, first loading
  \else
    \def\empty{}%
    \ifx\x\empty % LaTeX, first loading,
      % variable is initialized, but \ProvidesPackage not yet seen
    \else
      \expandafter\ifx\csname PackageInfo\endcsname\relax
        \def\x#1#2{%
          \immediate\write-1{Package #1 Info: #2.}%
        }%
      \else
        \def\x#1#2{\PackageInfo{#1}{#2, stopped}}%
      \fi
      \x{alphalph}{The package is already loaded}%
      \aftergroup\endinput
    \fi
  \fi
\endgroup%
%    \end{macrocode}
%    Package identification:
%    \begin{macrocode}
\begingroup\catcode61\catcode48\catcode32=10\relax%
  \catcode13=5 % ^^M
  \endlinechar=13 %
  \catcode35=6 % #
  \catcode39=12 % '
  \catcode40=12 % (
  \catcode41=12 % )
  \catcode44=12 % ,
  \catcode45=12 % -
  \catcode46=12 % .
  \catcode47=12 % /
  \catcode58=12 % :
  \catcode64=11 % @
  \catcode91=12 % [
  \catcode93=12 % ]
  \catcode123=1 % {
  \catcode125=2 % }
  \expandafter\ifx\csname ProvidesPackage\endcsname\relax
    \def\x#1#2#3[#4]{\endgroup
      \immediate\write-1{Package: #3 #4}%
      \xdef#1{#4}%
    }%
  \else
    \def\x#1#2[#3]{\endgroup
      #2[{#3}]%
      \ifx#1\@undefined
        \xdef#1{#3}%
      \fi
      \ifx#1\relax
        \xdef#1{#3}%
      \fi
    }%
  \fi
\expandafter\x\csname ver@alphalph.sty\endcsname
\ProvidesPackage{alphalph}%
  [2016/05/16 v2.5 Convert numbers to letters (HO)]%
%    \end{macrocode}
%
% \subsection{Catcodes}
%
%    \begin{macrocode}
\begingroup\catcode61\catcode48\catcode32=10\relax%
  \catcode13=5 % ^^M
  \endlinechar=13 %
  \catcode123=1 % {
  \catcode125=2 % }
  \catcode64=11 % @
  \def\x{\endgroup
    \expandafter\edef\csname AlPh@AtEnd\endcsname{%
      \endlinechar=\the\endlinechar\relax
      \catcode13=\the\catcode13\relax
      \catcode32=\the\catcode32\relax
      \catcode35=\the\catcode35\relax
      \catcode61=\the\catcode61\relax
      \catcode64=\the\catcode64\relax
      \catcode123=\the\catcode123\relax
      \catcode125=\the\catcode125\relax
    }%
  }%
\x\catcode61\catcode48\catcode32=10\relax%
\catcode13=5 % ^^M
\endlinechar=13 %
\catcode35=6 % #
\catcode64=11 % @
\catcode123=1 % {
\catcode125=2 % }
\def\TMP@EnsureCode#1#2{%
  \edef\AlPh@AtEnd{%
    \AlPh@AtEnd
    \catcode#1=\the\catcode#1\relax
  }%
  \catcode#1=#2\relax
}
\TMP@EnsureCode{33}{12}% !
\TMP@EnsureCode{39}{12}% '
\TMP@EnsureCode{40}{12}% (
\TMP@EnsureCode{41}{12}% )
\TMP@EnsureCode{43}{12}% +
\TMP@EnsureCode{44}{12}% ,
\TMP@EnsureCode{46}{12}% .
\TMP@EnsureCode{47}{12}% /
\TMP@EnsureCode{59}{12}% ;
\TMP@EnsureCode{60}{12}% <
\TMP@EnsureCode{62}{12}% >
\TMP@EnsureCode{91}{12}% [
\TMP@EnsureCode{93}{12}% ]
\TMP@EnsureCode{96}{12}% `
\TMP@EnsureCode{124}{12}% |
\edef\AlPh@AtEnd{\AlPh@AtEnd\noexpand\endinput}
%    \end{macrocode}
%
% \subsection{Package loading}
%
%    \begin{macrocode}
\begingroup\expandafter\expandafter\expandafter\endgroup
\expandafter\ifx\csname RequirePackage\endcsname\relax
  \input infwarerr.sty\relax
  \input intcalc.sty\relax
\else
  \RequirePackage{infwarerr}[2007/09/09]%
  \RequirePackage{intcalc}[2007/09/09]%
\fi
%    \end{macrocode}
%
% \subsection{\hologo{eTeX} detection}
%
%    \begin{macrocode}
\begingroup\expandafter\expandafter\expandafter\endgroup
\expandafter\ifx\csname numexpr\endcsname\relax
  \catcode124=9 % '!': ignore
  \catcode43=14 % '+': comment
\else
  \catcode124=14 % '!': comment
  \catcode43=9   % '+': ignore
\fi
%    \end{macrocode}
%
% \subsection{Help macros}
%
%    \begin{macro}{\AlPh@Error}
%    \begin{macrocode}
\def\AlPh@Error#1{%
  \begingroup
    \escapechar=92 % backslash
    \@PackageError{alphalph}{#1}\@ehc
  \endgroup
}
%    \end{macrocode}
%    \end{macro}
%
%    \begin{macro}{\AlPh@IfDefinable}
%    \begin{macrocode}
\begingroup\expandafter\expandafter\expandafter\endgroup
\expandafter\ifx\csname @ifdefinable\endcsname\relax
  \def\AlPh@IfDefinable#1#2{%
    \ifcase\ifx#1\@undefined\else\ifx#1\relax\else1\fi\fi0 %
      #2%
    \else
      \AlPh@Error{%
        Command \string#1 already defined%
      }%
    \fi
  }%
%    \end{macrocode}
%    \end{macro}
%    \begin{macrocode}
\else
%    \end{macrocode}
%    \begin{macro}{\AlPh@IfDefinable}
%    \begin{macrocode}
  \let\AlPh@IfDefinable\@ifdefinable
%    \end{macrocode}
%    \end{macro}
%    \begin{macrocode}
\fi
%    \end{macrocode}
%
%    \begin{macro}{\@ReturnAfterElseFi}
%    \begin{macro}{\@ReturnAfterFi}
%    The following commands moves the `then' and `else' part respectively
%    behind the |\if|-construct. This prevents a too deep |\if|-nesting
%    and so a \hologo{TeX} capacity error because of a limited input stack size.
%    I use this trick in several packages, so I don't prefix these internal
%    commands in order not to have the same macros with different names.
%    (It saves memory.)
%    \begin{macrocode}
\long\def\@ReturnAfterElseFi#1\else#2\fi{\fi#1}
\long\def\@ReturnAfterFi#1\fi{\fi#1}
%    \end{macrocode}
%    \end{macro}
%    \end{macro}
%
%    \begin{macro}{\@gobblefour}
%    \Hologo{LaTeX} defines commands for eating arguments.
%    Define \cs{@gobblefour} if it is not defined (\hologo{plainTeX}).
%    \begin{macrocode}
\expandafter\ifx\csname @gobblefour\endcsname\relax
  \long\def\@gobblefour#1#2#3#4{}%
\fi
%    \end{macrocode}
%    \end{macro}
%
%    \begin{macro}{AlPh@IfOptArg}
%    \begin{macrocode}
\begingroup\expandafter\expandafter\expandafter\endgroup
\expandafter\ifx\csname kernel@ifnextchar\endcsname\relax
  \begingroup\expandafter\expandafter\expandafter\endgroup
  \expandafter\ifx\csname @ifnextchar\endcsname\relax
    \def\AlPh@IfOptArg#1#2{%
      \def\AlPh@TempA{#1}%
      \def\AlPh@TempB{#2}%
      \futurelet\AlPh@Token\AlPh@IfOptArgNext
    }%
    \let\AlPh@BracketLeft=[%]
    \def\AlPh@IfOptArgNext{%
      \ifx\AlPh@Token\AlPh@BracketLeft
        \expandafter\AlPh@TempA
      \else
        \expandafter\AlPh@TempB
      \fi
    }%
  \else
    \def\AlPh@IfOptArg{\@ifnextchar[}%]
  \fi
\else
  \def\AlPh@IfOptArg{\kernel@ifnextchar[}%]
\fi
%    \end{macrocode}
%    \end{macro}
%
% \subsection{Symbol provider}
%
% \subsubsection{Alphabet}
%
%    The output of \cs{alphalph} and \cs{AlphAlph} should be
%    usable as part of command names (see \cs{@namedef}, \cs{csname}, \dots).
%    Unhappily some languages of package \xpackage{babel}
%    redefine \hologo{LaTeX}'s \cs{@alph} and \cs{@Alph} in a manner that
%    they cannot be used in expandable context any more.
%    Therefore package \xpackage{alphalph} provides its own
%    commands.
%
%    \begin{macro}{\alphalph@Alph}
%    \begin{macro}{\alphalph@alph}
%    The two commands \cmd{\AlPh@Alph} and \cmd{\AlPh@alph} convert a
%    number into a letter (uppercase and lowercase respectivly).
%    The character |@| is used as an error symbol,
%    if the number isn't in the range of 1 until 26.
%    Here we need no space after the number |#1|, because the error
%    symbol |@| for the zero case stops scanning the number.
%    This error symbol should not appear anywhere (except for bugs).
%    \begin{macrocode}
\def\alphalph@Alph#1{%
  \ifcase#1%
    @%
  \or A\or B\or C\or D\or E\or F\or G\or H\or I\or J\or K\or L\or M%
  \or N\or O\or P\or Q\or R\or S\or T\or U\or V\or W\or X\or Y\or Z%
  \else
    \AlPh@ctrerr
    @%
  \fi
}
\def\alphalph@alph#1{%
  \ifcase#1%
    @%
  \or a\or b\or c\or d\or e\or f\or g\or h\or i\or j\or k\or l\or m%
  \or n\or o\or p\or q\or r\or s\or t\or u\or v\or w\or x\or y\or z%
  \else
    \AlPh@ctrerr
    @%
  \fi
}
%    \end{macrocode}
%    \end{macro}
%    \end{macro}
%    \begin{macro}{\AlPh@ctrerr}
%    Macro \cs{AlPh@ctrerr} is used as hook for the algorithm
%    to get the available number of symbols.
%    \begin{macrocode}
\def\AlPh@ctrerr{}
%    \end{macrocode}
%    \end{macro}
%
% \subsection{Finding number of symbols}
%
%    \begin{macro}{\AlPh@GetNumberOfSymbols}
%    \noindent
%    |#1|: symbols provider
%    \begin{macrocode}
\def\AlPh@GetNumberOfSymbols#1{%
  \AlPh@TestNumber1!{#1}%
  \ifAlPh@Unavailable
    \def\AlPh@Number{0}%
    \AlPh@Error{No symbols found}%
  \else
    \def\AlPh@Number{1}%
    \AlPh@ExpSearch2!{#1}%
  \fi
}
%    \end{macrocode}
%    \end{macro}
%    \begin{macro}{\ifAlPh@Unavailable}
%    \begin{macrocode}
\let\ifAlPh@Unavailable\iffalse
\def\AlPh@Unavailabletrue{%
  \global\let\ifAlPh@Unavailable\iftrue
}
\def\AlPh@Unavailablefalse{%
  \global\let\ifAlPh@Unavailable\iffalse
}
%    \end{macrocode}
%    \end{macro}
%    \begin{macro}{\AlPh@TestNumber}
%    |#1|: number to be tested\\
%    |#2|: symbols provider
%    \begin{macrocode}
\def\AlPh@TestNumber#1!#2{%
  \AlPh@Unavailablefalse
  \begingroup
    \setbox0=\hbox{%
      \begingroup % color
        \let\@ctrerr\AlPh@Unavailabletrue
        \let\AlPh@ctrerr\AlPh@Unavailabletrue
        #2{#1}%
      \endgroup
    }%
    \ifdim\wd0=0pt %
      \AlPh@Unavailabletrue
    \fi
  \endgroup
}
%    \end{macrocode}
%    \end{macro}
%    \begin{macro}{\AlPh@ExpSearch}
%    |#1|: number to be tested\\
%    |#2|: symbols provider
%    \begin{macrocode}
\def\AlPh@ExpSearch#1!#2{%
  \let\AlPh@Next\relax
  \AlPh@TestNumber#1!{#2}%
  \ifAlPh@Unavailable
    \expandafter\AlPh@BinSearch\AlPh@Number!#1!{#2}%
  \else
    \def\AlPh@Number{#1}%
    \ifnum#1>1073741823 %
      \AlPh@TestNumber2147483647!{#2}%
      \ifAlPh@Unavailable
        \AlPh@BinSearch#1!2147483647!{#2}%
      \else
        \def\AlPh@Number{0}%
        \AlPh@Error{%
          Maximal symbol number not found%
        }%
      \fi
    \else
      \def\AlPh@Next{%
        \expandafter\AlPh@ExpSearch\number\intcalcShl{#1}!{#2}%
      }%
    \fi
  \fi
  \AlPh@Next
}
%    \end{macrocode}
%    \end{macro}
%    \begin{macro}{\AlPh@BinSearch}
%    |#1|: available number\\
%    |#2|: unavailable number, |#2| $>$ |#1|\\
%    |#3|: symbols provider
%    \begin{macrocode}
\def\AlPh@BinSearch#1!#2!#3{%
  \expandafter\AlPh@ProcessBinSearch
  \number\intcalcShr{\intcalcAdd{#1}{#2}}!%
  #1!#2!{#3}%
}
%    \end{macrocode}
%    \end{macro}
%    \begin{macro}{\AlPh@ProcessBinSearch}
%    |#1|: number to be tested, |#2| $\leq$ |#1| $\leq$ |#3|\\
%    |#2|: available number\\
%    |#3|: unavailable number\\
%    |#4|: symbols provider
%    \begin{macrocode}
\def\AlPh@ProcessBinSearch#1!#2!#3!#4{%
  \let\AlPh@Next\relax
  \ifnum#1>#2 %
    \ifnum#1<#3 %
      \AlPh@TestNumber#1!{#4}%
      \ifAlPh@Unavailable
        \def\AlPh@Next{%
          \AlPh@BinSearch#2!#1!{#4}%
        }%
      \else
        \def\AlPh@Next{%
          \AlPh@BinSearch#1!#3!{#4}%
        }%
      \fi
    \else
      \def\AlPh@Number{#2}%
    \fi
  \else
    \def\AlPh@Number{#2}%
  \fi
  \AlPh@Next
}
%    \end{macrocode}
%    \end{macro}
%
% \subsection{Methods}
%
%    The names of method macros start with \cs{AlPh@Method}.
%    These macros do the main job in converting a number to
%    its representation.
%    A method command is called with three arguments.
%    The first argument is the number of symbols. The
%    second argument is the basic macro for converting
%    a number with limited number range. The last parameter
%    is the number that needs converting.
%
% \subsubsection{Common methods}
%
%    \begin{macro}{\AlPh@CheckPositive}
%    \noindent
%    |#1|: number to be checked
%    |#2|: continuation macro\\
%    |#3|: number of symbols (hidden here)\\
%    |#4|: symbol provider (hidden here)
%    \begin{macrocode}
\def\AlPh@CheckPositive#1!#2{%
  \ifnum#1<1 %
    \expandafter\@gobblefour
  \fi
  #2{#1}%
}
%    \end{macrocode}
%    \end{macro}
%
% \subsubsection{Method `alph'}
%
%    \begin{macro}{\AlPh@Method@alph}
%    \noindent
%    |#1|: number of symbols\\
%    |#2|: symbols provider\\
%    |#3|: number to be converted
%    \begin{macrocode}
\def\AlPh@Method@alph#1#2#3{%
  \expandafter\AlPh@CheckPositive
|   \number#3!%
+   \the\numexpr#3!%
    \AlPh@ProcessAlph
    {#1}{#2}%
}
%    \end{macrocode}
%    \end{macro}
%    \begin{macro}{\AlPh@ProcessAlph}
%    |#1|: current number\\
%    |#2|: number of symbols\\
%    |#3|: symbols provider
%    \begin{macrocode}
\def\AlPh@ProcessAlph#1#2#3{%
  \ifnum#1>#2 %
    \@ReturnAfterElseFi{%
      \expandafter\AlPh@StepAlph\number
        \intcalcInc{%
          \intcalcMod{\intcalcDec{#1}}{#2}%
        }%
      \expandafter!\number
        \intcalcDiv{\intcalcDec{#1}}{#2}%
      !{#2}{#3}%
    }%
  \else
    \@ReturnAfterFi{%
      #3{#1}%
    }%
  \fi
}
%    \end{macrocode}
%    \end{macro}
%    \begin{macro}{\AlPh@StepAlph}
%    |#1|: current last digit\\
%    |#2|: new current number\\
%    |#3|: number of symbols\\
%    |#4|: symbols provider
%    \begin{macrocode}
\def\AlPh@StepAlph#1!#2!#3#4{%
  \AlPh@ProcessAlph{#2}{#3}{#4}%
  #4{#1}%
}
%    \end{macrocode}
%    \end{macro}
%
% \subsubsection{Method `wrap'}
%
%    \begin{macro}{\AlPh@Method@wrap}
%    \noindent
%    |#1|: number of symbols\\
%    |#2|: symbols provider\\
%    |#3|: number to be converted
%    \begin{macrocode}
\def\AlPh@Method@wrap#1#2#3{%
  \expandafter\AlPh@CheckPositive
|   \number#3!%
+   \the\numexpr#3!%
    \AlPh@ProcessWrap
    {#1}{#2}%
}
%    \end{macrocode}
%    \end{macro}
%
%    \begin{macro}{\AlPh@ProcessWrap}
%    |#1|: number to be converted\\
%    |#2|: number of symbols\\
%    |#3|: symbols provider
%    \begin{macrocode}
\def\AlPh@ProcessWrap#1#2#3{%
  \ifnum#1>#2 %
    \@ReturnAfterElseFi{%
      \expandafter\AlPh@StepWrap\number
        \intcalcInc{\intcalcMod{\intcalcDec{#1}}{#2}}%
      !{#3}%
    }%
  \else
    \@ReturnAfterFi{%
      #3{#1}%
    }%
  \fi
}
%    \end{macrocode}
%    \end{macro}
%    \begin{macro}{\AlPh@StepWrap}
%    |#1|: final number\\
%    |#2|: symbols provider
%    \begin{macrocode}
\def\AlPh@StepWrap#1!#2{%
  #2{#1}%
}
%    \end{macrocode}
%    \end{macro}
%
% \subsubsection{Method `mult'}
%
%    After the number of symbols is exhausted,
%    repetitions of the symbol are used.
%    \begin{gather*}
%      x \Def \text{number to be converted}\\
%      n \Def \text{number of symbols}\\
%      r \Def \text{repetition length}\\
%      s \Def \text{symbol slot}\\
%      r = ((x - 1) \div n) + 1\\
%      s = ((x - 1) \mod n) + 1
%    \end{gather*}
%    \begin{macro}{\AlPh@Method@mult}
%    \noindent
%    |#1|: number of symbols\\
%    |#2|: symbols provider\\
%    |#3|: number to be converted
%    \begin{macrocode}
\def\AlPh@Method@mult#1#2#3{%
  \expandafter\AlPh@CheckPositive
|   \number#3!%
+   \the\numexpr#3!%
    \AlPh@ProcessMult
    {#1}{#2}%
}
%    \end{macrocode}
%    \end{macro}
%
%    \begin{macro}{\AlPh@ProcessMult}
%    |#1|: number to be converted\\
%    |#2|: number of symbols\\
%    |#3|: symbols provider
%    \begin{macrocode}
\def\AlPh@ProcessMult#1#2#3{%
  \ifnum#1>#2 %
    \@ReturnAfterElseFi{%
      \expandafter\AlPh@StepMult\romannumeral
        \intcalcInc{\intcalcDiv{\intcalcDec{#1}}{#2}}%
        000%
      \expandafter!\number
        \intcalcInc{\intcalcMod{\intcalcDec{#1}}{#2}}%
      !{#3}%
    }%
  \else
    \@ReturnAfterFi{%
      #3{#1}%
    }%
  \fi
}
%    \end{macrocode}
%    \end{macro}
%    \begin{macro}{\AlPh@StepMult}
%    |#1#2|: repetitions coded as list of character `m'\\
%    |#3|: symbol slot\\
%    |#4|: symbols provider
%    \begin{macrocode}
\def\AlPh@StepMult#1#2!#3!#4{%
  \ifx\\#2\\%
  \else
    \@ReturnAfterFi{%
      \AlPh@StepMult#2!#3!{#4}%
    }%
  \fi
  #4{#3}%
}
%    \end{macrocode}
%    \end{macro}
%
% \subsection{User interface}
%
%    \begin{macro}{\newalphalph}
%    Macro \cs{newalphalph} had three arguments in versions below 2.0.
%    For the new method argument we use an optional argument an first
%    position.\\
%    |#1|: cmd\\
%    |[#2]|: method name: |alph| (default), |wrap|, |mult|\\ % hash-ok
%    |#3|: symbols provider\\
%    |#4|: number of symbols
%    \begin{macrocode}
\AlPh@IfDefinable\newalphalph{%
  \def\newalphalph#1{%
    \AlPh@IfOptArg{%
      \AlPh@newalphalph{#1}%
    }{%
      \AlPh@newalphalph{#1}[alph]%
    }%
  }%
}
%    \end{macrocode}
%    \end{macro}
%    \begin{macro}{\AlPh@newalphalph}
%    |#1|: cmd
%    |#2|: method name\\
%    |#3|: symbols provider\\
%    |#4|: number of symbols
%    \begin{macrocode}
\def\AlPh@newalphalph#1[#2]#3#4{%
  \begingroup\expandafter\expandafter\expandafter\endgroup
  \expandafter\ifx\csname AlPh@Method@#2\endcsname\relax
    \AlPh@Error{%
      Unknown method %
|     `#2'%
+     `\detokenize{#2}'%
    }%
  \else
    \ifx\\#4\\%
      \AlPh@GetNumberOfSymbols{#3}%
      \ifcase\AlPh@Number
      \else
        \begingroup
          \escapechar=92 % backslash
          \@PackageInfo{alphalph}{%
            Number of symbols for \string#1 is \AlPh@Number
          }%
        \endgroup
        \expandafter\AlPh@NewAlphAlph
        \csname AlPh@Method@#2\expandafter\endcsname
        \AlPh@Number!{#1}{#3}%
      \fi
    \else
      \expandafter\AlPh@NewAlphAlph
      \csname AlPh@Method@#2\expandafter\endcsname
|     \number#4!%
+     \the\numexpr#4!%
      {#1}{#3}%
    \fi
  \fi
}%
%    \end{macrocode}
%    \end{macro}
%    \begin{macro}{\AlPh@NewAlphAlph}
%    |#1|: method macro\\
%    |#2|: number of symbols\\
%    |#3|: cmd\\
%    |#4|: symbols provider
%    \begin{macrocode}
\def\AlPh@NewAlphAlph#1#2!#3#4{%
  \AlPh@IfDefinable#3{%
    \ifnum#2>0 %
      \def#3{#1{#2}{#4}}%
    \else
      \AlPh@Error{%
          Definition of \string#3 failed,\MessageBreak
          because number of symbols (#2) is not positive%
        }%
    \fi
  }%
}
%    \end{macrocode}
%    \end{macro}
%
%    \begin{macro}{\AlphAlph}
%    \begin{macrocode}
\newalphalph\AlphAlph\alphalph@Alph{26}
%    \end{macrocode}
%    \end{macro}
%    \begin{macro}{\alphalph}
%    \begin{macrocode}
\newalphalph\alphalph\alphalph@alph{26}
%    \end{macrocode}
%    \end{macro}
%
%    \begin{macrocode}
\AlPh@AtEnd%
%</package>
%    \end{macrocode}
%
% \section{Test}
%
% \subsection{Catcode checks for loading}
%
%    \begin{macrocode}
%<*test1>
%    \end{macrocode}
%    \begin{macrocode}
\catcode`\{=1 %
\catcode`\}=2 %
\catcode`\#=6 %
\catcode`\@=11 %
\expandafter\ifx\csname count@\endcsname\relax
  \countdef\count@=255 %
\fi
\expandafter\ifx\csname @gobble\endcsname\relax
  \long\def\@gobble#1{}%
\fi
\expandafter\ifx\csname @firstofone\endcsname\relax
  \long\def\@firstofone#1{#1}%
\fi
\expandafter\ifx\csname loop\endcsname\relax
  \expandafter\@firstofone
\else
  \expandafter\@gobble
\fi
{%
  \def\loop#1\repeat{%
    \def\body{#1}%
    \iterate
  }%
  \def\iterate{%
    \body
      \let\next\iterate
    \else
      \let\next\relax
    \fi
    \next
  }%
  \let\repeat=\fi
}%
\def\RestoreCatcodes{}
\count@=0 %
\loop
  \edef\RestoreCatcodes{%
    \RestoreCatcodes
    \catcode\the\count@=\the\catcode\count@\relax
  }%
\ifnum\count@<255 %
  \advance\count@ 1 %
\repeat

\def\RangeCatcodeInvalid#1#2{%
  \count@=#1\relax
  \loop
    \catcode\count@=15 %
  \ifnum\count@<#2\relax
    \advance\count@ 1 %
  \repeat
}
\def\RangeCatcodeCheck#1#2#3{%
  \count@=#1\relax
  \loop
    \ifnum#3=\catcode\count@
    \else
      \errmessage{%
        Character \the\count@\space
        with wrong catcode \the\catcode\count@\space
        instead of \number#3%
      }%
    \fi
  \ifnum\count@<#2\relax
    \advance\count@ 1 %
  \repeat
}
\def\space{ }
\expandafter\ifx\csname LoadCommand\endcsname\relax
  \def\LoadCommand{\input alphalph.sty\relax}%
\fi
\def\Test{%
  \RangeCatcodeInvalid{0}{47}%
  \RangeCatcodeInvalid{58}{64}%
  \RangeCatcodeInvalid{91}{96}%
  \RangeCatcodeInvalid{123}{255}%
  \catcode`\@=12 %
  \catcode`\\=0 %
  \catcode`\%=14 %
  \LoadCommand
  \RangeCatcodeCheck{0}{36}{15}%
  \RangeCatcodeCheck{37}{37}{14}%
  \RangeCatcodeCheck{38}{47}{15}%
  \RangeCatcodeCheck{48}{57}{12}%
  \RangeCatcodeCheck{58}{63}{15}%
  \RangeCatcodeCheck{64}{64}{12}%
  \RangeCatcodeCheck{65}{90}{11}%
  \RangeCatcodeCheck{91}{91}{15}%
  \RangeCatcodeCheck{92}{92}{0}%
  \RangeCatcodeCheck{93}{96}{15}%
  \RangeCatcodeCheck{97}{122}{11}%
  \RangeCatcodeCheck{123}{255}{15}%
  \RestoreCatcodes
}
\Test
\csname @@end\endcsname
\end
%    \end{macrocode}
%    \begin{macrocode}
%</test1>
%    \end{macrocode}
%
% \section{Macro tests}
%
%    \begin{macrocode}
%<*test2>
\NeedsTeXFormat{LaTeX2e}
\nofiles
\documentclass{article}
%<*noetex>
\makeatletter
\let\saved@numexpr\numexpr
\newcommand*{\DisableNumexpr}{%
  \let\numexpr\@undefined
}
\newcommand*{\RestoreNumexpr}{%
  \let\numexpr\saved@numexpr
}
\DisableNumexpr
%</noetex>
\usepackage{alphalph}[2016/05/16]
%<noetex>\RestoreNumexpr
\usepackage{qstest}
\IncludeTests{*}
\LogTests{log}{*}{*}

\newcommand*{\TestCmd}[3]{%
  \setbox0=\hbox{%
%<noetex>    \DisableNumexpr
    \edef\TestString{#1{#2}}%
    \expandafter\Expect\expandafter{\TestString}{#3}%
    \edef\TestString{#1{#2} }%
    \expandafter\Expect\expandafter{\TestString}{#3 }%
  }%
  \Expect*{\the\wd0}{0.0pt}%
}

\makeatletter
\newalphalph\LaTeXAlphAlph\@Alph{26}
\newalphalph\LaTeXalphalph\@alph{26}
\newalphalph\AlphWrap[wrap]\alphalph@Alph{26}
\newalphalph\alphwrap[wrap]\alphalph@alph{26}
\newalphalph\LaTeXAlphWrap[wrap]\@Alph{26}
\newalphalph\LaTeXalphwrap[wrap]\@alph{26}
\def\LastSymbol#1{%
  \ifx\\#1\\%
  \else
    \@LastSymbol#1\@nil
  \fi
}
\def\@LastSymbol#1#2\@nil{%
  \ifx\\#2\\%
    #1%
  \else
    \@LastSymbol#2\@nil
  \fi
}
\makeatother
\newcommand*{\TestAlph}[2]{%
  \uppercase{\TestCallCmd\AlphAlph{#2}}{#1}%
  \lowercase{\TestCallCmd\alphalph{#2}}{#1}%
  \uppercase{\TestCallCmd\LaTeXAlphAlph{#2}}{#1}%
  \lowercase{\TestCallCmd\LaTeXalphalph{#2}}{#1}%
  \edef\WrapString{\LastSymbol{#2}}%
  \expandafter\TestAlphWrap\expandafter{\WrapString}{#1}%
}
\newcommand*{\TestAlphWrap}[2]{%
  \uppercase{\TestCallCmd\AlphWrap{#1}}{#2}%
  \lowercase{\TestCallCmd\alphwrap{#1}}{#2}%
  \uppercase{\TestCallCmd\LaTeXAlphWrap{#1}}{#2}%
  \lowercase{\TestCallCmd\LaTeXalphwrap{#1}}{#2}%
}
\newcommand*{\TestCallCmd}[3]{%
  \TestCmd#1{#3}{#2}%
}
\begin{qstest}{AlphSymbols}{alphalph, AlphAlph, symbols}
  \TestAlph{1}{a}%
  \TestAlph{2}{b}%
  \TestAlph{3}{c}%
  \TestAlph{4}{d}%
  \TestAlph{5}{e}%
  \TestAlph{6}{f}%
  \TestAlph{7}{g}%
  \TestAlph{8}{h}%
  \TestAlph{9}{i}%
  \TestAlph{10}{j}%
  \TestAlph{11}{k}%
  \TestAlph{12}{l}%
  \TestAlph{13}{m}%
  \TestAlph{14}{n}%
  \TestAlph{15}{o}%
  \TestAlph{16}{p}%
  \TestAlph{17}{q}%
  \TestAlph{18}{r}%
  \TestAlph{19}{s}%
  \TestAlph{20}{t}%
  \TestAlph{21}{u}%
  \TestAlph{22}{v}%
  \TestAlph{23}{w}%
  \TestAlph{24}{x}%
  \TestAlph{25}{y}%
  \TestAlph{26}{z}%
\end{qstest}
\begin{qstest}{AlphRange}{alphalph, range}
  \TestAlph{0}{}%
  \TestAlph{-1}{}%
  \TestAlph{-2147483647}{}%
  \TestAlph{27}{aa}%
  \TestAlph{28}{ab}%
  \TestAlph{52}{az}%
  \TestAlph{53}{ba}%
  \TestAlph{78}{bz}%
  \TestAlph{79}{ca}%
  \TestAlph{702}{zz}%
  \TestAlph{703}{aaa}%
  \TestAlph{2147483647}{fxshrxw}%
\end{qstest}

\makeatletter
\newcommand*{\myvocals}[1]{%
  \ifcase#1X\or A\or E\or I\or O\or U\else Y\fi
}
\makeatother
\newalphalph\vocalsvocals\myvocals{5}
\newcommand*{\TestVocals}{%
  \TestCmd\vocalsvocals
}
\begin{qstest}{vocals}{vocals}
  \TestVocals{0}{}%
  \TestVocals{1}{A}%
  \TestVocals{2}{E}%
  \TestVocals{3}{I}%
  \TestVocals{4}{O}%
  \TestVocals{5}{U}%
  \TestVocals{6}{AA}%
  \TestVocals{7}{AE}%
  \TestVocals{8}{AI}%
  \TestVocals{9}{AO}%
  \TestVocals{10}{AU}%
  \TestVocals{11}{EA}%
  \TestVocals{24}{OO}%
  \TestVocals{25}{OU}%
  \TestVocals{26}{UA}%
  \TestVocals{29}{UO}%
  \TestVocals{30}{UU}%
  \TestVocals{31}{AAA}%
  \TestVocals{155}{UUU}%
  \TestVocals{156}{AAAA}%
  \TestVocals{2147483647}{AIIOOEEIOIIUOE}%
\end{qstest}

\makeatletter
\newalphalph\AlphMult[mult]{\alphalph@Alph}{26}
\newalphalph\alphmult[mult]{\alphalph@alph}{26}
\newalphalph\LaTeXAlphMult[mult]{\@Alph}{26}
\newalphalph\LaTeXalphmult[mult]{\@alph}{26}
\makeatother
\newcommand*{\TestMult}[2]{%
  \uppercase{\TestCallCmd\AlphMult{#2}}{#1}%
  \lowercase{\TestCallCmd\alphmult{#2}}{#1}%
  \uppercase{\TestCallCmd\LaTeXAlphMult{#2}}{#1}%
  \lowercase{\TestCallCmd\LaTeXalphmult{#2}}{#1}%
}
\begin{qstest}{mult}{mult}
  \TestMult{0}{}%
  \TestMult{-1}{}%
  \TestMult{-2147483647}{}%
  \TestMult{1}{a}%
  \TestMult{2}{b}%
  \TestMult{26}{z}%
  \TestMult{27}{aa}%
  \TestMult{28}{bb}%
  \TestMult{52}{zz}%
  \TestMult{53}{aaa}%
  \TestMult{54}{bbb}%
  \TestMult{259}{yyyyyyyyyy}%
  \TestMult{260}{zzzzzzzzzz}%
  \TestMult{261}{aaaaaaaaaaa}%
  \TestMult{262}{bbbbbbbbbbb}%
\end{qstest}

\def\myvocalsB#1{%
  \ifcase#1\or A\or E\or I\or O\or U\fi
}
\begin{qstest}{symbolnum}{symbolnum}
  \makeatletter
  \def\Test#1#2{%
    \let\TestCmd\relax
    \newalphalph\TestCmd{#1}{}%
    \Expect*{\AlPh@Number}{#2}%
  }%
  \Test\@alph{26}%
  \Test\@Alph{26}%
  \Test\@fnsymbol{9}%
  \Test\myvocalsB{5}%
  \Test\alphalph@alph{26}%
  \Test\alphalph@Alph{26}%
\end{qstest}

\begin{qstest}{list}{list}
  \makeatletter
  \def\catch#1\relax{%
    \def\FoundList{\catch#1}%
  }%
  \def\Test[#1]#2#3#4{%
    \let\testcmd\relax
    \newalphalph\testcmd[{#1}]{\catch}{#2}%
    \testcmd{#3}|\relax
    \expandafter\Expect\expandafter{\FoundList}{#4|}%
    %
    \let\SavedCatch\catch
    \def\catch{\noexpand\catch\noexpand\foo}%
    \edef\Result{#4|}%
    \@onelevel@sanitize\Result
    \let\catch\SavedCatch
    \let\testcmd\relax
    \newalphalph\testcmd[{#1}]{\catch\foo}{#2}%
    \testcmd{#3}|\relax
    \@onelevel@sanitize\FoundList
    \Expect*{\FoundList}*{\Result}%
  }%
  \Test[alph]{26}{3}{\catch{3}}%
  \Test[alph]{26}{12}{\catch{12}}%
  \Test[alph]{26}{27}{\catch{1}\catch{1}}%
  \Test[alph]{26}{78}{\catch{2}\catch{26}}%
  \Test[wrap]{26}{7}{\catch{7}}%
  \Test[wrap]{26}{14}{\catch{14}}%
  \Test[wrap]{26}{80}{\catch{2}}%
  \Test[wrap]{26}{700}{\catch{24}}%
  \Test[mult]{26}{4}{\catch{4}}%
  \Test[mult]{26}{17}{\catch{17}}%
  \Test[mult]{26}{54}{\catch{2}\catch{2}\catch{2}}%
\end{qstest}

\begin{document}
\end{document}
%</test2>
%    \end{macrocode}
%
% \section{Installation}
%
% \subsection{Download}
%
% \paragraph{Package.} This package is available on
% CTAN\footnote{\CTANpkg{alphalph}}:
% \begin{description}
% \item[\CTAN{macros/latex/contrib/oberdiek/alphalph.dtx}] The source file.
% \item[\CTAN{macros/latex/contrib/oberdiek/alphalph.pdf}] Documentation.
% \end{description}
%
%
% \paragraph{Bundle.} All the packages of the bundle `oberdiek'
% are also available in a TDS compliant ZIP archive. There
% the packages are already unpacked and the documentation files
% are generated. The files and directories obey the TDS standard.
% \begin{description}
% \item[\CTANinstall{install/macros/latex/contrib/oberdiek.tds.zip}]
% \end{description}
% \emph{TDS} refers to the standard ``A Directory Structure
% for \TeX\ Files'' (\CTAN{tds/tds.pdf}). Directories
% with \xfile{texmf} in their name are usually organized this way.
%
% \subsection{Bundle installation}
%
% \paragraph{Unpacking.} Unpack the \xfile{oberdiek.tds.zip} in the
% TDS tree (also known as \xfile{texmf} tree) of your choice.
% Example (linux):
% \begin{quote}
%   |unzip oberdiek.tds.zip -d ~/texmf|
% \end{quote}
%
% \paragraph{Script installation.}
% Check the directory \xfile{TDS:scripts/oberdiek/} for
% scripts that need further installation steps.
% Package \xpackage{attachfile2} comes with the Perl script
% \xfile{pdfatfi.pl} that should be installed in such a way
% that it can be called as \texttt{pdfatfi}.
% Example (linux):
% \begin{quote}
%   |chmod +x scripts/oberdiek/pdfatfi.pl|\\
%   |cp scripts/oberdiek/pdfatfi.pl /usr/local/bin/|
% \end{quote}
%
% \subsection{Package installation}
%
% \paragraph{Unpacking.} The \xfile{.dtx} file is a self-extracting
% \docstrip\ archive. The files are extracted by running the
% \xfile{.dtx} through \plainTeX:
% \begin{quote}
%   \verb|tex alphalph.dtx|
% \end{quote}
%
% \paragraph{TDS.} Now the different files must be moved into
% the different directories in your installation TDS tree
% (also known as \xfile{texmf} tree):
% \begin{quote}
% \def\t{^^A
% \begin{tabular}{@{}>{\ttfamily}l@{ $\rightarrow$ }>{\ttfamily}l@{}}
%   alphalph.sty & tex/generic/oberdiek/alphalph.sty\\
%   alphalph.pdf & doc/latex/oberdiek/alphalph.pdf\\
%   test/alphalph-test1.tex & doc/latex/oberdiek/test/alphalph-test1.tex\\
%   test/alphalph-test2.tex & doc/latex/oberdiek/test/alphalph-test2.tex\\
%   test/alphalph-test3.tex & doc/latex/oberdiek/test/alphalph-test3.tex\\
%   alphalph.dtx & source/latex/oberdiek/alphalph.dtx\\
% \end{tabular}^^A
% }^^A
% \sbox0{\t}^^A
% \ifdim\wd0>\linewidth
%   \begingroup
%     \advance\linewidth by\leftmargin
%     \advance\linewidth by\rightmargin
%   \edef\x{\endgroup
%     \def\noexpand\lw{\the\linewidth}^^A
%   }\x
%   \def\lwbox{^^A
%     \leavevmode
%     \hbox to \linewidth{^^A
%       \kern-\leftmargin\relax
%       \hss
%       \usebox0
%       \hss
%       \kern-\rightmargin\relax
%     }^^A
%   }^^A
%   \ifdim\wd0>\lw
%     \sbox0{\small\t}^^A
%     \ifdim\wd0>\linewidth
%       \ifdim\wd0>\lw
%         \sbox0{\footnotesize\t}^^A
%         \ifdim\wd0>\linewidth
%           \ifdim\wd0>\lw
%             \sbox0{\scriptsize\t}^^A
%             \ifdim\wd0>\linewidth
%               \ifdim\wd0>\lw
%                 \sbox0{\tiny\t}^^A
%                 \ifdim\wd0>\linewidth
%                   \lwbox
%                 \else
%                   \usebox0
%                 \fi
%               \else
%                 \lwbox
%               \fi
%             \else
%               \usebox0
%             \fi
%           \else
%             \lwbox
%           \fi
%         \else
%           \usebox0
%         \fi
%       \else
%         \lwbox
%       \fi
%     \else
%       \usebox0
%     \fi
%   \else
%     \lwbox
%   \fi
% \else
%   \usebox0
% \fi
% \end{quote}
% If you have a \xfile{docstrip.cfg} that configures and enables \docstrip's
% TDS installing feature, then some files can already be in the right
% place, see the documentation of \docstrip.
%
% \subsection{Refresh file name databases}
%
% If your \TeX~distribution
% (\teTeX, \mikTeX, \dots) relies on file name databases, you must refresh
% these. For example, \teTeX\ users run \verb|texhash| or
% \verb|mktexlsr|.
%
% \subsection{Some details for the interested}
%
% \paragraph{Attached source.}
%
% The PDF documentation on CTAN also includes the
% \xfile{.dtx} source file. It can be extracted by
% AcrobatReader 6 or higher. Another option is \textsf{pdftk},
% e.g. unpack the file into the current directory:
% \begin{quote}
%   \verb|pdftk alphalph.pdf unpack_files output .|
% \end{quote}
%
% \paragraph{Unpacking with \LaTeX.}
% The \xfile{.dtx} chooses its action depending on the format:
% \begin{description}
% \item[\plainTeX:] Run \docstrip\ and extract the files.
% \item[\LaTeX:] Generate the documentation.
% \end{description}
% If you insist on using \LaTeX\ for \docstrip\ (really,
% \docstrip\ does not need \LaTeX), then inform the autodetect routine
% about your intention:
% \begin{quote}
%   \verb|latex \let\install=y\input{alphalph.dtx}|
% \end{quote}
% Do not forget to quote the argument according to the demands
% of your shell.
%
% \paragraph{Generating the documentation.}
% You can use both the \xfile{.dtx} or the \xfile{.drv} to generate
% the documentation. The process can be configured by the
% configuration file \xfile{ltxdoc.cfg}. For instance, put this
% line into this file, if you want to have A4 as paper format:
% \begin{quote}
%   \verb|\PassOptionsToClass{a4paper}{article}|
% \end{quote}
% An example follows how to generate the
% documentation with pdf\LaTeX:
% \begin{quote}
%\begin{verbatim}
%pdflatex alphalph.dtx
%makeindex -s gind.ist alphalph.idx
%pdflatex alphalph.dtx
%makeindex -s gind.ist alphalph.idx
%pdflatex alphalph.dtx
%\end{verbatim}
% \end{quote}
%
% \begin{History}
%   \begin{Version}{1999/03/19 v0.1}
%   \item
%     The first version was built as a response to a
%     \URL{\link{question}}^^A
%     {https://groups.google.com/group/comp.text.tex/msg/17a74cd721641038}^^A
%     of \NameEmail{Will Douglas}{william.douglas@wolfson.ox.ac.uk}
%     and the
%     \URL{\link{request}}^^A
%     {https://groups.google.com/group/comp.text.tex/msg/8f9768825640315f}^^A
%     of \NameEmail{Donald Arsenau}{asnd@reg.triumf.ca},
%     published in the newsgroup
%     \href{news:comp.text.tex}{comp.text.tex}:
%     \URL{``\link{Re: alph counters \textgreater\ 26}''}^^A
%     {https://groups.google.com/group/comp.text.tex/msg/cec563eef8bf65d0}
%   \item
%     Copyright: LPPL (\CTAN{macros/latex/base/lppl.txt})
%   \end{Version}
%   \begin{Version}{1999/04/12 v1.0}
%   \item
%     Documentation added in dtx format.
%   \item
%     \hologo{eTeX} support added.
%   \end{Version}
%   \begin{Version}{1999/04/13 v1.1}
%   \item
%     Minor documentation change.
%   \item
%     First CTAN release.
%   \end{Version}
%   \begin{Version}{1999/06/26 v1.2}
%   \item
%     First generic code about \cmd{\ProvidesPackage} improved.
%   \item
%     Documentation: Installation part revised.
%   \end{Version}
%   \begin{Version}{2006/02/20 v1.3}
%   \item
%     Reload check (for \hologo{plainTeX})
%   \item
%     New DTX framework.
%   \item
%     LPPL 1.3
%   \end{Version}
%   \begin{Version}{2006/05/30 v1.4}
%   \item
%     \cs{newalphalph} added.
%   \end{Version}
%   \begin{Version}{2007/04/11 v1.5}
%   \item
%     Line ends sanitized.
%   \end{Version}
%   \begin{Version}{2007/09/09 v2.0}
%   \item
%     New implementation that uses package \cs{intcalc}.
%     This removes the dependency on \hologo{eTeX}.
%   \item
%     \cs{newalphalph} is extended to support new methods `wrap' and 'multi'.
%   \item
%     Documentation rewritten.
%   \end{Version}
%   \begin{Version}{2008/08/11 v2.1}
%   \item
%     Code is not changed.
%   \item
%     URLs updated from \texttt{www.dejanews.com}
%     to \texttt{groups.google.com}.
%   \end{Version}
%   \begin{Version}{2010/03/01 v2.2}
%   \item
%     Compatibility with \hologo{iniTeX}.
%   \end{Version}
%   \begin{Version}{2010/04/18 v2.3}
%   \item
%     Documentation fixes (Martin M\"unch).
%   \end{Version}
%   \begin{Version}{2011/05/13 v2.4}
%   \item
%     Documentation fixes (Jim Diamond) and using package
%     \xpackage{hologo} for the documentation.
%   \item
%     Catalogue file added.
%   \end{Version}
%   \begin{Version}{2016/05/16 v2.5}
%   \item
%     Documentation updates.
%   \end{Version}
% \end{History}
%
% \PrintIndex
%
% \Finale
\endinput
|
% \end{quote}
% Do not forget to quote the argument according to the demands
% of your shell.
%
% \paragraph{Generating the documentation.}
% You can use both the \xfile{.dtx} or the \xfile{.drv} to generate
% the documentation. The process can be configured by the
% configuration file \xfile{ltxdoc.cfg}. For instance, put this
% line into this file, if you want to have A4 as paper format:
% \begin{quote}
%   \verb|\PassOptionsToClass{a4paper}{article}|
% \end{quote}
% An example follows how to generate the
% documentation with pdf\LaTeX:
% \begin{quote}
%\begin{verbatim}
%pdflatex alphalph.dtx
%makeindex -s gind.ist alphalph.idx
%pdflatex alphalph.dtx
%makeindex -s gind.ist alphalph.idx
%pdflatex alphalph.dtx
%\end{verbatim}
% \end{quote}
%
% \begin{History}
%   \begin{Version}{1999/03/19 v0.1}
%   \item
%     The first version was built as a response to a
%     \URL{\link{question}}^^A
%     {https://groups.google.com/group/comp.text.tex/msg/17a74cd721641038}^^A
%     of \NameEmail{Will Douglas}{william.douglas@wolfson.ox.ac.uk}
%     and the
%     \URL{\link{request}}^^A
%     {https://groups.google.com/group/comp.text.tex/msg/8f9768825640315f}^^A
%     of \NameEmail{Donald Arsenau}{asnd@reg.triumf.ca},
%     published in the newsgroup
%     \href{news:comp.text.tex}{comp.text.tex}:
%     \URL{``\link{Re: alph counters \textgreater\ 26}''}^^A
%     {https://groups.google.com/group/comp.text.tex/msg/cec563eef8bf65d0}
%   \item
%     Copyright: LPPL (\CTAN{macros/latex/base/lppl.txt})
%   \end{Version}
%   \begin{Version}{1999/04/12 v1.0}
%   \item
%     Documentation added in dtx format.
%   \item
%     \hologo{eTeX} support added.
%   \end{Version}
%   \begin{Version}{1999/04/13 v1.1}
%   \item
%     Minor documentation change.
%   \item
%     First CTAN release.
%   \end{Version}
%   \begin{Version}{1999/06/26 v1.2}
%   \item
%     First generic code about \cmd{\ProvidesPackage} improved.
%   \item
%     Documentation: Installation part revised.
%   \end{Version}
%   \begin{Version}{2006/02/20 v1.3}
%   \item
%     Reload check (for \hologo{plainTeX})
%   \item
%     New DTX framework.
%   \item
%     LPPL 1.3
%   \end{Version}
%   \begin{Version}{2006/05/30 v1.4}
%   \item
%     \cs{newalphalph} added.
%   \end{Version}
%   \begin{Version}{2007/04/11 v1.5}
%   \item
%     Line ends sanitized.
%   \end{Version}
%   \begin{Version}{2007/09/09 v2.0}
%   \item
%     New implementation that uses package \cs{intcalc}.
%     This removes the dependency on \hologo{eTeX}.
%   \item
%     \cs{newalphalph} is extended to support new methods `wrap' and 'multi'.
%   \item
%     Documentation rewritten.
%   \end{Version}
%   \begin{Version}{2008/08/11 v2.1}
%   \item
%     Code is not changed.
%   \item
%     URLs updated from \texttt{www.dejanews.com}
%     to \texttt{groups.google.com}.
%   \end{Version}
%   \begin{Version}{2010/03/01 v2.2}
%   \item
%     Compatibility with \hologo{iniTeX}.
%   \end{Version}
%   \begin{Version}{2010/04/18 v2.3}
%   \item
%     Documentation fixes (Martin M\"unch).
%   \end{Version}
%   \begin{Version}{2011/05/13 v2.4}
%   \item
%     Documentation fixes (Jim Diamond) and using package
%     \xpackage{hologo} for the documentation.
%   \item
%     Catalogue file added.
%   \end{Version}
%   \begin{Version}{2016/05/16 v2.5}
%   \item
%     Documentation updates.
%   \end{Version}
% \end{History}
%
% \PrintIndex
%
% \Finale
\endinput
|
% \end{quote}
% Do not forget to quote the argument according to the demands
% of your shell.
%
% \paragraph{Generating the documentation.}
% You can use both the \xfile{.dtx} or the \xfile{.drv} to generate
% the documentation. The process can be configured by the
% configuration file \xfile{ltxdoc.cfg}. For instance, put this
% line into this file, if you want to have A4 as paper format:
% \begin{quote}
%   \verb|\PassOptionsToClass{a4paper}{article}|
% \end{quote}
% An example follows how to generate the
% documentation with pdf\LaTeX:
% \begin{quote}
%\begin{verbatim}
%pdflatex alphalph.dtx
%makeindex -s gind.ist alphalph.idx
%pdflatex alphalph.dtx
%makeindex -s gind.ist alphalph.idx
%pdflatex alphalph.dtx
%\end{verbatim}
% \end{quote}
%
% \begin{History}
%   \begin{Version}{1999/03/19 v0.1}
%   \item
%     The first version was built as a response to a
%     \URL{\link{question}}^^A
%     {https://groups.google.com/group/comp.text.tex/msg/17a74cd721641038}^^A
%     of \NameEmail{Will Douglas}{william.douglas@wolfson.ox.ac.uk}
%     and the
%     \URL{\link{request}}^^A
%     {https://groups.google.com/group/comp.text.tex/msg/8f9768825640315f}^^A
%     of \NameEmail{Donald Arsenau}{asnd@reg.triumf.ca},
%     published in the newsgroup
%     \href{news:comp.text.tex}{comp.text.tex}:
%     \URL{``\link{Re: alph counters \textgreater\ 26}''}^^A
%     {https://groups.google.com/group/comp.text.tex/msg/cec563eef8bf65d0}
%   \item
%     Copyright: LPPL (\CTAN{macros/latex/base/lppl.txt})
%   \end{Version}
%   \begin{Version}{1999/04/12 v1.0}
%   \item
%     Documentation added in dtx format.
%   \item
%     \hologo{eTeX} support added.
%   \end{Version}
%   \begin{Version}{1999/04/13 v1.1}
%   \item
%     Minor documentation change.
%   \item
%     First CTAN release.
%   \end{Version}
%   \begin{Version}{1999/06/26 v1.2}
%   \item
%     First generic code about \cmd{\ProvidesPackage} improved.
%   \item
%     Documentation: Installation part revised.
%   \end{Version}
%   \begin{Version}{2006/02/20 v1.3}
%   \item
%     Reload check (for \hologo{plainTeX})
%   \item
%     New DTX framework.
%   \item
%     LPPL 1.3
%   \end{Version}
%   \begin{Version}{2006/05/30 v1.4}
%   \item
%     \cs{newalphalph} added.
%   \end{Version}
%   \begin{Version}{2007/04/11 v1.5}
%   \item
%     Line ends sanitized.
%   \end{Version}
%   \begin{Version}{2007/09/09 v2.0}
%   \item
%     New implementation that uses package \cs{intcalc}.
%     This removes the dependency on \hologo{eTeX}.
%   \item
%     \cs{newalphalph} is extended to support new methods `wrap' and 'multi'.
%   \item
%     Documentation rewritten.
%   \end{Version}
%   \begin{Version}{2008/08/11 v2.1}
%   \item
%     Code is not changed.
%   \item
%     URLs updated from \texttt{www.dejanews.com}
%     to \texttt{groups.google.com}.
%   \end{Version}
%   \begin{Version}{2010/03/01 v2.2}
%   \item
%     Compatibility with \hologo{iniTeX}.
%   \end{Version}
%   \begin{Version}{2010/04/18 v2.3}
%   \item
%     Documentation fixes (Martin M\"unch).
%   \end{Version}
%   \begin{Version}{2011/05/13 v2.4}
%   \item
%     Documentation fixes (Jim Diamond) and using package
%     \xpackage{hologo} for the documentation.
%   \item
%     Catalogue file added.
%   \end{Version}
%   \begin{Version}{2016/05/16 v2.5}
%   \item
%     Documentation updates.
%   \end{Version}
% \end{History}
%
% \PrintIndex
%
% \Finale
\endinput
|
% \end{quote}
% Do not forget to quote the argument according to the demands
% of your shell.
%
% \paragraph{Generating the documentation.}
% You can use both the \xfile{.dtx} or the \xfile{.drv} to generate
% the documentation. The process can be configured by the
% configuration file \xfile{ltxdoc.cfg}. For instance, put this
% line into this file, if you want to have A4 as paper format:
% \begin{quote}
%   \verb|\PassOptionsToClass{a4paper}{article}|
% \end{quote}
% An example follows how to generate the
% documentation with pdf\LaTeX:
% \begin{quote}
%\begin{verbatim}
%pdflatex alphalph.dtx
%makeindex -s gind.ist alphalph.idx
%pdflatex alphalph.dtx
%makeindex -s gind.ist alphalph.idx
%pdflatex alphalph.dtx
%\end{verbatim}
% \end{quote}
%
% \begin{History}
%   \begin{Version}{1999/03/19 v0.1}
%   \item
%     The first version was built as a response to a
%     \URL{\link{question}}^^A
%     {https://groups.google.com/group/comp.text.tex/msg/17a74cd721641038}^^A
%     of \NameEmail{Will Douglas}{william.douglas@wolfson.ox.ac.uk}
%     and the
%     \URL{\link{request}}^^A
%     {https://groups.google.com/group/comp.text.tex/msg/8f9768825640315f}^^A
%     of \NameEmail{Donald Arsenau}{asnd@reg.triumf.ca},
%     published in the newsgroup
%     \href{news:comp.text.tex}{comp.text.tex}:
%     \URL{``\link{Re: alph counters \textgreater\ 26}''}^^A
%     {https://groups.google.com/group/comp.text.tex/msg/cec563eef8bf65d0}
%   \item
%     Copyright: LPPL (\CTAN{macros/latex/base/lppl.txt})
%   \end{Version}
%   \begin{Version}{1999/04/12 v1.0}
%   \item
%     Documentation added in dtx format.
%   \item
%     \hologo{eTeX} support added.
%   \end{Version}
%   \begin{Version}{1999/04/13 v1.1}
%   \item
%     Minor documentation change.
%   \item
%     First CTAN release.
%   \end{Version}
%   \begin{Version}{1999/06/26 v1.2}
%   \item
%     First generic code about \cmd{\ProvidesPackage} improved.
%   \item
%     Documentation: Installation part revised.
%   \end{Version}
%   \begin{Version}{2006/02/20 v1.3}
%   \item
%     Reload check (for \hologo{plainTeX})
%   \item
%     New DTX framework.
%   \item
%     LPPL 1.3
%   \end{Version}
%   \begin{Version}{2006/05/30 v1.4}
%   \item
%     \cs{newalphalph} added.
%   \end{Version}
%   \begin{Version}{2007/04/11 v1.5}
%   \item
%     Line ends sanitized.
%   \end{Version}
%   \begin{Version}{2007/09/09 v2.0}
%   \item
%     New implementation that uses package \cs{intcalc}.
%     This removes the dependency on \hologo{eTeX}.
%   \item
%     \cs{newalphalph} is extended to support new methods `wrap' and 'multi'.
%   \item
%     Documentation rewritten.
%   \end{Version}
%   \begin{Version}{2008/08/11 v2.1}
%   \item
%     Code is not changed.
%   \item
%     URLs updated from \texttt{www.dejanews.com}
%     to \texttt{groups.google.com}.
%   \end{Version}
%   \begin{Version}{2010/03/01 v2.2}
%   \item
%     Compatibility with \hologo{iniTeX}.
%   \end{Version}
%   \begin{Version}{2010/04/18 v2.3}
%   \item
%     Documentation fixes (Martin M\"unch).
%   \end{Version}
%   \begin{Version}{2011/05/13 v2.4}
%   \item
%     Documentation fixes (Jim Diamond) and using package
%     \xpackage{hologo} for the documentation.
%   \item
%     Catalogue file added.
%   \end{Version}
%   \begin{Version}{2016/05/16 v2.5}
%   \item
%     Documentation updates.
%   \end{Version}
% \end{History}
%
% \PrintIndex
%
% \Finale
\endinput
