% \iffalse meta-comment
%
% File: catchfile.dtx
% Version: 2016/05/16 v1.7
% Info: Catch the contents of a file
%
% Copyright (C) 2007, 2010, 2011 by
%    Heiko Oberdiek <heiko.oberdiek at googlemail.com>
%    2016
%    https://github.com/ho-tex/oberdiek/issues
%
% This work may be distributed and/or modified under the
% conditions of the LaTeX Project Public License, either
% version 1.3c of this license or (at your option) any later
% version. This version of this license is in
%    https://www.latex-project.org/lppl/lppl-1-3c.txt
% and the latest version of this license is in
%    https://www.latex-project.org/lppl.txt
% and version 1.3 or later is part of all distributions of
% LaTeX version 2005/12/01 or later.
%
% This work has the LPPL maintenance status "maintained".
%
% The Current Maintainers of this work are
% Heiko Oberdiek and the Oberdiek Package Support Group
% https://github.com/ho-tex/oberdiek/issues
%
% The Base Interpreter refers to any `TeX-Format',
% because some files are installed in TDS:tex/generic//.
%
% This work consists of the main source file catchfile.dtx
% and the derived files
%    catchfile.sty, catchfile.pdf, catchfile.ins, catchfile.drv,
%    catchfile-test1.tex, catchfile-test2.tex,
%    catchfile-test3.tex.
%
% Distribution:
%    CTAN:macros/latex/contrib/oberdiek/catchfile.dtx
%    CTAN:macros/latex/contrib/oberdiek/catchfile.pdf
%
% Unpacking:
%    (a) If catchfile.ins is present:
%           tex catchfile.ins
%    (b) Without catchfile.ins:
%           tex catchfile.dtx
%    (c) If you insist on using LaTeX
%           latex \let\install=y% \iffalse meta-comment
%
% File: catchfile.dtx
% Version: 2016/05/16 v1.7
% Info: Catch the contents of a file
%
% Copyright (C) 2007, 2010, 2011 by
%    Heiko Oberdiek <heiko.oberdiek at googlemail.com>
%    2016
%    https://github.com/ho-tex/oberdiek/issues
%
% This work may be distributed and/or modified under the
% conditions of the LaTeX Project Public License, either
% version 1.3c of this license or (at your option) any later
% version. This version of this license is in
%    http://www.latex-project.org/lppl/lppl-1-3c.txt
% and the latest version of this license is in
%    http://www.latex-project.org/lppl.txt
% and version 1.3 or later is part of all distributions of
% LaTeX version 2005/12/01 or later.
%
% This work has the LPPL maintenance status "maintained".
%
% This Current Maintainer of this work is Heiko Oberdiek.
%
% The Base Interpreter refers to any `TeX-Format',
% because some files are installed in TDS:tex/generic//.
%
% This work consists of the main source file catchfile.dtx
% and the derived files
%    catchfile.sty, catchfile.pdf, catchfile.ins, catchfile.drv,
%    catchfile-test1.tex, catchfile-test2.tex,
%    catchfile-test3.tex.
%
% Distribution:
%    CTAN:macros/latex/contrib/oberdiek/catchfile.dtx
%    CTAN:macros/latex/contrib/oberdiek/catchfile.pdf
%
% Unpacking:
%    (a) If catchfile.ins is present:
%           tex catchfile.ins
%    (b) Without catchfile.ins:
%           tex catchfile.dtx
%    (c) If you insist on using LaTeX
%           latex \let\install=y% \iffalse meta-comment
%
% File: catchfile.dtx
% Version: 2016/05/16 v1.7
% Info: Catch the contents of a file
%
% Copyright (C) 2007, 2010, 2011 by
%    Heiko Oberdiek <heiko.oberdiek at googlemail.com>
%    2016
%    https://github.com/ho-tex/oberdiek/issues
%
% This work may be distributed and/or modified under the
% conditions of the LaTeX Project Public License, either
% version 1.3c of this license or (at your option) any later
% version. This version of this license is in
%    http://www.latex-project.org/lppl/lppl-1-3c.txt
% and the latest version of this license is in
%    http://www.latex-project.org/lppl.txt
% and version 1.3 or later is part of all distributions of
% LaTeX version 2005/12/01 or later.
%
% This work has the LPPL maintenance status "maintained".
%
% This Current Maintainer of this work is Heiko Oberdiek.
%
% The Base Interpreter refers to any `TeX-Format',
% because some files are installed in TDS:tex/generic//.
%
% This work consists of the main source file catchfile.dtx
% and the derived files
%    catchfile.sty, catchfile.pdf, catchfile.ins, catchfile.drv,
%    catchfile-test1.tex, catchfile-test2.tex,
%    catchfile-test3.tex.
%
% Distribution:
%    CTAN:macros/latex/contrib/oberdiek/catchfile.dtx
%    CTAN:macros/latex/contrib/oberdiek/catchfile.pdf
%
% Unpacking:
%    (a) If catchfile.ins is present:
%           tex catchfile.ins
%    (b) Without catchfile.ins:
%           tex catchfile.dtx
%    (c) If you insist on using LaTeX
%           latex \let\install=y% \iffalse meta-comment
%
% File: catchfile.dtx
% Version: 2016/05/16 v1.7
% Info: Catch the contents of a file
%
% Copyright (C) 2007, 2010, 2011 by
%    Heiko Oberdiek <heiko.oberdiek at googlemail.com>
%    2016
%    https://github.com/ho-tex/oberdiek/issues
%
% This work may be distributed and/or modified under the
% conditions of the LaTeX Project Public License, either
% version 1.3c of this license or (at your option) any later
% version. This version of this license is in
%    http://www.latex-project.org/lppl/lppl-1-3c.txt
% and the latest version of this license is in
%    http://www.latex-project.org/lppl.txt
% and version 1.3 or later is part of all distributions of
% LaTeX version 2005/12/01 or later.
%
% This work has the LPPL maintenance status "maintained".
%
% This Current Maintainer of this work is Heiko Oberdiek.
%
% The Base Interpreter refers to any `TeX-Format',
% because some files are installed in TDS:tex/generic//.
%
% This work consists of the main source file catchfile.dtx
% and the derived files
%    catchfile.sty, catchfile.pdf, catchfile.ins, catchfile.drv,
%    catchfile-test1.tex, catchfile-test2.tex,
%    catchfile-test3.tex.
%
% Distribution:
%    CTAN:macros/latex/contrib/oberdiek/catchfile.dtx
%    CTAN:macros/latex/contrib/oberdiek/catchfile.pdf
%
% Unpacking:
%    (a) If catchfile.ins is present:
%           tex catchfile.ins
%    (b) Without catchfile.ins:
%           tex catchfile.dtx
%    (c) If you insist on using LaTeX
%           latex \let\install=y\input{catchfile.dtx}
%        (quote the arguments according to the demands of your shell)
%
% Documentation:
%    (a) If catchfile.drv is present:
%           latex catchfile.drv
%    (b) Without catchfile.drv:
%           latex catchfile.dtx; ...
%    The class ltxdoc loads the configuration file ltxdoc.cfg
%    if available. Here you can specify further options, e.g.
%    use A4 as paper format:
%       \PassOptionsToClass{a4paper}{article}
%
%    Programm calls to get the documentation (example):
%       pdflatex catchfile.dtx
%       makeindex -s gind.ist catchfile.idx
%       pdflatex catchfile.dtx
%       makeindex -s gind.ist catchfile.idx
%       pdflatex catchfile.dtx
%
% Installation:
%    TDS:tex/generic/oberdiek/catchfile.sty
%    TDS:doc/latex/oberdiek/catchfile.pdf
%    TDS:doc/latex/oberdiek/test/catchfile-test1.tex
%    TDS:doc/latex/oberdiek/test/catchfile-test2.tex
%    TDS:doc/latex/oberdiek/test/catchfile-test3.tex
%    TDS:source/latex/oberdiek/catchfile.dtx
%
%<*ignore>
\begingroup
  \catcode123=1 %
  \catcode125=2 %
  \def\x{LaTeX2e}%
\expandafter\endgroup
\ifcase 0\ifx\install y1\fi\expandafter
         \ifx\csname processbatchFile\endcsname\relax\else1\fi
         \ifx\fmtname\x\else 1\fi\relax
\else\csname fi\endcsname
%</ignore>
%<*install>
\input docstrip.tex
\Msg{************************************************************************}
\Msg{* Installation}
\Msg{* Package: catchfile 2016/05/16 v1.7 Catch the contents of a file (HO)}
\Msg{************************************************************************}

\keepsilent
\askforoverwritefalse

\let\MetaPrefix\relax
\preamble

This is a generated file.

Project: catchfile
Version: 2016/05/16 v1.7

Copyright (C) 2007, 2010, 2011 by
   Heiko Oberdiek <heiko.oberdiek at googlemail.com>

This work may be distributed and/or modified under the
conditions of the LaTeX Project Public License, either
version 1.3c of this license or (at your option) any later
version. This version of this license is in
   http://www.latex-project.org/lppl/lppl-1-3c.txt
and the latest version of this license is in
   http://www.latex-project.org/lppl.txt
and version 1.3 or later is part of all distributions of
LaTeX version 2005/12/01 or later.

This work has the LPPL maintenance status "maintained".

This Current Maintainer of this work is Heiko Oberdiek.

The Base Interpreter refers to any `TeX-Format',
because some files are installed in TDS:tex/generic//.

This work consists of the main source file catchfile.dtx
and the derived files
   catchfile.sty, catchfile.pdf, catchfile.ins, catchfile.drv,
   catchfile-test1.tex, catchfile-test2.tex,
   catchfile-test3.tex.

\endpreamble
\let\MetaPrefix\DoubleperCent

\generate{%
  \file{catchfile.ins}{\from{catchfile.dtx}{install}}%
  \file{catchfile.drv}{\from{catchfile.dtx}{driver}}%
  \usedir{tex/generic/oberdiek}%
  \file{catchfile.sty}{\from{catchfile.dtx}{package}}%
%  \usedir{doc/latex/oberdiek/test}%
%  \file{catchfile-test1.tex}{\from{catchfile.dtx}{test1}}%
%  \file{catchfile-test2.tex}{\from{catchfile.dtx}{test2}}%
%  \file{catchfile-test3.tex}{\from{catchfile.dtx}{test3}}%
  \nopreamble
  \nopostamble
%  \usedir{source/latex/oberdiek/catalogue}%
%  \file{catchfile.xml}{\from{catchfile.dtx}{catalogue}}%
}

\catcode32=13\relax% active space
\let =\space%
\Msg{************************************************************************}
\Msg{*}
\Msg{* To finish the installation you have to move the following}
\Msg{* file into a directory searched by TeX:}
\Msg{*}
\Msg{*     catchfile.sty}
\Msg{*}
\Msg{* To produce the documentation run the file `catchfile.drv'}
\Msg{* through LaTeX.}
\Msg{*}
\Msg{* Happy TeXing!}
\Msg{*}
\Msg{************************************************************************}

\endbatchfile
%</install>
%<*ignore>
\fi
%</ignore>
%<*driver>
\NeedsTeXFormat{LaTeX2e}
\ProvidesFile{catchfile.drv}%
  [2016/05/16 v1.7 Catch the contents of a file (HO)]%
\documentclass{ltxdoc}
\usepackage{holtxdoc}[2011/11/22]
\begin{document}
  \DocInput{catchfile.dtx}%
\end{document}
%</driver>
% \fi
%
%
% \CharacterTable
%  {Upper-case    \A\B\C\D\E\F\G\H\I\J\K\L\M\N\O\P\Q\R\S\T\U\V\W\X\Y\Z
%   Lower-case    \a\b\c\d\e\f\g\h\i\j\k\l\m\n\o\p\q\r\s\t\u\v\w\x\y\z
%   Digits        \0\1\2\3\4\5\6\7\8\9
%   Exclamation   \!     Double quote  \"     Hash (number) \#
%   Dollar        \$     Percent       \%     Ampersand     \&
%   Acute accent  \'     Left paren    \(     Right paren   \)
%   Asterisk      \*     Plus          \+     Comma         \,
%   Minus         \-     Point         \.     Solidus       \/
%   Colon         \:     Semicolon     \;     Less than     \<
%   Equals        \=     Greater than  \>     Question mark \?
%   Commercial at \@     Left bracket  \[     Backslash     \\
%   Right bracket \]     Circumflex    \^     Underscore    \_
%   Grave accent  \`     Left brace    \{     Vertical bar  \|
%   Right brace   \}     Tilde         \~}
%
% \GetFileInfo{catchfile.drv}
%
% \title{The \xpackage{catchfile} package}
% \date{2016/05/16 v1.7}
% \author{Heiko Oberdiek\thanks
% {Please report any issues at https://github.com/ho-tex/oberdiek/issues}\\
% \xemail{heiko.oberdiek at googlemail.com}}
%
% \maketitle
%
% \begin{abstract}
% This package catches the contents of a file and puts it in a macro.
% It requires \eTeX. Both \LaTeX\ and \plainTeX\ are supported.
% \end{abstract}
%
% \tableofcontents
%
% \section{Documentation}
%
% The package relies on \eTeX's \cs{everyeof}. Otherwise it
% aborts with an error message.
%
% \begin{declcs}{CatchFileDef} \M{cmd} \M{file name} \M{setup}\\
% \cs{CatchFileEdef} \M{cmd} \M{file name} \M{setup}
% \end{declcs}
% Macro \meta{cmd} is defined with the contents
% of file \meta{file name}.
% \cs{CatchFileDef} uses \cs{def}, \cs{CatchFileEdef} \cs{edef}
% for the definition.
% Additional setup code for setting
% catcodes or treatment of line ends can be given in code
% \meta{setup}. See the test files for an example.
%
% \StopEventually{
% }
%
% \section{Implementation}
%
%    \begin{macrocode}
%<*package>
%    \end{macrocode}
%
% \subsection{Reload check and package identification}
%    Reload check, especially if the package is not used with \LaTeX.
%    \begin{macrocode}
\begingroup\catcode61\catcode48\catcode32=10\relax%
  \catcode13=5 % ^^M
  \endlinechar=13 %
  \catcode35=6 % #
  \catcode39=12 % '
  \catcode44=12 % ,
  \catcode45=12 % -
  \catcode46=12 % .
  \catcode58=12 % :
  \catcode64=11 % @
  \catcode123=1 % {
  \catcode125=2 % }
  \expandafter\let\expandafter\x\csname ver@catchfile.sty\endcsname
  \ifx\x\relax % plain-TeX, first loading
  \else
    \def\empty{}%
    \ifx\x\empty % LaTeX, first loading,
      % variable is initialized, but \ProvidesPackage not yet seen
    \else
      \expandafter\ifx\csname PackageInfo\endcsname\relax
        \def\x#1#2{%
          \immediate\write-1{Package #1 Info: #2.}%
        }%
      \else
        \def\x#1#2{\PackageInfo{#1}{#2, stopped}}%
      \fi
      \x{catchfile}{The package is already loaded}%
      \aftergroup\endinput
    \fi
  \fi
\endgroup%
%    \end{macrocode}
%    Package identification:
%    \begin{macrocode}
\begingroup\catcode61\catcode48\catcode32=10\relax%
  \catcode13=5 % ^^M
  \endlinechar=13 %
  \catcode35=6 % #
  \catcode39=12 % '
  \catcode40=12 % (
  \catcode41=12 % )
  \catcode44=12 % ,
  \catcode45=12 % -
  \catcode46=12 % .
  \catcode47=12 % /
  \catcode58=12 % :
  \catcode64=11 % @
  \catcode91=12 % [
  \catcode93=12 % ]
  \catcode123=1 % {
  \catcode125=2 % }
  \expandafter\ifx\csname ProvidesPackage\endcsname\relax
    \def\x#1#2#3[#4]{\endgroup
      \immediate\write-1{Package: #3 #4}%
      \xdef#1{#4}%
    }%
  \else
    \def\x#1#2[#3]{\endgroup
      #2[{#3}]%
      \ifx#1\@undefined
        \xdef#1{#3}%
      \fi
      \ifx#1\relax
        \xdef#1{#3}%
      \fi
    }%
  \fi
\expandafter\x\csname ver@catchfile.sty\endcsname
\ProvidesPackage{catchfile}%
  [2016/05/16 v1.7 Catch the contents of a file (HO)]%
%    \end{macrocode}
%
% \subsection{Catcodes}
%
%    \begin{macrocode}
\begingroup\catcode61\catcode48\catcode32=10\relax%
  \catcode13=5 % ^^M
  \endlinechar=13 %
  \catcode123=1 % {
  \catcode125=2 % }
  \catcode64=11 % @
  \def\x{\endgroup
    \expandafter\edef\csname CatchFile@AtEnd\endcsname{%
      \endlinechar=\the\endlinechar\relax
      \catcode13=\the\catcode13\relax
      \catcode32=\the\catcode32\relax
      \catcode35=\the\catcode35\relax
      \catcode61=\the\catcode61\relax
      \catcode64=\the\catcode64\relax
      \catcode123=\the\catcode123\relax
      \catcode125=\the\catcode125\relax
    }%
  }%
\x\catcode61\catcode48\catcode32=10\relax%
\catcode13=5 % ^^M
\endlinechar=13 %
\catcode35=6 % #
\catcode64=11 % @
\catcode123=1 % {
\catcode125=2 % }
\def\TMP@EnsureCode#1#2{%
  \edef\CatchFile@AtEnd{%
    \CatchFile@AtEnd
    \catcode#1=\the\catcode#1\relax
  }%
  \catcode#1=#2\relax
}
\TMP@EnsureCode{39}{12}% '
\TMP@EnsureCode{44}{12}% ,
\TMP@EnsureCode{45}{12}% -
\TMP@EnsureCode{46}{12}% .
\TMP@EnsureCode{47}{12}% /
\TMP@EnsureCode{91}{12}% [
\TMP@EnsureCode{93}{12}% ]
\TMP@EnsureCode{96}{12}% `
\edef\CatchFile@AtEnd{\CatchFile@AtEnd\noexpand\endinput}
%    \end{macrocode}
%
% \subsection{Preparations}
%
%    \begin{macrocode}
\begingroup\expandafter\expandafter\expandafter\endgroup
\expandafter\ifx\csname RequirePackage\endcsname\relax
  \input infwarerr.sty\relax
  \input ltxcmds.sty\relax
\else
  \RequirePackage{infwarerr}[2007/09/09]%
  \RequirePackage{ltxcmds}[2010/03/09]%
\fi
%    \end{macrocode}
%
%    Check for \eTeX's \cs{everyeof}.
%    \begin{macrocode}
\begingroup
  \escapechar=92\relax
  \edef\TestString{\string\everyeof}%
  \edef\TestMeaning{\meaning\everyeof}%
  \ifx\TestString\TestMeaning
  \else
    \@PackageError{catchfile}{%
      Cannot find e-TeX's \string\everyeof,\MessageBreak
      package loading is aborted%
    }\@ehd
    \endgroup
    \expandafter\CatchFile@AtEnd
  \fi%
\endgroup
%    \end{macrocode}
%
% \subsection{Looking for primitive \cs{input}}
%
%    \begin{macro}{\CatchFile@Input}
%    The package needs the expandable primitive \cs{input}.
%    However there are formats that redefine it. For example,
%    \LaTeX's \cs{input} is not expandable, but it stores
%    the primitive in \cs{@@input}. The third possibility
%    is \cs{pdfprimitive}, introduced in \pdfTeX\ 1.40.0.
%
%    Thus we try to find the primitive and store it in
%    \cs{CatchFile@Input}. If it is used, it must be expanded
%    twice (because of the solution with \cs{pdfprimitive}).
%    \begin{macrocode}
\begingroup
  \def\Check#1#2#3#4\endgroup{%
    \edef\TestString{\string#1}%
    \edef\TestMeaning{\meaning#2}%
    \ifx\TestString\TestMeaning
      \endgroup
      \let\CatchFile@Primitive#2%
      \def\CatchFile@Input{\CatchFile@Primitive#3}%
    \else
      #4\endgroup
    \fi
  }%
  \Check\input\input{}%
  \Check\input\@@input{}%
  \Check\pdfprimitive\pdfprimitive\input
  \@PackageError{%
    Cannot find primitive \string\input,\MessageBreak
    package loading is aborted%
  }\@ehd
  \csname endgroup\endcsname
  \CatchFile@AtEnd%
\endgroup
%    \end{macrocode}
%    \end{macro}
%
% \subsection{Input file check}
%
%    \begin{macro}{\CatchFile@CheckFileExists}
%    \begin{macrocode}
\begingroup\expandafter\expandafter\expandafter\endgroup
\expandafter\ifx\csname IfFileExists\endcsname\relax
  \input pdftexcmds.sty\relax
  \begingroup\expandafter\expandafter\expandafter\endgroup
  \expandafter\ifx\csname pdf@filesize\endcsname\relax
    \def\CatchFile@CheckFileExists#1{%
      \expandafter\ifx\csname @inputcheck\endcsname\relax
        \csname newread\endcsname\@inputcheck
      \fi
      \openin\@inputcheck#1\relax
      \expandafter\closein\expandafter\@inputcheck
      \ifeof\@inputcheck
        \let\CatchFile@File\relax
      \else
        \def\CatchFile@File{#1}%
      \fi
    }%
  \else
    \def\CatchFile@CheckFileExists#1{%
      \expandafter\expandafter\expandafter\ifx
      \expandafter\expandafter\expandafter\relax\pdf@filesize{#1}\relax
        \let\CatchFile@File\relax
      \else
        \def\CatchFile@File{#1}%
      \fi
    }%
  \fi
\else
  \def\CatchFile@CheckFileExists#1{%
    \IfFileExists{#1}{%
      \expandafter\CatchFile@DefFile\@filef@und\@nil
      \begingroup\expandafter\expandafter\expandafter\endgroup
      \expandafter\ifx\csname @addtofilelist\endcsname\relax
      \else
        \@addtofilelist\CatchFile@File
      \fi
    }{%
      \let\CatchFile@File\relax
    }%
  }%
  \def\CatchFile@DefFile#1 \@nil{%
    \def\CatchFile@File{#1}%
  }%
\fi
%    \end{macrocode}
%    \end{macro}
%
%    \begin{macro}{\CatchFile@NotFound}
%    \begin{macrocode}
\def\CatchFile@NotFound#1#2{%
  \def#1{}%
  \@PackageError{catchfile}{%
    File `#2' not found%
  }\@ehc
}
%    \end{macrocode}
%    \end{macro}
%
% \subsection{Catch file contents}
%
%    \begin{macrocode}
\ltx@IfUndefined{RequirePackage}{%
  \input etexcmds.sty\relax
}{%
  \RequirePackage{etexcmds}[2010/01/28]%
}
%    \end{macrocode}
%
%    \begin{macro}{\CatchFileEdef}
%    \begin{macrocode}
\long\def\CatchFileEdef#1#2#3{%
  \CatchFile@CheckFileExists{#2}%
  \ifx\CatchFile@File\relax
    \CatchFile@NotFound{#1}{#2}%
  \else
    \begingroup
      \everyeof{\noexpand}%
      #3%
      \xdef\CatchFile@Contents{\CatchFile@Input\CatchFile@File\space}%
    \endgroup
    \let#1\CatchFile@Contents
  \fi
}
%    \end{macrocode}
%    \end{macro}
%
%    \begin{macro}{\CatchFileDef}
%    \begin{macrocode}
\long\def\CatchFileDef#1#2#3{%
  \CatchFile@CheckFileExists{#2}%
  \ifx\CatchFile@File\relax
    \CatchFile@NotFound{#1}{#2}%
  \else
    \begingroup
      \everyeof\expandafter{%
        \CatchFile@EOF
        \expandafter\CatchFile@Finish
        \noexpand
      }%
      \ltx@ifundefined{etex@unexpanded}{%
        \expandafter\long\expandafter\def\expandafter\CatchFile@Do
            \expandafter##\expandafter1\CatchFile@EOF{%
          \toks\ltx@zero{##1}%
          \xdef\CatchFile@gtemp{\the\toks\ltx@zero}%
          \def\CatchFile@Finish{%
            \endgroup
            \let#1\CatchFile@gtemp
            \global\let\CatchFile@gtemp\ltx@undefined
          }%
        }%
      }{%
        \expandafter\long\expandafter\def\expandafter\CatchFile@Do
            \expandafter##\expandafter1\CatchFile@EOF{%
          \edef\CatchFile@Finish{%
            \endgroup
            \etex@unexpanded{%
              \edef#1{\etex@unexpanded{##1}}%
            }%
          }%
        }%
      }%
      #3\relax
    \expandafter\expandafter\expandafter\CatchFile@Do
    \CatchFile@Input\CatchFile@File\relax
  \fi
}
%    \end{macrocode}
%    \cs{relax} after |#3| was added to make it more robust in
%    case the user uses something like
%\begin{verbatim}
%\CatchFileDef{\content}{\jobname.tt}{\endlinechar=-1}
%\end{verbatim}
%    that expands the following \cs{expandafter} after |#3| prematurely
%    (contribution of Martin Scharrer).
%    \end{macro}
%    \begin{macro}{\CatchFile@EOF}
%    If the file is read the catcode mappings are fixed.
%    This means that the same character cannot occur
%    inside the file with different catcodes.
%    Thus we use as end of file marker the at sign
%    twice with different catcodes.
%    \begin{macrocode}
\begingroup
  \lccode65=64 % lowercase('A') = '@'
  \lccode66=64 % lowercase('B') = '@'
  \catcode65=8 % catcode('A') = subscript
  \catcode66=3 % catcode('B') = math shift
\lowercase{\endgroup
  \def\CatchFile@EOF{AB}%
}
%    \end{macrocode}
%    \end{macro}
%
%    \begin{macrocode}
\CatchFile@AtEnd%
%</package>
%    \end{macrocode}
%
% \section{Test}
%
% \subsection{Catcode checks for loading}
%
%    \begin{macrocode}
%<*test1>
%    \end{macrocode}
%    \begin{macrocode}
\catcode`\{=1 %
\catcode`\}=2 %
\catcode`\#=6 %
\catcode`\@=11 %
\expandafter\ifx\csname count@\endcsname\relax
  \countdef\count@=255 %
\fi
\expandafter\ifx\csname @gobble\endcsname\relax
  \long\def\@gobble#1{}%
\fi
\expandafter\ifx\csname @firstofone\endcsname\relax
  \long\def\@firstofone#1{#1}%
\fi
\expandafter\ifx\csname loop\endcsname\relax
  \expandafter\@firstofone
\else
  \expandafter\@gobble
\fi
{%
  \def\loop#1\repeat{%
    \def\body{#1}%
    \iterate
  }%
  \def\iterate{%
    \body
      \let\next\iterate
    \else
      \let\next\relax
    \fi
    \next
  }%
  \let\repeat=\fi
}%
\def\RestoreCatcodes{}
\count@=0 %
\loop
  \edef\RestoreCatcodes{%
    \RestoreCatcodes
    \catcode\the\count@=\the\catcode\count@\relax
  }%
\ifnum\count@<255 %
  \advance\count@ 1 %
\repeat

\def\RangeCatcodeInvalid#1#2{%
  \count@=#1\relax
  \loop
    \catcode\count@=15 %
  \ifnum\count@<#2\relax
    \advance\count@ 1 %
  \repeat
}
\def\RangeCatcodeCheck#1#2#3{%
  \count@=#1\relax
  \loop
    \ifnum#3=\catcode\count@
    \else
      \errmessage{%
        Character \the\count@\space
        with wrong catcode \the\catcode\count@\space
        instead of \number#3%
      }%
    \fi
  \ifnum\count@<#2\relax
    \advance\count@ 1 %
  \repeat
}
\def\space{ }
\expandafter\ifx\csname LoadCommand\endcsname\relax
  \def\LoadCommand{\input catchfile.sty\relax}%
\fi
\def\Test{%
  \RangeCatcodeInvalid{0}{47}%
  \RangeCatcodeInvalid{58}{64}%
  \RangeCatcodeInvalid{91}{96}%
  \RangeCatcodeInvalid{123}{255}%
  \catcode`\@=12 %
  \catcode`\\=0 %
  \catcode`\%=14 %
  \LoadCommand
  \RangeCatcodeCheck{0}{36}{15}%
  \RangeCatcodeCheck{37}{37}{14}%
  \RangeCatcodeCheck{38}{47}{15}%
  \RangeCatcodeCheck{48}{57}{12}%
  \RangeCatcodeCheck{58}{63}{15}%
  \RangeCatcodeCheck{64}{64}{12}%
  \RangeCatcodeCheck{65}{90}{11}%
  \RangeCatcodeCheck{91}{91}{15}%
  \RangeCatcodeCheck{92}{92}{0}%
  \RangeCatcodeCheck{93}{96}{15}%
  \RangeCatcodeCheck{97}{122}{11}%
  \RangeCatcodeCheck{123}{255}{15}%
  \RestoreCatcodes
}
\Test
\csname @@end\endcsname
\end
%    \end{macrocode}
%    \begin{macrocode}
%</test1>
%    \end{macrocode}
%
% \subsection{\LaTeX}
%
%    \begin{macrocode}
%<*test2>
\NeedsTeXFormat{LaTeX2e}
\nofiles
\listfiles
\tracingnesting=2 %
\documentclass{minimal}
\usepackage{catchfile}[2016/05/16]
\makeatletter
\def\mysetup{%
  \let\do\@makeother
  \dospecials
}
\def\CheckContents{%
  \begingroup
    \expandafter\ifx\csname contents\endcsname\relax
      \PackageError{TEST}{\string\contents\space is not defined}\@ehc
    \fi
  \endgroup
}
\def\StartDisableUnexpanded{%
  \begingroup
    \let\etex@unexpanded\@undefined
}
\def\StopDisableUnexpanded{%
  \endgroup
}
\def\CheckCleanup{%
  \begingroup
    \edef\x{\the\toks0}%
    \def\y{ABC}%
    \ifx\x\y
    \else
      \PackageError{TEST}{\string\toks0 has changed}\@ehc
    \fi
    \ifx\CatchFile@gtemp\@undefined
    \else
      \PackageError{TEST}{Left over \string\CatchFile@gtemp}\@ehc
    \fi
  \endgroup
}
\def\Check{%
  \CheckContents
  \CheckCleanup
}
\makeatother
\begin{document}
\toks0{ABC}

\CatchFileDef\contents{catchfile.sty}\mysetup
\typeout{\meaning\contents}
\Check
\typeout{***********************************************************}%
\CatchFileDef\contents{catchfile.sty}{}
\typeout{\meaning\contents}
\Check
\typeout{***********************************************************}%

\StartDisableUnexpanded
\CatchFileDef\contents{catchfile.sty}\mysetup
\typeout{\meaning\contents}
\Check
\typeout{***********************************************************}%
\CatchFileDef\contents{catchfile.sty}{}
\typeout{\meaning\contents}
\Check
\typeout{***********************************************************}%
\StopDisableUnexpanded

\CatchFileEdef\contents{catchfile.sty}{%
  \mysetup
  \def\par{^^J}%
  \obeylines
}
\typeout{\contents}
\Check
\typeout{***********************************************************}%
\CatchFileEdef\contents{catchfile.sty}{%
  \catcode`\\=12 %
  \catcode`\#=12 %
  \def\par{^^J}%
  \obeylines
}
\typeout{\contents}
\Check
\typeout{***********************************************************}%
\end{document}
%</test2>
%    \end{macrocode}
%
% \subsection{\plainTeX}
%
%    \begin{macrocode}
%<*test3>
\def\msg#{\immediate\write16}
\newlinechar=10 %
\tracingnesting=2 %
\input catchfile.sty\relax

\def\mysetup{%
  \def\do##1{%
    \catcode`##1=12\relax
  }%
  \dospecials
}
\def\CheckContents{%
  \begingroup
    \expandafter\ifx\csname contents\endcsname\relax
      \def\space{ }%
      \errmessage{\string\contents\space is not defined.}%
    \fi
  \endgroup
}
\catcode`\@=11 %
\def\CheckCleanup{%
  \begingroup
    \edef\x{\the\toks0}%
    \def\y{ABC}%
    \ifx\x\y
    \else
      \errmessage{\string\toks0 has changed}%
    \fi
    \ifx\CatchFile@gtemp\@undefined
    \else
      \errmessage{Left over \string\CatchFile@gtemp}%
    \fi
  \endgroup
}
\def\Check{%
  \CheckContents
  \CheckCleanup
}
\def\StartDisableUnexpanded{%
  \begingroup
    \let\etex@unexpanded\@undefined
}
\def\StopDisableUnexpanded{%
  \endgroup
}
\catcode`\@=12 %

\toks0{ABC}

\CatchFileDef\contents{catchfile.sty}\mysetup
\msg{\meaning\contents}
\Check
\msg{***********************************************************}%
\CatchFileDef\contents{catchfile.sty}{}
\msg{\meaning\contents}
\Check
\msg{***********************************************************}%

\StartDisableUnexpanded
\CatchFileDef\contents{catchfile.sty}\mysetup
\msg{\meaning\contents}
\Check
\msg{***********************************************************}%
\CatchFileDef\contents{catchfile.sty}{}
\msg{\meaning\contents}
\Check
\msg{***********************************************************}%
\StopDisableUnexpanded

\CatchFileEdef\contents{catchfile.sty}{%
  \mysetup
  \def\par{^^J}%
  \obeylines
}
\msg{\contents}
\Check
\msg{***********************************************************}%
\CatchFileEdef\contents{catchfile.sty}{%
  \catcode`\\=12 %
  \catcode`\#=12 %
  \def\par{^^J}%
  \obeylines
}
\msg{\contents}
\Check
\msg{***********************************************************}%

\csname @@end\endcsname
\end
%</test3>
%    \end{macrocode}
%
% \section{Installation}
%
% \subsection{Download}
%
% \paragraph{Package.} This package is available on
% CTAN\footnote{\url{https://ctan.org/pkg/catchfile}}:
% \begin{description}
% \item[\CTAN{macros/latex/contrib/oberdiek/catchfile.dtx}] The source file.
% \item[\CTAN{macros/latex/contrib/oberdiek/catchfile.pdf}] Documentation.
% \end{description}
%
%
% \paragraph{Bundle.} All the packages of the bundle `oberdiek'
% are also available in a TDS compliant ZIP archive. There
% the packages are already unpacked and the documentation files
% are generated. The files and directories obey the TDS standard.
% \begin{description}
% \item[\CTANinstall{install/macros/latex/contrib/oberdiek.tds.zip}]
% \end{description}
% \emph{TDS} refers to the standard ``A Directory Structure
% for \TeX\ Files'' (\CTAN{tds/tds.pdf}). Directories
% with \xfile{texmf} in their name are usually organized this way.
%
% \subsection{Bundle installation}
%
% \paragraph{Unpacking.} Unpack the \xfile{oberdiek.tds.zip} in the
% TDS tree (also known as \xfile{texmf} tree) of your choice.
% Example (linux):
% \begin{quote}
%   |unzip oberdiek.tds.zip -d ~/texmf|
% \end{quote}
%
% \paragraph{Script installation.}
% Check the directory \xfile{TDS:scripts/oberdiek/} for
% scripts that need further installation steps.
% Package \xpackage{attachfile2} comes with the Perl script
% \xfile{pdfatfi.pl} that should be installed in such a way
% that it can be called as \texttt{pdfatfi}.
% Example (linux):
% \begin{quote}
%   |chmod +x scripts/oberdiek/pdfatfi.pl|\\
%   |cp scripts/oberdiek/pdfatfi.pl /usr/local/bin/|
% \end{quote}
%
% \subsection{Package installation}
%
% \paragraph{Unpacking.} The \xfile{.dtx} file is a self-extracting
% \docstrip\ archive. The files are extracted by running the
% \xfile{.dtx} through \plainTeX:
% \begin{quote}
%   \verb|tex catchfile.dtx|
% \end{quote}
%
% \paragraph{TDS.} Now the different files must be moved into
% the different directories in your installation TDS tree
% (also known as \xfile{texmf} tree):
% \begin{quote}
% \def\t{^^A
% \begin{tabular}{@{}>{\ttfamily}l@{ $\rightarrow$ }>{\ttfamily}l@{}}
%   catchfile.sty & tex/generic/oberdiek/catchfile.sty\\
%   catchfile.pdf & doc/latex/oberdiek/catchfile.pdf\\
%   test/catchfile-test1.tex & doc/latex/oberdiek/test/catchfile-test1.tex\\
%   test/catchfile-test2.tex & doc/latex/oberdiek/test/catchfile-test2.tex\\
%   test/catchfile-test3.tex & doc/latex/oberdiek/test/catchfile-test3.tex\\
%   catchfile.dtx & source/latex/oberdiek/catchfile.dtx\\
% \end{tabular}^^A
% }^^A
% \sbox0{\t}^^A
% \ifdim\wd0>\linewidth
%   \begingroup
%     \advance\linewidth by\leftmargin
%     \advance\linewidth by\rightmargin
%   \edef\x{\endgroup
%     \def\noexpand\lw{\the\linewidth}^^A
%   }\x
%   \def\lwbox{^^A
%     \leavevmode
%     \hbox to \linewidth{^^A
%       \kern-\leftmargin\relax
%       \hss
%       \usebox0
%       \hss
%       \kern-\rightmargin\relax
%     }^^A
%   }^^A
%   \ifdim\wd0>\lw
%     \sbox0{\small\t}^^A
%     \ifdim\wd0>\linewidth
%       \ifdim\wd0>\lw
%         \sbox0{\footnotesize\t}^^A
%         \ifdim\wd0>\linewidth
%           \ifdim\wd0>\lw
%             \sbox0{\scriptsize\t}^^A
%             \ifdim\wd0>\linewidth
%               \ifdim\wd0>\lw
%                 \sbox0{\tiny\t}^^A
%                 \ifdim\wd0>\linewidth
%                   \lwbox
%                 \else
%                   \usebox0
%                 \fi
%               \else
%                 \lwbox
%               \fi
%             \else
%               \usebox0
%             \fi
%           \else
%             \lwbox
%           \fi
%         \else
%           \usebox0
%         \fi
%       \else
%         \lwbox
%       \fi
%     \else
%       \usebox0
%     \fi
%   \else
%     \lwbox
%   \fi
% \else
%   \usebox0
% \fi
% \end{quote}
% If you have a \xfile{docstrip.cfg} that configures and enables \docstrip's
% TDS installing feature, then some files can already be in the right
% place, see the documentation of \docstrip.
%
% \subsection{Refresh file name databases}
%
% If your \TeX~distribution
% (\teTeX, \mikTeX, \dots) relies on file name databases, you must refresh
% these. For example, \teTeX\ users run \verb|texhash| or
% \verb|mktexlsr|.
%
% \subsection{Some details for the interested}
%
% \paragraph{Attached source.}
%
% The PDF documentation on CTAN also includes the
% \xfile{.dtx} source file. It can be extracted by
% AcrobatReader 6 or higher. Another option is \textsf{pdftk},
% e.g. unpack the file into the current directory:
% \begin{quote}
%   \verb|pdftk catchfile.pdf unpack_files output .|
% \end{quote}
%
% \paragraph{Unpacking with \LaTeX.}
% The \xfile{.dtx} chooses its action depending on the format:
% \begin{description}
% \item[\plainTeX:] Run \docstrip\ and extract the files.
% \item[\LaTeX:] Generate the documentation.
% \end{description}
% If you insist on using \LaTeX\ for \docstrip\ (really,
% \docstrip\ does not need \LaTeX), then inform the autodetect routine
% about your intention:
% \begin{quote}
%   \verb|latex \let\install=y\input{catchfile.dtx}|
% \end{quote}
% Do not forget to quote the argument according to the demands
% of your shell.
%
% \paragraph{Generating the documentation.}
% You can use both the \xfile{.dtx} or the \xfile{.drv} to generate
% the documentation. The process can be configured by the
% configuration file \xfile{ltxdoc.cfg}. For instance, put this
% line into this file, if you want to have A4 as paper format:
% \begin{quote}
%   \verb|\PassOptionsToClass{a4paper}{article}|
% \end{quote}
% An example follows how to generate the
% documentation with pdf\LaTeX:
% \begin{quote}
%\begin{verbatim}
%pdflatex catchfile.dtx
%makeindex -s gind.ist catchfile.idx
%pdflatex catchfile.dtx
%makeindex -s gind.ist catchfile.idx
%pdflatex catchfile.dtx
%\end{verbatim}
% \end{quote}
%
% \begin{History}
%   \begin{Version}{2007/05/30 v1.0}
%   \item
%     First version.
%   \end{Version}
%   \begin{Version}{2007/09/09 v1.1}
%   \item
%     Catcode section rewritten.
%   \end{Version}
%   \begin{Version}{2007/11/11 v1.2}
%   \item
%     Use of package \xpackage{pdftexcmds} for \LuaTeX\ support.
%   \end{Version}
%   \begin{Version}{2010/03/01 v1.3}
%   \item
%     Fix for unknown \cs{@PackageErrorNoLine}.
%   \end{Version}
%   \begin{Version}{2010/04/08 v1.4}
%   \item
%     \cs{closein} also added if \cs{ifeof} is true.
%   \end{Version}
%   \begin{Version}{2010/04/28 v1.5}
%   \item
%     \cs{CatchFileDef}: Getting rid of warning `end of semi simple group
%     entered at line \dots\ of
%     a different file (Florent Chervet).
%   \item
%     \cs{CatchFileDef}: Fix for error `Illegal parameter number in
%     definition of \dots' (HO) including improvement that uses
%     \cs{unexpanded} if available (Florent Chervet).
%   \end{Version}
%   \begin{Version}{2011/03/01 v1.6}
%   \item
%     \cs{relax} added after the setup argument of
%     \cs{CatchFileDef} to prevent premature file reading
%     (Martin Scharrer).
%   \end{Version}
%   \begin{Version}{2016/05/16 v1.7}
%   \item
%     Documentation updates.
%   \end{Version}
% \end{History}
%
% \PrintIndex
%
% \Finale
\endinput

%        (quote the arguments according to the demands of your shell)
%
% Documentation:
%    (a) If catchfile.drv is present:
%           latex catchfile.drv
%    (b) Without catchfile.drv:
%           latex catchfile.dtx; ...
%    The class ltxdoc loads the configuration file ltxdoc.cfg
%    if available. Here you can specify further options, e.g.
%    use A4 as paper format:
%       \PassOptionsToClass{a4paper}{article}
%
%    Programm calls to get the documentation (example):
%       pdflatex catchfile.dtx
%       makeindex -s gind.ist catchfile.idx
%       pdflatex catchfile.dtx
%       makeindex -s gind.ist catchfile.idx
%       pdflatex catchfile.dtx
%
% Installation:
%    TDS:tex/generic/oberdiek/catchfile.sty
%    TDS:doc/latex/oberdiek/catchfile.pdf
%    TDS:doc/latex/oberdiek/test/catchfile-test1.tex
%    TDS:doc/latex/oberdiek/test/catchfile-test2.tex
%    TDS:doc/latex/oberdiek/test/catchfile-test3.tex
%    TDS:source/latex/oberdiek/catchfile.dtx
%
%<*ignore>
\begingroup
  \catcode123=1 %
  \catcode125=2 %
  \def\x{LaTeX2e}%
\expandafter\endgroup
\ifcase 0\ifx\install y1\fi\expandafter
         \ifx\csname processbatchFile\endcsname\relax\else1\fi
         \ifx\fmtname\x\else 1\fi\relax
\else\csname fi\endcsname
%</ignore>
%<*install>
\input docstrip.tex
\Msg{************************************************************************}
\Msg{* Installation}
\Msg{* Package: catchfile 2016/05/16 v1.7 Catch the contents of a file (HO)}
\Msg{************************************************************************}

\keepsilent
\askforoverwritefalse

\let\MetaPrefix\relax
\preamble

This is a generated file.

Project: catchfile
Version: 2016/05/16 v1.7

Copyright (C) 2007, 2010, 2011 by
   Heiko Oberdiek <heiko.oberdiek at googlemail.com>

This work may be distributed and/or modified under the
conditions of the LaTeX Project Public License, either
version 1.3c of this license or (at your option) any later
version. This version of this license is in
   http://www.latex-project.org/lppl/lppl-1-3c.txt
and the latest version of this license is in
   http://www.latex-project.org/lppl.txt
and version 1.3 or later is part of all distributions of
LaTeX version 2005/12/01 or later.

This work has the LPPL maintenance status "maintained".

This Current Maintainer of this work is Heiko Oberdiek.

The Base Interpreter refers to any `TeX-Format',
because some files are installed in TDS:tex/generic//.

This work consists of the main source file catchfile.dtx
and the derived files
   catchfile.sty, catchfile.pdf, catchfile.ins, catchfile.drv,
   catchfile-test1.tex, catchfile-test2.tex,
   catchfile-test3.tex.

\endpreamble
\let\MetaPrefix\DoubleperCent

\generate{%
  \file{catchfile.ins}{\from{catchfile.dtx}{install}}%
  \file{catchfile.drv}{\from{catchfile.dtx}{driver}}%
  \usedir{tex/generic/oberdiek}%
  \file{catchfile.sty}{\from{catchfile.dtx}{package}}%
%  \usedir{doc/latex/oberdiek/test}%
%  \file{catchfile-test1.tex}{\from{catchfile.dtx}{test1}}%
%  \file{catchfile-test2.tex}{\from{catchfile.dtx}{test2}}%
%  \file{catchfile-test3.tex}{\from{catchfile.dtx}{test3}}%
  \nopreamble
  \nopostamble
%  \usedir{source/latex/oberdiek/catalogue}%
%  \file{catchfile.xml}{\from{catchfile.dtx}{catalogue}}%
}

\catcode32=13\relax% active space
\let =\space%
\Msg{************************************************************************}
\Msg{*}
\Msg{* To finish the installation you have to move the following}
\Msg{* file into a directory searched by TeX:}
\Msg{*}
\Msg{*     catchfile.sty}
\Msg{*}
\Msg{* To produce the documentation run the file `catchfile.drv'}
\Msg{* through LaTeX.}
\Msg{*}
\Msg{* Happy TeXing!}
\Msg{*}
\Msg{************************************************************************}

\endbatchfile
%</install>
%<*ignore>
\fi
%</ignore>
%<*driver>
\NeedsTeXFormat{LaTeX2e}
\ProvidesFile{catchfile.drv}%
  [2016/05/16 v1.7 Catch the contents of a file (HO)]%
\documentclass{ltxdoc}
\usepackage{holtxdoc}[2011/11/22]
\begin{document}
  \DocInput{catchfile.dtx}%
\end{document}
%</driver>
% \fi
%
%
% \CharacterTable
%  {Upper-case    \A\B\C\D\E\F\G\H\I\J\K\L\M\N\O\P\Q\R\S\T\U\V\W\X\Y\Z
%   Lower-case    \a\b\c\d\e\f\g\h\i\j\k\l\m\n\o\p\q\r\s\t\u\v\w\x\y\z
%   Digits        \0\1\2\3\4\5\6\7\8\9
%   Exclamation   \!     Double quote  \"     Hash (number) \#
%   Dollar        \$     Percent       \%     Ampersand     \&
%   Acute accent  \'     Left paren    \(     Right paren   \)
%   Asterisk      \*     Plus          \+     Comma         \,
%   Minus         \-     Point         \.     Solidus       \/
%   Colon         \:     Semicolon     \;     Less than     \<
%   Equals        \=     Greater than  \>     Question mark \?
%   Commercial at \@     Left bracket  \[     Backslash     \\
%   Right bracket \]     Circumflex    \^     Underscore    \_
%   Grave accent  \`     Left brace    \{     Vertical bar  \|
%   Right brace   \}     Tilde         \~}
%
% \GetFileInfo{catchfile.drv}
%
% \title{The \xpackage{catchfile} package}
% \date{2016/05/16 v1.7}
% \author{Heiko Oberdiek\thanks
% {Please report any issues at https://github.com/ho-tex/oberdiek/issues}\\
% \xemail{heiko.oberdiek at googlemail.com}}
%
% \maketitle
%
% \begin{abstract}
% This package catches the contents of a file and puts it in a macro.
% It requires \eTeX. Both \LaTeX\ and \plainTeX\ are supported.
% \end{abstract}
%
% \tableofcontents
%
% \section{Documentation}
%
% The package relies on \eTeX's \cs{everyeof}. Otherwise it
% aborts with an error message.
%
% \begin{declcs}{CatchFileDef} \M{cmd} \M{file name} \M{setup}\\
% \cs{CatchFileEdef} \M{cmd} \M{file name} \M{setup}
% \end{declcs}
% Macro \meta{cmd} is defined with the contents
% of file \meta{file name}.
% \cs{CatchFileDef} uses \cs{def}, \cs{CatchFileEdef} \cs{edef}
% for the definition.
% Additional setup code for setting
% catcodes or treatment of line ends can be given in code
% \meta{setup}. See the test files for an example.
%
% \StopEventually{
% }
%
% \section{Implementation}
%
%    \begin{macrocode}
%<*package>
%    \end{macrocode}
%
% \subsection{Reload check and package identification}
%    Reload check, especially if the package is not used with \LaTeX.
%    \begin{macrocode}
\begingroup\catcode61\catcode48\catcode32=10\relax%
  \catcode13=5 % ^^M
  \endlinechar=13 %
  \catcode35=6 % #
  \catcode39=12 % '
  \catcode44=12 % ,
  \catcode45=12 % -
  \catcode46=12 % .
  \catcode58=12 % :
  \catcode64=11 % @
  \catcode123=1 % {
  \catcode125=2 % }
  \expandafter\let\expandafter\x\csname ver@catchfile.sty\endcsname
  \ifx\x\relax % plain-TeX, first loading
  \else
    \def\empty{}%
    \ifx\x\empty % LaTeX, first loading,
      % variable is initialized, but \ProvidesPackage not yet seen
    \else
      \expandafter\ifx\csname PackageInfo\endcsname\relax
        \def\x#1#2{%
          \immediate\write-1{Package #1 Info: #2.}%
        }%
      \else
        \def\x#1#2{\PackageInfo{#1}{#2, stopped}}%
      \fi
      \x{catchfile}{The package is already loaded}%
      \aftergroup\endinput
    \fi
  \fi
\endgroup%
%    \end{macrocode}
%    Package identification:
%    \begin{macrocode}
\begingroup\catcode61\catcode48\catcode32=10\relax%
  \catcode13=5 % ^^M
  \endlinechar=13 %
  \catcode35=6 % #
  \catcode39=12 % '
  \catcode40=12 % (
  \catcode41=12 % )
  \catcode44=12 % ,
  \catcode45=12 % -
  \catcode46=12 % .
  \catcode47=12 % /
  \catcode58=12 % :
  \catcode64=11 % @
  \catcode91=12 % [
  \catcode93=12 % ]
  \catcode123=1 % {
  \catcode125=2 % }
  \expandafter\ifx\csname ProvidesPackage\endcsname\relax
    \def\x#1#2#3[#4]{\endgroup
      \immediate\write-1{Package: #3 #4}%
      \xdef#1{#4}%
    }%
  \else
    \def\x#1#2[#3]{\endgroup
      #2[{#3}]%
      \ifx#1\@undefined
        \xdef#1{#3}%
      \fi
      \ifx#1\relax
        \xdef#1{#3}%
      \fi
    }%
  \fi
\expandafter\x\csname ver@catchfile.sty\endcsname
\ProvidesPackage{catchfile}%
  [2016/05/16 v1.7 Catch the contents of a file (HO)]%
%    \end{macrocode}
%
% \subsection{Catcodes}
%
%    \begin{macrocode}
\begingroup\catcode61\catcode48\catcode32=10\relax%
  \catcode13=5 % ^^M
  \endlinechar=13 %
  \catcode123=1 % {
  \catcode125=2 % }
  \catcode64=11 % @
  \def\x{\endgroup
    \expandafter\edef\csname CatchFile@AtEnd\endcsname{%
      \endlinechar=\the\endlinechar\relax
      \catcode13=\the\catcode13\relax
      \catcode32=\the\catcode32\relax
      \catcode35=\the\catcode35\relax
      \catcode61=\the\catcode61\relax
      \catcode64=\the\catcode64\relax
      \catcode123=\the\catcode123\relax
      \catcode125=\the\catcode125\relax
    }%
  }%
\x\catcode61\catcode48\catcode32=10\relax%
\catcode13=5 % ^^M
\endlinechar=13 %
\catcode35=6 % #
\catcode64=11 % @
\catcode123=1 % {
\catcode125=2 % }
\def\TMP@EnsureCode#1#2{%
  \edef\CatchFile@AtEnd{%
    \CatchFile@AtEnd
    \catcode#1=\the\catcode#1\relax
  }%
  \catcode#1=#2\relax
}
\TMP@EnsureCode{39}{12}% '
\TMP@EnsureCode{44}{12}% ,
\TMP@EnsureCode{45}{12}% -
\TMP@EnsureCode{46}{12}% .
\TMP@EnsureCode{47}{12}% /
\TMP@EnsureCode{91}{12}% [
\TMP@EnsureCode{93}{12}% ]
\TMP@EnsureCode{96}{12}% `
\edef\CatchFile@AtEnd{\CatchFile@AtEnd\noexpand\endinput}
%    \end{macrocode}
%
% \subsection{Preparations}
%
%    \begin{macrocode}
\begingroup\expandafter\expandafter\expandafter\endgroup
\expandafter\ifx\csname RequirePackage\endcsname\relax
  \input infwarerr.sty\relax
  \input ltxcmds.sty\relax
\else
  \RequirePackage{infwarerr}[2007/09/09]%
  \RequirePackage{ltxcmds}[2010/03/09]%
\fi
%    \end{macrocode}
%
%    Check for \eTeX's \cs{everyeof}.
%    \begin{macrocode}
\begingroup
  \escapechar=92\relax
  \edef\TestString{\string\everyeof}%
  \edef\TestMeaning{\meaning\everyeof}%
  \ifx\TestString\TestMeaning
  \else
    \@PackageError{catchfile}{%
      Cannot find e-TeX's \string\everyeof,\MessageBreak
      package loading is aborted%
    }\@ehd
    \endgroup
    \expandafter\CatchFile@AtEnd
  \fi%
\endgroup
%    \end{macrocode}
%
% \subsection{Looking for primitive \cs{input}}
%
%    \begin{macro}{\CatchFile@Input}
%    The package needs the expandable primitive \cs{input}.
%    However there are formats that redefine it. For example,
%    \LaTeX's \cs{input} is not expandable, but it stores
%    the primitive in \cs{@@input}. The third possibility
%    is \cs{pdfprimitive}, introduced in \pdfTeX\ 1.40.0.
%
%    Thus we try to find the primitive and store it in
%    \cs{CatchFile@Input}. If it is used, it must be expanded
%    twice (because of the solution with \cs{pdfprimitive}).
%    \begin{macrocode}
\begingroup
  \def\Check#1#2#3#4\endgroup{%
    \edef\TestString{\string#1}%
    \edef\TestMeaning{\meaning#2}%
    \ifx\TestString\TestMeaning
      \endgroup
      \let\CatchFile@Primitive#2%
      \def\CatchFile@Input{\CatchFile@Primitive#3}%
    \else
      #4\endgroup
    \fi
  }%
  \Check\input\input{}%
  \Check\input\@@input{}%
  \Check\pdfprimitive\pdfprimitive\input
  \@PackageError{%
    Cannot find primitive \string\input,\MessageBreak
    package loading is aborted%
  }\@ehd
  \csname endgroup\endcsname
  \CatchFile@AtEnd%
\endgroup
%    \end{macrocode}
%    \end{macro}
%
% \subsection{Input file check}
%
%    \begin{macro}{\CatchFile@CheckFileExists}
%    \begin{macrocode}
\begingroup\expandafter\expandafter\expandafter\endgroup
\expandafter\ifx\csname IfFileExists\endcsname\relax
  \input pdftexcmds.sty\relax
  \begingroup\expandafter\expandafter\expandafter\endgroup
  \expandafter\ifx\csname pdf@filesize\endcsname\relax
    \def\CatchFile@CheckFileExists#1{%
      \expandafter\ifx\csname @inputcheck\endcsname\relax
        \csname newread\endcsname\@inputcheck
      \fi
      \openin\@inputcheck#1\relax
      \expandafter\closein\expandafter\@inputcheck
      \ifeof\@inputcheck
        \let\CatchFile@File\relax
      \else
        \def\CatchFile@File{#1}%
      \fi
    }%
  \else
    \def\CatchFile@CheckFileExists#1{%
      \expandafter\expandafter\expandafter\ifx
      \expandafter\expandafter\expandafter\relax\pdf@filesize{#1}\relax
        \let\CatchFile@File\relax
      \else
        \def\CatchFile@File{#1}%
      \fi
    }%
  \fi
\else
  \def\CatchFile@CheckFileExists#1{%
    \IfFileExists{#1}{%
      \expandafter\CatchFile@DefFile\@filef@und\@nil
      \begingroup\expandafter\expandafter\expandafter\endgroup
      \expandafter\ifx\csname @addtofilelist\endcsname\relax
      \else
        \@addtofilelist\CatchFile@File
      \fi
    }{%
      \let\CatchFile@File\relax
    }%
  }%
  \def\CatchFile@DefFile#1 \@nil{%
    \def\CatchFile@File{#1}%
  }%
\fi
%    \end{macrocode}
%    \end{macro}
%
%    \begin{macro}{\CatchFile@NotFound}
%    \begin{macrocode}
\def\CatchFile@NotFound#1#2{%
  \def#1{}%
  \@PackageError{catchfile}{%
    File `#2' not found%
  }\@ehc
}
%    \end{macrocode}
%    \end{macro}
%
% \subsection{Catch file contents}
%
%    \begin{macrocode}
\ltx@IfUndefined{RequirePackage}{%
  \input etexcmds.sty\relax
}{%
  \RequirePackage{etexcmds}[2010/01/28]%
}
%    \end{macrocode}
%
%    \begin{macro}{\CatchFileEdef}
%    \begin{macrocode}
\long\def\CatchFileEdef#1#2#3{%
  \CatchFile@CheckFileExists{#2}%
  \ifx\CatchFile@File\relax
    \CatchFile@NotFound{#1}{#2}%
  \else
    \begingroup
      \everyeof{\noexpand}%
      #3%
      \xdef\CatchFile@Contents{\CatchFile@Input\CatchFile@File\space}%
    \endgroup
    \let#1\CatchFile@Contents
  \fi
}
%    \end{macrocode}
%    \end{macro}
%
%    \begin{macro}{\CatchFileDef}
%    \begin{macrocode}
\long\def\CatchFileDef#1#2#3{%
  \CatchFile@CheckFileExists{#2}%
  \ifx\CatchFile@File\relax
    \CatchFile@NotFound{#1}{#2}%
  \else
    \begingroup
      \everyeof\expandafter{%
        \CatchFile@EOF
        \expandafter\CatchFile@Finish
        \noexpand
      }%
      \ltx@ifundefined{etex@unexpanded}{%
        \expandafter\long\expandafter\def\expandafter\CatchFile@Do
            \expandafter##\expandafter1\CatchFile@EOF{%
          \toks\ltx@zero{##1}%
          \xdef\CatchFile@gtemp{\the\toks\ltx@zero}%
          \def\CatchFile@Finish{%
            \endgroup
            \let#1\CatchFile@gtemp
            \global\let\CatchFile@gtemp\ltx@undefined
          }%
        }%
      }{%
        \expandafter\long\expandafter\def\expandafter\CatchFile@Do
            \expandafter##\expandafter1\CatchFile@EOF{%
          \edef\CatchFile@Finish{%
            \endgroup
            \etex@unexpanded{%
              \edef#1{\etex@unexpanded{##1}}%
            }%
          }%
        }%
      }%
      #3\relax
    \expandafter\expandafter\expandafter\CatchFile@Do
    \CatchFile@Input\CatchFile@File\relax
  \fi
}
%    \end{macrocode}
%    \cs{relax} after |#3| was added to make it more robust in
%    case the user uses something like
%\begin{verbatim}
%\CatchFileDef{\content}{\jobname.tt}{\endlinechar=-1}
%\end{verbatim}
%    that expands the following \cs{expandafter} after |#3| prematurely
%    (contribution of Martin Scharrer).
%    \end{macro}
%    \begin{macro}{\CatchFile@EOF}
%    If the file is read the catcode mappings are fixed.
%    This means that the same character cannot occur
%    inside the file with different catcodes.
%    Thus we use as end of file marker the at sign
%    twice with different catcodes.
%    \begin{macrocode}
\begingroup
  \lccode65=64 % lowercase('A') = '@'
  \lccode66=64 % lowercase('B') = '@'
  \catcode65=8 % catcode('A') = subscript
  \catcode66=3 % catcode('B') = math shift
\lowercase{\endgroup
  \def\CatchFile@EOF{AB}%
}
%    \end{macrocode}
%    \end{macro}
%
%    \begin{macrocode}
\CatchFile@AtEnd%
%</package>
%    \end{macrocode}
%
% \section{Test}
%
% \subsection{Catcode checks for loading}
%
%    \begin{macrocode}
%<*test1>
%    \end{macrocode}
%    \begin{macrocode}
\catcode`\{=1 %
\catcode`\}=2 %
\catcode`\#=6 %
\catcode`\@=11 %
\expandafter\ifx\csname count@\endcsname\relax
  \countdef\count@=255 %
\fi
\expandafter\ifx\csname @gobble\endcsname\relax
  \long\def\@gobble#1{}%
\fi
\expandafter\ifx\csname @firstofone\endcsname\relax
  \long\def\@firstofone#1{#1}%
\fi
\expandafter\ifx\csname loop\endcsname\relax
  \expandafter\@firstofone
\else
  \expandafter\@gobble
\fi
{%
  \def\loop#1\repeat{%
    \def\body{#1}%
    \iterate
  }%
  \def\iterate{%
    \body
      \let\next\iterate
    \else
      \let\next\relax
    \fi
    \next
  }%
  \let\repeat=\fi
}%
\def\RestoreCatcodes{}
\count@=0 %
\loop
  \edef\RestoreCatcodes{%
    \RestoreCatcodes
    \catcode\the\count@=\the\catcode\count@\relax
  }%
\ifnum\count@<255 %
  \advance\count@ 1 %
\repeat

\def\RangeCatcodeInvalid#1#2{%
  \count@=#1\relax
  \loop
    \catcode\count@=15 %
  \ifnum\count@<#2\relax
    \advance\count@ 1 %
  \repeat
}
\def\RangeCatcodeCheck#1#2#3{%
  \count@=#1\relax
  \loop
    \ifnum#3=\catcode\count@
    \else
      \errmessage{%
        Character \the\count@\space
        with wrong catcode \the\catcode\count@\space
        instead of \number#3%
      }%
    \fi
  \ifnum\count@<#2\relax
    \advance\count@ 1 %
  \repeat
}
\def\space{ }
\expandafter\ifx\csname LoadCommand\endcsname\relax
  \def\LoadCommand{\input catchfile.sty\relax}%
\fi
\def\Test{%
  \RangeCatcodeInvalid{0}{47}%
  \RangeCatcodeInvalid{58}{64}%
  \RangeCatcodeInvalid{91}{96}%
  \RangeCatcodeInvalid{123}{255}%
  \catcode`\@=12 %
  \catcode`\\=0 %
  \catcode`\%=14 %
  \LoadCommand
  \RangeCatcodeCheck{0}{36}{15}%
  \RangeCatcodeCheck{37}{37}{14}%
  \RangeCatcodeCheck{38}{47}{15}%
  \RangeCatcodeCheck{48}{57}{12}%
  \RangeCatcodeCheck{58}{63}{15}%
  \RangeCatcodeCheck{64}{64}{12}%
  \RangeCatcodeCheck{65}{90}{11}%
  \RangeCatcodeCheck{91}{91}{15}%
  \RangeCatcodeCheck{92}{92}{0}%
  \RangeCatcodeCheck{93}{96}{15}%
  \RangeCatcodeCheck{97}{122}{11}%
  \RangeCatcodeCheck{123}{255}{15}%
  \RestoreCatcodes
}
\Test
\csname @@end\endcsname
\end
%    \end{macrocode}
%    \begin{macrocode}
%</test1>
%    \end{macrocode}
%
% \subsection{\LaTeX}
%
%    \begin{macrocode}
%<*test2>
\NeedsTeXFormat{LaTeX2e}
\nofiles
\listfiles
\tracingnesting=2 %
\documentclass{minimal}
\usepackage{catchfile}[2016/05/16]
\makeatletter
\def\mysetup{%
  \let\do\@makeother
  \dospecials
}
\def\CheckContents{%
  \begingroup
    \expandafter\ifx\csname contents\endcsname\relax
      \PackageError{TEST}{\string\contents\space is not defined}\@ehc
    \fi
  \endgroup
}
\def\StartDisableUnexpanded{%
  \begingroup
    \let\etex@unexpanded\@undefined
}
\def\StopDisableUnexpanded{%
  \endgroup
}
\def\CheckCleanup{%
  \begingroup
    \edef\x{\the\toks0}%
    \def\y{ABC}%
    \ifx\x\y
    \else
      \PackageError{TEST}{\string\toks0 has changed}\@ehc
    \fi
    \ifx\CatchFile@gtemp\@undefined
    \else
      \PackageError{TEST}{Left over \string\CatchFile@gtemp}\@ehc
    \fi
  \endgroup
}
\def\Check{%
  \CheckContents
  \CheckCleanup
}
\makeatother
\begin{document}
\toks0{ABC}

\CatchFileDef\contents{catchfile.sty}\mysetup
\typeout{\meaning\contents}
\Check
\typeout{***********************************************************}%
\CatchFileDef\contents{catchfile.sty}{}
\typeout{\meaning\contents}
\Check
\typeout{***********************************************************}%

\StartDisableUnexpanded
\CatchFileDef\contents{catchfile.sty}\mysetup
\typeout{\meaning\contents}
\Check
\typeout{***********************************************************}%
\CatchFileDef\contents{catchfile.sty}{}
\typeout{\meaning\contents}
\Check
\typeout{***********************************************************}%
\StopDisableUnexpanded

\CatchFileEdef\contents{catchfile.sty}{%
  \mysetup
  \def\par{^^J}%
  \obeylines
}
\typeout{\contents}
\Check
\typeout{***********************************************************}%
\CatchFileEdef\contents{catchfile.sty}{%
  \catcode`\\=12 %
  \catcode`\#=12 %
  \def\par{^^J}%
  \obeylines
}
\typeout{\contents}
\Check
\typeout{***********************************************************}%
\end{document}
%</test2>
%    \end{macrocode}
%
% \subsection{\plainTeX}
%
%    \begin{macrocode}
%<*test3>
\def\msg#{\immediate\write16}
\newlinechar=10 %
\tracingnesting=2 %
\input catchfile.sty\relax

\def\mysetup{%
  \def\do##1{%
    \catcode`##1=12\relax
  }%
  \dospecials
}
\def\CheckContents{%
  \begingroup
    \expandafter\ifx\csname contents\endcsname\relax
      \def\space{ }%
      \errmessage{\string\contents\space is not defined.}%
    \fi
  \endgroup
}
\catcode`\@=11 %
\def\CheckCleanup{%
  \begingroup
    \edef\x{\the\toks0}%
    \def\y{ABC}%
    \ifx\x\y
    \else
      \errmessage{\string\toks0 has changed}%
    \fi
    \ifx\CatchFile@gtemp\@undefined
    \else
      \errmessage{Left over \string\CatchFile@gtemp}%
    \fi
  \endgroup
}
\def\Check{%
  \CheckContents
  \CheckCleanup
}
\def\StartDisableUnexpanded{%
  \begingroup
    \let\etex@unexpanded\@undefined
}
\def\StopDisableUnexpanded{%
  \endgroup
}
\catcode`\@=12 %

\toks0{ABC}

\CatchFileDef\contents{catchfile.sty}\mysetup
\msg{\meaning\contents}
\Check
\msg{***********************************************************}%
\CatchFileDef\contents{catchfile.sty}{}
\msg{\meaning\contents}
\Check
\msg{***********************************************************}%

\StartDisableUnexpanded
\CatchFileDef\contents{catchfile.sty}\mysetup
\msg{\meaning\contents}
\Check
\msg{***********************************************************}%
\CatchFileDef\contents{catchfile.sty}{}
\msg{\meaning\contents}
\Check
\msg{***********************************************************}%
\StopDisableUnexpanded

\CatchFileEdef\contents{catchfile.sty}{%
  \mysetup
  \def\par{^^J}%
  \obeylines
}
\msg{\contents}
\Check
\msg{***********************************************************}%
\CatchFileEdef\contents{catchfile.sty}{%
  \catcode`\\=12 %
  \catcode`\#=12 %
  \def\par{^^J}%
  \obeylines
}
\msg{\contents}
\Check
\msg{***********************************************************}%

\csname @@end\endcsname
\end
%</test3>
%    \end{macrocode}
%
% \section{Installation}
%
% \subsection{Download}
%
% \paragraph{Package.} This package is available on
% CTAN\footnote{\url{https://ctan.org/pkg/catchfile}}:
% \begin{description}
% \item[\CTAN{macros/latex/contrib/oberdiek/catchfile.dtx}] The source file.
% \item[\CTAN{macros/latex/contrib/oberdiek/catchfile.pdf}] Documentation.
% \end{description}
%
%
% \paragraph{Bundle.} All the packages of the bundle `oberdiek'
% are also available in a TDS compliant ZIP archive. There
% the packages are already unpacked and the documentation files
% are generated. The files and directories obey the TDS standard.
% \begin{description}
% \item[\CTANinstall{install/macros/latex/contrib/oberdiek.tds.zip}]
% \end{description}
% \emph{TDS} refers to the standard ``A Directory Structure
% for \TeX\ Files'' (\CTAN{tds/tds.pdf}). Directories
% with \xfile{texmf} in their name are usually organized this way.
%
% \subsection{Bundle installation}
%
% \paragraph{Unpacking.} Unpack the \xfile{oberdiek.tds.zip} in the
% TDS tree (also known as \xfile{texmf} tree) of your choice.
% Example (linux):
% \begin{quote}
%   |unzip oberdiek.tds.zip -d ~/texmf|
% \end{quote}
%
% \paragraph{Script installation.}
% Check the directory \xfile{TDS:scripts/oberdiek/} for
% scripts that need further installation steps.
% Package \xpackage{attachfile2} comes with the Perl script
% \xfile{pdfatfi.pl} that should be installed in such a way
% that it can be called as \texttt{pdfatfi}.
% Example (linux):
% \begin{quote}
%   |chmod +x scripts/oberdiek/pdfatfi.pl|\\
%   |cp scripts/oberdiek/pdfatfi.pl /usr/local/bin/|
% \end{quote}
%
% \subsection{Package installation}
%
% \paragraph{Unpacking.} The \xfile{.dtx} file is a self-extracting
% \docstrip\ archive. The files are extracted by running the
% \xfile{.dtx} through \plainTeX:
% \begin{quote}
%   \verb|tex catchfile.dtx|
% \end{quote}
%
% \paragraph{TDS.} Now the different files must be moved into
% the different directories in your installation TDS tree
% (also known as \xfile{texmf} tree):
% \begin{quote}
% \def\t{^^A
% \begin{tabular}{@{}>{\ttfamily}l@{ $\rightarrow$ }>{\ttfamily}l@{}}
%   catchfile.sty & tex/generic/oberdiek/catchfile.sty\\
%   catchfile.pdf & doc/latex/oberdiek/catchfile.pdf\\
%   test/catchfile-test1.tex & doc/latex/oberdiek/test/catchfile-test1.tex\\
%   test/catchfile-test2.tex & doc/latex/oberdiek/test/catchfile-test2.tex\\
%   test/catchfile-test3.tex & doc/latex/oberdiek/test/catchfile-test3.tex\\
%   catchfile.dtx & source/latex/oberdiek/catchfile.dtx\\
% \end{tabular}^^A
% }^^A
% \sbox0{\t}^^A
% \ifdim\wd0>\linewidth
%   \begingroup
%     \advance\linewidth by\leftmargin
%     \advance\linewidth by\rightmargin
%   \edef\x{\endgroup
%     \def\noexpand\lw{\the\linewidth}^^A
%   }\x
%   \def\lwbox{^^A
%     \leavevmode
%     \hbox to \linewidth{^^A
%       \kern-\leftmargin\relax
%       \hss
%       \usebox0
%       \hss
%       \kern-\rightmargin\relax
%     }^^A
%   }^^A
%   \ifdim\wd0>\lw
%     \sbox0{\small\t}^^A
%     \ifdim\wd0>\linewidth
%       \ifdim\wd0>\lw
%         \sbox0{\footnotesize\t}^^A
%         \ifdim\wd0>\linewidth
%           \ifdim\wd0>\lw
%             \sbox0{\scriptsize\t}^^A
%             \ifdim\wd0>\linewidth
%               \ifdim\wd0>\lw
%                 \sbox0{\tiny\t}^^A
%                 \ifdim\wd0>\linewidth
%                   \lwbox
%                 \else
%                   \usebox0
%                 \fi
%               \else
%                 \lwbox
%               \fi
%             \else
%               \usebox0
%             \fi
%           \else
%             \lwbox
%           \fi
%         \else
%           \usebox0
%         \fi
%       \else
%         \lwbox
%       \fi
%     \else
%       \usebox0
%     \fi
%   \else
%     \lwbox
%   \fi
% \else
%   \usebox0
% \fi
% \end{quote}
% If you have a \xfile{docstrip.cfg} that configures and enables \docstrip's
% TDS installing feature, then some files can already be in the right
% place, see the documentation of \docstrip.
%
% \subsection{Refresh file name databases}
%
% If your \TeX~distribution
% (\teTeX, \mikTeX, \dots) relies on file name databases, you must refresh
% these. For example, \teTeX\ users run \verb|texhash| or
% \verb|mktexlsr|.
%
% \subsection{Some details for the interested}
%
% \paragraph{Attached source.}
%
% The PDF documentation on CTAN also includes the
% \xfile{.dtx} source file. It can be extracted by
% AcrobatReader 6 or higher. Another option is \textsf{pdftk},
% e.g. unpack the file into the current directory:
% \begin{quote}
%   \verb|pdftk catchfile.pdf unpack_files output .|
% \end{quote}
%
% \paragraph{Unpacking with \LaTeX.}
% The \xfile{.dtx} chooses its action depending on the format:
% \begin{description}
% \item[\plainTeX:] Run \docstrip\ and extract the files.
% \item[\LaTeX:] Generate the documentation.
% \end{description}
% If you insist on using \LaTeX\ for \docstrip\ (really,
% \docstrip\ does not need \LaTeX), then inform the autodetect routine
% about your intention:
% \begin{quote}
%   \verb|latex \let\install=y% \iffalse meta-comment
%
% File: catchfile.dtx
% Version: 2016/05/16 v1.7
% Info: Catch the contents of a file
%
% Copyright (C) 2007, 2010, 2011 by
%    Heiko Oberdiek <heiko.oberdiek at googlemail.com>
%    2016
%    https://github.com/ho-tex/oberdiek/issues
%
% This work may be distributed and/or modified under the
% conditions of the LaTeX Project Public License, either
% version 1.3c of this license or (at your option) any later
% version. This version of this license is in
%    http://www.latex-project.org/lppl/lppl-1-3c.txt
% and the latest version of this license is in
%    http://www.latex-project.org/lppl.txt
% and version 1.3 or later is part of all distributions of
% LaTeX version 2005/12/01 or later.
%
% This work has the LPPL maintenance status "maintained".
%
% This Current Maintainer of this work is Heiko Oberdiek.
%
% The Base Interpreter refers to any `TeX-Format',
% because some files are installed in TDS:tex/generic//.
%
% This work consists of the main source file catchfile.dtx
% and the derived files
%    catchfile.sty, catchfile.pdf, catchfile.ins, catchfile.drv,
%    catchfile-test1.tex, catchfile-test2.tex,
%    catchfile-test3.tex.
%
% Distribution:
%    CTAN:macros/latex/contrib/oberdiek/catchfile.dtx
%    CTAN:macros/latex/contrib/oberdiek/catchfile.pdf
%
% Unpacking:
%    (a) If catchfile.ins is present:
%           tex catchfile.ins
%    (b) Without catchfile.ins:
%           tex catchfile.dtx
%    (c) If you insist on using LaTeX
%           latex \let\install=y\input{catchfile.dtx}
%        (quote the arguments according to the demands of your shell)
%
% Documentation:
%    (a) If catchfile.drv is present:
%           latex catchfile.drv
%    (b) Without catchfile.drv:
%           latex catchfile.dtx; ...
%    The class ltxdoc loads the configuration file ltxdoc.cfg
%    if available. Here you can specify further options, e.g.
%    use A4 as paper format:
%       \PassOptionsToClass{a4paper}{article}
%
%    Programm calls to get the documentation (example):
%       pdflatex catchfile.dtx
%       makeindex -s gind.ist catchfile.idx
%       pdflatex catchfile.dtx
%       makeindex -s gind.ist catchfile.idx
%       pdflatex catchfile.dtx
%
% Installation:
%    TDS:tex/generic/oberdiek/catchfile.sty
%    TDS:doc/latex/oberdiek/catchfile.pdf
%    TDS:doc/latex/oberdiek/test/catchfile-test1.tex
%    TDS:doc/latex/oberdiek/test/catchfile-test2.tex
%    TDS:doc/latex/oberdiek/test/catchfile-test3.tex
%    TDS:source/latex/oberdiek/catchfile.dtx
%
%<*ignore>
\begingroup
  \catcode123=1 %
  \catcode125=2 %
  \def\x{LaTeX2e}%
\expandafter\endgroup
\ifcase 0\ifx\install y1\fi\expandafter
         \ifx\csname processbatchFile\endcsname\relax\else1\fi
         \ifx\fmtname\x\else 1\fi\relax
\else\csname fi\endcsname
%</ignore>
%<*install>
\input docstrip.tex
\Msg{************************************************************************}
\Msg{* Installation}
\Msg{* Package: catchfile 2016/05/16 v1.7 Catch the contents of a file (HO)}
\Msg{************************************************************************}

\keepsilent
\askforoverwritefalse

\let\MetaPrefix\relax
\preamble

This is a generated file.

Project: catchfile
Version: 2016/05/16 v1.7

Copyright (C) 2007, 2010, 2011 by
   Heiko Oberdiek <heiko.oberdiek at googlemail.com>

This work may be distributed and/or modified under the
conditions of the LaTeX Project Public License, either
version 1.3c of this license or (at your option) any later
version. This version of this license is in
   http://www.latex-project.org/lppl/lppl-1-3c.txt
and the latest version of this license is in
   http://www.latex-project.org/lppl.txt
and version 1.3 or later is part of all distributions of
LaTeX version 2005/12/01 or later.

This work has the LPPL maintenance status "maintained".

This Current Maintainer of this work is Heiko Oberdiek.

The Base Interpreter refers to any `TeX-Format',
because some files are installed in TDS:tex/generic//.

This work consists of the main source file catchfile.dtx
and the derived files
   catchfile.sty, catchfile.pdf, catchfile.ins, catchfile.drv,
   catchfile-test1.tex, catchfile-test2.tex,
   catchfile-test3.tex.

\endpreamble
\let\MetaPrefix\DoubleperCent

\generate{%
  \file{catchfile.ins}{\from{catchfile.dtx}{install}}%
  \file{catchfile.drv}{\from{catchfile.dtx}{driver}}%
  \usedir{tex/generic/oberdiek}%
  \file{catchfile.sty}{\from{catchfile.dtx}{package}}%
%  \usedir{doc/latex/oberdiek/test}%
%  \file{catchfile-test1.tex}{\from{catchfile.dtx}{test1}}%
%  \file{catchfile-test2.tex}{\from{catchfile.dtx}{test2}}%
%  \file{catchfile-test3.tex}{\from{catchfile.dtx}{test3}}%
  \nopreamble
  \nopostamble
%  \usedir{source/latex/oberdiek/catalogue}%
%  \file{catchfile.xml}{\from{catchfile.dtx}{catalogue}}%
}

\catcode32=13\relax% active space
\let =\space%
\Msg{************************************************************************}
\Msg{*}
\Msg{* To finish the installation you have to move the following}
\Msg{* file into a directory searched by TeX:}
\Msg{*}
\Msg{*     catchfile.sty}
\Msg{*}
\Msg{* To produce the documentation run the file `catchfile.drv'}
\Msg{* through LaTeX.}
\Msg{*}
\Msg{* Happy TeXing!}
\Msg{*}
\Msg{************************************************************************}

\endbatchfile
%</install>
%<*ignore>
\fi
%</ignore>
%<*driver>
\NeedsTeXFormat{LaTeX2e}
\ProvidesFile{catchfile.drv}%
  [2016/05/16 v1.7 Catch the contents of a file (HO)]%
\documentclass{ltxdoc}
\usepackage{holtxdoc}[2011/11/22]
\begin{document}
  \DocInput{catchfile.dtx}%
\end{document}
%</driver>
% \fi
%
%
% \CharacterTable
%  {Upper-case    \A\B\C\D\E\F\G\H\I\J\K\L\M\N\O\P\Q\R\S\T\U\V\W\X\Y\Z
%   Lower-case    \a\b\c\d\e\f\g\h\i\j\k\l\m\n\o\p\q\r\s\t\u\v\w\x\y\z
%   Digits        \0\1\2\3\4\5\6\7\8\9
%   Exclamation   \!     Double quote  \"     Hash (number) \#
%   Dollar        \$     Percent       \%     Ampersand     \&
%   Acute accent  \'     Left paren    \(     Right paren   \)
%   Asterisk      \*     Plus          \+     Comma         \,
%   Minus         \-     Point         \.     Solidus       \/
%   Colon         \:     Semicolon     \;     Less than     \<
%   Equals        \=     Greater than  \>     Question mark \?
%   Commercial at \@     Left bracket  \[     Backslash     \\
%   Right bracket \]     Circumflex    \^     Underscore    \_
%   Grave accent  \`     Left brace    \{     Vertical bar  \|
%   Right brace   \}     Tilde         \~}
%
% \GetFileInfo{catchfile.drv}
%
% \title{The \xpackage{catchfile} package}
% \date{2016/05/16 v1.7}
% \author{Heiko Oberdiek\thanks
% {Please report any issues at https://github.com/ho-tex/oberdiek/issues}\\
% \xemail{heiko.oberdiek at googlemail.com}}
%
% \maketitle
%
% \begin{abstract}
% This package catches the contents of a file and puts it in a macro.
% It requires \eTeX. Both \LaTeX\ and \plainTeX\ are supported.
% \end{abstract}
%
% \tableofcontents
%
% \section{Documentation}
%
% The package relies on \eTeX's \cs{everyeof}. Otherwise it
% aborts with an error message.
%
% \begin{declcs}{CatchFileDef} \M{cmd} \M{file name} \M{setup}\\
% \cs{CatchFileEdef} \M{cmd} \M{file name} \M{setup}
% \end{declcs}
% Macro \meta{cmd} is defined with the contents
% of file \meta{file name}.
% \cs{CatchFileDef} uses \cs{def}, \cs{CatchFileEdef} \cs{edef}
% for the definition.
% Additional setup code for setting
% catcodes or treatment of line ends can be given in code
% \meta{setup}. See the test files for an example.
%
% \StopEventually{
% }
%
% \section{Implementation}
%
%    \begin{macrocode}
%<*package>
%    \end{macrocode}
%
% \subsection{Reload check and package identification}
%    Reload check, especially if the package is not used with \LaTeX.
%    \begin{macrocode}
\begingroup\catcode61\catcode48\catcode32=10\relax%
  \catcode13=5 % ^^M
  \endlinechar=13 %
  \catcode35=6 % #
  \catcode39=12 % '
  \catcode44=12 % ,
  \catcode45=12 % -
  \catcode46=12 % .
  \catcode58=12 % :
  \catcode64=11 % @
  \catcode123=1 % {
  \catcode125=2 % }
  \expandafter\let\expandafter\x\csname ver@catchfile.sty\endcsname
  \ifx\x\relax % plain-TeX, first loading
  \else
    \def\empty{}%
    \ifx\x\empty % LaTeX, first loading,
      % variable is initialized, but \ProvidesPackage not yet seen
    \else
      \expandafter\ifx\csname PackageInfo\endcsname\relax
        \def\x#1#2{%
          \immediate\write-1{Package #1 Info: #2.}%
        }%
      \else
        \def\x#1#2{\PackageInfo{#1}{#2, stopped}}%
      \fi
      \x{catchfile}{The package is already loaded}%
      \aftergroup\endinput
    \fi
  \fi
\endgroup%
%    \end{macrocode}
%    Package identification:
%    \begin{macrocode}
\begingroup\catcode61\catcode48\catcode32=10\relax%
  \catcode13=5 % ^^M
  \endlinechar=13 %
  \catcode35=6 % #
  \catcode39=12 % '
  \catcode40=12 % (
  \catcode41=12 % )
  \catcode44=12 % ,
  \catcode45=12 % -
  \catcode46=12 % .
  \catcode47=12 % /
  \catcode58=12 % :
  \catcode64=11 % @
  \catcode91=12 % [
  \catcode93=12 % ]
  \catcode123=1 % {
  \catcode125=2 % }
  \expandafter\ifx\csname ProvidesPackage\endcsname\relax
    \def\x#1#2#3[#4]{\endgroup
      \immediate\write-1{Package: #3 #4}%
      \xdef#1{#4}%
    }%
  \else
    \def\x#1#2[#3]{\endgroup
      #2[{#3}]%
      \ifx#1\@undefined
        \xdef#1{#3}%
      \fi
      \ifx#1\relax
        \xdef#1{#3}%
      \fi
    }%
  \fi
\expandafter\x\csname ver@catchfile.sty\endcsname
\ProvidesPackage{catchfile}%
  [2016/05/16 v1.7 Catch the contents of a file (HO)]%
%    \end{macrocode}
%
% \subsection{Catcodes}
%
%    \begin{macrocode}
\begingroup\catcode61\catcode48\catcode32=10\relax%
  \catcode13=5 % ^^M
  \endlinechar=13 %
  \catcode123=1 % {
  \catcode125=2 % }
  \catcode64=11 % @
  \def\x{\endgroup
    \expandafter\edef\csname CatchFile@AtEnd\endcsname{%
      \endlinechar=\the\endlinechar\relax
      \catcode13=\the\catcode13\relax
      \catcode32=\the\catcode32\relax
      \catcode35=\the\catcode35\relax
      \catcode61=\the\catcode61\relax
      \catcode64=\the\catcode64\relax
      \catcode123=\the\catcode123\relax
      \catcode125=\the\catcode125\relax
    }%
  }%
\x\catcode61\catcode48\catcode32=10\relax%
\catcode13=5 % ^^M
\endlinechar=13 %
\catcode35=6 % #
\catcode64=11 % @
\catcode123=1 % {
\catcode125=2 % }
\def\TMP@EnsureCode#1#2{%
  \edef\CatchFile@AtEnd{%
    \CatchFile@AtEnd
    \catcode#1=\the\catcode#1\relax
  }%
  \catcode#1=#2\relax
}
\TMP@EnsureCode{39}{12}% '
\TMP@EnsureCode{44}{12}% ,
\TMP@EnsureCode{45}{12}% -
\TMP@EnsureCode{46}{12}% .
\TMP@EnsureCode{47}{12}% /
\TMP@EnsureCode{91}{12}% [
\TMP@EnsureCode{93}{12}% ]
\TMP@EnsureCode{96}{12}% `
\edef\CatchFile@AtEnd{\CatchFile@AtEnd\noexpand\endinput}
%    \end{macrocode}
%
% \subsection{Preparations}
%
%    \begin{macrocode}
\begingroup\expandafter\expandafter\expandafter\endgroup
\expandafter\ifx\csname RequirePackage\endcsname\relax
  \input infwarerr.sty\relax
  \input ltxcmds.sty\relax
\else
  \RequirePackage{infwarerr}[2007/09/09]%
  \RequirePackage{ltxcmds}[2010/03/09]%
\fi
%    \end{macrocode}
%
%    Check for \eTeX's \cs{everyeof}.
%    \begin{macrocode}
\begingroup
  \escapechar=92\relax
  \edef\TestString{\string\everyeof}%
  \edef\TestMeaning{\meaning\everyeof}%
  \ifx\TestString\TestMeaning
  \else
    \@PackageError{catchfile}{%
      Cannot find e-TeX's \string\everyeof,\MessageBreak
      package loading is aborted%
    }\@ehd
    \endgroup
    \expandafter\CatchFile@AtEnd
  \fi%
\endgroup
%    \end{macrocode}
%
% \subsection{Looking for primitive \cs{input}}
%
%    \begin{macro}{\CatchFile@Input}
%    The package needs the expandable primitive \cs{input}.
%    However there are formats that redefine it. For example,
%    \LaTeX's \cs{input} is not expandable, but it stores
%    the primitive in \cs{@@input}. The third possibility
%    is \cs{pdfprimitive}, introduced in \pdfTeX\ 1.40.0.
%
%    Thus we try to find the primitive and store it in
%    \cs{CatchFile@Input}. If it is used, it must be expanded
%    twice (because of the solution with \cs{pdfprimitive}).
%    \begin{macrocode}
\begingroup
  \def\Check#1#2#3#4\endgroup{%
    \edef\TestString{\string#1}%
    \edef\TestMeaning{\meaning#2}%
    \ifx\TestString\TestMeaning
      \endgroup
      \let\CatchFile@Primitive#2%
      \def\CatchFile@Input{\CatchFile@Primitive#3}%
    \else
      #4\endgroup
    \fi
  }%
  \Check\input\input{}%
  \Check\input\@@input{}%
  \Check\pdfprimitive\pdfprimitive\input
  \@PackageError{%
    Cannot find primitive \string\input,\MessageBreak
    package loading is aborted%
  }\@ehd
  \csname endgroup\endcsname
  \CatchFile@AtEnd%
\endgroup
%    \end{macrocode}
%    \end{macro}
%
% \subsection{Input file check}
%
%    \begin{macro}{\CatchFile@CheckFileExists}
%    \begin{macrocode}
\begingroup\expandafter\expandafter\expandafter\endgroup
\expandafter\ifx\csname IfFileExists\endcsname\relax
  \input pdftexcmds.sty\relax
  \begingroup\expandafter\expandafter\expandafter\endgroup
  \expandafter\ifx\csname pdf@filesize\endcsname\relax
    \def\CatchFile@CheckFileExists#1{%
      \expandafter\ifx\csname @inputcheck\endcsname\relax
        \csname newread\endcsname\@inputcheck
      \fi
      \openin\@inputcheck#1\relax
      \expandafter\closein\expandafter\@inputcheck
      \ifeof\@inputcheck
        \let\CatchFile@File\relax
      \else
        \def\CatchFile@File{#1}%
      \fi
    }%
  \else
    \def\CatchFile@CheckFileExists#1{%
      \expandafter\expandafter\expandafter\ifx
      \expandafter\expandafter\expandafter\relax\pdf@filesize{#1}\relax
        \let\CatchFile@File\relax
      \else
        \def\CatchFile@File{#1}%
      \fi
    }%
  \fi
\else
  \def\CatchFile@CheckFileExists#1{%
    \IfFileExists{#1}{%
      \expandafter\CatchFile@DefFile\@filef@und\@nil
      \begingroup\expandafter\expandafter\expandafter\endgroup
      \expandafter\ifx\csname @addtofilelist\endcsname\relax
      \else
        \@addtofilelist\CatchFile@File
      \fi
    }{%
      \let\CatchFile@File\relax
    }%
  }%
  \def\CatchFile@DefFile#1 \@nil{%
    \def\CatchFile@File{#1}%
  }%
\fi
%    \end{macrocode}
%    \end{macro}
%
%    \begin{macro}{\CatchFile@NotFound}
%    \begin{macrocode}
\def\CatchFile@NotFound#1#2{%
  \def#1{}%
  \@PackageError{catchfile}{%
    File `#2' not found%
  }\@ehc
}
%    \end{macrocode}
%    \end{macro}
%
% \subsection{Catch file contents}
%
%    \begin{macrocode}
\ltx@IfUndefined{RequirePackage}{%
  \input etexcmds.sty\relax
}{%
  \RequirePackage{etexcmds}[2010/01/28]%
}
%    \end{macrocode}
%
%    \begin{macro}{\CatchFileEdef}
%    \begin{macrocode}
\long\def\CatchFileEdef#1#2#3{%
  \CatchFile@CheckFileExists{#2}%
  \ifx\CatchFile@File\relax
    \CatchFile@NotFound{#1}{#2}%
  \else
    \begingroup
      \everyeof{\noexpand}%
      #3%
      \xdef\CatchFile@Contents{\CatchFile@Input\CatchFile@File\space}%
    \endgroup
    \let#1\CatchFile@Contents
  \fi
}
%    \end{macrocode}
%    \end{macro}
%
%    \begin{macro}{\CatchFileDef}
%    \begin{macrocode}
\long\def\CatchFileDef#1#2#3{%
  \CatchFile@CheckFileExists{#2}%
  \ifx\CatchFile@File\relax
    \CatchFile@NotFound{#1}{#2}%
  \else
    \begingroup
      \everyeof\expandafter{%
        \CatchFile@EOF
        \expandafter\CatchFile@Finish
        \noexpand
      }%
      \ltx@ifundefined{etex@unexpanded}{%
        \expandafter\long\expandafter\def\expandafter\CatchFile@Do
            \expandafter##\expandafter1\CatchFile@EOF{%
          \toks\ltx@zero{##1}%
          \xdef\CatchFile@gtemp{\the\toks\ltx@zero}%
          \def\CatchFile@Finish{%
            \endgroup
            \let#1\CatchFile@gtemp
            \global\let\CatchFile@gtemp\ltx@undefined
          }%
        }%
      }{%
        \expandafter\long\expandafter\def\expandafter\CatchFile@Do
            \expandafter##\expandafter1\CatchFile@EOF{%
          \edef\CatchFile@Finish{%
            \endgroup
            \etex@unexpanded{%
              \edef#1{\etex@unexpanded{##1}}%
            }%
          }%
        }%
      }%
      #3\relax
    \expandafter\expandafter\expandafter\CatchFile@Do
    \CatchFile@Input\CatchFile@File\relax
  \fi
}
%    \end{macrocode}
%    \cs{relax} after |#3| was added to make it more robust in
%    case the user uses something like
%\begin{verbatim}
%\CatchFileDef{\content}{\jobname.tt}{\endlinechar=-1}
%\end{verbatim}
%    that expands the following \cs{expandafter} after |#3| prematurely
%    (contribution of Martin Scharrer).
%    \end{macro}
%    \begin{macro}{\CatchFile@EOF}
%    If the file is read the catcode mappings are fixed.
%    This means that the same character cannot occur
%    inside the file with different catcodes.
%    Thus we use as end of file marker the at sign
%    twice with different catcodes.
%    \begin{macrocode}
\begingroup
  \lccode65=64 % lowercase('A') = '@'
  \lccode66=64 % lowercase('B') = '@'
  \catcode65=8 % catcode('A') = subscript
  \catcode66=3 % catcode('B') = math shift
\lowercase{\endgroup
  \def\CatchFile@EOF{AB}%
}
%    \end{macrocode}
%    \end{macro}
%
%    \begin{macrocode}
\CatchFile@AtEnd%
%</package>
%    \end{macrocode}
%
% \section{Test}
%
% \subsection{Catcode checks for loading}
%
%    \begin{macrocode}
%<*test1>
%    \end{macrocode}
%    \begin{macrocode}
\catcode`\{=1 %
\catcode`\}=2 %
\catcode`\#=6 %
\catcode`\@=11 %
\expandafter\ifx\csname count@\endcsname\relax
  \countdef\count@=255 %
\fi
\expandafter\ifx\csname @gobble\endcsname\relax
  \long\def\@gobble#1{}%
\fi
\expandafter\ifx\csname @firstofone\endcsname\relax
  \long\def\@firstofone#1{#1}%
\fi
\expandafter\ifx\csname loop\endcsname\relax
  \expandafter\@firstofone
\else
  \expandafter\@gobble
\fi
{%
  \def\loop#1\repeat{%
    \def\body{#1}%
    \iterate
  }%
  \def\iterate{%
    \body
      \let\next\iterate
    \else
      \let\next\relax
    \fi
    \next
  }%
  \let\repeat=\fi
}%
\def\RestoreCatcodes{}
\count@=0 %
\loop
  \edef\RestoreCatcodes{%
    \RestoreCatcodes
    \catcode\the\count@=\the\catcode\count@\relax
  }%
\ifnum\count@<255 %
  \advance\count@ 1 %
\repeat

\def\RangeCatcodeInvalid#1#2{%
  \count@=#1\relax
  \loop
    \catcode\count@=15 %
  \ifnum\count@<#2\relax
    \advance\count@ 1 %
  \repeat
}
\def\RangeCatcodeCheck#1#2#3{%
  \count@=#1\relax
  \loop
    \ifnum#3=\catcode\count@
    \else
      \errmessage{%
        Character \the\count@\space
        with wrong catcode \the\catcode\count@\space
        instead of \number#3%
      }%
    \fi
  \ifnum\count@<#2\relax
    \advance\count@ 1 %
  \repeat
}
\def\space{ }
\expandafter\ifx\csname LoadCommand\endcsname\relax
  \def\LoadCommand{\input catchfile.sty\relax}%
\fi
\def\Test{%
  \RangeCatcodeInvalid{0}{47}%
  \RangeCatcodeInvalid{58}{64}%
  \RangeCatcodeInvalid{91}{96}%
  \RangeCatcodeInvalid{123}{255}%
  \catcode`\@=12 %
  \catcode`\\=0 %
  \catcode`\%=14 %
  \LoadCommand
  \RangeCatcodeCheck{0}{36}{15}%
  \RangeCatcodeCheck{37}{37}{14}%
  \RangeCatcodeCheck{38}{47}{15}%
  \RangeCatcodeCheck{48}{57}{12}%
  \RangeCatcodeCheck{58}{63}{15}%
  \RangeCatcodeCheck{64}{64}{12}%
  \RangeCatcodeCheck{65}{90}{11}%
  \RangeCatcodeCheck{91}{91}{15}%
  \RangeCatcodeCheck{92}{92}{0}%
  \RangeCatcodeCheck{93}{96}{15}%
  \RangeCatcodeCheck{97}{122}{11}%
  \RangeCatcodeCheck{123}{255}{15}%
  \RestoreCatcodes
}
\Test
\csname @@end\endcsname
\end
%    \end{macrocode}
%    \begin{macrocode}
%</test1>
%    \end{macrocode}
%
% \subsection{\LaTeX}
%
%    \begin{macrocode}
%<*test2>
\NeedsTeXFormat{LaTeX2e}
\nofiles
\listfiles
\tracingnesting=2 %
\documentclass{minimal}
\usepackage{catchfile}[2016/05/16]
\makeatletter
\def\mysetup{%
  \let\do\@makeother
  \dospecials
}
\def\CheckContents{%
  \begingroup
    \expandafter\ifx\csname contents\endcsname\relax
      \PackageError{TEST}{\string\contents\space is not defined}\@ehc
    \fi
  \endgroup
}
\def\StartDisableUnexpanded{%
  \begingroup
    \let\etex@unexpanded\@undefined
}
\def\StopDisableUnexpanded{%
  \endgroup
}
\def\CheckCleanup{%
  \begingroup
    \edef\x{\the\toks0}%
    \def\y{ABC}%
    \ifx\x\y
    \else
      \PackageError{TEST}{\string\toks0 has changed}\@ehc
    \fi
    \ifx\CatchFile@gtemp\@undefined
    \else
      \PackageError{TEST}{Left over \string\CatchFile@gtemp}\@ehc
    \fi
  \endgroup
}
\def\Check{%
  \CheckContents
  \CheckCleanup
}
\makeatother
\begin{document}
\toks0{ABC}

\CatchFileDef\contents{catchfile.sty}\mysetup
\typeout{\meaning\contents}
\Check
\typeout{***********************************************************}%
\CatchFileDef\contents{catchfile.sty}{}
\typeout{\meaning\contents}
\Check
\typeout{***********************************************************}%

\StartDisableUnexpanded
\CatchFileDef\contents{catchfile.sty}\mysetup
\typeout{\meaning\contents}
\Check
\typeout{***********************************************************}%
\CatchFileDef\contents{catchfile.sty}{}
\typeout{\meaning\contents}
\Check
\typeout{***********************************************************}%
\StopDisableUnexpanded

\CatchFileEdef\contents{catchfile.sty}{%
  \mysetup
  \def\par{^^J}%
  \obeylines
}
\typeout{\contents}
\Check
\typeout{***********************************************************}%
\CatchFileEdef\contents{catchfile.sty}{%
  \catcode`\\=12 %
  \catcode`\#=12 %
  \def\par{^^J}%
  \obeylines
}
\typeout{\contents}
\Check
\typeout{***********************************************************}%
\end{document}
%</test2>
%    \end{macrocode}
%
% \subsection{\plainTeX}
%
%    \begin{macrocode}
%<*test3>
\def\msg#{\immediate\write16}
\newlinechar=10 %
\tracingnesting=2 %
\input catchfile.sty\relax

\def\mysetup{%
  \def\do##1{%
    \catcode`##1=12\relax
  }%
  \dospecials
}
\def\CheckContents{%
  \begingroup
    \expandafter\ifx\csname contents\endcsname\relax
      \def\space{ }%
      \errmessage{\string\contents\space is not defined.}%
    \fi
  \endgroup
}
\catcode`\@=11 %
\def\CheckCleanup{%
  \begingroup
    \edef\x{\the\toks0}%
    \def\y{ABC}%
    \ifx\x\y
    \else
      \errmessage{\string\toks0 has changed}%
    \fi
    \ifx\CatchFile@gtemp\@undefined
    \else
      \errmessage{Left over \string\CatchFile@gtemp}%
    \fi
  \endgroup
}
\def\Check{%
  \CheckContents
  \CheckCleanup
}
\def\StartDisableUnexpanded{%
  \begingroup
    \let\etex@unexpanded\@undefined
}
\def\StopDisableUnexpanded{%
  \endgroup
}
\catcode`\@=12 %

\toks0{ABC}

\CatchFileDef\contents{catchfile.sty}\mysetup
\msg{\meaning\contents}
\Check
\msg{***********************************************************}%
\CatchFileDef\contents{catchfile.sty}{}
\msg{\meaning\contents}
\Check
\msg{***********************************************************}%

\StartDisableUnexpanded
\CatchFileDef\contents{catchfile.sty}\mysetup
\msg{\meaning\contents}
\Check
\msg{***********************************************************}%
\CatchFileDef\contents{catchfile.sty}{}
\msg{\meaning\contents}
\Check
\msg{***********************************************************}%
\StopDisableUnexpanded

\CatchFileEdef\contents{catchfile.sty}{%
  \mysetup
  \def\par{^^J}%
  \obeylines
}
\msg{\contents}
\Check
\msg{***********************************************************}%
\CatchFileEdef\contents{catchfile.sty}{%
  \catcode`\\=12 %
  \catcode`\#=12 %
  \def\par{^^J}%
  \obeylines
}
\msg{\contents}
\Check
\msg{***********************************************************}%

\csname @@end\endcsname
\end
%</test3>
%    \end{macrocode}
%
% \section{Installation}
%
% \subsection{Download}
%
% \paragraph{Package.} This package is available on
% CTAN\footnote{\url{https://ctan.org/pkg/catchfile}}:
% \begin{description}
% \item[\CTAN{macros/latex/contrib/oberdiek/catchfile.dtx}] The source file.
% \item[\CTAN{macros/latex/contrib/oberdiek/catchfile.pdf}] Documentation.
% \end{description}
%
%
% \paragraph{Bundle.} All the packages of the bundle `oberdiek'
% are also available in a TDS compliant ZIP archive. There
% the packages are already unpacked and the documentation files
% are generated. The files and directories obey the TDS standard.
% \begin{description}
% \item[\CTANinstall{install/macros/latex/contrib/oberdiek.tds.zip}]
% \end{description}
% \emph{TDS} refers to the standard ``A Directory Structure
% for \TeX\ Files'' (\CTAN{tds/tds.pdf}). Directories
% with \xfile{texmf} in their name are usually organized this way.
%
% \subsection{Bundle installation}
%
% \paragraph{Unpacking.} Unpack the \xfile{oberdiek.tds.zip} in the
% TDS tree (also known as \xfile{texmf} tree) of your choice.
% Example (linux):
% \begin{quote}
%   |unzip oberdiek.tds.zip -d ~/texmf|
% \end{quote}
%
% \paragraph{Script installation.}
% Check the directory \xfile{TDS:scripts/oberdiek/} for
% scripts that need further installation steps.
% Package \xpackage{attachfile2} comes with the Perl script
% \xfile{pdfatfi.pl} that should be installed in such a way
% that it can be called as \texttt{pdfatfi}.
% Example (linux):
% \begin{quote}
%   |chmod +x scripts/oberdiek/pdfatfi.pl|\\
%   |cp scripts/oberdiek/pdfatfi.pl /usr/local/bin/|
% \end{quote}
%
% \subsection{Package installation}
%
% \paragraph{Unpacking.} The \xfile{.dtx} file is a self-extracting
% \docstrip\ archive. The files are extracted by running the
% \xfile{.dtx} through \plainTeX:
% \begin{quote}
%   \verb|tex catchfile.dtx|
% \end{quote}
%
% \paragraph{TDS.} Now the different files must be moved into
% the different directories in your installation TDS tree
% (also known as \xfile{texmf} tree):
% \begin{quote}
% \def\t{^^A
% \begin{tabular}{@{}>{\ttfamily}l@{ $\rightarrow$ }>{\ttfamily}l@{}}
%   catchfile.sty & tex/generic/oberdiek/catchfile.sty\\
%   catchfile.pdf & doc/latex/oberdiek/catchfile.pdf\\
%   test/catchfile-test1.tex & doc/latex/oberdiek/test/catchfile-test1.tex\\
%   test/catchfile-test2.tex & doc/latex/oberdiek/test/catchfile-test2.tex\\
%   test/catchfile-test3.tex & doc/latex/oberdiek/test/catchfile-test3.tex\\
%   catchfile.dtx & source/latex/oberdiek/catchfile.dtx\\
% \end{tabular}^^A
% }^^A
% \sbox0{\t}^^A
% \ifdim\wd0>\linewidth
%   \begingroup
%     \advance\linewidth by\leftmargin
%     \advance\linewidth by\rightmargin
%   \edef\x{\endgroup
%     \def\noexpand\lw{\the\linewidth}^^A
%   }\x
%   \def\lwbox{^^A
%     \leavevmode
%     \hbox to \linewidth{^^A
%       \kern-\leftmargin\relax
%       \hss
%       \usebox0
%       \hss
%       \kern-\rightmargin\relax
%     }^^A
%   }^^A
%   \ifdim\wd0>\lw
%     \sbox0{\small\t}^^A
%     \ifdim\wd0>\linewidth
%       \ifdim\wd0>\lw
%         \sbox0{\footnotesize\t}^^A
%         \ifdim\wd0>\linewidth
%           \ifdim\wd0>\lw
%             \sbox0{\scriptsize\t}^^A
%             \ifdim\wd0>\linewidth
%               \ifdim\wd0>\lw
%                 \sbox0{\tiny\t}^^A
%                 \ifdim\wd0>\linewidth
%                   \lwbox
%                 \else
%                   \usebox0
%                 \fi
%               \else
%                 \lwbox
%               \fi
%             \else
%               \usebox0
%             \fi
%           \else
%             \lwbox
%           \fi
%         \else
%           \usebox0
%         \fi
%       \else
%         \lwbox
%       \fi
%     \else
%       \usebox0
%     \fi
%   \else
%     \lwbox
%   \fi
% \else
%   \usebox0
% \fi
% \end{quote}
% If you have a \xfile{docstrip.cfg} that configures and enables \docstrip's
% TDS installing feature, then some files can already be in the right
% place, see the documentation of \docstrip.
%
% \subsection{Refresh file name databases}
%
% If your \TeX~distribution
% (\teTeX, \mikTeX, \dots) relies on file name databases, you must refresh
% these. For example, \teTeX\ users run \verb|texhash| or
% \verb|mktexlsr|.
%
% \subsection{Some details for the interested}
%
% \paragraph{Attached source.}
%
% The PDF documentation on CTAN also includes the
% \xfile{.dtx} source file. It can be extracted by
% AcrobatReader 6 or higher. Another option is \textsf{pdftk},
% e.g. unpack the file into the current directory:
% \begin{quote}
%   \verb|pdftk catchfile.pdf unpack_files output .|
% \end{quote}
%
% \paragraph{Unpacking with \LaTeX.}
% The \xfile{.dtx} chooses its action depending on the format:
% \begin{description}
% \item[\plainTeX:] Run \docstrip\ and extract the files.
% \item[\LaTeX:] Generate the documentation.
% \end{description}
% If you insist on using \LaTeX\ for \docstrip\ (really,
% \docstrip\ does not need \LaTeX), then inform the autodetect routine
% about your intention:
% \begin{quote}
%   \verb|latex \let\install=y\input{catchfile.dtx}|
% \end{quote}
% Do not forget to quote the argument according to the demands
% of your shell.
%
% \paragraph{Generating the documentation.}
% You can use both the \xfile{.dtx} or the \xfile{.drv} to generate
% the documentation. The process can be configured by the
% configuration file \xfile{ltxdoc.cfg}. For instance, put this
% line into this file, if you want to have A4 as paper format:
% \begin{quote}
%   \verb|\PassOptionsToClass{a4paper}{article}|
% \end{quote}
% An example follows how to generate the
% documentation with pdf\LaTeX:
% \begin{quote}
%\begin{verbatim}
%pdflatex catchfile.dtx
%makeindex -s gind.ist catchfile.idx
%pdflatex catchfile.dtx
%makeindex -s gind.ist catchfile.idx
%pdflatex catchfile.dtx
%\end{verbatim}
% \end{quote}
%
% \begin{History}
%   \begin{Version}{2007/05/30 v1.0}
%   \item
%     First version.
%   \end{Version}
%   \begin{Version}{2007/09/09 v1.1}
%   \item
%     Catcode section rewritten.
%   \end{Version}
%   \begin{Version}{2007/11/11 v1.2}
%   \item
%     Use of package \xpackage{pdftexcmds} for \LuaTeX\ support.
%   \end{Version}
%   \begin{Version}{2010/03/01 v1.3}
%   \item
%     Fix for unknown \cs{@PackageErrorNoLine}.
%   \end{Version}
%   \begin{Version}{2010/04/08 v1.4}
%   \item
%     \cs{closein} also added if \cs{ifeof} is true.
%   \end{Version}
%   \begin{Version}{2010/04/28 v1.5}
%   \item
%     \cs{CatchFileDef}: Getting rid of warning `end of semi simple group
%     entered at line \dots\ of
%     a different file (Florent Chervet).
%   \item
%     \cs{CatchFileDef}: Fix for error `Illegal parameter number in
%     definition of \dots' (HO) including improvement that uses
%     \cs{unexpanded} if available (Florent Chervet).
%   \end{Version}
%   \begin{Version}{2011/03/01 v1.6}
%   \item
%     \cs{relax} added after the setup argument of
%     \cs{CatchFileDef} to prevent premature file reading
%     (Martin Scharrer).
%   \end{Version}
%   \begin{Version}{2016/05/16 v1.7}
%   \item
%     Documentation updates.
%   \end{Version}
% \end{History}
%
% \PrintIndex
%
% \Finale
\endinput
|
% \end{quote}
% Do not forget to quote the argument according to the demands
% of your shell.
%
% \paragraph{Generating the documentation.}
% You can use both the \xfile{.dtx} or the \xfile{.drv} to generate
% the documentation. The process can be configured by the
% configuration file \xfile{ltxdoc.cfg}. For instance, put this
% line into this file, if you want to have A4 as paper format:
% \begin{quote}
%   \verb|\PassOptionsToClass{a4paper}{article}|
% \end{quote}
% An example follows how to generate the
% documentation with pdf\LaTeX:
% \begin{quote}
%\begin{verbatim}
%pdflatex catchfile.dtx
%makeindex -s gind.ist catchfile.idx
%pdflatex catchfile.dtx
%makeindex -s gind.ist catchfile.idx
%pdflatex catchfile.dtx
%\end{verbatim}
% \end{quote}
%
% \begin{History}
%   \begin{Version}{2007/05/30 v1.0}
%   \item
%     First version.
%   \end{Version}
%   \begin{Version}{2007/09/09 v1.1}
%   \item
%     Catcode section rewritten.
%   \end{Version}
%   \begin{Version}{2007/11/11 v1.2}
%   \item
%     Use of package \xpackage{pdftexcmds} for \LuaTeX\ support.
%   \end{Version}
%   \begin{Version}{2010/03/01 v1.3}
%   \item
%     Fix for unknown \cs{@PackageErrorNoLine}.
%   \end{Version}
%   \begin{Version}{2010/04/08 v1.4}
%   \item
%     \cs{closein} also added if \cs{ifeof} is true.
%   \end{Version}
%   \begin{Version}{2010/04/28 v1.5}
%   \item
%     \cs{CatchFileDef}: Getting rid of warning `end of semi simple group
%     entered at line \dots\ of
%     a different file (Florent Chervet).
%   \item
%     \cs{CatchFileDef}: Fix for error `Illegal parameter number in
%     definition of \dots' (HO) including improvement that uses
%     \cs{unexpanded} if available (Florent Chervet).
%   \end{Version}
%   \begin{Version}{2011/03/01 v1.6}
%   \item
%     \cs{relax} added after the setup argument of
%     \cs{CatchFileDef} to prevent premature file reading
%     (Martin Scharrer).
%   \end{Version}
%   \begin{Version}{2016/05/16 v1.7}
%   \item
%     Documentation updates.
%   \end{Version}
% \end{History}
%
% \PrintIndex
%
% \Finale
\endinput

%        (quote the arguments according to the demands of your shell)
%
% Documentation:
%    (a) If catchfile.drv is present:
%           latex catchfile.drv
%    (b) Without catchfile.drv:
%           latex catchfile.dtx; ...
%    The class ltxdoc loads the configuration file ltxdoc.cfg
%    if available. Here you can specify further options, e.g.
%    use A4 as paper format:
%       \PassOptionsToClass{a4paper}{article}
%
%    Programm calls to get the documentation (example):
%       pdflatex catchfile.dtx
%       makeindex -s gind.ist catchfile.idx
%       pdflatex catchfile.dtx
%       makeindex -s gind.ist catchfile.idx
%       pdflatex catchfile.dtx
%
% Installation:
%    TDS:tex/generic/oberdiek/catchfile.sty
%    TDS:doc/latex/oberdiek/catchfile.pdf
%    TDS:doc/latex/oberdiek/test/catchfile-test1.tex
%    TDS:doc/latex/oberdiek/test/catchfile-test2.tex
%    TDS:doc/latex/oberdiek/test/catchfile-test3.tex
%    TDS:source/latex/oberdiek/catchfile.dtx
%
%<*ignore>
\begingroup
  \catcode123=1 %
  \catcode125=2 %
  \def\x{LaTeX2e}%
\expandafter\endgroup
\ifcase 0\ifx\install y1\fi\expandafter
         \ifx\csname processbatchFile\endcsname\relax\else1\fi
         \ifx\fmtname\x\else 1\fi\relax
\else\csname fi\endcsname
%</ignore>
%<*install>
\input docstrip.tex
\Msg{************************************************************************}
\Msg{* Installation}
\Msg{* Package: catchfile 2016/05/16 v1.7 Catch the contents of a file (HO)}
\Msg{************************************************************************}

\keepsilent
\askforoverwritefalse

\let\MetaPrefix\relax
\preamble

This is a generated file.

Project: catchfile
Version: 2016/05/16 v1.7

Copyright (C) 2007, 2010, 2011 by
   Heiko Oberdiek <heiko.oberdiek at googlemail.com>

This work may be distributed and/or modified under the
conditions of the LaTeX Project Public License, either
version 1.3c of this license or (at your option) any later
version. This version of this license is in
   http://www.latex-project.org/lppl/lppl-1-3c.txt
and the latest version of this license is in
   http://www.latex-project.org/lppl.txt
and version 1.3 or later is part of all distributions of
LaTeX version 2005/12/01 or later.

This work has the LPPL maintenance status "maintained".

This Current Maintainer of this work is Heiko Oberdiek.

The Base Interpreter refers to any `TeX-Format',
because some files are installed in TDS:tex/generic//.

This work consists of the main source file catchfile.dtx
and the derived files
   catchfile.sty, catchfile.pdf, catchfile.ins, catchfile.drv,
   catchfile-test1.tex, catchfile-test2.tex,
   catchfile-test3.tex.

\endpreamble
\let\MetaPrefix\DoubleperCent

\generate{%
  \file{catchfile.ins}{\from{catchfile.dtx}{install}}%
  \file{catchfile.drv}{\from{catchfile.dtx}{driver}}%
  \usedir{tex/generic/oberdiek}%
  \file{catchfile.sty}{\from{catchfile.dtx}{package}}%
%  \usedir{doc/latex/oberdiek/test}%
%  \file{catchfile-test1.tex}{\from{catchfile.dtx}{test1}}%
%  \file{catchfile-test2.tex}{\from{catchfile.dtx}{test2}}%
%  \file{catchfile-test3.tex}{\from{catchfile.dtx}{test3}}%
  \nopreamble
  \nopostamble
%  \usedir{source/latex/oberdiek/catalogue}%
%  \file{catchfile.xml}{\from{catchfile.dtx}{catalogue}}%
}

\catcode32=13\relax% active space
\let =\space%
\Msg{************************************************************************}
\Msg{*}
\Msg{* To finish the installation you have to move the following}
\Msg{* file into a directory searched by TeX:}
\Msg{*}
\Msg{*     catchfile.sty}
\Msg{*}
\Msg{* To produce the documentation run the file `catchfile.drv'}
\Msg{* through LaTeX.}
\Msg{*}
\Msg{* Happy TeXing!}
\Msg{*}
\Msg{************************************************************************}

\endbatchfile
%</install>
%<*ignore>
\fi
%</ignore>
%<*driver>
\NeedsTeXFormat{LaTeX2e}
\ProvidesFile{catchfile.drv}%
  [2016/05/16 v1.7 Catch the contents of a file (HO)]%
\documentclass{ltxdoc}
\usepackage{holtxdoc}[2011/11/22]
\begin{document}
  \DocInput{catchfile.dtx}%
\end{document}
%</driver>
% \fi
%
%
% \CharacterTable
%  {Upper-case    \A\B\C\D\E\F\G\H\I\J\K\L\M\N\O\P\Q\R\S\T\U\V\W\X\Y\Z
%   Lower-case    \a\b\c\d\e\f\g\h\i\j\k\l\m\n\o\p\q\r\s\t\u\v\w\x\y\z
%   Digits        \0\1\2\3\4\5\6\7\8\9
%   Exclamation   \!     Double quote  \"     Hash (number) \#
%   Dollar        \$     Percent       \%     Ampersand     \&
%   Acute accent  \'     Left paren    \(     Right paren   \)
%   Asterisk      \*     Plus          \+     Comma         \,
%   Minus         \-     Point         \.     Solidus       \/
%   Colon         \:     Semicolon     \;     Less than     \<
%   Equals        \=     Greater than  \>     Question mark \?
%   Commercial at \@     Left bracket  \[     Backslash     \\
%   Right bracket \]     Circumflex    \^     Underscore    \_
%   Grave accent  \`     Left brace    \{     Vertical bar  \|
%   Right brace   \}     Tilde         \~}
%
% \GetFileInfo{catchfile.drv}
%
% \title{The \xpackage{catchfile} package}
% \date{2016/05/16 v1.7}
% \author{Heiko Oberdiek\thanks
% {Please report any issues at https://github.com/ho-tex/oberdiek/issues}\\
% \xemail{heiko.oberdiek at googlemail.com}}
%
% \maketitle
%
% \begin{abstract}
% This package catches the contents of a file and puts it in a macro.
% It requires \eTeX. Both \LaTeX\ and \plainTeX\ are supported.
% \end{abstract}
%
% \tableofcontents
%
% \section{Documentation}
%
% The package relies on \eTeX's \cs{everyeof}. Otherwise it
% aborts with an error message.
%
% \begin{declcs}{CatchFileDef} \M{cmd} \M{file name} \M{setup}\\
% \cs{CatchFileEdef} \M{cmd} \M{file name} \M{setup}
% \end{declcs}
% Macro \meta{cmd} is defined with the contents
% of file \meta{file name}.
% \cs{CatchFileDef} uses \cs{def}, \cs{CatchFileEdef} \cs{edef}
% for the definition.
% Additional setup code for setting
% catcodes or treatment of line ends can be given in code
% \meta{setup}. See the test files for an example.
%
% \StopEventually{
% }
%
% \section{Implementation}
%
%    \begin{macrocode}
%<*package>
%    \end{macrocode}
%
% \subsection{Reload check and package identification}
%    Reload check, especially if the package is not used with \LaTeX.
%    \begin{macrocode}
\begingroup\catcode61\catcode48\catcode32=10\relax%
  \catcode13=5 % ^^M
  \endlinechar=13 %
  \catcode35=6 % #
  \catcode39=12 % '
  \catcode44=12 % ,
  \catcode45=12 % -
  \catcode46=12 % .
  \catcode58=12 % :
  \catcode64=11 % @
  \catcode123=1 % {
  \catcode125=2 % }
  \expandafter\let\expandafter\x\csname ver@catchfile.sty\endcsname
  \ifx\x\relax % plain-TeX, first loading
  \else
    \def\empty{}%
    \ifx\x\empty % LaTeX, first loading,
      % variable is initialized, but \ProvidesPackage not yet seen
    \else
      \expandafter\ifx\csname PackageInfo\endcsname\relax
        \def\x#1#2{%
          \immediate\write-1{Package #1 Info: #2.}%
        }%
      \else
        \def\x#1#2{\PackageInfo{#1}{#2, stopped}}%
      \fi
      \x{catchfile}{The package is already loaded}%
      \aftergroup\endinput
    \fi
  \fi
\endgroup%
%    \end{macrocode}
%    Package identification:
%    \begin{macrocode}
\begingroup\catcode61\catcode48\catcode32=10\relax%
  \catcode13=5 % ^^M
  \endlinechar=13 %
  \catcode35=6 % #
  \catcode39=12 % '
  \catcode40=12 % (
  \catcode41=12 % )
  \catcode44=12 % ,
  \catcode45=12 % -
  \catcode46=12 % .
  \catcode47=12 % /
  \catcode58=12 % :
  \catcode64=11 % @
  \catcode91=12 % [
  \catcode93=12 % ]
  \catcode123=1 % {
  \catcode125=2 % }
  \expandafter\ifx\csname ProvidesPackage\endcsname\relax
    \def\x#1#2#3[#4]{\endgroup
      \immediate\write-1{Package: #3 #4}%
      \xdef#1{#4}%
    }%
  \else
    \def\x#1#2[#3]{\endgroup
      #2[{#3}]%
      \ifx#1\@undefined
        \xdef#1{#3}%
      \fi
      \ifx#1\relax
        \xdef#1{#3}%
      \fi
    }%
  \fi
\expandafter\x\csname ver@catchfile.sty\endcsname
\ProvidesPackage{catchfile}%
  [2016/05/16 v1.7 Catch the contents of a file (HO)]%
%    \end{macrocode}
%
% \subsection{Catcodes}
%
%    \begin{macrocode}
\begingroup\catcode61\catcode48\catcode32=10\relax%
  \catcode13=5 % ^^M
  \endlinechar=13 %
  \catcode123=1 % {
  \catcode125=2 % }
  \catcode64=11 % @
  \def\x{\endgroup
    \expandafter\edef\csname CatchFile@AtEnd\endcsname{%
      \endlinechar=\the\endlinechar\relax
      \catcode13=\the\catcode13\relax
      \catcode32=\the\catcode32\relax
      \catcode35=\the\catcode35\relax
      \catcode61=\the\catcode61\relax
      \catcode64=\the\catcode64\relax
      \catcode123=\the\catcode123\relax
      \catcode125=\the\catcode125\relax
    }%
  }%
\x\catcode61\catcode48\catcode32=10\relax%
\catcode13=5 % ^^M
\endlinechar=13 %
\catcode35=6 % #
\catcode64=11 % @
\catcode123=1 % {
\catcode125=2 % }
\def\TMP@EnsureCode#1#2{%
  \edef\CatchFile@AtEnd{%
    \CatchFile@AtEnd
    \catcode#1=\the\catcode#1\relax
  }%
  \catcode#1=#2\relax
}
\TMP@EnsureCode{39}{12}% '
\TMP@EnsureCode{44}{12}% ,
\TMP@EnsureCode{45}{12}% -
\TMP@EnsureCode{46}{12}% .
\TMP@EnsureCode{47}{12}% /
\TMP@EnsureCode{91}{12}% [
\TMP@EnsureCode{93}{12}% ]
\TMP@EnsureCode{96}{12}% `
\edef\CatchFile@AtEnd{\CatchFile@AtEnd\noexpand\endinput}
%    \end{macrocode}
%
% \subsection{Preparations}
%
%    \begin{macrocode}
\begingroup\expandafter\expandafter\expandafter\endgroup
\expandafter\ifx\csname RequirePackage\endcsname\relax
  \input infwarerr.sty\relax
  \input ltxcmds.sty\relax
\else
  \RequirePackage{infwarerr}[2007/09/09]%
  \RequirePackage{ltxcmds}[2010/03/09]%
\fi
%    \end{macrocode}
%
%    Check for \eTeX's \cs{everyeof}.
%    \begin{macrocode}
\begingroup
  \escapechar=92\relax
  \edef\TestString{\string\everyeof}%
  \edef\TestMeaning{\meaning\everyeof}%
  \ifx\TestString\TestMeaning
  \else
    \@PackageError{catchfile}{%
      Cannot find e-TeX's \string\everyeof,\MessageBreak
      package loading is aborted%
    }\@ehd
    \endgroup
    \expandafter\CatchFile@AtEnd
  \fi%
\endgroup
%    \end{macrocode}
%
% \subsection{Looking for primitive \cs{input}}
%
%    \begin{macro}{\CatchFile@Input}
%    The package needs the expandable primitive \cs{input}.
%    However there are formats that redefine it. For example,
%    \LaTeX's \cs{input} is not expandable, but it stores
%    the primitive in \cs{@@input}. The third possibility
%    is \cs{pdfprimitive}, introduced in \pdfTeX\ 1.40.0.
%
%    Thus we try to find the primitive and store it in
%    \cs{CatchFile@Input}. If it is used, it must be expanded
%    twice (because of the solution with \cs{pdfprimitive}).
%    \begin{macrocode}
\begingroup
  \def\Check#1#2#3#4\endgroup{%
    \edef\TestString{\string#1}%
    \edef\TestMeaning{\meaning#2}%
    \ifx\TestString\TestMeaning
      \endgroup
      \let\CatchFile@Primitive#2%
      \def\CatchFile@Input{\CatchFile@Primitive#3}%
    \else
      #4\endgroup
    \fi
  }%
  \Check\input\input{}%
  \Check\input\@@input{}%
  \Check\pdfprimitive\pdfprimitive\input
  \@PackageError{%
    Cannot find primitive \string\input,\MessageBreak
    package loading is aborted%
  }\@ehd
  \csname endgroup\endcsname
  \CatchFile@AtEnd%
\endgroup
%    \end{macrocode}
%    \end{macro}
%
% \subsection{Input file check}
%
%    \begin{macro}{\CatchFile@CheckFileExists}
%    \begin{macrocode}
\begingroup\expandafter\expandafter\expandafter\endgroup
\expandafter\ifx\csname IfFileExists\endcsname\relax
  \input pdftexcmds.sty\relax
  \begingroup\expandafter\expandafter\expandafter\endgroup
  \expandafter\ifx\csname pdf@filesize\endcsname\relax
    \def\CatchFile@CheckFileExists#1{%
      \expandafter\ifx\csname @inputcheck\endcsname\relax
        \csname newread\endcsname\@inputcheck
      \fi
      \openin\@inputcheck#1\relax
      \expandafter\closein\expandafter\@inputcheck
      \ifeof\@inputcheck
        \let\CatchFile@File\relax
      \else
        \def\CatchFile@File{#1}%
      \fi
    }%
  \else
    \def\CatchFile@CheckFileExists#1{%
      \expandafter\expandafter\expandafter\ifx
      \expandafter\expandafter\expandafter\relax\pdf@filesize{#1}\relax
        \let\CatchFile@File\relax
      \else
        \def\CatchFile@File{#1}%
      \fi
    }%
  \fi
\else
  \def\CatchFile@CheckFileExists#1{%
    \IfFileExists{#1}{%
      \expandafter\CatchFile@DefFile\@filef@und\@nil
      \begingroup\expandafter\expandafter\expandafter\endgroup
      \expandafter\ifx\csname @addtofilelist\endcsname\relax
      \else
        \@addtofilelist\CatchFile@File
      \fi
    }{%
      \let\CatchFile@File\relax
    }%
  }%
  \def\CatchFile@DefFile#1 \@nil{%
    \def\CatchFile@File{#1}%
  }%
\fi
%    \end{macrocode}
%    \end{macro}
%
%    \begin{macro}{\CatchFile@NotFound}
%    \begin{macrocode}
\def\CatchFile@NotFound#1#2{%
  \def#1{}%
  \@PackageError{catchfile}{%
    File `#2' not found%
  }\@ehc
}
%    \end{macrocode}
%    \end{macro}
%
% \subsection{Catch file contents}
%
%    \begin{macrocode}
\ltx@IfUndefined{RequirePackage}{%
  \input etexcmds.sty\relax
}{%
  \RequirePackage{etexcmds}[2010/01/28]%
}
%    \end{macrocode}
%
%    \begin{macro}{\CatchFileEdef}
%    \begin{macrocode}
\long\def\CatchFileEdef#1#2#3{%
  \CatchFile@CheckFileExists{#2}%
  \ifx\CatchFile@File\relax
    \CatchFile@NotFound{#1}{#2}%
  \else
    \begingroup
      \everyeof{\noexpand}%
      #3%
      \xdef\CatchFile@Contents{\CatchFile@Input\CatchFile@File\space}%
    \endgroup
    \let#1\CatchFile@Contents
  \fi
}
%    \end{macrocode}
%    \end{macro}
%
%    \begin{macro}{\CatchFileDef}
%    \begin{macrocode}
\long\def\CatchFileDef#1#2#3{%
  \CatchFile@CheckFileExists{#2}%
  \ifx\CatchFile@File\relax
    \CatchFile@NotFound{#1}{#2}%
  \else
    \begingroup
      \everyeof\expandafter{%
        \CatchFile@EOF
        \expandafter\CatchFile@Finish
        \noexpand
      }%
      \ltx@ifundefined{etex@unexpanded}{%
        \expandafter\long\expandafter\def\expandafter\CatchFile@Do
            \expandafter##\expandafter1\CatchFile@EOF{%
          \toks\ltx@zero{##1}%
          \xdef\CatchFile@gtemp{\the\toks\ltx@zero}%
          \def\CatchFile@Finish{%
            \endgroup
            \let#1\CatchFile@gtemp
            \global\let\CatchFile@gtemp\ltx@undefined
          }%
        }%
      }{%
        \expandafter\long\expandafter\def\expandafter\CatchFile@Do
            \expandafter##\expandafter1\CatchFile@EOF{%
          \edef\CatchFile@Finish{%
            \endgroup
            \etex@unexpanded{%
              \edef#1{\etex@unexpanded{##1}}%
            }%
          }%
        }%
      }%
      #3\relax
    \expandafter\expandafter\expandafter\CatchFile@Do
    \CatchFile@Input\CatchFile@File\relax
  \fi
}
%    \end{macrocode}
%    \cs{relax} after |#3| was added to make it more robust in
%    case the user uses something like
%\begin{verbatim}
%\CatchFileDef{\content}{\jobname.tt}{\endlinechar=-1}
%\end{verbatim}
%    that expands the following \cs{expandafter} after |#3| prematurely
%    (contribution of Martin Scharrer).
%    \end{macro}
%    \begin{macro}{\CatchFile@EOF}
%    If the file is read the catcode mappings are fixed.
%    This means that the same character cannot occur
%    inside the file with different catcodes.
%    Thus we use as end of file marker the at sign
%    twice with different catcodes.
%    \begin{macrocode}
\begingroup
  \lccode65=64 % lowercase('A') = '@'
  \lccode66=64 % lowercase('B') = '@'
  \catcode65=8 % catcode('A') = subscript
  \catcode66=3 % catcode('B') = math shift
\lowercase{\endgroup
  \def\CatchFile@EOF{AB}%
}
%    \end{macrocode}
%    \end{macro}
%
%    \begin{macrocode}
\CatchFile@AtEnd%
%</package>
%    \end{macrocode}
%
% \section{Test}
%
% \subsection{Catcode checks for loading}
%
%    \begin{macrocode}
%<*test1>
%    \end{macrocode}
%    \begin{macrocode}
\catcode`\{=1 %
\catcode`\}=2 %
\catcode`\#=6 %
\catcode`\@=11 %
\expandafter\ifx\csname count@\endcsname\relax
  \countdef\count@=255 %
\fi
\expandafter\ifx\csname @gobble\endcsname\relax
  \long\def\@gobble#1{}%
\fi
\expandafter\ifx\csname @firstofone\endcsname\relax
  \long\def\@firstofone#1{#1}%
\fi
\expandafter\ifx\csname loop\endcsname\relax
  \expandafter\@firstofone
\else
  \expandafter\@gobble
\fi
{%
  \def\loop#1\repeat{%
    \def\body{#1}%
    \iterate
  }%
  \def\iterate{%
    \body
      \let\next\iterate
    \else
      \let\next\relax
    \fi
    \next
  }%
  \let\repeat=\fi
}%
\def\RestoreCatcodes{}
\count@=0 %
\loop
  \edef\RestoreCatcodes{%
    \RestoreCatcodes
    \catcode\the\count@=\the\catcode\count@\relax
  }%
\ifnum\count@<255 %
  \advance\count@ 1 %
\repeat

\def\RangeCatcodeInvalid#1#2{%
  \count@=#1\relax
  \loop
    \catcode\count@=15 %
  \ifnum\count@<#2\relax
    \advance\count@ 1 %
  \repeat
}
\def\RangeCatcodeCheck#1#2#3{%
  \count@=#1\relax
  \loop
    \ifnum#3=\catcode\count@
    \else
      \errmessage{%
        Character \the\count@\space
        with wrong catcode \the\catcode\count@\space
        instead of \number#3%
      }%
    \fi
  \ifnum\count@<#2\relax
    \advance\count@ 1 %
  \repeat
}
\def\space{ }
\expandafter\ifx\csname LoadCommand\endcsname\relax
  \def\LoadCommand{\input catchfile.sty\relax}%
\fi
\def\Test{%
  \RangeCatcodeInvalid{0}{47}%
  \RangeCatcodeInvalid{58}{64}%
  \RangeCatcodeInvalid{91}{96}%
  \RangeCatcodeInvalid{123}{255}%
  \catcode`\@=12 %
  \catcode`\\=0 %
  \catcode`\%=14 %
  \LoadCommand
  \RangeCatcodeCheck{0}{36}{15}%
  \RangeCatcodeCheck{37}{37}{14}%
  \RangeCatcodeCheck{38}{47}{15}%
  \RangeCatcodeCheck{48}{57}{12}%
  \RangeCatcodeCheck{58}{63}{15}%
  \RangeCatcodeCheck{64}{64}{12}%
  \RangeCatcodeCheck{65}{90}{11}%
  \RangeCatcodeCheck{91}{91}{15}%
  \RangeCatcodeCheck{92}{92}{0}%
  \RangeCatcodeCheck{93}{96}{15}%
  \RangeCatcodeCheck{97}{122}{11}%
  \RangeCatcodeCheck{123}{255}{15}%
  \RestoreCatcodes
}
\Test
\csname @@end\endcsname
\end
%    \end{macrocode}
%    \begin{macrocode}
%</test1>
%    \end{macrocode}
%
% \subsection{\LaTeX}
%
%    \begin{macrocode}
%<*test2>
\NeedsTeXFormat{LaTeX2e}
\nofiles
\listfiles
\tracingnesting=2 %
\documentclass{minimal}
\usepackage{catchfile}[2016/05/16]
\makeatletter
\def\mysetup{%
  \let\do\@makeother
  \dospecials
}
\def\CheckContents{%
  \begingroup
    \expandafter\ifx\csname contents\endcsname\relax
      \PackageError{TEST}{\string\contents\space is not defined}\@ehc
    \fi
  \endgroup
}
\def\StartDisableUnexpanded{%
  \begingroup
    \let\etex@unexpanded\@undefined
}
\def\StopDisableUnexpanded{%
  \endgroup
}
\def\CheckCleanup{%
  \begingroup
    \edef\x{\the\toks0}%
    \def\y{ABC}%
    \ifx\x\y
    \else
      \PackageError{TEST}{\string\toks0 has changed}\@ehc
    \fi
    \ifx\CatchFile@gtemp\@undefined
    \else
      \PackageError{TEST}{Left over \string\CatchFile@gtemp}\@ehc
    \fi
  \endgroup
}
\def\Check{%
  \CheckContents
  \CheckCleanup
}
\makeatother
\begin{document}
\toks0{ABC}

\CatchFileDef\contents{catchfile.sty}\mysetup
\typeout{\meaning\contents}
\Check
\typeout{***********************************************************}%
\CatchFileDef\contents{catchfile.sty}{}
\typeout{\meaning\contents}
\Check
\typeout{***********************************************************}%

\StartDisableUnexpanded
\CatchFileDef\contents{catchfile.sty}\mysetup
\typeout{\meaning\contents}
\Check
\typeout{***********************************************************}%
\CatchFileDef\contents{catchfile.sty}{}
\typeout{\meaning\contents}
\Check
\typeout{***********************************************************}%
\StopDisableUnexpanded

\CatchFileEdef\contents{catchfile.sty}{%
  \mysetup
  \def\par{^^J}%
  \obeylines
}
\typeout{\contents}
\Check
\typeout{***********************************************************}%
\CatchFileEdef\contents{catchfile.sty}{%
  \catcode`\\=12 %
  \catcode`\#=12 %
  \def\par{^^J}%
  \obeylines
}
\typeout{\contents}
\Check
\typeout{***********************************************************}%
\end{document}
%</test2>
%    \end{macrocode}
%
% \subsection{\plainTeX}
%
%    \begin{macrocode}
%<*test3>
\def\msg#{\immediate\write16}
\newlinechar=10 %
\tracingnesting=2 %
\input catchfile.sty\relax

\def\mysetup{%
  \def\do##1{%
    \catcode`##1=12\relax
  }%
  \dospecials
}
\def\CheckContents{%
  \begingroup
    \expandafter\ifx\csname contents\endcsname\relax
      \def\space{ }%
      \errmessage{\string\contents\space is not defined.}%
    \fi
  \endgroup
}
\catcode`\@=11 %
\def\CheckCleanup{%
  \begingroup
    \edef\x{\the\toks0}%
    \def\y{ABC}%
    \ifx\x\y
    \else
      \errmessage{\string\toks0 has changed}%
    \fi
    \ifx\CatchFile@gtemp\@undefined
    \else
      \errmessage{Left over \string\CatchFile@gtemp}%
    \fi
  \endgroup
}
\def\Check{%
  \CheckContents
  \CheckCleanup
}
\def\StartDisableUnexpanded{%
  \begingroup
    \let\etex@unexpanded\@undefined
}
\def\StopDisableUnexpanded{%
  \endgroup
}
\catcode`\@=12 %

\toks0{ABC}

\CatchFileDef\contents{catchfile.sty}\mysetup
\msg{\meaning\contents}
\Check
\msg{***********************************************************}%
\CatchFileDef\contents{catchfile.sty}{}
\msg{\meaning\contents}
\Check
\msg{***********************************************************}%

\StartDisableUnexpanded
\CatchFileDef\contents{catchfile.sty}\mysetup
\msg{\meaning\contents}
\Check
\msg{***********************************************************}%
\CatchFileDef\contents{catchfile.sty}{}
\msg{\meaning\contents}
\Check
\msg{***********************************************************}%
\StopDisableUnexpanded

\CatchFileEdef\contents{catchfile.sty}{%
  \mysetup
  \def\par{^^J}%
  \obeylines
}
\msg{\contents}
\Check
\msg{***********************************************************}%
\CatchFileEdef\contents{catchfile.sty}{%
  \catcode`\\=12 %
  \catcode`\#=12 %
  \def\par{^^J}%
  \obeylines
}
\msg{\contents}
\Check
\msg{***********************************************************}%

\csname @@end\endcsname
\end
%</test3>
%    \end{macrocode}
%
% \section{Installation}
%
% \subsection{Download}
%
% \paragraph{Package.} This package is available on
% CTAN\footnote{\url{https://ctan.org/pkg/catchfile}}:
% \begin{description}
% \item[\CTAN{macros/latex/contrib/oberdiek/catchfile.dtx}] The source file.
% \item[\CTAN{macros/latex/contrib/oberdiek/catchfile.pdf}] Documentation.
% \end{description}
%
%
% \paragraph{Bundle.} All the packages of the bundle `oberdiek'
% are also available in a TDS compliant ZIP archive. There
% the packages are already unpacked and the documentation files
% are generated. The files and directories obey the TDS standard.
% \begin{description}
% \item[\CTANinstall{install/macros/latex/contrib/oberdiek.tds.zip}]
% \end{description}
% \emph{TDS} refers to the standard ``A Directory Structure
% for \TeX\ Files'' (\CTAN{tds/tds.pdf}). Directories
% with \xfile{texmf} in their name are usually organized this way.
%
% \subsection{Bundle installation}
%
% \paragraph{Unpacking.} Unpack the \xfile{oberdiek.tds.zip} in the
% TDS tree (also known as \xfile{texmf} tree) of your choice.
% Example (linux):
% \begin{quote}
%   |unzip oberdiek.tds.zip -d ~/texmf|
% \end{quote}
%
% \paragraph{Script installation.}
% Check the directory \xfile{TDS:scripts/oberdiek/} for
% scripts that need further installation steps.
% Package \xpackage{attachfile2} comes with the Perl script
% \xfile{pdfatfi.pl} that should be installed in such a way
% that it can be called as \texttt{pdfatfi}.
% Example (linux):
% \begin{quote}
%   |chmod +x scripts/oberdiek/pdfatfi.pl|\\
%   |cp scripts/oberdiek/pdfatfi.pl /usr/local/bin/|
% \end{quote}
%
% \subsection{Package installation}
%
% \paragraph{Unpacking.} The \xfile{.dtx} file is a self-extracting
% \docstrip\ archive. The files are extracted by running the
% \xfile{.dtx} through \plainTeX:
% \begin{quote}
%   \verb|tex catchfile.dtx|
% \end{quote}
%
% \paragraph{TDS.} Now the different files must be moved into
% the different directories in your installation TDS tree
% (also known as \xfile{texmf} tree):
% \begin{quote}
% \def\t{^^A
% \begin{tabular}{@{}>{\ttfamily}l@{ $\rightarrow$ }>{\ttfamily}l@{}}
%   catchfile.sty & tex/generic/oberdiek/catchfile.sty\\
%   catchfile.pdf & doc/latex/oberdiek/catchfile.pdf\\
%   test/catchfile-test1.tex & doc/latex/oberdiek/test/catchfile-test1.tex\\
%   test/catchfile-test2.tex & doc/latex/oberdiek/test/catchfile-test2.tex\\
%   test/catchfile-test3.tex & doc/latex/oberdiek/test/catchfile-test3.tex\\
%   catchfile.dtx & source/latex/oberdiek/catchfile.dtx\\
% \end{tabular}^^A
% }^^A
% \sbox0{\t}^^A
% \ifdim\wd0>\linewidth
%   \begingroup
%     \advance\linewidth by\leftmargin
%     \advance\linewidth by\rightmargin
%   \edef\x{\endgroup
%     \def\noexpand\lw{\the\linewidth}^^A
%   }\x
%   \def\lwbox{^^A
%     \leavevmode
%     \hbox to \linewidth{^^A
%       \kern-\leftmargin\relax
%       \hss
%       \usebox0
%       \hss
%       \kern-\rightmargin\relax
%     }^^A
%   }^^A
%   \ifdim\wd0>\lw
%     \sbox0{\small\t}^^A
%     \ifdim\wd0>\linewidth
%       \ifdim\wd0>\lw
%         \sbox0{\footnotesize\t}^^A
%         \ifdim\wd0>\linewidth
%           \ifdim\wd0>\lw
%             \sbox0{\scriptsize\t}^^A
%             \ifdim\wd0>\linewidth
%               \ifdim\wd0>\lw
%                 \sbox0{\tiny\t}^^A
%                 \ifdim\wd0>\linewidth
%                   \lwbox
%                 \else
%                   \usebox0
%                 \fi
%               \else
%                 \lwbox
%               \fi
%             \else
%               \usebox0
%             \fi
%           \else
%             \lwbox
%           \fi
%         \else
%           \usebox0
%         \fi
%       \else
%         \lwbox
%       \fi
%     \else
%       \usebox0
%     \fi
%   \else
%     \lwbox
%   \fi
% \else
%   \usebox0
% \fi
% \end{quote}
% If you have a \xfile{docstrip.cfg} that configures and enables \docstrip's
% TDS installing feature, then some files can already be in the right
% place, see the documentation of \docstrip.
%
% \subsection{Refresh file name databases}
%
% If your \TeX~distribution
% (\teTeX, \mikTeX, \dots) relies on file name databases, you must refresh
% these. For example, \teTeX\ users run \verb|texhash| or
% \verb|mktexlsr|.
%
% \subsection{Some details for the interested}
%
% \paragraph{Attached source.}
%
% The PDF documentation on CTAN also includes the
% \xfile{.dtx} source file. It can be extracted by
% AcrobatReader 6 or higher. Another option is \textsf{pdftk},
% e.g. unpack the file into the current directory:
% \begin{quote}
%   \verb|pdftk catchfile.pdf unpack_files output .|
% \end{quote}
%
% \paragraph{Unpacking with \LaTeX.}
% The \xfile{.dtx} chooses its action depending on the format:
% \begin{description}
% \item[\plainTeX:] Run \docstrip\ and extract the files.
% \item[\LaTeX:] Generate the documentation.
% \end{description}
% If you insist on using \LaTeX\ for \docstrip\ (really,
% \docstrip\ does not need \LaTeX), then inform the autodetect routine
% about your intention:
% \begin{quote}
%   \verb|latex \let\install=y% \iffalse meta-comment
%
% File: catchfile.dtx
% Version: 2016/05/16 v1.7
% Info: Catch the contents of a file
%
% Copyright (C) 2007, 2010, 2011 by
%    Heiko Oberdiek <heiko.oberdiek at googlemail.com>
%    2016
%    https://github.com/ho-tex/oberdiek/issues
%
% This work may be distributed and/or modified under the
% conditions of the LaTeX Project Public License, either
% version 1.3c of this license or (at your option) any later
% version. This version of this license is in
%    http://www.latex-project.org/lppl/lppl-1-3c.txt
% and the latest version of this license is in
%    http://www.latex-project.org/lppl.txt
% and version 1.3 or later is part of all distributions of
% LaTeX version 2005/12/01 or later.
%
% This work has the LPPL maintenance status "maintained".
%
% This Current Maintainer of this work is Heiko Oberdiek.
%
% The Base Interpreter refers to any `TeX-Format',
% because some files are installed in TDS:tex/generic//.
%
% This work consists of the main source file catchfile.dtx
% and the derived files
%    catchfile.sty, catchfile.pdf, catchfile.ins, catchfile.drv,
%    catchfile-test1.tex, catchfile-test2.tex,
%    catchfile-test3.tex.
%
% Distribution:
%    CTAN:macros/latex/contrib/oberdiek/catchfile.dtx
%    CTAN:macros/latex/contrib/oberdiek/catchfile.pdf
%
% Unpacking:
%    (a) If catchfile.ins is present:
%           tex catchfile.ins
%    (b) Without catchfile.ins:
%           tex catchfile.dtx
%    (c) If you insist on using LaTeX
%           latex \let\install=y% \iffalse meta-comment
%
% File: catchfile.dtx
% Version: 2016/05/16 v1.7
% Info: Catch the contents of a file
%
% Copyright (C) 2007, 2010, 2011 by
%    Heiko Oberdiek <heiko.oberdiek at googlemail.com>
%    2016
%    https://github.com/ho-tex/oberdiek/issues
%
% This work may be distributed and/or modified under the
% conditions of the LaTeX Project Public License, either
% version 1.3c of this license or (at your option) any later
% version. This version of this license is in
%    http://www.latex-project.org/lppl/lppl-1-3c.txt
% and the latest version of this license is in
%    http://www.latex-project.org/lppl.txt
% and version 1.3 or later is part of all distributions of
% LaTeX version 2005/12/01 or later.
%
% This work has the LPPL maintenance status "maintained".
%
% This Current Maintainer of this work is Heiko Oberdiek.
%
% The Base Interpreter refers to any `TeX-Format',
% because some files are installed in TDS:tex/generic//.
%
% This work consists of the main source file catchfile.dtx
% and the derived files
%    catchfile.sty, catchfile.pdf, catchfile.ins, catchfile.drv,
%    catchfile-test1.tex, catchfile-test2.tex,
%    catchfile-test3.tex.
%
% Distribution:
%    CTAN:macros/latex/contrib/oberdiek/catchfile.dtx
%    CTAN:macros/latex/contrib/oberdiek/catchfile.pdf
%
% Unpacking:
%    (a) If catchfile.ins is present:
%           tex catchfile.ins
%    (b) Without catchfile.ins:
%           tex catchfile.dtx
%    (c) If you insist on using LaTeX
%           latex \let\install=y\input{catchfile.dtx}
%        (quote the arguments according to the demands of your shell)
%
% Documentation:
%    (a) If catchfile.drv is present:
%           latex catchfile.drv
%    (b) Without catchfile.drv:
%           latex catchfile.dtx; ...
%    The class ltxdoc loads the configuration file ltxdoc.cfg
%    if available. Here you can specify further options, e.g.
%    use A4 as paper format:
%       \PassOptionsToClass{a4paper}{article}
%
%    Programm calls to get the documentation (example):
%       pdflatex catchfile.dtx
%       makeindex -s gind.ist catchfile.idx
%       pdflatex catchfile.dtx
%       makeindex -s gind.ist catchfile.idx
%       pdflatex catchfile.dtx
%
% Installation:
%    TDS:tex/generic/oberdiek/catchfile.sty
%    TDS:doc/latex/oberdiek/catchfile.pdf
%    TDS:doc/latex/oberdiek/test/catchfile-test1.tex
%    TDS:doc/latex/oberdiek/test/catchfile-test2.tex
%    TDS:doc/latex/oberdiek/test/catchfile-test3.tex
%    TDS:source/latex/oberdiek/catchfile.dtx
%
%<*ignore>
\begingroup
  \catcode123=1 %
  \catcode125=2 %
  \def\x{LaTeX2e}%
\expandafter\endgroup
\ifcase 0\ifx\install y1\fi\expandafter
         \ifx\csname processbatchFile\endcsname\relax\else1\fi
         \ifx\fmtname\x\else 1\fi\relax
\else\csname fi\endcsname
%</ignore>
%<*install>
\input docstrip.tex
\Msg{************************************************************************}
\Msg{* Installation}
\Msg{* Package: catchfile 2016/05/16 v1.7 Catch the contents of a file (HO)}
\Msg{************************************************************************}

\keepsilent
\askforoverwritefalse

\let\MetaPrefix\relax
\preamble

This is a generated file.

Project: catchfile
Version: 2016/05/16 v1.7

Copyright (C) 2007, 2010, 2011 by
   Heiko Oberdiek <heiko.oberdiek at googlemail.com>

This work may be distributed and/or modified under the
conditions of the LaTeX Project Public License, either
version 1.3c of this license or (at your option) any later
version. This version of this license is in
   http://www.latex-project.org/lppl/lppl-1-3c.txt
and the latest version of this license is in
   http://www.latex-project.org/lppl.txt
and version 1.3 or later is part of all distributions of
LaTeX version 2005/12/01 or later.

This work has the LPPL maintenance status "maintained".

This Current Maintainer of this work is Heiko Oberdiek.

The Base Interpreter refers to any `TeX-Format',
because some files are installed in TDS:tex/generic//.

This work consists of the main source file catchfile.dtx
and the derived files
   catchfile.sty, catchfile.pdf, catchfile.ins, catchfile.drv,
   catchfile-test1.tex, catchfile-test2.tex,
   catchfile-test3.tex.

\endpreamble
\let\MetaPrefix\DoubleperCent

\generate{%
  \file{catchfile.ins}{\from{catchfile.dtx}{install}}%
  \file{catchfile.drv}{\from{catchfile.dtx}{driver}}%
  \usedir{tex/generic/oberdiek}%
  \file{catchfile.sty}{\from{catchfile.dtx}{package}}%
%  \usedir{doc/latex/oberdiek/test}%
%  \file{catchfile-test1.tex}{\from{catchfile.dtx}{test1}}%
%  \file{catchfile-test2.tex}{\from{catchfile.dtx}{test2}}%
%  \file{catchfile-test3.tex}{\from{catchfile.dtx}{test3}}%
  \nopreamble
  \nopostamble
%  \usedir{source/latex/oberdiek/catalogue}%
%  \file{catchfile.xml}{\from{catchfile.dtx}{catalogue}}%
}

\catcode32=13\relax% active space
\let =\space%
\Msg{************************************************************************}
\Msg{*}
\Msg{* To finish the installation you have to move the following}
\Msg{* file into a directory searched by TeX:}
\Msg{*}
\Msg{*     catchfile.sty}
\Msg{*}
\Msg{* To produce the documentation run the file `catchfile.drv'}
\Msg{* through LaTeX.}
\Msg{*}
\Msg{* Happy TeXing!}
\Msg{*}
\Msg{************************************************************************}

\endbatchfile
%</install>
%<*ignore>
\fi
%</ignore>
%<*driver>
\NeedsTeXFormat{LaTeX2e}
\ProvidesFile{catchfile.drv}%
  [2016/05/16 v1.7 Catch the contents of a file (HO)]%
\documentclass{ltxdoc}
\usepackage{holtxdoc}[2011/11/22]
\begin{document}
  \DocInput{catchfile.dtx}%
\end{document}
%</driver>
% \fi
%
%
% \CharacterTable
%  {Upper-case    \A\B\C\D\E\F\G\H\I\J\K\L\M\N\O\P\Q\R\S\T\U\V\W\X\Y\Z
%   Lower-case    \a\b\c\d\e\f\g\h\i\j\k\l\m\n\o\p\q\r\s\t\u\v\w\x\y\z
%   Digits        \0\1\2\3\4\5\6\7\8\9
%   Exclamation   \!     Double quote  \"     Hash (number) \#
%   Dollar        \$     Percent       \%     Ampersand     \&
%   Acute accent  \'     Left paren    \(     Right paren   \)
%   Asterisk      \*     Plus          \+     Comma         \,
%   Minus         \-     Point         \.     Solidus       \/
%   Colon         \:     Semicolon     \;     Less than     \<
%   Equals        \=     Greater than  \>     Question mark \?
%   Commercial at \@     Left bracket  \[     Backslash     \\
%   Right bracket \]     Circumflex    \^     Underscore    \_
%   Grave accent  \`     Left brace    \{     Vertical bar  \|
%   Right brace   \}     Tilde         \~}
%
% \GetFileInfo{catchfile.drv}
%
% \title{The \xpackage{catchfile} package}
% \date{2016/05/16 v1.7}
% \author{Heiko Oberdiek\thanks
% {Please report any issues at https://github.com/ho-tex/oberdiek/issues}\\
% \xemail{heiko.oberdiek at googlemail.com}}
%
% \maketitle
%
% \begin{abstract}
% This package catches the contents of a file and puts it in a macro.
% It requires \eTeX. Both \LaTeX\ and \plainTeX\ are supported.
% \end{abstract}
%
% \tableofcontents
%
% \section{Documentation}
%
% The package relies on \eTeX's \cs{everyeof}. Otherwise it
% aborts with an error message.
%
% \begin{declcs}{CatchFileDef} \M{cmd} \M{file name} \M{setup}\\
% \cs{CatchFileEdef} \M{cmd} \M{file name} \M{setup}
% \end{declcs}
% Macro \meta{cmd} is defined with the contents
% of file \meta{file name}.
% \cs{CatchFileDef} uses \cs{def}, \cs{CatchFileEdef} \cs{edef}
% for the definition.
% Additional setup code for setting
% catcodes or treatment of line ends can be given in code
% \meta{setup}. See the test files for an example.
%
% \StopEventually{
% }
%
% \section{Implementation}
%
%    \begin{macrocode}
%<*package>
%    \end{macrocode}
%
% \subsection{Reload check and package identification}
%    Reload check, especially if the package is not used with \LaTeX.
%    \begin{macrocode}
\begingroup\catcode61\catcode48\catcode32=10\relax%
  \catcode13=5 % ^^M
  \endlinechar=13 %
  \catcode35=6 % #
  \catcode39=12 % '
  \catcode44=12 % ,
  \catcode45=12 % -
  \catcode46=12 % .
  \catcode58=12 % :
  \catcode64=11 % @
  \catcode123=1 % {
  \catcode125=2 % }
  \expandafter\let\expandafter\x\csname ver@catchfile.sty\endcsname
  \ifx\x\relax % plain-TeX, first loading
  \else
    \def\empty{}%
    \ifx\x\empty % LaTeX, first loading,
      % variable is initialized, but \ProvidesPackage not yet seen
    \else
      \expandafter\ifx\csname PackageInfo\endcsname\relax
        \def\x#1#2{%
          \immediate\write-1{Package #1 Info: #2.}%
        }%
      \else
        \def\x#1#2{\PackageInfo{#1}{#2, stopped}}%
      \fi
      \x{catchfile}{The package is already loaded}%
      \aftergroup\endinput
    \fi
  \fi
\endgroup%
%    \end{macrocode}
%    Package identification:
%    \begin{macrocode}
\begingroup\catcode61\catcode48\catcode32=10\relax%
  \catcode13=5 % ^^M
  \endlinechar=13 %
  \catcode35=6 % #
  \catcode39=12 % '
  \catcode40=12 % (
  \catcode41=12 % )
  \catcode44=12 % ,
  \catcode45=12 % -
  \catcode46=12 % .
  \catcode47=12 % /
  \catcode58=12 % :
  \catcode64=11 % @
  \catcode91=12 % [
  \catcode93=12 % ]
  \catcode123=1 % {
  \catcode125=2 % }
  \expandafter\ifx\csname ProvidesPackage\endcsname\relax
    \def\x#1#2#3[#4]{\endgroup
      \immediate\write-1{Package: #3 #4}%
      \xdef#1{#4}%
    }%
  \else
    \def\x#1#2[#3]{\endgroup
      #2[{#3}]%
      \ifx#1\@undefined
        \xdef#1{#3}%
      \fi
      \ifx#1\relax
        \xdef#1{#3}%
      \fi
    }%
  \fi
\expandafter\x\csname ver@catchfile.sty\endcsname
\ProvidesPackage{catchfile}%
  [2016/05/16 v1.7 Catch the contents of a file (HO)]%
%    \end{macrocode}
%
% \subsection{Catcodes}
%
%    \begin{macrocode}
\begingroup\catcode61\catcode48\catcode32=10\relax%
  \catcode13=5 % ^^M
  \endlinechar=13 %
  \catcode123=1 % {
  \catcode125=2 % }
  \catcode64=11 % @
  \def\x{\endgroup
    \expandafter\edef\csname CatchFile@AtEnd\endcsname{%
      \endlinechar=\the\endlinechar\relax
      \catcode13=\the\catcode13\relax
      \catcode32=\the\catcode32\relax
      \catcode35=\the\catcode35\relax
      \catcode61=\the\catcode61\relax
      \catcode64=\the\catcode64\relax
      \catcode123=\the\catcode123\relax
      \catcode125=\the\catcode125\relax
    }%
  }%
\x\catcode61\catcode48\catcode32=10\relax%
\catcode13=5 % ^^M
\endlinechar=13 %
\catcode35=6 % #
\catcode64=11 % @
\catcode123=1 % {
\catcode125=2 % }
\def\TMP@EnsureCode#1#2{%
  \edef\CatchFile@AtEnd{%
    \CatchFile@AtEnd
    \catcode#1=\the\catcode#1\relax
  }%
  \catcode#1=#2\relax
}
\TMP@EnsureCode{39}{12}% '
\TMP@EnsureCode{44}{12}% ,
\TMP@EnsureCode{45}{12}% -
\TMP@EnsureCode{46}{12}% .
\TMP@EnsureCode{47}{12}% /
\TMP@EnsureCode{91}{12}% [
\TMP@EnsureCode{93}{12}% ]
\TMP@EnsureCode{96}{12}% `
\edef\CatchFile@AtEnd{\CatchFile@AtEnd\noexpand\endinput}
%    \end{macrocode}
%
% \subsection{Preparations}
%
%    \begin{macrocode}
\begingroup\expandafter\expandafter\expandafter\endgroup
\expandafter\ifx\csname RequirePackage\endcsname\relax
  \input infwarerr.sty\relax
  \input ltxcmds.sty\relax
\else
  \RequirePackage{infwarerr}[2007/09/09]%
  \RequirePackage{ltxcmds}[2010/03/09]%
\fi
%    \end{macrocode}
%
%    Check for \eTeX's \cs{everyeof}.
%    \begin{macrocode}
\begingroup
  \escapechar=92\relax
  \edef\TestString{\string\everyeof}%
  \edef\TestMeaning{\meaning\everyeof}%
  \ifx\TestString\TestMeaning
  \else
    \@PackageError{catchfile}{%
      Cannot find e-TeX's \string\everyeof,\MessageBreak
      package loading is aborted%
    }\@ehd
    \endgroup
    \expandafter\CatchFile@AtEnd
  \fi%
\endgroup
%    \end{macrocode}
%
% \subsection{Looking for primitive \cs{input}}
%
%    \begin{macro}{\CatchFile@Input}
%    The package needs the expandable primitive \cs{input}.
%    However there are formats that redefine it. For example,
%    \LaTeX's \cs{input} is not expandable, but it stores
%    the primitive in \cs{@@input}. The third possibility
%    is \cs{pdfprimitive}, introduced in \pdfTeX\ 1.40.0.
%
%    Thus we try to find the primitive and store it in
%    \cs{CatchFile@Input}. If it is used, it must be expanded
%    twice (because of the solution with \cs{pdfprimitive}).
%    \begin{macrocode}
\begingroup
  \def\Check#1#2#3#4\endgroup{%
    \edef\TestString{\string#1}%
    \edef\TestMeaning{\meaning#2}%
    \ifx\TestString\TestMeaning
      \endgroup
      \let\CatchFile@Primitive#2%
      \def\CatchFile@Input{\CatchFile@Primitive#3}%
    \else
      #4\endgroup
    \fi
  }%
  \Check\input\input{}%
  \Check\input\@@input{}%
  \Check\pdfprimitive\pdfprimitive\input
  \@PackageError{%
    Cannot find primitive \string\input,\MessageBreak
    package loading is aborted%
  }\@ehd
  \csname endgroup\endcsname
  \CatchFile@AtEnd%
\endgroup
%    \end{macrocode}
%    \end{macro}
%
% \subsection{Input file check}
%
%    \begin{macro}{\CatchFile@CheckFileExists}
%    \begin{macrocode}
\begingroup\expandafter\expandafter\expandafter\endgroup
\expandafter\ifx\csname IfFileExists\endcsname\relax
  \input pdftexcmds.sty\relax
  \begingroup\expandafter\expandafter\expandafter\endgroup
  \expandafter\ifx\csname pdf@filesize\endcsname\relax
    \def\CatchFile@CheckFileExists#1{%
      \expandafter\ifx\csname @inputcheck\endcsname\relax
        \csname newread\endcsname\@inputcheck
      \fi
      \openin\@inputcheck#1\relax
      \expandafter\closein\expandafter\@inputcheck
      \ifeof\@inputcheck
        \let\CatchFile@File\relax
      \else
        \def\CatchFile@File{#1}%
      \fi
    }%
  \else
    \def\CatchFile@CheckFileExists#1{%
      \expandafter\expandafter\expandafter\ifx
      \expandafter\expandafter\expandafter\relax\pdf@filesize{#1}\relax
        \let\CatchFile@File\relax
      \else
        \def\CatchFile@File{#1}%
      \fi
    }%
  \fi
\else
  \def\CatchFile@CheckFileExists#1{%
    \IfFileExists{#1}{%
      \expandafter\CatchFile@DefFile\@filef@und\@nil
      \begingroup\expandafter\expandafter\expandafter\endgroup
      \expandafter\ifx\csname @addtofilelist\endcsname\relax
      \else
        \@addtofilelist\CatchFile@File
      \fi
    }{%
      \let\CatchFile@File\relax
    }%
  }%
  \def\CatchFile@DefFile#1 \@nil{%
    \def\CatchFile@File{#1}%
  }%
\fi
%    \end{macrocode}
%    \end{macro}
%
%    \begin{macro}{\CatchFile@NotFound}
%    \begin{macrocode}
\def\CatchFile@NotFound#1#2{%
  \def#1{}%
  \@PackageError{catchfile}{%
    File `#2' not found%
  }\@ehc
}
%    \end{macrocode}
%    \end{macro}
%
% \subsection{Catch file contents}
%
%    \begin{macrocode}
\ltx@IfUndefined{RequirePackage}{%
  \input etexcmds.sty\relax
}{%
  \RequirePackage{etexcmds}[2010/01/28]%
}
%    \end{macrocode}
%
%    \begin{macro}{\CatchFileEdef}
%    \begin{macrocode}
\long\def\CatchFileEdef#1#2#3{%
  \CatchFile@CheckFileExists{#2}%
  \ifx\CatchFile@File\relax
    \CatchFile@NotFound{#1}{#2}%
  \else
    \begingroup
      \everyeof{\noexpand}%
      #3%
      \xdef\CatchFile@Contents{\CatchFile@Input\CatchFile@File\space}%
    \endgroup
    \let#1\CatchFile@Contents
  \fi
}
%    \end{macrocode}
%    \end{macro}
%
%    \begin{macro}{\CatchFileDef}
%    \begin{macrocode}
\long\def\CatchFileDef#1#2#3{%
  \CatchFile@CheckFileExists{#2}%
  \ifx\CatchFile@File\relax
    \CatchFile@NotFound{#1}{#2}%
  \else
    \begingroup
      \everyeof\expandafter{%
        \CatchFile@EOF
        \expandafter\CatchFile@Finish
        \noexpand
      }%
      \ltx@ifundefined{etex@unexpanded}{%
        \expandafter\long\expandafter\def\expandafter\CatchFile@Do
            \expandafter##\expandafter1\CatchFile@EOF{%
          \toks\ltx@zero{##1}%
          \xdef\CatchFile@gtemp{\the\toks\ltx@zero}%
          \def\CatchFile@Finish{%
            \endgroup
            \let#1\CatchFile@gtemp
            \global\let\CatchFile@gtemp\ltx@undefined
          }%
        }%
      }{%
        \expandafter\long\expandafter\def\expandafter\CatchFile@Do
            \expandafter##\expandafter1\CatchFile@EOF{%
          \edef\CatchFile@Finish{%
            \endgroup
            \etex@unexpanded{%
              \edef#1{\etex@unexpanded{##1}}%
            }%
          }%
        }%
      }%
      #3\relax
    \expandafter\expandafter\expandafter\CatchFile@Do
    \CatchFile@Input\CatchFile@File\relax
  \fi
}
%    \end{macrocode}
%    \cs{relax} after |#3| was added to make it more robust in
%    case the user uses something like
%\begin{verbatim}
%\CatchFileDef{\content}{\jobname.tt}{\endlinechar=-1}
%\end{verbatim}
%    that expands the following \cs{expandafter} after |#3| prematurely
%    (contribution of Martin Scharrer).
%    \end{macro}
%    \begin{macro}{\CatchFile@EOF}
%    If the file is read the catcode mappings are fixed.
%    This means that the same character cannot occur
%    inside the file with different catcodes.
%    Thus we use as end of file marker the at sign
%    twice with different catcodes.
%    \begin{macrocode}
\begingroup
  \lccode65=64 % lowercase('A') = '@'
  \lccode66=64 % lowercase('B') = '@'
  \catcode65=8 % catcode('A') = subscript
  \catcode66=3 % catcode('B') = math shift
\lowercase{\endgroup
  \def\CatchFile@EOF{AB}%
}
%    \end{macrocode}
%    \end{macro}
%
%    \begin{macrocode}
\CatchFile@AtEnd%
%</package>
%    \end{macrocode}
%
% \section{Test}
%
% \subsection{Catcode checks for loading}
%
%    \begin{macrocode}
%<*test1>
%    \end{macrocode}
%    \begin{macrocode}
\catcode`\{=1 %
\catcode`\}=2 %
\catcode`\#=6 %
\catcode`\@=11 %
\expandafter\ifx\csname count@\endcsname\relax
  \countdef\count@=255 %
\fi
\expandafter\ifx\csname @gobble\endcsname\relax
  \long\def\@gobble#1{}%
\fi
\expandafter\ifx\csname @firstofone\endcsname\relax
  \long\def\@firstofone#1{#1}%
\fi
\expandafter\ifx\csname loop\endcsname\relax
  \expandafter\@firstofone
\else
  \expandafter\@gobble
\fi
{%
  \def\loop#1\repeat{%
    \def\body{#1}%
    \iterate
  }%
  \def\iterate{%
    \body
      \let\next\iterate
    \else
      \let\next\relax
    \fi
    \next
  }%
  \let\repeat=\fi
}%
\def\RestoreCatcodes{}
\count@=0 %
\loop
  \edef\RestoreCatcodes{%
    \RestoreCatcodes
    \catcode\the\count@=\the\catcode\count@\relax
  }%
\ifnum\count@<255 %
  \advance\count@ 1 %
\repeat

\def\RangeCatcodeInvalid#1#2{%
  \count@=#1\relax
  \loop
    \catcode\count@=15 %
  \ifnum\count@<#2\relax
    \advance\count@ 1 %
  \repeat
}
\def\RangeCatcodeCheck#1#2#3{%
  \count@=#1\relax
  \loop
    \ifnum#3=\catcode\count@
    \else
      \errmessage{%
        Character \the\count@\space
        with wrong catcode \the\catcode\count@\space
        instead of \number#3%
      }%
    \fi
  \ifnum\count@<#2\relax
    \advance\count@ 1 %
  \repeat
}
\def\space{ }
\expandafter\ifx\csname LoadCommand\endcsname\relax
  \def\LoadCommand{\input catchfile.sty\relax}%
\fi
\def\Test{%
  \RangeCatcodeInvalid{0}{47}%
  \RangeCatcodeInvalid{58}{64}%
  \RangeCatcodeInvalid{91}{96}%
  \RangeCatcodeInvalid{123}{255}%
  \catcode`\@=12 %
  \catcode`\\=0 %
  \catcode`\%=14 %
  \LoadCommand
  \RangeCatcodeCheck{0}{36}{15}%
  \RangeCatcodeCheck{37}{37}{14}%
  \RangeCatcodeCheck{38}{47}{15}%
  \RangeCatcodeCheck{48}{57}{12}%
  \RangeCatcodeCheck{58}{63}{15}%
  \RangeCatcodeCheck{64}{64}{12}%
  \RangeCatcodeCheck{65}{90}{11}%
  \RangeCatcodeCheck{91}{91}{15}%
  \RangeCatcodeCheck{92}{92}{0}%
  \RangeCatcodeCheck{93}{96}{15}%
  \RangeCatcodeCheck{97}{122}{11}%
  \RangeCatcodeCheck{123}{255}{15}%
  \RestoreCatcodes
}
\Test
\csname @@end\endcsname
\end
%    \end{macrocode}
%    \begin{macrocode}
%</test1>
%    \end{macrocode}
%
% \subsection{\LaTeX}
%
%    \begin{macrocode}
%<*test2>
\NeedsTeXFormat{LaTeX2e}
\nofiles
\listfiles
\tracingnesting=2 %
\documentclass{minimal}
\usepackage{catchfile}[2016/05/16]
\makeatletter
\def\mysetup{%
  \let\do\@makeother
  \dospecials
}
\def\CheckContents{%
  \begingroup
    \expandafter\ifx\csname contents\endcsname\relax
      \PackageError{TEST}{\string\contents\space is not defined}\@ehc
    \fi
  \endgroup
}
\def\StartDisableUnexpanded{%
  \begingroup
    \let\etex@unexpanded\@undefined
}
\def\StopDisableUnexpanded{%
  \endgroup
}
\def\CheckCleanup{%
  \begingroup
    \edef\x{\the\toks0}%
    \def\y{ABC}%
    \ifx\x\y
    \else
      \PackageError{TEST}{\string\toks0 has changed}\@ehc
    \fi
    \ifx\CatchFile@gtemp\@undefined
    \else
      \PackageError{TEST}{Left over \string\CatchFile@gtemp}\@ehc
    \fi
  \endgroup
}
\def\Check{%
  \CheckContents
  \CheckCleanup
}
\makeatother
\begin{document}
\toks0{ABC}

\CatchFileDef\contents{catchfile.sty}\mysetup
\typeout{\meaning\contents}
\Check
\typeout{***********************************************************}%
\CatchFileDef\contents{catchfile.sty}{}
\typeout{\meaning\contents}
\Check
\typeout{***********************************************************}%

\StartDisableUnexpanded
\CatchFileDef\contents{catchfile.sty}\mysetup
\typeout{\meaning\contents}
\Check
\typeout{***********************************************************}%
\CatchFileDef\contents{catchfile.sty}{}
\typeout{\meaning\contents}
\Check
\typeout{***********************************************************}%
\StopDisableUnexpanded

\CatchFileEdef\contents{catchfile.sty}{%
  \mysetup
  \def\par{^^J}%
  \obeylines
}
\typeout{\contents}
\Check
\typeout{***********************************************************}%
\CatchFileEdef\contents{catchfile.sty}{%
  \catcode`\\=12 %
  \catcode`\#=12 %
  \def\par{^^J}%
  \obeylines
}
\typeout{\contents}
\Check
\typeout{***********************************************************}%
\end{document}
%</test2>
%    \end{macrocode}
%
% \subsection{\plainTeX}
%
%    \begin{macrocode}
%<*test3>
\def\msg#{\immediate\write16}
\newlinechar=10 %
\tracingnesting=2 %
\input catchfile.sty\relax

\def\mysetup{%
  \def\do##1{%
    \catcode`##1=12\relax
  }%
  \dospecials
}
\def\CheckContents{%
  \begingroup
    \expandafter\ifx\csname contents\endcsname\relax
      \def\space{ }%
      \errmessage{\string\contents\space is not defined.}%
    \fi
  \endgroup
}
\catcode`\@=11 %
\def\CheckCleanup{%
  \begingroup
    \edef\x{\the\toks0}%
    \def\y{ABC}%
    \ifx\x\y
    \else
      \errmessage{\string\toks0 has changed}%
    \fi
    \ifx\CatchFile@gtemp\@undefined
    \else
      \errmessage{Left over \string\CatchFile@gtemp}%
    \fi
  \endgroup
}
\def\Check{%
  \CheckContents
  \CheckCleanup
}
\def\StartDisableUnexpanded{%
  \begingroup
    \let\etex@unexpanded\@undefined
}
\def\StopDisableUnexpanded{%
  \endgroup
}
\catcode`\@=12 %

\toks0{ABC}

\CatchFileDef\contents{catchfile.sty}\mysetup
\msg{\meaning\contents}
\Check
\msg{***********************************************************}%
\CatchFileDef\contents{catchfile.sty}{}
\msg{\meaning\contents}
\Check
\msg{***********************************************************}%

\StartDisableUnexpanded
\CatchFileDef\contents{catchfile.sty}\mysetup
\msg{\meaning\contents}
\Check
\msg{***********************************************************}%
\CatchFileDef\contents{catchfile.sty}{}
\msg{\meaning\contents}
\Check
\msg{***********************************************************}%
\StopDisableUnexpanded

\CatchFileEdef\contents{catchfile.sty}{%
  \mysetup
  \def\par{^^J}%
  \obeylines
}
\msg{\contents}
\Check
\msg{***********************************************************}%
\CatchFileEdef\contents{catchfile.sty}{%
  \catcode`\\=12 %
  \catcode`\#=12 %
  \def\par{^^J}%
  \obeylines
}
\msg{\contents}
\Check
\msg{***********************************************************}%

\csname @@end\endcsname
\end
%</test3>
%    \end{macrocode}
%
% \section{Installation}
%
% \subsection{Download}
%
% \paragraph{Package.} This package is available on
% CTAN\footnote{\url{https://ctan.org/pkg/catchfile}}:
% \begin{description}
% \item[\CTAN{macros/latex/contrib/oberdiek/catchfile.dtx}] The source file.
% \item[\CTAN{macros/latex/contrib/oberdiek/catchfile.pdf}] Documentation.
% \end{description}
%
%
% \paragraph{Bundle.} All the packages of the bundle `oberdiek'
% are also available in a TDS compliant ZIP archive. There
% the packages are already unpacked and the documentation files
% are generated. The files and directories obey the TDS standard.
% \begin{description}
% \item[\CTANinstall{install/macros/latex/contrib/oberdiek.tds.zip}]
% \end{description}
% \emph{TDS} refers to the standard ``A Directory Structure
% for \TeX\ Files'' (\CTAN{tds/tds.pdf}). Directories
% with \xfile{texmf} in their name are usually organized this way.
%
% \subsection{Bundle installation}
%
% \paragraph{Unpacking.} Unpack the \xfile{oberdiek.tds.zip} in the
% TDS tree (also known as \xfile{texmf} tree) of your choice.
% Example (linux):
% \begin{quote}
%   |unzip oberdiek.tds.zip -d ~/texmf|
% \end{quote}
%
% \paragraph{Script installation.}
% Check the directory \xfile{TDS:scripts/oberdiek/} for
% scripts that need further installation steps.
% Package \xpackage{attachfile2} comes with the Perl script
% \xfile{pdfatfi.pl} that should be installed in such a way
% that it can be called as \texttt{pdfatfi}.
% Example (linux):
% \begin{quote}
%   |chmod +x scripts/oberdiek/pdfatfi.pl|\\
%   |cp scripts/oberdiek/pdfatfi.pl /usr/local/bin/|
% \end{quote}
%
% \subsection{Package installation}
%
% \paragraph{Unpacking.} The \xfile{.dtx} file is a self-extracting
% \docstrip\ archive. The files are extracted by running the
% \xfile{.dtx} through \plainTeX:
% \begin{quote}
%   \verb|tex catchfile.dtx|
% \end{quote}
%
% \paragraph{TDS.} Now the different files must be moved into
% the different directories in your installation TDS tree
% (also known as \xfile{texmf} tree):
% \begin{quote}
% \def\t{^^A
% \begin{tabular}{@{}>{\ttfamily}l@{ $\rightarrow$ }>{\ttfamily}l@{}}
%   catchfile.sty & tex/generic/oberdiek/catchfile.sty\\
%   catchfile.pdf & doc/latex/oberdiek/catchfile.pdf\\
%   test/catchfile-test1.tex & doc/latex/oberdiek/test/catchfile-test1.tex\\
%   test/catchfile-test2.tex & doc/latex/oberdiek/test/catchfile-test2.tex\\
%   test/catchfile-test3.tex & doc/latex/oberdiek/test/catchfile-test3.tex\\
%   catchfile.dtx & source/latex/oberdiek/catchfile.dtx\\
% \end{tabular}^^A
% }^^A
% \sbox0{\t}^^A
% \ifdim\wd0>\linewidth
%   \begingroup
%     \advance\linewidth by\leftmargin
%     \advance\linewidth by\rightmargin
%   \edef\x{\endgroup
%     \def\noexpand\lw{\the\linewidth}^^A
%   }\x
%   \def\lwbox{^^A
%     \leavevmode
%     \hbox to \linewidth{^^A
%       \kern-\leftmargin\relax
%       \hss
%       \usebox0
%       \hss
%       \kern-\rightmargin\relax
%     }^^A
%   }^^A
%   \ifdim\wd0>\lw
%     \sbox0{\small\t}^^A
%     \ifdim\wd0>\linewidth
%       \ifdim\wd0>\lw
%         \sbox0{\footnotesize\t}^^A
%         \ifdim\wd0>\linewidth
%           \ifdim\wd0>\lw
%             \sbox0{\scriptsize\t}^^A
%             \ifdim\wd0>\linewidth
%               \ifdim\wd0>\lw
%                 \sbox0{\tiny\t}^^A
%                 \ifdim\wd0>\linewidth
%                   \lwbox
%                 \else
%                   \usebox0
%                 \fi
%               \else
%                 \lwbox
%               \fi
%             \else
%               \usebox0
%             \fi
%           \else
%             \lwbox
%           \fi
%         \else
%           \usebox0
%         \fi
%       \else
%         \lwbox
%       \fi
%     \else
%       \usebox0
%     \fi
%   \else
%     \lwbox
%   \fi
% \else
%   \usebox0
% \fi
% \end{quote}
% If you have a \xfile{docstrip.cfg} that configures and enables \docstrip's
% TDS installing feature, then some files can already be in the right
% place, see the documentation of \docstrip.
%
% \subsection{Refresh file name databases}
%
% If your \TeX~distribution
% (\teTeX, \mikTeX, \dots) relies on file name databases, you must refresh
% these. For example, \teTeX\ users run \verb|texhash| or
% \verb|mktexlsr|.
%
% \subsection{Some details for the interested}
%
% \paragraph{Attached source.}
%
% The PDF documentation on CTAN also includes the
% \xfile{.dtx} source file. It can be extracted by
% AcrobatReader 6 or higher. Another option is \textsf{pdftk},
% e.g. unpack the file into the current directory:
% \begin{quote}
%   \verb|pdftk catchfile.pdf unpack_files output .|
% \end{quote}
%
% \paragraph{Unpacking with \LaTeX.}
% The \xfile{.dtx} chooses its action depending on the format:
% \begin{description}
% \item[\plainTeX:] Run \docstrip\ and extract the files.
% \item[\LaTeX:] Generate the documentation.
% \end{description}
% If you insist on using \LaTeX\ for \docstrip\ (really,
% \docstrip\ does not need \LaTeX), then inform the autodetect routine
% about your intention:
% \begin{quote}
%   \verb|latex \let\install=y\input{catchfile.dtx}|
% \end{quote}
% Do not forget to quote the argument according to the demands
% of your shell.
%
% \paragraph{Generating the documentation.}
% You can use both the \xfile{.dtx} or the \xfile{.drv} to generate
% the documentation. The process can be configured by the
% configuration file \xfile{ltxdoc.cfg}. For instance, put this
% line into this file, if you want to have A4 as paper format:
% \begin{quote}
%   \verb|\PassOptionsToClass{a4paper}{article}|
% \end{quote}
% An example follows how to generate the
% documentation with pdf\LaTeX:
% \begin{quote}
%\begin{verbatim}
%pdflatex catchfile.dtx
%makeindex -s gind.ist catchfile.idx
%pdflatex catchfile.dtx
%makeindex -s gind.ist catchfile.idx
%pdflatex catchfile.dtx
%\end{verbatim}
% \end{quote}
%
% \begin{History}
%   \begin{Version}{2007/05/30 v1.0}
%   \item
%     First version.
%   \end{Version}
%   \begin{Version}{2007/09/09 v1.1}
%   \item
%     Catcode section rewritten.
%   \end{Version}
%   \begin{Version}{2007/11/11 v1.2}
%   \item
%     Use of package \xpackage{pdftexcmds} for \LuaTeX\ support.
%   \end{Version}
%   \begin{Version}{2010/03/01 v1.3}
%   \item
%     Fix for unknown \cs{@PackageErrorNoLine}.
%   \end{Version}
%   \begin{Version}{2010/04/08 v1.4}
%   \item
%     \cs{closein} also added if \cs{ifeof} is true.
%   \end{Version}
%   \begin{Version}{2010/04/28 v1.5}
%   \item
%     \cs{CatchFileDef}: Getting rid of warning `end of semi simple group
%     entered at line \dots\ of
%     a different file (Florent Chervet).
%   \item
%     \cs{CatchFileDef}: Fix for error `Illegal parameter number in
%     definition of \dots' (HO) including improvement that uses
%     \cs{unexpanded} if available (Florent Chervet).
%   \end{Version}
%   \begin{Version}{2011/03/01 v1.6}
%   \item
%     \cs{relax} added after the setup argument of
%     \cs{CatchFileDef} to prevent premature file reading
%     (Martin Scharrer).
%   \end{Version}
%   \begin{Version}{2016/05/16 v1.7}
%   \item
%     Documentation updates.
%   \end{Version}
% \end{History}
%
% \PrintIndex
%
% \Finale
\endinput

%        (quote the arguments according to the demands of your shell)
%
% Documentation:
%    (a) If catchfile.drv is present:
%           latex catchfile.drv
%    (b) Without catchfile.drv:
%           latex catchfile.dtx; ...
%    The class ltxdoc loads the configuration file ltxdoc.cfg
%    if available. Here you can specify further options, e.g.
%    use A4 as paper format:
%       \PassOptionsToClass{a4paper}{article}
%
%    Programm calls to get the documentation (example):
%       pdflatex catchfile.dtx
%       makeindex -s gind.ist catchfile.idx
%       pdflatex catchfile.dtx
%       makeindex -s gind.ist catchfile.idx
%       pdflatex catchfile.dtx
%
% Installation:
%    TDS:tex/generic/oberdiek/catchfile.sty
%    TDS:doc/latex/oberdiek/catchfile.pdf
%    TDS:doc/latex/oberdiek/test/catchfile-test1.tex
%    TDS:doc/latex/oberdiek/test/catchfile-test2.tex
%    TDS:doc/latex/oberdiek/test/catchfile-test3.tex
%    TDS:source/latex/oberdiek/catchfile.dtx
%
%<*ignore>
\begingroup
  \catcode123=1 %
  \catcode125=2 %
  \def\x{LaTeX2e}%
\expandafter\endgroup
\ifcase 0\ifx\install y1\fi\expandafter
         \ifx\csname processbatchFile\endcsname\relax\else1\fi
         \ifx\fmtname\x\else 1\fi\relax
\else\csname fi\endcsname
%</ignore>
%<*install>
\input docstrip.tex
\Msg{************************************************************************}
\Msg{* Installation}
\Msg{* Package: catchfile 2016/05/16 v1.7 Catch the contents of a file (HO)}
\Msg{************************************************************************}

\keepsilent
\askforoverwritefalse

\let\MetaPrefix\relax
\preamble

This is a generated file.

Project: catchfile
Version: 2016/05/16 v1.7

Copyright (C) 2007, 2010, 2011 by
   Heiko Oberdiek <heiko.oberdiek at googlemail.com>

This work may be distributed and/or modified under the
conditions of the LaTeX Project Public License, either
version 1.3c of this license or (at your option) any later
version. This version of this license is in
   http://www.latex-project.org/lppl/lppl-1-3c.txt
and the latest version of this license is in
   http://www.latex-project.org/lppl.txt
and version 1.3 or later is part of all distributions of
LaTeX version 2005/12/01 or later.

This work has the LPPL maintenance status "maintained".

This Current Maintainer of this work is Heiko Oberdiek.

The Base Interpreter refers to any `TeX-Format',
because some files are installed in TDS:tex/generic//.

This work consists of the main source file catchfile.dtx
and the derived files
   catchfile.sty, catchfile.pdf, catchfile.ins, catchfile.drv,
   catchfile-test1.tex, catchfile-test2.tex,
   catchfile-test3.tex.

\endpreamble
\let\MetaPrefix\DoubleperCent

\generate{%
  \file{catchfile.ins}{\from{catchfile.dtx}{install}}%
  \file{catchfile.drv}{\from{catchfile.dtx}{driver}}%
  \usedir{tex/generic/oberdiek}%
  \file{catchfile.sty}{\from{catchfile.dtx}{package}}%
%  \usedir{doc/latex/oberdiek/test}%
%  \file{catchfile-test1.tex}{\from{catchfile.dtx}{test1}}%
%  \file{catchfile-test2.tex}{\from{catchfile.dtx}{test2}}%
%  \file{catchfile-test3.tex}{\from{catchfile.dtx}{test3}}%
  \nopreamble
  \nopostamble
%  \usedir{source/latex/oberdiek/catalogue}%
%  \file{catchfile.xml}{\from{catchfile.dtx}{catalogue}}%
}

\catcode32=13\relax% active space
\let =\space%
\Msg{************************************************************************}
\Msg{*}
\Msg{* To finish the installation you have to move the following}
\Msg{* file into a directory searched by TeX:}
\Msg{*}
\Msg{*     catchfile.sty}
\Msg{*}
\Msg{* To produce the documentation run the file `catchfile.drv'}
\Msg{* through LaTeX.}
\Msg{*}
\Msg{* Happy TeXing!}
\Msg{*}
\Msg{************************************************************************}

\endbatchfile
%</install>
%<*ignore>
\fi
%</ignore>
%<*driver>
\NeedsTeXFormat{LaTeX2e}
\ProvidesFile{catchfile.drv}%
  [2016/05/16 v1.7 Catch the contents of a file (HO)]%
\documentclass{ltxdoc}
\usepackage{holtxdoc}[2011/11/22]
\begin{document}
  \DocInput{catchfile.dtx}%
\end{document}
%</driver>
% \fi
%
%
% \CharacterTable
%  {Upper-case    \A\B\C\D\E\F\G\H\I\J\K\L\M\N\O\P\Q\R\S\T\U\V\W\X\Y\Z
%   Lower-case    \a\b\c\d\e\f\g\h\i\j\k\l\m\n\o\p\q\r\s\t\u\v\w\x\y\z
%   Digits        \0\1\2\3\4\5\6\7\8\9
%   Exclamation   \!     Double quote  \"     Hash (number) \#
%   Dollar        \$     Percent       \%     Ampersand     \&
%   Acute accent  \'     Left paren    \(     Right paren   \)
%   Asterisk      \*     Plus          \+     Comma         \,
%   Minus         \-     Point         \.     Solidus       \/
%   Colon         \:     Semicolon     \;     Less than     \<
%   Equals        \=     Greater than  \>     Question mark \?
%   Commercial at \@     Left bracket  \[     Backslash     \\
%   Right bracket \]     Circumflex    \^     Underscore    \_
%   Grave accent  \`     Left brace    \{     Vertical bar  \|
%   Right brace   \}     Tilde         \~}
%
% \GetFileInfo{catchfile.drv}
%
% \title{The \xpackage{catchfile} package}
% \date{2016/05/16 v1.7}
% \author{Heiko Oberdiek\thanks
% {Please report any issues at https://github.com/ho-tex/oberdiek/issues}\\
% \xemail{heiko.oberdiek at googlemail.com}}
%
% \maketitle
%
% \begin{abstract}
% This package catches the contents of a file and puts it in a macro.
% It requires \eTeX. Both \LaTeX\ and \plainTeX\ are supported.
% \end{abstract}
%
% \tableofcontents
%
% \section{Documentation}
%
% The package relies on \eTeX's \cs{everyeof}. Otherwise it
% aborts with an error message.
%
% \begin{declcs}{CatchFileDef} \M{cmd} \M{file name} \M{setup}\\
% \cs{CatchFileEdef} \M{cmd} \M{file name} \M{setup}
% \end{declcs}
% Macro \meta{cmd} is defined with the contents
% of file \meta{file name}.
% \cs{CatchFileDef} uses \cs{def}, \cs{CatchFileEdef} \cs{edef}
% for the definition.
% Additional setup code for setting
% catcodes or treatment of line ends can be given in code
% \meta{setup}. See the test files for an example.
%
% \StopEventually{
% }
%
% \section{Implementation}
%
%    \begin{macrocode}
%<*package>
%    \end{macrocode}
%
% \subsection{Reload check and package identification}
%    Reload check, especially if the package is not used with \LaTeX.
%    \begin{macrocode}
\begingroup\catcode61\catcode48\catcode32=10\relax%
  \catcode13=5 % ^^M
  \endlinechar=13 %
  \catcode35=6 % #
  \catcode39=12 % '
  \catcode44=12 % ,
  \catcode45=12 % -
  \catcode46=12 % .
  \catcode58=12 % :
  \catcode64=11 % @
  \catcode123=1 % {
  \catcode125=2 % }
  \expandafter\let\expandafter\x\csname ver@catchfile.sty\endcsname
  \ifx\x\relax % plain-TeX, first loading
  \else
    \def\empty{}%
    \ifx\x\empty % LaTeX, first loading,
      % variable is initialized, but \ProvidesPackage not yet seen
    \else
      \expandafter\ifx\csname PackageInfo\endcsname\relax
        \def\x#1#2{%
          \immediate\write-1{Package #1 Info: #2.}%
        }%
      \else
        \def\x#1#2{\PackageInfo{#1}{#2, stopped}}%
      \fi
      \x{catchfile}{The package is already loaded}%
      \aftergroup\endinput
    \fi
  \fi
\endgroup%
%    \end{macrocode}
%    Package identification:
%    \begin{macrocode}
\begingroup\catcode61\catcode48\catcode32=10\relax%
  \catcode13=5 % ^^M
  \endlinechar=13 %
  \catcode35=6 % #
  \catcode39=12 % '
  \catcode40=12 % (
  \catcode41=12 % )
  \catcode44=12 % ,
  \catcode45=12 % -
  \catcode46=12 % .
  \catcode47=12 % /
  \catcode58=12 % :
  \catcode64=11 % @
  \catcode91=12 % [
  \catcode93=12 % ]
  \catcode123=1 % {
  \catcode125=2 % }
  \expandafter\ifx\csname ProvidesPackage\endcsname\relax
    \def\x#1#2#3[#4]{\endgroup
      \immediate\write-1{Package: #3 #4}%
      \xdef#1{#4}%
    }%
  \else
    \def\x#1#2[#3]{\endgroup
      #2[{#3}]%
      \ifx#1\@undefined
        \xdef#1{#3}%
      \fi
      \ifx#1\relax
        \xdef#1{#3}%
      \fi
    }%
  \fi
\expandafter\x\csname ver@catchfile.sty\endcsname
\ProvidesPackage{catchfile}%
  [2016/05/16 v1.7 Catch the contents of a file (HO)]%
%    \end{macrocode}
%
% \subsection{Catcodes}
%
%    \begin{macrocode}
\begingroup\catcode61\catcode48\catcode32=10\relax%
  \catcode13=5 % ^^M
  \endlinechar=13 %
  \catcode123=1 % {
  \catcode125=2 % }
  \catcode64=11 % @
  \def\x{\endgroup
    \expandafter\edef\csname CatchFile@AtEnd\endcsname{%
      \endlinechar=\the\endlinechar\relax
      \catcode13=\the\catcode13\relax
      \catcode32=\the\catcode32\relax
      \catcode35=\the\catcode35\relax
      \catcode61=\the\catcode61\relax
      \catcode64=\the\catcode64\relax
      \catcode123=\the\catcode123\relax
      \catcode125=\the\catcode125\relax
    }%
  }%
\x\catcode61\catcode48\catcode32=10\relax%
\catcode13=5 % ^^M
\endlinechar=13 %
\catcode35=6 % #
\catcode64=11 % @
\catcode123=1 % {
\catcode125=2 % }
\def\TMP@EnsureCode#1#2{%
  \edef\CatchFile@AtEnd{%
    \CatchFile@AtEnd
    \catcode#1=\the\catcode#1\relax
  }%
  \catcode#1=#2\relax
}
\TMP@EnsureCode{39}{12}% '
\TMP@EnsureCode{44}{12}% ,
\TMP@EnsureCode{45}{12}% -
\TMP@EnsureCode{46}{12}% .
\TMP@EnsureCode{47}{12}% /
\TMP@EnsureCode{91}{12}% [
\TMP@EnsureCode{93}{12}% ]
\TMP@EnsureCode{96}{12}% `
\edef\CatchFile@AtEnd{\CatchFile@AtEnd\noexpand\endinput}
%    \end{macrocode}
%
% \subsection{Preparations}
%
%    \begin{macrocode}
\begingroup\expandafter\expandafter\expandafter\endgroup
\expandafter\ifx\csname RequirePackage\endcsname\relax
  \input infwarerr.sty\relax
  \input ltxcmds.sty\relax
\else
  \RequirePackage{infwarerr}[2007/09/09]%
  \RequirePackage{ltxcmds}[2010/03/09]%
\fi
%    \end{macrocode}
%
%    Check for \eTeX's \cs{everyeof}.
%    \begin{macrocode}
\begingroup
  \escapechar=92\relax
  \edef\TestString{\string\everyeof}%
  \edef\TestMeaning{\meaning\everyeof}%
  \ifx\TestString\TestMeaning
  \else
    \@PackageError{catchfile}{%
      Cannot find e-TeX's \string\everyeof,\MessageBreak
      package loading is aborted%
    }\@ehd
    \endgroup
    \expandafter\CatchFile@AtEnd
  \fi%
\endgroup
%    \end{macrocode}
%
% \subsection{Looking for primitive \cs{input}}
%
%    \begin{macro}{\CatchFile@Input}
%    The package needs the expandable primitive \cs{input}.
%    However there are formats that redefine it. For example,
%    \LaTeX's \cs{input} is not expandable, but it stores
%    the primitive in \cs{@@input}. The third possibility
%    is \cs{pdfprimitive}, introduced in \pdfTeX\ 1.40.0.
%
%    Thus we try to find the primitive and store it in
%    \cs{CatchFile@Input}. If it is used, it must be expanded
%    twice (because of the solution with \cs{pdfprimitive}).
%    \begin{macrocode}
\begingroup
  \def\Check#1#2#3#4\endgroup{%
    \edef\TestString{\string#1}%
    \edef\TestMeaning{\meaning#2}%
    \ifx\TestString\TestMeaning
      \endgroup
      \let\CatchFile@Primitive#2%
      \def\CatchFile@Input{\CatchFile@Primitive#3}%
    \else
      #4\endgroup
    \fi
  }%
  \Check\input\input{}%
  \Check\input\@@input{}%
  \Check\pdfprimitive\pdfprimitive\input
  \@PackageError{%
    Cannot find primitive \string\input,\MessageBreak
    package loading is aborted%
  }\@ehd
  \csname endgroup\endcsname
  \CatchFile@AtEnd%
\endgroup
%    \end{macrocode}
%    \end{macro}
%
% \subsection{Input file check}
%
%    \begin{macro}{\CatchFile@CheckFileExists}
%    \begin{macrocode}
\begingroup\expandafter\expandafter\expandafter\endgroup
\expandafter\ifx\csname IfFileExists\endcsname\relax
  \input pdftexcmds.sty\relax
  \begingroup\expandafter\expandafter\expandafter\endgroup
  \expandafter\ifx\csname pdf@filesize\endcsname\relax
    \def\CatchFile@CheckFileExists#1{%
      \expandafter\ifx\csname @inputcheck\endcsname\relax
        \csname newread\endcsname\@inputcheck
      \fi
      \openin\@inputcheck#1\relax
      \expandafter\closein\expandafter\@inputcheck
      \ifeof\@inputcheck
        \let\CatchFile@File\relax
      \else
        \def\CatchFile@File{#1}%
      \fi
    }%
  \else
    \def\CatchFile@CheckFileExists#1{%
      \expandafter\expandafter\expandafter\ifx
      \expandafter\expandafter\expandafter\relax\pdf@filesize{#1}\relax
        \let\CatchFile@File\relax
      \else
        \def\CatchFile@File{#1}%
      \fi
    }%
  \fi
\else
  \def\CatchFile@CheckFileExists#1{%
    \IfFileExists{#1}{%
      \expandafter\CatchFile@DefFile\@filef@und\@nil
      \begingroup\expandafter\expandafter\expandafter\endgroup
      \expandafter\ifx\csname @addtofilelist\endcsname\relax
      \else
        \@addtofilelist\CatchFile@File
      \fi
    }{%
      \let\CatchFile@File\relax
    }%
  }%
  \def\CatchFile@DefFile#1 \@nil{%
    \def\CatchFile@File{#1}%
  }%
\fi
%    \end{macrocode}
%    \end{macro}
%
%    \begin{macro}{\CatchFile@NotFound}
%    \begin{macrocode}
\def\CatchFile@NotFound#1#2{%
  \def#1{}%
  \@PackageError{catchfile}{%
    File `#2' not found%
  }\@ehc
}
%    \end{macrocode}
%    \end{macro}
%
% \subsection{Catch file contents}
%
%    \begin{macrocode}
\ltx@IfUndefined{RequirePackage}{%
  \input etexcmds.sty\relax
}{%
  \RequirePackage{etexcmds}[2010/01/28]%
}
%    \end{macrocode}
%
%    \begin{macro}{\CatchFileEdef}
%    \begin{macrocode}
\long\def\CatchFileEdef#1#2#3{%
  \CatchFile@CheckFileExists{#2}%
  \ifx\CatchFile@File\relax
    \CatchFile@NotFound{#1}{#2}%
  \else
    \begingroup
      \everyeof{\noexpand}%
      #3%
      \xdef\CatchFile@Contents{\CatchFile@Input\CatchFile@File\space}%
    \endgroup
    \let#1\CatchFile@Contents
  \fi
}
%    \end{macrocode}
%    \end{macro}
%
%    \begin{macro}{\CatchFileDef}
%    \begin{macrocode}
\long\def\CatchFileDef#1#2#3{%
  \CatchFile@CheckFileExists{#2}%
  \ifx\CatchFile@File\relax
    \CatchFile@NotFound{#1}{#2}%
  \else
    \begingroup
      \everyeof\expandafter{%
        \CatchFile@EOF
        \expandafter\CatchFile@Finish
        \noexpand
      }%
      \ltx@ifundefined{etex@unexpanded}{%
        \expandafter\long\expandafter\def\expandafter\CatchFile@Do
            \expandafter##\expandafter1\CatchFile@EOF{%
          \toks\ltx@zero{##1}%
          \xdef\CatchFile@gtemp{\the\toks\ltx@zero}%
          \def\CatchFile@Finish{%
            \endgroup
            \let#1\CatchFile@gtemp
            \global\let\CatchFile@gtemp\ltx@undefined
          }%
        }%
      }{%
        \expandafter\long\expandafter\def\expandafter\CatchFile@Do
            \expandafter##\expandafter1\CatchFile@EOF{%
          \edef\CatchFile@Finish{%
            \endgroup
            \etex@unexpanded{%
              \edef#1{\etex@unexpanded{##1}}%
            }%
          }%
        }%
      }%
      #3\relax
    \expandafter\expandafter\expandafter\CatchFile@Do
    \CatchFile@Input\CatchFile@File\relax
  \fi
}
%    \end{macrocode}
%    \cs{relax} after |#3| was added to make it more robust in
%    case the user uses something like
%\begin{verbatim}
%\CatchFileDef{\content}{\jobname.tt}{\endlinechar=-1}
%\end{verbatim}
%    that expands the following \cs{expandafter} after |#3| prematurely
%    (contribution of Martin Scharrer).
%    \end{macro}
%    \begin{macro}{\CatchFile@EOF}
%    If the file is read the catcode mappings are fixed.
%    This means that the same character cannot occur
%    inside the file with different catcodes.
%    Thus we use as end of file marker the at sign
%    twice with different catcodes.
%    \begin{macrocode}
\begingroup
  \lccode65=64 % lowercase('A') = '@'
  \lccode66=64 % lowercase('B') = '@'
  \catcode65=8 % catcode('A') = subscript
  \catcode66=3 % catcode('B') = math shift
\lowercase{\endgroup
  \def\CatchFile@EOF{AB}%
}
%    \end{macrocode}
%    \end{macro}
%
%    \begin{macrocode}
\CatchFile@AtEnd%
%</package>
%    \end{macrocode}
%
% \section{Test}
%
% \subsection{Catcode checks for loading}
%
%    \begin{macrocode}
%<*test1>
%    \end{macrocode}
%    \begin{macrocode}
\catcode`\{=1 %
\catcode`\}=2 %
\catcode`\#=6 %
\catcode`\@=11 %
\expandafter\ifx\csname count@\endcsname\relax
  \countdef\count@=255 %
\fi
\expandafter\ifx\csname @gobble\endcsname\relax
  \long\def\@gobble#1{}%
\fi
\expandafter\ifx\csname @firstofone\endcsname\relax
  \long\def\@firstofone#1{#1}%
\fi
\expandafter\ifx\csname loop\endcsname\relax
  \expandafter\@firstofone
\else
  \expandafter\@gobble
\fi
{%
  \def\loop#1\repeat{%
    \def\body{#1}%
    \iterate
  }%
  \def\iterate{%
    \body
      \let\next\iterate
    \else
      \let\next\relax
    \fi
    \next
  }%
  \let\repeat=\fi
}%
\def\RestoreCatcodes{}
\count@=0 %
\loop
  \edef\RestoreCatcodes{%
    \RestoreCatcodes
    \catcode\the\count@=\the\catcode\count@\relax
  }%
\ifnum\count@<255 %
  \advance\count@ 1 %
\repeat

\def\RangeCatcodeInvalid#1#2{%
  \count@=#1\relax
  \loop
    \catcode\count@=15 %
  \ifnum\count@<#2\relax
    \advance\count@ 1 %
  \repeat
}
\def\RangeCatcodeCheck#1#2#3{%
  \count@=#1\relax
  \loop
    \ifnum#3=\catcode\count@
    \else
      \errmessage{%
        Character \the\count@\space
        with wrong catcode \the\catcode\count@\space
        instead of \number#3%
      }%
    \fi
  \ifnum\count@<#2\relax
    \advance\count@ 1 %
  \repeat
}
\def\space{ }
\expandafter\ifx\csname LoadCommand\endcsname\relax
  \def\LoadCommand{\input catchfile.sty\relax}%
\fi
\def\Test{%
  \RangeCatcodeInvalid{0}{47}%
  \RangeCatcodeInvalid{58}{64}%
  \RangeCatcodeInvalid{91}{96}%
  \RangeCatcodeInvalid{123}{255}%
  \catcode`\@=12 %
  \catcode`\\=0 %
  \catcode`\%=14 %
  \LoadCommand
  \RangeCatcodeCheck{0}{36}{15}%
  \RangeCatcodeCheck{37}{37}{14}%
  \RangeCatcodeCheck{38}{47}{15}%
  \RangeCatcodeCheck{48}{57}{12}%
  \RangeCatcodeCheck{58}{63}{15}%
  \RangeCatcodeCheck{64}{64}{12}%
  \RangeCatcodeCheck{65}{90}{11}%
  \RangeCatcodeCheck{91}{91}{15}%
  \RangeCatcodeCheck{92}{92}{0}%
  \RangeCatcodeCheck{93}{96}{15}%
  \RangeCatcodeCheck{97}{122}{11}%
  \RangeCatcodeCheck{123}{255}{15}%
  \RestoreCatcodes
}
\Test
\csname @@end\endcsname
\end
%    \end{macrocode}
%    \begin{macrocode}
%</test1>
%    \end{macrocode}
%
% \subsection{\LaTeX}
%
%    \begin{macrocode}
%<*test2>
\NeedsTeXFormat{LaTeX2e}
\nofiles
\listfiles
\tracingnesting=2 %
\documentclass{minimal}
\usepackage{catchfile}[2016/05/16]
\makeatletter
\def\mysetup{%
  \let\do\@makeother
  \dospecials
}
\def\CheckContents{%
  \begingroup
    \expandafter\ifx\csname contents\endcsname\relax
      \PackageError{TEST}{\string\contents\space is not defined}\@ehc
    \fi
  \endgroup
}
\def\StartDisableUnexpanded{%
  \begingroup
    \let\etex@unexpanded\@undefined
}
\def\StopDisableUnexpanded{%
  \endgroup
}
\def\CheckCleanup{%
  \begingroup
    \edef\x{\the\toks0}%
    \def\y{ABC}%
    \ifx\x\y
    \else
      \PackageError{TEST}{\string\toks0 has changed}\@ehc
    \fi
    \ifx\CatchFile@gtemp\@undefined
    \else
      \PackageError{TEST}{Left over \string\CatchFile@gtemp}\@ehc
    \fi
  \endgroup
}
\def\Check{%
  \CheckContents
  \CheckCleanup
}
\makeatother
\begin{document}
\toks0{ABC}

\CatchFileDef\contents{catchfile.sty}\mysetup
\typeout{\meaning\contents}
\Check
\typeout{***********************************************************}%
\CatchFileDef\contents{catchfile.sty}{}
\typeout{\meaning\contents}
\Check
\typeout{***********************************************************}%

\StartDisableUnexpanded
\CatchFileDef\contents{catchfile.sty}\mysetup
\typeout{\meaning\contents}
\Check
\typeout{***********************************************************}%
\CatchFileDef\contents{catchfile.sty}{}
\typeout{\meaning\contents}
\Check
\typeout{***********************************************************}%
\StopDisableUnexpanded

\CatchFileEdef\contents{catchfile.sty}{%
  \mysetup
  \def\par{^^J}%
  \obeylines
}
\typeout{\contents}
\Check
\typeout{***********************************************************}%
\CatchFileEdef\contents{catchfile.sty}{%
  \catcode`\\=12 %
  \catcode`\#=12 %
  \def\par{^^J}%
  \obeylines
}
\typeout{\contents}
\Check
\typeout{***********************************************************}%
\end{document}
%</test2>
%    \end{macrocode}
%
% \subsection{\plainTeX}
%
%    \begin{macrocode}
%<*test3>
\def\msg#{\immediate\write16}
\newlinechar=10 %
\tracingnesting=2 %
\input catchfile.sty\relax

\def\mysetup{%
  \def\do##1{%
    \catcode`##1=12\relax
  }%
  \dospecials
}
\def\CheckContents{%
  \begingroup
    \expandafter\ifx\csname contents\endcsname\relax
      \def\space{ }%
      \errmessage{\string\contents\space is not defined.}%
    \fi
  \endgroup
}
\catcode`\@=11 %
\def\CheckCleanup{%
  \begingroup
    \edef\x{\the\toks0}%
    \def\y{ABC}%
    \ifx\x\y
    \else
      \errmessage{\string\toks0 has changed}%
    \fi
    \ifx\CatchFile@gtemp\@undefined
    \else
      \errmessage{Left over \string\CatchFile@gtemp}%
    \fi
  \endgroup
}
\def\Check{%
  \CheckContents
  \CheckCleanup
}
\def\StartDisableUnexpanded{%
  \begingroup
    \let\etex@unexpanded\@undefined
}
\def\StopDisableUnexpanded{%
  \endgroup
}
\catcode`\@=12 %

\toks0{ABC}

\CatchFileDef\contents{catchfile.sty}\mysetup
\msg{\meaning\contents}
\Check
\msg{***********************************************************}%
\CatchFileDef\contents{catchfile.sty}{}
\msg{\meaning\contents}
\Check
\msg{***********************************************************}%

\StartDisableUnexpanded
\CatchFileDef\contents{catchfile.sty}\mysetup
\msg{\meaning\contents}
\Check
\msg{***********************************************************}%
\CatchFileDef\contents{catchfile.sty}{}
\msg{\meaning\contents}
\Check
\msg{***********************************************************}%
\StopDisableUnexpanded

\CatchFileEdef\contents{catchfile.sty}{%
  \mysetup
  \def\par{^^J}%
  \obeylines
}
\msg{\contents}
\Check
\msg{***********************************************************}%
\CatchFileEdef\contents{catchfile.sty}{%
  \catcode`\\=12 %
  \catcode`\#=12 %
  \def\par{^^J}%
  \obeylines
}
\msg{\contents}
\Check
\msg{***********************************************************}%

\csname @@end\endcsname
\end
%</test3>
%    \end{macrocode}
%
% \section{Installation}
%
% \subsection{Download}
%
% \paragraph{Package.} This package is available on
% CTAN\footnote{\url{https://ctan.org/pkg/catchfile}}:
% \begin{description}
% \item[\CTAN{macros/latex/contrib/oberdiek/catchfile.dtx}] The source file.
% \item[\CTAN{macros/latex/contrib/oberdiek/catchfile.pdf}] Documentation.
% \end{description}
%
%
% \paragraph{Bundle.} All the packages of the bundle `oberdiek'
% are also available in a TDS compliant ZIP archive. There
% the packages are already unpacked and the documentation files
% are generated. The files and directories obey the TDS standard.
% \begin{description}
% \item[\CTANinstall{install/macros/latex/contrib/oberdiek.tds.zip}]
% \end{description}
% \emph{TDS} refers to the standard ``A Directory Structure
% for \TeX\ Files'' (\CTAN{tds/tds.pdf}). Directories
% with \xfile{texmf} in their name are usually organized this way.
%
% \subsection{Bundle installation}
%
% \paragraph{Unpacking.} Unpack the \xfile{oberdiek.tds.zip} in the
% TDS tree (also known as \xfile{texmf} tree) of your choice.
% Example (linux):
% \begin{quote}
%   |unzip oberdiek.tds.zip -d ~/texmf|
% \end{quote}
%
% \paragraph{Script installation.}
% Check the directory \xfile{TDS:scripts/oberdiek/} for
% scripts that need further installation steps.
% Package \xpackage{attachfile2} comes with the Perl script
% \xfile{pdfatfi.pl} that should be installed in such a way
% that it can be called as \texttt{pdfatfi}.
% Example (linux):
% \begin{quote}
%   |chmod +x scripts/oberdiek/pdfatfi.pl|\\
%   |cp scripts/oberdiek/pdfatfi.pl /usr/local/bin/|
% \end{quote}
%
% \subsection{Package installation}
%
% \paragraph{Unpacking.} The \xfile{.dtx} file is a self-extracting
% \docstrip\ archive. The files are extracted by running the
% \xfile{.dtx} through \plainTeX:
% \begin{quote}
%   \verb|tex catchfile.dtx|
% \end{quote}
%
% \paragraph{TDS.} Now the different files must be moved into
% the different directories in your installation TDS tree
% (also known as \xfile{texmf} tree):
% \begin{quote}
% \def\t{^^A
% \begin{tabular}{@{}>{\ttfamily}l@{ $\rightarrow$ }>{\ttfamily}l@{}}
%   catchfile.sty & tex/generic/oberdiek/catchfile.sty\\
%   catchfile.pdf & doc/latex/oberdiek/catchfile.pdf\\
%   test/catchfile-test1.tex & doc/latex/oberdiek/test/catchfile-test1.tex\\
%   test/catchfile-test2.tex & doc/latex/oberdiek/test/catchfile-test2.tex\\
%   test/catchfile-test3.tex & doc/latex/oberdiek/test/catchfile-test3.tex\\
%   catchfile.dtx & source/latex/oberdiek/catchfile.dtx\\
% \end{tabular}^^A
% }^^A
% \sbox0{\t}^^A
% \ifdim\wd0>\linewidth
%   \begingroup
%     \advance\linewidth by\leftmargin
%     \advance\linewidth by\rightmargin
%   \edef\x{\endgroup
%     \def\noexpand\lw{\the\linewidth}^^A
%   }\x
%   \def\lwbox{^^A
%     \leavevmode
%     \hbox to \linewidth{^^A
%       \kern-\leftmargin\relax
%       \hss
%       \usebox0
%       \hss
%       \kern-\rightmargin\relax
%     }^^A
%   }^^A
%   \ifdim\wd0>\lw
%     \sbox0{\small\t}^^A
%     \ifdim\wd0>\linewidth
%       \ifdim\wd0>\lw
%         \sbox0{\footnotesize\t}^^A
%         \ifdim\wd0>\linewidth
%           \ifdim\wd0>\lw
%             \sbox0{\scriptsize\t}^^A
%             \ifdim\wd0>\linewidth
%               \ifdim\wd0>\lw
%                 \sbox0{\tiny\t}^^A
%                 \ifdim\wd0>\linewidth
%                   \lwbox
%                 \else
%                   \usebox0
%                 \fi
%               \else
%                 \lwbox
%               \fi
%             \else
%               \usebox0
%             \fi
%           \else
%             \lwbox
%           \fi
%         \else
%           \usebox0
%         \fi
%       \else
%         \lwbox
%       \fi
%     \else
%       \usebox0
%     \fi
%   \else
%     \lwbox
%   \fi
% \else
%   \usebox0
% \fi
% \end{quote}
% If you have a \xfile{docstrip.cfg} that configures and enables \docstrip's
% TDS installing feature, then some files can already be in the right
% place, see the documentation of \docstrip.
%
% \subsection{Refresh file name databases}
%
% If your \TeX~distribution
% (\teTeX, \mikTeX, \dots) relies on file name databases, you must refresh
% these. For example, \teTeX\ users run \verb|texhash| or
% \verb|mktexlsr|.
%
% \subsection{Some details for the interested}
%
% \paragraph{Attached source.}
%
% The PDF documentation on CTAN also includes the
% \xfile{.dtx} source file. It can be extracted by
% AcrobatReader 6 or higher. Another option is \textsf{pdftk},
% e.g. unpack the file into the current directory:
% \begin{quote}
%   \verb|pdftk catchfile.pdf unpack_files output .|
% \end{quote}
%
% \paragraph{Unpacking with \LaTeX.}
% The \xfile{.dtx} chooses its action depending on the format:
% \begin{description}
% \item[\plainTeX:] Run \docstrip\ and extract the files.
% \item[\LaTeX:] Generate the documentation.
% \end{description}
% If you insist on using \LaTeX\ for \docstrip\ (really,
% \docstrip\ does not need \LaTeX), then inform the autodetect routine
% about your intention:
% \begin{quote}
%   \verb|latex \let\install=y% \iffalse meta-comment
%
% File: catchfile.dtx
% Version: 2016/05/16 v1.7
% Info: Catch the contents of a file
%
% Copyright (C) 2007, 2010, 2011 by
%    Heiko Oberdiek <heiko.oberdiek at googlemail.com>
%    2016
%    https://github.com/ho-tex/oberdiek/issues
%
% This work may be distributed and/or modified under the
% conditions of the LaTeX Project Public License, either
% version 1.3c of this license or (at your option) any later
% version. This version of this license is in
%    http://www.latex-project.org/lppl/lppl-1-3c.txt
% and the latest version of this license is in
%    http://www.latex-project.org/lppl.txt
% and version 1.3 or later is part of all distributions of
% LaTeX version 2005/12/01 or later.
%
% This work has the LPPL maintenance status "maintained".
%
% This Current Maintainer of this work is Heiko Oberdiek.
%
% The Base Interpreter refers to any `TeX-Format',
% because some files are installed in TDS:tex/generic//.
%
% This work consists of the main source file catchfile.dtx
% and the derived files
%    catchfile.sty, catchfile.pdf, catchfile.ins, catchfile.drv,
%    catchfile-test1.tex, catchfile-test2.tex,
%    catchfile-test3.tex.
%
% Distribution:
%    CTAN:macros/latex/contrib/oberdiek/catchfile.dtx
%    CTAN:macros/latex/contrib/oberdiek/catchfile.pdf
%
% Unpacking:
%    (a) If catchfile.ins is present:
%           tex catchfile.ins
%    (b) Without catchfile.ins:
%           tex catchfile.dtx
%    (c) If you insist on using LaTeX
%           latex \let\install=y\input{catchfile.dtx}
%        (quote the arguments according to the demands of your shell)
%
% Documentation:
%    (a) If catchfile.drv is present:
%           latex catchfile.drv
%    (b) Without catchfile.drv:
%           latex catchfile.dtx; ...
%    The class ltxdoc loads the configuration file ltxdoc.cfg
%    if available. Here you can specify further options, e.g.
%    use A4 as paper format:
%       \PassOptionsToClass{a4paper}{article}
%
%    Programm calls to get the documentation (example):
%       pdflatex catchfile.dtx
%       makeindex -s gind.ist catchfile.idx
%       pdflatex catchfile.dtx
%       makeindex -s gind.ist catchfile.idx
%       pdflatex catchfile.dtx
%
% Installation:
%    TDS:tex/generic/oberdiek/catchfile.sty
%    TDS:doc/latex/oberdiek/catchfile.pdf
%    TDS:doc/latex/oberdiek/test/catchfile-test1.tex
%    TDS:doc/latex/oberdiek/test/catchfile-test2.tex
%    TDS:doc/latex/oberdiek/test/catchfile-test3.tex
%    TDS:source/latex/oberdiek/catchfile.dtx
%
%<*ignore>
\begingroup
  \catcode123=1 %
  \catcode125=2 %
  \def\x{LaTeX2e}%
\expandafter\endgroup
\ifcase 0\ifx\install y1\fi\expandafter
         \ifx\csname processbatchFile\endcsname\relax\else1\fi
         \ifx\fmtname\x\else 1\fi\relax
\else\csname fi\endcsname
%</ignore>
%<*install>
\input docstrip.tex
\Msg{************************************************************************}
\Msg{* Installation}
\Msg{* Package: catchfile 2016/05/16 v1.7 Catch the contents of a file (HO)}
\Msg{************************************************************************}

\keepsilent
\askforoverwritefalse

\let\MetaPrefix\relax
\preamble

This is a generated file.

Project: catchfile
Version: 2016/05/16 v1.7

Copyright (C) 2007, 2010, 2011 by
   Heiko Oberdiek <heiko.oberdiek at googlemail.com>

This work may be distributed and/or modified under the
conditions of the LaTeX Project Public License, either
version 1.3c of this license or (at your option) any later
version. This version of this license is in
   http://www.latex-project.org/lppl/lppl-1-3c.txt
and the latest version of this license is in
   http://www.latex-project.org/lppl.txt
and version 1.3 or later is part of all distributions of
LaTeX version 2005/12/01 or later.

This work has the LPPL maintenance status "maintained".

This Current Maintainer of this work is Heiko Oberdiek.

The Base Interpreter refers to any `TeX-Format',
because some files are installed in TDS:tex/generic//.

This work consists of the main source file catchfile.dtx
and the derived files
   catchfile.sty, catchfile.pdf, catchfile.ins, catchfile.drv,
   catchfile-test1.tex, catchfile-test2.tex,
   catchfile-test3.tex.

\endpreamble
\let\MetaPrefix\DoubleperCent

\generate{%
  \file{catchfile.ins}{\from{catchfile.dtx}{install}}%
  \file{catchfile.drv}{\from{catchfile.dtx}{driver}}%
  \usedir{tex/generic/oberdiek}%
  \file{catchfile.sty}{\from{catchfile.dtx}{package}}%
%  \usedir{doc/latex/oberdiek/test}%
%  \file{catchfile-test1.tex}{\from{catchfile.dtx}{test1}}%
%  \file{catchfile-test2.tex}{\from{catchfile.dtx}{test2}}%
%  \file{catchfile-test3.tex}{\from{catchfile.dtx}{test3}}%
  \nopreamble
  \nopostamble
%  \usedir{source/latex/oberdiek/catalogue}%
%  \file{catchfile.xml}{\from{catchfile.dtx}{catalogue}}%
}

\catcode32=13\relax% active space
\let =\space%
\Msg{************************************************************************}
\Msg{*}
\Msg{* To finish the installation you have to move the following}
\Msg{* file into a directory searched by TeX:}
\Msg{*}
\Msg{*     catchfile.sty}
\Msg{*}
\Msg{* To produce the documentation run the file `catchfile.drv'}
\Msg{* through LaTeX.}
\Msg{*}
\Msg{* Happy TeXing!}
\Msg{*}
\Msg{************************************************************************}

\endbatchfile
%</install>
%<*ignore>
\fi
%</ignore>
%<*driver>
\NeedsTeXFormat{LaTeX2e}
\ProvidesFile{catchfile.drv}%
  [2016/05/16 v1.7 Catch the contents of a file (HO)]%
\documentclass{ltxdoc}
\usepackage{holtxdoc}[2011/11/22]
\begin{document}
  \DocInput{catchfile.dtx}%
\end{document}
%</driver>
% \fi
%
%
% \CharacterTable
%  {Upper-case    \A\B\C\D\E\F\G\H\I\J\K\L\M\N\O\P\Q\R\S\T\U\V\W\X\Y\Z
%   Lower-case    \a\b\c\d\e\f\g\h\i\j\k\l\m\n\o\p\q\r\s\t\u\v\w\x\y\z
%   Digits        \0\1\2\3\4\5\6\7\8\9
%   Exclamation   \!     Double quote  \"     Hash (number) \#
%   Dollar        \$     Percent       \%     Ampersand     \&
%   Acute accent  \'     Left paren    \(     Right paren   \)
%   Asterisk      \*     Plus          \+     Comma         \,
%   Minus         \-     Point         \.     Solidus       \/
%   Colon         \:     Semicolon     \;     Less than     \<
%   Equals        \=     Greater than  \>     Question mark \?
%   Commercial at \@     Left bracket  \[     Backslash     \\
%   Right bracket \]     Circumflex    \^     Underscore    \_
%   Grave accent  \`     Left brace    \{     Vertical bar  \|
%   Right brace   \}     Tilde         \~}
%
% \GetFileInfo{catchfile.drv}
%
% \title{The \xpackage{catchfile} package}
% \date{2016/05/16 v1.7}
% \author{Heiko Oberdiek\thanks
% {Please report any issues at https://github.com/ho-tex/oberdiek/issues}\\
% \xemail{heiko.oberdiek at googlemail.com}}
%
% \maketitle
%
% \begin{abstract}
% This package catches the contents of a file and puts it in a macro.
% It requires \eTeX. Both \LaTeX\ and \plainTeX\ are supported.
% \end{abstract}
%
% \tableofcontents
%
% \section{Documentation}
%
% The package relies on \eTeX's \cs{everyeof}. Otherwise it
% aborts with an error message.
%
% \begin{declcs}{CatchFileDef} \M{cmd} \M{file name} \M{setup}\\
% \cs{CatchFileEdef} \M{cmd} \M{file name} \M{setup}
% \end{declcs}
% Macro \meta{cmd} is defined with the contents
% of file \meta{file name}.
% \cs{CatchFileDef} uses \cs{def}, \cs{CatchFileEdef} \cs{edef}
% for the definition.
% Additional setup code for setting
% catcodes or treatment of line ends can be given in code
% \meta{setup}. See the test files for an example.
%
% \StopEventually{
% }
%
% \section{Implementation}
%
%    \begin{macrocode}
%<*package>
%    \end{macrocode}
%
% \subsection{Reload check and package identification}
%    Reload check, especially if the package is not used with \LaTeX.
%    \begin{macrocode}
\begingroup\catcode61\catcode48\catcode32=10\relax%
  \catcode13=5 % ^^M
  \endlinechar=13 %
  \catcode35=6 % #
  \catcode39=12 % '
  \catcode44=12 % ,
  \catcode45=12 % -
  \catcode46=12 % .
  \catcode58=12 % :
  \catcode64=11 % @
  \catcode123=1 % {
  \catcode125=2 % }
  \expandafter\let\expandafter\x\csname ver@catchfile.sty\endcsname
  \ifx\x\relax % plain-TeX, first loading
  \else
    \def\empty{}%
    \ifx\x\empty % LaTeX, first loading,
      % variable is initialized, but \ProvidesPackage not yet seen
    \else
      \expandafter\ifx\csname PackageInfo\endcsname\relax
        \def\x#1#2{%
          \immediate\write-1{Package #1 Info: #2.}%
        }%
      \else
        \def\x#1#2{\PackageInfo{#1}{#2, stopped}}%
      \fi
      \x{catchfile}{The package is already loaded}%
      \aftergroup\endinput
    \fi
  \fi
\endgroup%
%    \end{macrocode}
%    Package identification:
%    \begin{macrocode}
\begingroup\catcode61\catcode48\catcode32=10\relax%
  \catcode13=5 % ^^M
  \endlinechar=13 %
  \catcode35=6 % #
  \catcode39=12 % '
  \catcode40=12 % (
  \catcode41=12 % )
  \catcode44=12 % ,
  \catcode45=12 % -
  \catcode46=12 % .
  \catcode47=12 % /
  \catcode58=12 % :
  \catcode64=11 % @
  \catcode91=12 % [
  \catcode93=12 % ]
  \catcode123=1 % {
  \catcode125=2 % }
  \expandafter\ifx\csname ProvidesPackage\endcsname\relax
    \def\x#1#2#3[#4]{\endgroup
      \immediate\write-1{Package: #3 #4}%
      \xdef#1{#4}%
    }%
  \else
    \def\x#1#2[#3]{\endgroup
      #2[{#3}]%
      \ifx#1\@undefined
        \xdef#1{#3}%
      \fi
      \ifx#1\relax
        \xdef#1{#3}%
      \fi
    }%
  \fi
\expandafter\x\csname ver@catchfile.sty\endcsname
\ProvidesPackage{catchfile}%
  [2016/05/16 v1.7 Catch the contents of a file (HO)]%
%    \end{macrocode}
%
% \subsection{Catcodes}
%
%    \begin{macrocode}
\begingroup\catcode61\catcode48\catcode32=10\relax%
  \catcode13=5 % ^^M
  \endlinechar=13 %
  \catcode123=1 % {
  \catcode125=2 % }
  \catcode64=11 % @
  \def\x{\endgroup
    \expandafter\edef\csname CatchFile@AtEnd\endcsname{%
      \endlinechar=\the\endlinechar\relax
      \catcode13=\the\catcode13\relax
      \catcode32=\the\catcode32\relax
      \catcode35=\the\catcode35\relax
      \catcode61=\the\catcode61\relax
      \catcode64=\the\catcode64\relax
      \catcode123=\the\catcode123\relax
      \catcode125=\the\catcode125\relax
    }%
  }%
\x\catcode61\catcode48\catcode32=10\relax%
\catcode13=5 % ^^M
\endlinechar=13 %
\catcode35=6 % #
\catcode64=11 % @
\catcode123=1 % {
\catcode125=2 % }
\def\TMP@EnsureCode#1#2{%
  \edef\CatchFile@AtEnd{%
    \CatchFile@AtEnd
    \catcode#1=\the\catcode#1\relax
  }%
  \catcode#1=#2\relax
}
\TMP@EnsureCode{39}{12}% '
\TMP@EnsureCode{44}{12}% ,
\TMP@EnsureCode{45}{12}% -
\TMP@EnsureCode{46}{12}% .
\TMP@EnsureCode{47}{12}% /
\TMP@EnsureCode{91}{12}% [
\TMP@EnsureCode{93}{12}% ]
\TMP@EnsureCode{96}{12}% `
\edef\CatchFile@AtEnd{\CatchFile@AtEnd\noexpand\endinput}
%    \end{macrocode}
%
% \subsection{Preparations}
%
%    \begin{macrocode}
\begingroup\expandafter\expandafter\expandafter\endgroup
\expandafter\ifx\csname RequirePackage\endcsname\relax
  \input infwarerr.sty\relax
  \input ltxcmds.sty\relax
\else
  \RequirePackage{infwarerr}[2007/09/09]%
  \RequirePackage{ltxcmds}[2010/03/09]%
\fi
%    \end{macrocode}
%
%    Check for \eTeX's \cs{everyeof}.
%    \begin{macrocode}
\begingroup
  \escapechar=92\relax
  \edef\TestString{\string\everyeof}%
  \edef\TestMeaning{\meaning\everyeof}%
  \ifx\TestString\TestMeaning
  \else
    \@PackageError{catchfile}{%
      Cannot find e-TeX's \string\everyeof,\MessageBreak
      package loading is aborted%
    }\@ehd
    \endgroup
    \expandafter\CatchFile@AtEnd
  \fi%
\endgroup
%    \end{macrocode}
%
% \subsection{Looking for primitive \cs{input}}
%
%    \begin{macro}{\CatchFile@Input}
%    The package needs the expandable primitive \cs{input}.
%    However there are formats that redefine it. For example,
%    \LaTeX's \cs{input} is not expandable, but it stores
%    the primitive in \cs{@@input}. The third possibility
%    is \cs{pdfprimitive}, introduced in \pdfTeX\ 1.40.0.
%
%    Thus we try to find the primitive and store it in
%    \cs{CatchFile@Input}. If it is used, it must be expanded
%    twice (because of the solution with \cs{pdfprimitive}).
%    \begin{macrocode}
\begingroup
  \def\Check#1#2#3#4\endgroup{%
    \edef\TestString{\string#1}%
    \edef\TestMeaning{\meaning#2}%
    \ifx\TestString\TestMeaning
      \endgroup
      \let\CatchFile@Primitive#2%
      \def\CatchFile@Input{\CatchFile@Primitive#3}%
    \else
      #4\endgroup
    \fi
  }%
  \Check\input\input{}%
  \Check\input\@@input{}%
  \Check\pdfprimitive\pdfprimitive\input
  \@PackageError{%
    Cannot find primitive \string\input,\MessageBreak
    package loading is aborted%
  }\@ehd
  \csname endgroup\endcsname
  \CatchFile@AtEnd%
\endgroup
%    \end{macrocode}
%    \end{macro}
%
% \subsection{Input file check}
%
%    \begin{macro}{\CatchFile@CheckFileExists}
%    \begin{macrocode}
\begingroup\expandafter\expandafter\expandafter\endgroup
\expandafter\ifx\csname IfFileExists\endcsname\relax
  \input pdftexcmds.sty\relax
  \begingroup\expandafter\expandafter\expandafter\endgroup
  \expandafter\ifx\csname pdf@filesize\endcsname\relax
    \def\CatchFile@CheckFileExists#1{%
      \expandafter\ifx\csname @inputcheck\endcsname\relax
        \csname newread\endcsname\@inputcheck
      \fi
      \openin\@inputcheck#1\relax
      \expandafter\closein\expandafter\@inputcheck
      \ifeof\@inputcheck
        \let\CatchFile@File\relax
      \else
        \def\CatchFile@File{#1}%
      \fi
    }%
  \else
    \def\CatchFile@CheckFileExists#1{%
      \expandafter\expandafter\expandafter\ifx
      \expandafter\expandafter\expandafter\relax\pdf@filesize{#1}\relax
        \let\CatchFile@File\relax
      \else
        \def\CatchFile@File{#1}%
      \fi
    }%
  \fi
\else
  \def\CatchFile@CheckFileExists#1{%
    \IfFileExists{#1}{%
      \expandafter\CatchFile@DefFile\@filef@und\@nil
      \begingroup\expandafter\expandafter\expandafter\endgroup
      \expandafter\ifx\csname @addtofilelist\endcsname\relax
      \else
        \@addtofilelist\CatchFile@File
      \fi
    }{%
      \let\CatchFile@File\relax
    }%
  }%
  \def\CatchFile@DefFile#1 \@nil{%
    \def\CatchFile@File{#1}%
  }%
\fi
%    \end{macrocode}
%    \end{macro}
%
%    \begin{macro}{\CatchFile@NotFound}
%    \begin{macrocode}
\def\CatchFile@NotFound#1#2{%
  \def#1{}%
  \@PackageError{catchfile}{%
    File `#2' not found%
  }\@ehc
}
%    \end{macrocode}
%    \end{macro}
%
% \subsection{Catch file contents}
%
%    \begin{macrocode}
\ltx@IfUndefined{RequirePackage}{%
  \input etexcmds.sty\relax
}{%
  \RequirePackage{etexcmds}[2010/01/28]%
}
%    \end{macrocode}
%
%    \begin{macro}{\CatchFileEdef}
%    \begin{macrocode}
\long\def\CatchFileEdef#1#2#3{%
  \CatchFile@CheckFileExists{#2}%
  \ifx\CatchFile@File\relax
    \CatchFile@NotFound{#1}{#2}%
  \else
    \begingroup
      \everyeof{\noexpand}%
      #3%
      \xdef\CatchFile@Contents{\CatchFile@Input\CatchFile@File\space}%
    \endgroup
    \let#1\CatchFile@Contents
  \fi
}
%    \end{macrocode}
%    \end{macro}
%
%    \begin{macro}{\CatchFileDef}
%    \begin{macrocode}
\long\def\CatchFileDef#1#2#3{%
  \CatchFile@CheckFileExists{#2}%
  \ifx\CatchFile@File\relax
    \CatchFile@NotFound{#1}{#2}%
  \else
    \begingroup
      \everyeof\expandafter{%
        \CatchFile@EOF
        \expandafter\CatchFile@Finish
        \noexpand
      }%
      \ltx@ifundefined{etex@unexpanded}{%
        \expandafter\long\expandafter\def\expandafter\CatchFile@Do
            \expandafter##\expandafter1\CatchFile@EOF{%
          \toks\ltx@zero{##1}%
          \xdef\CatchFile@gtemp{\the\toks\ltx@zero}%
          \def\CatchFile@Finish{%
            \endgroup
            \let#1\CatchFile@gtemp
            \global\let\CatchFile@gtemp\ltx@undefined
          }%
        }%
      }{%
        \expandafter\long\expandafter\def\expandafter\CatchFile@Do
            \expandafter##\expandafter1\CatchFile@EOF{%
          \edef\CatchFile@Finish{%
            \endgroup
            \etex@unexpanded{%
              \edef#1{\etex@unexpanded{##1}}%
            }%
          }%
        }%
      }%
      #3\relax
    \expandafter\expandafter\expandafter\CatchFile@Do
    \CatchFile@Input\CatchFile@File\relax
  \fi
}
%    \end{macrocode}
%    \cs{relax} after |#3| was added to make it more robust in
%    case the user uses something like
%\begin{verbatim}
%\CatchFileDef{\content}{\jobname.tt}{\endlinechar=-1}
%\end{verbatim}
%    that expands the following \cs{expandafter} after |#3| prematurely
%    (contribution of Martin Scharrer).
%    \end{macro}
%    \begin{macro}{\CatchFile@EOF}
%    If the file is read the catcode mappings are fixed.
%    This means that the same character cannot occur
%    inside the file with different catcodes.
%    Thus we use as end of file marker the at sign
%    twice with different catcodes.
%    \begin{macrocode}
\begingroup
  \lccode65=64 % lowercase('A') = '@'
  \lccode66=64 % lowercase('B') = '@'
  \catcode65=8 % catcode('A') = subscript
  \catcode66=3 % catcode('B') = math shift
\lowercase{\endgroup
  \def\CatchFile@EOF{AB}%
}
%    \end{macrocode}
%    \end{macro}
%
%    \begin{macrocode}
\CatchFile@AtEnd%
%</package>
%    \end{macrocode}
%
% \section{Test}
%
% \subsection{Catcode checks for loading}
%
%    \begin{macrocode}
%<*test1>
%    \end{macrocode}
%    \begin{macrocode}
\catcode`\{=1 %
\catcode`\}=2 %
\catcode`\#=6 %
\catcode`\@=11 %
\expandafter\ifx\csname count@\endcsname\relax
  \countdef\count@=255 %
\fi
\expandafter\ifx\csname @gobble\endcsname\relax
  \long\def\@gobble#1{}%
\fi
\expandafter\ifx\csname @firstofone\endcsname\relax
  \long\def\@firstofone#1{#1}%
\fi
\expandafter\ifx\csname loop\endcsname\relax
  \expandafter\@firstofone
\else
  \expandafter\@gobble
\fi
{%
  \def\loop#1\repeat{%
    \def\body{#1}%
    \iterate
  }%
  \def\iterate{%
    \body
      \let\next\iterate
    \else
      \let\next\relax
    \fi
    \next
  }%
  \let\repeat=\fi
}%
\def\RestoreCatcodes{}
\count@=0 %
\loop
  \edef\RestoreCatcodes{%
    \RestoreCatcodes
    \catcode\the\count@=\the\catcode\count@\relax
  }%
\ifnum\count@<255 %
  \advance\count@ 1 %
\repeat

\def\RangeCatcodeInvalid#1#2{%
  \count@=#1\relax
  \loop
    \catcode\count@=15 %
  \ifnum\count@<#2\relax
    \advance\count@ 1 %
  \repeat
}
\def\RangeCatcodeCheck#1#2#3{%
  \count@=#1\relax
  \loop
    \ifnum#3=\catcode\count@
    \else
      \errmessage{%
        Character \the\count@\space
        with wrong catcode \the\catcode\count@\space
        instead of \number#3%
      }%
    \fi
  \ifnum\count@<#2\relax
    \advance\count@ 1 %
  \repeat
}
\def\space{ }
\expandafter\ifx\csname LoadCommand\endcsname\relax
  \def\LoadCommand{\input catchfile.sty\relax}%
\fi
\def\Test{%
  \RangeCatcodeInvalid{0}{47}%
  \RangeCatcodeInvalid{58}{64}%
  \RangeCatcodeInvalid{91}{96}%
  \RangeCatcodeInvalid{123}{255}%
  \catcode`\@=12 %
  \catcode`\\=0 %
  \catcode`\%=14 %
  \LoadCommand
  \RangeCatcodeCheck{0}{36}{15}%
  \RangeCatcodeCheck{37}{37}{14}%
  \RangeCatcodeCheck{38}{47}{15}%
  \RangeCatcodeCheck{48}{57}{12}%
  \RangeCatcodeCheck{58}{63}{15}%
  \RangeCatcodeCheck{64}{64}{12}%
  \RangeCatcodeCheck{65}{90}{11}%
  \RangeCatcodeCheck{91}{91}{15}%
  \RangeCatcodeCheck{92}{92}{0}%
  \RangeCatcodeCheck{93}{96}{15}%
  \RangeCatcodeCheck{97}{122}{11}%
  \RangeCatcodeCheck{123}{255}{15}%
  \RestoreCatcodes
}
\Test
\csname @@end\endcsname
\end
%    \end{macrocode}
%    \begin{macrocode}
%</test1>
%    \end{macrocode}
%
% \subsection{\LaTeX}
%
%    \begin{macrocode}
%<*test2>
\NeedsTeXFormat{LaTeX2e}
\nofiles
\listfiles
\tracingnesting=2 %
\documentclass{minimal}
\usepackage{catchfile}[2016/05/16]
\makeatletter
\def\mysetup{%
  \let\do\@makeother
  \dospecials
}
\def\CheckContents{%
  \begingroup
    \expandafter\ifx\csname contents\endcsname\relax
      \PackageError{TEST}{\string\contents\space is not defined}\@ehc
    \fi
  \endgroup
}
\def\StartDisableUnexpanded{%
  \begingroup
    \let\etex@unexpanded\@undefined
}
\def\StopDisableUnexpanded{%
  \endgroup
}
\def\CheckCleanup{%
  \begingroup
    \edef\x{\the\toks0}%
    \def\y{ABC}%
    \ifx\x\y
    \else
      \PackageError{TEST}{\string\toks0 has changed}\@ehc
    \fi
    \ifx\CatchFile@gtemp\@undefined
    \else
      \PackageError{TEST}{Left over \string\CatchFile@gtemp}\@ehc
    \fi
  \endgroup
}
\def\Check{%
  \CheckContents
  \CheckCleanup
}
\makeatother
\begin{document}
\toks0{ABC}

\CatchFileDef\contents{catchfile.sty}\mysetup
\typeout{\meaning\contents}
\Check
\typeout{***********************************************************}%
\CatchFileDef\contents{catchfile.sty}{}
\typeout{\meaning\contents}
\Check
\typeout{***********************************************************}%

\StartDisableUnexpanded
\CatchFileDef\contents{catchfile.sty}\mysetup
\typeout{\meaning\contents}
\Check
\typeout{***********************************************************}%
\CatchFileDef\contents{catchfile.sty}{}
\typeout{\meaning\contents}
\Check
\typeout{***********************************************************}%
\StopDisableUnexpanded

\CatchFileEdef\contents{catchfile.sty}{%
  \mysetup
  \def\par{^^J}%
  \obeylines
}
\typeout{\contents}
\Check
\typeout{***********************************************************}%
\CatchFileEdef\contents{catchfile.sty}{%
  \catcode`\\=12 %
  \catcode`\#=12 %
  \def\par{^^J}%
  \obeylines
}
\typeout{\contents}
\Check
\typeout{***********************************************************}%
\end{document}
%</test2>
%    \end{macrocode}
%
% \subsection{\plainTeX}
%
%    \begin{macrocode}
%<*test3>
\def\msg#{\immediate\write16}
\newlinechar=10 %
\tracingnesting=2 %
\input catchfile.sty\relax

\def\mysetup{%
  \def\do##1{%
    \catcode`##1=12\relax
  }%
  \dospecials
}
\def\CheckContents{%
  \begingroup
    \expandafter\ifx\csname contents\endcsname\relax
      \def\space{ }%
      \errmessage{\string\contents\space is not defined.}%
    \fi
  \endgroup
}
\catcode`\@=11 %
\def\CheckCleanup{%
  \begingroup
    \edef\x{\the\toks0}%
    \def\y{ABC}%
    \ifx\x\y
    \else
      \errmessage{\string\toks0 has changed}%
    \fi
    \ifx\CatchFile@gtemp\@undefined
    \else
      \errmessage{Left over \string\CatchFile@gtemp}%
    \fi
  \endgroup
}
\def\Check{%
  \CheckContents
  \CheckCleanup
}
\def\StartDisableUnexpanded{%
  \begingroup
    \let\etex@unexpanded\@undefined
}
\def\StopDisableUnexpanded{%
  \endgroup
}
\catcode`\@=12 %

\toks0{ABC}

\CatchFileDef\contents{catchfile.sty}\mysetup
\msg{\meaning\contents}
\Check
\msg{***********************************************************}%
\CatchFileDef\contents{catchfile.sty}{}
\msg{\meaning\contents}
\Check
\msg{***********************************************************}%

\StartDisableUnexpanded
\CatchFileDef\contents{catchfile.sty}\mysetup
\msg{\meaning\contents}
\Check
\msg{***********************************************************}%
\CatchFileDef\contents{catchfile.sty}{}
\msg{\meaning\contents}
\Check
\msg{***********************************************************}%
\StopDisableUnexpanded

\CatchFileEdef\contents{catchfile.sty}{%
  \mysetup
  \def\par{^^J}%
  \obeylines
}
\msg{\contents}
\Check
\msg{***********************************************************}%
\CatchFileEdef\contents{catchfile.sty}{%
  \catcode`\\=12 %
  \catcode`\#=12 %
  \def\par{^^J}%
  \obeylines
}
\msg{\contents}
\Check
\msg{***********************************************************}%

\csname @@end\endcsname
\end
%</test3>
%    \end{macrocode}
%
% \section{Installation}
%
% \subsection{Download}
%
% \paragraph{Package.} This package is available on
% CTAN\footnote{\url{https://ctan.org/pkg/catchfile}}:
% \begin{description}
% \item[\CTAN{macros/latex/contrib/oberdiek/catchfile.dtx}] The source file.
% \item[\CTAN{macros/latex/contrib/oberdiek/catchfile.pdf}] Documentation.
% \end{description}
%
%
% \paragraph{Bundle.} All the packages of the bundle `oberdiek'
% are also available in a TDS compliant ZIP archive. There
% the packages are already unpacked and the documentation files
% are generated. The files and directories obey the TDS standard.
% \begin{description}
% \item[\CTANinstall{install/macros/latex/contrib/oberdiek.tds.zip}]
% \end{description}
% \emph{TDS} refers to the standard ``A Directory Structure
% for \TeX\ Files'' (\CTAN{tds/tds.pdf}). Directories
% with \xfile{texmf} in their name are usually organized this way.
%
% \subsection{Bundle installation}
%
% \paragraph{Unpacking.} Unpack the \xfile{oberdiek.tds.zip} in the
% TDS tree (also known as \xfile{texmf} tree) of your choice.
% Example (linux):
% \begin{quote}
%   |unzip oberdiek.tds.zip -d ~/texmf|
% \end{quote}
%
% \paragraph{Script installation.}
% Check the directory \xfile{TDS:scripts/oberdiek/} for
% scripts that need further installation steps.
% Package \xpackage{attachfile2} comes with the Perl script
% \xfile{pdfatfi.pl} that should be installed in such a way
% that it can be called as \texttt{pdfatfi}.
% Example (linux):
% \begin{quote}
%   |chmod +x scripts/oberdiek/pdfatfi.pl|\\
%   |cp scripts/oberdiek/pdfatfi.pl /usr/local/bin/|
% \end{quote}
%
% \subsection{Package installation}
%
% \paragraph{Unpacking.} The \xfile{.dtx} file is a self-extracting
% \docstrip\ archive. The files are extracted by running the
% \xfile{.dtx} through \plainTeX:
% \begin{quote}
%   \verb|tex catchfile.dtx|
% \end{quote}
%
% \paragraph{TDS.} Now the different files must be moved into
% the different directories in your installation TDS tree
% (also known as \xfile{texmf} tree):
% \begin{quote}
% \def\t{^^A
% \begin{tabular}{@{}>{\ttfamily}l@{ $\rightarrow$ }>{\ttfamily}l@{}}
%   catchfile.sty & tex/generic/oberdiek/catchfile.sty\\
%   catchfile.pdf & doc/latex/oberdiek/catchfile.pdf\\
%   test/catchfile-test1.tex & doc/latex/oberdiek/test/catchfile-test1.tex\\
%   test/catchfile-test2.tex & doc/latex/oberdiek/test/catchfile-test2.tex\\
%   test/catchfile-test3.tex & doc/latex/oberdiek/test/catchfile-test3.tex\\
%   catchfile.dtx & source/latex/oberdiek/catchfile.dtx\\
% \end{tabular}^^A
% }^^A
% \sbox0{\t}^^A
% \ifdim\wd0>\linewidth
%   \begingroup
%     \advance\linewidth by\leftmargin
%     \advance\linewidth by\rightmargin
%   \edef\x{\endgroup
%     \def\noexpand\lw{\the\linewidth}^^A
%   }\x
%   \def\lwbox{^^A
%     \leavevmode
%     \hbox to \linewidth{^^A
%       \kern-\leftmargin\relax
%       \hss
%       \usebox0
%       \hss
%       \kern-\rightmargin\relax
%     }^^A
%   }^^A
%   \ifdim\wd0>\lw
%     \sbox0{\small\t}^^A
%     \ifdim\wd0>\linewidth
%       \ifdim\wd0>\lw
%         \sbox0{\footnotesize\t}^^A
%         \ifdim\wd0>\linewidth
%           \ifdim\wd0>\lw
%             \sbox0{\scriptsize\t}^^A
%             \ifdim\wd0>\linewidth
%               \ifdim\wd0>\lw
%                 \sbox0{\tiny\t}^^A
%                 \ifdim\wd0>\linewidth
%                   \lwbox
%                 \else
%                   \usebox0
%                 \fi
%               \else
%                 \lwbox
%               \fi
%             \else
%               \usebox0
%             \fi
%           \else
%             \lwbox
%           \fi
%         \else
%           \usebox0
%         \fi
%       \else
%         \lwbox
%       \fi
%     \else
%       \usebox0
%     \fi
%   \else
%     \lwbox
%   \fi
% \else
%   \usebox0
% \fi
% \end{quote}
% If you have a \xfile{docstrip.cfg} that configures and enables \docstrip's
% TDS installing feature, then some files can already be in the right
% place, see the documentation of \docstrip.
%
% \subsection{Refresh file name databases}
%
% If your \TeX~distribution
% (\teTeX, \mikTeX, \dots) relies on file name databases, you must refresh
% these. For example, \teTeX\ users run \verb|texhash| or
% \verb|mktexlsr|.
%
% \subsection{Some details for the interested}
%
% \paragraph{Attached source.}
%
% The PDF documentation on CTAN also includes the
% \xfile{.dtx} source file. It can be extracted by
% AcrobatReader 6 or higher. Another option is \textsf{pdftk},
% e.g. unpack the file into the current directory:
% \begin{quote}
%   \verb|pdftk catchfile.pdf unpack_files output .|
% \end{quote}
%
% \paragraph{Unpacking with \LaTeX.}
% The \xfile{.dtx} chooses its action depending on the format:
% \begin{description}
% \item[\plainTeX:] Run \docstrip\ and extract the files.
% \item[\LaTeX:] Generate the documentation.
% \end{description}
% If you insist on using \LaTeX\ for \docstrip\ (really,
% \docstrip\ does not need \LaTeX), then inform the autodetect routine
% about your intention:
% \begin{quote}
%   \verb|latex \let\install=y\input{catchfile.dtx}|
% \end{quote}
% Do not forget to quote the argument according to the demands
% of your shell.
%
% \paragraph{Generating the documentation.}
% You can use both the \xfile{.dtx} or the \xfile{.drv} to generate
% the documentation. The process can be configured by the
% configuration file \xfile{ltxdoc.cfg}. For instance, put this
% line into this file, if you want to have A4 as paper format:
% \begin{quote}
%   \verb|\PassOptionsToClass{a4paper}{article}|
% \end{quote}
% An example follows how to generate the
% documentation with pdf\LaTeX:
% \begin{quote}
%\begin{verbatim}
%pdflatex catchfile.dtx
%makeindex -s gind.ist catchfile.idx
%pdflatex catchfile.dtx
%makeindex -s gind.ist catchfile.idx
%pdflatex catchfile.dtx
%\end{verbatim}
% \end{quote}
%
% \begin{History}
%   \begin{Version}{2007/05/30 v1.0}
%   \item
%     First version.
%   \end{Version}
%   \begin{Version}{2007/09/09 v1.1}
%   \item
%     Catcode section rewritten.
%   \end{Version}
%   \begin{Version}{2007/11/11 v1.2}
%   \item
%     Use of package \xpackage{pdftexcmds} for \LuaTeX\ support.
%   \end{Version}
%   \begin{Version}{2010/03/01 v1.3}
%   \item
%     Fix for unknown \cs{@PackageErrorNoLine}.
%   \end{Version}
%   \begin{Version}{2010/04/08 v1.4}
%   \item
%     \cs{closein} also added if \cs{ifeof} is true.
%   \end{Version}
%   \begin{Version}{2010/04/28 v1.5}
%   \item
%     \cs{CatchFileDef}: Getting rid of warning `end of semi simple group
%     entered at line \dots\ of
%     a different file (Florent Chervet).
%   \item
%     \cs{CatchFileDef}: Fix for error `Illegal parameter number in
%     definition of \dots' (HO) including improvement that uses
%     \cs{unexpanded} if available (Florent Chervet).
%   \end{Version}
%   \begin{Version}{2011/03/01 v1.6}
%   \item
%     \cs{relax} added after the setup argument of
%     \cs{CatchFileDef} to prevent premature file reading
%     (Martin Scharrer).
%   \end{Version}
%   \begin{Version}{2016/05/16 v1.7}
%   \item
%     Documentation updates.
%   \end{Version}
% \end{History}
%
% \PrintIndex
%
% \Finale
\endinput
|
% \end{quote}
% Do not forget to quote the argument according to the demands
% of your shell.
%
% \paragraph{Generating the documentation.}
% You can use both the \xfile{.dtx} or the \xfile{.drv} to generate
% the documentation. The process can be configured by the
% configuration file \xfile{ltxdoc.cfg}. For instance, put this
% line into this file, if you want to have A4 as paper format:
% \begin{quote}
%   \verb|\PassOptionsToClass{a4paper}{article}|
% \end{quote}
% An example follows how to generate the
% documentation with pdf\LaTeX:
% \begin{quote}
%\begin{verbatim}
%pdflatex catchfile.dtx
%makeindex -s gind.ist catchfile.idx
%pdflatex catchfile.dtx
%makeindex -s gind.ist catchfile.idx
%pdflatex catchfile.dtx
%\end{verbatim}
% \end{quote}
%
% \begin{History}
%   \begin{Version}{2007/05/30 v1.0}
%   \item
%     First version.
%   \end{Version}
%   \begin{Version}{2007/09/09 v1.1}
%   \item
%     Catcode section rewritten.
%   \end{Version}
%   \begin{Version}{2007/11/11 v1.2}
%   \item
%     Use of package \xpackage{pdftexcmds} for \LuaTeX\ support.
%   \end{Version}
%   \begin{Version}{2010/03/01 v1.3}
%   \item
%     Fix for unknown \cs{@PackageErrorNoLine}.
%   \end{Version}
%   \begin{Version}{2010/04/08 v1.4}
%   \item
%     \cs{closein} also added if \cs{ifeof} is true.
%   \end{Version}
%   \begin{Version}{2010/04/28 v1.5}
%   \item
%     \cs{CatchFileDef}: Getting rid of warning `end of semi simple group
%     entered at line \dots\ of
%     a different file (Florent Chervet).
%   \item
%     \cs{CatchFileDef}: Fix for error `Illegal parameter number in
%     definition of \dots' (HO) including improvement that uses
%     \cs{unexpanded} if available (Florent Chervet).
%   \end{Version}
%   \begin{Version}{2011/03/01 v1.6}
%   \item
%     \cs{relax} added after the setup argument of
%     \cs{CatchFileDef} to prevent premature file reading
%     (Martin Scharrer).
%   \end{Version}
%   \begin{Version}{2016/05/16 v1.7}
%   \item
%     Documentation updates.
%   \end{Version}
% \end{History}
%
% \PrintIndex
%
% \Finale
\endinput
|
% \end{quote}
% Do not forget to quote the argument according to the demands
% of your shell.
%
% \paragraph{Generating the documentation.}
% You can use both the \xfile{.dtx} or the \xfile{.drv} to generate
% the documentation. The process can be configured by the
% configuration file \xfile{ltxdoc.cfg}. For instance, put this
% line into this file, if you want to have A4 as paper format:
% \begin{quote}
%   \verb|\PassOptionsToClass{a4paper}{article}|
% \end{quote}
% An example follows how to generate the
% documentation with pdf\LaTeX:
% \begin{quote}
%\begin{verbatim}
%pdflatex catchfile.dtx
%makeindex -s gind.ist catchfile.idx
%pdflatex catchfile.dtx
%makeindex -s gind.ist catchfile.idx
%pdflatex catchfile.dtx
%\end{verbatim}
% \end{quote}
%
% \begin{History}
%   \begin{Version}{2007/05/30 v1.0}
%   \item
%     First version.
%   \end{Version}
%   \begin{Version}{2007/09/09 v1.1}
%   \item
%     Catcode section rewritten.
%   \end{Version}
%   \begin{Version}{2007/11/11 v1.2}
%   \item
%     Use of package \xpackage{pdftexcmds} for \LuaTeX\ support.
%   \end{Version}
%   \begin{Version}{2010/03/01 v1.3}
%   \item
%     Fix for unknown \cs{@PackageErrorNoLine}.
%   \end{Version}
%   \begin{Version}{2010/04/08 v1.4}
%   \item
%     \cs{closein} also added if \cs{ifeof} is true.
%   \end{Version}
%   \begin{Version}{2010/04/28 v1.5}
%   \item
%     \cs{CatchFileDef}: Getting rid of warning `end of semi simple group
%     entered at line \dots\ of
%     a different file (Florent Chervet).
%   \item
%     \cs{CatchFileDef}: Fix for error `Illegal parameter number in
%     definition of \dots' (HO) including improvement that uses
%     \cs{unexpanded} if available (Florent Chervet).
%   \end{Version}
%   \begin{Version}{2011/03/01 v1.6}
%   \item
%     \cs{relax} added after the setup argument of
%     \cs{CatchFileDef} to prevent premature file reading
%     (Martin Scharrer).
%   \end{Version}
%   \begin{Version}{2016/05/16 v1.7}
%   \item
%     Documentation updates.
%   \end{Version}
% \end{History}
%
% \PrintIndex
%
% \Finale
\endinput

%        (quote the arguments according to the demands of your shell)
%
% Documentation:
%    (a) If catchfile.drv is present:
%           latex catchfile.drv
%    (b) Without catchfile.drv:
%           latex catchfile.dtx; ...
%    The class ltxdoc loads the configuration file ltxdoc.cfg
%    if available. Here you can specify further options, e.g.
%    use A4 as paper format:
%       \PassOptionsToClass{a4paper}{article}
%
%    Programm calls to get the documentation (example):
%       pdflatex catchfile.dtx
%       makeindex -s gind.ist catchfile.idx
%       pdflatex catchfile.dtx
%       makeindex -s gind.ist catchfile.idx
%       pdflatex catchfile.dtx
%
% Installation:
%    TDS:tex/generic/oberdiek/catchfile.sty
%    TDS:doc/latex/oberdiek/catchfile.pdf
%    TDS:doc/latex/oberdiek/test/catchfile-test1.tex
%    TDS:doc/latex/oberdiek/test/catchfile-test2.tex
%    TDS:doc/latex/oberdiek/test/catchfile-test3.tex
%    TDS:source/latex/oberdiek/catchfile.dtx
%
%<*ignore>
\begingroup
  \catcode123=1 %
  \catcode125=2 %
  \def\x{LaTeX2e}%
\expandafter\endgroup
\ifcase 0\ifx\install y1\fi\expandafter
         \ifx\csname processbatchFile\endcsname\relax\else1\fi
         \ifx\fmtname\x\else 1\fi\relax
\else\csname fi\endcsname
%</ignore>
%<*install>
\input docstrip.tex
\Msg{************************************************************************}
\Msg{* Installation}
\Msg{* Package: catchfile 2016/05/16 v1.7 Catch the contents of a file (HO)}
\Msg{************************************************************************}

\keepsilent
\askforoverwritefalse

\let\MetaPrefix\relax
\preamble

This is a generated file.

Project: catchfile
Version: 2016/05/16 v1.7

Copyright (C) 2007, 2010, 2011 by
   Heiko Oberdiek <heiko.oberdiek at googlemail.com>

This work may be distributed and/or modified under the
conditions of the LaTeX Project Public License, either
version 1.3c of this license or (at your option) any later
version. This version of this license is in
   https://www.latex-project.org/lppl/lppl-1-3c.txt
and the latest version of this license is in
   https://www.latex-project.org/lppl.txt
and version 1.3 or later is part of all distributions of
LaTeX version 2005/12/01 or later.

This work has the LPPL maintenance status "maintained".

The Current Maintainers of this work are
Heiko Oberdiek and the Oberdiek Package Support Group
https://github.com/ho-tex/oberdiek/issues


The Base Interpreter refers to any `TeX-Format',
because some files are installed in TDS:tex/generic//.

This work consists of the main source file catchfile.dtx
and the derived files
   catchfile.sty, catchfile.pdf, catchfile.ins, catchfile.drv,
   catchfile-test1.tex, catchfile-test2.tex,
   catchfile-test3.tex.

\endpreamble
\let\MetaPrefix\DoubleperCent

\generate{%
  \file{catchfile.ins}{\from{catchfile.dtx}{install}}%
  \file{catchfile.drv}{\from{catchfile.dtx}{driver}}%
  \usedir{tex/generic/oberdiek}%
  \file{catchfile.sty}{\from{catchfile.dtx}{package}}%
%  \usedir{doc/latex/oberdiek/test}%
%  \file{catchfile-test1.tex}{\from{catchfile.dtx}{test1}}%
%  \file{catchfile-test2.tex}{\from{catchfile.dtx}{test2}}%
%  \file{catchfile-test3.tex}{\from{catchfile.dtx}{test3}}%
  \nopreamble
  \nopostamble
%  \usedir{source/latex/oberdiek/catalogue}%
%  \file{catchfile.xml}{\from{catchfile.dtx}{catalogue}}%
}

\catcode32=13\relax% active space
\let =\space%
\Msg{************************************************************************}
\Msg{*}
\Msg{* To finish the installation you have to move the following}
\Msg{* file into a directory searched by TeX:}
\Msg{*}
\Msg{*     catchfile.sty}
\Msg{*}
\Msg{* To produce the documentation run the file `catchfile.drv'}
\Msg{* through LaTeX.}
\Msg{*}
\Msg{* Happy TeXing!}
\Msg{*}
\Msg{************************************************************************}

\endbatchfile
%</install>
%<*ignore>
\fi
%</ignore>
%<*driver>
\NeedsTeXFormat{LaTeX2e}
\ProvidesFile{catchfile.drv}%
  [2016/05/16 v1.7 Catch the contents of a file (HO)]%
\documentclass{ltxdoc}
\usepackage{holtxdoc}[2011/11/22]
\begin{document}
  \DocInput{catchfile.dtx}%
\end{document}
%</driver>
% \fi
%
%
% \CharacterTable
%  {Upper-case    \A\B\C\D\E\F\G\H\I\J\K\L\M\N\O\P\Q\R\S\T\U\V\W\X\Y\Z
%   Lower-case    \a\b\c\d\e\f\g\h\i\j\k\l\m\n\o\p\q\r\s\t\u\v\w\x\y\z
%   Digits        \0\1\2\3\4\5\6\7\8\9
%   Exclamation   \!     Double quote  \"     Hash (number) \#
%   Dollar        \$     Percent       \%     Ampersand     \&
%   Acute accent  \'     Left paren    \(     Right paren   \)
%   Asterisk      \*     Plus          \+     Comma         \,
%   Minus         \-     Point         \.     Solidus       \/
%   Colon         \:     Semicolon     \;     Less than     \<
%   Equals        \=     Greater than  \>     Question mark \?
%   Commercial at \@     Left bracket  \[     Backslash     \\
%   Right bracket \]     Circumflex    \^     Underscore    \_
%   Grave accent  \`     Left brace    \{     Vertical bar  \|
%   Right brace   \}     Tilde         \~}
%
% \GetFileInfo{catchfile.drv}
%
% \title{The \xpackage{catchfile} package}
% \date{2016/05/16 v1.7}
% \author{Heiko Oberdiek\thanks
% {Please report any issues at \url{https://github.com/ho-tex/oberdiek/issues}}}
%
% \maketitle
%
% \begin{abstract}
% This package catches the contents of a file and puts it in a macro.
% It requires \eTeX. Both \LaTeX\ and \plainTeX\ are supported.
% \end{abstract}
%
% \tableofcontents
%
% \section{Documentation}
%
% The package relies on \eTeX's \cs{everyeof}. Otherwise it
% aborts with an error message.
%
% \begin{declcs}{CatchFileDef} \M{cmd} \M{file name} \M{setup}\\
% \cs{CatchFileEdef} \M{cmd} \M{file name} \M{setup}
% \end{declcs}
% Macro \meta{cmd} is defined with the contents
% of file \meta{file name}.
% \cs{CatchFileDef} uses \cs{def}, \cs{CatchFileEdef} \cs{edef}
% for the definition.
% Additional setup code for setting
% catcodes or treatment of line ends can be given in code
% \meta{setup}. See the test files for an example.
%
% \StopEventually{
% }
%
% \section{Implementation}
%
%    \begin{macrocode}
%<*package>
%    \end{macrocode}
%
% \subsection{Reload check and package identification}
%    Reload check, especially if the package is not used with \LaTeX.
%    \begin{macrocode}
\begingroup\catcode61\catcode48\catcode32=10\relax%
  \catcode13=5 % ^^M
  \endlinechar=13 %
  \catcode35=6 % #
  \catcode39=12 % '
  \catcode44=12 % ,
  \catcode45=12 % -
  \catcode46=12 % .
  \catcode58=12 % :
  \catcode64=11 % @
  \catcode123=1 % {
  \catcode125=2 % }
  \expandafter\let\expandafter\x\csname ver@catchfile.sty\endcsname
  \ifx\x\relax % plain-TeX, first loading
  \else
    \def\empty{}%
    \ifx\x\empty % LaTeX, first loading,
      % variable is initialized, but \ProvidesPackage not yet seen
    \else
      \expandafter\ifx\csname PackageInfo\endcsname\relax
        \def\x#1#2{%
          \immediate\write-1{Package #1 Info: #2.}%
        }%
      \else
        \def\x#1#2{\PackageInfo{#1}{#2, stopped}}%
      \fi
      \x{catchfile}{The package is already loaded}%
      \aftergroup\endinput
    \fi
  \fi
\endgroup%
%    \end{macrocode}
%    Package identification:
%    \begin{macrocode}
\begingroup\catcode61\catcode48\catcode32=10\relax%
  \catcode13=5 % ^^M
  \endlinechar=13 %
  \catcode35=6 % #
  \catcode39=12 % '
  \catcode40=12 % (
  \catcode41=12 % )
  \catcode44=12 % ,
  \catcode45=12 % -
  \catcode46=12 % .
  \catcode47=12 % /
  \catcode58=12 % :
  \catcode64=11 % @
  \catcode91=12 % [
  \catcode93=12 % ]
  \catcode123=1 % {
  \catcode125=2 % }
  \expandafter\ifx\csname ProvidesPackage\endcsname\relax
    \def\x#1#2#3[#4]{\endgroup
      \immediate\write-1{Package: #3 #4}%
      \xdef#1{#4}%
    }%
  \else
    \def\x#1#2[#3]{\endgroup
      #2[{#3}]%
      \ifx#1\@undefined
        \xdef#1{#3}%
      \fi
      \ifx#1\relax
        \xdef#1{#3}%
      \fi
    }%
  \fi
\expandafter\x\csname ver@catchfile.sty\endcsname
\ProvidesPackage{catchfile}%
  [2016/05/16 v1.7 Catch the contents of a file (HO)]%
%    \end{macrocode}
%
% \subsection{Catcodes}
%
%    \begin{macrocode}
\begingroup\catcode61\catcode48\catcode32=10\relax%
  \catcode13=5 % ^^M
  \endlinechar=13 %
  \catcode123=1 % {
  \catcode125=2 % }
  \catcode64=11 % @
  \def\x{\endgroup
    \expandafter\edef\csname CatchFile@AtEnd\endcsname{%
      \endlinechar=\the\endlinechar\relax
      \catcode13=\the\catcode13\relax
      \catcode32=\the\catcode32\relax
      \catcode35=\the\catcode35\relax
      \catcode61=\the\catcode61\relax
      \catcode64=\the\catcode64\relax
      \catcode123=\the\catcode123\relax
      \catcode125=\the\catcode125\relax
    }%
  }%
\x\catcode61\catcode48\catcode32=10\relax%
\catcode13=5 % ^^M
\endlinechar=13 %
\catcode35=6 % #
\catcode64=11 % @
\catcode123=1 % {
\catcode125=2 % }
\def\TMP@EnsureCode#1#2{%
  \edef\CatchFile@AtEnd{%
    \CatchFile@AtEnd
    \catcode#1=\the\catcode#1\relax
  }%
  \catcode#1=#2\relax
}
\TMP@EnsureCode{39}{12}% '
\TMP@EnsureCode{44}{12}% ,
\TMP@EnsureCode{45}{12}% -
\TMP@EnsureCode{46}{12}% .
\TMP@EnsureCode{47}{12}% /
\TMP@EnsureCode{91}{12}% [
\TMP@EnsureCode{93}{12}% ]
\TMP@EnsureCode{96}{12}% `
\edef\CatchFile@AtEnd{\CatchFile@AtEnd\noexpand\endinput}
%    \end{macrocode}
%
% \subsection{Preparations}
%
%    \begin{macrocode}
\begingroup\expandafter\expandafter\expandafter\endgroup
\expandafter\ifx\csname RequirePackage\endcsname\relax
  \input infwarerr.sty\relax
  \input ltxcmds.sty\relax
\else
  \RequirePackage{infwarerr}[2007/09/09]%
  \RequirePackage{ltxcmds}[2010/03/09]%
\fi
%    \end{macrocode}
%
%    Check for \eTeX's \cs{everyeof}.
%    \begin{macrocode}
\begingroup
  \escapechar=92\relax
  \edef\TestString{\string\everyeof}%
  \edef\TestMeaning{\meaning\everyeof}%
  \ifx\TestString\TestMeaning
  \else
    \@PackageError{catchfile}{%
      Cannot find e-TeX's \string\everyeof,\MessageBreak
      package loading is aborted%
    }\@ehd
    \endgroup
    \expandafter\CatchFile@AtEnd
  \fi%
\endgroup
%    \end{macrocode}
%
% \subsection{Looking for primitive \cs{input}}
%
%    \begin{macro}{\CatchFile@Input}
%    The package needs the expandable primitive \cs{input}.
%    However there are formats that redefine it. For example,
%    \LaTeX's \cs{input} is not expandable, but it stores
%    the primitive in \cs{@@input}. The third possibility
%    is \cs{pdfprimitive}, introduced in \pdfTeX\ 1.40.0.
%
%    Thus we try to find the primitive and store it in
%    \cs{CatchFile@Input}. If it is used, it must be expanded
%    twice (because of the solution with \cs{pdfprimitive}).
%    \begin{macrocode}
\begingroup
  \def\Check#1#2#3#4\endgroup{%
    \edef\TestString{\string#1}%
    \edef\TestMeaning{\meaning#2}%
    \ifx\TestString\TestMeaning
      \endgroup
      \let\CatchFile@Primitive#2%
      \def\CatchFile@Input{\CatchFile@Primitive#3}%
    \else
      #4\endgroup
    \fi
  }%
  \Check\input\input{}%
  \Check\input\@@input{}%
  \Check\pdfprimitive\pdfprimitive\input
  \@PackageError{%
    Cannot find primitive \string\input,\MessageBreak
    package loading is aborted%
  }\@ehd
  \csname endgroup\endcsname
  \CatchFile@AtEnd%
\endgroup
%    \end{macrocode}
%    \end{macro}
%
% \subsection{Input file check}
%
%    \begin{macro}{\CatchFile@CheckFileExists}
%    \begin{macrocode}
\begingroup\expandafter\expandafter\expandafter\endgroup
\expandafter\ifx\csname IfFileExists\endcsname\relax
  \input pdftexcmds.sty\relax
  \begingroup\expandafter\expandafter\expandafter\endgroup
  \expandafter\ifx\csname pdf@filesize\endcsname\relax
    \def\CatchFile@CheckFileExists#1{%
      \expandafter\ifx\csname @inputcheck\endcsname\relax
        \csname newread\endcsname\@inputcheck
      \fi
      \openin\@inputcheck#1\relax
      \expandafter\closein\expandafter\@inputcheck
      \ifeof\@inputcheck
        \let\CatchFile@File\relax
      \else
        \def\CatchFile@File{#1}%
      \fi
    }%
  \else
    \def\CatchFile@CheckFileExists#1{%
      \expandafter\expandafter\expandafter\ifx
      \expandafter\expandafter\expandafter\relax\pdf@filesize{#1}\relax
        \let\CatchFile@File\relax
      \else
        \def\CatchFile@File{#1}%
      \fi
    }%
  \fi
\else
  \def\CatchFile@CheckFileExists#1{%
    \IfFileExists{#1}{%
      \expandafter\CatchFile@DefFile\@filef@und\@nil
      \begingroup\expandafter\expandafter\expandafter\endgroup
      \expandafter\ifx\csname @addtofilelist\endcsname\relax
      \else
        \@addtofilelist\CatchFile@File
      \fi
    }{%
      \let\CatchFile@File\relax
    }%
  }%
  \def\CatchFile@DefFile#1 \@nil{%
    \def\CatchFile@File{#1}%
  }%
\fi
%    \end{macrocode}
%    \end{macro}
%
%    \begin{macro}{\CatchFile@NotFound}
%    \begin{macrocode}
\def\CatchFile@NotFound#1#2{%
  \def#1{}%
  \@PackageError{catchfile}{%
    File `#2' not found%
  }\@ehc
}
%    \end{macrocode}
%    \end{macro}
%
% \subsection{Catch file contents}
%
%    \begin{macrocode}
\ltx@IfUndefined{RequirePackage}{%
  \input etexcmds.sty\relax
}{%
  \RequirePackage{etexcmds}[2010/01/28]%
}
%    \end{macrocode}
%
%    \begin{macro}{\CatchFileEdef}
%    \begin{macrocode}
\long\def\CatchFileEdef#1#2#3{%
  \CatchFile@CheckFileExists{#2}%
  \ifx\CatchFile@File\relax
    \CatchFile@NotFound{#1}{#2}%
  \else
    \begingroup
      \everyeof{\noexpand}%
      #3%
      \xdef\CatchFile@Contents{\CatchFile@Input\CatchFile@File\space}%
    \endgroup
    \let#1\CatchFile@Contents
  \fi
}
%    \end{macrocode}
%    \end{macro}
%
%    \begin{macro}{\CatchFileDef}
%    \begin{macrocode}
\long\def\CatchFileDef#1#2#3{%
  \CatchFile@CheckFileExists{#2}%
  \ifx\CatchFile@File\relax
    \CatchFile@NotFound{#1}{#2}%
  \else
    \begingroup
      \everyeof\expandafter{%
        \CatchFile@EOF
        \expandafter\CatchFile@Finish
        \noexpand
      }%
      \ltx@ifundefined{etex@unexpanded}{%
        \expandafter\long\expandafter\def\expandafter\CatchFile@Do
            \expandafter##\expandafter1\CatchFile@EOF{%
          \toks\ltx@zero{##1}%
          \xdef\CatchFile@gtemp{\the\toks\ltx@zero}%
          \def\CatchFile@Finish{%
            \endgroup
            \let#1\CatchFile@gtemp
            \global\let\CatchFile@gtemp\ltx@undefined
          }%
        }%
      }{%
        \expandafter\long\expandafter\def\expandafter\CatchFile@Do
            \expandafter##\expandafter1\CatchFile@EOF{%
          \edef\CatchFile@Finish{%
            \endgroup
            \etex@unexpanded{%
              \edef#1{\etex@unexpanded{##1}}%
            }%
          }%
        }%
      }%
      #3\relax
    \expandafter\expandafter\expandafter\CatchFile@Do
    \CatchFile@Input\CatchFile@File\relax
  \fi
}
%    \end{macrocode}
%    \cs{relax} after |#3| was added to make it more robust in
%    case the user uses something like
%\begin{verbatim}
%\CatchFileDef{\content}{\jobname.tt}{\endlinechar=-1}
%\end{verbatim}
%    that expands the following \cs{expandafter} after |#3| prematurely
%    (contribution of Martin Scharrer).
%    \end{macro}
%    \begin{macro}{\CatchFile@EOF}
%    If the file is read the catcode mappings are fixed.
%    This means that the same character cannot occur
%    inside the file with different catcodes.
%    Thus we use as end of file marker the at sign
%    twice with different catcodes.
%    \begin{macrocode}
\begingroup
  \lccode65=64 % lowercase('A') = '@'
  \lccode66=64 % lowercase('B') = '@'
  \catcode65=8 % catcode('A') = subscript
  \catcode66=3 % catcode('B') = math shift
\lowercase{\endgroup
  \def\CatchFile@EOF{AB}%
}
%    \end{macrocode}
%    \end{macro}
%
%    \begin{macrocode}
\CatchFile@AtEnd%
%</package>
%    \end{macrocode}
%
% \section{Test}
%
% \subsection{Catcode checks for loading}
%
%    \begin{macrocode}
%<*test1>
%    \end{macrocode}
%    \begin{macrocode}
\catcode`\{=1 %
\catcode`\}=2 %
\catcode`\#=6 %
\catcode`\@=11 %
\expandafter\ifx\csname count@\endcsname\relax
  \countdef\count@=255 %
\fi
\expandafter\ifx\csname @gobble\endcsname\relax
  \long\def\@gobble#1{}%
\fi
\expandafter\ifx\csname @firstofone\endcsname\relax
  \long\def\@firstofone#1{#1}%
\fi
\expandafter\ifx\csname loop\endcsname\relax
  \expandafter\@firstofone
\else
  \expandafter\@gobble
\fi
{%
  \def\loop#1\repeat{%
    \def\body{#1}%
    \iterate
  }%
  \def\iterate{%
    \body
      \let\next\iterate
    \else
      \let\next\relax
    \fi
    \next
  }%
  \let\repeat=\fi
}%
\def\RestoreCatcodes{}
\count@=0 %
\loop
  \edef\RestoreCatcodes{%
    \RestoreCatcodes
    \catcode\the\count@=\the\catcode\count@\relax
  }%
\ifnum\count@<255 %
  \advance\count@ 1 %
\repeat

\def\RangeCatcodeInvalid#1#2{%
  \count@=#1\relax
  \loop
    \catcode\count@=15 %
  \ifnum\count@<#2\relax
    \advance\count@ 1 %
  \repeat
}
\def\RangeCatcodeCheck#1#2#3{%
  \count@=#1\relax
  \loop
    \ifnum#3=\catcode\count@
    \else
      \errmessage{%
        Character \the\count@\space
        with wrong catcode \the\catcode\count@\space
        instead of \number#3%
      }%
    \fi
  \ifnum\count@<#2\relax
    \advance\count@ 1 %
  \repeat
}
\def\space{ }
\expandafter\ifx\csname LoadCommand\endcsname\relax
  \def\LoadCommand{\input catchfile.sty\relax}%
\fi
\def\Test{%
  \RangeCatcodeInvalid{0}{47}%
  \RangeCatcodeInvalid{58}{64}%
  \RangeCatcodeInvalid{91}{96}%
  \RangeCatcodeInvalid{123}{255}%
  \catcode`\@=12 %
  \catcode`\\=0 %
  \catcode`\%=14 %
  \LoadCommand
  \RangeCatcodeCheck{0}{36}{15}%
  \RangeCatcodeCheck{37}{37}{14}%
  \RangeCatcodeCheck{38}{47}{15}%
  \RangeCatcodeCheck{48}{57}{12}%
  \RangeCatcodeCheck{58}{63}{15}%
  \RangeCatcodeCheck{64}{64}{12}%
  \RangeCatcodeCheck{65}{90}{11}%
  \RangeCatcodeCheck{91}{91}{15}%
  \RangeCatcodeCheck{92}{92}{0}%
  \RangeCatcodeCheck{93}{96}{15}%
  \RangeCatcodeCheck{97}{122}{11}%
  \RangeCatcodeCheck{123}{255}{15}%
  \RestoreCatcodes
}
\Test
\csname @@end\endcsname
\end
%    \end{macrocode}
%    \begin{macrocode}
%</test1>
%    \end{macrocode}
%
% \subsection{\LaTeX}
%
%    \begin{macrocode}
%<*test2>
\NeedsTeXFormat{LaTeX2e}
\nofiles
\listfiles
\tracingnesting=2 %
\documentclass{minimal}
\usepackage{catchfile}[2016/05/16]
\makeatletter
\def\mysetup{%
  \let\do\@makeother
  \dospecials
}
\def\CheckContents{%
  \begingroup
    \expandafter\ifx\csname contents\endcsname\relax
      \PackageError{TEST}{\string\contents\space is not defined}\@ehc
    \fi
  \endgroup
}
\def\StartDisableUnexpanded{%
  \begingroup
    \let\etex@unexpanded\@undefined
}
\def\StopDisableUnexpanded{%
  \endgroup
}
\def\CheckCleanup{%
  \begingroup
    \edef\x{\the\toks0}%
    \def\y{ABC}%
    \ifx\x\y
    \else
      \PackageError{TEST}{\string\toks0 has changed}\@ehc
    \fi
    \ifx\CatchFile@gtemp\@undefined
    \else
      \PackageError{TEST}{Left over \string\CatchFile@gtemp}\@ehc
    \fi
  \endgroup
}
\def\Check{%
  \CheckContents
  \CheckCleanup
}
\makeatother
\begin{document}
\toks0{ABC}

\CatchFileDef\contents{catchfile.sty}\mysetup
\typeout{\meaning\contents}
\Check
\typeout{***********************************************************}%
\CatchFileDef\contents{catchfile.sty}{}
\typeout{\meaning\contents}
\Check
\typeout{***********************************************************}%

\StartDisableUnexpanded
\CatchFileDef\contents{catchfile.sty}\mysetup
\typeout{\meaning\contents}
\Check
\typeout{***********************************************************}%
\CatchFileDef\contents{catchfile.sty}{}
\typeout{\meaning\contents}
\Check
\typeout{***********************************************************}%
\StopDisableUnexpanded

\CatchFileEdef\contents{catchfile.sty}{%
  \mysetup
  \def\par{^^J}%
  \obeylines
}
\typeout{\contents}
\Check
\typeout{***********************************************************}%
\CatchFileEdef\contents{catchfile.sty}{%
  \catcode`\\=12 %
  \catcode`\#=12 %
  \def\par{^^J}%
  \obeylines
}
\typeout{\contents}
\Check
\typeout{***********************************************************}%
\end{document}
%</test2>
%    \end{macrocode}
%
% \subsection{\plainTeX}
%
%    \begin{macrocode}
%<*test3>
\def\msg#{\immediate\write16}
\newlinechar=10 %
\tracingnesting=2 %
\input catchfile.sty\relax

\def\mysetup{%
  \def\do##1{%
    \catcode`##1=12\relax
  }%
  \dospecials
}
\def\CheckContents{%
  \begingroup
    \expandafter\ifx\csname contents\endcsname\relax
      \def\space{ }%
      \errmessage{\string\contents\space is not defined.}%
    \fi
  \endgroup
}
\catcode`\@=11 %
\def\CheckCleanup{%
  \begingroup
    \edef\x{\the\toks0}%
    \def\y{ABC}%
    \ifx\x\y
    \else
      \errmessage{\string\toks0 has changed}%
    \fi
    \ifx\CatchFile@gtemp\@undefined
    \else
      \errmessage{Left over \string\CatchFile@gtemp}%
    \fi
  \endgroup
}
\def\Check{%
  \CheckContents
  \CheckCleanup
}
\def\StartDisableUnexpanded{%
  \begingroup
    \let\etex@unexpanded\@undefined
}
\def\StopDisableUnexpanded{%
  \endgroup
}
\catcode`\@=12 %

\toks0{ABC}

\CatchFileDef\contents{catchfile.sty}\mysetup
\msg{\meaning\contents}
\Check
\msg{***********************************************************}%
\CatchFileDef\contents{catchfile.sty}{}
\msg{\meaning\contents}
\Check
\msg{***********************************************************}%

\StartDisableUnexpanded
\CatchFileDef\contents{catchfile.sty}\mysetup
\msg{\meaning\contents}
\Check
\msg{***********************************************************}%
\CatchFileDef\contents{catchfile.sty}{}
\msg{\meaning\contents}
\Check
\msg{***********************************************************}%
\StopDisableUnexpanded

\CatchFileEdef\contents{catchfile.sty}{%
  \mysetup
  \def\par{^^J}%
  \obeylines
}
\msg{\contents}
\Check
\msg{***********************************************************}%
\CatchFileEdef\contents{catchfile.sty}{%
  \catcode`\\=12 %
  \catcode`\#=12 %
  \def\par{^^J}%
  \obeylines
}
\msg{\contents}
\Check
\msg{***********************************************************}%

\csname @@end\endcsname
\end
%</test3>
%    \end{macrocode}
%
% \section{Installation}
%
% \subsection{Download}
%
% \paragraph{Package.} This package is available on
% CTAN\footnote{\CTANpkg{catchfile}}:
% \begin{description}
% \item[\CTAN{macros/latex/contrib/oberdiek/catchfile.dtx}] The source file.
% \item[\CTAN{macros/latex/contrib/oberdiek/catchfile.pdf}] Documentation.
% \end{description}
%
%
% \paragraph{Bundle.} All the packages of the bundle `oberdiek'
% are also available in a TDS compliant ZIP archive. There
% the packages are already unpacked and the documentation files
% are generated. The files and directories obey the TDS standard.
% \begin{description}
% \item[\CTANinstall{install/macros/latex/contrib/oberdiek.tds.zip}]
% \end{description}
% \emph{TDS} refers to the standard ``A Directory Structure
% for \TeX\ Files'' (\CTAN{tds/tds.pdf}). Directories
% with \xfile{texmf} in their name are usually organized this way.
%
% \subsection{Bundle installation}
%
% \paragraph{Unpacking.} Unpack the \xfile{oberdiek.tds.zip} in the
% TDS tree (also known as \xfile{texmf} tree) of your choice.
% Example (linux):
% \begin{quote}
%   |unzip oberdiek.tds.zip -d ~/texmf|
% \end{quote}
%
% \paragraph{Script installation.}
% Check the directory \xfile{TDS:scripts/oberdiek/} for
% scripts that need further installation steps.
% Package \xpackage{attachfile2} comes with the Perl script
% \xfile{pdfatfi.pl} that should be installed in such a way
% that it can be called as \texttt{pdfatfi}.
% Example (linux):
% \begin{quote}
%   |chmod +x scripts/oberdiek/pdfatfi.pl|\\
%   |cp scripts/oberdiek/pdfatfi.pl /usr/local/bin/|
% \end{quote}
%
% \subsection{Package installation}
%
% \paragraph{Unpacking.} The \xfile{.dtx} file is a self-extracting
% \docstrip\ archive. The files are extracted by running the
% \xfile{.dtx} through \plainTeX:
% \begin{quote}
%   \verb|tex catchfile.dtx|
% \end{quote}
%
% \paragraph{TDS.} Now the different files must be moved into
% the different directories in your installation TDS tree
% (also known as \xfile{texmf} tree):
% \begin{quote}
% \def\t{^^A
% \begin{tabular}{@{}>{\ttfamily}l@{ $\rightarrow$ }>{\ttfamily}l@{}}
%   catchfile.sty & tex/generic/oberdiek/catchfile.sty\\
%   catchfile.pdf & doc/latex/oberdiek/catchfile.pdf\\
%   test/catchfile-test1.tex & doc/latex/oberdiek/test/catchfile-test1.tex\\
%   test/catchfile-test2.tex & doc/latex/oberdiek/test/catchfile-test2.tex\\
%   test/catchfile-test3.tex & doc/latex/oberdiek/test/catchfile-test3.tex\\
%   catchfile.dtx & source/latex/oberdiek/catchfile.dtx\\
% \end{tabular}^^A
% }^^A
% \sbox0{\t}^^A
% \ifdim\wd0>\linewidth
%   \begingroup
%     \advance\linewidth by\leftmargin
%     \advance\linewidth by\rightmargin
%   \edef\x{\endgroup
%     \def\noexpand\lw{\the\linewidth}^^A
%   }\x
%   \def\lwbox{^^A
%     \leavevmode
%     \hbox to \linewidth{^^A
%       \kern-\leftmargin\relax
%       \hss
%       \usebox0
%       \hss
%       \kern-\rightmargin\relax
%     }^^A
%   }^^A
%   \ifdim\wd0>\lw
%     \sbox0{\small\t}^^A
%     \ifdim\wd0>\linewidth
%       \ifdim\wd0>\lw
%         \sbox0{\footnotesize\t}^^A
%         \ifdim\wd0>\linewidth
%           \ifdim\wd0>\lw
%             \sbox0{\scriptsize\t}^^A
%             \ifdim\wd0>\linewidth
%               \ifdim\wd0>\lw
%                 \sbox0{\tiny\t}^^A
%                 \ifdim\wd0>\linewidth
%                   \lwbox
%                 \else
%                   \usebox0
%                 \fi
%               \else
%                 \lwbox
%               \fi
%             \else
%               \usebox0
%             \fi
%           \else
%             \lwbox
%           \fi
%         \else
%           \usebox0
%         \fi
%       \else
%         \lwbox
%       \fi
%     \else
%       \usebox0
%     \fi
%   \else
%     \lwbox
%   \fi
% \else
%   \usebox0
% \fi
% \end{quote}
% If you have a \xfile{docstrip.cfg} that configures and enables \docstrip's
% TDS installing feature, then some files can already be in the right
% place, see the documentation of \docstrip.
%
% \subsection{Refresh file name databases}
%
% If your \TeX~distribution
% (\teTeX, \mikTeX, \dots) relies on file name databases, you must refresh
% these. For example, \teTeX\ users run \verb|texhash| or
% \verb|mktexlsr|.
%
% \subsection{Some details for the interested}
%
% \paragraph{Attached source.}
%
% The PDF documentation on CTAN also includes the
% \xfile{.dtx} source file. It can be extracted by
% AcrobatReader 6 or higher. Another option is \textsf{pdftk},
% e.g. unpack the file into the current directory:
% \begin{quote}
%   \verb|pdftk catchfile.pdf unpack_files output .|
% \end{quote}
%
% \paragraph{Unpacking with \LaTeX.}
% The \xfile{.dtx} chooses its action depending on the format:
% \begin{description}
% \item[\plainTeX:] Run \docstrip\ and extract the files.
% \item[\LaTeX:] Generate the documentation.
% \end{description}
% If you insist on using \LaTeX\ for \docstrip\ (really,
% \docstrip\ does not need \LaTeX), then inform the autodetect routine
% about your intention:
% \begin{quote}
%   \verb|latex \let\install=y% \iffalse meta-comment
%
% File: catchfile.dtx
% Version: 2016/05/16 v1.7
% Info: Catch the contents of a file
%
% Copyright (C) 2007, 2010, 2011 by
%    Heiko Oberdiek <heiko.oberdiek at googlemail.com>
%    2016
%    https://github.com/ho-tex/oberdiek/issues
%
% This work may be distributed and/or modified under the
% conditions of the LaTeX Project Public License, either
% version 1.3c of this license or (at your option) any later
% version. This version of this license is in
%    http://www.latex-project.org/lppl/lppl-1-3c.txt
% and the latest version of this license is in
%    http://www.latex-project.org/lppl.txt
% and version 1.3 or later is part of all distributions of
% LaTeX version 2005/12/01 or later.
%
% This work has the LPPL maintenance status "maintained".
%
% This Current Maintainer of this work is Heiko Oberdiek.
%
% The Base Interpreter refers to any `TeX-Format',
% because some files are installed in TDS:tex/generic//.
%
% This work consists of the main source file catchfile.dtx
% and the derived files
%    catchfile.sty, catchfile.pdf, catchfile.ins, catchfile.drv,
%    catchfile-test1.tex, catchfile-test2.tex,
%    catchfile-test3.tex.
%
% Distribution:
%    CTAN:macros/latex/contrib/oberdiek/catchfile.dtx
%    CTAN:macros/latex/contrib/oberdiek/catchfile.pdf
%
% Unpacking:
%    (a) If catchfile.ins is present:
%           tex catchfile.ins
%    (b) Without catchfile.ins:
%           tex catchfile.dtx
%    (c) If you insist on using LaTeX
%           latex \let\install=y% \iffalse meta-comment
%
% File: catchfile.dtx
% Version: 2016/05/16 v1.7
% Info: Catch the contents of a file
%
% Copyright (C) 2007, 2010, 2011 by
%    Heiko Oberdiek <heiko.oberdiek at googlemail.com>
%    2016
%    https://github.com/ho-tex/oberdiek/issues
%
% This work may be distributed and/or modified under the
% conditions of the LaTeX Project Public License, either
% version 1.3c of this license or (at your option) any later
% version. This version of this license is in
%    http://www.latex-project.org/lppl/lppl-1-3c.txt
% and the latest version of this license is in
%    http://www.latex-project.org/lppl.txt
% and version 1.3 or later is part of all distributions of
% LaTeX version 2005/12/01 or later.
%
% This work has the LPPL maintenance status "maintained".
%
% This Current Maintainer of this work is Heiko Oberdiek.
%
% The Base Interpreter refers to any `TeX-Format',
% because some files are installed in TDS:tex/generic//.
%
% This work consists of the main source file catchfile.dtx
% and the derived files
%    catchfile.sty, catchfile.pdf, catchfile.ins, catchfile.drv,
%    catchfile-test1.tex, catchfile-test2.tex,
%    catchfile-test3.tex.
%
% Distribution:
%    CTAN:macros/latex/contrib/oberdiek/catchfile.dtx
%    CTAN:macros/latex/contrib/oberdiek/catchfile.pdf
%
% Unpacking:
%    (a) If catchfile.ins is present:
%           tex catchfile.ins
%    (b) Without catchfile.ins:
%           tex catchfile.dtx
%    (c) If you insist on using LaTeX
%           latex \let\install=y% \iffalse meta-comment
%
% File: catchfile.dtx
% Version: 2016/05/16 v1.7
% Info: Catch the contents of a file
%
% Copyright (C) 2007, 2010, 2011 by
%    Heiko Oberdiek <heiko.oberdiek at googlemail.com>
%    2016
%    https://github.com/ho-tex/oberdiek/issues
%
% This work may be distributed and/or modified under the
% conditions of the LaTeX Project Public License, either
% version 1.3c of this license or (at your option) any later
% version. This version of this license is in
%    http://www.latex-project.org/lppl/lppl-1-3c.txt
% and the latest version of this license is in
%    http://www.latex-project.org/lppl.txt
% and version 1.3 or later is part of all distributions of
% LaTeX version 2005/12/01 or later.
%
% This work has the LPPL maintenance status "maintained".
%
% This Current Maintainer of this work is Heiko Oberdiek.
%
% The Base Interpreter refers to any `TeX-Format',
% because some files are installed in TDS:tex/generic//.
%
% This work consists of the main source file catchfile.dtx
% and the derived files
%    catchfile.sty, catchfile.pdf, catchfile.ins, catchfile.drv,
%    catchfile-test1.tex, catchfile-test2.tex,
%    catchfile-test3.tex.
%
% Distribution:
%    CTAN:macros/latex/contrib/oberdiek/catchfile.dtx
%    CTAN:macros/latex/contrib/oberdiek/catchfile.pdf
%
% Unpacking:
%    (a) If catchfile.ins is present:
%           tex catchfile.ins
%    (b) Without catchfile.ins:
%           tex catchfile.dtx
%    (c) If you insist on using LaTeX
%           latex \let\install=y\input{catchfile.dtx}
%        (quote the arguments according to the demands of your shell)
%
% Documentation:
%    (a) If catchfile.drv is present:
%           latex catchfile.drv
%    (b) Without catchfile.drv:
%           latex catchfile.dtx; ...
%    The class ltxdoc loads the configuration file ltxdoc.cfg
%    if available. Here you can specify further options, e.g.
%    use A4 as paper format:
%       \PassOptionsToClass{a4paper}{article}
%
%    Programm calls to get the documentation (example):
%       pdflatex catchfile.dtx
%       makeindex -s gind.ist catchfile.idx
%       pdflatex catchfile.dtx
%       makeindex -s gind.ist catchfile.idx
%       pdflatex catchfile.dtx
%
% Installation:
%    TDS:tex/generic/oberdiek/catchfile.sty
%    TDS:doc/latex/oberdiek/catchfile.pdf
%    TDS:doc/latex/oberdiek/test/catchfile-test1.tex
%    TDS:doc/latex/oberdiek/test/catchfile-test2.tex
%    TDS:doc/latex/oberdiek/test/catchfile-test3.tex
%    TDS:source/latex/oberdiek/catchfile.dtx
%
%<*ignore>
\begingroup
  \catcode123=1 %
  \catcode125=2 %
  \def\x{LaTeX2e}%
\expandafter\endgroup
\ifcase 0\ifx\install y1\fi\expandafter
         \ifx\csname processbatchFile\endcsname\relax\else1\fi
         \ifx\fmtname\x\else 1\fi\relax
\else\csname fi\endcsname
%</ignore>
%<*install>
\input docstrip.tex
\Msg{************************************************************************}
\Msg{* Installation}
\Msg{* Package: catchfile 2016/05/16 v1.7 Catch the contents of a file (HO)}
\Msg{************************************************************************}

\keepsilent
\askforoverwritefalse

\let\MetaPrefix\relax
\preamble

This is a generated file.

Project: catchfile
Version: 2016/05/16 v1.7

Copyright (C) 2007, 2010, 2011 by
   Heiko Oberdiek <heiko.oberdiek at googlemail.com>

This work may be distributed and/or modified under the
conditions of the LaTeX Project Public License, either
version 1.3c of this license or (at your option) any later
version. This version of this license is in
   http://www.latex-project.org/lppl/lppl-1-3c.txt
and the latest version of this license is in
   http://www.latex-project.org/lppl.txt
and version 1.3 or later is part of all distributions of
LaTeX version 2005/12/01 or later.

This work has the LPPL maintenance status "maintained".

This Current Maintainer of this work is Heiko Oberdiek.

The Base Interpreter refers to any `TeX-Format',
because some files are installed in TDS:tex/generic//.

This work consists of the main source file catchfile.dtx
and the derived files
   catchfile.sty, catchfile.pdf, catchfile.ins, catchfile.drv,
   catchfile-test1.tex, catchfile-test2.tex,
   catchfile-test3.tex.

\endpreamble
\let\MetaPrefix\DoubleperCent

\generate{%
  \file{catchfile.ins}{\from{catchfile.dtx}{install}}%
  \file{catchfile.drv}{\from{catchfile.dtx}{driver}}%
  \usedir{tex/generic/oberdiek}%
  \file{catchfile.sty}{\from{catchfile.dtx}{package}}%
%  \usedir{doc/latex/oberdiek/test}%
%  \file{catchfile-test1.tex}{\from{catchfile.dtx}{test1}}%
%  \file{catchfile-test2.tex}{\from{catchfile.dtx}{test2}}%
%  \file{catchfile-test3.tex}{\from{catchfile.dtx}{test3}}%
  \nopreamble
  \nopostamble
%  \usedir{source/latex/oberdiek/catalogue}%
%  \file{catchfile.xml}{\from{catchfile.dtx}{catalogue}}%
}

\catcode32=13\relax% active space
\let =\space%
\Msg{************************************************************************}
\Msg{*}
\Msg{* To finish the installation you have to move the following}
\Msg{* file into a directory searched by TeX:}
\Msg{*}
\Msg{*     catchfile.sty}
\Msg{*}
\Msg{* To produce the documentation run the file `catchfile.drv'}
\Msg{* through LaTeX.}
\Msg{*}
\Msg{* Happy TeXing!}
\Msg{*}
\Msg{************************************************************************}

\endbatchfile
%</install>
%<*ignore>
\fi
%</ignore>
%<*driver>
\NeedsTeXFormat{LaTeX2e}
\ProvidesFile{catchfile.drv}%
  [2016/05/16 v1.7 Catch the contents of a file (HO)]%
\documentclass{ltxdoc}
\usepackage{holtxdoc}[2011/11/22]
\begin{document}
  \DocInput{catchfile.dtx}%
\end{document}
%</driver>
% \fi
%
%
% \CharacterTable
%  {Upper-case    \A\B\C\D\E\F\G\H\I\J\K\L\M\N\O\P\Q\R\S\T\U\V\W\X\Y\Z
%   Lower-case    \a\b\c\d\e\f\g\h\i\j\k\l\m\n\o\p\q\r\s\t\u\v\w\x\y\z
%   Digits        \0\1\2\3\4\5\6\7\8\9
%   Exclamation   \!     Double quote  \"     Hash (number) \#
%   Dollar        \$     Percent       \%     Ampersand     \&
%   Acute accent  \'     Left paren    \(     Right paren   \)
%   Asterisk      \*     Plus          \+     Comma         \,
%   Minus         \-     Point         \.     Solidus       \/
%   Colon         \:     Semicolon     \;     Less than     \<
%   Equals        \=     Greater than  \>     Question mark \?
%   Commercial at \@     Left bracket  \[     Backslash     \\
%   Right bracket \]     Circumflex    \^     Underscore    \_
%   Grave accent  \`     Left brace    \{     Vertical bar  \|
%   Right brace   \}     Tilde         \~}
%
% \GetFileInfo{catchfile.drv}
%
% \title{The \xpackage{catchfile} package}
% \date{2016/05/16 v1.7}
% \author{Heiko Oberdiek\thanks
% {Please report any issues at https://github.com/ho-tex/oberdiek/issues}\\
% \xemail{heiko.oberdiek at googlemail.com}}
%
% \maketitle
%
% \begin{abstract}
% This package catches the contents of a file and puts it in a macro.
% It requires \eTeX. Both \LaTeX\ and \plainTeX\ are supported.
% \end{abstract}
%
% \tableofcontents
%
% \section{Documentation}
%
% The package relies on \eTeX's \cs{everyeof}. Otherwise it
% aborts with an error message.
%
% \begin{declcs}{CatchFileDef} \M{cmd} \M{file name} \M{setup}\\
% \cs{CatchFileEdef} \M{cmd} \M{file name} \M{setup}
% \end{declcs}
% Macro \meta{cmd} is defined with the contents
% of file \meta{file name}.
% \cs{CatchFileDef} uses \cs{def}, \cs{CatchFileEdef} \cs{edef}
% for the definition.
% Additional setup code for setting
% catcodes or treatment of line ends can be given in code
% \meta{setup}. See the test files for an example.
%
% \StopEventually{
% }
%
% \section{Implementation}
%
%    \begin{macrocode}
%<*package>
%    \end{macrocode}
%
% \subsection{Reload check and package identification}
%    Reload check, especially if the package is not used with \LaTeX.
%    \begin{macrocode}
\begingroup\catcode61\catcode48\catcode32=10\relax%
  \catcode13=5 % ^^M
  \endlinechar=13 %
  \catcode35=6 % #
  \catcode39=12 % '
  \catcode44=12 % ,
  \catcode45=12 % -
  \catcode46=12 % .
  \catcode58=12 % :
  \catcode64=11 % @
  \catcode123=1 % {
  \catcode125=2 % }
  \expandafter\let\expandafter\x\csname ver@catchfile.sty\endcsname
  \ifx\x\relax % plain-TeX, first loading
  \else
    \def\empty{}%
    \ifx\x\empty % LaTeX, first loading,
      % variable is initialized, but \ProvidesPackage not yet seen
    \else
      \expandafter\ifx\csname PackageInfo\endcsname\relax
        \def\x#1#2{%
          \immediate\write-1{Package #1 Info: #2.}%
        }%
      \else
        \def\x#1#2{\PackageInfo{#1}{#2, stopped}}%
      \fi
      \x{catchfile}{The package is already loaded}%
      \aftergroup\endinput
    \fi
  \fi
\endgroup%
%    \end{macrocode}
%    Package identification:
%    \begin{macrocode}
\begingroup\catcode61\catcode48\catcode32=10\relax%
  \catcode13=5 % ^^M
  \endlinechar=13 %
  \catcode35=6 % #
  \catcode39=12 % '
  \catcode40=12 % (
  \catcode41=12 % )
  \catcode44=12 % ,
  \catcode45=12 % -
  \catcode46=12 % .
  \catcode47=12 % /
  \catcode58=12 % :
  \catcode64=11 % @
  \catcode91=12 % [
  \catcode93=12 % ]
  \catcode123=1 % {
  \catcode125=2 % }
  \expandafter\ifx\csname ProvidesPackage\endcsname\relax
    \def\x#1#2#3[#4]{\endgroup
      \immediate\write-1{Package: #3 #4}%
      \xdef#1{#4}%
    }%
  \else
    \def\x#1#2[#3]{\endgroup
      #2[{#3}]%
      \ifx#1\@undefined
        \xdef#1{#3}%
      \fi
      \ifx#1\relax
        \xdef#1{#3}%
      \fi
    }%
  \fi
\expandafter\x\csname ver@catchfile.sty\endcsname
\ProvidesPackage{catchfile}%
  [2016/05/16 v1.7 Catch the contents of a file (HO)]%
%    \end{macrocode}
%
% \subsection{Catcodes}
%
%    \begin{macrocode}
\begingroup\catcode61\catcode48\catcode32=10\relax%
  \catcode13=5 % ^^M
  \endlinechar=13 %
  \catcode123=1 % {
  \catcode125=2 % }
  \catcode64=11 % @
  \def\x{\endgroup
    \expandafter\edef\csname CatchFile@AtEnd\endcsname{%
      \endlinechar=\the\endlinechar\relax
      \catcode13=\the\catcode13\relax
      \catcode32=\the\catcode32\relax
      \catcode35=\the\catcode35\relax
      \catcode61=\the\catcode61\relax
      \catcode64=\the\catcode64\relax
      \catcode123=\the\catcode123\relax
      \catcode125=\the\catcode125\relax
    }%
  }%
\x\catcode61\catcode48\catcode32=10\relax%
\catcode13=5 % ^^M
\endlinechar=13 %
\catcode35=6 % #
\catcode64=11 % @
\catcode123=1 % {
\catcode125=2 % }
\def\TMP@EnsureCode#1#2{%
  \edef\CatchFile@AtEnd{%
    \CatchFile@AtEnd
    \catcode#1=\the\catcode#1\relax
  }%
  \catcode#1=#2\relax
}
\TMP@EnsureCode{39}{12}% '
\TMP@EnsureCode{44}{12}% ,
\TMP@EnsureCode{45}{12}% -
\TMP@EnsureCode{46}{12}% .
\TMP@EnsureCode{47}{12}% /
\TMP@EnsureCode{91}{12}% [
\TMP@EnsureCode{93}{12}% ]
\TMP@EnsureCode{96}{12}% `
\edef\CatchFile@AtEnd{\CatchFile@AtEnd\noexpand\endinput}
%    \end{macrocode}
%
% \subsection{Preparations}
%
%    \begin{macrocode}
\begingroup\expandafter\expandafter\expandafter\endgroup
\expandafter\ifx\csname RequirePackage\endcsname\relax
  \input infwarerr.sty\relax
  \input ltxcmds.sty\relax
\else
  \RequirePackage{infwarerr}[2007/09/09]%
  \RequirePackage{ltxcmds}[2010/03/09]%
\fi
%    \end{macrocode}
%
%    Check for \eTeX's \cs{everyeof}.
%    \begin{macrocode}
\begingroup
  \escapechar=92\relax
  \edef\TestString{\string\everyeof}%
  \edef\TestMeaning{\meaning\everyeof}%
  \ifx\TestString\TestMeaning
  \else
    \@PackageError{catchfile}{%
      Cannot find e-TeX's \string\everyeof,\MessageBreak
      package loading is aborted%
    }\@ehd
    \endgroup
    \expandafter\CatchFile@AtEnd
  \fi%
\endgroup
%    \end{macrocode}
%
% \subsection{Looking for primitive \cs{input}}
%
%    \begin{macro}{\CatchFile@Input}
%    The package needs the expandable primitive \cs{input}.
%    However there are formats that redefine it. For example,
%    \LaTeX's \cs{input} is not expandable, but it stores
%    the primitive in \cs{@@input}. The third possibility
%    is \cs{pdfprimitive}, introduced in \pdfTeX\ 1.40.0.
%
%    Thus we try to find the primitive and store it in
%    \cs{CatchFile@Input}. If it is used, it must be expanded
%    twice (because of the solution with \cs{pdfprimitive}).
%    \begin{macrocode}
\begingroup
  \def\Check#1#2#3#4\endgroup{%
    \edef\TestString{\string#1}%
    \edef\TestMeaning{\meaning#2}%
    \ifx\TestString\TestMeaning
      \endgroup
      \let\CatchFile@Primitive#2%
      \def\CatchFile@Input{\CatchFile@Primitive#3}%
    \else
      #4\endgroup
    \fi
  }%
  \Check\input\input{}%
  \Check\input\@@input{}%
  \Check\pdfprimitive\pdfprimitive\input
  \@PackageError{%
    Cannot find primitive \string\input,\MessageBreak
    package loading is aborted%
  }\@ehd
  \csname endgroup\endcsname
  \CatchFile@AtEnd%
\endgroup
%    \end{macrocode}
%    \end{macro}
%
% \subsection{Input file check}
%
%    \begin{macro}{\CatchFile@CheckFileExists}
%    \begin{macrocode}
\begingroup\expandafter\expandafter\expandafter\endgroup
\expandafter\ifx\csname IfFileExists\endcsname\relax
  \input pdftexcmds.sty\relax
  \begingroup\expandafter\expandafter\expandafter\endgroup
  \expandafter\ifx\csname pdf@filesize\endcsname\relax
    \def\CatchFile@CheckFileExists#1{%
      \expandafter\ifx\csname @inputcheck\endcsname\relax
        \csname newread\endcsname\@inputcheck
      \fi
      \openin\@inputcheck#1\relax
      \expandafter\closein\expandafter\@inputcheck
      \ifeof\@inputcheck
        \let\CatchFile@File\relax
      \else
        \def\CatchFile@File{#1}%
      \fi
    }%
  \else
    \def\CatchFile@CheckFileExists#1{%
      \expandafter\expandafter\expandafter\ifx
      \expandafter\expandafter\expandafter\relax\pdf@filesize{#1}\relax
        \let\CatchFile@File\relax
      \else
        \def\CatchFile@File{#1}%
      \fi
    }%
  \fi
\else
  \def\CatchFile@CheckFileExists#1{%
    \IfFileExists{#1}{%
      \expandafter\CatchFile@DefFile\@filef@und\@nil
      \begingroup\expandafter\expandafter\expandafter\endgroup
      \expandafter\ifx\csname @addtofilelist\endcsname\relax
      \else
        \@addtofilelist\CatchFile@File
      \fi
    }{%
      \let\CatchFile@File\relax
    }%
  }%
  \def\CatchFile@DefFile#1 \@nil{%
    \def\CatchFile@File{#1}%
  }%
\fi
%    \end{macrocode}
%    \end{macro}
%
%    \begin{macro}{\CatchFile@NotFound}
%    \begin{macrocode}
\def\CatchFile@NotFound#1#2{%
  \def#1{}%
  \@PackageError{catchfile}{%
    File `#2' not found%
  }\@ehc
}
%    \end{macrocode}
%    \end{macro}
%
% \subsection{Catch file contents}
%
%    \begin{macrocode}
\ltx@IfUndefined{RequirePackage}{%
  \input etexcmds.sty\relax
}{%
  \RequirePackage{etexcmds}[2010/01/28]%
}
%    \end{macrocode}
%
%    \begin{macro}{\CatchFileEdef}
%    \begin{macrocode}
\long\def\CatchFileEdef#1#2#3{%
  \CatchFile@CheckFileExists{#2}%
  \ifx\CatchFile@File\relax
    \CatchFile@NotFound{#1}{#2}%
  \else
    \begingroup
      \everyeof{\noexpand}%
      #3%
      \xdef\CatchFile@Contents{\CatchFile@Input\CatchFile@File\space}%
    \endgroup
    \let#1\CatchFile@Contents
  \fi
}
%    \end{macrocode}
%    \end{macro}
%
%    \begin{macro}{\CatchFileDef}
%    \begin{macrocode}
\long\def\CatchFileDef#1#2#3{%
  \CatchFile@CheckFileExists{#2}%
  \ifx\CatchFile@File\relax
    \CatchFile@NotFound{#1}{#2}%
  \else
    \begingroup
      \everyeof\expandafter{%
        \CatchFile@EOF
        \expandafter\CatchFile@Finish
        \noexpand
      }%
      \ltx@ifundefined{etex@unexpanded}{%
        \expandafter\long\expandafter\def\expandafter\CatchFile@Do
            \expandafter##\expandafter1\CatchFile@EOF{%
          \toks\ltx@zero{##1}%
          \xdef\CatchFile@gtemp{\the\toks\ltx@zero}%
          \def\CatchFile@Finish{%
            \endgroup
            \let#1\CatchFile@gtemp
            \global\let\CatchFile@gtemp\ltx@undefined
          }%
        }%
      }{%
        \expandafter\long\expandafter\def\expandafter\CatchFile@Do
            \expandafter##\expandafter1\CatchFile@EOF{%
          \edef\CatchFile@Finish{%
            \endgroup
            \etex@unexpanded{%
              \edef#1{\etex@unexpanded{##1}}%
            }%
          }%
        }%
      }%
      #3\relax
    \expandafter\expandafter\expandafter\CatchFile@Do
    \CatchFile@Input\CatchFile@File\relax
  \fi
}
%    \end{macrocode}
%    \cs{relax} after |#3| was added to make it more robust in
%    case the user uses something like
%\begin{verbatim}
%\CatchFileDef{\content}{\jobname.tt}{\endlinechar=-1}
%\end{verbatim}
%    that expands the following \cs{expandafter} after |#3| prematurely
%    (contribution of Martin Scharrer).
%    \end{macro}
%    \begin{macro}{\CatchFile@EOF}
%    If the file is read the catcode mappings are fixed.
%    This means that the same character cannot occur
%    inside the file with different catcodes.
%    Thus we use as end of file marker the at sign
%    twice with different catcodes.
%    \begin{macrocode}
\begingroup
  \lccode65=64 % lowercase('A') = '@'
  \lccode66=64 % lowercase('B') = '@'
  \catcode65=8 % catcode('A') = subscript
  \catcode66=3 % catcode('B') = math shift
\lowercase{\endgroup
  \def\CatchFile@EOF{AB}%
}
%    \end{macrocode}
%    \end{macro}
%
%    \begin{macrocode}
\CatchFile@AtEnd%
%</package>
%    \end{macrocode}
%
% \section{Test}
%
% \subsection{Catcode checks for loading}
%
%    \begin{macrocode}
%<*test1>
%    \end{macrocode}
%    \begin{macrocode}
\catcode`\{=1 %
\catcode`\}=2 %
\catcode`\#=6 %
\catcode`\@=11 %
\expandafter\ifx\csname count@\endcsname\relax
  \countdef\count@=255 %
\fi
\expandafter\ifx\csname @gobble\endcsname\relax
  \long\def\@gobble#1{}%
\fi
\expandafter\ifx\csname @firstofone\endcsname\relax
  \long\def\@firstofone#1{#1}%
\fi
\expandafter\ifx\csname loop\endcsname\relax
  \expandafter\@firstofone
\else
  \expandafter\@gobble
\fi
{%
  \def\loop#1\repeat{%
    \def\body{#1}%
    \iterate
  }%
  \def\iterate{%
    \body
      \let\next\iterate
    \else
      \let\next\relax
    \fi
    \next
  }%
  \let\repeat=\fi
}%
\def\RestoreCatcodes{}
\count@=0 %
\loop
  \edef\RestoreCatcodes{%
    \RestoreCatcodes
    \catcode\the\count@=\the\catcode\count@\relax
  }%
\ifnum\count@<255 %
  \advance\count@ 1 %
\repeat

\def\RangeCatcodeInvalid#1#2{%
  \count@=#1\relax
  \loop
    \catcode\count@=15 %
  \ifnum\count@<#2\relax
    \advance\count@ 1 %
  \repeat
}
\def\RangeCatcodeCheck#1#2#3{%
  \count@=#1\relax
  \loop
    \ifnum#3=\catcode\count@
    \else
      \errmessage{%
        Character \the\count@\space
        with wrong catcode \the\catcode\count@\space
        instead of \number#3%
      }%
    \fi
  \ifnum\count@<#2\relax
    \advance\count@ 1 %
  \repeat
}
\def\space{ }
\expandafter\ifx\csname LoadCommand\endcsname\relax
  \def\LoadCommand{\input catchfile.sty\relax}%
\fi
\def\Test{%
  \RangeCatcodeInvalid{0}{47}%
  \RangeCatcodeInvalid{58}{64}%
  \RangeCatcodeInvalid{91}{96}%
  \RangeCatcodeInvalid{123}{255}%
  \catcode`\@=12 %
  \catcode`\\=0 %
  \catcode`\%=14 %
  \LoadCommand
  \RangeCatcodeCheck{0}{36}{15}%
  \RangeCatcodeCheck{37}{37}{14}%
  \RangeCatcodeCheck{38}{47}{15}%
  \RangeCatcodeCheck{48}{57}{12}%
  \RangeCatcodeCheck{58}{63}{15}%
  \RangeCatcodeCheck{64}{64}{12}%
  \RangeCatcodeCheck{65}{90}{11}%
  \RangeCatcodeCheck{91}{91}{15}%
  \RangeCatcodeCheck{92}{92}{0}%
  \RangeCatcodeCheck{93}{96}{15}%
  \RangeCatcodeCheck{97}{122}{11}%
  \RangeCatcodeCheck{123}{255}{15}%
  \RestoreCatcodes
}
\Test
\csname @@end\endcsname
\end
%    \end{macrocode}
%    \begin{macrocode}
%</test1>
%    \end{macrocode}
%
% \subsection{\LaTeX}
%
%    \begin{macrocode}
%<*test2>
\NeedsTeXFormat{LaTeX2e}
\nofiles
\listfiles
\tracingnesting=2 %
\documentclass{minimal}
\usepackage{catchfile}[2016/05/16]
\makeatletter
\def\mysetup{%
  \let\do\@makeother
  \dospecials
}
\def\CheckContents{%
  \begingroup
    \expandafter\ifx\csname contents\endcsname\relax
      \PackageError{TEST}{\string\contents\space is not defined}\@ehc
    \fi
  \endgroup
}
\def\StartDisableUnexpanded{%
  \begingroup
    \let\etex@unexpanded\@undefined
}
\def\StopDisableUnexpanded{%
  \endgroup
}
\def\CheckCleanup{%
  \begingroup
    \edef\x{\the\toks0}%
    \def\y{ABC}%
    \ifx\x\y
    \else
      \PackageError{TEST}{\string\toks0 has changed}\@ehc
    \fi
    \ifx\CatchFile@gtemp\@undefined
    \else
      \PackageError{TEST}{Left over \string\CatchFile@gtemp}\@ehc
    \fi
  \endgroup
}
\def\Check{%
  \CheckContents
  \CheckCleanup
}
\makeatother
\begin{document}
\toks0{ABC}

\CatchFileDef\contents{catchfile.sty}\mysetup
\typeout{\meaning\contents}
\Check
\typeout{***********************************************************}%
\CatchFileDef\contents{catchfile.sty}{}
\typeout{\meaning\contents}
\Check
\typeout{***********************************************************}%

\StartDisableUnexpanded
\CatchFileDef\contents{catchfile.sty}\mysetup
\typeout{\meaning\contents}
\Check
\typeout{***********************************************************}%
\CatchFileDef\contents{catchfile.sty}{}
\typeout{\meaning\contents}
\Check
\typeout{***********************************************************}%
\StopDisableUnexpanded

\CatchFileEdef\contents{catchfile.sty}{%
  \mysetup
  \def\par{^^J}%
  \obeylines
}
\typeout{\contents}
\Check
\typeout{***********************************************************}%
\CatchFileEdef\contents{catchfile.sty}{%
  \catcode`\\=12 %
  \catcode`\#=12 %
  \def\par{^^J}%
  \obeylines
}
\typeout{\contents}
\Check
\typeout{***********************************************************}%
\end{document}
%</test2>
%    \end{macrocode}
%
% \subsection{\plainTeX}
%
%    \begin{macrocode}
%<*test3>
\def\msg#{\immediate\write16}
\newlinechar=10 %
\tracingnesting=2 %
\input catchfile.sty\relax

\def\mysetup{%
  \def\do##1{%
    \catcode`##1=12\relax
  }%
  \dospecials
}
\def\CheckContents{%
  \begingroup
    \expandafter\ifx\csname contents\endcsname\relax
      \def\space{ }%
      \errmessage{\string\contents\space is not defined.}%
    \fi
  \endgroup
}
\catcode`\@=11 %
\def\CheckCleanup{%
  \begingroup
    \edef\x{\the\toks0}%
    \def\y{ABC}%
    \ifx\x\y
    \else
      \errmessage{\string\toks0 has changed}%
    \fi
    \ifx\CatchFile@gtemp\@undefined
    \else
      \errmessage{Left over \string\CatchFile@gtemp}%
    \fi
  \endgroup
}
\def\Check{%
  \CheckContents
  \CheckCleanup
}
\def\StartDisableUnexpanded{%
  \begingroup
    \let\etex@unexpanded\@undefined
}
\def\StopDisableUnexpanded{%
  \endgroup
}
\catcode`\@=12 %

\toks0{ABC}

\CatchFileDef\contents{catchfile.sty}\mysetup
\msg{\meaning\contents}
\Check
\msg{***********************************************************}%
\CatchFileDef\contents{catchfile.sty}{}
\msg{\meaning\contents}
\Check
\msg{***********************************************************}%

\StartDisableUnexpanded
\CatchFileDef\contents{catchfile.sty}\mysetup
\msg{\meaning\contents}
\Check
\msg{***********************************************************}%
\CatchFileDef\contents{catchfile.sty}{}
\msg{\meaning\contents}
\Check
\msg{***********************************************************}%
\StopDisableUnexpanded

\CatchFileEdef\contents{catchfile.sty}{%
  \mysetup
  \def\par{^^J}%
  \obeylines
}
\msg{\contents}
\Check
\msg{***********************************************************}%
\CatchFileEdef\contents{catchfile.sty}{%
  \catcode`\\=12 %
  \catcode`\#=12 %
  \def\par{^^J}%
  \obeylines
}
\msg{\contents}
\Check
\msg{***********************************************************}%

\csname @@end\endcsname
\end
%</test3>
%    \end{macrocode}
%
% \section{Installation}
%
% \subsection{Download}
%
% \paragraph{Package.} This package is available on
% CTAN\footnote{\url{https://ctan.org/pkg/catchfile}}:
% \begin{description}
% \item[\CTAN{macros/latex/contrib/oberdiek/catchfile.dtx}] The source file.
% \item[\CTAN{macros/latex/contrib/oberdiek/catchfile.pdf}] Documentation.
% \end{description}
%
%
% \paragraph{Bundle.} All the packages of the bundle `oberdiek'
% are also available in a TDS compliant ZIP archive. There
% the packages are already unpacked and the documentation files
% are generated. The files and directories obey the TDS standard.
% \begin{description}
% \item[\CTANinstall{install/macros/latex/contrib/oberdiek.tds.zip}]
% \end{description}
% \emph{TDS} refers to the standard ``A Directory Structure
% for \TeX\ Files'' (\CTAN{tds/tds.pdf}). Directories
% with \xfile{texmf} in their name are usually organized this way.
%
% \subsection{Bundle installation}
%
% \paragraph{Unpacking.} Unpack the \xfile{oberdiek.tds.zip} in the
% TDS tree (also known as \xfile{texmf} tree) of your choice.
% Example (linux):
% \begin{quote}
%   |unzip oberdiek.tds.zip -d ~/texmf|
% \end{quote}
%
% \paragraph{Script installation.}
% Check the directory \xfile{TDS:scripts/oberdiek/} for
% scripts that need further installation steps.
% Package \xpackage{attachfile2} comes with the Perl script
% \xfile{pdfatfi.pl} that should be installed in such a way
% that it can be called as \texttt{pdfatfi}.
% Example (linux):
% \begin{quote}
%   |chmod +x scripts/oberdiek/pdfatfi.pl|\\
%   |cp scripts/oberdiek/pdfatfi.pl /usr/local/bin/|
% \end{quote}
%
% \subsection{Package installation}
%
% \paragraph{Unpacking.} The \xfile{.dtx} file is a self-extracting
% \docstrip\ archive. The files are extracted by running the
% \xfile{.dtx} through \plainTeX:
% \begin{quote}
%   \verb|tex catchfile.dtx|
% \end{quote}
%
% \paragraph{TDS.} Now the different files must be moved into
% the different directories in your installation TDS tree
% (also known as \xfile{texmf} tree):
% \begin{quote}
% \def\t{^^A
% \begin{tabular}{@{}>{\ttfamily}l@{ $\rightarrow$ }>{\ttfamily}l@{}}
%   catchfile.sty & tex/generic/oberdiek/catchfile.sty\\
%   catchfile.pdf & doc/latex/oberdiek/catchfile.pdf\\
%   test/catchfile-test1.tex & doc/latex/oberdiek/test/catchfile-test1.tex\\
%   test/catchfile-test2.tex & doc/latex/oberdiek/test/catchfile-test2.tex\\
%   test/catchfile-test3.tex & doc/latex/oberdiek/test/catchfile-test3.tex\\
%   catchfile.dtx & source/latex/oberdiek/catchfile.dtx\\
% \end{tabular}^^A
% }^^A
% \sbox0{\t}^^A
% \ifdim\wd0>\linewidth
%   \begingroup
%     \advance\linewidth by\leftmargin
%     \advance\linewidth by\rightmargin
%   \edef\x{\endgroup
%     \def\noexpand\lw{\the\linewidth}^^A
%   }\x
%   \def\lwbox{^^A
%     \leavevmode
%     \hbox to \linewidth{^^A
%       \kern-\leftmargin\relax
%       \hss
%       \usebox0
%       \hss
%       \kern-\rightmargin\relax
%     }^^A
%   }^^A
%   \ifdim\wd0>\lw
%     \sbox0{\small\t}^^A
%     \ifdim\wd0>\linewidth
%       \ifdim\wd0>\lw
%         \sbox0{\footnotesize\t}^^A
%         \ifdim\wd0>\linewidth
%           \ifdim\wd0>\lw
%             \sbox0{\scriptsize\t}^^A
%             \ifdim\wd0>\linewidth
%               \ifdim\wd0>\lw
%                 \sbox0{\tiny\t}^^A
%                 \ifdim\wd0>\linewidth
%                   \lwbox
%                 \else
%                   \usebox0
%                 \fi
%               \else
%                 \lwbox
%               \fi
%             \else
%               \usebox0
%             \fi
%           \else
%             \lwbox
%           \fi
%         \else
%           \usebox0
%         \fi
%       \else
%         \lwbox
%       \fi
%     \else
%       \usebox0
%     \fi
%   \else
%     \lwbox
%   \fi
% \else
%   \usebox0
% \fi
% \end{quote}
% If you have a \xfile{docstrip.cfg} that configures and enables \docstrip's
% TDS installing feature, then some files can already be in the right
% place, see the documentation of \docstrip.
%
% \subsection{Refresh file name databases}
%
% If your \TeX~distribution
% (\teTeX, \mikTeX, \dots) relies on file name databases, you must refresh
% these. For example, \teTeX\ users run \verb|texhash| or
% \verb|mktexlsr|.
%
% \subsection{Some details for the interested}
%
% \paragraph{Attached source.}
%
% The PDF documentation on CTAN also includes the
% \xfile{.dtx} source file. It can be extracted by
% AcrobatReader 6 or higher. Another option is \textsf{pdftk},
% e.g. unpack the file into the current directory:
% \begin{quote}
%   \verb|pdftk catchfile.pdf unpack_files output .|
% \end{quote}
%
% \paragraph{Unpacking with \LaTeX.}
% The \xfile{.dtx} chooses its action depending on the format:
% \begin{description}
% \item[\plainTeX:] Run \docstrip\ and extract the files.
% \item[\LaTeX:] Generate the documentation.
% \end{description}
% If you insist on using \LaTeX\ for \docstrip\ (really,
% \docstrip\ does not need \LaTeX), then inform the autodetect routine
% about your intention:
% \begin{quote}
%   \verb|latex \let\install=y\input{catchfile.dtx}|
% \end{quote}
% Do not forget to quote the argument according to the demands
% of your shell.
%
% \paragraph{Generating the documentation.}
% You can use both the \xfile{.dtx} or the \xfile{.drv} to generate
% the documentation. The process can be configured by the
% configuration file \xfile{ltxdoc.cfg}. For instance, put this
% line into this file, if you want to have A4 as paper format:
% \begin{quote}
%   \verb|\PassOptionsToClass{a4paper}{article}|
% \end{quote}
% An example follows how to generate the
% documentation with pdf\LaTeX:
% \begin{quote}
%\begin{verbatim}
%pdflatex catchfile.dtx
%makeindex -s gind.ist catchfile.idx
%pdflatex catchfile.dtx
%makeindex -s gind.ist catchfile.idx
%pdflatex catchfile.dtx
%\end{verbatim}
% \end{quote}
%
% \begin{History}
%   \begin{Version}{2007/05/30 v1.0}
%   \item
%     First version.
%   \end{Version}
%   \begin{Version}{2007/09/09 v1.1}
%   \item
%     Catcode section rewritten.
%   \end{Version}
%   \begin{Version}{2007/11/11 v1.2}
%   \item
%     Use of package \xpackage{pdftexcmds} for \LuaTeX\ support.
%   \end{Version}
%   \begin{Version}{2010/03/01 v1.3}
%   \item
%     Fix for unknown \cs{@PackageErrorNoLine}.
%   \end{Version}
%   \begin{Version}{2010/04/08 v1.4}
%   \item
%     \cs{closein} also added if \cs{ifeof} is true.
%   \end{Version}
%   \begin{Version}{2010/04/28 v1.5}
%   \item
%     \cs{CatchFileDef}: Getting rid of warning `end of semi simple group
%     entered at line \dots\ of
%     a different file (Florent Chervet).
%   \item
%     \cs{CatchFileDef}: Fix for error `Illegal parameter number in
%     definition of \dots' (HO) including improvement that uses
%     \cs{unexpanded} if available (Florent Chervet).
%   \end{Version}
%   \begin{Version}{2011/03/01 v1.6}
%   \item
%     \cs{relax} added after the setup argument of
%     \cs{CatchFileDef} to prevent premature file reading
%     (Martin Scharrer).
%   \end{Version}
%   \begin{Version}{2016/05/16 v1.7}
%   \item
%     Documentation updates.
%   \end{Version}
% \end{History}
%
% \PrintIndex
%
% \Finale
\endinput

%        (quote the arguments according to the demands of your shell)
%
% Documentation:
%    (a) If catchfile.drv is present:
%           latex catchfile.drv
%    (b) Without catchfile.drv:
%           latex catchfile.dtx; ...
%    The class ltxdoc loads the configuration file ltxdoc.cfg
%    if available. Here you can specify further options, e.g.
%    use A4 as paper format:
%       \PassOptionsToClass{a4paper}{article}
%
%    Programm calls to get the documentation (example):
%       pdflatex catchfile.dtx
%       makeindex -s gind.ist catchfile.idx
%       pdflatex catchfile.dtx
%       makeindex -s gind.ist catchfile.idx
%       pdflatex catchfile.dtx
%
% Installation:
%    TDS:tex/generic/oberdiek/catchfile.sty
%    TDS:doc/latex/oberdiek/catchfile.pdf
%    TDS:doc/latex/oberdiek/test/catchfile-test1.tex
%    TDS:doc/latex/oberdiek/test/catchfile-test2.tex
%    TDS:doc/latex/oberdiek/test/catchfile-test3.tex
%    TDS:source/latex/oberdiek/catchfile.dtx
%
%<*ignore>
\begingroup
  \catcode123=1 %
  \catcode125=2 %
  \def\x{LaTeX2e}%
\expandafter\endgroup
\ifcase 0\ifx\install y1\fi\expandafter
         \ifx\csname processbatchFile\endcsname\relax\else1\fi
         \ifx\fmtname\x\else 1\fi\relax
\else\csname fi\endcsname
%</ignore>
%<*install>
\input docstrip.tex
\Msg{************************************************************************}
\Msg{* Installation}
\Msg{* Package: catchfile 2016/05/16 v1.7 Catch the contents of a file (HO)}
\Msg{************************************************************************}

\keepsilent
\askforoverwritefalse

\let\MetaPrefix\relax
\preamble

This is a generated file.

Project: catchfile
Version: 2016/05/16 v1.7

Copyright (C) 2007, 2010, 2011 by
   Heiko Oberdiek <heiko.oberdiek at googlemail.com>

This work may be distributed and/or modified under the
conditions of the LaTeX Project Public License, either
version 1.3c of this license or (at your option) any later
version. This version of this license is in
   http://www.latex-project.org/lppl/lppl-1-3c.txt
and the latest version of this license is in
   http://www.latex-project.org/lppl.txt
and version 1.3 or later is part of all distributions of
LaTeX version 2005/12/01 or later.

This work has the LPPL maintenance status "maintained".

This Current Maintainer of this work is Heiko Oberdiek.

The Base Interpreter refers to any `TeX-Format',
because some files are installed in TDS:tex/generic//.

This work consists of the main source file catchfile.dtx
and the derived files
   catchfile.sty, catchfile.pdf, catchfile.ins, catchfile.drv,
   catchfile-test1.tex, catchfile-test2.tex,
   catchfile-test3.tex.

\endpreamble
\let\MetaPrefix\DoubleperCent

\generate{%
  \file{catchfile.ins}{\from{catchfile.dtx}{install}}%
  \file{catchfile.drv}{\from{catchfile.dtx}{driver}}%
  \usedir{tex/generic/oberdiek}%
  \file{catchfile.sty}{\from{catchfile.dtx}{package}}%
%  \usedir{doc/latex/oberdiek/test}%
%  \file{catchfile-test1.tex}{\from{catchfile.dtx}{test1}}%
%  \file{catchfile-test2.tex}{\from{catchfile.dtx}{test2}}%
%  \file{catchfile-test3.tex}{\from{catchfile.dtx}{test3}}%
  \nopreamble
  \nopostamble
%  \usedir{source/latex/oberdiek/catalogue}%
%  \file{catchfile.xml}{\from{catchfile.dtx}{catalogue}}%
}

\catcode32=13\relax% active space
\let =\space%
\Msg{************************************************************************}
\Msg{*}
\Msg{* To finish the installation you have to move the following}
\Msg{* file into a directory searched by TeX:}
\Msg{*}
\Msg{*     catchfile.sty}
\Msg{*}
\Msg{* To produce the documentation run the file `catchfile.drv'}
\Msg{* through LaTeX.}
\Msg{*}
\Msg{* Happy TeXing!}
\Msg{*}
\Msg{************************************************************************}

\endbatchfile
%</install>
%<*ignore>
\fi
%</ignore>
%<*driver>
\NeedsTeXFormat{LaTeX2e}
\ProvidesFile{catchfile.drv}%
  [2016/05/16 v1.7 Catch the contents of a file (HO)]%
\documentclass{ltxdoc}
\usepackage{holtxdoc}[2011/11/22]
\begin{document}
  \DocInput{catchfile.dtx}%
\end{document}
%</driver>
% \fi
%
%
% \CharacterTable
%  {Upper-case    \A\B\C\D\E\F\G\H\I\J\K\L\M\N\O\P\Q\R\S\T\U\V\W\X\Y\Z
%   Lower-case    \a\b\c\d\e\f\g\h\i\j\k\l\m\n\o\p\q\r\s\t\u\v\w\x\y\z
%   Digits        \0\1\2\3\4\5\6\7\8\9
%   Exclamation   \!     Double quote  \"     Hash (number) \#
%   Dollar        \$     Percent       \%     Ampersand     \&
%   Acute accent  \'     Left paren    \(     Right paren   \)
%   Asterisk      \*     Plus          \+     Comma         \,
%   Minus         \-     Point         \.     Solidus       \/
%   Colon         \:     Semicolon     \;     Less than     \<
%   Equals        \=     Greater than  \>     Question mark \?
%   Commercial at \@     Left bracket  \[     Backslash     \\
%   Right bracket \]     Circumflex    \^     Underscore    \_
%   Grave accent  \`     Left brace    \{     Vertical bar  \|
%   Right brace   \}     Tilde         \~}
%
% \GetFileInfo{catchfile.drv}
%
% \title{The \xpackage{catchfile} package}
% \date{2016/05/16 v1.7}
% \author{Heiko Oberdiek\thanks
% {Please report any issues at https://github.com/ho-tex/oberdiek/issues}\\
% \xemail{heiko.oberdiek at googlemail.com}}
%
% \maketitle
%
% \begin{abstract}
% This package catches the contents of a file and puts it in a macro.
% It requires \eTeX. Both \LaTeX\ and \plainTeX\ are supported.
% \end{abstract}
%
% \tableofcontents
%
% \section{Documentation}
%
% The package relies on \eTeX's \cs{everyeof}. Otherwise it
% aborts with an error message.
%
% \begin{declcs}{CatchFileDef} \M{cmd} \M{file name} \M{setup}\\
% \cs{CatchFileEdef} \M{cmd} \M{file name} \M{setup}
% \end{declcs}
% Macro \meta{cmd} is defined with the contents
% of file \meta{file name}.
% \cs{CatchFileDef} uses \cs{def}, \cs{CatchFileEdef} \cs{edef}
% for the definition.
% Additional setup code for setting
% catcodes or treatment of line ends can be given in code
% \meta{setup}. See the test files for an example.
%
% \StopEventually{
% }
%
% \section{Implementation}
%
%    \begin{macrocode}
%<*package>
%    \end{macrocode}
%
% \subsection{Reload check and package identification}
%    Reload check, especially if the package is not used with \LaTeX.
%    \begin{macrocode}
\begingroup\catcode61\catcode48\catcode32=10\relax%
  \catcode13=5 % ^^M
  \endlinechar=13 %
  \catcode35=6 % #
  \catcode39=12 % '
  \catcode44=12 % ,
  \catcode45=12 % -
  \catcode46=12 % .
  \catcode58=12 % :
  \catcode64=11 % @
  \catcode123=1 % {
  \catcode125=2 % }
  \expandafter\let\expandafter\x\csname ver@catchfile.sty\endcsname
  \ifx\x\relax % plain-TeX, first loading
  \else
    \def\empty{}%
    \ifx\x\empty % LaTeX, first loading,
      % variable is initialized, but \ProvidesPackage not yet seen
    \else
      \expandafter\ifx\csname PackageInfo\endcsname\relax
        \def\x#1#2{%
          \immediate\write-1{Package #1 Info: #2.}%
        }%
      \else
        \def\x#1#2{\PackageInfo{#1}{#2, stopped}}%
      \fi
      \x{catchfile}{The package is already loaded}%
      \aftergroup\endinput
    \fi
  \fi
\endgroup%
%    \end{macrocode}
%    Package identification:
%    \begin{macrocode}
\begingroup\catcode61\catcode48\catcode32=10\relax%
  \catcode13=5 % ^^M
  \endlinechar=13 %
  \catcode35=6 % #
  \catcode39=12 % '
  \catcode40=12 % (
  \catcode41=12 % )
  \catcode44=12 % ,
  \catcode45=12 % -
  \catcode46=12 % .
  \catcode47=12 % /
  \catcode58=12 % :
  \catcode64=11 % @
  \catcode91=12 % [
  \catcode93=12 % ]
  \catcode123=1 % {
  \catcode125=2 % }
  \expandafter\ifx\csname ProvidesPackage\endcsname\relax
    \def\x#1#2#3[#4]{\endgroup
      \immediate\write-1{Package: #3 #4}%
      \xdef#1{#4}%
    }%
  \else
    \def\x#1#2[#3]{\endgroup
      #2[{#3}]%
      \ifx#1\@undefined
        \xdef#1{#3}%
      \fi
      \ifx#1\relax
        \xdef#1{#3}%
      \fi
    }%
  \fi
\expandafter\x\csname ver@catchfile.sty\endcsname
\ProvidesPackage{catchfile}%
  [2016/05/16 v1.7 Catch the contents of a file (HO)]%
%    \end{macrocode}
%
% \subsection{Catcodes}
%
%    \begin{macrocode}
\begingroup\catcode61\catcode48\catcode32=10\relax%
  \catcode13=5 % ^^M
  \endlinechar=13 %
  \catcode123=1 % {
  \catcode125=2 % }
  \catcode64=11 % @
  \def\x{\endgroup
    \expandafter\edef\csname CatchFile@AtEnd\endcsname{%
      \endlinechar=\the\endlinechar\relax
      \catcode13=\the\catcode13\relax
      \catcode32=\the\catcode32\relax
      \catcode35=\the\catcode35\relax
      \catcode61=\the\catcode61\relax
      \catcode64=\the\catcode64\relax
      \catcode123=\the\catcode123\relax
      \catcode125=\the\catcode125\relax
    }%
  }%
\x\catcode61\catcode48\catcode32=10\relax%
\catcode13=5 % ^^M
\endlinechar=13 %
\catcode35=6 % #
\catcode64=11 % @
\catcode123=1 % {
\catcode125=2 % }
\def\TMP@EnsureCode#1#2{%
  \edef\CatchFile@AtEnd{%
    \CatchFile@AtEnd
    \catcode#1=\the\catcode#1\relax
  }%
  \catcode#1=#2\relax
}
\TMP@EnsureCode{39}{12}% '
\TMP@EnsureCode{44}{12}% ,
\TMP@EnsureCode{45}{12}% -
\TMP@EnsureCode{46}{12}% .
\TMP@EnsureCode{47}{12}% /
\TMP@EnsureCode{91}{12}% [
\TMP@EnsureCode{93}{12}% ]
\TMP@EnsureCode{96}{12}% `
\edef\CatchFile@AtEnd{\CatchFile@AtEnd\noexpand\endinput}
%    \end{macrocode}
%
% \subsection{Preparations}
%
%    \begin{macrocode}
\begingroup\expandafter\expandafter\expandafter\endgroup
\expandafter\ifx\csname RequirePackage\endcsname\relax
  \input infwarerr.sty\relax
  \input ltxcmds.sty\relax
\else
  \RequirePackage{infwarerr}[2007/09/09]%
  \RequirePackage{ltxcmds}[2010/03/09]%
\fi
%    \end{macrocode}
%
%    Check for \eTeX's \cs{everyeof}.
%    \begin{macrocode}
\begingroup
  \escapechar=92\relax
  \edef\TestString{\string\everyeof}%
  \edef\TestMeaning{\meaning\everyeof}%
  \ifx\TestString\TestMeaning
  \else
    \@PackageError{catchfile}{%
      Cannot find e-TeX's \string\everyeof,\MessageBreak
      package loading is aborted%
    }\@ehd
    \endgroup
    \expandafter\CatchFile@AtEnd
  \fi%
\endgroup
%    \end{macrocode}
%
% \subsection{Looking for primitive \cs{input}}
%
%    \begin{macro}{\CatchFile@Input}
%    The package needs the expandable primitive \cs{input}.
%    However there are formats that redefine it. For example,
%    \LaTeX's \cs{input} is not expandable, but it stores
%    the primitive in \cs{@@input}. The third possibility
%    is \cs{pdfprimitive}, introduced in \pdfTeX\ 1.40.0.
%
%    Thus we try to find the primitive and store it in
%    \cs{CatchFile@Input}. If it is used, it must be expanded
%    twice (because of the solution with \cs{pdfprimitive}).
%    \begin{macrocode}
\begingroup
  \def\Check#1#2#3#4\endgroup{%
    \edef\TestString{\string#1}%
    \edef\TestMeaning{\meaning#2}%
    \ifx\TestString\TestMeaning
      \endgroup
      \let\CatchFile@Primitive#2%
      \def\CatchFile@Input{\CatchFile@Primitive#3}%
    \else
      #4\endgroup
    \fi
  }%
  \Check\input\input{}%
  \Check\input\@@input{}%
  \Check\pdfprimitive\pdfprimitive\input
  \@PackageError{%
    Cannot find primitive \string\input,\MessageBreak
    package loading is aborted%
  }\@ehd
  \csname endgroup\endcsname
  \CatchFile@AtEnd%
\endgroup
%    \end{macrocode}
%    \end{macro}
%
% \subsection{Input file check}
%
%    \begin{macro}{\CatchFile@CheckFileExists}
%    \begin{macrocode}
\begingroup\expandafter\expandafter\expandafter\endgroup
\expandafter\ifx\csname IfFileExists\endcsname\relax
  \input pdftexcmds.sty\relax
  \begingroup\expandafter\expandafter\expandafter\endgroup
  \expandafter\ifx\csname pdf@filesize\endcsname\relax
    \def\CatchFile@CheckFileExists#1{%
      \expandafter\ifx\csname @inputcheck\endcsname\relax
        \csname newread\endcsname\@inputcheck
      \fi
      \openin\@inputcheck#1\relax
      \expandafter\closein\expandafter\@inputcheck
      \ifeof\@inputcheck
        \let\CatchFile@File\relax
      \else
        \def\CatchFile@File{#1}%
      \fi
    }%
  \else
    \def\CatchFile@CheckFileExists#1{%
      \expandafter\expandafter\expandafter\ifx
      \expandafter\expandafter\expandafter\relax\pdf@filesize{#1}\relax
        \let\CatchFile@File\relax
      \else
        \def\CatchFile@File{#1}%
      \fi
    }%
  \fi
\else
  \def\CatchFile@CheckFileExists#1{%
    \IfFileExists{#1}{%
      \expandafter\CatchFile@DefFile\@filef@und\@nil
      \begingroup\expandafter\expandafter\expandafter\endgroup
      \expandafter\ifx\csname @addtofilelist\endcsname\relax
      \else
        \@addtofilelist\CatchFile@File
      \fi
    }{%
      \let\CatchFile@File\relax
    }%
  }%
  \def\CatchFile@DefFile#1 \@nil{%
    \def\CatchFile@File{#1}%
  }%
\fi
%    \end{macrocode}
%    \end{macro}
%
%    \begin{macro}{\CatchFile@NotFound}
%    \begin{macrocode}
\def\CatchFile@NotFound#1#2{%
  \def#1{}%
  \@PackageError{catchfile}{%
    File `#2' not found%
  }\@ehc
}
%    \end{macrocode}
%    \end{macro}
%
% \subsection{Catch file contents}
%
%    \begin{macrocode}
\ltx@IfUndefined{RequirePackage}{%
  \input etexcmds.sty\relax
}{%
  \RequirePackage{etexcmds}[2010/01/28]%
}
%    \end{macrocode}
%
%    \begin{macro}{\CatchFileEdef}
%    \begin{macrocode}
\long\def\CatchFileEdef#1#2#3{%
  \CatchFile@CheckFileExists{#2}%
  \ifx\CatchFile@File\relax
    \CatchFile@NotFound{#1}{#2}%
  \else
    \begingroup
      \everyeof{\noexpand}%
      #3%
      \xdef\CatchFile@Contents{\CatchFile@Input\CatchFile@File\space}%
    \endgroup
    \let#1\CatchFile@Contents
  \fi
}
%    \end{macrocode}
%    \end{macro}
%
%    \begin{macro}{\CatchFileDef}
%    \begin{macrocode}
\long\def\CatchFileDef#1#2#3{%
  \CatchFile@CheckFileExists{#2}%
  \ifx\CatchFile@File\relax
    \CatchFile@NotFound{#1}{#2}%
  \else
    \begingroup
      \everyeof\expandafter{%
        \CatchFile@EOF
        \expandafter\CatchFile@Finish
        \noexpand
      }%
      \ltx@ifundefined{etex@unexpanded}{%
        \expandafter\long\expandafter\def\expandafter\CatchFile@Do
            \expandafter##\expandafter1\CatchFile@EOF{%
          \toks\ltx@zero{##1}%
          \xdef\CatchFile@gtemp{\the\toks\ltx@zero}%
          \def\CatchFile@Finish{%
            \endgroup
            \let#1\CatchFile@gtemp
            \global\let\CatchFile@gtemp\ltx@undefined
          }%
        }%
      }{%
        \expandafter\long\expandafter\def\expandafter\CatchFile@Do
            \expandafter##\expandafter1\CatchFile@EOF{%
          \edef\CatchFile@Finish{%
            \endgroup
            \etex@unexpanded{%
              \edef#1{\etex@unexpanded{##1}}%
            }%
          }%
        }%
      }%
      #3\relax
    \expandafter\expandafter\expandafter\CatchFile@Do
    \CatchFile@Input\CatchFile@File\relax
  \fi
}
%    \end{macrocode}
%    \cs{relax} after |#3| was added to make it more robust in
%    case the user uses something like
%\begin{verbatim}
%\CatchFileDef{\content}{\jobname.tt}{\endlinechar=-1}
%\end{verbatim}
%    that expands the following \cs{expandafter} after |#3| prematurely
%    (contribution of Martin Scharrer).
%    \end{macro}
%    \begin{macro}{\CatchFile@EOF}
%    If the file is read the catcode mappings are fixed.
%    This means that the same character cannot occur
%    inside the file with different catcodes.
%    Thus we use as end of file marker the at sign
%    twice with different catcodes.
%    \begin{macrocode}
\begingroup
  \lccode65=64 % lowercase('A') = '@'
  \lccode66=64 % lowercase('B') = '@'
  \catcode65=8 % catcode('A') = subscript
  \catcode66=3 % catcode('B') = math shift
\lowercase{\endgroup
  \def\CatchFile@EOF{AB}%
}
%    \end{macrocode}
%    \end{macro}
%
%    \begin{macrocode}
\CatchFile@AtEnd%
%</package>
%    \end{macrocode}
%
% \section{Test}
%
% \subsection{Catcode checks for loading}
%
%    \begin{macrocode}
%<*test1>
%    \end{macrocode}
%    \begin{macrocode}
\catcode`\{=1 %
\catcode`\}=2 %
\catcode`\#=6 %
\catcode`\@=11 %
\expandafter\ifx\csname count@\endcsname\relax
  \countdef\count@=255 %
\fi
\expandafter\ifx\csname @gobble\endcsname\relax
  \long\def\@gobble#1{}%
\fi
\expandafter\ifx\csname @firstofone\endcsname\relax
  \long\def\@firstofone#1{#1}%
\fi
\expandafter\ifx\csname loop\endcsname\relax
  \expandafter\@firstofone
\else
  \expandafter\@gobble
\fi
{%
  \def\loop#1\repeat{%
    \def\body{#1}%
    \iterate
  }%
  \def\iterate{%
    \body
      \let\next\iterate
    \else
      \let\next\relax
    \fi
    \next
  }%
  \let\repeat=\fi
}%
\def\RestoreCatcodes{}
\count@=0 %
\loop
  \edef\RestoreCatcodes{%
    \RestoreCatcodes
    \catcode\the\count@=\the\catcode\count@\relax
  }%
\ifnum\count@<255 %
  \advance\count@ 1 %
\repeat

\def\RangeCatcodeInvalid#1#2{%
  \count@=#1\relax
  \loop
    \catcode\count@=15 %
  \ifnum\count@<#2\relax
    \advance\count@ 1 %
  \repeat
}
\def\RangeCatcodeCheck#1#2#3{%
  \count@=#1\relax
  \loop
    \ifnum#3=\catcode\count@
    \else
      \errmessage{%
        Character \the\count@\space
        with wrong catcode \the\catcode\count@\space
        instead of \number#3%
      }%
    \fi
  \ifnum\count@<#2\relax
    \advance\count@ 1 %
  \repeat
}
\def\space{ }
\expandafter\ifx\csname LoadCommand\endcsname\relax
  \def\LoadCommand{\input catchfile.sty\relax}%
\fi
\def\Test{%
  \RangeCatcodeInvalid{0}{47}%
  \RangeCatcodeInvalid{58}{64}%
  \RangeCatcodeInvalid{91}{96}%
  \RangeCatcodeInvalid{123}{255}%
  \catcode`\@=12 %
  \catcode`\\=0 %
  \catcode`\%=14 %
  \LoadCommand
  \RangeCatcodeCheck{0}{36}{15}%
  \RangeCatcodeCheck{37}{37}{14}%
  \RangeCatcodeCheck{38}{47}{15}%
  \RangeCatcodeCheck{48}{57}{12}%
  \RangeCatcodeCheck{58}{63}{15}%
  \RangeCatcodeCheck{64}{64}{12}%
  \RangeCatcodeCheck{65}{90}{11}%
  \RangeCatcodeCheck{91}{91}{15}%
  \RangeCatcodeCheck{92}{92}{0}%
  \RangeCatcodeCheck{93}{96}{15}%
  \RangeCatcodeCheck{97}{122}{11}%
  \RangeCatcodeCheck{123}{255}{15}%
  \RestoreCatcodes
}
\Test
\csname @@end\endcsname
\end
%    \end{macrocode}
%    \begin{macrocode}
%</test1>
%    \end{macrocode}
%
% \subsection{\LaTeX}
%
%    \begin{macrocode}
%<*test2>
\NeedsTeXFormat{LaTeX2e}
\nofiles
\listfiles
\tracingnesting=2 %
\documentclass{minimal}
\usepackage{catchfile}[2016/05/16]
\makeatletter
\def\mysetup{%
  \let\do\@makeother
  \dospecials
}
\def\CheckContents{%
  \begingroup
    \expandafter\ifx\csname contents\endcsname\relax
      \PackageError{TEST}{\string\contents\space is not defined}\@ehc
    \fi
  \endgroup
}
\def\StartDisableUnexpanded{%
  \begingroup
    \let\etex@unexpanded\@undefined
}
\def\StopDisableUnexpanded{%
  \endgroup
}
\def\CheckCleanup{%
  \begingroup
    \edef\x{\the\toks0}%
    \def\y{ABC}%
    \ifx\x\y
    \else
      \PackageError{TEST}{\string\toks0 has changed}\@ehc
    \fi
    \ifx\CatchFile@gtemp\@undefined
    \else
      \PackageError{TEST}{Left over \string\CatchFile@gtemp}\@ehc
    \fi
  \endgroup
}
\def\Check{%
  \CheckContents
  \CheckCleanup
}
\makeatother
\begin{document}
\toks0{ABC}

\CatchFileDef\contents{catchfile.sty}\mysetup
\typeout{\meaning\contents}
\Check
\typeout{***********************************************************}%
\CatchFileDef\contents{catchfile.sty}{}
\typeout{\meaning\contents}
\Check
\typeout{***********************************************************}%

\StartDisableUnexpanded
\CatchFileDef\contents{catchfile.sty}\mysetup
\typeout{\meaning\contents}
\Check
\typeout{***********************************************************}%
\CatchFileDef\contents{catchfile.sty}{}
\typeout{\meaning\contents}
\Check
\typeout{***********************************************************}%
\StopDisableUnexpanded

\CatchFileEdef\contents{catchfile.sty}{%
  \mysetup
  \def\par{^^J}%
  \obeylines
}
\typeout{\contents}
\Check
\typeout{***********************************************************}%
\CatchFileEdef\contents{catchfile.sty}{%
  \catcode`\\=12 %
  \catcode`\#=12 %
  \def\par{^^J}%
  \obeylines
}
\typeout{\contents}
\Check
\typeout{***********************************************************}%
\end{document}
%</test2>
%    \end{macrocode}
%
% \subsection{\plainTeX}
%
%    \begin{macrocode}
%<*test3>
\def\msg#{\immediate\write16}
\newlinechar=10 %
\tracingnesting=2 %
\input catchfile.sty\relax

\def\mysetup{%
  \def\do##1{%
    \catcode`##1=12\relax
  }%
  \dospecials
}
\def\CheckContents{%
  \begingroup
    \expandafter\ifx\csname contents\endcsname\relax
      \def\space{ }%
      \errmessage{\string\contents\space is not defined.}%
    \fi
  \endgroup
}
\catcode`\@=11 %
\def\CheckCleanup{%
  \begingroup
    \edef\x{\the\toks0}%
    \def\y{ABC}%
    \ifx\x\y
    \else
      \errmessage{\string\toks0 has changed}%
    \fi
    \ifx\CatchFile@gtemp\@undefined
    \else
      \errmessage{Left over \string\CatchFile@gtemp}%
    \fi
  \endgroup
}
\def\Check{%
  \CheckContents
  \CheckCleanup
}
\def\StartDisableUnexpanded{%
  \begingroup
    \let\etex@unexpanded\@undefined
}
\def\StopDisableUnexpanded{%
  \endgroup
}
\catcode`\@=12 %

\toks0{ABC}

\CatchFileDef\contents{catchfile.sty}\mysetup
\msg{\meaning\contents}
\Check
\msg{***********************************************************}%
\CatchFileDef\contents{catchfile.sty}{}
\msg{\meaning\contents}
\Check
\msg{***********************************************************}%

\StartDisableUnexpanded
\CatchFileDef\contents{catchfile.sty}\mysetup
\msg{\meaning\contents}
\Check
\msg{***********************************************************}%
\CatchFileDef\contents{catchfile.sty}{}
\msg{\meaning\contents}
\Check
\msg{***********************************************************}%
\StopDisableUnexpanded

\CatchFileEdef\contents{catchfile.sty}{%
  \mysetup
  \def\par{^^J}%
  \obeylines
}
\msg{\contents}
\Check
\msg{***********************************************************}%
\CatchFileEdef\contents{catchfile.sty}{%
  \catcode`\\=12 %
  \catcode`\#=12 %
  \def\par{^^J}%
  \obeylines
}
\msg{\contents}
\Check
\msg{***********************************************************}%

\csname @@end\endcsname
\end
%</test3>
%    \end{macrocode}
%
% \section{Installation}
%
% \subsection{Download}
%
% \paragraph{Package.} This package is available on
% CTAN\footnote{\url{https://ctan.org/pkg/catchfile}}:
% \begin{description}
% \item[\CTAN{macros/latex/contrib/oberdiek/catchfile.dtx}] The source file.
% \item[\CTAN{macros/latex/contrib/oberdiek/catchfile.pdf}] Documentation.
% \end{description}
%
%
% \paragraph{Bundle.} All the packages of the bundle `oberdiek'
% are also available in a TDS compliant ZIP archive. There
% the packages are already unpacked and the documentation files
% are generated. The files and directories obey the TDS standard.
% \begin{description}
% \item[\CTANinstall{install/macros/latex/contrib/oberdiek.tds.zip}]
% \end{description}
% \emph{TDS} refers to the standard ``A Directory Structure
% for \TeX\ Files'' (\CTAN{tds/tds.pdf}). Directories
% with \xfile{texmf} in their name are usually organized this way.
%
% \subsection{Bundle installation}
%
% \paragraph{Unpacking.} Unpack the \xfile{oberdiek.tds.zip} in the
% TDS tree (also known as \xfile{texmf} tree) of your choice.
% Example (linux):
% \begin{quote}
%   |unzip oberdiek.tds.zip -d ~/texmf|
% \end{quote}
%
% \paragraph{Script installation.}
% Check the directory \xfile{TDS:scripts/oberdiek/} for
% scripts that need further installation steps.
% Package \xpackage{attachfile2} comes with the Perl script
% \xfile{pdfatfi.pl} that should be installed in such a way
% that it can be called as \texttt{pdfatfi}.
% Example (linux):
% \begin{quote}
%   |chmod +x scripts/oberdiek/pdfatfi.pl|\\
%   |cp scripts/oberdiek/pdfatfi.pl /usr/local/bin/|
% \end{quote}
%
% \subsection{Package installation}
%
% \paragraph{Unpacking.} The \xfile{.dtx} file is a self-extracting
% \docstrip\ archive. The files are extracted by running the
% \xfile{.dtx} through \plainTeX:
% \begin{quote}
%   \verb|tex catchfile.dtx|
% \end{quote}
%
% \paragraph{TDS.} Now the different files must be moved into
% the different directories in your installation TDS tree
% (also known as \xfile{texmf} tree):
% \begin{quote}
% \def\t{^^A
% \begin{tabular}{@{}>{\ttfamily}l@{ $\rightarrow$ }>{\ttfamily}l@{}}
%   catchfile.sty & tex/generic/oberdiek/catchfile.sty\\
%   catchfile.pdf & doc/latex/oberdiek/catchfile.pdf\\
%   test/catchfile-test1.tex & doc/latex/oberdiek/test/catchfile-test1.tex\\
%   test/catchfile-test2.tex & doc/latex/oberdiek/test/catchfile-test2.tex\\
%   test/catchfile-test3.tex & doc/latex/oberdiek/test/catchfile-test3.tex\\
%   catchfile.dtx & source/latex/oberdiek/catchfile.dtx\\
% \end{tabular}^^A
% }^^A
% \sbox0{\t}^^A
% \ifdim\wd0>\linewidth
%   \begingroup
%     \advance\linewidth by\leftmargin
%     \advance\linewidth by\rightmargin
%   \edef\x{\endgroup
%     \def\noexpand\lw{\the\linewidth}^^A
%   }\x
%   \def\lwbox{^^A
%     \leavevmode
%     \hbox to \linewidth{^^A
%       \kern-\leftmargin\relax
%       \hss
%       \usebox0
%       \hss
%       \kern-\rightmargin\relax
%     }^^A
%   }^^A
%   \ifdim\wd0>\lw
%     \sbox0{\small\t}^^A
%     \ifdim\wd0>\linewidth
%       \ifdim\wd0>\lw
%         \sbox0{\footnotesize\t}^^A
%         \ifdim\wd0>\linewidth
%           \ifdim\wd0>\lw
%             \sbox0{\scriptsize\t}^^A
%             \ifdim\wd0>\linewidth
%               \ifdim\wd0>\lw
%                 \sbox0{\tiny\t}^^A
%                 \ifdim\wd0>\linewidth
%                   \lwbox
%                 \else
%                   \usebox0
%                 \fi
%               \else
%                 \lwbox
%               \fi
%             \else
%               \usebox0
%             \fi
%           \else
%             \lwbox
%           \fi
%         \else
%           \usebox0
%         \fi
%       \else
%         \lwbox
%       \fi
%     \else
%       \usebox0
%     \fi
%   \else
%     \lwbox
%   \fi
% \else
%   \usebox0
% \fi
% \end{quote}
% If you have a \xfile{docstrip.cfg} that configures and enables \docstrip's
% TDS installing feature, then some files can already be in the right
% place, see the documentation of \docstrip.
%
% \subsection{Refresh file name databases}
%
% If your \TeX~distribution
% (\teTeX, \mikTeX, \dots) relies on file name databases, you must refresh
% these. For example, \teTeX\ users run \verb|texhash| or
% \verb|mktexlsr|.
%
% \subsection{Some details for the interested}
%
% \paragraph{Attached source.}
%
% The PDF documentation on CTAN also includes the
% \xfile{.dtx} source file. It can be extracted by
% AcrobatReader 6 or higher. Another option is \textsf{pdftk},
% e.g. unpack the file into the current directory:
% \begin{quote}
%   \verb|pdftk catchfile.pdf unpack_files output .|
% \end{quote}
%
% \paragraph{Unpacking with \LaTeX.}
% The \xfile{.dtx} chooses its action depending on the format:
% \begin{description}
% \item[\plainTeX:] Run \docstrip\ and extract the files.
% \item[\LaTeX:] Generate the documentation.
% \end{description}
% If you insist on using \LaTeX\ for \docstrip\ (really,
% \docstrip\ does not need \LaTeX), then inform the autodetect routine
% about your intention:
% \begin{quote}
%   \verb|latex \let\install=y% \iffalse meta-comment
%
% File: catchfile.dtx
% Version: 2016/05/16 v1.7
% Info: Catch the contents of a file
%
% Copyright (C) 2007, 2010, 2011 by
%    Heiko Oberdiek <heiko.oberdiek at googlemail.com>
%    2016
%    https://github.com/ho-tex/oberdiek/issues
%
% This work may be distributed and/or modified under the
% conditions of the LaTeX Project Public License, either
% version 1.3c of this license or (at your option) any later
% version. This version of this license is in
%    http://www.latex-project.org/lppl/lppl-1-3c.txt
% and the latest version of this license is in
%    http://www.latex-project.org/lppl.txt
% and version 1.3 or later is part of all distributions of
% LaTeX version 2005/12/01 or later.
%
% This work has the LPPL maintenance status "maintained".
%
% This Current Maintainer of this work is Heiko Oberdiek.
%
% The Base Interpreter refers to any `TeX-Format',
% because some files are installed in TDS:tex/generic//.
%
% This work consists of the main source file catchfile.dtx
% and the derived files
%    catchfile.sty, catchfile.pdf, catchfile.ins, catchfile.drv,
%    catchfile-test1.tex, catchfile-test2.tex,
%    catchfile-test3.tex.
%
% Distribution:
%    CTAN:macros/latex/contrib/oberdiek/catchfile.dtx
%    CTAN:macros/latex/contrib/oberdiek/catchfile.pdf
%
% Unpacking:
%    (a) If catchfile.ins is present:
%           tex catchfile.ins
%    (b) Without catchfile.ins:
%           tex catchfile.dtx
%    (c) If you insist on using LaTeX
%           latex \let\install=y\input{catchfile.dtx}
%        (quote the arguments according to the demands of your shell)
%
% Documentation:
%    (a) If catchfile.drv is present:
%           latex catchfile.drv
%    (b) Without catchfile.drv:
%           latex catchfile.dtx; ...
%    The class ltxdoc loads the configuration file ltxdoc.cfg
%    if available. Here you can specify further options, e.g.
%    use A4 as paper format:
%       \PassOptionsToClass{a4paper}{article}
%
%    Programm calls to get the documentation (example):
%       pdflatex catchfile.dtx
%       makeindex -s gind.ist catchfile.idx
%       pdflatex catchfile.dtx
%       makeindex -s gind.ist catchfile.idx
%       pdflatex catchfile.dtx
%
% Installation:
%    TDS:tex/generic/oberdiek/catchfile.sty
%    TDS:doc/latex/oberdiek/catchfile.pdf
%    TDS:doc/latex/oberdiek/test/catchfile-test1.tex
%    TDS:doc/latex/oberdiek/test/catchfile-test2.tex
%    TDS:doc/latex/oberdiek/test/catchfile-test3.tex
%    TDS:source/latex/oberdiek/catchfile.dtx
%
%<*ignore>
\begingroup
  \catcode123=1 %
  \catcode125=2 %
  \def\x{LaTeX2e}%
\expandafter\endgroup
\ifcase 0\ifx\install y1\fi\expandafter
         \ifx\csname processbatchFile\endcsname\relax\else1\fi
         \ifx\fmtname\x\else 1\fi\relax
\else\csname fi\endcsname
%</ignore>
%<*install>
\input docstrip.tex
\Msg{************************************************************************}
\Msg{* Installation}
\Msg{* Package: catchfile 2016/05/16 v1.7 Catch the contents of a file (HO)}
\Msg{************************************************************************}

\keepsilent
\askforoverwritefalse

\let\MetaPrefix\relax
\preamble

This is a generated file.

Project: catchfile
Version: 2016/05/16 v1.7

Copyright (C) 2007, 2010, 2011 by
   Heiko Oberdiek <heiko.oberdiek at googlemail.com>

This work may be distributed and/or modified under the
conditions of the LaTeX Project Public License, either
version 1.3c of this license or (at your option) any later
version. This version of this license is in
   http://www.latex-project.org/lppl/lppl-1-3c.txt
and the latest version of this license is in
   http://www.latex-project.org/lppl.txt
and version 1.3 or later is part of all distributions of
LaTeX version 2005/12/01 or later.

This work has the LPPL maintenance status "maintained".

This Current Maintainer of this work is Heiko Oberdiek.

The Base Interpreter refers to any `TeX-Format',
because some files are installed in TDS:tex/generic//.

This work consists of the main source file catchfile.dtx
and the derived files
   catchfile.sty, catchfile.pdf, catchfile.ins, catchfile.drv,
   catchfile-test1.tex, catchfile-test2.tex,
   catchfile-test3.tex.

\endpreamble
\let\MetaPrefix\DoubleperCent

\generate{%
  \file{catchfile.ins}{\from{catchfile.dtx}{install}}%
  \file{catchfile.drv}{\from{catchfile.dtx}{driver}}%
  \usedir{tex/generic/oberdiek}%
  \file{catchfile.sty}{\from{catchfile.dtx}{package}}%
%  \usedir{doc/latex/oberdiek/test}%
%  \file{catchfile-test1.tex}{\from{catchfile.dtx}{test1}}%
%  \file{catchfile-test2.tex}{\from{catchfile.dtx}{test2}}%
%  \file{catchfile-test3.tex}{\from{catchfile.dtx}{test3}}%
  \nopreamble
  \nopostamble
%  \usedir{source/latex/oberdiek/catalogue}%
%  \file{catchfile.xml}{\from{catchfile.dtx}{catalogue}}%
}

\catcode32=13\relax% active space
\let =\space%
\Msg{************************************************************************}
\Msg{*}
\Msg{* To finish the installation you have to move the following}
\Msg{* file into a directory searched by TeX:}
\Msg{*}
\Msg{*     catchfile.sty}
\Msg{*}
\Msg{* To produce the documentation run the file `catchfile.drv'}
\Msg{* through LaTeX.}
\Msg{*}
\Msg{* Happy TeXing!}
\Msg{*}
\Msg{************************************************************************}

\endbatchfile
%</install>
%<*ignore>
\fi
%</ignore>
%<*driver>
\NeedsTeXFormat{LaTeX2e}
\ProvidesFile{catchfile.drv}%
  [2016/05/16 v1.7 Catch the contents of a file (HO)]%
\documentclass{ltxdoc}
\usepackage{holtxdoc}[2011/11/22]
\begin{document}
  \DocInput{catchfile.dtx}%
\end{document}
%</driver>
% \fi
%
%
% \CharacterTable
%  {Upper-case    \A\B\C\D\E\F\G\H\I\J\K\L\M\N\O\P\Q\R\S\T\U\V\W\X\Y\Z
%   Lower-case    \a\b\c\d\e\f\g\h\i\j\k\l\m\n\o\p\q\r\s\t\u\v\w\x\y\z
%   Digits        \0\1\2\3\4\5\6\7\8\9
%   Exclamation   \!     Double quote  \"     Hash (number) \#
%   Dollar        \$     Percent       \%     Ampersand     \&
%   Acute accent  \'     Left paren    \(     Right paren   \)
%   Asterisk      \*     Plus          \+     Comma         \,
%   Minus         \-     Point         \.     Solidus       \/
%   Colon         \:     Semicolon     \;     Less than     \<
%   Equals        \=     Greater than  \>     Question mark \?
%   Commercial at \@     Left bracket  \[     Backslash     \\
%   Right bracket \]     Circumflex    \^     Underscore    \_
%   Grave accent  \`     Left brace    \{     Vertical bar  \|
%   Right brace   \}     Tilde         \~}
%
% \GetFileInfo{catchfile.drv}
%
% \title{The \xpackage{catchfile} package}
% \date{2016/05/16 v1.7}
% \author{Heiko Oberdiek\thanks
% {Please report any issues at https://github.com/ho-tex/oberdiek/issues}\\
% \xemail{heiko.oberdiek at googlemail.com}}
%
% \maketitle
%
% \begin{abstract}
% This package catches the contents of a file and puts it in a macro.
% It requires \eTeX. Both \LaTeX\ and \plainTeX\ are supported.
% \end{abstract}
%
% \tableofcontents
%
% \section{Documentation}
%
% The package relies on \eTeX's \cs{everyeof}. Otherwise it
% aborts with an error message.
%
% \begin{declcs}{CatchFileDef} \M{cmd} \M{file name} \M{setup}\\
% \cs{CatchFileEdef} \M{cmd} \M{file name} \M{setup}
% \end{declcs}
% Macro \meta{cmd} is defined with the contents
% of file \meta{file name}.
% \cs{CatchFileDef} uses \cs{def}, \cs{CatchFileEdef} \cs{edef}
% for the definition.
% Additional setup code for setting
% catcodes or treatment of line ends can be given in code
% \meta{setup}. See the test files for an example.
%
% \StopEventually{
% }
%
% \section{Implementation}
%
%    \begin{macrocode}
%<*package>
%    \end{macrocode}
%
% \subsection{Reload check and package identification}
%    Reload check, especially if the package is not used with \LaTeX.
%    \begin{macrocode}
\begingroup\catcode61\catcode48\catcode32=10\relax%
  \catcode13=5 % ^^M
  \endlinechar=13 %
  \catcode35=6 % #
  \catcode39=12 % '
  \catcode44=12 % ,
  \catcode45=12 % -
  \catcode46=12 % .
  \catcode58=12 % :
  \catcode64=11 % @
  \catcode123=1 % {
  \catcode125=2 % }
  \expandafter\let\expandafter\x\csname ver@catchfile.sty\endcsname
  \ifx\x\relax % plain-TeX, first loading
  \else
    \def\empty{}%
    \ifx\x\empty % LaTeX, first loading,
      % variable is initialized, but \ProvidesPackage not yet seen
    \else
      \expandafter\ifx\csname PackageInfo\endcsname\relax
        \def\x#1#2{%
          \immediate\write-1{Package #1 Info: #2.}%
        }%
      \else
        \def\x#1#2{\PackageInfo{#1}{#2, stopped}}%
      \fi
      \x{catchfile}{The package is already loaded}%
      \aftergroup\endinput
    \fi
  \fi
\endgroup%
%    \end{macrocode}
%    Package identification:
%    \begin{macrocode}
\begingroup\catcode61\catcode48\catcode32=10\relax%
  \catcode13=5 % ^^M
  \endlinechar=13 %
  \catcode35=6 % #
  \catcode39=12 % '
  \catcode40=12 % (
  \catcode41=12 % )
  \catcode44=12 % ,
  \catcode45=12 % -
  \catcode46=12 % .
  \catcode47=12 % /
  \catcode58=12 % :
  \catcode64=11 % @
  \catcode91=12 % [
  \catcode93=12 % ]
  \catcode123=1 % {
  \catcode125=2 % }
  \expandafter\ifx\csname ProvidesPackage\endcsname\relax
    \def\x#1#2#3[#4]{\endgroup
      \immediate\write-1{Package: #3 #4}%
      \xdef#1{#4}%
    }%
  \else
    \def\x#1#2[#3]{\endgroup
      #2[{#3}]%
      \ifx#1\@undefined
        \xdef#1{#3}%
      \fi
      \ifx#1\relax
        \xdef#1{#3}%
      \fi
    }%
  \fi
\expandafter\x\csname ver@catchfile.sty\endcsname
\ProvidesPackage{catchfile}%
  [2016/05/16 v1.7 Catch the contents of a file (HO)]%
%    \end{macrocode}
%
% \subsection{Catcodes}
%
%    \begin{macrocode}
\begingroup\catcode61\catcode48\catcode32=10\relax%
  \catcode13=5 % ^^M
  \endlinechar=13 %
  \catcode123=1 % {
  \catcode125=2 % }
  \catcode64=11 % @
  \def\x{\endgroup
    \expandafter\edef\csname CatchFile@AtEnd\endcsname{%
      \endlinechar=\the\endlinechar\relax
      \catcode13=\the\catcode13\relax
      \catcode32=\the\catcode32\relax
      \catcode35=\the\catcode35\relax
      \catcode61=\the\catcode61\relax
      \catcode64=\the\catcode64\relax
      \catcode123=\the\catcode123\relax
      \catcode125=\the\catcode125\relax
    }%
  }%
\x\catcode61\catcode48\catcode32=10\relax%
\catcode13=5 % ^^M
\endlinechar=13 %
\catcode35=6 % #
\catcode64=11 % @
\catcode123=1 % {
\catcode125=2 % }
\def\TMP@EnsureCode#1#2{%
  \edef\CatchFile@AtEnd{%
    \CatchFile@AtEnd
    \catcode#1=\the\catcode#1\relax
  }%
  \catcode#1=#2\relax
}
\TMP@EnsureCode{39}{12}% '
\TMP@EnsureCode{44}{12}% ,
\TMP@EnsureCode{45}{12}% -
\TMP@EnsureCode{46}{12}% .
\TMP@EnsureCode{47}{12}% /
\TMP@EnsureCode{91}{12}% [
\TMP@EnsureCode{93}{12}% ]
\TMP@EnsureCode{96}{12}% `
\edef\CatchFile@AtEnd{\CatchFile@AtEnd\noexpand\endinput}
%    \end{macrocode}
%
% \subsection{Preparations}
%
%    \begin{macrocode}
\begingroup\expandafter\expandafter\expandafter\endgroup
\expandafter\ifx\csname RequirePackage\endcsname\relax
  \input infwarerr.sty\relax
  \input ltxcmds.sty\relax
\else
  \RequirePackage{infwarerr}[2007/09/09]%
  \RequirePackage{ltxcmds}[2010/03/09]%
\fi
%    \end{macrocode}
%
%    Check for \eTeX's \cs{everyeof}.
%    \begin{macrocode}
\begingroup
  \escapechar=92\relax
  \edef\TestString{\string\everyeof}%
  \edef\TestMeaning{\meaning\everyeof}%
  \ifx\TestString\TestMeaning
  \else
    \@PackageError{catchfile}{%
      Cannot find e-TeX's \string\everyeof,\MessageBreak
      package loading is aborted%
    }\@ehd
    \endgroup
    \expandafter\CatchFile@AtEnd
  \fi%
\endgroup
%    \end{macrocode}
%
% \subsection{Looking for primitive \cs{input}}
%
%    \begin{macro}{\CatchFile@Input}
%    The package needs the expandable primitive \cs{input}.
%    However there are formats that redefine it. For example,
%    \LaTeX's \cs{input} is not expandable, but it stores
%    the primitive in \cs{@@input}. The third possibility
%    is \cs{pdfprimitive}, introduced in \pdfTeX\ 1.40.0.
%
%    Thus we try to find the primitive and store it in
%    \cs{CatchFile@Input}. If it is used, it must be expanded
%    twice (because of the solution with \cs{pdfprimitive}).
%    \begin{macrocode}
\begingroup
  \def\Check#1#2#3#4\endgroup{%
    \edef\TestString{\string#1}%
    \edef\TestMeaning{\meaning#2}%
    \ifx\TestString\TestMeaning
      \endgroup
      \let\CatchFile@Primitive#2%
      \def\CatchFile@Input{\CatchFile@Primitive#3}%
    \else
      #4\endgroup
    \fi
  }%
  \Check\input\input{}%
  \Check\input\@@input{}%
  \Check\pdfprimitive\pdfprimitive\input
  \@PackageError{%
    Cannot find primitive \string\input,\MessageBreak
    package loading is aborted%
  }\@ehd
  \csname endgroup\endcsname
  \CatchFile@AtEnd%
\endgroup
%    \end{macrocode}
%    \end{macro}
%
% \subsection{Input file check}
%
%    \begin{macro}{\CatchFile@CheckFileExists}
%    \begin{macrocode}
\begingroup\expandafter\expandafter\expandafter\endgroup
\expandafter\ifx\csname IfFileExists\endcsname\relax
  \input pdftexcmds.sty\relax
  \begingroup\expandafter\expandafter\expandafter\endgroup
  \expandafter\ifx\csname pdf@filesize\endcsname\relax
    \def\CatchFile@CheckFileExists#1{%
      \expandafter\ifx\csname @inputcheck\endcsname\relax
        \csname newread\endcsname\@inputcheck
      \fi
      \openin\@inputcheck#1\relax
      \expandafter\closein\expandafter\@inputcheck
      \ifeof\@inputcheck
        \let\CatchFile@File\relax
      \else
        \def\CatchFile@File{#1}%
      \fi
    }%
  \else
    \def\CatchFile@CheckFileExists#1{%
      \expandafter\expandafter\expandafter\ifx
      \expandafter\expandafter\expandafter\relax\pdf@filesize{#1}\relax
        \let\CatchFile@File\relax
      \else
        \def\CatchFile@File{#1}%
      \fi
    }%
  \fi
\else
  \def\CatchFile@CheckFileExists#1{%
    \IfFileExists{#1}{%
      \expandafter\CatchFile@DefFile\@filef@und\@nil
      \begingroup\expandafter\expandafter\expandafter\endgroup
      \expandafter\ifx\csname @addtofilelist\endcsname\relax
      \else
        \@addtofilelist\CatchFile@File
      \fi
    }{%
      \let\CatchFile@File\relax
    }%
  }%
  \def\CatchFile@DefFile#1 \@nil{%
    \def\CatchFile@File{#1}%
  }%
\fi
%    \end{macrocode}
%    \end{macro}
%
%    \begin{macro}{\CatchFile@NotFound}
%    \begin{macrocode}
\def\CatchFile@NotFound#1#2{%
  \def#1{}%
  \@PackageError{catchfile}{%
    File `#2' not found%
  }\@ehc
}
%    \end{macrocode}
%    \end{macro}
%
% \subsection{Catch file contents}
%
%    \begin{macrocode}
\ltx@IfUndefined{RequirePackage}{%
  \input etexcmds.sty\relax
}{%
  \RequirePackage{etexcmds}[2010/01/28]%
}
%    \end{macrocode}
%
%    \begin{macro}{\CatchFileEdef}
%    \begin{macrocode}
\long\def\CatchFileEdef#1#2#3{%
  \CatchFile@CheckFileExists{#2}%
  \ifx\CatchFile@File\relax
    \CatchFile@NotFound{#1}{#2}%
  \else
    \begingroup
      \everyeof{\noexpand}%
      #3%
      \xdef\CatchFile@Contents{\CatchFile@Input\CatchFile@File\space}%
    \endgroup
    \let#1\CatchFile@Contents
  \fi
}
%    \end{macrocode}
%    \end{macro}
%
%    \begin{macro}{\CatchFileDef}
%    \begin{macrocode}
\long\def\CatchFileDef#1#2#3{%
  \CatchFile@CheckFileExists{#2}%
  \ifx\CatchFile@File\relax
    \CatchFile@NotFound{#1}{#2}%
  \else
    \begingroup
      \everyeof\expandafter{%
        \CatchFile@EOF
        \expandafter\CatchFile@Finish
        \noexpand
      }%
      \ltx@ifundefined{etex@unexpanded}{%
        \expandafter\long\expandafter\def\expandafter\CatchFile@Do
            \expandafter##\expandafter1\CatchFile@EOF{%
          \toks\ltx@zero{##1}%
          \xdef\CatchFile@gtemp{\the\toks\ltx@zero}%
          \def\CatchFile@Finish{%
            \endgroup
            \let#1\CatchFile@gtemp
            \global\let\CatchFile@gtemp\ltx@undefined
          }%
        }%
      }{%
        \expandafter\long\expandafter\def\expandafter\CatchFile@Do
            \expandafter##\expandafter1\CatchFile@EOF{%
          \edef\CatchFile@Finish{%
            \endgroup
            \etex@unexpanded{%
              \edef#1{\etex@unexpanded{##1}}%
            }%
          }%
        }%
      }%
      #3\relax
    \expandafter\expandafter\expandafter\CatchFile@Do
    \CatchFile@Input\CatchFile@File\relax
  \fi
}
%    \end{macrocode}
%    \cs{relax} after |#3| was added to make it more robust in
%    case the user uses something like
%\begin{verbatim}
%\CatchFileDef{\content}{\jobname.tt}{\endlinechar=-1}
%\end{verbatim}
%    that expands the following \cs{expandafter} after |#3| prematurely
%    (contribution of Martin Scharrer).
%    \end{macro}
%    \begin{macro}{\CatchFile@EOF}
%    If the file is read the catcode mappings are fixed.
%    This means that the same character cannot occur
%    inside the file with different catcodes.
%    Thus we use as end of file marker the at sign
%    twice with different catcodes.
%    \begin{macrocode}
\begingroup
  \lccode65=64 % lowercase('A') = '@'
  \lccode66=64 % lowercase('B') = '@'
  \catcode65=8 % catcode('A') = subscript
  \catcode66=3 % catcode('B') = math shift
\lowercase{\endgroup
  \def\CatchFile@EOF{AB}%
}
%    \end{macrocode}
%    \end{macro}
%
%    \begin{macrocode}
\CatchFile@AtEnd%
%</package>
%    \end{macrocode}
%
% \section{Test}
%
% \subsection{Catcode checks for loading}
%
%    \begin{macrocode}
%<*test1>
%    \end{macrocode}
%    \begin{macrocode}
\catcode`\{=1 %
\catcode`\}=2 %
\catcode`\#=6 %
\catcode`\@=11 %
\expandafter\ifx\csname count@\endcsname\relax
  \countdef\count@=255 %
\fi
\expandafter\ifx\csname @gobble\endcsname\relax
  \long\def\@gobble#1{}%
\fi
\expandafter\ifx\csname @firstofone\endcsname\relax
  \long\def\@firstofone#1{#1}%
\fi
\expandafter\ifx\csname loop\endcsname\relax
  \expandafter\@firstofone
\else
  \expandafter\@gobble
\fi
{%
  \def\loop#1\repeat{%
    \def\body{#1}%
    \iterate
  }%
  \def\iterate{%
    \body
      \let\next\iterate
    \else
      \let\next\relax
    \fi
    \next
  }%
  \let\repeat=\fi
}%
\def\RestoreCatcodes{}
\count@=0 %
\loop
  \edef\RestoreCatcodes{%
    \RestoreCatcodes
    \catcode\the\count@=\the\catcode\count@\relax
  }%
\ifnum\count@<255 %
  \advance\count@ 1 %
\repeat

\def\RangeCatcodeInvalid#1#2{%
  \count@=#1\relax
  \loop
    \catcode\count@=15 %
  \ifnum\count@<#2\relax
    \advance\count@ 1 %
  \repeat
}
\def\RangeCatcodeCheck#1#2#3{%
  \count@=#1\relax
  \loop
    \ifnum#3=\catcode\count@
    \else
      \errmessage{%
        Character \the\count@\space
        with wrong catcode \the\catcode\count@\space
        instead of \number#3%
      }%
    \fi
  \ifnum\count@<#2\relax
    \advance\count@ 1 %
  \repeat
}
\def\space{ }
\expandafter\ifx\csname LoadCommand\endcsname\relax
  \def\LoadCommand{\input catchfile.sty\relax}%
\fi
\def\Test{%
  \RangeCatcodeInvalid{0}{47}%
  \RangeCatcodeInvalid{58}{64}%
  \RangeCatcodeInvalid{91}{96}%
  \RangeCatcodeInvalid{123}{255}%
  \catcode`\@=12 %
  \catcode`\\=0 %
  \catcode`\%=14 %
  \LoadCommand
  \RangeCatcodeCheck{0}{36}{15}%
  \RangeCatcodeCheck{37}{37}{14}%
  \RangeCatcodeCheck{38}{47}{15}%
  \RangeCatcodeCheck{48}{57}{12}%
  \RangeCatcodeCheck{58}{63}{15}%
  \RangeCatcodeCheck{64}{64}{12}%
  \RangeCatcodeCheck{65}{90}{11}%
  \RangeCatcodeCheck{91}{91}{15}%
  \RangeCatcodeCheck{92}{92}{0}%
  \RangeCatcodeCheck{93}{96}{15}%
  \RangeCatcodeCheck{97}{122}{11}%
  \RangeCatcodeCheck{123}{255}{15}%
  \RestoreCatcodes
}
\Test
\csname @@end\endcsname
\end
%    \end{macrocode}
%    \begin{macrocode}
%</test1>
%    \end{macrocode}
%
% \subsection{\LaTeX}
%
%    \begin{macrocode}
%<*test2>
\NeedsTeXFormat{LaTeX2e}
\nofiles
\listfiles
\tracingnesting=2 %
\documentclass{minimal}
\usepackage{catchfile}[2016/05/16]
\makeatletter
\def\mysetup{%
  \let\do\@makeother
  \dospecials
}
\def\CheckContents{%
  \begingroup
    \expandafter\ifx\csname contents\endcsname\relax
      \PackageError{TEST}{\string\contents\space is not defined}\@ehc
    \fi
  \endgroup
}
\def\StartDisableUnexpanded{%
  \begingroup
    \let\etex@unexpanded\@undefined
}
\def\StopDisableUnexpanded{%
  \endgroup
}
\def\CheckCleanup{%
  \begingroup
    \edef\x{\the\toks0}%
    \def\y{ABC}%
    \ifx\x\y
    \else
      \PackageError{TEST}{\string\toks0 has changed}\@ehc
    \fi
    \ifx\CatchFile@gtemp\@undefined
    \else
      \PackageError{TEST}{Left over \string\CatchFile@gtemp}\@ehc
    \fi
  \endgroup
}
\def\Check{%
  \CheckContents
  \CheckCleanup
}
\makeatother
\begin{document}
\toks0{ABC}

\CatchFileDef\contents{catchfile.sty}\mysetup
\typeout{\meaning\contents}
\Check
\typeout{***********************************************************}%
\CatchFileDef\contents{catchfile.sty}{}
\typeout{\meaning\contents}
\Check
\typeout{***********************************************************}%

\StartDisableUnexpanded
\CatchFileDef\contents{catchfile.sty}\mysetup
\typeout{\meaning\contents}
\Check
\typeout{***********************************************************}%
\CatchFileDef\contents{catchfile.sty}{}
\typeout{\meaning\contents}
\Check
\typeout{***********************************************************}%
\StopDisableUnexpanded

\CatchFileEdef\contents{catchfile.sty}{%
  \mysetup
  \def\par{^^J}%
  \obeylines
}
\typeout{\contents}
\Check
\typeout{***********************************************************}%
\CatchFileEdef\contents{catchfile.sty}{%
  \catcode`\\=12 %
  \catcode`\#=12 %
  \def\par{^^J}%
  \obeylines
}
\typeout{\contents}
\Check
\typeout{***********************************************************}%
\end{document}
%</test2>
%    \end{macrocode}
%
% \subsection{\plainTeX}
%
%    \begin{macrocode}
%<*test3>
\def\msg#{\immediate\write16}
\newlinechar=10 %
\tracingnesting=2 %
\input catchfile.sty\relax

\def\mysetup{%
  \def\do##1{%
    \catcode`##1=12\relax
  }%
  \dospecials
}
\def\CheckContents{%
  \begingroup
    \expandafter\ifx\csname contents\endcsname\relax
      \def\space{ }%
      \errmessage{\string\contents\space is not defined.}%
    \fi
  \endgroup
}
\catcode`\@=11 %
\def\CheckCleanup{%
  \begingroup
    \edef\x{\the\toks0}%
    \def\y{ABC}%
    \ifx\x\y
    \else
      \errmessage{\string\toks0 has changed}%
    \fi
    \ifx\CatchFile@gtemp\@undefined
    \else
      \errmessage{Left over \string\CatchFile@gtemp}%
    \fi
  \endgroup
}
\def\Check{%
  \CheckContents
  \CheckCleanup
}
\def\StartDisableUnexpanded{%
  \begingroup
    \let\etex@unexpanded\@undefined
}
\def\StopDisableUnexpanded{%
  \endgroup
}
\catcode`\@=12 %

\toks0{ABC}

\CatchFileDef\contents{catchfile.sty}\mysetup
\msg{\meaning\contents}
\Check
\msg{***********************************************************}%
\CatchFileDef\contents{catchfile.sty}{}
\msg{\meaning\contents}
\Check
\msg{***********************************************************}%

\StartDisableUnexpanded
\CatchFileDef\contents{catchfile.sty}\mysetup
\msg{\meaning\contents}
\Check
\msg{***********************************************************}%
\CatchFileDef\contents{catchfile.sty}{}
\msg{\meaning\contents}
\Check
\msg{***********************************************************}%
\StopDisableUnexpanded

\CatchFileEdef\contents{catchfile.sty}{%
  \mysetup
  \def\par{^^J}%
  \obeylines
}
\msg{\contents}
\Check
\msg{***********************************************************}%
\CatchFileEdef\contents{catchfile.sty}{%
  \catcode`\\=12 %
  \catcode`\#=12 %
  \def\par{^^J}%
  \obeylines
}
\msg{\contents}
\Check
\msg{***********************************************************}%

\csname @@end\endcsname
\end
%</test3>
%    \end{macrocode}
%
% \section{Installation}
%
% \subsection{Download}
%
% \paragraph{Package.} This package is available on
% CTAN\footnote{\url{https://ctan.org/pkg/catchfile}}:
% \begin{description}
% \item[\CTAN{macros/latex/contrib/oberdiek/catchfile.dtx}] The source file.
% \item[\CTAN{macros/latex/contrib/oberdiek/catchfile.pdf}] Documentation.
% \end{description}
%
%
% \paragraph{Bundle.} All the packages of the bundle `oberdiek'
% are also available in a TDS compliant ZIP archive. There
% the packages are already unpacked and the documentation files
% are generated. The files and directories obey the TDS standard.
% \begin{description}
% \item[\CTANinstall{install/macros/latex/contrib/oberdiek.tds.zip}]
% \end{description}
% \emph{TDS} refers to the standard ``A Directory Structure
% for \TeX\ Files'' (\CTAN{tds/tds.pdf}). Directories
% with \xfile{texmf} in their name are usually organized this way.
%
% \subsection{Bundle installation}
%
% \paragraph{Unpacking.} Unpack the \xfile{oberdiek.tds.zip} in the
% TDS tree (also known as \xfile{texmf} tree) of your choice.
% Example (linux):
% \begin{quote}
%   |unzip oberdiek.tds.zip -d ~/texmf|
% \end{quote}
%
% \paragraph{Script installation.}
% Check the directory \xfile{TDS:scripts/oberdiek/} for
% scripts that need further installation steps.
% Package \xpackage{attachfile2} comes with the Perl script
% \xfile{pdfatfi.pl} that should be installed in such a way
% that it can be called as \texttt{pdfatfi}.
% Example (linux):
% \begin{quote}
%   |chmod +x scripts/oberdiek/pdfatfi.pl|\\
%   |cp scripts/oberdiek/pdfatfi.pl /usr/local/bin/|
% \end{quote}
%
% \subsection{Package installation}
%
% \paragraph{Unpacking.} The \xfile{.dtx} file is a self-extracting
% \docstrip\ archive. The files are extracted by running the
% \xfile{.dtx} through \plainTeX:
% \begin{quote}
%   \verb|tex catchfile.dtx|
% \end{quote}
%
% \paragraph{TDS.} Now the different files must be moved into
% the different directories in your installation TDS tree
% (also known as \xfile{texmf} tree):
% \begin{quote}
% \def\t{^^A
% \begin{tabular}{@{}>{\ttfamily}l@{ $\rightarrow$ }>{\ttfamily}l@{}}
%   catchfile.sty & tex/generic/oberdiek/catchfile.sty\\
%   catchfile.pdf & doc/latex/oberdiek/catchfile.pdf\\
%   test/catchfile-test1.tex & doc/latex/oberdiek/test/catchfile-test1.tex\\
%   test/catchfile-test2.tex & doc/latex/oberdiek/test/catchfile-test2.tex\\
%   test/catchfile-test3.tex & doc/latex/oberdiek/test/catchfile-test3.tex\\
%   catchfile.dtx & source/latex/oberdiek/catchfile.dtx\\
% \end{tabular}^^A
% }^^A
% \sbox0{\t}^^A
% \ifdim\wd0>\linewidth
%   \begingroup
%     \advance\linewidth by\leftmargin
%     \advance\linewidth by\rightmargin
%   \edef\x{\endgroup
%     \def\noexpand\lw{\the\linewidth}^^A
%   }\x
%   \def\lwbox{^^A
%     \leavevmode
%     \hbox to \linewidth{^^A
%       \kern-\leftmargin\relax
%       \hss
%       \usebox0
%       \hss
%       \kern-\rightmargin\relax
%     }^^A
%   }^^A
%   \ifdim\wd0>\lw
%     \sbox0{\small\t}^^A
%     \ifdim\wd0>\linewidth
%       \ifdim\wd0>\lw
%         \sbox0{\footnotesize\t}^^A
%         \ifdim\wd0>\linewidth
%           \ifdim\wd0>\lw
%             \sbox0{\scriptsize\t}^^A
%             \ifdim\wd0>\linewidth
%               \ifdim\wd0>\lw
%                 \sbox0{\tiny\t}^^A
%                 \ifdim\wd0>\linewidth
%                   \lwbox
%                 \else
%                   \usebox0
%                 \fi
%               \else
%                 \lwbox
%               \fi
%             \else
%               \usebox0
%             \fi
%           \else
%             \lwbox
%           \fi
%         \else
%           \usebox0
%         \fi
%       \else
%         \lwbox
%       \fi
%     \else
%       \usebox0
%     \fi
%   \else
%     \lwbox
%   \fi
% \else
%   \usebox0
% \fi
% \end{quote}
% If you have a \xfile{docstrip.cfg} that configures and enables \docstrip's
% TDS installing feature, then some files can already be in the right
% place, see the documentation of \docstrip.
%
% \subsection{Refresh file name databases}
%
% If your \TeX~distribution
% (\teTeX, \mikTeX, \dots) relies on file name databases, you must refresh
% these. For example, \teTeX\ users run \verb|texhash| or
% \verb|mktexlsr|.
%
% \subsection{Some details for the interested}
%
% \paragraph{Attached source.}
%
% The PDF documentation on CTAN also includes the
% \xfile{.dtx} source file. It can be extracted by
% AcrobatReader 6 or higher. Another option is \textsf{pdftk},
% e.g. unpack the file into the current directory:
% \begin{quote}
%   \verb|pdftk catchfile.pdf unpack_files output .|
% \end{quote}
%
% \paragraph{Unpacking with \LaTeX.}
% The \xfile{.dtx} chooses its action depending on the format:
% \begin{description}
% \item[\plainTeX:] Run \docstrip\ and extract the files.
% \item[\LaTeX:] Generate the documentation.
% \end{description}
% If you insist on using \LaTeX\ for \docstrip\ (really,
% \docstrip\ does not need \LaTeX), then inform the autodetect routine
% about your intention:
% \begin{quote}
%   \verb|latex \let\install=y\input{catchfile.dtx}|
% \end{quote}
% Do not forget to quote the argument according to the demands
% of your shell.
%
% \paragraph{Generating the documentation.}
% You can use both the \xfile{.dtx} or the \xfile{.drv} to generate
% the documentation. The process can be configured by the
% configuration file \xfile{ltxdoc.cfg}. For instance, put this
% line into this file, if you want to have A4 as paper format:
% \begin{quote}
%   \verb|\PassOptionsToClass{a4paper}{article}|
% \end{quote}
% An example follows how to generate the
% documentation with pdf\LaTeX:
% \begin{quote}
%\begin{verbatim}
%pdflatex catchfile.dtx
%makeindex -s gind.ist catchfile.idx
%pdflatex catchfile.dtx
%makeindex -s gind.ist catchfile.idx
%pdflatex catchfile.dtx
%\end{verbatim}
% \end{quote}
%
% \begin{History}
%   \begin{Version}{2007/05/30 v1.0}
%   \item
%     First version.
%   \end{Version}
%   \begin{Version}{2007/09/09 v1.1}
%   \item
%     Catcode section rewritten.
%   \end{Version}
%   \begin{Version}{2007/11/11 v1.2}
%   \item
%     Use of package \xpackage{pdftexcmds} for \LuaTeX\ support.
%   \end{Version}
%   \begin{Version}{2010/03/01 v1.3}
%   \item
%     Fix for unknown \cs{@PackageErrorNoLine}.
%   \end{Version}
%   \begin{Version}{2010/04/08 v1.4}
%   \item
%     \cs{closein} also added if \cs{ifeof} is true.
%   \end{Version}
%   \begin{Version}{2010/04/28 v1.5}
%   \item
%     \cs{CatchFileDef}: Getting rid of warning `end of semi simple group
%     entered at line \dots\ of
%     a different file (Florent Chervet).
%   \item
%     \cs{CatchFileDef}: Fix for error `Illegal parameter number in
%     definition of \dots' (HO) including improvement that uses
%     \cs{unexpanded} if available (Florent Chervet).
%   \end{Version}
%   \begin{Version}{2011/03/01 v1.6}
%   \item
%     \cs{relax} added after the setup argument of
%     \cs{CatchFileDef} to prevent premature file reading
%     (Martin Scharrer).
%   \end{Version}
%   \begin{Version}{2016/05/16 v1.7}
%   \item
%     Documentation updates.
%   \end{Version}
% \end{History}
%
% \PrintIndex
%
% \Finale
\endinput
|
% \end{quote}
% Do not forget to quote the argument according to the demands
% of your shell.
%
% \paragraph{Generating the documentation.}
% You can use both the \xfile{.dtx} or the \xfile{.drv} to generate
% the documentation. The process can be configured by the
% configuration file \xfile{ltxdoc.cfg}. For instance, put this
% line into this file, if you want to have A4 as paper format:
% \begin{quote}
%   \verb|\PassOptionsToClass{a4paper}{article}|
% \end{quote}
% An example follows how to generate the
% documentation with pdf\LaTeX:
% \begin{quote}
%\begin{verbatim}
%pdflatex catchfile.dtx
%makeindex -s gind.ist catchfile.idx
%pdflatex catchfile.dtx
%makeindex -s gind.ist catchfile.idx
%pdflatex catchfile.dtx
%\end{verbatim}
% \end{quote}
%
% \begin{History}
%   \begin{Version}{2007/05/30 v1.0}
%   \item
%     First version.
%   \end{Version}
%   \begin{Version}{2007/09/09 v1.1}
%   \item
%     Catcode section rewritten.
%   \end{Version}
%   \begin{Version}{2007/11/11 v1.2}
%   \item
%     Use of package \xpackage{pdftexcmds} for \LuaTeX\ support.
%   \end{Version}
%   \begin{Version}{2010/03/01 v1.3}
%   \item
%     Fix for unknown \cs{@PackageErrorNoLine}.
%   \end{Version}
%   \begin{Version}{2010/04/08 v1.4}
%   \item
%     \cs{closein} also added if \cs{ifeof} is true.
%   \end{Version}
%   \begin{Version}{2010/04/28 v1.5}
%   \item
%     \cs{CatchFileDef}: Getting rid of warning `end of semi simple group
%     entered at line \dots\ of
%     a different file (Florent Chervet).
%   \item
%     \cs{CatchFileDef}: Fix for error `Illegal parameter number in
%     definition of \dots' (HO) including improvement that uses
%     \cs{unexpanded} if available (Florent Chervet).
%   \end{Version}
%   \begin{Version}{2011/03/01 v1.6}
%   \item
%     \cs{relax} added after the setup argument of
%     \cs{CatchFileDef} to prevent premature file reading
%     (Martin Scharrer).
%   \end{Version}
%   \begin{Version}{2016/05/16 v1.7}
%   \item
%     Documentation updates.
%   \end{Version}
% \end{History}
%
% \PrintIndex
%
% \Finale
\endinput

%        (quote the arguments according to the demands of your shell)
%
% Documentation:
%    (a) If catchfile.drv is present:
%           latex catchfile.drv
%    (b) Without catchfile.drv:
%           latex catchfile.dtx; ...
%    The class ltxdoc loads the configuration file ltxdoc.cfg
%    if available. Here you can specify further options, e.g.
%    use A4 as paper format:
%       \PassOptionsToClass{a4paper}{article}
%
%    Programm calls to get the documentation (example):
%       pdflatex catchfile.dtx
%       makeindex -s gind.ist catchfile.idx
%       pdflatex catchfile.dtx
%       makeindex -s gind.ist catchfile.idx
%       pdflatex catchfile.dtx
%
% Installation:
%    TDS:tex/generic/oberdiek/catchfile.sty
%    TDS:doc/latex/oberdiek/catchfile.pdf
%    TDS:doc/latex/oberdiek/test/catchfile-test1.tex
%    TDS:doc/latex/oberdiek/test/catchfile-test2.tex
%    TDS:doc/latex/oberdiek/test/catchfile-test3.tex
%    TDS:source/latex/oberdiek/catchfile.dtx
%
%<*ignore>
\begingroup
  \catcode123=1 %
  \catcode125=2 %
  \def\x{LaTeX2e}%
\expandafter\endgroup
\ifcase 0\ifx\install y1\fi\expandafter
         \ifx\csname processbatchFile\endcsname\relax\else1\fi
         \ifx\fmtname\x\else 1\fi\relax
\else\csname fi\endcsname
%</ignore>
%<*install>
\input docstrip.tex
\Msg{************************************************************************}
\Msg{* Installation}
\Msg{* Package: catchfile 2016/05/16 v1.7 Catch the contents of a file (HO)}
\Msg{************************************************************************}

\keepsilent
\askforoverwritefalse

\let\MetaPrefix\relax
\preamble

This is a generated file.

Project: catchfile
Version: 2016/05/16 v1.7

Copyright (C) 2007, 2010, 2011 by
   Heiko Oberdiek <heiko.oberdiek at googlemail.com>

This work may be distributed and/or modified under the
conditions of the LaTeX Project Public License, either
version 1.3c of this license or (at your option) any later
version. This version of this license is in
   http://www.latex-project.org/lppl/lppl-1-3c.txt
and the latest version of this license is in
   http://www.latex-project.org/lppl.txt
and version 1.3 or later is part of all distributions of
LaTeX version 2005/12/01 or later.

This work has the LPPL maintenance status "maintained".

This Current Maintainer of this work is Heiko Oberdiek.

The Base Interpreter refers to any `TeX-Format',
because some files are installed in TDS:tex/generic//.

This work consists of the main source file catchfile.dtx
and the derived files
   catchfile.sty, catchfile.pdf, catchfile.ins, catchfile.drv,
   catchfile-test1.tex, catchfile-test2.tex,
   catchfile-test3.tex.

\endpreamble
\let\MetaPrefix\DoubleperCent

\generate{%
  \file{catchfile.ins}{\from{catchfile.dtx}{install}}%
  \file{catchfile.drv}{\from{catchfile.dtx}{driver}}%
  \usedir{tex/generic/oberdiek}%
  \file{catchfile.sty}{\from{catchfile.dtx}{package}}%
%  \usedir{doc/latex/oberdiek/test}%
%  \file{catchfile-test1.tex}{\from{catchfile.dtx}{test1}}%
%  \file{catchfile-test2.tex}{\from{catchfile.dtx}{test2}}%
%  \file{catchfile-test3.tex}{\from{catchfile.dtx}{test3}}%
  \nopreamble
  \nopostamble
%  \usedir{source/latex/oberdiek/catalogue}%
%  \file{catchfile.xml}{\from{catchfile.dtx}{catalogue}}%
}

\catcode32=13\relax% active space
\let =\space%
\Msg{************************************************************************}
\Msg{*}
\Msg{* To finish the installation you have to move the following}
\Msg{* file into a directory searched by TeX:}
\Msg{*}
\Msg{*     catchfile.sty}
\Msg{*}
\Msg{* To produce the documentation run the file `catchfile.drv'}
\Msg{* through LaTeX.}
\Msg{*}
\Msg{* Happy TeXing!}
\Msg{*}
\Msg{************************************************************************}

\endbatchfile
%</install>
%<*ignore>
\fi
%</ignore>
%<*driver>
\NeedsTeXFormat{LaTeX2e}
\ProvidesFile{catchfile.drv}%
  [2016/05/16 v1.7 Catch the contents of a file (HO)]%
\documentclass{ltxdoc}
\usepackage{holtxdoc}[2011/11/22]
\begin{document}
  \DocInput{catchfile.dtx}%
\end{document}
%</driver>
% \fi
%
%
% \CharacterTable
%  {Upper-case    \A\B\C\D\E\F\G\H\I\J\K\L\M\N\O\P\Q\R\S\T\U\V\W\X\Y\Z
%   Lower-case    \a\b\c\d\e\f\g\h\i\j\k\l\m\n\o\p\q\r\s\t\u\v\w\x\y\z
%   Digits        \0\1\2\3\4\5\6\7\8\9
%   Exclamation   \!     Double quote  \"     Hash (number) \#
%   Dollar        \$     Percent       \%     Ampersand     \&
%   Acute accent  \'     Left paren    \(     Right paren   \)
%   Asterisk      \*     Plus          \+     Comma         \,
%   Minus         \-     Point         \.     Solidus       \/
%   Colon         \:     Semicolon     \;     Less than     \<
%   Equals        \=     Greater than  \>     Question mark \?
%   Commercial at \@     Left bracket  \[     Backslash     \\
%   Right bracket \]     Circumflex    \^     Underscore    \_
%   Grave accent  \`     Left brace    \{     Vertical bar  \|
%   Right brace   \}     Tilde         \~}
%
% \GetFileInfo{catchfile.drv}
%
% \title{The \xpackage{catchfile} package}
% \date{2016/05/16 v1.7}
% \author{Heiko Oberdiek\thanks
% {Please report any issues at https://github.com/ho-tex/oberdiek/issues}\\
% \xemail{heiko.oberdiek at googlemail.com}}
%
% \maketitle
%
% \begin{abstract}
% This package catches the contents of a file and puts it in a macro.
% It requires \eTeX. Both \LaTeX\ and \plainTeX\ are supported.
% \end{abstract}
%
% \tableofcontents
%
% \section{Documentation}
%
% The package relies on \eTeX's \cs{everyeof}. Otherwise it
% aborts with an error message.
%
% \begin{declcs}{CatchFileDef} \M{cmd} \M{file name} \M{setup}\\
% \cs{CatchFileEdef} \M{cmd} \M{file name} \M{setup}
% \end{declcs}
% Macro \meta{cmd} is defined with the contents
% of file \meta{file name}.
% \cs{CatchFileDef} uses \cs{def}, \cs{CatchFileEdef} \cs{edef}
% for the definition.
% Additional setup code for setting
% catcodes or treatment of line ends can be given in code
% \meta{setup}. See the test files for an example.
%
% \StopEventually{
% }
%
% \section{Implementation}
%
%    \begin{macrocode}
%<*package>
%    \end{macrocode}
%
% \subsection{Reload check and package identification}
%    Reload check, especially if the package is not used with \LaTeX.
%    \begin{macrocode}
\begingroup\catcode61\catcode48\catcode32=10\relax%
  \catcode13=5 % ^^M
  \endlinechar=13 %
  \catcode35=6 % #
  \catcode39=12 % '
  \catcode44=12 % ,
  \catcode45=12 % -
  \catcode46=12 % .
  \catcode58=12 % :
  \catcode64=11 % @
  \catcode123=1 % {
  \catcode125=2 % }
  \expandafter\let\expandafter\x\csname ver@catchfile.sty\endcsname
  \ifx\x\relax % plain-TeX, first loading
  \else
    \def\empty{}%
    \ifx\x\empty % LaTeX, first loading,
      % variable is initialized, but \ProvidesPackage not yet seen
    \else
      \expandafter\ifx\csname PackageInfo\endcsname\relax
        \def\x#1#2{%
          \immediate\write-1{Package #1 Info: #2.}%
        }%
      \else
        \def\x#1#2{\PackageInfo{#1}{#2, stopped}}%
      \fi
      \x{catchfile}{The package is already loaded}%
      \aftergroup\endinput
    \fi
  \fi
\endgroup%
%    \end{macrocode}
%    Package identification:
%    \begin{macrocode}
\begingroup\catcode61\catcode48\catcode32=10\relax%
  \catcode13=5 % ^^M
  \endlinechar=13 %
  \catcode35=6 % #
  \catcode39=12 % '
  \catcode40=12 % (
  \catcode41=12 % )
  \catcode44=12 % ,
  \catcode45=12 % -
  \catcode46=12 % .
  \catcode47=12 % /
  \catcode58=12 % :
  \catcode64=11 % @
  \catcode91=12 % [
  \catcode93=12 % ]
  \catcode123=1 % {
  \catcode125=2 % }
  \expandafter\ifx\csname ProvidesPackage\endcsname\relax
    \def\x#1#2#3[#4]{\endgroup
      \immediate\write-1{Package: #3 #4}%
      \xdef#1{#4}%
    }%
  \else
    \def\x#1#2[#3]{\endgroup
      #2[{#3}]%
      \ifx#1\@undefined
        \xdef#1{#3}%
      \fi
      \ifx#1\relax
        \xdef#1{#3}%
      \fi
    }%
  \fi
\expandafter\x\csname ver@catchfile.sty\endcsname
\ProvidesPackage{catchfile}%
  [2016/05/16 v1.7 Catch the contents of a file (HO)]%
%    \end{macrocode}
%
% \subsection{Catcodes}
%
%    \begin{macrocode}
\begingroup\catcode61\catcode48\catcode32=10\relax%
  \catcode13=5 % ^^M
  \endlinechar=13 %
  \catcode123=1 % {
  \catcode125=2 % }
  \catcode64=11 % @
  \def\x{\endgroup
    \expandafter\edef\csname CatchFile@AtEnd\endcsname{%
      \endlinechar=\the\endlinechar\relax
      \catcode13=\the\catcode13\relax
      \catcode32=\the\catcode32\relax
      \catcode35=\the\catcode35\relax
      \catcode61=\the\catcode61\relax
      \catcode64=\the\catcode64\relax
      \catcode123=\the\catcode123\relax
      \catcode125=\the\catcode125\relax
    }%
  }%
\x\catcode61\catcode48\catcode32=10\relax%
\catcode13=5 % ^^M
\endlinechar=13 %
\catcode35=6 % #
\catcode64=11 % @
\catcode123=1 % {
\catcode125=2 % }
\def\TMP@EnsureCode#1#2{%
  \edef\CatchFile@AtEnd{%
    \CatchFile@AtEnd
    \catcode#1=\the\catcode#1\relax
  }%
  \catcode#1=#2\relax
}
\TMP@EnsureCode{39}{12}% '
\TMP@EnsureCode{44}{12}% ,
\TMP@EnsureCode{45}{12}% -
\TMP@EnsureCode{46}{12}% .
\TMP@EnsureCode{47}{12}% /
\TMP@EnsureCode{91}{12}% [
\TMP@EnsureCode{93}{12}% ]
\TMP@EnsureCode{96}{12}% `
\edef\CatchFile@AtEnd{\CatchFile@AtEnd\noexpand\endinput}
%    \end{macrocode}
%
% \subsection{Preparations}
%
%    \begin{macrocode}
\begingroup\expandafter\expandafter\expandafter\endgroup
\expandafter\ifx\csname RequirePackage\endcsname\relax
  \input infwarerr.sty\relax
  \input ltxcmds.sty\relax
\else
  \RequirePackage{infwarerr}[2007/09/09]%
  \RequirePackage{ltxcmds}[2010/03/09]%
\fi
%    \end{macrocode}
%
%    Check for \eTeX's \cs{everyeof}.
%    \begin{macrocode}
\begingroup
  \escapechar=92\relax
  \edef\TestString{\string\everyeof}%
  \edef\TestMeaning{\meaning\everyeof}%
  \ifx\TestString\TestMeaning
  \else
    \@PackageError{catchfile}{%
      Cannot find e-TeX's \string\everyeof,\MessageBreak
      package loading is aborted%
    }\@ehd
    \endgroup
    \expandafter\CatchFile@AtEnd
  \fi%
\endgroup
%    \end{macrocode}
%
% \subsection{Looking for primitive \cs{input}}
%
%    \begin{macro}{\CatchFile@Input}
%    The package needs the expandable primitive \cs{input}.
%    However there are formats that redefine it. For example,
%    \LaTeX's \cs{input} is not expandable, but it stores
%    the primitive in \cs{@@input}. The third possibility
%    is \cs{pdfprimitive}, introduced in \pdfTeX\ 1.40.0.
%
%    Thus we try to find the primitive and store it in
%    \cs{CatchFile@Input}. If it is used, it must be expanded
%    twice (because of the solution with \cs{pdfprimitive}).
%    \begin{macrocode}
\begingroup
  \def\Check#1#2#3#4\endgroup{%
    \edef\TestString{\string#1}%
    \edef\TestMeaning{\meaning#2}%
    \ifx\TestString\TestMeaning
      \endgroup
      \let\CatchFile@Primitive#2%
      \def\CatchFile@Input{\CatchFile@Primitive#3}%
    \else
      #4\endgroup
    \fi
  }%
  \Check\input\input{}%
  \Check\input\@@input{}%
  \Check\pdfprimitive\pdfprimitive\input
  \@PackageError{%
    Cannot find primitive \string\input,\MessageBreak
    package loading is aborted%
  }\@ehd
  \csname endgroup\endcsname
  \CatchFile@AtEnd%
\endgroup
%    \end{macrocode}
%    \end{macro}
%
% \subsection{Input file check}
%
%    \begin{macro}{\CatchFile@CheckFileExists}
%    \begin{macrocode}
\begingroup\expandafter\expandafter\expandafter\endgroup
\expandafter\ifx\csname IfFileExists\endcsname\relax
  \input pdftexcmds.sty\relax
  \begingroup\expandafter\expandafter\expandafter\endgroup
  \expandafter\ifx\csname pdf@filesize\endcsname\relax
    \def\CatchFile@CheckFileExists#1{%
      \expandafter\ifx\csname @inputcheck\endcsname\relax
        \csname newread\endcsname\@inputcheck
      \fi
      \openin\@inputcheck#1\relax
      \expandafter\closein\expandafter\@inputcheck
      \ifeof\@inputcheck
        \let\CatchFile@File\relax
      \else
        \def\CatchFile@File{#1}%
      \fi
    }%
  \else
    \def\CatchFile@CheckFileExists#1{%
      \expandafter\expandafter\expandafter\ifx
      \expandafter\expandafter\expandafter\relax\pdf@filesize{#1}\relax
        \let\CatchFile@File\relax
      \else
        \def\CatchFile@File{#1}%
      \fi
    }%
  \fi
\else
  \def\CatchFile@CheckFileExists#1{%
    \IfFileExists{#1}{%
      \expandafter\CatchFile@DefFile\@filef@und\@nil
      \begingroup\expandafter\expandafter\expandafter\endgroup
      \expandafter\ifx\csname @addtofilelist\endcsname\relax
      \else
        \@addtofilelist\CatchFile@File
      \fi
    }{%
      \let\CatchFile@File\relax
    }%
  }%
  \def\CatchFile@DefFile#1 \@nil{%
    \def\CatchFile@File{#1}%
  }%
\fi
%    \end{macrocode}
%    \end{macro}
%
%    \begin{macro}{\CatchFile@NotFound}
%    \begin{macrocode}
\def\CatchFile@NotFound#1#2{%
  \def#1{}%
  \@PackageError{catchfile}{%
    File `#2' not found%
  }\@ehc
}
%    \end{macrocode}
%    \end{macro}
%
% \subsection{Catch file contents}
%
%    \begin{macrocode}
\ltx@IfUndefined{RequirePackage}{%
  \input etexcmds.sty\relax
}{%
  \RequirePackage{etexcmds}[2010/01/28]%
}
%    \end{macrocode}
%
%    \begin{macro}{\CatchFileEdef}
%    \begin{macrocode}
\long\def\CatchFileEdef#1#2#3{%
  \CatchFile@CheckFileExists{#2}%
  \ifx\CatchFile@File\relax
    \CatchFile@NotFound{#1}{#2}%
  \else
    \begingroup
      \everyeof{\noexpand}%
      #3%
      \xdef\CatchFile@Contents{\CatchFile@Input\CatchFile@File\space}%
    \endgroup
    \let#1\CatchFile@Contents
  \fi
}
%    \end{macrocode}
%    \end{macro}
%
%    \begin{macro}{\CatchFileDef}
%    \begin{macrocode}
\long\def\CatchFileDef#1#2#3{%
  \CatchFile@CheckFileExists{#2}%
  \ifx\CatchFile@File\relax
    \CatchFile@NotFound{#1}{#2}%
  \else
    \begingroup
      \everyeof\expandafter{%
        \CatchFile@EOF
        \expandafter\CatchFile@Finish
        \noexpand
      }%
      \ltx@ifundefined{etex@unexpanded}{%
        \expandafter\long\expandafter\def\expandafter\CatchFile@Do
            \expandafter##\expandafter1\CatchFile@EOF{%
          \toks\ltx@zero{##1}%
          \xdef\CatchFile@gtemp{\the\toks\ltx@zero}%
          \def\CatchFile@Finish{%
            \endgroup
            \let#1\CatchFile@gtemp
            \global\let\CatchFile@gtemp\ltx@undefined
          }%
        }%
      }{%
        \expandafter\long\expandafter\def\expandafter\CatchFile@Do
            \expandafter##\expandafter1\CatchFile@EOF{%
          \edef\CatchFile@Finish{%
            \endgroup
            \etex@unexpanded{%
              \edef#1{\etex@unexpanded{##1}}%
            }%
          }%
        }%
      }%
      #3\relax
    \expandafter\expandafter\expandafter\CatchFile@Do
    \CatchFile@Input\CatchFile@File\relax
  \fi
}
%    \end{macrocode}
%    \cs{relax} after |#3| was added to make it more robust in
%    case the user uses something like
%\begin{verbatim}
%\CatchFileDef{\content}{\jobname.tt}{\endlinechar=-1}
%\end{verbatim}
%    that expands the following \cs{expandafter} after |#3| prematurely
%    (contribution of Martin Scharrer).
%    \end{macro}
%    \begin{macro}{\CatchFile@EOF}
%    If the file is read the catcode mappings are fixed.
%    This means that the same character cannot occur
%    inside the file with different catcodes.
%    Thus we use as end of file marker the at sign
%    twice with different catcodes.
%    \begin{macrocode}
\begingroup
  \lccode65=64 % lowercase('A') = '@'
  \lccode66=64 % lowercase('B') = '@'
  \catcode65=8 % catcode('A') = subscript
  \catcode66=3 % catcode('B') = math shift
\lowercase{\endgroup
  \def\CatchFile@EOF{AB}%
}
%    \end{macrocode}
%    \end{macro}
%
%    \begin{macrocode}
\CatchFile@AtEnd%
%</package>
%    \end{macrocode}
%
% \section{Test}
%
% \subsection{Catcode checks for loading}
%
%    \begin{macrocode}
%<*test1>
%    \end{macrocode}
%    \begin{macrocode}
\catcode`\{=1 %
\catcode`\}=2 %
\catcode`\#=6 %
\catcode`\@=11 %
\expandafter\ifx\csname count@\endcsname\relax
  \countdef\count@=255 %
\fi
\expandafter\ifx\csname @gobble\endcsname\relax
  \long\def\@gobble#1{}%
\fi
\expandafter\ifx\csname @firstofone\endcsname\relax
  \long\def\@firstofone#1{#1}%
\fi
\expandafter\ifx\csname loop\endcsname\relax
  \expandafter\@firstofone
\else
  \expandafter\@gobble
\fi
{%
  \def\loop#1\repeat{%
    \def\body{#1}%
    \iterate
  }%
  \def\iterate{%
    \body
      \let\next\iterate
    \else
      \let\next\relax
    \fi
    \next
  }%
  \let\repeat=\fi
}%
\def\RestoreCatcodes{}
\count@=0 %
\loop
  \edef\RestoreCatcodes{%
    \RestoreCatcodes
    \catcode\the\count@=\the\catcode\count@\relax
  }%
\ifnum\count@<255 %
  \advance\count@ 1 %
\repeat

\def\RangeCatcodeInvalid#1#2{%
  \count@=#1\relax
  \loop
    \catcode\count@=15 %
  \ifnum\count@<#2\relax
    \advance\count@ 1 %
  \repeat
}
\def\RangeCatcodeCheck#1#2#3{%
  \count@=#1\relax
  \loop
    \ifnum#3=\catcode\count@
    \else
      \errmessage{%
        Character \the\count@\space
        with wrong catcode \the\catcode\count@\space
        instead of \number#3%
      }%
    \fi
  \ifnum\count@<#2\relax
    \advance\count@ 1 %
  \repeat
}
\def\space{ }
\expandafter\ifx\csname LoadCommand\endcsname\relax
  \def\LoadCommand{\input catchfile.sty\relax}%
\fi
\def\Test{%
  \RangeCatcodeInvalid{0}{47}%
  \RangeCatcodeInvalid{58}{64}%
  \RangeCatcodeInvalid{91}{96}%
  \RangeCatcodeInvalid{123}{255}%
  \catcode`\@=12 %
  \catcode`\\=0 %
  \catcode`\%=14 %
  \LoadCommand
  \RangeCatcodeCheck{0}{36}{15}%
  \RangeCatcodeCheck{37}{37}{14}%
  \RangeCatcodeCheck{38}{47}{15}%
  \RangeCatcodeCheck{48}{57}{12}%
  \RangeCatcodeCheck{58}{63}{15}%
  \RangeCatcodeCheck{64}{64}{12}%
  \RangeCatcodeCheck{65}{90}{11}%
  \RangeCatcodeCheck{91}{91}{15}%
  \RangeCatcodeCheck{92}{92}{0}%
  \RangeCatcodeCheck{93}{96}{15}%
  \RangeCatcodeCheck{97}{122}{11}%
  \RangeCatcodeCheck{123}{255}{15}%
  \RestoreCatcodes
}
\Test
\csname @@end\endcsname
\end
%    \end{macrocode}
%    \begin{macrocode}
%</test1>
%    \end{macrocode}
%
% \subsection{\LaTeX}
%
%    \begin{macrocode}
%<*test2>
\NeedsTeXFormat{LaTeX2e}
\nofiles
\listfiles
\tracingnesting=2 %
\documentclass{minimal}
\usepackage{catchfile}[2016/05/16]
\makeatletter
\def\mysetup{%
  \let\do\@makeother
  \dospecials
}
\def\CheckContents{%
  \begingroup
    \expandafter\ifx\csname contents\endcsname\relax
      \PackageError{TEST}{\string\contents\space is not defined}\@ehc
    \fi
  \endgroup
}
\def\StartDisableUnexpanded{%
  \begingroup
    \let\etex@unexpanded\@undefined
}
\def\StopDisableUnexpanded{%
  \endgroup
}
\def\CheckCleanup{%
  \begingroup
    \edef\x{\the\toks0}%
    \def\y{ABC}%
    \ifx\x\y
    \else
      \PackageError{TEST}{\string\toks0 has changed}\@ehc
    \fi
    \ifx\CatchFile@gtemp\@undefined
    \else
      \PackageError{TEST}{Left over \string\CatchFile@gtemp}\@ehc
    \fi
  \endgroup
}
\def\Check{%
  \CheckContents
  \CheckCleanup
}
\makeatother
\begin{document}
\toks0{ABC}

\CatchFileDef\contents{catchfile.sty}\mysetup
\typeout{\meaning\contents}
\Check
\typeout{***********************************************************}%
\CatchFileDef\contents{catchfile.sty}{}
\typeout{\meaning\contents}
\Check
\typeout{***********************************************************}%

\StartDisableUnexpanded
\CatchFileDef\contents{catchfile.sty}\mysetup
\typeout{\meaning\contents}
\Check
\typeout{***********************************************************}%
\CatchFileDef\contents{catchfile.sty}{}
\typeout{\meaning\contents}
\Check
\typeout{***********************************************************}%
\StopDisableUnexpanded

\CatchFileEdef\contents{catchfile.sty}{%
  \mysetup
  \def\par{^^J}%
  \obeylines
}
\typeout{\contents}
\Check
\typeout{***********************************************************}%
\CatchFileEdef\contents{catchfile.sty}{%
  \catcode`\\=12 %
  \catcode`\#=12 %
  \def\par{^^J}%
  \obeylines
}
\typeout{\contents}
\Check
\typeout{***********************************************************}%
\end{document}
%</test2>
%    \end{macrocode}
%
% \subsection{\plainTeX}
%
%    \begin{macrocode}
%<*test3>
\def\msg#{\immediate\write16}
\newlinechar=10 %
\tracingnesting=2 %
\input catchfile.sty\relax

\def\mysetup{%
  \def\do##1{%
    \catcode`##1=12\relax
  }%
  \dospecials
}
\def\CheckContents{%
  \begingroup
    \expandafter\ifx\csname contents\endcsname\relax
      \def\space{ }%
      \errmessage{\string\contents\space is not defined.}%
    \fi
  \endgroup
}
\catcode`\@=11 %
\def\CheckCleanup{%
  \begingroup
    \edef\x{\the\toks0}%
    \def\y{ABC}%
    \ifx\x\y
    \else
      \errmessage{\string\toks0 has changed}%
    \fi
    \ifx\CatchFile@gtemp\@undefined
    \else
      \errmessage{Left over \string\CatchFile@gtemp}%
    \fi
  \endgroup
}
\def\Check{%
  \CheckContents
  \CheckCleanup
}
\def\StartDisableUnexpanded{%
  \begingroup
    \let\etex@unexpanded\@undefined
}
\def\StopDisableUnexpanded{%
  \endgroup
}
\catcode`\@=12 %

\toks0{ABC}

\CatchFileDef\contents{catchfile.sty}\mysetup
\msg{\meaning\contents}
\Check
\msg{***********************************************************}%
\CatchFileDef\contents{catchfile.sty}{}
\msg{\meaning\contents}
\Check
\msg{***********************************************************}%

\StartDisableUnexpanded
\CatchFileDef\contents{catchfile.sty}\mysetup
\msg{\meaning\contents}
\Check
\msg{***********************************************************}%
\CatchFileDef\contents{catchfile.sty}{}
\msg{\meaning\contents}
\Check
\msg{***********************************************************}%
\StopDisableUnexpanded

\CatchFileEdef\contents{catchfile.sty}{%
  \mysetup
  \def\par{^^J}%
  \obeylines
}
\msg{\contents}
\Check
\msg{***********************************************************}%
\CatchFileEdef\contents{catchfile.sty}{%
  \catcode`\\=12 %
  \catcode`\#=12 %
  \def\par{^^J}%
  \obeylines
}
\msg{\contents}
\Check
\msg{***********************************************************}%

\csname @@end\endcsname
\end
%</test3>
%    \end{macrocode}
%
% \section{Installation}
%
% \subsection{Download}
%
% \paragraph{Package.} This package is available on
% CTAN\footnote{\url{https://ctan.org/pkg/catchfile}}:
% \begin{description}
% \item[\CTAN{macros/latex/contrib/oberdiek/catchfile.dtx}] The source file.
% \item[\CTAN{macros/latex/contrib/oberdiek/catchfile.pdf}] Documentation.
% \end{description}
%
%
% \paragraph{Bundle.} All the packages of the bundle `oberdiek'
% are also available in a TDS compliant ZIP archive. There
% the packages are already unpacked and the documentation files
% are generated. The files and directories obey the TDS standard.
% \begin{description}
% \item[\CTANinstall{install/macros/latex/contrib/oberdiek.tds.zip}]
% \end{description}
% \emph{TDS} refers to the standard ``A Directory Structure
% for \TeX\ Files'' (\CTAN{tds/tds.pdf}). Directories
% with \xfile{texmf} in their name are usually organized this way.
%
% \subsection{Bundle installation}
%
% \paragraph{Unpacking.} Unpack the \xfile{oberdiek.tds.zip} in the
% TDS tree (also known as \xfile{texmf} tree) of your choice.
% Example (linux):
% \begin{quote}
%   |unzip oberdiek.tds.zip -d ~/texmf|
% \end{quote}
%
% \paragraph{Script installation.}
% Check the directory \xfile{TDS:scripts/oberdiek/} for
% scripts that need further installation steps.
% Package \xpackage{attachfile2} comes with the Perl script
% \xfile{pdfatfi.pl} that should be installed in such a way
% that it can be called as \texttt{pdfatfi}.
% Example (linux):
% \begin{quote}
%   |chmod +x scripts/oberdiek/pdfatfi.pl|\\
%   |cp scripts/oberdiek/pdfatfi.pl /usr/local/bin/|
% \end{quote}
%
% \subsection{Package installation}
%
% \paragraph{Unpacking.} The \xfile{.dtx} file is a self-extracting
% \docstrip\ archive. The files are extracted by running the
% \xfile{.dtx} through \plainTeX:
% \begin{quote}
%   \verb|tex catchfile.dtx|
% \end{quote}
%
% \paragraph{TDS.} Now the different files must be moved into
% the different directories in your installation TDS tree
% (also known as \xfile{texmf} tree):
% \begin{quote}
% \def\t{^^A
% \begin{tabular}{@{}>{\ttfamily}l@{ $\rightarrow$ }>{\ttfamily}l@{}}
%   catchfile.sty & tex/generic/oberdiek/catchfile.sty\\
%   catchfile.pdf & doc/latex/oberdiek/catchfile.pdf\\
%   test/catchfile-test1.tex & doc/latex/oberdiek/test/catchfile-test1.tex\\
%   test/catchfile-test2.tex & doc/latex/oberdiek/test/catchfile-test2.tex\\
%   test/catchfile-test3.tex & doc/latex/oberdiek/test/catchfile-test3.tex\\
%   catchfile.dtx & source/latex/oberdiek/catchfile.dtx\\
% \end{tabular}^^A
% }^^A
% \sbox0{\t}^^A
% \ifdim\wd0>\linewidth
%   \begingroup
%     \advance\linewidth by\leftmargin
%     \advance\linewidth by\rightmargin
%   \edef\x{\endgroup
%     \def\noexpand\lw{\the\linewidth}^^A
%   }\x
%   \def\lwbox{^^A
%     \leavevmode
%     \hbox to \linewidth{^^A
%       \kern-\leftmargin\relax
%       \hss
%       \usebox0
%       \hss
%       \kern-\rightmargin\relax
%     }^^A
%   }^^A
%   \ifdim\wd0>\lw
%     \sbox0{\small\t}^^A
%     \ifdim\wd0>\linewidth
%       \ifdim\wd0>\lw
%         \sbox0{\footnotesize\t}^^A
%         \ifdim\wd0>\linewidth
%           \ifdim\wd0>\lw
%             \sbox0{\scriptsize\t}^^A
%             \ifdim\wd0>\linewidth
%               \ifdim\wd0>\lw
%                 \sbox0{\tiny\t}^^A
%                 \ifdim\wd0>\linewidth
%                   \lwbox
%                 \else
%                   \usebox0
%                 \fi
%               \else
%                 \lwbox
%               \fi
%             \else
%               \usebox0
%             \fi
%           \else
%             \lwbox
%           \fi
%         \else
%           \usebox0
%         \fi
%       \else
%         \lwbox
%       \fi
%     \else
%       \usebox0
%     \fi
%   \else
%     \lwbox
%   \fi
% \else
%   \usebox0
% \fi
% \end{quote}
% If you have a \xfile{docstrip.cfg} that configures and enables \docstrip's
% TDS installing feature, then some files can already be in the right
% place, see the documentation of \docstrip.
%
% \subsection{Refresh file name databases}
%
% If your \TeX~distribution
% (\teTeX, \mikTeX, \dots) relies on file name databases, you must refresh
% these. For example, \teTeX\ users run \verb|texhash| or
% \verb|mktexlsr|.
%
% \subsection{Some details for the interested}
%
% \paragraph{Attached source.}
%
% The PDF documentation on CTAN also includes the
% \xfile{.dtx} source file. It can be extracted by
% AcrobatReader 6 or higher. Another option is \textsf{pdftk},
% e.g. unpack the file into the current directory:
% \begin{quote}
%   \verb|pdftk catchfile.pdf unpack_files output .|
% \end{quote}
%
% \paragraph{Unpacking with \LaTeX.}
% The \xfile{.dtx} chooses its action depending on the format:
% \begin{description}
% \item[\plainTeX:] Run \docstrip\ and extract the files.
% \item[\LaTeX:] Generate the documentation.
% \end{description}
% If you insist on using \LaTeX\ for \docstrip\ (really,
% \docstrip\ does not need \LaTeX), then inform the autodetect routine
% about your intention:
% \begin{quote}
%   \verb|latex \let\install=y% \iffalse meta-comment
%
% File: catchfile.dtx
% Version: 2016/05/16 v1.7
% Info: Catch the contents of a file
%
% Copyright (C) 2007, 2010, 2011 by
%    Heiko Oberdiek <heiko.oberdiek at googlemail.com>
%    2016
%    https://github.com/ho-tex/oberdiek/issues
%
% This work may be distributed and/or modified under the
% conditions of the LaTeX Project Public License, either
% version 1.3c of this license or (at your option) any later
% version. This version of this license is in
%    http://www.latex-project.org/lppl/lppl-1-3c.txt
% and the latest version of this license is in
%    http://www.latex-project.org/lppl.txt
% and version 1.3 or later is part of all distributions of
% LaTeX version 2005/12/01 or later.
%
% This work has the LPPL maintenance status "maintained".
%
% This Current Maintainer of this work is Heiko Oberdiek.
%
% The Base Interpreter refers to any `TeX-Format',
% because some files are installed in TDS:tex/generic//.
%
% This work consists of the main source file catchfile.dtx
% and the derived files
%    catchfile.sty, catchfile.pdf, catchfile.ins, catchfile.drv,
%    catchfile-test1.tex, catchfile-test2.tex,
%    catchfile-test3.tex.
%
% Distribution:
%    CTAN:macros/latex/contrib/oberdiek/catchfile.dtx
%    CTAN:macros/latex/contrib/oberdiek/catchfile.pdf
%
% Unpacking:
%    (a) If catchfile.ins is present:
%           tex catchfile.ins
%    (b) Without catchfile.ins:
%           tex catchfile.dtx
%    (c) If you insist on using LaTeX
%           latex \let\install=y% \iffalse meta-comment
%
% File: catchfile.dtx
% Version: 2016/05/16 v1.7
% Info: Catch the contents of a file
%
% Copyright (C) 2007, 2010, 2011 by
%    Heiko Oberdiek <heiko.oberdiek at googlemail.com>
%    2016
%    https://github.com/ho-tex/oberdiek/issues
%
% This work may be distributed and/or modified under the
% conditions of the LaTeX Project Public License, either
% version 1.3c of this license or (at your option) any later
% version. This version of this license is in
%    http://www.latex-project.org/lppl/lppl-1-3c.txt
% and the latest version of this license is in
%    http://www.latex-project.org/lppl.txt
% and version 1.3 or later is part of all distributions of
% LaTeX version 2005/12/01 or later.
%
% This work has the LPPL maintenance status "maintained".
%
% This Current Maintainer of this work is Heiko Oberdiek.
%
% The Base Interpreter refers to any `TeX-Format',
% because some files are installed in TDS:tex/generic//.
%
% This work consists of the main source file catchfile.dtx
% and the derived files
%    catchfile.sty, catchfile.pdf, catchfile.ins, catchfile.drv,
%    catchfile-test1.tex, catchfile-test2.tex,
%    catchfile-test3.tex.
%
% Distribution:
%    CTAN:macros/latex/contrib/oberdiek/catchfile.dtx
%    CTAN:macros/latex/contrib/oberdiek/catchfile.pdf
%
% Unpacking:
%    (a) If catchfile.ins is present:
%           tex catchfile.ins
%    (b) Without catchfile.ins:
%           tex catchfile.dtx
%    (c) If you insist on using LaTeX
%           latex \let\install=y\input{catchfile.dtx}
%        (quote the arguments according to the demands of your shell)
%
% Documentation:
%    (a) If catchfile.drv is present:
%           latex catchfile.drv
%    (b) Without catchfile.drv:
%           latex catchfile.dtx; ...
%    The class ltxdoc loads the configuration file ltxdoc.cfg
%    if available. Here you can specify further options, e.g.
%    use A4 as paper format:
%       \PassOptionsToClass{a4paper}{article}
%
%    Programm calls to get the documentation (example):
%       pdflatex catchfile.dtx
%       makeindex -s gind.ist catchfile.idx
%       pdflatex catchfile.dtx
%       makeindex -s gind.ist catchfile.idx
%       pdflatex catchfile.dtx
%
% Installation:
%    TDS:tex/generic/oberdiek/catchfile.sty
%    TDS:doc/latex/oberdiek/catchfile.pdf
%    TDS:doc/latex/oberdiek/test/catchfile-test1.tex
%    TDS:doc/latex/oberdiek/test/catchfile-test2.tex
%    TDS:doc/latex/oberdiek/test/catchfile-test3.tex
%    TDS:source/latex/oberdiek/catchfile.dtx
%
%<*ignore>
\begingroup
  \catcode123=1 %
  \catcode125=2 %
  \def\x{LaTeX2e}%
\expandafter\endgroup
\ifcase 0\ifx\install y1\fi\expandafter
         \ifx\csname processbatchFile\endcsname\relax\else1\fi
         \ifx\fmtname\x\else 1\fi\relax
\else\csname fi\endcsname
%</ignore>
%<*install>
\input docstrip.tex
\Msg{************************************************************************}
\Msg{* Installation}
\Msg{* Package: catchfile 2016/05/16 v1.7 Catch the contents of a file (HO)}
\Msg{************************************************************************}

\keepsilent
\askforoverwritefalse

\let\MetaPrefix\relax
\preamble

This is a generated file.

Project: catchfile
Version: 2016/05/16 v1.7

Copyright (C) 2007, 2010, 2011 by
   Heiko Oberdiek <heiko.oberdiek at googlemail.com>

This work may be distributed and/or modified under the
conditions of the LaTeX Project Public License, either
version 1.3c of this license or (at your option) any later
version. This version of this license is in
   http://www.latex-project.org/lppl/lppl-1-3c.txt
and the latest version of this license is in
   http://www.latex-project.org/lppl.txt
and version 1.3 or later is part of all distributions of
LaTeX version 2005/12/01 or later.

This work has the LPPL maintenance status "maintained".

This Current Maintainer of this work is Heiko Oberdiek.

The Base Interpreter refers to any `TeX-Format',
because some files are installed in TDS:tex/generic//.

This work consists of the main source file catchfile.dtx
and the derived files
   catchfile.sty, catchfile.pdf, catchfile.ins, catchfile.drv,
   catchfile-test1.tex, catchfile-test2.tex,
   catchfile-test3.tex.

\endpreamble
\let\MetaPrefix\DoubleperCent

\generate{%
  \file{catchfile.ins}{\from{catchfile.dtx}{install}}%
  \file{catchfile.drv}{\from{catchfile.dtx}{driver}}%
  \usedir{tex/generic/oberdiek}%
  \file{catchfile.sty}{\from{catchfile.dtx}{package}}%
%  \usedir{doc/latex/oberdiek/test}%
%  \file{catchfile-test1.tex}{\from{catchfile.dtx}{test1}}%
%  \file{catchfile-test2.tex}{\from{catchfile.dtx}{test2}}%
%  \file{catchfile-test3.tex}{\from{catchfile.dtx}{test3}}%
  \nopreamble
  \nopostamble
%  \usedir{source/latex/oberdiek/catalogue}%
%  \file{catchfile.xml}{\from{catchfile.dtx}{catalogue}}%
}

\catcode32=13\relax% active space
\let =\space%
\Msg{************************************************************************}
\Msg{*}
\Msg{* To finish the installation you have to move the following}
\Msg{* file into a directory searched by TeX:}
\Msg{*}
\Msg{*     catchfile.sty}
\Msg{*}
\Msg{* To produce the documentation run the file `catchfile.drv'}
\Msg{* through LaTeX.}
\Msg{*}
\Msg{* Happy TeXing!}
\Msg{*}
\Msg{************************************************************************}

\endbatchfile
%</install>
%<*ignore>
\fi
%</ignore>
%<*driver>
\NeedsTeXFormat{LaTeX2e}
\ProvidesFile{catchfile.drv}%
  [2016/05/16 v1.7 Catch the contents of a file (HO)]%
\documentclass{ltxdoc}
\usepackage{holtxdoc}[2011/11/22]
\begin{document}
  \DocInput{catchfile.dtx}%
\end{document}
%</driver>
% \fi
%
%
% \CharacterTable
%  {Upper-case    \A\B\C\D\E\F\G\H\I\J\K\L\M\N\O\P\Q\R\S\T\U\V\W\X\Y\Z
%   Lower-case    \a\b\c\d\e\f\g\h\i\j\k\l\m\n\o\p\q\r\s\t\u\v\w\x\y\z
%   Digits        \0\1\2\3\4\5\6\7\8\9
%   Exclamation   \!     Double quote  \"     Hash (number) \#
%   Dollar        \$     Percent       \%     Ampersand     \&
%   Acute accent  \'     Left paren    \(     Right paren   \)
%   Asterisk      \*     Plus          \+     Comma         \,
%   Minus         \-     Point         \.     Solidus       \/
%   Colon         \:     Semicolon     \;     Less than     \<
%   Equals        \=     Greater than  \>     Question mark \?
%   Commercial at \@     Left bracket  \[     Backslash     \\
%   Right bracket \]     Circumflex    \^     Underscore    \_
%   Grave accent  \`     Left brace    \{     Vertical bar  \|
%   Right brace   \}     Tilde         \~}
%
% \GetFileInfo{catchfile.drv}
%
% \title{The \xpackage{catchfile} package}
% \date{2016/05/16 v1.7}
% \author{Heiko Oberdiek\thanks
% {Please report any issues at https://github.com/ho-tex/oberdiek/issues}\\
% \xemail{heiko.oberdiek at googlemail.com}}
%
% \maketitle
%
% \begin{abstract}
% This package catches the contents of a file and puts it in a macro.
% It requires \eTeX. Both \LaTeX\ and \plainTeX\ are supported.
% \end{abstract}
%
% \tableofcontents
%
% \section{Documentation}
%
% The package relies on \eTeX's \cs{everyeof}. Otherwise it
% aborts with an error message.
%
% \begin{declcs}{CatchFileDef} \M{cmd} \M{file name} \M{setup}\\
% \cs{CatchFileEdef} \M{cmd} \M{file name} \M{setup}
% \end{declcs}
% Macro \meta{cmd} is defined with the contents
% of file \meta{file name}.
% \cs{CatchFileDef} uses \cs{def}, \cs{CatchFileEdef} \cs{edef}
% for the definition.
% Additional setup code for setting
% catcodes or treatment of line ends can be given in code
% \meta{setup}. See the test files for an example.
%
% \StopEventually{
% }
%
% \section{Implementation}
%
%    \begin{macrocode}
%<*package>
%    \end{macrocode}
%
% \subsection{Reload check and package identification}
%    Reload check, especially if the package is not used with \LaTeX.
%    \begin{macrocode}
\begingroup\catcode61\catcode48\catcode32=10\relax%
  \catcode13=5 % ^^M
  \endlinechar=13 %
  \catcode35=6 % #
  \catcode39=12 % '
  \catcode44=12 % ,
  \catcode45=12 % -
  \catcode46=12 % .
  \catcode58=12 % :
  \catcode64=11 % @
  \catcode123=1 % {
  \catcode125=2 % }
  \expandafter\let\expandafter\x\csname ver@catchfile.sty\endcsname
  \ifx\x\relax % plain-TeX, first loading
  \else
    \def\empty{}%
    \ifx\x\empty % LaTeX, first loading,
      % variable is initialized, but \ProvidesPackage not yet seen
    \else
      \expandafter\ifx\csname PackageInfo\endcsname\relax
        \def\x#1#2{%
          \immediate\write-1{Package #1 Info: #2.}%
        }%
      \else
        \def\x#1#2{\PackageInfo{#1}{#2, stopped}}%
      \fi
      \x{catchfile}{The package is already loaded}%
      \aftergroup\endinput
    \fi
  \fi
\endgroup%
%    \end{macrocode}
%    Package identification:
%    \begin{macrocode}
\begingroup\catcode61\catcode48\catcode32=10\relax%
  \catcode13=5 % ^^M
  \endlinechar=13 %
  \catcode35=6 % #
  \catcode39=12 % '
  \catcode40=12 % (
  \catcode41=12 % )
  \catcode44=12 % ,
  \catcode45=12 % -
  \catcode46=12 % .
  \catcode47=12 % /
  \catcode58=12 % :
  \catcode64=11 % @
  \catcode91=12 % [
  \catcode93=12 % ]
  \catcode123=1 % {
  \catcode125=2 % }
  \expandafter\ifx\csname ProvidesPackage\endcsname\relax
    \def\x#1#2#3[#4]{\endgroup
      \immediate\write-1{Package: #3 #4}%
      \xdef#1{#4}%
    }%
  \else
    \def\x#1#2[#3]{\endgroup
      #2[{#3}]%
      \ifx#1\@undefined
        \xdef#1{#3}%
      \fi
      \ifx#1\relax
        \xdef#1{#3}%
      \fi
    }%
  \fi
\expandafter\x\csname ver@catchfile.sty\endcsname
\ProvidesPackage{catchfile}%
  [2016/05/16 v1.7 Catch the contents of a file (HO)]%
%    \end{macrocode}
%
% \subsection{Catcodes}
%
%    \begin{macrocode}
\begingroup\catcode61\catcode48\catcode32=10\relax%
  \catcode13=5 % ^^M
  \endlinechar=13 %
  \catcode123=1 % {
  \catcode125=2 % }
  \catcode64=11 % @
  \def\x{\endgroup
    \expandafter\edef\csname CatchFile@AtEnd\endcsname{%
      \endlinechar=\the\endlinechar\relax
      \catcode13=\the\catcode13\relax
      \catcode32=\the\catcode32\relax
      \catcode35=\the\catcode35\relax
      \catcode61=\the\catcode61\relax
      \catcode64=\the\catcode64\relax
      \catcode123=\the\catcode123\relax
      \catcode125=\the\catcode125\relax
    }%
  }%
\x\catcode61\catcode48\catcode32=10\relax%
\catcode13=5 % ^^M
\endlinechar=13 %
\catcode35=6 % #
\catcode64=11 % @
\catcode123=1 % {
\catcode125=2 % }
\def\TMP@EnsureCode#1#2{%
  \edef\CatchFile@AtEnd{%
    \CatchFile@AtEnd
    \catcode#1=\the\catcode#1\relax
  }%
  \catcode#1=#2\relax
}
\TMP@EnsureCode{39}{12}% '
\TMP@EnsureCode{44}{12}% ,
\TMP@EnsureCode{45}{12}% -
\TMP@EnsureCode{46}{12}% .
\TMP@EnsureCode{47}{12}% /
\TMP@EnsureCode{91}{12}% [
\TMP@EnsureCode{93}{12}% ]
\TMP@EnsureCode{96}{12}% `
\edef\CatchFile@AtEnd{\CatchFile@AtEnd\noexpand\endinput}
%    \end{macrocode}
%
% \subsection{Preparations}
%
%    \begin{macrocode}
\begingroup\expandafter\expandafter\expandafter\endgroup
\expandafter\ifx\csname RequirePackage\endcsname\relax
  \input infwarerr.sty\relax
  \input ltxcmds.sty\relax
\else
  \RequirePackage{infwarerr}[2007/09/09]%
  \RequirePackage{ltxcmds}[2010/03/09]%
\fi
%    \end{macrocode}
%
%    Check for \eTeX's \cs{everyeof}.
%    \begin{macrocode}
\begingroup
  \escapechar=92\relax
  \edef\TestString{\string\everyeof}%
  \edef\TestMeaning{\meaning\everyeof}%
  \ifx\TestString\TestMeaning
  \else
    \@PackageError{catchfile}{%
      Cannot find e-TeX's \string\everyeof,\MessageBreak
      package loading is aborted%
    }\@ehd
    \endgroup
    \expandafter\CatchFile@AtEnd
  \fi%
\endgroup
%    \end{macrocode}
%
% \subsection{Looking for primitive \cs{input}}
%
%    \begin{macro}{\CatchFile@Input}
%    The package needs the expandable primitive \cs{input}.
%    However there are formats that redefine it. For example,
%    \LaTeX's \cs{input} is not expandable, but it stores
%    the primitive in \cs{@@input}. The third possibility
%    is \cs{pdfprimitive}, introduced in \pdfTeX\ 1.40.0.
%
%    Thus we try to find the primitive and store it in
%    \cs{CatchFile@Input}. If it is used, it must be expanded
%    twice (because of the solution with \cs{pdfprimitive}).
%    \begin{macrocode}
\begingroup
  \def\Check#1#2#3#4\endgroup{%
    \edef\TestString{\string#1}%
    \edef\TestMeaning{\meaning#2}%
    \ifx\TestString\TestMeaning
      \endgroup
      \let\CatchFile@Primitive#2%
      \def\CatchFile@Input{\CatchFile@Primitive#3}%
    \else
      #4\endgroup
    \fi
  }%
  \Check\input\input{}%
  \Check\input\@@input{}%
  \Check\pdfprimitive\pdfprimitive\input
  \@PackageError{%
    Cannot find primitive \string\input,\MessageBreak
    package loading is aborted%
  }\@ehd
  \csname endgroup\endcsname
  \CatchFile@AtEnd%
\endgroup
%    \end{macrocode}
%    \end{macro}
%
% \subsection{Input file check}
%
%    \begin{macro}{\CatchFile@CheckFileExists}
%    \begin{macrocode}
\begingroup\expandafter\expandafter\expandafter\endgroup
\expandafter\ifx\csname IfFileExists\endcsname\relax
  \input pdftexcmds.sty\relax
  \begingroup\expandafter\expandafter\expandafter\endgroup
  \expandafter\ifx\csname pdf@filesize\endcsname\relax
    \def\CatchFile@CheckFileExists#1{%
      \expandafter\ifx\csname @inputcheck\endcsname\relax
        \csname newread\endcsname\@inputcheck
      \fi
      \openin\@inputcheck#1\relax
      \expandafter\closein\expandafter\@inputcheck
      \ifeof\@inputcheck
        \let\CatchFile@File\relax
      \else
        \def\CatchFile@File{#1}%
      \fi
    }%
  \else
    \def\CatchFile@CheckFileExists#1{%
      \expandafter\expandafter\expandafter\ifx
      \expandafter\expandafter\expandafter\relax\pdf@filesize{#1}\relax
        \let\CatchFile@File\relax
      \else
        \def\CatchFile@File{#1}%
      \fi
    }%
  \fi
\else
  \def\CatchFile@CheckFileExists#1{%
    \IfFileExists{#1}{%
      \expandafter\CatchFile@DefFile\@filef@und\@nil
      \begingroup\expandafter\expandafter\expandafter\endgroup
      \expandafter\ifx\csname @addtofilelist\endcsname\relax
      \else
        \@addtofilelist\CatchFile@File
      \fi
    }{%
      \let\CatchFile@File\relax
    }%
  }%
  \def\CatchFile@DefFile#1 \@nil{%
    \def\CatchFile@File{#1}%
  }%
\fi
%    \end{macrocode}
%    \end{macro}
%
%    \begin{macro}{\CatchFile@NotFound}
%    \begin{macrocode}
\def\CatchFile@NotFound#1#2{%
  \def#1{}%
  \@PackageError{catchfile}{%
    File `#2' not found%
  }\@ehc
}
%    \end{macrocode}
%    \end{macro}
%
% \subsection{Catch file contents}
%
%    \begin{macrocode}
\ltx@IfUndefined{RequirePackage}{%
  \input etexcmds.sty\relax
}{%
  \RequirePackage{etexcmds}[2010/01/28]%
}
%    \end{macrocode}
%
%    \begin{macro}{\CatchFileEdef}
%    \begin{macrocode}
\long\def\CatchFileEdef#1#2#3{%
  \CatchFile@CheckFileExists{#2}%
  \ifx\CatchFile@File\relax
    \CatchFile@NotFound{#1}{#2}%
  \else
    \begingroup
      \everyeof{\noexpand}%
      #3%
      \xdef\CatchFile@Contents{\CatchFile@Input\CatchFile@File\space}%
    \endgroup
    \let#1\CatchFile@Contents
  \fi
}
%    \end{macrocode}
%    \end{macro}
%
%    \begin{macro}{\CatchFileDef}
%    \begin{macrocode}
\long\def\CatchFileDef#1#2#3{%
  \CatchFile@CheckFileExists{#2}%
  \ifx\CatchFile@File\relax
    \CatchFile@NotFound{#1}{#2}%
  \else
    \begingroup
      \everyeof\expandafter{%
        \CatchFile@EOF
        \expandafter\CatchFile@Finish
        \noexpand
      }%
      \ltx@ifundefined{etex@unexpanded}{%
        \expandafter\long\expandafter\def\expandafter\CatchFile@Do
            \expandafter##\expandafter1\CatchFile@EOF{%
          \toks\ltx@zero{##1}%
          \xdef\CatchFile@gtemp{\the\toks\ltx@zero}%
          \def\CatchFile@Finish{%
            \endgroup
            \let#1\CatchFile@gtemp
            \global\let\CatchFile@gtemp\ltx@undefined
          }%
        }%
      }{%
        \expandafter\long\expandafter\def\expandafter\CatchFile@Do
            \expandafter##\expandafter1\CatchFile@EOF{%
          \edef\CatchFile@Finish{%
            \endgroup
            \etex@unexpanded{%
              \edef#1{\etex@unexpanded{##1}}%
            }%
          }%
        }%
      }%
      #3\relax
    \expandafter\expandafter\expandafter\CatchFile@Do
    \CatchFile@Input\CatchFile@File\relax
  \fi
}
%    \end{macrocode}
%    \cs{relax} after |#3| was added to make it more robust in
%    case the user uses something like
%\begin{verbatim}
%\CatchFileDef{\content}{\jobname.tt}{\endlinechar=-1}
%\end{verbatim}
%    that expands the following \cs{expandafter} after |#3| prematurely
%    (contribution of Martin Scharrer).
%    \end{macro}
%    \begin{macro}{\CatchFile@EOF}
%    If the file is read the catcode mappings are fixed.
%    This means that the same character cannot occur
%    inside the file with different catcodes.
%    Thus we use as end of file marker the at sign
%    twice with different catcodes.
%    \begin{macrocode}
\begingroup
  \lccode65=64 % lowercase('A') = '@'
  \lccode66=64 % lowercase('B') = '@'
  \catcode65=8 % catcode('A') = subscript
  \catcode66=3 % catcode('B') = math shift
\lowercase{\endgroup
  \def\CatchFile@EOF{AB}%
}
%    \end{macrocode}
%    \end{macro}
%
%    \begin{macrocode}
\CatchFile@AtEnd%
%</package>
%    \end{macrocode}
%
% \section{Test}
%
% \subsection{Catcode checks for loading}
%
%    \begin{macrocode}
%<*test1>
%    \end{macrocode}
%    \begin{macrocode}
\catcode`\{=1 %
\catcode`\}=2 %
\catcode`\#=6 %
\catcode`\@=11 %
\expandafter\ifx\csname count@\endcsname\relax
  \countdef\count@=255 %
\fi
\expandafter\ifx\csname @gobble\endcsname\relax
  \long\def\@gobble#1{}%
\fi
\expandafter\ifx\csname @firstofone\endcsname\relax
  \long\def\@firstofone#1{#1}%
\fi
\expandafter\ifx\csname loop\endcsname\relax
  \expandafter\@firstofone
\else
  \expandafter\@gobble
\fi
{%
  \def\loop#1\repeat{%
    \def\body{#1}%
    \iterate
  }%
  \def\iterate{%
    \body
      \let\next\iterate
    \else
      \let\next\relax
    \fi
    \next
  }%
  \let\repeat=\fi
}%
\def\RestoreCatcodes{}
\count@=0 %
\loop
  \edef\RestoreCatcodes{%
    \RestoreCatcodes
    \catcode\the\count@=\the\catcode\count@\relax
  }%
\ifnum\count@<255 %
  \advance\count@ 1 %
\repeat

\def\RangeCatcodeInvalid#1#2{%
  \count@=#1\relax
  \loop
    \catcode\count@=15 %
  \ifnum\count@<#2\relax
    \advance\count@ 1 %
  \repeat
}
\def\RangeCatcodeCheck#1#2#3{%
  \count@=#1\relax
  \loop
    \ifnum#3=\catcode\count@
    \else
      \errmessage{%
        Character \the\count@\space
        with wrong catcode \the\catcode\count@\space
        instead of \number#3%
      }%
    \fi
  \ifnum\count@<#2\relax
    \advance\count@ 1 %
  \repeat
}
\def\space{ }
\expandafter\ifx\csname LoadCommand\endcsname\relax
  \def\LoadCommand{\input catchfile.sty\relax}%
\fi
\def\Test{%
  \RangeCatcodeInvalid{0}{47}%
  \RangeCatcodeInvalid{58}{64}%
  \RangeCatcodeInvalid{91}{96}%
  \RangeCatcodeInvalid{123}{255}%
  \catcode`\@=12 %
  \catcode`\\=0 %
  \catcode`\%=14 %
  \LoadCommand
  \RangeCatcodeCheck{0}{36}{15}%
  \RangeCatcodeCheck{37}{37}{14}%
  \RangeCatcodeCheck{38}{47}{15}%
  \RangeCatcodeCheck{48}{57}{12}%
  \RangeCatcodeCheck{58}{63}{15}%
  \RangeCatcodeCheck{64}{64}{12}%
  \RangeCatcodeCheck{65}{90}{11}%
  \RangeCatcodeCheck{91}{91}{15}%
  \RangeCatcodeCheck{92}{92}{0}%
  \RangeCatcodeCheck{93}{96}{15}%
  \RangeCatcodeCheck{97}{122}{11}%
  \RangeCatcodeCheck{123}{255}{15}%
  \RestoreCatcodes
}
\Test
\csname @@end\endcsname
\end
%    \end{macrocode}
%    \begin{macrocode}
%</test1>
%    \end{macrocode}
%
% \subsection{\LaTeX}
%
%    \begin{macrocode}
%<*test2>
\NeedsTeXFormat{LaTeX2e}
\nofiles
\listfiles
\tracingnesting=2 %
\documentclass{minimal}
\usepackage{catchfile}[2016/05/16]
\makeatletter
\def\mysetup{%
  \let\do\@makeother
  \dospecials
}
\def\CheckContents{%
  \begingroup
    \expandafter\ifx\csname contents\endcsname\relax
      \PackageError{TEST}{\string\contents\space is not defined}\@ehc
    \fi
  \endgroup
}
\def\StartDisableUnexpanded{%
  \begingroup
    \let\etex@unexpanded\@undefined
}
\def\StopDisableUnexpanded{%
  \endgroup
}
\def\CheckCleanup{%
  \begingroup
    \edef\x{\the\toks0}%
    \def\y{ABC}%
    \ifx\x\y
    \else
      \PackageError{TEST}{\string\toks0 has changed}\@ehc
    \fi
    \ifx\CatchFile@gtemp\@undefined
    \else
      \PackageError{TEST}{Left over \string\CatchFile@gtemp}\@ehc
    \fi
  \endgroup
}
\def\Check{%
  \CheckContents
  \CheckCleanup
}
\makeatother
\begin{document}
\toks0{ABC}

\CatchFileDef\contents{catchfile.sty}\mysetup
\typeout{\meaning\contents}
\Check
\typeout{***********************************************************}%
\CatchFileDef\contents{catchfile.sty}{}
\typeout{\meaning\contents}
\Check
\typeout{***********************************************************}%

\StartDisableUnexpanded
\CatchFileDef\contents{catchfile.sty}\mysetup
\typeout{\meaning\contents}
\Check
\typeout{***********************************************************}%
\CatchFileDef\contents{catchfile.sty}{}
\typeout{\meaning\contents}
\Check
\typeout{***********************************************************}%
\StopDisableUnexpanded

\CatchFileEdef\contents{catchfile.sty}{%
  \mysetup
  \def\par{^^J}%
  \obeylines
}
\typeout{\contents}
\Check
\typeout{***********************************************************}%
\CatchFileEdef\contents{catchfile.sty}{%
  \catcode`\\=12 %
  \catcode`\#=12 %
  \def\par{^^J}%
  \obeylines
}
\typeout{\contents}
\Check
\typeout{***********************************************************}%
\end{document}
%</test2>
%    \end{macrocode}
%
% \subsection{\plainTeX}
%
%    \begin{macrocode}
%<*test3>
\def\msg#{\immediate\write16}
\newlinechar=10 %
\tracingnesting=2 %
\input catchfile.sty\relax

\def\mysetup{%
  \def\do##1{%
    \catcode`##1=12\relax
  }%
  \dospecials
}
\def\CheckContents{%
  \begingroup
    \expandafter\ifx\csname contents\endcsname\relax
      \def\space{ }%
      \errmessage{\string\contents\space is not defined.}%
    \fi
  \endgroup
}
\catcode`\@=11 %
\def\CheckCleanup{%
  \begingroup
    \edef\x{\the\toks0}%
    \def\y{ABC}%
    \ifx\x\y
    \else
      \errmessage{\string\toks0 has changed}%
    \fi
    \ifx\CatchFile@gtemp\@undefined
    \else
      \errmessage{Left over \string\CatchFile@gtemp}%
    \fi
  \endgroup
}
\def\Check{%
  \CheckContents
  \CheckCleanup
}
\def\StartDisableUnexpanded{%
  \begingroup
    \let\etex@unexpanded\@undefined
}
\def\StopDisableUnexpanded{%
  \endgroup
}
\catcode`\@=12 %

\toks0{ABC}

\CatchFileDef\contents{catchfile.sty}\mysetup
\msg{\meaning\contents}
\Check
\msg{***********************************************************}%
\CatchFileDef\contents{catchfile.sty}{}
\msg{\meaning\contents}
\Check
\msg{***********************************************************}%

\StartDisableUnexpanded
\CatchFileDef\contents{catchfile.sty}\mysetup
\msg{\meaning\contents}
\Check
\msg{***********************************************************}%
\CatchFileDef\contents{catchfile.sty}{}
\msg{\meaning\contents}
\Check
\msg{***********************************************************}%
\StopDisableUnexpanded

\CatchFileEdef\contents{catchfile.sty}{%
  \mysetup
  \def\par{^^J}%
  \obeylines
}
\msg{\contents}
\Check
\msg{***********************************************************}%
\CatchFileEdef\contents{catchfile.sty}{%
  \catcode`\\=12 %
  \catcode`\#=12 %
  \def\par{^^J}%
  \obeylines
}
\msg{\contents}
\Check
\msg{***********************************************************}%

\csname @@end\endcsname
\end
%</test3>
%    \end{macrocode}
%
% \section{Installation}
%
% \subsection{Download}
%
% \paragraph{Package.} This package is available on
% CTAN\footnote{\url{https://ctan.org/pkg/catchfile}}:
% \begin{description}
% \item[\CTAN{macros/latex/contrib/oberdiek/catchfile.dtx}] The source file.
% \item[\CTAN{macros/latex/contrib/oberdiek/catchfile.pdf}] Documentation.
% \end{description}
%
%
% \paragraph{Bundle.} All the packages of the bundle `oberdiek'
% are also available in a TDS compliant ZIP archive. There
% the packages are already unpacked and the documentation files
% are generated. The files and directories obey the TDS standard.
% \begin{description}
% \item[\CTANinstall{install/macros/latex/contrib/oberdiek.tds.zip}]
% \end{description}
% \emph{TDS} refers to the standard ``A Directory Structure
% for \TeX\ Files'' (\CTAN{tds/tds.pdf}). Directories
% with \xfile{texmf} in their name are usually organized this way.
%
% \subsection{Bundle installation}
%
% \paragraph{Unpacking.} Unpack the \xfile{oberdiek.tds.zip} in the
% TDS tree (also known as \xfile{texmf} tree) of your choice.
% Example (linux):
% \begin{quote}
%   |unzip oberdiek.tds.zip -d ~/texmf|
% \end{quote}
%
% \paragraph{Script installation.}
% Check the directory \xfile{TDS:scripts/oberdiek/} for
% scripts that need further installation steps.
% Package \xpackage{attachfile2} comes with the Perl script
% \xfile{pdfatfi.pl} that should be installed in such a way
% that it can be called as \texttt{pdfatfi}.
% Example (linux):
% \begin{quote}
%   |chmod +x scripts/oberdiek/pdfatfi.pl|\\
%   |cp scripts/oberdiek/pdfatfi.pl /usr/local/bin/|
% \end{quote}
%
% \subsection{Package installation}
%
% \paragraph{Unpacking.} The \xfile{.dtx} file is a self-extracting
% \docstrip\ archive. The files are extracted by running the
% \xfile{.dtx} through \plainTeX:
% \begin{quote}
%   \verb|tex catchfile.dtx|
% \end{quote}
%
% \paragraph{TDS.} Now the different files must be moved into
% the different directories in your installation TDS tree
% (also known as \xfile{texmf} tree):
% \begin{quote}
% \def\t{^^A
% \begin{tabular}{@{}>{\ttfamily}l@{ $\rightarrow$ }>{\ttfamily}l@{}}
%   catchfile.sty & tex/generic/oberdiek/catchfile.sty\\
%   catchfile.pdf & doc/latex/oberdiek/catchfile.pdf\\
%   test/catchfile-test1.tex & doc/latex/oberdiek/test/catchfile-test1.tex\\
%   test/catchfile-test2.tex & doc/latex/oberdiek/test/catchfile-test2.tex\\
%   test/catchfile-test3.tex & doc/latex/oberdiek/test/catchfile-test3.tex\\
%   catchfile.dtx & source/latex/oberdiek/catchfile.dtx\\
% \end{tabular}^^A
% }^^A
% \sbox0{\t}^^A
% \ifdim\wd0>\linewidth
%   \begingroup
%     \advance\linewidth by\leftmargin
%     \advance\linewidth by\rightmargin
%   \edef\x{\endgroup
%     \def\noexpand\lw{\the\linewidth}^^A
%   }\x
%   \def\lwbox{^^A
%     \leavevmode
%     \hbox to \linewidth{^^A
%       \kern-\leftmargin\relax
%       \hss
%       \usebox0
%       \hss
%       \kern-\rightmargin\relax
%     }^^A
%   }^^A
%   \ifdim\wd0>\lw
%     \sbox0{\small\t}^^A
%     \ifdim\wd0>\linewidth
%       \ifdim\wd0>\lw
%         \sbox0{\footnotesize\t}^^A
%         \ifdim\wd0>\linewidth
%           \ifdim\wd0>\lw
%             \sbox0{\scriptsize\t}^^A
%             \ifdim\wd0>\linewidth
%               \ifdim\wd0>\lw
%                 \sbox0{\tiny\t}^^A
%                 \ifdim\wd0>\linewidth
%                   \lwbox
%                 \else
%                   \usebox0
%                 \fi
%               \else
%                 \lwbox
%               \fi
%             \else
%               \usebox0
%             \fi
%           \else
%             \lwbox
%           \fi
%         \else
%           \usebox0
%         \fi
%       \else
%         \lwbox
%       \fi
%     \else
%       \usebox0
%     \fi
%   \else
%     \lwbox
%   \fi
% \else
%   \usebox0
% \fi
% \end{quote}
% If you have a \xfile{docstrip.cfg} that configures and enables \docstrip's
% TDS installing feature, then some files can already be in the right
% place, see the documentation of \docstrip.
%
% \subsection{Refresh file name databases}
%
% If your \TeX~distribution
% (\teTeX, \mikTeX, \dots) relies on file name databases, you must refresh
% these. For example, \teTeX\ users run \verb|texhash| or
% \verb|mktexlsr|.
%
% \subsection{Some details for the interested}
%
% \paragraph{Attached source.}
%
% The PDF documentation on CTAN also includes the
% \xfile{.dtx} source file. It can be extracted by
% AcrobatReader 6 or higher. Another option is \textsf{pdftk},
% e.g. unpack the file into the current directory:
% \begin{quote}
%   \verb|pdftk catchfile.pdf unpack_files output .|
% \end{quote}
%
% \paragraph{Unpacking with \LaTeX.}
% The \xfile{.dtx} chooses its action depending on the format:
% \begin{description}
% \item[\plainTeX:] Run \docstrip\ and extract the files.
% \item[\LaTeX:] Generate the documentation.
% \end{description}
% If you insist on using \LaTeX\ for \docstrip\ (really,
% \docstrip\ does not need \LaTeX), then inform the autodetect routine
% about your intention:
% \begin{quote}
%   \verb|latex \let\install=y\input{catchfile.dtx}|
% \end{quote}
% Do not forget to quote the argument according to the demands
% of your shell.
%
% \paragraph{Generating the documentation.}
% You can use both the \xfile{.dtx} or the \xfile{.drv} to generate
% the documentation. The process can be configured by the
% configuration file \xfile{ltxdoc.cfg}. For instance, put this
% line into this file, if you want to have A4 as paper format:
% \begin{quote}
%   \verb|\PassOptionsToClass{a4paper}{article}|
% \end{quote}
% An example follows how to generate the
% documentation with pdf\LaTeX:
% \begin{quote}
%\begin{verbatim}
%pdflatex catchfile.dtx
%makeindex -s gind.ist catchfile.idx
%pdflatex catchfile.dtx
%makeindex -s gind.ist catchfile.idx
%pdflatex catchfile.dtx
%\end{verbatim}
% \end{quote}
%
% \begin{History}
%   \begin{Version}{2007/05/30 v1.0}
%   \item
%     First version.
%   \end{Version}
%   \begin{Version}{2007/09/09 v1.1}
%   \item
%     Catcode section rewritten.
%   \end{Version}
%   \begin{Version}{2007/11/11 v1.2}
%   \item
%     Use of package \xpackage{pdftexcmds} for \LuaTeX\ support.
%   \end{Version}
%   \begin{Version}{2010/03/01 v1.3}
%   \item
%     Fix for unknown \cs{@PackageErrorNoLine}.
%   \end{Version}
%   \begin{Version}{2010/04/08 v1.4}
%   \item
%     \cs{closein} also added if \cs{ifeof} is true.
%   \end{Version}
%   \begin{Version}{2010/04/28 v1.5}
%   \item
%     \cs{CatchFileDef}: Getting rid of warning `end of semi simple group
%     entered at line \dots\ of
%     a different file (Florent Chervet).
%   \item
%     \cs{CatchFileDef}: Fix for error `Illegal parameter number in
%     definition of \dots' (HO) including improvement that uses
%     \cs{unexpanded} if available (Florent Chervet).
%   \end{Version}
%   \begin{Version}{2011/03/01 v1.6}
%   \item
%     \cs{relax} added after the setup argument of
%     \cs{CatchFileDef} to prevent premature file reading
%     (Martin Scharrer).
%   \end{Version}
%   \begin{Version}{2016/05/16 v1.7}
%   \item
%     Documentation updates.
%   \end{Version}
% \end{History}
%
% \PrintIndex
%
% \Finale
\endinput

%        (quote the arguments according to the demands of your shell)
%
% Documentation:
%    (a) If catchfile.drv is present:
%           latex catchfile.drv
%    (b) Without catchfile.drv:
%           latex catchfile.dtx; ...
%    The class ltxdoc loads the configuration file ltxdoc.cfg
%    if available. Here you can specify further options, e.g.
%    use A4 as paper format:
%       \PassOptionsToClass{a4paper}{article}
%
%    Programm calls to get the documentation (example):
%       pdflatex catchfile.dtx
%       makeindex -s gind.ist catchfile.idx
%       pdflatex catchfile.dtx
%       makeindex -s gind.ist catchfile.idx
%       pdflatex catchfile.dtx
%
% Installation:
%    TDS:tex/generic/oberdiek/catchfile.sty
%    TDS:doc/latex/oberdiek/catchfile.pdf
%    TDS:doc/latex/oberdiek/test/catchfile-test1.tex
%    TDS:doc/latex/oberdiek/test/catchfile-test2.tex
%    TDS:doc/latex/oberdiek/test/catchfile-test3.tex
%    TDS:source/latex/oberdiek/catchfile.dtx
%
%<*ignore>
\begingroup
  \catcode123=1 %
  \catcode125=2 %
  \def\x{LaTeX2e}%
\expandafter\endgroup
\ifcase 0\ifx\install y1\fi\expandafter
         \ifx\csname processbatchFile\endcsname\relax\else1\fi
         \ifx\fmtname\x\else 1\fi\relax
\else\csname fi\endcsname
%</ignore>
%<*install>
\input docstrip.tex
\Msg{************************************************************************}
\Msg{* Installation}
\Msg{* Package: catchfile 2016/05/16 v1.7 Catch the contents of a file (HO)}
\Msg{************************************************************************}

\keepsilent
\askforoverwritefalse

\let\MetaPrefix\relax
\preamble

This is a generated file.

Project: catchfile
Version: 2016/05/16 v1.7

Copyright (C) 2007, 2010, 2011 by
   Heiko Oberdiek <heiko.oberdiek at googlemail.com>

This work may be distributed and/or modified under the
conditions of the LaTeX Project Public License, either
version 1.3c of this license or (at your option) any later
version. This version of this license is in
   http://www.latex-project.org/lppl/lppl-1-3c.txt
and the latest version of this license is in
   http://www.latex-project.org/lppl.txt
and version 1.3 or later is part of all distributions of
LaTeX version 2005/12/01 or later.

This work has the LPPL maintenance status "maintained".

This Current Maintainer of this work is Heiko Oberdiek.

The Base Interpreter refers to any `TeX-Format',
because some files are installed in TDS:tex/generic//.

This work consists of the main source file catchfile.dtx
and the derived files
   catchfile.sty, catchfile.pdf, catchfile.ins, catchfile.drv,
   catchfile-test1.tex, catchfile-test2.tex,
   catchfile-test3.tex.

\endpreamble
\let\MetaPrefix\DoubleperCent

\generate{%
  \file{catchfile.ins}{\from{catchfile.dtx}{install}}%
  \file{catchfile.drv}{\from{catchfile.dtx}{driver}}%
  \usedir{tex/generic/oberdiek}%
  \file{catchfile.sty}{\from{catchfile.dtx}{package}}%
%  \usedir{doc/latex/oberdiek/test}%
%  \file{catchfile-test1.tex}{\from{catchfile.dtx}{test1}}%
%  \file{catchfile-test2.tex}{\from{catchfile.dtx}{test2}}%
%  \file{catchfile-test3.tex}{\from{catchfile.dtx}{test3}}%
  \nopreamble
  \nopostamble
%  \usedir{source/latex/oberdiek/catalogue}%
%  \file{catchfile.xml}{\from{catchfile.dtx}{catalogue}}%
}

\catcode32=13\relax% active space
\let =\space%
\Msg{************************************************************************}
\Msg{*}
\Msg{* To finish the installation you have to move the following}
\Msg{* file into a directory searched by TeX:}
\Msg{*}
\Msg{*     catchfile.sty}
\Msg{*}
\Msg{* To produce the documentation run the file `catchfile.drv'}
\Msg{* through LaTeX.}
\Msg{*}
\Msg{* Happy TeXing!}
\Msg{*}
\Msg{************************************************************************}

\endbatchfile
%</install>
%<*ignore>
\fi
%</ignore>
%<*driver>
\NeedsTeXFormat{LaTeX2e}
\ProvidesFile{catchfile.drv}%
  [2016/05/16 v1.7 Catch the contents of a file (HO)]%
\documentclass{ltxdoc}
\usepackage{holtxdoc}[2011/11/22]
\begin{document}
  \DocInput{catchfile.dtx}%
\end{document}
%</driver>
% \fi
%
%
% \CharacterTable
%  {Upper-case    \A\B\C\D\E\F\G\H\I\J\K\L\M\N\O\P\Q\R\S\T\U\V\W\X\Y\Z
%   Lower-case    \a\b\c\d\e\f\g\h\i\j\k\l\m\n\o\p\q\r\s\t\u\v\w\x\y\z
%   Digits        \0\1\2\3\4\5\6\7\8\9
%   Exclamation   \!     Double quote  \"     Hash (number) \#
%   Dollar        \$     Percent       \%     Ampersand     \&
%   Acute accent  \'     Left paren    \(     Right paren   \)
%   Asterisk      \*     Plus          \+     Comma         \,
%   Minus         \-     Point         \.     Solidus       \/
%   Colon         \:     Semicolon     \;     Less than     \<
%   Equals        \=     Greater than  \>     Question mark \?
%   Commercial at \@     Left bracket  \[     Backslash     \\
%   Right bracket \]     Circumflex    \^     Underscore    \_
%   Grave accent  \`     Left brace    \{     Vertical bar  \|
%   Right brace   \}     Tilde         \~}
%
% \GetFileInfo{catchfile.drv}
%
% \title{The \xpackage{catchfile} package}
% \date{2016/05/16 v1.7}
% \author{Heiko Oberdiek\thanks
% {Please report any issues at https://github.com/ho-tex/oberdiek/issues}\\
% \xemail{heiko.oberdiek at googlemail.com}}
%
% \maketitle
%
% \begin{abstract}
% This package catches the contents of a file and puts it in a macro.
% It requires \eTeX. Both \LaTeX\ and \plainTeX\ are supported.
% \end{abstract}
%
% \tableofcontents
%
% \section{Documentation}
%
% The package relies on \eTeX's \cs{everyeof}. Otherwise it
% aborts with an error message.
%
% \begin{declcs}{CatchFileDef} \M{cmd} \M{file name} \M{setup}\\
% \cs{CatchFileEdef} \M{cmd} \M{file name} \M{setup}
% \end{declcs}
% Macro \meta{cmd} is defined with the contents
% of file \meta{file name}.
% \cs{CatchFileDef} uses \cs{def}, \cs{CatchFileEdef} \cs{edef}
% for the definition.
% Additional setup code for setting
% catcodes or treatment of line ends can be given in code
% \meta{setup}. See the test files for an example.
%
% \StopEventually{
% }
%
% \section{Implementation}
%
%    \begin{macrocode}
%<*package>
%    \end{macrocode}
%
% \subsection{Reload check and package identification}
%    Reload check, especially if the package is not used with \LaTeX.
%    \begin{macrocode}
\begingroup\catcode61\catcode48\catcode32=10\relax%
  \catcode13=5 % ^^M
  \endlinechar=13 %
  \catcode35=6 % #
  \catcode39=12 % '
  \catcode44=12 % ,
  \catcode45=12 % -
  \catcode46=12 % .
  \catcode58=12 % :
  \catcode64=11 % @
  \catcode123=1 % {
  \catcode125=2 % }
  \expandafter\let\expandafter\x\csname ver@catchfile.sty\endcsname
  \ifx\x\relax % plain-TeX, first loading
  \else
    \def\empty{}%
    \ifx\x\empty % LaTeX, first loading,
      % variable is initialized, but \ProvidesPackage not yet seen
    \else
      \expandafter\ifx\csname PackageInfo\endcsname\relax
        \def\x#1#2{%
          \immediate\write-1{Package #1 Info: #2.}%
        }%
      \else
        \def\x#1#2{\PackageInfo{#1}{#2, stopped}}%
      \fi
      \x{catchfile}{The package is already loaded}%
      \aftergroup\endinput
    \fi
  \fi
\endgroup%
%    \end{macrocode}
%    Package identification:
%    \begin{macrocode}
\begingroup\catcode61\catcode48\catcode32=10\relax%
  \catcode13=5 % ^^M
  \endlinechar=13 %
  \catcode35=6 % #
  \catcode39=12 % '
  \catcode40=12 % (
  \catcode41=12 % )
  \catcode44=12 % ,
  \catcode45=12 % -
  \catcode46=12 % .
  \catcode47=12 % /
  \catcode58=12 % :
  \catcode64=11 % @
  \catcode91=12 % [
  \catcode93=12 % ]
  \catcode123=1 % {
  \catcode125=2 % }
  \expandafter\ifx\csname ProvidesPackage\endcsname\relax
    \def\x#1#2#3[#4]{\endgroup
      \immediate\write-1{Package: #3 #4}%
      \xdef#1{#4}%
    }%
  \else
    \def\x#1#2[#3]{\endgroup
      #2[{#3}]%
      \ifx#1\@undefined
        \xdef#1{#3}%
      \fi
      \ifx#1\relax
        \xdef#1{#3}%
      \fi
    }%
  \fi
\expandafter\x\csname ver@catchfile.sty\endcsname
\ProvidesPackage{catchfile}%
  [2016/05/16 v1.7 Catch the contents of a file (HO)]%
%    \end{macrocode}
%
% \subsection{Catcodes}
%
%    \begin{macrocode}
\begingroup\catcode61\catcode48\catcode32=10\relax%
  \catcode13=5 % ^^M
  \endlinechar=13 %
  \catcode123=1 % {
  \catcode125=2 % }
  \catcode64=11 % @
  \def\x{\endgroup
    \expandafter\edef\csname CatchFile@AtEnd\endcsname{%
      \endlinechar=\the\endlinechar\relax
      \catcode13=\the\catcode13\relax
      \catcode32=\the\catcode32\relax
      \catcode35=\the\catcode35\relax
      \catcode61=\the\catcode61\relax
      \catcode64=\the\catcode64\relax
      \catcode123=\the\catcode123\relax
      \catcode125=\the\catcode125\relax
    }%
  }%
\x\catcode61\catcode48\catcode32=10\relax%
\catcode13=5 % ^^M
\endlinechar=13 %
\catcode35=6 % #
\catcode64=11 % @
\catcode123=1 % {
\catcode125=2 % }
\def\TMP@EnsureCode#1#2{%
  \edef\CatchFile@AtEnd{%
    \CatchFile@AtEnd
    \catcode#1=\the\catcode#1\relax
  }%
  \catcode#1=#2\relax
}
\TMP@EnsureCode{39}{12}% '
\TMP@EnsureCode{44}{12}% ,
\TMP@EnsureCode{45}{12}% -
\TMP@EnsureCode{46}{12}% .
\TMP@EnsureCode{47}{12}% /
\TMP@EnsureCode{91}{12}% [
\TMP@EnsureCode{93}{12}% ]
\TMP@EnsureCode{96}{12}% `
\edef\CatchFile@AtEnd{\CatchFile@AtEnd\noexpand\endinput}
%    \end{macrocode}
%
% \subsection{Preparations}
%
%    \begin{macrocode}
\begingroup\expandafter\expandafter\expandafter\endgroup
\expandafter\ifx\csname RequirePackage\endcsname\relax
  \input infwarerr.sty\relax
  \input ltxcmds.sty\relax
\else
  \RequirePackage{infwarerr}[2007/09/09]%
  \RequirePackage{ltxcmds}[2010/03/09]%
\fi
%    \end{macrocode}
%
%    Check for \eTeX's \cs{everyeof}.
%    \begin{macrocode}
\begingroup
  \escapechar=92\relax
  \edef\TestString{\string\everyeof}%
  \edef\TestMeaning{\meaning\everyeof}%
  \ifx\TestString\TestMeaning
  \else
    \@PackageError{catchfile}{%
      Cannot find e-TeX's \string\everyeof,\MessageBreak
      package loading is aborted%
    }\@ehd
    \endgroup
    \expandafter\CatchFile@AtEnd
  \fi%
\endgroup
%    \end{macrocode}
%
% \subsection{Looking for primitive \cs{input}}
%
%    \begin{macro}{\CatchFile@Input}
%    The package needs the expandable primitive \cs{input}.
%    However there are formats that redefine it. For example,
%    \LaTeX's \cs{input} is not expandable, but it stores
%    the primitive in \cs{@@input}. The third possibility
%    is \cs{pdfprimitive}, introduced in \pdfTeX\ 1.40.0.
%
%    Thus we try to find the primitive and store it in
%    \cs{CatchFile@Input}. If it is used, it must be expanded
%    twice (because of the solution with \cs{pdfprimitive}).
%    \begin{macrocode}
\begingroup
  \def\Check#1#2#3#4\endgroup{%
    \edef\TestString{\string#1}%
    \edef\TestMeaning{\meaning#2}%
    \ifx\TestString\TestMeaning
      \endgroup
      \let\CatchFile@Primitive#2%
      \def\CatchFile@Input{\CatchFile@Primitive#3}%
    \else
      #4\endgroup
    \fi
  }%
  \Check\input\input{}%
  \Check\input\@@input{}%
  \Check\pdfprimitive\pdfprimitive\input
  \@PackageError{%
    Cannot find primitive \string\input,\MessageBreak
    package loading is aborted%
  }\@ehd
  \csname endgroup\endcsname
  \CatchFile@AtEnd%
\endgroup
%    \end{macrocode}
%    \end{macro}
%
% \subsection{Input file check}
%
%    \begin{macro}{\CatchFile@CheckFileExists}
%    \begin{macrocode}
\begingroup\expandafter\expandafter\expandafter\endgroup
\expandafter\ifx\csname IfFileExists\endcsname\relax
  \input pdftexcmds.sty\relax
  \begingroup\expandafter\expandafter\expandafter\endgroup
  \expandafter\ifx\csname pdf@filesize\endcsname\relax
    \def\CatchFile@CheckFileExists#1{%
      \expandafter\ifx\csname @inputcheck\endcsname\relax
        \csname newread\endcsname\@inputcheck
      \fi
      \openin\@inputcheck#1\relax
      \expandafter\closein\expandafter\@inputcheck
      \ifeof\@inputcheck
        \let\CatchFile@File\relax
      \else
        \def\CatchFile@File{#1}%
      \fi
    }%
  \else
    \def\CatchFile@CheckFileExists#1{%
      \expandafter\expandafter\expandafter\ifx
      \expandafter\expandafter\expandafter\relax\pdf@filesize{#1}\relax
        \let\CatchFile@File\relax
      \else
        \def\CatchFile@File{#1}%
      \fi
    }%
  \fi
\else
  \def\CatchFile@CheckFileExists#1{%
    \IfFileExists{#1}{%
      \expandafter\CatchFile@DefFile\@filef@und\@nil
      \begingroup\expandafter\expandafter\expandafter\endgroup
      \expandafter\ifx\csname @addtofilelist\endcsname\relax
      \else
        \@addtofilelist\CatchFile@File
      \fi
    }{%
      \let\CatchFile@File\relax
    }%
  }%
  \def\CatchFile@DefFile#1 \@nil{%
    \def\CatchFile@File{#1}%
  }%
\fi
%    \end{macrocode}
%    \end{macro}
%
%    \begin{macro}{\CatchFile@NotFound}
%    \begin{macrocode}
\def\CatchFile@NotFound#1#2{%
  \def#1{}%
  \@PackageError{catchfile}{%
    File `#2' not found%
  }\@ehc
}
%    \end{macrocode}
%    \end{macro}
%
% \subsection{Catch file contents}
%
%    \begin{macrocode}
\ltx@IfUndefined{RequirePackage}{%
  \input etexcmds.sty\relax
}{%
  \RequirePackage{etexcmds}[2010/01/28]%
}
%    \end{macrocode}
%
%    \begin{macro}{\CatchFileEdef}
%    \begin{macrocode}
\long\def\CatchFileEdef#1#2#3{%
  \CatchFile@CheckFileExists{#2}%
  \ifx\CatchFile@File\relax
    \CatchFile@NotFound{#1}{#2}%
  \else
    \begingroup
      \everyeof{\noexpand}%
      #3%
      \xdef\CatchFile@Contents{\CatchFile@Input\CatchFile@File\space}%
    \endgroup
    \let#1\CatchFile@Contents
  \fi
}
%    \end{macrocode}
%    \end{macro}
%
%    \begin{macro}{\CatchFileDef}
%    \begin{macrocode}
\long\def\CatchFileDef#1#2#3{%
  \CatchFile@CheckFileExists{#2}%
  \ifx\CatchFile@File\relax
    \CatchFile@NotFound{#1}{#2}%
  \else
    \begingroup
      \everyeof\expandafter{%
        \CatchFile@EOF
        \expandafter\CatchFile@Finish
        \noexpand
      }%
      \ltx@ifundefined{etex@unexpanded}{%
        \expandafter\long\expandafter\def\expandafter\CatchFile@Do
            \expandafter##\expandafter1\CatchFile@EOF{%
          \toks\ltx@zero{##1}%
          \xdef\CatchFile@gtemp{\the\toks\ltx@zero}%
          \def\CatchFile@Finish{%
            \endgroup
            \let#1\CatchFile@gtemp
            \global\let\CatchFile@gtemp\ltx@undefined
          }%
        }%
      }{%
        \expandafter\long\expandafter\def\expandafter\CatchFile@Do
            \expandafter##\expandafter1\CatchFile@EOF{%
          \edef\CatchFile@Finish{%
            \endgroup
            \etex@unexpanded{%
              \edef#1{\etex@unexpanded{##1}}%
            }%
          }%
        }%
      }%
      #3\relax
    \expandafter\expandafter\expandafter\CatchFile@Do
    \CatchFile@Input\CatchFile@File\relax
  \fi
}
%    \end{macrocode}
%    \cs{relax} after |#3| was added to make it more robust in
%    case the user uses something like
%\begin{verbatim}
%\CatchFileDef{\content}{\jobname.tt}{\endlinechar=-1}
%\end{verbatim}
%    that expands the following \cs{expandafter} after |#3| prematurely
%    (contribution of Martin Scharrer).
%    \end{macro}
%    \begin{macro}{\CatchFile@EOF}
%    If the file is read the catcode mappings are fixed.
%    This means that the same character cannot occur
%    inside the file with different catcodes.
%    Thus we use as end of file marker the at sign
%    twice with different catcodes.
%    \begin{macrocode}
\begingroup
  \lccode65=64 % lowercase('A') = '@'
  \lccode66=64 % lowercase('B') = '@'
  \catcode65=8 % catcode('A') = subscript
  \catcode66=3 % catcode('B') = math shift
\lowercase{\endgroup
  \def\CatchFile@EOF{AB}%
}
%    \end{macrocode}
%    \end{macro}
%
%    \begin{macrocode}
\CatchFile@AtEnd%
%</package>
%    \end{macrocode}
%
% \section{Test}
%
% \subsection{Catcode checks for loading}
%
%    \begin{macrocode}
%<*test1>
%    \end{macrocode}
%    \begin{macrocode}
\catcode`\{=1 %
\catcode`\}=2 %
\catcode`\#=6 %
\catcode`\@=11 %
\expandafter\ifx\csname count@\endcsname\relax
  \countdef\count@=255 %
\fi
\expandafter\ifx\csname @gobble\endcsname\relax
  \long\def\@gobble#1{}%
\fi
\expandafter\ifx\csname @firstofone\endcsname\relax
  \long\def\@firstofone#1{#1}%
\fi
\expandafter\ifx\csname loop\endcsname\relax
  \expandafter\@firstofone
\else
  \expandafter\@gobble
\fi
{%
  \def\loop#1\repeat{%
    \def\body{#1}%
    \iterate
  }%
  \def\iterate{%
    \body
      \let\next\iterate
    \else
      \let\next\relax
    \fi
    \next
  }%
  \let\repeat=\fi
}%
\def\RestoreCatcodes{}
\count@=0 %
\loop
  \edef\RestoreCatcodes{%
    \RestoreCatcodes
    \catcode\the\count@=\the\catcode\count@\relax
  }%
\ifnum\count@<255 %
  \advance\count@ 1 %
\repeat

\def\RangeCatcodeInvalid#1#2{%
  \count@=#1\relax
  \loop
    \catcode\count@=15 %
  \ifnum\count@<#2\relax
    \advance\count@ 1 %
  \repeat
}
\def\RangeCatcodeCheck#1#2#3{%
  \count@=#1\relax
  \loop
    \ifnum#3=\catcode\count@
    \else
      \errmessage{%
        Character \the\count@\space
        with wrong catcode \the\catcode\count@\space
        instead of \number#3%
      }%
    \fi
  \ifnum\count@<#2\relax
    \advance\count@ 1 %
  \repeat
}
\def\space{ }
\expandafter\ifx\csname LoadCommand\endcsname\relax
  \def\LoadCommand{\input catchfile.sty\relax}%
\fi
\def\Test{%
  \RangeCatcodeInvalid{0}{47}%
  \RangeCatcodeInvalid{58}{64}%
  \RangeCatcodeInvalid{91}{96}%
  \RangeCatcodeInvalid{123}{255}%
  \catcode`\@=12 %
  \catcode`\\=0 %
  \catcode`\%=14 %
  \LoadCommand
  \RangeCatcodeCheck{0}{36}{15}%
  \RangeCatcodeCheck{37}{37}{14}%
  \RangeCatcodeCheck{38}{47}{15}%
  \RangeCatcodeCheck{48}{57}{12}%
  \RangeCatcodeCheck{58}{63}{15}%
  \RangeCatcodeCheck{64}{64}{12}%
  \RangeCatcodeCheck{65}{90}{11}%
  \RangeCatcodeCheck{91}{91}{15}%
  \RangeCatcodeCheck{92}{92}{0}%
  \RangeCatcodeCheck{93}{96}{15}%
  \RangeCatcodeCheck{97}{122}{11}%
  \RangeCatcodeCheck{123}{255}{15}%
  \RestoreCatcodes
}
\Test
\csname @@end\endcsname
\end
%    \end{macrocode}
%    \begin{macrocode}
%</test1>
%    \end{macrocode}
%
% \subsection{\LaTeX}
%
%    \begin{macrocode}
%<*test2>
\NeedsTeXFormat{LaTeX2e}
\nofiles
\listfiles
\tracingnesting=2 %
\documentclass{minimal}
\usepackage{catchfile}[2016/05/16]
\makeatletter
\def\mysetup{%
  \let\do\@makeother
  \dospecials
}
\def\CheckContents{%
  \begingroup
    \expandafter\ifx\csname contents\endcsname\relax
      \PackageError{TEST}{\string\contents\space is not defined}\@ehc
    \fi
  \endgroup
}
\def\StartDisableUnexpanded{%
  \begingroup
    \let\etex@unexpanded\@undefined
}
\def\StopDisableUnexpanded{%
  \endgroup
}
\def\CheckCleanup{%
  \begingroup
    \edef\x{\the\toks0}%
    \def\y{ABC}%
    \ifx\x\y
    \else
      \PackageError{TEST}{\string\toks0 has changed}\@ehc
    \fi
    \ifx\CatchFile@gtemp\@undefined
    \else
      \PackageError{TEST}{Left over \string\CatchFile@gtemp}\@ehc
    \fi
  \endgroup
}
\def\Check{%
  \CheckContents
  \CheckCleanup
}
\makeatother
\begin{document}
\toks0{ABC}

\CatchFileDef\contents{catchfile.sty}\mysetup
\typeout{\meaning\contents}
\Check
\typeout{***********************************************************}%
\CatchFileDef\contents{catchfile.sty}{}
\typeout{\meaning\contents}
\Check
\typeout{***********************************************************}%

\StartDisableUnexpanded
\CatchFileDef\contents{catchfile.sty}\mysetup
\typeout{\meaning\contents}
\Check
\typeout{***********************************************************}%
\CatchFileDef\contents{catchfile.sty}{}
\typeout{\meaning\contents}
\Check
\typeout{***********************************************************}%
\StopDisableUnexpanded

\CatchFileEdef\contents{catchfile.sty}{%
  \mysetup
  \def\par{^^J}%
  \obeylines
}
\typeout{\contents}
\Check
\typeout{***********************************************************}%
\CatchFileEdef\contents{catchfile.sty}{%
  \catcode`\\=12 %
  \catcode`\#=12 %
  \def\par{^^J}%
  \obeylines
}
\typeout{\contents}
\Check
\typeout{***********************************************************}%
\end{document}
%</test2>
%    \end{macrocode}
%
% \subsection{\plainTeX}
%
%    \begin{macrocode}
%<*test3>
\def\msg#{\immediate\write16}
\newlinechar=10 %
\tracingnesting=2 %
\input catchfile.sty\relax

\def\mysetup{%
  \def\do##1{%
    \catcode`##1=12\relax
  }%
  \dospecials
}
\def\CheckContents{%
  \begingroup
    \expandafter\ifx\csname contents\endcsname\relax
      \def\space{ }%
      \errmessage{\string\contents\space is not defined.}%
    \fi
  \endgroup
}
\catcode`\@=11 %
\def\CheckCleanup{%
  \begingroup
    \edef\x{\the\toks0}%
    \def\y{ABC}%
    \ifx\x\y
    \else
      \errmessage{\string\toks0 has changed}%
    \fi
    \ifx\CatchFile@gtemp\@undefined
    \else
      \errmessage{Left over \string\CatchFile@gtemp}%
    \fi
  \endgroup
}
\def\Check{%
  \CheckContents
  \CheckCleanup
}
\def\StartDisableUnexpanded{%
  \begingroup
    \let\etex@unexpanded\@undefined
}
\def\StopDisableUnexpanded{%
  \endgroup
}
\catcode`\@=12 %

\toks0{ABC}

\CatchFileDef\contents{catchfile.sty}\mysetup
\msg{\meaning\contents}
\Check
\msg{***********************************************************}%
\CatchFileDef\contents{catchfile.sty}{}
\msg{\meaning\contents}
\Check
\msg{***********************************************************}%

\StartDisableUnexpanded
\CatchFileDef\contents{catchfile.sty}\mysetup
\msg{\meaning\contents}
\Check
\msg{***********************************************************}%
\CatchFileDef\contents{catchfile.sty}{}
\msg{\meaning\contents}
\Check
\msg{***********************************************************}%
\StopDisableUnexpanded

\CatchFileEdef\contents{catchfile.sty}{%
  \mysetup
  \def\par{^^J}%
  \obeylines
}
\msg{\contents}
\Check
\msg{***********************************************************}%
\CatchFileEdef\contents{catchfile.sty}{%
  \catcode`\\=12 %
  \catcode`\#=12 %
  \def\par{^^J}%
  \obeylines
}
\msg{\contents}
\Check
\msg{***********************************************************}%

\csname @@end\endcsname
\end
%</test3>
%    \end{macrocode}
%
% \section{Installation}
%
% \subsection{Download}
%
% \paragraph{Package.} This package is available on
% CTAN\footnote{\url{https://ctan.org/pkg/catchfile}}:
% \begin{description}
% \item[\CTAN{macros/latex/contrib/oberdiek/catchfile.dtx}] The source file.
% \item[\CTAN{macros/latex/contrib/oberdiek/catchfile.pdf}] Documentation.
% \end{description}
%
%
% \paragraph{Bundle.} All the packages of the bundle `oberdiek'
% are also available in a TDS compliant ZIP archive. There
% the packages are already unpacked and the documentation files
% are generated. The files and directories obey the TDS standard.
% \begin{description}
% \item[\CTANinstall{install/macros/latex/contrib/oberdiek.tds.zip}]
% \end{description}
% \emph{TDS} refers to the standard ``A Directory Structure
% for \TeX\ Files'' (\CTAN{tds/tds.pdf}). Directories
% with \xfile{texmf} in their name are usually organized this way.
%
% \subsection{Bundle installation}
%
% \paragraph{Unpacking.} Unpack the \xfile{oberdiek.tds.zip} in the
% TDS tree (also known as \xfile{texmf} tree) of your choice.
% Example (linux):
% \begin{quote}
%   |unzip oberdiek.tds.zip -d ~/texmf|
% \end{quote}
%
% \paragraph{Script installation.}
% Check the directory \xfile{TDS:scripts/oberdiek/} for
% scripts that need further installation steps.
% Package \xpackage{attachfile2} comes with the Perl script
% \xfile{pdfatfi.pl} that should be installed in such a way
% that it can be called as \texttt{pdfatfi}.
% Example (linux):
% \begin{quote}
%   |chmod +x scripts/oberdiek/pdfatfi.pl|\\
%   |cp scripts/oberdiek/pdfatfi.pl /usr/local/bin/|
% \end{quote}
%
% \subsection{Package installation}
%
% \paragraph{Unpacking.} The \xfile{.dtx} file is a self-extracting
% \docstrip\ archive. The files are extracted by running the
% \xfile{.dtx} through \plainTeX:
% \begin{quote}
%   \verb|tex catchfile.dtx|
% \end{quote}
%
% \paragraph{TDS.} Now the different files must be moved into
% the different directories in your installation TDS tree
% (also known as \xfile{texmf} tree):
% \begin{quote}
% \def\t{^^A
% \begin{tabular}{@{}>{\ttfamily}l@{ $\rightarrow$ }>{\ttfamily}l@{}}
%   catchfile.sty & tex/generic/oberdiek/catchfile.sty\\
%   catchfile.pdf & doc/latex/oberdiek/catchfile.pdf\\
%   test/catchfile-test1.tex & doc/latex/oberdiek/test/catchfile-test1.tex\\
%   test/catchfile-test2.tex & doc/latex/oberdiek/test/catchfile-test2.tex\\
%   test/catchfile-test3.tex & doc/latex/oberdiek/test/catchfile-test3.tex\\
%   catchfile.dtx & source/latex/oberdiek/catchfile.dtx\\
% \end{tabular}^^A
% }^^A
% \sbox0{\t}^^A
% \ifdim\wd0>\linewidth
%   \begingroup
%     \advance\linewidth by\leftmargin
%     \advance\linewidth by\rightmargin
%   \edef\x{\endgroup
%     \def\noexpand\lw{\the\linewidth}^^A
%   }\x
%   \def\lwbox{^^A
%     \leavevmode
%     \hbox to \linewidth{^^A
%       \kern-\leftmargin\relax
%       \hss
%       \usebox0
%       \hss
%       \kern-\rightmargin\relax
%     }^^A
%   }^^A
%   \ifdim\wd0>\lw
%     \sbox0{\small\t}^^A
%     \ifdim\wd0>\linewidth
%       \ifdim\wd0>\lw
%         \sbox0{\footnotesize\t}^^A
%         \ifdim\wd0>\linewidth
%           \ifdim\wd0>\lw
%             \sbox0{\scriptsize\t}^^A
%             \ifdim\wd0>\linewidth
%               \ifdim\wd0>\lw
%                 \sbox0{\tiny\t}^^A
%                 \ifdim\wd0>\linewidth
%                   \lwbox
%                 \else
%                   \usebox0
%                 \fi
%               \else
%                 \lwbox
%               \fi
%             \else
%               \usebox0
%             \fi
%           \else
%             \lwbox
%           \fi
%         \else
%           \usebox0
%         \fi
%       \else
%         \lwbox
%       \fi
%     \else
%       \usebox0
%     \fi
%   \else
%     \lwbox
%   \fi
% \else
%   \usebox0
% \fi
% \end{quote}
% If you have a \xfile{docstrip.cfg} that configures and enables \docstrip's
% TDS installing feature, then some files can already be in the right
% place, see the documentation of \docstrip.
%
% \subsection{Refresh file name databases}
%
% If your \TeX~distribution
% (\teTeX, \mikTeX, \dots) relies on file name databases, you must refresh
% these. For example, \teTeX\ users run \verb|texhash| or
% \verb|mktexlsr|.
%
% \subsection{Some details for the interested}
%
% \paragraph{Attached source.}
%
% The PDF documentation on CTAN also includes the
% \xfile{.dtx} source file. It can be extracted by
% AcrobatReader 6 or higher. Another option is \textsf{pdftk},
% e.g. unpack the file into the current directory:
% \begin{quote}
%   \verb|pdftk catchfile.pdf unpack_files output .|
% \end{quote}
%
% \paragraph{Unpacking with \LaTeX.}
% The \xfile{.dtx} chooses its action depending on the format:
% \begin{description}
% \item[\plainTeX:] Run \docstrip\ and extract the files.
% \item[\LaTeX:] Generate the documentation.
% \end{description}
% If you insist on using \LaTeX\ for \docstrip\ (really,
% \docstrip\ does not need \LaTeX), then inform the autodetect routine
% about your intention:
% \begin{quote}
%   \verb|latex \let\install=y% \iffalse meta-comment
%
% File: catchfile.dtx
% Version: 2016/05/16 v1.7
% Info: Catch the contents of a file
%
% Copyright (C) 2007, 2010, 2011 by
%    Heiko Oberdiek <heiko.oberdiek at googlemail.com>
%    2016
%    https://github.com/ho-tex/oberdiek/issues
%
% This work may be distributed and/or modified under the
% conditions of the LaTeX Project Public License, either
% version 1.3c of this license or (at your option) any later
% version. This version of this license is in
%    http://www.latex-project.org/lppl/lppl-1-3c.txt
% and the latest version of this license is in
%    http://www.latex-project.org/lppl.txt
% and version 1.3 or later is part of all distributions of
% LaTeX version 2005/12/01 or later.
%
% This work has the LPPL maintenance status "maintained".
%
% This Current Maintainer of this work is Heiko Oberdiek.
%
% The Base Interpreter refers to any `TeX-Format',
% because some files are installed in TDS:tex/generic//.
%
% This work consists of the main source file catchfile.dtx
% and the derived files
%    catchfile.sty, catchfile.pdf, catchfile.ins, catchfile.drv,
%    catchfile-test1.tex, catchfile-test2.tex,
%    catchfile-test3.tex.
%
% Distribution:
%    CTAN:macros/latex/contrib/oberdiek/catchfile.dtx
%    CTAN:macros/latex/contrib/oberdiek/catchfile.pdf
%
% Unpacking:
%    (a) If catchfile.ins is present:
%           tex catchfile.ins
%    (b) Without catchfile.ins:
%           tex catchfile.dtx
%    (c) If you insist on using LaTeX
%           latex \let\install=y\input{catchfile.dtx}
%        (quote the arguments according to the demands of your shell)
%
% Documentation:
%    (a) If catchfile.drv is present:
%           latex catchfile.drv
%    (b) Without catchfile.drv:
%           latex catchfile.dtx; ...
%    The class ltxdoc loads the configuration file ltxdoc.cfg
%    if available. Here you can specify further options, e.g.
%    use A4 as paper format:
%       \PassOptionsToClass{a4paper}{article}
%
%    Programm calls to get the documentation (example):
%       pdflatex catchfile.dtx
%       makeindex -s gind.ist catchfile.idx
%       pdflatex catchfile.dtx
%       makeindex -s gind.ist catchfile.idx
%       pdflatex catchfile.dtx
%
% Installation:
%    TDS:tex/generic/oberdiek/catchfile.sty
%    TDS:doc/latex/oberdiek/catchfile.pdf
%    TDS:doc/latex/oberdiek/test/catchfile-test1.tex
%    TDS:doc/latex/oberdiek/test/catchfile-test2.tex
%    TDS:doc/latex/oberdiek/test/catchfile-test3.tex
%    TDS:source/latex/oberdiek/catchfile.dtx
%
%<*ignore>
\begingroup
  \catcode123=1 %
  \catcode125=2 %
  \def\x{LaTeX2e}%
\expandafter\endgroup
\ifcase 0\ifx\install y1\fi\expandafter
         \ifx\csname processbatchFile\endcsname\relax\else1\fi
         \ifx\fmtname\x\else 1\fi\relax
\else\csname fi\endcsname
%</ignore>
%<*install>
\input docstrip.tex
\Msg{************************************************************************}
\Msg{* Installation}
\Msg{* Package: catchfile 2016/05/16 v1.7 Catch the contents of a file (HO)}
\Msg{************************************************************************}

\keepsilent
\askforoverwritefalse

\let\MetaPrefix\relax
\preamble

This is a generated file.

Project: catchfile
Version: 2016/05/16 v1.7

Copyright (C) 2007, 2010, 2011 by
   Heiko Oberdiek <heiko.oberdiek at googlemail.com>

This work may be distributed and/or modified under the
conditions of the LaTeX Project Public License, either
version 1.3c of this license or (at your option) any later
version. This version of this license is in
   http://www.latex-project.org/lppl/lppl-1-3c.txt
and the latest version of this license is in
   http://www.latex-project.org/lppl.txt
and version 1.3 or later is part of all distributions of
LaTeX version 2005/12/01 or later.

This work has the LPPL maintenance status "maintained".

This Current Maintainer of this work is Heiko Oberdiek.

The Base Interpreter refers to any `TeX-Format',
because some files are installed in TDS:tex/generic//.

This work consists of the main source file catchfile.dtx
and the derived files
   catchfile.sty, catchfile.pdf, catchfile.ins, catchfile.drv,
   catchfile-test1.tex, catchfile-test2.tex,
   catchfile-test3.tex.

\endpreamble
\let\MetaPrefix\DoubleperCent

\generate{%
  \file{catchfile.ins}{\from{catchfile.dtx}{install}}%
  \file{catchfile.drv}{\from{catchfile.dtx}{driver}}%
  \usedir{tex/generic/oberdiek}%
  \file{catchfile.sty}{\from{catchfile.dtx}{package}}%
%  \usedir{doc/latex/oberdiek/test}%
%  \file{catchfile-test1.tex}{\from{catchfile.dtx}{test1}}%
%  \file{catchfile-test2.tex}{\from{catchfile.dtx}{test2}}%
%  \file{catchfile-test3.tex}{\from{catchfile.dtx}{test3}}%
  \nopreamble
  \nopostamble
%  \usedir{source/latex/oberdiek/catalogue}%
%  \file{catchfile.xml}{\from{catchfile.dtx}{catalogue}}%
}

\catcode32=13\relax% active space
\let =\space%
\Msg{************************************************************************}
\Msg{*}
\Msg{* To finish the installation you have to move the following}
\Msg{* file into a directory searched by TeX:}
\Msg{*}
\Msg{*     catchfile.sty}
\Msg{*}
\Msg{* To produce the documentation run the file `catchfile.drv'}
\Msg{* through LaTeX.}
\Msg{*}
\Msg{* Happy TeXing!}
\Msg{*}
\Msg{************************************************************************}

\endbatchfile
%</install>
%<*ignore>
\fi
%</ignore>
%<*driver>
\NeedsTeXFormat{LaTeX2e}
\ProvidesFile{catchfile.drv}%
  [2016/05/16 v1.7 Catch the contents of a file (HO)]%
\documentclass{ltxdoc}
\usepackage{holtxdoc}[2011/11/22]
\begin{document}
  \DocInput{catchfile.dtx}%
\end{document}
%</driver>
% \fi
%
%
% \CharacterTable
%  {Upper-case    \A\B\C\D\E\F\G\H\I\J\K\L\M\N\O\P\Q\R\S\T\U\V\W\X\Y\Z
%   Lower-case    \a\b\c\d\e\f\g\h\i\j\k\l\m\n\o\p\q\r\s\t\u\v\w\x\y\z
%   Digits        \0\1\2\3\4\5\6\7\8\9
%   Exclamation   \!     Double quote  \"     Hash (number) \#
%   Dollar        \$     Percent       \%     Ampersand     \&
%   Acute accent  \'     Left paren    \(     Right paren   \)
%   Asterisk      \*     Plus          \+     Comma         \,
%   Minus         \-     Point         \.     Solidus       \/
%   Colon         \:     Semicolon     \;     Less than     \<
%   Equals        \=     Greater than  \>     Question mark \?
%   Commercial at \@     Left bracket  \[     Backslash     \\
%   Right bracket \]     Circumflex    \^     Underscore    \_
%   Grave accent  \`     Left brace    \{     Vertical bar  \|
%   Right brace   \}     Tilde         \~}
%
% \GetFileInfo{catchfile.drv}
%
% \title{The \xpackage{catchfile} package}
% \date{2016/05/16 v1.7}
% \author{Heiko Oberdiek\thanks
% {Please report any issues at https://github.com/ho-tex/oberdiek/issues}\\
% \xemail{heiko.oberdiek at googlemail.com}}
%
% \maketitle
%
% \begin{abstract}
% This package catches the contents of a file and puts it in a macro.
% It requires \eTeX. Both \LaTeX\ and \plainTeX\ are supported.
% \end{abstract}
%
% \tableofcontents
%
% \section{Documentation}
%
% The package relies on \eTeX's \cs{everyeof}. Otherwise it
% aborts with an error message.
%
% \begin{declcs}{CatchFileDef} \M{cmd} \M{file name} \M{setup}\\
% \cs{CatchFileEdef} \M{cmd} \M{file name} \M{setup}
% \end{declcs}
% Macro \meta{cmd} is defined with the contents
% of file \meta{file name}.
% \cs{CatchFileDef} uses \cs{def}, \cs{CatchFileEdef} \cs{edef}
% for the definition.
% Additional setup code for setting
% catcodes or treatment of line ends can be given in code
% \meta{setup}. See the test files for an example.
%
% \StopEventually{
% }
%
% \section{Implementation}
%
%    \begin{macrocode}
%<*package>
%    \end{macrocode}
%
% \subsection{Reload check and package identification}
%    Reload check, especially if the package is not used with \LaTeX.
%    \begin{macrocode}
\begingroup\catcode61\catcode48\catcode32=10\relax%
  \catcode13=5 % ^^M
  \endlinechar=13 %
  \catcode35=6 % #
  \catcode39=12 % '
  \catcode44=12 % ,
  \catcode45=12 % -
  \catcode46=12 % .
  \catcode58=12 % :
  \catcode64=11 % @
  \catcode123=1 % {
  \catcode125=2 % }
  \expandafter\let\expandafter\x\csname ver@catchfile.sty\endcsname
  \ifx\x\relax % plain-TeX, first loading
  \else
    \def\empty{}%
    \ifx\x\empty % LaTeX, first loading,
      % variable is initialized, but \ProvidesPackage not yet seen
    \else
      \expandafter\ifx\csname PackageInfo\endcsname\relax
        \def\x#1#2{%
          \immediate\write-1{Package #1 Info: #2.}%
        }%
      \else
        \def\x#1#2{\PackageInfo{#1}{#2, stopped}}%
      \fi
      \x{catchfile}{The package is already loaded}%
      \aftergroup\endinput
    \fi
  \fi
\endgroup%
%    \end{macrocode}
%    Package identification:
%    \begin{macrocode}
\begingroup\catcode61\catcode48\catcode32=10\relax%
  \catcode13=5 % ^^M
  \endlinechar=13 %
  \catcode35=6 % #
  \catcode39=12 % '
  \catcode40=12 % (
  \catcode41=12 % )
  \catcode44=12 % ,
  \catcode45=12 % -
  \catcode46=12 % .
  \catcode47=12 % /
  \catcode58=12 % :
  \catcode64=11 % @
  \catcode91=12 % [
  \catcode93=12 % ]
  \catcode123=1 % {
  \catcode125=2 % }
  \expandafter\ifx\csname ProvidesPackage\endcsname\relax
    \def\x#1#2#3[#4]{\endgroup
      \immediate\write-1{Package: #3 #4}%
      \xdef#1{#4}%
    }%
  \else
    \def\x#1#2[#3]{\endgroup
      #2[{#3}]%
      \ifx#1\@undefined
        \xdef#1{#3}%
      \fi
      \ifx#1\relax
        \xdef#1{#3}%
      \fi
    }%
  \fi
\expandafter\x\csname ver@catchfile.sty\endcsname
\ProvidesPackage{catchfile}%
  [2016/05/16 v1.7 Catch the contents of a file (HO)]%
%    \end{macrocode}
%
% \subsection{Catcodes}
%
%    \begin{macrocode}
\begingroup\catcode61\catcode48\catcode32=10\relax%
  \catcode13=5 % ^^M
  \endlinechar=13 %
  \catcode123=1 % {
  \catcode125=2 % }
  \catcode64=11 % @
  \def\x{\endgroup
    \expandafter\edef\csname CatchFile@AtEnd\endcsname{%
      \endlinechar=\the\endlinechar\relax
      \catcode13=\the\catcode13\relax
      \catcode32=\the\catcode32\relax
      \catcode35=\the\catcode35\relax
      \catcode61=\the\catcode61\relax
      \catcode64=\the\catcode64\relax
      \catcode123=\the\catcode123\relax
      \catcode125=\the\catcode125\relax
    }%
  }%
\x\catcode61\catcode48\catcode32=10\relax%
\catcode13=5 % ^^M
\endlinechar=13 %
\catcode35=6 % #
\catcode64=11 % @
\catcode123=1 % {
\catcode125=2 % }
\def\TMP@EnsureCode#1#2{%
  \edef\CatchFile@AtEnd{%
    \CatchFile@AtEnd
    \catcode#1=\the\catcode#1\relax
  }%
  \catcode#1=#2\relax
}
\TMP@EnsureCode{39}{12}% '
\TMP@EnsureCode{44}{12}% ,
\TMP@EnsureCode{45}{12}% -
\TMP@EnsureCode{46}{12}% .
\TMP@EnsureCode{47}{12}% /
\TMP@EnsureCode{91}{12}% [
\TMP@EnsureCode{93}{12}% ]
\TMP@EnsureCode{96}{12}% `
\edef\CatchFile@AtEnd{\CatchFile@AtEnd\noexpand\endinput}
%    \end{macrocode}
%
% \subsection{Preparations}
%
%    \begin{macrocode}
\begingroup\expandafter\expandafter\expandafter\endgroup
\expandafter\ifx\csname RequirePackage\endcsname\relax
  \input infwarerr.sty\relax
  \input ltxcmds.sty\relax
\else
  \RequirePackage{infwarerr}[2007/09/09]%
  \RequirePackage{ltxcmds}[2010/03/09]%
\fi
%    \end{macrocode}
%
%    Check for \eTeX's \cs{everyeof}.
%    \begin{macrocode}
\begingroup
  \escapechar=92\relax
  \edef\TestString{\string\everyeof}%
  \edef\TestMeaning{\meaning\everyeof}%
  \ifx\TestString\TestMeaning
  \else
    \@PackageError{catchfile}{%
      Cannot find e-TeX's \string\everyeof,\MessageBreak
      package loading is aborted%
    }\@ehd
    \endgroup
    \expandafter\CatchFile@AtEnd
  \fi%
\endgroup
%    \end{macrocode}
%
% \subsection{Looking for primitive \cs{input}}
%
%    \begin{macro}{\CatchFile@Input}
%    The package needs the expandable primitive \cs{input}.
%    However there are formats that redefine it. For example,
%    \LaTeX's \cs{input} is not expandable, but it stores
%    the primitive in \cs{@@input}. The third possibility
%    is \cs{pdfprimitive}, introduced in \pdfTeX\ 1.40.0.
%
%    Thus we try to find the primitive and store it in
%    \cs{CatchFile@Input}. If it is used, it must be expanded
%    twice (because of the solution with \cs{pdfprimitive}).
%    \begin{macrocode}
\begingroup
  \def\Check#1#2#3#4\endgroup{%
    \edef\TestString{\string#1}%
    \edef\TestMeaning{\meaning#2}%
    \ifx\TestString\TestMeaning
      \endgroup
      \let\CatchFile@Primitive#2%
      \def\CatchFile@Input{\CatchFile@Primitive#3}%
    \else
      #4\endgroup
    \fi
  }%
  \Check\input\input{}%
  \Check\input\@@input{}%
  \Check\pdfprimitive\pdfprimitive\input
  \@PackageError{%
    Cannot find primitive \string\input,\MessageBreak
    package loading is aborted%
  }\@ehd
  \csname endgroup\endcsname
  \CatchFile@AtEnd%
\endgroup
%    \end{macrocode}
%    \end{macro}
%
% \subsection{Input file check}
%
%    \begin{macro}{\CatchFile@CheckFileExists}
%    \begin{macrocode}
\begingroup\expandafter\expandafter\expandafter\endgroup
\expandafter\ifx\csname IfFileExists\endcsname\relax
  \input pdftexcmds.sty\relax
  \begingroup\expandafter\expandafter\expandafter\endgroup
  \expandafter\ifx\csname pdf@filesize\endcsname\relax
    \def\CatchFile@CheckFileExists#1{%
      \expandafter\ifx\csname @inputcheck\endcsname\relax
        \csname newread\endcsname\@inputcheck
      \fi
      \openin\@inputcheck#1\relax
      \expandafter\closein\expandafter\@inputcheck
      \ifeof\@inputcheck
        \let\CatchFile@File\relax
      \else
        \def\CatchFile@File{#1}%
      \fi
    }%
  \else
    \def\CatchFile@CheckFileExists#1{%
      \expandafter\expandafter\expandafter\ifx
      \expandafter\expandafter\expandafter\relax\pdf@filesize{#1}\relax
        \let\CatchFile@File\relax
      \else
        \def\CatchFile@File{#1}%
      \fi
    }%
  \fi
\else
  \def\CatchFile@CheckFileExists#1{%
    \IfFileExists{#1}{%
      \expandafter\CatchFile@DefFile\@filef@und\@nil
      \begingroup\expandafter\expandafter\expandafter\endgroup
      \expandafter\ifx\csname @addtofilelist\endcsname\relax
      \else
        \@addtofilelist\CatchFile@File
      \fi
    }{%
      \let\CatchFile@File\relax
    }%
  }%
  \def\CatchFile@DefFile#1 \@nil{%
    \def\CatchFile@File{#1}%
  }%
\fi
%    \end{macrocode}
%    \end{macro}
%
%    \begin{macro}{\CatchFile@NotFound}
%    \begin{macrocode}
\def\CatchFile@NotFound#1#2{%
  \def#1{}%
  \@PackageError{catchfile}{%
    File `#2' not found%
  }\@ehc
}
%    \end{macrocode}
%    \end{macro}
%
% \subsection{Catch file contents}
%
%    \begin{macrocode}
\ltx@IfUndefined{RequirePackage}{%
  \input etexcmds.sty\relax
}{%
  \RequirePackage{etexcmds}[2010/01/28]%
}
%    \end{macrocode}
%
%    \begin{macro}{\CatchFileEdef}
%    \begin{macrocode}
\long\def\CatchFileEdef#1#2#3{%
  \CatchFile@CheckFileExists{#2}%
  \ifx\CatchFile@File\relax
    \CatchFile@NotFound{#1}{#2}%
  \else
    \begingroup
      \everyeof{\noexpand}%
      #3%
      \xdef\CatchFile@Contents{\CatchFile@Input\CatchFile@File\space}%
    \endgroup
    \let#1\CatchFile@Contents
  \fi
}
%    \end{macrocode}
%    \end{macro}
%
%    \begin{macro}{\CatchFileDef}
%    \begin{macrocode}
\long\def\CatchFileDef#1#2#3{%
  \CatchFile@CheckFileExists{#2}%
  \ifx\CatchFile@File\relax
    \CatchFile@NotFound{#1}{#2}%
  \else
    \begingroup
      \everyeof\expandafter{%
        \CatchFile@EOF
        \expandafter\CatchFile@Finish
        \noexpand
      }%
      \ltx@ifundefined{etex@unexpanded}{%
        \expandafter\long\expandafter\def\expandafter\CatchFile@Do
            \expandafter##\expandafter1\CatchFile@EOF{%
          \toks\ltx@zero{##1}%
          \xdef\CatchFile@gtemp{\the\toks\ltx@zero}%
          \def\CatchFile@Finish{%
            \endgroup
            \let#1\CatchFile@gtemp
            \global\let\CatchFile@gtemp\ltx@undefined
          }%
        }%
      }{%
        \expandafter\long\expandafter\def\expandafter\CatchFile@Do
            \expandafter##\expandafter1\CatchFile@EOF{%
          \edef\CatchFile@Finish{%
            \endgroup
            \etex@unexpanded{%
              \edef#1{\etex@unexpanded{##1}}%
            }%
          }%
        }%
      }%
      #3\relax
    \expandafter\expandafter\expandafter\CatchFile@Do
    \CatchFile@Input\CatchFile@File\relax
  \fi
}
%    \end{macrocode}
%    \cs{relax} after |#3| was added to make it more robust in
%    case the user uses something like
%\begin{verbatim}
%\CatchFileDef{\content}{\jobname.tt}{\endlinechar=-1}
%\end{verbatim}
%    that expands the following \cs{expandafter} after |#3| prematurely
%    (contribution of Martin Scharrer).
%    \end{macro}
%    \begin{macro}{\CatchFile@EOF}
%    If the file is read the catcode mappings are fixed.
%    This means that the same character cannot occur
%    inside the file with different catcodes.
%    Thus we use as end of file marker the at sign
%    twice with different catcodes.
%    \begin{macrocode}
\begingroup
  \lccode65=64 % lowercase('A') = '@'
  \lccode66=64 % lowercase('B') = '@'
  \catcode65=8 % catcode('A') = subscript
  \catcode66=3 % catcode('B') = math shift
\lowercase{\endgroup
  \def\CatchFile@EOF{AB}%
}
%    \end{macrocode}
%    \end{macro}
%
%    \begin{macrocode}
\CatchFile@AtEnd%
%</package>
%    \end{macrocode}
%
% \section{Test}
%
% \subsection{Catcode checks for loading}
%
%    \begin{macrocode}
%<*test1>
%    \end{macrocode}
%    \begin{macrocode}
\catcode`\{=1 %
\catcode`\}=2 %
\catcode`\#=6 %
\catcode`\@=11 %
\expandafter\ifx\csname count@\endcsname\relax
  \countdef\count@=255 %
\fi
\expandafter\ifx\csname @gobble\endcsname\relax
  \long\def\@gobble#1{}%
\fi
\expandafter\ifx\csname @firstofone\endcsname\relax
  \long\def\@firstofone#1{#1}%
\fi
\expandafter\ifx\csname loop\endcsname\relax
  \expandafter\@firstofone
\else
  \expandafter\@gobble
\fi
{%
  \def\loop#1\repeat{%
    \def\body{#1}%
    \iterate
  }%
  \def\iterate{%
    \body
      \let\next\iterate
    \else
      \let\next\relax
    \fi
    \next
  }%
  \let\repeat=\fi
}%
\def\RestoreCatcodes{}
\count@=0 %
\loop
  \edef\RestoreCatcodes{%
    \RestoreCatcodes
    \catcode\the\count@=\the\catcode\count@\relax
  }%
\ifnum\count@<255 %
  \advance\count@ 1 %
\repeat

\def\RangeCatcodeInvalid#1#2{%
  \count@=#1\relax
  \loop
    \catcode\count@=15 %
  \ifnum\count@<#2\relax
    \advance\count@ 1 %
  \repeat
}
\def\RangeCatcodeCheck#1#2#3{%
  \count@=#1\relax
  \loop
    \ifnum#3=\catcode\count@
    \else
      \errmessage{%
        Character \the\count@\space
        with wrong catcode \the\catcode\count@\space
        instead of \number#3%
      }%
    \fi
  \ifnum\count@<#2\relax
    \advance\count@ 1 %
  \repeat
}
\def\space{ }
\expandafter\ifx\csname LoadCommand\endcsname\relax
  \def\LoadCommand{\input catchfile.sty\relax}%
\fi
\def\Test{%
  \RangeCatcodeInvalid{0}{47}%
  \RangeCatcodeInvalid{58}{64}%
  \RangeCatcodeInvalid{91}{96}%
  \RangeCatcodeInvalid{123}{255}%
  \catcode`\@=12 %
  \catcode`\\=0 %
  \catcode`\%=14 %
  \LoadCommand
  \RangeCatcodeCheck{0}{36}{15}%
  \RangeCatcodeCheck{37}{37}{14}%
  \RangeCatcodeCheck{38}{47}{15}%
  \RangeCatcodeCheck{48}{57}{12}%
  \RangeCatcodeCheck{58}{63}{15}%
  \RangeCatcodeCheck{64}{64}{12}%
  \RangeCatcodeCheck{65}{90}{11}%
  \RangeCatcodeCheck{91}{91}{15}%
  \RangeCatcodeCheck{92}{92}{0}%
  \RangeCatcodeCheck{93}{96}{15}%
  \RangeCatcodeCheck{97}{122}{11}%
  \RangeCatcodeCheck{123}{255}{15}%
  \RestoreCatcodes
}
\Test
\csname @@end\endcsname
\end
%    \end{macrocode}
%    \begin{macrocode}
%</test1>
%    \end{macrocode}
%
% \subsection{\LaTeX}
%
%    \begin{macrocode}
%<*test2>
\NeedsTeXFormat{LaTeX2e}
\nofiles
\listfiles
\tracingnesting=2 %
\documentclass{minimal}
\usepackage{catchfile}[2016/05/16]
\makeatletter
\def\mysetup{%
  \let\do\@makeother
  \dospecials
}
\def\CheckContents{%
  \begingroup
    \expandafter\ifx\csname contents\endcsname\relax
      \PackageError{TEST}{\string\contents\space is not defined}\@ehc
    \fi
  \endgroup
}
\def\StartDisableUnexpanded{%
  \begingroup
    \let\etex@unexpanded\@undefined
}
\def\StopDisableUnexpanded{%
  \endgroup
}
\def\CheckCleanup{%
  \begingroup
    \edef\x{\the\toks0}%
    \def\y{ABC}%
    \ifx\x\y
    \else
      \PackageError{TEST}{\string\toks0 has changed}\@ehc
    \fi
    \ifx\CatchFile@gtemp\@undefined
    \else
      \PackageError{TEST}{Left over \string\CatchFile@gtemp}\@ehc
    \fi
  \endgroup
}
\def\Check{%
  \CheckContents
  \CheckCleanup
}
\makeatother
\begin{document}
\toks0{ABC}

\CatchFileDef\contents{catchfile.sty}\mysetup
\typeout{\meaning\contents}
\Check
\typeout{***********************************************************}%
\CatchFileDef\contents{catchfile.sty}{}
\typeout{\meaning\contents}
\Check
\typeout{***********************************************************}%

\StartDisableUnexpanded
\CatchFileDef\contents{catchfile.sty}\mysetup
\typeout{\meaning\contents}
\Check
\typeout{***********************************************************}%
\CatchFileDef\contents{catchfile.sty}{}
\typeout{\meaning\contents}
\Check
\typeout{***********************************************************}%
\StopDisableUnexpanded

\CatchFileEdef\contents{catchfile.sty}{%
  \mysetup
  \def\par{^^J}%
  \obeylines
}
\typeout{\contents}
\Check
\typeout{***********************************************************}%
\CatchFileEdef\contents{catchfile.sty}{%
  \catcode`\\=12 %
  \catcode`\#=12 %
  \def\par{^^J}%
  \obeylines
}
\typeout{\contents}
\Check
\typeout{***********************************************************}%
\end{document}
%</test2>
%    \end{macrocode}
%
% \subsection{\plainTeX}
%
%    \begin{macrocode}
%<*test3>
\def\msg#{\immediate\write16}
\newlinechar=10 %
\tracingnesting=2 %
\input catchfile.sty\relax

\def\mysetup{%
  \def\do##1{%
    \catcode`##1=12\relax
  }%
  \dospecials
}
\def\CheckContents{%
  \begingroup
    \expandafter\ifx\csname contents\endcsname\relax
      \def\space{ }%
      \errmessage{\string\contents\space is not defined.}%
    \fi
  \endgroup
}
\catcode`\@=11 %
\def\CheckCleanup{%
  \begingroup
    \edef\x{\the\toks0}%
    \def\y{ABC}%
    \ifx\x\y
    \else
      \errmessage{\string\toks0 has changed}%
    \fi
    \ifx\CatchFile@gtemp\@undefined
    \else
      \errmessage{Left over \string\CatchFile@gtemp}%
    \fi
  \endgroup
}
\def\Check{%
  \CheckContents
  \CheckCleanup
}
\def\StartDisableUnexpanded{%
  \begingroup
    \let\etex@unexpanded\@undefined
}
\def\StopDisableUnexpanded{%
  \endgroup
}
\catcode`\@=12 %

\toks0{ABC}

\CatchFileDef\contents{catchfile.sty}\mysetup
\msg{\meaning\contents}
\Check
\msg{***********************************************************}%
\CatchFileDef\contents{catchfile.sty}{}
\msg{\meaning\contents}
\Check
\msg{***********************************************************}%

\StartDisableUnexpanded
\CatchFileDef\contents{catchfile.sty}\mysetup
\msg{\meaning\contents}
\Check
\msg{***********************************************************}%
\CatchFileDef\contents{catchfile.sty}{}
\msg{\meaning\contents}
\Check
\msg{***********************************************************}%
\StopDisableUnexpanded

\CatchFileEdef\contents{catchfile.sty}{%
  \mysetup
  \def\par{^^J}%
  \obeylines
}
\msg{\contents}
\Check
\msg{***********************************************************}%
\CatchFileEdef\contents{catchfile.sty}{%
  \catcode`\\=12 %
  \catcode`\#=12 %
  \def\par{^^J}%
  \obeylines
}
\msg{\contents}
\Check
\msg{***********************************************************}%

\csname @@end\endcsname
\end
%</test3>
%    \end{macrocode}
%
% \section{Installation}
%
% \subsection{Download}
%
% \paragraph{Package.} This package is available on
% CTAN\footnote{\url{https://ctan.org/pkg/catchfile}}:
% \begin{description}
% \item[\CTAN{macros/latex/contrib/oberdiek/catchfile.dtx}] The source file.
% \item[\CTAN{macros/latex/contrib/oberdiek/catchfile.pdf}] Documentation.
% \end{description}
%
%
% \paragraph{Bundle.} All the packages of the bundle `oberdiek'
% are also available in a TDS compliant ZIP archive. There
% the packages are already unpacked and the documentation files
% are generated. The files and directories obey the TDS standard.
% \begin{description}
% \item[\CTANinstall{install/macros/latex/contrib/oberdiek.tds.zip}]
% \end{description}
% \emph{TDS} refers to the standard ``A Directory Structure
% for \TeX\ Files'' (\CTAN{tds/tds.pdf}). Directories
% with \xfile{texmf} in their name are usually organized this way.
%
% \subsection{Bundle installation}
%
% \paragraph{Unpacking.} Unpack the \xfile{oberdiek.tds.zip} in the
% TDS tree (also known as \xfile{texmf} tree) of your choice.
% Example (linux):
% \begin{quote}
%   |unzip oberdiek.tds.zip -d ~/texmf|
% \end{quote}
%
% \paragraph{Script installation.}
% Check the directory \xfile{TDS:scripts/oberdiek/} for
% scripts that need further installation steps.
% Package \xpackage{attachfile2} comes with the Perl script
% \xfile{pdfatfi.pl} that should be installed in such a way
% that it can be called as \texttt{pdfatfi}.
% Example (linux):
% \begin{quote}
%   |chmod +x scripts/oberdiek/pdfatfi.pl|\\
%   |cp scripts/oberdiek/pdfatfi.pl /usr/local/bin/|
% \end{quote}
%
% \subsection{Package installation}
%
% \paragraph{Unpacking.} The \xfile{.dtx} file is a self-extracting
% \docstrip\ archive. The files are extracted by running the
% \xfile{.dtx} through \plainTeX:
% \begin{quote}
%   \verb|tex catchfile.dtx|
% \end{quote}
%
% \paragraph{TDS.} Now the different files must be moved into
% the different directories in your installation TDS tree
% (also known as \xfile{texmf} tree):
% \begin{quote}
% \def\t{^^A
% \begin{tabular}{@{}>{\ttfamily}l@{ $\rightarrow$ }>{\ttfamily}l@{}}
%   catchfile.sty & tex/generic/oberdiek/catchfile.sty\\
%   catchfile.pdf & doc/latex/oberdiek/catchfile.pdf\\
%   test/catchfile-test1.tex & doc/latex/oberdiek/test/catchfile-test1.tex\\
%   test/catchfile-test2.tex & doc/latex/oberdiek/test/catchfile-test2.tex\\
%   test/catchfile-test3.tex & doc/latex/oberdiek/test/catchfile-test3.tex\\
%   catchfile.dtx & source/latex/oberdiek/catchfile.dtx\\
% \end{tabular}^^A
% }^^A
% \sbox0{\t}^^A
% \ifdim\wd0>\linewidth
%   \begingroup
%     \advance\linewidth by\leftmargin
%     \advance\linewidth by\rightmargin
%   \edef\x{\endgroup
%     \def\noexpand\lw{\the\linewidth}^^A
%   }\x
%   \def\lwbox{^^A
%     \leavevmode
%     \hbox to \linewidth{^^A
%       \kern-\leftmargin\relax
%       \hss
%       \usebox0
%       \hss
%       \kern-\rightmargin\relax
%     }^^A
%   }^^A
%   \ifdim\wd0>\lw
%     \sbox0{\small\t}^^A
%     \ifdim\wd0>\linewidth
%       \ifdim\wd0>\lw
%         \sbox0{\footnotesize\t}^^A
%         \ifdim\wd0>\linewidth
%           \ifdim\wd0>\lw
%             \sbox0{\scriptsize\t}^^A
%             \ifdim\wd0>\linewidth
%               \ifdim\wd0>\lw
%                 \sbox0{\tiny\t}^^A
%                 \ifdim\wd0>\linewidth
%                   \lwbox
%                 \else
%                   \usebox0
%                 \fi
%               \else
%                 \lwbox
%               \fi
%             \else
%               \usebox0
%             \fi
%           \else
%             \lwbox
%           \fi
%         \else
%           \usebox0
%         \fi
%       \else
%         \lwbox
%       \fi
%     \else
%       \usebox0
%     \fi
%   \else
%     \lwbox
%   \fi
% \else
%   \usebox0
% \fi
% \end{quote}
% If you have a \xfile{docstrip.cfg} that configures and enables \docstrip's
% TDS installing feature, then some files can already be in the right
% place, see the documentation of \docstrip.
%
% \subsection{Refresh file name databases}
%
% If your \TeX~distribution
% (\teTeX, \mikTeX, \dots) relies on file name databases, you must refresh
% these. For example, \teTeX\ users run \verb|texhash| or
% \verb|mktexlsr|.
%
% \subsection{Some details for the interested}
%
% \paragraph{Attached source.}
%
% The PDF documentation on CTAN also includes the
% \xfile{.dtx} source file. It can be extracted by
% AcrobatReader 6 or higher. Another option is \textsf{pdftk},
% e.g. unpack the file into the current directory:
% \begin{quote}
%   \verb|pdftk catchfile.pdf unpack_files output .|
% \end{quote}
%
% \paragraph{Unpacking with \LaTeX.}
% The \xfile{.dtx} chooses its action depending on the format:
% \begin{description}
% \item[\plainTeX:] Run \docstrip\ and extract the files.
% \item[\LaTeX:] Generate the documentation.
% \end{description}
% If you insist on using \LaTeX\ for \docstrip\ (really,
% \docstrip\ does not need \LaTeX), then inform the autodetect routine
% about your intention:
% \begin{quote}
%   \verb|latex \let\install=y\input{catchfile.dtx}|
% \end{quote}
% Do not forget to quote the argument according to the demands
% of your shell.
%
% \paragraph{Generating the documentation.}
% You can use both the \xfile{.dtx} or the \xfile{.drv} to generate
% the documentation. The process can be configured by the
% configuration file \xfile{ltxdoc.cfg}. For instance, put this
% line into this file, if you want to have A4 as paper format:
% \begin{quote}
%   \verb|\PassOptionsToClass{a4paper}{article}|
% \end{quote}
% An example follows how to generate the
% documentation with pdf\LaTeX:
% \begin{quote}
%\begin{verbatim}
%pdflatex catchfile.dtx
%makeindex -s gind.ist catchfile.idx
%pdflatex catchfile.dtx
%makeindex -s gind.ist catchfile.idx
%pdflatex catchfile.dtx
%\end{verbatim}
% \end{quote}
%
% \begin{History}
%   \begin{Version}{2007/05/30 v1.0}
%   \item
%     First version.
%   \end{Version}
%   \begin{Version}{2007/09/09 v1.1}
%   \item
%     Catcode section rewritten.
%   \end{Version}
%   \begin{Version}{2007/11/11 v1.2}
%   \item
%     Use of package \xpackage{pdftexcmds} for \LuaTeX\ support.
%   \end{Version}
%   \begin{Version}{2010/03/01 v1.3}
%   \item
%     Fix for unknown \cs{@PackageErrorNoLine}.
%   \end{Version}
%   \begin{Version}{2010/04/08 v1.4}
%   \item
%     \cs{closein} also added if \cs{ifeof} is true.
%   \end{Version}
%   \begin{Version}{2010/04/28 v1.5}
%   \item
%     \cs{CatchFileDef}: Getting rid of warning `end of semi simple group
%     entered at line \dots\ of
%     a different file (Florent Chervet).
%   \item
%     \cs{CatchFileDef}: Fix for error `Illegal parameter number in
%     definition of \dots' (HO) including improvement that uses
%     \cs{unexpanded} if available (Florent Chervet).
%   \end{Version}
%   \begin{Version}{2011/03/01 v1.6}
%   \item
%     \cs{relax} added after the setup argument of
%     \cs{CatchFileDef} to prevent premature file reading
%     (Martin Scharrer).
%   \end{Version}
%   \begin{Version}{2016/05/16 v1.7}
%   \item
%     Documentation updates.
%   \end{Version}
% \end{History}
%
% \PrintIndex
%
% \Finale
\endinput
|
% \end{quote}
% Do not forget to quote the argument according to the demands
% of your shell.
%
% \paragraph{Generating the documentation.}
% You can use both the \xfile{.dtx} or the \xfile{.drv} to generate
% the documentation. The process can be configured by the
% configuration file \xfile{ltxdoc.cfg}. For instance, put this
% line into this file, if you want to have A4 as paper format:
% \begin{quote}
%   \verb|\PassOptionsToClass{a4paper}{article}|
% \end{quote}
% An example follows how to generate the
% documentation with pdf\LaTeX:
% \begin{quote}
%\begin{verbatim}
%pdflatex catchfile.dtx
%makeindex -s gind.ist catchfile.idx
%pdflatex catchfile.dtx
%makeindex -s gind.ist catchfile.idx
%pdflatex catchfile.dtx
%\end{verbatim}
% \end{quote}
%
% \begin{History}
%   \begin{Version}{2007/05/30 v1.0}
%   \item
%     First version.
%   \end{Version}
%   \begin{Version}{2007/09/09 v1.1}
%   \item
%     Catcode section rewritten.
%   \end{Version}
%   \begin{Version}{2007/11/11 v1.2}
%   \item
%     Use of package \xpackage{pdftexcmds} for \LuaTeX\ support.
%   \end{Version}
%   \begin{Version}{2010/03/01 v1.3}
%   \item
%     Fix for unknown \cs{@PackageErrorNoLine}.
%   \end{Version}
%   \begin{Version}{2010/04/08 v1.4}
%   \item
%     \cs{closein} also added if \cs{ifeof} is true.
%   \end{Version}
%   \begin{Version}{2010/04/28 v1.5}
%   \item
%     \cs{CatchFileDef}: Getting rid of warning `end of semi simple group
%     entered at line \dots\ of
%     a different file (Florent Chervet).
%   \item
%     \cs{CatchFileDef}: Fix for error `Illegal parameter number in
%     definition of \dots' (HO) including improvement that uses
%     \cs{unexpanded} if available (Florent Chervet).
%   \end{Version}
%   \begin{Version}{2011/03/01 v1.6}
%   \item
%     \cs{relax} added after the setup argument of
%     \cs{CatchFileDef} to prevent premature file reading
%     (Martin Scharrer).
%   \end{Version}
%   \begin{Version}{2016/05/16 v1.7}
%   \item
%     Documentation updates.
%   \end{Version}
% \end{History}
%
% \PrintIndex
%
% \Finale
\endinput
|
% \end{quote}
% Do not forget to quote the argument according to the demands
% of your shell.
%
% \paragraph{Generating the documentation.}
% You can use both the \xfile{.dtx} or the \xfile{.drv} to generate
% the documentation. The process can be configured by the
% configuration file \xfile{ltxdoc.cfg}. For instance, put this
% line into this file, if you want to have A4 as paper format:
% \begin{quote}
%   \verb|\PassOptionsToClass{a4paper}{article}|
% \end{quote}
% An example follows how to generate the
% documentation with pdf\LaTeX:
% \begin{quote}
%\begin{verbatim}
%pdflatex catchfile.dtx
%makeindex -s gind.ist catchfile.idx
%pdflatex catchfile.dtx
%makeindex -s gind.ist catchfile.idx
%pdflatex catchfile.dtx
%\end{verbatim}
% \end{quote}
%
% \begin{History}
%   \begin{Version}{2007/05/30 v1.0}
%   \item
%     First version.
%   \end{Version}
%   \begin{Version}{2007/09/09 v1.1}
%   \item
%     Catcode section rewritten.
%   \end{Version}
%   \begin{Version}{2007/11/11 v1.2}
%   \item
%     Use of package \xpackage{pdftexcmds} for \LuaTeX\ support.
%   \end{Version}
%   \begin{Version}{2010/03/01 v1.3}
%   \item
%     Fix for unknown \cs{@PackageErrorNoLine}.
%   \end{Version}
%   \begin{Version}{2010/04/08 v1.4}
%   \item
%     \cs{closein} also added if \cs{ifeof} is true.
%   \end{Version}
%   \begin{Version}{2010/04/28 v1.5}
%   \item
%     \cs{CatchFileDef}: Getting rid of warning `end of semi simple group
%     entered at line \dots\ of
%     a different file (Florent Chervet).
%   \item
%     \cs{CatchFileDef}: Fix for error `Illegal parameter number in
%     definition of \dots' (HO) including improvement that uses
%     \cs{unexpanded} if available (Florent Chervet).
%   \end{Version}
%   \begin{Version}{2011/03/01 v1.6}
%   \item
%     \cs{relax} added after the setup argument of
%     \cs{CatchFileDef} to prevent premature file reading
%     (Martin Scharrer).
%   \end{Version}
%   \begin{Version}{2016/05/16 v1.7}
%   \item
%     Documentation updates.
%   \end{Version}
% \end{History}
%
% \PrintIndex
%
% \Finale
\endinput
|
% \end{quote}
% Do not forget to quote the argument according to the demands
% of your shell.
%
% \paragraph{Generating the documentation.}
% You can use both the \xfile{.dtx} or the \xfile{.drv} to generate
% the documentation. The process can be configured by the
% configuration file \xfile{ltxdoc.cfg}. For instance, put this
% line into this file, if you want to have A4 as paper format:
% \begin{quote}
%   \verb|\PassOptionsToClass{a4paper}{article}|
% \end{quote}
% An example follows how to generate the
% documentation with pdf\LaTeX:
% \begin{quote}
%\begin{verbatim}
%pdflatex catchfile.dtx
%makeindex -s gind.ist catchfile.idx
%pdflatex catchfile.dtx
%makeindex -s gind.ist catchfile.idx
%pdflatex catchfile.dtx
%\end{verbatim}
% \end{quote}
%
% \begin{History}
%   \begin{Version}{2007/05/30 v1.0}
%   \item
%     First version.
%   \end{Version}
%   \begin{Version}{2007/09/09 v1.1}
%   \item
%     Catcode section rewritten.
%   \end{Version}
%   \begin{Version}{2007/11/11 v1.2}
%   \item
%     Use of package \xpackage{pdftexcmds} for \LuaTeX\ support.
%   \end{Version}
%   \begin{Version}{2010/03/01 v1.3}
%   \item
%     Fix for unknown \cs{@PackageErrorNoLine}.
%   \end{Version}
%   \begin{Version}{2010/04/08 v1.4}
%   \item
%     \cs{closein} also added if \cs{ifeof} is true.
%   \end{Version}
%   \begin{Version}{2010/04/28 v1.5}
%   \item
%     \cs{CatchFileDef}: Getting rid of warning `end of semi simple group
%     entered at line \dots\ of
%     a different file (Florent Chervet).
%   \item
%     \cs{CatchFileDef}: Fix for error `Illegal parameter number in
%     definition of \dots' (HO) including improvement that uses
%     \cs{unexpanded} if available (Florent Chervet).
%   \end{Version}
%   \begin{Version}{2011/03/01 v1.6}
%   \item
%     \cs{relax} added after the setup argument of
%     \cs{CatchFileDef} to prevent premature file reading
%     (Martin Scharrer).
%   \end{Version}
%   \begin{Version}{2016/05/16 v1.7}
%   \item
%     Documentation updates.
%   \end{Version}
% \end{History}
%
% \PrintIndex
%
% \Finale
\endinput
