% \iffalse meta-comment
%
% File: kvoptions.dtx
% Version: 2016/05/16 v3.12
% Info: Key value format for package options
%
% Copyright (C) 2004, 2006, 2007, 2009-2011 by
%    Heiko Oberdiek <heiko.oberdiek at googlemail.com>
%    2016
%    https://github.com/ho-tex/oberdiek/issues
%
% This work may be distributed and/or modified under the
% conditions of the LaTeX Project Public License, either
% version 1.3c of this license or (at your option) any later
% version. This version of this license is in
%    https://www.latex-project.org/lppl/lppl-1-3c.txt
% and the latest version of this license is in
%    https://www.latex-project.org/lppl.txt
% and version 1.3 or later is part of all distributions of
% LaTeX version 2005/12/01 or later.
%
% This work has the LPPL maintenance status "maintained".
%
% The Current Maintainers of this work are
% Heiko Oberdiek and the Oberdiek Package Support Group
% https://github.com/ho-tex/oberdiek/issues
%
% This work consists of the main source file kvoptions.dtx
% and the derived files
%    kvoptions.sty, kvoptions.pdf, kvoptions.ins, kvoptions.drv,
%    kvoptions-patch.sty, example-mycolorsetup.sty,
%    kvoptions-test1.tex, kvoptions-test2.tex,
%    kvoptions-test3.tex, kvoptions-test4.tex,
%    kvoptions-test4.sty.
%
% Distribution:
%    CTAN:macros/latex/contrib/oberdiek/kvoptions.dtx
%    CTAN:macros/latex/contrib/oberdiek/kvoptions.pdf
%
% Unpacking:
%    (a) If kvoptions.ins is present:
%           tex kvoptions.ins
%    (b) Without kvoptions.ins:
%           tex kvoptions.dtx
%    (c) If you insist on using LaTeX
%           latex \let\install=y% \iffalse meta-comment
%
% File: kvoptions.dtx
% Version: 2016/05/16 v3.12
% Info: Key value format for package options
%
% Copyright (C)
%    2004, 2006, 2007, 2009-2011 Heiko Oberdiek
%    2016-2019 Oberdiek Package Support Group
%    https://github.com/ho-tex/oberdiek/issues
%
% This work may be distributed and/or modified under the
% conditions of the LaTeX Project Public License, either
% version 1.3c of this license or (at your option) any later
% version. This version of this license is in
%    https://www.latex-project.org/lppl/lppl-1-3c.txt
% and the latest version of this license is in
%    https://www.latex-project.org/lppl.txt
% and version 1.3 or later is part of all distributions of
% LaTeX version 2005/12/01 or later.
%
% This work has the LPPL maintenance status "maintained".
%
% The Current Maintainers of this work are
% Heiko Oberdiek and the Oberdiek Package Support Group
% https://github.com/ho-tex/oberdiek/issues
%
% This work consists of the main source file kvoptions.dtx
% and the derived files
%    kvoptions.sty, kvoptions.pdf, kvoptions.ins, kvoptions.drv,
%    kvoptions-patch.sty, example-mycolorsetup.sty,
%    kvoptions-test1.tex, kvoptions-test2.tex,
%    kvoptions-test3.tex, kvoptions-test4.tex,
%    kvoptions-test4.sty.
%
% Distribution:
%    CTAN:macros/latex/contrib/oberdiek/kvoptions.dtx
%    CTAN:macros/latex/contrib/oberdiek/kvoptions.pdf
%
% Unpacking:
%    (a) If kvoptions.ins is present:
%           tex kvoptions.ins
%    (b) Without kvoptions.ins:
%           tex kvoptions.dtx
%    (c) If you insist on using LaTeX
%           latex \let\install=y% \iffalse meta-comment
%
% File: kvoptions.dtx
% Version: 2016/05/16 v3.12
% Info: Key value format for package options
%
% Copyright (C)
%    2004, 2006, 2007, 2009-2011 Heiko Oberdiek
%    2016-2019 Oberdiek Package Support Group
%    https://github.com/ho-tex/oberdiek/issues
%
% This work may be distributed and/or modified under the
% conditions of the LaTeX Project Public License, either
% version 1.3c of this license or (at your option) any later
% version. This version of this license is in
%    https://www.latex-project.org/lppl/lppl-1-3c.txt
% and the latest version of this license is in
%    https://www.latex-project.org/lppl.txt
% and version 1.3 or later is part of all distributions of
% LaTeX version 2005/12/01 or later.
%
% This work has the LPPL maintenance status "maintained".
%
% The Current Maintainers of this work are
% Heiko Oberdiek and the Oberdiek Package Support Group
% https://github.com/ho-tex/oberdiek/issues
%
% This work consists of the main source file kvoptions.dtx
% and the derived files
%    kvoptions.sty, kvoptions.pdf, kvoptions.ins, kvoptions.drv,
%    kvoptions-patch.sty, example-mycolorsetup.sty,
%    kvoptions-test1.tex, kvoptions-test2.tex,
%    kvoptions-test3.tex, kvoptions-test4.tex,
%    kvoptions-test4.sty.
%
% Distribution:
%    CTAN:macros/latex/contrib/oberdiek/kvoptions.dtx
%    CTAN:macros/latex/contrib/oberdiek/kvoptions.pdf
%
% Unpacking:
%    (a) If kvoptions.ins is present:
%           tex kvoptions.ins
%    (b) Without kvoptions.ins:
%           tex kvoptions.dtx
%    (c) If you insist on using LaTeX
%           latex \let\install=y% \iffalse meta-comment
%
% File: kvoptions.dtx
% Version: 2016/05/16 v3.12
% Info: Key value format for package options
%
% Copyright (C)
%    2004, 2006, 2007, 2009-2011 Heiko Oberdiek
%    2016-2019 Oberdiek Package Support Group
%    https://github.com/ho-tex/oberdiek/issues
%
% This work may be distributed and/or modified under the
% conditions of the LaTeX Project Public License, either
% version 1.3c of this license or (at your option) any later
% version. This version of this license is in
%    https://www.latex-project.org/lppl/lppl-1-3c.txt
% and the latest version of this license is in
%    https://www.latex-project.org/lppl.txt
% and version 1.3 or later is part of all distributions of
% LaTeX version 2005/12/01 or later.
%
% This work has the LPPL maintenance status "maintained".
%
% The Current Maintainers of this work are
% Heiko Oberdiek and the Oberdiek Package Support Group
% https://github.com/ho-tex/oberdiek/issues
%
% This work consists of the main source file kvoptions.dtx
% and the derived files
%    kvoptions.sty, kvoptions.pdf, kvoptions.ins, kvoptions.drv,
%    kvoptions-patch.sty, example-mycolorsetup.sty,
%    kvoptions-test1.tex, kvoptions-test2.tex,
%    kvoptions-test3.tex, kvoptions-test4.tex,
%    kvoptions-test4.sty.
%
% Distribution:
%    CTAN:macros/latex/contrib/oberdiek/kvoptions.dtx
%    CTAN:macros/latex/contrib/oberdiek/kvoptions.pdf
%
% Unpacking:
%    (a) If kvoptions.ins is present:
%           tex kvoptions.ins
%    (b) Without kvoptions.ins:
%           tex kvoptions.dtx
%    (c) If you insist on using LaTeX
%           latex \let\install=y\input{kvoptions.dtx}
%        (quote the arguments according to the demands of your shell)
%
% Documentation:
%    (a) If kvoptions.drv is present:
%           latex kvoptions.drv
%    (b) Without kvoptions.drv:
%           latex kvoptions.dtx; ...
%    The class ltxdoc loads the configuration file ltxdoc.cfg
%    if available. Here you can specify further options, e.g.
%    use A4 as paper format:
%       \PassOptionsToClass{a4paper}{article}
%
%    Programm calls to get the documentation (example):
%       pdflatex kvoptions.dtx
%       makeindex -s gind.ist kvoptions.idx
%       pdflatex kvoptions.dtx
%       makeindex -s gind.ist kvoptions.idx
%       pdflatex kvoptions.dtx
%
% Installation:
%    TDS:tex/latex/oberdiek/kvoptions.sty
%    TDS:tex/latex/oberdiek/kvoptions-patch.sty
%    TDS:doc/latex/oberdiek/kvoptions.pdf
%    TDS:doc/latex/oberdiek/example-mycolorsetup.sty
%    TDS:doc/latex/oberdiek/test/kvoptions-test1.tex
%    TDS:doc/latex/oberdiek/test/kvoptions-test2.tex
%    TDS:doc/latex/oberdiek/test/kvoptions-test3.tex
%    TDS:doc/latex/oberdiek/test/kvoptions-test4.tex
%    TDS:doc/latex/oberdiek/test/kvoptions-test4.sty
%    TDS:source/latex/oberdiek/kvoptions.dtx
%
%<*ignore>
\begingroup
  \catcode123=1 %
  \catcode125=2 %
  \def\x{LaTeX2e}%
\expandafter\endgroup
\ifcase 0\ifx\install y1\fi\expandafter
         \ifx\csname processbatchFile\endcsname\relax\else1\fi
         \ifx\fmtname\x\else 1\fi\relax
\else\csname fi\endcsname
%</ignore>
%<*install>
\input docstrip.tex
\Msg{************************************************************************}
\Msg{* Installation}
\Msg{* Package: kvoptions 2016/05/16 v3.12 Key value format for package options (HO)}
\Msg{************************************************************************}

\keepsilent
\askforoverwritefalse

\let\MetaPrefix\relax
\preamble

This is a generated file.

Project: kvoptions
Version: 2016/05/16 v3.12

Copyright (C)
   2004, 2006, 2007, 2009-2011 Heiko Oberdiek
   2016-2019 Oberdiek Package Support Group

This work may be distributed and/or modified under the
conditions of the LaTeX Project Public License, either
version 1.3c of this license or (at your option) any later
version. This version of this license is in
   https://www.latex-project.org/lppl/lppl-1-3c.txt
and the latest version of this license is in
   https://www.latex-project.org/lppl.txt
and version 1.3 or later is part of all distributions of
LaTeX version 2005/12/01 or later.

This work has the LPPL maintenance status "maintained".

The Current Maintainers of this work are
Heiko Oberdiek and the Oberdiek Package Support Group
https://github.com/ho-tex/oberdiek/issues


This work consists of the main source file kvoptions.dtx
and the derived files
   kvoptions.sty, kvoptions.pdf, kvoptions.ins, kvoptions.drv,
   kvoptions-patch.sty, example-mycolorsetup.sty,
   kvoptions-test1.tex, kvoptions-test2.tex,
   kvoptions-test3.tex, kvoptions-test4.tex,
   kvoptions-test4.sty.

\endpreamble
\let\MetaPrefix\DoubleperCent

\generate{%
  \file{kvoptions.ins}{\from{kvoptions.dtx}{install}}%
  \file{kvoptions.drv}{\from{kvoptions.dtx}{driver}}%
  \usedir{tex/latex/oberdiek}%
  \file{kvoptions.sty}{\from{kvoptions.dtx}{package}}%
  \file{kvoptions-patch.sty}{\from{kvoptions.dtx}{patch}}%
  \usedir{doc/latex/oberdiek}%
  \file{example-mycolorsetup.sty}{\from{kvoptions.dtx}{example}}%
%  \usedir{doc/latex/oberdiek/test}%
%  \file{kvoptions-test1.tex}{\from{kvoptions.dtx}{test1}}%
%  \file{kvoptions-test2.tex}{\from{kvoptions.dtx}{test2}}%
%  \file{kvoptions-test3.tex}{\from{kvoptions.dtx}{test3}}%
%  \file{kvoptions-test4.tex}{\from{kvoptions.dtx}{test4}}%
%  \file{kvoptions-test4.sty}{\from{kvoptions.dtx}{test4pkg}}%
  \nopreamble
  \nopostamble
%  \usedir{source/latex/oberdiek/catalogue}%
%  \file{kvoptions.xml}{\from{kvoptions.dtx}{catalogue}}%
}

\catcode32=13\relax% active space
\let =\space%
\Msg{************************************************************************}
\Msg{*}
\Msg{* To finish the installation you have to move the following}
\Msg{* files into a directory searched by TeX:}
\Msg{*}
\Msg{*     kvoptions.sty, kvoptions-patch.sty}
\Msg{*}
\Msg{* To produce the documentation run the file `kvoptions.drv'}
\Msg{* through LaTeX.}
\Msg{*}
\Msg{* Happy TeXing!}
\Msg{*}
\Msg{************************************************************************}

\endbatchfile
%</install>
%<*ignore>
\fi
%</ignore>
%<*driver>
\NeedsTeXFormat{LaTeX2e}
\ProvidesFile{kvoptions.drv}%
  [2016/05/16 v3.12 Key value format for package options (HO)]%
\documentclass{ltxdoc}
\usepackage{holtxdoc}[2011/11/22]
\begin{document}
  \DocInput{kvoptions.dtx}%
\end{document}
%</driver>
% \fi
%
%
% \CharacterTable
%  {Upper-case    \A\B\C\D\E\F\G\H\I\J\K\L\M\N\O\P\Q\R\S\T\U\V\W\X\Y\Z
%   Lower-case    \a\b\c\d\e\f\g\h\i\j\k\l\m\n\o\p\q\r\s\t\u\v\w\x\y\z
%   Digits        \0\1\2\3\4\5\6\7\8\9
%   Exclamation   \!     Double quote  \"     Hash (number) \#
%   Dollar        \$     Percent       \%     Ampersand     \&
%   Acute accent  \'     Left paren    \(     Right paren   \)
%   Asterisk      \*     Plus          \+     Comma         \,
%   Minus         \-     Point         \.     Solidus       \/
%   Colon         \:     Semicolon     \;     Less than     \<
%   Equals        \=     Greater than  \>     Question mark \?
%   Commercial at \@     Left bracket  \[     Backslash     \\
%   Right bracket \]     Circumflex    \^     Underscore    \_
%   Grave accent  \`     Left brace    \{     Vertical bar  \|
%   Right brace   \}     Tilde         \~}
%
% \GetFileInfo{kvoptions.drv}
%
% \title{The \xpackage{kvoptions} package}
% \date{2016/05/16 v3.12}
% \author{Heiko Oberdiek\thanks
% {Please report any issues at \url{https://github.com/ho-tex/oberdiek/issues}}}
%
% \maketitle
%
% \begin{abstract}
% This package is intended for package authors who want to
% use options in key value format for their package options.
% \end{abstract}
%
% \tableofcontents
%
% \def\M#1{\texttt{\{}\meta{#1}\texttt{\}}}
% \def\O#1{\texttt{[}\meta{#1}\texttt{]}}
%
% \section{Introduction}
%
% First I want to recommend the very good review article
% ``A guide to key-value methods'' by Joseph Wright \cite{tb94wright}.
% It introduces the different key-value packages and compares them.
%
% \subsection{The beginning}
%
% This package \xpackage{kvoptions} addresses class or package writers
% that want to allow their users to specify options
% as key value pairs, e.g.:
% \begin{quote}
% |\documentclass[verbose=false,name=me]{myclass}|\\
% |\usepackage[format=print]{mylayout}|
% \end{quote}
% Prominent example is package \xpackage{hyperref},
% probably the first package that offers this service.
% It's \cs{ProcessOptionsWithKV} is often copied und
% used in other packages, e.g. package \xpackage{helvet}
% that uses this interface for its option \xoption{scaled}.
%
% However copying code is not the most modern software
% development technique. And \xpackage{hyperref}'s
% code for \cs{ProcessOptionsWithKV} was changed
% to fix bugs. The version used in other packages
% depends on the time of copying and the awareness
% of \xpackage{hyperref}'s changes.
% Now the code is sourced out into this package and
% available for other package or class writers.
%
% \subsection{Overview}
%
% Package \xpackage{kvoptions} connects package
% \xpackage{\emph{k}ey\emph{v}al}
% with \LaTeX's package and class \emph{options}:
% \begin{center}
% \def\T#1{\bfseries #1}
% \renewcommand{\arraystretch}{1.5}
% \begin{tabular}{l|l|l}
%   \T{Package \xpackage{keyval}}
%   & \T{Package \xpackage{kvoptions}}
%   & \T{\LaTeX\ kernel}
%   \\
% \hline
%     \cs{define@key}
%   &
%     \renewcommand*{\arraystretch}{1}^^A
%     \begin{tabular}[t]{@{}l@{}}
%       \cs{DeclareVoidOption}\\
%       \cs{DeclareStringOption}\\
%       \cs{DeclareBoolOption}\\
%       \cs{DeclareComplementaryOption}\\
%       \cs{DisableKeyvalOption}
%     \end{tabular}
%   & \cs{DeclareOption}
%   \\
% \hline
%   & \cs{DeclareDefaultOption}
%   & \cs{DeclareOption*}
%   \\
% \hline
%   & \cs{ProcessKeyvalOptions}
%   & \cs{ProcessOptions*}
%   \\
% \hline
%   & Option \xoption{patch}
%   &
%     \renewcommand*{\arraystretch}{1}^^A
%     \begin{tabular}[t]{@{}l@{}}
%       Class/package\\
%       option system
%     \end{tabular}
%   \\
% \hline
%   & \cs{SetupKeyvalOptions} &
%   \\
% \end{tabular}
% \end{center}
%
% \hypersetup{bookmarksopenlevel=3}
% \section{Usage}
%
% \subsection{Process options}
%
% \subsubsection{\cs{ProcessKeyvalOptions}}
%
% \begin{declcs}{ProcessKeyvalOptions}\,\M{family}\\
%   \cs{ProcessKeyvalOptions} \*
% \end{declcs}
% This command evaluates the global or local options of
% the package that are defined with \xpackage{keyval}'s
% interface within the family \meta{family}. It acts the
% same way as \LaTeX's \cs{ProcessOptions*}. In a package
% unknown global options are ignored, in a class they
% are added to the unknown option list. The known global
% options and all local options are passed to \xpackage{keyval}'s
% \cs{setkeys} command for executing the options.
% Unknown options are reported to the user by an error.
%
% If the family name happens to be the same as the name
% of the package or class where \cs{ProcessKeyvalOptions}
% is used or the family name has previously been setup by
% \cs{SetupKeyvalOptions}, then \cs{ProcessKeyvalOptions}
% knows the family name already and you can use the star form
% without mandatory argument.
%
% \subsubsection{\cs{ProcessLocalKeyvalOptions}}
%
% \begin{declcs}{ProcessLocalKeyvalOptions}\,\M{family}\\
%   \cs{ProcessLocalKeyvalOptions} \*
% \end{declcs}
% This macro has the same syntax and works similar as
% \cs{ProcessKeyvalOptions}. However it ignores
% global options and only processes the local package options.
% Therefore it only can be used inside a package.
% An error is thrown, if it is used inside a class.
%
% Neither of the following macros are necessary for
% \cs{ProcessKeyvalOptions{}}. They just help the package/class
% author in common tasks.
%
% \subsubsection{\cs{SetupKeyvalOptions}}\label{setup}
%
% \begin{declcs}{SetupKeyvalOptions}\,\texttt{\{}\\
%   \hspace{1.5em}|family|\,|=|\,\meta{family}|,|\\
%   \hspace{1.5em}|prefix|\,|=|\,\meta{prefix}\\
%   \hspace{1.5em}|setkeys|\,|=|\,\meta{setkeys command}\\
%   \texttt{\}}
% \end{declcs}
% This command allows to configure the default assumptions
% that are based on the current package or class name.
% \LaTeX\ remembers this name in \cs{@currname}. The syntax
% description of the default looks a little weird, therefor
% an example is given for a package or class named |foobar|.
% \begin{quote}
% \begin{tabular}{@{}l|ll|l@{}}
%   Key & Default & (example) & Used by\\
% \hline
%   \rule{0pt}{1.2\ht\csname @arstrutbox\endcsname}^^A
%   |family| & \meta{\cs{@currname}} & (|foobar|)
%       & \cs{ProcessKeyvalOptions*} \\
%     &&& \cs{DeclareBoolOption} \\
%     &&& \cs{DeclareStringOption} \\
% \hline
%   \rule{0pt}{1.2\ht\csname @arstrutbox\endcsname}^^A
%   |prefix| & \meta{\cs{@currname}}|@| & (|foobar@|)
%       & \cs{DeclareBoolOption} \\
%     &&& \cs{DeclareStringOption} \\
%     &&& \cs{DeclareVoidOption} \\
% \hline
%   \rule{0pt}{1.2\ht\csname @arstrutbox\endcsname}^^A
%   |setkeys| & \cs{setkeys} & (\cs{kvsetkeys})
%       & \cs{ProcessKeyvalOptions} \\
%     &&& \cs{ProcessLocalKeyvalOptions}
% \end{tabular}
% \end{quote}
% Key |setkeys| was added in version 3.9. The original
% \cs{setkeys} of package \xpackage{keyval} is not reentrant.
% If an option is processed by this \cs{setkeys}, then
% the option should not call \cs{setkeys} again with a
% different family. Otherwise the next options of the
% first \cs{setkeys} call are processed with the wrong
% family. With key |setkeys| the macro \cs{kvsetkeys}
% can be set that does not have the problem of the
% original \cs{setkeys} of package {keyval}.
%
% Probably \cs{setkeys} of package \xpackage{xkeyval}
% is safe in this respect. But I haven't made a full
% analysis. At least it does not have the problem
% of the original \cs{setkeys}.
%
% \subsection{Option declarations}
%
% The options for \cs{ProcessKeyvalOptions} are defined
% by \xpackage{keyval}'s \cs{define@key}.
% Common purposes of such keys are boolean switches, they
% enable or disable something. Or they store a name or some kind of
% string in a macro. The following commands help the user. He
% declares what he wants and \xpackage{kvoptions} take care of
% the key definition, resource allocation and initialization.
%
% In order to avoid name clashes of macro names, internal
% commands are prefixed. Both the prefix and the family
% name for the defined keys can be configured by
% \cs{SetupKeyvalOptions}.
%
% \subsubsection{\cs{DeclareStringOption}}
%
% \begin{declcs}{DeclareStringOption}\,\O{init}\,\M{key}\,\O{default}
% \end{declcs}
% A macro is created that remembers the value of the key \meta{key}.
% The name of the macro consists of the option name \meta{key} that
% is prefixed by the prefix (see \ref{setup}).
% The initial contents of the
% macro can be given by the first optional argument \meta{init}.
% The default is empty.
%
% The the option \meta{key} is defined. The option code
% just stores its value in the macro. If the optional argument
% at the end of \cs{DeclareStringOption} is given, then option
% \meta{key} is defined with the default \meta{default}.
%
% Example for a package with the following two lines:
% \begin{quote}
% |\ProvidesPackage{foobar}|\\
% |\DeclareStringOption[me]{name}|
% \end{quote}
% Then \cs{DeclareStringOption} defines the macro with content |me|,
% note \LaTeX\ complains if the name of the macro already exists:
% \begin{quote}
% |\newcommand*{\foobar@name}{me}|
% \end{quote}
% The option definition is similar to:
% \begin{quote}
% |\define@key{foobar}{name}{%|\\
% |  \renewcommand*{\foobar@name}{#1}%|\\
% |}|
% \end{quote}
%
% \subsubsection{\cs{DeclareBoolOption}}
%
% \begin{declcs}{DeclareBoolOption}\,\O{init}\,\M{key}
% \end{declcs}
% A boolean switch is generated, initialized by value \meta{init}
% and the corresponding key \meta{key} is defined. If the
% initialization value is not given, |false| is used as default.
%
% The internal actions of \cs{DeclareBoolOption} are shown below.
% The example is given for a package author who has the following
% two lines in his package/class:
% \begin{quote}
% |\ProvidesPackage{foobar}|\\
% |\DeclareBoolOption{verbose}|
% \end{quote}
% First a new switch is created:
% \begin{quote}
%   |\newif\iffoobar@verbose|
% \end{quote}
% and initialized:
% \begin{quote}
%   |\foobar@verbosefalse|
% \end{quote}
% Finally the key is defined:
% \begin{quote}
%   |\define@key{foobar}{verbose}[true]{|\dots|}|
% \end{quote}
% The option code configures the boolean option in the following way:
% If the author specifies |true| or |false| then the switch is
% turned on or off respectivly. Also the option can be given without
% explicit value. Then the switch is enabled.
% Other values are reported as errors.
%
% Now the switch is ready to use in the package/class, e.g.:
% \begin{quote}
%   |\iffoobar@verbose|\\
%   \qquad\textit{\% print verbose message}\\
%   |\else|\\
%   \qquad\textit{\% be quiet}\\
%   |\fi|
% \end{quote}
% Users of package \cs{ifthen} can use the switch as
% boolean:
% \begin{quote}
%   |\boolean{foobar@verbose}|
% \end{quote}
%
% \subsubsection{\cs{DeclareComplementaryOption}}
%
% \begin{declcs}{DeclareComplementaryOption}\,\M{key}\,\M{parent}
% \end{declcs}
% Sometimes contrasting names are used to characterize the
% two states of a boolean switch, for example \xoption{draft}
% vs.\@ \xoption{final}. Both options behave like boolean
% options but they do not need two different switches, they
% should share one. \cs{DeclareComplementaryOption} allows this.
% The option \meta{key} shares the switch of option \meta{parent}.
% Example:
% \begin{quote}
%   |\DeclareBoolOption{draft}|\\
%   |\DeclareComplementaryOption{final}{draft}|
% \end{quote}
% Then |final| sets the switch of \xoption{draft} to |false|, and
% |final=false| enables the \xoption{draft} switch.
%
% \subsubsection{\cs{DeclareVoidOption}}
%
% \begin{declcs}{DeclareVoidOption}\,\M{key}\,\M{code}
% \end{declcs}
% \cs{ProcessKeyvalOptions} can be extended to recognize
% options that are declared in traditional way by \cs{DeclareOption}.
% But in case of the error that the user specifies a value, then
% this option would not recognized as key value option because
% of \cs{DeclareOption} and not detected as traditional option
% because of the value part. The user would get an unknown option
% error, difficult to understand.
%
% \cs{DeclareVoidOption} solves
% this problem. It defines the option \meta{key} as key value option.
% If the user specifies a value, a warning is given and the
% value is ignored.
%
% The code part \meta{code} is stored in a macro.
% The name of the macro consists of the option name \meta{key} that
% is prefixed by the prefix (see \ref{setup}).
% If the option is set, the macro will be executed. During the
% execution \cs{CurrentOption} is available with the current key name.
%
% \subsubsection{\cs{DeclareDefaultOption}}
%
% \begin{declcs}{DeclareDefaultOption}\,\M{code}
% \end{declcs}
% \begin{sloppypar}
% This command does not define a specific key, it is the equivalent
% to \LaTeX's \cs{DeclareOption*}. It allows the specification
% of a default action \meta{code} that is invoked if an unknown option
% is found.
% While \meta{code} is called,
% macro \cs{CurrentOption} contains the current option string.
% In addition \cs{CurrentOptionValue} contains the value part
% if the option string is parsable as key value pair, otherwise
% it is \cs{relax}. \cs{CurrentOptionKey} contains the key of
% the key value pair, or the whole option string, if it misses
% the equal sign.
% \end{sloppypar}
%
% Inside packages typical default actions are to pass unknown
% options to another package. Or an error message can be thrown
% by \cs{@unknownoptionerror}. This is the original error
% message that \LaTeX\ gives for unkown package options.
% This error message is easier to understand for the user
% as the error message from package \xpackage{keyval} that
% is given otherwise.
%
% A Class ignores unknown options and puts them on the
% unused option list. Let \LaTeX\ do the job and just
% call \cs{OptionNotUsed}. Or the options can be passed to
% another class that is later loaded.
%
% \subsubsection{Local options}
%
% \begin{declcs}{DeclareLocalOption} \M{option}\\
%   \cs{DeclareLocalOptions} \M{option list}
% \end{declcs}
% Both macros mark package options as local options. That means
% that they are ignored by \cs{ProcessKeyvalOptions} if they are given
% as global options. \cs{DeclareLocalOptions} takes one option,
% \cs{DeclareLocalOptions} expects a comma separated list of options.
%
% \subsubsection{Dynamic options}
%
% Options of \LaTeX's package/class system are cleared
% in \cs{ProcessOptions}. They modify the static model
% of a package. For example, depending on option \xoption{bookmarks}
% package \xpackage{hyperref} loads differently.
%
% Options, however,
% defined by \xpackage{keyval}'s \cs{define@key} remain
% defined, if the options are processed by \cs{setkeys}.
% Therefore these options can also be used to model
% the dynamic behaviour of a package. For example, in
% \xpackage{hyperref} the link colors can be changed
% everywhere until the end in |\end{document}|.
%
% However package \xpackage{color} that adds color support
% is necessary and it
% cannot be loaded after |\begin{document}|. Option \xoption{colorlinks}
% that loads \xpackage{color} should be active until |\begin{document}|
% and die in some way if it is too late for loading packages.
% With \cs{DisableKeyvalOption} the package/class author can
% specify and configure the death of an option and controls
% the life period of the option.
%
% \subsubsection{\cs{DisableKeyvalOption}}
%
% \begin{declcs}{DisableKeyvalOption}\,\O{options}\,\M{family}\,\M{key}
%   \\[0.5ex]
%   \meta{options}:\\
%   \hspace{1.5em}^^A
%   \begin{tabular}[t]{@{}l@{ \texttt{=} }l@{\qquad}>{default: }l@{}}
%     |action| & |undef|, |warning|, |error|, or |ignore|
%     & |undef|
%     \\
%     \multicolumn{1}{@{}l}{|global| or |local|} & & |global|
%     \\
%     |package| or |class| & \meta{name}
%     \\
%   \end{tabular}^^A
% \end{declcs}
% \cs{DisableKeyvalOption} can be called to mark the end when
% the option \meta{key} is no longer useful. The behaviour of
% an option after its death can be configured by \xoption{action}:
% \begin{description}
% \item[\texttt{undef}:] The option will be undefined,
%   If it is called, \cs{setkeys} reports an error because
%   of unknown key.
% \item[\texttt{error} or \texttt{warning}:] Use of the
%  option will cause an error or warning message.
%  Also these actions require that exclusivly either the
%  package or class name is given in options \xoption{package}
%  or \xoption{class}.
% \item[\texttt{ignore}:] The use of the option will silently
%  be ignored.
% \end{description}
% The option's death can be limited to the end of the current
% group, if option \xoption{local} is given. Default is
% \xoption{global}.
%
% The package/class author can wish the end of the option
% already during the package loading, then he will have
% static behaviour. In case of dynamic options \cs{DisableKeyvalOption}
% can be executed everywhere, also outside the package. Therefore
% the family name and the package/class name is usually unknown for
% \cs{DisableKeyvalOption}. Therefore the argument for the family
% name is mandatory and for some actions the package/class name
% must be provided.
%
% Usually a macro would configure the option death, Example:
%\begin{quote}
%\begin{verbatim}
%\ProvidesPackage{foobar}
%\DeclareBoolOption{color}
%\DeclareStringOption[red]{emphcolor}
%\ProcessKeyvalOptions*
%
%\newcommand*{\foobar@DisableOption}[2]{%
%  \DisableKeyvalueOption[
%    action={#1},
%    package=foobar
%  ]{foobar}{#2}%
%}
%
%\iffoobar@color
%  \RequirePackage{color}
%  \renewcommand*{\emph}[1]{\textcolor{\foobar@emphcolor}{#1}}
%\else
%  % Option emphcolor is not wrong, if we have color support.
%  % otherwise the option has no effect, but we don't want to
%  % remove it. Therefore action 'ignore' is the best choice:
%  \foobar@DisableOption{ignore}{emphcolor}
%\fi
%% No we don't need the option 'color'.
%\foobar@DisableOption{warning}{color}
%
%% With color support option 'emphcolor' will dynamically
%% change the color of \emph statements.
%\end{verbatim}
%\end{quote}
%
% \subsubsection{\cs{AddToKeyvalOption}}
%
% \begin{declcs}{AddToKeyvalOption}\,\M{family}\,\M{key}\,\M{code}\\
%   \cs{AddToKeyvalOption} \*\,\M{key}\,\M{code}
% \end{declcs}
% The code for an existing key \meta{key} of family \meta{family} is
% extended by code \meta{code}. In the starred form the current
% family setting is used, see \cs{ProcessKeyvalOptions*}.
%
% \subsection{Global vs. local options}
%
% Options that are given for \cs{documentclass} are called
% global options. They are known to the class and all packages.
% A package may make use of a global option and marks it as
% used. The advantage for the user is the freedom to specify
% options both in the \cs{documentclass} or \cs{usepackage}
% commands.
%
% However global options are shared with the class options
% and options of all other packages. Thus there can be the
% same option with different semantics for different
% packages and classes. As example, package \xpackage{bookmark}
% knows option \xoption{open} that specifies whether the bookmarks
% are opened or closed initially. It's values are |true| or |false|.
% Since KOMA-Script version 3.00 the KOMA classes also introduces
% option \xoption{open} with values |right| and |any| and a
% complete different meaning.
%
% Such conflicts can be resolved by marking all or part of options
% as local by \cs{DeclareLocalOption} or \cs{DeclareLocalOptions}.
% Then the packages ignores global occurences of these options.
% Package \xpackage{kvoptions} provides two methods:
% \begin{itemize}
% \item \cs{ProcessLocalKeyvalOptions} automatically uses all options
%   as local options. It ignores all global options.
% \item \cs{DeclareLocalOption} or \cs{DeclareLocalOptions} marks
%   options as local options. \cs{ProcessKeyvalOptions} will then
%   ignore global occurences for these local options.
% \end{itemize}
% Since version 1.5 package \xpackage{bookmark} uses the latter
% method. It checks global and local option places for driver options
% and limits all other options as local options. Thus the class
% option \xoption{open} of KOMA-Script is not misread as
% option for package \xpackage{bookmark}.
%
% \subsection{Summary of internal macros}
%
% The \cs{Declare}\texttt{\dots Option} commands define
% macros, additionally to the macros generated by the key definition.
% These macros can be used by the package/class author.
% The name of the macros starts with the prefix \meta{prefix}
% that can be configured by \cs{SetupKeyvalOptions}.
% \begin{center}
% \def\prefix{\cs{}\meta{prefix}\meta{key}}^^A
% \def\Hline{^^A
%   \hline
%   \rule{0pt}{1.2\ht\csname @arstrutbox\endcsname}^^A
% }^^A
% \begin{tabular}{@{}l|>{\ttfamily}l|l@{}}
%   Declare \meta{key} & \textrm{Defined macro} & Description\\
% \Hline
%   \cs{DeclareStringOption}
%    & \prefix & holds the string\\
% \Hline
%   \cs{DeclareBoolOption}
%     & \cs{if}\meta{prefix}\meta{key}& boolean switch\\
%   &\prefix false& disable switch\\
%   &\prefix true& enable switch\\
% \Hline
%   \cs{DeclareComplementaryOption}
%     & \prefix false& enable parent switch\\
%     & \prefix true& disable parent switch\\
% \Hline
%   \cs{DeclareVoidOption}
%   & \prefix & holds the action
% \end{tabular}
% \end{center}
%
% \subsection{\plainTeX}
%
% Package \xpackage{keyval} is also usable in \plainTeX\ with
% the help of file \xfile{miniltx.tex}. Some features of this
% package \xpackage{kvoptions} might also be useful for \plainTeX.
% If \LaTeX\ is not found, \cs{ProcessKeyvalOptions} and option
% \xoption{patch} are disabled. Before using the option declaration
% commands \cs{Declare...Option}, \cs{SetupKeyvalOptions} must be
% used.
%
% \hypersetup{bookmarksopenlevel=1}
%
% \section{Example}
%
% The following example defined a package that serves some
% private color management. A boolean option \xoption{print} enables
% print mode without colors. An option \xoption{emph} redefines
% \cs{emph} to print in the given color. And the driver
% can be specified by option \xoption{driver}.
%
%    \begin{macrocode}
%<*example>
    % Package identification
    % -----------------------
\NeedsTeXFormat{LaTeX2e}
\ProvidesPackage{example-mycolorsetup}[2016/05/16 Managing my colors]

\RequirePackage{ifpdf}
\RequirePackage{kvoptions}

    % Option declarations
    % -------------------

\SetupKeyvalOptions{
  family=MCS,
  prefix=MCS@
}
    % Use a shorter family name and prefix

    % Option print
\DeclareBoolOption{print}
    % is the same as
    % \DeclareBoolOption[false]{print}

    % Option driver
\ifpdf
  \DeclareStringOption[pdftex]{driver}
\else
  \DeclareStringOption[dvips]{driver}
\fi

    % Alternative interface for driver options
\DeclareVoidOption{dvips}{\SetupDriver}
\DeclareVoidOption{dvipdfm}{\SetupDriver}
\DeclareVoidOption{pdftex}{\SetupDriver}
    % In \SetupDriver we take the current option \CurrentOption
    % and pass it to the driver option.
    % The \expandafter commands expand \CurrentOption at the
    % time, when \SetupDriver is executed and \CurrentOption
    % has the correct meaning.
\newcommand*{\SetupDriver}{%
  \expandafter\@SetupDriver\expandafter{\CurrentOption}%
}
\newcommand*{\@SetupDriver}[1]{%
  \setkeys{MCS}{driver={#1}}%
}

    % Option emph
    % An empty value means, we want to have no color for \emph.
    % If the user specifies option emph without value, the red is used.
\DeclareStringOption{emph}[red]
    % is the same as
    % \DeclareStringOption[]{emph}[red]

    % Default option rule
\DeclareDefaultOption{%
  \ifx\CurrentOptionValue\relax
    \PackageWarningNoLine{\@currname}{%
      Unknown option `\CurrentOption'\MessageBreak
      is passed to package `color'%
    }%
    % Pass the option to package color.
    % Again it is better to expand \CurrentOption.
    \expandafter\PassOptionsToPackage
    \expandafter{\CurrentOption}{color}%
  \else
    % Package color does not take options with values.
    % We provide the standard LaTeX error.
    \@unknownoptionerror
  \fi
}

    % Process options
    % ---------------
\ProcessKeyvalOptions*

    % Implementation depending on option values
    % -----------------------------------------
    % Code for print mode
\ifMCS@print
  \PassOptionsToPackage{monochrome}{color}
    % tells package color to use black and white
\fi

\RequirePackage[\MCS@driver]{color}
    % load package color with the correct driver

    % \emph setup
\ifx\MCS@emph\@empty
    % \@empty is a predefined macro with empty contents.
    % the option value of option emph is empty, thus
    % we do not want a redefinition of \emph.
\else
  \renewcommand*{\emph}[1]{%
    \textcolor{\MCS@emph}{#1}%
  }
\fi
%</example>
%    \end{macrocode}
%
% \section{Package options}
%
% The package \xpackage{kvoptions} knows two package options
% \xoption{patch} and \xoption{debugshow}.
% The options of package \xpackage{kvoptions} are intended
% for authors, not for package/class writers. Inside a package
% it is too late for option \xoption{patch} and \xoption{debugshow}
% enables some messages that are perhaps useful for the debugging
% phase. Also \LaTeX\ is unhappy if a package is loaded
% later again with options that are previously not given.
% Thus package and class authors, stay with |\RequirePackage{kvoptions}|
% without options.
%
% Option \xoption{patch} loads package \xpackage{kvoptions-patch}.
%
% \subsection{Package \xpackage{kvoptions-patch}}
%
% \LaTeX's system of package/class options has some severe limitations
% that especially affects the value part if options are used as
% pair of key and value.
% \begin{itemize}
% \item Spaces are removed, regardless where:
%   \begin{quote}
%     |\documentclass[box=0 0 400 600]{article}|
%   \end{quote}
%   Now each package will see |box=00400600| as global option.
% \item In the previous case also braces would not help:
%   \begin{quote}
%     |\documentclass[box={0 0 400 600}]{article}|
%   \end{quote}
%   The result is an error message:
%   \begin{quote}
%     |! LaTeX Error: Missing \begin{document}.|
%   \end{quote}
%   As local option, however, it works if the package
%   knows about key value options (By using this package, for example).
% \item The requirements on robustness are extremly high.
%   \LaTeX\ expands the option. All that will not work as environment
%   name will break also as option. Even a \cs{relax} will generate
%   an error message:
%   \begin{quote}
%     |! Missing \endcsname inserted.|
%   \end{quote}
%   Of course, \LaTeX\ does not use its protecting mechanisms.
%   On contrary \cs{protect} itself will cause errors.
% \item The options are expanded. But perhaps the package will
%   do that, because it has to setup some things before?
%   Example \xpackage{hyperref}:
%   \begin{quote}
%     |\usepackage[pdfauthor=M\"uller]{hyperref}|
%   \end{quote}
%   Package \xpackage{hyperref} does not see |M\"uller| but
%   its expansion and it does not like it, you get many warnings
%   \begin{quote}
%     |Token not allowed in a PDFDocEncoded string|
%   \end{quote}
%   And the title becomes: |Mu127uller|.
%   Therefore such options must usually be given after package
%   \xpackage{hyperref}
%   is loaded:
%   \begin{quote}
%     |\usepackage{hyperref}|\\
%     |\hypersetup{pdfauthor=Fran\c coise M\"uller}|
%   \end{quote}
%   As package option it will even break with |Fran\c coise|
%   because of the cedilla |\c c|, it is not robust enough.
% \end{itemize}
% For users that do not want with this limitations the package
% offers package \xoption{kvoptions-patch}. It patches \LaTeX's option system
% and tries to teach it also to handle options that are given
% as pairs of key and value and to prevent expansion.
% It can already be used at the very beginning, before \cs{documentclass}:
% \begin{quote}
%   |\RequirePackage{kvoptions-patch}|\\
%   |\documentclass[pdfauthor=Fran\c coise M\"uller]{article}|\\
%   |\usepackage{hyperref}|
% \end{quote}
% The latest time is before the package where you want to use
% problematic values:
% \begin{quote}
%   |\usepackage{kvoptions-patch}|\\
%   |\usepackage[Fran\c coise M\"uller]{hyperref}|
% \end{quote}
% Some remarks:
% \begin{itemize}
% \item The patch requires \eTeX, its \cs{unexpanded} feature
%   is much too nice. It is possible to work around using token
%   registers. But the code becomes longer, slower, more difficult
%   to read and maintain. The package without option \xoption{patch}
%   works and will work without \eTeX.
% \item The code for the patch is quite long, there are many test
%   cases. Thus the probability for bugs is probably not too small.
% \item Since 2008/10/18 v3.0 package \xpackage{kvoptions-patch} is
%   available. Before option \xoption{patch} of package \xpackage{kvoptions}
%   must be used instead. I think, the solution as standalone package
%   \xpackage{kvoptions-patch} is cleaner and avoids option clashes.
% \end{itemize}
%
% \subsection{Option \xoption{debugshow}}
%
% The name of this option follows the convention of packages
% \xpackage{multicol}, \xpackage{tabularx}, and \xpackage{tracefnt}.
% Currently it prints the setting of boolean options, declared
% by \cs{DeclareBoolOption} in the \xfile{.log} file, if that
% boolean option is used.
% You can activate the option by
% \begin{itemize}
% \item |\PassOptionsToPackage{debugshow}{kvoptions}|\\
%       Put this somewhere before package \xpackage{kvoptions} is
%       loaded first, e.g. before \cs{documentclass}.
% \item |\RequirePackage[debugshow]{kvoptions}|\\
%       Before \cs{documentclass} even an author has to use
%       \cs{RequirePackage}. \cs{usepackage} only works after
%       \cs{documentclass}.
% \end{itemize}
% The preferred method is \cs{PassOptionsToPackage}, because
% it does not force the package loading and does not disturb,
% if the package is not loaded later at all.
%
% \section{Limitations}
%
% \subsection{Compatibility}
% \subsubsection{Package \xpackage{kvoptions-patch} vs. package \xpackage{xkvltxp}}
%
% Package \xpackage{xkvltxp} from the \xpackage{xkeyval} project
% has the same goal as package \xpackage{kvoptions-patch}
% and to patch \LaTeX's kernel commands in order to get better support
% for key value options.
% Of course they cannot be used both. The user must decide, which
% method he prefers. Package \xpackage{kvoptions-patch} aborts itself,
% if it detects that \xpackage{xkvltxp} is
% already loaded.
%
% However package \xpackage{xkvltxp} and \xpackage{kvoptions}
% can be used together, example:
% \begin{quote}
% |\usepackage{xkvltxp}|\\
% |\usepackage[...]{foobar} % |\texttt{\itshape foobar using kvoptions}
% \end{quote}
% The other way should work, too.
%
% Package \xpackage{kvoptions-patch} tries to catch more situations
% and to be more robust.
% For example, during the comparison of options it normalizes
% them by removing spaces around |=| and the value. Thus the
% following is not reported as option clash:
% \begin{quote}
%\begin{verbatim}
%\RequirePackage{kvoptions-patch}
%\documentclass{article}
%
%\usepackage[scaled=0.7]{helvet}
%\usepackage[scaled = 0.7]{helvet}
%
%\begin{document}
%\end{document}
%\end{verbatim}
% \end{quote}
%
% \subsection{Limitations}
%
% \subsubsection{Option comparisons}
%
% In some situations \LaTeX\ compares option lists, e.g. option clash check,
% \cs{@ifpackagewith}, or \cs{@ifclasswith}. Apart from catcode and
% sanitizing problems of option \xoption{patch}, there is another problem.
% \LaTeX\ does not know about the type and default values of
% options in key value style. Thus an option clash is reported,
% even if the key value has the same meaning:
% \begin{quote}
% |\usepackage[scaled]{helvet} %| \texttt{\textit{default is }.95}\\
% |\usepackage[.95]{helvet}|\\
% |\usepackage[0.95]{helvet}|
% \end{quote}
%
% \subsubsection{Option list parsing with package \xpackage{kvoptions-patch}}
%
% With package \xpackage{kvoptions-patch} the range of possible values
% in key value specifications is much large, for example
% the comma can be used, if enclosed in curly braces.
%
% Other packages, especially the packages that uses their
% own process option code can be surprised to find tokens
% inside options that they do not expect and errors would
% be the consequence. To avoid errors the options, especially
% the unused option list is sanitized. That means the list
% will only contain tokens with catcode 12 (other) and
% perhaps spaces (catcode 10). This allows a safe parsing
% for other packages. But a comma in the value part is
% no longer protected by curly braces because they have lost
% their special meaning. This is the price for compatibility.
%
% Example:
% \begin{quote}
% |\RequirePackage{kvoptions-patch}|\\
% |\documentclass[a={a,b,c},b]{article}|\\
% |\begin{document}|\\
% |\end{document}|
% \end{quote}
% Result:
% \begin{quote}
% |LaTeX Warning: Unused global option(s):|\\
% |    [a={a,c},b].|
% \end{quote}
%
% \StopEventually{
% }
%
% \section{Implementation}
%
% \subsection{Preamble}
%
%    \begin{macrocode}
%<*package>
%    \end{macrocode}
%
% \paragraph{Reload check and  identification.}
%    Reload check, especially if the package is not used with \LaTeX.
%    \begin{macrocode}
\begingroup\catcode61\catcode48\catcode32=10\relax%
  \catcode13=5 % ^^M
  \endlinechar=13 %
  \catcode35=6 % #
  \catcode39=12 % '
  \catcode44=12 % ,
  \catcode45=12 % -
  \catcode46=12 % .
  \catcode58=12 % :
  \catcode64=11 % @
  \catcode123=1 % {
  \catcode125=2 % }
  \expandafter\let\expandafter\x\csname ver@kvoptions.sty\endcsname
  \ifx\x\relax % plain-TeX, first loading
  \else
    \def\empty{}%
    \ifx\x\empty % LaTeX, first loading,
      % variable is initialized, but \ProvidesPackage not yet seen
    \else
      \expandafter\ifx\csname PackageInfo\endcsname\relax
        \def\x#1#2{%
          \immediate\write-1{Package #1 Info: #2.}%
        }%
      \else
        \def\x#1#2{\PackageInfo{#1}{#2, stopped}}%
      \fi
      \x{kvoptions}{The package is already loaded}%
      \aftergroup\endinput
    \fi
  \fi
\endgroup%
%    \end{macrocode}
%    Package identification:
%    \begin{macrocode}
\begingroup\catcode61\catcode48\catcode32=10\relax%
  \catcode13=5 % ^^M
  \endlinechar=13 %
  \catcode35=6 % #
  \catcode39=12 % '
  \catcode40=12 % (
  \catcode41=12 % )
  \catcode44=12 % ,
  \catcode45=12 % -
  \catcode46=12 % .
  \catcode47=12 % /
  \catcode58=12 % :
  \catcode64=11 % @
  \catcode91=12 % [
  \catcode93=12 % ]
  \catcode123=1 % {
  \catcode125=2 % }
  \expandafter\ifx\csname ProvidesPackage\endcsname\relax
    \def\x#1#2#3[#4]{\endgroup
      \immediate\write-1{Package: #3 #4}%
      \xdef#1{#4}%
    }%
  \else
    \def\x#1#2[#3]{\endgroup
      #2[{#3}]%
      \ifx#1\@undefined
        \xdef#1{#3}%
      \fi
      \ifx#1\relax
        \xdef#1{#3}%
      \fi
    }%
  \fi
\expandafter\x\csname ver@kvoptions.sty\endcsname
\ProvidesPackage{kvoptions}%
  [2016/05/16 v3.12 Key value format for package options (HO)]%
%    \end{macrocode}
%
% \paragraph{Catcodes}
%
%    \begin{macrocode}
\begingroup\catcode61\catcode48\catcode32=10\relax%
  \catcode13=5 % ^^M
  \endlinechar=13 %
  \catcode123=1 % {
  \catcode125=2 % }
  \catcode64=11 % @
  \def\x{\endgroup
    \expandafter\edef\csname KVO@AtEnd\endcsname{%
      \endlinechar=\the\endlinechar\relax
      \catcode13=\the\catcode13\relax
      \catcode32=\the\catcode32\relax
      \catcode35=\the\catcode35\relax
      \catcode61=\the\catcode61\relax
      \catcode64=\the\catcode64\relax
      \catcode123=\the\catcode123\relax
      \catcode125=\the\catcode125\relax
    }%
  }%
\x\catcode61\catcode48\catcode32=10\relax%
\catcode13=5 % ^^M
\endlinechar=13 %
\catcode35=6 % #
\catcode64=11 % @
\catcode123=1 % {
\catcode125=2 % }
\def\TMP@EnsureCode#1#2{%
  \edef\KVO@AtEnd{%
    \KVO@AtEnd
    \catcode#1=\the\catcode#1\relax
  }%
  \catcode#1=#2\relax
}
\TMP@EnsureCode{1}{14}% ^^A (comment)
\TMP@EnsureCode{2}{14}% ^^A (comment)
\TMP@EnsureCode{33}{12}% !
\TMP@EnsureCode{39}{12}% '
\TMP@EnsureCode{40}{12}% (
\TMP@EnsureCode{41}{12}% )
\TMP@EnsureCode{42}{12}% *
\TMP@EnsureCode{44}{12}% ,
\TMP@EnsureCode{45}{12}% -
\TMP@EnsureCode{46}{12}% .
\TMP@EnsureCode{47}{12}% /
\TMP@EnsureCode{58}{12}% :
\TMP@EnsureCode{62}{12}% >
\TMP@EnsureCode{91}{12}% [
\TMP@EnsureCode{93}{12}% ]
\TMP@EnsureCode{94}{7}% ^ (superscript)
\TMP@EnsureCode{96}{12}% `
\edef\KVO@AtEnd{\KVO@AtEnd\noexpand\endinput}
%    \end{macrocode}
%
% \paragraph{External resources.}
%    The package extends the support for key value pairs of
%    package \cs{keyval} to package options. Thus the package
%    needs to be loaded anyway. and we use it for
%    \cs{SetupKeyvalOptions}. AFAIK this does not disturb
%    users of \xpackage{xkeyval}.
%    \begin{macrocode}
\@ifundefined{define@key}{%
  \RequirePackage{keyval}\relax
}{}
%    \end{macrocode}
%
%    Macro \cs{DeclareLocalOptions} parses a comma separated key list
%    and uses \cs{comma@parse} of package \xpackage{kvsetkeys},
%    version 1.3.
%    \begin{macrocode}
\RequirePackage{ltxcmds}[2010/12/02]
\RequirePackage{kvsetkeys}[2007/09/29]
%    \end{macrocode}
%
% \paragraph{Provide macros for \plainTeX.}
%    \begin{macrocode}
\@ifundefined{@x@protect}{%
  \def\@x@protect#1\fi#2#3{%
    \fi\protect#1%
  }%
  \let\@typeset@protect\relax
}{}
\@ifundefined{@currname}{%
  \def\@currname{}%
}{}
\@ifundefined{@currext}{%
  \def\@currext{}%
}{}
%    \end{macrocode}
%
% \paragraph{Options}
%    Option \xoption{debugshow} enables additional lines of
%    code that prints information into the \xfile{.log}
%    file.
%    \begin{macrocode}
\DeclareOption{debugshow}{\catcode\@ne=9 }
%    \end{macrocode}
%
%    \begin{macrocode}
\DeclareOption{patch}{%
  \AtEndOfPackage{%
    \RequirePackage{kvoptions-patch}[2016/05/16]%
  }%
}
%    \end{macrocode}
%
%    Optionen auswerten:
%    \begin{macrocode}
\ProcessOptions\relax
%    \end{macrocode}
%
% \subsection{Option declaration macros}
%
% \subsubsection{\cs{SetupKeyvalOptions}}
%
%    The family for the key value pairs can be setup once
%    and is remembered later.
%    The package name seems a reasonable default for the
%    family key, if it is not set by the package author.
%
%    \begin{macro}{\KVO@family}
%    We cannot store the family setting in one macro,
%    because the package should be usable for many
%    other packages, too. Thus we remember the family
%    setting in a macro, whose name contains the
%    package name with extension, a key in \LaTeX's
%    class/package system.
%    \begin{macrocode}
\define@key{KVO}{family}{%
  \expandafter\edef\csname KVO@family@%
      \@currname.\@currext\endcsname{#1}%
}
\def\KVO@family{%
  \@ifundefined{KVO@family@\@currname.\@currext}{%
    \@currname
  }{%
    \csname KVO@family@\@currname.\@currext\endcsname
  }%
}
%    \end{macrocode}
%    \end{macro}
%
%    \begin{macro}{\KVO@prefix}
%    The value settings of options that are declared by \cs{DeclareBoolOption}
%    and \cs{DeclareStringOption} need to be saved in macros.
%    in the first case this is a switch \cs{if}\meta{prefix}\meta{key},
%    in the latter case a macro \cs{}\meta{prefix}\meta{key}.
%    The prefix can be configured, by \xoption{prefix} that is declared
%    here. The default is the package name with |@| appended.
%    \begin{macrocode}
\define@key{KVO}{prefix}{%
  \expandafter\edef\csname KVO@prefix@%
      \@currname.\@currext\endcsname{#1}%
}
\def\KVO@prefix{%
  \ltx@ifundefined{KVO@prefix@\@currname.\@currext}{%
    \@currname @%
  }{%
    \csname KVO@prefix@\@currname.\@currext\endcsname
  }%
}
%    \end{macrocode}
%    \end{macro}
%
%    \begin{macrocode}
\define@key{KVO}{setkeys}{%
  \expandafter\def\csname KVO@setkeys@%
      \@currname.\@currext\endcsname{#1}%
}
%    \end{macrocode}
%    \begin{macro}{\KVO@setkeys}
%    \begin{macrocode}
\def\KVO@setkeys{%
  \ltx@IfUndefined{KVO@setkeys@\@currname.\@currext}{%
    \setkeys
  }{%
    \csname KVO@setkeys@\@currname.\@currext\endcsname
  }%
}
%    \end{macrocode}
%    \end{macro}
%
%    \begin{macro}{\SetupKeyvalOptions}
%    The argument of \cs{SetupKeyvalOptions} expects a key value
%    list, known keys are \xoption{family} and \xoption{prefix}.
%    \begin{macrocode}
\newcommand*{\SetupKeyvalOptions}{%
  \kvsetkeys{KVO}%
}
%    \end{macrocode}
%    \end{macro}
%
% \subsubsection{\cs{DeclareBoolOption}}
%
%    \begin{macro}{\DeclareBoolOption}
%    Usually options of boolean type can be given by the
%    user without value and this means a setting to \emph{true}.
%    We follow this convention here. Also it simplifies the
%    user interface.
%
%    The switch is created and initialized with \emph{false}.
%    The default setting can be overwritten by the optional
%    argument.
%
%    \LaTeX's \cs{newif} does not check for already defined
%    macros, therefore we add this check here to prevent
%    the user from accidently redefining of \TeX's primitives
%    and other macros.
%    \begin{macrocode}
\newcommand*{\DeclareBoolOption}[2][false]{%
  \KVO@ifdefinable{if\KVO@prefix#2}{%
    \KVO@ifdefinable{\KVO@prefix#2true}{%
      \KVO@ifdefinable{\KVO@prefix#2false}{%
        \csname newif\expandafter\endcsname
        \csname if\KVO@prefix#2\endcsname
        \@ifundefined{\KVO@prefix#2#1}{%
          \PackageWarning{kvoptions}{%
            Initialization of option `#2' failed,\MessageBreak
            cannot set boolean option to `#1',\MessageBreak
            use `true' or `false', now using `false'%
          }%
        }{%
          \csname\KVO@prefix#2#1\endcsname
        }%
        \begingroup
          \edef\x{\endgroup
            \noexpand\define@key{\KVO@family}{#2}[true]{%
              \noexpand\KVO@boolkey{\@currname}%
              \ifx\@currext\@clsextension
                \noexpand\@clsextension
              \else
                \noexpand\@pkgextension
              \fi
              {\KVO@prefix}{#2}{####1}%
            }%
          }%
        \x
      }%
    }%
  }%
}
%    \end{macrocode}
%    \end{macro}
%
%    \begin{macro}{\DeclareComplementaryOption}
%    The first argument is the key name, the second the
%    key that must be a boolean option with the same
%    current family and prefix. A new switch is not
%    created for the new key, we have already a switch.
%    Instead we define switch setting commands to work
%    on the parent switch.
%    \begin{macrocode}
\newcommand*{\DeclareComplementaryOption}[2]{%
  \@ifundefined{if\KVO@prefix#2}{%
    \PackageError{kvoptions}{%
      Cannot generate option code for `#1',\MessageBreak
      parent switch `#2' does not exist%
    }{%
      You are inside %
      \ifx\@currext\@clsextension class\else package\fi\space
      `\@currname.\@currext'.\MessageBreak
      `\KVO@family' is used as familiy %
      for the keyval options.\MessageBreak
      `\KVO@prefix' serves as prefix %
      for internal switch macros.\MessageBreak
      \MessageBreak
      \@ehc
    }%
  }{%
    \KVO@ifdefinable{\KVO@prefix#1true}{%
      \KVO@ifdefinable{\KVO@prefix#1false}{%
        \expandafter\let\csname\KVO@prefix#1false\expandafter\endcsname
          \csname\KVO@prefix#2true\endcsname
        \expandafter\let\csname\KVO@prefix#1true\expandafter\endcsname
          \csname\KVO@prefix#2false\endcsname
%    \end{macrocode}
%    The same code part as in \cs{DeclareBoolOption} can
%    now be used.
%    \begin{macrocode}
        \begingroup
          \edef\x{\endgroup
            \noexpand\define@key{\KVO@family}{#1}[true]{%
              \noexpand\KVO@boolkey{\@currname}%
              \ifx\@currext\@clsextension
                \noexpand\@clsextension
              \else
                \noexpand\@pkgextension
              \fi
              {\KVO@prefix}{#1}{####1}%
            }%
          }%
        \x
      }%
    }%
  }%
}
%    \end{macrocode}
%    \end{macro}
%
%    \begin{macro}{\KVO@ifdefinable}
%    Generate the command token LaTeX's \cs{@ifdefinable} expects.
%    \begin{macrocode}
\def\KVO@ifdefinable#1{%
  \expandafter\@ifdefinable\csname #1\endcsname
}
%    \end{macrocode}
%    \end{macro}
%
%    \begin{macro}{\KVO@boolkey}
%    We check explicitly for |true| and |false| to prevent
%    the user from accidently calling other macros.
%    \begin{quote}
%    \begin{tabular}{@{}ll@{}}
%    |#1|& package/class name\\
%    |#2|& |\@pkgextension|/|\@clsextension|\\
%    |#3|& prefix \\
%    |#4|& key name\\
%    |#5|& new value
%    \end{tabular}
%    \end{quote}
%    \begin{macrocode}
\def\KVO@boolkey#1#2#3#4#5{%
  \edef\KVO@param{#5}%
  \ltx@onelevel@sanitize\KVO@param
  \ifx\KVO@param\KVO@true
    \expandafter\@firstofone
  \else
    \ifx\KVO@param\KVO@false
      \expandafter\expandafter\expandafter\@firstofone
    \else
      \ifx#2\@clsextension
        \expandafter\ClassWarning
      \else
        \expandafter\PackageWarning
      \fi
      {#1}{%
        Value `\KVO@param' is not supported by\MessageBreak
        option `#4'%
      }%
      \expandafter\expandafter\expandafter\@gobble
    \fi
  \fi
  {%
    ^^A\ifx#2\@clsextension
    ^^A  \expandafter\ClassInfo
    ^^A\else
    ^^A  \expandafter\PackageInfo
    ^^A\fi
    ^^A{#1}{[option] #4=\KVO@param}%
    \csname#3#4\KVO@param\endcsname
  }%
}
%    \end{macrocode}
%    \end{macro}
%    \begin{macro}{\KVO@true}
%    \begin{macro}{\KVO@false}
%    The macros \cs{KVO@true} and \cs{KVO@false} are used
%    for string comparisons. After \cs{ltx@onelevel@sanitize}
%    we have only tokens with catcode 12 (other).
%    \begin{macrocode}
\def\KVO@true{true}
\def\KVO@false{false}
\ltx@onelevel@sanitize\KVO@true
\ltx@onelevel@sanitize\KVO@false
%    \end{macrocode}
%    \end{macro}
%    \end{macro}
%
% \subsubsection{\cs{DeclareStringOption}}
%
%    \begin{macro}{\DeclareStringOption}
%    \begin{macrocode}
\newcommand*{\DeclareStringOption}[2][]{%
  \@ifnextchar[{%
    \KVO@DeclareStringOption{#1}{#2}@%
  }{%
    \KVO@DeclareStringOption{#1}{#2}{}[]%
  }%
}
%    \end{macrocode}
%    \end{macro}
%    \begin{macro}{\KVO@DeclareStringOption}
%    \begin{macrocode}
\def\KVO@DeclareStringOption#1#2#3[#4]{%
  \KVO@ifdefinable{\KVO@prefix#2}{%
    \@namedef{\KVO@prefix#2}{#1}%
    \begingroup
      \ifx\\#3\\%
        \toks@{}%
      \else
        \toks@{[{#4}]}%
      \fi
      \edef\x{\endgroup
        \noexpand\define@key{\KVO@family}{#2}\the\toks@{%
          ^^A\begingroup
          ^^A  \toks@{####1}%
          ^^A  \ifx\@currext\@clsextension
          ^^A    \noexpand\ClassInfo
          ^^A  \else
          ^^A    \noexpand\PackageInfo
          ^^A  \fi
          ^^A  {\@currname}{%
          ^^A    [option] #2={\noexpand\the\toks@}%
          ^^A  }%
          ^^A\endgroup
          \noexpand\def
          \expandafter\noexpand\csname\KVO@prefix#2\endcsname{####1}%
        }%
      }%
    \x
  }%
}
%    \end{macrocode}
%    \end{macro}
%
% \subsubsection{\cs{DeclareVoidOption}}
%
%    \begin{macro}{\DeclareVoidOption}
%    \begin{macrocode}
\newcommand*{\DeclareVoidOption}[2]{%
  \begingroup
    \let\next\@gobbletwo
    \KVO@ifdefinable{\KVO@prefix#1}{%
      \let\next\@firstofone
    }%
  \expandafter\endgroup
  \next{%
    \begingroup
      \edef\x{\endgroup
        \noexpand\define@key{\KVO@family}{#1}[\KVO@VOID@]{%
          \noexpand\KVO@voidkey{\@currname}%
          \ifx\@currext\@clsextension
            \noexpand\@clsextension
          \else
            \noexpand\@pkgextension
          \fi
          {#1}%
          {####1}%
          \expandafter\noexpand\csname\KVO@prefix#1\endcsname
        }%
      }%
    \x
    \begingroup
      \toks@{#2}%
    \expandafter\endgroup
    \expandafter\def
    \csname\KVO@prefix#1\expandafter\endcsname
    \expandafter{\the\toks@}%
  }%
}
\def\KVO@VOID@{@VOID@}
%    \end{macrocode}
%    \end{macro}
%    \begin{macro}{\KVO@voidkey}
%    \begin{tabular}{@{}ll@{}}
%    |#1|& package/class name\\
%    |#2|& |\@pkgextension|/|\@clsextension|\\
%    |#3|& key name\\
%    |#4|& default (|@VOID@|)\\
%    |#5|& macro with option code
%    \end{tabular}
%    \begin{macrocode}
\def\KVO@voidkey#1#2#3#4{%
  \def\CurrentOption{#3}%
  \begingroup
    \def\x{#4}%
  \expandafter\endgroup
  \ifx\x\KVO@VOID@
  \else
    \ifx#2\@clsextension
      \expandafter\ClassWarning
    \else
      \expandafter\PackageWarning
    \fi
    {#1}{%
      Unexpected value for option `#3'\MessageBreak
      is ignored%
    }%
  \fi
  ^^A\ifx#2\@clsextension
  ^^A  \expandafter\ClassInfo
  ^^A\else
  ^^A  \expandafter\PackageInfo
  ^^A\fi
  ^^A{#1}{[option] #3}%
}
%    \end{macrocode}
%    \end{macro}
%
% \subsubsection{\cs{DeclareDefaultOption}}
%
%    \begin{macro}{\DeclareDefaultOption}
%    \begin{macrocode}
\newcommand*{\DeclareDefaultOption}{%
  \@namedef{KVO@default@\@currname.\@currext}%
}
%    \end{macrocode}
%    \end{macro}
%
% \subsubsection{\cs{DeclareLocalOptions}}
%
%    \begin{macro}{\DeclareLocalOptions}
%    \begin{macrocode}
\newcommand*{\DeclareLocalOptions}[1]{%
  \comma@parse{#1}\KVO@DeclareLocalOption
}
%    \end{macrocode}
%    \end{macro}
%    \begin{macro}{\KVO@DeclareLocalOption}
%    \begin{macrocode}
\def\KVO@DeclareLocalOption#1{%
  \expandafter\def\csname KVO@local@\KVO@family @#1\endcsname{}%
}
%    \end{macrocode}
%    \end{macro}
%
% \subsection{Dynamic options}
%
% \subsubsection{\cs{DisableKeyvalOption}}
%
%    \begin{macrocode}
\SetupKeyvalOptions{%
  family=KVOdyn,%
  prefix=KVOdyn@%
}
\DeclareBoolOption[true]{global}
\DeclareComplementaryOption{local}{global}
\DeclareStringOption[undef]{action}
\let\KVOdyn@name\relax
\let\KVOdyn@ext\@empty
\define@key{KVOdyn}{class}{%
  \def\KVOdyn@name{#1}%
  \let\KVOdyn@ext\@clsextension
}
\define@key{KVOdyn}{package}{%
  \def\KVOdyn@name{#1}%
  \let\KVOdyn@ext\@pkgextension
}
\newcommand*{\DisableKeyvalOption}[3][]{%
  \begingroup
    \kvsetkeys{KVOdyn}{#1}%
    \def\x{\endgroup}%
    \@ifundefined{KVO@action@\KVOdyn@action}{%
      \PackageError{kvoptions}{%
        Unknown disable action %
        `\expandafter\strip@prefix\meaning\KVOdyn@action'\MessageBreak
        for option `#3' in keyval family '#2'%
      }\@ehc
    }{%
      \csname KVO@action@\KVOdyn@action\endcsname{#2}{#3}%
    }%
  \x
}
\def\KVO@action@undef#1#2{%
  \edef\x{\endgroup
    \ifKVOdyn@global\global\fi
    \let
    \expandafter\noexpand\csname KV@#1@#2\endcsname
    \relax
    \ifKVOdyn@global\global\fi
    \let
    \expandafter\noexpand\csname KV@#1@#2@default\endcsname
    \relax
  }%
  ^^A\PackageInfo{kvoptions}{%
  ^^A  [option] key `#2' of family `#1'\MessageBreak
  ^^A  is disabled (undef, \ifKVOdyn@global global\else local\fi)%
  ^^A}%
}
\def\KVO@action@ignore#1#2{%
  \edef\x{\endgroup
    \ifKVOdyn@global\global\fi
    \let
    \expandafter\noexpand\csname KV@#1@#2\endcsname
    \noexpand\@gobble
    \ifKVOdyn@global\global\fi
    \let
    \expandafter\noexpand\csname KV@#1@#2@default\endcsname
    \noexpand\@empty
  }%
  ^^A\PackageInfo{kvoptions}{%
  ^^A  [option] key `#2' of family `#1'\MessageBreak
  ^^A  is disabled (ignore, \ifKVOdyn@global global\else local\fi)%
  ^^A}%
}
\def\KVO@action@error{%
  \KVO@do@action{error}%
}
\def\KVO@action@warning{%
  \KVO@do@action{warning}%
}
%    \end{macrocode}
%    \begin{tabular}{@{}ll@{}}
%    |#1|& |error| or |warning|\\
%    |#2|& \meta{family}\\
%    |#3|& \meta{key}\\
%    \end{tabular}
%    \begin{macrocode}
\def\KVO@do@action#1#2#3{%
  \ifx\KVOdyn@name\relax
    \PackageError{kvoptions}{%
      Action type `#1' needs package/class name\MessageBreak
      for key `#3' in family `#2'%
    }\@ehc
  \else
    \edef\x{\endgroup
      \noexpand\define@key{#2}{#3}[]{%
        \expandafter\noexpand\csname KVO@disable@#1\endcsname
        {\KVOdyn@name}\noexpand\KVOdyn@ext{#3}%
      }%
      \ifKVOdyn@global
        \global\let
        \expandafter\noexpand\csname KV@#2@#3\endcsname
        \expandafter\noexpand\csname KV@#2@#3\endcsname
        \global\let
        \expandafter\noexpand\csname KV@#2@#3@default\endcsname
        \expandafter\noexpand\csname KV@#2@#3@default\endcsname
      \fi
    }%
    ^^A\ifx\KVOdyn@ext\@clsextension
    ^^A  \expandafter\ClassInfo
    ^^A\else
    ^^A   \expandafter\PackageInfo
    ^^A\fi
    ^^A{\KVOdyn@name}{%
    ^^A  [option] key `#3' of family `#2'\MessageBreak
    ^^A  is disabled (#1, \ifKVOdyn@global global\else local\fi)%
    ^^A}%
  \fi
}
\def\KVO@disable@error#1#2#3{%
  \ifx#2\@clsextension
    \expandafter\ClassError
  \else
    \expandafter\PackageError
  \fi
  {#1}{%
    Option `#3' is given too late,\MessageBreak
    now the option is ignored%
  }\@ehc
}
\def\KVO@disable@warning#1#2#3{%
  \ifx#2\@clsextension
    \expandafter\ClassWarning
  \else
    \expandafter\PackageWarning
  \fi
  {#1}{%
    Option `#3' is already consumed\MessageBreak
    and has no effect%
  }%
}
%    \end{macrocode}
%
% \subsection{Change option code}
%
% \subsubsection{\cs{AddToKeyvalOption}}
%
%    \begin{macro}{\AddToKeyvalOption}
%    \begin{macrocode}
\newcommand*{\AddToKeyvalOption}{%
  \@ifstar{%
    \begingroup
      \edef\x{\endgroup
        \noexpand\KVO@AddToKeyvalOption{\KVO@family}%
      }%
    \x
  }%
  \KVO@AddToKeyvalOption
}
%    \end{macrocode}
%    \end{macro}
%    \begin{macro}{\KVO@AddToKeyvalOption}
%    \begin{macrocode}
\def\KVO@AddToKeyvalOption#1#2{%
  \@ifundefined{KV@#1@#2}{%
    \PackageWarning{kvoptions}{%
      Key `#2' of family `#1' does not exist.\MessageBreak
      Ignoring \string\AddToKeyvalOption
    }%
    \@gobble
  }{%
    \edef\KVO@next{%
      \noexpand\KVO@@AddToKeyvalOption
      \expandafter\noexpand\csname KV@#1@#2\endcsname
    }%
    \afterassignment\KVO@next
    \def\KVO@temp##1%
  }%
}
%    \end{macrocode}
%    \end{macro}
%    \begin{macro}{\KVO@@AddToKeyvalOption}
%    \begin{macrocode}
\def\KVO@@AddToKeyvalOption#1{%
  \begingroup
    \toks@\expandafter{#1{##1}}%
    \toks@\expandafter{\the\expandafter\toks@\KVO@temp{##1}}%
    \edef\x{\endgroup
      \noexpand\def\noexpand#1####1{\the\toks@}%
    }%
  \x
}
%    \end{macrocode}
%    \end{macro}
%
% \subsection{Process options}
%
% \subsubsection{Get global options}
%
%    Package \xpackage{xkeyval} removes options with equal signs from
%    the global options (\cs{@classoptionslist}). The effect is that
%    other packages and classes will not see these global options anymore.
%    A bug-report was answered that this behaviour is ``by design''.
%    Thus I call it a design bug. Now getting the global options require
%    an algorithm instead of a simple macro call.
%    \begin{macrocode}
%</package>
%<*package|patch>
%    \end{macrocode}
%    \begin{macro}{\KVO@IfDefThen}
%    Call |#2| if command |#1| is defined and not \cs{relax}.
%    (Package \xpackage{kvoptions-patch} does not load
%    package \xpackage{ltxcmds}.)
%    \begin{macrocode}
\def\KVO@IfDefThen#1#2{%
  \ifx#1\ltx@undefined
  \else
    \ifx#1\relax
    \else
      #2%
    \fi
  \fi
}%
%    \end{macrocode}
%    \end{macro}
%    \begin{macro}{\KVO@GetClassOptionsList}
%    \begin{macrocode}
\def\KVO@GetClassOptionsList{%
  \let\KVO@classoptionslist\@classoptionslist
  \KVO@IfDefThen\@classoptionslist{%
    \KVO@IfDefThen\XKV@documentclass{%
      \ifx\XKV@documentclass\ltx@empty
      \else
        \KVO@IfDefThen\XKV@classoptionslist{%
          \ifx\XKV@classoptionslist\ltx@empty
          \else
            \let\KVO@classoptionslist\XKV@classoptionslist
          \fi
        }%
      \fi
    }%
  }%
}%
%    \end{macrocode}
%    \end{macro}
%    \begin{macrocode}
%</package|patch>
%<*package>
%    \end{macrocode}
%
% \subsubsection{\cs{ProcessKeyvalOptions}}
%
%    \begin{macro}{\ProcessKeyvalOptions}
%    If the optional star is given, we get the family
%    name and expand it for safety.
%    \begin{macrocode}
\newcommand*{\ProcessKeyvalOptions}{%
  \@ifstar{%
    \begingroup
      \edef\x{\endgroup
        \noexpand\KVO@ProcessKeyvalOptions{\KVO@family}%
      }%
    \x
  }%
  \KVO@ProcessKeyvalOptions
}
%    \end{macrocode}
%    \end{macro}
%
%    \begin{macrocode}
\def\KVO@ProcessKeyvalOptions#1{%
  \let\@tempc\relax
  \let\KVO@temp\@empty
%    \end{macrocode}
%    Add any global options that are known to KV to the start of
%    the list being built in |\KVO@temp| and mark them used (by
%    removing them from the unused option list).
%    \begin{macrocode}
  \ifx\@currext\@clsextension
  \else
    \KVO@GetClassOptionsList
    \ifx\KVO@classoptionslist\relax
    \else
      \@for\KVO@CurrentOption:=\KVO@classoptionslist\do{%
        \@ifundefined{KV@#1@\expandafter\KVO@getkey
                      \KVO@CurrentOption=\@nil}{%
        }{%
          \@ifundefined{KVO@local@#1@\expandafter\KVO@getkey
                        \KVO@CurrentOption=\@nil}{%
            \ifx\KVO@Patch Y%
              \edef\KVO@temp{%
                \etex@unexpanded\expandafter{%
                  \KVO@temp
                }%
                ,%
                \etex@unexpanded\expandafter{%
                  \KVO@CurrentOption
                }%
                ,%
              }%
              \ltx@onelevel@sanitize\KVO@CurrentOption
            \else
              \edef\KVO@temp{%
                \KVO@temp
                ,%
                \KVO@CurrentOption
                ,%
              }%
            \fi
            \@expandtwoargs\@removeelement\KVO@CurrentOption
              \@unusedoptionlist\@unusedoptionlist
          }{}%
        }%
      }%
    \fi
  \fi
%    \end{macrocode}
%     Now stick the package options at the end of the list and wrap
%     in a call to \cs{setkeys}. A class ignores unknown global
%     options, we must remove them to prevent error messages
%     from \cs{setkeys}.
%    \begin{macrocode}
  \begingroup
    \toks\tw@{}%
    \@ifundefined{opt@\@currname.\@currext}{%
      \toks@\expandafter{\KVO@temp}%
    }{%
      \toks@\expandafter\expandafter\expandafter{%
        \csname opt@\@currname.\@currext\endcsname
      }%
      \ifx\@currext\@clsextension
        \edef\CurrentOption{\the\toks@}%
        \toks@\expandafter{\KVO@temp}%
        \@for\CurrentOption:=\CurrentOption\do{%
          \@ifundefined{%
            KV@#1@\expandafter\KVO@getkey\CurrentOption=\@nil
          }{%
%    \end{macrocode}
%    A class puts not used options in the unused option list
%    unless there is a default handler.
%    \begin{macrocode}
            \@ifundefined{KVO@default@\@currname.\@currext}{%
              \ifx\KVO@Patch Y%
                \ltx@onelevel@sanitize\CurrentOption
              \fi
              \ifx\@unusedoptionlist\@empty
                \global\let\@unusedoptionlist\CurrentOption
              \else
                \expandafter\expandafter\expandafter\gdef
                \expandafter\expandafter\expandafter\@unusedoptionlist
                \expandafter\expandafter\expandafter{%
                  \expandafter\@unusedoptionlist
                  \expandafter,\CurrentOption
                }%
              \fi
            }{%
              \toks\tw@\expandafter{%
                \the\toks\expandafter\tw@\expandafter,\CurrentOption
              }%
            }%
          }{%
            \toks@\expandafter{%
              \the\expandafter\toks@\expandafter,\CurrentOption
            }%
          }%
        }%
      \else
%    \end{macrocode}
%    Without default action we pass all options to \cs{setkeys}.
%    Otherwise we have to check which options are known.
%    These are passed to \cs{setkeys}. For the others the default
%    action is performed.
%    \begin{macrocode}
        \@ifundefined{KVO@default@\@currname.\@currext}{%
          \toks@\expandafter\expandafter\expandafter{%
            \expandafter\KVO@temp\the\toks@
          }%
        }{%
          \edef\CurrentOption{\the\toks@}%
          \toks@\expandafter{\KVO@temp}%
          \@for\CurrentOption:=\CurrentOption\do{%
            \@ifundefined{%
              KV@#1@\expandafter\KVO@getkey\CurrentOption=\@nil
            }{%
              \toks\tw@\expandafter{%
                \the\toks\expandafter\tw@\expandafter,\CurrentOption
              }%
            }{%
              \toks@\expandafter{%
                \the\expandafter\toks@\expandafter,\CurrentOption
              }%
            }%
          }%
        }%
      \fi
    }%
    \edef\KVO@temp{\endgroup
      \noexpand\KVO@calldefault{\the\toks\tw@}%
      \noexpand\KVO@setkeys{#1}{\the\toks@}%
    }%
  \KVO@temp
%    \end{macrocode}
%    Some cleanup of \cs{ProcessOptions}.
%    \begin{macrocode}
  \let\CurrentOption\@empty
  \AtEndOfPackage{\let\@unprocessedoptions\relax}%
}
%    \end{macrocode}
%
% \subsubsection{\cs{ProcessLocalKeyvalOptions}}
%
%    \begin{macro}{\ProcessLocalKeyvalOptions}
%    If the optional star is given, we get the family
%    name and expand it for safety.
%    \begin{macrocode}
\newcommand*{\ProcessLocalKeyvalOptions}{%
  \@ifstar{%
    \begingroup
      \edef\x{\endgroup
        \noexpand\KVO@ProcessLocalKeyvalOptions{\KVO@family}%
      }%
    \x
  }%
  \KVO@ProcessLocalKeyvalOptions
}
%    \end{macrocode}
%    \end{macro}
%
%    \begin{macrocode}
\def\KVO@ProcessLocalKeyvalOptions#1{%
  \let\@tempc\relax
  \let\KVO@temp\@empty
%    \end{macrocode}
%    Check if \cs{ProcessLocalKeyvalOptions} is called inside
%    a package.
%    \begin{macrocode}
  \ifx\@currext\@pkgextension
  \else
    \PackageError{kvoptions}{%
      \string\ProcessLocalKeyvalOptions\space is intended for packages only%
    }\@ehc
  \fi
%    \end{macrocode}
%     The package options are put into toks register \cs{toks@}.
%    \begin{macrocode}
  \begingroup
    \toks\tw@{}%
    \@ifundefined{opt@\@currname.\@currext}{%
      \toks@\expandafter{\KVO@temp}%
    }{%
      \toks@\expandafter\expandafter\expandafter{%
        \csname opt@\@currname.\@currext\endcsname
      }%
%    \end{macrocode}
%    Without default action we pass all options to \cs{setkeys}.
%    Otherwise we have to check which options are known.
%    These are passed to \cs{setkeys}. For the others the default
%    action is performed.
%    \begin{macrocode}
      \@ifundefined{KVO@default@\@currname.\@currext}{%
        \toks@\expandafter\expandafter\expandafter{%
          \expandafter\KVO@temp\the\toks@
        }%
      }{%
        \edef\CurrentOption{\the\toks@}%
        \toks@\expandafter{\KVO@temp}%
        \@for\CurrentOption:=\CurrentOption\do{%
          \@ifundefined{%
            KV@#1@\expandafter\KVO@getkey\CurrentOption=\@nil
          }{%
            \toks\tw@\expandafter{%
              \the\toks\expandafter\tw@\expandafter,\CurrentOption
            }%
          }{%
            \toks@\expandafter{%
              \the\expandafter\toks@\expandafter,\CurrentOption
            }%
          }%
        }%
      }%
    }%
    \edef\KVO@temp{\endgroup
      \noexpand\KVO@calldefault{\the\toks\tw@}%
      \noexpand\KVO@setkeys{#1}{\the\toks@}%
    }%
  \KVO@temp
%    \end{macrocode}
%    Some cleanup of \cs{ProcessOptions}.
%    \begin{macrocode}
  \let\CurrentOption\@empty
  \AtEndOfPackage{\let\@unprocessedoptions\relax}%
}
%    \end{macrocode}
%
% \subsubsection{Helper macros}
%
%    \begin{macro}{\KVO@getkey}
%    Extract the key part of a key=value pair.
%    \begin{macrocode}
\def\KVO@getkey#1=#2\@nil{#1}
%    \end{macrocode}
%    \end{macro}
%
%    \begin{macro}{\KVO@calldefault}
%    \begin{macrocode}
\def\KVO@calldefault#1{%
  \begingroup
    \def\x{#1}%
  \expandafter\endgroup
  \ifx\x\@empty
  \else
    \@for\CurrentOption:=#1\do{%
      \ifx\CurrentOption\@empty
      \else
        \expandafter\KVO@setcurrents\CurrentOption=\@nil
        \@nameuse{KVO@default@\@currname.\@currext}%
      \fi
    }%
  \fi
}
%    \end{macrocode}
%    \end{macro}
%    \begin{macro}{\KVO@setcurrents}
%    Extract the key part of a key=value pair.
%    \begin{macrocode}
\def\KVO@setcurrents#1=#2\@nil{%
  \def\CurrentOptionValue{#2}%
  \ifx\CurrentOptionValue\@empty
    \let\CurrentOptionKey\CurrentOption
    \let\CurrentOptionValue\relax
  \else
    \edef\CurrentOptionKey{\zap@space#1 \@empty}%
    \expandafter\KVO@setcurrentvalue\CurrentOption\@nil
  \fi
}
%    \end{macrocode}
%    \end{macro}
%    \begin{macro}{\KV@setcurrentvalue}
%    Here the value part is parsed. Package \xpackage{keyval}'s
%    \cs{KV@@sp@def} helps in removing spaces at the begin and
%    end of the value.
%    \begin{macrocode}
\def\KVO@setcurrentvalue#1=#2\@nil{%
  \KV@@sp@def\CurrentOptionValue{#2}%
}
%    \end{macrocode}
%    \end{macro}
%
% \subsection{\plainTeX}
%
%    Disable \LaTeX\ stuff.
%    \begin{macrocode}
\begingroup\expandafter\expandafter\expandafter\endgroup
\expandafter\ifx\csname documentclass\endcsname\relax
  \def\ProcessKeyvalOptions{%
    \@ifstar{}\@gobble
  }%
\fi
%    \end{macrocode}
%
%    \begin{macrocode}
\KVO@AtEnd%
%</package>
%    \end{macrocode}
%
% \subsection{Package \xpackage{kvoptions-patch}}
%
%    \begin{macrocode}
%<*patch>
\NeedsTeXFormat{LaTeX2e}
\begingroup\catcode61\catcode48\catcode32=10\relax%
  \catcode13=5 % ^^M
  \endlinechar=13 %
  \catcode123=1 % {
  \catcode125=2 % }
  \catcode64=11 % @
  \def\x{\endgroup
    \expandafter\edef\csname KVO@AtEnd\endcsname{%
      \endlinechar=\the\endlinechar\relax
      \catcode13=\the\catcode13\relax
      \catcode32=\the\catcode32\relax
      \catcode35=\the\catcode35\relax
      \catcode61=\the\catcode61\relax
      \catcode64=\the\catcode64\relax
      \catcode123=\the\catcode123\relax
      \catcode125=\the\catcode125\relax
    }%
  }%
\x\catcode61\catcode48\catcode32=10\relax%
\catcode13=5 % ^^M
\endlinechar=13 %
\catcode35=6 % #
\catcode64=11 % @
\catcode123=1 % {
\catcode125=2 % }
\def\TMP@EnsureCode#1#2{%
  \edef\KVO@AtEnd{%
    \KVO@AtEnd
    \catcode#1=\the\catcode#1\relax
  }%
  \catcode#1=#2\relax
}
\TMP@EnsureCode{39}{12}% '
\TMP@EnsureCode{40}{12}% (
\TMP@EnsureCode{41}{12}% )
\TMP@EnsureCode{43}{12}% +
\TMP@EnsureCode{44}{12}% ,
\TMP@EnsureCode{45}{12}% -
\TMP@EnsureCode{46}{12}% .
\TMP@EnsureCode{47}{12}% /
\TMP@EnsureCode{58}{12}% :
\TMP@EnsureCode{60}{12}% <
\TMP@EnsureCode{62}{12}% >
\TMP@EnsureCode{91}{12}% [
\TMP@EnsureCode{93}{12}% ]
\TMP@EnsureCode{96}{12}% `
\TMP@EnsureCode{124}{12}% |
\edef\KVO@AtEnd{\KVO@AtEnd\noexpand\endinput}
\ProvidesPackage{kvoptions-patch}%
  [2016/05/16 v3.12 LaTeX patch for keyval options (HO)]%
%    \end{macrocode}
%
%    Check for \eTeX.
%    \begin{macrocode}
\begingroup\expandafter\expandafter\expandafter\endgroup
\expandafter\ifx\csname eTeXversion\endcsname\relax
  \PackageWarningNoLine{kvoptions-patch}{%
    Package loading is aborted, because e-TeX is missing%
  }%
  \expandafter\KVO@AtEnd
\fi%
%    \end{macrocode}
%
%    Package \xpackage{etexcmds} for \cs{etex@unexpanded}.
%    \begin{macrocode}
\RequirePackage{etexcmds}[2007/09/09]
\ifetex@unexpanded
\else
  \PackageError{kvoptions-patch}{%
    Could not find eTeX's \string\unexpanded.\MessageBreak
    Try adding \string\RequirePackage\string{etexcmds\string} %
    before \string\documentclass%
  }\@ehd
  \expandafter\KVO@AtEnd
\fi%
%    \end{macrocode}
%
%    Check for package \xpackage{xkvltxp}.
%    \begin{macrocode}
\@ifpackageloaded{xkvltxp}{%
  \PackageWarningNoLine{kvoptions}{%
    Option `patch' cannot be used together with\MessageBreak
    package `xkvltxp' that is already loaded.\MessageBreak
    Therefore package loading is aborted%
  }%
  \KVO@AtEnd
}{}%
%    \end{macrocode}
%
%    \begin{macrocode}
\def\@if@ptions#1#2#3{%
  \begingroup
    \KVO@normalize\KVO@temp{#3}%
    \edef\x{\endgroup
      \noexpand\@if@pti@ns{%
        \detokenize\expandafter\expandafter\expandafter{%
          \csname opt@#2.#1\endcsname
        }%
      }{%
        \detokenize\expandafter{\KVO@temp}%
      }%
    }%
  \x
}
%    \end{macrocode}
%
%    \begin{macrocode}
\def\@pass@ptions#1#2#3{%
  \KVO@normalize\KVO@temp{#2}%
  \@ifundefined{opt@#3.#1}{%
    \expandafter\gdef\csname opt@#3.#1%
          \expandafter\endcsname\expandafter{%
      \KVO@temp
    }%
  }{%
    \expandafter\gdef\csname opt@#3.#1%
          \expandafter\expandafter\expandafter\endcsname
          \expandafter\expandafter\expandafter{%
      \csname opt@#3.#1\expandafter\endcsname\expandafter,\KVO@temp
    }%
  }%
}
%    \end{macrocode}
%
%    \begin{macrocode}
\def\ProcessOptions{%
  \let\ds@\@empty
  \@ifundefined{opt@\@currname.\@currext}{%
    \let\@curroptions\@empty
  }{%
    \expandafter\expandafter\expandafter\def
    \expandafter\expandafter\expandafter\@curroptions
    \expandafter\expandafter\expandafter{%
      \csname opt@\@currname.\@currext\endcsname
    }%
  }%
  \@ifstar\KVO@xprocess@ptions\KVO@process@ptions
}
%    \end{macrocode}
%
%    \begin{macrocode}
\def\KVO@process@ptions{%
  \@for\CurrentOption:=\@declaredoptions\do{%
    \ifx\CurrentOption\@empty
    \else
      \begingroup
        \ifx\@currext\@clsextension
          \toks@{}%
        \else
          \KVO@GetClassOptionsList
          \toks@\expandafter{\KVO@classoptionslist,}%
        \fi
        \toks\tw@\expandafter{\@curroptions}%
        \edef\x{\endgroup
          \noexpand\in@{,\CurrentOption,}{,\the\toks@\the\toks\tw@,}%
        }%
      \x
      \ifin@
        \KVO@use@ption
        \expandafter\let\csname ds@\CurrentOption\endcsname\@empty
      \fi
    \fi
  }%
  \KVO@process@pti@ns
}
%    \end{macrocode}
%
%    \begin{macrocode}
\def\KVO@xprocess@ptions{%
  \ifx\@currext\@clsextension
  \else
    \KVO@GetClassOptionsList
    \@for\CurrentOption:=\KVO@classoptionslist\do{%
      \ifx\CurrentOption\@empty
      \else
        \KVO@in@\CurrentOption\@declaredoptions
        \ifin@
          \KVO@use@ption
          \expandafter\let\csname ds@\CurrentOption\endcsname\@empty
        \fi
      \fi
    }%
  \fi
  \KVO@process@pti@ns
}
%    \end{macrocode}
%
%    \begin{macrocode}
\def\KVO@in@#1#2{%
  \in@false
  \begingroup
    \@for\x:=#2\do{%
      \ifx\x#1\relax
        \in@true
      \fi
    }%
    \edef\x{\endgroup
      \ifin@
        \noexpand\in@true
      \fi
    }%
  \x
}
%    \end{macrocode}
%
%    \begin{macrocode}
\def\KVO@process@pti@ns{%
  \@for\CurrentOption:=\@curroptions\do{%
    \@ifundefined{ds@\KVO@SanitizedCurrentOption}{%
      \KVO@use@ption
      \default@ds
    }%
    \KVO@use@ption
  }%
  \@for\CurrentOption:=\@declaredoptions\do{%
    \expandafter\let\csname ds@\CurrentOption\endcsname\relax
  }%
  \let\CurrentOption\@empty
  \let\@fileswith@pti@ns\@@fileswith@pti@ns
  \AtEndOfPackage{\let\@unprocessedoptions\relax}%
}
%    \end{macrocode}
%
%    \begin{macrocode}
\def\KVO@use@ption{%
  \begingroup
    \edef\x{\endgroup
      \noexpand\@removeelement{%
        \detokenize\expandafter{\CurrentOption}%
      }{%
        \detokenize\expandafter{\@unusedoptionlist}%
      }%
    }%
  \x\@unusedoptionlist
  \csname ds@\KVO@SanitizedCurrentOption\endcsname
}
%    \end{macrocode}
%
%    \begin{macrocode}
\def\OptionNotUsed{%
  \ifx\@currext\@clsextension
    \xdef\@unusedoptionlist{%
      \ifx\@unusedoptionlist\@empty
      \else
        \detokenize\expandafter{\@unusedoptionlist,}%
      \fi
      \detokenize\expandafter{\CurrentOption}%
    }%
  \fi
}
%    \end{macrocode}
%
%    Variant of \cs{ExecuteOptions} that better protects \cs{CurrentOption}.
%    \begin{macrocode}
\def\CurrentOption@SaveLevel{0}
\def\ExecuteOptions{%
  \expandafter\KVO@ExecuteOptions
      \csname CurrentOption@\CurrentOption@SaveLevel\endcsname
}
\def\KVO@ExecuteOptions#1#2{%
  \let#1\CurrentOption
  \edef\CurrentOption@SaveLevel{%
    \the\numexpr\CurrentOption@SaveLevel+1%
  }%
  \@for\CurrentOption:=#2\do{%
    \csname ds@\CurrentOption\endcsname
  }%
  \edef\CurrentOption@SaveLevel{%
    \the\numexpr\CurrentOption@SaveLevel-1%
  }%
  \let\CurrentOption#1%
}
%    \end{macrocode}
%
%    \begin{macrocode}
\def\KVO@fileswith@pti@ns#1[#2]#3[#4]{%
  \ifx#1\@clsextension
    \ifx\@classoptionslist\relax
      \KVO@normalize\KVO@temp{#2}%
      \expandafter\gdef\expandafter\@classoptionslist\expandafter{%
        \KVO@temp
      }%
      \def\reserved@a{%
        \KVO@onefilewithoptions{#3}[{#2}][{#4}]#1%
        \@documentclasshook
      }%
    \else
      \def\reserved@a{%
        \KVO@onefilewithoptions{#3}[{#2}][{#4}]#1%
      }%
    \fi
  \else
    \begingroup
      \let\KVO@temp\relax
      \let\KVO@onefilewithoptions\relax
      \let\@pkgextension\relax
      \def\reserved@b##1,{%
        \ifx\@nil##1\relax
        \else
          \ifx\relax##1\relax
          \else
            \KVO@onefilewithoptions{##1}[{\KVO@temp}][{#4}]%
            \@pkgextension
          \fi
          \expandafter\reserved@b
        \fi
      }%
      \edef\reserved@a{\zap@space#3 \@empty}%
      \edef\reserved@a{\expandafter\reserved@b\reserved@a,\@nil,}%
      \toks@{#2}%
      \def\KVO@temp{\the\toks@}%
    \edef\reserved@a{\endgroup \reserved@a}%
  \fi
  \reserved@a
}
%    \end{macrocode}
%
%    \begin{macrocode}
\def\KVO@onefilewithoptions#1[#2][#3]#4{%
  \@pushfilename
  \xdef\@currname{#1}%
  \global\let\@currext#4%
  \expandafter\let\csname\@currname.\@currext-h@@k\endcsname\@empty
  \let\CurrentOption\@empty
  \@reset@ptions
  \makeatletter
  \def\reserved@a{%
    \@ifl@aded\@currext{#1}{%
      \@if@ptions\@currext{#1}{#2}{%
      }{%
        \begingroup
          \@ifundefined{opt@#1.\@currext}{%
            \def\x{}%
          }{%
            \edef\x{%
              \expandafter\expandafter\expandafter\strip@prefix
              \expandafter\meaning\csname opt@#1.\@currext\endcsname
            }%
          }%
          \def\y{#2}%
          \edef\y{\expandafter\strip@prefix\meaning\y}%
          \@latex@error{Option clash for \@cls@pkg\space #1}{%
            The package #1 has already been loaded %
            with options:\MessageBreak
            \space\space[\x]\MessageBreak
            There has now been an attempt to load it %
             with options\MessageBreak
            \space\space[\y]\MessageBreak
            Adding the global options:\MessageBreak
            \space\space
                 \x,\y\MessageBreak
            to your \noexpand\documentclass declaration may fix this.%
            \MessageBreak
            Try typing \space <return> \space to proceed.%
          }%
        \endgroup
      }%
    }{%
      \@pass@ptions\@currext{#2}{#1}%
      \global\expandafter
      \let\csname ver@\@currname.\@currext\endcsname\@empty
      \InputIfFileExists
        {\@currname.\@currext}%
        {}%
        {\@missingfileerror\@currname\@currext}%
      \let\@unprocessedoptions\@@unprocessedoptions
      \csname\@currname.\@currext-h@@k\endcsname
      \expandafter\let\csname\@currname.\@currext-h@@k\endcsname
              \@undefined
      \@unprocessedoptions
    }%
    \@ifl@ter\@currext{#1}{#3}{%
    }{%
      \@latex@warning@no@line{%
        You have requested,\on@line, %
        version\MessageBreak
          #3' of \@cls@pkg\space #1,\MessageBreak
        but only version\MessageBreak
         `\csname ver@#1.\@currext\endcsname'\MessageBreak
        is available%
      }%
    }%
    \ifx\@currext\@clsextension\let\LoadClass\@twoloadclasserror\fi
    \@popfilename
    \@reset@ptions
  }%
  \reserved@a
}
%    \end{macrocode}
%
%    \begin{macrocode}
\def\@unknownoptionerror{%
  \@latex@error{%
    Unknown option `\KVO@SanitizedCurrentOption' %
    for \@cls@pkg\space`\@currname'%
  }{%
    The option `\KVO@SanitizedCurrentOption' was not declared in %
    \@cls@pkg\space`\@currname', perhaps you\MessageBreak
    misspelled its name. %
    Try typing \space <return> %
    \space to proceed.%
  }%
}
%    \end{macrocode}
%
%    \begin{macrocode}
\def\@@unprocessedoptions{%
  \ifx\@currext\@pkgextension
    \@ifundefined{opt@\@currname.\@currext}{%
      \let\@curroptions\@empty
    }{%
      \expandafter\let\expandafter\@curroptions
          \csname opt@\@currname.\@currext\endcsname
    }%
    \@for\CurrentOption:=\@curroptions\do{%
        \ifx\CurrentOption\@empty\else\@unknownoptionerror\fi
    }%
  \fi
}
%    \end{macrocode}
%
%    \begin{macrocode}
\def\KVO@SanitizedCurrentOption{%
  \expandafter\strip@prefix\meaning\CurrentOption
}
%    \end{macrocode}
%
%    Normalize option list.
%    \begin{macrocode}
\def\KVO@normalize#1#2{%
  \let\KVO@result\@empty
  \KVO@splitcomma#2,\@nil
  \let#1\KVO@result
}
\def\KVO@splitcomma#1,#2\@nil{%
  \KVO@ifempty{#1}{}{%
    \KVO@checkkv#1=\@nil
  }%
  \KVO@ifempty{#2}{}{\KVO@splitcomma#2\@nil}%
}
\def\KVO@ifempty#1{%
  \expandafter\ifx\expandafter\\\detokenize{#1}\\%
    \expandafter\@firstoftwo
  \else
    \expandafter\@secondoftwo
  \fi
}
\def\KVO@checkkv#1=#2\@nil{%
  \KVO@ifempty{#2}{%
    % option without value
    \edef\KVO@x{\zap@space#1 \@empty}%
    \ifx\KVO@x\@empty
      % ignore empty option
    \else
      % append to list
      \edef\KVO@result{%
        \etex@unexpanded\expandafter{\KVO@result},\KVO@x
      }%
    \fi
  }{%
    % #1: "key", #2: "value="
    % add key part
    \edef\KVO@result{%
      \etex@unexpanded\expandafter{\KVO@result},%
      \zap@space#1 \@empty
    }%
    \futurelet\@let@token\KVO@checkfirsttok#2 \@nil| = \@nil|\KVO@nil
  }%
}
\def\KVO@checkfirsttok{%
  \ifx\@let@token\bgroup
    % no space at start
    \expandafter\KVO@removelastspace\expandafter=%
    % "<value><spaceopt>= \@nil"
  \else
    \expandafter\KVO@checkfirstA
  \fi
}
\def\KVO@checkfirstA#1 #2\@nil{%
  \KVO@ifempty{#2}{%
    \KVO@removelastspace=#1 \@nil
  }{%
    \KVO@ifempty{#1}{%
      \KVO@removelastspace=#2\@nil
    }{%
      \KVO@removelastspace=#1 #2\@nil
    }%
  }%
}
\def\KVO@removelastspace#1 = \@nil|#2\KVO@nil{%
  \KVO@ifempty{#2}{%
    \edef\KVO@result{%
      \etex@unexpanded\expandafter{\KVO@result}%
      \etex@unexpanded\expandafter{\KVO@removegarbage#1\KVO@nil}%
    }%
  }{%
    \edef\KVO@result{%
      \etex@unexpanded\expandafter{\KVO@result}%
      \etex@unexpanded{#1}%
    }%
  }%
}
\def\KVO@removegarbage#1= \@nil#2\KVO@nil{#1}%
%    \end{macrocode}
%
%    Arguments |#1| and |#2| are macros.
%    \begin{macrocode}
\def\KVO@removeelement#1#2{%
  \begingroup
    \toks@={}%
    \@for\x:=#2\do{%
      \ifx\x\@empty
      \else
        \ifx\x#1\relax
        \else
          \edef\t{\the\toks@}%
          \ifx\t\@empty
          \else
            \toks@\expandafter{\the\toks@,}%
          \fi
          \toks@\expandafter{\the\expandafter\toks@\x}%
        \fi
      \fi
    }%
    \edef\x{\endgroup
      \def\noexpand#2{\the\toks@}%
    }%
  \x
}
%    \end{macrocode}
%
%
%    \begin{macrocode}
\let\@@fileswith@pti@ns\KVO@fileswith@pti@ns
\ifx\@fileswith@pti@ns\@badrequireerror
\else
  \let\@fileswith@pti@ns\KVO@fileswith@pti@ns
\fi
%    \end{macrocode}
%
%    \begin{macro}{\KVO@Patch}
%    \begin{macrocode}
\let\KVO@Patch=Y
%    \end{macrocode}
%    \end{macro}
%
%    \begin{macrocode}
\KVO@AtEnd%
%</patch>
%    \end{macrocode}
%
% \section{Test}
%
% \subsection{Preface for standard catcode check}
%
%    \begin{macrocode}
%<*test1>
\input miniltx.tex\relax
%</test1>
%    \end{macrocode}
%
% \subsection{Catcode checks for loading}
%
%    \begin{macrocode}
%<*test1>
%    \end{macrocode}
%    \begin{macrocode}
\catcode`\{=1 %
\catcode`\}=2 %
\catcode`\#=6 %
\catcode`\@=11 %
\expandafter\ifx\csname count@\endcsname\relax
  \countdef\count@=255 %
\fi
\expandafter\ifx\csname @gobble\endcsname\relax
  \long\def\@gobble#1{}%
\fi
\expandafter\ifx\csname @firstofone\endcsname\relax
  \long\def\@firstofone#1{#1}%
\fi
\expandafter\ifx\csname loop\endcsname\relax
  \expandafter\@firstofone
\else
  \expandafter\@gobble
\fi
{%
  \def\loop#1\repeat{%
    \def\body{#1}%
    \iterate
  }%
  \def\iterate{%
    \body
      \let\next\iterate
    \else
      \let\next\relax
    \fi
    \next
  }%
  \let\repeat=\fi
}%
\def\RestoreCatcodes{}
\count@=0 %
\loop
  \edef\RestoreCatcodes{%
    \RestoreCatcodes
    \catcode\the\count@=\the\catcode\count@\relax
  }%
\ifnum\count@<255 %
  \advance\count@ 1 %
\repeat

\def\RangeCatcodeInvalid#1#2{%
  \count@=#1\relax
  \loop
    \catcode\count@=15 %
  \ifnum\count@<#2\relax
    \advance\count@ 1 %
  \repeat
}
\def\RangeCatcodeCheck#1#2#3{%
  \count@=#1\relax
  \loop
    \ifnum#3=\catcode\count@
    \else
      \errmessage{%
        Character \the\count@\space
        with wrong catcode \the\catcode\count@\space
        instead of \number#3%
      }%
    \fi
  \ifnum\count@<#2\relax
    \advance\count@ 1 %
  \repeat
}
\def\space{ }
\expandafter\ifx\csname LoadCommand\endcsname\relax
  \def\LoadCommand{\input kvoptions.sty\relax}%
\fi
\def\Test{%
  \RangeCatcodeInvalid{0}{47}%
  \RangeCatcodeInvalid{58}{64}%
  \RangeCatcodeInvalid{91}{96}%
  \RangeCatcodeInvalid{123}{255}%
  \catcode`\@=12 %
  \catcode`\\=0 %
  \catcode`\%=14 %
  \LoadCommand
  \RangeCatcodeCheck{0}{36}{15}%
  \RangeCatcodeCheck{37}{37}{14}%
  \RangeCatcodeCheck{38}{47}{15}%
  \RangeCatcodeCheck{48}{57}{12}%
  \RangeCatcodeCheck{58}{63}{15}%
  \RangeCatcodeCheck{64}{64}{12}%
  \RangeCatcodeCheck{65}{90}{11}%
  \RangeCatcodeCheck{91}{91}{15}%
  \RangeCatcodeCheck{92}{92}{0}%
  \RangeCatcodeCheck{93}{96}{15}%
  \RangeCatcodeCheck{97}{122}{11}%
  \RangeCatcodeCheck{123}{255}{15}%
  \RestoreCatcodes
}
\Test
\csname @@end\endcsname
\end
%    \end{macrocode}
%    \begin{macrocode}
%</test1>
%    \end{macrocode}
%
%    \begin{macrocode}
%<*test2>
\NeedsTeXFormat{LaTeX2e}
\makeatletter
\catcode`\@=11 %
\def\RestoreCatcodes{}
\count@=0 %
\loop
  \edef\RestoreCatcodes{%
    \RestoreCatcodes
    \catcode\the\count@=\the\catcode\count@\relax
  }%
\ifnum\count@<255 %
  \advance\count@\@ne
\repeat

\def\RangeCatcodeInvalid#1#2{%
  \count@=#1\relax
  \loop
    \catcode\count@=15 %
  \ifnum\count@<#2\relax
    \advance\count@\@ne
  \repeat
}
\def\Test#1{%
  \RangeCatcodeInvalid{0}{47}%
  \RangeCatcodeInvalid{58}{64}%
  \RangeCatcodeInvalid{91}{96}%
  \RangeCatcodeInvalid{123}{255}%
  \catcode`\@=12 %
  \catcode`\\=0 %
  \catcode`\{=1 %
  \catcode`\}=2 %
  \catcode`\#=6 %
  \catcode`\[=12 %
  \catcode`\]=12 %
  \catcode`\%=14 %
  \catcode`\ =10 %
  \catcode13=5 %
  #1\relax
  \RestoreCatcodes
}
\Test{\RequirePackage{kvoptions-patch}}%
\Test{\RequirePackage{kvoptions}}%
\csname @@end\endcsname
%</test2>
%    \end{macrocode}
%
%    \begin{macrocode}
%<*test3>
\NeedsTeXFormat{LaTeX2e}
\makeatletter
\RequirePackage{kvoptions}[2016/05/16]
\def\msg#{\immediate\write16}
\define@key{testfamily}{testkey}{%
  \msg{[testfamily/testkey/#1]}%
}
\define@key{testfamily}{testdefaultkey}[testdefault]{%
  \msg{[testfamily/testdefaultkey/#1]}%
}
\AddToKeyvalOption{testfamily}{testkey}{%
  \msg{[addition/#1]}%
}
\AddToKeyvalOption{testfamily}{testdefaultkey}{%
  \msg{[addition/#1]}%
}
\setkeys{testfamily}{%
  testkey=testA,%
  testdefaultkey=testB,%
  testdefaultkey,%
}
\SetupKeyvalOptions{%
  family=testfamily%
}
\AddToKeyvalOption*{testkey}{%
  \msg{[star addition/#1]}%
}
\AddToKeyvalOption*{testdefaultkey}{%
  \msg{[star addition/#1]}%
}
\setkeys{testfamily}{%
  testkey=testA,%
  testdefaultkey=testB,%
  testdefaultkey,%
}
\@@end
%</test3>
%    \end{macrocode}
%
%    \begin{macrocode}
%<*test4pkg>
\NeedsTeXFormat{LaTeX2e}
\ProvidesPackage{kvoptions-test4}[2016/05/16 package for testing]
\RequirePackage{kvoptions}[2016/05/16]
\SetupKeyvalOptions{%
  family=FOO,%
  prefix=foo,%
  setkeys=\kvsetkeys,%
}
\DeclareStringOption{str}
\define@key{FOO}{set}{%
  \setkeys{BAR}{strbar={#1}}%
}
\define@key{BAR}{strbar}{%
  \def\foostr{[BAR:#1]}%
}
\ProcessKeyvalOptions*
%</test4pkg>
%    \end{macrocode}
%    \begin{macrocode}
%<*test4>
\NeedsTeXFormat{LaTeX2e}
\ProvidesFile{kvoptions-test4.tex}[2016/05/16 test file]
\RequirePackage[%
  str=A,set=B,str=C,%
]{kvoptions-test4}[2016/05/16]
\def\TestExpected{C}
\ifx\foostr\TestExpected
  \typeout{* Test ok.}%
\else
  \typeout{* Result: [\foostr]}%
  \typeout{* Expected: [\TestExpected]}%
  \errmessage{Test failed!}%
\fi
\csname @@end\endcsname\end
%</test4>
%    \end{macrocode}
%
% \section{Installation}
%
% \subsection{Download}
%
% \paragraph{Package.} This package is available on
% CTAN\footnote{\CTANpkg{kvoptions}}:
% \begin{description}
% \item[\CTAN{macros/latex/contrib/oberdiek/kvoptions.dtx}] The source file.
% \item[\CTAN{macros/latex/contrib/oberdiek/kvoptions.pdf}] Documentation.
% \end{description}
%
%
% \paragraph{Bundle.} All the packages of the bundle `oberdiek'
% are also available in a TDS compliant ZIP archive. There
% the packages are already unpacked and the documentation files
% are generated. The files and directories obey the TDS standard.
% \begin{description}
% \item[\CTANinstall{install/macros/latex/contrib/oberdiek.tds.zip}]
% \end{description}
% \emph{TDS} refers to the standard ``A Directory Structure
% for \TeX\ Files'' (\CTAN{tds/tds.pdf}). Directories
% with \xfile{texmf} in their name are usually organized this way.
%
% \subsection{Bundle installation}
%
% \paragraph{Unpacking.} Unpack the \xfile{oberdiek.tds.zip} in the
% TDS tree (also known as \xfile{texmf} tree) of your choice.
% Example (linux):
% \begin{quote}
%   |unzip oberdiek.tds.zip -d ~/texmf|
% \end{quote}
%
% \paragraph{Script installation.}
% Check the directory \xfile{TDS:scripts/oberdiek/} for
% scripts that need further installation steps.
%
% \subsection{Package installation}
%
% \paragraph{Unpacking.} The \xfile{.dtx} file is a self-extracting
% \docstrip\ archive. The files are extracted by running the
% \xfile{.dtx} through \plainTeX:
% \begin{quote}
%   \verb|tex kvoptions.dtx|
% \end{quote}
%
% \paragraph{TDS.} Now the different files must be moved into
% the different directories in your installation TDS tree
% (also known as \xfile{texmf} tree):
% \begin{quote}
% \def\t{^^A
% \begin{tabular}{@{}>{\ttfamily}l@{ $\rightarrow$ }>{\ttfamily}l@{}}
%   kvoptions.sty & tex/latex/oberdiek/kvoptions.sty\\
%   kvoptions-patch.sty & tex/latex/oberdiek/kvoptions-patch.sty\\
%   kvoptions.pdf & doc/latex/oberdiek/kvoptions.pdf\\
%   example-mycolorsetup.sty & doc/latex/oberdiek/example-mycolorsetup.sty\\
%   test/kvoptions-test1.tex & doc/latex/oberdiek/test/kvoptions-test1.tex\\
%   test/kvoptions-test2.tex & doc/latex/oberdiek/test/kvoptions-test2.tex\\
%   test/kvoptions-test3.tex & doc/latex/oberdiek/test/kvoptions-test3.tex\\
%   test/kvoptions-test4.tex & doc/latex/oberdiek/test/kvoptions-test4.tex\\
%   test/kvoptions-test4.sty & doc/latex/oberdiek/test/kvoptions-test4.sty\\
%   kvoptions.dtx & source/latex/oberdiek/kvoptions.dtx\\
% \end{tabular}^^A
% }^^A
% \sbox0{\t}^^A
% \ifdim\wd0>\linewidth
%   \begingroup
%     \advance\linewidth by\leftmargin
%     \advance\linewidth by\rightmargin
%   \edef\x{\endgroup
%     \def\noexpand\lw{\the\linewidth}^^A
%   }\x
%   \def\lwbox{^^A
%     \leavevmode
%     \hbox to \linewidth{^^A
%       \kern-\leftmargin\relax
%       \hss
%       \usebox0
%       \hss
%       \kern-\rightmargin\relax
%     }^^A
%   }^^A
%   \ifdim\wd0>\lw
%     \sbox0{\small\t}^^A
%     \ifdim\wd0>\linewidth
%       \ifdim\wd0>\lw
%         \sbox0{\footnotesize\t}^^A
%         \ifdim\wd0>\linewidth
%           \ifdim\wd0>\lw
%             \sbox0{\scriptsize\t}^^A
%             \ifdim\wd0>\linewidth
%               \ifdim\wd0>\lw
%                 \sbox0{\tiny\t}^^A
%                 \ifdim\wd0>\linewidth
%                   \lwbox
%                 \else
%                   \usebox0
%                 \fi
%               \else
%                 \lwbox
%               \fi
%             \else
%               \usebox0
%             \fi
%           \else
%             \lwbox
%           \fi
%         \else
%           \usebox0
%         \fi
%       \else
%         \lwbox
%       \fi
%     \else
%       \usebox0
%     \fi
%   \else
%     \lwbox
%   \fi
% \else
%   \usebox0
% \fi
% \end{quote}
% If you have a \xfile{docstrip.cfg} that configures and enables \docstrip's
% TDS installing feature, then some files can already be in the right
% place, see the documentation of \docstrip.
%
% \subsection{Refresh file name databases}
%
% If your \TeX~distribution
% (\TeX\,Live, \mikTeX, \dots) relies on file name databases, you must refresh
% these. For example, \TeX\,Live\ users run \verb|texhash| or
% \verb|mktexlsr|.
%
% \subsection{Some details for the interested}
%
% \paragraph{Unpacking with \LaTeX.}
% The \xfile{.dtx} chooses its action depending on the format:
% \begin{description}
% \item[\plainTeX:] Run \docstrip\ and extract the files.
% \item[\LaTeX:] Generate the documentation.
% \end{description}
% If you insist on using \LaTeX\ for \docstrip\ (really,
% \docstrip\ does not need \LaTeX), then inform the autodetect routine
% about your intention:
% \begin{quote}
%   \verb|latex \let\install=y\input{kvoptions.dtx}|
% \end{quote}
% Do not forget to quote the argument according to the demands
% of your shell.
%
% \paragraph{Generating the documentation.}
% You can use both the \xfile{.dtx} or the \xfile{.drv} to generate
% the documentation. The process can be configured by the
% configuration file \xfile{ltxdoc.cfg}. For instance, put this
% line into this file, if you want to have A4 as paper format:
% \begin{quote}
%   \verb|\PassOptionsToClass{a4paper}{article}|
% \end{quote}
% An example follows how to generate the
% documentation with pdf\LaTeX:
% \begin{quote}
%\begin{verbatim}
%pdflatex kvoptions.dtx
%makeindex -s gind.ist kvoptions.idx
%pdflatex kvoptions.dtx
%makeindex -s gind.ist kvoptions.idx
%pdflatex kvoptions.dtx
%\end{verbatim}
% \end{quote}
%
% \begin{thebibliography}{99}
% \bibitem{tb94wright}
%   A guide to key-value methods, Joseph Wright, second draft for
%   \href{https://www.tug.org/tugboat}{TUGBoat}, 2009-03-17.
%   \url{https://www.texdev.net/uploads/2009/03/keyval.pdf}
% \bibpackage{ifthen}{David Carlisle}{2001/05/26}
%   {\CTANpkg{ifthen}}
% \bibpackage{helvet}{Sebastian Rahtz, Walter Schmidt}{2004/01/26}
%   {\CTANpkg{psfonts}}
% \bibpackage{hyperref}{Sebastian Rahtz, Heiko Oberdiek}{2006/02/12}
%   {\CTANpkg{hyperref}}
% \bibpackage{keyval}{David Carlisle}{1999/03/16}
%   {\CTANpkg{keyval}}
% \bibpackage{multicol}{Frank Mittelbach}{2004/02/14}
%   {\CTANpkg{multicol}}
% \bibpackage{tabularx}{David Carlisle}{1999/01/07}
%   {\CTANpkg{tabularx}}
% \bibpackage{tracefnt}{Frank Mittelbach, Rainer Sch\"opf}{1997/05/29}
%   {\CTANpkg{latex-base}}
% \bibpackage{xkeyval}{Hendri Adriaens}{2005/05/07}
%   {\CTANpkg{xkeyval}}
% \bibitem{clsguide}
%   The \LaTeX3 Project, \textit{\LaTeXe\ for class and package writers},
%   2003/12/09.
%   \CTANpkg{clsguide}
%
% \end{thebibliography}
%
% \begin{History}
%   \begin{Version}{0000/00/00 v0.0}
%   \item
%     Probably David Carlisle's code in \xpackage{hyperref}
%     was the start.
%   \end{Version}
%   \begin{Version}{2004/02/22 v1.0}
%   \item
%     The first version was never published. It also
%     has offered a patch to get rid of \LaTeX's option
%     expansion.
%   \end{Version}
%   \begin{Version}{2006/02/16 v2.0}
%   \item
%     Now the package is redesigned with an easier
%     user interface.
%   \item
%     \cs{ProcessKeyvalOptions} remains the central service, inherited
%     from \xpackage{hyperref}'s \cs{ProcessOptionsWithKV}.
%     Now the use inside classes is also supported.
%   \item
%     Provides help macros for boolean and simple string options.
%   \item
%     Fixes for the patch of \LaTeX{}. The patch is only enabled,
%     if the user requests it.
%   \end{Version}
%   \begin{Version}{2006/02/20 v2.1}
%   \item
%     Unused option list is sanitized to prevent problems
%     with other packages that uses own processing methods
%     for key value options. Disadvantage: the unused global
%     option detection is weakened.
%   \item
%     New option type by \cs{DeclareVoidOption} for options without
%     value.
%   \item
%     Default rule by \cs{DeclareDefaultOption}.
%   \item
%     Dynamic options: \cs{DisableKeyvalOption}.
%   \end{Version}
%   \begin{Version}{2006/06/01 v2.2}
%   \item
%     Fixes for option \xoption{patch}.
%   \end{Version}
%   \begin{Version}{2006/08/17 v2.3}
%   \item
%     \cs{DeclareBooleanOption} renamed to \cs{DeclareBoolOption}
%     to avoid a name clash with package \cs{ifoption}.
%   \end{Version}
%   \begin{Version}{2006/08/22 v2.4}
%   \item
%     Option \xoption{patch}: \cs{ExecuteOptions} does not
%     change the meaning of macro \cs{CurrentOption} at all.
%   \end{Version}
%   \begin{Version}{2007/04/11 v2.5}
%   \item
%     Line ends sanitized.
%   \end{Version}
%   \begin{Version}{2007/05/06 v2.6}
%   \item
%     Uses package \xpackage{etexcmds}.
%   \end{Version}
%   \begin{Version}{2007/06/11 v2.7}
%   \item
%     The patch part fixes LaTeX bug latex/3965.
%   \end{Version}
%   \begin{Version}{2007/10/02 v2.8}
%   \item
%     Compatibility for \plainTeX\ added.
%   \item
%     Typos in documentation fixed (Axel Sommerfeldt).
%   \end{Version}
%   \begin{Version}{2007/10/11 v2.9}
%   \item
%     Bug fix for option \xoption{patch}.
%   \end{Version}
%   \begin{Version}{2007/10/18 v3.0}
%   \item
%     New package \xpackage{kvoptions-patch}.
%   \end{Version}
%   \begin{Version}{2009/04/10 v3.1}
%   \item
%     Space by line end removed in definition of internal macro.
%   \end{Version}
%   \begin{Version}{2009/07/17 v3.2}
%   \item
%     \cs{ProcessLocalKeyvalOptions} added.
%   \item
%     \cs{DisableKeyvalOption} with the \texttt{action=ignore} option
%     fixed (Joseph Wright).
%   \end{Version}
%   \begin{Version}{2009/07/21 v3.3}
%   \item
%     \cs{DeclareLocalOption}, \cs{DeclareLocalOptions} added.
%   \end{Version}
%   \begin{Version}{2009/08/13 v3.4}
%   \item
%     Documentation addition: recommendation for Joseph Wright's
%     review article.
%   \item
%     Documentation addition: local/global options.
%   \end{Version}
%   \begin{Version}{2009/12/04 v3.5}
%   \item
%     \cs{AddToKeyvalOption} added.
%   \end{Version}
%   \begin{Version}{2009/12/08 v3.6}
%   \item
%     Fix: If a default handler is configured, it is now also
%     called for classes.
%   \end{Version}
%   \begin{Version}{2010/02/22 v3.7}
%   \item
%     Missing space in error message added.
%   \end{Version}
%   \begin{Version}{2010/07/23 v3.8}
%   \item
%     Documenation for package \xpackage{kvoptions-patch} improved.
%     No code changes.
%   \end{Version}
%   \begin{Version}{2010/12/02 v3.9}
%   \item
%     Key \texttt{setkeys} added for \cs{SetupKeyvalOptions}.
%   \end{Version}
%   \begin{Version}{2010/12/23 v3.10}
%   \item
%     \cs{DeclareVoidOption} also parses the second parameter as
%     \hologo{TeX} argument to improve compatibility with
%     \cs{DeclareOption}.
%   \end{Version}
%   \begin{Version}{2011/06/30 v3.11}
%   \item
%     Fix because of design bug in package \xpackage{xkeyval} that
%     removes global options with equal signs.
%   \end{Version}
%   \begin{Version}{2016/05/16 v3.12}
%   \item
%     Documentation updates.
%   \end{Version}
% \end{History}
%
% \PrintIndex
%
% \Finale
\endinput

%        (quote the arguments according to the demands of your shell)
%
% Documentation:
%    (a) If kvoptions.drv is present:
%           latex kvoptions.drv
%    (b) Without kvoptions.drv:
%           latex kvoptions.dtx; ...
%    The class ltxdoc loads the configuration file ltxdoc.cfg
%    if available. Here you can specify further options, e.g.
%    use A4 as paper format:
%       \PassOptionsToClass{a4paper}{article}
%
%    Programm calls to get the documentation (example):
%       pdflatex kvoptions.dtx
%       makeindex -s gind.ist kvoptions.idx
%       pdflatex kvoptions.dtx
%       makeindex -s gind.ist kvoptions.idx
%       pdflatex kvoptions.dtx
%
% Installation:
%    TDS:tex/latex/oberdiek/kvoptions.sty
%    TDS:tex/latex/oberdiek/kvoptions-patch.sty
%    TDS:doc/latex/oberdiek/kvoptions.pdf
%    TDS:doc/latex/oberdiek/example-mycolorsetup.sty
%    TDS:doc/latex/oberdiek/test/kvoptions-test1.tex
%    TDS:doc/latex/oberdiek/test/kvoptions-test2.tex
%    TDS:doc/latex/oberdiek/test/kvoptions-test3.tex
%    TDS:doc/latex/oberdiek/test/kvoptions-test4.tex
%    TDS:doc/latex/oberdiek/test/kvoptions-test4.sty
%    TDS:source/latex/oberdiek/kvoptions.dtx
%
%<*ignore>
\begingroup
  \catcode123=1 %
  \catcode125=2 %
  \def\x{LaTeX2e}%
\expandafter\endgroup
\ifcase 0\ifx\install y1\fi\expandafter
         \ifx\csname processbatchFile\endcsname\relax\else1\fi
         \ifx\fmtname\x\else 1\fi\relax
\else\csname fi\endcsname
%</ignore>
%<*install>
\input docstrip.tex
\Msg{************************************************************************}
\Msg{* Installation}
\Msg{* Package: kvoptions 2016/05/16 v3.12 Key value format for package options (HO)}
\Msg{************************************************************************}

\keepsilent
\askforoverwritefalse

\let\MetaPrefix\relax
\preamble

This is a generated file.

Project: kvoptions
Version: 2016/05/16 v3.12

Copyright (C)
   2004, 2006, 2007, 2009-2011 Heiko Oberdiek
   2016-2019 Oberdiek Package Support Group

This work may be distributed and/or modified under the
conditions of the LaTeX Project Public License, either
version 1.3c of this license or (at your option) any later
version. This version of this license is in
   https://www.latex-project.org/lppl/lppl-1-3c.txt
and the latest version of this license is in
   https://www.latex-project.org/lppl.txt
and version 1.3 or later is part of all distributions of
LaTeX version 2005/12/01 or later.

This work has the LPPL maintenance status "maintained".

The Current Maintainers of this work are
Heiko Oberdiek and the Oberdiek Package Support Group
https://github.com/ho-tex/oberdiek/issues


This work consists of the main source file kvoptions.dtx
and the derived files
   kvoptions.sty, kvoptions.pdf, kvoptions.ins, kvoptions.drv,
   kvoptions-patch.sty, example-mycolorsetup.sty,
   kvoptions-test1.tex, kvoptions-test2.tex,
   kvoptions-test3.tex, kvoptions-test4.tex,
   kvoptions-test4.sty.

\endpreamble
\let\MetaPrefix\DoubleperCent

\generate{%
  \file{kvoptions.ins}{\from{kvoptions.dtx}{install}}%
  \file{kvoptions.drv}{\from{kvoptions.dtx}{driver}}%
  \usedir{tex/latex/oberdiek}%
  \file{kvoptions.sty}{\from{kvoptions.dtx}{package}}%
  \file{kvoptions-patch.sty}{\from{kvoptions.dtx}{patch}}%
  \usedir{doc/latex/oberdiek}%
  \file{example-mycolorsetup.sty}{\from{kvoptions.dtx}{example}}%
%  \usedir{doc/latex/oberdiek/test}%
%  \file{kvoptions-test1.tex}{\from{kvoptions.dtx}{test1}}%
%  \file{kvoptions-test2.tex}{\from{kvoptions.dtx}{test2}}%
%  \file{kvoptions-test3.tex}{\from{kvoptions.dtx}{test3}}%
%  \file{kvoptions-test4.tex}{\from{kvoptions.dtx}{test4}}%
%  \file{kvoptions-test4.sty}{\from{kvoptions.dtx}{test4pkg}}%
  \nopreamble
  \nopostamble
%  \usedir{source/latex/oberdiek/catalogue}%
%  \file{kvoptions.xml}{\from{kvoptions.dtx}{catalogue}}%
}

\catcode32=13\relax% active space
\let =\space%
\Msg{************************************************************************}
\Msg{*}
\Msg{* To finish the installation you have to move the following}
\Msg{* files into a directory searched by TeX:}
\Msg{*}
\Msg{*     kvoptions.sty, kvoptions-patch.sty}
\Msg{*}
\Msg{* To produce the documentation run the file `kvoptions.drv'}
\Msg{* through LaTeX.}
\Msg{*}
\Msg{* Happy TeXing!}
\Msg{*}
\Msg{************************************************************************}

\endbatchfile
%</install>
%<*ignore>
\fi
%</ignore>
%<*driver>
\NeedsTeXFormat{LaTeX2e}
\ProvidesFile{kvoptions.drv}%
  [2016/05/16 v3.12 Key value format for package options (HO)]%
\documentclass{ltxdoc}
\usepackage{holtxdoc}[2011/11/22]
\begin{document}
  \DocInput{kvoptions.dtx}%
\end{document}
%</driver>
% \fi
%
%
% \CharacterTable
%  {Upper-case    \A\B\C\D\E\F\G\H\I\J\K\L\M\N\O\P\Q\R\S\T\U\V\W\X\Y\Z
%   Lower-case    \a\b\c\d\e\f\g\h\i\j\k\l\m\n\o\p\q\r\s\t\u\v\w\x\y\z
%   Digits        \0\1\2\3\4\5\6\7\8\9
%   Exclamation   \!     Double quote  \"     Hash (number) \#
%   Dollar        \$     Percent       \%     Ampersand     \&
%   Acute accent  \'     Left paren    \(     Right paren   \)
%   Asterisk      \*     Plus          \+     Comma         \,
%   Minus         \-     Point         \.     Solidus       \/
%   Colon         \:     Semicolon     \;     Less than     \<
%   Equals        \=     Greater than  \>     Question mark \?
%   Commercial at \@     Left bracket  \[     Backslash     \\
%   Right bracket \]     Circumflex    \^     Underscore    \_
%   Grave accent  \`     Left brace    \{     Vertical bar  \|
%   Right brace   \}     Tilde         \~}
%
% \GetFileInfo{kvoptions.drv}
%
% \title{The \xpackage{kvoptions} package}
% \date{2016/05/16 v3.12}
% \author{Heiko Oberdiek\thanks
% {Please report any issues at \url{https://github.com/ho-tex/oberdiek/issues}}}
%
% \maketitle
%
% \begin{abstract}
% This package is intended for package authors who want to
% use options in key value format for their package options.
% \end{abstract}
%
% \tableofcontents
%
% \def\M#1{\texttt{\{}\meta{#1}\texttt{\}}}
% \def\O#1{\texttt{[}\meta{#1}\texttt{]}}
%
% \section{Introduction}
%
% First I want to recommend the very good review article
% ``A guide to key-value methods'' by Joseph Wright \cite{tb94wright}.
% It introduces the different key-value packages and compares them.
%
% \subsection{The beginning}
%
% This package \xpackage{kvoptions} addresses class or package writers
% that want to allow their users to specify options
% as key value pairs, e.g.:
% \begin{quote}
% |\documentclass[verbose=false,name=me]{myclass}|\\
% |\usepackage[format=print]{mylayout}|
% \end{quote}
% Prominent example is package \xpackage{hyperref},
% probably the first package that offers this service.
% It's \cs{ProcessOptionsWithKV} is often copied und
% used in other packages, e.g. package \xpackage{helvet}
% that uses this interface for its option \xoption{scaled}.
%
% However copying code is not the most modern software
% development technique. And \xpackage{hyperref}'s
% code for \cs{ProcessOptionsWithKV} was changed
% to fix bugs. The version used in other packages
% depends on the time of copying and the awareness
% of \xpackage{hyperref}'s changes.
% Now the code is sourced out into this package and
% available for other package or class writers.
%
% \subsection{Overview}
%
% Package \xpackage{kvoptions} connects package
% \xpackage{\emph{k}ey\emph{v}al}
% with \LaTeX's package and class \emph{options}:
% \begin{center}
% \def\T#1{\bfseries #1}
% \renewcommand{\arraystretch}{1.5}
% \begin{tabular}{l|l|l}
%   \T{Package \xpackage{keyval}}
%   & \T{Package \xpackage{kvoptions}}
%   & \T{\LaTeX\ kernel}
%   \\
% \hline
%     \cs{define@key}
%   &
%     \renewcommand*{\arraystretch}{1}^^A
%     \begin{tabular}[t]{@{}l@{}}
%       \cs{DeclareVoidOption}\\
%       \cs{DeclareStringOption}\\
%       \cs{DeclareBoolOption}\\
%       \cs{DeclareComplementaryOption}\\
%       \cs{DisableKeyvalOption}
%     \end{tabular}
%   & \cs{DeclareOption}
%   \\
% \hline
%   & \cs{DeclareDefaultOption}
%   & \cs{DeclareOption*}
%   \\
% \hline
%   & \cs{ProcessKeyvalOptions}
%   & \cs{ProcessOptions*}
%   \\
% \hline
%   & Option \xoption{patch}
%   &
%     \renewcommand*{\arraystretch}{1}^^A
%     \begin{tabular}[t]{@{}l@{}}
%       Class/package\\
%       option system
%     \end{tabular}
%   \\
% \hline
%   & \cs{SetupKeyvalOptions} &
%   \\
% \end{tabular}
% \end{center}
%
% \hypersetup{bookmarksopenlevel=3}
% \section{Usage}
%
% \subsection{Process options}
%
% \subsubsection{\cs{ProcessKeyvalOptions}}
%
% \begin{declcs}{ProcessKeyvalOptions}\,\M{family}\\
%   \cs{ProcessKeyvalOptions} \*
% \end{declcs}
% This command evaluates the global or local options of
% the package that are defined with \xpackage{keyval}'s
% interface within the family \meta{family}. It acts the
% same way as \LaTeX's \cs{ProcessOptions*}. In a package
% unknown global options are ignored, in a class they
% are added to the unknown option list. The known global
% options and all local options are passed to \xpackage{keyval}'s
% \cs{setkeys} command for executing the options.
% Unknown options are reported to the user by an error.
%
% If the family name happens to be the same as the name
% of the package or class where \cs{ProcessKeyvalOptions}
% is used or the family name has previously been setup by
% \cs{SetupKeyvalOptions}, then \cs{ProcessKeyvalOptions}
% knows the family name already and you can use the star form
% without mandatory argument.
%
% \subsubsection{\cs{ProcessLocalKeyvalOptions}}
%
% \begin{declcs}{ProcessLocalKeyvalOptions}\,\M{family}\\
%   \cs{ProcessLocalKeyvalOptions} \*
% \end{declcs}
% This macro has the same syntax and works similar as
% \cs{ProcessKeyvalOptions}. However it ignores
% global options and only processes the local package options.
% Therefore it only can be used inside a package.
% An error is thrown, if it is used inside a class.
%
% Neither of the following macros are necessary for
% \cs{ProcessKeyvalOptions{}}. They just help the package/class
% author in common tasks.
%
% \subsubsection{\cs{SetupKeyvalOptions}}\label{setup}
%
% \begin{declcs}{SetupKeyvalOptions}\,\texttt{\{}\\
%   \hspace{1.5em}|family|\,|=|\,\meta{family}|,|\\
%   \hspace{1.5em}|prefix|\,|=|\,\meta{prefix}\\
%   \hspace{1.5em}|setkeys|\,|=|\,\meta{setkeys command}\\
%   \texttt{\}}
% \end{declcs}
% This command allows to configure the default assumptions
% that are based on the current package or class name.
% \LaTeX\ remembers this name in \cs{@currname}. The syntax
% description of the default looks a little weird, therefor
% an example is given for a package or class named |foobar|.
% \begin{quote}
% \begin{tabular}{@{}l|ll|l@{}}
%   Key & Default & (example) & Used by\\
% \hline
%   \rule{0pt}{1.2\ht\csname @arstrutbox\endcsname}^^A
%   |family| & \meta{\cs{@currname}} & (|foobar|)
%       & \cs{ProcessKeyvalOptions*} \\
%     &&& \cs{DeclareBoolOption} \\
%     &&& \cs{DeclareStringOption} \\
% \hline
%   \rule{0pt}{1.2\ht\csname @arstrutbox\endcsname}^^A
%   |prefix| & \meta{\cs{@currname}}|@| & (|foobar@|)
%       & \cs{DeclareBoolOption} \\
%     &&& \cs{DeclareStringOption} \\
%     &&& \cs{DeclareVoidOption} \\
% \hline
%   \rule{0pt}{1.2\ht\csname @arstrutbox\endcsname}^^A
%   |setkeys| & \cs{setkeys} & (\cs{kvsetkeys})
%       & \cs{ProcessKeyvalOptions} \\
%     &&& \cs{ProcessLocalKeyvalOptions}
% \end{tabular}
% \end{quote}
% Key |setkeys| was added in version 3.9. The original
% \cs{setkeys} of package \xpackage{keyval} is not reentrant.
% If an option is processed by this \cs{setkeys}, then
% the option should not call \cs{setkeys} again with a
% different family. Otherwise the next options of the
% first \cs{setkeys} call are processed with the wrong
% family. With key |setkeys| the macro \cs{kvsetkeys}
% can be set that does not have the problem of the
% original \cs{setkeys} of package {keyval}.
%
% Probably \cs{setkeys} of package \xpackage{xkeyval}
% is safe in this respect. But I haven't made a full
% analysis. At least it does not have the problem
% of the original \cs{setkeys}.
%
% \subsection{Option declarations}
%
% The options for \cs{ProcessKeyvalOptions} are defined
% by \xpackage{keyval}'s \cs{define@key}.
% Common purposes of such keys are boolean switches, they
% enable or disable something. Or they store a name or some kind of
% string in a macro. The following commands help the user. He
% declares what he wants and \xpackage{kvoptions} take care of
% the key definition, resource allocation and initialization.
%
% In order to avoid name clashes of macro names, internal
% commands are prefixed. Both the prefix and the family
% name for the defined keys can be configured by
% \cs{SetupKeyvalOptions}.
%
% \subsubsection{\cs{DeclareStringOption}}
%
% \begin{declcs}{DeclareStringOption}\,\O{init}\,\M{key}\,\O{default}
% \end{declcs}
% A macro is created that remembers the value of the key \meta{key}.
% The name of the macro consists of the option name \meta{key} that
% is prefixed by the prefix (see \ref{setup}).
% The initial contents of the
% macro can be given by the first optional argument \meta{init}.
% The default is empty.
%
% The the option \meta{key} is defined. The option code
% just stores its value in the macro. If the optional argument
% at the end of \cs{DeclareStringOption} is given, then option
% \meta{key} is defined with the default \meta{default}.
%
% Example for a package with the following two lines:
% \begin{quote}
% |\ProvidesPackage{foobar}|\\
% |\DeclareStringOption[me]{name}|
% \end{quote}
% Then \cs{DeclareStringOption} defines the macro with content |me|,
% note \LaTeX\ complains if the name of the macro already exists:
% \begin{quote}
% |\newcommand*{\foobar@name}{me}|
% \end{quote}
% The option definition is similar to:
% \begin{quote}
% |\define@key{foobar}{name}{%|\\
% |  \renewcommand*{\foobar@name}{#1}%|\\
% |}|
% \end{quote}
%
% \subsubsection{\cs{DeclareBoolOption}}
%
% \begin{declcs}{DeclareBoolOption}\,\O{init}\,\M{key}
% \end{declcs}
% A boolean switch is generated, initialized by value \meta{init}
% and the corresponding key \meta{key} is defined. If the
% initialization value is not given, |false| is used as default.
%
% The internal actions of \cs{DeclareBoolOption} are shown below.
% The example is given for a package author who has the following
% two lines in his package/class:
% \begin{quote}
% |\ProvidesPackage{foobar}|\\
% |\DeclareBoolOption{verbose}|
% \end{quote}
% First a new switch is created:
% \begin{quote}
%   |\newif\iffoobar@verbose|
% \end{quote}
% and initialized:
% \begin{quote}
%   |\foobar@verbosefalse|
% \end{quote}
% Finally the key is defined:
% \begin{quote}
%   |\define@key{foobar}{verbose}[true]{|\dots|}|
% \end{quote}
% The option code configures the boolean option in the following way:
% If the author specifies |true| or |false| then the switch is
% turned on or off respectivly. Also the option can be given without
% explicit value. Then the switch is enabled.
% Other values are reported as errors.
%
% Now the switch is ready to use in the package/class, e.g.:
% \begin{quote}
%   |\iffoobar@verbose|\\
%   \qquad\textit{\% print verbose message}\\
%   |\else|\\
%   \qquad\textit{\% be quiet}\\
%   |\fi|
% \end{quote}
% Users of package \cs{ifthen} can use the switch as
% boolean:
% \begin{quote}
%   |\boolean{foobar@verbose}|
% \end{quote}
%
% \subsubsection{\cs{DeclareComplementaryOption}}
%
% \begin{declcs}{DeclareComplementaryOption}\,\M{key}\,\M{parent}
% \end{declcs}
% Sometimes contrasting names are used to characterize the
% two states of a boolean switch, for example \xoption{draft}
% vs.\@ \xoption{final}. Both options behave like boolean
% options but they do not need two different switches, they
% should share one. \cs{DeclareComplementaryOption} allows this.
% The option \meta{key} shares the switch of option \meta{parent}.
% Example:
% \begin{quote}
%   |\DeclareBoolOption{draft}|\\
%   |\DeclareComplementaryOption{final}{draft}|
% \end{quote}
% Then |final| sets the switch of \xoption{draft} to |false|, and
% |final=false| enables the \xoption{draft} switch.
%
% \subsubsection{\cs{DeclareVoidOption}}
%
% \begin{declcs}{DeclareVoidOption}\,\M{key}\,\M{code}
% \end{declcs}
% \cs{ProcessKeyvalOptions} can be extended to recognize
% options that are declared in traditional way by \cs{DeclareOption}.
% But in case of the error that the user specifies a value, then
% this option would not recognized as key value option because
% of \cs{DeclareOption} and not detected as traditional option
% because of the value part. The user would get an unknown option
% error, difficult to understand.
%
% \cs{DeclareVoidOption} solves
% this problem. It defines the option \meta{key} as key value option.
% If the user specifies a value, a warning is given and the
% value is ignored.
%
% The code part \meta{code} is stored in a macro.
% The name of the macro consists of the option name \meta{key} that
% is prefixed by the prefix (see \ref{setup}).
% If the option is set, the macro will be executed. During the
% execution \cs{CurrentOption} is available with the current key name.
%
% \subsubsection{\cs{DeclareDefaultOption}}
%
% \begin{declcs}{DeclareDefaultOption}\,\M{code}
% \end{declcs}
% \begin{sloppypar}
% This command does not define a specific key, it is the equivalent
% to \LaTeX's \cs{DeclareOption*}. It allows the specification
% of a default action \meta{code} that is invoked if an unknown option
% is found.
% While \meta{code} is called,
% macro \cs{CurrentOption} contains the current option string.
% In addition \cs{CurrentOptionValue} contains the value part
% if the option string is parsable as key value pair, otherwise
% it is \cs{relax}. \cs{CurrentOptionKey} contains the key of
% the key value pair, or the whole option string, if it misses
% the equal sign.
% \end{sloppypar}
%
% Inside packages typical default actions are to pass unknown
% options to another package. Or an error message can be thrown
% by \cs{@unknownoptionerror}. This is the original error
% message that \LaTeX\ gives for unkown package options.
% This error message is easier to understand for the user
% as the error message from package \xpackage{keyval} that
% is given otherwise.
%
% A Class ignores unknown options and puts them on the
% unused option list. Let \LaTeX\ do the job and just
% call \cs{OptionNotUsed}. Or the options can be passed to
% another class that is later loaded.
%
% \subsubsection{Local options}
%
% \begin{declcs}{DeclareLocalOption} \M{option}\\
%   \cs{DeclareLocalOptions} \M{option list}
% \end{declcs}
% Both macros mark package options as local options. That means
% that they are ignored by \cs{ProcessKeyvalOptions} if they are given
% as global options. \cs{DeclareLocalOptions} takes one option,
% \cs{DeclareLocalOptions} expects a comma separated list of options.
%
% \subsubsection{Dynamic options}
%
% Options of \LaTeX's package/class system are cleared
% in \cs{ProcessOptions}. They modify the static model
% of a package. For example, depending on option \xoption{bookmarks}
% package \xpackage{hyperref} loads differently.
%
% Options, however,
% defined by \xpackage{keyval}'s \cs{define@key} remain
% defined, if the options are processed by \cs{setkeys}.
% Therefore these options can also be used to model
% the dynamic behaviour of a package. For example, in
% \xpackage{hyperref} the link colors can be changed
% everywhere until the end in |\end{document}|.
%
% However package \xpackage{color} that adds color support
% is necessary and it
% cannot be loaded after |\begin{document}|. Option \xoption{colorlinks}
% that loads \xpackage{color} should be active until |\begin{document}|
% and die in some way if it is too late for loading packages.
% With \cs{DisableKeyvalOption} the package/class author can
% specify and configure the death of an option and controls
% the life period of the option.
%
% \subsubsection{\cs{DisableKeyvalOption}}
%
% \begin{declcs}{DisableKeyvalOption}\,\O{options}\,\M{family}\,\M{key}
%   \\[0.5ex]
%   \meta{options}:\\
%   \hspace{1.5em}^^A
%   \begin{tabular}[t]{@{}l@{ \texttt{=} }l@{\qquad}>{default: }l@{}}
%     |action| & |undef|, |warning|, |error|, or |ignore|
%     & |undef|
%     \\
%     \multicolumn{1}{@{}l}{|global| or |local|} & & |global|
%     \\
%     |package| or |class| & \meta{name}
%     \\
%   \end{tabular}^^A
% \end{declcs}
% \cs{DisableKeyvalOption} can be called to mark the end when
% the option \meta{key} is no longer useful. The behaviour of
% an option after its death can be configured by \xoption{action}:
% \begin{description}
% \item[\texttt{undef}:] The option will be undefined,
%   If it is called, \cs{setkeys} reports an error because
%   of unknown key.
% \item[\texttt{error} or \texttt{warning}:] Use of the
%  option will cause an error or warning message.
%  Also these actions require that exclusivly either the
%  package or class name is given in options \xoption{package}
%  or \xoption{class}.
% \item[\texttt{ignore}:] The use of the option will silently
%  be ignored.
% \end{description}
% The option's death can be limited to the end of the current
% group, if option \xoption{local} is given. Default is
% \xoption{global}.
%
% The package/class author can wish the end of the option
% already during the package loading, then he will have
% static behaviour. In case of dynamic options \cs{DisableKeyvalOption}
% can be executed everywhere, also outside the package. Therefore
% the family name and the package/class name is usually unknown for
% \cs{DisableKeyvalOption}. Therefore the argument for the family
% name is mandatory and for some actions the package/class name
% must be provided.
%
% Usually a macro would configure the option death, Example:
%\begin{quote}
%\begin{verbatim}
%\ProvidesPackage{foobar}
%\DeclareBoolOption{color}
%\DeclareStringOption[red]{emphcolor}
%\ProcessKeyvalOptions*
%
%\newcommand*{\foobar@DisableOption}[2]{%
%  \DisableKeyvalueOption[
%    action={#1},
%    package=foobar
%  ]{foobar}{#2}%
%}
%
%\iffoobar@color
%  \RequirePackage{color}
%  \renewcommand*{\emph}[1]{\textcolor{\foobar@emphcolor}{#1}}
%\else
%  % Option emphcolor is not wrong, if we have color support.
%  % otherwise the option has no effect, but we don't want to
%  % remove it. Therefore action 'ignore' is the best choice:
%  \foobar@DisableOption{ignore}{emphcolor}
%\fi
%% No we don't need the option 'color'.
%\foobar@DisableOption{warning}{color}
%
%% With color support option 'emphcolor' will dynamically
%% change the color of \emph statements.
%\end{verbatim}
%\end{quote}
%
% \subsubsection{\cs{AddToKeyvalOption}}
%
% \begin{declcs}{AddToKeyvalOption}\,\M{family}\,\M{key}\,\M{code}\\
%   \cs{AddToKeyvalOption} \*\,\M{key}\,\M{code}
% \end{declcs}
% The code for an existing key \meta{key} of family \meta{family} is
% extended by code \meta{code}. In the starred form the current
% family setting is used, see \cs{ProcessKeyvalOptions*}.
%
% \subsection{Global vs. local options}
%
% Options that are given for \cs{documentclass} are called
% global options. They are known to the class and all packages.
% A package may make use of a global option and marks it as
% used. The advantage for the user is the freedom to specify
% options both in the \cs{documentclass} or \cs{usepackage}
% commands.
%
% However global options are shared with the class options
% and options of all other packages. Thus there can be the
% same option with different semantics for different
% packages and classes. As example, package \xpackage{bookmark}
% knows option \xoption{open} that specifies whether the bookmarks
% are opened or closed initially. It's values are |true| or |false|.
% Since KOMA-Script version 3.00 the KOMA classes also introduces
% option \xoption{open} with values |right| and |any| and a
% complete different meaning.
%
% Such conflicts can be resolved by marking all or part of options
% as local by \cs{DeclareLocalOption} or \cs{DeclareLocalOptions}.
% Then the packages ignores global occurences of these options.
% Package \xpackage{kvoptions} provides two methods:
% \begin{itemize}
% \item \cs{ProcessLocalKeyvalOptions} automatically uses all options
%   as local options. It ignores all global options.
% \item \cs{DeclareLocalOption} or \cs{DeclareLocalOptions} marks
%   options as local options. \cs{ProcessKeyvalOptions} will then
%   ignore global occurences for these local options.
% \end{itemize}
% Since version 1.5 package \xpackage{bookmark} uses the latter
% method. It checks global and local option places for driver options
% and limits all other options as local options. Thus the class
% option \xoption{open} of KOMA-Script is not misread as
% option for package \xpackage{bookmark}.
%
% \subsection{Summary of internal macros}
%
% The \cs{Declare}\texttt{\dots Option} commands define
% macros, additionally to the macros generated by the key definition.
% These macros can be used by the package/class author.
% The name of the macros starts with the prefix \meta{prefix}
% that can be configured by \cs{SetupKeyvalOptions}.
% \begin{center}
% \def\prefix{\cs{}\meta{prefix}\meta{key}}^^A
% \def\Hline{^^A
%   \hline
%   \rule{0pt}{1.2\ht\csname @arstrutbox\endcsname}^^A
% }^^A
% \begin{tabular}{@{}l|>{\ttfamily}l|l@{}}
%   Declare \meta{key} & \textrm{Defined macro} & Description\\
% \Hline
%   \cs{DeclareStringOption}
%    & \prefix & holds the string\\
% \Hline
%   \cs{DeclareBoolOption}
%     & \cs{if}\meta{prefix}\meta{key}& boolean switch\\
%   &\prefix false& disable switch\\
%   &\prefix true& enable switch\\
% \Hline
%   \cs{DeclareComplementaryOption}
%     & \prefix false& enable parent switch\\
%     & \prefix true& disable parent switch\\
% \Hline
%   \cs{DeclareVoidOption}
%   & \prefix & holds the action
% \end{tabular}
% \end{center}
%
% \subsection{\plainTeX}
%
% Package \xpackage{keyval} is also usable in \plainTeX\ with
% the help of file \xfile{miniltx.tex}. Some features of this
% package \xpackage{kvoptions} might also be useful for \plainTeX.
% If \LaTeX\ is not found, \cs{ProcessKeyvalOptions} and option
% \xoption{patch} are disabled. Before using the option declaration
% commands \cs{Declare...Option}, \cs{SetupKeyvalOptions} must be
% used.
%
% \hypersetup{bookmarksopenlevel=1}
%
% \section{Example}
%
% The following example defined a package that serves some
% private color management. A boolean option \xoption{print} enables
% print mode without colors. An option \xoption{emph} redefines
% \cs{emph} to print in the given color. And the driver
% can be specified by option \xoption{driver}.
%
%    \begin{macrocode}
%<*example>
    % Package identification
    % -----------------------
\NeedsTeXFormat{LaTeX2e}
\ProvidesPackage{example-mycolorsetup}[2016/05/16 Managing my colors]

\RequirePackage{ifpdf}
\RequirePackage{kvoptions}

    % Option declarations
    % -------------------

\SetupKeyvalOptions{
  family=MCS,
  prefix=MCS@
}
    % Use a shorter family name and prefix

    % Option print
\DeclareBoolOption{print}
    % is the same as
    % \DeclareBoolOption[false]{print}

    % Option driver
\ifpdf
  \DeclareStringOption[pdftex]{driver}
\else
  \DeclareStringOption[dvips]{driver}
\fi

    % Alternative interface for driver options
\DeclareVoidOption{dvips}{\SetupDriver}
\DeclareVoidOption{dvipdfm}{\SetupDriver}
\DeclareVoidOption{pdftex}{\SetupDriver}
    % In \SetupDriver we take the current option \CurrentOption
    % and pass it to the driver option.
    % The \expandafter commands expand \CurrentOption at the
    % time, when \SetupDriver is executed and \CurrentOption
    % has the correct meaning.
\newcommand*{\SetupDriver}{%
  \expandafter\@SetupDriver\expandafter{\CurrentOption}%
}
\newcommand*{\@SetupDriver}[1]{%
  \setkeys{MCS}{driver={#1}}%
}

    % Option emph
    % An empty value means, we want to have no color for \emph.
    % If the user specifies option emph without value, the red is used.
\DeclareStringOption{emph}[red]
    % is the same as
    % \DeclareStringOption[]{emph}[red]

    % Default option rule
\DeclareDefaultOption{%
  \ifx\CurrentOptionValue\relax
    \PackageWarningNoLine{\@currname}{%
      Unknown option `\CurrentOption'\MessageBreak
      is passed to package `color'%
    }%
    % Pass the option to package color.
    % Again it is better to expand \CurrentOption.
    \expandafter\PassOptionsToPackage
    \expandafter{\CurrentOption}{color}%
  \else
    % Package color does not take options with values.
    % We provide the standard LaTeX error.
    \@unknownoptionerror
  \fi
}

    % Process options
    % ---------------
\ProcessKeyvalOptions*

    % Implementation depending on option values
    % -----------------------------------------
    % Code for print mode
\ifMCS@print
  \PassOptionsToPackage{monochrome}{color}
    % tells package color to use black and white
\fi

\RequirePackage[\MCS@driver]{color}
    % load package color with the correct driver

    % \emph setup
\ifx\MCS@emph\@empty
    % \@empty is a predefined macro with empty contents.
    % the option value of option emph is empty, thus
    % we do not want a redefinition of \emph.
\else
  \renewcommand*{\emph}[1]{%
    \textcolor{\MCS@emph}{#1}%
  }
\fi
%</example>
%    \end{macrocode}
%
% \section{Package options}
%
% The package \xpackage{kvoptions} knows two package options
% \xoption{patch} and \xoption{debugshow}.
% The options of package \xpackage{kvoptions} are intended
% for authors, not for package/class writers. Inside a package
% it is too late for option \xoption{patch} and \xoption{debugshow}
% enables some messages that are perhaps useful for the debugging
% phase. Also \LaTeX\ is unhappy if a package is loaded
% later again with options that are previously not given.
% Thus package and class authors, stay with |\RequirePackage{kvoptions}|
% without options.
%
% Option \xoption{patch} loads package \xpackage{kvoptions-patch}.
%
% \subsection{Package \xpackage{kvoptions-patch}}
%
% \LaTeX's system of package/class options has some severe limitations
% that especially affects the value part if options are used as
% pair of key and value.
% \begin{itemize}
% \item Spaces are removed, regardless where:
%   \begin{quote}
%     |\documentclass[box=0 0 400 600]{article}|
%   \end{quote}
%   Now each package will see |box=00400600| as global option.
% \item In the previous case also braces would not help:
%   \begin{quote}
%     |\documentclass[box={0 0 400 600}]{article}|
%   \end{quote}
%   The result is an error message:
%   \begin{quote}
%     |! LaTeX Error: Missing \begin{document}.|
%   \end{quote}
%   As local option, however, it works if the package
%   knows about key value options (By using this package, for example).
% \item The requirements on robustness are extremly high.
%   \LaTeX\ expands the option. All that will not work as environment
%   name will break also as option. Even a \cs{relax} will generate
%   an error message:
%   \begin{quote}
%     |! Missing \endcsname inserted.|
%   \end{quote}
%   Of course, \LaTeX\ does not use its protecting mechanisms.
%   On contrary \cs{protect} itself will cause errors.
% \item The options are expanded. But perhaps the package will
%   do that, because it has to setup some things before?
%   Example \xpackage{hyperref}:
%   \begin{quote}
%     |\usepackage[pdfauthor=M\"uller]{hyperref}|
%   \end{quote}
%   Package \xpackage{hyperref} does not see |M\"uller| but
%   its expansion and it does not like it, you get many warnings
%   \begin{quote}
%     |Token not allowed in a PDFDocEncoded string|
%   \end{quote}
%   And the title becomes: |Mu127uller|.
%   Therefore such options must usually be given after package
%   \xpackage{hyperref}
%   is loaded:
%   \begin{quote}
%     |\usepackage{hyperref}|\\
%     |\hypersetup{pdfauthor=Fran\c coise M\"uller}|
%   \end{quote}
%   As package option it will even break with |Fran\c coise|
%   because of the cedilla |\c c|, it is not robust enough.
% \end{itemize}
% For users that do not want with this limitations the package
% offers package \xoption{kvoptions-patch}. It patches \LaTeX's option system
% and tries to teach it also to handle options that are given
% as pairs of key and value and to prevent expansion.
% It can already be used at the very beginning, before \cs{documentclass}:
% \begin{quote}
%   |\RequirePackage{kvoptions-patch}|\\
%   |\documentclass[pdfauthor=Fran\c coise M\"uller]{article}|\\
%   |\usepackage{hyperref}|
% \end{quote}
% The latest time is before the package where you want to use
% problematic values:
% \begin{quote}
%   |\usepackage{kvoptions-patch}|\\
%   |\usepackage[Fran\c coise M\"uller]{hyperref}|
% \end{quote}
% Some remarks:
% \begin{itemize}
% \item The patch requires \eTeX, its \cs{unexpanded} feature
%   is much too nice. It is possible to work around using token
%   registers. But the code becomes longer, slower, more difficult
%   to read and maintain. The package without option \xoption{patch}
%   works and will work without \eTeX.
% \item The code for the patch is quite long, there are many test
%   cases. Thus the probability for bugs is probably not too small.
% \item Since 2008/10/18 v3.0 package \xpackage{kvoptions-patch} is
%   available. Before option \xoption{patch} of package \xpackage{kvoptions}
%   must be used instead. I think, the solution as standalone package
%   \xpackage{kvoptions-patch} is cleaner and avoids option clashes.
% \end{itemize}
%
% \subsection{Option \xoption{debugshow}}
%
% The name of this option follows the convention of packages
% \xpackage{multicol}, \xpackage{tabularx}, and \xpackage{tracefnt}.
% Currently it prints the setting of boolean options, declared
% by \cs{DeclareBoolOption} in the \xfile{.log} file, if that
% boolean option is used.
% You can activate the option by
% \begin{itemize}
% \item |\PassOptionsToPackage{debugshow}{kvoptions}|\\
%       Put this somewhere before package \xpackage{kvoptions} is
%       loaded first, e.g. before \cs{documentclass}.
% \item |\RequirePackage[debugshow]{kvoptions}|\\
%       Before \cs{documentclass} even an author has to use
%       \cs{RequirePackage}. \cs{usepackage} only works after
%       \cs{documentclass}.
% \end{itemize}
% The preferred method is \cs{PassOptionsToPackage}, because
% it does not force the package loading and does not disturb,
% if the package is not loaded later at all.
%
% \section{Limitations}
%
% \subsection{Compatibility}
% \subsubsection{Package \xpackage{kvoptions-patch} vs. package \xpackage{xkvltxp}}
%
% Package \xpackage{xkvltxp} from the \xpackage{xkeyval} project
% has the same goal as package \xpackage{kvoptions-patch}
% and to patch \LaTeX's kernel commands in order to get better support
% for key value options.
% Of course they cannot be used both. The user must decide, which
% method he prefers. Package \xpackage{kvoptions-patch} aborts itself,
% if it detects that \xpackage{xkvltxp} is
% already loaded.
%
% However package \xpackage{xkvltxp} and \xpackage{kvoptions}
% can be used together, example:
% \begin{quote}
% |\usepackage{xkvltxp}|\\
% |\usepackage[...]{foobar} % |\texttt{\itshape foobar using kvoptions}
% \end{quote}
% The other way should work, too.
%
% Package \xpackage{kvoptions-patch} tries to catch more situations
% and to be more robust.
% For example, during the comparison of options it normalizes
% them by removing spaces around |=| and the value. Thus the
% following is not reported as option clash:
% \begin{quote}
%\begin{verbatim}
%\RequirePackage{kvoptions-patch}
%\documentclass{article}
%
%\usepackage[scaled=0.7]{helvet}
%\usepackage[scaled = 0.7]{helvet}
%
%\begin{document}
%\end{document}
%\end{verbatim}
% \end{quote}
%
% \subsection{Limitations}
%
% \subsubsection{Option comparisons}
%
% In some situations \LaTeX\ compares option lists, e.g. option clash check,
% \cs{@ifpackagewith}, or \cs{@ifclasswith}. Apart from catcode and
% sanitizing problems of option \xoption{patch}, there is another problem.
% \LaTeX\ does not know about the type and default values of
% options in key value style. Thus an option clash is reported,
% even if the key value has the same meaning:
% \begin{quote}
% |\usepackage[scaled]{helvet} %| \texttt{\textit{default is }.95}\\
% |\usepackage[.95]{helvet}|\\
% |\usepackage[0.95]{helvet}|
% \end{quote}
%
% \subsubsection{Option list parsing with package \xpackage{kvoptions-patch}}
%
% With package \xpackage{kvoptions-patch} the range of possible values
% in key value specifications is much large, for example
% the comma can be used, if enclosed in curly braces.
%
% Other packages, especially the packages that uses their
% own process option code can be surprised to find tokens
% inside options that they do not expect and errors would
% be the consequence. To avoid errors the options, especially
% the unused option list is sanitized. That means the list
% will only contain tokens with catcode 12 (other) and
% perhaps spaces (catcode 10). This allows a safe parsing
% for other packages. But a comma in the value part is
% no longer protected by curly braces because they have lost
% their special meaning. This is the price for compatibility.
%
% Example:
% \begin{quote}
% |\RequirePackage{kvoptions-patch}|\\
% |\documentclass[a={a,b,c},b]{article}|\\
% |\begin{document}|\\
% |\end{document}|
% \end{quote}
% Result:
% \begin{quote}
% |LaTeX Warning: Unused global option(s):|\\
% |    [a={a,c},b].|
% \end{quote}
%
% \StopEventually{
% }
%
% \section{Implementation}
%
% \subsection{Preamble}
%
%    \begin{macrocode}
%<*package>
%    \end{macrocode}
%
% \paragraph{Reload check and  identification.}
%    Reload check, especially if the package is not used with \LaTeX.
%    \begin{macrocode}
\begingroup\catcode61\catcode48\catcode32=10\relax%
  \catcode13=5 % ^^M
  \endlinechar=13 %
  \catcode35=6 % #
  \catcode39=12 % '
  \catcode44=12 % ,
  \catcode45=12 % -
  \catcode46=12 % .
  \catcode58=12 % :
  \catcode64=11 % @
  \catcode123=1 % {
  \catcode125=2 % }
  \expandafter\let\expandafter\x\csname ver@kvoptions.sty\endcsname
  \ifx\x\relax % plain-TeX, first loading
  \else
    \def\empty{}%
    \ifx\x\empty % LaTeX, first loading,
      % variable is initialized, but \ProvidesPackage not yet seen
    \else
      \expandafter\ifx\csname PackageInfo\endcsname\relax
        \def\x#1#2{%
          \immediate\write-1{Package #1 Info: #2.}%
        }%
      \else
        \def\x#1#2{\PackageInfo{#1}{#2, stopped}}%
      \fi
      \x{kvoptions}{The package is already loaded}%
      \aftergroup\endinput
    \fi
  \fi
\endgroup%
%    \end{macrocode}
%    Package identification:
%    \begin{macrocode}
\begingroup\catcode61\catcode48\catcode32=10\relax%
  \catcode13=5 % ^^M
  \endlinechar=13 %
  \catcode35=6 % #
  \catcode39=12 % '
  \catcode40=12 % (
  \catcode41=12 % )
  \catcode44=12 % ,
  \catcode45=12 % -
  \catcode46=12 % .
  \catcode47=12 % /
  \catcode58=12 % :
  \catcode64=11 % @
  \catcode91=12 % [
  \catcode93=12 % ]
  \catcode123=1 % {
  \catcode125=2 % }
  \expandafter\ifx\csname ProvidesPackage\endcsname\relax
    \def\x#1#2#3[#4]{\endgroup
      \immediate\write-1{Package: #3 #4}%
      \xdef#1{#4}%
    }%
  \else
    \def\x#1#2[#3]{\endgroup
      #2[{#3}]%
      \ifx#1\@undefined
        \xdef#1{#3}%
      \fi
      \ifx#1\relax
        \xdef#1{#3}%
      \fi
    }%
  \fi
\expandafter\x\csname ver@kvoptions.sty\endcsname
\ProvidesPackage{kvoptions}%
  [2016/05/16 v3.12 Key value format for package options (HO)]%
%    \end{macrocode}
%
% \paragraph{Catcodes}
%
%    \begin{macrocode}
\begingroup\catcode61\catcode48\catcode32=10\relax%
  \catcode13=5 % ^^M
  \endlinechar=13 %
  \catcode123=1 % {
  \catcode125=2 % }
  \catcode64=11 % @
  \def\x{\endgroup
    \expandafter\edef\csname KVO@AtEnd\endcsname{%
      \endlinechar=\the\endlinechar\relax
      \catcode13=\the\catcode13\relax
      \catcode32=\the\catcode32\relax
      \catcode35=\the\catcode35\relax
      \catcode61=\the\catcode61\relax
      \catcode64=\the\catcode64\relax
      \catcode123=\the\catcode123\relax
      \catcode125=\the\catcode125\relax
    }%
  }%
\x\catcode61\catcode48\catcode32=10\relax%
\catcode13=5 % ^^M
\endlinechar=13 %
\catcode35=6 % #
\catcode64=11 % @
\catcode123=1 % {
\catcode125=2 % }
\def\TMP@EnsureCode#1#2{%
  \edef\KVO@AtEnd{%
    \KVO@AtEnd
    \catcode#1=\the\catcode#1\relax
  }%
  \catcode#1=#2\relax
}
\TMP@EnsureCode{1}{14}% ^^A (comment)
\TMP@EnsureCode{2}{14}% ^^A (comment)
\TMP@EnsureCode{33}{12}% !
\TMP@EnsureCode{39}{12}% '
\TMP@EnsureCode{40}{12}% (
\TMP@EnsureCode{41}{12}% )
\TMP@EnsureCode{42}{12}% *
\TMP@EnsureCode{44}{12}% ,
\TMP@EnsureCode{45}{12}% -
\TMP@EnsureCode{46}{12}% .
\TMP@EnsureCode{47}{12}% /
\TMP@EnsureCode{58}{12}% :
\TMP@EnsureCode{62}{12}% >
\TMP@EnsureCode{91}{12}% [
\TMP@EnsureCode{93}{12}% ]
\TMP@EnsureCode{94}{7}% ^ (superscript)
\TMP@EnsureCode{96}{12}% `
\edef\KVO@AtEnd{\KVO@AtEnd\noexpand\endinput}
%    \end{macrocode}
%
% \paragraph{External resources.}
%    The package extends the support for key value pairs of
%    package \cs{keyval} to package options. Thus the package
%    needs to be loaded anyway. and we use it for
%    \cs{SetupKeyvalOptions}. AFAIK this does not disturb
%    users of \xpackage{xkeyval}.
%    \begin{macrocode}
\@ifundefined{define@key}{%
  \RequirePackage{keyval}\relax
}{}
%    \end{macrocode}
%
%    Macro \cs{DeclareLocalOptions} parses a comma separated key list
%    and uses \cs{comma@parse} of package \xpackage{kvsetkeys},
%    version 1.3.
%    \begin{macrocode}
\RequirePackage{ltxcmds}[2010/12/02]
\RequirePackage{kvsetkeys}[2007/09/29]
%    \end{macrocode}
%
% \paragraph{Provide macros for \plainTeX.}
%    \begin{macrocode}
\@ifundefined{@x@protect}{%
  \def\@x@protect#1\fi#2#3{%
    \fi\protect#1%
  }%
  \let\@typeset@protect\relax
}{}
\@ifundefined{@currname}{%
  \def\@currname{}%
}{}
\@ifundefined{@currext}{%
  \def\@currext{}%
}{}
%    \end{macrocode}
%
% \paragraph{Options}
%    Option \xoption{debugshow} enables additional lines of
%    code that prints information into the \xfile{.log}
%    file.
%    \begin{macrocode}
\DeclareOption{debugshow}{\catcode\@ne=9 }
%    \end{macrocode}
%
%    \begin{macrocode}
\DeclareOption{patch}{%
  \AtEndOfPackage{%
    \RequirePackage{kvoptions-patch}[2016/05/16]%
  }%
}
%    \end{macrocode}
%
%    Optionen auswerten:
%    \begin{macrocode}
\ProcessOptions\relax
%    \end{macrocode}
%
% \subsection{Option declaration macros}
%
% \subsubsection{\cs{SetupKeyvalOptions}}
%
%    The family for the key value pairs can be setup once
%    and is remembered later.
%    The package name seems a reasonable default for the
%    family key, if it is not set by the package author.
%
%    \begin{macro}{\KVO@family}
%    We cannot store the family setting in one macro,
%    because the package should be usable for many
%    other packages, too. Thus we remember the family
%    setting in a macro, whose name contains the
%    package name with extension, a key in \LaTeX's
%    class/package system.
%    \begin{macrocode}
\define@key{KVO}{family}{%
  \expandafter\edef\csname KVO@family@%
      \@currname.\@currext\endcsname{#1}%
}
\def\KVO@family{%
  \@ifundefined{KVO@family@\@currname.\@currext}{%
    \@currname
  }{%
    \csname KVO@family@\@currname.\@currext\endcsname
  }%
}
%    \end{macrocode}
%    \end{macro}
%
%    \begin{macro}{\KVO@prefix}
%    The value settings of options that are declared by \cs{DeclareBoolOption}
%    and \cs{DeclareStringOption} need to be saved in macros.
%    in the first case this is a switch \cs{if}\meta{prefix}\meta{key},
%    in the latter case a macro \cs{}\meta{prefix}\meta{key}.
%    The prefix can be configured, by \xoption{prefix} that is declared
%    here. The default is the package name with |@| appended.
%    \begin{macrocode}
\define@key{KVO}{prefix}{%
  \expandafter\edef\csname KVO@prefix@%
      \@currname.\@currext\endcsname{#1}%
}
\def\KVO@prefix{%
  \ltx@ifundefined{KVO@prefix@\@currname.\@currext}{%
    \@currname @%
  }{%
    \csname KVO@prefix@\@currname.\@currext\endcsname
  }%
}
%    \end{macrocode}
%    \end{macro}
%
%    \begin{macrocode}
\define@key{KVO}{setkeys}{%
  \expandafter\def\csname KVO@setkeys@%
      \@currname.\@currext\endcsname{#1}%
}
%    \end{macrocode}
%    \begin{macro}{\KVO@setkeys}
%    \begin{macrocode}
\def\KVO@setkeys{%
  \ltx@IfUndefined{KVO@setkeys@\@currname.\@currext}{%
    \setkeys
  }{%
    \csname KVO@setkeys@\@currname.\@currext\endcsname
  }%
}
%    \end{macrocode}
%    \end{macro}
%
%    \begin{macro}{\SetupKeyvalOptions}
%    The argument of \cs{SetupKeyvalOptions} expects a key value
%    list, known keys are \xoption{family} and \xoption{prefix}.
%    \begin{macrocode}
\newcommand*{\SetupKeyvalOptions}{%
  \kvsetkeys{KVO}%
}
%    \end{macrocode}
%    \end{macro}
%
% \subsubsection{\cs{DeclareBoolOption}}
%
%    \begin{macro}{\DeclareBoolOption}
%    Usually options of boolean type can be given by the
%    user without value and this means a setting to \emph{true}.
%    We follow this convention here. Also it simplifies the
%    user interface.
%
%    The switch is created and initialized with \emph{false}.
%    The default setting can be overwritten by the optional
%    argument.
%
%    \LaTeX's \cs{newif} does not check for already defined
%    macros, therefore we add this check here to prevent
%    the user from accidently redefining of \TeX's primitives
%    and other macros.
%    \begin{macrocode}
\newcommand*{\DeclareBoolOption}[2][false]{%
  \KVO@ifdefinable{if\KVO@prefix#2}{%
    \KVO@ifdefinable{\KVO@prefix#2true}{%
      \KVO@ifdefinable{\KVO@prefix#2false}{%
        \csname newif\expandafter\endcsname
        \csname if\KVO@prefix#2\endcsname
        \@ifundefined{\KVO@prefix#2#1}{%
          \PackageWarning{kvoptions}{%
            Initialization of option `#2' failed,\MessageBreak
            cannot set boolean option to `#1',\MessageBreak
            use `true' or `false', now using `false'%
          }%
        }{%
          \csname\KVO@prefix#2#1\endcsname
        }%
        \begingroup
          \edef\x{\endgroup
            \noexpand\define@key{\KVO@family}{#2}[true]{%
              \noexpand\KVO@boolkey{\@currname}%
              \ifx\@currext\@clsextension
                \noexpand\@clsextension
              \else
                \noexpand\@pkgextension
              \fi
              {\KVO@prefix}{#2}{####1}%
            }%
          }%
        \x
      }%
    }%
  }%
}
%    \end{macrocode}
%    \end{macro}
%
%    \begin{macro}{\DeclareComplementaryOption}
%    The first argument is the key name, the second the
%    key that must be a boolean option with the same
%    current family and prefix. A new switch is not
%    created for the new key, we have already a switch.
%    Instead we define switch setting commands to work
%    on the parent switch.
%    \begin{macrocode}
\newcommand*{\DeclareComplementaryOption}[2]{%
  \@ifundefined{if\KVO@prefix#2}{%
    \PackageError{kvoptions}{%
      Cannot generate option code for `#1',\MessageBreak
      parent switch `#2' does not exist%
    }{%
      You are inside %
      \ifx\@currext\@clsextension class\else package\fi\space
      `\@currname.\@currext'.\MessageBreak
      `\KVO@family' is used as familiy %
      for the keyval options.\MessageBreak
      `\KVO@prefix' serves as prefix %
      for internal switch macros.\MessageBreak
      \MessageBreak
      \@ehc
    }%
  }{%
    \KVO@ifdefinable{\KVO@prefix#1true}{%
      \KVO@ifdefinable{\KVO@prefix#1false}{%
        \expandafter\let\csname\KVO@prefix#1false\expandafter\endcsname
          \csname\KVO@prefix#2true\endcsname
        \expandafter\let\csname\KVO@prefix#1true\expandafter\endcsname
          \csname\KVO@prefix#2false\endcsname
%    \end{macrocode}
%    The same code part as in \cs{DeclareBoolOption} can
%    now be used.
%    \begin{macrocode}
        \begingroup
          \edef\x{\endgroup
            \noexpand\define@key{\KVO@family}{#1}[true]{%
              \noexpand\KVO@boolkey{\@currname}%
              \ifx\@currext\@clsextension
                \noexpand\@clsextension
              \else
                \noexpand\@pkgextension
              \fi
              {\KVO@prefix}{#1}{####1}%
            }%
          }%
        \x
      }%
    }%
  }%
}
%    \end{macrocode}
%    \end{macro}
%
%    \begin{macro}{\KVO@ifdefinable}
%    Generate the command token LaTeX's \cs{@ifdefinable} expects.
%    \begin{macrocode}
\def\KVO@ifdefinable#1{%
  \expandafter\@ifdefinable\csname #1\endcsname
}
%    \end{macrocode}
%    \end{macro}
%
%    \begin{macro}{\KVO@boolkey}
%    We check explicitly for |true| and |false| to prevent
%    the user from accidently calling other macros.
%    \begin{quote}
%    \begin{tabular}{@{}ll@{}}
%    |#1|& package/class name\\
%    |#2|& |\@pkgextension|/|\@clsextension|\\
%    |#3|& prefix \\
%    |#4|& key name\\
%    |#5|& new value
%    \end{tabular}
%    \end{quote}
%    \begin{macrocode}
\def\KVO@boolkey#1#2#3#4#5{%
  \edef\KVO@param{#5}%
  \ltx@onelevel@sanitize\KVO@param
  \ifx\KVO@param\KVO@true
    \expandafter\@firstofone
  \else
    \ifx\KVO@param\KVO@false
      \expandafter\expandafter\expandafter\@firstofone
    \else
      \ifx#2\@clsextension
        \expandafter\ClassWarning
      \else
        \expandafter\PackageWarning
      \fi
      {#1}{%
        Value `\KVO@param' is not supported by\MessageBreak
        option `#4'%
      }%
      \expandafter\expandafter\expandafter\@gobble
    \fi
  \fi
  {%
    ^^A\ifx#2\@clsextension
    ^^A  \expandafter\ClassInfo
    ^^A\else
    ^^A  \expandafter\PackageInfo
    ^^A\fi
    ^^A{#1}{[option] #4=\KVO@param}%
    \csname#3#4\KVO@param\endcsname
  }%
}
%    \end{macrocode}
%    \end{macro}
%    \begin{macro}{\KVO@true}
%    \begin{macro}{\KVO@false}
%    The macros \cs{KVO@true} and \cs{KVO@false} are used
%    for string comparisons. After \cs{ltx@onelevel@sanitize}
%    we have only tokens with catcode 12 (other).
%    \begin{macrocode}
\def\KVO@true{true}
\def\KVO@false{false}
\ltx@onelevel@sanitize\KVO@true
\ltx@onelevel@sanitize\KVO@false
%    \end{macrocode}
%    \end{macro}
%    \end{macro}
%
% \subsubsection{\cs{DeclareStringOption}}
%
%    \begin{macro}{\DeclareStringOption}
%    \begin{macrocode}
\newcommand*{\DeclareStringOption}[2][]{%
  \@ifnextchar[{%
    \KVO@DeclareStringOption{#1}{#2}@%
  }{%
    \KVO@DeclareStringOption{#1}{#2}{}[]%
  }%
}
%    \end{macrocode}
%    \end{macro}
%    \begin{macro}{\KVO@DeclareStringOption}
%    \begin{macrocode}
\def\KVO@DeclareStringOption#1#2#3[#4]{%
  \KVO@ifdefinable{\KVO@prefix#2}{%
    \@namedef{\KVO@prefix#2}{#1}%
    \begingroup
      \ifx\\#3\\%
        \toks@{}%
      \else
        \toks@{[{#4}]}%
      \fi
      \edef\x{\endgroup
        \noexpand\define@key{\KVO@family}{#2}\the\toks@{%
          ^^A\begingroup
          ^^A  \toks@{####1}%
          ^^A  \ifx\@currext\@clsextension
          ^^A    \noexpand\ClassInfo
          ^^A  \else
          ^^A    \noexpand\PackageInfo
          ^^A  \fi
          ^^A  {\@currname}{%
          ^^A    [option] #2={\noexpand\the\toks@}%
          ^^A  }%
          ^^A\endgroup
          \noexpand\def
          \expandafter\noexpand\csname\KVO@prefix#2\endcsname{####1}%
        }%
      }%
    \x
  }%
}
%    \end{macrocode}
%    \end{macro}
%
% \subsubsection{\cs{DeclareVoidOption}}
%
%    \begin{macro}{\DeclareVoidOption}
%    \begin{macrocode}
\newcommand*{\DeclareVoidOption}[2]{%
  \begingroup
    \let\next\@gobbletwo
    \KVO@ifdefinable{\KVO@prefix#1}{%
      \let\next\@firstofone
    }%
  \expandafter\endgroup
  \next{%
    \begingroup
      \edef\x{\endgroup
        \noexpand\define@key{\KVO@family}{#1}[\KVO@VOID@]{%
          \noexpand\KVO@voidkey{\@currname}%
          \ifx\@currext\@clsextension
            \noexpand\@clsextension
          \else
            \noexpand\@pkgextension
          \fi
          {#1}%
          {####1}%
          \expandafter\noexpand\csname\KVO@prefix#1\endcsname
        }%
      }%
    \x
    \begingroup
      \toks@{#2}%
    \expandafter\endgroup
    \expandafter\def
    \csname\KVO@prefix#1\expandafter\endcsname
    \expandafter{\the\toks@}%
  }%
}
\def\KVO@VOID@{@VOID@}
%    \end{macrocode}
%    \end{macro}
%    \begin{macro}{\KVO@voidkey}
%    \begin{tabular}{@{}ll@{}}
%    |#1|& package/class name\\
%    |#2|& |\@pkgextension|/|\@clsextension|\\
%    |#3|& key name\\
%    |#4|& default (|@VOID@|)\\
%    |#5|& macro with option code
%    \end{tabular}
%    \begin{macrocode}
\def\KVO@voidkey#1#2#3#4{%
  \def\CurrentOption{#3}%
  \begingroup
    \def\x{#4}%
  \expandafter\endgroup
  \ifx\x\KVO@VOID@
  \else
    \ifx#2\@clsextension
      \expandafter\ClassWarning
    \else
      \expandafter\PackageWarning
    \fi
    {#1}{%
      Unexpected value for option `#3'\MessageBreak
      is ignored%
    }%
  \fi
  ^^A\ifx#2\@clsextension
  ^^A  \expandafter\ClassInfo
  ^^A\else
  ^^A  \expandafter\PackageInfo
  ^^A\fi
  ^^A{#1}{[option] #3}%
}
%    \end{macrocode}
%    \end{macro}
%
% \subsubsection{\cs{DeclareDefaultOption}}
%
%    \begin{macro}{\DeclareDefaultOption}
%    \begin{macrocode}
\newcommand*{\DeclareDefaultOption}{%
  \@namedef{KVO@default@\@currname.\@currext}%
}
%    \end{macrocode}
%    \end{macro}
%
% \subsubsection{\cs{DeclareLocalOptions}}
%
%    \begin{macro}{\DeclareLocalOptions}
%    \begin{macrocode}
\newcommand*{\DeclareLocalOptions}[1]{%
  \comma@parse{#1}\KVO@DeclareLocalOption
}
%    \end{macrocode}
%    \end{macro}
%    \begin{macro}{\KVO@DeclareLocalOption}
%    \begin{macrocode}
\def\KVO@DeclareLocalOption#1{%
  \expandafter\def\csname KVO@local@\KVO@family @#1\endcsname{}%
}
%    \end{macrocode}
%    \end{macro}
%
% \subsection{Dynamic options}
%
% \subsubsection{\cs{DisableKeyvalOption}}
%
%    \begin{macrocode}
\SetupKeyvalOptions{%
  family=KVOdyn,%
  prefix=KVOdyn@%
}
\DeclareBoolOption[true]{global}
\DeclareComplementaryOption{local}{global}
\DeclareStringOption[undef]{action}
\let\KVOdyn@name\relax
\let\KVOdyn@ext\@empty
\define@key{KVOdyn}{class}{%
  \def\KVOdyn@name{#1}%
  \let\KVOdyn@ext\@clsextension
}
\define@key{KVOdyn}{package}{%
  \def\KVOdyn@name{#1}%
  \let\KVOdyn@ext\@pkgextension
}
\newcommand*{\DisableKeyvalOption}[3][]{%
  \begingroup
    \kvsetkeys{KVOdyn}{#1}%
    \def\x{\endgroup}%
    \@ifundefined{KVO@action@\KVOdyn@action}{%
      \PackageError{kvoptions}{%
        Unknown disable action %
        `\expandafter\strip@prefix\meaning\KVOdyn@action'\MessageBreak
        for option `#3' in keyval family '#2'%
      }\@ehc
    }{%
      \csname KVO@action@\KVOdyn@action\endcsname{#2}{#3}%
    }%
  \x
}
\def\KVO@action@undef#1#2{%
  \edef\x{\endgroup
    \ifKVOdyn@global\global\fi
    \let
    \expandafter\noexpand\csname KV@#1@#2\endcsname
    \relax
    \ifKVOdyn@global\global\fi
    \let
    \expandafter\noexpand\csname KV@#1@#2@default\endcsname
    \relax
  }%
  ^^A\PackageInfo{kvoptions}{%
  ^^A  [option] key `#2' of family `#1'\MessageBreak
  ^^A  is disabled (undef, \ifKVOdyn@global global\else local\fi)%
  ^^A}%
}
\def\KVO@action@ignore#1#2{%
  \edef\x{\endgroup
    \ifKVOdyn@global\global\fi
    \let
    \expandafter\noexpand\csname KV@#1@#2\endcsname
    \noexpand\@gobble
    \ifKVOdyn@global\global\fi
    \let
    \expandafter\noexpand\csname KV@#1@#2@default\endcsname
    \noexpand\@empty
  }%
  ^^A\PackageInfo{kvoptions}{%
  ^^A  [option] key `#2' of family `#1'\MessageBreak
  ^^A  is disabled (ignore, \ifKVOdyn@global global\else local\fi)%
  ^^A}%
}
\def\KVO@action@error{%
  \KVO@do@action{error}%
}
\def\KVO@action@warning{%
  \KVO@do@action{warning}%
}
%    \end{macrocode}
%    \begin{tabular}{@{}ll@{}}
%    |#1|& |error| or |warning|\\
%    |#2|& \meta{family}\\
%    |#3|& \meta{key}\\
%    \end{tabular}
%    \begin{macrocode}
\def\KVO@do@action#1#2#3{%
  \ifx\KVOdyn@name\relax
    \PackageError{kvoptions}{%
      Action type `#1' needs package/class name\MessageBreak
      for key `#3' in family `#2'%
    }\@ehc
  \else
    \edef\x{\endgroup
      \noexpand\define@key{#2}{#3}[]{%
        \expandafter\noexpand\csname KVO@disable@#1\endcsname
        {\KVOdyn@name}\noexpand\KVOdyn@ext{#3}%
      }%
      \ifKVOdyn@global
        \global\let
        \expandafter\noexpand\csname KV@#2@#3\endcsname
        \expandafter\noexpand\csname KV@#2@#3\endcsname
        \global\let
        \expandafter\noexpand\csname KV@#2@#3@default\endcsname
        \expandafter\noexpand\csname KV@#2@#3@default\endcsname
      \fi
    }%
    ^^A\ifx\KVOdyn@ext\@clsextension
    ^^A  \expandafter\ClassInfo
    ^^A\else
    ^^A   \expandafter\PackageInfo
    ^^A\fi
    ^^A{\KVOdyn@name}{%
    ^^A  [option] key `#3' of family `#2'\MessageBreak
    ^^A  is disabled (#1, \ifKVOdyn@global global\else local\fi)%
    ^^A}%
  \fi
}
\def\KVO@disable@error#1#2#3{%
  \ifx#2\@clsextension
    \expandafter\ClassError
  \else
    \expandafter\PackageError
  \fi
  {#1}{%
    Option `#3' is given too late,\MessageBreak
    now the option is ignored%
  }\@ehc
}
\def\KVO@disable@warning#1#2#3{%
  \ifx#2\@clsextension
    \expandafter\ClassWarning
  \else
    \expandafter\PackageWarning
  \fi
  {#1}{%
    Option `#3' is already consumed\MessageBreak
    and has no effect%
  }%
}
%    \end{macrocode}
%
% \subsection{Change option code}
%
% \subsubsection{\cs{AddToKeyvalOption}}
%
%    \begin{macro}{\AddToKeyvalOption}
%    \begin{macrocode}
\newcommand*{\AddToKeyvalOption}{%
  \@ifstar{%
    \begingroup
      \edef\x{\endgroup
        \noexpand\KVO@AddToKeyvalOption{\KVO@family}%
      }%
    \x
  }%
  \KVO@AddToKeyvalOption
}
%    \end{macrocode}
%    \end{macro}
%    \begin{macro}{\KVO@AddToKeyvalOption}
%    \begin{macrocode}
\def\KVO@AddToKeyvalOption#1#2{%
  \@ifundefined{KV@#1@#2}{%
    \PackageWarning{kvoptions}{%
      Key `#2' of family `#1' does not exist.\MessageBreak
      Ignoring \string\AddToKeyvalOption
    }%
    \@gobble
  }{%
    \edef\KVO@next{%
      \noexpand\KVO@@AddToKeyvalOption
      \expandafter\noexpand\csname KV@#1@#2\endcsname
    }%
    \afterassignment\KVO@next
    \def\KVO@temp##1%
  }%
}
%    \end{macrocode}
%    \end{macro}
%    \begin{macro}{\KVO@@AddToKeyvalOption}
%    \begin{macrocode}
\def\KVO@@AddToKeyvalOption#1{%
  \begingroup
    \toks@\expandafter{#1{##1}}%
    \toks@\expandafter{\the\expandafter\toks@\KVO@temp{##1}}%
    \edef\x{\endgroup
      \noexpand\def\noexpand#1####1{\the\toks@}%
    }%
  \x
}
%    \end{macrocode}
%    \end{macro}
%
% \subsection{Process options}
%
% \subsubsection{Get global options}
%
%    Package \xpackage{xkeyval} removes options with equal signs from
%    the global options (\cs{@classoptionslist}). The effect is that
%    other packages and classes will not see these global options anymore.
%    A bug-report was answered that this behaviour is ``by design''.
%    Thus I call it a design bug. Now getting the global options require
%    an algorithm instead of a simple macro call.
%    \begin{macrocode}
%</package>
%<*package|patch>
%    \end{macrocode}
%    \begin{macro}{\KVO@IfDefThen}
%    Call |#2| if command |#1| is defined and not \cs{relax}.
%    (Package \xpackage{kvoptions-patch} does not load
%    package \xpackage{ltxcmds}.)
%    \begin{macrocode}
\def\KVO@IfDefThen#1#2{%
  \ifx#1\ltx@undefined
  \else
    \ifx#1\relax
    \else
      #2%
    \fi
  \fi
}%
%    \end{macrocode}
%    \end{macro}
%    \begin{macro}{\KVO@GetClassOptionsList}
%    \begin{macrocode}
\def\KVO@GetClassOptionsList{%
  \let\KVO@classoptionslist\@classoptionslist
  \KVO@IfDefThen\@classoptionslist{%
    \KVO@IfDefThen\XKV@documentclass{%
      \ifx\XKV@documentclass\ltx@empty
      \else
        \KVO@IfDefThen\XKV@classoptionslist{%
          \ifx\XKV@classoptionslist\ltx@empty
          \else
            \let\KVO@classoptionslist\XKV@classoptionslist
          \fi
        }%
      \fi
    }%
  }%
}%
%    \end{macrocode}
%    \end{macro}
%    \begin{macrocode}
%</package|patch>
%<*package>
%    \end{macrocode}
%
% \subsubsection{\cs{ProcessKeyvalOptions}}
%
%    \begin{macro}{\ProcessKeyvalOptions}
%    If the optional star is given, we get the family
%    name and expand it for safety.
%    \begin{macrocode}
\newcommand*{\ProcessKeyvalOptions}{%
  \@ifstar{%
    \begingroup
      \edef\x{\endgroup
        \noexpand\KVO@ProcessKeyvalOptions{\KVO@family}%
      }%
    \x
  }%
  \KVO@ProcessKeyvalOptions
}
%    \end{macrocode}
%    \end{macro}
%
%    \begin{macrocode}
\def\KVO@ProcessKeyvalOptions#1{%
  \let\@tempc\relax
  \let\KVO@temp\@empty
%    \end{macrocode}
%    Add any global options that are known to KV to the start of
%    the list being built in |\KVO@temp| and mark them used (by
%    removing them from the unused option list).
%    \begin{macrocode}
  \ifx\@currext\@clsextension
  \else
    \KVO@GetClassOptionsList
    \ifx\KVO@classoptionslist\relax
    \else
      \@for\KVO@CurrentOption:=\KVO@classoptionslist\do{%
        \@ifundefined{KV@#1@\expandafter\KVO@getkey
                      \KVO@CurrentOption=\@nil}{%
        }{%
          \@ifundefined{KVO@local@#1@\expandafter\KVO@getkey
                        \KVO@CurrentOption=\@nil}{%
            \ifx\KVO@Patch Y%
              \edef\KVO@temp{%
                \etex@unexpanded\expandafter{%
                  \KVO@temp
                }%
                ,%
                \etex@unexpanded\expandafter{%
                  \KVO@CurrentOption
                }%
                ,%
              }%
              \ltx@onelevel@sanitize\KVO@CurrentOption
            \else
              \edef\KVO@temp{%
                \KVO@temp
                ,%
                \KVO@CurrentOption
                ,%
              }%
            \fi
            \@expandtwoargs\@removeelement\KVO@CurrentOption
              \@unusedoptionlist\@unusedoptionlist
          }{}%
        }%
      }%
    \fi
  \fi
%    \end{macrocode}
%     Now stick the package options at the end of the list and wrap
%     in a call to \cs{setkeys}. A class ignores unknown global
%     options, we must remove them to prevent error messages
%     from \cs{setkeys}.
%    \begin{macrocode}
  \begingroup
    \toks\tw@{}%
    \@ifundefined{opt@\@currname.\@currext}{%
      \toks@\expandafter{\KVO@temp}%
    }{%
      \toks@\expandafter\expandafter\expandafter{%
        \csname opt@\@currname.\@currext\endcsname
      }%
      \ifx\@currext\@clsextension
        \edef\CurrentOption{\the\toks@}%
        \toks@\expandafter{\KVO@temp}%
        \@for\CurrentOption:=\CurrentOption\do{%
          \@ifundefined{%
            KV@#1@\expandafter\KVO@getkey\CurrentOption=\@nil
          }{%
%    \end{macrocode}
%    A class puts not used options in the unused option list
%    unless there is a default handler.
%    \begin{macrocode}
            \@ifundefined{KVO@default@\@currname.\@currext}{%
              \ifx\KVO@Patch Y%
                \ltx@onelevel@sanitize\CurrentOption
              \fi
              \ifx\@unusedoptionlist\@empty
                \global\let\@unusedoptionlist\CurrentOption
              \else
                \expandafter\expandafter\expandafter\gdef
                \expandafter\expandafter\expandafter\@unusedoptionlist
                \expandafter\expandafter\expandafter{%
                  \expandafter\@unusedoptionlist
                  \expandafter,\CurrentOption
                }%
              \fi
            }{%
              \toks\tw@\expandafter{%
                \the\toks\expandafter\tw@\expandafter,\CurrentOption
              }%
            }%
          }{%
            \toks@\expandafter{%
              \the\expandafter\toks@\expandafter,\CurrentOption
            }%
          }%
        }%
      \else
%    \end{macrocode}
%    Without default action we pass all options to \cs{setkeys}.
%    Otherwise we have to check which options are known.
%    These are passed to \cs{setkeys}. For the others the default
%    action is performed.
%    \begin{macrocode}
        \@ifundefined{KVO@default@\@currname.\@currext}{%
          \toks@\expandafter\expandafter\expandafter{%
            \expandafter\KVO@temp\the\toks@
          }%
        }{%
          \edef\CurrentOption{\the\toks@}%
          \toks@\expandafter{\KVO@temp}%
          \@for\CurrentOption:=\CurrentOption\do{%
            \@ifundefined{%
              KV@#1@\expandafter\KVO@getkey\CurrentOption=\@nil
            }{%
              \toks\tw@\expandafter{%
                \the\toks\expandafter\tw@\expandafter,\CurrentOption
              }%
            }{%
              \toks@\expandafter{%
                \the\expandafter\toks@\expandafter,\CurrentOption
              }%
            }%
          }%
        }%
      \fi
    }%
    \edef\KVO@temp{\endgroup
      \noexpand\KVO@calldefault{\the\toks\tw@}%
      \noexpand\KVO@setkeys{#1}{\the\toks@}%
    }%
  \KVO@temp
%    \end{macrocode}
%    Some cleanup of \cs{ProcessOptions}.
%    \begin{macrocode}
  \let\CurrentOption\@empty
  \AtEndOfPackage{\let\@unprocessedoptions\relax}%
}
%    \end{macrocode}
%
% \subsubsection{\cs{ProcessLocalKeyvalOptions}}
%
%    \begin{macro}{\ProcessLocalKeyvalOptions}
%    If the optional star is given, we get the family
%    name and expand it for safety.
%    \begin{macrocode}
\newcommand*{\ProcessLocalKeyvalOptions}{%
  \@ifstar{%
    \begingroup
      \edef\x{\endgroup
        \noexpand\KVO@ProcessLocalKeyvalOptions{\KVO@family}%
      }%
    \x
  }%
  \KVO@ProcessLocalKeyvalOptions
}
%    \end{macrocode}
%    \end{macro}
%
%    \begin{macrocode}
\def\KVO@ProcessLocalKeyvalOptions#1{%
  \let\@tempc\relax
  \let\KVO@temp\@empty
%    \end{macrocode}
%    Check if \cs{ProcessLocalKeyvalOptions} is called inside
%    a package.
%    \begin{macrocode}
  \ifx\@currext\@pkgextension
  \else
    \PackageError{kvoptions}{%
      \string\ProcessLocalKeyvalOptions\space is intended for packages only%
    }\@ehc
  \fi
%    \end{macrocode}
%     The package options are put into toks register \cs{toks@}.
%    \begin{macrocode}
  \begingroup
    \toks\tw@{}%
    \@ifundefined{opt@\@currname.\@currext}{%
      \toks@\expandafter{\KVO@temp}%
    }{%
      \toks@\expandafter\expandafter\expandafter{%
        \csname opt@\@currname.\@currext\endcsname
      }%
%    \end{macrocode}
%    Without default action we pass all options to \cs{setkeys}.
%    Otherwise we have to check which options are known.
%    These are passed to \cs{setkeys}. For the others the default
%    action is performed.
%    \begin{macrocode}
      \@ifundefined{KVO@default@\@currname.\@currext}{%
        \toks@\expandafter\expandafter\expandafter{%
          \expandafter\KVO@temp\the\toks@
        }%
      }{%
        \edef\CurrentOption{\the\toks@}%
        \toks@\expandafter{\KVO@temp}%
        \@for\CurrentOption:=\CurrentOption\do{%
          \@ifundefined{%
            KV@#1@\expandafter\KVO@getkey\CurrentOption=\@nil
          }{%
            \toks\tw@\expandafter{%
              \the\toks\expandafter\tw@\expandafter,\CurrentOption
            }%
          }{%
            \toks@\expandafter{%
              \the\expandafter\toks@\expandafter,\CurrentOption
            }%
          }%
        }%
      }%
    }%
    \edef\KVO@temp{\endgroup
      \noexpand\KVO@calldefault{\the\toks\tw@}%
      \noexpand\KVO@setkeys{#1}{\the\toks@}%
    }%
  \KVO@temp
%    \end{macrocode}
%    Some cleanup of \cs{ProcessOptions}.
%    \begin{macrocode}
  \let\CurrentOption\@empty
  \AtEndOfPackage{\let\@unprocessedoptions\relax}%
}
%    \end{macrocode}
%
% \subsubsection{Helper macros}
%
%    \begin{macro}{\KVO@getkey}
%    Extract the key part of a key=value pair.
%    \begin{macrocode}
\def\KVO@getkey#1=#2\@nil{#1}
%    \end{macrocode}
%    \end{macro}
%
%    \begin{macro}{\KVO@calldefault}
%    \begin{macrocode}
\def\KVO@calldefault#1{%
  \begingroup
    \def\x{#1}%
  \expandafter\endgroup
  \ifx\x\@empty
  \else
    \@for\CurrentOption:=#1\do{%
      \ifx\CurrentOption\@empty
      \else
        \expandafter\KVO@setcurrents\CurrentOption=\@nil
        \@nameuse{KVO@default@\@currname.\@currext}%
      \fi
    }%
  \fi
}
%    \end{macrocode}
%    \end{macro}
%    \begin{macro}{\KVO@setcurrents}
%    Extract the key part of a key=value pair.
%    \begin{macrocode}
\def\KVO@setcurrents#1=#2\@nil{%
  \def\CurrentOptionValue{#2}%
  \ifx\CurrentOptionValue\@empty
    \let\CurrentOptionKey\CurrentOption
    \let\CurrentOptionValue\relax
  \else
    \edef\CurrentOptionKey{\zap@space#1 \@empty}%
    \expandafter\KVO@setcurrentvalue\CurrentOption\@nil
  \fi
}
%    \end{macrocode}
%    \end{macro}
%    \begin{macro}{\KV@setcurrentvalue}
%    Here the value part is parsed. Package \xpackage{keyval}'s
%    \cs{KV@@sp@def} helps in removing spaces at the begin and
%    end of the value.
%    \begin{macrocode}
\def\KVO@setcurrentvalue#1=#2\@nil{%
  \KV@@sp@def\CurrentOptionValue{#2}%
}
%    \end{macrocode}
%    \end{macro}
%
% \subsection{\plainTeX}
%
%    Disable \LaTeX\ stuff.
%    \begin{macrocode}
\begingroup\expandafter\expandafter\expandafter\endgroup
\expandafter\ifx\csname documentclass\endcsname\relax
  \def\ProcessKeyvalOptions{%
    \@ifstar{}\@gobble
  }%
\fi
%    \end{macrocode}
%
%    \begin{macrocode}
\KVO@AtEnd%
%</package>
%    \end{macrocode}
%
% \subsection{Package \xpackage{kvoptions-patch}}
%
%    \begin{macrocode}
%<*patch>
\NeedsTeXFormat{LaTeX2e}
\begingroup\catcode61\catcode48\catcode32=10\relax%
  \catcode13=5 % ^^M
  \endlinechar=13 %
  \catcode123=1 % {
  \catcode125=2 % }
  \catcode64=11 % @
  \def\x{\endgroup
    \expandafter\edef\csname KVO@AtEnd\endcsname{%
      \endlinechar=\the\endlinechar\relax
      \catcode13=\the\catcode13\relax
      \catcode32=\the\catcode32\relax
      \catcode35=\the\catcode35\relax
      \catcode61=\the\catcode61\relax
      \catcode64=\the\catcode64\relax
      \catcode123=\the\catcode123\relax
      \catcode125=\the\catcode125\relax
    }%
  }%
\x\catcode61\catcode48\catcode32=10\relax%
\catcode13=5 % ^^M
\endlinechar=13 %
\catcode35=6 % #
\catcode64=11 % @
\catcode123=1 % {
\catcode125=2 % }
\def\TMP@EnsureCode#1#2{%
  \edef\KVO@AtEnd{%
    \KVO@AtEnd
    \catcode#1=\the\catcode#1\relax
  }%
  \catcode#1=#2\relax
}
\TMP@EnsureCode{39}{12}% '
\TMP@EnsureCode{40}{12}% (
\TMP@EnsureCode{41}{12}% )
\TMP@EnsureCode{43}{12}% +
\TMP@EnsureCode{44}{12}% ,
\TMP@EnsureCode{45}{12}% -
\TMP@EnsureCode{46}{12}% .
\TMP@EnsureCode{47}{12}% /
\TMP@EnsureCode{58}{12}% :
\TMP@EnsureCode{60}{12}% <
\TMP@EnsureCode{62}{12}% >
\TMP@EnsureCode{91}{12}% [
\TMP@EnsureCode{93}{12}% ]
\TMP@EnsureCode{96}{12}% `
\TMP@EnsureCode{124}{12}% |
\edef\KVO@AtEnd{\KVO@AtEnd\noexpand\endinput}
\ProvidesPackage{kvoptions-patch}%
  [2016/05/16 v3.12 LaTeX patch for keyval options (HO)]%
%    \end{macrocode}
%
%    Check for \eTeX.
%    \begin{macrocode}
\begingroup\expandafter\expandafter\expandafter\endgroup
\expandafter\ifx\csname eTeXversion\endcsname\relax
  \PackageWarningNoLine{kvoptions-patch}{%
    Package loading is aborted, because e-TeX is missing%
  }%
  \expandafter\KVO@AtEnd
\fi%
%    \end{macrocode}
%
%    Package \xpackage{etexcmds} for \cs{etex@unexpanded}.
%    \begin{macrocode}
\RequirePackage{etexcmds}[2007/09/09]
\ifetex@unexpanded
\else
  \PackageError{kvoptions-patch}{%
    Could not find eTeX's \string\unexpanded.\MessageBreak
    Try adding \string\RequirePackage\string{etexcmds\string} %
    before \string\documentclass%
  }\@ehd
  \expandafter\KVO@AtEnd
\fi%
%    \end{macrocode}
%
%    Check for package \xpackage{xkvltxp}.
%    \begin{macrocode}
\@ifpackageloaded{xkvltxp}{%
  \PackageWarningNoLine{kvoptions}{%
    Option `patch' cannot be used together with\MessageBreak
    package `xkvltxp' that is already loaded.\MessageBreak
    Therefore package loading is aborted%
  }%
  \KVO@AtEnd
}{}%
%    \end{macrocode}
%
%    \begin{macrocode}
\def\@if@ptions#1#2#3{%
  \begingroup
    \KVO@normalize\KVO@temp{#3}%
    \edef\x{\endgroup
      \noexpand\@if@pti@ns{%
        \detokenize\expandafter\expandafter\expandafter{%
          \csname opt@#2.#1\endcsname
        }%
      }{%
        \detokenize\expandafter{\KVO@temp}%
      }%
    }%
  \x
}
%    \end{macrocode}
%
%    \begin{macrocode}
\def\@pass@ptions#1#2#3{%
  \KVO@normalize\KVO@temp{#2}%
  \@ifundefined{opt@#3.#1}{%
    \expandafter\gdef\csname opt@#3.#1%
          \expandafter\endcsname\expandafter{%
      \KVO@temp
    }%
  }{%
    \expandafter\gdef\csname opt@#3.#1%
          \expandafter\expandafter\expandafter\endcsname
          \expandafter\expandafter\expandafter{%
      \csname opt@#3.#1\expandafter\endcsname\expandafter,\KVO@temp
    }%
  }%
}
%    \end{macrocode}
%
%    \begin{macrocode}
\def\ProcessOptions{%
  \let\ds@\@empty
  \@ifundefined{opt@\@currname.\@currext}{%
    \let\@curroptions\@empty
  }{%
    \expandafter\expandafter\expandafter\def
    \expandafter\expandafter\expandafter\@curroptions
    \expandafter\expandafter\expandafter{%
      \csname opt@\@currname.\@currext\endcsname
    }%
  }%
  \@ifstar\KVO@xprocess@ptions\KVO@process@ptions
}
%    \end{macrocode}
%
%    \begin{macrocode}
\def\KVO@process@ptions{%
  \@for\CurrentOption:=\@declaredoptions\do{%
    \ifx\CurrentOption\@empty
    \else
      \begingroup
        \ifx\@currext\@clsextension
          \toks@{}%
        \else
          \KVO@GetClassOptionsList
          \toks@\expandafter{\KVO@classoptionslist,}%
        \fi
        \toks\tw@\expandafter{\@curroptions}%
        \edef\x{\endgroup
          \noexpand\in@{,\CurrentOption,}{,\the\toks@\the\toks\tw@,}%
        }%
      \x
      \ifin@
        \KVO@use@ption
        \expandafter\let\csname ds@\CurrentOption\endcsname\@empty
      \fi
    \fi
  }%
  \KVO@process@pti@ns
}
%    \end{macrocode}
%
%    \begin{macrocode}
\def\KVO@xprocess@ptions{%
  \ifx\@currext\@clsextension
  \else
    \KVO@GetClassOptionsList
    \@for\CurrentOption:=\KVO@classoptionslist\do{%
      \ifx\CurrentOption\@empty
      \else
        \KVO@in@\CurrentOption\@declaredoptions
        \ifin@
          \KVO@use@ption
          \expandafter\let\csname ds@\CurrentOption\endcsname\@empty
        \fi
      \fi
    }%
  \fi
  \KVO@process@pti@ns
}
%    \end{macrocode}
%
%    \begin{macrocode}
\def\KVO@in@#1#2{%
  \in@false
  \begingroup
    \@for\x:=#2\do{%
      \ifx\x#1\relax
        \in@true
      \fi
    }%
    \edef\x{\endgroup
      \ifin@
        \noexpand\in@true
      \fi
    }%
  \x
}
%    \end{macrocode}
%
%    \begin{macrocode}
\def\KVO@process@pti@ns{%
  \@for\CurrentOption:=\@curroptions\do{%
    \@ifundefined{ds@\KVO@SanitizedCurrentOption}{%
      \KVO@use@ption
      \default@ds
    }%
    \KVO@use@ption
  }%
  \@for\CurrentOption:=\@declaredoptions\do{%
    \expandafter\let\csname ds@\CurrentOption\endcsname\relax
  }%
  \let\CurrentOption\@empty
  \let\@fileswith@pti@ns\@@fileswith@pti@ns
  \AtEndOfPackage{\let\@unprocessedoptions\relax}%
}
%    \end{macrocode}
%
%    \begin{macrocode}
\def\KVO@use@ption{%
  \begingroup
    \edef\x{\endgroup
      \noexpand\@removeelement{%
        \detokenize\expandafter{\CurrentOption}%
      }{%
        \detokenize\expandafter{\@unusedoptionlist}%
      }%
    }%
  \x\@unusedoptionlist
  \csname ds@\KVO@SanitizedCurrentOption\endcsname
}
%    \end{macrocode}
%
%    \begin{macrocode}
\def\OptionNotUsed{%
  \ifx\@currext\@clsextension
    \xdef\@unusedoptionlist{%
      \ifx\@unusedoptionlist\@empty
      \else
        \detokenize\expandafter{\@unusedoptionlist,}%
      \fi
      \detokenize\expandafter{\CurrentOption}%
    }%
  \fi
}
%    \end{macrocode}
%
%    Variant of \cs{ExecuteOptions} that better protects \cs{CurrentOption}.
%    \begin{macrocode}
\def\CurrentOption@SaveLevel{0}
\def\ExecuteOptions{%
  \expandafter\KVO@ExecuteOptions
      \csname CurrentOption@\CurrentOption@SaveLevel\endcsname
}
\def\KVO@ExecuteOptions#1#2{%
  \let#1\CurrentOption
  \edef\CurrentOption@SaveLevel{%
    \the\numexpr\CurrentOption@SaveLevel+1%
  }%
  \@for\CurrentOption:=#2\do{%
    \csname ds@\CurrentOption\endcsname
  }%
  \edef\CurrentOption@SaveLevel{%
    \the\numexpr\CurrentOption@SaveLevel-1%
  }%
  \let\CurrentOption#1%
}
%    \end{macrocode}
%
%    \begin{macrocode}
\def\KVO@fileswith@pti@ns#1[#2]#3[#4]{%
  \ifx#1\@clsextension
    \ifx\@classoptionslist\relax
      \KVO@normalize\KVO@temp{#2}%
      \expandafter\gdef\expandafter\@classoptionslist\expandafter{%
        \KVO@temp
      }%
      \def\reserved@a{%
        \KVO@onefilewithoptions{#3}[{#2}][{#4}]#1%
        \@documentclasshook
      }%
    \else
      \def\reserved@a{%
        \KVO@onefilewithoptions{#3}[{#2}][{#4}]#1%
      }%
    \fi
  \else
    \begingroup
      \let\KVO@temp\relax
      \let\KVO@onefilewithoptions\relax
      \let\@pkgextension\relax
      \def\reserved@b##1,{%
        \ifx\@nil##1\relax
        \else
          \ifx\relax##1\relax
          \else
            \KVO@onefilewithoptions{##1}[{\KVO@temp}][{#4}]%
            \@pkgextension
          \fi
          \expandafter\reserved@b
        \fi
      }%
      \edef\reserved@a{\zap@space#3 \@empty}%
      \edef\reserved@a{\expandafter\reserved@b\reserved@a,\@nil,}%
      \toks@{#2}%
      \def\KVO@temp{\the\toks@}%
    \edef\reserved@a{\endgroup \reserved@a}%
  \fi
  \reserved@a
}
%    \end{macrocode}
%
%    \begin{macrocode}
\def\KVO@onefilewithoptions#1[#2][#3]#4{%
  \@pushfilename
  \xdef\@currname{#1}%
  \global\let\@currext#4%
  \expandafter\let\csname\@currname.\@currext-h@@k\endcsname\@empty
  \let\CurrentOption\@empty
  \@reset@ptions
  \makeatletter
  \def\reserved@a{%
    \@ifl@aded\@currext{#1}{%
      \@if@ptions\@currext{#1}{#2}{%
      }{%
        \begingroup
          \@ifundefined{opt@#1.\@currext}{%
            \def\x{}%
          }{%
            \edef\x{%
              \expandafter\expandafter\expandafter\strip@prefix
              \expandafter\meaning\csname opt@#1.\@currext\endcsname
            }%
          }%
          \def\y{#2}%
          \edef\y{\expandafter\strip@prefix\meaning\y}%
          \@latex@error{Option clash for \@cls@pkg\space #1}{%
            The package #1 has already been loaded %
            with options:\MessageBreak
            \space\space[\x]\MessageBreak
            There has now been an attempt to load it %
             with options\MessageBreak
            \space\space[\y]\MessageBreak
            Adding the global options:\MessageBreak
            \space\space
                 \x,\y\MessageBreak
            to your \noexpand\documentclass declaration may fix this.%
            \MessageBreak
            Try typing \space <return> \space to proceed.%
          }%
        \endgroup
      }%
    }{%
      \@pass@ptions\@currext{#2}{#1}%
      \global\expandafter
      \let\csname ver@\@currname.\@currext\endcsname\@empty
      \InputIfFileExists
        {\@currname.\@currext}%
        {}%
        {\@missingfileerror\@currname\@currext}%
      \let\@unprocessedoptions\@@unprocessedoptions
      \csname\@currname.\@currext-h@@k\endcsname
      \expandafter\let\csname\@currname.\@currext-h@@k\endcsname
              \@undefined
      \@unprocessedoptions
    }%
    \@ifl@ter\@currext{#1}{#3}{%
    }{%
      \@latex@warning@no@line{%
        You have requested,\on@line, %
        version\MessageBreak
          #3' of \@cls@pkg\space #1,\MessageBreak
        but only version\MessageBreak
         `\csname ver@#1.\@currext\endcsname'\MessageBreak
        is available%
      }%
    }%
    \ifx\@currext\@clsextension\let\LoadClass\@twoloadclasserror\fi
    \@popfilename
    \@reset@ptions
  }%
  \reserved@a
}
%    \end{macrocode}
%
%    \begin{macrocode}
\def\@unknownoptionerror{%
  \@latex@error{%
    Unknown option `\KVO@SanitizedCurrentOption' %
    for \@cls@pkg\space`\@currname'%
  }{%
    The option `\KVO@SanitizedCurrentOption' was not declared in %
    \@cls@pkg\space`\@currname', perhaps you\MessageBreak
    misspelled its name. %
    Try typing \space <return> %
    \space to proceed.%
  }%
}
%    \end{macrocode}
%
%    \begin{macrocode}
\def\@@unprocessedoptions{%
  \ifx\@currext\@pkgextension
    \@ifundefined{opt@\@currname.\@currext}{%
      \let\@curroptions\@empty
    }{%
      \expandafter\let\expandafter\@curroptions
          \csname opt@\@currname.\@currext\endcsname
    }%
    \@for\CurrentOption:=\@curroptions\do{%
        \ifx\CurrentOption\@empty\else\@unknownoptionerror\fi
    }%
  \fi
}
%    \end{macrocode}
%
%    \begin{macrocode}
\def\KVO@SanitizedCurrentOption{%
  \expandafter\strip@prefix\meaning\CurrentOption
}
%    \end{macrocode}
%
%    Normalize option list.
%    \begin{macrocode}
\def\KVO@normalize#1#2{%
  \let\KVO@result\@empty
  \KVO@splitcomma#2,\@nil
  \let#1\KVO@result
}
\def\KVO@splitcomma#1,#2\@nil{%
  \KVO@ifempty{#1}{}{%
    \KVO@checkkv#1=\@nil
  }%
  \KVO@ifempty{#2}{}{\KVO@splitcomma#2\@nil}%
}
\def\KVO@ifempty#1{%
  \expandafter\ifx\expandafter\\\detokenize{#1}\\%
    \expandafter\@firstoftwo
  \else
    \expandafter\@secondoftwo
  \fi
}
\def\KVO@checkkv#1=#2\@nil{%
  \KVO@ifempty{#2}{%
    % option without value
    \edef\KVO@x{\zap@space#1 \@empty}%
    \ifx\KVO@x\@empty
      % ignore empty option
    \else
      % append to list
      \edef\KVO@result{%
        \etex@unexpanded\expandafter{\KVO@result},\KVO@x
      }%
    \fi
  }{%
    % #1: "key", #2: "value="
    % add key part
    \edef\KVO@result{%
      \etex@unexpanded\expandafter{\KVO@result},%
      \zap@space#1 \@empty
    }%
    \futurelet\@let@token\KVO@checkfirsttok#2 \@nil| = \@nil|\KVO@nil
  }%
}
\def\KVO@checkfirsttok{%
  \ifx\@let@token\bgroup
    % no space at start
    \expandafter\KVO@removelastspace\expandafter=%
    % "<value><spaceopt>= \@nil"
  \else
    \expandafter\KVO@checkfirstA
  \fi
}
\def\KVO@checkfirstA#1 #2\@nil{%
  \KVO@ifempty{#2}{%
    \KVO@removelastspace=#1 \@nil
  }{%
    \KVO@ifempty{#1}{%
      \KVO@removelastspace=#2\@nil
    }{%
      \KVO@removelastspace=#1 #2\@nil
    }%
  }%
}
\def\KVO@removelastspace#1 = \@nil|#2\KVO@nil{%
  \KVO@ifempty{#2}{%
    \edef\KVO@result{%
      \etex@unexpanded\expandafter{\KVO@result}%
      \etex@unexpanded\expandafter{\KVO@removegarbage#1\KVO@nil}%
    }%
  }{%
    \edef\KVO@result{%
      \etex@unexpanded\expandafter{\KVO@result}%
      \etex@unexpanded{#1}%
    }%
  }%
}
\def\KVO@removegarbage#1= \@nil#2\KVO@nil{#1}%
%    \end{macrocode}
%
%    Arguments |#1| and |#2| are macros.
%    \begin{macrocode}
\def\KVO@removeelement#1#2{%
  \begingroup
    \toks@={}%
    \@for\x:=#2\do{%
      \ifx\x\@empty
      \else
        \ifx\x#1\relax
        \else
          \edef\t{\the\toks@}%
          \ifx\t\@empty
          \else
            \toks@\expandafter{\the\toks@,}%
          \fi
          \toks@\expandafter{\the\expandafter\toks@\x}%
        \fi
      \fi
    }%
    \edef\x{\endgroup
      \def\noexpand#2{\the\toks@}%
    }%
  \x
}
%    \end{macrocode}
%
%
%    \begin{macrocode}
\let\@@fileswith@pti@ns\KVO@fileswith@pti@ns
\ifx\@fileswith@pti@ns\@badrequireerror
\else
  \let\@fileswith@pti@ns\KVO@fileswith@pti@ns
\fi
%    \end{macrocode}
%
%    \begin{macro}{\KVO@Patch}
%    \begin{macrocode}
\let\KVO@Patch=Y
%    \end{macrocode}
%    \end{macro}
%
%    \begin{macrocode}
\KVO@AtEnd%
%</patch>
%    \end{macrocode}
%
% \section{Test}
%
% \subsection{Preface for standard catcode check}
%
%    \begin{macrocode}
%<*test1>
\input miniltx.tex\relax
%</test1>
%    \end{macrocode}
%
% \subsection{Catcode checks for loading}
%
%    \begin{macrocode}
%<*test1>
%    \end{macrocode}
%    \begin{macrocode}
\catcode`\{=1 %
\catcode`\}=2 %
\catcode`\#=6 %
\catcode`\@=11 %
\expandafter\ifx\csname count@\endcsname\relax
  \countdef\count@=255 %
\fi
\expandafter\ifx\csname @gobble\endcsname\relax
  \long\def\@gobble#1{}%
\fi
\expandafter\ifx\csname @firstofone\endcsname\relax
  \long\def\@firstofone#1{#1}%
\fi
\expandafter\ifx\csname loop\endcsname\relax
  \expandafter\@firstofone
\else
  \expandafter\@gobble
\fi
{%
  \def\loop#1\repeat{%
    \def\body{#1}%
    \iterate
  }%
  \def\iterate{%
    \body
      \let\next\iterate
    \else
      \let\next\relax
    \fi
    \next
  }%
  \let\repeat=\fi
}%
\def\RestoreCatcodes{}
\count@=0 %
\loop
  \edef\RestoreCatcodes{%
    \RestoreCatcodes
    \catcode\the\count@=\the\catcode\count@\relax
  }%
\ifnum\count@<255 %
  \advance\count@ 1 %
\repeat

\def\RangeCatcodeInvalid#1#2{%
  \count@=#1\relax
  \loop
    \catcode\count@=15 %
  \ifnum\count@<#2\relax
    \advance\count@ 1 %
  \repeat
}
\def\RangeCatcodeCheck#1#2#3{%
  \count@=#1\relax
  \loop
    \ifnum#3=\catcode\count@
    \else
      \errmessage{%
        Character \the\count@\space
        with wrong catcode \the\catcode\count@\space
        instead of \number#3%
      }%
    \fi
  \ifnum\count@<#2\relax
    \advance\count@ 1 %
  \repeat
}
\def\space{ }
\expandafter\ifx\csname LoadCommand\endcsname\relax
  \def\LoadCommand{\input kvoptions.sty\relax}%
\fi
\def\Test{%
  \RangeCatcodeInvalid{0}{47}%
  \RangeCatcodeInvalid{58}{64}%
  \RangeCatcodeInvalid{91}{96}%
  \RangeCatcodeInvalid{123}{255}%
  \catcode`\@=12 %
  \catcode`\\=0 %
  \catcode`\%=14 %
  \LoadCommand
  \RangeCatcodeCheck{0}{36}{15}%
  \RangeCatcodeCheck{37}{37}{14}%
  \RangeCatcodeCheck{38}{47}{15}%
  \RangeCatcodeCheck{48}{57}{12}%
  \RangeCatcodeCheck{58}{63}{15}%
  \RangeCatcodeCheck{64}{64}{12}%
  \RangeCatcodeCheck{65}{90}{11}%
  \RangeCatcodeCheck{91}{91}{15}%
  \RangeCatcodeCheck{92}{92}{0}%
  \RangeCatcodeCheck{93}{96}{15}%
  \RangeCatcodeCheck{97}{122}{11}%
  \RangeCatcodeCheck{123}{255}{15}%
  \RestoreCatcodes
}
\Test
\csname @@end\endcsname
\end
%    \end{macrocode}
%    \begin{macrocode}
%</test1>
%    \end{macrocode}
%
%    \begin{macrocode}
%<*test2>
\NeedsTeXFormat{LaTeX2e}
\makeatletter
\catcode`\@=11 %
\def\RestoreCatcodes{}
\count@=0 %
\loop
  \edef\RestoreCatcodes{%
    \RestoreCatcodes
    \catcode\the\count@=\the\catcode\count@\relax
  }%
\ifnum\count@<255 %
  \advance\count@\@ne
\repeat

\def\RangeCatcodeInvalid#1#2{%
  \count@=#1\relax
  \loop
    \catcode\count@=15 %
  \ifnum\count@<#2\relax
    \advance\count@\@ne
  \repeat
}
\def\Test#1{%
  \RangeCatcodeInvalid{0}{47}%
  \RangeCatcodeInvalid{58}{64}%
  \RangeCatcodeInvalid{91}{96}%
  \RangeCatcodeInvalid{123}{255}%
  \catcode`\@=12 %
  \catcode`\\=0 %
  \catcode`\{=1 %
  \catcode`\}=2 %
  \catcode`\#=6 %
  \catcode`\[=12 %
  \catcode`\]=12 %
  \catcode`\%=14 %
  \catcode`\ =10 %
  \catcode13=5 %
  #1\relax
  \RestoreCatcodes
}
\Test{\RequirePackage{kvoptions-patch}}%
\Test{\RequirePackage{kvoptions}}%
\csname @@end\endcsname
%</test2>
%    \end{macrocode}
%
%    \begin{macrocode}
%<*test3>
\NeedsTeXFormat{LaTeX2e}
\makeatletter
\RequirePackage{kvoptions}[2016/05/16]
\def\msg#{\immediate\write16}
\define@key{testfamily}{testkey}{%
  \msg{[testfamily/testkey/#1]}%
}
\define@key{testfamily}{testdefaultkey}[testdefault]{%
  \msg{[testfamily/testdefaultkey/#1]}%
}
\AddToKeyvalOption{testfamily}{testkey}{%
  \msg{[addition/#1]}%
}
\AddToKeyvalOption{testfamily}{testdefaultkey}{%
  \msg{[addition/#1]}%
}
\setkeys{testfamily}{%
  testkey=testA,%
  testdefaultkey=testB,%
  testdefaultkey,%
}
\SetupKeyvalOptions{%
  family=testfamily%
}
\AddToKeyvalOption*{testkey}{%
  \msg{[star addition/#1]}%
}
\AddToKeyvalOption*{testdefaultkey}{%
  \msg{[star addition/#1]}%
}
\setkeys{testfamily}{%
  testkey=testA,%
  testdefaultkey=testB,%
  testdefaultkey,%
}
\@@end
%</test3>
%    \end{macrocode}
%
%    \begin{macrocode}
%<*test4pkg>
\NeedsTeXFormat{LaTeX2e}
\ProvidesPackage{kvoptions-test4}[2016/05/16 package for testing]
\RequirePackage{kvoptions}[2016/05/16]
\SetupKeyvalOptions{%
  family=FOO,%
  prefix=foo,%
  setkeys=\kvsetkeys,%
}
\DeclareStringOption{str}
\define@key{FOO}{set}{%
  \setkeys{BAR}{strbar={#1}}%
}
\define@key{BAR}{strbar}{%
  \def\foostr{[BAR:#1]}%
}
\ProcessKeyvalOptions*
%</test4pkg>
%    \end{macrocode}
%    \begin{macrocode}
%<*test4>
\NeedsTeXFormat{LaTeX2e}
\ProvidesFile{kvoptions-test4.tex}[2016/05/16 test file]
\RequirePackage[%
  str=A,set=B,str=C,%
]{kvoptions-test4}[2016/05/16]
\def\TestExpected{C}
\ifx\foostr\TestExpected
  \typeout{* Test ok.}%
\else
  \typeout{* Result: [\foostr]}%
  \typeout{* Expected: [\TestExpected]}%
  \errmessage{Test failed!}%
\fi
\csname @@end\endcsname\end
%</test4>
%    \end{macrocode}
%
% \section{Installation}
%
% \subsection{Download}
%
% \paragraph{Package.} This package is available on
% CTAN\footnote{\CTANpkg{kvoptions}}:
% \begin{description}
% \item[\CTAN{macros/latex/contrib/oberdiek/kvoptions.dtx}] The source file.
% \item[\CTAN{macros/latex/contrib/oberdiek/kvoptions.pdf}] Documentation.
% \end{description}
%
%
% \paragraph{Bundle.} All the packages of the bundle `oberdiek'
% are also available in a TDS compliant ZIP archive. There
% the packages are already unpacked and the documentation files
% are generated. The files and directories obey the TDS standard.
% \begin{description}
% \item[\CTANinstall{install/macros/latex/contrib/oberdiek.tds.zip}]
% \end{description}
% \emph{TDS} refers to the standard ``A Directory Structure
% for \TeX\ Files'' (\CTAN{tds/tds.pdf}). Directories
% with \xfile{texmf} in their name are usually organized this way.
%
% \subsection{Bundle installation}
%
% \paragraph{Unpacking.} Unpack the \xfile{oberdiek.tds.zip} in the
% TDS tree (also known as \xfile{texmf} tree) of your choice.
% Example (linux):
% \begin{quote}
%   |unzip oberdiek.tds.zip -d ~/texmf|
% \end{quote}
%
% \paragraph{Script installation.}
% Check the directory \xfile{TDS:scripts/oberdiek/} for
% scripts that need further installation steps.
%
% \subsection{Package installation}
%
% \paragraph{Unpacking.} The \xfile{.dtx} file is a self-extracting
% \docstrip\ archive. The files are extracted by running the
% \xfile{.dtx} through \plainTeX:
% \begin{quote}
%   \verb|tex kvoptions.dtx|
% \end{quote}
%
% \paragraph{TDS.} Now the different files must be moved into
% the different directories in your installation TDS tree
% (also known as \xfile{texmf} tree):
% \begin{quote}
% \def\t{^^A
% \begin{tabular}{@{}>{\ttfamily}l@{ $\rightarrow$ }>{\ttfamily}l@{}}
%   kvoptions.sty & tex/latex/oberdiek/kvoptions.sty\\
%   kvoptions-patch.sty & tex/latex/oberdiek/kvoptions-patch.sty\\
%   kvoptions.pdf & doc/latex/oberdiek/kvoptions.pdf\\
%   example-mycolorsetup.sty & doc/latex/oberdiek/example-mycolorsetup.sty\\
%   test/kvoptions-test1.tex & doc/latex/oberdiek/test/kvoptions-test1.tex\\
%   test/kvoptions-test2.tex & doc/latex/oberdiek/test/kvoptions-test2.tex\\
%   test/kvoptions-test3.tex & doc/latex/oberdiek/test/kvoptions-test3.tex\\
%   test/kvoptions-test4.tex & doc/latex/oberdiek/test/kvoptions-test4.tex\\
%   test/kvoptions-test4.sty & doc/latex/oberdiek/test/kvoptions-test4.sty\\
%   kvoptions.dtx & source/latex/oberdiek/kvoptions.dtx\\
% \end{tabular}^^A
% }^^A
% \sbox0{\t}^^A
% \ifdim\wd0>\linewidth
%   \begingroup
%     \advance\linewidth by\leftmargin
%     \advance\linewidth by\rightmargin
%   \edef\x{\endgroup
%     \def\noexpand\lw{\the\linewidth}^^A
%   }\x
%   \def\lwbox{^^A
%     \leavevmode
%     \hbox to \linewidth{^^A
%       \kern-\leftmargin\relax
%       \hss
%       \usebox0
%       \hss
%       \kern-\rightmargin\relax
%     }^^A
%   }^^A
%   \ifdim\wd0>\lw
%     \sbox0{\small\t}^^A
%     \ifdim\wd0>\linewidth
%       \ifdim\wd0>\lw
%         \sbox0{\footnotesize\t}^^A
%         \ifdim\wd0>\linewidth
%           \ifdim\wd0>\lw
%             \sbox0{\scriptsize\t}^^A
%             \ifdim\wd0>\linewidth
%               \ifdim\wd0>\lw
%                 \sbox0{\tiny\t}^^A
%                 \ifdim\wd0>\linewidth
%                   \lwbox
%                 \else
%                   \usebox0
%                 \fi
%               \else
%                 \lwbox
%               \fi
%             \else
%               \usebox0
%             \fi
%           \else
%             \lwbox
%           \fi
%         \else
%           \usebox0
%         \fi
%       \else
%         \lwbox
%       \fi
%     \else
%       \usebox0
%     \fi
%   \else
%     \lwbox
%   \fi
% \else
%   \usebox0
% \fi
% \end{quote}
% If you have a \xfile{docstrip.cfg} that configures and enables \docstrip's
% TDS installing feature, then some files can already be in the right
% place, see the documentation of \docstrip.
%
% \subsection{Refresh file name databases}
%
% If your \TeX~distribution
% (\TeX\,Live, \mikTeX, \dots) relies on file name databases, you must refresh
% these. For example, \TeX\,Live\ users run \verb|texhash| or
% \verb|mktexlsr|.
%
% \subsection{Some details for the interested}
%
% \paragraph{Unpacking with \LaTeX.}
% The \xfile{.dtx} chooses its action depending on the format:
% \begin{description}
% \item[\plainTeX:] Run \docstrip\ and extract the files.
% \item[\LaTeX:] Generate the documentation.
% \end{description}
% If you insist on using \LaTeX\ for \docstrip\ (really,
% \docstrip\ does not need \LaTeX), then inform the autodetect routine
% about your intention:
% \begin{quote}
%   \verb|latex \let\install=y% \iffalse meta-comment
%
% File: kvoptions.dtx
% Version: 2016/05/16 v3.12
% Info: Key value format for package options
%
% Copyright (C)
%    2004, 2006, 2007, 2009-2011 Heiko Oberdiek
%    2016-2019 Oberdiek Package Support Group
%    https://github.com/ho-tex/oberdiek/issues
%
% This work may be distributed and/or modified under the
% conditions of the LaTeX Project Public License, either
% version 1.3c of this license or (at your option) any later
% version. This version of this license is in
%    https://www.latex-project.org/lppl/lppl-1-3c.txt
% and the latest version of this license is in
%    https://www.latex-project.org/lppl.txt
% and version 1.3 or later is part of all distributions of
% LaTeX version 2005/12/01 or later.
%
% This work has the LPPL maintenance status "maintained".
%
% The Current Maintainers of this work are
% Heiko Oberdiek and the Oberdiek Package Support Group
% https://github.com/ho-tex/oberdiek/issues
%
% This work consists of the main source file kvoptions.dtx
% and the derived files
%    kvoptions.sty, kvoptions.pdf, kvoptions.ins, kvoptions.drv,
%    kvoptions-patch.sty, example-mycolorsetup.sty,
%    kvoptions-test1.tex, kvoptions-test2.tex,
%    kvoptions-test3.tex, kvoptions-test4.tex,
%    kvoptions-test4.sty.
%
% Distribution:
%    CTAN:macros/latex/contrib/oberdiek/kvoptions.dtx
%    CTAN:macros/latex/contrib/oberdiek/kvoptions.pdf
%
% Unpacking:
%    (a) If kvoptions.ins is present:
%           tex kvoptions.ins
%    (b) Without kvoptions.ins:
%           tex kvoptions.dtx
%    (c) If you insist on using LaTeX
%           latex \let\install=y\input{kvoptions.dtx}
%        (quote the arguments according to the demands of your shell)
%
% Documentation:
%    (a) If kvoptions.drv is present:
%           latex kvoptions.drv
%    (b) Without kvoptions.drv:
%           latex kvoptions.dtx; ...
%    The class ltxdoc loads the configuration file ltxdoc.cfg
%    if available. Here you can specify further options, e.g.
%    use A4 as paper format:
%       \PassOptionsToClass{a4paper}{article}
%
%    Programm calls to get the documentation (example):
%       pdflatex kvoptions.dtx
%       makeindex -s gind.ist kvoptions.idx
%       pdflatex kvoptions.dtx
%       makeindex -s gind.ist kvoptions.idx
%       pdflatex kvoptions.dtx
%
% Installation:
%    TDS:tex/latex/oberdiek/kvoptions.sty
%    TDS:tex/latex/oberdiek/kvoptions-patch.sty
%    TDS:doc/latex/oberdiek/kvoptions.pdf
%    TDS:doc/latex/oberdiek/example-mycolorsetup.sty
%    TDS:doc/latex/oberdiek/test/kvoptions-test1.tex
%    TDS:doc/latex/oberdiek/test/kvoptions-test2.tex
%    TDS:doc/latex/oberdiek/test/kvoptions-test3.tex
%    TDS:doc/latex/oberdiek/test/kvoptions-test4.tex
%    TDS:doc/latex/oberdiek/test/kvoptions-test4.sty
%    TDS:source/latex/oberdiek/kvoptions.dtx
%
%<*ignore>
\begingroup
  \catcode123=1 %
  \catcode125=2 %
  \def\x{LaTeX2e}%
\expandafter\endgroup
\ifcase 0\ifx\install y1\fi\expandafter
         \ifx\csname processbatchFile\endcsname\relax\else1\fi
         \ifx\fmtname\x\else 1\fi\relax
\else\csname fi\endcsname
%</ignore>
%<*install>
\input docstrip.tex
\Msg{************************************************************************}
\Msg{* Installation}
\Msg{* Package: kvoptions 2016/05/16 v3.12 Key value format for package options (HO)}
\Msg{************************************************************************}

\keepsilent
\askforoverwritefalse

\let\MetaPrefix\relax
\preamble

This is a generated file.

Project: kvoptions
Version: 2016/05/16 v3.12

Copyright (C)
   2004, 2006, 2007, 2009-2011 Heiko Oberdiek
   2016-2019 Oberdiek Package Support Group

This work may be distributed and/or modified under the
conditions of the LaTeX Project Public License, either
version 1.3c of this license or (at your option) any later
version. This version of this license is in
   https://www.latex-project.org/lppl/lppl-1-3c.txt
and the latest version of this license is in
   https://www.latex-project.org/lppl.txt
and version 1.3 or later is part of all distributions of
LaTeX version 2005/12/01 or later.

This work has the LPPL maintenance status "maintained".

The Current Maintainers of this work are
Heiko Oberdiek and the Oberdiek Package Support Group
https://github.com/ho-tex/oberdiek/issues


This work consists of the main source file kvoptions.dtx
and the derived files
   kvoptions.sty, kvoptions.pdf, kvoptions.ins, kvoptions.drv,
   kvoptions-patch.sty, example-mycolorsetup.sty,
   kvoptions-test1.tex, kvoptions-test2.tex,
   kvoptions-test3.tex, kvoptions-test4.tex,
   kvoptions-test4.sty.

\endpreamble
\let\MetaPrefix\DoubleperCent

\generate{%
  \file{kvoptions.ins}{\from{kvoptions.dtx}{install}}%
  \file{kvoptions.drv}{\from{kvoptions.dtx}{driver}}%
  \usedir{tex/latex/oberdiek}%
  \file{kvoptions.sty}{\from{kvoptions.dtx}{package}}%
  \file{kvoptions-patch.sty}{\from{kvoptions.dtx}{patch}}%
  \usedir{doc/latex/oberdiek}%
  \file{example-mycolorsetup.sty}{\from{kvoptions.dtx}{example}}%
%  \usedir{doc/latex/oberdiek/test}%
%  \file{kvoptions-test1.tex}{\from{kvoptions.dtx}{test1}}%
%  \file{kvoptions-test2.tex}{\from{kvoptions.dtx}{test2}}%
%  \file{kvoptions-test3.tex}{\from{kvoptions.dtx}{test3}}%
%  \file{kvoptions-test4.tex}{\from{kvoptions.dtx}{test4}}%
%  \file{kvoptions-test4.sty}{\from{kvoptions.dtx}{test4pkg}}%
  \nopreamble
  \nopostamble
%  \usedir{source/latex/oberdiek/catalogue}%
%  \file{kvoptions.xml}{\from{kvoptions.dtx}{catalogue}}%
}

\catcode32=13\relax% active space
\let =\space%
\Msg{************************************************************************}
\Msg{*}
\Msg{* To finish the installation you have to move the following}
\Msg{* files into a directory searched by TeX:}
\Msg{*}
\Msg{*     kvoptions.sty, kvoptions-patch.sty}
\Msg{*}
\Msg{* To produce the documentation run the file `kvoptions.drv'}
\Msg{* through LaTeX.}
\Msg{*}
\Msg{* Happy TeXing!}
\Msg{*}
\Msg{************************************************************************}

\endbatchfile
%</install>
%<*ignore>
\fi
%</ignore>
%<*driver>
\NeedsTeXFormat{LaTeX2e}
\ProvidesFile{kvoptions.drv}%
  [2016/05/16 v3.12 Key value format for package options (HO)]%
\documentclass{ltxdoc}
\usepackage{holtxdoc}[2011/11/22]
\begin{document}
  \DocInput{kvoptions.dtx}%
\end{document}
%</driver>
% \fi
%
%
% \CharacterTable
%  {Upper-case    \A\B\C\D\E\F\G\H\I\J\K\L\M\N\O\P\Q\R\S\T\U\V\W\X\Y\Z
%   Lower-case    \a\b\c\d\e\f\g\h\i\j\k\l\m\n\o\p\q\r\s\t\u\v\w\x\y\z
%   Digits        \0\1\2\3\4\5\6\7\8\9
%   Exclamation   \!     Double quote  \"     Hash (number) \#
%   Dollar        \$     Percent       \%     Ampersand     \&
%   Acute accent  \'     Left paren    \(     Right paren   \)
%   Asterisk      \*     Plus          \+     Comma         \,
%   Minus         \-     Point         \.     Solidus       \/
%   Colon         \:     Semicolon     \;     Less than     \<
%   Equals        \=     Greater than  \>     Question mark \?
%   Commercial at \@     Left bracket  \[     Backslash     \\
%   Right bracket \]     Circumflex    \^     Underscore    \_
%   Grave accent  \`     Left brace    \{     Vertical bar  \|
%   Right brace   \}     Tilde         \~}
%
% \GetFileInfo{kvoptions.drv}
%
% \title{The \xpackage{kvoptions} package}
% \date{2016/05/16 v3.12}
% \author{Heiko Oberdiek\thanks
% {Please report any issues at \url{https://github.com/ho-tex/oberdiek/issues}}}
%
% \maketitle
%
% \begin{abstract}
% This package is intended for package authors who want to
% use options in key value format for their package options.
% \end{abstract}
%
% \tableofcontents
%
% \def\M#1{\texttt{\{}\meta{#1}\texttt{\}}}
% \def\O#1{\texttt{[}\meta{#1}\texttt{]}}
%
% \section{Introduction}
%
% First I want to recommend the very good review article
% ``A guide to key-value methods'' by Joseph Wright \cite{tb94wright}.
% It introduces the different key-value packages and compares them.
%
% \subsection{The beginning}
%
% This package \xpackage{kvoptions} addresses class or package writers
% that want to allow their users to specify options
% as key value pairs, e.g.:
% \begin{quote}
% |\documentclass[verbose=false,name=me]{myclass}|\\
% |\usepackage[format=print]{mylayout}|
% \end{quote}
% Prominent example is package \xpackage{hyperref},
% probably the first package that offers this service.
% It's \cs{ProcessOptionsWithKV} is often copied und
% used in other packages, e.g. package \xpackage{helvet}
% that uses this interface for its option \xoption{scaled}.
%
% However copying code is not the most modern software
% development technique. And \xpackage{hyperref}'s
% code for \cs{ProcessOptionsWithKV} was changed
% to fix bugs. The version used in other packages
% depends on the time of copying and the awareness
% of \xpackage{hyperref}'s changes.
% Now the code is sourced out into this package and
% available for other package or class writers.
%
% \subsection{Overview}
%
% Package \xpackage{kvoptions} connects package
% \xpackage{\emph{k}ey\emph{v}al}
% with \LaTeX's package and class \emph{options}:
% \begin{center}
% \def\T#1{\bfseries #1}
% \renewcommand{\arraystretch}{1.5}
% \begin{tabular}{l|l|l}
%   \T{Package \xpackage{keyval}}
%   & \T{Package \xpackage{kvoptions}}
%   & \T{\LaTeX\ kernel}
%   \\
% \hline
%     \cs{define@key}
%   &
%     \renewcommand*{\arraystretch}{1}^^A
%     \begin{tabular}[t]{@{}l@{}}
%       \cs{DeclareVoidOption}\\
%       \cs{DeclareStringOption}\\
%       \cs{DeclareBoolOption}\\
%       \cs{DeclareComplementaryOption}\\
%       \cs{DisableKeyvalOption}
%     \end{tabular}
%   & \cs{DeclareOption}
%   \\
% \hline
%   & \cs{DeclareDefaultOption}
%   & \cs{DeclareOption*}
%   \\
% \hline
%   & \cs{ProcessKeyvalOptions}
%   & \cs{ProcessOptions*}
%   \\
% \hline
%   & Option \xoption{patch}
%   &
%     \renewcommand*{\arraystretch}{1}^^A
%     \begin{tabular}[t]{@{}l@{}}
%       Class/package\\
%       option system
%     \end{tabular}
%   \\
% \hline
%   & \cs{SetupKeyvalOptions} &
%   \\
% \end{tabular}
% \end{center}
%
% \hypersetup{bookmarksopenlevel=3}
% \section{Usage}
%
% \subsection{Process options}
%
% \subsubsection{\cs{ProcessKeyvalOptions}}
%
% \begin{declcs}{ProcessKeyvalOptions}\,\M{family}\\
%   \cs{ProcessKeyvalOptions} \*
% \end{declcs}
% This command evaluates the global or local options of
% the package that are defined with \xpackage{keyval}'s
% interface within the family \meta{family}. It acts the
% same way as \LaTeX's \cs{ProcessOptions*}. In a package
% unknown global options are ignored, in a class they
% are added to the unknown option list. The known global
% options and all local options are passed to \xpackage{keyval}'s
% \cs{setkeys} command for executing the options.
% Unknown options are reported to the user by an error.
%
% If the family name happens to be the same as the name
% of the package or class where \cs{ProcessKeyvalOptions}
% is used or the family name has previously been setup by
% \cs{SetupKeyvalOptions}, then \cs{ProcessKeyvalOptions}
% knows the family name already and you can use the star form
% without mandatory argument.
%
% \subsubsection{\cs{ProcessLocalKeyvalOptions}}
%
% \begin{declcs}{ProcessLocalKeyvalOptions}\,\M{family}\\
%   \cs{ProcessLocalKeyvalOptions} \*
% \end{declcs}
% This macro has the same syntax and works similar as
% \cs{ProcessKeyvalOptions}. However it ignores
% global options and only processes the local package options.
% Therefore it only can be used inside a package.
% An error is thrown, if it is used inside a class.
%
% Neither of the following macros are necessary for
% \cs{ProcessKeyvalOptions{}}. They just help the package/class
% author in common tasks.
%
% \subsubsection{\cs{SetupKeyvalOptions}}\label{setup}
%
% \begin{declcs}{SetupKeyvalOptions}\,\texttt{\{}\\
%   \hspace{1.5em}|family|\,|=|\,\meta{family}|,|\\
%   \hspace{1.5em}|prefix|\,|=|\,\meta{prefix}\\
%   \hspace{1.5em}|setkeys|\,|=|\,\meta{setkeys command}\\
%   \texttt{\}}
% \end{declcs}
% This command allows to configure the default assumptions
% that are based on the current package or class name.
% \LaTeX\ remembers this name in \cs{@currname}. The syntax
% description of the default looks a little weird, therefor
% an example is given for a package or class named |foobar|.
% \begin{quote}
% \begin{tabular}{@{}l|ll|l@{}}
%   Key & Default & (example) & Used by\\
% \hline
%   \rule{0pt}{1.2\ht\csname @arstrutbox\endcsname}^^A
%   |family| & \meta{\cs{@currname}} & (|foobar|)
%       & \cs{ProcessKeyvalOptions*} \\
%     &&& \cs{DeclareBoolOption} \\
%     &&& \cs{DeclareStringOption} \\
% \hline
%   \rule{0pt}{1.2\ht\csname @arstrutbox\endcsname}^^A
%   |prefix| & \meta{\cs{@currname}}|@| & (|foobar@|)
%       & \cs{DeclareBoolOption} \\
%     &&& \cs{DeclareStringOption} \\
%     &&& \cs{DeclareVoidOption} \\
% \hline
%   \rule{0pt}{1.2\ht\csname @arstrutbox\endcsname}^^A
%   |setkeys| & \cs{setkeys} & (\cs{kvsetkeys})
%       & \cs{ProcessKeyvalOptions} \\
%     &&& \cs{ProcessLocalKeyvalOptions}
% \end{tabular}
% \end{quote}
% Key |setkeys| was added in version 3.9. The original
% \cs{setkeys} of package \xpackage{keyval} is not reentrant.
% If an option is processed by this \cs{setkeys}, then
% the option should not call \cs{setkeys} again with a
% different family. Otherwise the next options of the
% first \cs{setkeys} call are processed with the wrong
% family. With key |setkeys| the macro \cs{kvsetkeys}
% can be set that does not have the problem of the
% original \cs{setkeys} of package {keyval}.
%
% Probably \cs{setkeys} of package \xpackage{xkeyval}
% is safe in this respect. But I haven't made a full
% analysis. At least it does not have the problem
% of the original \cs{setkeys}.
%
% \subsection{Option declarations}
%
% The options for \cs{ProcessKeyvalOptions} are defined
% by \xpackage{keyval}'s \cs{define@key}.
% Common purposes of such keys are boolean switches, they
% enable or disable something. Or they store a name or some kind of
% string in a macro. The following commands help the user. He
% declares what he wants and \xpackage{kvoptions} take care of
% the key definition, resource allocation and initialization.
%
% In order to avoid name clashes of macro names, internal
% commands are prefixed. Both the prefix and the family
% name for the defined keys can be configured by
% \cs{SetupKeyvalOptions}.
%
% \subsubsection{\cs{DeclareStringOption}}
%
% \begin{declcs}{DeclareStringOption}\,\O{init}\,\M{key}\,\O{default}
% \end{declcs}
% A macro is created that remembers the value of the key \meta{key}.
% The name of the macro consists of the option name \meta{key} that
% is prefixed by the prefix (see \ref{setup}).
% The initial contents of the
% macro can be given by the first optional argument \meta{init}.
% The default is empty.
%
% The the option \meta{key} is defined. The option code
% just stores its value in the macro. If the optional argument
% at the end of \cs{DeclareStringOption} is given, then option
% \meta{key} is defined with the default \meta{default}.
%
% Example for a package with the following two lines:
% \begin{quote}
% |\ProvidesPackage{foobar}|\\
% |\DeclareStringOption[me]{name}|
% \end{quote}
% Then \cs{DeclareStringOption} defines the macro with content |me|,
% note \LaTeX\ complains if the name of the macro already exists:
% \begin{quote}
% |\newcommand*{\foobar@name}{me}|
% \end{quote}
% The option definition is similar to:
% \begin{quote}
% |\define@key{foobar}{name}{%|\\
% |  \renewcommand*{\foobar@name}{#1}%|\\
% |}|
% \end{quote}
%
% \subsubsection{\cs{DeclareBoolOption}}
%
% \begin{declcs}{DeclareBoolOption}\,\O{init}\,\M{key}
% \end{declcs}
% A boolean switch is generated, initialized by value \meta{init}
% and the corresponding key \meta{key} is defined. If the
% initialization value is not given, |false| is used as default.
%
% The internal actions of \cs{DeclareBoolOption} are shown below.
% The example is given for a package author who has the following
% two lines in his package/class:
% \begin{quote}
% |\ProvidesPackage{foobar}|\\
% |\DeclareBoolOption{verbose}|
% \end{quote}
% First a new switch is created:
% \begin{quote}
%   |\newif\iffoobar@verbose|
% \end{quote}
% and initialized:
% \begin{quote}
%   |\foobar@verbosefalse|
% \end{quote}
% Finally the key is defined:
% \begin{quote}
%   |\define@key{foobar}{verbose}[true]{|\dots|}|
% \end{quote}
% The option code configures the boolean option in the following way:
% If the author specifies |true| or |false| then the switch is
% turned on or off respectivly. Also the option can be given without
% explicit value. Then the switch is enabled.
% Other values are reported as errors.
%
% Now the switch is ready to use in the package/class, e.g.:
% \begin{quote}
%   |\iffoobar@verbose|\\
%   \qquad\textit{\% print verbose message}\\
%   |\else|\\
%   \qquad\textit{\% be quiet}\\
%   |\fi|
% \end{quote}
% Users of package \cs{ifthen} can use the switch as
% boolean:
% \begin{quote}
%   |\boolean{foobar@verbose}|
% \end{quote}
%
% \subsubsection{\cs{DeclareComplementaryOption}}
%
% \begin{declcs}{DeclareComplementaryOption}\,\M{key}\,\M{parent}
% \end{declcs}
% Sometimes contrasting names are used to characterize the
% two states of a boolean switch, for example \xoption{draft}
% vs.\@ \xoption{final}. Both options behave like boolean
% options but they do not need two different switches, they
% should share one. \cs{DeclareComplementaryOption} allows this.
% The option \meta{key} shares the switch of option \meta{parent}.
% Example:
% \begin{quote}
%   |\DeclareBoolOption{draft}|\\
%   |\DeclareComplementaryOption{final}{draft}|
% \end{quote}
% Then |final| sets the switch of \xoption{draft} to |false|, and
% |final=false| enables the \xoption{draft} switch.
%
% \subsubsection{\cs{DeclareVoidOption}}
%
% \begin{declcs}{DeclareVoidOption}\,\M{key}\,\M{code}
% \end{declcs}
% \cs{ProcessKeyvalOptions} can be extended to recognize
% options that are declared in traditional way by \cs{DeclareOption}.
% But in case of the error that the user specifies a value, then
% this option would not recognized as key value option because
% of \cs{DeclareOption} and not detected as traditional option
% because of the value part. The user would get an unknown option
% error, difficult to understand.
%
% \cs{DeclareVoidOption} solves
% this problem. It defines the option \meta{key} as key value option.
% If the user specifies a value, a warning is given and the
% value is ignored.
%
% The code part \meta{code} is stored in a macro.
% The name of the macro consists of the option name \meta{key} that
% is prefixed by the prefix (see \ref{setup}).
% If the option is set, the macro will be executed. During the
% execution \cs{CurrentOption} is available with the current key name.
%
% \subsubsection{\cs{DeclareDefaultOption}}
%
% \begin{declcs}{DeclareDefaultOption}\,\M{code}
% \end{declcs}
% \begin{sloppypar}
% This command does not define a specific key, it is the equivalent
% to \LaTeX's \cs{DeclareOption*}. It allows the specification
% of a default action \meta{code} that is invoked if an unknown option
% is found.
% While \meta{code} is called,
% macro \cs{CurrentOption} contains the current option string.
% In addition \cs{CurrentOptionValue} contains the value part
% if the option string is parsable as key value pair, otherwise
% it is \cs{relax}. \cs{CurrentOptionKey} contains the key of
% the key value pair, or the whole option string, if it misses
% the equal sign.
% \end{sloppypar}
%
% Inside packages typical default actions are to pass unknown
% options to another package. Or an error message can be thrown
% by \cs{@unknownoptionerror}. This is the original error
% message that \LaTeX\ gives for unkown package options.
% This error message is easier to understand for the user
% as the error message from package \xpackage{keyval} that
% is given otherwise.
%
% A Class ignores unknown options and puts them on the
% unused option list. Let \LaTeX\ do the job and just
% call \cs{OptionNotUsed}. Or the options can be passed to
% another class that is later loaded.
%
% \subsubsection{Local options}
%
% \begin{declcs}{DeclareLocalOption} \M{option}\\
%   \cs{DeclareLocalOptions} \M{option list}
% \end{declcs}
% Both macros mark package options as local options. That means
% that they are ignored by \cs{ProcessKeyvalOptions} if they are given
% as global options. \cs{DeclareLocalOptions} takes one option,
% \cs{DeclareLocalOptions} expects a comma separated list of options.
%
% \subsubsection{Dynamic options}
%
% Options of \LaTeX's package/class system are cleared
% in \cs{ProcessOptions}. They modify the static model
% of a package. For example, depending on option \xoption{bookmarks}
% package \xpackage{hyperref} loads differently.
%
% Options, however,
% defined by \xpackage{keyval}'s \cs{define@key} remain
% defined, if the options are processed by \cs{setkeys}.
% Therefore these options can also be used to model
% the dynamic behaviour of a package. For example, in
% \xpackage{hyperref} the link colors can be changed
% everywhere until the end in |\end{document}|.
%
% However package \xpackage{color} that adds color support
% is necessary and it
% cannot be loaded after |\begin{document}|. Option \xoption{colorlinks}
% that loads \xpackage{color} should be active until |\begin{document}|
% and die in some way if it is too late for loading packages.
% With \cs{DisableKeyvalOption} the package/class author can
% specify and configure the death of an option and controls
% the life period of the option.
%
% \subsubsection{\cs{DisableKeyvalOption}}
%
% \begin{declcs}{DisableKeyvalOption}\,\O{options}\,\M{family}\,\M{key}
%   \\[0.5ex]
%   \meta{options}:\\
%   \hspace{1.5em}^^A
%   \begin{tabular}[t]{@{}l@{ \texttt{=} }l@{\qquad}>{default: }l@{}}
%     |action| & |undef|, |warning|, |error|, or |ignore|
%     & |undef|
%     \\
%     \multicolumn{1}{@{}l}{|global| or |local|} & & |global|
%     \\
%     |package| or |class| & \meta{name}
%     \\
%   \end{tabular}^^A
% \end{declcs}
% \cs{DisableKeyvalOption} can be called to mark the end when
% the option \meta{key} is no longer useful. The behaviour of
% an option after its death can be configured by \xoption{action}:
% \begin{description}
% \item[\texttt{undef}:] The option will be undefined,
%   If it is called, \cs{setkeys} reports an error because
%   of unknown key.
% \item[\texttt{error} or \texttt{warning}:] Use of the
%  option will cause an error or warning message.
%  Also these actions require that exclusivly either the
%  package or class name is given in options \xoption{package}
%  or \xoption{class}.
% \item[\texttt{ignore}:] The use of the option will silently
%  be ignored.
% \end{description}
% The option's death can be limited to the end of the current
% group, if option \xoption{local} is given. Default is
% \xoption{global}.
%
% The package/class author can wish the end of the option
% already during the package loading, then he will have
% static behaviour. In case of dynamic options \cs{DisableKeyvalOption}
% can be executed everywhere, also outside the package. Therefore
% the family name and the package/class name is usually unknown for
% \cs{DisableKeyvalOption}. Therefore the argument for the family
% name is mandatory and for some actions the package/class name
% must be provided.
%
% Usually a macro would configure the option death, Example:
%\begin{quote}
%\begin{verbatim}
%\ProvidesPackage{foobar}
%\DeclareBoolOption{color}
%\DeclareStringOption[red]{emphcolor}
%\ProcessKeyvalOptions*
%
%\newcommand*{\foobar@DisableOption}[2]{%
%  \DisableKeyvalueOption[
%    action={#1},
%    package=foobar
%  ]{foobar}{#2}%
%}
%
%\iffoobar@color
%  \RequirePackage{color}
%  \renewcommand*{\emph}[1]{\textcolor{\foobar@emphcolor}{#1}}
%\else
%  % Option emphcolor is not wrong, if we have color support.
%  % otherwise the option has no effect, but we don't want to
%  % remove it. Therefore action 'ignore' is the best choice:
%  \foobar@DisableOption{ignore}{emphcolor}
%\fi
%% No we don't need the option 'color'.
%\foobar@DisableOption{warning}{color}
%
%% With color support option 'emphcolor' will dynamically
%% change the color of \emph statements.
%\end{verbatim}
%\end{quote}
%
% \subsubsection{\cs{AddToKeyvalOption}}
%
% \begin{declcs}{AddToKeyvalOption}\,\M{family}\,\M{key}\,\M{code}\\
%   \cs{AddToKeyvalOption} \*\,\M{key}\,\M{code}
% \end{declcs}
% The code for an existing key \meta{key} of family \meta{family} is
% extended by code \meta{code}. In the starred form the current
% family setting is used, see \cs{ProcessKeyvalOptions*}.
%
% \subsection{Global vs. local options}
%
% Options that are given for \cs{documentclass} are called
% global options. They are known to the class and all packages.
% A package may make use of a global option and marks it as
% used. The advantage for the user is the freedom to specify
% options both in the \cs{documentclass} or \cs{usepackage}
% commands.
%
% However global options are shared with the class options
% and options of all other packages. Thus there can be the
% same option with different semantics for different
% packages and classes. As example, package \xpackage{bookmark}
% knows option \xoption{open} that specifies whether the bookmarks
% are opened or closed initially. It's values are |true| or |false|.
% Since KOMA-Script version 3.00 the KOMA classes also introduces
% option \xoption{open} with values |right| and |any| and a
% complete different meaning.
%
% Such conflicts can be resolved by marking all or part of options
% as local by \cs{DeclareLocalOption} or \cs{DeclareLocalOptions}.
% Then the packages ignores global occurences of these options.
% Package \xpackage{kvoptions} provides two methods:
% \begin{itemize}
% \item \cs{ProcessLocalKeyvalOptions} automatically uses all options
%   as local options. It ignores all global options.
% \item \cs{DeclareLocalOption} or \cs{DeclareLocalOptions} marks
%   options as local options. \cs{ProcessKeyvalOptions} will then
%   ignore global occurences for these local options.
% \end{itemize}
% Since version 1.5 package \xpackage{bookmark} uses the latter
% method. It checks global and local option places for driver options
% and limits all other options as local options. Thus the class
% option \xoption{open} of KOMA-Script is not misread as
% option for package \xpackage{bookmark}.
%
% \subsection{Summary of internal macros}
%
% The \cs{Declare}\texttt{\dots Option} commands define
% macros, additionally to the macros generated by the key definition.
% These macros can be used by the package/class author.
% The name of the macros starts with the prefix \meta{prefix}
% that can be configured by \cs{SetupKeyvalOptions}.
% \begin{center}
% \def\prefix{\cs{}\meta{prefix}\meta{key}}^^A
% \def\Hline{^^A
%   \hline
%   \rule{0pt}{1.2\ht\csname @arstrutbox\endcsname}^^A
% }^^A
% \begin{tabular}{@{}l|>{\ttfamily}l|l@{}}
%   Declare \meta{key} & \textrm{Defined macro} & Description\\
% \Hline
%   \cs{DeclareStringOption}
%    & \prefix & holds the string\\
% \Hline
%   \cs{DeclareBoolOption}
%     & \cs{if}\meta{prefix}\meta{key}& boolean switch\\
%   &\prefix false& disable switch\\
%   &\prefix true& enable switch\\
% \Hline
%   \cs{DeclareComplementaryOption}
%     & \prefix false& enable parent switch\\
%     & \prefix true& disable parent switch\\
% \Hline
%   \cs{DeclareVoidOption}
%   & \prefix & holds the action
% \end{tabular}
% \end{center}
%
% \subsection{\plainTeX}
%
% Package \xpackage{keyval} is also usable in \plainTeX\ with
% the help of file \xfile{miniltx.tex}. Some features of this
% package \xpackage{kvoptions} might also be useful for \plainTeX.
% If \LaTeX\ is not found, \cs{ProcessKeyvalOptions} and option
% \xoption{patch} are disabled. Before using the option declaration
% commands \cs{Declare...Option}, \cs{SetupKeyvalOptions} must be
% used.
%
% \hypersetup{bookmarksopenlevel=1}
%
% \section{Example}
%
% The following example defined a package that serves some
% private color management. A boolean option \xoption{print} enables
% print mode without colors. An option \xoption{emph} redefines
% \cs{emph} to print in the given color. And the driver
% can be specified by option \xoption{driver}.
%
%    \begin{macrocode}
%<*example>
    % Package identification
    % -----------------------
\NeedsTeXFormat{LaTeX2e}
\ProvidesPackage{example-mycolorsetup}[2016/05/16 Managing my colors]

\RequirePackage{ifpdf}
\RequirePackage{kvoptions}

    % Option declarations
    % -------------------

\SetupKeyvalOptions{
  family=MCS,
  prefix=MCS@
}
    % Use a shorter family name and prefix

    % Option print
\DeclareBoolOption{print}
    % is the same as
    % \DeclareBoolOption[false]{print}

    % Option driver
\ifpdf
  \DeclareStringOption[pdftex]{driver}
\else
  \DeclareStringOption[dvips]{driver}
\fi

    % Alternative interface for driver options
\DeclareVoidOption{dvips}{\SetupDriver}
\DeclareVoidOption{dvipdfm}{\SetupDriver}
\DeclareVoidOption{pdftex}{\SetupDriver}
    % In \SetupDriver we take the current option \CurrentOption
    % and pass it to the driver option.
    % The \expandafter commands expand \CurrentOption at the
    % time, when \SetupDriver is executed and \CurrentOption
    % has the correct meaning.
\newcommand*{\SetupDriver}{%
  \expandafter\@SetupDriver\expandafter{\CurrentOption}%
}
\newcommand*{\@SetupDriver}[1]{%
  \setkeys{MCS}{driver={#1}}%
}

    % Option emph
    % An empty value means, we want to have no color for \emph.
    % If the user specifies option emph without value, the red is used.
\DeclareStringOption{emph}[red]
    % is the same as
    % \DeclareStringOption[]{emph}[red]

    % Default option rule
\DeclareDefaultOption{%
  \ifx\CurrentOptionValue\relax
    \PackageWarningNoLine{\@currname}{%
      Unknown option `\CurrentOption'\MessageBreak
      is passed to package `color'%
    }%
    % Pass the option to package color.
    % Again it is better to expand \CurrentOption.
    \expandafter\PassOptionsToPackage
    \expandafter{\CurrentOption}{color}%
  \else
    % Package color does not take options with values.
    % We provide the standard LaTeX error.
    \@unknownoptionerror
  \fi
}

    % Process options
    % ---------------
\ProcessKeyvalOptions*

    % Implementation depending on option values
    % -----------------------------------------
    % Code for print mode
\ifMCS@print
  \PassOptionsToPackage{monochrome}{color}
    % tells package color to use black and white
\fi

\RequirePackage[\MCS@driver]{color}
    % load package color with the correct driver

    % \emph setup
\ifx\MCS@emph\@empty
    % \@empty is a predefined macro with empty contents.
    % the option value of option emph is empty, thus
    % we do not want a redefinition of \emph.
\else
  \renewcommand*{\emph}[1]{%
    \textcolor{\MCS@emph}{#1}%
  }
\fi
%</example>
%    \end{macrocode}
%
% \section{Package options}
%
% The package \xpackage{kvoptions} knows two package options
% \xoption{patch} and \xoption{debugshow}.
% The options of package \xpackage{kvoptions} are intended
% for authors, not for package/class writers. Inside a package
% it is too late for option \xoption{patch} and \xoption{debugshow}
% enables some messages that are perhaps useful for the debugging
% phase. Also \LaTeX\ is unhappy if a package is loaded
% later again with options that are previously not given.
% Thus package and class authors, stay with |\RequirePackage{kvoptions}|
% without options.
%
% Option \xoption{patch} loads package \xpackage{kvoptions-patch}.
%
% \subsection{Package \xpackage{kvoptions-patch}}
%
% \LaTeX's system of package/class options has some severe limitations
% that especially affects the value part if options are used as
% pair of key and value.
% \begin{itemize}
% \item Spaces are removed, regardless where:
%   \begin{quote}
%     |\documentclass[box=0 0 400 600]{article}|
%   \end{quote}
%   Now each package will see |box=00400600| as global option.
% \item In the previous case also braces would not help:
%   \begin{quote}
%     |\documentclass[box={0 0 400 600}]{article}|
%   \end{quote}
%   The result is an error message:
%   \begin{quote}
%     |! LaTeX Error: Missing \begin{document}.|
%   \end{quote}
%   As local option, however, it works if the package
%   knows about key value options (By using this package, for example).
% \item The requirements on robustness are extremly high.
%   \LaTeX\ expands the option. All that will not work as environment
%   name will break also as option. Even a \cs{relax} will generate
%   an error message:
%   \begin{quote}
%     |! Missing \endcsname inserted.|
%   \end{quote}
%   Of course, \LaTeX\ does not use its protecting mechanisms.
%   On contrary \cs{protect} itself will cause errors.
% \item The options are expanded. But perhaps the package will
%   do that, because it has to setup some things before?
%   Example \xpackage{hyperref}:
%   \begin{quote}
%     |\usepackage[pdfauthor=M\"uller]{hyperref}|
%   \end{quote}
%   Package \xpackage{hyperref} does not see |M\"uller| but
%   its expansion and it does not like it, you get many warnings
%   \begin{quote}
%     |Token not allowed in a PDFDocEncoded string|
%   \end{quote}
%   And the title becomes: |Mu127uller|.
%   Therefore such options must usually be given after package
%   \xpackage{hyperref}
%   is loaded:
%   \begin{quote}
%     |\usepackage{hyperref}|\\
%     |\hypersetup{pdfauthor=Fran\c coise M\"uller}|
%   \end{quote}
%   As package option it will even break with |Fran\c coise|
%   because of the cedilla |\c c|, it is not robust enough.
% \end{itemize}
% For users that do not want with this limitations the package
% offers package \xoption{kvoptions-patch}. It patches \LaTeX's option system
% and tries to teach it also to handle options that are given
% as pairs of key and value and to prevent expansion.
% It can already be used at the very beginning, before \cs{documentclass}:
% \begin{quote}
%   |\RequirePackage{kvoptions-patch}|\\
%   |\documentclass[pdfauthor=Fran\c coise M\"uller]{article}|\\
%   |\usepackage{hyperref}|
% \end{quote}
% The latest time is before the package where you want to use
% problematic values:
% \begin{quote}
%   |\usepackage{kvoptions-patch}|\\
%   |\usepackage[Fran\c coise M\"uller]{hyperref}|
% \end{quote}
% Some remarks:
% \begin{itemize}
% \item The patch requires \eTeX, its \cs{unexpanded} feature
%   is much too nice. It is possible to work around using token
%   registers. But the code becomes longer, slower, more difficult
%   to read and maintain. The package without option \xoption{patch}
%   works and will work without \eTeX.
% \item The code for the patch is quite long, there are many test
%   cases. Thus the probability for bugs is probably not too small.
% \item Since 2008/10/18 v3.0 package \xpackage{kvoptions-patch} is
%   available. Before option \xoption{patch} of package \xpackage{kvoptions}
%   must be used instead. I think, the solution as standalone package
%   \xpackage{kvoptions-patch} is cleaner and avoids option clashes.
% \end{itemize}
%
% \subsection{Option \xoption{debugshow}}
%
% The name of this option follows the convention of packages
% \xpackage{multicol}, \xpackage{tabularx}, and \xpackage{tracefnt}.
% Currently it prints the setting of boolean options, declared
% by \cs{DeclareBoolOption} in the \xfile{.log} file, if that
% boolean option is used.
% You can activate the option by
% \begin{itemize}
% \item |\PassOptionsToPackage{debugshow}{kvoptions}|\\
%       Put this somewhere before package \xpackage{kvoptions} is
%       loaded first, e.g. before \cs{documentclass}.
% \item |\RequirePackage[debugshow]{kvoptions}|\\
%       Before \cs{documentclass} even an author has to use
%       \cs{RequirePackage}. \cs{usepackage} only works after
%       \cs{documentclass}.
% \end{itemize}
% The preferred method is \cs{PassOptionsToPackage}, because
% it does not force the package loading and does not disturb,
% if the package is not loaded later at all.
%
% \section{Limitations}
%
% \subsection{Compatibility}
% \subsubsection{Package \xpackage{kvoptions-patch} vs. package \xpackage{xkvltxp}}
%
% Package \xpackage{xkvltxp} from the \xpackage{xkeyval} project
% has the same goal as package \xpackage{kvoptions-patch}
% and to patch \LaTeX's kernel commands in order to get better support
% for key value options.
% Of course they cannot be used both. The user must decide, which
% method he prefers. Package \xpackage{kvoptions-patch} aborts itself,
% if it detects that \xpackage{xkvltxp} is
% already loaded.
%
% However package \xpackage{xkvltxp} and \xpackage{kvoptions}
% can be used together, example:
% \begin{quote}
% |\usepackage{xkvltxp}|\\
% |\usepackage[...]{foobar} % |\texttt{\itshape foobar using kvoptions}
% \end{quote}
% The other way should work, too.
%
% Package \xpackage{kvoptions-patch} tries to catch more situations
% and to be more robust.
% For example, during the comparison of options it normalizes
% them by removing spaces around |=| and the value. Thus the
% following is not reported as option clash:
% \begin{quote}
%\begin{verbatim}
%\RequirePackage{kvoptions-patch}
%\documentclass{article}
%
%\usepackage[scaled=0.7]{helvet}
%\usepackage[scaled = 0.7]{helvet}
%
%\begin{document}
%\end{document}
%\end{verbatim}
% \end{quote}
%
% \subsection{Limitations}
%
% \subsubsection{Option comparisons}
%
% In some situations \LaTeX\ compares option lists, e.g. option clash check,
% \cs{@ifpackagewith}, or \cs{@ifclasswith}. Apart from catcode and
% sanitizing problems of option \xoption{patch}, there is another problem.
% \LaTeX\ does not know about the type and default values of
% options in key value style. Thus an option clash is reported,
% even if the key value has the same meaning:
% \begin{quote}
% |\usepackage[scaled]{helvet} %| \texttt{\textit{default is }.95}\\
% |\usepackage[.95]{helvet}|\\
% |\usepackage[0.95]{helvet}|
% \end{quote}
%
% \subsubsection{Option list parsing with package \xpackage{kvoptions-patch}}
%
% With package \xpackage{kvoptions-patch} the range of possible values
% in key value specifications is much large, for example
% the comma can be used, if enclosed in curly braces.
%
% Other packages, especially the packages that uses their
% own process option code can be surprised to find tokens
% inside options that they do not expect and errors would
% be the consequence. To avoid errors the options, especially
% the unused option list is sanitized. That means the list
% will only contain tokens with catcode 12 (other) and
% perhaps spaces (catcode 10). This allows a safe parsing
% for other packages. But a comma in the value part is
% no longer protected by curly braces because they have lost
% their special meaning. This is the price for compatibility.
%
% Example:
% \begin{quote}
% |\RequirePackage{kvoptions-patch}|\\
% |\documentclass[a={a,b,c},b]{article}|\\
% |\begin{document}|\\
% |\end{document}|
% \end{quote}
% Result:
% \begin{quote}
% |LaTeX Warning: Unused global option(s):|\\
% |    [a={a,c},b].|
% \end{quote}
%
% \StopEventually{
% }
%
% \section{Implementation}
%
% \subsection{Preamble}
%
%    \begin{macrocode}
%<*package>
%    \end{macrocode}
%
% \paragraph{Reload check and  identification.}
%    Reload check, especially if the package is not used with \LaTeX.
%    \begin{macrocode}
\begingroup\catcode61\catcode48\catcode32=10\relax%
  \catcode13=5 % ^^M
  \endlinechar=13 %
  \catcode35=6 % #
  \catcode39=12 % '
  \catcode44=12 % ,
  \catcode45=12 % -
  \catcode46=12 % .
  \catcode58=12 % :
  \catcode64=11 % @
  \catcode123=1 % {
  \catcode125=2 % }
  \expandafter\let\expandafter\x\csname ver@kvoptions.sty\endcsname
  \ifx\x\relax % plain-TeX, first loading
  \else
    \def\empty{}%
    \ifx\x\empty % LaTeX, first loading,
      % variable is initialized, but \ProvidesPackage not yet seen
    \else
      \expandafter\ifx\csname PackageInfo\endcsname\relax
        \def\x#1#2{%
          \immediate\write-1{Package #1 Info: #2.}%
        }%
      \else
        \def\x#1#2{\PackageInfo{#1}{#2, stopped}}%
      \fi
      \x{kvoptions}{The package is already loaded}%
      \aftergroup\endinput
    \fi
  \fi
\endgroup%
%    \end{macrocode}
%    Package identification:
%    \begin{macrocode}
\begingroup\catcode61\catcode48\catcode32=10\relax%
  \catcode13=5 % ^^M
  \endlinechar=13 %
  \catcode35=6 % #
  \catcode39=12 % '
  \catcode40=12 % (
  \catcode41=12 % )
  \catcode44=12 % ,
  \catcode45=12 % -
  \catcode46=12 % .
  \catcode47=12 % /
  \catcode58=12 % :
  \catcode64=11 % @
  \catcode91=12 % [
  \catcode93=12 % ]
  \catcode123=1 % {
  \catcode125=2 % }
  \expandafter\ifx\csname ProvidesPackage\endcsname\relax
    \def\x#1#2#3[#4]{\endgroup
      \immediate\write-1{Package: #3 #4}%
      \xdef#1{#4}%
    }%
  \else
    \def\x#1#2[#3]{\endgroup
      #2[{#3}]%
      \ifx#1\@undefined
        \xdef#1{#3}%
      \fi
      \ifx#1\relax
        \xdef#1{#3}%
      \fi
    }%
  \fi
\expandafter\x\csname ver@kvoptions.sty\endcsname
\ProvidesPackage{kvoptions}%
  [2016/05/16 v3.12 Key value format for package options (HO)]%
%    \end{macrocode}
%
% \paragraph{Catcodes}
%
%    \begin{macrocode}
\begingroup\catcode61\catcode48\catcode32=10\relax%
  \catcode13=5 % ^^M
  \endlinechar=13 %
  \catcode123=1 % {
  \catcode125=2 % }
  \catcode64=11 % @
  \def\x{\endgroup
    \expandafter\edef\csname KVO@AtEnd\endcsname{%
      \endlinechar=\the\endlinechar\relax
      \catcode13=\the\catcode13\relax
      \catcode32=\the\catcode32\relax
      \catcode35=\the\catcode35\relax
      \catcode61=\the\catcode61\relax
      \catcode64=\the\catcode64\relax
      \catcode123=\the\catcode123\relax
      \catcode125=\the\catcode125\relax
    }%
  }%
\x\catcode61\catcode48\catcode32=10\relax%
\catcode13=5 % ^^M
\endlinechar=13 %
\catcode35=6 % #
\catcode64=11 % @
\catcode123=1 % {
\catcode125=2 % }
\def\TMP@EnsureCode#1#2{%
  \edef\KVO@AtEnd{%
    \KVO@AtEnd
    \catcode#1=\the\catcode#1\relax
  }%
  \catcode#1=#2\relax
}
\TMP@EnsureCode{1}{14}% ^^A (comment)
\TMP@EnsureCode{2}{14}% ^^A (comment)
\TMP@EnsureCode{33}{12}% !
\TMP@EnsureCode{39}{12}% '
\TMP@EnsureCode{40}{12}% (
\TMP@EnsureCode{41}{12}% )
\TMP@EnsureCode{42}{12}% *
\TMP@EnsureCode{44}{12}% ,
\TMP@EnsureCode{45}{12}% -
\TMP@EnsureCode{46}{12}% .
\TMP@EnsureCode{47}{12}% /
\TMP@EnsureCode{58}{12}% :
\TMP@EnsureCode{62}{12}% >
\TMP@EnsureCode{91}{12}% [
\TMP@EnsureCode{93}{12}% ]
\TMP@EnsureCode{94}{7}% ^ (superscript)
\TMP@EnsureCode{96}{12}% `
\edef\KVO@AtEnd{\KVO@AtEnd\noexpand\endinput}
%    \end{macrocode}
%
% \paragraph{External resources.}
%    The package extends the support for key value pairs of
%    package \cs{keyval} to package options. Thus the package
%    needs to be loaded anyway. and we use it for
%    \cs{SetupKeyvalOptions}. AFAIK this does not disturb
%    users of \xpackage{xkeyval}.
%    \begin{macrocode}
\@ifundefined{define@key}{%
  \RequirePackage{keyval}\relax
}{}
%    \end{macrocode}
%
%    Macro \cs{DeclareLocalOptions} parses a comma separated key list
%    and uses \cs{comma@parse} of package \xpackage{kvsetkeys},
%    version 1.3.
%    \begin{macrocode}
\RequirePackage{ltxcmds}[2010/12/02]
\RequirePackage{kvsetkeys}[2007/09/29]
%    \end{macrocode}
%
% \paragraph{Provide macros for \plainTeX.}
%    \begin{macrocode}
\@ifundefined{@x@protect}{%
  \def\@x@protect#1\fi#2#3{%
    \fi\protect#1%
  }%
  \let\@typeset@protect\relax
}{}
\@ifundefined{@currname}{%
  \def\@currname{}%
}{}
\@ifundefined{@currext}{%
  \def\@currext{}%
}{}
%    \end{macrocode}
%
% \paragraph{Options}
%    Option \xoption{debugshow} enables additional lines of
%    code that prints information into the \xfile{.log}
%    file.
%    \begin{macrocode}
\DeclareOption{debugshow}{\catcode\@ne=9 }
%    \end{macrocode}
%
%    \begin{macrocode}
\DeclareOption{patch}{%
  \AtEndOfPackage{%
    \RequirePackage{kvoptions-patch}[2016/05/16]%
  }%
}
%    \end{macrocode}
%
%    Optionen auswerten:
%    \begin{macrocode}
\ProcessOptions\relax
%    \end{macrocode}
%
% \subsection{Option declaration macros}
%
% \subsubsection{\cs{SetupKeyvalOptions}}
%
%    The family for the key value pairs can be setup once
%    and is remembered later.
%    The package name seems a reasonable default for the
%    family key, if it is not set by the package author.
%
%    \begin{macro}{\KVO@family}
%    We cannot store the family setting in one macro,
%    because the package should be usable for many
%    other packages, too. Thus we remember the family
%    setting in a macro, whose name contains the
%    package name with extension, a key in \LaTeX's
%    class/package system.
%    \begin{macrocode}
\define@key{KVO}{family}{%
  \expandafter\edef\csname KVO@family@%
      \@currname.\@currext\endcsname{#1}%
}
\def\KVO@family{%
  \@ifundefined{KVO@family@\@currname.\@currext}{%
    \@currname
  }{%
    \csname KVO@family@\@currname.\@currext\endcsname
  }%
}
%    \end{macrocode}
%    \end{macro}
%
%    \begin{macro}{\KVO@prefix}
%    The value settings of options that are declared by \cs{DeclareBoolOption}
%    and \cs{DeclareStringOption} need to be saved in macros.
%    in the first case this is a switch \cs{if}\meta{prefix}\meta{key},
%    in the latter case a macro \cs{}\meta{prefix}\meta{key}.
%    The prefix can be configured, by \xoption{prefix} that is declared
%    here. The default is the package name with |@| appended.
%    \begin{macrocode}
\define@key{KVO}{prefix}{%
  \expandafter\edef\csname KVO@prefix@%
      \@currname.\@currext\endcsname{#1}%
}
\def\KVO@prefix{%
  \ltx@ifundefined{KVO@prefix@\@currname.\@currext}{%
    \@currname @%
  }{%
    \csname KVO@prefix@\@currname.\@currext\endcsname
  }%
}
%    \end{macrocode}
%    \end{macro}
%
%    \begin{macrocode}
\define@key{KVO}{setkeys}{%
  \expandafter\def\csname KVO@setkeys@%
      \@currname.\@currext\endcsname{#1}%
}
%    \end{macrocode}
%    \begin{macro}{\KVO@setkeys}
%    \begin{macrocode}
\def\KVO@setkeys{%
  \ltx@IfUndefined{KVO@setkeys@\@currname.\@currext}{%
    \setkeys
  }{%
    \csname KVO@setkeys@\@currname.\@currext\endcsname
  }%
}
%    \end{macrocode}
%    \end{macro}
%
%    \begin{macro}{\SetupKeyvalOptions}
%    The argument of \cs{SetupKeyvalOptions} expects a key value
%    list, known keys are \xoption{family} and \xoption{prefix}.
%    \begin{macrocode}
\newcommand*{\SetupKeyvalOptions}{%
  \kvsetkeys{KVO}%
}
%    \end{macrocode}
%    \end{macro}
%
% \subsubsection{\cs{DeclareBoolOption}}
%
%    \begin{macro}{\DeclareBoolOption}
%    Usually options of boolean type can be given by the
%    user without value and this means a setting to \emph{true}.
%    We follow this convention here. Also it simplifies the
%    user interface.
%
%    The switch is created and initialized with \emph{false}.
%    The default setting can be overwritten by the optional
%    argument.
%
%    \LaTeX's \cs{newif} does not check for already defined
%    macros, therefore we add this check here to prevent
%    the user from accidently redefining of \TeX's primitives
%    and other macros.
%    \begin{macrocode}
\newcommand*{\DeclareBoolOption}[2][false]{%
  \KVO@ifdefinable{if\KVO@prefix#2}{%
    \KVO@ifdefinable{\KVO@prefix#2true}{%
      \KVO@ifdefinable{\KVO@prefix#2false}{%
        \csname newif\expandafter\endcsname
        \csname if\KVO@prefix#2\endcsname
        \@ifundefined{\KVO@prefix#2#1}{%
          \PackageWarning{kvoptions}{%
            Initialization of option `#2' failed,\MessageBreak
            cannot set boolean option to `#1',\MessageBreak
            use `true' or `false', now using `false'%
          }%
        }{%
          \csname\KVO@prefix#2#1\endcsname
        }%
        \begingroup
          \edef\x{\endgroup
            \noexpand\define@key{\KVO@family}{#2}[true]{%
              \noexpand\KVO@boolkey{\@currname}%
              \ifx\@currext\@clsextension
                \noexpand\@clsextension
              \else
                \noexpand\@pkgextension
              \fi
              {\KVO@prefix}{#2}{####1}%
            }%
          }%
        \x
      }%
    }%
  }%
}
%    \end{macrocode}
%    \end{macro}
%
%    \begin{macro}{\DeclareComplementaryOption}
%    The first argument is the key name, the second the
%    key that must be a boolean option with the same
%    current family and prefix. A new switch is not
%    created for the new key, we have already a switch.
%    Instead we define switch setting commands to work
%    on the parent switch.
%    \begin{macrocode}
\newcommand*{\DeclareComplementaryOption}[2]{%
  \@ifundefined{if\KVO@prefix#2}{%
    \PackageError{kvoptions}{%
      Cannot generate option code for `#1',\MessageBreak
      parent switch `#2' does not exist%
    }{%
      You are inside %
      \ifx\@currext\@clsextension class\else package\fi\space
      `\@currname.\@currext'.\MessageBreak
      `\KVO@family' is used as familiy %
      for the keyval options.\MessageBreak
      `\KVO@prefix' serves as prefix %
      for internal switch macros.\MessageBreak
      \MessageBreak
      \@ehc
    }%
  }{%
    \KVO@ifdefinable{\KVO@prefix#1true}{%
      \KVO@ifdefinable{\KVO@prefix#1false}{%
        \expandafter\let\csname\KVO@prefix#1false\expandafter\endcsname
          \csname\KVO@prefix#2true\endcsname
        \expandafter\let\csname\KVO@prefix#1true\expandafter\endcsname
          \csname\KVO@prefix#2false\endcsname
%    \end{macrocode}
%    The same code part as in \cs{DeclareBoolOption} can
%    now be used.
%    \begin{macrocode}
        \begingroup
          \edef\x{\endgroup
            \noexpand\define@key{\KVO@family}{#1}[true]{%
              \noexpand\KVO@boolkey{\@currname}%
              \ifx\@currext\@clsextension
                \noexpand\@clsextension
              \else
                \noexpand\@pkgextension
              \fi
              {\KVO@prefix}{#1}{####1}%
            }%
          }%
        \x
      }%
    }%
  }%
}
%    \end{macrocode}
%    \end{macro}
%
%    \begin{macro}{\KVO@ifdefinable}
%    Generate the command token LaTeX's \cs{@ifdefinable} expects.
%    \begin{macrocode}
\def\KVO@ifdefinable#1{%
  \expandafter\@ifdefinable\csname #1\endcsname
}
%    \end{macrocode}
%    \end{macro}
%
%    \begin{macro}{\KVO@boolkey}
%    We check explicitly for |true| and |false| to prevent
%    the user from accidently calling other macros.
%    \begin{quote}
%    \begin{tabular}{@{}ll@{}}
%    |#1|& package/class name\\
%    |#2|& |\@pkgextension|/|\@clsextension|\\
%    |#3|& prefix \\
%    |#4|& key name\\
%    |#5|& new value
%    \end{tabular}
%    \end{quote}
%    \begin{macrocode}
\def\KVO@boolkey#1#2#3#4#5{%
  \edef\KVO@param{#5}%
  \ltx@onelevel@sanitize\KVO@param
  \ifx\KVO@param\KVO@true
    \expandafter\@firstofone
  \else
    \ifx\KVO@param\KVO@false
      \expandafter\expandafter\expandafter\@firstofone
    \else
      \ifx#2\@clsextension
        \expandafter\ClassWarning
      \else
        \expandafter\PackageWarning
      \fi
      {#1}{%
        Value `\KVO@param' is not supported by\MessageBreak
        option `#4'%
      }%
      \expandafter\expandafter\expandafter\@gobble
    \fi
  \fi
  {%
    ^^A\ifx#2\@clsextension
    ^^A  \expandafter\ClassInfo
    ^^A\else
    ^^A  \expandafter\PackageInfo
    ^^A\fi
    ^^A{#1}{[option] #4=\KVO@param}%
    \csname#3#4\KVO@param\endcsname
  }%
}
%    \end{macrocode}
%    \end{macro}
%    \begin{macro}{\KVO@true}
%    \begin{macro}{\KVO@false}
%    The macros \cs{KVO@true} and \cs{KVO@false} are used
%    for string comparisons. After \cs{ltx@onelevel@sanitize}
%    we have only tokens with catcode 12 (other).
%    \begin{macrocode}
\def\KVO@true{true}
\def\KVO@false{false}
\ltx@onelevel@sanitize\KVO@true
\ltx@onelevel@sanitize\KVO@false
%    \end{macrocode}
%    \end{macro}
%    \end{macro}
%
% \subsubsection{\cs{DeclareStringOption}}
%
%    \begin{macro}{\DeclareStringOption}
%    \begin{macrocode}
\newcommand*{\DeclareStringOption}[2][]{%
  \@ifnextchar[{%
    \KVO@DeclareStringOption{#1}{#2}@%
  }{%
    \KVO@DeclareStringOption{#1}{#2}{}[]%
  }%
}
%    \end{macrocode}
%    \end{macro}
%    \begin{macro}{\KVO@DeclareStringOption}
%    \begin{macrocode}
\def\KVO@DeclareStringOption#1#2#3[#4]{%
  \KVO@ifdefinable{\KVO@prefix#2}{%
    \@namedef{\KVO@prefix#2}{#1}%
    \begingroup
      \ifx\\#3\\%
        \toks@{}%
      \else
        \toks@{[{#4}]}%
      \fi
      \edef\x{\endgroup
        \noexpand\define@key{\KVO@family}{#2}\the\toks@{%
          ^^A\begingroup
          ^^A  \toks@{####1}%
          ^^A  \ifx\@currext\@clsextension
          ^^A    \noexpand\ClassInfo
          ^^A  \else
          ^^A    \noexpand\PackageInfo
          ^^A  \fi
          ^^A  {\@currname}{%
          ^^A    [option] #2={\noexpand\the\toks@}%
          ^^A  }%
          ^^A\endgroup
          \noexpand\def
          \expandafter\noexpand\csname\KVO@prefix#2\endcsname{####1}%
        }%
      }%
    \x
  }%
}
%    \end{macrocode}
%    \end{macro}
%
% \subsubsection{\cs{DeclareVoidOption}}
%
%    \begin{macro}{\DeclareVoidOption}
%    \begin{macrocode}
\newcommand*{\DeclareVoidOption}[2]{%
  \begingroup
    \let\next\@gobbletwo
    \KVO@ifdefinable{\KVO@prefix#1}{%
      \let\next\@firstofone
    }%
  \expandafter\endgroup
  \next{%
    \begingroup
      \edef\x{\endgroup
        \noexpand\define@key{\KVO@family}{#1}[\KVO@VOID@]{%
          \noexpand\KVO@voidkey{\@currname}%
          \ifx\@currext\@clsextension
            \noexpand\@clsextension
          \else
            \noexpand\@pkgextension
          \fi
          {#1}%
          {####1}%
          \expandafter\noexpand\csname\KVO@prefix#1\endcsname
        }%
      }%
    \x
    \begingroup
      \toks@{#2}%
    \expandafter\endgroup
    \expandafter\def
    \csname\KVO@prefix#1\expandafter\endcsname
    \expandafter{\the\toks@}%
  }%
}
\def\KVO@VOID@{@VOID@}
%    \end{macrocode}
%    \end{macro}
%    \begin{macro}{\KVO@voidkey}
%    \begin{tabular}{@{}ll@{}}
%    |#1|& package/class name\\
%    |#2|& |\@pkgextension|/|\@clsextension|\\
%    |#3|& key name\\
%    |#4|& default (|@VOID@|)\\
%    |#5|& macro with option code
%    \end{tabular}
%    \begin{macrocode}
\def\KVO@voidkey#1#2#3#4{%
  \def\CurrentOption{#3}%
  \begingroup
    \def\x{#4}%
  \expandafter\endgroup
  \ifx\x\KVO@VOID@
  \else
    \ifx#2\@clsextension
      \expandafter\ClassWarning
    \else
      \expandafter\PackageWarning
    \fi
    {#1}{%
      Unexpected value for option `#3'\MessageBreak
      is ignored%
    }%
  \fi
  ^^A\ifx#2\@clsextension
  ^^A  \expandafter\ClassInfo
  ^^A\else
  ^^A  \expandafter\PackageInfo
  ^^A\fi
  ^^A{#1}{[option] #3}%
}
%    \end{macrocode}
%    \end{macro}
%
% \subsubsection{\cs{DeclareDefaultOption}}
%
%    \begin{macro}{\DeclareDefaultOption}
%    \begin{macrocode}
\newcommand*{\DeclareDefaultOption}{%
  \@namedef{KVO@default@\@currname.\@currext}%
}
%    \end{macrocode}
%    \end{macro}
%
% \subsubsection{\cs{DeclareLocalOptions}}
%
%    \begin{macro}{\DeclareLocalOptions}
%    \begin{macrocode}
\newcommand*{\DeclareLocalOptions}[1]{%
  \comma@parse{#1}\KVO@DeclareLocalOption
}
%    \end{macrocode}
%    \end{macro}
%    \begin{macro}{\KVO@DeclareLocalOption}
%    \begin{macrocode}
\def\KVO@DeclareLocalOption#1{%
  \expandafter\def\csname KVO@local@\KVO@family @#1\endcsname{}%
}
%    \end{macrocode}
%    \end{macro}
%
% \subsection{Dynamic options}
%
% \subsubsection{\cs{DisableKeyvalOption}}
%
%    \begin{macrocode}
\SetupKeyvalOptions{%
  family=KVOdyn,%
  prefix=KVOdyn@%
}
\DeclareBoolOption[true]{global}
\DeclareComplementaryOption{local}{global}
\DeclareStringOption[undef]{action}
\let\KVOdyn@name\relax
\let\KVOdyn@ext\@empty
\define@key{KVOdyn}{class}{%
  \def\KVOdyn@name{#1}%
  \let\KVOdyn@ext\@clsextension
}
\define@key{KVOdyn}{package}{%
  \def\KVOdyn@name{#1}%
  \let\KVOdyn@ext\@pkgextension
}
\newcommand*{\DisableKeyvalOption}[3][]{%
  \begingroup
    \kvsetkeys{KVOdyn}{#1}%
    \def\x{\endgroup}%
    \@ifundefined{KVO@action@\KVOdyn@action}{%
      \PackageError{kvoptions}{%
        Unknown disable action %
        `\expandafter\strip@prefix\meaning\KVOdyn@action'\MessageBreak
        for option `#3' in keyval family '#2'%
      }\@ehc
    }{%
      \csname KVO@action@\KVOdyn@action\endcsname{#2}{#3}%
    }%
  \x
}
\def\KVO@action@undef#1#2{%
  \edef\x{\endgroup
    \ifKVOdyn@global\global\fi
    \let
    \expandafter\noexpand\csname KV@#1@#2\endcsname
    \relax
    \ifKVOdyn@global\global\fi
    \let
    \expandafter\noexpand\csname KV@#1@#2@default\endcsname
    \relax
  }%
  ^^A\PackageInfo{kvoptions}{%
  ^^A  [option] key `#2' of family `#1'\MessageBreak
  ^^A  is disabled (undef, \ifKVOdyn@global global\else local\fi)%
  ^^A}%
}
\def\KVO@action@ignore#1#2{%
  \edef\x{\endgroup
    \ifKVOdyn@global\global\fi
    \let
    \expandafter\noexpand\csname KV@#1@#2\endcsname
    \noexpand\@gobble
    \ifKVOdyn@global\global\fi
    \let
    \expandafter\noexpand\csname KV@#1@#2@default\endcsname
    \noexpand\@empty
  }%
  ^^A\PackageInfo{kvoptions}{%
  ^^A  [option] key `#2' of family `#1'\MessageBreak
  ^^A  is disabled (ignore, \ifKVOdyn@global global\else local\fi)%
  ^^A}%
}
\def\KVO@action@error{%
  \KVO@do@action{error}%
}
\def\KVO@action@warning{%
  \KVO@do@action{warning}%
}
%    \end{macrocode}
%    \begin{tabular}{@{}ll@{}}
%    |#1|& |error| or |warning|\\
%    |#2|& \meta{family}\\
%    |#3|& \meta{key}\\
%    \end{tabular}
%    \begin{macrocode}
\def\KVO@do@action#1#2#3{%
  \ifx\KVOdyn@name\relax
    \PackageError{kvoptions}{%
      Action type `#1' needs package/class name\MessageBreak
      for key `#3' in family `#2'%
    }\@ehc
  \else
    \edef\x{\endgroup
      \noexpand\define@key{#2}{#3}[]{%
        \expandafter\noexpand\csname KVO@disable@#1\endcsname
        {\KVOdyn@name}\noexpand\KVOdyn@ext{#3}%
      }%
      \ifKVOdyn@global
        \global\let
        \expandafter\noexpand\csname KV@#2@#3\endcsname
        \expandafter\noexpand\csname KV@#2@#3\endcsname
        \global\let
        \expandafter\noexpand\csname KV@#2@#3@default\endcsname
        \expandafter\noexpand\csname KV@#2@#3@default\endcsname
      \fi
    }%
    ^^A\ifx\KVOdyn@ext\@clsextension
    ^^A  \expandafter\ClassInfo
    ^^A\else
    ^^A   \expandafter\PackageInfo
    ^^A\fi
    ^^A{\KVOdyn@name}{%
    ^^A  [option] key `#3' of family `#2'\MessageBreak
    ^^A  is disabled (#1, \ifKVOdyn@global global\else local\fi)%
    ^^A}%
  \fi
}
\def\KVO@disable@error#1#2#3{%
  \ifx#2\@clsextension
    \expandafter\ClassError
  \else
    \expandafter\PackageError
  \fi
  {#1}{%
    Option `#3' is given too late,\MessageBreak
    now the option is ignored%
  }\@ehc
}
\def\KVO@disable@warning#1#2#3{%
  \ifx#2\@clsextension
    \expandafter\ClassWarning
  \else
    \expandafter\PackageWarning
  \fi
  {#1}{%
    Option `#3' is already consumed\MessageBreak
    and has no effect%
  }%
}
%    \end{macrocode}
%
% \subsection{Change option code}
%
% \subsubsection{\cs{AddToKeyvalOption}}
%
%    \begin{macro}{\AddToKeyvalOption}
%    \begin{macrocode}
\newcommand*{\AddToKeyvalOption}{%
  \@ifstar{%
    \begingroup
      \edef\x{\endgroup
        \noexpand\KVO@AddToKeyvalOption{\KVO@family}%
      }%
    \x
  }%
  \KVO@AddToKeyvalOption
}
%    \end{macrocode}
%    \end{macro}
%    \begin{macro}{\KVO@AddToKeyvalOption}
%    \begin{macrocode}
\def\KVO@AddToKeyvalOption#1#2{%
  \@ifundefined{KV@#1@#2}{%
    \PackageWarning{kvoptions}{%
      Key `#2' of family `#1' does not exist.\MessageBreak
      Ignoring \string\AddToKeyvalOption
    }%
    \@gobble
  }{%
    \edef\KVO@next{%
      \noexpand\KVO@@AddToKeyvalOption
      \expandafter\noexpand\csname KV@#1@#2\endcsname
    }%
    \afterassignment\KVO@next
    \def\KVO@temp##1%
  }%
}
%    \end{macrocode}
%    \end{macro}
%    \begin{macro}{\KVO@@AddToKeyvalOption}
%    \begin{macrocode}
\def\KVO@@AddToKeyvalOption#1{%
  \begingroup
    \toks@\expandafter{#1{##1}}%
    \toks@\expandafter{\the\expandafter\toks@\KVO@temp{##1}}%
    \edef\x{\endgroup
      \noexpand\def\noexpand#1####1{\the\toks@}%
    }%
  \x
}
%    \end{macrocode}
%    \end{macro}
%
% \subsection{Process options}
%
% \subsubsection{Get global options}
%
%    Package \xpackage{xkeyval} removes options with equal signs from
%    the global options (\cs{@classoptionslist}). The effect is that
%    other packages and classes will not see these global options anymore.
%    A bug-report was answered that this behaviour is ``by design''.
%    Thus I call it a design bug. Now getting the global options require
%    an algorithm instead of a simple macro call.
%    \begin{macrocode}
%</package>
%<*package|patch>
%    \end{macrocode}
%    \begin{macro}{\KVO@IfDefThen}
%    Call |#2| if command |#1| is defined and not \cs{relax}.
%    (Package \xpackage{kvoptions-patch} does not load
%    package \xpackage{ltxcmds}.)
%    \begin{macrocode}
\def\KVO@IfDefThen#1#2{%
  \ifx#1\ltx@undefined
  \else
    \ifx#1\relax
    \else
      #2%
    \fi
  \fi
}%
%    \end{macrocode}
%    \end{macro}
%    \begin{macro}{\KVO@GetClassOptionsList}
%    \begin{macrocode}
\def\KVO@GetClassOptionsList{%
  \let\KVO@classoptionslist\@classoptionslist
  \KVO@IfDefThen\@classoptionslist{%
    \KVO@IfDefThen\XKV@documentclass{%
      \ifx\XKV@documentclass\ltx@empty
      \else
        \KVO@IfDefThen\XKV@classoptionslist{%
          \ifx\XKV@classoptionslist\ltx@empty
          \else
            \let\KVO@classoptionslist\XKV@classoptionslist
          \fi
        }%
      \fi
    }%
  }%
}%
%    \end{macrocode}
%    \end{macro}
%    \begin{macrocode}
%</package|patch>
%<*package>
%    \end{macrocode}
%
% \subsubsection{\cs{ProcessKeyvalOptions}}
%
%    \begin{macro}{\ProcessKeyvalOptions}
%    If the optional star is given, we get the family
%    name and expand it for safety.
%    \begin{macrocode}
\newcommand*{\ProcessKeyvalOptions}{%
  \@ifstar{%
    \begingroup
      \edef\x{\endgroup
        \noexpand\KVO@ProcessKeyvalOptions{\KVO@family}%
      }%
    \x
  }%
  \KVO@ProcessKeyvalOptions
}
%    \end{macrocode}
%    \end{macro}
%
%    \begin{macrocode}
\def\KVO@ProcessKeyvalOptions#1{%
  \let\@tempc\relax
  \let\KVO@temp\@empty
%    \end{macrocode}
%    Add any global options that are known to KV to the start of
%    the list being built in |\KVO@temp| and mark them used (by
%    removing them from the unused option list).
%    \begin{macrocode}
  \ifx\@currext\@clsextension
  \else
    \KVO@GetClassOptionsList
    \ifx\KVO@classoptionslist\relax
    \else
      \@for\KVO@CurrentOption:=\KVO@classoptionslist\do{%
        \@ifundefined{KV@#1@\expandafter\KVO@getkey
                      \KVO@CurrentOption=\@nil}{%
        }{%
          \@ifundefined{KVO@local@#1@\expandafter\KVO@getkey
                        \KVO@CurrentOption=\@nil}{%
            \ifx\KVO@Patch Y%
              \edef\KVO@temp{%
                \etex@unexpanded\expandafter{%
                  \KVO@temp
                }%
                ,%
                \etex@unexpanded\expandafter{%
                  \KVO@CurrentOption
                }%
                ,%
              }%
              \ltx@onelevel@sanitize\KVO@CurrentOption
            \else
              \edef\KVO@temp{%
                \KVO@temp
                ,%
                \KVO@CurrentOption
                ,%
              }%
            \fi
            \@expandtwoargs\@removeelement\KVO@CurrentOption
              \@unusedoptionlist\@unusedoptionlist
          }{}%
        }%
      }%
    \fi
  \fi
%    \end{macrocode}
%     Now stick the package options at the end of the list and wrap
%     in a call to \cs{setkeys}. A class ignores unknown global
%     options, we must remove them to prevent error messages
%     from \cs{setkeys}.
%    \begin{macrocode}
  \begingroup
    \toks\tw@{}%
    \@ifundefined{opt@\@currname.\@currext}{%
      \toks@\expandafter{\KVO@temp}%
    }{%
      \toks@\expandafter\expandafter\expandafter{%
        \csname opt@\@currname.\@currext\endcsname
      }%
      \ifx\@currext\@clsextension
        \edef\CurrentOption{\the\toks@}%
        \toks@\expandafter{\KVO@temp}%
        \@for\CurrentOption:=\CurrentOption\do{%
          \@ifundefined{%
            KV@#1@\expandafter\KVO@getkey\CurrentOption=\@nil
          }{%
%    \end{macrocode}
%    A class puts not used options in the unused option list
%    unless there is a default handler.
%    \begin{macrocode}
            \@ifundefined{KVO@default@\@currname.\@currext}{%
              \ifx\KVO@Patch Y%
                \ltx@onelevel@sanitize\CurrentOption
              \fi
              \ifx\@unusedoptionlist\@empty
                \global\let\@unusedoptionlist\CurrentOption
              \else
                \expandafter\expandafter\expandafter\gdef
                \expandafter\expandafter\expandafter\@unusedoptionlist
                \expandafter\expandafter\expandafter{%
                  \expandafter\@unusedoptionlist
                  \expandafter,\CurrentOption
                }%
              \fi
            }{%
              \toks\tw@\expandafter{%
                \the\toks\expandafter\tw@\expandafter,\CurrentOption
              }%
            }%
          }{%
            \toks@\expandafter{%
              \the\expandafter\toks@\expandafter,\CurrentOption
            }%
          }%
        }%
      \else
%    \end{macrocode}
%    Without default action we pass all options to \cs{setkeys}.
%    Otherwise we have to check which options are known.
%    These are passed to \cs{setkeys}. For the others the default
%    action is performed.
%    \begin{macrocode}
        \@ifundefined{KVO@default@\@currname.\@currext}{%
          \toks@\expandafter\expandafter\expandafter{%
            \expandafter\KVO@temp\the\toks@
          }%
        }{%
          \edef\CurrentOption{\the\toks@}%
          \toks@\expandafter{\KVO@temp}%
          \@for\CurrentOption:=\CurrentOption\do{%
            \@ifundefined{%
              KV@#1@\expandafter\KVO@getkey\CurrentOption=\@nil
            }{%
              \toks\tw@\expandafter{%
                \the\toks\expandafter\tw@\expandafter,\CurrentOption
              }%
            }{%
              \toks@\expandafter{%
                \the\expandafter\toks@\expandafter,\CurrentOption
              }%
            }%
          }%
        }%
      \fi
    }%
    \edef\KVO@temp{\endgroup
      \noexpand\KVO@calldefault{\the\toks\tw@}%
      \noexpand\KVO@setkeys{#1}{\the\toks@}%
    }%
  \KVO@temp
%    \end{macrocode}
%    Some cleanup of \cs{ProcessOptions}.
%    \begin{macrocode}
  \let\CurrentOption\@empty
  \AtEndOfPackage{\let\@unprocessedoptions\relax}%
}
%    \end{macrocode}
%
% \subsubsection{\cs{ProcessLocalKeyvalOptions}}
%
%    \begin{macro}{\ProcessLocalKeyvalOptions}
%    If the optional star is given, we get the family
%    name and expand it for safety.
%    \begin{macrocode}
\newcommand*{\ProcessLocalKeyvalOptions}{%
  \@ifstar{%
    \begingroup
      \edef\x{\endgroup
        \noexpand\KVO@ProcessLocalKeyvalOptions{\KVO@family}%
      }%
    \x
  }%
  \KVO@ProcessLocalKeyvalOptions
}
%    \end{macrocode}
%    \end{macro}
%
%    \begin{macrocode}
\def\KVO@ProcessLocalKeyvalOptions#1{%
  \let\@tempc\relax
  \let\KVO@temp\@empty
%    \end{macrocode}
%    Check if \cs{ProcessLocalKeyvalOptions} is called inside
%    a package.
%    \begin{macrocode}
  \ifx\@currext\@pkgextension
  \else
    \PackageError{kvoptions}{%
      \string\ProcessLocalKeyvalOptions\space is intended for packages only%
    }\@ehc
  \fi
%    \end{macrocode}
%     The package options are put into toks register \cs{toks@}.
%    \begin{macrocode}
  \begingroup
    \toks\tw@{}%
    \@ifundefined{opt@\@currname.\@currext}{%
      \toks@\expandafter{\KVO@temp}%
    }{%
      \toks@\expandafter\expandafter\expandafter{%
        \csname opt@\@currname.\@currext\endcsname
      }%
%    \end{macrocode}
%    Without default action we pass all options to \cs{setkeys}.
%    Otherwise we have to check which options are known.
%    These are passed to \cs{setkeys}. For the others the default
%    action is performed.
%    \begin{macrocode}
      \@ifundefined{KVO@default@\@currname.\@currext}{%
        \toks@\expandafter\expandafter\expandafter{%
          \expandafter\KVO@temp\the\toks@
        }%
      }{%
        \edef\CurrentOption{\the\toks@}%
        \toks@\expandafter{\KVO@temp}%
        \@for\CurrentOption:=\CurrentOption\do{%
          \@ifundefined{%
            KV@#1@\expandafter\KVO@getkey\CurrentOption=\@nil
          }{%
            \toks\tw@\expandafter{%
              \the\toks\expandafter\tw@\expandafter,\CurrentOption
            }%
          }{%
            \toks@\expandafter{%
              \the\expandafter\toks@\expandafter,\CurrentOption
            }%
          }%
        }%
      }%
    }%
    \edef\KVO@temp{\endgroup
      \noexpand\KVO@calldefault{\the\toks\tw@}%
      \noexpand\KVO@setkeys{#1}{\the\toks@}%
    }%
  \KVO@temp
%    \end{macrocode}
%    Some cleanup of \cs{ProcessOptions}.
%    \begin{macrocode}
  \let\CurrentOption\@empty
  \AtEndOfPackage{\let\@unprocessedoptions\relax}%
}
%    \end{macrocode}
%
% \subsubsection{Helper macros}
%
%    \begin{macro}{\KVO@getkey}
%    Extract the key part of a key=value pair.
%    \begin{macrocode}
\def\KVO@getkey#1=#2\@nil{#1}
%    \end{macrocode}
%    \end{macro}
%
%    \begin{macro}{\KVO@calldefault}
%    \begin{macrocode}
\def\KVO@calldefault#1{%
  \begingroup
    \def\x{#1}%
  \expandafter\endgroup
  \ifx\x\@empty
  \else
    \@for\CurrentOption:=#1\do{%
      \ifx\CurrentOption\@empty
      \else
        \expandafter\KVO@setcurrents\CurrentOption=\@nil
        \@nameuse{KVO@default@\@currname.\@currext}%
      \fi
    }%
  \fi
}
%    \end{macrocode}
%    \end{macro}
%    \begin{macro}{\KVO@setcurrents}
%    Extract the key part of a key=value pair.
%    \begin{macrocode}
\def\KVO@setcurrents#1=#2\@nil{%
  \def\CurrentOptionValue{#2}%
  \ifx\CurrentOptionValue\@empty
    \let\CurrentOptionKey\CurrentOption
    \let\CurrentOptionValue\relax
  \else
    \edef\CurrentOptionKey{\zap@space#1 \@empty}%
    \expandafter\KVO@setcurrentvalue\CurrentOption\@nil
  \fi
}
%    \end{macrocode}
%    \end{macro}
%    \begin{macro}{\KV@setcurrentvalue}
%    Here the value part is parsed. Package \xpackage{keyval}'s
%    \cs{KV@@sp@def} helps in removing spaces at the begin and
%    end of the value.
%    \begin{macrocode}
\def\KVO@setcurrentvalue#1=#2\@nil{%
  \KV@@sp@def\CurrentOptionValue{#2}%
}
%    \end{macrocode}
%    \end{macro}
%
% \subsection{\plainTeX}
%
%    Disable \LaTeX\ stuff.
%    \begin{macrocode}
\begingroup\expandafter\expandafter\expandafter\endgroup
\expandafter\ifx\csname documentclass\endcsname\relax
  \def\ProcessKeyvalOptions{%
    \@ifstar{}\@gobble
  }%
\fi
%    \end{macrocode}
%
%    \begin{macrocode}
\KVO@AtEnd%
%</package>
%    \end{macrocode}
%
% \subsection{Package \xpackage{kvoptions-patch}}
%
%    \begin{macrocode}
%<*patch>
\NeedsTeXFormat{LaTeX2e}
\begingroup\catcode61\catcode48\catcode32=10\relax%
  \catcode13=5 % ^^M
  \endlinechar=13 %
  \catcode123=1 % {
  \catcode125=2 % }
  \catcode64=11 % @
  \def\x{\endgroup
    \expandafter\edef\csname KVO@AtEnd\endcsname{%
      \endlinechar=\the\endlinechar\relax
      \catcode13=\the\catcode13\relax
      \catcode32=\the\catcode32\relax
      \catcode35=\the\catcode35\relax
      \catcode61=\the\catcode61\relax
      \catcode64=\the\catcode64\relax
      \catcode123=\the\catcode123\relax
      \catcode125=\the\catcode125\relax
    }%
  }%
\x\catcode61\catcode48\catcode32=10\relax%
\catcode13=5 % ^^M
\endlinechar=13 %
\catcode35=6 % #
\catcode64=11 % @
\catcode123=1 % {
\catcode125=2 % }
\def\TMP@EnsureCode#1#2{%
  \edef\KVO@AtEnd{%
    \KVO@AtEnd
    \catcode#1=\the\catcode#1\relax
  }%
  \catcode#1=#2\relax
}
\TMP@EnsureCode{39}{12}% '
\TMP@EnsureCode{40}{12}% (
\TMP@EnsureCode{41}{12}% )
\TMP@EnsureCode{43}{12}% +
\TMP@EnsureCode{44}{12}% ,
\TMP@EnsureCode{45}{12}% -
\TMP@EnsureCode{46}{12}% .
\TMP@EnsureCode{47}{12}% /
\TMP@EnsureCode{58}{12}% :
\TMP@EnsureCode{60}{12}% <
\TMP@EnsureCode{62}{12}% >
\TMP@EnsureCode{91}{12}% [
\TMP@EnsureCode{93}{12}% ]
\TMP@EnsureCode{96}{12}% `
\TMP@EnsureCode{124}{12}% |
\edef\KVO@AtEnd{\KVO@AtEnd\noexpand\endinput}
\ProvidesPackage{kvoptions-patch}%
  [2016/05/16 v3.12 LaTeX patch for keyval options (HO)]%
%    \end{macrocode}
%
%    Check for \eTeX.
%    \begin{macrocode}
\begingroup\expandafter\expandafter\expandafter\endgroup
\expandafter\ifx\csname eTeXversion\endcsname\relax
  \PackageWarningNoLine{kvoptions-patch}{%
    Package loading is aborted, because e-TeX is missing%
  }%
  \expandafter\KVO@AtEnd
\fi%
%    \end{macrocode}
%
%    Package \xpackage{etexcmds} for \cs{etex@unexpanded}.
%    \begin{macrocode}
\RequirePackage{etexcmds}[2007/09/09]
\ifetex@unexpanded
\else
  \PackageError{kvoptions-patch}{%
    Could not find eTeX's \string\unexpanded.\MessageBreak
    Try adding \string\RequirePackage\string{etexcmds\string} %
    before \string\documentclass%
  }\@ehd
  \expandafter\KVO@AtEnd
\fi%
%    \end{macrocode}
%
%    Check for package \xpackage{xkvltxp}.
%    \begin{macrocode}
\@ifpackageloaded{xkvltxp}{%
  \PackageWarningNoLine{kvoptions}{%
    Option `patch' cannot be used together with\MessageBreak
    package `xkvltxp' that is already loaded.\MessageBreak
    Therefore package loading is aborted%
  }%
  \KVO@AtEnd
}{}%
%    \end{macrocode}
%
%    \begin{macrocode}
\def\@if@ptions#1#2#3{%
  \begingroup
    \KVO@normalize\KVO@temp{#3}%
    \edef\x{\endgroup
      \noexpand\@if@pti@ns{%
        \detokenize\expandafter\expandafter\expandafter{%
          \csname opt@#2.#1\endcsname
        }%
      }{%
        \detokenize\expandafter{\KVO@temp}%
      }%
    }%
  \x
}
%    \end{macrocode}
%
%    \begin{macrocode}
\def\@pass@ptions#1#2#3{%
  \KVO@normalize\KVO@temp{#2}%
  \@ifundefined{opt@#3.#1}{%
    \expandafter\gdef\csname opt@#3.#1%
          \expandafter\endcsname\expandafter{%
      \KVO@temp
    }%
  }{%
    \expandafter\gdef\csname opt@#3.#1%
          \expandafter\expandafter\expandafter\endcsname
          \expandafter\expandafter\expandafter{%
      \csname opt@#3.#1\expandafter\endcsname\expandafter,\KVO@temp
    }%
  }%
}
%    \end{macrocode}
%
%    \begin{macrocode}
\def\ProcessOptions{%
  \let\ds@\@empty
  \@ifundefined{opt@\@currname.\@currext}{%
    \let\@curroptions\@empty
  }{%
    \expandafter\expandafter\expandafter\def
    \expandafter\expandafter\expandafter\@curroptions
    \expandafter\expandafter\expandafter{%
      \csname opt@\@currname.\@currext\endcsname
    }%
  }%
  \@ifstar\KVO@xprocess@ptions\KVO@process@ptions
}
%    \end{macrocode}
%
%    \begin{macrocode}
\def\KVO@process@ptions{%
  \@for\CurrentOption:=\@declaredoptions\do{%
    \ifx\CurrentOption\@empty
    \else
      \begingroup
        \ifx\@currext\@clsextension
          \toks@{}%
        \else
          \KVO@GetClassOptionsList
          \toks@\expandafter{\KVO@classoptionslist,}%
        \fi
        \toks\tw@\expandafter{\@curroptions}%
        \edef\x{\endgroup
          \noexpand\in@{,\CurrentOption,}{,\the\toks@\the\toks\tw@,}%
        }%
      \x
      \ifin@
        \KVO@use@ption
        \expandafter\let\csname ds@\CurrentOption\endcsname\@empty
      \fi
    \fi
  }%
  \KVO@process@pti@ns
}
%    \end{macrocode}
%
%    \begin{macrocode}
\def\KVO@xprocess@ptions{%
  \ifx\@currext\@clsextension
  \else
    \KVO@GetClassOptionsList
    \@for\CurrentOption:=\KVO@classoptionslist\do{%
      \ifx\CurrentOption\@empty
      \else
        \KVO@in@\CurrentOption\@declaredoptions
        \ifin@
          \KVO@use@ption
          \expandafter\let\csname ds@\CurrentOption\endcsname\@empty
        \fi
      \fi
    }%
  \fi
  \KVO@process@pti@ns
}
%    \end{macrocode}
%
%    \begin{macrocode}
\def\KVO@in@#1#2{%
  \in@false
  \begingroup
    \@for\x:=#2\do{%
      \ifx\x#1\relax
        \in@true
      \fi
    }%
    \edef\x{\endgroup
      \ifin@
        \noexpand\in@true
      \fi
    }%
  \x
}
%    \end{macrocode}
%
%    \begin{macrocode}
\def\KVO@process@pti@ns{%
  \@for\CurrentOption:=\@curroptions\do{%
    \@ifundefined{ds@\KVO@SanitizedCurrentOption}{%
      \KVO@use@ption
      \default@ds
    }%
    \KVO@use@ption
  }%
  \@for\CurrentOption:=\@declaredoptions\do{%
    \expandafter\let\csname ds@\CurrentOption\endcsname\relax
  }%
  \let\CurrentOption\@empty
  \let\@fileswith@pti@ns\@@fileswith@pti@ns
  \AtEndOfPackage{\let\@unprocessedoptions\relax}%
}
%    \end{macrocode}
%
%    \begin{macrocode}
\def\KVO@use@ption{%
  \begingroup
    \edef\x{\endgroup
      \noexpand\@removeelement{%
        \detokenize\expandafter{\CurrentOption}%
      }{%
        \detokenize\expandafter{\@unusedoptionlist}%
      }%
    }%
  \x\@unusedoptionlist
  \csname ds@\KVO@SanitizedCurrentOption\endcsname
}
%    \end{macrocode}
%
%    \begin{macrocode}
\def\OptionNotUsed{%
  \ifx\@currext\@clsextension
    \xdef\@unusedoptionlist{%
      \ifx\@unusedoptionlist\@empty
      \else
        \detokenize\expandafter{\@unusedoptionlist,}%
      \fi
      \detokenize\expandafter{\CurrentOption}%
    }%
  \fi
}
%    \end{macrocode}
%
%    Variant of \cs{ExecuteOptions} that better protects \cs{CurrentOption}.
%    \begin{macrocode}
\def\CurrentOption@SaveLevel{0}
\def\ExecuteOptions{%
  \expandafter\KVO@ExecuteOptions
      \csname CurrentOption@\CurrentOption@SaveLevel\endcsname
}
\def\KVO@ExecuteOptions#1#2{%
  \let#1\CurrentOption
  \edef\CurrentOption@SaveLevel{%
    \the\numexpr\CurrentOption@SaveLevel+1%
  }%
  \@for\CurrentOption:=#2\do{%
    \csname ds@\CurrentOption\endcsname
  }%
  \edef\CurrentOption@SaveLevel{%
    \the\numexpr\CurrentOption@SaveLevel-1%
  }%
  \let\CurrentOption#1%
}
%    \end{macrocode}
%
%    \begin{macrocode}
\def\KVO@fileswith@pti@ns#1[#2]#3[#4]{%
  \ifx#1\@clsextension
    \ifx\@classoptionslist\relax
      \KVO@normalize\KVO@temp{#2}%
      \expandafter\gdef\expandafter\@classoptionslist\expandafter{%
        \KVO@temp
      }%
      \def\reserved@a{%
        \KVO@onefilewithoptions{#3}[{#2}][{#4}]#1%
        \@documentclasshook
      }%
    \else
      \def\reserved@a{%
        \KVO@onefilewithoptions{#3}[{#2}][{#4}]#1%
      }%
    \fi
  \else
    \begingroup
      \let\KVO@temp\relax
      \let\KVO@onefilewithoptions\relax
      \let\@pkgextension\relax
      \def\reserved@b##1,{%
        \ifx\@nil##1\relax
        \else
          \ifx\relax##1\relax
          \else
            \KVO@onefilewithoptions{##1}[{\KVO@temp}][{#4}]%
            \@pkgextension
          \fi
          \expandafter\reserved@b
        \fi
      }%
      \edef\reserved@a{\zap@space#3 \@empty}%
      \edef\reserved@a{\expandafter\reserved@b\reserved@a,\@nil,}%
      \toks@{#2}%
      \def\KVO@temp{\the\toks@}%
    \edef\reserved@a{\endgroup \reserved@a}%
  \fi
  \reserved@a
}
%    \end{macrocode}
%
%    \begin{macrocode}
\def\KVO@onefilewithoptions#1[#2][#3]#4{%
  \@pushfilename
  \xdef\@currname{#1}%
  \global\let\@currext#4%
  \expandafter\let\csname\@currname.\@currext-h@@k\endcsname\@empty
  \let\CurrentOption\@empty
  \@reset@ptions
  \makeatletter
  \def\reserved@a{%
    \@ifl@aded\@currext{#1}{%
      \@if@ptions\@currext{#1}{#2}{%
      }{%
        \begingroup
          \@ifundefined{opt@#1.\@currext}{%
            \def\x{}%
          }{%
            \edef\x{%
              \expandafter\expandafter\expandafter\strip@prefix
              \expandafter\meaning\csname opt@#1.\@currext\endcsname
            }%
          }%
          \def\y{#2}%
          \edef\y{\expandafter\strip@prefix\meaning\y}%
          \@latex@error{Option clash for \@cls@pkg\space #1}{%
            The package #1 has already been loaded %
            with options:\MessageBreak
            \space\space[\x]\MessageBreak
            There has now been an attempt to load it %
             with options\MessageBreak
            \space\space[\y]\MessageBreak
            Adding the global options:\MessageBreak
            \space\space
                 \x,\y\MessageBreak
            to your \noexpand\documentclass declaration may fix this.%
            \MessageBreak
            Try typing \space <return> \space to proceed.%
          }%
        \endgroup
      }%
    }{%
      \@pass@ptions\@currext{#2}{#1}%
      \global\expandafter
      \let\csname ver@\@currname.\@currext\endcsname\@empty
      \InputIfFileExists
        {\@currname.\@currext}%
        {}%
        {\@missingfileerror\@currname\@currext}%
      \let\@unprocessedoptions\@@unprocessedoptions
      \csname\@currname.\@currext-h@@k\endcsname
      \expandafter\let\csname\@currname.\@currext-h@@k\endcsname
              \@undefined
      \@unprocessedoptions
    }%
    \@ifl@ter\@currext{#1}{#3}{%
    }{%
      \@latex@warning@no@line{%
        You have requested,\on@line, %
        version\MessageBreak
          #3' of \@cls@pkg\space #1,\MessageBreak
        but only version\MessageBreak
         `\csname ver@#1.\@currext\endcsname'\MessageBreak
        is available%
      }%
    }%
    \ifx\@currext\@clsextension\let\LoadClass\@twoloadclasserror\fi
    \@popfilename
    \@reset@ptions
  }%
  \reserved@a
}
%    \end{macrocode}
%
%    \begin{macrocode}
\def\@unknownoptionerror{%
  \@latex@error{%
    Unknown option `\KVO@SanitizedCurrentOption' %
    for \@cls@pkg\space`\@currname'%
  }{%
    The option `\KVO@SanitizedCurrentOption' was not declared in %
    \@cls@pkg\space`\@currname', perhaps you\MessageBreak
    misspelled its name. %
    Try typing \space <return> %
    \space to proceed.%
  }%
}
%    \end{macrocode}
%
%    \begin{macrocode}
\def\@@unprocessedoptions{%
  \ifx\@currext\@pkgextension
    \@ifundefined{opt@\@currname.\@currext}{%
      \let\@curroptions\@empty
    }{%
      \expandafter\let\expandafter\@curroptions
          \csname opt@\@currname.\@currext\endcsname
    }%
    \@for\CurrentOption:=\@curroptions\do{%
        \ifx\CurrentOption\@empty\else\@unknownoptionerror\fi
    }%
  \fi
}
%    \end{macrocode}
%
%    \begin{macrocode}
\def\KVO@SanitizedCurrentOption{%
  \expandafter\strip@prefix\meaning\CurrentOption
}
%    \end{macrocode}
%
%    Normalize option list.
%    \begin{macrocode}
\def\KVO@normalize#1#2{%
  \let\KVO@result\@empty
  \KVO@splitcomma#2,\@nil
  \let#1\KVO@result
}
\def\KVO@splitcomma#1,#2\@nil{%
  \KVO@ifempty{#1}{}{%
    \KVO@checkkv#1=\@nil
  }%
  \KVO@ifempty{#2}{}{\KVO@splitcomma#2\@nil}%
}
\def\KVO@ifempty#1{%
  \expandafter\ifx\expandafter\\\detokenize{#1}\\%
    \expandafter\@firstoftwo
  \else
    \expandafter\@secondoftwo
  \fi
}
\def\KVO@checkkv#1=#2\@nil{%
  \KVO@ifempty{#2}{%
    % option without value
    \edef\KVO@x{\zap@space#1 \@empty}%
    \ifx\KVO@x\@empty
      % ignore empty option
    \else
      % append to list
      \edef\KVO@result{%
        \etex@unexpanded\expandafter{\KVO@result},\KVO@x
      }%
    \fi
  }{%
    % #1: "key", #2: "value="
    % add key part
    \edef\KVO@result{%
      \etex@unexpanded\expandafter{\KVO@result},%
      \zap@space#1 \@empty
    }%
    \futurelet\@let@token\KVO@checkfirsttok#2 \@nil| = \@nil|\KVO@nil
  }%
}
\def\KVO@checkfirsttok{%
  \ifx\@let@token\bgroup
    % no space at start
    \expandafter\KVO@removelastspace\expandafter=%
    % "<value><spaceopt>= \@nil"
  \else
    \expandafter\KVO@checkfirstA
  \fi
}
\def\KVO@checkfirstA#1 #2\@nil{%
  \KVO@ifempty{#2}{%
    \KVO@removelastspace=#1 \@nil
  }{%
    \KVO@ifempty{#1}{%
      \KVO@removelastspace=#2\@nil
    }{%
      \KVO@removelastspace=#1 #2\@nil
    }%
  }%
}
\def\KVO@removelastspace#1 = \@nil|#2\KVO@nil{%
  \KVO@ifempty{#2}{%
    \edef\KVO@result{%
      \etex@unexpanded\expandafter{\KVO@result}%
      \etex@unexpanded\expandafter{\KVO@removegarbage#1\KVO@nil}%
    }%
  }{%
    \edef\KVO@result{%
      \etex@unexpanded\expandafter{\KVO@result}%
      \etex@unexpanded{#1}%
    }%
  }%
}
\def\KVO@removegarbage#1= \@nil#2\KVO@nil{#1}%
%    \end{macrocode}
%
%    Arguments |#1| and |#2| are macros.
%    \begin{macrocode}
\def\KVO@removeelement#1#2{%
  \begingroup
    \toks@={}%
    \@for\x:=#2\do{%
      \ifx\x\@empty
      \else
        \ifx\x#1\relax
        \else
          \edef\t{\the\toks@}%
          \ifx\t\@empty
          \else
            \toks@\expandafter{\the\toks@,}%
          \fi
          \toks@\expandafter{\the\expandafter\toks@\x}%
        \fi
      \fi
    }%
    \edef\x{\endgroup
      \def\noexpand#2{\the\toks@}%
    }%
  \x
}
%    \end{macrocode}
%
%
%    \begin{macrocode}
\let\@@fileswith@pti@ns\KVO@fileswith@pti@ns
\ifx\@fileswith@pti@ns\@badrequireerror
\else
  \let\@fileswith@pti@ns\KVO@fileswith@pti@ns
\fi
%    \end{macrocode}
%
%    \begin{macro}{\KVO@Patch}
%    \begin{macrocode}
\let\KVO@Patch=Y
%    \end{macrocode}
%    \end{macro}
%
%    \begin{macrocode}
\KVO@AtEnd%
%</patch>
%    \end{macrocode}
%
% \section{Test}
%
% \subsection{Preface for standard catcode check}
%
%    \begin{macrocode}
%<*test1>
\input miniltx.tex\relax
%</test1>
%    \end{macrocode}
%
% \subsection{Catcode checks for loading}
%
%    \begin{macrocode}
%<*test1>
%    \end{macrocode}
%    \begin{macrocode}
\catcode`\{=1 %
\catcode`\}=2 %
\catcode`\#=6 %
\catcode`\@=11 %
\expandafter\ifx\csname count@\endcsname\relax
  \countdef\count@=255 %
\fi
\expandafter\ifx\csname @gobble\endcsname\relax
  \long\def\@gobble#1{}%
\fi
\expandafter\ifx\csname @firstofone\endcsname\relax
  \long\def\@firstofone#1{#1}%
\fi
\expandafter\ifx\csname loop\endcsname\relax
  \expandafter\@firstofone
\else
  \expandafter\@gobble
\fi
{%
  \def\loop#1\repeat{%
    \def\body{#1}%
    \iterate
  }%
  \def\iterate{%
    \body
      \let\next\iterate
    \else
      \let\next\relax
    \fi
    \next
  }%
  \let\repeat=\fi
}%
\def\RestoreCatcodes{}
\count@=0 %
\loop
  \edef\RestoreCatcodes{%
    \RestoreCatcodes
    \catcode\the\count@=\the\catcode\count@\relax
  }%
\ifnum\count@<255 %
  \advance\count@ 1 %
\repeat

\def\RangeCatcodeInvalid#1#2{%
  \count@=#1\relax
  \loop
    \catcode\count@=15 %
  \ifnum\count@<#2\relax
    \advance\count@ 1 %
  \repeat
}
\def\RangeCatcodeCheck#1#2#3{%
  \count@=#1\relax
  \loop
    \ifnum#3=\catcode\count@
    \else
      \errmessage{%
        Character \the\count@\space
        with wrong catcode \the\catcode\count@\space
        instead of \number#3%
      }%
    \fi
  \ifnum\count@<#2\relax
    \advance\count@ 1 %
  \repeat
}
\def\space{ }
\expandafter\ifx\csname LoadCommand\endcsname\relax
  \def\LoadCommand{\input kvoptions.sty\relax}%
\fi
\def\Test{%
  \RangeCatcodeInvalid{0}{47}%
  \RangeCatcodeInvalid{58}{64}%
  \RangeCatcodeInvalid{91}{96}%
  \RangeCatcodeInvalid{123}{255}%
  \catcode`\@=12 %
  \catcode`\\=0 %
  \catcode`\%=14 %
  \LoadCommand
  \RangeCatcodeCheck{0}{36}{15}%
  \RangeCatcodeCheck{37}{37}{14}%
  \RangeCatcodeCheck{38}{47}{15}%
  \RangeCatcodeCheck{48}{57}{12}%
  \RangeCatcodeCheck{58}{63}{15}%
  \RangeCatcodeCheck{64}{64}{12}%
  \RangeCatcodeCheck{65}{90}{11}%
  \RangeCatcodeCheck{91}{91}{15}%
  \RangeCatcodeCheck{92}{92}{0}%
  \RangeCatcodeCheck{93}{96}{15}%
  \RangeCatcodeCheck{97}{122}{11}%
  \RangeCatcodeCheck{123}{255}{15}%
  \RestoreCatcodes
}
\Test
\csname @@end\endcsname
\end
%    \end{macrocode}
%    \begin{macrocode}
%</test1>
%    \end{macrocode}
%
%    \begin{macrocode}
%<*test2>
\NeedsTeXFormat{LaTeX2e}
\makeatletter
\catcode`\@=11 %
\def\RestoreCatcodes{}
\count@=0 %
\loop
  \edef\RestoreCatcodes{%
    \RestoreCatcodes
    \catcode\the\count@=\the\catcode\count@\relax
  }%
\ifnum\count@<255 %
  \advance\count@\@ne
\repeat

\def\RangeCatcodeInvalid#1#2{%
  \count@=#1\relax
  \loop
    \catcode\count@=15 %
  \ifnum\count@<#2\relax
    \advance\count@\@ne
  \repeat
}
\def\Test#1{%
  \RangeCatcodeInvalid{0}{47}%
  \RangeCatcodeInvalid{58}{64}%
  \RangeCatcodeInvalid{91}{96}%
  \RangeCatcodeInvalid{123}{255}%
  \catcode`\@=12 %
  \catcode`\\=0 %
  \catcode`\{=1 %
  \catcode`\}=2 %
  \catcode`\#=6 %
  \catcode`\[=12 %
  \catcode`\]=12 %
  \catcode`\%=14 %
  \catcode`\ =10 %
  \catcode13=5 %
  #1\relax
  \RestoreCatcodes
}
\Test{\RequirePackage{kvoptions-patch}}%
\Test{\RequirePackage{kvoptions}}%
\csname @@end\endcsname
%</test2>
%    \end{macrocode}
%
%    \begin{macrocode}
%<*test3>
\NeedsTeXFormat{LaTeX2e}
\makeatletter
\RequirePackage{kvoptions}[2016/05/16]
\def\msg#{\immediate\write16}
\define@key{testfamily}{testkey}{%
  \msg{[testfamily/testkey/#1]}%
}
\define@key{testfamily}{testdefaultkey}[testdefault]{%
  \msg{[testfamily/testdefaultkey/#1]}%
}
\AddToKeyvalOption{testfamily}{testkey}{%
  \msg{[addition/#1]}%
}
\AddToKeyvalOption{testfamily}{testdefaultkey}{%
  \msg{[addition/#1]}%
}
\setkeys{testfamily}{%
  testkey=testA,%
  testdefaultkey=testB,%
  testdefaultkey,%
}
\SetupKeyvalOptions{%
  family=testfamily%
}
\AddToKeyvalOption*{testkey}{%
  \msg{[star addition/#1]}%
}
\AddToKeyvalOption*{testdefaultkey}{%
  \msg{[star addition/#1]}%
}
\setkeys{testfamily}{%
  testkey=testA,%
  testdefaultkey=testB,%
  testdefaultkey,%
}
\@@end
%</test3>
%    \end{macrocode}
%
%    \begin{macrocode}
%<*test4pkg>
\NeedsTeXFormat{LaTeX2e}
\ProvidesPackage{kvoptions-test4}[2016/05/16 package for testing]
\RequirePackage{kvoptions}[2016/05/16]
\SetupKeyvalOptions{%
  family=FOO,%
  prefix=foo,%
  setkeys=\kvsetkeys,%
}
\DeclareStringOption{str}
\define@key{FOO}{set}{%
  \setkeys{BAR}{strbar={#1}}%
}
\define@key{BAR}{strbar}{%
  \def\foostr{[BAR:#1]}%
}
\ProcessKeyvalOptions*
%</test4pkg>
%    \end{macrocode}
%    \begin{macrocode}
%<*test4>
\NeedsTeXFormat{LaTeX2e}
\ProvidesFile{kvoptions-test4.tex}[2016/05/16 test file]
\RequirePackage[%
  str=A,set=B,str=C,%
]{kvoptions-test4}[2016/05/16]
\def\TestExpected{C}
\ifx\foostr\TestExpected
  \typeout{* Test ok.}%
\else
  \typeout{* Result: [\foostr]}%
  \typeout{* Expected: [\TestExpected]}%
  \errmessage{Test failed!}%
\fi
\csname @@end\endcsname\end
%</test4>
%    \end{macrocode}
%
% \section{Installation}
%
% \subsection{Download}
%
% \paragraph{Package.} This package is available on
% CTAN\footnote{\CTANpkg{kvoptions}}:
% \begin{description}
% \item[\CTAN{macros/latex/contrib/oberdiek/kvoptions.dtx}] The source file.
% \item[\CTAN{macros/latex/contrib/oberdiek/kvoptions.pdf}] Documentation.
% \end{description}
%
%
% \paragraph{Bundle.} All the packages of the bundle `oberdiek'
% are also available in a TDS compliant ZIP archive. There
% the packages are already unpacked and the documentation files
% are generated. The files and directories obey the TDS standard.
% \begin{description}
% \item[\CTANinstall{install/macros/latex/contrib/oberdiek.tds.zip}]
% \end{description}
% \emph{TDS} refers to the standard ``A Directory Structure
% for \TeX\ Files'' (\CTAN{tds/tds.pdf}). Directories
% with \xfile{texmf} in their name are usually organized this way.
%
% \subsection{Bundle installation}
%
% \paragraph{Unpacking.} Unpack the \xfile{oberdiek.tds.zip} in the
% TDS tree (also known as \xfile{texmf} tree) of your choice.
% Example (linux):
% \begin{quote}
%   |unzip oberdiek.tds.zip -d ~/texmf|
% \end{quote}
%
% \paragraph{Script installation.}
% Check the directory \xfile{TDS:scripts/oberdiek/} for
% scripts that need further installation steps.
%
% \subsection{Package installation}
%
% \paragraph{Unpacking.} The \xfile{.dtx} file is a self-extracting
% \docstrip\ archive. The files are extracted by running the
% \xfile{.dtx} through \plainTeX:
% \begin{quote}
%   \verb|tex kvoptions.dtx|
% \end{quote}
%
% \paragraph{TDS.} Now the different files must be moved into
% the different directories in your installation TDS tree
% (also known as \xfile{texmf} tree):
% \begin{quote}
% \def\t{^^A
% \begin{tabular}{@{}>{\ttfamily}l@{ $\rightarrow$ }>{\ttfamily}l@{}}
%   kvoptions.sty & tex/latex/oberdiek/kvoptions.sty\\
%   kvoptions-patch.sty & tex/latex/oberdiek/kvoptions-patch.sty\\
%   kvoptions.pdf & doc/latex/oberdiek/kvoptions.pdf\\
%   example-mycolorsetup.sty & doc/latex/oberdiek/example-mycolorsetup.sty\\
%   test/kvoptions-test1.tex & doc/latex/oberdiek/test/kvoptions-test1.tex\\
%   test/kvoptions-test2.tex & doc/latex/oberdiek/test/kvoptions-test2.tex\\
%   test/kvoptions-test3.tex & doc/latex/oberdiek/test/kvoptions-test3.tex\\
%   test/kvoptions-test4.tex & doc/latex/oberdiek/test/kvoptions-test4.tex\\
%   test/kvoptions-test4.sty & doc/latex/oberdiek/test/kvoptions-test4.sty\\
%   kvoptions.dtx & source/latex/oberdiek/kvoptions.dtx\\
% \end{tabular}^^A
% }^^A
% \sbox0{\t}^^A
% \ifdim\wd0>\linewidth
%   \begingroup
%     \advance\linewidth by\leftmargin
%     \advance\linewidth by\rightmargin
%   \edef\x{\endgroup
%     \def\noexpand\lw{\the\linewidth}^^A
%   }\x
%   \def\lwbox{^^A
%     \leavevmode
%     \hbox to \linewidth{^^A
%       \kern-\leftmargin\relax
%       \hss
%       \usebox0
%       \hss
%       \kern-\rightmargin\relax
%     }^^A
%   }^^A
%   \ifdim\wd0>\lw
%     \sbox0{\small\t}^^A
%     \ifdim\wd0>\linewidth
%       \ifdim\wd0>\lw
%         \sbox0{\footnotesize\t}^^A
%         \ifdim\wd0>\linewidth
%           \ifdim\wd0>\lw
%             \sbox0{\scriptsize\t}^^A
%             \ifdim\wd0>\linewidth
%               \ifdim\wd0>\lw
%                 \sbox0{\tiny\t}^^A
%                 \ifdim\wd0>\linewidth
%                   \lwbox
%                 \else
%                   \usebox0
%                 \fi
%               \else
%                 \lwbox
%               \fi
%             \else
%               \usebox0
%             \fi
%           \else
%             \lwbox
%           \fi
%         \else
%           \usebox0
%         \fi
%       \else
%         \lwbox
%       \fi
%     \else
%       \usebox0
%     \fi
%   \else
%     \lwbox
%   \fi
% \else
%   \usebox0
% \fi
% \end{quote}
% If you have a \xfile{docstrip.cfg} that configures and enables \docstrip's
% TDS installing feature, then some files can already be in the right
% place, see the documentation of \docstrip.
%
% \subsection{Refresh file name databases}
%
% If your \TeX~distribution
% (\TeX\,Live, \mikTeX, \dots) relies on file name databases, you must refresh
% these. For example, \TeX\,Live\ users run \verb|texhash| or
% \verb|mktexlsr|.
%
% \subsection{Some details for the interested}
%
% \paragraph{Unpacking with \LaTeX.}
% The \xfile{.dtx} chooses its action depending on the format:
% \begin{description}
% \item[\plainTeX:] Run \docstrip\ and extract the files.
% \item[\LaTeX:] Generate the documentation.
% \end{description}
% If you insist on using \LaTeX\ for \docstrip\ (really,
% \docstrip\ does not need \LaTeX), then inform the autodetect routine
% about your intention:
% \begin{quote}
%   \verb|latex \let\install=y\input{kvoptions.dtx}|
% \end{quote}
% Do not forget to quote the argument according to the demands
% of your shell.
%
% \paragraph{Generating the documentation.}
% You can use both the \xfile{.dtx} or the \xfile{.drv} to generate
% the documentation. The process can be configured by the
% configuration file \xfile{ltxdoc.cfg}. For instance, put this
% line into this file, if you want to have A4 as paper format:
% \begin{quote}
%   \verb|\PassOptionsToClass{a4paper}{article}|
% \end{quote}
% An example follows how to generate the
% documentation with pdf\LaTeX:
% \begin{quote}
%\begin{verbatim}
%pdflatex kvoptions.dtx
%makeindex -s gind.ist kvoptions.idx
%pdflatex kvoptions.dtx
%makeindex -s gind.ist kvoptions.idx
%pdflatex kvoptions.dtx
%\end{verbatim}
% \end{quote}
%
% \begin{thebibliography}{99}
% \bibitem{tb94wright}
%   A guide to key-value methods, Joseph Wright, second draft for
%   \href{https://www.tug.org/tugboat}{TUGBoat}, 2009-03-17.
%   \url{https://www.texdev.net/uploads/2009/03/keyval.pdf}
% \bibpackage{ifthen}{David Carlisle}{2001/05/26}
%   {\CTANpkg{ifthen}}
% \bibpackage{helvet}{Sebastian Rahtz, Walter Schmidt}{2004/01/26}
%   {\CTANpkg{psfonts}}
% \bibpackage{hyperref}{Sebastian Rahtz, Heiko Oberdiek}{2006/02/12}
%   {\CTANpkg{hyperref}}
% \bibpackage{keyval}{David Carlisle}{1999/03/16}
%   {\CTANpkg{keyval}}
% \bibpackage{multicol}{Frank Mittelbach}{2004/02/14}
%   {\CTANpkg{multicol}}
% \bibpackage{tabularx}{David Carlisle}{1999/01/07}
%   {\CTANpkg{tabularx}}
% \bibpackage{tracefnt}{Frank Mittelbach, Rainer Sch\"opf}{1997/05/29}
%   {\CTANpkg{latex-base}}
% \bibpackage{xkeyval}{Hendri Adriaens}{2005/05/07}
%   {\CTANpkg{xkeyval}}
% \bibitem{clsguide}
%   The \LaTeX3 Project, \textit{\LaTeXe\ for class and package writers},
%   2003/12/09.
%   \CTANpkg{clsguide}
%
% \end{thebibliography}
%
% \begin{History}
%   \begin{Version}{0000/00/00 v0.0}
%   \item
%     Probably David Carlisle's code in \xpackage{hyperref}
%     was the start.
%   \end{Version}
%   \begin{Version}{2004/02/22 v1.0}
%   \item
%     The first version was never published. It also
%     has offered a patch to get rid of \LaTeX's option
%     expansion.
%   \end{Version}
%   \begin{Version}{2006/02/16 v2.0}
%   \item
%     Now the package is redesigned with an easier
%     user interface.
%   \item
%     \cs{ProcessKeyvalOptions} remains the central service, inherited
%     from \xpackage{hyperref}'s \cs{ProcessOptionsWithKV}.
%     Now the use inside classes is also supported.
%   \item
%     Provides help macros for boolean and simple string options.
%   \item
%     Fixes for the patch of \LaTeX{}. The patch is only enabled,
%     if the user requests it.
%   \end{Version}
%   \begin{Version}{2006/02/20 v2.1}
%   \item
%     Unused option list is sanitized to prevent problems
%     with other packages that uses own processing methods
%     for key value options. Disadvantage: the unused global
%     option detection is weakened.
%   \item
%     New option type by \cs{DeclareVoidOption} for options without
%     value.
%   \item
%     Default rule by \cs{DeclareDefaultOption}.
%   \item
%     Dynamic options: \cs{DisableKeyvalOption}.
%   \end{Version}
%   \begin{Version}{2006/06/01 v2.2}
%   \item
%     Fixes for option \xoption{patch}.
%   \end{Version}
%   \begin{Version}{2006/08/17 v2.3}
%   \item
%     \cs{DeclareBooleanOption} renamed to \cs{DeclareBoolOption}
%     to avoid a name clash with package \cs{ifoption}.
%   \end{Version}
%   \begin{Version}{2006/08/22 v2.4}
%   \item
%     Option \xoption{patch}: \cs{ExecuteOptions} does not
%     change the meaning of macro \cs{CurrentOption} at all.
%   \end{Version}
%   \begin{Version}{2007/04/11 v2.5}
%   \item
%     Line ends sanitized.
%   \end{Version}
%   \begin{Version}{2007/05/06 v2.6}
%   \item
%     Uses package \xpackage{etexcmds}.
%   \end{Version}
%   \begin{Version}{2007/06/11 v2.7}
%   \item
%     The patch part fixes LaTeX bug latex/3965.
%   \end{Version}
%   \begin{Version}{2007/10/02 v2.8}
%   \item
%     Compatibility for \plainTeX\ added.
%   \item
%     Typos in documentation fixed (Axel Sommerfeldt).
%   \end{Version}
%   \begin{Version}{2007/10/11 v2.9}
%   \item
%     Bug fix for option \xoption{patch}.
%   \end{Version}
%   \begin{Version}{2007/10/18 v3.0}
%   \item
%     New package \xpackage{kvoptions-patch}.
%   \end{Version}
%   \begin{Version}{2009/04/10 v3.1}
%   \item
%     Space by line end removed in definition of internal macro.
%   \end{Version}
%   \begin{Version}{2009/07/17 v3.2}
%   \item
%     \cs{ProcessLocalKeyvalOptions} added.
%   \item
%     \cs{DisableKeyvalOption} with the \texttt{action=ignore} option
%     fixed (Joseph Wright).
%   \end{Version}
%   \begin{Version}{2009/07/21 v3.3}
%   \item
%     \cs{DeclareLocalOption}, \cs{DeclareLocalOptions} added.
%   \end{Version}
%   \begin{Version}{2009/08/13 v3.4}
%   \item
%     Documentation addition: recommendation for Joseph Wright's
%     review article.
%   \item
%     Documentation addition: local/global options.
%   \end{Version}
%   \begin{Version}{2009/12/04 v3.5}
%   \item
%     \cs{AddToKeyvalOption} added.
%   \end{Version}
%   \begin{Version}{2009/12/08 v3.6}
%   \item
%     Fix: If a default handler is configured, it is now also
%     called for classes.
%   \end{Version}
%   \begin{Version}{2010/02/22 v3.7}
%   \item
%     Missing space in error message added.
%   \end{Version}
%   \begin{Version}{2010/07/23 v3.8}
%   \item
%     Documenation for package \xpackage{kvoptions-patch} improved.
%     No code changes.
%   \end{Version}
%   \begin{Version}{2010/12/02 v3.9}
%   \item
%     Key \texttt{setkeys} added for \cs{SetupKeyvalOptions}.
%   \end{Version}
%   \begin{Version}{2010/12/23 v3.10}
%   \item
%     \cs{DeclareVoidOption} also parses the second parameter as
%     \hologo{TeX} argument to improve compatibility with
%     \cs{DeclareOption}.
%   \end{Version}
%   \begin{Version}{2011/06/30 v3.11}
%   \item
%     Fix because of design bug in package \xpackage{xkeyval} that
%     removes global options with equal signs.
%   \end{Version}
%   \begin{Version}{2016/05/16 v3.12}
%   \item
%     Documentation updates.
%   \end{Version}
% \end{History}
%
% \PrintIndex
%
% \Finale
\endinput
|
% \end{quote}
% Do not forget to quote the argument according to the demands
% of your shell.
%
% \paragraph{Generating the documentation.}
% You can use both the \xfile{.dtx} or the \xfile{.drv} to generate
% the documentation. The process can be configured by the
% configuration file \xfile{ltxdoc.cfg}. For instance, put this
% line into this file, if you want to have A4 as paper format:
% \begin{quote}
%   \verb|\PassOptionsToClass{a4paper}{article}|
% \end{quote}
% An example follows how to generate the
% documentation with pdf\LaTeX:
% \begin{quote}
%\begin{verbatim}
%pdflatex kvoptions.dtx
%makeindex -s gind.ist kvoptions.idx
%pdflatex kvoptions.dtx
%makeindex -s gind.ist kvoptions.idx
%pdflatex kvoptions.dtx
%\end{verbatim}
% \end{quote}
%
% \begin{thebibliography}{99}
% \bibitem{tb94wright}
%   A guide to key-value methods, Joseph Wright, second draft for
%   \href{https://www.tug.org/tugboat}{TUGBoat}, 2009-03-17.
%   \url{https://www.texdev.net/uploads/2009/03/keyval.pdf}
% \bibpackage{ifthen}{David Carlisle}{2001/05/26}
%   {\CTANpkg{ifthen}}
% \bibpackage{helvet}{Sebastian Rahtz, Walter Schmidt}{2004/01/26}
%   {\CTANpkg{psfonts}}
% \bibpackage{hyperref}{Sebastian Rahtz, Heiko Oberdiek}{2006/02/12}
%   {\CTANpkg{hyperref}}
% \bibpackage{keyval}{David Carlisle}{1999/03/16}
%   {\CTANpkg{keyval}}
% \bibpackage{multicol}{Frank Mittelbach}{2004/02/14}
%   {\CTANpkg{multicol}}
% \bibpackage{tabularx}{David Carlisle}{1999/01/07}
%   {\CTANpkg{tabularx}}
% \bibpackage{tracefnt}{Frank Mittelbach, Rainer Sch\"opf}{1997/05/29}
%   {\CTANpkg{latex-base}}
% \bibpackage{xkeyval}{Hendri Adriaens}{2005/05/07}
%   {\CTANpkg{xkeyval}}
% \bibitem{clsguide}
%   The \LaTeX3 Project, \textit{\LaTeXe\ for class and package writers},
%   2003/12/09.
%   \CTANpkg{clsguide}
%
% \end{thebibliography}
%
% \begin{History}
%   \begin{Version}{0000/00/00 v0.0}
%   \item
%     Probably David Carlisle's code in \xpackage{hyperref}
%     was the start.
%   \end{Version}
%   \begin{Version}{2004/02/22 v1.0}
%   \item
%     The first version was never published. It also
%     has offered a patch to get rid of \LaTeX's option
%     expansion.
%   \end{Version}
%   \begin{Version}{2006/02/16 v2.0}
%   \item
%     Now the package is redesigned with an easier
%     user interface.
%   \item
%     \cs{ProcessKeyvalOptions} remains the central service, inherited
%     from \xpackage{hyperref}'s \cs{ProcessOptionsWithKV}.
%     Now the use inside classes is also supported.
%   \item
%     Provides help macros for boolean and simple string options.
%   \item
%     Fixes for the patch of \LaTeX{}. The patch is only enabled,
%     if the user requests it.
%   \end{Version}
%   \begin{Version}{2006/02/20 v2.1}
%   \item
%     Unused option list is sanitized to prevent problems
%     with other packages that uses own processing methods
%     for key value options. Disadvantage: the unused global
%     option detection is weakened.
%   \item
%     New option type by \cs{DeclareVoidOption} for options without
%     value.
%   \item
%     Default rule by \cs{DeclareDefaultOption}.
%   \item
%     Dynamic options: \cs{DisableKeyvalOption}.
%   \end{Version}
%   \begin{Version}{2006/06/01 v2.2}
%   \item
%     Fixes for option \xoption{patch}.
%   \end{Version}
%   \begin{Version}{2006/08/17 v2.3}
%   \item
%     \cs{DeclareBooleanOption} renamed to \cs{DeclareBoolOption}
%     to avoid a name clash with package \cs{ifoption}.
%   \end{Version}
%   \begin{Version}{2006/08/22 v2.4}
%   \item
%     Option \xoption{patch}: \cs{ExecuteOptions} does not
%     change the meaning of macro \cs{CurrentOption} at all.
%   \end{Version}
%   \begin{Version}{2007/04/11 v2.5}
%   \item
%     Line ends sanitized.
%   \end{Version}
%   \begin{Version}{2007/05/06 v2.6}
%   \item
%     Uses package \xpackage{etexcmds}.
%   \end{Version}
%   \begin{Version}{2007/06/11 v2.7}
%   \item
%     The patch part fixes LaTeX bug latex/3965.
%   \end{Version}
%   \begin{Version}{2007/10/02 v2.8}
%   \item
%     Compatibility for \plainTeX\ added.
%   \item
%     Typos in documentation fixed (Axel Sommerfeldt).
%   \end{Version}
%   \begin{Version}{2007/10/11 v2.9}
%   \item
%     Bug fix for option \xoption{patch}.
%   \end{Version}
%   \begin{Version}{2007/10/18 v3.0}
%   \item
%     New package \xpackage{kvoptions-patch}.
%   \end{Version}
%   \begin{Version}{2009/04/10 v3.1}
%   \item
%     Space by line end removed in definition of internal macro.
%   \end{Version}
%   \begin{Version}{2009/07/17 v3.2}
%   \item
%     \cs{ProcessLocalKeyvalOptions} added.
%   \item
%     \cs{DisableKeyvalOption} with the \texttt{action=ignore} option
%     fixed (Joseph Wright).
%   \end{Version}
%   \begin{Version}{2009/07/21 v3.3}
%   \item
%     \cs{DeclareLocalOption}, \cs{DeclareLocalOptions} added.
%   \end{Version}
%   \begin{Version}{2009/08/13 v3.4}
%   \item
%     Documentation addition: recommendation for Joseph Wright's
%     review article.
%   \item
%     Documentation addition: local/global options.
%   \end{Version}
%   \begin{Version}{2009/12/04 v3.5}
%   \item
%     \cs{AddToKeyvalOption} added.
%   \end{Version}
%   \begin{Version}{2009/12/08 v3.6}
%   \item
%     Fix: If a default handler is configured, it is now also
%     called for classes.
%   \end{Version}
%   \begin{Version}{2010/02/22 v3.7}
%   \item
%     Missing space in error message added.
%   \end{Version}
%   \begin{Version}{2010/07/23 v3.8}
%   \item
%     Documenation for package \xpackage{kvoptions-patch} improved.
%     No code changes.
%   \end{Version}
%   \begin{Version}{2010/12/02 v3.9}
%   \item
%     Key \texttt{setkeys} added for \cs{SetupKeyvalOptions}.
%   \end{Version}
%   \begin{Version}{2010/12/23 v3.10}
%   \item
%     \cs{DeclareVoidOption} also parses the second parameter as
%     \hologo{TeX} argument to improve compatibility with
%     \cs{DeclareOption}.
%   \end{Version}
%   \begin{Version}{2011/06/30 v3.11}
%   \item
%     Fix because of design bug in package \xpackage{xkeyval} that
%     removes global options with equal signs.
%   \end{Version}
%   \begin{Version}{2016/05/16 v3.12}
%   \item
%     Documentation updates.
%   \end{Version}
% \end{History}
%
% \PrintIndex
%
% \Finale
\endinput

%        (quote the arguments according to the demands of your shell)
%
% Documentation:
%    (a) If kvoptions.drv is present:
%           latex kvoptions.drv
%    (b) Without kvoptions.drv:
%           latex kvoptions.dtx; ...
%    The class ltxdoc loads the configuration file ltxdoc.cfg
%    if available. Here you can specify further options, e.g.
%    use A4 as paper format:
%       \PassOptionsToClass{a4paper}{article}
%
%    Programm calls to get the documentation (example):
%       pdflatex kvoptions.dtx
%       makeindex -s gind.ist kvoptions.idx
%       pdflatex kvoptions.dtx
%       makeindex -s gind.ist kvoptions.idx
%       pdflatex kvoptions.dtx
%
% Installation:
%    TDS:tex/latex/oberdiek/kvoptions.sty
%    TDS:tex/latex/oberdiek/kvoptions-patch.sty
%    TDS:doc/latex/oberdiek/kvoptions.pdf
%    TDS:doc/latex/oberdiek/example-mycolorsetup.sty
%    TDS:doc/latex/oberdiek/test/kvoptions-test1.tex
%    TDS:doc/latex/oberdiek/test/kvoptions-test2.tex
%    TDS:doc/latex/oberdiek/test/kvoptions-test3.tex
%    TDS:doc/latex/oberdiek/test/kvoptions-test4.tex
%    TDS:doc/latex/oberdiek/test/kvoptions-test4.sty
%    TDS:source/latex/oberdiek/kvoptions.dtx
%
%<*ignore>
\begingroup
  \catcode123=1 %
  \catcode125=2 %
  \def\x{LaTeX2e}%
\expandafter\endgroup
\ifcase 0\ifx\install y1\fi\expandafter
         \ifx\csname processbatchFile\endcsname\relax\else1\fi
         \ifx\fmtname\x\else 1\fi\relax
\else\csname fi\endcsname
%</ignore>
%<*install>
\input docstrip.tex
\Msg{************************************************************************}
\Msg{* Installation}
\Msg{* Package: kvoptions 2016/05/16 v3.12 Key value format for package options (HO)}
\Msg{************************************************************************}

\keepsilent
\askforoverwritefalse

\let\MetaPrefix\relax
\preamble

This is a generated file.

Project: kvoptions
Version: 2016/05/16 v3.12

Copyright (C)
   2004, 2006, 2007, 2009-2011 Heiko Oberdiek
   2016-2019 Oberdiek Package Support Group

This work may be distributed and/or modified under the
conditions of the LaTeX Project Public License, either
version 1.3c of this license or (at your option) any later
version. This version of this license is in
   https://www.latex-project.org/lppl/lppl-1-3c.txt
and the latest version of this license is in
   https://www.latex-project.org/lppl.txt
and version 1.3 or later is part of all distributions of
LaTeX version 2005/12/01 or later.

This work has the LPPL maintenance status "maintained".

The Current Maintainers of this work are
Heiko Oberdiek and the Oberdiek Package Support Group
https://github.com/ho-tex/oberdiek/issues


This work consists of the main source file kvoptions.dtx
and the derived files
   kvoptions.sty, kvoptions.pdf, kvoptions.ins, kvoptions.drv,
   kvoptions-patch.sty, example-mycolorsetup.sty,
   kvoptions-test1.tex, kvoptions-test2.tex,
   kvoptions-test3.tex, kvoptions-test4.tex,
   kvoptions-test4.sty.

\endpreamble
\let\MetaPrefix\DoubleperCent

\generate{%
  \file{kvoptions.ins}{\from{kvoptions.dtx}{install}}%
  \file{kvoptions.drv}{\from{kvoptions.dtx}{driver}}%
  \usedir{tex/latex/oberdiek}%
  \file{kvoptions.sty}{\from{kvoptions.dtx}{package}}%
  \file{kvoptions-patch.sty}{\from{kvoptions.dtx}{patch}}%
  \usedir{doc/latex/oberdiek}%
  \file{example-mycolorsetup.sty}{\from{kvoptions.dtx}{example}}%
%  \usedir{doc/latex/oberdiek/test}%
%  \file{kvoptions-test1.tex}{\from{kvoptions.dtx}{test1}}%
%  \file{kvoptions-test2.tex}{\from{kvoptions.dtx}{test2}}%
%  \file{kvoptions-test3.tex}{\from{kvoptions.dtx}{test3}}%
%  \file{kvoptions-test4.tex}{\from{kvoptions.dtx}{test4}}%
%  \file{kvoptions-test4.sty}{\from{kvoptions.dtx}{test4pkg}}%
  \nopreamble
  \nopostamble
%  \usedir{source/latex/oberdiek/catalogue}%
%  \file{kvoptions.xml}{\from{kvoptions.dtx}{catalogue}}%
}

\catcode32=13\relax% active space
\let =\space%
\Msg{************************************************************************}
\Msg{*}
\Msg{* To finish the installation you have to move the following}
\Msg{* files into a directory searched by TeX:}
\Msg{*}
\Msg{*     kvoptions.sty, kvoptions-patch.sty}
\Msg{*}
\Msg{* To produce the documentation run the file `kvoptions.drv'}
\Msg{* through LaTeX.}
\Msg{*}
\Msg{* Happy TeXing!}
\Msg{*}
\Msg{************************************************************************}

\endbatchfile
%</install>
%<*ignore>
\fi
%</ignore>
%<*driver>
\NeedsTeXFormat{LaTeX2e}
\ProvidesFile{kvoptions.drv}%
  [2016/05/16 v3.12 Key value format for package options (HO)]%
\documentclass{ltxdoc}
\usepackage{holtxdoc}[2011/11/22]
\begin{document}
  \DocInput{kvoptions.dtx}%
\end{document}
%</driver>
% \fi
%
%
% \CharacterTable
%  {Upper-case    \A\B\C\D\E\F\G\H\I\J\K\L\M\N\O\P\Q\R\S\T\U\V\W\X\Y\Z
%   Lower-case    \a\b\c\d\e\f\g\h\i\j\k\l\m\n\o\p\q\r\s\t\u\v\w\x\y\z
%   Digits        \0\1\2\3\4\5\6\7\8\9
%   Exclamation   \!     Double quote  \"     Hash (number) \#
%   Dollar        \$     Percent       \%     Ampersand     \&
%   Acute accent  \'     Left paren    \(     Right paren   \)
%   Asterisk      \*     Plus          \+     Comma         \,
%   Minus         \-     Point         \.     Solidus       \/
%   Colon         \:     Semicolon     \;     Less than     \<
%   Equals        \=     Greater than  \>     Question mark \?
%   Commercial at \@     Left bracket  \[     Backslash     \\
%   Right bracket \]     Circumflex    \^     Underscore    \_
%   Grave accent  \`     Left brace    \{     Vertical bar  \|
%   Right brace   \}     Tilde         \~}
%
% \GetFileInfo{kvoptions.drv}
%
% \title{The \xpackage{kvoptions} package}
% \date{2016/05/16 v3.12}
% \author{Heiko Oberdiek\thanks
% {Please report any issues at \url{https://github.com/ho-tex/oberdiek/issues}}}
%
% \maketitle
%
% \begin{abstract}
% This package is intended for package authors who want to
% use options in key value format for their package options.
% \end{abstract}
%
% \tableofcontents
%
% \def\M#1{\texttt{\{}\meta{#1}\texttt{\}}}
% \def\O#1{\texttt{[}\meta{#1}\texttt{]}}
%
% \section{Introduction}
%
% First I want to recommend the very good review article
% ``A guide to key-value methods'' by Joseph Wright \cite{tb94wright}.
% It introduces the different key-value packages and compares them.
%
% \subsection{The beginning}
%
% This package \xpackage{kvoptions} addresses class or package writers
% that want to allow their users to specify options
% as key value pairs, e.g.:
% \begin{quote}
% |\documentclass[verbose=false,name=me]{myclass}|\\
% |\usepackage[format=print]{mylayout}|
% \end{quote}
% Prominent example is package \xpackage{hyperref},
% probably the first package that offers this service.
% It's \cs{ProcessOptionsWithKV} is often copied und
% used in other packages, e.g. package \xpackage{helvet}
% that uses this interface for its option \xoption{scaled}.
%
% However copying code is not the most modern software
% development technique. And \xpackage{hyperref}'s
% code for \cs{ProcessOptionsWithKV} was changed
% to fix bugs. The version used in other packages
% depends on the time of copying and the awareness
% of \xpackage{hyperref}'s changes.
% Now the code is sourced out into this package and
% available for other package or class writers.
%
% \subsection{Overview}
%
% Package \xpackage{kvoptions} connects package
% \xpackage{\emph{k}ey\emph{v}al}
% with \LaTeX's package and class \emph{options}:
% \begin{center}
% \def\T#1{\bfseries #1}
% \renewcommand{\arraystretch}{1.5}
% \begin{tabular}{l|l|l}
%   \T{Package \xpackage{keyval}}
%   & \T{Package \xpackage{kvoptions}}
%   & \T{\LaTeX\ kernel}
%   \\
% \hline
%     \cs{define@key}
%   &
%     \renewcommand*{\arraystretch}{1}^^A
%     \begin{tabular}[t]{@{}l@{}}
%       \cs{DeclareVoidOption}\\
%       \cs{DeclareStringOption}\\
%       \cs{DeclareBoolOption}\\
%       \cs{DeclareComplementaryOption}\\
%       \cs{DisableKeyvalOption}
%     \end{tabular}
%   & \cs{DeclareOption}
%   \\
% \hline
%   & \cs{DeclareDefaultOption}
%   & \cs{DeclareOption*}
%   \\
% \hline
%   & \cs{ProcessKeyvalOptions}
%   & \cs{ProcessOptions*}
%   \\
% \hline
%   & Option \xoption{patch}
%   &
%     \renewcommand*{\arraystretch}{1}^^A
%     \begin{tabular}[t]{@{}l@{}}
%       Class/package\\
%       option system
%     \end{tabular}
%   \\
% \hline
%   & \cs{SetupKeyvalOptions} &
%   \\
% \end{tabular}
% \end{center}
%
% \hypersetup{bookmarksopenlevel=3}
% \section{Usage}
%
% \subsection{Process options}
%
% \subsubsection{\cs{ProcessKeyvalOptions}}
%
% \begin{declcs}{ProcessKeyvalOptions}\,\M{family}\\
%   \cs{ProcessKeyvalOptions} \*
% \end{declcs}
% This command evaluates the global or local options of
% the package that are defined with \xpackage{keyval}'s
% interface within the family \meta{family}. It acts the
% same way as \LaTeX's \cs{ProcessOptions*}. In a package
% unknown global options are ignored, in a class they
% are added to the unknown option list. The known global
% options and all local options are passed to \xpackage{keyval}'s
% \cs{setkeys} command for executing the options.
% Unknown options are reported to the user by an error.
%
% If the family name happens to be the same as the name
% of the package or class where \cs{ProcessKeyvalOptions}
% is used or the family name has previously been setup by
% \cs{SetupKeyvalOptions}, then \cs{ProcessKeyvalOptions}
% knows the family name already and you can use the star form
% without mandatory argument.
%
% \subsubsection{\cs{ProcessLocalKeyvalOptions}}
%
% \begin{declcs}{ProcessLocalKeyvalOptions}\,\M{family}\\
%   \cs{ProcessLocalKeyvalOptions} \*
% \end{declcs}
% This macro has the same syntax and works similar as
% \cs{ProcessKeyvalOptions}. However it ignores
% global options and only processes the local package options.
% Therefore it only can be used inside a package.
% An error is thrown, if it is used inside a class.
%
% Neither of the following macros are necessary for
% \cs{ProcessKeyvalOptions{}}. They just help the package/class
% author in common tasks.
%
% \subsubsection{\cs{SetupKeyvalOptions}}\label{setup}
%
% \begin{declcs}{SetupKeyvalOptions}\,\texttt{\{}\\
%   \hspace{1.5em}|family|\,|=|\,\meta{family}|,|\\
%   \hspace{1.5em}|prefix|\,|=|\,\meta{prefix}\\
%   \hspace{1.5em}|setkeys|\,|=|\,\meta{setkeys command}\\
%   \texttt{\}}
% \end{declcs}
% This command allows to configure the default assumptions
% that are based on the current package or class name.
% \LaTeX\ remembers this name in \cs{@currname}. The syntax
% description of the default looks a little weird, therefor
% an example is given for a package or class named |foobar|.
% \begin{quote}
% \begin{tabular}{@{}l|ll|l@{}}
%   Key & Default & (example) & Used by\\
% \hline
%   \rule{0pt}{1.2\ht\csname @arstrutbox\endcsname}^^A
%   |family| & \meta{\cs{@currname}} & (|foobar|)
%       & \cs{ProcessKeyvalOptions*} \\
%     &&& \cs{DeclareBoolOption} \\
%     &&& \cs{DeclareStringOption} \\
% \hline
%   \rule{0pt}{1.2\ht\csname @arstrutbox\endcsname}^^A
%   |prefix| & \meta{\cs{@currname}}|@| & (|foobar@|)
%       & \cs{DeclareBoolOption} \\
%     &&& \cs{DeclareStringOption} \\
%     &&& \cs{DeclareVoidOption} \\
% \hline
%   \rule{0pt}{1.2\ht\csname @arstrutbox\endcsname}^^A
%   |setkeys| & \cs{setkeys} & (\cs{kvsetkeys})
%       & \cs{ProcessKeyvalOptions} \\
%     &&& \cs{ProcessLocalKeyvalOptions}
% \end{tabular}
% \end{quote}
% Key |setkeys| was added in version 3.9. The original
% \cs{setkeys} of package \xpackage{keyval} is not reentrant.
% If an option is processed by this \cs{setkeys}, then
% the option should not call \cs{setkeys} again with a
% different family. Otherwise the next options of the
% first \cs{setkeys} call are processed with the wrong
% family. With key |setkeys| the macro \cs{kvsetkeys}
% can be set that does not have the problem of the
% original \cs{setkeys} of package {keyval}.
%
% Probably \cs{setkeys} of package \xpackage{xkeyval}
% is safe in this respect. But I haven't made a full
% analysis. At least it does not have the problem
% of the original \cs{setkeys}.
%
% \subsection{Option declarations}
%
% The options for \cs{ProcessKeyvalOptions} are defined
% by \xpackage{keyval}'s \cs{define@key}.
% Common purposes of such keys are boolean switches, they
% enable or disable something. Or they store a name or some kind of
% string in a macro. The following commands help the user. He
% declares what he wants and \xpackage{kvoptions} take care of
% the key definition, resource allocation and initialization.
%
% In order to avoid name clashes of macro names, internal
% commands are prefixed. Both the prefix and the family
% name for the defined keys can be configured by
% \cs{SetupKeyvalOptions}.
%
% \subsubsection{\cs{DeclareStringOption}}
%
% \begin{declcs}{DeclareStringOption}\,\O{init}\,\M{key}\,\O{default}
% \end{declcs}
% A macro is created that remembers the value of the key \meta{key}.
% The name of the macro consists of the option name \meta{key} that
% is prefixed by the prefix (see \ref{setup}).
% The initial contents of the
% macro can be given by the first optional argument \meta{init}.
% The default is empty.
%
% The the option \meta{key} is defined. The option code
% just stores its value in the macro. If the optional argument
% at the end of \cs{DeclareStringOption} is given, then option
% \meta{key} is defined with the default \meta{default}.
%
% Example for a package with the following two lines:
% \begin{quote}
% |\ProvidesPackage{foobar}|\\
% |\DeclareStringOption[me]{name}|
% \end{quote}
% Then \cs{DeclareStringOption} defines the macro with content |me|,
% note \LaTeX\ complains if the name of the macro already exists:
% \begin{quote}
% |\newcommand*{\foobar@name}{me}|
% \end{quote}
% The option definition is similar to:
% \begin{quote}
% |\define@key{foobar}{name}{%|\\
% |  \renewcommand*{\foobar@name}{#1}%|\\
% |}|
% \end{quote}
%
% \subsubsection{\cs{DeclareBoolOption}}
%
% \begin{declcs}{DeclareBoolOption}\,\O{init}\,\M{key}
% \end{declcs}
% A boolean switch is generated, initialized by value \meta{init}
% and the corresponding key \meta{key} is defined. If the
% initialization value is not given, |false| is used as default.
%
% The internal actions of \cs{DeclareBoolOption} are shown below.
% The example is given for a package author who has the following
% two lines in his package/class:
% \begin{quote}
% |\ProvidesPackage{foobar}|\\
% |\DeclareBoolOption{verbose}|
% \end{quote}
% First a new switch is created:
% \begin{quote}
%   |\newif\iffoobar@verbose|
% \end{quote}
% and initialized:
% \begin{quote}
%   |\foobar@verbosefalse|
% \end{quote}
% Finally the key is defined:
% \begin{quote}
%   |\define@key{foobar}{verbose}[true]{|\dots|}|
% \end{quote}
% The option code configures the boolean option in the following way:
% If the author specifies |true| or |false| then the switch is
% turned on or off respectivly. Also the option can be given without
% explicit value. Then the switch is enabled.
% Other values are reported as errors.
%
% Now the switch is ready to use in the package/class, e.g.:
% \begin{quote}
%   |\iffoobar@verbose|\\
%   \qquad\textit{\% print verbose message}\\
%   |\else|\\
%   \qquad\textit{\% be quiet}\\
%   |\fi|
% \end{quote}
% Users of package \cs{ifthen} can use the switch as
% boolean:
% \begin{quote}
%   |\boolean{foobar@verbose}|
% \end{quote}
%
% \subsubsection{\cs{DeclareComplementaryOption}}
%
% \begin{declcs}{DeclareComplementaryOption}\,\M{key}\,\M{parent}
% \end{declcs}
% Sometimes contrasting names are used to characterize the
% two states of a boolean switch, for example \xoption{draft}
% vs.\@ \xoption{final}. Both options behave like boolean
% options but they do not need two different switches, they
% should share one. \cs{DeclareComplementaryOption} allows this.
% The option \meta{key} shares the switch of option \meta{parent}.
% Example:
% \begin{quote}
%   |\DeclareBoolOption{draft}|\\
%   |\DeclareComplementaryOption{final}{draft}|
% \end{quote}
% Then |final| sets the switch of \xoption{draft} to |false|, and
% |final=false| enables the \xoption{draft} switch.
%
% \subsubsection{\cs{DeclareVoidOption}}
%
% \begin{declcs}{DeclareVoidOption}\,\M{key}\,\M{code}
% \end{declcs}
% \cs{ProcessKeyvalOptions} can be extended to recognize
% options that are declared in traditional way by \cs{DeclareOption}.
% But in case of the error that the user specifies a value, then
% this option would not recognized as key value option because
% of \cs{DeclareOption} and not detected as traditional option
% because of the value part. The user would get an unknown option
% error, difficult to understand.
%
% \cs{DeclareVoidOption} solves
% this problem. It defines the option \meta{key} as key value option.
% If the user specifies a value, a warning is given and the
% value is ignored.
%
% The code part \meta{code} is stored in a macro.
% The name of the macro consists of the option name \meta{key} that
% is prefixed by the prefix (see \ref{setup}).
% If the option is set, the macro will be executed. During the
% execution \cs{CurrentOption} is available with the current key name.
%
% \subsubsection{\cs{DeclareDefaultOption}}
%
% \begin{declcs}{DeclareDefaultOption}\,\M{code}
% \end{declcs}
% \begin{sloppypar}
% This command does not define a specific key, it is the equivalent
% to \LaTeX's \cs{DeclareOption*}. It allows the specification
% of a default action \meta{code} that is invoked if an unknown option
% is found.
% While \meta{code} is called,
% macro \cs{CurrentOption} contains the current option string.
% In addition \cs{CurrentOptionValue} contains the value part
% if the option string is parsable as key value pair, otherwise
% it is \cs{relax}. \cs{CurrentOptionKey} contains the key of
% the key value pair, or the whole option string, if it misses
% the equal sign.
% \end{sloppypar}
%
% Inside packages typical default actions are to pass unknown
% options to another package. Or an error message can be thrown
% by \cs{@unknownoptionerror}. This is the original error
% message that \LaTeX\ gives for unkown package options.
% This error message is easier to understand for the user
% as the error message from package \xpackage{keyval} that
% is given otherwise.
%
% A Class ignores unknown options and puts them on the
% unused option list. Let \LaTeX\ do the job and just
% call \cs{OptionNotUsed}. Or the options can be passed to
% another class that is later loaded.
%
% \subsubsection{Local options}
%
% \begin{declcs}{DeclareLocalOption} \M{option}\\
%   \cs{DeclareLocalOptions} \M{option list}
% \end{declcs}
% Both macros mark package options as local options. That means
% that they are ignored by \cs{ProcessKeyvalOptions} if they are given
% as global options. \cs{DeclareLocalOptions} takes one option,
% \cs{DeclareLocalOptions} expects a comma separated list of options.
%
% \subsubsection{Dynamic options}
%
% Options of \LaTeX's package/class system are cleared
% in \cs{ProcessOptions}. They modify the static model
% of a package. For example, depending on option \xoption{bookmarks}
% package \xpackage{hyperref} loads differently.
%
% Options, however,
% defined by \xpackage{keyval}'s \cs{define@key} remain
% defined, if the options are processed by \cs{setkeys}.
% Therefore these options can also be used to model
% the dynamic behaviour of a package. For example, in
% \xpackage{hyperref} the link colors can be changed
% everywhere until the end in |\end{document}|.
%
% However package \xpackage{color} that adds color support
% is necessary and it
% cannot be loaded after |\begin{document}|. Option \xoption{colorlinks}
% that loads \xpackage{color} should be active until |\begin{document}|
% and die in some way if it is too late for loading packages.
% With \cs{DisableKeyvalOption} the package/class author can
% specify and configure the death of an option and controls
% the life period of the option.
%
% \subsubsection{\cs{DisableKeyvalOption}}
%
% \begin{declcs}{DisableKeyvalOption}\,\O{options}\,\M{family}\,\M{key}
%   \\[0.5ex]
%   \meta{options}:\\
%   \hspace{1.5em}^^A
%   \begin{tabular}[t]{@{}l@{ \texttt{=} }l@{\qquad}>{default: }l@{}}
%     |action| & |undef|, |warning|, |error|, or |ignore|
%     & |undef|
%     \\
%     \multicolumn{1}{@{}l}{|global| or |local|} & & |global|
%     \\
%     |package| or |class| & \meta{name}
%     \\
%   \end{tabular}^^A
% \end{declcs}
% \cs{DisableKeyvalOption} can be called to mark the end when
% the option \meta{key} is no longer useful. The behaviour of
% an option after its death can be configured by \xoption{action}:
% \begin{description}
% \item[\texttt{undef}:] The option will be undefined,
%   If it is called, \cs{setkeys} reports an error because
%   of unknown key.
% \item[\texttt{error} or \texttt{warning}:] Use of the
%  option will cause an error or warning message.
%  Also these actions require that exclusivly either the
%  package or class name is given in options \xoption{package}
%  or \xoption{class}.
% \item[\texttt{ignore}:] The use of the option will silently
%  be ignored.
% \end{description}
% The option's death can be limited to the end of the current
% group, if option \xoption{local} is given. Default is
% \xoption{global}.
%
% The package/class author can wish the end of the option
% already during the package loading, then he will have
% static behaviour. In case of dynamic options \cs{DisableKeyvalOption}
% can be executed everywhere, also outside the package. Therefore
% the family name and the package/class name is usually unknown for
% \cs{DisableKeyvalOption}. Therefore the argument for the family
% name is mandatory and for some actions the package/class name
% must be provided.
%
% Usually a macro would configure the option death, Example:
%\begin{quote}
%\begin{verbatim}
%\ProvidesPackage{foobar}
%\DeclareBoolOption{color}
%\DeclareStringOption[red]{emphcolor}
%\ProcessKeyvalOptions*
%
%\newcommand*{\foobar@DisableOption}[2]{%
%  \DisableKeyvalueOption[
%    action={#1},
%    package=foobar
%  ]{foobar}{#2}%
%}
%
%\iffoobar@color
%  \RequirePackage{color}
%  \renewcommand*{\emph}[1]{\textcolor{\foobar@emphcolor}{#1}}
%\else
%  % Option emphcolor is not wrong, if we have color support.
%  % otherwise the option has no effect, but we don't want to
%  % remove it. Therefore action 'ignore' is the best choice:
%  \foobar@DisableOption{ignore}{emphcolor}
%\fi
%% No we don't need the option 'color'.
%\foobar@DisableOption{warning}{color}
%
%% With color support option 'emphcolor' will dynamically
%% change the color of \emph statements.
%\end{verbatim}
%\end{quote}
%
% \subsubsection{\cs{AddToKeyvalOption}}
%
% \begin{declcs}{AddToKeyvalOption}\,\M{family}\,\M{key}\,\M{code}\\
%   \cs{AddToKeyvalOption} \*\,\M{key}\,\M{code}
% \end{declcs}
% The code for an existing key \meta{key} of family \meta{family} is
% extended by code \meta{code}. In the starred form the current
% family setting is used, see \cs{ProcessKeyvalOptions*}.
%
% \subsection{Global vs. local options}
%
% Options that are given for \cs{documentclass} are called
% global options. They are known to the class and all packages.
% A package may make use of a global option and marks it as
% used. The advantage for the user is the freedom to specify
% options both in the \cs{documentclass} or \cs{usepackage}
% commands.
%
% However global options are shared with the class options
% and options of all other packages. Thus there can be the
% same option with different semantics for different
% packages and classes. As example, package \xpackage{bookmark}
% knows option \xoption{open} that specifies whether the bookmarks
% are opened or closed initially. It's values are |true| or |false|.
% Since KOMA-Script version 3.00 the KOMA classes also introduces
% option \xoption{open} with values |right| and |any| and a
% complete different meaning.
%
% Such conflicts can be resolved by marking all or part of options
% as local by \cs{DeclareLocalOption} or \cs{DeclareLocalOptions}.
% Then the packages ignores global occurences of these options.
% Package \xpackage{kvoptions} provides two methods:
% \begin{itemize}
% \item \cs{ProcessLocalKeyvalOptions} automatically uses all options
%   as local options. It ignores all global options.
% \item \cs{DeclareLocalOption} or \cs{DeclareLocalOptions} marks
%   options as local options. \cs{ProcessKeyvalOptions} will then
%   ignore global occurences for these local options.
% \end{itemize}
% Since version 1.5 package \xpackage{bookmark} uses the latter
% method. It checks global and local option places for driver options
% and limits all other options as local options. Thus the class
% option \xoption{open} of KOMA-Script is not misread as
% option for package \xpackage{bookmark}.
%
% \subsection{Summary of internal macros}
%
% The \cs{Declare}\texttt{\dots Option} commands define
% macros, additionally to the macros generated by the key definition.
% These macros can be used by the package/class author.
% The name of the macros starts with the prefix \meta{prefix}
% that can be configured by \cs{SetupKeyvalOptions}.
% \begin{center}
% \def\prefix{\cs{}\meta{prefix}\meta{key}}^^A
% \def\Hline{^^A
%   \hline
%   \rule{0pt}{1.2\ht\csname @arstrutbox\endcsname}^^A
% }^^A
% \begin{tabular}{@{}l|>{\ttfamily}l|l@{}}
%   Declare \meta{key} & \textrm{Defined macro} & Description\\
% \Hline
%   \cs{DeclareStringOption}
%    & \prefix & holds the string\\
% \Hline
%   \cs{DeclareBoolOption}
%     & \cs{if}\meta{prefix}\meta{key}& boolean switch\\
%   &\prefix false& disable switch\\
%   &\prefix true& enable switch\\
% \Hline
%   \cs{DeclareComplementaryOption}
%     & \prefix false& enable parent switch\\
%     & \prefix true& disable parent switch\\
% \Hline
%   \cs{DeclareVoidOption}
%   & \prefix & holds the action
% \end{tabular}
% \end{center}
%
% \subsection{\plainTeX}
%
% Package \xpackage{keyval} is also usable in \plainTeX\ with
% the help of file \xfile{miniltx.tex}. Some features of this
% package \xpackage{kvoptions} might also be useful for \plainTeX.
% If \LaTeX\ is not found, \cs{ProcessKeyvalOptions} and option
% \xoption{patch} are disabled. Before using the option declaration
% commands \cs{Declare...Option}, \cs{SetupKeyvalOptions} must be
% used.
%
% \hypersetup{bookmarksopenlevel=1}
%
% \section{Example}
%
% The following example defined a package that serves some
% private color management. A boolean option \xoption{print} enables
% print mode without colors. An option \xoption{emph} redefines
% \cs{emph} to print in the given color. And the driver
% can be specified by option \xoption{driver}.
%
%    \begin{macrocode}
%<*example>
    % Package identification
    % -----------------------
\NeedsTeXFormat{LaTeX2e}
\ProvidesPackage{example-mycolorsetup}[2016/05/16 Managing my colors]

\RequirePackage{ifpdf}
\RequirePackage{kvoptions}

    % Option declarations
    % -------------------

\SetupKeyvalOptions{
  family=MCS,
  prefix=MCS@
}
    % Use a shorter family name and prefix

    % Option print
\DeclareBoolOption{print}
    % is the same as
    % \DeclareBoolOption[false]{print}

    % Option driver
\ifpdf
  \DeclareStringOption[pdftex]{driver}
\else
  \DeclareStringOption[dvips]{driver}
\fi

    % Alternative interface for driver options
\DeclareVoidOption{dvips}{\SetupDriver}
\DeclareVoidOption{dvipdfm}{\SetupDriver}
\DeclareVoidOption{pdftex}{\SetupDriver}
    % In \SetupDriver we take the current option \CurrentOption
    % and pass it to the driver option.
    % The \expandafter commands expand \CurrentOption at the
    % time, when \SetupDriver is executed and \CurrentOption
    % has the correct meaning.
\newcommand*{\SetupDriver}{%
  \expandafter\@SetupDriver\expandafter{\CurrentOption}%
}
\newcommand*{\@SetupDriver}[1]{%
  \setkeys{MCS}{driver={#1}}%
}

    % Option emph
    % An empty value means, we want to have no color for \emph.
    % If the user specifies option emph without value, the red is used.
\DeclareStringOption{emph}[red]
    % is the same as
    % \DeclareStringOption[]{emph}[red]

    % Default option rule
\DeclareDefaultOption{%
  \ifx\CurrentOptionValue\relax
    \PackageWarningNoLine{\@currname}{%
      Unknown option `\CurrentOption'\MessageBreak
      is passed to package `color'%
    }%
    % Pass the option to package color.
    % Again it is better to expand \CurrentOption.
    \expandafter\PassOptionsToPackage
    \expandafter{\CurrentOption}{color}%
  \else
    % Package color does not take options with values.
    % We provide the standard LaTeX error.
    \@unknownoptionerror
  \fi
}

    % Process options
    % ---------------
\ProcessKeyvalOptions*

    % Implementation depending on option values
    % -----------------------------------------
    % Code for print mode
\ifMCS@print
  \PassOptionsToPackage{monochrome}{color}
    % tells package color to use black and white
\fi

\RequirePackage[\MCS@driver]{color}
    % load package color with the correct driver

    % \emph setup
\ifx\MCS@emph\@empty
    % \@empty is a predefined macro with empty contents.
    % the option value of option emph is empty, thus
    % we do not want a redefinition of \emph.
\else
  \renewcommand*{\emph}[1]{%
    \textcolor{\MCS@emph}{#1}%
  }
\fi
%</example>
%    \end{macrocode}
%
% \section{Package options}
%
% The package \xpackage{kvoptions} knows two package options
% \xoption{patch} and \xoption{debugshow}.
% The options of package \xpackage{kvoptions} are intended
% for authors, not for package/class writers. Inside a package
% it is too late for option \xoption{patch} and \xoption{debugshow}
% enables some messages that are perhaps useful for the debugging
% phase. Also \LaTeX\ is unhappy if a package is loaded
% later again with options that are previously not given.
% Thus package and class authors, stay with |\RequirePackage{kvoptions}|
% without options.
%
% Option \xoption{patch} loads package \xpackage{kvoptions-patch}.
%
% \subsection{Package \xpackage{kvoptions-patch}}
%
% \LaTeX's system of package/class options has some severe limitations
% that especially affects the value part if options are used as
% pair of key and value.
% \begin{itemize}
% \item Spaces are removed, regardless where:
%   \begin{quote}
%     |\documentclass[box=0 0 400 600]{article}|
%   \end{quote}
%   Now each package will see |box=00400600| as global option.
% \item In the previous case also braces would not help:
%   \begin{quote}
%     |\documentclass[box={0 0 400 600}]{article}|
%   \end{quote}
%   The result is an error message:
%   \begin{quote}
%     |! LaTeX Error: Missing \begin{document}.|
%   \end{quote}
%   As local option, however, it works if the package
%   knows about key value options (By using this package, for example).
% \item The requirements on robustness are extremly high.
%   \LaTeX\ expands the option. All that will not work as environment
%   name will break also as option. Even a \cs{relax} will generate
%   an error message:
%   \begin{quote}
%     |! Missing \endcsname inserted.|
%   \end{quote}
%   Of course, \LaTeX\ does not use its protecting mechanisms.
%   On contrary \cs{protect} itself will cause errors.
% \item The options are expanded. But perhaps the package will
%   do that, because it has to setup some things before?
%   Example \xpackage{hyperref}:
%   \begin{quote}
%     |\usepackage[pdfauthor=M\"uller]{hyperref}|
%   \end{quote}
%   Package \xpackage{hyperref} does not see |M\"uller| but
%   its expansion and it does not like it, you get many warnings
%   \begin{quote}
%     |Token not allowed in a PDFDocEncoded string|
%   \end{quote}
%   And the title becomes: |Mu127uller|.
%   Therefore such options must usually be given after package
%   \xpackage{hyperref}
%   is loaded:
%   \begin{quote}
%     |\usepackage{hyperref}|\\
%     |\hypersetup{pdfauthor=Fran\c coise M\"uller}|
%   \end{quote}
%   As package option it will even break with |Fran\c coise|
%   because of the cedilla |\c c|, it is not robust enough.
% \end{itemize}
% For users that do not want with this limitations the package
% offers package \xoption{kvoptions-patch}. It patches \LaTeX's option system
% and tries to teach it also to handle options that are given
% as pairs of key and value and to prevent expansion.
% It can already be used at the very beginning, before \cs{documentclass}:
% \begin{quote}
%   |\RequirePackage{kvoptions-patch}|\\
%   |\documentclass[pdfauthor=Fran\c coise M\"uller]{article}|\\
%   |\usepackage{hyperref}|
% \end{quote}
% The latest time is before the package where you want to use
% problematic values:
% \begin{quote}
%   |\usepackage{kvoptions-patch}|\\
%   |\usepackage[Fran\c coise M\"uller]{hyperref}|
% \end{quote}
% Some remarks:
% \begin{itemize}
% \item The patch requires \eTeX, its \cs{unexpanded} feature
%   is much too nice. It is possible to work around using token
%   registers. But the code becomes longer, slower, more difficult
%   to read and maintain. The package without option \xoption{patch}
%   works and will work without \eTeX.
% \item The code for the patch is quite long, there are many test
%   cases. Thus the probability for bugs is probably not too small.
% \item Since 2008/10/18 v3.0 package \xpackage{kvoptions-patch} is
%   available. Before option \xoption{patch} of package \xpackage{kvoptions}
%   must be used instead. I think, the solution as standalone package
%   \xpackage{kvoptions-patch} is cleaner and avoids option clashes.
% \end{itemize}
%
% \subsection{Option \xoption{debugshow}}
%
% The name of this option follows the convention of packages
% \xpackage{multicol}, \xpackage{tabularx}, and \xpackage{tracefnt}.
% Currently it prints the setting of boolean options, declared
% by \cs{DeclareBoolOption} in the \xfile{.log} file, if that
% boolean option is used.
% You can activate the option by
% \begin{itemize}
% \item |\PassOptionsToPackage{debugshow}{kvoptions}|\\
%       Put this somewhere before package \xpackage{kvoptions} is
%       loaded first, e.g. before \cs{documentclass}.
% \item |\RequirePackage[debugshow]{kvoptions}|\\
%       Before \cs{documentclass} even an author has to use
%       \cs{RequirePackage}. \cs{usepackage} only works after
%       \cs{documentclass}.
% \end{itemize}
% The preferred method is \cs{PassOptionsToPackage}, because
% it does not force the package loading and does not disturb,
% if the package is not loaded later at all.
%
% \section{Limitations}
%
% \subsection{Compatibility}
% \subsubsection{Package \xpackage{kvoptions-patch} vs. package \xpackage{xkvltxp}}
%
% Package \xpackage{xkvltxp} from the \xpackage{xkeyval} project
% has the same goal as package \xpackage{kvoptions-patch}
% and to patch \LaTeX's kernel commands in order to get better support
% for key value options.
% Of course they cannot be used both. The user must decide, which
% method he prefers. Package \xpackage{kvoptions-patch} aborts itself,
% if it detects that \xpackage{xkvltxp} is
% already loaded.
%
% However package \xpackage{xkvltxp} and \xpackage{kvoptions}
% can be used together, example:
% \begin{quote}
% |\usepackage{xkvltxp}|\\
% |\usepackage[...]{foobar} % |\texttt{\itshape foobar using kvoptions}
% \end{quote}
% The other way should work, too.
%
% Package \xpackage{kvoptions-patch} tries to catch more situations
% and to be more robust.
% For example, during the comparison of options it normalizes
% them by removing spaces around |=| and the value. Thus the
% following is not reported as option clash:
% \begin{quote}
%\begin{verbatim}
%\RequirePackage{kvoptions-patch}
%\documentclass{article}
%
%\usepackage[scaled=0.7]{helvet}
%\usepackage[scaled = 0.7]{helvet}
%
%\begin{document}
%\end{document}
%\end{verbatim}
% \end{quote}
%
% \subsection{Limitations}
%
% \subsubsection{Option comparisons}
%
% In some situations \LaTeX\ compares option lists, e.g. option clash check,
% \cs{@ifpackagewith}, or \cs{@ifclasswith}. Apart from catcode and
% sanitizing problems of option \xoption{patch}, there is another problem.
% \LaTeX\ does not know about the type and default values of
% options in key value style. Thus an option clash is reported,
% even if the key value has the same meaning:
% \begin{quote}
% |\usepackage[scaled]{helvet} %| \texttt{\textit{default is }.95}\\
% |\usepackage[.95]{helvet}|\\
% |\usepackage[0.95]{helvet}|
% \end{quote}
%
% \subsubsection{Option list parsing with package \xpackage{kvoptions-patch}}
%
% With package \xpackage{kvoptions-patch} the range of possible values
% in key value specifications is much large, for example
% the comma can be used, if enclosed in curly braces.
%
% Other packages, especially the packages that uses their
% own process option code can be surprised to find tokens
% inside options that they do not expect and errors would
% be the consequence. To avoid errors the options, especially
% the unused option list is sanitized. That means the list
% will only contain tokens with catcode 12 (other) and
% perhaps spaces (catcode 10). This allows a safe parsing
% for other packages. But a comma in the value part is
% no longer protected by curly braces because they have lost
% their special meaning. This is the price for compatibility.
%
% Example:
% \begin{quote}
% |\RequirePackage{kvoptions-patch}|\\
% |\documentclass[a={a,b,c},b]{article}|\\
% |\begin{document}|\\
% |\end{document}|
% \end{quote}
% Result:
% \begin{quote}
% |LaTeX Warning: Unused global option(s):|\\
% |    [a={a,c},b].|
% \end{quote}
%
% \StopEventually{
% }
%
% \section{Implementation}
%
% \subsection{Preamble}
%
%    \begin{macrocode}
%<*package>
%    \end{macrocode}
%
% \paragraph{Reload check and  identification.}
%    Reload check, especially if the package is not used with \LaTeX.
%    \begin{macrocode}
\begingroup\catcode61\catcode48\catcode32=10\relax%
  \catcode13=5 % ^^M
  \endlinechar=13 %
  \catcode35=6 % #
  \catcode39=12 % '
  \catcode44=12 % ,
  \catcode45=12 % -
  \catcode46=12 % .
  \catcode58=12 % :
  \catcode64=11 % @
  \catcode123=1 % {
  \catcode125=2 % }
  \expandafter\let\expandafter\x\csname ver@kvoptions.sty\endcsname
  \ifx\x\relax % plain-TeX, first loading
  \else
    \def\empty{}%
    \ifx\x\empty % LaTeX, first loading,
      % variable is initialized, but \ProvidesPackage not yet seen
    \else
      \expandafter\ifx\csname PackageInfo\endcsname\relax
        \def\x#1#2{%
          \immediate\write-1{Package #1 Info: #2.}%
        }%
      \else
        \def\x#1#2{\PackageInfo{#1}{#2, stopped}}%
      \fi
      \x{kvoptions}{The package is already loaded}%
      \aftergroup\endinput
    \fi
  \fi
\endgroup%
%    \end{macrocode}
%    Package identification:
%    \begin{macrocode}
\begingroup\catcode61\catcode48\catcode32=10\relax%
  \catcode13=5 % ^^M
  \endlinechar=13 %
  \catcode35=6 % #
  \catcode39=12 % '
  \catcode40=12 % (
  \catcode41=12 % )
  \catcode44=12 % ,
  \catcode45=12 % -
  \catcode46=12 % .
  \catcode47=12 % /
  \catcode58=12 % :
  \catcode64=11 % @
  \catcode91=12 % [
  \catcode93=12 % ]
  \catcode123=1 % {
  \catcode125=2 % }
  \expandafter\ifx\csname ProvidesPackage\endcsname\relax
    \def\x#1#2#3[#4]{\endgroup
      \immediate\write-1{Package: #3 #4}%
      \xdef#1{#4}%
    }%
  \else
    \def\x#1#2[#3]{\endgroup
      #2[{#3}]%
      \ifx#1\@undefined
        \xdef#1{#3}%
      \fi
      \ifx#1\relax
        \xdef#1{#3}%
      \fi
    }%
  \fi
\expandafter\x\csname ver@kvoptions.sty\endcsname
\ProvidesPackage{kvoptions}%
  [2016/05/16 v3.12 Key value format for package options (HO)]%
%    \end{macrocode}
%
% \paragraph{Catcodes}
%
%    \begin{macrocode}
\begingroup\catcode61\catcode48\catcode32=10\relax%
  \catcode13=5 % ^^M
  \endlinechar=13 %
  \catcode123=1 % {
  \catcode125=2 % }
  \catcode64=11 % @
  \def\x{\endgroup
    \expandafter\edef\csname KVO@AtEnd\endcsname{%
      \endlinechar=\the\endlinechar\relax
      \catcode13=\the\catcode13\relax
      \catcode32=\the\catcode32\relax
      \catcode35=\the\catcode35\relax
      \catcode61=\the\catcode61\relax
      \catcode64=\the\catcode64\relax
      \catcode123=\the\catcode123\relax
      \catcode125=\the\catcode125\relax
    }%
  }%
\x\catcode61\catcode48\catcode32=10\relax%
\catcode13=5 % ^^M
\endlinechar=13 %
\catcode35=6 % #
\catcode64=11 % @
\catcode123=1 % {
\catcode125=2 % }
\def\TMP@EnsureCode#1#2{%
  \edef\KVO@AtEnd{%
    \KVO@AtEnd
    \catcode#1=\the\catcode#1\relax
  }%
  \catcode#1=#2\relax
}
\TMP@EnsureCode{1}{14}% ^^A (comment)
\TMP@EnsureCode{2}{14}% ^^A (comment)
\TMP@EnsureCode{33}{12}% !
\TMP@EnsureCode{39}{12}% '
\TMP@EnsureCode{40}{12}% (
\TMP@EnsureCode{41}{12}% )
\TMP@EnsureCode{42}{12}% *
\TMP@EnsureCode{44}{12}% ,
\TMP@EnsureCode{45}{12}% -
\TMP@EnsureCode{46}{12}% .
\TMP@EnsureCode{47}{12}% /
\TMP@EnsureCode{58}{12}% :
\TMP@EnsureCode{62}{12}% >
\TMP@EnsureCode{91}{12}% [
\TMP@EnsureCode{93}{12}% ]
\TMP@EnsureCode{94}{7}% ^ (superscript)
\TMP@EnsureCode{96}{12}% `
\edef\KVO@AtEnd{\KVO@AtEnd\noexpand\endinput}
%    \end{macrocode}
%
% \paragraph{External resources.}
%    The package extends the support for key value pairs of
%    package \cs{keyval} to package options. Thus the package
%    needs to be loaded anyway. and we use it for
%    \cs{SetupKeyvalOptions}. AFAIK this does not disturb
%    users of \xpackage{xkeyval}.
%    \begin{macrocode}
\@ifundefined{define@key}{%
  \RequirePackage{keyval}\relax
}{}
%    \end{macrocode}
%
%    Macro \cs{DeclareLocalOptions} parses a comma separated key list
%    and uses \cs{comma@parse} of package \xpackage{kvsetkeys},
%    version 1.3.
%    \begin{macrocode}
\RequirePackage{ltxcmds}[2010/12/02]
\RequirePackage{kvsetkeys}[2007/09/29]
%    \end{macrocode}
%
% \paragraph{Provide macros for \plainTeX.}
%    \begin{macrocode}
\@ifundefined{@x@protect}{%
  \def\@x@protect#1\fi#2#3{%
    \fi\protect#1%
  }%
  \let\@typeset@protect\relax
}{}
\@ifundefined{@currname}{%
  \def\@currname{}%
}{}
\@ifundefined{@currext}{%
  \def\@currext{}%
}{}
%    \end{macrocode}
%
% \paragraph{Options}
%    Option \xoption{debugshow} enables additional lines of
%    code that prints information into the \xfile{.log}
%    file.
%    \begin{macrocode}
\DeclareOption{debugshow}{\catcode\@ne=9 }
%    \end{macrocode}
%
%    \begin{macrocode}
\DeclareOption{patch}{%
  \AtEndOfPackage{%
    \RequirePackage{kvoptions-patch}[2016/05/16]%
  }%
}
%    \end{macrocode}
%
%    Optionen auswerten:
%    \begin{macrocode}
\ProcessOptions\relax
%    \end{macrocode}
%
% \subsection{Option declaration macros}
%
% \subsubsection{\cs{SetupKeyvalOptions}}
%
%    The family for the key value pairs can be setup once
%    and is remembered later.
%    The package name seems a reasonable default for the
%    family key, if it is not set by the package author.
%
%    \begin{macro}{\KVO@family}
%    We cannot store the family setting in one macro,
%    because the package should be usable for many
%    other packages, too. Thus we remember the family
%    setting in a macro, whose name contains the
%    package name with extension, a key in \LaTeX's
%    class/package system.
%    \begin{macrocode}
\define@key{KVO}{family}{%
  \expandafter\edef\csname KVO@family@%
      \@currname.\@currext\endcsname{#1}%
}
\def\KVO@family{%
  \@ifundefined{KVO@family@\@currname.\@currext}{%
    \@currname
  }{%
    \csname KVO@family@\@currname.\@currext\endcsname
  }%
}
%    \end{macrocode}
%    \end{macro}
%
%    \begin{macro}{\KVO@prefix}
%    The value settings of options that are declared by \cs{DeclareBoolOption}
%    and \cs{DeclareStringOption} need to be saved in macros.
%    in the first case this is a switch \cs{if}\meta{prefix}\meta{key},
%    in the latter case a macro \cs{}\meta{prefix}\meta{key}.
%    The prefix can be configured, by \xoption{prefix} that is declared
%    here. The default is the package name with |@| appended.
%    \begin{macrocode}
\define@key{KVO}{prefix}{%
  \expandafter\edef\csname KVO@prefix@%
      \@currname.\@currext\endcsname{#1}%
}
\def\KVO@prefix{%
  \ltx@ifundefined{KVO@prefix@\@currname.\@currext}{%
    \@currname @%
  }{%
    \csname KVO@prefix@\@currname.\@currext\endcsname
  }%
}
%    \end{macrocode}
%    \end{macro}
%
%    \begin{macrocode}
\define@key{KVO}{setkeys}{%
  \expandafter\def\csname KVO@setkeys@%
      \@currname.\@currext\endcsname{#1}%
}
%    \end{macrocode}
%    \begin{macro}{\KVO@setkeys}
%    \begin{macrocode}
\def\KVO@setkeys{%
  \ltx@IfUndefined{KVO@setkeys@\@currname.\@currext}{%
    \setkeys
  }{%
    \csname KVO@setkeys@\@currname.\@currext\endcsname
  }%
}
%    \end{macrocode}
%    \end{macro}
%
%    \begin{macro}{\SetupKeyvalOptions}
%    The argument of \cs{SetupKeyvalOptions} expects a key value
%    list, known keys are \xoption{family} and \xoption{prefix}.
%    \begin{macrocode}
\newcommand*{\SetupKeyvalOptions}{%
  \kvsetkeys{KVO}%
}
%    \end{macrocode}
%    \end{macro}
%
% \subsubsection{\cs{DeclareBoolOption}}
%
%    \begin{macro}{\DeclareBoolOption}
%    Usually options of boolean type can be given by the
%    user without value and this means a setting to \emph{true}.
%    We follow this convention here. Also it simplifies the
%    user interface.
%
%    The switch is created and initialized with \emph{false}.
%    The default setting can be overwritten by the optional
%    argument.
%
%    \LaTeX's \cs{newif} does not check for already defined
%    macros, therefore we add this check here to prevent
%    the user from accidently redefining of \TeX's primitives
%    and other macros.
%    \begin{macrocode}
\newcommand*{\DeclareBoolOption}[2][false]{%
  \KVO@ifdefinable{if\KVO@prefix#2}{%
    \KVO@ifdefinable{\KVO@prefix#2true}{%
      \KVO@ifdefinable{\KVO@prefix#2false}{%
        \csname newif\expandafter\endcsname
        \csname if\KVO@prefix#2\endcsname
        \@ifundefined{\KVO@prefix#2#1}{%
          \PackageWarning{kvoptions}{%
            Initialization of option `#2' failed,\MessageBreak
            cannot set boolean option to `#1',\MessageBreak
            use `true' or `false', now using `false'%
          }%
        }{%
          \csname\KVO@prefix#2#1\endcsname
        }%
        \begingroup
          \edef\x{\endgroup
            \noexpand\define@key{\KVO@family}{#2}[true]{%
              \noexpand\KVO@boolkey{\@currname}%
              \ifx\@currext\@clsextension
                \noexpand\@clsextension
              \else
                \noexpand\@pkgextension
              \fi
              {\KVO@prefix}{#2}{####1}%
            }%
          }%
        \x
      }%
    }%
  }%
}
%    \end{macrocode}
%    \end{macro}
%
%    \begin{macro}{\DeclareComplementaryOption}
%    The first argument is the key name, the second the
%    key that must be a boolean option with the same
%    current family and prefix. A new switch is not
%    created for the new key, we have already a switch.
%    Instead we define switch setting commands to work
%    on the parent switch.
%    \begin{macrocode}
\newcommand*{\DeclareComplementaryOption}[2]{%
  \@ifundefined{if\KVO@prefix#2}{%
    \PackageError{kvoptions}{%
      Cannot generate option code for `#1',\MessageBreak
      parent switch `#2' does not exist%
    }{%
      You are inside %
      \ifx\@currext\@clsextension class\else package\fi\space
      `\@currname.\@currext'.\MessageBreak
      `\KVO@family' is used as familiy %
      for the keyval options.\MessageBreak
      `\KVO@prefix' serves as prefix %
      for internal switch macros.\MessageBreak
      \MessageBreak
      \@ehc
    }%
  }{%
    \KVO@ifdefinable{\KVO@prefix#1true}{%
      \KVO@ifdefinable{\KVO@prefix#1false}{%
        \expandafter\let\csname\KVO@prefix#1false\expandafter\endcsname
          \csname\KVO@prefix#2true\endcsname
        \expandafter\let\csname\KVO@prefix#1true\expandafter\endcsname
          \csname\KVO@prefix#2false\endcsname
%    \end{macrocode}
%    The same code part as in \cs{DeclareBoolOption} can
%    now be used.
%    \begin{macrocode}
        \begingroup
          \edef\x{\endgroup
            \noexpand\define@key{\KVO@family}{#1}[true]{%
              \noexpand\KVO@boolkey{\@currname}%
              \ifx\@currext\@clsextension
                \noexpand\@clsextension
              \else
                \noexpand\@pkgextension
              \fi
              {\KVO@prefix}{#1}{####1}%
            }%
          }%
        \x
      }%
    }%
  }%
}
%    \end{macrocode}
%    \end{macro}
%
%    \begin{macro}{\KVO@ifdefinable}
%    Generate the command token LaTeX's \cs{@ifdefinable} expects.
%    \begin{macrocode}
\def\KVO@ifdefinable#1{%
  \expandafter\@ifdefinable\csname #1\endcsname
}
%    \end{macrocode}
%    \end{macro}
%
%    \begin{macro}{\KVO@boolkey}
%    We check explicitly for |true| and |false| to prevent
%    the user from accidently calling other macros.
%    \begin{quote}
%    \begin{tabular}{@{}ll@{}}
%    |#1|& package/class name\\
%    |#2|& |\@pkgextension|/|\@clsextension|\\
%    |#3|& prefix \\
%    |#4|& key name\\
%    |#5|& new value
%    \end{tabular}
%    \end{quote}
%    \begin{macrocode}
\def\KVO@boolkey#1#2#3#4#5{%
  \edef\KVO@param{#5}%
  \ltx@onelevel@sanitize\KVO@param
  \ifx\KVO@param\KVO@true
    \expandafter\@firstofone
  \else
    \ifx\KVO@param\KVO@false
      \expandafter\expandafter\expandafter\@firstofone
    \else
      \ifx#2\@clsextension
        \expandafter\ClassWarning
      \else
        \expandafter\PackageWarning
      \fi
      {#1}{%
        Value `\KVO@param' is not supported by\MessageBreak
        option `#4'%
      }%
      \expandafter\expandafter\expandafter\@gobble
    \fi
  \fi
  {%
    ^^A\ifx#2\@clsextension
    ^^A  \expandafter\ClassInfo
    ^^A\else
    ^^A  \expandafter\PackageInfo
    ^^A\fi
    ^^A{#1}{[option] #4=\KVO@param}%
    \csname#3#4\KVO@param\endcsname
  }%
}
%    \end{macrocode}
%    \end{macro}
%    \begin{macro}{\KVO@true}
%    \begin{macro}{\KVO@false}
%    The macros \cs{KVO@true} and \cs{KVO@false} are used
%    for string comparisons. After \cs{ltx@onelevel@sanitize}
%    we have only tokens with catcode 12 (other).
%    \begin{macrocode}
\def\KVO@true{true}
\def\KVO@false{false}
\ltx@onelevel@sanitize\KVO@true
\ltx@onelevel@sanitize\KVO@false
%    \end{macrocode}
%    \end{macro}
%    \end{macro}
%
% \subsubsection{\cs{DeclareStringOption}}
%
%    \begin{macro}{\DeclareStringOption}
%    \begin{macrocode}
\newcommand*{\DeclareStringOption}[2][]{%
  \@ifnextchar[{%
    \KVO@DeclareStringOption{#1}{#2}@%
  }{%
    \KVO@DeclareStringOption{#1}{#2}{}[]%
  }%
}
%    \end{macrocode}
%    \end{macro}
%    \begin{macro}{\KVO@DeclareStringOption}
%    \begin{macrocode}
\def\KVO@DeclareStringOption#1#2#3[#4]{%
  \KVO@ifdefinable{\KVO@prefix#2}{%
    \@namedef{\KVO@prefix#2}{#1}%
    \begingroup
      \ifx\\#3\\%
        \toks@{}%
      \else
        \toks@{[{#4}]}%
      \fi
      \edef\x{\endgroup
        \noexpand\define@key{\KVO@family}{#2}\the\toks@{%
          ^^A\begingroup
          ^^A  \toks@{####1}%
          ^^A  \ifx\@currext\@clsextension
          ^^A    \noexpand\ClassInfo
          ^^A  \else
          ^^A    \noexpand\PackageInfo
          ^^A  \fi
          ^^A  {\@currname}{%
          ^^A    [option] #2={\noexpand\the\toks@}%
          ^^A  }%
          ^^A\endgroup
          \noexpand\def
          \expandafter\noexpand\csname\KVO@prefix#2\endcsname{####1}%
        }%
      }%
    \x
  }%
}
%    \end{macrocode}
%    \end{macro}
%
% \subsubsection{\cs{DeclareVoidOption}}
%
%    \begin{macro}{\DeclareVoidOption}
%    \begin{macrocode}
\newcommand*{\DeclareVoidOption}[2]{%
  \begingroup
    \let\next\@gobbletwo
    \KVO@ifdefinable{\KVO@prefix#1}{%
      \let\next\@firstofone
    }%
  \expandafter\endgroup
  \next{%
    \begingroup
      \edef\x{\endgroup
        \noexpand\define@key{\KVO@family}{#1}[\KVO@VOID@]{%
          \noexpand\KVO@voidkey{\@currname}%
          \ifx\@currext\@clsextension
            \noexpand\@clsextension
          \else
            \noexpand\@pkgextension
          \fi
          {#1}%
          {####1}%
          \expandafter\noexpand\csname\KVO@prefix#1\endcsname
        }%
      }%
    \x
    \begingroup
      \toks@{#2}%
    \expandafter\endgroup
    \expandafter\def
    \csname\KVO@prefix#1\expandafter\endcsname
    \expandafter{\the\toks@}%
  }%
}
\def\KVO@VOID@{@VOID@}
%    \end{macrocode}
%    \end{macro}
%    \begin{macro}{\KVO@voidkey}
%    \begin{tabular}{@{}ll@{}}
%    |#1|& package/class name\\
%    |#2|& |\@pkgextension|/|\@clsextension|\\
%    |#3|& key name\\
%    |#4|& default (|@VOID@|)\\
%    |#5|& macro with option code
%    \end{tabular}
%    \begin{macrocode}
\def\KVO@voidkey#1#2#3#4{%
  \def\CurrentOption{#3}%
  \begingroup
    \def\x{#4}%
  \expandafter\endgroup
  \ifx\x\KVO@VOID@
  \else
    \ifx#2\@clsextension
      \expandafter\ClassWarning
    \else
      \expandafter\PackageWarning
    \fi
    {#1}{%
      Unexpected value for option `#3'\MessageBreak
      is ignored%
    }%
  \fi
  ^^A\ifx#2\@clsextension
  ^^A  \expandafter\ClassInfo
  ^^A\else
  ^^A  \expandafter\PackageInfo
  ^^A\fi
  ^^A{#1}{[option] #3}%
}
%    \end{macrocode}
%    \end{macro}
%
% \subsubsection{\cs{DeclareDefaultOption}}
%
%    \begin{macro}{\DeclareDefaultOption}
%    \begin{macrocode}
\newcommand*{\DeclareDefaultOption}{%
  \@namedef{KVO@default@\@currname.\@currext}%
}
%    \end{macrocode}
%    \end{macro}
%
% \subsubsection{\cs{DeclareLocalOptions}}
%
%    \begin{macro}{\DeclareLocalOptions}
%    \begin{macrocode}
\newcommand*{\DeclareLocalOptions}[1]{%
  \comma@parse{#1}\KVO@DeclareLocalOption
}
%    \end{macrocode}
%    \end{macro}
%    \begin{macro}{\KVO@DeclareLocalOption}
%    \begin{macrocode}
\def\KVO@DeclareLocalOption#1{%
  \expandafter\def\csname KVO@local@\KVO@family @#1\endcsname{}%
}
%    \end{macrocode}
%    \end{macro}
%
% \subsection{Dynamic options}
%
% \subsubsection{\cs{DisableKeyvalOption}}
%
%    \begin{macrocode}
\SetupKeyvalOptions{%
  family=KVOdyn,%
  prefix=KVOdyn@%
}
\DeclareBoolOption[true]{global}
\DeclareComplementaryOption{local}{global}
\DeclareStringOption[undef]{action}
\let\KVOdyn@name\relax
\let\KVOdyn@ext\@empty
\define@key{KVOdyn}{class}{%
  \def\KVOdyn@name{#1}%
  \let\KVOdyn@ext\@clsextension
}
\define@key{KVOdyn}{package}{%
  \def\KVOdyn@name{#1}%
  \let\KVOdyn@ext\@pkgextension
}
\newcommand*{\DisableKeyvalOption}[3][]{%
  \begingroup
    \kvsetkeys{KVOdyn}{#1}%
    \def\x{\endgroup}%
    \@ifundefined{KVO@action@\KVOdyn@action}{%
      \PackageError{kvoptions}{%
        Unknown disable action %
        `\expandafter\strip@prefix\meaning\KVOdyn@action'\MessageBreak
        for option `#3' in keyval family '#2'%
      }\@ehc
    }{%
      \csname KVO@action@\KVOdyn@action\endcsname{#2}{#3}%
    }%
  \x
}
\def\KVO@action@undef#1#2{%
  \edef\x{\endgroup
    \ifKVOdyn@global\global\fi
    \let
    \expandafter\noexpand\csname KV@#1@#2\endcsname
    \relax
    \ifKVOdyn@global\global\fi
    \let
    \expandafter\noexpand\csname KV@#1@#2@default\endcsname
    \relax
  }%
  ^^A\PackageInfo{kvoptions}{%
  ^^A  [option] key `#2' of family `#1'\MessageBreak
  ^^A  is disabled (undef, \ifKVOdyn@global global\else local\fi)%
  ^^A}%
}
\def\KVO@action@ignore#1#2{%
  \edef\x{\endgroup
    \ifKVOdyn@global\global\fi
    \let
    \expandafter\noexpand\csname KV@#1@#2\endcsname
    \noexpand\@gobble
    \ifKVOdyn@global\global\fi
    \let
    \expandafter\noexpand\csname KV@#1@#2@default\endcsname
    \noexpand\@empty
  }%
  ^^A\PackageInfo{kvoptions}{%
  ^^A  [option] key `#2' of family `#1'\MessageBreak
  ^^A  is disabled (ignore, \ifKVOdyn@global global\else local\fi)%
  ^^A}%
}
\def\KVO@action@error{%
  \KVO@do@action{error}%
}
\def\KVO@action@warning{%
  \KVO@do@action{warning}%
}
%    \end{macrocode}
%    \begin{tabular}{@{}ll@{}}
%    |#1|& |error| or |warning|\\
%    |#2|& \meta{family}\\
%    |#3|& \meta{key}\\
%    \end{tabular}
%    \begin{macrocode}
\def\KVO@do@action#1#2#3{%
  \ifx\KVOdyn@name\relax
    \PackageError{kvoptions}{%
      Action type `#1' needs package/class name\MessageBreak
      for key `#3' in family `#2'%
    }\@ehc
  \else
    \edef\x{\endgroup
      \noexpand\define@key{#2}{#3}[]{%
        \expandafter\noexpand\csname KVO@disable@#1\endcsname
        {\KVOdyn@name}\noexpand\KVOdyn@ext{#3}%
      }%
      \ifKVOdyn@global
        \global\let
        \expandafter\noexpand\csname KV@#2@#3\endcsname
        \expandafter\noexpand\csname KV@#2@#3\endcsname
        \global\let
        \expandafter\noexpand\csname KV@#2@#3@default\endcsname
        \expandafter\noexpand\csname KV@#2@#3@default\endcsname
      \fi
    }%
    ^^A\ifx\KVOdyn@ext\@clsextension
    ^^A  \expandafter\ClassInfo
    ^^A\else
    ^^A   \expandafter\PackageInfo
    ^^A\fi
    ^^A{\KVOdyn@name}{%
    ^^A  [option] key `#3' of family `#2'\MessageBreak
    ^^A  is disabled (#1, \ifKVOdyn@global global\else local\fi)%
    ^^A}%
  \fi
}
\def\KVO@disable@error#1#2#3{%
  \ifx#2\@clsextension
    \expandafter\ClassError
  \else
    \expandafter\PackageError
  \fi
  {#1}{%
    Option `#3' is given too late,\MessageBreak
    now the option is ignored%
  }\@ehc
}
\def\KVO@disable@warning#1#2#3{%
  \ifx#2\@clsextension
    \expandafter\ClassWarning
  \else
    \expandafter\PackageWarning
  \fi
  {#1}{%
    Option `#3' is already consumed\MessageBreak
    and has no effect%
  }%
}
%    \end{macrocode}
%
% \subsection{Change option code}
%
% \subsubsection{\cs{AddToKeyvalOption}}
%
%    \begin{macro}{\AddToKeyvalOption}
%    \begin{macrocode}
\newcommand*{\AddToKeyvalOption}{%
  \@ifstar{%
    \begingroup
      \edef\x{\endgroup
        \noexpand\KVO@AddToKeyvalOption{\KVO@family}%
      }%
    \x
  }%
  \KVO@AddToKeyvalOption
}
%    \end{macrocode}
%    \end{macro}
%    \begin{macro}{\KVO@AddToKeyvalOption}
%    \begin{macrocode}
\def\KVO@AddToKeyvalOption#1#2{%
  \@ifundefined{KV@#1@#2}{%
    \PackageWarning{kvoptions}{%
      Key `#2' of family `#1' does not exist.\MessageBreak
      Ignoring \string\AddToKeyvalOption
    }%
    \@gobble
  }{%
    \edef\KVO@next{%
      \noexpand\KVO@@AddToKeyvalOption
      \expandafter\noexpand\csname KV@#1@#2\endcsname
    }%
    \afterassignment\KVO@next
    \def\KVO@temp##1%
  }%
}
%    \end{macrocode}
%    \end{macro}
%    \begin{macro}{\KVO@@AddToKeyvalOption}
%    \begin{macrocode}
\def\KVO@@AddToKeyvalOption#1{%
  \begingroup
    \toks@\expandafter{#1{##1}}%
    \toks@\expandafter{\the\expandafter\toks@\KVO@temp{##1}}%
    \edef\x{\endgroup
      \noexpand\def\noexpand#1####1{\the\toks@}%
    }%
  \x
}
%    \end{macrocode}
%    \end{macro}
%
% \subsection{Process options}
%
% \subsubsection{Get global options}
%
%    Package \xpackage{xkeyval} removes options with equal signs from
%    the global options (\cs{@classoptionslist}). The effect is that
%    other packages and classes will not see these global options anymore.
%    A bug-report was answered that this behaviour is ``by design''.
%    Thus I call it a design bug. Now getting the global options require
%    an algorithm instead of a simple macro call.
%    \begin{macrocode}
%</package>
%<*package|patch>
%    \end{macrocode}
%    \begin{macro}{\KVO@IfDefThen}
%    Call |#2| if command |#1| is defined and not \cs{relax}.
%    (Package \xpackage{kvoptions-patch} does not load
%    package \xpackage{ltxcmds}.)
%    \begin{macrocode}
\def\KVO@IfDefThen#1#2{%
  \ifx#1\ltx@undefined
  \else
    \ifx#1\relax
    \else
      #2%
    \fi
  \fi
}%
%    \end{macrocode}
%    \end{macro}
%    \begin{macro}{\KVO@GetClassOptionsList}
%    \begin{macrocode}
\def\KVO@GetClassOptionsList{%
  \let\KVO@classoptionslist\@classoptionslist
  \KVO@IfDefThen\@classoptionslist{%
    \KVO@IfDefThen\XKV@documentclass{%
      \ifx\XKV@documentclass\ltx@empty
      \else
        \KVO@IfDefThen\XKV@classoptionslist{%
          \ifx\XKV@classoptionslist\ltx@empty
          \else
            \let\KVO@classoptionslist\XKV@classoptionslist
          \fi
        }%
      \fi
    }%
  }%
}%
%    \end{macrocode}
%    \end{macro}
%    \begin{macrocode}
%</package|patch>
%<*package>
%    \end{macrocode}
%
% \subsubsection{\cs{ProcessKeyvalOptions}}
%
%    \begin{macro}{\ProcessKeyvalOptions}
%    If the optional star is given, we get the family
%    name and expand it for safety.
%    \begin{macrocode}
\newcommand*{\ProcessKeyvalOptions}{%
  \@ifstar{%
    \begingroup
      \edef\x{\endgroup
        \noexpand\KVO@ProcessKeyvalOptions{\KVO@family}%
      }%
    \x
  }%
  \KVO@ProcessKeyvalOptions
}
%    \end{macrocode}
%    \end{macro}
%
%    \begin{macrocode}
\def\KVO@ProcessKeyvalOptions#1{%
  \let\@tempc\relax
  \let\KVO@temp\@empty
%    \end{macrocode}
%    Add any global options that are known to KV to the start of
%    the list being built in |\KVO@temp| and mark them used (by
%    removing them from the unused option list).
%    \begin{macrocode}
  \ifx\@currext\@clsextension
  \else
    \KVO@GetClassOptionsList
    \ifx\KVO@classoptionslist\relax
    \else
      \@for\KVO@CurrentOption:=\KVO@classoptionslist\do{%
        \@ifundefined{KV@#1@\expandafter\KVO@getkey
                      \KVO@CurrentOption=\@nil}{%
        }{%
          \@ifundefined{KVO@local@#1@\expandafter\KVO@getkey
                        \KVO@CurrentOption=\@nil}{%
            \ifx\KVO@Patch Y%
              \edef\KVO@temp{%
                \etex@unexpanded\expandafter{%
                  \KVO@temp
                }%
                ,%
                \etex@unexpanded\expandafter{%
                  \KVO@CurrentOption
                }%
                ,%
              }%
              \ltx@onelevel@sanitize\KVO@CurrentOption
            \else
              \edef\KVO@temp{%
                \KVO@temp
                ,%
                \KVO@CurrentOption
                ,%
              }%
            \fi
            \@expandtwoargs\@removeelement\KVO@CurrentOption
              \@unusedoptionlist\@unusedoptionlist
          }{}%
        }%
      }%
    \fi
  \fi
%    \end{macrocode}
%     Now stick the package options at the end of the list and wrap
%     in a call to \cs{setkeys}. A class ignores unknown global
%     options, we must remove them to prevent error messages
%     from \cs{setkeys}.
%    \begin{macrocode}
  \begingroup
    \toks\tw@{}%
    \@ifundefined{opt@\@currname.\@currext}{%
      \toks@\expandafter{\KVO@temp}%
    }{%
      \toks@\expandafter\expandafter\expandafter{%
        \csname opt@\@currname.\@currext\endcsname
      }%
      \ifx\@currext\@clsextension
        \edef\CurrentOption{\the\toks@}%
        \toks@\expandafter{\KVO@temp}%
        \@for\CurrentOption:=\CurrentOption\do{%
          \@ifundefined{%
            KV@#1@\expandafter\KVO@getkey\CurrentOption=\@nil
          }{%
%    \end{macrocode}
%    A class puts not used options in the unused option list
%    unless there is a default handler.
%    \begin{macrocode}
            \@ifundefined{KVO@default@\@currname.\@currext}{%
              \ifx\KVO@Patch Y%
                \ltx@onelevel@sanitize\CurrentOption
              \fi
              \ifx\@unusedoptionlist\@empty
                \global\let\@unusedoptionlist\CurrentOption
              \else
                \expandafter\expandafter\expandafter\gdef
                \expandafter\expandafter\expandafter\@unusedoptionlist
                \expandafter\expandafter\expandafter{%
                  \expandafter\@unusedoptionlist
                  \expandafter,\CurrentOption
                }%
              \fi
            }{%
              \toks\tw@\expandafter{%
                \the\toks\expandafter\tw@\expandafter,\CurrentOption
              }%
            }%
          }{%
            \toks@\expandafter{%
              \the\expandafter\toks@\expandafter,\CurrentOption
            }%
          }%
        }%
      \else
%    \end{macrocode}
%    Without default action we pass all options to \cs{setkeys}.
%    Otherwise we have to check which options are known.
%    These are passed to \cs{setkeys}. For the others the default
%    action is performed.
%    \begin{macrocode}
        \@ifundefined{KVO@default@\@currname.\@currext}{%
          \toks@\expandafter\expandafter\expandafter{%
            \expandafter\KVO@temp\the\toks@
          }%
        }{%
          \edef\CurrentOption{\the\toks@}%
          \toks@\expandafter{\KVO@temp}%
          \@for\CurrentOption:=\CurrentOption\do{%
            \@ifundefined{%
              KV@#1@\expandafter\KVO@getkey\CurrentOption=\@nil
            }{%
              \toks\tw@\expandafter{%
                \the\toks\expandafter\tw@\expandafter,\CurrentOption
              }%
            }{%
              \toks@\expandafter{%
                \the\expandafter\toks@\expandafter,\CurrentOption
              }%
            }%
          }%
        }%
      \fi
    }%
    \edef\KVO@temp{\endgroup
      \noexpand\KVO@calldefault{\the\toks\tw@}%
      \noexpand\KVO@setkeys{#1}{\the\toks@}%
    }%
  \KVO@temp
%    \end{macrocode}
%    Some cleanup of \cs{ProcessOptions}.
%    \begin{macrocode}
  \let\CurrentOption\@empty
  \AtEndOfPackage{\let\@unprocessedoptions\relax}%
}
%    \end{macrocode}
%
% \subsubsection{\cs{ProcessLocalKeyvalOptions}}
%
%    \begin{macro}{\ProcessLocalKeyvalOptions}
%    If the optional star is given, we get the family
%    name and expand it for safety.
%    \begin{macrocode}
\newcommand*{\ProcessLocalKeyvalOptions}{%
  \@ifstar{%
    \begingroup
      \edef\x{\endgroup
        \noexpand\KVO@ProcessLocalKeyvalOptions{\KVO@family}%
      }%
    \x
  }%
  \KVO@ProcessLocalKeyvalOptions
}
%    \end{macrocode}
%    \end{macro}
%
%    \begin{macrocode}
\def\KVO@ProcessLocalKeyvalOptions#1{%
  \let\@tempc\relax
  \let\KVO@temp\@empty
%    \end{macrocode}
%    Check if \cs{ProcessLocalKeyvalOptions} is called inside
%    a package.
%    \begin{macrocode}
  \ifx\@currext\@pkgextension
  \else
    \PackageError{kvoptions}{%
      \string\ProcessLocalKeyvalOptions\space is intended for packages only%
    }\@ehc
  \fi
%    \end{macrocode}
%     The package options are put into toks register \cs{toks@}.
%    \begin{macrocode}
  \begingroup
    \toks\tw@{}%
    \@ifundefined{opt@\@currname.\@currext}{%
      \toks@\expandafter{\KVO@temp}%
    }{%
      \toks@\expandafter\expandafter\expandafter{%
        \csname opt@\@currname.\@currext\endcsname
      }%
%    \end{macrocode}
%    Without default action we pass all options to \cs{setkeys}.
%    Otherwise we have to check which options are known.
%    These are passed to \cs{setkeys}. For the others the default
%    action is performed.
%    \begin{macrocode}
      \@ifundefined{KVO@default@\@currname.\@currext}{%
        \toks@\expandafter\expandafter\expandafter{%
          \expandafter\KVO@temp\the\toks@
        }%
      }{%
        \edef\CurrentOption{\the\toks@}%
        \toks@\expandafter{\KVO@temp}%
        \@for\CurrentOption:=\CurrentOption\do{%
          \@ifundefined{%
            KV@#1@\expandafter\KVO@getkey\CurrentOption=\@nil
          }{%
            \toks\tw@\expandafter{%
              \the\toks\expandafter\tw@\expandafter,\CurrentOption
            }%
          }{%
            \toks@\expandafter{%
              \the\expandafter\toks@\expandafter,\CurrentOption
            }%
          }%
        }%
      }%
    }%
    \edef\KVO@temp{\endgroup
      \noexpand\KVO@calldefault{\the\toks\tw@}%
      \noexpand\KVO@setkeys{#1}{\the\toks@}%
    }%
  \KVO@temp
%    \end{macrocode}
%    Some cleanup of \cs{ProcessOptions}.
%    \begin{macrocode}
  \let\CurrentOption\@empty
  \AtEndOfPackage{\let\@unprocessedoptions\relax}%
}
%    \end{macrocode}
%
% \subsubsection{Helper macros}
%
%    \begin{macro}{\KVO@getkey}
%    Extract the key part of a key=value pair.
%    \begin{macrocode}
\def\KVO@getkey#1=#2\@nil{#1}
%    \end{macrocode}
%    \end{macro}
%
%    \begin{macro}{\KVO@calldefault}
%    \begin{macrocode}
\def\KVO@calldefault#1{%
  \begingroup
    \def\x{#1}%
  \expandafter\endgroup
  \ifx\x\@empty
  \else
    \@for\CurrentOption:=#1\do{%
      \ifx\CurrentOption\@empty
      \else
        \expandafter\KVO@setcurrents\CurrentOption=\@nil
        \@nameuse{KVO@default@\@currname.\@currext}%
      \fi
    }%
  \fi
}
%    \end{macrocode}
%    \end{macro}
%    \begin{macro}{\KVO@setcurrents}
%    Extract the key part of a key=value pair.
%    \begin{macrocode}
\def\KVO@setcurrents#1=#2\@nil{%
  \def\CurrentOptionValue{#2}%
  \ifx\CurrentOptionValue\@empty
    \let\CurrentOptionKey\CurrentOption
    \let\CurrentOptionValue\relax
  \else
    \edef\CurrentOptionKey{\zap@space#1 \@empty}%
    \expandafter\KVO@setcurrentvalue\CurrentOption\@nil
  \fi
}
%    \end{macrocode}
%    \end{macro}
%    \begin{macro}{\KV@setcurrentvalue}
%    Here the value part is parsed. Package \xpackage{keyval}'s
%    \cs{KV@@sp@def} helps in removing spaces at the begin and
%    end of the value.
%    \begin{macrocode}
\def\KVO@setcurrentvalue#1=#2\@nil{%
  \KV@@sp@def\CurrentOptionValue{#2}%
}
%    \end{macrocode}
%    \end{macro}
%
% \subsection{\plainTeX}
%
%    Disable \LaTeX\ stuff.
%    \begin{macrocode}
\begingroup\expandafter\expandafter\expandafter\endgroup
\expandafter\ifx\csname documentclass\endcsname\relax
  \def\ProcessKeyvalOptions{%
    \@ifstar{}\@gobble
  }%
\fi
%    \end{macrocode}
%
%    \begin{macrocode}
\KVO@AtEnd%
%</package>
%    \end{macrocode}
%
% \subsection{Package \xpackage{kvoptions-patch}}
%
%    \begin{macrocode}
%<*patch>
\NeedsTeXFormat{LaTeX2e}
\begingroup\catcode61\catcode48\catcode32=10\relax%
  \catcode13=5 % ^^M
  \endlinechar=13 %
  \catcode123=1 % {
  \catcode125=2 % }
  \catcode64=11 % @
  \def\x{\endgroup
    \expandafter\edef\csname KVO@AtEnd\endcsname{%
      \endlinechar=\the\endlinechar\relax
      \catcode13=\the\catcode13\relax
      \catcode32=\the\catcode32\relax
      \catcode35=\the\catcode35\relax
      \catcode61=\the\catcode61\relax
      \catcode64=\the\catcode64\relax
      \catcode123=\the\catcode123\relax
      \catcode125=\the\catcode125\relax
    }%
  }%
\x\catcode61\catcode48\catcode32=10\relax%
\catcode13=5 % ^^M
\endlinechar=13 %
\catcode35=6 % #
\catcode64=11 % @
\catcode123=1 % {
\catcode125=2 % }
\def\TMP@EnsureCode#1#2{%
  \edef\KVO@AtEnd{%
    \KVO@AtEnd
    \catcode#1=\the\catcode#1\relax
  }%
  \catcode#1=#2\relax
}
\TMP@EnsureCode{39}{12}% '
\TMP@EnsureCode{40}{12}% (
\TMP@EnsureCode{41}{12}% )
\TMP@EnsureCode{43}{12}% +
\TMP@EnsureCode{44}{12}% ,
\TMP@EnsureCode{45}{12}% -
\TMP@EnsureCode{46}{12}% .
\TMP@EnsureCode{47}{12}% /
\TMP@EnsureCode{58}{12}% :
\TMP@EnsureCode{60}{12}% <
\TMP@EnsureCode{62}{12}% >
\TMP@EnsureCode{91}{12}% [
\TMP@EnsureCode{93}{12}% ]
\TMP@EnsureCode{96}{12}% `
\TMP@EnsureCode{124}{12}% |
\edef\KVO@AtEnd{\KVO@AtEnd\noexpand\endinput}
\ProvidesPackage{kvoptions-patch}%
  [2016/05/16 v3.12 LaTeX patch for keyval options (HO)]%
%    \end{macrocode}
%
%    Check for \eTeX.
%    \begin{macrocode}
\begingroup\expandafter\expandafter\expandafter\endgroup
\expandafter\ifx\csname eTeXversion\endcsname\relax
  \PackageWarningNoLine{kvoptions-patch}{%
    Package loading is aborted, because e-TeX is missing%
  }%
  \expandafter\KVO@AtEnd
\fi%
%    \end{macrocode}
%
%    Package \xpackage{etexcmds} for \cs{etex@unexpanded}.
%    \begin{macrocode}
\RequirePackage{etexcmds}[2007/09/09]
\ifetex@unexpanded
\else
  \PackageError{kvoptions-patch}{%
    Could not find eTeX's \string\unexpanded.\MessageBreak
    Try adding \string\RequirePackage\string{etexcmds\string} %
    before \string\documentclass%
  }\@ehd
  \expandafter\KVO@AtEnd
\fi%
%    \end{macrocode}
%
%    Check for package \xpackage{xkvltxp}.
%    \begin{macrocode}
\@ifpackageloaded{xkvltxp}{%
  \PackageWarningNoLine{kvoptions}{%
    Option `patch' cannot be used together with\MessageBreak
    package `xkvltxp' that is already loaded.\MessageBreak
    Therefore package loading is aborted%
  }%
  \KVO@AtEnd
}{}%
%    \end{macrocode}
%
%    \begin{macrocode}
\def\@if@ptions#1#2#3{%
  \begingroup
    \KVO@normalize\KVO@temp{#3}%
    \edef\x{\endgroup
      \noexpand\@if@pti@ns{%
        \detokenize\expandafter\expandafter\expandafter{%
          \csname opt@#2.#1\endcsname
        }%
      }{%
        \detokenize\expandafter{\KVO@temp}%
      }%
    }%
  \x
}
%    \end{macrocode}
%
%    \begin{macrocode}
\def\@pass@ptions#1#2#3{%
  \KVO@normalize\KVO@temp{#2}%
  \@ifundefined{opt@#3.#1}{%
    \expandafter\gdef\csname opt@#3.#1%
          \expandafter\endcsname\expandafter{%
      \KVO@temp
    }%
  }{%
    \expandafter\gdef\csname opt@#3.#1%
          \expandafter\expandafter\expandafter\endcsname
          \expandafter\expandafter\expandafter{%
      \csname opt@#3.#1\expandafter\endcsname\expandafter,\KVO@temp
    }%
  }%
}
%    \end{macrocode}
%
%    \begin{macrocode}
\def\ProcessOptions{%
  \let\ds@\@empty
  \@ifundefined{opt@\@currname.\@currext}{%
    \let\@curroptions\@empty
  }{%
    \expandafter\expandafter\expandafter\def
    \expandafter\expandafter\expandafter\@curroptions
    \expandafter\expandafter\expandafter{%
      \csname opt@\@currname.\@currext\endcsname
    }%
  }%
  \@ifstar\KVO@xprocess@ptions\KVO@process@ptions
}
%    \end{macrocode}
%
%    \begin{macrocode}
\def\KVO@process@ptions{%
  \@for\CurrentOption:=\@declaredoptions\do{%
    \ifx\CurrentOption\@empty
    \else
      \begingroup
        \ifx\@currext\@clsextension
          \toks@{}%
        \else
          \KVO@GetClassOptionsList
          \toks@\expandafter{\KVO@classoptionslist,}%
        \fi
        \toks\tw@\expandafter{\@curroptions}%
        \edef\x{\endgroup
          \noexpand\in@{,\CurrentOption,}{,\the\toks@\the\toks\tw@,}%
        }%
      \x
      \ifin@
        \KVO@use@ption
        \expandafter\let\csname ds@\CurrentOption\endcsname\@empty
      \fi
    \fi
  }%
  \KVO@process@pti@ns
}
%    \end{macrocode}
%
%    \begin{macrocode}
\def\KVO@xprocess@ptions{%
  \ifx\@currext\@clsextension
  \else
    \KVO@GetClassOptionsList
    \@for\CurrentOption:=\KVO@classoptionslist\do{%
      \ifx\CurrentOption\@empty
      \else
        \KVO@in@\CurrentOption\@declaredoptions
        \ifin@
          \KVO@use@ption
          \expandafter\let\csname ds@\CurrentOption\endcsname\@empty
        \fi
      \fi
    }%
  \fi
  \KVO@process@pti@ns
}
%    \end{macrocode}
%
%    \begin{macrocode}
\def\KVO@in@#1#2{%
  \in@false
  \begingroup
    \@for\x:=#2\do{%
      \ifx\x#1\relax
        \in@true
      \fi
    }%
    \edef\x{\endgroup
      \ifin@
        \noexpand\in@true
      \fi
    }%
  \x
}
%    \end{macrocode}
%
%    \begin{macrocode}
\def\KVO@process@pti@ns{%
  \@for\CurrentOption:=\@curroptions\do{%
    \@ifundefined{ds@\KVO@SanitizedCurrentOption}{%
      \KVO@use@ption
      \default@ds
    }%
    \KVO@use@ption
  }%
  \@for\CurrentOption:=\@declaredoptions\do{%
    \expandafter\let\csname ds@\CurrentOption\endcsname\relax
  }%
  \let\CurrentOption\@empty
  \let\@fileswith@pti@ns\@@fileswith@pti@ns
  \AtEndOfPackage{\let\@unprocessedoptions\relax}%
}
%    \end{macrocode}
%
%    \begin{macrocode}
\def\KVO@use@ption{%
  \begingroup
    \edef\x{\endgroup
      \noexpand\@removeelement{%
        \detokenize\expandafter{\CurrentOption}%
      }{%
        \detokenize\expandafter{\@unusedoptionlist}%
      }%
    }%
  \x\@unusedoptionlist
  \csname ds@\KVO@SanitizedCurrentOption\endcsname
}
%    \end{macrocode}
%
%    \begin{macrocode}
\def\OptionNotUsed{%
  \ifx\@currext\@clsextension
    \xdef\@unusedoptionlist{%
      \ifx\@unusedoptionlist\@empty
      \else
        \detokenize\expandafter{\@unusedoptionlist,}%
      \fi
      \detokenize\expandafter{\CurrentOption}%
    }%
  \fi
}
%    \end{macrocode}
%
%    Variant of \cs{ExecuteOptions} that better protects \cs{CurrentOption}.
%    \begin{macrocode}
\def\CurrentOption@SaveLevel{0}
\def\ExecuteOptions{%
  \expandafter\KVO@ExecuteOptions
      \csname CurrentOption@\CurrentOption@SaveLevel\endcsname
}
\def\KVO@ExecuteOptions#1#2{%
  \let#1\CurrentOption
  \edef\CurrentOption@SaveLevel{%
    \the\numexpr\CurrentOption@SaveLevel+1%
  }%
  \@for\CurrentOption:=#2\do{%
    \csname ds@\CurrentOption\endcsname
  }%
  \edef\CurrentOption@SaveLevel{%
    \the\numexpr\CurrentOption@SaveLevel-1%
  }%
  \let\CurrentOption#1%
}
%    \end{macrocode}
%
%    \begin{macrocode}
\def\KVO@fileswith@pti@ns#1[#2]#3[#4]{%
  \ifx#1\@clsextension
    \ifx\@classoptionslist\relax
      \KVO@normalize\KVO@temp{#2}%
      \expandafter\gdef\expandafter\@classoptionslist\expandafter{%
        \KVO@temp
      }%
      \def\reserved@a{%
        \KVO@onefilewithoptions{#3}[{#2}][{#4}]#1%
        \@documentclasshook
      }%
    \else
      \def\reserved@a{%
        \KVO@onefilewithoptions{#3}[{#2}][{#4}]#1%
      }%
    \fi
  \else
    \begingroup
      \let\KVO@temp\relax
      \let\KVO@onefilewithoptions\relax
      \let\@pkgextension\relax
      \def\reserved@b##1,{%
        \ifx\@nil##1\relax
        \else
          \ifx\relax##1\relax
          \else
            \KVO@onefilewithoptions{##1}[{\KVO@temp}][{#4}]%
            \@pkgextension
          \fi
          \expandafter\reserved@b
        \fi
      }%
      \edef\reserved@a{\zap@space#3 \@empty}%
      \edef\reserved@a{\expandafter\reserved@b\reserved@a,\@nil,}%
      \toks@{#2}%
      \def\KVO@temp{\the\toks@}%
    \edef\reserved@a{\endgroup \reserved@a}%
  \fi
  \reserved@a
}
%    \end{macrocode}
%
%    \begin{macrocode}
\def\KVO@onefilewithoptions#1[#2][#3]#4{%
  \@pushfilename
  \xdef\@currname{#1}%
  \global\let\@currext#4%
  \expandafter\let\csname\@currname.\@currext-h@@k\endcsname\@empty
  \let\CurrentOption\@empty
  \@reset@ptions
  \makeatletter
  \def\reserved@a{%
    \@ifl@aded\@currext{#1}{%
      \@if@ptions\@currext{#1}{#2}{%
      }{%
        \begingroup
          \@ifundefined{opt@#1.\@currext}{%
            \def\x{}%
          }{%
            \edef\x{%
              \expandafter\expandafter\expandafter\strip@prefix
              \expandafter\meaning\csname opt@#1.\@currext\endcsname
            }%
          }%
          \def\y{#2}%
          \edef\y{\expandafter\strip@prefix\meaning\y}%
          \@latex@error{Option clash for \@cls@pkg\space #1}{%
            The package #1 has already been loaded %
            with options:\MessageBreak
            \space\space[\x]\MessageBreak
            There has now been an attempt to load it %
             with options\MessageBreak
            \space\space[\y]\MessageBreak
            Adding the global options:\MessageBreak
            \space\space
                 \x,\y\MessageBreak
            to your \noexpand\documentclass declaration may fix this.%
            \MessageBreak
            Try typing \space <return> \space to proceed.%
          }%
        \endgroup
      }%
    }{%
      \@pass@ptions\@currext{#2}{#1}%
      \global\expandafter
      \let\csname ver@\@currname.\@currext\endcsname\@empty
      \InputIfFileExists
        {\@currname.\@currext}%
        {}%
        {\@missingfileerror\@currname\@currext}%
      \let\@unprocessedoptions\@@unprocessedoptions
      \csname\@currname.\@currext-h@@k\endcsname
      \expandafter\let\csname\@currname.\@currext-h@@k\endcsname
              \@undefined
      \@unprocessedoptions
    }%
    \@ifl@ter\@currext{#1}{#3}{%
    }{%
      \@latex@warning@no@line{%
        You have requested,\on@line, %
        version\MessageBreak
          #3' of \@cls@pkg\space #1,\MessageBreak
        but only version\MessageBreak
         `\csname ver@#1.\@currext\endcsname'\MessageBreak
        is available%
      }%
    }%
    \ifx\@currext\@clsextension\let\LoadClass\@twoloadclasserror\fi
    \@popfilename
    \@reset@ptions
  }%
  \reserved@a
}
%    \end{macrocode}
%
%    \begin{macrocode}
\def\@unknownoptionerror{%
  \@latex@error{%
    Unknown option `\KVO@SanitizedCurrentOption' %
    for \@cls@pkg\space`\@currname'%
  }{%
    The option `\KVO@SanitizedCurrentOption' was not declared in %
    \@cls@pkg\space`\@currname', perhaps you\MessageBreak
    misspelled its name. %
    Try typing \space <return> %
    \space to proceed.%
  }%
}
%    \end{macrocode}
%
%    \begin{macrocode}
\def\@@unprocessedoptions{%
  \ifx\@currext\@pkgextension
    \@ifundefined{opt@\@currname.\@currext}{%
      \let\@curroptions\@empty
    }{%
      \expandafter\let\expandafter\@curroptions
          \csname opt@\@currname.\@currext\endcsname
    }%
    \@for\CurrentOption:=\@curroptions\do{%
        \ifx\CurrentOption\@empty\else\@unknownoptionerror\fi
    }%
  \fi
}
%    \end{macrocode}
%
%    \begin{macrocode}
\def\KVO@SanitizedCurrentOption{%
  \expandafter\strip@prefix\meaning\CurrentOption
}
%    \end{macrocode}
%
%    Normalize option list.
%    \begin{macrocode}
\def\KVO@normalize#1#2{%
  \let\KVO@result\@empty
  \KVO@splitcomma#2,\@nil
  \let#1\KVO@result
}
\def\KVO@splitcomma#1,#2\@nil{%
  \KVO@ifempty{#1}{}{%
    \KVO@checkkv#1=\@nil
  }%
  \KVO@ifempty{#2}{}{\KVO@splitcomma#2\@nil}%
}
\def\KVO@ifempty#1{%
  \expandafter\ifx\expandafter\\\detokenize{#1}\\%
    \expandafter\@firstoftwo
  \else
    \expandafter\@secondoftwo
  \fi
}
\def\KVO@checkkv#1=#2\@nil{%
  \KVO@ifempty{#2}{%
    % option without value
    \edef\KVO@x{\zap@space#1 \@empty}%
    \ifx\KVO@x\@empty
      % ignore empty option
    \else
      % append to list
      \edef\KVO@result{%
        \etex@unexpanded\expandafter{\KVO@result},\KVO@x
      }%
    \fi
  }{%
    % #1: "key", #2: "value="
    % add key part
    \edef\KVO@result{%
      \etex@unexpanded\expandafter{\KVO@result},%
      \zap@space#1 \@empty
    }%
    \futurelet\@let@token\KVO@checkfirsttok#2 \@nil| = \@nil|\KVO@nil
  }%
}
\def\KVO@checkfirsttok{%
  \ifx\@let@token\bgroup
    % no space at start
    \expandafter\KVO@removelastspace\expandafter=%
    % "<value><spaceopt>= \@nil"
  \else
    \expandafter\KVO@checkfirstA
  \fi
}
\def\KVO@checkfirstA#1 #2\@nil{%
  \KVO@ifempty{#2}{%
    \KVO@removelastspace=#1 \@nil
  }{%
    \KVO@ifempty{#1}{%
      \KVO@removelastspace=#2\@nil
    }{%
      \KVO@removelastspace=#1 #2\@nil
    }%
  }%
}
\def\KVO@removelastspace#1 = \@nil|#2\KVO@nil{%
  \KVO@ifempty{#2}{%
    \edef\KVO@result{%
      \etex@unexpanded\expandafter{\KVO@result}%
      \etex@unexpanded\expandafter{\KVO@removegarbage#1\KVO@nil}%
    }%
  }{%
    \edef\KVO@result{%
      \etex@unexpanded\expandafter{\KVO@result}%
      \etex@unexpanded{#1}%
    }%
  }%
}
\def\KVO@removegarbage#1= \@nil#2\KVO@nil{#1}%
%    \end{macrocode}
%
%    Arguments |#1| and |#2| are macros.
%    \begin{macrocode}
\def\KVO@removeelement#1#2{%
  \begingroup
    \toks@={}%
    \@for\x:=#2\do{%
      \ifx\x\@empty
      \else
        \ifx\x#1\relax
        \else
          \edef\t{\the\toks@}%
          \ifx\t\@empty
          \else
            \toks@\expandafter{\the\toks@,}%
          \fi
          \toks@\expandafter{\the\expandafter\toks@\x}%
        \fi
      \fi
    }%
    \edef\x{\endgroup
      \def\noexpand#2{\the\toks@}%
    }%
  \x
}
%    \end{macrocode}
%
%
%    \begin{macrocode}
\let\@@fileswith@pti@ns\KVO@fileswith@pti@ns
\ifx\@fileswith@pti@ns\@badrequireerror
\else
  \let\@fileswith@pti@ns\KVO@fileswith@pti@ns
\fi
%    \end{macrocode}
%
%    \begin{macro}{\KVO@Patch}
%    \begin{macrocode}
\let\KVO@Patch=Y
%    \end{macrocode}
%    \end{macro}
%
%    \begin{macrocode}
\KVO@AtEnd%
%</patch>
%    \end{macrocode}
%
% \section{Test}
%
% \subsection{Preface for standard catcode check}
%
%    \begin{macrocode}
%<*test1>
\input miniltx.tex\relax
%</test1>
%    \end{macrocode}
%
% \subsection{Catcode checks for loading}
%
%    \begin{macrocode}
%<*test1>
%    \end{macrocode}
%    \begin{macrocode}
\catcode`\{=1 %
\catcode`\}=2 %
\catcode`\#=6 %
\catcode`\@=11 %
\expandafter\ifx\csname count@\endcsname\relax
  \countdef\count@=255 %
\fi
\expandafter\ifx\csname @gobble\endcsname\relax
  \long\def\@gobble#1{}%
\fi
\expandafter\ifx\csname @firstofone\endcsname\relax
  \long\def\@firstofone#1{#1}%
\fi
\expandafter\ifx\csname loop\endcsname\relax
  \expandafter\@firstofone
\else
  \expandafter\@gobble
\fi
{%
  \def\loop#1\repeat{%
    \def\body{#1}%
    \iterate
  }%
  \def\iterate{%
    \body
      \let\next\iterate
    \else
      \let\next\relax
    \fi
    \next
  }%
  \let\repeat=\fi
}%
\def\RestoreCatcodes{}
\count@=0 %
\loop
  \edef\RestoreCatcodes{%
    \RestoreCatcodes
    \catcode\the\count@=\the\catcode\count@\relax
  }%
\ifnum\count@<255 %
  \advance\count@ 1 %
\repeat

\def\RangeCatcodeInvalid#1#2{%
  \count@=#1\relax
  \loop
    \catcode\count@=15 %
  \ifnum\count@<#2\relax
    \advance\count@ 1 %
  \repeat
}
\def\RangeCatcodeCheck#1#2#3{%
  \count@=#1\relax
  \loop
    \ifnum#3=\catcode\count@
    \else
      \errmessage{%
        Character \the\count@\space
        with wrong catcode \the\catcode\count@\space
        instead of \number#3%
      }%
    \fi
  \ifnum\count@<#2\relax
    \advance\count@ 1 %
  \repeat
}
\def\space{ }
\expandafter\ifx\csname LoadCommand\endcsname\relax
  \def\LoadCommand{\input kvoptions.sty\relax}%
\fi
\def\Test{%
  \RangeCatcodeInvalid{0}{47}%
  \RangeCatcodeInvalid{58}{64}%
  \RangeCatcodeInvalid{91}{96}%
  \RangeCatcodeInvalid{123}{255}%
  \catcode`\@=12 %
  \catcode`\\=0 %
  \catcode`\%=14 %
  \LoadCommand
  \RangeCatcodeCheck{0}{36}{15}%
  \RangeCatcodeCheck{37}{37}{14}%
  \RangeCatcodeCheck{38}{47}{15}%
  \RangeCatcodeCheck{48}{57}{12}%
  \RangeCatcodeCheck{58}{63}{15}%
  \RangeCatcodeCheck{64}{64}{12}%
  \RangeCatcodeCheck{65}{90}{11}%
  \RangeCatcodeCheck{91}{91}{15}%
  \RangeCatcodeCheck{92}{92}{0}%
  \RangeCatcodeCheck{93}{96}{15}%
  \RangeCatcodeCheck{97}{122}{11}%
  \RangeCatcodeCheck{123}{255}{15}%
  \RestoreCatcodes
}
\Test
\csname @@end\endcsname
\end
%    \end{macrocode}
%    \begin{macrocode}
%</test1>
%    \end{macrocode}
%
%    \begin{macrocode}
%<*test2>
\NeedsTeXFormat{LaTeX2e}
\makeatletter
\catcode`\@=11 %
\def\RestoreCatcodes{}
\count@=0 %
\loop
  \edef\RestoreCatcodes{%
    \RestoreCatcodes
    \catcode\the\count@=\the\catcode\count@\relax
  }%
\ifnum\count@<255 %
  \advance\count@\@ne
\repeat

\def\RangeCatcodeInvalid#1#2{%
  \count@=#1\relax
  \loop
    \catcode\count@=15 %
  \ifnum\count@<#2\relax
    \advance\count@\@ne
  \repeat
}
\def\Test#1{%
  \RangeCatcodeInvalid{0}{47}%
  \RangeCatcodeInvalid{58}{64}%
  \RangeCatcodeInvalid{91}{96}%
  \RangeCatcodeInvalid{123}{255}%
  \catcode`\@=12 %
  \catcode`\\=0 %
  \catcode`\{=1 %
  \catcode`\}=2 %
  \catcode`\#=6 %
  \catcode`\[=12 %
  \catcode`\]=12 %
  \catcode`\%=14 %
  \catcode`\ =10 %
  \catcode13=5 %
  #1\relax
  \RestoreCatcodes
}
\Test{\RequirePackage{kvoptions-patch}}%
\Test{\RequirePackage{kvoptions}}%
\csname @@end\endcsname
%</test2>
%    \end{macrocode}
%
%    \begin{macrocode}
%<*test3>
\NeedsTeXFormat{LaTeX2e}
\makeatletter
\RequirePackage{kvoptions}[2016/05/16]
\def\msg#{\immediate\write16}
\define@key{testfamily}{testkey}{%
  \msg{[testfamily/testkey/#1]}%
}
\define@key{testfamily}{testdefaultkey}[testdefault]{%
  \msg{[testfamily/testdefaultkey/#1]}%
}
\AddToKeyvalOption{testfamily}{testkey}{%
  \msg{[addition/#1]}%
}
\AddToKeyvalOption{testfamily}{testdefaultkey}{%
  \msg{[addition/#1]}%
}
\setkeys{testfamily}{%
  testkey=testA,%
  testdefaultkey=testB,%
  testdefaultkey,%
}
\SetupKeyvalOptions{%
  family=testfamily%
}
\AddToKeyvalOption*{testkey}{%
  \msg{[star addition/#1]}%
}
\AddToKeyvalOption*{testdefaultkey}{%
  \msg{[star addition/#1]}%
}
\setkeys{testfamily}{%
  testkey=testA,%
  testdefaultkey=testB,%
  testdefaultkey,%
}
\@@end
%</test3>
%    \end{macrocode}
%
%    \begin{macrocode}
%<*test4pkg>
\NeedsTeXFormat{LaTeX2e}
\ProvidesPackage{kvoptions-test4}[2016/05/16 package for testing]
\RequirePackage{kvoptions}[2016/05/16]
\SetupKeyvalOptions{%
  family=FOO,%
  prefix=foo,%
  setkeys=\kvsetkeys,%
}
\DeclareStringOption{str}
\define@key{FOO}{set}{%
  \setkeys{BAR}{strbar={#1}}%
}
\define@key{BAR}{strbar}{%
  \def\foostr{[BAR:#1]}%
}
\ProcessKeyvalOptions*
%</test4pkg>
%    \end{macrocode}
%    \begin{macrocode}
%<*test4>
\NeedsTeXFormat{LaTeX2e}
\ProvidesFile{kvoptions-test4.tex}[2016/05/16 test file]
\RequirePackage[%
  str=A,set=B,str=C,%
]{kvoptions-test4}[2016/05/16]
\def\TestExpected{C}
\ifx\foostr\TestExpected
  \typeout{* Test ok.}%
\else
  \typeout{* Result: [\foostr]}%
  \typeout{* Expected: [\TestExpected]}%
  \errmessage{Test failed!}%
\fi
\csname @@end\endcsname\end
%</test4>
%    \end{macrocode}
%
% \section{Installation}
%
% \subsection{Download}
%
% \paragraph{Package.} This package is available on
% CTAN\footnote{\CTANpkg{kvoptions}}:
% \begin{description}
% \item[\CTAN{macros/latex/contrib/oberdiek/kvoptions.dtx}] The source file.
% \item[\CTAN{macros/latex/contrib/oberdiek/kvoptions.pdf}] Documentation.
% \end{description}
%
%
% \paragraph{Bundle.} All the packages of the bundle `oberdiek'
% are also available in a TDS compliant ZIP archive. There
% the packages are already unpacked and the documentation files
% are generated. The files and directories obey the TDS standard.
% \begin{description}
% \item[\CTANinstall{install/macros/latex/contrib/oberdiek.tds.zip}]
% \end{description}
% \emph{TDS} refers to the standard ``A Directory Structure
% for \TeX\ Files'' (\CTAN{tds/tds.pdf}). Directories
% with \xfile{texmf} in their name are usually organized this way.
%
% \subsection{Bundle installation}
%
% \paragraph{Unpacking.} Unpack the \xfile{oberdiek.tds.zip} in the
% TDS tree (also known as \xfile{texmf} tree) of your choice.
% Example (linux):
% \begin{quote}
%   |unzip oberdiek.tds.zip -d ~/texmf|
% \end{quote}
%
% \paragraph{Script installation.}
% Check the directory \xfile{TDS:scripts/oberdiek/} for
% scripts that need further installation steps.
%
% \subsection{Package installation}
%
% \paragraph{Unpacking.} The \xfile{.dtx} file is a self-extracting
% \docstrip\ archive. The files are extracted by running the
% \xfile{.dtx} through \plainTeX:
% \begin{quote}
%   \verb|tex kvoptions.dtx|
% \end{quote}
%
% \paragraph{TDS.} Now the different files must be moved into
% the different directories in your installation TDS tree
% (also known as \xfile{texmf} tree):
% \begin{quote}
% \def\t{^^A
% \begin{tabular}{@{}>{\ttfamily}l@{ $\rightarrow$ }>{\ttfamily}l@{}}
%   kvoptions.sty & tex/latex/oberdiek/kvoptions.sty\\
%   kvoptions-patch.sty & tex/latex/oberdiek/kvoptions-patch.sty\\
%   kvoptions.pdf & doc/latex/oberdiek/kvoptions.pdf\\
%   example-mycolorsetup.sty & doc/latex/oberdiek/example-mycolorsetup.sty\\
%   test/kvoptions-test1.tex & doc/latex/oberdiek/test/kvoptions-test1.tex\\
%   test/kvoptions-test2.tex & doc/latex/oberdiek/test/kvoptions-test2.tex\\
%   test/kvoptions-test3.tex & doc/latex/oberdiek/test/kvoptions-test3.tex\\
%   test/kvoptions-test4.tex & doc/latex/oberdiek/test/kvoptions-test4.tex\\
%   test/kvoptions-test4.sty & doc/latex/oberdiek/test/kvoptions-test4.sty\\
%   kvoptions.dtx & source/latex/oberdiek/kvoptions.dtx\\
% \end{tabular}^^A
% }^^A
% \sbox0{\t}^^A
% \ifdim\wd0>\linewidth
%   \begingroup
%     \advance\linewidth by\leftmargin
%     \advance\linewidth by\rightmargin
%   \edef\x{\endgroup
%     \def\noexpand\lw{\the\linewidth}^^A
%   }\x
%   \def\lwbox{^^A
%     \leavevmode
%     \hbox to \linewidth{^^A
%       \kern-\leftmargin\relax
%       \hss
%       \usebox0
%       \hss
%       \kern-\rightmargin\relax
%     }^^A
%   }^^A
%   \ifdim\wd0>\lw
%     \sbox0{\small\t}^^A
%     \ifdim\wd0>\linewidth
%       \ifdim\wd0>\lw
%         \sbox0{\footnotesize\t}^^A
%         \ifdim\wd0>\linewidth
%           \ifdim\wd0>\lw
%             \sbox0{\scriptsize\t}^^A
%             \ifdim\wd0>\linewidth
%               \ifdim\wd0>\lw
%                 \sbox0{\tiny\t}^^A
%                 \ifdim\wd0>\linewidth
%                   \lwbox
%                 \else
%                   \usebox0
%                 \fi
%               \else
%                 \lwbox
%               \fi
%             \else
%               \usebox0
%             \fi
%           \else
%             \lwbox
%           \fi
%         \else
%           \usebox0
%         \fi
%       \else
%         \lwbox
%       \fi
%     \else
%       \usebox0
%     \fi
%   \else
%     \lwbox
%   \fi
% \else
%   \usebox0
% \fi
% \end{quote}
% If you have a \xfile{docstrip.cfg} that configures and enables \docstrip's
% TDS installing feature, then some files can already be in the right
% place, see the documentation of \docstrip.
%
% \subsection{Refresh file name databases}
%
% If your \TeX~distribution
% (\TeX\,Live, \mikTeX, \dots) relies on file name databases, you must refresh
% these. For example, \TeX\,Live\ users run \verb|texhash| or
% \verb|mktexlsr|.
%
% \subsection{Some details for the interested}
%
% \paragraph{Unpacking with \LaTeX.}
% The \xfile{.dtx} chooses its action depending on the format:
% \begin{description}
% \item[\plainTeX:] Run \docstrip\ and extract the files.
% \item[\LaTeX:] Generate the documentation.
% \end{description}
% If you insist on using \LaTeX\ for \docstrip\ (really,
% \docstrip\ does not need \LaTeX), then inform the autodetect routine
% about your intention:
% \begin{quote}
%   \verb|latex \let\install=y% \iffalse meta-comment
%
% File: kvoptions.dtx
% Version: 2016/05/16 v3.12
% Info: Key value format for package options
%
% Copyright (C)
%    2004, 2006, 2007, 2009-2011 Heiko Oberdiek
%    2016-2019 Oberdiek Package Support Group
%    https://github.com/ho-tex/oberdiek/issues
%
% This work may be distributed and/or modified under the
% conditions of the LaTeX Project Public License, either
% version 1.3c of this license or (at your option) any later
% version. This version of this license is in
%    https://www.latex-project.org/lppl/lppl-1-3c.txt
% and the latest version of this license is in
%    https://www.latex-project.org/lppl.txt
% and version 1.3 or later is part of all distributions of
% LaTeX version 2005/12/01 or later.
%
% This work has the LPPL maintenance status "maintained".
%
% The Current Maintainers of this work are
% Heiko Oberdiek and the Oberdiek Package Support Group
% https://github.com/ho-tex/oberdiek/issues
%
% This work consists of the main source file kvoptions.dtx
% and the derived files
%    kvoptions.sty, kvoptions.pdf, kvoptions.ins, kvoptions.drv,
%    kvoptions-patch.sty, example-mycolorsetup.sty,
%    kvoptions-test1.tex, kvoptions-test2.tex,
%    kvoptions-test3.tex, kvoptions-test4.tex,
%    kvoptions-test4.sty.
%
% Distribution:
%    CTAN:macros/latex/contrib/oberdiek/kvoptions.dtx
%    CTAN:macros/latex/contrib/oberdiek/kvoptions.pdf
%
% Unpacking:
%    (a) If kvoptions.ins is present:
%           tex kvoptions.ins
%    (b) Without kvoptions.ins:
%           tex kvoptions.dtx
%    (c) If you insist on using LaTeX
%           latex \let\install=y% \iffalse meta-comment
%
% File: kvoptions.dtx
% Version: 2016/05/16 v3.12
% Info: Key value format for package options
%
% Copyright (C)
%    2004, 2006, 2007, 2009-2011 Heiko Oberdiek
%    2016-2019 Oberdiek Package Support Group
%    https://github.com/ho-tex/oberdiek/issues
%
% This work may be distributed and/or modified under the
% conditions of the LaTeX Project Public License, either
% version 1.3c of this license or (at your option) any later
% version. This version of this license is in
%    https://www.latex-project.org/lppl/lppl-1-3c.txt
% and the latest version of this license is in
%    https://www.latex-project.org/lppl.txt
% and version 1.3 or later is part of all distributions of
% LaTeX version 2005/12/01 or later.
%
% This work has the LPPL maintenance status "maintained".
%
% The Current Maintainers of this work are
% Heiko Oberdiek and the Oberdiek Package Support Group
% https://github.com/ho-tex/oberdiek/issues
%
% This work consists of the main source file kvoptions.dtx
% and the derived files
%    kvoptions.sty, kvoptions.pdf, kvoptions.ins, kvoptions.drv,
%    kvoptions-patch.sty, example-mycolorsetup.sty,
%    kvoptions-test1.tex, kvoptions-test2.tex,
%    kvoptions-test3.tex, kvoptions-test4.tex,
%    kvoptions-test4.sty.
%
% Distribution:
%    CTAN:macros/latex/contrib/oberdiek/kvoptions.dtx
%    CTAN:macros/latex/contrib/oberdiek/kvoptions.pdf
%
% Unpacking:
%    (a) If kvoptions.ins is present:
%           tex kvoptions.ins
%    (b) Without kvoptions.ins:
%           tex kvoptions.dtx
%    (c) If you insist on using LaTeX
%           latex \let\install=y\input{kvoptions.dtx}
%        (quote the arguments according to the demands of your shell)
%
% Documentation:
%    (a) If kvoptions.drv is present:
%           latex kvoptions.drv
%    (b) Without kvoptions.drv:
%           latex kvoptions.dtx; ...
%    The class ltxdoc loads the configuration file ltxdoc.cfg
%    if available. Here you can specify further options, e.g.
%    use A4 as paper format:
%       \PassOptionsToClass{a4paper}{article}
%
%    Programm calls to get the documentation (example):
%       pdflatex kvoptions.dtx
%       makeindex -s gind.ist kvoptions.idx
%       pdflatex kvoptions.dtx
%       makeindex -s gind.ist kvoptions.idx
%       pdflatex kvoptions.dtx
%
% Installation:
%    TDS:tex/latex/oberdiek/kvoptions.sty
%    TDS:tex/latex/oberdiek/kvoptions-patch.sty
%    TDS:doc/latex/oberdiek/kvoptions.pdf
%    TDS:doc/latex/oberdiek/example-mycolorsetup.sty
%    TDS:doc/latex/oberdiek/test/kvoptions-test1.tex
%    TDS:doc/latex/oberdiek/test/kvoptions-test2.tex
%    TDS:doc/latex/oberdiek/test/kvoptions-test3.tex
%    TDS:doc/latex/oberdiek/test/kvoptions-test4.tex
%    TDS:doc/latex/oberdiek/test/kvoptions-test4.sty
%    TDS:source/latex/oberdiek/kvoptions.dtx
%
%<*ignore>
\begingroup
  \catcode123=1 %
  \catcode125=2 %
  \def\x{LaTeX2e}%
\expandafter\endgroup
\ifcase 0\ifx\install y1\fi\expandafter
         \ifx\csname processbatchFile\endcsname\relax\else1\fi
         \ifx\fmtname\x\else 1\fi\relax
\else\csname fi\endcsname
%</ignore>
%<*install>
\input docstrip.tex
\Msg{************************************************************************}
\Msg{* Installation}
\Msg{* Package: kvoptions 2016/05/16 v3.12 Key value format for package options (HO)}
\Msg{************************************************************************}

\keepsilent
\askforoverwritefalse

\let\MetaPrefix\relax
\preamble

This is a generated file.

Project: kvoptions
Version: 2016/05/16 v3.12

Copyright (C)
   2004, 2006, 2007, 2009-2011 Heiko Oberdiek
   2016-2019 Oberdiek Package Support Group

This work may be distributed and/or modified under the
conditions of the LaTeX Project Public License, either
version 1.3c of this license or (at your option) any later
version. This version of this license is in
   https://www.latex-project.org/lppl/lppl-1-3c.txt
and the latest version of this license is in
   https://www.latex-project.org/lppl.txt
and version 1.3 or later is part of all distributions of
LaTeX version 2005/12/01 or later.

This work has the LPPL maintenance status "maintained".

The Current Maintainers of this work are
Heiko Oberdiek and the Oberdiek Package Support Group
https://github.com/ho-tex/oberdiek/issues


This work consists of the main source file kvoptions.dtx
and the derived files
   kvoptions.sty, kvoptions.pdf, kvoptions.ins, kvoptions.drv,
   kvoptions-patch.sty, example-mycolorsetup.sty,
   kvoptions-test1.tex, kvoptions-test2.tex,
   kvoptions-test3.tex, kvoptions-test4.tex,
   kvoptions-test4.sty.

\endpreamble
\let\MetaPrefix\DoubleperCent

\generate{%
  \file{kvoptions.ins}{\from{kvoptions.dtx}{install}}%
  \file{kvoptions.drv}{\from{kvoptions.dtx}{driver}}%
  \usedir{tex/latex/oberdiek}%
  \file{kvoptions.sty}{\from{kvoptions.dtx}{package}}%
  \file{kvoptions-patch.sty}{\from{kvoptions.dtx}{patch}}%
  \usedir{doc/latex/oberdiek}%
  \file{example-mycolorsetup.sty}{\from{kvoptions.dtx}{example}}%
%  \usedir{doc/latex/oberdiek/test}%
%  \file{kvoptions-test1.tex}{\from{kvoptions.dtx}{test1}}%
%  \file{kvoptions-test2.tex}{\from{kvoptions.dtx}{test2}}%
%  \file{kvoptions-test3.tex}{\from{kvoptions.dtx}{test3}}%
%  \file{kvoptions-test4.tex}{\from{kvoptions.dtx}{test4}}%
%  \file{kvoptions-test4.sty}{\from{kvoptions.dtx}{test4pkg}}%
  \nopreamble
  \nopostamble
%  \usedir{source/latex/oberdiek/catalogue}%
%  \file{kvoptions.xml}{\from{kvoptions.dtx}{catalogue}}%
}

\catcode32=13\relax% active space
\let =\space%
\Msg{************************************************************************}
\Msg{*}
\Msg{* To finish the installation you have to move the following}
\Msg{* files into a directory searched by TeX:}
\Msg{*}
\Msg{*     kvoptions.sty, kvoptions-patch.sty}
\Msg{*}
\Msg{* To produce the documentation run the file `kvoptions.drv'}
\Msg{* through LaTeX.}
\Msg{*}
\Msg{* Happy TeXing!}
\Msg{*}
\Msg{************************************************************************}

\endbatchfile
%</install>
%<*ignore>
\fi
%</ignore>
%<*driver>
\NeedsTeXFormat{LaTeX2e}
\ProvidesFile{kvoptions.drv}%
  [2016/05/16 v3.12 Key value format for package options (HO)]%
\documentclass{ltxdoc}
\usepackage{holtxdoc}[2011/11/22]
\begin{document}
  \DocInput{kvoptions.dtx}%
\end{document}
%</driver>
% \fi
%
%
% \CharacterTable
%  {Upper-case    \A\B\C\D\E\F\G\H\I\J\K\L\M\N\O\P\Q\R\S\T\U\V\W\X\Y\Z
%   Lower-case    \a\b\c\d\e\f\g\h\i\j\k\l\m\n\o\p\q\r\s\t\u\v\w\x\y\z
%   Digits        \0\1\2\3\4\5\6\7\8\9
%   Exclamation   \!     Double quote  \"     Hash (number) \#
%   Dollar        \$     Percent       \%     Ampersand     \&
%   Acute accent  \'     Left paren    \(     Right paren   \)
%   Asterisk      \*     Plus          \+     Comma         \,
%   Minus         \-     Point         \.     Solidus       \/
%   Colon         \:     Semicolon     \;     Less than     \<
%   Equals        \=     Greater than  \>     Question mark \?
%   Commercial at \@     Left bracket  \[     Backslash     \\
%   Right bracket \]     Circumflex    \^     Underscore    \_
%   Grave accent  \`     Left brace    \{     Vertical bar  \|
%   Right brace   \}     Tilde         \~}
%
% \GetFileInfo{kvoptions.drv}
%
% \title{The \xpackage{kvoptions} package}
% \date{2016/05/16 v3.12}
% \author{Heiko Oberdiek\thanks
% {Please report any issues at \url{https://github.com/ho-tex/oberdiek/issues}}}
%
% \maketitle
%
% \begin{abstract}
% This package is intended for package authors who want to
% use options in key value format for their package options.
% \end{abstract}
%
% \tableofcontents
%
% \def\M#1{\texttt{\{}\meta{#1}\texttt{\}}}
% \def\O#1{\texttt{[}\meta{#1}\texttt{]}}
%
% \section{Introduction}
%
% First I want to recommend the very good review article
% ``A guide to key-value methods'' by Joseph Wright \cite{tb94wright}.
% It introduces the different key-value packages and compares them.
%
% \subsection{The beginning}
%
% This package \xpackage{kvoptions} addresses class or package writers
% that want to allow their users to specify options
% as key value pairs, e.g.:
% \begin{quote}
% |\documentclass[verbose=false,name=me]{myclass}|\\
% |\usepackage[format=print]{mylayout}|
% \end{quote}
% Prominent example is package \xpackage{hyperref},
% probably the first package that offers this service.
% It's \cs{ProcessOptionsWithKV} is often copied und
% used in other packages, e.g. package \xpackage{helvet}
% that uses this interface for its option \xoption{scaled}.
%
% However copying code is not the most modern software
% development technique. And \xpackage{hyperref}'s
% code for \cs{ProcessOptionsWithKV} was changed
% to fix bugs. The version used in other packages
% depends on the time of copying and the awareness
% of \xpackage{hyperref}'s changes.
% Now the code is sourced out into this package and
% available for other package or class writers.
%
% \subsection{Overview}
%
% Package \xpackage{kvoptions} connects package
% \xpackage{\emph{k}ey\emph{v}al}
% with \LaTeX's package and class \emph{options}:
% \begin{center}
% \def\T#1{\bfseries #1}
% \renewcommand{\arraystretch}{1.5}
% \begin{tabular}{l|l|l}
%   \T{Package \xpackage{keyval}}
%   & \T{Package \xpackage{kvoptions}}
%   & \T{\LaTeX\ kernel}
%   \\
% \hline
%     \cs{define@key}
%   &
%     \renewcommand*{\arraystretch}{1}^^A
%     \begin{tabular}[t]{@{}l@{}}
%       \cs{DeclareVoidOption}\\
%       \cs{DeclareStringOption}\\
%       \cs{DeclareBoolOption}\\
%       \cs{DeclareComplementaryOption}\\
%       \cs{DisableKeyvalOption}
%     \end{tabular}
%   & \cs{DeclareOption}
%   \\
% \hline
%   & \cs{DeclareDefaultOption}
%   & \cs{DeclareOption*}
%   \\
% \hline
%   & \cs{ProcessKeyvalOptions}
%   & \cs{ProcessOptions*}
%   \\
% \hline
%   & Option \xoption{patch}
%   &
%     \renewcommand*{\arraystretch}{1}^^A
%     \begin{tabular}[t]{@{}l@{}}
%       Class/package\\
%       option system
%     \end{tabular}
%   \\
% \hline
%   & \cs{SetupKeyvalOptions} &
%   \\
% \end{tabular}
% \end{center}
%
% \hypersetup{bookmarksopenlevel=3}
% \section{Usage}
%
% \subsection{Process options}
%
% \subsubsection{\cs{ProcessKeyvalOptions}}
%
% \begin{declcs}{ProcessKeyvalOptions}\,\M{family}\\
%   \cs{ProcessKeyvalOptions} \*
% \end{declcs}
% This command evaluates the global or local options of
% the package that are defined with \xpackage{keyval}'s
% interface within the family \meta{family}. It acts the
% same way as \LaTeX's \cs{ProcessOptions*}. In a package
% unknown global options are ignored, in a class they
% are added to the unknown option list. The known global
% options and all local options are passed to \xpackage{keyval}'s
% \cs{setkeys} command for executing the options.
% Unknown options are reported to the user by an error.
%
% If the family name happens to be the same as the name
% of the package or class where \cs{ProcessKeyvalOptions}
% is used or the family name has previously been setup by
% \cs{SetupKeyvalOptions}, then \cs{ProcessKeyvalOptions}
% knows the family name already and you can use the star form
% without mandatory argument.
%
% \subsubsection{\cs{ProcessLocalKeyvalOptions}}
%
% \begin{declcs}{ProcessLocalKeyvalOptions}\,\M{family}\\
%   \cs{ProcessLocalKeyvalOptions} \*
% \end{declcs}
% This macro has the same syntax and works similar as
% \cs{ProcessKeyvalOptions}. However it ignores
% global options and only processes the local package options.
% Therefore it only can be used inside a package.
% An error is thrown, if it is used inside a class.
%
% Neither of the following macros are necessary for
% \cs{ProcessKeyvalOptions{}}. They just help the package/class
% author in common tasks.
%
% \subsubsection{\cs{SetupKeyvalOptions}}\label{setup}
%
% \begin{declcs}{SetupKeyvalOptions}\,\texttt{\{}\\
%   \hspace{1.5em}|family|\,|=|\,\meta{family}|,|\\
%   \hspace{1.5em}|prefix|\,|=|\,\meta{prefix}\\
%   \hspace{1.5em}|setkeys|\,|=|\,\meta{setkeys command}\\
%   \texttt{\}}
% \end{declcs}
% This command allows to configure the default assumptions
% that are based on the current package or class name.
% \LaTeX\ remembers this name in \cs{@currname}. The syntax
% description of the default looks a little weird, therefor
% an example is given for a package or class named |foobar|.
% \begin{quote}
% \begin{tabular}{@{}l|ll|l@{}}
%   Key & Default & (example) & Used by\\
% \hline
%   \rule{0pt}{1.2\ht\csname @arstrutbox\endcsname}^^A
%   |family| & \meta{\cs{@currname}} & (|foobar|)
%       & \cs{ProcessKeyvalOptions*} \\
%     &&& \cs{DeclareBoolOption} \\
%     &&& \cs{DeclareStringOption} \\
% \hline
%   \rule{0pt}{1.2\ht\csname @arstrutbox\endcsname}^^A
%   |prefix| & \meta{\cs{@currname}}|@| & (|foobar@|)
%       & \cs{DeclareBoolOption} \\
%     &&& \cs{DeclareStringOption} \\
%     &&& \cs{DeclareVoidOption} \\
% \hline
%   \rule{0pt}{1.2\ht\csname @arstrutbox\endcsname}^^A
%   |setkeys| & \cs{setkeys} & (\cs{kvsetkeys})
%       & \cs{ProcessKeyvalOptions} \\
%     &&& \cs{ProcessLocalKeyvalOptions}
% \end{tabular}
% \end{quote}
% Key |setkeys| was added in version 3.9. The original
% \cs{setkeys} of package \xpackage{keyval} is not reentrant.
% If an option is processed by this \cs{setkeys}, then
% the option should not call \cs{setkeys} again with a
% different family. Otherwise the next options of the
% first \cs{setkeys} call are processed with the wrong
% family. With key |setkeys| the macro \cs{kvsetkeys}
% can be set that does not have the problem of the
% original \cs{setkeys} of package {keyval}.
%
% Probably \cs{setkeys} of package \xpackage{xkeyval}
% is safe in this respect. But I haven't made a full
% analysis. At least it does not have the problem
% of the original \cs{setkeys}.
%
% \subsection{Option declarations}
%
% The options for \cs{ProcessKeyvalOptions} are defined
% by \xpackage{keyval}'s \cs{define@key}.
% Common purposes of such keys are boolean switches, they
% enable or disable something. Or they store a name or some kind of
% string in a macro. The following commands help the user. He
% declares what he wants and \xpackage{kvoptions} take care of
% the key definition, resource allocation and initialization.
%
% In order to avoid name clashes of macro names, internal
% commands are prefixed. Both the prefix and the family
% name for the defined keys can be configured by
% \cs{SetupKeyvalOptions}.
%
% \subsubsection{\cs{DeclareStringOption}}
%
% \begin{declcs}{DeclareStringOption}\,\O{init}\,\M{key}\,\O{default}
% \end{declcs}
% A macro is created that remembers the value of the key \meta{key}.
% The name of the macro consists of the option name \meta{key} that
% is prefixed by the prefix (see \ref{setup}).
% The initial contents of the
% macro can be given by the first optional argument \meta{init}.
% The default is empty.
%
% The the option \meta{key} is defined. The option code
% just stores its value in the macro. If the optional argument
% at the end of \cs{DeclareStringOption} is given, then option
% \meta{key} is defined with the default \meta{default}.
%
% Example for a package with the following two lines:
% \begin{quote}
% |\ProvidesPackage{foobar}|\\
% |\DeclareStringOption[me]{name}|
% \end{quote}
% Then \cs{DeclareStringOption} defines the macro with content |me|,
% note \LaTeX\ complains if the name of the macro already exists:
% \begin{quote}
% |\newcommand*{\foobar@name}{me}|
% \end{quote}
% The option definition is similar to:
% \begin{quote}
% |\define@key{foobar}{name}{%|\\
% |  \renewcommand*{\foobar@name}{#1}%|\\
% |}|
% \end{quote}
%
% \subsubsection{\cs{DeclareBoolOption}}
%
% \begin{declcs}{DeclareBoolOption}\,\O{init}\,\M{key}
% \end{declcs}
% A boolean switch is generated, initialized by value \meta{init}
% and the corresponding key \meta{key} is defined. If the
% initialization value is not given, |false| is used as default.
%
% The internal actions of \cs{DeclareBoolOption} are shown below.
% The example is given for a package author who has the following
% two lines in his package/class:
% \begin{quote}
% |\ProvidesPackage{foobar}|\\
% |\DeclareBoolOption{verbose}|
% \end{quote}
% First a new switch is created:
% \begin{quote}
%   |\newif\iffoobar@verbose|
% \end{quote}
% and initialized:
% \begin{quote}
%   |\foobar@verbosefalse|
% \end{quote}
% Finally the key is defined:
% \begin{quote}
%   |\define@key{foobar}{verbose}[true]{|\dots|}|
% \end{quote}
% The option code configures the boolean option in the following way:
% If the author specifies |true| or |false| then the switch is
% turned on or off respectivly. Also the option can be given without
% explicit value. Then the switch is enabled.
% Other values are reported as errors.
%
% Now the switch is ready to use in the package/class, e.g.:
% \begin{quote}
%   |\iffoobar@verbose|\\
%   \qquad\textit{\% print verbose message}\\
%   |\else|\\
%   \qquad\textit{\% be quiet}\\
%   |\fi|
% \end{quote}
% Users of package \cs{ifthen} can use the switch as
% boolean:
% \begin{quote}
%   |\boolean{foobar@verbose}|
% \end{quote}
%
% \subsubsection{\cs{DeclareComplementaryOption}}
%
% \begin{declcs}{DeclareComplementaryOption}\,\M{key}\,\M{parent}
% \end{declcs}
% Sometimes contrasting names are used to characterize the
% two states of a boolean switch, for example \xoption{draft}
% vs.\@ \xoption{final}. Both options behave like boolean
% options but they do not need two different switches, they
% should share one. \cs{DeclareComplementaryOption} allows this.
% The option \meta{key} shares the switch of option \meta{parent}.
% Example:
% \begin{quote}
%   |\DeclareBoolOption{draft}|\\
%   |\DeclareComplementaryOption{final}{draft}|
% \end{quote}
% Then |final| sets the switch of \xoption{draft} to |false|, and
% |final=false| enables the \xoption{draft} switch.
%
% \subsubsection{\cs{DeclareVoidOption}}
%
% \begin{declcs}{DeclareVoidOption}\,\M{key}\,\M{code}
% \end{declcs}
% \cs{ProcessKeyvalOptions} can be extended to recognize
% options that are declared in traditional way by \cs{DeclareOption}.
% But in case of the error that the user specifies a value, then
% this option would not recognized as key value option because
% of \cs{DeclareOption} and not detected as traditional option
% because of the value part. The user would get an unknown option
% error, difficult to understand.
%
% \cs{DeclareVoidOption} solves
% this problem. It defines the option \meta{key} as key value option.
% If the user specifies a value, a warning is given and the
% value is ignored.
%
% The code part \meta{code} is stored in a macro.
% The name of the macro consists of the option name \meta{key} that
% is prefixed by the prefix (see \ref{setup}).
% If the option is set, the macro will be executed. During the
% execution \cs{CurrentOption} is available with the current key name.
%
% \subsubsection{\cs{DeclareDefaultOption}}
%
% \begin{declcs}{DeclareDefaultOption}\,\M{code}
% \end{declcs}
% \begin{sloppypar}
% This command does not define a specific key, it is the equivalent
% to \LaTeX's \cs{DeclareOption*}. It allows the specification
% of a default action \meta{code} that is invoked if an unknown option
% is found.
% While \meta{code} is called,
% macro \cs{CurrentOption} contains the current option string.
% In addition \cs{CurrentOptionValue} contains the value part
% if the option string is parsable as key value pair, otherwise
% it is \cs{relax}. \cs{CurrentOptionKey} contains the key of
% the key value pair, or the whole option string, if it misses
% the equal sign.
% \end{sloppypar}
%
% Inside packages typical default actions are to pass unknown
% options to another package. Or an error message can be thrown
% by \cs{@unknownoptionerror}. This is the original error
% message that \LaTeX\ gives for unkown package options.
% This error message is easier to understand for the user
% as the error message from package \xpackage{keyval} that
% is given otherwise.
%
% A Class ignores unknown options and puts them on the
% unused option list. Let \LaTeX\ do the job and just
% call \cs{OptionNotUsed}. Or the options can be passed to
% another class that is later loaded.
%
% \subsubsection{Local options}
%
% \begin{declcs}{DeclareLocalOption} \M{option}\\
%   \cs{DeclareLocalOptions} \M{option list}
% \end{declcs}
% Both macros mark package options as local options. That means
% that they are ignored by \cs{ProcessKeyvalOptions} if they are given
% as global options. \cs{DeclareLocalOptions} takes one option,
% \cs{DeclareLocalOptions} expects a comma separated list of options.
%
% \subsubsection{Dynamic options}
%
% Options of \LaTeX's package/class system are cleared
% in \cs{ProcessOptions}. They modify the static model
% of a package. For example, depending on option \xoption{bookmarks}
% package \xpackage{hyperref} loads differently.
%
% Options, however,
% defined by \xpackage{keyval}'s \cs{define@key} remain
% defined, if the options are processed by \cs{setkeys}.
% Therefore these options can also be used to model
% the dynamic behaviour of a package. For example, in
% \xpackage{hyperref} the link colors can be changed
% everywhere until the end in |\end{document}|.
%
% However package \xpackage{color} that adds color support
% is necessary and it
% cannot be loaded after |\begin{document}|. Option \xoption{colorlinks}
% that loads \xpackage{color} should be active until |\begin{document}|
% and die in some way if it is too late for loading packages.
% With \cs{DisableKeyvalOption} the package/class author can
% specify and configure the death of an option and controls
% the life period of the option.
%
% \subsubsection{\cs{DisableKeyvalOption}}
%
% \begin{declcs}{DisableKeyvalOption}\,\O{options}\,\M{family}\,\M{key}
%   \\[0.5ex]
%   \meta{options}:\\
%   \hspace{1.5em}^^A
%   \begin{tabular}[t]{@{}l@{ \texttt{=} }l@{\qquad}>{default: }l@{}}
%     |action| & |undef|, |warning|, |error|, or |ignore|
%     & |undef|
%     \\
%     \multicolumn{1}{@{}l}{|global| or |local|} & & |global|
%     \\
%     |package| or |class| & \meta{name}
%     \\
%   \end{tabular}^^A
% \end{declcs}
% \cs{DisableKeyvalOption} can be called to mark the end when
% the option \meta{key} is no longer useful. The behaviour of
% an option after its death can be configured by \xoption{action}:
% \begin{description}
% \item[\texttt{undef}:] The option will be undefined,
%   If it is called, \cs{setkeys} reports an error because
%   of unknown key.
% \item[\texttt{error} or \texttt{warning}:] Use of the
%  option will cause an error or warning message.
%  Also these actions require that exclusivly either the
%  package or class name is given in options \xoption{package}
%  or \xoption{class}.
% \item[\texttt{ignore}:] The use of the option will silently
%  be ignored.
% \end{description}
% The option's death can be limited to the end of the current
% group, if option \xoption{local} is given. Default is
% \xoption{global}.
%
% The package/class author can wish the end of the option
% already during the package loading, then he will have
% static behaviour. In case of dynamic options \cs{DisableKeyvalOption}
% can be executed everywhere, also outside the package. Therefore
% the family name and the package/class name is usually unknown for
% \cs{DisableKeyvalOption}. Therefore the argument for the family
% name is mandatory and for some actions the package/class name
% must be provided.
%
% Usually a macro would configure the option death, Example:
%\begin{quote}
%\begin{verbatim}
%\ProvidesPackage{foobar}
%\DeclareBoolOption{color}
%\DeclareStringOption[red]{emphcolor}
%\ProcessKeyvalOptions*
%
%\newcommand*{\foobar@DisableOption}[2]{%
%  \DisableKeyvalueOption[
%    action={#1},
%    package=foobar
%  ]{foobar}{#2}%
%}
%
%\iffoobar@color
%  \RequirePackage{color}
%  \renewcommand*{\emph}[1]{\textcolor{\foobar@emphcolor}{#1}}
%\else
%  % Option emphcolor is not wrong, if we have color support.
%  % otherwise the option has no effect, but we don't want to
%  % remove it. Therefore action 'ignore' is the best choice:
%  \foobar@DisableOption{ignore}{emphcolor}
%\fi
%% No we don't need the option 'color'.
%\foobar@DisableOption{warning}{color}
%
%% With color support option 'emphcolor' will dynamically
%% change the color of \emph statements.
%\end{verbatim}
%\end{quote}
%
% \subsubsection{\cs{AddToKeyvalOption}}
%
% \begin{declcs}{AddToKeyvalOption}\,\M{family}\,\M{key}\,\M{code}\\
%   \cs{AddToKeyvalOption} \*\,\M{key}\,\M{code}
% \end{declcs}
% The code for an existing key \meta{key} of family \meta{family} is
% extended by code \meta{code}. In the starred form the current
% family setting is used, see \cs{ProcessKeyvalOptions*}.
%
% \subsection{Global vs. local options}
%
% Options that are given for \cs{documentclass} are called
% global options. They are known to the class and all packages.
% A package may make use of a global option and marks it as
% used. The advantage for the user is the freedom to specify
% options both in the \cs{documentclass} or \cs{usepackage}
% commands.
%
% However global options are shared with the class options
% and options of all other packages. Thus there can be the
% same option with different semantics for different
% packages and classes. As example, package \xpackage{bookmark}
% knows option \xoption{open} that specifies whether the bookmarks
% are opened or closed initially. It's values are |true| or |false|.
% Since KOMA-Script version 3.00 the KOMA classes also introduces
% option \xoption{open} with values |right| and |any| and a
% complete different meaning.
%
% Such conflicts can be resolved by marking all or part of options
% as local by \cs{DeclareLocalOption} or \cs{DeclareLocalOptions}.
% Then the packages ignores global occurences of these options.
% Package \xpackage{kvoptions} provides two methods:
% \begin{itemize}
% \item \cs{ProcessLocalKeyvalOptions} automatically uses all options
%   as local options. It ignores all global options.
% \item \cs{DeclareLocalOption} or \cs{DeclareLocalOptions} marks
%   options as local options. \cs{ProcessKeyvalOptions} will then
%   ignore global occurences for these local options.
% \end{itemize}
% Since version 1.5 package \xpackage{bookmark} uses the latter
% method. It checks global and local option places for driver options
% and limits all other options as local options. Thus the class
% option \xoption{open} of KOMA-Script is not misread as
% option for package \xpackage{bookmark}.
%
% \subsection{Summary of internal macros}
%
% The \cs{Declare}\texttt{\dots Option} commands define
% macros, additionally to the macros generated by the key definition.
% These macros can be used by the package/class author.
% The name of the macros starts with the prefix \meta{prefix}
% that can be configured by \cs{SetupKeyvalOptions}.
% \begin{center}
% \def\prefix{\cs{}\meta{prefix}\meta{key}}^^A
% \def\Hline{^^A
%   \hline
%   \rule{0pt}{1.2\ht\csname @arstrutbox\endcsname}^^A
% }^^A
% \begin{tabular}{@{}l|>{\ttfamily}l|l@{}}
%   Declare \meta{key} & \textrm{Defined macro} & Description\\
% \Hline
%   \cs{DeclareStringOption}
%    & \prefix & holds the string\\
% \Hline
%   \cs{DeclareBoolOption}
%     & \cs{if}\meta{prefix}\meta{key}& boolean switch\\
%   &\prefix false& disable switch\\
%   &\prefix true& enable switch\\
% \Hline
%   \cs{DeclareComplementaryOption}
%     & \prefix false& enable parent switch\\
%     & \prefix true& disable parent switch\\
% \Hline
%   \cs{DeclareVoidOption}
%   & \prefix & holds the action
% \end{tabular}
% \end{center}
%
% \subsection{\plainTeX}
%
% Package \xpackage{keyval} is also usable in \plainTeX\ with
% the help of file \xfile{miniltx.tex}. Some features of this
% package \xpackage{kvoptions} might also be useful for \plainTeX.
% If \LaTeX\ is not found, \cs{ProcessKeyvalOptions} and option
% \xoption{patch} are disabled. Before using the option declaration
% commands \cs{Declare...Option}, \cs{SetupKeyvalOptions} must be
% used.
%
% \hypersetup{bookmarksopenlevel=1}
%
% \section{Example}
%
% The following example defined a package that serves some
% private color management. A boolean option \xoption{print} enables
% print mode without colors. An option \xoption{emph} redefines
% \cs{emph} to print in the given color. And the driver
% can be specified by option \xoption{driver}.
%
%    \begin{macrocode}
%<*example>
    % Package identification
    % -----------------------
\NeedsTeXFormat{LaTeX2e}
\ProvidesPackage{example-mycolorsetup}[2016/05/16 Managing my colors]

\RequirePackage{ifpdf}
\RequirePackage{kvoptions}

    % Option declarations
    % -------------------

\SetupKeyvalOptions{
  family=MCS,
  prefix=MCS@
}
    % Use a shorter family name and prefix

    % Option print
\DeclareBoolOption{print}
    % is the same as
    % \DeclareBoolOption[false]{print}

    % Option driver
\ifpdf
  \DeclareStringOption[pdftex]{driver}
\else
  \DeclareStringOption[dvips]{driver}
\fi

    % Alternative interface for driver options
\DeclareVoidOption{dvips}{\SetupDriver}
\DeclareVoidOption{dvipdfm}{\SetupDriver}
\DeclareVoidOption{pdftex}{\SetupDriver}
    % In \SetupDriver we take the current option \CurrentOption
    % and pass it to the driver option.
    % The \expandafter commands expand \CurrentOption at the
    % time, when \SetupDriver is executed and \CurrentOption
    % has the correct meaning.
\newcommand*{\SetupDriver}{%
  \expandafter\@SetupDriver\expandafter{\CurrentOption}%
}
\newcommand*{\@SetupDriver}[1]{%
  \setkeys{MCS}{driver={#1}}%
}

    % Option emph
    % An empty value means, we want to have no color for \emph.
    % If the user specifies option emph without value, the red is used.
\DeclareStringOption{emph}[red]
    % is the same as
    % \DeclareStringOption[]{emph}[red]

    % Default option rule
\DeclareDefaultOption{%
  \ifx\CurrentOptionValue\relax
    \PackageWarningNoLine{\@currname}{%
      Unknown option `\CurrentOption'\MessageBreak
      is passed to package `color'%
    }%
    % Pass the option to package color.
    % Again it is better to expand \CurrentOption.
    \expandafter\PassOptionsToPackage
    \expandafter{\CurrentOption}{color}%
  \else
    % Package color does not take options with values.
    % We provide the standard LaTeX error.
    \@unknownoptionerror
  \fi
}

    % Process options
    % ---------------
\ProcessKeyvalOptions*

    % Implementation depending on option values
    % -----------------------------------------
    % Code for print mode
\ifMCS@print
  \PassOptionsToPackage{monochrome}{color}
    % tells package color to use black and white
\fi

\RequirePackage[\MCS@driver]{color}
    % load package color with the correct driver

    % \emph setup
\ifx\MCS@emph\@empty
    % \@empty is a predefined macro with empty contents.
    % the option value of option emph is empty, thus
    % we do not want a redefinition of \emph.
\else
  \renewcommand*{\emph}[1]{%
    \textcolor{\MCS@emph}{#1}%
  }
\fi
%</example>
%    \end{macrocode}
%
% \section{Package options}
%
% The package \xpackage{kvoptions} knows two package options
% \xoption{patch} and \xoption{debugshow}.
% The options of package \xpackage{kvoptions} are intended
% for authors, not for package/class writers. Inside a package
% it is too late for option \xoption{patch} and \xoption{debugshow}
% enables some messages that are perhaps useful for the debugging
% phase. Also \LaTeX\ is unhappy if a package is loaded
% later again with options that are previously not given.
% Thus package and class authors, stay with |\RequirePackage{kvoptions}|
% without options.
%
% Option \xoption{patch} loads package \xpackage{kvoptions-patch}.
%
% \subsection{Package \xpackage{kvoptions-patch}}
%
% \LaTeX's system of package/class options has some severe limitations
% that especially affects the value part if options are used as
% pair of key and value.
% \begin{itemize}
% \item Spaces are removed, regardless where:
%   \begin{quote}
%     |\documentclass[box=0 0 400 600]{article}|
%   \end{quote}
%   Now each package will see |box=00400600| as global option.
% \item In the previous case also braces would not help:
%   \begin{quote}
%     |\documentclass[box={0 0 400 600}]{article}|
%   \end{quote}
%   The result is an error message:
%   \begin{quote}
%     |! LaTeX Error: Missing \begin{document}.|
%   \end{quote}
%   As local option, however, it works if the package
%   knows about key value options (By using this package, for example).
% \item The requirements on robustness are extremly high.
%   \LaTeX\ expands the option. All that will not work as environment
%   name will break also as option. Even a \cs{relax} will generate
%   an error message:
%   \begin{quote}
%     |! Missing \endcsname inserted.|
%   \end{quote}
%   Of course, \LaTeX\ does not use its protecting mechanisms.
%   On contrary \cs{protect} itself will cause errors.
% \item The options are expanded. But perhaps the package will
%   do that, because it has to setup some things before?
%   Example \xpackage{hyperref}:
%   \begin{quote}
%     |\usepackage[pdfauthor=M\"uller]{hyperref}|
%   \end{quote}
%   Package \xpackage{hyperref} does not see |M\"uller| but
%   its expansion and it does not like it, you get many warnings
%   \begin{quote}
%     |Token not allowed in a PDFDocEncoded string|
%   \end{quote}
%   And the title becomes: |Mu127uller|.
%   Therefore such options must usually be given after package
%   \xpackage{hyperref}
%   is loaded:
%   \begin{quote}
%     |\usepackage{hyperref}|\\
%     |\hypersetup{pdfauthor=Fran\c coise M\"uller}|
%   \end{quote}
%   As package option it will even break with |Fran\c coise|
%   because of the cedilla |\c c|, it is not robust enough.
% \end{itemize}
% For users that do not want with this limitations the package
% offers package \xoption{kvoptions-patch}. It patches \LaTeX's option system
% and tries to teach it also to handle options that are given
% as pairs of key and value and to prevent expansion.
% It can already be used at the very beginning, before \cs{documentclass}:
% \begin{quote}
%   |\RequirePackage{kvoptions-patch}|\\
%   |\documentclass[pdfauthor=Fran\c coise M\"uller]{article}|\\
%   |\usepackage{hyperref}|
% \end{quote}
% The latest time is before the package where you want to use
% problematic values:
% \begin{quote}
%   |\usepackage{kvoptions-patch}|\\
%   |\usepackage[Fran\c coise M\"uller]{hyperref}|
% \end{quote}
% Some remarks:
% \begin{itemize}
% \item The patch requires \eTeX, its \cs{unexpanded} feature
%   is much too nice. It is possible to work around using token
%   registers. But the code becomes longer, slower, more difficult
%   to read and maintain. The package without option \xoption{patch}
%   works and will work without \eTeX.
% \item The code for the patch is quite long, there are many test
%   cases. Thus the probability for bugs is probably not too small.
% \item Since 2008/10/18 v3.0 package \xpackage{kvoptions-patch} is
%   available. Before option \xoption{patch} of package \xpackage{kvoptions}
%   must be used instead. I think, the solution as standalone package
%   \xpackage{kvoptions-patch} is cleaner and avoids option clashes.
% \end{itemize}
%
% \subsection{Option \xoption{debugshow}}
%
% The name of this option follows the convention of packages
% \xpackage{multicol}, \xpackage{tabularx}, and \xpackage{tracefnt}.
% Currently it prints the setting of boolean options, declared
% by \cs{DeclareBoolOption} in the \xfile{.log} file, if that
% boolean option is used.
% You can activate the option by
% \begin{itemize}
% \item |\PassOptionsToPackage{debugshow}{kvoptions}|\\
%       Put this somewhere before package \xpackage{kvoptions} is
%       loaded first, e.g. before \cs{documentclass}.
% \item |\RequirePackage[debugshow]{kvoptions}|\\
%       Before \cs{documentclass} even an author has to use
%       \cs{RequirePackage}. \cs{usepackage} only works after
%       \cs{documentclass}.
% \end{itemize}
% The preferred method is \cs{PassOptionsToPackage}, because
% it does not force the package loading and does not disturb,
% if the package is not loaded later at all.
%
% \section{Limitations}
%
% \subsection{Compatibility}
% \subsubsection{Package \xpackage{kvoptions-patch} vs. package \xpackage{xkvltxp}}
%
% Package \xpackage{xkvltxp} from the \xpackage{xkeyval} project
% has the same goal as package \xpackage{kvoptions-patch}
% and to patch \LaTeX's kernel commands in order to get better support
% for key value options.
% Of course they cannot be used both. The user must decide, which
% method he prefers. Package \xpackage{kvoptions-patch} aborts itself,
% if it detects that \xpackage{xkvltxp} is
% already loaded.
%
% However package \xpackage{xkvltxp} and \xpackage{kvoptions}
% can be used together, example:
% \begin{quote}
% |\usepackage{xkvltxp}|\\
% |\usepackage[...]{foobar} % |\texttt{\itshape foobar using kvoptions}
% \end{quote}
% The other way should work, too.
%
% Package \xpackage{kvoptions-patch} tries to catch more situations
% and to be more robust.
% For example, during the comparison of options it normalizes
% them by removing spaces around |=| and the value. Thus the
% following is not reported as option clash:
% \begin{quote}
%\begin{verbatim}
%\RequirePackage{kvoptions-patch}
%\documentclass{article}
%
%\usepackage[scaled=0.7]{helvet}
%\usepackage[scaled = 0.7]{helvet}
%
%\begin{document}
%\end{document}
%\end{verbatim}
% \end{quote}
%
% \subsection{Limitations}
%
% \subsubsection{Option comparisons}
%
% In some situations \LaTeX\ compares option lists, e.g. option clash check,
% \cs{@ifpackagewith}, or \cs{@ifclasswith}. Apart from catcode and
% sanitizing problems of option \xoption{patch}, there is another problem.
% \LaTeX\ does not know about the type and default values of
% options in key value style. Thus an option clash is reported,
% even if the key value has the same meaning:
% \begin{quote}
% |\usepackage[scaled]{helvet} %| \texttt{\textit{default is }.95}\\
% |\usepackage[.95]{helvet}|\\
% |\usepackage[0.95]{helvet}|
% \end{quote}
%
% \subsubsection{Option list parsing with package \xpackage{kvoptions-patch}}
%
% With package \xpackage{kvoptions-patch} the range of possible values
% in key value specifications is much large, for example
% the comma can be used, if enclosed in curly braces.
%
% Other packages, especially the packages that uses their
% own process option code can be surprised to find tokens
% inside options that they do not expect and errors would
% be the consequence. To avoid errors the options, especially
% the unused option list is sanitized. That means the list
% will only contain tokens with catcode 12 (other) and
% perhaps spaces (catcode 10). This allows a safe parsing
% for other packages. But a comma in the value part is
% no longer protected by curly braces because they have lost
% their special meaning. This is the price for compatibility.
%
% Example:
% \begin{quote}
% |\RequirePackage{kvoptions-patch}|\\
% |\documentclass[a={a,b,c},b]{article}|\\
% |\begin{document}|\\
% |\end{document}|
% \end{quote}
% Result:
% \begin{quote}
% |LaTeX Warning: Unused global option(s):|\\
% |    [a={a,c},b].|
% \end{quote}
%
% \StopEventually{
% }
%
% \section{Implementation}
%
% \subsection{Preamble}
%
%    \begin{macrocode}
%<*package>
%    \end{macrocode}
%
% \paragraph{Reload check and  identification.}
%    Reload check, especially if the package is not used with \LaTeX.
%    \begin{macrocode}
\begingroup\catcode61\catcode48\catcode32=10\relax%
  \catcode13=5 % ^^M
  \endlinechar=13 %
  \catcode35=6 % #
  \catcode39=12 % '
  \catcode44=12 % ,
  \catcode45=12 % -
  \catcode46=12 % .
  \catcode58=12 % :
  \catcode64=11 % @
  \catcode123=1 % {
  \catcode125=2 % }
  \expandafter\let\expandafter\x\csname ver@kvoptions.sty\endcsname
  \ifx\x\relax % plain-TeX, first loading
  \else
    \def\empty{}%
    \ifx\x\empty % LaTeX, first loading,
      % variable is initialized, but \ProvidesPackage not yet seen
    \else
      \expandafter\ifx\csname PackageInfo\endcsname\relax
        \def\x#1#2{%
          \immediate\write-1{Package #1 Info: #2.}%
        }%
      \else
        \def\x#1#2{\PackageInfo{#1}{#2, stopped}}%
      \fi
      \x{kvoptions}{The package is already loaded}%
      \aftergroup\endinput
    \fi
  \fi
\endgroup%
%    \end{macrocode}
%    Package identification:
%    \begin{macrocode}
\begingroup\catcode61\catcode48\catcode32=10\relax%
  \catcode13=5 % ^^M
  \endlinechar=13 %
  \catcode35=6 % #
  \catcode39=12 % '
  \catcode40=12 % (
  \catcode41=12 % )
  \catcode44=12 % ,
  \catcode45=12 % -
  \catcode46=12 % .
  \catcode47=12 % /
  \catcode58=12 % :
  \catcode64=11 % @
  \catcode91=12 % [
  \catcode93=12 % ]
  \catcode123=1 % {
  \catcode125=2 % }
  \expandafter\ifx\csname ProvidesPackage\endcsname\relax
    \def\x#1#2#3[#4]{\endgroup
      \immediate\write-1{Package: #3 #4}%
      \xdef#1{#4}%
    }%
  \else
    \def\x#1#2[#3]{\endgroup
      #2[{#3}]%
      \ifx#1\@undefined
        \xdef#1{#3}%
      \fi
      \ifx#1\relax
        \xdef#1{#3}%
      \fi
    }%
  \fi
\expandafter\x\csname ver@kvoptions.sty\endcsname
\ProvidesPackage{kvoptions}%
  [2016/05/16 v3.12 Key value format for package options (HO)]%
%    \end{macrocode}
%
% \paragraph{Catcodes}
%
%    \begin{macrocode}
\begingroup\catcode61\catcode48\catcode32=10\relax%
  \catcode13=5 % ^^M
  \endlinechar=13 %
  \catcode123=1 % {
  \catcode125=2 % }
  \catcode64=11 % @
  \def\x{\endgroup
    \expandafter\edef\csname KVO@AtEnd\endcsname{%
      \endlinechar=\the\endlinechar\relax
      \catcode13=\the\catcode13\relax
      \catcode32=\the\catcode32\relax
      \catcode35=\the\catcode35\relax
      \catcode61=\the\catcode61\relax
      \catcode64=\the\catcode64\relax
      \catcode123=\the\catcode123\relax
      \catcode125=\the\catcode125\relax
    }%
  }%
\x\catcode61\catcode48\catcode32=10\relax%
\catcode13=5 % ^^M
\endlinechar=13 %
\catcode35=6 % #
\catcode64=11 % @
\catcode123=1 % {
\catcode125=2 % }
\def\TMP@EnsureCode#1#2{%
  \edef\KVO@AtEnd{%
    \KVO@AtEnd
    \catcode#1=\the\catcode#1\relax
  }%
  \catcode#1=#2\relax
}
\TMP@EnsureCode{1}{14}% ^^A (comment)
\TMP@EnsureCode{2}{14}% ^^A (comment)
\TMP@EnsureCode{33}{12}% !
\TMP@EnsureCode{39}{12}% '
\TMP@EnsureCode{40}{12}% (
\TMP@EnsureCode{41}{12}% )
\TMP@EnsureCode{42}{12}% *
\TMP@EnsureCode{44}{12}% ,
\TMP@EnsureCode{45}{12}% -
\TMP@EnsureCode{46}{12}% .
\TMP@EnsureCode{47}{12}% /
\TMP@EnsureCode{58}{12}% :
\TMP@EnsureCode{62}{12}% >
\TMP@EnsureCode{91}{12}% [
\TMP@EnsureCode{93}{12}% ]
\TMP@EnsureCode{94}{7}% ^ (superscript)
\TMP@EnsureCode{96}{12}% `
\edef\KVO@AtEnd{\KVO@AtEnd\noexpand\endinput}
%    \end{macrocode}
%
% \paragraph{External resources.}
%    The package extends the support for key value pairs of
%    package \cs{keyval} to package options. Thus the package
%    needs to be loaded anyway. and we use it for
%    \cs{SetupKeyvalOptions}. AFAIK this does not disturb
%    users of \xpackage{xkeyval}.
%    \begin{macrocode}
\@ifundefined{define@key}{%
  \RequirePackage{keyval}\relax
}{}
%    \end{macrocode}
%
%    Macro \cs{DeclareLocalOptions} parses a comma separated key list
%    and uses \cs{comma@parse} of package \xpackage{kvsetkeys},
%    version 1.3.
%    \begin{macrocode}
\RequirePackage{ltxcmds}[2010/12/02]
\RequirePackage{kvsetkeys}[2007/09/29]
%    \end{macrocode}
%
% \paragraph{Provide macros for \plainTeX.}
%    \begin{macrocode}
\@ifundefined{@x@protect}{%
  \def\@x@protect#1\fi#2#3{%
    \fi\protect#1%
  }%
  \let\@typeset@protect\relax
}{}
\@ifundefined{@currname}{%
  \def\@currname{}%
}{}
\@ifundefined{@currext}{%
  \def\@currext{}%
}{}
%    \end{macrocode}
%
% \paragraph{Options}
%    Option \xoption{debugshow} enables additional lines of
%    code that prints information into the \xfile{.log}
%    file.
%    \begin{macrocode}
\DeclareOption{debugshow}{\catcode\@ne=9 }
%    \end{macrocode}
%
%    \begin{macrocode}
\DeclareOption{patch}{%
  \AtEndOfPackage{%
    \RequirePackage{kvoptions-patch}[2016/05/16]%
  }%
}
%    \end{macrocode}
%
%    Optionen auswerten:
%    \begin{macrocode}
\ProcessOptions\relax
%    \end{macrocode}
%
% \subsection{Option declaration macros}
%
% \subsubsection{\cs{SetupKeyvalOptions}}
%
%    The family for the key value pairs can be setup once
%    and is remembered later.
%    The package name seems a reasonable default for the
%    family key, if it is not set by the package author.
%
%    \begin{macro}{\KVO@family}
%    We cannot store the family setting in one macro,
%    because the package should be usable for many
%    other packages, too. Thus we remember the family
%    setting in a macro, whose name contains the
%    package name with extension, a key in \LaTeX's
%    class/package system.
%    \begin{macrocode}
\define@key{KVO}{family}{%
  \expandafter\edef\csname KVO@family@%
      \@currname.\@currext\endcsname{#1}%
}
\def\KVO@family{%
  \@ifundefined{KVO@family@\@currname.\@currext}{%
    \@currname
  }{%
    \csname KVO@family@\@currname.\@currext\endcsname
  }%
}
%    \end{macrocode}
%    \end{macro}
%
%    \begin{macro}{\KVO@prefix}
%    The value settings of options that are declared by \cs{DeclareBoolOption}
%    and \cs{DeclareStringOption} need to be saved in macros.
%    in the first case this is a switch \cs{if}\meta{prefix}\meta{key},
%    in the latter case a macro \cs{}\meta{prefix}\meta{key}.
%    The prefix can be configured, by \xoption{prefix} that is declared
%    here. The default is the package name with |@| appended.
%    \begin{macrocode}
\define@key{KVO}{prefix}{%
  \expandafter\edef\csname KVO@prefix@%
      \@currname.\@currext\endcsname{#1}%
}
\def\KVO@prefix{%
  \ltx@ifundefined{KVO@prefix@\@currname.\@currext}{%
    \@currname @%
  }{%
    \csname KVO@prefix@\@currname.\@currext\endcsname
  }%
}
%    \end{macrocode}
%    \end{macro}
%
%    \begin{macrocode}
\define@key{KVO}{setkeys}{%
  \expandafter\def\csname KVO@setkeys@%
      \@currname.\@currext\endcsname{#1}%
}
%    \end{macrocode}
%    \begin{macro}{\KVO@setkeys}
%    \begin{macrocode}
\def\KVO@setkeys{%
  \ltx@IfUndefined{KVO@setkeys@\@currname.\@currext}{%
    \setkeys
  }{%
    \csname KVO@setkeys@\@currname.\@currext\endcsname
  }%
}
%    \end{macrocode}
%    \end{macro}
%
%    \begin{macro}{\SetupKeyvalOptions}
%    The argument of \cs{SetupKeyvalOptions} expects a key value
%    list, known keys are \xoption{family} and \xoption{prefix}.
%    \begin{macrocode}
\newcommand*{\SetupKeyvalOptions}{%
  \kvsetkeys{KVO}%
}
%    \end{macrocode}
%    \end{macro}
%
% \subsubsection{\cs{DeclareBoolOption}}
%
%    \begin{macro}{\DeclareBoolOption}
%    Usually options of boolean type can be given by the
%    user without value and this means a setting to \emph{true}.
%    We follow this convention here. Also it simplifies the
%    user interface.
%
%    The switch is created and initialized with \emph{false}.
%    The default setting can be overwritten by the optional
%    argument.
%
%    \LaTeX's \cs{newif} does not check for already defined
%    macros, therefore we add this check here to prevent
%    the user from accidently redefining of \TeX's primitives
%    and other macros.
%    \begin{macrocode}
\newcommand*{\DeclareBoolOption}[2][false]{%
  \KVO@ifdefinable{if\KVO@prefix#2}{%
    \KVO@ifdefinable{\KVO@prefix#2true}{%
      \KVO@ifdefinable{\KVO@prefix#2false}{%
        \csname newif\expandafter\endcsname
        \csname if\KVO@prefix#2\endcsname
        \@ifundefined{\KVO@prefix#2#1}{%
          \PackageWarning{kvoptions}{%
            Initialization of option `#2' failed,\MessageBreak
            cannot set boolean option to `#1',\MessageBreak
            use `true' or `false', now using `false'%
          }%
        }{%
          \csname\KVO@prefix#2#1\endcsname
        }%
        \begingroup
          \edef\x{\endgroup
            \noexpand\define@key{\KVO@family}{#2}[true]{%
              \noexpand\KVO@boolkey{\@currname}%
              \ifx\@currext\@clsextension
                \noexpand\@clsextension
              \else
                \noexpand\@pkgextension
              \fi
              {\KVO@prefix}{#2}{####1}%
            }%
          }%
        \x
      }%
    }%
  }%
}
%    \end{macrocode}
%    \end{macro}
%
%    \begin{macro}{\DeclareComplementaryOption}
%    The first argument is the key name, the second the
%    key that must be a boolean option with the same
%    current family and prefix. A new switch is not
%    created for the new key, we have already a switch.
%    Instead we define switch setting commands to work
%    on the parent switch.
%    \begin{macrocode}
\newcommand*{\DeclareComplementaryOption}[2]{%
  \@ifundefined{if\KVO@prefix#2}{%
    \PackageError{kvoptions}{%
      Cannot generate option code for `#1',\MessageBreak
      parent switch `#2' does not exist%
    }{%
      You are inside %
      \ifx\@currext\@clsextension class\else package\fi\space
      `\@currname.\@currext'.\MessageBreak
      `\KVO@family' is used as familiy %
      for the keyval options.\MessageBreak
      `\KVO@prefix' serves as prefix %
      for internal switch macros.\MessageBreak
      \MessageBreak
      \@ehc
    }%
  }{%
    \KVO@ifdefinable{\KVO@prefix#1true}{%
      \KVO@ifdefinable{\KVO@prefix#1false}{%
        \expandafter\let\csname\KVO@prefix#1false\expandafter\endcsname
          \csname\KVO@prefix#2true\endcsname
        \expandafter\let\csname\KVO@prefix#1true\expandafter\endcsname
          \csname\KVO@prefix#2false\endcsname
%    \end{macrocode}
%    The same code part as in \cs{DeclareBoolOption} can
%    now be used.
%    \begin{macrocode}
        \begingroup
          \edef\x{\endgroup
            \noexpand\define@key{\KVO@family}{#1}[true]{%
              \noexpand\KVO@boolkey{\@currname}%
              \ifx\@currext\@clsextension
                \noexpand\@clsextension
              \else
                \noexpand\@pkgextension
              \fi
              {\KVO@prefix}{#1}{####1}%
            }%
          }%
        \x
      }%
    }%
  }%
}
%    \end{macrocode}
%    \end{macro}
%
%    \begin{macro}{\KVO@ifdefinable}
%    Generate the command token LaTeX's \cs{@ifdefinable} expects.
%    \begin{macrocode}
\def\KVO@ifdefinable#1{%
  \expandafter\@ifdefinable\csname #1\endcsname
}
%    \end{macrocode}
%    \end{macro}
%
%    \begin{macro}{\KVO@boolkey}
%    We check explicitly for |true| and |false| to prevent
%    the user from accidently calling other macros.
%    \begin{quote}
%    \begin{tabular}{@{}ll@{}}
%    |#1|& package/class name\\
%    |#2|& |\@pkgextension|/|\@clsextension|\\
%    |#3|& prefix \\
%    |#4|& key name\\
%    |#5|& new value
%    \end{tabular}
%    \end{quote}
%    \begin{macrocode}
\def\KVO@boolkey#1#2#3#4#5{%
  \edef\KVO@param{#5}%
  \ltx@onelevel@sanitize\KVO@param
  \ifx\KVO@param\KVO@true
    \expandafter\@firstofone
  \else
    \ifx\KVO@param\KVO@false
      \expandafter\expandafter\expandafter\@firstofone
    \else
      \ifx#2\@clsextension
        \expandafter\ClassWarning
      \else
        \expandafter\PackageWarning
      \fi
      {#1}{%
        Value `\KVO@param' is not supported by\MessageBreak
        option `#4'%
      }%
      \expandafter\expandafter\expandafter\@gobble
    \fi
  \fi
  {%
    ^^A\ifx#2\@clsextension
    ^^A  \expandafter\ClassInfo
    ^^A\else
    ^^A  \expandafter\PackageInfo
    ^^A\fi
    ^^A{#1}{[option] #4=\KVO@param}%
    \csname#3#4\KVO@param\endcsname
  }%
}
%    \end{macrocode}
%    \end{macro}
%    \begin{macro}{\KVO@true}
%    \begin{macro}{\KVO@false}
%    The macros \cs{KVO@true} and \cs{KVO@false} are used
%    for string comparisons. After \cs{ltx@onelevel@sanitize}
%    we have only tokens with catcode 12 (other).
%    \begin{macrocode}
\def\KVO@true{true}
\def\KVO@false{false}
\ltx@onelevel@sanitize\KVO@true
\ltx@onelevel@sanitize\KVO@false
%    \end{macrocode}
%    \end{macro}
%    \end{macro}
%
% \subsubsection{\cs{DeclareStringOption}}
%
%    \begin{macro}{\DeclareStringOption}
%    \begin{macrocode}
\newcommand*{\DeclareStringOption}[2][]{%
  \@ifnextchar[{%
    \KVO@DeclareStringOption{#1}{#2}@%
  }{%
    \KVO@DeclareStringOption{#1}{#2}{}[]%
  }%
}
%    \end{macrocode}
%    \end{macro}
%    \begin{macro}{\KVO@DeclareStringOption}
%    \begin{macrocode}
\def\KVO@DeclareStringOption#1#2#3[#4]{%
  \KVO@ifdefinable{\KVO@prefix#2}{%
    \@namedef{\KVO@prefix#2}{#1}%
    \begingroup
      \ifx\\#3\\%
        \toks@{}%
      \else
        \toks@{[{#4}]}%
      \fi
      \edef\x{\endgroup
        \noexpand\define@key{\KVO@family}{#2}\the\toks@{%
          ^^A\begingroup
          ^^A  \toks@{####1}%
          ^^A  \ifx\@currext\@clsextension
          ^^A    \noexpand\ClassInfo
          ^^A  \else
          ^^A    \noexpand\PackageInfo
          ^^A  \fi
          ^^A  {\@currname}{%
          ^^A    [option] #2={\noexpand\the\toks@}%
          ^^A  }%
          ^^A\endgroup
          \noexpand\def
          \expandafter\noexpand\csname\KVO@prefix#2\endcsname{####1}%
        }%
      }%
    \x
  }%
}
%    \end{macrocode}
%    \end{macro}
%
% \subsubsection{\cs{DeclareVoidOption}}
%
%    \begin{macro}{\DeclareVoidOption}
%    \begin{macrocode}
\newcommand*{\DeclareVoidOption}[2]{%
  \begingroup
    \let\next\@gobbletwo
    \KVO@ifdefinable{\KVO@prefix#1}{%
      \let\next\@firstofone
    }%
  \expandafter\endgroup
  \next{%
    \begingroup
      \edef\x{\endgroup
        \noexpand\define@key{\KVO@family}{#1}[\KVO@VOID@]{%
          \noexpand\KVO@voidkey{\@currname}%
          \ifx\@currext\@clsextension
            \noexpand\@clsextension
          \else
            \noexpand\@pkgextension
          \fi
          {#1}%
          {####1}%
          \expandafter\noexpand\csname\KVO@prefix#1\endcsname
        }%
      }%
    \x
    \begingroup
      \toks@{#2}%
    \expandafter\endgroup
    \expandafter\def
    \csname\KVO@prefix#1\expandafter\endcsname
    \expandafter{\the\toks@}%
  }%
}
\def\KVO@VOID@{@VOID@}
%    \end{macrocode}
%    \end{macro}
%    \begin{macro}{\KVO@voidkey}
%    \begin{tabular}{@{}ll@{}}
%    |#1|& package/class name\\
%    |#2|& |\@pkgextension|/|\@clsextension|\\
%    |#3|& key name\\
%    |#4|& default (|@VOID@|)\\
%    |#5|& macro with option code
%    \end{tabular}
%    \begin{macrocode}
\def\KVO@voidkey#1#2#3#4{%
  \def\CurrentOption{#3}%
  \begingroup
    \def\x{#4}%
  \expandafter\endgroup
  \ifx\x\KVO@VOID@
  \else
    \ifx#2\@clsextension
      \expandafter\ClassWarning
    \else
      \expandafter\PackageWarning
    \fi
    {#1}{%
      Unexpected value for option `#3'\MessageBreak
      is ignored%
    }%
  \fi
  ^^A\ifx#2\@clsextension
  ^^A  \expandafter\ClassInfo
  ^^A\else
  ^^A  \expandafter\PackageInfo
  ^^A\fi
  ^^A{#1}{[option] #3}%
}
%    \end{macrocode}
%    \end{macro}
%
% \subsubsection{\cs{DeclareDefaultOption}}
%
%    \begin{macro}{\DeclareDefaultOption}
%    \begin{macrocode}
\newcommand*{\DeclareDefaultOption}{%
  \@namedef{KVO@default@\@currname.\@currext}%
}
%    \end{macrocode}
%    \end{macro}
%
% \subsubsection{\cs{DeclareLocalOptions}}
%
%    \begin{macro}{\DeclareLocalOptions}
%    \begin{macrocode}
\newcommand*{\DeclareLocalOptions}[1]{%
  \comma@parse{#1}\KVO@DeclareLocalOption
}
%    \end{macrocode}
%    \end{macro}
%    \begin{macro}{\KVO@DeclareLocalOption}
%    \begin{macrocode}
\def\KVO@DeclareLocalOption#1{%
  \expandafter\def\csname KVO@local@\KVO@family @#1\endcsname{}%
}
%    \end{macrocode}
%    \end{macro}
%
% \subsection{Dynamic options}
%
% \subsubsection{\cs{DisableKeyvalOption}}
%
%    \begin{macrocode}
\SetupKeyvalOptions{%
  family=KVOdyn,%
  prefix=KVOdyn@%
}
\DeclareBoolOption[true]{global}
\DeclareComplementaryOption{local}{global}
\DeclareStringOption[undef]{action}
\let\KVOdyn@name\relax
\let\KVOdyn@ext\@empty
\define@key{KVOdyn}{class}{%
  \def\KVOdyn@name{#1}%
  \let\KVOdyn@ext\@clsextension
}
\define@key{KVOdyn}{package}{%
  \def\KVOdyn@name{#1}%
  \let\KVOdyn@ext\@pkgextension
}
\newcommand*{\DisableKeyvalOption}[3][]{%
  \begingroup
    \kvsetkeys{KVOdyn}{#1}%
    \def\x{\endgroup}%
    \@ifundefined{KVO@action@\KVOdyn@action}{%
      \PackageError{kvoptions}{%
        Unknown disable action %
        `\expandafter\strip@prefix\meaning\KVOdyn@action'\MessageBreak
        for option `#3' in keyval family '#2'%
      }\@ehc
    }{%
      \csname KVO@action@\KVOdyn@action\endcsname{#2}{#3}%
    }%
  \x
}
\def\KVO@action@undef#1#2{%
  \edef\x{\endgroup
    \ifKVOdyn@global\global\fi
    \let
    \expandafter\noexpand\csname KV@#1@#2\endcsname
    \relax
    \ifKVOdyn@global\global\fi
    \let
    \expandafter\noexpand\csname KV@#1@#2@default\endcsname
    \relax
  }%
  ^^A\PackageInfo{kvoptions}{%
  ^^A  [option] key `#2' of family `#1'\MessageBreak
  ^^A  is disabled (undef, \ifKVOdyn@global global\else local\fi)%
  ^^A}%
}
\def\KVO@action@ignore#1#2{%
  \edef\x{\endgroup
    \ifKVOdyn@global\global\fi
    \let
    \expandafter\noexpand\csname KV@#1@#2\endcsname
    \noexpand\@gobble
    \ifKVOdyn@global\global\fi
    \let
    \expandafter\noexpand\csname KV@#1@#2@default\endcsname
    \noexpand\@empty
  }%
  ^^A\PackageInfo{kvoptions}{%
  ^^A  [option] key `#2' of family `#1'\MessageBreak
  ^^A  is disabled (ignore, \ifKVOdyn@global global\else local\fi)%
  ^^A}%
}
\def\KVO@action@error{%
  \KVO@do@action{error}%
}
\def\KVO@action@warning{%
  \KVO@do@action{warning}%
}
%    \end{macrocode}
%    \begin{tabular}{@{}ll@{}}
%    |#1|& |error| or |warning|\\
%    |#2|& \meta{family}\\
%    |#3|& \meta{key}\\
%    \end{tabular}
%    \begin{macrocode}
\def\KVO@do@action#1#2#3{%
  \ifx\KVOdyn@name\relax
    \PackageError{kvoptions}{%
      Action type `#1' needs package/class name\MessageBreak
      for key `#3' in family `#2'%
    }\@ehc
  \else
    \edef\x{\endgroup
      \noexpand\define@key{#2}{#3}[]{%
        \expandafter\noexpand\csname KVO@disable@#1\endcsname
        {\KVOdyn@name}\noexpand\KVOdyn@ext{#3}%
      }%
      \ifKVOdyn@global
        \global\let
        \expandafter\noexpand\csname KV@#2@#3\endcsname
        \expandafter\noexpand\csname KV@#2@#3\endcsname
        \global\let
        \expandafter\noexpand\csname KV@#2@#3@default\endcsname
        \expandafter\noexpand\csname KV@#2@#3@default\endcsname
      \fi
    }%
    ^^A\ifx\KVOdyn@ext\@clsextension
    ^^A  \expandafter\ClassInfo
    ^^A\else
    ^^A   \expandafter\PackageInfo
    ^^A\fi
    ^^A{\KVOdyn@name}{%
    ^^A  [option] key `#3' of family `#2'\MessageBreak
    ^^A  is disabled (#1, \ifKVOdyn@global global\else local\fi)%
    ^^A}%
  \fi
}
\def\KVO@disable@error#1#2#3{%
  \ifx#2\@clsextension
    \expandafter\ClassError
  \else
    \expandafter\PackageError
  \fi
  {#1}{%
    Option `#3' is given too late,\MessageBreak
    now the option is ignored%
  }\@ehc
}
\def\KVO@disable@warning#1#2#3{%
  \ifx#2\@clsextension
    \expandafter\ClassWarning
  \else
    \expandafter\PackageWarning
  \fi
  {#1}{%
    Option `#3' is already consumed\MessageBreak
    and has no effect%
  }%
}
%    \end{macrocode}
%
% \subsection{Change option code}
%
% \subsubsection{\cs{AddToKeyvalOption}}
%
%    \begin{macro}{\AddToKeyvalOption}
%    \begin{macrocode}
\newcommand*{\AddToKeyvalOption}{%
  \@ifstar{%
    \begingroup
      \edef\x{\endgroup
        \noexpand\KVO@AddToKeyvalOption{\KVO@family}%
      }%
    \x
  }%
  \KVO@AddToKeyvalOption
}
%    \end{macrocode}
%    \end{macro}
%    \begin{macro}{\KVO@AddToKeyvalOption}
%    \begin{macrocode}
\def\KVO@AddToKeyvalOption#1#2{%
  \@ifundefined{KV@#1@#2}{%
    \PackageWarning{kvoptions}{%
      Key `#2' of family `#1' does not exist.\MessageBreak
      Ignoring \string\AddToKeyvalOption
    }%
    \@gobble
  }{%
    \edef\KVO@next{%
      \noexpand\KVO@@AddToKeyvalOption
      \expandafter\noexpand\csname KV@#1@#2\endcsname
    }%
    \afterassignment\KVO@next
    \def\KVO@temp##1%
  }%
}
%    \end{macrocode}
%    \end{macro}
%    \begin{macro}{\KVO@@AddToKeyvalOption}
%    \begin{macrocode}
\def\KVO@@AddToKeyvalOption#1{%
  \begingroup
    \toks@\expandafter{#1{##1}}%
    \toks@\expandafter{\the\expandafter\toks@\KVO@temp{##1}}%
    \edef\x{\endgroup
      \noexpand\def\noexpand#1####1{\the\toks@}%
    }%
  \x
}
%    \end{macrocode}
%    \end{macro}
%
% \subsection{Process options}
%
% \subsubsection{Get global options}
%
%    Package \xpackage{xkeyval} removes options with equal signs from
%    the global options (\cs{@classoptionslist}). The effect is that
%    other packages and classes will not see these global options anymore.
%    A bug-report was answered that this behaviour is ``by design''.
%    Thus I call it a design bug. Now getting the global options require
%    an algorithm instead of a simple macro call.
%    \begin{macrocode}
%</package>
%<*package|patch>
%    \end{macrocode}
%    \begin{macro}{\KVO@IfDefThen}
%    Call |#2| if command |#1| is defined and not \cs{relax}.
%    (Package \xpackage{kvoptions-patch} does not load
%    package \xpackage{ltxcmds}.)
%    \begin{macrocode}
\def\KVO@IfDefThen#1#2{%
  \ifx#1\ltx@undefined
  \else
    \ifx#1\relax
    \else
      #2%
    \fi
  \fi
}%
%    \end{macrocode}
%    \end{macro}
%    \begin{macro}{\KVO@GetClassOptionsList}
%    \begin{macrocode}
\def\KVO@GetClassOptionsList{%
  \let\KVO@classoptionslist\@classoptionslist
  \KVO@IfDefThen\@classoptionslist{%
    \KVO@IfDefThen\XKV@documentclass{%
      \ifx\XKV@documentclass\ltx@empty
      \else
        \KVO@IfDefThen\XKV@classoptionslist{%
          \ifx\XKV@classoptionslist\ltx@empty
          \else
            \let\KVO@classoptionslist\XKV@classoptionslist
          \fi
        }%
      \fi
    }%
  }%
}%
%    \end{macrocode}
%    \end{macro}
%    \begin{macrocode}
%</package|patch>
%<*package>
%    \end{macrocode}
%
% \subsubsection{\cs{ProcessKeyvalOptions}}
%
%    \begin{macro}{\ProcessKeyvalOptions}
%    If the optional star is given, we get the family
%    name and expand it for safety.
%    \begin{macrocode}
\newcommand*{\ProcessKeyvalOptions}{%
  \@ifstar{%
    \begingroup
      \edef\x{\endgroup
        \noexpand\KVO@ProcessKeyvalOptions{\KVO@family}%
      }%
    \x
  }%
  \KVO@ProcessKeyvalOptions
}
%    \end{macrocode}
%    \end{macro}
%
%    \begin{macrocode}
\def\KVO@ProcessKeyvalOptions#1{%
  \let\@tempc\relax
  \let\KVO@temp\@empty
%    \end{macrocode}
%    Add any global options that are known to KV to the start of
%    the list being built in |\KVO@temp| and mark them used (by
%    removing them from the unused option list).
%    \begin{macrocode}
  \ifx\@currext\@clsextension
  \else
    \KVO@GetClassOptionsList
    \ifx\KVO@classoptionslist\relax
    \else
      \@for\KVO@CurrentOption:=\KVO@classoptionslist\do{%
        \@ifundefined{KV@#1@\expandafter\KVO@getkey
                      \KVO@CurrentOption=\@nil}{%
        }{%
          \@ifundefined{KVO@local@#1@\expandafter\KVO@getkey
                        \KVO@CurrentOption=\@nil}{%
            \ifx\KVO@Patch Y%
              \edef\KVO@temp{%
                \etex@unexpanded\expandafter{%
                  \KVO@temp
                }%
                ,%
                \etex@unexpanded\expandafter{%
                  \KVO@CurrentOption
                }%
                ,%
              }%
              \ltx@onelevel@sanitize\KVO@CurrentOption
            \else
              \edef\KVO@temp{%
                \KVO@temp
                ,%
                \KVO@CurrentOption
                ,%
              }%
            \fi
            \@expandtwoargs\@removeelement\KVO@CurrentOption
              \@unusedoptionlist\@unusedoptionlist
          }{}%
        }%
      }%
    \fi
  \fi
%    \end{macrocode}
%     Now stick the package options at the end of the list and wrap
%     in a call to \cs{setkeys}. A class ignores unknown global
%     options, we must remove them to prevent error messages
%     from \cs{setkeys}.
%    \begin{macrocode}
  \begingroup
    \toks\tw@{}%
    \@ifundefined{opt@\@currname.\@currext}{%
      \toks@\expandafter{\KVO@temp}%
    }{%
      \toks@\expandafter\expandafter\expandafter{%
        \csname opt@\@currname.\@currext\endcsname
      }%
      \ifx\@currext\@clsextension
        \edef\CurrentOption{\the\toks@}%
        \toks@\expandafter{\KVO@temp}%
        \@for\CurrentOption:=\CurrentOption\do{%
          \@ifundefined{%
            KV@#1@\expandafter\KVO@getkey\CurrentOption=\@nil
          }{%
%    \end{macrocode}
%    A class puts not used options in the unused option list
%    unless there is a default handler.
%    \begin{macrocode}
            \@ifundefined{KVO@default@\@currname.\@currext}{%
              \ifx\KVO@Patch Y%
                \ltx@onelevel@sanitize\CurrentOption
              \fi
              \ifx\@unusedoptionlist\@empty
                \global\let\@unusedoptionlist\CurrentOption
              \else
                \expandafter\expandafter\expandafter\gdef
                \expandafter\expandafter\expandafter\@unusedoptionlist
                \expandafter\expandafter\expandafter{%
                  \expandafter\@unusedoptionlist
                  \expandafter,\CurrentOption
                }%
              \fi
            }{%
              \toks\tw@\expandafter{%
                \the\toks\expandafter\tw@\expandafter,\CurrentOption
              }%
            }%
          }{%
            \toks@\expandafter{%
              \the\expandafter\toks@\expandafter,\CurrentOption
            }%
          }%
        }%
      \else
%    \end{macrocode}
%    Without default action we pass all options to \cs{setkeys}.
%    Otherwise we have to check which options are known.
%    These are passed to \cs{setkeys}. For the others the default
%    action is performed.
%    \begin{macrocode}
        \@ifundefined{KVO@default@\@currname.\@currext}{%
          \toks@\expandafter\expandafter\expandafter{%
            \expandafter\KVO@temp\the\toks@
          }%
        }{%
          \edef\CurrentOption{\the\toks@}%
          \toks@\expandafter{\KVO@temp}%
          \@for\CurrentOption:=\CurrentOption\do{%
            \@ifundefined{%
              KV@#1@\expandafter\KVO@getkey\CurrentOption=\@nil
            }{%
              \toks\tw@\expandafter{%
                \the\toks\expandafter\tw@\expandafter,\CurrentOption
              }%
            }{%
              \toks@\expandafter{%
                \the\expandafter\toks@\expandafter,\CurrentOption
              }%
            }%
          }%
        }%
      \fi
    }%
    \edef\KVO@temp{\endgroup
      \noexpand\KVO@calldefault{\the\toks\tw@}%
      \noexpand\KVO@setkeys{#1}{\the\toks@}%
    }%
  \KVO@temp
%    \end{macrocode}
%    Some cleanup of \cs{ProcessOptions}.
%    \begin{macrocode}
  \let\CurrentOption\@empty
  \AtEndOfPackage{\let\@unprocessedoptions\relax}%
}
%    \end{macrocode}
%
% \subsubsection{\cs{ProcessLocalKeyvalOptions}}
%
%    \begin{macro}{\ProcessLocalKeyvalOptions}
%    If the optional star is given, we get the family
%    name and expand it for safety.
%    \begin{macrocode}
\newcommand*{\ProcessLocalKeyvalOptions}{%
  \@ifstar{%
    \begingroup
      \edef\x{\endgroup
        \noexpand\KVO@ProcessLocalKeyvalOptions{\KVO@family}%
      }%
    \x
  }%
  \KVO@ProcessLocalKeyvalOptions
}
%    \end{macrocode}
%    \end{macro}
%
%    \begin{macrocode}
\def\KVO@ProcessLocalKeyvalOptions#1{%
  \let\@tempc\relax
  \let\KVO@temp\@empty
%    \end{macrocode}
%    Check if \cs{ProcessLocalKeyvalOptions} is called inside
%    a package.
%    \begin{macrocode}
  \ifx\@currext\@pkgextension
  \else
    \PackageError{kvoptions}{%
      \string\ProcessLocalKeyvalOptions\space is intended for packages only%
    }\@ehc
  \fi
%    \end{macrocode}
%     The package options are put into toks register \cs{toks@}.
%    \begin{macrocode}
  \begingroup
    \toks\tw@{}%
    \@ifundefined{opt@\@currname.\@currext}{%
      \toks@\expandafter{\KVO@temp}%
    }{%
      \toks@\expandafter\expandafter\expandafter{%
        \csname opt@\@currname.\@currext\endcsname
      }%
%    \end{macrocode}
%    Without default action we pass all options to \cs{setkeys}.
%    Otherwise we have to check which options are known.
%    These are passed to \cs{setkeys}. For the others the default
%    action is performed.
%    \begin{macrocode}
      \@ifundefined{KVO@default@\@currname.\@currext}{%
        \toks@\expandafter\expandafter\expandafter{%
          \expandafter\KVO@temp\the\toks@
        }%
      }{%
        \edef\CurrentOption{\the\toks@}%
        \toks@\expandafter{\KVO@temp}%
        \@for\CurrentOption:=\CurrentOption\do{%
          \@ifundefined{%
            KV@#1@\expandafter\KVO@getkey\CurrentOption=\@nil
          }{%
            \toks\tw@\expandafter{%
              \the\toks\expandafter\tw@\expandafter,\CurrentOption
            }%
          }{%
            \toks@\expandafter{%
              \the\expandafter\toks@\expandafter,\CurrentOption
            }%
          }%
        }%
      }%
    }%
    \edef\KVO@temp{\endgroup
      \noexpand\KVO@calldefault{\the\toks\tw@}%
      \noexpand\KVO@setkeys{#1}{\the\toks@}%
    }%
  \KVO@temp
%    \end{macrocode}
%    Some cleanup of \cs{ProcessOptions}.
%    \begin{macrocode}
  \let\CurrentOption\@empty
  \AtEndOfPackage{\let\@unprocessedoptions\relax}%
}
%    \end{macrocode}
%
% \subsubsection{Helper macros}
%
%    \begin{macro}{\KVO@getkey}
%    Extract the key part of a key=value pair.
%    \begin{macrocode}
\def\KVO@getkey#1=#2\@nil{#1}
%    \end{macrocode}
%    \end{macro}
%
%    \begin{macro}{\KVO@calldefault}
%    \begin{macrocode}
\def\KVO@calldefault#1{%
  \begingroup
    \def\x{#1}%
  \expandafter\endgroup
  \ifx\x\@empty
  \else
    \@for\CurrentOption:=#1\do{%
      \ifx\CurrentOption\@empty
      \else
        \expandafter\KVO@setcurrents\CurrentOption=\@nil
        \@nameuse{KVO@default@\@currname.\@currext}%
      \fi
    }%
  \fi
}
%    \end{macrocode}
%    \end{macro}
%    \begin{macro}{\KVO@setcurrents}
%    Extract the key part of a key=value pair.
%    \begin{macrocode}
\def\KVO@setcurrents#1=#2\@nil{%
  \def\CurrentOptionValue{#2}%
  \ifx\CurrentOptionValue\@empty
    \let\CurrentOptionKey\CurrentOption
    \let\CurrentOptionValue\relax
  \else
    \edef\CurrentOptionKey{\zap@space#1 \@empty}%
    \expandafter\KVO@setcurrentvalue\CurrentOption\@nil
  \fi
}
%    \end{macrocode}
%    \end{macro}
%    \begin{macro}{\KV@setcurrentvalue}
%    Here the value part is parsed. Package \xpackage{keyval}'s
%    \cs{KV@@sp@def} helps in removing spaces at the begin and
%    end of the value.
%    \begin{macrocode}
\def\KVO@setcurrentvalue#1=#2\@nil{%
  \KV@@sp@def\CurrentOptionValue{#2}%
}
%    \end{macrocode}
%    \end{macro}
%
% \subsection{\plainTeX}
%
%    Disable \LaTeX\ stuff.
%    \begin{macrocode}
\begingroup\expandafter\expandafter\expandafter\endgroup
\expandafter\ifx\csname documentclass\endcsname\relax
  \def\ProcessKeyvalOptions{%
    \@ifstar{}\@gobble
  }%
\fi
%    \end{macrocode}
%
%    \begin{macrocode}
\KVO@AtEnd%
%</package>
%    \end{macrocode}
%
% \subsection{Package \xpackage{kvoptions-patch}}
%
%    \begin{macrocode}
%<*patch>
\NeedsTeXFormat{LaTeX2e}
\begingroup\catcode61\catcode48\catcode32=10\relax%
  \catcode13=5 % ^^M
  \endlinechar=13 %
  \catcode123=1 % {
  \catcode125=2 % }
  \catcode64=11 % @
  \def\x{\endgroup
    \expandafter\edef\csname KVO@AtEnd\endcsname{%
      \endlinechar=\the\endlinechar\relax
      \catcode13=\the\catcode13\relax
      \catcode32=\the\catcode32\relax
      \catcode35=\the\catcode35\relax
      \catcode61=\the\catcode61\relax
      \catcode64=\the\catcode64\relax
      \catcode123=\the\catcode123\relax
      \catcode125=\the\catcode125\relax
    }%
  }%
\x\catcode61\catcode48\catcode32=10\relax%
\catcode13=5 % ^^M
\endlinechar=13 %
\catcode35=6 % #
\catcode64=11 % @
\catcode123=1 % {
\catcode125=2 % }
\def\TMP@EnsureCode#1#2{%
  \edef\KVO@AtEnd{%
    \KVO@AtEnd
    \catcode#1=\the\catcode#1\relax
  }%
  \catcode#1=#2\relax
}
\TMP@EnsureCode{39}{12}% '
\TMP@EnsureCode{40}{12}% (
\TMP@EnsureCode{41}{12}% )
\TMP@EnsureCode{43}{12}% +
\TMP@EnsureCode{44}{12}% ,
\TMP@EnsureCode{45}{12}% -
\TMP@EnsureCode{46}{12}% .
\TMP@EnsureCode{47}{12}% /
\TMP@EnsureCode{58}{12}% :
\TMP@EnsureCode{60}{12}% <
\TMP@EnsureCode{62}{12}% >
\TMP@EnsureCode{91}{12}% [
\TMP@EnsureCode{93}{12}% ]
\TMP@EnsureCode{96}{12}% `
\TMP@EnsureCode{124}{12}% |
\edef\KVO@AtEnd{\KVO@AtEnd\noexpand\endinput}
\ProvidesPackage{kvoptions-patch}%
  [2016/05/16 v3.12 LaTeX patch for keyval options (HO)]%
%    \end{macrocode}
%
%    Check for \eTeX.
%    \begin{macrocode}
\begingroup\expandafter\expandafter\expandafter\endgroup
\expandafter\ifx\csname eTeXversion\endcsname\relax
  \PackageWarningNoLine{kvoptions-patch}{%
    Package loading is aborted, because e-TeX is missing%
  }%
  \expandafter\KVO@AtEnd
\fi%
%    \end{macrocode}
%
%    Package \xpackage{etexcmds} for \cs{etex@unexpanded}.
%    \begin{macrocode}
\RequirePackage{etexcmds}[2007/09/09]
\ifetex@unexpanded
\else
  \PackageError{kvoptions-patch}{%
    Could not find eTeX's \string\unexpanded.\MessageBreak
    Try adding \string\RequirePackage\string{etexcmds\string} %
    before \string\documentclass%
  }\@ehd
  \expandafter\KVO@AtEnd
\fi%
%    \end{macrocode}
%
%    Check for package \xpackage{xkvltxp}.
%    \begin{macrocode}
\@ifpackageloaded{xkvltxp}{%
  \PackageWarningNoLine{kvoptions}{%
    Option `patch' cannot be used together with\MessageBreak
    package `xkvltxp' that is already loaded.\MessageBreak
    Therefore package loading is aborted%
  }%
  \KVO@AtEnd
}{}%
%    \end{macrocode}
%
%    \begin{macrocode}
\def\@if@ptions#1#2#3{%
  \begingroup
    \KVO@normalize\KVO@temp{#3}%
    \edef\x{\endgroup
      \noexpand\@if@pti@ns{%
        \detokenize\expandafter\expandafter\expandafter{%
          \csname opt@#2.#1\endcsname
        }%
      }{%
        \detokenize\expandafter{\KVO@temp}%
      }%
    }%
  \x
}
%    \end{macrocode}
%
%    \begin{macrocode}
\def\@pass@ptions#1#2#3{%
  \KVO@normalize\KVO@temp{#2}%
  \@ifundefined{opt@#3.#1}{%
    \expandafter\gdef\csname opt@#3.#1%
          \expandafter\endcsname\expandafter{%
      \KVO@temp
    }%
  }{%
    \expandafter\gdef\csname opt@#3.#1%
          \expandafter\expandafter\expandafter\endcsname
          \expandafter\expandafter\expandafter{%
      \csname opt@#3.#1\expandafter\endcsname\expandafter,\KVO@temp
    }%
  }%
}
%    \end{macrocode}
%
%    \begin{macrocode}
\def\ProcessOptions{%
  \let\ds@\@empty
  \@ifundefined{opt@\@currname.\@currext}{%
    \let\@curroptions\@empty
  }{%
    \expandafter\expandafter\expandafter\def
    \expandafter\expandafter\expandafter\@curroptions
    \expandafter\expandafter\expandafter{%
      \csname opt@\@currname.\@currext\endcsname
    }%
  }%
  \@ifstar\KVO@xprocess@ptions\KVO@process@ptions
}
%    \end{macrocode}
%
%    \begin{macrocode}
\def\KVO@process@ptions{%
  \@for\CurrentOption:=\@declaredoptions\do{%
    \ifx\CurrentOption\@empty
    \else
      \begingroup
        \ifx\@currext\@clsextension
          \toks@{}%
        \else
          \KVO@GetClassOptionsList
          \toks@\expandafter{\KVO@classoptionslist,}%
        \fi
        \toks\tw@\expandafter{\@curroptions}%
        \edef\x{\endgroup
          \noexpand\in@{,\CurrentOption,}{,\the\toks@\the\toks\tw@,}%
        }%
      \x
      \ifin@
        \KVO@use@ption
        \expandafter\let\csname ds@\CurrentOption\endcsname\@empty
      \fi
    \fi
  }%
  \KVO@process@pti@ns
}
%    \end{macrocode}
%
%    \begin{macrocode}
\def\KVO@xprocess@ptions{%
  \ifx\@currext\@clsextension
  \else
    \KVO@GetClassOptionsList
    \@for\CurrentOption:=\KVO@classoptionslist\do{%
      \ifx\CurrentOption\@empty
      \else
        \KVO@in@\CurrentOption\@declaredoptions
        \ifin@
          \KVO@use@ption
          \expandafter\let\csname ds@\CurrentOption\endcsname\@empty
        \fi
      \fi
    }%
  \fi
  \KVO@process@pti@ns
}
%    \end{macrocode}
%
%    \begin{macrocode}
\def\KVO@in@#1#2{%
  \in@false
  \begingroup
    \@for\x:=#2\do{%
      \ifx\x#1\relax
        \in@true
      \fi
    }%
    \edef\x{\endgroup
      \ifin@
        \noexpand\in@true
      \fi
    }%
  \x
}
%    \end{macrocode}
%
%    \begin{macrocode}
\def\KVO@process@pti@ns{%
  \@for\CurrentOption:=\@curroptions\do{%
    \@ifundefined{ds@\KVO@SanitizedCurrentOption}{%
      \KVO@use@ption
      \default@ds
    }%
    \KVO@use@ption
  }%
  \@for\CurrentOption:=\@declaredoptions\do{%
    \expandafter\let\csname ds@\CurrentOption\endcsname\relax
  }%
  \let\CurrentOption\@empty
  \let\@fileswith@pti@ns\@@fileswith@pti@ns
  \AtEndOfPackage{\let\@unprocessedoptions\relax}%
}
%    \end{macrocode}
%
%    \begin{macrocode}
\def\KVO@use@ption{%
  \begingroup
    \edef\x{\endgroup
      \noexpand\@removeelement{%
        \detokenize\expandafter{\CurrentOption}%
      }{%
        \detokenize\expandafter{\@unusedoptionlist}%
      }%
    }%
  \x\@unusedoptionlist
  \csname ds@\KVO@SanitizedCurrentOption\endcsname
}
%    \end{macrocode}
%
%    \begin{macrocode}
\def\OptionNotUsed{%
  \ifx\@currext\@clsextension
    \xdef\@unusedoptionlist{%
      \ifx\@unusedoptionlist\@empty
      \else
        \detokenize\expandafter{\@unusedoptionlist,}%
      \fi
      \detokenize\expandafter{\CurrentOption}%
    }%
  \fi
}
%    \end{macrocode}
%
%    Variant of \cs{ExecuteOptions} that better protects \cs{CurrentOption}.
%    \begin{macrocode}
\def\CurrentOption@SaveLevel{0}
\def\ExecuteOptions{%
  \expandafter\KVO@ExecuteOptions
      \csname CurrentOption@\CurrentOption@SaveLevel\endcsname
}
\def\KVO@ExecuteOptions#1#2{%
  \let#1\CurrentOption
  \edef\CurrentOption@SaveLevel{%
    \the\numexpr\CurrentOption@SaveLevel+1%
  }%
  \@for\CurrentOption:=#2\do{%
    \csname ds@\CurrentOption\endcsname
  }%
  \edef\CurrentOption@SaveLevel{%
    \the\numexpr\CurrentOption@SaveLevel-1%
  }%
  \let\CurrentOption#1%
}
%    \end{macrocode}
%
%    \begin{macrocode}
\def\KVO@fileswith@pti@ns#1[#2]#3[#4]{%
  \ifx#1\@clsextension
    \ifx\@classoptionslist\relax
      \KVO@normalize\KVO@temp{#2}%
      \expandafter\gdef\expandafter\@classoptionslist\expandafter{%
        \KVO@temp
      }%
      \def\reserved@a{%
        \KVO@onefilewithoptions{#3}[{#2}][{#4}]#1%
        \@documentclasshook
      }%
    \else
      \def\reserved@a{%
        \KVO@onefilewithoptions{#3}[{#2}][{#4}]#1%
      }%
    \fi
  \else
    \begingroup
      \let\KVO@temp\relax
      \let\KVO@onefilewithoptions\relax
      \let\@pkgextension\relax
      \def\reserved@b##1,{%
        \ifx\@nil##1\relax
        \else
          \ifx\relax##1\relax
          \else
            \KVO@onefilewithoptions{##1}[{\KVO@temp}][{#4}]%
            \@pkgextension
          \fi
          \expandafter\reserved@b
        \fi
      }%
      \edef\reserved@a{\zap@space#3 \@empty}%
      \edef\reserved@a{\expandafter\reserved@b\reserved@a,\@nil,}%
      \toks@{#2}%
      \def\KVO@temp{\the\toks@}%
    \edef\reserved@a{\endgroup \reserved@a}%
  \fi
  \reserved@a
}
%    \end{macrocode}
%
%    \begin{macrocode}
\def\KVO@onefilewithoptions#1[#2][#3]#4{%
  \@pushfilename
  \xdef\@currname{#1}%
  \global\let\@currext#4%
  \expandafter\let\csname\@currname.\@currext-h@@k\endcsname\@empty
  \let\CurrentOption\@empty
  \@reset@ptions
  \makeatletter
  \def\reserved@a{%
    \@ifl@aded\@currext{#1}{%
      \@if@ptions\@currext{#1}{#2}{%
      }{%
        \begingroup
          \@ifundefined{opt@#1.\@currext}{%
            \def\x{}%
          }{%
            \edef\x{%
              \expandafter\expandafter\expandafter\strip@prefix
              \expandafter\meaning\csname opt@#1.\@currext\endcsname
            }%
          }%
          \def\y{#2}%
          \edef\y{\expandafter\strip@prefix\meaning\y}%
          \@latex@error{Option clash for \@cls@pkg\space #1}{%
            The package #1 has already been loaded %
            with options:\MessageBreak
            \space\space[\x]\MessageBreak
            There has now been an attempt to load it %
             with options\MessageBreak
            \space\space[\y]\MessageBreak
            Adding the global options:\MessageBreak
            \space\space
                 \x,\y\MessageBreak
            to your \noexpand\documentclass declaration may fix this.%
            \MessageBreak
            Try typing \space <return> \space to proceed.%
          }%
        \endgroup
      }%
    }{%
      \@pass@ptions\@currext{#2}{#1}%
      \global\expandafter
      \let\csname ver@\@currname.\@currext\endcsname\@empty
      \InputIfFileExists
        {\@currname.\@currext}%
        {}%
        {\@missingfileerror\@currname\@currext}%
      \let\@unprocessedoptions\@@unprocessedoptions
      \csname\@currname.\@currext-h@@k\endcsname
      \expandafter\let\csname\@currname.\@currext-h@@k\endcsname
              \@undefined
      \@unprocessedoptions
    }%
    \@ifl@ter\@currext{#1}{#3}{%
    }{%
      \@latex@warning@no@line{%
        You have requested,\on@line, %
        version\MessageBreak
          #3' of \@cls@pkg\space #1,\MessageBreak
        but only version\MessageBreak
         `\csname ver@#1.\@currext\endcsname'\MessageBreak
        is available%
      }%
    }%
    \ifx\@currext\@clsextension\let\LoadClass\@twoloadclasserror\fi
    \@popfilename
    \@reset@ptions
  }%
  \reserved@a
}
%    \end{macrocode}
%
%    \begin{macrocode}
\def\@unknownoptionerror{%
  \@latex@error{%
    Unknown option `\KVO@SanitizedCurrentOption' %
    for \@cls@pkg\space`\@currname'%
  }{%
    The option `\KVO@SanitizedCurrentOption' was not declared in %
    \@cls@pkg\space`\@currname', perhaps you\MessageBreak
    misspelled its name. %
    Try typing \space <return> %
    \space to proceed.%
  }%
}
%    \end{macrocode}
%
%    \begin{macrocode}
\def\@@unprocessedoptions{%
  \ifx\@currext\@pkgextension
    \@ifundefined{opt@\@currname.\@currext}{%
      \let\@curroptions\@empty
    }{%
      \expandafter\let\expandafter\@curroptions
          \csname opt@\@currname.\@currext\endcsname
    }%
    \@for\CurrentOption:=\@curroptions\do{%
        \ifx\CurrentOption\@empty\else\@unknownoptionerror\fi
    }%
  \fi
}
%    \end{macrocode}
%
%    \begin{macrocode}
\def\KVO@SanitizedCurrentOption{%
  \expandafter\strip@prefix\meaning\CurrentOption
}
%    \end{macrocode}
%
%    Normalize option list.
%    \begin{macrocode}
\def\KVO@normalize#1#2{%
  \let\KVO@result\@empty
  \KVO@splitcomma#2,\@nil
  \let#1\KVO@result
}
\def\KVO@splitcomma#1,#2\@nil{%
  \KVO@ifempty{#1}{}{%
    \KVO@checkkv#1=\@nil
  }%
  \KVO@ifempty{#2}{}{\KVO@splitcomma#2\@nil}%
}
\def\KVO@ifempty#1{%
  \expandafter\ifx\expandafter\\\detokenize{#1}\\%
    \expandafter\@firstoftwo
  \else
    \expandafter\@secondoftwo
  \fi
}
\def\KVO@checkkv#1=#2\@nil{%
  \KVO@ifempty{#2}{%
    % option without value
    \edef\KVO@x{\zap@space#1 \@empty}%
    \ifx\KVO@x\@empty
      % ignore empty option
    \else
      % append to list
      \edef\KVO@result{%
        \etex@unexpanded\expandafter{\KVO@result},\KVO@x
      }%
    \fi
  }{%
    % #1: "key", #2: "value="
    % add key part
    \edef\KVO@result{%
      \etex@unexpanded\expandafter{\KVO@result},%
      \zap@space#1 \@empty
    }%
    \futurelet\@let@token\KVO@checkfirsttok#2 \@nil| = \@nil|\KVO@nil
  }%
}
\def\KVO@checkfirsttok{%
  \ifx\@let@token\bgroup
    % no space at start
    \expandafter\KVO@removelastspace\expandafter=%
    % "<value><spaceopt>= \@nil"
  \else
    \expandafter\KVO@checkfirstA
  \fi
}
\def\KVO@checkfirstA#1 #2\@nil{%
  \KVO@ifempty{#2}{%
    \KVO@removelastspace=#1 \@nil
  }{%
    \KVO@ifempty{#1}{%
      \KVO@removelastspace=#2\@nil
    }{%
      \KVO@removelastspace=#1 #2\@nil
    }%
  }%
}
\def\KVO@removelastspace#1 = \@nil|#2\KVO@nil{%
  \KVO@ifempty{#2}{%
    \edef\KVO@result{%
      \etex@unexpanded\expandafter{\KVO@result}%
      \etex@unexpanded\expandafter{\KVO@removegarbage#1\KVO@nil}%
    }%
  }{%
    \edef\KVO@result{%
      \etex@unexpanded\expandafter{\KVO@result}%
      \etex@unexpanded{#1}%
    }%
  }%
}
\def\KVO@removegarbage#1= \@nil#2\KVO@nil{#1}%
%    \end{macrocode}
%
%    Arguments |#1| and |#2| are macros.
%    \begin{macrocode}
\def\KVO@removeelement#1#2{%
  \begingroup
    \toks@={}%
    \@for\x:=#2\do{%
      \ifx\x\@empty
      \else
        \ifx\x#1\relax
        \else
          \edef\t{\the\toks@}%
          \ifx\t\@empty
          \else
            \toks@\expandafter{\the\toks@,}%
          \fi
          \toks@\expandafter{\the\expandafter\toks@\x}%
        \fi
      \fi
    }%
    \edef\x{\endgroup
      \def\noexpand#2{\the\toks@}%
    }%
  \x
}
%    \end{macrocode}
%
%
%    \begin{macrocode}
\let\@@fileswith@pti@ns\KVO@fileswith@pti@ns
\ifx\@fileswith@pti@ns\@badrequireerror
\else
  \let\@fileswith@pti@ns\KVO@fileswith@pti@ns
\fi
%    \end{macrocode}
%
%    \begin{macro}{\KVO@Patch}
%    \begin{macrocode}
\let\KVO@Patch=Y
%    \end{macrocode}
%    \end{macro}
%
%    \begin{macrocode}
\KVO@AtEnd%
%</patch>
%    \end{macrocode}
%
% \section{Test}
%
% \subsection{Preface for standard catcode check}
%
%    \begin{macrocode}
%<*test1>
\input miniltx.tex\relax
%</test1>
%    \end{macrocode}
%
% \subsection{Catcode checks for loading}
%
%    \begin{macrocode}
%<*test1>
%    \end{macrocode}
%    \begin{macrocode}
\catcode`\{=1 %
\catcode`\}=2 %
\catcode`\#=6 %
\catcode`\@=11 %
\expandafter\ifx\csname count@\endcsname\relax
  \countdef\count@=255 %
\fi
\expandafter\ifx\csname @gobble\endcsname\relax
  \long\def\@gobble#1{}%
\fi
\expandafter\ifx\csname @firstofone\endcsname\relax
  \long\def\@firstofone#1{#1}%
\fi
\expandafter\ifx\csname loop\endcsname\relax
  \expandafter\@firstofone
\else
  \expandafter\@gobble
\fi
{%
  \def\loop#1\repeat{%
    \def\body{#1}%
    \iterate
  }%
  \def\iterate{%
    \body
      \let\next\iterate
    \else
      \let\next\relax
    \fi
    \next
  }%
  \let\repeat=\fi
}%
\def\RestoreCatcodes{}
\count@=0 %
\loop
  \edef\RestoreCatcodes{%
    \RestoreCatcodes
    \catcode\the\count@=\the\catcode\count@\relax
  }%
\ifnum\count@<255 %
  \advance\count@ 1 %
\repeat

\def\RangeCatcodeInvalid#1#2{%
  \count@=#1\relax
  \loop
    \catcode\count@=15 %
  \ifnum\count@<#2\relax
    \advance\count@ 1 %
  \repeat
}
\def\RangeCatcodeCheck#1#2#3{%
  \count@=#1\relax
  \loop
    \ifnum#3=\catcode\count@
    \else
      \errmessage{%
        Character \the\count@\space
        with wrong catcode \the\catcode\count@\space
        instead of \number#3%
      }%
    \fi
  \ifnum\count@<#2\relax
    \advance\count@ 1 %
  \repeat
}
\def\space{ }
\expandafter\ifx\csname LoadCommand\endcsname\relax
  \def\LoadCommand{\input kvoptions.sty\relax}%
\fi
\def\Test{%
  \RangeCatcodeInvalid{0}{47}%
  \RangeCatcodeInvalid{58}{64}%
  \RangeCatcodeInvalid{91}{96}%
  \RangeCatcodeInvalid{123}{255}%
  \catcode`\@=12 %
  \catcode`\\=0 %
  \catcode`\%=14 %
  \LoadCommand
  \RangeCatcodeCheck{0}{36}{15}%
  \RangeCatcodeCheck{37}{37}{14}%
  \RangeCatcodeCheck{38}{47}{15}%
  \RangeCatcodeCheck{48}{57}{12}%
  \RangeCatcodeCheck{58}{63}{15}%
  \RangeCatcodeCheck{64}{64}{12}%
  \RangeCatcodeCheck{65}{90}{11}%
  \RangeCatcodeCheck{91}{91}{15}%
  \RangeCatcodeCheck{92}{92}{0}%
  \RangeCatcodeCheck{93}{96}{15}%
  \RangeCatcodeCheck{97}{122}{11}%
  \RangeCatcodeCheck{123}{255}{15}%
  \RestoreCatcodes
}
\Test
\csname @@end\endcsname
\end
%    \end{macrocode}
%    \begin{macrocode}
%</test1>
%    \end{macrocode}
%
%    \begin{macrocode}
%<*test2>
\NeedsTeXFormat{LaTeX2e}
\makeatletter
\catcode`\@=11 %
\def\RestoreCatcodes{}
\count@=0 %
\loop
  \edef\RestoreCatcodes{%
    \RestoreCatcodes
    \catcode\the\count@=\the\catcode\count@\relax
  }%
\ifnum\count@<255 %
  \advance\count@\@ne
\repeat

\def\RangeCatcodeInvalid#1#2{%
  \count@=#1\relax
  \loop
    \catcode\count@=15 %
  \ifnum\count@<#2\relax
    \advance\count@\@ne
  \repeat
}
\def\Test#1{%
  \RangeCatcodeInvalid{0}{47}%
  \RangeCatcodeInvalid{58}{64}%
  \RangeCatcodeInvalid{91}{96}%
  \RangeCatcodeInvalid{123}{255}%
  \catcode`\@=12 %
  \catcode`\\=0 %
  \catcode`\{=1 %
  \catcode`\}=2 %
  \catcode`\#=6 %
  \catcode`\[=12 %
  \catcode`\]=12 %
  \catcode`\%=14 %
  \catcode`\ =10 %
  \catcode13=5 %
  #1\relax
  \RestoreCatcodes
}
\Test{\RequirePackage{kvoptions-patch}}%
\Test{\RequirePackage{kvoptions}}%
\csname @@end\endcsname
%</test2>
%    \end{macrocode}
%
%    \begin{macrocode}
%<*test3>
\NeedsTeXFormat{LaTeX2e}
\makeatletter
\RequirePackage{kvoptions}[2016/05/16]
\def\msg#{\immediate\write16}
\define@key{testfamily}{testkey}{%
  \msg{[testfamily/testkey/#1]}%
}
\define@key{testfamily}{testdefaultkey}[testdefault]{%
  \msg{[testfamily/testdefaultkey/#1]}%
}
\AddToKeyvalOption{testfamily}{testkey}{%
  \msg{[addition/#1]}%
}
\AddToKeyvalOption{testfamily}{testdefaultkey}{%
  \msg{[addition/#1]}%
}
\setkeys{testfamily}{%
  testkey=testA,%
  testdefaultkey=testB,%
  testdefaultkey,%
}
\SetupKeyvalOptions{%
  family=testfamily%
}
\AddToKeyvalOption*{testkey}{%
  \msg{[star addition/#1]}%
}
\AddToKeyvalOption*{testdefaultkey}{%
  \msg{[star addition/#1]}%
}
\setkeys{testfamily}{%
  testkey=testA,%
  testdefaultkey=testB,%
  testdefaultkey,%
}
\@@end
%</test3>
%    \end{macrocode}
%
%    \begin{macrocode}
%<*test4pkg>
\NeedsTeXFormat{LaTeX2e}
\ProvidesPackage{kvoptions-test4}[2016/05/16 package for testing]
\RequirePackage{kvoptions}[2016/05/16]
\SetupKeyvalOptions{%
  family=FOO,%
  prefix=foo,%
  setkeys=\kvsetkeys,%
}
\DeclareStringOption{str}
\define@key{FOO}{set}{%
  \setkeys{BAR}{strbar={#1}}%
}
\define@key{BAR}{strbar}{%
  \def\foostr{[BAR:#1]}%
}
\ProcessKeyvalOptions*
%</test4pkg>
%    \end{macrocode}
%    \begin{macrocode}
%<*test4>
\NeedsTeXFormat{LaTeX2e}
\ProvidesFile{kvoptions-test4.tex}[2016/05/16 test file]
\RequirePackage[%
  str=A,set=B,str=C,%
]{kvoptions-test4}[2016/05/16]
\def\TestExpected{C}
\ifx\foostr\TestExpected
  \typeout{* Test ok.}%
\else
  \typeout{* Result: [\foostr]}%
  \typeout{* Expected: [\TestExpected]}%
  \errmessage{Test failed!}%
\fi
\csname @@end\endcsname\end
%</test4>
%    \end{macrocode}
%
% \section{Installation}
%
% \subsection{Download}
%
% \paragraph{Package.} This package is available on
% CTAN\footnote{\CTANpkg{kvoptions}}:
% \begin{description}
% \item[\CTAN{macros/latex/contrib/oberdiek/kvoptions.dtx}] The source file.
% \item[\CTAN{macros/latex/contrib/oberdiek/kvoptions.pdf}] Documentation.
% \end{description}
%
%
% \paragraph{Bundle.} All the packages of the bundle `oberdiek'
% are also available in a TDS compliant ZIP archive. There
% the packages are already unpacked and the documentation files
% are generated. The files and directories obey the TDS standard.
% \begin{description}
% \item[\CTANinstall{install/macros/latex/contrib/oberdiek.tds.zip}]
% \end{description}
% \emph{TDS} refers to the standard ``A Directory Structure
% for \TeX\ Files'' (\CTAN{tds/tds.pdf}). Directories
% with \xfile{texmf} in their name are usually organized this way.
%
% \subsection{Bundle installation}
%
% \paragraph{Unpacking.} Unpack the \xfile{oberdiek.tds.zip} in the
% TDS tree (also known as \xfile{texmf} tree) of your choice.
% Example (linux):
% \begin{quote}
%   |unzip oberdiek.tds.zip -d ~/texmf|
% \end{quote}
%
% \paragraph{Script installation.}
% Check the directory \xfile{TDS:scripts/oberdiek/} for
% scripts that need further installation steps.
%
% \subsection{Package installation}
%
% \paragraph{Unpacking.} The \xfile{.dtx} file is a self-extracting
% \docstrip\ archive. The files are extracted by running the
% \xfile{.dtx} through \plainTeX:
% \begin{quote}
%   \verb|tex kvoptions.dtx|
% \end{quote}
%
% \paragraph{TDS.} Now the different files must be moved into
% the different directories in your installation TDS tree
% (also known as \xfile{texmf} tree):
% \begin{quote}
% \def\t{^^A
% \begin{tabular}{@{}>{\ttfamily}l@{ $\rightarrow$ }>{\ttfamily}l@{}}
%   kvoptions.sty & tex/latex/oberdiek/kvoptions.sty\\
%   kvoptions-patch.sty & tex/latex/oberdiek/kvoptions-patch.sty\\
%   kvoptions.pdf & doc/latex/oberdiek/kvoptions.pdf\\
%   example-mycolorsetup.sty & doc/latex/oberdiek/example-mycolorsetup.sty\\
%   test/kvoptions-test1.tex & doc/latex/oberdiek/test/kvoptions-test1.tex\\
%   test/kvoptions-test2.tex & doc/latex/oberdiek/test/kvoptions-test2.tex\\
%   test/kvoptions-test3.tex & doc/latex/oberdiek/test/kvoptions-test3.tex\\
%   test/kvoptions-test4.tex & doc/latex/oberdiek/test/kvoptions-test4.tex\\
%   test/kvoptions-test4.sty & doc/latex/oberdiek/test/kvoptions-test4.sty\\
%   kvoptions.dtx & source/latex/oberdiek/kvoptions.dtx\\
% \end{tabular}^^A
% }^^A
% \sbox0{\t}^^A
% \ifdim\wd0>\linewidth
%   \begingroup
%     \advance\linewidth by\leftmargin
%     \advance\linewidth by\rightmargin
%   \edef\x{\endgroup
%     \def\noexpand\lw{\the\linewidth}^^A
%   }\x
%   \def\lwbox{^^A
%     \leavevmode
%     \hbox to \linewidth{^^A
%       \kern-\leftmargin\relax
%       \hss
%       \usebox0
%       \hss
%       \kern-\rightmargin\relax
%     }^^A
%   }^^A
%   \ifdim\wd0>\lw
%     \sbox0{\small\t}^^A
%     \ifdim\wd0>\linewidth
%       \ifdim\wd0>\lw
%         \sbox0{\footnotesize\t}^^A
%         \ifdim\wd0>\linewidth
%           \ifdim\wd0>\lw
%             \sbox0{\scriptsize\t}^^A
%             \ifdim\wd0>\linewidth
%               \ifdim\wd0>\lw
%                 \sbox0{\tiny\t}^^A
%                 \ifdim\wd0>\linewidth
%                   \lwbox
%                 \else
%                   \usebox0
%                 \fi
%               \else
%                 \lwbox
%               \fi
%             \else
%               \usebox0
%             \fi
%           \else
%             \lwbox
%           \fi
%         \else
%           \usebox0
%         \fi
%       \else
%         \lwbox
%       \fi
%     \else
%       \usebox0
%     \fi
%   \else
%     \lwbox
%   \fi
% \else
%   \usebox0
% \fi
% \end{quote}
% If you have a \xfile{docstrip.cfg} that configures and enables \docstrip's
% TDS installing feature, then some files can already be in the right
% place, see the documentation of \docstrip.
%
% \subsection{Refresh file name databases}
%
% If your \TeX~distribution
% (\TeX\,Live, \mikTeX, \dots) relies on file name databases, you must refresh
% these. For example, \TeX\,Live\ users run \verb|texhash| or
% \verb|mktexlsr|.
%
% \subsection{Some details for the interested}
%
% \paragraph{Unpacking with \LaTeX.}
% The \xfile{.dtx} chooses its action depending on the format:
% \begin{description}
% \item[\plainTeX:] Run \docstrip\ and extract the files.
% \item[\LaTeX:] Generate the documentation.
% \end{description}
% If you insist on using \LaTeX\ for \docstrip\ (really,
% \docstrip\ does not need \LaTeX), then inform the autodetect routine
% about your intention:
% \begin{quote}
%   \verb|latex \let\install=y\input{kvoptions.dtx}|
% \end{quote}
% Do not forget to quote the argument according to the demands
% of your shell.
%
% \paragraph{Generating the documentation.}
% You can use both the \xfile{.dtx} or the \xfile{.drv} to generate
% the documentation. The process can be configured by the
% configuration file \xfile{ltxdoc.cfg}. For instance, put this
% line into this file, if you want to have A4 as paper format:
% \begin{quote}
%   \verb|\PassOptionsToClass{a4paper}{article}|
% \end{quote}
% An example follows how to generate the
% documentation with pdf\LaTeX:
% \begin{quote}
%\begin{verbatim}
%pdflatex kvoptions.dtx
%makeindex -s gind.ist kvoptions.idx
%pdflatex kvoptions.dtx
%makeindex -s gind.ist kvoptions.idx
%pdflatex kvoptions.dtx
%\end{verbatim}
% \end{quote}
%
% \begin{thebibliography}{99}
% \bibitem{tb94wright}
%   A guide to key-value methods, Joseph Wright, second draft for
%   \href{https://www.tug.org/tugboat}{TUGBoat}, 2009-03-17.
%   \url{https://www.texdev.net/uploads/2009/03/keyval.pdf}
% \bibpackage{ifthen}{David Carlisle}{2001/05/26}
%   {\CTANpkg{ifthen}}
% \bibpackage{helvet}{Sebastian Rahtz, Walter Schmidt}{2004/01/26}
%   {\CTANpkg{psfonts}}
% \bibpackage{hyperref}{Sebastian Rahtz, Heiko Oberdiek}{2006/02/12}
%   {\CTANpkg{hyperref}}
% \bibpackage{keyval}{David Carlisle}{1999/03/16}
%   {\CTANpkg{keyval}}
% \bibpackage{multicol}{Frank Mittelbach}{2004/02/14}
%   {\CTANpkg{multicol}}
% \bibpackage{tabularx}{David Carlisle}{1999/01/07}
%   {\CTANpkg{tabularx}}
% \bibpackage{tracefnt}{Frank Mittelbach, Rainer Sch\"opf}{1997/05/29}
%   {\CTANpkg{latex-base}}
% \bibpackage{xkeyval}{Hendri Adriaens}{2005/05/07}
%   {\CTANpkg{xkeyval}}
% \bibitem{clsguide}
%   The \LaTeX3 Project, \textit{\LaTeXe\ for class and package writers},
%   2003/12/09.
%   \CTANpkg{clsguide}
%
% \end{thebibliography}
%
% \begin{History}
%   \begin{Version}{0000/00/00 v0.0}
%   \item
%     Probably David Carlisle's code in \xpackage{hyperref}
%     was the start.
%   \end{Version}
%   \begin{Version}{2004/02/22 v1.0}
%   \item
%     The first version was never published. It also
%     has offered a patch to get rid of \LaTeX's option
%     expansion.
%   \end{Version}
%   \begin{Version}{2006/02/16 v2.0}
%   \item
%     Now the package is redesigned with an easier
%     user interface.
%   \item
%     \cs{ProcessKeyvalOptions} remains the central service, inherited
%     from \xpackage{hyperref}'s \cs{ProcessOptionsWithKV}.
%     Now the use inside classes is also supported.
%   \item
%     Provides help macros for boolean and simple string options.
%   \item
%     Fixes for the patch of \LaTeX{}. The patch is only enabled,
%     if the user requests it.
%   \end{Version}
%   \begin{Version}{2006/02/20 v2.1}
%   \item
%     Unused option list is sanitized to prevent problems
%     with other packages that uses own processing methods
%     for key value options. Disadvantage: the unused global
%     option detection is weakened.
%   \item
%     New option type by \cs{DeclareVoidOption} for options without
%     value.
%   \item
%     Default rule by \cs{DeclareDefaultOption}.
%   \item
%     Dynamic options: \cs{DisableKeyvalOption}.
%   \end{Version}
%   \begin{Version}{2006/06/01 v2.2}
%   \item
%     Fixes for option \xoption{patch}.
%   \end{Version}
%   \begin{Version}{2006/08/17 v2.3}
%   \item
%     \cs{DeclareBooleanOption} renamed to \cs{DeclareBoolOption}
%     to avoid a name clash with package \cs{ifoption}.
%   \end{Version}
%   \begin{Version}{2006/08/22 v2.4}
%   \item
%     Option \xoption{patch}: \cs{ExecuteOptions} does not
%     change the meaning of macro \cs{CurrentOption} at all.
%   \end{Version}
%   \begin{Version}{2007/04/11 v2.5}
%   \item
%     Line ends sanitized.
%   \end{Version}
%   \begin{Version}{2007/05/06 v2.6}
%   \item
%     Uses package \xpackage{etexcmds}.
%   \end{Version}
%   \begin{Version}{2007/06/11 v2.7}
%   \item
%     The patch part fixes LaTeX bug latex/3965.
%   \end{Version}
%   \begin{Version}{2007/10/02 v2.8}
%   \item
%     Compatibility for \plainTeX\ added.
%   \item
%     Typos in documentation fixed (Axel Sommerfeldt).
%   \end{Version}
%   \begin{Version}{2007/10/11 v2.9}
%   \item
%     Bug fix for option \xoption{patch}.
%   \end{Version}
%   \begin{Version}{2007/10/18 v3.0}
%   \item
%     New package \xpackage{kvoptions-patch}.
%   \end{Version}
%   \begin{Version}{2009/04/10 v3.1}
%   \item
%     Space by line end removed in definition of internal macro.
%   \end{Version}
%   \begin{Version}{2009/07/17 v3.2}
%   \item
%     \cs{ProcessLocalKeyvalOptions} added.
%   \item
%     \cs{DisableKeyvalOption} with the \texttt{action=ignore} option
%     fixed (Joseph Wright).
%   \end{Version}
%   \begin{Version}{2009/07/21 v3.3}
%   \item
%     \cs{DeclareLocalOption}, \cs{DeclareLocalOptions} added.
%   \end{Version}
%   \begin{Version}{2009/08/13 v3.4}
%   \item
%     Documentation addition: recommendation for Joseph Wright's
%     review article.
%   \item
%     Documentation addition: local/global options.
%   \end{Version}
%   \begin{Version}{2009/12/04 v3.5}
%   \item
%     \cs{AddToKeyvalOption} added.
%   \end{Version}
%   \begin{Version}{2009/12/08 v3.6}
%   \item
%     Fix: If a default handler is configured, it is now also
%     called for classes.
%   \end{Version}
%   \begin{Version}{2010/02/22 v3.7}
%   \item
%     Missing space in error message added.
%   \end{Version}
%   \begin{Version}{2010/07/23 v3.8}
%   \item
%     Documenation for package \xpackage{kvoptions-patch} improved.
%     No code changes.
%   \end{Version}
%   \begin{Version}{2010/12/02 v3.9}
%   \item
%     Key \texttt{setkeys} added for \cs{SetupKeyvalOptions}.
%   \end{Version}
%   \begin{Version}{2010/12/23 v3.10}
%   \item
%     \cs{DeclareVoidOption} also parses the second parameter as
%     \hologo{TeX} argument to improve compatibility with
%     \cs{DeclareOption}.
%   \end{Version}
%   \begin{Version}{2011/06/30 v3.11}
%   \item
%     Fix because of design bug in package \xpackage{xkeyval} that
%     removes global options with equal signs.
%   \end{Version}
%   \begin{Version}{2016/05/16 v3.12}
%   \item
%     Documentation updates.
%   \end{Version}
% \end{History}
%
% \PrintIndex
%
% \Finale
\endinput

%        (quote the arguments according to the demands of your shell)
%
% Documentation:
%    (a) If kvoptions.drv is present:
%           latex kvoptions.drv
%    (b) Without kvoptions.drv:
%           latex kvoptions.dtx; ...
%    The class ltxdoc loads the configuration file ltxdoc.cfg
%    if available. Here you can specify further options, e.g.
%    use A4 as paper format:
%       \PassOptionsToClass{a4paper}{article}
%
%    Programm calls to get the documentation (example):
%       pdflatex kvoptions.dtx
%       makeindex -s gind.ist kvoptions.idx
%       pdflatex kvoptions.dtx
%       makeindex -s gind.ist kvoptions.idx
%       pdflatex kvoptions.dtx
%
% Installation:
%    TDS:tex/latex/oberdiek/kvoptions.sty
%    TDS:tex/latex/oberdiek/kvoptions-patch.sty
%    TDS:doc/latex/oberdiek/kvoptions.pdf
%    TDS:doc/latex/oberdiek/example-mycolorsetup.sty
%    TDS:doc/latex/oberdiek/test/kvoptions-test1.tex
%    TDS:doc/latex/oberdiek/test/kvoptions-test2.tex
%    TDS:doc/latex/oberdiek/test/kvoptions-test3.tex
%    TDS:doc/latex/oberdiek/test/kvoptions-test4.tex
%    TDS:doc/latex/oberdiek/test/kvoptions-test4.sty
%    TDS:source/latex/oberdiek/kvoptions.dtx
%
%<*ignore>
\begingroup
  \catcode123=1 %
  \catcode125=2 %
  \def\x{LaTeX2e}%
\expandafter\endgroup
\ifcase 0\ifx\install y1\fi\expandafter
         \ifx\csname processbatchFile\endcsname\relax\else1\fi
         \ifx\fmtname\x\else 1\fi\relax
\else\csname fi\endcsname
%</ignore>
%<*install>
\input docstrip.tex
\Msg{************************************************************************}
\Msg{* Installation}
\Msg{* Package: kvoptions 2016/05/16 v3.12 Key value format for package options (HO)}
\Msg{************************************************************************}

\keepsilent
\askforoverwritefalse

\let\MetaPrefix\relax
\preamble

This is a generated file.

Project: kvoptions
Version: 2016/05/16 v3.12

Copyright (C)
   2004, 2006, 2007, 2009-2011 Heiko Oberdiek
   2016-2019 Oberdiek Package Support Group

This work may be distributed and/or modified under the
conditions of the LaTeX Project Public License, either
version 1.3c of this license or (at your option) any later
version. This version of this license is in
   https://www.latex-project.org/lppl/lppl-1-3c.txt
and the latest version of this license is in
   https://www.latex-project.org/lppl.txt
and version 1.3 or later is part of all distributions of
LaTeX version 2005/12/01 or later.

This work has the LPPL maintenance status "maintained".

The Current Maintainers of this work are
Heiko Oberdiek and the Oberdiek Package Support Group
https://github.com/ho-tex/oberdiek/issues


This work consists of the main source file kvoptions.dtx
and the derived files
   kvoptions.sty, kvoptions.pdf, kvoptions.ins, kvoptions.drv,
   kvoptions-patch.sty, example-mycolorsetup.sty,
   kvoptions-test1.tex, kvoptions-test2.tex,
   kvoptions-test3.tex, kvoptions-test4.tex,
   kvoptions-test4.sty.

\endpreamble
\let\MetaPrefix\DoubleperCent

\generate{%
  \file{kvoptions.ins}{\from{kvoptions.dtx}{install}}%
  \file{kvoptions.drv}{\from{kvoptions.dtx}{driver}}%
  \usedir{tex/latex/oberdiek}%
  \file{kvoptions.sty}{\from{kvoptions.dtx}{package}}%
  \file{kvoptions-patch.sty}{\from{kvoptions.dtx}{patch}}%
  \usedir{doc/latex/oberdiek}%
  \file{example-mycolorsetup.sty}{\from{kvoptions.dtx}{example}}%
%  \usedir{doc/latex/oberdiek/test}%
%  \file{kvoptions-test1.tex}{\from{kvoptions.dtx}{test1}}%
%  \file{kvoptions-test2.tex}{\from{kvoptions.dtx}{test2}}%
%  \file{kvoptions-test3.tex}{\from{kvoptions.dtx}{test3}}%
%  \file{kvoptions-test4.tex}{\from{kvoptions.dtx}{test4}}%
%  \file{kvoptions-test4.sty}{\from{kvoptions.dtx}{test4pkg}}%
  \nopreamble
  \nopostamble
%  \usedir{source/latex/oberdiek/catalogue}%
%  \file{kvoptions.xml}{\from{kvoptions.dtx}{catalogue}}%
}

\catcode32=13\relax% active space
\let =\space%
\Msg{************************************************************************}
\Msg{*}
\Msg{* To finish the installation you have to move the following}
\Msg{* files into a directory searched by TeX:}
\Msg{*}
\Msg{*     kvoptions.sty, kvoptions-patch.sty}
\Msg{*}
\Msg{* To produce the documentation run the file `kvoptions.drv'}
\Msg{* through LaTeX.}
\Msg{*}
\Msg{* Happy TeXing!}
\Msg{*}
\Msg{************************************************************************}

\endbatchfile
%</install>
%<*ignore>
\fi
%</ignore>
%<*driver>
\NeedsTeXFormat{LaTeX2e}
\ProvidesFile{kvoptions.drv}%
  [2016/05/16 v3.12 Key value format for package options (HO)]%
\documentclass{ltxdoc}
\usepackage{holtxdoc}[2011/11/22]
\begin{document}
  \DocInput{kvoptions.dtx}%
\end{document}
%</driver>
% \fi
%
%
% \CharacterTable
%  {Upper-case    \A\B\C\D\E\F\G\H\I\J\K\L\M\N\O\P\Q\R\S\T\U\V\W\X\Y\Z
%   Lower-case    \a\b\c\d\e\f\g\h\i\j\k\l\m\n\o\p\q\r\s\t\u\v\w\x\y\z
%   Digits        \0\1\2\3\4\5\6\7\8\9
%   Exclamation   \!     Double quote  \"     Hash (number) \#
%   Dollar        \$     Percent       \%     Ampersand     \&
%   Acute accent  \'     Left paren    \(     Right paren   \)
%   Asterisk      \*     Plus          \+     Comma         \,
%   Minus         \-     Point         \.     Solidus       \/
%   Colon         \:     Semicolon     \;     Less than     \<
%   Equals        \=     Greater than  \>     Question mark \?
%   Commercial at \@     Left bracket  \[     Backslash     \\
%   Right bracket \]     Circumflex    \^     Underscore    \_
%   Grave accent  \`     Left brace    \{     Vertical bar  \|
%   Right brace   \}     Tilde         \~}
%
% \GetFileInfo{kvoptions.drv}
%
% \title{The \xpackage{kvoptions} package}
% \date{2016/05/16 v3.12}
% \author{Heiko Oberdiek\thanks
% {Please report any issues at \url{https://github.com/ho-tex/oberdiek/issues}}}
%
% \maketitle
%
% \begin{abstract}
% This package is intended for package authors who want to
% use options in key value format for their package options.
% \end{abstract}
%
% \tableofcontents
%
% \def\M#1{\texttt{\{}\meta{#1}\texttt{\}}}
% \def\O#1{\texttt{[}\meta{#1}\texttt{]}}
%
% \section{Introduction}
%
% First I want to recommend the very good review article
% ``A guide to key-value methods'' by Joseph Wright \cite{tb94wright}.
% It introduces the different key-value packages and compares them.
%
% \subsection{The beginning}
%
% This package \xpackage{kvoptions} addresses class or package writers
% that want to allow their users to specify options
% as key value pairs, e.g.:
% \begin{quote}
% |\documentclass[verbose=false,name=me]{myclass}|\\
% |\usepackage[format=print]{mylayout}|
% \end{quote}
% Prominent example is package \xpackage{hyperref},
% probably the first package that offers this service.
% It's \cs{ProcessOptionsWithKV} is often copied und
% used in other packages, e.g. package \xpackage{helvet}
% that uses this interface for its option \xoption{scaled}.
%
% However copying code is not the most modern software
% development technique. And \xpackage{hyperref}'s
% code for \cs{ProcessOptionsWithKV} was changed
% to fix bugs. The version used in other packages
% depends on the time of copying and the awareness
% of \xpackage{hyperref}'s changes.
% Now the code is sourced out into this package and
% available for other package or class writers.
%
% \subsection{Overview}
%
% Package \xpackage{kvoptions} connects package
% \xpackage{\emph{k}ey\emph{v}al}
% with \LaTeX's package and class \emph{options}:
% \begin{center}
% \def\T#1{\bfseries #1}
% \renewcommand{\arraystretch}{1.5}
% \begin{tabular}{l|l|l}
%   \T{Package \xpackage{keyval}}
%   & \T{Package \xpackage{kvoptions}}
%   & \T{\LaTeX\ kernel}
%   \\
% \hline
%     \cs{define@key}
%   &
%     \renewcommand*{\arraystretch}{1}^^A
%     \begin{tabular}[t]{@{}l@{}}
%       \cs{DeclareVoidOption}\\
%       \cs{DeclareStringOption}\\
%       \cs{DeclareBoolOption}\\
%       \cs{DeclareComplementaryOption}\\
%       \cs{DisableKeyvalOption}
%     \end{tabular}
%   & \cs{DeclareOption}
%   \\
% \hline
%   & \cs{DeclareDefaultOption}
%   & \cs{DeclareOption*}
%   \\
% \hline
%   & \cs{ProcessKeyvalOptions}
%   & \cs{ProcessOptions*}
%   \\
% \hline
%   & Option \xoption{patch}
%   &
%     \renewcommand*{\arraystretch}{1}^^A
%     \begin{tabular}[t]{@{}l@{}}
%       Class/package\\
%       option system
%     \end{tabular}
%   \\
% \hline
%   & \cs{SetupKeyvalOptions} &
%   \\
% \end{tabular}
% \end{center}
%
% \hypersetup{bookmarksopenlevel=3}
% \section{Usage}
%
% \subsection{Process options}
%
% \subsubsection{\cs{ProcessKeyvalOptions}}
%
% \begin{declcs}{ProcessKeyvalOptions}\,\M{family}\\
%   \cs{ProcessKeyvalOptions} \*
% \end{declcs}
% This command evaluates the global or local options of
% the package that are defined with \xpackage{keyval}'s
% interface within the family \meta{family}. It acts the
% same way as \LaTeX's \cs{ProcessOptions*}. In a package
% unknown global options are ignored, in a class they
% are added to the unknown option list. The known global
% options and all local options are passed to \xpackage{keyval}'s
% \cs{setkeys} command for executing the options.
% Unknown options are reported to the user by an error.
%
% If the family name happens to be the same as the name
% of the package or class where \cs{ProcessKeyvalOptions}
% is used or the family name has previously been setup by
% \cs{SetupKeyvalOptions}, then \cs{ProcessKeyvalOptions}
% knows the family name already and you can use the star form
% without mandatory argument.
%
% \subsubsection{\cs{ProcessLocalKeyvalOptions}}
%
% \begin{declcs}{ProcessLocalKeyvalOptions}\,\M{family}\\
%   \cs{ProcessLocalKeyvalOptions} \*
% \end{declcs}
% This macro has the same syntax and works similar as
% \cs{ProcessKeyvalOptions}. However it ignores
% global options and only processes the local package options.
% Therefore it only can be used inside a package.
% An error is thrown, if it is used inside a class.
%
% Neither of the following macros are necessary for
% \cs{ProcessKeyvalOptions{}}. They just help the package/class
% author in common tasks.
%
% \subsubsection{\cs{SetupKeyvalOptions}}\label{setup}
%
% \begin{declcs}{SetupKeyvalOptions}\,\texttt{\{}\\
%   \hspace{1.5em}|family|\,|=|\,\meta{family}|,|\\
%   \hspace{1.5em}|prefix|\,|=|\,\meta{prefix}\\
%   \hspace{1.5em}|setkeys|\,|=|\,\meta{setkeys command}\\
%   \texttt{\}}
% \end{declcs}
% This command allows to configure the default assumptions
% that are based on the current package or class name.
% \LaTeX\ remembers this name in \cs{@currname}. The syntax
% description of the default looks a little weird, therefor
% an example is given for a package or class named |foobar|.
% \begin{quote}
% \begin{tabular}{@{}l|ll|l@{}}
%   Key & Default & (example) & Used by\\
% \hline
%   \rule{0pt}{1.2\ht\csname @arstrutbox\endcsname}^^A
%   |family| & \meta{\cs{@currname}} & (|foobar|)
%       & \cs{ProcessKeyvalOptions*} \\
%     &&& \cs{DeclareBoolOption} \\
%     &&& \cs{DeclareStringOption} \\
% \hline
%   \rule{0pt}{1.2\ht\csname @arstrutbox\endcsname}^^A
%   |prefix| & \meta{\cs{@currname}}|@| & (|foobar@|)
%       & \cs{DeclareBoolOption} \\
%     &&& \cs{DeclareStringOption} \\
%     &&& \cs{DeclareVoidOption} \\
% \hline
%   \rule{0pt}{1.2\ht\csname @arstrutbox\endcsname}^^A
%   |setkeys| & \cs{setkeys} & (\cs{kvsetkeys})
%       & \cs{ProcessKeyvalOptions} \\
%     &&& \cs{ProcessLocalKeyvalOptions}
% \end{tabular}
% \end{quote}
% Key |setkeys| was added in version 3.9. The original
% \cs{setkeys} of package \xpackage{keyval} is not reentrant.
% If an option is processed by this \cs{setkeys}, then
% the option should not call \cs{setkeys} again with a
% different family. Otherwise the next options of the
% first \cs{setkeys} call are processed with the wrong
% family. With key |setkeys| the macro \cs{kvsetkeys}
% can be set that does not have the problem of the
% original \cs{setkeys} of package {keyval}.
%
% Probably \cs{setkeys} of package \xpackage{xkeyval}
% is safe in this respect. But I haven't made a full
% analysis. At least it does not have the problem
% of the original \cs{setkeys}.
%
% \subsection{Option declarations}
%
% The options for \cs{ProcessKeyvalOptions} are defined
% by \xpackage{keyval}'s \cs{define@key}.
% Common purposes of such keys are boolean switches, they
% enable or disable something. Or they store a name or some kind of
% string in a macro. The following commands help the user. He
% declares what he wants and \xpackage{kvoptions} take care of
% the key definition, resource allocation and initialization.
%
% In order to avoid name clashes of macro names, internal
% commands are prefixed. Both the prefix and the family
% name for the defined keys can be configured by
% \cs{SetupKeyvalOptions}.
%
% \subsubsection{\cs{DeclareStringOption}}
%
% \begin{declcs}{DeclareStringOption}\,\O{init}\,\M{key}\,\O{default}
% \end{declcs}
% A macro is created that remembers the value of the key \meta{key}.
% The name of the macro consists of the option name \meta{key} that
% is prefixed by the prefix (see \ref{setup}).
% The initial contents of the
% macro can be given by the first optional argument \meta{init}.
% The default is empty.
%
% The the option \meta{key} is defined. The option code
% just stores its value in the macro. If the optional argument
% at the end of \cs{DeclareStringOption} is given, then option
% \meta{key} is defined with the default \meta{default}.
%
% Example for a package with the following two lines:
% \begin{quote}
% |\ProvidesPackage{foobar}|\\
% |\DeclareStringOption[me]{name}|
% \end{quote}
% Then \cs{DeclareStringOption} defines the macro with content |me|,
% note \LaTeX\ complains if the name of the macro already exists:
% \begin{quote}
% |\newcommand*{\foobar@name}{me}|
% \end{quote}
% The option definition is similar to:
% \begin{quote}
% |\define@key{foobar}{name}{%|\\
% |  \renewcommand*{\foobar@name}{#1}%|\\
% |}|
% \end{quote}
%
% \subsubsection{\cs{DeclareBoolOption}}
%
% \begin{declcs}{DeclareBoolOption}\,\O{init}\,\M{key}
% \end{declcs}
% A boolean switch is generated, initialized by value \meta{init}
% and the corresponding key \meta{key} is defined. If the
% initialization value is not given, |false| is used as default.
%
% The internal actions of \cs{DeclareBoolOption} are shown below.
% The example is given for a package author who has the following
% two lines in his package/class:
% \begin{quote}
% |\ProvidesPackage{foobar}|\\
% |\DeclareBoolOption{verbose}|
% \end{quote}
% First a new switch is created:
% \begin{quote}
%   |\newif\iffoobar@verbose|
% \end{quote}
% and initialized:
% \begin{quote}
%   |\foobar@verbosefalse|
% \end{quote}
% Finally the key is defined:
% \begin{quote}
%   |\define@key{foobar}{verbose}[true]{|\dots|}|
% \end{quote}
% The option code configures the boolean option in the following way:
% If the author specifies |true| or |false| then the switch is
% turned on or off respectivly. Also the option can be given without
% explicit value. Then the switch is enabled.
% Other values are reported as errors.
%
% Now the switch is ready to use in the package/class, e.g.:
% \begin{quote}
%   |\iffoobar@verbose|\\
%   \qquad\textit{\% print verbose message}\\
%   |\else|\\
%   \qquad\textit{\% be quiet}\\
%   |\fi|
% \end{quote}
% Users of package \cs{ifthen} can use the switch as
% boolean:
% \begin{quote}
%   |\boolean{foobar@verbose}|
% \end{quote}
%
% \subsubsection{\cs{DeclareComplementaryOption}}
%
% \begin{declcs}{DeclareComplementaryOption}\,\M{key}\,\M{parent}
% \end{declcs}
% Sometimes contrasting names are used to characterize the
% two states of a boolean switch, for example \xoption{draft}
% vs.\@ \xoption{final}. Both options behave like boolean
% options but they do not need two different switches, they
% should share one. \cs{DeclareComplementaryOption} allows this.
% The option \meta{key} shares the switch of option \meta{parent}.
% Example:
% \begin{quote}
%   |\DeclareBoolOption{draft}|\\
%   |\DeclareComplementaryOption{final}{draft}|
% \end{quote}
% Then |final| sets the switch of \xoption{draft} to |false|, and
% |final=false| enables the \xoption{draft} switch.
%
% \subsubsection{\cs{DeclareVoidOption}}
%
% \begin{declcs}{DeclareVoidOption}\,\M{key}\,\M{code}
% \end{declcs}
% \cs{ProcessKeyvalOptions} can be extended to recognize
% options that are declared in traditional way by \cs{DeclareOption}.
% But in case of the error that the user specifies a value, then
% this option would not recognized as key value option because
% of \cs{DeclareOption} and not detected as traditional option
% because of the value part. The user would get an unknown option
% error, difficult to understand.
%
% \cs{DeclareVoidOption} solves
% this problem. It defines the option \meta{key} as key value option.
% If the user specifies a value, a warning is given and the
% value is ignored.
%
% The code part \meta{code} is stored in a macro.
% The name of the macro consists of the option name \meta{key} that
% is prefixed by the prefix (see \ref{setup}).
% If the option is set, the macro will be executed. During the
% execution \cs{CurrentOption} is available with the current key name.
%
% \subsubsection{\cs{DeclareDefaultOption}}
%
% \begin{declcs}{DeclareDefaultOption}\,\M{code}
% \end{declcs}
% \begin{sloppypar}
% This command does not define a specific key, it is the equivalent
% to \LaTeX's \cs{DeclareOption*}. It allows the specification
% of a default action \meta{code} that is invoked if an unknown option
% is found.
% While \meta{code} is called,
% macro \cs{CurrentOption} contains the current option string.
% In addition \cs{CurrentOptionValue} contains the value part
% if the option string is parsable as key value pair, otherwise
% it is \cs{relax}. \cs{CurrentOptionKey} contains the key of
% the key value pair, or the whole option string, if it misses
% the equal sign.
% \end{sloppypar}
%
% Inside packages typical default actions are to pass unknown
% options to another package. Or an error message can be thrown
% by \cs{@unknownoptionerror}. This is the original error
% message that \LaTeX\ gives for unkown package options.
% This error message is easier to understand for the user
% as the error message from package \xpackage{keyval} that
% is given otherwise.
%
% A Class ignores unknown options and puts them on the
% unused option list. Let \LaTeX\ do the job and just
% call \cs{OptionNotUsed}. Or the options can be passed to
% another class that is later loaded.
%
% \subsubsection{Local options}
%
% \begin{declcs}{DeclareLocalOption} \M{option}\\
%   \cs{DeclareLocalOptions} \M{option list}
% \end{declcs}
% Both macros mark package options as local options. That means
% that they are ignored by \cs{ProcessKeyvalOptions} if they are given
% as global options. \cs{DeclareLocalOptions} takes one option,
% \cs{DeclareLocalOptions} expects a comma separated list of options.
%
% \subsubsection{Dynamic options}
%
% Options of \LaTeX's package/class system are cleared
% in \cs{ProcessOptions}. They modify the static model
% of a package. For example, depending on option \xoption{bookmarks}
% package \xpackage{hyperref} loads differently.
%
% Options, however,
% defined by \xpackage{keyval}'s \cs{define@key} remain
% defined, if the options are processed by \cs{setkeys}.
% Therefore these options can also be used to model
% the dynamic behaviour of a package. For example, in
% \xpackage{hyperref} the link colors can be changed
% everywhere until the end in |\end{document}|.
%
% However package \xpackage{color} that adds color support
% is necessary and it
% cannot be loaded after |\begin{document}|. Option \xoption{colorlinks}
% that loads \xpackage{color} should be active until |\begin{document}|
% and die in some way if it is too late for loading packages.
% With \cs{DisableKeyvalOption} the package/class author can
% specify and configure the death of an option and controls
% the life period of the option.
%
% \subsubsection{\cs{DisableKeyvalOption}}
%
% \begin{declcs}{DisableKeyvalOption}\,\O{options}\,\M{family}\,\M{key}
%   \\[0.5ex]
%   \meta{options}:\\
%   \hspace{1.5em}^^A
%   \begin{tabular}[t]{@{}l@{ \texttt{=} }l@{\qquad}>{default: }l@{}}
%     |action| & |undef|, |warning|, |error|, or |ignore|
%     & |undef|
%     \\
%     \multicolumn{1}{@{}l}{|global| or |local|} & & |global|
%     \\
%     |package| or |class| & \meta{name}
%     \\
%   \end{tabular}^^A
% \end{declcs}
% \cs{DisableKeyvalOption} can be called to mark the end when
% the option \meta{key} is no longer useful. The behaviour of
% an option after its death can be configured by \xoption{action}:
% \begin{description}
% \item[\texttt{undef}:] The option will be undefined,
%   If it is called, \cs{setkeys} reports an error because
%   of unknown key.
% \item[\texttt{error} or \texttt{warning}:] Use of the
%  option will cause an error or warning message.
%  Also these actions require that exclusivly either the
%  package or class name is given in options \xoption{package}
%  or \xoption{class}.
% \item[\texttt{ignore}:] The use of the option will silently
%  be ignored.
% \end{description}
% The option's death can be limited to the end of the current
% group, if option \xoption{local} is given. Default is
% \xoption{global}.
%
% The package/class author can wish the end of the option
% already during the package loading, then he will have
% static behaviour. In case of dynamic options \cs{DisableKeyvalOption}
% can be executed everywhere, also outside the package. Therefore
% the family name and the package/class name is usually unknown for
% \cs{DisableKeyvalOption}. Therefore the argument for the family
% name is mandatory and for some actions the package/class name
% must be provided.
%
% Usually a macro would configure the option death, Example:
%\begin{quote}
%\begin{verbatim}
%\ProvidesPackage{foobar}
%\DeclareBoolOption{color}
%\DeclareStringOption[red]{emphcolor}
%\ProcessKeyvalOptions*
%
%\newcommand*{\foobar@DisableOption}[2]{%
%  \DisableKeyvalueOption[
%    action={#1},
%    package=foobar
%  ]{foobar}{#2}%
%}
%
%\iffoobar@color
%  \RequirePackage{color}
%  \renewcommand*{\emph}[1]{\textcolor{\foobar@emphcolor}{#1}}
%\else
%  % Option emphcolor is not wrong, if we have color support.
%  % otherwise the option has no effect, but we don't want to
%  % remove it. Therefore action 'ignore' is the best choice:
%  \foobar@DisableOption{ignore}{emphcolor}
%\fi
%% No we don't need the option 'color'.
%\foobar@DisableOption{warning}{color}
%
%% With color support option 'emphcolor' will dynamically
%% change the color of \emph statements.
%\end{verbatim}
%\end{quote}
%
% \subsubsection{\cs{AddToKeyvalOption}}
%
% \begin{declcs}{AddToKeyvalOption}\,\M{family}\,\M{key}\,\M{code}\\
%   \cs{AddToKeyvalOption} \*\,\M{key}\,\M{code}
% \end{declcs}
% The code for an existing key \meta{key} of family \meta{family} is
% extended by code \meta{code}. In the starred form the current
% family setting is used, see \cs{ProcessKeyvalOptions*}.
%
% \subsection{Global vs. local options}
%
% Options that are given for \cs{documentclass} are called
% global options. They are known to the class and all packages.
% A package may make use of a global option and marks it as
% used. The advantage for the user is the freedom to specify
% options both in the \cs{documentclass} or \cs{usepackage}
% commands.
%
% However global options are shared with the class options
% and options of all other packages. Thus there can be the
% same option with different semantics for different
% packages and classes. As example, package \xpackage{bookmark}
% knows option \xoption{open} that specifies whether the bookmarks
% are opened or closed initially. It's values are |true| or |false|.
% Since KOMA-Script version 3.00 the KOMA classes also introduces
% option \xoption{open} with values |right| and |any| and a
% complete different meaning.
%
% Such conflicts can be resolved by marking all or part of options
% as local by \cs{DeclareLocalOption} or \cs{DeclareLocalOptions}.
% Then the packages ignores global occurences of these options.
% Package \xpackage{kvoptions} provides two methods:
% \begin{itemize}
% \item \cs{ProcessLocalKeyvalOptions} automatically uses all options
%   as local options. It ignores all global options.
% \item \cs{DeclareLocalOption} or \cs{DeclareLocalOptions} marks
%   options as local options. \cs{ProcessKeyvalOptions} will then
%   ignore global occurences for these local options.
% \end{itemize}
% Since version 1.5 package \xpackage{bookmark} uses the latter
% method. It checks global and local option places for driver options
% and limits all other options as local options. Thus the class
% option \xoption{open} of KOMA-Script is not misread as
% option for package \xpackage{bookmark}.
%
% \subsection{Summary of internal macros}
%
% The \cs{Declare}\texttt{\dots Option} commands define
% macros, additionally to the macros generated by the key definition.
% These macros can be used by the package/class author.
% The name of the macros starts with the prefix \meta{prefix}
% that can be configured by \cs{SetupKeyvalOptions}.
% \begin{center}
% \def\prefix{\cs{}\meta{prefix}\meta{key}}^^A
% \def\Hline{^^A
%   \hline
%   \rule{0pt}{1.2\ht\csname @arstrutbox\endcsname}^^A
% }^^A
% \begin{tabular}{@{}l|>{\ttfamily}l|l@{}}
%   Declare \meta{key} & \textrm{Defined macro} & Description\\
% \Hline
%   \cs{DeclareStringOption}
%    & \prefix & holds the string\\
% \Hline
%   \cs{DeclareBoolOption}
%     & \cs{if}\meta{prefix}\meta{key}& boolean switch\\
%   &\prefix false& disable switch\\
%   &\prefix true& enable switch\\
% \Hline
%   \cs{DeclareComplementaryOption}
%     & \prefix false& enable parent switch\\
%     & \prefix true& disable parent switch\\
% \Hline
%   \cs{DeclareVoidOption}
%   & \prefix & holds the action
% \end{tabular}
% \end{center}
%
% \subsection{\plainTeX}
%
% Package \xpackage{keyval} is also usable in \plainTeX\ with
% the help of file \xfile{miniltx.tex}. Some features of this
% package \xpackage{kvoptions} might also be useful for \plainTeX.
% If \LaTeX\ is not found, \cs{ProcessKeyvalOptions} and option
% \xoption{patch} are disabled. Before using the option declaration
% commands \cs{Declare...Option}, \cs{SetupKeyvalOptions} must be
% used.
%
% \hypersetup{bookmarksopenlevel=1}
%
% \section{Example}
%
% The following example defined a package that serves some
% private color management. A boolean option \xoption{print} enables
% print mode without colors. An option \xoption{emph} redefines
% \cs{emph} to print in the given color. And the driver
% can be specified by option \xoption{driver}.
%
%    \begin{macrocode}
%<*example>
    % Package identification
    % -----------------------
\NeedsTeXFormat{LaTeX2e}
\ProvidesPackage{example-mycolorsetup}[2016/05/16 Managing my colors]

\RequirePackage{ifpdf}
\RequirePackage{kvoptions}

    % Option declarations
    % -------------------

\SetupKeyvalOptions{
  family=MCS,
  prefix=MCS@
}
    % Use a shorter family name and prefix

    % Option print
\DeclareBoolOption{print}
    % is the same as
    % \DeclareBoolOption[false]{print}

    % Option driver
\ifpdf
  \DeclareStringOption[pdftex]{driver}
\else
  \DeclareStringOption[dvips]{driver}
\fi

    % Alternative interface for driver options
\DeclareVoidOption{dvips}{\SetupDriver}
\DeclareVoidOption{dvipdfm}{\SetupDriver}
\DeclareVoidOption{pdftex}{\SetupDriver}
    % In \SetupDriver we take the current option \CurrentOption
    % and pass it to the driver option.
    % The \expandafter commands expand \CurrentOption at the
    % time, when \SetupDriver is executed and \CurrentOption
    % has the correct meaning.
\newcommand*{\SetupDriver}{%
  \expandafter\@SetupDriver\expandafter{\CurrentOption}%
}
\newcommand*{\@SetupDriver}[1]{%
  \setkeys{MCS}{driver={#1}}%
}

    % Option emph
    % An empty value means, we want to have no color for \emph.
    % If the user specifies option emph without value, the red is used.
\DeclareStringOption{emph}[red]
    % is the same as
    % \DeclareStringOption[]{emph}[red]

    % Default option rule
\DeclareDefaultOption{%
  \ifx\CurrentOptionValue\relax
    \PackageWarningNoLine{\@currname}{%
      Unknown option `\CurrentOption'\MessageBreak
      is passed to package `color'%
    }%
    % Pass the option to package color.
    % Again it is better to expand \CurrentOption.
    \expandafter\PassOptionsToPackage
    \expandafter{\CurrentOption}{color}%
  \else
    % Package color does not take options with values.
    % We provide the standard LaTeX error.
    \@unknownoptionerror
  \fi
}

    % Process options
    % ---------------
\ProcessKeyvalOptions*

    % Implementation depending on option values
    % -----------------------------------------
    % Code for print mode
\ifMCS@print
  \PassOptionsToPackage{monochrome}{color}
    % tells package color to use black and white
\fi

\RequirePackage[\MCS@driver]{color}
    % load package color with the correct driver

    % \emph setup
\ifx\MCS@emph\@empty
    % \@empty is a predefined macro with empty contents.
    % the option value of option emph is empty, thus
    % we do not want a redefinition of \emph.
\else
  \renewcommand*{\emph}[1]{%
    \textcolor{\MCS@emph}{#1}%
  }
\fi
%</example>
%    \end{macrocode}
%
% \section{Package options}
%
% The package \xpackage{kvoptions} knows two package options
% \xoption{patch} and \xoption{debugshow}.
% The options of package \xpackage{kvoptions} are intended
% for authors, not for package/class writers. Inside a package
% it is too late for option \xoption{patch} and \xoption{debugshow}
% enables some messages that are perhaps useful for the debugging
% phase. Also \LaTeX\ is unhappy if a package is loaded
% later again with options that are previously not given.
% Thus package and class authors, stay with |\RequirePackage{kvoptions}|
% without options.
%
% Option \xoption{patch} loads package \xpackage{kvoptions-patch}.
%
% \subsection{Package \xpackage{kvoptions-patch}}
%
% \LaTeX's system of package/class options has some severe limitations
% that especially affects the value part if options are used as
% pair of key and value.
% \begin{itemize}
% \item Spaces are removed, regardless where:
%   \begin{quote}
%     |\documentclass[box=0 0 400 600]{article}|
%   \end{quote}
%   Now each package will see |box=00400600| as global option.
% \item In the previous case also braces would not help:
%   \begin{quote}
%     |\documentclass[box={0 0 400 600}]{article}|
%   \end{quote}
%   The result is an error message:
%   \begin{quote}
%     |! LaTeX Error: Missing \begin{document}.|
%   \end{quote}
%   As local option, however, it works if the package
%   knows about key value options (By using this package, for example).
% \item The requirements on robustness are extremly high.
%   \LaTeX\ expands the option. All that will not work as environment
%   name will break also as option. Even a \cs{relax} will generate
%   an error message:
%   \begin{quote}
%     |! Missing \endcsname inserted.|
%   \end{quote}
%   Of course, \LaTeX\ does not use its protecting mechanisms.
%   On contrary \cs{protect} itself will cause errors.
% \item The options are expanded. But perhaps the package will
%   do that, because it has to setup some things before?
%   Example \xpackage{hyperref}:
%   \begin{quote}
%     |\usepackage[pdfauthor=M\"uller]{hyperref}|
%   \end{quote}
%   Package \xpackage{hyperref} does not see |M\"uller| but
%   its expansion and it does not like it, you get many warnings
%   \begin{quote}
%     |Token not allowed in a PDFDocEncoded string|
%   \end{quote}
%   And the title becomes: |Mu127uller|.
%   Therefore such options must usually be given after package
%   \xpackage{hyperref}
%   is loaded:
%   \begin{quote}
%     |\usepackage{hyperref}|\\
%     |\hypersetup{pdfauthor=Fran\c coise M\"uller}|
%   \end{quote}
%   As package option it will even break with |Fran\c coise|
%   because of the cedilla |\c c|, it is not robust enough.
% \end{itemize}
% For users that do not want with this limitations the package
% offers package \xoption{kvoptions-patch}. It patches \LaTeX's option system
% and tries to teach it also to handle options that are given
% as pairs of key and value and to prevent expansion.
% It can already be used at the very beginning, before \cs{documentclass}:
% \begin{quote}
%   |\RequirePackage{kvoptions-patch}|\\
%   |\documentclass[pdfauthor=Fran\c coise M\"uller]{article}|\\
%   |\usepackage{hyperref}|
% \end{quote}
% The latest time is before the package where you want to use
% problematic values:
% \begin{quote}
%   |\usepackage{kvoptions-patch}|\\
%   |\usepackage[Fran\c coise M\"uller]{hyperref}|
% \end{quote}
% Some remarks:
% \begin{itemize}
% \item The patch requires \eTeX, its \cs{unexpanded} feature
%   is much too nice. It is possible to work around using token
%   registers. But the code becomes longer, slower, more difficult
%   to read and maintain. The package without option \xoption{patch}
%   works and will work without \eTeX.
% \item The code for the patch is quite long, there are many test
%   cases. Thus the probability for bugs is probably not too small.
% \item Since 2008/10/18 v3.0 package \xpackage{kvoptions-patch} is
%   available. Before option \xoption{patch} of package \xpackage{kvoptions}
%   must be used instead. I think, the solution as standalone package
%   \xpackage{kvoptions-patch} is cleaner and avoids option clashes.
% \end{itemize}
%
% \subsection{Option \xoption{debugshow}}
%
% The name of this option follows the convention of packages
% \xpackage{multicol}, \xpackage{tabularx}, and \xpackage{tracefnt}.
% Currently it prints the setting of boolean options, declared
% by \cs{DeclareBoolOption} in the \xfile{.log} file, if that
% boolean option is used.
% You can activate the option by
% \begin{itemize}
% \item |\PassOptionsToPackage{debugshow}{kvoptions}|\\
%       Put this somewhere before package \xpackage{kvoptions} is
%       loaded first, e.g. before \cs{documentclass}.
% \item |\RequirePackage[debugshow]{kvoptions}|\\
%       Before \cs{documentclass} even an author has to use
%       \cs{RequirePackage}. \cs{usepackage} only works after
%       \cs{documentclass}.
% \end{itemize}
% The preferred method is \cs{PassOptionsToPackage}, because
% it does not force the package loading and does not disturb,
% if the package is not loaded later at all.
%
% \section{Limitations}
%
% \subsection{Compatibility}
% \subsubsection{Package \xpackage{kvoptions-patch} vs. package \xpackage{xkvltxp}}
%
% Package \xpackage{xkvltxp} from the \xpackage{xkeyval} project
% has the same goal as package \xpackage{kvoptions-patch}
% and to patch \LaTeX's kernel commands in order to get better support
% for key value options.
% Of course they cannot be used both. The user must decide, which
% method he prefers. Package \xpackage{kvoptions-patch} aborts itself,
% if it detects that \xpackage{xkvltxp} is
% already loaded.
%
% However package \xpackage{xkvltxp} and \xpackage{kvoptions}
% can be used together, example:
% \begin{quote}
% |\usepackage{xkvltxp}|\\
% |\usepackage[...]{foobar} % |\texttt{\itshape foobar using kvoptions}
% \end{quote}
% The other way should work, too.
%
% Package \xpackage{kvoptions-patch} tries to catch more situations
% and to be more robust.
% For example, during the comparison of options it normalizes
% them by removing spaces around |=| and the value. Thus the
% following is not reported as option clash:
% \begin{quote}
%\begin{verbatim}
%\RequirePackage{kvoptions-patch}
%\documentclass{article}
%
%\usepackage[scaled=0.7]{helvet}
%\usepackage[scaled = 0.7]{helvet}
%
%\begin{document}
%\end{document}
%\end{verbatim}
% \end{quote}
%
% \subsection{Limitations}
%
% \subsubsection{Option comparisons}
%
% In some situations \LaTeX\ compares option lists, e.g. option clash check,
% \cs{@ifpackagewith}, or \cs{@ifclasswith}. Apart from catcode and
% sanitizing problems of option \xoption{patch}, there is another problem.
% \LaTeX\ does not know about the type and default values of
% options in key value style. Thus an option clash is reported,
% even if the key value has the same meaning:
% \begin{quote}
% |\usepackage[scaled]{helvet} %| \texttt{\textit{default is }.95}\\
% |\usepackage[.95]{helvet}|\\
% |\usepackage[0.95]{helvet}|
% \end{quote}
%
% \subsubsection{Option list parsing with package \xpackage{kvoptions-patch}}
%
% With package \xpackage{kvoptions-patch} the range of possible values
% in key value specifications is much large, for example
% the comma can be used, if enclosed in curly braces.
%
% Other packages, especially the packages that uses their
% own process option code can be surprised to find tokens
% inside options that they do not expect and errors would
% be the consequence. To avoid errors the options, especially
% the unused option list is sanitized. That means the list
% will only contain tokens with catcode 12 (other) and
% perhaps spaces (catcode 10). This allows a safe parsing
% for other packages. But a comma in the value part is
% no longer protected by curly braces because they have lost
% their special meaning. This is the price for compatibility.
%
% Example:
% \begin{quote}
% |\RequirePackage{kvoptions-patch}|\\
% |\documentclass[a={a,b,c},b]{article}|\\
% |\begin{document}|\\
% |\end{document}|
% \end{quote}
% Result:
% \begin{quote}
% |LaTeX Warning: Unused global option(s):|\\
% |    [a={a,c},b].|
% \end{quote}
%
% \StopEventually{
% }
%
% \section{Implementation}
%
% \subsection{Preamble}
%
%    \begin{macrocode}
%<*package>
%    \end{macrocode}
%
% \paragraph{Reload check and  identification.}
%    Reload check, especially if the package is not used with \LaTeX.
%    \begin{macrocode}
\begingroup\catcode61\catcode48\catcode32=10\relax%
  \catcode13=5 % ^^M
  \endlinechar=13 %
  \catcode35=6 % #
  \catcode39=12 % '
  \catcode44=12 % ,
  \catcode45=12 % -
  \catcode46=12 % .
  \catcode58=12 % :
  \catcode64=11 % @
  \catcode123=1 % {
  \catcode125=2 % }
  \expandafter\let\expandafter\x\csname ver@kvoptions.sty\endcsname
  \ifx\x\relax % plain-TeX, first loading
  \else
    \def\empty{}%
    \ifx\x\empty % LaTeX, first loading,
      % variable is initialized, but \ProvidesPackage not yet seen
    \else
      \expandafter\ifx\csname PackageInfo\endcsname\relax
        \def\x#1#2{%
          \immediate\write-1{Package #1 Info: #2.}%
        }%
      \else
        \def\x#1#2{\PackageInfo{#1}{#2, stopped}}%
      \fi
      \x{kvoptions}{The package is already loaded}%
      \aftergroup\endinput
    \fi
  \fi
\endgroup%
%    \end{macrocode}
%    Package identification:
%    \begin{macrocode}
\begingroup\catcode61\catcode48\catcode32=10\relax%
  \catcode13=5 % ^^M
  \endlinechar=13 %
  \catcode35=6 % #
  \catcode39=12 % '
  \catcode40=12 % (
  \catcode41=12 % )
  \catcode44=12 % ,
  \catcode45=12 % -
  \catcode46=12 % .
  \catcode47=12 % /
  \catcode58=12 % :
  \catcode64=11 % @
  \catcode91=12 % [
  \catcode93=12 % ]
  \catcode123=1 % {
  \catcode125=2 % }
  \expandafter\ifx\csname ProvidesPackage\endcsname\relax
    \def\x#1#2#3[#4]{\endgroup
      \immediate\write-1{Package: #3 #4}%
      \xdef#1{#4}%
    }%
  \else
    \def\x#1#2[#3]{\endgroup
      #2[{#3}]%
      \ifx#1\@undefined
        \xdef#1{#3}%
      \fi
      \ifx#1\relax
        \xdef#1{#3}%
      \fi
    }%
  \fi
\expandafter\x\csname ver@kvoptions.sty\endcsname
\ProvidesPackage{kvoptions}%
  [2016/05/16 v3.12 Key value format for package options (HO)]%
%    \end{macrocode}
%
% \paragraph{Catcodes}
%
%    \begin{macrocode}
\begingroup\catcode61\catcode48\catcode32=10\relax%
  \catcode13=5 % ^^M
  \endlinechar=13 %
  \catcode123=1 % {
  \catcode125=2 % }
  \catcode64=11 % @
  \def\x{\endgroup
    \expandafter\edef\csname KVO@AtEnd\endcsname{%
      \endlinechar=\the\endlinechar\relax
      \catcode13=\the\catcode13\relax
      \catcode32=\the\catcode32\relax
      \catcode35=\the\catcode35\relax
      \catcode61=\the\catcode61\relax
      \catcode64=\the\catcode64\relax
      \catcode123=\the\catcode123\relax
      \catcode125=\the\catcode125\relax
    }%
  }%
\x\catcode61\catcode48\catcode32=10\relax%
\catcode13=5 % ^^M
\endlinechar=13 %
\catcode35=6 % #
\catcode64=11 % @
\catcode123=1 % {
\catcode125=2 % }
\def\TMP@EnsureCode#1#2{%
  \edef\KVO@AtEnd{%
    \KVO@AtEnd
    \catcode#1=\the\catcode#1\relax
  }%
  \catcode#1=#2\relax
}
\TMP@EnsureCode{1}{14}% ^^A (comment)
\TMP@EnsureCode{2}{14}% ^^A (comment)
\TMP@EnsureCode{33}{12}% !
\TMP@EnsureCode{39}{12}% '
\TMP@EnsureCode{40}{12}% (
\TMP@EnsureCode{41}{12}% )
\TMP@EnsureCode{42}{12}% *
\TMP@EnsureCode{44}{12}% ,
\TMP@EnsureCode{45}{12}% -
\TMP@EnsureCode{46}{12}% .
\TMP@EnsureCode{47}{12}% /
\TMP@EnsureCode{58}{12}% :
\TMP@EnsureCode{62}{12}% >
\TMP@EnsureCode{91}{12}% [
\TMP@EnsureCode{93}{12}% ]
\TMP@EnsureCode{94}{7}% ^ (superscript)
\TMP@EnsureCode{96}{12}% `
\edef\KVO@AtEnd{\KVO@AtEnd\noexpand\endinput}
%    \end{macrocode}
%
% \paragraph{External resources.}
%    The package extends the support for key value pairs of
%    package \cs{keyval} to package options. Thus the package
%    needs to be loaded anyway. and we use it for
%    \cs{SetupKeyvalOptions}. AFAIK this does not disturb
%    users of \xpackage{xkeyval}.
%    \begin{macrocode}
\@ifundefined{define@key}{%
  \RequirePackage{keyval}\relax
}{}
%    \end{macrocode}
%
%    Macro \cs{DeclareLocalOptions} parses a comma separated key list
%    and uses \cs{comma@parse} of package \xpackage{kvsetkeys},
%    version 1.3.
%    \begin{macrocode}
\RequirePackage{ltxcmds}[2010/12/02]
\RequirePackage{kvsetkeys}[2007/09/29]
%    \end{macrocode}
%
% \paragraph{Provide macros for \plainTeX.}
%    \begin{macrocode}
\@ifundefined{@x@protect}{%
  \def\@x@protect#1\fi#2#3{%
    \fi\protect#1%
  }%
  \let\@typeset@protect\relax
}{}
\@ifundefined{@currname}{%
  \def\@currname{}%
}{}
\@ifundefined{@currext}{%
  \def\@currext{}%
}{}
%    \end{macrocode}
%
% \paragraph{Options}
%    Option \xoption{debugshow} enables additional lines of
%    code that prints information into the \xfile{.log}
%    file.
%    \begin{macrocode}
\DeclareOption{debugshow}{\catcode\@ne=9 }
%    \end{macrocode}
%
%    \begin{macrocode}
\DeclareOption{patch}{%
  \AtEndOfPackage{%
    \RequirePackage{kvoptions-patch}[2016/05/16]%
  }%
}
%    \end{macrocode}
%
%    Optionen auswerten:
%    \begin{macrocode}
\ProcessOptions\relax
%    \end{macrocode}
%
% \subsection{Option declaration macros}
%
% \subsubsection{\cs{SetupKeyvalOptions}}
%
%    The family for the key value pairs can be setup once
%    and is remembered later.
%    The package name seems a reasonable default for the
%    family key, if it is not set by the package author.
%
%    \begin{macro}{\KVO@family}
%    We cannot store the family setting in one macro,
%    because the package should be usable for many
%    other packages, too. Thus we remember the family
%    setting in a macro, whose name contains the
%    package name with extension, a key in \LaTeX's
%    class/package system.
%    \begin{macrocode}
\define@key{KVO}{family}{%
  \expandafter\edef\csname KVO@family@%
      \@currname.\@currext\endcsname{#1}%
}
\def\KVO@family{%
  \@ifundefined{KVO@family@\@currname.\@currext}{%
    \@currname
  }{%
    \csname KVO@family@\@currname.\@currext\endcsname
  }%
}
%    \end{macrocode}
%    \end{macro}
%
%    \begin{macro}{\KVO@prefix}
%    The value settings of options that are declared by \cs{DeclareBoolOption}
%    and \cs{DeclareStringOption} need to be saved in macros.
%    in the first case this is a switch \cs{if}\meta{prefix}\meta{key},
%    in the latter case a macro \cs{}\meta{prefix}\meta{key}.
%    The prefix can be configured, by \xoption{prefix} that is declared
%    here. The default is the package name with |@| appended.
%    \begin{macrocode}
\define@key{KVO}{prefix}{%
  \expandafter\edef\csname KVO@prefix@%
      \@currname.\@currext\endcsname{#1}%
}
\def\KVO@prefix{%
  \ltx@ifundefined{KVO@prefix@\@currname.\@currext}{%
    \@currname @%
  }{%
    \csname KVO@prefix@\@currname.\@currext\endcsname
  }%
}
%    \end{macrocode}
%    \end{macro}
%
%    \begin{macrocode}
\define@key{KVO}{setkeys}{%
  \expandafter\def\csname KVO@setkeys@%
      \@currname.\@currext\endcsname{#1}%
}
%    \end{macrocode}
%    \begin{macro}{\KVO@setkeys}
%    \begin{macrocode}
\def\KVO@setkeys{%
  \ltx@IfUndefined{KVO@setkeys@\@currname.\@currext}{%
    \setkeys
  }{%
    \csname KVO@setkeys@\@currname.\@currext\endcsname
  }%
}
%    \end{macrocode}
%    \end{macro}
%
%    \begin{macro}{\SetupKeyvalOptions}
%    The argument of \cs{SetupKeyvalOptions} expects a key value
%    list, known keys are \xoption{family} and \xoption{prefix}.
%    \begin{macrocode}
\newcommand*{\SetupKeyvalOptions}{%
  \kvsetkeys{KVO}%
}
%    \end{macrocode}
%    \end{macro}
%
% \subsubsection{\cs{DeclareBoolOption}}
%
%    \begin{macro}{\DeclareBoolOption}
%    Usually options of boolean type can be given by the
%    user without value and this means a setting to \emph{true}.
%    We follow this convention here. Also it simplifies the
%    user interface.
%
%    The switch is created and initialized with \emph{false}.
%    The default setting can be overwritten by the optional
%    argument.
%
%    \LaTeX's \cs{newif} does not check for already defined
%    macros, therefore we add this check here to prevent
%    the user from accidently redefining of \TeX's primitives
%    and other macros.
%    \begin{macrocode}
\newcommand*{\DeclareBoolOption}[2][false]{%
  \KVO@ifdefinable{if\KVO@prefix#2}{%
    \KVO@ifdefinable{\KVO@prefix#2true}{%
      \KVO@ifdefinable{\KVO@prefix#2false}{%
        \csname newif\expandafter\endcsname
        \csname if\KVO@prefix#2\endcsname
        \@ifundefined{\KVO@prefix#2#1}{%
          \PackageWarning{kvoptions}{%
            Initialization of option `#2' failed,\MessageBreak
            cannot set boolean option to `#1',\MessageBreak
            use `true' or `false', now using `false'%
          }%
        }{%
          \csname\KVO@prefix#2#1\endcsname
        }%
        \begingroup
          \edef\x{\endgroup
            \noexpand\define@key{\KVO@family}{#2}[true]{%
              \noexpand\KVO@boolkey{\@currname}%
              \ifx\@currext\@clsextension
                \noexpand\@clsextension
              \else
                \noexpand\@pkgextension
              \fi
              {\KVO@prefix}{#2}{####1}%
            }%
          }%
        \x
      }%
    }%
  }%
}
%    \end{macrocode}
%    \end{macro}
%
%    \begin{macro}{\DeclareComplementaryOption}
%    The first argument is the key name, the second the
%    key that must be a boolean option with the same
%    current family and prefix. A new switch is not
%    created for the new key, we have already a switch.
%    Instead we define switch setting commands to work
%    on the parent switch.
%    \begin{macrocode}
\newcommand*{\DeclareComplementaryOption}[2]{%
  \@ifundefined{if\KVO@prefix#2}{%
    \PackageError{kvoptions}{%
      Cannot generate option code for `#1',\MessageBreak
      parent switch `#2' does not exist%
    }{%
      You are inside %
      \ifx\@currext\@clsextension class\else package\fi\space
      `\@currname.\@currext'.\MessageBreak
      `\KVO@family' is used as familiy %
      for the keyval options.\MessageBreak
      `\KVO@prefix' serves as prefix %
      for internal switch macros.\MessageBreak
      \MessageBreak
      \@ehc
    }%
  }{%
    \KVO@ifdefinable{\KVO@prefix#1true}{%
      \KVO@ifdefinable{\KVO@prefix#1false}{%
        \expandafter\let\csname\KVO@prefix#1false\expandafter\endcsname
          \csname\KVO@prefix#2true\endcsname
        \expandafter\let\csname\KVO@prefix#1true\expandafter\endcsname
          \csname\KVO@prefix#2false\endcsname
%    \end{macrocode}
%    The same code part as in \cs{DeclareBoolOption} can
%    now be used.
%    \begin{macrocode}
        \begingroup
          \edef\x{\endgroup
            \noexpand\define@key{\KVO@family}{#1}[true]{%
              \noexpand\KVO@boolkey{\@currname}%
              \ifx\@currext\@clsextension
                \noexpand\@clsextension
              \else
                \noexpand\@pkgextension
              \fi
              {\KVO@prefix}{#1}{####1}%
            }%
          }%
        \x
      }%
    }%
  }%
}
%    \end{macrocode}
%    \end{macro}
%
%    \begin{macro}{\KVO@ifdefinable}
%    Generate the command token LaTeX's \cs{@ifdefinable} expects.
%    \begin{macrocode}
\def\KVO@ifdefinable#1{%
  \expandafter\@ifdefinable\csname #1\endcsname
}
%    \end{macrocode}
%    \end{macro}
%
%    \begin{macro}{\KVO@boolkey}
%    We check explicitly for |true| and |false| to prevent
%    the user from accidently calling other macros.
%    \begin{quote}
%    \begin{tabular}{@{}ll@{}}
%    |#1|& package/class name\\
%    |#2|& |\@pkgextension|/|\@clsextension|\\
%    |#3|& prefix \\
%    |#4|& key name\\
%    |#5|& new value
%    \end{tabular}
%    \end{quote}
%    \begin{macrocode}
\def\KVO@boolkey#1#2#3#4#5{%
  \edef\KVO@param{#5}%
  \ltx@onelevel@sanitize\KVO@param
  \ifx\KVO@param\KVO@true
    \expandafter\@firstofone
  \else
    \ifx\KVO@param\KVO@false
      \expandafter\expandafter\expandafter\@firstofone
    \else
      \ifx#2\@clsextension
        \expandafter\ClassWarning
      \else
        \expandafter\PackageWarning
      \fi
      {#1}{%
        Value `\KVO@param' is not supported by\MessageBreak
        option `#4'%
      }%
      \expandafter\expandafter\expandafter\@gobble
    \fi
  \fi
  {%
    ^^A\ifx#2\@clsextension
    ^^A  \expandafter\ClassInfo
    ^^A\else
    ^^A  \expandafter\PackageInfo
    ^^A\fi
    ^^A{#1}{[option] #4=\KVO@param}%
    \csname#3#4\KVO@param\endcsname
  }%
}
%    \end{macrocode}
%    \end{macro}
%    \begin{macro}{\KVO@true}
%    \begin{macro}{\KVO@false}
%    The macros \cs{KVO@true} and \cs{KVO@false} are used
%    for string comparisons. After \cs{ltx@onelevel@sanitize}
%    we have only tokens with catcode 12 (other).
%    \begin{macrocode}
\def\KVO@true{true}
\def\KVO@false{false}
\ltx@onelevel@sanitize\KVO@true
\ltx@onelevel@sanitize\KVO@false
%    \end{macrocode}
%    \end{macro}
%    \end{macro}
%
% \subsubsection{\cs{DeclareStringOption}}
%
%    \begin{macro}{\DeclareStringOption}
%    \begin{macrocode}
\newcommand*{\DeclareStringOption}[2][]{%
  \@ifnextchar[{%
    \KVO@DeclareStringOption{#1}{#2}@%
  }{%
    \KVO@DeclareStringOption{#1}{#2}{}[]%
  }%
}
%    \end{macrocode}
%    \end{macro}
%    \begin{macro}{\KVO@DeclareStringOption}
%    \begin{macrocode}
\def\KVO@DeclareStringOption#1#2#3[#4]{%
  \KVO@ifdefinable{\KVO@prefix#2}{%
    \@namedef{\KVO@prefix#2}{#1}%
    \begingroup
      \ifx\\#3\\%
        \toks@{}%
      \else
        \toks@{[{#4}]}%
      \fi
      \edef\x{\endgroup
        \noexpand\define@key{\KVO@family}{#2}\the\toks@{%
          ^^A\begingroup
          ^^A  \toks@{####1}%
          ^^A  \ifx\@currext\@clsextension
          ^^A    \noexpand\ClassInfo
          ^^A  \else
          ^^A    \noexpand\PackageInfo
          ^^A  \fi
          ^^A  {\@currname}{%
          ^^A    [option] #2={\noexpand\the\toks@}%
          ^^A  }%
          ^^A\endgroup
          \noexpand\def
          \expandafter\noexpand\csname\KVO@prefix#2\endcsname{####1}%
        }%
      }%
    \x
  }%
}
%    \end{macrocode}
%    \end{macro}
%
% \subsubsection{\cs{DeclareVoidOption}}
%
%    \begin{macro}{\DeclareVoidOption}
%    \begin{macrocode}
\newcommand*{\DeclareVoidOption}[2]{%
  \begingroup
    \let\next\@gobbletwo
    \KVO@ifdefinable{\KVO@prefix#1}{%
      \let\next\@firstofone
    }%
  \expandafter\endgroup
  \next{%
    \begingroup
      \edef\x{\endgroup
        \noexpand\define@key{\KVO@family}{#1}[\KVO@VOID@]{%
          \noexpand\KVO@voidkey{\@currname}%
          \ifx\@currext\@clsextension
            \noexpand\@clsextension
          \else
            \noexpand\@pkgextension
          \fi
          {#1}%
          {####1}%
          \expandafter\noexpand\csname\KVO@prefix#1\endcsname
        }%
      }%
    \x
    \begingroup
      \toks@{#2}%
    \expandafter\endgroup
    \expandafter\def
    \csname\KVO@prefix#1\expandafter\endcsname
    \expandafter{\the\toks@}%
  }%
}
\def\KVO@VOID@{@VOID@}
%    \end{macrocode}
%    \end{macro}
%    \begin{macro}{\KVO@voidkey}
%    \begin{tabular}{@{}ll@{}}
%    |#1|& package/class name\\
%    |#2|& |\@pkgextension|/|\@clsextension|\\
%    |#3|& key name\\
%    |#4|& default (|@VOID@|)\\
%    |#5|& macro with option code
%    \end{tabular}
%    \begin{macrocode}
\def\KVO@voidkey#1#2#3#4{%
  \def\CurrentOption{#3}%
  \begingroup
    \def\x{#4}%
  \expandafter\endgroup
  \ifx\x\KVO@VOID@
  \else
    \ifx#2\@clsextension
      \expandafter\ClassWarning
    \else
      \expandafter\PackageWarning
    \fi
    {#1}{%
      Unexpected value for option `#3'\MessageBreak
      is ignored%
    }%
  \fi
  ^^A\ifx#2\@clsextension
  ^^A  \expandafter\ClassInfo
  ^^A\else
  ^^A  \expandafter\PackageInfo
  ^^A\fi
  ^^A{#1}{[option] #3}%
}
%    \end{macrocode}
%    \end{macro}
%
% \subsubsection{\cs{DeclareDefaultOption}}
%
%    \begin{macro}{\DeclareDefaultOption}
%    \begin{macrocode}
\newcommand*{\DeclareDefaultOption}{%
  \@namedef{KVO@default@\@currname.\@currext}%
}
%    \end{macrocode}
%    \end{macro}
%
% \subsubsection{\cs{DeclareLocalOptions}}
%
%    \begin{macro}{\DeclareLocalOptions}
%    \begin{macrocode}
\newcommand*{\DeclareLocalOptions}[1]{%
  \comma@parse{#1}\KVO@DeclareLocalOption
}
%    \end{macrocode}
%    \end{macro}
%    \begin{macro}{\KVO@DeclareLocalOption}
%    \begin{macrocode}
\def\KVO@DeclareLocalOption#1{%
  \expandafter\def\csname KVO@local@\KVO@family @#1\endcsname{}%
}
%    \end{macrocode}
%    \end{macro}
%
% \subsection{Dynamic options}
%
% \subsubsection{\cs{DisableKeyvalOption}}
%
%    \begin{macrocode}
\SetupKeyvalOptions{%
  family=KVOdyn,%
  prefix=KVOdyn@%
}
\DeclareBoolOption[true]{global}
\DeclareComplementaryOption{local}{global}
\DeclareStringOption[undef]{action}
\let\KVOdyn@name\relax
\let\KVOdyn@ext\@empty
\define@key{KVOdyn}{class}{%
  \def\KVOdyn@name{#1}%
  \let\KVOdyn@ext\@clsextension
}
\define@key{KVOdyn}{package}{%
  \def\KVOdyn@name{#1}%
  \let\KVOdyn@ext\@pkgextension
}
\newcommand*{\DisableKeyvalOption}[3][]{%
  \begingroup
    \kvsetkeys{KVOdyn}{#1}%
    \def\x{\endgroup}%
    \@ifundefined{KVO@action@\KVOdyn@action}{%
      \PackageError{kvoptions}{%
        Unknown disable action %
        `\expandafter\strip@prefix\meaning\KVOdyn@action'\MessageBreak
        for option `#3' in keyval family '#2'%
      }\@ehc
    }{%
      \csname KVO@action@\KVOdyn@action\endcsname{#2}{#3}%
    }%
  \x
}
\def\KVO@action@undef#1#2{%
  \edef\x{\endgroup
    \ifKVOdyn@global\global\fi
    \let
    \expandafter\noexpand\csname KV@#1@#2\endcsname
    \relax
    \ifKVOdyn@global\global\fi
    \let
    \expandafter\noexpand\csname KV@#1@#2@default\endcsname
    \relax
  }%
  ^^A\PackageInfo{kvoptions}{%
  ^^A  [option] key `#2' of family `#1'\MessageBreak
  ^^A  is disabled (undef, \ifKVOdyn@global global\else local\fi)%
  ^^A}%
}
\def\KVO@action@ignore#1#2{%
  \edef\x{\endgroup
    \ifKVOdyn@global\global\fi
    \let
    \expandafter\noexpand\csname KV@#1@#2\endcsname
    \noexpand\@gobble
    \ifKVOdyn@global\global\fi
    \let
    \expandafter\noexpand\csname KV@#1@#2@default\endcsname
    \noexpand\@empty
  }%
  ^^A\PackageInfo{kvoptions}{%
  ^^A  [option] key `#2' of family `#1'\MessageBreak
  ^^A  is disabled (ignore, \ifKVOdyn@global global\else local\fi)%
  ^^A}%
}
\def\KVO@action@error{%
  \KVO@do@action{error}%
}
\def\KVO@action@warning{%
  \KVO@do@action{warning}%
}
%    \end{macrocode}
%    \begin{tabular}{@{}ll@{}}
%    |#1|& |error| or |warning|\\
%    |#2|& \meta{family}\\
%    |#3|& \meta{key}\\
%    \end{tabular}
%    \begin{macrocode}
\def\KVO@do@action#1#2#3{%
  \ifx\KVOdyn@name\relax
    \PackageError{kvoptions}{%
      Action type `#1' needs package/class name\MessageBreak
      for key `#3' in family `#2'%
    }\@ehc
  \else
    \edef\x{\endgroup
      \noexpand\define@key{#2}{#3}[]{%
        \expandafter\noexpand\csname KVO@disable@#1\endcsname
        {\KVOdyn@name}\noexpand\KVOdyn@ext{#3}%
      }%
      \ifKVOdyn@global
        \global\let
        \expandafter\noexpand\csname KV@#2@#3\endcsname
        \expandafter\noexpand\csname KV@#2@#3\endcsname
        \global\let
        \expandafter\noexpand\csname KV@#2@#3@default\endcsname
        \expandafter\noexpand\csname KV@#2@#3@default\endcsname
      \fi
    }%
    ^^A\ifx\KVOdyn@ext\@clsextension
    ^^A  \expandafter\ClassInfo
    ^^A\else
    ^^A   \expandafter\PackageInfo
    ^^A\fi
    ^^A{\KVOdyn@name}{%
    ^^A  [option] key `#3' of family `#2'\MessageBreak
    ^^A  is disabled (#1, \ifKVOdyn@global global\else local\fi)%
    ^^A}%
  \fi
}
\def\KVO@disable@error#1#2#3{%
  \ifx#2\@clsextension
    \expandafter\ClassError
  \else
    \expandafter\PackageError
  \fi
  {#1}{%
    Option `#3' is given too late,\MessageBreak
    now the option is ignored%
  }\@ehc
}
\def\KVO@disable@warning#1#2#3{%
  \ifx#2\@clsextension
    \expandafter\ClassWarning
  \else
    \expandafter\PackageWarning
  \fi
  {#1}{%
    Option `#3' is already consumed\MessageBreak
    and has no effect%
  }%
}
%    \end{macrocode}
%
% \subsection{Change option code}
%
% \subsubsection{\cs{AddToKeyvalOption}}
%
%    \begin{macro}{\AddToKeyvalOption}
%    \begin{macrocode}
\newcommand*{\AddToKeyvalOption}{%
  \@ifstar{%
    \begingroup
      \edef\x{\endgroup
        \noexpand\KVO@AddToKeyvalOption{\KVO@family}%
      }%
    \x
  }%
  \KVO@AddToKeyvalOption
}
%    \end{macrocode}
%    \end{macro}
%    \begin{macro}{\KVO@AddToKeyvalOption}
%    \begin{macrocode}
\def\KVO@AddToKeyvalOption#1#2{%
  \@ifundefined{KV@#1@#2}{%
    \PackageWarning{kvoptions}{%
      Key `#2' of family `#1' does not exist.\MessageBreak
      Ignoring \string\AddToKeyvalOption
    }%
    \@gobble
  }{%
    \edef\KVO@next{%
      \noexpand\KVO@@AddToKeyvalOption
      \expandafter\noexpand\csname KV@#1@#2\endcsname
    }%
    \afterassignment\KVO@next
    \def\KVO@temp##1%
  }%
}
%    \end{macrocode}
%    \end{macro}
%    \begin{macro}{\KVO@@AddToKeyvalOption}
%    \begin{macrocode}
\def\KVO@@AddToKeyvalOption#1{%
  \begingroup
    \toks@\expandafter{#1{##1}}%
    \toks@\expandafter{\the\expandafter\toks@\KVO@temp{##1}}%
    \edef\x{\endgroup
      \noexpand\def\noexpand#1####1{\the\toks@}%
    }%
  \x
}
%    \end{macrocode}
%    \end{macro}
%
% \subsection{Process options}
%
% \subsubsection{Get global options}
%
%    Package \xpackage{xkeyval} removes options with equal signs from
%    the global options (\cs{@classoptionslist}). The effect is that
%    other packages and classes will not see these global options anymore.
%    A bug-report was answered that this behaviour is ``by design''.
%    Thus I call it a design bug. Now getting the global options require
%    an algorithm instead of a simple macro call.
%    \begin{macrocode}
%</package>
%<*package|patch>
%    \end{macrocode}
%    \begin{macro}{\KVO@IfDefThen}
%    Call |#2| if command |#1| is defined and not \cs{relax}.
%    (Package \xpackage{kvoptions-patch} does not load
%    package \xpackage{ltxcmds}.)
%    \begin{macrocode}
\def\KVO@IfDefThen#1#2{%
  \ifx#1\ltx@undefined
  \else
    \ifx#1\relax
    \else
      #2%
    \fi
  \fi
}%
%    \end{macrocode}
%    \end{macro}
%    \begin{macro}{\KVO@GetClassOptionsList}
%    \begin{macrocode}
\def\KVO@GetClassOptionsList{%
  \let\KVO@classoptionslist\@classoptionslist
  \KVO@IfDefThen\@classoptionslist{%
    \KVO@IfDefThen\XKV@documentclass{%
      \ifx\XKV@documentclass\ltx@empty
      \else
        \KVO@IfDefThen\XKV@classoptionslist{%
          \ifx\XKV@classoptionslist\ltx@empty
          \else
            \let\KVO@classoptionslist\XKV@classoptionslist
          \fi
        }%
      \fi
    }%
  }%
}%
%    \end{macrocode}
%    \end{macro}
%    \begin{macrocode}
%</package|patch>
%<*package>
%    \end{macrocode}
%
% \subsubsection{\cs{ProcessKeyvalOptions}}
%
%    \begin{macro}{\ProcessKeyvalOptions}
%    If the optional star is given, we get the family
%    name and expand it for safety.
%    \begin{macrocode}
\newcommand*{\ProcessKeyvalOptions}{%
  \@ifstar{%
    \begingroup
      \edef\x{\endgroup
        \noexpand\KVO@ProcessKeyvalOptions{\KVO@family}%
      }%
    \x
  }%
  \KVO@ProcessKeyvalOptions
}
%    \end{macrocode}
%    \end{macro}
%
%    \begin{macrocode}
\def\KVO@ProcessKeyvalOptions#1{%
  \let\@tempc\relax
  \let\KVO@temp\@empty
%    \end{macrocode}
%    Add any global options that are known to KV to the start of
%    the list being built in |\KVO@temp| and mark them used (by
%    removing them from the unused option list).
%    \begin{macrocode}
  \ifx\@currext\@clsextension
  \else
    \KVO@GetClassOptionsList
    \ifx\KVO@classoptionslist\relax
    \else
      \@for\KVO@CurrentOption:=\KVO@classoptionslist\do{%
        \@ifundefined{KV@#1@\expandafter\KVO@getkey
                      \KVO@CurrentOption=\@nil}{%
        }{%
          \@ifundefined{KVO@local@#1@\expandafter\KVO@getkey
                        \KVO@CurrentOption=\@nil}{%
            \ifx\KVO@Patch Y%
              \edef\KVO@temp{%
                \etex@unexpanded\expandafter{%
                  \KVO@temp
                }%
                ,%
                \etex@unexpanded\expandafter{%
                  \KVO@CurrentOption
                }%
                ,%
              }%
              \ltx@onelevel@sanitize\KVO@CurrentOption
            \else
              \edef\KVO@temp{%
                \KVO@temp
                ,%
                \KVO@CurrentOption
                ,%
              }%
            \fi
            \@expandtwoargs\@removeelement\KVO@CurrentOption
              \@unusedoptionlist\@unusedoptionlist
          }{}%
        }%
      }%
    \fi
  \fi
%    \end{macrocode}
%     Now stick the package options at the end of the list and wrap
%     in a call to \cs{setkeys}. A class ignores unknown global
%     options, we must remove them to prevent error messages
%     from \cs{setkeys}.
%    \begin{macrocode}
  \begingroup
    \toks\tw@{}%
    \@ifundefined{opt@\@currname.\@currext}{%
      \toks@\expandafter{\KVO@temp}%
    }{%
      \toks@\expandafter\expandafter\expandafter{%
        \csname opt@\@currname.\@currext\endcsname
      }%
      \ifx\@currext\@clsextension
        \edef\CurrentOption{\the\toks@}%
        \toks@\expandafter{\KVO@temp}%
        \@for\CurrentOption:=\CurrentOption\do{%
          \@ifundefined{%
            KV@#1@\expandafter\KVO@getkey\CurrentOption=\@nil
          }{%
%    \end{macrocode}
%    A class puts not used options in the unused option list
%    unless there is a default handler.
%    \begin{macrocode}
            \@ifundefined{KVO@default@\@currname.\@currext}{%
              \ifx\KVO@Patch Y%
                \ltx@onelevel@sanitize\CurrentOption
              \fi
              \ifx\@unusedoptionlist\@empty
                \global\let\@unusedoptionlist\CurrentOption
              \else
                \expandafter\expandafter\expandafter\gdef
                \expandafter\expandafter\expandafter\@unusedoptionlist
                \expandafter\expandafter\expandafter{%
                  \expandafter\@unusedoptionlist
                  \expandafter,\CurrentOption
                }%
              \fi
            }{%
              \toks\tw@\expandafter{%
                \the\toks\expandafter\tw@\expandafter,\CurrentOption
              }%
            }%
          }{%
            \toks@\expandafter{%
              \the\expandafter\toks@\expandafter,\CurrentOption
            }%
          }%
        }%
      \else
%    \end{macrocode}
%    Without default action we pass all options to \cs{setkeys}.
%    Otherwise we have to check which options are known.
%    These are passed to \cs{setkeys}. For the others the default
%    action is performed.
%    \begin{macrocode}
        \@ifundefined{KVO@default@\@currname.\@currext}{%
          \toks@\expandafter\expandafter\expandafter{%
            \expandafter\KVO@temp\the\toks@
          }%
        }{%
          \edef\CurrentOption{\the\toks@}%
          \toks@\expandafter{\KVO@temp}%
          \@for\CurrentOption:=\CurrentOption\do{%
            \@ifundefined{%
              KV@#1@\expandafter\KVO@getkey\CurrentOption=\@nil
            }{%
              \toks\tw@\expandafter{%
                \the\toks\expandafter\tw@\expandafter,\CurrentOption
              }%
            }{%
              \toks@\expandafter{%
                \the\expandafter\toks@\expandafter,\CurrentOption
              }%
            }%
          }%
        }%
      \fi
    }%
    \edef\KVO@temp{\endgroup
      \noexpand\KVO@calldefault{\the\toks\tw@}%
      \noexpand\KVO@setkeys{#1}{\the\toks@}%
    }%
  \KVO@temp
%    \end{macrocode}
%    Some cleanup of \cs{ProcessOptions}.
%    \begin{macrocode}
  \let\CurrentOption\@empty
  \AtEndOfPackage{\let\@unprocessedoptions\relax}%
}
%    \end{macrocode}
%
% \subsubsection{\cs{ProcessLocalKeyvalOptions}}
%
%    \begin{macro}{\ProcessLocalKeyvalOptions}
%    If the optional star is given, we get the family
%    name and expand it for safety.
%    \begin{macrocode}
\newcommand*{\ProcessLocalKeyvalOptions}{%
  \@ifstar{%
    \begingroup
      \edef\x{\endgroup
        \noexpand\KVO@ProcessLocalKeyvalOptions{\KVO@family}%
      }%
    \x
  }%
  \KVO@ProcessLocalKeyvalOptions
}
%    \end{macrocode}
%    \end{macro}
%
%    \begin{macrocode}
\def\KVO@ProcessLocalKeyvalOptions#1{%
  \let\@tempc\relax
  \let\KVO@temp\@empty
%    \end{macrocode}
%    Check if \cs{ProcessLocalKeyvalOptions} is called inside
%    a package.
%    \begin{macrocode}
  \ifx\@currext\@pkgextension
  \else
    \PackageError{kvoptions}{%
      \string\ProcessLocalKeyvalOptions\space is intended for packages only%
    }\@ehc
  \fi
%    \end{macrocode}
%     The package options are put into toks register \cs{toks@}.
%    \begin{macrocode}
  \begingroup
    \toks\tw@{}%
    \@ifundefined{opt@\@currname.\@currext}{%
      \toks@\expandafter{\KVO@temp}%
    }{%
      \toks@\expandafter\expandafter\expandafter{%
        \csname opt@\@currname.\@currext\endcsname
      }%
%    \end{macrocode}
%    Without default action we pass all options to \cs{setkeys}.
%    Otherwise we have to check which options are known.
%    These are passed to \cs{setkeys}. For the others the default
%    action is performed.
%    \begin{macrocode}
      \@ifundefined{KVO@default@\@currname.\@currext}{%
        \toks@\expandafter\expandafter\expandafter{%
          \expandafter\KVO@temp\the\toks@
        }%
      }{%
        \edef\CurrentOption{\the\toks@}%
        \toks@\expandafter{\KVO@temp}%
        \@for\CurrentOption:=\CurrentOption\do{%
          \@ifundefined{%
            KV@#1@\expandafter\KVO@getkey\CurrentOption=\@nil
          }{%
            \toks\tw@\expandafter{%
              \the\toks\expandafter\tw@\expandafter,\CurrentOption
            }%
          }{%
            \toks@\expandafter{%
              \the\expandafter\toks@\expandafter,\CurrentOption
            }%
          }%
        }%
      }%
    }%
    \edef\KVO@temp{\endgroup
      \noexpand\KVO@calldefault{\the\toks\tw@}%
      \noexpand\KVO@setkeys{#1}{\the\toks@}%
    }%
  \KVO@temp
%    \end{macrocode}
%    Some cleanup of \cs{ProcessOptions}.
%    \begin{macrocode}
  \let\CurrentOption\@empty
  \AtEndOfPackage{\let\@unprocessedoptions\relax}%
}
%    \end{macrocode}
%
% \subsubsection{Helper macros}
%
%    \begin{macro}{\KVO@getkey}
%    Extract the key part of a key=value pair.
%    \begin{macrocode}
\def\KVO@getkey#1=#2\@nil{#1}
%    \end{macrocode}
%    \end{macro}
%
%    \begin{macro}{\KVO@calldefault}
%    \begin{macrocode}
\def\KVO@calldefault#1{%
  \begingroup
    \def\x{#1}%
  \expandafter\endgroup
  \ifx\x\@empty
  \else
    \@for\CurrentOption:=#1\do{%
      \ifx\CurrentOption\@empty
      \else
        \expandafter\KVO@setcurrents\CurrentOption=\@nil
        \@nameuse{KVO@default@\@currname.\@currext}%
      \fi
    }%
  \fi
}
%    \end{macrocode}
%    \end{macro}
%    \begin{macro}{\KVO@setcurrents}
%    Extract the key part of a key=value pair.
%    \begin{macrocode}
\def\KVO@setcurrents#1=#2\@nil{%
  \def\CurrentOptionValue{#2}%
  \ifx\CurrentOptionValue\@empty
    \let\CurrentOptionKey\CurrentOption
    \let\CurrentOptionValue\relax
  \else
    \edef\CurrentOptionKey{\zap@space#1 \@empty}%
    \expandafter\KVO@setcurrentvalue\CurrentOption\@nil
  \fi
}
%    \end{macrocode}
%    \end{macro}
%    \begin{macro}{\KV@setcurrentvalue}
%    Here the value part is parsed. Package \xpackage{keyval}'s
%    \cs{KV@@sp@def} helps in removing spaces at the begin and
%    end of the value.
%    \begin{macrocode}
\def\KVO@setcurrentvalue#1=#2\@nil{%
  \KV@@sp@def\CurrentOptionValue{#2}%
}
%    \end{macrocode}
%    \end{macro}
%
% \subsection{\plainTeX}
%
%    Disable \LaTeX\ stuff.
%    \begin{macrocode}
\begingroup\expandafter\expandafter\expandafter\endgroup
\expandafter\ifx\csname documentclass\endcsname\relax
  \def\ProcessKeyvalOptions{%
    \@ifstar{}\@gobble
  }%
\fi
%    \end{macrocode}
%
%    \begin{macrocode}
\KVO@AtEnd%
%</package>
%    \end{macrocode}
%
% \subsection{Package \xpackage{kvoptions-patch}}
%
%    \begin{macrocode}
%<*patch>
\NeedsTeXFormat{LaTeX2e}
\begingroup\catcode61\catcode48\catcode32=10\relax%
  \catcode13=5 % ^^M
  \endlinechar=13 %
  \catcode123=1 % {
  \catcode125=2 % }
  \catcode64=11 % @
  \def\x{\endgroup
    \expandafter\edef\csname KVO@AtEnd\endcsname{%
      \endlinechar=\the\endlinechar\relax
      \catcode13=\the\catcode13\relax
      \catcode32=\the\catcode32\relax
      \catcode35=\the\catcode35\relax
      \catcode61=\the\catcode61\relax
      \catcode64=\the\catcode64\relax
      \catcode123=\the\catcode123\relax
      \catcode125=\the\catcode125\relax
    }%
  }%
\x\catcode61\catcode48\catcode32=10\relax%
\catcode13=5 % ^^M
\endlinechar=13 %
\catcode35=6 % #
\catcode64=11 % @
\catcode123=1 % {
\catcode125=2 % }
\def\TMP@EnsureCode#1#2{%
  \edef\KVO@AtEnd{%
    \KVO@AtEnd
    \catcode#1=\the\catcode#1\relax
  }%
  \catcode#1=#2\relax
}
\TMP@EnsureCode{39}{12}% '
\TMP@EnsureCode{40}{12}% (
\TMP@EnsureCode{41}{12}% )
\TMP@EnsureCode{43}{12}% +
\TMP@EnsureCode{44}{12}% ,
\TMP@EnsureCode{45}{12}% -
\TMP@EnsureCode{46}{12}% .
\TMP@EnsureCode{47}{12}% /
\TMP@EnsureCode{58}{12}% :
\TMP@EnsureCode{60}{12}% <
\TMP@EnsureCode{62}{12}% >
\TMP@EnsureCode{91}{12}% [
\TMP@EnsureCode{93}{12}% ]
\TMP@EnsureCode{96}{12}% `
\TMP@EnsureCode{124}{12}% |
\edef\KVO@AtEnd{\KVO@AtEnd\noexpand\endinput}
\ProvidesPackage{kvoptions-patch}%
  [2016/05/16 v3.12 LaTeX patch for keyval options (HO)]%
%    \end{macrocode}
%
%    Check for \eTeX.
%    \begin{macrocode}
\begingroup\expandafter\expandafter\expandafter\endgroup
\expandafter\ifx\csname eTeXversion\endcsname\relax
  \PackageWarningNoLine{kvoptions-patch}{%
    Package loading is aborted, because e-TeX is missing%
  }%
  \expandafter\KVO@AtEnd
\fi%
%    \end{macrocode}
%
%    Package \xpackage{etexcmds} for \cs{etex@unexpanded}.
%    \begin{macrocode}
\RequirePackage{etexcmds}[2007/09/09]
\ifetex@unexpanded
\else
  \PackageError{kvoptions-patch}{%
    Could not find eTeX's \string\unexpanded.\MessageBreak
    Try adding \string\RequirePackage\string{etexcmds\string} %
    before \string\documentclass%
  }\@ehd
  \expandafter\KVO@AtEnd
\fi%
%    \end{macrocode}
%
%    Check for package \xpackage{xkvltxp}.
%    \begin{macrocode}
\@ifpackageloaded{xkvltxp}{%
  \PackageWarningNoLine{kvoptions}{%
    Option `patch' cannot be used together with\MessageBreak
    package `xkvltxp' that is already loaded.\MessageBreak
    Therefore package loading is aborted%
  }%
  \KVO@AtEnd
}{}%
%    \end{macrocode}
%
%    \begin{macrocode}
\def\@if@ptions#1#2#3{%
  \begingroup
    \KVO@normalize\KVO@temp{#3}%
    \edef\x{\endgroup
      \noexpand\@if@pti@ns{%
        \detokenize\expandafter\expandafter\expandafter{%
          \csname opt@#2.#1\endcsname
        }%
      }{%
        \detokenize\expandafter{\KVO@temp}%
      }%
    }%
  \x
}
%    \end{macrocode}
%
%    \begin{macrocode}
\def\@pass@ptions#1#2#3{%
  \KVO@normalize\KVO@temp{#2}%
  \@ifundefined{opt@#3.#1}{%
    \expandafter\gdef\csname opt@#3.#1%
          \expandafter\endcsname\expandafter{%
      \KVO@temp
    }%
  }{%
    \expandafter\gdef\csname opt@#3.#1%
          \expandafter\expandafter\expandafter\endcsname
          \expandafter\expandafter\expandafter{%
      \csname opt@#3.#1\expandafter\endcsname\expandafter,\KVO@temp
    }%
  }%
}
%    \end{macrocode}
%
%    \begin{macrocode}
\def\ProcessOptions{%
  \let\ds@\@empty
  \@ifundefined{opt@\@currname.\@currext}{%
    \let\@curroptions\@empty
  }{%
    \expandafter\expandafter\expandafter\def
    \expandafter\expandafter\expandafter\@curroptions
    \expandafter\expandafter\expandafter{%
      \csname opt@\@currname.\@currext\endcsname
    }%
  }%
  \@ifstar\KVO@xprocess@ptions\KVO@process@ptions
}
%    \end{macrocode}
%
%    \begin{macrocode}
\def\KVO@process@ptions{%
  \@for\CurrentOption:=\@declaredoptions\do{%
    \ifx\CurrentOption\@empty
    \else
      \begingroup
        \ifx\@currext\@clsextension
          \toks@{}%
        \else
          \KVO@GetClassOptionsList
          \toks@\expandafter{\KVO@classoptionslist,}%
        \fi
        \toks\tw@\expandafter{\@curroptions}%
        \edef\x{\endgroup
          \noexpand\in@{,\CurrentOption,}{,\the\toks@\the\toks\tw@,}%
        }%
      \x
      \ifin@
        \KVO@use@ption
        \expandafter\let\csname ds@\CurrentOption\endcsname\@empty
      \fi
    \fi
  }%
  \KVO@process@pti@ns
}
%    \end{macrocode}
%
%    \begin{macrocode}
\def\KVO@xprocess@ptions{%
  \ifx\@currext\@clsextension
  \else
    \KVO@GetClassOptionsList
    \@for\CurrentOption:=\KVO@classoptionslist\do{%
      \ifx\CurrentOption\@empty
      \else
        \KVO@in@\CurrentOption\@declaredoptions
        \ifin@
          \KVO@use@ption
          \expandafter\let\csname ds@\CurrentOption\endcsname\@empty
        \fi
      \fi
    }%
  \fi
  \KVO@process@pti@ns
}
%    \end{macrocode}
%
%    \begin{macrocode}
\def\KVO@in@#1#2{%
  \in@false
  \begingroup
    \@for\x:=#2\do{%
      \ifx\x#1\relax
        \in@true
      \fi
    }%
    \edef\x{\endgroup
      \ifin@
        \noexpand\in@true
      \fi
    }%
  \x
}
%    \end{macrocode}
%
%    \begin{macrocode}
\def\KVO@process@pti@ns{%
  \@for\CurrentOption:=\@curroptions\do{%
    \@ifundefined{ds@\KVO@SanitizedCurrentOption}{%
      \KVO@use@ption
      \default@ds
    }%
    \KVO@use@ption
  }%
  \@for\CurrentOption:=\@declaredoptions\do{%
    \expandafter\let\csname ds@\CurrentOption\endcsname\relax
  }%
  \let\CurrentOption\@empty
  \let\@fileswith@pti@ns\@@fileswith@pti@ns
  \AtEndOfPackage{\let\@unprocessedoptions\relax}%
}
%    \end{macrocode}
%
%    \begin{macrocode}
\def\KVO@use@ption{%
  \begingroup
    \edef\x{\endgroup
      \noexpand\@removeelement{%
        \detokenize\expandafter{\CurrentOption}%
      }{%
        \detokenize\expandafter{\@unusedoptionlist}%
      }%
    }%
  \x\@unusedoptionlist
  \csname ds@\KVO@SanitizedCurrentOption\endcsname
}
%    \end{macrocode}
%
%    \begin{macrocode}
\def\OptionNotUsed{%
  \ifx\@currext\@clsextension
    \xdef\@unusedoptionlist{%
      \ifx\@unusedoptionlist\@empty
      \else
        \detokenize\expandafter{\@unusedoptionlist,}%
      \fi
      \detokenize\expandafter{\CurrentOption}%
    }%
  \fi
}
%    \end{macrocode}
%
%    Variant of \cs{ExecuteOptions} that better protects \cs{CurrentOption}.
%    \begin{macrocode}
\def\CurrentOption@SaveLevel{0}
\def\ExecuteOptions{%
  \expandafter\KVO@ExecuteOptions
      \csname CurrentOption@\CurrentOption@SaveLevel\endcsname
}
\def\KVO@ExecuteOptions#1#2{%
  \let#1\CurrentOption
  \edef\CurrentOption@SaveLevel{%
    \the\numexpr\CurrentOption@SaveLevel+1%
  }%
  \@for\CurrentOption:=#2\do{%
    \csname ds@\CurrentOption\endcsname
  }%
  \edef\CurrentOption@SaveLevel{%
    \the\numexpr\CurrentOption@SaveLevel-1%
  }%
  \let\CurrentOption#1%
}
%    \end{macrocode}
%
%    \begin{macrocode}
\def\KVO@fileswith@pti@ns#1[#2]#3[#4]{%
  \ifx#1\@clsextension
    \ifx\@classoptionslist\relax
      \KVO@normalize\KVO@temp{#2}%
      \expandafter\gdef\expandafter\@classoptionslist\expandafter{%
        \KVO@temp
      }%
      \def\reserved@a{%
        \KVO@onefilewithoptions{#3}[{#2}][{#4}]#1%
        \@documentclasshook
      }%
    \else
      \def\reserved@a{%
        \KVO@onefilewithoptions{#3}[{#2}][{#4}]#1%
      }%
    \fi
  \else
    \begingroup
      \let\KVO@temp\relax
      \let\KVO@onefilewithoptions\relax
      \let\@pkgextension\relax
      \def\reserved@b##1,{%
        \ifx\@nil##1\relax
        \else
          \ifx\relax##1\relax
          \else
            \KVO@onefilewithoptions{##1}[{\KVO@temp}][{#4}]%
            \@pkgextension
          \fi
          \expandafter\reserved@b
        \fi
      }%
      \edef\reserved@a{\zap@space#3 \@empty}%
      \edef\reserved@a{\expandafter\reserved@b\reserved@a,\@nil,}%
      \toks@{#2}%
      \def\KVO@temp{\the\toks@}%
    \edef\reserved@a{\endgroup \reserved@a}%
  \fi
  \reserved@a
}
%    \end{macrocode}
%
%    \begin{macrocode}
\def\KVO@onefilewithoptions#1[#2][#3]#4{%
  \@pushfilename
  \xdef\@currname{#1}%
  \global\let\@currext#4%
  \expandafter\let\csname\@currname.\@currext-h@@k\endcsname\@empty
  \let\CurrentOption\@empty
  \@reset@ptions
  \makeatletter
  \def\reserved@a{%
    \@ifl@aded\@currext{#1}{%
      \@if@ptions\@currext{#1}{#2}{%
      }{%
        \begingroup
          \@ifundefined{opt@#1.\@currext}{%
            \def\x{}%
          }{%
            \edef\x{%
              \expandafter\expandafter\expandafter\strip@prefix
              \expandafter\meaning\csname opt@#1.\@currext\endcsname
            }%
          }%
          \def\y{#2}%
          \edef\y{\expandafter\strip@prefix\meaning\y}%
          \@latex@error{Option clash for \@cls@pkg\space #1}{%
            The package #1 has already been loaded %
            with options:\MessageBreak
            \space\space[\x]\MessageBreak
            There has now been an attempt to load it %
             with options\MessageBreak
            \space\space[\y]\MessageBreak
            Adding the global options:\MessageBreak
            \space\space
                 \x,\y\MessageBreak
            to your \noexpand\documentclass declaration may fix this.%
            \MessageBreak
            Try typing \space <return> \space to proceed.%
          }%
        \endgroup
      }%
    }{%
      \@pass@ptions\@currext{#2}{#1}%
      \global\expandafter
      \let\csname ver@\@currname.\@currext\endcsname\@empty
      \InputIfFileExists
        {\@currname.\@currext}%
        {}%
        {\@missingfileerror\@currname\@currext}%
      \let\@unprocessedoptions\@@unprocessedoptions
      \csname\@currname.\@currext-h@@k\endcsname
      \expandafter\let\csname\@currname.\@currext-h@@k\endcsname
              \@undefined
      \@unprocessedoptions
    }%
    \@ifl@ter\@currext{#1}{#3}{%
    }{%
      \@latex@warning@no@line{%
        You have requested,\on@line, %
        version\MessageBreak
          #3' of \@cls@pkg\space #1,\MessageBreak
        but only version\MessageBreak
         `\csname ver@#1.\@currext\endcsname'\MessageBreak
        is available%
      }%
    }%
    \ifx\@currext\@clsextension\let\LoadClass\@twoloadclasserror\fi
    \@popfilename
    \@reset@ptions
  }%
  \reserved@a
}
%    \end{macrocode}
%
%    \begin{macrocode}
\def\@unknownoptionerror{%
  \@latex@error{%
    Unknown option `\KVO@SanitizedCurrentOption' %
    for \@cls@pkg\space`\@currname'%
  }{%
    The option `\KVO@SanitizedCurrentOption' was not declared in %
    \@cls@pkg\space`\@currname', perhaps you\MessageBreak
    misspelled its name. %
    Try typing \space <return> %
    \space to proceed.%
  }%
}
%    \end{macrocode}
%
%    \begin{macrocode}
\def\@@unprocessedoptions{%
  \ifx\@currext\@pkgextension
    \@ifundefined{opt@\@currname.\@currext}{%
      \let\@curroptions\@empty
    }{%
      \expandafter\let\expandafter\@curroptions
          \csname opt@\@currname.\@currext\endcsname
    }%
    \@for\CurrentOption:=\@curroptions\do{%
        \ifx\CurrentOption\@empty\else\@unknownoptionerror\fi
    }%
  \fi
}
%    \end{macrocode}
%
%    \begin{macrocode}
\def\KVO@SanitizedCurrentOption{%
  \expandafter\strip@prefix\meaning\CurrentOption
}
%    \end{macrocode}
%
%    Normalize option list.
%    \begin{macrocode}
\def\KVO@normalize#1#2{%
  \let\KVO@result\@empty
  \KVO@splitcomma#2,\@nil
  \let#1\KVO@result
}
\def\KVO@splitcomma#1,#2\@nil{%
  \KVO@ifempty{#1}{}{%
    \KVO@checkkv#1=\@nil
  }%
  \KVO@ifempty{#2}{}{\KVO@splitcomma#2\@nil}%
}
\def\KVO@ifempty#1{%
  \expandafter\ifx\expandafter\\\detokenize{#1}\\%
    \expandafter\@firstoftwo
  \else
    \expandafter\@secondoftwo
  \fi
}
\def\KVO@checkkv#1=#2\@nil{%
  \KVO@ifempty{#2}{%
    % option without value
    \edef\KVO@x{\zap@space#1 \@empty}%
    \ifx\KVO@x\@empty
      % ignore empty option
    \else
      % append to list
      \edef\KVO@result{%
        \etex@unexpanded\expandafter{\KVO@result},\KVO@x
      }%
    \fi
  }{%
    % #1: "key", #2: "value="
    % add key part
    \edef\KVO@result{%
      \etex@unexpanded\expandafter{\KVO@result},%
      \zap@space#1 \@empty
    }%
    \futurelet\@let@token\KVO@checkfirsttok#2 \@nil| = \@nil|\KVO@nil
  }%
}
\def\KVO@checkfirsttok{%
  \ifx\@let@token\bgroup
    % no space at start
    \expandafter\KVO@removelastspace\expandafter=%
    % "<value><spaceopt>= \@nil"
  \else
    \expandafter\KVO@checkfirstA
  \fi
}
\def\KVO@checkfirstA#1 #2\@nil{%
  \KVO@ifempty{#2}{%
    \KVO@removelastspace=#1 \@nil
  }{%
    \KVO@ifempty{#1}{%
      \KVO@removelastspace=#2\@nil
    }{%
      \KVO@removelastspace=#1 #2\@nil
    }%
  }%
}
\def\KVO@removelastspace#1 = \@nil|#2\KVO@nil{%
  \KVO@ifempty{#2}{%
    \edef\KVO@result{%
      \etex@unexpanded\expandafter{\KVO@result}%
      \etex@unexpanded\expandafter{\KVO@removegarbage#1\KVO@nil}%
    }%
  }{%
    \edef\KVO@result{%
      \etex@unexpanded\expandafter{\KVO@result}%
      \etex@unexpanded{#1}%
    }%
  }%
}
\def\KVO@removegarbage#1= \@nil#2\KVO@nil{#1}%
%    \end{macrocode}
%
%    Arguments |#1| and |#2| are macros.
%    \begin{macrocode}
\def\KVO@removeelement#1#2{%
  \begingroup
    \toks@={}%
    \@for\x:=#2\do{%
      \ifx\x\@empty
      \else
        \ifx\x#1\relax
        \else
          \edef\t{\the\toks@}%
          \ifx\t\@empty
          \else
            \toks@\expandafter{\the\toks@,}%
          \fi
          \toks@\expandafter{\the\expandafter\toks@\x}%
        \fi
      \fi
    }%
    \edef\x{\endgroup
      \def\noexpand#2{\the\toks@}%
    }%
  \x
}
%    \end{macrocode}
%
%
%    \begin{macrocode}
\let\@@fileswith@pti@ns\KVO@fileswith@pti@ns
\ifx\@fileswith@pti@ns\@badrequireerror
\else
  \let\@fileswith@pti@ns\KVO@fileswith@pti@ns
\fi
%    \end{macrocode}
%
%    \begin{macro}{\KVO@Patch}
%    \begin{macrocode}
\let\KVO@Patch=Y
%    \end{macrocode}
%    \end{macro}
%
%    \begin{macrocode}
\KVO@AtEnd%
%</patch>
%    \end{macrocode}
%
% \section{Test}
%
% \subsection{Preface for standard catcode check}
%
%    \begin{macrocode}
%<*test1>
\input miniltx.tex\relax
%</test1>
%    \end{macrocode}
%
% \subsection{Catcode checks for loading}
%
%    \begin{macrocode}
%<*test1>
%    \end{macrocode}
%    \begin{macrocode}
\catcode`\{=1 %
\catcode`\}=2 %
\catcode`\#=6 %
\catcode`\@=11 %
\expandafter\ifx\csname count@\endcsname\relax
  \countdef\count@=255 %
\fi
\expandafter\ifx\csname @gobble\endcsname\relax
  \long\def\@gobble#1{}%
\fi
\expandafter\ifx\csname @firstofone\endcsname\relax
  \long\def\@firstofone#1{#1}%
\fi
\expandafter\ifx\csname loop\endcsname\relax
  \expandafter\@firstofone
\else
  \expandafter\@gobble
\fi
{%
  \def\loop#1\repeat{%
    \def\body{#1}%
    \iterate
  }%
  \def\iterate{%
    \body
      \let\next\iterate
    \else
      \let\next\relax
    \fi
    \next
  }%
  \let\repeat=\fi
}%
\def\RestoreCatcodes{}
\count@=0 %
\loop
  \edef\RestoreCatcodes{%
    \RestoreCatcodes
    \catcode\the\count@=\the\catcode\count@\relax
  }%
\ifnum\count@<255 %
  \advance\count@ 1 %
\repeat

\def\RangeCatcodeInvalid#1#2{%
  \count@=#1\relax
  \loop
    \catcode\count@=15 %
  \ifnum\count@<#2\relax
    \advance\count@ 1 %
  \repeat
}
\def\RangeCatcodeCheck#1#2#3{%
  \count@=#1\relax
  \loop
    \ifnum#3=\catcode\count@
    \else
      \errmessage{%
        Character \the\count@\space
        with wrong catcode \the\catcode\count@\space
        instead of \number#3%
      }%
    \fi
  \ifnum\count@<#2\relax
    \advance\count@ 1 %
  \repeat
}
\def\space{ }
\expandafter\ifx\csname LoadCommand\endcsname\relax
  \def\LoadCommand{\input kvoptions.sty\relax}%
\fi
\def\Test{%
  \RangeCatcodeInvalid{0}{47}%
  \RangeCatcodeInvalid{58}{64}%
  \RangeCatcodeInvalid{91}{96}%
  \RangeCatcodeInvalid{123}{255}%
  \catcode`\@=12 %
  \catcode`\\=0 %
  \catcode`\%=14 %
  \LoadCommand
  \RangeCatcodeCheck{0}{36}{15}%
  \RangeCatcodeCheck{37}{37}{14}%
  \RangeCatcodeCheck{38}{47}{15}%
  \RangeCatcodeCheck{48}{57}{12}%
  \RangeCatcodeCheck{58}{63}{15}%
  \RangeCatcodeCheck{64}{64}{12}%
  \RangeCatcodeCheck{65}{90}{11}%
  \RangeCatcodeCheck{91}{91}{15}%
  \RangeCatcodeCheck{92}{92}{0}%
  \RangeCatcodeCheck{93}{96}{15}%
  \RangeCatcodeCheck{97}{122}{11}%
  \RangeCatcodeCheck{123}{255}{15}%
  \RestoreCatcodes
}
\Test
\csname @@end\endcsname
\end
%    \end{macrocode}
%    \begin{macrocode}
%</test1>
%    \end{macrocode}
%
%    \begin{macrocode}
%<*test2>
\NeedsTeXFormat{LaTeX2e}
\makeatletter
\catcode`\@=11 %
\def\RestoreCatcodes{}
\count@=0 %
\loop
  \edef\RestoreCatcodes{%
    \RestoreCatcodes
    \catcode\the\count@=\the\catcode\count@\relax
  }%
\ifnum\count@<255 %
  \advance\count@\@ne
\repeat

\def\RangeCatcodeInvalid#1#2{%
  \count@=#1\relax
  \loop
    \catcode\count@=15 %
  \ifnum\count@<#2\relax
    \advance\count@\@ne
  \repeat
}
\def\Test#1{%
  \RangeCatcodeInvalid{0}{47}%
  \RangeCatcodeInvalid{58}{64}%
  \RangeCatcodeInvalid{91}{96}%
  \RangeCatcodeInvalid{123}{255}%
  \catcode`\@=12 %
  \catcode`\\=0 %
  \catcode`\{=1 %
  \catcode`\}=2 %
  \catcode`\#=6 %
  \catcode`\[=12 %
  \catcode`\]=12 %
  \catcode`\%=14 %
  \catcode`\ =10 %
  \catcode13=5 %
  #1\relax
  \RestoreCatcodes
}
\Test{\RequirePackage{kvoptions-patch}}%
\Test{\RequirePackage{kvoptions}}%
\csname @@end\endcsname
%</test2>
%    \end{macrocode}
%
%    \begin{macrocode}
%<*test3>
\NeedsTeXFormat{LaTeX2e}
\makeatletter
\RequirePackage{kvoptions}[2016/05/16]
\def\msg#{\immediate\write16}
\define@key{testfamily}{testkey}{%
  \msg{[testfamily/testkey/#1]}%
}
\define@key{testfamily}{testdefaultkey}[testdefault]{%
  \msg{[testfamily/testdefaultkey/#1]}%
}
\AddToKeyvalOption{testfamily}{testkey}{%
  \msg{[addition/#1]}%
}
\AddToKeyvalOption{testfamily}{testdefaultkey}{%
  \msg{[addition/#1]}%
}
\setkeys{testfamily}{%
  testkey=testA,%
  testdefaultkey=testB,%
  testdefaultkey,%
}
\SetupKeyvalOptions{%
  family=testfamily%
}
\AddToKeyvalOption*{testkey}{%
  \msg{[star addition/#1]}%
}
\AddToKeyvalOption*{testdefaultkey}{%
  \msg{[star addition/#1]}%
}
\setkeys{testfamily}{%
  testkey=testA,%
  testdefaultkey=testB,%
  testdefaultkey,%
}
\@@end
%</test3>
%    \end{macrocode}
%
%    \begin{macrocode}
%<*test4pkg>
\NeedsTeXFormat{LaTeX2e}
\ProvidesPackage{kvoptions-test4}[2016/05/16 package for testing]
\RequirePackage{kvoptions}[2016/05/16]
\SetupKeyvalOptions{%
  family=FOO,%
  prefix=foo,%
  setkeys=\kvsetkeys,%
}
\DeclareStringOption{str}
\define@key{FOO}{set}{%
  \setkeys{BAR}{strbar={#1}}%
}
\define@key{BAR}{strbar}{%
  \def\foostr{[BAR:#1]}%
}
\ProcessKeyvalOptions*
%</test4pkg>
%    \end{macrocode}
%    \begin{macrocode}
%<*test4>
\NeedsTeXFormat{LaTeX2e}
\ProvidesFile{kvoptions-test4.tex}[2016/05/16 test file]
\RequirePackage[%
  str=A,set=B,str=C,%
]{kvoptions-test4}[2016/05/16]
\def\TestExpected{C}
\ifx\foostr\TestExpected
  \typeout{* Test ok.}%
\else
  \typeout{* Result: [\foostr]}%
  \typeout{* Expected: [\TestExpected]}%
  \errmessage{Test failed!}%
\fi
\csname @@end\endcsname\end
%</test4>
%    \end{macrocode}
%
% \section{Installation}
%
% \subsection{Download}
%
% \paragraph{Package.} This package is available on
% CTAN\footnote{\CTANpkg{kvoptions}}:
% \begin{description}
% \item[\CTAN{macros/latex/contrib/oberdiek/kvoptions.dtx}] The source file.
% \item[\CTAN{macros/latex/contrib/oberdiek/kvoptions.pdf}] Documentation.
% \end{description}
%
%
% \paragraph{Bundle.} All the packages of the bundle `oberdiek'
% are also available in a TDS compliant ZIP archive. There
% the packages are already unpacked and the documentation files
% are generated. The files and directories obey the TDS standard.
% \begin{description}
% \item[\CTANinstall{install/macros/latex/contrib/oberdiek.tds.zip}]
% \end{description}
% \emph{TDS} refers to the standard ``A Directory Structure
% for \TeX\ Files'' (\CTAN{tds/tds.pdf}). Directories
% with \xfile{texmf} in their name are usually organized this way.
%
% \subsection{Bundle installation}
%
% \paragraph{Unpacking.} Unpack the \xfile{oberdiek.tds.zip} in the
% TDS tree (also known as \xfile{texmf} tree) of your choice.
% Example (linux):
% \begin{quote}
%   |unzip oberdiek.tds.zip -d ~/texmf|
% \end{quote}
%
% \paragraph{Script installation.}
% Check the directory \xfile{TDS:scripts/oberdiek/} for
% scripts that need further installation steps.
%
% \subsection{Package installation}
%
% \paragraph{Unpacking.} The \xfile{.dtx} file is a self-extracting
% \docstrip\ archive. The files are extracted by running the
% \xfile{.dtx} through \plainTeX:
% \begin{quote}
%   \verb|tex kvoptions.dtx|
% \end{quote}
%
% \paragraph{TDS.} Now the different files must be moved into
% the different directories in your installation TDS tree
% (also known as \xfile{texmf} tree):
% \begin{quote}
% \def\t{^^A
% \begin{tabular}{@{}>{\ttfamily}l@{ $\rightarrow$ }>{\ttfamily}l@{}}
%   kvoptions.sty & tex/latex/oberdiek/kvoptions.sty\\
%   kvoptions-patch.sty & tex/latex/oberdiek/kvoptions-patch.sty\\
%   kvoptions.pdf & doc/latex/oberdiek/kvoptions.pdf\\
%   example-mycolorsetup.sty & doc/latex/oberdiek/example-mycolorsetup.sty\\
%   test/kvoptions-test1.tex & doc/latex/oberdiek/test/kvoptions-test1.tex\\
%   test/kvoptions-test2.tex & doc/latex/oberdiek/test/kvoptions-test2.tex\\
%   test/kvoptions-test3.tex & doc/latex/oberdiek/test/kvoptions-test3.tex\\
%   test/kvoptions-test4.tex & doc/latex/oberdiek/test/kvoptions-test4.tex\\
%   test/kvoptions-test4.sty & doc/latex/oberdiek/test/kvoptions-test4.sty\\
%   kvoptions.dtx & source/latex/oberdiek/kvoptions.dtx\\
% \end{tabular}^^A
% }^^A
% \sbox0{\t}^^A
% \ifdim\wd0>\linewidth
%   \begingroup
%     \advance\linewidth by\leftmargin
%     \advance\linewidth by\rightmargin
%   \edef\x{\endgroup
%     \def\noexpand\lw{\the\linewidth}^^A
%   }\x
%   \def\lwbox{^^A
%     \leavevmode
%     \hbox to \linewidth{^^A
%       \kern-\leftmargin\relax
%       \hss
%       \usebox0
%       \hss
%       \kern-\rightmargin\relax
%     }^^A
%   }^^A
%   \ifdim\wd0>\lw
%     \sbox0{\small\t}^^A
%     \ifdim\wd0>\linewidth
%       \ifdim\wd0>\lw
%         \sbox0{\footnotesize\t}^^A
%         \ifdim\wd0>\linewidth
%           \ifdim\wd0>\lw
%             \sbox0{\scriptsize\t}^^A
%             \ifdim\wd0>\linewidth
%               \ifdim\wd0>\lw
%                 \sbox0{\tiny\t}^^A
%                 \ifdim\wd0>\linewidth
%                   \lwbox
%                 \else
%                   \usebox0
%                 \fi
%               \else
%                 \lwbox
%               \fi
%             \else
%               \usebox0
%             \fi
%           \else
%             \lwbox
%           \fi
%         \else
%           \usebox0
%         \fi
%       \else
%         \lwbox
%       \fi
%     \else
%       \usebox0
%     \fi
%   \else
%     \lwbox
%   \fi
% \else
%   \usebox0
% \fi
% \end{quote}
% If you have a \xfile{docstrip.cfg} that configures and enables \docstrip's
% TDS installing feature, then some files can already be in the right
% place, see the documentation of \docstrip.
%
% \subsection{Refresh file name databases}
%
% If your \TeX~distribution
% (\TeX\,Live, \mikTeX, \dots) relies on file name databases, you must refresh
% these. For example, \TeX\,Live\ users run \verb|texhash| or
% \verb|mktexlsr|.
%
% \subsection{Some details for the interested}
%
% \paragraph{Unpacking with \LaTeX.}
% The \xfile{.dtx} chooses its action depending on the format:
% \begin{description}
% \item[\plainTeX:] Run \docstrip\ and extract the files.
% \item[\LaTeX:] Generate the documentation.
% \end{description}
% If you insist on using \LaTeX\ for \docstrip\ (really,
% \docstrip\ does not need \LaTeX), then inform the autodetect routine
% about your intention:
% \begin{quote}
%   \verb|latex \let\install=y% \iffalse meta-comment
%
% File: kvoptions.dtx
% Version: 2016/05/16 v3.12
% Info: Key value format for package options
%
% Copyright (C)
%    2004, 2006, 2007, 2009-2011 Heiko Oberdiek
%    2016-2019 Oberdiek Package Support Group
%    https://github.com/ho-tex/oberdiek/issues
%
% This work may be distributed and/or modified under the
% conditions of the LaTeX Project Public License, either
% version 1.3c of this license or (at your option) any later
% version. This version of this license is in
%    https://www.latex-project.org/lppl/lppl-1-3c.txt
% and the latest version of this license is in
%    https://www.latex-project.org/lppl.txt
% and version 1.3 or later is part of all distributions of
% LaTeX version 2005/12/01 or later.
%
% This work has the LPPL maintenance status "maintained".
%
% The Current Maintainers of this work are
% Heiko Oberdiek and the Oberdiek Package Support Group
% https://github.com/ho-tex/oberdiek/issues
%
% This work consists of the main source file kvoptions.dtx
% and the derived files
%    kvoptions.sty, kvoptions.pdf, kvoptions.ins, kvoptions.drv,
%    kvoptions-patch.sty, example-mycolorsetup.sty,
%    kvoptions-test1.tex, kvoptions-test2.tex,
%    kvoptions-test3.tex, kvoptions-test4.tex,
%    kvoptions-test4.sty.
%
% Distribution:
%    CTAN:macros/latex/contrib/oberdiek/kvoptions.dtx
%    CTAN:macros/latex/contrib/oberdiek/kvoptions.pdf
%
% Unpacking:
%    (a) If kvoptions.ins is present:
%           tex kvoptions.ins
%    (b) Without kvoptions.ins:
%           tex kvoptions.dtx
%    (c) If you insist on using LaTeX
%           latex \let\install=y\input{kvoptions.dtx}
%        (quote the arguments according to the demands of your shell)
%
% Documentation:
%    (a) If kvoptions.drv is present:
%           latex kvoptions.drv
%    (b) Without kvoptions.drv:
%           latex kvoptions.dtx; ...
%    The class ltxdoc loads the configuration file ltxdoc.cfg
%    if available. Here you can specify further options, e.g.
%    use A4 as paper format:
%       \PassOptionsToClass{a4paper}{article}
%
%    Programm calls to get the documentation (example):
%       pdflatex kvoptions.dtx
%       makeindex -s gind.ist kvoptions.idx
%       pdflatex kvoptions.dtx
%       makeindex -s gind.ist kvoptions.idx
%       pdflatex kvoptions.dtx
%
% Installation:
%    TDS:tex/latex/oberdiek/kvoptions.sty
%    TDS:tex/latex/oberdiek/kvoptions-patch.sty
%    TDS:doc/latex/oberdiek/kvoptions.pdf
%    TDS:doc/latex/oberdiek/example-mycolorsetup.sty
%    TDS:doc/latex/oberdiek/test/kvoptions-test1.tex
%    TDS:doc/latex/oberdiek/test/kvoptions-test2.tex
%    TDS:doc/latex/oberdiek/test/kvoptions-test3.tex
%    TDS:doc/latex/oberdiek/test/kvoptions-test4.tex
%    TDS:doc/latex/oberdiek/test/kvoptions-test4.sty
%    TDS:source/latex/oberdiek/kvoptions.dtx
%
%<*ignore>
\begingroup
  \catcode123=1 %
  \catcode125=2 %
  \def\x{LaTeX2e}%
\expandafter\endgroup
\ifcase 0\ifx\install y1\fi\expandafter
         \ifx\csname processbatchFile\endcsname\relax\else1\fi
         \ifx\fmtname\x\else 1\fi\relax
\else\csname fi\endcsname
%</ignore>
%<*install>
\input docstrip.tex
\Msg{************************************************************************}
\Msg{* Installation}
\Msg{* Package: kvoptions 2016/05/16 v3.12 Key value format for package options (HO)}
\Msg{************************************************************************}

\keepsilent
\askforoverwritefalse

\let\MetaPrefix\relax
\preamble

This is a generated file.

Project: kvoptions
Version: 2016/05/16 v3.12

Copyright (C)
   2004, 2006, 2007, 2009-2011 Heiko Oberdiek
   2016-2019 Oberdiek Package Support Group

This work may be distributed and/or modified under the
conditions of the LaTeX Project Public License, either
version 1.3c of this license or (at your option) any later
version. This version of this license is in
   https://www.latex-project.org/lppl/lppl-1-3c.txt
and the latest version of this license is in
   https://www.latex-project.org/lppl.txt
and version 1.3 or later is part of all distributions of
LaTeX version 2005/12/01 or later.

This work has the LPPL maintenance status "maintained".

The Current Maintainers of this work are
Heiko Oberdiek and the Oberdiek Package Support Group
https://github.com/ho-tex/oberdiek/issues


This work consists of the main source file kvoptions.dtx
and the derived files
   kvoptions.sty, kvoptions.pdf, kvoptions.ins, kvoptions.drv,
   kvoptions-patch.sty, example-mycolorsetup.sty,
   kvoptions-test1.tex, kvoptions-test2.tex,
   kvoptions-test3.tex, kvoptions-test4.tex,
   kvoptions-test4.sty.

\endpreamble
\let\MetaPrefix\DoubleperCent

\generate{%
  \file{kvoptions.ins}{\from{kvoptions.dtx}{install}}%
  \file{kvoptions.drv}{\from{kvoptions.dtx}{driver}}%
  \usedir{tex/latex/oberdiek}%
  \file{kvoptions.sty}{\from{kvoptions.dtx}{package}}%
  \file{kvoptions-patch.sty}{\from{kvoptions.dtx}{patch}}%
  \usedir{doc/latex/oberdiek}%
  \file{example-mycolorsetup.sty}{\from{kvoptions.dtx}{example}}%
%  \usedir{doc/latex/oberdiek/test}%
%  \file{kvoptions-test1.tex}{\from{kvoptions.dtx}{test1}}%
%  \file{kvoptions-test2.tex}{\from{kvoptions.dtx}{test2}}%
%  \file{kvoptions-test3.tex}{\from{kvoptions.dtx}{test3}}%
%  \file{kvoptions-test4.tex}{\from{kvoptions.dtx}{test4}}%
%  \file{kvoptions-test4.sty}{\from{kvoptions.dtx}{test4pkg}}%
  \nopreamble
  \nopostamble
%  \usedir{source/latex/oberdiek/catalogue}%
%  \file{kvoptions.xml}{\from{kvoptions.dtx}{catalogue}}%
}

\catcode32=13\relax% active space
\let =\space%
\Msg{************************************************************************}
\Msg{*}
\Msg{* To finish the installation you have to move the following}
\Msg{* files into a directory searched by TeX:}
\Msg{*}
\Msg{*     kvoptions.sty, kvoptions-patch.sty}
\Msg{*}
\Msg{* To produce the documentation run the file `kvoptions.drv'}
\Msg{* through LaTeX.}
\Msg{*}
\Msg{* Happy TeXing!}
\Msg{*}
\Msg{************************************************************************}

\endbatchfile
%</install>
%<*ignore>
\fi
%</ignore>
%<*driver>
\NeedsTeXFormat{LaTeX2e}
\ProvidesFile{kvoptions.drv}%
  [2016/05/16 v3.12 Key value format for package options (HO)]%
\documentclass{ltxdoc}
\usepackage{holtxdoc}[2011/11/22]
\begin{document}
  \DocInput{kvoptions.dtx}%
\end{document}
%</driver>
% \fi
%
%
% \CharacterTable
%  {Upper-case    \A\B\C\D\E\F\G\H\I\J\K\L\M\N\O\P\Q\R\S\T\U\V\W\X\Y\Z
%   Lower-case    \a\b\c\d\e\f\g\h\i\j\k\l\m\n\o\p\q\r\s\t\u\v\w\x\y\z
%   Digits        \0\1\2\3\4\5\6\7\8\9
%   Exclamation   \!     Double quote  \"     Hash (number) \#
%   Dollar        \$     Percent       \%     Ampersand     \&
%   Acute accent  \'     Left paren    \(     Right paren   \)
%   Asterisk      \*     Plus          \+     Comma         \,
%   Minus         \-     Point         \.     Solidus       \/
%   Colon         \:     Semicolon     \;     Less than     \<
%   Equals        \=     Greater than  \>     Question mark \?
%   Commercial at \@     Left bracket  \[     Backslash     \\
%   Right bracket \]     Circumflex    \^     Underscore    \_
%   Grave accent  \`     Left brace    \{     Vertical bar  \|
%   Right brace   \}     Tilde         \~}
%
% \GetFileInfo{kvoptions.drv}
%
% \title{The \xpackage{kvoptions} package}
% \date{2016/05/16 v3.12}
% \author{Heiko Oberdiek\thanks
% {Please report any issues at \url{https://github.com/ho-tex/oberdiek/issues}}}
%
% \maketitle
%
% \begin{abstract}
% This package is intended for package authors who want to
% use options in key value format for their package options.
% \end{abstract}
%
% \tableofcontents
%
% \def\M#1{\texttt{\{}\meta{#1}\texttt{\}}}
% \def\O#1{\texttt{[}\meta{#1}\texttt{]}}
%
% \section{Introduction}
%
% First I want to recommend the very good review article
% ``A guide to key-value methods'' by Joseph Wright \cite{tb94wright}.
% It introduces the different key-value packages and compares them.
%
% \subsection{The beginning}
%
% This package \xpackage{kvoptions} addresses class or package writers
% that want to allow their users to specify options
% as key value pairs, e.g.:
% \begin{quote}
% |\documentclass[verbose=false,name=me]{myclass}|\\
% |\usepackage[format=print]{mylayout}|
% \end{quote}
% Prominent example is package \xpackage{hyperref},
% probably the first package that offers this service.
% It's \cs{ProcessOptionsWithKV} is often copied und
% used in other packages, e.g. package \xpackage{helvet}
% that uses this interface for its option \xoption{scaled}.
%
% However copying code is not the most modern software
% development technique. And \xpackage{hyperref}'s
% code for \cs{ProcessOptionsWithKV} was changed
% to fix bugs. The version used in other packages
% depends on the time of copying and the awareness
% of \xpackage{hyperref}'s changes.
% Now the code is sourced out into this package and
% available for other package or class writers.
%
% \subsection{Overview}
%
% Package \xpackage{kvoptions} connects package
% \xpackage{\emph{k}ey\emph{v}al}
% with \LaTeX's package and class \emph{options}:
% \begin{center}
% \def\T#1{\bfseries #1}
% \renewcommand{\arraystretch}{1.5}
% \begin{tabular}{l|l|l}
%   \T{Package \xpackage{keyval}}
%   & \T{Package \xpackage{kvoptions}}
%   & \T{\LaTeX\ kernel}
%   \\
% \hline
%     \cs{define@key}
%   &
%     \renewcommand*{\arraystretch}{1}^^A
%     \begin{tabular}[t]{@{}l@{}}
%       \cs{DeclareVoidOption}\\
%       \cs{DeclareStringOption}\\
%       \cs{DeclareBoolOption}\\
%       \cs{DeclareComplementaryOption}\\
%       \cs{DisableKeyvalOption}
%     \end{tabular}
%   & \cs{DeclareOption}
%   \\
% \hline
%   & \cs{DeclareDefaultOption}
%   & \cs{DeclareOption*}
%   \\
% \hline
%   & \cs{ProcessKeyvalOptions}
%   & \cs{ProcessOptions*}
%   \\
% \hline
%   & Option \xoption{patch}
%   &
%     \renewcommand*{\arraystretch}{1}^^A
%     \begin{tabular}[t]{@{}l@{}}
%       Class/package\\
%       option system
%     \end{tabular}
%   \\
% \hline
%   & \cs{SetupKeyvalOptions} &
%   \\
% \end{tabular}
% \end{center}
%
% \hypersetup{bookmarksopenlevel=3}
% \section{Usage}
%
% \subsection{Process options}
%
% \subsubsection{\cs{ProcessKeyvalOptions}}
%
% \begin{declcs}{ProcessKeyvalOptions}\,\M{family}\\
%   \cs{ProcessKeyvalOptions} \*
% \end{declcs}
% This command evaluates the global or local options of
% the package that are defined with \xpackage{keyval}'s
% interface within the family \meta{family}. It acts the
% same way as \LaTeX's \cs{ProcessOptions*}. In a package
% unknown global options are ignored, in a class they
% are added to the unknown option list. The known global
% options and all local options are passed to \xpackage{keyval}'s
% \cs{setkeys} command for executing the options.
% Unknown options are reported to the user by an error.
%
% If the family name happens to be the same as the name
% of the package or class where \cs{ProcessKeyvalOptions}
% is used or the family name has previously been setup by
% \cs{SetupKeyvalOptions}, then \cs{ProcessKeyvalOptions}
% knows the family name already and you can use the star form
% without mandatory argument.
%
% \subsubsection{\cs{ProcessLocalKeyvalOptions}}
%
% \begin{declcs}{ProcessLocalKeyvalOptions}\,\M{family}\\
%   \cs{ProcessLocalKeyvalOptions} \*
% \end{declcs}
% This macro has the same syntax and works similar as
% \cs{ProcessKeyvalOptions}. However it ignores
% global options and only processes the local package options.
% Therefore it only can be used inside a package.
% An error is thrown, if it is used inside a class.
%
% Neither of the following macros are necessary for
% \cs{ProcessKeyvalOptions{}}. They just help the package/class
% author in common tasks.
%
% \subsubsection{\cs{SetupKeyvalOptions}}\label{setup}
%
% \begin{declcs}{SetupKeyvalOptions}\,\texttt{\{}\\
%   \hspace{1.5em}|family|\,|=|\,\meta{family}|,|\\
%   \hspace{1.5em}|prefix|\,|=|\,\meta{prefix}\\
%   \hspace{1.5em}|setkeys|\,|=|\,\meta{setkeys command}\\
%   \texttt{\}}
% \end{declcs}
% This command allows to configure the default assumptions
% that are based on the current package or class name.
% \LaTeX\ remembers this name in \cs{@currname}. The syntax
% description of the default looks a little weird, therefor
% an example is given for a package or class named |foobar|.
% \begin{quote}
% \begin{tabular}{@{}l|ll|l@{}}
%   Key & Default & (example) & Used by\\
% \hline
%   \rule{0pt}{1.2\ht\csname @arstrutbox\endcsname}^^A
%   |family| & \meta{\cs{@currname}} & (|foobar|)
%       & \cs{ProcessKeyvalOptions*} \\
%     &&& \cs{DeclareBoolOption} \\
%     &&& \cs{DeclareStringOption} \\
% \hline
%   \rule{0pt}{1.2\ht\csname @arstrutbox\endcsname}^^A
%   |prefix| & \meta{\cs{@currname}}|@| & (|foobar@|)
%       & \cs{DeclareBoolOption} \\
%     &&& \cs{DeclareStringOption} \\
%     &&& \cs{DeclareVoidOption} \\
% \hline
%   \rule{0pt}{1.2\ht\csname @arstrutbox\endcsname}^^A
%   |setkeys| & \cs{setkeys} & (\cs{kvsetkeys})
%       & \cs{ProcessKeyvalOptions} \\
%     &&& \cs{ProcessLocalKeyvalOptions}
% \end{tabular}
% \end{quote}
% Key |setkeys| was added in version 3.9. The original
% \cs{setkeys} of package \xpackage{keyval} is not reentrant.
% If an option is processed by this \cs{setkeys}, then
% the option should not call \cs{setkeys} again with a
% different family. Otherwise the next options of the
% first \cs{setkeys} call are processed with the wrong
% family. With key |setkeys| the macro \cs{kvsetkeys}
% can be set that does not have the problem of the
% original \cs{setkeys} of package {keyval}.
%
% Probably \cs{setkeys} of package \xpackage{xkeyval}
% is safe in this respect. But I haven't made a full
% analysis. At least it does not have the problem
% of the original \cs{setkeys}.
%
% \subsection{Option declarations}
%
% The options for \cs{ProcessKeyvalOptions} are defined
% by \xpackage{keyval}'s \cs{define@key}.
% Common purposes of such keys are boolean switches, they
% enable or disable something. Or they store a name or some kind of
% string in a macro. The following commands help the user. He
% declares what he wants and \xpackage{kvoptions} take care of
% the key definition, resource allocation and initialization.
%
% In order to avoid name clashes of macro names, internal
% commands are prefixed. Both the prefix and the family
% name for the defined keys can be configured by
% \cs{SetupKeyvalOptions}.
%
% \subsubsection{\cs{DeclareStringOption}}
%
% \begin{declcs}{DeclareStringOption}\,\O{init}\,\M{key}\,\O{default}
% \end{declcs}
% A macro is created that remembers the value of the key \meta{key}.
% The name of the macro consists of the option name \meta{key} that
% is prefixed by the prefix (see \ref{setup}).
% The initial contents of the
% macro can be given by the first optional argument \meta{init}.
% The default is empty.
%
% The the option \meta{key} is defined. The option code
% just stores its value in the macro. If the optional argument
% at the end of \cs{DeclareStringOption} is given, then option
% \meta{key} is defined with the default \meta{default}.
%
% Example for a package with the following two lines:
% \begin{quote}
% |\ProvidesPackage{foobar}|\\
% |\DeclareStringOption[me]{name}|
% \end{quote}
% Then \cs{DeclareStringOption} defines the macro with content |me|,
% note \LaTeX\ complains if the name of the macro already exists:
% \begin{quote}
% |\newcommand*{\foobar@name}{me}|
% \end{quote}
% The option definition is similar to:
% \begin{quote}
% |\define@key{foobar}{name}{%|\\
% |  \renewcommand*{\foobar@name}{#1}%|\\
% |}|
% \end{quote}
%
% \subsubsection{\cs{DeclareBoolOption}}
%
% \begin{declcs}{DeclareBoolOption}\,\O{init}\,\M{key}
% \end{declcs}
% A boolean switch is generated, initialized by value \meta{init}
% and the corresponding key \meta{key} is defined. If the
% initialization value is not given, |false| is used as default.
%
% The internal actions of \cs{DeclareBoolOption} are shown below.
% The example is given for a package author who has the following
% two lines in his package/class:
% \begin{quote}
% |\ProvidesPackage{foobar}|\\
% |\DeclareBoolOption{verbose}|
% \end{quote}
% First a new switch is created:
% \begin{quote}
%   |\newif\iffoobar@verbose|
% \end{quote}
% and initialized:
% \begin{quote}
%   |\foobar@verbosefalse|
% \end{quote}
% Finally the key is defined:
% \begin{quote}
%   |\define@key{foobar}{verbose}[true]{|\dots|}|
% \end{quote}
% The option code configures the boolean option in the following way:
% If the author specifies |true| or |false| then the switch is
% turned on or off respectivly. Also the option can be given without
% explicit value. Then the switch is enabled.
% Other values are reported as errors.
%
% Now the switch is ready to use in the package/class, e.g.:
% \begin{quote}
%   |\iffoobar@verbose|\\
%   \qquad\textit{\% print verbose message}\\
%   |\else|\\
%   \qquad\textit{\% be quiet}\\
%   |\fi|
% \end{quote}
% Users of package \cs{ifthen} can use the switch as
% boolean:
% \begin{quote}
%   |\boolean{foobar@verbose}|
% \end{quote}
%
% \subsubsection{\cs{DeclareComplementaryOption}}
%
% \begin{declcs}{DeclareComplementaryOption}\,\M{key}\,\M{parent}
% \end{declcs}
% Sometimes contrasting names are used to characterize the
% two states of a boolean switch, for example \xoption{draft}
% vs.\@ \xoption{final}. Both options behave like boolean
% options but they do not need two different switches, they
% should share one. \cs{DeclareComplementaryOption} allows this.
% The option \meta{key} shares the switch of option \meta{parent}.
% Example:
% \begin{quote}
%   |\DeclareBoolOption{draft}|\\
%   |\DeclareComplementaryOption{final}{draft}|
% \end{quote}
% Then |final| sets the switch of \xoption{draft} to |false|, and
% |final=false| enables the \xoption{draft} switch.
%
% \subsubsection{\cs{DeclareVoidOption}}
%
% \begin{declcs}{DeclareVoidOption}\,\M{key}\,\M{code}
% \end{declcs}
% \cs{ProcessKeyvalOptions} can be extended to recognize
% options that are declared in traditional way by \cs{DeclareOption}.
% But in case of the error that the user specifies a value, then
% this option would not recognized as key value option because
% of \cs{DeclareOption} and not detected as traditional option
% because of the value part. The user would get an unknown option
% error, difficult to understand.
%
% \cs{DeclareVoidOption} solves
% this problem. It defines the option \meta{key} as key value option.
% If the user specifies a value, a warning is given and the
% value is ignored.
%
% The code part \meta{code} is stored in a macro.
% The name of the macro consists of the option name \meta{key} that
% is prefixed by the prefix (see \ref{setup}).
% If the option is set, the macro will be executed. During the
% execution \cs{CurrentOption} is available with the current key name.
%
% \subsubsection{\cs{DeclareDefaultOption}}
%
% \begin{declcs}{DeclareDefaultOption}\,\M{code}
% \end{declcs}
% \begin{sloppypar}
% This command does not define a specific key, it is the equivalent
% to \LaTeX's \cs{DeclareOption*}. It allows the specification
% of a default action \meta{code} that is invoked if an unknown option
% is found.
% While \meta{code} is called,
% macro \cs{CurrentOption} contains the current option string.
% In addition \cs{CurrentOptionValue} contains the value part
% if the option string is parsable as key value pair, otherwise
% it is \cs{relax}. \cs{CurrentOptionKey} contains the key of
% the key value pair, or the whole option string, if it misses
% the equal sign.
% \end{sloppypar}
%
% Inside packages typical default actions are to pass unknown
% options to another package. Or an error message can be thrown
% by \cs{@unknownoptionerror}. This is the original error
% message that \LaTeX\ gives for unkown package options.
% This error message is easier to understand for the user
% as the error message from package \xpackage{keyval} that
% is given otherwise.
%
% A Class ignores unknown options and puts them on the
% unused option list. Let \LaTeX\ do the job and just
% call \cs{OptionNotUsed}. Or the options can be passed to
% another class that is later loaded.
%
% \subsubsection{Local options}
%
% \begin{declcs}{DeclareLocalOption} \M{option}\\
%   \cs{DeclareLocalOptions} \M{option list}
% \end{declcs}
% Both macros mark package options as local options. That means
% that they are ignored by \cs{ProcessKeyvalOptions} if they are given
% as global options. \cs{DeclareLocalOptions} takes one option,
% \cs{DeclareLocalOptions} expects a comma separated list of options.
%
% \subsubsection{Dynamic options}
%
% Options of \LaTeX's package/class system are cleared
% in \cs{ProcessOptions}. They modify the static model
% of a package. For example, depending on option \xoption{bookmarks}
% package \xpackage{hyperref} loads differently.
%
% Options, however,
% defined by \xpackage{keyval}'s \cs{define@key} remain
% defined, if the options are processed by \cs{setkeys}.
% Therefore these options can also be used to model
% the dynamic behaviour of a package. For example, in
% \xpackage{hyperref} the link colors can be changed
% everywhere until the end in |\end{document}|.
%
% However package \xpackage{color} that adds color support
% is necessary and it
% cannot be loaded after |\begin{document}|. Option \xoption{colorlinks}
% that loads \xpackage{color} should be active until |\begin{document}|
% and die in some way if it is too late for loading packages.
% With \cs{DisableKeyvalOption} the package/class author can
% specify and configure the death of an option and controls
% the life period of the option.
%
% \subsubsection{\cs{DisableKeyvalOption}}
%
% \begin{declcs}{DisableKeyvalOption}\,\O{options}\,\M{family}\,\M{key}
%   \\[0.5ex]
%   \meta{options}:\\
%   \hspace{1.5em}^^A
%   \begin{tabular}[t]{@{}l@{ \texttt{=} }l@{\qquad}>{default: }l@{}}
%     |action| & |undef|, |warning|, |error|, or |ignore|
%     & |undef|
%     \\
%     \multicolumn{1}{@{}l}{|global| or |local|} & & |global|
%     \\
%     |package| or |class| & \meta{name}
%     \\
%   \end{tabular}^^A
% \end{declcs}
% \cs{DisableKeyvalOption} can be called to mark the end when
% the option \meta{key} is no longer useful. The behaviour of
% an option after its death can be configured by \xoption{action}:
% \begin{description}
% \item[\texttt{undef}:] The option will be undefined,
%   If it is called, \cs{setkeys} reports an error because
%   of unknown key.
% \item[\texttt{error} or \texttt{warning}:] Use of the
%  option will cause an error or warning message.
%  Also these actions require that exclusivly either the
%  package or class name is given in options \xoption{package}
%  or \xoption{class}.
% \item[\texttt{ignore}:] The use of the option will silently
%  be ignored.
% \end{description}
% The option's death can be limited to the end of the current
% group, if option \xoption{local} is given. Default is
% \xoption{global}.
%
% The package/class author can wish the end of the option
% already during the package loading, then he will have
% static behaviour. In case of dynamic options \cs{DisableKeyvalOption}
% can be executed everywhere, also outside the package. Therefore
% the family name and the package/class name is usually unknown for
% \cs{DisableKeyvalOption}. Therefore the argument for the family
% name is mandatory and for some actions the package/class name
% must be provided.
%
% Usually a macro would configure the option death, Example:
%\begin{quote}
%\begin{verbatim}
%\ProvidesPackage{foobar}
%\DeclareBoolOption{color}
%\DeclareStringOption[red]{emphcolor}
%\ProcessKeyvalOptions*
%
%\newcommand*{\foobar@DisableOption}[2]{%
%  \DisableKeyvalueOption[
%    action={#1},
%    package=foobar
%  ]{foobar}{#2}%
%}
%
%\iffoobar@color
%  \RequirePackage{color}
%  \renewcommand*{\emph}[1]{\textcolor{\foobar@emphcolor}{#1}}
%\else
%  % Option emphcolor is not wrong, if we have color support.
%  % otherwise the option has no effect, but we don't want to
%  % remove it. Therefore action 'ignore' is the best choice:
%  \foobar@DisableOption{ignore}{emphcolor}
%\fi
%% No we don't need the option 'color'.
%\foobar@DisableOption{warning}{color}
%
%% With color support option 'emphcolor' will dynamically
%% change the color of \emph statements.
%\end{verbatim}
%\end{quote}
%
% \subsubsection{\cs{AddToKeyvalOption}}
%
% \begin{declcs}{AddToKeyvalOption}\,\M{family}\,\M{key}\,\M{code}\\
%   \cs{AddToKeyvalOption} \*\,\M{key}\,\M{code}
% \end{declcs}
% The code for an existing key \meta{key} of family \meta{family} is
% extended by code \meta{code}. In the starred form the current
% family setting is used, see \cs{ProcessKeyvalOptions*}.
%
% \subsection{Global vs. local options}
%
% Options that are given for \cs{documentclass} are called
% global options. They are known to the class and all packages.
% A package may make use of a global option and marks it as
% used. The advantage for the user is the freedom to specify
% options both in the \cs{documentclass} or \cs{usepackage}
% commands.
%
% However global options are shared with the class options
% and options of all other packages. Thus there can be the
% same option with different semantics for different
% packages and classes. As example, package \xpackage{bookmark}
% knows option \xoption{open} that specifies whether the bookmarks
% are opened or closed initially. It's values are |true| or |false|.
% Since KOMA-Script version 3.00 the KOMA classes also introduces
% option \xoption{open} with values |right| and |any| and a
% complete different meaning.
%
% Such conflicts can be resolved by marking all or part of options
% as local by \cs{DeclareLocalOption} or \cs{DeclareLocalOptions}.
% Then the packages ignores global occurences of these options.
% Package \xpackage{kvoptions} provides two methods:
% \begin{itemize}
% \item \cs{ProcessLocalKeyvalOptions} automatically uses all options
%   as local options. It ignores all global options.
% \item \cs{DeclareLocalOption} or \cs{DeclareLocalOptions} marks
%   options as local options. \cs{ProcessKeyvalOptions} will then
%   ignore global occurences for these local options.
% \end{itemize}
% Since version 1.5 package \xpackage{bookmark} uses the latter
% method. It checks global and local option places for driver options
% and limits all other options as local options. Thus the class
% option \xoption{open} of KOMA-Script is not misread as
% option for package \xpackage{bookmark}.
%
% \subsection{Summary of internal macros}
%
% The \cs{Declare}\texttt{\dots Option} commands define
% macros, additionally to the macros generated by the key definition.
% These macros can be used by the package/class author.
% The name of the macros starts with the prefix \meta{prefix}
% that can be configured by \cs{SetupKeyvalOptions}.
% \begin{center}
% \def\prefix{\cs{}\meta{prefix}\meta{key}}^^A
% \def\Hline{^^A
%   \hline
%   \rule{0pt}{1.2\ht\csname @arstrutbox\endcsname}^^A
% }^^A
% \begin{tabular}{@{}l|>{\ttfamily}l|l@{}}
%   Declare \meta{key} & \textrm{Defined macro} & Description\\
% \Hline
%   \cs{DeclareStringOption}
%    & \prefix & holds the string\\
% \Hline
%   \cs{DeclareBoolOption}
%     & \cs{if}\meta{prefix}\meta{key}& boolean switch\\
%   &\prefix false& disable switch\\
%   &\prefix true& enable switch\\
% \Hline
%   \cs{DeclareComplementaryOption}
%     & \prefix false& enable parent switch\\
%     & \prefix true& disable parent switch\\
% \Hline
%   \cs{DeclareVoidOption}
%   & \prefix & holds the action
% \end{tabular}
% \end{center}
%
% \subsection{\plainTeX}
%
% Package \xpackage{keyval} is also usable in \plainTeX\ with
% the help of file \xfile{miniltx.tex}. Some features of this
% package \xpackage{kvoptions} might also be useful for \plainTeX.
% If \LaTeX\ is not found, \cs{ProcessKeyvalOptions} and option
% \xoption{patch} are disabled. Before using the option declaration
% commands \cs{Declare...Option}, \cs{SetupKeyvalOptions} must be
% used.
%
% \hypersetup{bookmarksopenlevel=1}
%
% \section{Example}
%
% The following example defined a package that serves some
% private color management. A boolean option \xoption{print} enables
% print mode without colors. An option \xoption{emph} redefines
% \cs{emph} to print in the given color. And the driver
% can be specified by option \xoption{driver}.
%
%    \begin{macrocode}
%<*example>
    % Package identification
    % -----------------------
\NeedsTeXFormat{LaTeX2e}
\ProvidesPackage{example-mycolorsetup}[2016/05/16 Managing my colors]

\RequirePackage{ifpdf}
\RequirePackage{kvoptions}

    % Option declarations
    % -------------------

\SetupKeyvalOptions{
  family=MCS,
  prefix=MCS@
}
    % Use a shorter family name and prefix

    % Option print
\DeclareBoolOption{print}
    % is the same as
    % \DeclareBoolOption[false]{print}

    % Option driver
\ifpdf
  \DeclareStringOption[pdftex]{driver}
\else
  \DeclareStringOption[dvips]{driver}
\fi

    % Alternative interface for driver options
\DeclareVoidOption{dvips}{\SetupDriver}
\DeclareVoidOption{dvipdfm}{\SetupDriver}
\DeclareVoidOption{pdftex}{\SetupDriver}
    % In \SetupDriver we take the current option \CurrentOption
    % and pass it to the driver option.
    % The \expandafter commands expand \CurrentOption at the
    % time, when \SetupDriver is executed and \CurrentOption
    % has the correct meaning.
\newcommand*{\SetupDriver}{%
  \expandafter\@SetupDriver\expandafter{\CurrentOption}%
}
\newcommand*{\@SetupDriver}[1]{%
  \setkeys{MCS}{driver={#1}}%
}

    % Option emph
    % An empty value means, we want to have no color for \emph.
    % If the user specifies option emph without value, the red is used.
\DeclareStringOption{emph}[red]
    % is the same as
    % \DeclareStringOption[]{emph}[red]

    % Default option rule
\DeclareDefaultOption{%
  \ifx\CurrentOptionValue\relax
    \PackageWarningNoLine{\@currname}{%
      Unknown option `\CurrentOption'\MessageBreak
      is passed to package `color'%
    }%
    % Pass the option to package color.
    % Again it is better to expand \CurrentOption.
    \expandafter\PassOptionsToPackage
    \expandafter{\CurrentOption}{color}%
  \else
    % Package color does not take options with values.
    % We provide the standard LaTeX error.
    \@unknownoptionerror
  \fi
}

    % Process options
    % ---------------
\ProcessKeyvalOptions*

    % Implementation depending on option values
    % -----------------------------------------
    % Code for print mode
\ifMCS@print
  \PassOptionsToPackage{monochrome}{color}
    % tells package color to use black and white
\fi

\RequirePackage[\MCS@driver]{color}
    % load package color with the correct driver

    % \emph setup
\ifx\MCS@emph\@empty
    % \@empty is a predefined macro with empty contents.
    % the option value of option emph is empty, thus
    % we do not want a redefinition of \emph.
\else
  \renewcommand*{\emph}[1]{%
    \textcolor{\MCS@emph}{#1}%
  }
\fi
%</example>
%    \end{macrocode}
%
% \section{Package options}
%
% The package \xpackage{kvoptions} knows two package options
% \xoption{patch} and \xoption{debugshow}.
% The options of package \xpackage{kvoptions} are intended
% for authors, not for package/class writers. Inside a package
% it is too late for option \xoption{patch} and \xoption{debugshow}
% enables some messages that are perhaps useful for the debugging
% phase. Also \LaTeX\ is unhappy if a package is loaded
% later again with options that are previously not given.
% Thus package and class authors, stay with |\RequirePackage{kvoptions}|
% without options.
%
% Option \xoption{patch} loads package \xpackage{kvoptions-patch}.
%
% \subsection{Package \xpackage{kvoptions-patch}}
%
% \LaTeX's system of package/class options has some severe limitations
% that especially affects the value part if options are used as
% pair of key and value.
% \begin{itemize}
% \item Spaces are removed, regardless where:
%   \begin{quote}
%     |\documentclass[box=0 0 400 600]{article}|
%   \end{quote}
%   Now each package will see |box=00400600| as global option.
% \item In the previous case also braces would not help:
%   \begin{quote}
%     |\documentclass[box={0 0 400 600}]{article}|
%   \end{quote}
%   The result is an error message:
%   \begin{quote}
%     |! LaTeX Error: Missing \begin{document}.|
%   \end{quote}
%   As local option, however, it works if the package
%   knows about key value options (By using this package, for example).
% \item The requirements on robustness are extremly high.
%   \LaTeX\ expands the option. All that will not work as environment
%   name will break also as option. Even a \cs{relax} will generate
%   an error message:
%   \begin{quote}
%     |! Missing \endcsname inserted.|
%   \end{quote}
%   Of course, \LaTeX\ does not use its protecting mechanisms.
%   On contrary \cs{protect} itself will cause errors.
% \item The options are expanded. But perhaps the package will
%   do that, because it has to setup some things before?
%   Example \xpackage{hyperref}:
%   \begin{quote}
%     |\usepackage[pdfauthor=M\"uller]{hyperref}|
%   \end{quote}
%   Package \xpackage{hyperref} does not see |M\"uller| but
%   its expansion and it does not like it, you get many warnings
%   \begin{quote}
%     |Token not allowed in a PDFDocEncoded string|
%   \end{quote}
%   And the title becomes: |Mu127uller|.
%   Therefore such options must usually be given after package
%   \xpackage{hyperref}
%   is loaded:
%   \begin{quote}
%     |\usepackage{hyperref}|\\
%     |\hypersetup{pdfauthor=Fran\c coise M\"uller}|
%   \end{quote}
%   As package option it will even break with |Fran\c coise|
%   because of the cedilla |\c c|, it is not robust enough.
% \end{itemize}
% For users that do not want with this limitations the package
% offers package \xoption{kvoptions-patch}. It patches \LaTeX's option system
% and tries to teach it also to handle options that are given
% as pairs of key and value and to prevent expansion.
% It can already be used at the very beginning, before \cs{documentclass}:
% \begin{quote}
%   |\RequirePackage{kvoptions-patch}|\\
%   |\documentclass[pdfauthor=Fran\c coise M\"uller]{article}|\\
%   |\usepackage{hyperref}|
% \end{quote}
% The latest time is before the package where you want to use
% problematic values:
% \begin{quote}
%   |\usepackage{kvoptions-patch}|\\
%   |\usepackage[Fran\c coise M\"uller]{hyperref}|
% \end{quote}
% Some remarks:
% \begin{itemize}
% \item The patch requires \eTeX, its \cs{unexpanded} feature
%   is much too nice. It is possible to work around using token
%   registers. But the code becomes longer, slower, more difficult
%   to read and maintain. The package without option \xoption{patch}
%   works and will work without \eTeX.
% \item The code for the patch is quite long, there are many test
%   cases. Thus the probability for bugs is probably not too small.
% \item Since 2008/10/18 v3.0 package \xpackage{kvoptions-patch} is
%   available. Before option \xoption{patch} of package \xpackage{kvoptions}
%   must be used instead. I think, the solution as standalone package
%   \xpackage{kvoptions-patch} is cleaner and avoids option clashes.
% \end{itemize}
%
% \subsection{Option \xoption{debugshow}}
%
% The name of this option follows the convention of packages
% \xpackage{multicol}, \xpackage{tabularx}, and \xpackage{tracefnt}.
% Currently it prints the setting of boolean options, declared
% by \cs{DeclareBoolOption} in the \xfile{.log} file, if that
% boolean option is used.
% You can activate the option by
% \begin{itemize}
% \item |\PassOptionsToPackage{debugshow}{kvoptions}|\\
%       Put this somewhere before package \xpackage{kvoptions} is
%       loaded first, e.g. before \cs{documentclass}.
% \item |\RequirePackage[debugshow]{kvoptions}|\\
%       Before \cs{documentclass} even an author has to use
%       \cs{RequirePackage}. \cs{usepackage} only works after
%       \cs{documentclass}.
% \end{itemize}
% The preferred method is \cs{PassOptionsToPackage}, because
% it does not force the package loading and does not disturb,
% if the package is not loaded later at all.
%
% \section{Limitations}
%
% \subsection{Compatibility}
% \subsubsection{Package \xpackage{kvoptions-patch} vs. package \xpackage{xkvltxp}}
%
% Package \xpackage{xkvltxp} from the \xpackage{xkeyval} project
% has the same goal as package \xpackage{kvoptions-patch}
% and to patch \LaTeX's kernel commands in order to get better support
% for key value options.
% Of course they cannot be used both. The user must decide, which
% method he prefers. Package \xpackage{kvoptions-patch} aborts itself,
% if it detects that \xpackage{xkvltxp} is
% already loaded.
%
% However package \xpackage{xkvltxp} and \xpackage{kvoptions}
% can be used together, example:
% \begin{quote}
% |\usepackage{xkvltxp}|\\
% |\usepackage[...]{foobar} % |\texttt{\itshape foobar using kvoptions}
% \end{quote}
% The other way should work, too.
%
% Package \xpackage{kvoptions-patch} tries to catch more situations
% and to be more robust.
% For example, during the comparison of options it normalizes
% them by removing spaces around |=| and the value. Thus the
% following is not reported as option clash:
% \begin{quote}
%\begin{verbatim}
%\RequirePackage{kvoptions-patch}
%\documentclass{article}
%
%\usepackage[scaled=0.7]{helvet}
%\usepackage[scaled = 0.7]{helvet}
%
%\begin{document}
%\end{document}
%\end{verbatim}
% \end{quote}
%
% \subsection{Limitations}
%
% \subsubsection{Option comparisons}
%
% In some situations \LaTeX\ compares option lists, e.g. option clash check,
% \cs{@ifpackagewith}, or \cs{@ifclasswith}. Apart from catcode and
% sanitizing problems of option \xoption{patch}, there is another problem.
% \LaTeX\ does not know about the type and default values of
% options in key value style. Thus an option clash is reported,
% even if the key value has the same meaning:
% \begin{quote}
% |\usepackage[scaled]{helvet} %| \texttt{\textit{default is }.95}\\
% |\usepackage[.95]{helvet}|\\
% |\usepackage[0.95]{helvet}|
% \end{quote}
%
% \subsubsection{Option list parsing with package \xpackage{kvoptions-patch}}
%
% With package \xpackage{kvoptions-patch} the range of possible values
% in key value specifications is much large, for example
% the comma can be used, if enclosed in curly braces.
%
% Other packages, especially the packages that uses their
% own process option code can be surprised to find tokens
% inside options that they do not expect and errors would
% be the consequence. To avoid errors the options, especially
% the unused option list is sanitized. That means the list
% will only contain tokens with catcode 12 (other) and
% perhaps spaces (catcode 10). This allows a safe parsing
% for other packages. But a comma in the value part is
% no longer protected by curly braces because they have lost
% their special meaning. This is the price for compatibility.
%
% Example:
% \begin{quote}
% |\RequirePackage{kvoptions-patch}|\\
% |\documentclass[a={a,b,c},b]{article}|\\
% |\begin{document}|\\
% |\end{document}|
% \end{quote}
% Result:
% \begin{quote}
% |LaTeX Warning: Unused global option(s):|\\
% |    [a={a,c},b].|
% \end{quote}
%
% \StopEventually{
% }
%
% \section{Implementation}
%
% \subsection{Preamble}
%
%    \begin{macrocode}
%<*package>
%    \end{macrocode}
%
% \paragraph{Reload check and  identification.}
%    Reload check, especially if the package is not used with \LaTeX.
%    \begin{macrocode}
\begingroup\catcode61\catcode48\catcode32=10\relax%
  \catcode13=5 % ^^M
  \endlinechar=13 %
  \catcode35=6 % #
  \catcode39=12 % '
  \catcode44=12 % ,
  \catcode45=12 % -
  \catcode46=12 % .
  \catcode58=12 % :
  \catcode64=11 % @
  \catcode123=1 % {
  \catcode125=2 % }
  \expandafter\let\expandafter\x\csname ver@kvoptions.sty\endcsname
  \ifx\x\relax % plain-TeX, first loading
  \else
    \def\empty{}%
    \ifx\x\empty % LaTeX, first loading,
      % variable is initialized, but \ProvidesPackage not yet seen
    \else
      \expandafter\ifx\csname PackageInfo\endcsname\relax
        \def\x#1#2{%
          \immediate\write-1{Package #1 Info: #2.}%
        }%
      \else
        \def\x#1#2{\PackageInfo{#1}{#2, stopped}}%
      \fi
      \x{kvoptions}{The package is already loaded}%
      \aftergroup\endinput
    \fi
  \fi
\endgroup%
%    \end{macrocode}
%    Package identification:
%    \begin{macrocode}
\begingroup\catcode61\catcode48\catcode32=10\relax%
  \catcode13=5 % ^^M
  \endlinechar=13 %
  \catcode35=6 % #
  \catcode39=12 % '
  \catcode40=12 % (
  \catcode41=12 % )
  \catcode44=12 % ,
  \catcode45=12 % -
  \catcode46=12 % .
  \catcode47=12 % /
  \catcode58=12 % :
  \catcode64=11 % @
  \catcode91=12 % [
  \catcode93=12 % ]
  \catcode123=1 % {
  \catcode125=2 % }
  \expandafter\ifx\csname ProvidesPackage\endcsname\relax
    \def\x#1#2#3[#4]{\endgroup
      \immediate\write-1{Package: #3 #4}%
      \xdef#1{#4}%
    }%
  \else
    \def\x#1#2[#3]{\endgroup
      #2[{#3}]%
      \ifx#1\@undefined
        \xdef#1{#3}%
      \fi
      \ifx#1\relax
        \xdef#1{#3}%
      \fi
    }%
  \fi
\expandafter\x\csname ver@kvoptions.sty\endcsname
\ProvidesPackage{kvoptions}%
  [2016/05/16 v3.12 Key value format for package options (HO)]%
%    \end{macrocode}
%
% \paragraph{Catcodes}
%
%    \begin{macrocode}
\begingroup\catcode61\catcode48\catcode32=10\relax%
  \catcode13=5 % ^^M
  \endlinechar=13 %
  \catcode123=1 % {
  \catcode125=2 % }
  \catcode64=11 % @
  \def\x{\endgroup
    \expandafter\edef\csname KVO@AtEnd\endcsname{%
      \endlinechar=\the\endlinechar\relax
      \catcode13=\the\catcode13\relax
      \catcode32=\the\catcode32\relax
      \catcode35=\the\catcode35\relax
      \catcode61=\the\catcode61\relax
      \catcode64=\the\catcode64\relax
      \catcode123=\the\catcode123\relax
      \catcode125=\the\catcode125\relax
    }%
  }%
\x\catcode61\catcode48\catcode32=10\relax%
\catcode13=5 % ^^M
\endlinechar=13 %
\catcode35=6 % #
\catcode64=11 % @
\catcode123=1 % {
\catcode125=2 % }
\def\TMP@EnsureCode#1#2{%
  \edef\KVO@AtEnd{%
    \KVO@AtEnd
    \catcode#1=\the\catcode#1\relax
  }%
  \catcode#1=#2\relax
}
\TMP@EnsureCode{1}{14}% ^^A (comment)
\TMP@EnsureCode{2}{14}% ^^A (comment)
\TMP@EnsureCode{33}{12}% !
\TMP@EnsureCode{39}{12}% '
\TMP@EnsureCode{40}{12}% (
\TMP@EnsureCode{41}{12}% )
\TMP@EnsureCode{42}{12}% *
\TMP@EnsureCode{44}{12}% ,
\TMP@EnsureCode{45}{12}% -
\TMP@EnsureCode{46}{12}% .
\TMP@EnsureCode{47}{12}% /
\TMP@EnsureCode{58}{12}% :
\TMP@EnsureCode{62}{12}% >
\TMP@EnsureCode{91}{12}% [
\TMP@EnsureCode{93}{12}% ]
\TMP@EnsureCode{94}{7}% ^ (superscript)
\TMP@EnsureCode{96}{12}% `
\edef\KVO@AtEnd{\KVO@AtEnd\noexpand\endinput}
%    \end{macrocode}
%
% \paragraph{External resources.}
%    The package extends the support for key value pairs of
%    package \cs{keyval} to package options. Thus the package
%    needs to be loaded anyway. and we use it for
%    \cs{SetupKeyvalOptions}. AFAIK this does not disturb
%    users of \xpackage{xkeyval}.
%    \begin{macrocode}
\@ifundefined{define@key}{%
  \RequirePackage{keyval}\relax
}{}
%    \end{macrocode}
%
%    Macro \cs{DeclareLocalOptions} parses a comma separated key list
%    and uses \cs{comma@parse} of package \xpackage{kvsetkeys},
%    version 1.3.
%    \begin{macrocode}
\RequirePackage{ltxcmds}[2010/12/02]
\RequirePackage{kvsetkeys}[2007/09/29]
%    \end{macrocode}
%
% \paragraph{Provide macros for \plainTeX.}
%    \begin{macrocode}
\@ifundefined{@x@protect}{%
  \def\@x@protect#1\fi#2#3{%
    \fi\protect#1%
  }%
  \let\@typeset@protect\relax
}{}
\@ifundefined{@currname}{%
  \def\@currname{}%
}{}
\@ifundefined{@currext}{%
  \def\@currext{}%
}{}
%    \end{macrocode}
%
% \paragraph{Options}
%    Option \xoption{debugshow} enables additional lines of
%    code that prints information into the \xfile{.log}
%    file.
%    \begin{macrocode}
\DeclareOption{debugshow}{\catcode\@ne=9 }
%    \end{macrocode}
%
%    \begin{macrocode}
\DeclareOption{patch}{%
  \AtEndOfPackage{%
    \RequirePackage{kvoptions-patch}[2016/05/16]%
  }%
}
%    \end{macrocode}
%
%    Optionen auswerten:
%    \begin{macrocode}
\ProcessOptions\relax
%    \end{macrocode}
%
% \subsection{Option declaration macros}
%
% \subsubsection{\cs{SetupKeyvalOptions}}
%
%    The family for the key value pairs can be setup once
%    and is remembered later.
%    The package name seems a reasonable default for the
%    family key, if it is not set by the package author.
%
%    \begin{macro}{\KVO@family}
%    We cannot store the family setting in one macro,
%    because the package should be usable for many
%    other packages, too. Thus we remember the family
%    setting in a macro, whose name contains the
%    package name with extension, a key in \LaTeX's
%    class/package system.
%    \begin{macrocode}
\define@key{KVO}{family}{%
  \expandafter\edef\csname KVO@family@%
      \@currname.\@currext\endcsname{#1}%
}
\def\KVO@family{%
  \@ifundefined{KVO@family@\@currname.\@currext}{%
    \@currname
  }{%
    \csname KVO@family@\@currname.\@currext\endcsname
  }%
}
%    \end{macrocode}
%    \end{macro}
%
%    \begin{macro}{\KVO@prefix}
%    The value settings of options that are declared by \cs{DeclareBoolOption}
%    and \cs{DeclareStringOption} need to be saved in macros.
%    in the first case this is a switch \cs{if}\meta{prefix}\meta{key},
%    in the latter case a macro \cs{}\meta{prefix}\meta{key}.
%    The prefix can be configured, by \xoption{prefix} that is declared
%    here. The default is the package name with |@| appended.
%    \begin{macrocode}
\define@key{KVO}{prefix}{%
  \expandafter\edef\csname KVO@prefix@%
      \@currname.\@currext\endcsname{#1}%
}
\def\KVO@prefix{%
  \ltx@ifundefined{KVO@prefix@\@currname.\@currext}{%
    \@currname @%
  }{%
    \csname KVO@prefix@\@currname.\@currext\endcsname
  }%
}
%    \end{macrocode}
%    \end{macro}
%
%    \begin{macrocode}
\define@key{KVO}{setkeys}{%
  \expandafter\def\csname KVO@setkeys@%
      \@currname.\@currext\endcsname{#1}%
}
%    \end{macrocode}
%    \begin{macro}{\KVO@setkeys}
%    \begin{macrocode}
\def\KVO@setkeys{%
  \ltx@IfUndefined{KVO@setkeys@\@currname.\@currext}{%
    \setkeys
  }{%
    \csname KVO@setkeys@\@currname.\@currext\endcsname
  }%
}
%    \end{macrocode}
%    \end{macro}
%
%    \begin{macro}{\SetupKeyvalOptions}
%    The argument of \cs{SetupKeyvalOptions} expects a key value
%    list, known keys are \xoption{family} and \xoption{prefix}.
%    \begin{macrocode}
\newcommand*{\SetupKeyvalOptions}{%
  \kvsetkeys{KVO}%
}
%    \end{macrocode}
%    \end{macro}
%
% \subsubsection{\cs{DeclareBoolOption}}
%
%    \begin{macro}{\DeclareBoolOption}
%    Usually options of boolean type can be given by the
%    user without value and this means a setting to \emph{true}.
%    We follow this convention here. Also it simplifies the
%    user interface.
%
%    The switch is created and initialized with \emph{false}.
%    The default setting can be overwritten by the optional
%    argument.
%
%    \LaTeX's \cs{newif} does not check for already defined
%    macros, therefore we add this check here to prevent
%    the user from accidently redefining of \TeX's primitives
%    and other macros.
%    \begin{macrocode}
\newcommand*{\DeclareBoolOption}[2][false]{%
  \KVO@ifdefinable{if\KVO@prefix#2}{%
    \KVO@ifdefinable{\KVO@prefix#2true}{%
      \KVO@ifdefinable{\KVO@prefix#2false}{%
        \csname newif\expandafter\endcsname
        \csname if\KVO@prefix#2\endcsname
        \@ifundefined{\KVO@prefix#2#1}{%
          \PackageWarning{kvoptions}{%
            Initialization of option `#2' failed,\MessageBreak
            cannot set boolean option to `#1',\MessageBreak
            use `true' or `false', now using `false'%
          }%
        }{%
          \csname\KVO@prefix#2#1\endcsname
        }%
        \begingroup
          \edef\x{\endgroup
            \noexpand\define@key{\KVO@family}{#2}[true]{%
              \noexpand\KVO@boolkey{\@currname}%
              \ifx\@currext\@clsextension
                \noexpand\@clsextension
              \else
                \noexpand\@pkgextension
              \fi
              {\KVO@prefix}{#2}{####1}%
            }%
          }%
        \x
      }%
    }%
  }%
}
%    \end{macrocode}
%    \end{macro}
%
%    \begin{macro}{\DeclareComplementaryOption}
%    The first argument is the key name, the second the
%    key that must be a boolean option with the same
%    current family and prefix. A new switch is not
%    created for the new key, we have already a switch.
%    Instead we define switch setting commands to work
%    on the parent switch.
%    \begin{macrocode}
\newcommand*{\DeclareComplementaryOption}[2]{%
  \@ifundefined{if\KVO@prefix#2}{%
    \PackageError{kvoptions}{%
      Cannot generate option code for `#1',\MessageBreak
      parent switch `#2' does not exist%
    }{%
      You are inside %
      \ifx\@currext\@clsextension class\else package\fi\space
      `\@currname.\@currext'.\MessageBreak
      `\KVO@family' is used as familiy %
      for the keyval options.\MessageBreak
      `\KVO@prefix' serves as prefix %
      for internal switch macros.\MessageBreak
      \MessageBreak
      \@ehc
    }%
  }{%
    \KVO@ifdefinable{\KVO@prefix#1true}{%
      \KVO@ifdefinable{\KVO@prefix#1false}{%
        \expandafter\let\csname\KVO@prefix#1false\expandafter\endcsname
          \csname\KVO@prefix#2true\endcsname
        \expandafter\let\csname\KVO@prefix#1true\expandafter\endcsname
          \csname\KVO@prefix#2false\endcsname
%    \end{macrocode}
%    The same code part as in \cs{DeclareBoolOption} can
%    now be used.
%    \begin{macrocode}
        \begingroup
          \edef\x{\endgroup
            \noexpand\define@key{\KVO@family}{#1}[true]{%
              \noexpand\KVO@boolkey{\@currname}%
              \ifx\@currext\@clsextension
                \noexpand\@clsextension
              \else
                \noexpand\@pkgextension
              \fi
              {\KVO@prefix}{#1}{####1}%
            }%
          }%
        \x
      }%
    }%
  }%
}
%    \end{macrocode}
%    \end{macro}
%
%    \begin{macro}{\KVO@ifdefinable}
%    Generate the command token LaTeX's \cs{@ifdefinable} expects.
%    \begin{macrocode}
\def\KVO@ifdefinable#1{%
  \expandafter\@ifdefinable\csname #1\endcsname
}
%    \end{macrocode}
%    \end{macro}
%
%    \begin{macro}{\KVO@boolkey}
%    We check explicitly for |true| and |false| to prevent
%    the user from accidently calling other macros.
%    \begin{quote}
%    \begin{tabular}{@{}ll@{}}
%    |#1|& package/class name\\
%    |#2|& |\@pkgextension|/|\@clsextension|\\
%    |#3|& prefix \\
%    |#4|& key name\\
%    |#5|& new value
%    \end{tabular}
%    \end{quote}
%    \begin{macrocode}
\def\KVO@boolkey#1#2#3#4#5{%
  \edef\KVO@param{#5}%
  \ltx@onelevel@sanitize\KVO@param
  \ifx\KVO@param\KVO@true
    \expandafter\@firstofone
  \else
    \ifx\KVO@param\KVO@false
      \expandafter\expandafter\expandafter\@firstofone
    \else
      \ifx#2\@clsextension
        \expandafter\ClassWarning
      \else
        \expandafter\PackageWarning
      \fi
      {#1}{%
        Value `\KVO@param' is not supported by\MessageBreak
        option `#4'%
      }%
      \expandafter\expandafter\expandafter\@gobble
    \fi
  \fi
  {%
    ^^A\ifx#2\@clsextension
    ^^A  \expandafter\ClassInfo
    ^^A\else
    ^^A  \expandafter\PackageInfo
    ^^A\fi
    ^^A{#1}{[option] #4=\KVO@param}%
    \csname#3#4\KVO@param\endcsname
  }%
}
%    \end{macrocode}
%    \end{macro}
%    \begin{macro}{\KVO@true}
%    \begin{macro}{\KVO@false}
%    The macros \cs{KVO@true} and \cs{KVO@false} are used
%    for string comparisons. After \cs{ltx@onelevel@sanitize}
%    we have only tokens with catcode 12 (other).
%    \begin{macrocode}
\def\KVO@true{true}
\def\KVO@false{false}
\ltx@onelevel@sanitize\KVO@true
\ltx@onelevel@sanitize\KVO@false
%    \end{macrocode}
%    \end{macro}
%    \end{macro}
%
% \subsubsection{\cs{DeclareStringOption}}
%
%    \begin{macro}{\DeclareStringOption}
%    \begin{macrocode}
\newcommand*{\DeclareStringOption}[2][]{%
  \@ifnextchar[{%
    \KVO@DeclareStringOption{#1}{#2}@%
  }{%
    \KVO@DeclareStringOption{#1}{#2}{}[]%
  }%
}
%    \end{macrocode}
%    \end{macro}
%    \begin{macro}{\KVO@DeclareStringOption}
%    \begin{macrocode}
\def\KVO@DeclareStringOption#1#2#3[#4]{%
  \KVO@ifdefinable{\KVO@prefix#2}{%
    \@namedef{\KVO@prefix#2}{#1}%
    \begingroup
      \ifx\\#3\\%
        \toks@{}%
      \else
        \toks@{[{#4}]}%
      \fi
      \edef\x{\endgroup
        \noexpand\define@key{\KVO@family}{#2}\the\toks@{%
          ^^A\begingroup
          ^^A  \toks@{####1}%
          ^^A  \ifx\@currext\@clsextension
          ^^A    \noexpand\ClassInfo
          ^^A  \else
          ^^A    \noexpand\PackageInfo
          ^^A  \fi
          ^^A  {\@currname}{%
          ^^A    [option] #2={\noexpand\the\toks@}%
          ^^A  }%
          ^^A\endgroup
          \noexpand\def
          \expandafter\noexpand\csname\KVO@prefix#2\endcsname{####1}%
        }%
      }%
    \x
  }%
}
%    \end{macrocode}
%    \end{macro}
%
% \subsubsection{\cs{DeclareVoidOption}}
%
%    \begin{macro}{\DeclareVoidOption}
%    \begin{macrocode}
\newcommand*{\DeclareVoidOption}[2]{%
  \begingroup
    \let\next\@gobbletwo
    \KVO@ifdefinable{\KVO@prefix#1}{%
      \let\next\@firstofone
    }%
  \expandafter\endgroup
  \next{%
    \begingroup
      \edef\x{\endgroup
        \noexpand\define@key{\KVO@family}{#1}[\KVO@VOID@]{%
          \noexpand\KVO@voidkey{\@currname}%
          \ifx\@currext\@clsextension
            \noexpand\@clsextension
          \else
            \noexpand\@pkgextension
          \fi
          {#1}%
          {####1}%
          \expandafter\noexpand\csname\KVO@prefix#1\endcsname
        }%
      }%
    \x
    \begingroup
      \toks@{#2}%
    \expandafter\endgroup
    \expandafter\def
    \csname\KVO@prefix#1\expandafter\endcsname
    \expandafter{\the\toks@}%
  }%
}
\def\KVO@VOID@{@VOID@}
%    \end{macrocode}
%    \end{macro}
%    \begin{macro}{\KVO@voidkey}
%    \begin{tabular}{@{}ll@{}}
%    |#1|& package/class name\\
%    |#2|& |\@pkgextension|/|\@clsextension|\\
%    |#3|& key name\\
%    |#4|& default (|@VOID@|)\\
%    |#5|& macro with option code
%    \end{tabular}
%    \begin{macrocode}
\def\KVO@voidkey#1#2#3#4{%
  \def\CurrentOption{#3}%
  \begingroup
    \def\x{#4}%
  \expandafter\endgroup
  \ifx\x\KVO@VOID@
  \else
    \ifx#2\@clsextension
      \expandafter\ClassWarning
    \else
      \expandafter\PackageWarning
    \fi
    {#1}{%
      Unexpected value for option `#3'\MessageBreak
      is ignored%
    }%
  \fi
  ^^A\ifx#2\@clsextension
  ^^A  \expandafter\ClassInfo
  ^^A\else
  ^^A  \expandafter\PackageInfo
  ^^A\fi
  ^^A{#1}{[option] #3}%
}
%    \end{macrocode}
%    \end{macro}
%
% \subsubsection{\cs{DeclareDefaultOption}}
%
%    \begin{macro}{\DeclareDefaultOption}
%    \begin{macrocode}
\newcommand*{\DeclareDefaultOption}{%
  \@namedef{KVO@default@\@currname.\@currext}%
}
%    \end{macrocode}
%    \end{macro}
%
% \subsubsection{\cs{DeclareLocalOptions}}
%
%    \begin{macro}{\DeclareLocalOptions}
%    \begin{macrocode}
\newcommand*{\DeclareLocalOptions}[1]{%
  \comma@parse{#1}\KVO@DeclareLocalOption
}
%    \end{macrocode}
%    \end{macro}
%    \begin{macro}{\KVO@DeclareLocalOption}
%    \begin{macrocode}
\def\KVO@DeclareLocalOption#1{%
  \expandafter\def\csname KVO@local@\KVO@family @#1\endcsname{}%
}
%    \end{macrocode}
%    \end{macro}
%
% \subsection{Dynamic options}
%
% \subsubsection{\cs{DisableKeyvalOption}}
%
%    \begin{macrocode}
\SetupKeyvalOptions{%
  family=KVOdyn,%
  prefix=KVOdyn@%
}
\DeclareBoolOption[true]{global}
\DeclareComplementaryOption{local}{global}
\DeclareStringOption[undef]{action}
\let\KVOdyn@name\relax
\let\KVOdyn@ext\@empty
\define@key{KVOdyn}{class}{%
  \def\KVOdyn@name{#1}%
  \let\KVOdyn@ext\@clsextension
}
\define@key{KVOdyn}{package}{%
  \def\KVOdyn@name{#1}%
  \let\KVOdyn@ext\@pkgextension
}
\newcommand*{\DisableKeyvalOption}[3][]{%
  \begingroup
    \kvsetkeys{KVOdyn}{#1}%
    \def\x{\endgroup}%
    \@ifundefined{KVO@action@\KVOdyn@action}{%
      \PackageError{kvoptions}{%
        Unknown disable action %
        `\expandafter\strip@prefix\meaning\KVOdyn@action'\MessageBreak
        for option `#3' in keyval family '#2'%
      }\@ehc
    }{%
      \csname KVO@action@\KVOdyn@action\endcsname{#2}{#3}%
    }%
  \x
}
\def\KVO@action@undef#1#2{%
  \edef\x{\endgroup
    \ifKVOdyn@global\global\fi
    \let
    \expandafter\noexpand\csname KV@#1@#2\endcsname
    \relax
    \ifKVOdyn@global\global\fi
    \let
    \expandafter\noexpand\csname KV@#1@#2@default\endcsname
    \relax
  }%
  ^^A\PackageInfo{kvoptions}{%
  ^^A  [option] key `#2' of family `#1'\MessageBreak
  ^^A  is disabled (undef, \ifKVOdyn@global global\else local\fi)%
  ^^A}%
}
\def\KVO@action@ignore#1#2{%
  \edef\x{\endgroup
    \ifKVOdyn@global\global\fi
    \let
    \expandafter\noexpand\csname KV@#1@#2\endcsname
    \noexpand\@gobble
    \ifKVOdyn@global\global\fi
    \let
    \expandafter\noexpand\csname KV@#1@#2@default\endcsname
    \noexpand\@empty
  }%
  ^^A\PackageInfo{kvoptions}{%
  ^^A  [option] key `#2' of family `#1'\MessageBreak
  ^^A  is disabled (ignore, \ifKVOdyn@global global\else local\fi)%
  ^^A}%
}
\def\KVO@action@error{%
  \KVO@do@action{error}%
}
\def\KVO@action@warning{%
  \KVO@do@action{warning}%
}
%    \end{macrocode}
%    \begin{tabular}{@{}ll@{}}
%    |#1|& |error| or |warning|\\
%    |#2|& \meta{family}\\
%    |#3|& \meta{key}\\
%    \end{tabular}
%    \begin{macrocode}
\def\KVO@do@action#1#2#3{%
  \ifx\KVOdyn@name\relax
    \PackageError{kvoptions}{%
      Action type `#1' needs package/class name\MessageBreak
      for key `#3' in family `#2'%
    }\@ehc
  \else
    \edef\x{\endgroup
      \noexpand\define@key{#2}{#3}[]{%
        \expandafter\noexpand\csname KVO@disable@#1\endcsname
        {\KVOdyn@name}\noexpand\KVOdyn@ext{#3}%
      }%
      \ifKVOdyn@global
        \global\let
        \expandafter\noexpand\csname KV@#2@#3\endcsname
        \expandafter\noexpand\csname KV@#2@#3\endcsname
        \global\let
        \expandafter\noexpand\csname KV@#2@#3@default\endcsname
        \expandafter\noexpand\csname KV@#2@#3@default\endcsname
      \fi
    }%
    ^^A\ifx\KVOdyn@ext\@clsextension
    ^^A  \expandafter\ClassInfo
    ^^A\else
    ^^A   \expandafter\PackageInfo
    ^^A\fi
    ^^A{\KVOdyn@name}{%
    ^^A  [option] key `#3' of family `#2'\MessageBreak
    ^^A  is disabled (#1, \ifKVOdyn@global global\else local\fi)%
    ^^A}%
  \fi
}
\def\KVO@disable@error#1#2#3{%
  \ifx#2\@clsextension
    \expandafter\ClassError
  \else
    \expandafter\PackageError
  \fi
  {#1}{%
    Option `#3' is given too late,\MessageBreak
    now the option is ignored%
  }\@ehc
}
\def\KVO@disable@warning#1#2#3{%
  \ifx#2\@clsextension
    \expandafter\ClassWarning
  \else
    \expandafter\PackageWarning
  \fi
  {#1}{%
    Option `#3' is already consumed\MessageBreak
    and has no effect%
  }%
}
%    \end{macrocode}
%
% \subsection{Change option code}
%
% \subsubsection{\cs{AddToKeyvalOption}}
%
%    \begin{macro}{\AddToKeyvalOption}
%    \begin{macrocode}
\newcommand*{\AddToKeyvalOption}{%
  \@ifstar{%
    \begingroup
      \edef\x{\endgroup
        \noexpand\KVO@AddToKeyvalOption{\KVO@family}%
      }%
    \x
  }%
  \KVO@AddToKeyvalOption
}
%    \end{macrocode}
%    \end{macro}
%    \begin{macro}{\KVO@AddToKeyvalOption}
%    \begin{macrocode}
\def\KVO@AddToKeyvalOption#1#2{%
  \@ifundefined{KV@#1@#2}{%
    \PackageWarning{kvoptions}{%
      Key `#2' of family `#1' does not exist.\MessageBreak
      Ignoring \string\AddToKeyvalOption
    }%
    \@gobble
  }{%
    \edef\KVO@next{%
      \noexpand\KVO@@AddToKeyvalOption
      \expandafter\noexpand\csname KV@#1@#2\endcsname
    }%
    \afterassignment\KVO@next
    \def\KVO@temp##1%
  }%
}
%    \end{macrocode}
%    \end{macro}
%    \begin{macro}{\KVO@@AddToKeyvalOption}
%    \begin{macrocode}
\def\KVO@@AddToKeyvalOption#1{%
  \begingroup
    \toks@\expandafter{#1{##1}}%
    \toks@\expandafter{\the\expandafter\toks@\KVO@temp{##1}}%
    \edef\x{\endgroup
      \noexpand\def\noexpand#1####1{\the\toks@}%
    }%
  \x
}
%    \end{macrocode}
%    \end{macro}
%
% \subsection{Process options}
%
% \subsubsection{Get global options}
%
%    Package \xpackage{xkeyval} removes options with equal signs from
%    the global options (\cs{@classoptionslist}). The effect is that
%    other packages and classes will not see these global options anymore.
%    A bug-report was answered that this behaviour is ``by design''.
%    Thus I call it a design bug. Now getting the global options require
%    an algorithm instead of a simple macro call.
%    \begin{macrocode}
%</package>
%<*package|patch>
%    \end{macrocode}
%    \begin{macro}{\KVO@IfDefThen}
%    Call |#2| if command |#1| is defined and not \cs{relax}.
%    (Package \xpackage{kvoptions-patch} does not load
%    package \xpackage{ltxcmds}.)
%    \begin{macrocode}
\def\KVO@IfDefThen#1#2{%
  \ifx#1\ltx@undefined
  \else
    \ifx#1\relax
    \else
      #2%
    \fi
  \fi
}%
%    \end{macrocode}
%    \end{macro}
%    \begin{macro}{\KVO@GetClassOptionsList}
%    \begin{macrocode}
\def\KVO@GetClassOptionsList{%
  \let\KVO@classoptionslist\@classoptionslist
  \KVO@IfDefThen\@classoptionslist{%
    \KVO@IfDefThen\XKV@documentclass{%
      \ifx\XKV@documentclass\ltx@empty
      \else
        \KVO@IfDefThen\XKV@classoptionslist{%
          \ifx\XKV@classoptionslist\ltx@empty
          \else
            \let\KVO@classoptionslist\XKV@classoptionslist
          \fi
        }%
      \fi
    }%
  }%
}%
%    \end{macrocode}
%    \end{macro}
%    \begin{macrocode}
%</package|patch>
%<*package>
%    \end{macrocode}
%
% \subsubsection{\cs{ProcessKeyvalOptions}}
%
%    \begin{macro}{\ProcessKeyvalOptions}
%    If the optional star is given, we get the family
%    name and expand it for safety.
%    \begin{macrocode}
\newcommand*{\ProcessKeyvalOptions}{%
  \@ifstar{%
    \begingroup
      \edef\x{\endgroup
        \noexpand\KVO@ProcessKeyvalOptions{\KVO@family}%
      }%
    \x
  }%
  \KVO@ProcessKeyvalOptions
}
%    \end{macrocode}
%    \end{macro}
%
%    \begin{macrocode}
\def\KVO@ProcessKeyvalOptions#1{%
  \let\@tempc\relax
  \let\KVO@temp\@empty
%    \end{macrocode}
%    Add any global options that are known to KV to the start of
%    the list being built in |\KVO@temp| and mark them used (by
%    removing them from the unused option list).
%    \begin{macrocode}
  \ifx\@currext\@clsextension
  \else
    \KVO@GetClassOptionsList
    \ifx\KVO@classoptionslist\relax
    \else
      \@for\KVO@CurrentOption:=\KVO@classoptionslist\do{%
        \@ifundefined{KV@#1@\expandafter\KVO@getkey
                      \KVO@CurrentOption=\@nil}{%
        }{%
          \@ifundefined{KVO@local@#1@\expandafter\KVO@getkey
                        \KVO@CurrentOption=\@nil}{%
            \ifx\KVO@Patch Y%
              \edef\KVO@temp{%
                \etex@unexpanded\expandafter{%
                  \KVO@temp
                }%
                ,%
                \etex@unexpanded\expandafter{%
                  \KVO@CurrentOption
                }%
                ,%
              }%
              \ltx@onelevel@sanitize\KVO@CurrentOption
            \else
              \edef\KVO@temp{%
                \KVO@temp
                ,%
                \KVO@CurrentOption
                ,%
              }%
            \fi
            \@expandtwoargs\@removeelement\KVO@CurrentOption
              \@unusedoptionlist\@unusedoptionlist
          }{}%
        }%
      }%
    \fi
  \fi
%    \end{macrocode}
%     Now stick the package options at the end of the list and wrap
%     in a call to \cs{setkeys}. A class ignores unknown global
%     options, we must remove them to prevent error messages
%     from \cs{setkeys}.
%    \begin{macrocode}
  \begingroup
    \toks\tw@{}%
    \@ifundefined{opt@\@currname.\@currext}{%
      \toks@\expandafter{\KVO@temp}%
    }{%
      \toks@\expandafter\expandafter\expandafter{%
        \csname opt@\@currname.\@currext\endcsname
      }%
      \ifx\@currext\@clsextension
        \edef\CurrentOption{\the\toks@}%
        \toks@\expandafter{\KVO@temp}%
        \@for\CurrentOption:=\CurrentOption\do{%
          \@ifundefined{%
            KV@#1@\expandafter\KVO@getkey\CurrentOption=\@nil
          }{%
%    \end{macrocode}
%    A class puts not used options in the unused option list
%    unless there is a default handler.
%    \begin{macrocode}
            \@ifundefined{KVO@default@\@currname.\@currext}{%
              \ifx\KVO@Patch Y%
                \ltx@onelevel@sanitize\CurrentOption
              \fi
              \ifx\@unusedoptionlist\@empty
                \global\let\@unusedoptionlist\CurrentOption
              \else
                \expandafter\expandafter\expandafter\gdef
                \expandafter\expandafter\expandafter\@unusedoptionlist
                \expandafter\expandafter\expandafter{%
                  \expandafter\@unusedoptionlist
                  \expandafter,\CurrentOption
                }%
              \fi
            }{%
              \toks\tw@\expandafter{%
                \the\toks\expandafter\tw@\expandafter,\CurrentOption
              }%
            }%
          }{%
            \toks@\expandafter{%
              \the\expandafter\toks@\expandafter,\CurrentOption
            }%
          }%
        }%
      \else
%    \end{macrocode}
%    Without default action we pass all options to \cs{setkeys}.
%    Otherwise we have to check which options are known.
%    These are passed to \cs{setkeys}. For the others the default
%    action is performed.
%    \begin{macrocode}
        \@ifundefined{KVO@default@\@currname.\@currext}{%
          \toks@\expandafter\expandafter\expandafter{%
            \expandafter\KVO@temp\the\toks@
          }%
        }{%
          \edef\CurrentOption{\the\toks@}%
          \toks@\expandafter{\KVO@temp}%
          \@for\CurrentOption:=\CurrentOption\do{%
            \@ifundefined{%
              KV@#1@\expandafter\KVO@getkey\CurrentOption=\@nil
            }{%
              \toks\tw@\expandafter{%
                \the\toks\expandafter\tw@\expandafter,\CurrentOption
              }%
            }{%
              \toks@\expandafter{%
                \the\expandafter\toks@\expandafter,\CurrentOption
              }%
            }%
          }%
        }%
      \fi
    }%
    \edef\KVO@temp{\endgroup
      \noexpand\KVO@calldefault{\the\toks\tw@}%
      \noexpand\KVO@setkeys{#1}{\the\toks@}%
    }%
  \KVO@temp
%    \end{macrocode}
%    Some cleanup of \cs{ProcessOptions}.
%    \begin{macrocode}
  \let\CurrentOption\@empty
  \AtEndOfPackage{\let\@unprocessedoptions\relax}%
}
%    \end{macrocode}
%
% \subsubsection{\cs{ProcessLocalKeyvalOptions}}
%
%    \begin{macro}{\ProcessLocalKeyvalOptions}
%    If the optional star is given, we get the family
%    name and expand it for safety.
%    \begin{macrocode}
\newcommand*{\ProcessLocalKeyvalOptions}{%
  \@ifstar{%
    \begingroup
      \edef\x{\endgroup
        \noexpand\KVO@ProcessLocalKeyvalOptions{\KVO@family}%
      }%
    \x
  }%
  \KVO@ProcessLocalKeyvalOptions
}
%    \end{macrocode}
%    \end{macro}
%
%    \begin{macrocode}
\def\KVO@ProcessLocalKeyvalOptions#1{%
  \let\@tempc\relax
  \let\KVO@temp\@empty
%    \end{macrocode}
%    Check if \cs{ProcessLocalKeyvalOptions} is called inside
%    a package.
%    \begin{macrocode}
  \ifx\@currext\@pkgextension
  \else
    \PackageError{kvoptions}{%
      \string\ProcessLocalKeyvalOptions\space is intended for packages only%
    }\@ehc
  \fi
%    \end{macrocode}
%     The package options are put into toks register \cs{toks@}.
%    \begin{macrocode}
  \begingroup
    \toks\tw@{}%
    \@ifundefined{opt@\@currname.\@currext}{%
      \toks@\expandafter{\KVO@temp}%
    }{%
      \toks@\expandafter\expandafter\expandafter{%
        \csname opt@\@currname.\@currext\endcsname
      }%
%    \end{macrocode}
%    Without default action we pass all options to \cs{setkeys}.
%    Otherwise we have to check which options are known.
%    These are passed to \cs{setkeys}. For the others the default
%    action is performed.
%    \begin{macrocode}
      \@ifundefined{KVO@default@\@currname.\@currext}{%
        \toks@\expandafter\expandafter\expandafter{%
          \expandafter\KVO@temp\the\toks@
        }%
      }{%
        \edef\CurrentOption{\the\toks@}%
        \toks@\expandafter{\KVO@temp}%
        \@for\CurrentOption:=\CurrentOption\do{%
          \@ifundefined{%
            KV@#1@\expandafter\KVO@getkey\CurrentOption=\@nil
          }{%
            \toks\tw@\expandafter{%
              \the\toks\expandafter\tw@\expandafter,\CurrentOption
            }%
          }{%
            \toks@\expandafter{%
              \the\expandafter\toks@\expandafter,\CurrentOption
            }%
          }%
        }%
      }%
    }%
    \edef\KVO@temp{\endgroup
      \noexpand\KVO@calldefault{\the\toks\tw@}%
      \noexpand\KVO@setkeys{#1}{\the\toks@}%
    }%
  \KVO@temp
%    \end{macrocode}
%    Some cleanup of \cs{ProcessOptions}.
%    \begin{macrocode}
  \let\CurrentOption\@empty
  \AtEndOfPackage{\let\@unprocessedoptions\relax}%
}
%    \end{macrocode}
%
% \subsubsection{Helper macros}
%
%    \begin{macro}{\KVO@getkey}
%    Extract the key part of a key=value pair.
%    \begin{macrocode}
\def\KVO@getkey#1=#2\@nil{#1}
%    \end{macrocode}
%    \end{macro}
%
%    \begin{macro}{\KVO@calldefault}
%    \begin{macrocode}
\def\KVO@calldefault#1{%
  \begingroup
    \def\x{#1}%
  \expandafter\endgroup
  \ifx\x\@empty
  \else
    \@for\CurrentOption:=#1\do{%
      \ifx\CurrentOption\@empty
      \else
        \expandafter\KVO@setcurrents\CurrentOption=\@nil
        \@nameuse{KVO@default@\@currname.\@currext}%
      \fi
    }%
  \fi
}
%    \end{macrocode}
%    \end{macro}
%    \begin{macro}{\KVO@setcurrents}
%    Extract the key part of a key=value pair.
%    \begin{macrocode}
\def\KVO@setcurrents#1=#2\@nil{%
  \def\CurrentOptionValue{#2}%
  \ifx\CurrentOptionValue\@empty
    \let\CurrentOptionKey\CurrentOption
    \let\CurrentOptionValue\relax
  \else
    \edef\CurrentOptionKey{\zap@space#1 \@empty}%
    \expandafter\KVO@setcurrentvalue\CurrentOption\@nil
  \fi
}
%    \end{macrocode}
%    \end{macro}
%    \begin{macro}{\KV@setcurrentvalue}
%    Here the value part is parsed. Package \xpackage{keyval}'s
%    \cs{KV@@sp@def} helps in removing spaces at the begin and
%    end of the value.
%    \begin{macrocode}
\def\KVO@setcurrentvalue#1=#2\@nil{%
  \KV@@sp@def\CurrentOptionValue{#2}%
}
%    \end{macrocode}
%    \end{macro}
%
% \subsection{\plainTeX}
%
%    Disable \LaTeX\ stuff.
%    \begin{macrocode}
\begingroup\expandafter\expandafter\expandafter\endgroup
\expandafter\ifx\csname documentclass\endcsname\relax
  \def\ProcessKeyvalOptions{%
    \@ifstar{}\@gobble
  }%
\fi
%    \end{macrocode}
%
%    \begin{macrocode}
\KVO@AtEnd%
%</package>
%    \end{macrocode}
%
% \subsection{Package \xpackage{kvoptions-patch}}
%
%    \begin{macrocode}
%<*patch>
\NeedsTeXFormat{LaTeX2e}
\begingroup\catcode61\catcode48\catcode32=10\relax%
  \catcode13=5 % ^^M
  \endlinechar=13 %
  \catcode123=1 % {
  \catcode125=2 % }
  \catcode64=11 % @
  \def\x{\endgroup
    \expandafter\edef\csname KVO@AtEnd\endcsname{%
      \endlinechar=\the\endlinechar\relax
      \catcode13=\the\catcode13\relax
      \catcode32=\the\catcode32\relax
      \catcode35=\the\catcode35\relax
      \catcode61=\the\catcode61\relax
      \catcode64=\the\catcode64\relax
      \catcode123=\the\catcode123\relax
      \catcode125=\the\catcode125\relax
    }%
  }%
\x\catcode61\catcode48\catcode32=10\relax%
\catcode13=5 % ^^M
\endlinechar=13 %
\catcode35=6 % #
\catcode64=11 % @
\catcode123=1 % {
\catcode125=2 % }
\def\TMP@EnsureCode#1#2{%
  \edef\KVO@AtEnd{%
    \KVO@AtEnd
    \catcode#1=\the\catcode#1\relax
  }%
  \catcode#1=#2\relax
}
\TMP@EnsureCode{39}{12}% '
\TMP@EnsureCode{40}{12}% (
\TMP@EnsureCode{41}{12}% )
\TMP@EnsureCode{43}{12}% +
\TMP@EnsureCode{44}{12}% ,
\TMP@EnsureCode{45}{12}% -
\TMP@EnsureCode{46}{12}% .
\TMP@EnsureCode{47}{12}% /
\TMP@EnsureCode{58}{12}% :
\TMP@EnsureCode{60}{12}% <
\TMP@EnsureCode{62}{12}% >
\TMP@EnsureCode{91}{12}% [
\TMP@EnsureCode{93}{12}% ]
\TMP@EnsureCode{96}{12}% `
\TMP@EnsureCode{124}{12}% |
\edef\KVO@AtEnd{\KVO@AtEnd\noexpand\endinput}
\ProvidesPackage{kvoptions-patch}%
  [2016/05/16 v3.12 LaTeX patch for keyval options (HO)]%
%    \end{macrocode}
%
%    Check for \eTeX.
%    \begin{macrocode}
\begingroup\expandafter\expandafter\expandafter\endgroup
\expandafter\ifx\csname eTeXversion\endcsname\relax
  \PackageWarningNoLine{kvoptions-patch}{%
    Package loading is aborted, because e-TeX is missing%
  }%
  \expandafter\KVO@AtEnd
\fi%
%    \end{macrocode}
%
%    Package \xpackage{etexcmds} for \cs{etex@unexpanded}.
%    \begin{macrocode}
\RequirePackage{etexcmds}[2007/09/09]
\ifetex@unexpanded
\else
  \PackageError{kvoptions-patch}{%
    Could not find eTeX's \string\unexpanded.\MessageBreak
    Try adding \string\RequirePackage\string{etexcmds\string} %
    before \string\documentclass%
  }\@ehd
  \expandafter\KVO@AtEnd
\fi%
%    \end{macrocode}
%
%    Check for package \xpackage{xkvltxp}.
%    \begin{macrocode}
\@ifpackageloaded{xkvltxp}{%
  \PackageWarningNoLine{kvoptions}{%
    Option `patch' cannot be used together with\MessageBreak
    package `xkvltxp' that is already loaded.\MessageBreak
    Therefore package loading is aborted%
  }%
  \KVO@AtEnd
}{}%
%    \end{macrocode}
%
%    \begin{macrocode}
\def\@if@ptions#1#2#3{%
  \begingroup
    \KVO@normalize\KVO@temp{#3}%
    \edef\x{\endgroup
      \noexpand\@if@pti@ns{%
        \detokenize\expandafter\expandafter\expandafter{%
          \csname opt@#2.#1\endcsname
        }%
      }{%
        \detokenize\expandafter{\KVO@temp}%
      }%
    }%
  \x
}
%    \end{macrocode}
%
%    \begin{macrocode}
\def\@pass@ptions#1#2#3{%
  \KVO@normalize\KVO@temp{#2}%
  \@ifundefined{opt@#3.#1}{%
    \expandafter\gdef\csname opt@#3.#1%
          \expandafter\endcsname\expandafter{%
      \KVO@temp
    }%
  }{%
    \expandafter\gdef\csname opt@#3.#1%
          \expandafter\expandafter\expandafter\endcsname
          \expandafter\expandafter\expandafter{%
      \csname opt@#3.#1\expandafter\endcsname\expandafter,\KVO@temp
    }%
  }%
}
%    \end{macrocode}
%
%    \begin{macrocode}
\def\ProcessOptions{%
  \let\ds@\@empty
  \@ifundefined{opt@\@currname.\@currext}{%
    \let\@curroptions\@empty
  }{%
    \expandafter\expandafter\expandafter\def
    \expandafter\expandafter\expandafter\@curroptions
    \expandafter\expandafter\expandafter{%
      \csname opt@\@currname.\@currext\endcsname
    }%
  }%
  \@ifstar\KVO@xprocess@ptions\KVO@process@ptions
}
%    \end{macrocode}
%
%    \begin{macrocode}
\def\KVO@process@ptions{%
  \@for\CurrentOption:=\@declaredoptions\do{%
    \ifx\CurrentOption\@empty
    \else
      \begingroup
        \ifx\@currext\@clsextension
          \toks@{}%
        \else
          \KVO@GetClassOptionsList
          \toks@\expandafter{\KVO@classoptionslist,}%
        \fi
        \toks\tw@\expandafter{\@curroptions}%
        \edef\x{\endgroup
          \noexpand\in@{,\CurrentOption,}{,\the\toks@\the\toks\tw@,}%
        }%
      \x
      \ifin@
        \KVO@use@ption
        \expandafter\let\csname ds@\CurrentOption\endcsname\@empty
      \fi
    \fi
  }%
  \KVO@process@pti@ns
}
%    \end{macrocode}
%
%    \begin{macrocode}
\def\KVO@xprocess@ptions{%
  \ifx\@currext\@clsextension
  \else
    \KVO@GetClassOptionsList
    \@for\CurrentOption:=\KVO@classoptionslist\do{%
      \ifx\CurrentOption\@empty
      \else
        \KVO@in@\CurrentOption\@declaredoptions
        \ifin@
          \KVO@use@ption
          \expandafter\let\csname ds@\CurrentOption\endcsname\@empty
        \fi
      \fi
    }%
  \fi
  \KVO@process@pti@ns
}
%    \end{macrocode}
%
%    \begin{macrocode}
\def\KVO@in@#1#2{%
  \in@false
  \begingroup
    \@for\x:=#2\do{%
      \ifx\x#1\relax
        \in@true
      \fi
    }%
    \edef\x{\endgroup
      \ifin@
        \noexpand\in@true
      \fi
    }%
  \x
}
%    \end{macrocode}
%
%    \begin{macrocode}
\def\KVO@process@pti@ns{%
  \@for\CurrentOption:=\@curroptions\do{%
    \@ifundefined{ds@\KVO@SanitizedCurrentOption}{%
      \KVO@use@ption
      \default@ds
    }%
    \KVO@use@ption
  }%
  \@for\CurrentOption:=\@declaredoptions\do{%
    \expandafter\let\csname ds@\CurrentOption\endcsname\relax
  }%
  \let\CurrentOption\@empty
  \let\@fileswith@pti@ns\@@fileswith@pti@ns
  \AtEndOfPackage{\let\@unprocessedoptions\relax}%
}
%    \end{macrocode}
%
%    \begin{macrocode}
\def\KVO@use@ption{%
  \begingroup
    \edef\x{\endgroup
      \noexpand\@removeelement{%
        \detokenize\expandafter{\CurrentOption}%
      }{%
        \detokenize\expandafter{\@unusedoptionlist}%
      }%
    }%
  \x\@unusedoptionlist
  \csname ds@\KVO@SanitizedCurrentOption\endcsname
}
%    \end{macrocode}
%
%    \begin{macrocode}
\def\OptionNotUsed{%
  \ifx\@currext\@clsextension
    \xdef\@unusedoptionlist{%
      \ifx\@unusedoptionlist\@empty
      \else
        \detokenize\expandafter{\@unusedoptionlist,}%
      \fi
      \detokenize\expandafter{\CurrentOption}%
    }%
  \fi
}
%    \end{macrocode}
%
%    Variant of \cs{ExecuteOptions} that better protects \cs{CurrentOption}.
%    \begin{macrocode}
\def\CurrentOption@SaveLevel{0}
\def\ExecuteOptions{%
  \expandafter\KVO@ExecuteOptions
      \csname CurrentOption@\CurrentOption@SaveLevel\endcsname
}
\def\KVO@ExecuteOptions#1#2{%
  \let#1\CurrentOption
  \edef\CurrentOption@SaveLevel{%
    \the\numexpr\CurrentOption@SaveLevel+1%
  }%
  \@for\CurrentOption:=#2\do{%
    \csname ds@\CurrentOption\endcsname
  }%
  \edef\CurrentOption@SaveLevel{%
    \the\numexpr\CurrentOption@SaveLevel-1%
  }%
  \let\CurrentOption#1%
}
%    \end{macrocode}
%
%    \begin{macrocode}
\def\KVO@fileswith@pti@ns#1[#2]#3[#4]{%
  \ifx#1\@clsextension
    \ifx\@classoptionslist\relax
      \KVO@normalize\KVO@temp{#2}%
      \expandafter\gdef\expandafter\@classoptionslist\expandafter{%
        \KVO@temp
      }%
      \def\reserved@a{%
        \KVO@onefilewithoptions{#3}[{#2}][{#4}]#1%
        \@documentclasshook
      }%
    \else
      \def\reserved@a{%
        \KVO@onefilewithoptions{#3}[{#2}][{#4}]#1%
      }%
    \fi
  \else
    \begingroup
      \let\KVO@temp\relax
      \let\KVO@onefilewithoptions\relax
      \let\@pkgextension\relax
      \def\reserved@b##1,{%
        \ifx\@nil##1\relax
        \else
          \ifx\relax##1\relax
          \else
            \KVO@onefilewithoptions{##1}[{\KVO@temp}][{#4}]%
            \@pkgextension
          \fi
          \expandafter\reserved@b
        \fi
      }%
      \edef\reserved@a{\zap@space#3 \@empty}%
      \edef\reserved@a{\expandafter\reserved@b\reserved@a,\@nil,}%
      \toks@{#2}%
      \def\KVO@temp{\the\toks@}%
    \edef\reserved@a{\endgroup \reserved@a}%
  \fi
  \reserved@a
}
%    \end{macrocode}
%
%    \begin{macrocode}
\def\KVO@onefilewithoptions#1[#2][#3]#4{%
  \@pushfilename
  \xdef\@currname{#1}%
  \global\let\@currext#4%
  \expandafter\let\csname\@currname.\@currext-h@@k\endcsname\@empty
  \let\CurrentOption\@empty
  \@reset@ptions
  \makeatletter
  \def\reserved@a{%
    \@ifl@aded\@currext{#1}{%
      \@if@ptions\@currext{#1}{#2}{%
      }{%
        \begingroup
          \@ifundefined{opt@#1.\@currext}{%
            \def\x{}%
          }{%
            \edef\x{%
              \expandafter\expandafter\expandafter\strip@prefix
              \expandafter\meaning\csname opt@#1.\@currext\endcsname
            }%
          }%
          \def\y{#2}%
          \edef\y{\expandafter\strip@prefix\meaning\y}%
          \@latex@error{Option clash for \@cls@pkg\space #1}{%
            The package #1 has already been loaded %
            with options:\MessageBreak
            \space\space[\x]\MessageBreak
            There has now been an attempt to load it %
             with options\MessageBreak
            \space\space[\y]\MessageBreak
            Adding the global options:\MessageBreak
            \space\space
                 \x,\y\MessageBreak
            to your \noexpand\documentclass declaration may fix this.%
            \MessageBreak
            Try typing \space <return> \space to proceed.%
          }%
        \endgroup
      }%
    }{%
      \@pass@ptions\@currext{#2}{#1}%
      \global\expandafter
      \let\csname ver@\@currname.\@currext\endcsname\@empty
      \InputIfFileExists
        {\@currname.\@currext}%
        {}%
        {\@missingfileerror\@currname\@currext}%
      \let\@unprocessedoptions\@@unprocessedoptions
      \csname\@currname.\@currext-h@@k\endcsname
      \expandafter\let\csname\@currname.\@currext-h@@k\endcsname
              \@undefined
      \@unprocessedoptions
    }%
    \@ifl@ter\@currext{#1}{#3}{%
    }{%
      \@latex@warning@no@line{%
        You have requested,\on@line, %
        version\MessageBreak
          #3' of \@cls@pkg\space #1,\MessageBreak
        but only version\MessageBreak
         `\csname ver@#1.\@currext\endcsname'\MessageBreak
        is available%
      }%
    }%
    \ifx\@currext\@clsextension\let\LoadClass\@twoloadclasserror\fi
    \@popfilename
    \@reset@ptions
  }%
  \reserved@a
}
%    \end{macrocode}
%
%    \begin{macrocode}
\def\@unknownoptionerror{%
  \@latex@error{%
    Unknown option `\KVO@SanitizedCurrentOption' %
    for \@cls@pkg\space`\@currname'%
  }{%
    The option `\KVO@SanitizedCurrentOption' was not declared in %
    \@cls@pkg\space`\@currname', perhaps you\MessageBreak
    misspelled its name. %
    Try typing \space <return> %
    \space to proceed.%
  }%
}
%    \end{macrocode}
%
%    \begin{macrocode}
\def\@@unprocessedoptions{%
  \ifx\@currext\@pkgextension
    \@ifundefined{opt@\@currname.\@currext}{%
      \let\@curroptions\@empty
    }{%
      \expandafter\let\expandafter\@curroptions
          \csname opt@\@currname.\@currext\endcsname
    }%
    \@for\CurrentOption:=\@curroptions\do{%
        \ifx\CurrentOption\@empty\else\@unknownoptionerror\fi
    }%
  \fi
}
%    \end{macrocode}
%
%    \begin{macrocode}
\def\KVO@SanitizedCurrentOption{%
  \expandafter\strip@prefix\meaning\CurrentOption
}
%    \end{macrocode}
%
%    Normalize option list.
%    \begin{macrocode}
\def\KVO@normalize#1#2{%
  \let\KVO@result\@empty
  \KVO@splitcomma#2,\@nil
  \let#1\KVO@result
}
\def\KVO@splitcomma#1,#2\@nil{%
  \KVO@ifempty{#1}{}{%
    \KVO@checkkv#1=\@nil
  }%
  \KVO@ifempty{#2}{}{\KVO@splitcomma#2\@nil}%
}
\def\KVO@ifempty#1{%
  \expandafter\ifx\expandafter\\\detokenize{#1}\\%
    \expandafter\@firstoftwo
  \else
    \expandafter\@secondoftwo
  \fi
}
\def\KVO@checkkv#1=#2\@nil{%
  \KVO@ifempty{#2}{%
    % option without value
    \edef\KVO@x{\zap@space#1 \@empty}%
    \ifx\KVO@x\@empty
      % ignore empty option
    \else
      % append to list
      \edef\KVO@result{%
        \etex@unexpanded\expandafter{\KVO@result},\KVO@x
      }%
    \fi
  }{%
    % #1: "key", #2: "value="
    % add key part
    \edef\KVO@result{%
      \etex@unexpanded\expandafter{\KVO@result},%
      \zap@space#1 \@empty
    }%
    \futurelet\@let@token\KVO@checkfirsttok#2 \@nil| = \@nil|\KVO@nil
  }%
}
\def\KVO@checkfirsttok{%
  \ifx\@let@token\bgroup
    % no space at start
    \expandafter\KVO@removelastspace\expandafter=%
    % "<value><spaceopt>= \@nil"
  \else
    \expandafter\KVO@checkfirstA
  \fi
}
\def\KVO@checkfirstA#1 #2\@nil{%
  \KVO@ifempty{#2}{%
    \KVO@removelastspace=#1 \@nil
  }{%
    \KVO@ifempty{#1}{%
      \KVO@removelastspace=#2\@nil
    }{%
      \KVO@removelastspace=#1 #2\@nil
    }%
  }%
}
\def\KVO@removelastspace#1 = \@nil|#2\KVO@nil{%
  \KVO@ifempty{#2}{%
    \edef\KVO@result{%
      \etex@unexpanded\expandafter{\KVO@result}%
      \etex@unexpanded\expandafter{\KVO@removegarbage#1\KVO@nil}%
    }%
  }{%
    \edef\KVO@result{%
      \etex@unexpanded\expandafter{\KVO@result}%
      \etex@unexpanded{#1}%
    }%
  }%
}
\def\KVO@removegarbage#1= \@nil#2\KVO@nil{#1}%
%    \end{macrocode}
%
%    Arguments |#1| and |#2| are macros.
%    \begin{macrocode}
\def\KVO@removeelement#1#2{%
  \begingroup
    \toks@={}%
    \@for\x:=#2\do{%
      \ifx\x\@empty
      \else
        \ifx\x#1\relax
        \else
          \edef\t{\the\toks@}%
          \ifx\t\@empty
          \else
            \toks@\expandafter{\the\toks@,}%
          \fi
          \toks@\expandafter{\the\expandafter\toks@\x}%
        \fi
      \fi
    }%
    \edef\x{\endgroup
      \def\noexpand#2{\the\toks@}%
    }%
  \x
}
%    \end{macrocode}
%
%
%    \begin{macrocode}
\let\@@fileswith@pti@ns\KVO@fileswith@pti@ns
\ifx\@fileswith@pti@ns\@badrequireerror
\else
  \let\@fileswith@pti@ns\KVO@fileswith@pti@ns
\fi
%    \end{macrocode}
%
%    \begin{macro}{\KVO@Patch}
%    \begin{macrocode}
\let\KVO@Patch=Y
%    \end{macrocode}
%    \end{macro}
%
%    \begin{macrocode}
\KVO@AtEnd%
%</patch>
%    \end{macrocode}
%
% \section{Test}
%
% \subsection{Preface for standard catcode check}
%
%    \begin{macrocode}
%<*test1>
\input miniltx.tex\relax
%</test1>
%    \end{macrocode}
%
% \subsection{Catcode checks for loading}
%
%    \begin{macrocode}
%<*test1>
%    \end{macrocode}
%    \begin{macrocode}
\catcode`\{=1 %
\catcode`\}=2 %
\catcode`\#=6 %
\catcode`\@=11 %
\expandafter\ifx\csname count@\endcsname\relax
  \countdef\count@=255 %
\fi
\expandafter\ifx\csname @gobble\endcsname\relax
  \long\def\@gobble#1{}%
\fi
\expandafter\ifx\csname @firstofone\endcsname\relax
  \long\def\@firstofone#1{#1}%
\fi
\expandafter\ifx\csname loop\endcsname\relax
  \expandafter\@firstofone
\else
  \expandafter\@gobble
\fi
{%
  \def\loop#1\repeat{%
    \def\body{#1}%
    \iterate
  }%
  \def\iterate{%
    \body
      \let\next\iterate
    \else
      \let\next\relax
    \fi
    \next
  }%
  \let\repeat=\fi
}%
\def\RestoreCatcodes{}
\count@=0 %
\loop
  \edef\RestoreCatcodes{%
    \RestoreCatcodes
    \catcode\the\count@=\the\catcode\count@\relax
  }%
\ifnum\count@<255 %
  \advance\count@ 1 %
\repeat

\def\RangeCatcodeInvalid#1#2{%
  \count@=#1\relax
  \loop
    \catcode\count@=15 %
  \ifnum\count@<#2\relax
    \advance\count@ 1 %
  \repeat
}
\def\RangeCatcodeCheck#1#2#3{%
  \count@=#1\relax
  \loop
    \ifnum#3=\catcode\count@
    \else
      \errmessage{%
        Character \the\count@\space
        with wrong catcode \the\catcode\count@\space
        instead of \number#3%
      }%
    \fi
  \ifnum\count@<#2\relax
    \advance\count@ 1 %
  \repeat
}
\def\space{ }
\expandafter\ifx\csname LoadCommand\endcsname\relax
  \def\LoadCommand{\input kvoptions.sty\relax}%
\fi
\def\Test{%
  \RangeCatcodeInvalid{0}{47}%
  \RangeCatcodeInvalid{58}{64}%
  \RangeCatcodeInvalid{91}{96}%
  \RangeCatcodeInvalid{123}{255}%
  \catcode`\@=12 %
  \catcode`\\=0 %
  \catcode`\%=14 %
  \LoadCommand
  \RangeCatcodeCheck{0}{36}{15}%
  \RangeCatcodeCheck{37}{37}{14}%
  \RangeCatcodeCheck{38}{47}{15}%
  \RangeCatcodeCheck{48}{57}{12}%
  \RangeCatcodeCheck{58}{63}{15}%
  \RangeCatcodeCheck{64}{64}{12}%
  \RangeCatcodeCheck{65}{90}{11}%
  \RangeCatcodeCheck{91}{91}{15}%
  \RangeCatcodeCheck{92}{92}{0}%
  \RangeCatcodeCheck{93}{96}{15}%
  \RangeCatcodeCheck{97}{122}{11}%
  \RangeCatcodeCheck{123}{255}{15}%
  \RestoreCatcodes
}
\Test
\csname @@end\endcsname
\end
%    \end{macrocode}
%    \begin{macrocode}
%</test1>
%    \end{macrocode}
%
%    \begin{macrocode}
%<*test2>
\NeedsTeXFormat{LaTeX2e}
\makeatletter
\catcode`\@=11 %
\def\RestoreCatcodes{}
\count@=0 %
\loop
  \edef\RestoreCatcodes{%
    \RestoreCatcodes
    \catcode\the\count@=\the\catcode\count@\relax
  }%
\ifnum\count@<255 %
  \advance\count@\@ne
\repeat

\def\RangeCatcodeInvalid#1#2{%
  \count@=#1\relax
  \loop
    \catcode\count@=15 %
  \ifnum\count@<#2\relax
    \advance\count@\@ne
  \repeat
}
\def\Test#1{%
  \RangeCatcodeInvalid{0}{47}%
  \RangeCatcodeInvalid{58}{64}%
  \RangeCatcodeInvalid{91}{96}%
  \RangeCatcodeInvalid{123}{255}%
  \catcode`\@=12 %
  \catcode`\\=0 %
  \catcode`\{=1 %
  \catcode`\}=2 %
  \catcode`\#=6 %
  \catcode`\[=12 %
  \catcode`\]=12 %
  \catcode`\%=14 %
  \catcode`\ =10 %
  \catcode13=5 %
  #1\relax
  \RestoreCatcodes
}
\Test{\RequirePackage{kvoptions-patch}}%
\Test{\RequirePackage{kvoptions}}%
\csname @@end\endcsname
%</test2>
%    \end{macrocode}
%
%    \begin{macrocode}
%<*test3>
\NeedsTeXFormat{LaTeX2e}
\makeatletter
\RequirePackage{kvoptions}[2016/05/16]
\def\msg#{\immediate\write16}
\define@key{testfamily}{testkey}{%
  \msg{[testfamily/testkey/#1]}%
}
\define@key{testfamily}{testdefaultkey}[testdefault]{%
  \msg{[testfamily/testdefaultkey/#1]}%
}
\AddToKeyvalOption{testfamily}{testkey}{%
  \msg{[addition/#1]}%
}
\AddToKeyvalOption{testfamily}{testdefaultkey}{%
  \msg{[addition/#1]}%
}
\setkeys{testfamily}{%
  testkey=testA,%
  testdefaultkey=testB,%
  testdefaultkey,%
}
\SetupKeyvalOptions{%
  family=testfamily%
}
\AddToKeyvalOption*{testkey}{%
  \msg{[star addition/#1]}%
}
\AddToKeyvalOption*{testdefaultkey}{%
  \msg{[star addition/#1]}%
}
\setkeys{testfamily}{%
  testkey=testA,%
  testdefaultkey=testB,%
  testdefaultkey,%
}
\@@end
%</test3>
%    \end{macrocode}
%
%    \begin{macrocode}
%<*test4pkg>
\NeedsTeXFormat{LaTeX2e}
\ProvidesPackage{kvoptions-test4}[2016/05/16 package for testing]
\RequirePackage{kvoptions}[2016/05/16]
\SetupKeyvalOptions{%
  family=FOO,%
  prefix=foo,%
  setkeys=\kvsetkeys,%
}
\DeclareStringOption{str}
\define@key{FOO}{set}{%
  \setkeys{BAR}{strbar={#1}}%
}
\define@key{BAR}{strbar}{%
  \def\foostr{[BAR:#1]}%
}
\ProcessKeyvalOptions*
%</test4pkg>
%    \end{macrocode}
%    \begin{macrocode}
%<*test4>
\NeedsTeXFormat{LaTeX2e}
\ProvidesFile{kvoptions-test4.tex}[2016/05/16 test file]
\RequirePackage[%
  str=A,set=B,str=C,%
]{kvoptions-test4}[2016/05/16]
\def\TestExpected{C}
\ifx\foostr\TestExpected
  \typeout{* Test ok.}%
\else
  \typeout{* Result: [\foostr]}%
  \typeout{* Expected: [\TestExpected]}%
  \errmessage{Test failed!}%
\fi
\csname @@end\endcsname\end
%</test4>
%    \end{macrocode}
%
% \section{Installation}
%
% \subsection{Download}
%
% \paragraph{Package.} This package is available on
% CTAN\footnote{\CTANpkg{kvoptions}}:
% \begin{description}
% \item[\CTAN{macros/latex/contrib/oberdiek/kvoptions.dtx}] The source file.
% \item[\CTAN{macros/latex/contrib/oberdiek/kvoptions.pdf}] Documentation.
% \end{description}
%
%
% \paragraph{Bundle.} All the packages of the bundle `oberdiek'
% are also available in a TDS compliant ZIP archive. There
% the packages are already unpacked and the documentation files
% are generated. The files and directories obey the TDS standard.
% \begin{description}
% \item[\CTANinstall{install/macros/latex/contrib/oberdiek.tds.zip}]
% \end{description}
% \emph{TDS} refers to the standard ``A Directory Structure
% for \TeX\ Files'' (\CTAN{tds/tds.pdf}). Directories
% with \xfile{texmf} in their name are usually organized this way.
%
% \subsection{Bundle installation}
%
% \paragraph{Unpacking.} Unpack the \xfile{oberdiek.tds.zip} in the
% TDS tree (also known as \xfile{texmf} tree) of your choice.
% Example (linux):
% \begin{quote}
%   |unzip oberdiek.tds.zip -d ~/texmf|
% \end{quote}
%
% \paragraph{Script installation.}
% Check the directory \xfile{TDS:scripts/oberdiek/} for
% scripts that need further installation steps.
%
% \subsection{Package installation}
%
% \paragraph{Unpacking.} The \xfile{.dtx} file is a self-extracting
% \docstrip\ archive. The files are extracted by running the
% \xfile{.dtx} through \plainTeX:
% \begin{quote}
%   \verb|tex kvoptions.dtx|
% \end{quote}
%
% \paragraph{TDS.} Now the different files must be moved into
% the different directories in your installation TDS tree
% (also known as \xfile{texmf} tree):
% \begin{quote}
% \def\t{^^A
% \begin{tabular}{@{}>{\ttfamily}l@{ $\rightarrow$ }>{\ttfamily}l@{}}
%   kvoptions.sty & tex/latex/oberdiek/kvoptions.sty\\
%   kvoptions-patch.sty & tex/latex/oberdiek/kvoptions-patch.sty\\
%   kvoptions.pdf & doc/latex/oberdiek/kvoptions.pdf\\
%   example-mycolorsetup.sty & doc/latex/oberdiek/example-mycolorsetup.sty\\
%   test/kvoptions-test1.tex & doc/latex/oberdiek/test/kvoptions-test1.tex\\
%   test/kvoptions-test2.tex & doc/latex/oberdiek/test/kvoptions-test2.tex\\
%   test/kvoptions-test3.tex & doc/latex/oberdiek/test/kvoptions-test3.tex\\
%   test/kvoptions-test4.tex & doc/latex/oberdiek/test/kvoptions-test4.tex\\
%   test/kvoptions-test4.sty & doc/latex/oberdiek/test/kvoptions-test4.sty\\
%   kvoptions.dtx & source/latex/oberdiek/kvoptions.dtx\\
% \end{tabular}^^A
% }^^A
% \sbox0{\t}^^A
% \ifdim\wd0>\linewidth
%   \begingroup
%     \advance\linewidth by\leftmargin
%     \advance\linewidth by\rightmargin
%   \edef\x{\endgroup
%     \def\noexpand\lw{\the\linewidth}^^A
%   }\x
%   \def\lwbox{^^A
%     \leavevmode
%     \hbox to \linewidth{^^A
%       \kern-\leftmargin\relax
%       \hss
%       \usebox0
%       \hss
%       \kern-\rightmargin\relax
%     }^^A
%   }^^A
%   \ifdim\wd0>\lw
%     \sbox0{\small\t}^^A
%     \ifdim\wd0>\linewidth
%       \ifdim\wd0>\lw
%         \sbox0{\footnotesize\t}^^A
%         \ifdim\wd0>\linewidth
%           \ifdim\wd0>\lw
%             \sbox0{\scriptsize\t}^^A
%             \ifdim\wd0>\linewidth
%               \ifdim\wd0>\lw
%                 \sbox0{\tiny\t}^^A
%                 \ifdim\wd0>\linewidth
%                   \lwbox
%                 \else
%                   \usebox0
%                 \fi
%               \else
%                 \lwbox
%               \fi
%             \else
%               \usebox0
%             \fi
%           \else
%             \lwbox
%           \fi
%         \else
%           \usebox0
%         \fi
%       \else
%         \lwbox
%       \fi
%     \else
%       \usebox0
%     \fi
%   \else
%     \lwbox
%   \fi
% \else
%   \usebox0
% \fi
% \end{quote}
% If you have a \xfile{docstrip.cfg} that configures and enables \docstrip's
% TDS installing feature, then some files can already be in the right
% place, see the documentation of \docstrip.
%
% \subsection{Refresh file name databases}
%
% If your \TeX~distribution
% (\TeX\,Live, \mikTeX, \dots) relies on file name databases, you must refresh
% these. For example, \TeX\,Live\ users run \verb|texhash| or
% \verb|mktexlsr|.
%
% \subsection{Some details for the interested}
%
% \paragraph{Unpacking with \LaTeX.}
% The \xfile{.dtx} chooses its action depending on the format:
% \begin{description}
% \item[\plainTeX:] Run \docstrip\ and extract the files.
% \item[\LaTeX:] Generate the documentation.
% \end{description}
% If you insist on using \LaTeX\ for \docstrip\ (really,
% \docstrip\ does not need \LaTeX), then inform the autodetect routine
% about your intention:
% \begin{quote}
%   \verb|latex \let\install=y\input{kvoptions.dtx}|
% \end{quote}
% Do not forget to quote the argument according to the demands
% of your shell.
%
% \paragraph{Generating the documentation.}
% You can use both the \xfile{.dtx} or the \xfile{.drv} to generate
% the documentation. The process can be configured by the
% configuration file \xfile{ltxdoc.cfg}. For instance, put this
% line into this file, if you want to have A4 as paper format:
% \begin{quote}
%   \verb|\PassOptionsToClass{a4paper}{article}|
% \end{quote}
% An example follows how to generate the
% documentation with pdf\LaTeX:
% \begin{quote}
%\begin{verbatim}
%pdflatex kvoptions.dtx
%makeindex -s gind.ist kvoptions.idx
%pdflatex kvoptions.dtx
%makeindex -s gind.ist kvoptions.idx
%pdflatex kvoptions.dtx
%\end{verbatim}
% \end{quote}
%
% \begin{thebibliography}{99}
% \bibitem{tb94wright}
%   A guide to key-value methods, Joseph Wright, second draft for
%   \href{https://www.tug.org/tugboat}{TUGBoat}, 2009-03-17.
%   \url{https://www.texdev.net/uploads/2009/03/keyval.pdf}
% \bibpackage{ifthen}{David Carlisle}{2001/05/26}
%   {\CTANpkg{ifthen}}
% \bibpackage{helvet}{Sebastian Rahtz, Walter Schmidt}{2004/01/26}
%   {\CTANpkg{psfonts}}
% \bibpackage{hyperref}{Sebastian Rahtz, Heiko Oberdiek}{2006/02/12}
%   {\CTANpkg{hyperref}}
% \bibpackage{keyval}{David Carlisle}{1999/03/16}
%   {\CTANpkg{keyval}}
% \bibpackage{multicol}{Frank Mittelbach}{2004/02/14}
%   {\CTANpkg{multicol}}
% \bibpackage{tabularx}{David Carlisle}{1999/01/07}
%   {\CTANpkg{tabularx}}
% \bibpackage{tracefnt}{Frank Mittelbach, Rainer Sch\"opf}{1997/05/29}
%   {\CTANpkg{latex-base}}
% \bibpackage{xkeyval}{Hendri Adriaens}{2005/05/07}
%   {\CTANpkg{xkeyval}}
% \bibitem{clsguide}
%   The \LaTeX3 Project, \textit{\LaTeXe\ for class and package writers},
%   2003/12/09.
%   \CTANpkg{clsguide}
%
% \end{thebibliography}
%
% \begin{History}
%   \begin{Version}{0000/00/00 v0.0}
%   \item
%     Probably David Carlisle's code in \xpackage{hyperref}
%     was the start.
%   \end{Version}
%   \begin{Version}{2004/02/22 v1.0}
%   \item
%     The first version was never published. It also
%     has offered a patch to get rid of \LaTeX's option
%     expansion.
%   \end{Version}
%   \begin{Version}{2006/02/16 v2.0}
%   \item
%     Now the package is redesigned with an easier
%     user interface.
%   \item
%     \cs{ProcessKeyvalOptions} remains the central service, inherited
%     from \xpackage{hyperref}'s \cs{ProcessOptionsWithKV}.
%     Now the use inside classes is also supported.
%   \item
%     Provides help macros for boolean and simple string options.
%   \item
%     Fixes for the patch of \LaTeX{}. The patch is only enabled,
%     if the user requests it.
%   \end{Version}
%   \begin{Version}{2006/02/20 v2.1}
%   \item
%     Unused option list is sanitized to prevent problems
%     with other packages that uses own processing methods
%     for key value options. Disadvantage: the unused global
%     option detection is weakened.
%   \item
%     New option type by \cs{DeclareVoidOption} for options without
%     value.
%   \item
%     Default rule by \cs{DeclareDefaultOption}.
%   \item
%     Dynamic options: \cs{DisableKeyvalOption}.
%   \end{Version}
%   \begin{Version}{2006/06/01 v2.2}
%   \item
%     Fixes for option \xoption{patch}.
%   \end{Version}
%   \begin{Version}{2006/08/17 v2.3}
%   \item
%     \cs{DeclareBooleanOption} renamed to \cs{DeclareBoolOption}
%     to avoid a name clash with package \cs{ifoption}.
%   \end{Version}
%   \begin{Version}{2006/08/22 v2.4}
%   \item
%     Option \xoption{patch}: \cs{ExecuteOptions} does not
%     change the meaning of macro \cs{CurrentOption} at all.
%   \end{Version}
%   \begin{Version}{2007/04/11 v2.5}
%   \item
%     Line ends sanitized.
%   \end{Version}
%   \begin{Version}{2007/05/06 v2.6}
%   \item
%     Uses package \xpackage{etexcmds}.
%   \end{Version}
%   \begin{Version}{2007/06/11 v2.7}
%   \item
%     The patch part fixes LaTeX bug latex/3965.
%   \end{Version}
%   \begin{Version}{2007/10/02 v2.8}
%   \item
%     Compatibility for \plainTeX\ added.
%   \item
%     Typos in documentation fixed (Axel Sommerfeldt).
%   \end{Version}
%   \begin{Version}{2007/10/11 v2.9}
%   \item
%     Bug fix for option \xoption{patch}.
%   \end{Version}
%   \begin{Version}{2007/10/18 v3.0}
%   \item
%     New package \xpackage{kvoptions-patch}.
%   \end{Version}
%   \begin{Version}{2009/04/10 v3.1}
%   \item
%     Space by line end removed in definition of internal macro.
%   \end{Version}
%   \begin{Version}{2009/07/17 v3.2}
%   \item
%     \cs{ProcessLocalKeyvalOptions} added.
%   \item
%     \cs{DisableKeyvalOption} with the \texttt{action=ignore} option
%     fixed (Joseph Wright).
%   \end{Version}
%   \begin{Version}{2009/07/21 v3.3}
%   \item
%     \cs{DeclareLocalOption}, \cs{DeclareLocalOptions} added.
%   \end{Version}
%   \begin{Version}{2009/08/13 v3.4}
%   \item
%     Documentation addition: recommendation for Joseph Wright's
%     review article.
%   \item
%     Documentation addition: local/global options.
%   \end{Version}
%   \begin{Version}{2009/12/04 v3.5}
%   \item
%     \cs{AddToKeyvalOption} added.
%   \end{Version}
%   \begin{Version}{2009/12/08 v3.6}
%   \item
%     Fix: If a default handler is configured, it is now also
%     called for classes.
%   \end{Version}
%   \begin{Version}{2010/02/22 v3.7}
%   \item
%     Missing space in error message added.
%   \end{Version}
%   \begin{Version}{2010/07/23 v3.8}
%   \item
%     Documenation for package \xpackage{kvoptions-patch} improved.
%     No code changes.
%   \end{Version}
%   \begin{Version}{2010/12/02 v3.9}
%   \item
%     Key \texttt{setkeys} added for \cs{SetupKeyvalOptions}.
%   \end{Version}
%   \begin{Version}{2010/12/23 v3.10}
%   \item
%     \cs{DeclareVoidOption} also parses the second parameter as
%     \hologo{TeX} argument to improve compatibility with
%     \cs{DeclareOption}.
%   \end{Version}
%   \begin{Version}{2011/06/30 v3.11}
%   \item
%     Fix because of design bug in package \xpackage{xkeyval} that
%     removes global options with equal signs.
%   \end{Version}
%   \begin{Version}{2016/05/16 v3.12}
%   \item
%     Documentation updates.
%   \end{Version}
% \end{History}
%
% \PrintIndex
%
% \Finale
\endinput
|
% \end{quote}
% Do not forget to quote the argument according to the demands
% of your shell.
%
% \paragraph{Generating the documentation.}
% You can use both the \xfile{.dtx} or the \xfile{.drv} to generate
% the documentation. The process can be configured by the
% configuration file \xfile{ltxdoc.cfg}. For instance, put this
% line into this file, if you want to have A4 as paper format:
% \begin{quote}
%   \verb|\PassOptionsToClass{a4paper}{article}|
% \end{quote}
% An example follows how to generate the
% documentation with pdf\LaTeX:
% \begin{quote}
%\begin{verbatim}
%pdflatex kvoptions.dtx
%makeindex -s gind.ist kvoptions.idx
%pdflatex kvoptions.dtx
%makeindex -s gind.ist kvoptions.idx
%pdflatex kvoptions.dtx
%\end{verbatim}
% \end{quote}
%
% \begin{thebibliography}{99}
% \bibitem{tb94wright}
%   A guide to key-value methods, Joseph Wright, second draft for
%   \href{https://www.tug.org/tugboat}{TUGBoat}, 2009-03-17.
%   \url{https://www.texdev.net/uploads/2009/03/keyval.pdf}
% \bibpackage{ifthen}{David Carlisle}{2001/05/26}
%   {\CTANpkg{ifthen}}
% \bibpackage{helvet}{Sebastian Rahtz, Walter Schmidt}{2004/01/26}
%   {\CTANpkg{psfonts}}
% \bibpackage{hyperref}{Sebastian Rahtz, Heiko Oberdiek}{2006/02/12}
%   {\CTANpkg{hyperref}}
% \bibpackage{keyval}{David Carlisle}{1999/03/16}
%   {\CTANpkg{keyval}}
% \bibpackage{multicol}{Frank Mittelbach}{2004/02/14}
%   {\CTANpkg{multicol}}
% \bibpackage{tabularx}{David Carlisle}{1999/01/07}
%   {\CTANpkg{tabularx}}
% \bibpackage{tracefnt}{Frank Mittelbach, Rainer Sch\"opf}{1997/05/29}
%   {\CTANpkg{latex-base}}
% \bibpackage{xkeyval}{Hendri Adriaens}{2005/05/07}
%   {\CTANpkg{xkeyval}}
% \bibitem{clsguide}
%   The \LaTeX3 Project, \textit{\LaTeXe\ for class and package writers},
%   2003/12/09.
%   \CTANpkg{clsguide}
%
% \end{thebibliography}
%
% \begin{History}
%   \begin{Version}{0000/00/00 v0.0}
%   \item
%     Probably David Carlisle's code in \xpackage{hyperref}
%     was the start.
%   \end{Version}
%   \begin{Version}{2004/02/22 v1.0}
%   \item
%     The first version was never published. It also
%     has offered a patch to get rid of \LaTeX's option
%     expansion.
%   \end{Version}
%   \begin{Version}{2006/02/16 v2.0}
%   \item
%     Now the package is redesigned with an easier
%     user interface.
%   \item
%     \cs{ProcessKeyvalOptions} remains the central service, inherited
%     from \xpackage{hyperref}'s \cs{ProcessOptionsWithKV}.
%     Now the use inside classes is also supported.
%   \item
%     Provides help macros for boolean and simple string options.
%   \item
%     Fixes for the patch of \LaTeX{}. The patch is only enabled,
%     if the user requests it.
%   \end{Version}
%   \begin{Version}{2006/02/20 v2.1}
%   \item
%     Unused option list is sanitized to prevent problems
%     with other packages that uses own processing methods
%     for key value options. Disadvantage: the unused global
%     option detection is weakened.
%   \item
%     New option type by \cs{DeclareVoidOption} for options without
%     value.
%   \item
%     Default rule by \cs{DeclareDefaultOption}.
%   \item
%     Dynamic options: \cs{DisableKeyvalOption}.
%   \end{Version}
%   \begin{Version}{2006/06/01 v2.2}
%   \item
%     Fixes for option \xoption{patch}.
%   \end{Version}
%   \begin{Version}{2006/08/17 v2.3}
%   \item
%     \cs{DeclareBooleanOption} renamed to \cs{DeclareBoolOption}
%     to avoid a name clash with package \cs{ifoption}.
%   \end{Version}
%   \begin{Version}{2006/08/22 v2.4}
%   \item
%     Option \xoption{patch}: \cs{ExecuteOptions} does not
%     change the meaning of macro \cs{CurrentOption} at all.
%   \end{Version}
%   \begin{Version}{2007/04/11 v2.5}
%   \item
%     Line ends sanitized.
%   \end{Version}
%   \begin{Version}{2007/05/06 v2.6}
%   \item
%     Uses package \xpackage{etexcmds}.
%   \end{Version}
%   \begin{Version}{2007/06/11 v2.7}
%   \item
%     The patch part fixes LaTeX bug latex/3965.
%   \end{Version}
%   \begin{Version}{2007/10/02 v2.8}
%   \item
%     Compatibility for \plainTeX\ added.
%   \item
%     Typos in documentation fixed (Axel Sommerfeldt).
%   \end{Version}
%   \begin{Version}{2007/10/11 v2.9}
%   \item
%     Bug fix for option \xoption{patch}.
%   \end{Version}
%   \begin{Version}{2007/10/18 v3.0}
%   \item
%     New package \xpackage{kvoptions-patch}.
%   \end{Version}
%   \begin{Version}{2009/04/10 v3.1}
%   \item
%     Space by line end removed in definition of internal macro.
%   \end{Version}
%   \begin{Version}{2009/07/17 v3.2}
%   \item
%     \cs{ProcessLocalKeyvalOptions} added.
%   \item
%     \cs{DisableKeyvalOption} with the \texttt{action=ignore} option
%     fixed (Joseph Wright).
%   \end{Version}
%   \begin{Version}{2009/07/21 v3.3}
%   \item
%     \cs{DeclareLocalOption}, \cs{DeclareLocalOptions} added.
%   \end{Version}
%   \begin{Version}{2009/08/13 v3.4}
%   \item
%     Documentation addition: recommendation for Joseph Wright's
%     review article.
%   \item
%     Documentation addition: local/global options.
%   \end{Version}
%   \begin{Version}{2009/12/04 v3.5}
%   \item
%     \cs{AddToKeyvalOption} added.
%   \end{Version}
%   \begin{Version}{2009/12/08 v3.6}
%   \item
%     Fix: If a default handler is configured, it is now also
%     called for classes.
%   \end{Version}
%   \begin{Version}{2010/02/22 v3.7}
%   \item
%     Missing space in error message added.
%   \end{Version}
%   \begin{Version}{2010/07/23 v3.8}
%   \item
%     Documenation for package \xpackage{kvoptions-patch} improved.
%     No code changes.
%   \end{Version}
%   \begin{Version}{2010/12/02 v3.9}
%   \item
%     Key \texttt{setkeys} added for \cs{SetupKeyvalOptions}.
%   \end{Version}
%   \begin{Version}{2010/12/23 v3.10}
%   \item
%     \cs{DeclareVoidOption} also parses the second parameter as
%     \hologo{TeX} argument to improve compatibility with
%     \cs{DeclareOption}.
%   \end{Version}
%   \begin{Version}{2011/06/30 v3.11}
%   \item
%     Fix because of design bug in package \xpackage{xkeyval} that
%     removes global options with equal signs.
%   \end{Version}
%   \begin{Version}{2016/05/16 v3.12}
%   \item
%     Documentation updates.
%   \end{Version}
% \end{History}
%
% \PrintIndex
%
% \Finale
\endinput
|
% \end{quote}
% Do not forget to quote the argument according to the demands
% of your shell.
%
% \paragraph{Generating the documentation.}
% You can use both the \xfile{.dtx} or the \xfile{.drv} to generate
% the documentation. The process can be configured by the
% configuration file \xfile{ltxdoc.cfg}. For instance, put this
% line into this file, if you want to have A4 as paper format:
% \begin{quote}
%   \verb|\PassOptionsToClass{a4paper}{article}|
% \end{quote}
% An example follows how to generate the
% documentation with pdf\LaTeX:
% \begin{quote}
%\begin{verbatim}
%pdflatex kvoptions.dtx
%makeindex -s gind.ist kvoptions.idx
%pdflatex kvoptions.dtx
%makeindex -s gind.ist kvoptions.idx
%pdflatex kvoptions.dtx
%\end{verbatim}
% \end{quote}
%
% \begin{thebibliography}{99}
% \bibitem{tb94wright}
%   A guide to key-value methods, Joseph Wright, second draft for
%   \href{https://www.tug.org/tugboat}{TUGBoat}, 2009-03-17.
%   \url{https://www.texdev.net/uploads/2009/03/keyval.pdf}
% \bibpackage{ifthen}{David Carlisle}{2001/05/26}
%   {\CTANpkg{ifthen}}
% \bibpackage{helvet}{Sebastian Rahtz, Walter Schmidt}{2004/01/26}
%   {\CTANpkg{psfonts}}
% \bibpackage{hyperref}{Sebastian Rahtz, Heiko Oberdiek}{2006/02/12}
%   {\CTANpkg{hyperref}}
% \bibpackage{keyval}{David Carlisle}{1999/03/16}
%   {\CTANpkg{keyval}}
% \bibpackage{multicol}{Frank Mittelbach}{2004/02/14}
%   {\CTANpkg{multicol}}
% \bibpackage{tabularx}{David Carlisle}{1999/01/07}
%   {\CTANpkg{tabularx}}
% \bibpackage{tracefnt}{Frank Mittelbach, Rainer Sch\"opf}{1997/05/29}
%   {\CTANpkg{latex-base}}
% \bibpackage{xkeyval}{Hendri Adriaens}{2005/05/07}
%   {\CTANpkg{xkeyval}}
% \bibitem{clsguide}
%   The \LaTeX3 Project, \textit{\LaTeXe\ for class and package writers},
%   2003/12/09.
%   \CTANpkg{clsguide}
%
% \end{thebibliography}
%
% \begin{History}
%   \begin{Version}{0000/00/00 v0.0}
%   \item
%     Probably David Carlisle's code in \xpackage{hyperref}
%     was the start.
%   \end{Version}
%   \begin{Version}{2004/02/22 v1.0}
%   \item
%     The first version was never published. It also
%     has offered a patch to get rid of \LaTeX's option
%     expansion.
%   \end{Version}
%   \begin{Version}{2006/02/16 v2.0}
%   \item
%     Now the package is redesigned with an easier
%     user interface.
%   \item
%     \cs{ProcessKeyvalOptions} remains the central service, inherited
%     from \xpackage{hyperref}'s \cs{ProcessOptionsWithKV}.
%     Now the use inside classes is also supported.
%   \item
%     Provides help macros for boolean and simple string options.
%   \item
%     Fixes for the patch of \LaTeX{}. The patch is only enabled,
%     if the user requests it.
%   \end{Version}
%   \begin{Version}{2006/02/20 v2.1}
%   \item
%     Unused option list is sanitized to prevent problems
%     with other packages that uses own processing methods
%     for key value options. Disadvantage: the unused global
%     option detection is weakened.
%   \item
%     New option type by \cs{DeclareVoidOption} for options without
%     value.
%   \item
%     Default rule by \cs{DeclareDefaultOption}.
%   \item
%     Dynamic options: \cs{DisableKeyvalOption}.
%   \end{Version}
%   \begin{Version}{2006/06/01 v2.2}
%   \item
%     Fixes for option \xoption{patch}.
%   \end{Version}
%   \begin{Version}{2006/08/17 v2.3}
%   \item
%     \cs{DeclareBooleanOption} renamed to \cs{DeclareBoolOption}
%     to avoid a name clash with package \cs{ifoption}.
%   \end{Version}
%   \begin{Version}{2006/08/22 v2.4}
%   \item
%     Option \xoption{patch}: \cs{ExecuteOptions} does not
%     change the meaning of macro \cs{CurrentOption} at all.
%   \end{Version}
%   \begin{Version}{2007/04/11 v2.5}
%   \item
%     Line ends sanitized.
%   \end{Version}
%   \begin{Version}{2007/05/06 v2.6}
%   \item
%     Uses package \xpackage{etexcmds}.
%   \end{Version}
%   \begin{Version}{2007/06/11 v2.7}
%   \item
%     The patch part fixes LaTeX bug latex/3965.
%   \end{Version}
%   \begin{Version}{2007/10/02 v2.8}
%   \item
%     Compatibility for \plainTeX\ added.
%   \item
%     Typos in documentation fixed (Axel Sommerfeldt).
%   \end{Version}
%   \begin{Version}{2007/10/11 v2.9}
%   \item
%     Bug fix for option \xoption{patch}.
%   \end{Version}
%   \begin{Version}{2007/10/18 v3.0}
%   \item
%     New package \xpackage{kvoptions-patch}.
%   \end{Version}
%   \begin{Version}{2009/04/10 v3.1}
%   \item
%     Space by line end removed in definition of internal macro.
%   \end{Version}
%   \begin{Version}{2009/07/17 v3.2}
%   \item
%     \cs{ProcessLocalKeyvalOptions} added.
%   \item
%     \cs{DisableKeyvalOption} with the \texttt{action=ignore} option
%     fixed (Joseph Wright).
%   \end{Version}
%   \begin{Version}{2009/07/21 v3.3}
%   \item
%     \cs{DeclareLocalOption}, \cs{DeclareLocalOptions} added.
%   \end{Version}
%   \begin{Version}{2009/08/13 v3.4}
%   \item
%     Documentation addition: recommendation for Joseph Wright's
%     review article.
%   \item
%     Documentation addition: local/global options.
%   \end{Version}
%   \begin{Version}{2009/12/04 v3.5}
%   \item
%     \cs{AddToKeyvalOption} added.
%   \end{Version}
%   \begin{Version}{2009/12/08 v3.6}
%   \item
%     Fix: If a default handler is configured, it is now also
%     called for classes.
%   \end{Version}
%   \begin{Version}{2010/02/22 v3.7}
%   \item
%     Missing space in error message added.
%   \end{Version}
%   \begin{Version}{2010/07/23 v3.8}
%   \item
%     Documenation for package \xpackage{kvoptions-patch} improved.
%     No code changes.
%   \end{Version}
%   \begin{Version}{2010/12/02 v3.9}
%   \item
%     Key \texttt{setkeys} added for \cs{SetupKeyvalOptions}.
%   \end{Version}
%   \begin{Version}{2010/12/23 v3.10}
%   \item
%     \cs{DeclareVoidOption} also parses the second parameter as
%     \hologo{TeX} argument to improve compatibility with
%     \cs{DeclareOption}.
%   \end{Version}
%   \begin{Version}{2011/06/30 v3.11}
%   \item
%     Fix because of design bug in package \xpackage{xkeyval} that
%     removes global options with equal signs.
%   \end{Version}
%   \begin{Version}{2016/05/16 v3.12}
%   \item
%     Documentation updates.
%   \end{Version}
% \end{History}
%
% \PrintIndex
%
% \Finale
\endinput

%        (quote the arguments according to the demands of your shell)
%
% Documentation:
%    (a) If kvoptions.drv is present:
%           latex kvoptions.drv
%    (b) Without kvoptions.drv:
%           latex kvoptions.dtx; ...
%    The class ltxdoc loads the configuration file ltxdoc.cfg
%    if available. Here you can specify further options, e.g.
%    use A4 as paper format:
%       \PassOptionsToClass{a4paper}{article}
%
%    Programm calls to get the documentation (example):
%       pdflatex kvoptions.dtx
%       makeindex -s gind.ist kvoptions.idx
%       pdflatex kvoptions.dtx
%       makeindex -s gind.ist kvoptions.idx
%       pdflatex kvoptions.dtx
%
% Installation:
%    TDS:tex/latex/oberdiek/kvoptions.sty
%    TDS:tex/latex/oberdiek/kvoptions-patch.sty
%    TDS:doc/latex/oberdiek/kvoptions.pdf
%    TDS:doc/latex/oberdiek/example-mycolorsetup.sty
%    TDS:doc/latex/oberdiek/test/kvoptions-test1.tex
%    TDS:doc/latex/oberdiek/test/kvoptions-test2.tex
%    TDS:doc/latex/oberdiek/test/kvoptions-test3.tex
%    TDS:doc/latex/oberdiek/test/kvoptions-test4.tex
%    TDS:doc/latex/oberdiek/test/kvoptions-test4.sty
%    TDS:source/latex/oberdiek/kvoptions.dtx
%
%<*ignore>
\begingroup
  \catcode123=1 %
  \catcode125=2 %
  \def\x{LaTeX2e}%
\expandafter\endgroup
\ifcase 0\ifx\install y1\fi\expandafter
         \ifx\csname processbatchFile\endcsname\relax\else1\fi
         \ifx\fmtname\x\else 1\fi\relax
\else\csname fi\endcsname
%</ignore>
%<*install>
\input docstrip.tex
\Msg{************************************************************************}
\Msg{* Installation}
\Msg{* Package: kvoptions 2016/05/16 v3.12 Key value format for package options (HO)}
\Msg{************************************************************************}

\keepsilent
\askforoverwritefalse

\let\MetaPrefix\relax
\preamble

This is a generated file.

Project: kvoptions
Version: 2016/05/16 v3.12

Copyright (C) 2004, 2006, 2007, 2009-2011 by
   Heiko Oberdiek <heiko.oberdiek at googlemail.com>

This work may be distributed and/or modified under the
conditions of the LaTeX Project Public License, either
version 1.3c of this license or (at your option) any later
version. This version of this license is in
   https://www.latex-project.org/lppl/lppl-1-3c.txt
and the latest version of this license is in
   https://www.latex-project.org/lppl.txt
and version 1.3 or later is part of all distributions of
LaTeX version 2005/12/01 or later.

This work has the LPPL maintenance status "maintained".

The Current Maintainers of this work are
Heiko Oberdiek and the Oberdiek Package Support Group
https://github.com/ho-tex/oberdiek/issues


This work consists of the main source file kvoptions.dtx
and the derived files
   kvoptions.sty, kvoptions.pdf, kvoptions.ins, kvoptions.drv,
   kvoptions-patch.sty, example-mycolorsetup.sty,
   kvoptions-test1.tex, kvoptions-test2.tex,
   kvoptions-test3.tex, kvoptions-test4.tex,
   kvoptions-test4.sty.

\endpreamble
\let\MetaPrefix\DoubleperCent

\generate{%
  \file{kvoptions.ins}{\from{kvoptions.dtx}{install}}%
  \file{kvoptions.drv}{\from{kvoptions.dtx}{driver}}%
  \usedir{tex/latex/oberdiek}%
  \file{kvoptions.sty}{\from{kvoptions.dtx}{package}}%
  \file{kvoptions-patch.sty}{\from{kvoptions.dtx}{patch}}%
  \usedir{doc/latex/oberdiek}%
  \file{example-mycolorsetup.sty}{\from{kvoptions.dtx}{example}}%
%  \usedir{doc/latex/oberdiek/test}%
%  \file{kvoptions-test1.tex}{\from{kvoptions.dtx}{test1}}%
%  \file{kvoptions-test2.tex}{\from{kvoptions.dtx}{test2}}%
%  \file{kvoptions-test3.tex}{\from{kvoptions.dtx}{test3}}%
%  \file{kvoptions-test4.tex}{\from{kvoptions.dtx}{test4}}%
%  \file{kvoptions-test4.sty}{\from{kvoptions.dtx}{test4pkg}}%
  \nopreamble
  \nopostamble
%  \usedir{source/latex/oberdiek/catalogue}%
%  \file{kvoptions.xml}{\from{kvoptions.dtx}{catalogue}}%
}

\catcode32=13\relax% active space
\let =\space%
\Msg{************************************************************************}
\Msg{*}
\Msg{* To finish the installation you have to move the following}
\Msg{* files into a directory searched by TeX:}
\Msg{*}
\Msg{*     kvoptions.sty, kvoptions-patch.sty}
\Msg{*}
\Msg{* To produce the documentation run the file `kvoptions.drv'}
\Msg{* through LaTeX.}
\Msg{*}
\Msg{* Happy TeXing!}
\Msg{*}
\Msg{************************************************************************}

\endbatchfile
%</install>
%<*ignore>
\fi
%</ignore>
%<*driver>
\NeedsTeXFormat{LaTeX2e}
\ProvidesFile{kvoptions.drv}%
  [2016/05/16 v3.12 Key value format for package options (HO)]%
\documentclass{ltxdoc}
\usepackage{holtxdoc}[2011/11/22]
\begin{document}
  \DocInput{kvoptions.dtx}%
\end{document}
%</driver>
% \fi
%
%
% \CharacterTable
%  {Upper-case    \A\B\C\D\E\F\G\H\I\J\K\L\M\N\O\P\Q\R\S\T\U\V\W\X\Y\Z
%   Lower-case    \a\b\c\d\e\f\g\h\i\j\k\l\m\n\o\p\q\r\s\t\u\v\w\x\y\z
%   Digits        \0\1\2\3\4\5\6\7\8\9
%   Exclamation   \!     Double quote  \"     Hash (number) \#
%   Dollar        \$     Percent       \%     Ampersand     \&
%   Acute accent  \'     Left paren    \(     Right paren   \)
%   Asterisk      \*     Plus          \+     Comma         \,
%   Minus         \-     Point         \.     Solidus       \/
%   Colon         \:     Semicolon     \;     Less than     \<
%   Equals        \=     Greater than  \>     Question mark \?
%   Commercial at \@     Left bracket  \[     Backslash     \\
%   Right bracket \]     Circumflex    \^     Underscore    \_
%   Grave accent  \`     Left brace    \{     Vertical bar  \|
%   Right brace   \}     Tilde         \~}
%
% \GetFileInfo{kvoptions.drv}
%
% \title{The \xpackage{kvoptions} package}
% \date{2016/05/16 v3.12}
% \author{Heiko Oberdiek\thanks
% {Please report any issues at \url{https://github.com/ho-tex/oberdiek/issues}}}
%
% \maketitle
%
% \begin{abstract}
% This package is intended for package authors who want to
% use options in key value format for their package options.
% \end{abstract}
%
% \tableofcontents
%
% \def\M#1{\texttt{\{}\meta{#1}\texttt{\}}}
% \def\O#1{\texttt{[}\meta{#1}\texttt{]}}
%
% \section{Introduction}
%
% First I want to recommend the very good review article
% ``A guide to key-value methods'' by Joseph Wright \cite{tb94wright}.
% It introduces the different key-value packages and compares them.
%
% \subsection{The beginning}
%
% This package \xpackage{kvoptions} addresses class or package writers
% that want to allow their users to specify options
% as key value pairs, e.g.:
% \begin{quote}
% |\documentclass[verbose=false,name=me]{myclass}|\\
% |\usepackage[format=print]{mylayout}|
% \end{quote}
% Prominent example is package \xpackage{hyperref},
% probably the first package that offers this service.
% It's \cs{ProcessOptionsWithKV} is often copied und
% used in other packages, e.g. package \xpackage{helvet}
% that uses this interface for its option \xoption{scaled}.
%
% However copying code is not the most modern software
% development technique. And \xpackage{hyperref}'s
% code for \cs{ProcessOptionsWithKV} was changed
% to fix bugs. The version used in other packages
% depends on the time of copying and the awareness
% of \xpackage{hyperref}'s changes.
% Now the code is sourced out into this package and
% available for other package or class writers.
%
% \subsection{Overview}
%
% Package \xpackage{kvoptions} connects package
% \xpackage{\emph{k}ey\emph{v}al}
% with \LaTeX's package and class \emph{options}:
% \begin{center}
% \def\T#1{\bfseries #1}
% \renewcommand{\arraystretch}{1.5}
% \begin{tabular}{l|l|l}
%   \T{Package \xpackage{keyval}}
%   & \T{Package \xpackage{kvoptions}}
%   & \T{\LaTeX\ kernel}
%   \\
% \hline
%     \cs{define@key}
%   &
%     \renewcommand*{\arraystretch}{1}^^A
%     \begin{tabular}[t]{@{}l@{}}
%       \cs{DeclareVoidOption}\\
%       \cs{DeclareStringOption}\\
%       \cs{DeclareBoolOption}\\
%       \cs{DeclareComplementaryOption}\\
%       \cs{DisableKeyvalOption}
%     \end{tabular}
%   & \cs{DeclareOption}
%   \\
% \hline
%   & \cs{DeclareDefaultOption}
%   & \cs{DeclareOption*}
%   \\
% \hline
%   & \cs{ProcessKeyvalOptions}
%   & \cs{ProcessOptions*}
%   \\
% \hline
%   & Option \xoption{patch}
%   &
%     \renewcommand*{\arraystretch}{1}^^A
%     \begin{tabular}[t]{@{}l@{}}
%       Class/package\\
%       option system
%     \end{tabular}
%   \\
% \hline
%   & \cs{SetupKeyvalOptions} &
%   \\
% \end{tabular}
% \end{center}
%
% \hypersetup{bookmarksopenlevel=3}
% \section{Usage}
%
% \subsection{Process options}
%
% \subsubsection{\cs{ProcessKeyvalOptions}}
%
% \begin{declcs}{ProcessKeyvalOptions}\,\M{family}\\
%   \cs{ProcessKeyvalOptions} \*
% \end{declcs}
% This command evaluates the global or local options of
% the package that are defined with \xpackage{keyval}'s
% interface within the family \meta{family}. It acts the
% same way as \LaTeX's \cs{ProcessOptions*}. In a package
% unknown global options are ignored, in a class they
% are added to the unknown option list. The known global
% options and all local options are passed to \xpackage{keyval}'s
% \cs{setkeys} command for executing the options.
% Unknown options are reported to the user by an error.
%
% If the family name happens to be the same as the name
% of the package or class where \cs{ProcessKeyvalOptions}
% is used or the family name has previously been setup by
% \cs{SetupKeyvalOptions}, then \cs{ProcessKeyvalOptions}
% knows the family name already and you can use the star form
% without mandatory argument.
%
% \subsubsection{\cs{ProcessLocalKeyvalOptions}}
%
% \begin{declcs}{ProcessLocalKeyvalOptions}\,\M{family}\\
%   \cs{ProcessLocalKeyvalOptions} \*
% \end{declcs}
% This macro has the same syntax and works similar as
% \cs{ProcessKeyvalOptions}. However it ignores
% global options and only processes the local package options.
% Therefore it only can be used inside a package.
% An error is thrown, if it is used inside a class.
%
% Neither of the following macros are necessary for
% \cs{ProcessKeyvalOptions{}}. They just help the package/class
% author in common tasks.
%
% \subsubsection{\cs{SetupKeyvalOptions}}\label{setup}
%
% \begin{declcs}{SetupKeyvalOptions}\,\texttt{\{}\\
%   \hspace{1.5em}|family|\,|=|\,\meta{family}|,|\\
%   \hspace{1.5em}|prefix|\,|=|\,\meta{prefix}\\
%   \hspace{1.5em}|setkeys|\,|=|\,\meta{setkeys command}\\
%   \texttt{\}}
% \end{declcs}
% This command allows to configure the default assumptions
% that are based on the current package or class name.
% \LaTeX\ remembers this name in \cs{@currname}. The syntax
% description of the default looks a little weird, therefor
% an example is given for a package or class named |foobar|.
% \begin{quote}
% \begin{tabular}{@{}l|ll|l@{}}
%   Key & Default & (example) & Used by\\
% \hline
%   \rule{0pt}{1.2\ht\csname @arstrutbox\endcsname}^^A
%   |family| & \meta{\cs{@currname}} & (|foobar|)
%       & \cs{ProcessKeyvalOptions*} \\
%     &&& \cs{DeclareBoolOption} \\
%     &&& \cs{DeclareStringOption} \\
% \hline
%   \rule{0pt}{1.2\ht\csname @arstrutbox\endcsname}^^A
%   |prefix| & \meta{\cs{@currname}}|@| & (|foobar@|)
%       & \cs{DeclareBoolOption} \\
%     &&& \cs{DeclareStringOption} \\
%     &&& \cs{DeclareVoidOption} \\
% \hline
%   \rule{0pt}{1.2\ht\csname @arstrutbox\endcsname}^^A
%   |setkeys| & \cs{setkeys} & (\cs{kvsetkeys})
%       & \cs{ProcessKeyvalOptions} \\
%     &&& \cs{ProcessLocalKeyvalOptions}
% \end{tabular}
% \end{quote}
% Key |setkeys| was added in version 3.9. The original
% \cs{setkeys} of package \xpackage{keyval} is not reentrant.
% If an option is processed by this \cs{setkeys}, then
% the option should not call \cs{setkeys} again with a
% different family. Otherwise the next options of the
% first \cs{setkeys} call are processed with the wrong
% family. With key |setkeys| the macro \cs{kvsetkeys}
% can be set that does not have the problem of the
% original \cs{setkeys} of package {keyval}.
%
% Probably \cs{setkeys} of package \xpackage{xkeyval}
% is safe in this respect. But I haven't made a full
% analysis. At least it does not have the problem
% of the original \cs{setkeys}.
%
% \subsection{Option declarations}
%
% The options for \cs{ProcessKeyvalOptions} are defined
% by \xpackage{keyval}'s \cs{define@key}.
% Common purposes of such keys are boolean switches, they
% enable or disable something. Or they store a name or some kind of
% string in a macro. The following commands help the user. He
% declares what he wants and \xpackage{kvoptions} take care of
% the key definition, resource allocation and initialization.
%
% In order to avoid name clashes of macro names, internal
% commands are prefixed. Both the prefix and the family
% name for the defined keys can be configured by
% \cs{SetupKeyvalOptions}.
%
% \subsubsection{\cs{DeclareStringOption}}
%
% \begin{declcs}{DeclareStringOption}\,\O{init}\,\M{key}\,\O{default}
% \end{declcs}
% A macro is created that remembers the value of the key \meta{key}.
% The name of the macro consists of the option name \meta{key} that
% is prefixed by the prefix (see \ref{setup}).
% The initial contents of the
% macro can be given by the first optional argument \meta{init}.
% The default is empty.
%
% The the option \meta{key} is defined. The option code
% just stores its value in the macro. If the optional argument
% at the end of \cs{DeclareStringOption} is given, then option
% \meta{key} is defined with the default \meta{default}.
%
% Example for a package with the following two lines:
% \begin{quote}
% |\ProvidesPackage{foobar}|\\
% |\DeclareStringOption[me]{name}|
% \end{quote}
% Then \cs{DeclareStringOption} defines the macro with content |me|,
% note \LaTeX\ complains if the name of the macro already exists:
% \begin{quote}
% |\newcommand*{\foobar@name}{me}|
% \end{quote}
% The option definition is similar to:
% \begin{quote}
% |\define@key{foobar}{name}{%|\\
% |  \renewcommand*{\foobar@name}{#1}%|\\
% |}|
% \end{quote}
%
% \subsubsection{\cs{DeclareBoolOption}}
%
% \begin{declcs}{DeclareBoolOption}\,\O{init}\,\M{key}
% \end{declcs}
% A boolean switch is generated, initialized by value \meta{init}
% and the corresponding key \meta{key} is defined. If the
% initialization value is not given, |false| is used as default.
%
% The internal actions of \cs{DeclareBoolOption} are shown below.
% The example is given for a package author who has the following
% two lines in his package/class:
% \begin{quote}
% |\ProvidesPackage{foobar}|\\
% |\DeclareBoolOption{verbose}|
% \end{quote}
% First a new switch is created:
% \begin{quote}
%   |\newif\iffoobar@verbose|
% \end{quote}
% and initialized:
% \begin{quote}
%   |\foobar@verbosefalse|
% \end{quote}
% Finally the key is defined:
% \begin{quote}
%   |\define@key{foobar}{verbose}[true]{|\dots|}|
% \end{quote}
% The option code configures the boolean option in the following way:
% If the author specifies |true| or |false| then the switch is
% turned on or off respectivly. Also the option can be given without
% explicit value. Then the switch is enabled.
% Other values are reported as errors.
%
% Now the switch is ready to use in the package/class, e.g.:
% \begin{quote}
%   |\iffoobar@verbose|\\
%   \qquad\textit{\% print verbose message}\\
%   |\else|\\
%   \qquad\textit{\% be quiet}\\
%   |\fi|
% \end{quote}
% Users of package \cs{ifthen} can use the switch as
% boolean:
% \begin{quote}
%   |\boolean{foobar@verbose}|
% \end{quote}
%
% \subsubsection{\cs{DeclareComplementaryOption}}
%
% \begin{declcs}{DeclareComplementaryOption}\,\M{key}\,\M{parent}
% \end{declcs}
% Sometimes contrasting names are used to characterize the
% two states of a boolean switch, for example \xoption{draft}
% vs.\@ \xoption{final}. Both options behave like boolean
% options but they do not need two different switches, they
% should share one. \cs{DeclareComplementaryOption} allows this.
% The option \meta{key} shares the switch of option \meta{parent}.
% Example:
% \begin{quote}
%   |\DeclareBoolOption{draft}|\\
%   |\DeclareComplementaryOption{final}{draft}|
% \end{quote}
% Then |final| sets the switch of \xoption{draft} to |false|, and
% |final=false| enables the \xoption{draft} switch.
%
% \subsubsection{\cs{DeclareVoidOption}}
%
% \begin{declcs}{DeclareVoidOption}\,\M{key}\,\M{code}
% \end{declcs}
% \cs{ProcessKeyvalOptions} can be extended to recognize
% options that are declared in traditional way by \cs{DeclareOption}.
% But in case of the error that the user specifies a value, then
% this option would not recognized as key value option because
% of \cs{DeclareOption} and not detected as traditional option
% because of the value part. The user would get an unknown option
% error, difficult to understand.
%
% \cs{DeclareVoidOption} solves
% this problem. It defines the option \meta{key} as key value option.
% If the user specifies a value, a warning is given and the
% value is ignored.
%
% The code part \meta{code} is stored in a macro.
% The name of the macro consists of the option name \meta{key} that
% is prefixed by the prefix (see \ref{setup}).
% If the option is set, the macro will be executed. During the
% execution \cs{CurrentOption} is available with the current key name.
%
% \subsubsection{\cs{DeclareDefaultOption}}
%
% \begin{declcs}{DeclareDefaultOption}\,\M{code}
% \end{declcs}
% \begin{sloppypar}
% This command does not define a specific key, it is the equivalent
% to \LaTeX's \cs{DeclareOption*}. It allows the specification
% of a default action \meta{code} that is invoked if an unknown option
% is found.
% While \meta{code} is called,
% macro \cs{CurrentOption} contains the current option string.
% In addition \cs{CurrentOptionValue} contains the value part
% if the option string is parsable as key value pair, otherwise
% it is \cs{relax}. \cs{CurrentOptionKey} contains the key of
% the key value pair, or the whole option string, if it misses
% the equal sign.
% \end{sloppypar}
%
% Inside packages typical default actions are to pass unknown
% options to another package. Or an error message can be thrown
% by \cs{@unknownoptionerror}. This is the original error
% message that \LaTeX\ gives for unkown package options.
% This error message is easier to understand for the user
% as the error message from package \xpackage{keyval} that
% is given otherwise.
%
% A Class ignores unknown options and puts them on the
% unused option list. Let \LaTeX\ do the job and just
% call \cs{OptionNotUsed}. Or the options can be passed to
% another class that is later loaded.
%
% \subsubsection{Local options}
%
% \begin{declcs}{DeclareLocalOption} \M{option}\\
%   \cs{DeclareLocalOptions} \M{option list}
% \end{declcs}
% Both macros mark package options as local options. That means
% that they are ignored by \cs{ProcessKeyvalOptions} if they are given
% as global options. \cs{DeclareLocalOptions} takes one option,
% \cs{DeclareLocalOptions} expects a comma separated list of options.
%
% \subsubsection{Dynamic options}
%
% Options of \LaTeX's package/class system are cleared
% in \cs{ProcessOptions}. They modify the static model
% of a package. For example, depending on option \xoption{bookmarks}
% package \xpackage{hyperref} loads differently.
%
% Options, however,
% defined by \xpackage{keyval}'s \cs{define@key} remain
% defined, if the options are processed by \cs{setkeys}.
% Therefore these options can also be used to model
% the dynamic behaviour of a package. For example, in
% \xpackage{hyperref} the link colors can be changed
% everywhere until the end in |\end{document}|.
%
% However package \xpackage{color} that adds color support
% is necessary and it
% cannot be loaded after |\begin{document}|. Option \xoption{colorlinks}
% that loads \xpackage{color} should be active until |\begin{document}|
% and die in some way if it is too late for loading packages.
% With \cs{DisableKeyvalOption} the package/class author can
% specify and configure the death of an option and controls
% the life period of the option.
%
% \subsubsection{\cs{DisableKeyvalOption}}
%
% \begin{declcs}{DisableKeyvalOption}\,\O{options}\,\M{family}\,\M{key}
%   \\[0.5ex]
%   \meta{options}:\\
%   \hspace{1.5em}^^A
%   \begin{tabular}[t]{@{}l@{ \texttt{=} }l@{\qquad}>{default: }l@{}}
%     |action| & |undef|, |warning|, |error|, or |ignore|
%     & |undef|
%     \\
%     \multicolumn{1}{@{}l}{|global| or |local|} & & |global|
%     \\
%     |package| or |class| & \meta{name}
%     \\
%   \end{tabular}^^A
% \end{declcs}
% \cs{DisableKeyvalOption} can be called to mark the end when
% the option \meta{key} is no longer useful. The behaviour of
% an option after its death can be configured by \xoption{action}:
% \begin{description}
% \item[\texttt{undef}:] The option will be undefined,
%   If it is called, \cs{setkeys} reports an error because
%   of unknown key.
% \item[\texttt{error} or \texttt{warning}:] Use of the
%  option will cause an error or warning message.
%  Also these actions require that exclusivly either the
%  package or class name is given in options \xoption{package}
%  or \xoption{class}.
% \item[\texttt{ignore}:] The use of the option will silently
%  be ignored.
% \end{description}
% The option's death can be limited to the end of the current
% group, if option \xoption{local} is given. Default is
% \xoption{global}.
%
% The package/class author can wish the end of the option
% already during the package loading, then he will have
% static behaviour. In case of dynamic options \cs{DisableKeyvalOption}
% can be executed everywhere, also outside the package. Therefore
% the family name and the package/class name is usually unknown for
% \cs{DisableKeyvalOption}. Therefore the argument for the family
% name is mandatory and for some actions the package/class name
% must be provided.
%
% Usually a macro would configure the option death, Example:
%\begin{quote}
%\begin{verbatim}
%\ProvidesPackage{foobar}
%\DeclareBoolOption{color}
%\DeclareStringOption[red]{emphcolor}
%\ProcessKeyvalOptions*
%
%\newcommand*{\foobar@DisableOption}[2]{%
%  \DisableKeyvalueOption[
%    action={#1},
%    package=foobar
%  ]{foobar}{#2}%
%}
%
%\iffoobar@color
%  \RequirePackage{color}
%  \renewcommand*{\emph}[1]{\textcolor{\foobar@emphcolor}{#1}}
%\else
%  % Option emphcolor is not wrong, if we have color support.
%  % otherwise the option has no effect, but we don't want to
%  % remove it. Therefore action 'ignore' is the best choice:
%  \foobar@DisableOption{ignore}{emphcolor}
%\fi
%% No we don't need the option 'color'.
%\foobar@DisableOption{warning}{color}
%
%% With color support option 'emphcolor' will dynamically
%% change the color of \emph statements.
%\end{verbatim}
%\end{quote}
%
% \subsubsection{\cs{AddToKeyvalOption}}
%
% \begin{declcs}{AddToKeyvalOption}\,\M{family}\,\M{key}\,\M{code}\\
%   \cs{AddToKeyvalOption} \*\,\M{key}\,\M{code}
% \end{declcs}
% The code for an existing key \meta{key} of family \meta{family} is
% extended by code \meta{code}. In the starred form the current
% family setting is used, see \cs{ProcessKeyvalOptions*}.
%
% \subsection{Global vs. local options}
%
% Options that are given for \cs{documentclass} are called
% global options. They are known to the class and all packages.
% A package may make use of a global option and marks it as
% used. The advantage for the user is the freedom to specify
% options both in the \cs{documentclass} or \cs{usepackage}
% commands.
%
% However global options are shared with the class options
% and options of all other packages. Thus there can be the
% same option with different semantics for different
% packages and classes. As example, package \xpackage{bookmark}
% knows option \xoption{open} that specifies whether the bookmarks
% are opened or closed initially. It's values are |true| or |false|.
% Since KOMA-Script version 3.00 the KOMA classes also introduces
% option \xoption{open} with values |right| and |any| and a
% complete different meaning.
%
% Such conflicts can be resolved by marking all or part of options
% as local by \cs{DeclareLocalOption} or \cs{DeclareLocalOptions}.
% Then the packages ignores global occurences of these options.
% Package \xpackage{kvoptions} provides two methods:
% \begin{itemize}
% \item \cs{ProcessLocalKeyvalOptions} automatically uses all options
%   as local options. It ignores all global options.
% \item \cs{DeclareLocalOption} or \cs{DeclareLocalOptions} marks
%   options as local options. \cs{ProcessKeyvalOptions} will then
%   ignore global occurences for these local options.
% \end{itemize}
% Since version 1.5 package \xpackage{bookmark} uses the latter
% method. It checks global and local option places for driver options
% and limits all other options as local options. Thus the class
% option \xoption{open} of KOMA-Script is not misread as
% option for package \xpackage{bookmark}.
%
% \subsection{Summary of internal macros}
%
% The \cs{Declare}\texttt{\dots Option} commands define
% macros, additionally to the macros generated by the key definition.
% These macros can be used by the package/class author.
% The name of the macros starts with the prefix \meta{prefix}
% that can be configured by \cs{SetupKeyvalOptions}.
% \begin{center}
% \def\prefix{\cs{}\meta{prefix}\meta{key}}^^A
% \def\Hline{^^A
%   \hline
%   \rule{0pt}{1.2\ht\csname @arstrutbox\endcsname}^^A
% }^^A
% \begin{tabular}{@{}l|>{\ttfamily}l|l@{}}
%   Declare \meta{key} & \textrm{Defined macro} & Description\\
% \Hline
%   \cs{DeclareStringOption}
%    & \prefix & holds the string\\
% \Hline
%   \cs{DeclareBoolOption}
%     & \cs{if}\meta{prefix}\meta{key}& boolean switch\\
%   &\prefix false& disable switch\\
%   &\prefix true& enable switch\\
% \Hline
%   \cs{DeclareComplementaryOption}
%     & \prefix false& enable parent switch\\
%     & \prefix true& disable parent switch\\
% \Hline
%   \cs{DeclareVoidOption}
%   & \prefix & holds the action
% \end{tabular}
% \end{center}
%
% \subsection{\plainTeX}
%
% Package \xpackage{keyval} is also usable in \plainTeX\ with
% the help of file \xfile{miniltx.tex}. Some features of this
% package \xpackage{kvoptions} might also be useful for \plainTeX.
% If \LaTeX\ is not found, \cs{ProcessKeyvalOptions} and option
% \xoption{patch} are disabled. Before using the option declaration
% commands \cs{Declare...Option}, \cs{SetupKeyvalOptions} must be
% used.
%
% \hypersetup{bookmarksopenlevel=1}
%
% \section{Example}
%
% The following example defined a package that serves some
% private color management. A boolean option \xoption{print} enables
% print mode without colors. An option \xoption{emph} redefines
% \cs{emph} to print in the given color. And the driver
% can be specified by option \xoption{driver}.
%
%    \begin{macrocode}
%<*example>
    % Package identification
    % -----------------------
\NeedsTeXFormat{LaTeX2e}
\ProvidesPackage{example-mycolorsetup}[2016/05/16 Managing my colors]

\RequirePackage{ifpdf}
\RequirePackage{kvoptions}

    % Option declarations
    % -------------------

\SetupKeyvalOptions{
  family=MCS,
  prefix=MCS@
}
    % Use a shorter family name and prefix

    % Option print
\DeclareBoolOption{print}
    % is the same as
    % \DeclareBoolOption[false]{print}

    % Option driver
\ifpdf
  \DeclareStringOption[pdftex]{driver}
\else
  \DeclareStringOption[dvips]{driver}
\fi

    % Alternative interface for driver options
\DeclareVoidOption{dvips}{\SetupDriver}
\DeclareVoidOption{dvipdfm}{\SetupDriver}
\DeclareVoidOption{pdftex}{\SetupDriver}
    % In \SetupDriver we take the current option \CurrentOption
    % and pass it to the driver option.
    % The \expandafter commands expand \CurrentOption at the
    % time, when \SetupDriver is executed and \CurrentOption
    % has the correct meaning.
\newcommand*{\SetupDriver}{%
  \expandafter\@SetupDriver\expandafter{\CurrentOption}%
}
\newcommand*{\@SetupDriver}[1]{%
  \setkeys{MCS}{driver={#1}}%
}

    % Option emph
    % An empty value means, we want to have no color for \emph.
    % If the user specifies option emph without value, the red is used.
\DeclareStringOption{emph}[red]
    % is the same as
    % \DeclareStringOption[]{emph}[red]

    % Default option rule
\DeclareDefaultOption{%
  \ifx\CurrentOptionValue\relax
    \PackageWarningNoLine{\@currname}{%
      Unknown option `\CurrentOption'\MessageBreak
      is passed to package `color'%
    }%
    % Pass the option to package color.
    % Again it is better to expand \CurrentOption.
    \expandafter\PassOptionsToPackage
    \expandafter{\CurrentOption}{color}%
  \else
    % Package color does not take options with values.
    % We provide the standard LaTeX error.
    \@unknownoptionerror
  \fi
}

    % Process options
    % ---------------
\ProcessKeyvalOptions*

    % Implementation depending on option values
    % -----------------------------------------
    % Code for print mode
\ifMCS@print
  \PassOptionsToPackage{monochrome}{color}
    % tells package color to use black and white
\fi

\RequirePackage[\MCS@driver]{color}
    % load package color with the correct driver

    % \emph setup
\ifx\MCS@emph\@empty
    % \@empty is a predefined macro with empty contents.
    % the option value of option emph is empty, thus
    % we do not want a redefinition of \emph.
\else
  \renewcommand*{\emph}[1]{%
    \textcolor{\MCS@emph}{#1}%
  }
\fi
%</example>
%    \end{macrocode}
%
% \section{Package options}
%
% The package \xpackage{kvoptions} knows two package options
% \xoption{patch} and \xoption{debugshow}.
% The options of package \xpackage{kvoptions} are intended
% for authors, not for package/class writers. Inside a package
% it is too late for option \xoption{patch} and \xoption{debugshow}
% enables some messages that are perhaps useful for the debugging
% phase. Also \LaTeX\ is unhappy if a package is loaded
% later again with options that are previously not given.
% Thus package and class authors, stay with |\RequirePackage{kvoptions}|
% without options.
%
% Option \xoption{patch} loads package \xpackage{kvoptions-patch}.
%
% \subsection{Package \xpackage{kvoptions-patch}}
%
% \LaTeX's system of package/class options has some severe limitations
% that especially affects the value part if options are used as
% pair of key and value.
% \begin{itemize}
% \item Spaces are removed, regardless where:
%   \begin{quote}
%     |\documentclass[box=0 0 400 600]{article}|
%   \end{quote}
%   Now each package will see |box=00400600| as global option.
% \item In the previous case also braces would not help:
%   \begin{quote}
%     |\documentclass[box={0 0 400 600}]{article}|
%   \end{quote}
%   The result is an error message:
%   \begin{quote}
%     |! LaTeX Error: Missing \begin{document}.|
%   \end{quote}
%   As local option, however, it works if the package
%   knows about key value options (By using this package, for example).
% \item The requirements on robustness are extremly high.
%   \LaTeX\ expands the option. All that will not work as environment
%   name will break also as option. Even a \cs{relax} will generate
%   an error message:
%   \begin{quote}
%     |! Missing \endcsname inserted.|
%   \end{quote}
%   Of course, \LaTeX\ does not use its protecting mechanisms.
%   On contrary \cs{protect} itself will cause errors.
% \item The options are expanded. But perhaps the package will
%   do that, because it has to setup some things before?
%   Example \xpackage{hyperref}:
%   \begin{quote}
%     |\usepackage[pdfauthor=M\"uller]{hyperref}|
%   \end{quote}
%   Package \xpackage{hyperref} does not see |M\"uller| but
%   its expansion and it does not like it, you get many warnings
%   \begin{quote}
%     |Token not allowed in a PDFDocEncoded string|
%   \end{quote}
%   And the title becomes: |Mu127uller|.
%   Therefore such options must usually be given after package
%   \xpackage{hyperref}
%   is loaded:
%   \begin{quote}
%     |\usepackage{hyperref}|\\
%     |\hypersetup{pdfauthor=Fran\c coise M\"uller}|
%   \end{quote}
%   As package option it will even break with |Fran\c coise|
%   because of the cedilla |\c c|, it is not robust enough.
% \end{itemize}
% For users that do not want with this limitations the package
% offers package \xoption{kvoptions-patch}. It patches \LaTeX's option system
% and tries to teach it also to handle options that are given
% as pairs of key and value and to prevent expansion.
% It can already be used at the very beginning, before \cs{documentclass}:
% \begin{quote}
%   |\RequirePackage{kvoptions-patch}|\\
%   |\documentclass[pdfauthor=Fran\c coise M\"uller]{article}|\\
%   |\usepackage{hyperref}|
% \end{quote}
% The latest time is before the package where you want to use
% problematic values:
% \begin{quote}
%   |\usepackage{kvoptions-patch}|\\
%   |\usepackage[Fran\c coise M\"uller]{hyperref}|
% \end{quote}
% Some remarks:
% \begin{itemize}
% \item The patch requires \eTeX, its \cs{unexpanded} feature
%   is much too nice. It is possible to work around using token
%   registers. But the code becomes longer, slower, more difficult
%   to read and maintain. The package without option \xoption{patch}
%   works and will work without \eTeX.
% \item The code for the patch is quite long, there are many test
%   cases. Thus the probability for bugs is probably not too small.
% \item Since 2008/10/18 v3.0 package \xpackage{kvoptions-patch} is
%   available. Before option \xoption{patch} of package \xpackage{kvoptions}
%   must be used instead. I think, the solution as standalone package
%   \xpackage{kvoptions-patch} is cleaner and avoids option clashes.
% \end{itemize}
%
% \subsection{Option \xoption{debugshow}}
%
% The name of this option follows the convention of packages
% \xpackage{multicol}, \xpackage{tabularx}, and \xpackage{tracefnt}.
% Currently it prints the setting of boolean options, declared
% by \cs{DeclareBoolOption} in the \xfile{.log} file, if that
% boolean option is used.
% You can activate the option by
% \begin{itemize}
% \item |\PassOptionsToPackage{debugshow}{kvoptions}|\\
%       Put this somewhere before package \xpackage{kvoptions} is
%       loaded first, e.g. before \cs{documentclass}.
% \item |\RequirePackage[debugshow]{kvoptions}|\\
%       Before \cs{documentclass} even an author has to use
%       \cs{RequirePackage}. \cs{usepackage} only works after
%       \cs{documentclass}.
% \end{itemize}
% The preferred method is \cs{PassOptionsToPackage}, because
% it does not force the package loading and does not disturb,
% if the package is not loaded later at all.
%
% \section{Limitations}
%
% \subsection{Compatibility}
% \subsubsection{Package \xpackage{kvoptions-patch} vs. package \xpackage{xkvltxp}}
%
% Package \xpackage{xkvltxp} from the \xpackage{xkeyval} project
% has the same goal as package \xpackage{kvoptions-patch}
% and to patch \LaTeX's kernel commands in order to get better support
% for key value options.
% Of course they cannot be used both. The user must decide, which
% method he prefers. Package \xpackage{kvoptions-patch} aborts itself,
% if it detects that \xpackage{xkvltxp} is
% already loaded.
%
% However package \xpackage{xkvltxp} and \xpackage{kvoptions}
% can be used together, example:
% \begin{quote}
% |\usepackage{xkvltxp}|\\
% |\usepackage[...]{foobar} % |\texttt{\itshape foobar using kvoptions}
% \end{quote}
% The other way should work, too.
%
% Package \xpackage{kvoptions-patch} tries to catch more situations
% and to be more robust.
% For example, during the comparison of options it normalizes
% them by removing spaces around |=| and the value. Thus the
% following is not reported as option clash:
% \begin{quote}
%\begin{verbatim}
%\RequirePackage{kvoptions-patch}
%\documentclass{article}
%
%\usepackage[scaled=0.7]{helvet}
%\usepackage[scaled = 0.7]{helvet}
%
%\begin{document}
%\end{document}
%\end{verbatim}
% \end{quote}
%
% \subsection{Limitations}
%
% \subsubsection{Option comparisons}
%
% In some situations \LaTeX\ compares option lists, e.g. option clash check,
% \cs{@ifpackagewith}, or \cs{@ifclasswith}. Apart from catcode and
% sanitizing problems of option \xoption{patch}, there is another problem.
% \LaTeX\ does not know about the type and default values of
% options in key value style. Thus an option clash is reported,
% even if the key value has the same meaning:
% \begin{quote}
% |\usepackage[scaled]{helvet} %| \texttt{\textit{default is }.95}\\
% |\usepackage[.95]{helvet}|\\
% |\usepackage[0.95]{helvet}|
% \end{quote}
%
% \subsubsection{Option list parsing with package \xpackage{kvoptions-patch}}
%
% With package \xpackage{kvoptions-patch} the range of possible values
% in key value specifications is much large, for example
% the comma can be used, if enclosed in curly braces.
%
% Other packages, especially the packages that uses their
% own process option code can be surprised to find tokens
% inside options that they do not expect and errors would
% be the consequence. To avoid errors the options, especially
% the unused option list is sanitized. That means the list
% will only contain tokens with catcode 12 (other) and
% perhaps spaces (catcode 10). This allows a safe parsing
% for other packages. But a comma in the value part is
% no longer protected by curly braces because they have lost
% their special meaning. This is the price for compatibility.
%
% Example:
% \begin{quote}
% |\RequirePackage{kvoptions-patch}|\\
% |\documentclass[a={a,b,c},b]{article}|\\
% |\begin{document}|\\
% |\end{document}|
% \end{quote}
% Result:
% \begin{quote}
% |LaTeX Warning: Unused global option(s):|\\
% |    [a={a,c},b].|
% \end{quote}
%
% \StopEventually{
% }
%
% \section{Implementation}
%
% \subsection{Preamble}
%
%    \begin{macrocode}
%<*package>
%    \end{macrocode}
%
% \paragraph{Reload check and  identification.}
%    Reload check, especially if the package is not used with \LaTeX.
%    \begin{macrocode}
\begingroup\catcode61\catcode48\catcode32=10\relax%
  \catcode13=5 % ^^M
  \endlinechar=13 %
  \catcode35=6 % #
  \catcode39=12 % '
  \catcode44=12 % ,
  \catcode45=12 % -
  \catcode46=12 % .
  \catcode58=12 % :
  \catcode64=11 % @
  \catcode123=1 % {
  \catcode125=2 % }
  \expandafter\let\expandafter\x\csname ver@kvoptions.sty\endcsname
  \ifx\x\relax % plain-TeX, first loading
  \else
    \def\empty{}%
    \ifx\x\empty % LaTeX, first loading,
      % variable is initialized, but \ProvidesPackage not yet seen
    \else
      \expandafter\ifx\csname PackageInfo\endcsname\relax
        \def\x#1#2{%
          \immediate\write-1{Package #1 Info: #2.}%
        }%
      \else
        \def\x#1#2{\PackageInfo{#1}{#2, stopped}}%
      \fi
      \x{kvoptions}{The package is already loaded}%
      \aftergroup\endinput
    \fi
  \fi
\endgroup%
%    \end{macrocode}
%    Package identification:
%    \begin{macrocode}
\begingroup\catcode61\catcode48\catcode32=10\relax%
  \catcode13=5 % ^^M
  \endlinechar=13 %
  \catcode35=6 % #
  \catcode39=12 % '
  \catcode40=12 % (
  \catcode41=12 % )
  \catcode44=12 % ,
  \catcode45=12 % -
  \catcode46=12 % .
  \catcode47=12 % /
  \catcode58=12 % :
  \catcode64=11 % @
  \catcode91=12 % [
  \catcode93=12 % ]
  \catcode123=1 % {
  \catcode125=2 % }
  \expandafter\ifx\csname ProvidesPackage\endcsname\relax
    \def\x#1#2#3[#4]{\endgroup
      \immediate\write-1{Package: #3 #4}%
      \xdef#1{#4}%
    }%
  \else
    \def\x#1#2[#3]{\endgroup
      #2[{#3}]%
      \ifx#1\@undefined
        \xdef#1{#3}%
      \fi
      \ifx#1\relax
        \xdef#1{#3}%
      \fi
    }%
  \fi
\expandafter\x\csname ver@kvoptions.sty\endcsname
\ProvidesPackage{kvoptions}%
  [2016/05/16 v3.12 Key value format for package options (HO)]%
%    \end{macrocode}
%
% \paragraph{Catcodes}
%
%    \begin{macrocode}
\begingroup\catcode61\catcode48\catcode32=10\relax%
  \catcode13=5 % ^^M
  \endlinechar=13 %
  \catcode123=1 % {
  \catcode125=2 % }
  \catcode64=11 % @
  \def\x{\endgroup
    \expandafter\edef\csname KVO@AtEnd\endcsname{%
      \endlinechar=\the\endlinechar\relax
      \catcode13=\the\catcode13\relax
      \catcode32=\the\catcode32\relax
      \catcode35=\the\catcode35\relax
      \catcode61=\the\catcode61\relax
      \catcode64=\the\catcode64\relax
      \catcode123=\the\catcode123\relax
      \catcode125=\the\catcode125\relax
    }%
  }%
\x\catcode61\catcode48\catcode32=10\relax%
\catcode13=5 % ^^M
\endlinechar=13 %
\catcode35=6 % #
\catcode64=11 % @
\catcode123=1 % {
\catcode125=2 % }
\def\TMP@EnsureCode#1#2{%
  \edef\KVO@AtEnd{%
    \KVO@AtEnd
    \catcode#1=\the\catcode#1\relax
  }%
  \catcode#1=#2\relax
}
\TMP@EnsureCode{1}{14}% ^^A (comment)
\TMP@EnsureCode{2}{14}% ^^A (comment)
\TMP@EnsureCode{33}{12}% !
\TMP@EnsureCode{39}{12}% '
\TMP@EnsureCode{40}{12}% (
\TMP@EnsureCode{41}{12}% )
\TMP@EnsureCode{42}{12}% *
\TMP@EnsureCode{44}{12}% ,
\TMP@EnsureCode{45}{12}% -
\TMP@EnsureCode{46}{12}% .
\TMP@EnsureCode{47}{12}% /
\TMP@EnsureCode{58}{12}% :
\TMP@EnsureCode{62}{12}% >
\TMP@EnsureCode{91}{12}% [
\TMP@EnsureCode{93}{12}% ]
\TMP@EnsureCode{94}{7}% ^ (superscript)
\TMP@EnsureCode{96}{12}% `
\edef\KVO@AtEnd{\KVO@AtEnd\noexpand\endinput}
%    \end{macrocode}
%
% \paragraph{External resources.}
%    The package extends the support for key value pairs of
%    package \cs{keyval} to package options. Thus the package
%    needs to be loaded anyway. and we use it for
%    \cs{SetupKeyvalOptions}. AFAIK this does not disturb
%    users of \xpackage{xkeyval}.
%    \begin{macrocode}
\@ifundefined{define@key}{%
  \RequirePackage{keyval}\relax
}{}
%    \end{macrocode}
%
%    Macro \cs{DeclareLocalOptions} parses a comma separated key list
%    and uses \cs{comma@parse} of package \xpackage{kvsetkeys},
%    version 1.3.
%    \begin{macrocode}
\RequirePackage{ltxcmds}[2010/12/02]
\RequirePackage{kvsetkeys}[2007/09/29]
%    \end{macrocode}
%
% \paragraph{Provide macros for \plainTeX.}
%    \begin{macrocode}
\@ifundefined{@x@protect}{%
  \def\@x@protect#1\fi#2#3{%
    \fi\protect#1%
  }%
  \let\@typeset@protect\relax
}{}
\@ifundefined{@currname}{%
  \def\@currname{}%
}{}
\@ifundefined{@currext}{%
  \def\@currext{}%
}{}
%    \end{macrocode}
%
% \paragraph{Options}
%    Option \xoption{debugshow} enables additional lines of
%    code that prints information into the \xfile{.log}
%    file.
%    \begin{macrocode}
\DeclareOption{debugshow}{\catcode\@ne=9 }
%    \end{macrocode}
%
%    \begin{macrocode}
\DeclareOption{patch}{%
  \AtEndOfPackage{%
    \RequirePackage{kvoptions-patch}[2016/05/16]%
  }%
}
%    \end{macrocode}
%
%    Optionen auswerten:
%    \begin{macrocode}
\ProcessOptions\relax
%    \end{macrocode}
%
% \subsection{Option declaration macros}
%
% \subsubsection{\cs{SetupKeyvalOptions}}
%
%    The family for the key value pairs can be setup once
%    and is remembered later.
%    The package name seems a reasonable default for the
%    family key, if it is not set by the package author.
%
%    \begin{macro}{\KVO@family}
%    We cannot store the family setting in one macro,
%    because the package should be usable for many
%    other packages, too. Thus we remember the family
%    setting in a macro, whose name contains the
%    package name with extension, a key in \LaTeX's
%    class/package system.
%    \begin{macrocode}
\define@key{KVO}{family}{%
  \expandafter\edef\csname KVO@family@%
      \@currname.\@currext\endcsname{#1}%
}
\def\KVO@family{%
  \@ifundefined{KVO@family@\@currname.\@currext}{%
    \@currname
  }{%
    \csname KVO@family@\@currname.\@currext\endcsname
  }%
}
%    \end{macrocode}
%    \end{macro}
%
%    \begin{macro}{\KVO@prefix}
%    The value settings of options that are declared by \cs{DeclareBoolOption}
%    and \cs{DeclareStringOption} need to be saved in macros.
%    in the first case this is a switch \cs{if}\meta{prefix}\meta{key},
%    in the latter case a macro \cs{}\meta{prefix}\meta{key}.
%    The prefix can be configured, by \xoption{prefix} that is declared
%    here. The default is the package name with |@| appended.
%    \begin{macrocode}
\define@key{KVO}{prefix}{%
  \expandafter\edef\csname KVO@prefix@%
      \@currname.\@currext\endcsname{#1}%
}
\def\KVO@prefix{%
  \ltx@ifundefined{KVO@prefix@\@currname.\@currext}{%
    \@currname @%
  }{%
    \csname KVO@prefix@\@currname.\@currext\endcsname
  }%
}
%    \end{macrocode}
%    \end{macro}
%
%    \begin{macrocode}
\define@key{KVO}{setkeys}{%
  \expandafter\def\csname KVO@setkeys@%
      \@currname.\@currext\endcsname{#1}%
}
%    \end{macrocode}
%    \begin{macro}{\KVO@setkeys}
%    \begin{macrocode}
\def\KVO@setkeys{%
  \ltx@IfUndefined{KVO@setkeys@\@currname.\@currext}{%
    \setkeys
  }{%
    \csname KVO@setkeys@\@currname.\@currext\endcsname
  }%
}
%    \end{macrocode}
%    \end{macro}
%
%    \begin{macro}{\SetupKeyvalOptions}
%    The argument of \cs{SetupKeyvalOptions} expects a key value
%    list, known keys are \xoption{family} and \xoption{prefix}.
%    \begin{macrocode}
\newcommand*{\SetupKeyvalOptions}{%
  \kvsetkeys{KVO}%
}
%    \end{macrocode}
%    \end{macro}
%
% \subsubsection{\cs{DeclareBoolOption}}
%
%    \begin{macro}{\DeclareBoolOption}
%    Usually options of boolean type can be given by the
%    user without value and this means a setting to \emph{true}.
%    We follow this convention here. Also it simplifies the
%    user interface.
%
%    The switch is created and initialized with \emph{false}.
%    The default setting can be overwritten by the optional
%    argument.
%
%    \LaTeX's \cs{newif} does not check for already defined
%    macros, therefore we add this check here to prevent
%    the user from accidently redefining of \TeX's primitives
%    and other macros.
%    \begin{macrocode}
\newcommand*{\DeclareBoolOption}[2][false]{%
  \KVO@ifdefinable{if\KVO@prefix#2}{%
    \KVO@ifdefinable{\KVO@prefix#2true}{%
      \KVO@ifdefinable{\KVO@prefix#2false}{%
        \csname newif\expandafter\endcsname
        \csname if\KVO@prefix#2\endcsname
        \@ifundefined{\KVO@prefix#2#1}{%
          \PackageWarning{kvoptions}{%
            Initialization of option `#2' failed,\MessageBreak
            cannot set boolean option to `#1',\MessageBreak
            use `true' or `false', now using `false'%
          }%
        }{%
          \csname\KVO@prefix#2#1\endcsname
        }%
        \begingroup
          \edef\x{\endgroup
            \noexpand\define@key{\KVO@family}{#2}[true]{%
              \noexpand\KVO@boolkey{\@currname}%
              \ifx\@currext\@clsextension
                \noexpand\@clsextension
              \else
                \noexpand\@pkgextension
              \fi
              {\KVO@prefix}{#2}{####1}%
            }%
          }%
        \x
      }%
    }%
  }%
}
%    \end{macrocode}
%    \end{macro}
%
%    \begin{macro}{\DeclareComplementaryOption}
%    The first argument is the key name, the second the
%    key that must be a boolean option with the same
%    current family and prefix. A new switch is not
%    created for the new key, we have already a switch.
%    Instead we define switch setting commands to work
%    on the parent switch.
%    \begin{macrocode}
\newcommand*{\DeclareComplementaryOption}[2]{%
  \@ifundefined{if\KVO@prefix#2}{%
    \PackageError{kvoptions}{%
      Cannot generate option code for `#1',\MessageBreak
      parent switch `#2' does not exist%
    }{%
      You are inside %
      \ifx\@currext\@clsextension class\else package\fi\space
      `\@currname.\@currext'.\MessageBreak
      `\KVO@family' is used as familiy %
      for the keyval options.\MessageBreak
      `\KVO@prefix' serves as prefix %
      for internal switch macros.\MessageBreak
      \MessageBreak
      \@ehc
    }%
  }{%
    \KVO@ifdefinable{\KVO@prefix#1true}{%
      \KVO@ifdefinable{\KVO@prefix#1false}{%
        \expandafter\let\csname\KVO@prefix#1false\expandafter\endcsname
          \csname\KVO@prefix#2true\endcsname
        \expandafter\let\csname\KVO@prefix#1true\expandafter\endcsname
          \csname\KVO@prefix#2false\endcsname
%    \end{macrocode}
%    The same code part as in \cs{DeclareBoolOption} can
%    now be used.
%    \begin{macrocode}
        \begingroup
          \edef\x{\endgroup
            \noexpand\define@key{\KVO@family}{#1}[true]{%
              \noexpand\KVO@boolkey{\@currname}%
              \ifx\@currext\@clsextension
                \noexpand\@clsextension
              \else
                \noexpand\@pkgextension
              \fi
              {\KVO@prefix}{#1}{####1}%
            }%
          }%
        \x
      }%
    }%
  }%
}
%    \end{macrocode}
%    \end{macro}
%
%    \begin{macro}{\KVO@ifdefinable}
%    Generate the command token LaTeX's \cs{@ifdefinable} expects.
%    \begin{macrocode}
\def\KVO@ifdefinable#1{%
  \expandafter\@ifdefinable\csname #1\endcsname
}
%    \end{macrocode}
%    \end{macro}
%
%    \begin{macro}{\KVO@boolkey}
%    We check explicitly for |true| and |false| to prevent
%    the user from accidently calling other macros.
%    \begin{quote}
%    \begin{tabular}{@{}ll@{}}
%    |#1|& package/class name\\
%    |#2|& |\@pkgextension|/|\@clsextension|\\
%    |#3|& prefix \\
%    |#4|& key name\\
%    |#5|& new value
%    \end{tabular}
%    \end{quote}
%    \begin{macrocode}
\def\KVO@boolkey#1#2#3#4#5{%
  \edef\KVO@param{#5}%
  \ltx@onelevel@sanitize\KVO@param
  \ifx\KVO@param\KVO@true
    \expandafter\@firstofone
  \else
    \ifx\KVO@param\KVO@false
      \expandafter\expandafter\expandafter\@firstofone
    \else
      \ifx#2\@clsextension
        \expandafter\ClassWarning
      \else
        \expandafter\PackageWarning
      \fi
      {#1}{%
        Value `\KVO@param' is not supported by\MessageBreak
        option `#4'%
      }%
      \expandafter\expandafter\expandafter\@gobble
    \fi
  \fi
  {%
    ^^A\ifx#2\@clsextension
    ^^A  \expandafter\ClassInfo
    ^^A\else
    ^^A  \expandafter\PackageInfo
    ^^A\fi
    ^^A{#1}{[option] #4=\KVO@param}%
    \csname#3#4\KVO@param\endcsname
  }%
}
%    \end{macrocode}
%    \end{macro}
%    \begin{macro}{\KVO@true}
%    \begin{macro}{\KVO@false}
%    The macros \cs{KVO@true} and \cs{KVO@false} are used
%    for string comparisons. After \cs{ltx@onelevel@sanitize}
%    we have only tokens with catcode 12 (other).
%    \begin{macrocode}
\def\KVO@true{true}
\def\KVO@false{false}
\ltx@onelevel@sanitize\KVO@true
\ltx@onelevel@sanitize\KVO@false
%    \end{macrocode}
%    \end{macro}
%    \end{macro}
%
% \subsubsection{\cs{DeclareStringOption}}
%
%    \begin{macro}{\DeclareStringOption}
%    \begin{macrocode}
\newcommand*{\DeclareStringOption}[2][]{%
  \@ifnextchar[{%
    \KVO@DeclareStringOption{#1}{#2}@%
  }{%
    \KVO@DeclareStringOption{#1}{#2}{}[]%
  }%
}
%    \end{macrocode}
%    \end{macro}
%    \begin{macro}{\KVO@DeclareStringOption}
%    \begin{macrocode}
\def\KVO@DeclareStringOption#1#2#3[#4]{%
  \KVO@ifdefinable{\KVO@prefix#2}{%
    \@namedef{\KVO@prefix#2}{#1}%
    \begingroup
      \ifx\\#3\\%
        \toks@{}%
      \else
        \toks@{[{#4}]}%
      \fi
      \edef\x{\endgroup
        \noexpand\define@key{\KVO@family}{#2}\the\toks@{%
          ^^A\begingroup
          ^^A  \toks@{####1}%
          ^^A  \ifx\@currext\@clsextension
          ^^A    \noexpand\ClassInfo
          ^^A  \else
          ^^A    \noexpand\PackageInfo
          ^^A  \fi
          ^^A  {\@currname}{%
          ^^A    [option] #2={\noexpand\the\toks@}%
          ^^A  }%
          ^^A\endgroup
          \noexpand\def
          \expandafter\noexpand\csname\KVO@prefix#2\endcsname{####1}%
        }%
      }%
    \x
  }%
}
%    \end{macrocode}
%    \end{macro}
%
% \subsubsection{\cs{DeclareVoidOption}}
%
%    \begin{macro}{\DeclareVoidOption}
%    \begin{macrocode}
\newcommand*{\DeclareVoidOption}[2]{%
  \begingroup
    \let\next\@gobbletwo
    \KVO@ifdefinable{\KVO@prefix#1}{%
      \let\next\@firstofone
    }%
  \expandafter\endgroup
  \next{%
    \begingroup
      \edef\x{\endgroup
        \noexpand\define@key{\KVO@family}{#1}[\KVO@VOID@]{%
          \noexpand\KVO@voidkey{\@currname}%
          \ifx\@currext\@clsextension
            \noexpand\@clsextension
          \else
            \noexpand\@pkgextension
          \fi
          {#1}%
          {####1}%
          \expandafter\noexpand\csname\KVO@prefix#1\endcsname
        }%
      }%
    \x
    \begingroup
      \toks@{#2}%
    \expandafter\endgroup
    \expandafter\def
    \csname\KVO@prefix#1\expandafter\endcsname
    \expandafter{\the\toks@}%
  }%
}
\def\KVO@VOID@{@VOID@}
%    \end{macrocode}
%    \end{macro}
%    \begin{macro}{\KVO@voidkey}
%    \begin{tabular}{@{}ll@{}}
%    |#1|& package/class name\\
%    |#2|& |\@pkgextension|/|\@clsextension|\\
%    |#3|& key name\\
%    |#4|& default (|@VOID@|)\\
%    |#5|& macro with option code
%    \end{tabular}
%    \begin{macrocode}
\def\KVO@voidkey#1#2#3#4{%
  \def\CurrentOption{#3}%
  \begingroup
    \def\x{#4}%
  \expandafter\endgroup
  \ifx\x\KVO@VOID@
  \else
    \ifx#2\@clsextension
      \expandafter\ClassWarning
    \else
      \expandafter\PackageWarning
    \fi
    {#1}{%
      Unexpected value for option `#3'\MessageBreak
      is ignored%
    }%
  \fi
  ^^A\ifx#2\@clsextension
  ^^A  \expandafter\ClassInfo
  ^^A\else
  ^^A  \expandafter\PackageInfo
  ^^A\fi
  ^^A{#1}{[option] #3}%
}
%    \end{macrocode}
%    \end{macro}
%
% \subsubsection{\cs{DeclareDefaultOption}}
%
%    \begin{macro}{\DeclareDefaultOption}
%    \begin{macrocode}
\newcommand*{\DeclareDefaultOption}{%
  \@namedef{KVO@default@\@currname.\@currext}%
}
%    \end{macrocode}
%    \end{macro}
%
% \subsubsection{\cs{DeclareLocalOptions}}
%
%    \begin{macro}{\DeclareLocalOptions}
%    \begin{macrocode}
\newcommand*{\DeclareLocalOptions}[1]{%
  \comma@parse{#1}\KVO@DeclareLocalOption
}
%    \end{macrocode}
%    \end{macro}
%    \begin{macro}{\KVO@DeclareLocalOption}
%    \begin{macrocode}
\def\KVO@DeclareLocalOption#1{%
  \expandafter\def\csname KVO@local@\KVO@family @#1\endcsname{}%
}
%    \end{macrocode}
%    \end{macro}
%
% \subsection{Dynamic options}
%
% \subsubsection{\cs{DisableKeyvalOption}}
%
%    \begin{macrocode}
\SetupKeyvalOptions{%
  family=KVOdyn,%
  prefix=KVOdyn@%
}
\DeclareBoolOption[true]{global}
\DeclareComplementaryOption{local}{global}
\DeclareStringOption[undef]{action}
\let\KVOdyn@name\relax
\let\KVOdyn@ext\@empty
\define@key{KVOdyn}{class}{%
  \def\KVOdyn@name{#1}%
  \let\KVOdyn@ext\@clsextension
}
\define@key{KVOdyn}{package}{%
  \def\KVOdyn@name{#1}%
  \let\KVOdyn@ext\@pkgextension
}
\newcommand*{\DisableKeyvalOption}[3][]{%
  \begingroup
    \kvsetkeys{KVOdyn}{#1}%
    \def\x{\endgroup}%
    \@ifundefined{KVO@action@\KVOdyn@action}{%
      \PackageError{kvoptions}{%
        Unknown disable action %
        `\expandafter\strip@prefix\meaning\KVOdyn@action'\MessageBreak
        for option `#3' in keyval family '#2'%
      }\@ehc
    }{%
      \csname KVO@action@\KVOdyn@action\endcsname{#2}{#3}%
    }%
  \x
}
\def\KVO@action@undef#1#2{%
  \edef\x{\endgroup
    \ifKVOdyn@global\global\fi
    \let
    \expandafter\noexpand\csname KV@#1@#2\endcsname
    \relax
    \ifKVOdyn@global\global\fi
    \let
    \expandafter\noexpand\csname KV@#1@#2@default\endcsname
    \relax
  }%
  ^^A\PackageInfo{kvoptions}{%
  ^^A  [option] key `#2' of family `#1'\MessageBreak
  ^^A  is disabled (undef, \ifKVOdyn@global global\else local\fi)%
  ^^A}%
}
\def\KVO@action@ignore#1#2{%
  \edef\x{\endgroup
    \ifKVOdyn@global\global\fi
    \let
    \expandafter\noexpand\csname KV@#1@#2\endcsname
    \noexpand\@gobble
    \ifKVOdyn@global\global\fi
    \let
    \expandafter\noexpand\csname KV@#1@#2@default\endcsname
    \noexpand\@empty
  }%
  ^^A\PackageInfo{kvoptions}{%
  ^^A  [option] key `#2' of family `#1'\MessageBreak
  ^^A  is disabled (ignore, \ifKVOdyn@global global\else local\fi)%
  ^^A}%
}
\def\KVO@action@error{%
  \KVO@do@action{error}%
}
\def\KVO@action@warning{%
  \KVO@do@action{warning}%
}
%    \end{macrocode}
%    \begin{tabular}{@{}ll@{}}
%    |#1|& |error| or |warning|\\
%    |#2|& \meta{family}\\
%    |#3|& \meta{key}\\
%    \end{tabular}
%    \begin{macrocode}
\def\KVO@do@action#1#2#3{%
  \ifx\KVOdyn@name\relax
    \PackageError{kvoptions}{%
      Action type `#1' needs package/class name\MessageBreak
      for key `#3' in family `#2'%
    }\@ehc
  \else
    \edef\x{\endgroup
      \noexpand\define@key{#2}{#3}[]{%
        \expandafter\noexpand\csname KVO@disable@#1\endcsname
        {\KVOdyn@name}\noexpand\KVOdyn@ext{#3}%
      }%
      \ifKVOdyn@global
        \global\let
        \expandafter\noexpand\csname KV@#2@#3\endcsname
        \expandafter\noexpand\csname KV@#2@#3\endcsname
        \global\let
        \expandafter\noexpand\csname KV@#2@#3@default\endcsname
        \expandafter\noexpand\csname KV@#2@#3@default\endcsname
      \fi
    }%
    ^^A\ifx\KVOdyn@ext\@clsextension
    ^^A  \expandafter\ClassInfo
    ^^A\else
    ^^A   \expandafter\PackageInfo
    ^^A\fi
    ^^A{\KVOdyn@name}{%
    ^^A  [option] key `#3' of family `#2'\MessageBreak
    ^^A  is disabled (#1, \ifKVOdyn@global global\else local\fi)%
    ^^A}%
  \fi
}
\def\KVO@disable@error#1#2#3{%
  \ifx#2\@clsextension
    \expandafter\ClassError
  \else
    \expandafter\PackageError
  \fi
  {#1}{%
    Option `#3' is given too late,\MessageBreak
    now the option is ignored%
  }\@ehc
}
\def\KVO@disable@warning#1#2#3{%
  \ifx#2\@clsextension
    \expandafter\ClassWarning
  \else
    \expandafter\PackageWarning
  \fi
  {#1}{%
    Option `#3' is already consumed\MessageBreak
    and has no effect%
  }%
}
%    \end{macrocode}
%
% \subsection{Change option code}
%
% \subsubsection{\cs{AddToKeyvalOption}}
%
%    \begin{macro}{\AddToKeyvalOption}
%    \begin{macrocode}
\newcommand*{\AddToKeyvalOption}{%
  \@ifstar{%
    \begingroup
      \edef\x{\endgroup
        \noexpand\KVO@AddToKeyvalOption{\KVO@family}%
      }%
    \x
  }%
  \KVO@AddToKeyvalOption
}
%    \end{macrocode}
%    \end{macro}
%    \begin{macro}{\KVO@AddToKeyvalOption}
%    \begin{macrocode}
\def\KVO@AddToKeyvalOption#1#2{%
  \@ifundefined{KV@#1@#2}{%
    \PackageWarning{kvoptions}{%
      Key `#2' of family `#1' does not exist.\MessageBreak
      Ignoring \string\AddToKeyvalOption
    }%
    \@gobble
  }{%
    \edef\KVO@next{%
      \noexpand\KVO@@AddToKeyvalOption
      \expandafter\noexpand\csname KV@#1@#2\endcsname
    }%
    \afterassignment\KVO@next
    \def\KVO@temp##1%
  }%
}
%    \end{macrocode}
%    \end{macro}
%    \begin{macro}{\KVO@@AddToKeyvalOption}
%    \begin{macrocode}
\def\KVO@@AddToKeyvalOption#1{%
  \begingroup
    \toks@\expandafter{#1{##1}}%
    \toks@\expandafter{\the\expandafter\toks@\KVO@temp{##1}}%
    \edef\x{\endgroup
      \noexpand\def\noexpand#1####1{\the\toks@}%
    }%
  \x
}
%    \end{macrocode}
%    \end{macro}
%
% \subsection{Process options}
%
% \subsubsection{Get global options}
%
%    Package \xpackage{xkeyval} removes options with equal signs from
%    the global options (\cs{@classoptionslist}). The effect is that
%    other packages and classes will not see these global options anymore.
%    A bug-report was answered that this behaviour is ``by design''.
%    Thus I call it a design bug. Now getting the global options require
%    an algorithm instead of a simple macro call.
%    \begin{macrocode}
%</package>
%<*package|patch>
%    \end{macrocode}
%    \begin{macro}{\KVO@IfDefThen}
%    Call |#2| if command |#1| is defined and not \cs{relax}.
%    (Package \xpackage{kvoptions-patch} does not load
%    package \xpackage{ltxcmds}.)
%    \begin{macrocode}
\def\KVO@IfDefThen#1#2{%
  \ifx#1\ltx@undefined
  \else
    \ifx#1\relax
    \else
      #2%
    \fi
  \fi
}%
%    \end{macrocode}
%    \end{macro}
%    \begin{macro}{\KVO@GetClassOptionsList}
%    \begin{macrocode}
\def\KVO@GetClassOptionsList{%
  \let\KVO@classoptionslist\@classoptionslist
  \KVO@IfDefThen\@classoptionslist{%
    \KVO@IfDefThen\XKV@documentclass{%
      \ifx\XKV@documentclass\ltx@empty
      \else
        \KVO@IfDefThen\XKV@classoptionslist{%
          \ifx\XKV@classoptionslist\ltx@empty
          \else
            \let\KVO@classoptionslist\XKV@classoptionslist
          \fi
        }%
      \fi
    }%
  }%
}%
%    \end{macrocode}
%    \end{macro}
%    \begin{macrocode}
%</package|patch>
%<*package>
%    \end{macrocode}
%
% \subsubsection{\cs{ProcessKeyvalOptions}}
%
%    \begin{macro}{\ProcessKeyvalOptions}
%    If the optional star is given, we get the family
%    name and expand it for safety.
%    \begin{macrocode}
\newcommand*{\ProcessKeyvalOptions}{%
  \@ifstar{%
    \begingroup
      \edef\x{\endgroup
        \noexpand\KVO@ProcessKeyvalOptions{\KVO@family}%
      }%
    \x
  }%
  \KVO@ProcessKeyvalOptions
}
%    \end{macrocode}
%    \end{macro}
%
%    \begin{macrocode}
\def\KVO@ProcessKeyvalOptions#1{%
  \let\@tempc\relax
  \let\KVO@temp\@empty
%    \end{macrocode}
%    Add any global options that are known to KV to the start of
%    the list being built in |\KVO@temp| and mark them used (by
%    removing them from the unused option list).
%    \begin{macrocode}
  \ifx\@currext\@clsextension
  \else
    \KVO@GetClassOptionsList
    \ifx\KVO@classoptionslist\relax
    \else
      \@for\KVO@CurrentOption:=\KVO@classoptionslist\do{%
        \@ifundefined{KV@#1@\expandafter\KVO@getkey
                      \KVO@CurrentOption=\@nil}{%
        }{%
          \@ifundefined{KVO@local@#1@\expandafter\KVO@getkey
                        \KVO@CurrentOption=\@nil}{%
            \ifx\KVO@Patch Y%
              \edef\KVO@temp{%
                \etex@unexpanded\expandafter{%
                  \KVO@temp
                }%
                ,%
                \etex@unexpanded\expandafter{%
                  \KVO@CurrentOption
                }%
                ,%
              }%
              \ltx@onelevel@sanitize\KVO@CurrentOption
            \else
              \edef\KVO@temp{%
                \KVO@temp
                ,%
                \KVO@CurrentOption
                ,%
              }%
            \fi
            \@expandtwoargs\@removeelement\KVO@CurrentOption
              \@unusedoptionlist\@unusedoptionlist
          }{}%
        }%
      }%
    \fi
  \fi
%    \end{macrocode}
%     Now stick the package options at the end of the list and wrap
%     in a call to \cs{setkeys}. A class ignores unknown global
%     options, we must remove them to prevent error messages
%     from \cs{setkeys}.
%    \begin{macrocode}
  \begingroup
    \toks\tw@{}%
    \@ifundefined{opt@\@currname.\@currext}{%
      \toks@\expandafter{\KVO@temp}%
    }{%
      \toks@\expandafter\expandafter\expandafter{%
        \csname opt@\@currname.\@currext\endcsname
      }%
      \ifx\@currext\@clsextension
        \edef\CurrentOption{\the\toks@}%
        \toks@\expandafter{\KVO@temp}%
        \@for\CurrentOption:=\CurrentOption\do{%
          \@ifundefined{%
            KV@#1@\expandafter\KVO@getkey\CurrentOption=\@nil
          }{%
%    \end{macrocode}
%    A class puts not used options in the unused option list
%    unless there is a default handler.
%    \begin{macrocode}
            \@ifundefined{KVO@default@\@currname.\@currext}{%
              \ifx\KVO@Patch Y%
                \ltx@onelevel@sanitize\CurrentOption
              \fi
              \ifx\@unusedoptionlist\@empty
                \global\let\@unusedoptionlist\CurrentOption
              \else
                \expandafter\expandafter\expandafter\gdef
                \expandafter\expandafter\expandafter\@unusedoptionlist
                \expandafter\expandafter\expandafter{%
                  \expandafter\@unusedoptionlist
                  \expandafter,\CurrentOption
                }%
              \fi
            }{%
              \toks\tw@\expandafter{%
                \the\toks\expandafter\tw@\expandafter,\CurrentOption
              }%
            }%
          }{%
            \toks@\expandafter{%
              \the\expandafter\toks@\expandafter,\CurrentOption
            }%
          }%
        }%
      \else
%    \end{macrocode}
%    Without default action we pass all options to \cs{setkeys}.
%    Otherwise we have to check which options are known.
%    These are passed to \cs{setkeys}. For the others the default
%    action is performed.
%    \begin{macrocode}
        \@ifundefined{KVO@default@\@currname.\@currext}{%
          \toks@\expandafter\expandafter\expandafter{%
            \expandafter\KVO@temp\the\toks@
          }%
        }{%
          \edef\CurrentOption{\the\toks@}%
          \toks@\expandafter{\KVO@temp}%
          \@for\CurrentOption:=\CurrentOption\do{%
            \@ifundefined{%
              KV@#1@\expandafter\KVO@getkey\CurrentOption=\@nil
            }{%
              \toks\tw@\expandafter{%
                \the\toks\expandafter\tw@\expandafter,\CurrentOption
              }%
            }{%
              \toks@\expandafter{%
                \the\expandafter\toks@\expandafter,\CurrentOption
              }%
            }%
          }%
        }%
      \fi
    }%
    \edef\KVO@temp{\endgroup
      \noexpand\KVO@calldefault{\the\toks\tw@}%
      \noexpand\KVO@setkeys{#1}{\the\toks@}%
    }%
  \KVO@temp
%    \end{macrocode}
%    Some cleanup of \cs{ProcessOptions}.
%    \begin{macrocode}
  \let\CurrentOption\@empty
  \AtEndOfPackage{\let\@unprocessedoptions\relax}%
}
%    \end{macrocode}
%
% \subsubsection{\cs{ProcessLocalKeyvalOptions}}
%
%    \begin{macro}{\ProcessLocalKeyvalOptions}
%    If the optional star is given, we get the family
%    name and expand it for safety.
%    \begin{macrocode}
\newcommand*{\ProcessLocalKeyvalOptions}{%
  \@ifstar{%
    \begingroup
      \edef\x{\endgroup
        \noexpand\KVO@ProcessLocalKeyvalOptions{\KVO@family}%
      }%
    \x
  }%
  \KVO@ProcessLocalKeyvalOptions
}
%    \end{macrocode}
%    \end{macro}
%
%    \begin{macrocode}
\def\KVO@ProcessLocalKeyvalOptions#1{%
  \let\@tempc\relax
  \let\KVO@temp\@empty
%    \end{macrocode}
%    Check if \cs{ProcessLocalKeyvalOptions} is called inside
%    a package.
%    \begin{macrocode}
  \ifx\@currext\@pkgextension
  \else
    \PackageError{kvoptions}{%
      \string\ProcessLocalKeyvalOptions\space is intended for packages only%
    }\@ehc
  \fi
%    \end{macrocode}
%     The package options are put into toks register \cs{toks@}.
%    \begin{macrocode}
  \begingroup
    \toks\tw@{}%
    \@ifundefined{opt@\@currname.\@currext}{%
      \toks@\expandafter{\KVO@temp}%
    }{%
      \toks@\expandafter\expandafter\expandafter{%
        \csname opt@\@currname.\@currext\endcsname
      }%
%    \end{macrocode}
%    Without default action we pass all options to \cs{setkeys}.
%    Otherwise we have to check which options are known.
%    These are passed to \cs{setkeys}. For the others the default
%    action is performed.
%    \begin{macrocode}
      \@ifundefined{KVO@default@\@currname.\@currext}{%
        \toks@\expandafter\expandafter\expandafter{%
          \expandafter\KVO@temp\the\toks@
        }%
      }{%
        \edef\CurrentOption{\the\toks@}%
        \toks@\expandafter{\KVO@temp}%
        \@for\CurrentOption:=\CurrentOption\do{%
          \@ifundefined{%
            KV@#1@\expandafter\KVO@getkey\CurrentOption=\@nil
          }{%
            \toks\tw@\expandafter{%
              \the\toks\expandafter\tw@\expandafter,\CurrentOption
            }%
          }{%
            \toks@\expandafter{%
              \the\expandafter\toks@\expandafter,\CurrentOption
            }%
          }%
        }%
      }%
    }%
    \edef\KVO@temp{\endgroup
      \noexpand\KVO@calldefault{\the\toks\tw@}%
      \noexpand\KVO@setkeys{#1}{\the\toks@}%
    }%
  \KVO@temp
%    \end{macrocode}
%    Some cleanup of \cs{ProcessOptions}.
%    \begin{macrocode}
  \let\CurrentOption\@empty
  \AtEndOfPackage{\let\@unprocessedoptions\relax}%
}
%    \end{macrocode}
%
% \subsubsection{Helper macros}
%
%    \begin{macro}{\KVO@getkey}
%    Extract the key part of a key=value pair.
%    \begin{macrocode}
\def\KVO@getkey#1=#2\@nil{#1}
%    \end{macrocode}
%    \end{macro}
%
%    \begin{macro}{\KVO@calldefault}
%    \begin{macrocode}
\def\KVO@calldefault#1{%
  \begingroup
    \def\x{#1}%
  \expandafter\endgroup
  \ifx\x\@empty
  \else
    \@for\CurrentOption:=#1\do{%
      \ifx\CurrentOption\@empty
      \else
        \expandafter\KVO@setcurrents\CurrentOption=\@nil
        \@nameuse{KVO@default@\@currname.\@currext}%
      \fi
    }%
  \fi
}
%    \end{macrocode}
%    \end{macro}
%    \begin{macro}{\KVO@setcurrents}
%    Extract the key part of a key=value pair.
%    \begin{macrocode}
\def\KVO@setcurrents#1=#2\@nil{%
  \def\CurrentOptionValue{#2}%
  \ifx\CurrentOptionValue\@empty
    \let\CurrentOptionKey\CurrentOption
    \let\CurrentOptionValue\relax
  \else
    \edef\CurrentOptionKey{\zap@space#1 \@empty}%
    \expandafter\KVO@setcurrentvalue\CurrentOption\@nil
  \fi
}
%    \end{macrocode}
%    \end{macro}
%    \begin{macro}{\KV@setcurrentvalue}
%    Here the value part is parsed. Package \xpackage{keyval}'s
%    \cs{KV@@sp@def} helps in removing spaces at the begin and
%    end of the value.
%    \begin{macrocode}
\def\KVO@setcurrentvalue#1=#2\@nil{%
  \KV@@sp@def\CurrentOptionValue{#2}%
}
%    \end{macrocode}
%    \end{macro}
%
% \subsection{\plainTeX}
%
%    Disable \LaTeX\ stuff.
%    \begin{macrocode}
\begingroup\expandafter\expandafter\expandafter\endgroup
\expandafter\ifx\csname documentclass\endcsname\relax
  \def\ProcessKeyvalOptions{%
    \@ifstar{}\@gobble
  }%
\fi
%    \end{macrocode}
%
%    \begin{macrocode}
\KVO@AtEnd%
%</package>
%    \end{macrocode}
%
% \subsection{Package \xpackage{kvoptions-patch}}
%
%    \begin{macrocode}
%<*patch>
\NeedsTeXFormat{LaTeX2e}
\begingroup\catcode61\catcode48\catcode32=10\relax%
  \catcode13=5 % ^^M
  \endlinechar=13 %
  \catcode123=1 % {
  \catcode125=2 % }
  \catcode64=11 % @
  \def\x{\endgroup
    \expandafter\edef\csname KVO@AtEnd\endcsname{%
      \endlinechar=\the\endlinechar\relax
      \catcode13=\the\catcode13\relax
      \catcode32=\the\catcode32\relax
      \catcode35=\the\catcode35\relax
      \catcode61=\the\catcode61\relax
      \catcode64=\the\catcode64\relax
      \catcode123=\the\catcode123\relax
      \catcode125=\the\catcode125\relax
    }%
  }%
\x\catcode61\catcode48\catcode32=10\relax%
\catcode13=5 % ^^M
\endlinechar=13 %
\catcode35=6 % #
\catcode64=11 % @
\catcode123=1 % {
\catcode125=2 % }
\def\TMP@EnsureCode#1#2{%
  \edef\KVO@AtEnd{%
    \KVO@AtEnd
    \catcode#1=\the\catcode#1\relax
  }%
  \catcode#1=#2\relax
}
\TMP@EnsureCode{39}{12}% '
\TMP@EnsureCode{40}{12}% (
\TMP@EnsureCode{41}{12}% )
\TMP@EnsureCode{43}{12}% +
\TMP@EnsureCode{44}{12}% ,
\TMP@EnsureCode{45}{12}% -
\TMP@EnsureCode{46}{12}% .
\TMP@EnsureCode{47}{12}% /
\TMP@EnsureCode{58}{12}% :
\TMP@EnsureCode{60}{12}% <
\TMP@EnsureCode{62}{12}% >
\TMP@EnsureCode{91}{12}% [
\TMP@EnsureCode{93}{12}% ]
\TMP@EnsureCode{96}{12}% `
\TMP@EnsureCode{124}{12}% |
\edef\KVO@AtEnd{\KVO@AtEnd\noexpand\endinput}
\ProvidesPackage{kvoptions-patch}%
  [2016/05/16 v3.12 LaTeX patch for keyval options (HO)]%
%    \end{macrocode}
%
%    Check for \eTeX.
%    \begin{macrocode}
\begingroup\expandafter\expandafter\expandafter\endgroup
\expandafter\ifx\csname eTeXversion\endcsname\relax
  \PackageWarningNoLine{kvoptions-patch}{%
    Package loading is aborted, because e-TeX is missing%
  }%
  \expandafter\KVO@AtEnd
\fi%
%    \end{macrocode}
%
%    Package \xpackage{etexcmds} for \cs{etex@unexpanded}.
%    \begin{macrocode}
\RequirePackage{etexcmds}[2007/09/09]
\ifetex@unexpanded
\else
  \PackageError{kvoptions-patch}{%
    Could not find eTeX's \string\unexpanded.\MessageBreak
    Try adding \string\RequirePackage\string{etexcmds\string} %
    before \string\documentclass%
  }\@ehd
  \expandafter\KVO@AtEnd
\fi%
%    \end{macrocode}
%
%    Check for package \xpackage{xkvltxp}.
%    \begin{macrocode}
\@ifpackageloaded{xkvltxp}{%
  \PackageWarningNoLine{kvoptions}{%
    Option `patch' cannot be used together with\MessageBreak
    package `xkvltxp' that is already loaded.\MessageBreak
    Therefore package loading is aborted%
  }%
  \KVO@AtEnd
}{}%
%    \end{macrocode}
%
%    \begin{macrocode}
\def\@if@ptions#1#2#3{%
  \begingroup
    \KVO@normalize\KVO@temp{#3}%
    \edef\x{\endgroup
      \noexpand\@if@pti@ns{%
        \detokenize\expandafter\expandafter\expandafter{%
          \csname opt@#2.#1\endcsname
        }%
      }{%
        \detokenize\expandafter{\KVO@temp}%
      }%
    }%
  \x
}
%    \end{macrocode}
%
%    \begin{macrocode}
\def\@pass@ptions#1#2#3{%
  \KVO@normalize\KVO@temp{#2}%
  \@ifundefined{opt@#3.#1}{%
    \expandafter\gdef\csname opt@#3.#1%
          \expandafter\endcsname\expandafter{%
      \KVO@temp
    }%
  }{%
    \expandafter\gdef\csname opt@#3.#1%
          \expandafter\expandafter\expandafter\endcsname
          \expandafter\expandafter\expandafter{%
      \csname opt@#3.#1\expandafter\endcsname\expandafter,\KVO@temp
    }%
  }%
}
%    \end{macrocode}
%
%    \begin{macrocode}
\def\ProcessOptions{%
  \let\ds@\@empty
  \@ifundefined{opt@\@currname.\@currext}{%
    \let\@curroptions\@empty
  }{%
    \expandafter\expandafter\expandafter\def
    \expandafter\expandafter\expandafter\@curroptions
    \expandafter\expandafter\expandafter{%
      \csname opt@\@currname.\@currext\endcsname
    }%
  }%
  \@ifstar\KVO@xprocess@ptions\KVO@process@ptions
}
%    \end{macrocode}
%
%    \begin{macrocode}
\def\KVO@process@ptions{%
  \@for\CurrentOption:=\@declaredoptions\do{%
    \ifx\CurrentOption\@empty
    \else
      \begingroup
        \ifx\@currext\@clsextension
          \toks@{}%
        \else
          \KVO@GetClassOptionsList
          \toks@\expandafter{\KVO@classoptionslist,}%
        \fi
        \toks\tw@\expandafter{\@curroptions}%
        \edef\x{\endgroup
          \noexpand\in@{,\CurrentOption,}{,\the\toks@\the\toks\tw@,}%
        }%
      \x
      \ifin@
        \KVO@use@ption
        \expandafter\let\csname ds@\CurrentOption\endcsname\@empty
      \fi
    \fi
  }%
  \KVO@process@pti@ns
}
%    \end{macrocode}
%
%    \begin{macrocode}
\def\KVO@xprocess@ptions{%
  \ifx\@currext\@clsextension
  \else
    \KVO@GetClassOptionsList
    \@for\CurrentOption:=\KVO@classoptionslist\do{%
      \ifx\CurrentOption\@empty
      \else
        \KVO@in@\CurrentOption\@declaredoptions
        \ifin@
          \KVO@use@ption
          \expandafter\let\csname ds@\CurrentOption\endcsname\@empty
        \fi
      \fi
    }%
  \fi
  \KVO@process@pti@ns
}
%    \end{macrocode}
%
%    \begin{macrocode}
\def\KVO@in@#1#2{%
  \in@false
  \begingroup
    \@for\x:=#2\do{%
      \ifx\x#1\relax
        \in@true
      \fi
    }%
    \edef\x{\endgroup
      \ifin@
        \noexpand\in@true
      \fi
    }%
  \x
}
%    \end{macrocode}
%
%    \begin{macrocode}
\def\KVO@process@pti@ns{%
  \@for\CurrentOption:=\@curroptions\do{%
    \@ifundefined{ds@\KVO@SanitizedCurrentOption}{%
      \KVO@use@ption
      \default@ds
    }%
    \KVO@use@ption
  }%
  \@for\CurrentOption:=\@declaredoptions\do{%
    \expandafter\let\csname ds@\CurrentOption\endcsname\relax
  }%
  \let\CurrentOption\@empty
  \let\@fileswith@pti@ns\@@fileswith@pti@ns
  \AtEndOfPackage{\let\@unprocessedoptions\relax}%
}
%    \end{macrocode}
%
%    \begin{macrocode}
\def\KVO@use@ption{%
  \begingroup
    \edef\x{\endgroup
      \noexpand\@removeelement{%
        \detokenize\expandafter{\CurrentOption}%
      }{%
        \detokenize\expandafter{\@unusedoptionlist}%
      }%
    }%
  \x\@unusedoptionlist
  \csname ds@\KVO@SanitizedCurrentOption\endcsname
}
%    \end{macrocode}
%
%    \begin{macrocode}
\def\OptionNotUsed{%
  \ifx\@currext\@clsextension
    \xdef\@unusedoptionlist{%
      \ifx\@unusedoptionlist\@empty
      \else
        \detokenize\expandafter{\@unusedoptionlist,}%
      \fi
      \detokenize\expandafter{\CurrentOption}%
    }%
  \fi
}
%    \end{macrocode}
%
%    Variant of \cs{ExecuteOptions} that better protects \cs{CurrentOption}.
%    \begin{macrocode}
\def\CurrentOption@SaveLevel{0}
\def\ExecuteOptions{%
  \expandafter\KVO@ExecuteOptions
      \csname CurrentOption@\CurrentOption@SaveLevel\endcsname
}
\def\KVO@ExecuteOptions#1#2{%
  \let#1\CurrentOption
  \edef\CurrentOption@SaveLevel{%
    \the\numexpr\CurrentOption@SaveLevel+1%
  }%
  \@for\CurrentOption:=#2\do{%
    \csname ds@\CurrentOption\endcsname
  }%
  \edef\CurrentOption@SaveLevel{%
    \the\numexpr\CurrentOption@SaveLevel-1%
  }%
  \let\CurrentOption#1%
}
%    \end{macrocode}
%
%    \begin{macrocode}
\def\KVO@fileswith@pti@ns#1[#2]#3[#4]{%
  \ifx#1\@clsextension
    \ifx\@classoptionslist\relax
      \KVO@normalize\KVO@temp{#2}%
      \expandafter\gdef\expandafter\@classoptionslist\expandafter{%
        \KVO@temp
      }%
      \def\reserved@a{%
        \KVO@onefilewithoptions{#3}[{#2}][{#4}]#1%
        \@documentclasshook
      }%
    \else
      \def\reserved@a{%
        \KVO@onefilewithoptions{#3}[{#2}][{#4}]#1%
      }%
    \fi
  \else
    \begingroup
      \let\KVO@temp\relax
      \let\KVO@onefilewithoptions\relax
      \let\@pkgextension\relax
      \def\reserved@b##1,{%
        \ifx\@nil##1\relax
        \else
          \ifx\relax##1\relax
          \else
            \KVO@onefilewithoptions{##1}[{\KVO@temp}][{#4}]%
            \@pkgextension
          \fi
          \expandafter\reserved@b
        \fi
      }%
      \edef\reserved@a{\zap@space#3 \@empty}%
      \edef\reserved@a{\expandafter\reserved@b\reserved@a,\@nil,}%
      \toks@{#2}%
      \def\KVO@temp{\the\toks@}%
    \edef\reserved@a{\endgroup \reserved@a}%
  \fi
  \reserved@a
}
%    \end{macrocode}
%
%    \begin{macrocode}
\def\KVO@onefilewithoptions#1[#2][#3]#4{%
  \@pushfilename
  \xdef\@currname{#1}%
  \global\let\@currext#4%
  \expandafter\let\csname\@currname.\@currext-h@@k\endcsname\@empty
  \let\CurrentOption\@empty
  \@reset@ptions
  \makeatletter
  \def\reserved@a{%
    \@ifl@aded\@currext{#1}{%
      \@if@ptions\@currext{#1}{#2}{%
      }{%
        \begingroup
          \@ifundefined{opt@#1.\@currext}{%
            \def\x{}%
          }{%
            \edef\x{%
              \expandafter\expandafter\expandafter\strip@prefix
              \expandafter\meaning\csname opt@#1.\@currext\endcsname
            }%
          }%
          \def\y{#2}%
          \edef\y{\expandafter\strip@prefix\meaning\y}%
          \@latex@error{Option clash for \@cls@pkg\space #1}{%
            The package #1 has already been loaded %
            with options:\MessageBreak
            \space\space[\x]\MessageBreak
            There has now been an attempt to load it %
             with options\MessageBreak
            \space\space[\y]\MessageBreak
            Adding the global options:\MessageBreak
            \space\space
                 \x,\y\MessageBreak
            to your \noexpand\documentclass declaration may fix this.%
            \MessageBreak
            Try typing \space <return> \space to proceed.%
          }%
        \endgroup
      }%
    }{%
      \@pass@ptions\@currext{#2}{#1}%
      \global\expandafter
      \let\csname ver@\@currname.\@currext\endcsname\@empty
      \InputIfFileExists
        {\@currname.\@currext}%
        {}%
        {\@missingfileerror\@currname\@currext}%
      \let\@unprocessedoptions\@@unprocessedoptions
      \csname\@currname.\@currext-h@@k\endcsname
      \expandafter\let\csname\@currname.\@currext-h@@k\endcsname
              \@undefined
      \@unprocessedoptions
    }%
    \@ifl@ter\@currext{#1}{#3}{%
    }{%
      \@latex@warning@no@line{%
        You have requested,\on@line, %
        version\MessageBreak
          #3' of \@cls@pkg\space #1,\MessageBreak
        but only version\MessageBreak
         `\csname ver@#1.\@currext\endcsname'\MessageBreak
        is available%
      }%
    }%
    \ifx\@currext\@clsextension\let\LoadClass\@twoloadclasserror\fi
    \@popfilename
    \@reset@ptions
  }%
  \reserved@a
}
%    \end{macrocode}
%
%    \begin{macrocode}
\def\@unknownoptionerror{%
  \@latex@error{%
    Unknown option `\KVO@SanitizedCurrentOption' %
    for \@cls@pkg\space`\@currname'%
  }{%
    The option `\KVO@SanitizedCurrentOption' was not declared in %
    \@cls@pkg\space`\@currname', perhaps you\MessageBreak
    misspelled its name. %
    Try typing \space <return> %
    \space to proceed.%
  }%
}
%    \end{macrocode}
%
%    \begin{macrocode}
\def\@@unprocessedoptions{%
  \ifx\@currext\@pkgextension
    \@ifundefined{opt@\@currname.\@currext}{%
      \let\@curroptions\@empty
    }{%
      \expandafter\let\expandafter\@curroptions
          \csname opt@\@currname.\@currext\endcsname
    }%
    \@for\CurrentOption:=\@curroptions\do{%
        \ifx\CurrentOption\@empty\else\@unknownoptionerror\fi
    }%
  \fi
}
%    \end{macrocode}
%
%    \begin{macrocode}
\def\KVO@SanitizedCurrentOption{%
  \expandafter\strip@prefix\meaning\CurrentOption
}
%    \end{macrocode}
%
%    Normalize option list.
%    \begin{macrocode}
\def\KVO@normalize#1#2{%
  \let\KVO@result\@empty
  \KVO@splitcomma#2,\@nil
  \let#1\KVO@result
}
\def\KVO@splitcomma#1,#2\@nil{%
  \KVO@ifempty{#1}{}{%
    \KVO@checkkv#1=\@nil
  }%
  \KVO@ifempty{#2}{}{\KVO@splitcomma#2\@nil}%
}
\def\KVO@ifempty#1{%
  \expandafter\ifx\expandafter\\\detokenize{#1}\\%
    \expandafter\@firstoftwo
  \else
    \expandafter\@secondoftwo
  \fi
}
\def\KVO@checkkv#1=#2\@nil{%
  \KVO@ifempty{#2}{%
    % option without value
    \edef\KVO@x{\zap@space#1 \@empty}%
    \ifx\KVO@x\@empty
      % ignore empty option
    \else
      % append to list
      \edef\KVO@result{%
        \etex@unexpanded\expandafter{\KVO@result},\KVO@x
      }%
    \fi
  }{%
    % #1: "key", #2: "value="
    % add key part
    \edef\KVO@result{%
      \etex@unexpanded\expandafter{\KVO@result},%
      \zap@space#1 \@empty
    }%
    \futurelet\@let@token\KVO@checkfirsttok#2 \@nil| = \@nil|\KVO@nil
  }%
}
\def\KVO@checkfirsttok{%
  \ifx\@let@token\bgroup
    % no space at start
    \expandafter\KVO@removelastspace\expandafter=%
    % "<value><spaceopt>= \@nil"
  \else
    \expandafter\KVO@checkfirstA
  \fi
}
\def\KVO@checkfirstA#1 #2\@nil{%
  \KVO@ifempty{#2}{%
    \KVO@removelastspace=#1 \@nil
  }{%
    \KVO@ifempty{#1}{%
      \KVO@removelastspace=#2\@nil
    }{%
      \KVO@removelastspace=#1 #2\@nil
    }%
  }%
}
\def\KVO@removelastspace#1 = \@nil|#2\KVO@nil{%
  \KVO@ifempty{#2}{%
    \edef\KVO@result{%
      \etex@unexpanded\expandafter{\KVO@result}%
      \etex@unexpanded\expandafter{\KVO@removegarbage#1\KVO@nil}%
    }%
  }{%
    \edef\KVO@result{%
      \etex@unexpanded\expandafter{\KVO@result}%
      \etex@unexpanded{#1}%
    }%
  }%
}
\def\KVO@removegarbage#1= \@nil#2\KVO@nil{#1}%
%    \end{macrocode}
%
%    Arguments |#1| and |#2| are macros.
%    \begin{macrocode}
\def\KVO@removeelement#1#2{%
  \begingroup
    \toks@={}%
    \@for\x:=#2\do{%
      \ifx\x\@empty
      \else
        \ifx\x#1\relax
        \else
          \edef\t{\the\toks@}%
          \ifx\t\@empty
          \else
            \toks@\expandafter{\the\toks@,}%
          \fi
          \toks@\expandafter{\the\expandafter\toks@\x}%
        \fi
      \fi
    }%
    \edef\x{\endgroup
      \def\noexpand#2{\the\toks@}%
    }%
  \x
}
%    \end{macrocode}
%
%
%    \begin{macrocode}
\let\@@fileswith@pti@ns\KVO@fileswith@pti@ns
\ifx\@fileswith@pti@ns\@badrequireerror
\else
  \let\@fileswith@pti@ns\KVO@fileswith@pti@ns
\fi
%    \end{macrocode}
%
%    \begin{macro}{\KVO@Patch}
%    \begin{macrocode}
\let\KVO@Patch=Y
%    \end{macrocode}
%    \end{macro}
%
%    \begin{macrocode}
\KVO@AtEnd%
%</patch>
%    \end{macrocode}
%
% \section{Test}
%
% \subsection{Preface for standard catcode check}
%
%    \begin{macrocode}
%<*test1>
\input miniltx.tex\relax
%</test1>
%    \end{macrocode}
%
% \subsection{Catcode checks for loading}
%
%    \begin{macrocode}
%<*test1>
%    \end{macrocode}
%    \begin{macrocode}
\catcode`\{=1 %
\catcode`\}=2 %
\catcode`\#=6 %
\catcode`\@=11 %
\expandafter\ifx\csname count@\endcsname\relax
  \countdef\count@=255 %
\fi
\expandafter\ifx\csname @gobble\endcsname\relax
  \long\def\@gobble#1{}%
\fi
\expandafter\ifx\csname @firstofone\endcsname\relax
  \long\def\@firstofone#1{#1}%
\fi
\expandafter\ifx\csname loop\endcsname\relax
  \expandafter\@firstofone
\else
  \expandafter\@gobble
\fi
{%
  \def\loop#1\repeat{%
    \def\body{#1}%
    \iterate
  }%
  \def\iterate{%
    \body
      \let\next\iterate
    \else
      \let\next\relax
    \fi
    \next
  }%
  \let\repeat=\fi
}%
\def\RestoreCatcodes{}
\count@=0 %
\loop
  \edef\RestoreCatcodes{%
    \RestoreCatcodes
    \catcode\the\count@=\the\catcode\count@\relax
  }%
\ifnum\count@<255 %
  \advance\count@ 1 %
\repeat

\def\RangeCatcodeInvalid#1#2{%
  \count@=#1\relax
  \loop
    \catcode\count@=15 %
  \ifnum\count@<#2\relax
    \advance\count@ 1 %
  \repeat
}
\def\RangeCatcodeCheck#1#2#3{%
  \count@=#1\relax
  \loop
    \ifnum#3=\catcode\count@
    \else
      \errmessage{%
        Character \the\count@\space
        with wrong catcode \the\catcode\count@\space
        instead of \number#3%
      }%
    \fi
  \ifnum\count@<#2\relax
    \advance\count@ 1 %
  \repeat
}
\def\space{ }
\expandafter\ifx\csname LoadCommand\endcsname\relax
  \def\LoadCommand{\input kvoptions.sty\relax}%
\fi
\def\Test{%
  \RangeCatcodeInvalid{0}{47}%
  \RangeCatcodeInvalid{58}{64}%
  \RangeCatcodeInvalid{91}{96}%
  \RangeCatcodeInvalid{123}{255}%
  \catcode`\@=12 %
  \catcode`\\=0 %
  \catcode`\%=14 %
  \LoadCommand
  \RangeCatcodeCheck{0}{36}{15}%
  \RangeCatcodeCheck{37}{37}{14}%
  \RangeCatcodeCheck{38}{47}{15}%
  \RangeCatcodeCheck{48}{57}{12}%
  \RangeCatcodeCheck{58}{63}{15}%
  \RangeCatcodeCheck{64}{64}{12}%
  \RangeCatcodeCheck{65}{90}{11}%
  \RangeCatcodeCheck{91}{91}{15}%
  \RangeCatcodeCheck{92}{92}{0}%
  \RangeCatcodeCheck{93}{96}{15}%
  \RangeCatcodeCheck{97}{122}{11}%
  \RangeCatcodeCheck{123}{255}{15}%
  \RestoreCatcodes
}
\Test
\csname @@end\endcsname
\end
%    \end{macrocode}
%    \begin{macrocode}
%</test1>
%    \end{macrocode}
%
%    \begin{macrocode}
%<*test2>
\NeedsTeXFormat{LaTeX2e}
\makeatletter
\catcode`\@=11 %
\def\RestoreCatcodes{}
\count@=0 %
\loop
  \edef\RestoreCatcodes{%
    \RestoreCatcodes
    \catcode\the\count@=\the\catcode\count@\relax
  }%
\ifnum\count@<255 %
  \advance\count@\@ne
\repeat

\def\RangeCatcodeInvalid#1#2{%
  \count@=#1\relax
  \loop
    \catcode\count@=15 %
  \ifnum\count@<#2\relax
    \advance\count@\@ne
  \repeat
}
\def\Test#1{%
  \RangeCatcodeInvalid{0}{47}%
  \RangeCatcodeInvalid{58}{64}%
  \RangeCatcodeInvalid{91}{96}%
  \RangeCatcodeInvalid{123}{255}%
  \catcode`\@=12 %
  \catcode`\\=0 %
  \catcode`\{=1 %
  \catcode`\}=2 %
  \catcode`\#=6 %
  \catcode`\[=12 %
  \catcode`\]=12 %
  \catcode`\%=14 %
  \catcode`\ =10 %
  \catcode13=5 %
  #1\relax
  \RestoreCatcodes
}
\Test{\RequirePackage{kvoptions-patch}}%
\Test{\RequirePackage{kvoptions}}%
\csname @@end\endcsname
%</test2>
%    \end{macrocode}
%
%    \begin{macrocode}
%<*test3>
\NeedsTeXFormat{LaTeX2e}
\makeatletter
\RequirePackage{kvoptions}[2016/05/16]
\def\msg#{\immediate\write16}
\define@key{testfamily}{testkey}{%
  \msg{[testfamily/testkey/#1]}%
}
\define@key{testfamily}{testdefaultkey}[testdefault]{%
  \msg{[testfamily/testdefaultkey/#1]}%
}
\AddToKeyvalOption{testfamily}{testkey}{%
  \msg{[addition/#1]}%
}
\AddToKeyvalOption{testfamily}{testdefaultkey}{%
  \msg{[addition/#1]}%
}
\setkeys{testfamily}{%
  testkey=testA,%
  testdefaultkey=testB,%
  testdefaultkey,%
}
\SetupKeyvalOptions{%
  family=testfamily%
}
\AddToKeyvalOption*{testkey}{%
  \msg{[star addition/#1]}%
}
\AddToKeyvalOption*{testdefaultkey}{%
  \msg{[star addition/#1]}%
}
\setkeys{testfamily}{%
  testkey=testA,%
  testdefaultkey=testB,%
  testdefaultkey,%
}
\@@end
%</test3>
%    \end{macrocode}
%
%    \begin{macrocode}
%<*test4pkg>
\NeedsTeXFormat{LaTeX2e}
\ProvidesPackage{kvoptions-test4}[2016/05/16 package for testing]
\RequirePackage{kvoptions}[2016/05/16]
\SetupKeyvalOptions{%
  family=FOO,%
  prefix=foo,%
  setkeys=\kvsetkeys,%
}
\DeclareStringOption{str}
\define@key{FOO}{set}{%
  \setkeys{BAR}{strbar={#1}}%
}
\define@key{BAR}{strbar}{%
  \def\foostr{[BAR:#1]}%
}
\ProcessKeyvalOptions*
%</test4pkg>
%    \end{macrocode}
%    \begin{macrocode}
%<*test4>
\NeedsTeXFormat{LaTeX2e}
\ProvidesFile{kvoptions-test4.tex}[2016/05/16 test file]
\RequirePackage[%
  str=A,set=B,str=C,%
]{kvoptions-test4}[2016/05/16]
\def\TestExpected{C}
\ifx\foostr\TestExpected
  \typeout{* Test ok.}%
\else
  \typeout{* Result: [\foostr]}%
  \typeout{* Expected: [\TestExpected]}%
  \errmessage{Test failed!}%
\fi
\csname @@end\endcsname\end
%</test4>
%    \end{macrocode}
%
% \section{Installation}
%
% \subsection{Download}
%
% \paragraph{Package.} This package is available on
% CTAN\footnote{\CTANpkg{kvoptions}}:
% \begin{description}
% \item[\CTAN{macros/latex/contrib/oberdiek/kvoptions.dtx}] The source file.
% \item[\CTAN{macros/latex/contrib/oberdiek/kvoptions.pdf}] Documentation.
% \end{description}
%
%
% \paragraph{Bundle.} All the packages of the bundle `oberdiek'
% are also available in a TDS compliant ZIP archive. There
% the packages are already unpacked and the documentation files
% are generated. The files and directories obey the TDS standard.
% \begin{description}
% \item[\CTANinstall{install/macros/latex/contrib/oberdiek.tds.zip}]
% \end{description}
% \emph{TDS} refers to the standard ``A Directory Structure
% for \TeX\ Files'' (\CTAN{tds/tds.pdf}). Directories
% with \xfile{texmf} in their name are usually organized this way.
%
% \subsection{Bundle installation}
%
% \paragraph{Unpacking.} Unpack the \xfile{oberdiek.tds.zip} in the
% TDS tree (also known as \xfile{texmf} tree) of your choice.
% Example (linux):
% \begin{quote}
%   |unzip oberdiek.tds.zip -d ~/texmf|
% \end{quote}
%
% \paragraph{Script installation.}
% Check the directory \xfile{TDS:scripts/oberdiek/} for
% scripts that need further installation steps.
% Package \xpackage{attachfile2} comes with the Perl script
% \xfile{pdfatfi.pl} that should be installed in such a way
% that it can be called as \texttt{pdfatfi}.
% Example (linux):
% \begin{quote}
%   |chmod +x scripts/oberdiek/pdfatfi.pl|\\
%   |cp scripts/oberdiek/pdfatfi.pl /usr/local/bin/|
% \end{quote}
%
% \subsection{Package installation}
%
% \paragraph{Unpacking.} The \xfile{.dtx} file is a self-extracting
% \docstrip\ archive. The files are extracted by running the
% \xfile{.dtx} through \plainTeX:
% \begin{quote}
%   \verb|tex kvoptions.dtx|
% \end{quote}
%
% \paragraph{TDS.} Now the different files must be moved into
% the different directories in your installation TDS tree
% (also known as \xfile{texmf} tree):
% \begin{quote}
% \def\t{^^A
% \begin{tabular}{@{}>{\ttfamily}l@{ $\rightarrow$ }>{\ttfamily}l@{}}
%   kvoptions.sty & tex/latex/oberdiek/kvoptions.sty\\
%   kvoptions-patch.sty & tex/latex/oberdiek/kvoptions-patch.sty\\
%   kvoptions.pdf & doc/latex/oberdiek/kvoptions.pdf\\
%   example-mycolorsetup.sty & doc/latex/oberdiek/example-mycolorsetup.sty\\
%   test/kvoptions-test1.tex & doc/latex/oberdiek/test/kvoptions-test1.tex\\
%   test/kvoptions-test2.tex & doc/latex/oberdiek/test/kvoptions-test2.tex\\
%   test/kvoptions-test3.tex & doc/latex/oberdiek/test/kvoptions-test3.tex\\
%   test/kvoptions-test4.tex & doc/latex/oberdiek/test/kvoptions-test4.tex\\
%   test/kvoptions-test4.sty & doc/latex/oberdiek/test/kvoptions-test4.sty\\
%   kvoptions.dtx & source/latex/oberdiek/kvoptions.dtx\\
% \end{tabular}^^A
% }^^A
% \sbox0{\t}^^A
% \ifdim\wd0>\linewidth
%   \begingroup
%     \advance\linewidth by\leftmargin
%     \advance\linewidth by\rightmargin
%   \edef\x{\endgroup
%     \def\noexpand\lw{\the\linewidth}^^A
%   }\x
%   \def\lwbox{^^A
%     \leavevmode
%     \hbox to \linewidth{^^A
%       \kern-\leftmargin\relax
%       \hss
%       \usebox0
%       \hss
%       \kern-\rightmargin\relax
%     }^^A
%   }^^A
%   \ifdim\wd0>\lw
%     \sbox0{\small\t}^^A
%     \ifdim\wd0>\linewidth
%       \ifdim\wd0>\lw
%         \sbox0{\footnotesize\t}^^A
%         \ifdim\wd0>\linewidth
%           \ifdim\wd0>\lw
%             \sbox0{\scriptsize\t}^^A
%             \ifdim\wd0>\linewidth
%               \ifdim\wd0>\lw
%                 \sbox0{\tiny\t}^^A
%                 \ifdim\wd0>\linewidth
%                   \lwbox
%                 \else
%                   \usebox0
%                 \fi
%               \else
%                 \lwbox
%               \fi
%             \else
%               \usebox0
%             \fi
%           \else
%             \lwbox
%           \fi
%         \else
%           \usebox0
%         \fi
%       \else
%         \lwbox
%       \fi
%     \else
%       \usebox0
%     \fi
%   \else
%     \lwbox
%   \fi
% \else
%   \usebox0
% \fi
% \end{quote}
% If you have a \xfile{docstrip.cfg} that configures and enables \docstrip's
% TDS installing feature, then some files can already be in the right
% place, see the documentation of \docstrip.
%
% \subsection{Refresh file name databases}
%
% If your \TeX~distribution
% (\teTeX, \mikTeX, \dots) relies on file name databases, you must refresh
% these. For example, \teTeX\ users run \verb|texhash| or
% \verb|mktexlsr|.
%
% \subsection{Some details for the interested}
%
% \paragraph{Attached source.}
%
% The PDF documentation on CTAN also includes the
% \xfile{.dtx} source file. It can be extracted by
% AcrobatReader 6 or higher. Another option is \textsf{pdftk},
% e.g. unpack the file into the current directory:
% \begin{quote}
%   \verb|pdftk kvoptions.pdf unpack_files output .|
% \end{quote}
%
% \paragraph{Unpacking with \LaTeX.}
% The \xfile{.dtx} chooses its action depending on the format:
% \begin{description}
% \item[\plainTeX:] Run \docstrip\ and extract the files.
% \item[\LaTeX:] Generate the documentation.
% \end{description}
% If you insist on using \LaTeX\ for \docstrip\ (really,
% \docstrip\ does not need \LaTeX), then inform the autodetect routine
% about your intention:
% \begin{quote}
%   \verb|latex \let\install=y% \iffalse meta-comment
%
% File: kvoptions.dtx
% Version: 2016/05/16 v3.12
% Info: Key value format for package options
%
% Copyright (C)
%    2004, 2006, 2007, 2009-2011 Heiko Oberdiek
%    2016-2019 Oberdiek Package Support Group
%    https://github.com/ho-tex/oberdiek/issues
%
% This work may be distributed and/or modified under the
% conditions of the LaTeX Project Public License, either
% version 1.3c of this license or (at your option) any later
% version. This version of this license is in
%    https://www.latex-project.org/lppl/lppl-1-3c.txt
% and the latest version of this license is in
%    https://www.latex-project.org/lppl.txt
% and version 1.3 or later is part of all distributions of
% LaTeX version 2005/12/01 or later.
%
% This work has the LPPL maintenance status "maintained".
%
% The Current Maintainers of this work are
% Heiko Oberdiek and the Oberdiek Package Support Group
% https://github.com/ho-tex/oberdiek/issues
%
% This work consists of the main source file kvoptions.dtx
% and the derived files
%    kvoptions.sty, kvoptions.pdf, kvoptions.ins, kvoptions.drv,
%    kvoptions-patch.sty, example-mycolorsetup.sty,
%    kvoptions-test1.tex, kvoptions-test2.tex,
%    kvoptions-test3.tex, kvoptions-test4.tex,
%    kvoptions-test4.sty.
%
% Distribution:
%    CTAN:macros/latex/contrib/oberdiek/kvoptions.dtx
%    CTAN:macros/latex/contrib/oberdiek/kvoptions.pdf
%
% Unpacking:
%    (a) If kvoptions.ins is present:
%           tex kvoptions.ins
%    (b) Without kvoptions.ins:
%           tex kvoptions.dtx
%    (c) If you insist on using LaTeX
%           latex \let\install=y% \iffalse meta-comment
%
% File: kvoptions.dtx
% Version: 2016/05/16 v3.12
% Info: Key value format for package options
%
% Copyright (C)
%    2004, 2006, 2007, 2009-2011 Heiko Oberdiek
%    2016-2019 Oberdiek Package Support Group
%    https://github.com/ho-tex/oberdiek/issues
%
% This work may be distributed and/or modified under the
% conditions of the LaTeX Project Public License, either
% version 1.3c of this license or (at your option) any later
% version. This version of this license is in
%    https://www.latex-project.org/lppl/lppl-1-3c.txt
% and the latest version of this license is in
%    https://www.latex-project.org/lppl.txt
% and version 1.3 or later is part of all distributions of
% LaTeX version 2005/12/01 or later.
%
% This work has the LPPL maintenance status "maintained".
%
% The Current Maintainers of this work are
% Heiko Oberdiek and the Oberdiek Package Support Group
% https://github.com/ho-tex/oberdiek/issues
%
% This work consists of the main source file kvoptions.dtx
% and the derived files
%    kvoptions.sty, kvoptions.pdf, kvoptions.ins, kvoptions.drv,
%    kvoptions-patch.sty, example-mycolorsetup.sty,
%    kvoptions-test1.tex, kvoptions-test2.tex,
%    kvoptions-test3.tex, kvoptions-test4.tex,
%    kvoptions-test4.sty.
%
% Distribution:
%    CTAN:macros/latex/contrib/oberdiek/kvoptions.dtx
%    CTAN:macros/latex/contrib/oberdiek/kvoptions.pdf
%
% Unpacking:
%    (a) If kvoptions.ins is present:
%           tex kvoptions.ins
%    (b) Without kvoptions.ins:
%           tex kvoptions.dtx
%    (c) If you insist on using LaTeX
%           latex \let\install=y% \iffalse meta-comment
%
% File: kvoptions.dtx
% Version: 2016/05/16 v3.12
% Info: Key value format for package options
%
% Copyright (C)
%    2004, 2006, 2007, 2009-2011 Heiko Oberdiek
%    2016-2019 Oberdiek Package Support Group
%    https://github.com/ho-tex/oberdiek/issues
%
% This work may be distributed and/or modified under the
% conditions of the LaTeX Project Public License, either
% version 1.3c of this license or (at your option) any later
% version. This version of this license is in
%    https://www.latex-project.org/lppl/lppl-1-3c.txt
% and the latest version of this license is in
%    https://www.latex-project.org/lppl.txt
% and version 1.3 or later is part of all distributions of
% LaTeX version 2005/12/01 or later.
%
% This work has the LPPL maintenance status "maintained".
%
% The Current Maintainers of this work are
% Heiko Oberdiek and the Oberdiek Package Support Group
% https://github.com/ho-tex/oberdiek/issues
%
% This work consists of the main source file kvoptions.dtx
% and the derived files
%    kvoptions.sty, kvoptions.pdf, kvoptions.ins, kvoptions.drv,
%    kvoptions-patch.sty, example-mycolorsetup.sty,
%    kvoptions-test1.tex, kvoptions-test2.tex,
%    kvoptions-test3.tex, kvoptions-test4.tex,
%    kvoptions-test4.sty.
%
% Distribution:
%    CTAN:macros/latex/contrib/oberdiek/kvoptions.dtx
%    CTAN:macros/latex/contrib/oberdiek/kvoptions.pdf
%
% Unpacking:
%    (a) If kvoptions.ins is present:
%           tex kvoptions.ins
%    (b) Without kvoptions.ins:
%           tex kvoptions.dtx
%    (c) If you insist on using LaTeX
%           latex \let\install=y\input{kvoptions.dtx}
%        (quote the arguments according to the demands of your shell)
%
% Documentation:
%    (a) If kvoptions.drv is present:
%           latex kvoptions.drv
%    (b) Without kvoptions.drv:
%           latex kvoptions.dtx; ...
%    The class ltxdoc loads the configuration file ltxdoc.cfg
%    if available. Here you can specify further options, e.g.
%    use A4 as paper format:
%       \PassOptionsToClass{a4paper}{article}
%
%    Programm calls to get the documentation (example):
%       pdflatex kvoptions.dtx
%       makeindex -s gind.ist kvoptions.idx
%       pdflatex kvoptions.dtx
%       makeindex -s gind.ist kvoptions.idx
%       pdflatex kvoptions.dtx
%
% Installation:
%    TDS:tex/latex/oberdiek/kvoptions.sty
%    TDS:tex/latex/oberdiek/kvoptions-patch.sty
%    TDS:doc/latex/oberdiek/kvoptions.pdf
%    TDS:doc/latex/oberdiek/example-mycolorsetup.sty
%    TDS:doc/latex/oberdiek/test/kvoptions-test1.tex
%    TDS:doc/latex/oberdiek/test/kvoptions-test2.tex
%    TDS:doc/latex/oberdiek/test/kvoptions-test3.tex
%    TDS:doc/latex/oberdiek/test/kvoptions-test4.tex
%    TDS:doc/latex/oberdiek/test/kvoptions-test4.sty
%    TDS:source/latex/oberdiek/kvoptions.dtx
%
%<*ignore>
\begingroup
  \catcode123=1 %
  \catcode125=2 %
  \def\x{LaTeX2e}%
\expandafter\endgroup
\ifcase 0\ifx\install y1\fi\expandafter
         \ifx\csname processbatchFile\endcsname\relax\else1\fi
         \ifx\fmtname\x\else 1\fi\relax
\else\csname fi\endcsname
%</ignore>
%<*install>
\input docstrip.tex
\Msg{************************************************************************}
\Msg{* Installation}
\Msg{* Package: kvoptions 2016/05/16 v3.12 Key value format for package options (HO)}
\Msg{************************************************************************}

\keepsilent
\askforoverwritefalse

\let\MetaPrefix\relax
\preamble

This is a generated file.

Project: kvoptions
Version: 2016/05/16 v3.12

Copyright (C)
   2004, 2006, 2007, 2009-2011 Heiko Oberdiek
   2016-2019 Oberdiek Package Support Group

This work may be distributed and/or modified under the
conditions of the LaTeX Project Public License, either
version 1.3c of this license or (at your option) any later
version. This version of this license is in
   https://www.latex-project.org/lppl/lppl-1-3c.txt
and the latest version of this license is in
   https://www.latex-project.org/lppl.txt
and version 1.3 or later is part of all distributions of
LaTeX version 2005/12/01 or later.

This work has the LPPL maintenance status "maintained".

The Current Maintainers of this work are
Heiko Oberdiek and the Oberdiek Package Support Group
https://github.com/ho-tex/oberdiek/issues


This work consists of the main source file kvoptions.dtx
and the derived files
   kvoptions.sty, kvoptions.pdf, kvoptions.ins, kvoptions.drv,
   kvoptions-patch.sty, example-mycolorsetup.sty,
   kvoptions-test1.tex, kvoptions-test2.tex,
   kvoptions-test3.tex, kvoptions-test4.tex,
   kvoptions-test4.sty.

\endpreamble
\let\MetaPrefix\DoubleperCent

\generate{%
  \file{kvoptions.ins}{\from{kvoptions.dtx}{install}}%
  \file{kvoptions.drv}{\from{kvoptions.dtx}{driver}}%
  \usedir{tex/latex/oberdiek}%
  \file{kvoptions.sty}{\from{kvoptions.dtx}{package}}%
  \file{kvoptions-patch.sty}{\from{kvoptions.dtx}{patch}}%
  \usedir{doc/latex/oberdiek}%
  \file{example-mycolorsetup.sty}{\from{kvoptions.dtx}{example}}%
%  \usedir{doc/latex/oberdiek/test}%
%  \file{kvoptions-test1.tex}{\from{kvoptions.dtx}{test1}}%
%  \file{kvoptions-test2.tex}{\from{kvoptions.dtx}{test2}}%
%  \file{kvoptions-test3.tex}{\from{kvoptions.dtx}{test3}}%
%  \file{kvoptions-test4.tex}{\from{kvoptions.dtx}{test4}}%
%  \file{kvoptions-test4.sty}{\from{kvoptions.dtx}{test4pkg}}%
  \nopreamble
  \nopostamble
%  \usedir{source/latex/oberdiek/catalogue}%
%  \file{kvoptions.xml}{\from{kvoptions.dtx}{catalogue}}%
}

\catcode32=13\relax% active space
\let =\space%
\Msg{************************************************************************}
\Msg{*}
\Msg{* To finish the installation you have to move the following}
\Msg{* files into a directory searched by TeX:}
\Msg{*}
\Msg{*     kvoptions.sty, kvoptions-patch.sty}
\Msg{*}
\Msg{* To produce the documentation run the file `kvoptions.drv'}
\Msg{* through LaTeX.}
\Msg{*}
\Msg{* Happy TeXing!}
\Msg{*}
\Msg{************************************************************************}

\endbatchfile
%</install>
%<*ignore>
\fi
%</ignore>
%<*driver>
\NeedsTeXFormat{LaTeX2e}
\ProvidesFile{kvoptions.drv}%
  [2016/05/16 v3.12 Key value format for package options (HO)]%
\documentclass{ltxdoc}
\usepackage{holtxdoc}[2011/11/22]
\begin{document}
  \DocInput{kvoptions.dtx}%
\end{document}
%</driver>
% \fi
%
%
% \CharacterTable
%  {Upper-case    \A\B\C\D\E\F\G\H\I\J\K\L\M\N\O\P\Q\R\S\T\U\V\W\X\Y\Z
%   Lower-case    \a\b\c\d\e\f\g\h\i\j\k\l\m\n\o\p\q\r\s\t\u\v\w\x\y\z
%   Digits        \0\1\2\3\4\5\6\7\8\9
%   Exclamation   \!     Double quote  \"     Hash (number) \#
%   Dollar        \$     Percent       \%     Ampersand     \&
%   Acute accent  \'     Left paren    \(     Right paren   \)
%   Asterisk      \*     Plus          \+     Comma         \,
%   Minus         \-     Point         \.     Solidus       \/
%   Colon         \:     Semicolon     \;     Less than     \<
%   Equals        \=     Greater than  \>     Question mark \?
%   Commercial at \@     Left bracket  \[     Backslash     \\
%   Right bracket \]     Circumflex    \^     Underscore    \_
%   Grave accent  \`     Left brace    \{     Vertical bar  \|
%   Right brace   \}     Tilde         \~}
%
% \GetFileInfo{kvoptions.drv}
%
% \title{The \xpackage{kvoptions} package}
% \date{2016/05/16 v3.12}
% \author{Heiko Oberdiek\thanks
% {Please report any issues at \url{https://github.com/ho-tex/oberdiek/issues}}}
%
% \maketitle
%
% \begin{abstract}
% This package is intended for package authors who want to
% use options in key value format for their package options.
% \end{abstract}
%
% \tableofcontents
%
% \def\M#1{\texttt{\{}\meta{#1}\texttt{\}}}
% \def\O#1{\texttt{[}\meta{#1}\texttt{]}}
%
% \section{Introduction}
%
% First I want to recommend the very good review article
% ``A guide to key-value methods'' by Joseph Wright \cite{tb94wright}.
% It introduces the different key-value packages and compares them.
%
% \subsection{The beginning}
%
% This package \xpackage{kvoptions} addresses class or package writers
% that want to allow their users to specify options
% as key value pairs, e.g.:
% \begin{quote}
% |\documentclass[verbose=false,name=me]{myclass}|\\
% |\usepackage[format=print]{mylayout}|
% \end{quote}
% Prominent example is package \xpackage{hyperref},
% probably the first package that offers this service.
% It's \cs{ProcessOptionsWithKV} is often copied und
% used in other packages, e.g. package \xpackage{helvet}
% that uses this interface for its option \xoption{scaled}.
%
% However copying code is not the most modern software
% development technique. And \xpackage{hyperref}'s
% code for \cs{ProcessOptionsWithKV} was changed
% to fix bugs. The version used in other packages
% depends on the time of copying and the awareness
% of \xpackage{hyperref}'s changes.
% Now the code is sourced out into this package and
% available for other package or class writers.
%
% \subsection{Overview}
%
% Package \xpackage{kvoptions} connects package
% \xpackage{\emph{k}ey\emph{v}al}
% with \LaTeX's package and class \emph{options}:
% \begin{center}
% \def\T#1{\bfseries #1}
% \renewcommand{\arraystretch}{1.5}
% \begin{tabular}{l|l|l}
%   \T{Package \xpackage{keyval}}
%   & \T{Package \xpackage{kvoptions}}
%   & \T{\LaTeX\ kernel}
%   \\
% \hline
%     \cs{define@key}
%   &
%     \renewcommand*{\arraystretch}{1}^^A
%     \begin{tabular}[t]{@{}l@{}}
%       \cs{DeclareVoidOption}\\
%       \cs{DeclareStringOption}\\
%       \cs{DeclareBoolOption}\\
%       \cs{DeclareComplementaryOption}\\
%       \cs{DisableKeyvalOption}
%     \end{tabular}
%   & \cs{DeclareOption}
%   \\
% \hline
%   & \cs{DeclareDefaultOption}
%   & \cs{DeclareOption*}
%   \\
% \hline
%   & \cs{ProcessKeyvalOptions}
%   & \cs{ProcessOptions*}
%   \\
% \hline
%   & Option \xoption{patch}
%   &
%     \renewcommand*{\arraystretch}{1}^^A
%     \begin{tabular}[t]{@{}l@{}}
%       Class/package\\
%       option system
%     \end{tabular}
%   \\
% \hline
%   & \cs{SetupKeyvalOptions} &
%   \\
% \end{tabular}
% \end{center}
%
% \hypersetup{bookmarksopenlevel=3}
% \section{Usage}
%
% \subsection{Process options}
%
% \subsubsection{\cs{ProcessKeyvalOptions}}
%
% \begin{declcs}{ProcessKeyvalOptions}\,\M{family}\\
%   \cs{ProcessKeyvalOptions} \*
% \end{declcs}
% This command evaluates the global or local options of
% the package that are defined with \xpackage{keyval}'s
% interface within the family \meta{family}. It acts the
% same way as \LaTeX's \cs{ProcessOptions*}. In a package
% unknown global options are ignored, in a class they
% are added to the unknown option list. The known global
% options and all local options are passed to \xpackage{keyval}'s
% \cs{setkeys} command for executing the options.
% Unknown options are reported to the user by an error.
%
% If the family name happens to be the same as the name
% of the package or class where \cs{ProcessKeyvalOptions}
% is used or the family name has previously been setup by
% \cs{SetupKeyvalOptions}, then \cs{ProcessKeyvalOptions}
% knows the family name already and you can use the star form
% without mandatory argument.
%
% \subsubsection{\cs{ProcessLocalKeyvalOptions}}
%
% \begin{declcs}{ProcessLocalKeyvalOptions}\,\M{family}\\
%   \cs{ProcessLocalKeyvalOptions} \*
% \end{declcs}
% This macro has the same syntax and works similar as
% \cs{ProcessKeyvalOptions}. However it ignores
% global options and only processes the local package options.
% Therefore it only can be used inside a package.
% An error is thrown, if it is used inside a class.
%
% Neither of the following macros are necessary for
% \cs{ProcessKeyvalOptions{}}. They just help the package/class
% author in common tasks.
%
% \subsubsection{\cs{SetupKeyvalOptions}}\label{setup}
%
% \begin{declcs}{SetupKeyvalOptions}\,\texttt{\{}\\
%   \hspace{1.5em}|family|\,|=|\,\meta{family}|,|\\
%   \hspace{1.5em}|prefix|\,|=|\,\meta{prefix}\\
%   \hspace{1.5em}|setkeys|\,|=|\,\meta{setkeys command}\\
%   \texttt{\}}
% \end{declcs}
% This command allows to configure the default assumptions
% that are based on the current package or class name.
% \LaTeX\ remembers this name in \cs{@currname}. The syntax
% description of the default looks a little weird, therefor
% an example is given for a package or class named |foobar|.
% \begin{quote}
% \begin{tabular}{@{}l|ll|l@{}}
%   Key & Default & (example) & Used by\\
% \hline
%   \rule{0pt}{1.2\ht\csname @arstrutbox\endcsname}^^A
%   |family| & \meta{\cs{@currname}} & (|foobar|)
%       & \cs{ProcessKeyvalOptions*} \\
%     &&& \cs{DeclareBoolOption} \\
%     &&& \cs{DeclareStringOption} \\
% \hline
%   \rule{0pt}{1.2\ht\csname @arstrutbox\endcsname}^^A
%   |prefix| & \meta{\cs{@currname}}|@| & (|foobar@|)
%       & \cs{DeclareBoolOption} \\
%     &&& \cs{DeclareStringOption} \\
%     &&& \cs{DeclareVoidOption} \\
% \hline
%   \rule{0pt}{1.2\ht\csname @arstrutbox\endcsname}^^A
%   |setkeys| & \cs{setkeys} & (\cs{kvsetkeys})
%       & \cs{ProcessKeyvalOptions} \\
%     &&& \cs{ProcessLocalKeyvalOptions}
% \end{tabular}
% \end{quote}
% Key |setkeys| was added in version 3.9. The original
% \cs{setkeys} of package \xpackage{keyval} is not reentrant.
% If an option is processed by this \cs{setkeys}, then
% the option should not call \cs{setkeys} again with a
% different family. Otherwise the next options of the
% first \cs{setkeys} call are processed with the wrong
% family. With key |setkeys| the macro \cs{kvsetkeys}
% can be set that does not have the problem of the
% original \cs{setkeys} of package {keyval}.
%
% Probably \cs{setkeys} of package \xpackage{xkeyval}
% is safe in this respect. But I haven't made a full
% analysis. At least it does not have the problem
% of the original \cs{setkeys}.
%
% \subsection{Option declarations}
%
% The options for \cs{ProcessKeyvalOptions} are defined
% by \xpackage{keyval}'s \cs{define@key}.
% Common purposes of such keys are boolean switches, they
% enable or disable something. Or they store a name or some kind of
% string in a macro. The following commands help the user. He
% declares what he wants and \xpackage{kvoptions} take care of
% the key definition, resource allocation and initialization.
%
% In order to avoid name clashes of macro names, internal
% commands are prefixed. Both the prefix and the family
% name for the defined keys can be configured by
% \cs{SetupKeyvalOptions}.
%
% \subsubsection{\cs{DeclareStringOption}}
%
% \begin{declcs}{DeclareStringOption}\,\O{init}\,\M{key}\,\O{default}
% \end{declcs}
% A macro is created that remembers the value of the key \meta{key}.
% The name of the macro consists of the option name \meta{key} that
% is prefixed by the prefix (see \ref{setup}).
% The initial contents of the
% macro can be given by the first optional argument \meta{init}.
% The default is empty.
%
% The the option \meta{key} is defined. The option code
% just stores its value in the macro. If the optional argument
% at the end of \cs{DeclareStringOption} is given, then option
% \meta{key} is defined with the default \meta{default}.
%
% Example for a package with the following two lines:
% \begin{quote}
% |\ProvidesPackage{foobar}|\\
% |\DeclareStringOption[me]{name}|
% \end{quote}
% Then \cs{DeclareStringOption} defines the macro with content |me|,
% note \LaTeX\ complains if the name of the macro already exists:
% \begin{quote}
% |\newcommand*{\foobar@name}{me}|
% \end{quote}
% The option definition is similar to:
% \begin{quote}
% |\define@key{foobar}{name}{%|\\
% |  \renewcommand*{\foobar@name}{#1}%|\\
% |}|
% \end{quote}
%
% \subsubsection{\cs{DeclareBoolOption}}
%
% \begin{declcs}{DeclareBoolOption}\,\O{init}\,\M{key}
% \end{declcs}
% A boolean switch is generated, initialized by value \meta{init}
% and the corresponding key \meta{key} is defined. If the
% initialization value is not given, |false| is used as default.
%
% The internal actions of \cs{DeclareBoolOption} are shown below.
% The example is given for a package author who has the following
% two lines in his package/class:
% \begin{quote}
% |\ProvidesPackage{foobar}|\\
% |\DeclareBoolOption{verbose}|
% \end{quote}
% First a new switch is created:
% \begin{quote}
%   |\newif\iffoobar@verbose|
% \end{quote}
% and initialized:
% \begin{quote}
%   |\foobar@verbosefalse|
% \end{quote}
% Finally the key is defined:
% \begin{quote}
%   |\define@key{foobar}{verbose}[true]{|\dots|}|
% \end{quote}
% The option code configures the boolean option in the following way:
% If the author specifies |true| or |false| then the switch is
% turned on or off respectivly. Also the option can be given without
% explicit value. Then the switch is enabled.
% Other values are reported as errors.
%
% Now the switch is ready to use in the package/class, e.g.:
% \begin{quote}
%   |\iffoobar@verbose|\\
%   \qquad\textit{\% print verbose message}\\
%   |\else|\\
%   \qquad\textit{\% be quiet}\\
%   |\fi|
% \end{quote}
% Users of package \cs{ifthen} can use the switch as
% boolean:
% \begin{quote}
%   |\boolean{foobar@verbose}|
% \end{quote}
%
% \subsubsection{\cs{DeclareComplementaryOption}}
%
% \begin{declcs}{DeclareComplementaryOption}\,\M{key}\,\M{parent}
% \end{declcs}
% Sometimes contrasting names are used to characterize the
% two states of a boolean switch, for example \xoption{draft}
% vs.\@ \xoption{final}. Both options behave like boolean
% options but they do not need two different switches, they
% should share one. \cs{DeclareComplementaryOption} allows this.
% The option \meta{key} shares the switch of option \meta{parent}.
% Example:
% \begin{quote}
%   |\DeclareBoolOption{draft}|\\
%   |\DeclareComplementaryOption{final}{draft}|
% \end{quote}
% Then |final| sets the switch of \xoption{draft} to |false|, and
% |final=false| enables the \xoption{draft} switch.
%
% \subsubsection{\cs{DeclareVoidOption}}
%
% \begin{declcs}{DeclareVoidOption}\,\M{key}\,\M{code}
% \end{declcs}
% \cs{ProcessKeyvalOptions} can be extended to recognize
% options that are declared in traditional way by \cs{DeclareOption}.
% But in case of the error that the user specifies a value, then
% this option would not recognized as key value option because
% of \cs{DeclareOption} and not detected as traditional option
% because of the value part. The user would get an unknown option
% error, difficult to understand.
%
% \cs{DeclareVoidOption} solves
% this problem. It defines the option \meta{key} as key value option.
% If the user specifies a value, a warning is given and the
% value is ignored.
%
% The code part \meta{code} is stored in a macro.
% The name of the macro consists of the option name \meta{key} that
% is prefixed by the prefix (see \ref{setup}).
% If the option is set, the macro will be executed. During the
% execution \cs{CurrentOption} is available with the current key name.
%
% \subsubsection{\cs{DeclareDefaultOption}}
%
% \begin{declcs}{DeclareDefaultOption}\,\M{code}
% \end{declcs}
% \begin{sloppypar}
% This command does not define a specific key, it is the equivalent
% to \LaTeX's \cs{DeclareOption*}. It allows the specification
% of a default action \meta{code} that is invoked if an unknown option
% is found.
% While \meta{code} is called,
% macro \cs{CurrentOption} contains the current option string.
% In addition \cs{CurrentOptionValue} contains the value part
% if the option string is parsable as key value pair, otherwise
% it is \cs{relax}. \cs{CurrentOptionKey} contains the key of
% the key value pair, or the whole option string, if it misses
% the equal sign.
% \end{sloppypar}
%
% Inside packages typical default actions are to pass unknown
% options to another package. Or an error message can be thrown
% by \cs{@unknownoptionerror}. This is the original error
% message that \LaTeX\ gives for unkown package options.
% This error message is easier to understand for the user
% as the error message from package \xpackage{keyval} that
% is given otherwise.
%
% A Class ignores unknown options and puts them on the
% unused option list. Let \LaTeX\ do the job and just
% call \cs{OptionNotUsed}. Or the options can be passed to
% another class that is later loaded.
%
% \subsubsection{Local options}
%
% \begin{declcs}{DeclareLocalOption} \M{option}\\
%   \cs{DeclareLocalOptions} \M{option list}
% \end{declcs}
% Both macros mark package options as local options. That means
% that they are ignored by \cs{ProcessKeyvalOptions} if they are given
% as global options. \cs{DeclareLocalOptions} takes one option,
% \cs{DeclareLocalOptions} expects a comma separated list of options.
%
% \subsubsection{Dynamic options}
%
% Options of \LaTeX's package/class system are cleared
% in \cs{ProcessOptions}. They modify the static model
% of a package. For example, depending on option \xoption{bookmarks}
% package \xpackage{hyperref} loads differently.
%
% Options, however,
% defined by \xpackage{keyval}'s \cs{define@key} remain
% defined, if the options are processed by \cs{setkeys}.
% Therefore these options can also be used to model
% the dynamic behaviour of a package. For example, in
% \xpackage{hyperref} the link colors can be changed
% everywhere until the end in |\end{document}|.
%
% However package \xpackage{color} that adds color support
% is necessary and it
% cannot be loaded after |\begin{document}|. Option \xoption{colorlinks}
% that loads \xpackage{color} should be active until |\begin{document}|
% and die in some way if it is too late for loading packages.
% With \cs{DisableKeyvalOption} the package/class author can
% specify and configure the death of an option and controls
% the life period of the option.
%
% \subsubsection{\cs{DisableKeyvalOption}}
%
% \begin{declcs}{DisableKeyvalOption}\,\O{options}\,\M{family}\,\M{key}
%   \\[0.5ex]
%   \meta{options}:\\
%   \hspace{1.5em}^^A
%   \begin{tabular}[t]{@{}l@{ \texttt{=} }l@{\qquad}>{default: }l@{}}
%     |action| & |undef|, |warning|, |error|, or |ignore|
%     & |undef|
%     \\
%     \multicolumn{1}{@{}l}{|global| or |local|} & & |global|
%     \\
%     |package| or |class| & \meta{name}
%     \\
%   \end{tabular}^^A
% \end{declcs}
% \cs{DisableKeyvalOption} can be called to mark the end when
% the option \meta{key} is no longer useful. The behaviour of
% an option after its death can be configured by \xoption{action}:
% \begin{description}
% \item[\texttt{undef}:] The option will be undefined,
%   If it is called, \cs{setkeys} reports an error because
%   of unknown key.
% \item[\texttt{error} or \texttt{warning}:] Use of the
%  option will cause an error or warning message.
%  Also these actions require that exclusivly either the
%  package or class name is given in options \xoption{package}
%  or \xoption{class}.
% \item[\texttt{ignore}:] The use of the option will silently
%  be ignored.
% \end{description}
% The option's death can be limited to the end of the current
% group, if option \xoption{local} is given. Default is
% \xoption{global}.
%
% The package/class author can wish the end of the option
% already during the package loading, then he will have
% static behaviour. In case of dynamic options \cs{DisableKeyvalOption}
% can be executed everywhere, also outside the package. Therefore
% the family name and the package/class name is usually unknown for
% \cs{DisableKeyvalOption}. Therefore the argument for the family
% name is mandatory and for some actions the package/class name
% must be provided.
%
% Usually a macro would configure the option death, Example:
%\begin{quote}
%\begin{verbatim}
%\ProvidesPackage{foobar}
%\DeclareBoolOption{color}
%\DeclareStringOption[red]{emphcolor}
%\ProcessKeyvalOptions*
%
%\newcommand*{\foobar@DisableOption}[2]{%
%  \DisableKeyvalueOption[
%    action={#1},
%    package=foobar
%  ]{foobar}{#2}%
%}
%
%\iffoobar@color
%  \RequirePackage{color}
%  \renewcommand*{\emph}[1]{\textcolor{\foobar@emphcolor}{#1}}
%\else
%  % Option emphcolor is not wrong, if we have color support.
%  % otherwise the option has no effect, but we don't want to
%  % remove it. Therefore action 'ignore' is the best choice:
%  \foobar@DisableOption{ignore}{emphcolor}
%\fi
%% No we don't need the option 'color'.
%\foobar@DisableOption{warning}{color}
%
%% With color support option 'emphcolor' will dynamically
%% change the color of \emph statements.
%\end{verbatim}
%\end{quote}
%
% \subsubsection{\cs{AddToKeyvalOption}}
%
% \begin{declcs}{AddToKeyvalOption}\,\M{family}\,\M{key}\,\M{code}\\
%   \cs{AddToKeyvalOption} \*\,\M{key}\,\M{code}
% \end{declcs}
% The code for an existing key \meta{key} of family \meta{family} is
% extended by code \meta{code}. In the starred form the current
% family setting is used, see \cs{ProcessKeyvalOptions*}.
%
% \subsection{Global vs. local options}
%
% Options that are given for \cs{documentclass} are called
% global options. They are known to the class and all packages.
% A package may make use of a global option and marks it as
% used. The advantage for the user is the freedom to specify
% options both in the \cs{documentclass} or \cs{usepackage}
% commands.
%
% However global options are shared with the class options
% and options of all other packages. Thus there can be the
% same option with different semantics for different
% packages and classes. As example, package \xpackage{bookmark}
% knows option \xoption{open} that specifies whether the bookmarks
% are opened or closed initially. It's values are |true| or |false|.
% Since KOMA-Script version 3.00 the KOMA classes also introduces
% option \xoption{open} with values |right| and |any| and a
% complete different meaning.
%
% Such conflicts can be resolved by marking all or part of options
% as local by \cs{DeclareLocalOption} or \cs{DeclareLocalOptions}.
% Then the packages ignores global occurences of these options.
% Package \xpackage{kvoptions} provides two methods:
% \begin{itemize}
% \item \cs{ProcessLocalKeyvalOptions} automatically uses all options
%   as local options. It ignores all global options.
% \item \cs{DeclareLocalOption} or \cs{DeclareLocalOptions} marks
%   options as local options. \cs{ProcessKeyvalOptions} will then
%   ignore global occurences for these local options.
% \end{itemize}
% Since version 1.5 package \xpackage{bookmark} uses the latter
% method. It checks global and local option places for driver options
% and limits all other options as local options. Thus the class
% option \xoption{open} of KOMA-Script is not misread as
% option for package \xpackage{bookmark}.
%
% \subsection{Summary of internal macros}
%
% The \cs{Declare}\texttt{\dots Option} commands define
% macros, additionally to the macros generated by the key definition.
% These macros can be used by the package/class author.
% The name of the macros starts with the prefix \meta{prefix}
% that can be configured by \cs{SetupKeyvalOptions}.
% \begin{center}
% \def\prefix{\cs{}\meta{prefix}\meta{key}}^^A
% \def\Hline{^^A
%   \hline
%   \rule{0pt}{1.2\ht\csname @arstrutbox\endcsname}^^A
% }^^A
% \begin{tabular}{@{}l|>{\ttfamily}l|l@{}}
%   Declare \meta{key} & \textrm{Defined macro} & Description\\
% \Hline
%   \cs{DeclareStringOption}
%    & \prefix & holds the string\\
% \Hline
%   \cs{DeclareBoolOption}
%     & \cs{if}\meta{prefix}\meta{key}& boolean switch\\
%   &\prefix false& disable switch\\
%   &\prefix true& enable switch\\
% \Hline
%   \cs{DeclareComplementaryOption}
%     & \prefix false& enable parent switch\\
%     & \prefix true& disable parent switch\\
% \Hline
%   \cs{DeclareVoidOption}
%   & \prefix & holds the action
% \end{tabular}
% \end{center}
%
% \subsection{\plainTeX}
%
% Package \xpackage{keyval} is also usable in \plainTeX\ with
% the help of file \xfile{miniltx.tex}. Some features of this
% package \xpackage{kvoptions} might also be useful for \plainTeX.
% If \LaTeX\ is not found, \cs{ProcessKeyvalOptions} and option
% \xoption{patch} are disabled. Before using the option declaration
% commands \cs{Declare...Option}, \cs{SetupKeyvalOptions} must be
% used.
%
% \hypersetup{bookmarksopenlevel=1}
%
% \section{Example}
%
% The following example defined a package that serves some
% private color management. A boolean option \xoption{print} enables
% print mode without colors. An option \xoption{emph} redefines
% \cs{emph} to print in the given color. And the driver
% can be specified by option \xoption{driver}.
%
%    \begin{macrocode}
%<*example>
    % Package identification
    % -----------------------
\NeedsTeXFormat{LaTeX2e}
\ProvidesPackage{example-mycolorsetup}[2016/05/16 Managing my colors]

\RequirePackage{ifpdf}
\RequirePackage{kvoptions}

    % Option declarations
    % -------------------

\SetupKeyvalOptions{
  family=MCS,
  prefix=MCS@
}
    % Use a shorter family name and prefix

    % Option print
\DeclareBoolOption{print}
    % is the same as
    % \DeclareBoolOption[false]{print}

    % Option driver
\ifpdf
  \DeclareStringOption[pdftex]{driver}
\else
  \DeclareStringOption[dvips]{driver}
\fi

    % Alternative interface for driver options
\DeclareVoidOption{dvips}{\SetupDriver}
\DeclareVoidOption{dvipdfm}{\SetupDriver}
\DeclareVoidOption{pdftex}{\SetupDriver}
    % In \SetupDriver we take the current option \CurrentOption
    % and pass it to the driver option.
    % The \expandafter commands expand \CurrentOption at the
    % time, when \SetupDriver is executed and \CurrentOption
    % has the correct meaning.
\newcommand*{\SetupDriver}{%
  \expandafter\@SetupDriver\expandafter{\CurrentOption}%
}
\newcommand*{\@SetupDriver}[1]{%
  \setkeys{MCS}{driver={#1}}%
}

    % Option emph
    % An empty value means, we want to have no color for \emph.
    % If the user specifies option emph without value, the red is used.
\DeclareStringOption{emph}[red]
    % is the same as
    % \DeclareStringOption[]{emph}[red]

    % Default option rule
\DeclareDefaultOption{%
  \ifx\CurrentOptionValue\relax
    \PackageWarningNoLine{\@currname}{%
      Unknown option `\CurrentOption'\MessageBreak
      is passed to package `color'%
    }%
    % Pass the option to package color.
    % Again it is better to expand \CurrentOption.
    \expandafter\PassOptionsToPackage
    \expandafter{\CurrentOption}{color}%
  \else
    % Package color does not take options with values.
    % We provide the standard LaTeX error.
    \@unknownoptionerror
  \fi
}

    % Process options
    % ---------------
\ProcessKeyvalOptions*

    % Implementation depending on option values
    % -----------------------------------------
    % Code for print mode
\ifMCS@print
  \PassOptionsToPackage{monochrome}{color}
    % tells package color to use black and white
\fi

\RequirePackage[\MCS@driver]{color}
    % load package color with the correct driver

    % \emph setup
\ifx\MCS@emph\@empty
    % \@empty is a predefined macro with empty contents.
    % the option value of option emph is empty, thus
    % we do not want a redefinition of \emph.
\else
  \renewcommand*{\emph}[1]{%
    \textcolor{\MCS@emph}{#1}%
  }
\fi
%</example>
%    \end{macrocode}
%
% \section{Package options}
%
% The package \xpackage{kvoptions} knows two package options
% \xoption{patch} and \xoption{debugshow}.
% The options of package \xpackage{kvoptions} are intended
% for authors, not for package/class writers. Inside a package
% it is too late for option \xoption{patch} and \xoption{debugshow}
% enables some messages that are perhaps useful for the debugging
% phase. Also \LaTeX\ is unhappy if a package is loaded
% later again with options that are previously not given.
% Thus package and class authors, stay with |\RequirePackage{kvoptions}|
% without options.
%
% Option \xoption{patch} loads package \xpackage{kvoptions-patch}.
%
% \subsection{Package \xpackage{kvoptions-patch}}
%
% \LaTeX's system of package/class options has some severe limitations
% that especially affects the value part if options are used as
% pair of key and value.
% \begin{itemize}
% \item Spaces are removed, regardless where:
%   \begin{quote}
%     |\documentclass[box=0 0 400 600]{article}|
%   \end{quote}
%   Now each package will see |box=00400600| as global option.
% \item In the previous case also braces would not help:
%   \begin{quote}
%     |\documentclass[box={0 0 400 600}]{article}|
%   \end{quote}
%   The result is an error message:
%   \begin{quote}
%     |! LaTeX Error: Missing \begin{document}.|
%   \end{quote}
%   As local option, however, it works if the package
%   knows about key value options (By using this package, for example).
% \item The requirements on robustness are extremly high.
%   \LaTeX\ expands the option. All that will not work as environment
%   name will break also as option. Even a \cs{relax} will generate
%   an error message:
%   \begin{quote}
%     |! Missing \endcsname inserted.|
%   \end{quote}
%   Of course, \LaTeX\ does not use its protecting mechanisms.
%   On contrary \cs{protect} itself will cause errors.
% \item The options are expanded. But perhaps the package will
%   do that, because it has to setup some things before?
%   Example \xpackage{hyperref}:
%   \begin{quote}
%     |\usepackage[pdfauthor=M\"uller]{hyperref}|
%   \end{quote}
%   Package \xpackage{hyperref} does not see |M\"uller| but
%   its expansion and it does not like it, you get many warnings
%   \begin{quote}
%     |Token not allowed in a PDFDocEncoded string|
%   \end{quote}
%   And the title becomes: |Mu127uller|.
%   Therefore such options must usually be given after package
%   \xpackage{hyperref}
%   is loaded:
%   \begin{quote}
%     |\usepackage{hyperref}|\\
%     |\hypersetup{pdfauthor=Fran\c coise M\"uller}|
%   \end{quote}
%   As package option it will even break with |Fran\c coise|
%   because of the cedilla |\c c|, it is not robust enough.
% \end{itemize}
% For users that do not want with this limitations the package
% offers package \xoption{kvoptions-patch}. It patches \LaTeX's option system
% and tries to teach it also to handle options that are given
% as pairs of key and value and to prevent expansion.
% It can already be used at the very beginning, before \cs{documentclass}:
% \begin{quote}
%   |\RequirePackage{kvoptions-patch}|\\
%   |\documentclass[pdfauthor=Fran\c coise M\"uller]{article}|\\
%   |\usepackage{hyperref}|
% \end{quote}
% The latest time is before the package where you want to use
% problematic values:
% \begin{quote}
%   |\usepackage{kvoptions-patch}|\\
%   |\usepackage[Fran\c coise M\"uller]{hyperref}|
% \end{quote}
% Some remarks:
% \begin{itemize}
% \item The patch requires \eTeX, its \cs{unexpanded} feature
%   is much too nice. It is possible to work around using token
%   registers. But the code becomes longer, slower, more difficult
%   to read and maintain. The package without option \xoption{patch}
%   works and will work without \eTeX.
% \item The code for the patch is quite long, there are many test
%   cases. Thus the probability for bugs is probably not too small.
% \item Since 2008/10/18 v3.0 package \xpackage{kvoptions-patch} is
%   available. Before option \xoption{patch} of package \xpackage{kvoptions}
%   must be used instead. I think, the solution as standalone package
%   \xpackage{kvoptions-patch} is cleaner and avoids option clashes.
% \end{itemize}
%
% \subsection{Option \xoption{debugshow}}
%
% The name of this option follows the convention of packages
% \xpackage{multicol}, \xpackage{tabularx}, and \xpackage{tracefnt}.
% Currently it prints the setting of boolean options, declared
% by \cs{DeclareBoolOption} in the \xfile{.log} file, if that
% boolean option is used.
% You can activate the option by
% \begin{itemize}
% \item |\PassOptionsToPackage{debugshow}{kvoptions}|\\
%       Put this somewhere before package \xpackage{kvoptions} is
%       loaded first, e.g. before \cs{documentclass}.
% \item |\RequirePackage[debugshow]{kvoptions}|\\
%       Before \cs{documentclass} even an author has to use
%       \cs{RequirePackage}. \cs{usepackage} only works after
%       \cs{documentclass}.
% \end{itemize}
% The preferred method is \cs{PassOptionsToPackage}, because
% it does not force the package loading and does not disturb,
% if the package is not loaded later at all.
%
% \section{Limitations}
%
% \subsection{Compatibility}
% \subsubsection{Package \xpackage{kvoptions-patch} vs. package \xpackage{xkvltxp}}
%
% Package \xpackage{xkvltxp} from the \xpackage{xkeyval} project
% has the same goal as package \xpackage{kvoptions-patch}
% and to patch \LaTeX's kernel commands in order to get better support
% for key value options.
% Of course they cannot be used both. The user must decide, which
% method he prefers. Package \xpackage{kvoptions-patch} aborts itself,
% if it detects that \xpackage{xkvltxp} is
% already loaded.
%
% However package \xpackage{xkvltxp} and \xpackage{kvoptions}
% can be used together, example:
% \begin{quote}
% |\usepackage{xkvltxp}|\\
% |\usepackage[...]{foobar} % |\texttt{\itshape foobar using kvoptions}
% \end{quote}
% The other way should work, too.
%
% Package \xpackage{kvoptions-patch} tries to catch more situations
% and to be more robust.
% For example, during the comparison of options it normalizes
% them by removing spaces around |=| and the value. Thus the
% following is not reported as option clash:
% \begin{quote}
%\begin{verbatim}
%\RequirePackage{kvoptions-patch}
%\documentclass{article}
%
%\usepackage[scaled=0.7]{helvet}
%\usepackage[scaled = 0.7]{helvet}
%
%\begin{document}
%\end{document}
%\end{verbatim}
% \end{quote}
%
% \subsection{Limitations}
%
% \subsubsection{Option comparisons}
%
% In some situations \LaTeX\ compares option lists, e.g. option clash check,
% \cs{@ifpackagewith}, or \cs{@ifclasswith}. Apart from catcode and
% sanitizing problems of option \xoption{patch}, there is another problem.
% \LaTeX\ does not know about the type and default values of
% options in key value style. Thus an option clash is reported,
% even if the key value has the same meaning:
% \begin{quote}
% |\usepackage[scaled]{helvet} %| \texttt{\textit{default is }.95}\\
% |\usepackage[.95]{helvet}|\\
% |\usepackage[0.95]{helvet}|
% \end{quote}
%
% \subsubsection{Option list parsing with package \xpackage{kvoptions-patch}}
%
% With package \xpackage{kvoptions-patch} the range of possible values
% in key value specifications is much large, for example
% the comma can be used, if enclosed in curly braces.
%
% Other packages, especially the packages that uses their
% own process option code can be surprised to find tokens
% inside options that they do not expect and errors would
% be the consequence. To avoid errors the options, especially
% the unused option list is sanitized. That means the list
% will only contain tokens with catcode 12 (other) and
% perhaps spaces (catcode 10). This allows a safe parsing
% for other packages. But a comma in the value part is
% no longer protected by curly braces because they have lost
% their special meaning. This is the price for compatibility.
%
% Example:
% \begin{quote}
% |\RequirePackage{kvoptions-patch}|\\
% |\documentclass[a={a,b,c},b]{article}|\\
% |\begin{document}|\\
% |\end{document}|
% \end{quote}
% Result:
% \begin{quote}
% |LaTeX Warning: Unused global option(s):|\\
% |    [a={a,c},b].|
% \end{quote}
%
% \StopEventually{
% }
%
% \section{Implementation}
%
% \subsection{Preamble}
%
%    \begin{macrocode}
%<*package>
%    \end{macrocode}
%
% \paragraph{Reload check and  identification.}
%    Reload check, especially if the package is not used with \LaTeX.
%    \begin{macrocode}
\begingroup\catcode61\catcode48\catcode32=10\relax%
  \catcode13=5 % ^^M
  \endlinechar=13 %
  \catcode35=6 % #
  \catcode39=12 % '
  \catcode44=12 % ,
  \catcode45=12 % -
  \catcode46=12 % .
  \catcode58=12 % :
  \catcode64=11 % @
  \catcode123=1 % {
  \catcode125=2 % }
  \expandafter\let\expandafter\x\csname ver@kvoptions.sty\endcsname
  \ifx\x\relax % plain-TeX, first loading
  \else
    \def\empty{}%
    \ifx\x\empty % LaTeX, first loading,
      % variable is initialized, but \ProvidesPackage not yet seen
    \else
      \expandafter\ifx\csname PackageInfo\endcsname\relax
        \def\x#1#2{%
          \immediate\write-1{Package #1 Info: #2.}%
        }%
      \else
        \def\x#1#2{\PackageInfo{#1}{#2, stopped}}%
      \fi
      \x{kvoptions}{The package is already loaded}%
      \aftergroup\endinput
    \fi
  \fi
\endgroup%
%    \end{macrocode}
%    Package identification:
%    \begin{macrocode}
\begingroup\catcode61\catcode48\catcode32=10\relax%
  \catcode13=5 % ^^M
  \endlinechar=13 %
  \catcode35=6 % #
  \catcode39=12 % '
  \catcode40=12 % (
  \catcode41=12 % )
  \catcode44=12 % ,
  \catcode45=12 % -
  \catcode46=12 % .
  \catcode47=12 % /
  \catcode58=12 % :
  \catcode64=11 % @
  \catcode91=12 % [
  \catcode93=12 % ]
  \catcode123=1 % {
  \catcode125=2 % }
  \expandafter\ifx\csname ProvidesPackage\endcsname\relax
    \def\x#1#2#3[#4]{\endgroup
      \immediate\write-1{Package: #3 #4}%
      \xdef#1{#4}%
    }%
  \else
    \def\x#1#2[#3]{\endgroup
      #2[{#3}]%
      \ifx#1\@undefined
        \xdef#1{#3}%
      \fi
      \ifx#1\relax
        \xdef#1{#3}%
      \fi
    }%
  \fi
\expandafter\x\csname ver@kvoptions.sty\endcsname
\ProvidesPackage{kvoptions}%
  [2016/05/16 v3.12 Key value format for package options (HO)]%
%    \end{macrocode}
%
% \paragraph{Catcodes}
%
%    \begin{macrocode}
\begingroup\catcode61\catcode48\catcode32=10\relax%
  \catcode13=5 % ^^M
  \endlinechar=13 %
  \catcode123=1 % {
  \catcode125=2 % }
  \catcode64=11 % @
  \def\x{\endgroup
    \expandafter\edef\csname KVO@AtEnd\endcsname{%
      \endlinechar=\the\endlinechar\relax
      \catcode13=\the\catcode13\relax
      \catcode32=\the\catcode32\relax
      \catcode35=\the\catcode35\relax
      \catcode61=\the\catcode61\relax
      \catcode64=\the\catcode64\relax
      \catcode123=\the\catcode123\relax
      \catcode125=\the\catcode125\relax
    }%
  }%
\x\catcode61\catcode48\catcode32=10\relax%
\catcode13=5 % ^^M
\endlinechar=13 %
\catcode35=6 % #
\catcode64=11 % @
\catcode123=1 % {
\catcode125=2 % }
\def\TMP@EnsureCode#1#2{%
  \edef\KVO@AtEnd{%
    \KVO@AtEnd
    \catcode#1=\the\catcode#1\relax
  }%
  \catcode#1=#2\relax
}
\TMP@EnsureCode{1}{14}% ^^A (comment)
\TMP@EnsureCode{2}{14}% ^^A (comment)
\TMP@EnsureCode{33}{12}% !
\TMP@EnsureCode{39}{12}% '
\TMP@EnsureCode{40}{12}% (
\TMP@EnsureCode{41}{12}% )
\TMP@EnsureCode{42}{12}% *
\TMP@EnsureCode{44}{12}% ,
\TMP@EnsureCode{45}{12}% -
\TMP@EnsureCode{46}{12}% .
\TMP@EnsureCode{47}{12}% /
\TMP@EnsureCode{58}{12}% :
\TMP@EnsureCode{62}{12}% >
\TMP@EnsureCode{91}{12}% [
\TMP@EnsureCode{93}{12}% ]
\TMP@EnsureCode{94}{7}% ^ (superscript)
\TMP@EnsureCode{96}{12}% `
\edef\KVO@AtEnd{\KVO@AtEnd\noexpand\endinput}
%    \end{macrocode}
%
% \paragraph{External resources.}
%    The package extends the support for key value pairs of
%    package \cs{keyval} to package options. Thus the package
%    needs to be loaded anyway. and we use it for
%    \cs{SetupKeyvalOptions}. AFAIK this does not disturb
%    users of \xpackage{xkeyval}.
%    \begin{macrocode}
\@ifundefined{define@key}{%
  \RequirePackage{keyval}\relax
}{}
%    \end{macrocode}
%
%    Macro \cs{DeclareLocalOptions} parses a comma separated key list
%    and uses \cs{comma@parse} of package \xpackage{kvsetkeys},
%    version 1.3.
%    \begin{macrocode}
\RequirePackage{ltxcmds}[2010/12/02]
\RequirePackage{kvsetkeys}[2007/09/29]
%    \end{macrocode}
%
% \paragraph{Provide macros for \plainTeX.}
%    \begin{macrocode}
\@ifundefined{@x@protect}{%
  \def\@x@protect#1\fi#2#3{%
    \fi\protect#1%
  }%
  \let\@typeset@protect\relax
}{}
\@ifundefined{@currname}{%
  \def\@currname{}%
}{}
\@ifundefined{@currext}{%
  \def\@currext{}%
}{}
%    \end{macrocode}
%
% \paragraph{Options}
%    Option \xoption{debugshow} enables additional lines of
%    code that prints information into the \xfile{.log}
%    file.
%    \begin{macrocode}
\DeclareOption{debugshow}{\catcode\@ne=9 }
%    \end{macrocode}
%
%    \begin{macrocode}
\DeclareOption{patch}{%
  \AtEndOfPackage{%
    \RequirePackage{kvoptions-patch}[2016/05/16]%
  }%
}
%    \end{macrocode}
%
%    Optionen auswerten:
%    \begin{macrocode}
\ProcessOptions\relax
%    \end{macrocode}
%
% \subsection{Option declaration macros}
%
% \subsubsection{\cs{SetupKeyvalOptions}}
%
%    The family for the key value pairs can be setup once
%    and is remembered later.
%    The package name seems a reasonable default for the
%    family key, if it is not set by the package author.
%
%    \begin{macro}{\KVO@family}
%    We cannot store the family setting in one macro,
%    because the package should be usable for many
%    other packages, too. Thus we remember the family
%    setting in a macro, whose name contains the
%    package name with extension, a key in \LaTeX's
%    class/package system.
%    \begin{macrocode}
\define@key{KVO}{family}{%
  \expandafter\edef\csname KVO@family@%
      \@currname.\@currext\endcsname{#1}%
}
\def\KVO@family{%
  \@ifundefined{KVO@family@\@currname.\@currext}{%
    \@currname
  }{%
    \csname KVO@family@\@currname.\@currext\endcsname
  }%
}
%    \end{macrocode}
%    \end{macro}
%
%    \begin{macro}{\KVO@prefix}
%    The value settings of options that are declared by \cs{DeclareBoolOption}
%    and \cs{DeclareStringOption} need to be saved in macros.
%    in the first case this is a switch \cs{if}\meta{prefix}\meta{key},
%    in the latter case a macro \cs{}\meta{prefix}\meta{key}.
%    The prefix can be configured, by \xoption{prefix} that is declared
%    here. The default is the package name with |@| appended.
%    \begin{macrocode}
\define@key{KVO}{prefix}{%
  \expandafter\edef\csname KVO@prefix@%
      \@currname.\@currext\endcsname{#1}%
}
\def\KVO@prefix{%
  \ltx@ifundefined{KVO@prefix@\@currname.\@currext}{%
    \@currname @%
  }{%
    \csname KVO@prefix@\@currname.\@currext\endcsname
  }%
}
%    \end{macrocode}
%    \end{macro}
%
%    \begin{macrocode}
\define@key{KVO}{setkeys}{%
  \expandafter\def\csname KVO@setkeys@%
      \@currname.\@currext\endcsname{#1}%
}
%    \end{macrocode}
%    \begin{macro}{\KVO@setkeys}
%    \begin{macrocode}
\def\KVO@setkeys{%
  \ltx@IfUndefined{KVO@setkeys@\@currname.\@currext}{%
    \setkeys
  }{%
    \csname KVO@setkeys@\@currname.\@currext\endcsname
  }%
}
%    \end{macrocode}
%    \end{macro}
%
%    \begin{macro}{\SetupKeyvalOptions}
%    The argument of \cs{SetupKeyvalOptions} expects a key value
%    list, known keys are \xoption{family} and \xoption{prefix}.
%    \begin{macrocode}
\newcommand*{\SetupKeyvalOptions}{%
  \kvsetkeys{KVO}%
}
%    \end{macrocode}
%    \end{macro}
%
% \subsubsection{\cs{DeclareBoolOption}}
%
%    \begin{macro}{\DeclareBoolOption}
%    Usually options of boolean type can be given by the
%    user without value and this means a setting to \emph{true}.
%    We follow this convention here. Also it simplifies the
%    user interface.
%
%    The switch is created and initialized with \emph{false}.
%    The default setting can be overwritten by the optional
%    argument.
%
%    \LaTeX's \cs{newif} does not check for already defined
%    macros, therefore we add this check here to prevent
%    the user from accidently redefining of \TeX's primitives
%    and other macros.
%    \begin{macrocode}
\newcommand*{\DeclareBoolOption}[2][false]{%
  \KVO@ifdefinable{if\KVO@prefix#2}{%
    \KVO@ifdefinable{\KVO@prefix#2true}{%
      \KVO@ifdefinable{\KVO@prefix#2false}{%
        \csname newif\expandafter\endcsname
        \csname if\KVO@prefix#2\endcsname
        \@ifundefined{\KVO@prefix#2#1}{%
          \PackageWarning{kvoptions}{%
            Initialization of option `#2' failed,\MessageBreak
            cannot set boolean option to `#1',\MessageBreak
            use `true' or `false', now using `false'%
          }%
        }{%
          \csname\KVO@prefix#2#1\endcsname
        }%
        \begingroup
          \edef\x{\endgroup
            \noexpand\define@key{\KVO@family}{#2}[true]{%
              \noexpand\KVO@boolkey{\@currname}%
              \ifx\@currext\@clsextension
                \noexpand\@clsextension
              \else
                \noexpand\@pkgextension
              \fi
              {\KVO@prefix}{#2}{####1}%
            }%
          }%
        \x
      }%
    }%
  }%
}
%    \end{macrocode}
%    \end{macro}
%
%    \begin{macro}{\DeclareComplementaryOption}
%    The first argument is the key name, the second the
%    key that must be a boolean option with the same
%    current family and prefix. A new switch is not
%    created for the new key, we have already a switch.
%    Instead we define switch setting commands to work
%    on the parent switch.
%    \begin{macrocode}
\newcommand*{\DeclareComplementaryOption}[2]{%
  \@ifundefined{if\KVO@prefix#2}{%
    \PackageError{kvoptions}{%
      Cannot generate option code for `#1',\MessageBreak
      parent switch `#2' does not exist%
    }{%
      You are inside %
      \ifx\@currext\@clsextension class\else package\fi\space
      `\@currname.\@currext'.\MessageBreak
      `\KVO@family' is used as familiy %
      for the keyval options.\MessageBreak
      `\KVO@prefix' serves as prefix %
      for internal switch macros.\MessageBreak
      \MessageBreak
      \@ehc
    }%
  }{%
    \KVO@ifdefinable{\KVO@prefix#1true}{%
      \KVO@ifdefinable{\KVO@prefix#1false}{%
        \expandafter\let\csname\KVO@prefix#1false\expandafter\endcsname
          \csname\KVO@prefix#2true\endcsname
        \expandafter\let\csname\KVO@prefix#1true\expandafter\endcsname
          \csname\KVO@prefix#2false\endcsname
%    \end{macrocode}
%    The same code part as in \cs{DeclareBoolOption} can
%    now be used.
%    \begin{macrocode}
        \begingroup
          \edef\x{\endgroup
            \noexpand\define@key{\KVO@family}{#1}[true]{%
              \noexpand\KVO@boolkey{\@currname}%
              \ifx\@currext\@clsextension
                \noexpand\@clsextension
              \else
                \noexpand\@pkgextension
              \fi
              {\KVO@prefix}{#1}{####1}%
            }%
          }%
        \x
      }%
    }%
  }%
}
%    \end{macrocode}
%    \end{macro}
%
%    \begin{macro}{\KVO@ifdefinable}
%    Generate the command token LaTeX's \cs{@ifdefinable} expects.
%    \begin{macrocode}
\def\KVO@ifdefinable#1{%
  \expandafter\@ifdefinable\csname #1\endcsname
}
%    \end{macrocode}
%    \end{macro}
%
%    \begin{macro}{\KVO@boolkey}
%    We check explicitly for |true| and |false| to prevent
%    the user from accidently calling other macros.
%    \begin{quote}
%    \begin{tabular}{@{}ll@{}}
%    |#1|& package/class name\\
%    |#2|& |\@pkgextension|/|\@clsextension|\\
%    |#3|& prefix \\
%    |#4|& key name\\
%    |#5|& new value
%    \end{tabular}
%    \end{quote}
%    \begin{macrocode}
\def\KVO@boolkey#1#2#3#4#5{%
  \edef\KVO@param{#5}%
  \ltx@onelevel@sanitize\KVO@param
  \ifx\KVO@param\KVO@true
    \expandafter\@firstofone
  \else
    \ifx\KVO@param\KVO@false
      \expandafter\expandafter\expandafter\@firstofone
    \else
      \ifx#2\@clsextension
        \expandafter\ClassWarning
      \else
        \expandafter\PackageWarning
      \fi
      {#1}{%
        Value `\KVO@param' is not supported by\MessageBreak
        option `#4'%
      }%
      \expandafter\expandafter\expandafter\@gobble
    \fi
  \fi
  {%
    ^^A\ifx#2\@clsextension
    ^^A  \expandafter\ClassInfo
    ^^A\else
    ^^A  \expandafter\PackageInfo
    ^^A\fi
    ^^A{#1}{[option] #4=\KVO@param}%
    \csname#3#4\KVO@param\endcsname
  }%
}
%    \end{macrocode}
%    \end{macro}
%    \begin{macro}{\KVO@true}
%    \begin{macro}{\KVO@false}
%    The macros \cs{KVO@true} and \cs{KVO@false} are used
%    for string comparisons. After \cs{ltx@onelevel@sanitize}
%    we have only tokens with catcode 12 (other).
%    \begin{macrocode}
\def\KVO@true{true}
\def\KVO@false{false}
\ltx@onelevel@sanitize\KVO@true
\ltx@onelevel@sanitize\KVO@false
%    \end{macrocode}
%    \end{macro}
%    \end{macro}
%
% \subsubsection{\cs{DeclareStringOption}}
%
%    \begin{macro}{\DeclareStringOption}
%    \begin{macrocode}
\newcommand*{\DeclareStringOption}[2][]{%
  \@ifnextchar[{%
    \KVO@DeclareStringOption{#1}{#2}@%
  }{%
    \KVO@DeclareStringOption{#1}{#2}{}[]%
  }%
}
%    \end{macrocode}
%    \end{macro}
%    \begin{macro}{\KVO@DeclareStringOption}
%    \begin{macrocode}
\def\KVO@DeclareStringOption#1#2#3[#4]{%
  \KVO@ifdefinable{\KVO@prefix#2}{%
    \@namedef{\KVO@prefix#2}{#1}%
    \begingroup
      \ifx\\#3\\%
        \toks@{}%
      \else
        \toks@{[{#4}]}%
      \fi
      \edef\x{\endgroup
        \noexpand\define@key{\KVO@family}{#2}\the\toks@{%
          ^^A\begingroup
          ^^A  \toks@{####1}%
          ^^A  \ifx\@currext\@clsextension
          ^^A    \noexpand\ClassInfo
          ^^A  \else
          ^^A    \noexpand\PackageInfo
          ^^A  \fi
          ^^A  {\@currname}{%
          ^^A    [option] #2={\noexpand\the\toks@}%
          ^^A  }%
          ^^A\endgroup
          \noexpand\def
          \expandafter\noexpand\csname\KVO@prefix#2\endcsname{####1}%
        }%
      }%
    \x
  }%
}
%    \end{macrocode}
%    \end{macro}
%
% \subsubsection{\cs{DeclareVoidOption}}
%
%    \begin{macro}{\DeclareVoidOption}
%    \begin{macrocode}
\newcommand*{\DeclareVoidOption}[2]{%
  \begingroup
    \let\next\@gobbletwo
    \KVO@ifdefinable{\KVO@prefix#1}{%
      \let\next\@firstofone
    }%
  \expandafter\endgroup
  \next{%
    \begingroup
      \edef\x{\endgroup
        \noexpand\define@key{\KVO@family}{#1}[\KVO@VOID@]{%
          \noexpand\KVO@voidkey{\@currname}%
          \ifx\@currext\@clsextension
            \noexpand\@clsextension
          \else
            \noexpand\@pkgextension
          \fi
          {#1}%
          {####1}%
          \expandafter\noexpand\csname\KVO@prefix#1\endcsname
        }%
      }%
    \x
    \begingroup
      \toks@{#2}%
    \expandafter\endgroup
    \expandafter\def
    \csname\KVO@prefix#1\expandafter\endcsname
    \expandafter{\the\toks@}%
  }%
}
\def\KVO@VOID@{@VOID@}
%    \end{macrocode}
%    \end{macro}
%    \begin{macro}{\KVO@voidkey}
%    \begin{tabular}{@{}ll@{}}
%    |#1|& package/class name\\
%    |#2|& |\@pkgextension|/|\@clsextension|\\
%    |#3|& key name\\
%    |#4|& default (|@VOID@|)\\
%    |#5|& macro with option code
%    \end{tabular}
%    \begin{macrocode}
\def\KVO@voidkey#1#2#3#4{%
  \def\CurrentOption{#3}%
  \begingroup
    \def\x{#4}%
  \expandafter\endgroup
  \ifx\x\KVO@VOID@
  \else
    \ifx#2\@clsextension
      \expandafter\ClassWarning
    \else
      \expandafter\PackageWarning
    \fi
    {#1}{%
      Unexpected value for option `#3'\MessageBreak
      is ignored%
    }%
  \fi
  ^^A\ifx#2\@clsextension
  ^^A  \expandafter\ClassInfo
  ^^A\else
  ^^A  \expandafter\PackageInfo
  ^^A\fi
  ^^A{#1}{[option] #3}%
}
%    \end{macrocode}
%    \end{macro}
%
% \subsubsection{\cs{DeclareDefaultOption}}
%
%    \begin{macro}{\DeclareDefaultOption}
%    \begin{macrocode}
\newcommand*{\DeclareDefaultOption}{%
  \@namedef{KVO@default@\@currname.\@currext}%
}
%    \end{macrocode}
%    \end{macro}
%
% \subsubsection{\cs{DeclareLocalOptions}}
%
%    \begin{macro}{\DeclareLocalOptions}
%    \begin{macrocode}
\newcommand*{\DeclareLocalOptions}[1]{%
  \comma@parse{#1}\KVO@DeclareLocalOption
}
%    \end{macrocode}
%    \end{macro}
%    \begin{macro}{\KVO@DeclareLocalOption}
%    \begin{macrocode}
\def\KVO@DeclareLocalOption#1{%
  \expandafter\def\csname KVO@local@\KVO@family @#1\endcsname{}%
}
%    \end{macrocode}
%    \end{macro}
%
% \subsection{Dynamic options}
%
% \subsubsection{\cs{DisableKeyvalOption}}
%
%    \begin{macrocode}
\SetupKeyvalOptions{%
  family=KVOdyn,%
  prefix=KVOdyn@%
}
\DeclareBoolOption[true]{global}
\DeclareComplementaryOption{local}{global}
\DeclareStringOption[undef]{action}
\let\KVOdyn@name\relax
\let\KVOdyn@ext\@empty
\define@key{KVOdyn}{class}{%
  \def\KVOdyn@name{#1}%
  \let\KVOdyn@ext\@clsextension
}
\define@key{KVOdyn}{package}{%
  \def\KVOdyn@name{#1}%
  \let\KVOdyn@ext\@pkgextension
}
\newcommand*{\DisableKeyvalOption}[3][]{%
  \begingroup
    \kvsetkeys{KVOdyn}{#1}%
    \def\x{\endgroup}%
    \@ifundefined{KVO@action@\KVOdyn@action}{%
      \PackageError{kvoptions}{%
        Unknown disable action %
        `\expandafter\strip@prefix\meaning\KVOdyn@action'\MessageBreak
        for option `#3' in keyval family '#2'%
      }\@ehc
    }{%
      \csname KVO@action@\KVOdyn@action\endcsname{#2}{#3}%
    }%
  \x
}
\def\KVO@action@undef#1#2{%
  \edef\x{\endgroup
    \ifKVOdyn@global\global\fi
    \let
    \expandafter\noexpand\csname KV@#1@#2\endcsname
    \relax
    \ifKVOdyn@global\global\fi
    \let
    \expandafter\noexpand\csname KV@#1@#2@default\endcsname
    \relax
  }%
  ^^A\PackageInfo{kvoptions}{%
  ^^A  [option] key `#2' of family `#1'\MessageBreak
  ^^A  is disabled (undef, \ifKVOdyn@global global\else local\fi)%
  ^^A}%
}
\def\KVO@action@ignore#1#2{%
  \edef\x{\endgroup
    \ifKVOdyn@global\global\fi
    \let
    \expandafter\noexpand\csname KV@#1@#2\endcsname
    \noexpand\@gobble
    \ifKVOdyn@global\global\fi
    \let
    \expandafter\noexpand\csname KV@#1@#2@default\endcsname
    \noexpand\@empty
  }%
  ^^A\PackageInfo{kvoptions}{%
  ^^A  [option] key `#2' of family `#1'\MessageBreak
  ^^A  is disabled (ignore, \ifKVOdyn@global global\else local\fi)%
  ^^A}%
}
\def\KVO@action@error{%
  \KVO@do@action{error}%
}
\def\KVO@action@warning{%
  \KVO@do@action{warning}%
}
%    \end{macrocode}
%    \begin{tabular}{@{}ll@{}}
%    |#1|& |error| or |warning|\\
%    |#2|& \meta{family}\\
%    |#3|& \meta{key}\\
%    \end{tabular}
%    \begin{macrocode}
\def\KVO@do@action#1#2#3{%
  \ifx\KVOdyn@name\relax
    \PackageError{kvoptions}{%
      Action type `#1' needs package/class name\MessageBreak
      for key `#3' in family `#2'%
    }\@ehc
  \else
    \edef\x{\endgroup
      \noexpand\define@key{#2}{#3}[]{%
        \expandafter\noexpand\csname KVO@disable@#1\endcsname
        {\KVOdyn@name}\noexpand\KVOdyn@ext{#3}%
      }%
      \ifKVOdyn@global
        \global\let
        \expandafter\noexpand\csname KV@#2@#3\endcsname
        \expandafter\noexpand\csname KV@#2@#3\endcsname
        \global\let
        \expandafter\noexpand\csname KV@#2@#3@default\endcsname
        \expandafter\noexpand\csname KV@#2@#3@default\endcsname
      \fi
    }%
    ^^A\ifx\KVOdyn@ext\@clsextension
    ^^A  \expandafter\ClassInfo
    ^^A\else
    ^^A   \expandafter\PackageInfo
    ^^A\fi
    ^^A{\KVOdyn@name}{%
    ^^A  [option] key `#3' of family `#2'\MessageBreak
    ^^A  is disabled (#1, \ifKVOdyn@global global\else local\fi)%
    ^^A}%
  \fi
}
\def\KVO@disable@error#1#2#3{%
  \ifx#2\@clsextension
    \expandafter\ClassError
  \else
    \expandafter\PackageError
  \fi
  {#1}{%
    Option `#3' is given too late,\MessageBreak
    now the option is ignored%
  }\@ehc
}
\def\KVO@disable@warning#1#2#3{%
  \ifx#2\@clsextension
    \expandafter\ClassWarning
  \else
    \expandafter\PackageWarning
  \fi
  {#1}{%
    Option `#3' is already consumed\MessageBreak
    and has no effect%
  }%
}
%    \end{macrocode}
%
% \subsection{Change option code}
%
% \subsubsection{\cs{AddToKeyvalOption}}
%
%    \begin{macro}{\AddToKeyvalOption}
%    \begin{macrocode}
\newcommand*{\AddToKeyvalOption}{%
  \@ifstar{%
    \begingroup
      \edef\x{\endgroup
        \noexpand\KVO@AddToKeyvalOption{\KVO@family}%
      }%
    \x
  }%
  \KVO@AddToKeyvalOption
}
%    \end{macrocode}
%    \end{macro}
%    \begin{macro}{\KVO@AddToKeyvalOption}
%    \begin{macrocode}
\def\KVO@AddToKeyvalOption#1#2{%
  \@ifundefined{KV@#1@#2}{%
    \PackageWarning{kvoptions}{%
      Key `#2' of family `#1' does not exist.\MessageBreak
      Ignoring \string\AddToKeyvalOption
    }%
    \@gobble
  }{%
    \edef\KVO@next{%
      \noexpand\KVO@@AddToKeyvalOption
      \expandafter\noexpand\csname KV@#1@#2\endcsname
    }%
    \afterassignment\KVO@next
    \def\KVO@temp##1%
  }%
}
%    \end{macrocode}
%    \end{macro}
%    \begin{macro}{\KVO@@AddToKeyvalOption}
%    \begin{macrocode}
\def\KVO@@AddToKeyvalOption#1{%
  \begingroup
    \toks@\expandafter{#1{##1}}%
    \toks@\expandafter{\the\expandafter\toks@\KVO@temp{##1}}%
    \edef\x{\endgroup
      \noexpand\def\noexpand#1####1{\the\toks@}%
    }%
  \x
}
%    \end{macrocode}
%    \end{macro}
%
% \subsection{Process options}
%
% \subsubsection{Get global options}
%
%    Package \xpackage{xkeyval} removes options with equal signs from
%    the global options (\cs{@classoptionslist}). The effect is that
%    other packages and classes will not see these global options anymore.
%    A bug-report was answered that this behaviour is ``by design''.
%    Thus I call it a design bug. Now getting the global options require
%    an algorithm instead of a simple macro call.
%    \begin{macrocode}
%</package>
%<*package|patch>
%    \end{macrocode}
%    \begin{macro}{\KVO@IfDefThen}
%    Call |#2| if command |#1| is defined and not \cs{relax}.
%    (Package \xpackage{kvoptions-patch} does not load
%    package \xpackage{ltxcmds}.)
%    \begin{macrocode}
\def\KVO@IfDefThen#1#2{%
  \ifx#1\ltx@undefined
  \else
    \ifx#1\relax
    \else
      #2%
    \fi
  \fi
}%
%    \end{macrocode}
%    \end{macro}
%    \begin{macro}{\KVO@GetClassOptionsList}
%    \begin{macrocode}
\def\KVO@GetClassOptionsList{%
  \let\KVO@classoptionslist\@classoptionslist
  \KVO@IfDefThen\@classoptionslist{%
    \KVO@IfDefThen\XKV@documentclass{%
      \ifx\XKV@documentclass\ltx@empty
      \else
        \KVO@IfDefThen\XKV@classoptionslist{%
          \ifx\XKV@classoptionslist\ltx@empty
          \else
            \let\KVO@classoptionslist\XKV@classoptionslist
          \fi
        }%
      \fi
    }%
  }%
}%
%    \end{macrocode}
%    \end{macro}
%    \begin{macrocode}
%</package|patch>
%<*package>
%    \end{macrocode}
%
% \subsubsection{\cs{ProcessKeyvalOptions}}
%
%    \begin{macro}{\ProcessKeyvalOptions}
%    If the optional star is given, we get the family
%    name and expand it for safety.
%    \begin{macrocode}
\newcommand*{\ProcessKeyvalOptions}{%
  \@ifstar{%
    \begingroup
      \edef\x{\endgroup
        \noexpand\KVO@ProcessKeyvalOptions{\KVO@family}%
      }%
    \x
  }%
  \KVO@ProcessKeyvalOptions
}
%    \end{macrocode}
%    \end{macro}
%
%    \begin{macrocode}
\def\KVO@ProcessKeyvalOptions#1{%
  \let\@tempc\relax
  \let\KVO@temp\@empty
%    \end{macrocode}
%    Add any global options that are known to KV to the start of
%    the list being built in |\KVO@temp| and mark them used (by
%    removing them from the unused option list).
%    \begin{macrocode}
  \ifx\@currext\@clsextension
  \else
    \KVO@GetClassOptionsList
    \ifx\KVO@classoptionslist\relax
    \else
      \@for\KVO@CurrentOption:=\KVO@classoptionslist\do{%
        \@ifundefined{KV@#1@\expandafter\KVO@getkey
                      \KVO@CurrentOption=\@nil}{%
        }{%
          \@ifundefined{KVO@local@#1@\expandafter\KVO@getkey
                        \KVO@CurrentOption=\@nil}{%
            \ifx\KVO@Patch Y%
              \edef\KVO@temp{%
                \etex@unexpanded\expandafter{%
                  \KVO@temp
                }%
                ,%
                \etex@unexpanded\expandafter{%
                  \KVO@CurrentOption
                }%
                ,%
              }%
              \ltx@onelevel@sanitize\KVO@CurrentOption
            \else
              \edef\KVO@temp{%
                \KVO@temp
                ,%
                \KVO@CurrentOption
                ,%
              }%
            \fi
            \@expandtwoargs\@removeelement\KVO@CurrentOption
              \@unusedoptionlist\@unusedoptionlist
          }{}%
        }%
      }%
    \fi
  \fi
%    \end{macrocode}
%     Now stick the package options at the end of the list and wrap
%     in a call to \cs{setkeys}. A class ignores unknown global
%     options, we must remove them to prevent error messages
%     from \cs{setkeys}.
%    \begin{macrocode}
  \begingroup
    \toks\tw@{}%
    \@ifundefined{opt@\@currname.\@currext}{%
      \toks@\expandafter{\KVO@temp}%
    }{%
      \toks@\expandafter\expandafter\expandafter{%
        \csname opt@\@currname.\@currext\endcsname
      }%
      \ifx\@currext\@clsextension
        \edef\CurrentOption{\the\toks@}%
        \toks@\expandafter{\KVO@temp}%
        \@for\CurrentOption:=\CurrentOption\do{%
          \@ifundefined{%
            KV@#1@\expandafter\KVO@getkey\CurrentOption=\@nil
          }{%
%    \end{macrocode}
%    A class puts not used options in the unused option list
%    unless there is a default handler.
%    \begin{macrocode}
            \@ifundefined{KVO@default@\@currname.\@currext}{%
              \ifx\KVO@Patch Y%
                \ltx@onelevel@sanitize\CurrentOption
              \fi
              \ifx\@unusedoptionlist\@empty
                \global\let\@unusedoptionlist\CurrentOption
              \else
                \expandafter\expandafter\expandafter\gdef
                \expandafter\expandafter\expandafter\@unusedoptionlist
                \expandafter\expandafter\expandafter{%
                  \expandafter\@unusedoptionlist
                  \expandafter,\CurrentOption
                }%
              \fi
            }{%
              \toks\tw@\expandafter{%
                \the\toks\expandafter\tw@\expandafter,\CurrentOption
              }%
            }%
          }{%
            \toks@\expandafter{%
              \the\expandafter\toks@\expandafter,\CurrentOption
            }%
          }%
        }%
      \else
%    \end{macrocode}
%    Without default action we pass all options to \cs{setkeys}.
%    Otherwise we have to check which options are known.
%    These are passed to \cs{setkeys}. For the others the default
%    action is performed.
%    \begin{macrocode}
        \@ifundefined{KVO@default@\@currname.\@currext}{%
          \toks@\expandafter\expandafter\expandafter{%
            \expandafter\KVO@temp\the\toks@
          }%
        }{%
          \edef\CurrentOption{\the\toks@}%
          \toks@\expandafter{\KVO@temp}%
          \@for\CurrentOption:=\CurrentOption\do{%
            \@ifundefined{%
              KV@#1@\expandafter\KVO@getkey\CurrentOption=\@nil
            }{%
              \toks\tw@\expandafter{%
                \the\toks\expandafter\tw@\expandafter,\CurrentOption
              }%
            }{%
              \toks@\expandafter{%
                \the\expandafter\toks@\expandafter,\CurrentOption
              }%
            }%
          }%
        }%
      \fi
    }%
    \edef\KVO@temp{\endgroup
      \noexpand\KVO@calldefault{\the\toks\tw@}%
      \noexpand\KVO@setkeys{#1}{\the\toks@}%
    }%
  \KVO@temp
%    \end{macrocode}
%    Some cleanup of \cs{ProcessOptions}.
%    \begin{macrocode}
  \let\CurrentOption\@empty
  \AtEndOfPackage{\let\@unprocessedoptions\relax}%
}
%    \end{macrocode}
%
% \subsubsection{\cs{ProcessLocalKeyvalOptions}}
%
%    \begin{macro}{\ProcessLocalKeyvalOptions}
%    If the optional star is given, we get the family
%    name and expand it for safety.
%    \begin{macrocode}
\newcommand*{\ProcessLocalKeyvalOptions}{%
  \@ifstar{%
    \begingroup
      \edef\x{\endgroup
        \noexpand\KVO@ProcessLocalKeyvalOptions{\KVO@family}%
      }%
    \x
  }%
  \KVO@ProcessLocalKeyvalOptions
}
%    \end{macrocode}
%    \end{macro}
%
%    \begin{macrocode}
\def\KVO@ProcessLocalKeyvalOptions#1{%
  \let\@tempc\relax
  \let\KVO@temp\@empty
%    \end{macrocode}
%    Check if \cs{ProcessLocalKeyvalOptions} is called inside
%    a package.
%    \begin{macrocode}
  \ifx\@currext\@pkgextension
  \else
    \PackageError{kvoptions}{%
      \string\ProcessLocalKeyvalOptions\space is intended for packages only%
    }\@ehc
  \fi
%    \end{macrocode}
%     The package options are put into toks register \cs{toks@}.
%    \begin{macrocode}
  \begingroup
    \toks\tw@{}%
    \@ifundefined{opt@\@currname.\@currext}{%
      \toks@\expandafter{\KVO@temp}%
    }{%
      \toks@\expandafter\expandafter\expandafter{%
        \csname opt@\@currname.\@currext\endcsname
      }%
%    \end{macrocode}
%    Without default action we pass all options to \cs{setkeys}.
%    Otherwise we have to check which options are known.
%    These are passed to \cs{setkeys}. For the others the default
%    action is performed.
%    \begin{macrocode}
      \@ifundefined{KVO@default@\@currname.\@currext}{%
        \toks@\expandafter\expandafter\expandafter{%
          \expandafter\KVO@temp\the\toks@
        }%
      }{%
        \edef\CurrentOption{\the\toks@}%
        \toks@\expandafter{\KVO@temp}%
        \@for\CurrentOption:=\CurrentOption\do{%
          \@ifundefined{%
            KV@#1@\expandafter\KVO@getkey\CurrentOption=\@nil
          }{%
            \toks\tw@\expandafter{%
              \the\toks\expandafter\tw@\expandafter,\CurrentOption
            }%
          }{%
            \toks@\expandafter{%
              \the\expandafter\toks@\expandafter,\CurrentOption
            }%
          }%
        }%
      }%
    }%
    \edef\KVO@temp{\endgroup
      \noexpand\KVO@calldefault{\the\toks\tw@}%
      \noexpand\KVO@setkeys{#1}{\the\toks@}%
    }%
  \KVO@temp
%    \end{macrocode}
%    Some cleanup of \cs{ProcessOptions}.
%    \begin{macrocode}
  \let\CurrentOption\@empty
  \AtEndOfPackage{\let\@unprocessedoptions\relax}%
}
%    \end{macrocode}
%
% \subsubsection{Helper macros}
%
%    \begin{macro}{\KVO@getkey}
%    Extract the key part of a key=value pair.
%    \begin{macrocode}
\def\KVO@getkey#1=#2\@nil{#1}
%    \end{macrocode}
%    \end{macro}
%
%    \begin{macro}{\KVO@calldefault}
%    \begin{macrocode}
\def\KVO@calldefault#1{%
  \begingroup
    \def\x{#1}%
  \expandafter\endgroup
  \ifx\x\@empty
  \else
    \@for\CurrentOption:=#1\do{%
      \ifx\CurrentOption\@empty
      \else
        \expandafter\KVO@setcurrents\CurrentOption=\@nil
        \@nameuse{KVO@default@\@currname.\@currext}%
      \fi
    }%
  \fi
}
%    \end{macrocode}
%    \end{macro}
%    \begin{macro}{\KVO@setcurrents}
%    Extract the key part of a key=value pair.
%    \begin{macrocode}
\def\KVO@setcurrents#1=#2\@nil{%
  \def\CurrentOptionValue{#2}%
  \ifx\CurrentOptionValue\@empty
    \let\CurrentOptionKey\CurrentOption
    \let\CurrentOptionValue\relax
  \else
    \edef\CurrentOptionKey{\zap@space#1 \@empty}%
    \expandafter\KVO@setcurrentvalue\CurrentOption\@nil
  \fi
}
%    \end{macrocode}
%    \end{macro}
%    \begin{macro}{\KV@setcurrentvalue}
%    Here the value part is parsed. Package \xpackage{keyval}'s
%    \cs{KV@@sp@def} helps in removing spaces at the begin and
%    end of the value.
%    \begin{macrocode}
\def\KVO@setcurrentvalue#1=#2\@nil{%
  \KV@@sp@def\CurrentOptionValue{#2}%
}
%    \end{macrocode}
%    \end{macro}
%
% \subsection{\plainTeX}
%
%    Disable \LaTeX\ stuff.
%    \begin{macrocode}
\begingroup\expandafter\expandafter\expandafter\endgroup
\expandafter\ifx\csname documentclass\endcsname\relax
  \def\ProcessKeyvalOptions{%
    \@ifstar{}\@gobble
  }%
\fi
%    \end{macrocode}
%
%    \begin{macrocode}
\KVO@AtEnd%
%</package>
%    \end{macrocode}
%
% \subsection{Package \xpackage{kvoptions-patch}}
%
%    \begin{macrocode}
%<*patch>
\NeedsTeXFormat{LaTeX2e}
\begingroup\catcode61\catcode48\catcode32=10\relax%
  \catcode13=5 % ^^M
  \endlinechar=13 %
  \catcode123=1 % {
  \catcode125=2 % }
  \catcode64=11 % @
  \def\x{\endgroup
    \expandafter\edef\csname KVO@AtEnd\endcsname{%
      \endlinechar=\the\endlinechar\relax
      \catcode13=\the\catcode13\relax
      \catcode32=\the\catcode32\relax
      \catcode35=\the\catcode35\relax
      \catcode61=\the\catcode61\relax
      \catcode64=\the\catcode64\relax
      \catcode123=\the\catcode123\relax
      \catcode125=\the\catcode125\relax
    }%
  }%
\x\catcode61\catcode48\catcode32=10\relax%
\catcode13=5 % ^^M
\endlinechar=13 %
\catcode35=6 % #
\catcode64=11 % @
\catcode123=1 % {
\catcode125=2 % }
\def\TMP@EnsureCode#1#2{%
  \edef\KVO@AtEnd{%
    \KVO@AtEnd
    \catcode#1=\the\catcode#1\relax
  }%
  \catcode#1=#2\relax
}
\TMP@EnsureCode{39}{12}% '
\TMP@EnsureCode{40}{12}% (
\TMP@EnsureCode{41}{12}% )
\TMP@EnsureCode{43}{12}% +
\TMP@EnsureCode{44}{12}% ,
\TMP@EnsureCode{45}{12}% -
\TMP@EnsureCode{46}{12}% .
\TMP@EnsureCode{47}{12}% /
\TMP@EnsureCode{58}{12}% :
\TMP@EnsureCode{60}{12}% <
\TMP@EnsureCode{62}{12}% >
\TMP@EnsureCode{91}{12}% [
\TMP@EnsureCode{93}{12}% ]
\TMP@EnsureCode{96}{12}% `
\TMP@EnsureCode{124}{12}% |
\edef\KVO@AtEnd{\KVO@AtEnd\noexpand\endinput}
\ProvidesPackage{kvoptions-patch}%
  [2016/05/16 v3.12 LaTeX patch for keyval options (HO)]%
%    \end{macrocode}
%
%    Check for \eTeX.
%    \begin{macrocode}
\begingroup\expandafter\expandafter\expandafter\endgroup
\expandafter\ifx\csname eTeXversion\endcsname\relax
  \PackageWarningNoLine{kvoptions-patch}{%
    Package loading is aborted, because e-TeX is missing%
  }%
  \expandafter\KVO@AtEnd
\fi%
%    \end{macrocode}
%
%    Package \xpackage{etexcmds} for \cs{etex@unexpanded}.
%    \begin{macrocode}
\RequirePackage{etexcmds}[2007/09/09]
\ifetex@unexpanded
\else
  \PackageError{kvoptions-patch}{%
    Could not find eTeX's \string\unexpanded.\MessageBreak
    Try adding \string\RequirePackage\string{etexcmds\string} %
    before \string\documentclass%
  }\@ehd
  \expandafter\KVO@AtEnd
\fi%
%    \end{macrocode}
%
%    Check for package \xpackage{xkvltxp}.
%    \begin{macrocode}
\@ifpackageloaded{xkvltxp}{%
  \PackageWarningNoLine{kvoptions}{%
    Option `patch' cannot be used together with\MessageBreak
    package `xkvltxp' that is already loaded.\MessageBreak
    Therefore package loading is aborted%
  }%
  \KVO@AtEnd
}{}%
%    \end{macrocode}
%
%    \begin{macrocode}
\def\@if@ptions#1#2#3{%
  \begingroup
    \KVO@normalize\KVO@temp{#3}%
    \edef\x{\endgroup
      \noexpand\@if@pti@ns{%
        \detokenize\expandafter\expandafter\expandafter{%
          \csname opt@#2.#1\endcsname
        }%
      }{%
        \detokenize\expandafter{\KVO@temp}%
      }%
    }%
  \x
}
%    \end{macrocode}
%
%    \begin{macrocode}
\def\@pass@ptions#1#2#3{%
  \KVO@normalize\KVO@temp{#2}%
  \@ifundefined{opt@#3.#1}{%
    \expandafter\gdef\csname opt@#3.#1%
          \expandafter\endcsname\expandafter{%
      \KVO@temp
    }%
  }{%
    \expandafter\gdef\csname opt@#3.#1%
          \expandafter\expandafter\expandafter\endcsname
          \expandafter\expandafter\expandafter{%
      \csname opt@#3.#1\expandafter\endcsname\expandafter,\KVO@temp
    }%
  }%
}
%    \end{macrocode}
%
%    \begin{macrocode}
\def\ProcessOptions{%
  \let\ds@\@empty
  \@ifundefined{opt@\@currname.\@currext}{%
    \let\@curroptions\@empty
  }{%
    \expandafter\expandafter\expandafter\def
    \expandafter\expandafter\expandafter\@curroptions
    \expandafter\expandafter\expandafter{%
      \csname opt@\@currname.\@currext\endcsname
    }%
  }%
  \@ifstar\KVO@xprocess@ptions\KVO@process@ptions
}
%    \end{macrocode}
%
%    \begin{macrocode}
\def\KVO@process@ptions{%
  \@for\CurrentOption:=\@declaredoptions\do{%
    \ifx\CurrentOption\@empty
    \else
      \begingroup
        \ifx\@currext\@clsextension
          \toks@{}%
        \else
          \KVO@GetClassOptionsList
          \toks@\expandafter{\KVO@classoptionslist,}%
        \fi
        \toks\tw@\expandafter{\@curroptions}%
        \edef\x{\endgroup
          \noexpand\in@{,\CurrentOption,}{,\the\toks@\the\toks\tw@,}%
        }%
      \x
      \ifin@
        \KVO@use@ption
        \expandafter\let\csname ds@\CurrentOption\endcsname\@empty
      \fi
    \fi
  }%
  \KVO@process@pti@ns
}
%    \end{macrocode}
%
%    \begin{macrocode}
\def\KVO@xprocess@ptions{%
  \ifx\@currext\@clsextension
  \else
    \KVO@GetClassOptionsList
    \@for\CurrentOption:=\KVO@classoptionslist\do{%
      \ifx\CurrentOption\@empty
      \else
        \KVO@in@\CurrentOption\@declaredoptions
        \ifin@
          \KVO@use@ption
          \expandafter\let\csname ds@\CurrentOption\endcsname\@empty
        \fi
      \fi
    }%
  \fi
  \KVO@process@pti@ns
}
%    \end{macrocode}
%
%    \begin{macrocode}
\def\KVO@in@#1#2{%
  \in@false
  \begingroup
    \@for\x:=#2\do{%
      \ifx\x#1\relax
        \in@true
      \fi
    }%
    \edef\x{\endgroup
      \ifin@
        \noexpand\in@true
      \fi
    }%
  \x
}
%    \end{macrocode}
%
%    \begin{macrocode}
\def\KVO@process@pti@ns{%
  \@for\CurrentOption:=\@curroptions\do{%
    \@ifundefined{ds@\KVO@SanitizedCurrentOption}{%
      \KVO@use@ption
      \default@ds
    }%
    \KVO@use@ption
  }%
  \@for\CurrentOption:=\@declaredoptions\do{%
    \expandafter\let\csname ds@\CurrentOption\endcsname\relax
  }%
  \let\CurrentOption\@empty
  \let\@fileswith@pti@ns\@@fileswith@pti@ns
  \AtEndOfPackage{\let\@unprocessedoptions\relax}%
}
%    \end{macrocode}
%
%    \begin{macrocode}
\def\KVO@use@ption{%
  \begingroup
    \edef\x{\endgroup
      \noexpand\@removeelement{%
        \detokenize\expandafter{\CurrentOption}%
      }{%
        \detokenize\expandafter{\@unusedoptionlist}%
      }%
    }%
  \x\@unusedoptionlist
  \csname ds@\KVO@SanitizedCurrentOption\endcsname
}
%    \end{macrocode}
%
%    \begin{macrocode}
\def\OptionNotUsed{%
  \ifx\@currext\@clsextension
    \xdef\@unusedoptionlist{%
      \ifx\@unusedoptionlist\@empty
      \else
        \detokenize\expandafter{\@unusedoptionlist,}%
      \fi
      \detokenize\expandafter{\CurrentOption}%
    }%
  \fi
}
%    \end{macrocode}
%
%    Variant of \cs{ExecuteOptions} that better protects \cs{CurrentOption}.
%    \begin{macrocode}
\def\CurrentOption@SaveLevel{0}
\def\ExecuteOptions{%
  \expandafter\KVO@ExecuteOptions
      \csname CurrentOption@\CurrentOption@SaveLevel\endcsname
}
\def\KVO@ExecuteOptions#1#2{%
  \let#1\CurrentOption
  \edef\CurrentOption@SaveLevel{%
    \the\numexpr\CurrentOption@SaveLevel+1%
  }%
  \@for\CurrentOption:=#2\do{%
    \csname ds@\CurrentOption\endcsname
  }%
  \edef\CurrentOption@SaveLevel{%
    \the\numexpr\CurrentOption@SaveLevel-1%
  }%
  \let\CurrentOption#1%
}
%    \end{macrocode}
%
%    \begin{macrocode}
\def\KVO@fileswith@pti@ns#1[#2]#3[#4]{%
  \ifx#1\@clsextension
    \ifx\@classoptionslist\relax
      \KVO@normalize\KVO@temp{#2}%
      \expandafter\gdef\expandafter\@classoptionslist\expandafter{%
        \KVO@temp
      }%
      \def\reserved@a{%
        \KVO@onefilewithoptions{#3}[{#2}][{#4}]#1%
        \@documentclasshook
      }%
    \else
      \def\reserved@a{%
        \KVO@onefilewithoptions{#3}[{#2}][{#4}]#1%
      }%
    \fi
  \else
    \begingroup
      \let\KVO@temp\relax
      \let\KVO@onefilewithoptions\relax
      \let\@pkgextension\relax
      \def\reserved@b##1,{%
        \ifx\@nil##1\relax
        \else
          \ifx\relax##1\relax
          \else
            \KVO@onefilewithoptions{##1}[{\KVO@temp}][{#4}]%
            \@pkgextension
          \fi
          \expandafter\reserved@b
        \fi
      }%
      \edef\reserved@a{\zap@space#3 \@empty}%
      \edef\reserved@a{\expandafter\reserved@b\reserved@a,\@nil,}%
      \toks@{#2}%
      \def\KVO@temp{\the\toks@}%
    \edef\reserved@a{\endgroup \reserved@a}%
  \fi
  \reserved@a
}
%    \end{macrocode}
%
%    \begin{macrocode}
\def\KVO@onefilewithoptions#1[#2][#3]#4{%
  \@pushfilename
  \xdef\@currname{#1}%
  \global\let\@currext#4%
  \expandafter\let\csname\@currname.\@currext-h@@k\endcsname\@empty
  \let\CurrentOption\@empty
  \@reset@ptions
  \makeatletter
  \def\reserved@a{%
    \@ifl@aded\@currext{#1}{%
      \@if@ptions\@currext{#1}{#2}{%
      }{%
        \begingroup
          \@ifundefined{opt@#1.\@currext}{%
            \def\x{}%
          }{%
            \edef\x{%
              \expandafter\expandafter\expandafter\strip@prefix
              \expandafter\meaning\csname opt@#1.\@currext\endcsname
            }%
          }%
          \def\y{#2}%
          \edef\y{\expandafter\strip@prefix\meaning\y}%
          \@latex@error{Option clash for \@cls@pkg\space #1}{%
            The package #1 has already been loaded %
            with options:\MessageBreak
            \space\space[\x]\MessageBreak
            There has now been an attempt to load it %
             with options\MessageBreak
            \space\space[\y]\MessageBreak
            Adding the global options:\MessageBreak
            \space\space
                 \x,\y\MessageBreak
            to your \noexpand\documentclass declaration may fix this.%
            \MessageBreak
            Try typing \space <return> \space to proceed.%
          }%
        \endgroup
      }%
    }{%
      \@pass@ptions\@currext{#2}{#1}%
      \global\expandafter
      \let\csname ver@\@currname.\@currext\endcsname\@empty
      \InputIfFileExists
        {\@currname.\@currext}%
        {}%
        {\@missingfileerror\@currname\@currext}%
      \let\@unprocessedoptions\@@unprocessedoptions
      \csname\@currname.\@currext-h@@k\endcsname
      \expandafter\let\csname\@currname.\@currext-h@@k\endcsname
              \@undefined
      \@unprocessedoptions
    }%
    \@ifl@ter\@currext{#1}{#3}{%
    }{%
      \@latex@warning@no@line{%
        You have requested,\on@line, %
        version\MessageBreak
          #3' of \@cls@pkg\space #1,\MessageBreak
        but only version\MessageBreak
         `\csname ver@#1.\@currext\endcsname'\MessageBreak
        is available%
      }%
    }%
    \ifx\@currext\@clsextension\let\LoadClass\@twoloadclasserror\fi
    \@popfilename
    \@reset@ptions
  }%
  \reserved@a
}
%    \end{macrocode}
%
%    \begin{macrocode}
\def\@unknownoptionerror{%
  \@latex@error{%
    Unknown option `\KVO@SanitizedCurrentOption' %
    for \@cls@pkg\space`\@currname'%
  }{%
    The option `\KVO@SanitizedCurrentOption' was not declared in %
    \@cls@pkg\space`\@currname', perhaps you\MessageBreak
    misspelled its name. %
    Try typing \space <return> %
    \space to proceed.%
  }%
}
%    \end{macrocode}
%
%    \begin{macrocode}
\def\@@unprocessedoptions{%
  \ifx\@currext\@pkgextension
    \@ifundefined{opt@\@currname.\@currext}{%
      \let\@curroptions\@empty
    }{%
      \expandafter\let\expandafter\@curroptions
          \csname opt@\@currname.\@currext\endcsname
    }%
    \@for\CurrentOption:=\@curroptions\do{%
        \ifx\CurrentOption\@empty\else\@unknownoptionerror\fi
    }%
  \fi
}
%    \end{macrocode}
%
%    \begin{macrocode}
\def\KVO@SanitizedCurrentOption{%
  \expandafter\strip@prefix\meaning\CurrentOption
}
%    \end{macrocode}
%
%    Normalize option list.
%    \begin{macrocode}
\def\KVO@normalize#1#2{%
  \let\KVO@result\@empty
  \KVO@splitcomma#2,\@nil
  \let#1\KVO@result
}
\def\KVO@splitcomma#1,#2\@nil{%
  \KVO@ifempty{#1}{}{%
    \KVO@checkkv#1=\@nil
  }%
  \KVO@ifempty{#2}{}{\KVO@splitcomma#2\@nil}%
}
\def\KVO@ifempty#1{%
  \expandafter\ifx\expandafter\\\detokenize{#1}\\%
    \expandafter\@firstoftwo
  \else
    \expandafter\@secondoftwo
  \fi
}
\def\KVO@checkkv#1=#2\@nil{%
  \KVO@ifempty{#2}{%
    % option without value
    \edef\KVO@x{\zap@space#1 \@empty}%
    \ifx\KVO@x\@empty
      % ignore empty option
    \else
      % append to list
      \edef\KVO@result{%
        \etex@unexpanded\expandafter{\KVO@result},\KVO@x
      }%
    \fi
  }{%
    % #1: "key", #2: "value="
    % add key part
    \edef\KVO@result{%
      \etex@unexpanded\expandafter{\KVO@result},%
      \zap@space#1 \@empty
    }%
    \futurelet\@let@token\KVO@checkfirsttok#2 \@nil| = \@nil|\KVO@nil
  }%
}
\def\KVO@checkfirsttok{%
  \ifx\@let@token\bgroup
    % no space at start
    \expandafter\KVO@removelastspace\expandafter=%
    % "<value><spaceopt>= \@nil"
  \else
    \expandafter\KVO@checkfirstA
  \fi
}
\def\KVO@checkfirstA#1 #2\@nil{%
  \KVO@ifempty{#2}{%
    \KVO@removelastspace=#1 \@nil
  }{%
    \KVO@ifempty{#1}{%
      \KVO@removelastspace=#2\@nil
    }{%
      \KVO@removelastspace=#1 #2\@nil
    }%
  }%
}
\def\KVO@removelastspace#1 = \@nil|#2\KVO@nil{%
  \KVO@ifempty{#2}{%
    \edef\KVO@result{%
      \etex@unexpanded\expandafter{\KVO@result}%
      \etex@unexpanded\expandafter{\KVO@removegarbage#1\KVO@nil}%
    }%
  }{%
    \edef\KVO@result{%
      \etex@unexpanded\expandafter{\KVO@result}%
      \etex@unexpanded{#1}%
    }%
  }%
}
\def\KVO@removegarbage#1= \@nil#2\KVO@nil{#1}%
%    \end{macrocode}
%
%    Arguments |#1| and |#2| are macros.
%    \begin{macrocode}
\def\KVO@removeelement#1#2{%
  \begingroup
    \toks@={}%
    \@for\x:=#2\do{%
      \ifx\x\@empty
      \else
        \ifx\x#1\relax
        \else
          \edef\t{\the\toks@}%
          \ifx\t\@empty
          \else
            \toks@\expandafter{\the\toks@,}%
          \fi
          \toks@\expandafter{\the\expandafter\toks@\x}%
        \fi
      \fi
    }%
    \edef\x{\endgroup
      \def\noexpand#2{\the\toks@}%
    }%
  \x
}
%    \end{macrocode}
%
%
%    \begin{macrocode}
\let\@@fileswith@pti@ns\KVO@fileswith@pti@ns
\ifx\@fileswith@pti@ns\@badrequireerror
\else
  \let\@fileswith@pti@ns\KVO@fileswith@pti@ns
\fi
%    \end{macrocode}
%
%    \begin{macro}{\KVO@Patch}
%    \begin{macrocode}
\let\KVO@Patch=Y
%    \end{macrocode}
%    \end{macro}
%
%    \begin{macrocode}
\KVO@AtEnd%
%</patch>
%    \end{macrocode}
%
% \section{Test}
%
% \subsection{Preface for standard catcode check}
%
%    \begin{macrocode}
%<*test1>
\input miniltx.tex\relax
%</test1>
%    \end{macrocode}
%
% \subsection{Catcode checks for loading}
%
%    \begin{macrocode}
%<*test1>
%    \end{macrocode}
%    \begin{macrocode}
\catcode`\{=1 %
\catcode`\}=2 %
\catcode`\#=6 %
\catcode`\@=11 %
\expandafter\ifx\csname count@\endcsname\relax
  \countdef\count@=255 %
\fi
\expandafter\ifx\csname @gobble\endcsname\relax
  \long\def\@gobble#1{}%
\fi
\expandafter\ifx\csname @firstofone\endcsname\relax
  \long\def\@firstofone#1{#1}%
\fi
\expandafter\ifx\csname loop\endcsname\relax
  \expandafter\@firstofone
\else
  \expandafter\@gobble
\fi
{%
  \def\loop#1\repeat{%
    \def\body{#1}%
    \iterate
  }%
  \def\iterate{%
    \body
      \let\next\iterate
    \else
      \let\next\relax
    \fi
    \next
  }%
  \let\repeat=\fi
}%
\def\RestoreCatcodes{}
\count@=0 %
\loop
  \edef\RestoreCatcodes{%
    \RestoreCatcodes
    \catcode\the\count@=\the\catcode\count@\relax
  }%
\ifnum\count@<255 %
  \advance\count@ 1 %
\repeat

\def\RangeCatcodeInvalid#1#2{%
  \count@=#1\relax
  \loop
    \catcode\count@=15 %
  \ifnum\count@<#2\relax
    \advance\count@ 1 %
  \repeat
}
\def\RangeCatcodeCheck#1#2#3{%
  \count@=#1\relax
  \loop
    \ifnum#3=\catcode\count@
    \else
      \errmessage{%
        Character \the\count@\space
        with wrong catcode \the\catcode\count@\space
        instead of \number#3%
      }%
    \fi
  \ifnum\count@<#2\relax
    \advance\count@ 1 %
  \repeat
}
\def\space{ }
\expandafter\ifx\csname LoadCommand\endcsname\relax
  \def\LoadCommand{\input kvoptions.sty\relax}%
\fi
\def\Test{%
  \RangeCatcodeInvalid{0}{47}%
  \RangeCatcodeInvalid{58}{64}%
  \RangeCatcodeInvalid{91}{96}%
  \RangeCatcodeInvalid{123}{255}%
  \catcode`\@=12 %
  \catcode`\\=0 %
  \catcode`\%=14 %
  \LoadCommand
  \RangeCatcodeCheck{0}{36}{15}%
  \RangeCatcodeCheck{37}{37}{14}%
  \RangeCatcodeCheck{38}{47}{15}%
  \RangeCatcodeCheck{48}{57}{12}%
  \RangeCatcodeCheck{58}{63}{15}%
  \RangeCatcodeCheck{64}{64}{12}%
  \RangeCatcodeCheck{65}{90}{11}%
  \RangeCatcodeCheck{91}{91}{15}%
  \RangeCatcodeCheck{92}{92}{0}%
  \RangeCatcodeCheck{93}{96}{15}%
  \RangeCatcodeCheck{97}{122}{11}%
  \RangeCatcodeCheck{123}{255}{15}%
  \RestoreCatcodes
}
\Test
\csname @@end\endcsname
\end
%    \end{macrocode}
%    \begin{macrocode}
%</test1>
%    \end{macrocode}
%
%    \begin{macrocode}
%<*test2>
\NeedsTeXFormat{LaTeX2e}
\makeatletter
\catcode`\@=11 %
\def\RestoreCatcodes{}
\count@=0 %
\loop
  \edef\RestoreCatcodes{%
    \RestoreCatcodes
    \catcode\the\count@=\the\catcode\count@\relax
  }%
\ifnum\count@<255 %
  \advance\count@\@ne
\repeat

\def\RangeCatcodeInvalid#1#2{%
  \count@=#1\relax
  \loop
    \catcode\count@=15 %
  \ifnum\count@<#2\relax
    \advance\count@\@ne
  \repeat
}
\def\Test#1{%
  \RangeCatcodeInvalid{0}{47}%
  \RangeCatcodeInvalid{58}{64}%
  \RangeCatcodeInvalid{91}{96}%
  \RangeCatcodeInvalid{123}{255}%
  \catcode`\@=12 %
  \catcode`\\=0 %
  \catcode`\{=1 %
  \catcode`\}=2 %
  \catcode`\#=6 %
  \catcode`\[=12 %
  \catcode`\]=12 %
  \catcode`\%=14 %
  \catcode`\ =10 %
  \catcode13=5 %
  #1\relax
  \RestoreCatcodes
}
\Test{\RequirePackage{kvoptions-patch}}%
\Test{\RequirePackage{kvoptions}}%
\csname @@end\endcsname
%</test2>
%    \end{macrocode}
%
%    \begin{macrocode}
%<*test3>
\NeedsTeXFormat{LaTeX2e}
\makeatletter
\RequirePackage{kvoptions}[2016/05/16]
\def\msg#{\immediate\write16}
\define@key{testfamily}{testkey}{%
  \msg{[testfamily/testkey/#1]}%
}
\define@key{testfamily}{testdefaultkey}[testdefault]{%
  \msg{[testfamily/testdefaultkey/#1]}%
}
\AddToKeyvalOption{testfamily}{testkey}{%
  \msg{[addition/#1]}%
}
\AddToKeyvalOption{testfamily}{testdefaultkey}{%
  \msg{[addition/#1]}%
}
\setkeys{testfamily}{%
  testkey=testA,%
  testdefaultkey=testB,%
  testdefaultkey,%
}
\SetupKeyvalOptions{%
  family=testfamily%
}
\AddToKeyvalOption*{testkey}{%
  \msg{[star addition/#1]}%
}
\AddToKeyvalOption*{testdefaultkey}{%
  \msg{[star addition/#1]}%
}
\setkeys{testfamily}{%
  testkey=testA,%
  testdefaultkey=testB,%
  testdefaultkey,%
}
\@@end
%</test3>
%    \end{macrocode}
%
%    \begin{macrocode}
%<*test4pkg>
\NeedsTeXFormat{LaTeX2e}
\ProvidesPackage{kvoptions-test4}[2016/05/16 package for testing]
\RequirePackage{kvoptions}[2016/05/16]
\SetupKeyvalOptions{%
  family=FOO,%
  prefix=foo,%
  setkeys=\kvsetkeys,%
}
\DeclareStringOption{str}
\define@key{FOO}{set}{%
  \setkeys{BAR}{strbar={#1}}%
}
\define@key{BAR}{strbar}{%
  \def\foostr{[BAR:#1]}%
}
\ProcessKeyvalOptions*
%</test4pkg>
%    \end{macrocode}
%    \begin{macrocode}
%<*test4>
\NeedsTeXFormat{LaTeX2e}
\ProvidesFile{kvoptions-test4.tex}[2016/05/16 test file]
\RequirePackage[%
  str=A,set=B,str=C,%
]{kvoptions-test4}[2016/05/16]
\def\TestExpected{C}
\ifx\foostr\TestExpected
  \typeout{* Test ok.}%
\else
  \typeout{* Result: [\foostr]}%
  \typeout{* Expected: [\TestExpected]}%
  \errmessage{Test failed!}%
\fi
\csname @@end\endcsname\end
%</test4>
%    \end{macrocode}
%
% \section{Installation}
%
% \subsection{Download}
%
% \paragraph{Package.} This package is available on
% CTAN\footnote{\CTANpkg{kvoptions}}:
% \begin{description}
% \item[\CTAN{macros/latex/contrib/oberdiek/kvoptions.dtx}] The source file.
% \item[\CTAN{macros/latex/contrib/oberdiek/kvoptions.pdf}] Documentation.
% \end{description}
%
%
% \paragraph{Bundle.} All the packages of the bundle `oberdiek'
% are also available in a TDS compliant ZIP archive. There
% the packages are already unpacked and the documentation files
% are generated. The files and directories obey the TDS standard.
% \begin{description}
% \item[\CTANinstall{install/macros/latex/contrib/oberdiek.tds.zip}]
% \end{description}
% \emph{TDS} refers to the standard ``A Directory Structure
% for \TeX\ Files'' (\CTAN{tds/tds.pdf}). Directories
% with \xfile{texmf} in their name are usually organized this way.
%
% \subsection{Bundle installation}
%
% \paragraph{Unpacking.} Unpack the \xfile{oberdiek.tds.zip} in the
% TDS tree (also known as \xfile{texmf} tree) of your choice.
% Example (linux):
% \begin{quote}
%   |unzip oberdiek.tds.zip -d ~/texmf|
% \end{quote}
%
% \paragraph{Script installation.}
% Check the directory \xfile{TDS:scripts/oberdiek/} for
% scripts that need further installation steps.
%
% \subsection{Package installation}
%
% \paragraph{Unpacking.} The \xfile{.dtx} file is a self-extracting
% \docstrip\ archive. The files are extracted by running the
% \xfile{.dtx} through \plainTeX:
% \begin{quote}
%   \verb|tex kvoptions.dtx|
% \end{quote}
%
% \paragraph{TDS.} Now the different files must be moved into
% the different directories in your installation TDS tree
% (also known as \xfile{texmf} tree):
% \begin{quote}
% \def\t{^^A
% \begin{tabular}{@{}>{\ttfamily}l@{ $\rightarrow$ }>{\ttfamily}l@{}}
%   kvoptions.sty & tex/latex/oberdiek/kvoptions.sty\\
%   kvoptions-patch.sty & tex/latex/oberdiek/kvoptions-patch.sty\\
%   kvoptions.pdf & doc/latex/oberdiek/kvoptions.pdf\\
%   example-mycolorsetup.sty & doc/latex/oberdiek/example-mycolorsetup.sty\\
%   test/kvoptions-test1.tex & doc/latex/oberdiek/test/kvoptions-test1.tex\\
%   test/kvoptions-test2.tex & doc/latex/oberdiek/test/kvoptions-test2.tex\\
%   test/kvoptions-test3.tex & doc/latex/oberdiek/test/kvoptions-test3.tex\\
%   test/kvoptions-test4.tex & doc/latex/oberdiek/test/kvoptions-test4.tex\\
%   test/kvoptions-test4.sty & doc/latex/oberdiek/test/kvoptions-test4.sty\\
%   kvoptions.dtx & source/latex/oberdiek/kvoptions.dtx\\
% \end{tabular}^^A
% }^^A
% \sbox0{\t}^^A
% \ifdim\wd0>\linewidth
%   \begingroup
%     \advance\linewidth by\leftmargin
%     \advance\linewidth by\rightmargin
%   \edef\x{\endgroup
%     \def\noexpand\lw{\the\linewidth}^^A
%   }\x
%   \def\lwbox{^^A
%     \leavevmode
%     \hbox to \linewidth{^^A
%       \kern-\leftmargin\relax
%       \hss
%       \usebox0
%       \hss
%       \kern-\rightmargin\relax
%     }^^A
%   }^^A
%   \ifdim\wd0>\lw
%     \sbox0{\small\t}^^A
%     \ifdim\wd0>\linewidth
%       \ifdim\wd0>\lw
%         \sbox0{\footnotesize\t}^^A
%         \ifdim\wd0>\linewidth
%           \ifdim\wd0>\lw
%             \sbox0{\scriptsize\t}^^A
%             \ifdim\wd0>\linewidth
%               \ifdim\wd0>\lw
%                 \sbox0{\tiny\t}^^A
%                 \ifdim\wd0>\linewidth
%                   \lwbox
%                 \else
%                   \usebox0
%                 \fi
%               \else
%                 \lwbox
%               \fi
%             \else
%               \usebox0
%             \fi
%           \else
%             \lwbox
%           \fi
%         \else
%           \usebox0
%         \fi
%       \else
%         \lwbox
%       \fi
%     \else
%       \usebox0
%     \fi
%   \else
%     \lwbox
%   \fi
% \else
%   \usebox0
% \fi
% \end{quote}
% If you have a \xfile{docstrip.cfg} that configures and enables \docstrip's
% TDS installing feature, then some files can already be in the right
% place, see the documentation of \docstrip.
%
% \subsection{Refresh file name databases}
%
% If your \TeX~distribution
% (\TeX\,Live, \mikTeX, \dots) relies on file name databases, you must refresh
% these. For example, \TeX\,Live\ users run \verb|texhash| or
% \verb|mktexlsr|.
%
% \subsection{Some details for the interested}
%
% \paragraph{Unpacking with \LaTeX.}
% The \xfile{.dtx} chooses its action depending on the format:
% \begin{description}
% \item[\plainTeX:] Run \docstrip\ and extract the files.
% \item[\LaTeX:] Generate the documentation.
% \end{description}
% If you insist on using \LaTeX\ for \docstrip\ (really,
% \docstrip\ does not need \LaTeX), then inform the autodetect routine
% about your intention:
% \begin{quote}
%   \verb|latex \let\install=y\input{kvoptions.dtx}|
% \end{quote}
% Do not forget to quote the argument according to the demands
% of your shell.
%
% \paragraph{Generating the documentation.}
% You can use both the \xfile{.dtx} or the \xfile{.drv} to generate
% the documentation. The process can be configured by the
% configuration file \xfile{ltxdoc.cfg}. For instance, put this
% line into this file, if you want to have A4 as paper format:
% \begin{quote}
%   \verb|\PassOptionsToClass{a4paper}{article}|
% \end{quote}
% An example follows how to generate the
% documentation with pdf\LaTeX:
% \begin{quote}
%\begin{verbatim}
%pdflatex kvoptions.dtx
%makeindex -s gind.ist kvoptions.idx
%pdflatex kvoptions.dtx
%makeindex -s gind.ist kvoptions.idx
%pdflatex kvoptions.dtx
%\end{verbatim}
% \end{quote}
%
% \begin{thebibliography}{99}
% \bibitem{tb94wright}
%   A guide to key-value methods, Joseph Wright, second draft for
%   \href{https://www.tug.org/tugboat}{TUGBoat}, 2009-03-17.
%   \url{https://www.texdev.net/uploads/2009/03/keyval.pdf}
% \bibpackage{ifthen}{David Carlisle}{2001/05/26}
%   {\CTANpkg{ifthen}}
% \bibpackage{helvet}{Sebastian Rahtz, Walter Schmidt}{2004/01/26}
%   {\CTANpkg{psfonts}}
% \bibpackage{hyperref}{Sebastian Rahtz, Heiko Oberdiek}{2006/02/12}
%   {\CTANpkg{hyperref}}
% \bibpackage{keyval}{David Carlisle}{1999/03/16}
%   {\CTANpkg{keyval}}
% \bibpackage{multicol}{Frank Mittelbach}{2004/02/14}
%   {\CTANpkg{multicol}}
% \bibpackage{tabularx}{David Carlisle}{1999/01/07}
%   {\CTANpkg{tabularx}}
% \bibpackage{tracefnt}{Frank Mittelbach, Rainer Sch\"opf}{1997/05/29}
%   {\CTANpkg{latex-base}}
% \bibpackage{xkeyval}{Hendri Adriaens}{2005/05/07}
%   {\CTANpkg{xkeyval}}
% \bibitem{clsguide}
%   The \LaTeX3 Project, \textit{\LaTeXe\ for class and package writers},
%   2003/12/09.
%   \CTANpkg{clsguide}
%
% \end{thebibliography}
%
% \begin{History}
%   \begin{Version}{0000/00/00 v0.0}
%   \item
%     Probably David Carlisle's code in \xpackage{hyperref}
%     was the start.
%   \end{Version}
%   \begin{Version}{2004/02/22 v1.0}
%   \item
%     The first version was never published. It also
%     has offered a patch to get rid of \LaTeX's option
%     expansion.
%   \end{Version}
%   \begin{Version}{2006/02/16 v2.0}
%   \item
%     Now the package is redesigned with an easier
%     user interface.
%   \item
%     \cs{ProcessKeyvalOptions} remains the central service, inherited
%     from \xpackage{hyperref}'s \cs{ProcessOptionsWithKV}.
%     Now the use inside classes is also supported.
%   \item
%     Provides help macros for boolean and simple string options.
%   \item
%     Fixes for the patch of \LaTeX{}. The patch is only enabled,
%     if the user requests it.
%   \end{Version}
%   \begin{Version}{2006/02/20 v2.1}
%   \item
%     Unused option list is sanitized to prevent problems
%     with other packages that uses own processing methods
%     for key value options. Disadvantage: the unused global
%     option detection is weakened.
%   \item
%     New option type by \cs{DeclareVoidOption} for options without
%     value.
%   \item
%     Default rule by \cs{DeclareDefaultOption}.
%   \item
%     Dynamic options: \cs{DisableKeyvalOption}.
%   \end{Version}
%   \begin{Version}{2006/06/01 v2.2}
%   \item
%     Fixes for option \xoption{patch}.
%   \end{Version}
%   \begin{Version}{2006/08/17 v2.3}
%   \item
%     \cs{DeclareBooleanOption} renamed to \cs{DeclareBoolOption}
%     to avoid a name clash with package \cs{ifoption}.
%   \end{Version}
%   \begin{Version}{2006/08/22 v2.4}
%   \item
%     Option \xoption{patch}: \cs{ExecuteOptions} does not
%     change the meaning of macro \cs{CurrentOption} at all.
%   \end{Version}
%   \begin{Version}{2007/04/11 v2.5}
%   \item
%     Line ends sanitized.
%   \end{Version}
%   \begin{Version}{2007/05/06 v2.6}
%   \item
%     Uses package \xpackage{etexcmds}.
%   \end{Version}
%   \begin{Version}{2007/06/11 v2.7}
%   \item
%     The patch part fixes LaTeX bug latex/3965.
%   \end{Version}
%   \begin{Version}{2007/10/02 v2.8}
%   \item
%     Compatibility for \plainTeX\ added.
%   \item
%     Typos in documentation fixed (Axel Sommerfeldt).
%   \end{Version}
%   \begin{Version}{2007/10/11 v2.9}
%   \item
%     Bug fix for option \xoption{patch}.
%   \end{Version}
%   \begin{Version}{2007/10/18 v3.0}
%   \item
%     New package \xpackage{kvoptions-patch}.
%   \end{Version}
%   \begin{Version}{2009/04/10 v3.1}
%   \item
%     Space by line end removed in definition of internal macro.
%   \end{Version}
%   \begin{Version}{2009/07/17 v3.2}
%   \item
%     \cs{ProcessLocalKeyvalOptions} added.
%   \item
%     \cs{DisableKeyvalOption} with the \texttt{action=ignore} option
%     fixed (Joseph Wright).
%   \end{Version}
%   \begin{Version}{2009/07/21 v3.3}
%   \item
%     \cs{DeclareLocalOption}, \cs{DeclareLocalOptions} added.
%   \end{Version}
%   \begin{Version}{2009/08/13 v3.4}
%   \item
%     Documentation addition: recommendation for Joseph Wright's
%     review article.
%   \item
%     Documentation addition: local/global options.
%   \end{Version}
%   \begin{Version}{2009/12/04 v3.5}
%   \item
%     \cs{AddToKeyvalOption} added.
%   \end{Version}
%   \begin{Version}{2009/12/08 v3.6}
%   \item
%     Fix: If a default handler is configured, it is now also
%     called for classes.
%   \end{Version}
%   \begin{Version}{2010/02/22 v3.7}
%   \item
%     Missing space in error message added.
%   \end{Version}
%   \begin{Version}{2010/07/23 v3.8}
%   \item
%     Documenation for package \xpackage{kvoptions-patch} improved.
%     No code changes.
%   \end{Version}
%   \begin{Version}{2010/12/02 v3.9}
%   \item
%     Key \texttt{setkeys} added for \cs{SetupKeyvalOptions}.
%   \end{Version}
%   \begin{Version}{2010/12/23 v3.10}
%   \item
%     \cs{DeclareVoidOption} also parses the second parameter as
%     \hologo{TeX} argument to improve compatibility with
%     \cs{DeclareOption}.
%   \end{Version}
%   \begin{Version}{2011/06/30 v3.11}
%   \item
%     Fix because of design bug in package \xpackage{xkeyval} that
%     removes global options with equal signs.
%   \end{Version}
%   \begin{Version}{2016/05/16 v3.12}
%   \item
%     Documentation updates.
%   \end{Version}
% \end{History}
%
% \PrintIndex
%
% \Finale
\endinput

%        (quote the arguments according to the demands of your shell)
%
% Documentation:
%    (a) If kvoptions.drv is present:
%           latex kvoptions.drv
%    (b) Without kvoptions.drv:
%           latex kvoptions.dtx; ...
%    The class ltxdoc loads the configuration file ltxdoc.cfg
%    if available. Here you can specify further options, e.g.
%    use A4 as paper format:
%       \PassOptionsToClass{a4paper}{article}
%
%    Programm calls to get the documentation (example):
%       pdflatex kvoptions.dtx
%       makeindex -s gind.ist kvoptions.idx
%       pdflatex kvoptions.dtx
%       makeindex -s gind.ist kvoptions.idx
%       pdflatex kvoptions.dtx
%
% Installation:
%    TDS:tex/latex/oberdiek/kvoptions.sty
%    TDS:tex/latex/oberdiek/kvoptions-patch.sty
%    TDS:doc/latex/oberdiek/kvoptions.pdf
%    TDS:doc/latex/oberdiek/example-mycolorsetup.sty
%    TDS:doc/latex/oberdiek/test/kvoptions-test1.tex
%    TDS:doc/latex/oberdiek/test/kvoptions-test2.tex
%    TDS:doc/latex/oberdiek/test/kvoptions-test3.tex
%    TDS:doc/latex/oberdiek/test/kvoptions-test4.tex
%    TDS:doc/latex/oberdiek/test/kvoptions-test4.sty
%    TDS:source/latex/oberdiek/kvoptions.dtx
%
%<*ignore>
\begingroup
  \catcode123=1 %
  \catcode125=2 %
  \def\x{LaTeX2e}%
\expandafter\endgroup
\ifcase 0\ifx\install y1\fi\expandafter
         \ifx\csname processbatchFile\endcsname\relax\else1\fi
         \ifx\fmtname\x\else 1\fi\relax
\else\csname fi\endcsname
%</ignore>
%<*install>
\input docstrip.tex
\Msg{************************************************************************}
\Msg{* Installation}
\Msg{* Package: kvoptions 2016/05/16 v3.12 Key value format for package options (HO)}
\Msg{************************************************************************}

\keepsilent
\askforoverwritefalse

\let\MetaPrefix\relax
\preamble

This is a generated file.

Project: kvoptions
Version: 2016/05/16 v3.12

Copyright (C)
   2004, 2006, 2007, 2009-2011 Heiko Oberdiek
   2016-2019 Oberdiek Package Support Group

This work may be distributed and/or modified under the
conditions of the LaTeX Project Public License, either
version 1.3c of this license or (at your option) any later
version. This version of this license is in
   https://www.latex-project.org/lppl/lppl-1-3c.txt
and the latest version of this license is in
   https://www.latex-project.org/lppl.txt
and version 1.3 or later is part of all distributions of
LaTeX version 2005/12/01 or later.

This work has the LPPL maintenance status "maintained".

The Current Maintainers of this work are
Heiko Oberdiek and the Oberdiek Package Support Group
https://github.com/ho-tex/oberdiek/issues


This work consists of the main source file kvoptions.dtx
and the derived files
   kvoptions.sty, kvoptions.pdf, kvoptions.ins, kvoptions.drv,
   kvoptions-patch.sty, example-mycolorsetup.sty,
   kvoptions-test1.tex, kvoptions-test2.tex,
   kvoptions-test3.tex, kvoptions-test4.tex,
   kvoptions-test4.sty.

\endpreamble
\let\MetaPrefix\DoubleperCent

\generate{%
  \file{kvoptions.ins}{\from{kvoptions.dtx}{install}}%
  \file{kvoptions.drv}{\from{kvoptions.dtx}{driver}}%
  \usedir{tex/latex/oberdiek}%
  \file{kvoptions.sty}{\from{kvoptions.dtx}{package}}%
  \file{kvoptions-patch.sty}{\from{kvoptions.dtx}{patch}}%
  \usedir{doc/latex/oberdiek}%
  \file{example-mycolorsetup.sty}{\from{kvoptions.dtx}{example}}%
%  \usedir{doc/latex/oberdiek/test}%
%  \file{kvoptions-test1.tex}{\from{kvoptions.dtx}{test1}}%
%  \file{kvoptions-test2.tex}{\from{kvoptions.dtx}{test2}}%
%  \file{kvoptions-test3.tex}{\from{kvoptions.dtx}{test3}}%
%  \file{kvoptions-test4.tex}{\from{kvoptions.dtx}{test4}}%
%  \file{kvoptions-test4.sty}{\from{kvoptions.dtx}{test4pkg}}%
  \nopreamble
  \nopostamble
%  \usedir{source/latex/oberdiek/catalogue}%
%  \file{kvoptions.xml}{\from{kvoptions.dtx}{catalogue}}%
}

\catcode32=13\relax% active space
\let =\space%
\Msg{************************************************************************}
\Msg{*}
\Msg{* To finish the installation you have to move the following}
\Msg{* files into a directory searched by TeX:}
\Msg{*}
\Msg{*     kvoptions.sty, kvoptions-patch.sty}
\Msg{*}
\Msg{* To produce the documentation run the file `kvoptions.drv'}
\Msg{* through LaTeX.}
\Msg{*}
\Msg{* Happy TeXing!}
\Msg{*}
\Msg{************************************************************************}

\endbatchfile
%</install>
%<*ignore>
\fi
%</ignore>
%<*driver>
\NeedsTeXFormat{LaTeX2e}
\ProvidesFile{kvoptions.drv}%
  [2016/05/16 v3.12 Key value format for package options (HO)]%
\documentclass{ltxdoc}
\usepackage{holtxdoc}[2011/11/22]
\begin{document}
  \DocInput{kvoptions.dtx}%
\end{document}
%</driver>
% \fi
%
%
% \CharacterTable
%  {Upper-case    \A\B\C\D\E\F\G\H\I\J\K\L\M\N\O\P\Q\R\S\T\U\V\W\X\Y\Z
%   Lower-case    \a\b\c\d\e\f\g\h\i\j\k\l\m\n\o\p\q\r\s\t\u\v\w\x\y\z
%   Digits        \0\1\2\3\4\5\6\7\8\9
%   Exclamation   \!     Double quote  \"     Hash (number) \#
%   Dollar        \$     Percent       \%     Ampersand     \&
%   Acute accent  \'     Left paren    \(     Right paren   \)
%   Asterisk      \*     Plus          \+     Comma         \,
%   Minus         \-     Point         \.     Solidus       \/
%   Colon         \:     Semicolon     \;     Less than     \<
%   Equals        \=     Greater than  \>     Question mark \?
%   Commercial at \@     Left bracket  \[     Backslash     \\
%   Right bracket \]     Circumflex    \^     Underscore    \_
%   Grave accent  \`     Left brace    \{     Vertical bar  \|
%   Right brace   \}     Tilde         \~}
%
% \GetFileInfo{kvoptions.drv}
%
% \title{The \xpackage{kvoptions} package}
% \date{2016/05/16 v3.12}
% \author{Heiko Oberdiek\thanks
% {Please report any issues at \url{https://github.com/ho-tex/oberdiek/issues}}}
%
% \maketitle
%
% \begin{abstract}
% This package is intended for package authors who want to
% use options in key value format for their package options.
% \end{abstract}
%
% \tableofcontents
%
% \def\M#1{\texttt{\{}\meta{#1}\texttt{\}}}
% \def\O#1{\texttt{[}\meta{#1}\texttt{]}}
%
% \section{Introduction}
%
% First I want to recommend the very good review article
% ``A guide to key-value methods'' by Joseph Wright \cite{tb94wright}.
% It introduces the different key-value packages and compares them.
%
% \subsection{The beginning}
%
% This package \xpackage{kvoptions} addresses class or package writers
% that want to allow their users to specify options
% as key value pairs, e.g.:
% \begin{quote}
% |\documentclass[verbose=false,name=me]{myclass}|\\
% |\usepackage[format=print]{mylayout}|
% \end{quote}
% Prominent example is package \xpackage{hyperref},
% probably the first package that offers this service.
% It's \cs{ProcessOptionsWithKV} is often copied und
% used in other packages, e.g. package \xpackage{helvet}
% that uses this interface for its option \xoption{scaled}.
%
% However copying code is not the most modern software
% development technique. And \xpackage{hyperref}'s
% code for \cs{ProcessOptionsWithKV} was changed
% to fix bugs. The version used in other packages
% depends on the time of copying and the awareness
% of \xpackage{hyperref}'s changes.
% Now the code is sourced out into this package and
% available for other package or class writers.
%
% \subsection{Overview}
%
% Package \xpackage{kvoptions} connects package
% \xpackage{\emph{k}ey\emph{v}al}
% with \LaTeX's package and class \emph{options}:
% \begin{center}
% \def\T#1{\bfseries #1}
% \renewcommand{\arraystretch}{1.5}
% \begin{tabular}{l|l|l}
%   \T{Package \xpackage{keyval}}
%   & \T{Package \xpackage{kvoptions}}
%   & \T{\LaTeX\ kernel}
%   \\
% \hline
%     \cs{define@key}
%   &
%     \renewcommand*{\arraystretch}{1}^^A
%     \begin{tabular}[t]{@{}l@{}}
%       \cs{DeclareVoidOption}\\
%       \cs{DeclareStringOption}\\
%       \cs{DeclareBoolOption}\\
%       \cs{DeclareComplementaryOption}\\
%       \cs{DisableKeyvalOption}
%     \end{tabular}
%   & \cs{DeclareOption}
%   \\
% \hline
%   & \cs{DeclareDefaultOption}
%   & \cs{DeclareOption*}
%   \\
% \hline
%   & \cs{ProcessKeyvalOptions}
%   & \cs{ProcessOptions*}
%   \\
% \hline
%   & Option \xoption{patch}
%   &
%     \renewcommand*{\arraystretch}{1}^^A
%     \begin{tabular}[t]{@{}l@{}}
%       Class/package\\
%       option system
%     \end{tabular}
%   \\
% \hline
%   & \cs{SetupKeyvalOptions} &
%   \\
% \end{tabular}
% \end{center}
%
% \hypersetup{bookmarksopenlevel=3}
% \section{Usage}
%
% \subsection{Process options}
%
% \subsubsection{\cs{ProcessKeyvalOptions}}
%
% \begin{declcs}{ProcessKeyvalOptions}\,\M{family}\\
%   \cs{ProcessKeyvalOptions} \*
% \end{declcs}
% This command evaluates the global or local options of
% the package that are defined with \xpackage{keyval}'s
% interface within the family \meta{family}. It acts the
% same way as \LaTeX's \cs{ProcessOptions*}. In a package
% unknown global options are ignored, in a class they
% are added to the unknown option list. The known global
% options and all local options are passed to \xpackage{keyval}'s
% \cs{setkeys} command for executing the options.
% Unknown options are reported to the user by an error.
%
% If the family name happens to be the same as the name
% of the package or class where \cs{ProcessKeyvalOptions}
% is used or the family name has previously been setup by
% \cs{SetupKeyvalOptions}, then \cs{ProcessKeyvalOptions}
% knows the family name already and you can use the star form
% without mandatory argument.
%
% \subsubsection{\cs{ProcessLocalKeyvalOptions}}
%
% \begin{declcs}{ProcessLocalKeyvalOptions}\,\M{family}\\
%   \cs{ProcessLocalKeyvalOptions} \*
% \end{declcs}
% This macro has the same syntax and works similar as
% \cs{ProcessKeyvalOptions}. However it ignores
% global options and only processes the local package options.
% Therefore it only can be used inside a package.
% An error is thrown, if it is used inside a class.
%
% Neither of the following macros are necessary for
% \cs{ProcessKeyvalOptions{}}. They just help the package/class
% author in common tasks.
%
% \subsubsection{\cs{SetupKeyvalOptions}}\label{setup}
%
% \begin{declcs}{SetupKeyvalOptions}\,\texttt{\{}\\
%   \hspace{1.5em}|family|\,|=|\,\meta{family}|,|\\
%   \hspace{1.5em}|prefix|\,|=|\,\meta{prefix}\\
%   \hspace{1.5em}|setkeys|\,|=|\,\meta{setkeys command}\\
%   \texttt{\}}
% \end{declcs}
% This command allows to configure the default assumptions
% that are based on the current package or class name.
% \LaTeX\ remembers this name in \cs{@currname}. The syntax
% description of the default looks a little weird, therefor
% an example is given for a package or class named |foobar|.
% \begin{quote}
% \begin{tabular}{@{}l|ll|l@{}}
%   Key & Default & (example) & Used by\\
% \hline
%   \rule{0pt}{1.2\ht\csname @arstrutbox\endcsname}^^A
%   |family| & \meta{\cs{@currname}} & (|foobar|)
%       & \cs{ProcessKeyvalOptions*} \\
%     &&& \cs{DeclareBoolOption} \\
%     &&& \cs{DeclareStringOption} \\
% \hline
%   \rule{0pt}{1.2\ht\csname @arstrutbox\endcsname}^^A
%   |prefix| & \meta{\cs{@currname}}|@| & (|foobar@|)
%       & \cs{DeclareBoolOption} \\
%     &&& \cs{DeclareStringOption} \\
%     &&& \cs{DeclareVoidOption} \\
% \hline
%   \rule{0pt}{1.2\ht\csname @arstrutbox\endcsname}^^A
%   |setkeys| & \cs{setkeys} & (\cs{kvsetkeys})
%       & \cs{ProcessKeyvalOptions} \\
%     &&& \cs{ProcessLocalKeyvalOptions}
% \end{tabular}
% \end{quote}
% Key |setkeys| was added in version 3.9. The original
% \cs{setkeys} of package \xpackage{keyval} is not reentrant.
% If an option is processed by this \cs{setkeys}, then
% the option should not call \cs{setkeys} again with a
% different family. Otherwise the next options of the
% first \cs{setkeys} call are processed with the wrong
% family. With key |setkeys| the macro \cs{kvsetkeys}
% can be set that does not have the problem of the
% original \cs{setkeys} of package {keyval}.
%
% Probably \cs{setkeys} of package \xpackage{xkeyval}
% is safe in this respect. But I haven't made a full
% analysis. At least it does not have the problem
% of the original \cs{setkeys}.
%
% \subsection{Option declarations}
%
% The options for \cs{ProcessKeyvalOptions} are defined
% by \xpackage{keyval}'s \cs{define@key}.
% Common purposes of such keys are boolean switches, they
% enable or disable something. Or they store a name or some kind of
% string in a macro. The following commands help the user. He
% declares what he wants and \xpackage{kvoptions} take care of
% the key definition, resource allocation and initialization.
%
% In order to avoid name clashes of macro names, internal
% commands are prefixed. Both the prefix and the family
% name for the defined keys can be configured by
% \cs{SetupKeyvalOptions}.
%
% \subsubsection{\cs{DeclareStringOption}}
%
% \begin{declcs}{DeclareStringOption}\,\O{init}\,\M{key}\,\O{default}
% \end{declcs}
% A macro is created that remembers the value of the key \meta{key}.
% The name of the macro consists of the option name \meta{key} that
% is prefixed by the prefix (see \ref{setup}).
% The initial contents of the
% macro can be given by the first optional argument \meta{init}.
% The default is empty.
%
% The the option \meta{key} is defined. The option code
% just stores its value in the macro. If the optional argument
% at the end of \cs{DeclareStringOption} is given, then option
% \meta{key} is defined with the default \meta{default}.
%
% Example for a package with the following two lines:
% \begin{quote}
% |\ProvidesPackage{foobar}|\\
% |\DeclareStringOption[me]{name}|
% \end{quote}
% Then \cs{DeclareStringOption} defines the macro with content |me|,
% note \LaTeX\ complains if the name of the macro already exists:
% \begin{quote}
% |\newcommand*{\foobar@name}{me}|
% \end{quote}
% The option definition is similar to:
% \begin{quote}
% |\define@key{foobar}{name}{%|\\
% |  \renewcommand*{\foobar@name}{#1}%|\\
% |}|
% \end{quote}
%
% \subsubsection{\cs{DeclareBoolOption}}
%
% \begin{declcs}{DeclareBoolOption}\,\O{init}\,\M{key}
% \end{declcs}
% A boolean switch is generated, initialized by value \meta{init}
% and the corresponding key \meta{key} is defined. If the
% initialization value is not given, |false| is used as default.
%
% The internal actions of \cs{DeclareBoolOption} are shown below.
% The example is given for a package author who has the following
% two lines in his package/class:
% \begin{quote}
% |\ProvidesPackage{foobar}|\\
% |\DeclareBoolOption{verbose}|
% \end{quote}
% First a new switch is created:
% \begin{quote}
%   |\newif\iffoobar@verbose|
% \end{quote}
% and initialized:
% \begin{quote}
%   |\foobar@verbosefalse|
% \end{quote}
% Finally the key is defined:
% \begin{quote}
%   |\define@key{foobar}{verbose}[true]{|\dots|}|
% \end{quote}
% The option code configures the boolean option in the following way:
% If the author specifies |true| or |false| then the switch is
% turned on or off respectivly. Also the option can be given without
% explicit value. Then the switch is enabled.
% Other values are reported as errors.
%
% Now the switch is ready to use in the package/class, e.g.:
% \begin{quote}
%   |\iffoobar@verbose|\\
%   \qquad\textit{\% print verbose message}\\
%   |\else|\\
%   \qquad\textit{\% be quiet}\\
%   |\fi|
% \end{quote}
% Users of package \cs{ifthen} can use the switch as
% boolean:
% \begin{quote}
%   |\boolean{foobar@verbose}|
% \end{quote}
%
% \subsubsection{\cs{DeclareComplementaryOption}}
%
% \begin{declcs}{DeclareComplementaryOption}\,\M{key}\,\M{parent}
% \end{declcs}
% Sometimes contrasting names are used to characterize the
% two states of a boolean switch, for example \xoption{draft}
% vs.\@ \xoption{final}. Both options behave like boolean
% options but they do not need two different switches, they
% should share one. \cs{DeclareComplementaryOption} allows this.
% The option \meta{key} shares the switch of option \meta{parent}.
% Example:
% \begin{quote}
%   |\DeclareBoolOption{draft}|\\
%   |\DeclareComplementaryOption{final}{draft}|
% \end{quote}
% Then |final| sets the switch of \xoption{draft} to |false|, and
% |final=false| enables the \xoption{draft} switch.
%
% \subsubsection{\cs{DeclareVoidOption}}
%
% \begin{declcs}{DeclareVoidOption}\,\M{key}\,\M{code}
% \end{declcs}
% \cs{ProcessKeyvalOptions} can be extended to recognize
% options that are declared in traditional way by \cs{DeclareOption}.
% But in case of the error that the user specifies a value, then
% this option would not recognized as key value option because
% of \cs{DeclareOption} and not detected as traditional option
% because of the value part. The user would get an unknown option
% error, difficult to understand.
%
% \cs{DeclareVoidOption} solves
% this problem. It defines the option \meta{key} as key value option.
% If the user specifies a value, a warning is given and the
% value is ignored.
%
% The code part \meta{code} is stored in a macro.
% The name of the macro consists of the option name \meta{key} that
% is prefixed by the prefix (see \ref{setup}).
% If the option is set, the macro will be executed. During the
% execution \cs{CurrentOption} is available with the current key name.
%
% \subsubsection{\cs{DeclareDefaultOption}}
%
% \begin{declcs}{DeclareDefaultOption}\,\M{code}
% \end{declcs}
% \begin{sloppypar}
% This command does not define a specific key, it is the equivalent
% to \LaTeX's \cs{DeclareOption*}. It allows the specification
% of a default action \meta{code} that is invoked if an unknown option
% is found.
% While \meta{code} is called,
% macro \cs{CurrentOption} contains the current option string.
% In addition \cs{CurrentOptionValue} contains the value part
% if the option string is parsable as key value pair, otherwise
% it is \cs{relax}. \cs{CurrentOptionKey} contains the key of
% the key value pair, or the whole option string, if it misses
% the equal sign.
% \end{sloppypar}
%
% Inside packages typical default actions are to pass unknown
% options to another package. Or an error message can be thrown
% by \cs{@unknownoptionerror}. This is the original error
% message that \LaTeX\ gives for unkown package options.
% This error message is easier to understand for the user
% as the error message from package \xpackage{keyval} that
% is given otherwise.
%
% A Class ignores unknown options and puts them on the
% unused option list. Let \LaTeX\ do the job and just
% call \cs{OptionNotUsed}. Or the options can be passed to
% another class that is later loaded.
%
% \subsubsection{Local options}
%
% \begin{declcs}{DeclareLocalOption} \M{option}\\
%   \cs{DeclareLocalOptions} \M{option list}
% \end{declcs}
% Both macros mark package options as local options. That means
% that they are ignored by \cs{ProcessKeyvalOptions} if they are given
% as global options. \cs{DeclareLocalOptions} takes one option,
% \cs{DeclareLocalOptions} expects a comma separated list of options.
%
% \subsubsection{Dynamic options}
%
% Options of \LaTeX's package/class system are cleared
% in \cs{ProcessOptions}. They modify the static model
% of a package. For example, depending on option \xoption{bookmarks}
% package \xpackage{hyperref} loads differently.
%
% Options, however,
% defined by \xpackage{keyval}'s \cs{define@key} remain
% defined, if the options are processed by \cs{setkeys}.
% Therefore these options can also be used to model
% the dynamic behaviour of a package. For example, in
% \xpackage{hyperref} the link colors can be changed
% everywhere until the end in |\end{document}|.
%
% However package \xpackage{color} that adds color support
% is necessary and it
% cannot be loaded after |\begin{document}|. Option \xoption{colorlinks}
% that loads \xpackage{color} should be active until |\begin{document}|
% and die in some way if it is too late for loading packages.
% With \cs{DisableKeyvalOption} the package/class author can
% specify and configure the death of an option and controls
% the life period of the option.
%
% \subsubsection{\cs{DisableKeyvalOption}}
%
% \begin{declcs}{DisableKeyvalOption}\,\O{options}\,\M{family}\,\M{key}
%   \\[0.5ex]
%   \meta{options}:\\
%   \hspace{1.5em}^^A
%   \begin{tabular}[t]{@{}l@{ \texttt{=} }l@{\qquad}>{default: }l@{}}
%     |action| & |undef|, |warning|, |error|, or |ignore|
%     & |undef|
%     \\
%     \multicolumn{1}{@{}l}{|global| or |local|} & & |global|
%     \\
%     |package| or |class| & \meta{name}
%     \\
%   \end{tabular}^^A
% \end{declcs}
% \cs{DisableKeyvalOption} can be called to mark the end when
% the option \meta{key} is no longer useful. The behaviour of
% an option after its death can be configured by \xoption{action}:
% \begin{description}
% \item[\texttt{undef}:] The option will be undefined,
%   If it is called, \cs{setkeys} reports an error because
%   of unknown key.
% \item[\texttt{error} or \texttt{warning}:] Use of the
%  option will cause an error or warning message.
%  Also these actions require that exclusivly either the
%  package or class name is given in options \xoption{package}
%  or \xoption{class}.
% \item[\texttt{ignore}:] The use of the option will silently
%  be ignored.
% \end{description}
% The option's death can be limited to the end of the current
% group, if option \xoption{local} is given. Default is
% \xoption{global}.
%
% The package/class author can wish the end of the option
% already during the package loading, then he will have
% static behaviour. In case of dynamic options \cs{DisableKeyvalOption}
% can be executed everywhere, also outside the package. Therefore
% the family name and the package/class name is usually unknown for
% \cs{DisableKeyvalOption}. Therefore the argument for the family
% name is mandatory and for some actions the package/class name
% must be provided.
%
% Usually a macro would configure the option death, Example:
%\begin{quote}
%\begin{verbatim}
%\ProvidesPackage{foobar}
%\DeclareBoolOption{color}
%\DeclareStringOption[red]{emphcolor}
%\ProcessKeyvalOptions*
%
%\newcommand*{\foobar@DisableOption}[2]{%
%  \DisableKeyvalueOption[
%    action={#1},
%    package=foobar
%  ]{foobar}{#2}%
%}
%
%\iffoobar@color
%  \RequirePackage{color}
%  \renewcommand*{\emph}[1]{\textcolor{\foobar@emphcolor}{#1}}
%\else
%  % Option emphcolor is not wrong, if we have color support.
%  % otherwise the option has no effect, but we don't want to
%  % remove it. Therefore action 'ignore' is the best choice:
%  \foobar@DisableOption{ignore}{emphcolor}
%\fi
%% No we don't need the option 'color'.
%\foobar@DisableOption{warning}{color}
%
%% With color support option 'emphcolor' will dynamically
%% change the color of \emph statements.
%\end{verbatim}
%\end{quote}
%
% \subsubsection{\cs{AddToKeyvalOption}}
%
% \begin{declcs}{AddToKeyvalOption}\,\M{family}\,\M{key}\,\M{code}\\
%   \cs{AddToKeyvalOption} \*\,\M{key}\,\M{code}
% \end{declcs}
% The code for an existing key \meta{key} of family \meta{family} is
% extended by code \meta{code}. In the starred form the current
% family setting is used, see \cs{ProcessKeyvalOptions*}.
%
% \subsection{Global vs. local options}
%
% Options that are given for \cs{documentclass} are called
% global options. They are known to the class and all packages.
% A package may make use of a global option and marks it as
% used. The advantage for the user is the freedom to specify
% options both in the \cs{documentclass} or \cs{usepackage}
% commands.
%
% However global options are shared with the class options
% and options of all other packages. Thus there can be the
% same option with different semantics for different
% packages and classes. As example, package \xpackage{bookmark}
% knows option \xoption{open} that specifies whether the bookmarks
% are opened or closed initially. It's values are |true| or |false|.
% Since KOMA-Script version 3.00 the KOMA classes also introduces
% option \xoption{open} with values |right| and |any| and a
% complete different meaning.
%
% Such conflicts can be resolved by marking all or part of options
% as local by \cs{DeclareLocalOption} or \cs{DeclareLocalOptions}.
% Then the packages ignores global occurences of these options.
% Package \xpackage{kvoptions} provides two methods:
% \begin{itemize}
% \item \cs{ProcessLocalKeyvalOptions} automatically uses all options
%   as local options. It ignores all global options.
% \item \cs{DeclareLocalOption} or \cs{DeclareLocalOptions} marks
%   options as local options. \cs{ProcessKeyvalOptions} will then
%   ignore global occurences for these local options.
% \end{itemize}
% Since version 1.5 package \xpackage{bookmark} uses the latter
% method. It checks global and local option places for driver options
% and limits all other options as local options. Thus the class
% option \xoption{open} of KOMA-Script is not misread as
% option for package \xpackage{bookmark}.
%
% \subsection{Summary of internal macros}
%
% The \cs{Declare}\texttt{\dots Option} commands define
% macros, additionally to the macros generated by the key definition.
% These macros can be used by the package/class author.
% The name of the macros starts with the prefix \meta{prefix}
% that can be configured by \cs{SetupKeyvalOptions}.
% \begin{center}
% \def\prefix{\cs{}\meta{prefix}\meta{key}}^^A
% \def\Hline{^^A
%   \hline
%   \rule{0pt}{1.2\ht\csname @arstrutbox\endcsname}^^A
% }^^A
% \begin{tabular}{@{}l|>{\ttfamily}l|l@{}}
%   Declare \meta{key} & \textrm{Defined macro} & Description\\
% \Hline
%   \cs{DeclareStringOption}
%    & \prefix & holds the string\\
% \Hline
%   \cs{DeclareBoolOption}
%     & \cs{if}\meta{prefix}\meta{key}& boolean switch\\
%   &\prefix false& disable switch\\
%   &\prefix true& enable switch\\
% \Hline
%   \cs{DeclareComplementaryOption}
%     & \prefix false& enable parent switch\\
%     & \prefix true& disable parent switch\\
% \Hline
%   \cs{DeclareVoidOption}
%   & \prefix & holds the action
% \end{tabular}
% \end{center}
%
% \subsection{\plainTeX}
%
% Package \xpackage{keyval} is also usable in \plainTeX\ with
% the help of file \xfile{miniltx.tex}. Some features of this
% package \xpackage{kvoptions} might also be useful for \plainTeX.
% If \LaTeX\ is not found, \cs{ProcessKeyvalOptions} and option
% \xoption{patch} are disabled. Before using the option declaration
% commands \cs{Declare...Option}, \cs{SetupKeyvalOptions} must be
% used.
%
% \hypersetup{bookmarksopenlevel=1}
%
% \section{Example}
%
% The following example defined a package that serves some
% private color management. A boolean option \xoption{print} enables
% print mode without colors. An option \xoption{emph} redefines
% \cs{emph} to print in the given color. And the driver
% can be specified by option \xoption{driver}.
%
%    \begin{macrocode}
%<*example>
    % Package identification
    % -----------------------
\NeedsTeXFormat{LaTeX2e}
\ProvidesPackage{example-mycolorsetup}[2016/05/16 Managing my colors]

\RequirePackage{ifpdf}
\RequirePackage{kvoptions}

    % Option declarations
    % -------------------

\SetupKeyvalOptions{
  family=MCS,
  prefix=MCS@
}
    % Use a shorter family name and prefix

    % Option print
\DeclareBoolOption{print}
    % is the same as
    % \DeclareBoolOption[false]{print}

    % Option driver
\ifpdf
  \DeclareStringOption[pdftex]{driver}
\else
  \DeclareStringOption[dvips]{driver}
\fi

    % Alternative interface for driver options
\DeclareVoidOption{dvips}{\SetupDriver}
\DeclareVoidOption{dvipdfm}{\SetupDriver}
\DeclareVoidOption{pdftex}{\SetupDriver}
    % In \SetupDriver we take the current option \CurrentOption
    % and pass it to the driver option.
    % The \expandafter commands expand \CurrentOption at the
    % time, when \SetupDriver is executed and \CurrentOption
    % has the correct meaning.
\newcommand*{\SetupDriver}{%
  \expandafter\@SetupDriver\expandafter{\CurrentOption}%
}
\newcommand*{\@SetupDriver}[1]{%
  \setkeys{MCS}{driver={#1}}%
}

    % Option emph
    % An empty value means, we want to have no color for \emph.
    % If the user specifies option emph without value, the red is used.
\DeclareStringOption{emph}[red]
    % is the same as
    % \DeclareStringOption[]{emph}[red]

    % Default option rule
\DeclareDefaultOption{%
  \ifx\CurrentOptionValue\relax
    \PackageWarningNoLine{\@currname}{%
      Unknown option `\CurrentOption'\MessageBreak
      is passed to package `color'%
    }%
    % Pass the option to package color.
    % Again it is better to expand \CurrentOption.
    \expandafter\PassOptionsToPackage
    \expandafter{\CurrentOption}{color}%
  \else
    % Package color does not take options with values.
    % We provide the standard LaTeX error.
    \@unknownoptionerror
  \fi
}

    % Process options
    % ---------------
\ProcessKeyvalOptions*

    % Implementation depending on option values
    % -----------------------------------------
    % Code for print mode
\ifMCS@print
  \PassOptionsToPackage{monochrome}{color}
    % tells package color to use black and white
\fi

\RequirePackage[\MCS@driver]{color}
    % load package color with the correct driver

    % \emph setup
\ifx\MCS@emph\@empty
    % \@empty is a predefined macro with empty contents.
    % the option value of option emph is empty, thus
    % we do not want a redefinition of \emph.
\else
  \renewcommand*{\emph}[1]{%
    \textcolor{\MCS@emph}{#1}%
  }
\fi
%</example>
%    \end{macrocode}
%
% \section{Package options}
%
% The package \xpackage{kvoptions} knows two package options
% \xoption{patch} and \xoption{debugshow}.
% The options of package \xpackage{kvoptions} are intended
% for authors, not for package/class writers. Inside a package
% it is too late for option \xoption{patch} and \xoption{debugshow}
% enables some messages that are perhaps useful for the debugging
% phase. Also \LaTeX\ is unhappy if a package is loaded
% later again with options that are previously not given.
% Thus package and class authors, stay with |\RequirePackage{kvoptions}|
% without options.
%
% Option \xoption{patch} loads package \xpackage{kvoptions-patch}.
%
% \subsection{Package \xpackage{kvoptions-patch}}
%
% \LaTeX's system of package/class options has some severe limitations
% that especially affects the value part if options are used as
% pair of key and value.
% \begin{itemize}
% \item Spaces are removed, regardless where:
%   \begin{quote}
%     |\documentclass[box=0 0 400 600]{article}|
%   \end{quote}
%   Now each package will see |box=00400600| as global option.
% \item In the previous case also braces would not help:
%   \begin{quote}
%     |\documentclass[box={0 0 400 600}]{article}|
%   \end{quote}
%   The result is an error message:
%   \begin{quote}
%     |! LaTeX Error: Missing \begin{document}.|
%   \end{quote}
%   As local option, however, it works if the package
%   knows about key value options (By using this package, for example).
% \item The requirements on robustness are extremly high.
%   \LaTeX\ expands the option. All that will not work as environment
%   name will break also as option. Even a \cs{relax} will generate
%   an error message:
%   \begin{quote}
%     |! Missing \endcsname inserted.|
%   \end{quote}
%   Of course, \LaTeX\ does not use its protecting mechanisms.
%   On contrary \cs{protect} itself will cause errors.
% \item The options are expanded. But perhaps the package will
%   do that, because it has to setup some things before?
%   Example \xpackage{hyperref}:
%   \begin{quote}
%     |\usepackage[pdfauthor=M\"uller]{hyperref}|
%   \end{quote}
%   Package \xpackage{hyperref} does not see |M\"uller| but
%   its expansion and it does not like it, you get many warnings
%   \begin{quote}
%     |Token not allowed in a PDFDocEncoded string|
%   \end{quote}
%   And the title becomes: |Mu127uller|.
%   Therefore such options must usually be given after package
%   \xpackage{hyperref}
%   is loaded:
%   \begin{quote}
%     |\usepackage{hyperref}|\\
%     |\hypersetup{pdfauthor=Fran\c coise M\"uller}|
%   \end{quote}
%   As package option it will even break with |Fran\c coise|
%   because of the cedilla |\c c|, it is not robust enough.
% \end{itemize}
% For users that do not want with this limitations the package
% offers package \xoption{kvoptions-patch}. It patches \LaTeX's option system
% and tries to teach it also to handle options that are given
% as pairs of key and value and to prevent expansion.
% It can already be used at the very beginning, before \cs{documentclass}:
% \begin{quote}
%   |\RequirePackage{kvoptions-patch}|\\
%   |\documentclass[pdfauthor=Fran\c coise M\"uller]{article}|\\
%   |\usepackage{hyperref}|
% \end{quote}
% The latest time is before the package where you want to use
% problematic values:
% \begin{quote}
%   |\usepackage{kvoptions-patch}|\\
%   |\usepackage[Fran\c coise M\"uller]{hyperref}|
% \end{quote}
% Some remarks:
% \begin{itemize}
% \item The patch requires \eTeX, its \cs{unexpanded} feature
%   is much too nice. It is possible to work around using token
%   registers. But the code becomes longer, slower, more difficult
%   to read and maintain. The package without option \xoption{patch}
%   works and will work without \eTeX.
% \item The code for the patch is quite long, there are many test
%   cases. Thus the probability for bugs is probably not too small.
% \item Since 2008/10/18 v3.0 package \xpackage{kvoptions-patch} is
%   available. Before option \xoption{patch} of package \xpackage{kvoptions}
%   must be used instead. I think, the solution as standalone package
%   \xpackage{kvoptions-patch} is cleaner and avoids option clashes.
% \end{itemize}
%
% \subsection{Option \xoption{debugshow}}
%
% The name of this option follows the convention of packages
% \xpackage{multicol}, \xpackage{tabularx}, and \xpackage{tracefnt}.
% Currently it prints the setting of boolean options, declared
% by \cs{DeclareBoolOption} in the \xfile{.log} file, if that
% boolean option is used.
% You can activate the option by
% \begin{itemize}
% \item |\PassOptionsToPackage{debugshow}{kvoptions}|\\
%       Put this somewhere before package \xpackage{kvoptions} is
%       loaded first, e.g. before \cs{documentclass}.
% \item |\RequirePackage[debugshow]{kvoptions}|\\
%       Before \cs{documentclass} even an author has to use
%       \cs{RequirePackage}. \cs{usepackage} only works after
%       \cs{documentclass}.
% \end{itemize}
% The preferred method is \cs{PassOptionsToPackage}, because
% it does not force the package loading and does not disturb,
% if the package is not loaded later at all.
%
% \section{Limitations}
%
% \subsection{Compatibility}
% \subsubsection{Package \xpackage{kvoptions-patch} vs. package \xpackage{xkvltxp}}
%
% Package \xpackage{xkvltxp} from the \xpackage{xkeyval} project
% has the same goal as package \xpackage{kvoptions-patch}
% and to patch \LaTeX's kernel commands in order to get better support
% for key value options.
% Of course they cannot be used both. The user must decide, which
% method he prefers. Package \xpackage{kvoptions-patch} aborts itself,
% if it detects that \xpackage{xkvltxp} is
% already loaded.
%
% However package \xpackage{xkvltxp} and \xpackage{kvoptions}
% can be used together, example:
% \begin{quote}
% |\usepackage{xkvltxp}|\\
% |\usepackage[...]{foobar} % |\texttt{\itshape foobar using kvoptions}
% \end{quote}
% The other way should work, too.
%
% Package \xpackage{kvoptions-patch} tries to catch more situations
% and to be more robust.
% For example, during the comparison of options it normalizes
% them by removing spaces around |=| and the value. Thus the
% following is not reported as option clash:
% \begin{quote}
%\begin{verbatim}
%\RequirePackage{kvoptions-patch}
%\documentclass{article}
%
%\usepackage[scaled=0.7]{helvet}
%\usepackage[scaled = 0.7]{helvet}
%
%\begin{document}
%\end{document}
%\end{verbatim}
% \end{quote}
%
% \subsection{Limitations}
%
% \subsubsection{Option comparisons}
%
% In some situations \LaTeX\ compares option lists, e.g. option clash check,
% \cs{@ifpackagewith}, or \cs{@ifclasswith}. Apart from catcode and
% sanitizing problems of option \xoption{patch}, there is another problem.
% \LaTeX\ does not know about the type and default values of
% options in key value style. Thus an option clash is reported,
% even if the key value has the same meaning:
% \begin{quote}
% |\usepackage[scaled]{helvet} %| \texttt{\textit{default is }.95}\\
% |\usepackage[.95]{helvet}|\\
% |\usepackage[0.95]{helvet}|
% \end{quote}
%
% \subsubsection{Option list parsing with package \xpackage{kvoptions-patch}}
%
% With package \xpackage{kvoptions-patch} the range of possible values
% in key value specifications is much large, for example
% the comma can be used, if enclosed in curly braces.
%
% Other packages, especially the packages that uses their
% own process option code can be surprised to find tokens
% inside options that they do not expect and errors would
% be the consequence. To avoid errors the options, especially
% the unused option list is sanitized. That means the list
% will only contain tokens with catcode 12 (other) and
% perhaps spaces (catcode 10). This allows a safe parsing
% for other packages. But a comma in the value part is
% no longer protected by curly braces because they have lost
% their special meaning. This is the price for compatibility.
%
% Example:
% \begin{quote}
% |\RequirePackage{kvoptions-patch}|\\
% |\documentclass[a={a,b,c},b]{article}|\\
% |\begin{document}|\\
% |\end{document}|
% \end{quote}
% Result:
% \begin{quote}
% |LaTeX Warning: Unused global option(s):|\\
% |    [a={a,c},b].|
% \end{quote}
%
% \StopEventually{
% }
%
% \section{Implementation}
%
% \subsection{Preamble}
%
%    \begin{macrocode}
%<*package>
%    \end{macrocode}
%
% \paragraph{Reload check and  identification.}
%    Reload check, especially if the package is not used with \LaTeX.
%    \begin{macrocode}
\begingroup\catcode61\catcode48\catcode32=10\relax%
  \catcode13=5 % ^^M
  \endlinechar=13 %
  \catcode35=6 % #
  \catcode39=12 % '
  \catcode44=12 % ,
  \catcode45=12 % -
  \catcode46=12 % .
  \catcode58=12 % :
  \catcode64=11 % @
  \catcode123=1 % {
  \catcode125=2 % }
  \expandafter\let\expandafter\x\csname ver@kvoptions.sty\endcsname
  \ifx\x\relax % plain-TeX, first loading
  \else
    \def\empty{}%
    \ifx\x\empty % LaTeX, first loading,
      % variable is initialized, but \ProvidesPackage not yet seen
    \else
      \expandafter\ifx\csname PackageInfo\endcsname\relax
        \def\x#1#2{%
          \immediate\write-1{Package #1 Info: #2.}%
        }%
      \else
        \def\x#1#2{\PackageInfo{#1}{#2, stopped}}%
      \fi
      \x{kvoptions}{The package is already loaded}%
      \aftergroup\endinput
    \fi
  \fi
\endgroup%
%    \end{macrocode}
%    Package identification:
%    \begin{macrocode}
\begingroup\catcode61\catcode48\catcode32=10\relax%
  \catcode13=5 % ^^M
  \endlinechar=13 %
  \catcode35=6 % #
  \catcode39=12 % '
  \catcode40=12 % (
  \catcode41=12 % )
  \catcode44=12 % ,
  \catcode45=12 % -
  \catcode46=12 % .
  \catcode47=12 % /
  \catcode58=12 % :
  \catcode64=11 % @
  \catcode91=12 % [
  \catcode93=12 % ]
  \catcode123=1 % {
  \catcode125=2 % }
  \expandafter\ifx\csname ProvidesPackage\endcsname\relax
    \def\x#1#2#3[#4]{\endgroup
      \immediate\write-1{Package: #3 #4}%
      \xdef#1{#4}%
    }%
  \else
    \def\x#1#2[#3]{\endgroup
      #2[{#3}]%
      \ifx#1\@undefined
        \xdef#1{#3}%
      \fi
      \ifx#1\relax
        \xdef#1{#3}%
      \fi
    }%
  \fi
\expandafter\x\csname ver@kvoptions.sty\endcsname
\ProvidesPackage{kvoptions}%
  [2016/05/16 v3.12 Key value format for package options (HO)]%
%    \end{macrocode}
%
% \paragraph{Catcodes}
%
%    \begin{macrocode}
\begingroup\catcode61\catcode48\catcode32=10\relax%
  \catcode13=5 % ^^M
  \endlinechar=13 %
  \catcode123=1 % {
  \catcode125=2 % }
  \catcode64=11 % @
  \def\x{\endgroup
    \expandafter\edef\csname KVO@AtEnd\endcsname{%
      \endlinechar=\the\endlinechar\relax
      \catcode13=\the\catcode13\relax
      \catcode32=\the\catcode32\relax
      \catcode35=\the\catcode35\relax
      \catcode61=\the\catcode61\relax
      \catcode64=\the\catcode64\relax
      \catcode123=\the\catcode123\relax
      \catcode125=\the\catcode125\relax
    }%
  }%
\x\catcode61\catcode48\catcode32=10\relax%
\catcode13=5 % ^^M
\endlinechar=13 %
\catcode35=6 % #
\catcode64=11 % @
\catcode123=1 % {
\catcode125=2 % }
\def\TMP@EnsureCode#1#2{%
  \edef\KVO@AtEnd{%
    \KVO@AtEnd
    \catcode#1=\the\catcode#1\relax
  }%
  \catcode#1=#2\relax
}
\TMP@EnsureCode{1}{14}% ^^A (comment)
\TMP@EnsureCode{2}{14}% ^^A (comment)
\TMP@EnsureCode{33}{12}% !
\TMP@EnsureCode{39}{12}% '
\TMP@EnsureCode{40}{12}% (
\TMP@EnsureCode{41}{12}% )
\TMP@EnsureCode{42}{12}% *
\TMP@EnsureCode{44}{12}% ,
\TMP@EnsureCode{45}{12}% -
\TMP@EnsureCode{46}{12}% .
\TMP@EnsureCode{47}{12}% /
\TMP@EnsureCode{58}{12}% :
\TMP@EnsureCode{62}{12}% >
\TMP@EnsureCode{91}{12}% [
\TMP@EnsureCode{93}{12}% ]
\TMP@EnsureCode{94}{7}% ^ (superscript)
\TMP@EnsureCode{96}{12}% `
\edef\KVO@AtEnd{\KVO@AtEnd\noexpand\endinput}
%    \end{macrocode}
%
% \paragraph{External resources.}
%    The package extends the support for key value pairs of
%    package \cs{keyval} to package options. Thus the package
%    needs to be loaded anyway. and we use it for
%    \cs{SetupKeyvalOptions}. AFAIK this does not disturb
%    users of \xpackage{xkeyval}.
%    \begin{macrocode}
\@ifundefined{define@key}{%
  \RequirePackage{keyval}\relax
}{}
%    \end{macrocode}
%
%    Macro \cs{DeclareLocalOptions} parses a comma separated key list
%    and uses \cs{comma@parse} of package \xpackage{kvsetkeys},
%    version 1.3.
%    \begin{macrocode}
\RequirePackage{ltxcmds}[2010/12/02]
\RequirePackage{kvsetkeys}[2007/09/29]
%    \end{macrocode}
%
% \paragraph{Provide macros for \plainTeX.}
%    \begin{macrocode}
\@ifundefined{@x@protect}{%
  \def\@x@protect#1\fi#2#3{%
    \fi\protect#1%
  }%
  \let\@typeset@protect\relax
}{}
\@ifundefined{@currname}{%
  \def\@currname{}%
}{}
\@ifundefined{@currext}{%
  \def\@currext{}%
}{}
%    \end{macrocode}
%
% \paragraph{Options}
%    Option \xoption{debugshow} enables additional lines of
%    code that prints information into the \xfile{.log}
%    file.
%    \begin{macrocode}
\DeclareOption{debugshow}{\catcode\@ne=9 }
%    \end{macrocode}
%
%    \begin{macrocode}
\DeclareOption{patch}{%
  \AtEndOfPackage{%
    \RequirePackage{kvoptions-patch}[2016/05/16]%
  }%
}
%    \end{macrocode}
%
%    Optionen auswerten:
%    \begin{macrocode}
\ProcessOptions\relax
%    \end{macrocode}
%
% \subsection{Option declaration macros}
%
% \subsubsection{\cs{SetupKeyvalOptions}}
%
%    The family for the key value pairs can be setup once
%    and is remembered later.
%    The package name seems a reasonable default for the
%    family key, if it is not set by the package author.
%
%    \begin{macro}{\KVO@family}
%    We cannot store the family setting in one macro,
%    because the package should be usable for many
%    other packages, too. Thus we remember the family
%    setting in a macro, whose name contains the
%    package name with extension, a key in \LaTeX's
%    class/package system.
%    \begin{macrocode}
\define@key{KVO}{family}{%
  \expandafter\edef\csname KVO@family@%
      \@currname.\@currext\endcsname{#1}%
}
\def\KVO@family{%
  \@ifundefined{KVO@family@\@currname.\@currext}{%
    \@currname
  }{%
    \csname KVO@family@\@currname.\@currext\endcsname
  }%
}
%    \end{macrocode}
%    \end{macro}
%
%    \begin{macro}{\KVO@prefix}
%    The value settings of options that are declared by \cs{DeclareBoolOption}
%    and \cs{DeclareStringOption} need to be saved in macros.
%    in the first case this is a switch \cs{if}\meta{prefix}\meta{key},
%    in the latter case a macro \cs{}\meta{prefix}\meta{key}.
%    The prefix can be configured, by \xoption{prefix} that is declared
%    here. The default is the package name with |@| appended.
%    \begin{macrocode}
\define@key{KVO}{prefix}{%
  \expandafter\edef\csname KVO@prefix@%
      \@currname.\@currext\endcsname{#1}%
}
\def\KVO@prefix{%
  \ltx@ifundefined{KVO@prefix@\@currname.\@currext}{%
    \@currname @%
  }{%
    \csname KVO@prefix@\@currname.\@currext\endcsname
  }%
}
%    \end{macrocode}
%    \end{macro}
%
%    \begin{macrocode}
\define@key{KVO}{setkeys}{%
  \expandafter\def\csname KVO@setkeys@%
      \@currname.\@currext\endcsname{#1}%
}
%    \end{macrocode}
%    \begin{macro}{\KVO@setkeys}
%    \begin{macrocode}
\def\KVO@setkeys{%
  \ltx@IfUndefined{KVO@setkeys@\@currname.\@currext}{%
    \setkeys
  }{%
    \csname KVO@setkeys@\@currname.\@currext\endcsname
  }%
}
%    \end{macrocode}
%    \end{macro}
%
%    \begin{macro}{\SetupKeyvalOptions}
%    The argument of \cs{SetupKeyvalOptions} expects a key value
%    list, known keys are \xoption{family} and \xoption{prefix}.
%    \begin{macrocode}
\newcommand*{\SetupKeyvalOptions}{%
  \kvsetkeys{KVO}%
}
%    \end{macrocode}
%    \end{macro}
%
% \subsubsection{\cs{DeclareBoolOption}}
%
%    \begin{macro}{\DeclareBoolOption}
%    Usually options of boolean type can be given by the
%    user without value and this means a setting to \emph{true}.
%    We follow this convention here. Also it simplifies the
%    user interface.
%
%    The switch is created and initialized with \emph{false}.
%    The default setting can be overwritten by the optional
%    argument.
%
%    \LaTeX's \cs{newif} does not check for already defined
%    macros, therefore we add this check here to prevent
%    the user from accidently redefining of \TeX's primitives
%    and other macros.
%    \begin{macrocode}
\newcommand*{\DeclareBoolOption}[2][false]{%
  \KVO@ifdefinable{if\KVO@prefix#2}{%
    \KVO@ifdefinable{\KVO@prefix#2true}{%
      \KVO@ifdefinable{\KVO@prefix#2false}{%
        \csname newif\expandafter\endcsname
        \csname if\KVO@prefix#2\endcsname
        \@ifundefined{\KVO@prefix#2#1}{%
          \PackageWarning{kvoptions}{%
            Initialization of option `#2' failed,\MessageBreak
            cannot set boolean option to `#1',\MessageBreak
            use `true' or `false', now using `false'%
          }%
        }{%
          \csname\KVO@prefix#2#1\endcsname
        }%
        \begingroup
          \edef\x{\endgroup
            \noexpand\define@key{\KVO@family}{#2}[true]{%
              \noexpand\KVO@boolkey{\@currname}%
              \ifx\@currext\@clsextension
                \noexpand\@clsextension
              \else
                \noexpand\@pkgextension
              \fi
              {\KVO@prefix}{#2}{####1}%
            }%
          }%
        \x
      }%
    }%
  }%
}
%    \end{macrocode}
%    \end{macro}
%
%    \begin{macro}{\DeclareComplementaryOption}
%    The first argument is the key name, the second the
%    key that must be a boolean option with the same
%    current family and prefix. A new switch is not
%    created for the new key, we have already a switch.
%    Instead we define switch setting commands to work
%    on the parent switch.
%    \begin{macrocode}
\newcommand*{\DeclareComplementaryOption}[2]{%
  \@ifundefined{if\KVO@prefix#2}{%
    \PackageError{kvoptions}{%
      Cannot generate option code for `#1',\MessageBreak
      parent switch `#2' does not exist%
    }{%
      You are inside %
      \ifx\@currext\@clsextension class\else package\fi\space
      `\@currname.\@currext'.\MessageBreak
      `\KVO@family' is used as familiy %
      for the keyval options.\MessageBreak
      `\KVO@prefix' serves as prefix %
      for internal switch macros.\MessageBreak
      \MessageBreak
      \@ehc
    }%
  }{%
    \KVO@ifdefinable{\KVO@prefix#1true}{%
      \KVO@ifdefinable{\KVO@prefix#1false}{%
        \expandafter\let\csname\KVO@prefix#1false\expandafter\endcsname
          \csname\KVO@prefix#2true\endcsname
        \expandafter\let\csname\KVO@prefix#1true\expandafter\endcsname
          \csname\KVO@prefix#2false\endcsname
%    \end{macrocode}
%    The same code part as in \cs{DeclareBoolOption} can
%    now be used.
%    \begin{macrocode}
        \begingroup
          \edef\x{\endgroup
            \noexpand\define@key{\KVO@family}{#1}[true]{%
              \noexpand\KVO@boolkey{\@currname}%
              \ifx\@currext\@clsextension
                \noexpand\@clsextension
              \else
                \noexpand\@pkgextension
              \fi
              {\KVO@prefix}{#1}{####1}%
            }%
          }%
        \x
      }%
    }%
  }%
}
%    \end{macrocode}
%    \end{macro}
%
%    \begin{macro}{\KVO@ifdefinable}
%    Generate the command token LaTeX's \cs{@ifdefinable} expects.
%    \begin{macrocode}
\def\KVO@ifdefinable#1{%
  \expandafter\@ifdefinable\csname #1\endcsname
}
%    \end{macrocode}
%    \end{macro}
%
%    \begin{macro}{\KVO@boolkey}
%    We check explicitly for |true| and |false| to prevent
%    the user from accidently calling other macros.
%    \begin{quote}
%    \begin{tabular}{@{}ll@{}}
%    |#1|& package/class name\\
%    |#2|& |\@pkgextension|/|\@clsextension|\\
%    |#3|& prefix \\
%    |#4|& key name\\
%    |#5|& new value
%    \end{tabular}
%    \end{quote}
%    \begin{macrocode}
\def\KVO@boolkey#1#2#3#4#5{%
  \edef\KVO@param{#5}%
  \ltx@onelevel@sanitize\KVO@param
  \ifx\KVO@param\KVO@true
    \expandafter\@firstofone
  \else
    \ifx\KVO@param\KVO@false
      \expandafter\expandafter\expandafter\@firstofone
    \else
      \ifx#2\@clsextension
        \expandafter\ClassWarning
      \else
        \expandafter\PackageWarning
      \fi
      {#1}{%
        Value `\KVO@param' is not supported by\MessageBreak
        option `#4'%
      }%
      \expandafter\expandafter\expandafter\@gobble
    \fi
  \fi
  {%
    ^^A\ifx#2\@clsextension
    ^^A  \expandafter\ClassInfo
    ^^A\else
    ^^A  \expandafter\PackageInfo
    ^^A\fi
    ^^A{#1}{[option] #4=\KVO@param}%
    \csname#3#4\KVO@param\endcsname
  }%
}
%    \end{macrocode}
%    \end{macro}
%    \begin{macro}{\KVO@true}
%    \begin{macro}{\KVO@false}
%    The macros \cs{KVO@true} and \cs{KVO@false} are used
%    for string comparisons. After \cs{ltx@onelevel@sanitize}
%    we have only tokens with catcode 12 (other).
%    \begin{macrocode}
\def\KVO@true{true}
\def\KVO@false{false}
\ltx@onelevel@sanitize\KVO@true
\ltx@onelevel@sanitize\KVO@false
%    \end{macrocode}
%    \end{macro}
%    \end{macro}
%
% \subsubsection{\cs{DeclareStringOption}}
%
%    \begin{macro}{\DeclareStringOption}
%    \begin{macrocode}
\newcommand*{\DeclareStringOption}[2][]{%
  \@ifnextchar[{%
    \KVO@DeclareStringOption{#1}{#2}@%
  }{%
    \KVO@DeclareStringOption{#1}{#2}{}[]%
  }%
}
%    \end{macrocode}
%    \end{macro}
%    \begin{macro}{\KVO@DeclareStringOption}
%    \begin{macrocode}
\def\KVO@DeclareStringOption#1#2#3[#4]{%
  \KVO@ifdefinable{\KVO@prefix#2}{%
    \@namedef{\KVO@prefix#2}{#1}%
    \begingroup
      \ifx\\#3\\%
        \toks@{}%
      \else
        \toks@{[{#4}]}%
      \fi
      \edef\x{\endgroup
        \noexpand\define@key{\KVO@family}{#2}\the\toks@{%
          ^^A\begingroup
          ^^A  \toks@{####1}%
          ^^A  \ifx\@currext\@clsextension
          ^^A    \noexpand\ClassInfo
          ^^A  \else
          ^^A    \noexpand\PackageInfo
          ^^A  \fi
          ^^A  {\@currname}{%
          ^^A    [option] #2={\noexpand\the\toks@}%
          ^^A  }%
          ^^A\endgroup
          \noexpand\def
          \expandafter\noexpand\csname\KVO@prefix#2\endcsname{####1}%
        }%
      }%
    \x
  }%
}
%    \end{macrocode}
%    \end{macro}
%
% \subsubsection{\cs{DeclareVoidOption}}
%
%    \begin{macro}{\DeclareVoidOption}
%    \begin{macrocode}
\newcommand*{\DeclareVoidOption}[2]{%
  \begingroup
    \let\next\@gobbletwo
    \KVO@ifdefinable{\KVO@prefix#1}{%
      \let\next\@firstofone
    }%
  \expandafter\endgroup
  \next{%
    \begingroup
      \edef\x{\endgroup
        \noexpand\define@key{\KVO@family}{#1}[\KVO@VOID@]{%
          \noexpand\KVO@voidkey{\@currname}%
          \ifx\@currext\@clsextension
            \noexpand\@clsextension
          \else
            \noexpand\@pkgextension
          \fi
          {#1}%
          {####1}%
          \expandafter\noexpand\csname\KVO@prefix#1\endcsname
        }%
      }%
    \x
    \begingroup
      \toks@{#2}%
    \expandafter\endgroup
    \expandafter\def
    \csname\KVO@prefix#1\expandafter\endcsname
    \expandafter{\the\toks@}%
  }%
}
\def\KVO@VOID@{@VOID@}
%    \end{macrocode}
%    \end{macro}
%    \begin{macro}{\KVO@voidkey}
%    \begin{tabular}{@{}ll@{}}
%    |#1|& package/class name\\
%    |#2|& |\@pkgextension|/|\@clsextension|\\
%    |#3|& key name\\
%    |#4|& default (|@VOID@|)\\
%    |#5|& macro with option code
%    \end{tabular}
%    \begin{macrocode}
\def\KVO@voidkey#1#2#3#4{%
  \def\CurrentOption{#3}%
  \begingroup
    \def\x{#4}%
  \expandafter\endgroup
  \ifx\x\KVO@VOID@
  \else
    \ifx#2\@clsextension
      \expandafter\ClassWarning
    \else
      \expandafter\PackageWarning
    \fi
    {#1}{%
      Unexpected value for option `#3'\MessageBreak
      is ignored%
    }%
  \fi
  ^^A\ifx#2\@clsextension
  ^^A  \expandafter\ClassInfo
  ^^A\else
  ^^A  \expandafter\PackageInfo
  ^^A\fi
  ^^A{#1}{[option] #3}%
}
%    \end{macrocode}
%    \end{macro}
%
% \subsubsection{\cs{DeclareDefaultOption}}
%
%    \begin{macro}{\DeclareDefaultOption}
%    \begin{macrocode}
\newcommand*{\DeclareDefaultOption}{%
  \@namedef{KVO@default@\@currname.\@currext}%
}
%    \end{macrocode}
%    \end{macro}
%
% \subsubsection{\cs{DeclareLocalOptions}}
%
%    \begin{macro}{\DeclareLocalOptions}
%    \begin{macrocode}
\newcommand*{\DeclareLocalOptions}[1]{%
  \comma@parse{#1}\KVO@DeclareLocalOption
}
%    \end{macrocode}
%    \end{macro}
%    \begin{macro}{\KVO@DeclareLocalOption}
%    \begin{macrocode}
\def\KVO@DeclareLocalOption#1{%
  \expandafter\def\csname KVO@local@\KVO@family @#1\endcsname{}%
}
%    \end{macrocode}
%    \end{macro}
%
% \subsection{Dynamic options}
%
% \subsubsection{\cs{DisableKeyvalOption}}
%
%    \begin{macrocode}
\SetupKeyvalOptions{%
  family=KVOdyn,%
  prefix=KVOdyn@%
}
\DeclareBoolOption[true]{global}
\DeclareComplementaryOption{local}{global}
\DeclareStringOption[undef]{action}
\let\KVOdyn@name\relax
\let\KVOdyn@ext\@empty
\define@key{KVOdyn}{class}{%
  \def\KVOdyn@name{#1}%
  \let\KVOdyn@ext\@clsextension
}
\define@key{KVOdyn}{package}{%
  \def\KVOdyn@name{#1}%
  \let\KVOdyn@ext\@pkgextension
}
\newcommand*{\DisableKeyvalOption}[3][]{%
  \begingroup
    \kvsetkeys{KVOdyn}{#1}%
    \def\x{\endgroup}%
    \@ifundefined{KVO@action@\KVOdyn@action}{%
      \PackageError{kvoptions}{%
        Unknown disable action %
        `\expandafter\strip@prefix\meaning\KVOdyn@action'\MessageBreak
        for option `#3' in keyval family '#2'%
      }\@ehc
    }{%
      \csname KVO@action@\KVOdyn@action\endcsname{#2}{#3}%
    }%
  \x
}
\def\KVO@action@undef#1#2{%
  \edef\x{\endgroup
    \ifKVOdyn@global\global\fi
    \let
    \expandafter\noexpand\csname KV@#1@#2\endcsname
    \relax
    \ifKVOdyn@global\global\fi
    \let
    \expandafter\noexpand\csname KV@#1@#2@default\endcsname
    \relax
  }%
  ^^A\PackageInfo{kvoptions}{%
  ^^A  [option] key `#2' of family `#1'\MessageBreak
  ^^A  is disabled (undef, \ifKVOdyn@global global\else local\fi)%
  ^^A}%
}
\def\KVO@action@ignore#1#2{%
  \edef\x{\endgroup
    \ifKVOdyn@global\global\fi
    \let
    \expandafter\noexpand\csname KV@#1@#2\endcsname
    \noexpand\@gobble
    \ifKVOdyn@global\global\fi
    \let
    \expandafter\noexpand\csname KV@#1@#2@default\endcsname
    \noexpand\@empty
  }%
  ^^A\PackageInfo{kvoptions}{%
  ^^A  [option] key `#2' of family `#1'\MessageBreak
  ^^A  is disabled (ignore, \ifKVOdyn@global global\else local\fi)%
  ^^A}%
}
\def\KVO@action@error{%
  \KVO@do@action{error}%
}
\def\KVO@action@warning{%
  \KVO@do@action{warning}%
}
%    \end{macrocode}
%    \begin{tabular}{@{}ll@{}}
%    |#1|& |error| or |warning|\\
%    |#2|& \meta{family}\\
%    |#3|& \meta{key}\\
%    \end{tabular}
%    \begin{macrocode}
\def\KVO@do@action#1#2#3{%
  \ifx\KVOdyn@name\relax
    \PackageError{kvoptions}{%
      Action type `#1' needs package/class name\MessageBreak
      for key `#3' in family `#2'%
    }\@ehc
  \else
    \edef\x{\endgroup
      \noexpand\define@key{#2}{#3}[]{%
        \expandafter\noexpand\csname KVO@disable@#1\endcsname
        {\KVOdyn@name}\noexpand\KVOdyn@ext{#3}%
      }%
      \ifKVOdyn@global
        \global\let
        \expandafter\noexpand\csname KV@#2@#3\endcsname
        \expandafter\noexpand\csname KV@#2@#3\endcsname
        \global\let
        \expandafter\noexpand\csname KV@#2@#3@default\endcsname
        \expandafter\noexpand\csname KV@#2@#3@default\endcsname
      \fi
    }%
    ^^A\ifx\KVOdyn@ext\@clsextension
    ^^A  \expandafter\ClassInfo
    ^^A\else
    ^^A   \expandafter\PackageInfo
    ^^A\fi
    ^^A{\KVOdyn@name}{%
    ^^A  [option] key `#3' of family `#2'\MessageBreak
    ^^A  is disabled (#1, \ifKVOdyn@global global\else local\fi)%
    ^^A}%
  \fi
}
\def\KVO@disable@error#1#2#3{%
  \ifx#2\@clsextension
    \expandafter\ClassError
  \else
    \expandafter\PackageError
  \fi
  {#1}{%
    Option `#3' is given too late,\MessageBreak
    now the option is ignored%
  }\@ehc
}
\def\KVO@disable@warning#1#2#3{%
  \ifx#2\@clsextension
    \expandafter\ClassWarning
  \else
    \expandafter\PackageWarning
  \fi
  {#1}{%
    Option `#3' is already consumed\MessageBreak
    and has no effect%
  }%
}
%    \end{macrocode}
%
% \subsection{Change option code}
%
% \subsubsection{\cs{AddToKeyvalOption}}
%
%    \begin{macro}{\AddToKeyvalOption}
%    \begin{macrocode}
\newcommand*{\AddToKeyvalOption}{%
  \@ifstar{%
    \begingroup
      \edef\x{\endgroup
        \noexpand\KVO@AddToKeyvalOption{\KVO@family}%
      }%
    \x
  }%
  \KVO@AddToKeyvalOption
}
%    \end{macrocode}
%    \end{macro}
%    \begin{macro}{\KVO@AddToKeyvalOption}
%    \begin{macrocode}
\def\KVO@AddToKeyvalOption#1#2{%
  \@ifundefined{KV@#1@#2}{%
    \PackageWarning{kvoptions}{%
      Key `#2' of family `#1' does not exist.\MessageBreak
      Ignoring \string\AddToKeyvalOption
    }%
    \@gobble
  }{%
    \edef\KVO@next{%
      \noexpand\KVO@@AddToKeyvalOption
      \expandafter\noexpand\csname KV@#1@#2\endcsname
    }%
    \afterassignment\KVO@next
    \def\KVO@temp##1%
  }%
}
%    \end{macrocode}
%    \end{macro}
%    \begin{macro}{\KVO@@AddToKeyvalOption}
%    \begin{macrocode}
\def\KVO@@AddToKeyvalOption#1{%
  \begingroup
    \toks@\expandafter{#1{##1}}%
    \toks@\expandafter{\the\expandafter\toks@\KVO@temp{##1}}%
    \edef\x{\endgroup
      \noexpand\def\noexpand#1####1{\the\toks@}%
    }%
  \x
}
%    \end{macrocode}
%    \end{macro}
%
% \subsection{Process options}
%
% \subsubsection{Get global options}
%
%    Package \xpackage{xkeyval} removes options with equal signs from
%    the global options (\cs{@classoptionslist}). The effect is that
%    other packages and classes will not see these global options anymore.
%    A bug-report was answered that this behaviour is ``by design''.
%    Thus I call it a design bug. Now getting the global options require
%    an algorithm instead of a simple macro call.
%    \begin{macrocode}
%</package>
%<*package|patch>
%    \end{macrocode}
%    \begin{macro}{\KVO@IfDefThen}
%    Call |#2| if command |#1| is defined and not \cs{relax}.
%    (Package \xpackage{kvoptions-patch} does not load
%    package \xpackage{ltxcmds}.)
%    \begin{macrocode}
\def\KVO@IfDefThen#1#2{%
  \ifx#1\ltx@undefined
  \else
    \ifx#1\relax
    \else
      #2%
    \fi
  \fi
}%
%    \end{macrocode}
%    \end{macro}
%    \begin{macro}{\KVO@GetClassOptionsList}
%    \begin{macrocode}
\def\KVO@GetClassOptionsList{%
  \let\KVO@classoptionslist\@classoptionslist
  \KVO@IfDefThen\@classoptionslist{%
    \KVO@IfDefThen\XKV@documentclass{%
      \ifx\XKV@documentclass\ltx@empty
      \else
        \KVO@IfDefThen\XKV@classoptionslist{%
          \ifx\XKV@classoptionslist\ltx@empty
          \else
            \let\KVO@classoptionslist\XKV@classoptionslist
          \fi
        }%
      \fi
    }%
  }%
}%
%    \end{macrocode}
%    \end{macro}
%    \begin{macrocode}
%</package|patch>
%<*package>
%    \end{macrocode}
%
% \subsubsection{\cs{ProcessKeyvalOptions}}
%
%    \begin{macro}{\ProcessKeyvalOptions}
%    If the optional star is given, we get the family
%    name and expand it for safety.
%    \begin{macrocode}
\newcommand*{\ProcessKeyvalOptions}{%
  \@ifstar{%
    \begingroup
      \edef\x{\endgroup
        \noexpand\KVO@ProcessKeyvalOptions{\KVO@family}%
      }%
    \x
  }%
  \KVO@ProcessKeyvalOptions
}
%    \end{macrocode}
%    \end{macro}
%
%    \begin{macrocode}
\def\KVO@ProcessKeyvalOptions#1{%
  \let\@tempc\relax
  \let\KVO@temp\@empty
%    \end{macrocode}
%    Add any global options that are known to KV to the start of
%    the list being built in |\KVO@temp| and mark them used (by
%    removing them from the unused option list).
%    \begin{macrocode}
  \ifx\@currext\@clsextension
  \else
    \KVO@GetClassOptionsList
    \ifx\KVO@classoptionslist\relax
    \else
      \@for\KVO@CurrentOption:=\KVO@classoptionslist\do{%
        \@ifundefined{KV@#1@\expandafter\KVO@getkey
                      \KVO@CurrentOption=\@nil}{%
        }{%
          \@ifundefined{KVO@local@#1@\expandafter\KVO@getkey
                        \KVO@CurrentOption=\@nil}{%
            \ifx\KVO@Patch Y%
              \edef\KVO@temp{%
                \etex@unexpanded\expandafter{%
                  \KVO@temp
                }%
                ,%
                \etex@unexpanded\expandafter{%
                  \KVO@CurrentOption
                }%
                ,%
              }%
              \ltx@onelevel@sanitize\KVO@CurrentOption
            \else
              \edef\KVO@temp{%
                \KVO@temp
                ,%
                \KVO@CurrentOption
                ,%
              }%
            \fi
            \@expandtwoargs\@removeelement\KVO@CurrentOption
              \@unusedoptionlist\@unusedoptionlist
          }{}%
        }%
      }%
    \fi
  \fi
%    \end{macrocode}
%     Now stick the package options at the end of the list and wrap
%     in a call to \cs{setkeys}. A class ignores unknown global
%     options, we must remove them to prevent error messages
%     from \cs{setkeys}.
%    \begin{macrocode}
  \begingroup
    \toks\tw@{}%
    \@ifundefined{opt@\@currname.\@currext}{%
      \toks@\expandafter{\KVO@temp}%
    }{%
      \toks@\expandafter\expandafter\expandafter{%
        \csname opt@\@currname.\@currext\endcsname
      }%
      \ifx\@currext\@clsextension
        \edef\CurrentOption{\the\toks@}%
        \toks@\expandafter{\KVO@temp}%
        \@for\CurrentOption:=\CurrentOption\do{%
          \@ifundefined{%
            KV@#1@\expandafter\KVO@getkey\CurrentOption=\@nil
          }{%
%    \end{macrocode}
%    A class puts not used options in the unused option list
%    unless there is a default handler.
%    \begin{macrocode}
            \@ifundefined{KVO@default@\@currname.\@currext}{%
              \ifx\KVO@Patch Y%
                \ltx@onelevel@sanitize\CurrentOption
              \fi
              \ifx\@unusedoptionlist\@empty
                \global\let\@unusedoptionlist\CurrentOption
              \else
                \expandafter\expandafter\expandafter\gdef
                \expandafter\expandafter\expandafter\@unusedoptionlist
                \expandafter\expandafter\expandafter{%
                  \expandafter\@unusedoptionlist
                  \expandafter,\CurrentOption
                }%
              \fi
            }{%
              \toks\tw@\expandafter{%
                \the\toks\expandafter\tw@\expandafter,\CurrentOption
              }%
            }%
          }{%
            \toks@\expandafter{%
              \the\expandafter\toks@\expandafter,\CurrentOption
            }%
          }%
        }%
      \else
%    \end{macrocode}
%    Without default action we pass all options to \cs{setkeys}.
%    Otherwise we have to check which options are known.
%    These are passed to \cs{setkeys}. For the others the default
%    action is performed.
%    \begin{macrocode}
        \@ifundefined{KVO@default@\@currname.\@currext}{%
          \toks@\expandafter\expandafter\expandafter{%
            \expandafter\KVO@temp\the\toks@
          }%
        }{%
          \edef\CurrentOption{\the\toks@}%
          \toks@\expandafter{\KVO@temp}%
          \@for\CurrentOption:=\CurrentOption\do{%
            \@ifundefined{%
              KV@#1@\expandafter\KVO@getkey\CurrentOption=\@nil
            }{%
              \toks\tw@\expandafter{%
                \the\toks\expandafter\tw@\expandafter,\CurrentOption
              }%
            }{%
              \toks@\expandafter{%
                \the\expandafter\toks@\expandafter,\CurrentOption
              }%
            }%
          }%
        }%
      \fi
    }%
    \edef\KVO@temp{\endgroup
      \noexpand\KVO@calldefault{\the\toks\tw@}%
      \noexpand\KVO@setkeys{#1}{\the\toks@}%
    }%
  \KVO@temp
%    \end{macrocode}
%    Some cleanup of \cs{ProcessOptions}.
%    \begin{macrocode}
  \let\CurrentOption\@empty
  \AtEndOfPackage{\let\@unprocessedoptions\relax}%
}
%    \end{macrocode}
%
% \subsubsection{\cs{ProcessLocalKeyvalOptions}}
%
%    \begin{macro}{\ProcessLocalKeyvalOptions}
%    If the optional star is given, we get the family
%    name and expand it for safety.
%    \begin{macrocode}
\newcommand*{\ProcessLocalKeyvalOptions}{%
  \@ifstar{%
    \begingroup
      \edef\x{\endgroup
        \noexpand\KVO@ProcessLocalKeyvalOptions{\KVO@family}%
      }%
    \x
  }%
  \KVO@ProcessLocalKeyvalOptions
}
%    \end{macrocode}
%    \end{macro}
%
%    \begin{macrocode}
\def\KVO@ProcessLocalKeyvalOptions#1{%
  \let\@tempc\relax
  \let\KVO@temp\@empty
%    \end{macrocode}
%    Check if \cs{ProcessLocalKeyvalOptions} is called inside
%    a package.
%    \begin{macrocode}
  \ifx\@currext\@pkgextension
  \else
    \PackageError{kvoptions}{%
      \string\ProcessLocalKeyvalOptions\space is intended for packages only%
    }\@ehc
  \fi
%    \end{macrocode}
%     The package options are put into toks register \cs{toks@}.
%    \begin{macrocode}
  \begingroup
    \toks\tw@{}%
    \@ifundefined{opt@\@currname.\@currext}{%
      \toks@\expandafter{\KVO@temp}%
    }{%
      \toks@\expandafter\expandafter\expandafter{%
        \csname opt@\@currname.\@currext\endcsname
      }%
%    \end{macrocode}
%    Without default action we pass all options to \cs{setkeys}.
%    Otherwise we have to check which options are known.
%    These are passed to \cs{setkeys}. For the others the default
%    action is performed.
%    \begin{macrocode}
      \@ifundefined{KVO@default@\@currname.\@currext}{%
        \toks@\expandafter\expandafter\expandafter{%
          \expandafter\KVO@temp\the\toks@
        }%
      }{%
        \edef\CurrentOption{\the\toks@}%
        \toks@\expandafter{\KVO@temp}%
        \@for\CurrentOption:=\CurrentOption\do{%
          \@ifundefined{%
            KV@#1@\expandafter\KVO@getkey\CurrentOption=\@nil
          }{%
            \toks\tw@\expandafter{%
              \the\toks\expandafter\tw@\expandafter,\CurrentOption
            }%
          }{%
            \toks@\expandafter{%
              \the\expandafter\toks@\expandafter,\CurrentOption
            }%
          }%
        }%
      }%
    }%
    \edef\KVO@temp{\endgroup
      \noexpand\KVO@calldefault{\the\toks\tw@}%
      \noexpand\KVO@setkeys{#1}{\the\toks@}%
    }%
  \KVO@temp
%    \end{macrocode}
%    Some cleanup of \cs{ProcessOptions}.
%    \begin{macrocode}
  \let\CurrentOption\@empty
  \AtEndOfPackage{\let\@unprocessedoptions\relax}%
}
%    \end{macrocode}
%
% \subsubsection{Helper macros}
%
%    \begin{macro}{\KVO@getkey}
%    Extract the key part of a key=value pair.
%    \begin{macrocode}
\def\KVO@getkey#1=#2\@nil{#1}
%    \end{macrocode}
%    \end{macro}
%
%    \begin{macro}{\KVO@calldefault}
%    \begin{macrocode}
\def\KVO@calldefault#1{%
  \begingroup
    \def\x{#1}%
  \expandafter\endgroup
  \ifx\x\@empty
  \else
    \@for\CurrentOption:=#1\do{%
      \ifx\CurrentOption\@empty
      \else
        \expandafter\KVO@setcurrents\CurrentOption=\@nil
        \@nameuse{KVO@default@\@currname.\@currext}%
      \fi
    }%
  \fi
}
%    \end{macrocode}
%    \end{macro}
%    \begin{macro}{\KVO@setcurrents}
%    Extract the key part of a key=value pair.
%    \begin{macrocode}
\def\KVO@setcurrents#1=#2\@nil{%
  \def\CurrentOptionValue{#2}%
  \ifx\CurrentOptionValue\@empty
    \let\CurrentOptionKey\CurrentOption
    \let\CurrentOptionValue\relax
  \else
    \edef\CurrentOptionKey{\zap@space#1 \@empty}%
    \expandafter\KVO@setcurrentvalue\CurrentOption\@nil
  \fi
}
%    \end{macrocode}
%    \end{macro}
%    \begin{macro}{\KV@setcurrentvalue}
%    Here the value part is parsed. Package \xpackage{keyval}'s
%    \cs{KV@@sp@def} helps in removing spaces at the begin and
%    end of the value.
%    \begin{macrocode}
\def\KVO@setcurrentvalue#1=#2\@nil{%
  \KV@@sp@def\CurrentOptionValue{#2}%
}
%    \end{macrocode}
%    \end{macro}
%
% \subsection{\plainTeX}
%
%    Disable \LaTeX\ stuff.
%    \begin{macrocode}
\begingroup\expandafter\expandafter\expandafter\endgroup
\expandafter\ifx\csname documentclass\endcsname\relax
  \def\ProcessKeyvalOptions{%
    \@ifstar{}\@gobble
  }%
\fi
%    \end{macrocode}
%
%    \begin{macrocode}
\KVO@AtEnd%
%</package>
%    \end{macrocode}
%
% \subsection{Package \xpackage{kvoptions-patch}}
%
%    \begin{macrocode}
%<*patch>
\NeedsTeXFormat{LaTeX2e}
\begingroup\catcode61\catcode48\catcode32=10\relax%
  \catcode13=5 % ^^M
  \endlinechar=13 %
  \catcode123=1 % {
  \catcode125=2 % }
  \catcode64=11 % @
  \def\x{\endgroup
    \expandafter\edef\csname KVO@AtEnd\endcsname{%
      \endlinechar=\the\endlinechar\relax
      \catcode13=\the\catcode13\relax
      \catcode32=\the\catcode32\relax
      \catcode35=\the\catcode35\relax
      \catcode61=\the\catcode61\relax
      \catcode64=\the\catcode64\relax
      \catcode123=\the\catcode123\relax
      \catcode125=\the\catcode125\relax
    }%
  }%
\x\catcode61\catcode48\catcode32=10\relax%
\catcode13=5 % ^^M
\endlinechar=13 %
\catcode35=6 % #
\catcode64=11 % @
\catcode123=1 % {
\catcode125=2 % }
\def\TMP@EnsureCode#1#2{%
  \edef\KVO@AtEnd{%
    \KVO@AtEnd
    \catcode#1=\the\catcode#1\relax
  }%
  \catcode#1=#2\relax
}
\TMP@EnsureCode{39}{12}% '
\TMP@EnsureCode{40}{12}% (
\TMP@EnsureCode{41}{12}% )
\TMP@EnsureCode{43}{12}% +
\TMP@EnsureCode{44}{12}% ,
\TMP@EnsureCode{45}{12}% -
\TMP@EnsureCode{46}{12}% .
\TMP@EnsureCode{47}{12}% /
\TMP@EnsureCode{58}{12}% :
\TMP@EnsureCode{60}{12}% <
\TMP@EnsureCode{62}{12}% >
\TMP@EnsureCode{91}{12}% [
\TMP@EnsureCode{93}{12}% ]
\TMP@EnsureCode{96}{12}% `
\TMP@EnsureCode{124}{12}% |
\edef\KVO@AtEnd{\KVO@AtEnd\noexpand\endinput}
\ProvidesPackage{kvoptions-patch}%
  [2016/05/16 v3.12 LaTeX patch for keyval options (HO)]%
%    \end{macrocode}
%
%    Check for \eTeX.
%    \begin{macrocode}
\begingroup\expandafter\expandafter\expandafter\endgroup
\expandafter\ifx\csname eTeXversion\endcsname\relax
  \PackageWarningNoLine{kvoptions-patch}{%
    Package loading is aborted, because e-TeX is missing%
  }%
  \expandafter\KVO@AtEnd
\fi%
%    \end{macrocode}
%
%    Package \xpackage{etexcmds} for \cs{etex@unexpanded}.
%    \begin{macrocode}
\RequirePackage{etexcmds}[2007/09/09]
\ifetex@unexpanded
\else
  \PackageError{kvoptions-patch}{%
    Could not find eTeX's \string\unexpanded.\MessageBreak
    Try adding \string\RequirePackage\string{etexcmds\string} %
    before \string\documentclass%
  }\@ehd
  \expandafter\KVO@AtEnd
\fi%
%    \end{macrocode}
%
%    Check for package \xpackage{xkvltxp}.
%    \begin{macrocode}
\@ifpackageloaded{xkvltxp}{%
  \PackageWarningNoLine{kvoptions}{%
    Option `patch' cannot be used together with\MessageBreak
    package `xkvltxp' that is already loaded.\MessageBreak
    Therefore package loading is aborted%
  }%
  \KVO@AtEnd
}{}%
%    \end{macrocode}
%
%    \begin{macrocode}
\def\@if@ptions#1#2#3{%
  \begingroup
    \KVO@normalize\KVO@temp{#3}%
    \edef\x{\endgroup
      \noexpand\@if@pti@ns{%
        \detokenize\expandafter\expandafter\expandafter{%
          \csname opt@#2.#1\endcsname
        }%
      }{%
        \detokenize\expandafter{\KVO@temp}%
      }%
    }%
  \x
}
%    \end{macrocode}
%
%    \begin{macrocode}
\def\@pass@ptions#1#2#3{%
  \KVO@normalize\KVO@temp{#2}%
  \@ifundefined{opt@#3.#1}{%
    \expandafter\gdef\csname opt@#3.#1%
          \expandafter\endcsname\expandafter{%
      \KVO@temp
    }%
  }{%
    \expandafter\gdef\csname opt@#3.#1%
          \expandafter\expandafter\expandafter\endcsname
          \expandafter\expandafter\expandafter{%
      \csname opt@#3.#1\expandafter\endcsname\expandafter,\KVO@temp
    }%
  }%
}
%    \end{macrocode}
%
%    \begin{macrocode}
\def\ProcessOptions{%
  \let\ds@\@empty
  \@ifundefined{opt@\@currname.\@currext}{%
    \let\@curroptions\@empty
  }{%
    \expandafter\expandafter\expandafter\def
    \expandafter\expandafter\expandafter\@curroptions
    \expandafter\expandafter\expandafter{%
      \csname opt@\@currname.\@currext\endcsname
    }%
  }%
  \@ifstar\KVO@xprocess@ptions\KVO@process@ptions
}
%    \end{macrocode}
%
%    \begin{macrocode}
\def\KVO@process@ptions{%
  \@for\CurrentOption:=\@declaredoptions\do{%
    \ifx\CurrentOption\@empty
    \else
      \begingroup
        \ifx\@currext\@clsextension
          \toks@{}%
        \else
          \KVO@GetClassOptionsList
          \toks@\expandafter{\KVO@classoptionslist,}%
        \fi
        \toks\tw@\expandafter{\@curroptions}%
        \edef\x{\endgroup
          \noexpand\in@{,\CurrentOption,}{,\the\toks@\the\toks\tw@,}%
        }%
      \x
      \ifin@
        \KVO@use@ption
        \expandafter\let\csname ds@\CurrentOption\endcsname\@empty
      \fi
    \fi
  }%
  \KVO@process@pti@ns
}
%    \end{macrocode}
%
%    \begin{macrocode}
\def\KVO@xprocess@ptions{%
  \ifx\@currext\@clsextension
  \else
    \KVO@GetClassOptionsList
    \@for\CurrentOption:=\KVO@classoptionslist\do{%
      \ifx\CurrentOption\@empty
      \else
        \KVO@in@\CurrentOption\@declaredoptions
        \ifin@
          \KVO@use@ption
          \expandafter\let\csname ds@\CurrentOption\endcsname\@empty
        \fi
      \fi
    }%
  \fi
  \KVO@process@pti@ns
}
%    \end{macrocode}
%
%    \begin{macrocode}
\def\KVO@in@#1#2{%
  \in@false
  \begingroup
    \@for\x:=#2\do{%
      \ifx\x#1\relax
        \in@true
      \fi
    }%
    \edef\x{\endgroup
      \ifin@
        \noexpand\in@true
      \fi
    }%
  \x
}
%    \end{macrocode}
%
%    \begin{macrocode}
\def\KVO@process@pti@ns{%
  \@for\CurrentOption:=\@curroptions\do{%
    \@ifundefined{ds@\KVO@SanitizedCurrentOption}{%
      \KVO@use@ption
      \default@ds
    }%
    \KVO@use@ption
  }%
  \@for\CurrentOption:=\@declaredoptions\do{%
    \expandafter\let\csname ds@\CurrentOption\endcsname\relax
  }%
  \let\CurrentOption\@empty
  \let\@fileswith@pti@ns\@@fileswith@pti@ns
  \AtEndOfPackage{\let\@unprocessedoptions\relax}%
}
%    \end{macrocode}
%
%    \begin{macrocode}
\def\KVO@use@ption{%
  \begingroup
    \edef\x{\endgroup
      \noexpand\@removeelement{%
        \detokenize\expandafter{\CurrentOption}%
      }{%
        \detokenize\expandafter{\@unusedoptionlist}%
      }%
    }%
  \x\@unusedoptionlist
  \csname ds@\KVO@SanitizedCurrentOption\endcsname
}
%    \end{macrocode}
%
%    \begin{macrocode}
\def\OptionNotUsed{%
  \ifx\@currext\@clsextension
    \xdef\@unusedoptionlist{%
      \ifx\@unusedoptionlist\@empty
      \else
        \detokenize\expandafter{\@unusedoptionlist,}%
      \fi
      \detokenize\expandafter{\CurrentOption}%
    }%
  \fi
}
%    \end{macrocode}
%
%    Variant of \cs{ExecuteOptions} that better protects \cs{CurrentOption}.
%    \begin{macrocode}
\def\CurrentOption@SaveLevel{0}
\def\ExecuteOptions{%
  \expandafter\KVO@ExecuteOptions
      \csname CurrentOption@\CurrentOption@SaveLevel\endcsname
}
\def\KVO@ExecuteOptions#1#2{%
  \let#1\CurrentOption
  \edef\CurrentOption@SaveLevel{%
    \the\numexpr\CurrentOption@SaveLevel+1%
  }%
  \@for\CurrentOption:=#2\do{%
    \csname ds@\CurrentOption\endcsname
  }%
  \edef\CurrentOption@SaveLevel{%
    \the\numexpr\CurrentOption@SaveLevel-1%
  }%
  \let\CurrentOption#1%
}
%    \end{macrocode}
%
%    \begin{macrocode}
\def\KVO@fileswith@pti@ns#1[#2]#3[#4]{%
  \ifx#1\@clsextension
    \ifx\@classoptionslist\relax
      \KVO@normalize\KVO@temp{#2}%
      \expandafter\gdef\expandafter\@classoptionslist\expandafter{%
        \KVO@temp
      }%
      \def\reserved@a{%
        \KVO@onefilewithoptions{#3}[{#2}][{#4}]#1%
        \@documentclasshook
      }%
    \else
      \def\reserved@a{%
        \KVO@onefilewithoptions{#3}[{#2}][{#4}]#1%
      }%
    \fi
  \else
    \begingroup
      \let\KVO@temp\relax
      \let\KVO@onefilewithoptions\relax
      \let\@pkgextension\relax
      \def\reserved@b##1,{%
        \ifx\@nil##1\relax
        \else
          \ifx\relax##1\relax
          \else
            \KVO@onefilewithoptions{##1}[{\KVO@temp}][{#4}]%
            \@pkgextension
          \fi
          \expandafter\reserved@b
        \fi
      }%
      \edef\reserved@a{\zap@space#3 \@empty}%
      \edef\reserved@a{\expandafter\reserved@b\reserved@a,\@nil,}%
      \toks@{#2}%
      \def\KVO@temp{\the\toks@}%
    \edef\reserved@a{\endgroup \reserved@a}%
  \fi
  \reserved@a
}
%    \end{macrocode}
%
%    \begin{macrocode}
\def\KVO@onefilewithoptions#1[#2][#3]#4{%
  \@pushfilename
  \xdef\@currname{#1}%
  \global\let\@currext#4%
  \expandafter\let\csname\@currname.\@currext-h@@k\endcsname\@empty
  \let\CurrentOption\@empty
  \@reset@ptions
  \makeatletter
  \def\reserved@a{%
    \@ifl@aded\@currext{#1}{%
      \@if@ptions\@currext{#1}{#2}{%
      }{%
        \begingroup
          \@ifundefined{opt@#1.\@currext}{%
            \def\x{}%
          }{%
            \edef\x{%
              \expandafter\expandafter\expandafter\strip@prefix
              \expandafter\meaning\csname opt@#1.\@currext\endcsname
            }%
          }%
          \def\y{#2}%
          \edef\y{\expandafter\strip@prefix\meaning\y}%
          \@latex@error{Option clash for \@cls@pkg\space #1}{%
            The package #1 has already been loaded %
            with options:\MessageBreak
            \space\space[\x]\MessageBreak
            There has now been an attempt to load it %
             with options\MessageBreak
            \space\space[\y]\MessageBreak
            Adding the global options:\MessageBreak
            \space\space
                 \x,\y\MessageBreak
            to your \noexpand\documentclass declaration may fix this.%
            \MessageBreak
            Try typing \space <return> \space to proceed.%
          }%
        \endgroup
      }%
    }{%
      \@pass@ptions\@currext{#2}{#1}%
      \global\expandafter
      \let\csname ver@\@currname.\@currext\endcsname\@empty
      \InputIfFileExists
        {\@currname.\@currext}%
        {}%
        {\@missingfileerror\@currname\@currext}%
      \let\@unprocessedoptions\@@unprocessedoptions
      \csname\@currname.\@currext-h@@k\endcsname
      \expandafter\let\csname\@currname.\@currext-h@@k\endcsname
              \@undefined
      \@unprocessedoptions
    }%
    \@ifl@ter\@currext{#1}{#3}{%
    }{%
      \@latex@warning@no@line{%
        You have requested,\on@line, %
        version\MessageBreak
          #3' of \@cls@pkg\space #1,\MessageBreak
        but only version\MessageBreak
         `\csname ver@#1.\@currext\endcsname'\MessageBreak
        is available%
      }%
    }%
    \ifx\@currext\@clsextension\let\LoadClass\@twoloadclasserror\fi
    \@popfilename
    \@reset@ptions
  }%
  \reserved@a
}
%    \end{macrocode}
%
%    \begin{macrocode}
\def\@unknownoptionerror{%
  \@latex@error{%
    Unknown option `\KVO@SanitizedCurrentOption' %
    for \@cls@pkg\space`\@currname'%
  }{%
    The option `\KVO@SanitizedCurrentOption' was not declared in %
    \@cls@pkg\space`\@currname', perhaps you\MessageBreak
    misspelled its name. %
    Try typing \space <return> %
    \space to proceed.%
  }%
}
%    \end{macrocode}
%
%    \begin{macrocode}
\def\@@unprocessedoptions{%
  \ifx\@currext\@pkgextension
    \@ifundefined{opt@\@currname.\@currext}{%
      \let\@curroptions\@empty
    }{%
      \expandafter\let\expandafter\@curroptions
          \csname opt@\@currname.\@currext\endcsname
    }%
    \@for\CurrentOption:=\@curroptions\do{%
        \ifx\CurrentOption\@empty\else\@unknownoptionerror\fi
    }%
  \fi
}
%    \end{macrocode}
%
%    \begin{macrocode}
\def\KVO@SanitizedCurrentOption{%
  \expandafter\strip@prefix\meaning\CurrentOption
}
%    \end{macrocode}
%
%    Normalize option list.
%    \begin{macrocode}
\def\KVO@normalize#1#2{%
  \let\KVO@result\@empty
  \KVO@splitcomma#2,\@nil
  \let#1\KVO@result
}
\def\KVO@splitcomma#1,#2\@nil{%
  \KVO@ifempty{#1}{}{%
    \KVO@checkkv#1=\@nil
  }%
  \KVO@ifempty{#2}{}{\KVO@splitcomma#2\@nil}%
}
\def\KVO@ifempty#1{%
  \expandafter\ifx\expandafter\\\detokenize{#1}\\%
    \expandafter\@firstoftwo
  \else
    \expandafter\@secondoftwo
  \fi
}
\def\KVO@checkkv#1=#2\@nil{%
  \KVO@ifempty{#2}{%
    % option without value
    \edef\KVO@x{\zap@space#1 \@empty}%
    \ifx\KVO@x\@empty
      % ignore empty option
    \else
      % append to list
      \edef\KVO@result{%
        \etex@unexpanded\expandafter{\KVO@result},\KVO@x
      }%
    \fi
  }{%
    % #1: "key", #2: "value="
    % add key part
    \edef\KVO@result{%
      \etex@unexpanded\expandafter{\KVO@result},%
      \zap@space#1 \@empty
    }%
    \futurelet\@let@token\KVO@checkfirsttok#2 \@nil| = \@nil|\KVO@nil
  }%
}
\def\KVO@checkfirsttok{%
  \ifx\@let@token\bgroup
    % no space at start
    \expandafter\KVO@removelastspace\expandafter=%
    % "<value><spaceopt>= \@nil"
  \else
    \expandafter\KVO@checkfirstA
  \fi
}
\def\KVO@checkfirstA#1 #2\@nil{%
  \KVO@ifempty{#2}{%
    \KVO@removelastspace=#1 \@nil
  }{%
    \KVO@ifempty{#1}{%
      \KVO@removelastspace=#2\@nil
    }{%
      \KVO@removelastspace=#1 #2\@nil
    }%
  }%
}
\def\KVO@removelastspace#1 = \@nil|#2\KVO@nil{%
  \KVO@ifempty{#2}{%
    \edef\KVO@result{%
      \etex@unexpanded\expandafter{\KVO@result}%
      \etex@unexpanded\expandafter{\KVO@removegarbage#1\KVO@nil}%
    }%
  }{%
    \edef\KVO@result{%
      \etex@unexpanded\expandafter{\KVO@result}%
      \etex@unexpanded{#1}%
    }%
  }%
}
\def\KVO@removegarbage#1= \@nil#2\KVO@nil{#1}%
%    \end{macrocode}
%
%    Arguments |#1| and |#2| are macros.
%    \begin{macrocode}
\def\KVO@removeelement#1#2{%
  \begingroup
    \toks@={}%
    \@for\x:=#2\do{%
      \ifx\x\@empty
      \else
        \ifx\x#1\relax
        \else
          \edef\t{\the\toks@}%
          \ifx\t\@empty
          \else
            \toks@\expandafter{\the\toks@,}%
          \fi
          \toks@\expandafter{\the\expandafter\toks@\x}%
        \fi
      \fi
    }%
    \edef\x{\endgroup
      \def\noexpand#2{\the\toks@}%
    }%
  \x
}
%    \end{macrocode}
%
%
%    \begin{macrocode}
\let\@@fileswith@pti@ns\KVO@fileswith@pti@ns
\ifx\@fileswith@pti@ns\@badrequireerror
\else
  \let\@fileswith@pti@ns\KVO@fileswith@pti@ns
\fi
%    \end{macrocode}
%
%    \begin{macro}{\KVO@Patch}
%    \begin{macrocode}
\let\KVO@Patch=Y
%    \end{macrocode}
%    \end{macro}
%
%    \begin{macrocode}
\KVO@AtEnd%
%</patch>
%    \end{macrocode}
%
% \section{Test}
%
% \subsection{Preface for standard catcode check}
%
%    \begin{macrocode}
%<*test1>
\input miniltx.tex\relax
%</test1>
%    \end{macrocode}
%
% \subsection{Catcode checks for loading}
%
%    \begin{macrocode}
%<*test1>
%    \end{macrocode}
%    \begin{macrocode}
\catcode`\{=1 %
\catcode`\}=2 %
\catcode`\#=6 %
\catcode`\@=11 %
\expandafter\ifx\csname count@\endcsname\relax
  \countdef\count@=255 %
\fi
\expandafter\ifx\csname @gobble\endcsname\relax
  \long\def\@gobble#1{}%
\fi
\expandafter\ifx\csname @firstofone\endcsname\relax
  \long\def\@firstofone#1{#1}%
\fi
\expandafter\ifx\csname loop\endcsname\relax
  \expandafter\@firstofone
\else
  \expandafter\@gobble
\fi
{%
  \def\loop#1\repeat{%
    \def\body{#1}%
    \iterate
  }%
  \def\iterate{%
    \body
      \let\next\iterate
    \else
      \let\next\relax
    \fi
    \next
  }%
  \let\repeat=\fi
}%
\def\RestoreCatcodes{}
\count@=0 %
\loop
  \edef\RestoreCatcodes{%
    \RestoreCatcodes
    \catcode\the\count@=\the\catcode\count@\relax
  }%
\ifnum\count@<255 %
  \advance\count@ 1 %
\repeat

\def\RangeCatcodeInvalid#1#2{%
  \count@=#1\relax
  \loop
    \catcode\count@=15 %
  \ifnum\count@<#2\relax
    \advance\count@ 1 %
  \repeat
}
\def\RangeCatcodeCheck#1#2#3{%
  \count@=#1\relax
  \loop
    \ifnum#3=\catcode\count@
    \else
      \errmessage{%
        Character \the\count@\space
        with wrong catcode \the\catcode\count@\space
        instead of \number#3%
      }%
    \fi
  \ifnum\count@<#2\relax
    \advance\count@ 1 %
  \repeat
}
\def\space{ }
\expandafter\ifx\csname LoadCommand\endcsname\relax
  \def\LoadCommand{\input kvoptions.sty\relax}%
\fi
\def\Test{%
  \RangeCatcodeInvalid{0}{47}%
  \RangeCatcodeInvalid{58}{64}%
  \RangeCatcodeInvalid{91}{96}%
  \RangeCatcodeInvalid{123}{255}%
  \catcode`\@=12 %
  \catcode`\\=0 %
  \catcode`\%=14 %
  \LoadCommand
  \RangeCatcodeCheck{0}{36}{15}%
  \RangeCatcodeCheck{37}{37}{14}%
  \RangeCatcodeCheck{38}{47}{15}%
  \RangeCatcodeCheck{48}{57}{12}%
  \RangeCatcodeCheck{58}{63}{15}%
  \RangeCatcodeCheck{64}{64}{12}%
  \RangeCatcodeCheck{65}{90}{11}%
  \RangeCatcodeCheck{91}{91}{15}%
  \RangeCatcodeCheck{92}{92}{0}%
  \RangeCatcodeCheck{93}{96}{15}%
  \RangeCatcodeCheck{97}{122}{11}%
  \RangeCatcodeCheck{123}{255}{15}%
  \RestoreCatcodes
}
\Test
\csname @@end\endcsname
\end
%    \end{macrocode}
%    \begin{macrocode}
%</test1>
%    \end{macrocode}
%
%    \begin{macrocode}
%<*test2>
\NeedsTeXFormat{LaTeX2e}
\makeatletter
\catcode`\@=11 %
\def\RestoreCatcodes{}
\count@=0 %
\loop
  \edef\RestoreCatcodes{%
    \RestoreCatcodes
    \catcode\the\count@=\the\catcode\count@\relax
  }%
\ifnum\count@<255 %
  \advance\count@\@ne
\repeat

\def\RangeCatcodeInvalid#1#2{%
  \count@=#1\relax
  \loop
    \catcode\count@=15 %
  \ifnum\count@<#2\relax
    \advance\count@\@ne
  \repeat
}
\def\Test#1{%
  \RangeCatcodeInvalid{0}{47}%
  \RangeCatcodeInvalid{58}{64}%
  \RangeCatcodeInvalid{91}{96}%
  \RangeCatcodeInvalid{123}{255}%
  \catcode`\@=12 %
  \catcode`\\=0 %
  \catcode`\{=1 %
  \catcode`\}=2 %
  \catcode`\#=6 %
  \catcode`\[=12 %
  \catcode`\]=12 %
  \catcode`\%=14 %
  \catcode`\ =10 %
  \catcode13=5 %
  #1\relax
  \RestoreCatcodes
}
\Test{\RequirePackage{kvoptions-patch}}%
\Test{\RequirePackage{kvoptions}}%
\csname @@end\endcsname
%</test2>
%    \end{macrocode}
%
%    \begin{macrocode}
%<*test3>
\NeedsTeXFormat{LaTeX2e}
\makeatletter
\RequirePackage{kvoptions}[2016/05/16]
\def\msg#{\immediate\write16}
\define@key{testfamily}{testkey}{%
  \msg{[testfamily/testkey/#1]}%
}
\define@key{testfamily}{testdefaultkey}[testdefault]{%
  \msg{[testfamily/testdefaultkey/#1]}%
}
\AddToKeyvalOption{testfamily}{testkey}{%
  \msg{[addition/#1]}%
}
\AddToKeyvalOption{testfamily}{testdefaultkey}{%
  \msg{[addition/#1]}%
}
\setkeys{testfamily}{%
  testkey=testA,%
  testdefaultkey=testB,%
  testdefaultkey,%
}
\SetupKeyvalOptions{%
  family=testfamily%
}
\AddToKeyvalOption*{testkey}{%
  \msg{[star addition/#1]}%
}
\AddToKeyvalOption*{testdefaultkey}{%
  \msg{[star addition/#1]}%
}
\setkeys{testfamily}{%
  testkey=testA,%
  testdefaultkey=testB,%
  testdefaultkey,%
}
\@@end
%</test3>
%    \end{macrocode}
%
%    \begin{macrocode}
%<*test4pkg>
\NeedsTeXFormat{LaTeX2e}
\ProvidesPackage{kvoptions-test4}[2016/05/16 package for testing]
\RequirePackage{kvoptions}[2016/05/16]
\SetupKeyvalOptions{%
  family=FOO,%
  prefix=foo,%
  setkeys=\kvsetkeys,%
}
\DeclareStringOption{str}
\define@key{FOO}{set}{%
  \setkeys{BAR}{strbar={#1}}%
}
\define@key{BAR}{strbar}{%
  \def\foostr{[BAR:#1]}%
}
\ProcessKeyvalOptions*
%</test4pkg>
%    \end{macrocode}
%    \begin{macrocode}
%<*test4>
\NeedsTeXFormat{LaTeX2e}
\ProvidesFile{kvoptions-test4.tex}[2016/05/16 test file]
\RequirePackage[%
  str=A,set=B,str=C,%
]{kvoptions-test4}[2016/05/16]
\def\TestExpected{C}
\ifx\foostr\TestExpected
  \typeout{* Test ok.}%
\else
  \typeout{* Result: [\foostr]}%
  \typeout{* Expected: [\TestExpected]}%
  \errmessage{Test failed!}%
\fi
\csname @@end\endcsname\end
%</test4>
%    \end{macrocode}
%
% \section{Installation}
%
% \subsection{Download}
%
% \paragraph{Package.} This package is available on
% CTAN\footnote{\CTANpkg{kvoptions}}:
% \begin{description}
% \item[\CTAN{macros/latex/contrib/oberdiek/kvoptions.dtx}] The source file.
% \item[\CTAN{macros/latex/contrib/oberdiek/kvoptions.pdf}] Documentation.
% \end{description}
%
%
% \paragraph{Bundle.} All the packages of the bundle `oberdiek'
% are also available in a TDS compliant ZIP archive. There
% the packages are already unpacked and the documentation files
% are generated. The files and directories obey the TDS standard.
% \begin{description}
% \item[\CTANinstall{install/macros/latex/contrib/oberdiek.tds.zip}]
% \end{description}
% \emph{TDS} refers to the standard ``A Directory Structure
% for \TeX\ Files'' (\CTAN{tds/tds.pdf}). Directories
% with \xfile{texmf} in their name are usually organized this way.
%
% \subsection{Bundle installation}
%
% \paragraph{Unpacking.} Unpack the \xfile{oberdiek.tds.zip} in the
% TDS tree (also known as \xfile{texmf} tree) of your choice.
% Example (linux):
% \begin{quote}
%   |unzip oberdiek.tds.zip -d ~/texmf|
% \end{quote}
%
% \paragraph{Script installation.}
% Check the directory \xfile{TDS:scripts/oberdiek/} for
% scripts that need further installation steps.
%
% \subsection{Package installation}
%
% \paragraph{Unpacking.} The \xfile{.dtx} file is a self-extracting
% \docstrip\ archive. The files are extracted by running the
% \xfile{.dtx} through \plainTeX:
% \begin{quote}
%   \verb|tex kvoptions.dtx|
% \end{quote}
%
% \paragraph{TDS.} Now the different files must be moved into
% the different directories in your installation TDS tree
% (also known as \xfile{texmf} tree):
% \begin{quote}
% \def\t{^^A
% \begin{tabular}{@{}>{\ttfamily}l@{ $\rightarrow$ }>{\ttfamily}l@{}}
%   kvoptions.sty & tex/latex/oberdiek/kvoptions.sty\\
%   kvoptions-patch.sty & tex/latex/oberdiek/kvoptions-patch.sty\\
%   kvoptions.pdf & doc/latex/oberdiek/kvoptions.pdf\\
%   example-mycolorsetup.sty & doc/latex/oberdiek/example-mycolorsetup.sty\\
%   test/kvoptions-test1.tex & doc/latex/oberdiek/test/kvoptions-test1.tex\\
%   test/kvoptions-test2.tex & doc/latex/oberdiek/test/kvoptions-test2.tex\\
%   test/kvoptions-test3.tex & doc/latex/oberdiek/test/kvoptions-test3.tex\\
%   test/kvoptions-test4.tex & doc/latex/oberdiek/test/kvoptions-test4.tex\\
%   test/kvoptions-test4.sty & doc/latex/oberdiek/test/kvoptions-test4.sty\\
%   kvoptions.dtx & source/latex/oberdiek/kvoptions.dtx\\
% \end{tabular}^^A
% }^^A
% \sbox0{\t}^^A
% \ifdim\wd0>\linewidth
%   \begingroup
%     \advance\linewidth by\leftmargin
%     \advance\linewidth by\rightmargin
%   \edef\x{\endgroup
%     \def\noexpand\lw{\the\linewidth}^^A
%   }\x
%   \def\lwbox{^^A
%     \leavevmode
%     \hbox to \linewidth{^^A
%       \kern-\leftmargin\relax
%       \hss
%       \usebox0
%       \hss
%       \kern-\rightmargin\relax
%     }^^A
%   }^^A
%   \ifdim\wd0>\lw
%     \sbox0{\small\t}^^A
%     \ifdim\wd0>\linewidth
%       \ifdim\wd0>\lw
%         \sbox0{\footnotesize\t}^^A
%         \ifdim\wd0>\linewidth
%           \ifdim\wd0>\lw
%             \sbox0{\scriptsize\t}^^A
%             \ifdim\wd0>\linewidth
%               \ifdim\wd0>\lw
%                 \sbox0{\tiny\t}^^A
%                 \ifdim\wd0>\linewidth
%                   \lwbox
%                 \else
%                   \usebox0
%                 \fi
%               \else
%                 \lwbox
%               \fi
%             \else
%               \usebox0
%             \fi
%           \else
%             \lwbox
%           \fi
%         \else
%           \usebox0
%         \fi
%       \else
%         \lwbox
%       \fi
%     \else
%       \usebox0
%     \fi
%   \else
%     \lwbox
%   \fi
% \else
%   \usebox0
% \fi
% \end{quote}
% If you have a \xfile{docstrip.cfg} that configures and enables \docstrip's
% TDS installing feature, then some files can already be in the right
% place, see the documentation of \docstrip.
%
% \subsection{Refresh file name databases}
%
% If your \TeX~distribution
% (\TeX\,Live, \mikTeX, \dots) relies on file name databases, you must refresh
% these. For example, \TeX\,Live\ users run \verb|texhash| or
% \verb|mktexlsr|.
%
% \subsection{Some details for the interested}
%
% \paragraph{Unpacking with \LaTeX.}
% The \xfile{.dtx} chooses its action depending on the format:
% \begin{description}
% \item[\plainTeX:] Run \docstrip\ and extract the files.
% \item[\LaTeX:] Generate the documentation.
% \end{description}
% If you insist on using \LaTeX\ for \docstrip\ (really,
% \docstrip\ does not need \LaTeX), then inform the autodetect routine
% about your intention:
% \begin{quote}
%   \verb|latex \let\install=y% \iffalse meta-comment
%
% File: kvoptions.dtx
% Version: 2016/05/16 v3.12
% Info: Key value format for package options
%
% Copyright (C)
%    2004, 2006, 2007, 2009-2011 Heiko Oberdiek
%    2016-2019 Oberdiek Package Support Group
%    https://github.com/ho-tex/oberdiek/issues
%
% This work may be distributed and/or modified under the
% conditions of the LaTeX Project Public License, either
% version 1.3c of this license or (at your option) any later
% version. This version of this license is in
%    https://www.latex-project.org/lppl/lppl-1-3c.txt
% and the latest version of this license is in
%    https://www.latex-project.org/lppl.txt
% and version 1.3 or later is part of all distributions of
% LaTeX version 2005/12/01 or later.
%
% This work has the LPPL maintenance status "maintained".
%
% The Current Maintainers of this work are
% Heiko Oberdiek and the Oberdiek Package Support Group
% https://github.com/ho-tex/oberdiek/issues
%
% This work consists of the main source file kvoptions.dtx
% and the derived files
%    kvoptions.sty, kvoptions.pdf, kvoptions.ins, kvoptions.drv,
%    kvoptions-patch.sty, example-mycolorsetup.sty,
%    kvoptions-test1.tex, kvoptions-test2.tex,
%    kvoptions-test3.tex, kvoptions-test4.tex,
%    kvoptions-test4.sty.
%
% Distribution:
%    CTAN:macros/latex/contrib/oberdiek/kvoptions.dtx
%    CTAN:macros/latex/contrib/oberdiek/kvoptions.pdf
%
% Unpacking:
%    (a) If kvoptions.ins is present:
%           tex kvoptions.ins
%    (b) Without kvoptions.ins:
%           tex kvoptions.dtx
%    (c) If you insist on using LaTeX
%           latex \let\install=y\input{kvoptions.dtx}
%        (quote the arguments according to the demands of your shell)
%
% Documentation:
%    (a) If kvoptions.drv is present:
%           latex kvoptions.drv
%    (b) Without kvoptions.drv:
%           latex kvoptions.dtx; ...
%    The class ltxdoc loads the configuration file ltxdoc.cfg
%    if available. Here you can specify further options, e.g.
%    use A4 as paper format:
%       \PassOptionsToClass{a4paper}{article}
%
%    Programm calls to get the documentation (example):
%       pdflatex kvoptions.dtx
%       makeindex -s gind.ist kvoptions.idx
%       pdflatex kvoptions.dtx
%       makeindex -s gind.ist kvoptions.idx
%       pdflatex kvoptions.dtx
%
% Installation:
%    TDS:tex/latex/oberdiek/kvoptions.sty
%    TDS:tex/latex/oberdiek/kvoptions-patch.sty
%    TDS:doc/latex/oberdiek/kvoptions.pdf
%    TDS:doc/latex/oberdiek/example-mycolorsetup.sty
%    TDS:doc/latex/oberdiek/test/kvoptions-test1.tex
%    TDS:doc/latex/oberdiek/test/kvoptions-test2.tex
%    TDS:doc/latex/oberdiek/test/kvoptions-test3.tex
%    TDS:doc/latex/oberdiek/test/kvoptions-test4.tex
%    TDS:doc/latex/oberdiek/test/kvoptions-test4.sty
%    TDS:source/latex/oberdiek/kvoptions.dtx
%
%<*ignore>
\begingroup
  \catcode123=1 %
  \catcode125=2 %
  \def\x{LaTeX2e}%
\expandafter\endgroup
\ifcase 0\ifx\install y1\fi\expandafter
         \ifx\csname processbatchFile\endcsname\relax\else1\fi
         \ifx\fmtname\x\else 1\fi\relax
\else\csname fi\endcsname
%</ignore>
%<*install>
\input docstrip.tex
\Msg{************************************************************************}
\Msg{* Installation}
\Msg{* Package: kvoptions 2016/05/16 v3.12 Key value format for package options (HO)}
\Msg{************************************************************************}

\keepsilent
\askforoverwritefalse

\let\MetaPrefix\relax
\preamble

This is a generated file.

Project: kvoptions
Version: 2016/05/16 v3.12

Copyright (C)
   2004, 2006, 2007, 2009-2011 Heiko Oberdiek
   2016-2019 Oberdiek Package Support Group

This work may be distributed and/or modified under the
conditions of the LaTeX Project Public License, either
version 1.3c of this license or (at your option) any later
version. This version of this license is in
   https://www.latex-project.org/lppl/lppl-1-3c.txt
and the latest version of this license is in
   https://www.latex-project.org/lppl.txt
and version 1.3 or later is part of all distributions of
LaTeX version 2005/12/01 or later.

This work has the LPPL maintenance status "maintained".

The Current Maintainers of this work are
Heiko Oberdiek and the Oberdiek Package Support Group
https://github.com/ho-tex/oberdiek/issues


This work consists of the main source file kvoptions.dtx
and the derived files
   kvoptions.sty, kvoptions.pdf, kvoptions.ins, kvoptions.drv,
   kvoptions-patch.sty, example-mycolorsetup.sty,
   kvoptions-test1.tex, kvoptions-test2.tex,
   kvoptions-test3.tex, kvoptions-test4.tex,
   kvoptions-test4.sty.

\endpreamble
\let\MetaPrefix\DoubleperCent

\generate{%
  \file{kvoptions.ins}{\from{kvoptions.dtx}{install}}%
  \file{kvoptions.drv}{\from{kvoptions.dtx}{driver}}%
  \usedir{tex/latex/oberdiek}%
  \file{kvoptions.sty}{\from{kvoptions.dtx}{package}}%
  \file{kvoptions-patch.sty}{\from{kvoptions.dtx}{patch}}%
  \usedir{doc/latex/oberdiek}%
  \file{example-mycolorsetup.sty}{\from{kvoptions.dtx}{example}}%
%  \usedir{doc/latex/oberdiek/test}%
%  \file{kvoptions-test1.tex}{\from{kvoptions.dtx}{test1}}%
%  \file{kvoptions-test2.tex}{\from{kvoptions.dtx}{test2}}%
%  \file{kvoptions-test3.tex}{\from{kvoptions.dtx}{test3}}%
%  \file{kvoptions-test4.tex}{\from{kvoptions.dtx}{test4}}%
%  \file{kvoptions-test4.sty}{\from{kvoptions.dtx}{test4pkg}}%
  \nopreamble
  \nopostamble
%  \usedir{source/latex/oberdiek/catalogue}%
%  \file{kvoptions.xml}{\from{kvoptions.dtx}{catalogue}}%
}

\catcode32=13\relax% active space
\let =\space%
\Msg{************************************************************************}
\Msg{*}
\Msg{* To finish the installation you have to move the following}
\Msg{* files into a directory searched by TeX:}
\Msg{*}
\Msg{*     kvoptions.sty, kvoptions-patch.sty}
\Msg{*}
\Msg{* To produce the documentation run the file `kvoptions.drv'}
\Msg{* through LaTeX.}
\Msg{*}
\Msg{* Happy TeXing!}
\Msg{*}
\Msg{************************************************************************}

\endbatchfile
%</install>
%<*ignore>
\fi
%</ignore>
%<*driver>
\NeedsTeXFormat{LaTeX2e}
\ProvidesFile{kvoptions.drv}%
  [2016/05/16 v3.12 Key value format for package options (HO)]%
\documentclass{ltxdoc}
\usepackage{holtxdoc}[2011/11/22]
\begin{document}
  \DocInput{kvoptions.dtx}%
\end{document}
%</driver>
% \fi
%
%
% \CharacterTable
%  {Upper-case    \A\B\C\D\E\F\G\H\I\J\K\L\M\N\O\P\Q\R\S\T\U\V\W\X\Y\Z
%   Lower-case    \a\b\c\d\e\f\g\h\i\j\k\l\m\n\o\p\q\r\s\t\u\v\w\x\y\z
%   Digits        \0\1\2\3\4\5\6\7\8\9
%   Exclamation   \!     Double quote  \"     Hash (number) \#
%   Dollar        \$     Percent       \%     Ampersand     \&
%   Acute accent  \'     Left paren    \(     Right paren   \)
%   Asterisk      \*     Plus          \+     Comma         \,
%   Minus         \-     Point         \.     Solidus       \/
%   Colon         \:     Semicolon     \;     Less than     \<
%   Equals        \=     Greater than  \>     Question mark \?
%   Commercial at \@     Left bracket  \[     Backslash     \\
%   Right bracket \]     Circumflex    \^     Underscore    \_
%   Grave accent  \`     Left brace    \{     Vertical bar  \|
%   Right brace   \}     Tilde         \~}
%
% \GetFileInfo{kvoptions.drv}
%
% \title{The \xpackage{kvoptions} package}
% \date{2016/05/16 v3.12}
% \author{Heiko Oberdiek\thanks
% {Please report any issues at \url{https://github.com/ho-tex/oberdiek/issues}}}
%
% \maketitle
%
% \begin{abstract}
% This package is intended for package authors who want to
% use options in key value format for their package options.
% \end{abstract}
%
% \tableofcontents
%
% \def\M#1{\texttt{\{}\meta{#1}\texttt{\}}}
% \def\O#1{\texttt{[}\meta{#1}\texttt{]}}
%
% \section{Introduction}
%
% First I want to recommend the very good review article
% ``A guide to key-value methods'' by Joseph Wright \cite{tb94wright}.
% It introduces the different key-value packages and compares them.
%
% \subsection{The beginning}
%
% This package \xpackage{kvoptions} addresses class or package writers
% that want to allow their users to specify options
% as key value pairs, e.g.:
% \begin{quote}
% |\documentclass[verbose=false,name=me]{myclass}|\\
% |\usepackage[format=print]{mylayout}|
% \end{quote}
% Prominent example is package \xpackage{hyperref},
% probably the first package that offers this service.
% It's \cs{ProcessOptionsWithKV} is often copied und
% used in other packages, e.g. package \xpackage{helvet}
% that uses this interface for its option \xoption{scaled}.
%
% However copying code is not the most modern software
% development technique. And \xpackage{hyperref}'s
% code for \cs{ProcessOptionsWithKV} was changed
% to fix bugs. The version used in other packages
% depends on the time of copying and the awareness
% of \xpackage{hyperref}'s changes.
% Now the code is sourced out into this package and
% available for other package or class writers.
%
% \subsection{Overview}
%
% Package \xpackage{kvoptions} connects package
% \xpackage{\emph{k}ey\emph{v}al}
% with \LaTeX's package and class \emph{options}:
% \begin{center}
% \def\T#1{\bfseries #1}
% \renewcommand{\arraystretch}{1.5}
% \begin{tabular}{l|l|l}
%   \T{Package \xpackage{keyval}}
%   & \T{Package \xpackage{kvoptions}}
%   & \T{\LaTeX\ kernel}
%   \\
% \hline
%     \cs{define@key}
%   &
%     \renewcommand*{\arraystretch}{1}^^A
%     \begin{tabular}[t]{@{}l@{}}
%       \cs{DeclareVoidOption}\\
%       \cs{DeclareStringOption}\\
%       \cs{DeclareBoolOption}\\
%       \cs{DeclareComplementaryOption}\\
%       \cs{DisableKeyvalOption}
%     \end{tabular}
%   & \cs{DeclareOption}
%   \\
% \hline
%   & \cs{DeclareDefaultOption}
%   & \cs{DeclareOption*}
%   \\
% \hline
%   & \cs{ProcessKeyvalOptions}
%   & \cs{ProcessOptions*}
%   \\
% \hline
%   & Option \xoption{patch}
%   &
%     \renewcommand*{\arraystretch}{1}^^A
%     \begin{tabular}[t]{@{}l@{}}
%       Class/package\\
%       option system
%     \end{tabular}
%   \\
% \hline
%   & \cs{SetupKeyvalOptions} &
%   \\
% \end{tabular}
% \end{center}
%
% \hypersetup{bookmarksopenlevel=3}
% \section{Usage}
%
% \subsection{Process options}
%
% \subsubsection{\cs{ProcessKeyvalOptions}}
%
% \begin{declcs}{ProcessKeyvalOptions}\,\M{family}\\
%   \cs{ProcessKeyvalOptions} \*
% \end{declcs}
% This command evaluates the global or local options of
% the package that are defined with \xpackage{keyval}'s
% interface within the family \meta{family}. It acts the
% same way as \LaTeX's \cs{ProcessOptions*}. In a package
% unknown global options are ignored, in a class they
% are added to the unknown option list. The known global
% options and all local options are passed to \xpackage{keyval}'s
% \cs{setkeys} command for executing the options.
% Unknown options are reported to the user by an error.
%
% If the family name happens to be the same as the name
% of the package or class where \cs{ProcessKeyvalOptions}
% is used or the family name has previously been setup by
% \cs{SetupKeyvalOptions}, then \cs{ProcessKeyvalOptions}
% knows the family name already and you can use the star form
% without mandatory argument.
%
% \subsubsection{\cs{ProcessLocalKeyvalOptions}}
%
% \begin{declcs}{ProcessLocalKeyvalOptions}\,\M{family}\\
%   \cs{ProcessLocalKeyvalOptions} \*
% \end{declcs}
% This macro has the same syntax and works similar as
% \cs{ProcessKeyvalOptions}. However it ignores
% global options and only processes the local package options.
% Therefore it only can be used inside a package.
% An error is thrown, if it is used inside a class.
%
% Neither of the following macros are necessary for
% \cs{ProcessKeyvalOptions{}}. They just help the package/class
% author in common tasks.
%
% \subsubsection{\cs{SetupKeyvalOptions}}\label{setup}
%
% \begin{declcs}{SetupKeyvalOptions}\,\texttt{\{}\\
%   \hspace{1.5em}|family|\,|=|\,\meta{family}|,|\\
%   \hspace{1.5em}|prefix|\,|=|\,\meta{prefix}\\
%   \hspace{1.5em}|setkeys|\,|=|\,\meta{setkeys command}\\
%   \texttt{\}}
% \end{declcs}
% This command allows to configure the default assumptions
% that are based on the current package or class name.
% \LaTeX\ remembers this name in \cs{@currname}. The syntax
% description of the default looks a little weird, therefor
% an example is given for a package or class named |foobar|.
% \begin{quote}
% \begin{tabular}{@{}l|ll|l@{}}
%   Key & Default & (example) & Used by\\
% \hline
%   \rule{0pt}{1.2\ht\csname @arstrutbox\endcsname}^^A
%   |family| & \meta{\cs{@currname}} & (|foobar|)
%       & \cs{ProcessKeyvalOptions*} \\
%     &&& \cs{DeclareBoolOption} \\
%     &&& \cs{DeclareStringOption} \\
% \hline
%   \rule{0pt}{1.2\ht\csname @arstrutbox\endcsname}^^A
%   |prefix| & \meta{\cs{@currname}}|@| & (|foobar@|)
%       & \cs{DeclareBoolOption} \\
%     &&& \cs{DeclareStringOption} \\
%     &&& \cs{DeclareVoidOption} \\
% \hline
%   \rule{0pt}{1.2\ht\csname @arstrutbox\endcsname}^^A
%   |setkeys| & \cs{setkeys} & (\cs{kvsetkeys})
%       & \cs{ProcessKeyvalOptions} \\
%     &&& \cs{ProcessLocalKeyvalOptions}
% \end{tabular}
% \end{quote}
% Key |setkeys| was added in version 3.9. The original
% \cs{setkeys} of package \xpackage{keyval} is not reentrant.
% If an option is processed by this \cs{setkeys}, then
% the option should not call \cs{setkeys} again with a
% different family. Otherwise the next options of the
% first \cs{setkeys} call are processed with the wrong
% family. With key |setkeys| the macro \cs{kvsetkeys}
% can be set that does not have the problem of the
% original \cs{setkeys} of package {keyval}.
%
% Probably \cs{setkeys} of package \xpackage{xkeyval}
% is safe in this respect. But I haven't made a full
% analysis. At least it does not have the problem
% of the original \cs{setkeys}.
%
% \subsection{Option declarations}
%
% The options for \cs{ProcessKeyvalOptions} are defined
% by \xpackage{keyval}'s \cs{define@key}.
% Common purposes of such keys are boolean switches, they
% enable or disable something. Or they store a name or some kind of
% string in a macro. The following commands help the user. He
% declares what he wants and \xpackage{kvoptions} take care of
% the key definition, resource allocation and initialization.
%
% In order to avoid name clashes of macro names, internal
% commands are prefixed. Both the prefix and the family
% name for the defined keys can be configured by
% \cs{SetupKeyvalOptions}.
%
% \subsubsection{\cs{DeclareStringOption}}
%
% \begin{declcs}{DeclareStringOption}\,\O{init}\,\M{key}\,\O{default}
% \end{declcs}
% A macro is created that remembers the value of the key \meta{key}.
% The name of the macro consists of the option name \meta{key} that
% is prefixed by the prefix (see \ref{setup}).
% The initial contents of the
% macro can be given by the first optional argument \meta{init}.
% The default is empty.
%
% The the option \meta{key} is defined. The option code
% just stores its value in the macro. If the optional argument
% at the end of \cs{DeclareStringOption} is given, then option
% \meta{key} is defined with the default \meta{default}.
%
% Example for a package with the following two lines:
% \begin{quote}
% |\ProvidesPackage{foobar}|\\
% |\DeclareStringOption[me]{name}|
% \end{quote}
% Then \cs{DeclareStringOption} defines the macro with content |me|,
% note \LaTeX\ complains if the name of the macro already exists:
% \begin{quote}
% |\newcommand*{\foobar@name}{me}|
% \end{quote}
% The option definition is similar to:
% \begin{quote}
% |\define@key{foobar}{name}{%|\\
% |  \renewcommand*{\foobar@name}{#1}%|\\
% |}|
% \end{quote}
%
% \subsubsection{\cs{DeclareBoolOption}}
%
% \begin{declcs}{DeclareBoolOption}\,\O{init}\,\M{key}
% \end{declcs}
% A boolean switch is generated, initialized by value \meta{init}
% and the corresponding key \meta{key} is defined. If the
% initialization value is not given, |false| is used as default.
%
% The internal actions of \cs{DeclareBoolOption} are shown below.
% The example is given for a package author who has the following
% two lines in his package/class:
% \begin{quote}
% |\ProvidesPackage{foobar}|\\
% |\DeclareBoolOption{verbose}|
% \end{quote}
% First a new switch is created:
% \begin{quote}
%   |\newif\iffoobar@verbose|
% \end{quote}
% and initialized:
% \begin{quote}
%   |\foobar@verbosefalse|
% \end{quote}
% Finally the key is defined:
% \begin{quote}
%   |\define@key{foobar}{verbose}[true]{|\dots|}|
% \end{quote}
% The option code configures the boolean option in the following way:
% If the author specifies |true| or |false| then the switch is
% turned on or off respectivly. Also the option can be given without
% explicit value. Then the switch is enabled.
% Other values are reported as errors.
%
% Now the switch is ready to use in the package/class, e.g.:
% \begin{quote}
%   |\iffoobar@verbose|\\
%   \qquad\textit{\% print verbose message}\\
%   |\else|\\
%   \qquad\textit{\% be quiet}\\
%   |\fi|
% \end{quote}
% Users of package \cs{ifthen} can use the switch as
% boolean:
% \begin{quote}
%   |\boolean{foobar@verbose}|
% \end{quote}
%
% \subsubsection{\cs{DeclareComplementaryOption}}
%
% \begin{declcs}{DeclareComplementaryOption}\,\M{key}\,\M{parent}
% \end{declcs}
% Sometimes contrasting names are used to characterize the
% two states of a boolean switch, for example \xoption{draft}
% vs.\@ \xoption{final}. Both options behave like boolean
% options but they do not need two different switches, they
% should share one. \cs{DeclareComplementaryOption} allows this.
% The option \meta{key} shares the switch of option \meta{parent}.
% Example:
% \begin{quote}
%   |\DeclareBoolOption{draft}|\\
%   |\DeclareComplementaryOption{final}{draft}|
% \end{quote}
% Then |final| sets the switch of \xoption{draft} to |false|, and
% |final=false| enables the \xoption{draft} switch.
%
% \subsubsection{\cs{DeclareVoidOption}}
%
% \begin{declcs}{DeclareVoidOption}\,\M{key}\,\M{code}
% \end{declcs}
% \cs{ProcessKeyvalOptions} can be extended to recognize
% options that are declared in traditional way by \cs{DeclareOption}.
% But in case of the error that the user specifies a value, then
% this option would not recognized as key value option because
% of \cs{DeclareOption} and not detected as traditional option
% because of the value part. The user would get an unknown option
% error, difficult to understand.
%
% \cs{DeclareVoidOption} solves
% this problem. It defines the option \meta{key} as key value option.
% If the user specifies a value, a warning is given and the
% value is ignored.
%
% The code part \meta{code} is stored in a macro.
% The name of the macro consists of the option name \meta{key} that
% is prefixed by the prefix (see \ref{setup}).
% If the option is set, the macro will be executed. During the
% execution \cs{CurrentOption} is available with the current key name.
%
% \subsubsection{\cs{DeclareDefaultOption}}
%
% \begin{declcs}{DeclareDefaultOption}\,\M{code}
% \end{declcs}
% \begin{sloppypar}
% This command does not define a specific key, it is the equivalent
% to \LaTeX's \cs{DeclareOption*}. It allows the specification
% of a default action \meta{code} that is invoked if an unknown option
% is found.
% While \meta{code} is called,
% macro \cs{CurrentOption} contains the current option string.
% In addition \cs{CurrentOptionValue} contains the value part
% if the option string is parsable as key value pair, otherwise
% it is \cs{relax}. \cs{CurrentOptionKey} contains the key of
% the key value pair, or the whole option string, if it misses
% the equal sign.
% \end{sloppypar}
%
% Inside packages typical default actions are to pass unknown
% options to another package. Or an error message can be thrown
% by \cs{@unknownoptionerror}. This is the original error
% message that \LaTeX\ gives for unkown package options.
% This error message is easier to understand for the user
% as the error message from package \xpackage{keyval} that
% is given otherwise.
%
% A Class ignores unknown options and puts them on the
% unused option list. Let \LaTeX\ do the job and just
% call \cs{OptionNotUsed}. Or the options can be passed to
% another class that is later loaded.
%
% \subsubsection{Local options}
%
% \begin{declcs}{DeclareLocalOption} \M{option}\\
%   \cs{DeclareLocalOptions} \M{option list}
% \end{declcs}
% Both macros mark package options as local options. That means
% that they are ignored by \cs{ProcessKeyvalOptions} if they are given
% as global options. \cs{DeclareLocalOptions} takes one option,
% \cs{DeclareLocalOptions} expects a comma separated list of options.
%
% \subsubsection{Dynamic options}
%
% Options of \LaTeX's package/class system are cleared
% in \cs{ProcessOptions}. They modify the static model
% of a package. For example, depending on option \xoption{bookmarks}
% package \xpackage{hyperref} loads differently.
%
% Options, however,
% defined by \xpackage{keyval}'s \cs{define@key} remain
% defined, if the options are processed by \cs{setkeys}.
% Therefore these options can also be used to model
% the dynamic behaviour of a package. For example, in
% \xpackage{hyperref} the link colors can be changed
% everywhere until the end in |\end{document}|.
%
% However package \xpackage{color} that adds color support
% is necessary and it
% cannot be loaded after |\begin{document}|. Option \xoption{colorlinks}
% that loads \xpackage{color} should be active until |\begin{document}|
% and die in some way if it is too late for loading packages.
% With \cs{DisableKeyvalOption} the package/class author can
% specify and configure the death of an option and controls
% the life period of the option.
%
% \subsubsection{\cs{DisableKeyvalOption}}
%
% \begin{declcs}{DisableKeyvalOption}\,\O{options}\,\M{family}\,\M{key}
%   \\[0.5ex]
%   \meta{options}:\\
%   \hspace{1.5em}^^A
%   \begin{tabular}[t]{@{}l@{ \texttt{=} }l@{\qquad}>{default: }l@{}}
%     |action| & |undef|, |warning|, |error|, or |ignore|
%     & |undef|
%     \\
%     \multicolumn{1}{@{}l}{|global| or |local|} & & |global|
%     \\
%     |package| or |class| & \meta{name}
%     \\
%   \end{tabular}^^A
% \end{declcs}
% \cs{DisableKeyvalOption} can be called to mark the end when
% the option \meta{key} is no longer useful. The behaviour of
% an option after its death can be configured by \xoption{action}:
% \begin{description}
% \item[\texttt{undef}:] The option will be undefined,
%   If it is called, \cs{setkeys} reports an error because
%   of unknown key.
% \item[\texttt{error} or \texttt{warning}:] Use of the
%  option will cause an error or warning message.
%  Also these actions require that exclusivly either the
%  package or class name is given in options \xoption{package}
%  or \xoption{class}.
% \item[\texttt{ignore}:] The use of the option will silently
%  be ignored.
% \end{description}
% The option's death can be limited to the end of the current
% group, if option \xoption{local} is given. Default is
% \xoption{global}.
%
% The package/class author can wish the end of the option
% already during the package loading, then he will have
% static behaviour. In case of dynamic options \cs{DisableKeyvalOption}
% can be executed everywhere, also outside the package. Therefore
% the family name and the package/class name is usually unknown for
% \cs{DisableKeyvalOption}. Therefore the argument for the family
% name is mandatory and for some actions the package/class name
% must be provided.
%
% Usually a macro would configure the option death, Example:
%\begin{quote}
%\begin{verbatim}
%\ProvidesPackage{foobar}
%\DeclareBoolOption{color}
%\DeclareStringOption[red]{emphcolor}
%\ProcessKeyvalOptions*
%
%\newcommand*{\foobar@DisableOption}[2]{%
%  \DisableKeyvalueOption[
%    action={#1},
%    package=foobar
%  ]{foobar}{#2}%
%}
%
%\iffoobar@color
%  \RequirePackage{color}
%  \renewcommand*{\emph}[1]{\textcolor{\foobar@emphcolor}{#1}}
%\else
%  % Option emphcolor is not wrong, if we have color support.
%  % otherwise the option has no effect, but we don't want to
%  % remove it. Therefore action 'ignore' is the best choice:
%  \foobar@DisableOption{ignore}{emphcolor}
%\fi
%% No we don't need the option 'color'.
%\foobar@DisableOption{warning}{color}
%
%% With color support option 'emphcolor' will dynamically
%% change the color of \emph statements.
%\end{verbatim}
%\end{quote}
%
% \subsubsection{\cs{AddToKeyvalOption}}
%
% \begin{declcs}{AddToKeyvalOption}\,\M{family}\,\M{key}\,\M{code}\\
%   \cs{AddToKeyvalOption} \*\,\M{key}\,\M{code}
% \end{declcs}
% The code for an existing key \meta{key} of family \meta{family} is
% extended by code \meta{code}. In the starred form the current
% family setting is used, see \cs{ProcessKeyvalOptions*}.
%
% \subsection{Global vs. local options}
%
% Options that are given for \cs{documentclass} are called
% global options. They are known to the class and all packages.
% A package may make use of a global option and marks it as
% used. The advantage for the user is the freedom to specify
% options both in the \cs{documentclass} or \cs{usepackage}
% commands.
%
% However global options are shared with the class options
% and options of all other packages. Thus there can be the
% same option with different semantics for different
% packages and classes. As example, package \xpackage{bookmark}
% knows option \xoption{open} that specifies whether the bookmarks
% are opened or closed initially. It's values are |true| or |false|.
% Since KOMA-Script version 3.00 the KOMA classes also introduces
% option \xoption{open} with values |right| and |any| and a
% complete different meaning.
%
% Such conflicts can be resolved by marking all or part of options
% as local by \cs{DeclareLocalOption} or \cs{DeclareLocalOptions}.
% Then the packages ignores global occurences of these options.
% Package \xpackage{kvoptions} provides two methods:
% \begin{itemize}
% \item \cs{ProcessLocalKeyvalOptions} automatically uses all options
%   as local options. It ignores all global options.
% \item \cs{DeclareLocalOption} or \cs{DeclareLocalOptions} marks
%   options as local options. \cs{ProcessKeyvalOptions} will then
%   ignore global occurences for these local options.
% \end{itemize}
% Since version 1.5 package \xpackage{bookmark} uses the latter
% method. It checks global and local option places for driver options
% and limits all other options as local options. Thus the class
% option \xoption{open} of KOMA-Script is not misread as
% option for package \xpackage{bookmark}.
%
% \subsection{Summary of internal macros}
%
% The \cs{Declare}\texttt{\dots Option} commands define
% macros, additionally to the macros generated by the key definition.
% These macros can be used by the package/class author.
% The name of the macros starts with the prefix \meta{prefix}
% that can be configured by \cs{SetupKeyvalOptions}.
% \begin{center}
% \def\prefix{\cs{}\meta{prefix}\meta{key}}^^A
% \def\Hline{^^A
%   \hline
%   \rule{0pt}{1.2\ht\csname @arstrutbox\endcsname}^^A
% }^^A
% \begin{tabular}{@{}l|>{\ttfamily}l|l@{}}
%   Declare \meta{key} & \textrm{Defined macro} & Description\\
% \Hline
%   \cs{DeclareStringOption}
%    & \prefix & holds the string\\
% \Hline
%   \cs{DeclareBoolOption}
%     & \cs{if}\meta{prefix}\meta{key}& boolean switch\\
%   &\prefix false& disable switch\\
%   &\prefix true& enable switch\\
% \Hline
%   \cs{DeclareComplementaryOption}
%     & \prefix false& enable parent switch\\
%     & \prefix true& disable parent switch\\
% \Hline
%   \cs{DeclareVoidOption}
%   & \prefix & holds the action
% \end{tabular}
% \end{center}
%
% \subsection{\plainTeX}
%
% Package \xpackage{keyval} is also usable in \plainTeX\ with
% the help of file \xfile{miniltx.tex}. Some features of this
% package \xpackage{kvoptions} might also be useful for \plainTeX.
% If \LaTeX\ is not found, \cs{ProcessKeyvalOptions} and option
% \xoption{patch} are disabled. Before using the option declaration
% commands \cs{Declare...Option}, \cs{SetupKeyvalOptions} must be
% used.
%
% \hypersetup{bookmarksopenlevel=1}
%
% \section{Example}
%
% The following example defined a package that serves some
% private color management. A boolean option \xoption{print} enables
% print mode without colors. An option \xoption{emph} redefines
% \cs{emph} to print in the given color. And the driver
% can be specified by option \xoption{driver}.
%
%    \begin{macrocode}
%<*example>
    % Package identification
    % -----------------------
\NeedsTeXFormat{LaTeX2e}
\ProvidesPackage{example-mycolorsetup}[2016/05/16 Managing my colors]

\RequirePackage{ifpdf}
\RequirePackage{kvoptions}

    % Option declarations
    % -------------------

\SetupKeyvalOptions{
  family=MCS,
  prefix=MCS@
}
    % Use a shorter family name and prefix

    % Option print
\DeclareBoolOption{print}
    % is the same as
    % \DeclareBoolOption[false]{print}

    % Option driver
\ifpdf
  \DeclareStringOption[pdftex]{driver}
\else
  \DeclareStringOption[dvips]{driver}
\fi

    % Alternative interface for driver options
\DeclareVoidOption{dvips}{\SetupDriver}
\DeclareVoidOption{dvipdfm}{\SetupDriver}
\DeclareVoidOption{pdftex}{\SetupDriver}
    % In \SetupDriver we take the current option \CurrentOption
    % and pass it to the driver option.
    % The \expandafter commands expand \CurrentOption at the
    % time, when \SetupDriver is executed and \CurrentOption
    % has the correct meaning.
\newcommand*{\SetupDriver}{%
  \expandafter\@SetupDriver\expandafter{\CurrentOption}%
}
\newcommand*{\@SetupDriver}[1]{%
  \setkeys{MCS}{driver={#1}}%
}

    % Option emph
    % An empty value means, we want to have no color for \emph.
    % If the user specifies option emph without value, the red is used.
\DeclareStringOption{emph}[red]
    % is the same as
    % \DeclareStringOption[]{emph}[red]

    % Default option rule
\DeclareDefaultOption{%
  \ifx\CurrentOptionValue\relax
    \PackageWarningNoLine{\@currname}{%
      Unknown option `\CurrentOption'\MessageBreak
      is passed to package `color'%
    }%
    % Pass the option to package color.
    % Again it is better to expand \CurrentOption.
    \expandafter\PassOptionsToPackage
    \expandafter{\CurrentOption}{color}%
  \else
    % Package color does not take options with values.
    % We provide the standard LaTeX error.
    \@unknownoptionerror
  \fi
}

    % Process options
    % ---------------
\ProcessKeyvalOptions*

    % Implementation depending on option values
    % -----------------------------------------
    % Code for print mode
\ifMCS@print
  \PassOptionsToPackage{monochrome}{color}
    % tells package color to use black and white
\fi

\RequirePackage[\MCS@driver]{color}
    % load package color with the correct driver

    % \emph setup
\ifx\MCS@emph\@empty
    % \@empty is a predefined macro with empty contents.
    % the option value of option emph is empty, thus
    % we do not want a redefinition of \emph.
\else
  \renewcommand*{\emph}[1]{%
    \textcolor{\MCS@emph}{#1}%
  }
\fi
%</example>
%    \end{macrocode}
%
% \section{Package options}
%
% The package \xpackage{kvoptions} knows two package options
% \xoption{patch} and \xoption{debugshow}.
% The options of package \xpackage{kvoptions} are intended
% for authors, not for package/class writers. Inside a package
% it is too late for option \xoption{patch} and \xoption{debugshow}
% enables some messages that are perhaps useful for the debugging
% phase. Also \LaTeX\ is unhappy if a package is loaded
% later again with options that are previously not given.
% Thus package and class authors, stay with |\RequirePackage{kvoptions}|
% without options.
%
% Option \xoption{patch} loads package \xpackage{kvoptions-patch}.
%
% \subsection{Package \xpackage{kvoptions-patch}}
%
% \LaTeX's system of package/class options has some severe limitations
% that especially affects the value part if options are used as
% pair of key and value.
% \begin{itemize}
% \item Spaces are removed, regardless where:
%   \begin{quote}
%     |\documentclass[box=0 0 400 600]{article}|
%   \end{quote}
%   Now each package will see |box=00400600| as global option.
% \item In the previous case also braces would not help:
%   \begin{quote}
%     |\documentclass[box={0 0 400 600}]{article}|
%   \end{quote}
%   The result is an error message:
%   \begin{quote}
%     |! LaTeX Error: Missing \begin{document}.|
%   \end{quote}
%   As local option, however, it works if the package
%   knows about key value options (By using this package, for example).
% \item The requirements on robustness are extremly high.
%   \LaTeX\ expands the option. All that will not work as environment
%   name will break also as option. Even a \cs{relax} will generate
%   an error message:
%   \begin{quote}
%     |! Missing \endcsname inserted.|
%   \end{quote}
%   Of course, \LaTeX\ does not use its protecting mechanisms.
%   On contrary \cs{protect} itself will cause errors.
% \item The options are expanded. But perhaps the package will
%   do that, because it has to setup some things before?
%   Example \xpackage{hyperref}:
%   \begin{quote}
%     |\usepackage[pdfauthor=M\"uller]{hyperref}|
%   \end{quote}
%   Package \xpackage{hyperref} does not see |M\"uller| but
%   its expansion and it does not like it, you get many warnings
%   \begin{quote}
%     |Token not allowed in a PDFDocEncoded string|
%   \end{quote}
%   And the title becomes: |Mu127uller|.
%   Therefore such options must usually be given after package
%   \xpackage{hyperref}
%   is loaded:
%   \begin{quote}
%     |\usepackage{hyperref}|\\
%     |\hypersetup{pdfauthor=Fran\c coise M\"uller}|
%   \end{quote}
%   As package option it will even break with |Fran\c coise|
%   because of the cedilla |\c c|, it is not robust enough.
% \end{itemize}
% For users that do not want with this limitations the package
% offers package \xoption{kvoptions-patch}. It patches \LaTeX's option system
% and tries to teach it also to handle options that are given
% as pairs of key and value and to prevent expansion.
% It can already be used at the very beginning, before \cs{documentclass}:
% \begin{quote}
%   |\RequirePackage{kvoptions-patch}|\\
%   |\documentclass[pdfauthor=Fran\c coise M\"uller]{article}|\\
%   |\usepackage{hyperref}|
% \end{quote}
% The latest time is before the package where you want to use
% problematic values:
% \begin{quote}
%   |\usepackage{kvoptions-patch}|\\
%   |\usepackage[Fran\c coise M\"uller]{hyperref}|
% \end{quote}
% Some remarks:
% \begin{itemize}
% \item The patch requires \eTeX, its \cs{unexpanded} feature
%   is much too nice. It is possible to work around using token
%   registers. But the code becomes longer, slower, more difficult
%   to read and maintain. The package without option \xoption{patch}
%   works and will work without \eTeX.
% \item The code for the patch is quite long, there are many test
%   cases. Thus the probability for bugs is probably not too small.
% \item Since 2008/10/18 v3.0 package \xpackage{kvoptions-patch} is
%   available. Before option \xoption{patch} of package \xpackage{kvoptions}
%   must be used instead. I think, the solution as standalone package
%   \xpackage{kvoptions-patch} is cleaner and avoids option clashes.
% \end{itemize}
%
% \subsection{Option \xoption{debugshow}}
%
% The name of this option follows the convention of packages
% \xpackage{multicol}, \xpackage{tabularx}, and \xpackage{tracefnt}.
% Currently it prints the setting of boolean options, declared
% by \cs{DeclareBoolOption} in the \xfile{.log} file, if that
% boolean option is used.
% You can activate the option by
% \begin{itemize}
% \item |\PassOptionsToPackage{debugshow}{kvoptions}|\\
%       Put this somewhere before package \xpackage{kvoptions} is
%       loaded first, e.g. before \cs{documentclass}.
% \item |\RequirePackage[debugshow]{kvoptions}|\\
%       Before \cs{documentclass} even an author has to use
%       \cs{RequirePackage}. \cs{usepackage} only works after
%       \cs{documentclass}.
% \end{itemize}
% The preferred method is \cs{PassOptionsToPackage}, because
% it does not force the package loading and does not disturb,
% if the package is not loaded later at all.
%
% \section{Limitations}
%
% \subsection{Compatibility}
% \subsubsection{Package \xpackage{kvoptions-patch} vs. package \xpackage{xkvltxp}}
%
% Package \xpackage{xkvltxp} from the \xpackage{xkeyval} project
% has the same goal as package \xpackage{kvoptions-patch}
% and to patch \LaTeX's kernel commands in order to get better support
% for key value options.
% Of course they cannot be used both. The user must decide, which
% method he prefers. Package \xpackage{kvoptions-patch} aborts itself,
% if it detects that \xpackage{xkvltxp} is
% already loaded.
%
% However package \xpackage{xkvltxp} and \xpackage{kvoptions}
% can be used together, example:
% \begin{quote}
% |\usepackage{xkvltxp}|\\
% |\usepackage[...]{foobar} % |\texttt{\itshape foobar using kvoptions}
% \end{quote}
% The other way should work, too.
%
% Package \xpackage{kvoptions-patch} tries to catch more situations
% and to be more robust.
% For example, during the comparison of options it normalizes
% them by removing spaces around |=| and the value. Thus the
% following is not reported as option clash:
% \begin{quote}
%\begin{verbatim}
%\RequirePackage{kvoptions-patch}
%\documentclass{article}
%
%\usepackage[scaled=0.7]{helvet}
%\usepackage[scaled = 0.7]{helvet}
%
%\begin{document}
%\end{document}
%\end{verbatim}
% \end{quote}
%
% \subsection{Limitations}
%
% \subsubsection{Option comparisons}
%
% In some situations \LaTeX\ compares option lists, e.g. option clash check,
% \cs{@ifpackagewith}, or \cs{@ifclasswith}. Apart from catcode and
% sanitizing problems of option \xoption{patch}, there is another problem.
% \LaTeX\ does not know about the type and default values of
% options in key value style. Thus an option clash is reported,
% even if the key value has the same meaning:
% \begin{quote}
% |\usepackage[scaled]{helvet} %| \texttt{\textit{default is }.95}\\
% |\usepackage[.95]{helvet}|\\
% |\usepackage[0.95]{helvet}|
% \end{quote}
%
% \subsubsection{Option list parsing with package \xpackage{kvoptions-patch}}
%
% With package \xpackage{kvoptions-patch} the range of possible values
% in key value specifications is much large, for example
% the comma can be used, if enclosed in curly braces.
%
% Other packages, especially the packages that uses their
% own process option code can be surprised to find tokens
% inside options that they do not expect and errors would
% be the consequence. To avoid errors the options, especially
% the unused option list is sanitized. That means the list
% will only contain tokens with catcode 12 (other) and
% perhaps spaces (catcode 10). This allows a safe parsing
% for other packages. But a comma in the value part is
% no longer protected by curly braces because they have lost
% their special meaning. This is the price for compatibility.
%
% Example:
% \begin{quote}
% |\RequirePackage{kvoptions-patch}|\\
% |\documentclass[a={a,b,c},b]{article}|\\
% |\begin{document}|\\
% |\end{document}|
% \end{quote}
% Result:
% \begin{quote}
% |LaTeX Warning: Unused global option(s):|\\
% |    [a={a,c},b].|
% \end{quote}
%
% \StopEventually{
% }
%
% \section{Implementation}
%
% \subsection{Preamble}
%
%    \begin{macrocode}
%<*package>
%    \end{macrocode}
%
% \paragraph{Reload check and  identification.}
%    Reload check, especially if the package is not used with \LaTeX.
%    \begin{macrocode}
\begingroup\catcode61\catcode48\catcode32=10\relax%
  \catcode13=5 % ^^M
  \endlinechar=13 %
  \catcode35=6 % #
  \catcode39=12 % '
  \catcode44=12 % ,
  \catcode45=12 % -
  \catcode46=12 % .
  \catcode58=12 % :
  \catcode64=11 % @
  \catcode123=1 % {
  \catcode125=2 % }
  \expandafter\let\expandafter\x\csname ver@kvoptions.sty\endcsname
  \ifx\x\relax % plain-TeX, first loading
  \else
    \def\empty{}%
    \ifx\x\empty % LaTeX, first loading,
      % variable is initialized, but \ProvidesPackage not yet seen
    \else
      \expandafter\ifx\csname PackageInfo\endcsname\relax
        \def\x#1#2{%
          \immediate\write-1{Package #1 Info: #2.}%
        }%
      \else
        \def\x#1#2{\PackageInfo{#1}{#2, stopped}}%
      \fi
      \x{kvoptions}{The package is already loaded}%
      \aftergroup\endinput
    \fi
  \fi
\endgroup%
%    \end{macrocode}
%    Package identification:
%    \begin{macrocode}
\begingroup\catcode61\catcode48\catcode32=10\relax%
  \catcode13=5 % ^^M
  \endlinechar=13 %
  \catcode35=6 % #
  \catcode39=12 % '
  \catcode40=12 % (
  \catcode41=12 % )
  \catcode44=12 % ,
  \catcode45=12 % -
  \catcode46=12 % .
  \catcode47=12 % /
  \catcode58=12 % :
  \catcode64=11 % @
  \catcode91=12 % [
  \catcode93=12 % ]
  \catcode123=1 % {
  \catcode125=2 % }
  \expandafter\ifx\csname ProvidesPackage\endcsname\relax
    \def\x#1#2#3[#4]{\endgroup
      \immediate\write-1{Package: #3 #4}%
      \xdef#1{#4}%
    }%
  \else
    \def\x#1#2[#3]{\endgroup
      #2[{#3}]%
      \ifx#1\@undefined
        \xdef#1{#3}%
      \fi
      \ifx#1\relax
        \xdef#1{#3}%
      \fi
    }%
  \fi
\expandafter\x\csname ver@kvoptions.sty\endcsname
\ProvidesPackage{kvoptions}%
  [2016/05/16 v3.12 Key value format for package options (HO)]%
%    \end{macrocode}
%
% \paragraph{Catcodes}
%
%    \begin{macrocode}
\begingroup\catcode61\catcode48\catcode32=10\relax%
  \catcode13=5 % ^^M
  \endlinechar=13 %
  \catcode123=1 % {
  \catcode125=2 % }
  \catcode64=11 % @
  \def\x{\endgroup
    \expandafter\edef\csname KVO@AtEnd\endcsname{%
      \endlinechar=\the\endlinechar\relax
      \catcode13=\the\catcode13\relax
      \catcode32=\the\catcode32\relax
      \catcode35=\the\catcode35\relax
      \catcode61=\the\catcode61\relax
      \catcode64=\the\catcode64\relax
      \catcode123=\the\catcode123\relax
      \catcode125=\the\catcode125\relax
    }%
  }%
\x\catcode61\catcode48\catcode32=10\relax%
\catcode13=5 % ^^M
\endlinechar=13 %
\catcode35=6 % #
\catcode64=11 % @
\catcode123=1 % {
\catcode125=2 % }
\def\TMP@EnsureCode#1#2{%
  \edef\KVO@AtEnd{%
    \KVO@AtEnd
    \catcode#1=\the\catcode#1\relax
  }%
  \catcode#1=#2\relax
}
\TMP@EnsureCode{1}{14}% ^^A (comment)
\TMP@EnsureCode{2}{14}% ^^A (comment)
\TMP@EnsureCode{33}{12}% !
\TMP@EnsureCode{39}{12}% '
\TMP@EnsureCode{40}{12}% (
\TMP@EnsureCode{41}{12}% )
\TMP@EnsureCode{42}{12}% *
\TMP@EnsureCode{44}{12}% ,
\TMP@EnsureCode{45}{12}% -
\TMP@EnsureCode{46}{12}% .
\TMP@EnsureCode{47}{12}% /
\TMP@EnsureCode{58}{12}% :
\TMP@EnsureCode{62}{12}% >
\TMP@EnsureCode{91}{12}% [
\TMP@EnsureCode{93}{12}% ]
\TMP@EnsureCode{94}{7}% ^ (superscript)
\TMP@EnsureCode{96}{12}% `
\edef\KVO@AtEnd{\KVO@AtEnd\noexpand\endinput}
%    \end{macrocode}
%
% \paragraph{External resources.}
%    The package extends the support for key value pairs of
%    package \cs{keyval} to package options. Thus the package
%    needs to be loaded anyway. and we use it for
%    \cs{SetupKeyvalOptions}. AFAIK this does not disturb
%    users of \xpackage{xkeyval}.
%    \begin{macrocode}
\@ifundefined{define@key}{%
  \RequirePackage{keyval}\relax
}{}
%    \end{macrocode}
%
%    Macro \cs{DeclareLocalOptions} parses a comma separated key list
%    and uses \cs{comma@parse} of package \xpackage{kvsetkeys},
%    version 1.3.
%    \begin{macrocode}
\RequirePackage{ltxcmds}[2010/12/02]
\RequirePackage{kvsetkeys}[2007/09/29]
%    \end{macrocode}
%
% \paragraph{Provide macros for \plainTeX.}
%    \begin{macrocode}
\@ifundefined{@x@protect}{%
  \def\@x@protect#1\fi#2#3{%
    \fi\protect#1%
  }%
  \let\@typeset@protect\relax
}{}
\@ifundefined{@currname}{%
  \def\@currname{}%
}{}
\@ifundefined{@currext}{%
  \def\@currext{}%
}{}
%    \end{macrocode}
%
% \paragraph{Options}
%    Option \xoption{debugshow} enables additional lines of
%    code that prints information into the \xfile{.log}
%    file.
%    \begin{macrocode}
\DeclareOption{debugshow}{\catcode\@ne=9 }
%    \end{macrocode}
%
%    \begin{macrocode}
\DeclareOption{patch}{%
  \AtEndOfPackage{%
    \RequirePackage{kvoptions-patch}[2016/05/16]%
  }%
}
%    \end{macrocode}
%
%    Optionen auswerten:
%    \begin{macrocode}
\ProcessOptions\relax
%    \end{macrocode}
%
% \subsection{Option declaration macros}
%
% \subsubsection{\cs{SetupKeyvalOptions}}
%
%    The family for the key value pairs can be setup once
%    and is remembered later.
%    The package name seems a reasonable default for the
%    family key, if it is not set by the package author.
%
%    \begin{macro}{\KVO@family}
%    We cannot store the family setting in one macro,
%    because the package should be usable for many
%    other packages, too. Thus we remember the family
%    setting in a macro, whose name contains the
%    package name with extension, a key in \LaTeX's
%    class/package system.
%    \begin{macrocode}
\define@key{KVO}{family}{%
  \expandafter\edef\csname KVO@family@%
      \@currname.\@currext\endcsname{#1}%
}
\def\KVO@family{%
  \@ifundefined{KVO@family@\@currname.\@currext}{%
    \@currname
  }{%
    \csname KVO@family@\@currname.\@currext\endcsname
  }%
}
%    \end{macrocode}
%    \end{macro}
%
%    \begin{macro}{\KVO@prefix}
%    The value settings of options that are declared by \cs{DeclareBoolOption}
%    and \cs{DeclareStringOption} need to be saved in macros.
%    in the first case this is a switch \cs{if}\meta{prefix}\meta{key},
%    in the latter case a macro \cs{}\meta{prefix}\meta{key}.
%    The prefix can be configured, by \xoption{prefix} that is declared
%    here. The default is the package name with |@| appended.
%    \begin{macrocode}
\define@key{KVO}{prefix}{%
  \expandafter\edef\csname KVO@prefix@%
      \@currname.\@currext\endcsname{#1}%
}
\def\KVO@prefix{%
  \ltx@ifundefined{KVO@prefix@\@currname.\@currext}{%
    \@currname @%
  }{%
    \csname KVO@prefix@\@currname.\@currext\endcsname
  }%
}
%    \end{macrocode}
%    \end{macro}
%
%    \begin{macrocode}
\define@key{KVO}{setkeys}{%
  \expandafter\def\csname KVO@setkeys@%
      \@currname.\@currext\endcsname{#1}%
}
%    \end{macrocode}
%    \begin{macro}{\KVO@setkeys}
%    \begin{macrocode}
\def\KVO@setkeys{%
  \ltx@IfUndefined{KVO@setkeys@\@currname.\@currext}{%
    \setkeys
  }{%
    \csname KVO@setkeys@\@currname.\@currext\endcsname
  }%
}
%    \end{macrocode}
%    \end{macro}
%
%    \begin{macro}{\SetupKeyvalOptions}
%    The argument of \cs{SetupKeyvalOptions} expects a key value
%    list, known keys are \xoption{family} and \xoption{prefix}.
%    \begin{macrocode}
\newcommand*{\SetupKeyvalOptions}{%
  \kvsetkeys{KVO}%
}
%    \end{macrocode}
%    \end{macro}
%
% \subsubsection{\cs{DeclareBoolOption}}
%
%    \begin{macro}{\DeclareBoolOption}
%    Usually options of boolean type can be given by the
%    user without value and this means a setting to \emph{true}.
%    We follow this convention here. Also it simplifies the
%    user interface.
%
%    The switch is created and initialized with \emph{false}.
%    The default setting can be overwritten by the optional
%    argument.
%
%    \LaTeX's \cs{newif} does not check for already defined
%    macros, therefore we add this check here to prevent
%    the user from accidently redefining of \TeX's primitives
%    and other macros.
%    \begin{macrocode}
\newcommand*{\DeclareBoolOption}[2][false]{%
  \KVO@ifdefinable{if\KVO@prefix#2}{%
    \KVO@ifdefinable{\KVO@prefix#2true}{%
      \KVO@ifdefinable{\KVO@prefix#2false}{%
        \csname newif\expandafter\endcsname
        \csname if\KVO@prefix#2\endcsname
        \@ifundefined{\KVO@prefix#2#1}{%
          \PackageWarning{kvoptions}{%
            Initialization of option `#2' failed,\MessageBreak
            cannot set boolean option to `#1',\MessageBreak
            use `true' or `false', now using `false'%
          }%
        }{%
          \csname\KVO@prefix#2#1\endcsname
        }%
        \begingroup
          \edef\x{\endgroup
            \noexpand\define@key{\KVO@family}{#2}[true]{%
              \noexpand\KVO@boolkey{\@currname}%
              \ifx\@currext\@clsextension
                \noexpand\@clsextension
              \else
                \noexpand\@pkgextension
              \fi
              {\KVO@prefix}{#2}{####1}%
            }%
          }%
        \x
      }%
    }%
  }%
}
%    \end{macrocode}
%    \end{macro}
%
%    \begin{macro}{\DeclareComplementaryOption}
%    The first argument is the key name, the second the
%    key that must be a boolean option with the same
%    current family and prefix. A new switch is not
%    created for the new key, we have already a switch.
%    Instead we define switch setting commands to work
%    on the parent switch.
%    \begin{macrocode}
\newcommand*{\DeclareComplementaryOption}[2]{%
  \@ifundefined{if\KVO@prefix#2}{%
    \PackageError{kvoptions}{%
      Cannot generate option code for `#1',\MessageBreak
      parent switch `#2' does not exist%
    }{%
      You are inside %
      \ifx\@currext\@clsextension class\else package\fi\space
      `\@currname.\@currext'.\MessageBreak
      `\KVO@family' is used as familiy %
      for the keyval options.\MessageBreak
      `\KVO@prefix' serves as prefix %
      for internal switch macros.\MessageBreak
      \MessageBreak
      \@ehc
    }%
  }{%
    \KVO@ifdefinable{\KVO@prefix#1true}{%
      \KVO@ifdefinable{\KVO@prefix#1false}{%
        \expandafter\let\csname\KVO@prefix#1false\expandafter\endcsname
          \csname\KVO@prefix#2true\endcsname
        \expandafter\let\csname\KVO@prefix#1true\expandafter\endcsname
          \csname\KVO@prefix#2false\endcsname
%    \end{macrocode}
%    The same code part as in \cs{DeclareBoolOption} can
%    now be used.
%    \begin{macrocode}
        \begingroup
          \edef\x{\endgroup
            \noexpand\define@key{\KVO@family}{#1}[true]{%
              \noexpand\KVO@boolkey{\@currname}%
              \ifx\@currext\@clsextension
                \noexpand\@clsextension
              \else
                \noexpand\@pkgextension
              \fi
              {\KVO@prefix}{#1}{####1}%
            }%
          }%
        \x
      }%
    }%
  }%
}
%    \end{macrocode}
%    \end{macro}
%
%    \begin{macro}{\KVO@ifdefinable}
%    Generate the command token LaTeX's \cs{@ifdefinable} expects.
%    \begin{macrocode}
\def\KVO@ifdefinable#1{%
  \expandafter\@ifdefinable\csname #1\endcsname
}
%    \end{macrocode}
%    \end{macro}
%
%    \begin{macro}{\KVO@boolkey}
%    We check explicitly for |true| and |false| to prevent
%    the user from accidently calling other macros.
%    \begin{quote}
%    \begin{tabular}{@{}ll@{}}
%    |#1|& package/class name\\
%    |#2|& |\@pkgextension|/|\@clsextension|\\
%    |#3|& prefix \\
%    |#4|& key name\\
%    |#5|& new value
%    \end{tabular}
%    \end{quote}
%    \begin{macrocode}
\def\KVO@boolkey#1#2#3#4#5{%
  \edef\KVO@param{#5}%
  \ltx@onelevel@sanitize\KVO@param
  \ifx\KVO@param\KVO@true
    \expandafter\@firstofone
  \else
    \ifx\KVO@param\KVO@false
      \expandafter\expandafter\expandafter\@firstofone
    \else
      \ifx#2\@clsextension
        \expandafter\ClassWarning
      \else
        \expandafter\PackageWarning
      \fi
      {#1}{%
        Value `\KVO@param' is not supported by\MessageBreak
        option `#4'%
      }%
      \expandafter\expandafter\expandafter\@gobble
    \fi
  \fi
  {%
    ^^A\ifx#2\@clsextension
    ^^A  \expandafter\ClassInfo
    ^^A\else
    ^^A  \expandafter\PackageInfo
    ^^A\fi
    ^^A{#1}{[option] #4=\KVO@param}%
    \csname#3#4\KVO@param\endcsname
  }%
}
%    \end{macrocode}
%    \end{macro}
%    \begin{macro}{\KVO@true}
%    \begin{macro}{\KVO@false}
%    The macros \cs{KVO@true} and \cs{KVO@false} are used
%    for string comparisons. After \cs{ltx@onelevel@sanitize}
%    we have only tokens with catcode 12 (other).
%    \begin{macrocode}
\def\KVO@true{true}
\def\KVO@false{false}
\ltx@onelevel@sanitize\KVO@true
\ltx@onelevel@sanitize\KVO@false
%    \end{macrocode}
%    \end{macro}
%    \end{macro}
%
% \subsubsection{\cs{DeclareStringOption}}
%
%    \begin{macro}{\DeclareStringOption}
%    \begin{macrocode}
\newcommand*{\DeclareStringOption}[2][]{%
  \@ifnextchar[{%
    \KVO@DeclareStringOption{#1}{#2}@%
  }{%
    \KVO@DeclareStringOption{#1}{#2}{}[]%
  }%
}
%    \end{macrocode}
%    \end{macro}
%    \begin{macro}{\KVO@DeclareStringOption}
%    \begin{macrocode}
\def\KVO@DeclareStringOption#1#2#3[#4]{%
  \KVO@ifdefinable{\KVO@prefix#2}{%
    \@namedef{\KVO@prefix#2}{#1}%
    \begingroup
      \ifx\\#3\\%
        \toks@{}%
      \else
        \toks@{[{#4}]}%
      \fi
      \edef\x{\endgroup
        \noexpand\define@key{\KVO@family}{#2}\the\toks@{%
          ^^A\begingroup
          ^^A  \toks@{####1}%
          ^^A  \ifx\@currext\@clsextension
          ^^A    \noexpand\ClassInfo
          ^^A  \else
          ^^A    \noexpand\PackageInfo
          ^^A  \fi
          ^^A  {\@currname}{%
          ^^A    [option] #2={\noexpand\the\toks@}%
          ^^A  }%
          ^^A\endgroup
          \noexpand\def
          \expandafter\noexpand\csname\KVO@prefix#2\endcsname{####1}%
        }%
      }%
    \x
  }%
}
%    \end{macrocode}
%    \end{macro}
%
% \subsubsection{\cs{DeclareVoidOption}}
%
%    \begin{macro}{\DeclareVoidOption}
%    \begin{macrocode}
\newcommand*{\DeclareVoidOption}[2]{%
  \begingroup
    \let\next\@gobbletwo
    \KVO@ifdefinable{\KVO@prefix#1}{%
      \let\next\@firstofone
    }%
  \expandafter\endgroup
  \next{%
    \begingroup
      \edef\x{\endgroup
        \noexpand\define@key{\KVO@family}{#1}[\KVO@VOID@]{%
          \noexpand\KVO@voidkey{\@currname}%
          \ifx\@currext\@clsextension
            \noexpand\@clsextension
          \else
            \noexpand\@pkgextension
          \fi
          {#1}%
          {####1}%
          \expandafter\noexpand\csname\KVO@prefix#1\endcsname
        }%
      }%
    \x
    \begingroup
      \toks@{#2}%
    \expandafter\endgroup
    \expandafter\def
    \csname\KVO@prefix#1\expandafter\endcsname
    \expandafter{\the\toks@}%
  }%
}
\def\KVO@VOID@{@VOID@}
%    \end{macrocode}
%    \end{macro}
%    \begin{macro}{\KVO@voidkey}
%    \begin{tabular}{@{}ll@{}}
%    |#1|& package/class name\\
%    |#2|& |\@pkgextension|/|\@clsextension|\\
%    |#3|& key name\\
%    |#4|& default (|@VOID@|)\\
%    |#5|& macro with option code
%    \end{tabular}
%    \begin{macrocode}
\def\KVO@voidkey#1#2#3#4{%
  \def\CurrentOption{#3}%
  \begingroup
    \def\x{#4}%
  \expandafter\endgroup
  \ifx\x\KVO@VOID@
  \else
    \ifx#2\@clsextension
      \expandafter\ClassWarning
    \else
      \expandafter\PackageWarning
    \fi
    {#1}{%
      Unexpected value for option `#3'\MessageBreak
      is ignored%
    }%
  \fi
  ^^A\ifx#2\@clsextension
  ^^A  \expandafter\ClassInfo
  ^^A\else
  ^^A  \expandafter\PackageInfo
  ^^A\fi
  ^^A{#1}{[option] #3}%
}
%    \end{macrocode}
%    \end{macro}
%
% \subsubsection{\cs{DeclareDefaultOption}}
%
%    \begin{macro}{\DeclareDefaultOption}
%    \begin{macrocode}
\newcommand*{\DeclareDefaultOption}{%
  \@namedef{KVO@default@\@currname.\@currext}%
}
%    \end{macrocode}
%    \end{macro}
%
% \subsubsection{\cs{DeclareLocalOptions}}
%
%    \begin{macro}{\DeclareLocalOptions}
%    \begin{macrocode}
\newcommand*{\DeclareLocalOptions}[1]{%
  \comma@parse{#1}\KVO@DeclareLocalOption
}
%    \end{macrocode}
%    \end{macro}
%    \begin{macro}{\KVO@DeclareLocalOption}
%    \begin{macrocode}
\def\KVO@DeclareLocalOption#1{%
  \expandafter\def\csname KVO@local@\KVO@family @#1\endcsname{}%
}
%    \end{macrocode}
%    \end{macro}
%
% \subsection{Dynamic options}
%
% \subsubsection{\cs{DisableKeyvalOption}}
%
%    \begin{macrocode}
\SetupKeyvalOptions{%
  family=KVOdyn,%
  prefix=KVOdyn@%
}
\DeclareBoolOption[true]{global}
\DeclareComplementaryOption{local}{global}
\DeclareStringOption[undef]{action}
\let\KVOdyn@name\relax
\let\KVOdyn@ext\@empty
\define@key{KVOdyn}{class}{%
  \def\KVOdyn@name{#1}%
  \let\KVOdyn@ext\@clsextension
}
\define@key{KVOdyn}{package}{%
  \def\KVOdyn@name{#1}%
  \let\KVOdyn@ext\@pkgextension
}
\newcommand*{\DisableKeyvalOption}[3][]{%
  \begingroup
    \kvsetkeys{KVOdyn}{#1}%
    \def\x{\endgroup}%
    \@ifundefined{KVO@action@\KVOdyn@action}{%
      \PackageError{kvoptions}{%
        Unknown disable action %
        `\expandafter\strip@prefix\meaning\KVOdyn@action'\MessageBreak
        for option `#3' in keyval family '#2'%
      }\@ehc
    }{%
      \csname KVO@action@\KVOdyn@action\endcsname{#2}{#3}%
    }%
  \x
}
\def\KVO@action@undef#1#2{%
  \edef\x{\endgroup
    \ifKVOdyn@global\global\fi
    \let
    \expandafter\noexpand\csname KV@#1@#2\endcsname
    \relax
    \ifKVOdyn@global\global\fi
    \let
    \expandafter\noexpand\csname KV@#1@#2@default\endcsname
    \relax
  }%
  ^^A\PackageInfo{kvoptions}{%
  ^^A  [option] key `#2' of family `#1'\MessageBreak
  ^^A  is disabled (undef, \ifKVOdyn@global global\else local\fi)%
  ^^A}%
}
\def\KVO@action@ignore#1#2{%
  \edef\x{\endgroup
    \ifKVOdyn@global\global\fi
    \let
    \expandafter\noexpand\csname KV@#1@#2\endcsname
    \noexpand\@gobble
    \ifKVOdyn@global\global\fi
    \let
    \expandafter\noexpand\csname KV@#1@#2@default\endcsname
    \noexpand\@empty
  }%
  ^^A\PackageInfo{kvoptions}{%
  ^^A  [option] key `#2' of family `#1'\MessageBreak
  ^^A  is disabled (ignore, \ifKVOdyn@global global\else local\fi)%
  ^^A}%
}
\def\KVO@action@error{%
  \KVO@do@action{error}%
}
\def\KVO@action@warning{%
  \KVO@do@action{warning}%
}
%    \end{macrocode}
%    \begin{tabular}{@{}ll@{}}
%    |#1|& |error| or |warning|\\
%    |#2|& \meta{family}\\
%    |#3|& \meta{key}\\
%    \end{tabular}
%    \begin{macrocode}
\def\KVO@do@action#1#2#3{%
  \ifx\KVOdyn@name\relax
    \PackageError{kvoptions}{%
      Action type `#1' needs package/class name\MessageBreak
      for key `#3' in family `#2'%
    }\@ehc
  \else
    \edef\x{\endgroup
      \noexpand\define@key{#2}{#3}[]{%
        \expandafter\noexpand\csname KVO@disable@#1\endcsname
        {\KVOdyn@name}\noexpand\KVOdyn@ext{#3}%
      }%
      \ifKVOdyn@global
        \global\let
        \expandafter\noexpand\csname KV@#2@#3\endcsname
        \expandafter\noexpand\csname KV@#2@#3\endcsname
        \global\let
        \expandafter\noexpand\csname KV@#2@#3@default\endcsname
        \expandafter\noexpand\csname KV@#2@#3@default\endcsname
      \fi
    }%
    ^^A\ifx\KVOdyn@ext\@clsextension
    ^^A  \expandafter\ClassInfo
    ^^A\else
    ^^A   \expandafter\PackageInfo
    ^^A\fi
    ^^A{\KVOdyn@name}{%
    ^^A  [option] key `#3' of family `#2'\MessageBreak
    ^^A  is disabled (#1, \ifKVOdyn@global global\else local\fi)%
    ^^A}%
  \fi
}
\def\KVO@disable@error#1#2#3{%
  \ifx#2\@clsextension
    \expandafter\ClassError
  \else
    \expandafter\PackageError
  \fi
  {#1}{%
    Option `#3' is given too late,\MessageBreak
    now the option is ignored%
  }\@ehc
}
\def\KVO@disable@warning#1#2#3{%
  \ifx#2\@clsextension
    \expandafter\ClassWarning
  \else
    \expandafter\PackageWarning
  \fi
  {#1}{%
    Option `#3' is already consumed\MessageBreak
    and has no effect%
  }%
}
%    \end{macrocode}
%
% \subsection{Change option code}
%
% \subsubsection{\cs{AddToKeyvalOption}}
%
%    \begin{macro}{\AddToKeyvalOption}
%    \begin{macrocode}
\newcommand*{\AddToKeyvalOption}{%
  \@ifstar{%
    \begingroup
      \edef\x{\endgroup
        \noexpand\KVO@AddToKeyvalOption{\KVO@family}%
      }%
    \x
  }%
  \KVO@AddToKeyvalOption
}
%    \end{macrocode}
%    \end{macro}
%    \begin{macro}{\KVO@AddToKeyvalOption}
%    \begin{macrocode}
\def\KVO@AddToKeyvalOption#1#2{%
  \@ifundefined{KV@#1@#2}{%
    \PackageWarning{kvoptions}{%
      Key `#2' of family `#1' does not exist.\MessageBreak
      Ignoring \string\AddToKeyvalOption
    }%
    \@gobble
  }{%
    \edef\KVO@next{%
      \noexpand\KVO@@AddToKeyvalOption
      \expandafter\noexpand\csname KV@#1@#2\endcsname
    }%
    \afterassignment\KVO@next
    \def\KVO@temp##1%
  }%
}
%    \end{macrocode}
%    \end{macro}
%    \begin{macro}{\KVO@@AddToKeyvalOption}
%    \begin{macrocode}
\def\KVO@@AddToKeyvalOption#1{%
  \begingroup
    \toks@\expandafter{#1{##1}}%
    \toks@\expandafter{\the\expandafter\toks@\KVO@temp{##1}}%
    \edef\x{\endgroup
      \noexpand\def\noexpand#1####1{\the\toks@}%
    }%
  \x
}
%    \end{macrocode}
%    \end{macro}
%
% \subsection{Process options}
%
% \subsubsection{Get global options}
%
%    Package \xpackage{xkeyval} removes options with equal signs from
%    the global options (\cs{@classoptionslist}). The effect is that
%    other packages and classes will not see these global options anymore.
%    A bug-report was answered that this behaviour is ``by design''.
%    Thus I call it a design bug. Now getting the global options require
%    an algorithm instead of a simple macro call.
%    \begin{macrocode}
%</package>
%<*package|patch>
%    \end{macrocode}
%    \begin{macro}{\KVO@IfDefThen}
%    Call |#2| if command |#1| is defined and not \cs{relax}.
%    (Package \xpackage{kvoptions-patch} does not load
%    package \xpackage{ltxcmds}.)
%    \begin{macrocode}
\def\KVO@IfDefThen#1#2{%
  \ifx#1\ltx@undefined
  \else
    \ifx#1\relax
    \else
      #2%
    \fi
  \fi
}%
%    \end{macrocode}
%    \end{macro}
%    \begin{macro}{\KVO@GetClassOptionsList}
%    \begin{macrocode}
\def\KVO@GetClassOptionsList{%
  \let\KVO@classoptionslist\@classoptionslist
  \KVO@IfDefThen\@classoptionslist{%
    \KVO@IfDefThen\XKV@documentclass{%
      \ifx\XKV@documentclass\ltx@empty
      \else
        \KVO@IfDefThen\XKV@classoptionslist{%
          \ifx\XKV@classoptionslist\ltx@empty
          \else
            \let\KVO@classoptionslist\XKV@classoptionslist
          \fi
        }%
      \fi
    }%
  }%
}%
%    \end{macrocode}
%    \end{macro}
%    \begin{macrocode}
%</package|patch>
%<*package>
%    \end{macrocode}
%
% \subsubsection{\cs{ProcessKeyvalOptions}}
%
%    \begin{macro}{\ProcessKeyvalOptions}
%    If the optional star is given, we get the family
%    name and expand it for safety.
%    \begin{macrocode}
\newcommand*{\ProcessKeyvalOptions}{%
  \@ifstar{%
    \begingroup
      \edef\x{\endgroup
        \noexpand\KVO@ProcessKeyvalOptions{\KVO@family}%
      }%
    \x
  }%
  \KVO@ProcessKeyvalOptions
}
%    \end{macrocode}
%    \end{macro}
%
%    \begin{macrocode}
\def\KVO@ProcessKeyvalOptions#1{%
  \let\@tempc\relax
  \let\KVO@temp\@empty
%    \end{macrocode}
%    Add any global options that are known to KV to the start of
%    the list being built in |\KVO@temp| and mark them used (by
%    removing them from the unused option list).
%    \begin{macrocode}
  \ifx\@currext\@clsextension
  \else
    \KVO@GetClassOptionsList
    \ifx\KVO@classoptionslist\relax
    \else
      \@for\KVO@CurrentOption:=\KVO@classoptionslist\do{%
        \@ifundefined{KV@#1@\expandafter\KVO@getkey
                      \KVO@CurrentOption=\@nil}{%
        }{%
          \@ifundefined{KVO@local@#1@\expandafter\KVO@getkey
                        \KVO@CurrentOption=\@nil}{%
            \ifx\KVO@Patch Y%
              \edef\KVO@temp{%
                \etex@unexpanded\expandafter{%
                  \KVO@temp
                }%
                ,%
                \etex@unexpanded\expandafter{%
                  \KVO@CurrentOption
                }%
                ,%
              }%
              \ltx@onelevel@sanitize\KVO@CurrentOption
            \else
              \edef\KVO@temp{%
                \KVO@temp
                ,%
                \KVO@CurrentOption
                ,%
              }%
            \fi
            \@expandtwoargs\@removeelement\KVO@CurrentOption
              \@unusedoptionlist\@unusedoptionlist
          }{}%
        }%
      }%
    \fi
  \fi
%    \end{macrocode}
%     Now stick the package options at the end of the list and wrap
%     in a call to \cs{setkeys}. A class ignores unknown global
%     options, we must remove them to prevent error messages
%     from \cs{setkeys}.
%    \begin{macrocode}
  \begingroup
    \toks\tw@{}%
    \@ifundefined{opt@\@currname.\@currext}{%
      \toks@\expandafter{\KVO@temp}%
    }{%
      \toks@\expandafter\expandafter\expandafter{%
        \csname opt@\@currname.\@currext\endcsname
      }%
      \ifx\@currext\@clsextension
        \edef\CurrentOption{\the\toks@}%
        \toks@\expandafter{\KVO@temp}%
        \@for\CurrentOption:=\CurrentOption\do{%
          \@ifundefined{%
            KV@#1@\expandafter\KVO@getkey\CurrentOption=\@nil
          }{%
%    \end{macrocode}
%    A class puts not used options in the unused option list
%    unless there is a default handler.
%    \begin{macrocode}
            \@ifundefined{KVO@default@\@currname.\@currext}{%
              \ifx\KVO@Patch Y%
                \ltx@onelevel@sanitize\CurrentOption
              \fi
              \ifx\@unusedoptionlist\@empty
                \global\let\@unusedoptionlist\CurrentOption
              \else
                \expandafter\expandafter\expandafter\gdef
                \expandafter\expandafter\expandafter\@unusedoptionlist
                \expandafter\expandafter\expandafter{%
                  \expandafter\@unusedoptionlist
                  \expandafter,\CurrentOption
                }%
              \fi
            }{%
              \toks\tw@\expandafter{%
                \the\toks\expandafter\tw@\expandafter,\CurrentOption
              }%
            }%
          }{%
            \toks@\expandafter{%
              \the\expandafter\toks@\expandafter,\CurrentOption
            }%
          }%
        }%
      \else
%    \end{macrocode}
%    Without default action we pass all options to \cs{setkeys}.
%    Otherwise we have to check which options are known.
%    These are passed to \cs{setkeys}. For the others the default
%    action is performed.
%    \begin{macrocode}
        \@ifundefined{KVO@default@\@currname.\@currext}{%
          \toks@\expandafter\expandafter\expandafter{%
            \expandafter\KVO@temp\the\toks@
          }%
        }{%
          \edef\CurrentOption{\the\toks@}%
          \toks@\expandafter{\KVO@temp}%
          \@for\CurrentOption:=\CurrentOption\do{%
            \@ifundefined{%
              KV@#1@\expandafter\KVO@getkey\CurrentOption=\@nil
            }{%
              \toks\tw@\expandafter{%
                \the\toks\expandafter\tw@\expandafter,\CurrentOption
              }%
            }{%
              \toks@\expandafter{%
                \the\expandafter\toks@\expandafter,\CurrentOption
              }%
            }%
          }%
        }%
      \fi
    }%
    \edef\KVO@temp{\endgroup
      \noexpand\KVO@calldefault{\the\toks\tw@}%
      \noexpand\KVO@setkeys{#1}{\the\toks@}%
    }%
  \KVO@temp
%    \end{macrocode}
%    Some cleanup of \cs{ProcessOptions}.
%    \begin{macrocode}
  \let\CurrentOption\@empty
  \AtEndOfPackage{\let\@unprocessedoptions\relax}%
}
%    \end{macrocode}
%
% \subsubsection{\cs{ProcessLocalKeyvalOptions}}
%
%    \begin{macro}{\ProcessLocalKeyvalOptions}
%    If the optional star is given, we get the family
%    name and expand it for safety.
%    \begin{macrocode}
\newcommand*{\ProcessLocalKeyvalOptions}{%
  \@ifstar{%
    \begingroup
      \edef\x{\endgroup
        \noexpand\KVO@ProcessLocalKeyvalOptions{\KVO@family}%
      }%
    \x
  }%
  \KVO@ProcessLocalKeyvalOptions
}
%    \end{macrocode}
%    \end{macro}
%
%    \begin{macrocode}
\def\KVO@ProcessLocalKeyvalOptions#1{%
  \let\@tempc\relax
  \let\KVO@temp\@empty
%    \end{macrocode}
%    Check if \cs{ProcessLocalKeyvalOptions} is called inside
%    a package.
%    \begin{macrocode}
  \ifx\@currext\@pkgextension
  \else
    \PackageError{kvoptions}{%
      \string\ProcessLocalKeyvalOptions\space is intended for packages only%
    }\@ehc
  \fi
%    \end{macrocode}
%     The package options are put into toks register \cs{toks@}.
%    \begin{macrocode}
  \begingroup
    \toks\tw@{}%
    \@ifundefined{opt@\@currname.\@currext}{%
      \toks@\expandafter{\KVO@temp}%
    }{%
      \toks@\expandafter\expandafter\expandafter{%
        \csname opt@\@currname.\@currext\endcsname
      }%
%    \end{macrocode}
%    Without default action we pass all options to \cs{setkeys}.
%    Otherwise we have to check which options are known.
%    These are passed to \cs{setkeys}. For the others the default
%    action is performed.
%    \begin{macrocode}
      \@ifundefined{KVO@default@\@currname.\@currext}{%
        \toks@\expandafter\expandafter\expandafter{%
          \expandafter\KVO@temp\the\toks@
        }%
      }{%
        \edef\CurrentOption{\the\toks@}%
        \toks@\expandafter{\KVO@temp}%
        \@for\CurrentOption:=\CurrentOption\do{%
          \@ifundefined{%
            KV@#1@\expandafter\KVO@getkey\CurrentOption=\@nil
          }{%
            \toks\tw@\expandafter{%
              \the\toks\expandafter\tw@\expandafter,\CurrentOption
            }%
          }{%
            \toks@\expandafter{%
              \the\expandafter\toks@\expandafter,\CurrentOption
            }%
          }%
        }%
      }%
    }%
    \edef\KVO@temp{\endgroup
      \noexpand\KVO@calldefault{\the\toks\tw@}%
      \noexpand\KVO@setkeys{#1}{\the\toks@}%
    }%
  \KVO@temp
%    \end{macrocode}
%    Some cleanup of \cs{ProcessOptions}.
%    \begin{macrocode}
  \let\CurrentOption\@empty
  \AtEndOfPackage{\let\@unprocessedoptions\relax}%
}
%    \end{macrocode}
%
% \subsubsection{Helper macros}
%
%    \begin{macro}{\KVO@getkey}
%    Extract the key part of a key=value pair.
%    \begin{macrocode}
\def\KVO@getkey#1=#2\@nil{#1}
%    \end{macrocode}
%    \end{macro}
%
%    \begin{macro}{\KVO@calldefault}
%    \begin{macrocode}
\def\KVO@calldefault#1{%
  \begingroup
    \def\x{#1}%
  \expandafter\endgroup
  \ifx\x\@empty
  \else
    \@for\CurrentOption:=#1\do{%
      \ifx\CurrentOption\@empty
      \else
        \expandafter\KVO@setcurrents\CurrentOption=\@nil
        \@nameuse{KVO@default@\@currname.\@currext}%
      \fi
    }%
  \fi
}
%    \end{macrocode}
%    \end{macro}
%    \begin{macro}{\KVO@setcurrents}
%    Extract the key part of a key=value pair.
%    \begin{macrocode}
\def\KVO@setcurrents#1=#2\@nil{%
  \def\CurrentOptionValue{#2}%
  \ifx\CurrentOptionValue\@empty
    \let\CurrentOptionKey\CurrentOption
    \let\CurrentOptionValue\relax
  \else
    \edef\CurrentOptionKey{\zap@space#1 \@empty}%
    \expandafter\KVO@setcurrentvalue\CurrentOption\@nil
  \fi
}
%    \end{macrocode}
%    \end{macro}
%    \begin{macro}{\KV@setcurrentvalue}
%    Here the value part is parsed. Package \xpackage{keyval}'s
%    \cs{KV@@sp@def} helps in removing spaces at the begin and
%    end of the value.
%    \begin{macrocode}
\def\KVO@setcurrentvalue#1=#2\@nil{%
  \KV@@sp@def\CurrentOptionValue{#2}%
}
%    \end{macrocode}
%    \end{macro}
%
% \subsection{\plainTeX}
%
%    Disable \LaTeX\ stuff.
%    \begin{macrocode}
\begingroup\expandafter\expandafter\expandafter\endgroup
\expandafter\ifx\csname documentclass\endcsname\relax
  \def\ProcessKeyvalOptions{%
    \@ifstar{}\@gobble
  }%
\fi
%    \end{macrocode}
%
%    \begin{macrocode}
\KVO@AtEnd%
%</package>
%    \end{macrocode}
%
% \subsection{Package \xpackage{kvoptions-patch}}
%
%    \begin{macrocode}
%<*patch>
\NeedsTeXFormat{LaTeX2e}
\begingroup\catcode61\catcode48\catcode32=10\relax%
  \catcode13=5 % ^^M
  \endlinechar=13 %
  \catcode123=1 % {
  \catcode125=2 % }
  \catcode64=11 % @
  \def\x{\endgroup
    \expandafter\edef\csname KVO@AtEnd\endcsname{%
      \endlinechar=\the\endlinechar\relax
      \catcode13=\the\catcode13\relax
      \catcode32=\the\catcode32\relax
      \catcode35=\the\catcode35\relax
      \catcode61=\the\catcode61\relax
      \catcode64=\the\catcode64\relax
      \catcode123=\the\catcode123\relax
      \catcode125=\the\catcode125\relax
    }%
  }%
\x\catcode61\catcode48\catcode32=10\relax%
\catcode13=5 % ^^M
\endlinechar=13 %
\catcode35=6 % #
\catcode64=11 % @
\catcode123=1 % {
\catcode125=2 % }
\def\TMP@EnsureCode#1#2{%
  \edef\KVO@AtEnd{%
    \KVO@AtEnd
    \catcode#1=\the\catcode#1\relax
  }%
  \catcode#1=#2\relax
}
\TMP@EnsureCode{39}{12}% '
\TMP@EnsureCode{40}{12}% (
\TMP@EnsureCode{41}{12}% )
\TMP@EnsureCode{43}{12}% +
\TMP@EnsureCode{44}{12}% ,
\TMP@EnsureCode{45}{12}% -
\TMP@EnsureCode{46}{12}% .
\TMP@EnsureCode{47}{12}% /
\TMP@EnsureCode{58}{12}% :
\TMP@EnsureCode{60}{12}% <
\TMP@EnsureCode{62}{12}% >
\TMP@EnsureCode{91}{12}% [
\TMP@EnsureCode{93}{12}% ]
\TMP@EnsureCode{96}{12}% `
\TMP@EnsureCode{124}{12}% |
\edef\KVO@AtEnd{\KVO@AtEnd\noexpand\endinput}
\ProvidesPackage{kvoptions-patch}%
  [2016/05/16 v3.12 LaTeX patch for keyval options (HO)]%
%    \end{macrocode}
%
%    Check for \eTeX.
%    \begin{macrocode}
\begingroup\expandafter\expandafter\expandafter\endgroup
\expandafter\ifx\csname eTeXversion\endcsname\relax
  \PackageWarningNoLine{kvoptions-patch}{%
    Package loading is aborted, because e-TeX is missing%
  }%
  \expandafter\KVO@AtEnd
\fi%
%    \end{macrocode}
%
%    Package \xpackage{etexcmds} for \cs{etex@unexpanded}.
%    \begin{macrocode}
\RequirePackage{etexcmds}[2007/09/09]
\ifetex@unexpanded
\else
  \PackageError{kvoptions-patch}{%
    Could not find eTeX's \string\unexpanded.\MessageBreak
    Try adding \string\RequirePackage\string{etexcmds\string} %
    before \string\documentclass%
  }\@ehd
  \expandafter\KVO@AtEnd
\fi%
%    \end{macrocode}
%
%    Check for package \xpackage{xkvltxp}.
%    \begin{macrocode}
\@ifpackageloaded{xkvltxp}{%
  \PackageWarningNoLine{kvoptions}{%
    Option `patch' cannot be used together with\MessageBreak
    package `xkvltxp' that is already loaded.\MessageBreak
    Therefore package loading is aborted%
  }%
  \KVO@AtEnd
}{}%
%    \end{macrocode}
%
%    \begin{macrocode}
\def\@if@ptions#1#2#3{%
  \begingroup
    \KVO@normalize\KVO@temp{#3}%
    \edef\x{\endgroup
      \noexpand\@if@pti@ns{%
        \detokenize\expandafter\expandafter\expandafter{%
          \csname opt@#2.#1\endcsname
        }%
      }{%
        \detokenize\expandafter{\KVO@temp}%
      }%
    }%
  \x
}
%    \end{macrocode}
%
%    \begin{macrocode}
\def\@pass@ptions#1#2#3{%
  \KVO@normalize\KVO@temp{#2}%
  \@ifundefined{opt@#3.#1}{%
    \expandafter\gdef\csname opt@#3.#1%
          \expandafter\endcsname\expandafter{%
      \KVO@temp
    }%
  }{%
    \expandafter\gdef\csname opt@#3.#1%
          \expandafter\expandafter\expandafter\endcsname
          \expandafter\expandafter\expandafter{%
      \csname opt@#3.#1\expandafter\endcsname\expandafter,\KVO@temp
    }%
  }%
}
%    \end{macrocode}
%
%    \begin{macrocode}
\def\ProcessOptions{%
  \let\ds@\@empty
  \@ifundefined{opt@\@currname.\@currext}{%
    \let\@curroptions\@empty
  }{%
    \expandafter\expandafter\expandafter\def
    \expandafter\expandafter\expandafter\@curroptions
    \expandafter\expandafter\expandafter{%
      \csname opt@\@currname.\@currext\endcsname
    }%
  }%
  \@ifstar\KVO@xprocess@ptions\KVO@process@ptions
}
%    \end{macrocode}
%
%    \begin{macrocode}
\def\KVO@process@ptions{%
  \@for\CurrentOption:=\@declaredoptions\do{%
    \ifx\CurrentOption\@empty
    \else
      \begingroup
        \ifx\@currext\@clsextension
          \toks@{}%
        \else
          \KVO@GetClassOptionsList
          \toks@\expandafter{\KVO@classoptionslist,}%
        \fi
        \toks\tw@\expandafter{\@curroptions}%
        \edef\x{\endgroup
          \noexpand\in@{,\CurrentOption,}{,\the\toks@\the\toks\tw@,}%
        }%
      \x
      \ifin@
        \KVO@use@ption
        \expandafter\let\csname ds@\CurrentOption\endcsname\@empty
      \fi
    \fi
  }%
  \KVO@process@pti@ns
}
%    \end{macrocode}
%
%    \begin{macrocode}
\def\KVO@xprocess@ptions{%
  \ifx\@currext\@clsextension
  \else
    \KVO@GetClassOptionsList
    \@for\CurrentOption:=\KVO@classoptionslist\do{%
      \ifx\CurrentOption\@empty
      \else
        \KVO@in@\CurrentOption\@declaredoptions
        \ifin@
          \KVO@use@ption
          \expandafter\let\csname ds@\CurrentOption\endcsname\@empty
        \fi
      \fi
    }%
  \fi
  \KVO@process@pti@ns
}
%    \end{macrocode}
%
%    \begin{macrocode}
\def\KVO@in@#1#2{%
  \in@false
  \begingroup
    \@for\x:=#2\do{%
      \ifx\x#1\relax
        \in@true
      \fi
    }%
    \edef\x{\endgroup
      \ifin@
        \noexpand\in@true
      \fi
    }%
  \x
}
%    \end{macrocode}
%
%    \begin{macrocode}
\def\KVO@process@pti@ns{%
  \@for\CurrentOption:=\@curroptions\do{%
    \@ifundefined{ds@\KVO@SanitizedCurrentOption}{%
      \KVO@use@ption
      \default@ds
    }%
    \KVO@use@ption
  }%
  \@for\CurrentOption:=\@declaredoptions\do{%
    \expandafter\let\csname ds@\CurrentOption\endcsname\relax
  }%
  \let\CurrentOption\@empty
  \let\@fileswith@pti@ns\@@fileswith@pti@ns
  \AtEndOfPackage{\let\@unprocessedoptions\relax}%
}
%    \end{macrocode}
%
%    \begin{macrocode}
\def\KVO@use@ption{%
  \begingroup
    \edef\x{\endgroup
      \noexpand\@removeelement{%
        \detokenize\expandafter{\CurrentOption}%
      }{%
        \detokenize\expandafter{\@unusedoptionlist}%
      }%
    }%
  \x\@unusedoptionlist
  \csname ds@\KVO@SanitizedCurrentOption\endcsname
}
%    \end{macrocode}
%
%    \begin{macrocode}
\def\OptionNotUsed{%
  \ifx\@currext\@clsextension
    \xdef\@unusedoptionlist{%
      \ifx\@unusedoptionlist\@empty
      \else
        \detokenize\expandafter{\@unusedoptionlist,}%
      \fi
      \detokenize\expandafter{\CurrentOption}%
    }%
  \fi
}
%    \end{macrocode}
%
%    Variant of \cs{ExecuteOptions} that better protects \cs{CurrentOption}.
%    \begin{macrocode}
\def\CurrentOption@SaveLevel{0}
\def\ExecuteOptions{%
  \expandafter\KVO@ExecuteOptions
      \csname CurrentOption@\CurrentOption@SaveLevel\endcsname
}
\def\KVO@ExecuteOptions#1#2{%
  \let#1\CurrentOption
  \edef\CurrentOption@SaveLevel{%
    \the\numexpr\CurrentOption@SaveLevel+1%
  }%
  \@for\CurrentOption:=#2\do{%
    \csname ds@\CurrentOption\endcsname
  }%
  \edef\CurrentOption@SaveLevel{%
    \the\numexpr\CurrentOption@SaveLevel-1%
  }%
  \let\CurrentOption#1%
}
%    \end{macrocode}
%
%    \begin{macrocode}
\def\KVO@fileswith@pti@ns#1[#2]#3[#4]{%
  \ifx#1\@clsextension
    \ifx\@classoptionslist\relax
      \KVO@normalize\KVO@temp{#2}%
      \expandafter\gdef\expandafter\@classoptionslist\expandafter{%
        \KVO@temp
      }%
      \def\reserved@a{%
        \KVO@onefilewithoptions{#3}[{#2}][{#4}]#1%
        \@documentclasshook
      }%
    \else
      \def\reserved@a{%
        \KVO@onefilewithoptions{#3}[{#2}][{#4}]#1%
      }%
    \fi
  \else
    \begingroup
      \let\KVO@temp\relax
      \let\KVO@onefilewithoptions\relax
      \let\@pkgextension\relax
      \def\reserved@b##1,{%
        \ifx\@nil##1\relax
        \else
          \ifx\relax##1\relax
          \else
            \KVO@onefilewithoptions{##1}[{\KVO@temp}][{#4}]%
            \@pkgextension
          \fi
          \expandafter\reserved@b
        \fi
      }%
      \edef\reserved@a{\zap@space#3 \@empty}%
      \edef\reserved@a{\expandafter\reserved@b\reserved@a,\@nil,}%
      \toks@{#2}%
      \def\KVO@temp{\the\toks@}%
    \edef\reserved@a{\endgroup \reserved@a}%
  \fi
  \reserved@a
}
%    \end{macrocode}
%
%    \begin{macrocode}
\def\KVO@onefilewithoptions#1[#2][#3]#4{%
  \@pushfilename
  \xdef\@currname{#1}%
  \global\let\@currext#4%
  \expandafter\let\csname\@currname.\@currext-h@@k\endcsname\@empty
  \let\CurrentOption\@empty
  \@reset@ptions
  \makeatletter
  \def\reserved@a{%
    \@ifl@aded\@currext{#1}{%
      \@if@ptions\@currext{#1}{#2}{%
      }{%
        \begingroup
          \@ifundefined{opt@#1.\@currext}{%
            \def\x{}%
          }{%
            \edef\x{%
              \expandafter\expandafter\expandafter\strip@prefix
              \expandafter\meaning\csname opt@#1.\@currext\endcsname
            }%
          }%
          \def\y{#2}%
          \edef\y{\expandafter\strip@prefix\meaning\y}%
          \@latex@error{Option clash for \@cls@pkg\space #1}{%
            The package #1 has already been loaded %
            with options:\MessageBreak
            \space\space[\x]\MessageBreak
            There has now been an attempt to load it %
             with options\MessageBreak
            \space\space[\y]\MessageBreak
            Adding the global options:\MessageBreak
            \space\space
                 \x,\y\MessageBreak
            to your \noexpand\documentclass declaration may fix this.%
            \MessageBreak
            Try typing \space <return> \space to proceed.%
          }%
        \endgroup
      }%
    }{%
      \@pass@ptions\@currext{#2}{#1}%
      \global\expandafter
      \let\csname ver@\@currname.\@currext\endcsname\@empty
      \InputIfFileExists
        {\@currname.\@currext}%
        {}%
        {\@missingfileerror\@currname\@currext}%
      \let\@unprocessedoptions\@@unprocessedoptions
      \csname\@currname.\@currext-h@@k\endcsname
      \expandafter\let\csname\@currname.\@currext-h@@k\endcsname
              \@undefined
      \@unprocessedoptions
    }%
    \@ifl@ter\@currext{#1}{#3}{%
    }{%
      \@latex@warning@no@line{%
        You have requested,\on@line, %
        version\MessageBreak
          #3' of \@cls@pkg\space #1,\MessageBreak
        but only version\MessageBreak
         `\csname ver@#1.\@currext\endcsname'\MessageBreak
        is available%
      }%
    }%
    \ifx\@currext\@clsextension\let\LoadClass\@twoloadclasserror\fi
    \@popfilename
    \@reset@ptions
  }%
  \reserved@a
}
%    \end{macrocode}
%
%    \begin{macrocode}
\def\@unknownoptionerror{%
  \@latex@error{%
    Unknown option `\KVO@SanitizedCurrentOption' %
    for \@cls@pkg\space`\@currname'%
  }{%
    The option `\KVO@SanitizedCurrentOption' was not declared in %
    \@cls@pkg\space`\@currname', perhaps you\MessageBreak
    misspelled its name. %
    Try typing \space <return> %
    \space to proceed.%
  }%
}
%    \end{macrocode}
%
%    \begin{macrocode}
\def\@@unprocessedoptions{%
  \ifx\@currext\@pkgextension
    \@ifundefined{opt@\@currname.\@currext}{%
      \let\@curroptions\@empty
    }{%
      \expandafter\let\expandafter\@curroptions
          \csname opt@\@currname.\@currext\endcsname
    }%
    \@for\CurrentOption:=\@curroptions\do{%
        \ifx\CurrentOption\@empty\else\@unknownoptionerror\fi
    }%
  \fi
}
%    \end{macrocode}
%
%    \begin{macrocode}
\def\KVO@SanitizedCurrentOption{%
  \expandafter\strip@prefix\meaning\CurrentOption
}
%    \end{macrocode}
%
%    Normalize option list.
%    \begin{macrocode}
\def\KVO@normalize#1#2{%
  \let\KVO@result\@empty
  \KVO@splitcomma#2,\@nil
  \let#1\KVO@result
}
\def\KVO@splitcomma#1,#2\@nil{%
  \KVO@ifempty{#1}{}{%
    \KVO@checkkv#1=\@nil
  }%
  \KVO@ifempty{#2}{}{\KVO@splitcomma#2\@nil}%
}
\def\KVO@ifempty#1{%
  \expandafter\ifx\expandafter\\\detokenize{#1}\\%
    \expandafter\@firstoftwo
  \else
    \expandafter\@secondoftwo
  \fi
}
\def\KVO@checkkv#1=#2\@nil{%
  \KVO@ifempty{#2}{%
    % option without value
    \edef\KVO@x{\zap@space#1 \@empty}%
    \ifx\KVO@x\@empty
      % ignore empty option
    \else
      % append to list
      \edef\KVO@result{%
        \etex@unexpanded\expandafter{\KVO@result},\KVO@x
      }%
    \fi
  }{%
    % #1: "key", #2: "value="
    % add key part
    \edef\KVO@result{%
      \etex@unexpanded\expandafter{\KVO@result},%
      \zap@space#1 \@empty
    }%
    \futurelet\@let@token\KVO@checkfirsttok#2 \@nil| = \@nil|\KVO@nil
  }%
}
\def\KVO@checkfirsttok{%
  \ifx\@let@token\bgroup
    % no space at start
    \expandafter\KVO@removelastspace\expandafter=%
    % "<value><spaceopt>= \@nil"
  \else
    \expandafter\KVO@checkfirstA
  \fi
}
\def\KVO@checkfirstA#1 #2\@nil{%
  \KVO@ifempty{#2}{%
    \KVO@removelastspace=#1 \@nil
  }{%
    \KVO@ifempty{#1}{%
      \KVO@removelastspace=#2\@nil
    }{%
      \KVO@removelastspace=#1 #2\@nil
    }%
  }%
}
\def\KVO@removelastspace#1 = \@nil|#2\KVO@nil{%
  \KVO@ifempty{#2}{%
    \edef\KVO@result{%
      \etex@unexpanded\expandafter{\KVO@result}%
      \etex@unexpanded\expandafter{\KVO@removegarbage#1\KVO@nil}%
    }%
  }{%
    \edef\KVO@result{%
      \etex@unexpanded\expandafter{\KVO@result}%
      \etex@unexpanded{#1}%
    }%
  }%
}
\def\KVO@removegarbage#1= \@nil#2\KVO@nil{#1}%
%    \end{macrocode}
%
%    Arguments |#1| and |#2| are macros.
%    \begin{macrocode}
\def\KVO@removeelement#1#2{%
  \begingroup
    \toks@={}%
    \@for\x:=#2\do{%
      \ifx\x\@empty
      \else
        \ifx\x#1\relax
        \else
          \edef\t{\the\toks@}%
          \ifx\t\@empty
          \else
            \toks@\expandafter{\the\toks@,}%
          \fi
          \toks@\expandafter{\the\expandafter\toks@\x}%
        \fi
      \fi
    }%
    \edef\x{\endgroup
      \def\noexpand#2{\the\toks@}%
    }%
  \x
}
%    \end{macrocode}
%
%
%    \begin{macrocode}
\let\@@fileswith@pti@ns\KVO@fileswith@pti@ns
\ifx\@fileswith@pti@ns\@badrequireerror
\else
  \let\@fileswith@pti@ns\KVO@fileswith@pti@ns
\fi
%    \end{macrocode}
%
%    \begin{macro}{\KVO@Patch}
%    \begin{macrocode}
\let\KVO@Patch=Y
%    \end{macrocode}
%    \end{macro}
%
%    \begin{macrocode}
\KVO@AtEnd%
%</patch>
%    \end{macrocode}
%
% \section{Test}
%
% \subsection{Preface for standard catcode check}
%
%    \begin{macrocode}
%<*test1>
\input miniltx.tex\relax
%</test1>
%    \end{macrocode}
%
% \subsection{Catcode checks for loading}
%
%    \begin{macrocode}
%<*test1>
%    \end{macrocode}
%    \begin{macrocode}
\catcode`\{=1 %
\catcode`\}=2 %
\catcode`\#=6 %
\catcode`\@=11 %
\expandafter\ifx\csname count@\endcsname\relax
  \countdef\count@=255 %
\fi
\expandafter\ifx\csname @gobble\endcsname\relax
  \long\def\@gobble#1{}%
\fi
\expandafter\ifx\csname @firstofone\endcsname\relax
  \long\def\@firstofone#1{#1}%
\fi
\expandafter\ifx\csname loop\endcsname\relax
  \expandafter\@firstofone
\else
  \expandafter\@gobble
\fi
{%
  \def\loop#1\repeat{%
    \def\body{#1}%
    \iterate
  }%
  \def\iterate{%
    \body
      \let\next\iterate
    \else
      \let\next\relax
    \fi
    \next
  }%
  \let\repeat=\fi
}%
\def\RestoreCatcodes{}
\count@=0 %
\loop
  \edef\RestoreCatcodes{%
    \RestoreCatcodes
    \catcode\the\count@=\the\catcode\count@\relax
  }%
\ifnum\count@<255 %
  \advance\count@ 1 %
\repeat

\def\RangeCatcodeInvalid#1#2{%
  \count@=#1\relax
  \loop
    \catcode\count@=15 %
  \ifnum\count@<#2\relax
    \advance\count@ 1 %
  \repeat
}
\def\RangeCatcodeCheck#1#2#3{%
  \count@=#1\relax
  \loop
    \ifnum#3=\catcode\count@
    \else
      \errmessage{%
        Character \the\count@\space
        with wrong catcode \the\catcode\count@\space
        instead of \number#3%
      }%
    \fi
  \ifnum\count@<#2\relax
    \advance\count@ 1 %
  \repeat
}
\def\space{ }
\expandafter\ifx\csname LoadCommand\endcsname\relax
  \def\LoadCommand{\input kvoptions.sty\relax}%
\fi
\def\Test{%
  \RangeCatcodeInvalid{0}{47}%
  \RangeCatcodeInvalid{58}{64}%
  \RangeCatcodeInvalid{91}{96}%
  \RangeCatcodeInvalid{123}{255}%
  \catcode`\@=12 %
  \catcode`\\=0 %
  \catcode`\%=14 %
  \LoadCommand
  \RangeCatcodeCheck{0}{36}{15}%
  \RangeCatcodeCheck{37}{37}{14}%
  \RangeCatcodeCheck{38}{47}{15}%
  \RangeCatcodeCheck{48}{57}{12}%
  \RangeCatcodeCheck{58}{63}{15}%
  \RangeCatcodeCheck{64}{64}{12}%
  \RangeCatcodeCheck{65}{90}{11}%
  \RangeCatcodeCheck{91}{91}{15}%
  \RangeCatcodeCheck{92}{92}{0}%
  \RangeCatcodeCheck{93}{96}{15}%
  \RangeCatcodeCheck{97}{122}{11}%
  \RangeCatcodeCheck{123}{255}{15}%
  \RestoreCatcodes
}
\Test
\csname @@end\endcsname
\end
%    \end{macrocode}
%    \begin{macrocode}
%</test1>
%    \end{macrocode}
%
%    \begin{macrocode}
%<*test2>
\NeedsTeXFormat{LaTeX2e}
\makeatletter
\catcode`\@=11 %
\def\RestoreCatcodes{}
\count@=0 %
\loop
  \edef\RestoreCatcodes{%
    \RestoreCatcodes
    \catcode\the\count@=\the\catcode\count@\relax
  }%
\ifnum\count@<255 %
  \advance\count@\@ne
\repeat

\def\RangeCatcodeInvalid#1#2{%
  \count@=#1\relax
  \loop
    \catcode\count@=15 %
  \ifnum\count@<#2\relax
    \advance\count@\@ne
  \repeat
}
\def\Test#1{%
  \RangeCatcodeInvalid{0}{47}%
  \RangeCatcodeInvalid{58}{64}%
  \RangeCatcodeInvalid{91}{96}%
  \RangeCatcodeInvalid{123}{255}%
  \catcode`\@=12 %
  \catcode`\\=0 %
  \catcode`\{=1 %
  \catcode`\}=2 %
  \catcode`\#=6 %
  \catcode`\[=12 %
  \catcode`\]=12 %
  \catcode`\%=14 %
  \catcode`\ =10 %
  \catcode13=5 %
  #1\relax
  \RestoreCatcodes
}
\Test{\RequirePackage{kvoptions-patch}}%
\Test{\RequirePackage{kvoptions}}%
\csname @@end\endcsname
%</test2>
%    \end{macrocode}
%
%    \begin{macrocode}
%<*test3>
\NeedsTeXFormat{LaTeX2e}
\makeatletter
\RequirePackage{kvoptions}[2016/05/16]
\def\msg#{\immediate\write16}
\define@key{testfamily}{testkey}{%
  \msg{[testfamily/testkey/#1]}%
}
\define@key{testfamily}{testdefaultkey}[testdefault]{%
  \msg{[testfamily/testdefaultkey/#1]}%
}
\AddToKeyvalOption{testfamily}{testkey}{%
  \msg{[addition/#1]}%
}
\AddToKeyvalOption{testfamily}{testdefaultkey}{%
  \msg{[addition/#1]}%
}
\setkeys{testfamily}{%
  testkey=testA,%
  testdefaultkey=testB,%
  testdefaultkey,%
}
\SetupKeyvalOptions{%
  family=testfamily%
}
\AddToKeyvalOption*{testkey}{%
  \msg{[star addition/#1]}%
}
\AddToKeyvalOption*{testdefaultkey}{%
  \msg{[star addition/#1]}%
}
\setkeys{testfamily}{%
  testkey=testA,%
  testdefaultkey=testB,%
  testdefaultkey,%
}
\@@end
%</test3>
%    \end{macrocode}
%
%    \begin{macrocode}
%<*test4pkg>
\NeedsTeXFormat{LaTeX2e}
\ProvidesPackage{kvoptions-test4}[2016/05/16 package for testing]
\RequirePackage{kvoptions}[2016/05/16]
\SetupKeyvalOptions{%
  family=FOO,%
  prefix=foo,%
  setkeys=\kvsetkeys,%
}
\DeclareStringOption{str}
\define@key{FOO}{set}{%
  \setkeys{BAR}{strbar={#1}}%
}
\define@key{BAR}{strbar}{%
  \def\foostr{[BAR:#1]}%
}
\ProcessKeyvalOptions*
%</test4pkg>
%    \end{macrocode}
%    \begin{macrocode}
%<*test4>
\NeedsTeXFormat{LaTeX2e}
\ProvidesFile{kvoptions-test4.tex}[2016/05/16 test file]
\RequirePackage[%
  str=A,set=B,str=C,%
]{kvoptions-test4}[2016/05/16]
\def\TestExpected{C}
\ifx\foostr\TestExpected
  \typeout{* Test ok.}%
\else
  \typeout{* Result: [\foostr]}%
  \typeout{* Expected: [\TestExpected]}%
  \errmessage{Test failed!}%
\fi
\csname @@end\endcsname\end
%</test4>
%    \end{macrocode}
%
% \section{Installation}
%
% \subsection{Download}
%
% \paragraph{Package.} This package is available on
% CTAN\footnote{\CTANpkg{kvoptions}}:
% \begin{description}
% \item[\CTAN{macros/latex/contrib/oberdiek/kvoptions.dtx}] The source file.
% \item[\CTAN{macros/latex/contrib/oberdiek/kvoptions.pdf}] Documentation.
% \end{description}
%
%
% \paragraph{Bundle.} All the packages of the bundle `oberdiek'
% are also available in a TDS compliant ZIP archive. There
% the packages are already unpacked and the documentation files
% are generated. The files and directories obey the TDS standard.
% \begin{description}
% \item[\CTANinstall{install/macros/latex/contrib/oberdiek.tds.zip}]
% \end{description}
% \emph{TDS} refers to the standard ``A Directory Structure
% for \TeX\ Files'' (\CTAN{tds/tds.pdf}). Directories
% with \xfile{texmf} in their name are usually organized this way.
%
% \subsection{Bundle installation}
%
% \paragraph{Unpacking.} Unpack the \xfile{oberdiek.tds.zip} in the
% TDS tree (also known as \xfile{texmf} tree) of your choice.
% Example (linux):
% \begin{quote}
%   |unzip oberdiek.tds.zip -d ~/texmf|
% \end{quote}
%
% \paragraph{Script installation.}
% Check the directory \xfile{TDS:scripts/oberdiek/} for
% scripts that need further installation steps.
%
% \subsection{Package installation}
%
% \paragraph{Unpacking.} The \xfile{.dtx} file is a self-extracting
% \docstrip\ archive. The files are extracted by running the
% \xfile{.dtx} through \plainTeX:
% \begin{quote}
%   \verb|tex kvoptions.dtx|
% \end{quote}
%
% \paragraph{TDS.} Now the different files must be moved into
% the different directories in your installation TDS tree
% (also known as \xfile{texmf} tree):
% \begin{quote}
% \def\t{^^A
% \begin{tabular}{@{}>{\ttfamily}l@{ $\rightarrow$ }>{\ttfamily}l@{}}
%   kvoptions.sty & tex/latex/oberdiek/kvoptions.sty\\
%   kvoptions-patch.sty & tex/latex/oberdiek/kvoptions-patch.sty\\
%   kvoptions.pdf & doc/latex/oberdiek/kvoptions.pdf\\
%   example-mycolorsetup.sty & doc/latex/oberdiek/example-mycolorsetup.sty\\
%   test/kvoptions-test1.tex & doc/latex/oberdiek/test/kvoptions-test1.tex\\
%   test/kvoptions-test2.tex & doc/latex/oberdiek/test/kvoptions-test2.tex\\
%   test/kvoptions-test3.tex & doc/latex/oberdiek/test/kvoptions-test3.tex\\
%   test/kvoptions-test4.tex & doc/latex/oberdiek/test/kvoptions-test4.tex\\
%   test/kvoptions-test4.sty & doc/latex/oberdiek/test/kvoptions-test4.sty\\
%   kvoptions.dtx & source/latex/oberdiek/kvoptions.dtx\\
% \end{tabular}^^A
% }^^A
% \sbox0{\t}^^A
% \ifdim\wd0>\linewidth
%   \begingroup
%     \advance\linewidth by\leftmargin
%     \advance\linewidth by\rightmargin
%   \edef\x{\endgroup
%     \def\noexpand\lw{\the\linewidth}^^A
%   }\x
%   \def\lwbox{^^A
%     \leavevmode
%     \hbox to \linewidth{^^A
%       \kern-\leftmargin\relax
%       \hss
%       \usebox0
%       \hss
%       \kern-\rightmargin\relax
%     }^^A
%   }^^A
%   \ifdim\wd0>\lw
%     \sbox0{\small\t}^^A
%     \ifdim\wd0>\linewidth
%       \ifdim\wd0>\lw
%         \sbox0{\footnotesize\t}^^A
%         \ifdim\wd0>\linewidth
%           \ifdim\wd0>\lw
%             \sbox0{\scriptsize\t}^^A
%             \ifdim\wd0>\linewidth
%               \ifdim\wd0>\lw
%                 \sbox0{\tiny\t}^^A
%                 \ifdim\wd0>\linewidth
%                   \lwbox
%                 \else
%                   \usebox0
%                 \fi
%               \else
%                 \lwbox
%               \fi
%             \else
%               \usebox0
%             \fi
%           \else
%             \lwbox
%           \fi
%         \else
%           \usebox0
%         \fi
%       \else
%         \lwbox
%       \fi
%     \else
%       \usebox0
%     \fi
%   \else
%     \lwbox
%   \fi
% \else
%   \usebox0
% \fi
% \end{quote}
% If you have a \xfile{docstrip.cfg} that configures and enables \docstrip's
% TDS installing feature, then some files can already be in the right
% place, see the documentation of \docstrip.
%
% \subsection{Refresh file name databases}
%
% If your \TeX~distribution
% (\TeX\,Live, \mikTeX, \dots) relies on file name databases, you must refresh
% these. For example, \TeX\,Live\ users run \verb|texhash| or
% \verb|mktexlsr|.
%
% \subsection{Some details for the interested}
%
% \paragraph{Unpacking with \LaTeX.}
% The \xfile{.dtx} chooses its action depending on the format:
% \begin{description}
% \item[\plainTeX:] Run \docstrip\ and extract the files.
% \item[\LaTeX:] Generate the documentation.
% \end{description}
% If you insist on using \LaTeX\ for \docstrip\ (really,
% \docstrip\ does not need \LaTeX), then inform the autodetect routine
% about your intention:
% \begin{quote}
%   \verb|latex \let\install=y\input{kvoptions.dtx}|
% \end{quote}
% Do not forget to quote the argument according to the demands
% of your shell.
%
% \paragraph{Generating the documentation.}
% You can use both the \xfile{.dtx} or the \xfile{.drv} to generate
% the documentation. The process can be configured by the
% configuration file \xfile{ltxdoc.cfg}. For instance, put this
% line into this file, if you want to have A4 as paper format:
% \begin{quote}
%   \verb|\PassOptionsToClass{a4paper}{article}|
% \end{quote}
% An example follows how to generate the
% documentation with pdf\LaTeX:
% \begin{quote}
%\begin{verbatim}
%pdflatex kvoptions.dtx
%makeindex -s gind.ist kvoptions.idx
%pdflatex kvoptions.dtx
%makeindex -s gind.ist kvoptions.idx
%pdflatex kvoptions.dtx
%\end{verbatim}
% \end{quote}
%
% \begin{thebibliography}{99}
% \bibitem{tb94wright}
%   A guide to key-value methods, Joseph Wright, second draft for
%   \href{https://www.tug.org/tugboat}{TUGBoat}, 2009-03-17.
%   \url{https://www.texdev.net/uploads/2009/03/keyval.pdf}
% \bibpackage{ifthen}{David Carlisle}{2001/05/26}
%   {\CTANpkg{ifthen}}
% \bibpackage{helvet}{Sebastian Rahtz, Walter Schmidt}{2004/01/26}
%   {\CTANpkg{psfonts}}
% \bibpackage{hyperref}{Sebastian Rahtz, Heiko Oberdiek}{2006/02/12}
%   {\CTANpkg{hyperref}}
% \bibpackage{keyval}{David Carlisle}{1999/03/16}
%   {\CTANpkg{keyval}}
% \bibpackage{multicol}{Frank Mittelbach}{2004/02/14}
%   {\CTANpkg{multicol}}
% \bibpackage{tabularx}{David Carlisle}{1999/01/07}
%   {\CTANpkg{tabularx}}
% \bibpackage{tracefnt}{Frank Mittelbach, Rainer Sch\"opf}{1997/05/29}
%   {\CTANpkg{latex-base}}
% \bibpackage{xkeyval}{Hendri Adriaens}{2005/05/07}
%   {\CTANpkg{xkeyval}}
% \bibitem{clsguide}
%   The \LaTeX3 Project, \textit{\LaTeXe\ for class and package writers},
%   2003/12/09.
%   \CTANpkg{clsguide}
%
% \end{thebibliography}
%
% \begin{History}
%   \begin{Version}{0000/00/00 v0.0}
%   \item
%     Probably David Carlisle's code in \xpackage{hyperref}
%     was the start.
%   \end{Version}
%   \begin{Version}{2004/02/22 v1.0}
%   \item
%     The first version was never published. It also
%     has offered a patch to get rid of \LaTeX's option
%     expansion.
%   \end{Version}
%   \begin{Version}{2006/02/16 v2.0}
%   \item
%     Now the package is redesigned with an easier
%     user interface.
%   \item
%     \cs{ProcessKeyvalOptions} remains the central service, inherited
%     from \xpackage{hyperref}'s \cs{ProcessOptionsWithKV}.
%     Now the use inside classes is also supported.
%   \item
%     Provides help macros for boolean and simple string options.
%   \item
%     Fixes for the patch of \LaTeX{}. The patch is only enabled,
%     if the user requests it.
%   \end{Version}
%   \begin{Version}{2006/02/20 v2.1}
%   \item
%     Unused option list is sanitized to prevent problems
%     with other packages that uses own processing methods
%     for key value options. Disadvantage: the unused global
%     option detection is weakened.
%   \item
%     New option type by \cs{DeclareVoidOption} for options without
%     value.
%   \item
%     Default rule by \cs{DeclareDefaultOption}.
%   \item
%     Dynamic options: \cs{DisableKeyvalOption}.
%   \end{Version}
%   \begin{Version}{2006/06/01 v2.2}
%   \item
%     Fixes for option \xoption{patch}.
%   \end{Version}
%   \begin{Version}{2006/08/17 v2.3}
%   \item
%     \cs{DeclareBooleanOption} renamed to \cs{DeclareBoolOption}
%     to avoid a name clash with package \cs{ifoption}.
%   \end{Version}
%   \begin{Version}{2006/08/22 v2.4}
%   \item
%     Option \xoption{patch}: \cs{ExecuteOptions} does not
%     change the meaning of macro \cs{CurrentOption} at all.
%   \end{Version}
%   \begin{Version}{2007/04/11 v2.5}
%   \item
%     Line ends sanitized.
%   \end{Version}
%   \begin{Version}{2007/05/06 v2.6}
%   \item
%     Uses package \xpackage{etexcmds}.
%   \end{Version}
%   \begin{Version}{2007/06/11 v2.7}
%   \item
%     The patch part fixes LaTeX bug latex/3965.
%   \end{Version}
%   \begin{Version}{2007/10/02 v2.8}
%   \item
%     Compatibility for \plainTeX\ added.
%   \item
%     Typos in documentation fixed (Axel Sommerfeldt).
%   \end{Version}
%   \begin{Version}{2007/10/11 v2.9}
%   \item
%     Bug fix for option \xoption{patch}.
%   \end{Version}
%   \begin{Version}{2007/10/18 v3.0}
%   \item
%     New package \xpackage{kvoptions-patch}.
%   \end{Version}
%   \begin{Version}{2009/04/10 v3.1}
%   \item
%     Space by line end removed in definition of internal macro.
%   \end{Version}
%   \begin{Version}{2009/07/17 v3.2}
%   \item
%     \cs{ProcessLocalKeyvalOptions} added.
%   \item
%     \cs{DisableKeyvalOption} with the \texttt{action=ignore} option
%     fixed (Joseph Wright).
%   \end{Version}
%   \begin{Version}{2009/07/21 v3.3}
%   \item
%     \cs{DeclareLocalOption}, \cs{DeclareLocalOptions} added.
%   \end{Version}
%   \begin{Version}{2009/08/13 v3.4}
%   \item
%     Documentation addition: recommendation for Joseph Wright's
%     review article.
%   \item
%     Documentation addition: local/global options.
%   \end{Version}
%   \begin{Version}{2009/12/04 v3.5}
%   \item
%     \cs{AddToKeyvalOption} added.
%   \end{Version}
%   \begin{Version}{2009/12/08 v3.6}
%   \item
%     Fix: If a default handler is configured, it is now also
%     called for classes.
%   \end{Version}
%   \begin{Version}{2010/02/22 v3.7}
%   \item
%     Missing space in error message added.
%   \end{Version}
%   \begin{Version}{2010/07/23 v3.8}
%   \item
%     Documenation for package \xpackage{kvoptions-patch} improved.
%     No code changes.
%   \end{Version}
%   \begin{Version}{2010/12/02 v3.9}
%   \item
%     Key \texttt{setkeys} added for \cs{SetupKeyvalOptions}.
%   \end{Version}
%   \begin{Version}{2010/12/23 v3.10}
%   \item
%     \cs{DeclareVoidOption} also parses the second parameter as
%     \hologo{TeX} argument to improve compatibility with
%     \cs{DeclareOption}.
%   \end{Version}
%   \begin{Version}{2011/06/30 v3.11}
%   \item
%     Fix because of design bug in package \xpackage{xkeyval} that
%     removes global options with equal signs.
%   \end{Version}
%   \begin{Version}{2016/05/16 v3.12}
%   \item
%     Documentation updates.
%   \end{Version}
% \end{History}
%
% \PrintIndex
%
% \Finale
\endinput
|
% \end{quote}
% Do not forget to quote the argument according to the demands
% of your shell.
%
% \paragraph{Generating the documentation.}
% You can use both the \xfile{.dtx} or the \xfile{.drv} to generate
% the documentation. The process can be configured by the
% configuration file \xfile{ltxdoc.cfg}. For instance, put this
% line into this file, if you want to have A4 as paper format:
% \begin{quote}
%   \verb|\PassOptionsToClass{a4paper}{article}|
% \end{quote}
% An example follows how to generate the
% documentation with pdf\LaTeX:
% \begin{quote}
%\begin{verbatim}
%pdflatex kvoptions.dtx
%makeindex -s gind.ist kvoptions.idx
%pdflatex kvoptions.dtx
%makeindex -s gind.ist kvoptions.idx
%pdflatex kvoptions.dtx
%\end{verbatim}
% \end{quote}
%
% \begin{thebibliography}{99}
% \bibitem{tb94wright}
%   A guide to key-value methods, Joseph Wright, second draft for
%   \href{https://www.tug.org/tugboat}{TUGBoat}, 2009-03-17.
%   \url{https://www.texdev.net/uploads/2009/03/keyval.pdf}
% \bibpackage{ifthen}{David Carlisle}{2001/05/26}
%   {\CTANpkg{ifthen}}
% \bibpackage{helvet}{Sebastian Rahtz, Walter Schmidt}{2004/01/26}
%   {\CTANpkg{psfonts}}
% \bibpackage{hyperref}{Sebastian Rahtz, Heiko Oberdiek}{2006/02/12}
%   {\CTANpkg{hyperref}}
% \bibpackage{keyval}{David Carlisle}{1999/03/16}
%   {\CTANpkg{keyval}}
% \bibpackage{multicol}{Frank Mittelbach}{2004/02/14}
%   {\CTANpkg{multicol}}
% \bibpackage{tabularx}{David Carlisle}{1999/01/07}
%   {\CTANpkg{tabularx}}
% \bibpackage{tracefnt}{Frank Mittelbach, Rainer Sch\"opf}{1997/05/29}
%   {\CTANpkg{latex-base}}
% \bibpackage{xkeyval}{Hendri Adriaens}{2005/05/07}
%   {\CTANpkg{xkeyval}}
% \bibitem{clsguide}
%   The \LaTeX3 Project, \textit{\LaTeXe\ for class and package writers},
%   2003/12/09.
%   \CTANpkg{clsguide}
%
% \end{thebibliography}
%
% \begin{History}
%   \begin{Version}{0000/00/00 v0.0}
%   \item
%     Probably David Carlisle's code in \xpackage{hyperref}
%     was the start.
%   \end{Version}
%   \begin{Version}{2004/02/22 v1.0}
%   \item
%     The first version was never published. It also
%     has offered a patch to get rid of \LaTeX's option
%     expansion.
%   \end{Version}
%   \begin{Version}{2006/02/16 v2.0}
%   \item
%     Now the package is redesigned with an easier
%     user interface.
%   \item
%     \cs{ProcessKeyvalOptions} remains the central service, inherited
%     from \xpackage{hyperref}'s \cs{ProcessOptionsWithKV}.
%     Now the use inside classes is also supported.
%   \item
%     Provides help macros for boolean and simple string options.
%   \item
%     Fixes for the patch of \LaTeX{}. The patch is only enabled,
%     if the user requests it.
%   \end{Version}
%   \begin{Version}{2006/02/20 v2.1}
%   \item
%     Unused option list is sanitized to prevent problems
%     with other packages that uses own processing methods
%     for key value options. Disadvantage: the unused global
%     option detection is weakened.
%   \item
%     New option type by \cs{DeclareVoidOption} for options without
%     value.
%   \item
%     Default rule by \cs{DeclareDefaultOption}.
%   \item
%     Dynamic options: \cs{DisableKeyvalOption}.
%   \end{Version}
%   \begin{Version}{2006/06/01 v2.2}
%   \item
%     Fixes for option \xoption{patch}.
%   \end{Version}
%   \begin{Version}{2006/08/17 v2.3}
%   \item
%     \cs{DeclareBooleanOption} renamed to \cs{DeclareBoolOption}
%     to avoid a name clash with package \cs{ifoption}.
%   \end{Version}
%   \begin{Version}{2006/08/22 v2.4}
%   \item
%     Option \xoption{patch}: \cs{ExecuteOptions} does not
%     change the meaning of macro \cs{CurrentOption} at all.
%   \end{Version}
%   \begin{Version}{2007/04/11 v2.5}
%   \item
%     Line ends sanitized.
%   \end{Version}
%   \begin{Version}{2007/05/06 v2.6}
%   \item
%     Uses package \xpackage{etexcmds}.
%   \end{Version}
%   \begin{Version}{2007/06/11 v2.7}
%   \item
%     The patch part fixes LaTeX bug latex/3965.
%   \end{Version}
%   \begin{Version}{2007/10/02 v2.8}
%   \item
%     Compatibility for \plainTeX\ added.
%   \item
%     Typos in documentation fixed (Axel Sommerfeldt).
%   \end{Version}
%   \begin{Version}{2007/10/11 v2.9}
%   \item
%     Bug fix for option \xoption{patch}.
%   \end{Version}
%   \begin{Version}{2007/10/18 v3.0}
%   \item
%     New package \xpackage{kvoptions-patch}.
%   \end{Version}
%   \begin{Version}{2009/04/10 v3.1}
%   \item
%     Space by line end removed in definition of internal macro.
%   \end{Version}
%   \begin{Version}{2009/07/17 v3.2}
%   \item
%     \cs{ProcessLocalKeyvalOptions} added.
%   \item
%     \cs{DisableKeyvalOption} with the \texttt{action=ignore} option
%     fixed (Joseph Wright).
%   \end{Version}
%   \begin{Version}{2009/07/21 v3.3}
%   \item
%     \cs{DeclareLocalOption}, \cs{DeclareLocalOptions} added.
%   \end{Version}
%   \begin{Version}{2009/08/13 v3.4}
%   \item
%     Documentation addition: recommendation for Joseph Wright's
%     review article.
%   \item
%     Documentation addition: local/global options.
%   \end{Version}
%   \begin{Version}{2009/12/04 v3.5}
%   \item
%     \cs{AddToKeyvalOption} added.
%   \end{Version}
%   \begin{Version}{2009/12/08 v3.6}
%   \item
%     Fix: If a default handler is configured, it is now also
%     called for classes.
%   \end{Version}
%   \begin{Version}{2010/02/22 v3.7}
%   \item
%     Missing space in error message added.
%   \end{Version}
%   \begin{Version}{2010/07/23 v3.8}
%   \item
%     Documenation for package \xpackage{kvoptions-patch} improved.
%     No code changes.
%   \end{Version}
%   \begin{Version}{2010/12/02 v3.9}
%   \item
%     Key \texttt{setkeys} added for \cs{SetupKeyvalOptions}.
%   \end{Version}
%   \begin{Version}{2010/12/23 v3.10}
%   \item
%     \cs{DeclareVoidOption} also parses the second parameter as
%     \hologo{TeX} argument to improve compatibility with
%     \cs{DeclareOption}.
%   \end{Version}
%   \begin{Version}{2011/06/30 v3.11}
%   \item
%     Fix because of design bug in package \xpackage{xkeyval} that
%     removes global options with equal signs.
%   \end{Version}
%   \begin{Version}{2016/05/16 v3.12}
%   \item
%     Documentation updates.
%   \end{Version}
% \end{History}
%
% \PrintIndex
%
% \Finale
\endinput

%        (quote the arguments according to the demands of your shell)
%
% Documentation:
%    (a) If kvoptions.drv is present:
%           latex kvoptions.drv
%    (b) Without kvoptions.drv:
%           latex kvoptions.dtx; ...
%    The class ltxdoc loads the configuration file ltxdoc.cfg
%    if available. Here you can specify further options, e.g.
%    use A4 as paper format:
%       \PassOptionsToClass{a4paper}{article}
%
%    Programm calls to get the documentation (example):
%       pdflatex kvoptions.dtx
%       makeindex -s gind.ist kvoptions.idx
%       pdflatex kvoptions.dtx
%       makeindex -s gind.ist kvoptions.idx
%       pdflatex kvoptions.dtx
%
% Installation:
%    TDS:tex/latex/oberdiek/kvoptions.sty
%    TDS:tex/latex/oberdiek/kvoptions-patch.sty
%    TDS:doc/latex/oberdiek/kvoptions.pdf
%    TDS:doc/latex/oberdiek/example-mycolorsetup.sty
%    TDS:doc/latex/oberdiek/test/kvoptions-test1.tex
%    TDS:doc/latex/oberdiek/test/kvoptions-test2.tex
%    TDS:doc/latex/oberdiek/test/kvoptions-test3.tex
%    TDS:doc/latex/oberdiek/test/kvoptions-test4.tex
%    TDS:doc/latex/oberdiek/test/kvoptions-test4.sty
%    TDS:source/latex/oberdiek/kvoptions.dtx
%
%<*ignore>
\begingroup
  \catcode123=1 %
  \catcode125=2 %
  \def\x{LaTeX2e}%
\expandafter\endgroup
\ifcase 0\ifx\install y1\fi\expandafter
         \ifx\csname processbatchFile\endcsname\relax\else1\fi
         \ifx\fmtname\x\else 1\fi\relax
\else\csname fi\endcsname
%</ignore>
%<*install>
\input docstrip.tex
\Msg{************************************************************************}
\Msg{* Installation}
\Msg{* Package: kvoptions 2016/05/16 v3.12 Key value format for package options (HO)}
\Msg{************************************************************************}

\keepsilent
\askforoverwritefalse

\let\MetaPrefix\relax
\preamble

This is a generated file.

Project: kvoptions
Version: 2016/05/16 v3.12

Copyright (C)
   2004, 2006, 2007, 2009-2011 Heiko Oberdiek
   2016-2019 Oberdiek Package Support Group

This work may be distributed and/or modified under the
conditions of the LaTeX Project Public License, either
version 1.3c of this license or (at your option) any later
version. This version of this license is in
   https://www.latex-project.org/lppl/lppl-1-3c.txt
and the latest version of this license is in
   https://www.latex-project.org/lppl.txt
and version 1.3 or later is part of all distributions of
LaTeX version 2005/12/01 or later.

This work has the LPPL maintenance status "maintained".

The Current Maintainers of this work are
Heiko Oberdiek and the Oberdiek Package Support Group
https://github.com/ho-tex/oberdiek/issues


This work consists of the main source file kvoptions.dtx
and the derived files
   kvoptions.sty, kvoptions.pdf, kvoptions.ins, kvoptions.drv,
   kvoptions-patch.sty, example-mycolorsetup.sty,
   kvoptions-test1.tex, kvoptions-test2.tex,
   kvoptions-test3.tex, kvoptions-test4.tex,
   kvoptions-test4.sty.

\endpreamble
\let\MetaPrefix\DoubleperCent

\generate{%
  \file{kvoptions.ins}{\from{kvoptions.dtx}{install}}%
  \file{kvoptions.drv}{\from{kvoptions.dtx}{driver}}%
  \usedir{tex/latex/oberdiek}%
  \file{kvoptions.sty}{\from{kvoptions.dtx}{package}}%
  \file{kvoptions-patch.sty}{\from{kvoptions.dtx}{patch}}%
  \usedir{doc/latex/oberdiek}%
  \file{example-mycolorsetup.sty}{\from{kvoptions.dtx}{example}}%
%  \usedir{doc/latex/oberdiek/test}%
%  \file{kvoptions-test1.tex}{\from{kvoptions.dtx}{test1}}%
%  \file{kvoptions-test2.tex}{\from{kvoptions.dtx}{test2}}%
%  \file{kvoptions-test3.tex}{\from{kvoptions.dtx}{test3}}%
%  \file{kvoptions-test4.tex}{\from{kvoptions.dtx}{test4}}%
%  \file{kvoptions-test4.sty}{\from{kvoptions.dtx}{test4pkg}}%
  \nopreamble
  \nopostamble
%  \usedir{source/latex/oberdiek/catalogue}%
%  \file{kvoptions.xml}{\from{kvoptions.dtx}{catalogue}}%
}

\catcode32=13\relax% active space
\let =\space%
\Msg{************************************************************************}
\Msg{*}
\Msg{* To finish the installation you have to move the following}
\Msg{* files into a directory searched by TeX:}
\Msg{*}
\Msg{*     kvoptions.sty, kvoptions-patch.sty}
\Msg{*}
\Msg{* To produce the documentation run the file `kvoptions.drv'}
\Msg{* through LaTeX.}
\Msg{*}
\Msg{* Happy TeXing!}
\Msg{*}
\Msg{************************************************************************}

\endbatchfile
%</install>
%<*ignore>
\fi
%</ignore>
%<*driver>
\NeedsTeXFormat{LaTeX2e}
\ProvidesFile{kvoptions.drv}%
  [2016/05/16 v3.12 Key value format for package options (HO)]%
\documentclass{ltxdoc}
\usepackage{holtxdoc}[2011/11/22]
\begin{document}
  \DocInput{kvoptions.dtx}%
\end{document}
%</driver>
% \fi
%
%
% \CharacterTable
%  {Upper-case    \A\B\C\D\E\F\G\H\I\J\K\L\M\N\O\P\Q\R\S\T\U\V\W\X\Y\Z
%   Lower-case    \a\b\c\d\e\f\g\h\i\j\k\l\m\n\o\p\q\r\s\t\u\v\w\x\y\z
%   Digits        \0\1\2\3\4\5\6\7\8\9
%   Exclamation   \!     Double quote  \"     Hash (number) \#
%   Dollar        \$     Percent       \%     Ampersand     \&
%   Acute accent  \'     Left paren    \(     Right paren   \)
%   Asterisk      \*     Plus          \+     Comma         \,
%   Minus         \-     Point         \.     Solidus       \/
%   Colon         \:     Semicolon     \;     Less than     \<
%   Equals        \=     Greater than  \>     Question mark \?
%   Commercial at \@     Left bracket  \[     Backslash     \\
%   Right bracket \]     Circumflex    \^     Underscore    \_
%   Grave accent  \`     Left brace    \{     Vertical bar  \|
%   Right brace   \}     Tilde         \~}
%
% \GetFileInfo{kvoptions.drv}
%
% \title{The \xpackage{kvoptions} package}
% \date{2016/05/16 v3.12}
% \author{Heiko Oberdiek\thanks
% {Please report any issues at \url{https://github.com/ho-tex/oberdiek/issues}}}
%
% \maketitle
%
% \begin{abstract}
% This package is intended for package authors who want to
% use options in key value format for their package options.
% \end{abstract}
%
% \tableofcontents
%
% \def\M#1{\texttt{\{}\meta{#1}\texttt{\}}}
% \def\O#1{\texttt{[}\meta{#1}\texttt{]}}
%
% \section{Introduction}
%
% First I want to recommend the very good review article
% ``A guide to key-value methods'' by Joseph Wright \cite{tb94wright}.
% It introduces the different key-value packages and compares them.
%
% \subsection{The beginning}
%
% This package \xpackage{kvoptions} addresses class or package writers
% that want to allow their users to specify options
% as key value pairs, e.g.:
% \begin{quote}
% |\documentclass[verbose=false,name=me]{myclass}|\\
% |\usepackage[format=print]{mylayout}|
% \end{quote}
% Prominent example is package \xpackage{hyperref},
% probably the first package that offers this service.
% It's \cs{ProcessOptionsWithKV} is often copied und
% used in other packages, e.g. package \xpackage{helvet}
% that uses this interface for its option \xoption{scaled}.
%
% However copying code is not the most modern software
% development technique. And \xpackage{hyperref}'s
% code for \cs{ProcessOptionsWithKV} was changed
% to fix bugs. The version used in other packages
% depends on the time of copying and the awareness
% of \xpackage{hyperref}'s changes.
% Now the code is sourced out into this package and
% available for other package or class writers.
%
% \subsection{Overview}
%
% Package \xpackage{kvoptions} connects package
% \xpackage{\emph{k}ey\emph{v}al}
% with \LaTeX's package and class \emph{options}:
% \begin{center}
% \def\T#1{\bfseries #1}
% \renewcommand{\arraystretch}{1.5}
% \begin{tabular}{l|l|l}
%   \T{Package \xpackage{keyval}}
%   & \T{Package \xpackage{kvoptions}}
%   & \T{\LaTeX\ kernel}
%   \\
% \hline
%     \cs{define@key}
%   &
%     \renewcommand*{\arraystretch}{1}^^A
%     \begin{tabular}[t]{@{}l@{}}
%       \cs{DeclareVoidOption}\\
%       \cs{DeclareStringOption}\\
%       \cs{DeclareBoolOption}\\
%       \cs{DeclareComplementaryOption}\\
%       \cs{DisableKeyvalOption}
%     \end{tabular}
%   & \cs{DeclareOption}
%   \\
% \hline
%   & \cs{DeclareDefaultOption}
%   & \cs{DeclareOption*}
%   \\
% \hline
%   & \cs{ProcessKeyvalOptions}
%   & \cs{ProcessOptions*}
%   \\
% \hline
%   & Option \xoption{patch}
%   &
%     \renewcommand*{\arraystretch}{1}^^A
%     \begin{tabular}[t]{@{}l@{}}
%       Class/package\\
%       option system
%     \end{tabular}
%   \\
% \hline
%   & \cs{SetupKeyvalOptions} &
%   \\
% \end{tabular}
% \end{center}
%
% \hypersetup{bookmarksopenlevel=3}
% \section{Usage}
%
% \subsection{Process options}
%
% \subsubsection{\cs{ProcessKeyvalOptions}}
%
% \begin{declcs}{ProcessKeyvalOptions}\,\M{family}\\
%   \cs{ProcessKeyvalOptions} \*
% \end{declcs}
% This command evaluates the global or local options of
% the package that are defined with \xpackage{keyval}'s
% interface within the family \meta{family}. It acts the
% same way as \LaTeX's \cs{ProcessOptions*}. In a package
% unknown global options are ignored, in a class they
% are added to the unknown option list. The known global
% options and all local options are passed to \xpackage{keyval}'s
% \cs{setkeys} command for executing the options.
% Unknown options are reported to the user by an error.
%
% If the family name happens to be the same as the name
% of the package or class where \cs{ProcessKeyvalOptions}
% is used or the family name has previously been setup by
% \cs{SetupKeyvalOptions}, then \cs{ProcessKeyvalOptions}
% knows the family name already and you can use the star form
% without mandatory argument.
%
% \subsubsection{\cs{ProcessLocalKeyvalOptions}}
%
% \begin{declcs}{ProcessLocalKeyvalOptions}\,\M{family}\\
%   \cs{ProcessLocalKeyvalOptions} \*
% \end{declcs}
% This macro has the same syntax and works similar as
% \cs{ProcessKeyvalOptions}. However it ignores
% global options and only processes the local package options.
% Therefore it only can be used inside a package.
% An error is thrown, if it is used inside a class.
%
% Neither of the following macros are necessary for
% \cs{ProcessKeyvalOptions{}}. They just help the package/class
% author in common tasks.
%
% \subsubsection{\cs{SetupKeyvalOptions}}\label{setup}
%
% \begin{declcs}{SetupKeyvalOptions}\,\texttt{\{}\\
%   \hspace{1.5em}|family|\,|=|\,\meta{family}|,|\\
%   \hspace{1.5em}|prefix|\,|=|\,\meta{prefix}\\
%   \hspace{1.5em}|setkeys|\,|=|\,\meta{setkeys command}\\
%   \texttt{\}}
% \end{declcs}
% This command allows to configure the default assumptions
% that are based on the current package or class name.
% \LaTeX\ remembers this name in \cs{@currname}. The syntax
% description of the default looks a little weird, therefor
% an example is given for a package or class named |foobar|.
% \begin{quote}
% \begin{tabular}{@{}l|ll|l@{}}
%   Key & Default & (example) & Used by\\
% \hline
%   \rule{0pt}{1.2\ht\csname @arstrutbox\endcsname}^^A
%   |family| & \meta{\cs{@currname}} & (|foobar|)
%       & \cs{ProcessKeyvalOptions*} \\
%     &&& \cs{DeclareBoolOption} \\
%     &&& \cs{DeclareStringOption} \\
% \hline
%   \rule{0pt}{1.2\ht\csname @arstrutbox\endcsname}^^A
%   |prefix| & \meta{\cs{@currname}}|@| & (|foobar@|)
%       & \cs{DeclareBoolOption} \\
%     &&& \cs{DeclareStringOption} \\
%     &&& \cs{DeclareVoidOption} \\
% \hline
%   \rule{0pt}{1.2\ht\csname @arstrutbox\endcsname}^^A
%   |setkeys| & \cs{setkeys} & (\cs{kvsetkeys})
%       & \cs{ProcessKeyvalOptions} \\
%     &&& \cs{ProcessLocalKeyvalOptions}
% \end{tabular}
% \end{quote}
% Key |setkeys| was added in version 3.9. The original
% \cs{setkeys} of package \xpackage{keyval} is not reentrant.
% If an option is processed by this \cs{setkeys}, then
% the option should not call \cs{setkeys} again with a
% different family. Otherwise the next options of the
% first \cs{setkeys} call are processed with the wrong
% family. With key |setkeys| the macro \cs{kvsetkeys}
% can be set that does not have the problem of the
% original \cs{setkeys} of package {keyval}.
%
% Probably \cs{setkeys} of package \xpackage{xkeyval}
% is safe in this respect. But I haven't made a full
% analysis. At least it does not have the problem
% of the original \cs{setkeys}.
%
% \subsection{Option declarations}
%
% The options for \cs{ProcessKeyvalOptions} are defined
% by \xpackage{keyval}'s \cs{define@key}.
% Common purposes of such keys are boolean switches, they
% enable or disable something. Or they store a name or some kind of
% string in a macro. The following commands help the user. He
% declares what he wants and \xpackage{kvoptions} take care of
% the key definition, resource allocation and initialization.
%
% In order to avoid name clashes of macro names, internal
% commands are prefixed. Both the prefix and the family
% name for the defined keys can be configured by
% \cs{SetupKeyvalOptions}.
%
% \subsubsection{\cs{DeclareStringOption}}
%
% \begin{declcs}{DeclareStringOption}\,\O{init}\,\M{key}\,\O{default}
% \end{declcs}
% A macro is created that remembers the value of the key \meta{key}.
% The name of the macro consists of the option name \meta{key} that
% is prefixed by the prefix (see \ref{setup}).
% The initial contents of the
% macro can be given by the first optional argument \meta{init}.
% The default is empty.
%
% The the option \meta{key} is defined. The option code
% just stores its value in the macro. If the optional argument
% at the end of \cs{DeclareStringOption} is given, then option
% \meta{key} is defined with the default \meta{default}.
%
% Example for a package with the following two lines:
% \begin{quote}
% |\ProvidesPackage{foobar}|\\
% |\DeclareStringOption[me]{name}|
% \end{quote}
% Then \cs{DeclareStringOption} defines the macro with content |me|,
% note \LaTeX\ complains if the name of the macro already exists:
% \begin{quote}
% |\newcommand*{\foobar@name}{me}|
% \end{quote}
% The option definition is similar to:
% \begin{quote}
% |\define@key{foobar}{name}{%|\\
% |  \renewcommand*{\foobar@name}{#1}%|\\
% |}|
% \end{quote}
%
% \subsubsection{\cs{DeclareBoolOption}}
%
% \begin{declcs}{DeclareBoolOption}\,\O{init}\,\M{key}
% \end{declcs}
% A boolean switch is generated, initialized by value \meta{init}
% and the corresponding key \meta{key} is defined. If the
% initialization value is not given, |false| is used as default.
%
% The internal actions of \cs{DeclareBoolOption} are shown below.
% The example is given for a package author who has the following
% two lines in his package/class:
% \begin{quote}
% |\ProvidesPackage{foobar}|\\
% |\DeclareBoolOption{verbose}|
% \end{quote}
% First a new switch is created:
% \begin{quote}
%   |\newif\iffoobar@verbose|
% \end{quote}
% and initialized:
% \begin{quote}
%   |\foobar@verbosefalse|
% \end{quote}
% Finally the key is defined:
% \begin{quote}
%   |\define@key{foobar}{verbose}[true]{|\dots|}|
% \end{quote}
% The option code configures the boolean option in the following way:
% If the author specifies |true| or |false| then the switch is
% turned on or off respectivly. Also the option can be given without
% explicit value. Then the switch is enabled.
% Other values are reported as errors.
%
% Now the switch is ready to use in the package/class, e.g.:
% \begin{quote}
%   |\iffoobar@verbose|\\
%   \qquad\textit{\% print verbose message}\\
%   |\else|\\
%   \qquad\textit{\% be quiet}\\
%   |\fi|
% \end{quote}
% Users of package \cs{ifthen} can use the switch as
% boolean:
% \begin{quote}
%   |\boolean{foobar@verbose}|
% \end{quote}
%
% \subsubsection{\cs{DeclareComplementaryOption}}
%
% \begin{declcs}{DeclareComplementaryOption}\,\M{key}\,\M{parent}
% \end{declcs}
% Sometimes contrasting names are used to characterize the
% two states of a boolean switch, for example \xoption{draft}
% vs.\@ \xoption{final}. Both options behave like boolean
% options but they do not need two different switches, they
% should share one. \cs{DeclareComplementaryOption} allows this.
% The option \meta{key} shares the switch of option \meta{parent}.
% Example:
% \begin{quote}
%   |\DeclareBoolOption{draft}|\\
%   |\DeclareComplementaryOption{final}{draft}|
% \end{quote}
% Then |final| sets the switch of \xoption{draft} to |false|, and
% |final=false| enables the \xoption{draft} switch.
%
% \subsubsection{\cs{DeclareVoidOption}}
%
% \begin{declcs}{DeclareVoidOption}\,\M{key}\,\M{code}
% \end{declcs}
% \cs{ProcessKeyvalOptions} can be extended to recognize
% options that are declared in traditional way by \cs{DeclareOption}.
% But in case of the error that the user specifies a value, then
% this option would not recognized as key value option because
% of \cs{DeclareOption} and not detected as traditional option
% because of the value part. The user would get an unknown option
% error, difficult to understand.
%
% \cs{DeclareVoidOption} solves
% this problem. It defines the option \meta{key} as key value option.
% If the user specifies a value, a warning is given and the
% value is ignored.
%
% The code part \meta{code} is stored in a macro.
% The name of the macro consists of the option name \meta{key} that
% is prefixed by the prefix (see \ref{setup}).
% If the option is set, the macro will be executed. During the
% execution \cs{CurrentOption} is available with the current key name.
%
% \subsubsection{\cs{DeclareDefaultOption}}
%
% \begin{declcs}{DeclareDefaultOption}\,\M{code}
% \end{declcs}
% \begin{sloppypar}
% This command does not define a specific key, it is the equivalent
% to \LaTeX's \cs{DeclareOption*}. It allows the specification
% of a default action \meta{code} that is invoked if an unknown option
% is found.
% While \meta{code} is called,
% macro \cs{CurrentOption} contains the current option string.
% In addition \cs{CurrentOptionValue} contains the value part
% if the option string is parsable as key value pair, otherwise
% it is \cs{relax}. \cs{CurrentOptionKey} contains the key of
% the key value pair, or the whole option string, if it misses
% the equal sign.
% \end{sloppypar}
%
% Inside packages typical default actions are to pass unknown
% options to another package. Or an error message can be thrown
% by \cs{@unknownoptionerror}. This is the original error
% message that \LaTeX\ gives for unkown package options.
% This error message is easier to understand for the user
% as the error message from package \xpackage{keyval} that
% is given otherwise.
%
% A Class ignores unknown options and puts them on the
% unused option list. Let \LaTeX\ do the job and just
% call \cs{OptionNotUsed}. Or the options can be passed to
% another class that is later loaded.
%
% \subsubsection{Local options}
%
% \begin{declcs}{DeclareLocalOption} \M{option}\\
%   \cs{DeclareLocalOptions} \M{option list}
% \end{declcs}
% Both macros mark package options as local options. That means
% that they are ignored by \cs{ProcessKeyvalOptions} if they are given
% as global options. \cs{DeclareLocalOptions} takes one option,
% \cs{DeclareLocalOptions} expects a comma separated list of options.
%
% \subsubsection{Dynamic options}
%
% Options of \LaTeX's package/class system are cleared
% in \cs{ProcessOptions}. They modify the static model
% of a package. For example, depending on option \xoption{bookmarks}
% package \xpackage{hyperref} loads differently.
%
% Options, however,
% defined by \xpackage{keyval}'s \cs{define@key} remain
% defined, if the options are processed by \cs{setkeys}.
% Therefore these options can also be used to model
% the dynamic behaviour of a package. For example, in
% \xpackage{hyperref} the link colors can be changed
% everywhere until the end in |\end{document}|.
%
% However package \xpackage{color} that adds color support
% is necessary and it
% cannot be loaded after |\begin{document}|. Option \xoption{colorlinks}
% that loads \xpackage{color} should be active until |\begin{document}|
% and die in some way if it is too late for loading packages.
% With \cs{DisableKeyvalOption} the package/class author can
% specify and configure the death of an option and controls
% the life period of the option.
%
% \subsubsection{\cs{DisableKeyvalOption}}
%
% \begin{declcs}{DisableKeyvalOption}\,\O{options}\,\M{family}\,\M{key}
%   \\[0.5ex]
%   \meta{options}:\\
%   \hspace{1.5em}^^A
%   \begin{tabular}[t]{@{}l@{ \texttt{=} }l@{\qquad}>{default: }l@{}}
%     |action| & |undef|, |warning|, |error|, or |ignore|
%     & |undef|
%     \\
%     \multicolumn{1}{@{}l}{|global| or |local|} & & |global|
%     \\
%     |package| or |class| & \meta{name}
%     \\
%   \end{tabular}^^A
% \end{declcs}
% \cs{DisableKeyvalOption} can be called to mark the end when
% the option \meta{key} is no longer useful. The behaviour of
% an option after its death can be configured by \xoption{action}:
% \begin{description}
% \item[\texttt{undef}:] The option will be undefined,
%   If it is called, \cs{setkeys} reports an error because
%   of unknown key.
% \item[\texttt{error} or \texttt{warning}:] Use of the
%  option will cause an error or warning message.
%  Also these actions require that exclusivly either the
%  package or class name is given in options \xoption{package}
%  or \xoption{class}.
% \item[\texttt{ignore}:] The use of the option will silently
%  be ignored.
% \end{description}
% The option's death can be limited to the end of the current
% group, if option \xoption{local} is given. Default is
% \xoption{global}.
%
% The package/class author can wish the end of the option
% already during the package loading, then he will have
% static behaviour. In case of dynamic options \cs{DisableKeyvalOption}
% can be executed everywhere, also outside the package. Therefore
% the family name and the package/class name is usually unknown for
% \cs{DisableKeyvalOption}. Therefore the argument for the family
% name is mandatory and for some actions the package/class name
% must be provided.
%
% Usually a macro would configure the option death, Example:
%\begin{quote}
%\begin{verbatim}
%\ProvidesPackage{foobar}
%\DeclareBoolOption{color}
%\DeclareStringOption[red]{emphcolor}
%\ProcessKeyvalOptions*
%
%\newcommand*{\foobar@DisableOption}[2]{%
%  \DisableKeyvalueOption[
%    action={#1},
%    package=foobar
%  ]{foobar}{#2}%
%}
%
%\iffoobar@color
%  \RequirePackage{color}
%  \renewcommand*{\emph}[1]{\textcolor{\foobar@emphcolor}{#1}}
%\else
%  % Option emphcolor is not wrong, if we have color support.
%  % otherwise the option has no effect, but we don't want to
%  % remove it. Therefore action 'ignore' is the best choice:
%  \foobar@DisableOption{ignore}{emphcolor}
%\fi
%% No we don't need the option 'color'.
%\foobar@DisableOption{warning}{color}
%
%% With color support option 'emphcolor' will dynamically
%% change the color of \emph statements.
%\end{verbatim}
%\end{quote}
%
% \subsubsection{\cs{AddToKeyvalOption}}
%
% \begin{declcs}{AddToKeyvalOption}\,\M{family}\,\M{key}\,\M{code}\\
%   \cs{AddToKeyvalOption} \*\,\M{key}\,\M{code}
% \end{declcs}
% The code for an existing key \meta{key} of family \meta{family} is
% extended by code \meta{code}. In the starred form the current
% family setting is used, see \cs{ProcessKeyvalOptions*}.
%
% \subsection{Global vs. local options}
%
% Options that are given for \cs{documentclass} are called
% global options. They are known to the class and all packages.
% A package may make use of a global option and marks it as
% used. The advantage for the user is the freedom to specify
% options both in the \cs{documentclass} or \cs{usepackage}
% commands.
%
% However global options are shared with the class options
% and options of all other packages. Thus there can be the
% same option with different semantics for different
% packages and classes. As example, package \xpackage{bookmark}
% knows option \xoption{open} that specifies whether the bookmarks
% are opened or closed initially. It's values are |true| or |false|.
% Since KOMA-Script version 3.00 the KOMA classes also introduces
% option \xoption{open} with values |right| and |any| and a
% complete different meaning.
%
% Such conflicts can be resolved by marking all or part of options
% as local by \cs{DeclareLocalOption} or \cs{DeclareLocalOptions}.
% Then the packages ignores global occurences of these options.
% Package \xpackage{kvoptions} provides two methods:
% \begin{itemize}
% \item \cs{ProcessLocalKeyvalOptions} automatically uses all options
%   as local options. It ignores all global options.
% \item \cs{DeclareLocalOption} or \cs{DeclareLocalOptions} marks
%   options as local options. \cs{ProcessKeyvalOptions} will then
%   ignore global occurences for these local options.
% \end{itemize}
% Since version 1.5 package \xpackage{bookmark} uses the latter
% method. It checks global and local option places for driver options
% and limits all other options as local options. Thus the class
% option \xoption{open} of KOMA-Script is not misread as
% option for package \xpackage{bookmark}.
%
% \subsection{Summary of internal macros}
%
% The \cs{Declare}\texttt{\dots Option} commands define
% macros, additionally to the macros generated by the key definition.
% These macros can be used by the package/class author.
% The name of the macros starts with the prefix \meta{prefix}
% that can be configured by \cs{SetupKeyvalOptions}.
% \begin{center}
% \def\prefix{\cs{}\meta{prefix}\meta{key}}^^A
% \def\Hline{^^A
%   \hline
%   \rule{0pt}{1.2\ht\csname @arstrutbox\endcsname}^^A
% }^^A
% \begin{tabular}{@{}l|>{\ttfamily}l|l@{}}
%   Declare \meta{key} & \textrm{Defined macro} & Description\\
% \Hline
%   \cs{DeclareStringOption}
%    & \prefix & holds the string\\
% \Hline
%   \cs{DeclareBoolOption}
%     & \cs{if}\meta{prefix}\meta{key}& boolean switch\\
%   &\prefix false& disable switch\\
%   &\prefix true& enable switch\\
% \Hline
%   \cs{DeclareComplementaryOption}
%     & \prefix false& enable parent switch\\
%     & \prefix true& disable parent switch\\
% \Hline
%   \cs{DeclareVoidOption}
%   & \prefix & holds the action
% \end{tabular}
% \end{center}
%
% \subsection{\plainTeX}
%
% Package \xpackage{keyval} is also usable in \plainTeX\ with
% the help of file \xfile{miniltx.tex}. Some features of this
% package \xpackage{kvoptions} might also be useful for \plainTeX.
% If \LaTeX\ is not found, \cs{ProcessKeyvalOptions} and option
% \xoption{patch} are disabled. Before using the option declaration
% commands \cs{Declare...Option}, \cs{SetupKeyvalOptions} must be
% used.
%
% \hypersetup{bookmarksopenlevel=1}
%
% \section{Example}
%
% The following example defined a package that serves some
% private color management. A boolean option \xoption{print} enables
% print mode without colors. An option \xoption{emph} redefines
% \cs{emph} to print in the given color. And the driver
% can be specified by option \xoption{driver}.
%
%    \begin{macrocode}
%<*example>
    % Package identification
    % -----------------------
\NeedsTeXFormat{LaTeX2e}
\ProvidesPackage{example-mycolorsetup}[2016/05/16 Managing my colors]

\RequirePackage{ifpdf}
\RequirePackage{kvoptions}

    % Option declarations
    % -------------------

\SetupKeyvalOptions{
  family=MCS,
  prefix=MCS@
}
    % Use a shorter family name and prefix

    % Option print
\DeclareBoolOption{print}
    % is the same as
    % \DeclareBoolOption[false]{print}

    % Option driver
\ifpdf
  \DeclareStringOption[pdftex]{driver}
\else
  \DeclareStringOption[dvips]{driver}
\fi

    % Alternative interface for driver options
\DeclareVoidOption{dvips}{\SetupDriver}
\DeclareVoidOption{dvipdfm}{\SetupDriver}
\DeclareVoidOption{pdftex}{\SetupDriver}
    % In \SetupDriver we take the current option \CurrentOption
    % and pass it to the driver option.
    % The \expandafter commands expand \CurrentOption at the
    % time, when \SetupDriver is executed and \CurrentOption
    % has the correct meaning.
\newcommand*{\SetupDriver}{%
  \expandafter\@SetupDriver\expandafter{\CurrentOption}%
}
\newcommand*{\@SetupDriver}[1]{%
  \setkeys{MCS}{driver={#1}}%
}

    % Option emph
    % An empty value means, we want to have no color for \emph.
    % If the user specifies option emph without value, the red is used.
\DeclareStringOption{emph}[red]
    % is the same as
    % \DeclareStringOption[]{emph}[red]

    % Default option rule
\DeclareDefaultOption{%
  \ifx\CurrentOptionValue\relax
    \PackageWarningNoLine{\@currname}{%
      Unknown option `\CurrentOption'\MessageBreak
      is passed to package `color'%
    }%
    % Pass the option to package color.
    % Again it is better to expand \CurrentOption.
    \expandafter\PassOptionsToPackage
    \expandafter{\CurrentOption}{color}%
  \else
    % Package color does not take options with values.
    % We provide the standard LaTeX error.
    \@unknownoptionerror
  \fi
}

    % Process options
    % ---------------
\ProcessKeyvalOptions*

    % Implementation depending on option values
    % -----------------------------------------
    % Code for print mode
\ifMCS@print
  \PassOptionsToPackage{monochrome}{color}
    % tells package color to use black and white
\fi

\RequirePackage[\MCS@driver]{color}
    % load package color with the correct driver

    % \emph setup
\ifx\MCS@emph\@empty
    % \@empty is a predefined macro with empty contents.
    % the option value of option emph is empty, thus
    % we do not want a redefinition of \emph.
\else
  \renewcommand*{\emph}[1]{%
    \textcolor{\MCS@emph}{#1}%
  }
\fi
%</example>
%    \end{macrocode}
%
% \section{Package options}
%
% The package \xpackage{kvoptions} knows two package options
% \xoption{patch} and \xoption{debugshow}.
% The options of package \xpackage{kvoptions} are intended
% for authors, not for package/class writers. Inside a package
% it is too late for option \xoption{patch} and \xoption{debugshow}
% enables some messages that are perhaps useful for the debugging
% phase. Also \LaTeX\ is unhappy if a package is loaded
% later again with options that are previously not given.
% Thus package and class authors, stay with |\RequirePackage{kvoptions}|
% without options.
%
% Option \xoption{patch} loads package \xpackage{kvoptions-patch}.
%
% \subsection{Package \xpackage{kvoptions-patch}}
%
% \LaTeX's system of package/class options has some severe limitations
% that especially affects the value part if options are used as
% pair of key and value.
% \begin{itemize}
% \item Spaces are removed, regardless where:
%   \begin{quote}
%     |\documentclass[box=0 0 400 600]{article}|
%   \end{quote}
%   Now each package will see |box=00400600| as global option.
% \item In the previous case also braces would not help:
%   \begin{quote}
%     |\documentclass[box={0 0 400 600}]{article}|
%   \end{quote}
%   The result is an error message:
%   \begin{quote}
%     |! LaTeX Error: Missing \begin{document}.|
%   \end{quote}
%   As local option, however, it works if the package
%   knows about key value options (By using this package, for example).
% \item The requirements on robustness are extremly high.
%   \LaTeX\ expands the option. All that will not work as environment
%   name will break also as option. Even a \cs{relax} will generate
%   an error message:
%   \begin{quote}
%     |! Missing \endcsname inserted.|
%   \end{quote}
%   Of course, \LaTeX\ does not use its protecting mechanisms.
%   On contrary \cs{protect} itself will cause errors.
% \item The options are expanded. But perhaps the package will
%   do that, because it has to setup some things before?
%   Example \xpackage{hyperref}:
%   \begin{quote}
%     |\usepackage[pdfauthor=M\"uller]{hyperref}|
%   \end{quote}
%   Package \xpackage{hyperref} does not see |M\"uller| but
%   its expansion and it does not like it, you get many warnings
%   \begin{quote}
%     |Token not allowed in a PDFDocEncoded string|
%   \end{quote}
%   And the title becomes: |Mu127uller|.
%   Therefore such options must usually be given after package
%   \xpackage{hyperref}
%   is loaded:
%   \begin{quote}
%     |\usepackage{hyperref}|\\
%     |\hypersetup{pdfauthor=Fran\c coise M\"uller}|
%   \end{quote}
%   As package option it will even break with |Fran\c coise|
%   because of the cedilla |\c c|, it is not robust enough.
% \end{itemize}
% For users that do not want with this limitations the package
% offers package \xoption{kvoptions-patch}. It patches \LaTeX's option system
% and tries to teach it also to handle options that are given
% as pairs of key and value and to prevent expansion.
% It can already be used at the very beginning, before \cs{documentclass}:
% \begin{quote}
%   |\RequirePackage{kvoptions-patch}|\\
%   |\documentclass[pdfauthor=Fran\c coise M\"uller]{article}|\\
%   |\usepackage{hyperref}|
% \end{quote}
% The latest time is before the package where you want to use
% problematic values:
% \begin{quote}
%   |\usepackage{kvoptions-patch}|\\
%   |\usepackage[Fran\c coise M\"uller]{hyperref}|
% \end{quote}
% Some remarks:
% \begin{itemize}
% \item The patch requires \eTeX, its \cs{unexpanded} feature
%   is much too nice. It is possible to work around using token
%   registers. But the code becomes longer, slower, more difficult
%   to read and maintain. The package without option \xoption{patch}
%   works and will work without \eTeX.
% \item The code for the patch is quite long, there are many test
%   cases. Thus the probability for bugs is probably not too small.
% \item Since 2008/10/18 v3.0 package \xpackage{kvoptions-patch} is
%   available. Before option \xoption{patch} of package \xpackage{kvoptions}
%   must be used instead. I think, the solution as standalone package
%   \xpackage{kvoptions-patch} is cleaner and avoids option clashes.
% \end{itemize}
%
% \subsection{Option \xoption{debugshow}}
%
% The name of this option follows the convention of packages
% \xpackage{multicol}, \xpackage{tabularx}, and \xpackage{tracefnt}.
% Currently it prints the setting of boolean options, declared
% by \cs{DeclareBoolOption} in the \xfile{.log} file, if that
% boolean option is used.
% You can activate the option by
% \begin{itemize}
% \item |\PassOptionsToPackage{debugshow}{kvoptions}|\\
%       Put this somewhere before package \xpackage{kvoptions} is
%       loaded first, e.g. before \cs{documentclass}.
% \item |\RequirePackage[debugshow]{kvoptions}|\\
%       Before \cs{documentclass} even an author has to use
%       \cs{RequirePackage}. \cs{usepackage} only works after
%       \cs{documentclass}.
% \end{itemize}
% The preferred method is \cs{PassOptionsToPackage}, because
% it does not force the package loading and does not disturb,
% if the package is not loaded later at all.
%
% \section{Limitations}
%
% \subsection{Compatibility}
% \subsubsection{Package \xpackage{kvoptions-patch} vs. package \xpackage{xkvltxp}}
%
% Package \xpackage{xkvltxp} from the \xpackage{xkeyval} project
% has the same goal as package \xpackage{kvoptions-patch}
% and to patch \LaTeX's kernel commands in order to get better support
% for key value options.
% Of course they cannot be used both. The user must decide, which
% method he prefers. Package \xpackage{kvoptions-patch} aborts itself,
% if it detects that \xpackage{xkvltxp} is
% already loaded.
%
% However package \xpackage{xkvltxp} and \xpackage{kvoptions}
% can be used together, example:
% \begin{quote}
% |\usepackage{xkvltxp}|\\
% |\usepackage[...]{foobar} % |\texttt{\itshape foobar using kvoptions}
% \end{quote}
% The other way should work, too.
%
% Package \xpackage{kvoptions-patch} tries to catch more situations
% and to be more robust.
% For example, during the comparison of options it normalizes
% them by removing spaces around |=| and the value. Thus the
% following is not reported as option clash:
% \begin{quote}
%\begin{verbatim}
%\RequirePackage{kvoptions-patch}
%\documentclass{article}
%
%\usepackage[scaled=0.7]{helvet}
%\usepackage[scaled = 0.7]{helvet}
%
%\begin{document}
%\end{document}
%\end{verbatim}
% \end{quote}
%
% \subsection{Limitations}
%
% \subsubsection{Option comparisons}
%
% In some situations \LaTeX\ compares option lists, e.g. option clash check,
% \cs{@ifpackagewith}, or \cs{@ifclasswith}. Apart from catcode and
% sanitizing problems of option \xoption{patch}, there is another problem.
% \LaTeX\ does not know about the type and default values of
% options in key value style. Thus an option clash is reported,
% even if the key value has the same meaning:
% \begin{quote}
% |\usepackage[scaled]{helvet} %| \texttt{\textit{default is }.95}\\
% |\usepackage[.95]{helvet}|\\
% |\usepackage[0.95]{helvet}|
% \end{quote}
%
% \subsubsection{Option list parsing with package \xpackage{kvoptions-patch}}
%
% With package \xpackage{kvoptions-patch} the range of possible values
% in key value specifications is much large, for example
% the comma can be used, if enclosed in curly braces.
%
% Other packages, especially the packages that uses their
% own process option code can be surprised to find tokens
% inside options that they do not expect and errors would
% be the consequence. To avoid errors the options, especially
% the unused option list is sanitized. That means the list
% will only contain tokens with catcode 12 (other) and
% perhaps spaces (catcode 10). This allows a safe parsing
% for other packages. But a comma in the value part is
% no longer protected by curly braces because they have lost
% their special meaning. This is the price for compatibility.
%
% Example:
% \begin{quote}
% |\RequirePackage{kvoptions-patch}|\\
% |\documentclass[a={a,b,c},b]{article}|\\
% |\begin{document}|\\
% |\end{document}|
% \end{quote}
% Result:
% \begin{quote}
% |LaTeX Warning: Unused global option(s):|\\
% |    [a={a,c},b].|
% \end{quote}
%
% \StopEventually{
% }
%
% \section{Implementation}
%
% \subsection{Preamble}
%
%    \begin{macrocode}
%<*package>
%    \end{macrocode}
%
% \paragraph{Reload check and  identification.}
%    Reload check, especially if the package is not used with \LaTeX.
%    \begin{macrocode}
\begingroup\catcode61\catcode48\catcode32=10\relax%
  \catcode13=5 % ^^M
  \endlinechar=13 %
  \catcode35=6 % #
  \catcode39=12 % '
  \catcode44=12 % ,
  \catcode45=12 % -
  \catcode46=12 % .
  \catcode58=12 % :
  \catcode64=11 % @
  \catcode123=1 % {
  \catcode125=2 % }
  \expandafter\let\expandafter\x\csname ver@kvoptions.sty\endcsname
  \ifx\x\relax % plain-TeX, first loading
  \else
    \def\empty{}%
    \ifx\x\empty % LaTeX, first loading,
      % variable is initialized, but \ProvidesPackage not yet seen
    \else
      \expandafter\ifx\csname PackageInfo\endcsname\relax
        \def\x#1#2{%
          \immediate\write-1{Package #1 Info: #2.}%
        }%
      \else
        \def\x#1#2{\PackageInfo{#1}{#2, stopped}}%
      \fi
      \x{kvoptions}{The package is already loaded}%
      \aftergroup\endinput
    \fi
  \fi
\endgroup%
%    \end{macrocode}
%    Package identification:
%    \begin{macrocode}
\begingroup\catcode61\catcode48\catcode32=10\relax%
  \catcode13=5 % ^^M
  \endlinechar=13 %
  \catcode35=6 % #
  \catcode39=12 % '
  \catcode40=12 % (
  \catcode41=12 % )
  \catcode44=12 % ,
  \catcode45=12 % -
  \catcode46=12 % .
  \catcode47=12 % /
  \catcode58=12 % :
  \catcode64=11 % @
  \catcode91=12 % [
  \catcode93=12 % ]
  \catcode123=1 % {
  \catcode125=2 % }
  \expandafter\ifx\csname ProvidesPackage\endcsname\relax
    \def\x#1#2#3[#4]{\endgroup
      \immediate\write-1{Package: #3 #4}%
      \xdef#1{#4}%
    }%
  \else
    \def\x#1#2[#3]{\endgroup
      #2[{#3}]%
      \ifx#1\@undefined
        \xdef#1{#3}%
      \fi
      \ifx#1\relax
        \xdef#1{#3}%
      \fi
    }%
  \fi
\expandafter\x\csname ver@kvoptions.sty\endcsname
\ProvidesPackage{kvoptions}%
  [2016/05/16 v3.12 Key value format for package options (HO)]%
%    \end{macrocode}
%
% \paragraph{Catcodes}
%
%    \begin{macrocode}
\begingroup\catcode61\catcode48\catcode32=10\relax%
  \catcode13=5 % ^^M
  \endlinechar=13 %
  \catcode123=1 % {
  \catcode125=2 % }
  \catcode64=11 % @
  \def\x{\endgroup
    \expandafter\edef\csname KVO@AtEnd\endcsname{%
      \endlinechar=\the\endlinechar\relax
      \catcode13=\the\catcode13\relax
      \catcode32=\the\catcode32\relax
      \catcode35=\the\catcode35\relax
      \catcode61=\the\catcode61\relax
      \catcode64=\the\catcode64\relax
      \catcode123=\the\catcode123\relax
      \catcode125=\the\catcode125\relax
    }%
  }%
\x\catcode61\catcode48\catcode32=10\relax%
\catcode13=5 % ^^M
\endlinechar=13 %
\catcode35=6 % #
\catcode64=11 % @
\catcode123=1 % {
\catcode125=2 % }
\def\TMP@EnsureCode#1#2{%
  \edef\KVO@AtEnd{%
    \KVO@AtEnd
    \catcode#1=\the\catcode#1\relax
  }%
  \catcode#1=#2\relax
}
\TMP@EnsureCode{1}{14}% ^^A (comment)
\TMP@EnsureCode{2}{14}% ^^A (comment)
\TMP@EnsureCode{33}{12}% !
\TMP@EnsureCode{39}{12}% '
\TMP@EnsureCode{40}{12}% (
\TMP@EnsureCode{41}{12}% )
\TMP@EnsureCode{42}{12}% *
\TMP@EnsureCode{44}{12}% ,
\TMP@EnsureCode{45}{12}% -
\TMP@EnsureCode{46}{12}% .
\TMP@EnsureCode{47}{12}% /
\TMP@EnsureCode{58}{12}% :
\TMP@EnsureCode{62}{12}% >
\TMP@EnsureCode{91}{12}% [
\TMP@EnsureCode{93}{12}% ]
\TMP@EnsureCode{94}{7}% ^ (superscript)
\TMP@EnsureCode{96}{12}% `
\edef\KVO@AtEnd{\KVO@AtEnd\noexpand\endinput}
%    \end{macrocode}
%
% \paragraph{External resources.}
%    The package extends the support for key value pairs of
%    package \cs{keyval} to package options. Thus the package
%    needs to be loaded anyway. and we use it for
%    \cs{SetupKeyvalOptions}. AFAIK this does not disturb
%    users of \xpackage{xkeyval}.
%    \begin{macrocode}
\@ifundefined{define@key}{%
  \RequirePackage{keyval}\relax
}{}
%    \end{macrocode}
%
%    Macro \cs{DeclareLocalOptions} parses a comma separated key list
%    and uses \cs{comma@parse} of package \xpackage{kvsetkeys},
%    version 1.3.
%    \begin{macrocode}
\RequirePackage{ltxcmds}[2010/12/02]
\RequirePackage{kvsetkeys}[2007/09/29]
%    \end{macrocode}
%
% \paragraph{Provide macros for \plainTeX.}
%    \begin{macrocode}
\@ifundefined{@x@protect}{%
  \def\@x@protect#1\fi#2#3{%
    \fi\protect#1%
  }%
  \let\@typeset@protect\relax
}{}
\@ifundefined{@currname}{%
  \def\@currname{}%
}{}
\@ifundefined{@currext}{%
  \def\@currext{}%
}{}
%    \end{macrocode}
%
% \paragraph{Options}
%    Option \xoption{debugshow} enables additional lines of
%    code that prints information into the \xfile{.log}
%    file.
%    \begin{macrocode}
\DeclareOption{debugshow}{\catcode\@ne=9 }
%    \end{macrocode}
%
%    \begin{macrocode}
\DeclareOption{patch}{%
  \AtEndOfPackage{%
    \RequirePackage{kvoptions-patch}[2016/05/16]%
  }%
}
%    \end{macrocode}
%
%    Optionen auswerten:
%    \begin{macrocode}
\ProcessOptions\relax
%    \end{macrocode}
%
% \subsection{Option declaration macros}
%
% \subsubsection{\cs{SetupKeyvalOptions}}
%
%    The family for the key value pairs can be setup once
%    and is remembered later.
%    The package name seems a reasonable default for the
%    family key, if it is not set by the package author.
%
%    \begin{macro}{\KVO@family}
%    We cannot store the family setting in one macro,
%    because the package should be usable for many
%    other packages, too. Thus we remember the family
%    setting in a macro, whose name contains the
%    package name with extension, a key in \LaTeX's
%    class/package system.
%    \begin{macrocode}
\define@key{KVO}{family}{%
  \expandafter\edef\csname KVO@family@%
      \@currname.\@currext\endcsname{#1}%
}
\def\KVO@family{%
  \@ifundefined{KVO@family@\@currname.\@currext}{%
    \@currname
  }{%
    \csname KVO@family@\@currname.\@currext\endcsname
  }%
}
%    \end{macrocode}
%    \end{macro}
%
%    \begin{macro}{\KVO@prefix}
%    The value settings of options that are declared by \cs{DeclareBoolOption}
%    and \cs{DeclareStringOption} need to be saved in macros.
%    in the first case this is a switch \cs{if}\meta{prefix}\meta{key},
%    in the latter case a macro \cs{}\meta{prefix}\meta{key}.
%    The prefix can be configured, by \xoption{prefix} that is declared
%    here. The default is the package name with |@| appended.
%    \begin{macrocode}
\define@key{KVO}{prefix}{%
  \expandafter\edef\csname KVO@prefix@%
      \@currname.\@currext\endcsname{#1}%
}
\def\KVO@prefix{%
  \ltx@ifundefined{KVO@prefix@\@currname.\@currext}{%
    \@currname @%
  }{%
    \csname KVO@prefix@\@currname.\@currext\endcsname
  }%
}
%    \end{macrocode}
%    \end{macro}
%
%    \begin{macrocode}
\define@key{KVO}{setkeys}{%
  \expandafter\def\csname KVO@setkeys@%
      \@currname.\@currext\endcsname{#1}%
}
%    \end{macrocode}
%    \begin{macro}{\KVO@setkeys}
%    \begin{macrocode}
\def\KVO@setkeys{%
  \ltx@IfUndefined{KVO@setkeys@\@currname.\@currext}{%
    \setkeys
  }{%
    \csname KVO@setkeys@\@currname.\@currext\endcsname
  }%
}
%    \end{macrocode}
%    \end{macro}
%
%    \begin{macro}{\SetupKeyvalOptions}
%    The argument of \cs{SetupKeyvalOptions} expects a key value
%    list, known keys are \xoption{family} and \xoption{prefix}.
%    \begin{macrocode}
\newcommand*{\SetupKeyvalOptions}{%
  \kvsetkeys{KVO}%
}
%    \end{macrocode}
%    \end{macro}
%
% \subsubsection{\cs{DeclareBoolOption}}
%
%    \begin{macro}{\DeclareBoolOption}
%    Usually options of boolean type can be given by the
%    user without value and this means a setting to \emph{true}.
%    We follow this convention here. Also it simplifies the
%    user interface.
%
%    The switch is created and initialized with \emph{false}.
%    The default setting can be overwritten by the optional
%    argument.
%
%    \LaTeX's \cs{newif} does not check for already defined
%    macros, therefore we add this check here to prevent
%    the user from accidently redefining of \TeX's primitives
%    and other macros.
%    \begin{macrocode}
\newcommand*{\DeclareBoolOption}[2][false]{%
  \KVO@ifdefinable{if\KVO@prefix#2}{%
    \KVO@ifdefinable{\KVO@prefix#2true}{%
      \KVO@ifdefinable{\KVO@prefix#2false}{%
        \csname newif\expandafter\endcsname
        \csname if\KVO@prefix#2\endcsname
        \@ifundefined{\KVO@prefix#2#1}{%
          \PackageWarning{kvoptions}{%
            Initialization of option `#2' failed,\MessageBreak
            cannot set boolean option to `#1',\MessageBreak
            use `true' or `false', now using `false'%
          }%
        }{%
          \csname\KVO@prefix#2#1\endcsname
        }%
        \begingroup
          \edef\x{\endgroup
            \noexpand\define@key{\KVO@family}{#2}[true]{%
              \noexpand\KVO@boolkey{\@currname}%
              \ifx\@currext\@clsextension
                \noexpand\@clsextension
              \else
                \noexpand\@pkgextension
              \fi
              {\KVO@prefix}{#2}{####1}%
            }%
          }%
        \x
      }%
    }%
  }%
}
%    \end{macrocode}
%    \end{macro}
%
%    \begin{macro}{\DeclareComplementaryOption}
%    The first argument is the key name, the second the
%    key that must be a boolean option with the same
%    current family and prefix. A new switch is not
%    created for the new key, we have already a switch.
%    Instead we define switch setting commands to work
%    on the parent switch.
%    \begin{macrocode}
\newcommand*{\DeclareComplementaryOption}[2]{%
  \@ifundefined{if\KVO@prefix#2}{%
    \PackageError{kvoptions}{%
      Cannot generate option code for `#1',\MessageBreak
      parent switch `#2' does not exist%
    }{%
      You are inside %
      \ifx\@currext\@clsextension class\else package\fi\space
      `\@currname.\@currext'.\MessageBreak
      `\KVO@family' is used as familiy %
      for the keyval options.\MessageBreak
      `\KVO@prefix' serves as prefix %
      for internal switch macros.\MessageBreak
      \MessageBreak
      \@ehc
    }%
  }{%
    \KVO@ifdefinable{\KVO@prefix#1true}{%
      \KVO@ifdefinable{\KVO@prefix#1false}{%
        \expandafter\let\csname\KVO@prefix#1false\expandafter\endcsname
          \csname\KVO@prefix#2true\endcsname
        \expandafter\let\csname\KVO@prefix#1true\expandafter\endcsname
          \csname\KVO@prefix#2false\endcsname
%    \end{macrocode}
%    The same code part as in \cs{DeclareBoolOption} can
%    now be used.
%    \begin{macrocode}
        \begingroup
          \edef\x{\endgroup
            \noexpand\define@key{\KVO@family}{#1}[true]{%
              \noexpand\KVO@boolkey{\@currname}%
              \ifx\@currext\@clsextension
                \noexpand\@clsextension
              \else
                \noexpand\@pkgextension
              \fi
              {\KVO@prefix}{#1}{####1}%
            }%
          }%
        \x
      }%
    }%
  }%
}
%    \end{macrocode}
%    \end{macro}
%
%    \begin{macro}{\KVO@ifdefinable}
%    Generate the command token LaTeX's \cs{@ifdefinable} expects.
%    \begin{macrocode}
\def\KVO@ifdefinable#1{%
  \expandafter\@ifdefinable\csname #1\endcsname
}
%    \end{macrocode}
%    \end{macro}
%
%    \begin{macro}{\KVO@boolkey}
%    We check explicitly for |true| and |false| to prevent
%    the user from accidently calling other macros.
%    \begin{quote}
%    \begin{tabular}{@{}ll@{}}
%    |#1|& package/class name\\
%    |#2|& |\@pkgextension|/|\@clsextension|\\
%    |#3|& prefix \\
%    |#4|& key name\\
%    |#5|& new value
%    \end{tabular}
%    \end{quote}
%    \begin{macrocode}
\def\KVO@boolkey#1#2#3#4#5{%
  \edef\KVO@param{#5}%
  \ltx@onelevel@sanitize\KVO@param
  \ifx\KVO@param\KVO@true
    \expandafter\@firstofone
  \else
    \ifx\KVO@param\KVO@false
      \expandafter\expandafter\expandafter\@firstofone
    \else
      \ifx#2\@clsextension
        \expandafter\ClassWarning
      \else
        \expandafter\PackageWarning
      \fi
      {#1}{%
        Value `\KVO@param' is not supported by\MessageBreak
        option `#4'%
      }%
      \expandafter\expandafter\expandafter\@gobble
    \fi
  \fi
  {%
    ^^A\ifx#2\@clsextension
    ^^A  \expandafter\ClassInfo
    ^^A\else
    ^^A  \expandafter\PackageInfo
    ^^A\fi
    ^^A{#1}{[option] #4=\KVO@param}%
    \csname#3#4\KVO@param\endcsname
  }%
}
%    \end{macrocode}
%    \end{macro}
%    \begin{macro}{\KVO@true}
%    \begin{macro}{\KVO@false}
%    The macros \cs{KVO@true} and \cs{KVO@false} are used
%    for string comparisons. After \cs{ltx@onelevel@sanitize}
%    we have only tokens with catcode 12 (other).
%    \begin{macrocode}
\def\KVO@true{true}
\def\KVO@false{false}
\ltx@onelevel@sanitize\KVO@true
\ltx@onelevel@sanitize\KVO@false
%    \end{macrocode}
%    \end{macro}
%    \end{macro}
%
% \subsubsection{\cs{DeclareStringOption}}
%
%    \begin{macro}{\DeclareStringOption}
%    \begin{macrocode}
\newcommand*{\DeclareStringOption}[2][]{%
  \@ifnextchar[{%
    \KVO@DeclareStringOption{#1}{#2}@%
  }{%
    \KVO@DeclareStringOption{#1}{#2}{}[]%
  }%
}
%    \end{macrocode}
%    \end{macro}
%    \begin{macro}{\KVO@DeclareStringOption}
%    \begin{macrocode}
\def\KVO@DeclareStringOption#1#2#3[#4]{%
  \KVO@ifdefinable{\KVO@prefix#2}{%
    \@namedef{\KVO@prefix#2}{#1}%
    \begingroup
      \ifx\\#3\\%
        \toks@{}%
      \else
        \toks@{[{#4}]}%
      \fi
      \edef\x{\endgroup
        \noexpand\define@key{\KVO@family}{#2}\the\toks@{%
          ^^A\begingroup
          ^^A  \toks@{####1}%
          ^^A  \ifx\@currext\@clsextension
          ^^A    \noexpand\ClassInfo
          ^^A  \else
          ^^A    \noexpand\PackageInfo
          ^^A  \fi
          ^^A  {\@currname}{%
          ^^A    [option] #2={\noexpand\the\toks@}%
          ^^A  }%
          ^^A\endgroup
          \noexpand\def
          \expandafter\noexpand\csname\KVO@prefix#2\endcsname{####1}%
        }%
      }%
    \x
  }%
}
%    \end{macrocode}
%    \end{macro}
%
% \subsubsection{\cs{DeclareVoidOption}}
%
%    \begin{macro}{\DeclareVoidOption}
%    \begin{macrocode}
\newcommand*{\DeclareVoidOption}[2]{%
  \begingroup
    \let\next\@gobbletwo
    \KVO@ifdefinable{\KVO@prefix#1}{%
      \let\next\@firstofone
    }%
  \expandafter\endgroup
  \next{%
    \begingroup
      \edef\x{\endgroup
        \noexpand\define@key{\KVO@family}{#1}[\KVO@VOID@]{%
          \noexpand\KVO@voidkey{\@currname}%
          \ifx\@currext\@clsextension
            \noexpand\@clsextension
          \else
            \noexpand\@pkgextension
          \fi
          {#1}%
          {####1}%
          \expandafter\noexpand\csname\KVO@prefix#1\endcsname
        }%
      }%
    \x
    \begingroup
      \toks@{#2}%
    \expandafter\endgroup
    \expandafter\def
    \csname\KVO@prefix#1\expandafter\endcsname
    \expandafter{\the\toks@}%
  }%
}
\def\KVO@VOID@{@VOID@}
%    \end{macrocode}
%    \end{macro}
%    \begin{macro}{\KVO@voidkey}
%    \begin{tabular}{@{}ll@{}}
%    |#1|& package/class name\\
%    |#2|& |\@pkgextension|/|\@clsextension|\\
%    |#3|& key name\\
%    |#4|& default (|@VOID@|)\\
%    |#5|& macro with option code
%    \end{tabular}
%    \begin{macrocode}
\def\KVO@voidkey#1#2#3#4{%
  \def\CurrentOption{#3}%
  \begingroup
    \def\x{#4}%
  \expandafter\endgroup
  \ifx\x\KVO@VOID@
  \else
    \ifx#2\@clsextension
      \expandafter\ClassWarning
    \else
      \expandafter\PackageWarning
    \fi
    {#1}{%
      Unexpected value for option `#3'\MessageBreak
      is ignored%
    }%
  \fi
  ^^A\ifx#2\@clsextension
  ^^A  \expandafter\ClassInfo
  ^^A\else
  ^^A  \expandafter\PackageInfo
  ^^A\fi
  ^^A{#1}{[option] #3}%
}
%    \end{macrocode}
%    \end{macro}
%
% \subsubsection{\cs{DeclareDefaultOption}}
%
%    \begin{macro}{\DeclareDefaultOption}
%    \begin{macrocode}
\newcommand*{\DeclareDefaultOption}{%
  \@namedef{KVO@default@\@currname.\@currext}%
}
%    \end{macrocode}
%    \end{macro}
%
% \subsubsection{\cs{DeclareLocalOptions}}
%
%    \begin{macro}{\DeclareLocalOptions}
%    \begin{macrocode}
\newcommand*{\DeclareLocalOptions}[1]{%
  \comma@parse{#1}\KVO@DeclareLocalOption
}
%    \end{macrocode}
%    \end{macro}
%    \begin{macro}{\KVO@DeclareLocalOption}
%    \begin{macrocode}
\def\KVO@DeclareLocalOption#1{%
  \expandafter\def\csname KVO@local@\KVO@family @#1\endcsname{}%
}
%    \end{macrocode}
%    \end{macro}
%
% \subsection{Dynamic options}
%
% \subsubsection{\cs{DisableKeyvalOption}}
%
%    \begin{macrocode}
\SetupKeyvalOptions{%
  family=KVOdyn,%
  prefix=KVOdyn@%
}
\DeclareBoolOption[true]{global}
\DeclareComplementaryOption{local}{global}
\DeclareStringOption[undef]{action}
\let\KVOdyn@name\relax
\let\KVOdyn@ext\@empty
\define@key{KVOdyn}{class}{%
  \def\KVOdyn@name{#1}%
  \let\KVOdyn@ext\@clsextension
}
\define@key{KVOdyn}{package}{%
  \def\KVOdyn@name{#1}%
  \let\KVOdyn@ext\@pkgextension
}
\newcommand*{\DisableKeyvalOption}[3][]{%
  \begingroup
    \kvsetkeys{KVOdyn}{#1}%
    \def\x{\endgroup}%
    \@ifundefined{KVO@action@\KVOdyn@action}{%
      \PackageError{kvoptions}{%
        Unknown disable action %
        `\expandafter\strip@prefix\meaning\KVOdyn@action'\MessageBreak
        for option `#3' in keyval family '#2'%
      }\@ehc
    }{%
      \csname KVO@action@\KVOdyn@action\endcsname{#2}{#3}%
    }%
  \x
}
\def\KVO@action@undef#1#2{%
  \edef\x{\endgroup
    \ifKVOdyn@global\global\fi
    \let
    \expandafter\noexpand\csname KV@#1@#2\endcsname
    \relax
    \ifKVOdyn@global\global\fi
    \let
    \expandafter\noexpand\csname KV@#1@#2@default\endcsname
    \relax
  }%
  ^^A\PackageInfo{kvoptions}{%
  ^^A  [option] key `#2' of family `#1'\MessageBreak
  ^^A  is disabled (undef, \ifKVOdyn@global global\else local\fi)%
  ^^A}%
}
\def\KVO@action@ignore#1#2{%
  \edef\x{\endgroup
    \ifKVOdyn@global\global\fi
    \let
    \expandafter\noexpand\csname KV@#1@#2\endcsname
    \noexpand\@gobble
    \ifKVOdyn@global\global\fi
    \let
    \expandafter\noexpand\csname KV@#1@#2@default\endcsname
    \noexpand\@empty
  }%
  ^^A\PackageInfo{kvoptions}{%
  ^^A  [option] key `#2' of family `#1'\MessageBreak
  ^^A  is disabled (ignore, \ifKVOdyn@global global\else local\fi)%
  ^^A}%
}
\def\KVO@action@error{%
  \KVO@do@action{error}%
}
\def\KVO@action@warning{%
  \KVO@do@action{warning}%
}
%    \end{macrocode}
%    \begin{tabular}{@{}ll@{}}
%    |#1|& |error| or |warning|\\
%    |#2|& \meta{family}\\
%    |#3|& \meta{key}\\
%    \end{tabular}
%    \begin{macrocode}
\def\KVO@do@action#1#2#3{%
  \ifx\KVOdyn@name\relax
    \PackageError{kvoptions}{%
      Action type `#1' needs package/class name\MessageBreak
      for key `#3' in family `#2'%
    }\@ehc
  \else
    \edef\x{\endgroup
      \noexpand\define@key{#2}{#3}[]{%
        \expandafter\noexpand\csname KVO@disable@#1\endcsname
        {\KVOdyn@name}\noexpand\KVOdyn@ext{#3}%
      }%
      \ifKVOdyn@global
        \global\let
        \expandafter\noexpand\csname KV@#2@#3\endcsname
        \expandafter\noexpand\csname KV@#2@#3\endcsname
        \global\let
        \expandafter\noexpand\csname KV@#2@#3@default\endcsname
        \expandafter\noexpand\csname KV@#2@#3@default\endcsname
      \fi
    }%
    ^^A\ifx\KVOdyn@ext\@clsextension
    ^^A  \expandafter\ClassInfo
    ^^A\else
    ^^A   \expandafter\PackageInfo
    ^^A\fi
    ^^A{\KVOdyn@name}{%
    ^^A  [option] key `#3' of family `#2'\MessageBreak
    ^^A  is disabled (#1, \ifKVOdyn@global global\else local\fi)%
    ^^A}%
  \fi
}
\def\KVO@disable@error#1#2#3{%
  \ifx#2\@clsextension
    \expandafter\ClassError
  \else
    \expandafter\PackageError
  \fi
  {#1}{%
    Option `#3' is given too late,\MessageBreak
    now the option is ignored%
  }\@ehc
}
\def\KVO@disable@warning#1#2#3{%
  \ifx#2\@clsextension
    \expandafter\ClassWarning
  \else
    \expandafter\PackageWarning
  \fi
  {#1}{%
    Option `#3' is already consumed\MessageBreak
    and has no effect%
  }%
}
%    \end{macrocode}
%
% \subsection{Change option code}
%
% \subsubsection{\cs{AddToKeyvalOption}}
%
%    \begin{macro}{\AddToKeyvalOption}
%    \begin{macrocode}
\newcommand*{\AddToKeyvalOption}{%
  \@ifstar{%
    \begingroup
      \edef\x{\endgroup
        \noexpand\KVO@AddToKeyvalOption{\KVO@family}%
      }%
    \x
  }%
  \KVO@AddToKeyvalOption
}
%    \end{macrocode}
%    \end{macro}
%    \begin{macro}{\KVO@AddToKeyvalOption}
%    \begin{macrocode}
\def\KVO@AddToKeyvalOption#1#2{%
  \@ifundefined{KV@#1@#2}{%
    \PackageWarning{kvoptions}{%
      Key `#2' of family `#1' does not exist.\MessageBreak
      Ignoring \string\AddToKeyvalOption
    }%
    \@gobble
  }{%
    \edef\KVO@next{%
      \noexpand\KVO@@AddToKeyvalOption
      \expandafter\noexpand\csname KV@#1@#2\endcsname
    }%
    \afterassignment\KVO@next
    \def\KVO@temp##1%
  }%
}
%    \end{macrocode}
%    \end{macro}
%    \begin{macro}{\KVO@@AddToKeyvalOption}
%    \begin{macrocode}
\def\KVO@@AddToKeyvalOption#1{%
  \begingroup
    \toks@\expandafter{#1{##1}}%
    \toks@\expandafter{\the\expandafter\toks@\KVO@temp{##1}}%
    \edef\x{\endgroup
      \noexpand\def\noexpand#1####1{\the\toks@}%
    }%
  \x
}
%    \end{macrocode}
%    \end{macro}
%
% \subsection{Process options}
%
% \subsubsection{Get global options}
%
%    Package \xpackage{xkeyval} removes options with equal signs from
%    the global options (\cs{@classoptionslist}). The effect is that
%    other packages and classes will not see these global options anymore.
%    A bug-report was answered that this behaviour is ``by design''.
%    Thus I call it a design bug. Now getting the global options require
%    an algorithm instead of a simple macro call.
%    \begin{macrocode}
%</package>
%<*package|patch>
%    \end{macrocode}
%    \begin{macro}{\KVO@IfDefThen}
%    Call |#2| if command |#1| is defined and not \cs{relax}.
%    (Package \xpackage{kvoptions-patch} does not load
%    package \xpackage{ltxcmds}.)
%    \begin{macrocode}
\def\KVO@IfDefThen#1#2{%
  \ifx#1\ltx@undefined
  \else
    \ifx#1\relax
    \else
      #2%
    \fi
  \fi
}%
%    \end{macrocode}
%    \end{macro}
%    \begin{macro}{\KVO@GetClassOptionsList}
%    \begin{macrocode}
\def\KVO@GetClassOptionsList{%
  \let\KVO@classoptionslist\@classoptionslist
  \KVO@IfDefThen\@classoptionslist{%
    \KVO@IfDefThen\XKV@documentclass{%
      \ifx\XKV@documentclass\ltx@empty
      \else
        \KVO@IfDefThen\XKV@classoptionslist{%
          \ifx\XKV@classoptionslist\ltx@empty
          \else
            \let\KVO@classoptionslist\XKV@classoptionslist
          \fi
        }%
      \fi
    }%
  }%
}%
%    \end{macrocode}
%    \end{macro}
%    \begin{macrocode}
%</package|patch>
%<*package>
%    \end{macrocode}
%
% \subsubsection{\cs{ProcessKeyvalOptions}}
%
%    \begin{macro}{\ProcessKeyvalOptions}
%    If the optional star is given, we get the family
%    name and expand it for safety.
%    \begin{macrocode}
\newcommand*{\ProcessKeyvalOptions}{%
  \@ifstar{%
    \begingroup
      \edef\x{\endgroup
        \noexpand\KVO@ProcessKeyvalOptions{\KVO@family}%
      }%
    \x
  }%
  \KVO@ProcessKeyvalOptions
}
%    \end{macrocode}
%    \end{macro}
%
%    \begin{macrocode}
\def\KVO@ProcessKeyvalOptions#1{%
  \let\@tempc\relax
  \let\KVO@temp\@empty
%    \end{macrocode}
%    Add any global options that are known to KV to the start of
%    the list being built in |\KVO@temp| and mark them used (by
%    removing them from the unused option list).
%    \begin{macrocode}
  \ifx\@currext\@clsextension
  \else
    \KVO@GetClassOptionsList
    \ifx\KVO@classoptionslist\relax
    \else
      \@for\KVO@CurrentOption:=\KVO@classoptionslist\do{%
        \@ifundefined{KV@#1@\expandafter\KVO@getkey
                      \KVO@CurrentOption=\@nil}{%
        }{%
          \@ifundefined{KVO@local@#1@\expandafter\KVO@getkey
                        \KVO@CurrentOption=\@nil}{%
            \ifx\KVO@Patch Y%
              \edef\KVO@temp{%
                \etex@unexpanded\expandafter{%
                  \KVO@temp
                }%
                ,%
                \etex@unexpanded\expandafter{%
                  \KVO@CurrentOption
                }%
                ,%
              }%
              \ltx@onelevel@sanitize\KVO@CurrentOption
            \else
              \edef\KVO@temp{%
                \KVO@temp
                ,%
                \KVO@CurrentOption
                ,%
              }%
            \fi
            \@expandtwoargs\@removeelement\KVO@CurrentOption
              \@unusedoptionlist\@unusedoptionlist
          }{}%
        }%
      }%
    \fi
  \fi
%    \end{macrocode}
%     Now stick the package options at the end of the list and wrap
%     in a call to \cs{setkeys}. A class ignores unknown global
%     options, we must remove them to prevent error messages
%     from \cs{setkeys}.
%    \begin{macrocode}
  \begingroup
    \toks\tw@{}%
    \@ifundefined{opt@\@currname.\@currext}{%
      \toks@\expandafter{\KVO@temp}%
    }{%
      \toks@\expandafter\expandafter\expandafter{%
        \csname opt@\@currname.\@currext\endcsname
      }%
      \ifx\@currext\@clsextension
        \edef\CurrentOption{\the\toks@}%
        \toks@\expandafter{\KVO@temp}%
        \@for\CurrentOption:=\CurrentOption\do{%
          \@ifundefined{%
            KV@#1@\expandafter\KVO@getkey\CurrentOption=\@nil
          }{%
%    \end{macrocode}
%    A class puts not used options in the unused option list
%    unless there is a default handler.
%    \begin{macrocode}
            \@ifundefined{KVO@default@\@currname.\@currext}{%
              \ifx\KVO@Patch Y%
                \ltx@onelevel@sanitize\CurrentOption
              \fi
              \ifx\@unusedoptionlist\@empty
                \global\let\@unusedoptionlist\CurrentOption
              \else
                \expandafter\expandafter\expandafter\gdef
                \expandafter\expandafter\expandafter\@unusedoptionlist
                \expandafter\expandafter\expandafter{%
                  \expandafter\@unusedoptionlist
                  \expandafter,\CurrentOption
                }%
              \fi
            }{%
              \toks\tw@\expandafter{%
                \the\toks\expandafter\tw@\expandafter,\CurrentOption
              }%
            }%
          }{%
            \toks@\expandafter{%
              \the\expandafter\toks@\expandafter,\CurrentOption
            }%
          }%
        }%
      \else
%    \end{macrocode}
%    Without default action we pass all options to \cs{setkeys}.
%    Otherwise we have to check which options are known.
%    These are passed to \cs{setkeys}. For the others the default
%    action is performed.
%    \begin{macrocode}
        \@ifundefined{KVO@default@\@currname.\@currext}{%
          \toks@\expandafter\expandafter\expandafter{%
            \expandafter\KVO@temp\the\toks@
          }%
        }{%
          \edef\CurrentOption{\the\toks@}%
          \toks@\expandafter{\KVO@temp}%
          \@for\CurrentOption:=\CurrentOption\do{%
            \@ifundefined{%
              KV@#1@\expandafter\KVO@getkey\CurrentOption=\@nil
            }{%
              \toks\tw@\expandafter{%
                \the\toks\expandafter\tw@\expandafter,\CurrentOption
              }%
            }{%
              \toks@\expandafter{%
                \the\expandafter\toks@\expandafter,\CurrentOption
              }%
            }%
          }%
        }%
      \fi
    }%
    \edef\KVO@temp{\endgroup
      \noexpand\KVO@calldefault{\the\toks\tw@}%
      \noexpand\KVO@setkeys{#1}{\the\toks@}%
    }%
  \KVO@temp
%    \end{macrocode}
%    Some cleanup of \cs{ProcessOptions}.
%    \begin{macrocode}
  \let\CurrentOption\@empty
  \AtEndOfPackage{\let\@unprocessedoptions\relax}%
}
%    \end{macrocode}
%
% \subsubsection{\cs{ProcessLocalKeyvalOptions}}
%
%    \begin{macro}{\ProcessLocalKeyvalOptions}
%    If the optional star is given, we get the family
%    name and expand it for safety.
%    \begin{macrocode}
\newcommand*{\ProcessLocalKeyvalOptions}{%
  \@ifstar{%
    \begingroup
      \edef\x{\endgroup
        \noexpand\KVO@ProcessLocalKeyvalOptions{\KVO@family}%
      }%
    \x
  }%
  \KVO@ProcessLocalKeyvalOptions
}
%    \end{macrocode}
%    \end{macro}
%
%    \begin{macrocode}
\def\KVO@ProcessLocalKeyvalOptions#1{%
  \let\@tempc\relax
  \let\KVO@temp\@empty
%    \end{macrocode}
%    Check if \cs{ProcessLocalKeyvalOptions} is called inside
%    a package.
%    \begin{macrocode}
  \ifx\@currext\@pkgextension
  \else
    \PackageError{kvoptions}{%
      \string\ProcessLocalKeyvalOptions\space is intended for packages only%
    }\@ehc
  \fi
%    \end{macrocode}
%     The package options are put into toks register \cs{toks@}.
%    \begin{macrocode}
  \begingroup
    \toks\tw@{}%
    \@ifundefined{opt@\@currname.\@currext}{%
      \toks@\expandafter{\KVO@temp}%
    }{%
      \toks@\expandafter\expandafter\expandafter{%
        \csname opt@\@currname.\@currext\endcsname
      }%
%    \end{macrocode}
%    Without default action we pass all options to \cs{setkeys}.
%    Otherwise we have to check which options are known.
%    These are passed to \cs{setkeys}. For the others the default
%    action is performed.
%    \begin{macrocode}
      \@ifundefined{KVO@default@\@currname.\@currext}{%
        \toks@\expandafter\expandafter\expandafter{%
          \expandafter\KVO@temp\the\toks@
        }%
      }{%
        \edef\CurrentOption{\the\toks@}%
        \toks@\expandafter{\KVO@temp}%
        \@for\CurrentOption:=\CurrentOption\do{%
          \@ifundefined{%
            KV@#1@\expandafter\KVO@getkey\CurrentOption=\@nil
          }{%
            \toks\tw@\expandafter{%
              \the\toks\expandafter\tw@\expandafter,\CurrentOption
            }%
          }{%
            \toks@\expandafter{%
              \the\expandafter\toks@\expandafter,\CurrentOption
            }%
          }%
        }%
      }%
    }%
    \edef\KVO@temp{\endgroup
      \noexpand\KVO@calldefault{\the\toks\tw@}%
      \noexpand\KVO@setkeys{#1}{\the\toks@}%
    }%
  \KVO@temp
%    \end{macrocode}
%    Some cleanup of \cs{ProcessOptions}.
%    \begin{macrocode}
  \let\CurrentOption\@empty
  \AtEndOfPackage{\let\@unprocessedoptions\relax}%
}
%    \end{macrocode}
%
% \subsubsection{Helper macros}
%
%    \begin{macro}{\KVO@getkey}
%    Extract the key part of a key=value pair.
%    \begin{macrocode}
\def\KVO@getkey#1=#2\@nil{#1}
%    \end{macrocode}
%    \end{macro}
%
%    \begin{macro}{\KVO@calldefault}
%    \begin{macrocode}
\def\KVO@calldefault#1{%
  \begingroup
    \def\x{#1}%
  \expandafter\endgroup
  \ifx\x\@empty
  \else
    \@for\CurrentOption:=#1\do{%
      \ifx\CurrentOption\@empty
      \else
        \expandafter\KVO@setcurrents\CurrentOption=\@nil
        \@nameuse{KVO@default@\@currname.\@currext}%
      \fi
    }%
  \fi
}
%    \end{macrocode}
%    \end{macro}
%    \begin{macro}{\KVO@setcurrents}
%    Extract the key part of a key=value pair.
%    \begin{macrocode}
\def\KVO@setcurrents#1=#2\@nil{%
  \def\CurrentOptionValue{#2}%
  \ifx\CurrentOptionValue\@empty
    \let\CurrentOptionKey\CurrentOption
    \let\CurrentOptionValue\relax
  \else
    \edef\CurrentOptionKey{\zap@space#1 \@empty}%
    \expandafter\KVO@setcurrentvalue\CurrentOption\@nil
  \fi
}
%    \end{macrocode}
%    \end{macro}
%    \begin{macro}{\KV@setcurrentvalue}
%    Here the value part is parsed. Package \xpackage{keyval}'s
%    \cs{KV@@sp@def} helps in removing spaces at the begin and
%    end of the value.
%    \begin{macrocode}
\def\KVO@setcurrentvalue#1=#2\@nil{%
  \KV@@sp@def\CurrentOptionValue{#2}%
}
%    \end{macrocode}
%    \end{macro}
%
% \subsection{\plainTeX}
%
%    Disable \LaTeX\ stuff.
%    \begin{macrocode}
\begingroup\expandafter\expandafter\expandafter\endgroup
\expandafter\ifx\csname documentclass\endcsname\relax
  \def\ProcessKeyvalOptions{%
    \@ifstar{}\@gobble
  }%
\fi
%    \end{macrocode}
%
%    \begin{macrocode}
\KVO@AtEnd%
%</package>
%    \end{macrocode}
%
% \subsection{Package \xpackage{kvoptions-patch}}
%
%    \begin{macrocode}
%<*patch>
\NeedsTeXFormat{LaTeX2e}
\begingroup\catcode61\catcode48\catcode32=10\relax%
  \catcode13=5 % ^^M
  \endlinechar=13 %
  \catcode123=1 % {
  \catcode125=2 % }
  \catcode64=11 % @
  \def\x{\endgroup
    \expandafter\edef\csname KVO@AtEnd\endcsname{%
      \endlinechar=\the\endlinechar\relax
      \catcode13=\the\catcode13\relax
      \catcode32=\the\catcode32\relax
      \catcode35=\the\catcode35\relax
      \catcode61=\the\catcode61\relax
      \catcode64=\the\catcode64\relax
      \catcode123=\the\catcode123\relax
      \catcode125=\the\catcode125\relax
    }%
  }%
\x\catcode61\catcode48\catcode32=10\relax%
\catcode13=5 % ^^M
\endlinechar=13 %
\catcode35=6 % #
\catcode64=11 % @
\catcode123=1 % {
\catcode125=2 % }
\def\TMP@EnsureCode#1#2{%
  \edef\KVO@AtEnd{%
    \KVO@AtEnd
    \catcode#1=\the\catcode#1\relax
  }%
  \catcode#1=#2\relax
}
\TMP@EnsureCode{39}{12}% '
\TMP@EnsureCode{40}{12}% (
\TMP@EnsureCode{41}{12}% )
\TMP@EnsureCode{43}{12}% +
\TMP@EnsureCode{44}{12}% ,
\TMP@EnsureCode{45}{12}% -
\TMP@EnsureCode{46}{12}% .
\TMP@EnsureCode{47}{12}% /
\TMP@EnsureCode{58}{12}% :
\TMP@EnsureCode{60}{12}% <
\TMP@EnsureCode{62}{12}% >
\TMP@EnsureCode{91}{12}% [
\TMP@EnsureCode{93}{12}% ]
\TMP@EnsureCode{96}{12}% `
\TMP@EnsureCode{124}{12}% |
\edef\KVO@AtEnd{\KVO@AtEnd\noexpand\endinput}
\ProvidesPackage{kvoptions-patch}%
  [2016/05/16 v3.12 LaTeX patch for keyval options (HO)]%
%    \end{macrocode}
%
%    Check for \eTeX.
%    \begin{macrocode}
\begingroup\expandafter\expandafter\expandafter\endgroup
\expandafter\ifx\csname eTeXversion\endcsname\relax
  \PackageWarningNoLine{kvoptions-patch}{%
    Package loading is aborted, because e-TeX is missing%
  }%
  \expandafter\KVO@AtEnd
\fi%
%    \end{macrocode}
%
%    Package \xpackage{etexcmds} for \cs{etex@unexpanded}.
%    \begin{macrocode}
\RequirePackage{etexcmds}[2007/09/09]
\ifetex@unexpanded
\else
  \PackageError{kvoptions-patch}{%
    Could not find eTeX's \string\unexpanded.\MessageBreak
    Try adding \string\RequirePackage\string{etexcmds\string} %
    before \string\documentclass%
  }\@ehd
  \expandafter\KVO@AtEnd
\fi%
%    \end{macrocode}
%
%    Check for package \xpackage{xkvltxp}.
%    \begin{macrocode}
\@ifpackageloaded{xkvltxp}{%
  \PackageWarningNoLine{kvoptions}{%
    Option `patch' cannot be used together with\MessageBreak
    package `xkvltxp' that is already loaded.\MessageBreak
    Therefore package loading is aborted%
  }%
  \KVO@AtEnd
}{}%
%    \end{macrocode}
%
%    \begin{macrocode}
\def\@if@ptions#1#2#3{%
  \begingroup
    \KVO@normalize\KVO@temp{#3}%
    \edef\x{\endgroup
      \noexpand\@if@pti@ns{%
        \detokenize\expandafter\expandafter\expandafter{%
          \csname opt@#2.#1\endcsname
        }%
      }{%
        \detokenize\expandafter{\KVO@temp}%
      }%
    }%
  \x
}
%    \end{macrocode}
%
%    \begin{macrocode}
\def\@pass@ptions#1#2#3{%
  \KVO@normalize\KVO@temp{#2}%
  \@ifundefined{opt@#3.#1}{%
    \expandafter\gdef\csname opt@#3.#1%
          \expandafter\endcsname\expandafter{%
      \KVO@temp
    }%
  }{%
    \expandafter\gdef\csname opt@#3.#1%
          \expandafter\expandafter\expandafter\endcsname
          \expandafter\expandafter\expandafter{%
      \csname opt@#3.#1\expandafter\endcsname\expandafter,\KVO@temp
    }%
  }%
}
%    \end{macrocode}
%
%    \begin{macrocode}
\def\ProcessOptions{%
  \let\ds@\@empty
  \@ifundefined{opt@\@currname.\@currext}{%
    \let\@curroptions\@empty
  }{%
    \expandafter\expandafter\expandafter\def
    \expandafter\expandafter\expandafter\@curroptions
    \expandafter\expandafter\expandafter{%
      \csname opt@\@currname.\@currext\endcsname
    }%
  }%
  \@ifstar\KVO@xprocess@ptions\KVO@process@ptions
}
%    \end{macrocode}
%
%    \begin{macrocode}
\def\KVO@process@ptions{%
  \@for\CurrentOption:=\@declaredoptions\do{%
    \ifx\CurrentOption\@empty
    \else
      \begingroup
        \ifx\@currext\@clsextension
          \toks@{}%
        \else
          \KVO@GetClassOptionsList
          \toks@\expandafter{\KVO@classoptionslist,}%
        \fi
        \toks\tw@\expandafter{\@curroptions}%
        \edef\x{\endgroup
          \noexpand\in@{,\CurrentOption,}{,\the\toks@\the\toks\tw@,}%
        }%
      \x
      \ifin@
        \KVO@use@ption
        \expandafter\let\csname ds@\CurrentOption\endcsname\@empty
      \fi
    \fi
  }%
  \KVO@process@pti@ns
}
%    \end{macrocode}
%
%    \begin{macrocode}
\def\KVO@xprocess@ptions{%
  \ifx\@currext\@clsextension
  \else
    \KVO@GetClassOptionsList
    \@for\CurrentOption:=\KVO@classoptionslist\do{%
      \ifx\CurrentOption\@empty
      \else
        \KVO@in@\CurrentOption\@declaredoptions
        \ifin@
          \KVO@use@ption
          \expandafter\let\csname ds@\CurrentOption\endcsname\@empty
        \fi
      \fi
    }%
  \fi
  \KVO@process@pti@ns
}
%    \end{macrocode}
%
%    \begin{macrocode}
\def\KVO@in@#1#2{%
  \in@false
  \begingroup
    \@for\x:=#2\do{%
      \ifx\x#1\relax
        \in@true
      \fi
    }%
    \edef\x{\endgroup
      \ifin@
        \noexpand\in@true
      \fi
    }%
  \x
}
%    \end{macrocode}
%
%    \begin{macrocode}
\def\KVO@process@pti@ns{%
  \@for\CurrentOption:=\@curroptions\do{%
    \@ifundefined{ds@\KVO@SanitizedCurrentOption}{%
      \KVO@use@ption
      \default@ds
    }%
    \KVO@use@ption
  }%
  \@for\CurrentOption:=\@declaredoptions\do{%
    \expandafter\let\csname ds@\CurrentOption\endcsname\relax
  }%
  \let\CurrentOption\@empty
  \let\@fileswith@pti@ns\@@fileswith@pti@ns
  \AtEndOfPackage{\let\@unprocessedoptions\relax}%
}
%    \end{macrocode}
%
%    \begin{macrocode}
\def\KVO@use@ption{%
  \begingroup
    \edef\x{\endgroup
      \noexpand\@removeelement{%
        \detokenize\expandafter{\CurrentOption}%
      }{%
        \detokenize\expandafter{\@unusedoptionlist}%
      }%
    }%
  \x\@unusedoptionlist
  \csname ds@\KVO@SanitizedCurrentOption\endcsname
}
%    \end{macrocode}
%
%    \begin{macrocode}
\def\OptionNotUsed{%
  \ifx\@currext\@clsextension
    \xdef\@unusedoptionlist{%
      \ifx\@unusedoptionlist\@empty
      \else
        \detokenize\expandafter{\@unusedoptionlist,}%
      \fi
      \detokenize\expandafter{\CurrentOption}%
    }%
  \fi
}
%    \end{macrocode}
%
%    Variant of \cs{ExecuteOptions} that better protects \cs{CurrentOption}.
%    \begin{macrocode}
\def\CurrentOption@SaveLevel{0}
\def\ExecuteOptions{%
  \expandafter\KVO@ExecuteOptions
      \csname CurrentOption@\CurrentOption@SaveLevel\endcsname
}
\def\KVO@ExecuteOptions#1#2{%
  \let#1\CurrentOption
  \edef\CurrentOption@SaveLevel{%
    \the\numexpr\CurrentOption@SaveLevel+1%
  }%
  \@for\CurrentOption:=#2\do{%
    \csname ds@\CurrentOption\endcsname
  }%
  \edef\CurrentOption@SaveLevel{%
    \the\numexpr\CurrentOption@SaveLevel-1%
  }%
  \let\CurrentOption#1%
}
%    \end{macrocode}
%
%    \begin{macrocode}
\def\KVO@fileswith@pti@ns#1[#2]#3[#4]{%
  \ifx#1\@clsextension
    \ifx\@classoptionslist\relax
      \KVO@normalize\KVO@temp{#2}%
      \expandafter\gdef\expandafter\@classoptionslist\expandafter{%
        \KVO@temp
      }%
      \def\reserved@a{%
        \KVO@onefilewithoptions{#3}[{#2}][{#4}]#1%
        \@documentclasshook
      }%
    \else
      \def\reserved@a{%
        \KVO@onefilewithoptions{#3}[{#2}][{#4}]#1%
      }%
    \fi
  \else
    \begingroup
      \let\KVO@temp\relax
      \let\KVO@onefilewithoptions\relax
      \let\@pkgextension\relax
      \def\reserved@b##1,{%
        \ifx\@nil##1\relax
        \else
          \ifx\relax##1\relax
          \else
            \KVO@onefilewithoptions{##1}[{\KVO@temp}][{#4}]%
            \@pkgextension
          \fi
          \expandafter\reserved@b
        \fi
      }%
      \edef\reserved@a{\zap@space#3 \@empty}%
      \edef\reserved@a{\expandafter\reserved@b\reserved@a,\@nil,}%
      \toks@{#2}%
      \def\KVO@temp{\the\toks@}%
    \edef\reserved@a{\endgroup \reserved@a}%
  \fi
  \reserved@a
}
%    \end{macrocode}
%
%    \begin{macrocode}
\def\KVO@onefilewithoptions#1[#2][#3]#4{%
  \@pushfilename
  \xdef\@currname{#1}%
  \global\let\@currext#4%
  \expandafter\let\csname\@currname.\@currext-h@@k\endcsname\@empty
  \let\CurrentOption\@empty
  \@reset@ptions
  \makeatletter
  \def\reserved@a{%
    \@ifl@aded\@currext{#1}{%
      \@if@ptions\@currext{#1}{#2}{%
      }{%
        \begingroup
          \@ifundefined{opt@#1.\@currext}{%
            \def\x{}%
          }{%
            \edef\x{%
              \expandafter\expandafter\expandafter\strip@prefix
              \expandafter\meaning\csname opt@#1.\@currext\endcsname
            }%
          }%
          \def\y{#2}%
          \edef\y{\expandafter\strip@prefix\meaning\y}%
          \@latex@error{Option clash for \@cls@pkg\space #1}{%
            The package #1 has already been loaded %
            with options:\MessageBreak
            \space\space[\x]\MessageBreak
            There has now been an attempt to load it %
             with options\MessageBreak
            \space\space[\y]\MessageBreak
            Adding the global options:\MessageBreak
            \space\space
                 \x,\y\MessageBreak
            to your \noexpand\documentclass declaration may fix this.%
            \MessageBreak
            Try typing \space <return> \space to proceed.%
          }%
        \endgroup
      }%
    }{%
      \@pass@ptions\@currext{#2}{#1}%
      \global\expandafter
      \let\csname ver@\@currname.\@currext\endcsname\@empty
      \InputIfFileExists
        {\@currname.\@currext}%
        {}%
        {\@missingfileerror\@currname\@currext}%
      \let\@unprocessedoptions\@@unprocessedoptions
      \csname\@currname.\@currext-h@@k\endcsname
      \expandafter\let\csname\@currname.\@currext-h@@k\endcsname
              \@undefined
      \@unprocessedoptions
    }%
    \@ifl@ter\@currext{#1}{#3}{%
    }{%
      \@latex@warning@no@line{%
        You have requested,\on@line, %
        version\MessageBreak
          #3' of \@cls@pkg\space #1,\MessageBreak
        but only version\MessageBreak
         `\csname ver@#1.\@currext\endcsname'\MessageBreak
        is available%
      }%
    }%
    \ifx\@currext\@clsextension\let\LoadClass\@twoloadclasserror\fi
    \@popfilename
    \@reset@ptions
  }%
  \reserved@a
}
%    \end{macrocode}
%
%    \begin{macrocode}
\def\@unknownoptionerror{%
  \@latex@error{%
    Unknown option `\KVO@SanitizedCurrentOption' %
    for \@cls@pkg\space`\@currname'%
  }{%
    The option `\KVO@SanitizedCurrentOption' was not declared in %
    \@cls@pkg\space`\@currname', perhaps you\MessageBreak
    misspelled its name. %
    Try typing \space <return> %
    \space to proceed.%
  }%
}
%    \end{macrocode}
%
%    \begin{macrocode}
\def\@@unprocessedoptions{%
  \ifx\@currext\@pkgextension
    \@ifundefined{opt@\@currname.\@currext}{%
      \let\@curroptions\@empty
    }{%
      \expandafter\let\expandafter\@curroptions
          \csname opt@\@currname.\@currext\endcsname
    }%
    \@for\CurrentOption:=\@curroptions\do{%
        \ifx\CurrentOption\@empty\else\@unknownoptionerror\fi
    }%
  \fi
}
%    \end{macrocode}
%
%    \begin{macrocode}
\def\KVO@SanitizedCurrentOption{%
  \expandafter\strip@prefix\meaning\CurrentOption
}
%    \end{macrocode}
%
%    Normalize option list.
%    \begin{macrocode}
\def\KVO@normalize#1#2{%
  \let\KVO@result\@empty
  \KVO@splitcomma#2,\@nil
  \let#1\KVO@result
}
\def\KVO@splitcomma#1,#2\@nil{%
  \KVO@ifempty{#1}{}{%
    \KVO@checkkv#1=\@nil
  }%
  \KVO@ifempty{#2}{}{\KVO@splitcomma#2\@nil}%
}
\def\KVO@ifempty#1{%
  \expandafter\ifx\expandafter\\\detokenize{#1}\\%
    \expandafter\@firstoftwo
  \else
    \expandafter\@secondoftwo
  \fi
}
\def\KVO@checkkv#1=#2\@nil{%
  \KVO@ifempty{#2}{%
    % option without value
    \edef\KVO@x{\zap@space#1 \@empty}%
    \ifx\KVO@x\@empty
      % ignore empty option
    \else
      % append to list
      \edef\KVO@result{%
        \etex@unexpanded\expandafter{\KVO@result},\KVO@x
      }%
    \fi
  }{%
    % #1: "key", #2: "value="
    % add key part
    \edef\KVO@result{%
      \etex@unexpanded\expandafter{\KVO@result},%
      \zap@space#1 \@empty
    }%
    \futurelet\@let@token\KVO@checkfirsttok#2 \@nil| = \@nil|\KVO@nil
  }%
}
\def\KVO@checkfirsttok{%
  \ifx\@let@token\bgroup
    % no space at start
    \expandafter\KVO@removelastspace\expandafter=%
    % "<value><spaceopt>= \@nil"
  \else
    \expandafter\KVO@checkfirstA
  \fi
}
\def\KVO@checkfirstA#1 #2\@nil{%
  \KVO@ifempty{#2}{%
    \KVO@removelastspace=#1 \@nil
  }{%
    \KVO@ifempty{#1}{%
      \KVO@removelastspace=#2\@nil
    }{%
      \KVO@removelastspace=#1 #2\@nil
    }%
  }%
}
\def\KVO@removelastspace#1 = \@nil|#2\KVO@nil{%
  \KVO@ifempty{#2}{%
    \edef\KVO@result{%
      \etex@unexpanded\expandafter{\KVO@result}%
      \etex@unexpanded\expandafter{\KVO@removegarbage#1\KVO@nil}%
    }%
  }{%
    \edef\KVO@result{%
      \etex@unexpanded\expandafter{\KVO@result}%
      \etex@unexpanded{#1}%
    }%
  }%
}
\def\KVO@removegarbage#1= \@nil#2\KVO@nil{#1}%
%    \end{macrocode}
%
%    Arguments |#1| and |#2| are macros.
%    \begin{macrocode}
\def\KVO@removeelement#1#2{%
  \begingroup
    \toks@={}%
    \@for\x:=#2\do{%
      \ifx\x\@empty
      \else
        \ifx\x#1\relax
        \else
          \edef\t{\the\toks@}%
          \ifx\t\@empty
          \else
            \toks@\expandafter{\the\toks@,}%
          \fi
          \toks@\expandafter{\the\expandafter\toks@\x}%
        \fi
      \fi
    }%
    \edef\x{\endgroup
      \def\noexpand#2{\the\toks@}%
    }%
  \x
}
%    \end{macrocode}
%
%
%    \begin{macrocode}
\let\@@fileswith@pti@ns\KVO@fileswith@pti@ns
\ifx\@fileswith@pti@ns\@badrequireerror
\else
  \let\@fileswith@pti@ns\KVO@fileswith@pti@ns
\fi
%    \end{macrocode}
%
%    \begin{macro}{\KVO@Patch}
%    \begin{macrocode}
\let\KVO@Patch=Y
%    \end{macrocode}
%    \end{macro}
%
%    \begin{macrocode}
\KVO@AtEnd%
%</patch>
%    \end{macrocode}
%
% \section{Test}
%
% \subsection{Preface for standard catcode check}
%
%    \begin{macrocode}
%<*test1>
\input miniltx.tex\relax
%</test1>
%    \end{macrocode}
%
% \subsection{Catcode checks for loading}
%
%    \begin{macrocode}
%<*test1>
%    \end{macrocode}
%    \begin{macrocode}
\catcode`\{=1 %
\catcode`\}=2 %
\catcode`\#=6 %
\catcode`\@=11 %
\expandafter\ifx\csname count@\endcsname\relax
  \countdef\count@=255 %
\fi
\expandafter\ifx\csname @gobble\endcsname\relax
  \long\def\@gobble#1{}%
\fi
\expandafter\ifx\csname @firstofone\endcsname\relax
  \long\def\@firstofone#1{#1}%
\fi
\expandafter\ifx\csname loop\endcsname\relax
  \expandafter\@firstofone
\else
  \expandafter\@gobble
\fi
{%
  \def\loop#1\repeat{%
    \def\body{#1}%
    \iterate
  }%
  \def\iterate{%
    \body
      \let\next\iterate
    \else
      \let\next\relax
    \fi
    \next
  }%
  \let\repeat=\fi
}%
\def\RestoreCatcodes{}
\count@=0 %
\loop
  \edef\RestoreCatcodes{%
    \RestoreCatcodes
    \catcode\the\count@=\the\catcode\count@\relax
  }%
\ifnum\count@<255 %
  \advance\count@ 1 %
\repeat

\def\RangeCatcodeInvalid#1#2{%
  \count@=#1\relax
  \loop
    \catcode\count@=15 %
  \ifnum\count@<#2\relax
    \advance\count@ 1 %
  \repeat
}
\def\RangeCatcodeCheck#1#2#3{%
  \count@=#1\relax
  \loop
    \ifnum#3=\catcode\count@
    \else
      \errmessage{%
        Character \the\count@\space
        with wrong catcode \the\catcode\count@\space
        instead of \number#3%
      }%
    \fi
  \ifnum\count@<#2\relax
    \advance\count@ 1 %
  \repeat
}
\def\space{ }
\expandafter\ifx\csname LoadCommand\endcsname\relax
  \def\LoadCommand{\input kvoptions.sty\relax}%
\fi
\def\Test{%
  \RangeCatcodeInvalid{0}{47}%
  \RangeCatcodeInvalid{58}{64}%
  \RangeCatcodeInvalid{91}{96}%
  \RangeCatcodeInvalid{123}{255}%
  \catcode`\@=12 %
  \catcode`\\=0 %
  \catcode`\%=14 %
  \LoadCommand
  \RangeCatcodeCheck{0}{36}{15}%
  \RangeCatcodeCheck{37}{37}{14}%
  \RangeCatcodeCheck{38}{47}{15}%
  \RangeCatcodeCheck{48}{57}{12}%
  \RangeCatcodeCheck{58}{63}{15}%
  \RangeCatcodeCheck{64}{64}{12}%
  \RangeCatcodeCheck{65}{90}{11}%
  \RangeCatcodeCheck{91}{91}{15}%
  \RangeCatcodeCheck{92}{92}{0}%
  \RangeCatcodeCheck{93}{96}{15}%
  \RangeCatcodeCheck{97}{122}{11}%
  \RangeCatcodeCheck{123}{255}{15}%
  \RestoreCatcodes
}
\Test
\csname @@end\endcsname
\end
%    \end{macrocode}
%    \begin{macrocode}
%</test1>
%    \end{macrocode}
%
%    \begin{macrocode}
%<*test2>
\NeedsTeXFormat{LaTeX2e}
\makeatletter
\catcode`\@=11 %
\def\RestoreCatcodes{}
\count@=0 %
\loop
  \edef\RestoreCatcodes{%
    \RestoreCatcodes
    \catcode\the\count@=\the\catcode\count@\relax
  }%
\ifnum\count@<255 %
  \advance\count@\@ne
\repeat

\def\RangeCatcodeInvalid#1#2{%
  \count@=#1\relax
  \loop
    \catcode\count@=15 %
  \ifnum\count@<#2\relax
    \advance\count@\@ne
  \repeat
}
\def\Test#1{%
  \RangeCatcodeInvalid{0}{47}%
  \RangeCatcodeInvalid{58}{64}%
  \RangeCatcodeInvalid{91}{96}%
  \RangeCatcodeInvalid{123}{255}%
  \catcode`\@=12 %
  \catcode`\\=0 %
  \catcode`\{=1 %
  \catcode`\}=2 %
  \catcode`\#=6 %
  \catcode`\[=12 %
  \catcode`\]=12 %
  \catcode`\%=14 %
  \catcode`\ =10 %
  \catcode13=5 %
  #1\relax
  \RestoreCatcodes
}
\Test{\RequirePackage{kvoptions-patch}}%
\Test{\RequirePackage{kvoptions}}%
\csname @@end\endcsname
%</test2>
%    \end{macrocode}
%
%    \begin{macrocode}
%<*test3>
\NeedsTeXFormat{LaTeX2e}
\makeatletter
\RequirePackage{kvoptions}[2016/05/16]
\def\msg#{\immediate\write16}
\define@key{testfamily}{testkey}{%
  \msg{[testfamily/testkey/#1]}%
}
\define@key{testfamily}{testdefaultkey}[testdefault]{%
  \msg{[testfamily/testdefaultkey/#1]}%
}
\AddToKeyvalOption{testfamily}{testkey}{%
  \msg{[addition/#1]}%
}
\AddToKeyvalOption{testfamily}{testdefaultkey}{%
  \msg{[addition/#1]}%
}
\setkeys{testfamily}{%
  testkey=testA,%
  testdefaultkey=testB,%
  testdefaultkey,%
}
\SetupKeyvalOptions{%
  family=testfamily%
}
\AddToKeyvalOption*{testkey}{%
  \msg{[star addition/#1]}%
}
\AddToKeyvalOption*{testdefaultkey}{%
  \msg{[star addition/#1]}%
}
\setkeys{testfamily}{%
  testkey=testA,%
  testdefaultkey=testB,%
  testdefaultkey,%
}
\@@end
%</test3>
%    \end{macrocode}
%
%    \begin{macrocode}
%<*test4pkg>
\NeedsTeXFormat{LaTeX2e}
\ProvidesPackage{kvoptions-test4}[2016/05/16 package for testing]
\RequirePackage{kvoptions}[2016/05/16]
\SetupKeyvalOptions{%
  family=FOO,%
  prefix=foo,%
  setkeys=\kvsetkeys,%
}
\DeclareStringOption{str}
\define@key{FOO}{set}{%
  \setkeys{BAR}{strbar={#1}}%
}
\define@key{BAR}{strbar}{%
  \def\foostr{[BAR:#1]}%
}
\ProcessKeyvalOptions*
%</test4pkg>
%    \end{macrocode}
%    \begin{macrocode}
%<*test4>
\NeedsTeXFormat{LaTeX2e}
\ProvidesFile{kvoptions-test4.tex}[2016/05/16 test file]
\RequirePackage[%
  str=A,set=B,str=C,%
]{kvoptions-test4}[2016/05/16]
\def\TestExpected{C}
\ifx\foostr\TestExpected
  \typeout{* Test ok.}%
\else
  \typeout{* Result: [\foostr]}%
  \typeout{* Expected: [\TestExpected]}%
  \errmessage{Test failed!}%
\fi
\csname @@end\endcsname\end
%</test4>
%    \end{macrocode}
%
% \section{Installation}
%
% \subsection{Download}
%
% \paragraph{Package.} This package is available on
% CTAN\footnote{\CTANpkg{kvoptions}}:
% \begin{description}
% \item[\CTAN{macros/latex/contrib/oberdiek/kvoptions.dtx}] The source file.
% \item[\CTAN{macros/latex/contrib/oberdiek/kvoptions.pdf}] Documentation.
% \end{description}
%
%
% \paragraph{Bundle.} All the packages of the bundle `oberdiek'
% are also available in a TDS compliant ZIP archive. There
% the packages are already unpacked and the documentation files
% are generated. The files and directories obey the TDS standard.
% \begin{description}
% \item[\CTANinstall{install/macros/latex/contrib/oberdiek.tds.zip}]
% \end{description}
% \emph{TDS} refers to the standard ``A Directory Structure
% for \TeX\ Files'' (\CTAN{tds/tds.pdf}). Directories
% with \xfile{texmf} in their name are usually organized this way.
%
% \subsection{Bundle installation}
%
% \paragraph{Unpacking.} Unpack the \xfile{oberdiek.tds.zip} in the
% TDS tree (also known as \xfile{texmf} tree) of your choice.
% Example (linux):
% \begin{quote}
%   |unzip oberdiek.tds.zip -d ~/texmf|
% \end{quote}
%
% \paragraph{Script installation.}
% Check the directory \xfile{TDS:scripts/oberdiek/} for
% scripts that need further installation steps.
%
% \subsection{Package installation}
%
% \paragraph{Unpacking.} The \xfile{.dtx} file is a self-extracting
% \docstrip\ archive. The files are extracted by running the
% \xfile{.dtx} through \plainTeX:
% \begin{quote}
%   \verb|tex kvoptions.dtx|
% \end{quote}
%
% \paragraph{TDS.} Now the different files must be moved into
% the different directories in your installation TDS tree
% (also known as \xfile{texmf} tree):
% \begin{quote}
% \def\t{^^A
% \begin{tabular}{@{}>{\ttfamily}l@{ $\rightarrow$ }>{\ttfamily}l@{}}
%   kvoptions.sty & tex/latex/oberdiek/kvoptions.sty\\
%   kvoptions-patch.sty & tex/latex/oberdiek/kvoptions-patch.sty\\
%   kvoptions.pdf & doc/latex/oberdiek/kvoptions.pdf\\
%   example-mycolorsetup.sty & doc/latex/oberdiek/example-mycolorsetup.sty\\
%   test/kvoptions-test1.tex & doc/latex/oberdiek/test/kvoptions-test1.tex\\
%   test/kvoptions-test2.tex & doc/latex/oberdiek/test/kvoptions-test2.tex\\
%   test/kvoptions-test3.tex & doc/latex/oberdiek/test/kvoptions-test3.tex\\
%   test/kvoptions-test4.tex & doc/latex/oberdiek/test/kvoptions-test4.tex\\
%   test/kvoptions-test4.sty & doc/latex/oberdiek/test/kvoptions-test4.sty\\
%   kvoptions.dtx & source/latex/oberdiek/kvoptions.dtx\\
% \end{tabular}^^A
% }^^A
% \sbox0{\t}^^A
% \ifdim\wd0>\linewidth
%   \begingroup
%     \advance\linewidth by\leftmargin
%     \advance\linewidth by\rightmargin
%   \edef\x{\endgroup
%     \def\noexpand\lw{\the\linewidth}^^A
%   }\x
%   \def\lwbox{^^A
%     \leavevmode
%     \hbox to \linewidth{^^A
%       \kern-\leftmargin\relax
%       \hss
%       \usebox0
%       \hss
%       \kern-\rightmargin\relax
%     }^^A
%   }^^A
%   \ifdim\wd0>\lw
%     \sbox0{\small\t}^^A
%     \ifdim\wd0>\linewidth
%       \ifdim\wd0>\lw
%         \sbox0{\footnotesize\t}^^A
%         \ifdim\wd0>\linewidth
%           \ifdim\wd0>\lw
%             \sbox0{\scriptsize\t}^^A
%             \ifdim\wd0>\linewidth
%               \ifdim\wd0>\lw
%                 \sbox0{\tiny\t}^^A
%                 \ifdim\wd0>\linewidth
%                   \lwbox
%                 \else
%                   \usebox0
%                 \fi
%               \else
%                 \lwbox
%               \fi
%             \else
%               \usebox0
%             \fi
%           \else
%             \lwbox
%           \fi
%         \else
%           \usebox0
%         \fi
%       \else
%         \lwbox
%       \fi
%     \else
%       \usebox0
%     \fi
%   \else
%     \lwbox
%   \fi
% \else
%   \usebox0
% \fi
% \end{quote}
% If you have a \xfile{docstrip.cfg} that configures and enables \docstrip's
% TDS installing feature, then some files can already be in the right
% place, see the documentation of \docstrip.
%
% \subsection{Refresh file name databases}
%
% If your \TeX~distribution
% (\TeX\,Live, \mikTeX, \dots) relies on file name databases, you must refresh
% these. For example, \TeX\,Live\ users run \verb|texhash| or
% \verb|mktexlsr|.
%
% \subsection{Some details for the interested}
%
% \paragraph{Unpacking with \LaTeX.}
% The \xfile{.dtx} chooses its action depending on the format:
% \begin{description}
% \item[\plainTeX:] Run \docstrip\ and extract the files.
% \item[\LaTeX:] Generate the documentation.
% \end{description}
% If you insist on using \LaTeX\ for \docstrip\ (really,
% \docstrip\ does not need \LaTeX), then inform the autodetect routine
% about your intention:
% \begin{quote}
%   \verb|latex \let\install=y% \iffalse meta-comment
%
% File: kvoptions.dtx
% Version: 2016/05/16 v3.12
% Info: Key value format for package options
%
% Copyright (C)
%    2004, 2006, 2007, 2009-2011 Heiko Oberdiek
%    2016-2019 Oberdiek Package Support Group
%    https://github.com/ho-tex/oberdiek/issues
%
% This work may be distributed and/or modified under the
% conditions of the LaTeX Project Public License, either
% version 1.3c of this license or (at your option) any later
% version. This version of this license is in
%    https://www.latex-project.org/lppl/lppl-1-3c.txt
% and the latest version of this license is in
%    https://www.latex-project.org/lppl.txt
% and version 1.3 or later is part of all distributions of
% LaTeX version 2005/12/01 or later.
%
% This work has the LPPL maintenance status "maintained".
%
% The Current Maintainers of this work are
% Heiko Oberdiek and the Oberdiek Package Support Group
% https://github.com/ho-tex/oberdiek/issues
%
% This work consists of the main source file kvoptions.dtx
% and the derived files
%    kvoptions.sty, kvoptions.pdf, kvoptions.ins, kvoptions.drv,
%    kvoptions-patch.sty, example-mycolorsetup.sty,
%    kvoptions-test1.tex, kvoptions-test2.tex,
%    kvoptions-test3.tex, kvoptions-test4.tex,
%    kvoptions-test4.sty.
%
% Distribution:
%    CTAN:macros/latex/contrib/oberdiek/kvoptions.dtx
%    CTAN:macros/latex/contrib/oberdiek/kvoptions.pdf
%
% Unpacking:
%    (a) If kvoptions.ins is present:
%           tex kvoptions.ins
%    (b) Without kvoptions.ins:
%           tex kvoptions.dtx
%    (c) If you insist on using LaTeX
%           latex \let\install=y% \iffalse meta-comment
%
% File: kvoptions.dtx
% Version: 2016/05/16 v3.12
% Info: Key value format for package options
%
% Copyright (C)
%    2004, 2006, 2007, 2009-2011 Heiko Oberdiek
%    2016-2019 Oberdiek Package Support Group
%    https://github.com/ho-tex/oberdiek/issues
%
% This work may be distributed and/or modified under the
% conditions of the LaTeX Project Public License, either
% version 1.3c of this license or (at your option) any later
% version. This version of this license is in
%    https://www.latex-project.org/lppl/lppl-1-3c.txt
% and the latest version of this license is in
%    https://www.latex-project.org/lppl.txt
% and version 1.3 or later is part of all distributions of
% LaTeX version 2005/12/01 or later.
%
% This work has the LPPL maintenance status "maintained".
%
% The Current Maintainers of this work are
% Heiko Oberdiek and the Oberdiek Package Support Group
% https://github.com/ho-tex/oberdiek/issues
%
% This work consists of the main source file kvoptions.dtx
% and the derived files
%    kvoptions.sty, kvoptions.pdf, kvoptions.ins, kvoptions.drv,
%    kvoptions-patch.sty, example-mycolorsetup.sty,
%    kvoptions-test1.tex, kvoptions-test2.tex,
%    kvoptions-test3.tex, kvoptions-test4.tex,
%    kvoptions-test4.sty.
%
% Distribution:
%    CTAN:macros/latex/contrib/oberdiek/kvoptions.dtx
%    CTAN:macros/latex/contrib/oberdiek/kvoptions.pdf
%
% Unpacking:
%    (a) If kvoptions.ins is present:
%           tex kvoptions.ins
%    (b) Without kvoptions.ins:
%           tex kvoptions.dtx
%    (c) If you insist on using LaTeX
%           latex \let\install=y\input{kvoptions.dtx}
%        (quote the arguments according to the demands of your shell)
%
% Documentation:
%    (a) If kvoptions.drv is present:
%           latex kvoptions.drv
%    (b) Without kvoptions.drv:
%           latex kvoptions.dtx; ...
%    The class ltxdoc loads the configuration file ltxdoc.cfg
%    if available. Here you can specify further options, e.g.
%    use A4 as paper format:
%       \PassOptionsToClass{a4paper}{article}
%
%    Programm calls to get the documentation (example):
%       pdflatex kvoptions.dtx
%       makeindex -s gind.ist kvoptions.idx
%       pdflatex kvoptions.dtx
%       makeindex -s gind.ist kvoptions.idx
%       pdflatex kvoptions.dtx
%
% Installation:
%    TDS:tex/latex/oberdiek/kvoptions.sty
%    TDS:tex/latex/oberdiek/kvoptions-patch.sty
%    TDS:doc/latex/oberdiek/kvoptions.pdf
%    TDS:doc/latex/oberdiek/example-mycolorsetup.sty
%    TDS:doc/latex/oberdiek/test/kvoptions-test1.tex
%    TDS:doc/latex/oberdiek/test/kvoptions-test2.tex
%    TDS:doc/latex/oberdiek/test/kvoptions-test3.tex
%    TDS:doc/latex/oberdiek/test/kvoptions-test4.tex
%    TDS:doc/latex/oberdiek/test/kvoptions-test4.sty
%    TDS:source/latex/oberdiek/kvoptions.dtx
%
%<*ignore>
\begingroup
  \catcode123=1 %
  \catcode125=2 %
  \def\x{LaTeX2e}%
\expandafter\endgroup
\ifcase 0\ifx\install y1\fi\expandafter
         \ifx\csname processbatchFile\endcsname\relax\else1\fi
         \ifx\fmtname\x\else 1\fi\relax
\else\csname fi\endcsname
%</ignore>
%<*install>
\input docstrip.tex
\Msg{************************************************************************}
\Msg{* Installation}
\Msg{* Package: kvoptions 2016/05/16 v3.12 Key value format for package options (HO)}
\Msg{************************************************************************}

\keepsilent
\askforoverwritefalse

\let\MetaPrefix\relax
\preamble

This is a generated file.

Project: kvoptions
Version: 2016/05/16 v3.12

Copyright (C)
   2004, 2006, 2007, 2009-2011 Heiko Oberdiek
   2016-2019 Oberdiek Package Support Group

This work may be distributed and/or modified under the
conditions of the LaTeX Project Public License, either
version 1.3c of this license or (at your option) any later
version. This version of this license is in
   https://www.latex-project.org/lppl/lppl-1-3c.txt
and the latest version of this license is in
   https://www.latex-project.org/lppl.txt
and version 1.3 or later is part of all distributions of
LaTeX version 2005/12/01 or later.

This work has the LPPL maintenance status "maintained".

The Current Maintainers of this work are
Heiko Oberdiek and the Oberdiek Package Support Group
https://github.com/ho-tex/oberdiek/issues


This work consists of the main source file kvoptions.dtx
and the derived files
   kvoptions.sty, kvoptions.pdf, kvoptions.ins, kvoptions.drv,
   kvoptions-patch.sty, example-mycolorsetup.sty,
   kvoptions-test1.tex, kvoptions-test2.tex,
   kvoptions-test3.tex, kvoptions-test4.tex,
   kvoptions-test4.sty.

\endpreamble
\let\MetaPrefix\DoubleperCent

\generate{%
  \file{kvoptions.ins}{\from{kvoptions.dtx}{install}}%
  \file{kvoptions.drv}{\from{kvoptions.dtx}{driver}}%
  \usedir{tex/latex/oberdiek}%
  \file{kvoptions.sty}{\from{kvoptions.dtx}{package}}%
  \file{kvoptions-patch.sty}{\from{kvoptions.dtx}{patch}}%
  \usedir{doc/latex/oberdiek}%
  \file{example-mycolorsetup.sty}{\from{kvoptions.dtx}{example}}%
%  \usedir{doc/latex/oberdiek/test}%
%  \file{kvoptions-test1.tex}{\from{kvoptions.dtx}{test1}}%
%  \file{kvoptions-test2.tex}{\from{kvoptions.dtx}{test2}}%
%  \file{kvoptions-test3.tex}{\from{kvoptions.dtx}{test3}}%
%  \file{kvoptions-test4.tex}{\from{kvoptions.dtx}{test4}}%
%  \file{kvoptions-test4.sty}{\from{kvoptions.dtx}{test4pkg}}%
  \nopreamble
  \nopostamble
%  \usedir{source/latex/oberdiek/catalogue}%
%  \file{kvoptions.xml}{\from{kvoptions.dtx}{catalogue}}%
}

\catcode32=13\relax% active space
\let =\space%
\Msg{************************************************************************}
\Msg{*}
\Msg{* To finish the installation you have to move the following}
\Msg{* files into a directory searched by TeX:}
\Msg{*}
\Msg{*     kvoptions.sty, kvoptions-patch.sty}
\Msg{*}
\Msg{* To produce the documentation run the file `kvoptions.drv'}
\Msg{* through LaTeX.}
\Msg{*}
\Msg{* Happy TeXing!}
\Msg{*}
\Msg{************************************************************************}

\endbatchfile
%</install>
%<*ignore>
\fi
%</ignore>
%<*driver>
\NeedsTeXFormat{LaTeX2e}
\ProvidesFile{kvoptions.drv}%
  [2016/05/16 v3.12 Key value format for package options (HO)]%
\documentclass{ltxdoc}
\usepackage{holtxdoc}[2011/11/22]
\begin{document}
  \DocInput{kvoptions.dtx}%
\end{document}
%</driver>
% \fi
%
%
% \CharacterTable
%  {Upper-case    \A\B\C\D\E\F\G\H\I\J\K\L\M\N\O\P\Q\R\S\T\U\V\W\X\Y\Z
%   Lower-case    \a\b\c\d\e\f\g\h\i\j\k\l\m\n\o\p\q\r\s\t\u\v\w\x\y\z
%   Digits        \0\1\2\3\4\5\6\7\8\9
%   Exclamation   \!     Double quote  \"     Hash (number) \#
%   Dollar        \$     Percent       \%     Ampersand     \&
%   Acute accent  \'     Left paren    \(     Right paren   \)
%   Asterisk      \*     Plus          \+     Comma         \,
%   Minus         \-     Point         \.     Solidus       \/
%   Colon         \:     Semicolon     \;     Less than     \<
%   Equals        \=     Greater than  \>     Question mark \?
%   Commercial at \@     Left bracket  \[     Backslash     \\
%   Right bracket \]     Circumflex    \^     Underscore    \_
%   Grave accent  \`     Left brace    \{     Vertical bar  \|
%   Right brace   \}     Tilde         \~}
%
% \GetFileInfo{kvoptions.drv}
%
% \title{The \xpackage{kvoptions} package}
% \date{2016/05/16 v3.12}
% \author{Heiko Oberdiek\thanks
% {Please report any issues at \url{https://github.com/ho-tex/oberdiek/issues}}}
%
% \maketitle
%
% \begin{abstract}
% This package is intended for package authors who want to
% use options in key value format for their package options.
% \end{abstract}
%
% \tableofcontents
%
% \def\M#1{\texttt{\{}\meta{#1}\texttt{\}}}
% \def\O#1{\texttt{[}\meta{#1}\texttt{]}}
%
% \section{Introduction}
%
% First I want to recommend the very good review article
% ``A guide to key-value methods'' by Joseph Wright \cite{tb94wright}.
% It introduces the different key-value packages and compares them.
%
% \subsection{The beginning}
%
% This package \xpackage{kvoptions} addresses class or package writers
% that want to allow their users to specify options
% as key value pairs, e.g.:
% \begin{quote}
% |\documentclass[verbose=false,name=me]{myclass}|\\
% |\usepackage[format=print]{mylayout}|
% \end{quote}
% Prominent example is package \xpackage{hyperref},
% probably the first package that offers this service.
% It's \cs{ProcessOptionsWithKV} is often copied und
% used in other packages, e.g. package \xpackage{helvet}
% that uses this interface for its option \xoption{scaled}.
%
% However copying code is not the most modern software
% development technique. And \xpackage{hyperref}'s
% code for \cs{ProcessOptionsWithKV} was changed
% to fix bugs. The version used in other packages
% depends on the time of copying and the awareness
% of \xpackage{hyperref}'s changes.
% Now the code is sourced out into this package and
% available for other package or class writers.
%
% \subsection{Overview}
%
% Package \xpackage{kvoptions} connects package
% \xpackage{\emph{k}ey\emph{v}al}
% with \LaTeX's package and class \emph{options}:
% \begin{center}
% \def\T#1{\bfseries #1}
% \renewcommand{\arraystretch}{1.5}
% \begin{tabular}{l|l|l}
%   \T{Package \xpackage{keyval}}
%   & \T{Package \xpackage{kvoptions}}
%   & \T{\LaTeX\ kernel}
%   \\
% \hline
%     \cs{define@key}
%   &
%     \renewcommand*{\arraystretch}{1}^^A
%     \begin{tabular}[t]{@{}l@{}}
%       \cs{DeclareVoidOption}\\
%       \cs{DeclareStringOption}\\
%       \cs{DeclareBoolOption}\\
%       \cs{DeclareComplementaryOption}\\
%       \cs{DisableKeyvalOption}
%     \end{tabular}
%   & \cs{DeclareOption}
%   \\
% \hline
%   & \cs{DeclareDefaultOption}
%   & \cs{DeclareOption*}
%   \\
% \hline
%   & \cs{ProcessKeyvalOptions}
%   & \cs{ProcessOptions*}
%   \\
% \hline
%   & Option \xoption{patch}
%   &
%     \renewcommand*{\arraystretch}{1}^^A
%     \begin{tabular}[t]{@{}l@{}}
%       Class/package\\
%       option system
%     \end{tabular}
%   \\
% \hline
%   & \cs{SetupKeyvalOptions} &
%   \\
% \end{tabular}
% \end{center}
%
% \hypersetup{bookmarksopenlevel=3}
% \section{Usage}
%
% \subsection{Process options}
%
% \subsubsection{\cs{ProcessKeyvalOptions}}
%
% \begin{declcs}{ProcessKeyvalOptions}\,\M{family}\\
%   \cs{ProcessKeyvalOptions} \*
% \end{declcs}
% This command evaluates the global or local options of
% the package that are defined with \xpackage{keyval}'s
% interface within the family \meta{family}. It acts the
% same way as \LaTeX's \cs{ProcessOptions*}. In a package
% unknown global options are ignored, in a class they
% are added to the unknown option list. The known global
% options and all local options are passed to \xpackage{keyval}'s
% \cs{setkeys} command for executing the options.
% Unknown options are reported to the user by an error.
%
% If the family name happens to be the same as the name
% of the package or class where \cs{ProcessKeyvalOptions}
% is used or the family name has previously been setup by
% \cs{SetupKeyvalOptions}, then \cs{ProcessKeyvalOptions}
% knows the family name already and you can use the star form
% without mandatory argument.
%
% \subsubsection{\cs{ProcessLocalKeyvalOptions}}
%
% \begin{declcs}{ProcessLocalKeyvalOptions}\,\M{family}\\
%   \cs{ProcessLocalKeyvalOptions} \*
% \end{declcs}
% This macro has the same syntax and works similar as
% \cs{ProcessKeyvalOptions}. However it ignores
% global options and only processes the local package options.
% Therefore it only can be used inside a package.
% An error is thrown, if it is used inside a class.
%
% Neither of the following macros are necessary for
% \cs{ProcessKeyvalOptions{}}. They just help the package/class
% author in common tasks.
%
% \subsubsection{\cs{SetupKeyvalOptions}}\label{setup}
%
% \begin{declcs}{SetupKeyvalOptions}\,\texttt{\{}\\
%   \hspace{1.5em}|family|\,|=|\,\meta{family}|,|\\
%   \hspace{1.5em}|prefix|\,|=|\,\meta{prefix}\\
%   \hspace{1.5em}|setkeys|\,|=|\,\meta{setkeys command}\\
%   \texttt{\}}
% \end{declcs}
% This command allows to configure the default assumptions
% that are based on the current package or class name.
% \LaTeX\ remembers this name in \cs{@currname}. The syntax
% description of the default looks a little weird, therefor
% an example is given for a package or class named |foobar|.
% \begin{quote}
% \begin{tabular}{@{}l|ll|l@{}}
%   Key & Default & (example) & Used by\\
% \hline
%   \rule{0pt}{1.2\ht\csname @arstrutbox\endcsname}^^A
%   |family| & \meta{\cs{@currname}} & (|foobar|)
%       & \cs{ProcessKeyvalOptions*} \\
%     &&& \cs{DeclareBoolOption} \\
%     &&& \cs{DeclareStringOption} \\
% \hline
%   \rule{0pt}{1.2\ht\csname @arstrutbox\endcsname}^^A
%   |prefix| & \meta{\cs{@currname}}|@| & (|foobar@|)
%       & \cs{DeclareBoolOption} \\
%     &&& \cs{DeclareStringOption} \\
%     &&& \cs{DeclareVoidOption} \\
% \hline
%   \rule{0pt}{1.2\ht\csname @arstrutbox\endcsname}^^A
%   |setkeys| & \cs{setkeys} & (\cs{kvsetkeys})
%       & \cs{ProcessKeyvalOptions} \\
%     &&& \cs{ProcessLocalKeyvalOptions}
% \end{tabular}
% \end{quote}
% Key |setkeys| was added in version 3.9. The original
% \cs{setkeys} of package \xpackage{keyval} is not reentrant.
% If an option is processed by this \cs{setkeys}, then
% the option should not call \cs{setkeys} again with a
% different family. Otherwise the next options of the
% first \cs{setkeys} call are processed with the wrong
% family. With key |setkeys| the macro \cs{kvsetkeys}
% can be set that does not have the problem of the
% original \cs{setkeys} of package {keyval}.
%
% Probably \cs{setkeys} of package \xpackage{xkeyval}
% is safe in this respect. But I haven't made a full
% analysis. At least it does not have the problem
% of the original \cs{setkeys}.
%
% \subsection{Option declarations}
%
% The options for \cs{ProcessKeyvalOptions} are defined
% by \xpackage{keyval}'s \cs{define@key}.
% Common purposes of such keys are boolean switches, they
% enable or disable something. Or they store a name or some kind of
% string in a macro. The following commands help the user. He
% declares what he wants and \xpackage{kvoptions} take care of
% the key definition, resource allocation and initialization.
%
% In order to avoid name clashes of macro names, internal
% commands are prefixed. Both the prefix and the family
% name for the defined keys can be configured by
% \cs{SetupKeyvalOptions}.
%
% \subsubsection{\cs{DeclareStringOption}}
%
% \begin{declcs}{DeclareStringOption}\,\O{init}\,\M{key}\,\O{default}
% \end{declcs}
% A macro is created that remembers the value of the key \meta{key}.
% The name of the macro consists of the option name \meta{key} that
% is prefixed by the prefix (see \ref{setup}).
% The initial contents of the
% macro can be given by the first optional argument \meta{init}.
% The default is empty.
%
% The the option \meta{key} is defined. The option code
% just stores its value in the macro. If the optional argument
% at the end of \cs{DeclareStringOption} is given, then option
% \meta{key} is defined with the default \meta{default}.
%
% Example for a package with the following two lines:
% \begin{quote}
% |\ProvidesPackage{foobar}|\\
% |\DeclareStringOption[me]{name}|
% \end{quote}
% Then \cs{DeclareStringOption} defines the macro with content |me|,
% note \LaTeX\ complains if the name of the macro already exists:
% \begin{quote}
% |\newcommand*{\foobar@name}{me}|
% \end{quote}
% The option definition is similar to:
% \begin{quote}
% |\define@key{foobar}{name}{%|\\
% |  \renewcommand*{\foobar@name}{#1}%|\\
% |}|
% \end{quote}
%
% \subsubsection{\cs{DeclareBoolOption}}
%
% \begin{declcs}{DeclareBoolOption}\,\O{init}\,\M{key}
% \end{declcs}
% A boolean switch is generated, initialized by value \meta{init}
% and the corresponding key \meta{key} is defined. If the
% initialization value is not given, |false| is used as default.
%
% The internal actions of \cs{DeclareBoolOption} are shown below.
% The example is given for a package author who has the following
% two lines in his package/class:
% \begin{quote}
% |\ProvidesPackage{foobar}|\\
% |\DeclareBoolOption{verbose}|
% \end{quote}
% First a new switch is created:
% \begin{quote}
%   |\newif\iffoobar@verbose|
% \end{quote}
% and initialized:
% \begin{quote}
%   |\foobar@verbosefalse|
% \end{quote}
% Finally the key is defined:
% \begin{quote}
%   |\define@key{foobar}{verbose}[true]{|\dots|}|
% \end{quote}
% The option code configures the boolean option in the following way:
% If the author specifies |true| or |false| then the switch is
% turned on or off respectivly. Also the option can be given without
% explicit value. Then the switch is enabled.
% Other values are reported as errors.
%
% Now the switch is ready to use in the package/class, e.g.:
% \begin{quote}
%   |\iffoobar@verbose|\\
%   \qquad\textit{\% print verbose message}\\
%   |\else|\\
%   \qquad\textit{\% be quiet}\\
%   |\fi|
% \end{quote}
% Users of package \cs{ifthen} can use the switch as
% boolean:
% \begin{quote}
%   |\boolean{foobar@verbose}|
% \end{quote}
%
% \subsubsection{\cs{DeclareComplementaryOption}}
%
% \begin{declcs}{DeclareComplementaryOption}\,\M{key}\,\M{parent}
% \end{declcs}
% Sometimes contrasting names are used to characterize the
% two states of a boolean switch, for example \xoption{draft}
% vs.\@ \xoption{final}. Both options behave like boolean
% options but they do not need two different switches, they
% should share one. \cs{DeclareComplementaryOption} allows this.
% The option \meta{key} shares the switch of option \meta{parent}.
% Example:
% \begin{quote}
%   |\DeclareBoolOption{draft}|\\
%   |\DeclareComplementaryOption{final}{draft}|
% \end{quote}
% Then |final| sets the switch of \xoption{draft} to |false|, and
% |final=false| enables the \xoption{draft} switch.
%
% \subsubsection{\cs{DeclareVoidOption}}
%
% \begin{declcs}{DeclareVoidOption}\,\M{key}\,\M{code}
% \end{declcs}
% \cs{ProcessKeyvalOptions} can be extended to recognize
% options that are declared in traditional way by \cs{DeclareOption}.
% But in case of the error that the user specifies a value, then
% this option would not recognized as key value option because
% of \cs{DeclareOption} and not detected as traditional option
% because of the value part. The user would get an unknown option
% error, difficult to understand.
%
% \cs{DeclareVoidOption} solves
% this problem. It defines the option \meta{key} as key value option.
% If the user specifies a value, a warning is given and the
% value is ignored.
%
% The code part \meta{code} is stored in a macro.
% The name of the macro consists of the option name \meta{key} that
% is prefixed by the prefix (see \ref{setup}).
% If the option is set, the macro will be executed. During the
% execution \cs{CurrentOption} is available with the current key name.
%
% \subsubsection{\cs{DeclareDefaultOption}}
%
% \begin{declcs}{DeclareDefaultOption}\,\M{code}
% \end{declcs}
% \begin{sloppypar}
% This command does not define a specific key, it is the equivalent
% to \LaTeX's \cs{DeclareOption*}. It allows the specification
% of a default action \meta{code} that is invoked if an unknown option
% is found.
% While \meta{code} is called,
% macro \cs{CurrentOption} contains the current option string.
% In addition \cs{CurrentOptionValue} contains the value part
% if the option string is parsable as key value pair, otherwise
% it is \cs{relax}. \cs{CurrentOptionKey} contains the key of
% the key value pair, or the whole option string, if it misses
% the equal sign.
% \end{sloppypar}
%
% Inside packages typical default actions are to pass unknown
% options to another package. Or an error message can be thrown
% by \cs{@unknownoptionerror}. This is the original error
% message that \LaTeX\ gives for unkown package options.
% This error message is easier to understand for the user
% as the error message from package \xpackage{keyval} that
% is given otherwise.
%
% A Class ignores unknown options and puts them on the
% unused option list. Let \LaTeX\ do the job and just
% call \cs{OptionNotUsed}. Or the options can be passed to
% another class that is later loaded.
%
% \subsubsection{Local options}
%
% \begin{declcs}{DeclareLocalOption} \M{option}\\
%   \cs{DeclareLocalOptions} \M{option list}
% \end{declcs}
% Both macros mark package options as local options. That means
% that they are ignored by \cs{ProcessKeyvalOptions} if they are given
% as global options. \cs{DeclareLocalOptions} takes one option,
% \cs{DeclareLocalOptions} expects a comma separated list of options.
%
% \subsubsection{Dynamic options}
%
% Options of \LaTeX's package/class system are cleared
% in \cs{ProcessOptions}. They modify the static model
% of a package. For example, depending on option \xoption{bookmarks}
% package \xpackage{hyperref} loads differently.
%
% Options, however,
% defined by \xpackage{keyval}'s \cs{define@key} remain
% defined, if the options are processed by \cs{setkeys}.
% Therefore these options can also be used to model
% the dynamic behaviour of a package. For example, in
% \xpackage{hyperref} the link colors can be changed
% everywhere until the end in |\end{document}|.
%
% However package \xpackage{color} that adds color support
% is necessary and it
% cannot be loaded after |\begin{document}|. Option \xoption{colorlinks}
% that loads \xpackage{color} should be active until |\begin{document}|
% and die in some way if it is too late for loading packages.
% With \cs{DisableKeyvalOption} the package/class author can
% specify and configure the death of an option and controls
% the life period of the option.
%
% \subsubsection{\cs{DisableKeyvalOption}}
%
% \begin{declcs}{DisableKeyvalOption}\,\O{options}\,\M{family}\,\M{key}
%   \\[0.5ex]
%   \meta{options}:\\
%   \hspace{1.5em}^^A
%   \begin{tabular}[t]{@{}l@{ \texttt{=} }l@{\qquad}>{default: }l@{}}
%     |action| & |undef|, |warning|, |error|, or |ignore|
%     & |undef|
%     \\
%     \multicolumn{1}{@{}l}{|global| or |local|} & & |global|
%     \\
%     |package| or |class| & \meta{name}
%     \\
%   \end{tabular}^^A
% \end{declcs}
% \cs{DisableKeyvalOption} can be called to mark the end when
% the option \meta{key} is no longer useful. The behaviour of
% an option after its death can be configured by \xoption{action}:
% \begin{description}
% \item[\texttt{undef}:] The option will be undefined,
%   If it is called, \cs{setkeys} reports an error because
%   of unknown key.
% \item[\texttt{error} or \texttt{warning}:] Use of the
%  option will cause an error or warning message.
%  Also these actions require that exclusivly either the
%  package or class name is given in options \xoption{package}
%  or \xoption{class}.
% \item[\texttt{ignore}:] The use of the option will silently
%  be ignored.
% \end{description}
% The option's death can be limited to the end of the current
% group, if option \xoption{local} is given. Default is
% \xoption{global}.
%
% The package/class author can wish the end of the option
% already during the package loading, then he will have
% static behaviour. In case of dynamic options \cs{DisableKeyvalOption}
% can be executed everywhere, also outside the package. Therefore
% the family name and the package/class name is usually unknown for
% \cs{DisableKeyvalOption}. Therefore the argument for the family
% name is mandatory and for some actions the package/class name
% must be provided.
%
% Usually a macro would configure the option death, Example:
%\begin{quote}
%\begin{verbatim}
%\ProvidesPackage{foobar}
%\DeclareBoolOption{color}
%\DeclareStringOption[red]{emphcolor}
%\ProcessKeyvalOptions*
%
%\newcommand*{\foobar@DisableOption}[2]{%
%  \DisableKeyvalueOption[
%    action={#1},
%    package=foobar
%  ]{foobar}{#2}%
%}
%
%\iffoobar@color
%  \RequirePackage{color}
%  \renewcommand*{\emph}[1]{\textcolor{\foobar@emphcolor}{#1}}
%\else
%  % Option emphcolor is not wrong, if we have color support.
%  % otherwise the option has no effect, but we don't want to
%  % remove it. Therefore action 'ignore' is the best choice:
%  \foobar@DisableOption{ignore}{emphcolor}
%\fi
%% No we don't need the option 'color'.
%\foobar@DisableOption{warning}{color}
%
%% With color support option 'emphcolor' will dynamically
%% change the color of \emph statements.
%\end{verbatim}
%\end{quote}
%
% \subsubsection{\cs{AddToKeyvalOption}}
%
% \begin{declcs}{AddToKeyvalOption}\,\M{family}\,\M{key}\,\M{code}\\
%   \cs{AddToKeyvalOption} \*\,\M{key}\,\M{code}
% \end{declcs}
% The code for an existing key \meta{key} of family \meta{family} is
% extended by code \meta{code}. In the starred form the current
% family setting is used, see \cs{ProcessKeyvalOptions*}.
%
% \subsection{Global vs. local options}
%
% Options that are given for \cs{documentclass} are called
% global options. They are known to the class and all packages.
% A package may make use of a global option and marks it as
% used. The advantage for the user is the freedom to specify
% options both in the \cs{documentclass} or \cs{usepackage}
% commands.
%
% However global options are shared with the class options
% and options of all other packages. Thus there can be the
% same option with different semantics for different
% packages and classes. As example, package \xpackage{bookmark}
% knows option \xoption{open} that specifies whether the bookmarks
% are opened or closed initially. It's values are |true| or |false|.
% Since KOMA-Script version 3.00 the KOMA classes also introduces
% option \xoption{open} with values |right| and |any| and a
% complete different meaning.
%
% Such conflicts can be resolved by marking all or part of options
% as local by \cs{DeclareLocalOption} or \cs{DeclareLocalOptions}.
% Then the packages ignores global occurences of these options.
% Package \xpackage{kvoptions} provides two methods:
% \begin{itemize}
% \item \cs{ProcessLocalKeyvalOptions} automatically uses all options
%   as local options. It ignores all global options.
% \item \cs{DeclareLocalOption} or \cs{DeclareLocalOptions} marks
%   options as local options. \cs{ProcessKeyvalOptions} will then
%   ignore global occurences for these local options.
% \end{itemize}
% Since version 1.5 package \xpackage{bookmark} uses the latter
% method. It checks global and local option places for driver options
% and limits all other options as local options. Thus the class
% option \xoption{open} of KOMA-Script is not misread as
% option for package \xpackage{bookmark}.
%
% \subsection{Summary of internal macros}
%
% The \cs{Declare}\texttt{\dots Option} commands define
% macros, additionally to the macros generated by the key definition.
% These macros can be used by the package/class author.
% The name of the macros starts with the prefix \meta{prefix}
% that can be configured by \cs{SetupKeyvalOptions}.
% \begin{center}
% \def\prefix{\cs{}\meta{prefix}\meta{key}}^^A
% \def\Hline{^^A
%   \hline
%   \rule{0pt}{1.2\ht\csname @arstrutbox\endcsname}^^A
% }^^A
% \begin{tabular}{@{}l|>{\ttfamily}l|l@{}}
%   Declare \meta{key} & \textrm{Defined macro} & Description\\
% \Hline
%   \cs{DeclareStringOption}
%    & \prefix & holds the string\\
% \Hline
%   \cs{DeclareBoolOption}
%     & \cs{if}\meta{prefix}\meta{key}& boolean switch\\
%   &\prefix false& disable switch\\
%   &\prefix true& enable switch\\
% \Hline
%   \cs{DeclareComplementaryOption}
%     & \prefix false& enable parent switch\\
%     & \prefix true& disable parent switch\\
% \Hline
%   \cs{DeclareVoidOption}
%   & \prefix & holds the action
% \end{tabular}
% \end{center}
%
% \subsection{\plainTeX}
%
% Package \xpackage{keyval} is also usable in \plainTeX\ with
% the help of file \xfile{miniltx.tex}. Some features of this
% package \xpackage{kvoptions} might also be useful for \plainTeX.
% If \LaTeX\ is not found, \cs{ProcessKeyvalOptions} and option
% \xoption{patch} are disabled. Before using the option declaration
% commands \cs{Declare...Option}, \cs{SetupKeyvalOptions} must be
% used.
%
% \hypersetup{bookmarksopenlevel=1}
%
% \section{Example}
%
% The following example defined a package that serves some
% private color management. A boolean option \xoption{print} enables
% print mode without colors. An option \xoption{emph} redefines
% \cs{emph} to print in the given color. And the driver
% can be specified by option \xoption{driver}.
%
%    \begin{macrocode}
%<*example>
    % Package identification
    % -----------------------
\NeedsTeXFormat{LaTeX2e}
\ProvidesPackage{example-mycolorsetup}[2016/05/16 Managing my colors]

\RequirePackage{ifpdf}
\RequirePackage{kvoptions}

    % Option declarations
    % -------------------

\SetupKeyvalOptions{
  family=MCS,
  prefix=MCS@
}
    % Use a shorter family name and prefix

    % Option print
\DeclareBoolOption{print}
    % is the same as
    % \DeclareBoolOption[false]{print}

    % Option driver
\ifpdf
  \DeclareStringOption[pdftex]{driver}
\else
  \DeclareStringOption[dvips]{driver}
\fi

    % Alternative interface for driver options
\DeclareVoidOption{dvips}{\SetupDriver}
\DeclareVoidOption{dvipdfm}{\SetupDriver}
\DeclareVoidOption{pdftex}{\SetupDriver}
    % In \SetupDriver we take the current option \CurrentOption
    % and pass it to the driver option.
    % The \expandafter commands expand \CurrentOption at the
    % time, when \SetupDriver is executed and \CurrentOption
    % has the correct meaning.
\newcommand*{\SetupDriver}{%
  \expandafter\@SetupDriver\expandafter{\CurrentOption}%
}
\newcommand*{\@SetupDriver}[1]{%
  \setkeys{MCS}{driver={#1}}%
}

    % Option emph
    % An empty value means, we want to have no color for \emph.
    % If the user specifies option emph without value, the red is used.
\DeclareStringOption{emph}[red]
    % is the same as
    % \DeclareStringOption[]{emph}[red]

    % Default option rule
\DeclareDefaultOption{%
  \ifx\CurrentOptionValue\relax
    \PackageWarningNoLine{\@currname}{%
      Unknown option `\CurrentOption'\MessageBreak
      is passed to package `color'%
    }%
    % Pass the option to package color.
    % Again it is better to expand \CurrentOption.
    \expandafter\PassOptionsToPackage
    \expandafter{\CurrentOption}{color}%
  \else
    % Package color does not take options with values.
    % We provide the standard LaTeX error.
    \@unknownoptionerror
  \fi
}

    % Process options
    % ---------------
\ProcessKeyvalOptions*

    % Implementation depending on option values
    % -----------------------------------------
    % Code for print mode
\ifMCS@print
  \PassOptionsToPackage{monochrome}{color}
    % tells package color to use black and white
\fi

\RequirePackage[\MCS@driver]{color}
    % load package color with the correct driver

    % \emph setup
\ifx\MCS@emph\@empty
    % \@empty is a predefined macro with empty contents.
    % the option value of option emph is empty, thus
    % we do not want a redefinition of \emph.
\else
  \renewcommand*{\emph}[1]{%
    \textcolor{\MCS@emph}{#1}%
  }
\fi
%</example>
%    \end{macrocode}
%
% \section{Package options}
%
% The package \xpackage{kvoptions} knows two package options
% \xoption{patch} and \xoption{debugshow}.
% The options of package \xpackage{kvoptions} are intended
% for authors, not for package/class writers. Inside a package
% it is too late for option \xoption{patch} and \xoption{debugshow}
% enables some messages that are perhaps useful for the debugging
% phase. Also \LaTeX\ is unhappy if a package is loaded
% later again with options that are previously not given.
% Thus package and class authors, stay with |\RequirePackage{kvoptions}|
% without options.
%
% Option \xoption{patch} loads package \xpackage{kvoptions-patch}.
%
% \subsection{Package \xpackage{kvoptions-patch}}
%
% \LaTeX's system of package/class options has some severe limitations
% that especially affects the value part if options are used as
% pair of key and value.
% \begin{itemize}
% \item Spaces are removed, regardless where:
%   \begin{quote}
%     |\documentclass[box=0 0 400 600]{article}|
%   \end{quote}
%   Now each package will see |box=00400600| as global option.
% \item In the previous case also braces would not help:
%   \begin{quote}
%     |\documentclass[box={0 0 400 600}]{article}|
%   \end{quote}
%   The result is an error message:
%   \begin{quote}
%     |! LaTeX Error: Missing \begin{document}.|
%   \end{quote}
%   As local option, however, it works if the package
%   knows about key value options (By using this package, for example).
% \item The requirements on robustness are extremly high.
%   \LaTeX\ expands the option. All that will not work as environment
%   name will break also as option. Even a \cs{relax} will generate
%   an error message:
%   \begin{quote}
%     |! Missing \endcsname inserted.|
%   \end{quote}
%   Of course, \LaTeX\ does not use its protecting mechanisms.
%   On contrary \cs{protect} itself will cause errors.
% \item The options are expanded. But perhaps the package will
%   do that, because it has to setup some things before?
%   Example \xpackage{hyperref}:
%   \begin{quote}
%     |\usepackage[pdfauthor=M\"uller]{hyperref}|
%   \end{quote}
%   Package \xpackage{hyperref} does not see |M\"uller| but
%   its expansion and it does not like it, you get many warnings
%   \begin{quote}
%     |Token not allowed in a PDFDocEncoded string|
%   \end{quote}
%   And the title becomes: |Mu127uller|.
%   Therefore such options must usually be given after package
%   \xpackage{hyperref}
%   is loaded:
%   \begin{quote}
%     |\usepackage{hyperref}|\\
%     |\hypersetup{pdfauthor=Fran\c coise M\"uller}|
%   \end{quote}
%   As package option it will even break with |Fran\c coise|
%   because of the cedilla |\c c|, it is not robust enough.
% \end{itemize}
% For users that do not want with this limitations the package
% offers package \xoption{kvoptions-patch}. It patches \LaTeX's option system
% and tries to teach it also to handle options that are given
% as pairs of key and value and to prevent expansion.
% It can already be used at the very beginning, before \cs{documentclass}:
% \begin{quote}
%   |\RequirePackage{kvoptions-patch}|\\
%   |\documentclass[pdfauthor=Fran\c coise M\"uller]{article}|\\
%   |\usepackage{hyperref}|
% \end{quote}
% The latest time is before the package where you want to use
% problematic values:
% \begin{quote}
%   |\usepackage{kvoptions-patch}|\\
%   |\usepackage[Fran\c coise M\"uller]{hyperref}|
% \end{quote}
% Some remarks:
% \begin{itemize}
% \item The patch requires \eTeX, its \cs{unexpanded} feature
%   is much too nice. It is possible to work around using token
%   registers. But the code becomes longer, slower, more difficult
%   to read and maintain. The package without option \xoption{patch}
%   works and will work without \eTeX.
% \item The code for the patch is quite long, there are many test
%   cases. Thus the probability for bugs is probably not too small.
% \item Since 2008/10/18 v3.0 package \xpackage{kvoptions-patch} is
%   available. Before option \xoption{patch} of package \xpackage{kvoptions}
%   must be used instead. I think, the solution as standalone package
%   \xpackage{kvoptions-patch} is cleaner and avoids option clashes.
% \end{itemize}
%
% \subsection{Option \xoption{debugshow}}
%
% The name of this option follows the convention of packages
% \xpackage{multicol}, \xpackage{tabularx}, and \xpackage{tracefnt}.
% Currently it prints the setting of boolean options, declared
% by \cs{DeclareBoolOption} in the \xfile{.log} file, if that
% boolean option is used.
% You can activate the option by
% \begin{itemize}
% \item |\PassOptionsToPackage{debugshow}{kvoptions}|\\
%       Put this somewhere before package \xpackage{kvoptions} is
%       loaded first, e.g. before \cs{documentclass}.
% \item |\RequirePackage[debugshow]{kvoptions}|\\
%       Before \cs{documentclass} even an author has to use
%       \cs{RequirePackage}. \cs{usepackage} only works after
%       \cs{documentclass}.
% \end{itemize}
% The preferred method is \cs{PassOptionsToPackage}, because
% it does not force the package loading and does not disturb,
% if the package is not loaded later at all.
%
% \section{Limitations}
%
% \subsection{Compatibility}
% \subsubsection{Package \xpackage{kvoptions-patch} vs. package \xpackage{xkvltxp}}
%
% Package \xpackage{xkvltxp} from the \xpackage{xkeyval} project
% has the same goal as package \xpackage{kvoptions-patch}
% and to patch \LaTeX's kernel commands in order to get better support
% for key value options.
% Of course they cannot be used both. The user must decide, which
% method he prefers. Package \xpackage{kvoptions-patch} aborts itself,
% if it detects that \xpackage{xkvltxp} is
% already loaded.
%
% However package \xpackage{xkvltxp} and \xpackage{kvoptions}
% can be used together, example:
% \begin{quote}
% |\usepackage{xkvltxp}|\\
% |\usepackage[...]{foobar} % |\texttt{\itshape foobar using kvoptions}
% \end{quote}
% The other way should work, too.
%
% Package \xpackage{kvoptions-patch} tries to catch more situations
% and to be more robust.
% For example, during the comparison of options it normalizes
% them by removing spaces around |=| and the value. Thus the
% following is not reported as option clash:
% \begin{quote}
%\begin{verbatim}
%\RequirePackage{kvoptions-patch}
%\documentclass{article}
%
%\usepackage[scaled=0.7]{helvet}
%\usepackage[scaled = 0.7]{helvet}
%
%\begin{document}
%\end{document}
%\end{verbatim}
% \end{quote}
%
% \subsection{Limitations}
%
% \subsubsection{Option comparisons}
%
% In some situations \LaTeX\ compares option lists, e.g. option clash check,
% \cs{@ifpackagewith}, or \cs{@ifclasswith}. Apart from catcode and
% sanitizing problems of option \xoption{patch}, there is another problem.
% \LaTeX\ does not know about the type and default values of
% options in key value style. Thus an option clash is reported,
% even if the key value has the same meaning:
% \begin{quote}
% |\usepackage[scaled]{helvet} %| \texttt{\textit{default is }.95}\\
% |\usepackage[.95]{helvet}|\\
% |\usepackage[0.95]{helvet}|
% \end{quote}
%
% \subsubsection{Option list parsing with package \xpackage{kvoptions-patch}}
%
% With package \xpackage{kvoptions-patch} the range of possible values
% in key value specifications is much large, for example
% the comma can be used, if enclosed in curly braces.
%
% Other packages, especially the packages that uses their
% own process option code can be surprised to find tokens
% inside options that they do not expect and errors would
% be the consequence. To avoid errors the options, especially
% the unused option list is sanitized. That means the list
% will only contain tokens with catcode 12 (other) and
% perhaps spaces (catcode 10). This allows a safe parsing
% for other packages. But a comma in the value part is
% no longer protected by curly braces because they have lost
% their special meaning. This is the price for compatibility.
%
% Example:
% \begin{quote}
% |\RequirePackage{kvoptions-patch}|\\
% |\documentclass[a={a,b,c},b]{article}|\\
% |\begin{document}|\\
% |\end{document}|
% \end{quote}
% Result:
% \begin{quote}
% |LaTeX Warning: Unused global option(s):|\\
% |    [a={a,c},b].|
% \end{quote}
%
% \StopEventually{
% }
%
% \section{Implementation}
%
% \subsection{Preamble}
%
%    \begin{macrocode}
%<*package>
%    \end{macrocode}
%
% \paragraph{Reload check and  identification.}
%    Reload check, especially if the package is not used with \LaTeX.
%    \begin{macrocode}
\begingroup\catcode61\catcode48\catcode32=10\relax%
  \catcode13=5 % ^^M
  \endlinechar=13 %
  \catcode35=6 % #
  \catcode39=12 % '
  \catcode44=12 % ,
  \catcode45=12 % -
  \catcode46=12 % .
  \catcode58=12 % :
  \catcode64=11 % @
  \catcode123=1 % {
  \catcode125=2 % }
  \expandafter\let\expandafter\x\csname ver@kvoptions.sty\endcsname
  \ifx\x\relax % plain-TeX, first loading
  \else
    \def\empty{}%
    \ifx\x\empty % LaTeX, first loading,
      % variable is initialized, but \ProvidesPackage not yet seen
    \else
      \expandafter\ifx\csname PackageInfo\endcsname\relax
        \def\x#1#2{%
          \immediate\write-1{Package #1 Info: #2.}%
        }%
      \else
        \def\x#1#2{\PackageInfo{#1}{#2, stopped}}%
      \fi
      \x{kvoptions}{The package is already loaded}%
      \aftergroup\endinput
    \fi
  \fi
\endgroup%
%    \end{macrocode}
%    Package identification:
%    \begin{macrocode}
\begingroup\catcode61\catcode48\catcode32=10\relax%
  \catcode13=5 % ^^M
  \endlinechar=13 %
  \catcode35=6 % #
  \catcode39=12 % '
  \catcode40=12 % (
  \catcode41=12 % )
  \catcode44=12 % ,
  \catcode45=12 % -
  \catcode46=12 % .
  \catcode47=12 % /
  \catcode58=12 % :
  \catcode64=11 % @
  \catcode91=12 % [
  \catcode93=12 % ]
  \catcode123=1 % {
  \catcode125=2 % }
  \expandafter\ifx\csname ProvidesPackage\endcsname\relax
    \def\x#1#2#3[#4]{\endgroup
      \immediate\write-1{Package: #3 #4}%
      \xdef#1{#4}%
    }%
  \else
    \def\x#1#2[#3]{\endgroup
      #2[{#3}]%
      \ifx#1\@undefined
        \xdef#1{#3}%
      \fi
      \ifx#1\relax
        \xdef#1{#3}%
      \fi
    }%
  \fi
\expandafter\x\csname ver@kvoptions.sty\endcsname
\ProvidesPackage{kvoptions}%
  [2016/05/16 v3.12 Key value format for package options (HO)]%
%    \end{macrocode}
%
% \paragraph{Catcodes}
%
%    \begin{macrocode}
\begingroup\catcode61\catcode48\catcode32=10\relax%
  \catcode13=5 % ^^M
  \endlinechar=13 %
  \catcode123=1 % {
  \catcode125=2 % }
  \catcode64=11 % @
  \def\x{\endgroup
    \expandafter\edef\csname KVO@AtEnd\endcsname{%
      \endlinechar=\the\endlinechar\relax
      \catcode13=\the\catcode13\relax
      \catcode32=\the\catcode32\relax
      \catcode35=\the\catcode35\relax
      \catcode61=\the\catcode61\relax
      \catcode64=\the\catcode64\relax
      \catcode123=\the\catcode123\relax
      \catcode125=\the\catcode125\relax
    }%
  }%
\x\catcode61\catcode48\catcode32=10\relax%
\catcode13=5 % ^^M
\endlinechar=13 %
\catcode35=6 % #
\catcode64=11 % @
\catcode123=1 % {
\catcode125=2 % }
\def\TMP@EnsureCode#1#2{%
  \edef\KVO@AtEnd{%
    \KVO@AtEnd
    \catcode#1=\the\catcode#1\relax
  }%
  \catcode#1=#2\relax
}
\TMP@EnsureCode{1}{14}% ^^A (comment)
\TMP@EnsureCode{2}{14}% ^^A (comment)
\TMP@EnsureCode{33}{12}% !
\TMP@EnsureCode{39}{12}% '
\TMP@EnsureCode{40}{12}% (
\TMP@EnsureCode{41}{12}% )
\TMP@EnsureCode{42}{12}% *
\TMP@EnsureCode{44}{12}% ,
\TMP@EnsureCode{45}{12}% -
\TMP@EnsureCode{46}{12}% .
\TMP@EnsureCode{47}{12}% /
\TMP@EnsureCode{58}{12}% :
\TMP@EnsureCode{62}{12}% >
\TMP@EnsureCode{91}{12}% [
\TMP@EnsureCode{93}{12}% ]
\TMP@EnsureCode{94}{7}% ^ (superscript)
\TMP@EnsureCode{96}{12}% `
\edef\KVO@AtEnd{\KVO@AtEnd\noexpand\endinput}
%    \end{macrocode}
%
% \paragraph{External resources.}
%    The package extends the support for key value pairs of
%    package \cs{keyval} to package options. Thus the package
%    needs to be loaded anyway. and we use it for
%    \cs{SetupKeyvalOptions}. AFAIK this does not disturb
%    users of \xpackage{xkeyval}.
%    \begin{macrocode}
\@ifundefined{define@key}{%
  \RequirePackage{keyval}\relax
}{}
%    \end{macrocode}
%
%    Macro \cs{DeclareLocalOptions} parses a comma separated key list
%    and uses \cs{comma@parse} of package \xpackage{kvsetkeys},
%    version 1.3.
%    \begin{macrocode}
\RequirePackage{ltxcmds}[2010/12/02]
\RequirePackage{kvsetkeys}[2007/09/29]
%    \end{macrocode}
%
% \paragraph{Provide macros for \plainTeX.}
%    \begin{macrocode}
\@ifundefined{@x@protect}{%
  \def\@x@protect#1\fi#2#3{%
    \fi\protect#1%
  }%
  \let\@typeset@protect\relax
}{}
\@ifundefined{@currname}{%
  \def\@currname{}%
}{}
\@ifundefined{@currext}{%
  \def\@currext{}%
}{}
%    \end{macrocode}
%
% \paragraph{Options}
%    Option \xoption{debugshow} enables additional lines of
%    code that prints information into the \xfile{.log}
%    file.
%    \begin{macrocode}
\DeclareOption{debugshow}{\catcode\@ne=9 }
%    \end{macrocode}
%
%    \begin{macrocode}
\DeclareOption{patch}{%
  \AtEndOfPackage{%
    \RequirePackage{kvoptions-patch}[2016/05/16]%
  }%
}
%    \end{macrocode}
%
%    Optionen auswerten:
%    \begin{macrocode}
\ProcessOptions\relax
%    \end{macrocode}
%
% \subsection{Option declaration macros}
%
% \subsubsection{\cs{SetupKeyvalOptions}}
%
%    The family for the key value pairs can be setup once
%    and is remembered later.
%    The package name seems a reasonable default for the
%    family key, if it is not set by the package author.
%
%    \begin{macro}{\KVO@family}
%    We cannot store the family setting in one macro,
%    because the package should be usable for many
%    other packages, too. Thus we remember the family
%    setting in a macro, whose name contains the
%    package name with extension, a key in \LaTeX's
%    class/package system.
%    \begin{macrocode}
\define@key{KVO}{family}{%
  \expandafter\edef\csname KVO@family@%
      \@currname.\@currext\endcsname{#1}%
}
\def\KVO@family{%
  \@ifundefined{KVO@family@\@currname.\@currext}{%
    \@currname
  }{%
    \csname KVO@family@\@currname.\@currext\endcsname
  }%
}
%    \end{macrocode}
%    \end{macro}
%
%    \begin{macro}{\KVO@prefix}
%    The value settings of options that are declared by \cs{DeclareBoolOption}
%    and \cs{DeclareStringOption} need to be saved in macros.
%    in the first case this is a switch \cs{if}\meta{prefix}\meta{key},
%    in the latter case a macro \cs{}\meta{prefix}\meta{key}.
%    The prefix can be configured, by \xoption{prefix} that is declared
%    here. The default is the package name with |@| appended.
%    \begin{macrocode}
\define@key{KVO}{prefix}{%
  \expandafter\edef\csname KVO@prefix@%
      \@currname.\@currext\endcsname{#1}%
}
\def\KVO@prefix{%
  \ltx@ifundefined{KVO@prefix@\@currname.\@currext}{%
    \@currname @%
  }{%
    \csname KVO@prefix@\@currname.\@currext\endcsname
  }%
}
%    \end{macrocode}
%    \end{macro}
%
%    \begin{macrocode}
\define@key{KVO}{setkeys}{%
  \expandafter\def\csname KVO@setkeys@%
      \@currname.\@currext\endcsname{#1}%
}
%    \end{macrocode}
%    \begin{macro}{\KVO@setkeys}
%    \begin{macrocode}
\def\KVO@setkeys{%
  \ltx@IfUndefined{KVO@setkeys@\@currname.\@currext}{%
    \setkeys
  }{%
    \csname KVO@setkeys@\@currname.\@currext\endcsname
  }%
}
%    \end{macrocode}
%    \end{macro}
%
%    \begin{macro}{\SetupKeyvalOptions}
%    The argument of \cs{SetupKeyvalOptions} expects a key value
%    list, known keys are \xoption{family} and \xoption{prefix}.
%    \begin{macrocode}
\newcommand*{\SetupKeyvalOptions}{%
  \kvsetkeys{KVO}%
}
%    \end{macrocode}
%    \end{macro}
%
% \subsubsection{\cs{DeclareBoolOption}}
%
%    \begin{macro}{\DeclareBoolOption}
%    Usually options of boolean type can be given by the
%    user without value and this means a setting to \emph{true}.
%    We follow this convention here. Also it simplifies the
%    user interface.
%
%    The switch is created and initialized with \emph{false}.
%    The default setting can be overwritten by the optional
%    argument.
%
%    \LaTeX's \cs{newif} does not check for already defined
%    macros, therefore we add this check here to prevent
%    the user from accidently redefining of \TeX's primitives
%    and other macros.
%    \begin{macrocode}
\newcommand*{\DeclareBoolOption}[2][false]{%
  \KVO@ifdefinable{if\KVO@prefix#2}{%
    \KVO@ifdefinable{\KVO@prefix#2true}{%
      \KVO@ifdefinable{\KVO@prefix#2false}{%
        \csname newif\expandafter\endcsname
        \csname if\KVO@prefix#2\endcsname
        \@ifundefined{\KVO@prefix#2#1}{%
          \PackageWarning{kvoptions}{%
            Initialization of option `#2' failed,\MessageBreak
            cannot set boolean option to `#1',\MessageBreak
            use `true' or `false', now using `false'%
          }%
        }{%
          \csname\KVO@prefix#2#1\endcsname
        }%
        \begingroup
          \edef\x{\endgroup
            \noexpand\define@key{\KVO@family}{#2}[true]{%
              \noexpand\KVO@boolkey{\@currname}%
              \ifx\@currext\@clsextension
                \noexpand\@clsextension
              \else
                \noexpand\@pkgextension
              \fi
              {\KVO@prefix}{#2}{####1}%
            }%
          }%
        \x
      }%
    }%
  }%
}
%    \end{macrocode}
%    \end{macro}
%
%    \begin{macro}{\DeclareComplementaryOption}
%    The first argument is the key name, the second the
%    key that must be a boolean option with the same
%    current family and prefix. A new switch is not
%    created for the new key, we have already a switch.
%    Instead we define switch setting commands to work
%    on the parent switch.
%    \begin{macrocode}
\newcommand*{\DeclareComplementaryOption}[2]{%
  \@ifundefined{if\KVO@prefix#2}{%
    \PackageError{kvoptions}{%
      Cannot generate option code for `#1',\MessageBreak
      parent switch `#2' does not exist%
    }{%
      You are inside %
      \ifx\@currext\@clsextension class\else package\fi\space
      `\@currname.\@currext'.\MessageBreak
      `\KVO@family' is used as familiy %
      for the keyval options.\MessageBreak
      `\KVO@prefix' serves as prefix %
      for internal switch macros.\MessageBreak
      \MessageBreak
      \@ehc
    }%
  }{%
    \KVO@ifdefinable{\KVO@prefix#1true}{%
      \KVO@ifdefinable{\KVO@prefix#1false}{%
        \expandafter\let\csname\KVO@prefix#1false\expandafter\endcsname
          \csname\KVO@prefix#2true\endcsname
        \expandafter\let\csname\KVO@prefix#1true\expandafter\endcsname
          \csname\KVO@prefix#2false\endcsname
%    \end{macrocode}
%    The same code part as in \cs{DeclareBoolOption} can
%    now be used.
%    \begin{macrocode}
        \begingroup
          \edef\x{\endgroup
            \noexpand\define@key{\KVO@family}{#1}[true]{%
              \noexpand\KVO@boolkey{\@currname}%
              \ifx\@currext\@clsextension
                \noexpand\@clsextension
              \else
                \noexpand\@pkgextension
              \fi
              {\KVO@prefix}{#1}{####1}%
            }%
          }%
        \x
      }%
    }%
  }%
}
%    \end{macrocode}
%    \end{macro}
%
%    \begin{macro}{\KVO@ifdefinable}
%    Generate the command token LaTeX's \cs{@ifdefinable} expects.
%    \begin{macrocode}
\def\KVO@ifdefinable#1{%
  \expandafter\@ifdefinable\csname #1\endcsname
}
%    \end{macrocode}
%    \end{macro}
%
%    \begin{macro}{\KVO@boolkey}
%    We check explicitly for |true| and |false| to prevent
%    the user from accidently calling other macros.
%    \begin{quote}
%    \begin{tabular}{@{}ll@{}}
%    |#1|& package/class name\\
%    |#2|& |\@pkgextension|/|\@clsextension|\\
%    |#3|& prefix \\
%    |#4|& key name\\
%    |#5|& new value
%    \end{tabular}
%    \end{quote}
%    \begin{macrocode}
\def\KVO@boolkey#1#2#3#4#5{%
  \edef\KVO@param{#5}%
  \ltx@onelevel@sanitize\KVO@param
  \ifx\KVO@param\KVO@true
    \expandafter\@firstofone
  \else
    \ifx\KVO@param\KVO@false
      \expandafter\expandafter\expandafter\@firstofone
    \else
      \ifx#2\@clsextension
        \expandafter\ClassWarning
      \else
        \expandafter\PackageWarning
      \fi
      {#1}{%
        Value `\KVO@param' is not supported by\MessageBreak
        option `#4'%
      }%
      \expandafter\expandafter\expandafter\@gobble
    \fi
  \fi
  {%
    ^^A\ifx#2\@clsextension
    ^^A  \expandafter\ClassInfo
    ^^A\else
    ^^A  \expandafter\PackageInfo
    ^^A\fi
    ^^A{#1}{[option] #4=\KVO@param}%
    \csname#3#4\KVO@param\endcsname
  }%
}
%    \end{macrocode}
%    \end{macro}
%    \begin{macro}{\KVO@true}
%    \begin{macro}{\KVO@false}
%    The macros \cs{KVO@true} and \cs{KVO@false} are used
%    for string comparisons. After \cs{ltx@onelevel@sanitize}
%    we have only tokens with catcode 12 (other).
%    \begin{macrocode}
\def\KVO@true{true}
\def\KVO@false{false}
\ltx@onelevel@sanitize\KVO@true
\ltx@onelevel@sanitize\KVO@false
%    \end{macrocode}
%    \end{macro}
%    \end{macro}
%
% \subsubsection{\cs{DeclareStringOption}}
%
%    \begin{macro}{\DeclareStringOption}
%    \begin{macrocode}
\newcommand*{\DeclareStringOption}[2][]{%
  \@ifnextchar[{%
    \KVO@DeclareStringOption{#1}{#2}@%
  }{%
    \KVO@DeclareStringOption{#1}{#2}{}[]%
  }%
}
%    \end{macrocode}
%    \end{macro}
%    \begin{macro}{\KVO@DeclareStringOption}
%    \begin{macrocode}
\def\KVO@DeclareStringOption#1#2#3[#4]{%
  \KVO@ifdefinable{\KVO@prefix#2}{%
    \@namedef{\KVO@prefix#2}{#1}%
    \begingroup
      \ifx\\#3\\%
        \toks@{}%
      \else
        \toks@{[{#4}]}%
      \fi
      \edef\x{\endgroup
        \noexpand\define@key{\KVO@family}{#2}\the\toks@{%
          ^^A\begingroup
          ^^A  \toks@{####1}%
          ^^A  \ifx\@currext\@clsextension
          ^^A    \noexpand\ClassInfo
          ^^A  \else
          ^^A    \noexpand\PackageInfo
          ^^A  \fi
          ^^A  {\@currname}{%
          ^^A    [option] #2={\noexpand\the\toks@}%
          ^^A  }%
          ^^A\endgroup
          \noexpand\def
          \expandafter\noexpand\csname\KVO@prefix#2\endcsname{####1}%
        }%
      }%
    \x
  }%
}
%    \end{macrocode}
%    \end{macro}
%
% \subsubsection{\cs{DeclareVoidOption}}
%
%    \begin{macro}{\DeclareVoidOption}
%    \begin{macrocode}
\newcommand*{\DeclareVoidOption}[2]{%
  \begingroup
    \let\next\@gobbletwo
    \KVO@ifdefinable{\KVO@prefix#1}{%
      \let\next\@firstofone
    }%
  \expandafter\endgroup
  \next{%
    \begingroup
      \edef\x{\endgroup
        \noexpand\define@key{\KVO@family}{#1}[\KVO@VOID@]{%
          \noexpand\KVO@voidkey{\@currname}%
          \ifx\@currext\@clsextension
            \noexpand\@clsextension
          \else
            \noexpand\@pkgextension
          \fi
          {#1}%
          {####1}%
          \expandafter\noexpand\csname\KVO@prefix#1\endcsname
        }%
      }%
    \x
    \begingroup
      \toks@{#2}%
    \expandafter\endgroup
    \expandafter\def
    \csname\KVO@prefix#1\expandafter\endcsname
    \expandafter{\the\toks@}%
  }%
}
\def\KVO@VOID@{@VOID@}
%    \end{macrocode}
%    \end{macro}
%    \begin{macro}{\KVO@voidkey}
%    \begin{tabular}{@{}ll@{}}
%    |#1|& package/class name\\
%    |#2|& |\@pkgextension|/|\@clsextension|\\
%    |#3|& key name\\
%    |#4|& default (|@VOID@|)\\
%    |#5|& macro with option code
%    \end{tabular}
%    \begin{macrocode}
\def\KVO@voidkey#1#2#3#4{%
  \def\CurrentOption{#3}%
  \begingroup
    \def\x{#4}%
  \expandafter\endgroup
  \ifx\x\KVO@VOID@
  \else
    \ifx#2\@clsextension
      \expandafter\ClassWarning
    \else
      \expandafter\PackageWarning
    \fi
    {#1}{%
      Unexpected value for option `#3'\MessageBreak
      is ignored%
    }%
  \fi
  ^^A\ifx#2\@clsextension
  ^^A  \expandafter\ClassInfo
  ^^A\else
  ^^A  \expandafter\PackageInfo
  ^^A\fi
  ^^A{#1}{[option] #3}%
}
%    \end{macrocode}
%    \end{macro}
%
% \subsubsection{\cs{DeclareDefaultOption}}
%
%    \begin{macro}{\DeclareDefaultOption}
%    \begin{macrocode}
\newcommand*{\DeclareDefaultOption}{%
  \@namedef{KVO@default@\@currname.\@currext}%
}
%    \end{macrocode}
%    \end{macro}
%
% \subsubsection{\cs{DeclareLocalOptions}}
%
%    \begin{macro}{\DeclareLocalOptions}
%    \begin{macrocode}
\newcommand*{\DeclareLocalOptions}[1]{%
  \comma@parse{#1}\KVO@DeclareLocalOption
}
%    \end{macrocode}
%    \end{macro}
%    \begin{macro}{\KVO@DeclareLocalOption}
%    \begin{macrocode}
\def\KVO@DeclareLocalOption#1{%
  \expandafter\def\csname KVO@local@\KVO@family @#1\endcsname{}%
}
%    \end{macrocode}
%    \end{macro}
%
% \subsection{Dynamic options}
%
% \subsubsection{\cs{DisableKeyvalOption}}
%
%    \begin{macrocode}
\SetupKeyvalOptions{%
  family=KVOdyn,%
  prefix=KVOdyn@%
}
\DeclareBoolOption[true]{global}
\DeclareComplementaryOption{local}{global}
\DeclareStringOption[undef]{action}
\let\KVOdyn@name\relax
\let\KVOdyn@ext\@empty
\define@key{KVOdyn}{class}{%
  \def\KVOdyn@name{#1}%
  \let\KVOdyn@ext\@clsextension
}
\define@key{KVOdyn}{package}{%
  \def\KVOdyn@name{#1}%
  \let\KVOdyn@ext\@pkgextension
}
\newcommand*{\DisableKeyvalOption}[3][]{%
  \begingroup
    \kvsetkeys{KVOdyn}{#1}%
    \def\x{\endgroup}%
    \@ifundefined{KVO@action@\KVOdyn@action}{%
      \PackageError{kvoptions}{%
        Unknown disable action %
        `\expandafter\strip@prefix\meaning\KVOdyn@action'\MessageBreak
        for option `#3' in keyval family '#2'%
      }\@ehc
    }{%
      \csname KVO@action@\KVOdyn@action\endcsname{#2}{#3}%
    }%
  \x
}
\def\KVO@action@undef#1#2{%
  \edef\x{\endgroup
    \ifKVOdyn@global\global\fi
    \let
    \expandafter\noexpand\csname KV@#1@#2\endcsname
    \relax
    \ifKVOdyn@global\global\fi
    \let
    \expandafter\noexpand\csname KV@#1@#2@default\endcsname
    \relax
  }%
  ^^A\PackageInfo{kvoptions}{%
  ^^A  [option] key `#2' of family `#1'\MessageBreak
  ^^A  is disabled (undef, \ifKVOdyn@global global\else local\fi)%
  ^^A}%
}
\def\KVO@action@ignore#1#2{%
  \edef\x{\endgroup
    \ifKVOdyn@global\global\fi
    \let
    \expandafter\noexpand\csname KV@#1@#2\endcsname
    \noexpand\@gobble
    \ifKVOdyn@global\global\fi
    \let
    \expandafter\noexpand\csname KV@#1@#2@default\endcsname
    \noexpand\@empty
  }%
  ^^A\PackageInfo{kvoptions}{%
  ^^A  [option] key `#2' of family `#1'\MessageBreak
  ^^A  is disabled (ignore, \ifKVOdyn@global global\else local\fi)%
  ^^A}%
}
\def\KVO@action@error{%
  \KVO@do@action{error}%
}
\def\KVO@action@warning{%
  \KVO@do@action{warning}%
}
%    \end{macrocode}
%    \begin{tabular}{@{}ll@{}}
%    |#1|& |error| or |warning|\\
%    |#2|& \meta{family}\\
%    |#3|& \meta{key}\\
%    \end{tabular}
%    \begin{macrocode}
\def\KVO@do@action#1#2#3{%
  \ifx\KVOdyn@name\relax
    \PackageError{kvoptions}{%
      Action type `#1' needs package/class name\MessageBreak
      for key `#3' in family `#2'%
    }\@ehc
  \else
    \edef\x{\endgroup
      \noexpand\define@key{#2}{#3}[]{%
        \expandafter\noexpand\csname KVO@disable@#1\endcsname
        {\KVOdyn@name}\noexpand\KVOdyn@ext{#3}%
      }%
      \ifKVOdyn@global
        \global\let
        \expandafter\noexpand\csname KV@#2@#3\endcsname
        \expandafter\noexpand\csname KV@#2@#3\endcsname
        \global\let
        \expandafter\noexpand\csname KV@#2@#3@default\endcsname
        \expandafter\noexpand\csname KV@#2@#3@default\endcsname
      \fi
    }%
    ^^A\ifx\KVOdyn@ext\@clsextension
    ^^A  \expandafter\ClassInfo
    ^^A\else
    ^^A   \expandafter\PackageInfo
    ^^A\fi
    ^^A{\KVOdyn@name}{%
    ^^A  [option] key `#3' of family `#2'\MessageBreak
    ^^A  is disabled (#1, \ifKVOdyn@global global\else local\fi)%
    ^^A}%
  \fi
}
\def\KVO@disable@error#1#2#3{%
  \ifx#2\@clsextension
    \expandafter\ClassError
  \else
    \expandafter\PackageError
  \fi
  {#1}{%
    Option `#3' is given too late,\MessageBreak
    now the option is ignored%
  }\@ehc
}
\def\KVO@disable@warning#1#2#3{%
  \ifx#2\@clsextension
    \expandafter\ClassWarning
  \else
    \expandafter\PackageWarning
  \fi
  {#1}{%
    Option `#3' is already consumed\MessageBreak
    and has no effect%
  }%
}
%    \end{macrocode}
%
% \subsection{Change option code}
%
% \subsubsection{\cs{AddToKeyvalOption}}
%
%    \begin{macro}{\AddToKeyvalOption}
%    \begin{macrocode}
\newcommand*{\AddToKeyvalOption}{%
  \@ifstar{%
    \begingroup
      \edef\x{\endgroup
        \noexpand\KVO@AddToKeyvalOption{\KVO@family}%
      }%
    \x
  }%
  \KVO@AddToKeyvalOption
}
%    \end{macrocode}
%    \end{macro}
%    \begin{macro}{\KVO@AddToKeyvalOption}
%    \begin{macrocode}
\def\KVO@AddToKeyvalOption#1#2{%
  \@ifundefined{KV@#1@#2}{%
    \PackageWarning{kvoptions}{%
      Key `#2' of family `#1' does not exist.\MessageBreak
      Ignoring \string\AddToKeyvalOption
    }%
    \@gobble
  }{%
    \edef\KVO@next{%
      \noexpand\KVO@@AddToKeyvalOption
      \expandafter\noexpand\csname KV@#1@#2\endcsname
    }%
    \afterassignment\KVO@next
    \def\KVO@temp##1%
  }%
}
%    \end{macrocode}
%    \end{macro}
%    \begin{macro}{\KVO@@AddToKeyvalOption}
%    \begin{macrocode}
\def\KVO@@AddToKeyvalOption#1{%
  \begingroup
    \toks@\expandafter{#1{##1}}%
    \toks@\expandafter{\the\expandafter\toks@\KVO@temp{##1}}%
    \edef\x{\endgroup
      \noexpand\def\noexpand#1####1{\the\toks@}%
    }%
  \x
}
%    \end{macrocode}
%    \end{macro}
%
% \subsection{Process options}
%
% \subsubsection{Get global options}
%
%    Package \xpackage{xkeyval} removes options with equal signs from
%    the global options (\cs{@classoptionslist}). The effect is that
%    other packages and classes will not see these global options anymore.
%    A bug-report was answered that this behaviour is ``by design''.
%    Thus I call it a design bug. Now getting the global options require
%    an algorithm instead of a simple macro call.
%    \begin{macrocode}
%</package>
%<*package|patch>
%    \end{macrocode}
%    \begin{macro}{\KVO@IfDefThen}
%    Call |#2| if command |#1| is defined and not \cs{relax}.
%    (Package \xpackage{kvoptions-patch} does not load
%    package \xpackage{ltxcmds}.)
%    \begin{macrocode}
\def\KVO@IfDefThen#1#2{%
  \ifx#1\ltx@undefined
  \else
    \ifx#1\relax
    \else
      #2%
    \fi
  \fi
}%
%    \end{macrocode}
%    \end{macro}
%    \begin{macro}{\KVO@GetClassOptionsList}
%    \begin{macrocode}
\def\KVO@GetClassOptionsList{%
  \let\KVO@classoptionslist\@classoptionslist
  \KVO@IfDefThen\@classoptionslist{%
    \KVO@IfDefThen\XKV@documentclass{%
      \ifx\XKV@documentclass\ltx@empty
      \else
        \KVO@IfDefThen\XKV@classoptionslist{%
          \ifx\XKV@classoptionslist\ltx@empty
          \else
            \let\KVO@classoptionslist\XKV@classoptionslist
          \fi
        }%
      \fi
    }%
  }%
}%
%    \end{macrocode}
%    \end{macro}
%    \begin{macrocode}
%</package|patch>
%<*package>
%    \end{macrocode}
%
% \subsubsection{\cs{ProcessKeyvalOptions}}
%
%    \begin{macro}{\ProcessKeyvalOptions}
%    If the optional star is given, we get the family
%    name and expand it for safety.
%    \begin{macrocode}
\newcommand*{\ProcessKeyvalOptions}{%
  \@ifstar{%
    \begingroup
      \edef\x{\endgroup
        \noexpand\KVO@ProcessKeyvalOptions{\KVO@family}%
      }%
    \x
  }%
  \KVO@ProcessKeyvalOptions
}
%    \end{macrocode}
%    \end{macro}
%
%    \begin{macrocode}
\def\KVO@ProcessKeyvalOptions#1{%
  \let\@tempc\relax
  \let\KVO@temp\@empty
%    \end{macrocode}
%    Add any global options that are known to KV to the start of
%    the list being built in |\KVO@temp| and mark them used (by
%    removing them from the unused option list).
%    \begin{macrocode}
  \ifx\@currext\@clsextension
  \else
    \KVO@GetClassOptionsList
    \ifx\KVO@classoptionslist\relax
    \else
      \@for\KVO@CurrentOption:=\KVO@classoptionslist\do{%
        \@ifundefined{KV@#1@\expandafter\KVO@getkey
                      \KVO@CurrentOption=\@nil}{%
        }{%
          \@ifundefined{KVO@local@#1@\expandafter\KVO@getkey
                        \KVO@CurrentOption=\@nil}{%
            \ifx\KVO@Patch Y%
              \edef\KVO@temp{%
                \etex@unexpanded\expandafter{%
                  \KVO@temp
                }%
                ,%
                \etex@unexpanded\expandafter{%
                  \KVO@CurrentOption
                }%
                ,%
              }%
              \ltx@onelevel@sanitize\KVO@CurrentOption
            \else
              \edef\KVO@temp{%
                \KVO@temp
                ,%
                \KVO@CurrentOption
                ,%
              }%
            \fi
            \@expandtwoargs\@removeelement\KVO@CurrentOption
              \@unusedoptionlist\@unusedoptionlist
          }{}%
        }%
      }%
    \fi
  \fi
%    \end{macrocode}
%     Now stick the package options at the end of the list and wrap
%     in a call to \cs{setkeys}. A class ignores unknown global
%     options, we must remove them to prevent error messages
%     from \cs{setkeys}.
%    \begin{macrocode}
  \begingroup
    \toks\tw@{}%
    \@ifundefined{opt@\@currname.\@currext}{%
      \toks@\expandafter{\KVO@temp}%
    }{%
      \toks@\expandafter\expandafter\expandafter{%
        \csname opt@\@currname.\@currext\endcsname
      }%
      \ifx\@currext\@clsextension
        \edef\CurrentOption{\the\toks@}%
        \toks@\expandafter{\KVO@temp}%
        \@for\CurrentOption:=\CurrentOption\do{%
          \@ifundefined{%
            KV@#1@\expandafter\KVO@getkey\CurrentOption=\@nil
          }{%
%    \end{macrocode}
%    A class puts not used options in the unused option list
%    unless there is a default handler.
%    \begin{macrocode}
            \@ifundefined{KVO@default@\@currname.\@currext}{%
              \ifx\KVO@Patch Y%
                \ltx@onelevel@sanitize\CurrentOption
              \fi
              \ifx\@unusedoptionlist\@empty
                \global\let\@unusedoptionlist\CurrentOption
              \else
                \expandafter\expandafter\expandafter\gdef
                \expandafter\expandafter\expandafter\@unusedoptionlist
                \expandafter\expandafter\expandafter{%
                  \expandafter\@unusedoptionlist
                  \expandafter,\CurrentOption
                }%
              \fi
            }{%
              \toks\tw@\expandafter{%
                \the\toks\expandafter\tw@\expandafter,\CurrentOption
              }%
            }%
          }{%
            \toks@\expandafter{%
              \the\expandafter\toks@\expandafter,\CurrentOption
            }%
          }%
        }%
      \else
%    \end{macrocode}
%    Without default action we pass all options to \cs{setkeys}.
%    Otherwise we have to check which options are known.
%    These are passed to \cs{setkeys}. For the others the default
%    action is performed.
%    \begin{macrocode}
        \@ifundefined{KVO@default@\@currname.\@currext}{%
          \toks@\expandafter\expandafter\expandafter{%
            \expandafter\KVO@temp\the\toks@
          }%
        }{%
          \edef\CurrentOption{\the\toks@}%
          \toks@\expandafter{\KVO@temp}%
          \@for\CurrentOption:=\CurrentOption\do{%
            \@ifundefined{%
              KV@#1@\expandafter\KVO@getkey\CurrentOption=\@nil
            }{%
              \toks\tw@\expandafter{%
                \the\toks\expandafter\tw@\expandafter,\CurrentOption
              }%
            }{%
              \toks@\expandafter{%
                \the\expandafter\toks@\expandafter,\CurrentOption
              }%
            }%
          }%
        }%
      \fi
    }%
    \edef\KVO@temp{\endgroup
      \noexpand\KVO@calldefault{\the\toks\tw@}%
      \noexpand\KVO@setkeys{#1}{\the\toks@}%
    }%
  \KVO@temp
%    \end{macrocode}
%    Some cleanup of \cs{ProcessOptions}.
%    \begin{macrocode}
  \let\CurrentOption\@empty
  \AtEndOfPackage{\let\@unprocessedoptions\relax}%
}
%    \end{macrocode}
%
% \subsubsection{\cs{ProcessLocalKeyvalOptions}}
%
%    \begin{macro}{\ProcessLocalKeyvalOptions}
%    If the optional star is given, we get the family
%    name and expand it for safety.
%    \begin{macrocode}
\newcommand*{\ProcessLocalKeyvalOptions}{%
  \@ifstar{%
    \begingroup
      \edef\x{\endgroup
        \noexpand\KVO@ProcessLocalKeyvalOptions{\KVO@family}%
      }%
    \x
  }%
  \KVO@ProcessLocalKeyvalOptions
}
%    \end{macrocode}
%    \end{macro}
%
%    \begin{macrocode}
\def\KVO@ProcessLocalKeyvalOptions#1{%
  \let\@tempc\relax
  \let\KVO@temp\@empty
%    \end{macrocode}
%    Check if \cs{ProcessLocalKeyvalOptions} is called inside
%    a package.
%    \begin{macrocode}
  \ifx\@currext\@pkgextension
  \else
    \PackageError{kvoptions}{%
      \string\ProcessLocalKeyvalOptions\space is intended for packages only%
    }\@ehc
  \fi
%    \end{macrocode}
%     The package options are put into toks register \cs{toks@}.
%    \begin{macrocode}
  \begingroup
    \toks\tw@{}%
    \@ifundefined{opt@\@currname.\@currext}{%
      \toks@\expandafter{\KVO@temp}%
    }{%
      \toks@\expandafter\expandafter\expandafter{%
        \csname opt@\@currname.\@currext\endcsname
      }%
%    \end{macrocode}
%    Without default action we pass all options to \cs{setkeys}.
%    Otherwise we have to check which options are known.
%    These are passed to \cs{setkeys}. For the others the default
%    action is performed.
%    \begin{macrocode}
      \@ifundefined{KVO@default@\@currname.\@currext}{%
        \toks@\expandafter\expandafter\expandafter{%
          \expandafter\KVO@temp\the\toks@
        }%
      }{%
        \edef\CurrentOption{\the\toks@}%
        \toks@\expandafter{\KVO@temp}%
        \@for\CurrentOption:=\CurrentOption\do{%
          \@ifundefined{%
            KV@#1@\expandafter\KVO@getkey\CurrentOption=\@nil
          }{%
            \toks\tw@\expandafter{%
              \the\toks\expandafter\tw@\expandafter,\CurrentOption
            }%
          }{%
            \toks@\expandafter{%
              \the\expandafter\toks@\expandafter,\CurrentOption
            }%
          }%
        }%
      }%
    }%
    \edef\KVO@temp{\endgroup
      \noexpand\KVO@calldefault{\the\toks\tw@}%
      \noexpand\KVO@setkeys{#1}{\the\toks@}%
    }%
  \KVO@temp
%    \end{macrocode}
%    Some cleanup of \cs{ProcessOptions}.
%    \begin{macrocode}
  \let\CurrentOption\@empty
  \AtEndOfPackage{\let\@unprocessedoptions\relax}%
}
%    \end{macrocode}
%
% \subsubsection{Helper macros}
%
%    \begin{macro}{\KVO@getkey}
%    Extract the key part of a key=value pair.
%    \begin{macrocode}
\def\KVO@getkey#1=#2\@nil{#1}
%    \end{macrocode}
%    \end{macro}
%
%    \begin{macro}{\KVO@calldefault}
%    \begin{macrocode}
\def\KVO@calldefault#1{%
  \begingroup
    \def\x{#1}%
  \expandafter\endgroup
  \ifx\x\@empty
  \else
    \@for\CurrentOption:=#1\do{%
      \ifx\CurrentOption\@empty
      \else
        \expandafter\KVO@setcurrents\CurrentOption=\@nil
        \@nameuse{KVO@default@\@currname.\@currext}%
      \fi
    }%
  \fi
}
%    \end{macrocode}
%    \end{macro}
%    \begin{macro}{\KVO@setcurrents}
%    Extract the key part of a key=value pair.
%    \begin{macrocode}
\def\KVO@setcurrents#1=#2\@nil{%
  \def\CurrentOptionValue{#2}%
  \ifx\CurrentOptionValue\@empty
    \let\CurrentOptionKey\CurrentOption
    \let\CurrentOptionValue\relax
  \else
    \edef\CurrentOptionKey{\zap@space#1 \@empty}%
    \expandafter\KVO@setcurrentvalue\CurrentOption\@nil
  \fi
}
%    \end{macrocode}
%    \end{macro}
%    \begin{macro}{\KV@setcurrentvalue}
%    Here the value part is parsed. Package \xpackage{keyval}'s
%    \cs{KV@@sp@def} helps in removing spaces at the begin and
%    end of the value.
%    \begin{macrocode}
\def\KVO@setcurrentvalue#1=#2\@nil{%
  \KV@@sp@def\CurrentOptionValue{#2}%
}
%    \end{macrocode}
%    \end{macro}
%
% \subsection{\plainTeX}
%
%    Disable \LaTeX\ stuff.
%    \begin{macrocode}
\begingroup\expandafter\expandafter\expandafter\endgroup
\expandafter\ifx\csname documentclass\endcsname\relax
  \def\ProcessKeyvalOptions{%
    \@ifstar{}\@gobble
  }%
\fi
%    \end{macrocode}
%
%    \begin{macrocode}
\KVO@AtEnd%
%</package>
%    \end{macrocode}
%
% \subsection{Package \xpackage{kvoptions-patch}}
%
%    \begin{macrocode}
%<*patch>
\NeedsTeXFormat{LaTeX2e}
\begingroup\catcode61\catcode48\catcode32=10\relax%
  \catcode13=5 % ^^M
  \endlinechar=13 %
  \catcode123=1 % {
  \catcode125=2 % }
  \catcode64=11 % @
  \def\x{\endgroup
    \expandafter\edef\csname KVO@AtEnd\endcsname{%
      \endlinechar=\the\endlinechar\relax
      \catcode13=\the\catcode13\relax
      \catcode32=\the\catcode32\relax
      \catcode35=\the\catcode35\relax
      \catcode61=\the\catcode61\relax
      \catcode64=\the\catcode64\relax
      \catcode123=\the\catcode123\relax
      \catcode125=\the\catcode125\relax
    }%
  }%
\x\catcode61\catcode48\catcode32=10\relax%
\catcode13=5 % ^^M
\endlinechar=13 %
\catcode35=6 % #
\catcode64=11 % @
\catcode123=1 % {
\catcode125=2 % }
\def\TMP@EnsureCode#1#2{%
  \edef\KVO@AtEnd{%
    \KVO@AtEnd
    \catcode#1=\the\catcode#1\relax
  }%
  \catcode#1=#2\relax
}
\TMP@EnsureCode{39}{12}% '
\TMP@EnsureCode{40}{12}% (
\TMP@EnsureCode{41}{12}% )
\TMP@EnsureCode{43}{12}% +
\TMP@EnsureCode{44}{12}% ,
\TMP@EnsureCode{45}{12}% -
\TMP@EnsureCode{46}{12}% .
\TMP@EnsureCode{47}{12}% /
\TMP@EnsureCode{58}{12}% :
\TMP@EnsureCode{60}{12}% <
\TMP@EnsureCode{62}{12}% >
\TMP@EnsureCode{91}{12}% [
\TMP@EnsureCode{93}{12}% ]
\TMP@EnsureCode{96}{12}% `
\TMP@EnsureCode{124}{12}% |
\edef\KVO@AtEnd{\KVO@AtEnd\noexpand\endinput}
\ProvidesPackage{kvoptions-patch}%
  [2016/05/16 v3.12 LaTeX patch for keyval options (HO)]%
%    \end{macrocode}
%
%    Check for \eTeX.
%    \begin{macrocode}
\begingroup\expandafter\expandafter\expandafter\endgroup
\expandafter\ifx\csname eTeXversion\endcsname\relax
  \PackageWarningNoLine{kvoptions-patch}{%
    Package loading is aborted, because e-TeX is missing%
  }%
  \expandafter\KVO@AtEnd
\fi%
%    \end{macrocode}
%
%    Package \xpackage{etexcmds} for \cs{etex@unexpanded}.
%    \begin{macrocode}
\RequirePackage{etexcmds}[2007/09/09]
\ifetex@unexpanded
\else
  \PackageError{kvoptions-patch}{%
    Could not find eTeX's \string\unexpanded.\MessageBreak
    Try adding \string\RequirePackage\string{etexcmds\string} %
    before \string\documentclass%
  }\@ehd
  \expandafter\KVO@AtEnd
\fi%
%    \end{macrocode}
%
%    Check for package \xpackage{xkvltxp}.
%    \begin{macrocode}
\@ifpackageloaded{xkvltxp}{%
  \PackageWarningNoLine{kvoptions}{%
    Option `patch' cannot be used together with\MessageBreak
    package `xkvltxp' that is already loaded.\MessageBreak
    Therefore package loading is aborted%
  }%
  \KVO@AtEnd
}{}%
%    \end{macrocode}
%
%    \begin{macrocode}
\def\@if@ptions#1#2#3{%
  \begingroup
    \KVO@normalize\KVO@temp{#3}%
    \edef\x{\endgroup
      \noexpand\@if@pti@ns{%
        \detokenize\expandafter\expandafter\expandafter{%
          \csname opt@#2.#1\endcsname
        }%
      }{%
        \detokenize\expandafter{\KVO@temp}%
      }%
    }%
  \x
}
%    \end{macrocode}
%
%    \begin{macrocode}
\def\@pass@ptions#1#2#3{%
  \KVO@normalize\KVO@temp{#2}%
  \@ifundefined{opt@#3.#1}{%
    \expandafter\gdef\csname opt@#3.#1%
          \expandafter\endcsname\expandafter{%
      \KVO@temp
    }%
  }{%
    \expandafter\gdef\csname opt@#3.#1%
          \expandafter\expandafter\expandafter\endcsname
          \expandafter\expandafter\expandafter{%
      \csname opt@#3.#1\expandafter\endcsname\expandafter,\KVO@temp
    }%
  }%
}
%    \end{macrocode}
%
%    \begin{macrocode}
\def\ProcessOptions{%
  \let\ds@\@empty
  \@ifundefined{opt@\@currname.\@currext}{%
    \let\@curroptions\@empty
  }{%
    \expandafter\expandafter\expandafter\def
    \expandafter\expandafter\expandafter\@curroptions
    \expandafter\expandafter\expandafter{%
      \csname opt@\@currname.\@currext\endcsname
    }%
  }%
  \@ifstar\KVO@xprocess@ptions\KVO@process@ptions
}
%    \end{macrocode}
%
%    \begin{macrocode}
\def\KVO@process@ptions{%
  \@for\CurrentOption:=\@declaredoptions\do{%
    \ifx\CurrentOption\@empty
    \else
      \begingroup
        \ifx\@currext\@clsextension
          \toks@{}%
        \else
          \KVO@GetClassOptionsList
          \toks@\expandafter{\KVO@classoptionslist,}%
        \fi
        \toks\tw@\expandafter{\@curroptions}%
        \edef\x{\endgroup
          \noexpand\in@{,\CurrentOption,}{,\the\toks@\the\toks\tw@,}%
        }%
      \x
      \ifin@
        \KVO@use@ption
        \expandafter\let\csname ds@\CurrentOption\endcsname\@empty
      \fi
    \fi
  }%
  \KVO@process@pti@ns
}
%    \end{macrocode}
%
%    \begin{macrocode}
\def\KVO@xprocess@ptions{%
  \ifx\@currext\@clsextension
  \else
    \KVO@GetClassOptionsList
    \@for\CurrentOption:=\KVO@classoptionslist\do{%
      \ifx\CurrentOption\@empty
      \else
        \KVO@in@\CurrentOption\@declaredoptions
        \ifin@
          \KVO@use@ption
          \expandafter\let\csname ds@\CurrentOption\endcsname\@empty
        \fi
      \fi
    }%
  \fi
  \KVO@process@pti@ns
}
%    \end{macrocode}
%
%    \begin{macrocode}
\def\KVO@in@#1#2{%
  \in@false
  \begingroup
    \@for\x:=#2\do{%
      \ifx\x#1\relax
        \in@true
      \fi
    }%
    \edef\x{\endgroup
      \ifin@
        \noexpand\in@true
      \fi
    }%
  \x
}
%    \end{macrocode}
%
%    \begin{macrocode}
\def\KVO@process@pti@ns{%
  \@for\CurrentOption:=\@curroptions\do{%
    \@ifundefined{ds@\KVO@SanitizedCurrentOption}{%
      \KVO@use@ption
      \default@ds
    }%
    \KVO@use@ption
  }%
  \@for\CurrentOption:=\@declaredoptions\do{%
    \expandafter\let\csname ds@\CurrentOption\endcsname\relax
  }%
  \let\CurrentOption\@empty
  \let\@fileswith@pti@ns\@@fileswith@pti@ns
  \AtEndOfPackage{\let\@unprocessedoptions\relax}%
}
%    \end{macrocode}
%
%    \begin{macrocode}
\def\KVO@use@ption{%
  \begingroup
    \edef\x{\endgroup
      \noexpand\@removeelement{%
        \detokenize\expandafter{\CurrentOption}%
      }{%
        \detokenize\expandafter{\@unusedoptionlist}%
      }%
    }%
  \x\@unusedoptionlist
  \csname ds@\KVO@SanitizedCurrentOption\endcsname
}
%    \end{macrocode}
%
%    \begin{macrocode}
\def\OptionNotUsed{%
  \ifx\@currext\@clsextension
    \xdef\@unusedoptionlist{%
      \ifx\@unusedoptionlist\@empty
      \else
        \detokenize\expandafter{\@unusedoptionlist,}%
      \fi
      \detokenize\expandafter{\CurrentOption}%
    }%
  \fi
}
%    \end{macrocode}
%
%    Variant of \cs{ExecuteOptions} that better protects \cs{CurrentOption}.
%    \begin{macrocode}
\def\CurrentOption@SaveLevel{0}
\def\ExecuteOptions{%
  \expandafter\KVO@ExecuteOptions
      \csname CurrentOption@\CurrentOption@SaveLevel\endcsname
}
\def\KVO@ExecuteOptions#1#2{%
  \let#1\CurrentOption
  \edef\CurrentOption@SaveLevel{%
    \the\numexpr\CurrentOption@SaveLevel+1%
  }%
  \@for\CurrentOption:=#2\do{%
    \csname ds@\CurrentOption\endcsname
  }%
  \edef\CurrentOption@SaveLevel{%
    \the\numexpr\CurrentOption@SaveLevel-1%
  }%
  \let\CurrentOption#1%
}
%    \end{macrocode}
%
%    \begin{macrocode}
\def\KVO@fileswith@pti@ns#1[#2]#3[#4]{%
  \ifx#1\@clsextension
    \ifx\@classoptionslist\relax
      \KVO@normalize\KVO@temp{#2}%
      \expandafter\gdef\expandafter\@classoptionslist\expandafter{%
        \KVO@temp
      }%
      \def\reserved@a{%
        \KVO@onefilewithoptions{#3}[{#2}][{#4}]#1%
        \@documentclasshook
      }%
    \else
      \def\reserved@a{%
        \KVO@onefilewithoptions{#3}[{#2}][{#4}]#1%
      }%
    \fi
  \else
    \begingroup
      \let\KVO@temp\relax
      \let\KVO@onefilewithoptions\relax
      \let\@pkgextension\relax
      \def\reserved@b##1,{%
        \ifx\@nil##1\relax
        \else
          \ifx\relax##1\relax
          \else
            \KVO@onefilewithoptions{##1}[{\KVO@temp}][{#4}]%
            \@pkgextension
          \fi
          \expandafter\reserved@b
        \fi
      }%
      \edef\reserved@a{\zap@space#3 \@empty}%
      \edef\reserved@a{\expandafter\reserved@b\reserved@a,\@nil,}%
      \toks@{#2}%
      \def\KVO@temp{\the\toks@}%
    \edef\reserved@a{\endgroup \reserved@a}%
  \fi
  \reserved@a
}
%    \end{macrocode}
%
%    \begin{macrocode}
\def\KVO@onefilewithoptions#1[#2][#3]#4{%
  \@pushfilename
  \xdef\@currname{#1}%
  \global\let\@currext#4%
  \expandafter\let\csname\@currname.\@currext-h@@k\endcsname\@empty
  \let\CurrentOption\@empty
  \@reset@ptions
  \makeatletter
  \def\reserved@a{%
    \@ifl@aded\@currext{#1}{%
      \@if@ptions\@currext{#1}{#2}{%
      }{%
        \begingroup
          \@ifundefined{opt@#1.\@currext}{%
            \def\x{}%
          }{%
            \edef\x{%
              \expandafter\expandafter\expandafter\strip@prefix
              \expandafter\meaning\csname opt@#1.\@currext\endcsname
            }%
          }%
          \def\y{#2}%
          \edef\y{\expandafter\strip@prefix\meaning\y}%
          \@latex@error{Option clash for \@cls@pkg\space #1}{%
            The package #1 has already been loaded %
            with options:\MessageBreak
            \space\space[\x]\MessageBreak
            There has now been an attempt to load it %
             with options\MessageBreak
            \space\space[\y]\MessageBreak
            Adding the global options:\MessageBreak
            \space\space
                 \x,\y\MessageBreak
            to your \noexpand\documentclass declaration may fix this.%
            \MessageBreak
            Try typing \space <return> \space to proceed.%
          }%
        \endgroup
      }%
    }{%
      \@pass@ptions\@currext{#2}{#1}%
      \global\expandafter
      \let\csname ver@\@currname.\@currext\endcsname\@empty
      \InputIfFileExists
        {\@currname.\@currext}%
        {}%
        {\@missingfileerror\@currname\@currext}%
      \let\@unprocessedoptions\@@unprocessedoptions
      \csname\@currname.\@currext-h@@k\endcsname
      \expandafter\let\csname\@currname.\@currext-h@@k\endcsname
              \@undefined
      \@unprocessedoptions
    }%
    \@ifl@ter\@currext{#1}{#3}{%
    }{%
      \@latex@warning@no@line{%
        You have requested,\on@line, %
        version\MessageBreak
          #3' of \@cls@pkg\space #1,\MessageBreak
        but only version\MessageBreak
         `\csname ver@#1.\@currext\endcsname'\MessageBreak
        is available%
      }%
    }%
    \ifx\@currext\@clsextension\let\LoadClass\@twoloadclasserror\fi
    \@popfilename
    \@reset@ptions
  }%
  \reserved@a
}
%    \end{macrocode}
%
%    \begin{macrocode}
\def\@unknownoptionerror{%
  \@latex@error{%
    Unknown option `\KVO@SanitizedCurrentOption' %
    for \@cls@pkg\space`\@currname'%
  }{%
    The option `\KVO@SanitizedCurrentOption' was not declared in %
    \@cls@pkg\space`\@currname', perhaps you\MessageBreak
    misspelled its name. %
    Try typing \space <return> %
    \space to proceed.%
  }%
}
%    \end{macrocode}
%
%    \begin{macrocode}
\def\@@unprocessedoptions{%
  \ifx\@currext\@pkgextension
    \@ifundefined{opt@\@currname.\@currext}{%
      \let\@curroptions\@empty
    }{%
      \expandafter\let\expandafter\@curroptions
          \csname opt@\@currname.\@currext\endcsname
    }%
    \@for\CurrentOption:=\@curroptions\do{%
        \ifx\CurrentOption\@empty\else\@unknownoptionerror\fi
    }%
  \fi
}
%    \end{macrocode}
%
%    \begin{macrocode}
\def\KVO@SanitizedCurrentOption{%
  \expandafter\strip@prefix\meaning\CurrentOption
}
%    \end{macrocode}
%
%    Normalize option list.
%    \begin{macrocode}
\def\KVO@normalize#1#2{%
  \let\KVO@result\@empty
  \KVO@splitcomma#2,\@nil
  \let#1\KVO@result
}
\def\KVO@splitcomma#1,#2\@nil{%
  \KVO@ifempty{#1}{}{%
    \KVO@checkkv#1=\@nil
  }%
  \KVO@ifempty{#2}{}{\KVO@splitcomma#2\@nil}%
}
\def\KVO@ifempty#1{%
  \expandafter\ifx\expandafter\\\detokenize{#1}\\%
    \expandafter\@firstoftwo
  \else
    \expandafter\@secondoftwo
  \fi
}
\def\KVO@checkkv#1=#2\@nil{%
  \KVO@ifempty{#2}{%
    % option without value
    \edef\KVO@x{\zap@space#1 \@empty}%
    \ifx\KVO@x\@empty
      % ignore empty option
    \else
      % append to list
      \edef\KVO@result{%
        \etex@unexpanded\expandafter{\KVO@result},\KVO@x
      }%
    \fi
  }{%
    % #1: "key", #2: "value="
    % add key part
    \edef\KVO@result{%
      \etex@unexpanded\expandafter{\KVO@result},%
      \zap@space#1 \@empty
    }%
    \futurelet\@let@token\KVO@checkfirsttok#2 \@nil| = \@nil|\KVO@nil
  }%
}
\def\KVO@checkfirsttok{%
  \ifx\@let@token\bgroup
    % no space at start
    \expandafter\KVO@removelastspace\expandafter=%
    % "<value><spaceopt>= \@nil"
  \else
    \expandafter\KVO@checkfirstA
  \fi
}
\def\KVO@checkfirstA#1 #2\@nil{%
  \KVO@ifempty{#2}{%
    \KVO@removelastspace=#1 \@nil
  }{%
    \KVO@ifempty{#1}{%
      \KVO@removelastspace=#2\@nil
    }{%
      \KVO@removelastspace=#1 #2\@nil
    }%
  }%
}
\def\KVO@removelastspace#1 = \@nil|#2\KVO@nil{%
  \KVO@ifempty{#2}{%
    \edef\KVO@result{%
      \etex@unexpanded\expandafter{\KVO@result}%
      \etex@unexpanded\expandafter{\KVO@removegarbage#1\KVO@nil}%
    }%
  }{%
    \edef\KVO@result{%
      \etex@unexpanded\expandafter{\KVO@result}%
      \etex@unexpanded{#1}%
    }%
  }%
}
\def\KVO@removegarbage#1= \@nil#2\KVO@nil{#1}%
%    \end{macrocode}
%
%    Arguments |#1| and |#2| are macros.
%    \begin{macrocode}
\def\KVO@removeelement#1#2{%
  \begingroup
    \toks@={}%
    \@for\x:=#2\do{%
      \ifx\x\@empty
      \else
        \ifx\x#1\relax
        \else
          \edef\t{\the\toks@}%
          \ifx\t\@empty
          \else
            \toks@\expandafter{\the\toks@,}%
          \fi
          \toks@\expandafter{\the\expandafter\toks@\x}%
        \fi
      \fi
    }%
    \edef\x{\endgroup
      \def\noexpand#2{\the\toks@}%
    }%
  \x
}
%    \end{macrocode}
%
%
%    \begin{macrocode}
\let\@@fileswith@pti@ns\KVO@fileswith@pti@ns
\ifx\@fileswith@pti@ns\@badrequireerror
\else
  \let\@fileswith@pti@ns\KVO@fileswith@pti@ns
\fi
%    \end{macrocode}
%
%    \begin{macro}{\KVO@Patch}
%    \begin{macrocode}
\let\KVO@Patch=Y
%    \end{macrocode}
%    \end{macro}
%
%    \begin{macrocode}
\KVO@AtEnd%
%</patch>
%    \end{macrocode}
%
% \section{Test}
%
% \subsection{Preface for standard catcode check}
%
%    \begin{macrocode}
%<*test1>
\input miniltx.tex\relax
%</test1>
%    \end{macrocode}
%
% \subsection{Catcode checks for loading}
%
%    \begin{macrocode}
%<*test1>
%    \end{macrocode}
%    \begin{macrocode}
\catcode`\{=1 %
\catcode`\}=2 %
\catcode`\#=6 %
\catcode`\@=11 %
\expandafter\ifx\csname count@\endcsname\relax
  \countdef\count@=255 %
\fi
\expandafter\ifx\csname @gobble\endcsname\relax
  \long\def\@gobble#1{}%
\fi
\expandafter\ifx\csname @firstofone\endcsname\relax
  \long\def\@firstofone#1{#1}%
\fi
\expandafter\ifx\csname loop\endcsname\relax
  \expandafter\@firstofone
\else
  \expandafter\@gobble
\fi
{%
  \def\loop#1\repeat{%
    \def\body{#1}%
    \iterate
  }%
  \def\iterate{%
    \body
      \let\next\iterate
    \else
      \let\next\relax
    \fi
    \next
  }%
  \let\repeat=\fi
}%
\def\RestoreCatcodes{}
\count@=0 %
\loop
  \edef\RestoreCatcodes{%
    \RestoreCatcodes
    \catcode\the\count@=\the\catcode\count@\relax
  }%
\ifnum\count@<255 %
  \advance\count@ 1 %
\repeat

\def\RangeCatcodeInvalid#1#2{%
  \count@=#1\relax
  \loop
    \catcode\count@=15 %
  \ifnum\count@<#2\relax
    \advance\count@ 1 %
  \repeat
}
\def\RangeCatcodeCheck#1#2#3{%
  \count@=#1\relax
  \loop
    \ifnum#3=\catcode\count@
    \else
      \errmessage{%
        Character \the\count@\space
        with wrong catcode \the\catcode\count@\space
        instead of \number#3%
      }%
    \fi
  \ifnum\count@<#2\relax
    \advance\count@ 1 %
  \repeat
}
\def\space{ }
\expandafter\ifx\csname LoadCommand\endcsname\relax
  \def\LoadCommand{\input kvoptions.sty\relax}%
\fi
\def\Test{%
  \RangeCatcodeInvalid{0}{47}%
  \RangeCatcodeInvalid{58}{64}%
  \RangeCatcodeInvalid{91}{96}%
  \RangeCatcodeInvalid{123}{255}%
  \catcode`\@=12 %
  \catcode`\\=0 %
  \catcode`\%=14 %
  \LoadCommand
  \RangeCatcodeCheck{0}{36}{15}%
  \RangeCatcodeCheck{37}{37}{14}%
  \RangeCatcodeCheck{38}{47}{15}%
  \RangeCatcodeCheck{48}{57}{12}%
  \RangeCatcodeCheck{58}{63}{15}%
  \RangeCatcodeCheck{64}{64}{12}%
  \RangeCatcodeCheck{65}{90}{11}%
  \RangeCatcodeCheck{91}{91}{15}%
  \RangeCatcodeCheck{92}{92}{0}%
  \RangeCatcodeCheck{93}{96}{15}%
  \RangeCatcodeCheck{97}{122}{11}%
  \RangeCatcodeCheck{123}{255}{15}%
  \RestoreCatcodes
}
\Test
\csname @@end\endcsname
\end
%    \end{macrocode}
%    \begin{macrocode}
%</test1>
%    \end{macrocode}
%
%    \begin{macrocode}
%<*test2>
\NeedsTeXFormat{LaTeX2e}
\makeatletter
\catcode`\@=11 %
\def\RestoreCatcodes{}
\count@=0 %
\loop
  \edef\RestoreCatcodes{%
    \RestoreCatcodes
    \catcode\the\count@=\the\catcode\count@\relax
  }%
\ifnum\count@<255 %
  \advance\count@\@ne
\repeat

\def\RangeCatcodeInvalid#1#2{%
  \count@=#1\relax
  \loop
    \catcode\count@=15 %
  \ifnum\count@<#2\relax
    \advance\count@\@ne
  \repeat
}
\def\Test#1{%
  \RangeCatcodeInvalid{0}{47}%
  \RangeCatcodeInvalid{58}{64}%
  \RangeCatcodeInvalid{91}{96}%
  \RangeCatcodeInvalid{123}{255}%
  \catcode`\@=12 %
  \catcode`\\=0 %
  \catcode`\{=1 %
  \catcode`\}=2 %
  \catcode`\#=6 %
  \catcode`\[=12 %
  \catcode`\]=12 %
  \catcode`\%=14 %
  \catcode`\ =10 %
  \catcode13=5 %
  #1\relax
  \RestoreCatcodes
}
\Test{\RequirePackage{kvoptions-patch}}%
\Test{\RequirePackage{kvoptions}}%
\csname @@end\endcsname
%</test2>
%    \end{macrocode}
%
%    \begin{macrocode}
%<*test3>
\NeedsTeXFormat{LaTeX2e}
\makeatletter
\RequirePackage{kvoptions}[2016/05/16]
\def\msg#{\immediate\write16}
\define@key{testfamily}{testkey}{%
  \msg{[testfamily/testkey/#1]}%
}
\define@key{testfamily}{testdefaultkey}[testdefault]{%
  \msg{[testfamily/testdefaultkey/#1]}%
}
\AddToKeyvalOption{testfamily}{testkey}{%
  \msg{[addition/#1]}%
}
\AddToKeyvalOption{testfamily}{testdefaultkey}{%
  \msg{[addition/#1]}%
}
\setkeys{testfamily}{%
  testkey=testA,%
  testdefaultkey=testB,%
  testdefaultkey,%
}
\SetupKeyvalOptions{%
  family=testfamily%
}
\AddToKeyvalOption*{testkey}{%
  \msg{[star addition/#1]}%
}
\AddToKeyvalOption*{testdefaultkey}{%
  \msg{[star addition/#1]}%
}
\setkeys{testfamily}{%
  testkey=testA,%
  testdefaultkey=testB,%
  testdefaultkey,%
}
\@@end
%</test3>
%    \end{macrocode}
%
%    \begin{macrocode}
%<*test4pkg>
\NeedsTeXFormat{LaTeX2e}
\ProvidesPackage{kvoptions-test4}[2016/05/16 package for testing]
\RequirePackage{kvoptions}[2016/05/16]
\SetupKeyvalOptions{%
  family=FOO,%
  prefix=foo,%
  setkeys=\kvsetkeys,%
}
\DeclareStringOption{str}
\define@key{FOO}{set}{%
  \setkeys{BAR}{strbar={#1}}%
}
\define@key{BAR}{strbar}{%
  \def\foostr{[BAR:#1]}%
}
\ProcessKeyvalOptions*
%</test4pkg>
%    \end{macrocode}
%    \begin{macrocode}
%<*test4>
\NeedsTeXFormat{LaTeX2e}
\ProvidesFile{kvoptions-test4.tex}[2016/05/16 test file]
\RequirePackage[%
  str=A,set=B,str=C,%
]{kvoptions-test4}[2016/05/16]
\def\TestExpected{C}
\ifx\foostr\TestExpected
  \typeout{* Test ok.}%
\else
  \typeout{* Result: [\foostr]}%
  \typeout{* Expected: [\TestExpected]}%
  \errmessage{Test failed!}%
\fi
\csname @@end\endcsname\end
%</test4>
%    \end{macrocode}
%
% \section{Installation}
%
% \subsection{Download}
%
% \paragraph{Package.} This package is available on
% CTAN\footnote{\CTANpkg{kvoptions}}:
% \begin{description}
% \item[\CTAN{macros/latex/contrib/oberdiek/kvoptions.dtx}] The source file.
% \item[\CTAN{macros/latex/contrib/oberdiek/kvoptions.pdf}] Documentation.
% \end{description}
%
%
% \paragraph{Bundle.} All the packages of the bundle `oberdiek'
% are also available in a TDS compliant ZIP archive. There
% the packages are already unpacked and the documentation files
% are generated. The files and directories obey the TDS standard.
% \begin{description}
% \item[\CTANinstall{install/macros/latex/contrib/oberdiek.tds.zip}]
% \end{description}
% \emph{TDS} refers to the standard ``A Directory Structure
% for \TeX\ Files'' (\CTAN{tds/tds.pdf}). Directories
% with \xfile{texmf} in their name are usually organized this way.
%
% \subsection{Bundle installation}
%
% \paragraph{Unpacking.} Unpack the \xfile{oberdiek.tds.zip} in the
% TDS tree (also known as \xfile{texmf} tree) of your choice.
% Example (linux):
% \begin{quote}
%   |unzip oberdiek.tds.zip -d ~/texmf|
% \end{quote}
%
% \paragraph{Script installation.}
% Check the directory \xfile{TDS:scripts/oberdiek/} for
% scripts that need further installation steps.
%
% \subsection{Package installation}
%
% \paragraph{Unpacking.} The \xfile{.dtx} file is a self-extracting
% \docstrip\ archive. The files are extracted by running the
% \xfile{.dtx} through \plainTeX:
% \begin{quote}
%   \verb|tex kvoptions.dtx|
% \end{quote}
%
% \paragraph{TDS.} Now the different files must be moved into
% the different directories in your installation TDS tree
% (also known as \xfile{texmf} tree):
% \begin{quote}
% \def\t{^^A
% \begin{tabular}{@{}>{\ttfamily}l@{ $\rightarrow$ }>{\ttfamily}l@{}}
%   kvoptions.sty & tex/latex/oberdiek/kvoptions.sty\\
%   kvoptions-patch.sty & tex/latex/oberdiek/kvoptions-patch.sty\\
%   kvoptions.pdf & doc/latex/oberdiek/kvoptions.pdf\\
%   example-mycolorsetup.sty & doc/latex/oberdiek/example-mycolorsetup.sty\\
%   test/kvoptions-test1.tex & doc/latex/oberdiek/test/kvoptions-test1.tex\\
%   test/kvoptions-test2.tex & doc/latex/oberdiek/test/kvoptions-test2.tex\\
%   test/kvoptions-test3.tex & doc/latex/oberdiek/test/kvoptions-test3.tex\\
%   test/kvoptions-test4.tex & doc/latex/oberdiek/test/kvoptions-test4.tex\\
%   test/kvoptions-test4.sty & doc/latex/oberdiek/test/kvoptions-test4.sty\\
%   kvoptions.dtx & source/latex/oberdiek/kvoptions.dtx\\
% \end{tabular}^^A
% }^^A
% \sbox0{\t}^^A
% \ifdim\wd0>\linewidth
%   \begingroup
%     \advance\linewidth by\leftmargin
%     \advance\linewidth by\rightmargin
%   \edef\x{\endgroup
%     \def\noexpand\lw{\the\linewidth}^^A
%   }\x
%   \def\lwbox{^^A
%     \leavevmode
%     \hbox to \linewidth{^^A
%       \kern-\leftmargin\relax
%       \hss
%       \usebox0
%       \hss
%       \kern-\rightmargin\relax
%     }^^A
%   }^^A
%   \ifdim\wd0>\lw
%     \sbox0{\small\t}^^A
%     \ifdim\wd0>\linewidth
%       \ifdim\wd0>\lw
%         \sbox0{\footnotesize\t}^^A
%         \ifdim\wd0>\linewidth
%           \ifdim\wd0>\lw
%             \sbox0{\scriptsize\t}^^A
%             \ifdim\wd0>\linewidth
%               \ifdim\wd0>\lw
%                 \sbox0{\tiny\t}^^A
%                 \ifdim\wd0>\linewidth
%                   \lwbox
%                 \else
%                   \usebox0
%                 \fi
%               \else
%                 \lwbox
%               \fi
%             \else
%               \usebox0
%             \fi
%           \else
%             \lwbox
%           \fi
%         \else
%           \usebox0
%         \fi
%       \else
%         \lwbox
%       \fi
%     \else
%       \usebox0
%     \fi
%   \else
%     \lwbox
%   \fi
% \else
%   \usebox0
% \fi
% \end{quote}
% If you have a \xfile{docstrip.cfg} that configures and enables \docstrip's
% TDS installing feature, then some files can already be in the right
% place, see the documentation of \docstrip.
%
% \subsection{Refresh file name databases}
%
% If your \TeX~distribution
% (\TeX\,Live, \mikTeX, \dots) relies on file name databases, you must refresh
% these. For example, \TeX\,Live\ users run \verb|texhash| or
% \verb|mktexlsr|.
%
% \subsection{Some details for the interested}
%
% \paragraph{Unpacking with \LaTeX.}
% The \xfile{.dtx} chooses its action depending on the format:
% \begin{description}
% \item[\plainTeX:] Run \docstrip\ and extract the files.
% \item[\LaTeX:] Generate the documentation.
% \end{description}
% If you insist on using \LaTeX\ for \docstrip\ (really,
% \docstrip\ does not need \LaTeX), then inform the autodetect routine
% about your intention:
% \begin{quote}
%   \verb|latex \let\install=y\input{kvoptions.dtx}|
% \end{quote}
% Do not forget to quote the argument according to the demands
% of your shell.
%
% \paragraph{Generating the documentation.}
% You can use both the \xfile{.dtx} or the \xfile{.drv} to generate
% the documentation. The process can be configured by the
% configuration file \xfile{ltxdoc.cfg}. For instance, put this
% line into this file, if you want to have A4 as paper format:
% \begin{quote}
%   \verb|\PassOptionsToClass{a4paper}{article}|
% \end{quote}
% An example follows how to generate the
% documentation with pdf\LaTeX:
% \begin{quote}
%\begin{verbatim}
%pdflatex kvoptions.dtx
%makeindex -s gind.ist kvoptions.idx
%pdflatex kvoptions.dtx
%makeindex -s gind.ist kvoptions.idx
%pdflatex kvoptions.dtx
%\end{verbatim}
% \end{quote}
%
% \begin{thebibliography}{99}
% \bibitem{tb94wright}
%   A guide to key-value methods, Joseph Wright, second draft for
%   \href{https://www.tug.org/tugboat}{TUGBoat}, 2009-03-17.
%   \url{https://www.texdev.net/uploads/2009/03/keyval.pdf}
% \bibpackage{ifthen}{David Carlisle}{2001/05/26}
%   {\CTANpkg{ifthen}}
% \bibpackage{helvet}{Sebastian Rahtz, Walter Schmidt}{2004/01/26}
%   {\CTANpkg{psfonts}}
% \bibpackage{hyperref}{Sebastian Rahtz, Heiko Oberdiek}{2006/02/12}
%   {\CTANpkg{hyperref}}
% \bibpackage{keyval}{David Carlisle}{1999/03/16}
%   {\CTANpkg{keyval}}
% \bibpackage{multicol}{Frank Mittelbach}{2004/02/14}
%   {\CTANpkg{multicol}}
% \bibpackage{tabularx}{David Carlisle}{1999/01/07}
%   {\CTANpkg{tabularx}}
% \bibpackage{tracefnt}{Frank Mittelbach, Rainer Sch\"opf}{1997/05/29}
%   {\CTANpkg{latex-base}}
% \bibpackage{xkeyval}{Hendri Adriaens}{2005/05/07}
%   {\CTANpkg{xkeyval}}
% \bibitem{clsguide}
%   The \LaTeX3 Project, \textit{\LaTeXe\ for class and package writers},
%   2003/12/09.
%   \CTANpkg{clsguide}
%
% \end{thebibliography}
%
% \begin{History}
%   \begin{Version}{0000/00/00 v0.0}
%   \item
%     Probably David Carlisle's code in \xpackage{hyperref}
%     was the start.
%   \end{Version}
%   \begin{Version}{2004/02/22 v1.0}
%   \item
%     The first version was never published. It also
%     has offered a patch to get rid of \LaTeX's option
%     expansion.
%   \end{Version}
%   \begin{Version}{2006/02/16 v2.0}
%   \item
%     Now the package is redesigned with an easier
%     user interface.
%   \item
%     \cs{ProcessKeyvalOptions} remains the central service, inherited
%     from \xpackage{hyperref}'s \cs{ProcessOptionsWithKV}.
%     Now the use inside classes is also supported.
%   \item
%     Provides help macros for boolean and simple string options.
%   \item
%     Fixes for the patch of \LaTeX{}. The patch is only enabled,
%     if the user requests it.
%   \end{Version}
%   \begin{Version}{2006/02/20 v2.1}
%   \item
%     Unused option list is sanitized to prevent problems
%     with other packages that uses own processing methods
%     for key value options. Disadvantage: the unused global
%     option detection is weakened.
%   \item
%     New option type by \cs{DeclareVoidOption} for options without
%     value.
%   \item
%     Default rule by \cs{DeclareDefaultOption}.
%   \item
%     Dynamic options: \cs{DisableKeyvalOption}.
%   \end{Version}
%   \begin{Version}{2006/06/01 v2.2}
%   \item
%     Fixes for option \xoption{patch}.
%   \end{Version}
%   \begin{Version}{2006/08/17 v2.3}
%   \item
%     \cs{DeclareBooleanOption} renamed to \cs{DeclareBoolOption}
%     to avoid a name clash with package \cs{ifoption}.
%   \end{Version}
%   \begin{Version}{2006/08/22 v2.4}
%   \item
%     Option \xoption{patch}: \cs{ExecuteOptions} does not
%     change the meaning of macro \cs{CurrentOption} at all.
%   \end{Version}
%   \begin{Version}{2007/04/11 v2.5}
%   \item
%     Line ends sanitized.
%   \end{Version}
%   \begin{Version}{2007/05/06 v2.6}
%   \item
%     Uses package \xpackage{etexcmds}.
%   \end{Version}
%   \begin{Version}{2007/06/11 v2.7}
%   \item
%     The patch part fixes LaTeX bug latex/3965.
%   \end{Version}
%   \begin{Version}{2007/10/02 v2.8}
%   \item
%     Compatibility for \plainTeX\ added.
%   \item
%     Typos in documentation fixed (Axel Sommerfeldt).
%   \end{Version}
%   \begin{Version}{2007/10/11 v2.9}
%   \item
%     Bug fix for option \xoption{patch}.
%   \end{Version}
%   \begin{Version}{2007/10/18 v3.0}
%   \item
%     New package \xpackage{kvoptions-patch}.
%   \end{Version}
%   \begin{Version}{2009/04/10 v3.1}
%   \item
%     Space by line end removed in definition of internal macro.
%   \end{Version}
%   \begin{Version}{2009/07/17 v3.2}
%   \item
%     \cs{ProcessLocalKeyvalOptions} added.
%   \item
%     \cs{DisableKeyvalOption} with the \texttt{action=ignore} option
%     fixed (Joseph Wright).
%   \end{Version}
%   \begin{Version}{2009/07/21 v3.3}
%   \item
%     \cs{DeclareLocalOption}, \cs{DeclareLocalOptions} added.
%   \end{Version}
%   \begin{Version}{2009/08/13 v3.4}
%   \item
%     Documentation addition: recommendation for Joseph Wright's
%     review article.
%   \item
%     Documentation addition: local/global options.
%   \end{Version}
%   \begin{Version}{2009/12/04 v3.5}
%   \item
%     \cs{AddToKeyvalOption} added.
%   \end{Version}
%   \begin{Version}{2009/12/08 v3.6}
%   \item
%     Fix: If a default handler is configured, it is now also
%     called for classes.
%   \end{Version}
%   \begin{Version}{2010/02/22 v3.7}
%   \item
%     Missing space in error message added.
%   \end{Version}
%   \begin{Version}{2010/07/23 v3.8}
%   \item
%     Documenation for package \xpackage{kvoptions-patch} improved.
%     No code changes.
%   \end{Version}
%   \begin{Version}{2010/12/02 v3.9}
%   \item
%     Key \texttt{setkeys} added for \cs{SetupKeyvalOptions}.
%   \end{Version}
%   \begin{Version}{2010/12/23 v3.10}
%   \item
%     \cs{DeclareVoidOption} also parses the second parameter as
%     \hologo{TeX} argument to improve compatibility with
%     \cs{DeclareOption}.
%   \end{Version}
%   \begin{Version}{2011/06/30 v3.11}
%   \item
%     Fix because of design bug in package \xpackage{xkeyval} that
%     removes global options with equal signs.
%   \end{Version}
%   \begin{Version}{2016/05/16 v3.12}
%   \item
%     Documentation updates.
%   \end{Version}
% \end{History}
%
% \PrintIndex
%
% \Finale
\endinput

%        (quote the arguments according to the demands of your shell)
%
% Documentation:
%    (a) If kvoptions.drv is present:
%           latex kvoptions.drv
%    (b) Without kvoptions.drv:
%           latex kvoptions.dtx; ...
%    The class ltxdoc loads the configuration file ltxdoc.cfg
%    if available. Here you can specify further options, e.g.
%    use A4 as paper format:
%       \PassOptionsToClass{a4paper}{article}
%
%    Programm calls to get the documentation (example):
%       pdflatex kvoptions.dtx
%       makeindex -s gind.ist kvoptions.idx
%       pdflatex kvoptions.dtx
%       makeindex -s gind.ist kvoptions.idx
%       pdflatex kvoptions.dtx
%
% Installation:
%    TDS:tex/latex/oberdiek/kvoptions.sty
%    TDS:tex/latex/oberdiek/kvoptions-patch.sty
%    TDS:doc/latex/oberdiek/kvoptions.pdf
%    TDS:doc/latex/oberdiek/example-mycolorsetup.sty
%    TDS:doc/latex/oberdiek/test/kvoptions-test1.tex
%    TDS:doc/latex/oberdiek/test/kvoptions-test2.tex
%    TDS:doc/latex/oberdiek/test/kvoptions-test3.tex
%    TDS:doc/latex/oberdiek/test/kvoptions-test4.tex
%    TDS:doc/latex/oberdiek/test/kvoptions-test4.sty
%    TDS:source/latex/oberdiek/kvoptions.dtx
%
%<*ignore>
\begingroup
  \catcode123=1 %
  \catcode125=2 %
  \def\x{LaTeX2e}%
\expandafter\endgroup
\ifcase 0\ifx\install y1\fi\expandafter
         \ifx\csname processbatchFile\endcsname\relax\else1\fi
         \ifx\fmtname\x\else 1\fi\relax
\else\csname fi\endcsname
%</ignore>
%<*install>
\input docstrip.tex
\Msg{************************************************************************}
\Msg{* Installation}
\Msg{* Package: kvoptions 2016/05/16 v3.12 Key value format for package options (HO)}
\Msg{************************************************************************}

\keepsilent
\askforoverwritefalse

\let\MetaPrefix\relax
\preamble

This is a generated file.

Project: kvoptions
Version: 2016/05/16 v3.12

Copyright (C)
   2004, 2006, 2007, 2009-2011 Heiko Oberdiek
   2016-2019 Oberdiek Package Support Group

This work may be distributed and/or modified under the
conditions of the LaTeX Project Public License, either
version 1.3c of this license or (at your option) any later
version. This version of this license is in
   https://www.latex-project.org/lppl/lppl-1-3c.txt
and the latest version of this license is in
   https://www.latex-project.org/lppl.txt
and version 1.3 or later is part of all distributions of
LaTeX version 2005/12/01 or later.

This work has the LPPL maintenance status "maintained".

The Current Maintainers of this work are
Heiko Oberdiek and the Oberdiek Package Support Group
https://github.com/ho-tex/oberdiek/issues


This work consists of the main source file kvoptions.dtx
and the derived files
   kvoptions.sty, kvoptions.pdf, kvoptions.ins, kvoptions.drv,
   kvoptions-patch.sty, example-mycolorsetup.sty,
   kvoptions-test1.tex, kvoptions-test2.tex,
   kvoptions-test3.tex, kvoptions-test4.tex,
   kvoptions-test4.sty.

\endpreamble
\let\MetaPrefix\DoubleperCent

\generate{%
  \file{kvoptions.ins}{\from{kvoptions.dtx}{install}}%
  \file{kvoptions.drv}{\from{kvoptions.dtx}{driver}}%
  \usedir{tex/latex/oberdiek}%
  \file{kvoptions.sty}{\from{kvoptions.dtx}{package}}%
  \file{kvoptions-patch.sty}{\from{kvoptions.dtx}{patch}}%
  \usedir{doc/latex/oberdiek}%
  \file{example-mycolorsetup.sty}{\from{kvoptions.dtx}{example}}%
%  \usedir{doc/latex/oberdiek/test}%
%  \file{kvoptions-test1.tex}{\from{kvoptions.dtx}{test1}}%
%  \file{kvoptions-test2.tex}{\from{kvoptions.dtx}{test2}}%
%  \file{kvoptions-test3.tex}{\from{kvoptions.dtx}{test3}}%
%  \file{kvoptions-test4.tex}{\from{kvoptions.dtx}{test4}}%
%  \file{kvoptions-test4.sty}{\from{kvoptions.dtx}{test4pkg}}%
  \nopreamble
  \nopostamble
%  \usedir{source/latex/oberdiek/catalogue}%
%  \file{kvoptions.xml}{\from{kvoptions.dtx}{catalogue}}%
}

\catcode32=13\relax% active space
\let =\space%
\Msg{************************************************************************}
\Msg{*}
\Msg{* To finish the installation you have to move the following}
\Msg{* files into a directory searched by TeX:}
\Msg{*}
\Msg{*     kvoptions.sty, kvoptions-patch.sty}
\Msg{*}
\Msg{* To produce the documentation run the file `kvoptions.drv'}
\Msg{* through LaTeX.}
\Msg{*}
\Msg{* Happy TeXing!}
\Msg{*}
\Msg{************************************************************************}

\endbatchfile
%</install>
%<*ignore>
\fi
%</ignore>
%<*driver>
\NeedsTeXFormat{LaTeX2e}
\ProvidesFile{kvoptions.drv}%
  [2016/05/16 v3.12 Key value format for package options (HO)]%
\documentclass{ltxdoc}
\usepackage{holtxdoc}[2011/11/22]
\begin{document}
  \DocInput{kvoptions.dtx}%
\end{document}
%</driver>
% \fi
%
%
% \CharacterTable
%  {Upper-case    \A\B\C\D\E\F\G\H\I\J\K\L\M\N\O\P\Q\R\S\T\U\V\W\X\Y\Z
%   Lower-case    \a\b\c\d\e\f\g\h\i\j\k\l\m\n\o\p\q\r\s\t\u\v\w\x\y\z
%   Digits        \0\1\2\3\4\5\6\7\8\9
%   Exclamation   \!     Double quote  \"     Hash (number) \#
%   Dollar        \$     Percent       \%     Ampersand     \&
%   Acute accent  \'     Left paren    \(     Right paren   \)
%   Asterisk      \*     Plus          \+     Comma         \,
%   Minus         \-     Point         \.     Solidus       \/
%   Colon         \:     Semicolon     \;     Less than     \<
%   Equals        \=     Greater than  \>     Question mark \?
%   Commercial at \@     Left bracket  \[     Backslash     \\
%   Right bracket \]     Circumflex    \^     Underscore    \_
%   Grave accent  \`     Left brace    \{     Vertical bar  \|
%   Right brace   \}     Tilde         \~}
%
% \GetFileInfo{kvoptions.drv}
%
% \title{The \xpackage{kvoptions} package}
% \date{2016/05/16 v3.12}
% \author{Heiko Oberdiek\thanks
% {Please report any issues at \url{https://github.com/ho-tex/oberdiek/issues}}}
%
% \maketitle
%
% \begin{abstract}
% This package is intended for package authors who want to
% use options in key value format for their package options.
% \end{abstract}
%
% \tableofcontents
%
% \def\M#1{\texttt{\{}\meta{#1}\texttt{\}}}
% \def\O#1{\texttt{[}\meta{#1}\texttt{]}}
%
% \section{Introduction}
%
% First I want to recommend the very good review article
% ``A guide to key-value methods'' by Joseph Wright \cite{tb94wright}.
% It introduces the different key-value packages and compares them.
%
% \subsection{The beginning}
%
% This package \xpackage{kvoptions} addresses class or package writers
% that want to allow their users to specify options
% as key value pairs, e.g.:
% \begin{quote}
% |\documentclass[verbose=false,name=me]{myclass}|\\
% |\usepackage[format=print]{mylayout}|
% \end{quote}
% Prominent example is package \xpackage{hyperref},
% probably the first package that offers this service.
% It's \cs{ProcessOptionsWithKV} is often copied und
% used in other packages, e.g. package \xpackage{helvet}
% that uses this interface for its option \xoption{scaled}.
%
% However copying code is not the most modern software
% development technique. And \xpackage{hyperref}'s
% code for \cs{ProcessOptionsWithKV} was changed
% to fix bugs. The version used in other packages
% depends on the time of copying and the awareness
% of \xpackage{hyperref}'s changes.
% Now the code is sourced out into this package and
% available for other package or class writers.
%
% \subsection{Overview}
%
% Package \xpackage{kvoptions} connects package
% \xpackage{\emph{k}ey\emph{v}al}
% with \LaTeX's package and class \emph{options}:
% \begin{center}
% \def\T#1{\bfseries #1}
% \renewcommand{\arraystretch}{1.5}
% \begin{tabular}{l|l|l}
%   \T{Package \xpackage{keyval}}
%   & \T{Package \xpackage{kvoptions}}
%   & \T{\LaTeX\ kernel}
%   \\
% \hline
%     \cs{define@key}
%   &
%     \renewcommand*{\arraystretch}{1}^^A
%     \begin{tabular}[t]{@{}l@{}}
%       \cs{DeclareVoidOption}\\
%       \cs{DeclareStringOption}\\
%       \cs{DeclareBoolOption}\\
%       \cs{DeclareComplementaryOption}\\
%       \cs{DisableKeyvalOption}
%     \end{tabular}
%   & \cs{DeclareOption}
%   \\
% \hline
%   & \cs{DeclareDefaultOption}
%   & \cs{DeclareOption*}
%   \\
% \hline
%   & \cs{ProcessKeyvalOptions}
%   & \cs{ProcessOptions*}
%   \\
% \hline
%   & Option \xoption{patch}
%   &
%     \renewcommand*{\arraystretch}{1}^^A
%     \begin{tabular}[t]{@{}l@{}}
%       Class/package\\
%       option system
%     \end{tabular}
%   \\
% \hline
%   & \cs{SetupKeyvalOptions} &
%   \\
% \end{tabular}
% \end{center}
%
% \hypersetup{bookmarksopenlevel=3}
% \section{Usage}
%
% \subsection{Process options}
%
% \subsubsection{\cs{ProcessKeyvalOptions}}
%
% \begin{declcs}{ProcessKeyvalOptions}\,\M{family}\\
%   \cs{ProcessKeyvalOptions} \*
% \end{declcs}
% This command evaluates the global or local options of
% the package that are defined with \xpackage{keyval}'s
% interface within the family \meta{family}. It acts the
% same way as \LaTeX's \cs{ProcessOptions*}. In a package
% unknown global options are ignored, in a class they
% are added to the unknown option list. The known global
% options and all local options are passed to \xpackage{keyval}'s
% \cs{setkeys} command for executing the options.
% Unknown options are reported to the user by an error.
%
% If the family name happens to be the same as the name
% of the package or class where \cs{ProcessKeyvalOptions}
% is used or the family name has previously been setup by
% \cs{SetupKeyvalOptions}, then \cs{ProcessKeyvalOptions}
% knows the family name already and you can use the star form
% without mandatory argument.
%
% \subsubsection{\cs{ProcessLocalKeyvalOptions}}
%
% \begin{declcs}{ProcessLocalKeyvalOptions}\,\M{family}\\
%   \cs{ProcessLocalKeyvalOptions} \*
% \end{declcs}
% This macro has the same syntax and works similar as
% \cs{ProcessKeyvalOptions}. However it ignores
% global options and only processes the local package options.
% Therefore it only can be used inside a package.
% An error is thrown, if it is used inside a class.
%
% Neither of the following macros are necessary for
% \cs{ProcessKeyvalOptions{}}. They just help the package/class
% author in common tasks.
%
% \subsubsection{\cs{SetupKeyvalOptions}}\label{setup}
%
% \begin{declcs}{SetupKeyvalOptions}\,\texttt{\{}\\
%   \hspace{1.5em}|family|\,|=|\,\meta{family}|,|\\
%   \hspace{1.5em}|prefix|\,|=|\,\meta{prefix}\\
%   \hspace{1.5em}|setkeys|\,|=|\,\meta{setkeys command}\\
%   \texttt{\}}
% \end{declcs}
% This command allows to configure the default assumptions
% that are based on the current package or class name.
% \LaTeX\ remembers this name in \cs{@currname}. The syntax
% description of the default looks a little weird, therefor
% an example is given for a package or class named |foobar|.
% \begin{quote}
% \begin{tabular}{@{}l|ll|l@{}}
%   Key & Default & (example) & Used by\\
% \hline
%   \rule{0pt}{1.2\ht\csname @arstrutbox\endcsname}^^A
%   |family| & \meta{\cs{@currname}} & (|foobar|)
%       & \cs{ProcessKeyvalOptions*} \\
%     &&& \cs{DeclareBoolOption} \\
%     &&& \cs{DeclareStringOption} \\
% \hline
%   \rule{0pt}{1.2\ht\csname @arstrutbox\endcsname}^^A
%   |prefix| & \meta{\cs{@currname}}|@| & (|foobar@|)
%       & \cs{DeclareBoolOption} \\
%     &&& \cs{DeclareStringOption} \\
%     &&& \cs{DeclareVoidOption} \\
% \hline
%   \rule{0pt}{1.2\ht\csname @arstrutbox\endcsname}^^A
%   |setkeys| & \cs{setkeys} & (\cs{kvsetkeys})
%       & \cs{ProcessKeyvalOptions} \\
%     &&& \cs{ProcessLocalKeyvalOptions}
% \end{tabular}
% \end{quote}
% Key |setkeys| was added in version 3.9. The original
% \cs{setkeys} of package \xpackage{keyval} is not reentrant.
% If an option is processed by this \cs{setkeys}, then
% the option should not call \cs{setkeys} again with a
% different family. Otherwise the next options of the
% first \cs{setkeys} call are processed with the wrong
% family. With key |setkeys| the macro \cs{kvsetkeys}
% can be set that does not have the problem of the
% original \cs{setkeys} of package {keyval}.
%
% Probably \cs{setkeys} of package \xpackage{xkeyval}
% is safe in this respect. But I haven't made a full
% analysis. At least it does not have the problem
% of the original \cs{setkeys}.
%
% \subsection{Option declarations}
%
% The options for \cs{ProcessKeyvalOptions} are defined
% by \xpackage{keyval}'s \cs{define@key}.
% Common purposes of such keys are boolean switches, they
% enable or disable something. Or they store a name or some kind of
% string in a macro. The following commands help the user. He
% declares what he wants and \xpackage{kvoptions} take care of
% the key definition, resource allocation and initialization.
%
% In order to avoid name clashes of macro names, internal
% commands are prefixed. Both the prefix and the family
% name for the defined keys can be configured by
% \cs{SetupKeyvalOptions}.
%
% \subsubsection{\cs{DeclareStringOption}}
%
% \begin{declcs}{DeclareStringOption}\,\O{init}\,\M{key}\,\O{default}
% \end{declcs}
% A macro is created that remembers the value of the key \meta{key}.
% The name of the macro consists of the option name \meta{key} that
% is prefixed by the prefix (see \ref{setup}).
% The initial contents of the
% macro can be given by the first optional argument \meta{init}.
% The default is empty.
%
% The the option \meta{key} is defined. The option code
% just stores its value in the macro. If the optional argument
% at the end of \cs{DeclareStringOption} is given, then option
% \meta{key} is defined with the default \meta{default}.
%
% Example for a package with the following two lines:
% \begin{quote}
% |\ProvidesPackage{foobar}|\\
% |\DeclareStringOption[me]{name}|
% \end{quote}
% Then \cs{DeclareStringOption} defines the macro with content |me|,
% note \LaTeX\ complains if the name of the macro already exists:
% \begin{quote}
% |\newcommand*{\foobar@name}{me}|
% \end{quote}
% The option definition is similar to:
% \begin{quote}
% |\define@key{foobar}{name}{%|\\
% |  \renewcommand*{\foobar@name}{#1}%|\\
% |}|
% \end{quote}
%
% \subsubsection{\cs{DeclareBoolOption}}
%
% \begin{declcs}{DeclareBoolOption}\,\O{init}\,\M{key}
% \end{declcs}
% A boolean switch is generated, initialized by value \meta{init}
% and the corresponding key \meta{key} is defined. If the
% initialization value is not given, |false| is used as default.
%
% The internal actions of \cs{DeclareBoolOption} are shown below.
% The example is given for a package author who has the following
% two lines in his package/class:
% \begin{quote}
% |\ProvidesPackage{foobar}|\\
% |\DeclareBoolOption{verbose}|
% \end{quote}
% First a new switch is created:
% \begin{quote}
%   |\newif\iffoobar@verbose|
% \end{quote}
% and initialized:
% \begin{quote}
%   |\foobar@verbosefalse|
% \end{quote}
% Finally the key is defined:
% \begin{quote}
%   |\define@key{foobar}{verbose}[true]{|\dots|}|
% \end{quote}
% The option code configures the boolean option in the following way:
% If the author specifies |true| or |false| then the switch is
% turned on or off respectivly. Also the option can be given without
% explicit value. Then the switch is enabled.
% Other values are reported as errors.
%
% Now the switch is ready to use in the package/class, e.g.:
% \begin{quote}
%   |\iffoobar@verbose|\\
%   \qquad\textit{\% print verbose message}\\
%   |\else|\\
%   \qquad\textit{\% be quiet}\\
%   |\fi|
% \end{quote}
% Users of package \cs{ifthen} can use the switch as
% boolean:
% \begin{quote}
%   |\boolean{foobar@verbose}|
% \end{quote}
%
% \subsubsection{\cs{DeclareComplementaryOption}}
%
% \begin{declcs}{DeclareComplementaryOption}\,\M{key}\,\M{parent}
% \end{declcs}
% Sometimes contrasting names are used to characterize the
% two states of a boolean switch, for example \xoption{draft}
% vs.\@ \xoption{final}. Both options behave like boolean
% options but they do not need two different switches, they
% should share one. \cs{DeclareComplementaryOption} allows this.
% The option \meta{key} shares the switch of option \meta{parent}.
% Example:
% \begin{quote}
%   |\DeclareBoolOption{draft}|\\
%   |\DeclareComplementaryOption{final}{draft}|
% \end{quote}
% Then |final| sets the switch of \xoption{draft} to |false|, and
% |final=false| enables the \xoption{draft} switch.
%
% \subsubsection{\cs{DeclareVoidOption}}
%
% \begin{declcs}{DeclareVoidOption}\,\M{key}\,\M{code}
% \end{declcs}
% \cs{ProcessKeyvalOptions} can be extended to recognize
% options that are declared in traditional way by \cs{DeclareOption}.
% But in case of the error that the user specifies a value, then
% this option would not recognized as key value option because
% of \cs{DeclareOption} and not detected as traditional option
% because of the value part. The user would get an unknown option
% error, difficult to understand.
%
% \cs{DeclareVoidOption} solves
% this problem. It defines the option \meta{key} as key value option.
% If the user specifies a value, a warning is given and the
% value is ignored.
%
% The code part \meta{code} is stored in a macro.
% The name of the macro consists of the option name \meta{key} that
% is prefixed by the prefix (see \ref{setup}).
% If the option is set, the macro will be executed. During the
% execution \cs{CurrentOption} is available with the current key name.
%
% \subsubsection{\cs{DeclareDefaultOption}}
%
% \begin{declcs}{DeclareDefaultOption}\,\M{code}
% \end{declcs}
% \begin{sloppypar}
% This command does not define a specific key, it is the equivalent
% to \LaTeX's \cs{DeclareOption*}. It allows the specification
% of a default action \meta{code} that is invoked if an unknown option
% is found.
% While \meta{code} is called,
% macro \cs{CurrentOption} contains the current option string.
% In addition \cs{CurrentOptionValue} contains the value part
% if the option string is parsable as key value pair, otherwise
% it is \cs{relax}. \cs{CurrentOptionKey} contains the key of
% the key value pair, or the whole option string, if it misses
% the equal sign.
% \end{sloppypar}
%
% Inside packages typical default actions are to pass unknown
% options to another package. Or an error message can be thrown
% by \cs{@unknownoptionerror}. This is the original error
% message that \LaTeX\ gives for unkown package options.
% This error message is easier to understand for the user
% as the error message from package \xpackage{keyval} that
% is given otherwise.
%
% A Class ignores unknown options and puts them on the
% unused option list. Let \LaTeX\ do the job and just
% call \cs{OptionNotUsed}. Or the options can be passed to
% another class that is later loaded.
%
% \subsubsection{Local options}
%
% \begin{declcs}{DeclareLocalOption} \M{option}\\
%   \cs{DeclareLocalOptions} \M{option list}
% \end{declcs}
% Both macros mark package options as local options. That means
% that they are ignored by \cs{ProcessKeyvalOptions} if they are given
% as global options. \cs{DeclareLocalOptions} takes one option,
% \cs{DeclareLocalOptions} expects a comma separated list of options.
%
% \subsubsection{Dynamic options}
%
% Options of \LaTeX's package/class system are cleared
% in \cs{ProcessOptions}. They modify the static model
% of a package. For example, depending on option \xoption{bookmarks}
% package \xpackage{hyperref} loads differently.
%
% Options, however,
% defined by \xpackage{keyval}'s \cs{define@key} remain
% defined, if the options are processed by \cs{setkeys}.
% Therefore these options can also be used to model
% the dynamic behaviour of a package. For example, in
% \xpackage{hyperref} the link colors can be changed
% everywhere until the end in |\end{document}|.
%
% However package \xpackage{color} that adds color support
% is necessary and it
% cannot be loaded after |\begin{document}|. Option \xoption{colorlinks}
% that loads \xpackage{color} should be active until |\begin{document}|
% and die in some way if it is too late for loading packages.
% With \cs{DisableKeyvalOption} the package/class author can
% specify and configure the death of an option and controls
% the life period of the option.
%
% \subsubsection{\cs{DisableKeyvalOption}}
%
% \begin{declcs}{DisableKeyvalOption}\,\O{options}\,\M{family}\,\M{key}
%   \\[0.5ex]
%   \meta{options}:\\
%   \hspace{1.5em}^^A
%   \begin{tabular}[t]{@{}l@{ \texttt{=} }l@{\qquad}>{default: }l@{}}
%     |action| & |undef|, |warning|, |error|, or |ignore|
%     & |undef|
%     \\
%     \multicolumn{1}{@{}l}{|global| or |local|} & & |global|
%     \\
%     |package| or |class| & \meta{name}
%     \\
%   \end{tabular}^^A
% \end{declcs}
% \cs{DisableKeyvalOption} can be called to mark the end when
% the option \meta{key} is no longer useful. The behaviour of
% an option after its death can be configured by \xoption{action}:
% \begin{description}
% \item[\texttt{undef}:] The option will be undefined,
%   If it is called, \cs{setkeys} reports an error because
%   of unknown key.
% \item[\texttt{error} or \texttt{warning}:] Use of the
%  option will cause an error or warning message.
%  Also these actions require that exclusivly either the
%  package or class name is given in options \xoption{package}
%  or \xoption{class}.
% \item[\texttt{ignore}:] The use of the option will silently
%  be ignored.
% \end{description}
% The option's death can be limited to the end of the current
% group, if option \xoption{local} is given. Default is
% \xoption{global}.
%
% The package/class author can wish the end of the option
% already during the package loading, then he will have
% static behaviour. In case of dynamic options \cs{DisableKeyvalOption}
% can be executed everywhere, also outside the package. Therefore
% the family name and the package/class name is usually unknown for
% \cs{DisableKeyvalOption}. Therefore the argument for the family
% name is mandatory and for some actions the package/class name
% must be provided.
%
% Usually a macro would configure the option death, Example:
%\begin{quote}
%\begin{verbatim}
%\ProvidesPackage{foobar}
%\DeclareBoolOption{color}
%\DeclareStringOption[red]{emphcolor}
%\ProcessKeyvalOptions*
%
%\newcommand*{\foobar@DisableOption}[2]{%
%  \DisableKeyvalueOption[
%    action={#1},
%    package=foobar
%  ]{foobar}{#2}%
%}
%
%\iffoobar@color
%  \RequirePackage{color}
%  \renewcommand*{\emph}[1]{\textcolor{\foobar@emphcolor}{#1}}
%\else
%  % Option emphcolor is not wrong, if we have color support.
%  % otherwise the option has no effect, but we don't want to
%  % remove it. Therefore action 'ignore' is the best choice:
%  \foobar@DisableOption{ignore}{emphcolor}
%\fi
%% No we don't need the option 'color'.
%\foobar@DisableOption{warning}{color}
%
%% With color support option 'emphcolor' will dynamically
%% change the color of \emph statements.
%\end{verbatim}
%\end{quote}
%
% \subsubsection{\cs{AddToKeyvalOption}}
%
% \begin{declcs}{AddToKeyvalOption}\,\M{family}\,\M{key}\,\M{code}\\
%   \cs{AddToKeyvalOption} \*\,\M{key}\,\M{code}
% \end{declcs}
% The code for an existing key \meta{key} of family \meta{family} is
% extended by code \meta{code}. In the starred form the current
% family setting is used, see \cs{ProcessKeyvalOptions*}.
%
% \subsection{Global vs. local options}
%
% Options that are given for \cs{documentclass} are called
% global options. They are known to the class and all packages.
% A package may make use of a global option and marks it as
% used. The advantage for the user is the freedom to specify
% options both in the \cs{documentclass} or \cs{usepackage}
% commands.
%
% However global options are shared with the class options
% and options of all other packages. Thus there can be the
% same option with different semantics for different
% packages and classes. As example, package \xpackage{bookmark}
% knows option \xoption{open} that specifies whether the bookmarks
% are opened or closed initially. It's values are |true| or |false|.
% Since KOMA-Script version 3.00 the KOMA classes also introduces
% option \xoption{open} with values |right| and |any| and a
% complete different meaning.
%
% Such conflicts can be resolved by marking all or part of options
% as local by \cs{DeclareLocalOption} or \cs{DeclareLocalOptions}.
% Then the packages ignores global occurences of these options.
% Package \xpackage{kvoptions} provides two methods:
% \begin{itemize}
% \item \cs{ProcessLocalKeyvalOptions} automatically uses all options
%   as local options. It ignores all global options.
% \item \cs{DeclareLocalOption} or \cs{DeclareLocalOptions} marks
%   options as local options. \cs{ProcessKeyvalOptions} will then
%   ignore global occurences for these local options.
% \end{itemize}
% Since version 1.5 package \xpackage{bookmark} uses the latter
% method. It checks global and local option places for driver options
% and limits all other options as local options. Thus the class
% option \xoption{open} of KOMA-Script is not misread as
% option for package \xpackage{bookmark}.
%
% \subsection{Summary of internal macros}
%
% The \cs{Declare}\texttt{\dots Option} commands define
% macros, additionally to the macros generated by the key definition.
% These macros can be used by the package/class author.
% The name of the macros starts with the prefix \meta{prefix}
% that can be configured by \cs{SetupKeyvalOptions}.
% \begin{center}
% \def\prefix{\cs{}\meta{prefix}\meta{key}}^^A
% \def\Hline{^^A
%   \hline
%   \rule{0pt}{1.2\ht\csname @arstrutbox\endcsname}^^A
% }^^A
% \begin{tabular}{@{}l|>{\ttfamily}l|l@{}}
%   Declare \meta{key} & \textrm{Defined macro} & Description\\
% \Hline
%   \cs{DeclareStringOption}
%    & \prefix & holds the string\\
% \Hline
%   \cs{DeclareBoolOption}
%     & \cs{if}\meta{prefix}\meta{key}& boolean switch\\
%   &\prefix false& disable switch\\
%   &\prefix true& enable switch\\
% \Hline
%   \cs{DeclareComplementaryOption}
%     & \prefix false& enable parent switch\\
%     & \prefix true& disable parent switch\\
% \Hline
%   \cs{DeclareVoidOption}
%   & \prefix & holds the action
% \end{tabular}
% \end{center}
%
% \subsection{\plainTeX}
%
% Package \xpackage{keyval} is also usable in \plainTeX\ with
% the help of file \xfile{miniltx.tex}. Some features of this
% package \xpackage{kvoptions} might also be useful for \plainTeX.
% If \LaTeX\ is not found, \cs{ProcessKeyvalOptions} and option
% \xoption{patch} are disabled. Before using the option declaration
% commands \cs{Declare...Option}, \cs{SetupKeyvalOptions} must be
% used.
%
% \hypersetup{bookmarksopenlevel=1}
%
% \section{Example}
%
% The following example defined a package that serves some
% private color management. A boolean option \xoption{print} enables
% print mode without colors. An option \xoption{emph} redefines
% \cs{emph} to print in the given color. And the driver
% can be specified by option \xoption{driver}.
%
%    \begin{macrocode}
%<*example>
    % Package identification
    % -----------------------
\NeedsTeXFormat{LaTeX2e}
\ProvidesPackage{example-mycolorsetup}[2016/05/16 Managing my colors]

\RequirePackage{ifpdf}
\RequirePackage{kvoptions}

    % Option declarations
    % -------------------

\SetupKeyvalOptions{
  family=MCS,
  prefix=MCS@
}
    % Use a shorter family name and prefix

    % Option print
\DeclareBoolOption{print}
    % is the same as
    % \DeclareBoolOption[false]{print}

    % Option driver
\ifpdf
  \DeclareStringOption[pdftex]{driver}
\else
  \DeclareStringOption[dvips]{driver}
\fi

    % Alternative interface for driver options
\DeclareVoidOption{dvips}{\SetupDriver}
\DeclareVoidOption{dvipdfm}{\SetupDriver}
\DeclareVoidOption{pdftex}{\SetupDriver}
    % In \SetupDriver we take the current option \CurrentOption
    % and pass it to the driver option.
    % The \expandafter commands expand \CurrentOption at the
    % time, when \SetupDriver is executed and \CurrentOption
    % has the correct meaning.
\newcommand*{\SetupDriver}{%
  \expandafter\@SetupDriver\expandafter{\CurrentOption}%
}
\newcommand*{\@SetupDriver}[1]{%
  \setkeys{MCS}{driver={#1}}%
}

    % Option emph
    % An empty value means, we want to have no color for \emph.
    % If the user specifies option emph without value, the red is used.
\DeclareStringOption{emph}[red]
    % is the same as
    % \DeclareStringOption[]{emph}[red]

    % Default option rule
\DeclareDefaultOption{%
  \ifx\CurrentOptionValue\relax
    \PackageWarningNoLine{\@currname}{%
      Unknown option `\CurrentOption'\MessageBreak
      is passed to package `color'%
    }%
    % Pass the option to package color.
    % Again it is better to expand \CurrentOption.
    \expandafter\PassOptionsToPackage
    \expandafter{\CurrentOption}{color}%
  \else
    % Package color does not take options with values.
    % We provide the standard LaTeX error.
    \@unknownoptionerror
  \fi
}

    % Process options
    % ---------------
\ProcessKeyvalOptions*

    % Implementation depending on option values
    % -----------------------------------------
    % Code for print mode
\ifMCS@print
  \PassOptionsToPackage{monochrome}{color}
    % tells package color to use black and white
\fi

\RequirePackage[\MCS@driver]{color}
    % load package color with the correct driver

    % \emph setup
\ifx\MCS@emph\@empty
    % \@empty is a predefined macro with empty contents.
    % the option value of option emph is empty, thus
    % we do not want a redefinition of \emph.
\else
  \renewcommand*{\emph}[1]{%
    \textcolor{\MCS@emph}{#1}%
  }
\fi
%</example>
%    \end{macrocode}
%
% \section{Package options}
%
% The package \xpackage{kvoptions} knows two package options
% \xoption{patch} and \xoption{debugshow}.
% The options of package \xpackage{kvoptions} are intended
% for authors, not for package/class writers. Inside a package
% it is too late for option \xoption{patch} and \xoption{debugshow}
% enables some messages that are perhaps useful for the debugging
% phase. Also \LaTeX\ is unhappy if a package is loaded
% later again with options that are previously not given.
% Thus package and class authors, stay with |\RequirePackage{kvoptions}|
% without options.
%
% Option \xoption{patch} loads package \xpackage{kvoptions-patch}.
%
% \subsection{Package \xpackage{kvoptions-patch}}
%
% \LaTeX's system of package/class options has some severe limitations
% that especially affects the value part if options are used as
% pair of key and value.
% \begin{itemize}
% \item Spaces are removed, regardless where:
%   \begin{quote}
%     |\documentclass[box=0 0 400 600]{article}|
%   \end{quote}
%   Now each package will see |box=00400600| as global option.
% \item In the previous case also braces would not help:
%   \begin{quote}
%     |\documentclass[box={0 0 400 600}]{article}|
%   \end{quote}
%   The result is an error message:
%   \begin{quote}
%     |! LaTeX Error: Missing \begin{document}.|
%   \end{quote}
%   As local option, however, it works if the package
%   knows about key value options (By using this package, for example).
% \item The requirements on robustness are extremly high.
%   \LaTeX\ expands the option. All that will not work as environment
%   name will break also as option. Even a \cs{relax} will generate
%   an error message:
%   \begin{quote}
%     |! Missing \endcsname inserted.|
%   \end{quote}
%   Of course, \LaTeX\ does not use its protecting mechanisms.
%   On contrary \cs{protect} itself will cause errors.
% \item The options are expanded. But perhaps the package will
%   do that, because it has to setup some things before?
%   Example \xpackage{hyperref}:
%   \begin{quote}
%     |\usepackage[pdfauthor=M\"uller]{hyperref}|
%   \end{quote}
%   Package \xpackage{hyperref} does not see |M\"uller| but
%   its expansion and it does not like it, you get many warnings
%   \begin{quote}
%     |Token not allowed in a PDFDocEncoded string|
%   \end{quote}
%   And the title becomes: |Mu127uller|.
%   Therefore such options must usually be given after package
%   \xpackage{hyperref}
%   is loaded:
%   \begin{quote}
%     |\usepackage{hyperref}|\\
%     |\hypersetup{pdfauthor=Fran\c coise M\"uller}|
%   \end{quote}
%   As package option it will even break with |Fran\c coise|
%   because of the cedilla |\c c|, it is not robust enough.
% \end{itemize}
% For users that do not want with this limitations the package
% offers package \xoption{kvoptions-patch}. It patches \LaTeX's option system
% and tries to teach it also to handle options that are given
% as pairs of key and value and to prevent expansion.
% It can already be used at the very beginning, before \cs{documentclass}:
% \begin{quote}
%   |\RequirePackage{kvoptions-patch}|\\
%   |\documentclass[pdfauthor=Fran\c coise M\"uller]{article}|\\
%   |\usepackage{hyperref}|
% \end{quote}
% The latest time is before the package where you want to use
% problematic values:
% \begin{quote}
%   |\usepackage{kvoptions-patch}|\\
%   |\usepackage[Fran\c coise M\"uller]{hyperref}|
% \end{quote}
% Some remarks:
% \begin{itemize}
% \item The patch requires \eTeX, its \cs{unexpanded} feature
%   is much too nice. It is possible to work around using token
%   registers. But the code becomes longer, slower, more difficult
%   to read and maintain. The package without option \xoption{patch}
%   works and will work without \eTeX.
% \item The code for the patch is quite long, there are many test
%   cases. Thus the probability for bugs is probably not too small.
% \item Since 2008/10/18 v3.0 package \xpackage{kvoptions-patch} is
%   available. Before option \xoption{patch} of package \xpackage{kvoptions}
%   must be used instead. I think, the solution as standalone package
%   \xpackage{kvoptions-patch} is cleaner and avoids option clashes.
% \end{itemize}
%
% \subsection{Option \xoption{debugshow}}
%
% The name of this option follows the convention of packages
% \xpackage{multicol}, \xpackage{tabularx}, and \xpackage{tracefnt}.
% Currently it prints the setting of boolean options, declared
% by \cs{DeclareBoolOption} in the \xfile{.log} file, if that
% boolean option is used.
% You can activate the option by
% \begin{itemize}
% \item |\PassOptionsToPackage{debugshow}{kvoptions}|\\
%       Put this somewhere before package \xpackage{kvoptions} is
%       loaded first, e.g. before \cs{documentclass}.
% \item |\RequirePackage[debugshow]{kvoptions}|\\
%       Before \cs{documentclass} even an author has to use
%       \cs{RequirePackage}. \cs{usepackage} only works after
%       \cs{documentclass}.
% \end{itemize}
% The preferred method is \cs{PassOptionsToPackage}, because
% it does not force the package loading and does not disturb,
% if the package is not loaded later at all.
%
% \section{Limitations}
%
% \subsection{Compatibility}
% \subsubsection{Package \xpackage{kvoptions-patch} vs. package \xpackage{xkvltxp}}
%
% Package \xpackage{xkvltxp} from the \xpackage{xkeyval} project
% has the same goal as package \xpackage{kvoptions-patch}
% and to patch \LaTeX's kernel commands in order to get better support
% for key value options.
% Of course they cannot be used both. The user must decide, which
% method he prefers. Package \xpackage{kvoptions-patch} aborts itself,
% if it detects that \xpackage{xkvltxp} is
% already loaded.
%
% However package \xpackage{xkvltxp} and \xpackage{kvoptions}
% can be used together, example:
% \begin{quote}
% |\usepackage{xkvltxp}|\\
% |\usepackage[...]{foobar} % |\texttt{\itshape foobar using kvoptions}
% \end{quote}
% The other way should work, too.
%
% Package \xpackage{kvoptions-patch} tries to catch more situations
% and to be more robust.
% For example, during the comparison of options it normalizes
% them by removing spaces around |=| and the value. Thus the
% following is not reported as option clash:
% \begin{quote}
%\begin{verbatim}
%\RequirePackage{kvoptions-patch}
%\documentclass{article}
%
%\usepackage[scaled=0.7]{helvet}
%\usepackage[scaled = 0.7]{helvet}
%
%\begin{document}
%\end{document}
%\end{verbatim}
% \end{quote}
%
% \subsection{Limitations}
%
% \subsubsection{Option comparisons}
%
% In some situations \LaTeX\ compares option lists, e.g. option clash check,
% \cs{@ifpackagewith}, or \cs{@ifclasswith}. Apart from catcode and
% sanitizing problems of option \xoption{patch}, there is another problem.
% \LaTeX\ does not know about the type and default values of
% options in key value style. Thus an option clash is reported,
% even if the key value has the same meaning:
% \begin{quote}
% |\usepackage[scaled]{helvet} %| \texttt{\textit{default is }.95}\\
% |\usepackage[.95]{helvet}|\\
% |\usepackage[0.95]{helvet}|
% \end{quote}
%
% \subsubsection{Option list parsing with package \xpackage{kvoptions-patch}}
%
% With package \xpackage{kvoptions-patch} the range of possible values
% in key value specifications is much large, for example
% the comma can be used, if enclosed in curly braces.
%
% Other packages, especially the packages that uses their
% own process option code can be surprised to find tokens
% inside options that they do not expect and errors would
% be the consequence. To avoid errors the options, especially
% the unused option list is sanitized. That means the list
% will only contain tokens with catcode 12 (other) and
% perhaps spaces (catcode 10). This allows a safe parsing
% for other packages. But a comma in the value part is
% no longer protected by curly braces because they have lost
% their special meaning. This is the price for compatibility.
%
% Example:
% \begin{quote}
% |\RequirePackage{kvoptions-patch}|\\
% |\documentclass[a={a,b,c},b]{article}|\\
% |\begin{document}|\\
% |\end{document}|
% \end{quote}
% Result:
% \begin{quote}
% |LaTeX Warning: Unused global option(s):|\\
% |    [a={a,c},b].|
% \end{quote}
%
% \StopEventually{
% }
%
% \section{Implementation}
%
% \subsection{Preamble}
%
%    \begin{macrocode}
%<*package>
%    \end{macrocode}
%
% \paragraph{Reload check and  identification.}
%    Reload check, especially if the package is not used with \LaTeX.
%    \begin{macrocode}
\begingroup\catcode61\catcode48\catcode32=10\relax%
  \catcode13=5 % ^^M
  \endlinechar=13 %
  \catcode35=6 % #
  \catcode39=12 % '
  \catcode44=12 % ,
  \catcode45=12 % -
  \catcode46=12 % .
  \catcode58=12 % :
  \catcode64=11 % @
  \catcode123=1 % {
  \catcode125=2 % }
  \expandafter\let\expandafter\x\csname ver@kvoptions.sty\endcsname
  \ifx\x\relax % plain-TeX, first loading
  \else
    \def\empty{}%
    \ifx\x\empty % LaTeX, first loading,
      % variable is initialized, but \ProvidesPackage not yet seen
    \else
      \expandafter\ifx\csname PackageInfo\endcsname\relax
        \def\x#1#2{%
          \immediate\write-1{Package #1 Info: #2.}%
        }%
      \else
        \def\x#1#2{\PackageInfo{#1}{#2, stopped}}%
      \fi
      \x{kvoptions}{The package is already loaded}%
      \aftergroup\endinput
    \fi
  \fi
\endgroup%
%    \end{macrocode}
%    Package identification:
%    \begin{macrocode}
\begingroup\catcode61\catcode48\catcode32=10\relax%
  \catcode13=5 % ^^M
  \endlinechar=13 %
  \catcode35=6 % #
  \catcode39=12 % '
  \catcode40=12 % (
  \catcode41=12 % )
  \catcode44=12 % ,
  \catcode45=12 % -
  \catcode46=12 % .
  \catcode47=12 % /
  \catcode58=12 % :
  \catcode64=11 % @
  \catcode91=12 % [
  \catcode93=12 % ]
  \catcode123=1 % {
  \catcode125=2 % }
  \expandafter\ifx\csname ProvidesPackage\endcsname\relax
    \def\x#1#2#3[#4]{\endgroup
      \immediate\write-1{Package: #3 #4}%
      \xdef#1{#4}%
    }%
  \else
    \def\x#1#2[#3]{\endgroup
      #2[{#3}]%
      \ifx#1\@undefined
        \xdef#1{#3}%
      \fi
      \ifx#1\relax
        \xdef#1{#3}%
      \fi
    }%
  \fi
\expandafter\x\csname ver@kvoptions.sty\endcsname
\ProvidesPackage{kvoptions}%
  [2016/05/16 v3.12 Key value format for package options (HO)]%
%    \end{macrocode}
%
% \paragraph{Catcodes}
%
%    \begin{macrocode}
\begingroup\catcode61\catcode48\catcode32=10\relax%
  \catcode13=5 % ^^M
  \endlinechar=13 %
  \catcode123=1 % {
  \catcode125=2 % }
  \catcode64=11 % @
  \def\x{\endgroup
    \expandafter\edef\csname KVO@AtEnd\endcsname{%
      \endlinechar=\the\endlinechar\relax
      \catcode13=\the\catcode13\relax
      \catcode32=\the\catcode32\relax
      \catcode35=\the\catcode35\relax
      \catcode61=\the\catcode61\relax
      \catcode64=\the\catcode64\relax
      \catcode123=\the\catcode123\relax
      \catcode125=\the\catcode125\relax
    }%
  }%
\x\catcode61\catcode48\catcode32=10\relax%
\catcode13=5 % ^^M
\endlinechar=13 %
\catcode35=6 % #
\catcode64=11 % @
\catcode123=1 % {
\catcode125=2 % }
\def\TMP@EnsureCode#1#2{%
  \edef\KVO@AtEnd{%
    \KVO@AtEnd
    \catcode#1=\the\catcode#1\relax
  }%
  \catcode#1=#2\relax
}
\TMP@EnsureCode{1}{14}% ^^A (comment)
\TMP@EnsureCode{2}{14}% ^^A (comment)
\TMP@EnsureCode{33}{12}% !
\TMP@EnsureCode{39}{12}% '
\TMP@EnsureCode{40}{12}% (
\TMP@EnsureCode{41}{12}% )
\TMP@EnsureCode{42}{12}% *
\TMP@EnsureCode{44}{12}% ,
\TMP@EnsureCode{45}{12}% -
\TMP@EnsureCode{46}{12}% .
\TMP@EnsureCode{47}{12}% /
\TMP@EnsureCode{58}{12}% :
\TMP@EnsureCode{62}{12}% >
\TMP@EnsureCode{91}{12}% [
\TMP@EnsureCode{93}{12}% ]
\TMP@EnsureCode{94}{7}% ^ (superscript)
\TMP@EnsureCode{96}{12}% `
\edef\KVO@AtEnd{\KVO@AtEnd\noexpand\endinput}
%    \end{macrocode}
%
% \paragraph{External resources.}
%    The package extends the support for key value pairs of
%    package \cs{keyval} to package options. Thus the package
%    needs to be loaded anyway. and we use it for
%    \cs{SetupKeyvalOptions}. AFAIK this does not disturb
%    users of \xpackage{xkeyval}.
%    \begin{macrocode}
\@ifundefined{define@key}{%
  \RequirePackage{keyval}\relax
}{}
%    \end{macrocode}
%
%    Macro \cs{DeclareLocalOptions} parses a comma separated key list
%    and uses \cs{comma@parse} of package \xpackage{kvsetkeys},
%    version 1.3.
%    \begin{macrocode}
\RequirePackage{ltxcmds}[2010/12/02]
\RequirePackage{kvsetkeys}[2007/09/29]
%    \end{macrocode}
%
% \paragraph{Provide macros for \plainTeX.}
%    \begin{macrocode}
\@ifundefined{@x@protect}{%
  \def\@x@protect#1\fi#2#3{%
    \fi\protect#1%
  }%
  \let\@typeset@protect\relax
}{}
\@ifundefined{@currname}{%
  \def\@currname{}%
}{}
\@ifundefined{@currext}{%
  \def\@currext{}%
}{}
%    \end{macrocode}
%
% \paragraph{Options}
%    Option \xoption{debugshow} enables additional lines of
%    code that prints information into the \xfile{.log}
%    file.
%    \begin{macrocode}
\DeclareOption{debugshow}{\catcode\@ne=9 }
%    \end{macrocode}
%
%    \begin{macrocode}
\DeclareOption{patch}{%
  \AtEndOfPackage{%
    \RequirePackage{kvoptions-patch}[2016/05/16]%
  }%
}
%    \end{macrocode}
%
%    Optionen auswerten:
%    \begin{macrocode}
\ProcessOptions\relax
%    \end{macrocode}
%
% \subsection{Option declaration macros}
%
% \subsubsection{\cs{SetupKeyvalOptions}}
%
%    The family for the key value pairs can be setup once
%    and is remembered later.
%    The package name seems a reasonable default for the
%    family key, if it is not set by the package author.
%
%    \begin{macro}{\KVO@family}
%    We cannot store the family setting in one macro,
%    because the package should be usable for many
%    other packages, too. Thus we remember the family
%    setting in a macro, whose name contains the
%    package name with extension, a key in \LaTeX's
%    class/package system.
%    \begin{macrocode}
\define@key{KVO}{family}{%
  \expandafter\edef\csname KVO@family@%
      \@currname.\@currext\endcsname{#1}%
}
\def\KVO@family{%
  \@ifundefined{KVO@family@\@currname.\@currext}{%
    \@currname
  }{%
    \csname KVO@family@\@currname.\@currext\endcsname
  }%
}
%    \end{macrocode}
%    \end{macro}
%
%    \begin{macro}{\KVO@prefix}
%    The value settings of options that are declared by \cs{DeclareBoolOption}
%    and \cs{DeclareStringOption} need to be saved in macros.
%    in the first case this is a switch \cs{if}\meta{prefix}\meta{key},
%    in the latter case a macro \cs{}\meta{prefix}\meta{key}.
%    The prefix can be configured, by \xoption{prefix} that is declared
%    here. The default is the package name with |@| appended.
%    \begin{macrocode}
\define@key{KVO}{prefix}{%
  \expandafter\edef\csname KVO@prefix@%
      \@currname.\@currext\endcsname{#1}%
}
\def\KVO@prefix{%
  \ltx@ifundefined{KVO@prefix@\@currname.\@currext}{%
    \@currname @%
  }{%
    \csname KVO@prefix@\@currname.\@currext\endcsname
  }%
}
%    \end{macrocode}
%    \end{macro}
%
%    \begin{macrocode}
\define@key{KVO}{setkeys}{%
  \expandafter\def\csname KVO@setkeys@%
      \@currname.\@currext\endcsname{#1}%
}
%    \end{macrocode}
%    \begin{macro}{\KVO@setkeys}
%    \begin{macrocode}
\def\KVO@setkeys{%
  \ltx@IfUndefined{KVO@setkeys@\@currname.\@currext}{%
    \setkeys
  }{%
    \csname KVO@setkeys@\@currname.\@currext\endcsname
  }%
}
%    \end{macrocode}
%    \end{macro}
%
%    \begin{macro}{\SetupKeyvalOptions}
%    The argument of \cs{SetupKeyvalOptions} expects a key value
%    list, known keys are \xoption{family} and \xoption{prefix}.
%    \begin{macrocode}
\newcommand*{\SetupKeyvalOptions}{%
  \kvsetkeys{KVO}%
}
%    \end{macrocode}
%    \end{macro}
%
% \subsubsection{\cs{DeclareBoolOption}}
%
%    \begin{macro}{\DeclareBoolOption}
%    Usually options of boolean type can be given by the
%    user without value and this means a setting to \emph{true}.
%    We follow this convention here. Also it simplifies the
%    user interface.
%
%    The switch is created and initialized with \emph{false}.
%    The default setting can be overwritten by the optional
%    argument.
%
%    \LaTeX's \cs{newif} does not check for already defined
%    macros, therefore we add this check here to prevent
%    the user from accidently redefining of \TeX's primitives
%    and other macros.
%    \begin{macrocode}
\newcommand*{\DeclareBoolOption}[2][false]{%
  \KVO@ifdefinable{if\KVO@prefix#2}{%
    \KVO@ifdefinable{\KVO@prefix#2true}{%
      \KVO@ifdefinable{\KVO@prefix#2false}{%
        \csname newif\expandafter\endcsname
        \csname if\KVO@prefix#2\endcsname
        \@ifundefined{\KVO@prefix#2#1}{%
          \PackageWarning{kvoptions}{%
            Initialization of option `#2' failed,\MessageBreak
            cannot set boolean option to `#1',\MessageBreak
            use `true' or `false', now using `false'%
          }%
        }{%
          \csname\KVO@prefix#2#1\endcsname
        }%
        \begingroup
          \edef\x{\endgroup
            \noexpand\define@key{\KVO@family}{#2}[true]{%
              \noexpand\KVO@boolkey{\@currname}%
              \ifx\@currext\@clsextension
                \noexpand\@clsextension
              \else
                \noexpand\@pkgextension
              \fi
              {\KVO@prefix}{#2}{####1}%
            }%
          }%
        \x
      }%
    }%
  }%
}
%    \end{macrocode}
%    \end{macro}
%
%    \begin{macro}{\DeclareComplementaryOption}
%    The first argument is the key name, the second the
%    key that must be a boolean option with the same
%    current family and prefix. A new switch is not
%    created for the new key, we have already a switch.
%    Instead we define switch setting commands to work
%    on the parent switch.
%    \begin{macrocode}
\newcommand*{\DeclareComplementaryOption}[2]{%
  \@ifundefined{if\KVO@prefix#2}{%
    \PackageError{kvoptions}{%
      Cannot generate option code for `#1',\MessageBreak
      parent switch `#2' does not exist%
    }{%
      You are inside %
      \ifx\@currext\@clsextension class\else package\fi\space
      `\@currname.\@currext'.\MessageBreak
      `\KVO@family' is used as familiy %
      for the keyval options.\MessageBreak
      `\KVO@prefix' serves as prefix %
      for internal switch macros.\MessageBreak
      \MessageBreak
      \@ehc
    }%
  }{%
    \KVO@ifdefinable{\KVO@prefix#1true}{%
      \KVO@ifdefinable{\KVO@prefix#1false}{%
        \expandafter\let\csname\KVO@prefix#1false\expandafter\endcsname
          \csname\KVO@prefix#2true\endcsname
        \expandafter\let\csname\KVO@prefix#1true\expandafter\endcsname
          \csname\KVO@prefix#2false\endcsname
%    \end{macrocode}
%    The same code part as in \cs{DeclareBoolOption} can
%    now be used.
%    \begin{macrocode}
        \begingroup
          \edef\x{\endgroup
            \noexpand\define@key{\KVO@family}{#1}[true]{%
              \noexpand\KVO@boolkey{\@currname}%
              \ifx\@currext\@clsextension
                \noexpand\@clsextension
              \else
                \noexpand\@pkgextension
              \fi
              {\KVO@prefix}{#1}{####1}%
            }%
          }%
        \x
      }%
    }%
  }%
}
%    \end{macrocode}
%    \end{macro}
%
%    \begin{macro}{\KVO@ifdefinable}
%    Generate the command token LaTeX's \cs{@ifdefinable} expects.
%    \begin{macrocode}
\def\KVO@ifdefinable#1{%
  \expandafter\@ifdefinable\csname #1\endcsname
}
%    \end{macrocode}
%    \end{macro}
%
%    \begin{macro}{\KVO@boolkey}
%    We check explicitly for |true| and |false| to prevent
%    the user from accidently calling other macros.
%    \begin{quote}
%    \begin{tabular}{@{}ll@{}}
%    |#1|& package/class name\\
%    |#2|& |\@pkgextension|/|\@clsextension|\\
%    |#3|& prefix \\
%    |#4|& key name\\
%    |#5|& new value
%    \end{tabular}
%    \end{quote}
%    \begin{macrocode}
\def\KVO@boolkey#1#2#3#4#5{%
  \edef\KVO@param{#5}%
  \ltx@onelevel@sanitize\KVO@param
  \ifx\KVO@param\KVO@true
    \expandafter\@firstofone
  \else
    \ifx\KVO@param\KVO@false
      \expandafter\expandafter\expandafter\@firstofone
    \else
      \ifx#2\@clsextension
        \expandafter\ClassWarning
      \else
        \expandafter\PackageWarning
      \fi
      {#1}{%
        Value `\KVO@param' is not supported by\MessageBreak
        option `#4'%
      }%
      \expandafter\expandafter\expandafter\@gobble
    \fi
  \fi
  {%
    ^^A\ifx#2\@clsextension
    ^^A  \expandafter\ClassInfo
    ^^A\else
    ^^A  \expandafter\PackageInfo
    ^^A\fi
    ^^A{#1}{[option] #4=\KVO@param}%
    \csname#3#4\KVO@param\endcsname
  }%
}
%    \end{macrocode}
%    \end{macro}
%    \begin{macro}{\KVO@true}
%    \begin{macro}{\KVO@false}
%    The macros \cs{KVO@true} and \cs{KVO@false} are used
%    for string comparisons. After \cs{ltx@onelevel@sanitize}
%    we have only tokens with catcode 12 (other).
%    \begin{macrocode}
\def\KVO@true{true}
\def\KVO@false{false}
\ltx@onelevel@sanitize\KVO@true
\ltx@onelevel@sanitize\KVO@false
%    \end{macrocode}
%    \end{macro}
%    \end{macro}
%
% \subsubsection{\cs{DeclareStringOption}}
%
%    \begin{macro}{\DeclareStringOption}
%    \begin{macrocode}
\newcommand*{\DeclareStringOption}[2][]{%
  \@ifnextchar[{%
    \KVO@DeclareStringOption{#1}{#2}@%
  }{%
    \KVO@DeclareStringOption{#1}{#2}{}[]%
  }%
}
%    \end{macrocode}
%    \end{macro}
%    \begin{macro}{\KVO@DeclareStringOption}
%    \begin{macrocode}
\def\KVO@DeclareStringOption#1#2#3[#4]{%
  \KVO@ifdefinable{\KVO@prefix#2}{%
    \@namedef{\KVO@prefix#2}{#1}%
    \begingroup
      \ifx\\#3\\%
        \toks@{}%
      \else
        \toks@{[{#4}]}%
      \fi
      \edef\x{\endgroup
        \noexpand\define@key{\KVO@family}{#2}\the\toks@{%
          ^^A\begingroup
          ^^A  \toks@{####1}%
          ^^A  \ifx\@currext\@clsextension
          ^^A    \noexpand\ClassInfo
          ^^A  \else
          ^^A    \noexpand\PackageInfo
          ^^A  \fi
          ^^A  {\@currname}{%
          ^^A    [option] #2={\noexpand\the\toks@}%
          ^^A  }%
          ^^A\endgroup
          \noexpand\def
          \expandafter\noexpand\csname\KVO@prefix#2\endcsname{####1}%
        }%
      }%
    \x
  }%
}
%    \end{macrocode}
%    \end{macro}
%
% \subsubsection{\cs{DeclareVoidOption}}
%
%    \begin{macro}{\DeclareVoidOption}
%    \begin{macrocode}
\newcommand*{\DeclareVoidOption}[2]{%
  \begingroup
    \let\next\@gobbletwo
    \KVO@ifdefinable{\KVO@prefix#1}{%
      \let\next\@firstofone
    }%
  \expandafter\endgroup
  \next{%
    \begingroup
      \edef\x{\endgroup
        \noexpand\define@key{\KVO@family}{#1}[\KVO@VOID@]{%
          \noexpand\KVO@voidkey{\@currname}%
          \ifx\@currext\@clsextension
            \noexpand\@clsextension
          \else
            \noexpand\@pkgextension
          \fi
          {#1}%
          {####1}%
          \expandafter\noexpand\csname\KVO@prefix#1\endcsname
        }%
      }%
    \x
    \begingroup
      \toks@{#2}%
    \expandafter\endgroup
    \expandafter\def
    \csname\KVO@prefix#1\expandafter\endcsname
    \expandafter{\the\toks@}%
  }%
}
\def\KVO@VOID@{@VOID@}
%    \end{macrocode}
%    \end{macro}
%    \begin{macro}{\KVO@voidkey}
%    \begin{tabular}{@{}ll@{}}
%    |#1|& package/class name\\
%    |#2|& |\@pkgextension|/|\@clsextension|\\
%    |#3|& key name\\
%    |#4|& default (|@VOID@|)\\
%    |#5|& macro with option code
%    \end{tabular}
%    \begin{macrocode}
\def\KVO@voidkey#1#2#3#4{%
  \def\CurrentOption{#3}%
  \begingroup
    \def\x{#4}%
  \expandafter\endgroup
  \ifx\x\KVO@VOID@
  \else
    \ifx#2\@clsextension
      \expandafter\ClassWarning
    \else
      \expandafter\PackageWarning
    \fi
    {#1}{%
      Unexpected value for option `#3'\MessageBreak
      is ignored%
    }%
  \fi
  ^^A\ifx#2\@clsextension
  ^^A  \expandafter\ClassInfo
  ^^A\else
  ^^A  \expandafter\PackageInfo
  ^^A\fi
  ^^A{#1}{[option] #3}%
}
%    \end{macrocode}
%    \end{macro}
%
% \subsubsection{\cs{DeclareDefaultOption}}
%
%    \begin{macro}{\DeclareDefaultOption}
%    \begin{macrocode}
\newcommand*{\DeclareDefaultOption}{%
  \@namedef{KVO@default@\@currname.\@currext}%
}
%    \end{macrocode}
%    \end{macro}
%
% \subsubsection{\cs{DeclareLocalOptions}}
%
%    \begin{macro}{\DeclareLocalOptions}
%    \begin{macrocode}
\newcommand*{\DeclareLocalOptions}[1]{%
  \comma@parse{#1}\KVO@DeclareLocalOption
}
%    \end{macrocode}
%    \end{macro}
%    \begin{macro}{\KVO@DeclareLocalOption}
%    \begin{macrocode}
\def\KVO@DeclareLocalOption#1{%
  \expandafter\def\csname KVO@local@\KVO@family @#1\endcsname{}%
}
%    \end{macrocode}
%    \end{macro}
%
% \subsection{Dynamic options}
%
% \subsubsection{\cs{DisableKeyvalOption}}
%
%    \begin{macrocode}
\SetupKeyvalOptions{%
  family=KVOdyn,%
  prefix=KVOdyn@%
}
\DeclareBoolOption[true]{global}
\DeclareComplementaryOption{local}{global}
\DeclareStringOption[undef]{action}
\let\KVOdyn@name\relax
\let\KVOdyn@ext\@empty
\define@key{KVOdyn}{class}{%
  \def\KVOdyn@name{#1}%
  \let\KVOdyn@ext\@clsextension
}
\define@key{KVOdyn}{package}{%
  \def\KVOdyn@name{#1}%
  \let\KVOdyn@ext\@pkgextension
}
\newcommand*{\DisableKeyvalOption}[3][]{%
  \begingroup
    \kvsetkeys{KVOdyn}{#1}%
    \def\x{\endgroup}%
    \@ifundefined{KVO@action@\KVOdyn@action}{%
      \PackageError{kvoptions}{%
        Unknown disable action %
        `\expandafter\strip@prefix\meaning\KVOdyn@action'\MessageBreak
        for option `#3' in keyval family '#2'%
      }\@ehc
    }{%
      \csname KVO@action@\KVOdyn@action\endcsname{#2}{#3}%
    }%
  \x
}
\def\KVO@action@undef#1#2{%
  \edef\x{\endgroup
    \ifKVOdyn@global\global\fi
    \let
    \expandafter\noexpand\csname KV@#1@#2\endcsname
    \relax
    \ifKVOdyn@global\global\fi
    \let
    \expandafter\noexpand\csname KV@#1@#2@default\endcsname
    \relax
  }%
  ^^A\PackageInfo{kvoptions}{%
  ^^A  [option] key `#2' of family `#1'\MessageBreak
  ^^A  is disabled (undef, \ifKVOdyn@global global\else local\fi)%
  ^^A}%
}
\def\KVO@action@ignore#1#2{%
  \edef\x{\endgroup
    \ifKVOdyn@global\global\fi
    \let
    \expandafter\noexpand\csname KV@#1@#2\endcsname
    \noexpand\@gobble
    \ifKVOdyn@global\global\fi
    \let
    \expandafter\noexpand\csname KV@#1@#2@default\endcsname
    \noexpand\@empty
  }%
  ^^A\PackageInfo{kvoptions}{%
  ^^A  [option] key `#2' of family `#1'\MessageBreak
  ^^A  is disabled (ignore, \ifKVOdyn@global global\else local\fi)%
  ^^A}%
}
\def\KVO@action@error{%
  \KVO@do@action{error}%
}
\def\KVO@action@warning{%
  \KVO@do@action{warning}%
}
%    \end{macrocode}
%    \begin{tabular}{@{}ll@{}}
%    |#1|& |error| or |warning|\\
%    |#2|& \meta{family}\\
%    |#3|& \meta{key}\\
%    \end{tabular}
%    \begin{macrocode}
\def\KVO@do@action#1#2#3{%
  \ifx\KVOdyn@name\relax
    \PackageError{kvoptions}{%
      Action type `#1' needs package/class name\MessageBreak
      for key `#3' in family `#2'%
    }\@ehc
  \else
    \edef\x{\endgroup
      \noexpand\define@key{#2}{#3}[]{%
        \expandafter\noexpand\csname KVO@disable@#1\endcsname
        {\KVOdyn@name}\noexpand\KVOdyn@ext{#3}%
      }%
      \ifKVOdyn@global
        \global\let
        \expandafter\noexpand\csname KV@#2@#3\endcsname
        \expandafter\noexpand\csname KV@#2@#3\endcsname
        \global\let
        \expandafter\noexpand\csname KV@#2@#3@default\endcsname
        \expandafter\noexpand\csname KV@#2@#3@default\endcsname
      \fi
    }%
    ^^A\ifx\KVOdyn@ext\@clsextension
    ^^A  \expandafter\ClassInfo
    ^^A\else
    ^^A   \expandafter\PackageInfo
    ^^A\fi
    ^^A{\KVOdyn@name}{%
    ^^A  [option] key `#3' of family `#2'\MessageBreak
    ^^A  is disabled (#1, \ifKVOdyn@global global\else local\fi)%
    ^^A}%
  \fi
}
\def\KVO@disable@error#1#2#3{%
  \ifx#2\@clsextension
    \expandafter\ClassError
  \else
    \expandafter\PackageError
  \fi
  {#1}{%
    Option `#3' is given too late,\MessageBreak
    now the option is ignored%
  }\@ehc
}
\def\KVO@disable@warning#1#2#3{%
  \ifx#2\@clsextension
    \expandafter\ClassWarning
  \else
    \expandafter\PackageWarning
  \fi
  {#1}{%
    Option `#3' is already consumed\MessageBreak
    and has no effect%
  }%
}
%    \end{macrocode}
%
% \subsection{Change option code}
%
% \subsubsection{\cs{AddToKeyvalOption}}
%
%    \begin{macro}{\AddToKeyvalOption}
%    \begin{macrocode}
\newcommand*{\AddToKeyvalOption}{%
  \@ifstar{%
    \begingroup
      \edef\x{\endgroup
        \noexpand\KVO@AddToKeyvalOption{\KVO@family}%
      }%
    \x
  }%
  \KVO@AddToKeyvalOption
}
%    \end{macrocode}
%    \end{macro}
%    \begin{macro}{\KVO@AddToKeyvalOption}
%    \begin{macrocode}
\def\KVO@AddToKeyvalOption#1#2{%
  \@ifundefined{KV@#1@#2}{%
    \PackageWarning{kvoptions}{%
      Key `#2' of family `#1' does not exist.\MessageBreak
      Ignoring \string\AddToKeyvalOption
    }%
    \@gobble
  }{%
    \edef\KVO@next{%
      \noexpand\KVO@@AddToKeyvalOption
      \expandafter\noexpand\csname KV@#1@#2\endcsname
    }%
    \afterassignment\KVO@next
    \def\KVO@temp##1%
  }%
}
%    \end{macrocode}
%    \end{macro}
%    \begin{macro}{\KVO@@AddToKeyvalOption}
%    \begin{macrocode}
\def\KVO@@AddToKeyvalOption#1{%
  \begingroup
    \toks@\expandafter{#1{##1}}%
    \toks@\expandafter{\the\expandafter\toks@\KVO@temp{##1}}%
    \edef\x{\endgroup
      \noexpand\def\noexpand#1####1{\the\toks@}%
    }%
  \x
}
%    \end{macrocode}
%    \end{macro}
%
% \subsection{Process options}
%
% \subsubsection{Get global options}
%
%    Package \xpackage{xkeyval} removes options with equal signs from
%    the global options (\cs{@classoptionslist}). The effect is that
%    other packages and classes will not see these global options anymore.
%    A bug-report was answered that this behaviour is ``by design''.
%    Thus I call it a design bug. Now getting the global options require
%    an algorithm instead of a simple macro call.
%    \begin{macrocode}
%</package>
%<*package|patch>
%    \end{macrocode}
%    \begin{macro}{\KVO@IfDefThen}
%    Call |#2| if command |#1| is defined and not \cs{relax}.
%    (Package \xpackage{kvoptions-patch} does not load
%    package \xpackage{ltxcmds}.)
%    \begin{macrocode}
\def\KVO@IfDefThen#1#2{%
  \ifx#1\ltx@undefined
  \else
    \ifx#1\relax
    \else
      #2%
    \fi
  \fi
}%
%    \end{macrocode}
%    \end{macro}
%    \begin{macro}{\KVO@GetClassOptionsList}
%    \begin{macrocode}
\def\KVO@GetClassOptionsList{%
  \let\KVO@classoptionslist\@classoptionslist
  \KVO@IfDefThen\@classoptionslist{%
    \KVO@IfDefThen\XKV@documentclass{%
      \ifx\XKV@documentclass\ltx@empty
      \else
        \KVO@IfDefThen\XKV@classoptionslist{%
          \ifx\XKV@classoptionslist\ltx@empty
          \else
            \let\KVO@classoptionslist\XKV@classoptionslist
          \fi
        }%
      \fi
    }%
  }%
}%
%    \end{macrocode}
%    \end{macro}
%    \begin{macrocode}
%</package|patch>
%<*package>
%    \end{macrocode}
%
% \subsubsection{\cs{ProcessKeyvalOptions}}
%
%    \begin{macro}{\ProcessKeyvalOptions}
%    If the optional star is given, we get the family
%    name and expand it for safety.
%    \begin{macrocode}
\newcommand*{\ProcessKeyvalOptions}{%
  \@ifstar{%
    \begingroup
      \edef\x{\endgroup
        \noexpand\KVO@ProcessKeyvalOptions{\KVO@family}%
      }%
    \x
  }%
  \KVO@ProcessKeyvalOptions
}
%    \end{macrocode}
%    \end{macro}
%
%    \begin{macrocode}
\def\KVO@ProcessKeyvalOptions#1{%
  \let\@tempc\relax
  \let\KVO@temp\@empty
%    \end{macrocode}
%    Add any global options that are known to KV to the start of
%    the list being built in |\KVO@temp| and mark them used (by
%    removing them from the unused option list).
%    \begin{macrocode}
  \ifx\@currext\@clsextension
  \else
    \KVO@GetClassOptionsList
    \ifx\KVO@classoptionslist\relax
    \else
      \@for\KVO@CurrentOption:=\KVO@classoptionslist\do{%
        \@ifundefined{KV@#1@\expandafter\KVO@getkey
                      \KVO@CurrentOption=\@nil}{%
        }{%
          \@ifundefined{KVO@local@#1@\expandafter\KVO@getkey
                        \KVO@CurrentOption=\@nil}{%
            \ifx\KVO@Patch Y%
              \edef\KVO@temp{%
                \etex@unexpanded\expandafter{%
                  \KVO@temp
                }%
                ,%
                \etex@unexpanded\expandafter{%
                  \KVO@CurrentOption
                }%
                ,%
              }%
              \ltx@onelevel@sanitize\KVO@CurrentOption
            \else
              \edef\KVO@temp{%
                \KVO@temp
                ,%
                \KVO@CurrentOption
                ,%
              }%
            \fi
            \@expandtwoargs\@removeelement\KVO@CurrentOption
              \@unusedoptionlist\@unusedoptionlist
          }{}%
        }%
      }%
    \fi
  \fi
%    \end{macrocode}
%     Now stick the package options at the end of the list and wrap
%     in a call to \cs{setkeys}. A class ignores unknown global
%     options, we must remove them to prevent error messages
%     from \cs{setkeys}.
%    \begin{macrocode}
  \begingroup
    \toks\tw@{}%
    \@ifundefined{opt@\@currname.\@currext}{%
      \toks@\expandafter{\KVO@temp}%
    }{%
      \toks@\expandafter\expandafter\expandafter{%
        \csname opt@\@currname.\@currext\endcsname
      }%
      \ifx\@currext\@clsextension
        \edef\CurrentOption{\the\toks@}%
        \toks@\expandafter{\KVO@temp}%
        \@for\CurrentOption:=\CurrentOption\do{%
          \@ifundefined{%
            KV@#1@\expandafter\KVO@getkey\CurrentOption=\@nil
          }{%
%    \end{macrocode}
%    A class puts not used options in the unused option list
%    unless there is a default handler.
%    \begin{macrocode}
            \@ifundefined{KVO@default@\@currname.\@currext}{%
              \ifx\KVO@Patch Y%
                \ltx@onelevel@sanitize\CurrentOption
              \fi
              \ifx\@unusedoptionlist\@empty
                \global\let\@unusedoptionlist\CurrentOption
              \else
                \expandafter\expandafter\expandafter\gdef
                \expandafter\expandafter\expandafter\@unusedoptionlist
                \expandafter\expandafter\expandafter{%
                  \expandafter\@unusedoptionlist
                  \expandafter,\CurrentOption
                }%
              \fi
            }{%
              \toks\tw@\expandafter{%
                \the\toks\expandafter\tw@\expandafter,\CurrentOption
              }%
            }%
          }{%
            \toks@\expandafter{%
              \the\expandafter\toks@\expandafter,\CurrentOption
            }%
          }%
        }%
      \else
%    \end{macrocode}
%    Without default action we pass all options to \cs{setkeys}.
%    Otherwise we have to check which options are known.
%    These are passed to \cs{setkeys}. For the others the default
%    action is performed.
%    \begin{macrocode}
        \@ifundefined{KVO@default@\@currname.\@currext}{%
          \toks@\expandafter\expandafter\expandafter{%
            \expandafter\KVO@temp\the\toks@
          }%
        }{%
          \edef\CurrentOption{\the\toks@}%
          \toks@\expandafter{\KVO@temp}%
          \@for\CurrentOption:=\CurrentOption\do{%
            \@ifundefined{%
              KV@#1@\expandafter\KVO@getkey\CurrentOption=\@nil
            }{%
              \toks\tw@\expandafter{%
                \the\toks\expandafter\tw@\expandafter,\CurrentOption
              }%
            }{%
              \toks@\expandafter{%
                \the\expandafter\toks@\expandafter,\CurrentOption
              }%
            }%
          }%
        }%
      \fi
    }%
    \edef\KVO@temp{\endgroup
      \noexpand\KVO@calldefault{\the\toks\tw@}%
      \noexpand\KVO@setkeys{#1}{\the\toks@}%
    }%
  \KVO@temp
%    \end{macrocode}
%    Some cleanup of \cs{ProcessOptions}.
%    \begin{macrocode}
  \let\CurrentOption\@empty
  \AtEndOfPackage{\let\@unprocessedoptions\relax}%
}
%    \end{macrocode}
%
% \subsubsection{\cs{ProcessLocalKeyvalOptions}}
%
%    \begin{macro}{\ProcessLocalKeyvalOptions}
%    If the optional star is given, we get the family
%    name and expand it for safety.
%    \begin{macrocode}
\newcommand*{\ProcessLocalKeyvalOptions}{%
  \@ifstar{%
    \begingroup
      \edef\x{\endgroup
        \noexpand\KVO@ProcessLocalKeyvalOptions{\KVO@family}%
      }%
    \x
  }%
  \KVO@ProcessLocalKeyvalOptions
}
%    \end{macrocode}
%    \end{macro}
%
%    \begin{macrocode}
\def\KVO@ProcessLocalKeyvalOptions#1{%
  \let\@tempc\relax
  \let\KVO@temp\@empty
%    \end{macrocode}
%    Check if \cs{ProcessLocalKeyvalOptions} is called inside
%    a package.
%    \begin{macrocode}
  \ifx\@currext\@pkgextension
  \else
    \PackageError{kvoptions}{%
      \string\ProcessLocalKeyvalOptions\space is intended for packages only%
    }\@ehc
  \fi
%    \end{macrocode}
%     The package options are put into toks register \cs{toks@}.
%    \begin{macrocode}
  \begingroup
    \toks\tw@{}%
    \@ifundefined{opt@\@currname.\@currext}{%
      \toks@\expandafter{\KVO@temp}%
    }{%
      \toks@\expandafter\expandafter\expandafter{%
        \csname opt@\@currname.\@currext\endcsname
      }%
%    \end{macrocode}
%    Without default action we pass all options to \cs{setkeys}.
%    Otherwise we have to check which options are known.
%    These are passed to \cs{setkeys}. For the others the default
%    action is performed.
%    \begin{macrocode}
      \@ifundefined{KVO@default@\@currname.\@currext}{%
        \toks@\expandafter\expandafter\expandafter{%
          \expandafter\KVO@temp\the\toks@
        }%
      }{%
        \edef\CurrentOption{\the\toks@}%
        \toks@\expandafter{\KVO@temp}%
        \@for\CurrentOption:=\CurrentOption\do{%
          \@ifundefined{%
            KV@#1@\expandafter\KVO@getkey\CurrentOption=\@nil
          }{%
            \toks\tw@\expandafter{%
              \the\toks\expandafter\tw@\expandafter,\CurrentOption
            }%
          }{%
            \toks@\expandafter{%
              \the\expandafter\toks@\expandafter,\CurrentOption
            }%
          }%
        }%
      }%
    }%
    \edef\KVO@temp{\endgroup
      \noexpand\KVO@calldefault{\the\toks\tw@}%
      \noexpand\KVO@setkeys{#1}{\the\toks@}%
    }%
  \KVO@temp
%    \end{macrocode}
%    Some cleanup of \cs{ProcessOptions}.
%    \begin{macrocode}
  \let\CurrentOption\@empty
  \AtEndOfPackage{\let\@unprocessedoptions\relax}%
}
%    \end{macrocode}
%
% \subsubsection{Helper macros}
%
%    \begin{macro}{\KVO@getkey}
%    Extract the key part of a key=value pair.
%    \begin{macrocode}
\def\KVO@getkey#1=#2\@nil{#1}
%    \end{macrocode}
%    \end{macro}
%
%    \begin{macro}{\KVO@calldefault}
%    \begin{macrocode}
\def\KVO@calldefault#1{%
  \begingroup
    \def\x{#1}%
  \expandafter\endgroup
  \ifx\x\@empty
  \else
    \@for\CurrentOption:=#1\do{%
      \ifx\CurrentOption\@empty
      \else
        \expandafter\KVO@setcurrents\CurrentOption=\@nil
        \@nameuse{KVO@default@\@currname.\@currext}%
      \fi
    }%
  \fi
}
%    \end{macrocode}
%    \end{macro}
%    \begin{macro}{\KVO@setcurrents}
%    Extract the key part of a key=value pair.
%    \begin{macrocode}
\def\KVO@setcurrents#1=#2\@nil{%
  \def\CurrentOptionValue{#2}%
  \ifx\CurrentOptionValue\@empty
    \let\CurrentOptionKey\CurrentOption
    \let\CurrentOptionValue\relax
  \else
    \edef\CurrentOptionKey{\zap@space#1 \@empty}%
    \expandafter\KVO@setcurrentvalue\CurrentOption\@nil
  \fi
}
%    \end{macrocode}
%    \end{macro}
%    \begin{macro}{\KV@setcurrentvalue}
%    Here the value part is parsed. Package \xpackage{keyval}'s
%    \cs{KV@@sp@def} helps in removing spaces at the begin and
%    end of the value.
%    \begin{macrocode}
\def\KVO@setcurrentvalue#1=#2\@nil{%
  \KV@@sp@def\CurrentOptionValue{#2}%
}
%    \end{macrocode}
%    \end{macro}
%
% \subsection{\plainTeX}
%
%    Disable \LaTeX\ stuff.
%    \begin{macrocode}
\begingroup\expandafter\expandafter\expandafter\endgroup
\expandafter\ifx\csname documentclass\endcsname\relax
  \def\ProcessKeyvalOptions{%
    \@ifstar{}\@gobble
  }%
\fi
%    \end{macrocode}
%
%    \begin{macrocode}
\KVO@AtEnd%
%</package>
%    \end{macrocode}
%
% \subsection{Package \xpackage{kvoptions-patch}}
%
%    \begin{macrocode}
%<*patch>
\NeedsTeXFormat{LaTeX2e}
\begingroup\catcode61\catcode48\catcode32=10\relax%
  \catcode13=5 % ^^M
  \endlinechar=13 %
  \catcode123=1 % {
  \catcode125=2 % }
  \catcode64=11 % @
  \def\x{\endgroup
    \expandafter\edef\csname KVO@AtEnd\endcsname{%
      \endlinechar=\the\endlinechar\relax
      \catcode13=\the\catcode13\relax
      \catcode32=\the\catcode32\relax
      \catcode35=\the\catcode35\relax
      \catcode61=\the\catcode61\relax
      \catcode64=\the\catcode64\relax
      \catcode123=\the\catcode123\relax
      \catcode125=\the\catcode125\relax
    }%
  }%
\x\catcode61\catcode48\catcode32=10\relax%
\catcode13=5 % ^^M
\endlinechar=13 %
\catcode35=6 % #
\catcode64=11 % @
\catcode123=1 % {
\catcode125=2 % }
\def\TMP@EnsureCode#1#2{%
  \edef\KVO@AtEnd{%
    \KVO@AtEnd
    \catcode#1=\the\catcode#1\relax
  }%
  \catcode#1=#2\relax
}
\TMP@EnsureCode{39}{12}% '
\TMP@EnsureCode{40}{12}% (
\TMP@EnsureCode{41}{12}% )
\TMP@EnsureCode{43}{12}% +
\TMP@EnsureCode{44}{12}% ,
\TMP@EnsureCode{45}{12}% -
\TMP@EnsureCode{46}{12}% .
\TMP@EnsureCode{47}{12}% /
\TMP@EnsureCode{58}{12}% :
\TMP@EnsureCode{60}{12}% <
\TMP@EnsureCode{62}{12}% >
\TMP@EnsureCode{91}{12}% [
\TMP@EnsureCode{93}{12}% ]
\TMP@EnsureCode{96}{12}% `
\TMP@EnsureCode{124}{12}% |
\edef\KVO@AtEnd{\KVO@AtEnd\noexpand\endinput}
\ProvidesPackage{kvoptions-patch}%
  [2016/05/16 v3.12 LaTeX patch for keyval options (HO)]%
%    \end{macrocode}
%
%    Check for \eTeX.
%    \begin{macrocode}
\begingroup\expandafter\expandafter\expandafter\endgroup
\expandafter\ifx\csname eTeXversion\endcsname\relax
  \PackageWarningNoLine{kvoptions-patch}{%
    Package loading is aborted, because e-TeX is missing%
  }%
  \expandafter\KVO@AtEnd
\fi%
%    \end{macrocode}
%
%    Package \xpackage{etexcmds} for \cs{etex@unexpanded}.
%    \begin{macrocode}
\RequirePackage{etexcmds}[2007/09/09]
\ifetex@unexpanded
\else
  \PackageError{kvoptions-patch}{%
    Could not find eTeX's \string\unexpanded.\MessageBreak
    Try adding \string\RequirePackage\string{etexcmds\string} %
    before \string\documentclass%
  }\@ehd
  \expandafter\KVO@AtEnd
\fi%
%    \end{macrocode}
%
%    Check for package \xpackage{xkvltxp}.
%    \begin{macrocode}
\@ifpackageloaded{xkvltxp}{%
  \PackageWarningNoLine{kvoptions}{%
    Option `patch' cannot be used together with\MessageBreak
    package `xkvltxp' that is already loaded.\MessageBreak
    Therefore package loading is aborted%
  }%
  \KVO@AtEnd
}{}%
%    \end{macrocode}
%
%    \begin{macrocode}
\def\@if@ptions#1#2#3{%
  \begingroup
    \KVO@normalize\KVO@temp{#3}%
    \edef\x{\endgroup
      \noexpand\@if@pti@ns{%
        \detokenize\expandafter\expandafter\expandafter{%
          \csname opt@#2.#1\endcsname
        }%
      }{%
        \detokenize\expandafter{\KVO@temp}%
      }%
    }%
  \x
}
%    \end{macrocode}
%
%    \begin{macrocode}
\def\@pass@ptions#1#2#3{%
  \KVO@normalize\KVO@temp{#2}%
  \@ifundefined{opt@#3.#1}{%
    \expandafter\gdef\csname opt@#3.#1%
          \expandafter\endcsname\expandafter{%
      \KVO@temp
    }%
  }{%
    \expandafter\gdef\csname opt@#3.#1%
          \expandafter\expandafter\expandafter\endcsname
          \expandafter\expandafter\expandafter{%
      \csname opt@#3.#1\expandafter\endcsname\expandafter,\KVO@temp
    }%
  }%
}
%    \end{macrocode}
%
%    \begin{macrocode}
\def\ProcessOptions{%
  \let\ds@\@empty
  \@ifundefined{opt@\@currname.\@currext}{%
    \let\@curroptions\@empty
  }{%
    \expandafter\expandafter\expandafter\def
    \expandafter\expandafter\expandafter\@curroptions
    \expandafter\expandafter\expandafter{%
      \csname opt@\@currname.\@currext\endcsname
    }%
  }%
  \@ifstar\KVO@xprocess@ptions\KVO@process@ptions
}
%    \end{macrocode}
%
%    \begin{macrocode}
\def\KVO@process@ptions{%
  \@for\CurrentOption:=\@declaredoptions\do{%
    \ifx\CurrentOption\@empty
    \else
      \begingroup
        \ifx\@currext\@clsextension
          \toks@{}%
        \else
          \KVO@GetClassOptionsList
          \toks@\expandafter{\KVO@classoptionslist,}%
        \fi
        \toks\tw@\expandafter{\@curroptions}%
        \edef\x{\endgroup
          \noexpand\in@{,\CurrentOption,}{,\the\toks@\the\toks\tw@,}%
        }%
      \x
      \ifin@
        \KVO@use@ption
        \expandafter\let\csname ds@\CurrentOption\endcsname\@empty
      \fi
    \fi
  }%
  \KVO@process@pti@ns
}
%    \end{macrocode}
%
%    \begin{macrocode}
\def\KVO@xprocess@ptions{%
  \ifx\@currext\@clsextension
  \else
    \KVO@GetClassOptionsList
    \@for\CurrentOption:=\KVO@classoptionslist\do{%
      \ifx\CurrentOption\@empty
      \else
        \KVO@in@\CurrentOption\@declaredoptions
        \ifin@
          \KVO@use@ption
          \expandafter\let\csname ds@\CurrentOption\endcsname\@empty
        \fi
      \fi
    }%
  \fi
  \KVO@process@pti@ns
}
%    \end{macrocode}
%
%    \begin{macrocode}
\def\KVO@in@#1#2{%
  \in@false
  \begingroup
    \@for\x:=#2\do{%
      \ifx\x#1\relax
        \in@true
      \fi
    }%
    \edef\x{\endgroup
      \ifin@
        \noexpand\in@true
      \fi
    }%
  \x
}
%    \end{macrocode}
%
%    \begin{macrocode}
\def\KVO@process@pti@ns{%
  \@for\CurrentOption:=\@curroptions\do{%
    \@ifundefined{ds@\KVO@SanitizedCurrentOption}{%
      \KVO@use@ption
      \default@ds
    }%
    \KVO@use@ption
  }%
  \@for\CurrentOption:=\@declaredoptions\do{%
    \expandafter\let\csname ds@\CurrentOption\endcsname\relax
  }%
  \let\CurrentOption\@empty
  \let\@fileswith@pti@ns\@@fileswith@pti@ns
  \AtEndOfPackage{\let\@unprocessedoptions\relax}%
}
%    \end{macrocode}
%
%    \begin{macrocode}
\def\KVO@use@ption{%
  \begingroup
    \edef\x{\endgroup
      \noexpand\@removeelement{%
        \detokenize\expandafter{\CurrentOption}%
      }{%
        \detokenize\expandafter{\@unusedoptionlist}%
      }%
    }%
  \x\@unusedoptionlist
  \csname ds@\KVO@SanitizedCurrentOption\endcsname
}
%    \end{macrocode}
%
%    \begin{macrocode}
\def\OptionNotUsed{%
  \ifx\@currext\@clsextension
    \xdef\@unusedoptionlist{%
      \ifx\@unusedoptionlist\@empty
      \else
        \detokenize\expandafter{\@unusedoptionlist,}%
      \fi
      \detokenize\expandafter{\CurrentOption}%
    }%
  \fi
}
%    \end{macrocode}
%
%    Variant of \cs{ExecuteOptions} that better protects \cs{CurrentOption}.
%    \begin{macrocode}
\def\CurrentOption@SaveLevel{0}
\def\ExecuteOptions{%
  \expandafter\KVO@ExecuteOptions
      \csname CurrentOption@\CurrentOption@SaveLevel\endcsname
}
\def\KVO@ExecuteOptions#1#2{%
  \let#1\CurrentOption
  \edef\CurrentOption@SaveLevel{%
    \the\numexpr\CurrentOption@SaveLevel+1%
  }%
  \@for\CurrentOption:=#2\do{%
    \csname ds@\CurrentOption\endcsname
  }%
  \edef\CurrentOption@SaveLevel{%
    \the\numexpr\CurrentOption@SaveLevel-1%
  }%
  \let\CurrentOption#1%
}
%    \end{macrocode}
%
%    \begin{macrocode}
\def\KVO@fileswith@pti@ns#1[#2]#3[#4]{%
  \ifx#1\@clsextension
    \ifx\@classoptionslist\relax
      \KVO@normalize\KVO@temp{#2}%
      \expandafter\gdef\expandafter\@classoptionslist\expandafter{%
        \KVO@temp
      }%
      \def\reserved@a{%
        \KVO@onefilewithoptions{#3}[{#2}][{#4}]#1%
        \@documentclasshook
      }%
    \else
      \def\reserved@a{%
        \KVO@onefilewithoptions{#3}[{#2}][{#4}]#1%
      }%
    \fi
  \else
    \begingroup
      \let\KVO@temp\relax
      \let\KVO@onefilewithoptions\relax
      \let\@pkgextension\relax
      \def\reserved@b##1,{%
        \ifx\@nil##1\relax
        \else
          \ifx\relax##1\relax
          \else
            \KVO@onefilewithoptions{##1}[{\KVO@temp}][{#4}]%
            \@pkgextension
          \fi
          \expandafter\reserved@b
        \fi
      }%
      \edef\reserved@a{\zap@space#3 \@empty}%
      \edef\reserved@a{\expandafter\reserved@b\reserved@a,\@nil,}%
      \toks@{#2}%
      \def\KVO@temp{\the\toks@}%
    \edef\reserved@a{\endgroup \reserved@a}%
  \fi
  \reserved@a
}
%    \end{macrocode}
%
%    \begin{macrocode}
\def\KVO@onefilewithoptions#1[#2][#3]#4{%
  \@pushfilename
  \xdef\@currname{#1}%
  \global\let\@currext#4%
  \expandafter\let\csname\@currname.\@currext-h@@k\endcsname\@empty
  \let\CurrentOption\@empty
  \@reset@ptions
  \makeatletter
  \def\reserved@a{%
    \@ifl@aded\@currext{#1}{%
      \@if@ptions\@currext{#1}{#2}{%
      }{%
        \begingroup
          \@ifundefined{opt@#1.\@currext}{%
            \def\x{}%
          }{%
            \edef\x{%
              \expandafter\expandafter\expandafter\strip@prefix
              \expandafter\meaning\csname opt@#1.\@currext\endcsname
            }%
          }%
          \def\y{#2}%
          \edef\y{\expandafter\strip@prefix\meaning\y}%
          \@latex@error{Option clash for \@cls@pkg\space #1}{%
            The package #1 has already been loaded %
            with options:\MessageBreak
            \space\space[\x]\MessageBreak
            There has now been an attempt to load it %
             with options\MessageBreak
            \space\space[\y]\MessageBreak
            Adding the global options:\MessageBreak
            \space\space
                 \x,\y\MessageBreak
            to your \noexpand\documentclass declaration may fix this.%
            \MessageBreak
            Try typing \space <return> \space to proceed.%
          }%
        \endgroup
      }%
    }{%
      \@pass@ptions\@currext{#2}{#1}%
      \global\expandafter
      \let\csname ver@\@currname.\@currext\endcsname\@empty
      \InputIfFileExists
        {\@currname.\@currext}%
        {}%
        {\@missingfileerror\@currname\@currext}%
      \let\@unprocessedoptions\@@unprocessedoptions
      \csname\@currname.\@currext-h@@k\endcsname
      \expandafter\let\csname\@currname.\@currext-h@@k\endcsname
              \@undefined
      \@unprocessedoptions
    }%
    \@ifl@ter\@currext{#1}{#3}{%
    }{%
      \@latex@warning@no@line{%
        You have requested,\on@line, %
        version\MessageBreak
          #3' of \@cls@pkg\space #1,\MessageBreak
        but only version\MessageBreak
         `\csname ver@#1.\@currext\endcsname'\MessageBreak
        is available%
      }%
    }%
    \ifx\@currext\@clsextension\let\LoadClass\@twoloadclasserror\fi
    \@popfilename
    \@reset@ptions
  }%
  \reserved@a
}
%    \end{macrocode}
%
%    \begin{macrocode}
\def\@unknownoptionerror{%
  \@latex@error{%
    Unknown option `\KVO@SanitizedCurrentOption' %
    for \@cls@pkg\space`\@currname'%
  }{%
    The option `\KVO@SanitizedCurrentOption' was not declared in %
    \@cls@pkg\space`\@currname', perhaps you\MessageBreak
    misspelled its name. %
    Try typing \space <return> %
    \space to proceed.%
  }%
}
%    \end{macrocode}
%
%    \begin{macrocode}
\def\@@unprocessedoptions{%
  \ifx\@currext\@pkgextension
    \@ifundefined{opt@\@currname.\@currext}{%
      \let\@curroptions\@empty
    }{%
      \expandafter\let\expandafter\@curroptions
          \csname opt@\@currname.\@currext\endcsname
    }%
    \@for\CurrentOption:=\@curroptions\do{%
        \ifx\CurrentOption\@empty\else\@unknownoptionerror\fi
    }%
  \fi
}
%    \end{macrocode}
%
%    \begin{macrocode}
\def\KVO@SanitizedCurrentOption{%
  \expandafter\strip@prefix\meaning\CurrentOption
}
%    \end{macrocode}
%
%    Normalize option list.
%    \begin{macrocode}
\def\KVO@normalize#1#2{%
  \let\KVO@result\@empty
  \KVO@splitcomma#2,\@nil
  \let#1\KVO@result
}
\def\KVO@splitcomma#1,#2\@nil{%
  \KVO@ifempty{#1}{}{%
    \KVO@checkkv#1=\@nil
  }%
  \KVO@ifempty{#2}{}{\KVO@splitcomma#2\@nil}%
}
\def\KVO@ifempty#1{%
  \expandafter\ifx\expandafter\\\detokenize{#1}\\%
    \expandafter\@firstoftwo
  \else
    \expandafter\@secondoftwo
  \fi
}
\def\KVO@checkkv#1=#2\@nil{%
  \KVO@ifempty{#2}{%
    % option without value
    \edef\KVO@x{\zap@space#1 \@empty}%
    \ifx\KVO@x\@empty
      % ignore empty option
    \else
      % append to list
      \edef\KVO@result{%
        \etex@unexpanded\expandafter{\KVO@result},\KVO@x
      }%
    \fi
  }{%
    % #1: "key", #2: "value="
    % add key part
    \edef\KVO@result{%
      \etex@unexpanded\expandafter{\KVO@result},%
      \zap@space#1 \@empty
    }%
    \futurelet\@let@token\KVO@checkfirsttok#2 \@nil| = \@nil|\KVO@nil
  }%
}
\def\KVO@checkfirsttok{%
  \ifx\@let@token\bgroup
    % no space at start
    \expandafter\KVO@removelastspace\expandafter=%
    % "<value><spaceopt>= \@nil"
  \else
    \expandafter\KVO@checkfirstA
  \fi
}
\def\KVO@checkfirstA#1 #2\@nil{%
  \KVO@ifempty{#2}{%
    \KVO@removelastspace=#1 \@nil
  }{%
    \KVO@ifempty{#1}{%
      \KVO@removelastspace=#2\@nil
    }{%
      \KVO@removelastspace=#1 #2\@nil
    }%
  }%
}
\def\KVO@removelastspace#1 = \@nil|#2\KVO@nil{%
  \KVO@ifempty{#2}{%
    \edef\KVO@result{%
      \etex@unexpanded\expandafter{\KVO@result}%
      \etex@unexpanded\expandafter{\KVO@removegarbage#1\KVO@nil}%
    }%
  }{%
    \edef\KVO@result{%
      \etex@unexpanded\expandafter{\KVO@result}%
      \etex@unexpanded{#1}%
    }%
  }%
}
\def\KVO@removegarbage#1= \@nil#2\KVO@nil{#1}%
%    \end{macrocode}
%
%    Arguments |#1| and |#2| are macros.
%    \begin{macrocode}
\def\KVO@removeelement#1#2{%
  \begingroup
    \toks@={}%
    \@for\x:=#2\do{%
      \ifx\x\@empty
      \else
        \ifx\x#1\relax
        \else
          \edef\t{\the\toks@}%
          \ifx\t\@empty
          \else
            \toks@\expandafter{\the\toks@,}%
          \fi
          \toks@\expandafter{\the\expandafter\toks@\x}%
        \fi
      \fi
    }%
    \edef\x{\endgroup
      \def\noexpand#2{\the\toks@}%
    }%
  \x
}
%    \end{macrocode}
%
%
%    \begin{macrocode}
\let\@@fileswith@pti@ns\KVO@fileswith@pti@ns
\ifx\@fileswith@pti@ns\@badrequireerror
\else
  \let\@fileswith@pti@ns\KVO@fileswith@pti@ns
\fi
%    \end{macrocode}
%
%    \begin{macro}{\KVO@Patch}
%    \begin{macrocode}
\let\KVO@Patch=Y
%    \end{macrocode}
%    \end{macro}
%
%    \begin{macrocode}
\KVO@AtEnd%
%</patch>
%    \end{macrocode}
%
% \section{Test}
%
% \subsection{Preface for standard catcode check}
%
%    \begin{macrocode}
%<*test1>
\input miniltx.tex\relax
%</test1>
%    \end{macrocode}
%
% \subsection{Catcode checks for loading}
%
%    \begin{macrocode}
%<*test1>
%    \end{macrocode}
%    \begin{macrocode}
\catcode`\{=1 %
\catcode`\}=2 %
\catcode`\#=6 %
\catcode`\@=11 %
\expandafter\ifx\csname count@\endcsname\relax
  \countdef\count@=255 %
\fi
\expandafter\ifx\csname @gobble\endcsname\relax
  \long\def\@gobble#1{}%
\fi
\expandafter\ifx\csname @firstofone\endcsname\relax
  \long\def\@firstofone#1{#1}%
\fi
\expandafter\ifx\csname loop\endcsname\relax
  \expandafter\@firstofone
\else
  \expandafter\@gobble
\fi
{%
  \def\loop#1\repeat{%
    \def\body{#1}%
    \iterate
  }%
  \def\iterate{%
    \body
      \let\next\iterate
    \else
      \let\next\relax
    \fi
    \next
  }%
  \let\repeat=\fi
}%
\def\RestoreCatcodes{}
\count@=0 %
\loop
  \edef\RestoreCatcodes{%
    \RestoreCatcodes
    \catcode\the\count@=\the\catcode\count@\relax
  }%
\ifnum\count@<255 %
  \advance\count@ 1 %
\repeat

\def\RangeCatcodeInvalid#1#2{%
  \count@=#1\relax
  \loop
    \catcode\count@=15 %
  \ifnum\count@<#2\relax
    \advance\count@ 1 %
  \repeat
}
\def\RangeCatcodeCheck#1#2#3{%
  \count@=#1\relax
  \loop
    \ifnum#3=\catcode\count@
    \else
      \errmessage{%
        Character \the\count@\space
        with wrong catcode \the\catcode\count@\space
        instead of \number#3%
      }%
    \fi
  \ifnum\count@<#2\relax
    \advance\count@ 1 %
  \repeat
}
\def\space{ }
\expandafter\ifx\csname LoadCommand\endcsname\relax
  \def\LoadCommand{\input kvoptions.sty\relax}%
\fi
\def\Test{%
  \RangeCatcodeInvalid{0}{47}%
  \RangeCatcodeInvalid{58}{64}%
  \RangeCatcodeInvalid{91}{96}%
  \RangeCatcodeInvalid{123}{255}%
  \catcode`\@=12 %
  \catcode`\\=0 %
  \catcode`\%=14 %
  \LoadCommand
  \RangeCatcodeCheck{0}{36}{15}%
  \RangeCatcodeCheck{37}{37}{14}%
  \RangeCatcodeCheck{38}{47}{15}%
  \RangeCatcodeCheck{48}{57}{12}%
  \RangeCatcodeCheck{58}{63}{15}%
  \RangeCatcodeCheck{64}{64}{12}%
  \RangeCatcodeCheck{65}{90}{11}%
  \RangeCatcodeCheck{91}{91}{15}%
  \RangeCatcodeCheck{92}{92}{0}%
  \RangeCatcodeCheck{93}{96}{15}%
  \RangeCatcodeCheck{97}{122}{11}%
  \RangeCatcodeCheck{123}{255}{15}%
  \RestoreCatcodes
}
\Test
\csname @@end\endcsname
\end
%    \end{macrocode}
%    \begin{macrocode}
%</test1>
%    \end{macrocode}
%
%    \begin{macrocode}
%<*test2>
\NeedsTeXFormat{LaTeX2e}
\makeatletter
\catcode`\@=11 %
\def\RestoreCatcodes{}
\count@=0 %
\loop
  \edef\RestoreCatcodes{%
    \RestoreCatcodes
    \catcode\the\count@=\the\catcode\count@\relax
  }%
\ifnum\count@<255 %
  \advance\count@\@ne
\repeat

\def\RangeCatcodeInvalid#1#2{%
  \count@=#1\relax
  \loop
    \catcode\count@=15 %
  \ifnum\count@<#2\relax
    \advance\count@\@ne
  \repeat
}
\def\Test#1{%
  \RangeCatcodeInvalid{0}{47}%
  \RangeCatcodeInvalid{58}{64}%
  \RangeCatcodeInvalid{91}{96}%
  \RangeCatcodeInvalid{123}{255}%
  \catcode`\@=12 %
  \catcode`\\=0 %
  \catcode`\{=1 %
  \catcode`\}=2 %
  \catcode`\#=6 %
  \catcode`\[=12 %
  \catcode`\]=12 %
  \catcode`\%=14 %
  \catcode`\ =10 %
  \catcode13=5 %
  #1\relax
  \RestoreCatcodes
}
\Test{\RequirePackage{kvoptions-patch}}%
\Test{\RequirePackage{kvoptions}}%
\csname @@end\endcsname
%</test2>
%    \end{macrocode}
%
%    \begin{macrocode}
%<*test3>
\NeedsTeXFormat{LaTeX2e}
\makeatletter
\RequirePackage{kvoptions}[2016/05/16]
\def\msg#{\immediate\write16}
\define@key{testfamily}{testkey}{%
  \msg{[testfamily/testkey/#1]}%
}
\define@key{testfamily}{testdefaultkey}[testdefault]{%
  \msg{[testfamily/testdefaultkey/#1]}%
}
\AddToKeyvalOption{testfamily}{testkey}{%
  \msg{[addition/#1]}%
}
\AddToKeyvalOption{testfamily}{testdefaultkey}{%
  \msg{[addition/#1]}%
}
\setkeys{testfamily}{%
  testkey=testA,%
  testdefaultkey=testB,%
  testdefaultkey,%
}
\SetupKeyvalOptions{%
  family=testfamily%
}
\AddToKeyvalOption*{testkey}{%
  \msg{[star addition/#1]}%
}
\AddToKeyvalOption*{testdefaultkey}{%
  \msg{[star addition/#1]}%
}
\setkeys{testfamily}{%
  testkey=testA,%
  testdefaultkey=testB,%
  testdefaultkey,%
}
\@@end
%</test3>
%    \end{macrocode}
%
%    \begin{macrocode}
%<*test4pkg>
\NeedsTeXFormat{LaTeX2e}
\ProvidesPackage{kvoptions-test4}[2016/05/16 package for testing]
\RequirePackage{kvoptions}[2016/05/16]
\SetupKeyvalOptions{%
  family=FOO,%
  prefix=foo,%
  setkeys=\kvsetkeys,%
}
\DeclareStringOption{str}
\define@key{FOO}{set}{%
  \setkeys{BAR}{strbar={#1}}%
}
\define@key{BAR}{strbar}{%
  \def\foostr{[BAR:#1]}%
}
\ProcessKeyvalOptions*
%</test4pkg>
%    \end{macrocode}
%    \begin{macrocode}
%<*test4>
\NeedsTeXFormat{LaTeX2e}
\ProvidesFile{kvoptions-test4.tex}[2016/05/16 test file]
\RequirePackage[%
  str=A,set=B,str=C,%
]{kvoptions-test4}[2016/05/16]
\def\TestExpected{C}
\ifx\foostr\TestExpected
  \typeout{* Test ok.}%
\else
  \typeout{* Result: [\foostr]}%
  \typeout{* Expected: [\TestExpected]}%
  \errmessage{Test failed!}%
\fi
\csname @@end\endcsname\end
%</test4>
%    \end{macrocode}
%
% \section{Installation}
%
% \subsection{Download}
%
% \paragraph{Package.} This package is available on
% CTAN\footnote{\CTANpkg{kvoptions}}:
% \begin{description}
% \item[\CTAN{macros/latex/contrib/oberdiek/kvoptions.dtx}] The source file.
% \item[\CTAN{macros/latex/contrib/oberdiek/kvoptions.pdf}] Documentation.
% \end{description}
%
%
% \paragraph{Bundle.} All the packages of the bundle `oberdiek'
% are also available in a TDS compliant ZIP archive. There
% the packages are already unpacked and the documentation files
% are generated. The files and directories obey the TDS standard.
% \begin{description}
% \item[\CTANinstall{install/macros/latex/contrib/oberdiek.tds.zip}]
% \end{description}
% \emph{TDS} refers to the standard ``A Directory Structure
% for \TeX\ Files'' (\CTAN{tds/tds.pdf}). Directories
% with \xfile{texmf} in their name are usually organized this way.
%
% \subsection{Bundle installation}
%
% \paragraph{Unpacking.} Unpack the \xfile{oberdiek.tds.zip} in the
% TDS tree (also known as \xfile{texmf} tree) of your choice.
% Example (linux):
% \begin{quote}
%   |unzip oberdiek.tds.zip -d ~/texmf|
% \end{quote}
%
% \paragraph{Script installation.}
% Check the directory \xfile{TDS:scripts/oberdiek/} for
% scripts that need further installation steps.
%
% \subsection{Package installation}
%
% \paragraph{Unpacking.} The \xfile{.dtx} file is a self-extracting
% \docstrip\ archive. The files are extracted by running the
% \xfile{.dtx} through \plainTeX:
% \begin{quote}
%   \verb|tex kvoptions.dtx|
% \end{quote}
%
% \paragraph{TDS.} Now the different files must be moved into
% the different directories in your installation TDS tree
% (also known as \xfile{texmf} tree):
% \begin{quote}
% \def\t{^^A
% \begin{tabular}{@{}>{\ttfamily}l@{ $\rightarrow$ }>{\ttfamily}l@{}}
%   kvoptions.sty & tex/latex/oberdiek/kvoptions.sty\\
%   kvoptions-patch.sty & tex/latex/oberdiek/kvoptions-patch.sty\\
%   kvoptions.pdf & doc/latex/oberdiek/kvoptions.pdf\\
%   example-mycolorsetup.sty & doc/latex/oberdiek/example-mycolorsetup.sty\\
%   test/kvoptions-test1.tex & doc/latex/oberdiek/test/kvoptions-test1.tex\\
%   test/kvoptions-test2.tex & doc/latex/oberdiek/test/kvoptions-test2.tex\\
%   test/kvoptions-test3.tex & doc/latex/oberdiek/test/kvoptions-test3.tex\\
%   test/kvoptions-test4.tex & doc/latex/oberdiek/test/kvoptions-test4.tex\\
%   test/kvoptions-test4.sty & doc/latex/oberdiek/test/kvoptions-test4.sty\\
%   kvoptions.dtx & source/latex/oberdiek/kvoptions.dtx\\
% \end{tabular}^^A
% }^^A
% \sbox0{\t}^^A
% \ifdim\wd0>\linewidth
%   \begingroup
%     \advance\linewidth by\leftmargin
%     \advance\linewidth by\rightmargin
%   \edef\x{\endgroup
%     \def\noexpand\lw{\the\linewidth}^^A
%   }\x
%   \def\lwbox{^^A
%     \leavevmode
%     \hbox to \linewidth{^^A
%       \kern-\leftmargin\relax
%       \hss
%       \usebox0
%       \hss
%       \kern-\rightmargin\relax
%     }^^A
%   }^^A
%   \ifdim\wd0>\lw
%     \sbox0{\small\t}^^A
%     \ifdim\wd0>\linewidth
%       \ifdim\wd0>\lw
%         \sbox0{\footnotesize\t}^^A
%         \ifdim\wd0>\linewidth
%           \ifdim\wd0>\lw
%             \sbox0{\scriptsize\t}^^A
%             \ifdim\wd0>\linewidth
%               \ifdim\wd0>\lw
%                 \sbox0{\tiny\t}^^A
%                 \ifdim\wd0>\linewidth
%                   \lwbox
%                 \else
%                   \usebox0
%                 \fi
%               \else
%                 \lwbox
%               \fi
%             \else
%               \usebox0
%             \fi
%           \else
%             \lwbox
%           \fi
%         \else
%           \usebox0
%         \fi
%       \else
%         \lwbox
%       \fi
%     \else
%       \usebox0
%     \fi
%   \else
%     \lwbox
%   \fi
% \else
%   \usebox0
% \fi
% \end{quote}
% If you have a \xfile{docstrip.cfg} that configures and enables \docstrip's
% TDS installing feature, then some files can already be in the right
% place, see the documentation of \docstrip.
%
% \subsection{Refresh file name databases}
%
% If your \TeX~distribution
% (\TeX\,Live, \mikTeX, \dots) relies on file name databases, you must refresh
% these. For example, \TeX\,Live\ users run \verb|texhash| or
% \verb|mktexlsr|.
%
% \subsection{Some details for the interested}
%
% \paragraph{Unpacking with \LaTeX.}
% The \xfile{.dtx} chooses its action depending on the format:
% \begin{description}
% \item[\plainTeX:] Run \docstrip\ and extract the files.
% \item[\LaTeX:] Generate the documentation.
% \end{description}
% If you insist on using \LaTeX\ for \docstrip\ (really,
% \docstrip\ does not need \LaTeX), then inform the autodetect routine
% about your intention:
% \begin{quote}
%   \verb|latex \let\install=y% \iffalse meta-comment
%
% File: kvoptions.dtx
% Version: 2016/05/16 v3.12
% Info: Key value format for package options
%
% Copyright (C)
%    2004, 2006, 2007, 2009-2011 Heiko Oberdiek
%    2016-2019 Oberdiek Package Support Group
%    https://github.com/ho-tex/oberdiek/issues
%
% This work may be distributed and/or modified under the
% conditions of the LaTeX Project Public License, either
% version 1.3c of this license or (at your option) any later
% version. This version of this license is in
%    https://www.latex-project.org/lppl/lppl-1-3c.txt
% and the latest version of this license is in
%    https://www.latex-project.org/lppl.txt
% and version 1.3 or later is part of all distributions of
% LaTeX version 2005/12/01 or later.
%
% This work has the LPPL maintenance status "maintained".
%
% The Current Maintainers of this work are
% Heiko Oberdiek and the Oberdiek Package Support Group
% https://github.com/ho-tex/oberdiek/issues
%
% This work consists of the main source file kvoptions.dtx
% and the derived files
%    kvoptions.sty, kvoptions.pdf, kvoptions.ins, kvoptions.drv,
%    kvoptions-patch.sty, example-mycolorsetup.sty,
%    kvoptions-test1.tex, kvoptions-test2.tex,
%    kvoptions-test3.tex, kvoptions-test4.tex,
%    kvoptions-test4.sty.
%
% Distribution:
%    CTAN:macros/latex/contrib/oberdiek/kvoptions.dtx
%    CTAN:macros/latex/contrib/oberdiek/kvoptions.pdf
%
% Unpacking:
%    (a) If kvoptions.ins is present:
%           tex kvoptions.ins
%    (b) Without kvoptions.ins:
%           tex kvoptions.dtx
%    (c) If you insist on using LaTeX
%           latex \let\install=y\input{kvoptions.dtx}
%        (quote the arguments according to the demands of your shell)
%
% Documentation:
%    (a) If kvoptions.drv is present:
%           latex kvoptions.drv
%    (b) Without kvoptions.drv:
%           latex kvoptions.dtx; ...
%    The class ltxdoc loads the configuration file ltxdoc.cfg
%    if available. Here you can specify further options, e.g.
%    use A4 as paper format:
%       \PassOptionsToClass{a4paper}{article}
%
%    Programm calls to get the documentation (example):
%       pdflatex kvoptions.dtx
%       makeindex -s gind.ist kvoptions.idx
%       pdflatex kvoptions.dtx
%       makeindex -s gind.ist kvoptions.idx
%       pdflatex kvoptions.dtx
%
% Installation:
%    TDS:tex/latex/oberdiek/kvoptions.sty
%    TDS:tex/latex/oberdiek/kvoptions-patch.sty
%    TDS:doc/latex/oberdiek/kvoptions.pdf
%    TDS:doc/latex/oberdiek/example-mycolorsetup.sty
%    TDS:doc/latex/oberdiek/test/kvoptions-test1.tex
%    TDS:doc/latex/oberdiek/test/kvoptions-test2.tex
%    TDS:doc/latex/oberdiek/test/kvoptions-test3.tex
%    TDS:doc/latex/oberdiek/test/kvoptions-test4.tex
%    TDS:doc/latex/oberdiek/test/kvoptions-test4.sty
%    TDS:source/latex/oberdiek/kvoptions.dtx
%
%<*ignore>
\begingroup
  \catcode123=1 %
  \catcode125=2 %
  \def\x{LaTeX2e}%
\expandafter\endgroup
\ifcase 0\ifx\install y1\fi\expandafter
         \ifx\csname processbatchFile\endcsname\relax\else1\fi
         \ifx\fmtname\x\else 1\fi\relax
\else\csname fi\endcsname
%</ignore>
%<*install>
\input docstrip.tex
\Msg{************************************************************************}
\Msg{* Installation}
\Msg{* Package: kvoptions 2016/05/16 v3.12 Key value format for package options (HO)}
\Msg{************************************************************************}

\keepsilent
\askforoverwritefalse

\let\MetaPrefix\relax
\preamble

This is a generated file.

Project: kvoptions
Version: 2016/05/16 v3.12

Copyright (C)
   2004, 2006, 2007, 2009-2011 Heiko Oberdiek
   2016-2019 Oberdiek Package Support Group

This work may be distributed and/or modified under the
conditions of the LaTeX Project Public License, either
version 1.3c of this license or (at your option) any later
version. This version of this license is in
   https://www.latex-project.org/lppl/lppl-1-3c.txt
and the latest version of this license is in
   https://www.latex-project.org/lppl.txt
and version 1.3 or later is part of all distributions of
LaTeX version 2005/12/01 or later.

This work has the LPPL maintenance status "maintained".

The Current Maintainers of this work are
Heiko Oberdiek and the Oberdiek Package Support Group
https://github.com/ho-tex/oberdiek/issues


This work consists of the main source file kvoptions.dtx
and the derived files
   kvoptions.sty, kvoptions.pdf, kvoptions.ins, kvoptions.drv,
   kvoptions-patch.sty, example-mycolorsetup.sty,
   kvoptions-test1.tex, kvoptions-test2.tex,
   kvoptions-test3.tex, kvoptions-test4.tex,
   kvoptions-test4.sty.

\endpreamble
\let\MetaPrefix\DoubleperCent

\generate{%
  \file{kvoptions.ins}{\from{kvoptions.dtx}{install}}%
  \file{kvoptions.drv}{\from{kvoptions.dtx}{driver}}%
  \usedir{tex/latex/oberdiek}%
  \file{kvoptions.sty}{\from{kvoptions.dtx}{package}}%
  \file{kvoptions-patch.sty}{\from{kvoptions.dtx}{patch}}%
  \usedir{doc/latex/oberdiek}%
  \file{example-mycolorsetup.sty}{\from{kvoptions.dtx}{example}}%
%  \usedir{doc/latex/oberdiek/test}%
%  \file{kvoptions-test1.tex}{\from{kvoptions.dtx}{test1}}%
%  \file{kvoptions-test2.tex}{\from{kvoptions.dtx}{test2}}%
%  \file{kvoptions-test3.tex}{\from{kvoptions.dtx}{test3}}%
%  \file{kvoptions-test4.tex}{\from{kvoptions.dtx}{test4}}%
%  \file{kvoptions-test4.sty}{\from{kvoptions.dtx}{test4pkg}}%
  \nopreamble
  \nopostamble
%  \usedir{source/latex/oberdiek/catalogue}%
%  \file{kvoptions.xml}{\from{kvoptions.dtx}{catalogue}}%
}

\catcode32=13\relax% active space
\let =\space%
\Msg{************************************************************************}
\Msg{*}
\Msg{* To finish the installation you have to move the following}
\Msg{* files into a directory searched by TeX:}
\Msg{*}
\Msg{*     kvoptions.sty, kvoptions-patch.sty}
\Msg{*}
\Msg{* To produce the documentation run the file `kvoptions.drv'}
\Msg{* through LaTeX.}
\Msg{*}
\Msg{* Happy TeXing!}
\Msg{*}
\Msg{************************************************************************}

\endbatchfile
%</install>
%<*ignore>
\fi
%</ignore>
%<*driver>
\NeedsTeXFormat{LaTeX2e}
\ProvidesFile{kvoptions.drv}%
  [2016/05/16 v3.12 Key value format for package options (HO)]%
\documentclass{ltxdoc}
\usepackage{holtxdoc}[2011/11/22]
\begin{document}
  \DocInput{kvoptions.dtx}%
\end{document}
%</driver>
% \fi
%
%
% \CharacterTable
%  {Upper-case    \A\B\C\D\E\F\G\H\I\J\K\L\M\N\O\P\Q\R\S\T\U\V\W\X\Y\Z
%   Lower-case    \a\b\c\d\e\f\g\h\i\j\k\l\m\n\o\p\q\r\s\t\u\v\w\x\y\z
%   Digits        \0\1\2\3\4\5\6\7\8\9
%   Exclamation   \!     Double quote  \"     Hash (number) \#
%   Dollar        \$     Percent       \%     Ampersand     \&
%   Acute accent  \'     Left paren    \(     Right paren   \)
%   Asterisk      \*     Plus          \+     Comma         \,
%   Minus         \-     Point         \.     Solidus       \/
%   Colon         \:     Semicolon     \;     Less than     \<
%   Equals        \=     Greater than  \>     Question mark \?
%   Commercial at \@     Left bracket  \[     Backslash     \\
%   Right bracket \]     Circumflex    \^     Underscore    \_
%   Grave accent  \`     Left brace    \{     Vertical bar  \|
%   Right brace   \}     Tilde         \~}
%
% \GetFileInfo{kvoptions.drv}
%
% \title{The \xpackage{kvoptions} package}
% \date{2016/05/16 v3.12}
% \author{Heiko Oberdiek\thanks
% {Please report any issues at \url{https://github.com/ho-tex/oberdiek/issues}}}
%
% \maketitle
%
% \begin{abstract}
% This package is intended for package authors who want to
% use options in key value format for their package options.
% \end{abstract}
%
% \tableofcontents
%
% \def\M#1{\texttt{\{}\meta{#1}\texttt{\}}}
% \def\O#1{\texttt{[}\meta{#1}\texttt{]}}
%
% \section{Introduction}
%
% First I want to recommend the very good review article
% ``A guide to key-value methods'' by Joseph Wright \cite{tb94wright}.
% It introduces the different key-value packages and compares them.
%
% \subsection{The beginning}
%
% This package \xpackage{kvoptions} addresses class or package writers
% that want to allow their users to specify options
% as key value pairs, e.g.:
% \begin{quote}
% |\documentclass[verbose=false,name=me]{myclass}|\\
% |\usepackage[format=print]{mylayout}|
% \end{quote}
% Prominent example is package \xpackage{hyperref},
% probably the first package that offers this service.
% It's \cs{ProcessOptionsWithKV} is often copied und
% used in other packages, e.g. package \xpackage{helvet}
% that uses this interface for its option \xoption{scaled}.
%
% However copying code is not the most modern software
% development technique. And \xpackage{hyperref}'s
% code for \cs{ProcessOptionsWithKV} was changed
% to fix bugs. The version used in other packages
% depends on the time of copying and the awareness
% of \xpackage{hyperref}'s changes.
% Now the code is sourced out into this package and
% available for other package or class writers.
%
% \subsection{Overview}
%
% Package \xpackage{kvoptions} connects package
% \xpackage{\emph{k}ey\emph{v}al}
% with \LaTeX's package and class \emph{options}:
% \begin{center}
% \def\T#1{\bfseries #1}
% \renewcommand{\arraystretch}{1.5}
% \begin{tabular}{l|l|l}
%   \T{Package \xpackage{keyval}}
%   & \T{Package \xpackage{kvoptions}}
%   & \T{\LaTeX\ kernel}
%   \\
% \hline
%     \cs{define@key}
%   &
%     \renewcommand*{\arraystretch}{1}^^A
%     \begin{tabular}[t]{@{}l@{}}
%       \cs{DeclareVoidOption}\\
%       \cs{DeclareStringOption}\\
%       \cs{DeclareBoolOption}\\
%       \cs{DeclareComplementaryOption}\\
%       \cs{DisableKeyvalOption}
%     \end{tabular}
%   & \cs{DeclareOption}
%   \\
% \hline
%   & \cs{DeclareDefaultOption}
%   & \cs{DeclareOption*}
%   \\
% \hline
%   & \cs{ProcessKeyvalOptions}
%   & \cs{ProcessOptions*}
%   \\
% \hline
%   & Option \xoption{patch}
%   &
%     \renewcommand*{\arraystretch}{1}^^A
%     \begin{tabular}[t]{@{}l@{}}
%       Class/package\\
%       option system
%     \end{tabular}
%   \\
% \hline
%   & \cs{SetupKeyvalOptions} &
%   \\
% \end{tabular}
% \end{center}
%
% \hypersetup{bookmarksopenlevel=3}
% \section{Usage}
%
% \subsection{Process options}
%
% \subsubsection{\cs{ProcessKeyvalOptions}}
%
% \begin{declcs}{ProcessKeyvalOptions}\,\M{family}\\
%   \cs{ProcessKeyvalOptions} \*
% \end{declcs}
% This command evaluates the global or local options of
% the package that are defined with \xpackage{keyval}'s
% interface within the family \meta{family}. It acts the
% same way as \LaTeX's \cs{ProcessOptions*}. In a package
% unknown global options are ignored, in a class they
% are added to the unknown option list. The known global
% options and all local options are passed to \xpackage{keyval}'s
% \cs{setkeys} command for executing the options.
% Unknown options are reported to the user by an error.
%
% If the family name happens to be the same as the name
% of the package or class where \cs{ProcessKeyvalOptions}
% is used or the family name has previously been setup by
% \cs{SetupKeyvalOptions}, then \cs{ProcessKeyvalOptions}
% knows the family name already and you can use the star form
% without mandatory argument.
%
% \subsubsection{\cs{ProcessLocalKeyvalOptions}}
%
% \begin{declcs}{ProcessLocalKeyvalOptions}\,\M{family}\\
%   \cs{ProcessLocalKeyvalOptions} \*
% \end{declcs}
% This macro has the same syntax and works similar as
% \cs{ProcessKeyvalOptions}. However it ignores
% global options and only processes the local package options.
% Therefore it only can be used inside a package.
% An error is thrown, if it is used inside a class.
%
% Neither of the following macros are necessary for
% \cs{ProcessKeyvalOptions{}}. They just help the package/class
% author in common tasks.
%
% \subsubsection{\cs{SetupKeyvalOptions}}\label{setup}
%
% \begin{declcs}{SetupKeyvalOptions}\,\texttt{\{}\\
%   \hspace{1.5em}|family|\,|=|\,\meta{family}|,|\\
%   \hspace{1.5em}|prefix|\,|=|\,\meta{prefix}\\
%   \hspace{1.5em}|setkeys|\,|=|\,\meta{setkeys command}\\
%   \texttt{\}}
% \end{declcs}
% This command allows to configure the default assumptions
% that are based on the current package or class name.
% \LaTeX\ remembers this name in \cs{@currname}. The syntax
% description of the default looks a little weird, therefor
% an example is given for a package or class named |foobar|.
% \begin{quote}
% \begin{tabular}{@{}l|ll|l@{}}
%   Key & Default & (example) & Used by\\
% \hline
%   \rule{0pt}{1.2\ht\csname @arstrutbox\endcsname}^^A
%   |family| & \meta{\cs{@currname}} & (|foobar|)
%       & \cs{ProcessKeyvalOptions*} \\
%     &&& \cs{DeclareBoolOption} \\
%     &&& \cs{DeclareStringOption} \\
% \hline
%   \rule{0pt}{1.2\ht\csname @arstrutbox\endcsname}^^A
%   |prefix| & \meta{\cs{@currname}}|@| & (|foobar@|)
%       & \cs{DeclareBoolOption} \\
%     &&& \cs{DeclareStringOption} \\
%     &&& \cs{DeclareVoidOption} \\
% \hline
%   \rule{0pt}{1.2\ht\csname @arstrutbox\endcsname}^^A
%   |setkeys| & \cs{setkeys} & (\cs{kvsetkeys})
%       & \cs{ProcessKeyvalOptions} \\
%     &&& \cs{ProcessLocalKeyvalOptions}
% \end{tabular}
% \end{quote}
% Key |setkeys| was added in version 3.9. The original
% \cs{setkeys} of package \xpackage{keyval} is not reentrant.
% If an option is processed by this \cs{setkeys}, then
% the option should not call \cs{setkeys} again with a
% different family. Otherwise the next options of the
% first \cs{setkeys} call are processed with the wrong
% family. With key |setkeys| the macro \cs{kvsetkeys}
% can be set that does not have the problem of the
% original \cs{setkeys} of package {keyval}.
%
% Probably \cs{setkeys} of package \xpackage{xkeyval}
% is safe in this respect. But I haven't made a full
% analysis. At least it does not have the problem
% of the original \cs{setkeys}.
%
% \subsection{Option declarations}
%
% The options for \cs{ProcessKeyvalOptions} are defined
% by \xpackage{keyval}'s \cs{define@key}.
% Common purposes of such keys are boolean switches, they
% enable or disable something. Or they store a name or some kind of
% string in a macro. The following commands help the user. He
% declares what he wants and \xpackage{kvoptions} take care of
% the key definition, resource allocation and initialization.
%
% In order to avoid name clashes of macro names, internal
% commands are prefixed. Both the prefix and the family
% name for the defined keys can be configured by
% \cs{SetupKeyvalOptions}.
%
% \subsubsection{\cs{DeclareStringOption}}
%
% \begin{declcs}{DeclareStringOption}\,\O{init}\,\M{key}\,\O{default}
% \end{declcs}
% A macro is created that remembers the value of the key \meta{key}.
% The name of the macro consists of the option name \meta{key} that
% is prefixed by the prefix (see \ref{setup}).
% The initial contents of the
% macro can be given by the first optional argument \meta{init}.
% The default is empty.
%
% The the option \meta{key} is defined. The option code
% just stores its value in the macro. If the optional argument
% at the end of \cs{DeclareStringOption} is given, then option
% \meta{key} is defined with the default \meta{default}.
%
% Example for a package with the following two lines:
% \begin{quote}
% |\ProvidesPackage{foobar}|\\
% |\DeclareStringOption[me]{name}|
% \end{quote}
% Then \cs{DeclareStringOption} defines the macro with content |me|,
% note \LaTeX\ complains if the name of the macro already exists:
% \begin{quote}
% |\newcommand*{\foobar@name}{me}|
% \end{quote}
% The option definition is similar to:
% \begin{quote}
% |\define@key{foobar}{name}{%|\\
% |  \renewcommand*{\foobar@name}{#1}%|\\
% |}|
% \end{quote}
%
% \subsubsection{\cs{DeclareBoolOption}}
%
% \begin{declcs}{DeclareBoolOption}\,\O{init}\,\M{key}
% \end{declcs}
% A boolean switch is generated, initialized by value \meta{init}
% and the corresponding key \meta{key} is defined. If the
% initialization value is not given, |false| is used as default.
%
% The internal actions of \cs{DeclareBoolOption} are shown below.
% The example is given for a package author who has the following
% two lines in his package/class:
% \begin{quote}
% |\ProvidesPackage{foobar}|\\
% |\DeclareBoolOption{verbose}|
% \end{quote}
% First a new switch is created:
% \begin{quote}
%   |\newif\iffoobar@verbose|
% \end{quote}
% and initialized:
% \begin{quote}
%   |\foobar@verbosefalse|
% \end{quote}
% Finally the key is defined:
% \begin{quote}
%   |\define@key{foobar}{verbose}[true]{|\dots|}|
% \end{quote}
% The option code configures the boolean option in the following way:
% If the author specifies |true| or |false| then the switch is
% turned on or off respectivly. Also the option can be given without
% explicit value. Then the switch is enabled.
% Other values are reported as errors.
%
% Now the switch is ready to use in the package/class, e.g.:
% \begin{quote}
%   |\iffoobar@verbose|\\
%   \qquad\textit{\% print verbose message}\\
%   |\else|\\
%   \qquad\textit{\% be quiet}\\
%   |\fi|
% \end{quote}
% Users of package \cs{ifthen} can use the switch as
% boolean:
% \begin{quote}
%   |\boolean{foobar@verbose}|
% \end{quote}
%
% \subsubsection{\cs{DeclareComplementaryOption}}
%
% \begin{declcs}{DeclareComplementaryOption}\,\M{key}\,\M{parent}
% \end{declcs}
% Sometimes contrasting names are used to characterize the
% two states of a boolean switch, for example \xoption{draft}
% vs.\@ \xoption{final}. Both options behave like boolean
% options but they do not need two different switches, they
% should share one. \cs{DeclareComplementaryOption} allows this.
% The option \meta{key} shares the switch of option \meta{parent}.
% Example:
% \begin{quote}
%   |\DeclareBoolOption{draft}|\\
%   |\DeclareComplementaryOption{final}{draft}|
% \end{quote}
% Then |final| sets the switch of \xoption{draft} to |false|, and
% |final=false| enables the \xoption{draft} switch.
%
% \subsubsection{\cs{DeclareVoidOption}}
%
% \begin{declcs}{DeclareVoidOption}\,\M{key}\,\M{code}
% \end{declcs}
% \cs{ProcessKeyvalOptions} can be extended to recognize
% options that are declared in traditional way by \cs{DeclareOption}.
% But in case of the error that the user specifies a value, then
% this option would not recognized as key value option because
% of \cs{DeclareOption} and not detected as traditional option
% because of the value part. The user would get an unknown option
% error, difficult to understand.
%
% \cs{DeclareVoidOption} solves
% this problem. It defines the option \meta{key} as key value option.
% If the user specifies a value, a warning is given and the
% value is ignored.
%
% The code part \meta{code} is stored in a macro.
% The name of the macro consists of the option name \meta{key} that
% is prefixed by the prefix (see \ref{setup}).
% If the option is set, the macro will be executed. During the
% execution \cs{CurrentOption} is available with the current key name.
%
% \subsubsection{\cs{DeclareDefaultOption}}
%
% \begin{declcs}{DeclareDefaultOption}\,\M{code}
% \end{declcs}
% \begin{sloppypar}
% This command does not define a specific key, it is the equivalent
% to \LaTeX's \cs{DeclareOption*}. It allows the specification
% of a default action \meta{code} that is invoked if an unknown option
% is found.
% While \meta{code} is called,
% macro \cs{CurrentOption} contains the current option string.
% In addition \cs{CurrentOptionValue} contains the value part
% if the option string is parsable as key value pair, otherwise
% it is \cs{relax}. \cs{CurrentOptionKey} contains the key of
% the key value pair, or the whole option string, if it misses
% the equal sign.
% \end{sloppypar}
%
% Inside packages typical default actions are to pass unknown
% options to another package. Or an error message can be thrown
% by \cs{@unknownoptionerror}. This is the original error
% message that \LaTeX\ gives for unkown package options.
% This error message is easier to understand for the user
% as the error message from package \xpackage{keyval} that
% is given otherwise.
%
% A Class ignores unknown options and puts them on the
% unused option list. Let \LaTeX\ do the job and just
% call \cs{OptionNotUsed}. Or the options can be passed to
% another class that is later loaded.
%
% \subsubsection{Local options}
%
% \begin{declcs}{DeclareLocalOption} \M{option}\\
%   \cs{DeclareLocalOptions} \M{option list}
% \end{declcs}
% Both macros mark package options as local options. That means
% that they are ignored by \cs{ProcessKeyvalOptions} if they are given
% as global options. \cs{DeclareLocalOptions} takes one option,
% \cs{DeclareLocalOptions} expects a comma separated list of options.
%
% \subsubsection{Dynamic options}
%
% Options of \LaTeX's package/class system are cleared
% in \cs{ProcessOptions}. They modify the static model
% of a package. For example, depending on option \xoption{bookmarks}
% package \xpackage{hyperref} loads differently.
%
% Options, however,
% defined by \xpackage{keyval}'s \cs{define@key} remain
% defined, if the options are processed by \cs{setkeys}.
% Therefore these options can also be used to model
% the dynamic behaviour of a package. For example, in
% \xpackage{hyperref} the link colors can be changed
% everywhere until the end in |\end{document}|.
%
% However package \xpackage{color} that adds color support
% is necessary and it
% cannot be loaded after |\begin{document}|. Option \xoption{colorlinks}
% that loads \xpackage{color} should be active until |\begin{document}|
% and die in some way if it is too late for loading packages.
% With \cs{DisableKeyvalOption} the package/class author can
% specify and configure the death of an option and controls
% the life period of the option.
%
% \subsubsection{\cs{DisableKeyvalOption}}
%
% \begin{declcs}{DisableKeyvalOption}\,\O{options}\,\M{family}\,\M{key}
%   \\[0.5ex]
%   \meta{options}:\\
%   \hspace{1.5em}^^A
%   \begin{tabular}[t]{@{}l@{ \texttt{=} }l@{\qquad}>{default: }l@{}}
%     |action| & |undef|, |warning|, |error|, or |ignore|
%     & |undef|
%     \\
%     \multicolumn{1}{@{}l}{|global| or |local|} & & |global|
%     \\
%     |package| or |class| & \meta{name}
%     \\
%   \end{tabular}^^A
% \end{declcs}
% \cs{DisableKeyvalOption} can be called to mark the end when
% the option \meta{key} is no longer useful. The behaviour of
% an option after its death can be configured by \xoption{action}:
% \begin{description}
% \item[\texttt{undef}:] The option will be undefined,
%   If it is called, \cs{setkeys} reports an error because
%   of unknown key.
% \item[\texttt{error} or \texttt{warning}:] Use of the
%  option will cause an error or warning message.
%  Also these actions require that exclusivly either the
%  package or class name is given in options \xoption{package}
%  or \xoption{class}.
% \item[\texttt{ignore}:] The use of the option will silently
%  be ignored.
% \end{description}
% The option's death can be limited to the end of the current
% group, if option \xoption{local} is given. Default is
% \xoption{global}.
%
% The package/class author can wish the end of the option
% already during the package loading, then he will have
% static behaviour. In case of dynamic options \cs{DisableKeyvalOption}
% can be executed everywhere, also outside the package. Therefore
% the family name and the package/class name is usually unknown for
% \cs{DisableKeyvalOption}. Therefore the argument for the family
% name is mandatory and for some actions the package/class name
% must be provided.
%
% Usually a macro would configure the option death, Example:
%\begin{quote}
%\begin{verbatim}
%\ProvidesPackage{foobar}
%\DeclareBoolOption{color}
%\DeclareStringOption[red]{emphcolor}
%\ProcessKeyvalOptions*
%
%\newcommand*{\foobar@DisableOption}[2]{%
%  \DisableKeyvalueOption[
%    action={#1},
%    package=foobar
%  ]{foobar}{#2}%
%}
%
%\iffoobar@color
%  \RequirePackage{color}
%  \renewcommand*{\emph}[1]{\textcolor{\foobar@emphcolor}{#1}}
%\else
%  % Option emphcolor is not wrong, if we have color support.
%  % otherwise the option has no effect, but we don't want to
%  % remove it. Therefore action 'ignore' is the best choice:
%  \foobar@DisableOption{ignore}{emphcolor}
%\fi
%% No we don't need the option 'color'.
%\foobar@DisableOption{warning}{color}
%
%% With color support option 'emphcolor' will dynamically
%% change the color of \emph statements.
%\end{verbatim}
%\end{quote}
%
% \subsubsection{\cs{AddToKeyvalOption}}
%
% \begin{declcs}{AddToKeyvalOption}\,\M{family}\,\M{key}\,\M{code}\\
%   \cs{AddToKeyvalOption} \*\,\M{key}\,\M{code}
% \end{declcs}
% The code for an existing key \meta{key} of family \meta{family} is
% extended by code \meta{code}. In the starred form the current
% family setting is used, see \cs{ProcessKeyvalOptions*}.
%
% \subsection{Global vs. local options}
%
% Options that are given for \cs{documentclass} are called
% global options. They are known to the class and all packages.
% A package may make use of a global option and marks it as
% used. The advantage for the user is the freedom to specify
% options both in the \cs{documentclass} or \cs{usepackage}
% commands.
%
% However global options are shared with the class options
% and options of all other packages. Thus there can be the
% same option with different semantics for different
% packages and classes. As example, package \xpackage{bookmark}
% knows option \xoption{open} that specifies whether the bookmarks
% are opened or closed initially. It's values are |true| or |false|.
% Since KOMA-Script version 3.00 the KOMA classes also introduces
% option \xoption{open} with values |right| and |any| and a
% complete different meaning.
%
% Such conflicts can be resolved by marking all or part of options
% as local by \cs{DeclareLocalOption} or \cs{DeclareLocalOptions}.
% Then the packages ignores global occurences of these options.
% Package \xpackage{kvoptions} provides two methods:
% \begin{itemize}
% \item \cs{ProcessLocalKeyvalOptions} automatically uses all options
%   as local options. It ignores all global options.
% \item \cs{DeclareLocalOption} or \cs{DeclareLocalOptions} marks
%   options as local options. \cs{ProcessKeyvalOptions} will then
%   ignore global occurences for these local options.
% \end{itemize}
% Since version 1.5 package \xpackage{bookmark} uses the latter
% method. It checks global and local option places for driver options
% and limits all other options as local options. Thus the class
% option \xoption{open} of KOMA-Script is not misread as
% option for package \xpackage{bookmark}.
%
% \subsection{Summary of internal macros}
%
% The \cs{Declare}\texttt{\dots Option} commands define
% macros, additionally to the macros generated by the key definition.
% These macros can be used by the package/class author.
% The name of the macros starts with the prefix \meta{prefix}
% that can be configured by \cs{SetupKeyvalOptions}.
% \begin{center}
% \def\prefix{\cs{}\meta{prefix}\meta{key}}^^A
% \def\Hline{^^A
%   \hline
%   \rule{0pt}{1.2\ht\csname @arstrutbox\endcsname}^^A
% }^^A
% \begin{tabular}{@{}l|>{\ttfamily}l|l@{}}
%   Declare \meta{key} & \textrm{Defined macro} & Description\\
% \Hline
%   \cs{DeclareStringOption}
%    & \prefix & holds the string\\
% \Hline
%   \cs{DeclareBoolOption}
%     & \cs{if}\meta{prefix}\meta{key}& boolean switch\\
%   &\prefix false& disable switch\\
%   &\prefix true& enable switch\\
% \Hline
%   \cs{DeclareComplementaryOption}
%     & \prefix false& enable parent switch\\
%     & \prefix true& disable parent switch\\
% \Hline
%   \cs{DeclareVoidOption}
%   & \prefix & holds the action
% \end{tabular}
% \end{center}
%
% \subsection{\plainTeX}
%
% Package \xpackage{keyval} is also usable in \plainTeX\ with
% the help of file \xfile{miniltx.tex}. Some features of this
% package \xpackage{kvoptions} might also be useful for \plainTeX.
% If \LaTeX\ is not found, \cs{ProcessKeyvalOptions} and option
% \xoption{patch} are disabled. Before using the option declaration
% commands \cs{Declare...Option}, \cs{SetupKeyvalOptions} must be
% used.
%
% \hypersetup{bookmarksopenlevel=1}
%
% \section{Example}
%
% The following example defined a package that serves some
% private color management. A boolean option \xoption{print} enables
% print mode without colors. An option \xoption{emph} redefines
% \cs{emph} to print in the given color. And the driver
% can be specified by option \xoption{driver}.
%
%    \begin{macrocode}
%<*example>
    % Package identification
    % -----------------------
\NeedsTeXFormat{LaTeX2e}
\ProvidesPackage{example-mycolorsetup}[2016/05/16 Managing my colors]

\RequirePackage{ifpdf}
\RequirePackage{kvoptions}

    % Option declarations
    % -------------------

\SetupKeyvalOptions{
  family=MCS,
  prefix=MCS@
}
    % Use a shorter family name and prefix

    % Option print
\DeclareBoolOption{print}
    % is the same as
    % \DeclareBoolOption[false]{print}

    % Option driver
\ifpdf
  \DeclareStringOption[pdftex]{driver}
\else
  \DeclareStringOption[dvips]{driver}
\fi

    % Alternative interface for driver options
\DeclareVoidOption{dvips}{\SetupDriver}
\DeclareVoidOption{dvipdfm}{\SetupDriver}
\DeclareVoidOption{pdftex}{\SetupDriver}
    % In \SetupDriver we take the current option \CurrentOption
    % and pass it to the driver option.
    % The \expandafter commands expand \CurrentOption at the
    % time, when \SetupDriver is executed and \CurrentOption
    % has the correct meaning.
\newcommand*{\SetupDriver}{%
  \expandafter\@SetupDriver\expandafter{\CurrentOption}%
}
\newcommand*{\@SetupDriver}[1]{%
  \setkeys{MCS}{driver={#1}}%
}

    % Option emph
    % An empty value means, we want to have no color for \emph.
    % If the user specifies option emph without value, the red is used.
\DeclareStringOption{emph}[red]
    % is the same as
    % \DeclareStringOption[]{emph}[red]

    % Default option rule
\DeclareDefaultOption{%
  \ifx\CurrentOptionValue\relax
    \PackageWarningNoLine{\@currname}{%
      Unknown option `\CurrentOption'\MessageBreak
      is passed to package `color'%
    }%
    % Pass the option to package color.
    % Again it is better to expand \CurrentOption.
    \expandafter\PassOptionsToPackage
    \expandafter{\CurrentOption}{color}%
  \else
    % Package color does not take options with values.
    % We provide the standard LaTeX error.
    \@unknownoptionerror
  \fi
}

    % Process options
    % ---------------
\ProcessKeyvalOptions*

    % Implementation depending on option values
    % -----------------------------------------
    % Code for print mode
\ifMCS@print
  \PassOptionsToPackage{monochrome}{color}
    % tells package color to use black and white
\fi

\RequirePackage[\MCS@driver]{color}
    % load package color with the correct driver

    % \emph setup
\ifx\MCS@emph\@empty
    % \@empty is a predefined macro with empty contents.
    % the option value of option emph is empty, thus
    % we do not want a redefinition of \emph.
\else
  \renewcommand*{\emph}[1]{%
    \textcolor{\MCS@emph}{#1}%
  }
\fi
%</example>
%    \end{macrocode}
%
% \section{Package options}
%
% The package \xpackage{kvoptions} knows two package options
% \xoption{patch} and \xoption{debugshow}.
% The options of package \xpackage{kvoptions} are intended
% for authors, not for package/class writers. Inside a package
% it is too late for option \xoption{patch} and \xoption{debugshow}
% enables some messages that are perhaps useful for the debugging
% phase. Also \LaTeX\ is unhappy if a package is loaded
% later again with options that are previously not given.
% Thus package and class authors, stay with |\RequirePackage{kvoptions}|
% without options.
%
% Option \xoption{patch} loads package \xpackage{kvoptions-patch}.
%
% \subsection{Package \xpackage{kvoptions-patch}}
%
% \LaTeX's system of package/class options has some severe limitations
% that especially affects the value part if options are used as
% pair of key and value.
% \begin{itemize}
% \item Spaces are removed, regardless where:
%   \begin{quote}
%     |\documentclass[box=0 0 400 600]{article}|
%   \end{quote}
%   Now each package will see |box=00400600| as global option.
% \item In the previous case also braces would not help:
%   \begin{quote}
%     |\documentclass[box={0 0 400 600}]{article}|
%   \end{quote}
%   The result is an error message:
%   \begin{quote}
%     |! LaTeX Error: Missing \begin{document}.|
%   \end{quote}
%   As local option, however, it works if the package
%   knows about key value options (By using this package, for example).
% \item The requirements on robustness are extremly high.
%   \LaTeX\ expands the option. All that will not work as environment
%   name will break also as option. Even a \cs{relax} will generate
%   an error message:
%   \begin{quote}
%     |! Missing \endcsname inserted.|
%   \end{quote}
%   Of course, \LaTeX\ does not use its protecting mechanisms.
%   On contrary \cs{protect} itself will cause errors.
% \item The options are expanded. But perhaps the package will
%   do that, because it has to setup some things before?
%   Example \xpackage{hyperref}:
%   \begin{quote}
%     |\usepackage[pdfauthor=M\"uller]{hyperref}|
%   \end{quote}
%   Package \xpackage{hyperref} does not see |M\"uller| but
%   its expansion and it does not like it, you get many warnings
%   \begin{quote}
%     |Token not allowed in a PDFDocEncoded string|
%   \end{quote}
%   And the title becomes: |Mu127uller|.
%   Therefore such options must usually be given after package
%   \xpackage{hyperref}
%   is loaded:
%   \begin{quote}
%     |\usepackage{hyperref}|\\
%     |\hypersetup{pdfauthor=Fran\c coise M\"uller}|
%   \end{quote}
%   As package option it will even break with |Fran\c coise|
%   because of the cedilla |\c c|, it is not robust enough.
% \end{itemize}
% For users that do not want with this limitations the package
% offers package \xoption{kvoptions-patch}. It patches \LaTeX's option system
% and tries to teach it also to handle options that are given
% as pairs of key and value and to prevent expansion.
% It can already be used at the very beginning, before \cs{documentclass}:
% \begin{quote}
%   |\RequirePackage{kvoptions-patch}|\\
%   |\documentclass[pdfauthor=Fran\c coise M\"uller]{article}|\\
%   |\usepackage{hyperref}|
% \end{quote}
% The latest time is before the package where you want to use
% problematic values:
% \begin{quote}
%   |\usepackage{kvoptions-patch}|\\
%   |\usepackage[Fran\c coise M\"uller]{hyperref}|
% \end{quote}
% Some remarks:
% \begin{itemize}
% \item The patch requires \eTeX, its \cs{unexpanded} feature
%   is much too nice. It is possible to work around using token
%   registers. But the code becomes longer, slower, more difficult
%   to read and maintain. The package without option \xoption{patch}
%   works and will work without \eTeX.
% \item The code for the patch is quite long, there are many test
%   cases. Thus the probability for bugs is probably not too small.
% \item Since 2008/10/18 v3.0 package \xpackage{kvoptions-patch} is
%   available. Before option \xoption{patch} of package \xpackage{kvoptions}
%   must be used instead. I think, the solution as standalone package
%   \xpackage{kvoptions-patch} is cleaner and avoids option clashes.
% \end{itemize}
%
% \subsection{Option \xoption{debugshow}}
%
% The name of this option follows the convention of packages
% \xpackage{multicol}, \xpackage{tabularx}, and \xpackage{tracefnt}.
% Currently it prints the setting of boolean options, declared
% by \cs{DeclareBoolOption} in the \xfile{.log} file, if that
% boolean option is used.
% You can activate the option by
% \begin{itemize}
% \item |\PassOptionsToPackage{debugshow}{kvoptions}|\\
%       Put this somewhere before package \xpackage{kvoptions} is
%       loaded first, e.g. before \cs{documentclass}.
% \item |\RequirePackage[debugshow]{kvoptions}|\\
%       Before \cs{documentclass} even an author has to use
%       \cs{RequirePackage}. \cs{usepackage} only works after
%       \cs{documentclass}.
% \end{itemize}
% The preferred method is \cs{PassOptionsToPackage}, because
% it does not force the package loading and does not disturb,
% if the package is not loaded later at all.
%
% \section{Limitations}
%
% \subsection{Compatibility}
% \subsubsection{Package \xpackage{kvoptions-patch} vs. package \xpackage{xkvltxp}}
%
% Package \xpackage{xkvltxp} from the \xpackage{xkeyval} project
% has the same goal as package \xpackage{kvoptions-patch}
% and to patch \LaTeX's kernel commands in order to get better support
% for key value options.
% Of course they cannot be used both. The user must decide, which
% method he prefers. Package \xpackage{kvoptions-patch} aborts itself,
% if it detects that \xpackage{xkvltxp} is
% already loaded.
%
% However package \xpackage{xkvltxp} and \xpackage{kvoptions}
% can be used together, example:
% \begin{quote}
% |\usepackage{xkvltxp}|\\
% |\usepackage[...]{foobar} % |\texttt{\itshape foobar using kvoptions}
% \end{quote}
% The other way should work, too.
%
% Package \xpackage{kvoptions-patch} tries to catch more situations
% and to be more robust.
% For example, during the comparison of options it normalizes
% them by removing spaces around |=| and the value. Thus the
% following is not reported as option clash:
% \begin{quote}
%\begin{verbatim}
%\RequirePackage{kvoptions-patch}
%\documentclass{article}
%
%\usepackage[scaled=0.7]{helvet}
%\usepackage[scaled = 0.7]{helvet}
%
%\begin{document}
%\end{document}
%\end{verbatim}
% \end{quote}
%
% \subsection{Limitations}
%
% \subsubsection{Option comparisons}
%
% In some situations \LaTeX\ compares option lists, e.g. option clash check,
% \cs{@ifpackagewith}, or \cs{@ifclasswith}. Apart from catcode and
% sanitizing problems of option \xoption{patch}, there is another problem.
% \LaTeX\ does not know about the type and default values of
% options in key value style. Thus an option clash is reported,
% even if the key value has the same meaning:
% \begin{quote}
% |\usepackage[scaled]{helvet} %| \texttt{\textit{default is }.95}\\
% |\usepackage[.95]{helvet}|\\
% |\usepackage[0.95]{helvet}|
% \end{quote}
%
% \subsubsection{Option list parsing with package \xpackage{kvoptions-patch}}
%
% With package \xpackage{kvoptions-patch} the range of possible values
% in key value specifications is much large, for example
% the comma can be used, if enclosed in curly braces.
%
% Other packages, especially the packages that uses their
% own process option code can be surprised to find tokens
% inside options that they do not expect and errors would
% be the consequence. To avoid errors the options, especially
% the unused option list is sanitized. That means the list
% will only contain tokens with catcode 12 (other) and
% perhaps spaces (catcode 10). This allows a safe parsing
% for other packages. But a comma in the value part is
% no longer protected by curly braces because they have lost
% their special meaning. This is the price for compatibility.
%
% Example:
% \begin{quote}
% |\RequirePackage{kvoptions-patch}|\\
% |\documentclass[a={a,b,c},b]{article}|\\
% |\begin{document}|\\
% |\end{document}|
% \end{quote}
% Result:
% \begin{quote}
% |LaTeX Warning: Unused global option(s):|\\
% |    [a={a,c},b].|
% \end{quote}
%
% \StopEventually{
% }
%
% \section{Implementation}
%
% \subsection{Preamble}
%
%    \begin{macrocode}
%<*package>
%    \end{macrocode}
%
% \paragraph{Reload check and  identification.}
%    Reload check, especially if the package is not used with \LaTeX.
%    \begin{macrocode}
\begingroup\catcode61\catcode48\catcode32=10\relax%
  \catcode13=5 % ^^M
  \endlinechar=13 %
  \catcode35=6 % #
  \catcode39=12 % '
  \catcode44=12 % ,
  \catcode45=12 % -
  \catcode46=12 % .
  \catcode58=12 % :
  \catcode64=11 % @
  \catcode123=1 % {
  \catcode125=2 % }
  \expandafter\let\expandafter\x\csname ver@kvoptions.sty\endcsname
  \ifx\x\relax % plain-TeX, first loading
  \else
    \def\empty{}%
    \ifx\x\empty % LaTeX, first loading,
      % variable is initialized, but \ProvidesPackage not yet seen
    \else
      \expandafter\ifx\csname PackageInfo\endcsname\relax
        \def\x#1#2{%
          \immediate\write-1{Package #1 Info: #2.}%
        }%
      \else
        \def\x#1#2{\PackageInfo{#1}{#2, stopped}}%
      \fi
      \x{kvoptions}{The package is already loaded}%
      \aftergroup\endinput
    \fi
  \fi
\endgroup%
%    \end{macrocode}
%    Package identification:
%    \begin{macrocode}
\begingroup\catcode61\catcode48\catcode32=10\relax%
  \catcode13=5 % ^^M
  \endlinechar=13 %
  \catcode35=6 % #
  \catcode39=12 % '
  \catcode40=12 % (
  \catcode41=12 % )
  \catcode44=12 % ,
  \catcode45=12 % -
  \catcode46=12 % .
  \catcode47=12 % /
  \catcode58=12 % :
  \catcode64=11 % @
  \catcode91=12 % [
  \catcode93=12 % ]
  \catcode123=1 % {
  \catcode125=2 % }
  \expandafter\ifx\csname ProvidesPackage\endcsname\relax
    \def\x#1#2#3[#4]{\endgroup
      \immediate\write-1{Package: #3 #4}%
      \xdef#1{#4}%
    }%
  \else
    \def\x#1#2[#3]{\endgroup
      #2[{#3}]%
      \ifx#1\@undefined
        \xdef#1{#3}%
      \fi
      \ifx#1\relax
        \xdef#1{#3}%
      \fi
    }%
  \fi
\expandafter\x\csname ver@kvoptions.sty\endcsname
\ProvidesPackage{kvoptions}%
  [2016/05/16 v3.12 Key value format for package options (HO)]%
%    \end{macrocode}
%
% \paragraph{Catcodes}
%
%    \begin{macrocode}
\begingroup\catcode61\catcode48\catcode32=10\relax%
  \catcode13=5 % ^^M
  \endlinechar=13 %
  \catcode123=1 % {
  \catcode125=2 % }
  \catcode64=11 % @
  \def\x{\endgroup
    \expandafter\edef\csname KVO@AtEnd\endcsname{%
      \endlinechar=\the\endlinechar\relax
      \catcode13=\the\catcode13\relax
      \catcode32=\the\catcode32\relax
      \catcode35=\the\catcode35\relax
      \catcode61=\the\catcode61\relax
      \catcode64=\the\catcode64\relax
      \catcode123=\the\catcode123\relax
      \catcode125=\the\catcode125\relax
    }%
  }%
\x\catcode61\catcode48\catcode32=10\relax%
\catcode13=5 % ^^M
\endlinechar=13 %
\catcode35=6 % #
\catcode64=11 % @
\catcode123=1 % {
\catcode125=2 % }
\def\TMP@EnsureCode#1#2{%
  \edef\KVO@AtEnd{%
    \KVO@AtEnd
    \catcode#1=\the\catcode#1\relax
  }%
  \catcode#1=#2\relax
}
\TMP@EnsureCode{1}{14}% ^^A (comment)
\TMP@EnsureCode{2}{14}% ^^A (comment)
\TMP@EnsureCode{33}{12}% !
\TMP@EnsureCode{39}{12}% '
\TMP@EnsureCode{40}{12}% (
\TMP@EnsureCode{41}{12}% )
\TMP@EnsureCode{42}{12}% *
\TMP@EnsureCode{44}{12}% ,
\TMP@EnsureCode{45}{12}% -
\TMP@EnsureCode{46}{12}% .
\TMP@EnsureCode{47}{12}% /
\TMP@EnsureCode{58}{12}% :
\TMP@EnsureCode{62}{12}% >
\TMP@EnsureCode{91}{12}% [
\TMP@EnsureCode{93}{12}% ]
\TMP@EnsureCode{94}{7}% ^ (superscript)
\TMP@EnsureCode{96}{12}% `
\edef\KVO@AtEnd{\KVO@AtEnd\noexpand\endinput}
%    \end{macrocode}
%
% \paragraph{External resources.}
%    The package extends the support for key value pairs of
%    package \cs{keyval} to package options. Thus the package
%    needs to be loaded anyway. and we use it for
%    \cs{SetupKeyvalOptions}. AFAIK this does not disturb
%    users of \xpackage{xkeyval}.
%    \begin{macrocode}
\@ifundefined{define@key}{%
  \RequirePackage{keyval}\relax
}{}
%    \end{macrocode}
%
%    Macro \cs{DeclareLocalOptions} parses a comma separated key list
%    and uses \cs{comma@parse} of package \xpackage{kvsetkeys},
%    version 1.3.
%    \begin{macrocode}
\RequirePackage{ltxcmds}[2010/12/02]
\RequirePackage{kvsetkeys}[2007/09/29]
%    \end{macrocode}
%
% \paragraph{Provide macros for \plainTeX.}
%    \begin{macrocode}
\@ifundefined{@x@protect}{%
  \def\@x@protect#1\fi#2#3{%
    \fi\protect#1%
  }%
  \let\@typeset@protect\relax
}{}
\@ifundefined{@currname}{%
  \def\@currname{}%
}{}
\@ifundefined{@currext}{%
  \def\@currext{}%
}{}
%    \end{macrocode}
%
% \paragraph{Options}
%    Option \xoption{debugshow} enables additional lines of
%    code that prints information into the \xfile{.log}
%    file.
%    \begin{macrocode}
\DeclareOption{debugshow}{\catcode\@ne=9 }
%    \end{macrocode}
%
%    \begin{macrocode}
\DeclareOption{patch}{%
  \AtEndOfPackage{%
    \RequirePackage{kvoptions-patch}[2016/05/16]%
  }%
}
%    \end{macrocode}
%
%    Optionen auswerten:
%    \begin{macrocode}
\ProcessOptions\relax
%    \end{macrocode}
%
% \subsection{Option declaration macros}
%
% \subsubsection{\cs{SetupKeyvalOptions}}
%
%    The family for the key value pairs can be setup once
%    and is remembered later.
%    The package name seems a reasonable default for the
%    family key, if it is not set by the package author.
%
%    \begin{macro}{\KVO@family}
%    We cannot store the family setting in one macro,
%    because the package should be usable for many
%    other packages, too. Thus we remember the family
%    setting in a macro, whose name contains the
%    package name with extension, a key in \LaTeX's
%    class/package system.
%    \begin{macrocode}
\define@key{KVO}{family}{%
  \expandafter\edef\csname KVO@family@%
      \@currname.\@currext\endcsname{#1}%
}
\def\KVO@family{%
  \@ifundefined{KVO@family@\@currname.\@currext}{%
    \@currname
  }{%
    \csname KVO@family@\@currname.\@currext\endcsname
  }%
}
%    \end{macrocode}
%    \end{macro}
%
%    \begin{macro}{\KVO@prefix}
%    The value settings of options that are declared by \cs{DeclareBoolOption}
%    and \cs{DeclareStringOption} need to be saved in macros.
%    in the first case this is a switch \cs{if}\meta{prefix}\meta{key},
%    in the latter case a macro \cs{}\meta{prefix}\meta{key}.
%    The prefix can be configured, by \xoption{prefix} that is declared
%    here. The default is the package name with |@| appended.
%    \begin{macrocode}
\define@key{KVO}{prefix}{%
  \expandafter\edef\csname KVO@prefix@%
      \@currname.\@currext\endcsname{#1}%
}
\def\KVO@prefix{%
  \ltx@ifundefined{KVO@prefix@\@currname.\@currext}{%
    \@currname @%
  }{%
    \csname KVO@prefix@\@currname.\@currext\endcsname
  }%
}
%    \end{macrocode}
%    \end{macro}
%
%    \begin{macrocode}
\define@key{KVO}{setkeys}{%
  \expandafter\def\csname KVO@setkeys@%
      \@currname.\@currext\endcsname{#1}%
}
%    \end{macrocode}
%    \begin{macro}{\KVO@setkeys}
%    \begin{macrocode}
\def\KVO@setkeys{%
  \ltx@IfUndefined{KVO@setkeys@\@currname.\@currext}{%
    \setkeys
  }{%
    \csname KVO@setkeys@\@currname.\@currext\endcsname
  }%
}
%    \end{macrocode}
%    \end{macro}
%
%    \begin{macro}{\SetupKeyvalOptions}
%    The argument of \cs{SetupKeyvalOptions} expects a key value
%    list, known keys are \xoption{family} and \xoption{prefix}.
%    \begin{macrocode}
\newcommand*{\SetupKeyvalOptions}{%
  \kvsetkeys{KVO}%
}
%    \end{macrocode}
%    \end{macro}
%
% \subsubsection{\cs{DeclareBoolOption}}
%
%    \begin{macro}{\DeclareBoolOption}
%    Usually options of boolean type can be given by the
%    user without value and this means a setting to \emph{true}.
%    We follow this convention here. Also it simplifies the
%    user interface.
%
%    The switch is created and initialized with \emph{false}.
%    The default setting can be overwritten by the optional
%    argument.
%
%    \LaTeX's \cs{newif} does not check for already defined
%    macros, therefore we add this check here to prevent
%    the user from accidently redefining of \TeX's primitives
%    and other macros.
%    \begin{macrocode}
\newcommand*{\DeclareBoolOption}[2][false]{%
  \KVO@ifdefinable{if\KVO@prefix#2}{%
    \KVO@ifdefinable{\KVO@prefix#2true}{%
      \KVO@ifdefinable{\KVO@prefix#2false}{%
        \csname newif\expandafter\endcsname
        \csname if\KVO@prefix#2\endcsname
        \@ifundefined{\KVO@prefix#2#1}{%
          \PackageWarning{kvoptions}{%
            Initialization of option `#2' failed,\MessageBreak
            cannot set boolean option to `#1',\MessageBreak
            use `true' or `false', now using `false'%
          }%
        }{%
          \csname\KVO@prefix#2#1\endcsname
        }%
        \begingroup
          \edef\x{\endgroup
            \noexpand\define@key{\KVO@family}{#2}[true]{%
              \noexpand\KVO@boolkey{\@currname}%
              \ifx\@currext\@clsextension
                \noexpand\@clsextension
              \else
                \noexpand\@pkgextension
              \fi
              {\KVO@prefix}{#2}{####1}%
            }%
          }%
        \x
      }%
    }%
  }%
}
%    \end{macrocode}
%    \end{macro}
%
%    \begin{macro}{\DeclareComplementaryOption}
%    The first argument is the key name, the second the
%    key that must be a boolean option with the same
%    current family and prefix. A new switch is not
%    created for the new key, we have already a switch.
%    Instead we define switch setting commands to work
%    on the parent switch.
%    \begin{macrocode}
\newcommand*{\DeclareComplementaryOption}[2]{%
  \@ifundefined{if\KVO@prefix#2}{%
    \PackageError{kvoptions}{%
      Cannot generate option code for `#1',\MessageBreak
      parent switch `#2' does not exist%
    }{%
      You are inside %
      \ifx\@currext\@clsextension class\else package\fi\space
      `\@currname.\@currext'.\MessageBreak
      `\KVO@family' is used as familiy %
      for the keyval options.\MessageBreak
      `\KVO@prefix' serves as prefix %
      for internal switch macros.\MessageBreak
      \MessageBreak
      \@ehc
    }%
  }{%
    \KVO@ifdefinable{\KVO@prefix#1true}{%
      \KVO@ifdefinable{\KVO@prefix#1false}{%
        \expandafter\let\csname\KVO@prefix#1false\expandafter\endcsname
          \csname\KVO@prefix#2true\endcsname
        \expandafter\let\csname\KVO@prefix#1true\expandafter\endcsname
          \csname\KVO@prefix#2false\endcsname
%    \end{macrocode}
%    The same code part as in \cs{DeclareBoolOption} can
%    now be used.
%    \begin{macrocode}
        \begingroup
          \edef\x{\endgroup
            \noexpand\define@key{\KVO@family}{#1}[true]{%
              \noexpand\KVO@boolkey{\@currname}%
              \ifx\@currext\@clsextension
                \noexpand\@clsextension
              \else
                \noexpand\@pkgextension
              \fi
              {\KVO@prefix}{#1}{####1}%
            }%
          }%
        \x
      }%
    }%
  }%
}
%    \end{macrocode}
%    \end{macro}
%
%    \begin{macro}{\KVO@ifdefinable}
%    Generate the command token LaTeX's \cs{@ifdefinable} expects.
%    \begin{macrocode}
\def\KVO@ifdefinable#1{%
  \expandafter\@ifdefinable\csname #1\endcsname
}
%    \end{macrocode}
%    \end{macro}
%
%    \begin{macro}{\KVO@boolkey}
%    We check explicitly for |true| and |false| to prevent
%    the user from accidently calling other macros.
%    \begin{quote}
%    \begin{tabular}{@{}ll@{}}
%    |#1|& package/class name\\
%    |#2|& |\@pkgextension|/|\@clsextension|\\
%    |#3|& prefix \\
%    |#4|& key name\\
%    |#5|& new value
%    \end{tabular}
%    \end{quote}
%    \begin{macrocode}
\def\KVO@boolkey#1#2#3#4#5{%
  \edef\KVO@param{#5}%
  \ltx@onelevel@sanitize\KVO@param
  \ifx\KVO@param\KVO@true
    \expandafter\@firstofone
  \else
    \ifx\KVO@param\KVO@false
      \expandafter\expandafter\expandafter\@firstofone
    \else
      \ifx#2\@clsextension
        \expandafter\ClassWarning
      \else
        \expandafter\PackageWarning
      \fi
      {#1}{%
        Value `\KVO@param' is not supported by\MessageBreak
        option `#4'%
      }%
      \expandafter\expandafter\expandafter\@gobble
    \fi
  \fi
  {%
    ^^A\ifx#2\@clsextension
    ^^A  \expandafter\ClassInfo
    ^^A\else
    ^^A  \expandafter\PackageInfo
    ^^A\fi
    ^^A{#1}{[option] #4=\KVO@param}%
    \csname#3#4\KVO@param\endcsname
  }%
}
%    \end{macrocode}
%    \end{macro}
%    \begin{macro}{\KVO@true}
%    \begin{macro}{\KVO@false}
%    The macros \cs{KVO@true} and \cs{KVO@false} are used
%    for string comparisons. After \cs{ltx@onelevel@sanitize}
%    we have only tokens with catcode 12 (other).
%    \begin{macrocode}
\def\KVO@true{true}
\def\KVO@false{false}
\ltx@onelevel@sanitize\KVO@true
\ltx@onelevel@sanitize\KVO@false
%    \end{macrocode}
%    \end{macro}
%    \end{macro}
%
% \subsubsection{\cs{DeclareStringOption}}
%
%    \begin{macro}{\DeclareStringOption}
%    \begin{macrocode}
\newcommand*{\DeclareStringOption}[2][]{%
  \@ifnextchar[{%
    \KVO@DeclareStringOption{#1}{#2}@%
  }{%
    \KVO@DeclareStringOption{#1}{#2}{}[]%
  }%
}
%    \end{macrocode}
%    \end{macro}
%    \begin{macro}{\KVO@DeclareStringOption}
%    \begin{macrocode}
\def\KVO@DeclareStringOption#1#2#3[#4]{%
  \KVO@ifdefinable{\KVO@prefix#2}{%
    \@namedef{\KVO@prefix#2}{#1}%
    \begingroup
      \ifx\\#3\\%
        \toks@{}%
      \else
        \toks@{[{#4}]}%
      \fi
      \edef\x{\endgroup
        \noexpand\define@key{\KVO@family}{#2}\the\toks@{%
          ^^A\begingroup
          ^^A  \toks@{####1}%
          ^^A  \ifx\@currext\@clsextension
          ^^A    \noexpand\ClassInfo
          ^^A  \else
          ^^A    \noexpand\PackageInfo
          ^^A  \fi
          ^^A  {\@currname}{%
          ^^A    [option] #2={\noexpand\the\toks@}%
          ^^A  }%
          ^^A\endgroup
          \noexpand\def
          \expandafter\noexpand\csname\KVO@prefix#2\endcsname{####1}%
        }%
      }%
    \x
  }%
}
%    \end{macrocode}
%    \end{macro}
%
% \subsubsection{\cs{DeclareVoidOption}}
%
%    \begin{macro}{\DeclareVoidOption}
%    \begin{macrocode}
\newcommand*{\DeclareVoidOption}[2]{%
  \begingroup
    \let\next\@gobbletwo
    \KVO@ifdefinable{\KVO@prefix#1}{%
      \let\next\@firstofone
    }%
  \expandafter\endgroup
  \next{%
    \begingroup
      \edef\x{\endgroup
        \noexpand\define@key{\KVO@family}{#1}[\KVO@VOID@]{%
          \noexpand\KVO@voidkey{\@currname}%
          \ifx\@currext\@clsextension
            \noexpand\@clsextension
          \else
            \noexpand\@pkgextension
          \fi
          {#1}%
          {####1}%
          \expandafter\noexpand\csname\KVO@prefix#1\endcsname
        }%
      }%
    \x
    \begingroup
      \toks@{#2}%
    \expandafter\endgroup
    \expandafter\def
    \csname\KVO@prefix#1\expandafter\endcsname
    \expandafter{\the\toks@}%
  }%
}
\def\KVO@VOID@{@VOID@}
%    \end{macrocode}
%    \end{macro}
%    \begin{macro}{\KVO@voidkey}
%    \begin{tabular}{@{}ll@{}}
%    |#1|& package/class name\\
%    |#2|& |\@pkgextension|/|\@clsextension|\\
%    |#3|& key name\\
%    |#4|& default (|@VOID@|)\\
%    |#5|& macro with option code
%    \end{tabular}
%    \begin{macrocode}
\def\KVO@voidkey#1#2#3#4{%
  \def\CurrentOption{#3}%
  \begingroup
    \def\x{#4}%
  \expandafter\endgroup
  \ifx\x\KVO@VOID@
  \else
    \ifx#2\@clsextension
      \expandafter\ClassWarning
    \else
      \expandafter\PackageWarning
    \fi
    {#1}{%
      Unexpected value for option `#3'\MessageBreak
      is ignored%
    }%
  \fi
  ^^A\ifx#2\@clsextension
  ^^A  \expandafter\ClassInfo
  ^^A\else
  ^^A  \expandafter\PackageInfo
  ^^A\fi
  ^^A{#1}{[option] #3}%
}
%    \end{macrocode}
%    \end{macro}
%
% \subsubsection{\cs{DeclareDefaultOption}}
%
%    \begin{macro}{\DeclareDefaultOption}
%    \begin{macrocode}
\newcommand*{\DeclareDefaultOption}{%
  \@namedef{KVO@default@\@currname.\@currext}%
}
%    \end{macrocode}
%    \end{macro}
%
% \subsubsection{\cs{DeclareLocalOptions}}
%
%    \begin{macro}{\DeclareLocalOptions}
%    \begin{macrocode}
\newcommand*{\DeclareLocalOptions}[1]{%
  \comma@parse{#1}\KVO@DeclareLocalOption
}
%    \end{macrocode}
%    \end{macro}
%    \begin{macro}{\KVO@DeclareLocalOption}
%    \begin{macrocode}
\def\KVO@DeclareLocalOption#1{%
  \expandafter\def\csname KVO@local@\KVO@family @#1\endcsname{}%
}
%    \end{macrocode}
%    \end{macro}
%
% \subsection{Dynamic options}
%
% \subsubsection{\cs{DisableKeyvalOption}}
%
%    \begin{macrocode}
\SetupKeyvalOptions{%
  family=KVOdyn,%
  prefix=KVOdyn@%
}
\DeclareBoolOption[true]{global}
\DeclareComplementaryOption{local}{global}
\DeclareStringOption[undef]{action}
\let\KVOdyn@name\relax
\let\KVOdyn@ext\@empty
\define@key{KVOdyn}{class}{%
  \def\KVOdyn@name{#1}%
  \let\KVOdyn@ext\@clsextension
}
\define@key{KVOdyn}{package}{%
  \def\KVOdyn@name{#1}%
  \let\KVOdyn@ext\@pkgextension
}
\newcommand*{\DisableKeyvalOption}[3][]{%
  \begingroup
    \kvsetkeys{KVOdyn}{#1}%
    \def\x{\endgroup}%
    \@ifundefined{KVO@action@\KVOdyn@action}{%
      \PackageError{kvoptions}{%
        Unknown disable action %
        `\expandafter\strip@prefix\meaning\KVOdyn@action'\MessageBreak
        for option `#3' in keyval family '#2'%
      }\@ehc
    }{%
      \csname KVO@action@\KVOdyn@action\endcsname{#2}{#3}%
    }%
  \x
}
\def\KVO@action@undef#1#2{%
  \edef\x{\endgroup
    \ifKVOdyn@global\global\fi
    \let
    \expandafter\noexpand\csname KV@#1@#2\endcsname
    \relax
    \ifKVOdyn@global\global\fi
    \let
    \expandafter\noexpand\csname KV@#1@#2@default\endcsname
    \relax
  }%
  ^^A\PackageInfo{kvoptions}{%
  ^^A  [option] key `#2' of family `#1'\MessageBreak
  ^^A  is disabled (undef, \ifKVOdyn@global global\else local\fi)%
  ^^A}%
}
\def\KVO@action@ignore#1#2{%
  \edef\x{\endgroup
    \ifKVOdyn@global\global\fi
    \let
    \expandafter\noexpand\csname KV@#1@#2\endcsname
    \noexpand\@gobble
    \ifKVOdyn@global\global\fi
    \let
    \expandafter\noexpand\csname KV@#1@#2@default\endcsname
    \noexpand\@empty
  }%
  ^^A\PackageInfo{kvoptions}{%
  ^^A  [option] key `#2' of family `#1'\MessageBreak
  ^^A  is disabled (ignore, \ifKVOdyn@global global\else local\fi)%
  ^^A}%
}
\def\KVO@action@error{%
  \KVO@do@action{error}%
}
\def\KVO@action@warning{%
  \KVO@do@action{warning}%
}
%    \end{macrocode}
%    \begin{tabular}{@{}ll@{}}
%    |#1|& |error| or |warning|\\
%    |#2|& \meta{family}\\
%    |#3|& \meta{key}\\
%    \end{tabular}
%    \begin{macrocode}
\def\KVO@do@action#1#2#3{%
  \ifx\KVOdyn@name\relax
    \PackageError{kvoptions}{%
      Action type `#1' needs package/class name\MessageBreak
      for key `#3' in family `#2'%
    }\@ehc
  \else
    \edef\x{\endgroup
      \noexpand\define@key{#2}{#3}[]{%
        \expandafter\noexpand\csname KVO@disable@#1\endcsname
        {\KVOdyn@name}\noexpand\KVOdyn@ext{#3}%
      }%
      \ifKVOdyn@global
        \global\let
        \expandafter\noexpand\csname KV@#2@#3\endcsname
        \expandafter\noexpand\csname KV@#2@#3\endcsname
        \global\let
        \expandafter\noexpand\csname KV@#2@#3@default\endcsname
        \expandafter\noexpand\csname KV@#2@#3@default\endcsname
      \fi
    }%
    ^^A\ifx\KVOdyn@ext\@clsextension
    ^^A  \expandafter\ClassInfo
    ^^A\else
    ^^A   \expandafter\PackageInfo
    ^^A\fi
    ^^A{\KVOdyn@name}{%
    ^^A  [option] key `#3' of family `#2'\MessageBreak
    ^^A  is disabled (#1, \ifKVOdyn@global global\else local\fi)%
    ^^A}%
  \fi
}
\def\KVO@disable@error#1#2#3{%
  \ifx#2\@clsextension
    \expandafter\ClassError
  \else
    \expandafter\PackageError
  \fi
  {#1}{%
    Option `#3' is given too late,\MessageBreak
    now the option is ignored%
  }\@ehc
}
\def\KVO@disable@warning#1#2#3{%
  \ifx#2\@clsextension
    \expandafter\ClassWarning
  \else
    \expandafter\PackageWarning
  \fi
  {#1}{%
    Option `#3' is already consumed\MessageBreak
    and has no effect%
  }%
}
%    \end{macrocode}
%
% \subsection{Change option code}
%
% \subsubsection{\cs{AddToKeyvalOption}}
%
%    \begin{macro}{\AddToKeyvalOption}
%    \begin{macrocode}
\newcommand*{\AddToKeyvalOption}{%
  \@ifstar{%
    \begingroup
      \edef\x{\endgroup
        \noexpand\KVO@AddToKeyvalOption{\KVO@family}%
      }%
    \x
  }%
  \KVO@AddToKeyvalOption
}
%    \end{macrocode}
%    \end{macro}
%    \begin{macro}{\KVO@AddToKeyvalOption}
%    \begin{macrocode}
\def\KVO@AddToKeyvalOption#1#2{%
  \@ifundefined{KV@#1@#2}{%
    \PackageWarning{kvoptions}{%
      Key `#2' of family `#1' does not exist.\MessageBreak
      Ignoring \string\AddToKeyvalOption
    }%
    \@gobble
  }{%
    \edef\KVO@next{%
      \noexpand\KVO@@AddToKeyvalOption
      \expandafter\noexpand\csname KV@#1@#2\endcsname
    }%
    \afterassignment\KVO@next
    \def\KVO@temp##1%
  }%
}
%    \end{macrocode}
%    \end{macro}
%    \begin{macro}{\KVO@@AddToKeyvalOption}
%    \begin{macrocode}
\def\KVO@@AddToKeyvalOption#1{%
  \begingroup
    \toks@\expandafter{#1{##1}}%
    \toks@\expandafter{\the\expandafter\toks@\KVO@temp{##1}}%
    \edef\x{\endgroup
      \noexpand\def\noexpand#1####1{\the\toks@}%
    }%
  \x
}
%    \end{macrocode}
%    \end{macro}
%
% \subsection{Process options}
%
% \subsubsection{Get global options}
%
%    Package \xpackage{xkeyval} removes options with equal signs from
%    the global options (\cs{@classoptionslist}). The effect is that
%    other packages and classes will not see these global options anymore.
%    A bug-report was answered that this behaviour is ``by design''.
%    Thus I call it a design bug. Now getting the global options require
%    an algorithm instead of a simple macro call.
%    \begin{macrocode}
%</package>
%<*package|patch>
%    \end{macrocode}
%    \begin{macro}{\KVO@IfDefThen}
%    Call |#2| if command |#1| is defined and not \cs{relax}.
%    (Package \xpackage{kvoptions-patch} does not load
%    package \xpackage{ltxcmds}.)
%    \begin{macrocode}
\def\KVO@IfDefThen#1#2{%
  \ifx#1\ltx@undefined
  \else
    \ifx#1\relax
    \else
      #2%
    \fi
  \fi
}%
%    \end{macrocode}
%    \end{macro}
%    \begin{macro}{\KVO@GetClassOptionsList}
%    \begin{macrocode}
\def\KVO@GetClassOptionsList{%
  \let\KVO@classoptionslist\@classoptionslist
  \KVO@IfDefThen\@classoptionslist{%
    \KVO@IfDefThen\XKV@documentclass{%
      \ifx\XKV@documentclass\ltx@empty
      \else
        \KVO@IfDefThen\XKV@classoptionslist{%
          \ifx\XKV@classoptionslist\ltx@empty
          \else
            \let\KVO@classoptionslist\XKV@classoptionslist
          \fi
        }%
      \fi
    }%
  }%
}%
%    \end{macrocode}
%    \end{macro}
%    \begin{macrocode}
%</package|patch>
%<*package>
%    \end{macrocode}
%
% \subsubsection{\cs{ProcessKeyvalOptions}}
%
%    \begin{macro}{\ProcessKeyvalOptions}
%    If the optional star is given, we get the family
%    name and expand it for safety.
%    \begin{macrocode}
\newcommand*{\ProcessKeyvalOptions}{%
  \@ifstar{%
    \begingroup
      \edef\x{\endgroup
        \noexpand\KVO@ProcessKeyvalOptions{\KVO@family}%
      }%
    \x
  }%
  \KVO@ProcessKeyvalOptions
}
%    \end{macrocode}
%    \end{macro}
%
%    \begin{macrocode}
\def\KVO@ProcessKeyvalOptions#1{%
  \let\@tempc\relax
  \let\KVO@temp\@empty
%    \end{macrocode}
%    Add any global options that are known to KV to the start of
%    the list being built in |\KVO@temp| and mark them used (by
%    removing them from the unused option list).
%    \begin{macrocode}
  \ifx\@currext\@clsextension
  \else
    \KVO@GetClassOptionsList
    \ifx\KVO@classoptionslist\relax
    \else
      \@for\KVO@CurrentOption:=\KVO@classoptionslist\do{%
        \@ifundefined{KV@#1@\expandafter\KVO@getkey
                      \KVO@CurrentOption=\@nil}{%
        }{%
          \@ifundefined{KVO@local@#1@\expandafter\KVO@getkey
                        \KVO@CurrentOption=\@nil}{%
            \ifx\KVO@Patch Y%
              \edef\KVO@temp{%
                \etex@unexpanded\expandafter{%
                  \KVO@temp
                }%
                ,%
                \etex@unexpanded\expandafter{%
                  \KVO@CurrentOption
                }%
                ,%
              }%
              \ltx@onelevel@sanitize\KVO@CurrentOption
            \else
              \edef\KVO@temp{%
                \KVO@temp
                ,%
                \KVO@CurrentOption
                ,%
              }%
            \fi
            \@expandtwoargs\@removeelement\KVO@CurrentOption
              \@unusedoptionlist\@unusedoptionlist
          }{}%
        }%
      }%
    \fi
  \fi
%    \end{macrocode}
%     Now stick the package options at the end of the list and wrap
%     in a call to \cs{setkeys}. A class ignores unknown global
%     options, we must remove them to prevent error messages
%     from \cs{setkeys}.
%    \begin{macrocode}
  \begingroup
    \toks\tw@{}%
    \@ifundefined{opt@\@currname.\@currext}{%
      \toks@\expandafter{\KVO@temp}%
    }{%
      \toks@\expandafter\expandafter\expandafter{%
        \csname opt@\@currname.\@currext\endcsname
      }%
      \ifx\@currext\@clsextension
        \edef\CurrentOption{\the\toks@}%
        \toks@\expandafter{\KVO@temp}%
        \@for\CurrentOption:=\CurrentOption\do{%
          \@ifundefined{%
            KV@#1@\expandafter\KVO@getkey\CurrentOption=\@nil
          }{%
%    \end{macrocode}
%    A class puts not used options in the unused option list
%    unless there is a default handler.
%    \begin{macrocode}
            \@ifundefined{KVO@default@\@currname.\@currext}{%
              \ifx\KVO@Patch Y%
                \ltx@onelevel@sanitize\CurrentOption
              \fi
              \ifx\@unusedoptionlist\@empty
                \global\let\@unusedoptionlist\CurrentOption
              \else
                \expandafter\expandafter\expandafter\gdef
                \expandafter\expandafter\expandafter\@unusedoptionlist
                \expandafter\expandafter\expandafter{%
                  \expandafter\@unusedoptionlist
                  \expandafter,\CurrentOption
                }%
              \fi
            }{%
              \toks\tw@\expandafter{%
                \the\toks\expandafter\tw@\expandafter,\CurrentOption
              }%
            }%
          }{%
            \toks@\expandafter{%
              \the\expandafter\toks@\expandafter,\CurrentOption
            }%
          }%
        }%
      \else
%    \end{macrocode}
%    Without default action we pass all options to \cs{setkeys}.
%    Otherwise we have to check which options are known.
%    These are passed to \cs{setkeys}. For the others the default
%    action is performed.
%    \begin{macrocode}
        \@ifundefined{KVO@default@\@currname.\@currext}{%
          \toks@\expandafter\expandafter\expandafter{%
            \expandafter\KVO@temp\the\toks@
          }%
        }{%
          \edef\CurrentOption{\the\toks@}%
          \toks@\expandafter{\KVO@temp}%
          \@for\CurrentOption:=\CurrentOption\do{%
            \@ifundefined{%
              KV@#1@\expandafter\KVO@getkey\CurrentOption=\@nil
            }{%
              \toks\tw@\expandafter{%
                \the\toks\expandafter\tw@\expandafter,\CurrentOption
              }%
            }{%
              \toks@\expandafter{%
                \the\expandafter\toks@\expandafter,\CurrentOption
              }%
            }%
          }%
        }%
      \fi
    }%
    \edef\KVO@temp{\endgroup
      \noexpand\KVO@calldefault{\the\toks\tw@}%
      \noexpand\KVO@setkeys{#1}{\the\toks@}%
    }%
  \KVO@temp
%    \end{macrocode}
%    Some cleanup of \cs{ProcessOptions}.
%    \begin{macrocode}
  \let\CurrentOption\@empty
  \AtEndOfPackage{\let\@unprocessedoptions\relax}%
}
%    \end{macrocode}
%
% \subsubsection{\cs{ProcessLocalKeyvalOptions}}
%
%    \begin{macro}{\ProcessLocalKeyvalOptions}
%    If the optional star is given, we get the family
%    name and expand it for safety.
%    \begin{macrocode}
\newcommand*{\ProcessLocalKeyvalOptions}{%
  \@ifstar{%
    \begingroup
      \edef\x{\endgroup
        \noexpand\KVO@ProcessLocalKeyvalOptions{\KVO@family}%
      }%
    \x
  }%
  \KVO@ProcessLocalKeyvalOptions
}
%    \end{macrocode}
%    \end{macro}
%
%    \begin{macrocode}
\def\KVO@ProcessLocalKeyvalOptions#1{%
  \let\@tempc\relax
  \let\KVO@temp\@empty
%    \end{macrocode}
%    Check if \cs{ProcessLocalKeyvalOptions} is called inside
%    a package.
%    \begin{macrocode}
  \ifx\@currext\@pkgextension
  \else
    \PackageError{kvoptions}{%
      \string\ProcessLocalKeyvalOptions\space is intended for packages only%
    }\@ehc
  \fi
%    \end{macrocode}
%     The package options are put into toks register \cs{toks@}.
%    \begin{macrocode}
  \begingroup
    \toks\tw@{}%
    \@ifundefined{opt@\@currname.\@currext}{%
      \toks@\expandafter{\KVO@temp}%
    }{%
      \toks@\expandafter\expandafter\expandafter{%
        \csname opt@\@currname.\@currext\endcsname
      }%
%    \end{macrocode}
%    Without default action we pass all options to \cs{setkeys}.
%    Otherwise we have to check which options are known.
%    These are passed to \cs{setkeys}. For the others the default
%    action is performed.
%    \begin{macrocode}
      \@ifundefined{KVO@default@\@currname.\@currext}{%
        \toks@\expandafter\expandafter\expandafter{%
          \expandafter\KVO@temp\the\toks@
        }%
      }{%
        \edef\CurrentOption{\the\toks@}%
        \toks@\expandafter{\KVO@temp}%
        \@for\CurrentOption:=\CurrentOption\do{%
          \@ifundefined{%
            KV@#1@\expandafter\KVO@getkey\CurrentOption=\@nil
          }{%
            \toks\tw@\expandafter{%
              \the\toks\expandafter\tw@\expandafter,\CurrentOption
            }%
          }{%
            \toks@\expandafter{%
              \the\expandafter\toks@\expandafter,\CurrentOption
            }%
          }%
        }%
      }%
    }%
    \edef\KVO@temp{\endgroup
      \noexpand\KVO@calldefault{\the\toks\tw@}%
      \noexpand\KVO@setkeys{#1}{\the\toks@}%
    }%
  \KVO@temp
%    \end{macrocode}
%    Some cleanup of \cs{ProcessOptions}.
%    \begin{macrocode}
  \let\CurrentOption\@empty
  \AtEndOfPackage{\let\@unprocessedoptions\relax}%
}
%    \end{macrocode}
%
% \subsubsection{Helper macros}
%
%    \begin{macro}{\KVO@getkey}
%    Extract the key part of a key=value pair.
%    \begin{macrocode}
\def\KVO@getkey#1=#2\@nil{#1}
%    \end{macrocode}
%    \end{macro}
%
%    \begin{macro}{\KVO@calldefault}
%    \begin{macrocode}
\def\KVO@calldefault#1{%
  \begingroup
    \def\x{#1}%
  \expandafter\endgroup
  \ifx\x\@empty
  \else
    \@for\CurrentOption:=#1\do{%
      \ifx\CurrentOption\@empty
      \else
        \expandafter\KVO@setcurrents\CurrentOption=\@nil
        \@nameuse{KVO@default@\@currname.\@currext}%
      \fi
    }%
  \fi
}
%    \end{macrocode}
%    \end{macro}
%    \begin{macro}{\KVO@setcurrents}
%    Extract the key part of a key=value pair.
%    \begin{macrocode}
\def\KVO@setcurrents#1=#2\@nil{%
  \def\CurrentOptionValue{#2}%
  \ifx\CurrentOptionValue\@empty
    \let\CurrentOptionKey\CurrentOption
    \let\CurrentOptionValue\relax
  \else
    \edef\CurrentOptionKey{\zap@space#1 \@empty}%
    \expandafter\KVO@setcurrentvalue\CurrentOption\@nil
  \fi
}
%    \end{macrocode}
%    \end{macro}
%    \begin{macro}{\KV@setcurrentvalue}
%    Here the value part is parsed. Package \xpackage{keyval}'s
%    \cs{KV@@sp@def} helps in removing spaces at the begin and
%    end of the value.
%    \begin{macrocode}
\def\KVO@setcurrentvalue#1=#2\@nil{%
  \KV@@sp@def\CurrentOptionValue{#2}%
}
%    \end{macrocode}
%    \end{macro}
%
% \subsection{\plainTeX}
%
%    Disable \LaTeX\ stuff.
%    \begin{macrocode}
\begingroup\expandafter\expandafter\expandafter\endgroup
\expandafter\ifx\csname documentclass\endcsname\relax
  \def\ProcessKeyvalOptions{%
    \@ifstar{}\@gobble
  }%
\fi
%    \end{macrocode}
%
%    \begin{macrocode}
\KVO@AtEnd%
%</package>
%    \end{macrocode}
%
% \subsection{Package \xpackage{kvoptions-patch}}
%
%    \begin{macrocode}
%<*patch>
\NeedsTeXFormat{LaTeX2e}
\begingroup\catcode61\catcode48\catcode32=10\relax%
  \catcode13=5 % ^^M
  \endlinechar=13 %
  \catcode123=1 % {
  \catcode125=2 % }
  \catcode64=11 % @
  \def\x{\endgroup
    \expandafter\edef\csname KVO@AtEnd\endcsname{%
      \endlinechar=\the\endlinechar\relax
      \catcode13=\the\catcode13\relax
      \catcode32=\the\catcode32\relax
      \catcode35=\the\catcode35\relax
      \catcode61=\the\catcode61\relax
      \catcode64=\the\catcode64\relax
      \catcode123=\the\catcode123\relax
      \catcode125=\the\catcode125\relax
    }%
  }%
\x\catcode61\catcode48\catcode32=10\relax%
\catcode13=5 % ^^M
\endlinechar=13 %
\catcode35=6 % #
\catcode64=11 % @
\catcode123=1 % {
\catcode125=2 % }
\def\TMP@EnsureCode#1#2{%
  \edef\KVO@AtEnd{%
    \KVO@AtEnd
    \catcode#1=\the\catcode#1\relax
  }%
  \catcode#1=#2\relax
}
\TMP@EnsureCode{39}{12}% '
\TMP@EnsureCode{40}{12}% (
\TMP@EnsureCode{41}{12}% )
\TMP@EnsureCode{43}{12}% +
\TMP@EnsureCode{44}{12}% ,
\TMP@EnsureCode{45}{12}% -
\TMP@EnsureCode{46}{12}% .
\TMP@EnsureCode{47}{12}% /
\TMP@EnsureCode{58}{12}% :
\TMP@EnsureCode{60}{12}% <
\TMP@EnsureCode{62}{12}% >
\TMP@EnsureCode{91}{12}% [
\TMP@EnsureCode{93}{12}% ]
\TMP@EnsureCode{96}{12}% `
\TMP@EnsureCode{124}{12}% |
\edef\KVO@AtEnd{\KVO@AtEnd\noexpand\endinput}
\ProvidesPackage{kvoptions-patch}%
  [2016/05/16 v3.12 LaTeX patch for keyval options (HO)]%
%    \end{macrocode}
%
%    Check for \eTeX.
%    \begin{macrocode}
\begingroup\expandafter\expandafter\expandafter\endgroup
\expandafter\ifx\csname eTeXversion\endcsname\relax
  \PackageWarningNoLine{kvoptions-patch}{%
    Package loading is aborted, because e-TeX is missing%
  }%
  \expandafter\KVO@AtEnd
\fi%
%    \end{macrocode}
%
%    Package \xpackage{etexcmds} for \cs{etex@unexpanded}.
%    \begin{macrocode}
\RequirePackage{etexcmds}[2007/09/09]
\ifetex@unexpanded
\else
  \PackageError{kvoptions-patch}{%
    Could not find eTeX's \string\unexpanded.\MessageBreak
    Try adding \string\RequirePackage\string{etexcmds\string} %
    before \string\documentclass%
  }\@ehd
  \expandafter\KVO@AtEnd
\fi%
%    \end{macrocode}
%
%    Check for package \xpackage{xkvltxp}.
%    \begin{macrocode}
\@ifpackageloaded{xkvltxp}{%
  \PackageWarningNoLine{kvoptions}{%
    Option `patch' cannot be used together with\MessageBreak
    package `xkvltxp' that is already loaded.\MessageBreak
    Therefore package loading is aborted%
  }%
  \KVO@AtEnd
}{}%
%    \end{macrocode}
%
%    \begin{macrocode}
\def\@if@ptions#1#2#3{%
  \begingroup
    \KVO@normalize\KVO@temp{#3}%
    \edef\x{\endgroup
      \noexpand\@if@pti@ns{%
        \detokenize\expandafter\expandafter\expandafter{%
          \csname opt@#2.#1\endcsname
        }%
      }{%
        \detokenize\expandafter{\KVO@temp}%
      }%
    }%
  \x
}
%    \end{macrocode}
%
%    \begin{macrocode}
\def\@pass@ptions#1#2#3{%
  \KVO@normalize\KVO@temp{#2}%
  \@ifundefined{opt@#3.#1}{%
    \expandafter\gdef\csname opt@#3.#1%
          \expandafter\endcsname\expandafter{%
      \KVO@temp
    }%
  }{%
    \expandafter\gdef\csname opt@#3.#1%
          \expandafter\expandafter\expandafter\endcsname
          \expandafter\expandafter\expandafter{%
      \csname opt@#3.#1\expandafter\endcsname\expandafter,\KVO@temp
    }%
  }%
}
%    \end{macrocode}
%
%    \begin{macrocode}
\def\ProcessOptions{%
  \let\ds@\@empty
  \@ifundefined{opt@\@currname.\@currext}{%
    \let\@curroptions\@empty
  }{%
    \expandafter\expandafter\expandafter\def
    \expandafter\expandafter\expandafter\@curroptions
    \expandafter\expandafter\expandafter{%
      \csname opt@\@currname.\@currext\endcsname
    }%
  }%
  \@ifstar\KVO@xprocess@ptions\KVO@process@ptions
}
%    \end{macrocode}
%
%    \begin{macrocode}
\def\KVO@process@ptions{%
  \@for\CurrentOption:=\@declaredoptions\do{%
    \ifx\CurrentOption\@empty
    \else
      \begingroup
        \ifx\@currext\@clsextension
          \toks@{}%
        \else
          \KVO@GetClassOptionsList
          \toks@\expandafter{\KVO@classoptionslist,}%
        \fi
        \toks\tw@\expandafter{\@curroptions}%
        \edef\x{\endgroup
          \noexpand\in@{,\CurrentOption,}{,\the\toks@\the\toks\tw@,}%
        }%
      \x
      \ifin@
        \KVO@use@ption
        \expandafter\let\csname ds@\CurrentOption\endcsname\@empty
      \fi
    \fi
  }%
  \KVO@process@pti@ns
}
%    \end{macrocode}
%
%    \begin{macrocode}
\def\KVO@xprocess@ptions{%
  \ifx\@currext\@clsextension
  \else
    \KVO@GetClassOptionsList
    \@for\CurrentOption:=\KVO@classoptionslist\do{%
      \ifx\CurrentOption\@empty
      \else
        \KVO@in@\CurrentOption\@declaredoptions
        \ifin@
          \KVO@use@ption
          \expandafter\let\csname ds@\CurrentOption\endcsname\@empty
        \fi
      \fi
    }%
  \fi
  \KVO@process@pti@ns
}
%    \end{macrocode}
%
%    \begin{macrocode}
\def\KVO@in@#1#2{%
  \in@false
  \begingroup
    \@for\x:=#2\do{%
      \ifx\x#1\relax
        \in@true
      \fi
    }%
    \edef\x{\endgroup
      \ifin@
        \noexpand\in@true
      \fi
    }%
  \x
}
%    \end{macrocode}
%
%    \begin{macrocode}
\def\KVO@process@pti@ns{%
  \@for\CurrentOption:=\@curroptions\do{%
    \@ifundefined{ds@\KVO@SanitizedCurrentOption}{%
      \KVO@use@ption
      \default@ds
    }%
    \KVO@use@ption
  }%
  \@for\CurrentOption:=\@declaredoptions\do{%
    \expandafter\let\csname ds@\CurrentOption\endcsname\relax
  }%
  \let\CurrentOption\@empty
  \let\@fileswith@pti@ns\@@fileswith@pti@ns
  \AtEndOfPackage{\let\@unprocessedoptions\relax}%
}
%    \end{macrocode}
%
%    \begin{macrocode}
\def\KVO@use@ption{%
  \begingroup
    \edef\x{\endgroup
      \noexpand\@removeelement{%
        \detokenize\expandafter{\CurrentOption}%
      }{%
        \detokenize\expandafter{\@unusedoptionlist}%
      }%
    }%
  \x\@unusedoptionlist
  \csname ds@\KVO@SanitizedCurrentOption\endcsname
}
%    \end{macrocode}
%
%    \begin{macrocode}
\def\OptionNotUsed{%
  \ifx\@currext\@clsextension
    \xdef\@unusedoptionlist{%
      \ifx\@unusedoptionlist\@empty
      \else
        \detokenize\expandafter{\@unusedoptionlist,}%
      \fi
      \detokenize\expandafter{\CurrentOption}%
    }%
  \fi
}
%    \end{macrocode}
%
%    Variant of \cs{ExecuteOptions} that better protects \cs{CurrentOption}.
%    \begin{macrocode}
\def\CurrentOption@SaveLevel{0}
\def\ExecuteOptions{%
  \expandafter\KVO@ExecuteOptions
      \csname CurrentOption@\CurrentOption@SaveLevel\endcsname
}
\def\KVO@ExecuteOptions#1#2{%
  \let#1\CurrentOption
  \edef\CurrentOption@SaveLevel{%
    \the\numexpr\CurrentOption@SaveLevel+1%
  }%
  \@for\CurrentOption:=#2\do{%
    \csname ds@\CurrentOption\endcsname
  }%
  \edef\CurrentOption@SaveLevel{%
    \the\numexpr\CurrentOption@SaveLevel-1%
  }%
  \let\CurrentOption#1%
}
%    \end{macrocode}
%
%    \begin{macrocode}
\def\KVO@fileswith@pti@ns#1[#2]#3[#4]{%
  \ifx#1\@clsextension
    \ifx\@classoptionslist\relax
      \KVO@normalize\KVO@temp{#2}%
      \expandafter\gdef\expandafter\@classoptionslist\expandafter{%
        \KVO@temp
      }%
      \def\reserved@a{%
        \KVO@onefilewithoptions{#3}[{#2}][{#4}]#1%
        \@documentclasshook
      }%
    \else
      \def\reserved@a{%
        \KVO@onefilewithoptions{#3}[{#2}][{#4}]#1%
      }%
    \fi
  \else
    \begingroup
      \let\KVO@temp\relax
      \let\KVO@onefilewithoptions\relax
      \let\@pkgextension\relax
      \def\reserved@b##1,{%
        \ifx\@nil##1\relax
        \else
          \ifx\relax##1\relax
          \else
            \KVO@onefilewithoptions{##1}[{\KVO@temp}][{#4}]%
            \@pkgextension
          \fi
          \expandafter\reserved@b
        \fi
      }%
      \edef\reserved@a{\zap@space#3 \@empty}%
      \edef\reserved@a{\expandafter\reserved@b\reserved@a,\@nil,}%
      \toks@{#2}%
      \def\KVO@temp{\the\toks@}%
    \edef\reserved@a{\endgroup \reserved@a}%
  \fi
  \reserved@a
}
%    \end{macrocode}
%
%    \begin{macrocode}
\def\KVO@onefilewithoptions#1[#2][#3]#4{%
  \@pushfilename
  \xdef\@currname{#1}%
  \global\let\@currext#4%
  \expandafter\let\csname\@currname.\@currext-h@@k\endcsname\@empty
  \let\CurrentOption\@empty
  \@reset@ptions
  \makeatletter
  \def\reserved@a{%
    \@ifl@aded\@currext{#1}{%
      \@if@ptions\@currext{#1}{#2}{%
      }{%
        \begingroup
          \@ifundefined{opt@#1.\@currext}{%
            \def\x{}%
          }{%
            \edef\x{%
              \expandafter\expandafter\expandafter\strip@prefix
              \expandafter\meaning\csname opt@#1.\@currext\endcsname
            }%
          }%
          \def\y{#2}%
          \edef\y{\expandafter\strip@prefix\meaning\y}%
          \@latex@error{Option clash for \@cls@pkg\space #1}{%
            The package #1 has already been loaded %
            with options:\MessageBreak
            \space\space[\x]\MessageBreak
            There has now been an attempt to load it %
             with options\MessageBreak
            \space\space[\y]\MessageBreak
            Adding the global options:\MessageBreak
            \space\space
                 \x,\y\MessageBreak
            to your \noexpand\documentclass declaration may fix this.%
            \MessageBreak
            Try typing \space <return> \space to proceed.%
          }%
        \endgroup
      }%
    }{%
      \@pass@ptions\@currext{#2}{#1}%
      \global\expandafter
      \let\csname ver@\@currname.\@currext\endcsname\@empty
      \InputIfFileExists
        {\@currname.\@currext}%
        {}%
        {\@missingfileerror\@currname\@currext}%
      \let\@unprocessedoptions\@@unprocessedoptions
      \csname\@currname.\@currext-h@@k\endcsname
      \expandafter\let\csname\@currname.\@currext-h@@k\endcsname
              \@undefined
      \@unprocessedoptions
    }%
    \@ifl@ter\@currext{#1}{#3}{%
    }{%
      \@latex@warning@no@line{%
        You have requested,\on@line, %
        version\MessageBreak
          #3' of \@cls@pkg\space #1,\MessageBreak
        but only version\MessageBreak
         `\csname ver@#1.\@currext\endcsname'\MessageBreak
        is available%
      }%
    }%
    \ifx\@currext\@clsextension\let\LoadClass\@twoloadclasserror\fi
    \@popfilename
    \@reset@ptions
  }%
  \reserved@a
}
%    \end{macrocode}
%
%    \begin{macrocode}
\def\@unknownoptionerror{%
  \@latex@error{%
    Unknown option `\KVO@SanitizedCurrentOption' %
    for \@cls@pkg\space`\@currname'%
  }{%
    The option `\KVO@SanitizedCurrentOption' was not declared in %
    \@cls@pkg\space`\@currname', perhaps you\MessageBreak
    misspelled its name. %
    Try typing \space <return> %
    \space to proceed.%
  }%
}
%    \end{macrocode}
%
%    \begin{macrocode}
\def\@@unprocessedoptions{%
  \ifx\@currext\@pkgextension
    \@ifundefined{opt@\@currname.\@currext}{%
      \let\@curroptions\@empty
    }{%
      \expandafter\let\expandafter\@curroptions
          \csname opt@\@currname.\@currext\endcsname
    }%
    \@for\CurrentOption:=\@curroptions\do{%
        \ifx\CurrentOption\@empty\else\@unknownoptionerror\fi
    }%
  \fi
}
%    \end{macrocode}
%
%    \begin{macrocode}
\def\KVO@SanitizedCurrentOption{%
  \expandafter\strip@prefix\meaning\CurrentOption
}
%    \end{macrocode}
%
%    Normalize option list.
%    \begin{macrocode}
\def\KVO@normalize#1#2{%
  \let\KVO@result\@empty
  \KVO@splitcomma#2,\@nil
  \let#1\KVO@result
}
\def\KVO@splitcomma#1,#2\@nil{%
  \KVO@ifempty{#1}{}{%
    \KVO@checkkv#1=\@nil
  }%
  \KVO@ifempty{#2}{}{\KVO@splitcomma#2\@nil}%
}
\def\KVO@ifempty#1{%
  \expandafter\ifx\expandafter\\\detokenize{#1}\\%
    \expandafter\@firstoftwo
  \else
    \expandafter\@secondoftwo
  \fi
}
\def\KVO@checkkv#1=#2\@nil{%
  \KVO@ifempty{#2}{%
    % option without value
    \edef\KVO@x{\zap@space#1 \@empty}%
    \ifx\KVO@x\@empty
      % ignore empty option
    \else
      % append to list
      \edef\KVO@result{%
        \etex@unexpanded\expandafter{\KVO@result},\KVO@x
      }%
    \fi
  }{%
    % #1: "key", #2: "value="
    % add key part
    \edef\KVO@result{%
      \etex@unexpanded\expandafter{\KVO@result},%
      \zap@space#1 \@empty
    }%
    \futurelet\@let@token\KVO@checkfirsttok#2 \@nil| = \@nil|\KVO@nil
  }%
}
\def\KVO@checkfirsttok{%
  \ifx\@let@token\bgroup
    % no space at start
    \expandafter\KVO@removelastspace\expandafter=%
    % "<value><spaceopt>= \@nil"
  \else
    \expandafter\KVO@checkfirstA
  \fi
}
\def\KVO@checkfirstA#1 #2\@nil{%
  \KVO@ifempty{#2}{%
    \KVO@removelastspace=#1 \@nil
  }{%
    \KVO@ifempty{#1}{%
      \KVO@removelastspace=#2\@nil
    }{%
      \KVO@removelastspace=#1 #2\@nil
    }%
  }%
}
\def\KVO@removelastspace#1 = \@nil|#2\KVO@nil{%
  \KVO@ifempty{#2}{%
    \edef\KVO@result{%
      \etex@unexpanded\expandafter{\KVO@result}%
      \etex@unexpanded\expandafter{\KVO@removegarbage#1\KVO@nil}%
    }%
  }{%
    \edef\KVO@result{%
      \etex@unexpanded\expandafter{\KVO@result}%
      \etex@unexpanded{#1}%
    }%
  }%
}
\def\KVO@removegarbage#1= \@nil#2\KVO@nil{#1}%
%    \end{macrocode}
%
%    Arguments |#1| and |#2| are macros.
%    \begin{macrocode}
\def\KVO@removeelement#1#2{%
  \begingroup
    \toks@={}%
    \@for\x:=#2\do{%
      \ifx\x\@empty
      \else
        \ifx\x#1\relax
        \else
          \edef\t{\the\toks@}%
          \ifx\t\@empty
          \else
            \toks@\expandafter{\the\toks@,}%
          \fi
          \toks@\expandafter{\the\expandafter\toks@\x}%
        \fi
      \fi
    }%
    \edef\x{\endgroup
      \def\noexpand#2{\the\toks@}%
    }%
  \x
}
%    \end{macrocode}
%
%
%    \begin{macrocode}
\let\@@fileswith@pti@ns\KVO@fileswith@pti@ns
\ifx\@fileswith@pti@ns\@badrequireerror
\else
  \let\@fileswith@pti@ns\KVO@fileswith@pti@ns
\fi
%    \end{macrocode}
%
%    \begin{macro}{\KVO@Patch}
%    \begin{macrocode}
\let\KVO@Patch=Y
%    \end{macrocode}
%    \end{macro}
%
%    \begin{macrocode}
\KVO@AtEnd%
%</patch>
%    \end{macrocode}
%
% \section{Test}
%
% \subsection{Preface for standard catcode check}
%
%    \begin{macrocode}
%<*test1>
\input miniltx.tex\relax
%</test1>
%    \end{macrocode}
%
% \subsection{Catcode checks for loading}
%
%    \begin{macrocode}
%<*test1>
%    \end{macrocode}
%    \begin{macrocode}
\catcode`\{=1 %
\catcode`\}=2 %
\catcode`\#=6 %
\catcode`\@=11 %
\expandafter\ifx\csname count@\endcsname\relax
  \countdef\count@=255 %
\fi
\expandafter\ifx\csname @gobble\endcsname\relax
  \long\def\@gobble#1{}%
\fi
\expandafter\ifx\csname @firstofone\endcsname\relax
  \long\def\@firstofone#1{#1}%
\fi
\expandafter\ifx\csname loop\endcsname\relax
  \expandafter\@firstofone
\else
  \expandafter\@gobble
\fi
{%
  \def\loop#1\repeat{%
    \def\body{#1}%
    \iterate
  }%
  \def\iterate{%
    \body
      \let\next\iterate
    \else
      \let\next\relax
    \fi
    \next
  }%
  \let\repeat=\fi
}%
\def\RestoreCatcodes{}
\count@=0 %
\loop
  \edef\RestoreCatcodes{%
    \RestoreCatcodes
    \catcode\the\count@=\the\catcode\count@\relax
  }%
\ifnum\count@<255 %
  \advance\count@ 1 %
\repeat

\def\RangeCatcodeInvalid#1#2{%
  \count@=#1\relax
  \loop
    \catcode\count@=15 %
  \ifnum\count@<#2\relax
    \advance\count@ 1 %
  \repeat
}
\def\RangeCatcodeCheck#1#2#3{%
  \count@=#1\relax
  \loop
    \ifnum#3=\catcode\count@
    \else
      \errmessage{%
        Character \the\count@\space
        with wrong catcode \the\catcode\count@\space
        instead of \number#3%
      }%
    \fi
  \ifnum\count@<#2\relax
    \advance\count@ 1 %
  \repeat
}
\def\space{ }
\expandafter\ifx\csname LoadCommand\endcsname\relax
  \def\LoadCommand{\input kvoptions.sty\relax}%
\fi
\def\Test{%
  \RangeCatcodeInvalid{0}{47}%
  \RangeCatcodeInvalid{58}{64}%
  \RangeCatcodeInvalid{91}{96}%
  \RangeCatcodeInvalid{123}{255}%
  \catcode`\@=12 %
  \catcode`\\=0 %
  \catcode`\%=14 %
  \LoadCommand
  \RangeCatcodeCheck{0}{36}{15}%
  \RangeCatcodeCheck{37}{37}{14}%
  \RangeCatcodeCheck{38}{47}{15}%
  \RangeCatcodeCheck{48}{57}{12}%
  \RangeCatcodeCheck{58}{63}{15}%
  \RangeCatcodeCheck{64}{64}{12}%
  \RangeCatcodeCheck{65}{90}{11}%
  \RangeCatcodeCheck{91}{91}{15}%
  \RangeCatcodeCheck{92}{92}{0}%
  \RangeCatcodeCheck{93}{96}{15}%
  \RangeCatcodeCheck{97}{122}{11}%
  \RangeCatcodeCheck{123}{255}{15}%
  \RestoreCatcodes
}
\Test
\csname @@end\endcsname
\end
%    \end{macrocode}
%    \begin{macrocode}
%</test1>
%    \end{macrocode}
%
%    \begin{macrocode}
%<*test2>
\NeedsTeXFormat{LaTeX2e}
\makeatletter
\catcode`\@=11 %
\def\RestoreCatcodes{}
\count@=0 %
\loop
  \edef\RestoreCatcodes{%
    \RestoreCatcodes
    \catcode\the\count@=\the\catcode\count@\relax
  }%
\ifnum\count@<255 %
  \advance\count@\@ne
\repeat

\def\RangeCatcodeInvalid#1#2{%
  \count@=#1\relax
  \loop
    \catcode\count@=15 %
  \ifnum\count@<#2\relax
    \advance\count@\@ne
  \repeat
}
\def\Test#1{%
  \RangeCatcodeInvalid{0}{47}%
  \RangeCatcodeInvalid{58}{64}%
  \RangeCatcodeInvalid{91}{96}%
  \RangeCatcodeInvalid{123}{255}%
  \catcode`\@=12 %
  \catcode`\\=0 %
  \catcode`\{=1 %
  \catcode`\}=2 %
  \catcode`\#=6 %
  \catcode`\[=12 %
  \catcode`\]=12 %
  \catcode`\%=14 %
  \catcode`\ =10 %
  \catcode13=5 %
  #1\relax
  \RestoreCatcodes
}
\Test{\RequirePackage{kvoptions-patch}}%
\Test{\RequirePackage{kvoptions}}%
\csname @@end\endcsname
%</test2>
%    \end{macrocode}
%
%    \begin{macrocode}
%<*test3>
\NeedsTeXFormat{LaTeX2e}
\makeatletter
\RequirePackage{kvoptions}[2016/05/16]
\def\msg#{\immediate\write16}
\define@key{testfamily}{testkey}{%
  \msg{[testfamily/testkey/#1]}%
}
\define@key{testfamily}{testdefaultkey}[testdefault]{%
  \msg{[testfamily/testdefaultkey/#1]}%
}
\AddToKeyvalOption{testfamily}{testkey}{%
  \msg{[addition/#1]}%
}
\AddToKeyvalOption{testfamily}{testdefaultkey}{%
  \msg{[addition/#1]}%
}
\setkeys{testfamily}{%
  testkey=testA,%
  testdefaultkey=testB,%
  testdefaultkey,%
}
\SetupKeyvalOptions{%
  family=testfamily%
}
\AddToKeyvalOption*{testkey}{%
  \msg{[star addition/#1]}%
}
\AddToKeyvalOption*{testdefaultkey}{%
  \msg{[star addition/#1]}%
}
\setkeys{testfamily}{%
  testkey=testA,%
  testdefaultkey=testB,%
  testdefaultkey,%
}
\@@end
%</test3>
%    \end{macrocode}
%
%    \begin{macrocode}
%<*test4pkg>
\NeedsTeXFormat{LaTeX2e}
\ProvidesPackage{kvoptions-test4}[2016/05/16 package for testing]
\RequirePackage{kvoptions}[2016/05/16]
\SetupKeyvalOptions{%
  family=FOO,%
  prefix=foo,%
  setkeys=\kvsetkeys,%
}
\DeclareStringOption{str}
\define@key{FOO}{set}{%
  \setkeys{BAR}{strbar={#1}}%
}
\define@key{BAR}{strbar}{%
  \def\foostr{[BAR:#1]}%
}
\ProcessKeyvalOptions*
%</test4pkg>
%    \end{macrocode}
%    \begin{macrocode}
%<*test4>
\NeedsTeXFormat{LaTeX2e}
\ProvidesFile{kvoptions-test4.tex}[2016/05/16 test file]
\RequirePackage[%
  str=A,set=B,str=C,%
]{kvoptions-test4}[2016/05/16]
\def\TestExpected{C}
\ifx\foostr\TestExpected
  \typeout{* Test ok.}%
\else
  \typeout{* Result: [\foostr]}%
  \typeout{* Expected: [\TestExpected]}%
  \errmessage{Test failed!}%
\fi
\csname @@end\endcsname\end
%</test4>
%    \end{macrocode}
%
% \section{Installation}
%
% \subsection{Download}
%
% \paragraph{Package.} This package is available on
% CTAN\footnote{\CTANpkg{kvoptions}}:
% \begin{description}
% \item[\CTAN{macros/latex/contrib/oberdiek/kvoptions.dtx}] The source file.
% \item[\CTAN{macros/latex/contrib/oberdiek/kvoptions.pdf}] Documentation.
% \end{description}
%
%
% \paragraph{Bundle.} All the packages of the bundle `oberdiek'
% are also available in a TDS compliant ZIP archive. There
% the packages are already unpacked and the documentation files
% are generated. The files and directories obey the TDS standard.
% \begin{description}
% \item[\CTANinstall{install/macros/latex/contrib/oberdiek.tds.zip}]
% \end{description}
% \emph{TDS} refers to the standard ``A Directory Structure
% for \TeX\ Files'' (\CTAN{tds/tds.pdf}). Directories
% with \xfile{texmf} in their name are usually organized this way.
%
% \subsection{Bundle installation}
%
% \paragraph{Unpacking.} Unpack the \xfile{oberdiek.tds.zip} in the
% TDS tree (also known as \xfile{texmf} tree) of your choice.
% Example (linux):
% \begin{quote}
%   |unzip oberdiek.tds.zip -d ~/texmf|
% \end{quote}
%
% \paragraph{Script installation.}
% Check the directory \xfile{TDS:scripts/oberdiek/} for
% scripts that need further installation steps.
%
% \subsection{Package installation}
%
% \paragraph{Unpacking.} The \xfile{.dtx} file is a self-extracting
% \docstrip\ archive. The files are extracted by running the
% \xfile{.dtx} through \plainTeX:
% \begin{quote}
%   \verb|tex kvoptions.dtx|
% \end{quote}
%
% \paragraph{TDS.} Now the different files must be moved into
% the different directories in your installation TDS tree
% (also known as \xfile{texmf} tree):
% \begin{quote}
% \def\t{^^A
% \begin{tabular}{@{}>{\ttfamily}l@{ $\rightarrow$ }>{\ttfamily}l@{}}
%   kvoptions.sty & tex/latex/oberdiek/kvoptions.sty\\
%   kvoptions-patch.sty & tex/latex/oberdiek/kvoptions-patch.sty\\
%   kvoptions.pdf & doc/latex/oberdiek/kvoptions.pdf\\
%   example-mycolorsetup.sty & doc/latex/oberdiek/example-mycolorsetup.sty\\
%   test/kvoptions-test1.tex & doc/latex/oberdiek/test/kvoptions-test1.tex\\
%   test/kvoptions-test2.tex & doc/latex/oberdiek/test/kvoptions-test2.tex\\
%   test/kvoptions-test3.tex & doc/latex/oberdiek/test/kvoptions-test3.tex\\
%   test/kvoptions-test4.tex & doc/latex/oberdiek/test/kvoptions-test4.tex\\
%   test/kvoptions-test4.sty & doc/latex/oberdiek/test/kvoptions-test4.sty\\
%   kvoptions.dtx & source/latex/oberdiek/kvoptions.dtx\\
% \end{tabular}^^A
% }^^A
% \sbox0{\t}^^A
% \ifdim\wd0>\linewidth
%   \begingroup
%     \advance\linewidth by\leftmargin
%     \advance\linewidth by\rightmargin
%   \edef\x{\endgroup
%     \def\noexpand\lw{\the\linewidth}^^A
%   }\x
%   \def\lwbox{^^A
%     \leavevmode
%     \hbox to \linewidth{^^A
%       \kern-\leftmargin\relax
%       \hss
%       \usebox0
%       \hss
%       \kern-\rightmargin\relax
%     }^^A
%   }^^A
%   \ifdim\wd0>\lw
%     \sbox0{\small\t}^^A
%     \ifdim\wd0>\linewidth
%       \ifdim\wd0>\lw
%         \sbox0{\footnotesize\t}^^A
%         \ifdim\wd0>\linewidth
%           \ifdim\wd0>\lw
%             \sbox0{\scriptsize\t}^^A
%             \ifdim\wd0>\linewidth
%               \ifdim\wd0>\lw
%                 \sbox0{\tiny\t}^^A
%                 \ifdim\wd0>\linewidth
%                   \lwbox
%                 \else
%                   \usebox0
%                 \fi
%               \else
%                 \lwbox
%               \fi
%             \else
%               \usebox0
%             \fi
%           \else
%             \lwbox
%           \fi
%         \else
%           \usebox0
%         \fi
%       \else
%         \lwbox
%       \fi
%     \else
%       \usebox0
%     \fi
%   \else
%     \lwbox
%   \fi
% \else
%   \usebox0
% \fi
% \end{quote}
% If you have a \xfile{docstrip.cfg} that configures and enables \docstrip's
% TDS installing feature, then some files can already be in the right
% place, see the documentation of \docstrip.
%
% \subsection{Refresh file name databases}
%
% If your \TeX~distribution
% (\TeX\,Live, \mikTeX, \dots) relies on file name databases, you must refresh
% these. For example, \TeX\,Live\ users run \verb|texhash| or
% \verb|mktexlsr|.
%
% \subsection{Some details for the interested}
%
% \paragraph{Unpacking with \LaTeX.}
% The \xfile{.dtx} chooses its action depending on the format:
% \begin{description}
% \item[\plainTeX:] Run \docstrip\ and extract the files.
% \item[\LaTeX:] Generate the documentation.
% \end{description}
% If you insist on using \LaTeX\ for \docstrip\ (really,
% \docstrip\ does not need \LaTeX), then inform the autodetect routine
% about your intention:
% \begin{quote}
%   \verb|latex \let\install=y\input{kvoptions.dtx}|
% \end{quote}
% Do not forget to quote the argument according to the demands
% of your shell.
%
% \paragraph{Generating the documentation.}
% You can use both the \xfile{.dtx} or the \xfile{.drv} to generate
% the documentation. The process can be configured by the
% configuration file \xfile{ltxdoc.cfg}. For instance, put this
% line into this file, if you want to have A4 as paper format:
% \begin{quote}
%   \verb|\PassOptionsToClass{a4paper}{article}|
% \end{quote}
% An example follows how to generate the
% documentation with pdf\LaTeX:
% \begin{quote}
%\begin{verbatim}
%pdflatex kvoptions.dtx
%makeindex -s gind.ist kvoptions.idx
%pdflatex kvoptions.dtx
%makeindex -s gind.ist kvoptions.idx
%pdflatex kvoptions.dtx
%\end{verbatim}
% \end{quote}
%
% \begin{thebibliography}{99}
% \bibitem{tb94wright}
%   A guide to key-value methods, Joseph Wright, second draft for
%   \href{https://www.tug.org/tugboat}{TUGBoat}, 2009-03-17.
%   \url{https://www.texdev.net/uploads/2009/03/keyval.pdf}
% \bibpackage{ifthen}{David Carlisle}{2001/05/26}
%   {\CTANpkg{ifthen}}
% \bibpackage{helvet}{Sebastian Rahtz, Walter Schmidt}{2004/01/26}
%   {\CTANpkg{psfonts}}
% \bibpackage{hyperref}{Sebastian Rahtz, Heiko Oberdiek}{2006/02/12}
%   {\CTANpkg{hyperref}}
% \bibpackage{keyval}{David Carlisle}{1999/03/16}
%   {\CTANpkg{keyval}}
% \bibpackage{multicol}{Frank Mittelbach}{2004/02/14}
%   {\CTANpkg{multicol}}
% \bibpackage{tabularx}{David Carlisle}{1999/01/07}
%   {\CTANpkg{tabularx}}
% \bibpackage{tracefnt}{Frank Mittelbach, Rainer Sch\"opf}{1997/05/29}
%   {\CTANpkg{latex-base}}
% \bibpackage{xkeyval}{Hendri Adriaens}{2005/05/07}
%   {\CTANpkg{xkeyval}}
% \bibitem{clsguide}
%   The \LaTeX3 Project, \textit{\LaTeXe\ for class and package writers},
%   2003/12/09.
%   \CTANpkg{clsguide}
%
% \end{thebibliography}
%
% \begin{History}
%   \begin{Version}{0000/00/00 v0.0}
%   \item
%     Probably David Carlisle's code in \xpackage{hyperref}
%     was the start.
%   \end{Version}
%   \begin{Version}{2004/02/22 v1.0}
%   \item
%     The first version was never published. It also
%     has offered a patch to get rid of \LaTeX's option
%     expansion.
%   \end{Version}
%   \begin{Version}{2006/02/16 v2.0}
%   \item
%     Now the package is redesigned with an easier
%     user interface.
%   \item
%     \cs{ProcessKeyvalOptions} remains the central service, inherited
%     from \xpackage{hyperref}'s \cs{ProcessOptionsWithKV}.
%     Now the use inside classes is also supported.
%   \item
%     Provides help macros for boolean and simple string options.
%   \item
%     Fixes for the patch of \LaTeX{}. The patch is only enabled,
%     if the user requests it.
%   \end{Version}
%   \begin{Version}{2006/02/20 v2.1}
%   \item
%     Unused option list is sanitized to prevent problems
%     with other packages that uses own processing methods
%     for key value options. Disadvantage: the unused global
%     option detection is weakened.
%   \item
%     New option type by \cs{DeclareVoidOption} for options without
%     value.
%   \item
%     Default rule by \cs{DeclareDefaultOption}.
%   \item
%     Dynamic options: \cs{DisableKeyvalOption}.
%   \end{Version}
%   \begin{Version}{2006/06/01 v2.2}
%   \item
%     Fixes for option \xoption{patch}.
%   \end{Version}
%   \begin{Version}{2006/08/17 v2.3}
%   \item
%     \cs{DeclareBooleanOption} renamed to \cs{DeclareBoolOption}
%     to avoid a name clash with package \cs{ifoption}.
%   \end{Version}
%   \begin{Version}{2006/08/22 v2.4}
%   \item
%     Option \xoption{patch}: \cs{ExecuteOptions} does not
%     change the meaning of macro \cs{CurrentOption} at all.
%   \end{Version}
%   \begin{Version}{2007/04/11 v2.5}
%   \item
%     Line ends sanitized.
%   \end{Version}
%   \begin{Version}{2007/05/06 v2.6}
%   \item
%     Uses package \xpackage{etexcmds}.
%   \end{Version}
%   \begin{Version}{2007/06/11 v2.7}
%   \item
%     The patch part fixes LaTeX bug latex/3965.
%   \end{Version}
%   \begin{Version}{2007/10/02 v2.8}
%   \item
%     Compatibility for \plainTeX\ added.
%   \item
%     Typos in documentation fixed (Axel Sommerfeldt).
%   \end{Version}
%   \begin{Version}{2007/10/11 v2.9}
%   \item
%     Bug fix for option \xoption{patch}.
%   \end{Version}
%   \begin{Version}{2007/10/18 v3.0}
%   \item
%     New package \xpackage{kvoptions-patch}.
%   \end{Version}
%   \begin{Version}{2009/04/10 v3.1}
%   \item
%     Space by line end removed in definition of internal macro.
%   \end{Version}
%   \begin{Version}{2009/07/17 v3.2}
%   \item
%     \cs{ProcessLocalKeyvalOptions} added.
%   \item
%     \cs{DisableKeyvalOption} with the \texttt{action=ignore} option
%     fixed (Joseph Wright).
%   \end{Version}
%   \begin{Version}{2009/07/21 v3.3}
%   \item
%     \cs{DeclareLocalOption}, \cs{DeclareLocalOptions} added.
%   \end{Version}
%   \begin{Version}{2009/08/13 v3.4}
%   \item
%     Documentation addition: recommendation for Joseph Wright's
%     review article.
%   \item
%     Documentation addition: local/global options.
%   \end{Version}
%   \begin{Version}{2009/12/04 v3.5}
%   \item
%     \cs{AddToKeyvalOption} added.
%   \end{Version}
%   \begin{Version}{2009/12/08 v3.6}
%   \item
%     Fix: If a default handler is configured, it is now also
%     called for classes.
%   \end{Version}
%   \begin{Version}{2010/02/22 v3.7}
%   \item
%     Missing space in error message added.
%   \end{Version}
%   \begin{Version}{2010/07/23 v3.8}
%   \item
%     Documenation for package \xpackage{kvoptions-patch} improved.
%     No code changes.
%   \end{Version}
%   \begin{Version}{2010/12/02 v3.9}
%   \item
%     Key \texttt{setkeys} added for \cs{SetupKeyvalOptions}.
%   \end{Version}
%   \begin{Version}{2010/12/23 v3.10}
%   \item
%     \cs{DeclareVoidOption} also parses the second parameter as
%     \hologo{TeX} argument to improve compatibility with
%     \cs{DeclareOption}.
%   \end{Version}
%   \begin{Version}{2011/06/30 v3.11}
%   \item
%     Fix because of design bug in package \xpackage{xkeyval} that
%     removes global options with equal signs.
%   \end{Version}
%   \begin{Version}{2016/05/16 v3.12}
%   \item
%     Documentation updates.
%   \end{Version}
% \end{History}
%
% \PrintIndex
%
% \Finale
\endinput
|
% \end{quote}
% Do not forget to quote the argument according to the demands
% of your shell.
%
% \paragraph{Generating the documentation.}
% You can use both the \xfile{.dtx} or the \xfile{.drv} to generate
% the documentation. The process can be configured by the
% configuration file \xfile{ltxdoc.cfg}. For instance, put this
% line into this file, if you want to have A4 as paper format:
% \begin{quote}
%   \verb|\PassOptionsToClass{a4paper}{article}|
% \end{quote}
% An example follows how to generate the
% documentation with pdf\LaTeX:
% \begin{quote}
%\begin{verbatim}
%pdflatex kvoptions.dtx
%makeindex -s gind.ist kvoptions.idx
%pdflatex kvoptions.dtx
%makeindex -s gind.ist kvoptions.idx
%pdflatex kvoptions.dtx
%\end{verbatim}
% \end{quote}
%
% \begin{thebibliography}{99}
% \bibitem{tb94wright}
%   A guide to key-value methods, Joseph Wright, second draft for
%   \href{https://www.tug.org/tugboat}{TUGBoat}, 2009-03-17.
%   \url{https://www.texdev.net/uploads/2009/03/keyval.pdf}
% \bibpackage{ifthen}{David Carlisle}{2001/05/26}
%   {\CTANpkg{ifthen}}
% \bibpackage{helvet}{Sebastian Rahtz, Walter Schmidt}{2004/01/26}
%   {\CTANpkg{psfonts}}
% \bibpackage{hyperref}{Sebastian Rahtz, Heiko Oberdiek}{2006/02/12}
%   {\CTANpkg{hyperref}}
% \bibpackage{keyval}{David Carlisle}{1999/03/16}
%   {\CTANpkg{keyval}}
% \bibpackage{multicol}{Frank Mittelbach}{2004/02/14}
%   {\CTANpkg{multicol}}
% \bibpackage{tabularx}{David Carlisle}{1999/01/07}
%   {\CTANpkg{tabularx}}
% \bibpackage{tracefnt}{Frank Mittelbach, Rainer Sch\"opf}{1997/05/29}
%   {\CTANpkg{latex-base}}
% \bibpackage{xkeyval}{Hendri Adriaens}{2005/05/07}
%   {\CTANpkg{xkeyval}}
% \bibitem{clsguide}
%   The \LaTeX3 Project, \textit{\LaTeXe\ for class and package writers},
%   2003/12/09.
%   \CTANpkg{clsguide}
%
% \end{thebibliography}
%
% \begin{History}
%   \begin{Version}{0000/00/00 v0.0}
%   \item
%     Probably David Carlisle's code in \xpackage{hyperref}
%     was the start.
%   \end{Version}
%   \begin{Version}{2004/02/22 v1.0}
%   \item
%     The first version was never published. It also
%     has offered a patch to get rid of \LaTeX's option
%     expansion.
%   \end{Version}
%   \begin{Version}{2006/02/16 v2.0}
%   \item
%     Now the package is redesigned with an easier
%     user interface.
%   \item
%     \cs{ProcessKeyvalOptions} remains the central service, inherited
%     from \xpackage{hyperref}'s \cs{ProcessOptionsWithKV}.
%     Now the use inside classes is also supported.
%   \item
%     Provides help macros for boolean and simple string options.
%   \item
%     Fixes for the patch of \LaTeX{}. The patch is only enabled,
%     if the user requests it.
%   \end{Version}
%   \begin{Version}{2006/02/20 v2.1}
%   \item
%     Unused option list is sanitized to prevent problems
%     with other packages that uses own processing methods
%     for key value options. Disadvantage: the unused global
%     option detection is weakened.
%   \item
%     New option type by \cs{DeclareVoidOption} for options without
%     value.
%   \item
%     Default rule by \cs{DeclareDefaultOption}.
%   \item
%     Dynamic options: \cs{DisableKeyvalOption}.
%   \end{Version}
%   \begin{Version}{2006/06/01 v2.2}
%   \item
%     Fixes for option \xoption{patch}.
%   \end{Version}
%   \begin{Version}{2006/08/17 v2.3}
%   \item
%     \cs{DeclareBooleanOption} renamed to \cs{DeclareBoolOption}
%     to avoid a name clash with package \cs{ifoption}.
%   \end{Version}
%   \begin{Version}{2006/08/22 v2.4}
%   \item
%     Option \xoption{patch}: \cs{ExecuteOptions} does not
%     change the meaning of macro \cs{CurrentOption} at all.
%   \end{Version}
%   \begin{Version}{2007/04/11 v2.5}
%   \item
%     Line ends sanitized.
%   \end{Version}
%   \begin{Version}{2007/05/06 v2.6}
%   \item
%     Uses package \xpackage{etexcmds}.
%   \end{Version}
%   \begin{Version}{2007/06/11 v2.7}
%   \item
%     The patch part fixes LaTeX bug latex/3965.
%   \end{Version}
%   \begin{Version}{2007/10/02 v2.8}
%   \item
%     Compatibility for \plainTeX\ added.
%   \item
%     Typos in documentation fixed (Axel Sommerfeldt).
%   \end{Version}
%   \begin{Version}{2007/10/11 v2.9}
%   \item
%     Bug fix for option \xoption{patch}.
%   \end{Version}
%   \begin{Version}{2007/10/18 v3.0}
%   \item
%     New package \xpackage{kvoptions-patch}.
%   \end{Version}
%   \begin{Version}{2009/04/10 v3.1}
%   \item
%     Space by line end removed in definition of internal macro.
%   \end{Version}
%   \begin{Version}{2009/07/17 v3.2}
%   \item
%     \cs{ProcessLocalKeyvalOptions} added.
%   \item
%     \cs{DisableKeyvalOption} with the \texttt{action=ignore} option
%     fixed (Joseph Wright).
%   \end{Version}
%   \begin{Version}{2009/07/21 v3.3}
%   \item
%     \cs{DeclareLocalOption}, \cs{DeclareLocalOptions} added.
%   \end{Version}
%   \begin{Version}{2009/08/13 v3.4}
%   \item
%     Documentation addition: recommendation for Joseph Wright's
%     review article.
%   \item
%     Documentation addition: local/global options.
%   \end{Version}
%   \begin{Version}{2009/12/04 v3.5}
%   \item
%     \cs{AddToKeyvalOption} added.
%   \end{Version}
%   \begin{Version}{2009/12/08 v3.6}
%   \item
%     Fix: If a default handler is configured, it is now also
%     called for classes.
%   \end{Version}
%   \begin{Version}{2010/02/22 v3.7}
%   \item
%     Missing space in error message added.
%   \end{Version}
%   \begin{Version}{2010/07/23 v3.8}
%   \item
%     Documenation for package \xpackage{kvoptions-patch} improved.
%     No code changes.
%   \end{Version}
%   \begin{Version}{2010/12/02 v3.9}
%   \item
%     Key \texttt{setkeys} added for \cs{SetupKeyvalOptions}.
%   \end{Version}
%   \begin{Version}{2010/12/23 v3.10}
%   \item
%     \cs{DeclareVoidOption} also parses the second parameter as
%     \hologo{TeX} argument to improve compatibility with
%     \cs{DeclareOption}.
%   \end{Version}
%   \begin{Version}{2011/06/30 v3.11}
%   \item
%     Fix because of design bug in package \xpackage{xkeyval} that
%     removes global options with equal signs.
%   \end{Version}
%   \begin{Version}{2016/05/16 v3.12}
%   \item
%     Documentation updates.
%   \end{Version}
% \end{History}
%
% \PrintIndex
%
% \Finale
\endinput
|
% \end{quote}
% Do not forget to quote the argument according to the demands
% of your shell.
%
% \paragraph{Generating the documentation.}
% You can use both the \xfile{.dtx} or the \xfile{.drv} to generate
% the documentation. The process can be configured by the
% configuration file \xfile{ltxdoc.cfg}. For instance, put this
% line into this file, if you want to have A4 as paper format:
% \begin{quote}
%   \verb|\PassOptionsToClass{a4paper}{article}|
% \end{quote}
% An example follows how to generate the
% documentation with pdf\LaTeX:
% \begin{quote}
%\begin{verbatim}
%pdflatex kvoptions.dtx
%makeindex -s gind.ist kvoptions.idx
%pdflatex kvoptions.dtx
%makeindex -s gind.ist kvoptions.idx
%pdflatex kvoptions.dtx
%\end{verbatim}
% \end{quote}
%
% \begin{thebibliography}{99}
% \bibitem{tb94wright}
%   A guide to key-value methods, Joseph Wright, second draft for
%   \href{https://www.tug.org/tugboat}{TUGBoat}, 2009-03-17.
%   \url{https://www.texdev.net/uploads/2009/03/keyval.pdf}
% \bibpackage{ifthen}{David Carlisle}{2001/05/26}
%   {\CTANpkg{ifthen}}
% \bibpackage{helvet}{Sebastian Rahtz, Walter Schmidt}{2004/01/26}
%   {\CTANpkg{psfonts}}
% \bibpackage{hyperref}{Sebastian Rahtz, Heiko Oberdiek}{2006/02/12}
%   {\CTANpkg{hyperref}}
% \bibpackage{keyval}{David Carlisle}{1999/03/16}
%   {\CTANpkg{keyval}}
% \bibpackage{multicol}{Frank Mittelbach}{2004/02/14}
%   {\CTANpkg{multicol}}
% \bibpackage{tabularx}{David Carlisle}{1999/01/07}
%   {\CTANpkg{tabularx}}
% \bibpackage{tracefnt}{Frank Mittelbach, Rainer Sch\"opf}{1997/05/29}
%   {\CTANpkg{latex-base}}
% \bibpackage{xkeyval}{Hendri Adriaens}{2005/05/07}
%   {\CTANpkg{xkeyval}}
% \bibitem{clsguide}
%   The \LaTeX3 Project, \textit{\LaTeXe\ for class and package writers},
%   2003/12/09.
%   \CTANpkg{clsguide}
%
% \end{thebibliography}
%
% \begin{History}
%   \begin{Version}{0000/00/00 v0.0}
%   \item
%     Probably David Carlisle's code in \xpackage{hyperref}
%     was the start.
%   \end{Version}
%   \begin{Version}{2004/02/22 v1.0}
%   \item
%     The first version was never published. It also
%     has offered a patch to get rid of \LaTeX's option
%     expansion.
%   \end{Version}
%   \begin{Version}{2006/02/16 v2.0}
%   \item
%     Now the package is redesigned with an easier
%     user interface.
%   \item
%     \cs{ProcessKeyvalOptions} remains the central service, inherited
%     from \xpackage{hyperref}'s \cs{ProcessOptionsWithKV}.
%     Now the use inside classes is also supported.
%   \item
%     Provides help macros for boolean and simple string options.
%   \item
%     Fixes for the patch of \LaTeX{}. The patch is only enabled,
%     if the user requests it.
%   \end{Version}
%   \begin{Version}{2006/02/20 v2.1}
%   \item
%     Unused option list is sanitized to prevent problems
%     with other packages that uses own processing methods
%     for key value options. Disadvantage: the unused global
%     option detection is weakened.
%   \item
%     New option type by \cs{DeclareVoidOption} for options without
%     value.
%   \item
%     Default rule by \cs{DeclareDefaultOption}.
%   \item
%     Dynamic options: \cs{DisableKeyvalOption}.
%   \end{Version}
%   \begin{Version}{2006/06/01 v2.2}
%   \item
%     Fixes for option \xoption{patch}.
%   \end{Version}
%   \begin{Version}{2006/08/17 v2.3}
%   \item
%     \cs{DeclareBooleanOption} renamed to \cs{DeclareBoolOption}
%     to avoid a name clash with package \cs{ifoption}.
%   \end{Version}
%   \begin{Version}{2006/08/22 v2.4}
%   \item
%     Option \xoption{patch}: \cs{ExecuteOptions} does not
%     change the meaning of macro \cs{CurrentOption} at all.
%   \end{Version}
%   \begin{Version}{2007/04/11 v2.5}
%   \item
%     Line ends sanitized.
%   \end{Version}
%   \begin{Version}{2007/05/06 v2.6}
%   \item
%     Uses package \xpackage{etexcmds}.
%   \end{Version}
%   \begin{Version}{2007/06/11 v2.7}
%   \item
%     The patch part fixes LaTeX bug latex/3965.
%   \end{Version}
%   \begin{Version}{2007/10/02 v2.8}
%   \item
%     Compatibility for \plainTeX\ added.
%   \item
%     Typos in documentation fixed (Axel Sommerfeldt).
%   \end{Version}
%   \begin{Version}{2007/10/11 v2.9}
%   \item
%     Bug fix for option \xoption{patch}.
%   \end{Version}
%   \begin{Version}{2007/10/18 v3.0}
%   \item
%     New package \xpackage{kvoptions-patch}.
%   \end{Version}
%   \begin{Version}{2009/04/10 v3.1}
%   \item
%     Space by line end removed in definition of internal macro.
%   \end{Version}
%   \begin{Version}{2009/07/17 v3.2}
%   \item
%     \cs{ProcessLocalKeyvalOptions} added.
%   \item
%     \cs{DisableKeyvalOption} with the \texttt{action=ignore} option
%     fixed (Joseph Wright).
%   \end{Version}
%   \begin{Version}{2009/07/21 v3.3}
%   \item
%     \cs{DeclareLocalOption}, \cs{DeclareLocalOptions} added.
%   \end{Version}
%   \begin{Version}{2009/08/13 v3.4}
%   \item
%     Documentation addition: recommendation for Joseph Wright's
%     review article.
%   \item
%     Documentation addition: local/global options.
%   \end{Version}
%   \begin{Version}{2009/12/04 v3.5}
%   \item
%     \cs{AddToKeyvalOption} added.
%   \end{Version}
%   \begin{Version}{2009/12/08 v3.6}
%   \item
%     Fix: If a default handler is configured, it is now also
%     called for classes.
%   \end{Version}
%   \begin{Version}{2010/02/22 v3.7}
%   \item
%     Missing space in error message added.
%   \end{Version}
%   \begin{Version}{2010/07/23 v3.8}
%   \item
%     Documenation for package \xpackage{kvoptions-patch} improved.
%     No code changes.
%   \end{Version}
%   \begin{Version}{2010/12/02 v3.9}
%   \item
%     Key \texttt{setkeys} added for \cs{SetupKeyvalOptions}.
%   \end{Version}
%   \begin{Version}{2010/12/23 v3.10}
%   \item
%     \cs{DeclareVoidOption} also parses the second parameter as
%     \hologo{TeX} argument to improve compatibility with
%     \cs{DeclareOption}.
%   \end{Version}
%   \begin{Version}{2011/06/30 v3.11}
%   \item
%     Fix because of design bug in package \xpackage{xkeyval} that
%     removes global options with equal signs.
%   \end{Version}
%   \begin{Version}{2016/05/16 v3.12}
%   \item
%     Documentation updates.
%   \end{Version}
% \end{History}
%
% \PrintIndex
%
% \Finale
\endinput
|
% \end{quote}
% Do not forget to quote the argument according to the demands
% of your shell.
%
% \paragraph{Generating the documentation.}
% You can use both the \xfile{.dtx} or the \xfile{.drv} to generate
% the documentation. The process can be configured by the
% configuration file \xfile{ltxdoc.cfg}. For instance, put this
% line into this file, if you want to have A4 as paper format:
% \begin{quote}
%   \verb|\PassOptionsToClass{a4paper}{article}|
% \end{quote}
% An example follows how to generate the
% documentation with pdf\LaTeX:
% \begin{quote}
%\begin{verbatim}
%pdflatex kvoptions.dtx
%makeindex -s gind.ist kvoptions.idx
%pdflatex kvoptions.dtx
%makeindex -s gind.ist kvoptions.idx
%pdflatex kvoptions.dtx
%\end{verbatim}
% \end{quote}
%
% \begin{thebibliography}{99}
% \bibitem{tb94wright}
%   A guide to key-value methods, Joseph Wright, second draft for
%   \href{https://www.tug.org/tugboat}{TUGBoat}, 2009-03-17.
%   \url{https://www.texdev.net/uploads/2009/03/keyval.pdf}
% \bibpackage{ifthen}{David Carlisle}{2001/05/26}
%   {\CTANpkg{ifthen}}
% \bibpackage{helvet}{Sebastian Rahtz, Walter Schmidt}{2004/01/26}
%   {\CTANpkg{psfonts}}
% \bibpackage{hyperref}{Sebastian Rahtz, Heiko Oberdiek}{2006/02/12}
%   {\CTANpkg{hyperref}}
% \bibpackage{keyval}{David Carlisle}{1999/03/16}
%   {\CTANpkg{keyval}}
% \bibpackage{multicol}{Frank Mittelbach}{2004/02/14}
%   {\CTANpkg{multicol}}
% \bibpackage{tabularx}{David Carlisle}{1999/01/07}
%   {\CTANpkg{tabularx}}
% \bibpackage{tracefnt}{Frank Mittelbach, Rainer Sch\"opf}{1997/05/29}
%   {\CTANpkg{ltfsstrc}}
% \bibpackage{xkeyval}{Hendri Adriaens}{2005/05/07}
%   {\CTANpkg{xkeyval}}
% \bibitem{clsguide}
%   The \LaTeX3 Project, \textit{\LaTeXe\ for class and package writers},
%   2003/12/09.
%   \CTAN{macros/latex/doc/clsguide.pdf}
%
% \end{thebibliography}
%
% \begin{History}
%   \begin{Version}{0000/00/00 v0.0}
%   \item
%     Probably David Carlisle's code in \xpackage{hyperref}
%     was the start.
%   \end{Version}
%   \begin{Version}{2004/02/22 v1.0}
%   \item
%     The first version was never published. It also
%     has offered a patch to get rid of \LaTeX's option
%     expansion.
%   \end{Version}
%   \begin{Version}{2006/02/16 v2.0}
%   \item
%     Now the package is redesigned with an easier
%     user interface.
%   \item
%     \cs{ProcessKeyvalOptions} remains the central service, inherited
%     from \xpackage{hyperref}'s \cs{ProcessOptionsWithKV}.
%     Now the use inside classes is also supported.
%   \item
%     Provides help macros for boolean and simple string options.
%   \item
%     Fixes for the patch of \LaTeX{}. The patch is only enabled,
%     if the user requests it.
%   \end{Version}
%   \begin{Version}{2006/02/20 v2.1}
%   \item
%     Unused option list is sanitized to prevent problems
%     with other packages that uses own processing methods
%     for key value options. Disadvantage: the unused global
%     option detection is weakened.
%   \item
%     New option type by \cs{DeclareVoidOption} for options without
%     value.
%   \item
%     Default rule by \cs{DeclareDefaultOption}.
%   \item
%     Dynamic options: \cs{DisableKeyvalOption}.
%   \end{Version}
%   \begin{Version}{2006/06/01 v2.2}
%   \item
%     Fixes for option \xoption{patch}.
%   \end{Version}
%   \begin{Version}{2006/08/17 v2.3}
%   \item
%     \cs{DeclareBooleanOption} renamed to \cs{DeclareBoolOption}
%     to avoid a name clash with package \cs{ifoption}.
%   \end{Version}
%   \begin{Version}{2006/08/22 v2.4}
%   \item
%     Option \xoption{patch}: \cs{ExecuteOptions} does not
%     change the meaning of macro \cs{CurrentOption} at all.
%   \end{Version}
%   \begin{Version}{2007/04/11 v2.5}
%   \item
%     Line ends sanitized.
%   \end{Version}
%   \begin{Version}{2007/05/06 v2.6}
%   \item
%     Uses package \xpackage{etexcmds}.
%   \end{Version}
%   \begin{Version}{2007/06/11 v2.7}
%   \item
%     The patch part fixes LaTeX bug latex/3965.
%   \end{Version}
%   \begin{Version}{2007/10/02 v2.8}
%   \item
%     Compatibility for \plainTeX\ added.
%   \item
%     Typos in documentation fixed (Axel Sommerfeldt).
%   \end{Version}
%   \begin{Version}{2007/10/11 v2.9}
%   \item
%     Bug fix for option \xoption{patch}.
%   \end{Version}
%   \begin{Version}{2007/10/18 v3.0}
%   \item
%     New package \xpackage{kvoptions-patch}.
%   \end{Version}
%   \begin{Version}{2009/04/10 v3.1}
%   \item
%     Space by line end removed in definition of internal macro.
%   \end{Version}
%   \begin{Version}{2009/07/17 v3.2}
%   \item
%     \cs{ProcessLocalKeyvalOptions} added.
%   \item
%     \cs{DisableKeyvalOption} with the \texttt{action=ignore} option
%     fixed (Joseph Wright).
%   \end{Version}
%   \begin{Version}{2009/07/21 v3.3}
%   \item
%     \cs{DeclareLocalOption}, \cs{DeclareLocalOptions} added.
%   \end{Version}
%   \begin{Version}{2009/08/13 v3.4}
%   \item
%     Documentation addition: recommendation for Joseph Wright's
%     review article.
%   \item
%     Documentation addition: local/global options.
%   \end{Version}
%   \begin{Version}{2009/12/04 v3.5}
%   \item
%     \cs{AddToKeyvalOption} added.
%   \end{Version}
%   \begin{Version}{2009/12/08 v3.6}
%   \item
%     Fix: If a default handler is configured, it is now also
%     called for classes.
%   \end{Version}
%   \begin{Version}{2010/02/22 v3.7}
%   \item
%     Missing space in error message added.
%   \end{Version}
%   \begin{Version}{2010/07/23 v3.8}
%   \item
%     Documenation for package \xpackage{kvoptions-patch} improved.
%     No code changes.
%   \end{Version}
%   \begin{Version}{2010/12/02 v3.9}
%   \item
%     Key \texttt{setkeys} added for \cs{SetupKeyvalOptions}.
%   \end{Version}
%   \begin{Version}{2010/12/23 v3.10}
%   \item
%     \cs{DeclareVoidOption} also parses the second parameter as
%     \hologo{TeX} argument to improve compatibility with
%     \cs{DeclareOption}.
%   \end{Version}
%   \begin{Version}{2011/06/30 v3.11}
%   \item
%     Fix because of design bug in package \xpackage{xkeyval} that
%     removes global options with equal signs.
%   \end{Version}
%   \begin{Version}{2016/05/16 v3.12}
%   \item
%     Documentation updates.
%   \end{Version}
% \end{History}
%
% \PrintIndex
%
% \Finale
\endinput
