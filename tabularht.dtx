% \iffalse meta-comment
%
% File: tabularht.dtx
% Version: 2016/05/16 v2.6
% Info: Tabular with height specified
%
% Copyright (C) 2005-2007 by
%    Heiko Oberdiek <heiko.oberdiek at googlemail.com>
%    2016
%    https://github.com/ho-tex/oberdiek/issues
%
% This work may be distributed and/or modified under the
% conditions of the LaTeX Project Public License, either
% version 1.3c of this license or (at your option) any later
% version. This version of this license is in
%    https://www.latex-project.org/lppl/lppl-1-3c.txt
% and the latest version of this license is in
%    https://www.latex-project.org/lppl.txt
% and version 1.3 or later is part of all distributions of
% LaTeX version 2005/12/01 or later.
%
% This work has the LPPL maintenance status "maintained".
%
% The Current Maintainers of this work are
% Heiko Oberdiek and the Oberdiek Package Support Group
% https://github.com/ho-tex/oberdiek/issues
%
% This work consists of the main source file tabularht.dtx
% and the derived files
%    tabularht.sty, tabularht.pdf, tabularht.ins, tabularht.drv,
%    tabularht-example1.tex, tabularht-example2.tex.
%
% Distribution:
%    CTAN:macros/latex/contrib/oberdiek/tabularht.dtx
%    CTAN:macros/latex/contrib/oberdiek/tabularht.pdf
%
% Unpacking:
%    (a) If tabularht.ins is present:
%           tex tabularht.ins
%    (b) Without tabularht.ins:
%           tex tabularht.dtx
%    (c) If you insist on using LaTeX
%           latex \let\install=y% \iffalse meta-comment
%
% File: tabularht.dtx
% Version: 2016/05/16 v2.6
% Info: Tabular with height specified
%
% Copyright (C) 2005-2007 by
%    Heiko Oberdiek <heiko.oberdiek at googlemail.com>
%    2016
%    https://github.com/ho-tex/oberdiek/issues
%
% This work may be distributed and/or modified under the
% conditions of the LaTeX Project Public License, either
% version 1.3c of this license or (at your option) any later
% version. This version of this license is in
%    https://www.latex-project.org/lppl/lppl-1-3c.txt
% and the latest version of this license is in
%    https://www.latex-project.org/lppl.txt
% and version 1.3 or later is part of all distributions of
% LaTeX version 2005/12/01 or later.
%
% This work has the LPPL maintenance status "maintained".
%
% The Current Maintainers of this work are
% Heiko Oberdiek and the Oberdiek Package Support Group
% https://github.com/ho-tex/oberdiek/issues
%
% This work consists of the main source file tabularht.dtx
% and the derived files
%    tabularht.sty, tabularht.pdf, tabularht.ins, tabularht.drv,
%    tabularht-example1.tex, tabularht-example2.tex.
%
% Distribution:
%    CTAN:macros/latex/contrib/oberdiek/tabularht.dtx
%    CTAN:macros/latex/contrib/oberdiek/tabularht.pdf
%
% Unpacking:
%    (a) If tabularht.ins is present:
%           tex tabularht.ins
%    (b) Without tabularht.ins:
%           tex tabularht.dtx
%    (c) If you insist on using LaTeX
%           latex \let\install=y% \iffalse meta-comment
%
% File: tabularht.dtx
% Version: 2016/05/16 v2.6
% Info: Tabular with height specified
%
% Copyright (C) 2005-2007 by
%    Heiko Oberdiek <heiko.oberdiek at googlemail.com>
%    2016
%    https://github.com/ho-tex/oberdiek/issues
%
% This work may be distributed and/or modified under the
% conditions of the LaTeX Project Public License, either
% version 1.3c of this license or (at your option) any later
% version. This version of this license is in
%    https://www.latex-project.org/lppl/lppl-1-3c.txt
% and the latest version of this license is in
%    https://www.latex-project.org/lppl.txt
% and version 1.3 or later is part of all distributions of
% LaTeX version 2005/12/01 or later.
%
% This work has the LPPL maintenance status "maintained".
%
% The Current Maintainers of this work are
% Heiko Oberdiek and the Oberdiek Package Support Group
% https://github.com/ho-tex/oberdiek/issues
%
% This work consists of the main source file tabularht.dtx
% and the derived files
%    tabularht.sty, tabularht.pdf, tabularht.ins, tabularht.drv,
%    tabularht-example1.tex, tabularht-example2.tex.
%
% Distribution:
%    CTAN:macros/latex/contrib/oberdiek/tabularht.dtx
%    CTAN:macros/latex/contrib/oberdiek/tabularht.pdf
%
% Unpacking:
%    (a) If tabularht.ins is present:
%           tex tabularht.ins
%    (b) Without tabularht.ins:
%           tex tabularht.dtx
%    (c) If you insist on using LaTeX
%           latex \let\install=y% \iffalse meta-comment
%
% File: tabularht.dtx
% Version: 2016/05/16 v2.6
% Info: Tabular with height specified
%
% Copyright (C) 2005-2007 by
%    Heiko Oberdiek <heiko.oberdiek at googlemail.com>
%    2016
%    https://github.com/ho-tex/oberdiek/issues
%
% This work may be distributed and/or modified under the
% conditions of the LaTeX Project Public License, either
% version 1.3c of this license or (at your option) any later
% version. This version of this license is in
%    https://www.latex-project.org/lppl/lppl-1-3c.txt
% and the latest version of this license is in
%    https://www.latex-project.org/lppl.txt
% and version 1.3 or later is part of all distributions of
% LaTeX version 2005/12/01 or later.
%
% This work has the LPPL maintenance status "maintained".
%
% The Current Maintainers of this work are
% Heiko Oberdiek and the Oberdiek Package Support Group
% https://github.com/ho-tex/oberdiek/issues
%
% This work consists of the main source file tabularht.dtx
% and the derived files
%    tabularht.sty, tabularht.pdf, tabularht.ins, tabularht.drv,
%    tabularht-example1.tex, tabularht-example2.tex.
%
% Distribution:
%    CTAN:macros/latex/contrib/oberdiek/tabularht.dtx
%    CTAN:macros/latex/contrib/oberdiek/tabularht.pdf
%
% Unpacking:
%    (a) If tabularht.ins is present:
%           tex tabularht.ins
%    (b) Without tabularht.ins:
%           tex tabularht.dtx
%    (c) If you insist on using LaTeX
%           latex \let\install=y\input{tabularht.dtx}
%        (quote the arguments according to the demands of your shell)
%
% Documentation:
%    (a) If tabularht.drv is present:
%           latex tabularht.drv
%    (b) Without tabularht.drv:
%           latex tabularht.dtx; ...
%    The class ltxdoc loads the configuration file ltxdoc.cfg
%    if available. Here you can specify further options, e.g.
%    use A4 as paper format:
%       \PassOptionsToClass{a4paper}{article}
%
%    Programm calls to get the documentation (example):
%       pdflatex tabularht.dtx
%       makeindex -s gind.ist tabularht.idx
%       pdflatex tabularht.dtx
%       makeindex -s gind.ist tabularht.idx
%       pdflatex tabularht.dtx
%
% Installation:
%    TDS:tex/latex/oberdiek/tabularht.sty
%    TDS:doc/latex/oberdiek/tabularht.pdf
%    TDS:doc/latex/oberdiek/tabularht-example1.tex
%    TDS:doc/latex/oberdiek/tabularht-example2.tex
%    TDS:source/latex/oberdiek/tabularht.dtx
%
%<*ignore>
\begingroup
  \catcode123=1 %
  \catcode125=2 %
  \def\x{LaTeX2e}%
\expandafter\endgroup
\ifcase 0\ifx\install y1\fi\expandafter
         \ifx\csname processbatchFile\endcsname\relax\else1\fi
         \ifx\fmtname\x\else 1\fi\relax
\else\csname fi\endcsname
%</ignore>
%<*install>
\input docstrip.tex
\Msg{************************************************************************}
\Msg{* Installation}
\Msg{* Package: tabularht 2016/05/16 v2.6 Tabular with height specified (HO)}
\Msg{************************************************************************}

\keepsilent
\askforoverwritefalse

\let\MetaPrefix\relax
\preamble

This is a generated file.

Project: tabularht
Version: 2016/05/16 v2.6

Copyright (C) 2005-2007 by
   Heiko Oberdiek <heiko.oberdiek at googlemail.com>

This work may be distributed and/or modified under the
conditions of the LaTeX Project Public License, either
version 1.3c of this license or (at your option) any later
version. This version of this license is in
   https://www.latex-project.org/lppl/lppl-1-3c.txt
and the latest version of this license is in
   https://www.latex-project.org/lppl.txt
and version 1.3 or later is part of all distributions of
LaTeX version 2005/12/01 or later.

This work has the LPPL maintenance status "maintained".

The Current Maintainers of this work are
Heiko Oberdiek and the Oberdiek Package Support Group
https://github.com/ho-tex/oberdiek/issues


This work consists of the main source file tabularht.dtx
and the derived files
   tabularht.sty, tabularht.pdf, tabularht.ins, tabularht.drv,
   tabularht-example1.tex, tabularht-example2.tex.

\endpreamble
\let\MetaPrefix\DoubleperCent

\generate{%
  \file{tabularht.ins}{\from{tabularht.dtx}{install}}%
  \file{tabularht.drv}{\from{tabularht.dtx}{driver}}%
  \usedir{tex/latex/oberdiek}%
  \file{tabularht.sty}{\from{tabularht.dtx}{package}}%
  \usedir{doc/latex/oberdiek}%
  \file{tabularht-example1.tex}{\from{tabularht.dtx}{example1}}%
  \file{tabularht-example2.tex}{\from{tabularht.dtx}{example2}}%
  \nopreamble
  \nopostamble
%  \usedir{source/latex/oberdiek/catalogue}%
%  \file{tabularht.xml}{\from{tabularht.dtx}{catalogue}}%
}

\catcode32=13\relax% active space
\let =\space%
\Msg{************************************************************************}
\Msg{*}
\Msg{* To finish the installation you have to move the following}
\Msg{* file into a directory searched by TeX:}
\Msg{*}
\Msg{*     tabularht.sty}
\Msg{*}
\Msg{* To produce the documentation run the file `tabularht.drv'}
\Msg{* through LaTeX.}
\Msg{*}
\Msg{* Happy TeXing!}
\Msg{*}
\Msg{************************************************************************}

\endbatchfile
%</install>
%<*ignore>
\fi
%</ignore>
%<*driver>
\NeedsTeXFormat{LaTeX2e}
\ProvidesFile{tabularht.drv}%
  [2016/05/16 v2.6 Tabular with height specified (HO)]%
\documentclass{ltxdoc}
\usepackage{holtxdoc}[2011/11/22]
\begin{document}
  \DocInput{tabularht.dtx}%
\end{document}
%</driver>
% \fi
%
%
% \CharacterTable
%  {Upper-case    \A\B\C\D\E\F\G\H\I\J\K\L\M\N\O\P\Q\R\S\T\U\V\W\X\Y\Z
%   Lower-case    \a\b\c\d\e\f\g\h\i\j\k\l\m\n\o\p\q\r\s\t\u\v\w\x\y\z
%   Digits        \0\1\2\3\4\5\6\7\8\9
%   Exclamation   \!     Double quote  \"     Hash (number) \#
%   Dollar        \$     Percent       \%     Ampersand     \&
%   Acute accent  \'     Left paren    \(     Right paren   \)
%   Asterisk      \*     Plus          \+     Comma         \,
%   Minus         \-     Point         \.     Solidus       \/
%   Colon         \:     Semicolon     \;     Less than     \<
%   Equals        \=     Greater than  \>     Question mark \?
%   Commercial at \@     Left bracket  \[     Backslash     \\
%   Right bracket \]     Circumflex    \^     Underscore    \_
%   Grave accent  \`     Left brace    \{     Vertical bar  \|
%   Right brace   \}     Tilde         \~}
%
% \GetFileInfo{tabularht.drv}
%
% \title{The \xpackage{tabularht} package}
% \date{2016/05/16 v2.6}
% \author{Heiko Oberdiek\thanks
% {Please report any issues at \url{https://github.com/ho-tex/oberdiek/issues}}}
%
% \maketitle
%
% \begin{abstract}
% This package defines some environments that adds
% a height specification to tabular and array.
% \end{abstract}
%
% \tableofcontents
%
% \section{Usage}
%
% \begin{quote}
%   |\usepackage{tabularht}|
% \end{quote}
% The package provides the following environments
% that extend the tabular/array environment by
% a height specification as first argument:
% \begin{itemize}
% \item |tabularht|, |tabularht*|
% \item |arrayht|
% \item |tabularhtx|
%       (if package \xpackage{tabularx} is loaded)
% \end{itemize}
% The height argument allows a length specification,
% package \xpackage{calc} is supported if used. This means,
% the tabular will have the specified height.
% You can also use the prefixes |to=| and |spread=|.
% |to=| is the default, |spread=| means, the natural
% height of the tabular box is changed by the length
% after |spread=|.
%
% Examples:
% \begin{quote}
% \begin{tabular}{@{}l@{ $\rightarrow$ }l@{}}
%   |\begin{tabularht}{1in}|
%   & height is 1in\\
%   |\begin{tabularht}{to=1in}|
%   & height is 1in\\
%   |\begin{tabularht}{spread=0pt}|
%   & natural height, same as |\begin{tabular}|\\
%   |\begin{tabularht}{spread=1in}|
%   & natural height increased by 1in
% \end{tabular}
% \end{quote}
%
% Hint:  See also package \xpackage{tabularkv}, it provides an interface,
%        where most parameters for the environments can be given
%        by key-value pairs.
%
% \begin{declcs}{interrowspace}\,|{|\dots|}|
% \end{declcs}
% Adds space between table rows. It is essentially
% the same as |\noalign{\vspace{...}}|.
% \begin{declcs}{interrowfill}
% \end{declcs}
% Short for |\interrowspace{\fill}|
% \begin{declcs}{interrowstart}\,\dots\,\cs{interrowstop}
% \end{declcs}
% Marker commands, useful for option \xoption{vlines}.
%
% \subsection{Option \xoption{vlines}}
%
% Warning: This stuff is experimental.
%
% \vspace{2ex}
% \noindent
% Vertical lines are interrupted, if space is inserted
% in \cs{noalign}, \cs{interrowspace}, \cs{addlinespace}
% (\xpackage{booktabs}),
% between double \cs{hline}s. This option tries to detect
% and add the vertical lines. The lines in a tabular
% with \xpackage{tabularht} support (environments of this package)
% are numbered from left to right. The gap that is
% controlled by \cs{interrowspace} or inbetween \cs{interrowstart}
% and \cs{interrowstop} is then filled with the detected
% vertical lines.
%
% If only a limited selection of the lines should be
% drawn, the commands know an optional argument with a
% list of line numbers, e.g.
% \begin{quote}
%\begin{verbatim}
%\begin{tabularht}{50mm}{|l|l|}
%  Hello & World\\
%  \interrowfill[1,3]
%  Foo & Bar
%\end{tabularht}
%\end{verbatim}
% \end{quote}
% There are three lines, but the middle line is not drawn
% in the gap between the first and second row.
% Zero can be used to suppress all lines:
% \begin{quote}
% |\interrowspace[0]{10mm}|
% \end{quote}
% The syntax of the commands with the optional argument
% with the line number list \meta{list}. \meta{list} is a comma
% separated list of numbers, \meta{height} means the
% height specification described above with the optional
% prefixes |to=| or |spread=|.
% \begin{quote}
%   \cs{interrowspace}\,|[|\meta{list}|]|\,|{|\meta{height}|}|\\
%   \cs{interrowfill}\,|[|\meta{list}|]|\\
%   \cs{interrowstart}\,|[|\meta{list}|]|\,\dots\,\cs{interrowstop}
% \end{quote}
%
% Option \xoption{vlines} is driver dependent and uses \eTeX\ features.
% \begin{description}
% \item[\xoption{pdftex}:] \pdfTeX\ in PDF mode.
%   Here the positions of the lines are written with the
%   help of the \cs{pdfsavepos} feature into the \xfile{.aux} file(s).
%   Therefore you need two LaTeX runs to get the lines.
% \item[\xoption{dvips}:]
%   Here, PostScript's currentpoint it used to get
%   the line positions. The lines are then drawn at the end
%   of the page. Thus one \LaTeX/dvips run is sufficient
%   for this option.
% \item[Other drivers:]\mbox{}
%   \begin{description}
%   \item[PostScript drivers:] probably possible, an end of page
%     hook would be nice.
%   \item[\VTeX:] with GeX (PostScript interpreter) probably possible.
%   \item[dvipdfm:] no idea. The big problem is, how to get the
%     current position?
%   \end{description}
% \end{description}
%
% \subsection{Limitations}
%
% \begin{itemize}
% \item
%   Vertical lines are interrupted by |\noalign{\vfill}|.
% \end{itemize}
%
% \subsection{Compatibility}
%
% \begin{itemize}
% \item
%   \xpackage{array}, \xpackage{delarray}, \xpackage{tabularx}
%   are supported.
% \item There can be problems with packages that
%   redefine \cs{@array} (or \cs{@@array}, \cs{@tabarray}) and
%   \cs{@arrayrule} (for option \xoption{vlines}).
% \item \xpackage{colortbl}: it should at least work, but there isn't
%   support for filling the gaps with color, neither
%   the rules nor the backgrounds.
% \end{itemize}
%
% \subsection{Examples}
%
% \subsubsection{Example 1}
%
%    \begin{macrocode}
%<*example1>
\documentclass{article}
\usepackage{tabularht}

\begin{document}
\fbox{%
  \begin{tabularht*}{1in}{4in}{@{}l@{\extracolsep{\fill}}r@{}}%
    upper left corner & upper right corner\\%
    \noalign{\vfill}%
    \multicolumn{2}{@{}c@{}}{bounding box}\\%
    \noalign{\vfill}%
    lower left corner & lower right corner\\%
  \end{tabularht*}%
}
\end{document}
%</example1>
%    \end{macrocode}
%
% \subsubsection{Example 2}
%
%    \begin{macrocode}
%<*example2>
\documentclass{article}
\usepackage{booktabs}
\usepackage[dvips,vlines]{tabularht}

\begin{document}

\begin{tabularht}{spread=0pt}{|l|l|}
  \hline
  First&Line\\%
  \hline
\interrowstart
  \addlinespace[10mm]%
\interrowstop
  \hline
  Second&Line\\%
\interrowstart
  \hline
  \hline
\interrowstop
  Third&Line\\%
  \hline
\interrowspace{10mm}
  \hline
  Fourth&Line\\%
  \hline
\end{tabularht}

\end{document}
%</example2>
%    \end{macrocode}
%
% \StopEventually{
% }
%
% \section{Implementation}
%
%    \begin{macrocode}
%<*package>
%    \end{macrocode}
%    Package identification.
%    \begin{macrocode}
\NeedsTeXFormat{LaTeX2e}
\ProvidesPackage{tabularht}%
  [2016/05/16 v2.6 Tabular with height specified (HO)]
%    \end{macrocode}
%
% \subsection{Environments}
%
%    \begin{macrocode}
\let\@toarrayheight\@empty
\let\tabH@array@init\@empty

\toks@={%
  \begingroup
    \long\def\x#1\vcenter\fi\fi\bgroup#2\@sharp#3#4\@nil{%
      \endgroup
      \gdef\@array[##1]##2{%
        \tabH@array@init
        #1%
        \vcenter\fi\fi
        \@toarrayheight
        \bgroup
        \let\@toarrayheight\@empty
        #2\@sharp###3#4%
      }%
    }%
  \expandafter\x\@array[#1]{#2}\@nil % hash-ok
}
\edef\tabH@patch@array{\the\toks@}
\def\tabH@patch@@array{%
  \ifx\@array\@@array
    \def\reserved@a{\let\@@array\@array}%
  \else
    \let\reserved@a\relax
  \fi
  \tabH@patch@array
  \reserved@a
}
\tabH@patch@@array

\@ifpackageloaded{array}{}{%
  \AtBeginDocument{%
    \@ifpackageloaded{array}{%
      \tabH@patch@@array
    }{}%
  }%
}

\def\tabH@setheight#1{%
  \tabH@@setheight#1==\@nil
}
\def\tabH@@setheight#1=#2=#3\@nil{%
  \ifx\\#2#3\\%
    \setlength{\dimen@}{#1}%
    \edef\@toarrayheight{to\the\dimen@}%
  \else
    \edef\tabH@temp{\zap@space#1 \@empty}%
    \ifx\tabH@temp\tabH@to
    \else
      \ifx\tabH@temp\tabH@spread
      \else
        \PackageError{tabularht}{%
          Unknown height specifier %
          `\expandafter\strip@prefix\meaning\tabH@temp'%
        }{%
          The height dimension for tabular height can be prefixed%
          \MessageBreak
          with `to=' or `spread=', default is `to='.%
        }%
        \let\tabH@temp\tabH@to
      \fi
    \fi
    \setlength{\dimen@}{#2}%
    \edef\@toarrayheight{\tabH@temp\the\dimen@}%
  \fi
}
\def\tabH@to{to}
\def\tabH@spread{spread}
%    \end{macrocode}
%    First argument is the height of the table,
%    then the orignal arguments for tabular follow.
%    \begin{macrocode}
\newenvironment{tabularht}[1]{%
  \tabH@setheight{#1}%
  \tabular
}{%
  \endtabular
}

\newenvironment{tabularht*}[1]{%
  \tabH@setheight{#1}%
  \@nameuse{tabular*}%
}{%
  \@nameuse{endtabular*}%
}

\newenvironment{tabularhtx}[1]{%
  \tabH@setheight{#1}%
  \tabularx
}{%
  \endtabularx
}

\newenvironment{arrayht}[1]{%
  \tabH@setheight{#1}%
  \array
}{%
  \endarray
}

\def\interrowspace{%
  \noalign\bgroup
    \tabH@interrowspace
}
\newcommand*{\tabH@interrowspace}[2][]{%
    \tabH@vspace{#1}{#2}%
  \egroup
}
\def\interrowfill{%
  \noalign\bgroup
    \tabH@interrowfill
}
\newcommand*{\tabH@interrowfill}[1][]{%
    \tabH@vspace{#1}{\fill}%
  \egroup
}
\def\tabH@vspace#1#2{%
  \tabH@vspace@start{#1}%
  \vspace{#2}%
  \tabH@vspace@stop
}
\let\tabH@vspace@start\@gobble
\let\tabH@vspace@stop\@empty

\newcommand*{\interrowstart}{%
  \noalign\bgroup
    \tabH@interrowstart
}
\newcommand*{\tabH@interrowstart}[1][]{%
    \tabH@vspace@start{#1}%
  \egroup
}
\newcommand*{\interrowstop}{%
  \noalign{\tabH@vspace@stop}%
}
%    \end{macrocode}
%
% \subsection{Options}
%
%    \begin{macrocode}
\providecommand*{\tabH@driver}{}

\DeclareOption{vlines}{%
  \let\tabH@temp\relax
}
\DeclareOption{pdftex}{}
\DeclareOption{dvips}{%
  \def\tabH@driver{dvips}%
}
\ProcessOptions*\relax

\ifx\tabH@temp\relax
\else
  \expandafter\endinput
\fi

\begingroup
  \@ifundefined{eTeXversion}{%
    \PackageError{tabularht}{%
      Option `vlines' requires eTeX%
    }{%
      Use of eTeX is recommended for LaTeX, see ltnews16.%
    }%
    \endgroup
    \endinput
  }{}%
\endgroup
%    \end{macrocode}
%
% \subsection{Option vlines, driver independent stuff}
%
%    \begin{macrocode}
\begingroup
  \let\@addtoreset\@gobbletwo
  \newcounter{tabH@unique}%
\endgroup
\let\tabH@currenttab\@empty

\def\tabH@array@init{%
  \ifx\@toarrayheight\@empty
    % ignore vertical lines of nested tabular environments
    \let\tabH@currenttab\@empty
  \else
    \stepcounter{tabH@unique}%
    \edef\tabH@currenttab{\the\c@tabH@unique}%
  \fi
}

\renewcommand*{\@arrayrule}{%
  \@addtopreamble{%
    \hskip -.5\arrayrulewidth
    \ifx\tabH@currenttab\@empty
    \else
      \tabH@vrule{\tabH@currenttab}%
    \fi
    \begingroup
      \expandafter\ifx\csname CT@arc@\endcsname\relax
      \else
        \expandafter\CT@arc@
      \fi
      \vline
    \endgroup
    \hskip -.5\arrayrulewidth
  }%
}
\let\tabH@arrayrule\@arrayrule
\AtBeginDocument{%
  \@ifpackageloaded{colortbl}{%
    \let\@arrayrule\tabH@arrayrule
  }{}%
}

\let\tabH@vrule\@gobble
%    \end{macrocode}
%
% \subsection{Driver pdftex}
%
%    \begin{macrocode}
\RequirePackage{ifpdf}
\ifpdf
  \begingroup
    \@ifundefined{pdfsavepos}{%
      \PackageError{tabularht}{%
        Your pdfTeX is too old%
      }{%
        \string\pdfsavepos\space is missing.%
      }%
      \endgroup
      \csname fi\endcsname
      \endinput
    }{}%

    \let\on@line\@empty
    \PackageInfo{tabularht}{%
      Using driver `pdftex' because of pdfTeX in PDF mode%
    }%
  \endgroup

  \protected\def\tabH@vrule#1{%
    \if@filesw
      \pdfsavepos
      \protected@write\@auxout{%
        \let\tabH@lastxpos\relax
      }{%
        \tabH@aux@vrule{#1}{\tabH@lastxpos}%
      }%
    \fi
  }%

  \def\tabH@lastxpos{\the\pdflastxpos}%
  \def\tabH@lastypos{\the\pdflastypos}%

  % The .aux file contains three commands:
  % \tabH@aux@vrule{tabular id}{x position}
  % \tabH@aux@vstart{tabular id}{row id}{x position}{y position}
  % \tabH@aux@vstop{y position}
  %
  \AtBeginDocument{%
    % The .aux files are read the first time before
    % \AtBeginDocument and later at \end{document}.
    % \tabH@aux@done is a marker to distinguish
    % between these two readings. Only in the first
    % case we need the \tabH@aux@... commands.
    \let\tabH@aux@done\@empty
    \if@filesw
      \immediate\write\@mainaux{%
        \@percentchar\@percentchar BeginProlog: tabularht%
      }%
      % items in the aux file are executed,
      % if tabularht is loaded
      % and during the aux file read at \begin{document} only
      \immediate\write\@mainaux{%
        \detokenize{%
          % the \tabH@aux@... commands are needed only if
          % tabularht is loaded with driver pdftex.
          \@ifundefined{tabH@aux@vrule}\@secondoftwo\@firstofone
          {%
            % disable commands except for the first .aux files reading
            \@ifundefined{tabH@aux@done}\@gobble\@firstofone
          }%
          {%
            \let\tabH@aux@vrule\@gobbletwo
            \let\tabH@aux@vstart\@gobblefour
            \let\tabH@aux@vstop\@gobble
          }%
        }%
      }%
      \immediate\write\@mainaux{%
        \@percentchar\@percentchar EndProlog: tabularht%
      }%
    \fi
  }%

  % the x positions of vrules are stored in
  % \tabH@<tabcount>list with distinct values
  \protected\def\tabH@aux@vrule#1#2{%
    \@ifundefined{tabH@#1list}{%
      \expandafter\xdef\csname tabH@#1list\endcsname{%
        \noexpand\do{#2}%
      }%
    }{%
      \begingroup
        \def\x{#2}%
        \let\y\@undefined
        \let\do\tabH@do@add
        \expandafter\xdef\csname tabH@#1list\endcsname{%
          \csname tabH@#1list\endcsname\@empty
          \ifx\y\@undefined
            \noexpand\do{\x}%
          \fi
        }%
      \endgroup
    }%
  }%
  \def\tabH@do@add#1{%
    \ifx\y\@undefined
      \ifnum#1<\x\space
      \else
        \expandafter\ifx\csname y\endcsname\relax\fi
        \ifnum#1>\x\space
          \noexpand\do{\x}%
        \fi
      \fi
    \fi
    \noexpand\do{#1}%
  }%

  \def\tabH@vspace@start#1{%
    \if@filesw
      \stepcounter{tabH@unique}%
      \edef\tabH@currentrow{\the\c@tabH@unique}%
      \pdfsavepos
      \protected@write\@auxout{%
        \let\tabH@lastxpos\relax
        \let\tabH@lastypos\relax
      }{%
        \tabH@aux@vstart{\tabH@currenttab}{\tabH@currentrow}%
                        {\tabH@lastxpos}{\tabH@lastypos}%
      }%
    \fi
    \begingroup
      \edef\a{tabH@\tabH@currenttab row\tabH@currentrow}%
      \expandafter\let\expandafter\x\csname\a x\endcsname
      \ifx\x\relax
      \else
        \expandafter\let\expandafter\y\csname\a y\endcsname
        \expandafter\let\expandafter\l
            \csname tabH@\tabH@currenttab list\endcsname
        \ifx\l\relax
        \else
          \def\f{#1}%
          \ifx\f\@empty
            \let\do\tabH@do@set
          \else
            \count@=\z@
            \let\do\tabH@do@filter
          \fi
          \setbox\z@=\hbox{\l}%
          \wd\z@=\z@
          \dp\z@=\z@
          \copy\z@
        \fi
      \fi
    \endgroup
  }%
  \def\tabH@vspace@stop{%
    \if@filesw
      \pdfsavepos
      \protected@write\@auxout{%
        \let\tabH@lastypos\relax
      }{%
        \tabH@aux@vstop{\tabH@lastypos}%
      }%
    \fi
  }%
  \def\tabH@do@set#1{%
    \hbox to \z@{%
      \hskip \dimexpr #1sp - \x sp\relax
      \vrule \@width\arrayrulewidth
             \@depth\dimexpr \y sp\relax
      \hss
    }%
  }%
  \def\tabH@do@filter{%
    \@tempswafalse
    \advance\count@\@ne
    \@for\e:=\f\do{%
      \ifnum\e=\count@
        \@tempswatrue
      \fi
    }%
    \if@tempswa
      \expandafter\tabH@do@set
    \else
      \expandafter\@gobble
    \fi
  }%

  \protected\def\tabH@aux@vstart#1#2#3#4{%
    \def\tabH@current@vstart{{#1}{#2}{#3}{#4}}%
  }%
  \protected\def\tabH@aux@vstop{%
    \expandafter\tabH@aux@v\tabH@current@vstart
  }%
  \def\tabH@aux@v#1#2#3#4#5{%
    \expandafter\gdef\csname tabH@#1row#2x\endcsname{#3}%
    \expandafter\xdef\csname tabH@#1row#2y\endcsname{%
      \the\numexpr #4 - #5\relax
    }%
  }%

  \csname fi\endcsname
  \endinput

\fi
%    \end{macrocode}
%
% \subsection{DVI drivers}
%
%    \begin{macrocode}
\ifx\tabH@driver\@empty
  \PackageError{tabularht}{%
    Missing DVI driver, option `vlines' disabled%
  }{%
    Supported DVI drivers: dvips.%
  }%
  \expandafter\endinput
\fi

\def\tabH@driver@dvips{%
  \def\tabH@literalps##1{\special{ps:SDict begin ##1 end}}%
  \def\tabH@headerps##1{\special{! ##1}}%
}

\@onelevel@sanitize\tabH@driver
\@ifundefined{tabH@driver@\tabH@driver}{%
  \PackageError{tabularht}{%
    Unsupported driver `\tabH@driver'%
  }{%
    Supported DVI drivers: dvips.%
  }%
  \endinput
}{}

\begingroup
  \let\on@line\@empty
  \PackageInfo{tabularht}{%
    Using driver `\tabH@driver'%
  }%
\endgroup
\csname tabH@driver@\tabH@driver\endcsname

\protected\def\tabH@vrule#1#2\vrule#3\arrayrulewidth{%
  #2% \fi or empty
  % hack to get rid of maxdrift rounding of dvips,
  % thus simulate a large motion
  \kern1in\relax
  \tabH@literalps{%
    #1 tabH.vrule %
    Resolution neg 0 translate%
  }%
  \vrule#3\arrayrulewidth
  \tabH@literalps{Resolution 0 translate}%
  \kern-1in\relax
}

\def\tabH@vspace@start#1{%
  \begingroup
    \let\y\@empty
    \@for\x:=#1\do{%
      \ifx\y\@empty
        \edef\y{\x}%
      \else
        \edef\y{\y\space\x}%
      \fi
    }%
    \tabH@literalps{\tabH@currenttab[\y]currentpoint exch pop}%
  \endgroup
}
\def\tabH@vspace@stop{%
  \tabH@literalps{%
    currentpoint exch pop %
    \number\dimexpr\arrayrulewidth\relax\space
    tabH.vspace%
  }%
}

\tabH@headerps{%
  userdict begin%
    /tabH.list 10 dict def%
    /tabH.job [] def %
  end%
  /tabH.vrule{%
    10 string cvs cvn dup tabH.list exch known{%
      tabH.list exch dup [ exch tabH.list exch get %
      currentpoint pop round exch true exch{%
        % tabH.list key [ ... x true i
        % tabH.list key [ ... false i
        exch{%
          % ... [ ... x i
          2 copy lt{false}{%
            2 copy eq{pop false}{exch true}ifelse%
          }ifelse%
        }{false}ifelse%
      }forall %
      pop%
      ]put%
    }{%
      tabH.list exch[currentpoint pop round]put%
    }ifelse%
  }bind def%
  % <tab num> <cols array> <ytop> <ybottom> <rulewidth[sp]>
  /tabH.vspace{%
    userdict begin %
      10 dict dup begin %
        exch 65536 div Resolution mul 72.27 div %
        % dvips uses a poor man's ceil function
        % see dopage.c before "drawrule": (int)(... + 0.9999999)
        0.9999999 add truncate%
        /rulewidth exch def %
        exch/ybottom exch def %
        exch/ytop exch def %
        exch/cols exch def %
        exch/tabkey exch 10 string cvs cvn def %
      end%
      /tabH.job exch[exch userdict/tabH.job get aload pop]def %
    end%
  }bind def %
  % Now we do the work at the end of the page.
  % Unhappily "eop-hook" cannot be used, because "eop"
  % executes "restore" before, so that all data are lost.
  TeXDict begin%
    /eop%
    [%
      {%
        tabH.job{%
          begin%
           /colarray %
             tabH.list tabkey known{tabH.list tabkey get}{[]}ifelse %
           def %
           cols length 0 eq not{%
             /colarray[%
               cols{1 sub %
                 dup 0 lt{pop}{%
                   dup colarray length ge{pop}{%
                     colarray exch get%
                   }ifelse%
                 }ifelse%
               }forall%
             ]def%
           }if %
           colarray{%
             % (rulewidth) == rulewidth == % debug
             Resolution sub %
             ytop rulewidth ytop ybottom sub v%
           }forall %
          end%
        }forall%
        % tabH.list{== ==}forall % debug
      }bind aload pop %
      TeXDict /eop get aload pop%
    ]cvx def %
  end%
}
%    \end{macrocode}
%
%    \begin{macrocode}
%</package>
%    \end{macrocode}
%
% \section{Installation}
%
% \subsection{Download}
%
% \paragraph{Package.} This package is available on
% CTAN\footnote{\CTANpkg{tabularht}}:
% \begin{description}
% \item[\CTAN{macros/latex/contrib/oberdiek/tabularht.dtx}] The source file.
% \item[\CTAN{macros/latex/contrib/oberdiek/tabularht.pdf}] Documentation.
% \end{description}
%
%
% \paragraph{Bundle.} All the packages of the bundle `oberdiek'
% are also available in a TDS compliant ZIP archive. There
% the packages are already unpacked and the documentation files
% are generated. The files and directories obey the TDS standard.
% \begin{description}
% \item[\CTANinstall{install/macros/latex/contrib/oberdiek.tds.zip}]
% \end{description}
% \emph{TDS} refers to the standard ``A Directory Structure
% for \TeX\ Files'' (\CTAN{tds/tds.pdf}). Directories
% with \xfile{texmf} in their name are usually organized this way.
%
% \subsection{Bundle installation}
%
% \paragraph{Unpacking.} Unpack the \xfile{oberdiek.tds.zip} in the
% TDS tree (also known as \xfile{texmf} tree) of your choice.
% Example (linux):
% \begin{quote}
%   |unzip oberdiek.tds.zip -d ~/texmf|
% \end{quote}
%
% \paragraph{Script installation.}
% Check the directory \xfile{TDS:scripts/oberdiek/} for
% scripts that need further installation steps.
%
% \subsection{Package installation}
%
% \paragraph{Unpacking.} The \xfile{.dtx} file is a self-extracting
% \docstrip\ archive. The files are extracted by running the
% \xfile{.dtx} through \plainTeX:
% \begin{quote}
%   \verb|tex tabularht.dtx|
% \end{quote}
%
% \paragraph{TDS.} Now the different files must be moved into
% the different directories in your installation TDS tree
% (also known as \xfile{texmf} tree):
% \begin{quote}
% \def\t{^^A
% \begin{tabular}{@{}>{\ttfamily}l@{ $\rightarrow$ }>{\ttfamily}l@{}}
%   tabularht.sty & tex/latex/oberdiek/tabularht.sty\\
%   tabularht.pdf & doc/latex/oberdiek/tabularht.pdf\\
%   tabularht-example1.tex & doc/latex/oberdiek/tabularht-example1.tex\\
%   tabularht-example2.tex & doc/latex/oberdiek/tabularht-example2.tex\\
%   tabularht.dtx & source/latex/oberdiek/tabularht.dtx\\
% \end{tabular}^^A
% }^^A
% \sbox0{\t}^^A
% \ifdim\wd0>\linewidth
%   \begingroup
%     \advance\linewidth by\leftmargin
%     \advance\linewidth by\rightmargin
%   \edef\x{\endgroup
%     \def\noexpand\lw{\the\linewidth}^^A
%   }\x
%   \def\lwbox{^^A
%     \leavevmode
%     \hbox to \linewidth{^^A
%       \kern-\leftmargin\relax
%       \hss
%       \usebox0
%       \hss
%       \kern-\rightmargin\relax
%     }^^A
%   }^^A
%   \ifdim\wd0>\lw
%     \sbox0{\small\t}^^A
%     \ifdim\wd0>\linewidth
%       \ifdim\wd0>\lw
%         \sbox0{\footnotesize\t}^^A
%         \ifdim\wd0>\linewidth
%           \ifdim\wd0>\lw
%             \sbox0{\scriptsize\t}^^A
%             \ifdim\wd0>\linewidth
%               \ifdim\wd0>\lw
%                 \sbox0{\tiny\t}^^A
%                 \ifdim\wd0>\linewidth
%                   \lwbox
%                 \else
%                   \usebox0
%                 \fi
%               \else
%                 \lwbox
%               \fi
%             \else
%               \usebox0
%             \fi
%           \else
%             \lwbox
%           \fi
%         \else
%           \usebox0
%         \fi
%       \else
%         \lwbox
%       \fi
%     \else
%       \usebox0
%     \fi
%   \else
%     \lwbox
%   \fi
% \else
%   \usebox0
% \fi
% \end{quote}
% If you have a \xfile{docstrip.cfg} that configures and enables \docstrip's
% TDS installing feature, then some files can already be in the right
% place, see the documentation of \docstrip.
%
% \subsection{Refresh file name databases}
%
% If your \TeX~distribution
% (\teTeX, \mikTeX, \dots) relies on file name databases, you must refresh
% these. For example, \teTeX\ users run \verb|texhash| or
% \verb|mktexlsr|.
%
% \subsection{Some details for the interested}
%
% \paragraph{Unpacking with \LaTeX.}
% The \xfile{.dtx} chooses its action depending on the format:
% \begin{description}
% \item[\plainTeX:] Run \docstrip\ and extract the files.
% \item[\LaTeX:] Generate the documentation.
% \end{description}
% If you insist on using \LaTeX\ for \docstrip\ (really,
% \docstrip\ does not need \LaTeX), then inform the autodetect routine
% about your intention:
% \begin{quote}
%   \verb|latex \let\install=y\input{tabularht.dtx}|
% \end{quote}
% Do not forget to quote the argument according to the demands
% of your shell.
%
% \paragraph{Generating the documentation.}
% You can use both the \xfile{.dtx} or the \xfile{.drv} to generate
% the documentation. The process can be configured by the
% configuration file \xfile{ltxdoc.cfg}. For instance, put this
% line into this file, if you want to have A4 as paper format:
% \begin{quote}
%   \verb|\PassOptionsToClass{a4paper}{article}|
% \end{quote}
% An example follows how to generate the
% documentation with pdf\LaTeX:
% \begin{quote}
%\begin{verbatim}
%pdflatex tabularht.dtx
%makeindex -s gind.ist tabularht.idx
%pdflatex tabularht.dtx
%makeindex -s gind.ist tabularht.idx
%pdflatex tabularht.dtx
%\end{verbatim}
% \end{quote}
%
% \begin{History}
%   \begin{Version}{2005/09/22 v1.0}
%   \item
%     First public version.
%   \end{Version}
%   \begin{Version}{2005/10/16 v2.0}
%   \item
%     Height specification allows
%     |to=...| or |spread=...|, default is |to=|.
%   \item
%     Option \xoption{vlines} added,
%     drivers \xoption{pdftex} and \xoption{dvips}.
%   \item
%     \cs{interrowspace}, \cs{interrowfil}, and
%     \cs{interrowstart}\dots\cs{interrowstop} added.
%   \end{Version}
%   \begin{Version}{2005/10/18 v2.1}
%   \item
%     Fix for package \xpackage{colortbl}, but the colors
%     of \xpackage{colortbl} remain unsupported.
%   \end{Version}
%   \begin{Version}{2006/02/20 v2.2}
%   \item
%     Code is not changed.
%   \item
%     DTX framework.
%   \end{Version}
%   \begin{Version}{2006/12/22 v2.3}
%   \item
%     Documentation fix.
%   \item
%     Fix in code of option vlines.
%   \end{Version}
%   \begin{Version}{2007/03/21 v2.4}
%   \item
%     Fix: Counter |tabh@unique| must not be changed by \cs{include}.
%   \end{Version}
%   \begin{Version}{2007/04/11 v2.5}
%   \item
%     Line ends sanitized.
%   \end{Version}
%   \begin{Version}{2016/05/16 v2.6}
%   \item
%     Documentation updates.
%   \end{Version}
% \end{History}
%
% \PrintIndex
%
% \Finale
\endinput

%        (quote the arguments according to the demands of your shell)
%
% Documentation:
%    (a) If tabularht.drv is present:
%           latex tabularht.drv
%    (b) Without tabularht.drv:
%           latex tabularht.dtx; ...
%    The class ltxdoc loads the configuration file ltxdoc.cfg
%    if available. Here you can specify further options, e.g.
%    use A4 as paper format:
%       \PassOptionsToClass{a4paper}{article}
%
%    Programm calls to get the documentation (example):
%       pdflatex tabularht.dtx
%       makeindex -s gind.ist tabularht.idx
%       pdflatex tabularht.dtx
%       makeindex -s gind.ist tabularht.idx
%       pdflatex tabularht.dtx
%
% Installation:
%    TDS:tex/latex/oberdiek/tabularht.sty
%    TDS:doc/latex/oberdiek/tabularht.pdf
%    TDS:doc/latex/oberdiek/tabularht-example1.tex
%    TDS:doc/latex/oberdiek/tabularht-example2.tex
%    TDS:source/latex/oberdiek/tabularht.dtx
%
%<*ignore>
\begingroup
  \catcode123=1 %
  \catcode125=2 %
  \def\x{LaTeX2e}%
\expandafter\endgroup
\ifcase 0\ifx\install y1\fi\expandafter
         \ifx\csname processbatchFile\endcsname\relax\else1\fi
         \ifx\fmtname\x\else 1\fi\relax
\else\csname fi\endcsname
%</ignore>
%<*install>
\input docstrip.tex
\Msg{************************************************************************}
\Msg{* Installation}
\Msg{* Package: tabularht 2016/05/16 v2.6 Tabular with height specified (HO)}
\Msg{************************************************************************}

\keepsilent
\askforoverwritefalse

\let\MetaPrefix\relax
\preamble

This is a generated file.

Project: tabularht
Version: 2016/05/16 v2.6

Copyright (C) 2005-2007 by
   Heiko Oberdiek <heiko.oberdiek at googlemail.com>

This work may be distributed and/or modified under the
conditions of the LaTeX Project Public License, either
version 1.3c of this license or (at your option) any later
version. This version of this license is in
   https://www.latex-project.org/lppl/lppl-1-3c.txt
and the latest version of this license is in
   https://www.latex-project.org/lppl.txt
and version 1.3 or later is part of all distributions of
LaTeX version 2005/12/01 or later.

This work has the LPPL maintenance status "maintained".

The Current Maintainers of this work are
Heiko Oberdiek and the Oberdiek Package Support Group
https://github.com/ho-tex/oberdiek/issues


This work consists of the main source file tabularht.dtx
and the derived files
   tabularht.sty, tabularht.pdf, tabularht.ins, tabularht.drv,
   tabularht-example1.tex, tabularht-example2.tex.

\endpreamble
\let\MetaPrefix\DoubleperCent

\generate{%
  \file{tabularht.ins}{\from{tabularht.dtx}{install}}%
  \file{tabularht.drv}{\from{tabularht.dtx}{driver}}%
  \usedir{tex/latex/oberdiek}%
  \file{tabularht.sty}{\from{tabularht.dtx}{package}}%
  \usedir{doc/latex/oberdiek}%
  \file{tabularht-example1.tex}{\from{tabularht.dtx}{example1}}%
  \file{tabularht-example2.tex}{\from{tabularht.dtx}{example2}}%
  \nopreamble
  \nopostamble
%  \usedir{source/latex/oberdiek/catalogue}%
%  \file{tabularht.xml}{\from{tabularht.dtx}{catalogue}}%
}

\catcode32=13\relax% active space
\let =\space%
\Msg{************************************************************************}
\Msg{*}
\Msg{* To finish the installation you have to move the following}
\Msg{* file into a directory searched by TeX:}
\Msg{*}
\Msg{*     tabularht.sty}
\Msg{*}
\Msg{* To produce the documentation run the file `tabularht.drv'}
\Msg{* through LaTeX.}
\Msg{*}
\Msg{* Happy TeXing!}
\Msg{*}
\Msg{************************************************************************}

\endbatchfile
%</install>
%<*ignore>
\fi
%</ignore>
%<*driver>
\NeedsTeXFormat{LaTeX2e}
\ProvidesFile{tabularht.drv}%
  [2016/05/16 v2.6 Tabular with height specified (HO)]%
\documentclass{ltxdoc}
\usepackage{holtxdoc}[2011/11/22]
\begin{document}
  \DocInput{tabularht.dtx}%
\end{document}
%</driver>
% \fi
%
%
% \CharacterTable
%  {Upper-case    \A\B\C\D\E\F\G\H\I\J\K\L\M\N\O\P\Q\R\S\T\U\V\W\X\Y\Z
%   Lower-case    \a\b\c\d\e\f\g\h\i\j\k\l\m\n\o\p\q\r\s\t\u\v\w\x\y\z
%   Digits        \0\1\2\3\4\5\6\7\8\9
%   Exclamation   \!     Double quote  \"     Hash (number) \#
%   Dollar        \$     Percent       \%     Ampersand     \&
%   Acute accent  \'     Left paren    \(     Right paren   \)
%   Asterisk      \*     Plus          \+     Comma         \,
%   Minus         \-     Point         \.     Solidus       \/
%   Colon         \:     Semicolon     \;     Less than     \<
%   Equals        \=     Greater than  \>     Question mark \?
%   Commercial at \@     Left bracket  \[     Backslash     \\
%   Right bracket \]     Circumflex    \^     Underscore    \_
%   Grave accent  \`     Left brace    \{     Vertical bar  \|
%   Right brace   \}     Tilde         \~}
%
% \GetFileInfo{tabularht.drv}
%
% \title{The \xpackage{tabularht} package}
% \date{2016/05/16 v2.6}
% \author{Heiko Oberdiek\thanks
% {Please report any issues at \url{https://github.com/ho-tex/oberdiek/issues}}}
%
% \maketitle
%
% \begin{abstract}
% This package defines some environments that adds
% a height specification to tabular and array.
% \end{abstract}
%
% \tableofcontents
%
% \section{Usage}
%
% \begin{quote}
%   |\usepackage{tabularht}|
% \end{quote}
% The package provides the following environments
% that extend the tabular/array environment by
% a height specification as first argument:
% \begin{itemize}
% \item |tabularht|, |tabularht*|
% \item |arrayht|
% \item |tabularhtx|
%       (if package \xpackage{tabularx} is loaded)
% \end{itemize}
% The height argument allows a length specification,
% package \xpackage{calc} is supported if used. This means,
% the tabular will have the specified height.
% You can also use the prefixes |to=| and |spread=|.
% |to=| is the default, |spread=| means, the natural
% height of the tabular box is changed by the length
% after |spread=|.
%
% Examples:
% \begin{quote}
% \begin{tabular}{@{}l@{ $\rightarrow$ }l@{}}
%   |\begin{tabularht}{1in}|
%   & height is 1in\\
%   |\begin{tabularht}{to=1in}|
%   & height is 1in\\
%   |\begin{tabularht}{spread=0pt}|
%   & natural height, same as |\begin{tabular}|\\
%   |\begin{tabularht}{spread=1in}|
%   & natural height increased by 1in
% \end{tabular}
% \end{quote}
%
% Hint:  See also package \xpackage{tabularkv}, it provides an interface,
%        where most parameters for the environments can be given
%        by key-value pairs.
%
% \begin{declcs}{interrowspace}\,|{|\dots|}|
% \end{declcs}
% Adds space between table rows. It is essentially
% the same as |\noalign{\vspace{...}}|.
% \begin{declcs}{interrowfill}
% \end{declcs}
% Short for |\interrowspace{\fill}|
% \begin{declcs}{interrowstart}\,\dots\,\cs{interrowstop}
% \end{declcs}
% Marker commands, useful for option \xoption{vlines}.
%
% \subsection{Option \xoption{vlines}}
%
% Warning: This stuff is experimental.
%
% \vspace{2ex}
% \noindent
% Vertical lines are interrupted, if space is inserted
% in \cs{noalign}, \cs{interrowspace}, \cs{addlinespace}
% (\xpackage{booktabs}),
% between double \cs{hline}s. This option tries to detect
% and add the vertical lines. The lines in a tabular
% with \xpackage{tabularht} support (environments of this package)
% are numbered from left to right. The gap that is
% controlled by \cs{interrowspace} or inbetween \cs{interrowstart}
% and \cs{interrowstop} is then filled with the detected
% vertical lines.
%
% If only a limited selection of the lines should be
% drawn, the commands know an optional argument with a
% list of line numbers, e.g.
% \begin{quote}
%\begin{verbatim}
%\begin{tabularht}{50mm}{|l|l|}
%  Hello & World\\
%  \interrowfill[1,3]
%  Foo & Bar
%\end{tabularht}
%\end{verbatim}
% \end{quote}
% There are three lines, but the middle line is not drawn
% in the gap between the first and second row.
% Zero can be used to suppress all lines:
% \begin{quote}
% |\interrowspace[0]{10mm}|
% \end{quote}
% The syntax of the commands with the optional argument
% with the line number list \meta{list}. \meta{list} is a comma
% separated list of numbers, \meta{height} means the
% height specification described above with the optional
% prefixes |to=| or |spread=|.
% \begin{quote}
%   \cs{interrowspace}\,|[|\meta{list}|]|\,|{|\meta{height}|}|\\
%   \cs{interrowfill}\,|[|\meta{list}|]|\\
%   \cs{interrowstart}\,|[|\meta{list}|]|\,\dots\,\cs{interrowstop}
% \end{quote}
%
% Option \xoption{vlines} is driver dependent and uses \eTeX\ features.
% \begin{description}
% \item[\xoption{pdftex}:] \pdfTeX\ in PDF mode.
%   Here the positions of the lines are written with the
%   help of the \cs{pdfsavepos} feature into the \xfile{.aux} file(s).
%   Therefore you need two LaTeX runs to get the lines.
% \item[\xoption{dvips}:]
%   Here, PostScript's currentpoint it used to get
%   the line positions. The lines are then drawn at the end
%   of the page. Thus one \LaTeX/dvips run is sufficient
%   for this option.
% \item[Other drivers:]\mbox{}
%   \begin{description}
%   \item[PostScript drivers:] probably possible, an end of page
%     hook would be nice.
%   \item[\VTeX:] with GeX (PostScript interpreter) probably possible.
%   \item[dvipdfm:] no idea. The big problem is, how to get the
%     current position?
%   \end{description}
% \end{description}
%
% \subsection{Limitations}
%
% \begin{itemize}
% \item
%   Vertical lines are interrupted by |\noalign{\vfill}|.
% \end{itemize}
%
% \subsection{Compatibility}
%
% \begin{itemize}
% \item
%   \xpackage{array}, \xpackage{delarray}, \xpackage{tabularx}
%   are supported.
% \item There can be problems with packages that
%   redefine \cs{@array} (or \cs{@@array}, \cs{@tabarray}) and
%   \cs{@arrayrule} (for option \xoption{vlines}).
% \item \xpackage{colortbl}: it should at least work, but there isn't
%   support for filling the gaps with color, neither
%   the rules nor the backgrounds.
% \end{itemize}
%
% \subsection{Examples}
%
% \subsubsection{Example 1}
%
%    \begin{macrocode}
%<*example1>
\documentclass{article}
\usepackage{tabularht}

\begin{document}
\fbox{%
  \begin{tabularht*}{1in}{4in}{@{}l@{\extracolsep{\fill}}r@{}}%
    upper left corner & upper right corner\\%
    \noalign{\vfill}%
    \multicolumn{2}{@{}c@{}}{bounding box}\\%
    \noalign{\vfill}%
    lower left corner & lower right corner\\%
  \end{tabularht*}%
}
\end{document}
%</example1>
%    \end{macrocode}
%
% \subsubsection{Example 2}
%
%    \begin{macrocode}
%<*example2>
\documentclass{article}
\usepackage{booktabs}
\usepackage[dvips,vlines]{tabularht}

\begin{document}

\begin{tabularht}{spread=0pt}{|l|l|}
  \hline
  First&Line\\%
  \hline
\interrowstart
  \addlinespace[10mm]%
\interrowstop
  \hline
  Second&Line\\%
\interrowstart
  \hline
  \hline
\interrowstop
  Third&Line\\%
  \hline
\interrowspace{10mm}
  \hline
  Fourth&Line\\%
  \hline
\end{tabularht}

\end{document}
%</example2>
%    \end{macrocode}
%
% \StopEventually{
% }
%
% \section{Implementation}
%
%    \begin{macrocode}
%<*package>
%    \end{macrocode}
%    Package identification.
%    \begin{macrocode}
\NeedsTeXFormat{LaTeX2e}
\ProvidesPackage{tabularht}%
  [2016/05/16 v2.6 Tabular with height specified (HO)]
%    \end{macrocode}
%
% \subsection{Environments}
%
%    \begin{macrocode}
\let\@toarrayheight\@empty
\let\tabH@array@init\@empty

\toks@={%
  \begingroup
    \long\def\x#1\vcenter\fi\fi\bgroup#2\@sharp#3#4\@nil{%
      \endgroup
      \gdef\@array[##1]##2{%
        \tabH@array@init
        #1%
        \vcenter\fi\fi
        \@toarrayheight
        \bgroup
        \let\@toarrayheight\@empty
        #2\@sharp###3#4%
      }%
    }%
  \expandafter\x\@array[#1]{#2}\@nil % hash-ok
}
\edef\tabH@patch@array{\the\toks@}
\def\tabH@patch@@array{%
  \ifx\@array\@@array
    \def\reserved@a{\let\@@array\@array}%
  \else
    \let\reserved@a\relax
  \fi
  \tabH@patch@array
  \reserved@a
}
\tabH@patch@@array

\@ifpackageloaded{array}{}{%
  \AtBeginDocument{%
    \@ifpackageloaded{array}{%
      \tabH@patch@@array
    }{}%
  }%
}

\def\tabH@setheight#1{%
  \tabH@@setheight#1==\@nil
}
\def\tabH@@setheight#1=#2=#3\@nil{%
  \ifx\\#2#3\\%
    \setlength{\dimen@}{#1}%
    \edef\@toarrayheight{to\the\dimen@}%
  \else
    \edef\tabH@temp{\zap@space#1 \@empty}%
    \ifx\tabH@temp\tabH@to
    \else
      \ifx\tabH@temp\tabH@spread
      \else
        \PackageError{tabularht}{%
          Unknown height specifier %
          `\expandafter\strip@prefix\meaning\tabH@temp'%
        }{%
          The height dimension for tabular height can be prefixed%
          \MessageBreak
          with `to=' or `spread=', default is `to='.%
        }%
        \let\tabH@temp\tabH@to
      \fi
    \fi
    \setlength{\dimen@}{#2}%
    \edef\@toarrayheight{\tabH@temp\the\dimen@}%
  \fi
}
\def\tabH@to{to}
\def\tabH@spread{spread}
%    \end{macrocode}
%    First argument is the height of the table,
%    then the orignal arguments for tabular follow.
%    \begin{macrocode}
\newenvironment{tabularht}[1]{%
  \tabH@setheight{#1}%
  \tabular
}{%
  \endtabular
}

\newenvironment{tabularht*}[1]{%
  \tabH@setheight{#1}%
  \@nameuse{tabular*}%
}{%
  \@nameuse{endtabular*}%
}

\newenvironment{tabularhtx}[1]{%
  \tabH@setheight{#1}%
  \tabularx
}{%
  \endtabularx
}

\newenvironment{arrayht}[1]{%
  \tabH@setheight{#1}%
  \array
}{%
  \endarray
}

\def\interrowspace{%
  \noalign\bgroup
    \tabH@interrowspace
}
\newcommand*{\tabH@interrowspace}[2][]{%
    \tabH@vspace{#1}{#2}%
  \egroup
}
\def\interrowfill{%
  \noalign\bgroup
    \tabH@interrowfill
}
\newcommand*{\tabH@interrowfill}[1][]{%
    \tabH@vspace{#1}{\fill}%
  \egroup
}
\def\tabH@vspace#1#2{%
  \tabH@vspace@start{#1}%
  \vspace{#2}%
  \tabH@vspace@stop
}
\let\tabH@vspace@start\@gobble
\let\tabH@vspace@stop\@empty

\newcommand*{\interrowstart}{%
  \noalign\bgroup
    \tabH@interrowstart
}
\newcommand*{\tabH@interrowstart}[1][]{%
    \tabH@vspace@start{#1}%
  \egroup
}
\newcommand*{\interrowstop}{%
  \noalign{\tabH@vspace@stop}%
}
%    \end{macrocode}
%
% \subsection{Options}
%
%    \begin{macrocode}
\providecommand*{\tabH@driver}{}

\DeclareOption{vlines}{%
  \let\tabH@temp\relax
}
\DeclareOption{pdftex}{}
\DeclareOption{dvips}{%
  \def\tabH@driver{dvips}%
}
\ProcessOptions*\relax

\ifx\tabH@temp\relax
\else
  \expandafter\endinput
\fi

\begingroup
  \@ifundefined{eTeXversion}{%
    \PackageError{tabularht}{%
      Option `vlines' requires eTeX%
    }{%
      Use of eTeX is recommended for LaTeX, see ltnews16.%
    }%
    \endgroup
    \endinput
  }{}%
\endgroup
%    \end{macrocode}
%
% \subsection{Option vlines, driver independent stuff}
%
%    \begin{macrocode}
\begingroup
  \let\@addtoreset\@gobbletwo
  \newcounter{tabH@unique}%
\endgroup
\let\tabH@currenttab\@empty

\def\tabH@array@init{%
  \ifx\@toarrayheight\@empty
    % ignore vertical lines of nested tabular environments
    \let\tabH@currenttab\@empty
  \else
    \stepcounter{tabH@unique}%
    \edef\tabH@currenttab{\the\c@tabH@unique}%
  \fi
}

\renewcommand*{\@arrayrule}{%
  \@addtopreamble{%
    \hskip -.5\arrayrulewidth
    \ifx\tabH@currenttab\@empty
    \else
      \tabH@vrule{\tabH@currenttab}%
    \fi
    \begingroup
      \expandafter\ifx\csname CT@arc@\endcsname\relax
      \else
        \expandafter\CT@arc@
      \fi
      \vline
    \endgroup
    \hskip -.5\arrayrulewidth
  }%
}
\let\tabH@arrayrule\@arrayrule
\AtBeginDocument{%
  \@ifpackageloaded{colortbl}{%
    \let\@arrayrule\tabH@arrayrule
  }{}%
}

\let\tabH@vrule\@gobble
%    \end{macrocode}
%
% \subsection{Driver pdftex}
%
%    \begin{macrocode}
\RequirePackage{ifpdf}
\ifpdf
  \begingroup
    \@ifundefined{pdfsavepos}{%
      \PackageError{tabularht}{%
        Your pdfTeX is too old%
      }{%
        \string\pdfsavepos\space is missing.%
      }%
      \endgroup
      \csname fi\endcsname
      \endinput
    }{}%

    \let\on@line\@empty
    \PackageInfo{tabularht}{%
      Using driver `pdftex' because of pdfTeX in PDF mode%
    }%
  \endgroup

  \protected\def\tabH@vrule#1{%
    \if@filesw
      \pdfsavepos
      \protected@write\@auxout{%
        \let\tabH@lastxpos\relax
      }{%
        \tabH@aux@vrule{#1}{\tabH@lastxpos}%
      }%
    \fi
  }%

  \def\tabH@lastxpos{\the\pdflastxpos}%
  \def\tabH@lastypos{\the\pdflastypos}%

  % The .aux file contains three commands:
  % \tabH@aux@vrule{tabular id}{x position}
  % \tabH@aux@vstart{tabular id}{row id}{x position}{y position}
  % \tabH@aux@vstop{y position}
  %
  \AtBeginDocument{%
    % The .aux files are read the first time before
    % \AtBeginDocument and later at \end{document}.
    % \tabH@aux@done is a marker to distinguish
    % between these two readings. Only in the first
    % case we need the \tabH@aux@... commands.
    \let\tabH@aux@done\@empty
    \if@filesw
      \immediate\write\@mainaux{%
        \@percentchar\@percentchar BeginProlog: tabularht%
      }%
      % items in the aux file are executed,
      % if tabularht is loaded
      % and during the aux file read at \begin{document} only
      \immediate\write\@mainaux{%
        \detokenize{%
          % the \tabH@aux@... commands are needed only if
          % tabularht is loaded with driver pdftex.
          \@ifundefined{tabH@aux@vrule}\@secondoftwo\@firstofone
          {%
            % disable commands except for the first .aux files reading
            \@ifundefined{tabH@aux@done}\@gobble\@firstofone
          }%
          {%
            \let\tabH@aux@vrule\@gobbletwo
            \let\tabH@aux@vstart\@gobblefour
            \let\tabH@aux@vstop\@gobble
          }%
        }%
      }%
      \immediate\write\@mainaux{%
        \@percentchar\@percentchar EndProlog: tabularht%
      }%
    \fi
  }%

  % the x positions of vrules are stored in
  % \tabH@<tabcount>list with distinct values
  \protected\def\tabH@aux@vrule#1#2{%
    \@ifundefined{tabH@#1list}{%
      \expandafter\xdef\csname tabH@#1list\endcsname{%
        \noexpand\do{#2}%
      }%
    }{%
      \begingroup
        \def\x{#2}%
        \let\y\@undefined
        \let\do\tabH@do@add
        \expandafter\xdef\csname tabH@#1list\endcsname{%
          \csname tabH@#1list\endcsname\@empty
          \ifx\y\@undefined
            \noexpand\do{\x}%
          \fi
        }%
      \endgroup
    }%
  }%
  \def\tabH@do@add#1{%
    \ifx\y\@undefined
      \ifnum#1<\x\space
      \else
        \expandafter\ifx\csname y\endcsname\relax\fi
        \ifnum#1>\x\space
          \noexpand\do{\x}%
        \fi
      \fi
    \fi
    \noexpand\do{#1}%
  }%

  \def\tabH@vspace@start#1{%
    \if@filesw
      \stepcounter{tabH@unique}%
      \edef\tabH@currentrow{\the\c@tabH@unique}%
      \pdfsavepos
      \protected@write\@auxout{%
        \let\tabH@lastxpos\relax
        \let\tabH@lastypos\relax
      }{%
        \tabH@aux@vstart{\tabH@currenttab}{\tabH@currentrow}%
                        {\tabH@lastxpos}{\tabH@lastypos}%
      }%
    \fi
    \begingroup
      \edef\a{tabH@\tabH@currenttab row\tabH@currentrow}%
      \expandafter\let\expandafter\x\csname\a x\endcsname
      \ifx\x\relax
      \else
        \expandafter\let\expandafter\y\csname\a y\endcsname
        \expandafter\let\expandafter\l
            \csname tabH@\tabH@currenttab list\endcsname
        \ifx\l\relax
        \else
          \def\f{#1}%
          \ifx\f\@empty
            \let\do\tabH@do@set
          \else
            \count@=\z@
            \let\do\tabH@do@filter
          \fi
          \setbox\z@=\hbox{\l}%
          \wd\z@=\z@
          \dp\z@=\z@
          \copy\z@
        \fi
      \fi
    \endgroup
  }%
  \def\tabH@vspace@stop{%
    \if@filesw
      \pdfsavepos
      \protected@write\@auxout{%
        \let\tabH@lastypos\relax
      }{%
        \tabH@aux@vstop{\tabH@lastypos}%
      }%
    \fi
  }%
  \def\tabH@do@set#1{%
    \hbox to \z@{%
      \hskip \dimexpr #1sp - \x sp\relax
      \vrule \@width\arrayrulewidth
             \@depth\dimexpr \y sp\relax
      \hss
    }%
  }%
  \def\tabH@do@filter{%
    \@tempswafalse
    \advance\count@\@ne
    \@for\e:=\f\do{%
      \ifnum\e=\count@
        \@tempswatrue
      \fi
    }%
    \if@tempswa
      \expandafter\tabH@do@set
    \else
      \expandafter\@gobble
    \fi
  }%

  \protected\def\tabH@aux@vstart#1#2#3#4{%
    \def\tabH@current@vstart{{#1}{#2}{#3}{#4}}%
  }%
  \protected\def\tabH@aux@vstop{%
    \expandafter\tabH@aux@v\tabH@current@vstart
  }%
  \def\tabH@aux@v#1#2#3#4#5{%
    \expandafter\gdef\csname tabH@#1row#2x\endcsname{#3}%
    \expandafter\xdef\csname tabH@#1row#2y\endcsname{%
      \the\numexpr #4 - #5\relax
    }%
  }%

  \csname fi\endcsname
  \endinput

\fi
%    \end{macrocode}
%
% \subsection{DVI drivers}
%
%    \begin{macrocode}
\ifx\tabH@driver\@empty
  \PackageError{tabularht}{%
    Missing DVI driver, option `vlines' disabled%
  }{%
    Supported DVI drivers: dvips.%
  }%
  \expandafter\endinput
\fi

\def\tabH@driver@dvips{%
  \def\tabH@literalps##1{\special{ps:SDict begin ##1 end}}%
  \def\tabH@headerps##1{\special{! ##1}}%
}

\@onelevel@sanitize\tabH@driver
\@ifundefined{tabH@driver@\tabH@driver}{%
  \PackageError{tabularht}{%
    Unsupported driver `\tabH@driver'%
  }{%
    Supported DVI drivers: dvips.%
  }%
  \endinput
}{}

\begingroup
  \let\on@line\@empty
  \PackageInfo{tabularht}{%
    Using driver `\tabH@driver'%
  }%
\endgroup
\csname tabH@driver@\tabH@driver\endcsname

\protected\def\tabH@vrule#1#2\vrule#3\arrayrulewidth{%
  #2% \fi or empty
  % hack to get rid of maxdrift rounding of dvips,
  % thus simulate a large motion
  \kern1in\relax
  \tabH@literalps{%
    #1 tabH.vrule %
    Resolution neg 0 translate%
  }%
  \vrule#3\arrayrulewidth
  \tabH@literalps{Resolution 0 translate}%
  \kern-1in\relax
}

\def\tabH@vspace@start#1{%
  \begingroup
    \let\y\@empty
    \@for\x:=#1\do{%
      \ifx\y\@empty
        \edef\y{\x}%
      \else
        \edef\y{\y\space\x}%
      \fi
    }%
    \tabH@literalps{\tabH@currenttab[\y]currentpoint exch pop}%
  \endgroup
}
\def\tabH@vspace@stop{%
  \tabH@literalps{%
    currentpoint exch pop %
    \number\dimexpr\arrayrulewidth\relax\space
    tabH.vspace%
  }%
}

\tabH@headerps{%
  userdict begin%
    /tabH.list 10 dict def%
    /tabH.job [] def %
  end%
  /tabH.vrule{%
    10 string cvs cvn dup tabH.list exch known{%
      tabH.list exch dup [ exch tabH.list exch get %
      currentpoint pop round exch true exch{%
        % tabH.list key [ ... x true i
        % tabH.list key [ ... false i
        exch{%
          % ... [ ... x i
          2 copy lt{false}{%
            2 copy eq{pop false}{exch true}ifelse%
          }ifelse%
        }{false}ifelse%
      }forall %
      pop%
      ]put%
    }{%
      tabH.list exch[currentpoint pop round]put%
    }ifelse%
  }bind def%
  % <tab num> <cols array> <ytop> <ybottom> <rulewidth[sp]>
  /tabH.vspace{%
    userdict begin %
      10 dict dup begin %
        exch 65536 div Resolution mul 72.27 div %
        % dvips uses a poor man's ceil function
        % see dopage.c before "drawrule": (int)(... + 0.9999999)
        0.9999999 add truncate%
        /rulewidth exch def %
        exch/ybottom exch def %
        exch/ytop exch def %
        exch/cols exch def %
        exch/tabkey exch 10 string cvs cvn def %
      end%
      /tabH.job exch[exch userdict/tabH.job get aload pop]def %
    end%
  }bind def %
  % Now we do the work at the end of the page.
  % Unhappily "eop-hook" cannot be used, because "eop"
  % executes "restore" before, so that all data are lost.
  TeXDict begin%
    /eop%
    [%
      {%
        tabH.job{%
          begin%
           /colarray %
             tabH.list tabkey known{tabH.list tabkey get}{[]}ifelse %
           def %
           cols length 0 eq not{%
             /colarray[%
               cols{1 sub %
                 dup 0 lt{pop}{%
                   dup colarray length ge{pop}{%
                     colarray exch get%
                   }ifelse%
                 }ifelse%
               }forall%
             ]def%
           }if %
           colarray{%
             % (rulewidth) == rulewidth == % debug
             Resolution sub %
             ytop rulewidth ytop ybottom sub v%
           }forall %
          end%
        }forall%
        % tabH.list{== ==}forall % debug
      }bind aload pop %
      TeXDict /eop get aload pop%
    ]cvx def %
  end%
}
%    \end{macrocode}
%
%    \begin{macrocode}
%</package>
%    \end{macrocode}
%
% \section{Installation}
%
% \subsection{Download}
%
% \paragraph{Package.} This package is available on
% CTAN\footnote{\CTANpkg{tabularht}}:
% \begin{description}
% \item[\CTAN{macros/latex/contrib/oberdiek/tabularht.dtx}] The source file.
% \item[\CTAN{macros/latex/contrib/oberdiek/tabularht.pdf}] Documentation.
% \end{description}
%
%
% \paragraph{Bundle.} All the packages of the bundle `oberdiek'
% are also available in a TDS compliant ZIP archive. There
% the packages are already unpacked and the documentation files
% are generated. The files and directories obey the TDS standard.
% \begin{description}
% \item[\CTANinstall{install/macros/latex/contrib/oberdiek.tds.zip}]
% \end{description}
% \emph{TDS} refers to the standard ``A Directory Structure
% for \TeX\ Files'' (\CTAN{tds/tds.pdf}). Directories
% with \xfile{texmf} in their name are usually organized this way.
%
% \subsection{Bundle installation}
%
% \paragraph{Unpacking.} Unpack the \xfile{oberdiek.tds.zip} in the
% TDS tree (also known as \xfile{texmf} tree) of your choice.
% Example (linux):
% \begin{quote}
%   |unzip oberdiek.tds.zip -d ~/texmf|
% \end{quote}
%
% \paragraph{Script installation.}
% Check the directory \xfile{TDS:scripts/oberdiek/} for
% scripts that need further installation steps.
%
% \subsection{Package installation}
%
% \paragraph{Unpacking.} The \xfile{.dtx} file is a self-extracting
% \docstrip\ archive. The files are extracted by running the
% \xfile{.dtx} through \plainTeX:
% \begin{quote}
%   \verb|tex tabularht.dtx|
% \end{quote}
%
% \paragraph{TDS.} Now the different files must be moved into
% the different directories in your installation TDS tree
% (also known as \xfile{texmf} tree):
% \begin{quote}
% \def\t{^^A
% \begin{tabular}{@{}>{\ttfamily}l@{ $\rightarrow$ }>{\ttfamily}l@{}}
%   tabularht.sty & tex/latex/oberdiek/tabularht.sty\\
%   tabularht.pdf & doc/latex/oberdiek/tabularht.pdf\\
%   tabularht-example1.tex & doc/latex/oberdiek/tabularht-example1.tex\\
%   tabularht-example2.tex & doc/latex/oberdiek/tabularht-example2.tex\\
%   tabularht.dtx & source/latex/oberdiek/tabularht.dtx\\
% \end{tabular}^^A
% }^^A
% \sbox0{\t}^^A
% \ifdim\wd0>\linewidth
%   \begingroup
%     \advance\linewidth by\leftmargin
%     \advance\linewidth by\rightmargin
%   \edef\x{\endgroup
%     \def\noexpand\lw{\the\linewidth}^^A
%   }\x
%   \def\lwbox{^^A
%     \leavevmode
%     \hbox to \linewidth{^^A
%       \kern-\leftmargin\relax
%       \hss
%       \usebox0
%       \hss
%       \kern-\rightmargin\relax
%     }^^A
%   }^^A
%   \ifdim\wd0>\lw
%     \sbox0{\small\t}^^A
%     \ifdim\wd0>\linewidth
%       \ifdim\wd0>\lw
%         \sbox0{\footnotesize\t}^^A
%         \ifdim\wd0>\linewidth
%           \ifdim\wd0>\lw
%             \sbox0{\scriptsize\t}^^A
%             \ifdim\wd0>\linewidth
%               \ifdim\wd0>\lw
%                 \sbox0{\tiny\t}^^A
%                 \ifdim\wd0>\linewidth
%                   \lwbox
%                 \else
%                   \usebox0
%                 \fi
%               \else
%                 \lwbox
%               \fi
%             \else
%               \usebox0
%             \fi
%           \else
%             \lwbox
%           \fi
%         \else
%           \usebox0
%         \fi
%       \else
%         \lwbox
%       \fi
%     \else
%       \usebox0
%     \fi
%   \else
%     \lwbox
%   \fi
% \else
%   \usebox0
% \fi
% \end{quote}
% If you have a \xfile{docstrip.cfg} that configures and enables \docstrip's
% TDS installing feature, then some files can already be in the right
% place, see the documentation of \docstrip.
%
% \subsection{Refresh file name databases}
%
% If your \TeX~distribution
% (\teTeX, \mikTeX, \dots) relies on file name databases, you must refresh
% these. For example, \teTeX\ users run \verb|texhash| or
% \verb|mktexlsr|.
%
% \subsection{Some details for the interested}
%
% \paragraph{Unpacking with \LaTeX.}
% The \xfile{.dtx} chooses its action depending on the format:
% \begin{description}
% \item[\plainTeX:] Run \docstrip\ and extract the files.
% \item[\LaTeX:] Generate the documentation.
% \end{description}
% If you insist on using \LaTeX\ for \docstrip\ (really,
% \docstrip\ does not need \LaTeX), then inform the autodetect routine
% about your intention:
% \begin{quote}
%   \verb|latex \let\install=y% \iffalse meta-comment
%
% File: tabularht.dtx
% Version: 2016/05/16 v2.6
% Info: Tabular with height specified
%
% Copyright (C) 2005-2007 by
%    Heiko Oberdiek <heiko.oberdiek at googlemail.com>
%    2016
%    https://github.com/ho-tex/oberdiek/issues
%
% This work may be distributed and/or modified under the
% conditions of the LaTeX Project Public License, either
% version 1.3c of this license or (at your option) any later
% version. This version of this license is in
%    https://www.latex-project.org/lppl/lppl-1-3c.txt
% and the latest version of this license is in
%    https://www.latex-project.org/lppl.txt
% and version 1.3 or later is part of all distributions of
% LaTeX version 2005/12/01 or later.
%
% This work has the LPPL maintenance status "maintained".
%
% The Current Maintainers of this work are
% Heiko Oberdiek and the Oberdiek Package Support Group
% https://github.com/ho-tex/oberdiek/issues
%
% This work consists of the main source file tabularht.dtx
% and the derived files
%    tabularht.sty, tabularht.pdf, tabularht.ins, tabularht.drv,
%    tabularht-example1.tex, tabularht-example2.tex.
%
% Distribution:
%    CTAN:macros/latex/contrib/oberdiek/tabularht.dtx
%    CTAN:macros/latex/contrib/oberdiek/tabularht.pdf
%
% Unpacking:
%    (a) If tabularht.ins is present:
%           tex tabularht.ins
%    (b) Without tabularht.ins:
%           tex tabularht.dtx
%    (c) If you insist on using LaTeX
%           latex \let\install=y\input{tabularht.dtx}
%        (quote the arguments according to the demands of your shell)
%
% Documentation:
%    (a) If tabularht.drv is present:
%           latex tabularht.drv
%    (b) Without tabularht.drv:
%           latex tabularht.dtx; ...
%    The class ltxdoc loads the configuration file ltxdoc.cfg
%    if available. Here you can specify further options, e.g.
%    use A4 as paper format:
%       \PassOptionsToClass{a4paper}{article}
%
%    Programm calls to get the documentation (example):
%       pdflatex tabularht.dtx
%       makeindex -s gind.ist tabularht.idx
%       pdflatex tabularht.dtx
%       makeindex -s gind.ist tabularht.idx
%       pdflatex tabularht.dtx
%
% Installation:
%    TDS:tex/latex/oberdiek/tabularht.sty
%    TDS:doc/latex/oberdiek/tabularht.pdf
%    TDS:doc/latex/oberdiek/tabularht-example1.tex
%    TDS:doc/latex/oberdiek/tabularht-example2.tex
%    TDS:source/latex/oberdiek/tabularht.dtx
%
%<*ignore>
\begingroup
  \catcode123=1 %
  \catcode125=2 %
  \def\x{LaTeX2e}%
\expandafter\endgroup
\ifcase 0\ifx\install y1\fi\expandafter
         \ifx\csname processbatchFile\endcsname\relax\else1\fi
         \ifx\fmtname\x\else 1\fi\relax
\else\csname fi\endcsname
%</ignore>
%<*install>
\input docstrip.tex
\Msg{************************************************************************}
\Msg{* Installation}
\Msg{* Package: tabularht 2016/05/16 v2.6 Tabular with height specified (HO)}
\Msg{************************************************************************}

\keepsilent
\askforoverwritefalse

\let\MetaPrefix\relax
\preamble

This is a generated file.

Project: tabularht
Version: 2016/05/16 v2.6

Copyright (C) 2005-2007 by
   Heiko Oberdiek <heiko.oberdiek at googlemail.com>

This work may be distributed and/or modified under the
conditions of the LaTeX Project Public License, either
version 1.3c of this license or (at your option) any later
version. This version of this license is in
   https://www.latex-project.org/lppl/lppl-1-3c.txt
and the latest version of this license is in
   https://www.latex-project.org/lppl.txt
and version 1.3 or later is part of all distributions of
LaTeX version 2005/12/01 or later.

This work has the LPPL maintenance status "maintained".

The Current Maintainers of this work are
Heiko Oberdiek and the Oberdiek Package Support Group
https://github.com/ho-tex/oberdiek/issues


This work consists of the main source file tabularht.dtx
and the derived files
   tabularht.sty, tabularht.pdf, tabularht.ins, tabularht.drv,
   tabularht-example1.tex, tabularht-example2.tex.

\endpreamble
\let\MetaPrefix\DoubleperCent

\generate{%
  \file{tabularht.ins}{\from{tabularht.dtx}{install}}%
  \file{tabularht.drv}{\from{tabularht.dtx}{driver}}%
  \usedir{tex/latex/oberdiek}%
  \file{tabularht.sty}{\from{tabularht.dtx}{package}}%
  \usedir{doc/latex/oberdiek}%
  \file{tabularht-example1.tex}{\from{tabularht.dtx}{example1}}%
  \file{tabularht-example2.tex}{\from{tabularht.dtx}{example2}}%
  \nopreamble
  \nopostamble
%  \usedir{source/latex/oberdiek/catalogue}%
%  \file{tabularht.xml}{\from{tabularht.dtx}{catalogue}}%
}

\catcode32=13\relax% active space
\let =\space%
\Msg{************************************************************************}
\Msg{*}
\Msg{* To finish the installation you have to move the following}
\Msg{* file into a directory searched by TeX:}
\Msg{*}
\Msg{*     tabularht.sty}
\Msg{*}
\Msg{* To produce the documentation run the file `tabularht.drv'}
\Msg{* through LaTeX.}
\Msg{*}
\Msg{* Happy TeXing!}
\Msg{*}
\Msg{************************************************************************}

\endbatchfile
%</install>
%<*ignore>
\fi
%</ignore>
%<*driver>
\NeedsTeXFormat{LaTeX2e}
\ProvidesFile{tabularht.drv}%
  [2016/05/16 v2.6 Tabular with height specified (HO)]%
\documentclass{ltxdoc}
\usepackage{holtxdoc}[2011/11/22]
\begin{document}
  \DocInput{tabularht.dtx}%
\end{document}
%</driver>
% \fi
%
%
% \CharacterTable
%  {Upper-case    \A\B\C\D\E\F\G\H\I\J\K\L\M\N\O\P\Q\R\S\T\U\V\W\X\Y\Z
%   Lower-case    \a\b\c\d\e\f\g\h\i\j\k\l\m\n\o\p\q\r\s\t\u\v\w\x\y\z
%   Digits        \0\1\2\3\4\5\6\7\8\9
%   Exclamation   \!     Double quote  \"     Hash (number) \#
%   Dollar        \$     Percent       \%     Ampersand     \&
%   Acute accent  \'     Left paren    \(     Right paren   \)
%   Asterisk      \*     Plus          \+     Comma         \,
%   Minus         \-     Point         \.     Solidus       \/
%   Colon         \:     Semicolon     \;     Less than     \<
%   Equals        \=     Greater than  \>     Question mark \?
%   Commercial at \@     Left bracket  \[     Backslash     \\
%   Right bracket \]     Circumflex    \^     Underscore    \_
%   Grave accent  \`     Left brace    \{     Vertical bar  \|
%   Right brace   \}     Tilde         \~}
%
% \GetFileInfo{tabularht.drv}
%
% \title{The \xpackage{tabularht} package}
% \date{2016/05/16 v2.6}
% \author{Heiko Oberdiek\thanks
% {Please report any issues at \url{https://github.com/ho-tex/oberdiek/issues}}}
%
% \maketitle
%
% \begin{abstract}
% This package defines some environments that adds
% a height specification to tabular and array.
% \end{abstract}
%
% \tableofcontents
%
% \section{Usage}
%
% \begin{quote}
%   |\usepackage{tabularht}|
% \end{quote}
% The package provides the following environments
% that extend the tabular/array environment by
% a height specification as first argument:
% \begin{itemize}
% \item |tabularht|, |tabularht*|
% \item |arrayht|
% \item |tabularhtx|
%       (if package \xpackage{tabularx} is loaded)
% \end{itemize}
% The height argument allows a length specification,
% package \xpackage{calc} is supported if used. This means,
% the tabular will have the specified height.
% You can also use the prefixes |to=| and |spread=|.
% |to=| is the default, |spread=| means, the natural
% height of the tabular box is changed by the length
% after |spread=|.
%
% Examples:
% \begin{quote}
% \begin{tabular}{@{}l@{ $\rightarrow$ }l@{}}
%   |\begin{tabularht}{1in}|
%   & height is 1in\\
%   |\begin{tabularht}{to=1in}|
%   & height is 1in\\
%   |\begin{tabularht}{spread=0pt}|
%   & natural height, same as |\begin{tabular}|\\
%   |\begin{tabularht}{spread=1in}|
%   & natural height increased by 1in
% \end{tabular}
% \end{quote}
%
% Hint:  See also package \xpackage{tabularkv}, it provides an interface,
%        where most parameters for the environments can be given
%        by key-value pairs.
%
% \begin{declcs}{interrowspace}\,|{|\dots|}|
% \end{declcs}
% Adds space between table rows. It is essentially
% the same as |\noalign{\vspace{...}}|.
% \begin{declcs}{interrowfill}
% \end{declcs}
% Short for |\interrowspace{\fill}|
% \begin{declcs}{interrowstart}\,\dots\,\cs{interrowstop}
% \end{declcs}
% Marker commands, useful for option \xoption{vlines}.
%
% \subsection{Option \xoption{vlines}}
%
% Warning: This stuff is experimental.
%
% \vspace{2ex}
% \noindent
% Vertical lines are interrupted, if space is inserted
% in \cs{noalign}, \cs{interrowspace}, \cs{addlinespace}
% (\xpackage{booktabs}),
% between double \cs{hline}s. This option tries to detect
% and add the vertical lines. The lines in a tabular
% with \xpackage{tabularht} support (environments of this package)
% are numbered from left to right. The gap that is
% controlled by \cs{interrowspace} or inbetween \cs{interrowstart}
% and \cs{interrowstop} is then filled with the detected
% vertical lines.
%
% If only a limited selection of the lines should be
% drawn, the commands know an optional argument with a
% list of line numbers, e.g.
% \begin{quote}
%\begin{verbatim}
%\begin{tabularht}{50mm}{|l|l|}
%  Hello & World\\
%  \interrowfill[1,3]
%  Foo & Bar
%\end{tabularht}
%\end{verbatim}
% \end{quote}
% There are three lines, but the middle line is not drawn
% in the gap between the first and second row.
% Zero can be used to suppress all lines:
% \begin{quote}
% |\interrowspace[0]{10mm}|
% \end{quote}
% The syntax of the commands with the optional argument
% with the line number list \meta{list}. \meta{list} is a comma
% separated list of numbers, \meta{height} means the
% height specification described above with the optional
% prefixes |to=| or |spread=|.
% \begin{quote}
%   \cs{interrowspace}\,|[|\meta{list}|]|\,|{|\meta{height}|}|\\
%   \cs{interrowfill}\,|[|\meta{list}|]|\\
%   \cs{interrowstart}\,|[|\meta{list}|]|\,\dots\,\cs{interrowstop}
% \end{quote}
%
% Option \xoption{vlines} is driver dependent and uses \eTeX\ features.
% \begin{description}
% \item[\xoption{pdftex}:] \pdfTeX\ in PDF mode.
%   Here the positions of the lines are written with the
%   help of the \cs{pdfsavepos} feature into the \xfile{.aux} file(s).
%   Therefore you need two LaTeX runs to get the lines.
% \item[\xoption{dvips}:]
%   Here, PostScript's currentpoint it used to get
%   the line positions. The lines are then drawn at the end
%   of the page. Thus one \LaTeX/dvips run is sufficient
%   for this option.
% \item[Other drivers:]\mbox{}
%   \begin{description}
%   \item[PostScript drivers:] probably possible, an end of page
%     hook would be nice.
%   \item[\VTeX:] with GeX (PostScript interpreter) probably possible.
%   \item[dvipdfm:] no idea. The big problem is, how to get the
%     current position?
%   \end{description}
% \end{description}
%
% \subsection{Limitations}
%
% \begin{itemize}
% \item
%   Vertical lines are interrupted by |\noalign{\vfill}|.
% \end{itemize}
%
% \subsection{Compatibility}
%
% \begin{itemize}
% \item
%   \xpackage{array}, \xpackage{delarray}, \xpackage{tabularx}
%   are supported.
% \item There can be problems with packages that
%   redefine \cs{@array} (or \cs{@@array}, \cs{@tabarray}) and
%   \cs{@arrayrule} (for option \xoption{vlines}).
% \item \xpackage{colortbl}: it should at least work, but there isn't
%   support for filling the gaps with color, neither
%   the rules nor the backgrounds.
% \end{itemize}
%
% \subsection{Examples}
%
% \subsubsection{Example 1}
%
%    \begin{macrocode}
%<*example1>
\documentclass{article}
\usepackage{tabularht}

\begin{document}
\fbox{%
  \begin{tabularht*}{1in}{4in}{@{}l@{\extracolsep{\fill}}r@{}}%
    upper left corner & upper right corner\\%
    \noalign{\vfill}%
    \multicolumn{2}{@{}c@{}}{bounding box}\\%
    \noalign{\vfill}%
    lower left corner & lower right corner\\%
  \end{tabularht*}%
}
\end{document}
%</example1>
%    \end{macrocode}
%
% \subsubsection{Example 2}
%
%    \begin{macrocode}
%<*example2>
\documentclass{article}
\usepackage{booktabs}
\usepackage[dvips,vlines]{tabularht}

\begin{document}

\begin{tabularht}{spread=0pt}{|l|l|}
  \hline
  First&Line\\%
  \hline
\interrowstart
  \addlinespace[10mm]%
\interrowstop
  \hline
  Second&Line\\%
\interrowstart
  \hline
  \hline
\interrowstop
  Third&Line\\%
  \hline
\interrowspace{10mm}
  \hline
  Fourth&Line\\%
  \hline
\end{tabularht}

\end{document}
%</example2>
%    \end{macrocode}
%
% \StopEventually{
% }
%
% \section{Implementation}
%
%    \begin{macrocode}
%<*package>
%    \end{macrocode}
%    Package identification.
%    \begin{macrocode}
\NeedsTeXFormat{LaTeX2e}
\ProvidesPackage{tabularht}%
  [2016/05/16 v2.6 Tabular with height specified (HO)]
%    \end{macrocode}
%
% \subsection{Environments}
%
%    \begin{macrocode}
\let\@toarrayheight\@empty
\let\tabH@array@init\@empty

\toks@={%
  \begingroup
    \long\def\x#1\vcenter\fi\fi\bgroup#2\@sharp#3#4\@nil{%
      \endgroup
      \gdef\@array[##1]##2{%
        \tabH@array@init
        #1%
        \vcenter\fi\fi
        \@toarrayheight
        \bgroup
        \let\@toarrayheight\@empty
        #2\@sharp###3#4%
      }%
    }%
  \expandafter\x\@array[#1]{#2}\@nil % hash-ok
}
\edef\tabH@patch@array{\the\toks@}
\def\tabH@patch@@array{%
  \ifx\@array\@@array
    \def\reserved@a{\let\@@array\@array}%
  \else
    \let\reserved@a\relax
  \fi
  \tabH@patch@array
  \reserved@a
}
\tabH@patch@@array

\@ifpackageloaded{array}{}{%
  \AtBeginDocument{%
    \@ifpackageloaded{array}{%
      \tabH@patch@@array
    }{}%
  }%
}

\def\tabH@setheight#1{%
  \tabH@@setheight#1==\@nil
}
\def\tabH@@setheight#1=#2=#3\@nil{%
  \ifx\\#2#3\\%
    \setlength{\dimen@}{#1}%
    \edef\@toarrayheight{to\the\dimen@}%
  \else
    \edef\tabH@temp{\zap@space#1 \@empty}%
    \ifx\tabH@temp\tabH@to
    \else
      \ifx\tabH@temp\tabH@spread
      \else
        \PackageError{tabularht}{%
          Unknown height specifier %
          `\expandafter\strip@prefix\meaning\tabH@temp'%
        }{%
          The height dimension for tabular height can be prefixed%
          \MessageBreak
          with `to=' or `spread=', default is `to='.%
        }%
        \let\tabH@temp\tabH@to
      \fi
    \fi
    \setlength{\dimen@}{#2}%
    \edef\@toarrayheight{\tabH@temp\the\dimen@}%
  \fi
}
\def\tabH@to{to}
\def\tabH@spread{spread}
%    \end{macrocode}
%    First argument is the height of the table,
%    then the orignal arguments for tabular follow.
%    \begin{macrocode}
\newenvironment{tabularht}[1]{%
  \tabH@setheight{#1}%
  \tabular
}{%
  \endtabular
}

\newenvironment{tabularht*}[1]{%
  \tabH@setheight{#1}%
  \@nameuse{tabular*}%
}{%
  \@nameuse{endtabular*}%
}

\newenvironment{tabularhtx}[1]{%
  \tabH@setheight{#1}%
  \tabularx
}{%
  \endtabularx
}

\newenvironment{arrayht}[1]{%
  \tabH@setheight{#1}%
  \array
}{%
  \endarray
}

\def\interrowspace{%
  \noalign\bgroup
    \tabH@interrowspace
}
\newcommand*{\tabH@interrowspace}[2][]{%
    \tabH@vspace{#1}{#2}%
  \egroup
}
\def\interrowfill{%
  \noalign\bgroup
    \tabH@interrowfill
}
\newcommand*{\tabH@interrowfill}[1][]{%
    \tabH@vspace{#1}{\fill}%
  \egroup
}
\def\tabH@vspace#1#2{%
  \tabH@vspace@start{#1}%
  \vspace{#2}%
  \tabH@vspace@stop
}
\let\tabH@vspace@start\@gobble
\let\tabH@vspace@stop\@empty

\newcommand*{\interrowstart}{%
  \noalign\bgroup
    \tabH@interrowstart
}
\newcommand*{\tabH@interrowstart}[1][]{%
    \tabH@vspace@start{#1}%
  \egroup
}
\newcommand*{\interrowstop}{%
  \noalign{\tabH@vspace@stop}%
}
%    \end{macrocode}
%
% \subsection{Options}
%
%    \begin{macrocode}
\providecommand*{\tabH@driver}{}

\DeclareOption{vlines}{%
  \let\tabH@temp\relax
}
\DeclareOption{pdftex}{}
\DeclareOption{dvips}{%
  \def\tabH@driver{dvips}%
}
\ProcessOptions*\relax

\ifx\tabH@temp\relax
\else
  \expandafter\endinput
\fi

\begingroup
  \@ifundefined{eTeXversion}{%
    \PackageError{tabularht}{%
      Option `vlines' requires eTeX%
    }{%
      Use of eTeX is recommended for LaTeX, see ltnews16.%
    }%
    \endgroup
    \endinput
  }{}%
\endgroup
%    \end{macrocode}
%
% \subsection{Option vlines, driver independent stuff}
%
%    \begin{macrocode}
\begingroup
  \let\@addtoreset\@gobbletwo
  \newcounter{tabH@unique}%
\endgroup
\let\tabH@currenttab\@empty

\def\tabH@array@init{%
  \ifx\@toarrayheight\@empty
    % ignore vertical lines of nested tabular environments
    \let\tabH@currenttab\@empty
  \else
    \stepcounter{tabH@unique}%
    \edef\tabH@currenttab{\the\c@tabH@unique}%
  \fi
}

\renewcommand*{\@arrayrule}{%
  \@addtopreamble{%
    \hskip -.5\arrayrulewidth
    \ifx\tabH@currenttab\@empty
    \else
      \tabH@vrule{\tabH@currenttab}%
    \fi
    \begingroup
      \expandafter\ifx\csname CT@arc@\endcsname\relax
      \else
        \expandafter\CT@arc@
      \fi
      \vline
    \endgroup
    \hskip -.5\arrayrulewidth
  }%
}
\let\tabH@arrayrule\@arrayrule
\AtBeginDocument{%
  \@ifpackageloaded{colortbl}{%
    \let\@arrayrule\tabH@arrayrule
  }{}%
}

\let\tabH@vrule\@gobble
%    \end{macrocode}
%
% \subsection{Driver pdftex}
%
%    \begin{macrocode}
\RequirePackage{ifpdf}
\ifpdf
  \begingroup
    \@ifundefined{pdfsavepos}{%
      \PackageError{tabularht}{%
        Your pdfTeX is too old%
      }{%
        \string\pdfsavepos\space is missing.%
      }%
      \endgroup
      \csname fi\endcsname
      \endinput
    }{}%

    \let\on@line\@empty
    \PackageInfo{tabularht}{%
      Using driver `pdftex' because of pdfTeX in PDF mode%
    }%
  \endgroup

  \protected\def\tabH@vrule#1{%
    \if@filesw
      \pdfsavepos
      \protected@write\@auxout{%
        \let\tabH@lastxpos\relax
      }{%
        \tabH@aux@vrule{#1}{\tabH@lastxpos}%
      }%
    \fi
  }%

  \def\tabH@lastxpos{\the\pdflastxpos}%
  \def\tabH@lastypos{\the\pdflastypos}%

  % The .aux file contains three commands:
  % \tabH@aux@vrule{tabular id}{x position}
  % \tabH@aux@vstart{tabular id}{row id}{x position}{y position}
  % \tabH@aux@vstop{y position}
  %
  \AtBeginDocument{%
    % The .aux files are read the first time before
    % \AtBeginDocument and later at \end{document}.
    % \tabH@aux@done is a marker to distinguish
    % between these two readings. Only in the first
    % case we need the \tabH@aux@... commands.
    \let\tabH@aux@done\@empty
    \if@filesw
      \immediate\write\@mainaux{%
        \@percentchar\@percentchar BeginProlog: tabularht%
      }%
      % items in the aux file are executed,
      % if tabularht is loaded
      % and during the aux file read at \begin{document} only
      \immediate\write\@mainaux{%
        \detokenize{%
          % the \tabH@aux@... commands are needed only if
          % tabularht is loaded with driver pdftex.
          \@ifundefined{tabH@aux@vrule}\@secondoftwo\@firstofone
          {%
            % disable commands except for the first .aux files reading
            \@ifundefined{tabH@aux@done}\@gobble\@firstofone
          }%
          {%
            \let\tabH@aux@vrule\@gobbletwo
            \let\tabH@aux@vstart\@gobblefour
            \let\tabH@aux@vstop\@gobble
          }%
        }%
      }%
      \immediate\write\@mainaux{%
        \@percentchar\@percentchar EndProlog: tabularht%
      }%
    \fi
  }%

  % the x positions of vrules are stored in
  % \tabH@<tabcount>list with distinct values
  \protected\def\tabH@aux@vrule#1#2{%
    \@ifundefined{tabH@#1list}{%
      \expandafter\xdef\csname tabH@#1list\endcsname{%
        \noexpand\do{#2}%
      }%
    }{%
      \begingroup
        \def\x{#2}%
        \let\y\@undefined
        \let\do\tabH@do@add
        \expandafter\xdef\csname tabH@#1list\endcsname{%
          \csname tabH@#1list\endcsname\@empty
          \ifx\y\@undefined
            \noexpand\do{\x}%
          \fi
        }%
      \endgroup
    }%
  }%
  \def\tabH@do@add#1{%
    \ifx\y\@undefined
      \ifnum#1<\x\space
      \else
        \expandafter\ifx\csname y\endcsname\relax\fi
        \ifnum#1>\x\space
          \noexpand\do{\x}%
        \fi
      \fi
    \fi
    \noexpand\do{#1}%
  }%

  \def\tabH@vspace@start#1{%
    \if@filesw
      \stepcounter{tabH@unique}%
      \edef\tabH@currentrow{\the\c@tabH@unique}%
      \pdfsavepos
      \protected@write\@auxout{%
        \let\tabH@lastxpos\relax
        \let\tabH@lastypos\relax
      }{%
        \tabH@aux@vstart{\tabH@currenttab}{\tabH@currentrow}%
                        {\tabH@lastxpos}{\tabH@lastypos}%
      }%
    \fi
    \begingroup
      \edef\a{tabH@\tabH@currenttab row\tabH@currentrow}%
      \expandafter\let\expandafter\x\csname\a x\endcsname
      \ifx\x\relax
      \else
        \expandafter\let\expandafter\y\csname\a y\endcsname
        \expandafter\let\expandafter\l
            \csname tabH@\tabH@currenttab list\endcsname
        \ifx\l\relax
        \else
          \def\f{#1}%
          \ifx\f\@empty
            \let\do\tabH@do@set
          \else
            \count@=\z@
            \let\do\tabH@do@filter
          \fi
          \setbox\z@=\hbox{\l}%
          \wd\z@=\z@
          \dp\z@=\z@
          \copy\z@
        \fi
      \fi
    \endgroup
  }%
  \def\tabH@vspace@stop{%
    \if@filesw
      \pdfsavepos
      \protected@write\@auxout{%
        \let\tabH@lastypos\relax
      }{%
        \tabH@aux@vstop{\tabH@lastypos}%
      }%
    \fi
  }%
  \def\tabH@do@set#1{%
    \hbox to \z@{%
      \hskip \dimexpr #1sp - \x sp\relax
      \vrule \@width\arrayrulewidth
             \@depth\dimexpr \y sp\relax
      \hss
    }%
  }%
  \def\tabH@do@filter{%
    \@tempswafalse
    \advance\count@\@ne
    \@for\e:=\f\do{%
      \ifnum\e=\count@
        \@tempswatrue
      \fi
    }%
    \if@tempswa
      \expandafter\tabH@do@set
    \else
      \expandafter\@gobble
    \fi
  }%

  \protected\def\tabH@aux@vstart#1#2#3#4{%
    \def\tabH@current@vstart{{#1}{#2}{#3}{#4}}%
  }%
  \protected\def\tabH@aux@vstop{%
    \expandafter\tabH@aux@v\tabH@current@vstart
  }%
  \def\tabH@aux@v#1#2#3#4#5{%
    \expandafter\gdef\csname tabH@#1row#2x\endcsname{#3}%
    \expandafter\xdef\csname tabH@#1row#2y\endcsname{%
      \the\numexpr #4 - #5\relax
    }%
  }%

  \csname fi\endcsname
  \endinput

\fi
%    \end{macrocode}
%
% \subsection{DVI drivers}
%
%    \begin{macrocode}
\ifx\tabH@driver\@empty
  \PackageError{tabularht}{%
    Missing DVI driver, option `vlines' disabled%
  }{%
    Supported DVI drivers: dvips.%
  }%
  \expandafter\endinput
\fi

\def\tabH@driver@dvips{%
  \def\tabH@literalps##1{\special{ps:SDict begin ##1 end}}%
  \def\tabH@headerps##1{\special{! ##1}}%
}

\@onelevel@sanitize\tabH@driver
\@ifundefined{tabH@driver@\tabH@driver}{%
  \PackageError{tabularht}{%
    Unsupported driver `\tabH@driver'%
  }{%
    Supported DVI drivers: dvips.%
  }%
  \endinput
}{}

\begingroup
  \let\on@line\@empty
  \PackageInfo{tabularht}{%
    Using driver `\tabH@driver'%
  }%
\endgroup
\csname tabH@driver@\tabH@driver\endcsname

\protected\def\tabH@vrule#1#2\vrule#3\arrayrulewidth{%
  #2% \fi or empty
  % hack to get rid of maxdrift rounding of dvips,
  % thus simulate a large motion
  \kern1in\relax
  \tabH@literalps{%
    #1 tabH.vrule %
    Resolution neg 0 translate%
  }%
  \vrule#3\arrayrulewidth
  \tabH@literalps{Resolution 0 translate}%
  \kern-1in\relax
}

\def\tabH@vspace@start#1{%
  \begingroup
    \let\y\@empty
    \@for\x:=#1\do{%
      \ifx\y\@empty
        \edef\y{\x}%
      \else
        \edef\y{\y\space\x}%
      \fi
    }%
    \tabH@literalps{\tabH@currenttab[\y]currentpoint exch pop}%
  \endgroup
}
\def\tabH@vspace@stop{%
  \tabH@literalps{%
    currentpoint exch pop %
    \number\dimexpr\arrayrulewidth\relax\space
    tabH.vspace%
  }%
}

\tabH@headerps{%
  userdict begin%
    /tabH.list 10 dict def%
    /tabH.job [] def %
  end%
  /tabH.vrule{%
    10 string cvs cvn dup tabH.list exch known{%
      tabH.list exch dup [ exch tabH.list exch get %
      currentpoint pop round exch true exch{%
        % tabH.list key [ ... x true i
        % tabH.list key [ ... false i
        exch{%
          % ... [ ... x i
          2 copy lt{false}{%
            2 copy eq{pop false}{exch true}ifelse%
          }ifelse%
        }{false}ifelse%
      }forall %
      pop%
      ]put%
    }{%
      tabH.list exch[currentpoint pop round]put%
    }ifelse%
  }bind def%
  % <tab num> <cols array> <ytop> <ybottom> <rulewidth[sp]>
  /tabH.vspace{%
    userdict begin %
      10 dict dup begin %
        exch 65536 div Resolution mul 72.27 div %
        % dvips uses a poor man's ceil function
        % see dopage.c before "drawrule": (int)(... + 0.9999999)
        0.9999999 add truncate%
        /rulewidth exch def %
        exch/ybottom exch def %
        exch/ytop exch def %
        exch/cols exch def %
        exch/tabkey exch 10 string cvs cvn def %
      end%
      /tabH.job exch[exch userdict/tabH.job get aload pop]def %
    end%
  }bind def %
  % Now we do the work at the end of the page.
  % Unhappily "eop-hook" cannot be used, because "eop"
  % executes "restore" before, so that all data are lost.
  TeXDict begin%
    /eop%
    [%
      {%
        tabH.job{%
          begin%
           /colarray %
             tabH.list tabkey known{tabH.list tabkey get}{[]}ifelse %
           def %
           cols length 0 eq not{%
             /colarray[%
               cols{1 sub %
                 dup 0 lt{pop}{%
                   dup colarray length ge{pop}{%
                     colarray exch get%
                   }ifelse%
                 }ifelse%
               }forall%
             ]def%
           }if %
           colarray{%
             % (rulewidth) == rulewidth == % debug
             Resolution sub %
             ytop rulewidth ytop ybottom sub v%
           }forall %
          end%
        }forall%
        % tabH.list{== ==}forall % debug
      }bind aload pop %
      TeXDict /eop get aload pop%
    ]cvx def %
  end%
}
%    \end{macrocode}
%
%    \begin{macrocode}
%</package>
%    \end{macrocode}
%
% \section{Installation}
%
% \subsection{Download}
%
% \paragraph{Package.} This package is available on
% CTAN\footnote{\CTANpkg{tabularht}}:
% \begin{description}
% \item[\CTAN{macros/latex/contrib/oberdiek/tabularht.dtx}] The source file.
% \item[\CTAN{macros/latex/contrib/oberdiek/tabularht.pdf}] Documentation.
% \end{description}
%
%
% \paragraph{Bundle.} All the packages of the bundle `oberdiek'
% are also available in a TDS compliant ZIP archive. There
% the packages are already unpacked and the documentation files
% are generated. The files and directories obey the TDS standard.
% \begin{description}
% \item[\CTANinstall{install/macros/latex/contrib/oberdiek.tds.zip}]
% \end{description}
% \emph{TDS} refers to the standard ``A Directory Structure
% for \TeX\ Files'' (\CTAN{tds/tds.pdf}). Directories
% with \xfile{texmf} in their name are usually organized this way.
%
% \subsection{Bundle installation}
%
% \paragraph{Unpacking.} Unpack the \xfile{oberdiek.tds.zip} in the
% TDS tree (also known as \xfile{texmf} tree) of your choice.
% Example (linux):
% \begin{quote}
%   |unzip oberdiek.tds.zip -d ~/texmf|
% \end{quote}
%
% \paragraph{Script installation.}
% Check the directory \xfile{TDS:scripts/oberdiek/} for
% scripts that need further installation steps.
%
% \subsection{Package installation}
%
% \paragraph{Unpacking.} The \xfile{.dtx} file is a self-extracting
% \docstrip\ archive. The files are extracted by running the
% \xfile{.dtx} through \plainTeX:
% \begin{quote}
%   \verb|tex tabularht.dtx|
% \end{quote}
%
% \paragraph{TDS.} Now the different files must be moved into
% the different directories in your installation TDS tree
% (also known as \xfile{texmf} tree):
% \begin{quote}
% \def\t{^^A
% \begin{tabular}{@{}>{\ttfamily}l@{ $\rightarrow$ }>{\ttfamily}l@{}}
%   tabularht.sty & tex/latex/oberdiek/tabularht.sty\\
%   tabularht.pdf & doc/latex/oberdiek/tabularht.pdf\\
%   tabularht-example1.tex & doc/latex/oberdiek/tabularht-example1.tex\\
%   tabularht-example2.tex & doc/latex/oberdiek/tabularht-example2.tex\\
%   tabularht.dtx & source/latex/oberdiek/tabularht.dtx\\
% \end{tabular}^^A
% }^^A
% \sbox0{\t}^^A
% \ifdim\wd0>\linewidth
%   \begingroup
%     \advance\linewidth by\leftmargin
%     \advance\linewidth by\rightmargin
%   \edef\x{\endgroup
%     \def\noexpand\lw{\the\linewidth}^^A
%   }\x
%   \def\lwbox{^^A
%     \leavevmode
%     \hbox to \linewidth{^^A
%       \kern-\leftmargin\relax
%       \hss
%       \usebox0
%       \hss
%       \kern-\rightmargin\relax
%     }^^A
%   }^^A
%   \ifdim\wd0>\lw
%     \sbox0{\small\t}^^A
%     \ifdim\wd0>\linewidth
%       \ifdim\wd0>\lw
%         \sbox0{\footnotesize\t}^^A
%         \ifdim\wd0>\linewidth
%           \ifdim\wd0>\lw
%             \sbox0{\scriptsize\t}^^A
%             \ifdim\wd0>\linewidth
%               \ifdim\wd0>\lw
%                 \sbox0{\tiny\t}^^A
%                 \ifdim\wd0>\linewidth
%                   \lwbox
%                 \else
%                   \usebox0
%                 \fi
%               \else
%                 \lwbox
%               \fi
%             \else
%               \usebox0
%             \fi
%           \else
%             \lwbox
%           \fi
%         \else
%           \usebox0
%         \fi
%       \else
%         \lwbox
%       \fi
%     \else
%       \usebox0
%     \fi
%   \else
%     \lwbox
%   \fi
% \else
%   \usebox0
% \fi
% \end{quote}
% If you have a \xfile{docstrip.cfg} that configures and enables \docstrip's
% TDS installing feature, then some files can already be in the right
% place, see the documentation of \docstrip.
%
% \subsection{Refresh file name databases}
%
% If your \TeX~distribution
% (\teTeX, \mikTeX, \dots) relies on file name databases, you must refresh
% these. For example, \teTeX\ users run \verb|texhash| or
% \verb|mktexlsr|.
%
% \subsection{Some details for the interested}
%
% \paragraph{Unpacking with \LaTeX.}
% The \xfile{.dtx} chooses its action depending on the format:
% \begin{description}
% \item[\plainTeX:] Run \docstrip\ and extract the files.
% \item[\LaTeX:] Generate the documentation.
% \end{description}
% If you insist on using \LaTeX\ for \docstrip\ (really,
% \docstrip\ does not need \LaTeX), then inform the autodetect routine
% about your intention:
% \begin{quote}
%   \verb|latex \let\install=y\input{tabularht.dtx}|
% \end{quote}
% Do not forget to quote the argument according to the demands
% of your shell.
%
% \paragraph{Generating the documentation.}
% You can use both the \xfile{.dtx} or the \xfile{.drv} to generate
% the documentation. The process can be configured by the
% configuration file \xfile{ltxdoc.cfg}. For instance, put this
% line into this file, if you want to have A4 as paper format:
% \begin{quote}
%   \verb|\PassOptionsToClass{a4paper}{article}|
% \end{quote}
% An example follows how to generate the
% documentation with pdf\LaTeX:
% \begin{quote}
%\begin{verbatim}
%pdflatex tabularht.dtx
%makeindex -s gind.ist tabularht.idx
%pdflatex tabularht.dtx
%makeindex -s gind.ist tabularht.idx
%pdflatex tabularht.dtx
%\end{verbatim}
% \end{quote}
%
% \begin{History}
%   \begin{Version}{2005/09/22 v1.0}
%   \item
%     First public version.
%   \end{Version}
%   \begin{Version}{2005/10/16 v2.0}
%   \item
%     Height specification allows
%     |to=...| or |spread=...|, default is |to=|.
%   \item
%     Option \xoption{vlines} added,
%     drivers \xoption{pdftex} and \xoption{dvips}.
%   \item
%     \cs{interrowspace}, \cs{interrowfil}, and
%     \cs{interrowstart}\dots\cs{interrowstop} added.
%   \end{Version}
%   \begin{Version}{2005/10/18 v2.1}
%   \item
%     Fix for package \xpackage{colortbl}, but the colors
%     of \xpackage{colortbl} remain unsupported.
%   \end{Version}
%   \begin{Version}{2006/02/20 v2.2}
%   \item
%     Code is not changed.
%   \item
%     DTX framework.
%   \end{Version}
%   \begin{Version}{2006/12/22 v2.3}
%   \item
%     Documentation fix.
%   \item
%     Fix in code of option vlines.
%   \end{Version}
%   \begin{Version}{2007/03/21 v2.4}
%   \item
%     Fix: Counter |tabh@unique| must not be changed by \cs{include}.
%   \end{Version}
%   \begin{Version}{2007/04/11 v2.5}
%   \item
%     Line ends sanitized.
%   \end{Version}
%   \begin{Version}{2016/05/16 v2.6}
%   \item
%     Documentation updates.
%   \end{Version}
% \end{History}
%
% \PrintIndex
%
% \Finale
\endinput
|
% \end{quote}
% Do not forget to quote the argument according to the demands
% of your shell.
%
% \paragraph{Generating the documentation.}
% You can use both the \xfile{.dtx} or the \xfile{.drv} to generate
% the documentation. The process can be configured by the
% configuration file \xfile{ltxdoc.cfg}. For instance, put this
% line into this file, if you want to have A4 as paper format:
% \begin{quote}
%   \verb|\PassOptionsToClass{a4paper}{article}|
% \end{quote}
% An example follows how to generate the
% documentation with pdf\LaTeX:
% \begin{quote}
%\begin{verbatim}
%pdflatex tabularht.dtx
%makeindex -s gind.ist tabularht.idx
%pdflatex tabularht.dtx
%makeindex -s gind.ist tabularht.idx
%pdflatex tabularht.dtx
%\end{verbatim}
% \end{quote}
%
% \begin{History}
%   \begin{Version}{2005/09/22 v1.0}
%   \item
%     First public version.
%   \end{Version}
%   \begin{Version}{2005/10/16 v2.0}
%   \item
%     Height specification allows
%     |to=...| or |spread=...|, default is |to=|.
%   \item
%     Option \xoption{vlines} added,
%     drivers \xoption{pdftex} and \xoption{dvips}.
%   \item
%     \cs{interrowspace}, \cs{interrowfil}, and
%     \cs{interrowstart}\dots\cs{interrowstop} added.
%   \end{Version}
%   \begin{Version}{2005/10/18 v2.1}
%   \item
%     Fix for package \xpackage{colortbl}, but the colors
%     of \xpackage{colortbl} remain unsupported.
%   \end{Version}
%   \begin{Version}{2006/02/20 v2.2}
%   \item
%     Code is not changed.
%   \item
%     DTX framework.
%   \end{Version}
%   \begin{Version}{2006/12/22 v2.3}
%   \item
%     Documentation fix.
%   \item
%     Fix in code of option vlines.
%   \end{Version}
%   \begin{Version}{2007/03/21 v2.4}
%   \item
%     Fix: Counter |tabh@unique| must not be changed by \cs{include}.
%   \end{Version}
%   \begin{Version}{2007/04/11 v2.5}
%   \item
%     Line ends sanitized.
%   \end{Version}
%   \begin{Version}{2016/05/16 v2.6}
%   \item
%     Documentation updates.
%   \end{Version}
% \end{History}
%
% \PrintIndex
%
% \Finale
\endinput

%        (quote the arguments according to the demands of your shell)
%
% Documentation:
%    (a) If tabularht.drv is present:
%           latex tabularht.drv
%    (b) Without tabularht.drv:
%           latex tabularht.dtx; ...
%    The class ltxdoc loads the configuration file ltxdoc.cfg
%    if available. Here you can specify further options, e.g.
%    use A4 as paper format:
%       \PassOptionsToClass{a4paper}{article}
%
%    Programm calls to get the documentation (example):
%       pdflatex tabularht.dtx
%       makeindex -s gind.ist tabularht.idx
%       pdflatex tabularht.dtx
%       makeindex -s gind.ist tabularht.idx
%       pdflatex tabularht.dtx
%
% Installation:
%    TDS:tex/latex/oberdiek/tabularht.sty
%    TDS:doc/latex/oberdiek/tabularht.pdf
%    TDS:doc/latex/oberdiek/tabularht-example1.tex
%    TDS:doc/latex/oberdiek/tabularht-example2.tex
%    TDS:source/latex/oberdiek/tabularht.dtx
%
%<*ignore>
\begingroup
  \catcode123=1 %
  \catcode125=2 %
  \def\x{LaTeX2e}%
\expandafter\endgroup
\ifcase 0\ifx\install y1\fi\expandafter
         \ifx\csname processbatchFile\endcsname\relax\else1\fi
         \ifx\fmtname\x\else 1\fi\relax
\else\csname fi\endcsname
%</ignore>
%<*install>
\input docstrip.tex
\Msg{************************************************************************}
\Msg{* Installation}
\Msg{* Package: tabularht 2016/05/16 v2.6 Tabular with height specified (HO)}
\Msg{************************************************************************}

\keepsilent
\askforoverwritefalse

\let\MetaPrefix\relax
\preamble

This is a generated file.

Project: tabularht
Version: 2016/05/16 v2.6

Copyright (C) 2005-2007 by
   Heiko Oberdiek <heiko.oberdiek at googlemail.com>

This work may be distributed and/or modified under the
conditions of the LaTeX Project Public License, either
version 1.3c of this license or (at your option) any later
version. This version of this license is in
   https://www.latex-project.org/lppl/lppl-1-3c.txt
and the latest version of this license is in
   https://www.latex-project.org/lppl.txt
and version 1.3 or later is part of all distributions of
LaTeX version 2005/12/01 or later.

This work has the LPPL maintenance status "maintained".

The Current Maintainers of this work are
Heiko Oberdiek and the Oberdiek Package Support Group
https://github.com/ho-tex/oberdiek/issues


This work consists of the main source file tabularht.dtx
and the derived files
   tabularht.sty, tabularht.pdf, tabularht.ins, tabularht.drv,
   tabularht-example1.tex, tabularht-example2.tex.

\endpreamble
\let\MetaPrefix\DoubleperCent

\generate{%
  \file{tabularht.ins}{\from{tabularht.dtx}{install}}%
  \file{tabularht.drv}{\from{tabularht.dtx}{driver}}%
  \usedir{tex/latex/oberdiek}%
  \file{tabularht.sty}{\from{tabularht.dtx}{package}}%
  \usedir{doc/latex/oberdiek}%
  \file{tabularht-example1.tex}{\from{tabularht.dtx}{example1}}%
  \file{tabularht-example2.tex}{\from{tabularht.dtx}{example2}}%
  \nopreamble
  \nopostamble
%  \usedir{source/latex/oberdiek/catalogue}%
%  \file{tabularht.xml}{\from{tabularht.dtx}{catalogue}}%
}

\catcode32=13\relax% active space
\let =\space%
\Msg{************************************************************************}
\Msg{*}
\Msg{* To finish the installation you have to move the following}
\Msg{* file into a directory searched by TeX:}
\Msg{*}
\Msg{*     tabularht.sty}
\Msg{*}
\Msg{* To produce the documentation run the file `tabularht.drv'}
\Msg{* through LaTeX.}
\Msg{*}
\Msg{* Happy TeXing!}
\Msg{*}
\Msg{************************************************************************}

\endbatchfile
%</install>
%<*ignore>
\fi
%</ignore>
%<*driver>
\NeedsTeXFormat{LaTeX2e}
\ProvidesFile{tabularht.drv}%
  [2016/05/16 v2.6 Tabular with height specified (HO)]%
\documentclass{ltxdoc}
\usepackage{holtxdoc}[2011/11/22]
\begin{document}
  \DocInput{tabularht.dtx}%
\end{document}
%</driver>
% \fi
%
%
% \CharacterTable
%  {Upper-case    \A\B\C\D\E\F\G\H\I\J\K\L\M\N\O\P\Q\R\S\T\U\V\W\X\Y\Z
%   Lower-case    \a\b\c\d\e\f\g\h\i\j\k\l\m\n\o\p\q\r\s\t\u\v\w\x\y\z
%   Digits        \0\1\2\3\4\5\6\7\8\9
%   Exclamation   \!     Double quote  \"     Hash (number) \#
%   Dollar        \$     Percent       \%     Ampersand     \&
%   Acute accent  \'     Left paren    \(     Right paren   \)
%   Asterisk      \*     Plus          \+     Comma         \,
%   Minus         \-     Point         \.     Solidus       \/
%   Colon         \:     Semicolon     \;     Less than     \<
%   Equals        \=     Greater than  \>     Question mark \?
%   Commercial at \@     Left bracket  \[     Backslash     \\
%   Right bracket \]     Circumflex    \^     Underscore    \_
%   Grave accent  \`     Left brace    \{     Vertical bar  \|
%   Right brace   \}     Tilde         \~}
%
% \GetFileInfo{tabularht.drv}
%
% \title{The \xpackage{tabularht} package}
% \date{2016/05/16 v2.6}
% \author{Heiko Oberdiek\thanks
% {Please report any issues at \url{https://github.com/ho-tex/oberdiek/issues}}}
%
% \maketitle
%
% \begin{abstract}
% This package defines some environments that adds
% a height specification to tabular and array.
% \end{abstract}
%
% \tableofcontents
%
% \section{Usage}
%
% \begin{quote}
%   |\usepackage{tabularht}|
% \end{quote}
% The package provides the following environments
% that extend the tabular/array environment by
% a height specification as first argument:
% \begin{itemize}
% \item |tabularht|, |tabularht*|
% \item |arrayht|
% \item |tabularhtx|
%       (if package \xpackage{tabularx} is loaded)
% \end{itemize}
% The height argument allows a length specification,
% package \xpackage{calc} is supported if used. This means,
% the tabular will have the specified height.
% You can also use the prefixes |to=| and |spread=|.
% |to=| is the default, |spread=| means, the natural
% height of the tabular box is changed by the length
% after |spread=|.
%
% Examples:
% \begin{quote}
% \begin{tabular}{@{}l@{ $\rightarrow$ }l@{}}
%   |\begin{tabularht}{1in}|
%   & height is 1in\\
%   |\begin{tabularht}{to=1in}|
%   & height is 1in\\
%   |\begin{tabularht}{spread=0pt}|
%   & natural height, same as |\begin{tabular}|\\
%   |\begin{tabularht}{spread=1in}|
%   & natural height increased by 1in
% \end{tabular}
% \end{quote}
%
% Hint:  See also package \xpackage{tabularkv}, it provides an interface,
%        where most parameters for the environments can be given
%        by key-value pairs.
%
% \begin{declcs}{interrowspace}\,|{|\dots|}|
% \end{declcs}
% Adds space between table rows. It is essentially
% the same as |\noalign{\vspace{...}}|.
% \begin{declcs}{interrowfill}
% \end{declcs}
% Short for |\interrowspace{\fill}|
% \begin{declcs}{interrowstart}\,\dots\,\cs{interrowstop}
% \end{declcs}
% Marker commands, useful for option \xoption{vlines}.
%
% \subsection{Option \xoption{vlines}}
%
% Warning: This stuff is experimental.
%
% \vspace{2ex}
% \noindent
% Vertical lines are interrupted, if space is inserted
% in \cs{noalign}, \cs{interrowspace}, \cs{addlinespace}
% (\xpackage{booktabs}),
% between double \cs{hline}s. This option tries to detect
% and add the vertical lines. The lines in a tabular
% with \xpackage{tabularht} support (environments of this package)
% are numbered from left to right. The gap that is
% controlled by \cs{interrowspace} or inbetween \cs{interrowstart}
% and \cs{interrowstop} is then filled with the detected
% vertical lines.
%
% If only a limited selection of the lines should be
% drawn, the commands know an optional argument with a
% list of line numbers, e.g.
% \begin{quote}
%\begin{verbatim}
%\begin{tabularht}{50mm}{|l|l|}
%  Hello & World\\
%  \interrowfill[1,3]
%  Foo & Bar
%\end{tabularht}
%\end{verbatim}
% \end{quote}
% There are three lines, but the middle line is not drawn
% in the gap between the first and second row.
% Zero can be used to suppress all lines:
% \begin{quote}
% |\interrowspace[0]{10mm}|
% \end{quote}
% The syntax of the commands with the optional argument
% with the line number list \meta{list}. \meta{list} is a comma
% separated list of numbers, \meta{height} means the
% height specification described above with the optional
% prefixes |to=| or |spread=|.
% \begin{quote}
%   \cs{interrowspace}\,|[|\meta{list}|]|\,|{|\meta{height}|}|\\
%   \cs{interrowfill}\,|[|\meta{list}|]|\\
%   \cs{interrowstart}\,|[|\meta{list}|]|\,\dots\,\cs{interrowstop}
% \end{quote}
%
% Option \xoption{vlines} is driver dependent and uses \eTeX\ features.
% \begin{description}
% \item[\xoption{pdftex}:] \pdfTeX\ in PDF mode.
%   Here the positions of the lines are written with the
%   help of the \cs{pdfsavepos} feature into the \xfile{.aux} file(s).
%   Therefore you need two LaTeX runs to get the lines.
% \item[\xoption{dvips}:]
%   Here, PostScript's currentpoint it used to get
%   the line positions. The lines are then drawn at the end
%   of the page. Thus one \LaTeX/dvips run is sufficient
%   for this option.
% \item[Other drivers:]\mbox{}
%   \begin{description}
%   \item[PostScript drivers:] probably possible, an end of page
%     hook would be nice.
%   \item[\VTeX:] with GeX (PostScript interpreter) probably possible.
%   \item[dvipdfm:] no idea. The big problem is, how to get the
%     current position?
%   \end{description}
% \end{description}
%
% \subsection{Limitations}
%
% \begin{itemize}
% \item
%   Vertical lines are interrupted by |\noalign{\vfill}|.
% \end{itemize}
%
% \subsection{Compatibility}
%
% \begin{itemize}
% \item
%   \xpackage{array}, \xpackage{delarray}, \xpackage{tabularx}
%   are supported.
% \item There can be problems with packages that
%   redefine \cs{@array} (or \cs{@@array}, \cs{@tabarray}) and
%   \cs{@arrayrule} (for option \xoption{vlines}).
% \item \xpackage{colortbl}: it should at least work, but there isn't
%   support for filling the gaps with color, neither
%   the rules nor the backgrounds.
% \end{itemize}
%
% \subsection{Examples}
%
% \subsubsection{Example 1}
%
%    \begin{macrocode}
%<*example1>
\documentclass{article}
\usepackage{tabularht}

\begin{document}
\fbox{%
  \begin{tabularht*}{1in}{4in}{@{}l@{\extracolsep{\fill}}r@{}}%
    upper left corner & upper right corner\\%
    \noalign{\vfill}%
    \multicolumn{2}{@{}c@{}}{bounding box}\\%
    \noalign{\vfill}%
    lower left corner & lower right corner\\%
  \end{tabularht*}%
}
\end{document}
%</example1>
%    \end{macrocode}
%
% \subsubsection{Example 2}
%
%    \begin{macrocode}
%<*example2>
\documentclass{article}
\usepackage{booktabs}
\usepackage[dvips,vlines]{tabularht}

\begin{document}

\begin{tabularht}{spread=0pt}{|l|l|}
  \hline
  First&Line\\%
  \hline
\interrowstart
  \addlinespace[10mm]%
\interrowstop
  \hline
  Second&Line\\%
\interrowstart
  \hline
  \hline
\interrowstop
  Third&Line\\%
  \hline
\interrowspace{10mm}
  \hline
  Fourth&Line\\%
  \hline
\end{tabularht}

\end{document}
%</example2>
%    \end{macrocode}
%
% \StopEventually{
% }
%
% \section{Implementation}
%
%    \begin{macrocode}
%<*package>
%    \end{macrocode}
%    Package identification.
%    \begin{macrocode}
\NeedsTeXFormat{LaTeX2e}
\ProvidesPackage{tabularht}%
  [2016/05/16 v2.6 Tabular with height specified (HO)]
%    \end{macrocode}
%
% \subsection{Environments}
%
%    \begin{macrocode}
\let\@toarrayheight\@empty
\let\tabH@array@init\@empty

\toks@={%
  \begingroup
    \long\def\x#1\vcenter\fi\fi\bgroup#2\@sharp#3#4\@nil{%
      \endgroup
      \gdef\@array[##1]##2{%
        \tabH@array@init
        #1%
        \vcenter\fi\fi
        \@toarrayheight
        \bgroup
        \let\@toarrayheight\@empty
        #2\@sharp###3#4%
      }%
    }%
  \expandafter\x\@array[#1]{#2}\@nil % hash-ok
}
\edef\tabH@patch@array{\the\toks@}
\def\tabH@patch@@array{%
  \ifx\@array\@@array
    \def\reserved@a{\let\@@array\@array}%
  \else
    \let\reserved@a\relax
  \fi
  \tabH@patch@array
  \reserved@a
}
\tabH@patch@@array

\@ifpackageloaded{array}{}{%
  \AtBeginDocument{%
    \@ifpackageloaded{array}{%
      \tabH@patch@@array
    }{}%
  }%
}

\def\tabH@setheight#1{%
  \tabH@@setheight#1==\@nil
}
\def\tabH@@setheight#1=#2=#3\@nil{%
  \ifx\\#2#3\\%
    \setlength{\dimen@}{#1}%
    \edef\@toarrayheight{to\the\dimen@}%
  \else
    \edef\tabH@temp{\zap@space#1 \@empty}%
    \ifx\tabH@temp\tabH@to
    \else
      \ifx\tabH@temp\tabH@spread
      \else
        \PackageError{tabularht}{%
          Unknown height specifier %
          `\expandafter\strip@prefix\meaning\tabH@temp'%
        }{%
          The height dimension for tabular height can be prefixed%
          \MessageBreak
          with `to=' or `spread=', default is `to='.%
        }%
        \let\tabH@temp\tabH@to
      \fi
    \fi
    \setlength{\dimen@}{#2}%
    \edef\@toarrayheight{\tabH@temp\the\dimen@}%
  \fi
}
\def\tabH@to{to}
\def\tabH@spread{spread}
%    \end{macrocode}
%    First argument is the height of the table,
%    then the orignal arguments for tabular follow.
%    \begin{macrocode}
\newenvironment{tabularht}[1]{%
  \tabH@setheight{#1}%
  \tabular
}{%
  \endtabular
}

\newenvironment{tabularht*}[1]{%
  \tabH@setheight{#1}%
  \@nameuse{tabular*}%
}{%
  \@nameuse{endtabular*}%
}

\newenvironment{tabularhtx}[1]{%
  \tabH@setheight{#1}%
  \tabularx
}{%
  \endtabularx
}

\newenvironment{arrayht}[1]{%
  \tabH@setheight{#1}%
  \array
}{%
  \endarray
}

\def\interrowspace{%
  \noalign\bgroup
    \tabH@interrowspace
}
\newcommand*{\tabH@interrowspace}[2][]{%
    \tabH@vspace{#1}{#2}%
  \egroup
}
\def\interrowfill{%
  \noalign\bgroup
    \tabH@interrowfill
}
\newcommand*{\tabH@interrowfill}[1][]{%
    \tabH@vspace{#1}{\fill}%
  \egroup
}
\def\tabH@vspace#1#2{%
  \tabH@vspace@start{#1}%
  \vspace{#2}%
  \tabH@vspace@stop
}
\let\tabH@vspace@start\@gobble
\let\tabH@vspace@stop\@empty

\newcommand*{\interrowstart}{%
  \noalign\bgroup
    \tabH@interrowstart
}
\newcommand*{\tabH@interrowstart}[1][]{%
    \tabH@vspace@start{#1}%
  \egroup
}
\newcommand*{\interrowstop}{%
  \noalign{\tabH@vspace@stop}%
}
%    \end{macrocode}
%
% \subsection{Options}
%
%    \begin{macrocode}
\providecommand*{\tabH@driver}{}

\DeclareOption{vlines}{%
  \let\tabH@temp\relax
}
\DeclareOption{pdftex}{}
\DeclareOption{dvips}{%
  \def\tabH@driver{dvips}%
}
\ProcessOptions*\relax

\ifx\tabH@temp\relax
\else
  \expandafter\endinput
\fi

\begingroup
  \@ifundefined{eTeXversion}{%
    \PackageError{tabularht}{%
      Option `vlines' requires eTeX%
    }{%
      Use of eTeX is recommended for LaTeX, see ltnews16.%
    }%
    \endgroup
    \endinput
  }{}%
\endgroup
%    \end{macrocode}
%
% \subsection{Option vlines, driver independent stuff}
%
%    \begin{macrocode}
\begingroup
  \let\@addtoreset\@gobbletwo
  \newcounter{tabH@unique}%
\endgroup
\let\tabH@currenttab\@empty

\def\tabH@array@init{%
  \ifx\@toarrayheight\@empty
    % ignore vertical lines of nested tabular environments
    \let\tabH@currenttab\@empty
  \else
    \stepcounter{tabH@unique}%
    \edef\tabH@currenttab{\the\c@tabH@unique}%
  \fi
}

\renewcommand*{\@arrayrule}{%
  \@addtopreamble{%
    \hskip -.5\arrayrulewidth
    \ifx\tabH@currenttab\@empty
    \else
      \tabH@vrule{\tabH@currenttab}%
    \fi
    \begingroup
      \expandafter\ifx\csname CT@arc@\endcsname\relax
      \else
        \expandafter\CT@arc@
      \fi
      \vline
    \endgroup
    \hskip -.5\arrayrulewidth
  }%
}
\let\tabH@arrayrule\@arrayrule
\AtBeginDocument{%
  \@ifpackageloaded{colortbl}{%
    \let\@arrayrule\tabH@arrayrule
  }{}%
}

\let\tabH@vrule\@gobble
%    \end{macrocode}
%
% \subsection{Driver pdftex}
%
%    \begin{macrocode}
\RequirePackage{ifpdf}
\ifpdf
  \begingroup
    \@ifundefined{pdfsavepos}{%
      \PackageError{tabularht}{%
        Your pdfTeX is too old%
      }{%
        \string\pdfsavepos\space is missing.%
      }%
      \endgroup
      \csname fi\endcsname
      \endinput
    }{}%

    \let\on@line\@empty
    \PackageInfo{tabularht}{%
      Using driver `pdftex' because of pdfTeX in PDF mode%
    }%
  \endgroup

  \protected\def\tabH@vrule#1{%
    \if@filesw
      \pdfsavepos
      \protected@write\@auxout{%
        \let\tabH@lastxpos\relax
      }{%
        \tabH@aux@vrule{#1}{\tabH@lastxpos}%
      }%
    \fi
  }%

  \def\tabH@lastxpos{\the\pdflastxpos}%
  \def\tabH@lastypos{\the\pdflastypos}%

  % The .aux file contains three commands:
  % \tabH@aux@vrule{tabular id}{x position}
  % \tabH@aux@vstart{tabular id}{row id}{x position}{y position}
  % \tabH@aux@vstop{y position}
  %
  \AtBeginDocument{%
    % The .aux files are read the first time before
    % \AtBeginDocument and later at \end{document}.
    % \tabH@aux@done is a marker to distinguish
    % between these two readings. Only in the first
    % case we need the \tabH@aux@... commands.
    \let\tabH@aux@done\@empty
    \if@filesw
      \immediate\write\@mainaux{%
        \@percentchar\@percentchar BeginProlog: tabularht%
      }%
      % items in the aux file are executed,
      % if tabularht is loaded
      % and during the aux file read at \begin{document} only
      \immediate\write\@mainaux{%
        \detokenize{%
          % the \tabH@aux@... commands are needed only if
          % tabularht is loaded with driver pdftex.
          \@ifundefined{tabH@aux@vrule}\@secondoftwo\@firstofone
          {%
            % disable commands except for the first .aux files reading
            \@ifundefined{tabH@aux@done}\@gobble\@firstofone
          }%
          {%
            \let\tabH@aux@vrule\@gobbletwo
            \let\tabH@aux@vstart\@gobblefour
            \let\tabH@aux@vstop\@gobble
          }%
        }%
      }%
      \immediate\write\@mainaux{%
        \@percentchar\@percentchar EndProlog: tabularht%
      }%
    \fi
  }%

  % the x positions of vrules are stored in
  % \tabH@<tabcount>list with distinct values
  \protected\def\tabH@aux@vrule#1#2{%
    \@ifundefined{tabH@#1list}{%
      \expandafter\xdef\csname tabH@#1list\endcsname{%
        \noexpand\do{#2}%
      }%
    }{%
      \begingroup
        \def\x{#2}%
        \let\y\@undefined
        \let\do\tabH@do@add
        \expandafter\xdef\csname tabH@#1list\endcsname{%
          \csname tabH@#1list\endcsname\@empty
          \ifx\y\@undefined
            \noexpand\do{\x}%
          \fi
        }%
      \endgroup
    }%
  }%
  \def\tabH@do@add#1{%
    \ifx\y\@undefined
      \ifnum#1<\x\space
      \else
        \expandafter\ifx\csname y\endcsname\relax\fi
        \ifnum#1>\x\space
          \noexpand\do{\x}%
        \fi
      \fi
    \fi
    \noexpand\do{#1}%
  }%

  \def\tabH@vspace@start#1{%
    \if@filesw
      \stepcounter{tabH@unique}%
      \edef\tabH@currentrow{\the\c@tabH@unique}%
      \pdfsavepos
      \protected@write\@auxout{%
        \let\tabH@lastxpos\relax
        \let\tabH@lastypos\relax
      }{%
        \tabH@aux@vstart{\tabH@currenttab}{\tabH@currentrow}%
                        {\tabH@lastxpos}{\tabH@lastypos}%
      }%
    \fi
    \begingroup
      \edef\a{tabH@\tabH@currenttab row\tabH@currentrow}%
      \expandafter\let\expandafter\x\csname\a x\endcsname
      \ifx\x\relax
      \else
        \expandafter\let\expandafter\y\csname\a y\endcsname
        \expandafter\let\expandafter\l
            \csname tabH@\tabH@currenttab list\endcsname
        \ifx\l\relax
        \else
          \def\f{#1}%
          \ifx\f\@empty
            \let\do\tabH@do@set
          \else
            \count@=\z@
            \let\do\tabH@do@filter
          \fi
          \setbox\z@=\hbox{\l}%
          \wd\z@=\z@
          \dp\z@=\z@
          \copy\z@
        \fi
      \fi
    \endgroup
  }%
  \def\tabH@vspace@stop{%
    \if@filesw
      \pdfsavepos
      \protected@write\@auxout{%
        \let\tabH@lastypos\relax
      }{%
        \tabH@aux@vstop{\tabH@lastypos}%
      }%
    \fi
  }%
  \def\tabH@do@set#1{%
    \hbox to \z@{%
      \hskip \dimexpr #1sp - \x sp\relax
      \vrule \@width\arrayrulewidth
             \@depth\dimexpr \y sp\relax
      \hss
    }%
  }%
  \def\tabH@do@filter{%
    \@tempswafalse
    \advance\count@\@ne
    \@for\e:=\f\do{%
      \ifnum\e=\count@
        \@tempswatrue
      \fi
    }%
    \if@tempswa
      \expandafter\tabH@do@set
    \else
      \expandafter\@gobble
    \fi
  }%

  \protected\def\tabH@aux@vstart#1#2#3#4{%
    \def\tabH@current@vstart{{#1}{#2}{#3}{#4}}%
  }%
  \protected\def\tabH@aux@vstop{%
    \expandafter\tabH@aux@v\tabH@current@vstart
  }%
  \def\tabH@aux@v#1#2#3#4#5{%
    \expandafter\gdef\csname tabH@#1row#2x\endcsname{#3}%
    \expandafter\xdef\csname tabH@#1row#2y\endcsname{%
      \the\numexpr #4 - #5\relax
    }%
  }%

  \csname fi\endcsname
  \endinput

\fi
%    \end{macrocode}
%
% \subsection{DVI drivers}
%
%    \begin{macrocode}
\ifx\tabH@driver\@empty
  \PackageError{tabularht}{%
    Missing DVI driver, option `vlines' disabled%
  }{%
    Supported DVI drivers: dvips.%
  }%
  \expandafter\endinput
\fi

\def\tabH@driver@dvips{%
  \def\tabH@literalps##1{\special{ps:SDict begin ##1 end}}%
  \def\tabH@headerps##1{\special{! ##1}}%
}

\@onelevel@sanitize\tabH@driver
\@ifundefined{tabH@driver@\tabH@driver}{%
  \PackageError{tabularht}{%
    Unsupported driver `\tabH@driver'%
  }{%
    Supported DVI drivers: dvips.%
  }%
  \endinput
}{}

\begingroup
  \let\on@line\@empty
  \PackageInfo{tabularht}{%
    Using driver `\tabH@driver'%
  }%
\endgroup
\csname tabH@driver@\tabH@driver\endcsname

\protected\def\tabH@vrule#1#2\vrule#3\arrayrulewidth{%
  #2% \fi or empty
  % hack to get rid of maxdrift rounding of dvips,
  % thus simulate a large motion
  \kern1in\relax
  \tabH@literalps{%
    #1 tabH.vrule %
    Resolution neg 0 translate%
  }%
  \vrule#3\arrayrulewidth
  \tabH@literalps{Resolution 0 translate}%
  \kern-1in\relax
}

\def\tabH@vspace@start#1{%
  \begingroup
    \let\y\@empty
    \@for\x:=#1\do{%
      \ifx\y\@empty
        \edef\y{\x}%
      \else
        \edef\y{\y\space\x}%
      \fi
    }%
    \tabH@literalps{\tabH@currenttab[\y]currentpoint exch pop}%
  \endgroup
}
\def\tabH@vspace@stop{%
  \tabH@literalps{%
    currentpoint exch pop %
    \number\dimexpr\arrayrulewidth\relax\space
    tabH.vspace%
  }%
}

\tabH@headerps{%
  userdict begin%
    /tabH.list 10 dict def%
    /tabH.job [] def %
  end%
  /tabH.vrule{%
    10 string cvs cvn dup tabH.list exch known{%
      tabH.list exch dup [ exch tabH.list exch get %
      currentpoint pop round exch true exch{%
        % tabH.list key [ ... x true i
        % tabH.list key [ ... false i
        exch{%
          % ... [ ... x i
          2 copy lt{false}{%
            2 copy eq{pop false}{exch true}ifelse%
          }ifelse%
        }{false}ifelse%
      }forall %
      pop%
      ]put%
    }{%
      tabH.list exch[currentpoint pop round]put%
    }ifelse%
  }bind def%
  % <tab num> <cols array> <ytop> <ybottom> <rulewidth[sp]>
  /tabH.vspace{%
    userdict begin %
      10 dict dup begin %
        exch 65536 div Resolution mul 72.27 div %
        % dvips uses a poor man's ceil function
        % see dopage.c before "drawrule": (int)(... + 0.9999999)
        0.9999999 add truncate%
        /rulewidth exch def %
        exch/ybottom exch def %
        exch/ytop exch def %
        exch/cols exch def %
        exch/tabkey exch 10 string cvs cvn def %
      end%
      /tabH.job exch[exch userdict/tabH.job get aload pop]def %
    end%
  }bind def %
  % Now we do the work at the end of the page.
  % Unhappily "eop-hook" cannot be used, because "eop"
  % executes "restore" before, so that all data are lost.
  TeXDict begin%
    /eop%
    [%
      {%
        tabH.job{%
          begin%
           /colarray %
             tabH.list tabkey known{tabH.list tabkey get}{[]}ifelse %
           def %
           cols length 0 eq not{%
             /colarray[%
               cols{1 sub %
                 dup 0 lt{pop}{%
                   dup colarray length ge{pop}{%
                     colarray exch get%
                   }ifelse%
                 }ifelse%
               }forall%
             ]def%
           }if %
           colarray{%
             % (rulewidth) == rulewidth == % debug
             Resolution sub %
             ytop rulewidth ytop ybottom sub v%
           }forall %
          end%
        }forall%
        % tabH.list{== ==}forall % debug
      }bind aload pop %
      TeXDict /eop get aload pop%
    ]cvx def %
  end%
}
%    \end{macrocode}
%
%    \begin{macrocode}
%</package>
%    \end{macrocode}
%
% \section{Installation}
%
% \subsection{Download}
%
% \paragraph{Package.} This package is available on
% CTAN\footnote{\CTANpkg{tabularht}}:
% \begin{description}
% \item[\CTAN{macros/latex/contrib/oberdiek/tabularht.dtx}] The source file.
% \item[\CTAN{macros/latex/contrib/oberdiek/tabularht.pdf}] Documentation.
% \end{description}
%
%
% \paragraph{Bundle.} All the packages of the bundle `oberdiek'
% are also available in a TDS compliant ZIP archive. There
% the packages are already unpacked and the documentation files
% are generated. The files and directories obey the TDS standard.
% \begin{description}
% \item[\CTANinstall{install/macros/latex/contrib/oberdiek.tds.zip}]
% \end{description}
% \emph{TDS} refers to the standard ``A Directory Structure
% for \TeX\ Files'' (\CTAN{tds/tds.pdf}). Directories
% with \xfile{texmf} in their name are usually organized this way.
%
% \subsection{Bundle installation}
%
% \paragraph{Unpacking.} Unpack the \xfile{oberdiek.tds.zip} in the
% TDS tree (also known as \xfile{texmf} tree) of your choice.
% Example (linux):
% \begin{quote}
%   |unzip oberdiek.tds.zip -d ~/texmf|
% \end{quote}
%
% \paragraph{Script installation.}
% Check the directory \xfile{TDS:scripts/oberdiek/} for
% scripts that need further installation steps.
%
% \subsection{Package installation}
%
% \paragraph{Unpacking.} The \xfile{.dtx} file is a self-extracting
% \docstrip\ archive. The files are extracted by running the
% \xfile{.dtx} through \plainTeX:
% \begin{quote}
%   \verb|tex tabularht.dtx|
% \end{quote}
%
% \paragraph{TDS.} Now the different files must be moved into
% the different directories in your installation TDS tree
% (also known as \xfile{texmf} tree):
% \begin{quote}
% \def\t{^^A
% \begin{tabular}{@{}>{\ttfamily}l@{ $\rightarrow$ }>{\ttfamily}l@{}}
%   tabularht.sty & tex/latex/oberdiek/tabularht.sty\\
%   tabularht.pdf & doc/latex/oberdiek/tabularht.pdf\\
%   tabularht-example1.tex & doc/latex/oberdiek/tabularht-example1.tex\\
%   tabularht-example2.tex & doc/latex/oberdiek/tabularht-example2.tex\\
%   tabularht.dtx & source/latex/oberdiek/tabularht.dtx\\
% \end{tabular}^^A
% }^^A
% \sbox0{\t}^^A
% \ifdim\wd0>\linewidth
%   \begingroup
%     \advance\linewidth by\leftmargin
%     \advance\linewidth by\rightmargin
%   \edef\x{\endgroup
%     \def\noexpand\lw{\the\linewidth}^^A
%   }\x
%   \def\lwbox{^^A
%     \leavevmode
%     \hbox to \linewidth{^^A
%       \kern-\leftmargin\relax
%       \hss
%       \usebox0
%       \hss
%       \kern-\rightmargin\relax
%     }^^A
%   }^^A
%   \ifdim\wd0>\lw
%     \sbox0{\small\t}^^A
%     \ifdim\wd0>\linewidth
%       \ifdim\wd0>\lw
%         \sbox0{\footnotesize\t}^^A
%         \ifdim\wd0>\linewidth
%           \ifdim\wd0>\lw
%             \sbox0{\scriptsize\t}^^A
%             \ifdim\wd0>\linewidth
%               \ifdim\wd0>\lw
%                 \sbox0{\tiny\t}^^A
%                 \ifdim\wd0>\linewidth
%                   \lwbox
%                 \else
%                   \usebox0
%                 \fi
%               \else
%                 \lwbox
%               \fi
%             \else
%               \usebox0
%             \fi
%           \else
%             \lwbox
%           \fi
%         \else
%           \usebox0
%         \fi
%       \else
%         \lwbox
%       \fi
%     \else
%       \usebox0
%     \fi
%   \else
%     \lwbox
%   \fi
% \else
%   \usebox0
% \fi
% \end{quote}
% If you have a \xfile{docstrip.cfg} that configures and enables \docstrip's
% TDS installing feature, then some files can already be in the right
% place, see the documentation of \docstrip.
%
% \subsection{Refresh file name databases}
%
% If your \TeX~distribution
% (\teTeX, \mikTeX, \dots) relies on file name databases, you must refresh
% these. For example, \teTeX\ users run \verb|texhash| or
% \verb|mktexlsr|.
%
% \subsection{Some details for the interested}
%
% \paragraph{Unpacking with \LaTeX.}
% The \xfile{.dtx} chooses its action depending on the format:
% \begin{description}
% \item[\plainTeX:] Run \docstrip\ and extract the files.
% \item[\LaTeX:] Generate the documentation.
% \end{description}
% If you insist on using \LaTeX\ for \docstrip\ (really,
% \docstrip\ does not need \LaTeX), then inform the autodetect routine
% about your intention:
% \begin{quote}
%   \verb|latex \let\install=y% \iffalse meta-comment
%
% File: tabularht.dtx
% Version: 2016/05/16 v2.6
% Info: Tabular with height specified
%
% Copyright (C) 2005-2007 by
%    Heiko Oberdiek <heiko.oberdiek at googlemail.com>
%    2016
%    https://github.com/ho-tex/oberdiek/issues
%
% This work may be distributed and/or modified under the
% conditions of the LaTeX Project Public License, either
% version 1.3c of this license or (at your option) any later
% version. This version of this license is in
%    https://www.latex-project.org/lppl/lppl-1-3c.txt
% and the latest version of this license is in
%    https://www.latex-project.org/lppl.txt
% and version 1.3 or later is part of all distributions of
% LaTeX version 2005/12/01 or later.
%
% This work has the LPPL maintenance status "maintained".
%
% The Current Maintainers of this work are
% Heiko Oberdiek and the Oberdiek Package Support Group
% https://github.com/ho-tex/oberdiek/issues
%
% This work consists of the main source file tabularht.dtx
% and the derived files
%    tabularht.sty, tabularht.pdf, tabularht.ins, tabularht.drv,
%    tabularht-example1.tex, tabularht-example2.tex.
%
% Distribution:
%    CTAN:macros/latex/contrib/oberdiek/tabularht.dtx
%    CTAN:macros/latex/contrib/oberdiek/tabularht.pdf
%
% Unpacking:
%    (a) If tabularht.ins is present:
%           tex tabularht.ins
%    (b) Without tabularht.ins:
%           tex tabularht.dtx
%    (c) If you insist on using LaTeX
%           latex \let\install=y% \iffalse meta-comment
%
% File: tabularht.dtx
% Version: 2016/05/16 v2.6
% Info: Tabular with height specified
%
% Copyright (C) 2005-2007 by
%    Heiko Oberdiek <heiko.oberdiek at googlemail.com>
%    2016
%    https://github.com/ho-tex/oberdiek/issues
%
% This work may be distributed and/or modified under the
% conditions of the LaTeX Project Public License, either
% version 1.3c of this license or (at your option) any later
% version. This version of this license is in
%    https://www.latex-project.org/lppl/lppl-1-3c.txt
% and the latest version of this license is in
%    https://www.latex-project.org/lppl.txt
% and version 1.3 or later is part of all distributions of
% LaTeX version 2005/12/01 or later.
%
% This work has the LPPL maintenance status "maintained".
%
% The Current Maintainers of this work are
% Heiko Oberdiek and the Oberdiek Package Support Group
% https://github.com/ho-tex/oberdiek/issues
%
% This work consists of the main source file tabularht.dtx
% and the derived files
%    tabularht.sty, tabularht.pdf, tabularht.ins, tabularht.drv,
%    tabularht-example1.tex, tabularht-example2.tex.
%
% Distribution:
%    CTAN:macros/latex/contrib/oberdiek/tabularht.dtx
%    CTAN:macros/latex/contrib/oberdiek/tabularht.pdf
%
% Unpacking:
%    (a) If tabularht.ins is present:
%           tex tabularht.ins
%    (b) Without tabularht.ins:
%           tex tabularht.dtx
%    (c) If you insist on using LaTeX
%           latex \let\install=y\input{tabularht.dtx}
%        (quote the arguments according to the demands of your shell)
%
% Documentation:
%    (a) If tabularht.drv is present:
%           latex tabularht.drv
%    (b) Without tabularht.drv:
%           latex tabularht.dtx; ...
%    The class ltxdoc loads the configuration file ltxdoc.cfg
%    if available. Here you can specify further options, e.g.
%    use A4 as paper format:
%       \PassOptionsToClass{a4paper}{article}
%
%    Programm calls to get the documentation (example):
%       pdflatex tabularht.dtx
%       makeindex -s gind.ist tabularht.idx
%       pdflatex tabularht.dtx
%       makeindex -s gind.ist tabularht.idx
%       pdflatex tabularht.dtx
%
% Installation:
%    TDS:tex/latex/oberdiek/tabularht.sty
%    TDS:doc/latex/oberdiek/tabularht.pdf
%    TDS:doc/latex/oberdiek/tabularht-example1.tex
%    TDS:doc/latex/oberdiek/tabularht-example2.tex
%    TDS:source/latex/oberdiek/tabularht.dtx
%
%<*ignore>
\begingroup
  \catcode123=1 %
  \catcode125=2 %
  \def\x{LaTeX2e}%
\expandafter\endgroup
\ifcase 0\ifx\install y1\fi\expandafter
         \ifx\csname processbatchFile\endcsname\relax\else1\fi
         \ifx\fmtname\x\else 1\fi\relax
\else\csname fi\endcsname
%</ignore>
%<*install>
\input docstrip.tex
\Msg{************************************************************************}
\Msg{* Installation}
\Msg{* Package: tabularht 2016/05/16 v2.6 Tabular with height specified (HO)}
\Msg{************************************************************************}

\keepsilent
\askforoverwritefalse

\let\MetaPrefix\relax
\preamble

This is a generated file.

Project: tabularht
Version: 2016/05/16 v2.6

Copyright (C) 2005-2007 by
   Heiko Oberdiek <heiko.oberdiek at googlemail.com>

This work may be distributed and/or modified under the
conditions of the LaTeX Project Public License, either
version 1.3c of this license or (at your option) any later
version. This version of this license is in
   https://www.latex-project.org/lppl/lppl-1-3c.txt
and the latest version of this license is in
   https://www.latex-project.org/lppl.txt
and version 1.3 or later is part of all distributions of
LaTeX version 2005/12/01 or later.

This work has the LPPL maintenance status "maintained".

The Current Maintainers of this work are
Heiko Oberdiek and the Oberdiek Package Support Group
https://github.com/ho-tex/oberdiek/issues


This work consists of the main source file tabularht.dtx
and the derived files
   tabularht.sty, tabularht.pdf, tabularht.ins, tabularht.drv,
   tabularht-example1.tex, tabularht-example2.tex.

\endpreamble
\let\MetaPrefix\DoubleperCent

\generate{%
  \file{tabularht.ins}{\from{tabularht.dtx}{install}}%
  \file{tabularht.drv}{\from{tabularht.dtx}{driver}}%
  \usedir{tex/latex/oberdiek}%
  \file{tabularht.sty}{\from{tabularht.dtx}{package}}%
  \usedir{doc/latex/oberdiek}%
  \file{tabularht-example1.tex}{\from{tabularht.dtx}{example1}}%
  \file{tabularht-example2.tex}{\from{tabularht.dtx}{example2}}%
  \nopreamble
  \nopostamble
%  \usedir{source/latex/oberdiek/catalogue}%
%  \file{tabularht.xml}{\from{tabularht.dtx}{catalogue}}%
}

\catcode32=13\relax% active space
\let =\space%
\Msg{************************************************************************}
\Msg{*}
\Msg{* To finish the installation you have to move the following}
\Msg{* file into a directory searched by TeX:}
\Msg{*}
\Msg{*     tabularht.sty}
\Msg{*}
\Msg{* To produce the documentation run the file `tabularht.drv'}
\Msg{* through LaTeX.}
\Msg{*}
\Msg{* Happy TeXing!}
\Msg{*}
\Msg{************************************************************************}

\endbatchfile
%</install>
%<*ignore>
\fi
%</ignore>
%<*driver>
\NeedsTeXFormat{LaTeX2e}
\ProvidesFile{tabularht.drv}%
  [2016/05/16 v2.6 Tabular with height specified (HO)]%
\documentclass{ltxdoc}
\usepackage{holtxdoc}[2011/11/22]
\begin{document}
  \DocInput{tabularht.dtx}%
\end{document}
%</driver>
% \fi
%
%
% \CharacterTable
%  {Upper-case    \A\B\C\D\E\F\G\H\I\J\K\L\M\N\O\P\Q\R\S\T\U\V\W\X\Y\Z
%   Lower-case    \a\b\c\d\e\f\g\h\i\j\k\l\m\n\o\p\q\r\s\t\u\v\w\x\y\z
%   Digits        \0\1\2\3\4\5\6\7\8\9
%   Exclamation   \!     Double quote  \"     Hash (number) \#
%   Dollar        \$     Percent       \%     Ampersand     \&
%   Acute accent  \'     Left paren    \(     Right paren   \)
%   Asterisk      \*     Plus          \+     Comma         \,
%   Minus         \-     Point         \.     Solidus       \/
%   Colon         \:     Semicolon     \;     Less than     \<
%   Equals        \=     Greater than  \>     Question mark \?
%   Commercial at \@     Left bracket  \[     Backslash     \\
%   Right bracket \]     Circumflex    \^     Underscore    \_
%   Grave accent  \`     Left brace    \{     Vertical bar  \|
%   Right brace   \}     Tilde         \~}
%
% \GetFileInfo{tabularht.drv}
%
% \title{The \xpackage{tabularht} package}
% \date{2016/05/16 v2.6}
% \author{Heiko Oberdiek\thanks
% {Please report any issues at \url{https://github.com/ho-tex/oberdiek/issues}}}
%
% \maketitle
%
% \begin{abstract}
% This package defines some environments that adds
% a height specification to tabular and array.
% \end{abstract}
%
% \tableofcontents
%
% \section{Usage}
%
% \begin{quote}
%   |\usepackage{tabularht}|
% \end{quote}
% The package provides the following environments
% that extend the tabular/array environment by
% a height specification as first argument:
% \begin{itemize}
% \item |tabularht|, |tabularht*|
% \item |arrayht|
% \item |tabularhtx|
%       (if package \xpackage{tabularx} is loaded)
% \end{itemize}
% The height argument allows a length specification,
% package \xpackage{calc} is supported if used. This means,
% the tabular will have the specified height.
% You can also use the prefixes |to=| and |spread=|.
% |to=| is the default, |spread=| means, the natural
% height of the tabular box is changed by the length
% after |spread=|.
%
% Examples:
% \begin{quote}
% \begin{tabular}{@{}l@{ $\rightarrow$ }l@{}}
%   |\begin{tabularht}{1in}|
%   & height is 1in\\
%   |\begin{tabularht}{to=1in}|
%   & height is 1in\\
%   |\begin{tabularht}{spread=0pt}|
%   & natural height, same as |\begin{tabular}|\\
%   |\begin{tabularht}{spread=1in}|
%   & natural height increased by 1in
% \end{tabular}
% \end{quote}
%
% Hint:  See also package \xpackage{tabularkv}, it provides an interface,
%        where most parameters for the environments can be given
%        by key-value pairs.
%
% \begin{declcs}{interrowspace}\,|{|\dots|}|
% \end{declcs}
% Adds space between table rows. It is essentially
% the same as |\noalign{\vspace{...}}|.
% \begin{declcs}{interrowfill}
% \end{declcs}
% Short for |\interrowspace{\fill}|
% \begin{declcs}{interrowstart}\,\dots\,\cs{interrowstop}
% \end{declcs}
% Marker commands, useful for option \xoption{vlines}.
%
% \subsection{Option \xoption{vlines}}
%
% Warning: This stuff is experimental.
%
% \vspace{2ex}
% \noindent
% Vertical lines are interrupted, if space is inserted
% in \cs{noalign}, \cs{interrowspace}, \cs{addlinespace}
% (\xpackage{booktabs}),
% between double \cs{hline}s. This option tries to detect
% and add the vertical lines. The lines in a tabular
% with \xpackage{tabularht} support (environments of this package)
% are numbered from left to right. The gap that is
% controlled by \cs{interrowspace} or inbetween \cs{interrowstart}
% and \cs{interrowstop} is then filled with the detected
% vertical lines.
%
% If only a limited selection of the lines should be
% drawn, the commands know an optional argument with a
% list of line numbers, e.g.
% \begin{quote}
%\begin{verbatim}
%\begin{tabularht}{50mm}{|l|l|}
%  Hello & World\\
%  \interrowfill[1,3]
%  Foo & Bar
%\end{tabularht}
%\end{verbatim}
% \end{quote}
% There are three lines, but the middle line is not drawn
% in the gap between the first and second row.
% Zero can be used to suppress all lines:
% \begin{quote}
% |\interrowspace[0]{10mm}|
% \end{quote}
% The syntax of the commands with the optional argument
% with the line number list \meta{list}. \meta{list} is a comma
% separated list of numbers, \meta{height} means the
% height specification described above with the optional
% prefixes |to=| or |spread=|.
% \begin{quote}
%   \cs{interrowspace}\,|[|\meta{list}|]|\,|{|\meta{height}|}|\\
%   \cs{interrowfill}\,|[|\meta{list}|]|\\
%   \cs{interrowstart}\,|[|\meta{list}|]|\,\dots\,\cs{interrowstop}
% \end{quote}
%
% Option \xoption{vlines} is driver dependent and uses \eTeX\ features.
% \begin{description}
% \item[\xoption{pdftex}:] \pdfTeX\ in PDF mode.
%   Here the positions of the lines are written with the
%   help of the \cs{pdfsavepos} feature into the \xfile{.aux} file(s).
%   Therefore you need two LaTeX runs to get the lines.
% \item[\xoption{dvips}:]
%   Here, PostScript's currentpoint it used to get
%   the line positions. The lines are then drawn at the end
%   of the page. Thus one \LaTeX/dvips run is sufficient
%   for this option.
% \item[Other drivers:]\mbox{}
%   \begin{description}
%   \item[PostScript drivers:] probably possible, an end of page
%     hook would be nice.
%   \item[\VTeX:] with GeX (PostScript interpreter) probably possible.
%   \item[dvipdfm:] no idea. The big problem is, how to get the
%     current position?
%   \end{description}
% \end{description}
%
% \subsection{Limitations}
%
% \begin{itemize}
% \item
%   Vertical lines are interrupted by |\noalign{\vfill}|.
% \end{itemize}
%
% \subsection{Compatibility}
%
% \begin{itemize}
% \item
%   \xpackage{array}, \xpackage{delarray}, \xpackage{tabularx}
%   are supported.
% \item There can be problems with packages that
%   redefine \cs{@array} (or \cs{@@array}, \cs{@tabarray}) and
%   \cs{@arrayrule} (for option \xoption{vlines}).
% \item \xpackage{colortbl}: it should at least work, but there isn't
%   support for filling the gaps with color, neither
%   the rules nor the backgrounds.
% \end{itemize}
%
% \subsection{Examples}
%
% \subsubsection{Example 1}
%
%    \begin{macrocode}
%<*example1>
\documentclass{article}
\usepackage{tabularht}

\begin{document}
\fbox{%
  \begin{tabularht*}{1in}{4in}{@{}l@{\extracolsep{\fill}}r@{}}%
    upper left corner & upper right corner\\%
    \noalign{\vfill}%
    \multicolumn{2}{@{}c@{}}{bounding box}\\%
    \noalign{\vfill}%
    lower left corner & lower right corner\\%
  \end{tabularht*}%
}
\end{document}
%</example1>
%    \end{macrocode}
%
% \subsubsection{Example 2}
%
%    \begin{macrocode}
%<*example2>
\documentclass{article}
\usepackage{booktabs}
\usepackage[dvips,vlines]{tabularht}

\begin{document}

\begin{tabularht}{spread=0pt}{|l|l|}
  \hline
  First&Line\\%
  \hline
\interrowstart
  \addlinespace[10mm]%
\interrowstop
  \hline
  Second&Line\\%
\interrowstart
  \hline
  \hline
\interrowstop
  Third&Line\\%
  \hline
\interrowspace{10mm}
  \hline
  Fourth&Line\\%
  \hline
\end{tabularht}

\end{document}
%</example2>
%    \end{macrocode}
%
% \StopEventually{
% }
%
% \section{Implementation}
%
%    \begin{macrocode}
%<*package>
%    \end{macrocode}
%    Package identification.
%    \begin{macrocode}
\NeedsTeXFormat{LaTeX2e}
\ProvidesPackage{tabularht}%
  [2016/05/16 v2.6 Tabular with height specified (HO)]
%    \end{macrocode}
%
% \subsection{Environments}
%
%    \begin{macrocode}
\let\@toarrayheight\@empty
\let\tabH@array@init\@empty

\toks@={%
  \begingroup
    \long\def\x#1\vcenter\fi\fi\bgroup#2\@sharp#3#4\@nil{%
      \endgroup
      \gdef\@array[##1]##2{%
        \tabH@array@init
        #1%
        \vcenter\fi\fi
        \@toarrayheight
        \bgroup
        \let\@toarrayheight\@empty
        #2\@sharp###3#4%
      }%
    }%
  \expandafter\x\@array[#1]{#2}\@nil % hash-ok
}
\edef\tabH@patch@array{\the\toks@}
\def\tabH@patch@@array{%
  \ifx\@array\@@array
    \def\reserved@a{\let\@@array\@array}%
  \else
    \let\reserved@a\relax
  \fi
  \tabH@patch@array
  \reserved@a
}
\tabH@patch@@array

\@ifpackageloaded{array}{}{%
  \AtBeginDocument{%
    \@ifpackageloaded{array}{%
      \tabH@patch@@array
    }{}%
  }%
}

\def\tabH@setheight#1{%
  \tabH@@setheight#1==\@nil
}
\def\tabH@@setheight#1=#2=#3\@nil{%
  \ifx\\#2#3\\%
    \setlength{\dimen@}{#1}%
    \edef\@toarrayheight{to\the\dimen@}%
  \else
    \edef\tabH@temp{\zap@space#1 \@empty}%
    \ifx\tabH@temp\tabH@to
    \else
      \ifx\tabH@temp\tabH@spread
      \else
        \PackageError{tabularht}{%
          Unknown height specifier %
          `\expandafter\strip@prefix\meaning\tabH@temp'%
        }{%
          The height dimension for tabular height can be prefixed%
          \MessageBreak
          with `to=' or `spread=', default is `to='.%
        }%
        \let\tabH@temp\tabH@to
      \fi
    \fi
    \setlength{\dimen@}{#2}%
    \edef\@toarrayheight{\tabH@temp\the\dimen@}%
  \fi
}
\def\tabH@to{to}
\def\tabH@spread{spread}
%    \end{macrocode}
%    First argument is the height of the table,
%    then the orignal arguments for tabular follow.
%    \begin{macrocode}
\newenvironment{tabularht}[1]{%
  \tabH@setheight{#1}%
  \tabular
}{%
  \endtabular
}

\newenvironment{tabularht*}[1]{%
  \tabH@setheight{#1}%
  \@nameuse{tabular*}%
}{%
  \@nameuse{endtabular*}%
}

\newenvironment{tabularhtx}[1]{%
  \tabH@setheight{#1}%
  \tabularx
}{%
  \endtabularx
}

\newenvironment{arrayht}[1]{%
  \tabH@setheight{#1}%
  \array
}{%
  \endarray
}

\def\interrowspace{%
  \noalign\bgroup
    \tabH@interrowspace
}
\newcommand*{\tabH@interrowspace}[2][]{%
    \tabH@vspace{#1}{#2}%
  \egroup
}
\def\interrowfill{%
  \noalign\bgroup
    \tabH@interrowfill
}
\newcommand*{\tabH@interrowfill}[1][]{%
    \tabH@vspace{#1}{\fill}%
  \egroup
}
\def\tabH@vspace#1#2{%
  \tabH@vspace@start{#1}%
  \vspace{#2}%
  \tabH@vspace@stop
}
\let\tabH@vspace@start\@gobble
\let\tabH@vspace@stop\@empty

\newcommand*{\interrowstart}{%
  \noalign\bgroup
    \tabH@interrowstart
}
\newcommand*{\tabH@interrowstart}[1][]{%
    \tabH@vspace@start{#1}%
  \egroup
}
\newcommand*{\interrowstop}{%
  \noalign{\tabH@vspace@stop}%
}
%    \end{macrocode}
%
% \subsection{Options}
%
%    \begin{macrocode}
\providecommand*{\tabH@driver}{}

\DeclareOption{vlines}{%
  \let\tabH@temp\relax
}
\DeclareOption{pdftex}{}
\DeclareOption{dvips}{%
  \def\tabH@driver{dvips}%
}
\ProcessOptions*\relax

\ifx\tabH@temp\relax
\else
  \expandafter\endinput
\fi

\begingroup
  \@ifundefined{eTeXversion}{%
    \PackageError{tabularht}{%
      Option `vlines' requires eTeX%
    }{%
      Use of eTeX is recommended for LaTeX, see ltnews16.%
    }%
    \endgroup
    \endinput
  }{}%
\endgroup
%    \end{macrocode}
%
% \subsection{Option vlines, driver independent stuff}
%
%    \begin{macrocode}
\begingroup
  \let\@addtoreset\@gobbletwo
  \newcounter{tabH@unique}%
\endgroup
\let\tabH@currenttab\@empty

\def\tabH@array@init{%
  \ifx\@toarrayheight\@empty
    % ignore vertical lines of nested tabular environments
    \let\tabH@currenttab\@empty
  \else
    \stepcounter{tabH@unique}%
    \edef\tabH@currenttab{\the\c@tabH@unique}%
  \fi
}

\renewcommand*{\@arrayrule}{%
  \@addtopreamble{%
    \hskip -.5\arrayrulewidth
    \ifx\tabH@currenttab\@empty
    \else
      \tabH@vrule{\tabH@currenttab}%
    \fi
    \begingroup
      \expandafter\ifx\csname CT@arc@\endcsname\relax
      \else
        \expandafter\CT@arc@
      \fi
      \vline
    \endgroup
    \hskip -.5\arrayrulewidth
  }%
}
\let\tabH@arrayrule\@arrayrule
\AtBeginDocument{%
  \@ifpackageloaded{colortbl}{%
    \let\@arrayrule\tabH@arrayrule
  }{}%
}

\let\tabH@vrule\@gobble
%    \end{macrocode}
%
% \subsection{Driver pdftex}
%
%    \begin{macrocode}
\RequirePackage{ifpdf}
\ifpdf
  \begingroup
    \@ifundefined{pdfsavepos}{%
      \PackageError{tabularht}{%
        Your pdfTeX is too old%
      }{%
        \string\pdfsavepos\space is missing.%
      }%
      \endgroup
      \csname fi\endcsname
      \endinput
    }{}%

    \let\on@line\@empty
    \PackageInfo{tabularht}{%
      Using driver `pdftex' because of pdfTeX in PDF mode%
    }%
  \endgroup

  \protected\def\tabH@vrule#1{%
    \if@filesw
      \pdfsavepos
      \protected@write\@auxout{%
        \let\tabH@lastxpos\relax
      }{%
        \tabH@aux@vrule{#1}{\tabH@lastxpos}%
      }%
    \fi
  }%

  \def\tabH@lastxpos{\the\pdflastxpos}%
  \def\tabH@lastypos{\the\pdflastypos}%

  % The .aux file contains three commands:
  % \tabH@aux@vrule{tabular id}{x position}
  % \tabH@aux@vstart{tabular id}{row id}{x position}{y position}
  % \tabH@aux@vstop{y position}
  %
  \AtBeginDocument{%
    % The .aux files are read the first time before
    % \AtBeginDocument and later at \end{document}.
    % \tabH@aux@done is a marker to distinguish
    % between these two readings. Only in the first
    % case we need the \tabH@aux@... commands.
    \let\tabH@aux@done\@empty
    \if@filesw
      \immediate\write\@mainaux{%
        \@percentchar\@percentchar BeginProlog: tabularht%
      }%
      % items in the aux file are executed,
      % if tabularht is loaded
      % and during the aux file read at \begin{document} only
      \immediate\write\@mainaux{%
        \detokenize{%
          % the \tabH@aux@... commands are needed only if
          % tabularht is loaded with driver pdftex.
          \@ifundefined{tabH@aux@vrule}\@secondoftwo\@firstofone
          {%
            % disable commands except for the first .aux files reading
            \@ifundefined{tabH@aux@done}\@gobble\@firstofone
          }%
          {%
            \let\tabH@aux@vrule\@gobbletwo
            \let\tabH@aux@vstart\@gobblefour
            \let\tabH@aux@vstop\@gobble
          }%
        }%
      }%
      \immediate\write\@mainaux{%
        \@percentchar\@percentchar EndProlog: tabularht%
      }%
    \fi
  }%

  % the x positions of vrules are stored in
  % \tabH@<tabcount>list with distinct values
  \protected\def\tabH@aux@vrule#1#2{%
    \@ifundefined{tabH@#1list}{%
      \expandafter\xdef\csname tabH@#1list\endcsname{%
        \noexpand\do{#2}%
      }%
    }{%
      \begingroup
        \def\x{#2}%
        \let\y\@undefined
        \let\do\tabH@do@add
        \expandafter\xdef\csname tabH@#1list\endcsname{%
          \csname tabH@#1list\endcsname\@empty
          \ifx\y\@undefined
            \noexpand\do{\x}%
          \fi
        }%
      \endgroup
    }%
  }%
  \def\tabH@do@add#1{%
    \ifx\y\@undefined
      \ifnum#1<\x\space
      \else
        \expandafter\ifx\csname y\endcsname\relax\fi
        \ifnum#1>\x\space
          \noexpand\do{\x}%
        \fi
      \fi
    \fi
    \noexpand\do{#1}%
  }%

  \def\tabH@vspace@start#1{%
    \if@filesw
      \stepcounter{tabH@unique}%
      \edef\tabH@currentrow{\the\c@tabH@unique}%
      \pdfsavepos
      \protected@write\@auxout{%
        \let\tabH@lastxpos\relax
        \let\tabH@lastypos\relax
      }{%
        \tabH@aux@vstart{\tabH@currenttab}{\tabH@currentrow}%
                        {\tabH@lastxpos}{\tabH@lastypos}%
      }%
    \fi
    \begingroup
      \edef\a{tabH@\tabH@currenttab row\tabH@currentrow}%
      \expandafter\let\expandafter\x\csname\a x\endcsname
      \ifx\x\relax
      \else
        \expandafter\let\expandafter\y\csname\a y\endcsname
        \expandafter\let\expandafter\l
            \csname tabH@\tabH@currenttab list\endcsname
        \ifx\l\relax
        \else
          \def\f{#1}%
          \ifx\f\@empty
            \let\do\tabH@do@set
          \else
            \count@=\z@
            \let\do\tabH@do@filter
          \fi
          \setbox\z@=\hbox{\l}%
          \wd\z@=\z@
          \dp\z@=\z@
          \copy\z@
        \fi
      \fi
    \endgroup
  }%
  \def\tabH@vspace@stop{%
    \if@filesw
      \pdfsavepos
      \protected@write\@auxout{%
        \let\tabH@lastypos\relax
      }{%
        \tabH@aux@vstop{\tabH@lastypos}%
      }%
    \fi
  }%
  \def\tabH@do@set#1{%
    \hbox to \z@{%
      \hskip \dimexpr #1sp - \x sp\relax
      \vrule \@width\arrayrulewidth
             \@depth\dimexpr \y sp\relax
      \hss
    }%
  }%
  \def\tabH@do@filter{%
    \@tempswafalse
    \advance\count@\@ne
    \@for\e:=\f\do{%
      \ifnum\e=\count@
        \@tempswatrue
      \fi
    }%
    \if@tempswa
      \expandafter\tabH@do@set
    \else
      \expandafter\@gobble
    \fi
  }%

  \protected\def\tabH@aux@vstart#1#2#3#4{%
    \def\tabH@current@vstart{{#1}{#2}{#3}{#4}}%
  }%
  \protected\def\tabH@aux@vstop{%
    \expandafter\tabH@aux@v\tabH@current@vstart
  }%
  \def\tabH@aux@v#1#2#3#4#5{%
    \expandafter\gdef\csname tabH@#1row#2x\endcsname{#3}%
    \expandafter\xdef\csname tabH@#1row#2y\endcsname{%
      \the\numexpr #4 - #5\relax
    }%
  }%

  \csname fi\endcsname
  \endinput

\fi
%    \end{macrocode}
%
% \subsection{DVI drivers}
%
%    \begin{macrocode}
\ifx\tabH@driver\@empty
  \PackageError{tabularht}{%
    Missing DVI driver, option `vlines' disabled%
  }{%
    Supported DVI drivers: dvips.%
  }%
  \expandafter\endinput
\fi

\def\tabH@driver@dvips{%
  \def\tabH@literalps##1{\special{ps:SDict begin ##1 end}}%
  \def\tabH@headerps##1{\special{! ##1}}%
}

\@onelevel@sanitize\tabH@driver
\@ifundefined{tabH@driver@\tabH@driver}{%
  \PackageError{tabularht}{%
    Unsupported driver `\tabH@driver'%
  }{%
    Supported DVI drivers: dvips.%
  }%
  \endinput
}{}

\begingroup
  \let\on@line\@empty
  \PackageInfo{tabularht}{%
    Using driver `\tabH@driver'%
  }%
\endgroup
\csname tabH@driver@\tabH@driver\endcsname

\protected\def\tabH@vrule#1#2\vrule#3\arrayrulewidth{%
  #2% \fi or empty
  % hack to get rid of maxdrift rounding of dvips,
  % thus simulate a large motion
  \kern1in\relax
  \tabH@literalps{%
    #1 tabH.vrule %
    Resolution neg 0 translate%
  }%
  \vrule#3\arrayrulewidth
  \tabH@literalps{Resolution 0 translate}%
  \kern-1in\relax
}

\def\tabH@vspace@start#1{%
  \begingroup
    \let\y\@empty
    \@for\x:=#1\do{%
      \ifx\y\@empty
        \edef\y{\x}%
      \else
        \edef\y{\y\space\x}%
      \fi
    }%
    \tabH@literalps{\tabH@currenttab[\y]currentpoint exch pop}%
  \endgroup
}
\def\tabH@vspace@stop{%
  \tabH@literalps{%
    currentpoint exch pop %
    \number\dimexpr\arrayrulewidth\relax\space
    tabH.vspace%
  }%
}

\tabH@headerps{%
  userdict begin%
    /tabH.list 10 dict def%
    /tabH.job [] def %
  end%
  /tabH.vrule{%
    10 string cvs cvn dup tabH.list exch known{%
      tabH.list exch dup [ exch tabH.list exch get %
      currentpoint pop round exch true exch{%
        % tabH.list key [ ... x true i
        % tabH.list key [ ... false i
        exch{%
          % ... [ ... x i
          2 copy lt{false}{%
            2 copy eq{pop false}{exch true}ifelse%
          }ifelse%
        }{false}ifelse%
      }forall %
      pop%
      ]put%
    }{%
      tabH.list exch[currentpoint pop round]put%
    }ifelse%
  }bind def%
  % <tab num> <cols array> <ytop> <ybottom> <rulewidth[sp]>
  /tabH.vspace{%
    userdict begin %
      10 dict dup begin %
        exch 65536 div Resolution mul 72.27 div %
        % dvips uses a poor man's ceil function
        % see dopage.c before "drawrule": (int)(... + 0.9999999)
        0.9999999 add truncate%
        /rulewidth exch def %
        exch/ybottom exch def %
        exch/ytop exch def %
        exch/cols exch def %
        exch/tabkey exch 10 string cvs cvn def %
      end%
      /tabH.job exch[exch userdict/tabH.job get aload pop]def %
    end%
  }bind def %
  % Now we do the work at the end of the page.
  % Unhappily "eop-hook" cannot be used, because "eop"
  % executes "restore" before, so that all data are lost.
  TeXDict begin%
    /eop%
    [%
      {%
        tabH.job{%
          begin%
           /colarray %
             tabH.list tabkey known{tabH.list tabkey get}{[]}ifelse %
           def %
           cols length 0 eq not{%
             /colarray[%
               cols{1 sub %
                 dup 0 lt{pop}{%
                   dup colarray length ge{pop}{%
                     colarray exch get%
                   }ifelse%
                 }ifelse%
               }forall%
             ]def%
           }if %
           colarray{%
             % (rulewidth) == rulewidth == % debug
             Resolution sub %
             ytop rulewidth ytop ybottom sub v%
           }forall %
          end%
        }forall%
        % tabH.list{== ==}forall % debug
      }bind aload pop %
      TeXDict /eop get aload pop%
    ]cvx def %
  end%
}
%    \end{macrocode}
%
%    \begin{macrocode}
%</package>
%    \end{macrocode}
%
% \section{Installation}
%
% \subsection{Download}
%
% \paragraph{Package.} This package is available on
% CTAN\footnote{\CTANpkg{tabularht}}:
% \begin{description}
% \item[\CTAN{macros/latex/contrib/oberdiek/tabularht.dtx}] The source file.
% \item[\CTAN{macros/latex/contrib/oberdiek/tabularht.pdf}] Documentation.
% \end{description}
%
%
% \paragraph{Bundle.} All the packages of the bundle `oberdiek'
% are also available in a TDS compliant ZIP archive. There
% the packages are already unpacked and the documentation files
% are generated. The files and directories obey the TDS standard.
% \begin{description}
% \item[\CTANinstall{install/macros/latex/contrib/oberdiek.tds.zip}]
% \end{description}
% \emph{TDS} refers to the standard ``A Directory Structure
% for \TeX\ Files'' (\CTAN{tds/tds.pdf}). Directories
% with \xfile{texmf} in their name are usually organized this way.
%
% \subsection{Bundle installation}
%
% \paragraph{Unpacking.} Unpack the \xfile{oberdiek.tds.zip} in the
% TDS tree (also known as \xfile{texmf} tree) of your choice.
% Example (linux):
% \begin{quote}
%   |unzip oberdiek.tds.zip -d ~/texmf|
% \end{quote}
%
% \paragraph{Script installation.}
% Check the directory \xfile{TDS:scripts/oberdiek/} for
% scripts that need further installation steps.
%
% \subsection{Package installation}
%
% \paragraph{Unpacking.} The \xfile{.dtx} file is a self-extracting
% \docstrip\ archive. The files are extracted by running the
% \xfile{.dtx} through \plainTeX:
% \begin{quote}
%   \verb|tex tabularht.dtx|
% \end{quote}
%
% \paragraph{TDS.} Now the different files must be moved into
% the different directories in your installation TDS tree
% (also known as \xfile{texmf} tree):
% \begin{quote}
% \def\t{^^A
% \begin{tabular}{@{}>{\ttfamily}l@{ $\rightarrow$ }>{\ttfamily}l@{}}
%   tabularht.sty & tex/latex/oberdiek/tabularht.sty\\
%   tabularht.pdf & doc/latex/oberdiek/tabularht.pdf\\
%   tabularht-example1.tex & doc/latex/oberdiek/tabularht-example1.tex\\
%   tabularht-example2.tex & doc/latex/oberdiek/tabularht-example2.tex\\
%   tabularht.dtx & source/latex/oberdiek/tabularht.dtx\\
% \end{tabular}^^A
% }^^A
% \sbox0{\t}^^A
% \ifdim\wd0>\linewidth
%   \begingroup
%     \advance\linewidth by\leftmargin
%     \advance\linewidth by\rightmargin
%   \edef\x{\endgroup
%     \def\noexpand\lw{\the\linewidth}^^A
%   }\x
%   \def\lwbox{^^A
%     \leavevmode
%     \hbox to \linewidth{^^A
%       \kern-\leftmargin\relax
%       \hss
%       \usebox0
%       \hss
%       \kern-\rightmargin\relax
%     }^^A
%   }^^A
%   \ifdim\wd0>\lw
%     \sbox0{\small\t}^^A
%     \ifdim\wd0>\linewidth
%       \ifdim\wd0>\lw
%         \sbox0{\footnotesize\t}^^A
%         \ifdim\wd0>\linewidth
%           \ifdim\wd0>\lw
%             \sbox0{\scriptsize\t}^^A
%             \ifdim\wd0>\linewidth
%               \ifdim\wd0>\lw
%                 \sbox0{\tiny\t}^^A
%                 \ifdim\wd0>\linewidth
%                   \lwbox
%                 \else
%                   \usebox0
%                 \fi
%               \else
%                 \lwbox
%               \fi
%             \else
%               \usebox0
%             \fi
%           \else
%             \lwbox
%           \fi
%         \else
%           \usebox0
%         \fi
%       \else
%         \lwbox
%       \fi
%     \else
%       \usebox0
%     \fi
%   \else
%     \lwbox
%   \fi
% \else
%   \usebox0
% \fi
% \end{quote}
% If you have a \xfile{docstrip.cfg} that configures and enables \docstrip's
% TDS installing feature, then some files can already be in the right
% place, see the documentation of \docstrip.
%
% \subsection{Refresh file name databases}
%
% If your \TeX~distribution
% (\teTeX, \mikTeX, \dots) relies on file name databases, you must refresh
% these. For example, \teTeX\ users run \verb|texhash| or
% \verb|mktexlsr|.
%
% \subsection{Some details for the interested}
%
% \paragraph{Unpacking with \LaTeX.}
% The \xfile{.dtx} chooses its action depending on the format:
% \begin{description}
% \item[\plainTeX:] Run \docstrip\ and extract the files.
% \item[\LaTeX:] Generate the documentation.
% \end{description}
% If you insist on using \LaTeX\ for \docstrip\ (really,
% \docstrip\ does not need \LaTeX), then inform the autodetect routine
% about your intention:
% \begin{quote}
%   \verb|latex \let\install=y\input{tabularht.dtx}|
% \end{quote}
% Do not forget to quote the argument according to the demands
% of your shell.
%
% \paragraph{Generating the documentation.}
% You can use both the \xfile{.dtx} or the \xfile{.drv} to generate
% the documentation. The process can be configured by the
% configuration file \xfile{ltxdoc.cfg}. For instance, put this
% line into this file, if you want to have A4 as paper format:
% \begin{quote}
%   \verb|\PassOptionsToClass{a4paper}{article}|
% \end{quote}
% An example follows how to generate the
% documentation with pdf\LaTeX:
% \begin{quote}
%\begin{verbatim}
%pdflatex tabularht.dtx
%makeindex -s gind.ist tabularht.idx
%pdflatex tabularht.dtx
%makeindex -s gind.ist tabularht.idx
%pdflatex tabularht.dtx
%\end{verbatim}
% \end{quote}
%
% \begin{History}
%   \begin{Version}{2005/09/22 v1.0}
%   \item
%     First public version.
%   \end{Version}
%   \begin{Version}{2005/10/16 v2.0}
%   \item
%     Height specification allows
%     |to=...| or |spread=...|, default is |to=|.
%   \item
%     Option \xoption{vlines} added,
%     drivers \xoption{pdftex} and \xoption{dvips}.
%   \item
%     \cs{interrowspace}, \cs{interrowfil}, and
%     \cs{interrowstart}\dots\cs{interrowstop} added.
%   \end{Version}
%   \begin{Version}{2005/10/18 v2.1}
%   \item
%     Fix for package \xpackage{colortbl}, but the colors
%     of \xpackage{colortbl} remain unsupported.
%   \end{Version}
%   \begin{Version}{2006/02/20 v2.2}
%   \item
%     Code is not changed.
%   \item
%     DTX framework.
%   \end{Version}
%   \begin{Version}{2006/12/22 v2.3}
%   \item
%     Documentation fix.
%   \item
%     Fix in code of option vlines.
%   \end{Version}
%   \begin{Version}{2007/03/21 v2.4}
%   \item
%     Fix: Counter |tabh@unique| must not be changed by \cs{include}.
%   \end{Version}
%   \begin{Version}{2007/04/11 v2.5}
%   \item
%     Line ends sanitized.
%   \end{Version}
%   \begin{Version}{2016/05/16 v2.6}
%   \item
%     Documentation updates.
%   \end{Version}
% \end{History}
%
% \PrintIndex
%
% \Finale
\endinput

%        (quote the arguments according to the demands of your shell)
%
% Documentation:
%    (a) If tabularht.drv is present:
%           latex tabularht.drv
%    (b) Without tabularht.drv:
%           latex tabularht.dtx; ...
%    The class ltxdoc loads the configuration file ltxdoc.cfg
%    if available. Here you can specify further options, e.g.
%    use A4 as paper format:
%       \PassOptionsToClass{a4paper}{article}
%
%    Programm calls to get the documentation (example):
%       pdflatex tabularht.dtx
%       makeindex -s gind.ist tabularht.idx
%       pdflatex tabularht.dtx
%       makeindex -s gind.ist tabularht.idx
%       pdflatex tabularht.dtx
%
% Installation:
%    TDS:tex/latex/oberdiek/tabularht.sty
%    TDS:doc/latex/oberdiek/tabularht.pdf
%    TDS:doc/latex/oberdiek/tabularht-example1.tex
%    TDS:doc/latex/oberdiek/tabularht-example2.tex
%    TDS:source/latex/oberdiek/tabularht.dtx
%
%<*ignore>
\begingroup
  \catcode123=1 %
  \catcode125=2 %
  \def\x{LaTeX2e}%
\expandafter\endgroup
\ifcase 0\ifx\install y1\fi\expandafter
         \ifx\csname processbatchFile\endcsname\relax\else1\fi
         \ifx\fmtname\x\else 1\fi\relax
\else\csname fi\endcsname
%</ignore>
%<*install>
\input docstrip.tex
\Msg{************************************************************************}
\Msg{* Installation}
\Msg{* Package: tabularht 2016/05/16 v2.6 Tabular with height specified (HO)}
\Msg{************************************************************************}

\keepsilent
\askforoverwritefalse

\let\MetaPrefix\relax
\preamble

This is a generated file.

Project: tabularht
Version: 2016/05/16 v2.6

Copyright (C) 2005-2007 by
   Heiko Oberdiek <heiko.oberdiek at googlemail.com>

This work may be distributed and/or modified under the
conditions of the LaTeX Project Public License, either
version 1.3c of this license or (at your option) any later
version. This version of this license is in
   https://www.latex-project.org/lppl/lppl-1-3c.txt
and the latest version of this license is in
   https://www.latex-project.org/lppl.txt
and version 1.3 or later is part of all distributions of
LaTeX version 2005/12/01 or later.

This work has the LPPL maintenance status "maintained".

The Current Maintainers of this work are
Heiko Oberdiek and the Oberdiek Package Support Group
https://github.com/ho-tex/oberdiek/issues


This work consists of the main source file tabularht.dtx
and the derived files
   tabularht.sty, tabularht.pdf, tabularht.ins, tabularht.drv,
   tabularht-example1.tex, tabularht-example2.tex.

\endpreamble
\let\MetaPrefix\DoubleperCent

\generate{%
  \file{tabularht.ins}{\from{tabularht.dtx}{install}}%
  \file{tabularht.drv}{\from{tabularht.dtx}{driver}}%
  \usedir{tex/latex/oberdiek}%
  \file{tabularht.sty}{\from{tabularht.dtx}{package}}%
  \usedir{doc/latex/oberdiek}%
  \file{tabularht-example1.tex}{\from{tabularht.dtx}{example1}}%
  \file{tabularht-example2.tex}{\from{tabularht.dtx}{example2}}%
  \nopreamble
  \nopostamble
%  \usedir{source/latex/oberdiek/catalogue}%
%  \file{tabularht.xml}{\from{tabularht.dtx}{catalogue}}%
}

\catcode32=13\relax% active space
\let =\space%
\Msg{************************************************************************}
\Msg{*}
\Msg{* To finish the installation you have to move the following}
\Msg{* file into a directory searched by TeX:}
\Msg{*}
\Msg{*     tabularht.sty}
\Msg{*}
\Msg{* To produce the documentation run the file `tabularht.drv'}
\Msg{* through LaTeX.}
\Msg{*}
\Msg{* Happy TeXing!}
\Msg{*}
\Msg{************************************************************************}

\endbatchfile
%</install>
%<*ignore>
\fi
%</ignore>
%<*driver>
\NeedsTeXFormat{LaTeX2e}
\ProvidesFile{tabularht.drv}%
  [2016/05/16 v2.6 Tabular with height specified (HO)]%
\documentclass{ltxdoc}
\usepackage{holtxdoc}[2011/11/22]
\begin{document}
  \DocInput{tabularht.dtx}%
\end{document}
%</driver>
% \fi
%
%
% \CharacterTable
%  {Upper-case    \A\B\C\D\E\F\G\H\I\J\K\L\M\N\O\P\Q\R\S\T\U\V\W\X\Y\Z
%   Lower-case    \a\b\c\d\e\f\g\h\i\j\k\l\m\n\o\p\q\r\s\t\u\v\w\x\y\z
%   Digits        \0\1\2\3\4\5\6\7\8\9
%   Exclamation   \!     Double quote  \"     Hash (number) \#
%   Dollar        \$     Percent       \%     Ampersand     \&
%   Acute accent  \'     Left paren    \(     Right paren   \)
%   Asterisk      \*     Plus          \+     Comma         \,
%   Minus         \-     Point         \.     Solidus       \/
%   Colon         \:     Semicolon     \;     Less than     \<
%   Equals        \=     Greater than  \>     Question mark \?
%   Commercial at \@     Left bracket  \[     Backslash     \\
%   Right bracket \]     Circumflex    \^     Underscore    \_
%   Grave accent  \`     Left brace    \{     Vertical bar  \|
%   Right brace   \}     Tilde         \~}
%
% \GetFileInfo{tabularht.drv}
%
% \title{The \xpackage{tabularht} package}
% \date{2016/05/16 v2.6}
% \author{Heiko Oberdiek\thanks
% {Please report any issues at \url{https://github.com/ho-tex/oberdiek/issues}}}
%
% \maketitle
%
% \begin{abstract}
% This package defines some environments that adds
% a height specification to tabular and array.
% \end{abstract}
%
% \tableofcontents
%
% \section{Usage}
%
% \begin{quote}
%   |\usepackage{tabularht}|
% \end{quote}
% The package provides the following environments
% that extend the tabular/array environment by
% a height specification as first argument:
% \begin{itemize}
% \item |tabularht|, |tabularht*|
% \item |arrayht|
% \item |tabularhtx|
%       (if package \xpackage{tabularx} is loaded)
% \end{itemize}
% The height argument allows a length specification,
% package \xpackage{calc} is supported if used. This means,
% the tabular will have the specified height.
% You can also use the prefixes |to=| and |spread=|.
% |to=| is the default, |spread=| means, the natural
% height of the tabular box is changed by the length
% after |spread=|.
%
% Examples:
% \begin{quote}
% \begin{tabular}{@{}l@{ $\rightarrow$ }l@{}}
%   |\begin{tabularht}{1in}|
%   & height is 1in\\
%   |\begin{tabularht}{to=1in}|
%   & height is 1in\\
%   |\begin{tabularht}{spread=0pt}|
%   & natural height, same as |\begin{tabular}|\\
%   |\begin{tabularht}{spread=1in}|
%   & natural height increased by 1in
% \end{tabular}
% \end{quote}
%
% Hint:  See also package \xpackage{tabularkv}, it provides an interface,
%        where most parameters for the environments can be given
%        by key-value pairs.
%
% \begin{declcs}{interrowspace}\,|{|\dots|}|
% \end{declcs}
% Adds space between table rows. It is essentially
% the same as |\noalign{\vspace{...}}|.
% \begin{declcs}{interrowfill}
% \end{declcs}
% Short for |\interrowspace{\fill}|
% \begin{declcs}{interrowstart}\,\dots\,\cs{interrowstop}
% \end{declcs}
% Marker commands, useful for option \xoption{vlines}.
%
% \subsection{Option \xoption{vlines}}
%
% Warning: This stuff is experimental.
%
% \vspace{2ex}
% \noindent
% Vertical lines are interrupted, if space is inserted
% in \cs{noalign}, \cs{interrowspace}, \cs{addlinespace}
% (\xpackage{booktabs}),
% between double \cs{hline}s. This option tries to detect
% and add the vertical lines. The lines in a tabular
% with \xpackage{tabularht} support (environments of this package)
% are numbered from left to right. The gap that is
% controlled by \cs{interrowspace} or inbetween \cs{interrowstart}
% and \cs{interrowstop} is then filled with the detected
% vertical lines.
%
% If only a limited selection of the lines should be
% drawn, the commands know an optional argument with a
% list of line numbers, e.g.
% \begin{quote}
%\begin{verbatim}
%\begin{tabularht}{50mm}{|l|l|}
%  Hello & World\\
%  \interrowfill[1,3]
%  Foo & Bar
%\end{tabularht}
%\end{verbatim}
% \end{quote}
% There are three lines, but the middle line is not drawn
% in the gap between the first and second row.
% Zero can be used to suppress all lines:
% \begin{quote}
% |\interrowspace[0]{10mm}|
% \end{quote}
% The syntax of the commands with the optional argument
% with the line number list \meta{list}. \meta{list} is a comma
% separated list of numbers, \meta{height} means the
% height specification described above with the optional
% prefixes |to=| or |spread=|.
% \begin{quote}
%   \cs{interrowspace}\,|[|\meta{list}|]|\,|{|\meta{height}|}|\\
%   \cs{interrowfill}\,|[|\meta{list}|]|\\
%   \cs{interrowstart}\,|[|\meta{list}|]|\,\dots\,\cs{interrowstop}
% \end{quote}
%
% Option \xoption{vlines} is driver dependent and uses \eTeX\ features.
% \begin{description}
% \item[\xoption{pdftex}:] \pdfTeX\ in PDF mode.
%   Here the positions of the lines are written with the
%   help of the \cs{pdfsavepos} feature into the \xfile{.aux} file(s).
%   Therefore you need two LaTeX runs to get the lines.
% \item[\xoption{dvips}:]
%   Here, PostScript's currentpoint it used to get
%   the line positions. The lines are then drawn at the end
%   of the page. Thus one \LaTeX/dvips run is sufficient
%   for this option.
% \item[Other drivers:]\mbox{}
%   \begin{description}
%   \item[PostScript drivers:] probably possible, an end of page
%     hook would be nice.
%   \item[\VTeX:] with GeX (PostScript interpreter) probably possible.
%   \item[dvipdfm:] no idea. The big problem is, how to get the
%     current position?
%   \end{description}
% \end{description}
%
% \subsection{Limitations}
%
% \begin{itemize}
% \item
%   Vertical lines are interrupted by |\noalign{\vfill}|.
% \end{itemize}
%
% \subsection{Compatibility}
%
% \begin{itemize}
% \item
%   \xpackage{array}, \xpackage{delarray}, \xpackage{tabularx}
%   are supported.
% \item There can be problems with packages that
%   redefine \cs{@array} (or \cs{@@array}, \cs{@tabarray}) and
%   \cs{@arrayrule} (for option \xoption{vlines}).
% \item \xpackage{colortbl}: it should at least work, but there isn't
%   support for filling the gaps with color, neither
%   the rules nor the backgrounds.
% \end{itemize}
%
% \subsection{Examples}
%
% \subsubsection{Example 1}
%
%    \begin{macrocode}
%<*example1>
\documentclass{article}
\usepackage{tabularht}

\begin{document}
\fbox{%
  \begin{tabularht*}{1in}{4in}{@{}l@{\extracolsep{\fill}}r@{}}%
    upper left corner & upper right corner\\%
    \noalign{\vfill}%
    \multicolumn{2}{@{}c@{}}{bounding box}\\%
    \noalign{\vfill}%
    lower left corner & lower right corner\\%
  \end{tabularht*}%
}
\end{document}
%</example1>
%    \end{macrocode}
%
% \subsubsection{Example 2}
%
%    \begin{macrocode}
%<*example2>
\documentclass{article}
\usepackage{booktabs}
\usepackage[dvips,vlines]{tabularht}

\begin{document}

\begin{tabularht}{spread=0pt}{|l|l|}
  \hline
  First&Line\\%
  \hline
\interrowstart
  \addlinespace[10mm]%
\interrowstop
  \hline
  Second&Line\\%
\interrowstart
  \hline
  \hline
\interrowstop
  Third&Line\\%
  \hline
\interrowspace{10mm}
  \hline
  Fourth&Line\\%
  \hline
\end{tabularht}

\end{document}
%</example2>
%    \end{macrocode}
%
% \StopEventually{
% }
%
% \section{Implementation}
%
%    \begin{macrocode}
%<*package>
%    \end{macrocode}
%    Package identification.
%    \begin{macrocode}
\NeedsTeXFormat{LaTeX2e}
\ProvidesPackage{tabularht}%
  [2016/05/16 v2.6 Tabular with height specified (HO)]
%    \end{macrocode}
%
% \subsection{Environments}
%
%    \begin{macrocode}
\let\@toarrayheight\@empty
\let\tabH@array@init\@empty

\toks@={%
  \begingroup
    \long\def\x#1\vcenter\fi\fi\bgroup#2\@sharp#3#4\@nil{%
      \endgroup
      \gdef\@array[##1]##2{%
        \tabH@array@init
        #1%
        \vcenter\fi\fi
        \@toarrayheight
        \bgroup
        \let\@toarrayheight\@empty
        #2\@sharp###3#4%
      }%
    }%
  \expandafter\x\@array[#1]{#2}\@nil % hash-ok
}
\edef\tabH@patch@array{\the\toks@}
\def\tabH@patch@@array{%
  \ifx\@array\@@array
    \def\reserved@a{\let\@@array\@array}%
  \else
    \let\reserved@a\relax
  \fi
  \tabH@patch@array
  \reserved@a
}
\tabH@patch@@array

\@ifpackageloaded{array}{}{%
  \AtBeginDocument{%
    \@ifpackageloaded{array}{%
      \tabH@patch@@array
    }{}%
  }%
}

\def\tabH@setheight#1{%
  \tabH@@setheight#1==\@nil
}
\def\tabH@@setheight#1=#2=#3\@nil{%
  \ifx\\#2#3\\%
    \setlength{\dimen@}{#1}%
    \edef\@toarrayheight{to\the\dimen@}%
  \else
    \edef\tabH@temp{\zap@space#1 \@empty}%
    \ifx\tabH@temp\tabH@to
    \else
      \ifx\tabH@temp\tabH@spread
      \else
        \PackageError{tabularht}{%
          Unknown height specifier %
          `\expandafter\strip@prefix\meaning\tabH@temp'%
        }{%
          The height dimension for tabular height can be prefixed%
          \MessageBreak
          with `to=' or `spread=', default is `to='.%
        }%
        \let\tabH@temp\tabH@to
      \fi
    \fi
    \setlength{\dimen@}{#2}%
    \edef\@toarrayheight{\tabH@temp\the\dimen@}%
  \fi
}
\def\tabH@to{to}
\def\tabH@spread{spread}
%    \end{macrocode}
%    First argument is the height of the table,
%    then the orignal arguments for tabular follow.
%    \begin{macrocode}
\newenvironment{tabularht}[1]{%
  \tabH@setheight{#1}%
  \tabular
}{%
  \endtabular
}

\newenvironment{tabularht*}[1]{%
  \tabH@setheight{#1}%
  \@nameuse{tabular*}%
}{%
  \@nameuse{endtabular*}%
}

\newenvironment{tabularhtx}[1]{%
  \tabH@setheight{#1}%
  \tabularx
}{%
  \endtabularx
}

\newenvironment{arrayht}[1]{%
  \tabH@setheight{#1}%
  \array
}{%
  \endarray
}

\def\interrowspace{%
  \noalign\bgroup
    \tabH@interrowspace
}
\newcommand*{\tabH@interrowspace}[2][]{%
    \tabH@vspace{#1}{#2}%
  \egroup
}
\def\interrowfill{%
  \noalign\bgroup
    \tabH@interrowfill
}
\newcommand*{\tabH@interrowfill}[1][]{%
    \tabH@vspace{#1}{\fill}%
  \egroup
}
\def\tabH@vspace#1#2{%
  \tabH@vspace@start{#1}%
  \vspace{#2}%
  \tabH@vspace@stop
}
\let\tabH@vspace@start\@gobble
\let\tabH@vspace@stop\@empty

\newcommand*{\interrowstart}{%
  \noalign\bgroup
    \tabH@interrowstart
}
\newcommand*{\tabH@interrowstart}[1][]{%
    \tabH@vspace@start{#1}%
  \egroup
}
\newcommand*{\interrowstop}{%
  \noalign{\tabH@vspace@stop}%
}
%    \end{macrocode}
%
% \subsection{Options}
%
%    \begin{macrocode}
\providecommand*{\tabH@driver}{}

\DeclareOption{vlines}{%
  \let\tabH@temp\relax
}
\DeclareOption{pdftex}{}
\DeclareOption{dvips}{%
  \def\tabH@driver{dvips}%
}
\ProcessOptions*\relax

\ifx\tabH@temp\relax
\else
  \expandafter\endinput
\fi

\begingroup
  \@ifundefined{eTeXversion}{%
    \PackageError{tabularht}{%
      Option `vlines' requires eTeX%
    }{%
      Use of eTeX is recommended for LaTeX, see ltnews16.%
    }%
    \endgroup
    \endinput
  }{}%
\endgroup
%    \end{macrocode}
%
% \subsection{Option vlines, driver independent stuff}
%
%    \begin{macrocode}
\begingroup
  \let\@addtoreset\@gobbletwo
  \newcounter{tabH@unique}%
\endgroup
\let\tabH@currenttab\@empty

\def\tabH@array@init{%
  \ifx\@toarrayheight\@empty
    % ignore vertical lines of nested tabular environments
    \let\tabH@currenttab\@empty
  \else
    \stepcounter{tabH@unique}%
    \edef\tabH@currenttab{\the\c@tabH@unique}%
  \fi
}

\renewcommand*{\@arrayrule}{%
  \@addtopreamble{%
    \hskip -.5\arrayrulewidth
    \ifx\tabH@currenttab\@empty
    \else
      \tabH@vrule{\tabH@currenttab}%
    \fi
    \begingroup
      \expandafter\ifx\csname CT@arc@\endcsname\relax
      \else
        \expandafter\CT@arc@
      \fi
      \vline
    \endgroup
    \hskip -.5\arrayrulewidth
  }%
}
\let\tabH@arrayrule\@arrayrule
\AtBeginDocument{%
  \@ifpackageloaded{colortbl}{%
    \let\@arrayrule\tabH@arrayrule
  }{}%
}

\let\tabH@vrule\@gobble
%    \end{macrocode}
%
% \subsection{Driver pdftex}
%
%    \begin{macrocode}
\RequirePackage{ifpdf}
\ifpdf
  \begingroup
    \@ifundefined{pdfsavepos}{%
      \PackageError{tabularht}{%
        Your pdfTeX is too old%
      }{%
        \string\pdfsavepos\space is missing.%
      }%
      \endgroup
      \csname fi\endcsname
      \endinput
    }{}%

    \let\on@line\@empty
    \PackageInfo{tabularht}{%
      Using driver `pdftex' because of pdfTeX in PDF mode%
    }%
  \endgroup

  \protected\def\tabH@vrule#1{%
    \if@filesw
      \pdfsavepos
      \protected@write\@auxout{%
        \let\tabH@lastxpos\relax
      }{%
        \tabH@aux@vrule{#1}{\tabH@lastxpos}%
      }%
    \fi
  }%

  \def\tabH@lastxpos{\the\pdflastxpos}%
  \def\tabH@lastypos{\the\pdflastypos}%

  % The .aux file contains three commands:
  % \tabH@aux@vrule{tabular id}{x position}
  % \tabH@aux@vstart{tabular id}{row id}{x position}{y position}
  % \tabH@aux@vstop{y position}
  %
  \AtBeginDocument{%
    % The .aux files are read the first time before
    % \AtBeginDocument and later at \end{document}.
    % \tabH@aux@done is a marker to distinguish
    % between these two readings. Only in the first
    % case we need the \tabH@aux@... commands.
    \let\tabH@aux@done\@empty
    \if@filesw
      \immediate\write\@mainaux{%
        \@percentchar\@percentchar BeginProlog: tabularht%
      }%
      % items in the aux file are executed,
      % if tabularht is loaded
      % and during the aux file read at \begin{document} only
      \immediate\write\@mainaux{%
        \detokenize{%
          % the \tabH@aux@... commands are needed only if
          % tabularht is loaded with driver pdftex.
          \@ifundefined{tabH@aux@vrule}\@secondoftwo\@firstofone
          {%
            % disable commands except for the first .aux files reading
            \@ifundefined{tabH@aux@done}\@gobble\@firstofone
          }%
          {%
            \let\tabH@aux@vrule\@gobbletwo
            \let\tabH@aux@vstart\@gobblefour
            \let\tabH@aux@vstop\@gobble
          }%
        }%
      }%
      \immediate\write\@mainaux{%
        \@percentchar\@percentchar EndProlog: tabularht%
      }%
    \fi
  }%

  % the x positions of vrules are stored in
  % \tabH@<tabcount>list with distinct values
  \protected\def\tabH@aux@vrule#1#2{%
    \@ifundefined{tabH@#1list}{%
      \expandafter\xdef\csname tabH@#1list\endcsname{%
        \noexpand\do{#2}%
      }%
    }{%
      \begingroup
        \def\x{#2}%
        \let\y\@undefined
        \let\do\tabH@do@add
        \expandafter\xdef\csname tabH@#1list\endcsname{%
          \csname tabH@#1list\endcsname\@empty
          \ifx\y\@undefined
            \noexpand\do{\x}%
          \fi
        }%
      \endgroup
    }%
  }%
  \def\tabH@do@add#1{%
    \ifx\y\@undefined
      \ifnum#1<\x\space
      \else
        \expandafter\ifx\csname y\endcsname\relax\fi
        \ifnum#1>\x\space
          \noexpand\do{\x}%
        \fi
      \fi
    \fi
    \noexpand\do{#1}%
  }%

  \def\tabH@vspace@start#1{%
    \if@filesw
      \stepcounter{tabH@unique}%
      \edef\tabH@currentrow{\the\c@tabH@unique}%
      \pdfsavepos
      \protected@write\@auxout{%
        \let\tabH@lastxpos\relax
        \let\tabH@lastypos\relax
      }{%
        \tabH@aux@vstart{\tabH@currenttab}{\tabH@currentrow}%
                        {\tabH@lastxpos}{\tabH@lastypos}%
      }%
    \fi
    \begingroup
      \edef\a{tabH@\tabH@currenttab row\tabH@currentrow}%
      \expandafter\let\expandafter\x\csname\a x\endcsname
      \ifx\x\relax
      \else
        \expandafter\let\expandafter\y\csname\a y\endcsname
        \expandafter\let\expandafter\l
            \csname tabH@\tabH@currenttab list\endcsname
        \ifx\l\relax
        \else
          \def\f{#1}%
          \ifx\f\@empty
            \let\do\tabH@do@set
          \else
            \count@=\z@
            \let\do\tabH@do@filter
          \fi
          \setbox\z@=\hbox{\l}%
          \wd\z@=\z@
          \dp\z@=\z@
          \copy\z@
        \fi
      \fi
    \endgroup
  }%
  \def\tabH@vspace@stop{%
    \if@filesw
      \pdfsavepos
      \protected@write\@auxout{%
        \let\tabH@lastypos\relax
      }{%
        \tabH@aux@vstop{\tabH@lastypos}%
      }%
    \fi
  }%
  \def\tabH@do@set#1{%
    \hbox to \z@{%
      \hskip \dimexpr #1sp - \x sp\relax
      \vrule \@width\arrayrulewidth
             \@depth\dimexpr \y sp\relax
      \hss
    }%
  }%
  \def\tabH@do@filter{%
    \@tempswafalse
    \advance\count@\@ne
    \@for\e:=\f\do{%
      \ifnum\e=\count@
        \@tempswatrue
      \fi
    }%
    \if@tempswa
      \expandafter\tabH@do@set
    \else
      \expandafter\@gobble
    \fi
  }%

  \protected\def\tabH@aux@vstart#1#2#3#4{%
    \def\tabH@current@vstart{{#1}{#2}{#3}{#4}}%
  }%
  \protected\def\tabH@aux@vstop{%
    \expandafter\tabH@aux@v\tabH@current@vstart
  }%
  \def\tabH@aux@v#1#2#3#4#5{%
    \expandafter\gdef\csname tabH@#1row#2x\endcsname{#3}%
    \expandafter\xdef\csname tabH@#1row#2y\endcsname{%
      \the\numexpr #4 - #5\relax
    }%
  }%

  \csname fi\endcsname
  \endinput

\fi
%    \end{macrocode}
%
% \subsection{DVI drivers}
%
%    \begin{macrocode}
\ifx\tabH@driver\@empty
  \PackageError{tabularht}{%
    Missing DVI driver, option `vlines' disabled%
  }{%
    Supported DVI drivers: dvips.%
  }%
  \expandafter\endinput
\fi

\def\tabH@driver@dvips{%
  \def\tabH@literalps##1{\special{ps:SDict begin ##1 end}}%
  \def\tabH@headerps##1{\special{! ##1}}%
}

\@onelevel@sanitize\tabH@driver
\@ifundefined{tabH@driver@\tabH@driver}{%
  \PackageError{tabularht}{%
    Unsupported driver `\tabH@driver'%
  }{%
    Supported DVI drivers: dvips.%
  }%
  \endinput
}{}

\begingroup
  \let\on@line\@empty
  \PackageInfo{tabularht}{%
    Using driver `\tabH@driver'%
  }%
\endgroup
\csname tabH@driver@\tabH@driver\endcsname

\protected\def\tabH@vrule#1#2\vrule#3\arrayrulewidth{%
  #2% \fi or empty
  % hack to get rid of maxdrift rounding of dvips,
  % thus simulate a large motion
  \kern1in\relax
  \tabH@literalps{%
    #1 tabH.vrule %
    Resolution neg 0 translate%
  }%
  \vrule#3\arrayrulewidth
  \tabH@literalps{Resolution 0 translate}%
  \kern-1in\relax
}

\def\tabH@vspace@start#1{%
  \begingroup
    \let\y\@empty
    \@for\x:=#1\do{%
      \ifx\y\@empty
        \edef\y{\x}%
      \else
        \edef\y{\y\space\x}%
      \fi
    }%
    \tabH@literalps{\tabH@currenttab[\y]currentpoint exch pop}%
  \endgroup
}
\def\tabH@vspace@stop{%
  \tabH@literalps{%
    currentpoint exch pop %
    \number\dimexpr\arrayrulewidth\relax\space
    tabH.vspace%
  }%
}

\tabH@headerps{%
  userdict begin%
    /tabH.list 10 dict def%
    /tabH.job [] def %
  end%
  /tabH.vrule{%
    10 string cvs cvn dup tabH.list exch known{%
      tabH.list exch dup [ exch tabH.list exch get %
      currentpoint pop round exch true exch{%
        % tabH.list key [ ... x true i
        % tabH.list key [ ... false i
        exch{%
          % ... [ ... x i
          2 copy lt{false}{%
            2 copy eq{pop false}{exch true}ifelse%
          }ifelse%
        }{false}ifelse%
      }forall %
      pop%
      ]put%
    }{%
      tabH.list exch[currentpoint pop round]put%
    }ifelse%
  }bind def%
  % <tab num> <cols array> <ytop> <ybottom> <rulewidth[sp]>
  /tabH.vspace{%
    userdict begin %
      10 dict dup begin %
        exch 65536 div Resolution mul 72.27 div %
        % dvips uses a poor man's ceil function
        % see dopage.c before "drawrule": (int)(... + 0.9999999)
        0.9999999 add truncate%
        /rulewidth exch def %
        exch/ybottom exch def %
        exch/ytop exch def %
        exch/cols exch def %
        exch/tabkey exch 10 string cvs cvn def %
      end%
      /tabH.job exch[exch userdict/tabH.job get aload pop]def %
    end%
  }bind def %
  % Now we do the work at the end of the page.
  % Unhappily "eop-hook" cannot be used, because "eop"
  % executes "restore" before, so that all data are lost.
  TeXDict begin%
    /eop%
    [%
      {%
        tabH.job{%
          begin%
           /colarray %
             tabH.list tabkey known{tabH.list tabkey get}{[]}ifelse %
           def %
           cols length 0 eq not{%
             /colarray[%
               cols{1 sub %
                 dup 0 lt{pop}{%
                   dup colarray length ge{pop}{%
                     colarray exch get%
                   }ifelse%
                 }ifelse%
               }forall%
             ]def%
           }if %
           colarray{%
             % (rulewidth) == rulewidth == % debug
             Resolution sub %
             ytop rulewidth ytop ybottom sub v%
           }forall %
          end%
        }forall%
        % tabH.list{== ==}forall % debug
      }bind aload pop %
      TeXDict /eop get aload pop%
    ]cvx def %
  end%
}
%    \end{macrocode}
%
%    \begin{macrocode}
%</package>
%    \end{macrocode}
%
% \section{Installation}
%
% \subsection{Download}
%
% \paragraph{Package.} This package is available on
% CTAN\footnote{\CTANpkg{tabularht}}:
% \begin{description}
% \item[\CTAN{macros/latex/contrib/oberdiek/tabularht.dtx}] The source file.
% \item[\CTAN{macros/latex/contrib/oberdiek/tabularht.pdf}] Documentation.
% \end{description}
%
%
% \paragraph{Bundle.} All the packages of the bundle `oberdiek'
% are also available in a TDS compliant ZIP archive. There
% the packages are already unpacked and the documentation files
% are generated. The files and directories obey the TDS standard.
% \begin{description}
% \item[\CTANinstall{install/macros/latex/contrib/oberdiek.tds.zip}]
% \end{description}
% \emph{TDS} refers to the standard ``A Directory Structure
% for \TeX\ Files'' (\CTAN{tds/tds.pdf}). Directories
% with \xfile{texmf} in their name are usually organized this way.
%
% \subsection{Bundle installation}
%
% \paragraph{Unpacking.} Unpack the \xfile{oberdiek.tds.zip} in the
% TDS tree (also known as \xfile{texmf} tree) of your choice.
% Example (linux):
% \begin{quote}
%   |unzip oberdiek.tds.zip -d ~/texmf|
% \end{quote}
%
% \paragraph{Script installation.}
% Check the directory \xfile{TDS:scripts/oberdiek/} for
% scripts that need further installation steps.
%
% \subsection{Package installation}
%
% \paragraph{Unpacking.} The \xfile{.dtx} file is a self-extracting
% \docstrip\ archive. The files are extracted by running the
% \xfile{.dtx} through \plainTeX:
% \begin{quote}
%   \verb|tex tabularht.dtx|
% \end{quote}
%
% \paragraph{TDS.} Now the different files must be moved into
% the different directories in your installation TDS tree
% (also known as \xfile{texmf} tree):
% \begin{quote}
% \def\t{^^A
% \begin{tabular}{@{}>{\ttfamily}l@{ $\rightarrow$ }>{\ttfamily}l@{}}
%   tabularht.sty & tex/latex/oberdiek/tabularht.sty\\
%   tabularht.pdf & doc/latex/oberdiek/tabularht.pdf\\
%   tabularht-example1.tex & doc/latex/oberdiek/tabularht-example1.tex\\
%   tabularht-example2.tex & doc/latex/oberdiek/tabularht-example2.tex\\
%   tabularht.dtx & source/latex/oberdiek/tabularht.dtx\\
% \end{tabular}^^A
% }^^A
% \sbox0{\t}^^A
% \ifdim\wd0>\linewidth
%   \begingroup
%     \advance\linewidth by\leftmargin
%     \advance\linewidth by\rightmargin
%   \edef\x{\endgroup
%     \def\noexpand\lw{\the\linewidth}^^A
%   }\x
%   \def\lwbox{^^A
%     \leavevmode
%     \hbox to \linewidth{^^A
%       \kern-\leftmargin\relax
%       \hss
%       \usebox0
%       \hss
%       \kern-\rightmargin\relax
%     }^^A
%   }^^A
%   \ifdim\wd0>\lw
%     \sbox0{\small\t}^^A
%     \ifdim\wd0>\linewidth
%       \ifdim\wd0>\lw
%         \sbox0{\footnotesize\t}^^A
%         \ifdim\wd0>\linewidth
%           \ifdim\wd0>\lw
%             \sbox0{\scriptsize\t}^^A
%             \ifdim\wd0>\linewidth
%               \ifdim\wd0>\lw
%                 \sbox0{\tiny\t}^^A
%                 \ifdim\wd0>\linewidth
%                   \lwbox
%                 \else
%                   \usebox0
%                 \fi
%               \else
%                 \lwbox
%               \fi
%             \else
%               \usebox0
%             \fi
%           \else
%             \lwbox
%           \fi
%         \else
%           \usebox0
%         \fi
%       \else
%         \lwbox
%       \fi
%     \else
%       \usebox0
%     \fi
%   \else
%     \lwbox
%   \fi
% \else
%   \usebox0
% \fi
% \end{quote}
% If you have a \xfile{docstrip.cfg} that configures and enables \docstrip's
% TDS installing feature, then some files can already be in the right
% place, see the documentation of \docstrip.
%
% \subsection{Refresh file name databases}
%
% If your \TeX~distribution
% (\teTeX, \mikTeX, \dots) relies on file name databases, you must refresh
% these. For example, \teTeX\ users run \verb|texhash| or
% \verb|mktexlsr|.
%
% \subsection{Some details for the interested}
%
% \paragraph{Unpacking with \LaTeX.}
% The \xfile{.dtx} chooses its action depending on the format:
% \begin{description}
% \item[\plainTeX:] Run \docstrip\ and extract the files.
% \item[\LaTeX:] Generate the documentation.
% \end{description}
% If you insist on using \LaTeX\ for \docstrip\ (really,
% \docstrip\ does not need \LaTeX), then inform the autodetect routine
% about your intention:
% \begin{quote}
%   \verb|latex \let\install=y% \iffalse meta-comment
%
% File: tabularht.dtx
% Version: 2016/05/16 v2.6
% Info: Tabular with height specified
%
% Copyright (C) 2005-2007 by
%    Heiko Oberdiek <heiko.oberdiek at googlemail.com>
%    2016
%    https://github.com/ho-tex/oberdiek/issues
%
% This work may be distributed and/or modified under the
% conditions of the LaTeX Project Public License, either
% version 1.3c of this license or (at your option) any later
% version. This version of this license is in
%    https://www.latex-project.org/lppl/lppl-1-3c.txt
% and the latest version of this license is in
%    https://www.latex-project.org/lppl.txt
% and version 1.3 or later is part of all distributions of
% LaTeX version 2005/12/01 or later.
%
% This work has the LPPL maintenance status "maintained".
%
% The Current Maintainers of this work are
% Heiko Oberdiek and the Oberdiek Package Support Group
% https://github.com/ho-tex/oberdiek/issues
%
% This work consists of the main source file tabularht.dtx
% and the derived files
%    tabularht.sty, tabularht.pdf, tabularht.ins, tabularht.drv,
%    tabularht-example1.tex, tabularht-example2.tex.
%
% Distribution:
%    CTAN:macros/latex/contrib/oberdiek/tabularht.dtx
%    CTAN:macros/latex/contrib/oberdiek/tabularht.pdf
%
% Unpacking:
%    (a) If tabularht.ins is present:
%           tex tabularht.ins
%    (b) Without tabularht.ins:
%           tex tabularht.dtx
%    (c) If you insist on using LaTeX
%           latex \let\install=y\input{tabularht.dtx}
%        (quote the arguments according to the demands of your shell)
%
% Documentation:
%    (a) If tabularht.drv is present:
%           latex tabularht.drv
%    (b) Without tabularht.drv:
%           latex tabularht.dtx; ...
%    The class ltxdoc loads the configuration file ltxdoc.cfg
%    if available. Here you can specify further options, e.g.
%    use A4 as paper format:
%       \PassOptionsToClass{a4paper}{article}
%
%    Programm calls to get the documentation (example):
%       pdflatex tabularht.dtx
%       makeindex -s gind.ist tabularht.idx
%       pdflatex tabularht.dtx
%       makeindex -s gind.ist tabularht.idx
%       pdflatex tabularht.dtx
%
% Installation:
%    TDS:tex/latex/oberdiek/tabularht.sty
%    TDS:doc/latex/oberdiek/tabularht.pdf
%    TDS:doc/latex/oberdiek/tabularht-example1.tex
%    TDS:doc/latex/oberdiek/tabularht-example2.tex
%    TDS:source/latex/oberdiek/tabularht.dtx
%
%<*ignore>
\begingroup
  \catcode123=1 %
  \catcode125=2 %
  \def\x{LaTeX2e}%
\expandafter\endgroup
\ifcase 0\ifx\install y1\fi\expandafter
         \ifx\csname processbatchFile\endcsname\relax\else1\fi
         \ifx\fmtname\x\else 1\fi\relax
\else\csname fi\endcsname
%</ignore>
%<*install>
\input docstrip.tex
\Msg{************************************************************************}
\Msg{* Installation}
\Msg{* Package: tabularht 2016/05/16 v2.6 Tabular with height specified (HO)}
\Msg{************************************************************************}

\keepsilent
\askforoverwritefalse

\let\MetaPrefix\relax
\preamble

This is a generated file.

Project: tabularht
Version: 2016/05/16 v2.6

Copyright (C) 2005-2007 by
   Heiko Oberdiek <heiko.oberdiek at googlemail.com>

This work may be distributed and/or modified under the
conditions of the LaTeX Project Public License, either
version 1.3c of this license or (at your option) any later
version. This version of this license is in
   https://www.latex-project.org/lppl/lppl-1-3c.txt
and the latest version of this license is in
   https://www.latex-project.org/lppl.txt
and version 1.3 or later is part of all distributions of
LaTeX version 2005/12/01 or later.

This work has the LPPL maintenance status "maintained".

The Current Maintainers of this work are
Heiko Oberdiek and the Oberdiek Package Support Group
https://github.com/ho-tex/oberdiek/issues


This work consists of the main source file tabularht.dtx
and the derived files
   tabularht.sty, tabularht.pdf, tabularht.ins, tabularht.drv,
   tabularht-example1.tex, tabularht-example2.tex.

\endpreamble
\let\MetaPrefix\DoubleperCent

\generate{%
  \file{tabularht.ins}{\from{tabularht.dtx}{install}}%
  \file{tabularht.drv}{\from{tabularht.dtx}{driver}}%
  \usedir{tex/latex/oberdiek}%
  \file{tabularht.sty}{\from{tabularht.dtx}{package}}%
  \usedir{doc/latex/oberdiek}%
  \file{tabularht-example1.tex}{\from{tabularht.dtx}{example1}}%
  \file{tabularht-example2.tex}{\from{tabularht.dtx}{example2}}%
  \nopreamble
  \nopostamble
%  \usedir{source/latex/oberdiek/catalogue}%
%  \file{tabularht.xml}{\from{tabularht.dtx}{catalogue}}%
}

\catcode32=13\relax% active space
\let =\space%
\Msg{************************************************************************}
\Msg{*}
\Msg{* To finish the installation you have to move the following}
\Msg{* file into a directory searched by TeX:}
\Msg{*}
\Msg{*     tabularht.sty}
\Msg{*}
\Msg{* To produce the documentation run the file `tabularht.drv'}
\Msg{* through LaTeX.}
\Msg{*}
\Msg{* Happy TeXing!}
\Msg{*}
\Msg{************************************************************************}

\endbatchfile
%</install>
%<*ignore>
\fi
%</ignore>
%<*driver>
\NeedsTeXFormat{LaTeX2e}
\ProvidesFile{tabularht.drv}%
  [2016/05/16 v2.6 Tabular with height specified (HO)]%
\documentclass{ltxdoc}
\usepackage{holtxdoc}[2011/11/22]
\begin{document}
  \DocInput{tabularht.dtx}%
\end{document}
%</driver>
% \fi
%
%
% \CharacterTable
%  {Upper-case    \A\B\C\D\E\F\G\H\I\J\K\L\M\N\O\P\Q\R\S\T\U\V\W\X\Y\Z
%   Lower-case    \a\b\c\d\e\f\g\h\i\j\k\l\m\n\o\p\q\r\s\t\u\v\w\x\y\z
%   Digits        \0\1\2\3\4\5\6\7\8\9
%   Exclamation   \!     Double quote  \"     Hash (number) \#
%   Dollar        \$     Percent       \%     Ampersand     \&
%   Acute accent  \'     Left paren    \(     Right paren   \)
%   Asterisk      \*     Plus          \+     Comma         \,
%   Minus         \-     Point         \.     Solidus       \/
%   Colon         \:     Semicolon     \;     Less than     \<
%   Equals        \=     Greater than  \>     Question mark \?
%   Commercial at \@     Left bracket  \[     Backslash     \\
%   Right bracket \]     Circumflex    \^     Underscore    \_
%   Grave accent  \`     Left brace    \{     Vertical bar  \|
%   Right brace   \}     Tilde         \~}
%
% \GetFileInfo{tabularht.drv}
%
% \title{The \xpackage{tabularht} package}
% \date{2016/05/16 v2.6}
% \author{Heiko Oberdiek\thanks
% {Please report any issues at \url{https://github.com/ho-tex/oberdiek/issues}}}
%
% \maketitle
%
% \begin{abstract}
% This package defines some environments that adds
% a height specification to tabular and array.
% \end{abstract}
%
% \tableofcontents
%
% \section{Usage}
%
% \begin{quote}
%   |\usepackage{tabularht}|
% \end{quote}
% The package provides the following environments
% that extend the tabular/array environment by
% a height specification as first argument:
% \begin{itemize}
% \item |tabularht|, |tabularht*|
% \item |arrayht|
% \item |tabularhtx|
%       (if package \xpackage{tabularx} is loaded)
% \end{itemize}
% The height argument allows a length specification,
% package \xpackage{calc} is supported if used. This means,
% the tabular will have the specified height.
% You can also use the prefixes |to=| and |spread=|.
% |to=| is the default, |spread=| means, the natural
% height of the tabular box is changed by the length
% after |spread=|.
%
% Examples:
% \begin{quote}
% \begin{tabular}{@{}l@{ $\rightarrow$ }l@{}}
%   |\begin{tabularht}{1in}|
%   & height is 1in\\
%   |\begin{tabularht}{to=1in}|
%   & height is 1in\\
%   |\begin{tabularht}{spread=0pt}|
%   & natural height, same as |\begin{tabular}|\\
%   |\begin{tabularht}{spread=1in}|
%   & natural height increased by 1in
% \end{tabular}
% \end{quote}
%
% Hint:  See also package \xpackage{tabularkv}, it provides an interface,
%        where most parameters for the environments can be given
%        by key-value pairs.
%
% \begin{declcs}{interrowspace}\,|{|\dots|}|
% \end{declcs}
% Adds space between table rows. It is essentially
% the same as |\noalign{\vspace{...}}|.
% \begin{declcs}{interrowfill}
% \end{declcs}
% Short for |\interrowspace{\fill}|
% \begin{declcs}{interrowstart}\,\dots\,\cs{interrowstop}
% \end{declcs}
% Marker commands, useful for option \xoption{vlines}.
%
% \subsection{Option \xoption{vlines}}
%
% Warning: This stuff is experimental.
%
% \vspace{2ex}
% \noindent
% Vertical lines are interrupted, if space is inserted
% in \cs{noalign}, \cs{interrowspace}, \cs{addlinespace}
% (\xpackage{booktabs}),
% between double \cs{hline}s. This option tries to detect
% and add the vertical lines. The lines in a tabular
% with \xpackage{tabularht} support (environments of this package)
% are numbered from left to right. The gap that is
% controlled by \cs{interrowspace} or inbetween \cs{interrowstart}
% and \cs{interrowstop} is then filled with the detected
% vertical lines.
%
% If only a limited selection of the lines should be
% drawn, the commands know an optional argument with a
% list of line numbers, e.g.
% \begin{quote}
%\begin{verbatim}
%\begin{tabularht}{50mm}{|l|l|}
%  Hello & World\\
%  \interrowfill[1,3]
%  Foo & Bar
%\end{tabularht}
%\end{verbatim}
% \end{quote}
% There are three lines, but the middle line is not drawn
% in the gap between the first and second row.
% Zero can be used to suppress all lines:
% \begin{quote}
% |\interrowspace[0]{10mm}|
% \end{quote}
% The syntax of the commands with the optional argument
% with the line number list \meta{list}. \meta{list} is a comma
% separated list of numbers, \meta{height} means the
% height specification described above with the optional
% prefixes |to=| or |spread=|.
% \begin{quote}
%   \cs{interrowspace}\,|[|\meta{list}|]|\,|{|\meta{height}|}|\\
%   \cs{interrowfill}\,|[|\meta{list}|]|\\
%   \cs{interrowstart}\,|[|\meta{list}|]|\,\dots\,\cs{interrowstop}
% \end{quote}
%
% Option \xoption{vlines} is driver dependent and uses \eTeX\ features.
% \begin{description}
% \item[\xoption{pdftex}:] \pdfTeX\ in PDF mode.
%   Here the positions of the lines are written with the
%   help of the \cs{pdfsavepos} feature into the \xfile{.aux} file(s).
%   Therefore you need two LaTeX runs to get the lines.
% \item[\xoption{dvips}:]
%   Here, PostScript's currentpoint it used to get
%   the line positions. The lines are then drawn at the end
%   of the page. Thus one \LaTeX/dvips run is sufficient
%   for this option.
% \item[Other drivers:]\mbox{}
%   \begin{description}
%   \item[PostScript drivers:] probably possible, an end of page
%     hook would be nice.
%   \item[\VTeX:] with GeX (PostScript interpreter) probably possible.
%   \item[dvipdfm:] no idea. The big problem is, how to get the
%     current position?
%   \end{description}
% \end{description}
%
% \subsection{Limitations}
%
% \begin{itemize}
% \item
%   Vertical lines are interrupted by |\noalign{\vfill}|.
% \end{itemize}
%
% \subsection{Compatibility}
%
% \begin{itemize}
% \item
%   \xpackage{array}, \xpackage{delarray}, \xpackage{tabularx}
%   are supported.
% \item There can be problems with packages that
%   redefine \cs{@array} (or \cs{@@array}, \cs{@tabarray}) and
%   \cs{@arrayrule} (for option \xoption{vlines}).
% \item \xpackage{colortbl}: it should at least work, but there isn't
%   support for filling the gaps with color, neither
%   the rules nor the backgrounds.
% \end{itemize}
%
% \subsection{Examples}
%
% \subsubsection{Example 1}
%
%    \begin{macrocode}
%<*example1>
\documentclass{article}
\usepackage{tabularht}

\begin{document}
\fbox{%
  \begin{tabularht*}{1in}{4in}{@{}l@{\extracolsep{\fill}}r@{}}%
    upper left corner & upper right corner\\%
    \noalign{\vfill}%
    \multicolumn{2}{@{}c@{}}{bounding box}\\%
    \noalign{\vfill}%
    lower left corner & lower right corner\\%
  \end{tabularht*}%
}
\end{document}
%</example1>
%    \end{macrocode}
%
% \subsubsection{Example 2}
%
%    \begin{macrocode}
%<*example2>
\documentclass{article}
\usepackage{booktabs}
\usepackage[dvips,vlines]{tabularht}

\begin{document}

\begin{tabularht}{spread=0pt}{|l|l|}
  \hline
  First&Line\\%
  \hline
\interrowstart
  \addlinespace[10mm]%
\interrowstop
  \hline
  Second&Line\\%
\interrowstart
  \hline
  \hline
\interrowstop
  Third&Line\\%
  \hline
\interrowspace{10mm}
  \hline
  Fourth&Line\\%
  \hline
\end{tabularht}

\end{document}
%</example2>
%    \end{macrocode}
%
% \StopEventually{
% }
%
% \section{Implementation}
%
%    \begin{macrocode}
%<*package>
%    \end{macrocode}
%    Package identification.
%    \begin{macrocode}
\NeedsTeXFormat{LaTeX2e}
\ProvidesPackage{tabularht}%
  [2016/05/16 v2.6 Tabular with height specified (HO)]
%    \end{macrocode}
%
% \subsection{Environments}
%
%    \begin{macrocode}
\let\@toarrayheight\@empty
\let\tabH@array@init\@empty

\toks@={%
  \begingroup
    \long\def\x#1\vcenter\fi\fi\bgroup#2\@sharp#3#4\@nil{%
      \endgroup
      \gdef\@array[##1]##2{%
        \tabH@array@init
        #1%
        \vcenter\fi\fi
        \@toarrayheight
        \bgroup
        \let\@toarrayheight\@empty
        #2\@sharp###3#4%
      }%
    }%
  \expandafter\x\@array[#1]{#2}\@nil % hash-ok
}
\edef\tabH@patch@array{\the\toks@}
\def\tabH@patch@@array{%
  \ifx\@array\@@array
    \def\reserved@a{\let\@@array\@array}%
  \else
    \let\reserved@a\relax
  \fi
  \tabH@patch@array
  \reserved@a
}
\tabH@patch@@array

\@ifpackageloaded{array}{}{%
  \AtBeginDocument{%
    \@ifpackageloaded{array}{%
      \tabH@patch@@array
    }{}%
  }%
}

\def\tabH@setheight#1{%
  \tabH@@setheight#1==\@nil
}
\def\tabH@@setheight#1=#2=#3\@nil{%
  \ifx\\#2#3\\%
    \setlength{\dimen@}{#1}%
    \edef\@toarrayheight{to\the\dimen@}%
  \else
    \edef\tabH@temp{\zap@space#1 \@empty}%
    \ifx\tabH@temp\tabH@to
    \else
      \ifx\tabH@temp\tabH@spread
      \else
        \PackageError{tabularht}{%
          Unknown height specifier %
          `\expandafter\strip@prefix\meaning\tabH@temp'%
        }{%
          The height dimension for tabular height can be prefixed%
          \MessageBreak
          with `to=' or `spread=', default is `to='.%
        }%
        \let\tabH@temp\tabH@to
      \fi
    \fi
    \setlength{\dimen@}{#2}%
    \edef\@toarrayheight{\tabH@temp\the\dimen@}%
  \fi
}
\def\tabH@to{to}
\def\tabH@spread{spread}
%    \end{macrocode}
%    First argument is the height of the table,
%    then the orignal arguments for tabular follow.
%    \begin{macrocode}
\newenvironment{tabularht}[1]{%
  \tabH@setheight{#1}%
  \tabular
}{%
  \endtabular
}

\newenvironment{tabularht*}[1]{%
  \tabH@setheight{#1}%
  \@nameuse{tabular*}%
}{%
  \@nameuse{endtabular*}%
}

\newenvironment{tabularhtx}[1]{%
  \tabH@setheight{#1}%
  \tabularx
}{%
  \endtabularx
}

\newenvironment{arrayht}[1]{%
  \tabH@setheight{#1}%
  \array
}{%
  \endarray
}

\def\interrowspace{%
  \noalign\bgroup
    \tabH@interrowspace
}
\newcommand*{\tabH@interrowspace}[2][]{%
    \tabH@vspace{#1}{#2}%
  \egroup
}
\def\interrowfill{%
  \noalign\bgroup
    \tabH@interrowfill
}
\newcommand*{\tabH@interrowfill}[1][]{%
    \tabH@vspace{#1}{\fill}%
  \egroup
}
\def\tabH@vspace#1#2{%
  \tabH@vspace@start{#1}%
  \vspace{#2}%
  \tabH@vspace@stop
}
\let\tabH@vspace@start\@gobble
\let\tabH@vspace@stop\@empty

\newcommand*{\interrowstart}{%
  \noalign\bgroup
    \tabH@interrowstart
}
\newcommand*{\tabH@interrowstart}[1][]{%
    \tabH@vspace@start{#1}%
  \egroup
}
\newcommand*{\interrowstop}{%
  \noalign{\tabH@vspace@stop}%
}
%    \end{macrocode}
%
% \subsection{Options}
%
%    \begin{macrocode}
\providecommand*{\tabH@driver}{}

\DeclareOption{vlines}{%
  \let\tabH@temp\relax
}
\DeclareOption{pdftex}{}
\DeclareOption{dvips}{%
  \def\tabH@driver{dvips}%
}
\ProcessOptions*\relax

\ifx\tabH@temp\relax
\else
  \expandafter\endinput
\fi

\begingroup
  \@ifundefined{eTeXversion}{%
    \PackageError{tabularht}{%
      Option `vlines' requires eTeX%
    }{%
      Use of eTeX is recommended for LaTeX, see ltnews16.%
    }%
    \endgroup
    \endinput
  }{}%
\endgroup
%    \end{macrocode}
%
% \subsection{Option vlines, driver independent stuff}
%
%    \begin{macrocode}
\begingroup
  \let\@addtoreset\@gobbletwo
  \newcounter{tabH@unique}%
\endgroup
\let\tabH@currenttab\@empty

\def\tabH@array@init{%
  \ifx\@toarrayheight\@empty
    % ignore vertical lines of nested tabular environments
    \let\tabH@currenttab\@empty
  \else
    \stepcounter{tabH@unique}%
    \edef\tabH@currenttab{\the\c@tabH@unique}%
  \fi
}

\renewcommand*{\@arrayrule}{%
  \@addtopreamble{%
    \hskip -.5\arrayrulewidth
    \ifx\tabH@currenttab\@empty
    \else
      \tabH@vrule{\tabH@currenttab}%
    \fi
    \begingroup
      \expandafter\ifx\csname CT@arc@\endcsname\relax
      \else
        \expandafter\CT@arc@
      \fi
      \vline
    \endgroup
    \hskip -.5\arrayrulewidth
  }%
}
\let\tabH@arrayrule\@arrayrule
\AtBeginDocument{%
  \@ifpackageloaded{colortbl}{%
    \let\@arrayrule\tabH@arrayrule
  }{}%
}

\let\tabH@vrule\@gobble
%    \end{macrocode}
%
% \subsection{Driver pdftex}
%
%    \begin{macrocode}
\RequirePackage{ifpdf}
\ifpdf
  \begingroup
    \@ifundefined{pdfsavepos}{%
      \PackageError{tabularht}{%
        Your pdfTeX is too old%
      }{%
        \string\pdfsavepos\space is missing.%
      }%
      \endgroup
      \csname fi\endcsname
      \endinput
    }{}%

    \let\on@line\@empty
    \PackageInfo{tabularht}{%
      Using driver `pdftex' because of pdfTeX in PDF mode%
    }%
  \endgroup

  \protected\def\tabH@vrule#1{%
    \if@filesw
      \pdfsavepos
      \protected@write\@auxout{%
        \let\tabH@lastxpos\relax
      }{%
        \tabH@aux@vrule{#1}{\tabH@lastxpos}%
      }%
    \fi
  }%

  \def\tabH@lastxpos{\the\pdflastxpos}%
  \def\tabH@lastypos{\the\pdflastypos}%

  % The .aux file contains three commands:
  % \tabH@aux@vrule{tabular id}{x position}
  % \tabH@aux@vstart{tabular id}{row id}{x position}{y position}
  % \tabH@aux@vstop{y position}
  %
  \AtBeginDocument{%
    % The .aux files are read the first time before
    % \AtBeginDocument and later at \end{document}.
    % \tabH@aux@done is a marker to distinguish
    % between these two readings. Only in the first
    % case we need the \tabH@aux@... commands.
    \let\tabH@aux@done\@empty
    \if@filesw
      \immediate\write\@mainaux{%
        \@percentchar\@percentchar BeginProlog: tabularht%
      }%
      % items in the aux file are executed,
      % if tabularht is loaded
      % and during the aux file read at \begin{document} only
      \immediate\write\@mainaux{%
        \detokenize{%
          % the \tabH@aux@... commands are needed only if
          % tabularht is loaded with driver pdftex.
          \@ifundefined{tabH@aux@vrule}\@secondoftwo\@firstofone
          {%
            % disable commands except for the first .aux files reading
            \@ifundefined{tabH@aux@done}\@gobble\@firstofone
          }%
          {%
            \let\tabH@aux@vrule\@gobbletwo
            \let\tabH@aux@vstart\@gobblefour
            \let\tabH@aux@vstop\@gobble
          }%
        }%
      }%
      \immediate\write\@mainaux{%
        \@percentchar\@percentchar EndProlog: tabularht%
      }%
    \fi
  }%

  % the x positions of vrules are stored in
  % \tabH@<tabcount>list with distinct values
  \protected\def\tabH@aux@vrule#1#2{%
    \@ifundefined{tabH@#1list}{%
      \expandafter\xdef\csname tabH@#1list\endcsname{%
        \noexpand\do{#2}%
      }%
    }{%
      \begingroup
        \def\x{#2}%
        \let\y\@undefined
        \let\do\tabH@do@add
        \expandafter\xdef\csname tabH@#1list\endcsname{%
          \csname tabH@#1list\endcsname\@empty
          \ifx\y\@undefined
            \noexpand\do{\x}%
          \fi
        }%
      \endgroup
    }%
  }%
  \def\tabH@do@add#1{%
    \ifx\y\@undefined
      \ifnum#1<\x\space
      \else
        \expandafter\ifx\csname y\endcsname\relax\fi
        \ifnum#1>\x\space
          \noexpand\do{\x}%
        \fi
      \fi
    \fi
    \noexpand\do{#1}%
  }%

  \def\tabH@vspace@start#1{%
    \if@filesw
      \stepcounter{tabH@unique}%
      \edef\tabH@currentrow{\the\c@tabH@unique}%
      \pdfsavepos
      \protected@write\@auxout{%
        \let\tabH@lastxpos\relax
        \let\tabH@lastypos\relax
      }{%
        \tabH@aux@vstart{\tabH@currenttab}{\tabH@currentrow}%
                        {\tabH@lastxpos}{\tabH@lastypos}%
      }%
    \fi
    \begingroup
      \edef\a{tabH@\tabH@currenttab row\tabH@currentrow}%
      \expandafter\let\expandafter\x\csname\a x\endcsname
      \ifx\x\relax
      \else
        \expandafter\let\expandafter\y\csname\a y\endcsname
        \expandafter\let\expandafter\l
            \csname tabH@\tabH@currenttab list\endcsname
        \ifx\l\relax
        \else
          \def\f{#1}%
          \ifx\f\@empty
            \let\do\tabH@do@set
          \else
            \count@=\z@
            \let\do\tabH@do@filter
          \fi
          \setbox\z@=\hbox{\l}%
          \wd\z@=\z@
          \dp\z@=\z@
          \copy\z@
        \fi
      \fi
    \endgroup
  }%
  \def\tabH@vspace@stop{%
    \if@filesw
      \pdfsavepos
      \protected@write\@auxout{%
        \let\tabH@lastypos\relax
      }{%
        \tabH@aux@vstop{\tabH@lastypos}%
      }%
    \fi
  }%
  \def\tabH@do@set#1{%
    \hbox to \z@{%
      \hskip \dimexpr #1sp - \x sp\relax
      \vrule \@width\arrayrulewidth
             \@depth\dimexpr \y sp\relax
      \hss
    }%
  }%
  \def\tabH@do@filter{%
    \@tempswafalse
    \advance\count@\@ne
    \@for\e:=\f\do{%
      \ifnum\e=\count@
        \@tempswatrue
      \fi
    }%
    \if@tempswa
      \expandafter\tabH@do@set
    \else
      \expandafter\@gobble
    \fi
  }%

  \protected\def\tabH@aux@vstart#1#2#3#4{%
    \def\tabH@current@vstart{{#1}{#2}{#3}{#4}}%
  }%
  \protected\def\tabH@aux@vstop{%
    \expandafter\tabH@aux@v\tabH@current@vstart
  }%
  \def\tabH@aux@v#1#2#3#4#5{%
    \expandafter\gdef\csname tabH@#1row#2x\endcsname{#3}%
    \expandafter\xdef\csname tabH@#1row#2y\endcsname{%
      \the\numexpr #4 - #5\relax
    }%
  }%

  \csname fi\endcsname
  \endinput

\fi
%    \end{macrocode}
%
% \subsection{DVI drivers}
%
%    \begin{macrocode}
\ifx\tabH@driver\@empty
  \PackageError{tabularht}{%
    Missing DVI driver, option `vlines' disabled%
  }{%
    Supported DVI drivers: dvips.%
  }%
  \expandafter\endinput
\fi

\def\tabH@driver@dvips{%
  \def\tabH@literalps##1{\special{ps:SDict begin ##1 end}}%
  \def\tabH@headerps##1{\special{! ##1}}%
}

\@onelevel@sanitize\tabH@driver
\@ifundefined{tabH@driver@\tabH@driver}{%
  \PackageError{tabularht}{%
    Unsupported driver `\tabH@driver'%
  }{%
    Supported DVI drivers: dvips.%
  }%
  \endinput
}{}

\begingroup
  \let\on@line\@empty
  \PackageInfo{tabularht}{%
    Using driver `\tabH@driver'%
  }%
\endgroup
\csname tabH@driver@\tabH@driver\endcsname

\protected\def\tabH@vrule#1#2\vrule#3\arrayrulewidth{%
  #2% \fi or empty
  % hack to get rid of maxdrift rounding of dvips,
  % thus simulate a large motion
  \kern1in\relax
  \tabH@literalps{%
    #1 tabH.vrule %
    Resolution neg 0 translate%
  }%
  \vrule#3\arrayrulewidth
  \tabH@literalps{Resolution 0 translate}%
  \kern-1in\relax
}

\def\tabH@vspace@start#1{%
  \begingroup
    \let\y\@empty
    \@for\x:=#1\do{%
      \ifx\y\@empty
        \edef\y{\x}%
      \else
        \edef\y{\y\space\x}%
      \fi
    }%
    \tabH@literalps{\tabH@currenttab[\y]currentpoint exch pop}%
  \endgroup
}
\def\tabH@vspace@stop{%
  \tabH@literalps{%
    currentpoint exch pop %
    \number\dimexpr\arrayrulewidth\relax\space
    tabH.vspace%
  }%
}

\tabH@headerps{%
  userdict begin%
    /tabH.list 10 dict def%
    /tabH.job [] def %
  end%
  /tabH.vrule{%
    10 string cvs cvn dup tabH.list exch known{%
      tabH.list exch dup [ exch tabH.list exch get %
      currentpoint pop round exch true exch{%
        % tabH.list key [ ... x true i
        % tabH.list key [ ... false i
        exch{%
          % ... [ ... x i
          2 copy lt{false}{%
            2 copy eq{pop false}{exch true}ifelse%
          }ifelse%
        }{false}ifelse%
      }forall %
      pop%
      ]put%
    }{%
      tabH.list exch[currentpoint pop round]put%
    }ifelse%
  }bind def%
  % <tab num> <cols array> <ytop> <ybottom> <rulewidth[sp]>
  /tabH.vspace{%
    userdict begin %
      10 dict dup begin %
        exch 65536 div Resolution mul 72.27 div %
        % dvips uses a poor man's ceil function
        % see dopage.c before "drawrule": (int)(... + 0.9999999)
        0.9999999 add truncate%
        /rulewidth exch def %
        exch/ybottom exch def %
        exch/ytop exch def %
        exch/cols exch def %
        exch/tabkey exch 10 string cvs cvn def %
      end%
      /tabH.job exch[exch userdict/tabH.job get aload pop]def %
    end%
  }bind def %
  % Now we do the work at the end of the page.
  % Unhappily "eop-hook" cannot be used, because "eop"
  % executes "restore" before, so that all data are lost.
  TeXDict begin%
    /eop%
    [%
      {%
        tabH.job{%
          begin%
           /colarray %
             tabH.list tabkey known{tabH.list tabkey get}{[]}ifelse %
           def %
           cols length 0 eq not{%
             /colarray[%
               cols{1 sub %
                 dup 0 lt{pop}{%
                   dup colarray length ge{pop}{%
                     colarray exch get%
                   }ifelse%
                 }ifelse%
               }forall%
             ]def%
           }if %
           colarray{%
             % (rulewidth) == rulewidth == % debug
             Resolution sub %
             ytop rulewidth ytop ybottom sub v%
           }forall %
          end%
        }forall%
        % tabH.list{== ==}forall % debug
      }bind aload pop %
      TeXDict /eop get aload pop%
    ]cvx def %
  end%
}
%    \end{macrocode}
%
%    \begin{macrocode}
%</package>
%    \end{macrocode}
%
% \section{Installation}
%
% \subsection{Download}
%
% \paragraph{Package.} This package is available on
% CTAN\footnote{\CTANpkg{tabularht}}:
% \begin{description}
% \item[\CTAN{macros/latex/contrib/oberdiek/tabularht.dtx}] The source file.
% \item[\CTAN{macros/latex/contrib/oberdiek/tabularht.pdf}] Documentation.
% \end{description}
%
%
% \paragraph{Bundle.} All the packages of the bundle `oberdiek'
% are also available in a TDS compliant ZIP archive. There
% the packages are already unpacked and the documentation files
% are generated. The files and directories obey the TDS standard.
% \begin{description}
% \item[\CTANinstall{install/macros/latex/contrib/oberdiek.tds.zip}]
% \end{description}
% \emph{TDS} refers to the standard ``A Directory Structure
% for \TeX\ Files'' (\CTAN{tds/tds.pdf}). Directories
% with \xfile{texmf} in their name are usually organized this way.
%
% \subsection{Bundle installation}
%
% \paragraph{Unpacking.} Unpack the \xfile{oberdiek.tds.zip} in the
% TDS tree (also known as \xfile{texmf} tree) of your choice.
% Example (linux):
% \begin{quote}
%   |unzip oberdiek.tds.zip -d ~/texmf|
% \end{quote}
%
% \paragraph{Script installation.}
% Check the directory \xfile{TDS:scripts/oberdiek/} for
% scripts that need further installation steps.
%
% \subsection{Package installation}
%
% \paragraph{Unpacking.} The \xfile{.dtx} file is a self-extracting
% \docstrip\ archive. The files are extracted by running the
% \xfile{.dtx} through \plainTeX:
% \begin{quote}
%   \verb|tex tabularht.dtx|
% \end{quote}
%
% \paragraph{TDS.} Now the different files must be moved into
% the different directories in your installation TDS tree
% (also known as \xfile{texmf} tree):
% \begin{quote}
% \def\t{^^A
% \begin{tabular}{@{}>{\ttfamily}l@{ $\rightarrow$ }>{\ttfamily}l@{}}
%   tabularht.sty & tex/latex/oberdiek/tabularht.sty\\
%   tabularht.pdf & doc/latex/oberdiek/tabularht.pdf\\
%   tabularht-example1.tex & doc/latex/oberdiek/tabularht-example1.tex\\
%   tabularht-example2.tex & doc/latex/oberdiek/tabularht-example2.tex\\
%   tabularht.dtx & source/latex/oberdiek/tabularht.dtx\\
% \end{tabular}^^A
% }^^A
% \sbox0{\t}^^A
% \ifdim\wd0>\linewidth
%   \begingroup
%     \advance\linewidth by\leftmargin
%     \advance\linewidth by\rightmargin
%   \edef\x{\endgroup
%     \def\noexpand\lw{\the\linewidth}^^A
%   }\x
%   \def\lwbox{^^A
%     \leavevmode
%     \hbox to \linewidth{^^A
%       \kern-\leftmargin\relax
%       \hss
%       \usebox0
%       \hss
%       \kern-\rightmargin\relax
%     }^^A
%   }^^A
%   \ifdim\wd0>\lw
%     \sbox0{\small\t}^^A
%     \ifdim\wd0>\linewidth
%       \ifdim\wd0>\lw
%         \sbox0{\footnotesize\t}^^A
%         \ifdim\wd0>\linewidth
%           \ifdim\wd0>\lw
%             \sbox0{\scriptsize\t}^^A
%             \ifdim\wd0>\linewidth
%               \ifdim\wd0>\lw
%                 \sbox0{\tiny\t}^^A
%                 \ifdim\wd0>\linewidth
%                   \lwbox
%                 \else
%                   \usebox0
%                 \fi
%               \else
%                 \lwbox
%               \fi
%             \else
%               \usebox0
%             \fi
%           \else
%             \lwbox
%           \fi
%         \else
%           \usebox0
%         \fi
%       \else
%         \lwbox
%       \fi
%     \else
%       \usebox0
%     \fi
%   \else
%     \lwbox
%   \fi
% \else
%   \usebox0
% \fi
% \end{quote}
% If you have a \xfile{docstrip.cfg} that configures and enables \docstrip's
% TDS installing feature, then some files can already be in the right
% place, see the documentation of \docstrip.
%
% \subsection{Refresh file name databases}
%
% If your \TeX~distribution
% (\teTeX, \mikTeX, \dots) relies on file name databases, you must refresh
% these. For example, \teTeX\ users run \verb|texhash| or
% \verb|mktexlsr|.
%
% \subsection{Some details for the interested}
%
% \paragraph{Unpacking with \LaTeX.}
% The \xfile{.dtx} chooses its action depending on the format:
% \begin{description}
% \item[\plainTeX:] Run \docstrip\ and extract the files.
% \item[\LaTeX:] Generate the documentation.
% \end{description}
% If you insist on using \LaTeX\ for \docstrip\ (really,
% \docstrip\ does not need \LaTeX), then inform the autodetect routine
% about your intention:
% \begin{quote}
%   \verb|latex \let\install=y\input{tabularht.dtx}|
% \end{quote}
% Do not forget to quote the argument according to the demands
% of your shell.
%
% \paragraph{Generating the documentation.}
% You can use both the \xfile{.dtx} or the \xfile{.drv} to generate
% the documentation. The process can be configured by the
% configuration file \xfile{ltxdoc.cfg}. For instance, put this
% line into this file, if you want to have A4 as paper format:
% \begin{quote}
%   \verb|\PassOptionsToClass{a4paper}{article}|
% \end{quote}
% An example follows how to generate the
% documentation with pdf\LaTeX:
% \begin{quote}
%\begin{verbatim}
%pdflatex tabularht.dtx
%makeindex -s gind.ist tabularht.idx
%pdflatex tabularht.dtx
%makeindex -s gind.ist tabularht.idx
%pdflatex tabularht.dtx
%\end{verbatim}
% \end{quote}
%
% \begin{History}
%   \begin{Version}{2005/09/22 v1.0}
%   \item
%     First public version.
%   \end{Version}
%   \begin{Version}{2005/10/16 v2.0}
%   \item
%     Height specification allows
%     |to=...| or |spread=...|, default is |to=|.
%   \item
%     Option \xoption{vlines} added,
%     drivers \xoption{pdftex} and \xoption{dvips}.
%   \item
%     \cs{interrowspace}, \cs{interrowfil}, and
%     \cs{interrowstart}\dots\cs{interrowstop} added.
%   \end{Version}
%   \begin{Version}{2005/10/18 v2.1}
%   \item
%     Fix for package \xpackage{colortbl}, but the colors
%     of \xpackage{colortbl} remain unsupported.
%   \end{Version}
%   \begin{Version}{2006/02/20 v2.2}
%   \item
%     Code is not changed.
%   \item
%     DTX framework.
%   \end{Version}
%   \begin{Version}{2006/12/22 v2.3}
%   \item
%     Documentation fix.
%   \item
%     Fix in code of option vlines.
%   \end{Version}
%   \begin{Version}{2007/03/21 v2.4}
%   \item
%     Fix: Counter |tabh@unique| must not be changed by \cs{include}.
%   \end{Version}
%   \begin{Version}{2007/04/11 v2.5}
%   \item
%     Line ends sanitized.
%   \end{Version}
%   \begin{Version}{2016/05/16 v2.6}
%   \item
%     Documentation updates.
%   \end{Version}
% \end{History}
%
% \PrintIndex
%
% \Finale
\endinput
|
% \end{quote}
% Do not forget to quote the argument according to the demands
% of your shell.
%
% \paragraph{Generating the documentation.}
% You can use both the \xfile{.dtx} or the \xfile{.drv} to generate
% the documentation. The process can be configured by the
% configuration file \xfile{ltxdoc.cfg}. For instance, put this
% line into this file, if you want to have A4 as paper format:
% \begin{quote}
%   \verb|\PassOptionsToClass{a4paper}{article}|
% \end{quote}
% An example follows how to generate the
% documentation with pdf\LaTeX:
% \begin{quote}
%\begin{verbatim}
%pdflatex tabularht.dtx
%makeindex -s gind.ist tabularht.idx
%pdflatex tabularht.dtx
%makeindex -s gind.ist tabularht.idx
%pdflatex tabularht.dtx
%\end{verbatim}
% \end{quote}
%
% \begin{History}
%   \begin{Version}{2005/09/22 v1.0}
%   \item
%     First public version.
%   \end{Version}
%   \begin{Version}{2005/10/16 v2.0}
%   \item
%     Height specification allows
%     |to=...| or |spread=...|, default is |to=|.
%   \item
%     Option \xoption{vlines} added,
%     drivers \xoption{pdftex} and \xoption{dvips}.
%   \item
%     \cs{interrowspace}, \cs{interrowfil}, and
%     \cs{interrowstart}\dots\cs{interrowstop} added.
%   \end{Version}
%   \begin{Version}{2005/10/18 v2.1}
%   \item
%     Fix for package \xpackage{colortbl}, but the colors
%     of \xpackage{colortbl} remain unsupported.
%   \end{Version}
%   \begin{Version}{2006/02/20 v2.2}
%   \item
%     Code is not changed.
%   \item
%     DTX framework.
%   \end{Version}
%   \begin{Version}{2006/12/22 v2.3}
%   \item
%     Documentation fix.
%   \item
%     Fix in code of option vlines.
%   \end{Version}
%   \begin{Version}{2007/03/21 v2.4}
%   \item
%     Fix: Counter |tabh@unique| must not be changed by \cs{include}.
%   \end{Version}
%   \begin{Version}{2007/04/11 v2.5}
%   \item
%     Line ends sanitized.
%   \end{Version}
%   \begin{Version}{2016/05/16 v2.6}
%   \item
%     Documentation updates.
%   \end{Version}
% \end{History}
%
% \PrintIndex
%
% \Finale
\endinput
|
% \end{quote}
% Do not forget to quote the argument according to the demands
% of your shell.
%
% \paragraph{Generating the documentation.}
% You can use both the \xfile{.dtx} or the \xfile{.drv} to generate
% the documentation. The process can be configured by the
% configuration file \xfile{ltxdoc.cfg}. For instance, put this
% line into this file, if you want to have A4 as paper format:
% \begin{quote}
%   \verb|\PassOptionsToClass{a4paper}{article}|
% \end{quote}
% An example follows how to generate the
% documentation with pdf\LaTeX:
% \begin{quote}
%\begin{verbatim}
%pdflatex tabularht.dtx
%makeindex -s gind.ist tabularht.idx
%pdflatex tabularht.dtx
%makeindex -s gind.ist tabularht.idx
%pdflatex tabularht.dtx
%\end{verbatim}
% \end{quote}
%
% \begin{History}
%   \begin{Version}{2005/09/22 v1.0}
%   \item
%     First public version.
%   \end{Version}
%   \begin{Version}{2005/10/16 v2.0}
%   \item
%     Height specification allows
%     |to=...| or |spread=...|, default is |to=|.
%   \item
%     Option \xoption{vlines} added,
%     drivers \xoption{pdftex} and \xoption{dvips}.
%   \item
%     \cs{interrowspace}, \cs{interrowfil}, and
%     \cs{interrowstart}\dots\cs{interrowstop} added.
%   \end{Version}
%   \begin{Version}{2005/10/18 v2.1}
%   \item
%     Fix for package \xpackage{colortbl}, but the colors
%     of \xpackage{colortbl} remain unsupported.
%   \end{Version}
%   \begin{Version}{2006/02/20 v2.2}
%   \item
%     Code is not changed.
%   \item
%     DTX framework.
%   \end{Version}
%   \begin{Version}{2006/12/22 v2.3}
%   \item
%     Documentation fix.
%   \item
%     Fix in code of option vlines.
%   \end{Version}
%   \begin{Version}{2007/03/21 v2.4}
%   \item
%     Fix: Counter |tabh@unique| must not be changed by \cs{include}.
%   \end{Version}
%   \begin{Version}{2007/04/11 v2.5}
%   \item
%     Line ends sanitized.
%   \end{Version}
%   \begin{Version}{2016/05/16 v2.6}
%   \item
%     Documentation updates.
%   \end{Version}
% \end{History}
%
% \PrintIndex
%
% \Finale
\endinput

%        (quote the arguments according to the demands of your shell)
%
% Documentation:
%    (a) If tabularht.drv is present:
%           latex tabularht.drv
%    (b) Without tabularht.drv:
%           latex tabularht.dtx; ...
%    The class ltxdoc loads the configuration file ltxdoc.cfg
%    if available. Here you can specify further options, e.g.
%    use A4 as paper format:
%       \PassOptionsToClass{a4paper}{article}
%
%    Programm calls to get the documentation (example):
%       pdflatex tabularht.dtx
%       makeindex -s gind.ist tabularht.idx
%       pdflatex tabularht.dtx
%       makeindex -s gind.ist tabularht.idx
%       pdflatex tabularht.dtx
%
% Installation:
%    TDS:tex/latex/oberdiek/tabularht.sty
%    TDS:doc/latex/oberdiek/tabularht.pdf
%    TDS:doc/latex/oberdiek/tabularht-example1.tex
%    TDS:doc/latex/oberdiek/tabularht-example2.tex
%    TDS:source/latex/oberdiek/tabularht.dtx
%
%<*ignore>
\begingroup
  \catcode123=1 %
  \catcode125=2 %
  \def\x{LaTeX2e}%
\expandafter\endgroup
\ifcase 0\ifx\install y1\fi\expandafter
         \ifx\csname processbatchFile\endcsname\relax\else1\fi
         \ifx\fmtname\x\else 1\fi\relax
\else\csname fi\endcsname
%</ignore>
%<*install>
\input docstrip.tex
\Msg{************************************************************************}
\Msg{* Installation}
\Msg{* Package: tabularht 2016/05/16 v2.6 Tabular with height specified (HO)}
\Msg{************************************************************************}

\keepsilent
\askforoverwritefalse

\let\MetaPrefix\relax
\preamble

This is a generated file.

Project: tabularht
Version: 2016/05/16 v2.6

Copyright (C) 2005-2007 by
   Heiko Oberdiek <heiko.oberdiek at googlemail.com>

This work may be distributed and/or modified under the
conditions of the LaTeX Project Public License, either
version 1.3c of this license or (at your option) any later
version. This version of this license is in
   https://www.latex-project.org/lppl/lppl-1-3c.txt
and the latest version of this license is in
   https://www.latex-project.org/lppl.txt
and version 1.3 or later is part of all distributions of
LaTeX version 2005/12/01 or later.

This work has the LPPL maintenance status "maintained".

The Current Maintainers of this work are
Heiko Oberdiek and the Oberdiek Package Support Group
https://github.com/ho-tex/oberdiek/issues


This work consists of the main source file tabularht.dtx
and the derived files
   tabularht.sty, tabularht.pdf, tabularht.ins, tabularht.drv,
   tabularht-example1.tex, tabularht-example2.tex.

\endpreamble
\let\MetaPrefix\DoubleperCent

\generate{%
  \file{tabularht.ins}{\from{tabularht.dtx}{install}}%
  \file{tabularht.drv}{\from{tabularht.dtx}{driver}}%
  \usedir{tex/latex/oberdiek}%
  \file{tabularht.sty}{\from{tabularht.dtx}{package}}%
  \usedir{doc/latex/oberdiek}%
  \file{tabularht-example1.tex}{\from{tabularht.dtx}{example1}}%
  \file{tabularht-example2.tex}{\from{tabularht.dtx}{example2}}%
  \nopreamble
  \nopostamble
%  \usedir{source/latex/oberdiek/catalogue}%
%  \file{tabularht.xml}{\from{tabularht.dtx}{catalogue}}%
}

\catcode32=13\relax% active space
\let =\space%
\Msg{************************************************************************}
\Msg{*}
\Msg{* To finish the installation you have to move the following}
\Msg{* file into a directory searched by TeX:}
\Msg{*}
\Msg{*     tabularht.sty}
\Msg{*}
\Msg{* To produce the documentation run the file `tabularht.drv'}
\Msg{* through LaTeX.}
\Msg{*}
\Msg{* Happy TeXing!}
\Msg{*}
\Msg{************************************************************************}

\endbatchfile
%</install>
%<*ignore>
\fi
%</ignore>
%<*driver>
\NeedsTeXFormat{LaTeX2e}
\ProvidesFile{tabularht.drv}%
  [2016/05/16 v2.6 Tabular with height specified (HO)]%
\documentclass{ltxdoc}
\usepackage{holtxdoc}[2011/11/22]
\begin{document}
  \DocInput{tabularht.dtx}%
\end{document}
%</driver>
% \fi
%
%
% \CharacterTable
%  {Upper-case    \A\B\C\D\E\F\G\H\I\J\K\L\M\N\O\P\Q\R\S\T\U\V\W\X\Y\Z
%   Lower-case    \a\b\c\d\e\f\g\h\i\j\k\l\m\n\o\p\q\r\s\t\u\v\w\x\y\z
%   Digits        \0\1\2\3\4\5\6\7\8\9
%   Exclamation   \!     Double quote  \"     Hash (number) \#
%   Dollar        \$     Percent       \%     Ampersand     \&
%   Acute accent  \'     Left paren    \(     Right paren   \)
%   Asterisk      \*     Plus          \+     Comma         \,
%   Minus         \-     Point         \.     Solidus       \/
%   Colon         \:     Semicolon     \;     Less than     \<
%   Equals        \=     Greater than  \>     Question mark \?
%   Commercial at \@     Left bracket  \[     Backslash     \\
%   Right bracket \]     Circumflex    \^     Underscore    \_
%   Grave accent  \`     Left brace    \{     Vertical bar  \|
%   Right brace   \}     Tilde         \~}
%
% \GetFileInfo{tabularht.drv}
%
% \title{The \xpackage{tabularht} package}
% \date{2016/05/16 v2.6}
% \author{Heiko Oberdiek\thanks
% {Please report any issues at \url{https://github.com/ho-tex/oberdiek/issues}}}
%
% \maketitle
%
% \begin{abstract}
% This package defines some environments that adds
% a height specification to tabular and array.
% \end{abstract}
%
% \tableofcontents
%
% \section{Usage}
%
% \begin{quote}
%   |\usepackage{tabularht}|
% \end{quote}
% The package provides the following environments
% that extend the tabular/array environment by
% a height specification as first argument:
% \begin{itemize}
% \item |tabularht|, |tabularht*|
% \item |arrayht|
% \item |tabularhtx|
%       (if package \xpackage{tabularx} is loaded)
% \end{itemize}
% The height argument allows a length specification,
% package \xpackage{calc} is supported if used. This means,
% the tabular will have the specified height.
% You can also use the prefixes |to=| and |spread=|.
% |to=| is the default, |spread=| means, the natural
% height of the tabular box is changed by the length
% after |spread=|.
%
% Examples:
% \begin{quote}
% \begin{tabular}{@{}l@{ $\rightarrow$ }l@{}}
%   |\begin{tabularht}{1in}|
%   & height is 1in\\
%   |\begin{tabularht}{to=1in}|
%   & height is 1in\\
%   |\begin{tabularht}{spread=0pt}|
%   & natural height, same as |\begin{tabular}|\\
%   |\begin{tabularht}{spread=1in}|
%   & natural height increased by 1in
% \end{tabular}
% \end{quote}
%
% Hint:  See also package \xpackage{tabularkv}, it provides an interface,
%        where most parameters for the environments can be given
%        by key-value pairs.
%
% \begin{declcs}{interrowspace}\,|{|\dots|}|
% \end{declcs}
% Adds space between table rows. It is essentially
% the same as |\noalign{\vspace{...}}|.
% \begin{declcs}{interrowfill}
% \end{declcs}
% Short for |\interrowspace{\fill}|
% \begin{declcs}{interrowstart}\,\dots\,\cs{interrowstop}
% \end{declcs}
% Marker commands, useful for option \xoption{vlines}.
%
% \subsection{Option \xoption{vlines}}
%
% Warning: This stuff is experimental.
%
% \vspace{2ex}
% \noindent
% Vertical lines are interrupted, if space is inserted
% in \cs{noalign}, \cs{interrowspace}, \cs{addlinespace}
% (\xpackage{booktabs}),
% between double \cs{hline}s. This option tries to detect
% and add the vertical lines. The lines in a tabular
% with \xpackage{tabularht} support (environments of this package)
% are numbered from left to right. The gap that is
% controlled by \cs{interrowspace} or inbetween \cs{interrowstart}
% and \cs{interrowstop} is then filled with the detected
% vertical lines.
%
% If only a limited selection of the lines should be
% drawn, the commands know an optional argument with a
% list of line numbers, e.g.
% \begin{quote}
%\begin{verbatim}
%\begin{tabularht}{50mm}{|l|l|}
%  Hello & World\\
%  \interrowfill[1,3]
%  Foo & Bar
%\end{tabularht}
%\end{verbatim}
% \end{quote}
% There are three lines, but the middle line is not drawn
% in the gap between the first and second row.
% Zero can be used to suppress all lines:
% \begin{quote}
% |\interrowspace[0]{10mm}|
% \end{quote}
% The syntax of the commands with the optional argument
% with the line number list \meta{list}. \meta{list} is a comma
% separated list of numbers, \meta{height} means the
% height specification described above with the optional
% prefixes |to=| or |spread=|.
% \begin{quote}
%   \cs{interrowspace}\,|[|\meta{list}|]|\,|{|\meta{height}|}|\\
%   \cs{interrowfill}\,|[|\meta{list}|]|\\
%   \cs{interrowstart}\,|[|\meta{list}|]|\,\dots\,\cs{interrowstop}
% \end{quote}
%
% Option \xoption{vlines} is driver dependent and uses \eTeX\ features.
% \begin{description}
% \item[\xoption{pdftex}:] \pdfTeX\ in PDF mode.
%   Here the positions of the lines are written with the
%   help of the \cs{pdfsavepos} feature into the \xfile{.aux} file(s).
%   Therefore you need two LaTeX runs to get the lines.
% \item[\xoption{dvips}:]
%   Here, PostScript's currentpoint it used to get
%   the line positions. The lines are then drawn at the end
%   of the page. Thus one \LaTeX/dvips run is sufficient
%   for this option.
% \item[Other drivers:]\mbox{}
%   \begin{description}
%   \item[PostScript drivers:] probably possible, an end of page
%     hook would be nice.
%   \item[\VTeX:] with GeX (PostScript interpreter) probably possible.
%   \item[dvipdfm:] no idea. The big problem is, how to get the
%     current position?
%   \end{description}
% \end{description}
%
% \subsection{Limitations}
%
% \begin{itemize}
% \item
%   Vertical lines are interrupted by |\noalign{\vfill}|.
% \end{itemize}
%
% \subsection{Compatibility}
%
% \begin{itemize}
% \item
%   \xpackage{array}, \xpackage{delarray}, \xpackage{tabularx}
%   are supported.
% \item There can be problems with packages that
%   redefine \cs{@array} (or \cs{@@array}, \cs{@tabarray}) and
%   \cs{@arrayrule} (for option \xoption{vlines}).
% \item \xpackage{colortbl}: it should at least work, but there isn't
%   support for filling the gaps with color, neither
%   the rules nor the backgrounds.
% \end{itemize}
%
% \subsection{Examples}
%
% \subsubsection{Example 1}
%
%    \begin{macrocode}
%<*example1>
\documentclass{article}
\usepackage{tabularht}

\begin{document}
\fbox{%
  \begin{tabularht*}{1in}{4in}{@{}l@{\extracolsep{\fill}}r@{}}%
    upper left corner & upper right corner\\%
    \noalign{\vfill}%
    \multicolumn{2}{@{}c@{}}{bounding box}\\%
    \noalign{\vfill}%
    lower left corner & lower right corner\\%
  \end{tabularht*}%
}
\end{document}
%</example1>
%    \end{macrocode}
%
% \subsubsection{Example 2}
%
%    \begin{macrocode}
%<*example2>
\documentclass{article}
\usepackage{booktabs}
\usepackage[dvips,vlines]{tabularht}

\begin{document}

\begin{tabularht}{spread=0pt}{|l|l|}
  \hline
  First&Line\\%
  \hline
\interrowstart
  \addlinespace[10mm]%
\interrowstop
  \hline
  Second&Line\\%
\interrowstart
  \hline
  \hline
\interrowstop
  Third&Line\\%
  \hline
\interrowspace{10mm}
  \hline
  Fourth&Line\\%
  \hline
\end{tabularht}

\end{document}
%</example2>
%    \end{macrocode}
%
% \StopEventually{
% }
%
% \section{Implementation}
%
%    \begin{macrocode}
%<*package>
%    \end{macrocode}
%    Package identification.
%    \begin{macrocode}
\NeedsTeXFormat{LaTeX2e}
\ProvidesPackage{tabularht}%
  [2016/05/16 v2.6 Tabular with height specified (HO)]
%    \end{macrocode}
%
% \subsection{Environments}
%
%    \begin{macrocode}
\let\@toarrayheight\@empty
\let\tabH@array@init\@empty

\toks@={%
  \begingroup
    \long\def\x#1\vcenter\fi\fi\bgroup#2\@sharp#3#4\@nil{%
      \endgroup
      \gdef\@array[##1]##2{%
        \tabH@array@init
        #1%
        \vcenter\fi\fi
        \@toarrayheight
        \bgroup
        \let\@toarrayheight\@empty
        #2\@sharp###3#4%
      }%
    }%
  \expandafter\x\@array[#1]{#2}\@nil % hash-ok
}
\edef\tabH@patch@array{\the\toks@}
\def\tabH@patch@@array{%
  \ifx\@array\@@array
    \def\reserved@a{\let\@@array\@array}%
  \else
    \let\reserved@a\relax
  \fi
  \tabH@patch@array
  \reserved@a
}
\tabH@patch@@array

\@ifpackageloaded{array}{}{%
  \AtBeginDocument{%
    \@ifpackageloaded{array}{%
      \tabH@patch@@array
    }{}%
  }%
}

\def\tabH@setheight#1{%
  \tabH@@setheight#1==\@nil
}
\def\tabH@@setheight#1=#2=#3\@nil{%
  \ifx\\#2#3\\%
    \setlength{\dimen@}{#1}%
    \edef\@toarrayheight{to\the\dimen@}%
  \else
    \edef\tabH@temp{\zap@space#1 \@empty}%
    \ifx\tabH@temp\tabH@to
    \else
      \ifx\tabH@temp\tabH@spread
      \else
        \PackageError{tabularht}{%
          Unknown height specifier %
          `\expandafter\strip@prefix\meaning\tabH@temp'%
        }{%
          The height dimension for tabular height can be prefixed%
          \MessageBreak
          with `to=' or `spread=', default is `to='.%
        }%
        \let\tabH@temp\tabH@to
      \fi
    \fi
    \setlength{\dimen@}{#2}%
    \edef\@toarrayheight{\tabH@temp\the\dimen@}%
  \fi
}
\def\tabH@to{to}
\def\tabH@spread{spread}
%    \end{macrocode}
%    First argument is the height of the table,
%    then the orignal arguments for tabular follow.
%    \begin{macrocode}
\newenvironment{tabularht}[1]{%
  \tabH@setheight{#1}%
  \tabular
}{%
  \endtabular
}

\newenvironment{tabularht*}[1]{%
  \tabH@setheight{#1}%
  \@nameuse{tabular*}%
}{%
  \@nameuse{endtabular*}%
}

\newenvironment{tabularhtx}[1]{%
  \tabH@setheight{#1}%
  \tabularx
}{%
  \endtabularx
}

\newenvironment{arrayht}[1]{%
  \tabH@setheight{#1}%
  \array
}{%
  \endarray
}

\def\interrowspace{%
  \noalign\bgroup
    \tabH@interrowspace
}
\newcommand*{\tabH@interrowspace}[2][]{%
    \tabH@vspace{#1}{#2}%
  \egroup
}
\def\interrowfill{%
  \noalign\bgroup
    \tabH@interrowfill
}
\newcommand*{\tabH@interrowfill}[1][]{%
    \tabH@vspace{#1}{\fill}%
  \egroup
}
\def\tabH@vspace#1#2{%
  \tabH@vspace@start{#1}%
  \vspace{#2}%
  \tabH@vspace@stop
}
\let\tabH@vspace@start\@gobble
\let\tabH@vspace@stop\@empty

\newcommand*{\interrowstart}{%
  \noalign\bgroup
    \tabH@interrowstart
}
\newcommand*{\tabH@interrowstart}[1][]{%
    \tabH@vspace@start{#1}%
  \egroup
}
\newcommand*{\interrowstop}{%
  \noalign{\tabH@vspace@stop}%
}
%    \end{macrocode}
%
% \subsection{Options}
%
%    \begin{macrocode}
\providecommand*{\tabH@driver}{}

\DeclareOption{vlines}{%
  \let\tabH@temp\relax
}
\DeclareOption{pdftex}{}
\DeclareOption{dvips}{%
  \def\tabH@driver{dvips}%
}
\ProcessOptions*\relax

\ifx\tabH@temp\relax
\else
  \expandafter\endinput
\fi

\begingroup
  \@ifundefined{eTeXversion}{%
    \PackageError{tabularht}{%
      Option `vlines' requires eTeX%
    }{%
      Use of eTeX is recommended for LaTeX, see ltnews16.%
    }%
    \endgroup
    \endinput
  }{}%
\endgroup
%    \end{macrocode}
%
% \subsection{Option vlines, driver independent stuff}
%
%    \begin{macrocode}
\begingroup
  \let\@addtoreset\@gobbletwo
  \newcounter{tabH@unique}%
\endgroup
\let\tabH@currenttab\@empty

\def\tabH@array@init{%
  \ifx\@toarrayheight\@empty
    % ignore vertical lines of nested tabular environments
    \let\tabH@currenttab\@empty
  \else
    \stepcounter{tabH@unique}%
    \edef\tabH@currenttab{\the\c@tabH@unique}%
  \fi
}

\renewcommand*{\@arrayrule}{%
  \@addtopreamble{%
    \hskip -.5\arrayrulewidth
    \ifx\tabH@currenttab\@empty
    \else
      \tabH@vrule{\tabH@currenttab}%
    \fi
    \begingroup
      \expandafter\ifx\csname CT@arc@\endcsname\relax
      \else
        \expandafter\CT@arc@
      \fi
      \vline
    \endgroup
    \hskip -.5\arrayrulewidth
  }%
}
\let\tabH@arrayrule\@arrayrule
\AtBeginDocument{%
  \@ifpackageloaded{colortbl}{%
    \let\@arrayrule\tabH@arrayrule
  }{}%
}

\let\tabH@vrule\@gobble
%    \end{macrocode}
%
% \subsection{Driver pdftex}
%
%    \begin{macrocode}
\RequirePackage{ifpdf}
\ifpdf
  \begingroup
    \@ifundefined{pdfsavepos}{%
      \PackageError{tabularht}{%
        Your pdfTeX is too old%
      }{%
        \string\pdfsavepos\space is missing.%
      }%
      \endgroup
      \csname fi\endcsname
      \endinput
    }{}%

    \let\on@line\@empty
    \PackageInfo{tabularht}{%
      Using driver `pdftex' because of pdfTeX in PDF mode%
    }%
  \endgroup

  \protected\def\tabH@vrule#1{%
    \if@filesw
      \pdfsavepos
      \protected@write\@auxout{%
        \let\tabH@lastxpos\relax
      }{%
        \tabH@aux@vrule{#1}{\tabH@lastxpos}%
      }%
    \fi
  }%

  \def\tabH@lastxpos{\the\pdflastxpos}%
  \def\tabH@lastypos{\the\pdflastypos}%

  % The .aux file contains three commands:
  % \tabH@aux@vrule{tabular id}{x position}
  % \tabH@aux@vstart{tabular id}{row id}{x position}{y position}
  % \tabH@aux@vstop{y position}
  %
  \AtBeginDocument{%
    % The .aux files are read the first time before
    % \AtBeginDocument and later at \end{document}.
    % \tabH@aux@done is a marker to distinguish
    % between these two readings. Only in the first
    % case we need the \tabH@aux@... commands.
    \let\tabH@aux@done\@empty
    \if@filesw
      \immediate\write\@mainaux{%
        \@percentchar\@percentchar BeginProlog: tabularht%
      }%
      % items in the aux file are executed,
      % if tabularht is loaded
      % and during the aux file read at \begin{document} only
      \immediate\write\@mainaux{%
        \detokenize{%
          % the \tabH@aux@... commands are needed only if
          % tabularht is loaded with driver pdftex.
          \@ifundefined{tabH@aux@vrule}\@secondoftwo\@firstofone
          {%
            % disable commands except for the first .aux files reading
            \@ifundefined{tabH@aux@done}\@gobble\@firstofone
          }%
          {%
            \let\tabH@aux@vrule\@gobbletwo
            \let\tabH@aux@vstart\@gobblefour
            \let\tabH@aux@vstop\@gobble
          }%
        }%
      }%
      \immediate\write\@mainaux{%
        \@percentchar\@percentchar EndProlog: tabularht%
      }%
    \fi
  }%

  % the x positions of vrules are stored in
  % \tabH@<tabcount>list with distinct values
  \protected\def\tabH@aux@vrule#1#2{%
    \@ifundefined{tabH@#1list}{%
      \expandafter\xdef\csname tabH@#1list\endcsname{%
        \noexpand\do{#2}%
      }%
    }{%
      \begingroup
        \def\x{#2}%
        \let\y\@undefined
        \let\do\tabH@do@add
        \expandafter\xdef\csname tabH@#1list\endcsname{%
          \csname tabH@#1list\endcsname\@empty
          \ifx\y\@undefined
            \noexpand\do{\x}%
          \fi
        }%
      \endgroup
    }%
  }%
  \def\tabH@do@add#1{%
    \ifx\y\@undefined
      \ifnum#1<\x\space
      \else
        \expandafter\ifx\csname y\endcsname\relax\fi
        \ifnum#1>\x\space
          \noexpand\do{\x}%
        \fi
      \fi
    \fi
    \noexpand\do{#1}%
  }%

  \def\tabH@vspace@start#1{%
    \if@filesw
      \stepcounter{tabH@unique}%
      \edef\tabH@currentrow{\the\c@tabH@unique}%
      \pdfsavepos
      \protected@write\@auxout{%
        \let\tabH@lastxpos\relax
        \let\tabH@lastypos\relax
      }{%
        \tabH@aux@vstart{\tabH@currenttab}{\tabH@currentrow}%
                        {\tabH@lastxpos}{\tabH@lastypos}%
      }%
    \fi
    \begingroup
      \edef\a{tabH@\tabH@currenttab row\tabH@currentrow}%
      \expandafter\let\expandafter\x\csname\a x\endcsname
      \ifx\x\relax
      \else
        \expandafter\let\expandafter\y\csname\a y\endcsname
        \expandafter\let\expandafter\l
            \csname tabH@\tabH@currenttab list\endcsname
        \ifx\l\relax
        \else
          \def\f{#1}%
          \ifx\f\@empty
            \let\do\tabH@do@set
          \else
            \count@=\z@
            \let\do\tabH@do@filter
          \fi
          \setbox\z@=\hbox{\l}%
          \wd\z@=\z@
          \dp\z@=\z@
          \copy\z@
        \fi
      \fi
    \endgroup
  }%
  \def\tabH@vspace@stop{%
    \if@filesw
      \pdfsavepos
      \protected@write\@auxout{%
        \let\tabH@lastypos\relax
      }{%
        \tabH@aux@vstop{\tabH@lastypos}%
      }%
    \fi
  }%
  \def\tabH@do@set#1{%
    \hbox to \z@{%
      \hskip \dimexpr #1sp - \x sp\relax
      \vrule \@width\arrayrulewidth
             \@depth\dimexpr \y sp\relax
      \hss
    }%
  }%
  \def\tabH@do@filter{%
    \@tempswafalse
    \advance\count@\@ne
    \@for\e:=\f\do{%
      \ifnum\e=\count@
        \@tempswatrue
      \fi
    }%
    \if@tempswa
      \expandafter\tabH@do@set
    \else
      \expandafter\@gobble
    \fi
  }%

  \protected\def\tabH@aux@vstart#1#2#3#4{%
    \def\tabH@current@vstart{{#1}{#2}{#3}{#4}}%
  }%
  \protected\def\tabH@aux@vstop{%
    \expandafter\tabH@aux@v\tabH@current@vstart
  }%
  \def\tabH@aux@v#1#2#3#4#5{%
    \expandafter\gdef\csname tabH@#1row#2x\endcsname{#3}%
    \expandafter\xdef\csname tabH@#1row#2y\endcsname{%
      \the\numexpr #4 - #5\relax
    }%
  }%

  \csname fi\endcsname
  \endinput

\fi
%    \end{macrocode}
%
% \subsection{DVI drivers}
%
%    \begin{macrocode}
\ifx\tabH@driver\@empty
  \PackageError{tabularht}{%
    Missing DVI driver, option `vlines' disabled%
  }{%
    Supported DVI drivers: dvips.%
  }%
  \expandafter\endinput
\fi

\def\tabH@driver@dvips{%
  \def\tabH@literalps##1{\special{ps:SDict begin ##1 end}}%
  \def\tabH@headerps##1{\special{! ##1}}%
}

\@onelevel@sanitize\tabH@driver
\@ifundefined{tabH@driver@\tabH@driver}{%
  \PackageError{tabularht}{%
    Unsupported driver `\tabH@driver'%
  }{%
    Supported DVI drivers: dvips.%
  }%
  \endinput
}{}

\begingroup
  \let\on@line\@empty
  \PackageInfo{tabularht}{%
    Using driver `\tabH@driver'%
  }%
\endgroup
\csname tabH@driver@\tabH@driver\endcsname

\protected\def\tabH@vrule#1#2\vrule#3\arrayrulewidth{%
  #2% \fi or empty
  % hack to get rid of maxdrift rounding of dvips,
  % thus simulate a large motion
  \kern1in\relax
  \tabH@literalps{%
    #1 tabH.vrule %
    Resolution neg 0 translate%
  }%
  \vrule#3\arrayrulewidth
  \tabH@literalps{Resolution 0 translate}%
  \kern-1in\relax
}

\def\tabH@vspace@start#1{%
  \begingroup
    \let\y\@empty
    \@for\x:=#1\do{%
      \ifx\y\@empty
        \edef\y{\x}%
      \else
        \edef\y{\y\space\x}%
      \fi
    }%
    \tabH@literalps{\tabH@currenttab[\y]currentpoint exch pop}%
  \endgroup
}
\def\tabH@vspace@stop{%
  \tabH@literalps{%
    currentpoint exch pop %
    \number\dimexpr\arrayrulewidth\relax\space
    tabH.vspace%
  }%
}

\tabH@headerps{%
  userdict begin%
    /tabH.list 10 dict def%
    /tabH.job [] def %
  end%
  /tabH.vrule{%
    10 string cvs cvn dup tabH.list exch known{%
      tabH.list exch dup [ exch tabH.list exch get %
      currentpoint pop round exch true exch{%
        % tabH.list key [ ... x true i
        % tabH.list key [ ... false i
        exch{%
          % ... [ ... x i
          2 copy lt{false}{%
            2 copy eq{pop false}{exch true}ifelse%
          }ifelse%
        }{false}ifelse%
      }forall %
      pop%
      ]put%
    }{%
      tabH.list exch[currentpoint pop round]put%
    }ifelse%
  }bind def%
  % <tab num> <cols array> <ytop> <ybottom> <rulewidth[sp]>
  /tabH.vspace{%
    userdict begin %
      10 dict dup begin %
        exch 65536 div Resolution mul 72.27 div %
        % dvips uses a poor man's ceil function
        % see dopage.c before "drawrule": (int)(... + 0.9999999)
        0.9999999 add truncate%
        /rulewidth exch def %
        exch/ybottom exch def %
        exch/ytop exch def %
        exch/cols exch def %
        exch/tabkey exch 10 string cvs cvn def %
      end%
      /tabH.job exch[exch userdict/tabH.job get aload pop]def %
    end%
  }bind def %
  % Now we do the work at the end of the page.
  % Unhappily "eop-hook" cannot be used, because "eop"
  % executes "restore" before, so that all data are lost.
  TeXDict begin%
    /eop%
    [%
      {%
        tabH.job{%
          begin%
           /colarray %
             tabH.list tabkey known{tabH.list tabkey get}{[]}ifelse %
           def %
           cols length 0 eq not{%
             /colarray[%
               cols{1 sub %
                 dup 0 lt{pop}{%
                   dup colarray length ge{pop}{%
                     colarray exch get%
                   }ifelse%
                 }ifelse%
               }forall%
             ]def%
           }if %
           colarray{%
             % (rulewidth) == rulewidth == % debug
             Resolution sub %
             ytop rulewidth ytop ybottom sub v%
           }forall %
          end%
        }forall%
        % tabH.list{== ==}forall % debug
      }bind aload pop %
      TeXDict /eop get aload pop%
    ]cvx def %
  end%
}
%    \end{macrocode}
%
%    \begin{macrocode}
%</package>
%    \end{macrocode}
%
% \section{Installation}
%
% \subsection{Download}
%
% \paragraph{Package.} This package is available on
% CTAN\footnote{\CTANpkg{tabularht}}:
% \begin{description}
% \item[\CTAN{macros/latex/contrib/oberdiek/tabularht.dtx}] The source file.
% \item[\CTAN{macros/latex/contrib/oberdiek/tabularht.pdf}] Documentation.
% \end{description}
%
%
% \paragraph{Bundle.} All the packages of the bundle `oberdiek'
% are also available in a TDS compliant ZIP archive. There
% the packages are already unpacked and the documentation files
% are generated. The files and directories obey the TDS standard.
% \begin{description}
% \item[\CTANinstall{install/macros/latex/contrib/oberdiek.tds.zip}]
% \end{description}
% \emph{TDS} refers to the standard ``A Directory Structure
% for \TeX\ Files'' (\CTAN{tds/tds.pdf}). Directories
% with \xfile{texmf} in their name are usually organized this way.
%
% \subsection{Bundle installation}
%
% \paragraph{Unpacking.} Unpack the \xfile{oberdiek.tds.zip} in the
% TDS tree (also known as \xfile{texmf} tree) of your choice.
% Example (linux):
% \begin{quote}
%   |unzip oberdiek.tds.zip -d ~/texmf|
% \end{quote}
%
% \paragraph{Script installation.}
% Check the directory \xfile{TDS:scripts/oberdiek/} for
% scripts that need further installation steps.
% Package \xpackage{attachfile2} comes with the Perl script
% \xfile{pdfatfi.pl} that should be installed in such a way
% that it can be called as \texttt{pdfatfi}.
% Example (linux):
% \begin{quote}
%   |chmod +x scripts/oberdiek/pdfatfi.pl|\\
%   |cp scripts/oberdiek/pdfatfi.pl /usr/local/bin/|
% \end{quote}
%
% \subsection{Package installation}
%
% \paragraph{Unpacking.} The \xfile{.dtx} file is a self-extracting
% \docstrip\ archive. The files are extracted by running the
% \xfile{.dtx} through \plainTeX:
% \begin{quote}
%   \verb|tex tabularht.dtx|
% \end{quote}
%
% \paragraph{TDS.} Now the different files must be moved into
% the different directories in your installation TDS tree
% (also known as \xfile{texmf} tree):
% \begin{quote}
% \def\t{^^A
% \begin{tabular}{@{}>{\ttfamily}l@{ $\rightarrow$ }>{\ttfamily}l@{}}
%   tabularht.sty & tex/latex/oberdiek/tabularht.sty\\
%   tabularht.pdf & doc/latex/oberdiek/tabularht.pdf\\
%   tabularht-example1.tex & doc/latex/oberdiek/tabularht-example1.tex\\
%   tabularht-example2.tex & doc/latex/oberdiek/tabularht-example2.tex\\
%   tabularht.dtx & source/latex/oberdiek/tabularht.dtx\\
% \end{tabular}^^A
% }^^A
% \sbox0{\t}^^A
% \ifdim\wd0>\linewidth
%   \begingroup
%     \advance\linewidth by\leftmargin
%     \advance\linewidth by\rightmargin
%   \edef\x{\endgroup
%     \def\noexpand\lw{\the\linewidth}^^A
%   }\x
%   \def\lwbox{^^A
%     \leavevmode
%     \hbox to \linewidth{^^A
%       \kern-\leftmargin\relax
%       \hss
%       \usebox0
%       \hss
%       \kern-\rightmargin\relax
%     }^^A
%   }^^A
%   \ifdim\wd0>\lw
%     \sbox0{\small\t}^^A
%     \ifdim\wd0>\linewidth
%       \ifdim\wd0>\lw
%         \sbox0{\footnotesize\t}^^A
%         \ifdim\wd0>\linewidth
%           \ifdim\wd0>\lw
%             \sbox0{\scriptsize\t}^^A
%             \ifdim\wd0>\linewidth
%               \ifdim\wd0>\lw
%                 \sbox0{\tiny\t}^^A
%                 \ifdim\wd0>\linewidth
%                   \lwbox
%                 \else
%                   \usebox0
%                 \fi
%               \else
%                 \lwbox
%               \fi
%             \else
%               \usebox0
%             \fi
%           \else
%             \lwbox
%           \fi
%         \else
%           \usebox0
%         \fi
%       \else
%         \lwbox
%       \fi
%     \else
%       \usebox0
%     \fi
%   \else
%     \lwbox
%   \fi
% \else
%   \usebox0
% \fi
% \end{quote}
% If you have a \xfile{docstrip.cfg} that configures and enables \docstrip's
% TDS installing feature, then some files can already be in the right
% place, see the documentation of \docstrip.
%
% \subsection{Refresh file name databases}
%
% If your \TeX~distribution
% (\teTeX, \mikTeX, \dots) relies on file name databases, you must refresh
% these. For example, \teTeX\ users run \verb|texhash| or
% \verb|mktexlsr|.
%
% \subsection{Some details for the interested}
%
% \paragraph{Attached source.}
%
% The PDF documentation on CTAN also includes the
% \xfile{.dtx} source file. It can be extracted by
% AcrobatReader 6 or higher. Another option is \textsf{pdftk},
% e.g. unpack the file into the current directory:
% \begin{quote}
%   \verb|pdftk tabularht.pdf unpack_files output .|
% \end{quote}
%
% \paragraph{Unpacking with \LaTeX.}
% The \xfile{.dtx} chooses its action depending on the format:
% \begin{description}
% \item[\plainTeX:] Run \docstrip\ and extract the files.
% \item[\LaTeX:] Generate the documentation.
% \end{description}
% If you insist on using \LaTeX\ for \docstrip\ (really,
% \docstrip\ does not need \LaTeX), then inform the autodetect routine
% about your intention:
% \begin{quote}
%   \verb|latex \let\install=y% \iffalse meta-comment
%
% File: tabularht.dtx
% Version: 2016/05/16 v2.6
% Info: Tabular with height specified
%
% Copyright (C) 2005-2007 by
%    Heiko Oberdiek <heiko.oberdiek at googlemail.com>
%    2016
%    https://github.com/ho-tex/oberdiek/issues
%
% This work may be distributed and/or modified under the
% conditions of the LaTeX Project Public License, either
% version 1.3c of this license or (at your option) any later
% version. This version of this license is in
%    https://www.latex-project.org/lppl/lppl-1-3c.txt
% and the latest version of this license is in
%    https://www.latex-project.org/lppl.txt
% and version 1.3 or later is part of all distributions of
% LaTeX version 2005/12/01 or later.
%
% This work has the LPPL maintenance status "maintained".
%
% The Current Maintainers of this work are
% Heiko Oberdiek and the Oberdiek Package Support Group
% https://github.com/ho-tex/oberdiek/issues
%
% This work consists of the main source file tabularht.dtx
% and the derived files
%    tabularht.sty, tabularht.pdf, tabularht.ins, tabularht.drv,
%    tabularht-example1.tex, tabularht-example2.tex.
%
% Distribution:
%    CTAN:macros/latex/contrib/oberdiek/tabularht.dtx
%    CTAN:macros/latex/contrib/oberdiek/tabularht.pdf
%
% Unpacking:
%    (a) If tabularht.ins is present:
%           tex tabularht.ins
%    (b) Without tabularht.ins:
%           tex tabularht.dtx
%    (c) If you insist on using LaTeX
%           latex \let\install=y% \iffalse meta-comment
%
% File: tabularht.dtx
% Version: 2016/05/16 v2.6
% Info: Tabular with height specified
%
% Copyright (C) 2005-2007 by
%    Heiko Oberdiek <heiko.oberdiek at googlemail.com>
%    2016
%    https://github.com/ho-tex/oberdiek/issues
%
% This work may be distributed and/or modified under the
% conditions of the LaTeX Project Public License, either
% version 1.3c of this license or (at your option) any later
% version. This version of this license is in
%    https://www.latex-project.org/lppl/lppl-1-3c.txt
% and the latest version of this license is in
%    https://www.latex-project.org/lppl.txt
% and version 1.3 or later is part of all distributions of
% LaTeX version 2005/12/01 or later.
%
% This work has the LPPL maintenance status "maintained".
%
% The Current Maintainers of this work are
% Heiko Oberdiek and the Oberdiek Package Support Group
% https://github.com/ho-tex/oberdiek/issues
%
% This work consists of the main source file tabularht.dtx
% and the derived files
%    tabularht.sty, tabularht.pdf, tabularht.ins, tabularht.drv,
%    tabularht-example1.tex, tabularht-example2.tex.
%
% Distribution:
%    CTAN:macros/latex/contrib/oberdiek/tabularht.dtx
%    CTAN:macros/latex/contrib/oberdiek/tabularht.pdf
%
% Unpacking:
%    (a) If tabularht.ins is present:
%           tex tabularht.ins
%    (b) Without tabularht.ins:
%           tex tabularht.dtx
%    (c) If you insist on using LaTeX
%           latex \let\install=y% \iffalse meta-comment
%
% File: tabularht.dtx
% Version: 2016/05/16 v2.6
% Info: Tabular with height specified
%
% Copyright (C) 2005-2007 by
%    Heiko Oberdiek <heiko.oberdiek at googlemail.com>
%    2016
%    https://github.com/ho-tex/oberdiek/issues
%
% This work may be distributed and/or modified under the
% conditions of the LaTeX Project Public License, either
% version 1.3c of this license or (at your option) any later
% version. This version of this license is in
%    https://www.latex-project.org/lppl/lppl-1-3c.txt
% and the latest version of this license is in
%    https://www.latex-project.org/lppl.txt
% and version 1.3 or later is part of all distributions of
% LaTeX version 2005/12/01 or later.
%
% This work has the LPPL maintenance status "maintained".
%
% The Current Maintainers of this work are
% Heiko Oberdiek and the Oberdiek Package Support Group
% https://github.com/ho-tex/oberdiek/issues
%
% This work consists of the main source file tabularht.dtx
% and the derived files
%    tabularht.sty, tabularht.pdf, tabularht.ins, tabularht.drv,
%    tabularht-example1.tex, tabularht-example2.tex.
%
% Distribution:
%    CTAN:macros/latex/contrib/oberdiek/tabularht.dtx
%    CTAN:macros/latex/contrib/oberdiek/tabularht.pdf
%
% Unpacking:
%    (a) If tabularht.ins is present:
%           tex tabularht.ins
%    (b) Without tabularht.ins:
%           tex tabularht.dtx
%    (c) If you insist on using LaTeX
%           latex \let\install=y\input{tabularht.dtx}
%        (quote the arguments according to the demands of your shell)
%
% Documentation:
%    (a) If tabularht.drv is present:
%           latex tabularht.drv
%    (b) Without tabularht.drv:
%           latex tabularht.dtx; ...
%    The class ltxdoc loads the configuration file ltxdoc.cfg
%    if available. Here you can specify further options, e.g.
%    use A4 as paper format:
%       \PassOptionsToClass{a4paper}{article}
%
%    Programm calls to get the documentation (example):
%       pdflatex tabularht.dtx
%       makeindex -s gind.ist tabularht.idx
%       pdflatex tabularht.dtx
%       makeindex -s gind.ist tabularht.idx
%       pdflatex tabularht.dtx
%
% Installation:
%    TDS:tex/latex/oberdiek/tabularht.sty
%    TDS:doc/latex/oberdiek/tabularht.pdf
%    TDS:doc/latex/oberdiek/tabularht-example1.tex
%    TDS:doc/latex/oberdiek/tabularht-example2.tex
%    TDS:source/latex/oberdiek/tabularht.dtx
%
%<*ignore>
\begingroup
  \catcode123=1 %
  \catcode125=2 %
  \def\x{LaTeX2e}%
\expandafter\endgroup
\ifcase 0\ifx\install y1\fi\expandafter
         \ifx\csname processbatchFile\endcsname\relax\else1\fi
         \ifx\fmtname\x\else 1\fi\relax
\else\csname fi\endcsname
%</ignore>
%<*install>
\input docstrip.tex
\Msg{************************************************************************}
\Msg{* Installation}
\Msg{* Package: tabularht 2016/05/16 v2.6 Tabular with height specified (HO)}
\Msg{************************************************************************}

\keepsilent
\askforoverwritefalse

\let\MetaPrefix\relax
\preamble

This is a generated file.

Project: tabularht
Version: 2016/05/16 v2.6

Copyright (C) 2005-2007 by
   Heiko Oberdiek <heiko.oberdiek at googlemail.com>

This work may be distributed and/or modified under the
conditions of the LaTeX Project Public License, either
version 1.3c of this license or (at your option) any later
version. This version of this license is in
   https://www.latex-project.org/lppl/lppl-1-3c.txt
and the latest version of this license is in
   https://www.latex-project.org/lppl.txt
and version 1.3 or later is part of all distributions of
LaTeX version 2005/12/01 or later.

This work has the LPPL maintenance status "maintained".

The Current Maintainers of this work are
Heiko Oberdiek and the Oberdiek Package Support Group
https://github.com/ho-tex/oberdiek/issues


This work consists of the main source file tabularht.dtx
and the derived files
   tabularht.sty, tabularht.pdf, tabularht.ins, tabularht.drv,
   tabularht-example1.tex, tabularht-example2.tex.

\endpreamble
\let\MetaPrefix\DoubleperCent

\generate{%
  \file{tabularht.ins}{\from{tabularht.dtx}{install}}%
  \file{tabularht.drv}{\from{tabularht.dtx}{driver}}%
  \usedir{tex/latex/oberdiek}%
  \file{tabularht.sty}{\from{tabularht.dtx}{package}}%
  \usedir{doc/latex/oberdiek}%
  \file{tabularht-example1.tex}{\from{tabularht.dtx}{example1}}%
  \file{tabularht-example2.tex}{\from{tabularht.dtx}{example2}}%
  \nopreamble
  \nopostamble
%  \usedir{source/latex/oberdiek/catalogue}%
%  \file{tabularht.xml}{\from{tabularht.dtx}{catalogue}}%
}

\catcode32=13\relax% active space
\let =\space%
\Msg{************************************************************************}
\Msg{*}
\Msg{* To finish the installation you have to move the following}
\Msg{* file into a directory searched by TeX:}
\Msg{*}
\Msg{*     tabularht.sty}
\Msg{*}
\Msg{* To produce the documentation run the file `tabularht.drv'}
\Msg{* through LaTeX.}
\Msg{*}
\Msg{* Happy TeXing!}
\Msg{*}
\Msg{************************************************************************}

\endbatchfile
%</install>
%<*ignore>
\fi
%</ignore>
%<*driver>
\NeedsTeXFormat{LaTeX2e}
\ProvidesFile{tabularht.drv}%
  [2016/05/16 v2.6 Tabular with height specified (HO)]%
\documentclass{ltxdoc}
\usepackage{holtxdoc}[2011/11/22]
\begin{document}
  \DocInput{tabularht.dtx}%
\end{document}
%</driver>
% \fi
%
%
% \CharacterTable
%  {Upper-case    \A\B\C\D\E\F\G\H\I\J\K\L\M\N\O\P\Q\R\S\T\U\V\W\X\Y\Z
%   Lower-case    \a\b\c\d\e\f\g\h\i\j\k\l\m\n\o\p\q\r\s\t\u\v\w\x\y\z
%   Digits        \0\1\2\3\4\5\6\7\8\9
%   Exclamation   \!     Double quote  \"     Hash (number) \#
%   Dollar        \$     Percent       \%     Ampersand     \&
%   Acute accent  \'     Left paren    \(     Right paren   \)
%   Asterisk      \*     Plus          \+     Comma         \,
%   Minus         \-     Point         \.     Solidus       \/
%   Colon         \:     Semicolon     \;     Less than     \<
%   Equals        \=     Greater than  \>     Question mark \?
%   Commercial at \@     Left bracket  \[     Backslash     \\
%   Right bracket \]     Circumflex    \^     Underscore    \_
%   Grave accent  \`     Left brace    \{     Vertical bar  \|
%   Right brace   \}     Tilde         \~}
%
% \GetFileInfo{tabularht.drv}
%
% \title{The \xpackage{tabularht} package}
% \date{2016/05/16 v2.6}
% \author{Heiko Oberdiek\thanks
% {Please report any issues at \url{https://github.com/ho-tex/oberdiek/issues}}}
%
% \maketitle
%
% \begin{abstract}
% This package defines some environments that adds
% a height specification to tabular and array.
% \end{abstract}
%
% \tableofcontents
%
% \section{Usage}
%
% \begin{quote}
%   |\usepackage{tabularht}|
% \end{quote}
% The package provides the following environments
% that extend the tabular/array environment by
% a height specification as first argument:
% \begin{itemize}
% \item |tabularht|, |tabularht*|
% \item |arrayht|
% \item |tabularhtx|
%       (if package \xpackage{tabularx} is loaded)
% \end{itemize}
% The height argument allows a length specification,
% package \xpackage{calc} is supported if used. This means,
% the tabular will have the specified height.
% You can also use the prefixes |to=| and |spread=|.
% |to=| is the default, |spread=| means, the natural
% height of the tabular box is changed by the length
% after |spread=|.
%
% Examples:
% \begin{quote}
% \begin{tabular}{@{}l@{ $\rightarrow$ }l@{}}
%   |\begin{tabularht}{1in}|
%   & height is 1in\\
%   |\begin{tabularht}{to=1in}|
%   & height is 1in\\
%   |\begin{tabularht}{spread=0pt}|
%   & natural height, same as |\begin{tabular}|\\
%   |\begin{tabularht}{spread=1in}|
%   & natural height increased by 1in
% \end{tabular}
% \end{quote}
%
% Hint:  See also package \xpackage{tabularkv}, it provides an interface,
%        where most parameters for the environments can be given
%        by key-value pairs.
%
% \begin{declcs}{interrowspace}\,|{|\dots|}|
% \end{declcs}
% Adds space between table rows. It is essentially
% the same as |\noalign{\vspace{...}}|.
% \begin{declcs}{interrowfill}
% \end{declcs}
% Short for |\interrowspace{\fill}|
% \begin{declcs}{interrowstart}\,\dots\,\cs{interrowstop}
% \end{declcs}
% Marker commands, useful for option \xoption{vlines}.
%
% \subsection{Option \xoption{vlines}}
%
% Warning: This stuff is experimental.
%
% \vspace{2ex}
% \noindent
% Vertical lines are interrupted, if space is inserted
% in \cs{noalign}, \cs{interrowspace}, \cs{addlinespace}
% (\xpackage{booktabs}),
% between double \cs{hline}s. This option tries to detect
% and add the vertical lines. The lines in a tabular
% with \xpackage{tabularht} support (environments of this package)
% are numbered from left to right. The gap that is
% controlled by \cs{interrowspace} or inbetween \cs{interrowstart}
% and \cs{interrowstop} is then filled with the detected
% vertical lines.
%
% If only a limited selection of the lines should be
% drawn, the commands know an optional argument with a
% list of line numbers, e.g.
% \begin{quote}
%\begin{verbatim}
%\begin{tabularht}{50mm}{|l|l|}
%  Hello & World\\
%  \interrowfill[1,3]
%  Foo & Bar
%\end{tabularht}
%\end{verbatim}
% \end{quote}
% There are three lines, but the middle line is not drawn
% in the gap between the first and second row.
% Zero can be used to suppress all lines:
% \begin{quote}
% |\interrowspace[0]{10mm}|
% \end{quote}
% The syntax of the commands with the optional argument
% with the line number list \meta{list}. \meta{list} is a comma
% separated list of numbers, \meta{height} means the
% height specification described above with the optional
% prefixes |to=| or |spread=|.
% \begin{quote}
%   \cs{interrowspace}\,|[|\meta{list}|]|\,|{|\meta{height}|}|\\
%   \cs{interrowfill}\,|[|\meta{list}|]|\\
%   \cs{interrowstart}\,|[|\meta{list}|]|\,\dots\,\cs{interrowstop}
% \end{quote}
%
% Option \xoption{vlines} is driver dependent and uses \eTeX\ features.
% \begin{description}
% \item[\xoption{pdftex}:] \pdfTeX\ in PDF mode.
%   Here the positions of the lines are written with the
%   help of the \cs{pdfsavepos} feature into the \xfile{.aux} file(s).
%   Therefore you need two LaTeX runs to get the lines.
% \item[\xoption{dvips}:]
%   Here, PostScript's currentpoint it used to get
%   the line positions. The lines are then drawn at the end
%   of the page. Thus one \LaTeX/dvips run is sufficient
%   for this option.
% \item[Other drivers:]\mbox{}
%   \begin{description}
%   \item[PostScript drivers:] probably possible, an end of page
%     hook would be nice.
%   \item[\VTeX:] with GeX (PostScript interpreter) probably possible.
%   \item[dvipdfm:] no idea. The big problem is, how to get the
%     current position?
%   \end{description}
% \end{description}
%
% \subsection{Limitations}
%
% \begin{itemize}
% \item
%   Vertical lines are interrupted by |\noalign{\vfill}|.
% \end{itemize}
%
% \subsection{Compatibility}
%
% \begin{itemize}
% \item
%   \xpackage{array}, \xpackage{delarray}, \xpackage{tabularx}
%   are supported.
% \item There can be problems with packages that
%   redefine \cs{@array} (or \cs{@@array}, \cs{@tabarray}) and
%   \cs{@arrayrule} (for option \xoption{vlines}).
% \item \xpackage{colortbl}: it should at least work, but there isn't
%   support for filling the gaps with color, neither
%   the rules nor the backgrounds.
% \end{itemize}
%
% \subsection{Examples}
%
% \subsubsection{Example 1}
%
%    \begin{macrocode}
%<*example1>
\documentclass{article}
\usepackage{tabularht}

\begin{document}
\fbox{%
  \begin{tabularht*}{1in}{4in}{@{}l@{\extracolsep{\fill}}r@{}}%
    upper left corner & upper right corner\\%
    \noalign{\vfill}%
    \multicolumn{2}{@{}c@{}}{bounding box}\\%
    \noalign{\vfill}%
    lower left corner & lower right corner\\%
  \end{tabularht*}%
}
\end{document}
%</example1>
%    \end{macrocode}
%
% \subsubsection{Example 2}
%
%    \begin{macrocode}
%<*example2>
\documentclass{article}
\usepackage{booktabs}
\usepackage[dvips,vlines]{tabularht}

\begin{document}

\begin{tabularht}{spread=0pt}{|l|l|}
  \hline
  First&Line\\%
  \hline
\interrowstart
  \addlinespace[10mm]%
\interrowstop
  \hline
  Second&Line\\%
\interrowstart
  \hline
  \hline
\interrowstop
  Third&Line\\%
  \hline
\interrowspace{10mm}
  \hline
  Fourth&Line\\%
  \hline
\end{tabularht}

\end{document}
%</example2>
%    \end{macrocode}
%
% \StopEventually{
% }
%
% \section{Implementation}
%
%    \begin{macrocode}
%<*package>
%    \end{macrocode}
%    Package identification.
%    \begin{macrocode}
\NeedsTeXFormat{LaTeX2e}
\ProvidesPackage{tabularht}%
  [2016/05/16 v2.6 Tabular with height specified (HO)]
%    \end{macrocode}
%
% \subsection{Environments}
%
%    \begin{macrocode}
\let\@toarrayheight\@empty
\let\tabH@array@init\@empty

\toks@={%
  \begingroup
    \long\def\x#1\vcenter\fi\fi\bgroup#2\@sharp#3#4\@nil{%
      \endgroup
      \gdef\@array[##1]##2{%
        \tabH@array@init
        #1%
        \vcenter\fi\fi
        \@toarrayheight
        \bgroup
        \let\@toarrayheight\@empty
        #2\@sharp###3#4%
      }%
    }%
  \expandafter\x\@array[#1]{#2}\@nil % hash-ok
}
\edef\tabH@patch@array{\the\toks@}
\def\tabH@patch@@array{%
  \ifx\@array\@@array
    \def\reserved@a{\let\@@array\@array}%
  \else
    \let\reserved@a\relax
  \fi
  \tabH@patch@array
  \reserved@a
}
\tabH@patch@@array

\@ifpackageloaded{array}{}{%
  \AtBeginDocument{%
    \@ifpackageloaded{array}{%
      \tabH@patch@@array
    }{}%
  }%
}

\def\tabH@setheight#1{%
  \tabH@@setheight#1==\@nil
}
\def\tabH@@setheight#1=#2=#3\@nil{%
  \ifx\\#2#3\\%
    \setlength{\dimen@}{#1}%
    \edef\@toarrayheight{to\the\dimen@}%
  \else
    \edef\tabH@temp{\zap@space#1 \@empty}%
    \ifx\tabH@temp\tabH@to
    \else
      \ifx\tabH@temp\tabH@spread
      \else
        \PackageError{tabularht}{%
          Unknown height specifier %
          `\expandafter\strip@prefix\meaning\tabH@temp'%
        }{%
          The height dimension for tabular height can be prefixed%
          \MessageBreak
          with `to=' or `spread=', default is `to='.%
        }%
        \let\tabH@temp\tabH@to
      \fi
    \fi
    \setlength{\dimen@}{#2}%
    \edef\@toarrayheight{\tabH@temp\the\dimen@}%
  \fi
}
\def\tabH@to{to}
\def\tabH@spread{spread}
%    \end{macrocode}
%    First argument is the height of the table,
%    then the orignal arguments for tabular follow.
%    \begin{macrocode}
\newenvironment{tabularht}[1]{%
  \tabH@setheight{#1}%
  \tabular
}{%
  \endtabular
}

\newenvironment{tabularht*}[1]{%
  \tabH@setheight{#1}%
  \@nameuse{tabular*}%
}{%
  \@nameuse{endtabular*}%
}

\newenvironment{tabularhtx}[1]{%
  \tabH@setheight{#1}%
  \tabularx
}{%
  \endtabularx
}

\newenvironment{arrayht}[1]{%
  \tabH@setheight{#1}%
  \array
}{%
  \endarray
}

\def\interrowspace{%
  \noalign\bgroup
    \tabH@interrowspace
}
\newcommand*{\tabH@interrowspace}[2][]{%
    \tabH@vspace{#1}{#2}%
  \egroup
}
\def\interrowfill{%
  \noalign\bgroup
    \tabH@interrowfill
}
\newcommand*{\tabH@interrowfill}[1][]{%
    \tabH@vspace{#1}{\fill}%
  \egroup
}
\def\tabH@vspace#1#2{%
  \tabH@vspace@start{#1}%
  \vspace{#2}%
  \tabH@vspace@stop
}
\let\tabH@vspace@start\@gobble
\let\tabH@vspace@stop\@empty

\newcommand*{\interrowstart}{%
  \noalign\bgroup
    \tabH@interrowstart
}
\newcommand*{\tabH@interrowstart}[1][]{%
    \tabH@vspace@start{#1}%
  \egroup
}
\newcommand*{\interrowstop}{%
  \noalign{\tabH@vspace@stop}%
}
%    \end{macrocode}
%
% \subsection{Options}
%
%    \begin{macrocode}
\providecommand*{\tabH@driver}{}

\DeclareOption{vlines}{%
  \let\tabH@temp\relax
}
\DeclareOption{pdftex}{}
\DeclareOption{dvips}{%
  \def\tabH@driver{dvips}%
}
\ProcessOptions*\relax

\ifx\tabH@temp\relax
\else
  \expandafter\endinput
\fi

\begingroup
  \@ifundefined{eTeXversion}{%
    \PackageError{tabularht}{%
      Option `vlines' requires eTeX%
    }{%
      Use of eTeX is recommended for LaTeX, see ltnews16.%
    }%
    \endgroup
    \endinput
  }{}%
\endgroup
%    \end{macrocode}
%
% \subsection{Option vlines, driver independent stuff}
%
%    \begin{macrocode}
\begingroup
  \let\@addtoreset\@gobbletwo
  \newcounter{tabH@unique}%
\endgroup
\let\tabH@currenttab\@empty

\def\tabH@array@init{%
  \ifx\@toarrayheight\@empty
    % ignore vertical lines of nested tabular environments
    \let\tabH@currenttab\@empty
  \else
    \stepcounter{tabH@unique}%
    \edef\tabH@currenttab{\the\c@tabH@unique}%
  \fi
}

\renewcommand*{\@arrayrule}{%
  \@addtopreamble{%
    \hskip -.5\arrayrulewidth
    \ifx\tabH@currenttab\@empty
    \else
      \tabH@vrule{\tabH@currenttab}%
    \fi
    \begingroup
      \expandafter\ifx\csname CT@arc@\endcsname\relax
      \else
        \expandafter\CT@arc@
      \fi
      \vline
    \endgroup
    \hskip -.5\arrayrulewidth
  }%
}
\let\tabH@arrayrule\@arrayrule
\AtBeginDocument{%
  \@ifpackageloaded{colortbl}{%
    \let\@arrayrule\tabH@arrayrule
  }{}%
}

\let\tabH@vrule\@gobble
%    \end{macrocode}
%
% \subsection{Driver pdftex}
%
%    \begin{macrocode}
\RequirePackage{ifpdf}
\ifpdf
  \begingroup
    \@ifundefined{pdfsavepos}{%
      \PackageError{tabularht}{%
        Your pdfTeX is too old%
      }{%
        \string\pdfsavepos\space is missing.%
      }%
      \endgroup
      \csname fi\endcsname
      \endinput
    }{}%

    \let\on@line\@empty
    \PackageInfo{tabularht}{%
      Using driver `pdftex' because of pdfTeX in PDF mode%
    }%
  \endgroup

  \protected\def\tabH@vrule#1{%
    \if@filesw
      \pdfsavepos
      \protected@write\@auxout{%
        \let\tabH@lastxpos\relax
      }{%
        \tabH@aux@vrule{#1}{\tabH@lastxpos}%
      }%
    \fi
  }%

  \def\tabH@lastxpos{\the\pdflastxpos}%
  \def\tabH@lastypos{\the\pdflastypos}%

  % The .aux file contains three commands:
  % \tabH@aux@vrule{tabular id}{x position}
  % \tabH@aux@vstart{tabular id}{row id}{x position}{y position}
  % \tabH@aux@vstop{y position}
  %
  \AtBeginDocument{%
    % The .aux files are read the first time before
    % \AtBeginDocument and later at \end{document}.
    % \tabH@aux@done is a marker to distinguish
    % between these two readings. Only in the first
    % case we need the \tabH@aux@... commands.
    \let\tabH@aux@done\@empty
    \if@filesw
      \immediate\write\@mainaux{%
        \@percentchar\@percentchar BeginProlog: tabularht%
      }%
      % items in the aux file are executed,
      % if tabularht is loaded
      % and during the aux file read at \begin{document} only
      \immediate\write\@mainaux{%
        \detokenize{%
          % the \tabH@aux@... commands are needed only if
          % tabularht is loaded with driver pdftex.
          \@ifundefined{tabH@aux@vrule}\@secondoftwo\@firstofone
          {%
            % disable commands except for the first .aux files reading
            \@ifundefined{tabH@aux@done}\@gobble\@firstofone
          }%
          {%
            \let\tabH@aux@vrule\@gobbletwo
            \let\tabH@aux@vstart\@gobblefour
            \let\tabH@aux@vstop\@gobble
          }%
        }%
      }%
      \immediate\write\@mainaux{%
        \@percentchar\@percentchar EndProlog: tabularht%
      }%
    \fi
  }%

  % the x positions of vrules are stored in
  % \tabH@<tabcount>list with distinct values
  \protected\def\tabH@aux@vrule#1#2{%
    \@ifundefined{tabH@#1list}{%
      \expandafter\xdef\csname tabH@#1list\endcsname{%
        \noexpand\do{#2}%
      }%
    }{%
      \begingroup
        \def\x{#2}%
        \let\y\@undefined
        \let\do\tabH@do@add
        \expandafter\xdef\csname tabH@#1list\endcsname{%
          \csname tabH@#1list\endcsname\@empty
          \ifx\y\@undefined
            \noexpand\do{\x}%
          \fi
        }%
      \endgroup
    }%
  }%
  \def\tabH@do@add#1{%
    \ifx\y\@undefined
      \ifnum#1<\x\space
      \else
        \expandafter\ifx\csname y\endcsname\relax\fi
        \ifnum#1>\x\space
          \noexpand\do{\x}%
        \fi
      \fi
    \fi
    \noexpand\do{#1}%
  }%

  \def\tabH@vspace@start#1{%
    \if@filesw
      \stepcounter{tabH@unique}%
      \edef\tabH@currentrow{\the\c@tabH@unique}%
      \pdfsavepos
      \protected@write\@auxout{%
        \let\tabH@lastxpos\relax
        \let\tabH@lastypos\relax
      }{%
        \tabH@aux@vstart{\tabH@currenttab}{\tabH@currentrow}%
                        {\tabH@lastxpos}{\tabH@lastypos}%
      }%
    \fi
    \begingroup
      \edef\a{tabH@\tabH@currenttab row\tabH@currentrow}%
      \expandafter\let\expandafter\x\csname\a x\endcsname
      \ifx\x\relax
      \else
        \expandafter\let\expandafter\y\csname\a y\endcsname
        \expandafter\let\expandafter\l
            \csname tabH@\tabH@currenttab list\endcsname
        \ifx\l\relax
        \else
          \def\f{#1}%
          \ifx\f\@empty
            \let\do\tabH@do@set
          \else
            \count@=\z@
            \let\do\tabH@do@filter
          \fi
          \setbox\z@=\hbox{\l}%
          \wd\z@=\z@
          \dp\z@=\z@
          \copy\z@
        \fi
      \fi
    \endgroup
  }%
  \def\tabH@vspace@stop{%
    \if@filesw
      \pdfsavepos
      \protected@write\@auxout{%
        \let\tabH@lastypos\relax
      }{%
        \tabH@aux@vstop{\tabH@lastypos}%
      }%
    \fi
  }%
  \def\tabH@do@set#1{%
    \hbox to \z@{%
      \hskip \dimexpr #1sp - \x sp\relax
      \vrule \@width\arrayrulewidth
             \@depth\dimexpr \y sp\relax
      \hss
    }%
  }%
  \def\tabH@do@filter{%
    \@tempswafalse
    \advance\count@\@ne
    \@for\e:=\f\do{%
      \ifnum\e=\count@
        \@tempswatrue
      \fi
    }%
    \if@tempswa
      \expandafter\tabH@do@set
    \else
      \expandafter\@gobble
    \fi
  }%

  \protected\def\tabH@aux@vstart#1#2#3#4{%
    \def\tabH@current@vstart{{#1}{#2}{#3}{#4}}%
  }%
  \protected\def\tabH@aux@vstop{%
    \expandafter\tabH@aux@v\tabH@current@vstart
  }%
  \def\tabH@aux@v#1#2#3#4#5{%
    \expandafter\gdef\csname tabH@#1row#2x\endcsname{#3}%
    \expandafter\xdef\csname tabH@#1row#2y\endcsname{%
      \the\numexpr #4 - #5\relax
    }%
  }%

  \csname fi\endcsname
  \endinput

\fi
%    \end{macrocode}
%
% \subsection{DVI drivers}
%
%    \begin{macrocode}
\ifx\tabH@driver\@empty
  \PackageError{tabularht}{%
    Missing DVI driver, option `vlines' disabled%
  }{%
    Supported DVI drivers: dvips.%
  }%
  \expandafter\endinput
\fi

\def\tabH@driver@dvips{%
  \def\tabH@literalps##1{\special{ps:SDict begin ##1 end}}%
  \def\tabH@headerps##1{\special{! ##1}}%
}

\@onelevel@sanitize\tabH@driver
\@ifundefined{tabH@driver@\tabH@driver}{%
  \PackageError{tabularht}{%
    Unsupported driver `\tabH@driver'%
  }{%
    Supported DVI drivers: dvips.%
  }%
  \endinput
}{}

\begingroup
  \let\on@line\@empty
  \PackageInfo{tabularht}{%
    Using driver `\tabH@driver'%
  }%
\endgroup
\csname tabH@driver@\tabH@driver\endcsname

\protected\def\tabH@vrule#1#2\vrule#3\arrayrulewidth{%
  #2% \fi or empty
  % hack to get rid of maxdrift rounding of dvips,
  % thus simulate a large motion
  \kern1in\relax
  \tabH@literalps{%
    #1 tabH.vrule %
    Resolution neg 0 translate%
  }%
  \vrule#3\arrayrulewidth
  \tabH@literalps{Resolution 0 translate}%
  \kern-1in\relax
}

\def\tabH@vspace@start#1{%
  \begingroup
    \let\y\@empty
    \@for\x:=#1\do{%
      \ifx\y\@empty
        \edef\y{\x}%
      \else
        \edef\y{\y\space\x}%
      \fi
    }%
    \tabH@literalps{\tabH@currenttab[\y]currentpoint exch pop}%
  \endgroup
}
\def\tabH@vspace@stop{%
  \tabH@literalps{%
    currentpoint exch pop %
    \number\dimexpr\arrayrulewidth\relax\space
    tabH.vspace%
  }%
}

\tabH@headerps{%
  userdict begin%
    /tabH.list 10 dict def%
    /tabH.job [] def %
  end%
  /tabH.vrule{%
    10 string cvs cvn dup tabH.list exch known{%
      tabH.list exch dup [ exch tabH.list exch get %
      currentpoint pop round exch true exch{%
        % tabH.list key [ ... x true i
        % tabH.list key [ ... false i
        exch{%
          % ... [ ... x i
          2 copy lt{false}{%
            2 copy eq{pop false}{exch true}ifelse%
          }ifelse%
        }{false}ifelse%
      }forall %
      pop%
      ]put%
    }{%
      tabH.list exch[currentpoint pop round]put%
    }ifelse%
  }bind def%
  % <tab num> <cols array> <ytop> <ybottom> <rulewidth[sp]>
  /tabH.vspace{%
    userdict begin %
      10 dict dup begin %
        exch 65536 div Resolution mul 72.27 div %
        % dvips uses a poor man's ceil function
        % see dopage.c before "drawrule": (int)(... + 0.9999999)
        0.9999999 add truncate%
        /rulewidth exch def %
        exch/ybottom exch def %
        exch/ytop exch def %
        exch/cols exch def %
        exch/tabkey exch 10 string cvs cvn def %
      end%
      /tabH.job exch[exch userdict/tabH.job get aload pop]def %
    end%
  }bind def %
  % Now we do the work at the end of the page.
  % Unhappily "eop-hook" cannot be used, because "eop"
  % executes "restore" before, so that all data are lost.
  TeXDict begin%
    /eop%
    [%
      {%
        tabH.job{%
          begin%
           /colarray %
             tabH.list tabkey known{tabH.list tabkey get}{[]}ifelse %
           def %
           cols length 0 eq not{%
             /colarray[%
               cols{1 sub %
                 dup 0 lt{pop}{%
                   dup colarray length ge{pop}{%
                     colarray exch get%
                   }ifelse%
                 }ifelse%
               }forall%
             ]def%
           }if %
           colarray{%
             % (rulewidth) == rulewidth == % debug
             Resolution sub %
             ytop rulewidth ytop ybottom sub v%
           }forall %
          end%
        }forall%
        % tabH.list{== ==}forall % debug
      }bind aload pop %
      TeXDict /eop get aload pop%
    ]cvx def %
  end%
}
%    \end{macrocode}
%
%    \begin{macrocode}
%</package>
%    \end{macrocode}
%
% \section{Installation}
%
% \subsection{Download}
%
% \paragraph{Package.} This package is available on
% CTAN\footnote{\CTANpkg{tabularht}}:
% \begin{description}
% \item[\CTAN{macros/latex/contrib/oberdiek/tabularht.dtx}] The source file.
% \item[\CTAN{macros/latex/contrib/oberdiek/tabularht.pdf}] Documentation.
% \end{description}
%
%
% \paragraph{Bundle.} All the packages of the bundle `oberdiek'
% are also available in a TDS compliant ZIP archive. There
% the packages are already unpacked and the documentation files
% are generated. The files and directories obey the TDS standard.
% \begin{description}
% \item[\CTANinstall{install/macros/latex/contrib/oberdiek.tds.zip}]
% \end{description}
% \emph{TDS} refers to the standard ``A Directory Structure
% for \TeX\ Files'' (\CTAN{tds/tds.pdf}). Directories
% with \xfile{texmf} in their name are usually organized this way.
%
% \subsection{Bundle installation}
%
% \paragraph{Unpacking.} Unpack the \xfile{oberdiek.tds.zip} in the
% TDS tree (also known as \xfile{texmf} tree) of your choice.
% Example (linux):
% \begin{quote}
%   |unzip oberdiek.tds.zip -d ~/texmf|
% \end{quote}
%
% \paragraph{Script installation.}
% Check the directory \xfile{TDS:scripts/oberdiek/} for
% scripts that need further installation steps.
%
% \subsection{Package installation}
%
% \paragraph{Unpacking.} The \xfile{.dtx} file is a self-extracting
% \docstrip\ archive. The files are extracted by running the
% \xfile{.dtx} through \plainTeX:
% \begin{quote}
%   \verb|tex tabularht.dtx|
% \end{quote}
%
% \paragraph{TDS.} Now the different files must be moved into
% the different directories in your installation TDS tree
% (also known as \xfile{texmf} tree):
% \begin{quote}
% \def\t{^^A
% \begin{tabular}{@{}>{\ttfamily}l@{ $\rightarrow$ }>{\ttfamily}l@{}}
%   tabularht.sty & tex/latex/oberdiek/tabularht.sty\\
%   tabularht.pdf & doc/latex/oberdiek/tabularht.pdf\\
%   tabularht-example1.tex & doc/latex/oberdiek/tabularht-example1.tex\\
%   tabularht-example2.tex & doc/latex/oberdiek/tabularht-example2.tex\\
%   tabularht.dtx & source/latex/oberdiek/tabularht.dtx\\
% \end{tabular}^^A
% }^^A
% \sbox0{\t}^^A
% \ifdim\wd0>\linewidth
%   \begingroup
%     \advance\linewidth by\leftmargin
%     \advance\linewidth by\rightmargin
%   \edef\x{\endgroup
%     \def\noexpand\lw{\the\linewidth}^^A
%   }\x
%   \def\lwbox{^^A
%     \leavevmode
%     \hbox to \linewidth{^^A
%       \kern-\leftmargin\relax
%       \hss
%       \usebox0
%       \hss
%       \kern-\rightmargin\relax
%     }^^A
%   }^^A
%   \ifdim\wd0>\lw
%     \sbox0{\small\t}^^A
%     \ifdim\wd0>\linewidth
%       \ifdim\wd0>\lw
%         \sbox0{\footnotesize\t}^^A
%         \ifdim\wd0>\linewidth
%           \ifdim\wd0>\lw
%             \sbox0{\scriptsize\t}^^A
%             \ifdim\wd0>\linewidth
%               \ifdim\wd0>\lw
%                 \sbox0{\tiny\t}^^A
%                 \ifdim\wd0>\linewidth
%                   \lwbox
%                 \else
%                   \usebox0
%                 \fi
%               \else
%                 \lwbox
%               \fi
%             \else
%               \usebox0
%             \fi
%           \else
%             \lwbox
%           \fi
%         \else
%           \usebox0
%         \fi
%       \else
%         \lwbox
%       \fi
%     \else
%       \usebox0
%     \fi
%   \else
%     \lwbox
%   \fi
% \else
%   \usebox0
% \fi
% \end{quote}
% If you have a \xfile{docstrip.cfg} that configures and enables \docstrip's
% TDS installing feature, then some files can already be in the right
% place, see the documentation of \docstrip.
%
% \subsection{Refresh file name databases}
%
% If your \TeX~distribution
% (\teTeX, \mikTeX, \dots) relies on file name databases, you must refresh
% these. For example, \teTeX\ users run \verb|texhash| or
% \verb|mktexlsr|.
%
% \subsection{Some details for the interested}
%
% \paragraph{Unpacking with \LaTeX.}
% The \xfile{.dtx} chooses its action depending on the format:
% \begin{description}
% \item[\plainTeX:] Run \docstrip\ and extract the files.
% \item[\LaTeX:] Generate the documentation.
% \end{description}
% If you insist on using \LaTeX\ for \docstrip\ (really,
% \docstrip\ does not need \LaTeX), then inform the autodetect routine
% about your intention:
% \begin{quote}
%   \verb|latex \let\install=y\input{tabularht.dtx}|
% \end{quote}
% Do not forget to quote the argument according to the demands
% of your shell.
%
% \paragraph{Generating the documentation.}
% You can use both the \xfile{.dtx} or the \xfile{.drv} to generate
% the documentation. The process can be configured by the
% configuration file \xfile{ltxdoc.cfg}. For instance, put this
% line into this file, if you want to have A4 as paper format:
% \begin{quote}
%   \verb|\PassOptionsToClass{a4paper}{article}|
% \end{quote}
% An example follows how to generate the
% documentation with pdf\LaTeX:
% \begin{quote}
%\begin{verbatim}
%pdflatex tabularht.dtx
%makeindex -s gind.ist tabularht.idx
%pdflatex tabularht.dtx
%makeindex -s gind.ist tabularht.idx
%pdflatex tabularht.dtx
%\end{verbatim}
% \end{quote}
%
% \begin{History}
%   \begin{Version}{2005/09/22 v1.0}
%   \item
%     First public version.
%   \end{Version}
%   \begin{Version}{2005/10/16 v2.0}
%   \item
%     Height specification allows
%     |to=...| or |spread=...|, default is |to=|.
%   \item
%     Option \xoption{vlines} added,
%     drivers \xoption{pdftex} and \xoption{dvips}.
%   \item
%     \cs{interrowspace}, \cs{interrowfil}, and
%     \cs{interrowstart}\dots\cs{interrowstop} added.
%   \end{Version}
%   \begin{Version}{2005/10/18 v2.1}
%   \item
%     Fix for package \xpackage{colortbl}, but the colors
%     of \xpackage{colortbl} remain unsupported.
%   \end{Version}
%   \begin{Version}{2006/02/20 v2.2}
%   \item
%     Code is not changed.
%   \item
%     DTX framework.
%   \end{Version}
%   \begin{Version}{2006/12/22 v2.3}
%   \item
%     Documentation fix.
%   \item
%     Fix in code of option vlines.
%   \end{Version}
%   \begin{Version}{2007/03/21 v2.4}
%   \item
%     Fix: Counter |tabh@unique| must not be changed by \cs{include}.
%   \end{Version}
%   \begin{Version}{2007/04/11 v2.5}
%   \item
%     Line ends sanitized.
%   \end{Version}
%   \begin{Version}{2016/05/16 v2.6}
%   \item
%     Documentation updates.
%   \end{Version}
% \end{History}
%
% \PrintIndex
%
% \Finale
\endinput

%        (quote the arguments according to the demands of your shell)
%
% Documentation:
%    (a) If tabularht.drv is present:
%           latex tabularht.drv
%    (b) Without tabularht.drv:
%           latex tabularht.dtx; ...
%    The class ltxdoc loads the configuration file ltxdoc.cfg
%    if available. Here you can specify further options, e.g.
%    use A4 as paper format:
%       \PassOptionsToClass{a4paper}{article}
%
%    Programm calls to get the documentation (example):
%       pdflatex tabularht.dtx
%       makeindex -s gind.ist tabularht.idx
%       pdflatex tabularht.dtx
%       makeindex -s gind.ist tabularht.idx
%       pdflatex tabularht.dtx
%
% Installation:
%    TDS:tex/latex/oberdiek/tabularht.sty
%    TDS:doc/latex/oberdiek/tabularht.pdf
%    TDS:doc/latex/oberdiek/tabularht-example1.tex
%    TDS:doc/latex/oberdiek/tabularht-example2.tex
%    TDS:source/latex/oberdiek/tabularht.dtx
%
%<*ignore>
\begingroup
  \catcode123=1 %
  \catcode125=2 %
  \def\x{LaTeX2e}%
\expandafter\endgroup
\ifcase 0\ifx\install y1\fi\expandafter
         \ifx\csname processbatchFile\endcsname\relax\else1\fi
         \ifx\fmtname\x\else 1\fi\relax
\else\csname fi\endcsname
%</ignore>
%<*install>
\input docstrip.tex
\Msg{************************************************************************}
\Msg{* Installation}
\Msg{* Package: tabularht 2016/05/16 v2.6 Tabular with height specified (HO)}
\Msg{************************************************************************}

\keepsilent
\askforoverwritefalse

\let\MetaPrefix\relax
\preamble

This is a generated file.

Project: tabularht
Version: 2016/05/16 v2.6

Copyright (C) 2005-2007 by
   Heiko Oberdiek <heiko.oberdiek at googlemail.com>

This work may be distributed and/or modified under the
conditions of the LaTeX Project Public License, either
version 1.3c of this license or (at your option) any later
version. This version of this license is in
   https://www.latex-project.org/lppl/lppl-1-3c.txt
and the latest version of this license is in
   https://www.latex-project.org/lppl.txt
and version 1.3 or later is part of all distributions of
LaTeX version 2005/12/01 or later.

This work has the LPPL maintenance status "maintained".

The Current Maintainers of this work are
Heiko Oberdiek and the Oberdiek Package Support Group
https://github.com/ho-tex/oberdiek/issues


This work consists of the main source file tabularht.dtx
and the derived files
   tabularht.sty, tabularht.pdf, tabularht.ins, tabularht.drv,
   tabularht-example1.tex, tabularht-example2.tex.

\endpreamble
\let\MetaPrefix\DoubleperCent

\generate{%
  \file{tabularht.ins}{\from{tabularht.dtx}{install}}%
  \file{tabularht.drv}{\from{tabularht.dtx}{driver}}%
  \usedir{tex/latex/oberdiek}%
  \file{tabularht.sty}{\from{tabularht.dtx}{package}}%
  \usedir{doc/latex/oberdiek}%
  \file{tabularht-example1.tex}{\from{tabularht.dtx}{example1}}%
  \file{tabularht-example2.tex}{\from{tabularht.dtx}{example2}}%
  \nopreamble
  \nopostamble
%  \usedir{source/latex/oberdiek/catalogue}%
%  \file{tabularht.xml}{\from{tabularht.dtx}{catalogue}}%
}

\catcode32=13\relax% active space
\let =\space%
\Msg{************************************************************************}
\Msg{*}
\Msg{* To finish the installation you have to move the following}
\Msg{* file into a directory searched by TeX:}
\Msg{*}
\Msg{*     tabularht.sty}
\Msg{*}
\Msg{* To produce the documentation run the file `tabularht.drv'}
\Msg{* through LaTeX.}
\Msg{*}
\Msg{* Happy TeXing!}
\Msg{*}
\Msg{************************************************************************}

\endbatchfile
%</install>
%<*ignore>
\fi
%</ignore>
%<*driver>
\NeedsTeXFormat{LaTeX2e}
\ProvidesFile{tabularht.drv}%
  [2016/05/16 v2.6 Tabular with height specified (HO)]%
\documentclass{ltxdoc}
\usepackage{holtxdoc}[2011/11/22]
\begin{document}
  \DocInput{tabularht.dtx}%
\end{document}
%</driver>
% \fi
%
%
% \CharacterTable
%  {Upper-case    \A\B\C\D\E\F\G\H\I\J\K\L\M\N\O\P\Q\R\S\T\U\V\W\X\Y\Z
%   Lower-case    \a\b\c\d\e\f\g\h\i\j\k\l\m\n\o\p\q\r\s\t\u\v\w\x\y\z
%   Digits        \0\1\2\3\4\5\6\7\8\9
%   Exclamation   \!     Double quote  \"     Hash (number) \#
%   Dollar        \$     Percent       \%     Ampersand     \&
%   Acute accent  \'     Left paren    \(     Right paren   \)
%   Asterisk      \*     Plus          \+     Comma         \,
%   Minus         \-     Point         \.     Solidus       \/
%   Colon         \:     Semicolon     \;     Less than     \<
%   Equals        \=     Greater than  \>     Question mark \?
%   Commercial at \@     Left bracket  \[     Backslash     \\
%   Right bracket \]     Circumflex    \^     Underscore    \_
%   Grave accent  \`     Left brace    \{     Vertical bar  \|
%   Right brace   \}     Tilde         \~}
%
% \GetFileInfo{tabularht.drv}
%
% \title{The \xpackage{tabularht} package}
% \date{2016/05/16 v2.6}
% \author{Heiko Oberdiek\thanks
% {Please report any issues at \url{https://github.com/ho-tex/oberdiek/issues}}}
%
% \maketitle
%
% \begin{abstract}
% This package defines some environments that adds
% a height specification to tabular and array.
% \end{abstract}
%
% \tableofcontents
%
% \section{Usage}
%
% \begin{quote}
%   |\usepackage{tabularht}|
% \end{quote}
% The package provides the following environments
% that extend the tabular/array environment by
% a height specification as first argument:
% \begin{itemize}
% \item |tabularht|, |tabularht*|
% \item |arrayht|
% \item |tabularhtx|
%       (if package \xpackage{tabularx} is loaded)
% \end{itemize}
% The height argument allows a length specification,
% package \xpackage{calc} is supported if used. This means,
% the tabular will have the specified height.
% You can also use the prefixes |to=| and |spread=|.
% |to=| is the default, |spread=| means, the natural
% height of the tabular box is changed by the length
% after |spread=|.
%
% Examples:
% \begin{quote}
% \begin{tabular}{@{}l@{ $\rightarrow$ }l@{}}
%   |\begin{tabularht}{1in}|
%   & height is 1in\\
%   |\begin{tabularht}{to=1in}|
%   & height is 1in\\
%   |\begin{tabularht}{spread=0pt}|
%   & natural height, same as |\begin{tabular}|\\
%   |\begin{tabularht}{spread=1in}|
%   & natural height increased by 1in
% \end{tabular}
% \end{quote}
%
% Hint:  See also package \xpackage{tabularkv}, it provides an interface,
%        where most parameters for the environments can be given
%        by key-value pairs.
%
% \begin{declcs}{interrowspace}\,|{|\dots|}|
% \end{declcs}
% Adds space between table rows. It is essentially
% the same as |\noalign{\vspace{...}}|.
% \begin{declcs}{interrowfill}
% \end{declcs}
% Short for |\interrowspace{\fill}|
% \begin{declcs}{interrowstart}\,\dots\,\cs{interrowstop}
% \end{declcs}
% Marker commands, useful for option \xoption{vlines}.
%
% \subsection{Option \xoption{vlines}}
%
% Warning: This stuff is experimental.
%
% \vspace{2ex}
% \noindent
% Vertical lines are interrupted, if space is inserted
% in \cs{noalign}, \cs{interrowspace}, \cs{addlinespace}
% (\xpackage{booktabs}),
% between double \cs{hline}s. This option tries to detect
% and add the vertical lines. The lines in a tabular
% with \xpackage{tabularht} support (environments of this package)
% are numbered from left to right. The gap that is
% controlled by \cs{interrowspace} or inbetween \cs{interrowstart}
% and \cs{interrowstop} is then filled with the detected
% vertical lines.
%
% If only a limited selection of the lines should be
% drawn, the commands know an optional argument with a
% list of line numbers, e.g.
% \begin{quote}
%\begin{verbatim}
%\begin{tabularht}{50mm}{|l|l|}
%  Hello & World\\
%  \interrowfill[1,3]
%  Foo & Bar
%\end{tabularht}
%\end{verbatim}
% \end{quote}
% There are three lines, but the middle line is not drawn
% in the gap between the first and second row.
% Zero can be used to suppress all lines:
% \begin{quote}
% |\interrowspace[0]{10mm}|
% \end{quote}
% The syntax of the commands with the optional argument
% with the line number list \meta{list}. \meta{list} is a comma
% separated list of numbers, \meta{height} means the
% height specification described above with the optional
% prefixes |to=| or |spread=|.
% \begin{quote}
%   \cs{interrowspace}\,|[|\meta{list}|]|\,|{|\meta{height}|}|\\
%   \cs{interrowfill}\,|[|\meta{list}|]|\\
%   \cs{interrowstart}\,|[|\meta{list}|]|\,\dots\,\cs{interrowstop}
% \end{quote}
%
% Option \xoption{vlines} is driver dependent and uses \eTeX\ features.
% \begin{description}
% \item[\xoption{pdftex}:] \pdfTeX\ in PDF mode.
%   Here the positions of the lines are written with the
%   help of the \cs{pdfsavepos} feature into the \xfile{.aux} file(s).
%   Therefore you need two LaTeX runs to get the lines.
% \item[\xoption{dvips}:]
%   Here, PostScript's currentpoint it used to get
%   the line positions. The lines are then drawn at the end
%   of the page. Thus one \LaTeX/dvips run is sufficient
%   for this option.
% \item[Other drivers:]\mbox{}
%   \begin{description}
%   \item[PostScript drivers:] probably possible, an end of page
%     hook would be nice.
%   \item[\VTeX:] with GeX (PostScript interpreter) probably possible.
%   \item[dvipdfm:] no idea. The big problem is, how to get the
%     current position?
%   \end{description}
% \end{description}
%
% \subsection{Limitations}
%
% \begin{itemize}
% \item
%   Vertical lines are interrupted by |\noalign{\vfill}|.
% \end{itemize}
%
% \subsection{Compatibility}
%
% \begin{itemize}
% \item
%   \xpackage{array}, \xpackage{delarray}, \xpackage{tabularx}
%   are supported.
% \item There can be problems with packages that
%   redefine \cs{@array} (or \cs{@@array}, \cs{@tabarray}) and
%   \cs{@arrayrule} (for option \xoption{vlines}).
% \item \xpackage{colortbl}: it should at least work, but there isn't
%   support for filling the gaps with color, neither
%   the rules nor the backgrounds.
% \end{itemize}
%
% \subsection{Examples}
%
% \subsubsection{Example 1}
%
%    \begin{macrocode}
%<*example1>
\documentclass{article}
\usepackage{tabularht}

\begin{document}
\fbox{%
  \begin{tabularht*}{1in}{4in}{@{}l@{\extracolsep{\fill}}r@{}}%
    upper left corner & upper right corner\\%
    \noalign{\vfill}%
    \multicolumn{2}{@{}c@{}}{bounding box}\\%
    \noalign{\vfill}%
    lower left corner & lower right corner\\%
  \end{tabularht*}%
}
\end{document}
%</example1>
%    \end{macrocode}
%
% \subsubsection{Example 2}
%
%    \begin{macrocode}
%<*example2>
\documentclass{article}
\usepackage{booktabs}
\usepackage[dvips,vlines]{tabularht}

\begin{document}

\begin{tabularht}{spread=0pt}{|l|l|}
  \hline
  First&Line\\%
  \hline
\interrowstart
  \addlinespace[10mm]%
\interrowstop
  \hline
  Second&Line\\%
\interrowstart
  \hline
  \hline
\interrowstop
  Third&Line\\%
  \hline
\interrowspace{10mm}
  \hline
  Fourth&Line\\%
  \hline
\end{tabularht}

\end{document}
%</example2>
%    \end{macrocode}
%
% \StopEventually{
% }
%
% \section{Implementation}
%
%    \begin{macrocode}
%<*package>
%    \end{macrocode}
%    Package identification.
%    \begin{macrocode}
\NeedsTeXFormat{LaTeX2e}
\ProvidesPackage{tabularht}%
  [2016/05/16 v2.6 Tabular with height specified (HO)]
%    \end{macrocode}
%
% \subsection{Environments}
%
%    \begin{macrocode}
\let\@toarrayheight\@empty
\let\tabH@array@init\@empty

\toks@={%
  \begingroup
    \long\def\x#1\vcenter\fi\fi\bgroup#2\@sharp#3#4\@nil{%
      \endgroup
      \gdef\@array[##1]##2{%
        \tabH@array@init
        #1%
        \vcenter\fi\fi
        \@toarrayheight
        \bgroup
        \let\@toarrayheight\@empty
        #2\@sharp###3#4%
      }%
    }%
  \expandafter\x\@array[#1]{#2}\@nil % hash-ok
}
\edef\tabH@patch@array{\the\toks@}
\def\tabH@patch@@array{%
  \ifx\@array\@@array
    \def\reserved@a{\let\@@array\@array}%
  \else
    \let\reserved@a\relax
  \fi
  \tabH@patch@array
  \reserved@a
}
\tabH@patch@@array

\@ifpackageloaded{array}{}{%
  \AtBeginDocument{%
    \@ifpackageloaded{array}{%
      \tabH@patch@@array
    }{}%
  }%
}

\def\tabH@setheight#1{%
  \tabH@@setheight#1==\@nil
}
\def\tabH@@setheight#1=#2=#3\@nil{%
  \ifx\\#2#3\\%
    \setlength{\dimen@}{#1}%
    \edef\@toarrayheight{to\the\dimen@}%
  \else
    \edef\tabH@temp{\zap@space#1 \@empty}%
    \ifx\tabH@temp\tabH@to
    \else
      \ifx\tabH@temp\tabH@spread
      \else
        \PackageError{tabularht}{%
          Unknown height specifier %
          `\expandafter\strip@prefix\meaning\tabH@temp'%
        }{%
          The height dimension for tabular height can be prefixed%
          \MessageBreak
          with `to=' or `spread=', default is `to='.%
        }%
        \let\tabH@temp\tabH@to
      \fi
    \fi
    \setlength{\dimen@}{#2}%
    \edef\@toarrayheight{\tabH@temp\the\dimen@}%
  \fi
}
\def\tabH@to{to}
\def\tabH@spread{spread}
%    \end{macrocode}
%    First argument is the height of the table,
%    then the orignal arguments for tabular follow.
%    \begin{macrocode}
\newenvironment{tabularht}[1]{%
  \tabH@setheight{#1}%
  \tabular
}{%
  \endtabular
}

\newenvironment{tabularht*}[1]{%
  \tabH@setheight{#1}%
  \@nameuse{tabular*}%
}{%
  \@nameuse{endtabular*}%
}

\newenvironment{tabularhtx}[1]{%
  \tabH@setheight{#1}%
  \tabularx
}{%
  \endtabularx
}

\newenvironment{arrayht}[1]{%
  \tabH@setheight{#1}%
  \array
}{%
  \endarray
}

\def\interrowspace{%
  \noalign\bgroup
    \tabH@interrowspace
}
\newcommand*{\tabH@interrowspace}[2][]{%
    \tabH@vspace{#1}{#2}%
  \egroup
}
\def\interrowfill{%
  \noalign\bgroup
    \tabH@interrowfill
}
\newcommand*{\tabH@interrowfill}[1][]{%
    \tabH@vspace{#1}{\fill}%
  \egroup
}
\def\tabH@vspace#1#2{%
  \tabH@vspace@start{#1}%
  \vspace{#2}%
  \tabH@vspace@stop
}
\let\tabH@vspace@start\@gobble
\let\tabH@vspace@stop\@empty

\newcommand*{\interrowstart}{%
  \noalign\bgroup
    \tabH@interrowstart
}
\newcommand*{\tabH@interrowstart}[1][]{%
    \tabH@vspace@start{#1}%
  \egroup
}
\newcommand*{\interrowstop}{%
  \noalign{\tabH@vspace@stop}%
}
%    \end{macrocode}
%
% \subsection{Options}
%
%    \begin{macrocode}
\providecommand*{\tabH@driver}{}

\DeclareOption{vlines}{%
  \let\tabH@temp\relax
}
\DeclareOption{pdftex}{}
\DeclareOption{dvips}{%
  \def\tabH@driver{dvips}%
}
\ProcessOptions*\relax

\ifx\tabH@temp\relax
\else
  \expandafter\endinput
\fi

\begingroup
  \@ifundefined{eTeXversion}{%
    \PackageError{tabularht}{%
      Option `vlines' requires eTeX%
    }{%
      Use of eTeX is recommended for LaTeX, see ltnews16.%
    }%
    \endgroup
    \endinput
  }{}%
\endgroup
%    \end{macrocode}
%
% \subsection{Option vlines, driver independent stuff}
%
%    \begin{macrocode}
\begingroup
  \let\@addtoreset\@gobbletwo
  \newcounter{tabH@unique}%
\endgroup
\let\tabH@currenttab\@empty

\def\tabH@array@init{%
  \ifx\@toarrayheight\@empty
    % ignore vertical lines of nested tabular environments
    \let\tabH@currenttab\@empty
  \else
    \stepcounter{tabH@unique}%
    \edef\tabH@currenttab{\the\c@tabH@unique}%
  \fi
}

\renewcommand*{\@arrayrule}{%
  \@addtopreamble{%
    \hskip -.5\arrayrulewidth
    \ifx\tabH@currenttab\@empty
    \else
      \tabH@vrule{\tabH@currenttab}%
    \fi
    \begingroup
      \expandafter\ifx\csname CT@arc@\endcsname\relax
      \else
        \expandafter\CT@arc@
      \fi
      \vline
    \endgroup
    \hskip -.5\arrayrulewidth
  }%
}
\let\tabH@arrayrule\@arrayrule
\AtBeginDocument{%
  \@ifpackageloaded{colortbl}{%
    \let\@arrayrule\tabH@arrayrule
  }{}%
}

\let\tabH@vrule\@gobble
%    \end{macrocode}
%
% \subsection{Driver pdftex}
%
%    \begin{macrocode}
\RequirePackage{ifpdf}
\ifpdf
  \begingroup
    \@ifundefined{pdfsavepos}{%
      \PackageError{tabularht}{%
        Your pdfTeX is too old%
      }{%
        \string\pdfsavepos\space is missing.%
      }%
      \endgroup
      \csname fi\endcsname
      \endinput
    }{}%

    \let\on@line\@empty
    \PackageInfo{tabularht}{%
      Using driver `pdftex' because of pdfTeX in PDF mode%
    }%
  \endgroup

  \protected\def\tabH@vrule#1{%
    \if@filesw
      \pdfsavepos
      \protected@write\@auxout{%
        \let\tabH@lastxpos\relax
      }{%
        \tabH@aux@vrule{#1}{\tabH@lastxpos}%
      }%
    \fi
  }%

  \def\tabH@lastxpos{\the\pdflastxpos}%
  \def\tabH@lastypos{\the\pdflastypos}%

  % The .aux file contains three commands:
  % \tabH@aux@vrule{tabular id}{x position}
  % \tabH@aux@vstart{tabular id}{row id}{x position}{y position}
  % \tabH@aux@vstop{y position}
  %
  \AtBeginDocument{%
    % The .aux files are read the first time before
    % \AtBeginDocument and later at \end{document}.
    % \tabH@aux@done is a marker to distinguish
    % between these two readings. Only in the first
    % case we need the \tabH@aux@... commands.
    \let\tabH@aux@done\@empty
    \if@filesw
      \immediate\write\@mainaux{%
        \@percentchar\@percentchar BeginProlog: tabularht%
      }%
      % items in the aux file are executed,
      % if tabularht is loaded
      % and during the aux file read at \begin{document} only
      \immediate\write\@mainaux{%
        \detokenize{%
          % the \tabH@aux@... commands are needed only if
          % tabularht is loaded with driver pdftex.
          \@ifundefined{tabH@aux@vrule}\@secondoftwo\@firstofone
          {%
            % disable commands except for the first .aux files reading
            \@ifundefined{tabH@aux@done}\@gobble\@firstofone
          }%
          {%
            \let\tabH@aux@vrule\@gobbletwo
            \let\tabH@aux@vstart\@gobblefour
            \let\tabH@aux@vstop\@gobble
          }%
        }%
      }%
      \immediate\write\@mainaux{%
        \@percentchar\@percentchar EndProlog: tabularht%
      }%
    \fi
  }%

  % the x positions of vrules are stored in
  % \tabH@<tabcount>list with distinct values
  \protected\def\tabH@aux@vrule#1#2{%
    \@ifundefined{tabH@#1list}{%
      \expandafter\xdef\csname tabH@#1list\endcsname{%
        \noexpand\do{#2}%
      }%
    }{%
      \begingroup
        \def\x{#2}%
        \let\y\@undefined
        \let\do\tabH@do@add
        \expandafter\xdef\csname tabH@#1list\endcsname{%
          \csname tabH@#1list\endcsname\@empty
          \ifx\y\@undefined
            \noexpand\do{\x}%
          \fi
        }%
      \endgroup
    }%
  }%
  \def\tabH@do@add#1{%
    \ifx\y\@undefined
      \ifnum#1<\x\space
      \else
        \expandafter\ifx\csname y\endcsname\relax\fi
        \ifnum#1>\x\space
          \noexpand\do{\x}%
        \fi
      \fi
    \fi
    \noexpand\do{#1}%
  }%

  \def\tabH@vspace@start#1{%
    \if@filesw
      \stepcounter{tabH@unique}%
      \edef\tabH@currentrow{\the\c@tabH@unique}%
      \pdfsavepos
      \protected@write\@auxout{%
        \let\tabH@lastxpos\relax
        \let\tabH@lastypos\relax
      }{%
        \tabH@aux@vstart{\tabH@currenttab}{\tabH@currentrow}%
                        {\tabH@lastxpos}{\tabH@lastypos}%
      }%
    \fi
    \begingroup
      \edef\a{tabH@\tabH@currenttab row\tabH@currentrow}%
      \expandafter\let\expandafter\x\csname\a x\endcsname
      \ifx\x\relax
      \else
        \expandafter\let\expandafter\y\csname\a y\endcsname
        \expandafter\let\expandafter\l
            \csname tabH@\tabH@currenttab list\endcsname
        \ifx\l\relax
        \else
          \def\f{#1}%
          \ifx\f\@empty
            \let\do\tabH@do@set
          \else
            \count@=\z@
            \let\do\tabH@do@filter
          \fi
          \setbox\z@=\hbox{\l}%
          \wd\z@=\z@
          \dp\z@=\z@
          \copy\z@
        \fi
      \fi
    \endgroup
  }%
  \def\tabH@vspace@stop{%
    \if@filesw
      \pdfsavepos
      \protected@write\@auxout{%
        \let\tabH@lastypos\relax
      }{%
        \tabH@aux@vstop{\tabH@lastypos}%
      }%
    \fi
  }%
  \def\tabH@do@set#1{%
    \hbox to \z@{%
      \hskip \dimexpr #1sp - \x sp\relax
      \vrule \@width\arrayrulewidth
             \@depth\dimexpr \y sp\relax
      \hss
    }%
  }%
  \def\tabH@do@filter{%
    \@tempswafalse
    \advance\count@\@ne
    \@for\e:=\f\do{%
      \ifnum\e=\count@
        \@tempswatrue
      \fi
    }%
    \if@tempswa
      \expandafter\tabH@do@set
    \else
      \expandafter\@gobble
    \fi
  }%

  \protected\def\tabH@aux@vstart#1#2#3#4{%
    \def\tabH@current@vstart{{#1}{#2}{#3}{#4}}%
  }%
  \protected\def\tabH@aux@vstop{%
    \expandafter\tabH@aux@v\tabH@current@vstart
  }%
  \def\tabH@aux@v#1#2#3#4#5{%
    \expandafter\gdef\csname tabH@#1row#2x\endcsname{#3}%
    \expandafter\xdef\csname tabH@#1row#2y\endcsname{%
      \the\numexpr #4 - #5\relax
    }%
  }%

  \csname fi\endcsname
  \endinput

\fi
%    \end{macrocode}
%
% \subsection{DVI drivers}
%
%    \begin{macrocode}
\ifx\tabH@driver\@empty
  \PackageError{tabularht}{%
    Missing DVI driver, option `vlines' disabled%
  }{%
    Supported DVI drivers: dvips.%
  }%
  \expandafter\endinput
\fi

\def\tabH@driver@dvips{%
  \def\tabH@literalps##1{\special{ps:SDict begin ##1 end}}%
  \def\tabH@headerps##1{\special{! ##1}}%
}

\@onelevel@sanitize\tabH@driver
\@ifundefined{tabH@driver@\tabH@driver}{%
  \PackageError{tabularht}{%
    Unsupported driver `\tabH@driver'%
  }{%
    Supported DVI drivers: dvips.%
  }%
  \endinput
}{}

\begingroup
  \let\on@line\@empty
  \PackageInfo{tabularht}{%
    Using driver `\tabH@driver'%
  }%
\endgroup
\csname tabH@driver@\tabH@driver\endcsname

\protected\def\tabH@vrule#1#2\vrule#3\arrayrulewidth{%
  #2% \fi or empty
  % hack to get rid of maxdrift rounding of dvips,
  % thus simulate a large motion
  \kern1in\relax
  \tabH@literalps{%
    #1 tabH.vrule %
    Resolution neg 0 translate%
  }%
  \vrule#3\arrayrulewidth
  \tabH@literalps{Resolution 0 translate}%
  \kern-1in\relax
}

\def\tabH@vspace@start#1{%
  \begingroup
    \let\y\@empty
    \@for\x:=#1\do{%
      \ifx\y\@empty
        \edef\y{\x}%
      \else
        \edef\y{\y\space\x}%
      \fi
    }%
    \tabH@literalps{\tabH@currenttab[\y]currentpoint exch pop}%
  \endgroup
}
\def\tabH@vspace@stop{%
  \tabH@literalps{%
    currentpoint exch pop %
    \number\dimexpr\arrayrulewidth\relax\space
    tabH.vspace%
  }%
}

\tabH@headerps{%
  userdict begin%
    /tabH.list 10 dict def%
    /tabH.job [] def %
  end%
  /tabH.vrule{%
    10 string cvs cvn dup tabH.list exch known{%
      tabH.list exch dup [ exch tabH.list exch get %
      currentpoint pop round exch true exch{%
        % tabH.list key [ ... x true i
        % tabH.list key [ ... false i
        exch{%
          % ... [ ... x i
          2 copy lt{false}{%
            2 copy eq{pop false}{exch true}ifelse%
          }ifelse%
        }{false}ifelse%
      }forall %
      pop%
      ]put%
    }{%
      tabH.list exch[currentpoint pop round]put%
    }ifelse%
  }bind def%
  % <tab num> <cols array> <ytop> <ybottom> <rulewidth[sp]>
  /tabH.vspace{%
    userdict begin %
      10 dict dup begin %
        exch 65536 div Resolution mul 72.27 div %
        % dvips uses a poor man's ceil function
        % see dopage.c before "drawrule": (int)(... + 0.9999999)
        0.9999999 add truncate%
        /rulewidth exch def %
        exch/ybottom exch def %
        exch/ytop exch def %
        exch/cols exch def %
        exch/tabkey exch 10 string cvs cvn def %
      end%
      /tabH.job exch[exch userdict/tabH.job get aload pop]def %
    end%
  }bind def %
  % Now we do the work at the end of the page.
  % Unhappily "eop-hook" cannot be used, because "eop"
  % executes "restore" before, so that all data are lost.
  TeXDict begin%
    /eop%
    [%
      {%
        tabH.job{%
          begin%
           /colarray %
             tabH.list tabkey known{tabH.list tabkey get}{[]}ifelse %
           def %
           cols length 0 eq not{%
             /colarray[%
               cols{1 sub %
                 dup 0 lt{pop}{%
                   dup colarray length ge{pop}{%
                     colarray exch get%
                   }ifelse%
                 }ifelse%
               }forall%
             ]def%
           }if %
           colarray{%
             % (rulewidth) == rulewidth == % debug
             Resolution sub %
             ytop rulewidth ytop ybottom sub v%
           }forall %
          end%
        }forall%
        % tabH.list{== ==}forall % debug
      }bind aload pop %
      TeXDict /eop get aload pop%
    ]cvx def %
  end%
}
%    \end{macrocode}
%
%    \begin{macrocode}
%</package>
%    \end{macrocode}
%
% \section{Installation}
%
% \subsection{Download}
%
% \paragraph{Package.} This package is available on
% CTAN\footnote{\CTANpkg{tabularht}}:
% \begin{description}
% \item[\CTAN{macros/latex/contrib/oberdiek/tabularht.dtx}] The source file.
% \item[\CTAN{macros/latex/contrib/oberdiek/tabularht.pdf}] Documentation.
% \end{description}
%
%
% \paragraph{Bundle.} All the packages of the bundle `oberdiek'
% are also available in a TDS compliant ZIP archive. There
% the packages are already unpacked and the documentation files
% are generated. The files and directories obey the TDS standard.
% \begin{description}
% \item[\CTANinstall{install/macros/latex/contrib/oberdiek.tds.zip}]
% \end{description}
% \emph{TDS} refers to the standard ``A Directory Structure
% for \TeX\ Files'' (\CTAN{tds/tds.pdf}). Directories
% with \xfile{texmf} in their name are usually organized this way.
%
% \subsection{Bundle installation}
%
% \paragraph{Unpacking.} Unpack the \xfile{oberdiek.tds.zip} in the
% TDS tree (also known as \xfile{texmf} tree) of your choice.
% Example (linux):
% \begin{quote}
%   |unzip oberdiek.tds.zip -d ~/texmf|
% \end{quote}
%
% \paragraph{Script installation.}
% Check the directory \xfile{TDS:scripts/oberdiek/} for
% scripts that need further installation steps.
%
% \subsection{Package installation}
%
% \paragraph{Unpacking.} The \xfile{.dtx} file is a self-extracting
% \docstrip\ archive. The files are extracted by running the
% \xfile{.dtx} through \plainTeX:
% \begin{quote}
%   \verb|tex tabularht.dtx|
% \end{quote}
%
% \paragraph{TDS.} Now the different files must be moved into
% the different directories in your installation TDS tree
% (also known as \xfile{texmf} tree):
% \begin{quote}
% \def\t{^^A
% \begin{tabular}{@{}>{\ttfamily}l@{ $\rightarrow$ }>{\ttfamily}l@{}}
%   tabularht.sty & tex/latex/oberdiek/tabularht.sty\\
%   tabularht.pdf & doc/latex/oberdiek/tabularht.pdf\\
%   tabularht-example1.tex & doc/latex/oberdiek/tabularht-example1.tex\\
%   tabularht-example2.tex & doc/latex/oberdiek/tabularht-example2.tex\\
%   tabularht.dtx & source/latex/oberdiek/tabularht.dtx\\
% \end{tabular}^^A
% }^^A
% \sbox0{\t}^^A
% \ifdim\wd0>\linewidth
%   \begingroup
%     \advance\linewidth by\leftmargin
%     \advance\linewidth by\rightmargin
%   \edef\x{\endgroup
%     \def\noexpand\lw{\the\linewidth}^^A
%   }\x
%   \def\lwbox{^^A
%     \leavevmode
%     \hbox to \linewidth{^^A
%       \kern-\leftmargin\relax
%       \hss
%       \usebox0
%       \hss
%       \kern-\rightmargin\relax
%     }^^A
%   }^^A
%   \ifdim\wd0>\lw
%     \sbox0{\small\t}^^A
%     \ifdim\wd0>\linewidth
%       \ifdim\wd0>\lw
%         \sbox0{\footnotesize\t}^^A
%         \ifdim\wd0>\linewidth
%           \ifdim\wd0>\lw
%             \sbox0{\scriptsize\t}^^A
%             \ifdim\wd0>\linewidth
%               \ifdim\wd0>\lw
%                 \sbox0{\tiny\t}^^A
%                 \ifdim\wd0>\linewidth
%                   \lwbox
%                 \else
%                   \usebox0
%                 \fi
%               \else
%                 \lwbox
%               \fi
%             \else
%               \usebox0
%             \fi
%           \else
%             \lwbox
%           \fi
%         \else
%           \usebox0
%         \fi
%       \else
%         \lwbox
%       \fi
%     \else
%       \usebox0
%     \fi
%   \else
%     \lwbox
%   \fi
% \else
%   \usebox0
% \fi
% \end{quote}
% If you have a \xfile{docstrip.cfg} that configures and enables \docstrip's
% TDS installing feature, then some files can already be in the right
% place, see the documentation of \docstrip.
%
% \subsection{Refresh file name databases}
%
% If your \TeX~distribution
% (\teTeX, \mikTeX, \dots) relies on file name databases, you must refresh
% these. For example, \teTeX\ users run \verb|texhash| or
% \verb|mktexlsr|.
%
% \subsection{Some details for the interested}
%
% \paragraph{Unpacking with \LaTeX.}
% The \xfile{.dtx} chooses its action depending on the format:
% \begin{description}
% \item[\plainTeX:] Run \docstrip\ and extract the files.
% \item[\LaTeX:] Generate the documentation.
% \end{description}
% If you insist on using \LaTeX\ for \docstrip\ (really,
% \docstrip\ does not need \LaTeX), then inform the autodetect routine
% about your intention:
% \begin{quote}
%   \verb|latex \let\install=y% \iffalse meta-comment
%
% File: tabularht.dtx
% Version: 2016/05/16 v2.6
% Info: Tabular with height specified
%
% Copyright (C) 2005-2007 by
%    Heiko Oberdiek <heiko.oberdiek at googlemail.com>
%    2016
%    https://github.com/ho-tex/oberdiek/issues
%
% This work may be distributed and/or modified under the
% conditions of the LaTeX Project Public License, either
% version 1.3c of this license or (at your option) any later
% version. This version of this license is in
%    https://www.latex-project.org/lppl/lppl-1-3c.txt
% and the latest version of this license is in
%    https://www.latex-project.org/lppl.txt
% and version 1.3 or later is part of all distributions of
% LaTeX version 2005/12/01 or later.
%
% This work has the LPPL maintenance status "maintained".
%
% The Current Maintainers of this work are
% Heiko Oberdiek and the Oberdiek Package Support Group
% https://github.com/ho-tex/oberdiek/issues
%
% This work consists of the main source file tabularht.dtx
% and the derived files
%    tabularht.sty, tabularht.pdf, tabularht.ins, tabularht.drv,
%    tabularht-example1.tex, tabularht-example2.tex.
%
% Distribution:
%    CTAN:macros/latex/contrib/oberdiek/tabularht.dtx
%    CTAN:macros/latex/contrib/oberdiek/tabularht.pdf
%
% Unpacking:
%    (a) If tabularht.ins is present:
%           tex tabularht.ins
%    (b) Without tabularht.ins:
%           tex tabularht.dtx
%    (c) If you insist on using LaTeX
%           latex \let\install=y\input{tabularht.dtx}
%        (quote the arguments according to the demands of your shell)
%
% Documentation:
%    (a) If tabularht.drv is present:
%           latex tabularht.drv
%    (b) Without tabularht.drv:
%           latex tabularht.dtx; ...
%    The class ltxdoc loads the configuration file ltxdoc.cfg
%    if available. Here you can specify further options, e.g.
%    use A4 as paper format:
%       \PassOptionsToClass{a4paper}{article}
%
%    Programm calls to get the documentation (example):
%       pdflatex tabularht.dtx
%       makeindex -s gind.ist tabularht.idx
%       pdflatex tabularht.dtx
%       makeindex -s gind.ist tabularht.idx
%       pdflatex tabularht.dtx
%
% Installation:
%    TDS:tex/latex/oberdiek/tabularht.sty
%    TDS:doc/latex/oberdiek/tabularht.pdf
%    TDS:doc/latex/oberdiek/tabularht-example1.tex
%    TDS:doc/latex/oberdiek/tabularht-example2.tex
%    TDS:source/latex/oberdiek/tabularht.dtx
%
%<*ignore>
\begingroup
  \catcode123=1 %
  \catcode125=2 %
  \def\x{LaTeX2e}%
\expandafter\endgroup
\ifcase 0\ifx\install y1\fi\expandafter
         \ifx\csname processbatchFile\endcsname\relax\else1\fi
         \ifx\fmtname\x\else 1\fi\relax
\else\csname fi\endcsname
%</ignore>
%<*install>
\input docstrip.tex
\Msg{************************************************************************}
\Msg{* Installation}
\Msg{* Package: tabularht 2016/05/16 v2.6 Tabular with height specified (HO)}
\Msg{************************************************************************}

\keepsilent
\askforoverwritefalse

\let\MetaPrefix\relax
\preamble

This is a generated file.

Project: tabularht
Version: 2016/05/16 v2.6

Copyright (C) 2005-2007 by
   Heiko Oberdiek <heiko.oberdiek at googlemail.com>

This work may be distributed and/or modified under the
conditions of the LaTeX Project Public License, either
version 1.3c of this license or (at your option) any later
version. This version of this license is in
   https://www.latex-project.org/lppl/lppl-1-3c.txt
and the latest version of this license is in
   https://www.latex-project.org/lppl.txt
and version 1.3 or later is part of all distributions of
LaTeX version 2005/12/01 or later.

This work has the LPPL maintenance status "maintained".

The Current Maintainers of this work are
Heiko Oberdiek and the Oberdiek Package Support Group
https://github.com/ho-tex/oberdiek/issues


This work consists of the main source file tabularht.dtx
and the derived files
   tabularht.sty, tabularht.pdf, tabularht.ins, tabularht.drv,
   tabularht-example1.tex, tabularht-example2.tex.

\endpreamble
\let\MetaPrefix\DoubleperCent

\generate{%
  \file{tabularht.ins}{\from{tabularht.dtx}{install}}%
  \file{tabularht.drv}{\from{tabularht.dtx}{driver}}%
  \usedir{tex/latex/oberdiek}%
  \file{tabularht.sty}{\from{tabularht.dtx}{package}}%
  \usedir{doc/latex/oberdiek}%
  \file{tabularht-example1.tex}{\from{tabularht.dtx}{example1}}%
  \file{tabularht-example2.tex}{\from{tabularht.dtx}{example2}}%
  \nopreamble
  \nopostamble
%  \usedir{source/latex/oberdiek/catalogue}%
%  \file{tabularht.xml}{\from{tabularht.dtx}{catalogue}}%
}

\catcode32=13\relax% active space
\let =\space%
\Msg{************************************************************************}
\Msg{*}
\Msg{* To finish the installation you have to move the following}
\Msg{* file into a directory searched by TeX:}
\Msg{*}
\Msg{*     tabularht.sty}
\Msg{*}
\Msg{* To produce the documentation run the file `tabularht.drv'}
\Msg{* through LaTeX.}
\Msg{*}
\Msg{* Happy TeXing!}
\Msg{*}
\Msg{************************************************************************}

\endbatchfile
%</install>
%<*ignore>
\fi
%</ignore>
%<*driver>
\NeedsTeXFormat{LaTeX2e}
\ProvidesFile{tabularht.drv}%
  [2016/05/16 v2.6 Tabular with height specified (HO)]%
\documentclass{ltxdoc}
\usepackage{holtxdoc}[2011/11/22]
\begin{document}
  \DocInput{tabularht.dtx}%
\end{document}
%</driver>
% \fi
%
%
% \CharacterTable
%  {Upper-case    \A\B\C\D\E\F\G\H\I\J\K\L\M\N\O\P\Q\R\S\T\U\V\W\X\Y\Z
%   Lower-case    \a\b\c\d\e\f\g\h\i\j\k\l\m\n\o\p\q\r\s\t\u\v\w\x\y\z
%   Digits        \0\1\2\3\4\5\6\7\8\9
%   Exclamation   \!     Double quote  \"     Hash (number) \#
%   Dollar        \$     Percent       \%     Ampersand     \&
%   Acute accent  \'     Left paren    \(     Right paren   \)
%   Asterisk      \*     Plus          \+     Comma         \,
%   Minus         \-     Point         \.     Solidus       \/
%   Colon         \:     Semicolon     \;     Less than     \<
%   Equals        \=     Greater than  \>     Question mark \?
%   Commercial at \@     Left bracket  \[     Backslash     \\
%   Right bracket \]     Circumflex    \^     Underscore    \_
%   Grave accent  \`     Left brace    \{     Vertical bar  \|
%   Right brace   \}     Tilde         \~}
%
% \GetFileInfo{tabularht.drv}
%
% \title{The \xpackage{tabularht} package}
% \date{2016/05/16 v2.6}
% \author{Heiko Oberdiek\thanks
% {Please report any issues at \url{https://github.com/ho-tex/oberdiek/issues}}}
%
% \maketitle
%
% \begin{abstract}
% This package defines some environments that adds
% a height specification to tabular and array.
% \end{abstract}
%
% \tableofcontents
%
% \section{Usage}
%
% \begin{quote}
%   |\usepackage{tabularht}|
% \end{quote}
% The package provides the following environments
% that extend the tabular/array environment by
% a height specification as first argument:
% \begin{itemize}
% \item |tabularht|, |tabularht*|
% \item |arrayht|
% \item |tabularhtx|
%       (if package \xpackage{tabularx} is loaded)
% \end{itemize}
% The height argument allows a length specification,
% package \xpackage{calc} is supported if used. This means,
% the tabular will have the specified height.
% You can also use the prefixes |to=| and |spread=|.
% |to=| is the default, |spread=| means, the natural
% height of the tabular box is changed by the length
% after |spread=|.
%
% Examples:
% \begin{quote}
% \begin{tabular}{@{}l@{ $\rightarrow$ }l@{}}
%   |\begin{tabularht}{1in}|
%   & height is 1in\\
%   |\begin{tabularht}{to=1in}|
%   & height is 1in\\
%   |\begin{tabularht}{spread=0pt}|
%   & natural height, same as |\begin{tabular}|\\
%   |\begin{tabularht}{spread=1in}|
%   & natural height increased by 1in
% \end{tabular}
% \end{quote}
%
% Hint:  See also package \xpackage{tabularkv}, it provides an interface,
%        where most parameters for the environments can be given
%        by key-value pairs.
%
% \begin{declcs}{interrowspace}\,|{|\dots|}|
% \end{declcs}
% Adds space between table rows. It is essentially
% the same as |\noalign{\vspace{...}}|.
% \begin{declcs}{interrowfill}
% \end{declcs}
% Short for |\interrowspace{\fill}|
% \begin{declcs}{interrowstart}\,\dots\,\cs{interrowstop}
% \end{declcs}
% Marker commands, useful for option \xoption{vlines}.
%
% \subsection{Option \xoption{vlines}}
%
% Warning: This stuff is experimental.
%
% \vspace{2ex}
% \noindent
% Vertical lines are interrupted, if space is inserted
% in \cs{noalign}, \cs{interrowspace}, \cs{addlinespace}
% (\xpackage{booktabs}),
% between double \cs{hline}s. This option tries to detect
% and add the vertical lines. The lines in a tabular
% with \xpackage{tabularht} support (environments of this package)
% are numbered from left to right. The gap that is
% controlled by \cs{interrowspace} or inbetween \cs{interrowstart}
% and \cs{interrowstop} is then filled with the detected
% vertical lines.
%
% If only a limited selection of the lines should be
% drawn, the commands know an optional argument with a
% list of line numbers, e.g.
% \begin{quote}
%\begin{verbatim}
%\begin{tabularht}{50mm}{|l|l|}
%  Hello & World\\
%  \interrowfill[1,3]
%  Foo & Bar
%\end{tabularht}
%\end{verbatim}
% \end{quote}
% There are three lines, but the middle line is not drawn
% in the gap between the first and second row.
% Zero can be used to suppress all lines:
% \begin{quote}
% |\interrowspace[0]{10mm}|
% \end{quote}
% The syntax of the commands with the optional argument
% with the line number list \meta{list}. \meta{list} is a comma
% separated list of numbers, \meta{height} means the
% height specification described above with the optional
% prefixes |to=| or |spread=|.
% \begin{quote}
%   \cs{interrowspace}\,|[|\meta{list}|]|\,|{|\meta{height}|}|\\
%   \cs{interrowfill}\,|[|\meta{list}|]|\\
%   \cs{interrowstart}\,|[|\meta{list}|]|\,\dots\,\cs{interrowstop}
% \end{quote}
%
% Option \xoption{vlines} is driver dependent and uses \eTeX\ features.
% \begin{description}
% \item[\xoption{pdftex}:] \pdfTeX\ in PDF mode.
%   Here the positions of the lines are written with the
%   help of the \cs{pdfsavepos} feature into the \xfile{.aux} file(s).
%   Therefore you need two LaTeX runs to get the lines.
% \item[\xoption{dvips}:]
%   Here, PostScript's currentpoint it used to get
%   the line positions. The lines are then drawn at the end
%   of the page. Thus one \LaTeX/dvips run is sufficient
%   for this option.
% \item[Other drivers:]\mbox{}
%   \begin{description}
%   \item[PostScript drivers:] probably possible, an end of page
%     hook would be nice.
%   \item[\VTeX:] with GeX (PostScript interpreter) probably possible.
%   \item[dvipdfm:] no idea. The big problem is, how to get the
%     current position?
%   \end{description}
% \end{description}
%
% \subsection{Limitations}
%
% \begin{itemize}
% \item
%   Vertical lines are interrupted by |\noalign{\vfill}|.
% \end{itemize}
%
% \subsection{Compatibility}
%
% \begin{itemize}
% \item
%   \xpackage{array}, \xpackage{delarray}, \xpackage{tabularx}
%   are supported.
% \item There can be problems with packages that
%   redefine \cs{@array} (or \cs{@@array}, \cs{@tabarray}) and
%   \cs{@arrayrule} (for option \xoption{vlines}).
% \item \xpackage{colortbl}: it should at least work, but there isn't
%   support for filling the gaps with color, neither
%   the rules nor the backgrounds.
% \end{itemize}
%
% \subsection{Examples}
%
% \subsubsection{Example 1}
%
%    \begin{macrocode}
%<*example1>
\documentclass{article}
\usepackage{tabularht}

\begin{document}
\fbox{%
  \begin{tabularht*}{1in}{4in}{@{}l@{\extracolsep{\fill}}r@{}}%
    upper left corner & upper right corner\\%
    \noalign{\vfill}%
    \multicolumn{2}{@{}c@{}}{bounding box}\\%
    \noalign{\vfill}%
    lower left corner & lower right corner\\%
  \end{tabularht*}%
}
\end{document}
%</example1>
%    \end{macrocode}
%
% \subsubsection{Example 2}
%
%    \begin{macrocode}
%<*example2>
\documentclass{article}
\usepackage{booktabs}
\usepackage[dvips,vlines]{tabularht}

\begin{document}

\begin{tabularht}{spread=0pt}{|l|l|}
  \hline
  First&Line\\%
  \hline
\interrowstart
  \addlinespace[10mm]%
\interrowstop
  \hline
  Second&Line\\%
\interrowstart
  \hline
  \hline
\interrowstop
  Third&Line\\%
  \hline
\interrowspace{10mm}
  \hline
  Fourth&Line\\%
  \hline
\end{tabularht}

\end{document}
%</example2>
%    \end{macrocode}
%
% \StopEventually{
% }
%
% \section{Implementation}
%
%    \begin{macrocode}
%<*package>
%    \end{macrocode}
%    Package identification.
%    \begin{macrocode}
\NeedsTeXFormat{LaTeX2e}
\ProvidesPackage{tabularht}%
  [2016/05/16 v2.6 Tabular with height specified (HO)]
%    \end{macrocode}
%
% \subsection{Environments}
%
%    \begin{macrocode}
\let\@toarrayheight\@empty
\let\tabH@array@init\@empty

\toks@={%
  \begingroup
    \long\def\x#1\vcenter\fi\fi\bgroup#2\@sharp#3#4\@nil{%
      \endgroup
      \gdef\@array[##1]##2{%
        \tabH@array@init
        #1%
        \vcenter\fi\fi
        \@toarrayheight
        \bgroup
        \let\@toarrayheight\@empty
        #2\@sharp###3#4%
      }%
    }%
  \expandafter\x\@array[#1]{#2}\@nil % hash-ok
}
\edef\tabH@patch@array{\the\toks@}
\def\tabH@patch@@array{%
  \ifx\@array\@@array
    \def\reserved@a{\let\@@array\@array}%
  \else
    \let\reserved@a\relax
  \fi
  \tabH@patch@array
  \reserved@a
}
\tabH@patch@@array

\@ifpackageloaded{array}{}{%
  \AtBeginDocument{%
    \@ifpackageloaded{array}{%
      \tabH@patch@@array
    }{}%
  }%
}

\def\tabH@setheight#1{%
  \tabH@@setheight#1==\@nil
}
\def\tabH@@setheight#1=#2=#3\@nil{%
  \ifx\\#2#3\\%
    \setlength{\dimen@}{#1}%
    \edef\@toarrayheight{to\the\dimen@}%
  \else
    \edef\tabH@temp{\zap@space#1 \@empty}%
    \ifx\tabH@temp\tabH@to
    \else
      \ifx\tabH@temp\tabH@spread
      \else
        \PackageError{tabularht}{%
          Unknown height specifier %
          `\expandafter\strip@prefix\meaning\tabH@temp'%
        }{%
          The height dimension for tabular height can be prefixed%
          \MessageBreak
          with `to=' or `spread=', default is `to='.%
        }%
        \let\tabH@temp\tabH@to
      \fi
    \fi
    \setlength{\dimen@}{#2}%
    \edef\@toarrayheight{\tabH@temp\the\dimen@}%
  \fi
}
\def\tabH@to{to}
\def\tabH@spread{spread}
%    \end{macrocode}
%    First argument is the height of the table,
%    then the orignal arguments for tabular follow.
%    \begin{macrocode}
\newenvironment{tabularht}[1]{%
  \tabH@setheight{#1}%
  \tabular
}{%
  \endtabular
}

\newenvironment{tabularht*}[1]{%
  \tabH@setheight{#1}%
  \@nameuse{tabular*}%
}{%
  \@nameuse{endtabular*}%
}

\newenvironment{tabularhtx}[1]{%
  \tabH@setheight{#1}%
  \tabularx
}{%
  \endtabularx
}

\newenvironment{arrayht}[1]{%
  \tabH@setheight{#1}%
  \array
}{%
  \endarray
}

\def\interrowspace{%
  \noalign\bgroup
    \tabH@interrowspace
}
\newcommand*{\tabH@interrowspace}[2][]{%
    \tabH@vspace{#1}{#2}%
  \egroup
}
\def\interrowfill{%
  \noalign\bgroup
    \tabH@interrowfill
}
\newcommand*{\tabH@interrowfill}[1][]{%
    \tabH@vspace{#1}{\fill}%
  \egroup
}
\def\tabH@vspace#1#2{%
  \tabH@vspace@start{#1}%
  \vspace{#2}%
  \tabH@vspace@stop
}
\let\tabH@vspace@start\@gobble
\let\tabH@vspace@stop\@empty

\newcommand*{\interrowstart}{%
  \noalign\bgroup
    \tabH@interrowstart
}
\newcommand*{\tabH@interrowstart}[1][]{%
    \tabH@vspace@start{#1}%
  \egroup
}
\newcommand*{\interrowstop}{%
  \noalign{\tabH@vspace@stop}%
}
%    \end{macrocode}
%
% \subsection{Options}
%
%    \begin{macrocode}
\providecommand*{\tabH@driver}{}

\DeclareOption{vlines}{%
  \let\tabH@temp\relax
}
\DeclareOption{pdftex}{}
\DeclareOption{dvips}{%
  \def\tabH@driver{dvips}%
}
\ProcessOptions*\relax

\ifx\tabH@temp\relax
\else
  \expandafter\endinput
\fi

\begingroup
  \@ifundefined{eTeXversion}{%
    \PackageError{tabularht}{%
      Option `vlines' requires eTeX%
    }{%
      Use of eTeX is recommended for LaTeX, see ltnews16.%
    }%
    \endgroup
    \endinput
  }{}%
\endgroup
%    \end{macrocode}
%
% \subsection{Option vlines, driver independent stuff}
%
%    \begin{macrocode}
\begingroup
  \let\@addtoreset\@gobbletwo
  \newcounter{tabH@unique}%
\endgroup
\let\tabH@currenttab\@empty

\def\tabH@array@init{%
  \ifx\@toarrayheight\@empty
    % ignore vertical lines of nested tabular environments
    \let\tabH@currenttab\@empty
  \else
    \stepcounter{tabH@unique}%
    \edef\tabH@currenttab{\the\c@tabH@unique}%
  \fi
}

\renewcommand*{\@arrayrule}{%
  \@addtopreamble{%
    \hskip -.5\arrayrulewidth
    \ifx\tabH@currenttab\@empty
    \else
      \tabH@vrule{\tabH@currenttab}%
    \fi
    \begingroup
      \expandafter\ifx\csname CT@arc@\endcsname\relax
      \else
        \expandafter\CT@arc@
      \fi
      \vline
    \endgroup
    \hskip -.5\arrayrulewidth
  }%
}
\let\tabH@arrayrule\@arrayrule
\AtBeginDocument{%
  \@ifpackageloaded{colortbl}{%
    \let\@arrayrule\tabH@arrayrule
  }{}%
}

\let\tabH@vrule\@gobble
%    \end{macrocode}
%
% \subsection{Driver pdftex}
%
%    \begin{macrocode}
\RequirePackage{ifpdf}
\ifpdf
  \begingroup
    \@ifundefined{pdfsavepos}{%
      \PackageError{tabularht}{%
        Your pdfTeX is too old%
      }{%
        \string\pdfsavepos\space is missing.%
      }%
      \endgroup
      \csname fi\endcsname
      \endinput
    }{}%

    \let\on@line\@empty
    \PackageInfo{tabularht}{%
      Using driver `pdftex' because of pdfTeX in PDF mode%
    }%
  \endgroup

  \protected\def\tabH@vrule#1{%
    \if@filesw
      \pdfsavepos
      \protected@write\@auxout{%
        \let\tabH@lastxpos\relax
      }{%
        \tabH@aux@vrule{#1}{\tabH@lastxpos}%
      }%
    \fi
  }%

  \def\tabH@lastxpos{\the\pdflastxpos}%
  \def\tabH@lastypos{\the\pdflastypos}%

  % The .aux file contains three commands:
  % \tabH@aux@vrule{tabular id}{x position}
  % \tabH@aux@vstart{tabular id}{row id}{x position}{y position}
  % \tabH@aux@vstop{y position}
  %
  \AtBeginDocument{%
    % The .aux files are read the first time before
    % \AtBeginDocument and later at \end{document}.
    % \tabH@aux@done is a marker to distinguish
    % between these two readings. Only in the first
    % case we need the \tabH@aux@... commands.
    \let\tabH@aux@done\@empty
    \if@filesw
      \immediate\write\@mainaux{%
        \@percentchar\@percentchar BeginProlog: tabularht%
      }%
      % items in the aux file are executed,
      % if tabularht is loaded
      % and during the aux file read at \begin{document} only
      \immediate\write\@mainaux{%
        \detokenize{%
          % the \tabH@aux@... commands are needed only if
          % tabularht is loaded with driver pdftex.
          \@ifundefined{tabH@aux@vrule}\@secondoftwo\@firstofone
          {%
            % disable commands except for the first .aux files reading
            \@ifundefined{tabH@aux@done}\@gobble\@firstofone
          }%
          {%
            \let\tabH@aux@vrule\@gobbletwo
            \let\tabH@aux@vstart\@gobblefour
            \let\tabH@aux@vstop\@gobble
          }%
        }%
      }%
      \immediate\write\@mainaux{%
        \@percentchar\@percentchar EndProlog: tabularht%
      }%
    \fi
  }%

  % the x positions of vrules are stored in
  % \tabH@<tabcount>list with distinct values
  \protected\def\tabH@aux@vrule#1#2{%
    \@ifundefined{tabH@#1list}{%
      \expandafter\xdef\csname tabH@#1list\endcsname{%
        \noexpand\do{#2}%
      }%
    }{%
      \begingroup
        \def\x{#2}%
        \let\y\@undefined
        \let\do\tabH@do@add
        \expandafter\xdef\csname tabH@#1list\endcsname{%
          \csname tabH@#1list\endcsname\@empty
          \ifx\y\@undefined
            \noexpand\do{\x}%
          \fi
        }%
      \endgroup
    }%
  }%
  \def\tabH@do@add#1{%
    \ifx\y\@undefined
      \ifnum#1<\x\space
      \else
        \expandafter\ifx\csname y\endcsname\relax\fi
        \ifnum#1>\x\space
          \noexpand\do{\x}%
        \fi
      \fi
    \fi
    \noexpand\do{#1}%
  }%

  \def\tabH@vspace@start#1{%
    \if@filesw
      \stepcounter{tabH@unique}%
      \edef\tabH@currentrow{\the\c@tabH@unique}%
      \pdfsavepos
      \protected@write\@auxout{%
        \let\tabH@lastxpos\relax
        \let\tabH@lastypos\relax
      }{%
        \tabH@aux@vstart{\tabH@currenttab}{\tabH@currentrow}%
                        {\tabH@lastxpos}{\tabH@lastypos}%
      }%
    \fi
    \begingroup
      \edef\a{tabH@\tabH@currenttab row\tabH@currentrow}%
      \expandafter\let\expandafter\x\csname\a x\endcsname
      \ifx\x\relax
      \else
        \expandafter\let\expandafter\y\csname\a y\endcsname
        \expandafter\let\expandafter\l
            \csname tabH@\tabH@currenttab list\endcsname
        \ifx\l\relax
        \else
          \def\f{#1}%
          \ifx\f\@empty
            \let\do\tabH@do@set
          \else
            \count@=\z@
            \let\do\tabH@do@filter
          \fi
          \setbox\z@=\hbox{\l}%
          \wd\z@=\z@
          \dp\z@=\z@
          \copy\z@
        \fi
      \fi
    \endgroup
  }%
  \def\tabH@vspace@stop{%
    \if@filesw
      \pdfsavepos
      \protected@write\@auxout{%
        \let\tabH@lastypos\relax
      }{%
        \tabH@aux@vstop{\tabH@lastypos}%
      }%
    \fi
  }%
  \def\tabH@do@set#1{%
    \hbox to \z@{%
      \hskip \dimexpr #1sp - \x sp\relax
      \vrule \@width\arrayrulewidth
             \@depth\dimexpr \y sp\relax
      \hss
    }%
  }%
  \def\tabH@do@filter{%
    \@tempswafalse
    \advance\count@\@ne
    \@for\e:=\f\do{%
      \ifnum\e=\count@
        \@tempswatrue
      \fi
    }%
    \if@tempswa
      \expandafter\tabH@do@set
    \else
      \expandafter\@gobble
    \fi
  }%

  \protected\def\tabH@aux@vstart#1#2#3#4{%
    \def\tabH@current@vstart{{#1}{#2}{#3}{#4}}%
  }%
  \protected\def\tabH@aux@vstop{%
    \expandafter\tabH@aux@v\tabH@current@vstart
  }%
  \def\tabH@aux@v#1#2#3#4#5{%
    \expandafter\gdef\csname tabH@#1row#2x\endcsname{#3}%
    \expandafter\xdef\csname tabH@#1row#2y\endcsname{%
      \the\numexpr #4 - #5\relax
    }%
  }%

  \csname fi\endcsname
  \endinput

\fi
%    \end{macrocode}
%
% \subsection{DVI drivers}
%
%    \begin{macrocode}
\ifx\tabH@driver\@empty
  \PackageError{tabularht}{%
    Missing DVI driver, option `vlines' disabled%
  }{%
    Supported DVI drivers: dvips.%
  }%
  \expandafter\endinput
\fi

\def\tabH@driver@dvips{%
  \def\tabH@literalps##1{\special{ps:SDict begin ##1 end}}%
  \def\tabH@headerps##1{\special{! ##1}}%
}

\@onelevel@sanitize\tabH@driver
\@ifundefined{tabH@driver@\tabH@driver}{%
  \PackageError{tabularht}{%
    Unsupported driver `\tabH@driver'%
  }{%
    Supported DVI drivers: dvips.%
  }%
  \endinput
}{}

\begingroup
  \let\on@line\@empty
  \PackageInfo{tabularht}{%
    Using driver `\tabH@driver'%
  }%
\endgroup
\csname tabH@driver@\tabH@driver\endcsname

\protected\def\tabH@vrule#1#2\vrule#3\arrayrulewidth{%
  #2% \fi or empty
  % hack to get rid of maxdrift rounding of dvips,
  % thus simulate a large motion
  \kern1in\relax
  \tabH@literalps{%
    #1 tabH.vrule %
    Resolution neg 0 translate%
  }%
  \vrule#3\arrayrulewidth
  \tabH@literalps{Resolution 0 translate}%
  \kern-1in\relax
}

\def\tabH@vspace@start#1{%
  \begingroup
    \let\y\@empty
    \@for\x:=#1\do{%
      \ifx\y\@empty
        \edef\y{\x}%
      \else
        \edef\y{\y\space\x}%
      \fi
    }%
    \tabH@literalps{\tabH@currenttab[\y]currentpoint exch pop}%
  \endgroup
}
\def\tabH@vspace@stop{%
  \tabH@literalps{%
    currentpoint exch pop %
    \number\dimexpr\arrayrulewidth\relax\space
    tabH.vspace%
  }%
}

\tabH@headerps{%
  userdict begin%
    /tabH.list 10 dict def%
    /tabH.job [] def %
  end%
  /tabH.vrule{%
    10 string cvs cvn dup tabH.list exch known{%
      tabH.list exch dup [ exch tabH.list exch get %
      currentpoint pop round exch true exch{%
        % tabH.list key [ ... x true i
        % tabH.list key [ ... false i
        exch{%
          % ... [ ... x i
          2 copy lt{false}{%
            2 copy eq{pop false}{exch true}ifelse%
          }ifelse%
        }{false}ifelse%
      }forall %
      pop%
      ]put%
    }{%
      tabH.list exch[currentpoint pop round]put%
    }ifelse%
  }bind def%
  % <tab num> <cols array> <ytop> <ybottom> <rulewidth[sp]>
  /tabH.vspace{%
    userdict begin %
      10 dict dup begin %
        exch 65536 div Resolution mul 72.27 div %
        % dvips uses a poor man's ceil function
        % see dopage.c before "drawrule": (int)(... + 0.9999999)
        0.9999999 add truncate%
        /rulewidth exch def %
        exch/ybottom exch def %
        exch/ytop exch def %
        exch/cols exch def %
        exch/tabkey exch 10 string cvs cvn def %
      end%
      /tabH.job exch[exch userdict/tabH.job get aload pop]def %
    end%
  }bind def %
  % Now we do the work at the end of the page.
  % Unhappily "eop-hook" cannot be used, because "eop"
  % executes "restore" before, so that all data are lost.
  TeXDict begin%
    /eop%
    [%
      {%
        tabH.job{%
          begin%
           /colarray %
             tabH.list tabkey known{tabH.list tabkey get}{[]}ifelse %
           def %
           cols length 0 eq not{%
             /colarray[%
               cols{1 sub %
                 dup 0 lt{pop}{%
                   dup colarray length ge{pop}{%
                     colarray exch get%
                   }ifelse%
                 }ifelse%
               }forall%
             ]def%
           }if %
           colarray{%
             % (rulewidth) == rulewidth == % debug
             Resolution sub %
             ytop rulewidth ytop ybottom sub v%
           }forall %
          end%
        }forall%
        % tabH.list{== ==}forall % debug
      }bind aload pop %
      TeXDict /eop get aload pop%
    ]cvx def %
  end%
}
%    \end{macrocode}
%
%    \begin{macrocode}
%</package>
%    \end{macrocode}
%
% \section{Installation}
%
% \subsection{Download}
%
% \paragraph{Package.} This package is available on
% CTAN\footnote{\CTANpkg{tabularht}}:
% \begin{description}
% \item[\CTAN{macros/latex/contrib/oberdiek/tabularht.dtx}] The source file.
% \item[\CTAN{macros/latex/contrib/oberdiek/tabularht.pdf}] Documentation.
% \end{description}
%
%
% \paragraph{Bundle.} All the packages of the bundle `oberdiek'
% are also available in a TDS compliant ZIP archive. There
% the packages are already unpacked and the documentation files
% are generated. The files and directories obey the TDS standard.
% \begin{description}
% \item[\CTANinstall{install/macros/latex/contrib/oberdiek.tds.zip}]
% \end{description}
% \emph{TDS} refers to the standard ``A Directory Structure
% for \TeX\ Files'' (\CTAN{tds/tds.pdf}). Directories
% with \xfile{texmf} in their name are usually organized this way.
%
% \subsection{Bundle installation}
%
% \paragraph{Unpacking.} Unpack the \xfile{oberdiek.tds.zip} in the
% TDS tree (also known as \xfile{texmf} tree) of your choice.
% Example (linux):
% \begin{quote}
%   |unzip oberdiek.tds.zip -d ~/texmf|
% \end{quote}
%
% \paragraph{Script installation.}
% Check the directory \xfile{TDS:scripts/oberdiek/} for
% scripts that need further installation steps.
%
% \subsection{Package installation}
%
% \paragraph{Unpacking.} The \xfile{.dtx} file is a self-extracting
% \docstrip\ archive. The files are extracted by running the
% \xfile{.dtx} through \plainTeX:
% \begin{quote}
%   \verb|tex tabularht.dtx|
% \end{quote}
%
% \paragraph{TDS.} Now the different files must be moved into
% the different directories in your installation TDS tree
% (also known as \xfile{texmf} tree):
% \begin{quote}
% \def\t{^^A
% \begin{tabular}{@{}>{\ttfamily}l@{ $\rightarrow$ }>{\ttfamily}l@{}}
%   tabularht.sty & tex/latex/oberdiek/tabularht.sty\\
%   tabularht.pdf & doc/latex/oberdiek/tabularht.pdf\\
%   tabularht-example1.tex & doc/latex/oberdiek/tabularht-example1.tex\\
%   tabularht-example2.tex & doc/latex/oberdiek/tabularht-example2.tex\\
%   tabularht.dtx & source/latex/oberdiek/tabularht.dtx\\
% \end{tabular}^^A
% }^^A
% \sbox0{\t}^^A
% \ifdim\wd0>\linewidth
%   \begingroup
%     \advance\linewidth by\leftmargin
%     \advance\linewidth by\rightmargin
%   \edef\x{\endgroup
%     \def\noexpand\lw{\the\linewidth}^^A
%   }\x
%   \def\lwbox{^^A
%     \leavevmode
%     \hbox to \linewidth{^^A
%       \kern-\leftmargin\relax
%       \hss
%       \usebox0
%       \hss
%       \kern-\rightmargin\relax
%     }^^A
%   }^^A
%   \ifdim\wd0>\lw
%     \sbox0{\small\t}^^A
%     \ifdim\wd0>\linewidth
%       \ifdim\wd0>\lw
%         \sbox0{\footnotesize\t}^^A
%         \ifdim\wd0>\linewidth
%           \ifdim\wd0>\lw
%             \sbox0{\scriptsize\t}^^A
%             \ifdim\wd0>\linewidth
%               \ifdim\wd0>\lw
%                 \sbox0{\tiny\t}^^A
%                 \ifdim\wd0>\linewidth
%                   \lwbox
%                 \else
%                   \usebox0
%                 \fi
%               \else
%                 \lwbox
%               \fi
%             \else
%               \usebox0
%             \fi
%           \else
%             \lwbox
%           \fi
%         \else
%           \usebox0
%         \fi
%       \else
%         \lwbox
%       \fi
%     \else
%       \usebox0
%     \fi
%   \else
%     \lwbox
%   \fi
% \else
%   \usebox0
% \fi
% \end{quote}
% If you have a \xfile{docstrip.cfg} that configures and enables \docstrip's
% TDS installing feature, then some files can already be in the right
% place, see the documentation of \docstrip.
%
% \subsection{Refresh file name databases}
%
% If your \TeX~distribution
% (\teTeX, \mikTeX, \dots) relies on file name databases, you must refresh
% these. For example, \teTeX\ users run \verb|texhash| or
% \verb|mktexlsr|.
%
% \subsection{Some details for the interested}
%
% \paragraph{Unpacking with \LaTeX.}
% The \xfile{.dtx} chooses its action depending on the format:
% \begin{description}
% \item[\plainTeX:] Run \docstrip\ and extract the files.
% \item[\LaTeX:] Generate the documentation.
% \end{description}
% If you insist on using \LaTeX\ for \docstrip\ (really,
% \docstrip\ does not need \LaTeX), then inform the autodetect routine
% about your intention:
% \begin{quote}
%   \verb|latex \let\install=y\input{tabularht.dtx}|
% \end{quote}
% Do not forget to quote the argument according to the demands
% of your shell.
%
% \paragraph{Generating the documentation.}
% You can use both the \xfile{.dtx} or the \xfile{.drv} to generate
% the documentation. The process can be configured by the
% configuration file \xfile{ltxdoc.cfg}. For instance, put this
% line into this file, if you want to have A4 as paper format:
% \begin{quote}
%   \verb|\PassOptionsToClass{a4paper}{article}|
% \end{quote}
% An example follows how to generate the
% documentation with pdf\LaTeX:
% \begin{quote}
%\begin{verbatim}
%pdflatex tabularht.dtx
%makeindex -s gind.ist tabularht.idx
%pdflatex tabularht.dtx
%makeindex -s gind.ist tabularht.idx
%pdflatex tabularht.dtx
%\end{verbatim}
% \end{quote}
%
% \begin{History}
%   \begin{Version}{2005/09/22 v1.0}
%   \item
%     First public version.
%   \end{Version}
%   \begin{Version}{2005/10/16 v2.0}
%   \item
%     Height specification allows
%     |to=...| or |spread=...|, default is |to=|.
%   \item
%     Option \xoption{vlines} added,
%     drivers \xoption{pdftex} and \xoption{dvips}.
%   \item
%     \cs{interrowspace}, \cs{interrowfil}, and
%     \cs{interrowstart}\dots\cs{interrowstop} added.
%   \end{Version}
%   \begin{Version}{2005/10/18 v2.1}
%   \item
%     Fix for package \xpackage{colortbl}, but the colors
%     of \xpackage{colortbl} remain unsupported.
%   \end{Version}
%   \begin{Version}{2006/02/20 v2.2}
%   \item
%     Code is not changed.
%   \item
%     DTX framework.
%   \end{Version}
%   \begin{Version}{2006/12/22 v2.3}
%   \item
%     Documentation fix.
%   \item
%     Fix in code of option vlines.
%   \end{Version}
%   \begin{Version}{2007/03/21 v2.4}
%   \item
%     Fix: Counter |tabh@unique| must not be changed by \cs{include}.
%   \end{Version}
%   \begin{Version}{2007/04/11 v2.5}
%   \item
%     Line ends sanitized.
%   \end{Version}
%   \begin{Version}{2016/05/16 v2.6}
%   \item
%     Documentation updates.
%   \end{Version}
% \end{History}
%
% \PrintIndex
%
% \Finale
\endinput
|
% \end{quote}
% Do not forget to quote the argument according to the demands
% of your shell.
%
% \paragraph{Generating the documentation.}
% You can use both the \xfile{.dtx} or the \xfile{.drv} to generate
% the documentation. The process can be configured by the
% configuration file \xfile{ltxdoc.cfg}. For instance, put this
% line into this file, if you want to have A4 as paper format:
% \begin{quote}
%   \verb|\PassOptionsToClass{a4paper}{article}|
% \end{quote}
% An example follows how to generate the
% documentation with pdf\LaTeX:
% \begin{quote}
%\begin{verbatim}
%pdflatex tabularht.dtx
%makeindex -s gind.ist tabularht.idx
%pdflatex tabularht.dtx
%makeindex -s gind.ist tabularht.idx
%pdflatex tabularht.dtx
%\end{verbatim}
% \end{quote}
%
% \begin{History}
%   \begin{Version}{2005/09/22 v1.0}
%   \item
%     First public version.
%   \end{Version}
%   \begin{Version}{2005/10/16 v2.0}
%   \item
%     Height specification allows
%     |to=...| or |spread=...|, default is |to=|.
%   \item
%     Option \xoption{vlines} added,
%     drivers \xoption{pdftex} and \xoption{dvips}.
%   \item
%     \cs{interrowspace}, \cs{interrowfil}, and
%     \cs{interrowstart}\dots\cs{interrowstop} added.
%   \end{Version}
%   \begin{Version}{2005/10/18 v2.1}
%   \item
%     Fix for package \xpackage{colortbl}, but the colors
%     of \xpackage{colortbl} remain unsupported.
%   \end{Version}
%   \begin{Version}{2006/02/20 v2.2}
%   \item
%     Code is not changed.
%   \item
%     DTX framework.
%   \end{Version}
%   \begin{Version}{2006/12/22 v2.3}
%   \item
%     Documentation fix.
%   \item
%     Fix in code of option vlines.
%   \end{Version}
%   \begin{Version}{2007/03/21 v2.4}
%   \item
%     Fix: Counter |tabh@unique| must not be changed by \cs{include}.
%   \end{Version}
%   \begin{Version}{2007/04/11 v2.5}
%   \item
%     Line ends sanitized.
%   \end{Version}
%   \begin{Version}{2016/05/16 v2.6}
%   \item
%     Documentation updates.
%   \end{Version}
% \end{History}
%
% \PrintIndex
%
% \Finale
\endinput

%        (quote the arguments according to the demands of your shell)
%
% Documentation:
%    (a) If tabularht.drv is present:
%           latex tabularht.drv
%    (b) Without tabularht.drv:
%           latex tabularht.dtx; ...
%    The class ltxdoc loads the configuration file ltxdoc.cfg
%    if available. Here you can specify further options, e.g.
%    use A4 as paper format:
%       \PassOptionsToClass{a4paper}{article}
%
%    Programm calls to get the documentation (example):
%       pdflatex tabularht.dtx
%       makeindex -s gind.ist tabularht.idx
%       pdflatex tabularht.dtx
%       makeindex -s gind.ist tabularht.idx
%       pdflatex tabularht.dtx
%
% Installation:
%    TDS:tex/latex/oberdiek/tabularht.sty
%    TDS:doc/latex/oberdiek/tabularht.pdf
%    TDS:doc/latex/oberdiek/tabularht-example1.tex
%    TDS:doc/latex/oberdiek/tabularht-example2.tex
%    TDS:source/latex/oberdiek/tabularht.dtx
%
%<*ignore>
\begingroup
  \catcode123=1 %
  \catcode125=2 %
  \def\x{LaTeX2e}%
\expandafter\endgroup
\ifcase 0\ifx\install y1\fi\expandafter
         \ifx\csname processbatchFile\endcsname\relax\else1\fi
         \ifx\fmtname\x\else 1\fi\relax
\else\csname fi\endcsname
%</ignore>
%<*install>
\input docstrip.tex
\Msg{************************************************************************}
\Msg{* Installation}
\Msg{* Package: tabularht 2016/05/16 v2.6 Tabular with height specified (HO)}
\Msg{************************************************************************}

\keepsilent
\askforoverwritefalse

\let\MetaPrefix\relax
\preamble

This is a generated file.

Project: tabularht
Version: 2016/05/16 v2.6

Copyright (C) 2005-2007 by
   Heiko Oberdiek <heiko.oberdiek at googlemail.com>

This work may be distributed and/or modified under the
conditions of the LaTeX Project Public License, either
version 1.3c of this license or (at your option) any later
version. This version of this license is in
   https://www.latex-project.org/lppl/lppl-1-3c.txt
and the latest version of this license is in
   https://www.latex-project.org/lppl.txt
and version 1.3 or later is part of all distributions of
LaTeX version 2005/12/01 or later.

This work has the LPPL maintenance status "maintained".

The Current Maintainers of this work are
Heiko Oberdiek and the Oberdiek Package Support Group
https://github.com/ho-tex/oberdiek/issues


This work consists of the main source file tabularht.dtx
and the derived files
   tabularht.sty, tabularht.pdf, tabularht.ins, tabularht.drv,
   tabularht-example1.tex, tabularht-example2.tex.

\endpreamble
\let\MetaPrefix\DoubleperCent

\generate{%
  \file{tabularht.ins}{\from{tabularht.dtx}{install}}%
  \file{tabularht.drv}{\from{tabularht.dtx}{driver}}%
  \usedir{tex/latex/oberdiek}%
  \file{tabularht.sty}{\from{tabularht.dtx}{package}}%
  \usedir{doc/latex/oberdiek}%
  \file{tabularht-example1.tex}{\from{tabularht.dtx}{example1}}%
  \file{tabularht-example2.tex}{\from{tabularht.dtx}{example2}}%
  \nopreamble
  \nopostamble
%  \usedir{source/latex/oberdiek/catalogue}%
%  \file{tabularht.xml}{\from{tabularht.dtx}{catalogue}}%
}

\catcode32=13\relax% active space
\let =\space%
\Msg{************************************************************************}
\Msg{*}
\Msg{* To finish the installation you have to move the following}
\Msg{* file into a directory searched by TeX:}
\Msg{*}
\Msg{*     tabularht.sty}
\Msg{*}
\Msg{* To produce the documentation run the file `tabularht.drv'}
\Msg{* through LaTeX.}
\Msg{*}
\Msg{* Happy TeXing!}
\Msg{*}
\Msg{************************************************************************}

\endbatchfile
%</install>
%<*ignore>
\fi
%</ignore>
%<*driver>
\NeedsTeXFormat{LaTeX2e}
\ProvidesFile{tabularht.drv}%
  [2016/05/16 v2.6 Tabular with height specified (HO)]%
\documentclass{ltxdoc}
\usepackage{holtxdoc}[2011/11/22]
\begin{document}
  \DocInput{tabularht.dtx}%
\end{document}
%</driver>
% \fi
%
%
% \CharacterTable
%  {Upper-case    \A\B\C\D\E\F\G\H\I\J\K\L\M\N\O\P\Q\R\S\T\U\V\W\X\Y\Z
%   Lower-case    \a\b\c\d\e\f\g\h\i\j\k\l\m\n\o\p\q\r\s\t\u\v\w\x\y\z
%   Digits        \0\1\2\3\4\5\6\7\8\9
%   Exclamation   \!     Double quote  \"     Hash (number) \#
%   Dollar        \$     Percent       \%     Ampersand     \&
%   Acute accent  \'     Left paren    \(     Right paren   \)
%   Asterisk      \*     Plus          \+     Comma         \,
%   Minus         \-     Point         \.     Solidus       \/
%   Colon         \:     Semicolon     \;     Less than     \<
%   Equals        \=     Greater than  \>     Question mark \?
%   Commercial at \@     Left bracket  \[     Backslash     \\
%   Right bracket \]     Circumflex    \^     Underscore    \_
%   Grave accent  \`     Left brace    \{     Vertical bar  \|
%   Right brace   \}     Tilde         \~}
%
% \GetFileInfo{tabularht.drv}
%
% \title{The \xpackage{tabularht} package}
% \date{2016/05/16 v2.6}
% \author{Heiko Oberdiek\thanks
% {Please report any issues at \url{https://github.com/ho-tex/oberdiek/issues}}}
%
% \maketitle
%
% \begin{abstract}
% This package defines some environments that adds
% a height specification to tabular and array.
% \end{abstract}
%
% \tableofcontents
%
% \section{Usage}
%
% \begin{quote}
%   |\usepackage{tabularht}|
% \end{quote}
% The package provides the following environments
% that extend the tabular/array environment by
% a height specification as first argument:
% \begin{itemize}
% \item |tabularht|, |tabularht*|
% \item |arrayht|
% \item |tabularhtx|
%       (if package \xpackage{tabularx} is loaded)
% \end{itemize}
% The height argument allows a length specification,
% package \xpackage{calc} is supported if used. This means,
% the tabular will have the specified height.
% You can also use the prefixes |to=| and |spread=|.
% |to=| is the default, |spread=| means, the natural
% height of the tabular box is changed by the length
% after |spread=|.
%
% Examples:
% \begin{quote}
% \begin{tabular}{@{}l@{ $\rightarrow$ }l@{}}
%   |\begin{tabularht}{1in}|
%   & height is 1in\\
%   |\begin{tabularht}{to=1in}|
%   & height is 1in\\
%   |\begin{tabularht}{spread=0pt}|
%   & natural height, same as |\begin{tabular}|\\
%   |\begin{tabularht}{spread=1in}|
%   & natural height increased by 1in
% \end{tabular}
% \end{quote}
%
% Hint:  See also package \xpackage{tabularkv}, it provides an interface,
%        where most parameters for the environments can be given
%        by key-value pairs.
%
% \begin{declcs}{interrowspace}\,|{|\dots|}|
% \end{declcs}
% Adds space between table rows. It is essentially
% the same as |\noalign{\vspace{...}}|.
% \begin{declcs}{interrowfill}
% \end{declcs}
% Short for |\interrowspace{\fill}|
% \begin{declcs}{interrowstart}\,\dots\,\cs{interrowstop}
% \end{declcs}
% Marker commands, useful for option \xoption{vlines}.
%
% \subsection{Option \xoption{vlines}}
%
% Warning: This stuff is experimental.
%
% \vspace{2ex}
% \noindent
% Vertical lines are interrupted, if space is inserted
% in \cs{noalign}, \cs{interrowspace}, \cs{addlinespace}
% (\xpackage{booktabs}),
% between double \cs{hline}s. This option tries to detect
% and add the vertical lines. The lines in a tabular
% with \xpackage{tabularht} support (environments of this package)
% are numbered from left to right. The gap that is
% controlled by \cs{interrowspace} or inbetween \cs{interrowstart}
% and \cs{interrowstop} is then filled with the detected
% vertical lines.
%
% If only a limited selection of the lines should be
% drawn, the commands know an optional argument with a
% list of line numbers, e.g.
% \begin{quote}
%\begin{verbatim}
%\begin{tabularht}{50mm}{|l|l|}
%  Hello & World\\
%  \interrowfill[1,3]
%  Foo & Bar
%\end{tabularht}
%\end{verbatim}
% \end{quote}
% There are three lines, but the middle line is not drawn
% in the gap between the first and second row.
% Zero can be used to suppress all lines:
% \begin{quote}
% |\interrowspace[0]{10mm}|
% \end{quote}
% The syntax of the commands with the optional argument
% with the line number list \meta{list}. \meta{list} is a comma
% separated list of numbers, \meta{height} means the
% height specification described above with the optional
% prefixes |to=| or |spread=|.
% \begin{quote}
%   \cs{interrowspace}\,|[|\meta{list}|]|\,|{|\meta{height}|}|\\
%   \cs{interrowfill}\,|[|\meta{list}|]|\\
%   \cs{interrowstart}\,|[|\meta{list}|]|\,\dots\,\cs{interrowstop}
% \end{quote}
%
% Option \xoption{vlines} is driver dependent and uses \eTeX\ features.
% \begin{description}
% \item[\xoption{pdftex}:] \pdfTeX\ in PDF mode.
%   Here the positions of the lines are written with the
%   help of the \cs{pdfsavepos} feature into the \xfile{.aux} file(s).
%   Therefore you need two LaTeX runs to get the lines.
% \item[\xoption{dvips}:]
%   Here, PostScript's currentpoint it used to get
%   the line positions. The lines are then drawn at the end
%   of the page. Thus one \LaTeX/dvips run is sufficient
%   for this option.
% \item[Other drivers:]\mbox{}
%   \begin{description}
%   \item[PostScript drivers:] probably possible, an end of page
%     hook would be nice.
%   \item[\VTeX:] with GeX (PostScript interpreter) probably possible.
%   \item[dvipdfm:] no idea. The big problem is, how to get the
%     current position?
%   \end{description}
% \end{description}
%
% \subsection{Limitations}
%
% \begin{itemize}
% \item
%   Vertical lines are interrupted by |\noalign{\vfill}|.
% \end{itemize}
%
% \subsection{Compatibility}
%
% \begin{itemize}
% \item
%   \xpackage{array}, \xpackage{delarray}, \xpackage{tabularx}
%   are supported.
% \item There can be problems with packages that
%   redefine \cs{@array} (or \cs{@@array}, \cs{@tabarray}) and
%   \cs{@arrayrule} (for option \xoption{vlines}).
% \item \xpackage{colortbl}: it should at least work, but there isn't
%   support for filling the gaps with color, neither
%   the rules nor the backgrounds.
% \end{itemize}
%
% \subsection{Examples}
%
% \subsubsection{Example 1}
%
%    \begin{macrocode}
%<*example1>
\documentclass{article}
\usepackage{tabularht}

\begin{document}
\fbox{%
  \begin{tabularht*}{1in}{4in}{@{}l@{\extracolsep{\fill}}r@{}}%
    upper left corner & upper right corner\\%
    \noalign{\vfill}%
    \multicolumn{2}{@{}c@{}}{bounding box}\\%
    \noalign{\vfill}%
    lower left corner & lower right corner\\%
  \end{tabularht*}%
}
\end{document}
%</example1>
%    \end{macrocode}
%
% \subsubsection{Example 2}
%
%    \begin{macrocode}
%<*example2>
\documentclass{article}
\usepackage{booktabs}
\usepackage[dvips,vlines]{tabularht}

\begin{document}

\begin{tabularht}{spread=0pt}{|l|l|}
  \hline
  First&Line\\%
  \hline
\interrowstart
  \addlinespace[10mm]%
\interrowstop
  \hline
  Second&Line\\%
\interrowstart
  \hline
  \hline
\interrowstop
  Third&Line\\%
  \hline
\interrowspace{10mm}
  \hline
  Fourth&Line\\%
  \hline
\end{tabularht}

\end{document}
%</example2>
%    \end{macrocode}
%
% \StopEventually{
% }
%
% \section{Implementation}
%
%    \begin{macrocode}
%<*package>
%    \end{macrocode}
%    Package identification.
%    \begin{macrocode}
\NeedsTeXFormat{LaTeX2e}
\ProvidesPackage{tabularht}%
  [2016/05/16 v2.6 Tabular with height specified (HO)]
%    \end{macrocode}
%
% \subsection{Environments}
%
%    \begin{macrocode}
\let\@toarrayheight\@empty
\let\tabH@array@init\@empty

\toks@={%
  \begingroup
    \long\def\x#1\vcenter\fi\fi\bgroup#2\@sharp#3#4\@nil{%
      \endgroup
      \gdef\@array[##1]##2{%
        \tabH@array@init
        #1%
        \vcenter\fi\fi
        \@toarrayheight
        \bgroup
        \let\@toarrayheight\@empty
        #2\@sharp###3#4%
      }%
    }%
  \expandafter\x\@array[#1]{#2}\@nil % hash-ok
}
\edef\tabH@patch@array{\the\toks@}
\def\tabH@patch@@array{%
  \ifx\@array\@@array
    \def\reserved@a{\let\@@array\@array}%
  \else
    \let\reserved@a\relax
  \fi
  \tabH@patch@array
  \reserved@a
}
\tabH@patch@@array

\@ifpackageloaded{array}{}{%
  \AtBeginDocument{%
    \@ifpackageloaded{array}{%
      \tabH@patch@@array
    }{}%
  }%
}

\def\tabH@setheight#1{%
  \tabH@@setheight#1==\@nil
}
\def\tabH@@setheight#1=#2=#3\@nil{%
  \ifx\\#2#3\\%
    \setlength{\dimen@}{#1}%
    \edef\@toarrayheight{to\the\dimen@}%
  \else
    \edef\tabH@temp{\zap@space#1 \@empty}%
    \ifx\tabH@temp\tabH@to
    \else
      \ifx\tabH@temp\tabH@spread
      \else
        \PackageError{tabularht}{%
          Unknown height specifier %
          `\expandafter\strip@prefix\meaning\tabH@temp'%
        }{%
          The height dimension for tabular height can be prefixed%
          \MessageBreak
          with `to=' or `spread=', default is `to='.%
        }%
        \let\tabH@temp\tabH@to
      \fi
    \fi
    \setlength{\dimen@}{#2}%
    \edef\@toarrayheight{\tabH@temp\the\dimen@}%
  \fi
}
\def\tabH@to{to}
\def\tabH@spread{spread}
%    \end{macrocode}
%    First argument is the height of the table,
%    then the orignal arguments for tabular follow.
%    \begin{macrocode}
\newenvironment{tabularht}[1]{%
  \tabH@setheight{#1}%
  \tabular
}{%
  \endtabular
}

\newenvironment{tabularht*}[1]{%
  \tabH@setheight{#1}%
  \@nameuse{tabular*}%
}{%
  \@nameuse{endtabular*}%
}

\newenvironment{tabularhtx}[1]{%
  \tabH@setheight{#1}%
  \tabularx
}{%
  \endtabularx
}

\newenvironment{arrayht}[1]{%
  \tabH@setheight{#1}%
  \array
}{%
  \endarray
}

\def\interrowspace{%
  \noalign\bgroup
    \tabH@interrowspace
}
\newcommand*{\tabH@interrowspace}[2][]{%
    \tabH@vspace{#1}{#2}%
  \egroup
}
\def\interrowfill{%
  \noalign\bgroup
    \tabH@interrowfill
}
\newcommand*{\tabH@interrowfill}[1][]{%
    \tabH@vspace{#1}{\fill}%
  \egroup
}
\def\tabH@vspace#1#2{%
  \tabH@vspace@start{#1}%
  \vspace{#2}%
  \tabH@vspace@stop
}
\let\tabH@vspace@start\@gobble
\let\tabH@vspace@stop\@empty

\newcommand*{\interrowstart}{%
  \noalign\bgroup
    \tabH@interrowstart
}
\newcommand*{\tabH@interrowstart}[1][]{%
    \tabH@vspace@start{#1}%
  \egroup
}
\newcommand*{\interrowstop}{%
  \noalign{\tabH@vspace@stop}%
}
%    \end{macrocode}
%
% \subsection{Options}
%
%    \begin{macrocode}
\providecommand*{\tabH@driver}{}

\DeclareOption{vlines}{%
  \let\tabH@temp\relax
}
\DeclareOption{pdftex}{}
\DeclareOption{dvips}{%
  \def\tabH@driver{dvips}%
}
\ProcessOptions*\relax

\ifx\tabH@temp\relax
\else
  \expandafter\endinput
\fi

\begingroup
  \@ifundefined{eTeXversion}{%
    \PackageError{tabularht}{%
      Option `vlines' requires eTeX%
    }{%
      Use of eTeX is recommended for LaTeX, see ltnews16.%
    }%
    \endgroup
    \endinput
  }{}%
\endgroup
%    \end{macrocode}
%
% \subsection{Option vlines, driver independent stuff}
%
%    \begin{macrocode}
\begingroup
  \let\@addtoreset\@gobbletwo
  \newcounter{tabH@unique}%
\endgroup
\let\tabH@currenttab\@empty

\def\tabH@array@init{%
  \ifx\@toarrayheight\@empty
    % ignore vertical lines of nested tabular environments
    \let\tabH@currenttab\@empty
  \else
    \stepcounter{tabH@unique}%
    \edef\tabH@currenttab{\the\c@tabH@unique}%
  \fi
}

\renewcommand*{\@arrayrule}{%
  \@addtopreamble{%
    \hskip -.5\arrayrulewidth
    \ifx\tabH@currenttab\@empty
    \else
      \tabH@vrule{\tabH@currenttab}%
    \fi
    \begingroup
      \expandafter\ifx\csname CT@arc@\endcsname\relax
      \else
        \expandafter\CT@arc@
      \fi
      \vline
    \endgroup
    \hskip -.5\arrayrulewidth
  }%
}
\let\tabH@arrayrule\@arrayrule
\AtBeginDocument{%
  \@ifpackageloaded{colortbl}{%
    \let\@arrayrule\tabH@arrayrule
  }{}%
}

\let\tabH@vrule\@gobble
%    \end{macrocode}
%
% \subsection{Driver pdftex}
%
%    \begin{macrocode}
\RequirePackage{ifpdf}
\ifpdf
  \begingroup
    \@ifundefined{pdfsavepos}{%
      \PackageError{tabularht}{%
        Your pdfTeX is too old%
      }{%
        \string\pdfsavepos\space is missing.%
      }%
      \endgroup
      \csname fi\endcsname
      \endinput
    }{}%

    \let\on@line\@empty
    \PackageInfo{tabularht}{%
      Using driver `pdftex' because of pdfTeX in PDF mode%
    }%
  \endgroup

  \protected\def\tabH@vrule#1{%
    \if@filesw
      \pdfsavepos
      \protected@write\@auxout{%
        \let\tabH@lastxpos\relax
      }{%
        \tabH@aux@vrule{#1}{\tabH@lastxpos}%
      }%
    \fi
  }%

  \def\tabH@lastxpos{\the\pdflastxpos}%
  \def\tabH@lastypos{\the\pdflastypos}%

  % The .aux file contains three commands:
  % \tabH@aux@vrule{tabular id}{x position}
  % \tabH@aux@vstart{tabular id}{row id}{x position}{y position}
  % \tabH@aux@vstop{y position}
  %
  \AtBeginDocument{%
    % The .aux files are read the first time before
    % \AtBeginDocument and later at \end{document}.
    % \tabH@aux@done is a marker to distinguish
    % between these two readings. Only in the first
    % case we need the \tabH@aux@... commands.
    \let\tabH@aux@done\@empty
    \if@filesw
      \immediate\write\@mainaux{%
        \@percentchar\@percentchar BeginProlog: tabularht%
      }%
      % items in the aux file are executed,
      % if tabularht is loaded
      % and during the aux file read at \begin{document} only
      \immediate\write\@mainaux{%
        \detokenize{%
          % the \tabH@aux@... commands are needed only if
          % tabularht is loaded with driver pdftex.
          \@ifundefined{tabH@aux@vrule}\@secondoftwo\@firstofone
          {%
            % disable commands except for the first .aux files reading
            \@ifundefined{tabH@aux@done}\@gobble\@firstofone
          }%
          {%
            \let\tabH@aux@vrule\@gobbletwo
            \let\tabH@aux@vstart\@gobblefour
            \let\tabH@aux@vstop\@gobble
          }%
        }%
      }%
      \immediate\write\@mainaux{%
        \@percentchar\@percentchar EndProlog: tabularht%
      }%
    \fi
  }%

  % the x positions of vrules are stored in
  % \tabH@<tabcount>list with distinct values
  \protected\def\tabH@aux@vrule#1#2{%
    \@ifundefined{tabH@#1list}{%
      \expandafter\xdef\csname tabH@#1list\endcsname{%
        \noexpand\do{#2}%
      }%
    }{%
      \begingroup
        \def\x{#2}%
        \let\y\@undefined
        \let\do\tabH@do@add
        \expandafter\xdef\csname tabH@#1list\endcsname{%
          \csname tabH@#1list\endcsname\@empty
          \ifx\y\@undefined
            \noexpand\do{\x}%
          \fi
        }%
      \endgroup
    }%
  }%
  \def\tabH@do@add#1{%
    \ifx\y\@undefined
      \ifnum#1<\x\space
      \else
        \expandafter\ifx\csname y\endcsname\relax\fi
        \ifnum#1>\x\space
          \noexpand\do{\x}%
        \fi
      \fi
    \fi
    \noexpand\do{#1}%
  }%

  \def\tabH@vspace@start#1{%
    \if@filesw
      \stepcounter{tabH@unique}%
      \edef\tabH@currentrow{\the\c@tabH@unique}%
      \pdfsavepos
      \protected@write\@auxout{%
        \let\tabH@lastxpos\relax
        \let\tabH@lastypos\relax
      }{%
        \tabH@aux@vstart{\tabH@currenttab}{\tabH@currentrow}%
                        {\tabH@lastxpos}{\tabH@lastypos}%
      }%
    \fi
    \begingroup
      \edef\a{tabH@\tabH@currenttab row\tabH@currentrow}%
      \expandafter\let\expandafter\x\csname\a x\endcsname
      \ifx\x\relax
      \else
        \expandafter\let\expandafter\y\csname\a y\endcsname
        \expandafter\let\expandafter\l
            \csname tabH@\tabH@currenttab list\endcsname
        \ifx\l\relax
        \else
          \def\f{#1}%
          \ifx\f\@empty
            \let\do\tabH@do@set
          \else
            \count@=\z@
            \let\do\tabH@do@filter
          \fi
          \setbox\z@=\hbox{\l}%
          \wd\z@=\z@
          \dp\z@=\z@
          \copy\z@
        \fi
      \fi
    \endgroup
  }%
  \def\tabH@vspace@stop{%
    \if@filesw
      \pdfsavepos
      \protected@write\@auxout{%
        \let\tabH@lastypos\relax
      }{%
        \tabH@aux@vstop{\tabH@lastypos}%
      }%
    \fi
  }%
  \def\tabH@do@set#1{%
    \hbox to \z@{%
      \hskip \dimexpr #1sp - \x sp\relax
      \vrule \@width\arrayrulewidth
             \@depth\dimexpr \y sp\relax
      \hss
    }%
  }%
  \def\tabH@do@filter{%
    \@tempswafalse
    \advance\count@\@ne
    \@for\e:=\f\do{%
      \ifnum\e=\count@
        \@tempswatrue
      \fi
    }%
    \if@tempswa
      \expandafter\tabH@do@set
    \else
      \expandafter\@gobble
    \fi
  }%

  \protected\def\tabH@aux@vstart#1#2#3#4{%
    \def\tabH@current@vstart{{#1}{#2}{#3}{#4}}%
  }%
  \protected\def\tabH@aux@vstop{%
    \expandafter\tabH@aux@v\tabH@current@vstart
  }%
  \def\tabH@aux@v#1#2#3#4#5{%
    \expandafter\gdef\csname tabH@#1row#2x\endcsname{#3}%
    \expandafter\xdef\csname tabH@#1row#2y\endcsname{%
      \the\numexpr #4 - #5\relax
    }%
  }%

  \csname fi\endcsname
  \endinput

\fi
%    \end{macrocode}
%
% \subsection{DVI drivers}
%
%    \begin{macrocode}
\ifx\tabH@driver\@empty
  \PackageError{tabularht}{%
    Missing DVI driver, option `vlines' disabled%
  }{%
    Supported DVI drivers: dvips.%
  }%
  \expandafter\endinput
\fi

\def\tabH@driver@dvips{%
  \def\tabH@literalps##1{\special{ps:SDict begin ##1 end}}%
  \def\tabH@headerps##1{\special{! ##1}}%
}

\@onelevel@sanitize\tabH@driver
\@ifundefined{tabH@driver@\tabH@driver}{%
  \PackageError{tabularht}{%
    Unsupported driver `\tabH@driver'%
  }{%
    Supported DVI drivers: dvips.%
  }%
  \endinput
}{}

\begingroup
  \let\on@line\@empty
  \PackageInfo{tabularht}{%
    Using driver `\tabH@driver'%
  }%
\endgroup
\csname tabH@driver@\tabH@driver\endcsname

\protected\def\tabH@vrule#1#2\vrule#3\arrayrulewidth{%
  #2% \fi or empty
  % hack to get rid of maxdrift rounding of dvips,
  % thus simulate a large motion
  \kern1in\relax
  \tabH@literalps{%
    #1 tabH.vrule %
    Resolution neg 0 translate%
  }%
  \vrule#3\arrayrulewidth
  \tabH@literalps{Resolution 0 translate}%
  \kern-1in\relax
}

\def\tabH@vspace@start#1{%
  \begingroup
    \let\y\@empty
    \@for\x:=#1\do{%
      \ifx\y\@empty
        \edef\y{\x}%
      \else
        \edef\y{\y\space\x}%
      \fi
    }%
    \tabH@literalps{\tabH@currenttab[\y]currentpoint exch pop}%
  \endgroup
}
\def\tabH@vspace@stop{%
  \tabH@literalps{%
    currentpoint exch pop %
    \number\dimexpr\arrayrulewidth\relax\space
    tabH.vspace%
  }%
}

\tabH@headerps{%
  userdict begin%
    /tabH.list 10 dict def%
    /tabH.job [] def %
  end%
  /tabH.vrule{%
    10 string cvs cvn dup tabH.list exch known{%
      tabH.list exch dup [ exch tabH.list exch get %
      currentpoint pop round exch true exch{%
        % tabH.list key [ ... x true i
        % tabH.list key [ ... false i
        exch{%
          % ... [ ... x i
          2 copy lt{false}{%
            2 copy eq{pop false}{exch true}ifelse%
          }ifelse%
        }{false}ifelse%
      }forall %
      pop%
      ]put%
    }{%
      tabH.list exch[currentpoint pop round]put%
    }ifelse%
  }bind def%
  % <tab num> <cols array> <ytop> <ybottom> <rulewidth[sp]>
  /tabH.vspace{%
    userdict begin %
      10 dict dup begin %
        exch 65536 div Resolution mul 72.27 div %
        % dvips uses a poor man's ceil function
        % see dopage.c before "drawrule": (int)(... + 0.9999999)
        0.9999999 add truncate%
        /rulewidth exch def %
        exch/ybottom exch def %
        exch/ytop exch def %
        exch/cols exch def %
        exch/tabkey exch 10 string cvs cvn def %
      end%
      /tabH.job exch[exch userdict/tabH.job get aload pop]def %
    end%
  }bind def %
  % Now we do the work at the end of the page.
  % Unhappily "eop-hook" cannot be used, because "eop"
  % executes "restore" before, so that all data are lost.
  TeXDict begin%
    /eop%
    [%
      {%
        tabH.job{%
          begin%
           /colarray %
             tabH.list tabkey known{tabH.list tabkey get}{[]}ifelse %
           def %
           cols length 0 eq not{%
             /colarray[%
               cols{1 sub %
                 dup 0 lt{pop}{%
                   dup colarray length ge{pop}{%
                     colarray exch get%
                   }ifelse%
                 }ifelse%
               }forall%
             ]def%
           }if %
           colarray{%
             % (rulewidth) == rulewidth == % debug
             Resolution sub %
             ytop rulewidth ytop ybottom sub v%
           }forall %
          end%
        }forall%
        % tabH.list{== ==}forall % debug
      }bind aload pop %
      TeXDict /eop get aload pop%
    ]cvx def %
  end%
}
%    \end{macrocode}
%
%    \begin{macrocode}
%</package>
%    \end{macrocode}
%
% \section{Installation}
%
% \subsection{Download}
%
% \paragraph{Package.} This package is available on
% CTAN\footnote{\CTANpkg{tabularht}}:
% \begin{description}
% \item[\CTAN{macros/latex/contrib/oberdiek/tabularht.dtx}] The source file.
% \item[\CTAN{macros/latex/contrib/oberdiek/tabularht.pdf}] Documentation.
% \end{description}
%
%
% \paragraph{Bundle.} All the packages of the bundle `oberdiek'
% are also available in a TDS compliant ZIP archive. There
% the packages are already unpacked and the documentation files
% are generated. The files and directories obey the TDS standard.
% \begin{description}
% \item[\CTANinstall{install/macros/latex/contrib/oberdiek.tds.zip}]
% \end{description}
% \emph{TDS} refers to the standard ``A Directory Structure
% for \TeX\ Files'' (\CTAN{tds/tds.pdf}). Directories
% with \xfile{texmf} in their name are usually organized this way.
%
% \subsection{Bundle installation}
%
% \paragraph{Unpacking.} Unpack the \xfile{oberdiek.tds.zip} in the
% TDS tree (also known as \xfile{texmf} tree) of your choice.
% Example (linux):
% \begin{quote}
%   |unzip oberdiek.tds.zip -d ~/texmf|
% \end{quote}
%
% \paragraph{Script installation.}
% Check the directory \xfile{TDS:scripts/oberdiek/} for
% scripts that need further installation steps.
%
% \subsection{Package installation}
%
% \paragraph{Unpacking.} The \xfile{.dtx} file is a self-extracting
% \docstrip\ archive. The files are extracted by running the
% \xfile{.dtx} through \plainTeX:
% \begin{quote}
%   \verb|tex tabularht.dtx|
% \end{quote}
%
% \paragraph{TDS.} Now the different files must be moved into
% the different directories in your installation TDS tree
% (also known as \xfile{texmf} tree):
% \begin{quote}
% \def\t{^^A
% \begin{tabular}{@{}>{\ttfamily}l@{ $\rightarrow$ }>{\ttfamily}l@{}}
%   tabularht.sty & tex/latex/oberdiek/tabularht.sty\\
%   tabularht.pdf & doc/latex/oberdiek/tabularht.pdf\\
%   tabularht-example1.tex & doc/latex/oberdiek/tabularht-example1.tex\\
%   tabularht-example2.tex & doc/latex/oberdiek/tabularht-example2.tex\\
%   tabularht.dtx & source/latex/oberdiek/tabularht.dtx\\
% \end{tabular}^^A
% }^^A
% \sbox0{\t}^^A
% \ifdim\wd0>\linewidth
%   \begingroup
%     \advance\linewidth by\leftmargin
%     \advance\linewidth by\rightmargin
%   \edef\x{\endgroup
%     \def\noexpand\lw{\the\linewidth}^^A
%   }\x
%   \def\lwbox{^^A
%     \leavevmode
%     \hbox to \linewidth{^^A
%       \kern-\leftmargin\relax
%       \hss
%       \usebox0
%       \hss
%       \kern-\rightmargin\relax
%     }^^A
%   }^^A
%   \ifdim\wd0>\lw
%     \sbox0{\small\t}^^A
%     \ifdim\wd0>\linewidth
%       \ifdim\wd0>\lw
%         \sbox0{\footnotesize\t}^^A
%         \ifdim\wd0>\linewidth
%           \ifdim\wd0>\lw
%             \sbox0{\scriptsize\t}^^A
%             \ifdim\wd0>\linewidth
%               \ifdim\wd0>\lw
%                 \sbox0{\tiny\t}^^A
%                 \ifdim\wd0>\linewidth
%                   \lwbox
%                 \else
%                   \usebox0
%                 \fi
%               \else
%                 \lwbox
%               \fi
%             \else
%               \usebox0
%             \fi
%           \else
%             \lwbox
%           \fi
%         \else
%           \usebox0
%         \fi
%       \else
%         \lwbox
%       \fi
%     \else
%       \usebox0
%     \fi
%   \else
%     \lwbox
%   \fi
% \else
%   \usebox0
% \fi
% \end{quote}
% If you have a \xfile{docstrip.cfg} that configures and enables \docstrip's
% TDS installing feature, then some files can already be in the right
% place, see the documentation of \docstrip.
%
% \subsection{Refresh file name databases}
%
% If your \TeX~distribution
% (\teTeX, \mikTeX, \dots) relies on file name databases, you must refresh
% these. For example, \teTeX\ users run \verb|texhash| or
% \verb|mktexlsr|.
%
% \subsection{Some details for the interested}
%
% \paragraph{Unpacking with \LaTeX.}
% The \xfile{.dtx} chooses its action depending on the format:
% \begin{description}
% \item[\plainTeX:] Run \docstrip\ and extract the files.
% \item[\LaTeX:] Generate the documentation.
% \end{description}
% If you insist on using \LaTeX\ for \docstrip\ (really,
% \docstrip\ does not need \LaTeX), then inform the autodetect routine
% about your intention:
% \begin{quote}
%   \verb|latex \let\install=y% \iffalse meta-comment
%
% File: tabularht.dtx
% Version: 2016/05/16 v2.6
% Info: Tabular with height specified
%
% Copyright (C) 2005-2007 by
%    Heiko Oberdiek <heiko.oberdiek at googlemail.com>
%    2016
%    https://github.com/ho-tex/oberdiek/issues
%
% This work may be distributed and/or modified under the
% conditions of the LaTeX Project Public License, either
% version 1.3c of this license or (at your option) any later
% version. This version of this license is in
%    https://www.latex-project.org/lppl/lppl-1-3c.txt
% and the latest version of this license is in
%    https://www.latex-project.org/lppl.txt
% and version 1.3 or later is part of all distributions of
% LaTeX version 2005/12/01 or later.
%
% This work has the LPPL maintenance status "maintained".
%
% The Current Maintainers of this work are
% Heiko Oberdiek and the Oberdiek Package Support Group
% https://github.com/ho-tex/oberdiek/issues
%
% This work consists of the main source file tabularht.dtx
% and the derived files
%    tabularht.sty, tabularht.pdf, tabularht.ins, tabularht.drv,
%    tabularht-example1.tex, tabularht-example2.tex.
%
% Distribution:
%    CTAN:macros/latex/contrib/oberdiek/tabularht.dtx
%    CTAN:macros/latex/contrib/oberdiek/tabularht.pdf
%
% Unpacking:
%    (a) If tabularht.ins is present:
%           tex tabularht.ins
%    (b) Without tabularht.ins:
%           tex tabularht.dtx
%    (c) If you insist on using LaTeX
%           latex \let\install=y% \iffalse meta-comment
%
% File: tabularht.dtx
% Version: 2016/05/16 v2.6
% Info: Tabular with height specified
%
% Copyright (C) 2005-2007 by
%    Heiko Oberdiek <heiko.oberdiek at googlemail.com>
%    2016
%    https://github.com/ho-tex/oberdiek/issues
%
% This work may be distributed and/or modified under the
% conditions of the LaTeX Project Public License, either
% version 1.3c of this license or (at your option) any later
% version. This version of this license is in
%    https://www.latex-project.org/lppl/lppl-1-3c.txt
% and the latest version of this license is in
%    https://www.latex-project.org/lppl.txt
% and version 1.3 or later is part of all distributions of
% LaTeX version 2005/12/01 or later.
%
% This work has the LPPL maintenance status "maintained".
%
% The Current Maintainers of this work are
% Heiko Oberdiek and the Oberdiek Package Support Group
% https://github.com/ho-tex/oberdiek/issues
%
% This work consists of the main source file tabularht.dtx
% and the derived files
%    tabularht.sty, tabularht.pdf, tabularht.ins, tabularht.drv,
%    tabularht-example1.tex, tabularht-example2.tex.
%
% Distribution:
%    CTAN:macros/latex/contrib/oberdiek/tabularht.dtx
%    CTAN:macros/latex/contrib/oberdiek/tabularht.pdf
%
% Unpacking:
%    (a) If tabularht.ins is present:
%           tex tabularht.ins
%    (b) Without tabularht.ins:
%           tex tabularht.dtx
%    (c) If you insist on using LaTeX
%           latex \let\install=y\input{tabularht.dtx}
%        (quote the arguments according to the demands of your shell)
%
% Documentation:
%    (a) If tabularht.drv is present:
%           latex tabularht.drv
%    (b) Without tabularht.drv:
%           latex tabularht.dtx; ...
%    The class ltxdoc loads the configuration file ltxdoc.cfg
%    if available. Here you can specify further options, e.g.
%    use A4 as paper format:
%       \PassOptionsToClass{a4paper}{article}
%
%    Programm calls to get the documentation (example):
%       pdflatex tabularht.dtx
%       makeindex -s gind.ist tabularht.idx
%       pdflatex tabularht.dtx
%       makeindex -s gind.ist tabularht.idx
%       pdflatex tabularht.dtx
%
% Installation:
%    TDS:tex/latex/oberdiek/tabularht.sty
%    TDS:doc/latex/oberdiek/tabularht.pdf
%    TDS:doc/latex/oberdiek/tabularht-example1.tex
%    TDS:doc/latex/oberdiek/tabularht-example2.tex
%    TDS:source/latex/oberdiek/tabularht.dtx
%
%<*ignore>
\begingroup
  \catcode123=1 %
  \catcode125=2 %
  \def\x{LaTeX2e}%
\expandafter\endgroup
\ifcase 0\ifx\install y1\fi\expandafter
         \ifx\csname processbatchFile\endcsname\relax\else1\fi
         \ifx\fmtname\x\else 1\fi\relax
\else\csname fi\endcsname
%</ignore>
%<*install>
\input docstrip.tex
\Msg{************************************************************************}
\Msg{* Installation}
\Msg{* Package: tabularht 2016/05/16 v2.6 Tabular with height specified (HO)}
\Msg{************************************************************************}

\keepsilent
\askforoverwritefalse

\let\MetaPrefix\relax
\preamble

This is a generated file.

Project: tabularht
Version: 2016/05/16 v2.6

Copyright (C) 2005-2007 by
   Heiko Oberdiek <heiko.oberdiek at googlemail.com>

This work may be distributed and/or modified under the
conditions of the LaTeX Project Public License, either
version 1.3c of this license or (at your option) any later
version. This version of this license is in
   https://www.latex-project.org/lppl/lppl-1-3c.txt
and the latest version of this license is in
   https://www.latex-project.org/lppl.txt
and version 1.3 or later is part of all distributions of
LaTeX version 2005/12/01 or later.

This work has the LPPL maintenance status "maintained".

The Current Maintainers of this work are
Heiko Oberdiek and the Oberdiek Package Support Group
https://github.com/ho-tex/oberdiek/issues


This work consists of the main source file tabularht.dtx
and the derived files
   tabularht.sty, tabularht.pdf, tabularht.ins, tabularht.drv,
   tabularht-example1.tex, tabularht-example2.tex.

\endpreamble
\let\MetaPrefix\DoubleperCent

\generate{%
  \file{tabularht.ins}{\from{tabularht.dtx}{install}}%
  \file{tabularht.drv}{\from{tabularht.dtx}{driver}}%
  \usedir{tex/latex/oberdiek}%
  \file{tabularht.sty}{\from{tabularht.dtx}{package}}%
  \usedir{doc/latex/oberdiek}%
  \file{tabularht-example1.tex}{\from{tabularht.dtx}{example1}}%
  \file{tabularht-example2.tex}{\from{tabularht.dtx}{example2}}%
  \nopreamble
  \nopostamble
%  \usedir{source/latex/oberdiek/catalogue}%
%  \file{tabularht.xml}{\from{tabularht.dtx}{catalogue}}%
}

\catcode32=13\relax% active space
\let =\space%
\Msg{************************************************************************}
\Msg{*}
\Msg{* To finish the installation you have to move the following}
\Msg{* file into a directory searched by TeX:}
\Msg{*}
\Msg{*     tabularht.sty}
\Msg{*}
\Msg{* To produce the documentation run the file `tabularht.drv'}
\Msg{* through LaTeX.}
\Msg{*}
\Msg{* Happy TeXing!}
\Msg{*}
\Msg{************************************************************************}

\endbatchfile
%</install>
%<*ignore>
\fi
%</ignore>
%<*driver>
\NeedsTeXFormat{LaTeX2e}
\ProvidesFile{tabularht.drv}%
  [2016/05/16 v2.6 Tabular with height specified (HO)]%
\documentclass{ltxdoc}
\usepackage{holtxdoc}[2011/11/22]
\begin{document}
  \DocInput{tabularht.dtx}%
\end{document}
%</driver>
% \fi
%
%
% \CharacterTable
%  {Upper-case    \A\B\C\D\E\F\G\H\I\J\K\L\M\N\O\P\Q\R\S\T\U\V\W\X\Y\Z
%   Lower-case    \a\b\c\d\e\f\g\h\i\j\k\l\m\n\o\p\q\r\s\t\u\v\w\x\y\z
%   Digits        \0\1\2\3\4\5\6\7\8\9
%   Exclamation   \!     Double quote  \"     Hash (number) \#
%   Dollar        \$     Percent       \%     Ampersand     \&
%   Acute accent  \'     Left paren    \(     Right paren   \)
%   Asterisk      \*     Plus          \+     Comma         \,
%   Minus         \-     Point         \.     Solidus       \/
%   Colon         \:     Semicolon     \;     Less than     \<
%   Equals        \=     Greater than  \>     Question mark \?
%   Commercial at \@     Left bracket  \[     Backslash     \\
%   Right bracket \]     Circumflex    \^     Underscore    \_
%   Grave accent  \`     Left brace    \{     Vertical bar  \|
%   Right brace   \}     Tilde         \~}
%
% \GetFileInfo{tabularht.drv}
%
% \title{The \xpackage{tabularht} package}
% \date{2016/05/16 v2.6}
% \author{Heiko Oberdiek\thanks
% {Please report any issues at \url{https://github.com/ho-tex/oberdiek/issues}}}
%
% \maketitle
%
% \begin{abstract}
% This package defines some environments that adds
% a height specification to tabular and array.
% \end{abstract}
%
% \tableofcontents
%
% \section{Usage}
%
% \begin{quote}
%   |\usepackage{tabularht}|
% \end{quote}
% The package provides the following environments
% that extend the tabular/array environment by
% a height specification as first argument:
% \begin{itemize}
% \item |tabularht|, |tabularht*|
% \item |arrayht|
% \item |tabularhtx|
%       (if package \xpackage{tabularx} is loaded)
% \end{itemize}
% The height argument allows a length specification,
% package \xpackage{calc} is supported if used. This means,
% the tabular will have the specified height.
% You can also use the prefixes |to=| and |spread=|.
% |to=| is the default, |spread=| means, the natural
% height of the tabular box is changed by the length
% after |spread=|.
%
% Examples:
% \begin{quote}
% \begin{tabular}{@{}l@{ $\rightarrow$ }l@{}}
%   |\begin{tabularht}{1in}|
%   & height is 1in\\
%   |\begin{tabularht}{to=1in}|
%   & height is 1in\\
%   |\begin{tabularht}{spread=0pt}|
%   & natural height, same as |\begin{tabular}|\\
%   |\begin{tabularht}{spread=1in}|
%   & natural height increased by 1in
% \end{tabular}
% \end{quote}
%
% Hint:  See also package \xpackage{tabularkv}, it provides an interface,
%        where most parameters for the environments can be given
%        by key-value pairs.
%
% \begin{declcs}{interrowspace}\,|{|\dots|}|
% \end{declcs}
% Adds space between table rows. It is essentially
% the same as |\noalign{\vspace{...}}|.
% \begin{declcs}{interrowfill}
% \end{declcs}
% Short for |\interrowspace{\fill}|
% \begin{declcs}{interrowstart}\,\dots\,\cs{interrowstop}
% \end{declcs}
% Marker commands, useful for option \xoption{vlines}.
%
% \subsection{Option \xoption{vlines}}
%
% Warning: This stuff is experimental.
%
% \vspace{2ex}
% \noindent
% Vertical lines are interrupted, if space is inserted
% in \cs{noalign}, \cs{interrowspace}, \cs{addlinespace}
% (\xpackage{booktabs}),
% between double \cs{hline}s. This option tries to detect
% and add the vertical lines. The lines in a tabular
% with \xpackage{tabularht} support (environments of this package)
% are numbered from left to right. The gap that is
% controlled by \cs{interrowspace} or inbetween \cs{interrowstart}
% and \cs{interrowstop} is then filled with the detected
% vertical lines.
%
% If only a limited selection of the lines should be
% drawn, the commands know an optional argument with a
% list of line numbers, e.g.
% \begin{quote}
%\begin{verbatim}
%\begin{tabularht}{50mm}{|l|l|}
%  Hello & World\\
%  \interrowfill[1,3]
%  Foo & Bar
%\end{tabularht}
%\end{verbatim}
% \end{quote}
% There are three lines, but the middle line is not drawn
% in the gap between the first and second row.
% Zero can be used to suppress all lines:
% \begin{quote}
% |\interrowspace[0]{10mm}|
% \end{quote}
% The syntax of the commands with the optional argument
% with the line number list \meta{list}. \meta{list} is a comma
% separated list of numbers, \meta{height} means the
% height specification described above with the optional
% prefixes |to=| or |spread=|.
% \begin{quote}
%   \cs{interrowspace}\,|[|\meta{list}|]|\,|{|\meta{height}|}|\\
%   \cs{interrowfill}\,|[|\meta{list}|]|\\
%   \cs{interrowstart}\,|[|\meta{list}|]|\,\dots\,\cs{interrowstop}
% \end{quote}
%
% Option \xoption{vlines} is driver dependent and uses \eTeX\ features.
% \begin{description}
% \item[\xoption{pdftex}:] \pdfTeX\ in PDF mode.
%   Here the positions of the lines are written with the
%   help of the \cs{pdfsavepos} feature into the \xfile{.aux} file(s).
%   Therefore you need two LaTeX runs to get the lines.
% \item[\xoption{dvips}:]
%   Here, PostScript's currentpoint it used to get
%   the line positions. The lines are then drawn at the end
%   of the page. Thus one \LaTeX/dvips run is sufficient
%   for this option.
% \item[Other drivers:]\mbox{}
%   \begin{description}
%   \item[PostScript drivers:] probably possible, an end of page
%     hook would be nice.
%   \item[\VTeX:] with GeX (PostScript interpreter) probably possible.
%   \item[dvipdfm:] no idea. The big problem is, how to get the
%     current position?
%   \end{description}
% \end{description}
%
% \subsection{Limitations}
%
% \begin{itemize}
% \item
%   Vertical lines are interrupted by |\noalign{\vfill}|.
% \end{itemize}
%
% \subsection{Compatibility}
%
% \begin{itemize}
% \item
%   \xpackage{array}, \xpackage{delarray}, \xpackage{tabularx}
%   are supported.
% \item There can be problems with packages that
%   redefine \cs{@array} (or \cs{@@array}, \cs{@tabarray}) and
%   \cs{@arrayrule} (for option \xoption{vlines}).
% \item \xpackage{colortbl}: it should at least work, but there isn't
%   support for filling the gaps with color, neither
%   the rules nor the backgrounds.
% \end{itemize}
%
% \subsection{Examples}
%
% \subsubsection{Example 1}
%
%    \begin{macrocode}
%<*example1>
\documentclass{article}
\usepackage{tabularht}

\begin{document}
\fbox{%
  \begin{tabularht*}{1in}{4in}{@{}l@{\extracolsep{\fill}}r@{}}%
    upper left corner & upper right corner\\%
    \noalign{\vfill}%
    \multicolumn{2}{@{}c@{}}{bounding box}\\%
    \noalign{\vfill}%
    lower left corner & lower right corner\\%
  \end{tabularht*}%
}
\end{document}
%</example1>
%    \end{macrocode}
%
% \subsubsection{Example 2}
%
%    \begin{macrocode}
%<*example2>
\documentclass{article}
\usepackage{booktabs}
\usepackage[dvips,vlines]{tabularht}

\begin{document}

\begin{tabularht}{spread=0pt}{|l|l|}
  \hline
  First&Line\\%
  \hline
\interrowstart
  \addlinespace[10mm]%
\interrowstop
  \hline
  Second&Line\\%
\interrowstart
  \hline
  \hline
\interrowstop
  Third&Line\\%
  \hline
\interrowspace{10mm}
  \hline
  Fourth&Line\\%
  \hline
\end{tabularht}

\end{document}
%</example2>
%    \end{macrocode}
%
% \StopEventually{
% }
%
% \section{Implementation}
%
%    \begin{macrocode}
%<*package>
%    \end{macrocode}
%    Package identification.
%    \begin{macrocode}
\NeedsTeXFormat{LaTeX2e}
\ProvidesPackage{tabularht}%
  [2016/05/16 v2.6 Tabular with height specified (HO)]
%    \end{macrocode}
%
% \subsection{Environments}
%
%    \begin{macrocode}
\let\@toarrayheight\@empty
\let\tabH@array@init\@empty

\toks@={%
  \begingroup
    \long\def\x#1\vcenter\fi\fi\bgroup#2\@sharp#3#4\@nil{%
      \endgroup
      \gdef\@array[##1]##2{%
        \tabH@array@init
        #1%
        \vcenter\fi\fi
        \@toarrayheight
        \bgroup
        \let\@toarrayheight\@empty
        #2\@sharp###3#4%
      }%
    }%
  \expandafter\x\@array[#1]{#2}\@nil % hash-ok
}
\edef\tabH@patch@array{\the\toks@}
\def\tabH@patch@@array{%
  \ifx\@array\@@array
    \def\reserved@a{\let\@@array\@array}%
  \else
    \let\reserved@a\relax
  \fi
  \tabH@patch@array
  \reserved@a
}
\tabH@patch@@array

\@ifpackageloaded{array}{}{%
  \AtBeginDocument{%
    \@ifpackageloaded{array}{%
      \tabH@patch@@array
    }{}%
  }%
}

\def\tabH@setheight#1{%
  \tabH@@setheight#1==\@nil
}
\def\tabH@@setheight#1=#2=#3\@nil{%
  \ifx\\#2#3\\%
    \setlength{\dimen@}{#1}%
    \edef\@toarrayheight{to\the\dimen@}%
  \else
    \edef\tabH@temp{\zap@space#1 \@empty}%
    \ifx\tabH@temp\tabH@to
    \else
      \ifx\tabH@temp\tabH@spread
      \else
        \PackageError{tabularht}{%
          Unknown height specifier %
          `\expandafter\strip@prefix\meaning\tabH@temp'%
        }{%
          The height dimension for tabular height can be prefixed%
          \MessageBreak
          with `to=' or `spread=', default is `to='.%
        }%
        \let\tabH@temp\tabH@to
      \fi
    \fi
    \setlength{\dimen@}{#2}%
    \edef\@toarrayheight{\tabH@temp\the\dimen@}%
  \fi
}
\def\tabH@to{to}
\def\tabH@spread{spread}
%    \end{macrocode}
%    First argument is the height of the table,
%    then the orignal arguments for tabular follow.
%    \begin{macrocode}
\newenvironment{tabularht}[1]{%
  \tabH@setheight{#1}%
  \tabular
}{%
  \endtabular
}

\newenvironment{tabularht*}[1]{%
  \tabH@setheight{#1}%
  \@nameuse{tabular*}%
}{%
  \@nameuse{endtabular*}%
}

\newenvironment{tabularhtx}[1]{%
  \tabH@setheight{#1}%
  \tabularx
}{%
  \endtabularx
}

\newenvironment{arrayht}[1]{%
  \tabH@setheight{#1}%
  \array
}{%
  \endarray
}

\def\interrowspace{%
  \noalign\bgroup
    \tabH@interrowspace
}
\newcommand*{\tabH@interrowspace}[2][]{%
    \tabH@vspace{#1}{#2}%
  \egroup
}
\def\interrowfill{%
  \noalign\bgroup
    \tabH@interrowfill
}
\newcommand*{\tabH@interrowfill}[1][]{%
    \tabH@vspace{#1}{\fill}%
  \egroup
}
\def\tabH@vspace#1#2{%
  \tabH@vspace@start{#1}%
  \vspace{#2}%
  \tabH@vspace@stop
}
\let\tabH@vspace@start\@gobble
\let\tabH@vspace@stop\@empty

\newcommand*{\interrowstart}{%
  \noalign\bgroup
    \tabH@interrowstart
}
\newcommand*{\tabH@interrowstart}[1][]{%
    \tabH@vspace@start{#1}%
  \egroup
}
\newcommand*{\interrowstop}{%
  \noalign{\tabH@vspace@stop}%
}
%    \end{macrocode}
%
% \subsection{Options}
%
%    \begin{macrocode}
\providecommand*{\tabH@driver}{}

\DeclareOption{vlines}{%
  \let\tabH@temp\relax
}
\DeclareOption{pdftex}{}
\DeclareOption{dvips}{%
  \def\tabH@driver{dvips}%
}
\ProcessOptions*\relax

\ifx\tabH@temp\relax
\else
  \expandafter\endinput
\fi

\begingroup
  \@ifundefined{eTeXversion}{%
    \PackageError{tabularht}{%
      Option `vlines' requires eTeX%
    }{%
      Use of eTeX is recommended for LaTeX, see ltnews16.%
    }%
    \endgroup
    \endinput
  }{}%
\endgroup
%    \end{macrocode}
%
% \subsection{Option vlines, driver independent stuff}
%
%    \begin{macrocode}
\begingroup
  \let\@addtoreset\@gobbletwo
  \newcounter{tabH@unique}%
\endgroup
\let\tabH@currenttab\@empty

\def\tabH@array@init{%
  \ifx\@toarrayheight\@empty
    % ignore vertical lines of nested tabular environments
    \let\tabH@currenttab\@empty
  \else
    \stepcounter{tabH@unique}%
    \edef\tabH@currenttab{\the\c@tabH@unique}%
  \fi
}

\renewcommand*{\@arrayrule}{%
  \@addtopreamble{%
    \hskip -.5\arrayrulewidth
    \ifx\tabH@currenttab\@empty
    \else
      \tabH@vrule{\tabH@currenttab}%
    \fi
    \begingroup
      \expandafter\ifx\csname CT@arc@\endcsname\relax
      \else
        \expandafter\CT@arc@
      \fi
      \vline
    \endgroup
    \hskip -.5\arrayrulewidth
  }%
}
\let\tabH@arrayrule\@arrayrule
\AtBeginDocument{%
  \@ifpackageloaded{colortbl}{%
    \let\@arrayrule\tabH@arrayrule
  }{}%
}

\let\tabH@vrule\@gobble
%    \end{macrocode}
%
% \subsection{Driver pdftex}
%
%    \begin{macrocode}
\RequirePackage{ifpdf}
\ifpdf
  \begingroup
    \@ifundefined{pdfsavepos}{%
      \PackageError{tabularht}{%
        Your pdfTeX is too old%
      }{%
        \string\pdfsavepos\space is missing.%
      }%
      \endgroup
      \csname fi\endcsname
      \endinput
    }{}%

    \let\on@line\@empty
    \PackageInfo{tabularht}{%
      Using driver `pdftex' because of pdfTeX in PDF mode%
    }%
  \endgroup

  \protected\def\tabH@vrule#1{%
    \if@filesw
      \pdfsavepos
      \protected@write\@auxout{%
        \let\tabH@lastxpos\relax
      }{%
        \tabH@aux@vrule{#1}{\tabH@lastxpos}%
      }%
    \fi
  }%

  \def\tabH@lastxpos{\the\pdflastxpos}%
  \def\tabH@lastypos{\the\pdflastypos}%

  % The .aux file contains three commands:
  % \tabH@aux@vrule{tabular id}{x position}
  % \tabH@aux@vstart{tabular id}{row id}{x position}{y position}
  % \tabH@aux@vstop{y position}
  %
  \AtBeginDocument{%
    % The .aux files are read the first time before
    % \AtBeginDocument and later at \end{document}.
    % \tabH@aux@done is a marker to distinguish
    % between these two readings. Only in the first
    % case we need the \tabH@aux@... commands.
    \let\tabH@aux@done\@empty
    \if@filesw
      \immediate\write\@mainaux{%
        \@percentchar\@percentchar BeginProlog: tabularht%
      }%
      % items in the aux file are executed,
      % if tabularht is loaded
      % and during the aux file read at \begin{document} only
      \immediate\write\@mainaux{%
        \detokenize{%
          % the \tabH@aux@... commands are needed only if
          % tabularht is loaded with driver pdftex.
          \@ifundefined{tabH@aux@vrule}\@secondoftwo\@firstofone
          {%
            % disable commands except for the first .aux files reading
            \@ifundefined{tabH@aux@done}\@gobble\@firstofone
          }%
          {%
            \let\tabH@aux@vrule\@gobbletwo
            \let\tabH@aux@vstart\@gobblefour
            \let\tabH@aux@vstop\@gobble
          }%
        }%
      }%
      \immediate\write\@mainaux{%
        \@percentchar\@percentchar EndProlog: tabularht%
      }%
    \fi
  }%

  % the x positions of vrules are stored in
  % \tabH@<tabcount>list with distinct values
  \protected\def\tabH@aux@vrule#1#2{%
    \@ifundefined{tabH@#1list}{%
      \expandafter\xdef\csname tabH@#1list\endcsname{%
        \noexpand\do{#2}%
      }%
    }{%
      \begingroup
        \def\x{#2}%
        \let\y\@undefined
        \let\do\tabH@do@add
        \expandafter\xdef\csname tabH@#1list\endcsname{%
          \csname tabH@#1list\endcsname\@empty
          \ifx\y\@undefined
            \noexpand\do{\x}%
          \fi
        }%
      \endgroup
    }%
  }%
  \def\tabH@do@add#1{%
    \ifx\y\@undefined
      \ifnum#1<\x\space
      \else
        \expandafter\ifx\csname y\endcsname\relax\fi
        \ifnum#1>\x\space
          \noexpand\do{\x}%
        \fi
      \fi
    \fi
    \noexpand\do{#1}%
  }%

  \def\tabH@vspace@start#1{%
    \if@filesw
      \stepcounter{tabH@unique}%
      \edef\tabH@currentrow{\the\c@tabH@unique}%
      \pdfsavepos
      \protected@write\@auxout{%
        \let\tabH@lastxpos\relax
        \let\tabH@lastypos\relax
      }{%
        \tabH@aux@vstart{\tabH@currenttab}{\tabH@currentrow}%
                        {\tabH@lastxpos}{\tabH@lastypos}%
      }%
    \fi
    \begingroup
      \edef\a{tabH@\tabH@currenttab row\tabH@currentrow}%
      \expandafter\let\expandafter\x\csname\a x\endcsname
      \ifx\x\relax
      \else
        \expandafter\let\expandafter\y\csname\a y\endcsname
        \expandafter\let\expandafter\l
            \csname tabH@\tabH@currenttab list\endcsname
        \ifx\l\relax
        \else
          \def\f{#1}%
          \ifx\f\@empty
            \let\do\tabH@do@set
          \else
            \count@=\z@
            \let\do\tabH@do@filter
          \fi
          \setbox\z@=\hbox{\l}%
          \wd\z@=\z@
          \dp\z@=\z@
          \copy\z@
        \fi
      \fi
    \endgroup
  }%
  \def\tabH@vspace@stop{%
    \if@filesw
      \pdfsavepos
      \protected@write\@auxout{%
        \let\tabH@lastypos\relax
      }{%
        \tabH@aux@vstop{\tabH@lastypos}%
      }%
    \fi
  }%
  \def\tabH@do@set#1{%
    \hbox to \z@{%
      \hskip \dimexpr #1sp - \x sp\relax
      \vrule \@width\arrayrulewidth
             \@depth\dimexpr \y sp\relax
      \hss
    }%
  }%
  \def\tabH@do@filter{%
    \@tempswafalse
    \advance\count@\@ne
    \@for\e:=\f\do{%
      \ifnum\e=\count@
        \@tempswatrue
      \fi
    }%
    \if@tempswa
      \expandafter\tabH@do@set
    \else
      \expandafter\@gobble
    \fi
  }%

  \protected\def\tabH@aux@vstart#1#2#3#4{%
    \def\tabH@current@vstart{{#1}{#2}{#3}{#4}}%
  }%
  \protected\def\tabH@aux@vstop{%
    \expandafter\tabH@aux@v\tabH@current@vstart
  }%
  \def\tabH@aux@v#1#2#3#4#5{%
    \expandafter\gdef\csname tabH@#1row#2x\endcsname{#3}%
    \expandafter\xdef\csname tabH@#1row#2y\endcsname{%
      \the\numexpr #4 - #5\relax
    }%
  }%

  \csname fi\endcsname
  \endinput

\fi
%    \end{macrocode}
%
% \subsection{DVI drivers}
%
%    \begin{macrocode}
\ifx\tabH@driver\@empty
  \PackageError{tabularht}{%
    Missing DVI driver, option `vlines' disabled%
  }{%
    Supported DVI drivers: dvips.%
  }%
  \expandafter\endinput
\fi

\def\tabH@driver@dvips{%
  \def\tabH@literalps##1{\special{ps:SDict begin ##1 end}}%
  \def\tabH@headerps##1{\special{! ##1}}%
}

\@onelevel@sanitize\tabH@driver
\@ifundefined{tabH@driver@\tabH@driver}{%
  \PackageError{tabularht}{%
    Unsupported driver `\tabH@driver'%
  }{%
    Supported DVI drivers: dvips.%
  }%
  \endinput
}{}

\begingroup
  \let\on@line\@empty
  \PackageInfo{tabularht}{%
    Using driver `\tabH@driver'%
  }%
\endgroup
\csname tabH@driver@\tabH@driver\endcsname

\protected\def\tabH@vrule#1#2\vrule#3\arrayrulewidth{%
  #2% \fi or empty
  % hack to get rid of maxdrift rounding of dvips,
  % thus simulate a large motion
  \kern1in\relax
  \tabH@literalps{%
    #1 tabH.vrule %
    Resolution neg 0 translate%
  }%
  \vrule#3\arrayrulewidth
  \tabH@literalps{Resolution 0 translate}%
  \kern-1in\relax
}

\def\tabH@vspace@start#1{%
  \begingroup
    \let\y\@empty
    \@for\x:=#1\do{%
      \ifx\y\@empty
        \edef\y{\x}%
      \else
        \edef\y{\y\space\x}%
      \fi
    }%
    \tabH@literalps{\tabH@currenttab[\y]currentpoint exch pop}%
  \endgroup
}
\def\tabH@vspace@stop{%
  \tabH@literalps{%
    currentpoint exch pop %
    \number\dimexpr\arrayrulewidth\relax\space
    tabH.vspace%
  }%
}

\tabH@headerps{%
  userdict begin%
    /tabH.list 10 dict def%
    /tabH.job [] def %
  end%
  /tabH.vrule{%
    10 string cvs cvn dup tabH.list exch known{%
      tabH.list exch dup [ exch tabH.list exch get %
      currentpoint pop round exch true exch{%
        % tabH.list key [ ... x true i
        % tabH.list key [ ... false i
        exch{%
          % ... [ ... x i
          2 copy lt{false}{%
            2 copy eq{pop false}{exch true}ifelse%
          }ifelse%
        }{false}ifelse%
      }forall %
      pop%
      ]put%
    }{%
      tabH.list exch[currentpoint pop round]put%
    }ifelse%
  }bind def%
  % <tab num> <cols array> <ytop> <ybottom> <rulewidth[sp]>
  /tabH.vspace{%
    userdict begin %
      10 dict dup begin %
        exch 65536 div Resolution mul 72.27 div %
        % dvips uses a poor man's ceil function
        % see dopage.c before "drawrule": (int)(... + 0.9999999)
        0.9999999 add truncate%
        /rulewidth exch def %
        exch/ybottom exch def %
        exch/ytop exch def %
        exch/cols exch def %
        exch/tabkey exch 10 string cvs cvn def %
      end%
      /tabH.job exch[exch userdict/tabH.job get aload pop]def %
    end%
  }bind def %
  % Now we do the work at the end of the page.
  % Unhappily "eop-hook" cannot be used, because "eop"
  % executes "restore" before, so that all data are lost.
  TeXDict begin%
    /eop%
    [%
      {%
        tabH.job{%
          begin%
           /colarray %
             tabH.list tabkey known{tabH.list tabkey get}{[]}ifelse %
           def %
           cols length 0 eq not{%
             /colarray[%
               cols{1 sub %
                 dup 0 lt{pop}{%
                   dup colarray length ge{pop}{%
                     colarray exch get%
                   }ifelse%
                 }ifelse%
               }forall%
             ]def%
           }if %
           colarray{%
             % (rulewidth) == rulewidth == % debug
             Resolution sub %
             ytop rulewidth ytop ybottom sub v%
           }forall %
          end%
        }forall%
        % tabH.list{== ==}forall % debug
      }bind aload pop %
      TeXDict /eop get aload pop%
    ]cvx def %
  end%
}
%    \end{macrocode}
%
%    \begin{macrocode}
%</package>
%    \end{macrocode}
%
% \section{Installation}
%
% \subsection{Download}
%
% \paragraph{Package.} This package is available on
% CTAN\footnote{\CTANpkg{tabularht}}:
% \begin{description}
% \item[\CTAN{macros/latex/contrib/oberdiek/tabularht.dtx}] The source file.
% \item[\CTAN{macros/latex/contrib/oberdiek/tabularht.pdf}] Documentation.
% \end{description}
%
%
% \paragraph{Bundle.} All the packages of the bundle `oberdiek'
% are also available in a TDS compliant ZIP archive. There
% the packages are already unpacked and the documentation files
% are generated. The files and directories obey the TDS standard.
% \begin{description}
% \item[\CTANinstall{install/macros/latex/contrib/oberdiek.tds.zip}]
% \end{description}
% \emph{TDS} refers to the standard ``A Directory Structure
% for \TeX\ Files'' (\CTAN{tds/tds.pdf}). Directories
% with \xfile{texmf} in their name are usually organized this way.
%
% \subsection{Bundle installation}
%
% \paragraph{Unpacking.} Unpack the \xfile{oberdiek.tds.zip} in the
% TDS tree (also known as \xfile{texmf} tree) of your choice.
% Example (linux):
% \begin{quote}
%   |unzip oberdiek.tds.zip -d ~/texmf|
% \end{quote}
%
% \paragraph{Script installation.}
% Check the directory \xfile{TDS:scripts/oberdiek/} for
% scripts that need further installation steps.
%
% \subsection{Package installation}
%
% \paragraph{Unpacking.} The \xfile{.dtx} file is a self-extracting
% \docstrip\ archive. The files are extracted by running the
% \xfile{.dtx} through \plainTeX:
% \begin{quote}
%   \verb|tex tabularht.dtx|
% \end{quote}
%
% \paragraph{TDS.} Now the different files must be moved into
% the different directories in your installation TDS tree
% (also known as \xfile{texmf} tree):
% \begin{quote}
% \def\t{^^A
% \begin{tabular}{@{}>{\ttfamily}l@{ $\rightarrow$ }>{\ttfamily}l@{}}
%   tabularht.sty & tex/latex/oberdiek/tabularht.sty\\
%   tabularht.pdf & doc/latex/oberdiek/tabularht.pdf\\
%   tabularht-example1.tex & doc/latex/oberdiek/tabularht-example1.tex\\
%   tabularht-example2.tex & doc/latex/oberdiek/tabularht-example2.tex\\
%   tabularht.dtx & source/latex/oberdiek/tabularht.dtx\\
% \end{tabular}^^A
% }^^A
% \sbox0{\t}^^A
% \ifdim\wd0>\linewidth
%   \begingroup
%     \advance\linewidth by\leftmargin
%     \advance\linewidth by\rightmargin
%   \edef\x{\endgroup
%     \def\noexpand\lw{\the\linewidth}^^A
%   }\x
%   \def\lwbox{^^A
%     \leavevmode
%     \hbox to \linewidth{^^A
%       \kern-\leftmargin\relax
%       \hss
%       \usebox0
%       \hss
%       \kern-\rightmargin\relax
%     }^^A
%   }^^A
%   \ifdim\wd0>\lw
%     \sbox0{\small\t}^^A
%     \ifdim\wd0>\linewidth
%       \ifdim\wd0>\lw
%         \sbox0{\footnotesize\t}^^A
%         \ifdim\wd0>\linewidth
%           \ifdim\wd0>\lw
%             \sbox0{\scriptsize\t}^^A
%             \ifdim\wd0>\linewidth
%               \ifdim\wd0>\lw
%                 \sbox0{\tiny\t}^^A
%                 \ifdim\wd0>\linewidth
%                   \lwbox
%                 \else
%                   \usebox0
%                 \fi
%               \else
%                 \lwbox
%               \fi
%             \else
%               \usebox0
%             \fi
%           \else
%             \lwbox
%           \fi
%         \else
%           \usebox0
%         \fi
%       \else
%         \lwbox
%       \fi
%     \else
%       \usebox0
%     \fi
%   \else
%     \lwbox
%   \fi
% \else
%   \usebox0
% \fi
% \end{quote}
% If you have a \xfile{docstrip.cfg} that configures and enables \docstrip's
% TDS installing feature, then some files can already be in the right
% place, see the documentation of \docstrip.
%
% \subsection{Refresh file name databases}
%
% If your \TeX~distribution
% (\teTeX, \mikTeX, \dots) relies on file name databases, you must refresh
% these. For example, \teTeX\ users run \verb|texhash| or
% \verb|mktexlsr|.
%
% \subsection{Some details for the interested}
%
% \paragraph{Unpacking with \LaTeX.}
% The \xfile{.dtx} chooses its action depending on the format:
% \begin{description}
% \item[\plainTeX:] Run \docstrip\ and extract the files.
% \item[\LaTeX:] Generate the documentation.
% \end{description}
% If you insist on using \LaTeX\ for \docstrip\ (really,
% \docstrip\ does not need \LaTeX), then inform the autodetect routine
% about your intention:
% \begin{quote}
%   \verb|latex \let\install=y\input{tabularht.dtx}|
% \end{quote}
% Do not forget to quote the argument according to the demands
% of your shell.
%
% \paragraph{Generating the documentation.}
% You can use both the \xfile{.dtx} or the \xfile{.drv} to generate
% the documentation. The process can be configured by the
% configuration file \xfile{ltxdoc.cfg}. For instance, put this
% line into this file, if you want to have A4 as paper format:
% \begin{quote}
%   \verb|\PassOptionsToClass{a4paper}{article}|
% \end{quote}
% An example follows how to generate the
% documentation with pdf\LaTeX:
% \begin{quote}
%\begin{verbatim}
%pdflatex tabularht.dtx
%makeindex -s gind.ist tabularht.idx
%pdflatex tabularht.dtx
%makeindex -s gind.ist tabularht.idx
%pdflatex tabularht.dtx
%\end{verbatim}
% \end{quote}
%
% \begin{History}
%   \begin{Version}{2005/09/22 v1.0}
%   \item
%     First public version.
%   \end{Version}
%   \begin{Version}{2005/10/16 v2.0}
%   \item
%     Height specification allows
%     |to=...| or |spread=...|, default is |to=|.
%   \item
%     Option \xoption{vlines} added,
%     drivers \xoption{pdftex} and \xoption{dvips}.
%   \item
%     \cs{interrowspace}, \cs{interrowfil}, and
%     \cs{interrowstart}\dots\cs{interrowstop} added.
%   \end{Version}
%   \begin{Version}{2005/10/18 v2.1}
%   \item
%     Fix for package \xpackage{colortbl}, but the colors
%     of \xpackage{colortbl} remain unsupported.
%   \end{Version}
%   \begin{Version}{2006/02/20 v2.2}
%   \item
%     Code is not changed.
%   \item
%     DTX framework.
%   \end{Version}
%   \begin{Version}{2006/12/22 v2.3}
%   \item
%     Documentation fix.
%   \item
%     Fix in code of option vlines.
%   \end{Version}
%   \begin{Version}{2007/03/21 v2.4}
%   \item
%     Fix: Counter |tabh@unique| must not be changed by \cs{include}.
%   \end{Version}
%   \begin{Version}{2007/04/11 v2.5}
%   \item
%     Line ends sanitized.
%   \end{Version}
%   \begin{Version}{2016/05/16 v2.6}
%   \item
%     Documentation updates.
%   \end{Version}
% \end{History}
%
% \PrintIndex
%
% \Finale
\endinput

%        (quote the arguments according to the demands of your shell)
%
% Documentation:
%    (a) If tabularht.drv is present:
%           latex tabularht.drv
%    (b) Without tabularht.drv:
%           latex tabularht.dtx; ...
%    The class ltxdoc loads the configuration file ltxdoc.cfg
%    if available. Here you can specify further options, e.g.
%    use A4 as paper format:
%       \PassOptionsToClass{a4paper}{article}
%
%    Programm calls to get the documentation (example):
%       pdflatex tabularht.dtx
%       makeindex -s gind.ist tabularht.idx
%       pdflatex tabularht.dtx
%       makeindex -s gind.ist tabularht.idx
%       pdflatex tabularht.dtx
%
% Installation:
%    TDS:tex/latex/oberdiek/tabularht.sty
%    TDS:doc/latex/oberdiek/tabularht.pdf
%    TDS:doc/latex/oberdiek/tabularht-example1.tex
%    TDS:doc/latex/oberdiek/tabularht-example2.tex
%    TDS:source/latex/oberdiek/tabularht.dtx
%
%<*ignore>
\begingroup
  \catcode123=1 %
  \catcode125=2 %
  \def\x{LaTeX2e}%
\expandafter\endgroup
\ifcase 0\ifx\install y1\fi\expandafter
         \ifx\csname processbatchFile\endcsname\relax\else1\fi
         \ifx\fmtname\x\else 1\fi\relax
\else\csname fi\endcsname
%</ignore>
%<*install>
\input docstrip.tex
\Msg{************************************************************************}
\Msg{* Installation}
\Msg{* Package: tabularht 2016/05/16 v2.6 Tabular with height specified (HO)}
\Msg{************************************************************************}

\keepsilent
\askforoverwritefalse

\let\MetaPrefix\relax
\preamble

This is a generated file.

Project: tabularht
Version: 2016/05/16 v2.6

Copyright (C) 2005-2007 by
   Heiko Oberdiek <heiko.oberdiek at googlemail.com>

This work may be distributed and/or modified under the
conditions of the LaTeX Project Public License, either
version 1.3c of this license or (at your option) any later
version. This version of this license is in
   https://www.latex-project.org/lppl/lppl-1-3c.txt
and the latest version of this license is in
   https://www.latex-project.org/lppl.txt
and version 1.3 or later is part of all distributions of
LaTeX version 2005/12/01 or later.

This work has the LPPL maintenance status "maintained".

The Current Maintainers of this work are
Heiko Oberdiek and the Oberdiek Package Support Group
https://github.com/ho-tex/oberdiek/issues


This work consists of the main source file tabularht.dtx
and the derived files
   tabularht.sty, tabularht.pdf, tabularht.ins, tabularht.drv,
   tabularht-example1.tex, tabularht-example2.tex.

\endpreamble
\let\MetaPrefix\DoubleperCent

\generate{%
  \file{tabularht.ins}{\from{tabularht.dtx}{install}}%
  \file{tabularht.drv}{\from{tabularht.dtx}{driver}}%
  \usedir{tex/latex/oberdiek}%
  \file{tabularht.sty}{\from{tabularht.dtx}{package}}%
  \usedir{doc/latex/oberdiek}%
  \file{tabularht-example1.tex}{\from{tabularht.dtx}{example1}}%
  \file{tabularht-example2.tex}{\from{tabularht.dtx}{example2}}%
  \nopreamble
  \nopostamble
%  \usedir{source/latex/oberdiek/catalogue}%
%  \file{tabularht.xml}{\from{tabularht.dtx}{catalogue}}%
}

\catcode32=13\relax% active space
\let =\space%
\Msg{************************************************************************}
\Msg{*}
\Msg{* To finish the installation you have to move the following}
\Msg{* file into a directory searched by TeX:}
\Msg{*}
\Msg{*     tabularht.sty}
\Msg{*}
\Msg{* To produce the documentation run the file `tabularht.drv'}
\Msg{* through LaTeX.}
\Msg{*}
\Msg{* Happy TeXing!}
\Msg{*}
\Msg{************************************************************************}

\endbatchfile
%</install>
%<*ignore>
\fi
%</ignore>
%<*driver>
\NeedsTeXFormat{LaTeX2e}
\ProvidesFile{tabularht.drv}%
  [2016/05/16 v2.6 Tabular with height specified (HO)]%
\documentclass{ltxdoc}
\usepackage{holtxdoc}[2011/11/22]
\begin{document}
  \DocInput{tabularht.dtx}%
\end{document}
%</driver>
% \fi
%
%
% \CharacterTable
%  {Upper-case    \A\B\C\D\E\F\G\H\I\J\K\L\M\N\O\P\Q\R\S\T\U\V\W\X\Y\Z
%   Lower-case    \a\b\c\d\e\f\g\h\i\j\k\l\m\n\o\p\q\r\s\t\u\v\w\x\y\z
%   Digits        \0\1\2\3\4\5\6\7\8\9
%   Exclamation   \!     Double quote  \"     Hash (number) \#
%   Dollar        \$     Percent       \%     Ampersand     \&
%   Acute accent  \'     Left paren    \(     Right paren   \)
%   Asterisk      \*     Plus          \+     Comma         \,
%   Minus         \-     Point         \.     Solidus       \/
%   Colon         \:     Semicolon     \;     Less than     \<
%   Equals        \=     Greater than  \>     Question mark \?
%   Commercial at \@     Left bracket  \[     Backslash     \\
%   Right bracket \]     Circumflex    \^     Underscore    \_
%   Grave accent  \`     Left brace    \{     Vertical bar  \|
%   Right brace   \}     Tilde         \~}
%
% \GetFileInfo{tabularht.drv}
%
% \title{The \xpackage{tabularht} package}
% \date{2016/05/16 v2.6}
% \author{Heiko Oberdiek\thanks
% {Please report any issues at \url{https://github.com/ho-tex/oberdiek/issues}}}
%
% \maketitle
%
% \begin{abstract}
% This package defines some environments that adds
% a height specification to tabular and array.
% \end{abstract}
%
% \tableofcontents
%
% \section{Usage}
%
% \begin{quote}
%   |\usepackage{tabularht}|
% \end{quote}
% The package provides the following environments
% that extend the tabular/array environment by
% a height specification as first argument:
% \begin{itemize}
% \item |tabularht|, |tabularht*|
% \item |arrayht|
% \item |tabularhtx|
%       (if package \xpackage{tabularx} is loaded)
% \end{itemize}
% The height argument allows a length specification,
% package \xpackage{calc} is supported if used. This means,
% the tabular will have the specified height.
% You can also use the prefixes |to=| and |spread=|.
% |to=| is the default, |spread=| means, the natural
% height of the tabular box is changed by the length
% after |spread=|.
%
% Examples:
% \begin{quote}
% \begin{tabular}{@{}l@{ $\rightarrow$ }l@{}}
%   |\begin{tabularht}{1in}|
%   & height is 1in\\
%   |\begin{tabularht}{to=1in}|
%   & height is 1in\\
%   |\begin{tabularht}{spread=0pt}|
%   & natural height, same as |\begin{tabular}|\\
%   |\begin{tabularht}{spread=1in}|
%   & natural height increased by 1in
% \end{tabular}
% \end{quote}
%
% Hint:  See also package \xpackage{tabularkv}, it provides an interface,
%        where most parameters for the environments can be given
%        by key-value pairs.
%
% \begin{declcs}{interrowspace}\,|{|\dots|}|
% \end{declcs}
% Adds space between table rows. It is essentially
% the same as |\noalign{\vspace{...}}|.
% \begin{declcs}{interrowfill}
% \end{declcs}
% Short for |\interrowspace{\fill}|
% \begin{declcs}{interrowstart}\,\dots\,\cs{interrowstop}
% \end{declcs}
% Marker commands, useful for option \xoption{vlines}.
%
% \subsection{Option \xoption{vlines}}
%
% Warning: This stuff is experimental.
%
% \vspace{2ex}
% \noindent
% Vertical lines are interrupted, if space is inserted
% in \cs{noalign}, \cs{interrowspace}, \cs{addlinespace}
% (\xpackage{booktabs}),
% between double \cs{hline}s. This option tries to detect
% and add the vertical lines. The lines in a tabular
% with \xpackage{tabularht} support (environments of this package)
% are numbered from left to right. The gap that is
% controlled by \cs{interrowspace} or inbetween \cs{interrowstart}
% and \cs{interrowstop} is then filled with the detected
% vertical lines.
%
% If only a limited selection of the lines should be
% drawn, the commands know an optional argument with a
% list of line numbers, e.g.
% \begin{quote}
%\begin{verbatim}
%\begin{tabularht}{50mm}{|l|l|}
%  Hello & World\\
%  \interrowfill[1,3]
%  Foo & Bar
%\end{tabularht}
%\end{verbatim}
% \end{quote}
% There are three lines, but the middle line is not drawn
% in the gap between the first and second row.
% Zero can be used to suppress all lines:
% \begin{quote}
% |\interrowspace[0]{10mm}|
% \end{quote}
% The syntax of the commands with the optional argument
% with the line number list \meta{list}. \meta{list} is a comma
% separated list of numbers, \meta{height} means the
% height specification described above with the optional
% prefixes |to=| or |spread=|.
% \begin{quote}
%   \cs{interrowspace}\,|[|\meta{list}|]|\,|{|\meta{height}|}|\\
%   \cs{interrowfill}\,|[|\meta{list}|]|\\
%   \cs{interrowstart}\,|[|\meta{list}|]|\,\dots\,\cs{interrowstop}
% \end{quote}
%
% Option \xoption{vlines} is driver dependent and uses \eTeX\ features.
% \begin{description}
% \item[\xoption{pdftex}:] \pdfTeX\ in PDF mode.
%   Here the positions of the lines are written with the
%   help of the \cs{pdfsavepos} feature into the \xfile{.aux} file(s).
%   Therefore you need two LaTeX runs to get the lines.
% \item[\xoption{dvips}:]
%   Here, PostScript's currentpoint it used to get
%   the line positions. The lines are then drawn at the end
%   of the page. Thus one \LaTeX/dvips run is sufficient
%   for this option.
% \item[Other drivers:]\mbox{}
%   \begin{description}
%   \item[PostScript drivers:] probably possible, an end of page
%     hook would be nice.
%   \item[\VTeX:] with GeX (PostScript interpreter) probably possible.
%   \item[dvipdfm:] no idea. The big problem is, how to get the
%     current position?
%   \end{description}
% \end{description}
%
% \subsection{Limitations}
%
% \begin{itemize}
% \item
%   Vertical lines are interrupted by |\noalign{\vfill}|.
% \end{itemize}
%
% \subsection{Compatibility}
%
% \begin{itemize}
% \item
%   \xpackage{array}, \xpackage{delarray}, \xpackage{tabularx}
%   are supported.
% \item There can be problems with packages that
%   redefine \cs{@array} (or \cs{@@array}, \cs{@tabarray}) and
%   \cs{@arrayrule} (for option \xoption{vlines}).
% \item \xpackage{colortbl}: it should at least work, but there isn't
%   support for filling the gaps with color, neither
%   the rules nor the backgrounds.
% \end{itemize}
%
% \subsection{Examples}
%
% \subsubsection{Example 1}
%
%    \begin{macrocode}
%<*example1>
\documentclass{article}
\usepackage{tabularht}

\begin{document}
\fbox{%
  \begin{tabularht*}{1in}{4in}{@{}l@{\extracolsep{\fill}}r@{}}%
    upper left corner & upper right corner\\%
    \noalign{\vfill}%
    \multicolumn{2}{@{}c@{}}{bounding box}\\%
    \noalign{\vfill}%
    lower left corner & lower right corner\\%
  \end{tabularht*}%
}
\end{document}
%</example1>
%    \end{macrocode}
%
% \subsubsection{Example 2}
%
%    \begin{macrocode}
%<*example2>
\documentclass{article}
\usepackage{booktabs}
\usepackage[dvips,vlines]{tabularht}

\begin{document}

\begin{tabularht}{spread=0pt}{|l|l|}
  \hline
  First&Line\\%
  \hline
\interrowstart
  \addlinespace[10mm]%
\interrowstop
  \hline
  Second&Line\\%
\interrowstart
  \hline
  \hline
\interrowstop
  Third&Line\\%
  \hline
\interrowspace{10mm}
  \hline
  Fourth&Line\\%
  \hline
\end{tabularht}

\end{document}
%</example2>
%    \end{macrocode}
%
% \StopEventually{
% }
%
% \section{Implementation}
%
%    \begin{macrocode}
%<*package>
%    \end{macrocode}
%    Package identification.
%    \begin{macrocode}
\NeedsTeXFormat{LaTeX2e}
\ProvidesPackage{tabularht}%
  [2016/05/16 v2.6 Tabular with height specified (HO)]
%    \end{macrocode}
%
% \subsection{Environments}
%
%    \begin{macrocode}
\let\@toarrayheight\@empty
\let\tabH@array@init\@empty

\toks@={%
  \begingroup
    \long\def\x#1\vcenter\fi\fi\bgroup#2\@sharp#3#4\@nil{%
      \endgroup
      \gdef\@array[##1]##2{%
        \tabH@array@init
        #1%
        \vcenter\fi\fi
        \@toarrayheight
        \bgroup
        \let\@toarrayheight\@empty
        #2\@sharp###3#4%
      }%
    }%
  \expandafter\x\@array[#1]{#2}\@nil % hash-ok
}
\edef\tabH@patch@array{\the\toks@}
\def\tabH@patch@@array{%
  \ifx\@array\@@array
    \def\reserved@a{\let\@@array\@array}%
  \else
    \let\reserved@a\relax
  \fi
  \tabH@patch@array
  \reserved@a
}
\tabH@patch@@array

\@ifpackageloaded{array}{}{%
  \AtBeginDocument{%
    \@ifpackageloaded{array}{%
      \tabH@patch@@array
    }{}%
  }%
}

\def\tabH@setheight#1{%
  \tabH@@setheight#1==\@nil
}
\def\tabH@@setheight#1=#2=#3\@nil{%
  \ifx\\#2#3\\%
    \setlength{\dimen@}{#1}%
    \edef\@toarrayheight{to\the\dimen@}%
  \else
    \edef\tabH@temp{\zap@space#1 \@empty}%
    \ifx\tabH@temp\tabH@to
    \else
      \ifx\tabH@temp\tabH@spread
      \else
        \PackageError{tabularht}{%
          Unknown height specifier %
          `\expandafter\strip@prefix\meaning\tabH@temp'%
        }{%
          The height dimension for tabular height can be prefixed%
          \MessageBreak
          with `to=' or `spread=', default is `to='.%
        }%
        \let\tabH@temp\tabH@to
      \fi
    \fi
    \setlength{\dimen@}{#2}%
    \edef\@toarrayheight{\tabH@temp\the\dimen@}%
  \fi
}
\def\tabH@to{to}
\def\tabH@spread{spread}
%    \end{macrocode}
%    First argument is the height of the table,
%    then the orignal arguments for tabular follow.
%    \begin{macrocode}
\newenvironment{tabularht}[1]{%
  \tabH@setheight{#1}%
  \tabular
}{%
  \endtabular
}

\newenvironment{tabularht*}[1]{%
  \tabH@setheight{#1}%
  \@nameuse{tabular*}%
}{%
  \@nameuse{endtabular*}%
}

\newenvironment{tabularhtx}[1]{%
  \tabH@setheight{#1}%
  \tabularx
}{%
  \endtabularx
}

\newenvironment{arrayht}[1]{%
  \tabH@setheight{#1}%
  \array
}{%
  \endarray
}

\def\interrowspace{%
  \noalign\bgroup
    \tabH@interrowspace
}
\newcommand*{\tabH@interrowspace}[2][]{%
    \tabH@vspace{#1}{#2}%
  \egroup
}
\def\interrowfill{%
  \noalign\bgroup
    \tabH@interrowfill
}
\newcommand*{\tabH@interrowfill}[1][]{%
    \tabH@vspace{#1}{\fill}%
  \egroup
}
\def\tabH@vspace#1#2{%
  \tabH@vspace@start{#1}%
  \vspace{#2}%
  \tabH@vspace@stop
}
\let\tabH@vspace@start\@gobble
\let\tabH@vspace@stop\@empty

\newcommand*{\interrowstart}{%
  \noalign\bgroup
    \tabH@interrowstart
}
\newcommand*{\tabH@interrowstart}[1][]{%
    \tabH@vspace@start{#1}%
  \egroup
}
\newcommand*{\interrowstop}{%
  \noalign{\tabH@vspace@stop}%
}
%    \end{macrocode}
%
% \subsection{Options}
%
%    \begin{macrocode}
\providecommand*{\tabH@driver}{}

\DeclareOption{vlines}{%
  \let\tabH@temp\relax
}
\DeclareOption{pdftex}{}
\DeclareOption{dvips}{%
  \def\tabH@driver{dvips}%
}
\ProcessOptions*\relax

\ifx\tabH@temp\relax
\else
  \expandafter\endinput
\fi

\begingroup
  \@ifundefined{eTeXversion}{%
    \PackageError{tabularht}{%
      Option `vlines' requires eTeX%
    }{%
      Use of eTeX is recommended for LaTeX, see ltnews16.%
    }%
    \endgroup
    \endinput
  }{}%
\endgroup
%    \end{macrocode}
%
% \subsection{Option vlines, driver independent stuff}
%
%    \begin{macrocode}
\begingroup
  \let\@addtoreset\@gobbletwo
  \newcounter{tabH@unique}%
\endgroup
\let\tabH@currenttab\@empty

\def\tabH@array@init{%
  \ifx\@toarrayheight\@empty
    % ignore vertical lines of nested tabular environments
    \let\tabH@currenttab\@empty
  \else
    \stepcounter{tabH@unique}%
    \edef\tabH@currenttab{\the\c@tabH@unique}%
  \fi
}

\renewcommand*{\@arrayrule}{%
  \@addtopreamble{%
    \hskip -.5\arrayrulewidth
    \ifx\tabH@currenttab\@empty
    \else
      \tabH@vrule{\tabH@currenttab}%
    \fi
    \begingroup
      \expandafter\ifx\csname CT@arc@\endcsname\relax
      \else
        \expandafter\CT@arc@
      \fi
      \vline
    \endgroup
    \hskip -.5\arrayrulewidth
  }%
}
\let\tabH@arrayrule\@arrayrule
\AtBeginDocument{%
  \@ifpackageloaded{colortbl}{%
    \let\@arrayrule\tabH@arrayrule
  }{}%
}

\let\tabH@vrule\@gobble
%    \end{macrocode}
%
% \subsection{Driver pdftex}
%
%    \begin{macrocode}
\RequirePackage{ifpdf}
\ifpdf
  \begingroup
    \@ifundefined{pdfsavepos}{%
      \PackageError{tabularht}{%
        Your pdfTeX is too old%
      }{%
        \string\pdfsavepos\space is missing.%
      }%
      \endgroup
      \csname fi\endcsname
      \endinput
    }{}%

    \let\on@line\@empty
    \PackageInfo{tabularht}{%
      Using driver `pdftex' because of pdfTeX in PDF mode%
    }%
  \endgroup

  \protected\def\tabH@vrule#1{%
    \if@filesw
      \pdfsavepos
      \protected@write\@auxout{%
        \let\tabH@lastxpos\relax
      }{%
        \tabH@aux@vrule{#1}{\tabH@lastxpos}%
      }%
    \fi
  }%

  \def\tabH@lastxpos{\the\pdflastxpos}%
  \def\tabH@lastypos{\the\pdflastypos}%

  % The .aux file contains three commands:
  % \tabH@aux@vrule{tabular id}{x position}
  % \tabH@aux@vstart{tabular id}{row id}{x position}{y position}
  % \tabH@aux@vstop{y position}
  %
  \AtBeginDocument{%
    % The .aux files are read the first time before
    % \AtBeginDocument and later at \end{document}.
    % \tabH@aux@done is a marker to distinguish
    % between these two readings. Only in the first
    % case we need the \tabH@aux@... commands.
    \let\tabH@aux@done\@empty
    \if@filesw
      \immediate\write\@mainaux{%
        \@percentchar\@percentchar BeginProlog: tabularht%
      }%
      % items in the aux file are executed,
      % if tabularht is loaded
      % and during the aux file read at \begin{document} only
      \immediate\write\@mainaux{%
        \detokenize{%
          % the \tabH@aux@... commands are needed only if
          % tabularht is loaded with driver pdftex.
          \@ifundefined{tabH@aux@vrule}\@secondoftwo\@firstofone
          {%
            % disable commands except for the first .aux files reading
            \@ifundefined{tabH@aux@done}\@gobble\@firstofone
          }%
          {%
            \let\tabH@aux@vrule\@gobbletwo
            \let\tabH@aux@vstart\@gobblefour
            \let\tabH@aux@vstop\@gobble
          }%
        }%
      }%
      \immediate\write\@mainaux{%
        \@percentchar\@percentchar EndProlog: tabularht%
      }%
    \fi
  }%

  % the x positions of vrules are stored in
  % \tabH@<tabcount>list with distinct values
  \protected\def\tabH@aux@vrule#1#2{%
    \@ifundefined{tabH@#1list}{%
      \expandafter\xdef\csname tabH@#1list\endcsname{%
        \noexpand\do{#2}%
      }%
    }{%
      \begingroup
        \def\x{#2}%
        \let\y\@undefined
        \let\do\tabH@do@add
        \expandafter\xdef\csname tabH@#1list\endcsname{%
          \csname tabH@#1list\endcsname\@empty
          \ifx\y\@undefined
            \noexpand\do{\x}%
          \fi
        }%
      \endgroup
    }%
  }%
  \def\tabH@do@add#1{%
    \ifx\y\@undefined
      \ifnum#1<\x\space
      \else
        \expandafter\ifx\csname y\endcsname\relax\fi
        \ifnum#1>\x\space
          \noexpand\do{\x}%
        \fi
      \fi
    \fi
    \noexpand\do{#1}%
  }%

  \def\tabH@vspace@start#1{%
    \if@filesw
      \stepcounter{tabH@unique}%
      \edef\tabH@currentrow{\the\c@tabH@unique}%
      \pdfsavepos
      \protected@write\@auxout{%
        \let\tabH@lastxpos\relax
        \let\tabH@lastypos\relax
      }{%
        \tabH@aux@vstart{\tabH@currenttab}{\tabH@currentrow}%
                        {\tabH@lastxpos}{\tabH@lastypos}%
      }%
    \fi
    \begingroup
      \edef\a{tabH@\tabH@currenttab row\tabH@currentrow}%
      \expandafter\let\expandafter\x\csname\a x\endcsname
      \ifx\x\relax
      \else
        \expandafter\let\expandafter\y\csname\a y\endcsname
        \expandafter\let\expandafter\l
            \csname tabH@\tabH@currenttab list\endcsname
        \ifx\l\relax
        \else
          \def\f{#1}%
          \ifx\f\@empty
            \let\do\tabH@do@set
          \else
            \count@=\z@
            \let\do\tabH@do@filter
          \fi
          \setbox\z@=\hbox{\l}%
          \wd\z@=\z@
          \dp\z@=\z@
          \copy\z@
        \fi
      \fi
    \endgroup
  }%
  \def\tabH@vspace@stop{%
    \if@filesw
      \pdfsavepos
      \protected@write\@auxout{%
        \let\tabH@lastypos\relax
      }{%
        \tabH@aux@vstop{\tabH@lastypos}%
      }%
    \fi
  }%
  \def\tabH@do@set#1{%
    \hbox to \z@{%
      \hskip \dimexpr #1sp - \x sp\relax
      \vrule \@width\arrayrulewidth
             \@depth\dimexpr \y sp\relax
      \hss
    }%
  }%
  \def\tabH@do@filter{%
    \@tempswafalse
    \advance\count@\@ne
    \@for\e:=\f\do{%
      \ifnum\e=\count@
        \@tempswatrue
      \fi
    }%
    \if@tempswa
      \expandafter\tabH@do@set
    \else
      \expandafter\@gobble
    \fi
  }%

  \protected\def\tabH@aux@vstart#1#2#3#4{%
    \def\tabH@current@vstart{{#1}{#2}{#3}{#4}}%
  }%
  \protected\def\tabH@aux@vstop{%
    \expandafter\tabH@aux@v\tabH@current@vstart
  }%
  \def\tabH@aux@v#1#2#3#4#5{%
    \expandafter\gdef\csname tabH@#1row#2x\endcsname{#3}%
    \expandafter\xdef\csname tabH@#1row#2y\endcsname{%
      \the\numexpr #4 - #5\relax
    }%
  }%

  \csname fi\endcsname
  \endinput

\fi
%    \end{macrocode}
%
% \subsection{DVI drivers}
%
%    \begin{macrocode}
\ifx\tabH@driver\@empty
  \PackageError{tabularht}{%
    Missing DVI driver, option `vlines' disabled%
  }{%
    Supported DVI drivers: dvips.%
  }%
  \expandafter\endinput
\fi

\def\tabH@driver@dvips{%
  \def\tabH@literalps##1{\special{ps:SDict begin ##1 end}}%
  \def\tabH@headerps##1{\special{! ##1}}%
}

\@onelevel@sanitize\tabH@driver
\@ifundefined{tabH@driver@\tabH@driver}{%
  \PackageError{tabularht}{%
    Unsupported driver `\tabH@driver'%
  }{%
    Supported DVI drivers: dvips.%
  }%
  \endinput
}{}

\begingroup
  \let\on@line\@empty
  \PackageInfo{tabularht}{%
    Using driver `\tabH@driver'%
  }%
\endgroup
\csname tabH@driver@\tabH@driver\endcsname

\protected\def\tabH@vrule#1#2\vrule#3\arrayrulewidth{%
  #2% \fi or empty
  % hack to get rid of maxdrift rounding of dvips,
  % thus simulate a large motion
  \kern1in\relax
  \tabH@literalps{%
    #1 tabH.vrule %
    Resolution neg 0 translate%
  }%
  \vrule#3\arrayrulewidth
  \tabH@literalps{Resolution 0 translate}%
  \kern-1in\relax
}

\def\tabH@vspace@start#1{%
  \begingroup
    \let\y\@empty
    \@for\x:=#1\do{%
      \ifx\y\@empty
        \edef\y{\x}%
      \else
        \edef\y{\y\space\x}%
      \fi
    }%
    \tabH@literalps{\tabH@currenttab[\y]currentpoint exch pop}%
  \endgroup
}
\def\tabH@vspace@stop{%
  \tabH@literalps{%
    currentpoint exch pop %
    \number\dimexpr\arrayrulewidth\relax\space
    tabH.vspace%
  }%
}

\tabH@headerps{%
  userdict begin%
    /tabH.list 10 dict def%
    /tabH.job [] def %
  end%
  /tabH.vrule{%
    10 string cvs cvn dup tabH.list exch known{%
      tabH.list exch dup [ exch tabH.list exch get %
      currentpoint pop round exch true exch{%
        % tabH.list key [ ... x true i
        % tabH.list key [ ... false i
        exch{%
          % ... [ ... x i
          2 copy lt{false}{%
            2 copy eq{pop false}{exch true}ifelse%
          }ifelse%
        }{false}ifelse%
      }forall %
      pop%
      ]put%
    }{%
      tabH.list exch[currentpoint pop round]put%
    }ifelse%
  }bind def%
  % <tab num> <cols array> <ytop> <ybottom> <rulewidth[sp]>
  /tabH.vspace{%
    userdict begin %
      10 dict dup begin %
        exch 65536 div Resolution mul 72.27 div %
        % dvips uses a poor man's ceil function
        % see dopage.c before "drawrule": (int)(... + 0.9999999)
        0.9999999 add truncate%
        /rulewidth exch def %
        exch/ybottom exch def %
        exch/ytop exch def %
        exch/cols exch def %
        exch/tabkey exch 10 string cvs cvn def %
      end%
      /tabH.job exch[exch userdict/tabH.job get aload pop]def %
    end%
  }bind def %
  % Now we do the work at the end of the page.
  % Unhappily "eop-hook" cannot be used, because "eop"
  % executes "restore" before, so that all data are lost.
  TeXDict begin%
    /eop%
    [%
      {%
        tabH.job{%
          begin%
           /colarray %
             tabH.list tabkey known{tabH.list tabkey get}{[]}ifelse %
           def %
           cols length 0 eq not{%
             /colarray[%
               cols{1 sub %
                 dup 0 lt{pop}{%
                   dup colarray length ge{pop}{%
                     colarray exch get%
                   }ifelse%
                 }ifelse%
               }forall%
             ]def%
           }if %
           colarray{%
             % (rulewidth) == rulewidth == % debug
             Resolution sub %
             ytop rulewidth ytop ybottom sub v%
           }forall %
          end%
        }forall%
        % tabH.list{== ==}forall % debug
      }bind aload pop %
      TeXDict /eop get aload pop%
    ]cvx def %
  end%
}
%    \end{macrocode}
%
%    \begin{macrocode}
%</package>
%    \end{macrocode}
%
% \section{Installation}
%
% \subsection{Download}
%
% \paragraph{Package.} This package is available on
% CTAN\footnote{\CTANpkg{tabularht}}:
% \begin{description}
% \item[\CTAN{macros/latex/contrib/oberdiek/tabularht.dtx}] The source file.
% \item[\CTAN{macros/latex/contrib/oberdiek/tabularht.pdf}] Documentation.
% \end{description}
%
%
% \paragraph{Bundle.} All the packages of the bundle `oberdiek'
% are also available in a TDS compliant ZIP archive. There
% the packages are already unpacked and the documentation files
% are generated. The files and directories obey the TDS standard.
% \begin{description}
% \item[\CTANinstall{install/macros/latex/contrib/oberdiek.tds.zip}]
% \end{description}
% \emph{TDS} refers to the standard ``A Directory Structure
% for \TeX\ Files'' (\CTAN{tds/tds.pdf}). Directories
% with \xfile{texmf} in their name are usually organized this way.
%
% \subsection{Bundle installation}
%
% \paragraph{Unpacking.} Unpack the \xfile{oberdiek.tds.zip} in the
% TDS tree (also known as \xfile{texmf} tree) of your choice.
% Example (linux):
% \begin{quote}
%   |unzip oberdiek.tds.zip -d ~/texmf|
% \end{quote}
%
% \paragraph{Script installation.}
% Check the directory \xfile{TDS:scripts/oberdiek/} for
% scripts that need further installation steps.
%
% \subsection{Package installation}
%
% \paragraph{Unpacking.} The \xfile{.dtx} file is a self-extracting
% \docstrip\ archive. The files are extracted by running the
% \xfile{.dtx} through \plainTeX:
% \begin{quote}
%   \verb|tex tabularht.dtx|
% \end{quote}
%
% \paragraph{TDS.} Now the different files must be moved into
% the different directories in your installation TDS tree
% (also known as \xfile{texmf} tree):
% \begin{quote}
% \def\t{^^A
% \begin{tabular}{@{}>{\ttfamily}l@{ $\rightarrow$ }>{\ttfamily}l@{}}
%   tabularht.sty & tex/latex/oberdiek/tabularht.sty\\
%   tabularht.pdf & doc/latex/oberdiek/tabularht.pdf\\
%   tabularht-example1.tex & doc/latex/oberdiek/tabularht-example1.tex\\
%   tabularht-example2.tex & doc/latex/oberdiek/tabularht-example2.tex\\
%   tabularht.dtx & source/latex/oberdiek/tabularht.dtx\\
% \end{tabular}^^A
% }^^A
% \sbox0{\t}^^A
% \ifdim\wd0>\linewidth
%   \begingroup
%     \advance\linewidth by\leftmargin
%     \advance\linewidth by\rightmargin
%   \edef\x{\endgroup
%     \def\noexpand\lw{\the\linewidth}^^A
%   }\x
%   \def\lwbox{^^A
%     \leavevmode
%     \hbox to \linewidth{^^A
%       \kern-\leftmargin\relax
%       \hss
%       \usebox0
%       \hss
%       \kern-\rightmargin\relax
%     }^^A
%   }^^A
%   \ifdim\wd0>\lw
%     \sbox0{\small\t}^^A
%     \ifdim\wd0>\linewidth
%       \ifdim\wd0>\lw
%         \sbox0{\footnotesize\t}^^A
%         \ifdim\wd0>\linewidth
%           \ifdim\wd0>\lw
%             \sbox0{\scriptsize\t}^^A
%             \ifdim\wd0>\linewidth
%               \ifdim\wd0>\lw
%                 \sbox0{\tiny\t}^^A
%                 \ifdim\wd0>\linewidth
%                   \lwbox
%                 \else
%                   \usebox0
%                 \fi
%               \else
%                 \lwbox
%               \fi
%             \else
%               \usebox0
%             \fi
%           \else
%             \lwbox
%           \fi
%         \else
%           \usebox0
%         \fi
%       \else
%         \lwbox
%       \fi
%     \else
%       \usebox0
%     \fi
%   \else
%     \lwbox
%   \fi
% \else
%   \usebox0
% \fi
% \end{quote}
% If you have a \xfile{docstrip.cfg} that configures and enables \docstrip's
% TDS installing feature, then some files can already be in the right
% place, see the documentation of \docstrip.
%
% \subsection{Refresh file name databases}
%
% If your \TeX~distribution
% (\teTeX, \mikTeX, \dots) relies on file name databases, you must refresh
% these. For example, \teTeX\ users run \verb|texhash| or
% \verb|mktexlsr|.
%
% \subsection{Some details for the interested}
%
% \paragraph{Unpacking with \LaTeX.}
% The \xfile{.dtx} chooses its action depending on the format:
% \begin{description}
% \item[\plainTeX:] Run \docstrip\ and extract the files.
% \item[\LaTeX:] Generate the documentation.
% \end{description}
% If you insist on using \LaTeX\ for \docstrip\ (really,
% \docstrip\ does not need \LaTeX), then inform the autodetect routine
% about your intention:
% \begin{quote}
%   \verb|latex \let\install=y% \iffalse meta-comment
%
% File: tabularht.dtx
% Version: 2016/05/16 v2.6
% Info: Tabular with height specified
%
% Copyright (C) 2005-2007 by
%    Heiko Oberdiek <heiko.oberdiek at googlemail.com>
%    2016
%    https://github.com/ho-tex/oberdiek/issues
%
% This work may be distributed and/or modified under the
% conditions of the LaTeX Project Public License, either
% version 1.3c of this license or (at your option) any later
% version. This version of this license is in
%    https://www.latex-project.org/lppl/lppl-1-3c.txt
% and the latest version of this license is in
%    https://www.latex-project.org/lppl.txt
% and version 1.3 or later is part of all distributions of
% LaTeX version 2005/12/01 or later.
%
% This work has the LPPL maintenance status "maintained".
%
% The Current Maintainers of this work are
% Heiko Oberdiek and the Oberdiek Package Support Group
% https://github.com/ho-tex/oberdiek/issues
%
% This work consists of the main source file tabularht.dtx
% and the derived files
%    tabularht.sty, tabularht.pdf, tabularht.ins, tabularht.drv,
%    tabularht-example1.tex, tabularht-example2.tex.
%
% Distribution:
%    CTAN:macros/latex/contrib/oberdiek/tabularht.dtx
%    CTAN:macros/latex/contrib/oberdiek/tabularht.pdf
%
% Unpacking:
%    (a) If tabularht.ins is present:
%           tex tabularht.ins
%    (b) Without tabularht.ins:
%           tex tabularht.dtx
%    (c) If you insist on using LaTeX
%           latex \let\install=y\input{tabularht.dtx}
%        (quote the arguments according to the demands of your shell)
%
% Documentation:
%    (a) If tabularht.drv is present:
%           latex tabularht.drv
%    (b) Without tabularht.drv:
%           latex tabularht.dtx; ...
%    The class ltxdoc loads the configuration file ltxdoc.cfg
%    if available. Here you can specify further options, e.g.
%    use A4 as paper format:
%       \PassOptionsToClass{a4paper}{article}
%
%    Programm calls to get the documentation (example):
%       pdflatex tabularht.dtx
%       makeindex -s gind.ist tabularht.idx
%       pdflatex tabularht.dtx
%       makeindex -s gind.ist tabularht.idx
%       pdflatex tabularht.dtx
%
% Installation:
%    TDS:tex/latex/oberdiek/tabularht.sty
%    TDS:doc/latex/oberdiek/tabularht.pdf
%    TDS:doc/latex/oberdiek/tabularht-example1.tex
%    TDS:doc/latex/oberdiek/tabularht-example2.tex
%    TDS:source/latex/oberdiek/tabularht.dtx
%
%<*ignore>
\begingroup
  \catcode123=1 %
  \catcode125=2 %
  \def\x{LaTeX2e}%
\expandafter\endgroup
\ifcase 0\ifx\install y1\fi\expandafter
         \ifx\csname processbatchFile\endcsname\relax\else1\fi
         \ifx\fmtname\x\else 1\fi\relax
\else\csname fi\endcsname
%</ignore>
%<*install>
\input docstrip.tex
\Msg{************************************************************************}
\Msg{* Installation}
\Msg{* Package: tabularht 2016/05/16 v2.6 Tabular with height specified (HO)}
\Msg{************************************************************************}

\keepsilent
\askforoverwritefalse

\let\MetaPrefix\relax
\preamble

This is a generated file.

Project: tabularht
Version: 2016/05/16 v2.6

Copyright (C) 2005-2007 by
   Heiko Oberdiek <heiko.oberdiek at googlemail.com>

This work may be distributed and/or modified under the
conditions of the LaTeX Project Public License, either
version 1.3c of this license or (at your option) any later
version. This version of this license is in
   https://www.latex-project.org/lppl/lppl-1-3c.txt
and the latest version of this license is in
   https://www.latex-project.org/lppl.txt
and version 1.3 or later is part of all distributions of
LaTeX version 2005/12/01 or later.

This work has the LPPL maintenance status "maintained".

The Current Maintainers of this work are
Heiko Oberdiek and the Oberdiek Package Support Group
https://github.com/ho-tex/oberdiek/issues


This work consists of the main source file tabularht.dtx
and the derived files
   tabularht.sty, tabularht.pdf, tabularht.ins, tabularht.drv,
   tabularht-example1.tex, tabularht-example2.tex.

\endpreamble
\let\MetaPrefix\DoubleperCent

\generate{%
  \file{tabularht.ins}{\from{tabularht.dtx}{install}}%
  \file{tabularht.drv}{\from{tabularht.dtx}{driver}}%
  \usedir{tex/latex/oberdiek}%
  \file{tabularht.sty}{\from{tabularht.dtx}{package}}%
  \usedir{doc/latex/oberdiek}%
  \file{tabularht-example1.tex}{\from{tabularht.dtx}{example1}}%
  \file{tabularht-example2.tex}{\from{tabularht.dtx}{example2}}%
  \nopreamble
  \nopostamble
%  \usedir{source/latex/oberdiek/catalogue}%
%  \file{tabularht.xml}{\from{tabularht.dtx}{catalogue}}%
}

\catcode32=13\relax% active space
\let =\space%
\Msg{************************************************************************}
\Msg{*}
\Msg{* To finish the installation you have to move the following}
\Msg{* file into a directory searched by TeX:}
\Msg{*}
\Msg{*     tabularht.sty}
\Msg{*}
\Msg{* To produce the documentation run the file `tabularht.drv'}
\Msg{* through LaTeX.}
\Msg{*}
\Msg{* Happy TeXing!}
\Msg{*}
\Msg{************************************************************************}

\endbatchfile
%</install>
%<*ignore>
\fi
%</ignore>
%<*driver>
\NeedsTeXFormat{LaTeX2e}
\ProvidesFile{tabularht.drv}%
  [2016/05/16 v2.6 Tabular with height specified (HO)]%
\documentclass{ltxdoc}
\usepackage{holtxdoc}[2011/11/22]
\begin{document}
  \DocInput{tabularht.dtx}%
\end{document}
%</driver>
% \fi
%
%
% \CharacterTable
%  {Upper-case    \A\B\C\D\E\F\G\H\I\J\K\L\M\N\O\P\Q\R\S\T\U\V\W\X\Y\Z
%   Lower-case    \a\b\c\d\e\f\g\h\i\j\k\l\m\n\o\p\q\r\s\t\u\v\w\x\y\z
%   Digits        \0\1\2\3\4\5\6\7\8\9
%   Exclamation   \!     Double quote  \"     Hash (number) \#
%   Dollar        \$     Percent       \%     Ampersand     \&
%   Acute accent  \'     Left paren    \(     Right paren   \)
%   Asterisk      \*     Plus          \+     Comma         \,
%   Minus         \-     Point         \.     Solidus       \/
%   Colon         \:     Semicolon     \;     Less than     \<
%   Equals        \=     Greater than  \>     Question mark \?
%   Commercial at \@     Left bracket  \[     Backslash     \\
%   Right bracket \]     Circumflex    \^     Underscore    \_
%   Grave accent  \`     Left brace    \{     Vertical bar  \|
%   Right brace   \}     Tilde         \~}
%
% \GetFileInfo{tabularht.drv}
%
% \title{The \xpackage{tabularht} package}
% \date{2016/05/16 v2.6}
% \author{Heiko Oberdiek\thanks
% {Please report any issues at \url{https://github.com/ho-tex/oberdiek/issues}}}
%
% \maketitle
%
% \begin{abstract}
% This package defines some environments that adds
% a height specification to tabular and array.
% \end{abstract}
%
% \tableofcontents
%
% \section{Usage}
%
% \begin{quote}
%   |\usepackage{tabularht}|
% \end{quote}
% The package provides the following environments
% that extend the tabular/array environment by
% a height specification as first argument:
% \begin{itemize}
% \item |tabularht|, |tabularht*|
% \item |arrayht|
% \item |tabularhtx|
%       (if package \xpackage{tabularx} is loaded)
% \end{itemize}
% The height argument allows a length specification,
% package \xpackage{calc} is supported if used. This means,
% the tabular will have the specified height.
% You can also use the prefixes |to=| and |spread=|.
% |to=| is the default, |spread=| means, the natural
% height of the tabular box is changed by the length
% after |spread=|.
%
% Examples:
% \begin{quote}
% \begin{tabular}{@{}l@{ $\rightarrow$ }l@{}}
%   |\begin{tabularht}{1in}|
%   & height is 1in\\
%   |\begin{tabularht}{to=1in}|
%   & height is 1in\\
%   |\begin{tabularht}{spread=0pt}|
%   & natural height, same as |\begin{tabular}|\\
%   |\begin{tabularht}{spread=1in}|
%   & natural height increased by 1in
% \end{tabular}
% \end{quote}
%
% Hint:  See also package \xpackage{tabularkv}, it provides an interface,
%        where most parameters for the environments can be given
%        by key-value pairs.
%
% \begin{declcs}{interrowspace}\,|{|\dots|}|
% \end{declcs}
% Adds space between table rows. It is essentially
% the same as |\noalign{\vspace{...}}|.
% \begin{declcs}{interrowfill}
% \end{declcs}
% Short for |\interrowspace{\fill}|
% \begin{declcs}{interrowstart}\,\dots\,\cs{interrowstop}
% \end{declcs}
% Marker commands, useful for option \xoption{vlines}.
%
% \subsection{Option \xoption{vlines}}
%
% Warning: This stuff is experimental.
%
% \vspace{2ex}
% \noindent
% Vertical lines are interrupted, if space is inserted
% in \cs{noalign}, \cs{interrowspace}, \cs{addlinespace}
% (\xpackage{booktabs}),
% between double \cs{hline}s. This option tries to detect
% and add the vertical lines. The lines in a tabular
% with \xpackage{tabularht} support (environments of this package)
% are numbered from left to right. The gap that is
% controlled by \cs{interrowspace} or inbetween \cs{interrowstart}
% and \cs{interrowstop} is then filled with the detected
% vertical lines.
%
% If only a limited selection of the lines should be
% drawn, the commands know an optional argument with a
% list of line numbers, e.g.
% \begin{quote}
%\begin{verbatim}
%\begin{tabularht}{50mm}{|l|l|}
%  Hello & World\\
%  \interrowfill[1,3]
%  Foo & Bar
%\end{tabularht}
%\end{verbatim}
% \end{quote}
% There are three lines, but the middle line is not drawn
% in the gap between the first and second row.
% Zero can be used to suppress all lines:
% \begin{quote}
% |\interrowspace[0]{10mm}|
% \end{quote}
% The syntax of the commands with the optional argument
% with the line number list \meta{list}. \meta{list} is a comma
% separated list of numbers, \meta{height} means the
% height specification described above with the optional
% prefixes |to=| or |spread=|.
% \begin{quote}
%   \cs{interrowspace}\,|[|\meta{list}|]|\,|{|\meta{height}|}|\\
%   \cs{interrowfill}\,|[|\meta{list}|]|\\
%   \cs{interrowstart}\,|[|\meta{list}|]|\,\dots\,\cs{interrowstop}
% \end{quote}
%
% Option \xoption{vlines} is driver dependent and uses \eTeX\ features.
% \begin{description}
% \item[\xoption{pdftex}:] \pdfTeX\ in PDF mode.
%   Here the positions of the lines are written with the
%   help of the \cs{pdfsavepos} feature into the \xfile{.aux} file(s).
%   Therefore you need two LaTeX runs to get the lines.
% \item[\xoption{dvips}:]
%   Here, PostScript's currentpoint it used to get
%   the line positions. The lines are then drawn at the end
%   of the page. Thus one \LaTeX/dvips run is sufficient
%   for this option.
% \item[Other drivers:]\mbox{}
%   \begin{description}
%   \item[PostScript drivers:] probably possible, an end of page
%     hook would be nice.
%   \item[\VTeX:] with GeX (PostScript interpreter) probably possible.
%   \item[dvipdfm:] no idea. The big problem is, how to get the
%     current position?
%   \end{description}
% \end{description}
%
% \subsection{Limitations}
%
% \begin{itemize}
% \item
%   Vertical lines are interrupted by |\noalign{\vfill}|.
% \end{itemize}
%
% \subsection{Compatibility}
%
% \begin{itemize}
% \item
%   \xpackage{array}, \xpackage{delarray}, \xpackage{tabularx}
%   are supported.
% \item There can be problems with packages that
%   redefine \cs{@array} (or \cs{@@array}, \cs{@tabarray}) and
%   \cs{@arrayrule} (for option \xoption{vlines}).
% \item \xpackage{colortbl}: it should at least work, but there isn't
%   support for filling the gaps with color, neither
%   the rules nor the backgrounds.
% \end{itemize}
%
% \subsection{Examples}
%
% \subsubsection{Example 1}
%
%    \begin{macrocode}
%<*example1>
\documentclass{article}
\usepackage{tabularht}

\begin{document}
\fbox{%
  \begin{tabularht*}{1in}{4in}{@{}l@{\extracolsep{\fill}}r@{}}%
    upper left corner & upper right corner\\%
    \noalign{\vfill}%
    \multicolumn{2}{@{}c@{}}{bounding box}\\%
    \noalign{\vfill}%
    lower left corner & lower right corner\\%
  \end{tabularht*}%
}
\end{document}
%</example1>
%    \end{macrocode}
%
% \subsubsection{Example 2}
%
%    \begin{macrocode}
%<*example2>
\documentclass{article}
\usepackage{booktabs}
\usepackage[dvips,vlines]{tabularht}

\begin{document}

\begin{tabularht}{spread=0pt}{|l|l|}
  \hline
  First&Line\\%
  \hline
\interrowstart
  \addlinespace[10mm]%
\interrowstop
  \hline
  Second&Line\\%
\interrowstart
  \hline
  \hline
\interrowstop
  Third&Line\\%
  \hline
\interrowspace{10mm}
  \hline
  Fourth&Line\\%
  \hline
\end{tabularht}

\end{document}
%</example2>
%    \end{macrocode}
%
% \StopEventually{
% }
%
% \section{Implementation}
%
%    \begin{macrocode}
%<*package>
%    \end{macrocode}
%    Package identification.
%    \begin{macrocode}
\NeedsTeXFormat{LaTeX2e}
\ProvidesPackage{tabularht}%
  [2016/05/16 v2.6 Tabular with height specified (HO)]
%    \end{macrocode}
%
% \subsection{Environments}
%
%    \begin{macrocode}
\let\@toarrayheight\@empty
\let\tabH@array@init\@empty

\toks@={%
  \begingroup
    \long\def\x#1\vcenter\fi\fi\bgroup#2\@sharp#3#4\@nil{%
      \endgroup
      \gdef\@array[##1]##2{%
        \tabH@array@init
        #1%
        \vcenter\fi\fi
        \@toarrayheight
        \bgroup
        \let\@toarrayheight\@empty
        #2\@sharp###3#4%
      }%
    }%
  \expandafter\x\@array[#1]{#2}\@nil % hash-ok
}
\edef\tabH@patch@array{\the\toks@}
\def\tabH@patch@@array{%
  \ifx\@array\@@array
    \def\reserved@a{\let\@@array\@array}%
  \else
    \let\reserved@a\relax
  \fi
  \tabH@patch@array
  \reserved@a
}
\tabH@patch@@array

\@ifpackageloaded{array}{}{%
  \AtBeginDocument{%
    \@ifpackageloaded{array}{%
      \tabH@patch@@array
    }{}%
  }%
}

\def\tabH@setheight#1{%
  \tabH@@setheight#1==\@nil
}
\def\tabH@@setheight#1=#2=#3\@nil{%
  \ifx\\#2#3\\%
    \setlength{\dimen@}{#1}%
    \edef\@toarrayheight{to\the\dimen@}%
  \else
    \edef\tabH@temp{\zap@space#1 \@empty}%
    \ifx\tabH@temp\tabH@to
    \else
      \ifx\tabH@temp\tabH@spread
      \else
        \PackageError{tabularht}{%
          Unknown height specifier %
          `\expandafter\strip@prefix\meaning\tabH@temp'%
        }{%
          The height dimension for tabular height can be prefixed%
          \MessageBreak
          with `to=' or `spread=', default is `to='.%
        }%
        \let\tabH@temp\tabH@to
      \fi
    \fi
    \setlength{\dimen@}{#2}%
    \edef\@toarrayheight{\tabH@temp\the\dimen@}%
  \fi
}
\def\tabH@to{to}
\def\tabH@spread{spread}
%    \end{macrocode}
%    First argument is the height of the table,
%    then the orignal arguments for tabular follow.
%    \begin{macrocode}
\newenvironment{tabularht}[1]{%
  \tabH@setheight{#1}%
  \tabular
}{%
  \endtabular
}

\newenvironment{tabularht*}[1]{%
  \tabH@setheight{#1}%
  \@nameuse{tabular*}%
}{%
  \@nameuse{endtabular*}%
}

\newenvironment{tabularhtx}[1]{%
  \tabH@setheight{#1}%
  \tabularx
}{%
  \endtabularx
}

\newenvironment{arrayht}[1]{%
  \tabH@setheight{#1}%
  \array
}{%
  \endarray
}

\def\interrowspace{%
  \noalign\bgroup
    \tabH@interrowspace
}
\newcommand*{\tabH@interrowspace}[2][]{%
    \tabH@vspace{#1}{#2}%
  \egroup
}
\def\interrowfill{%
  \noalign\bgroup
    \tabH@interrowfill
}
\newcommand*{\tabH@interrowfill}[1][]{%
    \tabH@vspace{#1}{\fill}%
  \egroup
}
\def\tabH@vspace#1#2{%
  \tabH@vspace@start{#1}%
  \vspace{#2}%
  \tabH@vspace@stop
}
\let\tabH@vspace@start\@gobble
\let\tabH@vspace@stop\@empty

\newcommand*{\interrowstart}{%
  \noalign\bgroup
    \tabH@interrowstart
}
\newcommand*{\tabH@interrowstart}[1][]{%
    \tabH@vspace@start{#1}%
  \egroup
}
\newcommand*{\interrowstop}{%
  \noalign{\tabH@vspace@stop}%
}
%    \end{macrocode}
%
% \subsection{Options}
%
%    \begin{macrocode}
\providecommand*{\tabH@driver}{}

\DeclareOption{vlines}{%
  \let\tabH@temp\relax
}
\DeclareOption{pdftex}{}
\DeclareOption{dvips}{%
  \def\tabH@driver{dvips}%
}
\ProcessOptions*\relax

\ifx\tabH@temp\relax
\else
  \expandafter\endinput
\fi

\begingroup
  \@ifundefined{eTeXversion}{%
    \PackageError{tabularht}{%
      Option `vlines' requires eTeX%
    }{%
      Use of eTeX is recommended for LaTeX, see ltnews16.%
    }%
    \endgroup
    \endinput
  }{}%
\endgroup
%    \end{macrocode}
%
% \subsection{Option vlines, driver independent stuff}
%
%    \begin{macrocode}
\begingroup
  \let\@addtoreset\@gobbletwo
  \newcounter{tabH@unique}%
\endgroup
\let\tabH@currenttab\@empty

\def\tabH@array@init{%
  \ifx\@toarrayheight\@empty
    % ignore vertical lines of nested tabular environments
    \let\tabH@currenttab\@empty
  \else
    \stepcounter{tabH@unique}%
    \edef\tabH@currenttab{\the\c@tabH@unique}%
  \fi
}

\renewcommand*{\@arrayrule}{%
  \@addtopreamble{%
    \hskip -.5\arrayrulewidth
    \ifx\tabH@currenttab\@empty
    \else
      \tabH@vrule{\tabH@currenttab}%
    \fi
    \begingroup
      \expandafter\ifx\csname CT@arc@\endcsname\relax
      \else
        \expandafter\CT@arc@
      \fi
      \vline
    \endgroup
    \hskip -.5\arrayrulewidth
  }%
}
\let\tabH@arrayrule\@arrayrule
\AtBeginDocument{%
  \@ifpackageloaded{colortbl}{%
    \let\@arrayrule\tabH@arrayrule
  }{}%
}

\let\tabH@vrule\@gobble
%    \end{macrocode}
%
% \subsection{Driver pdftex}
%
%    \begin{macrocode}
\RequirePackage{ifpdf}
\ifpdf
  \begingroup
    \@ifundefined{pdfsavepos}{%
      \PackageError{tabularht}{%
        Your pdfTeX is too old%
      }{%
        \string\pdfsavepos\space is missing.%
      }%
      \endgroup
      \csname fi\endcsname
      \endinput
    }{}%

    \let\on@line\@empty
    \PackageInfo{tabularht}{%
      Using driver `pdftex' because of pdfTeX in PDF mode%
    }%
  \endgroup

  \protected\def\tabH@vrule#1{%
    \if@filesw
      \pdfsavepos
      \protected@write\@auxout{%
        \let\tabH@lastxpos\relax
      }{%
        \tabH@aux@vrule{#1}{\tabH@lastxpos}%
      }%
    \fi
  }%

  \def\tabH@lastxpos{\the\pdflastxpos}%
  \def\tabH@lastypos{\the\pdflastypos}%

  % The .aux file contains three commands:
  % \tabH@aux@vrule{tabular id}{x position}
  % \tabH@aux@vstart{tabular id}{row id}{x position}{y position}
  % \tabH@aux@vstop{y position}
  %
  \AtBeginDocument{%
    % The .aux files are read the first time before
    % \AtBeginDocument and later at \end{document}.
    % \tabH@aux@done is a marker to distinguish
    % between these two readings. Only in the first
    % case we need the \tabH@aux@... commands.
    \let\tabH@aux@done\@empty
    \if@filesw
      \immediate\write\@mainaux{%
        \@percentchar\@percentchar BeginProlog: tabularht%
      }%
      % items in the aux file are executed,
      % if tabularht is loaded
      % and during the aux file read at \begin{document} only
      \immediate\write\@mainaux{%
        \detokenize{%
          % the \tabH@aux@... commands are needed only if
          % tabularht is loaded with driver pdftex.
          \@ifundefined{tabH@aux@vrule}\@secondoftwo\@firstofone
          {%
            % disable commands except for the first .aux files reading
            \@ifundefined{tabH@aux@done}\@gobble\@firstofone
          }%
          {%
            \let\tabH@aux@vrule\@gobbletwo
            \let\tabH@aux@vstart\@gobblefour
            \let\tabH@aux@vstop\@gobble
          }%
        }%
      }%
      \immediate\write\@mainaux{%
        \@percentchar\@percentchar EndProlog: tabularht%
      }%
    \fi
  }%

  % the x positions of vrules are stored in
  % \tabH@<tabcount>list with distinct values
  \protected\def\tabH@aux@vrule#1#2{%
    \@ifundefined{tabH@#1list}{%
      \expandafter\xdef\csname tabH@#1list\endcsname{%
        \noexpand\do{#2}%
      }%
    }{%
      \begingroup
        \def\x{#2}%
        \let\y\@undefined
        \let\do\tabH@do@add
        \expandafter\xdef\csname tabH@#1list\endcsname{%
          \csname tabH@#1list\endcsname\@empty
          \ifx\y\@undefined
            \noexpand\do{\x}%
          \fi
        }%
      \endgroup
    }%
  }%
  \def\tabH@do@add#1{%
    \ifx\y\@undefined
      \ifnum#1<\x\space
      \else
        \expandafter\ifx\csname y\endcsname\relax\fi
        \ifnum#1>\x\space
          \noexpand\do{\x}%
        \fi
      \fi
    \fi
    \noexpand\do{#1}%
  }%

  \def\tabH@vspace@start#1{%
    \if@filesw
      \stepcounter{tabH@unique}%
      \edef\tabH@currentrow{\the\c@tabH@unique}%
      \pdfsavepos
      \protected@write\@auxout{%
        \let\tabH@lastxpos\relax
        \let\tabH@lastypos\relax
      }{%
        \tabH@aux@vstart{\tabH@currenttab}{\tabH@currentrow}%
                        {\tabH@lastxpos}{\tabH@lastypos}%
      }%
    \fi
    \begingroup
      \edef\a{tabH@\tabH@currenttab row\tabH@currentrow}%
      \expandafter\let\expandafter\x\csname\a x\endcsname
      \ifx\x\relax
      \else
        \expandafter\let\expandafter\y\csname\a y\endcsname
        \expandafter\let\expandafter\l
            \csname tabH@\tabH@currenttab list\endcsname
        \ifx\l\relax
        \else
          \def\f{#1}%
          \ifx\f\@empty
            \let\do\tabH@do@set
          \else
            \count@=\z@
            \let\do\tabH@do@filter
          \fi
          \setbox\z@=\hbox{\l}%
          \wd\z@=\z@
          \dp\z@=\z@
          \copy\z@
        \fi
      \fi
    \endgroup
  }%
  \def\tabH@vspace@stop{%
    \if@filesw
      \pdfsavepos
      \protected@write\@auxout{%
        \let\tabH@lastypos\relax
      }{%
        \tabH@aux@vstop{\tabH@lastypos}%
      }%
    \fi
  }%
  \def\tabH@do@set#1{%
    \hbox to \z@{%
      \hskip \dimexpr #1sp - \x sp\relax
      \vrule \@width\arrayrulewidth
             \@depth\dimexpr \y sp\relax
      \hss
    }%
  }%
  \def\tabH@do@filter{%
    \@tempswafalse
    \advance\count@\@ne
    \@for\e:=\f\do{%
      \ifnum\e=\count@
        \@tempswatrue
      \fi
    }%
    \if@tempswa
      \expandafter\tabH@do@set
    \else
      \expandafter\@gobble
    \fi
  }%

  \protected\def\tabH@aux@vstart#1#2#3#4{%
    \def\tabH@current@vstart{{#1}{#2}{#3}{#4}}%
  }%
  \protected\def\tabH@aux@vstop{%
    \expandafter\tabH@aux@v\tabH@current@vstart
  }%
  \def\tabH@aux@v#1#2#3#4#5{%
    \expandafter\gdef\csname tabH@#1row#2x\endcsname{#3}%
    \expandafter\xdef\csname tabH@#1row#2y\endcsname{%
      \the\numexpr #4 - #5\relax
    }%
  }%

  \csname fi\endcsname
  \endinput

\fi
%    \end{macrocode}
%
% \subsection{DVI drivers}
%
%    \begin{macrocode}
\ifx\tabH@driver\@empty
  \PackageError{tabularht}{%
    Missing DVI driver, option `vlines' disabled%
  }{%
    Supported DVI drivers: dvips.%
  }%
  \expandafter\endinput
\fi

\def\tabH@driver@dvips{%
  \def\tabH@literalps##1{\special{ps:SDict begin ##1 end}}%
  \def\tabH@headerps##1{\special{! ##1}}%
}

\@onelevel@sanitize\tabH@driver
\@ifundefined{tabH@driver@\tabH@driver}{%
  \PackageError{tabularht}{%
    Unsupported driver `\tabH@driver'%
  }{%
    Supported DVI drivers: dvips.%
  }%
  \endinput
}{}

\begingroup
  \let\on@line\@empty
  \PackageInfo{tabularht}{%
    Using driver `\tabH@driver'%
  }%
\endgroup
\csname tabH@driver@\tabH@driver\endcsname

\protected\def\tabH@vrule#1#2\vrule#3\arrayrulewidth{%
  #2% \fi or empty
  % hack to get rid of maxdrift rounding of dvips,
  % thus simulate a large motion
  \kern1in\relax
  \tabH@literalps{%
    #1 tabH.vrule %
    Resolution neg 0 translate%
  }%
  \vrule#3\arrayrulewidth
  \tabH@literalps{Resolution 0 translate}%
  \kern-1in\relax
}

\def\tabH@vspace@start#1{%
  \begingroup
    \let\y\@empty
    \@for\x:=#1\do{%
      \ifx\y\@empty
        \edef\y{\x}%
      \else
        \edef\y{\y\space\x}%
      \fi
    }%
    \tabH@literalps{\tabH@currenttab[\y]currentpoint exch pop}%
  \endgroup
}
\def\tabH@vspace@stop{%
  \tabH@literalps{%
    currentpoint exch pop %
    \number\dimexpr\arrayrulewidth\relax\space
    tabH.vspace%
  }%
}

\tabH@headerps{%
  userdict begin%
    /tabH.list 10 dict def%
    /tabH.job [] def %
  end%
  /tabH.vrule{%
    10 string cvs cvn dup tabH.list exch known{%
      tabH.list exch dup [ exch tabH.list exch get %
      currentpoint pop round exch true exch{%
        % tabH.list key [ ... x true i
        % tabH.list key [ ... false i
        exch{%
          % ... [ ... x i
          2 copy lt{false}{%
            2 copy eq{pop false}{exch true}ifelse%
          }ifelse%
        }{false}ifelse%
      }forall %
      pop%
      ]put%
    }{%
      tabH.list exch[currentpoint pop round]put%
    }ifelse%
  }bind def%
  % <tab num> <cols array> <ytop> <ybottom> <rulewidth[sp]>
  /tabH.vspace{%
    userdict begin %
      10 dict dup begin %
        exch 65536 div Resolution mul 72.27 div %
        % dvips uses a poor man's ceil function
        % see dopage.c before "drawrule": (int)(... + 0.9999999)
        0.9999999 add truncate%
        /rulewidth exch def %
        exch/ybottom exch def %
        exch/ytop exch def %
        exch/cols exch def %
        exch/tabkey exch 10 string cvs cvn def %
      end%
      /tabH.job exch[exch userdict/tabH.job get aload pop]def %
    end%
  }bind def %
  % Now we do the work at the end of the page.
  % Unhappily "eop-hook" cannot be used, because "eop"
  % executes "restore" before, so that all data are lost.
  TeXDict begin%
    /eop%
    [%
      {%
        tabH.job{%
          begin%
           /colarray %
             tabH.list tabkey known{tabH.list tabkey get}{[]}ifelse %
           def %
           cols length 0 eq not{%
             /colarray[%
               cols{1 sub %
                 dup 0 lt{pop}{%
                   dup colarray length ge{pop}{%
                     colarray exch get%
                   }ifelse%
                 }ifelse%
               }forall%
             ]def%
           }if %
           colarray{%
             % (rulewidth) == rulewidth == % debug
             Resolution sub %
             ytop rulewidth ytop ybottom sub v%
           }forall %
          end%
        }forall%
        % tabH.list{== ==}forall % debug
      }bind aload pop %
      TeXDict /eop get aload pop%
    ]cvx def %
  end%
}
%    \end{macrocode}
%
%    \begin{macrocode}
%</package>
%    \end{macrocode}
%
% \section{Installation}
%
% \subsection{Download}
%
% \paragraph{Package.} This package is available on
% CTAN\footnote{\CTANpkg{tabularht}}:
% \begin{description}
% \item[\CTAN{macros/latex/contrib/oberdiek/tabularht.dtx}] The source file.
% \item[\CTAN{macros/latex/contrib/oberdiek/tabularht.pdf}] Documentation.
% \end{description}
%
%
% \paragraph{Bundle.} All the packages of the bundle `oberdiek'
% are also available in a TDS compliant ZIP archive. There
% the packages are already unpacked and the documentation files
% are generated. The files and directories obey the TDS standard.
% \begin{description}
% \item[\CTANinstall{install/macros/latex/contrib/oberdiek.tds.zip}]
% \end{description}
% \emph{TDS} refers to the standard ``A Directory Structure
% for \TeX\ Files'' (\CTAN{tds/tds.pdf}). Directories
% with \xfile{texmf} in their name are usually organized this way.
%
% \subsection{Bundle installation}
%
% \paragraph{Unpacking.} Unpack the \xfile{oberdiek.tds.zip} in the
% TDS tree (also known as \xfile{texmf} tree) of your choice.
% Example (linux):
% \begin{quote}
%   |unzip oberdiek.tds.zip -d ~/texmf|
% \end{quote}
%
% \paragraph{Script installation.}
% Check the directory \xfile{TDS:scripts/oberdiek/} for
% scripts that need further installation steps.
%
% \subsection{Package installation}
%
% \paragraph{Unpacking.} The \xfile{.dtx} file is a self-extracting
% \docstrip\ archive. The files are extracted by running the
% \xfile{.dtx} through \plainTeX:
% \begin{quote}
%   \verb|tex tabularht.dtx|
% \end{quote}
%
% \paragraph{TDS.} Now the different files must be moved into
% the different directories in your installation TDS tree
% (also known as \xfile{texmf} tree):
% \begin{quote}
% \def\t{^^A
% \begin{tabular}{@{}>{\ttfamily}l@{ $\rightarrow$ }>{\ttfamily}l@{}}
%   tabularht.sty & tex/latex/oberdiek/tabularht.sty\\
%   tabularht.pdf & doc/latex/oberdiek/tabularht.pdf\\
%   tabularht-example1.tex & doc/latex/oberdiek/tabularht-example1.tex\\
%   tabularht-example2.tex & doc/latex/oberdiek/tabularht-example2.tex\\
%   tabularht.dtx & source/latex/oberdiek/tabularht.dtx\\
% \end{tabular}^^A
% }^^A
% \sbox0{\t}^^A
% \ifdim\wd0>\linewidth
%   \begingroup
%     \advance\linewidth by\leftmargin
%     \advance\linewidth by\rightmargin
%   \edef\x{\endgroup
%     \def\noexpand\lw{\the\linewidth}^^A
%   }\x
%   \def\lwbox{^^A
%     \leavevmode
%     \hbox to \linewidth{^^A
%       \kern-\leftmargin\relax
%       \hss
%       \usebox0
%       \hss
%       \kern-\rightmargin\relax
%     }^^A
%   }^^A
%   \ifdim\wd0>\lw
%     \sbox0{\small\t}^^A
%     \ifdim\wd0>\linewidth
%       \ifdim\wd0>\lw
%         \sbox0{\footnotesize\t}^^A
%         \ifdim\wd0>\linewidth
%           \ifdim\wd0>\lw
%             \sbox0{\scriptsize\t}^^A
%             \ifdim\wd0>\linewidth
%               \ifdim\wd0>\lw
%                 \sbox0{\tiny\t}^^A
%                 \ifdim\wd0>\linewidth
%                   \lwbox
%                 \else
%                   \usebox0
%                 \fi
%               \else
%                 \lwbox
%               \fi
%             \else
%               \usebox0
%             \fi
%           \else
%             \lwbox
%           \fi
%         \else
%           \usebox0
%         \fi
%       \else
%         \lwbox
%       \fi
%     \else
%       \usebox0
%     \fi
%   \else
%     \lwbox
%   \fi
% \else
%   \usebox0
% \fi
% \end{quote}
% If you have a \xfile{docstrip.cfg} that configures and enables \docstrip's
% TDS installing feature, then some files can already be in the right
% place, see the documentation of \docstrip.
%
% \subsection{Refresh file name databases}
%
% If your \TeX~distribution
% (\teTeX, \mikTeX, \dots) relies on file name databases, you must refresh
% these. For example, \teTeX\ users run \verb|texhash| or
% \verb|mktexlsr|.
%
% \subsection{Some details for the interested}
%
% \paragraph{Unpacking with \LaTeX.}
% The \xfile{.dtx} chooses its action depending on the format:
% \begin{description}
% \item[\plainTeX:] Run \docstrip\ and extract the files.
% \item[\LaTeX:] Generate the documentation.
% \end{description}
% If you insist on using \LaTeX\ for \docstrip\ (really,
% \docstrip\ does not need \LaTeX), then inform the autodetect routine
% about your intention:
% \begin{quote}
%   \verb|latex \let\install=y\input{tabularht.dtx}|
% \end{quote}
% Do not forget to quote the argument according to the demands
% of your shell.
%
% \paragraph{Generating the documentation.}
% You can use both the \xfile{.dtx} or the \xfile{.drv} to generate
% the documentation. The process can be configured by the
% configuration file \xfile{ltxdoc.cfg}. For instance, put this
% line into this file, if you want to have A4 as paper format:
% \begin{quote}
%   \verb|\PassOptionsToClass{a4paper}{article}|
% \end{quote}
% An example follows how to generate the
% documentation with pdf\LaTeX:
% \begin{quote}
%\begin{verbatim}
%pdflatex tabularht.dtx
%makeindex -s gind.ist tabularht.idx
%pdflatex tabularht.dtx
%makeindex -s gind.ist tabularht.idx
%pdflatex tabularht.dtx
%\end{verbatim}
% \end{quote}
%
% \begin{History}
%   \begin{Version}{2005/09/22 v1.0}
%   \item
%     First public version.
%   \end{Version}
%   \begin{Version}{2005/10/16 v2.0}
%   \item
%     Height specification allows
%     |to=...| or |spread=...|, default is |to=|.
%   \item
%     Option \xoption{vlines} added,
%     drivers \xoption{pdftex} and \xoption{dvips}.
%   \item
%     \cs{interrowspace}, \cs{interrowfil}, and
%     \cs{interrowstart}\dots\cs{interrowstop} added.
%   \end{Version}
%   \begin{Version}{2005/10/18 v2.1}
%   \item
%     Fix for package \xpackage{colortbl}, but the colors
%     of \xpackage{colortbl} remain unsupported.
%   \end{Version}
%   \begin{Version}{2006/02/20 v2.2}
%   \item
%     Code is not changed.
%   \item
%     DTX framework.
%   \end{Version}
%   \begin{Version}{2006/12/22 v2.3}
%   \item
%     Documentation fix.
%   \item
%     Fix in code of option vlines.
%   \end{Version}
%   \begin{Version}{2007/03/21 v2.4}
%   \item
%     Fix: Counter |tabh@unique| must not be changed by \cs{include}.
%   \end{Version}
%   \begin{Version}{2007/04/11 v2.5}
%   \item
%     Line ends sanitized.
%   \end{Version}
%   \begin{Version}{2016/05/16 v2.6}
%   \item
%     Documentation updates.
%   \end{Version}
% \end{History}
%
% \PrintIndex
%
% \Finale
\endinput
|
% \end{quote}
% Do not forget to quote the argument according to the demands
% of your shell.
%
% \paragraph{Generating the documentation.}
% You can use both the \xfile{.dtx} or the \xfile{.drv} to generate
% the documentation. The process can be configured by the
% configuration file \xfile{ltxdoc.cfg}. For instance, put this
% line into this file, if you want to have A4 as paper format:
% \begin{quote}
%   \verb|\PassOptionsToClass{a4paper}{article}|
% \end{quote}
% An example follows how to generate the
% documentation with pdf\LaTeX:
% \begin{quote}
%\begin{verbatim}
%pdflatex tabularht.dtx
%makeindex -s gind.ist tabularht.idx
%pdflatex tabularht.dtx
%makeindex -s gind.ist tabularht.idx
%pdflatex tabularht.dtx
%\end{verbatim}
% \end{quote}
%
% \begin{History}
%   \begin{Version}{2005/09/22 v1.0}
%   \item
%     First public version.
%   \end{Version}
%   \begin{Version}{2005/10/16 v2.0}
%   \item
%     Height specification allows
%     |to=...| or |spread=...|, default is |to=|.
%   \item
%     Option \xoption{vlines} added,
%     drivers \xoption{pdftex} and \xoption{dvips}.
%   \item
%     \cs{interrowspace}, \cs{interrowfil}, and
%     \cs{interrowstart}\dots\cs{interrowstop} added.
%   \end{Version}
%   \begin{Version}{2005/10/18 v2.1}
%   \item
%     Fix for package \xpackage{colortbl}, but the colors
%     of \xpackage{colortbl} remain unsupported.
%   \end{Version}
%   \begin{Version}{2006/02/20 v2.2}
%   \item
%     Code is not changed.
%   \item
%     DTX framework.
%   \end{Version}
%   \begin{Version}{2006/12/22 v2.3}
%   \item
%     Documentation fix.
%   \item
%     Fix in code of option vlines.
%   \end{Version}
%   \begin{Version}{2007/03/21 v2.4}
%   \item
%     Fix: Counter |tabh@unique| must not be changed by \cs{include}.
%   \end{Version}
%   \begin{Version}{2007/04/11 v2.5}
%   \item
%     Line ends sanitized.
%   \end{Version}
%   \begin{Version}{2016/05/16 v2.6}
%   \item
%     Documentation updates.
%   \end{Version}
% \end{History}
%
% \PrintIndex
%
% \Finale
\endinput
|
% \end{quote}
% Do not forget to quote the argument according to the demands
% of your shell.
%
% \paragraph{Generating the documentation.}
% You can use both the \xfile{.dtx} or the \xfile{.drv} to generate
% the documentation. The process can be configured by the
% configuration file \xfile{ltxdoc.cfg}. For instance, put this
% line into this file, if you want to have A4 as paper format:
% \begin{quote}
%   \verb|\PassOptionsToClass{a4paper}{article}|
% \end{quote}
% An example follows how to generate the
% documentation with pdf\LaTeX:
% \begin{quote}
%\begin{verbatim}
%pdflatex tabularht.dtx
%makeindex -s gind.ist tabularht.idx
%pdflatex tabularht.dtx
%makeindex -s gind.ist tabularht.idx
%pdflatex tabularht.dtx
%\end{verbatim}
% \end{quote}
%
% \begin{History}
%   \begin{Version}{2005/09/22 v1.0}
%   \item
%     First public version.
%   \end{Version}
%   \begin{Version}{2005/10/16 v2.0}
%   \item
%     Height specification allows
%     |to=...| or |spread=...|, default is |to=|.
%   \item
%     Option \xoption{vlines} added,
%     drivers \xoption{pdftex} and \xoption{dvips}.
%   \item
%     \cs{interrowspace}, \cs{interrowfil}, and
%     \cs{interrowstart}\dots\cs{interrowstop} added.
%   \end{Version}
%   \begin{Version}{2005/10/18 v2.1}
%   \item
%     Fix for package \xpackage{colortbl}, but the colors
%     of \xpackage{colortbl} remain unsupported.
%   \end{Version}
%   \begin{Version}{2006/02/20 v2.2}
%   \item
%     Code is not changed.
%   \item
%     DTX framework.
%   \end{Version}
%   \begin{Version}{2006/12/22 v2.3}
%   \item
%     Documentation fix.
%   \item
%     Fix in code of option vlines.
%   \end{Version}
%   \begin{Version}{2007/03/21 v2.4}
%   \item
%     Fix: Counter |tabh@unique| must not be changed by \cs{include}.
%   \end{Version}
%   \begin{Version}{2007/04/11 v2.5}
%   \item
%     Line ends sanitized.
%   \end{Version}
%   \begin{Version}{2016/05/16 v2.6}
%   \item
%     Documentation updates.
%   \end{Version}
% \end{History}
%
% \PrintIndex
%
% \Finale
\endinput
|
% \end{quote}
% Do not forget to quote the argument according to the demands
% of your shell.
%
% \paragraph{Generating the documentation.}
% You can use both the \xfile{.dtx} or the \xfile{.drv} to generate
% the documentation. The process can be configured by the
% configuration file \xfile{ltxdoc.cfg}. For instance, put this
% line into this file, if you want to have A4 as paper format:
% \begin{quote}
%   \verb|\PassOptionsToClass{a4paper}{article}|
% \end{quote}
% An example follows how to generate the
% documentation with pdf\LaTeX:
% \begin{quote}
%\begin{verbatim}
%pdflatex tabularht.dtx
%makeindex -s gind.ist tabularht.idx
%pdflatex tabularht.dtx
%makeindex -s gind.ist tabularht.idx
%pdflatex tabularht.dtx
%\end{verbatim}
% \end{quote}
%
% \begin{History}
%   \begin{Version}{2005/09/22 v1.0}
%   \item
%     First public version.
%   \end{Version}
%   \begin{Version}{2005/10/16 v2.0}
%   \item
%     Height specification allows
%     |to=...| or |spread=...|, default is |to=|.
%   \item
%     Option \xoption{vlines} added,
%     drivers \xoption{pdftex} and \xoption{dvips}.
%   \item
%     \cs{interrowspace}, \cs{interrowfil}, and
%     \cs{interrowstart}\dots\cs{interrowstop} added.
%   \end{Version}
%   \begin{Version}{2005/10/18 v2.1}
%   \item
%     Fix for package \xpackage{colortbl}, but the colors
%     of \xpackage{colortbl} remain unsupported.
%   \end{Version}
%   \begin{Version}{2006/02/20 v2.2}
%   \item
%     Code is not changed.
%   \item
%     DTX framework.
%   \end{Version}
%   \begin{Version}{2006/12/22 v2.3}
%   \item
%     Documentation fix.
%   \item
%     Fix in code of option vlines.
%   \end{Version}
%   \begin{Version}{2007/03/21 v2.4}
%   \item
%     Fix: Counter |tabh@unique| must not be changed by \cs{include}.
%   \end{Version}
%   \begin{Version}{2007/04/11 v2.5}
%   \item
%     Line ends sanitized.
%   \end{Version}
%   \begin{Version}{2016/05/16 v2.6}
%   \item
%     Documentation updates.
%   \end{Version}
% \end{History}
%
% \PrintIndex
%
% \Finale
\endinput
