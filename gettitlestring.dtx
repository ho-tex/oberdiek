% \iffalse meta-comment
%
% File: gettitlestring.dtx
% Version: 2016/05/16 v1.5
% Info: Cleanup title references
%
% Copyright (C) 2009, 2010 by
%    Heiko Oberdiek <heiko.oberdiek at googlemail.com>
%    2016
%    https://github.com/ho-tex/oberdiek/issues
%
% This work may be distributed and/or modified under the
% conditions of the LaTeX Project Public License, either
% version 1.3c of this license or (at your option) any later
% version. This version of this license is in
%    https://www.latex-project.org/lppl/lppl-1-3c.txt
% and the latest version of this license is in
%    https://www.latex-project.org/lppl.txt
% and version 1.3 or later is part of all distributions of
% LaTeX version 2005/12/01 or later.
%
% This work has the LPPL maintenance status "maintained".
%
% The Current Maintainers of this work are
% Heiko Oberdiek and the Oberdiek Package Support Group
% https://github.com/ho-tex/oberdiek/issues
%
% The Base Interpreter refers to any `TeX-Format',
% because some files are installed in TDS:tex/generic//.
%
% This work consists of the main source file gettitlestring.dtx
% and the derived files
%    gettitlestring.sty, gettitlestring.pdf, gettitlestring.ins,
%    gettitlestring.drv, gettitlestring-test1.tex,
%    gettitlestring-test2.tex.
%
% Distribution:
%    CTAN:macros/latex/contrib/oberdiek/gettitlestring.dtx
%    CTAN:macros/latex/contrib/oberdiek/gettitlestring.pdf
%
% Unpacking:
%    (a) If gettitlestring.ins is present:
%           tex gettitlestring.ins
%    (b) Without gettitlestring.ins:
%           tex gettitlestring.dtx
%    (c) If you insist on using LaTeX
%           latex \let\install=y% \iffalse meta-comment
%
% File: gettitlestring.dtx
% Version: 2016/05/16 v1.5
% Info: Cleanup title references
%
% Copyright (C) 2009, 2010 by
%    Heiko Oberdiek <heiko.oberdiek at googlemail.com>
%    2016
%    https://github.com/ho-tex/oberdiek/issues
%
% This work may be distributed and/or modified under the
% conditions of the LaTeX Project Public License, either
% version 1.3c of this license or (at your option) any later
% version. This version of this license is in
%    http://www.latex-project.org/lppl/lppl-1-3c.txt
% and the latest version of this license is in
%    http://www.latex-project.org/lppl.txt
% and version 1.3 or later is part of all distributions of
% LaTeX version 2005/12/01 or later.
%
% This work has the LPPL maintenance status "maintained".
%
% This Current Maintainer of this work is Heiko Oberdiek.
%
% The Base Interpreter refers to any `TeX-Format',
% because some files are installed in TDS:tex/generic//.
%
% This work consists of the main source file gettitlestring.dtx
% and the derived files
%    gettitlestring.sty, gettitlestring.pdf, gettitlestring.ins,
%    gettitlestring.drv, gettitlestring-test1.tex,
%    gettitlestring-test2.tex.
%
% Distribution:
%    CTAN:macros/latex/contrib/oberdiek/gettitlestring.dtx
%    CTAN:macros/latex/contrib/oberdiek/gettitlestring.pdf
%
% Unpacking:
%    (a) If gettitlestring.ins is present:
%           tex gettitlestring.ins
%    (b) Without gettitlestring.ins:
%           tex gettitlestring.dtx
%    (c) If you insist on using LaTeX
%           latex \let\install=y% \iffalse meta-comment
%
% File: gettitlestring.dtx
% Version: 2016/05/16 v1.5
% Info: Cleanup title references
%
% Copyright (C) 2009, 2010 by
%    Heiko Oberdiek <heiko.oberdiek at googlemail.com>
%    2016
%    https://github.com/ho-tex/oberdiek/issues
%
% This work may be distributed and/or modified under the
% conditions of the LaTeX Project Public License, either
% version 1.3c of this license or (at your option) any later
% version. This version of this license is in
%    http://www.latex-project.org/lppl/lppl-1-3c.txt
% and the latest version of this license is in
%    http://www.latex-project.org/lppl.txt
% and version 1.3 or later is part of all distributions of
% LaTeX version 2005/12/01 or later.
%
% This work has the LPPL maintenance status "maintained".
%
% This Current Maintainer of this work is Heiko Oberdiek.
%
% The Base Interpreter refers to any `TeX-Format',
% because some files are installed in TDS:tex/generic//.
%
% This work consists of the main source file gettitlestring.dtx
% and the derived files
%    gettitlestring.sty, gettitlestring.pdf, gettitlestring.ins,
%    gettitlestring.drv, gettitlestring-test1.tex,
%    gettitlestring-test2.tex.
%
% Distribution:
%    CTAN:macros/latex/contrib/oberdiek/gettitlestring.dtx
%    CTAN:macros/latex/contrib/oberdiek/gettitlestring.pdf
%
% Unpacking:
%    (a) If gettitlestring.ins is present:
%           tex gettitlestring.ins
%    (b) Without gettitlestring.ins:
%           tex gettitlestring.dtx
%    (c) If you insist on using LaTeX
%           latex \let\install=y% \iffalse meta-comment
%
% File: gettitlestring.dtx
% Version: 2016/05/16 v1.5
% Info: Cleanup title references
%
% Copyright (C) 2009, 2010 by
%    Heiko Oberdiek <heiko.oberdiek at googlemail.com>
%    2016
%    https://github.com/ho-tex/oberdiek/issues
%
% This work may be distributed and/or modified under the
% conditions of the LaTeX Project Public License, either
% version 1.3c of this license or (at your option) any later
% version. This version of this license is in
%    http://www.latex-project.org/lppl/lppl-1-3c.txt
% and the latest version of this license is in
%    http://www.latex-project.org/lppl.txt
% and version 1.3 or later is part of all distributions of
% LaTeX version 2005/12/01 or later.
%
% This work has the LPPL maintenance status "maintained".
%
% This Current Maintainer of this work is Heiko Oberdiek.
%
% The Base Interpreter refers to any `TeX-Format',
% because some files are installed in TDS:tex/generic//.
%
% This work consists of the main source file gettitlestring.dtx
% and the derived files
%    gettitlestring.sty, gettitlestring.pdf, gettitlestring.ins,
%    gettitlestring.drv, gettitlestring-test1.tex,
%    gettitlestring-test2.tex.
%
% Distribution:
%    CTAN:macros/latex/contrib/oberdiek/gettitlestring.dtx
%    CTAN:macros/latex/contrib/oberdiek/gettitlestring.pdf
%
% Unpacking:
%    (a) If gettitlestring.ins is present:
%           tex gettitlestring.ins
%    (b) Without gettitlestring.ins:
%           tex gettitlestring.dtx
%    (c) If you insist on using LaTeX
%           latex \let\install=y\input{gettitlestring.dtx}
%        (quote the arguments according to the demands of your shell)
%
% Documentation:
%    (a) If gettitlestring.drv is present:
%           latex gettitlestring.drv
%    (b) Without gettitlestring.drv:
%           latex gettitlestring.dtx; ...
%    The class ltxdoc loads the configuration file ltxdoc.cfg
%    if available. Here you can specify further options, e.g.
%    use A4 as paper format:
%       \PassOptionsToClass{a4paper}{article}
%
%    Programm calls to get the documentation (example):
%       pdflatex gettitlestring.dtx
%       makeindex -s gind.ist gettitlestring.idx
%       pdflatex gettitlestring.dtx
%       makeindex -s gind.ist gettitlestring.idx
%       pdflatex gettitlestring.dtx
%
% Installation:
%    TDS:tex/generic/oberdiek/gettitlestring.sty
%    TDS:doc/latex/oberdiek/gettitlestring.pdf
%    TDS:doc/latex/oberdiek/test/gettitlestring-test1.tex
%    TDS:doc/latex/oberdiek/test/gettitlestring-test2.tex
%    TDS:source/latex/oberdiek/gettitlestring.dtx
%
%<*ignore>
\begingroup
  \catcode123=1 %
  \catcode125=2 %
  \def\x{LaTeX2e}%
\expandafter\endgroup
\ifcase 0\ifx\install y1\fi\expandafter
         \ifx\csname processbatchFile\endcsname\relax\else1\fi
         \ifx\fmtname\x\else 1\fi\relax
\else\csname fi\endcsname
%</ignore>
%<*install>
\input docstrip.tex
\Msg{************************************************************************}
\Msg{* Installation}
\Msg{* Package: gettitlestring 2016/05/16 v1.5 Cleanup title references (HO)}
\Msg{************************************************************************}

\keepsilent
\askforoverwritefalse

\let\MetaPrefix\relax
\preamble

This is a generated file.

Project: gettitlestring
Version: 2016/05/16 v1.5

Copyright (C) 2009, 2010 by
   Heiko Oberdiek <heiko.oberdiek at googlemail.com>

This work may be distributed and/or modified under the
conditions of the LaTeX Project Public License, either
version 1.3c of this license or (at your option) any later
version. This version of this license is in
   http://www.latex-project.org/lppl/lppl-1-3c.txt
and the latest version of this license is in
   http://www.latex-project.org/lppl.txt
and version 1.3 or later is part of all distributions of
LaTeX version 2005/12/01 or later.

This work has the LPPL maintenance status "maintained".

This Current Maintainer of this work is Heiko Oberdiek.

The Base Interpreter refers to any `TeX-Format',
because some files are installed in TDS:tex/generic//.

This work consists of the main source file gettitlestring.dtx
and the derived files
   gettitlestring.sty, gettitlestring.pdf, gettitlestring.ins,
   gettitlestring.drv, gettitlestring-test1.tex,
   gettitlestring-test2.tex.

\endpreamble
\let\MetaPrefix\DoubleperCent

\generate{%
  \file{gettitlestring.ins}{\from{gettitlestring.dtx}{install}}%
  \file{gettitlestring.drv}{\from{gettitlestring.dtx}{driver}}%
  \usedir{tex/generic/oberdiek}%
  \file{gettitlestring.sty}{\from{gettitlestring.dtx}{package}}%
%  \usedir{doc/latex/oberdiek/test}%
%  \file{gettitlestring-test1.tex}{\from{gettitlestring.dtx}{test1}}%
%  \file{gettitlestring-test2.tex}{\from{gettitlestring.dtx}{test2}}%
  \nopreamble
  \nopostamble
%  \usedir{source/latex/oberdiek/catalogue}%
%  \file{gettitlestring.xml}{\from{gettitlestring.dtx}{catalogue}}%
}

\catcode32=13\relax% active space
\let =\space%
\Msg{************************************************************************}
\Msg{*}
\Msg{* To finish the installation you have to move the following}
\Msg{* file into a directory searched by TeX:}
\Msg{*}
\Msg{*     gettitlestring.sty}
\Msg{*}
\Msg{* To produce the documentation run the file `gettitlestring.drv'}
\Msg{* through LaTeX.}
\Msg{*}
\Msg{* Happy TeXing!}
\Msg{*}
\Msg{************************************************************************}

\endbatchfile
%</install>
%<*ignore>
\fi
%</ignore>
%<*driver>
\NeedsTeXFormat{LaTeX2e}
\ProvidesFile{gettitlestring.drv}%
  [2016/05/16 v1.5 Cleanup title references (HO)]%
\documentclass{ltxdoc}
\usepackage{holtxdoc}[2011/11/22]
\begin{document}
  \DocInput{gettitlestring.dtx}%
\end{document}
%</driver>
% \fi
%
%
% \CharacterTable
%  {Upper-case    \A\B\C\D\E\F\G\H\I\J\K\L\M\N\O\P\Q\R\S\T\U\V\W\X\Y\Z
%   Lower-case    \a\b\c\d\e\f\g\h\i\j\k\l\m\n\o\p\q\r\s\t\u\v\w\x\y\z
%   Digits        \0\1\2\3\4\5\6\7\8\9
%   Exclamation   \!     Double quote  \"     Hash (number) \#
%   Dollar        \$     Percent       \%     Ampersand     \&
%   Acute accent  \'     Left paren    \(     Right paren   \)
%   Asterisk      \*     Plus          \+     Comma         \,
%   Minus         \-     Point         \.     Solidus       \/
%   Colon         \:     Semicolon     \;     Less than     \<
%   Equals        \=     Greater than  \>     Question mark \?
%   Commercial at \@     Left bracket  \[     Backslash     \\
%   Right bracket \]     Circumflex    \^     Underscore    \_
%   Grave accent  \`     Left brace    \{     Vertical bar  \|
%   Right brace   \}     Tilde         \~}
%
% \GetFileInfo{gettitlestring.drv}
%
% \title{The \xpackage{gettitlestring} package}
% \date{2016/05/16 v1.5}
% \author{Heiko Oberdiek\thanks
% {Please report any issues at \url{https://github.com/ho-tex/oberdiek/issues}}\\
% \xemail{heiko.oberdiek at googlemail.com}}
%
% \maketitle
%
% \begin{abstract}
% The \LaTeX\ package addresses packages that are dealing with
% references to titles (\cs{section}, \cs{caption}, \dots).
% The package tries to remove \cs{label} and other
% commands from title strings.
% \end{abstract}
%
% \tableofcontents
%
% \section{Documentation}
%
% \subsection{Macros}
%
% \begin{declcs}{GetTitleStringSetup} \M{key value list}
% \end{declcs}
% The options are given as comma separated key value pairs.
% See section \ref{sec:options}.
%
% \begin{declcs}{GetTitleString} \M{text}\\
% \cs{GetTitleStringExpand} \M{text}\\
% \cs{GetTitleStringNonExpand} \M{text}
% \end{declcs}
% Macro \cs{GetTitleString} tries to remove unwanted stuff from \meta{text}
% the result is stored in Macro \cs{GetTitleStringResult}.
% Two methods are available:
% \begin{description}
% \item[\cs{GetTitleStringExpand}:]
% The \meta{text} is expanded in a context where the unwanted
% macros are redefined to remove themselves.
% This is the method used in packages \xpackage{titleref}~\cite{titleref},
% \xpackage{zref-titleref}~\cite{zref}
% or class \xclass{memoir}~\cite{memoir}.
% \cs{protect} is supported, but fragile material might break.
% \item[\cs{GetTitleStringNonExpand}:]
% The \meta{text} is not expanded. Thus the removal of unwanted
% material is more difficult. It is especially removed at the
% start of the \meta{text} and spaces are removed from the end.
% Currently only \cs{label} is removed in the whole string,
% if it is not hidden inside curly braces or part of macro
% definitions. Thus the removal of unwanted stuff might not be
% complete, but fragile material will not break.
% (But the result string can break at a later time, of course).
% \end{description}
% Option \xoption{expand} controls which method is used by
% macro \cs{GetTitleString}.
%
% \begin{declcs}{GetTitleStringDisableCommands} \M{code}
% \end{declcs}
% The \meta{code} is called right before the
% text is expanded in \cs{GetTitleStringExpand}.
% Additional definitions can be given for macros that
% should be removed.
% Keep in mind that expansion means that the definitions
% must work in expandable context. Macros like
% \cs{@ifstar} or \cs{@ifnextchar} or optional arguments
% will not work. The macro names in \meta{code} may contain
% the at sign |@|, it has catcode 11 (letter).
%
% \subsection{Options}\label{sec:options}
%
% \begin{description}
% \item[\xoption{expand}:] Boolean option, takes values |true| or |false|.
% No value means |true|. The option specifies the method to remove
% unwanted stuff from the title string, see below.
% \end{description}
% Options can be set at the following places:
% \begin{itemize}
% \item \cs{usepackage}
% \item Configuration file \xfile{gettitlestring.cfg}.
% \item \cs{GetTitleStringSetup}
% \end{itemize}
%
% \StopEventually{
% }
%
% \section{Implementation}
%    \begin{macrocode}
%<*package>
%    \end{macrocode}
%    Reload check, especially if the package is not used with \LaTeX.
%    \begin{macrocode}
\begingroup\catcode61\catcode48\catcode32=10\relax%
  \catcode13=5 % ^^M
  \endlinechar=13 %
  \catcode35=6 % #
  \catcode39=12 % '
  \catcode44=12 % ,
  \catcode45=12 % -
  \catcode46=12 % .
  \catcode58=12 % :
  \catcode64=11 % @
  \catcode123=1 % {
  \catcode125=2 % }
  \expandafter\let\expandafter\x\csname ver@gettitlestring.sty\endcsname
  \ifx\x\relax % plain-TeX, first loading
  \else
    \def\empty{}%
    \ifx\x\empty % LaTeX, first loading,
      % variable is initialized, but \ProvidesPackage not yet seen
    \else
      \expandafter\ifx\csname PackageInfo\endcsname\relax
        \def\x#1#2{%
          \immediate\write-1{Package #1 Info: #2.}%
        }%
      \else
        \def\x#1#2{\PackageInfo{#1}{#2, stopped}}%
      \fi
      \x{gettitlestring}{The package is already loaded}%
      \aftergroup\endinput
    \fi
  \fi
\endgroup%
%    \end{macrocode}
%    Package identification:
%    \begin{macrocode}
\begingroup\catcode61\catcode48\catcode32=10\relax%
  \catcode13=5 % ^^M
  \endlinechar=13 %
  \catcode35=6 % #
  \catcode39=12 % '
  \catcode40=12 % (
  \catcode41=12 % )
  \catcode44=12 % ,
  \catcode45=12 % -
  \catcode46=12 % .
  \catcode47=12 % /
  \catcode58=12 % :
  \catcode64=11 % @
  \catcode91=12 % [
  \catcode93=12 % ]
  \catcode123=1 % {
  \catcode125=2 % }
  \expandafter\ifx\csname ProvidesPackage\endcsname\relax
    \def\x#1#2#3[#4]{\endgroup
      \immediate\write-1{Package: #3 #4}%
      \xdef#1{#4}%
    }%
  \else
    \def\x#1#2[#3]{\endgroup
      #2[{#3}]%
      \ifx#1\@undefined
        \xdef#1{#3}%
      \fi
      \ifx#1\relax
        \xdef#1{#3}%
      \fi
    }%
  \fi
\expandafter\x\csname ver@gettitlestring.sty\endcsname
\ProvidesPackage{gettitlestring}%
  [2016/05/16 v1.5 Cleanup title references (HO)]%
%    \end{macrocode}
%
%    \begin{macrocode}
\begingroup\catcode61\catcode48\catcode32=10\relax%
  \catcode13=5 % ^^M
  \endlinechar=13 %
  \catcode123=1 % {
  \catcode125=2 % }
  \catcode64=11 % @
  \def\x{\endgroup
    \expandafter\edef\csname GTS@AtEnd\endcsname{%
      \endlinechar=\the\endlinechar\relax
      \catcode13=\the\catcode13\relax
      \catcode32=\the\catcode32\relax
      \catcode35=\the\catcode35\relax
      \catcode61=\the\catcode61\relax
      \catcode64=\the\catcode64\relax
      \catcode123=\the\catcode123\relax
      \catcode125=\the\catcode125\relax
    }%
  }%
\x\catcode61\catcode48\catcode32=10\relax%
\catcode13=5 % ^^M
\endlinechar=13 %
\catcode35=6 % #
\catcode64=11 % @
\catcode123=1 % {
\catcode125=2 % }
\def\TMP@EnsureCode#1#2{%
  \edef\GTS@AtEnd{%
    \GTS@AtEnd
    \catcode#1=\the\catcode#1\relax
  }%
  \catcode#1=#2\relax
}
\TMP@EnsureCode{42}{12}% *
\TMP@EnsureCode{44}{12}% ,
\TMP@EnsureCode{45}{12}% -
\TMP@EnsureCode{46}{12}% .
\TMP@EnsureCode{47}{12}% /
\TMP@EnsureCode{91}{12}% [
\TMP@EnsureCode{93}{12}% ]
\edef\GTS@AtEnd{\GTS@AtEnd\noexpand\endinput}
%    \end{macrocode}
%
% \subsection{Options}
%
%    \begin{macrocode}
\RequirePackage{kvoptions}[2009/07/17]
\SetupKeyvalOptions{%
  family=gettitlestring,%
  prefix=GTS@%
}
\newcommand*{\GetTitleStringSetup}{%
  \setkeys{gettitlestring}%
}
\DeclareBoolOption{expand}
\InputIfFileExists{gettitlestring.cfg}{}{}
\ProcessKeyvalOptions*\relax
%    \end{macrocode}
%
% \subsection{\cs{GetTitleString}}
%
%    \begin{macro}{\GetTitleString}
%    \begin{macrocode}
\newcommand*{\GetTitleString}{%
  \ifGTS@expand
    \expandafter\GetTitleStringExpand
  \else
    \expandafter\GetTitleStringNonExpand
  \fi
}
%    \end{macrocode}
%    \end{macro}
%    \begin{macro}{\GetTitleStringExpand}
%    \begin{macrocode}
\newcommand{\GetTitleStringExpand}[1]{%
  \def\GetTitleStringResult{#1}%
  \begingroup
    \GTS@DisablePredefinedCmds
    \GTS@DisableHook
    \edef\x{\endgroup
      \noexpand\def\noexpand\GetTitleStringResult{%
        \GetTitleStringResult
      }%
    }%
  \x
}
%    \end{macrocode}
%    \end{macro}
%    \begin{macro}{\GetTitleString}
%    \begin{macrocode}
\newcommand{\GetTitleStringNonExpand}[1]{%
  \def\GetTitleStringResult{#1}%
  \global\let\GTS@GlobalString\GetTitleStringResult
  \begingroup
    \GTS@RemoveLeft
    \GTS@RemoveRight
  \endgroup
  \let\GetTitleStringResult\GTS@GlobalString
}
%    \end{macrocode}
%    \end{macro}
%
% \subsubsection{Expand method}
%
%    \begin{macro}{\GTS@DisablePredefinedCmds}
%    \begin{macrocode}
\def\GTS@DisablePredefinedCmds{%
  \let\label\@gobble
  \let\zlabel\@gobble
  \let\zref@label\@gobble
  \let\zref@labelbylist\@gobbletwo
  \let\zref@labelbyprops\@gobbletwo
  \let\index\@gobble
  \let\glossary\@gobble
  \let\markboth\@gobbletwo
  \let\@mkboth\@gobbletwo
  \let\markright\@gobble
  \let\phantomsection\@empty
  \def\addcontentsline{\expandafter\@gobble\@gobbletwo}%
  \let\raggedright\@empty
  \let\raggedleft\@empty
  \let\centering\@empty
  \let\protect\@unexpandable@protect
  \let\enit@format\@empty % package enumitem
}
%    \end{macrocode}
%    \end{macro}
%
%    \begin{macro}{\GTS@DisableHook}
%    \begin{macrocode}
\providecommand*{\GTS@DisableHook}{}
%    \end{macrocode}
%    \end{macro}
%    \begin{macro}{\GetTitleStringDisableCommands}
%    \begin{macrocode}
\def\GetTitleStringDisableCommands{%
  \begingroup
    \makeatletter
    \GTS@DisableCommands
}
%    \end{macrocode}
%    \end{macro}
%    \begin{macro}{\GTS@DisableCommands}
%    \begin{macrocode}
\long\def\GTS@DisableCommands#1{%
    \toks0=\expandafter{\GTS@DisableHook}%
    \toks2={#1}%
    \xdef\GTS@GlobalString{\the\toks0 \the\toks2}%
  \endgroup
  \let\GTS@DisableHook\GTS@GlobalString
}
%    \end{macrocode}
%    \end{macro}
%
% \subsubsection{Non-expand method}
%
%    \begin{macrocode}
\def\GTS@RemoveLeft{%
  \toks@\expandafter\expandafter\expandafter{%
    \expandafter\GTS@Car\GTS@GlobalString{}{}{}{}\GTS@Nil
  }%
  \edef\GTS@Token{\the\toks@}%
  \GTS@PredefinedLeftCmds
  \expandafter\futurelet\expandafter\GTS@Token
  \expandafter\GTS@TestLeftSpace\GTS@GlobalString\GTS@Nil
  \GTS@End
}
\def\GTS@End{}
\long\def\GTS@TestLeft#1#2{%
  \def\GTS@temp{#1}%
  \ifx\GTS@temp\GTS@Token
    \toks@\expandafter\expandafter\expandafter{%
      \expandafter#2\GTS@GlobalString\GTS@Nil
    }%
    \expandafter\GTS@TestLeftEnd
  \fi
}
\long\def\GTS@TestLeftEnd#1\GTS@End{%
  \xdef\GTS@GlobalString{\the\toks@}%
  \GTS@RemoveLeft
}
\long\def\GTS@Car#1#2\GTS@Nil{#1}
\long\def\GTS@Cdr#1#2\GTS@Nil{#2}
\long\def\GTS@CdrTwo#1#2#3\GTS@Nil{#3}
\long\def\GTS@CdrThree#1#2#3#4\GTS@Nil{#4}
\long\def\GTS@CdrFour#1#2#3#4#5\GTS@Nil{#5}
\long\def\GTS@TestLeftSpace#1\GTS@Nil{%
  \ifx\GTS@Token\@sptoken
    \toks@\expandafter{%
      \romannumeral-0\GTS@GlobalString
    }%
    \expandafter\GTS@TestLeftEnd
  \fi
}
%    \end{macrocode}
%    \begin{macro}{\GTS@PredefinedLeftCmds}
%    \begin{macrocode}
\def\GTS@PredefinedLeftCmds{%
  \GTS@TestLeft\Hy@phantomsection\GTS@Cdr
  \GTS@TestLeft\Hy@SectionAnchor\GTS@Cdr
  \GTS@TestLeft\Hy@SectionAnchorHref\GTS@CdrTwo
  \GTS@TestLeft\label\GTS@CdrTwo
  \GTS@TestLeft\zlabel\GTS@CdrTwo
  \GTS@TestLeft\index\GTS@CdrTwo
  \GTS@TestLeft\glossary\GTS@CdrTwo
  \GTS@TestLeft\markboth\GTS@CdrThree
  \GTS@TestLeft\@mkboth\GTS@CdrThree
  \GTS@TestLeft\addcontentsline\GTS@CdrFour
  \GTS@TestLeft\enit@format\GTS@Cdr % package enumitem
}
%    \end{macrocode}
%    \end{macro}
%
%    \begin{macrocode}
\def\GTS@RemoveRight{%
  \toks@{}%
  \expandafter\GTS@TestRightLabel\GTS@GlobalString
      \label{}\GTS@Nil\@nil
  \GTS@RemoveRightSpace
}
\begingroup
  \def\GTS@temp#1{\endgroup
    \def\GTS@RemoveRightSpace{%
      \expandafter\GTS@TestRightSpace\GTS@GlobalString
          \GTS@Nil#1\GTS@Nil\@nil
    }%
  }%
\GTS@temp{ }
\def\GTS@TestRightSpace#1 \GTS@Nil#2\@nil{%
  \ifx\relax#2\relax
  \else
    \gdef\GTS@GlobalString{#1}%
    \expandafter\GTS@RemoveRightSpace
  \fi
}
\def\GTS@TestRightLabel#1\label#2#3\GTS@Nil#4\@nil{%
  \def\GTS@temp{#3}%
  \ifx\GTS@temp\@empty
    \expandafter\gdef\expandafter\GTS@GlobalString\expandafter{%
      \the\toks@
      #1%
    }%
    \expandafter\@gobble
  \else
    \expandafter\@firstofone
  \fi
  {%
    \toks@\expandafter{\the\toks@#1}%
    \GTS@TestRightLabel#3\GTS@Nil\@nil
  }%
}
%    \end{macrocode}
%
%    \begin{macrocode}
\GTS@AtEnd%
%</package>
%    \end{macrocode}
%
% \section{Test}
%
% \subsection{Catcode checks for loading}
%
%    \begin{macrocode}
%<*test1>
%    \end{macrocode}
%    \begin{macrocode}
\catcode`\{=1 %
\catcode`\}=2 %
\catcode`\#=6 %
\catcode`\@=11 %
\expandafter\ifx\csname count@\endcsname\relax
  \countdef\count@=255 %
\fi
\expandafter\ifx\csname @gobble\endcsname\relax
  \long\def\@gobble#1{}%
\fi
\expandafter\ifx\csname @firstofone\endcsname\relax
  \long\def\@firstofone#1{#1}%
\fi
\expandafter\ifx\csname loop\endcsname\relax
  \expandafter\@firstofone
\else
  \expandafter\@gobble
\fi
{%
  \def\loop#1\repeat{%
    \def\body{#1}%
    \iterate
  }%
  \def\iterate{%
    \body
      \let\next\iterate
    \else
      \let\next\relax
    \fi
    \next
  }%
  \let\repeat=\fi
}%
\def\RestoreCatcodes{}
\count@=0 %
\loop
  \edef\RestoreCatcodes{%
    \RestoreCatcodes
    \catcode\the\count@=\the\catcode\count@\relax
  }%
\ifnum\count@<255 %
  \advance\count@ 1 %
\repeat

\def\RangeCatcodeInvalid#1#2{%
  \count@=#1\relax
  \loop
    \catcode\count@=15 %
  \ifnum\count@<#2\relax
    \advance\count@ 1 %
  \repeat
}
\def\RangeCatcodeCheck#1#2#3{%
  \count@=#1\relax
  \loop
    \ifnum#3=\catcode\count@
    \else
      \errmessage{%
        Character \the\count@\space
        with wrong catcode \the\catcode\count@\space
        instead of \number#3%
      }%
    \fi
  \ifnum\count@<#2\relax
    \advance\count@ 1 %
  \repeat
}
\def\space{ }
\expandafter\ifx\csname LoadCommand\endcsname\relax
  \def\LoadCommand{\input gettitlestring.sty\relax}%
\fi
\def\Test{%
  \RangeCatcodeInvalid{0}{47}%
  \RangeCatcodeInvalid{58}{64}%
  \RangeCatcodeInvalid{91}{96}%
  \RangeCatcodeInvalid{123}{255}%
  \catcode`\@=12 %
  \catcode`\\=0 %
  \catcode`\%=14 %
  \LoadCommand
  \RangeCatcodeCheck{0}{36}{15}%
  \RangeCatcodeCheck{37}{37}{14}%
  \RangeCatcodeCheck{38}{47}{15}%
  \RangeCatcodeCheck{48}{57}{12}%
  \RangeCatcodeCheck{58}{63}{15}%
  \RangeCatcodeCheck{64}{64}{12}%
  \RangeCatcodeCheck{65}{90}{11}%
  \RangeCatcodeCheck{91}{91}{15}%
  \RangeCatcodeCheck{92}{92}{0}%
  \RangeCatcodeCheck{93}{96}{15}%
  \RangeCatcodeCheck{97}{122}{11}%
  \RangeCatcodeCheck{123}{255}{15}%
  \RestoreCatcodes
}
\Test
\csname @@end\endcsname
\end
%    \end{macrocode}
%    \begin{macrocode}
%</test1>
%    \end{macrocode}
%
% \subsection{Test of non-expand method}
%
%    \begin{macrocode}
%<*test2>
\NeedsTeXFormat{LaTeX2e}
\documentclass{minimal}
\usepackage{gettitlestring}[2016/05/16]
\usepackage{qstest}
\IncludeTests{*}
\LogTests{log}{*}{*}
\begin{document}
\begin{qstest}{non-expand}{non-expand}
  \def\test#1#2{%
    \sbox0{%
      \GetTitleString{#1}%
      \Expect{#2}*{\GetTitleStringResult}%
    }%
    \Expect{0.0pt}*{\the\wd0}%
  }%
  \test{}{}%
  \test{ }{}%
  \test{ x }{x}%
  \test{ x y }{x y}%
  \test{ \relax}{\relax}%
  \test{\label{f}a}{a}%
  \test{ \label{f}a}{a}%
  \test{\label{f} a}{a}%
  \test{ \label{f} a}{a}%
  \test{a\label{f}}{a}%
  \test{a\label{f} }{a}%
  \test{a \label{f}}{a}%
  \test{a \label{f} }{a}%
  \test{a\label{f}b\label{g}}{ab}%
  \test{a \label{f}b \label{g} }{a b}%
  \test{a\label{f} b \label{g} }{a b}%
\end{qstest}
\end{document}
%</test2>
%    \end{macrocode}
%
% \section{Installation}
%
% \subsection{Download}
%
% \paragraph{Package.} This package is available on
% CTAN\footnote{\CTANpkg{gettitlestring}}:
% \begin{description}
% \item[\CTAN{macros/latex/contrib/oberdiek/gettitlestring.dtx}] The source file.
% \item[\CTAN{macros/latex/contrib/oberdiek/gettitlestring.pdf}] Documentation.
% \end{description}
%
%
% \paragraph{Bundle.} All the packages of the bundle `oberdiek'
% are also available in a TDS compliant ZIP archive. There
% the packages are already unpacked and the documentation files
% are generated. The files and directories obey the TDS standard.
% \begin{description}
% \item[\CTANinstall{install/macros/latex/contrib/oberdiek.tds.zip}]
% \end{description}
% \emph{TDS} refers to the standard ``A Directory Structure
% for \TeX\ Files'' (\CTAN{tds/tds.pdf}). Directories
% with \xfile{texmf} in their name are usually organized this way.
%
% \subsection{Bundle installation}
%
% \paragraph{Unpacking.} Unpack the \xfile{oberdiek.tds.zip} in the
% TDS tree (also known as \xfile{texmf} tree) of your choice.
% Example (linux):
% \begin{quote}
%   |unzip oberdiek.tds.zip -d ~/texmf|
% \end{quote}
%
% \paragraph{Script installation.}
% Check the directory \xfile{TDS:scripts/oberdiek/} for
% scripts that need further installation steps.
% Package \xpackage{attachfile2} comes with the Perl script
% \xfile{pdfatfi.pl} that should be installed in such a way
% that it can be called as \texttt{pdfatfi}.
% Example (linux):
% \begin{quote}
%   |chmod +x scripts/oberdiek/pdfatfi.pl|\\
%   |cp scripts/oberdiek/pdfatfi.pl /usr/local/bin/|
% \end{quote}
%
% \subsection{Package installation}
%
% \paragraph{Unpacking.} The \xfile{.dtx} file is a self-extracting
% \docstrip\ archive. The files are extracted by running the
% \xfile{.dtx} through \plainTeX:
% \begin{quote}
%   \verb|tex gettitlestring.dtx|
% \end{quote}
%
% \paragraph{TDS.} Now the different files must be moved into
% the different directories in your installation TDS tree
% (also known as \xfile{texmf} tree):
% \begin{quote}
% \def\t{^^A
% \begin{tabular}{@{}>{\ttfamily}l@{ $\rightarrow$ }>{\ttfamily}l@{}}
%   gettitlestring.sty & tex/generic/oberdiek/gettitlestring.sty\\
%   gettitlestring.pdf & doc/latex/oberdiek/gettitlestring.pdf\\
%   test/gettitlestring-test1.tex & doc/latex/oberdiek/test/gettitlestring-test1.tex\\
%   test/gettitlestring-test2.tex & doc/latex/oberdiek/test/gettitlestring-test2.tex\\
%   gettitlestring.dtx & source/latex/oberdiek/gettitlestring.dtx\\
% \end{tabular}^^A
% }^^A
% \sbox0{\t}^^A
% \ifdim\wd0>\linewidth
%   \begingroup
%     \advance\linewidth by\leftmargin
%     \advance\linewidth by\rightmargin
%   \edef\x{\endgroup
%     \def\noexpand\lw{\the\linewidth}^^A
%   }\x
%   \def\lwbox{^^A
%     \leavevmode
%     \hbox to \linewidth{^^A
%       \kern-\leftmargin\relax
%       \hss
%       \usebox0
%       \hss
%       \kern-\rightmargin\relax
%     }^^A
%   }^^A
%   \ifdim\wd0>\lw
%     \sbox0{\small\t}^^A
%     \ifdim\wd0>\linewidth
%       \ifdim\wd0>\lw
%         \sbox0{\footnotesize\t}^^A
%         \ifdim\wd0>\linewidth
%           \ifdim\wd0>\lw
%             \sbox0{\scriptsize\t}^^A
%             \ifdim\wd0>\linewidth
%               \ifdim\wd0>\lw
%                 \sbox0{\tiny\t}^^A
%                 \ifdim\wd0>\linewidth
%                   \lwbox
%                 \else
%                   \usebox0
%                 \fi
%               \else
%                 \lwbox
%               \fi
%             \else
%               \usebox0
%             \fi
%           \else
%             \lwbox
%           \fi
%         \else
%           \usebox0
%         \fi
%       \else
%         \lwbox
%       \fi
%     \else
%       \usebox0
%     \fi
%   \else
%     \lwbox
%   \fi
% \else
%   \usebox0
% \fi
% \end{quote}
% If you have a \xfile{docstrip.cfg} that configures and enables \docstrip's
% TDS installing feature, then some files can already be in the right
% place, see the documentation of \docstrip.
%
% \subsection{Refresh file name databases}
%
% If your \TeX~distribution
% (\teTeX, \mikTeX, \dots) relies on file name databases, you must refresh
% these. For example, \teTeX\ users run \verb|texhash| or
% \verb|mktexlsr|.
%
% \subsection{Some details for the interested}
%
% \paragraph{Attached source.}
%
% The PDF documentation on CTAN also includes the
% \xfile{.dtx} source file. It can be extracted by
% AcrobatReader 6 or higher. Another option is \textsf{pdftk},
% e.g. unpack the file into the current directory:
% \begin{quote}
%   \verb|pdftk gettitlestring.pdf unpack_files output .|
% \end{quote}
%
% \paragraph{Unpacking with \LaTeX.}
% The \xfile{.dtx} chooses its action depending on the format:
% \begin{description}
% \item[\plainTeX:] Run \docstrip\ and extract the files.
% \item[\LaTeX:] Generate the documentation.
% \end{description}
% If you insist on using \LaTeX\ for \docstrip\ (really,
% \docstrip\ does not need \LaTeX), then inform the autodetect routine
% about your intention:
% \begin{quote}
%   \verb|latex \let\install=y\input{gettitlestring.dtx}|
% \end{quote}
% Do not forget to quote the argument according to the demands
% of your shell.
%
% \paragraph{Generating the documentation.}
% You can use both the \xfile{.dtx} or the \xfile{.drv} to generate
% the documentation. The process can be configured by the
% configuration file \xfile{ltxdoc.cfg}. For instance, put this
% line into this file, if you want to have A4 as paper format:
% \begin{quote}
%   \verb|\PassOptionsToClass{a4paper}{article}|
% \end{quote}
% An example follows how to generate the
% documentation with pdf\LaTeX:
% \begin{quote}
%\begin{verbatim}
%pdflatex gettitlestring.dtx
%makeindex -s gind.ist gettitlestring.idx
%pdflatex gettitlestring.dtx
%makeindex -s gind.ist gettitlestring.idx
%pdflatex gettitlestring.dtx
%\end{verbatim}
% \end{quote}
%
% \begin{thebibliography}{9}
%
% \bibitem{memoir}
% Peter Wilson, Lars Madsen:
% \textit{The Memoir Class};
% 2009/11/17 v1.61803398c;
% \CTANpkg{memoir}
%
% \bibitem{titleref}
% Donald Arsenau:
% \textit{Titleref.sty};
% 2001/04/05 ver 3.1;
% \CTAN{macros/latex/contrib/misc/titleref.sty}
%
% \bibitem{zref}
% Heiko Oberdiek:
% \textit{The \xpackage{zref} package};
% 2009/12/08 v2.7;
% \CTAN{macros/latex/contrib/oberdiek/zref.pdf}
%
% \end{thebibliography}
%
% \begin{History}
%   \begin{Version}{2009/12/08 v1.0}
%   \item
%     The first version.
%   \end{Version}
%   \begin{Version}{2009/12/12 v1.1}
%   \item
%     Short info shortened.
%   \end{Version}
%   \begin{Version}{2009/12/13 v1.2}
%   \item
%     Forgotten third argument for \cs{InputIfFileExists} added.
%   \end{Version}
%   \begin{Version}{2009/12/18 v1.3}
%   \item
%     \cs{Hy@SectionAnchorHref} added for filtering
%     (hyperref 2009/12/18 v6.79w).
%   \end{Version}
%   \begin{Version}{2010/12/03 v1.4}
%   \item
%     Support of package \xpackage{enumitem}: removing
%     \cs{enit@format} from title string (problem report by GL).
%   \end{Version}
%   \begin{Version}{2016/05/16 v1.5}
%   \item
%     Documentation updates.
%   \end{Version}
% \end{History}
%
% \PrintIndex
%
% \Finale
\endinput

%        (quote the arguments according to the demands of your shell)
%
% Documentation:
%    (a) If gettitlestring.drv is present:
%           latex gettitlestring.drv
%    (b) Without gettitlestring.drv:
%           latex gettitlestring.dtx; ...
%    The class ltxdoc loads the configuration file ltxdoc.cfg
%    if available. Here you can specify further options, e.g.
%    use A4 as paper format:
%       \PassOptionsToClass{a4paper}{article}
%
%    Programm calls to get the documentation (example):
%       pdflatex gettitlestring.dtx
%       makeindex -s gind.ist gettitlestring.idx
%       pdflatex gettitlestring.dtx
%       makeindex -s gind.ist gettitlestring.idx
%       pdflatex gettitlestring.dtx
%
% Installation:
%    TDS:tex/generic/oberdiek/gettitlestring.sty
%    TDS:doc/latex/oberdiek/gettitlestring.pdf
%    TDS:doc/latex/oberdiek/test/gettitlestring-test1.tex
%    TDS:doc/latex/oberdiek/test/gettitlestring-test2.tex
%    TDS:source/latex/oberdiek/gettitlestring.dtx
%
%<*ignore>
\begingroup
  \catcode123=1 %
  \catcode125=2 %
  \def\x{LaTeX2e}%
\expandafter\endgroup
\ifcase 0\ifx\install y1\fi\expandafter
         \ifx\csname processbatchFile\endcsname\relax\else1\fi
         \ifx\fmtname\x\else 1\fi\relax
\else\csname fi\endcsname
%</ignore>
%<*install>
\input docstrip.tex
\Msg{************************************************************************}
\Msg{* Installation}
\Msg{* Package: gettitlestring 2016/05/16 v1.5 Cleanup title references (HO)}
\Msg{************************************************************************}

\keepsilent
\askforoverwritefalse

\let\MetaPrefix\relax
\preamble

This is a generated file.

Project: gettitlestring
Version: 2016/05/16 v1.5

Copyright (C) 2009, 2010 by
   Heiko Oberdiek <heiko.oberdiek at googlemail.com>

This work may be distributed and/or modified under the
conditions of the LaTeX Project Public License, either
version 1.3c of this license or (at your option) any later
version. This version of this license is in
   http://www.latex-project.org/lppl/lppl-1-3c.txt
and the latest version of this license is in
   http://www.latex-project.org/lppl.txt
and version 1.3 or later is part of all distributions of
LaTeX version 2005/12/01 or later.

This work has the LPPL maintenance status "maintained".

This Current Maintainer of this work is Heiko Oberdiek.

The Base Interpreter refers to any `TeX-Format',
because some files are installed in TDS:tex/generic//.

This work consists of the main source file gettitlestring.dtx
and the derived files
   gettitlestring.sty, gettitlestring.pdf, gettitlestring.ins,
   gettitlestring.drv, gettitlestring-test1.tex,
   gettitlestring-test2.tex.

\endpreamble
\let\MetaPrefix\DoubleperCent

\generate{%
  \file{gettitlestring.ins}{\from{gettitlestring.dtx}{install}}%
  \file{gettitlestring.drv}{\from{gettitlestring.dtx}{driver}}%
  \usedir{tex/generic/oberdiek}%
  \file{gettitlestring.sty}{\from{gettitlestring.dtx}{package}}%
%  \usedir{doc/latex/oberdiek/test}%
%  \file{gettitlestring-test1.tex}{\from{gettitlestring.dtx}{test1}}%
%  \file{gettitlestring-test2.tex}{\from{gettitlestring.dtx}{test2}}%
  \nopreamble
  \nopostamble
%  \usedir{source/latex/oberdiek/catalogue}%
%  \file{gettitlestring.xml}{\from{gettitlestring.dtx}{catalogue}}%
}

\catcode32=13\relax% active space
\let =\space%
\Msg{************************************************************************}
\Msg{*}
\Msg{* To finish the installation you have to move the following}
\Msg{* file into a directory searched by TeX:}
\Msg{*}
\Msg{*     gettitlestring.sty}
\Msg{*}
\Msg{* To produce the documentation run the file `gettitlestring.drv'}
\Msg{* through LaTeX.}
\Msg{*}
\Msg{* Happy TeXing!}
\Msg{*}
\Msg{************************************************************************}

\endbatchfile
%</install>
%<*ignore>
\fi
%</ignore>
%<*driver>
\NeedsTeXFormat{LaTeX2e}
\ProvidesFile{gettitlestring.drv}%
  [2016/05/16 v1.5 Cleanup title references (HO)]%
\documentclass{ltxdoc}
\usepackage{holtxdoc}[2011/11/22]
\begin{document}
  \DocInput{gettitlestring.dtx}%
\end{document}
%</driver>
% \fi
%
%
% \CharacterTable
%  {Upper-case    \A\B\C\D\E\F\G\H\I\J\K\L\M\N\O\P\Q\R\S\T\U\V\W\X\Y\Z
%   Lower-case    \a\b\c\d\e\f\g\h\i\j\k\l\m\n\o\p\q\r\s\t\u\v\w\x\y\z
%   Digits        \0\1\2\3\4\5\6\7\8\9
%   Exclamation   \!     Double quote  \"     Hash (number) \#
%   Dollar        \$     Percent       \%     Ampersand     \&
%   Acute accent  \'     Left paren    \(     Right paren   \)
%   Asterisk      \*     Plus          \+     Comma         \,
%   Minus         \-     Point         \.     Solidus       \/
%   Colon         \:     Semicolon     \;     Less than     \<
%   Equals        \=     Greater than  \>     Question mark \?
%   Commercial at \@     Left bracket  \[     Backslash     \\
%   Right bracket \]     Circumflex    \^     Underscore    \_
%   Grave accent  \`     Left brace    \{     Vertical bar  \|
%   Right brace   \}     Tilde         \~}
%
% \GetFileInfo{gettitlestring.drv}
%
% \title{The \xpackage{gettitlestring} package}
% \date{2016/05/16 v1.5}
% \author{Heiko Oberdiek\thanks
% {Please report any issues at \url{https://github.com/ho-tex/oberdiek/issues}}\\
% \xemail{heiko.oberdiek at googlemail.com}}
%
% \maketitle
%
% \begin{abstract}
% The \LaTeX\ package addresses packages that are dealing with
% references to titles (\cs{section}, \cs{caption}, \dots).
% The package tries to remove \cs{label} and other
% commands from title strings.
% \end{abstract}
%
% \tableofcontents
%
% \section{Documentation}
%
% \subsection{Macros}
%
% \begin{declcs}{GetTitleStringSetup} \M{key value list}
% \end{declcs}
% The options are given as comma separated key value pairs.
% See section \ref{sec:options}.
%
% \begin{declcs}{GetTitleString} \M{text}\\
% \cs{GetTitleStringExpand} \M{text}\\
% \cs{GetTitleStringNonExpand} \M{text}
% \end{declcs}
% Macro \cs{GetTitleString} tries to remove unwanted stuff from \meta{text}
% the result is stored in Macro \cs{GetTitleStringResult}.
% Two methods are available:
% \begin{description}
% \item[\cs{GetTitleStringExpand}:]
% The \meta{text} is expanded in a context where the unwanted
% macros are redefined to remove themselves.
% This is the method used in packages \xpackage{titleref}~\cite{titleref},
% \xpackage{zref-titleref}~\cite{zref}
% or class \xclass{memoir}~\cite{memoir}.
% \cs{protect} is supported, but fragile material might break.
% \item[\cs{GetTitleStringNonExpand}:]
% The \meta{text} is not expanded. Thus the removal of unwanted
% material is more difficult. It is especially removed at the
% start of the \meta{text} and spaces are removed from the end.
% Currently only \cs{label} is removed in the whole string,
% if it is not hidden inside curly braces or part of macro
% definitions. Thus the removal of unwanted stuff might not be
% complete, but fragile material will not break.
% (But the result string can break at a later time, of course).
% \end{description}
% Option \xoption{expand} controls which method is used by
% macro \cs{GetTitleString}.
%
% \begin{declcs}{GetTitleStringDisableCommands} \M{code}
% \end{declcs}
% The \meta{code} is called right before the
% text is expanded in \cs{GetTitleStringExpand}.
% Additional definitions can be given for macros that
% should be removed.
% Keep in mind that expansion means that the definitions
% must work in expandable context. Macros like
% \cs{@ifstar} or \cs{@ifnextchar} or optional arguments
% will not work. The macro names in \meta{code} may contain
% the at sign |@|, it has catcode 11 (letter).
%
% \subsection{Options}\label{sec:options}
%
% \begin{description}
% \item[\xoption{expand}:] Boolean option, takes values |true| or |false|.
% No value means |true|. The option specifies the method to remove
% unwanted stuff from the title string, see below.
% \end{description}
% Options can be set at the following places:
% \begin{itemize}
% \item \cs{usepackage}
% \item Configuration file \xfile{gettitlestring.cfg}.
% \item \cs{GetTitleStringSetup}
% \end{itemize}
%
% \StopEventually{
% }
%
% \section{Implementation}
%    \begin{macrocode}
%<*package>
%    \end{macrocode}
%    Reload check, especially if the package is not used with \LaTeX.
%    \begin{macrocode}
\begingroup\catcode61\catcode48\catcode32=10\relax%
  \catcode13=5 % ^^M
  \endlinechar=13 %
  \catcode35=6 % #
  \catcode39=12 % '
  \catcode44=12 % ,
  \catcode45=12 % -
  \catcode46=12 % .
  \catcode58=12 % :
  \catcode64=11 % @
  \catcode123=1 % {
  \catcode125=2 % }
  \expandafter\let\expandafter\x\csname ver@gettitlestring.sty\endcsname
  \ifx\x\relax % plain-TeX, first loading
  \else
    \def\empty{}%
    \ifx\x\empty % LaTeX, first loading,
      % variable is initialized, but \ProvidesPackage not yet seen
    \else
      \expandafter\ifx\csname PackageInfo\endcsname\relax
        \def\x#1#2{%
          \immediate\write-1{Package #1 Info: #2.}%
        }%
      \else
        \def\x#1#2{\PackageInfo{#1}{#2, stopped}}%
      \fi
      \x{gettitlestring}{The package is already loaded}%
      \aftergroup\endinput
    \fi
  \fi
\endgroup%
%    \end{macrocode}
%    Package identification:
%    \begin{macrocode}
\begingroup\catcode61\catcode48\catcode32=10\relax%
  \catcode13=5 % ^^M
  \endlinechar=13 %
  \catcode35=6 % #
  \catcode39=12 % '
  \catcode40=12 % (
  \catcode41=12 % )
  \catcode44=12 % ,
  \catcode45=12 % -
  \catcode46=12 % .
  \catcode47=12 % /
  \catcode58=12 % :
  \catcode64=11 % @
  \catcode91=12 % [
  \catcode93=12 % ]
  \catcode123=1 % {
  \catcode125=2 % }
  \expandafter\ifx\csname ProvidesPackage\endcsname\relax
    \def\x#1#2#3[#4]{\endgroup
      \immediate\write-1{Package: #3 #4}%
      \xdef#1{#4}%
    }%
  \else
    \def\x#1#2[#3]{\endgroup
      #2[{#3}]%
      \ifx#1\@undefined
        \xdef#1{#3}%
      \fi
      \ifx#1\relax
        \xdef#1{#3}%
      \fi
    }%
  \fi
\expandafter\x\csname ver@gettitlestring.sty\endcsname
\ProvidesPackage{gettitlestring}%
  [2016/05/16 v1.5 Cleanup title references (HO)]%
%    \end{macrocode}
%
%    \begin{macrocode}
\begingroup\catcode61\catcode48\catcode32=10\relax%
  \catcode13=5 % ^^M
  \endlinechar=13 %
  \catcode123=1 % {
  \catcode125=2 % }
  \catcode64=11 % @
  \def\x{\endgroup
    \expandafter\edef\csname GTS@AtEnd\endcsname{%
      \endlinechar=\the\endlinechar\relax
      \catcode13=\the\catcode13\relax
      \catcode32=\the\catcode32\relax
      \catcode35=\the\catcode35\relax
      \catcode61=\the\catcode61\relax
      \catcode64=\the\catcode64\relax
      \catcode123=\the\catcode123\relax
      \catcode125=\the\catcode125\relax
    }%
  }%
\x\catcode61\catcode48\catcode32=10\relax%
\catcode13=5 % ^^M
\endlinechar=13 %
\catcode35=6 % #
\catcode64=11 % @
\catcode123=1 % {
\catcode125=2 % }
\def\TMP@EnsureCode#1#2{%
  \edef\GTS@AtEnd{%
    \GTS@AtEnd
    \catcode#1=\the\catcode#1\relax
  }%
  \catcode#1=#2\relax
}
\TMP@EnsureCode{42}{12}% *
\TMP@EnsureCode{44}{12}% ,
\TMP@EnsureCode{45}{12}% -
\TMP@EnsureCode{46}{12}% .
\TMP@EnsureCode{47}{12}% /
\TMP@EnsureCode{91}{12}% [
\TMP@EnsureCode{93}{12}% ]
\edef\GTS@AtEnd{\GTS@AtEnd\noexpand\endinput}
%    \end{macrocode}
%
% \subsection{Options}
%
%    \begin{macrocode}
\RequirePackage{kvoptions}[2009/07/17]
\SetupKeyvalOptions{%
  family=gettitlestring,%
  prefix=GTS@%
}
\newcommand*{\GetTitleStringSetup}{%
  \setkeys{gettitlestring}%
}
\DeclareBoolOption{expand}
\InputIfFileExists{gettitlestring.cfg}{}{}
\ProcessKeyvalOptions*\relax
%    \end{macrocode}
%
% \subsection{\cs{GetTitleString}}
%
%    \begin{macro}{\GetTitleString}
%    \begin{macrocode}
\newcommand*{\GetTitleString}{%
  \ifGTS@expand
    \expandafter\GetTitleStringExpand
  \else
    \expandafter\GetTitleStringNonExpand
  \fi
}
%    \end{macrocode}
%    \end{macro}
%    \begin{macro}{\GetTitleStringExpand}
%    \begin{macrocode}
\newcommand{\GetTitleStringExpand}[1]{%
  \def\GetTitleStringResult{#1}%
  \begingroup
    \GTS@DisablePredefinedCmds
    \GTS@DisableHook
    \edef\x{\endgroup
      \noexpand\def\noexpand\GetTitleStringResult{%
        \GetTitleStringResult
      }%
    }%
  \x
}
%    \end{macrocode}
%    \end{macro}
%    \begin{macro}{\GetTitleString}
%    \begin{macrocode}
\newcommand{\GetTitleStringNonExpand}[1]{%
  \def\GetTitleStringResult{#1}%
  \global\let\GTS@GlobalString\GetTitleStringResult
  \begingroup
    \GTS@RemoveLeft
    \GTS@RemoveRight
  \endgroup
  \let\GetTitleStringResult\GTS@GlobalString
}
%    \end{macrocode}
%    \end{macro}
%
% \subsubsection{Expand method}
%
%    \begin{macro}{\GTS@DisablePredefinedCmds}
%    \begin{macrocode}
\def\GTS@DisablePredefinedCmds{%
  \let\label\@gobble
  \let\zlabel\@gobble
  \let\zref@label\@gobble
  \let\zref@labelbylist\@gobbletwo
  \let\zref@labelbyprops\@gobbletwo
  \let\index\@gobble
  \let\glossary\@gobble
  \let\markboth\@gobbletwo
  \let\@mkboth\@gobbletwo
  \let\markright\@gobble
  \let\phantomsection\@empty
  \def\addcontentsline{\expandafter\@gobble\@gobbletwo}%
  \let\raggedright\@empty
  \let\raggedleft\@empty
  \let\centering\@empty
  \let\protect\@unexpandable@protect
  \let\enit@format\@empty % package enumitem
}
%    \end{macrocode}
%    \end{macro}
%
%    \begin{macro}{\GTS@DisableHook}
%    \begin{macrocode}
\providecommand*{\GTS@DisableHook}{}
%    \end{macrocode}
%    \end{macro}
%    \begin{macro}{\GetTitleStringDisableCommands}
%    \begin{macrocode}
\def\GetTitleStringDisableCommands{%
  \begingroup
    \makeatletter
    \GTS@DisableCommands
}
%    \end{macrocode}
%    \end{macro}
%    \begin{macro}{\GTS@DisableCommands}
%    \begin{macrocode}
\long\def\GTS@DisableCommands#1{%
    \toks0=\expandafter{\GTS@DisableHook}%
    \toks2={#1}%
    \xdef\GTS@GlobalString{\the\toks0 \the\toks2}%
  \endgroup
  \let\GTS@DisableHook\GTS@GlobalString
}
%    \end{macrocode}
%    \end{macro}
%
% \subsubsection{Non-expand method}
%
%    \begin{macrocode}
\def\GTS@RemoveLeft{%
  \toks@\expandafter\expandafter\expandafter{%
    \expandafter\GTS@Car\GTS@GlobalString{}{}{}{}\GTS@Nil
  }%
  \edef\GTS@Token{\the\toks@}%
  \GTS@PredefinedLeftCmds
  \expandafter\futurelet\expandafter\GTS@Token
  \expandafter\GTS@TestLeftSpace\GTS@GlobalString\GTS@Nil
  \GTS@End
}
\def\GTS@End{}
\long\def\GTS@TestLeft#1#2{%
  \def\GTS@temp{#1}%
  \ifx\GTS@temp\GTS@Token
    \toks@\expandafter\expandafter\expandafter{%
      \expandafter#2\GTS@GlobalString\GTS@Nil
    }%
    \expandafter\GTS@TestLeftEnd
  \fi
}
\long\def\GTS@TestLeftEnd#1\GTS@End{%
  \xdef\GTS@GlobalString{\the\toks@}%
  \GTS@RemoveLeft
}
\long\def\GTS@Car#1#2\GTS@Nil{#1}
\long\def\GTS@Cdr#1#2\GTS@Nil{#2}
\long\def\GTS@CdrTwo#1#2#3\GTS@Nil{#3}
\long\def\GTS@CdrThree#1#2#3#4\GTS@Nil{#4}
\long\def\GTS@CdrFour#1#2#3#4#5\GTS@Nil{#5}
\long\def\GTS@TestLeftSpace#1\GTS@Nil{%
  \ifx\GTS@Token\@sptoken
    \toks@\expandafter{%
      \romannumeral-0\GTS@GlobalString
    }%
    \expandafter\GTS@TestLeftEnd
  \fi
}
%    \end{macrocode}
%    \begin{macro}{\GTS@PredefinedLeftCmds}
%    \begin{macrocode}
\def\GTS@PredefinedLeftCmds{%
  \GTS@TestLeft\Hy@phantomsection\GTS@Cdr
  \GTS@TestLeft\Hy@SectionAnchor\GTS@Cdr
  \GTS@TestLeft\Hy@SectionAnchorHref\GTS@CdrTwo
  \GTS@TestLeft\label\GTS@CdrTwo
  \GTS@TestLeft\zlabel\GTS@CdrTwo
  \GTS@TestLeft\index\GTS@CdrTwo
  \GTS@TestLeft\glossary\GTS@CdrTwo
  \GTS@TestLeft\markboth\GTS@CdrThree
  \GTS@TestLeft\@mkboth\GTS@CdrThree
  \GTS@TestLeft\addcontentsline\GTS@CdrFour
  \GTS@TestLeft\enit@format\GTS@Cdr % package enumitem
}
%    \end{macrocode}
%    \end{macro}
%
%    \begin{macrocode}
\def\GTS@RemoveRight{%
  \toks@{}%
  \expandafter\GTS@TestRightLabel\GTS@GlobalString
      \label{}\GTS@Nil\@nil
  \GTS@RemoveRightSpace
}
\begingroup
  \def\GTS@temp#1{\endgroup
    \def\GTS@RemoveRightSpace{%
      \expandafter\GTS@TestRightSpace\GTS@GlobalString
          \GTS@Nil#1\GTS@Nil\@nil
    }%
  }%
\GTS@temp{ }
\def\GTS@TestRightSpace#1 \GTS@Nil#2\@nil{%
  \ifx\relax#2\relax
  \else
    \gdef\GTS@GlobalString{#1}%
    \expandafter\GTS@RemoveRightSpace
  \fi
}
\def\GTS@TestRightLabel#1\label#2#3\GTS@Nil#4\@nil{%
  \def\GTS@temp{#3}%
  \ifx\GTS@temp\@empty
    \expandafter\gdef\expandafter\GTS@GlobalString\expandafter{%
      \the\toks@
      #1%
    }%
    \expandafter\@gobble
  \else
    \expandafter\@firstofone
  \fi
  {%
    \toks@\expandafter{\the\toks@#1}%
    \GTS@TestRightLabel#3\GTS@Nil\@nil
  }%
}
%    \end{macrocode}
%
%    \begin{macrocode}
\GTS@AtEnd%
%</package>
%    \end{macrocode}
%
% \section{Test}
%
% \subsection{Catcode checks for loading}
%
%    \begin{macrocode}
%<*test1>
%    \end{macrocode}
%    \begin{macrocode}
\catcode`\{=1 %
\catcode`\}=2 %
\catcode`\#=6 %
\catcode`\@=11 %
\expandafter\ifx\csname count@\endcsname\relax
  \countdef\count@=255 %
\fi
\expandafter\ifx\csname @gobble\endcsname\relax
  \long\def\@gobble#1{}%
\fi
\expandafter\ifx\csname @firstofone\endcsname\relax
  \long\def\@firstofone#1{#1}%
\fi
\expandafter\ifx\csname loop\endcsname\relax
  \expandafter\@firstofone
\else
  \expandafter\@gobble
\fi
{%
  \def\loop#1\repeat{%
    \def\body{#1}%
    \iterate
  }%
  \def\iterate{%
    \body
      \let\next\iterate
    \else
      \let\next\relax
    \fi
    \next
  }%
  \let\repeat=\fi
}%
\def\RestoreCatcodes{}
\count@=0 %
\loop
  \edef\RestoreCatcodes{%
    \RestoreCatcodes
    \catcode\the\count@=\the\catcode\count@\relax
  }%
\ifnum\count@<255 %
  \advance\count@ 1 %
\repeat

\def\RangeCatcodeInvalid#1#2{%
  \count@=#1\relax
  \loop
    \catcode\count@=15 %
  \ifnum\count@<#2\relax
    \advance\count@ 1 %
  \repeat
}
\def\RangeCatcodeCheck#1#2#3{%
  \count@=#1\relax
  \loop
    \ifnum#3=\catcode\count@
    \else
      \errmessage{%
        Character \the\count@\space
        with wrong catcode \the\catcode\count@\space
        instead of \number#3%
      }%
    \fi
  \ifnum\count@<#2\relax
    \advance\count@ 1 %
  \repeat
}
\def\space{ }
\expandafter\ifx\csname LoadCommand\endcsname\relax
  \def\LoadCommand{\input gettitlestring.sty\relax}%
\fi
\def\Test{%
  \RangeCatcodeInvalid{0}{47}%
  \RangeCatcodeInvalid{58}{64}%
  \RangeCatcodeInvalid{91}{96}%
  \RangeCatcodeInvalid{123}{255}%
  \catcode`\@=12 %
  \catcode`\\=0 %
  \catcode`\%=14 %
  \LoadCommand
  \RangeCatcodeCheck{0}{36}{15}%
  \RangeCatcodeCheck{37}{37}{14}%
  \RangeCatcodeCheck{38}{47}{15}%
  \RangeCatcodeCheck{48}{57}{12}%
  \RangeCatcodeCheck{58}{63}{15}%
  \RangeCatcodeCheck{64}{64}{12}%
  \RangeCatcodeCheck{65}{90}{11}%
  \RangeCatcodeCheck{91}{91}{15}%
  \RangeCatcodeCheck{92}{92}{0}%
  \RangeCatcodeCheck{93}{96}{15}%
  \RangeCatcodeCheck{97}{122}{11}%
  \RangeCatcodeCheck{123}{255}{15}%
  \RestoreCatcodes
}
\Test
\csname @@end\endcsname
\end
%    \end{macrocode}
%    \begin{macrocode}
%</test1>
%    \end{macrocode}
%
% \subsection{Test of non-expand method}
%
%    \begin{macrocode}
%<*test2>
\NeedsTeXFormat{LaTeX2e}
\documentclass{minimal}
\usepackage{gettitlestring}[2016/05/16]
\usepackage{qstest}
\IncludeTests{*}
\LogTests{log}{*}{*}
\begin{document}
\begin{qstest}{non-expand}{non-expand}
  \def\test#1#2{%
    \sbox0{%
      \GetTitleString{#1}%
      \Expect{#2}*{\GetTitleStringResult}%
    }%
    \Expect{0.0pt}*{\the\wd0}%
  }%
  \test{}{}%
  \test{ }{}%
  \test{ x }{x}%
  \test{ x y }{x y}%
  \test{ \relax}{\relax}%
  \test{\label{f}a}{a}%
  \test{ \label{f}a}{a}%
  \test{\label{f} a}{a}%
  \test{ \label{f} a}{a}%
  \test{a\label{f}}{a}%
  \test{a\label{f} }{a}%
  \test{a \label{f}}{a}%
  \test{a \label{f} }{a}%
  \test{a\label{f}b\label{g}}{ab}%
  \test{a \label{f}b \label{g} }{a b}%
  \test{a\label{f} b \label{g} }{a b}%
\end{qstest}
\end{document}
%</test2>
%    \end{macrocode}
%
% \section{Installation}
%
% \subsection{Download}
%
% \paragraph{Package.} This package is available on
% CTAN\footnote{\CTANpkg{gettitlestring}}:
% \begin{description}
% \item[\CTAN{macros/latex/contrib/oberdiek/gettitlestring.dtx}] The source file.
% \item[\CTAN{macros/latex/contrib/oberdiek/gettitlestring.pdf}] Documentation.
% \end{description}
%
%
% \paragraph{Bundle.} All the packages of the bundle `oberdiek'
% are also available in a TDS compliant ZIP archive. There
% the packages are already unpacked and the documentation files
% are generated. The files and directories obey the TDS standard.
% \begin{description}
% \item[\CTANinstall{install/macros/latex/contrib/oberdiek.tds.zip}]
% \end{description}
% \emph{TDS} refers to the standard ``A Directory Structure
% for \TeX\ Files'' (\CTAN{tds/tds.pdf}). Directories
% with \xfile{texmf} in their name are usually organized this way.
%
% \subsection{Bundle installation}
%
% \paragraph{Unpacking.} Unpack the \xfile{oberdiek.tds.zip} in the
% TDS tree (also known as \xfile{texmf} tree) of your choice.
% Example (linux):
% \begin{quote}
%   |unzip oberdiek.tds.zip -d ~/texmf|
% \end{quote}
%
% \paragraph{Script installation.}
% Check the directory \xfile{TDS:scripts/oberdiek/} for
% scripts that need further installation steps.
% Package \xpackage{attachfile2} comes with the Perl script
% \xfile{pdfatfi.pl} that should be installed in such a way
% that it can be called as \texttt{pdfatfi}.
% Example (linux):
% \begin{quote}
%   |chmod +x scripts/oberdiek/pdfatfi.pl|\\
%   |cp scripts/oberdiek/pdfatfi.pl /usr/local/bin/|
% \end{quote}
%
% \subsection{Package installation}
%
% \paragraph{Unpacking.} The \xfile{.dtx} file is a self-extracting
% \docstrip\ archive. The files are extracted by running the
% \xfile{.dtx} through \plainTeX:
% \begin{quote}
%   \verb|tex gettitlestring.dtx|
% \end{quote}
%
% \paragraph{TDS.} Now the different files must be moved into
% the different directories in your installation TDS tree
% (also known as \xfile{texmf} tree):
% \begin{quote}
% \def\t{^^A
% \begin{tabular}{@{}>{\ttfamily}l@{ $\rightarrow$ }>{\ttfamily}l@{}}
%   gettitlestring.sty & tex/generic/oberdiek/gettitlestring.sty\\
%   gettitlestring.pdf & doc/latex/oberdiek/gettitlestring.pdf\\
%   test/gettitlestring-test1.tex & doc/latex/oberdiek/test/gettitlestring-test1.tex\\
%   test/gettitlestring-test2.tex & doc/latex/oberdiek/test/gettitlestring-test2.tex\\
%   gettitlestring.dtx & source/latex/oberdiek/gettitlestring.dtx\\
% \end{tabular}^^A
% }^^A
% \sbox0{\t}^^A
% \ifdim\wd0>\linewidth
%   \begingroup
%     \advance\linewidth by\leftmargin
%     \advance\linewidth by\rightmargin
%   \edef\x{\endgroup
%     \def\noexpand\lw{\the\linewidth}^^A
%   }\x
%   \def\lwbox{^^A
%     \leavevmode
%     \hbox to \linewidth{^^A
%       \kern-\leftmargin\relax
%       \hss
%       \usebox0
%       \hss
%       \kern-\rightmargin\relax
%     }^^A
%   }^^A
%   \ifdim\wd0>\lw
%     \sbox0{\small\t}^^A
%     \ifdim\wd0>\linewidth
%       \ifdim\wd0>\lw
%         \sbox0{\footnotesize\t}^^A
%         \ifdim\wd0>\linewidth
%           \ifdim\wd0>\lw
%             \sbox0{\scriptsize\t}^^A
%             \ifdim\wd0>\linewidth
%               \ifdim\wd0>\lw
%                 \sbox0{\tiny\t}^^A
%                 \ifdim\wd0>\linewidth
%                   \lwbox
%                 \else
%                   \usebox0
%                 \fi
%               \else
%                 \lwbox
%               \fi
%             \else
%               \usebox0
%             \fi
%           \else
%             \lwbox
%           \fi
%         \else
%           \usebox0
%         \fi
%       \else
%         \lwbox
%       \fi
%     \else
%       \usebox0
%     \fi
%   \else
%     \lwbox
%   \fi
% \else
%   \usebox0
% \fi
% \end{quote}
% If you have a \xfile{docstrip.cfg} that configures and enables \docstrip's
% TDS installing feature, then some files can already be in the right
% place, see the documentation of \docstrip.
%
% \subsection{Refresh file name databases}
%
% If your \TeX~distribution
% (\teTeX, \mikTeX, \dots) relies on file name databases, you must refresh
% these. For example, \teTeX\ users run \verb|texhash| or
% \verb|mktexlsr|.
%
% \subsection{Some details for the interested}
%
% \paragraph{Attached source.}
%
% The PDF documentation on CTAN also includes the
% \xfile{.dtx} source file. It can be extracted by
% AcrobatReader 6 or higher. Another option is \textsf{pdftk},
% e.g. unpack the file into the current directory:
% \begin{quote}
%   \verb|pdftk gettitlestring.pdf unpack_files output .|
% \end{quote}
%
% \paragraph{Unpacking with \LaTeX.}
% The \xfile{.dtx} chooses its action depending on the format:
% \begin{description}
% \item[\plainTeX:] Run \docstrip\ and extract the files.
% \item[\LaTeX:] Generate the documentation.
% \end{description}
% If you insist on using \LaTeX\ for \docstrip\ (really,
% \docstrip\ does not need \LaTeX), then inform the autodetect routine
% about your intention:
% \begin{quote}
%   \verb|latex \let\install=y% \iffalse meta-comment
%
% File: gettitlestring.dtx
% Version: 2016/05/16 v1.5
% Info: Cleanup title references
%
% Copyright (C) 2009, 2010 by
%    Heiko Oberdiek <heiko.oberdiek at googlemail.com>
%    2016
%    https://github.com/ho-tex/oberdiek/issues
%
% This work may be distributed and/or modified under the
% conditions of the LaTeX Project Public License, either
% version 1.3c of this license or (at your option) any later
% version. This version of this license is in
%    http://www.latex-project.org/lppl/lppl-1-3c.txt
% and the latest version of this license is in
%    http://www.latex-project.org/lppl.txt
% and version 1.3 or later is part of all distributions of
% LaTeX version 2005/12/01 or later.
%
% This work has the LPPL maintenance status "maintained".
%
% This Current Maintainer of this work is Heiko Oberdiek.
%
% The Base Interpreter refers to any `TeX-Format',
% because some files are installed in TDS:tex/generic//.
%
% This work consists of the main source file gettitlestring.dtx
% and the derived files
%    gettitlestring.sty, gettitlestring.pdf, gettitlestring.ins,
%    gettitlestring.drv, gettitlestring-test1.tex,
%    gettitlestring-test2.tex.
%
% Distribution:
%    CTAN:macros/latex/contrib/oberdiek/gettitlestring.dtx
%    CTAN:macros/latex/contrib/oberdiek/gettitlestring.pdf
%
% Unpacking:
%    (a) If gettitlestring.ins is present:
%           tex gettitlestring.ins
%    (b) Without gettitlestring.ins:
%           tex gettitlestring.dtx
%    (c) If you insist on using LaTeX
%           latex \let\install=y\input{gettitlestring.dtx}
%        (quote the arguments according to the demands of your shell)
%
% Documentation:
%    (a) If gettitlestring.drv is present:
%           latex gettitlestring.drv
%    (b) Without gettitlestring.drv:
%           latex gettitlestring.dtx; ...
%    The class ltxdoc loads the configuration file ltxdoc.cfg
%    if available. Here you can specify further options, e.g.
%    use A4 as paper format:
%       \PassOptionsToClass{a4paper}{article}
%
%    Programm calls to get the documentation (example):
%       pdflatex gettitlestring.dtx
%       makeindex -s gind.ist gettitlestring.idx
%       pdflatex gettitlestring.dtx
%       makeindex -s gind.ist gettitlestring.idx
%       pdflatex gettitlestring.dtx
%
% Installation:
%    TDS:tex/generic/oberdiek/gettitlestring.sty
%    TDS:doc/latex/oberdiek/gettitlestring.pdf
%    TDS:doc/latex/oberdiek/test/gettitlestring-test1.tex
%    TDS:doc/latex/oberdiek/test/gettitlestring-test2.tex
%    TDS:source/latex/oberdiek/gettitlestring.dtx
%
%<*ignore>
\begingroup
  \catcode123=1 %
  \catcode125=2 %
  \def\x{LaTeX2e}%
\expandafter\endgroup
\ifcase 0\ifx\install y1\fi\expandafter
         \ifx\csname processbatchFile\endcsname\relax\else1\fi
         \ifx\fmtname\x\else 1\fi\relax
\else\csname fi\endcsname
%</ignore>
%<*install>
\input docstrip.tex
\Msg{************************************************************************}
\Msg{* Installation}
\Msg{* Package: gettitlestring 2016/05/16 v1.5 Cleanup title references (HO)}
\Msg{************************************************************************}

\keepsilent
\askforoverwritefalse

\let\MetaPrefix\relax
\preamble

This is a generated file.

Project: gettitlestring
Version: 2016/05/16 v1.5

Copyright (C) 2009, 2010 by
   Heiko Oberdiek <heiko.oberdiek at googlemail.com>

This work may be distributed and/or modified under the
conditions of the LaTeX Project Public License, either
version 1.3c of this license or (at your option) any later
version. This version of this license is in
   http://www.latex-project.org/lppl/lppl-1-3c.txt
and the latest version of this license is in
   http://www.latex-project.org/lppl.txt
and version 1.3 or later is part of all distributions of
LaTeX version 2005/12/01 or later.

This work has the LPPL maintenance status "maintained".

This Current Maintainer of this work is Heiko Oberdiek.

The Base Interpreter refers to any `TeX-Format',
because some files are installed in TDS:tex/generic//.

This work consists of the main source file gettitlestring.dtx
and the derived files
   gettitlestring.sty, gettitlestring.pdf, gettitlestring.ins,
   gettitlestring.drv, gettitlestring-test1.tex,
   gettitlestring-test2.tex.

\endpreamble
\let\MetaPrefix\DoubleperCent

\generate{%
  \file{gettitlestring.ins}{\from{gettitlestring.dtx}{install}}%
  \file{gettitlestring.drv}{\from{gettitlestring.dtx}{driver}}%
  \usedir{tex/generic/oberdiek}%
  \file{gettitlestring.sty}{\from{gettitlestring.dtx}{package}}%
%  \usedir{doc/latex/oberdiek/test}%
%  \file{gettitlestring-test1.tex}{\from{gettitlestring.dtx}{test1}}%
%  \file{gettitlestring-test2.tex}{\from{gettitlestring.dtx}{test2}}%
  \nopreamble
  \nopostamble
%  \usedir{source/latex/oberdiek/catalogue}%
%  \file{gettitlestring.xml}{\from{gettitlestring.dtx}{catalogue}}%
}

\catcode32=13\relax% active space
\let =\space%
\Msg{************************************************************************}
\Msg{*}
\Msg{* To finish the installation you have to move the following}
\Msg{* file into a directory searched by TeX:}
\Msg{*}
\Msg{*     gettitlestring.sty}
\Msg{*}
\Msg{* To produce the documentation run the file `gettitlestring.drv'}
\Msg{* through LaTeX.}
\Msg{*}
\Msg{* Happy TeXing!}
\Msg{*}
\Msg{************************************************************************}

\endbatchfile
%</install>
%<*ignore>
\fi
%</ignore>
%<*driver>
\NeedsTeXFormat{LaTeX2e}
\ProvidesFile{gettitlestring.drv}%
  [2016/05/16 v1.5 Cleanup title references (HO)]%
\documentclass{ltxdoc}
\usepackage{holtxdoc}[2011/11/22]
\begin{document}
  \DocInput{gettitlestring.dtx}%
\end{document}
%</driver>
% \fi
%
%
% \CharacterTable
%  {Upper-case    \A\B\C\D\E\F\G\H\I\J\K\L\M\N\O\P\Q\R\S\T\U\V\W\X\Y\Z
%   Lower-case    \a\b\c\d\e\f\g\h\i\j\k\l\m\n\o\p\q\r\s\t\u\v\w\x\y\z
%   Digits        \0\1\2\3\4\5\6\7\8\9
%   Exclamation   \!     Double quote  \"     Hash (number) \#
%   Dollar        \$     Percent       \%     Ampersand     \&
%   Acute accent  \'     Left paren    \(     Right paren   \)
%   Asterisk      \*     Plus          \+     Comma         \,
%   Minus         \-     Point         \.     Solidus       \/
%   Colon         \:     Semicolon     \;     Less than     \<
%   Equals        \=     Greater than  \>     Question mark \?
%   Commercial at \@     Left bracket  \[     Backslash     \\
%   Right bracket \]     Circumflex    \^     Underscore    \_
%   Grave accent  \`     Left brace    \{     Vertical bar  \|
%   Right brace   \}     Tilde         \~}
%
% \GetFileInfo{gettitlestring.drv}
%
% \title{The \xpackage{gettitlestring} package}
% \date{2016/05/16 v1.5}
% \author{Heiko Oberdiek\thanks
% {Please report any issues at \url{https://github.com/ho-tex/oberdiek/issues}}\\
% \xemail{heiko.oberdiek at googlemail.com}}
%
% \maketitle
%
% \begin{abstract}
% The \LaTeX\ package addresses packages that are dealing with
% references to titles (\cs{section}, \cs{caption}, \dots).
% The package tries to remove \cs{label} and other
% commands from title strings.
% \end{abstract}
%
% \tableofcontents
%
% \section{Documentation}
%
% \subsection{Macros}
%
% \begin{declcs}{GetTitleStringSetup} \M{key value list}
% \end{declcs}
% The options are given as comma separated key value pairs.
% See section \ref{sec:options}.
%
% \begin{declcs}{GetTitleString} \M{text}\\
% \cs{GetTitleStringExpand} \M{text}\\
% \cs{GetTitleStringNonExpand} \M{text}
% \end{declcs}
% Macro \cs{GetTitleString} tries to remove unwanted stuff from \meta{text}
% the result is stored in Macro \cs{GetTitleStringResult}.
% Two methods are available:
% \begin{description}
% \item[\cs{GetTitleStringExpand}:]
% The \meta{text} is expanded in a context where the unwanted
% macros are redefined to remove themselves.
% This is the method used in packages \xpackage{titleref}~\cite{titleref},
% \xpackage{zref-titleref}~\cite{zref}
% or class \xclass{memoir}~\cite{memoir}.
% \cs{protect} is supported, but fragile material might break.
% \item[\cs{GetTitleStringNonExpand}:]
% The \meta{text} is not expanded. Thus the removal of unwanted
% material is more difficult. It is especially removed at the
% start of the \meta{text} and spaces are removed from the end.
% Currently only \cs{label} is removed in the whole string,
% if it is not hidden inside curly braces or part of macro
% definitions. Thus the removal of unwanted stuff might not be
% complete, but fragile material will not break.
% (But the result string can break at a later time, of course).
% \end{description}
% Option \xoption{expand} controls which method is used by
% macro \cs{GetTitleString}.
%
% \begin{declcs}{GetTitleStringDisableCommands} \M{code}
% \end{declcs}
% The \meta{code} is called right before the
% text is expanded in \cs{GetTitleStringExpand}.
% Additional definitions can be given for macros that
% should be removed.
% Keep in mind that expansion means that the definitions
% must work in expandable context. Macros like
% \cs{@ifstar} or \cs{@ifnextchar} or optional arguments
% will not work. The macro names in \meta{code} may contain
% the at sign |@|, it has catcode 11 (letter).
%
% \subsection{Options}\label{sec:options}
%
% \begin{description}
% \item[\xoption{expand}:] Boolean option, takes values |true| or |false|.
% No value means |true|. The option specifies the method to remove
% unwanted stuff from the title string, see below.
% \end{description}
% Options can be set at the following places:
% \begin{itemize}
% \item \cs{usepackage}
% \item Configuration file \xfile{gettitlestring.cfg}.
% \item \cs{GetTitleStringSetup}
% \end{itemize}
%
% \StopEventually{
% }
%
% \section{Implementation}
%    \begin{macrocode}
%<*package>
%    \end{macrocode}
%    Reload check, especially if the package is not used with \LaTeX.
%    \begin{macrocode}
\begingroup\catcode61\catcode48\catcode32=10\relax%
  \catcode13=5 % ^^M
  \endlinechar=13 %
  \catcode35=6 % #
  \catcode39=12 % '
  \catcode44=12 % ,
  \catcode45=12 % -
  \catcode46=12 % .
  \catcode58=12 % :
  \catcode64=11 % @
  \catcode123=1 % {
  \catcode125=2 % }
  \expandafter\let\expandafter\x\csname ver@gettitlestring.sty\endcsname
  \ifx\x\relax % plain-TeX, first loading
  \else
    \def\empty{}%
    \ifx\x\empty % LaTeX, first loading,
      % variable is initialized, but \ProvidesPackage not yet seen
    \else
      \expandafter\ifx\csname PackageInfo\endcsname\relax
        \def\x#1#2{%
          \immediate\write-1{Package #1 Info: #2.}%
        }%
      \else
        \def\x#1#2{\PackageInfo{#1}{#2, stopped}}%
      \fi
      \x{gettitlestring}{The package is already loaded}%
      \aftergroup\endinput
    \fi
  \fi
\endgroup%
%    \end{macrocode}
%    Package identification:
%    \begin{macrocode}
\begingroup\catcode61\catcode48\catcode32=10\relax%
  \catcode13=5 % ^^M
  \endlinechar=13 %
  \catcode35=6 % #
  \catcode39=12 % '
  \catcode40=12 % (
  \catcode41=12 % )
  \catcode44=12 % ,
  \catcode45=12 % -
  \catcode46=12 % .
  \catcode47=12 % /
  \catcode58=12 % :
  \catcode64=11 % @
  \catcode91=12 % [
  \catcode93=12 % ]
  \catcode123=1 % {
  \catcode125=2 % }
  \expandafter\ifx\csname ProvidesPackage\endcsname\relax
    \def\x#1#2#3[#4]{\endgroup
      \immediate\write-1{Package: #3 #4}%
      \xdef#1{#4}%
    }%
  \else
    \def\x#1#2[#3]{\endgroup
      #2[{#3}]%
      \ifx#1\@undefined
        \xdef#1{#3}%
      \fi
      \ifx#1\relax
        \xdef#1{#3}%
      \fi
    }%
  \fi
\expandafter\x\csname ver@gettitlestring.sty\endcsname
\ProvidesPackage{gettitlestring}%
  [2016/05/16 v1.5 Cleanup title references (HO)]%
%    \end{macrocode}
%
%    \begin{macrocode}
\begingroup\catcode61\catcode48\catcode32=10\relax%
  \catcode13=5 % ^^M
  \endlinechar=13 %
  \catcode123=1 % {
  \catcode125=2 % }
  \catcode64=11 % @
  \def\x{\endgroup
    \expandafter\edef\csname GTS@AtEnd\endcsname{%
      \endlinechar=\the\endlinechar\relax
      \catcode13=\the\catcode13\relax
      \catcode32=\the\catcode32\relax
      \catcode35=\the\catcode35\relax
      \catcode61=\the\catcode61\relax
      \catcode64=\the\catcode64\relax
      \catcode123=\the\catcode123\relax
      \catcode125=\the\catcode125\relax
    }%
  }%
\x\catcode61\catcode48\catcode32=10\relax%
\catcode13=5 % ^^M
\endlinechar=13 %
\catcode35=6 % #
\catcode64=11 % @
\catcode123=1 % {
\catcode125=2 % }
\def\TMP@EnsureCode#1#2{%
  \edef\GTS@AtEnd{%
    \GTS@AtEnd
    \catcode#1=\the\catcode#1\relax
  }%
  \catcode#1=#2\relax
}
\TMP@EnsureCode{42}{12}% *
\TMP@EnsureCode{44}{12}% ,
\TMP@EnsureCode{45}{12}% -
\TMP@EnsureCode{46}{12}% .
\TMP@EnsureCode{47}{12}% /
\TMP@EnsureCode{91}{12}% [
\TMP@EnsureCode{93}{12}% ]
\edef\GTS@AtEnd{\GTS@AtEnd\noexpand\endinput}
%    \end{macrocode}
%
% \subsection{Options}
%
%    \begin{macrocode}
\RequirePackage{kvoptions}[2009/07/17]
\SetupKeyvalOptions{%
  family=gettitlestring,%
  prefix=GTS@%
}
\newcommand*{\GetTitleStringSetup}{%
  \setkeys{gettitlestring}%
}
\DeclareBoolOption{expand}
\InputIfFileExists{gettitlestring.cfg}{}{}
\ProcessKeyvalOptions*\relax
%    \end{macrocode}
%
% \subsection{\cs{GetTitleString}}
%
%    \begin{macro}{\GetTitleString}
%    \begin{macrocode}
\newcommand*{\GetTitleString}{%
  \ifGTS@expand
    \expandafter\GetTitleStringExpand
  \else
    \expandafter\GetTitleStringNonExpand
  \fi
}
%    \end{macrocode}
%    \end{macro}
%    \begin{macro}{\GetTitleStringExpand}
%    \begin{macrocode}
\newcommand{\GetTitleStringExpand}[1]{%
  \def\GetTitleStringResult{#1}%
  \begingroup
    \GTS@DisablePredefinedCmds
    \GTS@DisableHook
    \edef\x{\endgroup
      \noexpand\def\noexpand\GetTitleStringResult{%
        \GetTitleStringResult
      }%
    }%
  \x
}
%    \end{macrocode}
%    \end{macro}
%    \begin{macro}{\GetTitleString}
%    \begin{macrocode}
\newcommand{\GetTitleStringNonExpand}[1]{%
  \def\GetTitleStringResult{#1}%
  \global\let\GTS@GlobalString\GetTitleStringResult
  \begingroup
    \GTS@RemoveLeft
    \GTS@RemoveRight
  \endgroup
  \let\GetTitleStringResult\GTS@GlobalString
}
%    \end{macrocode}
%    \end{macro}
%
% \subsubsection{Expand method}
%
%    \begin{macro}{\GTS@DisablePredefinedCmds}
%    \begin{macrocode}
\def\GTS@DisablePredefinedCmds{%
  \let\label\@gobble
  \let\zlabel\@gobble
  \let\zref@label\@gobble
  \let\zref@labelbylist\@gobbletwo
  \let\zref@labelbyprops\@gobbletwo
  \let\index\@gobble
  \let\glossary\@gobble
  \let\markboth\@gobbletwo
  \let\@mkboth\@gobbletwo
  \let\markright\@gobble
  \let\phantomsection\@empty
  \def\addcontentsline{\expandafter\@gobble\@gobbletwo}%
  \let\raggedright\@empty
  \let\raggedleft\@empty
  \let\centering\@empty
  \let\protect\@unexpandable@protect
  \let\enit@format\@empty % package enumitem
}
%    \end{macrocode}
%    \end{macro}
%
%    \begin{macro}{\GTS@DisableHook}
%    \begin{macrocode}
\providecommand*{\GTS@DisableHook}{}
%    \end{macrocode}
%    \end{macro}
%    \begin{macro}{\GetTitleStringDisableCommands}
%    \begin{macrocode}
\def\GetTitleStringDisableCommands{%
  \begingroup
    \makeatletter
    \GTS@DisableCommands
}
%    \end{macrocode}
%    \end{macro}
%    \begin{macro}{\GTS@DisableCommands}
%    \begin{macrocode}
\long\def\GTS@DisableCommands#1{%
    \toks0=\expandafter{\GTS@DisableHook}%
    \toks2={#1}%
    \xdef\GTS@GlobalString{\the\toks0 \the\toks2}%
  \endgroup
  \let\GTS@DisableHook\GTS@GlobalString
}
%    \end{macrocode}
%    \end{macro}
%
% \subsubsection{Non-expand method}
%
%    \begin{macrocode}
\def\GTS@RemoveLeft{%
  \toks@\expandafter\expandafter\expandafter{%
    \expandafter\GTS@Car\GTS@GlobalString{}{}{}{}\GTS@Nil
  }%
  \edef\GTS@Token{\the\toks@}%
  \GTS@PredefinedLeftCmds
  \expandafter\futurelet\expandafter\GTS@Token
  \expandafter\GTS@TestLeftSpace\GTS@GlobalString\GTS@Nil
  \GTS@End
}
\def\GTS@End{}
\long\def\GTS@TestLeft#1#2{%
  \def\GTS@temp{#1}%
  \ifx\GTS@temp\GTS@Token
    \toks@\expandafter\expandafter\expandafter{%
      \expandafter#2\GTS@GlobalString\GTS@Nil
    }%
    \expandafter\GTS@TestLeftEnd
  \fi
}
\long\def\GTS@TestLeftEnd#1\GTS@End{%
  \xdef\GTS@GlobalString{\the\toks@}%
  \GTS@RemoveLeft
}
\long\def\GTS@Car#1#2\GTS@Nil{#1}
\long\def\GTS@Cdr#1#2\GTS@Nil{#2}
\long\def\GTS@CdrTwo#1#2#3\GTS@Nil{#3}
\long\def\GTS@CdrThree#1#2#3#4\GTS@Nil{#4}
\long\def\GTS@CdrFour#1#2#3#4#5\GTS@Nil{#5}
\long\def\GTS@TestLeftSpace#1\GTS@Nil{%
  \ifx\GTS@Token\@sptoken
    \toks@\expandafter{%
      \romannumeral-0\GTS@GlobalString
    }%
    \expandafter\GTS@TestLeftEnd
  \fi
}
%    \end{macrocode}
%    \begin{macro}{\GTS@PredefinedLeftCmds}
%    \begin{macrocode}
\def\GTS@PredefinedLeftCmds{%
  \GTS@TestLeft\Hy@phantomsection\GTS@Cdr
  \GTS@TestLeft\Hy@SectionAnchor\GTS@Cdr
  \GTS@TestLeft\Hy@SectionAnchorHref\GTS@CdrTwo
  \GTS@TestLeft\label\GTS@CdrTwo
  \GTS@TestLeft\zlabel\GTS@CdrTwo
  \GTS@TestLeft\index\GTS@CdrTwo
  \GTS@TestLeft\glossary\GTS@CdrTwo
  \GTS@TestLeft\markboth\GTS@CdrThree
  \GTS@TestLeft\@mkboth\GTS@CdrThree
  \GTS@TestLeft\addcontentsline\GTS@CdrFour
  \GTS@TestLeft\enit@format\GTS@Cdr % package enumitem
}
%    \end{macrocode}
%    \end{macro}
%
%    \begin{macrocode}
\def\GTS@RemoveRight{%
  \toks@{}%
  \expandafter\GTS@TestRightLabel\GTS@GlobalString
      \label{}\GTS@Nil\@nil
  \GTS@RemoveRightSpace
}
\begingroup
  \def\GTS@temp#1{\endgroup
    \def\GTS@RemoveRightSpace{%
      \expandafter\GTS@TestRightSpace\GTS@GlobalString
          \GTS@Nil#1\GTS@Nil\@nil
    }%
  }%
\GTS@temp{ }
\def\GTS@TestRightSpace#1 \GTS@Nil#2\@nil{%
  \ifx\relax#2\relax
  \else
    \gdef\GTS@GlobalString{#1}%
    \expandafter\GTS@RemoveRightSpace
  \fi
}
\def\GTS@TestRightLabel#1\label#2#3\GTS@Nil#4\@nil{%
  \def\GTS@temp{#3}%
  \ifx\GTS@temp\@empty
    \expandafter\gdef\expandafter\GTS@GlobalString\expandafter{%
      \the\toks@
      #1%
    }%
    \expandafter\@gobble
  \else
    \expandafter\@firstofone
  \fi
  {%
    \toks@\expandafter{\the\toks@#1}%
    \GTS@TestRightLabel#3\GTS@Nil\@nil
  }%
}
%    \end{macrocode}
%
%    \begin{macrocode}
\GTS@AtEnd%
%</package>
%    \end{macrocode}
%
% \section{Test}
%
% \subsection{Catcode checks for loading}
%
%    \begin{macrocode}
%<*test1>
%    \end{macrocode}
%    \begin{macrocode}
\catcode`\{=1 %
\catcode`\}=2 %
\catcode`\#=6 %
\catcode`\@=11 %
\expandafter\ifx\csname count@\endcsname\relax
  \countdef\count@=255 %
\fi
\expandafter\ifx\csname @gobble\endcsname\relax
  \long\def\@gobble#1{}%
\fi
\expandafter\ifx\csname @firstofone\endcsname\relax
  \long\def\@firstofone#1{#1}%
\fi
\expandafter\ifx\csname loop\endcsname\relax
  \expandafter\@firstofone
\else
  \expandafter\@gobble
\fi
{%
  \def\loop#1\repeat{%
    \def\body{#1}%
    \iterate
  }%
  \def\iterate{%
    \body
      \let\next\iterate
    \else
      \let\next\relax
    \fi
    \next
  }%
  \let\repeat=\fi
}%
\def\RestoreCatcodes{}
\count@=0 %
\loop
  \edef\RestoreCatcodes{%
    \RestoreCatcodes
    \catcode\the\count@=\the\catcode\count@\relax
  }%
\ifnum\count@<255 %
  \advance\count@ 1 %
\repeat

\def\RangeCatcodeInvalid#1#2{%
  \count@=#1\relax
  \loop
    \catcode\count@=15 %
  \ifnum\count@<#2\relax
    \advance\count@ 1 %
  \repeat
}
\def\RangeCatcodeCheck#1#2#3{%
  \count@=#1\relax
  \loop
    \ifnum#3=\catcode\count@
    \else
      \errmessage{%
        Character \the\count@\space
        with wrong catcode \the\catcode\count@\space
        instead of \number#3%
      }%
    \fi
  \ifnum\count@<#2\relax
    \advance\count@ 1 %
  \repeat
}
\def\space{ }
\expandafter\ifx\csname LoadCommand\endcsname\relax
  \def\LoadCommand{\input gettitlestring.sty\relax}%
\fi
\def\Test{%
  \RangeCatcodeInvalid{0}{47}%
  \RangeCatcodeInvalid{58}{64}%
  \RangeCatcodeInvalid{91}{96}%
  \RangeCatcodeInvalid{123}{255}%
  \catcode`\@=12 %
  \catcode`\\=0 %
  \catcode`\%=14 %
  \LoadCommand
  \RangeCatcodeCheck{0}{36}{15}%
  \RangeCatcodeCheck{37}{37}{14}%
  \RangeCatcodeCheck{38}{47}{15}%
  \RangeCatcodeCheck{48}{57}{12}%
  \RangeCatcodeCheck{58}{63}{15}%
  \RangeCatcodeCheck{64}{64}{12}%
  \RangeCatcodeCheck{65}{90}{11}%
  \RangeCatcodeCheck{91}{91}{15}%
  \RangeCatcodeCheck{92}{92}{0}%
  \RangeCatcodeCheck{93}{96}{15}%
  \RangeCatcodeCheck{97}{122}{11}%
  \RangeCatcodeCheck{123}{255}{15}%
  \RestoreCatcodes
}
\Test
\csname @@end\endcsname
\end
%    \end{macrocode}
%    \begin{macrocode}
%</test1>
%    \end{macrocode}
%
% \subsection{Test of non-expand method}
%
%    \begin{macrocode}
%<*test2>
\NeedsTeXFormat{LaTeX2e}
\documentclass{minimal}
\usepackage{gettitlestring}[2016/05/16]
\usepackage{qstest}
\IncludeTests{*}
\LogTests{log}{*}{*}
\begin{document}
\begin{qstest}{non-expand}{non-expand}
  \def\test#1#2{%
    \sbox0{%
      \GetTitleString{#1}%
      \Expect{#2}*{\GetTitleStringResult}%
    }%
    \Expect{0.0pt}*{\the\wd0}%
  }%
  \test{}{}%
  \test{ }{}%
  \test{ x }{x}%
  \test{ x y }{x y}%
  \test{ \relax}{\relax}%
  \test{\label{f}a}{a}%
  \test{ \label{f}a}{a}%
  \test{\label{f} a}{a}%
  \test{ \label{f} a}{a}%
  \test{a\label{f}}{a}%
  \test{a\label{f} }{a}%
  \test{a \label{f}}{a}%
  \test{a \label{f} }{a}%
  \test{a\label{f}b\label{g}}{ab}%
  \test{a \label{f}b \label{g} }{a b}%
  \test{a\label{f} b \label{g} }{a b}%
\end{qstest}
\end{document}
%</test2>
%    \end{macrocode}
%
% \section{Installation}
%
% \subsection{Download}
%
% \paragraph{Package.} This package is available on
% CTAN\footnote{\CTANpkg{gettitlestring}}:
% \begin{description}
% \item[\CTAN{macros/latex/contrib/oberdiek/gettitlestring.dtx}] The source file.
% \item[\CTAN{macros/latex/contrib/oberdiek/gettitlestring.pdf}] Documentation.
% \end{description}
%
%
% \paragraph{Bundle.} All the packages of the bundle `oberdiek'
% are also available in a TDS compliant ZIP archive. There
% the packages are already unpacked and the documentation files
% are generated. The files and directories obey the TDS standard.
% \begin{description}
% \item[\CTANinstall{install/macros/latex/contrib/oberdiek.tds.zip}]
% \end{description}
% \emph{TDS} refers to the standard ``A Directory Structure
% for \TeX\ Files'' (\CTAN{tds/tds.pdf}). Directories
% with \xfile{texmf} in their name are usually organized this way.
%
% \subsection{Bundle installation}
%
% \paragraph{Unpacking.} Unpack the \xfile{oberdiek.tds.zip} in the
% TDS tree (also known as \xfile{texmf} tree) of your choice.
% Example (linux):
% \begin{quote}
%   |unzip oberdiek.tds.zip -d ~/texmf|
% \end{quote}
%
% \paragraph{Script installation.}
% Check the directory \xfile{TDS:scripts/oberdiek/} for
% scripts that need further installation steps.
% Package \xpackage{attachfile2} comes with the Perl script
% \xfile{pdfatfi.pl} that should be installed in such a way
% that it can be called as \texttt{pdfatfi}.
% Example (linux):
% \begin{quote}
%   |chmod +x scripts/oberdiek/pdfatfi.pl|\\
%   |cp scripts/oberdiek/pdfatfi.pl /usr/local/bin/|
% \end{quote}
%
% \subsection{Package installation}
%
% \paragraph{Unpacking.} The \xfile{.dtx} file is a self-extracting
% \docstrip\ archive. The files are extracted by running the
% \xfile{.dtx} through \plainTeX:
% \begin{quote}
%   \verb|tex gettitlestring.dtx|
% \end{quote}
%
% \paragraph{TDS.} Now the different files must be moved into
% the different directories in your installation TDS tree
% (also known as \xfile{texmf} tree):
% \begin{quote}
% \def\t{^^A
% \begin{tabular}{@{}>{\ttfamily}l@{ $\rightarrow$ }>{\ttfamily}l@{}}
%   gettitlestring.sty & tex/generic/oberdiek/gettitlestring.sty\\
%   gettitlestring.pdf & doc/latex/oberdiek/gettitlestring.pdf\\
%   test/gettitlestring-test1.tex & doc/latex/oberdiek/test/gettitlestring-test1.tex\\
%   test/gettitlestring-test2.tex & doc/latex/oberdiek/test/gettitlestring-test2.tex\\
%   gettitlestring.dtx & source/latex/oberdiek/gettitlestring.dtx\\
% \end{tabular}^^A
% }^^A
% \sbox0{\t}^^A
% \ifdim\wd0>\linewidth
%   \begingroup
%     \advance\linewidth by\leftmargin
%     \advance\linewidth by\rightmargin
%   \edef\x{\endgroup
%     \def\noexpand\lw{\the\linewidth}^^A
%   }\x
%   \def\lwbox{^^A
%     \leavevmode
%     \hbox to \linewidth{^^A
%       \kern-\leftmargin\relax
%       \hss
%       \usebox0
%       \hss
%       \kern-\rightmargin\relax
%     }^^A
%   }^^A
%   \ifdim\wd0>\lw
%     \sbox0{\small\t}^^A
%     \ifdim\wd0>\linewidth
%       \ifdim\wd0>\lw
%         \sbox0{\footnotesize\t}^^A
%         \ifdim\wd0>\linewidth
%           \ifdim\wd0>\lw
%             \sbox0{\scriptsize\t}^^A
%             \ifdim\wd0>\linewidth
%               \ifdim\wd0>\lw
%                 \sbox0{\tiny\t}^^A
%                 \ifdim\wd0>\linewidth
%                   \lwbox
%                 \else
%                   \usebox0
%                 \fi
%               \else
%                 \lwbox
%               \fi
%             \else
%               \usebox0
%             \fi
%           \else
%             \lwbox
%           \fi
%         \else
%           \usebox0
%         \fi
%       \else
%         \lwbox
%       \fi
%     \else
%       \usebox0
%     \fi
%   \else
%     \lwbox
%   \fi
% \else
%   \usebox0
% \fi
% \end{quote}
% If you have a \xfile{docstrip.cfg} that configures and enables \docstrip's
% TDS installing feature, then some files can already be in the right
% place, see the documentation of \docstrip.
%
% \subsection{Refresh file name databases}
%
% If your \TeX~distribution
% (\teTeX, \mikTeX, \dots) relies on file name databases, you must refresh
% these. For example, \teTeX\ users run \verb|texhash| or
% \verb|mktexlsr|.
%
% \subsection{Some details for the interested}
%
% \paragraph{Attached source.}
%
% The PDF documentation on CTAN also includes the
% \xfile{.dtx} source file. It can be extracted by
% AcrobatReader 6 or higher. Another option is \textsf{pdftk},
% e.g. unpack the file into the current directory:
% \begin{quote}
%   \verb|pdftk gettitlestring.pdf unpack_files output .|
% \end{quote}
%
% \paragraph{Unpacking with \LaTeX.}
% The \xfile{.dtx} chooses its action depending on the format:
% \begin{description}
% \item[\plainTeX:] Run \docstrip\ and extract the files.
% \item[\LaTeX:] Generate the documentation.
% \end{description}
% If you insist on using \LaTeX\ for \docstrip\ (really,
% \docstrip\ does not need \LaTeX), then inform the autodetect routine
% about your intention:
% \begin{quote}
%   \verb|latex \let\install=y\input{gettitlestring.dtx}|
% \end{quote}
% Do not forget to quote the argument according to the demands
% of your shell.
%
% \paragraph{Generating the documentation.}
% You can use both the \xfile{.dtx} or the \xfile{.drv} to generate
% the documentation. The process can be configured by the
% configuration file \xfile{ltxdoc.cfg}. For instance, put this
% line into this file, if you want to have A4 as paper format:
% \begin{quote}
%   \verb|\PassOptionsToClass{a4paper}{article}|
% \end{quote}
% An example follows how to generate the
% documentation with pdf\LaTeX:
% \begin{quote}
%\begin{verbatim}
%pdflatex gettitlestring.dtx
%makeindex -s gind.ist gettitlestring.idx
%pdflatex gettitlestring.dtx
%makeindex -s gind.ist gettitlestring.idx
%pdflatex gettitlestring.dtx
%\end{verbatim}
% \end{quote}
%
% \begin{thebibliography}{9}
%
% \bibitem{memoir}
% Peter Wilson, Lars Madsen:
% \textit{The Memoir Class};
% 2009/11/17 v1.61803398c;
% \CTANpkg{memoir}
%
% \bibitem{titleref}
% Donald Arsenau:
% \textit{Titleref.sty};
% 2001/04/05 ver 3.1;
% \CTAN{macros/latex/contrib/misc/titleref.sty}
%
% \bibitem{zref}
% Heiko Oberdiek:
% \textit{The \xpackage{zref} package};
% 2009/12/08 v2.7;
% \CTAN{macros/latex/contrib/oberdiek/zref.pdf}
%
% \end{thebibliography}
%
% \begin{History}
%   \begin{Version}{2009/12/08 v1.0}
%   \item
%     The first version.
%   \end{Version}
%   \begin{Version}{2009/12/12 v1.1}
%   \item
%     Short info shortened.
%   \end{Version}
%   \begin{Version}{2009/12/13 v1.2}
%   \item
%     Forgotten third argument for \cs{InputIfFileExists} added.
%   \end{Version}
%   \begin{Version}{2009/12/18 v1.3}
%   \item
%     \cs{Hy@SectionAnchorHref} added for filtering
%     (hyperref 2009/12/18 v6.79w).
%   \end{Version}
%   \begin{Version}{2010/12/03 v1.4}
%   \item
%     Support of package \xpackage{enumitem}: removing
%     \cs{enit@format} from title string (problem report by GL).
%   \end{Version}
%   \begin{Version}{2016/05/16 v1.5}
%   \item
%     Documentation updates.
%   \end{Version}
% \end{History}
%
% \PrintIndex
%
% \Finale
\endinput
|
% \end{quote}
% Do not forget to quote the argument according to the demands
% of your shell.
%
% \paragraph{Generating the documentation.}
% You can use both the \xfile{.dtx} or the \xfile{.drv} to generate
% the documentation. The process can be configured by the
% configuration file \xfile{ltxdoc.cfg}. For instance, put this
% line into this file, if you want to have A4 as paper format:
% \begin{quote}
%   \verb|\PassOptionsToClass{a4paper}{article}|
% \end{quote}
% An example follows how to generate the
% documentation with pdf\LaTeX:
% \begin{quote}
%\begin{verbatim}
%pdflatex gettitlestring.dtx
%makeindex -s gind.ist gettitlestring.idx
%pdflatex gettitlestring.dtx
%makeindex -s gind.ist gettitlestring.idx
%pdflatex gettitlestring.dtx
%\end{verbatim}
% \end{quote}
%
% \begin{thebibliography}{9}
%
% \bibitem{memoir}
% Peter Wilson, Lars Madsen:
% \textit{The Memoir Class};
% 2009/11/17 v1.61803398c;
% \CTANpkg{memoir}
%
% \bibitem{titleref}
% Donald Arsenau:
% \textit{Titleref.sty};
% 2001/04/05 ver 3.1;
% \CTAN{macros/latex/contrib/misc/titleref.sty}
%
% \bibitem{zref}
% Heiko Oberdiek:
% \textit{The \xpackage{zref} package};
% 2009/12/08 v2.7;
% \CTAN{macros/latex/contrib/oberdiek/zref.pdf}
%
% \end{thebibliography}
%
% \begin{History}
%   \begin{Version}{2009/12/08 v1.0}
%   \item
%     The first version.
%   \end{Version}
%   \begin{Version}{2009/12/12 v1.1}
%   \item
%     Short info shortened.
%   \end{Version}
%   \begin{Version}{2009/12/13 v1.2}
%   \item
%     Forgotten third argument for \cs{InputIfFileExists} added.
%   \end{Version}
%   \begin{Version}{2009/12/18 v1.3}
%   \item
%     \cs{Hy@SectionAnchorHref} added for filtering
%     (hyperref 2009/12/18 v6.79w).
%   \end{Version}
%   \begin{Version}{2010/12/03 v1.4}
%   \item
%     Support of package \xpackage{enumitem}: removing
%     \cs{enit@format} from title string (problem report by GL).
%   \end{Version}
%   \begin{Version}{2016/05/16 v1.5}
%   \item
%     Documentation updates.
%   \end{Version}
% \end{History}
%
% \PrintIndex
%
% \Finale
\endinput

%        (quote the arguments according to the demands of your shell)
%
% Documentation:
%    (a) If gettitlestring.drv is present:
%           latex gettitlestring.drv
%    (b) Without gettitlestring.drv:
%           latex gettitlestring.dtx; ...
%    The class ltxdoc loads the configuration file ltxdoc.cfg
%    if available. Here you can specify further options, e.g.
%    use A4 as paper format:
%       \PassOptionsToClass{a4paper}{article}
%
%    Programm calls to get the documentation (example):
%       pdflatex gettitlestring.dtx
%       makeindex -s gind.ist gettitlestring.idx
%       pdflatex gettitlestring.dtx
%       makeindex -s gind.ist gettitlestring.idx
%       pdflatex gettitlestring.dtx
%
% Installation:
%    TDS:tex/generic/oberdiek/gettitlestring.sty
%    TDS:doc/latex/oberdiek/gettitlestring.pdf
%    TDS:doc/latex/oberdiek/test/gettitlestring-test1.tex
%    TDS:doc/latex/oberdiek/test/gettitlestring-test2.tex
%    TDS:source/latex/oberdiek/gettitlestring.dtx
%
%<*ignore>
\begingroup
  \catcode123=1 %
  \catcode125=2 %
  \def\x{LaTeX2e}%
\expandafter\endgroup
\ifcase 0\ifx\install y1\fi\expandafter
         \ifx\csname processbatchFile\endcsname\relax\else1\fi
         \ifx\fmtname\x\else 1\fi\relax
\else\csname fi\endcsname
%</ignore>
%<*install>
\input docstrip.tex
\Msg{************************************************************************}
\Msg{* Installation}
\Msg{* Package: gettitlestring 2016/05/16 v1.5 Cleanup title references (HO)}
\Msg{************************************************************************}

\keepsilent
\askforoverwritefalse

\let\MetaPrefix\relax
\preamble

This is a generated file.

Project: gettitlestring
Version: 2016/05/16 v1.5

Copyright (C) 2009, 2010 by
   Heiko Oberdiek <heiko.oberdiek at googlemail.com>

This work may be distributed and/or modified under the
conditions of the LaTeX Project Public License, either
version 1.3c of this license or (at your option) any later
version. This version of this license is in
   http://www.latex-project.org/lppl/lppl-1-3c.txt
and the latest version of this license is in
   http://www.latex-project.org/lppl.txt
and version 1.3 or later is part of all distributions of
LaTeX version 2005/12/01 or later.

This work has the LPPL maintenance status "maintained".

This Current Maintainer of this work is Heiko Oberdiek.

The Base Interpreter refers to any `TeX-Format',
because some files are installed in TDS:tex/generic//.

This work consists of the main source file gettitlestring.dtx
and the derived files
   gettitlestring.sty, gettitlestring.pdf, gettitlestring.ins,
   gettitlestring.drv, gettitlestring-test1.tex,
   gettitlestring-test2.tex.

\endpreamble
\let\MetaPrefix\DoubleperCent

\generate{%
  \file{gettitlestring.ins}{\from{gettitlestring.dtx}{install}}%
  \file{gettitlestring.drv}{\from{gettitlestring.dtx}{driver}}%
  \usedir{tex/generic/oberdiek}%
  \file{gettitlestring.sty}{\from{gettitlestring.dtx}{package}}%
%  \usedir{doc/latex/oberdiek/test}%
%  \file{gettitlestring-test1.tex}{\from{gettitlestring.dtx}{test1}}%
%  \file{gettitlestring-test2.tex}{\from{gettitlestring.dtx}{test2}}%
  \nopreamble
  \nopostamble
%  \usedir{source/latex/oberdiek/catalogue}%
%  \file{gettitlestring.xml}{\from{gettitlestring.dtx}{catalogue}}%
}

\catcode32=13\relax% active space
\let =\space%
\Msg{************************************************************************}
\Msg{*}
\Msg{* To finish the installation you have to move the following}
\Msg{* file into a directory searched by TeX:}
\Msg{*}
\Msg{*     gettitlestring.sty}
\Msg{*}
\Msg{* To produce the documentation run the file `gettitlestring.drv'}
\Msg{* through LaTeX.}
\Msg{*}
\Msg{* Happy TeXing!}
\Msg{*}
\Msg{************************************************************************}

\endbatchfile
%</install>
%<*ignore>
\fi
%</ignore>
%<*driver>
\NeedsTeXFormat{LaTeX2e}
\ProvidesFile{gettitlestring.drv}%
  [2016/05/16 v1.5 Cleanup title references (HO)]%
\documentclass{ltxdoc}
\usepackage{holtxdoc}[2011/11/22]
\begin{document}
  \DocInput{gettitlestring.dtx}%
\end{document}
%</driver>
% \fi
%
%
% \CharacterTable
%  {Upper-case    \A\B\C\D\E\F\G\H\I\J\K\L\M\N\O\P\Q\R\S\T\U\V\W\X\Y\Z
%   Lower-case    \a\b\c\d\e\f\g\h\i\j\k\l\m\n\o\p\q\r\s\t\u\v\w\x\y\z
%   Digits        \0\1\2\3\4\5\6\7\8\9
%   Exclamation   \!     Double quote  \"     Hash (number) \#
%   Dollar        \$     Percent       \%     Ampersand     \&
%   Acute accent  \'     Left paren    \(     Right paren   \)
%   Asterisk      \*     Plus          \+     Comma         \,
%   Minus         \-     Point         \.     Solidus       \/
%   Colon         \:     Semicolon     \;     Less than     \<
%   Equals        \=     Greater than  \>     Question mark \?
%   Commercial at \@     Left bracket  \[     Backslash     \\
%   Right bracket \]     Circumflex    \^     Underscore    \_
%   Grave accent  \`     Left brace    \{     Vertical bar  \|
%   Right brace   \}     Tilde         \~}
%
% \GetFileInfo{gettitlestring.drv}
%
% \title{The \xpackage{gettitlestring} package}
% \date{2016/05/16 v1.5}
% \author{Heiko Oberdiek\thanks
% {Please report any issues at \url{https://github.com/ho-tex/oberdiek/issues}}\\
% \xemail{heiko.oberdiek at googlemail.com}}
%
% \maketitle
%
% \begin{abstract}
% The \LaTeX\ package addresses packages that are dealing with
% references to titles (\cs{section}, \cs{caption}, \dots).
% The package tries to remove \cs{label} and other
% commands from title strings.
% \end{abstract}
%
% \tableofcontents
%
% \section{Documentation}
%
% \subsection{Macros}
%
% \begin{declcs}{GetTitleStringSetup} \M{key value list}
% \end{declcs}
% The options are given as comma separated key value pairs.
% See section \ref{sec:options}.
%
% \begin{declcs}{GetTitleString} \M{text}\\
% \cs{GetTitleStringExpand} \M{text}\\
% \cs{GetTitleStringNonExpand} \M{text}
% \end{declcs}
% Macro \cs{GetTitleString} tries to remove unwanted stuff from \meta{text}
% the result is stored in Macro \cs{GetTitleStringResult}.
% Two methods are available:
% \begin{description}
% \item[\cs{GetTitleStringExpand}:]
% The \meta{text} is expanded in a context where the unwanted
% macros are redefined to remove themselves.
% This is the method used in packages \xpackage{titleref}~\cite{titleref},
% \xpackage{zref-titleref}~\cite{zref}
% or class \xclass{memoir}~\cite{memoir}.
% \cs{protect} is supported, but fragile material might break.
% \item[\cs{GetTitleStringNonExpand}:]
% The \meta{text} is not expanded. Thus the removal of unwanted
% material is more difficult. It is especially removed at the
% start of the \meta{text} and spaces are removed from the end.
% Currently only \cs{label} is removed in the whole string,
% if it is not hidden inside curly braces or part of macro
% definitions. Thus the removal of unwanted stuff might not be
% complete, but fragile material will not break.
% (But the result string can break at a later time, of course).
% \end{description}
% Option \xoption{expand} controls which method is used by
% macro \cs{GetTitleString}.
%
% \begin{declcs}{GetTitleStringDisableCommands} \M{code}
% \end{declcs}
% The \meta{code} is called right before the
% text is expanded in \cs{GetTitleStringExpand}.
% Additional definitions can be given for macros that
% should be removed.
% Keep in mind that expansion means that the definitions
% must work in expandable context. Macros like
% \cs{@ifstar} or \cs{@ifnextchar} or optional arguments
% will not work. The macro names in \meta{code} may contain
% the at sign |@|, it has catcode 11 (letter).
%
% \subsection{Options}\label{sec:options}
%
% \begin{description}
% \item[\xoption{expand}:] Boolean option, takes values |true| or |false|.
% No value means |true|. The option specifies the method to remove
% unwanted stuff from the title string, see below.
% \end{description}
% Options can be set at the following places:
% \begin{itemize}
% \item \cs{usepackage}
% \item Configuration file \xfile{gettitlestring.cfg}.
% \item \cs{GetTitleStringSetup}
% \end{itemize}
%
% \StopEventually{
% }
%
% \section{Implementation}
%    \begin{macrocode}
%<*package>
%    \end{macrocode}
%    Reload check, especially if the package is not used with \LaTeX.
%    \begin{macrocode}
\begingroup\catcode61\catcode48\catcode32=10\relax%
  \catcode13=5 % ^^M
  \endlinechar=13 %
  \catcode35=6 % #
  \catcode39=12 % '
  \catcode44=12 % ,
  \catcode45=12 % -
  \catcode46=12 % .
  \catcode58=12 % :
  \catcode64=11 % @
  \catcode123=1 % {
  \catcode125=2 % }
  \expandafter\let\expandafter\x\csname ver@gettitlestring.sty\endcsname
  \ifx\x\relax % plain-TeX, first loading
  \else
    \def\empty{}%
    \ifx\x\empty % LaTeX, first loading,
      % variable is initialized, but \ProvidesPackage not yet seen
    \else
      \expandafter\ifx\csname PackageInfo\endcsname\relax
        \def\x#1#2{%
          \immediate\write-1{Package #1 Info: #2.}%
        }%
      \else
        \def\x#1#2{\PackageInfo{#1}{#2, stopped}}%
      \fi
      \x{gettitlestring}{The package is already loaded}%
      \aftergroup\endinput
    \fi
  \fi
\endgroup%
%    \end{macrocode}
%    Package identification:
%    \begin{macrocode}
\begingroup\catcode61\catcode48\catcode32=10\relax%
  \catcode13=5 % ^^M
  \endlinechar=13 %
  \catcode35=6 % #
  \catcode39=12 % '
  \catcode40=12 % (
  \catcode41=12 % )
  \catcode44=12 % ,
  \catcode45=12 % -
  \catcode46=12 % .
  \catcode47=12 % /
  \catcode58=12 % :
  \catcode64=11 % @
  \catcode91=12 % [
  \catcode93=12 % ]
  \catcode123=1 % {
  \catcode125=2 % }
  \expandafter\ifx\csname ProvidesPackage\endcsname\relax
    \def\x#1#2#3[#4]{\endgroup
      \immediate\write-1{Package: #3 #4}%
      \xdef#1{#4}%
    }%
  \else
    \def\x#1#2[#3]{\endgroup
      #2[{#3}]%
      \ifx#1\@undefined
        \xdef#1{#3}%
      \fi
      \ifx#1\relax
        \xdef#1{#3}%
      \fi
    }%
  \fi
\expandafter\x\csname ver@gettitlestring.sty\endcsname
\ProvidesPackage{gettitlestring}%
  [2016/05/16 v1.5 Cleanup title references (HO)]%
%    \end{macrocode}
%
%    \begin{macrocode}
\begingroup\catcode61\catcode48\catcode32=10\relax%
  \catcode13=5 % ^^M
  \endlinechar=13 %
  \catcode123=1 % {
  \catcode125=2 % }
  \catcode64=11 % @
  \def\x{\endgroup
    \expandafter\edef\csname GTS@AtEnd\endcsname{%
      \endlinechar=\the\endlinechar\relax
      \catcode13=\the\catcode13\relax
      \catcode32=\the\catcode32\relax
      \catcode35=\the\catcode35\relax
      \catcode61=\the\catcode61\relax
      \catcode64=\the\catcode64\relax
      \catcode123=\the\catcode123\relax
      \catcode125=\the\catcode125\relax
    }%
  }%
\x\catcode61\catcode48\catcode32=10\relax%
\catcode13=5 % ^^M
\endlinechar=13 %
\catcode35=6 % #
\catcode64=11 % @
\catcode123=1 % {
\catcode125=2 % }
\def\TMP@EnsureCode#1#2{%
  \edef\GTS@AtEnd{%
    \GTS@AtEnd
    \catcode#1=\the\catcode#1\relax
  }%
  \catcode#1=#2\relax
}
\TMP@EnsureCode{42}{12}% *
\TMP@EnsureCode{44}{12}% ,
\TMP@EnsureCode{45}{12}% -
\TMP@EnsureCode{46}{12}% .
\TMP@EnsureCode{47}{12}% /
\TMP@EnsureCode{91}{12}% [
\TMP@EnsureCode{93}{12}% ]
\edef\GTS@AtEnd{\GTS@AtEnd\noexpand\endinput}
%    \end{macrocode}
%
% \subsection{Options}
%
%    \begin{macrocode}
\RequirePackage{kvoptions}[2009/07/17]
\SetupKeyvalOptions{%
  family=gettitlestring,%
  prefix=GTS@%
}
\newcommand*{\GetTitleStringSetup}{%
  \setkeys{gettitlestring}%
}
\DeclareBoolOption{expand}
\InputIfFileExists{gettitlestring.cfg}{}{}
\ProcessKeyvalOptions*\relax
%    \end{macrocode}
%
% \subsection{\cs{GetTitleString}}
%
%    \begin{macro}{\GetTitleString}
%    \begin{macrocode}
\newcommand*{\GetTitleString}{%
  \ifGTS@expand
    \expandafter\GetTitleStringExpand
  \else
    \expandafter\GetTitleStringNonExpand
  \fi
}
%    \end{macrocode}
%    \end{macro}
%    \begin{macro}{\GetTitleStringExpand}
%    \begin{macrocode}
\newcommand{\GetTitleStringExpand}[1]{%
  \def\GetTitleStringResult{#1}%
  \begingroup
    \GTS@DisablePredefinedCmds
    \GTS@DisableHook
    \edef\x{\endgroup
      \noexpand\def\noexpand\GetTitleStringResult{%
        \GetTitleStringResult
      }%
    }%
  \x
}
%    \end{macrocode}
%    \end{macro}
%    \begin{macro}{\GetTitleString}
%    \begin{macrocode}
\newcommand{\GetTitleStringNonExpand}[1]{%
  \def\GetTitleStringResult{#1}%
  \global\let\GTS@GlobalString\GetTitleStringResult
  \begingroup
    \GTS@RemoveLeft
    \GTS@RemoveRight
  \endgroup
  \let\GetTitleStringResult\GTS@GlobalString
}
%    \end{macrocode}
%    \end{macro}
%
% \subsubsection{Expand method}
%
%    \begin{macro}{\GTS@DisablePredefinedCmds}
%    \begin{macrocode}
\def\GTS@DisablePredefinedCmds{%
  \let\label\@gobble
  \let\zlabel\@gobble
  \let\zref@label\@gobble
  \let\zref@labelbylist\@gobbletwo
  \let\zref@labelbyprops\@gobbletwo
  \let\index\@gobble
  \let\glossary\@gobble
  \let\markboth\@gobbletwo
  \let\@mkboth\@gobbletwo
  \let\markright\@gobble
  \let\phantomsection\@empty
  \def\addcontentsline{\expandafter\@gobble\@gobbletwo}%
  \let\raggedright\@empty
  \let\raggedleft\@empty
  \let\centering\@empty
  \let\protect\@unexpandable@protect
  \let\enit@format\@empty % package enumitem
}
%    \end{macrocode}
%    \end{macro}
%
%    \begin{macro}{\GTS@DisableHook}
%    \begin{macrocode}
\providecommand*{\GTS@DisableHook}{}
%    \end{macrocode}
%    \end{macro}
%    \begin{macro}{\GetTitleStringDisableCommands}
%    \begin{macrocode}
\def\GetTitleStringDisableCommands{%
  \begingroup
    \makeatletter
    \GTS@DisableCommands
}
%    \end{macrocode}
%    \end{macro}
%    \begin{macro}{\GTS@DisableCommands}
%    \begin{macrocode}
\long\def\GTS@DisableCommands#1{%
    \toks0=\expandafter{\GTS@DisableHook}%
    \toks2={#1}%
    \xdef\GTS@GlobalString{\the\toks0 \the\toks2}%
  \endgroup
  \let\GTS@DisableHook\GTS@GlobalString
}
%    \end{macrocode}
%    \end{macro}
%
% \subsubsection{Non-expand method}
%
%    \begin{macrocode}
\def\GTS@RemoveLeft{%
  \toks@\expandafter\expandafter\expandafter{%
    \expandafter\GTS@Car\GTS@GlobalString{}{}{}{}\GTS@Nil
  }%
  \edef\GTS@Token{\the\toks@}%
  \GTS@PredefinedLeftCmds
  \expandafter\futurelet\expandafter\GTS@Token
  \expandafter\GTS@TestLeftSpace\GTS@GlobalString\GTS@Nil
  \GTS@End
}
\def\GTS@End{}
\long\def\GTS@TestLeft#1#2{%
  \def\GTS@temp{#1}%
  \ifx\GTS@temp\GTS@Token
    \toks@\expandafter\expandafter\expandafter{%
      \expandafter#2\GTS@GlobalString\GTS@Nil
    }%
    \expandafter\GTS@TestLeftEnd
  \fi
}
\long\def\GTS@TestLeftEnd#1\GTS@End{%
  \xdef\GTS@GlobalString{\the\toks@}%
  \GTS@RemoveLeft
}
\long\def\GTS@Car#1#2\GTS@Nil{#1}
\long\def\GTS@Cdr#1#2\GTS@Nil{#2}
\long\def\GTS@CdrTwo#1#2#3\GTS@Nil{#3}
\long\def\GTS@CdrThree#1#2#3#4\GTS@Nil{#4}
\long\def\GTS@CdrFour#1#2#3#4#5\GTS@Nil{#5}
\long\def\GTS@TestLeftSpace#1\GTS@Nil{%
  \ifx\GTS@Token\@sptoken
    \toks@\expandafter{%
      \romannumeral-0\GTS@GlobalString
    }%
    \expandafter\GTS@TestLeftEnd
  \fi
}
%    \end{macrocode}
%    \begin{macro}{\GTS@PredefinedLeftCmds}
%    \begin{macrocode}
\def\GTS@PredefinedLeftCmds{%
  \GTS@TestLeft\Hy@phantomsection\GTS@Cdr
  \GTS@TestLeft\Hy@SectionAnchor\GTS@Cdr
  \GTS@TestLeft\Hy@SectionAnchorHref\GTS@CdrTwo
  \GTS@TestLeft\label\GTS@CdrTwo
  \GTS@TestLeft\zlabel\GTS@CdrTwo
  \GTS@TestLeft\index\GTS@CdrTwo
  \GTS@TestLeft\glossary\GTS@CdrTwo
  \GTS@TestLeft\markboth\GTS@CdrThree
  \GTS@TestLeft\@mkboth\GTS@CdrThree
  \GTS@TestLeft\addcontentsline\GTS@CdrFour
  \GTS@TestLeft\enit@format\GTS@Cdr % package enumitem
}
%    \end{macrocode}
%    \end{macro}
%
%    \begin{macrocode}
\def\GTS@RemoveRight{%
  \toks@{}%
  \expandafter\GTS@TestRightLabel\GTS@GlobalString
      \label{}\GTS@Nil\@nil
  \GTS@RemoveRightSpace
}
\begingroup
  \def\GTS@temp#1{\endgroup
    \def\GTS@RemoveRightSpace{%
      \expandafter\GTS@TestRightSpace\GTS@GlobalString
          \GTS@Nil#1\GTS@Nil\@nil
    }%
  }%
\GTS@temp{ }
\def\GTS@TestRightSpace#1 \GTS@Nil#2\@nil{%
  \ifx\relax#2\relax
  \else
    \gdef\GTS@GlobalString{#1}%
    \expandafter\GTS@RemoveRightSpace
  \fi
}
\def\GTS@TestRightLabel#1\label#2#3\GTS@Nil#4\@nil{%
  \def\GTS@temp{#3}%
  \ifx\GTS@temp\@empty
    \expandafter\gdef\expandafter\GTS@GlobalString\expandafter{%
      \the\toks@
      #1%
    }%
    \expandafter\@gobble
  \else
    \expandafter\@firstofone
  \fi
  {%
    \toks@\expandafter{\the\toks@#1}%
    \GTS@TestRightLabel#3\GTS@Nil\@nil
  }%
}
%    \end{macrocode}
%
%    \begin{macrocode}
\GTS@AtEnd%
%</package>
%    \end{macrocode}
%
% \section{Test}
%
% \subsection{Catcode checks for loading}
%
%    \begin{macrocode}
%<*test1>
%    \end{macrocode}
%    \begin{macrocode}
\catcode`\{=1 %
\catcode`\}=2 %
\catcode`\#=6 %
\catcode`\@=11 %
\expandafter\ifx\csname count@\endcsname\relax
  \countdef\count@=255 %
\fi
\expandafter\ifx\csname @gobble\endcsname\relax
  \long\def\@gobble#1{}%
\fi
\expandafter\ifx\csname @firstofone\endcsname\relax
  \long\def\@firstofone#1{#1}%
\fi
\expandafter\ifx\csname loop\endcsname\relax
  \expandafter\@firstofone
\else
  \expandafter\@gobble
\fi
{%
  \def\loop#1\repeat{%
    \def\body{#1}%
    \iterate
  }%
  \def\iterate{%
    \body
      \let\next\iterate
    \else
      \let\next\relax
    \fi
    \next
  }%
  \let\repeat=\fi
}%
\def\RestoreCatcodes{}
\count@=0 %
\loop
  \edef\RestoreCatcodes{%
    \RestoreCatcodes
    \catcode\the\count@=\the\catcode\count@\relax
  }%
\ifnum\count@<255 %
  \advance\count@ 1 %
\repeat

\def\RangeCatcodeInvalid#1#2{%
  \count@=#1\relax
  \loop
    \catcode\count@=15 %
  \ifnum\count@<#2\relax
    \advance\count@ 1 %
  \repeat
}
\def\RangeCatcodeCheck#1#2#3{%
  \count@=#1\relax
  \loop
    \ifnum#3=\catcode\count@
    \else
      \errmessage{%
        Character \the\count@\space
        with wrong catcode \the\catcode\count@\space
        instead of \number#3%
      }%
    \fi
  \ifnum\count@<#2\relax
    \advance\count@ 1 %
  \repeat
}
\def\space{ }
\expandafter\ifx\csname LoadCommand\endcsname\relax
  \def\LoadCommand{\input gettitlestring.sty\relax}%
\fi
\def\Test{%
  \RangeCatcodeInvalid{0}{47}%
  \RangeCatcodeInvalid{58}{64}%
  \RangeCatcodeInvalid{91}{96}%
  \RangeCatcodeInvalid{123}{255}%
  \catcode`\@=12 %
  \catcode`\\=0 %
  \catcode`\%=14 %
  \LoadCommand
  \RangeCatcodeCheck{0}{36}{15}%
  \RangeCatcodeCheck{37}{37}{14}%
  \RangeCatcodeCheck{38}{47}{15}%
  \RangeCatcodeCheck{48}{57}{12}%
  \RangeCatcodeCheck{58}{63}{15}%
  \RangeCatcodeCheck{64}{64}{12}%
  \RangeCatcodeCheck{65}{90}{11}%
  \RangeCatcodeCheck{91}{91}{15}%
  \RangeCatcodeCheck{92}{92}{0}%
  \RangeCatcodeCheck{93}{96}{15}%
  \RangeCatcodeCheck{97}{122}{11}%
  \RangeCatcodeCheck{123}{255}{15}%
  \RestoreCatcodes
}
\Test
\csname @@end\endcsname
\end
%    \end{macrocode}
%    \begin{macrocode}
%</test1>
%    \end{macrocode}
%
% \subsection{Test of non-expand method}
%
%    \begin{macrocode}
%<*test2>
\NeedsTeXFormat{LaTeX2e}
\documentclass{minimal}
\usepackage{gettitlestring}[2016/05/16]
\usepackage{qstest}
\IncludeTests{*}
\LogTests{log}{*}{*}
\begin{document}
\begin{qstest}{non-expand}{non-expand}
  \def\test#1#2{%
    \sbox0{%
      \GetTitleString{#1}%
      \Expect{#2}*{\GetTitleStringResult}%
    }%
    \Expect{0.0pt}*{\the\wd0}%
  }%
  \test{}{}%
  \test{ }{}%
  \test{ x }{x}%
  \test{ x y }{x y}%
  \test{ \relax}{\relax}%
  \test{\label{f}a}{a}%
  \test{ \label{f}a}{a}%
  \test{\label{f} a}{a}%
  \test{ \label{f} a}{a}%
  \test{a\label{f}}{a}%
  \test{a\label{f} }{a}%
  \test{a \label{f}}{a}%
  \test{a \label{f} }{a}%
  \test{a\label{f}b\label{g}}{ab}%
  \test{a \label{f}b \label{g} }{a b}%
  \test{a\label{f} b \label{g} }{a b}%
\end{qstest}
\end{document}
%</test2>
%    \end{macrocode}
%
% \section{Installation}
%
% \subsection{Download}
%
% \paragraph{Package.} This package is available on
% CTAN\footnote{\CTANpkg{gettitlestring}}:
% \begin{description}
% \item[\CTAN{macros/latex/contrib/oberdiek/gettitlestring.dtx}] The source file.
% \item[\CTAN{macros/latex/contrib/oberdiek/gettitlestring.pdf}] Documentation.
% \end{description}
%
%
% \paragraph{Bundle.} All the packages of the bundle `oberdiek'
% are also available in a TDS compliant ZIP archive. There
% the packages are already unpacked and the documentation files
% are generated. The files and directories obey the TDS standard.
% \begin{description}
% \item[\CTANinstall{install/macros/latex/contrib/oberdiek.tds.zip}]
% \end{description}
% \emph{TDS} refers to the standard ``A Directory Structure
% for \TeX\ Files'' (\CTAN{tds/tds.pdf}). Directories
% with \xfile{texmf} in their name are usually organized this way.
%
% \subsection{Bundle installation}
%
% \paragraph{Unpacking.} Unpack the \xfile{oberdiek.tds.zip} in the
% TDS tree (also known as \xfile{texmf} tree) of your choice.
% Example (linux):
% \begin{quote}
%   |unzip oberdiek.tds.zip -d ~/texmf|
% \end{quote}
%
% \paragraph{Script installation.}
% Check the directory \xfile{TDS:scripts/oberdiek/} for
% scripts that need further installation steps.
% Package \xpackage{attachfile2} comes with the Perl script
% \xfile{pdfatfi.pl} that should be installed in such a way
% that it can be called as \texttt{pdfatfi}.
% Example (linux):
% \begin{quote}
%   |chmod +x scripts/oberdiek/pdfatfi.pl|\\
%   |cp scripts/oberdiek/pdfatfi.pl /usr/local/bin/|
% \end{quote}
%
% \subsection{Package installation}
%
% \paragraph{Unpacking.} The \xfile{.dtx} file is a self-extracting
% \docstrip\ archive. The files are extracted by running the
% \xfile{.dtx} through \plainTeX:
% \begin{quote}
%   \verb|tex gettitlestring.dtx|
% \end{quote}
%
% \paragraph{TDS.} Now the different files must be moved into
% the different directories in your installation TDS tree
% (also known as \xfile{texmf} tree):
% \begin{quote}
% \def\t{^^A
% \begin{tabular}{@{}>{\ttfamily}l@{ $\rightarrow$ }>{\ttfamily}l@{}}
%   gettitlestring.sty & tex/generic/oberdiek/gettitlestring.sty\\
%   gettitlestring.pdf & doc/latex/oberdiek/gettitlestring.pdf\\
%   test/gettitlestring-test1.tex & doc/latex/oberdiek/test/gettitlestring-test1.tex\\
%   test/gettitlestring-test2.tex & doc/latex/oberdiek/test/gettitlestring-test2.tex\\
%   gettitlestring.dtx & source/latex/oberdiek/gettitlestring.dtx\\
% \end{tabular}^^A
% }^^A
% \sbox0{\t}^^A
% \ifdim\wd0>\linewidth
%   \begingroup
%     \advance\linewidth by\leftmargin
%     \advance\linewidth by\rightmargin
%   \edef\x{\endgroup
%     \def\noexpand\lw{\the\linewidth}^^A
%   }\x
%   \def\lwbox{^^A
%     \leavevmode
%     \hbox to \linewidth{^^A
%       \kern-\leftmargin\relax
%       \hss
%       \usebox0
%       \hss
%       \kern-\rightmargin\relax
%     }^^A
%   }^^A
%   \ifdim\wd0>\lw
%     \sbox0{\small\t}^^A
%     \ifdim\wd0>\linewidth
%       \ifdim\wd0>\lw
%         \sbox0{\footnotesize\t}^^A
%         \ifdim\wd0>\linewidth
%           \ifdim\wd0>\lw
%             \sbox0{\scriptsize\t}^^A
%             \ifdim\wd0>\linewidth
%               \ifdim\wd0>\lw
%                 \sbox0{\tiny\t}^^A
%                 \ifdim\wd0>\linewidth
%                   \lwbox
%                 \else
%                   \usebox0
%                 \fi
%               \else
%                 \lwbox
%               \fi
%             \else
%               \usebox0
%             \fi
%           \else
%             \lwbox
%           \fi
%         \else
%           \usebox0
%         \fi
%       \else
%         \lwbox
%       \fi
%     \else
%       \usebox0
%     \fi
%   \else
%     \lwbox
%   \fi
% \else
%   \usebox0
% \fi
% \end{quote}
% If you have a \xfile{docstrip.cfg} that configures and enables \docstrip's
% TDS installing feature, then some files can already be in the right
% place, see the documentation of \docstrip.
%
% \subsection{Refresh file name databases}
%
% If your \TeX~distribution
% (\teTeX, \mikTeX, \dots) relies on file name databases, you must refresh
% these. For example, \teTeX\ users run \verb|texhash| or
% \verb|mktexlsr|.
%
% \subsection{Some details for the interested}
%
% \paragraph{Attached source.}
%
% The PDF documentation on CTAN also includes the
% \xfile{.dtx} source file. It can be extracted by
% AcrobatReader 6 or higher. Another option is \textsf{pdftk},
% e.g. unpack the file into the current directory:
% \begin{quote}
%   \verb|pdftk gettitlestring.pdf unpack_files output .|
% \end{quote}
%
% \paragraph{Unpacking with \LaTeX.}
% The \xfile{.dtx} chooses its action depending on the format:
% \begin{description}
% \item[\plainTeX:] Run \docstrip\ and extract the files.
% \item[\LaTeX:] Generate the documentation.
% \end{description}
% If you insist on using \LaTeX\ for \docstrip\ (really,
% \docstrip\ does not need \LaTeX), then inform the autodetect routine
% about your intention:
% \begin{quote}
%   \verb|latex \let\install=y% \iffalse meta-comment
%
% File: gettitlestring.dtx
% Version: 2016/05/16 v1.5
% Info: Cleanup title references
%
% Copyright (C) 2009, 2010 by
%    Heiko Oberdiek <heiko.oberdiek at googlemail.com>
%    2016
%    https://github.com/ho-tex/oberdiek/issues
%
% This work may be distributed and/or modified under the
% conditions of the LaTeX Project Public License, either
% version 1.3c of this license or (at your option) any later
% version. This version of this license is in
%    http://www.latex-project.org/lppl/lppl-1-3c.txt
% and the latest version of this license is in
%    http://www.latex-project.org/lppl.txt
% and version 1.3 or later is part of all distributions of
% LaTeX version 2005/12/01 or later.
%
% This work has the LPPL maintenance status "maintained".
%
% This Current Maintainer of this work is Heiko Oberdiek.
%
% The Base Interpreter refers to any `TeX-Format',
% because some files are installed in TDS:tex/generic//.
%
% This work consists of the main source file gettitlestring.dtx
% and the derived files
%    gettitlestring.sty, gettitlestring.pdf, gettitlestring.ins,
%    gettitlestring.drv, gettitlestring-test1.tex,
%    gettitlestring-test2.tex.
%
% Distribution:
%    CTAN:macros/latex/contrib/oberdiek/gettitlestring.dtx
%    CTAN:macros/latex/contrib/oberdiek/gettitlestring.pdf
%
% Unpacking:
%    (a) If gettitlestring.ins is present:
%           tex gettitlestring.ins
%    (b) Without gettitlestring.ins:
%           tex gettitlestring.dtx
%    (c) If you insist on using LaTeX
%           latex \let\install=y% \iffalse meta-comment
%
% File: gettitlestring.dtx
% Version: 2016/05/16 v1.5
% Info: Cleanup title references
%
% Copyright (C) 2009, 2010 by
%    Heiko Oberdiek <heiko.oberdiek at googlemail.com>
%    2016
%    https://github.com/ho-tex/oberdiek/issues
%
% This work may be distributed and/or modified under the
% conditions of the LaTeX Project Public License, either
% version 1.3c of this license or (at your option) any later
% version. This version of this license is in
%    http://www.latex-project.org/lppl/lppl-1-3c.txt
% and the latest version of this license is in
%    http://www.latex-project.org/lppl.txt
% and version 1.3 or later is part of all distributions of
% LaTeX version 2005/12/01 or later.
%
% This work has the LPPL maintenance status "maintained".
%
% This Current Maintainer of this work is Heiko Oberdiek.
%
% The Base Interpreter refers to any `TeX-Format',
% because some files are installed in TDS:tex/generic//.
%
% This work consists of the main source file gettitlestring.dtx
% and the derived files
%    gettitlestring.sty, gettitlestring.pdf, gettitlestring.ins,
%    gettitlestring.drv, gettitlestring-test1.tex,
%    gettitlestring-test2.tex.
%
% Distribution:
%    CTAN:macros/latex/contrib/oberdiek/gettitlestring.dtx
%    CTAN:macros/latex/contrib/oberdiek/gettitlestring.pdf
%
% Unpacking:
%    (a) If gettitlestring.ins is present:
%           tex gettitlestring.ins
%    (b) Without gettitlestring.ins:
%           tex gettitlestring.dtx
%    (c) If you insist on using LaTeX
%           latex \let\install=y\input{gettitlestring.dtx}
%        (quote the arguments according to the demands of your shell)
%
% Documentation:
%    (a) If gettitlestring.drv is present:
%           latex gettitlestring.drv
%    (b) Without gettitlestring.drv:
%           latex gettitlestring.dtx; ...
%    The class ltxdoc loads the configuration file ltxdoc.cfg
%    if available. Here you can specify further options, e.g.
%    use A4 as paper format:
%       \PassOptionsToClass{a4paper}{article}
%
%    Programm calls to get the documentation (example):
%       pdflatex gettitlestring.dtx
%       makeindex -s gind.ist gettitlestring.idx
%       pdflatex gettitlestring.dtx
%       makeindex -s gind.ist gettitlestring.idx
%       pdflatex gettitlestring.dtx
%
% Installation:
%    TDS:tex/generic/oberdiek/gettitlestring.sty
%    TDS:doc/latex/oberdiek/gettitlestring.pdf
%    TDS:doc/latex/oberdiek/test/gettitlestring-test1.tex
%    TDS:doc/latex/oberdiek/test/gettitlestring-test2.tex
%    TDS:source/latex/oberdiek/gettitlestring.dtx
%
%<*ignore>
\begingroup
  \catcode123=1 %
  \catcode125=2 %
  \def\x{LaTeX2e}%
\expandafter\endgroup
\ifcase 0\ifx\install y1\fi\expandafter
         \ifx\csname processbatchFile\endcsname\relax\else1\fi
         \ifx\fmtname\x\else 1\fi\relax
\else\csname fi\endcsname
%</ignore>
%<*install>
\input docstrip.tex
\Msg{************************************************************************}
\Msg{* Installation}
\Msg{* Package: gettitlestring 2016/05/16 v1.5 Cleanup title references (HO)}
\Msg{************************************************************************}

\keepsilent
\askforoverwritefalse

\let\MetaPrefix\relax
\preamble

This is a generated file.

Project: gettitlestring
Version: 2016/05/16 v1.5

Copyright (C) 2009, 2010 by
   Heiko Oberdiek <heiko.oberdiek at googlemail.com>

This work may be distributed and/or modified under the
conditions of the LaTeX Project Public License, either
version 1.3c of this license or (at your option) any later
version. This version of this license is in
   http://www.latex-project.org/lppl/lppl-1-3c.txt
and the latest version of this license is in
   http://www.latex-project.org/lppl.txt
and version 1.3 or later is part of all distributions of
LaTeX version 2005/12/01 or later.

This work has the LPPL maintenance status "maintained".

This Current Maintainer of this work is Heiko Oberdiek.

The Base Interpreter refers to any `TeX-Format',
because some files are installed in TDS:tex/generic//.

This work consists of the main source file gettitlestring.dtx
and the derived files
   gettitlestring.sty, gettitlestring.pdf, gettitlestring.ins,
   gettitlestring.drv, gettitlestring-test1.tex,
   gettitlestring-test2.tex.

\endpreamble
\let\MetaPrefix\DoubleperCent

\generate{%
  \file{gettitlestring.ins}{\from{gettitlestring.dtx}{install}}%
  \file{gettitlestring.drv}{\from{gettitlestring.dtx}{driver}}%
  \usedir{tex/generic/oberdiek}%
  \file{gettitlestring.sty}{\from{gettitlestring.dtx}{package}}%
%  \usedir{doc/latex/oberdiek/test}%
%  \file{gettitlestring-test1.tex}{\from{gettitlestring.dtx}{test1}}%
%  \file{gettitlestring-test2.tex}{\from{gettitlestring.dtx}{test2}}%
  \nopreamble
  \nopostamble
%  \usedir{source/latex/oberdiek/catalogue}%
%  \file{gettitlestring.xml}{\from{gettitlestring.dtx}{catalogue}}%
}

\catcode32=13\relax% active space
\let =\space%
\Msg{************************************************************************}
\Msg{*}
\Msg{* To finish the installation you have to move the following}
\Msg{* file into a directory searched by TeX:}
\Msg{*}
\Msg{*     gettitlestring.sty}
\Msg{*}
\Msg{* To produce the documentation run the file `gettitlestring.drv'}
\Msg{* through LaTeX.}
\Msg{*}
\Msg{* Happy TeXing!}
\Msg{*}
\Msg{************************************************************************}

\endbatchfile
%</install>
%<*ignore>
\fi
%</ignore>
%<*driver>
\NeedsTeXFormat{LaTeX2e}
\ProvidesFile{gettitlestring.drv}%
  [2016/05/16 v1.5 Cleanup title references (HO)]%
\documentclass{ltxdoc}
\usepackage{holtxdoc}[2011/11/22]
\begin{document}
  \DocInput{gettitlestring.dtx}%
\end{document}
%</driver>
% \fi
%
%
% \CharacterTable
%  {Upper-case    \A\B\C\D\E\F\G\H\I\J\K\L\M\N\O\P\Q\R\S\T\U\V\W\X\Y\Z
%   Lower-case    \a\b\c\d\e\f\g\h\i\j\k\l\m\n\o\p\q\r\s\t\u\v\w\x\y\z
%   Digits        \0\1\2\3\4\5\6\7\8\9
%   Exclamation   \!     Double quote  \"     Hash (number) \#
%   Dollar        \$     Percent       \%     Ampersand     \&
%   Acute accent  \'     Left paren    \(     Right paren   \)
%   Asterisk      \*     Plus          \+     Comma         \,
%   Minus         \-     Point         \.     Solidus       \/
%   Colon         \:     Semicolon     \;     Less than     \<
%   Equals        \=     Greater than  \>     Question mark \?
%   Commercial at \@     Left bracket  \[     Backslash     \\
%   Right bracket \]     Circumflex    \^     Underscore    \_
%   Grave accent  \`     Left brace    \{     Vertical bar  \|
%   Right brace   \}     Tilde         \~}
%
% \GetFileInfo{gettitlestring.drv}
%
% \title{The \xpackage{gettitlestring} package}
% \date{2016/05/16 v1.5}
% \author{Heiko Oberdiek\thanks
% {Please report any issues at \url{https://github.com/ho-tex/oberdiek/issues}}\\
% \xemail{heiko.oberdiek at googlemail.com}}
%
% \maketitle
%
% \begin{abstract}
% The \LaTeX\ package addresses packages that are dealing with
% references to titles (\cs{section}, \cs{caption}, \dots).
% The package tries to remove \cs{label} and other
% commands from title strings.
% \end{abstract}
%
% \tableofcontents
%
% \section{Documentation}
%
% \subsection{Macros}
%
% \begin{declcs}{GetTitleStringSetup} \M{key value list}
% \end{declcs}
% The options are given as comma separated key value pairs.
% See section \ref{sec:options}.
%
% \begin{declcs}{GetTitleString} \M{text}\\
% \cs{GetTitleStringExpand} \M{text}\\
% \cs{GetTitleStringNonExpand} \M{text}
% \end{declcs}
% Macro \cs{GetTitleString} tries to remove unwanted stuff from \meta{text}
% the result is stored in Macro \cs{GetTitleStringResult}.
% Two methods are available:
% \begin{description}
% \item[\cs{GetTitleStringExpand}:]
% The \meta{text} is expanded in a context where the unwanted
% macros are redefined to remove themselves.
% This is the method used in packages \xpackage{titleref}~\cite{titleref},
% \xpackage{zref-titleref}~\cite{zref}
% or class \xclass{memoir}~\cite{memoir}.
% \cs{protect} is supported, but fragile material might break.
% \item[\cs{GetTitleStringNonExpand}:]
% The \meta{text} is not expanded. Thus the removal of unwanted
% material is more difficult. It is especially removed at the
% start of the \meta{text} and spaces are removed from the end.
% Currently only \cs{label} is removed in the whole string,
% if it is not hidden inside curly braces or part of macro
% definitions. Thus the removal of unwanted stuff might not be
% complete, but fragile material will not break.
% (But the result string can break at a later time, of course).
% \end{description}
% Option \xoption{expand} controls which method is used by
% macro \cs{GetTitleString}.
%
% \begin{declcs}{GetTitleStringDisableCommands} \M{code}
% \end{declcs}
% The \meta{code} is called right before the
% text is expanded in \cs{GetTitleStringExpand}.
% Additional definitions can be given for macros that
% should be removed.
% Keep in mind that expansion means that the definitions
% must work in expandable context. Macros like
% \cs{@ifstar} or \cs{@ifnextchar} or optional arguments
% will not work. The macro names in \meta{code} may contain
% the at sign |@|, it has catcode 11 (letter).
%
% \subsection{Options}\label{sec:options}
%
% \begin{description}
% \item[\xoption{expand}:] Boolean option, takes values |true| or |false|.
% No value means |true|. The option specifies the method to remove
% unwanted stuff from the title string, see below.
% \end{description}
% Options can be set at the following places:
% \begin{itemize}
% \item \cs{usepackage}
% \item Configuration file \xfile{gettitlestring.cfg}.
% \item \cs{GetTitleStringSetup}
% \end{itemize}
%
% \StopEventually{
% }
%
% \section{Implementation}
%    \begin{macrocode}
%<*package>
%    \end{macrocode}
%    Reload check, especially if the package is not used with \LaTeX.
%    \begin{macrocode}
\begingroup\catcode61\catcode48\catcode32=10\relax%
  \catcode13=5 % ^^M
  \endlinechar=13 %
  \catcode35=6 % #
  \catcode39=12 % '
  \catcode44=12 % ,
  \catcode45=12 % -
  \catcode46=12 % .
  \catcode58=12 % :
  \catcode64=11 % @
  \catcode123=1 % {
  \catcode125=2 % }
  \expandafter\let\expandafter\x\csname ver@gettitlestring.sty\endcsname
  \ifx\x\relax % plain-TeX, first loading
  \else
    \def\empty{}%
    \ifx\x\empty % LaTeX, first loading,
      % variable is initialized, but \ProvidesPackage not yet seen
    \else
      \expandafter\ifx\csname PackageInfo\endcsname\relax
        \def\x#1#2{%
          \immediate\write-1{Package #1 Info: #2.}%
        }%
      \else
        \def\x#1#2{\PackageInfo{#1}{#2, stopped}}%
      \fi
      \x{gettitlestring}{The package is already loaded}%
      \aftergroup\endinput
    \fi
  \fi
\endgroup%
%    \end{macrocode}
%    Package identification:
%    \begin{macrocode}
\begingroup\catcode61\catcode48\catcode32=10\relax%
  \catcode13=5 % ^^M
  \endlinechar=13 %
  \catcode35=6 % #
  \catcode39=12 % '
  \catcode40=12 % (
  \catcode41=12 % )
  \catcode44=12 % ,
  \catcode45=12 % -
  \catcode46=12 % .
  \catcode47=12 % /
  \catcode58=12 % :
  \catcode64=11 % @
  \catcode91=12 % [
  \catcode93=12 % ]
  \catcode123=1 % {
  \catcode125=2 % }
  \expandafter\ifx\csname ProvidesPackage\endcsname\relax
    \def\x#1#2#3[#4]{\endgroup
      \immediate\write-1{Package: #3 #4}%
      \xdef#1{#4}%
    }%
  \else
    \def\x#1#2[#3]{\endgroup
      #2[{#3}]%
      \ifx#1\@undefined
        \xdef#1{#3}%
      \fi
      \ifx#1\relax
        \xdef#1{#3}%
      \fi
    }%
  \fi
\expandafter\x\csname ver@gettitlestring.sty\endcsname
\ProvidesPackage{gettitlestring}%
  [2016/05/16 v1.5 Cleanup title references (HO)]%
%    \end{macrocode}
%
%    \begin{macrocode}
\begingroup\catcode61\catcode48\catcode32=10\relax%
  \catcode13=5 % ^^M
  \endlinechar=13 %
  \catcode123=1 % {
  \catcode125=2 % }
  \catcode64=11 % @
  \def\x{\endgroup
    \expandafter\edef\csname GTS@AtEnd\endcsname{%
      \endlinechar=\the\endlinechar\relax
      \catcode13=\the\catcode13\relax
      \catcode32=\the\catcode32\relax
      \catcode35=\the\catcode35\relax
      \catcode61=\the\catcode61\relax
      \catcode64=\the\catcode64\relax
      \catcode123=\the\catcode123\relax
      \catcode125=\the\catcode125\relax
    }%
  }%
\x\catcode61\catcode48\catcode32=10\relax%
\catcode13=5 % ^^M
\endlinechar=13 %
\catcode35=6 % #
\catcode64=11 % @
\catcode123=1 % {
\catcode125=2 % }
\def\TMP@EnsureCode#1#2{%
  \edef\GTS@AtEnd{%
    \GTS@AtEnd
    \catcode#1=\the\catcode#1\relax
  }%
  \catcode#1=#2\relax
}
\TMP@EnsureCode{42}{12}% *
\TMP@EnsureCode{44}{12}% ,
\TMP@EnsureCode{45}{12}% -
\TMP@EnsureCode{46}{12}% .
\TMP@EnsureCode{47}{12}% /
\TMP@EnsureCode{91}{12}% [
\TMP@EnsureCode{93}{12}% ]
\edef\GTS@AtEnd{\GTS@AtEnd\noexpand\endinput}
%    \end{macrocode}
%
% \subsection{Options}
%
%    \begin{macrocode}
\RequirePackage{kvoptions}[2009/07/17]
\SetupKeyvalOptions{%
  family=gettitlestring,%
  prefix=GTS@%
}
\newcommand*{\GetTitleStringSetup}{%
  \setkeys{gettitlestring}%
}
\DeclareBoolOption{expand}
\InputIfFileExists{gettitlestring.cfg}{}{}
\ProcessKeyvalOptions*\relax
%    \end{macrocode}
%
% \subsection{\cs{GetTitleString}}
%
%    \begin{macro}{\GetTitleString}
%    \begin{macrocode}
\newcommand*{\GetTitleString}{%
  \ifGTS@expand
    \expandafter\GetTitleStringExpand
  \else
    \expandafter\GetTitleStringNonExpand
  \fi
}
%    \end{macrocode}
%    \end{macro}
%    \begin{macro}{\GetTitleStringExpand}
%    \begin{macrocode}
\newcommand{\GetTitleStringExpand}[1]{%
  \def\GetTitleStringResult{#1}%
  \begingroup
    \GTS@DisablePredefinedCmds
    \GTS@DisableHook
    \edef\x{\endgroup
      \noexpand\def\noexpand\GetTitleStringResult{%
        \GetTitleStringResult
      }%
    }%
  \x
}
%    \end{macrocode}
%    \end{macro}
%    \begin{macro}{\GetTitleString}
%    \begin{macrocode}
\newcommand{\GetTitleStringNonExpand}[1]{%
  \def\GetTitleStringResult{#1}%
  \global\let\GTS@GlobalString\GetTitleStringResult
  \begingroup
    \GTS@RemoveLeft
    \GTS@RemoveRight
  \endgroup
  \let\GetTitleStringResult\GTS@GlobalString
}
%    \end{macrocode}
%    \end{macro}
%
% \subsubsection{Expand method}
%
%    \begin{macro}{\GTS@DisablePredefinedCmds}
%    \begin{macrocode}
\def\GTS@DisablePredefinedCmds{%
  \let\label\@gobble
  \let\zlabel\@gobble
  \let\zref@label\@gobble
  \let\zref@labelbylist\@gobbletwo
  \let\zref@labelbyprops\@gobbletwo
  \let\index\@gobble
  \let\glossary\@gobble
  \let\markboth\@gobbletwo
  \let\@mkboth\@gobbletwo
  \let\markright\@gobble
  \let\phantomsection\@empty
  \def\addcontentsline{\expandafter\@gobble\@gobbletwo}%
  \let\raggedright\@empty
  \let\raggedleft\@empty
  \let\centering\@empty
  \let\protect\@unexpandable@protect
  \let\enit@format\@empty % package enumitem
}
%    \end{macrocode}
%    \end{macro}
%
%    \begin{macro}{\GTS@DisableHook}
%    \begin{macrocode}
\providecommand*{\GTS@DisableHook}{}
%    \end{macrocode}
%    \end{macro}
%    \begin{macro}{\GetTitleStringDisableCommands}
%    \begin{macrocode}
\def\GetTitleStringDisableCommands{%
  \begingroup
    \makeatletter
    \GTS@DisableCommands
}
%    \end{macrocode}
%    \end{macro}
%    \begin{macro}{\GTS@DisableCommands}
%    \begin{macrocode}
\long\def\GTS@DisableCommands#1{%
    \toks0=\expandafter{\GTS@DisableHook}%
    \toks2={#1}%
    \xdef\GTS@GlobalString{\the\toks0 \the\toks2}%
  \endgroup
  \let\GTS@DisableHook\GTS@GlobalString
}
%    \end{macrocode}
%    \end{macro}
%
% \subsubsection{Non-expand method}
%
%    \begin{macrocode}
\def\GTS@RemoveLeft{%
  \toks@\expandafter\expandafter\expandafter{%
    \expandafter\GTS@Car\GTS@GlobalString{}{}{}{}\GTS@Nil
  }%
  \edef\GTS@Token{\the\toks@}%
  \GTS@PredefinedLeftCmds
  \expandafter\futurelet\expandafter\GTS@Token
  \expandafter\GTS@TestLeftSpace\GTS@GlobalString\GTS@Nil
  \GTS@End
}
\def\GTS@End{}
\long\def\GTS@TestLeft#1#2{%
  \def\GTS@temp{#1}%
  \ifx\GTS@temp\GTS@Token
    \toks@\expandafter\expandafter\expandafter{%
      \expandafter#2\GTS@GlobalString\GTS@Nil
    }%
    \expandafter\GTS@TestLeftEnd
  \fi
}
\long\def\GTS@TestLeftEnd#1\GTS@End{%
  \xdef\GTS@GlobalString{\the\toks@}%
  \GTS@RemoveLeft
}
\long\def\GTS@Car#1#2\GTS@Nil{#1}
\long\def\GTS@Cdr#1#2\GTS@Nil{#2}
\long\def\GTS@CdrTwo#1#2#3\GTS@Nil{#3}
\long\def\GTS@CdrThree#1#2#3#4\GTS@Nil{#4}
\long\def\GTS@CdrFour#1#2#3#4#5\GTS@Nil{#5}
\long\def\GTS@TestLeftSpace#1\GTS@Nil{%
  \ifx\GTS@Token\@sptoken
    \toks@\expandafter{%
      \romannumeral-0\GTS@GlobalString
    }%
    \expandafter\GTS@TestLeftEnd
  \fi
}
%    \end{macrocode}
%    \begin{macro}{\GTS@PredefinedLeftCmds}
%    \begin{macrocode}
\def\GTS@PredefinedLeftCmds{%
  \GTS@TestLeft\Hy@phantomsection\GTS@Cdr
  \GTS@TestLeft\Hy@SectionAnchor\GTS@Cdr
  \GTS@TestLeft\Hy@SectionAnchorHref\GTS@CdrTwo
  \GTS@TestLeft\label\GTS@CdrTwo
  \GTS@TestLeft\zlabel\GTS@CdrTwo
  \GTS@TestLeft\index\GTS@CdrTwo
  \GTS@TestLeft\glossary\GTS@CdrTwo
  \GTS@TestLeft\markboth\GTS@CdrThree
  \GTS@TestLeft\@mkboth\GTS@CdrThree
  \GTS@TestLeft\addcontentsline\GTS@CdrFour
  \GTS@TestLeft\enit@format\GTS@Cdr % package enumitem
}
%    \end{macrocode}
%    \end{macro}
%
%    \begin{macrocode}
\def\GTS@RemoveRight{%
  \toks@{}%
  \expandafter\GTS@TestRightLabel\GTS@GlobalString
      \label{}\GTS@Nil\@nil
  \GTS@RemoveRightSpace
}
\begingroup
  \def\GTS@temp#1{\endgroup
    \def\GTS@RemoveRightSpace{%
      \expandafter\GTS@TestRightSpace\GTS@GlobalString
          \GTS@Nil#1\GTS@Nil\@nil
    }%
  }%
\GTS@temp{ }
\def\GTS@TestRightSpace#1 \GTS@Nil#2\@nil{%
  \ifx\relax#2\relax
  \else
    \gdef\GTS@GlobalString{#1}%
    \expandafter\GTS@RemoveRightSpace
  \fi
}
\def\GTS@TestRightLabel#1\label#2#3\GTS@Nil#4\@nil{%
  \def\GTS@temp{#3}%
  \ifx\GTS@temp\@empty
    \expandafter\gdef\expandafter\GTS@GlobalString\expandafter{%
      \the\toks@
      #1%
    }%
    \expandafter\@gobble
  \else
    \expandafter\@firstofone
  \fi
  {%
    \toks@\expandafter{\the\toks@#1}%
    \GTS@TestRightLabel#3\GTS@Nil\@nil
  }%
}
%    \end{macrocode}
%
%    \begin{macrocode}
\GTS@AtEnd%
%</package>
%    \end{macrocode}
%
% \section{Test}
%
% \subsection{Catcode checks for loading}
%
%    \begin{macrocode}
%<*test1>
%    \end{macrocode}
%    \begin{macrocode}
\catcode`\{=1 %
\catcode`\}=2 %
\catcode`\#=6 %
\catcode`\@=11 %
\expandafter\ifx\csname count@\endcsname\relax
  \countdef\count@=255 %
\fi
\expandafter\ifx\csname @gobble\endcsname\relax
  \long\def\@gobble#1{}%
\fi
\expandafter\ifx\csname @firstofone\endcsname\relax
  \long\def\@firstofone#1{#1}%
\fi
\expandafter\ifx\csname loop\endcsname\relax
  \expandafter\@firstofone
\else
  \expandafter\@gobble
\fi
{%
  \def\loop#1\repeat{%
    \def\body{#1}%
    \iterate
  }%
  \def\iterate{%
    \body
      \let\next\iterate
    \else
      \let\next\relax
    \fi
    \next
  }%
  \let\repeat=\fi
}%
\def\RestoreCatcodes{}
\count@=0 %
\loop
  \edef\RestoreCatcodes{%
    \RestoreCatcodes
    \catcode\the\count@=\the\catcode\count@\relax
  }%
\ifnum\count@<255 %
  \advance\count@ 1 %
\repeat

\def\RangeCatcodeInvalid#1#2{%
  \count@=#1\relax
  \loop
    \catcode\count@=15 %
  \ifnum\count@<#2\relax
    \advance\count@ 1 %
  \repeat
}
\def\RangeCatcodeCheck#1#2#3{%
  \count@=#1\relax
  \loop
    \ifnum#3=\catcode\count@
    \else
      \errmessage{%
        Character \the\count@\space
        with wrong catcode \the\catcode\count@\space
        instead of \number#3%
      }%
    \fi
  \ifnum\count@<#2\relax
    \advance\count@ 1 %
  \repeat
}
\def\space{ }
\expandafter\ifx\csname LoadCommand\endcsname\relax
  \def\LoadCommand{\input gettitlestring.sty\relax}%
\fi
\def\Test{%
  \RangeCatcodeInvalid{0}{47}%
  \RangeCatcodeInvalid{58}{64}%
  \RangeCatcodeInvalid{91}{96}%
  \RangeCatcodeInvalid{123}{255}%
  \catcode`\@=12 %
  \catcode`\\=0 %
  \catcode`\%=14 %
  \LoadCommand
  \RangeCatcodeCheck{0}{36}{15}%
  \RangeCatcodeCheck{37}{37}{14}%
  \RangeCatcodeCheck{38}{47}{15}%
  \RangeCatcodeCheck{48}{57}{12}%
  \RangeCatcodeCheck{58}{63}{15}%
  \RangeCatcodeCheck{64}{64}{12}%
  \RangeCatcodeCheck{65}{90}{11}%
  \RangeCatcodeCheck{91}{91}{15}%
  \RangeCatcodeCheck{92}{92}{0}%
  \RangeCatcodeCheck{93}{96}{15}%
  \RangeCatcodeCheck{97}{122}{11}%
  \RangeCatcodeCheck{123}{255}{15}%
  \RestoreCatcodes
}
\Test
\csname @@end\endcsname
\end
%    \end{macrocode}
%    \begin{macrocode}
%</test1>
%    \end{macrocode}
%
% \subsection{Test of non-expand method}
%
%    \begin{macrocode}
%<*test2>
\NeedsTeXFormat{LaTeX2e}
\documentclass{minimal}
\usepackage{gettitlestring}[2016/05/16]
\usepackage{qstest}
\IncludeTests{*}
\LogTests{log}{*}{*}
\begin{document}
\begin{qstest}{non-expand}{non-expand}
  \def\test#1#2{%
    \sbox0{%
      \GetTitleString{#1}%
      \Expect{#2}*{\GetTitleStringResult}%
    }%
    \Expect{0.0pt}*{\the\wd0}%
  }%
  \test{}{}%
  \test{ }{}%
  \test{ x }{x}%
  \test{ x y }{x y}%
  \test{ \relax}{\relax}%
  \test{\label{f}a}{a}%
  \test{ \label{f}a}{a}%
  \test{\label{f} a}{a}%
  \test{ \label{f} a}{a}%
  \test{a\label{f}}{a}%
  \test{a\label{f} }{a}%
  \test{a \label{f}}{a}%
  \test{a \label{f} }{a}%
  \test{a\label{f}b\label{g}}{ab}%
  \test{a \label{f}b \label{g} }{a b}%
  \test{a\label{f} b \label{g} }{a b}%
\end{qstest}
\end{document}
%</test2>
%    \end{macrocode}
%
% \section{Installation}
%
% \subsection{Download}
%
% \paragraph{Package.} This package is available on
% CTAN\footnote{\CTANpkg{gettitlestring}}:
% \begin{description}
% \item[\CTAN{macros/latex/contrib/oberdiek/gettitlestring.dtx}] The source file.
% \item[\CTAN{macros/latex/contrib/oberdiek/gettitlestring.pdf}] Documentation.
% \end{description}
%
%
% \paragraph{Bundle.} All the packages of the bundle `oberdiek'
% are also available in a TDS compliant ZIP archive. There
% the packages are already unpacked and the documentation files
% are generated. The files and directories obey the TDS standard.
% \begin{description}
% \item[\CTANinstall{install/macros/latex/contrib/oberdiek.tds.zip}]
% \end{description}
% \emph{TDS} refers to the standard ``A Directory Structure
% for \TeX\ Files'' (\CTAN{tds/tds.pdf}). Directories
% with \xfile{texmf} in their name are usually organized this way.
%
% \subsection{Bundle installation}
%
% \paragraph{Unpacking.} Unpack the \xfile{oberdiek.tds.zip} in the
% TDS tree (also known as \xfile{texmf} tree) of your choice.
% Example (linux):
% \begin{quote}
%   |unzip oberdiek.tds.zip -d ~/texmf|
% \end{quote}
%
% \paragraph{Script installation.}
% Check the directory \xfile{TDS:scripts/oberdiek/} for
% scripts that need further installation steps.
% Package \xpackage{attachfile2} comes with the Perl script
% \xfile{pdfatfi.pl} that should be installed in such a way
% that it can be called as \texttt{pdfatfi}.
% Example (linux):
% \begin{quote}
%   |chmod +x scripts/oberdiek/pdfatfi.pl|\\
%   |cp scripts/oberdiek/pdfatfi.pl /usr/local/bin/|
% \end{quote}
%
% \subsection{Package installation}
%
% \paragraph{Unpacking.} The \xfile{.dtx} file is a self-extracting
% \docstrip\ archive. The files are extracted by running the
% \xfile{.dtx} through \plainTeX:
% \begin{quote}
%   \verb|tex gettitlestring.dtx|
% \end{quote}
%
% \paragraph{TDS.} Now the different files must be moved into
% the different directories in your installation TDS tree
% (also known as \xfile{texmf} tree):
% \begin{quote}
% \def\t{^^A
% \begin{tabular}{@{}>{\ttfamily}l@{ $\rightarrow$ }>{\ttfamily}l@{}}
%   gettitlestring.sty & tex/generic/oberdiek/gettitlestring.sty\\
%   gettitlestring.pdf & doc/latex/oberdiek/gettitlestring.pdf\\
%   test/gettitlestring-test1.tex & doc/latex/oberdiek/test/gettitlestring-test1.tex\\
%   test/gettitlestring-test2.tex & doc/latex/oberdiek/test/gettitlestring-test2.tex\\
%   gettitlestring.dtx & source/latex/oberdiek/gettitlestring.dtx\\
% \end{tabular}^^A
% }^^A
% \sbox0{\t}^^A
% \ifdim\wd0>\linewidth
%   \begingroup
%     \advance\linewidth by\leftmargin
%     \advance\linewidth by\rightmargin
%   \edef\x{\endgroup
%     \def\noexpand\lw{\the\linewidth}^^A
%   }\x
%   \def\lwbox{^^A
%     \leavevmode
%     \hbox to \linewidth{^^A
%       \kern-\leftmargin\relax
%       \hss
%       \usebox0
%       \hss
%       \kern-\rightmargin\relax
%     }^^A
%   }^^A
%   \ifdim\wd0>\lw
%     \sbox0{\small\t}^^A
%     \ifdim\wd0>\linewidth
%       \ifdim\wd0>\lw
%         \sbox0{\footnotesize\t}^^A
%         \ifdim\wd0>\linewidth
%           \ifdim\wd0>\lw
%             \sbox0{\scriptsize\t}^^A
%             \ifdim\wd0>\linewidth
%               \ifdim\wd0>\lw
%                 \sbox0{\tiny\t}^^A
%                 \ifdim\wd0>\linewidth
%                   \lwbox
%                 \else
%                   \usebox0
%                 \fi
%               \else
%                 \lwbox
%               \fi
%             \else
%               \usebox0
%             \fi
%           \else
%             \lwbox
%           \fi
%         \else
%           \usebox0
%         \fi
%       \else
%         \lwbox
%       \fi
%     \else
%       \usebox0
%     \fi
%   \else
%     \lwbox
%   \fi
% \else
%   \usebox0
% \fi
% \end{quote}
% If you have a \xfile{docstrip.cfg} that configures and enables \docstrip's
% TDS installing feature, then some files can already be in the right
% place, see the documentation of \docstrip.
%
% \subsection{Refresh file name databases}
%
% If your \TeX~distribution
% (\teTeX, \mikTeX, \dots) relies on file name databases, you must refresh
% these. For example, \teTeX\ users run \verb|texhash| or
% \verb|mktexlsr|.
%
% \subsection{Some details for the interested}
%
% \paragraph{Attached source.}
%
% The PDF documentation on CTAN also includes the
% \xfile{.dtx} source file. It can be extracted by
% AcrobatReader 6 or higher. Another option is \textsf{pdftk},
% e.g. unpack the file into the current directory:
% \begin{quote}
%   \verb|pdftk gettitlestring.pdf unpack_files output .|
% \end{quote}
%
% \paragraph{Unpacking with \LaTeX.}
% The \xfile{.dtx} chooses its action depending on the format:
% \begin{description}
% \item[\plainTeX:] Run \docstrip\ and extract the files.
% \item[\LaTeX:] Generate the documentation.
% \end{description}
% If you insist on using \LaTeX\ for \docstrip\ (really,
% \docstrip\ does not need \LaTeX), then inform the autodetect routine
% about your intention:
% \begin{quote}
%   \verb|latex \let\install=y\input{gettitlestring.dtx}|
% \end{quote}
% Do not forget to quote the argument according to the demands
% of your shell.
%
% \paragraph{Generating the documentation.}
% You can use both the \xfile{.dtx} or the \xfile{.drv} to generate
% the documentation. The process can be configured by the
% configuration file \xfile{ltxdoc.cfg}. For instance, put this
% line into this file, if you want to have A4 as paper format:
% \begin{quote}
%   \verb|\PassOptionsToClass{a4paper}{article}|
% \end{quote}
% An example follows how to generate the
% documentation with pdf\LaTeX:
% \begin{quote}
%\begin{verbatim}
%pdflatex gettitlestring.dtx
%makeindex -s gind.ist gettitlestring.idx
%pdflatex gettitlestring.dtx
%makeindex -s gind.ist gettitlestring.idx
%pdflatex gettitlestring.dtx
%\end{verbatim}
% \end{quote}
%
% \begin{thebibliography}{9}
%
% \bibitem{memoir}
% Peter Wilson, Lars Madsen:
% \textit{The Memoir Class};
% 2009/11/17 v1.61803398c;
% \CTANpkg{memoir}
%
% \bibitem{titleref}
% Donald Arsenau:
% \textit{Titleref.sty};
% 2001/04/05 ver 3.1;
% \CTAN{macros/latex/contrib/misc/titleref.sty}
%
% \bibitem{zref}
% Heiko Oberdiek:
% \textit{The \xpackage{zref} package};
% 2009/12/08 v2.7;
% \CTAN{macros/latex/contrib/oberdiek/zref.pdf}
%
% \end{thebibliography}
%
% \begin{History}
%   \begin{Version}{2009/12/08 v1.0}
%   \item
%     The first version.
%   \end{Version}
%   \begin{Version}{2009/12/12 v1.1}
%   \item
%     Short info shortened.
%   \end{Version}
%   \begin{Version}{2009/12/13 v1.2}
%   \item
%     Forgotten third argument for \cs{InputIfFileExists} added.
%   \end{Version}
%   \begin{Version}{2009/12/18 v1.3}
%   \item
%     \cs{Hy@SectionAnchorHref} added for filtering
%     (hyperref 2009/12/18 v6.79w).
%   \end{Version}
%   \begin{Version}{2010/12/03 v1.4}
%   \item
%     Support of package \xpackage{enumitem}: removing
%     \cs{enit@format} from title string (problem report by GL).
%   \end{Version}
%   \begin{Version}{2016/05/16 v1.5}
%   \item
%     Documentation updates.
%   \end{Version}
% \end{History}
%
% \PrintIndex
%
% \Finale
\endinput

%        (quote the arguments according to the demands of your shell)
%
% Documentation:
%    (a) If gettitlestring.drv is present:
%           latex gettitlestring.drv
%    (b) Without gettitlestring.drv:
%           latex gettitlestring.dtx; ...
%    The class ltxdoc loads the configuration file ltxdoc.cfg
%    if available. Here you can specify further options, e.g.
%    use A4 as paper format:
%       \PassOptionsToClass{a4paper}{article}
%
%    Programm calls to get the documentation (example):
%       pdflatex gettitlestring.dtx
%       makeindex -s gind.ist gettitlestring.idx
%       pdflatex gettitlestring.dtx
%       makeindex -s gind.ist gettitlestring.idx
%       pdflatex gettitlestring.dtx
%
% Installation:
%    TDS:tex/generic/oberdiek/gettitlestring.sty
%    TDS:doc/latex/oberdiek/gettitlestring.pdf
%    TDS:doc/latex/oberdiek/test/gettitlestring-test1.tex
%    TDS:doc/latex/oberdiek/test/gettitlestring-test2.tex
%    TDS:source/latex/oberdiek/gettitlestring.dtx
%
%<*ignore>
\begingroup
  \catcode123=1 %
  \catcode125=2 %
  \def\x{LaTeX2e}%
\expandafter\endgroup
\ifcase 0\ifx\install y1\fi\expandafter
         \ifx\csname processbatchFile\endcsname\relax\else1\fi
         \ifx\fmtname\x\else 1\fi\relax
\else\csname fi\endcsname
%</ignore>
%<*install>
\input docstrip.tex
\Msg{************************************************************************}
\Msg{* Installation}
\Msg{* Package: gettitlestring 2016/05/16 v1.5 Cleanup title references (HO)}
\Msg{************************************************************************}

\keepsilent
\askforoverwritefalse

\let\MetaPrefix\relax
\preamble

This is a generated file.

Project: gettitlestring
Version: 2016/05/16 v1.5

Copyright (C) 2009, 2010 by
   Heiko Oberdiek <heiko.oberdiek at googlemail.com>

This work may be distributed and/or modified under the
conditions of the LaTeX Project Public License, either
version 1.3c of this license or (at your option) any later
version. This version of this license is in
   http://www.latex-project.org/lppl/lppl-1-3c.txt
and the latest version of this license is in
   http://www.latex-project.org/lppl.txt
and version 1.3 or later is part of all distributions of
LaTeX version 2005/12/01 or later.

This work has the LPPL maintenance status "maintained".

This Current Maintainer of this work is Heiko Oberdiek.

The Base Interpreter refers to any `TeX-Format',
because some files are installed in TDS:tex/generic//.

This work consists of the main source file gettitlestring.dtx
and the derived files
   gettitlestring.sty, gettitlestring.pdf, gettitlestring.ins,
   gettitlestring.drv, gettitlestring-test1.tex,
   gettitlestring-test2.tex.

\endpreamble
\let\MetaPrefix\DoubleperCent

\generate{%
  \file{gettitlestring.ins}{\from{gettitlestring.dtx}{install}}%
  \file{gettitlestring.drv}{\from{gettitlestring.dtx}{driver}}%
  \usedir{tex/generic/oberdiek}%
  \file{gettitlestring.sty}{\from{gettitlestring.dtx}{package}}%
%  \usedir{doc/latex/oberdiek/test}%
%  \file{gettitlestring-test1.tex}{\from{gettitlestring.dtx}{test1}}%
%  \file{gettitlestring-test2.tex}{\from{gettitlestring.dtx}{test2}}%
  \nopreamble
  \nopostamble
%  \usedir{source/latex/oberdiek/catalogue}%
%  \file{gettitlestring.xml}{\from{gettitlestring.dtx}{catalogue}}%
}

\catcode32=13\relax% active space
\let =\space%
\Msg{************************************************************************}
\Msg{*}
\Msg{* To finish the installation you have to move the following}
\Msg{* file into a directory searched by TeX:}
\Msg{*}
\Msg{*     gettitlestring.sty}
\Msg{*}
\Msg{* To produce the documentation run the file `gettitlestring.drv'}
\Msg{* through LaTeX.}
\Msg{*}
\Msg{* Happy TeXing!}
\Msg{*}
\Msg{************************************************************************}

\endbatchfile
%</install>
%<*ignore>
\fi
%</ignore>
%<*driver>
\NeedsTeXFormat{LaTeX2e}
\ProvidesFile{gettitlestring.drv}%
  [2016/05/16 v1.5 Cleanup title references (HO)]%
\documentclass{ltxdoc}
\usepackage{holtxdoc}[2011/11/22]
\begin{document}
  \DocInput{gettitlestring.dtx}%
\end{document}
%</driver>
% \fi
%
%
% \CharacterTable
%  {Upper-case    \A\B\C\D\E\F\G\H\I\J\K\L\M\N\O\P\Q\R\S\T\U\V\W\X\Y\Z
%   Lower-case    \a\b\c\d\e\f\g\h\i\j\k\l\m\n\o\p\q\r\s\t\u\v\w\x\y\z
%   Digits        \0\1\2\3\4\5\6\7\8\9
%   Exclamation   \!     Double quote  \"     Hash (number) \#
%   Dollar        \$     Percent       \%     Ampersand     \&
%   Acute accent  \'     Left paren    \(     Right paren   \)
%   Asterisk      \*     Plus          \+     Comma         \,
%   Minus         \-     Point         \.     Solidus       \/
%   Colon         \:     Semicolon     \;     Less than     \<
%   Equals        \=     Greater than  \>     Question mark \?
%   Commercial at \@     Left bracket  \[     Backslash     \\
%   Right bracket \]     Circumflex    \^     Underscore    \_
%   Grave accent  \`     Left brace    \{     Vertical bar  \|
%   Right brace   \}     Tilde         \~}
%
% \GetFileInfo{gettitlestring.drv}
%
% \title{The \xpackage{gettitlestring} package}
% \date{2016/05/16 v1.5}
% \author{Heiko Oberdiek\thanks
% {Please report any issues at \url{https://github.com/ho-tex/oberdiek/issues}}\\
% \xemail{heiko.oberdiek at googlemail.com}}
%
% \maketitle
%
% \begin{abstract}
% The \LaTeX\ package addresses packages that are dealing with
% references to titles (\cs{section}, \cs{caption}, \dots).
% The package tries to remove \cs{label} and other
% commands from title strings.
% \end{abstract}
%
% \tableofcontents
%
% \section{Documentation}
%
% \subsection{Macros}
%
% \begin{declcs}{GetTitleStringSetup} \M{key value list}
% \end{declcs}
% The options are given as comma separated key value pairs.
% See section \ref{sec:options}.
%
% \begin{declcs}{GetTitleString} \M{text}\\
% \cs{GetTitleStringExpand} \M{text}\\
% \cs{GetTitleStringNonExpand} \M{text}
% \end{declcs}
% Macro \cs{GetTitleString} tries to remove unwanted stuff from \meta{text}
% the result is stored in Macro \cs{GetTitleStringResult}.
% Two methods are available:
% \begin{description}
% \item[\cs{GetTitleStringExpand}:]
% The \meta{text} is expanded in a context where the unwanted
% macros are redefined to remove themselves.
% This is the method used in packages \xpackage{titleref}~\cite{titleref},
% \xpackage{zref-titleref}~\cite{zref}
% or class \xclass{memoir}~\cite{memoir}.
% \cs{protect} is supported, but fragile material might break.
% \item[\cs{GetTitleStringNonExpand}:]
% The \meta{text} is not expanded. Thus the removal of unwanted
% material is more difficult. It is especially removed at the
% start of the \meta{text} and spaces are removed from the end.
% Currently only \cs{label} is removed in the whole string,
% if it is not hidden inside curly braces or part of macro
% definitions. Thus the removal of unwanted stuff might not be
% complete, but fragile material will not break.
% (But the result string can break at a later time, of course).
% \end{description}
% Option \xoption{expand} controls which method is used by
% macro \cs{GetTitleString}.
%
% \begin{declcs}{GetTitleStringDisableCommands} \M{code}
% \end{declcs}
% The \meta{code} is called right before the
% text is expanded in \cs{GetTitleStringExpand}.
% Additional definitions can be given for macros that
% should be removed.
% Keep in mind that expansion means that the definitions
% must work in expandable context. Macros like
% \cs{@ifstar} or \cs{@ifnextchar} or optional arguments
% will not work. The macro names in \meta{code} may contain
% the at sign |@|, it has catcode 11 (letter).
%
% \subsection{Options}\label{sec:options}
%
% \begin{description}
% \item[\xoption{expand}:] Boolean option, takes values |true| or |false|.
% No value means |true|. The option specifies the method to remove
% unwanted stuff from the title string, see below.
% \end{description}
% Options can be set at the following places:
% \begin{itemize}
% \item \cs{usepackage}
% \item Configuration file \xfile{gettitlestring.cfg}.
% \item \cs{GetTitleStringSetup}
% \end{itemize}
%
% \StopEventually{
% }
%
% \section{Implementation}
%    \begin{macrocode}
%<*package>
%    \end{macrocode}
%    Reload check, especially if the package is not used with \LaTeX.
%    \begin{macrocode}
\begingroup\catcode61\catcode48\catcode32=10\relax%
  \catcode13=5 % ^^M
  \endlinechar=13 %
  \catcode35=6 % #
  \catcode39=12 % '
  \catcode44=12 % ,
  \catcode45=12 % -
  \catcode46=12 % .
  \catcode58=12 % :
  \catcode64=11 % @
  \catcode123=1 % {
  \catcode125=2 % }
  \expandafter\let\expandafter\x\csname ver@gettitlestring.sty\endcsname
  \ifx\x\relax % plain-TeX, first loading
  \else
    \def\empty{}%
    \ifx\x\empty % LaTeX, first loading,
      % variable is initialized, but \ProvidesPackage not yet seen
    \else
      \expandafter\ifx\csname PackageInfo\endcsname\relax
        \def\x#1#2{%
          \immediate\write-1{Package #1 Info: #2.}%
        }%
      \else
        \def\x#1#2{\PackageInfo{#1}{#2, stopped}}%
      \fi
      \x{gettitlestring}{The package is already loaded}%
      \aftergroup\endinput
    \fi
  \fi
\endgroup%
%    \end{macrocode}
%    Package identification:
%    \begin{macrocode}
\begingroup\catcode61\catcode48\catcode32=10\relax%
  \catcode13=5 % ^^M
  \endlinechar=13 %
  \catcode35=6 % #
  \catcode39=12 % '
  \catcode40=12 % (
  \catcode41=12 % )
  \catcode44=12 % ,
  \catcode45=12 % -
  \catcode46=12 % .
  \catcode47=12 % /
  \catcode58=12 % :
  \catcode64=11 % @
  \catcode91=12 % [
  \catcode93=12 % ]
  \catcode123=1 % {
  \catcode125=2 % }
  \expandafter\ifx\csname ProvidesPackage\endcsname\relax
    \def\x#1#2#3[#4]{\endgroup
      \immediate\write-1{Package: #3 #4}%
      \xdef#1{#4}%
    }%
  \else
    \def\x#1#2[#3]{\endgroup
      #2[{#3}]%
      \ifx#1\@undefined
        \xdef#1{#3}%
      \fi
      \ifx#1\relax
        \xdef#1{#3}%
      \fi
    }%
  \fi
\expandafter\x\csname ver@gettitlestring.sty\endcsname
\ProvidesPackage{gettitlestring}%
  [2016/05/16 v1.5 Cleanup title references (HO)]%
%    \end{macrocode}
%
%    \begin{macrocode}
\begingroup\catcode61\catcode48\catcode32=10\relax%
  \catcode13=5 % ^^M
  \endlinechar=13 %
  \catcode123=1 % {
  \catcode125=2 % }
  \catcode64=11 % @
  \def\x{\endgroup
    \expandafter\edef\csname GTS@AtEnd\endcsname{%
      \endlinechar=\the\endlinechar\relax
      \catcode13=\the\catcode13\relax
      \catcode32=\the\catcode32\relax
      \catcode35=\the\catcode35\relax
      \catcode61=\the\catcode61\relax
      \catcode64=\the\catcode64\relax
      \catcode123=\the\catcode123\relax
      \catcode125=\the\catcode125\relax
    }%
  }%
\x\catcode61\catcode48\catcode32=10\relax%
\catcode13=5 % ^^M
\endlinechar=13 %
\catcode35=6 % #
\catcode64=11 % @
\catcode123=1 % {
\catcode125=2 % }
\def\TMP@EnsureCode#1#2{%
  \edef\GTS@AtEnd{%
    \GTS@AtEnd
    \catcode#1=\the\catcode#1\relax
  }%
  \catcode#1=#2\relax
}
\TMP@EnsureCode{42}{12}% *
\TMP@EnsureCode{44}{12}% ,
\TMP@EnsureCode{45}{12}% -
\TMP@EnsureCode{46}{12}% .
\TMP@EnsureCode{47}{12}% /
\TMP@EnsureCode{91}{12}% [
\TMP@EnsureCode{93}{12}% ]
\edef\GTS@AtEnd{\GTS@AtEnd\noexpand\endinput}
%    \end{macrocode}
%
% \subsection{Options}
%
%    \begin{macrocode}
\RequirePackage{kvoptions}[2009/07/17]
\SetupKeyvalOptions{%
  family=gettitlestring,%
  prefix=GTS@%
}
\newcommand*{\GetTitleStringSetup}{%
  \setkeys{gettitlestring}%
}
\DeclareBoolOption{expand}
\InputIfFileExists{gettitlestring.cfg}{}{}
\ProcessKeyvalOptions*\relax
%    \end{macrocode}
%
% \subsection{\cs{GetTitleString}}
%
%    \begin{macro}{\GetTitleString}
%    \begin{macrocode}
\newcommand*{\GetTitleString}{%
  \ifGTS@expand
    \expandafter\GetTitleStringExpand
  \else
    \expandafter\GetTitleStringNonExpand
  \fi
}
%    \end{macrocode}
%    \end{macro}
%    \begin{macro}{\GetTitleStringExpand}
%    \begin{macrocode}
\newcommand{\GetTitleStringExpand}[1]{%
  \def\GetTitleStringResult{#1}%
  \begingroup
    \GTS@DisablePredefinedCmds
    \GTS@DisableHook
    \edef\x{\endgroup
      \noexpand\def\noexpand\GetTitleStringResult{%
        \GetTitleStringResult
      }%
    }%
  \x
}
%    \end{macrocode}
%    \end{macro}
%    \begin{macro}{\GetTitleString}
%    \begin{macrocode}
\newcommand{\GetTitleStringNonExpand}[1]{%
  \def\GetTitleStringResult{#1}%
  \global\let\GTS@GlobalString\GetTitleStringResult
  \begingroup
    \GTS@RemoveLeft
    \GTS@RemoveRight
  \endgroup
  \let\GetTitleStringResult\GTS@GlobalString
}
%    \end{macrocode}
%    \end{macro}
%
% \subsubsection{Expand method}
%
%    \begin{macro}{\GTS@DisablePredefinedCmds}
%    \begin{macrocode}
\def\GTS@DisablePredefinedCmds{%
  \let\label\@gobble
  \let\zlabel\@gobble
  \let\zref@label\@gobble
  \let\zref@labelbylist\@gobbletwo
  \let\zref@labelbyprops\@gobbletwo
  \let\index\@gobble
  \let\glossary\@gobble
  \let\markboth\@gobbletwo
  \let\@mkboth\@gobbletwo
  \let\markright\@gobble
  \let\phantomsection\@empty
  \def\addcontentsline{\expandafter\@gobble\@gobbletwo}%
  \let\raggedright\@empty
  \let\raggedleft\@empty
  \let\centering\@empty
  \let\protect\@unexpandable@protect
  \let\enit@format\@empty % package enumitem
}
%    \end{macrocode}
%    \end{macro}
%
%    \begin{macro}{\GTS@DisableHook}
%    \begin{macrocode}
\providecommand*{\GTS@DisableHook}{}
%    \end{macrocode}
%    \end{macro}
%    \begin{macro}{\GetTitleStringDisableCommands}
%    \begin{macrocode}
\def\GetTitleStringDisableCommands{%
  \begingroup
    \makeatletter
    \GTS@DisableCommands
}
%    \end{macrocode}
%    \end{macro}
%    \begin{macro}{\GTS@DisableCommands}
%    \begin{macrocode}
\long\def\GTS@DisableCommands#1{%
    \toks0=\expandafter{\GTS@DisableHook}%
    \toks2={#1}%
    \xdef\GTS@GlobalString{\the\toks0 \the\toks2}%
  \endgroup
  \let\GTS@DisableHook\GTS@GlobalString
}
%    \end{macrocode}
%    \end{macro}
%
% \subsubsection{Non-expand method}
%
%    \begin{macrocode}
\def\GTS@RemoveLeft{%
  \toks@\expandafter\expandafter\expandafter{%
    \expandafter\GTS@Car\GTS@GlobalString{}{}{}{}\GTS@Nil
  }%
  \edef\GTS@Token{\the\toks@}%
  \GTS@PredefinedLeftCmds
  \expandafter\futurelet\expandafter\GTS@Token
  \expandafter\GTS@TestLeftSpace\GTS@GlobalString\GTS@Nil
  \GTS@End
}
\def\GTS@End{}
\long\def\GTS@TestLeft#1#2{%
  \def\GTS@temp{#1}%
  \ifx\GTS@temp\GTS@Token
    \toks@\expandafter\expandafter\expandafter{%
      \expandafter#2\GTS@GlobalString\GTS@Nil
    }%
    \expandafter\GTS@TestLeftEnd
  \fi
}
\long\def\GTS@TestLeftEnd#1\GTS@End{%
  \xdef\GTS@GlobalString{\the\toks@}%
  \GTS@RemoveLeft
}
\long\def\GTS@Car#1#2\GTS@Nil{#1}
\long\def\GTS@Cdr#1#2\GTS@Nil{#2}
\long\def\GTS@CdrTwo#1#2#3\GTS@Nil{#3}
\long\def\GTS@CdrThree#1#2#3#4\GTS@Nil{#4}
\long\def\GTS@CdrFour#1#2#3#4#5\GTS@Nil{#5}
\long\def\GTS@TestLeftSpace#1\GTS@Nil{%
  \ifx\GTS@Token\@sptoken
    \toks@\expandafter{%
      \romannumeral-0\GTS@GlobalString
    }%
    \expandafter\GTS@TestLeftEnd
  \fi
}
%    \end{macrocode}
%    \begin{macro}{\GTS@PredefinedLeftCmds}
%    \begin{macrocode}
\def\GTS@PredefinedLeftCmds{%
  \GTS@TestLeft\Hy@phantomsection\GTS@Cdr
  \GTS@TestLeft\Hy@SectionAnchor\GTS@Cdr
  \GTS@TestLeft\Hy@SectionAnchorHref\GTS@CdrTwo
  \GTS@TestLeft\label\GTS@CdrTwo
  \GTS@TestLeft\zlabel\GTS@CdrTwo
  \GTS@TestLeft\index\GTS@CdrTwo
  \GTS@TestLeft\glossary\GTS@CdrTwo
  \GTS@TestLeft\markboth\GTS@CdrThree
  \GTS@TestLeft\@mkboth\GTS@CdrThree
  \GTS@TestLeft\addcontentsline\GTS@CdrFour
  \GTS@TestLeft\enit@format\GTS@Cdr % package enumitem
}
%    \end{macrocode}
%    \end{macro}
%
%    \begin{macrocode}
\def\GTS@RemoveRight{%
  \toks@{}%
  \expandafter\GTS@TestRightLabel\GTS@GlobalString
      \label{}\GTS@Nil\@nil
  \GTS@RemoveRightSpace
}
\begingroup
  \def\GTS@temp#1{\endgroup
    \def\GTS@RemoveRightSpace{%
      \expandafter\GTS@TestRightSpace\GTS@GlobalString
          \GTS@Nil#1\GTS@Nil\@nil
    }%
  }%
\GTS@temp{ }
\def\GTS@TestRightSpace#1 \GTS@Nil#2\@nil{%
  \ifx\relax#2\relax
  \else
    \gdef\GTS@GlobalString{#1}%
    \expandafter\GTS@RemoveRightSpace
  \fi
}
\def\GTS@TestRightLabel#1\label#2#3\GTS@Nil#4\@nil{%
  \def\GTS@temp{#3}%
  \ifx\GTS@temp\@empty
    \expandafter\gdef\expandafter\GTS@GlobalString\expandafter{%
      \the\toks@
      #1%
    }%
    \expandafter\@gobble
  \else
    \expandafter\@firstofone
  \fi
  {%
    \toks@\expandafter{\the\toks@#1}%
    \GTS@TestRightLabel#3\GTS@Nil\@nil
  }%
}
%    \end{macrocode}
%
%    \begin{macrocode}
\GTS@AtEnd%
%</package>
%    \end{macrocode}
%
% \section{Test}
%
% \subsection{Catcode checks for loading}
%
%    \begin{macrocode}
%<*test1>
%    \end{macrocode}
%    \begin{macrocode}
\catcode`\{=1 %
\catcode`\}=2 %
\catcode`\#=6 %
\catcode`\@=11 %
\expandafter\ifx\csname count@\endcsname\relax
  \countdef\count@=255 %
\fi
\expandafter\ifx\csname @gobble\endcsname\relax
  \long\def\@gobble#1{}%
\fi
\expandafter\ifx\csname @firstofone\endcsname\relax
  \long\def\@firstofone#1{#1}%
\fi
\expandafter\ifx\csname loop\endcsname\relax
  \expandafter\@firstofone
\else
  \expandafter\@gobble
\fi
{%
  \def\loop#1\repeat{%
    \def\body{#1}%
    \iterate
  }%
  \def\iterate{%
    \body
      \let\next\iterate
    \else
      \let\next\relax
    \fi
    \next
  }%
  \let\repeat=\fi
}%
\def\RestoreCatcodes{}
\count@=0 %
\loop
  \edef\RestoreCatcodes{%
    \RestoreCatcodes
    \catcode\the\count@=\the\catcode\count@\relax
  }%
\ifnum\count@<255 %
  \advance\count@ 1 %
\repeat

\def\RangeCatcodeInvalid#1#2{%
  \count@=#1\relax
  \loop
    \catcode\count@=15 %
  \ifnum\count@<#2\relax
    \advance\count@ 1 %
  \repeat
}
\def\RangeCatcodeCheck#1#2#3{%
  \count@=#1\relax
  \loop
    \ifnum#3=\catcode\count@
    \else
      \errmessage{%
        Character \the\count@\space
        with wrong catcode \the\catcode\count@\space
        instead of \number#3%
      }%
    \fi
  \ifnum\count@<#2\relax
    \advance\count@ 1 %
  \repeat
}
\def\space{ }
\expandafter\ifx\csname LoadCommand\endcsname\relax
  \def\LoadCommand{\input gettitlestring.sty\relax}%
\fi
\def\Test{%
  \RangeCatcodeInvalid{0}{47}%
  \RangeCatcodeInvalid{58}{64}%
  \RangeCatcodeInvalid{91}{96}%
  \RangeCatcodeInvalid{123}{255}%
  \catcode`\@=12 %
  \catcode`\\=0 %
  \catcode`\%=14 %
  \LoadCommand
  \RangeCatcodeCheck{0}{36}{15}%
  \RangeCatcodeCheck{37}{37}{14}%
  \RangeCatcodeCheck{38}{47}{15}%
  \RangeCatcodeCheck{48}{57}{12}%
  \RangeCatcodeCheck{58}{63}{15}%
  \RangeCatcodeCheck{64}{64}{12}%
  \RangeCatcodeCheck{65}{90}{11}%
  \RangeCatcodeCheck{91}{91}{15}%
  \RangeCatcodeCheck{92}{92}{0}%
  \RangeCatcodeCheck{93}{96}{15}%
  \RangeCatcodeCheck{97}{122}{11}%
  \RangeCatcodeCheck{123}{255}{15}%
  \RestoreCatcodes
}
\Test
\csname @@end\endcsname
\end
%    \end{macrocode}
%    \begin{macrocode}
%</test1>
%    \end{macrocode}
%
% \subsection{Test of non-expand method}
%
%    \begin{macrocode}
%<*test2>
\NeedsTeXFormat{LaTeX2e}
\documentclass{minimal}
\usepackage{gettitlestring}[2016/05/16]
\usepackage{qstest}
\IncludeTests{*}
\LogTests{log}{*}{*}
\begin{document}
\begin{qstest}{non-expand}{non-expand}
  \def\test#1#2{%
    \sbox0{%
      \GetTitleString{#1}%
      \Expect{#2}*{\GetTitleStringResult}%
    }%
    \Expect{0.0pt}*{\the\wd0}%
  }%
  \test{}{}%
  \test{ }{}%
  \test{ x }{x}%
  \test{ x y }{x y}%
  \test{ \relax}{\relax}%
  \test{\label{f}a}{a}%
  \test{ \label{f}a}{a}%
  \test{\label{f} a}{a}%
  \test{ \label{f} a}{a}%
  \test{a\label{f}}{a}%
  \test{a\label{f} }{a}%
  \test{a \label{f}}{a}%
  \test{a \label{f} }{a}%
  \test{a\label{f}b\label{g}}{ab}%
  \test{a \label{f}b \label{g} }{a b}%
  \test{a\label{f} b \label{g} }{a b}%
\end{qstest}
\end{document}
%</test2>
%    \end{macrocode}
%
% \section{Installation}
%
% \subsection{Download}
%
% \paragraph{Package.} This package is available on
% CTAN\footnote{\CTANpkg{gettitlestring}}:
% \begin{description}
% \item[\CTAN{macros/latex/contrib/oberdiek/gettitlestring.dtx}] The source file.
% \item[\CTAN{macros/latex/contrib/oberdiek/gettitlestring.pdf}] Documentation.
% \end{description}
%
%
% \paragraph{Bundle.} All the packages of the bundle `oberdiek'
% are also available in a TDS compliant ZIP archive. There
% the packages are already unpacked and the documentation files
% are generated. The files and directories obey the TDS standard.
% \begin{description}
% \item[\CTANinstall{install/macros/latex/contrib/oberdiek.tds.zip}]
% \end{description}
% \emph{TDS} refers to the standard ``A Directory Structure
% for \TeX\ Files'' (\CTAN{tds/tds.pdf}). Directories
% with \xfile{texmf} in their name are usually organized this way.
%
% \subsection{Bundle installation}
%
% \paragraph{Unpacking.} Unpack the \xfile{oberdiek.tds.zip} in the
% TDS tree (also known as \xfile{texmf} tree) of your choice.
% Example (linux):
% \begin{quote}
%   |unzip oberdiek.tds.zip -d ~/texmf|
% \end{quote}
%
% \paragraph{Script installation.}
% Check the directory \xfile{TDS:scripts/oberdiek/} for
% scripts that need further installation steps.
% Package \xpackage{attachfile2} comes with the Perl script
% \xfile{pdfatfi.pl} that should be installed in such a way
% that it can be called as \texttt{pdfatfi}.
% Example (linux):
% \begin{quote}
%   |chmod +x scripts/oberdiek/pdfatfi.pl|\\
%   |cp scripts/oberdiek/pdfatfi.pl /usr/local/bin/|
% \end{quote}
%
% \subsection{Package installation}
%
% \paragraph{Unpacking.} The \xfile{.dtx} file is a self-extracting
% \docstrip\ archive. The files are extracted by running the
% \xfile{.dtx} through \plainTeX:
% \begin{quote}
%   \verb|tex gettitlestring.dtx|
% \end{quote}
%
% \paragraph{TDS.} Now the different files must be moved into
% the different directories in your installation TDS tree
% (also known as \xfile{texmf} tree):
% \begin{quote}
% \def\t{^^A
% \begin{tabular}{@{}>{\ttfamily}l@{ $\rightarrow$ }>{\ttfamily}l@{}}
%   gettitlestring.sty & tex/generic/oberdiek/gettitlestring.sty\\
%   gettitlestring.pdf & doc/latex/oberdiek/gettitlestring.pdf\\
%   test/gettitlestring-test1.tex & doc/latex/oberdiek/test/gettitlestring-test1.tex\\
%   test/gettitlestring-test2.tex & doc/latex/oberdiek/test/gettitlestring-test2.tex\\
%   gettitlestring.dtx & source/latex/oberdiek/gettitlestring.dtx\\
% \end{tabular}^^A
% }^^A
% \sbox0{\t}^^A
% \ifdim\wd0>\linewidth
%   \begingroup
%     \advance\linewidth by\leftmargin
%     \advance\linewidth by\rightmargin
%   \edef\x{\endgroup
%     \def\noexpand\lw{\the\linewidth}^^A
%   }\x
%   \def\lwbox{^^A
%     \leavevmode
%     \hbox to \linewidth{^^A
%       \kern-\leftmargin\relax
%       \hss
%       \usebox0
%       \hss
%       \kern-\rightmargin\relax
%     }^^A
%   }^^A
%   \ifdim\wd0>\lw
%     \sbox0{\small\t}^^A
%     \ifdim\wd0>\linewidth
%       \ifdim\wd0>\lw
%         \sbox0{\footnotesize\t}^^A
%         \ifdim\wd0>\linewidth
%           \ifdim\wd0>\lw
%             \sbox0{\scriptsize\t}^^A
%             \ifdim\wd0>\linewidth
%               \ifdim\wd0>\lw
%                 \sbox0{\tiny\t}^^A
%                 \ifdim\wd0>\linewidth
%                   \lwbox
%                 \else
%                   \usebox0
%                 \fi
%               \else
%                 \lwbox
%               \fi
%             \else
%               \usebox0
%             \fi
%           \else
%             \lwbox
%           \fi
%         \else
%           \usebox0
%         \fi
%       \else
%         \lwbox
%       \fi
%     \else
%       \usebox0
%     \fi
%   \else
%     \lwbox
%   \fi
% \else
%   \usebox0
% \fi
% \end{quote}
% If you have a \xfile{docstrip.cfg} that configures and enables \docstrip's
% TDS installing feature, then some files can already be in the right
% place, see the documentation of \docstrip.
%
% \subsection{Refresh file name databases}
%
% If your \TeX~distribution
% (\teTeX, \mikTeX, \dots) relies on file name databases, you must refresh
% these. For example, \teTeX\ users run \verb|texhash| or
% \verb|mktexlsr|.
%
% \subsection{Some details for the interested}
%
% \paragraph{Attached source.}
%
% The PDF documentation on CTAN also includes the
% \xfile{.dtx} source file. It can be extracted by
% AcrobatReader 6 or higher. Another option is \textsf{pdftk},
% e.g. unpack the file into the current directory:
% \begin{quote}
%   \verb|pdftk gettitlestring.pdf unpack_files output .|
% \end{quote}
%
% \paragraph{Unpacking with \LaTeX.}
% The \xfile{.dtx} chooses its action depending on the format:
% \begin{description}
% \item[\plainTeX:] Run \docstrip\ and extract the files.
% \item[\LaTeX:] Generate the documentation.
% \end{description}
% If you insist on using \LaTeX\ for \docstrip\ (really,
% \docstrip\ does not need \LaTeX), then inform the autodetect routine
% about your intention:
% \begin{quote}
%   \verb|latex \let\install=y% \iffalse meta-comment
%
% File: gettitlestring.dtx
% Version: 2016/05/16 v1.5
% Info: Cleanup title references
%
% Copyright (C) 2009, 2010 by
%    Heiko Oberdiek <heiko.oberdiek at googlemail.com>
%    2016
%    https://github.com/ho-tex/oberdiek/issues
%
% This work may be distributed and/or modified under the
% conditions of the LaTeX Project Public License, either
% version 1.3c of this license or (at your option) any later
% version. This version of this license is in
%    http://www.latex-project.org/lppl/lppl-1-3c.txt
% and the latest version of this license is in
%    http://www.latex-project.org/lppl.txt
% and version 1.3 or later is part of all distributions of
% LaTeX version 2005/12/01 or later.
%
% This work has the LPPL maintenance status "maintained".
%
% This Current Maintainer of this work is Heiko Oberdiek.
%
% The Base Interpreter refers to any `TeX-Format',
% because some files are installed in TDS:tex/generic//.
%
% This work consists of the main source file gettitlestring.dtx
% and the derived files
%    gettitlestring.sty, gettitlestring.pdf, gettitlestring.ins,
%    gettitlestring.drv, gettitlestring-test1.tex,
%    gettitlestring-test2.tex.
%
% Distribution:
%    CTAN:macros/latex/contrib/oberdiek/gettitlestring.dtx
%    CTAN:macros/latex/contrib/oberdiek/gettitlestring.pdf
%
% Unpacking:
%    (a) If gettitlestring.ins is present:
%           tex gettitlestring.ins
%    (b) Without gettitlestring.ins:
%           tex gettitlestring.dtx
%    (c) If you insist on using LaTeX
%           latex \let\install=y\input{gettitlestring.dtx}
%        (quote the arguments according to the demands of your shell)
%
% Documentation:
%    (a) If gettitlestring.drv is present:
%           latex gettitlestring.drv
%    (b) Without gettitlestring.drv:
%           latex gettitlestring.dtx; ...
%    The class ltxdoc loads the configuration file ltxdoc.cfg
%    if available. Here you can specify further options, e.g.
%    use A4 as paper format:
%       \PassOptionsToClass{a4paper}{article}
%
%    Programm calls to get the documentation (example):
%       pdflatex gettitlestring.dtx
%       makeindex -s gind.ist gettitlestring.idx
%       pdflatex gettitlestring.dtx
%       makeindex -s gind.ist gettitlestring.idx
%       pdflatex gettitlestring.dtx
%
% Installation:
%    TDS:tex/generic/oberdiek/gettitlestring.sty
%    TDS:doc/latex/oberdiek/gettitlestring.pdf
%    TDS:doc/latex/oberdiek/test/gettitlestring-test1.tex
%    TDS:doc/latex/oberdiek/test/gettitlestring-test2.tex
%    TDS:source/latex/oberdiek/gettitlestring.dtx
%
%<*ignore>
\begingroup
  \catcode123=1 %
  \catcode125=2 %
  \def\x{LaTeX2e}%
\expandafter\endgroup
\ifcase 0\ifx\install y1\fi\expandafter
         \ifx\csname processbatchFile\endcsname\relax\else1\fi
         \ifx\fmtname\x\else 1\fi\relax
\else\csname fi\endcsname
%</ignore>
%<*install>
\input docstrip.tex
\Msg{************************************************************************}
\Msg{* Installation}
\Msg{* Package: gettitlestring 2016/05/16 v1.5 Cleanup title references (HO)}
\Msg{************************************************************************}

\keepsilent
\askforoverwritefalse

\let\MetaPrefix\relax
\preamble

This is a generated file.

Project: gettitlestring
Version: 2016/05/16 v1.5

Copyright (C) 2009, 2010 by
   Heiko Oberdiek <heiko.oberdiek at googlemail.com>

This work may be distributed and/or modified under the
conditions of the LaTeX Project Public License, either
version 1.3c of this license or (at your option) any later
version. This version of this license is in
   http://www.latex-project.org/lppl/lppl-1-3c.txt
and the latest version of this license is in
   http://www.latex-project.org/lppl.txt
and version 1.3 or later is part of all distributions of
LaTeX version 2005/12/01 or later.

This work has the LPPL maintenance status "maintained".

This Current Maintainer of this work is Heiko Oberdiek.

The Base Interpreter refers to any `TeX-Format',
because some files are installed in TDS:tex/generic//.

This work consists of the main source file gettitlestring.dtx
and the derived files
   gettitlestring.sty, gettitlestring.pdf, gettitlestring.ins,
   gettitlestring.drv, gettitlestring-test1.tex,
   gettitlestring-test2.tex.

\endpreamble
\let\MetaPrefix\DoubleperCent

\generate{%
  \file{gettitlestring.ins}{\from{gettitlestring.dtx}{install}}%
  \file{gettitlestring.drv}{\from{gettitlestring.dtx}{driver}}%
  \usedir{tex/generic/oberdiek}%
  \file{gettitlestring.sty}{\from{gettitlestring.dtx}{package}}%
%  \usedir{doc/latex/oberdiek/test}%
%  \file{gettitlestring-test1.tex}{\from{gettitlestring.dtx}{test1}}%
%  \file{gettitlestring-test2.tex}{\from{gettitlestring.dtx}{test2}}%
  \nopreamble
  \nopostamble
%  \usedir{source/latex/oberdiek/catalogue}%
%  \file{gettitlestring.xml}{\from{gettitlestring.dtx}{catalogue}}%
}

\catcode32=13\relax% active space
\let =\space%
\Msg{************************************************************************}
\Msg{*}
\Msg{* To finish the installation you have to move the following}
\Msg{* file into a directory searched by TeX:}
\Msg{*}
\Msg{*     gettitlestring.sty}
\Msg{*}
\Msg{* To produce the documentation run the file `gettitlestring.drv'}
\Msg{* through LaTeX.}
\Msg{*}
\Msg{* Happy TeXing!}
\Msg{*}
\Msg{************************************************************************}

\endbatchfile
%</install>
%<*ignore>
\fi
%</ignore>
%<*driver>
\NeedsTeXFormat{LaTeX2e}
\ProvidesFile{gettitlestring.drv}%
  [2016/05/16 v1.5 Cleanup title references (HO)]%
\documentclass{ltxdoc}
\usepackage{holtxdoc}[2011/11/22]
\begin{document}
  \DocInput{gettitlestring.dtx}%
\end{document}
%</driver>
% \fi
%
%
% \CharacterTable
%  {Upper-case    \A\B\C\D\E\F\G\H\I\J\K\L\M\N\O\P\Q\R\S\T\U\V\W\X\Y\Z
%   Lower-case    \a\b\c\d\e\f\g\h\i\j\k\l\m\n\o\p\q\r\s\t\u\v\w\x\y\z
%   Digits        \0\1\2\3\4\5\6\7\8\9
%   Exclamation   \!     Double quote  \"     Hash (number) \#
%   Dollar        \$     Percent       \%     Ampersand     \&
%   Acute accent  \'     Left paren    \(     Right paren   \)
%   Asterisk      \*     Plus          \+     Comma         \,
%   Minus         \-     Point         \.     Solidus       \/
%   Colon         \:     Semicolon     \;     Less than     \<
%   Equals        \=     Greater than  \>     Question mark \?
%   Commercial at \@     Left bracket  \[     Backslash     \\
%   Right bracket \]     Circumflex    \^     Underscore    \_
%   Grave accent  \`     Left brace    \{     Vertical bar  \|
%   Right brace   \}     Tilde         \~}
%
% \GetFileInfo{gettitlestring.drv}
%
% \title{The \xpackage{gettitlestring} package}
% \date{2016/05/16 v1.5}
% \author{Heiko Oberdiek\thanks
% {Please report any issues at \url{https://github.com/ho-tex/oberdiek/issues}}\\
% \xemail{heiko.oberdiek at googlemail.com}}
%
% \maketitle
%
% \begin{abstract}
% The \LaTeX\ package addresses packages that are dealing with
% references to titles (\cs{section}, \cs{caption}, \dots).
% The package tries to remove \cs{label} and other
% commands from title strings.
% \end{abstract}
%
% \tableofcontents
%
% \section{Documentation}
%
% \subsection{Macros}
%
% \begin{declcs}{GetTitleStringSetup} \M{key value list}
% \end{declcs}
% The options are given as comma separated key value pairs.
% See section \ref{sec:options}.
%
% \begin{declcs}{GetTitleString} \M{text}\\
% \cs{GetTitleStringExpand} \M{text}\\
% \cs{GetTitleStringNonExpand} \M{text}
% \end{declcs}
% Macro \cs{GetTitleString} tries to remove unwanted stuff from \meta{text}
% the result is stored in Macro \cs{GetTitleStringResult}.
% Two methods are available:
% \begin{description}
% \item[\cs{GetTitleStringExpand}:]
% The \meta{text} is expanded in a context where the unwanted
% macros are redefined to remove themselves.
% This is the method used in packages \xpackage{titleref}~\cite{titleref},
% \xpackage{zref-titleref}~\cite{zref}
% or class \xclass{memoir}~\cite{memoir}.
% \cs{protect} is supported, but fragile material might break.
% \item[\cs{GetTitleStringNonExpand}:]
% The \meta{text} is not expanded. Thus the removal of unwanted
% material is more difficult. It is especially removed at the
% start of the \meta{text} and spaces are removed from the end.
% Currently only \cs{label} is removed in the whole string,
% if it is not hidden inside curly braces or part of macro
% definitions. Thus the removal of unwanted stuff might not be
% complete, but fragile material will not break.
% (But the result string can break at a later time, of course).
% \end{description}
% Option \xoption{expand} controls which method is used by
% macro \cs{GetTitleString}.
%
% \begin{declcs}{GetTitleStringDisableCommands} \M{code}
% \end{declcs}
% The \meta{code} is called right before the
% text is expanded in \cs{GetTitleStringExpand}.
% Additional definitions can be given for macros that
% should be removed.
% Keep in mind that expansion means that the definitions
% must work in expandable context. Macros like
% \cs{@ifstar} or \cs{@ifnextchar} or optional arguments
% will not work. The macro names in \meta{code} may contain
% the at sign |@|, it has catcode 11 (letter).
%
% \subsection{Options}\label{sec:options}
%
% \begin{description}
% \item[\xoption{expand}:] Boolean option, takes values |true| or |false|.
% No value means |true|. The option specifies the method to remove
% unwanted stuff from the title string, see below.
% \end{description}
% Options can be set at the following places:
% \begin{itemize}
% \item \cs{usepackage}
% \item Configuration file \xfile{gettitlestring.cfg}.
% \item \cs{GetTitleStringSetup}
% \end{itemize}
%
% \StopEventually{
% }
%
% \section{Implementation}
%    \begin{macrocode}
%<*package>
%    \end{macrocode}
%    Reload check, especially if the package is not used with \LaTeX.
%    \begin{macrocode}
\begingroup\catcode61\catcode48\catcode32=10\relax%
  \catcode13=5 % ^^M
  \endlinechar=13 %
  \catcode35=6 % #
  \catcode39=12 % '
  \catcode44=12 % ,
  \catcode45=12 % -
  \catcode46=12 % .
  \catcode58=12 % :
  \catcode64=11 % @
  \catcode123=1 % {
  \catcode125=2 % }
  \expandafter\let\expandafter\x\csname ver@gettitlestring.sty\endcsname
  \ifx\x\relax % plain-TeX, first loading
  \else
    \def\empty{}%
    \ifx\x\empty % LaTeX, first loading,
      % variable is initialized, but \ProvidesPackage not yet seen
    \else
      \expandafter\ifx\csname PackageInfo\endcsname\relax
        \def\x#1#2{%
          \immediate\write-1{Package #1 Info: #2.}%
        }%
      \else
        \def\x#1#2{\PackageInfo{#1}{#2, stopped}}%
      \fi
      \x{gettitlestring}{The package is already loaded}%
      \aftergroup\endinput
    \fi
  \fi
\endgroup%
%    \end{macrocode}
%    Package identification:
%    \begin{macrocode}
\begingroup\catcode61\catcode48\catcode32=10\relax%
  \catcode13=5 % ^^M
  \endlinechar=13 %
  \catcode35=6 % #
  \catcode39=12 % '
  \catcode40=12 % (
  \catcode41=12 % )
  \catcode44=12 % ,
  \catcode45=12 % -
  \catcode46=12 % .
  \catcode47=12 % /
  \catcode58=12 % :
  \catcode64=11 % @
  \catcode91=12 % [
  \catcode93=12 % ]
  \catcode123=1 % {
  \catcode125=2 % }
  \expandafter\ifx\csname ProvidesPackage\endcsname\relax
    \def\x#1#2#3[#4]{\endgroup
      \immediate\write-1{Package: #3 #4}%
      \xdef#1{#4}%
    }%
  \else
    \def\x#1#2[#3]{\endgroup
      #2[{#3}]%
      \ifx#1\@undefined
        \xdef#1{#3}%
      \fi
      \ifx#1\relax
        \xdef#1{#3}%
      \fi
    }%
  \fi
\expandafter\x\csname ver@gettitlestring.sty\endcsname
\ProvidesPackage{gettitlestring}%
  [2016/05/16 v1.5 Cleanup title references (HO)]%
%    \end{macrocode}
%
%    \begin{macrocode}
\begingroup\catcode61\catcode48\catcode32=10\relax%
  \catcode13=5 % ^^M
  \endlinechar=13 %
  \catcode123=1 % {
  \catcode125=2 % }
  \catcode64=11 % @
  \def\x{\endgroup
    \expandafter\edef\csname GTS@AtEnd\endcsname{%
      \endlinechar=\the\endlinechar\relax
      \catcode13=\the\catcode13\relax
      \catcode32=\the\catcode32\relax
      \catcode35=\the\catcode35\relax
      \catcode61=\the\catcode61\relax
      \catcode64=\the\catcode64\relax
      \catcode123=\the\catcode123\relax
      \catcode125=\the\catcode125\relax
    }%
  }%
\x\catcode61\catcode48\catcode32=10\relax%
\catcode13=5 % ^^M
\endlinechar=13 %
\catcode35=6 % #
\catcode64=11 % @
\catcode123=1 % {
\catcode125=2 % }
\def\TMP@EnsureCode#1#2{%
  \edef\GTS@AtEnd{%
    \GTS@AtEnd
    \catcode#1=\the\catcode#1\relax
  }%
  \catcode#1=#2\relax
}
\TMP@EnsureCode{42}{12}% *
\TMP@EnsureCode{44}{12}% ,
\TMP@EnsureCode{45}{12}% -
\TMP@EnsureCode{46}{12}% .
\TMP@EnsureCode{47}{12}% /
\TMP@EnsureCode{91}{12}% [
\TMP@EnsureCode{93}{12}% ]
\edef\GTS@AtEnd{\GTS@AtEnd\noexpand\endinput}
%    \end{macrocode}
%
% \subsection{Options}
%
%    \begin{macrocode}
\RequirePackage{kvoptions}[2009/07/17]
\SetupKeyvalOptions{%
  family=gettitlestring,%
  prefix=GTS@%
}
\newcommand*{\GetTitleStringSetup}{%
  \setkeys{gettitlestring}%
}
\DeclareBoolOption{expand}
\InputIfFileExists{gettitlestring.cfg}{}{}
\ProcessKeyvalOptions*\relax
%    \end{macrocode}
%
% \subsection{\cs{GetTitleString}}
%
%    \begin{macro}{\GetTitleString}
%    \begin{macrocode}
\newcommand*{\GetTitleString}{%
  \ifGTS@expand
    \expandafter\GetTitleStringExpand
  \else
    \expandafter\GetTitleStringNonExpand
  \fi
}
%    \end{macrocode}
%    \end{macro}
%    \begin{macro}{\GetTitleStringExpand}
%    \begin{macrocode}
\newcommand{\GetTitleStringExpand}[1]{%
  \def\GetTitleStringResult{#1}%
  \begingroup
    \GTS@DisablePredefinedCmds
    \GTS@DisableHook
    \edef\x{\endgroup
      \noexpand\def\noexpand\GetTitleStringResult{%
        \GetTitleStringResult
      }%
    }%
  \x
}
%    \end{macrocode}
%    \end{macro}
%    \begin{macro}{\GetTitleString}
%    \begin{macrocode}
\newcommand{\GetTitleStringNonExpand}[1]{%
  \def\GetTitleStringResult{#1}%
  \global\let\GTS@GlobalString\GetTitleStringResult
  \begingroup
    \GTS@RemoveLeft
    \GTS@RemoveRight
  \endgroup
  \let\GetTitleStringResult\GTS@GlobalString
}
%    \end{macrocode}
%    \end{macro}
%
% \subsubsection{Expand method}
%
%    \begin{macro}{\GTS@DisablePredefinedCmds}
%    \begin{macrocode}
\def\GTS@DisablePredefinedCmds{%
  \let\label\@gobble
  \let\zlabel\@gobble
  \let\zref@label\@gobble
  \let\zref@labelbylist\@gobbletwo
  \let\zref@labelbyprops\@gobbletwo
  \let\index\@gobble
  \let\glossary\@gobble
  \let\markboth\@gobbletwo
  \let\@mkboth\@gobbletwo
  \let\markright\@gobble
  \let\phantomsection\@empty
  \def\addcontentsline{\expandafter\@gobble\@gobbletwo}%
  \let\raggedright\@empty
  \let\raggedleft\@empty
  \let\centering\@empty
  \let\protect\@unexpandable@protect
  \let\enit@format\@empty % package enumitem
}
%    \end{macrocode}
%    \end{macro}
%
%    \begin{macro}{\GTS@DisableHook}
%    \begin{macrocode}
\providecommand*{\GTS@DisableHook}{}
%    \end{macrocode}
%    \end{macro}
%    \begin{macro}{\GetTitleStringDisableCommands}
%    \begin{macrocode}
\def\GetTitleStringDisableCommands{%
  \begingroup
    \makeatletter
    \GTS@DisableCommands
}
%    \end{macrocode}
%    \end{macro}
%    \begin{macro}{\GTS@DisableCommands}
%    \begin{macrocode}
\long\def\GTS@DisableCommands#1{%
    \toks0=\expandafter{\GTS@DisableHook}%
    \toks2={#1}%
    \xdef\GTS@GlobalString{\the\toks0 \the\toks2}%
  \endgroup
  \let\GTS@DisableHook\GTS@GlobalString
}
%    \end{macrocode}
%    \end{macro}
%
% \subsubsection{Non-expand method}
%
%    \begin{macrocode}
\def\GTS@RemoveLeft{%
  \toks@\expandafter\expandafter\expandafter{%
    \expandafter\GTS@Car\GTS@GlobalString{}{}{}{}\GTS@Nil
  }%
  \edef\GTS@Token{\the\toks@}%
  \GTS@PredefinedLeftCmds
  \expandafter\futurelet\expandafter\GTS@Token
  \expandafter\GTS@TestLeftSpace\GTS@GlobalString\GTS@Nil
  \GTS@End
}
\def\GTS@End{}
\long\def\GTS@TestLeft#1#2{%
  \def\GTS@temp{#1}%
  \ifx\GTS@temp\GTS@Token
    \toks@\expandafter\expandafter\expandafter{%
      \expandafter#2\GTS@GlobalString\GTS@Nil
    }%
    \expandafter\GTS@TestLeftEnd
  \fi
}
\long\def\GTS@TestLeftEnd#1\GTS@End{%
  \xdef\GTS@GlobalString{\the\toks@}%
  \GTS@RemoveLeft
}
\long\def\GTS@Car#1#2\GTS@Nil{#1}
\long\def\GTS@Cdr#1#2\GTS@Nil{#2}
\long\def\GTS@CdrTwo#1#2#3\GTS@Nil{#3}
\long\def\GTS@CdrThree#1#2#3#4\GTS@Nil{#4}
\long\def\GTS@CdrFour#1#2#3#4#5\GTS@Nil{#5}
\long\def\GTS@TestLeftSpace#1\GTS@Nil{%
  \ifx\GTS@Token\@sptoken
    \toks@\expandafter{%
      \romannumeral-0\GTS@GlobalString
    }%
    \expandafter\GTS@TestLeftEnd
  \fi
}
%    \end{macrocode}
%    \begin{macro}{\GTS@PredefinedLeftCmds}
%    \begin{macrocode}
\def\GTS@PredefinedLeftCmds{%
  \GTS@TestLeft\Hy@phantomsection\GTS@Cdr
  \GTS@TestLeft\Hy@SectionAnchor\GTS@Cdr
  \GTS@TestLeft\Hy@SectionAnchorHref\GTS@CdrTwo
  \GTS@TestLeft\label\GTS@CdrTwo
  \GTS@TestLeft\zlabel\GTS@CdrTwo
  \GTS@TestLeft\index\GTS@CdrTwo
  \GTS@TestLeft\glossary\GTS@CdrTwo
  \GTS@TestLeft\markboth\GTS@CdrThree
  \GTS@TestLeft\@mkboth\GTS@CdrThree
  \GTS@TestLeft\addcontentsline\GTS@CdrFour
  \GTS@TestLeft\enit@format\GTS@Cdr % package enumitem
}
%    \end{macrocode}
%    \end{macro}
%
%    \begin{macrocode}
\def\GTS@RemoveRight{%
  \toks@{}%
  \expandafter\GTS@TestRightLabel\GTS@GlobalString
      \label{}\GTS@Nil\@nil
  \GTS@RemoveRightSpace
}
\begingroup
  \def\GTS@temp#1{\endgroup
    \def\GTS@RemoveRightSpace{%
      \expandafter\GTS@TestRightSpace\GTS@GlobalString
          \GTS@Nil#1\GTS@Nil\@nil
    }%
  }%
\GTS@temp{ }
\def\GTS@TestRightSpace#1 \GTS@Nil#2\@nil{%
  \ifx\relax#2\relax
  \else
    \gdef\GTS@GlobalString{#1}%
    \expandafter\GTS@RemoveRightSpace
  \fi
}
\def\GTS@TestRightLabel#1\label#2#3\GTS@Nil#4\@nil{%
  \def\GTS@temp{#3}%
  \ifx\GTS@temp\@empty
    \expandafter\gdef\expandafter\GTS@GlobalString\expandafter{%
      \the\toks@
      #1%
    }%
    \expandafter\@gobble
  \else
    \expandafter\@firstofone
  \fi
  {%
    \toks@\expandafter{\the\toks@#1}%
    \GTS@TestRightLabel#3\GTS@Nil\@nil
  }%
}
%    \end{macrocode}
%
%    \begin{macrocode}
\GTS@AtEnd%
%</package>
%    \end{macrocode}
%
% \section{Test}
%
% \subsection{Catcode checks for loading}
%
%    \begin{macrocode}
%<*test1>
%    \end{macrocode}
%    \begin{macrocode}
\catcode`\{=1 %
\catcode`\}=2 %
\catcode`\#=6 %
\catcode`\@=11 %
\expandafter\ifx\csname count@\endcsname\relax
  \countdef\count@=255 %
\fi
\expandafter\ifx\csname @gobble\endcsname\relax
  \long\def\@gobble#1{}%
\fi
\expandafter\ifx\csname @firstofone\endcsname\relax
  \long\def\@firstofone#1{#1}%
\fi
\expandafter\ifx\csname loop\endcsname\relax
  \expandafter\@firstofone
\else
  \expandafter\@gobble
\fi
{%
  \def\loop#1\repeat{%
    \def\body{#1}%
    \iterate
  }%
  \def\iterate{%
    \body
      \let\next\iterate
    \else
      \let\next\relax
    \fi
    \next
  }%
  \let\repeat=\fi
}%
\def\RestoreCatcodes{}
\count@=0 %
\loop
  \edef\RestoreCatcodes{%
    \RestoreCatcodes
    \catcode\the\count@=\the\catcode\count@\relax
  }%
\ifnum\count@<255 %
  \advance\count@ 1 %
\repeat

\def\RangeCatcodeInvalid#1#2{%
  \count@=#1\relax
  \loop
    \catcode\count@=15 %
  \ifnum\count@<#2\relax
    \advance\count@ 1 %
  \repeat
}
\def\RangeCatcodeCheck#1#2#3{%
  \count@=#1\relax
  \loop
    \ifnum#3=\catcode\count@
    \else
      \errmessage{%
        Character \the\count@\space
        with wrong catcode \the\catcode\count@\space
        instead of \number#3%
      }%
    \fi
  \ifnum\count@<#2\relax
    \advance\count@ 1 %
  \repeat
}
\def\space{ }
\expandafter\ifx\csname LoadCommand\endcsname\relax
  \def\LoadCommand{\input gettitlestring.sty\relax}%
\fi
\def\Test{%
  \RangeCatcodeInvalid{0}{47}%
  \RangeCatcodeInvalid{58}{64}%
  \RangeCatcodeInvalid{91}{96}%
  \RangeCatcodeInvalid{123}{255}%
  \catcode`\@=12 %
  \catcode`\\=0 %
  \catcode`\%=14 %
  \LoadCommand
  \RangeCatcodeCheck{0}{36}{15}%
  \RangeCatcodeCheck{37}{37}{14}%
  \RangeCatcodeCheck{38}{47}{15}%
  \RangeCatcodeCheck{48}{57}{12}%
  \RangeCatcodeCheck{58}{63}{15}%
  \RangeCatcodeCheck{64}{64}{12}%
  \RangeCatcodeCheck{65}{90}{11}%
  \RangeCatcodeCheck{91}{91}{15}%
  \RangeCatcodeCheck{92}{92}{0}%
  \RangeCatcodeCheck{93}{96}{15}%
  \RangeCatcodeCheck{97}{122}{11}%
  \RangeCatcodeCheck{123}{255}{15}%
  \RestoreCatcodes
}
\Test
\csname @@end\endcsname
\end
%    \end{macrocode}
%    \begin{macrocode}
%</test1>
%    \end{macrocode}
%
% \subsection{Test of non-expand method}
%
%    \begin{macrocode}
%<*test2>
\NeedsTeXFormat{LaTeX2e}
\documentclass{minimal}
\usepackage{gettitlestring}[2016/05/16]
\usepackage{qstest}
\IncludeTests{*}
\LogTests{log}{*}{*}
\begin{document}
\begin{qstest}{non-expand}{non-expand}
  \def\test#1#2{%
    \sbox0{%
      \GetTitleString{#1}%
      \Expect{#2}*{\GetTitleStringResult}%
    }%
    \Expect{0.0pt}*{\the\wd0}%
  }%
  \test{}{}%
  \test{ }{}%
  \test{ x }{x}%
  \test{ x y }{x y}%
  \test{ \relax}{\relax}%
  \test{\label{f}a}{a}%
  \test{ \label{f}a}{a}%
  \test{\label{f} a}{a}%
  \test{ \label{f} a}{a}%
  \test{a\label{f}}{a}%
  \test{a\label{f} }{a}%
  \test{a \label{f}}{a}%
  \test{a \label{f} }{a}%
  \test{a\label{f}b\label{g}}{ab}%
  \test{a \label{f}b \label{g} }{a b}%
  \test{a\label{f} b \label{g} }{a b}%
\end{qstest}
\end{document}
%</test2>
%    \end{macrocode}
%
% \section{Installation}
%
% \subsection{Download}
%
% \paragraph{Package.} This package is available on
% CTAN\footnote{\CTANpkg{gettitlestring}}:
% \begin{description}
% \item[\CTAN{macros/latex/contrib/oberdiek/gettitlestring.dtx}] The source file.
% \item[\CTAN{macros/latex/contrib/oberdiek/gettitlestring.pdf}] Documentation.
% \end{description}
%
%
% \paragraph{Bundle.} All the packages of the bundle `oberdiek'
% are also available in a TDS compliant ZIP archive. There
% the packages are already unpacked and the documentation files
% are generated. The files and directories obey the TDS standard.
% \begin{description}
% \item[\CTANinstall{install/macros/latex/contrib/oberdiek.tds.zip}]
% \end{description}
% \emph{TDS} refers to the standard ``A Directory Structure
% for \TeX\ Files'' (\CTAN{tds/tds.pdf}). Directories
% with \xfile{texmf} in their name are usually organized this way.
%
% \subsection{Bundle installation}
%
% \paragraph{Unpacking.} Unpack the \xfile{oberdiek.tds.zip} in the
% TDS tree (also known as \xfile{texmf} tree) of your choice.
% Example (linux):
% \begin{quote}
%   |unzip oberdiek.tds.zip -d ~/texmf|
% \end{quote}
%
% \paragraph{Script installation.}
% Check the directory \xfile{TDS:scripts/oberdiek/} for
% scripts that need further installation steps.
% Package \xpackage{attachfile2} comes with the Perl script
% \xfile{pdfatfi.pl} that should be installed in such a way
% that it can be called as \texttt{pdfatfi}.
% Example (linux):
% \begin{quote}
%   |chmod +x scripts/oberdiek/pdfatfi.pl|\\
%   |cp scripts/oberdiek/pdfatfi.pl /usr/local/bin/|
% \end{quote}
%
% \subsection{Package installation}
%
% \paragraph{Unpacking.} The \xfile{.dtx} file is a self-extracting
% \docstrip\ archive. The files are extracted by running the
% \xfile{.dtx} through \plainTeX:
% \begin{quote}
%   \verb|tex gettitlestring.dtx|
% \end{quote}
%
% \paragraph{TDS.} Now the different files must be moved into
% the different directories in your installation TDS tree
% (also known as \xfile{texmf} tree):
% \begin{quote}
% \def\t{^^A
% \begin{tabular}{@{}>{\ttfamily}l@{ $\rightarrow$ }>{\ttfamily}l@{}}
%   gettitlestring.sty & tex/generic/oberdiek/gettitlestring.sty\\
%   gettitlestring.pdf & doc/latex/oberdiek/gettitlestring.pdf\\
%   test/gettitlestring-test1.tex & doc/latex/oberdiek/test/gettitlestring-test1.tex\\
%   test/gettitlestring-test2.tex & doc/latex/oberdiek/test/gettitlestring-test2.tex\\
%   gettitlestring.dtx & source/latex/oberdiek/gettitlestring.dtx\\
% \end{tabular}^^A
% }^^A
% \sbox0{\t}^^A
% \ifdim\wd0>\linewidth
%   \begingroup
%     \advance\linewidth by\leftmargin
%     \advance\linewidth by\rightmargin
%   \edef\x{\endgroup
%     \def\noexpand\lw{\the\linewidth}^^A
%   }\x
%   \def\lwbox{^^A
%     \leavevmode
%     \hbox to \linewidth{^^A
%       \kern-\leftmargin\relax
%       \hss
%       \usebox0
%       \hss
%       \kern-\rightmargin\relax
%     }^^A
%   }^^A
%   \ifdim\wd0>\lw
%     \sbox0{\small\t}^^A
%     \ifdim\wd0>\linewidth
%       \ifdim\wd0>\lw
%         \sbox0{\footnotesize\t}^^A
%         \ifdim\wd0>\linewidth
%           \ifdim\wd0>\lw
%             \sbox0{\scriptsize\t}^^A
%             \ifdim\wd0>\linewidth
%               \ifdim\wd0>\lw
%                 \sbox0{\tiny\t}^^A
%                 \ifdim\wd0>\linewidth
%                   \lwbox
%                 \else
%                   \usebox0
%                 \fi
%               \else
%                 \lwbox
%               \fi
%             \else
%               \usebox0
%             \fi
%           \else
%             \lwbox
%           \fi
%         \else
%           \usebox0
%         \fi
%       \else
%         \lwbox
%       \fi
%     \else
%       \usebox0
%     \fi
%   \else
%     \lwbox
%   \fi
% \else
%   \usebox0
% \fi
% \end{quote}
% If you have a \xfile{docstrip.cfg} that configures and enables \docstrip's
% TDS installing feature, then some files can already be in the right
% place, see the documentation of \docstrip.
%
% \subsection{Refresh file name databases}
%
% If your \TeX~distribution
% (\teTeX, \mikTeX, \dots) relies on file name databases, you must refresh
% these. For example, \teTeX\ users run \verb|texhash| or
% \verb|mktexlsr|.
%
% \subsection{Some details for the interested}
%
% \paragraph{Attached source.}
%
% The PDF documentation on CTAN also includes the
% \xfile{.dtx} source file. It can be extracted by
% AcrobatReader 6 or higher. Another option is \textsf{pdftk},
% e.g. unpack the file into the current directory:
% \begin{quote}
%   \verb|pdftk gettitlestring.pdf unpack_files output .|
% \end{quote}
%
% \paragraph{Unpacking with \LaTeX.}
% The \xfile{.dtx} chooses its action depending on the format:
% \begin{description}
% \item[\plainTeX:] Run \docstrip\ and extract the files.
% \item[\LaTeX:] Generate the documentation.
% \end{description}
% If you insist on using \LaTeX\ for \docstrip\ (really,
% \docstrip\ does not need \LaTeX), then inform the autodetect routine
% about your intention:
% \begin{quote}
%   \verb|latex \let\install=y\input{gettitlestring.dtx}|
% \end{quote}
% Do not forget to quote the argument according to the demands
% of your shell.
%
% \paragraph{Generating the documentation.}
% You can use both the \xfile{.dtx} or the \xfile{.drv} to generate
% the documentation. The process can be configured by the
% configuration file \xfile{ltxdoc.cfg}. For instance, put this
% line into this file, if you want to have A4 as paper format:
% \begin{quote}
%   \verb|\PassOptionsToClass{a4paper}{article}|
% \end{quote}
% An example follows how to generate the
% documentation with pdf\LaTeX:
% \begin{quote}
%\begin{verbatim}
%pdflatex gettitlestring.dtx
%makeindex -s gind.ist gettitlestring.idx
%pdflatex gettitlestring.dtx
%makeindex -s gind.ist gettitlestring.idx
%pdflatex gettitlestring.dtx
%\end{verbatim}
% \end{quote}
%
% \begin{thebibliography}{9}
%
% \bibitem{memoir}
% Peter Wilson, Lars Madsen:
% \textit{The Memoir Class};
% 2009/11/17 v1.61803398c;
% \CTANpkg{memoir}
%
% \bibitem{titleref}
% Donald Arsenau:
% \textit{Titleref.sty};
% 2001/04/05 ver 3.1;
% \CTAN{macros/latex/contrib/misc/titleref.sty}
%
% \bibitem{zref}
% Heiko Oberdiek:
% \textit{The \xpackage{zref} package};
% 2009/12/08 v2.7;
% \CTAN{macros/latex/contrib/oberdiek/zref.pdf}
%
% \end{thebibliography}
%
% \begin{History}
%   \begin{Version}{2009/12/08 v1.0}
%   \item
%     The first version.
%   \end{Version}
%   \begin{Version}{2009/12/12 v1.1}
%   \item
%     Short info shortened.
%   \end{Version}
%   \begin{Version}{2009/12/13 v1.2}
%   \item
%     Forgotten third argument for \cs{InputIfFileExists} added.
%   \end{Version}
%   \begin{Version}{2009/12/18 v1.3}
%   \item
%     \cs{Hy@SectionAnchorHref} added for filtering
%     (hyperref 2009/12/18 v6.79w).
%   \end{Version}
%   \begin{Version}{2010/12/03 v1.4}
%   \item
%     Support of package \xpackage{enumitem}: removing
%     \cs{enit@format} from title string (problem report by GL).
%   \end{Version}
%   \begin{Version}{2016/05/16 v1.5}
%   \item
%     Documentation updates.
%   \end{Version}
% \end{History}
%
% \PrintIndex
%
% \Finale
\endinput
|
% \end{quote}
% Do not forget to quote the argument according to the demands
% of your shell.
%
% \paragraph{Generating the documentation.}
% You can use both the \xfile{.dtx} or the \xfile{.drv} to generate
% the documentation. The process can be configured by the
% configuration file \xfile{ltxdoc.cfg}. For instance, put this
% line into this file, if you want to have A4 as paper format:
% \begin{quote}
%   \verb|\PassOptionsToClass{a4paper}{article}|
% \end{quote}
% An example follows how to generate the
% documentation with pdf\LaTeX:
% \begin{quote}
%\begin{verbatim}
%pdflatex gettitlestring.dtx
%makeindex -s gind.ist gettitlestring.idx
%pdflatex gettitlestring.dtx
%makeindex -s gind.ist gettitlestring.idx
%pdflatex gettitlestring.dtx
%\end{verbatim}
% \end{quote}
%
% \begin{thebibliography}{9}
%
% \bibitem{memoir}
% Peter Wilson, Lars Madsen:
% \textit{The Memoir Class};
% 2009/11/17 v1.61803398c;
% \CTANpkg{memoir}
%
% \bibitem{titleref}
% Donald Arsenau:
% \textit{Titleref.sty};
% 2001/04/05 ver 3.1;
% \CTAN{macros/latex/contrib/misc/titleref.sty}
%
% \bibitem{zref}
% Heiko Oberdiek:
% \textit{The \xpackage{zref} package};
% 2009/12/08 v2.7;
% \CTAN{macros/latex/contrib/oberdiek/zref.pdf}
%
% \end{thebibliography}
%
% \begin{History}
%   \begin{Version}{2009/12/08 v1.0}
%   \item
%     The first version.
%   \end{Version}
%   \begin{Version}{2009/12/12 v1.1}
%   \item
%     Short info shortened.
%   \end{Version}
%   \begin{Version}{2009/12/13 v1.2}
%   \item
%     Forgotten third argument for \cs{InputIfFileExists} added.
%   \end{Version}
%   \begin{Version}{2009/12/18 v1.3}
%   \item
%     \cs{Hy@SectionAnchorHref} added for filtering
%     (hyperref 2009/12/18 v6.79w).
%   \end{Version}
%   \begin{Version}{2010/12/03 v1.4}
%   \item
%     Support of package \xpackage{enumitem}: removing
%     \cs{enit@format} from title string (problem report by GL).
%   \end{Version}
%   \begin{Version}{2016/05/16 v1.5}
%   \item
%     Documentation updates.
%   \end{Version}
% \end{History}
%
% \PrintIndex
%
% \Finale
\endinput
|
% \end{quote}
% Do not forget to quote the argument according to the demands
% of your shell.
%
% \paragraph{Generating the documentation.}
% You can use both the \xfile{.dtx} or the \xfile{.drv} to generate
% the documentation. The process can be configured by the
% configuration file \xfile{ltxdoc.cfg}. For instance, put this
% line into this file, if you want to have A4 as paper format:
% \begin{quote}
%   \verb|\PassOptionsToClass{a4paper}{article}|
% \end{quote}
% An example follows how to generate the
% documentation with pdf\LaTeX:
% \begin{quote}
%\begin{verbatim}
%pdflatex gettitlestring.dtx
%makeindex -s gind.ist gettitlestring.idx
%pdflatex gettitlestring.dtx
%makeindex -s gind.ist gettitlestring.idx
%pdflatex gettitlestring.dtx
%\end{verbatim}
% \end{quote}
%
% \begin{thebibliography}{9}
%
% \bibitem{memoir}
% Peter Wilson, Lars Madsen:
% \textit{The Memoir Class};
% 2009/11/17 v1.61803398c;
% \CTANpkg{memoir}
%
% \bibitem{titleref}
% Donald Arsenau:
% \textit{Titleref.sty};
% 2001/04/05 ver 3.1;
% \CTAN{macros/latex/contrib/misc/titleref.sty}
%
% \bibitem{zref}
% Heiko Oberdiek:
% \textit{The \xpackage{zref} package};
% 2009/12/08 v2.7;
% \CTAN{macros/latex/contrib/oberdiek/zref.pdf}
%
% \end{thebibliography}
%
% \begin{History}
%   \begin{Version}{2009/12/08 v1.0}
%   \item
%     The first version.
%   \end{Version}
%   \begin{Version}{2009/12/12 v1.1}
%   \item
%     Short info shortened.
%   \end{Version}
%   \begin{Version}{2009/12/13 v1.2}
%   \item
%     Forgotten third argument for \cs{InputIfFileExists} added.
%   \end{Version}
%   \begin{Version}{2009/12/18 v1.3}
%   \item
%     \cs{Hy@SectionAnchorHref} added for filtering
%     (hyperref 2009/12/18 v6.79w).
%   \end{Version}
%   \begin{Version}{2010/12/03 v1.4}
%   \item
%     Support of package \xpackage{enumitem}: removing
%     \cs{enit@format} from title string (problem report by GL).
%   \end{Version}
%   \begin{Version}{2016/05/16 v1.5}
%   \item
%     Documentation updates.
%   \end{Version}
% \end{History}
%
% \PrintIndex
%
% \Finale
\endinput

%        (quote the arguments according to the demands of your shell)
%
% Documentation:
%    (a) If gettitlestring.drv is present:
%           latex gettitlestring.drv
%    (b) Without gettitlestring.drv:
%           latex gettitlestring.dtx; ...
%    The class ltxdoc loads the configuration file ltxdoc.cfg
%    if available. Here you can specify further options, e.g.
%    use A4 as paper format:
%       \PassOptionsToClass{a4paper}{article}
%
%    Programm calls to get the documentation (example):
%       pdflatex gettitlestring.dtx
%       makeindex -s gind.ist gettitlestring.idx
%       pdflatex gettitlestring.dtx
%       makeindex -s gind.ist gettitlestring.idx
%       pdflatex gettitlestring.dtx
%
% Installation:
%    TDS:tex/generic/oberdiek/gettitlestring.sty
%    TDS:doc/latex/oberdiek/gettitlestring.pdf
%    TDS:doc/latex/oberdiek/test/gettitlestring-test1.tex
%    TDS:doc/latex/oberdiek/test/gettitlestring-test2.tex
%    TDS:source/latex/oberdiek/gettitlestring.dtx
%
%<*ignore>
\begingroup
  \catcode123=1 %
  \catcode125=2 %
  \def\x{LaTeX2e}%
\expandafter\endgroup
\ifcase 0\ifx\install y1\fi\expandafter
         \ifx\csname processbatchFile\endcsname\relax\else1\fi
         \ifx\fmtname\x\else 1\fi\relax
\else\csname fi\endcsname
%</ignore>
%<*install>
\input docstrip.tex
\Msg{************************************************************************}
\Msg{* Installation}
\Msg{* Package: gettitlestring 2016/05/16 v1.5 Cleanup title references (HO)}
\Msg{************************************************************************}

\keepsilent
\askforoverwritefalse

\let\MetaPrefix\relax
\preamble

This is a generated file.

Project: gettitlestring
Version: 2016/05/16 v1.5

Copyright (C) 2009, 2010 by
   Heiko Oberdiek <heiko.oberdiek at googlemail.com>

This work may be distributed and/or modified under the
conditions of the LaTeX Project Public License, either
version 1.3c of this license or (at your option) any later
version. This version of this license is in
   https://www.latex-project.org/lppl/lppl-1-3c.txt
and the latest version of this license is in
   https://www.latex-project.org/lppl.txt
and version 1.3 or later is part of all distributions of
LaTeX version 2005/12/01 or later.

This work has the LPPL maintenance status "maintained".

The Current Maintainers of this work are
Heiko Oberdiek and the Oberdiek Package Support Group
https://github.com/ho-tex/oberdiek/issues


The Base Interpreter refers to any `TeX-Format',
because some files are installed in TDS:tex/generic//.

This work consists of the main source file gettitlestring.dtx
and the derived files
   gettitlestring.sty, gettitlestring.pdf, gettitlestring.ins,
   gettitlestring.drv, gettitlestring-test1.tex,
   gettitlestring-test2.tex.

\endpreamble
\let\MetaPrefix\DoubleperCent

\generate{%
  \file{gettitlestring.ins}{\from{gettitlestring.dtx}{install}}%
  \file{gettitlestring.drv}{\from{gettitlestring.dtx}{driver}}%
  \usedir{tex/generic/oberdiek}%
  \file{gettitlestring.sty}{\from{gettitlestring.dtx}{package}}%
%  \usedir{doc/latex/oberdiek/test}%
%  \file{gettitlestring-test1.tex}{\from{gettitlestring.dtx}{test1}}%
%  \file{gettitlestring-test2.tex}{\from{gettitlestring.dtx}{test2}}%
  \nopreamble
  \nopostamble
%  \usedir{source/latex/oberdiek/catalogue}%
%  \file{gettitlestring.xml}{\from{gettitlestring.dtx}{catalogue}}%
}

\catcode32=13\relax% active space
\let =\space%
\Msg{************************************************************************}
\Msg{*}
\Msg{* To finish the installation you have to move the following}
\Msg{* file into a directory searched by TeX:}
\Msg{*}
\Msg{*     gettitlestring.sty}
\Msg{*}
\Msg{* To produce the documentation run the file `gettitlestring.drv'}
\Msg{* through LaTeX.}
\Msg{*}
\Msg{* Happy TeXing!}
\Msg{*}
\Msg{************************************************************************}

\endbatchfile
%</install>
%<*ignore>
\fi
%</ignore>
%<*driver>
\NeedsTeXFormat{LaTeX2e}
\ProvidesFile{gettitlestring.drv}%
  [2016/05/16 v1.5 Cleanup title references (HO)]%
\documentclass{ltxdoc}
\usepackage{holtxdoc}[2011/11/22]
\begin{document}
  \DocInput{gettitlestring.dtx}%
\end{document}
%</driver>
% \fi
%
%
% \CharacterTable
%  {Upper-case    \A\B\C\D\E\F\G\H\I\J\K\L\M\N\O\P\Q\R\S\T\U\V\W\X\Y\Z
%   Lower-case    \a\b\c\d\e\f\g\h\i\j\k\l\m\n\o\p\q\r\s\t\u\v\w\x\y\z
%   Digits        \0\1\2\3\4\5\6\7\8\9
%   Exclamation   \!     Double quote  \"     Hash (number) \#
%   Dollar        \$     Percent       \%     Ampersand     \&
%   Acute accent  \'     Left paren    \(     Right paren   \)
%   Asterisk      \*     Plus          \+     Comma         \,
%   Minus         \-     Point         \.     Solidus       \/
%   Colon         \:     Semicolon     \;     Less than     \<
%   Equals        \=     Greater than  \>     Question mark \?
%   Commercial at \@     Left bracket  \[     Backslash     \\
%   Right bracket \]     Circumflex    \^     Underscore    \_
%   Grave accent  \`     Left brace    \{     Vertical bar  \|
%   Right brace   \}     Tilde         \~}
%
% \GetFileInfo{gettitlestring.drv}
%
% \title{The \xpackage{gettitlestring} package}
% \date{2016/05/16 v1.5}
% \author{Heiko Oberdiek\thanks
% {Please report any issues at \url{https://github.com/ho-tex/oberdiek/issues}}}
%
% \maketitle
%
% \begin{abstract}
% The \LaTeX\ package addresses packages that are dealing with
% references to titles (\cs{section}, \cs{caption}, \dots).
% The package tries to remove \cs{label} and other
% commands from title strings.
% \end{abstract}
%
% \tableofcontents
%
% \section{Documentation}
%
% \subsection{Macros}
%
% \begin{declcs}{GetTitleStringSetup} \M{key value list}
% \end{declcs}
% The options are given as comma separated key value pairs.
% See section \ref{sec:options}.
%
% \begin{declcs}{GetTitleString} \M{text}\\
% \cs{GetTitleStringExpand} \M{text}\\
% \cs{GetTitleStringNonExpand} \M{text}
% \end{declcs}
% Macro \cs{GetTitleString} tries to remove unwanted stuff from \meta{text}
% the result is stored in Macro \cs{GetTitleStringResult}.
% Two methods are available:
% \begin{description}
% \item[\cs{GetTitleStringExpand}:]
% The \meta{text} is expanded in a context where the unwanted
% macros are redefined to remove themselves.
% This is the method used in packages \xpackage{titleref}~\cite{titleref},
% \xpackage{zref-titleref}~\cite{zref}
% or class \xclass{memoir}~\cite{memoir}.
% \cs{protect} is supported, but fragile material might break.
% \item[\cs{GetTitleStringNonExpand}:]
% The \meta{text} is not expanded. Thus the removal of unwanted
% material is more difficult. It is especially removed at the
% start of the \meta{text} and spaces are removed from the end.
% Currently only \cs{label} is removed in the whole string,
% if it is not hidden inside curly braces or part of macro
% definitions. Thus the removal of unwanted stuff might not be
% complete, but fragile material will not break.
% (But the result string can break at a later time, of course).
% \end{description}
% Option \xoption{expand} controls which method is used by
% macro \cs{GetTitleString}.
%
% \begin{declcs}{GetTitleStringDisableCommands} \M{code}
% \end{declcs}
% The \meta{code} is called right before the
% text is expanded in \cs{GetTitleStringExpand}.
% Additional definitions can be given for macros that
% should be removed.
% Keep in mind that expansion means that the definitions
% must work in expandable context. Macros like
% \cs{@ifstar} or \cs{@ifnextchar} or optional arguments
% will not work. The macro names in \meta{code} may contain
% the at sign |@|, it has catcode 11 (letter).
%
% \subsection{Options}\label{sec:options}
%
% \begin{description}
% \item[\xoption{expand}:] Boolean option, takes values |true| or |false|.
% No value means |true|. The option specifies the method to remove
% unwanted stuff from the title string, see below.
% \end{description}
% Options can be set at the following places:
% \begin{itemize}
% \item \cs{usepackage}
% \item Configuration file \xfile{gettitlestring.cfg}.
% \item \cs{GetTitleStringSetup}
% \end{itemize}
%
% \StopEventually{
% }
%
% \section{Implementation}
%    \begin{macrocode}
%<*package>
%    \end{macrocode}
%    Reload check, especially if the package is not used with \LaTeX.
%    \begin{macrocode}
\begingroup\catcode61\catcode48\catcode32=10\relax%
  \catcode13=5 % ^^M
  \endlinechar=13 %
  \catcode35=6 % #
  \catcode39=12 % '
  \catcode44=12 % ,
  \catcode45=12 % -
  \catcode46=12 % .
  \catcode58=12 % :
  \catcode64=11 % @
  \catcode123=1 % {
  \catcode125=2 % }
  \expandafter\let\expandafter\x\csname ver@gettitlestring.sty\endcsname
  \ifx\x\relax % plain-TeX, first loading
  \else
    \def\empty{}%
    \ifx\x\empty % LaTeX, first loading,
      % variable is initialized, but \ProvidesPackage not yet seen
    \else
      \expandafter\ifx\csname PackageInfo\endcsname\relax
        \def\x#1#2{%
          \immediate\write-1{Package #1 Info: #2.}%
        }%
      \else
        \def\x#1#2{\PackageInfo{#1}{#2, stopped}}%
      \fi
      \x{gettitlestring}{The package is already loaded}%
      \aftergroup\endinput
    \fi
  \fi
\endgroup%
%    \end{macrocode}
%    Package identification:
%    \begin{macrocode}
\begingroup\catcode61\catcode48\catcode32=10\relax%
  \catcode13=5 % ^^M
  \endlinechar=13 %
  \catcode35=6 % #
  \catcode39=12 % '
  \catcode40=12 % (
  \catcode41=12 % )
  \catcode44=12 % ,
  \catcode45=12 % -
  \catcode46=12 % .
  \catcode47=12 % /
  \catcode58=12 % :
  \catcode64=11 % @
  \catcode91=12 % [
  \catcode93=12 % ]
  \catcode123=1 % {
  \catcode125=2 % }
  \expandafter\ifx\csname ProvidesPackage\endcsname\relax
    \def\x#1#2#3[#4]{\endgroup
      \immediate\write-1{Package: #3 #4}%
      \xdef#1{#4}%
    }%
  \else
    \def\x#1#2[#3]{\endgroup
      #2[{#3}]%
      \ifx#1\@undefined
        \xdef#1{#3}%
      \fi
      \ifx#1\relax
        \xdef#1{#3}%
      \fi
    }%
  \fi
\expandafter\x\csname ver@gettitlestring.sty\endcsname
\ProvidesPackage{gettitlestring}%
  [2016/05/16 v1.5 Cleanup title references (HO)]%
%    \end{macrocode}
%
%    \begin{macrocode}
\begingroup\catcode61\catcode48\catcode32=10\relax%
  \catcode13=5 % ^^M
  \endlinechar=13 %
  \catcode123=1 % {
  \catcode125=2 % }
  \catcode64=11 % @
  \def\x{\endgroup
    \expandafter\edef\csname GTS@AtEnd\endcsname{%
      \endlinechar=\the\endlinechar\relax
      \catcode13=\the\catcode13\relax
      \catcode32=\the\catcode32\relax
      \catcode35=\the\catcode35\relax
      \catcode61=\the\catcode61\relax
      \catcode64=\the\catcode64\relax
      \catcode123=\the\catcode123\relax
      \catcode125=\the\catcode125\relax
    }%
  }%
\x\catcode61\catcode48\catcode32=10\relax%
\catcode13=5 % ^^M
\endlinechar=13 %
\catcode35=6 % #
\catcode64=11 % @
\catcode123=1 % {
\catcode125=2 % }
\def\TMP@EnsureCode#1#2{%
  \edef\GTS@AtEnd{%
    \GTS@AtEnd
    \catcode#1=\the\catcode#1\relax
  }%
  \catcode#1=#2\relax
}
\TMP@EnsureCode{42}{12}% *
\TMP@EnsureCode{44}{12}% ,
\TMP@EnsureCode{45}{12}% -
\TMP@EnsureCode{46}{12}% .
\TMP@EnsureCode{47}{12}% /
\TMP@EnsureCode{91}{12}% [
\TMP@EnsureCode{93}{12}% ]
\edef\GTS@AtEnd{\GTS@AtEnd\noexpand\endinput}
%    \end{macrocode}
%
% \subsection{Options}
%
%    \begin{macrocode}
\RequirePackage{kvoptions}[2009/07/17]
\SetupKeyvalOptions{%
  family=gettitlestring,%
  prefix=GTS@%
}
\newcommand*{\GetTitleStringSetup}{%
  \setkeys{gettitlestring}%
}
\DeclareBoolOption{expand}
\InputIfFileExists{gettitlestring.cfg}{}{}
\ProcessKeyvalOptions*\relax
%    \end{macrocode}
%
% \subsection{\cs{GetTitleString}}
%
%    \begin{macro}{\GetTitleString}
%    \begin{macrocode}
\newcommand*{\GetTitleString}{%
  \ifGTS@expand
    \expandafter\GetTitleStringExpand
  \else
    \expandafter\GetTitleStringNonExpand
  \fi
}
%    \end{macrocode}
%    \end{macro}
%    \begin{macro}{\GetTitleStringExpand}
%    \begin{macrocode}
\newcommand{\GetTitleStringExpand}[1]{%
  \def\GetTitleStringResult{#1}%
  \begingroup
    \GTS@DisablePredefinedCmds
    \GTS@DisableHook
    \edef\x{\endgroup
      \noexpand\def\noexpand\GetTitleStringResult{%
        \GetTitleStringResult
      }%
    }%
  \x
}
%    \end{macrocode}
%    \end{macro}
%    \begin{macro}{\GetTitleString}
%    \begin{macrocode}
\newcommand{\GetTitleStringNonExpand}[1]{%
  \def\GetTitleStringResult{#1}%
  \global\let\GTS@GlobalString\GetTitleStringResult
  \begingroup
    \GTS@RemoveLeft
    \GTS@RemoveRight
  \endgroup
  \let\GetTitleStringResult\GTS@GlobalString
}
%    \end{macrocode}
%    \end{macro}
%
% \subsubsection{Expand method}
%
%    \begin{macro}{\GTS@DisablePredefinedCmds}
%    \begin{macrocode}
\def\GTS@DisablePredefinedCmds{%
  \let\label\@gobble
  \let\zlabel\@gobble
  \let\zref@label\@gobble
  \let\zref@labelbylist\@gobbletwo
  \let\zref@labelbyprops\@gobbletwo
  \let\index\@gobble
  \let\glossary\@gobble
  \let\markboth\@gobbletwo
  \let\@mkboth\@gobbletwo
  \let\markright\@gobble
  \let\phantomsection\@empty
  \def\addcontentsline{\expandafter\@gobble\@gobbletwo}%
  \let\raggedright\@empty
  \let\raggedleft\@empty
  \let\centering\@empty
  \let\protect\@unexpandable@protect
  \let\enit@format\@empty % package enumitem
}
%    \end{macrocode}
%    \end{macro}
%
%    \begin{macro}{\GTS@DisableHook}
%    \begin{macrocode}
\providecommand*{\GTS@DisableHook}{}
%    \end{macrocode}
%    \end{macro}
%    \begin{macro}{\GetTitleStringDisableCommands}
%    \begin{macrocode}
\def\GetTitleStringDisableCommands{%
  \begingroup
    \makeatletter
    \GTS@DisableCommands
}
%    \end{macrocode}
%    \end{macro}
%    \begin{macro}{\GTS@DisableCommands}
%    \begin{macrocode}
\long\def\GTS@DisableCommands#1{%
    \toks0=\expandafter{\GTS@DisableHook}%
    \toks2={#1}%
    \xdef\GTS@GlobalString{\the\toks0 \the\toks2}%
  \endgroup
  \let\GTS@DisableHook\GTS@GlobalString
}
%    \end{macrocode}
%    \end{macro}
%
% \subsubsection{Non-expand method}
%
%    \begin{macrocode}
\def\GTS@RemoveLeft{%
  \toks@\expandafter\expandafter\expandafter{%
    \expandafter\GTS@Car\GTS@GlobalString{}{}{}{}\GTS@Nil
  }%
  \edef\GTS@Token{\the\toks@}%
  \GTS@PredefinedLeftCmds
  \expandafter\futurelet\expandafter\GTS@Token
  \expandafter\GTS@TestLeftSpace\GTS@GlobalString\GTS@Nil
  \GTS@End
}
\def\GTS@End{}
\long\def\GTS@TestLeft#1#2{%
  \def\GTS@temp{#1}%
  \ifx\GTS@temp\GTS@Token
    \toks@\expandafter\expandafter\expandafter{%
      \expandafter#2\GTS@GlobalString\GTS@Nil
    }%
    \expandafter\GTS@TestLeftEnd
  \fi
}
\long\def\GTS@TestLeftEnd#1\GTS@End{%
  \xdef\GTS@GlobalString{\the\toks@}%
  \GTS@RemoveLeft
}
\long\def\GTS@Car#1#2\GTS@Nil{#1}
\long\def\GTS@Cdr#1#2\GTS@Nil{#2}
\long\def\GTS@CdrTwo#1#2#3\GTS@Nil{#3}
\long\def\GTS@CdrThree#1#2#3#4\GTS@Nil{#4}
\long\def\GTS@CdrFour#1#2#3#4#5\GTS@Nil{#5}
\long\def\GTS@TestLeftSpace#1\GTS@Nil{%
  \ifx\GTS@Token\@sptoken
    \toks@\expandafter{%
      \romannumeral-0\GTS@GlobalString
    }%
    \expandafter\GTS@TestLeftEnd
  \fi
}
%    \end{macrocode}
%    \begin{macro}{\GTS@PredefinedLeftCmds}
%    \begin{macrocode}
\def\GTS@PredefinedLeftCmds{%
  \GTS@TestLeft\Hy@phantomsection\GTS@Cdr
  \GTS@TestLeft\Hy@SectionAnchor\GTS@Cdr
  \GTS@TestLeft\Hy@SectionAnchorHref\GTS@CdrTwo
  \GTS@TestLeft\label\GTS@CdrTwo
  \GTS@TestLeft\zlabel\GTS@CdrTwo
  \GTS@TestLeft\index\GTS@CdrTwo
  \GTS@TestLeft\glossary\GTS@CdrTwo
  \GTS@TestLeft\markboth\GTS@CdrThree
  \GTS@TestLeft\@mkboth\GTS@CdrThree
  \GTS@TestLeft\addcontentsline\GTS@CdrFour
  \GTS@TestLeft\enit@format\GTS@Cdr % package enumitem
}
%    \end{macrocode}
%    \end{macro}
%
%    \begin{macrocode}
\def\GTS@RemoveRight{%
  \toks@{}%
  \expandafter\GTS@TestRightLabel\GTS@GlobalString
      \label{}\GTS@Nil\@nil
  \GTS@RemoveRightSpace
}
\begingroup
  \def\GTS@temp#1{\endgroup
    \def\GTS@RemoveRightSpace{%
      \expandafter\GTS@TestRightSpace\GTS@GlobalString
          \GTS@Nil#1\GTS@Nil\@nil
    }%
  }%
\GTS@temp{ }
\def\GTS@TestRightSpace#1 \GTS@Nil#2\@nil{%
  \ifx\relax#2\relax
  \else
    \gdef\GTS@GlobalString{#1}%
    \expandafter\GTS@RemoveRightSpace
  \fi
}
\def\GTS@TestRightLabel#1\label#2#3\GTS@Nil#4\@nil{%
  \def\GTS@temp{#3}%
  \ifx\GTS@temp\@empty
    \expandafter\gdef\expandafter\GTS@GlobalString\expandafter{%
      \the\toks@
      #1%
    }%
    \expandafter\@gobble
  \else
    \expandafter\@firstofone
  \fi
  {%
    \toks@\expandafter{\the\toks@#1}%
    \GTS@TestRightLabel#3\GTS@Nil\@nil
  }%
}
%    \end{macrocode}
%
%    \begin{macrocode}
\GTS@AtEnd%
%</package>
%    \end{macrocode}
%
% \section{Test}
%
% \subsection{Catcode checks for loading}
%
%    \begin{macrocode}
%<*test1>
%    \end{macrocode}
%    \begin{macrocode}
\catcode`\{=1 %
\catcode`\}=2 %
\catcode`\#=6 %
\catcode`\@=11 %
\expandafter\ifx\csname count@\endcsname\relax
  \countdef\count@=255 %
\fi
\expandafter\ifx\csname @gobble\endcsname\relax
  \long\def\@gobble#1{}%
\fi
\expandafter\ifx\csname @firstofone\endcsname\relax
  \long\def\@firstofone#1{#1}%
\fi
\expandafter\ifx\csname loop\endcsname\relax
  \expandafter\@firstofone
\else
  \expandafter\@gobble
\fi
{%
  \def\loop#1\repeat{%
    \def\body{#1}%
    \iterate
  }%
  \def\iterate{%
    \body
      \let\next\iterate
    \else
      \let\next\relax
    \fi
    \next
  }%
  \let\repeat=\fi
}%
\def\RestoreCatcodes{}
\count@=0 %
\loop
  \edef\RestoreCatcodes{%
    \RestoreCatcodes
    \catcode\the\count@=\the\catcode\count@\relax
  }%
\ifnum\count@<255 %
  \advance\count@ 1 %
\repeat

\def\RangeCatcodeInvalid#1#2{%
  \count@=#1\relax
  \loop
    \catcode\count@=15 %
  \ifnum\count@<#2\relax
    \advance\count@ 1 %
  \repeat
}
\def\RangeCatcodeCheck#1#2#3{%
  \count@=#1\relax
  \loop
    \ifnum#3=\catcode\count@
    \else
      \errmessage{%
        Character \the\count@\space
        with wrong catcode \the\catcode\count@\space
        instead of \number#3%
      }%
    \fi
  \ifnum\count@<#2\relax
    \advance\count@ 1 %
  \repeat
}
\def\space{ }
\expandafter\ifx\csname LoadCommand\endcsname\relax
  \def\LoadCommand{\input gettitlestring.sty\relax}%
\fi
\def\Test{%
  \RangeCatcodeInvalid{0}{47}%
  \RangeCatcodeInvalid{58}{64}%
  \RangeCatcodeInvalid{91}{96}%
  \RangeCatcodeInvalid{123}{255}%
  \catcode`\@=12 %
  \catcode`\\=0 %
  \catcode`\%=14 %
  \LoadCommand
  \RangeCatcodeCheck{0}{36}{15}%
  \RangeCatcodeCheck{37}{37}{14}%
  \RangeCatcodeCheck{38}{47}{15}%
  \RangeCatcodeCheck{48}{57}{12}%
  \RangeCatcodeCheck{58}{63}{15}%
  \RangeCatcodeCheck{64}{64}{12}%
  \RangeCatcodeCheck{65}{90}{11}%
  \RangeCatcodeCheck{91}{91}{15}%
  \RangeCatcodeCheck{92}{92}{0}%
  \RangeCatcodeCheck{93}{96}{15}%
  \RangeCatcodeCheck{97}{122}{11}%
  \RangeCatcodeCheck{123}{255}{15}%
  \RestoreCatcodes
}
\Test
\csname @@end\endcsname
\end
%    \end{macrocode}
%    \begin{macrocode}
%</test1>
%    \end{macrocode}
%
% \subsection{Test of non-expand method}
%
%    \begin{macrocode}
%<*test2>
\NeedsTeXFormat{LaTeX2e}
\documentclass{minimal}
\usepackage{gettitlestring}[2016/05/16]
\usepackage{qstest}
\IncludeTests{*}
\LogTests{log}{*}{*}
\begin{document}
\begin{qstest}{non-expand}{non-expand}
  \def\test#1#2{%
    \sbox0{%
      \GetTitleString{#1}%
      \Expect{#2}*{\GetTitleStringResult}%
    }%
    \Expect{0.0pt}*{\the\wd0}%
  }%
  \test{}{}%
  \test{ }{}%
  \test{ x }{x}%
  \test{ x y }{x y}%
  \test{ \relax}{\relax}%
  \test{\label{f}a}{a}%
  \test{ \label{f}a}{a}%
  \test{\label{f} a}{a}%
  \test{ \label{f} a}{a}%
  \test{a\label{f}}{a}%
  \test{a\label{f} }{a}%
  \test{a \label{f}}{a}%
  \test{a \label{f} }{a}%
  \test{a\label{f}b\label{g}}{ab}%
  \test{a \label{f}b \label{g} }{a b}%
  \test{a\label{f} b \label{g} }{a b}%
\end{qstest}
\end{document}
%</test2>
%    \end{macrocode}
%
% \section{Installation}
%
% \subsection{Download}
%
% \paragraph{Package.} This package is available on
% CTAN\footnote{\CTANpkg{gettitlestring}}:
% \begin{description}
% \item[\CTAN{macros/latex/contrib/oberdiek/gettitlestring.dtx}] The source file.
% \item[\CTAN{macros/latex/contrib/oberdiek/gettitlestring.pdf}] Documentation.
% \end{description}
%
%
% \paragraph{Bundle.} All the packages of the bundle `oberdiek'
% are also available in a TDS compliant ZIP archive. There
% the packages are already unpacked and the documentation files
% are generated. The files and directories obey the TDS standard.
% \begin{description}
% \item[\CTANinstall{install/macros/latex/contrib/oberdiek.tds.zip}]
% \end{description}
% \emph{TDS} refers to the standard ``A Directory Structure
% for \TeX\ Files'' (\CTAN{tds/tds.pdf}). Directories
% with \xfile{texmf} in their name are usually organized this way.
%
% \subsection{Bundle installation}
%
% \paragraph{Unpacking.} Unpack the \xfile{oberdiek.tds.zip} in the
% TDS tree (also known as \xfile{texmf} tree) of your choice.
% Example (linux):
% \begin{quote}
%   |unzip oberdiek.tds.zip -d ~/texmf|
% \end{quote}
%
% \paragraph{Script installation.}
% Check the directory \xfile{TDS:scripts/oberdiek/} for
% scripts that need further installation steps.
%
% \subsection{Package installation}
%
% \paragraph{Unpacking.} The \xfile{.dtx} file is a self-extracting
% \docstrip\ archive. The files are extracted by running the
% \xfile{.dtx} through \plainTeX:
% \begin{quote}
%   \verb|tex gettitlestring.dtx|
% \end{quote}
%
% \paragraph{TDS.} Now the different files must be moved into
% the different directories in your installation TDS tree
% (also known as \xfile{texmf} tree):
% \begin{quote}
% \def\t{^^A
% \begin{tabular}{@{}>{\ttfamily}l@{ $\rightarrow$ }>{\ttfamily}l@{}}
%   gettitlestring.sty & tex/generic/oberdiek/gettitlestring.sty\\
%   gettitlestring.pdf & doc/latex/oberdiek/gettitlestring.pdf\\
%   test/gettitlestring-test1.tex & doc/latex/oberdiek/test/gettitlestring-test1.tex\\
%   test/gettitlestring-test2.tex & doc/latex/oberdiek/test/gettitlestring-test2.tex\\
%   gettitlestring.dtx & source/latex/oberdiek/gettitlestring.dtx\\
% \end{tabular}^^A
% }^^A
% \sbox0{\t}^^A
% \ifdim\wd0>\linewidth
%   \begingroup
%     \advance\linewidth by\leftmargin
%     \advance\linewidth by\rightmargin
%   \edef\x{\endgroup
%     \def\noexpand\lw{\the\linewidth}^^A
%   }\x
%   \def\lwbox{^^A
%     \leavevmode
%     \hbox to \linewidth{^^A
%       \kern-\leftmargin\relax
%       \hss
%       \usebox0
%       \hss
%       \kern-\rightmargin\relax
%     }^^A
%   }^^A
%   \ifdim\wd0>\lw
%     \sbox0{\small\t}^^A
%     \ifdim\wd0>\linewidth
%       \ifdim\wd0>\lw
%         \sbox0{\footnotesize\t}^^A
%         \ifdim\wd0>\linewidth
%           \ifdim\wd0>\lw
%             \sbox0{\scriptsize\t}^^A
%             \ifdim\wd0>\linewidth
%               \ifdim\wd0>\lw
%                 \sbox0{\tiny\t}^^A
%                 \ifdim\wd0>\linewidth
%                   \lwbox
%                 \else
%                   \usebox0
%                 \fi
%               \else
%                 \lwbox
%               \fi
%             \else
%               \usebox0
%             \fi
%           \else
%             \lwbox
%           \fi
%         \else
%           \usebox0
%         \fi
%       \else
%         \lwbox
%       \fi
%     \else
%       \usebox0
%     \fi
%   \else
%     \lwbox
%   \fi
% \else
%   \usebox0
% \fi
% \end{quote}
% If you have a \xfile{docstrip.cfg} that configures and enables \docstrip's
% TDS installing feature, then some files can already be in the right
% place, see the documentation of \docstrip.
%
% \subsection{Refresh file name databases}
%
% If your \TeX~distribution
% (\teTeX, \mikTeX, \dots) relies on file name databases, you must refresh
% these. For example, \teTeX\ users run \verb|texhash| or
% \verb|mktexlsr|.
%
% \subsection{Some details for the interested}
%
% \paragraph{Unpacking with \LaTeX.}
% The \xfile{.dtx} chooses its action depending on the format:
% \begin{description}
% \item[\plainTeX:] Run \docstrip\ and extract the files.
% \item[\LaTeX:] Generate the documentation.
% \end{description}
% If you insist on using \LaTeX\ for \docstrip\ (really,
% \docstrip\ does not need \LaTeX), then inform the autodetect routine
% about your intention:
% \begin{quote}
%   \verb|latex \let\install=y% \iffalse meta-comment
%
% File: gettitlestring.dtx
% Version: 2016/05/16 v1.5
% Info: Cleanup title references
%
% Copyright (C) 2009, 2010 by
%    Heiko Oberdiek <heiko.oberdiek at googlemail.com>
%    2016
%    https://github.com/ho-tex/oberdiek/issues
%
% This work may be distributed and/or modified under the
% conditions of the LaTeX Project Public License, either
% version 1.3c of this license or (at your option) any later
% version. This version of this license is in
%    http://www.latex-project.org/lppl/lppl-1-3c.txt
% and the latest version of this license is in
%    http://www.latex-project.org/lppl.txt
% and version 1.3 or later is part of all distributions of
% LaTeX version 2005/12/01 or later.
%
% This work has the LPPL maintenance status "maintained".
%
% This Current Maintainer of this work is Heiko Oberdiek.
%
% The Base Interpreter refers to any `TeX-Format',
% because some files are installed in TDS:tex/generic//.
%
% This work consists of the main source file gettitlestring.dtx
% and the derived files
%    gettitlestring.sty, gettitlestring.pdf, gettitlestring.ins,
%    gettitlestring.drv, gettitlestring-test1.tex,
%    gettitlestring-test2.tex.
%
% Distribution:
%    CTAN:macros/latex/contrib/oberdiek/gettitlestring.dtx
%    CTAN:macros/latex/contrib/oberdiek/gettitlestring.pdf
%
% Unpacking:
%    (a) If gettitlestring.ins is present:
%           tex gettitlestring.ins
%    (b) Without gettitlestring.ins:
%           tex gettitlestring.dtx
%    (c) If you insist on using LaTeX
%           latex \let\install=y% \iffalse meta-comment
%
% File: gettitlestring.dtx
% Version: 2016/05/16 v1.5
% Info: Cleanup title references
%
% Copyright (C) 2009, 2010 by
%    Heiko Oberdiek <heiko.oberdiek at googlemail.com>
%    2016
%    https://github.com/ho-tex/oberdiek/issues
%
% This work may be distributed and/or modified under the
% conditions of the LaTeX Project Public License, either
% version 1.3c of this license or (at your option) any later
% version. This version of this license is in
%    http://www.latex-project.org/lppl/lppl-1-3c.txt
% and the latest version of this license is in
%    http://www.latex-project.org/lppl.txt
% and version 1.3 or later is part of all distributions of
% LaTeX version 2005/12/01 or later.
%
% This work has the LPPL maintenance status "maintained".
%
% This Current Maintainer of this work is Heiko Oberdiek.
%
% The Base Interpreter refers to any `TeX-Format',
% because some files are installed in TDS:tex/generic//.
%
% This work consists of the main source file gettitlestring.dtx
% and the derived files
%    gettitlestring.sty, gettitlestring.pdf, gettitlestring.ins,
%    gettitlestring.drv, gettitlestring-test1.tex,
%    gettitlestring-test2.tex.
%
% Distribution:
%    CTAN:macros/latex/contrib/oberdiek/gettitlestring.dtx
%    CTAN:macros/latex/contrib/oberdiek/gettitlestring.pdf
%
% Unpacking:
%    (a) If gettitlestring.ins is present:
%           tex gettitlestring.ins
%    (b) Without gettitlestring.ins:
%           tex gettitlestring.dtx
%    (c) If you insist on using LaTeX
%           latex \let\install=y% \iffalse meta-comment
%
% File: gettitlestring.dtx
% Version: 2016/05/16 v1.5
% Info: Cleanup title references
%
% Copyright (C) 2009, 2010 by
%    Heiko Oberdiek <heiko.oberdiek at googlemail.com>
%    2016
%    https://github.com/ho-tex/oberdiek/issues
%
% This work may be distributed and/or modified under the
% conditions of the LaTeX Project Public License, either
% version 1.3c of this license or (at your option) any later
% version. This version of this license is in
%    http://www.latex-project.org/lppl/lppl-1-3c.txt
% and the latest version of this license is in
%    http://www.latex-project.org/lppl.txt
% and version 1.3 or later is part of all distributions of
% LaTeX version 2005/12/01 or later.
%
% This work has the LPPL maintenance status "maintained".
%
% This Current Maintainer of this work is Heiko Oberdiek.
%
% The Base Interpreter refers to any `TeX-Format',
% because some files are installed in TDS:tex/generic//.
%
% This work consists of the main source file gettitlestring.dtx
% and the derived files
%    gettitlestring.sty, gettitlestring.pdf, gettitlestring.ins,
%    gettitlestring.drv, gettitlestring-test1.tex,
%    gettitlestring-test2.tex.
%
% Distribution:
%    CTAN:macros/latex/contrib/oberdiek/gettitlestring.dtx
%    CTAN:macros/latex/contrib/oberdiek/gettitlestring.pdf
%
% Unpacking:
%    (a) If gettitlestring.ins is present:
%           tex gettitlestring.ins
%    (b) Without gettitlestring.ins:
%           tex gettitlestring.dtx
%    (c) If you insist on using LaTeX
%           latex \let\install=y\input{gettitlestring.dtx}
%        (quote the arguments according to the demands of your shell)
%
% Documentation:
%    (a) If gettitlestring.drv is present:
%           latex gettitlestring.drv
%    (b) Without gettitlestring.drv:
%           latex gettitlestring.dtx; ...
%    The class ltxdoc loads the configuration file ltxdoc.cfg
%    if available. Here you can specify further options, e.g.
%    use A4 as paper format:
%       \PassOptionsToClass{a4paper}{article}
%
%    Programm calls to get the documentation (example):
%       pdflatex gettitlestring.dtx
%       makeindex -s gind.ist gettitlestring.idx
%       pdflatex gettitlestring.dtx
%       makeindex -s gind.ist gettitlestring.idx
%       pdflatex gettitlestring.dtx
%
% Installation:
%    TDS:tex/generic/oberdiek/gettitlestring.sty
%    TDS:doc/latex/oberdiek/gettitlestring.pdf
%    TDS:doc/latex/oberdiek/test/gettitlestring-test1.tex
%    TDS:doc/latex/oberdiek/test/gettitlestring-test2.tex
%    TDS:source/latex/oberdiek/gettitlestring.dtx
%
%<*ignore>
\begingroup
  \catcode123=1 %
  \catcode125=2 %
  \def\x{LaTeX2e}%
\expandafter\endgroup
\ifcase 0\ifx\install y1\fi\expandafter
         \ifx\csname processbatchFile\endcsname\relax\else1\fi
         \ifx\fmtname\x\else 1\fi\relax
\else\csname fi\endcsname
%</ignore>
%<*install>
\input docstrip.tex
\Msg{************************************************************************}
\Msg{* Installation}
\Msg{* Package: gettitlestring 2016/05/16 v1.5 Cleanup title references (HO)}
\Msg{************************************************************************}

\keepsilent
\askforoverwritefalse

\let\MetaPrefix\relax
\preamble

This is a generated file.

Project: gettitlestring
Version: 2016/05/16 v1.5

Copyright (C) 2009, 2010 by
   Heiko Oberdiek <heiko.oberdiek at googlemail.com>

This work may be distributed and/or modified under the
conditions of the LaTeX Project Public License, either
version 1.3c of this license or (at your option) any later
version. This version of this license is in
   http://www.latex-project.org/lppl/lppl-1-3c.txt
and the latest version of this license is in
   http://www.latex-project.org/lppl.txt
and version 1.3 or later is part of all distributions of
LaTeX version 2005/12/01 or later.

This work has the LPPL maintenance status "maintained".

This Current Maintainer of this work is Heiko Oberdiek.

The Base Interpreter refers to any `TeX-Format',
because some files are installed in TDS:tex/generic//.

This work consists of the main source file gettitlestring.dtx
and the derived files
   gettitlestring.sty, gettitlestring.pdf, gettitlestring.ins,
   gettitlestring.drv, gettitlestring-test1.tex,
   gettitlestring-test2.tex.

\endpreamble
\let\MetaPrefix\DoubleperCent

\generate{%
  \file{gettitlestring.ins}{\from{gettitlestring.dtx}{install}}%
  \file{gettitlestring.drv}{\from{gettitlestring.dtx}{driver}}%
  \usedir{tex/generic/oberdiek}%
  \file{gettitlestring.sty}{\from{gettitlestring.dtx}{package}}%
%  \usedir{doc/latex/oberdiek/test}%
%  \file{gettitlestring-test1.tex}{\from{gettitlestring.dtx}{test1}}%
%  \file{gettitlestring-test2.tex}{\from{gettitlestring.dtx}{test2}}%
  \nopreamble
  \nopostamble
%  \usedir{source/latex/oberdiek/catalogue}%
%  \file{gettitlestring.xml}{\from{gettitlestring.dtx}{catalogue}}%
}

\catcode32=13\relax% active space
\let =\space%
\Msg{************************************************************************}
\Msg{*}
\Msg{* To finish the installation you have to move the following}
\Msg{* file into a directory searched by TeX:}
\Msg{*}
\Msg{*     gettitlestring.sty}
\Msg{*}
\Msg{* To produce the documentation run the file `gettitlestring.drv'}
\Msg{* through LaTeX.}
\Msg{*}
\Msg{* Happy TeXing!}
\Msg{*}
\Msg{************************************************************************}

\endbatchfile
%</install>
%<*ignore>
\fi
%</ignore>
%<*driver>
\NeedsTeXFormat{LaTeX2e}
\ProvidesFile{gettitlestring.drv}%
  [2016/05/16 v1.5 Cleanup title references (HO)]%
\documentclass{ltxdoc}
\usepackage{holtxdoc}[2011/11/22]
\begin{document}
  \DocInput{gettitlestring.dtx}%
\end{document}
%</driver>
% \fi
%
%
% \CharacterTable
%  {Upper-case    \A\B\C\D\E\F\G\H\I\J\K\L\M\N\O\P\Q\R\S\T\U\V\W\X\Y\Z
%   Lower-case    \a\b\c\d\e\f\g\h\i\j\k\l\m\n\o\p\q\r\s\t\u\v\w\x\y\z
%   Digits        \0\1\2\3\4\5\6\7\8\9
%   Exclamation   \!     Double quote  \"     Hash (number) \#
%   Dollar        \$     Percent       \%     Ampersand     \&
%   Acute accent  \'     Left paren    \(     Right paren   \)
%   Asterisk      \*     Plus          \+     Comma         \,
%   Minus         \-     Point         \.     Solidus       \/
%   Colon         \:     Semicolon     \;     Less than     \<
%   Equals        \=     Greater than  \>     Question mark \?
%   Commercial at \@     Left bracket  \[     Backslash     \\
%   Right bracket \]     Circumflex    \^     Underscore    \_
%   Grave accent  \`     Left brace    \{     Vertical bar  \|
%   Right brace   \}     Tilde         \~}
%
% \GetFileInfo{gettitlestring.drv}
%
% \title{The \xpackage{gettitlestring} package}
% \date{2016/05/16 v1.5}
% \author{Heiko Oberdiek\thanks
% {Please report any issues at \url{https://github.com/ho-tex/oberdiek/issues}}\\
% \xemail{heiko.oberdiek at googlemail.com}}
%
% \maketitle
%
% \begin{abstract}
% The \LaTeX\ package addresses packages that are dealing with
% references to titles (\cs{section}, \cs{caption}, \dots).
% The package tries to remove \cs{label} and other
% commands from title strings.
% \end{abstract}
%
% \tableofcontents
%
% \section{Documentation}
%
% \subsection{Macros}
%
% \begin{declcs}{GetTitleStringSetup} \M{key value list}
% \end{declcs}
% The options are given as comma separated key value pairs.
% See section \ref{sec:options}.
%
% \begin{declcs}{GetTitleString} \M{text}\\
% \cs{GetTitleStringExpand} \M{text}\\
% \cs{GetTitleStringNonExpand} \M{text}
% \end{declcs}
% Macro \cs{GetTitleString} tries to remove unwanted stuff from \meta{text}
% the result is stored in Macro \cs{GetTitleStringResult}.
% Two methods are available:
% \begin{description}
% \item[\cs{GetTitleStringExpand}:]
% The \meta{text} is expanded in a context where the unwanted
% macros are redefined to remove themselves.
% This is the method used in packages \xpackage{titleref}~\cite{titleref},
% \xpackage{zref-titleref}~\cite{zref}
% or class \xclass{memoir}~\cite{memoir}.
% \cs{protect} is supported, but fragile material might break.
% \item[\cs{GetTitleStringNonExpand}:]
% The \meta{text} is not expanded. Thus the removal of unwanted
% material is more difficult. It is especially removed at the
% start of the \meta{text} and spaces are removed from the end.
% Currently only \cs{label} is removed in the whole string,
% if it is not hidden inside curly braces or part of macro
% definitions. Thus the removal of unwanted stuff might not be
% complete, but fragile material will not break.
% (But the result string can break at a later time, of course).
% \end{description}
% Option \xoption{expand} controls which method is used by
% macro \cs{GetTitleString}.
%
% \begin{declcs}{GetTitleStringDisableCommands} \M{code}
% \end{declcs}
% The \meta{code} is called right before the
% text is expanded in \cs{GetTitleStringExpand}.
% Additional definitions can be given for macros that
% should be removed.
% Keep in mind that expansion means that the definitions
% must work in expandable context. Macros like
% \cs{@ifstar} or \cs{@ifnextchar} or optional arguments
% will not work. The macro names in \meta{code} may contain
% the at sign |@|, it has catcode 11 (letter).
%
% \subsection{Options}\label{sec:options}
%
% \begin{description}
% \item[\xoption{expand}:] Boolean option, takes values |true| or |false|.
% No value means |true|. The option specifies the method to remove
% unwanted stuff from the title string, see below.
% \end{description}
% Options can be set at the following places:
% \begin{itemize}
% \item \cs{usepackage}
% \item Configuration file \xfile{gettitlestring.cfg}.
% \item \cs{GetTitleStringSetup}
% \end{itemize}
%
% \StopEventually{
% }
%
% \section{Implementation}
%    \begin{macrocode}
%<*package>
%    \end{macrocode}
%    Reload check, especially if the package is not used with \LaTeX.
%    \begin{macrocode}
\begingroup\catcode61\catcode48\catcode32=10\relax%
  \catcode13=5 % ^^M
  \endlinechar=13 %
  \catcode35=6 % #
  \catcode39=12 % '
  \catcode44=12 % ,
  \catcode45=12 % -
  \catcode46=12 % .
  \catcode58=12 % :
  \catcode64=11 % @
  \catcode123=1 % {
  \catcode125=2 % }
  \expandafter\let\expandafter\x\csname ver@gettitlestring.sty\endcsname
  \ifx\x\relax % plain-TeX, first loading
  \else
    \def\empty{}%
    \ifx\x\empty % LaTeX, first loading,
      % variable is initialized, but \ProvidesPackage not yet seen
    \else
      \expandafter\ifx\csname PackageInfo\endcsname\relax
        \def\x#1#2{%
          \immediate\write-1{Package #1 Info: #2.}%
        }%
      \else
        \def\x#1#2{\PackageInfo{#1}{#2, stopped}}%
      \fi
      \x{gettitlestring}{The package is already loaded}%
      \aftergroup\endinput
    \fi
  \fi
\endgroup%
%    \end{macrocode}
%    Package identification:
%    \begin{macrocode}
\begingroup\catcode61\catcode48\catcode32=10\relax%
  \catcode13=5 % ^^M
  \endlinechar=13 %
  \catcode35=6 % #
  \catcode39=12 % '
  \catcode40=12 % (
  \catcode41=12 % )
  \catcode44=12 % ,
  \catcode45=12 % -
  \catcode46=12 % .
  \catcode47=12 % /
  \catcode58=12 % :
  \catcode64=11 % @
  \catcode91=12 % [
  \catcode93=12 % ]
  \catcode123=1 % {
  \catcode125=2 % }
  \expandafter\ifx\csname ProvidesPackage\endcsname\relax
    \def\x#1#2#3[#4]{\endgroup
      \immediate\write-1{Package: #3 #4}%
      \xdef#1{#4}%
    }%
  \else
    \def\x#1#2[#3]{\endgroup
      #2[{#3}]%
      \ifx#1\@undefined
        \xdef#1{#3}%
      \fi
      \ifx#1\relax
        \xdef#1{#3}%
      \fi
    }%
  \fi
\expandafter\x\csname ver@gettitlestring.sty\endcsname
\ProvidesPackage{gettitlestring}%
  [2016/05/16 v1.5 Cleanup title references (HO)]%
%    \end{macrocode}
%
%    \begin{macrocode}
\begingroup\catcode61\catcode48\catcode32=10\relax%
  \catcode13=5 % ^^M
  \endlinechar=13 %
  \catcode123=1 % {
  \catcode125=2 % }
  \catcode64=11 % @
  \def\x{\endgroup
    \expandafter\edef\csname GTS@AtEnd\endcsname{%
      \endlinechar=\the\endlinechar\relax
      \catcode13=\the\catcode13\relax
      \catcode32=\the\catcode32\relax
      \catcode35=\the\catcode35\relax
      \catcode61=\the\catcode61\relax
      \catcode64=\the\catcode64\relax
      \catcode123=\the\catcode123\relax
      \catcode125=\the\catcode125\relax
    }%
  }%
\x\catcode61\catcode48\catcode32=10\relax%
\catcode13=5 % ^^M
\endlinechar=13 %
\catcode35=6 % #
\catcode64=11 % @
\catcode123=1 % {
\catcode125=2 % }
\def\TMP@EnsureCode#1#2{%
  \edef\GTS@AtEnd{%
    \GTS@AtEnd
    \catcode#1=\the\catcode#1\relax
  }%
  \catcode#1=#2\relax
}
\TMP@EnsureCode{42}{12}% *
\TMP@EnsureCode{44}{12}% ,
\TMP@EnsureCode{45}{12}% -
\TMP@EnsureCode{46}{12}% .
\TMP@EnsureCode{47}{12}% /
\TMP@EnsureCode{91}{12}% [
\TMP@EnsureCode{93}{12}% ]
\edef\GTS@AtEnd{\GTS@AtEnd\noexpand\endinput}
%    \end{macrocode}
%
% \subsection{Options}
%
%    \begin{macrocode}
\RequirePackage{kvoptions}[2009/07/17]
\SetupKeyvalOptions{%
  family=gettitlestring,%
  prefix=GTS@%
}
\newcommand*{\GetTitleStringSetup}{%
  \setkeys{gettitlestring}%
}
\DeclareBoolOption{expand}
\InputIfFileExists{gettitlestring.cfg}{}{}
\ProcessKeyvalOptions*\relax
%    \end{macrocode}
%
% \subsection{\cs{GetTitleString}}
%
%    \begin{macro}{\GetTitleString}
%    \begin{macrocode}
\newcommand*{\GetTitleString}{%
  \ifGTS@expand
    \expandafter\GetTitleStringExpand
  \else
    \expandafter\GetTitleStringNonExpand
  \fi
}
%    \end{macrocode}
%    \end{macro}
%    \begin{macro}{\GetTitleStringExpand}
%    \begin{macrocode}
\newcommand{\GetTitleStringExpand}[1]{%
  \def\GetTitleStringResult{#1}%
  \begingroup
    \GTS@DisablePredefinedCmds
    \GTS@DisableHook
    \edef\x{\endgroup
      \noexpand\def\noexpand\GetTitleStringResult{%
        \GetTitleStringResult
      }%
    }%
  \x
}
%    \end{macrocode}
%    \end{macro}
%    \begin{macro}{\GetTitleString}
%    \begin{macrocode}
\newcommand{\GetTitleStringNonExpand}[1]{%
  \def\GetTitleStringResult{#1}%
  \global\let\GTS@GlobalString\GetTitleStringResult
  \begingroup
    \GTS@RemoveLeft
    \GTS@RemoveRight
  \endgroup
  \let\GetTitleStringResult\GTS@GlobalString
}
%    \end{macrocode}
%    \end{macro}
%
% \subsubsection{Expand method}
%
%    \begin{macro}{\GTS@DisablePredefinedCmds}
%    \begin{macrocode}
\def\GTS@DisablePredefinedCmds{%
  \let\label\@gobble
  \let\zlabel\@gobble
  \let\zref@label\@gobble
  \let\zref@labelbylist\@gobbletwo
  \let\zref@labelbyprops\@gobbletwo
  \let\index\@gobble
  \let\glossary\@gobble
  \let\markboth\@gobbletwo
  \let\@mkboth\@gobbletwo
  \let\markright\@gobble
  \let\phantomsection\@empty
  \def\addcontentsline{\expandafter\@gobble\@gobbletwo}%
  \let\raggedright\@empty
  \let\raggedleft\@empty
  \let\centering\@empty
  \let\protect\@unexpandable@protect
  \let\enit@format\@empty % package enumitem
}
%    \end{macrocode}
%    \end{macro}
%
%    \begin{macro}{\GTS@DisableHook}
%    \begin{macrocode}
\providecommand*{\GTS@DisableHook}{}
%    \end{macrocode}
%    \end{macro}
%    \begin{macro}{\GetTitleStringDisableCommands}
%    \begin{macrocode}
\def\GetTitleStringDisableCommands{%
  \begingroup
    \makeatletter
    \GTS@DisableCommands
}
%    \end{macrocode}
%    \end{macro}
%    \begin{macro}{\GTS@DisableCommands}
%    \begin{macrocode}
\long\def\GTS@DisableCommands#1{%
    \toks0=\expandafter{\GTS@DisableHook}%
    \toks2={#1}%
    \xdef\GTS@GlobalString{\the\toks0 \the\toks2}%
  \endgroup
  \let\GTS@DisableHook\GTS@GlobalString
}
%    \end{macrocode}
%    \end{macro}
%
% \subsubsection{Non-expand method}
%
%    \begin{macrocode}
\def\GTS@RemoveLeft{%
  \toks@\expandafter\expandafter\expandafter{%
    \expandafter\GTS@Car\GTS@GlobalString{}{}{}{}\GTS@Nil
  }%
  \edef\GTS@Token{\the\toks@}%
  \GTS@PredefinedLeftCmds
  \expandafter\futurelet\expandafter\GTS@Token
  \expandafter\GTS@TestLeftSpace\GTS@GlobalString\GTS@Nil
  \GTS@End
}
\def\GTS@End{}
\long\def\GTS@TestLeft#1#2{%
  \def\GTS@temp{#1}%
  \ifx\GTS@temp\GTS@Token
    \toks@\expandafter\expandafter\expandafter{%
      \expandafter#2\GTS@GlobalString\GTS@Nil
    }%
    \expandafter\GTS@TestLeftEnd
  \fi
}
\long\def\GTS@TestLeftEnd#1\GTS@End{%
  \xdef\GTS@GlobalString{\the\toks@}%
  \GTS@RemoveLeft
}
\long\def\GTS@Car#1#2\GTS@Nil{#1}
\long\def\GTS@Cdr#1#2\GTS@Nil{#2}
\long\def\GTS@CdrTwo#1#2#3\GTS@Nil{#3}
\long\def\GTS@CdrThree#1#2#3#4\GTS@Nil{#4}
\long\def\GTS@CdrFour#1#2#3#4#5\GTS@Nil{#5}
\long\def\GTS@TestLeftSpace#1\GTS@Nil{%
  \ifx\GTS@Token\@sptoken
    \toks@\expandafter{%
      \romannumeral-0\GTS@GlobalString
    }%
    \expandafter\GTS@TestLeftEnd
  \fi
}
%    \end{macrocode}
%    \begin{macro}{\GTS@PredefinedLeftCmds}
%    \begin{macrocode}
\def\GTS@PredefinedLeftCmds{%
  \GTS@TestLeft\Hy@phantomsection\GTS@Cdr
  \GTS@TestLeft\Hy@SectionAnchor\GTS@Cdr
  \GTS@TestLeft\Hy@SectionAnchorHref\GTS@CdrTwo
  \GTS@TestLeft\label\GTS@CdrTwo
  \GTS@TestLeft\zlabel\GTS@CdrTwo
  \GTS@TestLeft\index\GTS@CdrTwo
  \GTS@TestLeft\glossary\GTS@CdrTwo
  \GTS@TestLeft\markboth\GTS@CdrThree
  \GTS@TestLeft\@mkboth\GTS@CdrThree
  \GTS@TestLeft\addcontentsline\GTS@CdrFour
  \GTS@TestLeft\enit@format\GTS@Cdr % package enumitem
}
%    \end{macrocode}
%    \end{macro}
%
%    \begin{macrocode}
\def\GTS@RemoveRight{%
  \toks@{}%
  \expandafter\GTS@TestRightLabel\GTS@GlobalString
      \label{}\GTS@Nil\@nil
  \GTS@RemoveRightSpace
}
\begingroup
  \def\GTS@temp#1{\endgroup
    \def\GTS@RemoveRightSpace{%
      \expandafter\GTS@TestRightSpace\GTS@GlobalString
          \GTS@Nil#1\GTS@Nil\@nil
    }%
  }%
\GTS@temp{ }
\def\GTS@TestRightSpace#1 \GTS@Nil#2\@nil{%
  \ifx\relax#2\relax
  \else
    \gdef\GTS@GlobalString{#1}%
    \expandafter\GTS@RemoveRightSpace
  \fi
}
\def\GTS@TestRightLabel#1\label#2#3\GTS@Nil#4\@nil{%
  \def\GTS@temp{#3}%
  \ifx\GTS@temp\@empty
    \expandafter\gdef\expandafter\GTS@GlobalString\expandafter{%
      \the\toks@
      #1%
    }%
    \expandafter\@gobble
  \else
    \expandafter\@firstofone
  \fi
  {%
    \toks@\expandafter{\the\toks@#1}%
    \GTS@TestRightLabel#3\GTS@Nil\@nil
  }%
}
%    \end{macrocode}
%
%    \begin{macrocode}
\GTS@AtEnd%
%</package>
%    \end{macrocode}
%
% \section{Test}
%
% \subsection{Catcode checks for loading}
%
%    \begin{macrocode}
%<*test1>
%    \end{macrocode}
%    \begin{macrocode}
\catcode`\{=1 %
\catcode`\}=2 %
\catcode`\#=6 %
\catcode`\@=11 %
\expandafter\ifx\csname count@\endcsname\relax
  \countdef\count@=255 %
\fi
\expandafter\ifx\csname @gobble\endcsname\relax
  \long\def\@gobble#1{}%
\fi
\expandafter\ifx\csname @firstofone\endcsname\relax
  \long\def\@firstofone#1{#1}%
\fi
\expandafter\ifx\csname loop\endcsname\relax
  \expandafter\@firstofone
\else
  \expandafter\@gobble
\fi
{%
  \def\loop#1\repeat{%
    \def\body{#1}%
    \iterate
  }%
  \def\iterate{%
    \body
      \let\next\iterate
    \else
      \let\next\relax
    \fi
    \next
  }%
  \let\repeat=\fi
}%
\def\RestoreCatcodes{}
\count@=0 %
\loop
  \edef\RestoreCatcodes{%
    \RestoreCatcodes
    \catcode\the\count@=\the\catcode\count@\relax
  }%
\ifnum\count@<255 %
  \advance\count@ 1 %
\repeat

\def\RangeCatcodeInvalid#1#2{%
  \count@=#1\relax
  \loop
    \catcode\count@=15 %
  \ifnum\count@<#2\relax
    \advance\count@ 1 %
  \repeat
}
\def\RangeCatcodeCheck#1#2#3{%
  \count@=#1\relax
  \loop
    \ifnum#3=\catcode\count@
    \else
      \errmessage{%
        Character \the\count@\space
        with wrong catcode \the\catcode\count@\space
        instead of \number#3%
      }%
    \fi
  \ifnum\count@<#2\relax
    \advance\count@ 1 %
  \repeat
}
\def\space{ }
\expandafter\ifx\csname LoadCommand\endcsname\relax
  \def\LoadCommand{\input gettitlestring.sty\relax}%
\fi
\def\Test{%
  \RangeCatcodeInvalid{0}{47}%
  \RangeCatcodeInvalid{58}{64}%
  \RangeCatcodeInvalid{91}{96}%
  \RangeCatcodeInvalid{123}{255}%
  \catcode`\@=12 %
  \catcode`\\=0 %
  \catcode`\%=14 %
  \LoadCommand
  \RangeCatcodeCheck{0}{36}{15}%
  \RangeCatcodeCheck{37}{37}{14}%
  \RangeCatcodeCheck{38}{47}{15}%
  \RangeCatcodeCheck{48}{57}{12}%
  \RangeCatcodeCheck{58}{63}{15}%
  \RangeCatcodeCheck{64}{64}{12}%
  \RangeCatcodeCheck{65}{90}{11}%
  \RangeCatcodeCheck{91}{91}{15}%
  \RangeCatcodeCheck{92}{92}{0}%
  \RangeCatcodeCheck{93}{96}{15}%
  \RangeCatcodeCheck{97}{122}{11}%
  \RangeCatcodeCheck{123}{255}{15}%
  \RestoreCatcodes
}
\Test
\csname @@end\endcsname
\end
%    \end{macrocode}
%    \begin{macrocode}
%</test1>
%    \end{macrocode}
%
% \subsection{Test of non-expand method}
%
%    \begin{macrocode}
%<*test2>
\NeedsTeXFormat{LaTeX2e}
\documentclass{minimal}
\usepackage{gettitlestring}[2016/05/16]
\usepackage{qstest}
\IncludeTests{*}
\LogTests{log}{*}{*}
\begin{document}
\begin{qstest}{non-expand}{non-expand}
  \def\test#1#2{%
    \sbox0{%
      \GetTitleString{#1}%
      \Expect{#2}*{\GetTitleStringResult}%
    }%
    \Expect{0.0pt}*{\the\wd0}%
  }%
  \test{}{}%
  \test{ }{}%
  \test{ x }{x}%
  \test{ x y }{x y}%
  \test{ \relax}{\relax}%
  \test{\label{f}a}{a}%
  \test{ \label{f}a}{a}%
  \test{\label{f} a}{a}%
  \test{ \label{f} a}{a}%
  \test{a\label{f}}{a}%
  \test{a\label{f} }{a}%
  \test{a \label{f}}{a}%
  \test{a \label{f} }{a}%
  \test{a\label{f}b\label{g}}{ab}%
  \test{a \label{f}b \label{g} }{a b}%
  \test{a\label{f} b \label{g} }{a b}%
\end{qstest}
\end{document}
%</test2>
%    \end{macrocode}
%
% \section{Installation}
%
% \subsection{Download}
%
% \paragraph{Package.} This package is available on
% CTAN\footnote{\CTANpkg{gettitlestring}}:
% \begin{description}
% \item[\CTAN{macros/latex/contrib/oberdiek/gettitlestring.dtx}] The source file.
% \item[\CTAN{macros/latex/contrib/oberdiek/gettitlestring.pdf}] Documentation.
% \end{description}
%
%
% \paragraph{Bundle.} All the packages of the bundle `oberdiek'
% are also available in a TDS compliant ZIP archive. There
% the packages are already unpacked and the documentation files
% are generated. The files and directories obey the TDS standard.
% \begin{description}
% \item[\CTANinstall{install/macros/latex/contrib/oberdiek.tds.zip}]
% \end{description}
% \emph{TDS} refers to the standard ``A Directory Structure
% for \TeX\ Files'' (\CTAN{tds/tds.pdf}). Directories
% with \xfile{texmf} in their name are usually organized this way.
%
% \subsection{Bundle installation}
%
% \paragraph{Unpacking.} Unpack the \xfile{oberdiek.tds.zip} in the
% TDS tree (also known as \xfile{texmf} tree) of your choice.
% Example (linux):
% \begin{quote}
%   |unzip oberdiek.tds.zip -d ~/texmf|
% \end{quote}
%
% \paragraph{Script installation.}
% Check the directory \xfile{TDS:scripts/oberdiek/} for
% scripts that need further installation steps.
% Package \xpackage{attachfile2} comes with the Perl script
% \xfile{pdfatfi.pl} that should be installed in such a way
% that it can be called as \texttt{pdfatfi}.
% Example (linux):
% \begin{quote}
%   |chmod +x scripts/oberdiek/pdfatfi.pl|\\
%   |cp scripts/oberdiek/pdfatfi.pl /usr/local/bin/|
% \end{quote}
%
% \subsection{Package installation}
%
% \paragraph{Unpacking.} The \xfile{.dtx} file is a self-extracting
% \docstrip\ archive. The files are extracted by running the
% \xfile{.dtx} through \plainTeX:
% \begin{quote}
%   \verb|tex gettitlestring.dtx|
% \end{quote}
%
% \paragraph{TDS.} Now the different files must be moved into
% the different directories in your installation TDS tree
% (also known as \xfile{texmf} tree):
% \begin{quote}
% \def\t{^^A
% \begin{tabular}{@{}>{\ttfamily}l@{ $\rightarrow$ }>{\ttfamily}l@{}}
%   gettitlestring.sty & tex/generic/oberdiek/gettitlestring.sty\\
%   gettitlestring.pdf & doc/latex/oberdiek/gettitlestring.pdf\\
%   test/gettitlestring-test1.tex & doc/latex/oberdiek/test/gettitlestring-test1.tex\\
%   test/gettitlestring-test2.tex & doc/latex/oberdiek/test/gettitlestring-test2.tex\\
%   gettitlestring.dtx & source/latex/oberdiek/gettitlestring.dtx\\
% \end{tabular}^^A
% }^^A
% \sbox0{\t}^^A
% \ifdim\wd0>\linewidth
%   \begingroup
%     \advance\linewidth by\leftmargin
%     \advance\linewidth by\rightmargin
%   \edef\x{\endgroup
%     \def\noexpand\lw{\the\linewidth}^^A
%   }\x
%   \def\lwbox{^^A
%     \leavevmode
%     \hbox to \linewidth{^^A
%       \kern-\leftmargin\relax
%       \hss
%       \usebox0
%       \hss
%       \kern-\rightmargin\relax
%     }^^A
%   }^^A
%   \ifdim\wd0>\lw
%     \sbox0{\small\t}^^A
%     \ifdim\wd0>\linewidth
%       \ifdim\wd0>\lw
%         \sbox0{\footnotesize\t}^^A
%         \ifdim\wd0>\linewidth
%           \ifdim\wd0>\lw
%             \sbox0{\scriptsize\t}^^A
%             \ifdim\wd0>\linewidth
%               \ifdim\wd0>\lw
%                 \sbox0{\tiny\t}^^A
%                 \ifdim\wd0>\linewidth
%                   \lwbox
%                 \else
%                   \usebox0
%                 \fi
%               \else
%                 \lwbox
%               \fi
%             \else
%               \usebox0
%             \fi
%           \else
%             \lwbox
%           \fi
%         \else
%           \usebox0
%         \fi
%       \else
%         \lwbox
%       \fi
%     \else
%       \usebox0
%     \fi
%   \else
%     \lwbox
%   \fi
% \else
%   \usebox0
% \fi
% \end{quote}
% If you have a \xfile{docstrip.cfg} that configures and enables \docstrip's
% TDS installing feature, then some files can already be in the right
% place, see the documentation of \docstrip.
%
% \subsection{Refresh file name databases}
%
% If your \TeX~distribution
% (\teTeX, \mikTeX, \dots) relies on file name databases, you must refresh
% these. For example, \teTeX\ users run \verb|texhash| or
% \verb|mktexlsr|.
%
% \subsection{Some details for the interested}
%
% \paragraph{Attached source.}
%
% The PDF documentation on CTAN also includes the
% \xfile{.dtx} source file. It can be extracted by
% AcrobatReader 6 or higher. Another option is \textsf{pdftk},
% e.g. unpack the file into the current directory:
% \begin{quote}
%   \verb|pdftk gettitlestring.pdf unpack_files output .|
% \end{quote}
%
% \paragraph{Unpacking with \LaTeX.}
% The \xfile{.dtx} chooses its action depending on the format:
% \begin{description}
% \item[\plainTeX:] Run \docstrip\ and extract the files.
% \item[\LaTeX:] Generate the documentation.
% \end{description}
% If you insist on using \LaTeX\ for \docstrip\ (really,
% \docstrip\ does not need \LaTeX), then inform the autodetect routine
% about your intention:
% \begin{quote}
%   \verb|latex \let\install=y\input{gettitlestring.dtx}|
% \end{quote}
% Do not forget to quote the argument according to the demands
% of your shell.
%
% \paragraph{Generating the documentation.}
% You can use both the \xfile{.dtx} or the \xfile{.drv} to generate
% the documentation. The process can be configured by the
% configuration file \xfile{ltxdoc.cfg}. For instance, put this
% line into this file, if you want to have A4 as paper format:
% \begin{quote}
%   \verb|\PassOptionsToClass{a4paper}{article}|
% \end{quote}
% An example follows how to generate the
% documentation with pdf\LaTeX:
% \begin{quote}
%\begin{verbatim}
%pdflatex gettitlestring.dtx
%makeindex -s gind.ist gettitlestring.idx
%pdflatex gettitlestring.dtx
%makeindex -s gind.ist gettitlestring.idx
%pdflatex gettitlestring.dtx
%\end{verbatim}
% \end{quote}
%
% \begin{thebibliography}{9}
%
% \bibitem{memoir}
% Peter Wilson, Lars Madsen:
% \textit{The Memoir Class};
% 2009/11/17 v1.61803398c;
% \CTANpkg{memoir}
%
% \bibitem{titleref}
% Donald Arsenau:
% \textit{Titleref.sty};
% 2001/04/05 ver 3.1;
% \CTAN{macros/latex/contrib/misc/titleref.sty}
%
% \bibitem{zref}
% Heiko Oberdiek:
% \textit{The \xpackage{zref} package};
% 2009/12/08 v2.7;
% \CTAN{macros/latex/contrib/oberdiek/zref.pdf}
%
% \end{thebibliography}
%
% \begin{History}
%   \begin{Version}{2009/12/08 v1.0}
%   \item
%     The first version.
%   \end{Version}
%   \begin{Version}{2009/12/12 v1.1}
%   \item
%     Short info shortened.
%   \end{Version}
%   \begin{Version}{2009/12/13 v1.2}
%   \item
%     Forgotten third argument for \cs{InputIfFileExists} added.
%   \end{Version}
%   \begin{Version}{2009/12/18 v1.3}
%   \item
%     \cs{Hy@SectionAnchorHref} added for filtering
%     (hyperref 2009/12/18 v6.79w).
%   \end{Version}
%   \begin{Version}{2010/12/03 v1.4}
%   \item
%     Support of package \xpackage{enumitem}: removing
%     \cs{enit@format} from title string (problem report by GL).
%   \end{Version}
%   \begin{Version}{2016/05/16 v1.5}
%   \item
%     Documentation updates.
%   \end{Version}
% \end{History}
%
% \PrintIndex
%
% \Finale
\endinput

%        (quote the arguments according to the demands of your shell)
%
% Documentation:
%    (a) If gettitlestring.drv is present:
%           latex gettitlestring.drv
%    (b) Without gettitlestring.drv:
%           latex gettitlestring.dtx; ...
%    The class ltxdoc loads the configuration file ltxdoc.cfg
%    if available. Here you can specify further options, e.g.
%    use A4 as paper format:
%       \PassOptionsToClass{a4paper}{article}
%
%    Programm calls to get the documentation (example):
%       pdflatex gettitlestring.dtx
%       makeindex -s gind.ist gettitlestring.idx
%       pdflatex gettitlestring.dtx
%       makeindex -s gind.ist gettitlestring.idx
%       pdflatex gettitlestring.dtx
%
% Installation:
%    TDS:tex/generic/oberdiek/gettitlestring.sty
%    TDS:doc/latex/oberdiek/gettitlestring.pdf
%    TDS:doc/latex/oberdiek/test/gettitlestring-test1.tex
%    TDS:doc/latex/oberdiek/test/gettitlestring-test2.tex
%    TDS:source/latex/oberdiek/gettitlestring.dtx
%
%<*ignore>
\begingroup
  \catcode123=1 %
  \catcode125=2 %
  \def\x{LaTeX2e}%
\expandafter\endgroup
\ifcase 0\ifx\install y1\fi\expandafter
         \ifx\csname processbatchFile\endcsname\relax\else1\fi
         \ifx\fmtname\x\else 1\fi\relax
\else\csname fi\endcsname
%</ignore>
%<*install>
\input docstrip.tex
\Msg{************************************************************************}
\Msg{* Installation}
\Msg{* Package: gettitlestring 2016/05/16 v1.5 Cleanup title references (HO)}
\Msg{************************************************************************}

\keepsilent
\askforoverwritefalse

\let\MetaPrefix\relax
\preamble

This is a generated file.

Project: gettitlestring
Version: 2016/05/16 v1.5

Copyright (C) 2009, 2010 by
   Heiko Oberdiek <heiko.oberdiek at googlemail.com>

This work may be distributed and/or modified under the
conditions of the LaTeX Project Public License, either
version 1.3c of this license or (at your option) any later
version. This version of this license is in
   http://www.latex-project.org/lppl/lppl-1-3c.txt
and the latest version of this license is in
   http://www.latex-project.org/lppl.txt
and version 1.3 or later is part of all distributions of
LaTeX version 2005/12/01 or later.

This work has the LPPL maintenance status "maintained".

This Current Maintainer of this work is Heiko Oberdiek.

The Base Interpreter refers to any `TeX-Format',
because some files are installed in TDS:tex/generic//.

This work consists of the main source file gettitlestring.dtx
and the derived files
   gettitlestring.sty, gettitlestring.pdf, gettitlestring.ins,
   gettitlestring.drv, gettitlestring-test1.tex,
   gettitlestring-test2.tex.

\endpreamble
\let\MetaPrefix\DoubleperCent

\generate{%
  \file{gettitlestring.ins}{\from{gettitlestring.dtx}{install}}%
  \file{gettitlestring.drv}{\from{gettitlestring.dtx}{driver}}%
  \usedir{tex/generic/oberdiek}%
  \file{gettitlestring.sty}{\from{gettitlestring.dtx}{package}}%
%  \usedir{doc/latex/oberdiek/test}%
%  \file{gettitlestring-test1.tex}{\from{gettitlestring.dtx}{test1}}%
%  \file{gettitlestring-test2.tex}{\from{gettitlestring.dtx}{test2}}%
  \nopreamble
  \nopostamble
%  \usedir{source/latex/oberdiek/catalogue}%
%  \file{gettitlestring.xml}{\from{gettitlestring.dtx}{catalogue}}%
}

\catcode32=13\relax% active space
\let =\space%
\Msg{************************************************************************}
\Msg{*}
\Msg{* To finish the installation you have to move the following}
\Msg{* file into a directory searched by TeX:}
\Msg{*}
\Msg{*     gettitlestring.sty}
\Msg{*}
\Msg{* To produce the documentation run the file `gettitlestring.drv'}
\Msg{* through LaTeX.}
\Msg{*}
\Msg{* Happy TeXing!}
\Msg{*}
\Msg{************************************************************************}

\endbatchfile
%</install>
%<*ignore>
\fi
%</ignore>
%<*driver>
\NeedsTeXFormat{LaTeX2e}
\ProvidesFile{gettitlestring.drv}%
  [2016/05/16 v1.5 Cleanup title references (HO)]%
\documentclass{ltxdoc}
\usepackage{holtxdoc}[2011/11/22]
\begin{document}
  \DocInput{gettitlestring.dtx}%
\end{document}
%</driver>
% \fi
%
%
% \CharacterTable
%  {Upper-case    \A\B\C\D\E\F\G\H\I\J\K\L\M\N\O\P\Q\R\S\T\U\V\W\X\Y\Z
%   Lower-case    \a\b\c\d\e\f\g\h\i\j\k\l\m\n\o\p\q\r\s\t\u\v\w\x\y\z
%   Digits        \0\1\2\3\4\5\6\7\8\9
%   Exclamation   \!     Double quote  \"     Hash (number) \#
%   Dollar        \$     Percent       \%     Ampersand     \&
%   Acute accent  \'     Left paren    \(     Right paren   \)
%   Asterisk      \*     Plus          \+     Comma         \,
%   Minus         \-     Point         \.     Solidus       \/
%   Colon         \:     Semicolon     \;     Less than     \<
%   Equals        \=     Greater than  \>     Question mark \?
%   Commercial at \@     Left bracket  \[     Backslash     \\
%   Right bracket \]     Circumflex    \^     Underscore    \_
%   Grave accent  \`     Left brace    \{     Vertical bar  \|
%   Right brace   \}     Tilde         \~}
%
% \GetFileInfo{gettitlestring.drv}
%
% \title{The \xpackage{gettitlestring} package}
% \date{2016/05/16 v1.5}
% \author{Heiko Oberdiek\thanks
% {Please report any issues at \url{https://github.com/ho-tex/oberdiek/issues}}\\
% \xemail{heiko.oberdiek at googlemail.com}}
%
% \maketitle
%
% \begin{abstract}
% The \LaTeX\ package addresses packages that are dealing with
% references to titles (\cs{section}, \cs{caption}, \dots).
% The package tries to remove \cs{label} and other
% commands from title strings.
% \end{abstract}
%
% \tableofcontents
%
% \section{Documentation}
%
% \subsection{Macros}
%
% \begin{declcs}{GetTitleStringSetup} \M{key value list}
% \end{declcs}
% The options are given as comma separated key value pairs.
% See section \ref{sec:options}.
%
% \begin{declcs}{GetTitleString} \M{text}\\
% \cs{GetTitleStringExpand} \M{text}\\
% \cs{GetTitleStringNonExpand} \M{text}
% \end{declcs}
% Macro \cs{GetTitleString} tries to remove unwanted stuff from \meta{text}
% the result is stored in Macro \cs{GetTitleStringResult}.
% Two methods are available:
% \begin{description}
% \item[\cs{GetTitleStringExpand}:]
% The \meta{text} is expanded in a context where the unwanted
% macros are redefined to remove themselves.
% This is the method used in packages \xpackage{titleref}~\cite{titleref},
% \xpackage{zref-titleref}~\cite{zref}
% or class \xclass{memoir}~\cite{memoir}.
% \cs{protect} is supported, but fragile material might break.
% \item[\cs{GetTitleStringNonExpand}:]
% The \meta{text} is not expanded. Thus the removal of unwanted
% material is more difficult. It is especially removed at the
% start of the \meta{text} and spaces are removed from the end.
% Currently only \cs{label} is removed in the whole string,
% if it is not hidden inside curly braces or part of macro
% definitions. Thus the removal of unwanted stuff might not be
% complete, but fragile material will not break.
% (But the result string can break at a later time, of course).
% \end{description}
% Option \xoption{expand} controls which method is used by
% macro \cs{GetTitleString}.
%
% \begin{declcs}{GetTitleStringDisableCommands} \M{code}
% \end{declcs}
% The \meta{code} is called right before the
% text is expanded in \cs{GetTitleStringExpand}.
% Additional definitions can be given for macros that
% should be removed.
% Keep in mind that expansion means that the definitions
% must work in expandable context. Macros like
% \cs{@ifstar} or \cs{@ifnextchar} or optional arguments
% will not work. The macro names in \meta{code} may contain
% the at sign |@|, it has catcode 11 (letter).
%
% \subsection{Options}\label{sec:options}
%
% \begin{description}
% \item[\xoption{expand}:] Boolean option, takes values |true| or |false|.
% No value means |true|. The option specifies the method to remove
% unwanted stuff from the title string, see below.
% \end{description}
% Options can be set at the following places:
% \begin{itemize}
% \item \cs{usepackage}
% \item Configuration file \xfile{gettitlestring.cfg}.
% \item \cs{GetTitleStringSetup}
% \end{itemize}
%
% \StopEventually{
% }
%
% \section{Implementation}
%    \begin{macrocode}
%<*package>
%    \end{macrocode}
%    Reload check, especially if the package is not used with \LaTeX.
%    \begin{macrocode}
\begingroup\catcode61\catcode48\catcode32=10\relax%
  \catcode13=5 % ^^M
  \endlinechar=13 %
  \catcode35=6 % #
  \catcode39=12 % '
  \catcode44=12 % ,
  \catcode45=12 % -
  \catcode46=12 % .
  \catcode58=12 % :
  \catcode64=11 % @
  \catcode123=1 % {
  \catcode125=2 % }
  \expandafter\let\expandafter\x\csname ver@gettitlestring.sty\endcsname
  \ifx\x\relax % plain-TeX, first loading
  \else
    \def\empty{}%
    \ifx\x\empty % LaTeX, first loading,
      % variable is initialized, but \ProvidesPackage not yet seen
    \else
      \expandafter\ifx\csname PackageInfo\endcsname\relax
        \def\x#1#2{%
          \immediate\write-1{Package #1 Info: #2.}%
        }%
      \else
        \def\x#1#2{\PackageInfo{#1}{#2, stopped}}%
      \fi
      \x{gettitlestring}{The package is already loaded}%
      \aftergroup\endinput
    \fi
  \fi
\endgroup%
%    \end{macrocode}
%    Package identification:
%    \begin{macrocode}
\begingroup\catcode61\catcode48\catcode32=10\relax%
  \catcode13=5 % ^^M
  \endlinechar=13 %
  \catcode35=6 % #
  \catcode39=12 % '
  \catcode40=12 % (
  \catcode41=12 % )
  \catcode44=12 % ,
  \catcode45=12 % -
  \catcode46=12 % .
  \catcode47=12 % /
  \catcode58=12 % :
  \catcode64=11 % @
  \catcode91=12 % [
  \catcode93=12 % ]
  \catcode123=1 % {
  \catcode125=2 % }
  \expandafter\ifx\csname ProvidesPackage\endcsname\relax
    \def\x#1#2#3[#4]{\endgroup
      \immediate\write-1{Package: #3 #4}%
      \xdef#1{#4}%
    }%
  \else
    \def\x#1#2[#3]{\endgroup
      #2[{#3}]%
      \ifx#1\@undefined
        \xdef#1{#3}%
      \fi
      \ifx#1\relax
        \xdef#1{#3}%
      \fi
    }%
  \fi
\expandafter\x\csname ver@gettitlestring.sty\endcsname
\ProvidesPackage{gettitlestring}%
  [2016/05/16 v1.5 Cleanup title references (HO)]%
%    \end{macrocode}
%
%    \begin{macrocode}
\begingroup\catcode61\catcode48\catcode32=10\relax%
  \catcode13=5 % ^^M
  \endlinechar=13 %
  \catcode123=1 % {
  \catcode125=2 % }
  \catcode64=11 % @
  \def\x{\endgroup
    \expandafter\edef\csname GTS@AtEnd\endcsname{%
      \endlinechar=\the\endlinechar\relax
      \catcode13=\the\catcode13\relax
      \catcode32=\the\catcode32\relax
      \catcode35=\the\catcode35\relax
      \catcode61=\the\catcode61\relax
      \catcode64=\the\catcode64\relax
      \catcode123=\the\catcode123\relax
      \catcode125=\the\catcode125\relax
    }%
  }%
\x\catcode61\catcode48\catcode32=10\relax%
\catcode13=5 % ^^M
\endlinechar=13 %
\catcode35=6 % #
\catcode64=11 % @
\catcode123=1 % {
\catcode125=2 % }
\def\TMP@EnsureCode#1#2{%
  \edef\GTS@AtEnd{%
    \GTS@AtEnd
    \catcode#1=\the\catcode#1\relax
  }%
  \catcode#1=#2\relax
}
\TMP@EnsureCode{42}{12}% *
\TMP@EnsureCode{44}{12}% ,
\TMP@EnsureCode{45}{12}% -
\TMP@EnsureCode{46}{12}% .
\TMP@EnsureCode{47}{12}% /
\TMP@EnsureCode{91}{12}% [
\TMP@EnsureCode{93}{12}% ]
\edef\GTS@AtEnd{\GTS@AtEnd\noexpand\endinput}
%    \end{macrocode}
%
% \subsection{Options}
%
%    \begin{macrocode}
\RequirePackage{kvoptions}[2009/07/17]
\SetupKeyvalOptions{%
  family=gettitlestring,%
  prefix=GTS@%
}
\newcommand*{\GetTitleStringSetup}{%
  \setkeys{gettitlestring}%
}
\DeclareBoolOption{expand}
\InputIfFileExists{gettitlestring.cfg}{}{}
\ProcessKeyvalOptions*\relax
%    \end{macrocode}
%
% \subsection{\cs{GetTitleString}}
%
%    \begin{macro}{\GetTitleString}
%    \begin{macrocode}
\newcommand*{\GetTitleString}{%
  \ifGTS@expand
    \expandafter\GetTitleStringExpand
  \else
    \expandafter\GetTitleStringNonExpand
  \fi
}
%    \end{macrocode}
%    \end{macro}
%    \begin{macro}{\GetTitleStringExpand}
%    \begin{macrocode}
\newcommand{\GetTitleStringExpand}[1]{%
  \def\GetTitleStringResult{#1}%
  \begingroup
    \GTS@DisablePredefinedCmds
    \GTS@DisableHook
    \edef\x{\endgroup
      \noexpand\def\noexpand\GetTitleStringResult{%
        \GetTitleStringResult
      }%
    }%
  \x
}
%    \end{macrocode}
%    \end{macro}
%    \begin{macro}{\GetTitleString}
%    \begin{macrocode}
\newcommand{\GetTitleStringNonExpand}[1]{%
  \def\GetTitleStringResult{#1}%
  \global\let\GTS@GlobalString\GetTitleStringResult
  \begingroup
    \GTS@RemoveLeft
    \GTS@RemoveRight
  \endgroup
  \let\GetTitleStringResult\GTS@GlobalString
}
%    \end{macrocode}
%    \end{macro}
%
% \subsubsection{Expand method}
%
%    \begin{macro}{\GTS@DisablePredefinedCmds}
%    \begin{macrocode}
\def\GTS@DisablePredefinedCmds{%
  \let\label\@gobble
  \let\zlabel\@gobble
  \let\zref@label\@gobble
  \let\zref@labelbylist\@gobbletwo
  \let\zref@labelbyprops\@gobbletwo
  \let\index\@gobble
  \let\glossary\@gobble
  \let\markboth\@gobbletwo
  \let\@mkboth\@gobbletwo
  \let\markright\@gobble
  \let\phantomsection\@empty
  \def\addcontentsline{\expandafter\@gobble\@gobbletwo}%
  \let\raggedright\@empty
  \let\raggedleft\@empty
  \let\centering\@empty
  \let\protect\@unexpandable@protect
  \let\enit@format\@empty % package enumitem
}
%    \end{macrocode}
%    \end{macro}
%
%    \begin{macro}{\GTS@DisableHook}
%    \begin{macrocode}
\providecommand*{\GTS@DisableHook}{}
%    \end{macrocode}
%    \end{macro}
%    \begin{macro}{\GetTitleStringDisableCommands}
%    \begin{macrocode}
\def\GetTitleStringDisableCommands{%
  \begingroup
    \makeatletter
    \GTS@DisableCommands
}
%    \end{macrocode}
%    \end{macro}
%    \begin{macro}{\GTS@DisableCommands}
%    \begin{macrocode}
\long\def\GTS@DisableCommands#1{%
    \toks0=\expandafter{\GTS@DisableHook}%
    \toks2={#1}%
    \xdef\GTS@GlobalString{\the\toks0 \the\toks2}%
  \endgroup
  \let\GTS@DisableHook\GTS@GlobalString
}
%    \end{macrocode}
%    \end{macro}
%
% \subsubsection{Non-expand method}
%
%    \begin{macrocode}
\def\GTS@RemoveLeft{%
  \toks@\expandafter\expandafter\expandafter{%
    \expandafter\GTS@Car\GTS@GlobalString{}{}{}{}\GTS@Nil
  }%
  \edef\GTS@Token{\the\toks@}%
  \GTS@PredefinedLeftCmds
  \expandafter\futurelet\expandafter\GTS@Token
  \expandafter\GTS@TestLeftSpace\GTS@GlobalString\GTS@Nil
  \GTS@End
}
\def\GTS@End{}
\long\def\GTS@TestLeft#1#2{%
  \def\GTS@temp{#1}%
  \ifx\GTS@temp\GTS@Token
    \toks@\expandafter\expandafter\expandafter{%
      \expandafter#2\GTS@GlobalString\GTS@Nil
    }%
    \expandafter\GTS@TestLeftEnd
  \fi
}
\long\def\GTS@TestLeftEnd#1\GTS@End{%
  \xdef\GTS@GlobalString{\the\toks@}%
  \GTS@RemoveLeft
}
\long\def\GTS@Car#1#2\GTS@Nil{#1}
\long\def\GTS@Cdr#1#2\GTS@Nil{#2}
\long\def\GTS@CdrTwo#1#2#3\GTS@Nil{#3}
\long\def\GTS@CdrThree#1#2#3#4\GTS@Nil{#4}
\long\def\GTS@CdrFour#1#2#3#4#5\GTS@Nil{#5}
\long\def\GTS@TestLeftSpace#1\GTS@Nil{%
  \ifx\GTS@Token\@sptoken
    \toks@\expandafter{%
      \romannumeral-0\GTS@GlobalString
    }%
    \expandafter\GTS@TestLeftEnd
  \fi
}
%    \end{macrocode}
%    \begin{macro}{\GTS@PredefinedLeftCmds}
%    \begin{macrocode}
\def\GTS@PredefinedLeftCmds{%
  \GTS@TestLeft\Hy@phantomsection\GTS@Cdr
  \GTS@TestLeft\Hy@SectionAnchor\GTS@Cdr
  \GTS@TestLeft\Hy@SectionAnchorHref\GTS@CdrTwo
  \GTS@TestLeft\label\GTS@CdrTwo
  \GTS@TestLeft\zlabel\GTS@CdrTwo
  \GTS@TestLeft\index\GTS@CdrTwo
  \GTS@TestLeft\glossary\GTS@CdrTwo
  \GTS@TestLeft\markboth\GTS@CdrThree
  \GTS@TestLeft\@mkboth\GTS@CdrThree
  \GTS@TestLeft\addcontentsline\GTS@CdrFour
  \GTS@TestLeft\enit@format\GTS@Cdr % package enumitem
}
%    \end{macrocode}
%    \end{macro}
%
%    \begin{macrocode}
\def\GTS@RemoveRight{%
  \toks@{}%
  \expandafter\GTS@TestRightLabel\GTS@GlobalString
      \label{}\GTS@Nil\@nil
  \GTS@RemoveRightSpace
}
\begingroup
  \def\GTS@temp#1{\endgroup
    \def\GTS@RemoveRightSpace{%
      \expandafter\GTS@TestRightSpace\GTS@GlobalString
          \GTS@Nil#1\GTS@Nil\@nil
    }%
  }%
\GTS@temp{ }
\def\GTS@TestRightSpace#1 \GTS@Nil#2\@nil{%
  \ifx\relax#2\relax
  \else
    \gdef\GTS@GlobalString{#1}%
    \expandafter\GTS@RemoveRightSpace
  \fi
}
\def\GTS@TestRightLabel#1\label#2#3\GTS@Nil#4\@nil{%
  \def\GTS@temp{#3}%
  \ifx\GTS@temp\@empty
    \expandafter\gdef\expandafter\GTS@GlobalString\expandafter{%
      \the\toks@
      #1%
    }%
    \expandafter\@gobble
  \else
    \expandafter\@firstofone
  \fi
  {%
    \toks@\expandafter{\the\toks@#1}%
    \GTS@TestRightLabel#3\GTS@Nil\@nil
  }%
}
%    \end{macrocode}
%
%    \begin{macrocode}
\GTS@AtEnd%
%</package>
%    \end{macrocode}
%
% \section{Test}
%
% \subsection{Catcode checks for loading}
%
%    \begin{macrocode}
%<*test1>
%    \end{macrocode}
%    \begin{macrocode}
\catcode`\{=1 %
\catcode`\}=2 %
\catcode`\#=6 %
\catcode`\@=11 %
\expandafter\ifx\csname count@\endcsname\relax
  \countdef\count@=255 %
\fi
\expandafter\ifx\csname @gobble\endcsname\relax
  \long\def\@gobble#1{}%
\fi
\expandafter\ifx\csname @firstofone\endcsname\relax
  \long\def\@firstofone#1{#1}%
\fi
\expandafter\ifx\csname loop\endcsname\relax
  \expandafter\@firstofone
\else
  \expandafter\@gobble
\fi
{%
  \def\loop#1\repeat{%
    \def\body{#1}%
    \iterate
  }%
  \def\iterate{%
    \body
      \let\next\iterate
    \else
      \let\next\relax
    \fi
    \next
  }%
  \let\repeat=\fi
}%
\def\RestoreCatcodes{}
\count@=0 %
\loop
  \edef\RestoreCatcodes{%
    \RestoreCatcodes
    \catcode\the\count@=\the\catcode\count@\relax
  }%
\ifnum\count@<255 %
  \advance\count@ 1 %
\repeat

\def\RangeCatcodeInvalid#1#2{%
  \count@=#1\relax
  \loop
    \catcode\count@=15 %
  \ifnum\count@<#2\relax
    \advance\count@ 1 %
  \repeat
}
\def\RangeCatcodeCheck#1#2#3{%
  \count@=#1\relax
  \loop
    \ifnum#3=\catcode\count@
    \else
      \errmessage{%
        Character \the\count@\space
        with wrong catcode \the\catcode\count@\space
        instead of \number#3%
      }%
    \fi
  \ifnum\count@<#2\relax
    \advance\count@ 1 %
  \repeat
}
\def\space{ }
\expandafter\ifx\csname LoadCommand\endcsname\relax
  \def\LoadCommand{\input gettitlestring.sty\relax}%
\fi
\def\Test{%
  \RangeCatcodeInvalid{0}{47}%
  \RangeCatcodeInvalid{58}{64}%
  \RangeCatcodeInvalid{91}{96}%
  \RangeCatcodeInvalid{123}{255}%
  \catcode`\@=12 %
  \catcode`\\=0 %
  \catcode`\%=14 %
  \LoadCommand
  \RangeCatcodeCheck{0}{36}{15}%
  \RangeCatcodeCheck{37}{37}{14}%
  \RangeCatcodeCheck{38}{47}{15}%
  \RangeCatcodeCheck{48}{57}{12}%
  \RangeCatcodeCheck{58}{63}{15}%
  \RangeCatcodeCheck{64}{64}{12}%
  \RangeCatcodeCheck{65}{90}{11}%
  \RangeCatcodeCheck{91}{91}{15}%
  \RangeCatcodeCheck{92}{92}{0}%
  \RangeCatcodeCheck{93}{96}{15}%
  \RangeCatcodeCheck{97}{122}{11}%
  \RangeCatcodeCheck{123}{255}{15}%
  \RestoreCatcodes
}
\Test
\csname @@end\endcsname
\end
%    \end{macrocode}
%    \begin{macrocode}
%</test1>
%    \end{macrocode}
%
% \subsection{Test of non-expand method}
%
%    \begin{macrocode}
%<*test2>
\NeedsTeXFormat{LaTeX2e}
\documentclass{minimal}
\usepackage{gettitlestring}[2016/05/16]
\usepackage{qstest}
\IncludeTests{*}
\LogTests{log}{*}{*}
\begin{document}
\begin{qstest}{non-expand}{non-expand}
  \def\test#1#2{%
    \sbox0{%
      \GetTitleString{#1}%
      \Expect{#2}*{\GetTitleStringResult}%
    }%
    \Expect{0.0pt}*{\the\wd0}%
  }%
  \test{}{}%
  \test{ }{}%
  \test{ x }{x}%
  \test{ x y }{x y}%
  \test{ \relax}{\relax}%
  \test{\label{f}a}{a}%
  \test{ \label{f}a}{a}%
  \test{\label{f} a}{a}%
  \test{ \label{f} a}{a}%
  \test{a\label{f}}{a}%
  \test{a\label{f} }{a}%
  \test{a \label{f}}{a}%
  \test{a \label{f} }{a}%
  \test{a\label{f}b\label{g}}{ab}%
  \test{a \label{f}b \label{g} }{a b}%
  \test{a\label{f} b \label{g} }{a b}%
\end{qstest}
\end{document}
%</test2>
%    \end{macrocode}
%
% \section{Installation}
%
% \subsection{Download}
%
% \paragraph{Package.} This package is available on
% CTAN\footnote{\CTANpkg{gettitlestring}}:
% \begin{description}
% \item[\CTAN{macros/latex/contrib/oberdiek/gettitlestring.dtx}] The source file.
% \item[\CTAN{macros/latex/contrib/oberdiek/gettitlestring.pdf}] Documentation.
% \end{description}
%
%
% \paragraph{Bundle.} All the packages of the bundle `oberdiek'
% are also available in a TDS compliant ZIP archive. There
% the packages are already unpacked and the documentation files
% are generated. The files and directories obey the TDS standard.
% \begin{description}
% \item[\CTANinstall{install/macros/latex/contrib/oberdiek.tds.zip}]
% \end{description}
% \emph{TDS} refers to the standard ``A Directory Structure
% for \TeX\ Files'' (\CTAN{tds/tds.pdf}). Directories
% with \xfile{texmf} in their name are usually organized this way.
%
% \subsection{Bundle installation}
%
% \paragraph{Unpacking.} Unpack the \xfile{oberdiek.tds.zip} in the
% TDS tree (also known as \xfile{texmf} tree) of your choice.
% Example (linux):
% \begin{quote}
%   |unzip oberdiek.tds.zip -d ~/texmf|
% \end{quote}
%
% \paragraph{Script installation.}
% Check the directory \xfile{TDS:scripts/oberdiek/} for
% scripts that need further installation steps.
% Package \xpackage{attachfile2} comes with the Perl script
% \xfile{pdfatfi.pl} that should be installed in such a way
% that it can be called as \texttt{pdfatfi}.
% Example (linux):
% \begin{quote}
%   |chmod +x scripts/oberdiek/pdfatfi.pl|\\
%   |cp scripts/oberdiek/pdfatfi.pl /usr/local/bin/|
% \end{quote}
%
% \subsection{Package installation}
%
% \paragraph{Unpacking.} The \xfile{.dtx} file is a self-extracting
% \docstrip\ archive. The files are extracted by running the
% \xfile{.dtx} through \plainTeX:
% \begin{quote}
%   \verb|tex gettitlestring.dtx|
% \end{quote}
%
% \paragraph{TDS.} Now the different files must be moved into
% the different directories in your installation TDS tree
% (also known as \xfile{texmf} tree):
% \begin{quote}
% \def\t{^^A
% \begin{tabular}{@{}>{\ttfamily}l@{ $\rightarrow$ }>{\ttfamily}l@{}}
%   gettitlestring.sty & tex/generic/oberdiek/gettitlestring.sty\\
%   gettitlestring.pdf & doc/latex/oberdiek/gettitlestring.pdf\\
%   test/gettitlestring-test1.tex & doc/latex/oberdiek/test/gettitlestring-test1.tex\\
%   test/gettitlestring-test2.tex & doc/latex/oberdiek/test/gettitlestring-test2.tex\\
%   gettitlestring.dtx & source/latex/oberdiek/gettitlestring.dtx\\
% \end{tabular}^^A
% }^^A
% \sbox0{\t}^^A
% \ifdim\wd0>\linewidth
%   \begingroup
%     \advance\linewidth by\leftmargin
%     \advance\linewidth by\rightmargin
%   \edef\x{\endgroup
%     \def\noexpand\lw{\the\linewidth}^^A
%   }\x
%   \def\lwbox{^^A
%     \leavevmode
%     \hbox to \linewidth{^^A
%       \kern-\leftmargin\relax
%       \hss
%       \usebox0
%       \hss
%       \kern-\rightmargin\relax
%     }^^A
%   }^^A
%   \ifdim\wd0>\lw
%     \sbox0{\small\t}^^A
%     \ifdim\wd0>\linewidth
%       \ifdim\wd0>\lw
%         \sbox0{\footnotesize\t}^^A
%         \ifdim\wd0>\linewidth
%           \ifdim\wd0>\lw
%             \sbox0{\scriptsize\t}^^A
%             \ifdim\wd0>\linewidth
%               \ifdim\wd0>\lw
%                 \sbox0{\tiny\t}^^A
%                 \ifdim\wd0>\linewidth
%                   \lwbox
%                 \else
%                   \usebox0
%                 \fi
%               \else
%                 \lwbox
%               \fi
%             \else
%               \usebox0
%             \fi
%           \else
%             \lwbox
%           \fi
%         \else
%           \usebox0
%         \fi
%       \else
%         \lwbox
%       \fi
%     \else
%       \usebox0
%     \fi
%   \else
%     \lwbox
%   \fi
% \else
%   \usebox0
% \fi
% \end{quote}
% If you have a \xfile{docstrip.cfg} that configures and enables \docstrip's
% TDS installing feature, then some files can already be in the right
% place, see the documentation of \docstrip.
%
% \subsection{Refresh file name databases}
%
% If your \TeX~distribution
% (\teTeX, \mikTeX, \dots) relies on file name databases, you must refresh
% these. For example, \teTeX\ users run \verb|texhash| or
% \verb|mktexlsr|.
%
% \subsection{Some details for the interested}
%
% \paragraph{Attached source.}
%
% The PDF documentation on CTAN also includes the
% \xfile{.dtx} source file. It can be extracted by
% AcrobatReader 6 or higher. Another option is \textsf{pdftk},
% e.g. unpack the file into the current directory:
% \begin{quote}
%   \verb|pdftk gettitlestring.pdf unpack_files output .|
% \end{quote}
%
% \paragraph{Unpacking with \LaTeX.}
% The \xfile{.dtx} chooses its action depending on the format:
% \begin{description}
% \item[\plainTeX:] Run \docstrip\ and extract the files.
% \item[\LaTeX:] Generate the documentation.
% \end{description}
% If you insist on using \LaTeX\ for \docstrip\ (really,
% \docstrip\ does not need \LaTeX), then inform the autodetect routine
% about your intention:
% \begin{quote}
%   \verb|latex \let\install=y% \iffalse meta-comment
%
% File: gettitlestring.dtx
% Version: 2016/05/16 v1.5
% Info: Cleanup title references
%
% Copyright (C) 2009, 2010 by
%    Heiko Oberdiek <heiko.oberdiek at googlemail.com>
%    2016
%    https://github.com/ho-tex/oberdiek/issues
%
% This work may be distributed and/or modified under the
% conditions of the LaTeX Project Public License, either
% version 1.3c of this license or (at your option) any later
% version. This version of this license is in
%    http://www.latex-project.org/lppl/lppl-1-3c.txt
% and the latest version of this license is in
%    http://www.latex-project.org/lppl.txt
% and version 1.3 or later is part of all distributions of
% LaTeX version 2005/12/01 or later.
%
% This work has the LPPL maintenance status "maintained".
%
% This Current Maintainer of this work is Heiko Oberdiek.
%
% The Base Interpreter refers to any `TeX-Format',
% because some files are installed in TDS:tex/generic//.
%
% This work consists of the main source file gettitlestring.dtx
% and the derived files
%    gettitlestring.sty, gettitlestring.pdf, gettitlestring.ins,
%    gettitlestring.drv, gettitlestring-test1.tex,
%    gettitlestring-test2.tex.
%
% Distribution:
%    CTAN:macros/latex/contrib/oberdiek/gettitlestring.dtx
%    CTAN:macros/latex/contrib/oberdiek/gettitlestring.pdf
%
% Unpacking:
%    (a) If gettitlestring.ins is present:
%           tex gettitlestring.ins
%    (b) Without gettitlestring.ins:
%           tex gettitlestring.dtx
%    (c) If you insist on using LaTeX
%           latex \let\install=y\input{gettitlestring.dtx}
%        (quote the arguments according to the demands of your shell)
%
% Documentation:
%    (a) If gettitlestring.drv is present:
%           latex gettitlestring.drv
%    (b) Without gettitlestring.drv:
%           latex gettitlestring.dtx; ...
%    The class ltxdoc loads the configuration file ltxdoc.cfg
%    if available. Here you can specify further options, e.g.
%    use A4 as paper format:
%       \PassOptionsToClass{a4paper}{article}
%
%    Programm calls to get the documentation (example):
%       pdflatex gettitlestring.dtx
%       makeindex -s gind.ist gettitlestring.idx
%       pdflatex gettitlestring.dtx
%       makeindex -s gind.ist gettitlestring.idx
%       pdflatex gettitlestring.dtx
%
% Installation:
%    TDS:tex/generic/oberdiek/gettitlestring.sty
%    TDS:doc/latex/oberdiek/gettitlestring.pdf
%    TDS:doc/latex/oberdiek/test/gettitlestring-test1.tex
%    TDS:doc/latex/oberdiek/test/gettitlestring-test2.tex
%    TDS:source/latex/oberdiek/gettitlestring.dtx
%
%<*ignore>
\begingroup
  \catcode123=1 %
  \catcode125=2 %
  \def\x{LaTeX2e}%
\expandafter\endgroup
\ifcase 0\ifx\install y1\fi\expandafter
         \ifx\csname processbatchFile\endcsname\relax\else1\fi
         \ifx\fmtname\x\else 1\fi\relax
\else\csname fi\endcsname
%</ignore>
%<*install>
\input docstrip.tex
\Msg{************************************************************************}
\Msg{* Installation}
\Msg{* Package: gettitlestring 2016/05/16 v1.5 Cleanup title references (HO)}
\Msg{************************************************************************}

\keepsilent
\askforoverwritefalse

\let\MetaPrefix\relax
\preamble

This is a generated file.

Project: gettitlestring
Version: 2016/05/16 v1.5

Copyright (C) 2009, 2010 by
   Heiko Oberdiek <heiko.oberdiek at googlemail.com>

This work may be distributed and/or modified under the
conditions of the LaTeX Project Public License, either
version 1.3c of this license or (at your option) any later
version. This version of this license is in
   http://www.latex-project.org/lppl/lppl-1-3c.txt
and the latest version of this license is in
   http://www.latex-project.org/lppl.txt
and version 1.3 or later is part of all distributions of
LaTeX version 2005/12/01 or later.

This work has the LPPL maintenance status "maintained".

This Current Maintainer of this work is Heiko Oberdiek.

The Base Interpreter refers to any `TeX-Format',
because some files are installed in TDS:tex/generic//.

This work consists of the main source file gettitlestring.dtx
and the derived files
   gettitlestring.sty, gettitlestring.pdf, gettitlestring.ins,
   gettitlestring.drv, gettitlestring-test1.tex,
   gettitlestring-test2.tex.

\endpreamble
\let\MetaPrefix\DoubleperCent

\generate{%
  \file{gettitlestring.ins}{\from{gettitlestring.dtx}{install}}%
  \file{gettitlestring.drv}{\from{gettitlestring.dtx}{driver}}%
  \usedir{tex/generic/oberdiek}%
  \file{gettitlestring.sty}{\from{gettitlestring.dtx}{package}}%
%  \usedir{doc/latex/oberdiek/test}%
%  \file{gettitlestring-test1.tex}{\from{gettitlestring.dtx}{test1}}%
%  \file{gettitlestring-test2.tex}{\from{gettitlestring.dtx}{test2}}%
  \nopreamble
  \nopostamble
%  \usedir{source/latex/oberdiek/catalogue}%
%  \file{gettitlestring.xml}{\from{gettitlestring.dtx}{catalogue}}%
}

\catcode32=13\relax% active space
\let =\space%
\Msg{************************************************************************}
\Msg{*}
\Msg{* To finish the installation you have to move the following}
\Msg{* file into a directory searched by TeX:}
\Msg{*}
\Msg{*     gettitlestring.sty}
\Msg{*}
\Msg{* To produce the documentation run the file `gettitlestring.drv'}
\Msg{* through LaTeX.}
\Msg{*}
\Msg{* Happy TeXing!}
\Msg{*}
\Msg{************************************************************************}

\endbatchfile
%</install>
%<*ignore>
\fi
%</ignore>
%<*driver>
\NeedsTeXFormat{LaTeX2e}
\ProvidesFile{gettitlestring.drv}%
  [2016/05/16 v1.5 Cleanup title references (HO)]%
\documentclass{ltxdoc}
\usepackage{holtxdoc}[2011/11/22]
\begin{document}
  \DocInput{gettitlestring.dtx}%
\end{document}
%</driver>
% \fi
%
%
% \CharacterTable
%  {Upper-case    \A\B\C\D\E\F\G\H\I\J\K\L\M\N\O\P\Q\R\S\T\U\V\W\X\Y\Z
%   Lower-case    \a\b\c\d\e\f\g\h\i\j\k\l\m\n\o\p\q\r\s\t\u\v\w\x\y\z
%   Digits        \0\1\2\3\4\5\6\7\8\9
%   Exclamation   \!     Double quote  \"     Hash (number) \#
%   Dollar        \$     Percent       \%     Ampersand     \&
%   Acute accent  \'     Left paren    \(     Right paren   \)
%   Asterisk      \*     Plus          \+     Comma         \,
%   Minus         \-     Point         \.     Solidus       \/
%   Colon         \:     Semicolon     \;     Less than     \<
%   Equals        \=     Greater than  \>     Question mark \?
%   Commercial at \@     Left bracket  \[     Backslash     \\
%   Right bracket \]     Circumflex    \^     Underscore    \_
%   Grave accent  \`     Left brace    \{     Vertical bar  \|
%   Right brace   \}     Tilde         \~}
%
% \GetFileInfo{gettitlestring.drv}
%
% \title{The \xpackage{gettitlestring} package}
% \date{2016/05/16 v1.5}
% \author{Heiko Oberdiek\thanks
% {Please report any issues at \url{https://github.com/ho-tex/oberdiek/issues}}\\
% \xemail{heiko.oberdiek at googlemail.com}}
%
% \maketitle
%
% \begin{abstract}
% The \LaTeX\ package addresses packages that are dealing with
% references to titles (\cs{section}, \cs{caption}, \dots).
% The package tries to remove \cs{label} and other
% commands from title strings.
% \end{abstract}
%
% \tableofcontents
%
% \section{Documentation}
%
% \subsection{Macros}
%
% \begin{declcs}{GetTitleStringSetup} \M{key value list}
% \end{declcs}
% The options are given as comma separated key value pairs.
% See section \ref{sec:options}.
%
% \begin{declcs}{GetTitleString} \M{text}\\
% \cs{GetTitleStringExpand} \M{text}\\
% \cs{GetTitleStringNonExpand} \M{text}
% \end{declcs}
% Macro \cs{GetTitleString} tries to remove unwanted stuff from \meta{text}
% the result is stored in Macro \cs{GetTitleStringResult}.
% Two methods are available:
% \begin{description}
% \item[\cs{GetTitleStringExpand}:]
% The \meta{text} is expanded in a context where the unwanted
% macros are redefined to remove themselves.
% This is the method used in packages \xpackage{titleref}~\cite{titleref},
% \xpackage{zref-titleref}~\cite{zref}
% or class \xclass{memoir}~\cite{memoir}.
% \cs{protect} is supported, but fragile material might break.
% \item[\cs{GetTitleStringNonExpand}:]
% The \meta{text} is not expanded. Thus the removal of unwanted
% material is more difficult. It is especially removed at the
% start of the \meta{text} and spaces are removed from the end.
% Currently only \cs{label} is removed in the whole string,
% if it is not hidden inside curly braces or part of macro
% definitions. Thus the removal of unwanted stuff might not be
% complete, but fragile material will not break.
% (But the result string can break at a later time, of course).
% \end{description}
% Option \xoption{expand} controls which method is used by
% macro \cs{GetTitleString}.
%
% \begin{declcs}{GetTitleStringDisableCommands} \M{code}
% \end{declcs}
% The \meta{code} is called right before the
% text is expanded in \cs{GetTitleStringExpand}.
% Additional definitions can be given for macros that
% should be removed.
% Keep in mind that expansion means that the definitions
% must work in expandable context. Macros like
% \cs{@ifstar} or \cs{@ifnextchar} or optional arguments
% will not work. The macro names in \meta{code} may contain
% the at sign |@|, it has catcode 11 (letter).
%
% \subsection{Options}\label{sec:options}
%
% \begin{description}
% \item[\xoption{expand}:] Boolean option, takes values |true| or |false|.
% No value means |true|. The option specifies the method to remove
% unwanted stuff from the title string, see below.
% \end{description}
% Options can be set at the following places:
% \begin{itemize}
% \item \cs{usepackage}
% \item Configuration file \xfile{gettitlestring.cfg}.
% \item \cs{GetTitleStringSetup}
% \end{itemize}
%
% \StopEventually{
% }
%
% \section{Implementation}
%    \begin{macrocode}
%<*package>
%    \end{macrocode}
%    Reload check, especially if the package is not used with \LaTeX.
%    \begin{macrocode}
\begingroup\catcode61\catcode48\catcode32=10\relax%
  \catcode13=5 % ^^M
  \endlinechar=13 %
  \catcode35=6 % #
  \catcode39=12 % '
  \catcode44=12 % ,
  \catcode45=12 % -
  \catcode46=12 % .
  \catcode58=12 % :
  \catcode64=11 % @
  \catcode123=1 % {
  \catcode125=2 % }
  \expandafter\let\expandafter\x\csname ver@gettitlestring.sty\endcsname
  \ifx\x\relax % plain-TeX, first loading
  \else
    \def\empty{}%
    \ifx\x\empty % LaTeX, first loading,
      % variable is initialized, but \ProvidesPackage not yet seen
    \else
      \expandafter\ifx\csname PackageInfo\endcsname\relax
        \def\x#1#2{%
          \immediate\write-1{Package #1 Info: #2.}%
        }%
      \else
        \def\x#1#2{\PackageInfo{#1}{#2, stopped}}%
      \fi
      \x{gettitlestring}{The package is already loaded}%
      \aftergroup\endinput
    \fi
  \fi
\endgroup%
%    \end{macrocode}
%    Package identification:
%    \begin{macrocode}
\begingroup\catcode61\catcode48\catcode32=10\relax%
  \catcode13=5 % ^^M
  \endlinechar=13 %
  \catcode35=6 % #
  \catcode39=12 % '
  \catcode40=12 % (
  \catcode41=12 % )
  \catcode44=12 % ,
  \catcode45=12 % -
  \catcode46=12 % .
  \catcode47=12 % /
  \catcode58=12 % :
  \catcode64=11 % @
  \catcode91=12 % [
  \catcode93=12 % ]
  \catcode123=1 % {
  \catcode125=2 % }
  \expandafter\ifx\csname ProvidesPackage\endcsname\relax
    \def\x#1#2#3[#4]{\endgroup
      \immediate\write-1{Package: #3 #4}%
      \xdef#1{#4}%
    }%
  \else
    \def\x#1#2[#3]{\endgroup
      #2[{#3}]%
      \ifx#1\@undefined
        \xdef#1{#3}%
      \fi
      \ifx#1\relax
        \xdef#1{#3}%
      \fi
    }%
  \fi
\expandafter\x\csname ver@gettitlestring.sty\endcsname
\ProvidesPackage{gettitlestring}%
  [2016/05/16 v1.5 Cleanup title references (HO)]%
%    \end{macrocode}
%
%    \begin{macrocode}
\begingroup\catcode61\catcode48\catcode32=10\relax%
  \catcode13=5 % ^^M
  \endlinechar=13 %
  \catcode123=1 % {
  \catcode125=2 % }
  \catcode64=11 % @
  \def\x{\endgroup
    \expandafter\edef\csname GTS@AtEnd\endcsname{%
      \endlinechar=\the\endlinechar\relax
      \catcode13=\the\catcode13\relax
      \catcode32=\the\catcode32\relax
      \catcode35=\the\catcode35\relax
      \catcode61=\the\catcode61\relax
      \catcode64=\the\catcode64\relax
      \catcode123=\the\catcode123\relax
      \catcode125=\the\catcode125\relax
    }%
  }%
\x\catcode61\catcode48\catcode32=10\relax%
\catcode13=5 % ^^M
\endlinechar=13 %
\catcode35=6 % #
\catcode64=11 % @
\catcode123=1 % {
\catcode125=2 % }
\def\TMP@EnsureCode#1#2{%
  \edef\GTS@AtEnd{%
    \GTS@AtEnd
    \catcode#1=\the\catcode#1\relax
  }%
  \catcode#1=#2\relax
}
\TMP@EnsureCode{42}{12}% *
\TMP@EnsureCode{44}{12}% ,
\TMP@EnsureCode{45}{12}% -
\TMP@EnsureCode{46}{12}% .
\TMP@EnsureCode{47}{12}% /
\TMP@EnsureCode{91}{12}% [
\TMP@EnsureCode{93}{12}% ]
\edef\GTS@AtEnd{\GTS@AtEnd\noexpand\endinput}
%    \end{macrocode}
%
% \subsection{Options}
%
%    \begin{macrocode}
\RequirePackage{kvoptions}[2009/07/17]
\SetupKeyvalOptions{%
  family=gettitlestring,%
  prefix=GTS@%
}
\newcommand*{\GetTitleStringSetup}{%
  \setkeys{gettitlestring}%
}
\DeclareBoolOption{expand}
\InputIfFileExists{gettitlestring.cfg}{}{}
\ProcessKeyvalOptions*\relax
%    \end{macrocode}
%
% \subsection{\cs{GetTitleString}}
%
%    \begin{macro}{\GetTitleString}
%    \begin{macrocode}
\newcommand*{\GetTitleString}{%
  \ifGTS@expand
    \expandafter\GetTitleStringExpand
  \else
    \expandafter\GetTitleStringNonExpand
  \fi
}
%    \end{macrocode}
%    \end{macro}
%    \begin{macro}{\GetTitleStringExpand}
%    \begin{macrocode}
\newcommand{\GetTitleStringExpand}[1]{%
  \def\GetTitleStringResult{#1}%
  \begingroup
    \GTS@DisablePredefinedCmds
    \GTS@DisableHook
    \edef\x{\endgroup
      \noexpand\def\noexpand\GetTitleStringResult{%
        \GetTitleStringResult
      }%
    }%
  \x
}
%    \end{macrocode}
%    \end{macro}
%    \begin{macro}{\GetTitleString}
%    \begin{macrocode}
\newcommand{\GetTitleStringNonExpand}[1]{%
  \def\GetTitleStringResult{#1}%
  \global\let\GTS@GlobalString\GetTitleStringResult
  \begingroup
    \GTS@RemoveLeft
    \GTS@RemoveRight
  \endgroup
  \let\GetTitleStringResult\GTS@GlobalString
}
%    \end{macrocode}
%    \end{macro}
%
% \subsubsection{Expand method}
%
%    \begin{macro}{\GTS@DisablePredefinedCmds}
%    \begin{macrocode}
\def\GTS@DisablePredefinedCmds{%
  \let\label\@gobble
  \let\zlabel\@gobble
  \let\zref@label\@gobble
  \let\zref@labelbylist\@gobbletwo
  \let\zref@labelbyprops\@gobbletwo
  \let\index\@gobble
  \let\glossary\@gobble
  \let\markboth\@gobbletwo
  \let\@mkboth\@gobbletwo
  \let\markright\@gobble
  \let\phantomsection\@empty
  \def\addcontentsline{\expandafter\@gobble\@gobbletwo}%
  \let\raggedright\@empty
  \let\raggedleft\@empty
  \let\centering\@empty
  \let\protect\@unexpandable@protect
  \let\enit@format\@empty % package enumitem
}
%    \end{macrocode}
%    \end{macro}
%
%    \begin{macro}{\GTS@DisableHook}
%    \begin{macrocode}
\providecommand*{\GTS@DisableHook}{}
%    \end{macrocode}
%    \end{macro}
%    \begin{macro}{\GetTitleStringDisableCommands}
%    \begin{macrocode}
\def\GetTitleStringDisableCommands{%
  \begingroup
    \makeatletter
    \GTS@DisableCommands
}
%    \end{macrocode}
%    \end{macro}
%    \begin{macro}{\GTS@DisableCommands}
%    \begin{macrocode}
\long\def\GTS@DisableCommands#1{%
    \toks0=\expandafter{\GTS@DisableHook}%
    \toks2={#1}%
    \xdef\GTS@GlobalString{\the\toks0 \the\toks2}%
  \endgroup
  \let\GTS@DisableHook\GTS@GlobalString
}
%    \end{macrocode}
%    \end{macro}
%
% \subsubsection{Non-expand method}
%
%    \begin{macrocode}
\def\GTS@RemoveLeft{%
  \toks@\expandafter\expandafter\expandafter{%
    \expandafter\GTS@Car\GTS@GlobalString{}{}{}{}\GTS@Nil
  }%
  \edef\GTS@Token{\the\toks@}%
  \GTS@PredefinedLeftCmds
  \expandafter\futurelet\expandafter\GTS@Token
  \expandafter\GTS@TestLeftSpace\GTS@GlobalString\GTS@Nil
  \GTS@End
}
\def\GTS@End{}
\long\def\GTS@TestLeft#1#2{%
  \def\GTS@temp{#1}%
  \ifx\GTS@temp\GTS@Token
    \toks@\expandafter\expandafter\expandafter{%
      \expandafter#2\GTS@GlobalString\GTS@Nil
    }%
    \expandafter\GTS@TestLeftEnd
  \fi
}
\long\def\GTS@TestLeftEnd#1\GTS@End{%
  \xdef\GTS@GlobalString{\the\toks@}%
  \GTS@RemoveLeft
}
\long\def\GTS@Car#1#2\GTS@Nil{#1}
\long\def\GTS@Cdr#1#2\GTS@Nil{#2}
\long\def\GTS@CdrTwo#1#2#3\GTS@Nil{#3}
\long\def\GTS@CdrThree#1#2#3#4\GTS@Nil{#4}
\long\def\GTS@CdrFour#1#2#3#4#5\GTS@Nil{#5}
\long\def\GTS@TestLeftSpace#1\GTS@Nil{%
  \ifx\GTS@Token\@sptoken
    \toks@\expandafter{%
      \romannumeral-0\GTS@GlobalString
    }%
    \expandafter\GTS@TestLeftEnd
  \fi
}
%    \end{macrocode}
%    \begin{macro}{\GTS@PredefinedLeftCmds}
%    \begin{macrocode}
\def\GTS@PredefinedLeftCmds{%
  \GTS@TestLeft\Hy@phantomsection\GTS@Cdr
  \GTS@TestLeft\Hy@SectionAnchor\GTS@Cdr
  \GTS@TestLeft\Hy@SectionAnchorHref\GTS@CdrTwo
  \GTS@TestLeft\label\GTS@CdrTwo
  \GTS@TestLeft\zlabel\GTS@CdrTwo
  \GTS@TestLeft\index\GTS@CdrTwo
  \GTS@TestLeft\glossary\GTS@CdrTwo
  \GTS@TestLeft\markboth\GTS@CdrThree
  \GTS@TestLeft\@mkboth\GTS@CdrThree
  \GTS@TestLeft\addcontentsline\GTS@CdrFour
  \GTS@TestLeft\enit@format\GTS@Cdr % package enumitem
}
%    \end{macrocode}
%    \end{macro}
%
%    \begin{macrocode}
\def\GTS@RemoveRight{%
  \toks@{}%
  \expandafter\GTS@TestRightLabel\GTS@GlobalString
      \label{}\GTS@Nil\@nil
  \GTS@RemoveRightSpace
}
\begingroup
  \def\GTS@temp#1{\endgroup
    \def\GTS@RemoveRightSpace{%
      \expandafter\GTS@TestRightSpace\GTS@GlobalString
          \GTS@Nil#1\GTS@Nil\@nil
    }%
  }%
\GTS@temp{ }
\def\GTS@TestRightSpace#1 \GTS@Nil#2\@nil{%
  \ifx\relax#2\relax
  \else
    \gdef\GTS@GlobalString{#1}%
    \expandafter\GTS@RemoveRightSpace
  \fi
}
\def\GTS@TestRightLabel#1\label#2#3\GTS@Nil#4\@nil{%
  \def\GTS@temp{#3}%
  \ifx\GTS@temp\@empty
    \expandafter\gdef\expandafter\GTS@GlobalString\expandafter{%
      \the\toks@
      #1%
    }%
    \expandafter\@gobble
  \else
    \expandafter\@firstofone
  \fi
  {%
    \toks@\expandafter{\the\toks@#1}%
    \GTS@TestRightLabel#3\GTS@Nil\@nil
  }%
}
%    \end{macrocode}
%
%    \begin{macrocode}
\GTS@AtEnd%
%</package>
%    \end{macrocode}
%
% \section{Test}
%
% \subsection{Catcode checks for loading}
%
%    \begin{macrocode}
%<*test1>
%    \end{macrocode}
%    \begin{macrocode}
\catcode`\{=1 %
\catcode`\}=2 %
\catcode`\#=6 %
\catcode`\@=11 %
\expandafter\ifx\csname count@\endcsname\relax
  \countdef\count@=255 %
\fi
\expandafter\ifx\csname @gobble\endcsname\relax
  \long\def\@gobble#1{}%
\fi
\expandafter\ifx\csname @firstofone\endcsname\relax
  \long\def\@firstofone#1{#1}%
\fi
\expandafter\ifx\csname loop\endcsname\relax
  \expandafter\@firstofone
\else
  \expandafter\@gobble
\fi
{%
  \def\loop#1\repeat{%
    \def\body{#1}%
    \iterate
  }%
  \def\iterate{%
    \body
      \let\next\iterate
    \else
      \let\next\relax
    \fi
    \next
  }%
  \let\repeat=\fi
}%
\def\RestoreCatcodes{}
\count@=0 %
\loop
  \edef\RestoreCatcodes{%
    \RestoreCatcodes
    \catcode\the\count@=\the\catcode\count@\relax
  }%
\ifnum\count@<255 %
  \advance\count@ 1 %
\repeat

\def\RangeCatcodeInvalid#1#2{%
  \count@=#1\relax
  \loop
    \catcode\count@=15 %
  \ifnum\count@<#2\relax
    \advance\count@ 1 %
  \repeat
}
\def\RangeCatcodeCheck#1#2#3{%
  \count@=#1\relax
  \loop
    \ifnum#3=\catcode\count@
    \else
      \errmessage{%
        Character \the\count@\space
        with wrong catcode \the\catcode\count@\space
        instead of \number#3%
      }%
    \fi
  \ifnum\count@<#2\relax
    \advance\count@ 1 %
  \repeat
}
\def\space{ }
\expandafter\ifx\csname LoadCommand\endcsname\relax
  \def\LoadCommand{\input gettitlestring.sty\relax}%
\fi
\def\Test{%
  \RangeCatcodeInvalid{0}{47}%
  \RangeCatcodeInvalid{58}{64}%
  \RangeCatcodeInvalid{91}{96}%
  \RangeCatcodeInvalid{123}{255}%
  \catcode`\@=12 %
  \catcode`\\=0 %
  \catcode`\%=14 %
  \LoadCommand
  \RangeCatcodeCheck{0}{36}{15}%
  \RangeCatcodeCheck{37}{37}{14}%
  \RangeCatcodeCheck{38}{47}{15}%
  \RangeCatcodeCheck{48}{57}{12}%
  \RangeCatcodeCheck{58}{63}{15}%
  \RangeCatcodeCheck{64}{64}{12}%
  \RangeCatcodeCheck{65}{90}{11}%
  \RangeCatcodeCheck{91}{91}{15}%
  \RangeCatcodeCheck{92}{92}{0}%
  \RangeCatcodeCheck{93}{96}{15}%
  \RangeCatcodeCheck{97}{122}{11}%
  \RangeCatcodeCheck{123}{255}{15}%
  \RestoreCatcodes
}
\Test
\csname @@end\endcsname
\end
%    \end{macrocode}
%    \begin{macrocode}
%</test1>
%    \end{macrocode}
%
% \subsection{Test of non-expand method}
%
%    \begin{macrocode}
%<*test2>
\NeedsTeXFormat{LaTeX2e}
\documentclass{minimal}
\usepackage{gettitlestring}[2016/05/16]
\usepackage{qstest}
\IncludeTests{*}
\LogTests{log}{*}{*}
\begin{document}
\begin{qstest}{non-expand}{non-expand}
  \def\test#1#2{%
    \sbox0{%
      \GetTitleString{#1}%
      \Expect{#2}*{\GetTitleStringResult}%
    }%
    \Expect{0.0pt}*{\the\wd0}%
  }%
  \test{}{}%
  \test{ }{}%
  \test{ x }{x}%
  \test{ x y }{x y}%
  \test{ \relax}{\relax}%
  \test{\label{f}a}{a}%
  \test{ \label{f}a}{a}%
  \test{\label{f} a}{a}%
  \test{ \label{f} a}{a}%
  \test{a\label{f}}{a}%
  \test{a\label{f} }{a}%
  \test{a \label{f}}{a}%
  \test{a \label{f} }{a}%
  \test{a\label{f}b\label{g}}{ab}%
  \test{a \label{f}b \label{g} }{a b}%
  \test{a\label{f} b \label{g} }{a b}%
\end{qstest}
\end{document}
%</test2>
%    \end{macrocode}
%
% \section{Installation}
%
% \subsection{Download}
%
% \paragraph{Package.} This package is available on
% CTAN\footnote{\CTANpkg{gettitlestring}}:
% \begin{description}
% \item[\CTAN{macros/latex/contrib/oberdiek/gettitlestring.dtx}] The source file.
% \item[\CTAN{macros/latex/contrib/oberdiek/gettitlestring.pdf}] Documentation.
% \end{description}
%
%
% \paragraph{Bundle.} All the packages of the bundle `oberdiek'
% are also available in a TDS compliant ZIP archive. There
% the packages are already unpacked and the documentation files
% are generated. The files and directories obey the TDS standard.
% \begin{description}
% \item[\CTANinstall{install/macros/latex/contrib/oberdiek.tds.zip}]
% \end{description}
% \emph{TDS} refers to the standard ``A Directory Structure
% for \TeX\ Files'' (\CTAN{tds/tds.pdf}). Directories
% with \xfile{texmf} in their name are usually organized this way.
%
% \subsection{Bundle installation}
%
% \paragraph{Unpacking.} Unpack the \xfile{oberdiek.tds.zip} in the
% TDS tree (also known as \xfile{texmf} tree) of your choice.
% Example (linux):
% \begin{quote}
%   |unzip oberdiek.tds.zip -d ~/texmf|
% \end{quote}
%
% \paragraph{Script installation.}
% Check the directory \xfile{TDS:scripts/oberdiek/} for
% scripts that need further installation steps.
% Package \xpackage{attachfile2} comes with the Perl script
% \xfile{pdfatfi.pl} that should be installed in such a way
% that it can be called as \texttt{pdfatfi}.
% Example (linux):
% \begin{quote}
%   |chmod +x scripts/oberdiek/pdfatfi.pl|\\
%   |cp scripts/oberdiek/pdfatfi.pl /usr/local/bin/|
% \end{quote}
%
% \subsection{Package installation}
%
% \paragraph{Unpacking.} The \xfile{.dtx} file is a self-extracting
% \docstrip\ archive. The files are extracted by running the
% \xfile{.dtx} through \plainTeX:
% \begin{quote}
%   \verb|tex gettitlestring.dtx|
% \end{quote}
%
% \paragraph{TDS.} Now the different files must be moved into
% the different directories in your installation TDS tree
% (also known as \xfile{texmf} tree):
% \begin{quote}
% \def\t{^^A
% \begin{tabular}{@{}>{\ttfamily}l@{ $\rightarrow$ }>{\ttfamily}l@{}}
%   gettitlestring.sty & tex/generic/oberdiek/gettitlestring.sty\\
%   gettitlestring.pdf & doc/latex/oberdiek/gettitlestring.pdf\\
%   test/gettitlestring-test1.tex & doc/latex/oberdiek/test/gettitlestring-test1.tex\\
%   test/gettitlestring-test2.tex & doc/latex/oberdiek/test/gettitlestring-test2.tex\\
%   gettitlestring.dtx & source/latex/oberdiek/gettitlestring.dtx\\
% \end{tabular}^^A
% }^^A
% \sbox0{\t}^^A
% \ifdim\wd0>\linewidth
%   \begingroup
%     \advance\linewidth by\leftmargin
%     \advance\linewidth by\rightmargin
%   \edef\x{\endgroup
%     \def\noexpand\lw{\the\linewidth}^^A
%   }\x
%   \def\lwbox{^^A
%     \leavevmode
%     \hbox to \linewidth{^^A
%       \kern-\leftmargin\relax
%       \hss
%       \usebox0
%       \hss
%       \kern-\rightmargin\relax
%     }^^A
%   }^^A
%   \ifdim\wd0>\lw
%     \sbox0{\small\t}^^A
%     \ifdim\wd0>\linewidth
%       \ifdim\wd0>\lw
%         \sbox0{\footnotesize\t}^^A
%         \ifdim\wd0>\linewidth
%           \ifdim\wd0>\lw
%             \sbox0{\scriptsize\t}^^A
%             \ifdim\wd0>\linewidth
%               \ifdim\wd0>\lw
%                 \sbox0{\tiny\t}^^A
%                 \ifdim\wd0>\linewidth
%                   \lwbox
%                 \else
%                   \usebox0
%                 \fi
%               \else
%                 \lwbox
%               \fi
%             \else
%               \usebox0
%             \fi
%           \else
%             \lwbox
%           \fi
%         \else
%           \usebox0
%         \fi
%       \else
%         \lwbox
%       \fi
%     \else
%       \usebox0
%     \fi
%   \else
%     \lwbox
%   \fi
% \else
%   \usebox0
% \fi
% \end{quote}
% If you have a \xfile{docstrip.cfg} that configures and enables \docstrip's
% TDS installing feature, then some files can already be in the right
% place, see the documentation of \docstrip.
%
% \subsection{Refresh file name databases}
%
% If your \TeX~distribution
% (\teTeX, \mikTeX, \dots) relies on file name databases, you must refresh
% these. For example, \teTeX\ users run \verb|texhash| or
% \verb|mktexlsr|.
%
% \subsection{Some details for the interested}
%
% \paragraph{Attached source.}
%
% The PDF documentation on CTAN also includes the
% \xfile{.dtx} source file. It can be extracted by
% AcrobatReader 6 or higher. Another option is \textsf{pdftk},
% e.g. unpack the file into the current directory:
% \begin{quote}
%   \verb|pdftk gettitlestring.pdf unpack_files output .|
% \end{quote}
%
% \paragraph{Unpacking with \LaTeX.}
% The \xfile{.dtx} chooses its action depending on the format:
% \begin{description}
% \item[\plainTeX:] Run \docstrip\ and extract the files.
% \item[\LaTeX:] Generate the documentation.
% \end{description}
% If you insist on using \LaTeX\ for \docstrip\ (really,
% \docstrip\ does not need \LaTeX), then inform the autodetect routine
% about your intention:
% \begin{quote}
%   \verb|latex \let\install=y\input{gettitlestring.dtx}|
% \end{quote}
% Do not forget to quote the argument according to the demands
% of your shell.
%
% \paragraph{Generating the documentation.}
% You can use both the \xfile{.dtx} or the \xfile{.drv} to generate
% the documentation. The process can be configured by the
% configuration file \xfile{ltxdoc.cfg}. For instance, put this
% line into this file, if you want to have A4 as paper format:
% \begin{quote}
%   \verb|\PassOptionsToClass{a4paper}{article}|
% \end{quote}
% An example follows how to generate the
% documentation with pdf\LaTeX:
% \begin{quote}
%\begin{verbatim}
%pdflatex gettitlestring.dtx
%makeindex -s gind.ist gettitlestring.idx
%pdflatex gettitlestring.dtx
%makeindex -s gind.ist gettitlestring.idx
%pdflatex gettitlestring.dtx
%\end{verbatim}
% \end{quote}
%
% \begin{thebibliography}{9}
%
% \bibitem{memoir}
% Peter Wilson, Lars Madsen:
% \textit{The Memoir Class};
% 2009/11/17 v1.61803398c;
% \CTANpkg{memoir}
%
% \bibitem{titleref}
% Donald Arsenau:
% \textit{Titleref.sty};
% 2001/04/05 ver 3.1;
% \CTAN{macros/latex/contrib/misc/titleref.sty}
%
% \bibitem{zref}
% Heiko Oberdiek:
% \textit{The \xpackage{zref} package};
% 2009/12/08 v2.7;
% \CTAN{macros/latex/contrib/oberdiek/zref.pdf}
%
% \end{thebibliography}
%
% \begin{History}
%   \begin{Version}{2009/12/08 v1.0}
%   \item
%     The first version.
%   \end{Version}
%   \begin{Version}{2009/12/12 v1.1}
%   \item
%     Short info shortened.
%   \end{Version}
%   \begin{Version}{2009/12/13 v1.2}
%   \item
%     Forgotten third argument for \cs{InputIfFileExists} added.
%   \end{Version}
%   \begin{Version}{2009/12/18 v1.3}
%   \item
%     \cs{Hy@SectionAnchorHref} added for filtering
%     (hyperref 2009/12/18 v6.79w).
%   \end{Version}
%   \begin{Version}{2010/12/03 v1.4}
%   \item
%     Support of package \xpackage{enumitem}: removing
%     \cs{enit@format} from title string (problem report by GL).
%   \end{Version}
%   \begin{Version}{2016/05/16 v1.5}
%   \item
%     Documentation updates.
%   \end{Version}
% \end{History}
%
% \PrintIndex
%
% \Finale
\endinput
|
% \end{quote}
% Do not forget to quote the argument according to the demands
% of your shell.
%
% \paragraph{Generating the documentation.}
% You can use both the \xfile{.dtx} or the \xfile{.drv} to generate
% the documentation. The process can be configured by the
% configuration file \xfile{ltxdoc.cfg}. For instance, put this
% line into this file, if you want to have A4 as paper format:
% \begin{quote}
%   \verb|\PassOptionsToClass{a4paper}{article}|
% \end{quote}
% An example follows how to generate the
% documentation with pdf\LaTeX:
% \begin{quote}
%\begin{verbatim}
%pdflatex gettitlestring.dtx
%makeindex -s gind.ist gettitlestring.idx
%pdflatex gettitlestring.dtx
%makeindex -s gind.ist gettitlestring.idx
%pdflatex gettitlestring.dtx
%\end{verbatim}
% \end{quote}
%
% \begin{thebibliography}{9}
%
% \bibitem{memoir}
% Peter Wilson, Lars Madsen:
% \textit{The Memoir Class};
% 2009/11/17 v1.61803398c;
% \CTANpkg{memoir}
%
% \bibitem{titleref}
% Donald Arsenau:
% \textit{Titleref.sty};
% 2001/04/05 ver 3.1;
% \CTAN{macros/latex/contrib/misc/titleref.sty}
%
% \bibitem{zref}
% Heiko Oberdiek:
% \textit{The \xpackage{zref} package};
% 2009/12/08 v2.7;
% \CTAN{macros/latex/contrib/oberdiek/zref.pdf}
%
% \end{thebibliography}
%
% \begin{History}
%   \begin{Version}{2009/12/08 v1.0}
%   \item
%     The first version.
%   \end{Version}
%   \begin{Version}{2009/12/12 v1.1}
%   \item
%     Short info shortened.
%   \end{Version}
%   \begin{Version}{2009/12/13 v1.2}
%   \item
%     Forgotten third argument for \cs{InputIfFileExists} added.
%   \end{Version}
%   \begin{Version}{2009/12/18 v1.3}
%   \item
%     \cs{Hy@SectionAnchorHref} added for filtering
%     (hyperref 2009/12/18 v6.79w).
%   \end{Version}
%   \begin{Version}{2010/12/03 v1.4}
%   \item
%     Support of package \xpackage{enumitem}: removing
%     \cs{enit@format} from title string (problem report by GL).
%   \end{Version}
%   \begin{Version}{2016/05/16 v1.5}
%   \item
%     Documentation updates.
%   \end{Version}
% \end{History}
%
% \PrintIndex
%
% \Finale
\endinput

%        (quote the arguments according to the demands of your shell)
%
% Documentation:
%    (a) If gettitlestring.drv is present:
%           latex gettitlestring.drv
%    (b) Without gettitlestring.drv:
%           latex gettitlestring.dtx; ...
%    The class ltxdoc loads the configuration file ltxdoc.cfg
%    if available. Here you can specify further options, e.g.
%    use A4 as paper format:
%       \PassOptionsToClass{a4paper}{article}
%
%    Programm calls to get the documentation (example):
%       pdflatex gettitlestring.dtx
%       makeindex -s gind.ist gettitlestring.idx
%       pdflatex gettitlestring.dtx
%       makeindex -s gind.ist gettitlestring.idx
%       pdflatex gettitlestring.dtx
%
% Installation:
%    TDS:tex/generic/oberdiek/gettitlestring.sty
%    TDS:doc/latex/oberdiek/gettitlestring.pdf
%    TDS:doc/latex/oberdiek/test/gettitlestring-test1.tex
%    TDS:doc/latex/oberdiek/test/gettitlestring-test2.tex
%    TDS:source/latex/oberdiek/gettitlestring.dtx
%
%<*ignore>
\begingroup
  \catcode123=1 %
  \catcode125=2 %
  \def\x{LaTeX2e}%
\expandafter\endgroup
\ifcase 0\ifx\install y1\fi\expandafter
         \ifx\csname processbatchFile\endcsname\relax\else1\fi
         \ifx\fmtname\x\else 1\fi\relax
\else\csname fi\endcsname
%</ignore>
%<*install>
\input docstrip.tex
\Msg{************************************************************************}
\Msg{* Installation}
\Msg{* Package: gettitlestring 2016/05/16 v1.5 Cleanup title references (HO)}
\Msg{************************************************************************}

\keepsilent
\askforoverwritefalse

\let\MetaPrefix\relax
\preamble

This is a generated file.

Project: gettitlestring
Version: 2016/05/16 v1.5

Copyright (C) 2009, 2010 by
   Heiko Oberdiek <heiko.oberdiek at googlemail.com>

This work may be distributed and/or modified under the
conditions of the LaTeX Project Public License, either
version 1.3c of this license or (at your option) any later
version. This version of this license is in
   http://www.latex-project.org/lppl/lppl-1-3c.txt
and the latest version of this license is in
   http://www.latex-project.org/lppl.txt
and version 1.3 or later is part of all distributions of
LaTeX version 2005/12/01 or later.

This work has the LPPL maintenance status "maintained".

This Current Maintainer of this work is Heiko Oberdiek.

The Base Interpreter refers to any `TeX-Format',
because some files are installed in TDS:tex/generic//.

This work consists of the main source file gettitlestring.dtx
and the derived files
   gettitlestring.sty, gettitlestring.pdf, gettitlestring.ins,
   gettitlestring.drv, gettitlestring-test1.tex,
   gettitlestring-test2.tex.

\endpreamble
\let\MetaPrefix\DoubleperCent

\generate{%
  \file{gettitlestring.ins}{\from{gettitlestring.dtx}{install}}%
  \file{gettitlestring.drv}{\from{gettitlestring.dtx}{driver}}%
  \usedir{tex/generic/oberdiek}%
  \file{gettitlestring.sty}{\from{gettitlestring.dtx}{package}}%
%  \usedir{doc/latex/oberdiek/test}%
%  \file{gettitlestring-test1.tex}{\from{gettitlestring.dtx}{test1}}%
%  \file{gettitlestring-test2.tex}{\from{gettitlestring.dtx}{test2}}%
  \nopreamble
  \nopostamble
%  \usedir{source/latex/oberdiek/catalogue}%
%  \file{gettitlestring.xml}{\from{gettitlestring.dtx}{catalogue}}%
}

\catcode32=13\relax% active space
\let =\space%
\Msg{************************************************************************}
\Msg{*}
\Msg{* To finish the installation you have to move the following}
\Msg{* file into a directory searched by TeX:}
\Msg{*}
\Msg{*     gettitlestring.sty}
\Msg{*}
\Msg{* To produce the documentation run the file `gettitlestring.drv'}
\Msg{* through LaTeX.}
\Msg{*}
\Msg{* Happy TeXing!}
\Msg{*}
\Msg{************************************************************************}

\endbatchfile
%</install>
%<*ignore>
\fi
%</ignore>
%<*driver>
\NeedsTeXFormat{LaTeX2e}
\ProvidesFile{gettitlestring.drv}%
  [2016/05/16 v1.5 Cleanup title references (HO)]%
\documentclass{ltxdoc}
\usepackage{holtxdoc}[2011/11/22]
\begin{document}
  \DocInput{gettitlestring.dtx}%
\end{document}
%</driver>
% \fi
%
%
% \CharacterTable
%  {Upper-case    \A\B\C\D\E\F\G\H\I\J\K\L\M\N\O\P\Q\R\S\T\U\V\W\X\Y\Z
%   Lower-case    \a\b\c\d\e\f\g\h\i\j\k\l\m\n\o\p\q\r\s\t\u\v\w\x\y\z
%   Digits        \0\1\2\3\4\5\6\7\8\9
%   Exclamation   \!     Double quote  \"     Hash (number) \#
%   Dollar        \$     Percent       \%     Ampersand     \&
%   Acute accent  \'     Left paren    \(     Right paren   \)
%   Asterisk      \*     Plus          \+     Comma         \,
%   Minus         \-     Point         \.     Solidus       \/
%   Colon         \:     Semicolon     \;     Less than     \<
%   Equals        \=     Greater than  \>     Question mark \?
%   Commercial at \@     Left bracket  \[     Backslash     \\
%   Right bracket \]     Circumflex    \^     Underscore    \_
%   Grave accent  \`     Left brace    \{     Vertical bar  \|
%   Right brace   \}     Tilde         \~}
%
% \GetFileInfo{gettitlestring.drv}
%
% \title{The \xpackage{gettitlestring} package}
% \date{2016/05/16 v1.5}
% \author{Heiko Oberdiek\thanks
% {Please report any issues at \url{https://github.com/ho-tex/oberdiek/issues}}\\
% \xemail{heiko.oberdiek at googlemail.com}}
%
% \maketitle
%
% \begin{abstract}
% The \LaTeX\ package addresses packages that are dealing with
% references to titles (\cs{section}, \cs{caption}, \dots).
% The package tries to remove \cs{label} and other
% commands from title strings.
% \end{abstract}
%
% \tableofcontents
%
% \section{Documentation}
%
% \subsection{Macros}
%
% \begin{declcs}{GetTitleStringSetup} \M{key value list}
% \end{declcs}
% The options are given as comma separated key value pairs.
% See section \ref{sec:options}.
%
% \begin{declcs}{GetTitleString} \M{text}\\
% \cs{GetTitleStringExpand} \M{text}\\
% \cs{GetTitleStringNonExpand} \M{text}
% \end{declcs}
% Macro \cs{GetTitleString} tries to remove unwanted stuff from \meta{text}
% the result is stored in Macro \cs{GetTitleStringResult}.
% Two methods are available:
% \begin{description}
% \item[\cs{GetTitleStringExpand}:]
% The \meta{text} is expanded in a context where the unwanted
% macros are redefined to remove themselves.
% This is the method used in packages \xpackage{titleref}~\cite{titleref},
% \xpackage{zref-titleref}~\cite{zref}
% or class \xclass{memoir}~\cite{memoir}.
% \cs{protect} is supported, but fragile material might break.
% \item[\cs{GetTitleStringNonExpand}:]
% The \meta{text} is not expanded. Thus the removal of unwanted
% material is more difficult. It is especially removed at the
% start of the \meta{text} and spaces are removed from the end.
% Currently only \cs{label} is removed in the whole string,
% if it is not hidden inside curly braces or part of macro
% definitions. Thus the removal of unwanted stuff might not be
% complete, but fragile material will not break.
% (But the result string can break at a later time, of course).
% \end{description}
% Option \xoption{expand} controls which method is used by
% macro \cs{GetTitleString}.
%
% \begin{declcs}{GetTitleStringDisableCommands} \M{code}
% \end{declcs}
% The \meta{code} is called right before the
% text is expanded in \cs{GetTitleStringExpand}.
% Additional definitions can be given for macros that
% should be removed.
% Keep in mind that expansion means that the definitions
% must work in expandable context. Macros like
% \cs{@ifstar} or \cs{@ifnextchar} or optional arguments
% will not work. The macro names in \meta{code} may contain
% the at sign |@|, it has catcode 11 (letter).
%
% \subsection{Options}\label{sec:options}
%
% \begin{description}
% \item[\xoption{expand}:] Boolean option, takes values |true| or |false|.
% No value means |true|. The option specifies the method to remove
% unwanted stuff from the title string, see below.
% \end{description}
% Options can be set at the following places:
% \begin{itemize}
% \item \cs{usepackage}
% \item Configuration file \xfile{gettitlestring.cfg}.
% \item \cs{GetTitleStringSetup}
% \end{itemize}
%
% \StopEventually{
% }
%
% \section{Implementation}
%    \begin{macrocode}
%<*package>
%    \end{macrocode}
%    Reload check, especially if the package is not used with \LaTeX.
%    \begin{macrocode}
\begingroup\catcode61\catcode48\catcode32=10\relax%
  \catcode13=5 % ^^M
  \endlinechar=13 %
  \catcode35=6 % #
  \catcode39=12 % '
  \catcode44=12 % ,
  \catcode45=12 % -
  \catcode46=12 % .
  \catcode58=12 % :
  \catcode64=11 % @
  \catcode123=1 % {
  \catcode125=2 % }
  \expandafter\let\expandafter\x\csname ver@gettitlestring.sty\endcsname
  \ifx\x\relax % plain-TeX, first loading
  \else
    \def\empty{}%
    \ifx\x\empty % LaTeX, first loading,
      % variable is initialized, but \ProvidesPackage not yet seen
    \else
      \expandafter\ifx\csname PackageInfo\endcsname\relax
        \def\x#1#2{%
          \immediate\write-1{Package #1 Info: #2.}%
        }%
      \else
        \def\x#1#2{\PackageInfo{#1}{#2, stopped}}%
      \fi
      \x{gettitlestring}{The package is already loaded}%
      \aftergroup\endinput
    \fi
  \fi
\endgroup%
%    \end{macrocode}
%    Package identification:
%    \begin{macrocode}
\begingroup\catcode61\catcode48\catcode32=10\relax%
  \catcode13=5 % ^^M
  \endlinechar=13 %
  \catcode35=6 % #
  \catcode39=12 % '
  \catcode40=12 % (
  \catcode41=12 % )
  \catcode44=12 % ,
  \catcode45=12 % -
  \catcode46=12 % .
  \catcode47=12 % /
  \catcode58=12 % :
  \catcode64=11 % @
  \catcode91=12 % [
  \catcode93=12 % ]
  \catcode123=1 % {
  \catcode125=2 % }
  \expandafter\ifx\csname ProvidesPackage\endcsname\relax
    \def\x#1#2#3[#4]{\endgroup
      \immediate\write-1{Package: #3 #4}%
      \xdef#1{#4}%
    }%
  \else
    \def\x#1#2[#3]{\endgroup
      #2[{#3}]%
      \ifx#1\@undefined
        \xdef#1{#3}%
      \fi
      \ifx#1\relax
        \xdef#1{#3}%
      \fi
    }%
  \fi
\expandafter\x\csname ver@gettitlestring.sty\endcsname
\ProvidesPackage{gettitlestring}%
  [2016/05/16 v1.5 Cleanup title references (HO)]%
%    \end{macrocode}
%
%    \begin{macrocode}
\begingroup\catcode61\catcode48\catcode32=10\relax%
  \catcode13=5 % ^^M
  \endlinechar=13 %
  \catcode123=1 % {
  \catcode125=2 % }
  \catcode64=11 % @
  \def\x{\endgroup
    \expandafter\edef\csname GTS@AtEnd\endcsname{%
      \endlinechar=\the\endlinechar\relax
      \catcode13=\the\catcode13\relax
      \catcode32=\the\catcode32\relax
      \catcode35=\the\catcode35\relax
      \catcode61=\the\catcode61\relax
      \catcode64=\the\catcode64\relax
      \catcode123=\the\catcode123\relax
      \catcode125=\the\catcode125\relax
    }%
  }%
\x\catcode61\catcode48\catcode32=10\relax%
\catcode13=5 % ^^M
\endlinechar=13 %
\catcode35=6 % #
\catcode64=11 % @
\catcode123=1 % {
\catcode125=2 % }
\def\TMP@EnsureCode#1#2{%
  \edef\GTS@AtEnd{%
    \GTS@AtEnd
    \catcode#1=\the\catcode#1\relax
  }%
  \catcode#1=#2\relax
}
\TMP@EnsureCode{42}{12}% *
\TMP@EnsureCode{44}{12}% ,
\TMP@EnsureCode{45}{12}% -
\TMP@EnsureCode{46}{12}% .
\TMP@EnsureCode{47}{12}% /
\TMP@EnsureCode{91}{12}% [
\TMP@EnsureCode{93}{12}% ]
\edef\GTS@AtEnd{\GTS@AtEnd\noexpand\endinput}
%    \end{macrocode}
%
% \subsection{Options}
%
%    \begin{macrocode}
\RequirePackage{kvoptions}[2009/07/17]
\SetupKeyvalOptions{%
  family=gettitlestring,%
  prefix=GTS@%
}
\newcommand*{\GetTitleStringSetup}{%
  \setkeys{gettitlestring}%
}
\DeclareBoolOption{expand}
\InputIfFileExists{gettitlestring.cfg}{}{}
\ProcessKeyvalOptions*\relax
%    \end{macrocode}
%
% \subsection{\cs{GetTitleString}}
%
%    \begin{macro}{\GetTitleString}
%    \begin{macrocode}
\newcommand*{\GetTitleString}{%
  \ifGTS@expand
    \expandafter\GetTitleStringExpand
  \else
    \expandafter\GetTitleStringNonExpand
  \fi
}
%    \end{macrocode}
%    \end{macro}
%    \begin{macro}{\GetTitleStringExpand}
%    \begin{macrocode}
\newcommand{\GetTitleStringExpand}[1]{%
  \def\GetTitleStringResult{#1}%
  \begingroup
    \GTS@DisablePredefinedCmds
    \GTS@DisableHook
    \edef\x{\endgroup
      \noexpand\def\noexpand\GetTitleStringResult{%
        \GetTitleStringResult
      }%
    }%
  \x
}
%    \end{macrocode}
%    \end{macro}
%    \begin{macro}{\GetTitleString}
%    \begin{macrocode}
\newcommand{\GetTitleStringNonExpand}[1]{%
  \def\GetTitleStringResult{#1}%
  \global\let\GTS@GlobalString\GetTitleStringResult
  \begingroup
    \GTS@RemoveLeft
    \GTS@RemoveRight
  \endgroup
  \let\GetTitleStringResult\GTS@GlobalString
}
%    \end{macrocode}
%    \end{macro}
%
% \subsubsection{Expand method}
%
%    \begin{macro}{\GTS@DisablePredefinedCmds}
%    \begin{macrocode}
\def\GTS@DisablePredefinedCmds{%
  \let\label\@gobble
  \let\zlabel\@gobble
  \let\zref@label\@gobble
  \let\zref@labelbylist\@gobbletwo
  \let\zref@labelbyprops\@gobbletwo
  \let\index\@gobble
  \let\glossary\@gobble
  \let\markboth\@gobbletwo
  \let\@mkboth\@gobbletwo
  \let\markright\@gobble
  \let\phantomsection\@empty
  \def\addcontentsline{\expandafter\@gobble\@gobbletwo}%
  \let\raggedright\@empty
  \let\raggedleft\@empty
  \let\centering\@empty
  \let\protect\@unexpandable@protect
  \let\enit@format\@empty % package enumitem
}
%    \end{macrocode}
%    \end{macro}
%
%    \begin{macro}{\GTS@DisableHook}
%    \begin{macrocode}
\providecommand*{\GTS@DisableHook}{}
%    \end{macrocode}
%    \end{macro}
%    \begin{macro}{\GetTitleStringDisableCommands}
%    \begin{macrocode}
\def\GetTitleStringDisableCommands{%
  \begingroup
    \makeatletter
    \GTS@DisableCommands
}
%    \end{macrocode}
%    \end{macro}
%    \begin{macro}{\GTS@DisableCommands}
%    \begin{macrocode}
\long\def\GTS@DisableCommands#1{%
    \toks0=\expandafter{\GTS@DisableHook}%
    \toks2={#1}%
    \xdef\GTS@GlobalString{\the\toks0 \the\toks2}%
  \endgroup
  \let\GTS@DisableHook\GTS@GlobalString
}
%    \end{macrocode}
%    \end{macro}
%
% \subsubsection{Non-expand method}
%
%    \begin{macrocode}
\def\GTS@RemoveLeft{%
  \toks@\expandafter\expandafter\expandafter{%
    \expandafter\GTS@Car\GTS@GlobalString{}{}{}{}\GTS@Nil
  }%
  \edef\GTS@Token{\the\toks@}%
  \GTS@PredefinedLeftCmds
  \expandafter\futurelet\expandafter\GTS@Token
  \expandafter\GTS@TestLeftSpace\GTS@GlobalString\GTS@Nil
  \GTS@End
}
\def\GTS@End{}
\long\def\GTS@TestLeft#1#2{%
  \def\GTS@temp{#1}%
  \ifx\GTS@temp\GTS@Token
    \toks@\expandafter\expandafter\expandafter{%
      \expandafter#2\GTS@GlobalString\GTS@Nil
    }%
    \expandafter\GTS@TestLeftEnd
  \fi
}
\long\def\GTS@TestLeftEnd#1\GTS@End{%
  \xdef\GTS@GlobalString{\the\toks@}%
  \GTS@RemoveLeft
}
\long\def\GTS@Car#1#2\GTS@Nil{#1}
\long\def\GTS@Cdr#1#2\GTS@Nil{#2}
\long\def\GTS@CdrTwo#1#2#3\GTS@Nil{#3}
\long\def\GTS@CdrThree#1#2#3#4\GTS@Nil{#4}
\long\def\GTS@CdrFour#1#2#3#4#5\GTS@Nil{#5}
\long\def\GTS@TestLeftSpace#1\GTS@Nil{%
  \ifx\GTS@Token\@sptoken
    \toks@\expandafter{%
      \romannumeral-0\GTS@GlobalString
    }%
    \expandafter\GTS@TestLeftEnd
  \fi
}
%    \end{macrocode}
%    \begin{macro}{\GTS@PredefinedLeftCmds}
%    \begin{macrocode}
\def\GTS@PredefinedLeftCmds{%
  \GTS@TestLeft\Hy@phantomsection\GTS@Cdr
  \GTS@TestLeft\Hy@SectionAnchor\GTS@Cdr
  \GTS@TestLeft\Hy@SectionAnchorHref\GTS@CdrTwo
  \GTS@TestLeft\label\GTS@CdrTwo
  \GTS@TestLeft\zlabel\GTS@CdrTwo
  \GTS@TestLeft\index\GTS@CdrTwo
  \GTS@TestLeft\glossary\GTS@CdrTwo
  \GTS@TestLeft\markboth\GTS@CdrThree
  \GTS@TestLeft\@mkboth\GTS@CdrThree
  \GTS@TestLeft\addcontentsline\GTS@CdrFour
  \GTS@TestLeft\enit@format\GTS@Cdr % package enumitem
}
%    \end{macrocode}
%    \end{macro}
%
%    \begin{macrocode}
\def\GTS@RemoveRight{%
  \toks@{}%
  \expandafter\GTS@TestRightLabel\GTS@GlobalString
      \label{}\GTS@Nil\@nil
  \GTS@RemoveRightSpace
}
\begingroup
  \def\GTS@temp#1{\endgroup
    \def\GTS@RemoveRightSpace{%
      \expandafter\GTS@TestRightSpace\GTS@GlobalString
          \GTS@Nil#1\GTS@Nil\@nil
    }%
  }%
\GTS@temp{ }
\def\GTS@TestRightSpace#1 \GTS@Nil#2\@nil{%
  \ifx\relax#2\relax
  \else
    \gdef\GTS@GlobalString{#1}%
    \expandafter\GTS@RemoveRightSpace
  \fi
}
\def\GTS@TestRightLabel#1\label#2#3\GTS@Nil#4\@nil{%
  \def\GTS@temp{#3}%
  \ifx\GTS@temp\@empty
    \expandafter\gdef\expandafter\GTS@GlobalString\expandafter{%
      \the\toks@
      #1%
    }%
    \expandafter\@gobble
  \else
    \expandafter\@firstofone
  \fi
  {%
    \toks@\expandafter{\the\toks@#1}%
    \GTS@TestRightLabel#3\GTS@Nil\@nil
  }%
}
%    \end{macrocode}
%
%    \begin{macrocode}
\GTS@AtEnd%
%</package>
%    \end{macrocode}
%
% \section{Test}
%
% \subsection{Catcode checks for loading}
%
%    \begin{macrocode}
%<*test1>
%    \end{macrocode}
%    \begin{macrocode}
\catcode`\{=1 %
\catcode`\}=2 %
\catcode`\#=6 %
\catcode`\@=11 %
\expandafter\ifx\csname count@\endcsname\relax
  \countdef\count@=255 %
\fi
\expandafter\ifx\csname @gobble\endcsname\relax
  \long\def\@gobble#1{}%
\fi
\expandafter\ifx\csname @firstofone\endcsname\relax
  \long\def\@firstofone#1{#1}%
\fi
\expandafter\ifx\csname loop\endcsname\relax
  \expandafter\@firstofone
\else
  \expandafter\@gobble
\fi
{%
  \def\loop#1\repeat{%
    \def\body{#1}%
    \iterate
  }%
  \def\iterate{%
    \body
      \let\next\iterate
    \else
      \let\next\relax
    \fi
    \next
  }%
  \let\repeat=\fi
}%
\def\RestoreCatcodes{}
\count@=0 %
\loop
  \edef\RestoreCatcodes{%
    \RestoreCatcodes
    \catcode\the\count@=\the\catcode\count@\relax
  }%
\ifnum\count@<255 %
  \advance\count@ 1 %
\repeat

\def\RangeCatcodeInvalid#1#2{%
  \count@=#1\relax
  \loop
    \catcode\count@=15 %
  \ifnum\count@<#2\relax
    \advance\count@ 1 %
  \repeat
}
\def\RangeCatcodeCheck#1#2#3{%
  \count@=#1\relax
  \loop
    \ifnum#3=\catcode\count@
    \else
      \errmessage{%
        Character \the\count@\space
        with wrong catcode \the\catcode\count@\space
        instead of \number#3%
      }%
    \fi
  \ifnum\count@<#2\relax
    \advance\count@ 1 %
  \repeat
}
\def\space{ }
\expandafter\ifx\csname LoadCommand\endcsname\relax
  \def\LoadCommand{\input gettitlestring.sty\relax}%
\fi
\def\Test{%
  \RangeCatcodeInvalid{0}{47}%
  \RangeCatcodeInvalid{58}{64}%
  \RangeCatcodeInvalid{91}{96}%
  \RangeCatcodeInvalid{123}{255}%
  \catcode`\@=12 %
  \catcode`\\=0 %
  \catcode`\%=14 %
  \LoadCommand
  \RangeCatcodeCheck{0}{36}{15}%
  \RangeCatcodeCheck{37}{37}{14}%
  \RangeCatcodeCheck{38}{47}{15}%
  \RangeCatcodeCheck{48}{57}{12}%
  \RangeCatcodeCheck{58}{63}{15}%
  \RangeCatcodeCheck{64}{64}{12}%
  \RangeCatcodeCheck{65}{90}{11}%
  \RangeCatcodeCheck{91}{91}{15}%
  \RangeCatcodeCheck{92}{92}{0}%
  \RangeCatcodeCheck{93}{96}{15}%
  \RangeCatcodeCheck{97}{122}{11}%
  \RangeCatcodeCheck{123}{255}{15}%
  \RestoreCatcodes
}
\Test
\csname @@end\endcsname
\end
%    \end{macrocode}
%    \begin{macrocode}
%</test1>
%    \end{macrocode}
%
% \subsection{Test of non-expand method}
%
%    \begin{macrocode}
%<*test2>
\NeedsTeXFormat{LaTeX2e}
\documentclass{minimal}
\usepackage{gettitlestring}[2016/05/16]
\usepackage{qstest}
\IncludeTests{*}
\LogTests{log}{*}{*}
\begin{document}
\begin{qstest}{non-expand}{non-expand}
  \def\test#1#2{%
    \sbox0{%
      \GetTitleString{#1}%
      \Expect{#2}*{\GetTitleStringResult}%
    }%
    \Expect{0.0pt}*{\the\wd0}%
  }%
  \test{}{}%
  \test{ }{}%
  \test{ x }{x}%
  \test{ x y }{x y}%
  \test{ \relax}{\relax}%
  \test{\label{f}a}{a}%
  \test{ \label{f}a}{a}%
  \test{\label{f} a}{a}%
  \test{ \label{f} a}{a}%
  \test{a\label{f}}{a}%
  \test{a\label{f} }{a}%
  \test{a \label{f}}{a}%
  \test{a \label{f} }{a}%
  \test{a\label{f}b\label{g}}{ab}%
  \test{a \label{f}b \label{g} }{a b}%
  \test{a\label{f} b \label{g} }{a b}%
\end{qstest}
\end{document}
%</test2>
%    \end{macrocode}
%
% \section{Installation}
%
% \subsection{Download}
%
% \paragraph{Package.} This package is available on
% CTAN\footnote{\CTANpkg{gettitlestring}}:
% \begin{description}
% \item[\CTAN{macros/latex/contrib/oberdiek/gettitlestring.dtx}] The source file.
% \item[\CTAN{macros/latex/contrib/oberdiek/gettitlestring.pdf}] Documentation.
% \end{description}
%
%
% \paragraph{Bundle.} All the packages of the bundle `oberdiek'
% are also available in a TDS compliant ZIP archive. There
% the packages are already unpacked and the documentation files
% are generated. The files and directories obey the TDS standard.
% \begin{description}
% \item[\CTANinstall{install/macros/latex/contrib/oberdiek.tds.zip}]
% \end{description}
% \emph{TDS} refers to the standard ``A Directory Structure
% for \TeX\ Files'' (\CTAN{tds/tds.pdf}). Directories
% with \xfile{texmf} in their name are usually organized this way.
%
% \subsection{Bundle installation}
%
% \paragraph{Unpacking.} Unpack the \xfile{oberdiek.tds.zip} in the
% TDS tree (also known as \xfile{texmf} tree) of your choice.
% Example (linux):
% \begin{quote}
%   |unzip oberdiek.tds.zip -d ~/texmf|
% \end{quote}
%
% \paragraph{Script installation.}
% Check the directory \xfile{TDS:scripts/oberdiek/} for
% scripts that need further installation steps.
% Package \xpackage{attachfile2} comes with the Perl script
% \xfile{pdfatfi.pl} that should be installed in such a way
% that it can be called as \texttt{pdfatfi}.
% Example (linux):
% \begin{quote}
%   |chmod +x scripts/oberdiek/pdfatfi.pl|\\
%   |cp scripts/oberdiek/pdfatfi.pl /usr/local/bin/|
% \end{quote}
%
% \subsection{Package installation}
%
% \paragraph{Unpacking.} The \xfile{.dtx} file is a self-extracting
% \docstrip\ archive. The files are extracted by running the
% \xfile{.dtx} through \plainTeX:
% \begin{quote}
%   \verb|tex gettitlestring.dtx|
% \end{quote}
%
% \paragraph{TDS.} Now the different files must be moved into
% the different directories in your installation TDS tree
% (also known as \xfile{texmf} tree):
% \begin{quote}
% \def\t{^^A
% \begin{tabular}{@{}>{\ttfamily}l@{ $\rightarrow$ }>{\ttfamily}l@{}}
%   gettitlestring.sty & tex/generic/oberdiek/gettitlestring.sty\\
%   gettitlestring.pdf & doc/latex/oberdiek/gettitlestring.pdf\\
%   test/gettitlestring-test1.tex & doc/latex/oberdiek/test/gettitlestring-test1.tex\\
%   test/gettitlestring-test2.tex & doc/latex/oberdiek/test/gettitlestring-test2.tex\\
%   gettitlestring.dtx & source/latex/oberdiek/gettitlestring.dtx\\
% \end{tabular}^^A
% }^^A
% \sbox0{\t}^^A
% \ifdim\wd0>\linewidth
%   \begingroup
%     \advance\linewidth by\leftmargin
%     \advance\linewidth by\rightmargin
%   \edef\x{\endgroup
%     \def\noexpand\lw{\the\linewidth}^^A
%   }\x
%   \def\lwbox{^^A
%     \leavevmode
%     \hbox to \linewidth{^^A
%       \kern-\leftmargin\relax
%       \hss
%       \usebox0
%       \hss
%       \kern-\rightmargin\relax
%     }^^A
%   }^^A
%   \ifdim\wd0>\lw
%     \sbox0{\small\t}^^A
%     \ifdim\wd0>\linewidth
%       \ifdim\wd0>\lw
%         \sbox0{\footnotesize\t}^^A
%         \ifdim\wd0>\linewidth
%           \ifdim\wd0>\lw
%             \sbox0{\scriptsize\t}^^A
%             \ifdim\wd0>\linewidth
%               \ifdim\wd0>\lw
%                 \sbox0{\tiny\t}^^A
%                 \ifdim\wd0>\linewidth
%                   \lwbox
%                 \else
%                   \usebox0
%                 \fi
%               \else
%                 \lwbox
%               \fi
%             \else
%               \usebox0
%             \fi
%           \else
%             \lwbox
%           \fi
%         \else
%           \usebox0
%         \fi
%       \else
%         \lwbox
%       \fi
%     \else
%       \usebox0
%     \fi
%   \else
%     \lwbox
%   \fi
% \else
%   \usebox0
% \fi
% \end{quote}
% If you have a \xfile{docstrip.cfg} that configures and enables \docstrip's
% TDS installing feature, then some files can already be in the right
% place, see the documentation of \docstrip.
%
% \subsection{Refresh file name databases}
%
% If your \TeX~distribution
% (\teTeX, \mikTeX, \dots) relies on file name databases, you must refresh
% these. For example, \teTeX\ users run \verb|texhash| or
% \verb|mktexlsr|.
%
% \subsection{Some details for the interested}
%
% \paragraph{Attached source.}
%
% The PDF documentation on CTAN also includes the
% \xfile{.dtx} source file. It can be extracted by
% AcrobatReader 6 or higher. Another option is \textsf{pdftk},
% e.g. unpack the file into the current directory:
% \begin{quote}
%   \verb|pdftk gettitlestring.pdf unpack_files output .|
% \end{quote}
%
% \paragraph{Unpacking with \LaTeX.}
% The \xfile{.dtx} chooses its action depending on the format:
% \begin{description}
% \item[\plainTeX:] Run \docstrip\ and extract the files.
% \item[\LaTeX:] Generate the documentation.
% \end{description}
% If you insist on using \LaTeX\ for \docstrip\ (really,
% \docstrip\ does not need \LaTeX), then inform the autodetect routine
% about your intention:
% \begin{quote}
%   \verb|latex \let\install=y% \iffalse meta-comment
%
% File: gettitlestring.dtx
% Version: 2016/05/16 v1.5
% Info: Cleanup title references
%
% Copyright (C) 2009, 2010 by
%    Heiko Oberdiek <heiko.oberdiek at googlemail.com>
%    2016
%    https://github.com/ho-tex/oberdiek/issues
%
% This work may be distributed and/or modified under the
% conditions of the LaTeX Project Public License, either
% version 1.3c of this license or (at your option) any later
% version. This version of this license is in
%    http://www.latex-project.org/lppl/lppl-1-3c.txt
% and the latest version of this license is in
%    http://www.latex-project.org/lppl.txt
% and version 1.3 or later is part of all distributions of
% LaTeX version 2005/12/01 or later.
%
% This work has the LPPL maintenance status "maintained".
%
% This Current Maintainer of this work is Heiko Oberdiek.
%
% The Base Interpreter refers to any `TeX-Format',
% because some files are installed in TDS:tex/generic//.
%
% This work consists of the main source file gettitlestring.dtx
% and the derived files
%    gettitlestring.sty, gettitlestring.pdf, gettitlestring.ins,
%    gettitlestring.drv, gettitlestring-test1.tex,
%    gettitlestring-test2.tex.
%
% Distribution:
%    CTAN:macros/latex/contrib/oberdiek/gettitlestring.dtx
%    CTAN:macros/latex/contrib/oberdiek/gettitlestring.pdf
%
% Unpacking:
%    (a) If gettitlestring.ins is present:
%           tex gettitlestring.ins
%    (b) Without gettitlestring.ins:
%           tex gettitlestring.dtx
%    (c) If you insist on using LaTeX
%           latex \let\install=y% \iffalse meta-comment
%
% File: gettitlestring.dtx
% Version: 2016/05/16 v1.5
% Info: Cleanup title references
%
% Copyright (C) 2009, 2010 by
%    Heiko Oberdiek <heiko.oberdiek at googlemail.com>
%    2016
%    https://github.com/ho-tex/oberdiek/issues
%
% This work may be distributed and/or modified under the
% conditions of the LaTeX Project Public License, either
% version 1.3c of this license or (at your option) any later
% version. This version of this license is in
%    http://www.latex-project.org/lppl/lppl-1-3c.txt
% and the latest version of this license is in
%    http://www.latex-project.org/lppl.txt
% and version 1.3 or later is part of all distributions of
% LaTeX version 2005/12/01 or later.
%
% This work has the LPPL maintenance status "maintained".
%
% This Current Maintainer of this work is Heiko Oberdiek.
%
% The Base Interpreter refers to any `TeX-Format',
% because some files are installed in TDS:tex/generic//.
%
% This work consists of the main source file gettitlestring.dtx
% and the derived files
%    gettitlestring.sty, gettitlestring.pdf, gettitlestring.ins,
%    gettitlestring.drv, gettitlestring-test1.tex,
%    gettitlestring-test2.tex.
%
% Distribution:
%    CTAN:macros/latex/contrib/oberdiek/gettitlestring.dtx
%    CTAN:macros/latex/contrib/oberdiek/gettitlestring.pdf
%
% Unpacking:
%    (a) If gettitlestring.ins is present:
%           tex gettitlestring.ins
%    (b) Without gettitlestring.ins:
%           tex gettitlestring.dtx
%    (c) If you insist on using LaTeX
%           latex \let\install=y\input{gettitlestring.dtx}
%        (quote the arguments according to the demands of your shell)
%
% Documentation:
%    (a) If gettitlestring.drv is present:
%           latex gettitlestring.drv
%    (b) Without gettitlestring.drv:
%           latex gettitlestring.dtx; ...
%    The class ltxdoc loads the configuration file ltxdoc.cfg
%    if available. Here you can specify further options, e.g.
%    use A4 as paper format:
%       \PassOptionsToClass{a4paper}{article}
%
%    Programm calls to get the documentation (example):
%       pdflatex gettitlestring.dtx
%       makeindex -s gind.ist gettitlestring.idx
%       pdflatex gettitlestring.dtx
%       makeindex -s gind.ist gettitlestring.idx
%       pdflatex gettitlestring.dtx
%
% Installation:
%    TDS:tex/generic/oberdiek/gettitlestring.sty
%    TDS:doc/latex/oberdiek/gettitlestring.pdf
%    TDS:doc/latex/oberdiek/test/gettitlestring-test1.tex
%    TDS:doc/latex/oberdiek/test/gettitlestring-test2.tex
%    TDS:source/latex/oberdiek/gettitlestring.dtx
%
%<*ignore>
\begingroup
  \catcode123=1 %
  \catcode125=2 %
  \def\x{LaTeX2e}%
\expandafter\endgroup
\ifcase 0\ifx\install y1\fi\expandafter
         \ifx\csname processbatchFile\endcsname\relax\else1\fi
         \ifx\fmtname\x\else 1\fi\relax
\else\csname fi\endcsname
%</ignore>
%<*install>
\input docstrip.tex
\Msg{************************************************************************}
\Msg{* Installation}
\Msg{* Package: gettitlestring 2016/05/16 v1.5 Cleanup title references (HO)}
\Msg{************************************************************************}

\keepsilent
\askforoverwritefalse

\let\MetaPrefix\relax
\preamble

This is a generated file.

Project: gettitlestring
Version: 2016/05/16 v1.5

Copyright (C) 2009, 2010 by
   Heiko Oberdiek <heiko.oberdiek at googlemail.com>

This work may be distributed and/or modified under the
conditions of the LaTeX Project Public License, either
version 1.3c of this license or (at your option) any later
version. This version of this license is in
   http://www.latex-project.org/lppl/lppl-1-3c.txt
and the latest version of this license is in
   http://www.latex-project.org/lppl.txt
and version 1.3 or later is part of all distributions of
LaTeX version 2005/12/01 or later.

This work has the LPPL maintenance status "maintained".

This Current Maintainer of this work is Heiko Oberdiek.

The Base Interpreter refers to any `TeX-Format',
because some files are installed in TDS:tex/generic//.

This work consists of the main source file gettitlestring.dtx
and the derived files
   gettitlestring.sty, gettitlestring.pdf, gettitlestring.ins,
   gettitlestring.drv, gettitlestring-test1.tex,
   gettitlestring-test2.tex.

\endpreamble
\let\MetaPrefix\DoubleperCent

\generate{%
  \file{gettitlestring.ins}{\from{gettitlestring.dtx}{install}}%
  \file{gettitlestring.drv}{\from{gettitlestring.dtx}{driver}}%
  \usedir{tex/generic/oberdiek}%
  \file{gettitlestring.sty}{\from{gettitlestring.dtx}{package}}%
%  \usedir{doc/latex/oberdiek/test}%
%  \file{gettitlestring-test1.tex}{\from{gettitlestring.dtx}{test1}}%
%  \file{gettitlestring-test2.tex}{\from{gettitlestring.dtx}{test2}}%
  \nopreamble
  \nopostamble
%  \usedir{source/latex/oberdiek/catalogue}%
%  \file{gettitlestring.xml}{\from{gettitlestring.dtx}{catalogue}}%
}

\catcode32=13\relax% active space
\let =\space%
\Msg{************************************************************************}
\Msg{*}
\Msg{* To finish the installation you have to move the following}
\Msg{* file into a directory searched by TeX:}
\Msg{*}
\Msg{*     gettitlestring.sty}
\Msg{*}
\Msg{* To produce the documentation run the file `gettitlestring.drv'}
\Msg{* through LaTeX.}
\Msg{*}
\Msg{* Happy TeXing!}
\Msg{*}
\Msg{************************************************************************}

\endbatchfile
%</install>
%<*ignore>
\fi
%</ignore>
%<*driver>
\NeedsTeXFormat{LaTeX2e}
\ProvidesFile{gettitlestring.drv}%
  [2016/05/16 v1.5 Cleanup title references (HO)]%
\documentclass{ltxdoc}
\usepackage{holtxdoc}[2011/11/22]
\begin{document}
  \DocInput{gettitlestring.dtx}%
\end{document}
%</driver>
% \fi
%
%
% \CharacterTable
%  {Upper-case    \A\B\C\D\E\F\G\H\I\J\K\L\M\N\O\P\Q\R\S\T\U\V\W\X\Y\Z
%   Lower-case    \a\b\c\d\e\f\g\h\i\j\k\l\m\n\o\p\q\r\s\t\u\v\w\x\y\z
%   Digits        \0\1\2\3\4\5\6\7\8\9
%   Exclamation   \!     Double quote  \"     Hash (number) \#
%   Dollar        \$     Percent       \%     Ampersand     \&
%   Acute accent  \'     Left paren    \(     Right paren   \)
%   Asterisk      \*     Plus          \+     Comma         \,
%   Minus         \-     Point         \.     Solidus       \/
%   Colon         \:     Semicolon     \;     Less than     \<
%   Equals        \=     Greater than  \>     Question mark \?
%   Commercial at \@     Left bracket  \[     Backslash     \\
%   Right bracket \]     Circumflex    \^     Underscore    \_
%   Grave accent  \`     Left brace    \{     Vertical bar  \|
%   Right brace   \}     Tilde         \~}
%
% \GetFileInfo{gettitlestring.drv}
%
% \title{The \xpackage{gettitlestring} package}
% \date{2016/05/16 v1.5}
% \author{Heiko Oberdiek\thanks
% {Please report any issues at \url{https://github.com/ho-tex/oberdiek/issues}}\\
% \xemail{heiko.oberdiek at googlemail.com}}
%
% \maketitle
%
% \begin{abstract}
% The \LaTeX\ package addresses packages that are dealing with
% references to titles (\cs{section}, \cs{caption}, \dots).
% The package tries to remove \cs{label} and other
% commands from title strings.
% \end{abstract}
%
% \tableofcontents
%
% \section{Documentation}
%
% \subsection{Macros}
%
% \begin{declcs}{GetTitleStringSetup} \M{key value list}
% \end{declcs}
% The options are given as comma separated key value pairs.
% See section \ref{sec:options}.
%
% \begin{declcs}{GetTitleString} \M{text}\\
% \cs{GetTitleStringExpand} \M{text}\\
% \cs{GetTitleStringNonExpand} \M{text}
% \end{declcs}
% Macro \cs{GetTitleString} tries to remove unwanted stuff from \meta{text}
% the result is stored in Macro \cs{GetTitleStringResult}.
% Two methods are available:
% \begin{description}
% \item[\cs{GetTitleStringExpand}:]
% The \meta{text} is expanded in a context where the unwanted
% macros are redefined to remove themselves.
% This is the method used in packages \xpackage{titleref}~\cite{titleref},
% \xpackage{zref-titleref}~\cite{zref}
% or class \xclass{memoir}~\cite{memoir}.
% \cs{protect} is supported, but fragile material might break.
% \item[\cs{GetTitleStringNonExpand}:]
% The \meta{text} is not expanded. Thus the removal of unwanted
% material is more difficult. It is especially removed at the
% start of the \meta{text} and spaces are removed from the end.
% Currently only \cs{label} is removed in the whole string,
% if it is not hidden inside curly braces or part of macro
% definitions. Thus the removal of unwanted stuff might not be
% complete, but fragile material will not break.
% (But the result string can break at a later time, of course).
% \end{description}
% Option \xoption{expand} controls which method is used by
% macro \cs{GetTitleString}.
%
% \begin{declcs}{GetTitleStringDisableCommands} \M{code}
% \end{declcs}
% The \meta{code} is called right before the
% text is expanded in \cs{GetTitleStringExpand}.
% Additional definitions can be given for macros that
% should be removed.
% Keep in mind that expansion means that the definitions
% must work in expandable context. Macros like
% \cs{@ifstar} or \cs{@ifnextchar} or optional arguments
% will not work. The macro names in \meta{code} may contain
% the at sign |@|, it has catcode 11 (letter).
%
% \subsection{Options}\label{sec:options}
%
% \begin{description}
% \item[\xoption{expand}:] Boolean option, takes values |true| or |false|.
% No value means |true|. The option specifies the method to remove
% unwanted stuff from the title string, see below.
% \end{description}
% Options can be set at the following places:
% \begin{itemize}
% \item \cs{usepackage}
% \item Configuration file \xfile{gettitlestring.cfg}.
% \item \cs{GetTitleStringSetup}
% \end{itemize}
%
% \StopEventually{
% }
%
% \section{Implementation}
%    \begin{macrocode}
%<*package>
%    \end{macrocode}
%    Reload check, especially if the package is not used with \LaTeX.
%    \begin{macrocode}
\begingroup\catcode61\catcode48\catcode32=10\relax%
  \catcode13=5 % ^^M
  \endlinechar=13 %
  \catcode35=6 % #
  \catcode39=12 % '
  \catcode44=12 % ,
  \catcode45=12 % -
  \catcode46=12 % .
  \catcode58=12 % :
  \catcode64=11 % @
  \catcode123=1 % {
  \catcode125=2 % }
  \expandafter\let\expandafter\x\csname ver@gettitlestring.sty\endcsname
  \ifx\x\relax % plain-TeX, first loading
  \else
    \def\empty{}%
    \ifx\x\empty % LaTeX, first loading,
      % variable is initialized, but \ProvidesPackage not yet seen
    \else
      \expandafter\ifx\csname PackageInfo\endcsname\relax
        \def\x#1#2{%
          \immediate\write-1{Package #1 Info: #2.}%
        }%
      \else
        \def\x#1#2{\PackageInfo{#1}{#2, stopped}}%
      \fi
      \x{gettitlestring}{The package is already loaded}%
      \aftergroup\endinput
    \fi
  \fi
\endgroup%
%    \end{macrocode}
%    Package identification:
%    \begin{macrocode}
\begingroup\catcode61\catcode48\catcode32=10\relax%
  \catcode13=5 % ^^M
  \endlinechar=13 %
  \catcode35=6 % #
  \catcode39=12 % '
  \catcode40=12 % (
  \catcode41=12 % )
  \catcode44=12 % ,
  \catcode45=12 % -
  \catcode46=12 % .
  \catcode47=12 % /
  \catcode58=12 % :
  \catcode64=11 % @
  \catcode91=12 % [
  \catcode93=12 % ]
  \catcode123=1 % {
  \catcode125=2 % }
  \expandafter\ifx\csname ProvidesPackage\endcsname\relax
    \def\x#1#2#3[#4]{\endgroup
      \immediate\write-1{Package: #3 #4}%
      \xdef#1{#4}%
    }%
  \else
    \def\x#1#2[#3]{\endgroup
      #2[{#3}]%
      \ifx#1\@undefined
        \xdef#1{#3}%
      \fi
      \ifx#1\relax
        \xdef#1{#3}%
      \fi
    }%
  \fi
\expandafter\x\csname ver@gettitlestring.sty\endcsname
\ProvidesPackage{gettitlestring}%
  [2016/05/16 v1.5 Cleanup title references (HO)]%
%    \end{macrocode}
%
%    \begin{macrocode}
\begingroup\catcode61\catcode48\catcode32=10\relax%
  \catcode13=5 % ^^M
  \endlinechar=13 %
  \catcode123=1 % {
  \catcode125=2 % }
  \catcode64=11 % @
  \def\x{\endgroup
    \expandafter\edef\csname GTS@AtEnd\endcsname{%
      \endlinechar=\the\endlinechar\relax
      \catcode13=\the\catcode13\relax
      \catcode32=\the\catcode32\relax
      \catcode35=\the\catcode35\relax
      \catcode61=\the\catcode61\relax
      \catcode64=\the\catcode64\relax
      \catcode123=\the\catcode123\relax
      \catcode125=\the\catcode125\relax
    }%
  }%
\x\catcode61\catcode48\catcode32=10\relax%
\catcode13=5 % ^^M
\endlinechar=13 %
\catcode35=6 % #
\catcode64=11 % @
\catcode123=1 % {
\catcode125=2 % }
\def\TMP@EnsureCode#1#2{%
  \edef\GTS@AtEnd{%
    \GTS@AtEnd
    \catcode#1=\the\catcode#1\relax
  }%
  \catcode#1=#2\relax
}
\TMP@EnsureCode{42}{12}% *
\TMP@EnsureCode{44}{12}% ,
\TMP@EnsureCode{45}{12}% -
\TMP@EnsureCode{46}{12}% .
\TMP@EnsureCode{47}{12}% /
\TMP@EnsureCode{91}{12}% [
\TMP@EnsureCode{93}{12}% ]
\edef\GTS@AtEnd{\GTS@AtEnd\noexpand\endinput}
%    \end{macrocode}
%
% \subsection{Options}
%
%    \begin{macrocode}
\RequirePackage{kvoptions}[2009/07/17]
\SetupKeyvalOptions{%
  family=gettitlestring,%
  prefix=GTS@%
}
\newcommand*{\GetTitleStringSetup}{%
  \setkeys{gettitlestring}%
}
\DeclareBoolOption{expand}
\InputIfFileExists{gettitlestring.cfg}{}{}
\ProcessKeyvalOptions*\relax
%    \end{macrocode}
%
% \subsection{\cs{GetTitleString}}
%
%    \begin{macro}{\GetTitleString}
%    \begin{macrocode}
\newcommand*{\GetTitleString}{%
  \ifGTS@expand
    \expandafter\GetTitleStringExpand
  \else
    \expandafter\GetTitleStringNonExpand
  \fi
}
%    \end{macrocode}
%    \end{macro}
%    \begin{macro}{\GetTitleStringExpand}
%    \begin{macrocode}
\newcommand{\GetTitleStringExpand}[1]{%
  \def\GetTitleStringResult{#1}%
  \begingroup
    \GTS@DisablePredefinedCmds
    \GTS@DisableHook
    \edef\x{\endgroup
      \noexpand\def\noexpand\GetTitleStringResult{%
        \GetTitleStringResult
      }%
    }%
  \x
}
%    \end{macrocode}
%    \end{macro}
%    \begin{macro}{\GetTitleString}
%    \begin{macrocode}
\newcommand{\GetTitleStringNonExpand}[1]{%
  \def\GetTitleStringResult{#1}%
  \global\let\GTS@GlobalString\GetTitleStringResult
  \begingroup
    \GTS@RemoveLeft
    \GTS@RemoveRight
  \endgroup
  \let\GetTitleStringResult\GTS@GlobalString
}
%    \end{macrocode}
%    \end{macro}
%
% \subsubsection{Expand method}
%
%    \begin{macro}{\GTS@DisablePredefinedCmds}
%    \begin{macrocode}
\def\GTS@DisablePredefinedCmds{%
  \let\label\@gobble
  \let\zlabel\@gobble
  \let\zref@label\@gobble
  \let\zref@labelbylist\@gobbletwo
  \let\zref@labelbyprops\@gobbletwo
  \let\index\@gobble
  \let\glossary\@gobble
  \let\markboth\@gobbletwo
  \let\@mkboth\@gobbletwo
  \let\markright\@gobble
  \let\phantomsection\@empty
  \def\addcontentsline{\expandafter\@gobble\@gobbletwo}%
  \let\raggedright\@empty
  \let\raggedleft\@empty
  \let\centering\@empty
  \let\protect\@unexpandable@protect
  \let\enit@format\@empty % package enumitem
}
%    \end{macrocode}
%    \end{macro}
%
%    \begin{macro}{\GTS@DisableHook}
%    \begin{macrocode}
\providecommand*{\GTS@DisableHook}{}
%    \end{macrocode}
%    \end{macro}
%    \begin{macro}{\GetTitleStringDisableCommands}
%    \begin{macrocode}
\def\GetTitleStringDisableCommands{%
  \begingroup
    \makeatletter
    \GTS@DisableCommands
}
%    \end{macrocode}
%    \end{macro}
%    \begin{macro}{\GTS@DisableCommands}
%    \begin{macrocode}
\long\def\GTS@DisableCommands#1{%
    \toks0=\expandafter{\GTS@DisableHook}%
    \toks2={#1}%
    \xdef\GTS@GlobalString{\the\toks0 \the\toks2}%
  \endgroup
  \let\GTS@DisableHook\GTS@GlobalString
}
%    \end{macrocode}
%    \end{macro}
%
% \subsubsection{Non-expand method}
%
%    \begin{macrocode}
\def\GTS@RemoveLeft{%
  \toks@\expandafter\expandafter\expandafter{%
    \expandafter\GTS@Car\GTS@GlobalString{}{}{}{}\GTS@Nil
  }%
  \edef\GTS@Token{\the\toks@}%
  \GTS@PredefinedLeftCmds
  \expandafter\futurelet\expandafter\GTS@Token
  \expandafter\GTS@TestLeftSpace\GTS@GlobalString\GTS@Nil
  \GTS@End
}
\def\GTS@End{}
\long\def\GTS@TestLeft#1#2{%
  \def\GTS@temp{#1}%
  \ifx\GTS@temp\GTS@Token
    \toks@\expandafter\expandafter\expandafter{%
      \expandafter#2\GTS@GlobalString\GTS@Nil
    }%
    \expandafter\GTS@TestLeftEnd
  \fi
}
\long\def\GTS@TestLeftEnd#1\GTS@End{%
  \xdef\GTS@GlobalString{\the\toks@}%
  \GTS@RemoveLeft
}
\long\def\GTS@Car#1#2\GTS@Nil{#1}
\long\def\GTS@Cdr#1#2\GTS@Nil{#2}
\long\def\GTS@CdrTwo#1#2#3\GTS@Nil{#3}
\long\def\GTS@CdrThree#1#2#3#4\GTS@Nil{#4}
\long\def\GTS@CdrFour#1#2#3#4#5\GTS@Nil{#5}
\long\def\GTS@TestLeftSpace#1\GTS@Nil{%
  \ifx\GTS@Token\@sptoken
    \toks@\expandafter{%
      \romannumeral-0\GTS@GlobalString
    }%
    \expandafter\GTS@TestLeftEnd
  \fi
}
%    \end{macrocode}
%    \begin{macro}{\GTS@PredefinedLeftCmds}
%    \begin{macrocode}
\def\GTS@PredefinedLeftCmds{%
  \GTS@TestLeft\Hy@phantomsection\GTS@Cdr
  \GTS@TestLeft\Hy@SectionAnchor\GTS@Cdr
  \GTS@TestLeft\Hy@SectionAnchorHref\GTS@CdrTwo
  \GTS@TestLeft\label\GTS@CdrTwo
  \GTS@TestLeft\zlabel\GTS@CdrTwo
  \GTS@TestLeft\index\GTS@CdrTwo
  \GTS@TestLeft\glossary\GTS@CdrTwo
  \GTS@TestLeft\markboth\GTS@CdrThree
  \GTS@TestLeft\@mkboth\GTS@CdrThree
  \GTS@TestLeft\addcontentsline\GTS@CdrFour
  \GTS@TestLeft\enit@format\GTS@Cdr % package enumitem
}
%    \end{macrocode}
%    \end{macro}
%
%    \begin{macrocode}
\def\GTS@RemoveRight{%
  \toks@{}%
  \expandafter\GTS@TestRightLabel\GTS@GlobalString
      \label{}\GTS@Nil\@nil
  \GTS@RemoveRightSpace
}
\begingroup
  \def\GTS@temp#1{\endgroup
    \def\GTS@RemoveRightSpace{%
      \expandafter\GTS@TestRightSpace\GTS@GlobalString
          \GTS@Nil#1\GTS@Nil\@nil
    }%
  }%
\GTS@temp{ }
\def\GTS@TestRightSpace#1 \GTS@Nil#2\@nil{%
  \ifx\relax#2\relax
  \else
    \gdef\GTS@GlobalString{#1}%
    \expandafter\GTS@RemoveRightSpace
  \fi
}
\def\GTS@TestRightLabel#1\label#2#3\GTS@Nil#4\@nil{%
  \def\GTS@temp{#3}%
  \ifx\GTS@temp\@empty
    \expandafter\gdef\expandafter\GTS@GlobalString\expandafter{%
      \the\toks@
      #1%
    }%
    \expandafter\@gobble
  \else
    \expandafter\@firstofone
  \fi
  {%
    \toks@\expandafter{\the\toks@#1}%
    \GTS@TestRightLabel#3\GTS@Nil\@nil
  }%
}
%    \end{macrocode}
%
%    \begin{macrocode}
\GTS@AtEnd%
%</package>
%    \end{macrocode}
%
% \section{Test}
%
% \subsection{Catcode checks for loading}
%
%    \begin{macrocode}
%<*test1>
%    \end{macrocode}
%    \begin{macrocode}
\catcode`\{=1 %
\catcode`\}=2 %
\catcode`\#=6 %
\catcode`\@=11 %
\expandafter\ifx\csname count@\endcsname\relax
  \countdef\count@=255 %
\fi
\expandafter\ifx\csname @gobble\endcsname\relax
  \long\def\@gobble#1{}%
\fi
\expandafter\ifx\csname @firstofone\endcsname\relax
  \long\def\@firstofone#1{#1}%
\fi
\expandafter\ifx\csname loop\endcsname\relax
  \expandafter\@firstofone
\else
  \expandafter\@gobble
\fi
{%
  \def\loop#1\repeat{%
    \def\body{#1}%
    \iterate
  }%
  \def\iterate{%
    \body
      \let\next\iterate
    \else
      \let\next\relax
    \fi
    \next
  }%
  \let\repeat=\fi
}%
\def\RestoreCatcodes{}
\count@=0 %
\loop
  \edef\RestoreCatcodes{%
    \RestoreCatcodes
    \catcode\the\count@=\the\catcode\count@\relax
  }%
\ifnum\count@<255 %
  \advance\count@ 1 %
\repeat

\def\RangeCatcodeInvalid#1#2{%
  \count@=#1\relax
  \loop
    \catcode\count@=15 %
  \ifnum\count@<#2\relax
    \advance\count@ 1 %
  \repeat
}
\def\RangeCatcodeCheck#1#2#3{%
  \count@=#1\relax
  \loop
    \ifnum#3=\catcode\count@
    \else
      \errmessage{%
        Character \the\count@\space
        with wrong catcode \the\catcode\count@\space
        instead of \number#3%
      }%
    \fi
  \ifnum\count@<#2\relax
    \advance\count@ 1 %
  \repeat
}
\def\space{ }
\expandafter\ifx\csname LoadCommand\endcsname\relax
  \def\LoadCommand{\input gettitlestring.sty\relax}%
\fi
\def\Test{%
  \RangeCatcodeInvalid{0}{47}%
  \RangeCatcodeInvalid{58}{64}%
  \RangeCatcodeInvalid{91}{96}%
  \RangeCatcodeInvalid{123}{255}%
  \catcode`\@=12 %
  \catcode`\\=0 %
  \catcode`\%=14 %
  \LoadCommand
  \RangeCatcodeCheck{0}{36}{15}%
  \RangeCatcodeCheck{37}{37}{14}%
  \RangeCatcodeCheck{38}{47}{15}%
  \RangeCatcodeCheck{48}{57}{12}%
  \RangeCatcodeCheck{58}{63}{15}%
  \RangeCatcodeCheck{64}{64}{12}%
  \RangeCatcodeCheck{65}{90}{11}%
  \RangeCatcodeCheck{91}{91}{15}%
  \RangeCatcodeCheck{92}{92}{0}%
  \RangeCatcodeCheck{93}{96}{15}%
  \RangeCatcodeCheck{97}{122}{11}%
  \RangeCatcodeCheck{123}{255}{15}%
  \RestoreCatcodes
}
\Test
\csname @@end\endcsname
\end
%    \end{macrocode}
%    \begin{macrocode}
%</test1>
%    \end{macrocode}
%
% \subsection{Test of non-expand method}
%
%    \begin{macrocode}
%<*test2>
\NeedsTeXFormat{LaTeX2e}
\documentclass{minimal}
\usepackage{gettitlestring}[2016/05/16]
\usepackage{qstest}
\IncludeTests{*}
\LogTests{log}{*}{*}
\begin{document}
\begin{qstest}{non-expand}{non-expand}
  \def\test#1#2{%
    \sbox0{%
      \GetTitleString{#1}%
      \Expect{#2}*{\GetTitleStringResult}%
    }%
    \Expect{0.0pt}*{\the\wd0}%
  }%
  \test{}{}%
  \test{ }{}%
  \test{ x }{x}%
  \test{ x y }{x y}%
  \test{ \relax}{\relax}%
  \test{\label{f}a}{a}%
  \test{ \label{f}a}{a}%
  \test{\label{f} a}{a}%
  \test{ \label{f} a}{a}%
  \test{a\label{f}}{a}%
  \test{a\label{f} }{a}%
  \test{a \label{f}}{a}%
  \test{a \label{f} }{a}%
  \test{a\label{f}b\label{g}}{ab}%
  \test{a \label{f}b \label{g} }{a b}%
  \test{a\label{f} b \label{g} }{a b}%
\end{qstest}
\end{document}
%</test2>
%    \end{macrocode}
%
% \section{Installation}
%
% \subsection{Download}
%
% \paragraph{Package.} This package is available on
% CTAN\footnote{\CTANpkg{gettitlestring}}:
% \begin{description}
% \item[\CTAN{macros/latex/contrib/oberdiek/gettitlestring.dtx}] The source file.
% \item[\CTAN{macros/latex/contrib/oberdiek/gettitlestring.pdf}] Documentation.
% \end{description}
%
%
% \paragraph{Bundle.} All the packages of the bundle `oberdiek'
% are also available in a TDS compliant ZIP archive. There
% the packages are already unpacked and the documentation files
% are generated. The files and directories obey the TDS standard.
% \begin{description}
% \item[\CTANinstall{install/macros/latex/contrib/oberdiek.tds.zip}]
% \end{description}
% \emph{TDS} refers to the standard ``A Directory Structure
% for \TeX\ Files'' (\CTAN{tds/tds.pdf}). Directories
% with \xfile{texmf} in their name are usually organized this way.
%
% \subsection{Bundle installation}
%
% \paragraph{Unpacking.} Unpack the \xfile{oberdiek.tds.zip} in the
% TDS tree (also known as \xfile{texmf} tree) of your choice.
% Example (linux):
% \begin{quote}
%   |unzip oberdiek.tds.zip -d ~/texmf|
% \end{quote}
%
% \paragraph{Script installation.}
% Check the directory \xfile{TDS:scripts/oberdiek/} for
% scripts that need further installation steps.
% Package \xpackage{attachfile2} comes with the Perl script
% \xfile{pdfatfi.pl} that should be installed in such a way
% that it can be called as \texttt{pdfatfi}.
% Example (linux):
% \begin{quote}
%   |chmod +x scripts/oberdiek/pdfatfi.pl|\\
%   |cp scripts/oberdiek/pdfatfi.pl /usr/local/bin/|
% \end{quote}
%
% \subsection{Package installation}
%
% \paragraph{Unpacking.} The \xfile{.dtx} file is a self-extracting
% \docstrip\ archive. The files are extracted by running the
% \xfile{.dtx} through \plainTeX:
% \begin{quote}
%   \verb|tex gettitlestring.dtx|
% \end{quote}
%
% \paragraph{TDS.} Now the different files must be moved into
% the different directories in your installation TDS tree
% (also known as \xfile{texmf} tree):
% \begin{quote}
% \def\t{^^A
% \begin{tabular}{@{}>{\ttfamily}l@{ $\rightarrow$ }>{\ttfamily}l@{}}
%   gettitlestring.sty & tex/generic/oberdiek/gettitlestring.sty\\
%   gettitlestring.pdf & doc/latex/oberdiek/gettitlestring.pdf\\
%   test/gettitlestring-test1.tex & doc/latex/oberdiek/test/gettitlestring-test1.tex\\
%   test/gettitlestring-test2.tex & doc/latex/oberdiek/test/gettitlestring-test2.tex\\
%   gettitlestring.dtx & source/latex/oberdiek/gettitlestring.dtx\\
% \end{tabular}^^A
% }^^A
% \sbox0{\t}^^A
% \ifdim\wd0>\linewidth
%   \begingroup
%     \advance\linewidth by\leftmargin
%     \advance\linewidth by\rightmargin
%   \edef\x{\endgroup
%     \def\noexpand\lw{\the\linewidth}^^A
%   }\x
%   \def\lwbox{^^A
%     \leavevmode
%     \hbox to \linewidth{^^A
%       \kern-\leftmargin\relax
%       \hss
%       \usebox0
%       \hss
%       \kern-\rightmargin\relax
%     }^^A
%   }^^A
%   \ifdim\wd0>\lw
%     \sbox0{\small\t}^^A
%     \ifdim\wd0>\linewidth
%       \ifdim\wd0>\lw
%         \sbox0{\footnotesize\t}^^A
%         \ifdim\wd0>\linewidth
%           \ifdim\wd0>\lw
%             \sbox0{\scriptsize\t}^^A
%             \ifdim\wd0>\linewidth
%               \ifdim\wd0>\lw
%                 \sbox0{\tiny\t}^^A
%                 \ifdim\wd0>\linewidth
%                   \lwbox
%                 \else
%                   \usebox0
%                 \fi
%               \else
%                 \lwbox
%               \fi
%             \else
%               \usebox0
%             \fi
%           \else
%             \lwbox
%           \fi
%         \else
%           \usebox0
%         \fi
%       \else
%         \lwbox
%       \fi
%     \else
%       \usebox0
%     \fi
%   \else
%     \lwbox
%   \fi
% \else
%   \usebox0
% \fi
% \end{quote}
% If you have a \xfile{docstrip.cfg} that configures and enables \docstrip's
% TDS installing feature, then some files can already be in the right
% place, see the documentation of \docstrip.
%
% \subsection{Refresh file name databases}
%
% If your \TeX~distribution
% (\teTeX, \mikTeX, \dots) relies on file name databases, you must refresh
% these. For example, \teTeX\ users run \verb|texhash| or
% \verb|mktexlsr|.
%
% \subsection{Some details for the interested}
%
% \paragraph{Attached source.}
%
% The PDF documentation on CTAN also includes the
% \xfile{.dtx} source file. It can be extracted by
% AcrobatReader 6 or higher. Another option is \textsf{pdftk},
% e.g. unpack the file into the current directory:
% \begin{quote}
%   \verb|pdftk gettitlestring.pdf unpack_files output .|
% \end{quote}
%
% \paragraph{Unpacking with \LaTeX.}
% The \xfile{.dtx} chooses its action depending on the format:
% \begin{description}
% \item[\plainTeX:] Run \docstrip\ and extract the files.
% \item[\LaTeX:] Generate the documentation.
% \end{description}
% If you insist on using \LaTeX\ for \docstrip\ (really,
% \docstrip\ does not need \LaTeX), then inform the autodetect routine
% about your intention:
% \begin{quote}
%   \verb|latex \let\install=y\input{gettitlestring.dtx}|
% \end{quote}
% Do not forget to quote the argument according to the demands
% of your shell.
%
% \paragraph{Generating the documentation.}
% You can use both the \xfile{.dtx} or the \xfile{.drv} to generate
% the documentation. The process can be configured by the
% configuration file \xfile{ltxdoc.cfg}. For instance, put this
% line into this file, if you want to have A4 as paper format:
% \begin{quote}
%   \verb|\PassOptionsToClass{a4paper}{article}|
% \end{quote}
% An example follows how to generate the
% documentation with pdf\LaTeX:
% \begin{quote}
%\begin{verbatim}
%pdflatex gettitlestring.dtx
%makeindex -s gind.ist gettitlestring.idx
%pdflatex gettitlestring.dtx
%makeindex -s gind.ist gettitlestring.idx
%pdflatex gettitlestring.dtx
%\end{verbatim}
% \end{quote}
%
% \begin{thebibliography}{9}
%
% \bibitem{memoir}
% Peter Wilson, Lars Madsen:
% \textit{The Memoir Class};
% 2009/11/17 v1.61803398c;
% \CTANpkg{memoir}
%
% \bibitem{titleref}
% Donald Arsenau:
% \textit{Titleref.sty};
% 2001/04/05 ver 3.1;
% \CTAN{macros/latex/contrib/misc/titleref.sty}
%
% \bibitem{zref}
% Heiko Oberdiek:
% \textit{The \xpackage{zref} package};
% 2009/12/08 v2.7;
% \CTAN{macros/latex/contrib/oberdiek/zref.pdf}
%
% \end{thebibliography}
%
% \begin{History}
%   \begin{Version}{2009/12/08 v1.0}
%   \item
%     The first version.
%   \end{Version}
%   \begin{Version}{2009/12/12 v1.1}
%   \item
%     Short info shortened.
%   \end{Version}
%   \begin{Version}{2009/12/13 v1.2}
%   \item
%     Forgotten third argument for \cs{InputIfFileExists} added.
%   \end{Version}
%   \begin{Version}{2009/12/18 v1.3}
%   \item
%     \cs{Hy@SectionAnchorHref} added for filtering
%     (hyperref 2009/12/18 v6.79w).
%   \end{Version}
%   \begin{Version}{2010/12/03 v1.4}
%   \item
%     Support of package \xpackage{enumitem}: removing
%     \cs{enit@format} from title string (problem report by GL).
%   \end{Version}
%   \begin{Version}{2016/05/16 v1.5}
%   \item
%     Documentation updates.
%   \end{Version}
% \end{History}
%
% \PrintIndex
%
% \Finale
\endinput

%        (quote the arguments according to the demands of your shell)
%
% Documentation:
%    (a) If gettitlestring.drv is present:
%           latex gettitlestring.drv
%    (b) Without gettitlestring.drv:
%           latex gettitlestring.dtx; ...
%    The class ltxdoc loads the configuration file ltxdoc.cfg
%    if available. Here you can specify further options, e.g.
%    use A4 as paper format:
%       \PassOptionsToClass{a4paper}{article}
%
%    Programm calls to get the documentation (example):
%       pdflatex gettitlestring.dtx
%       makeindex -s gind.ist gettitlestring.idx
%       pdflatex gettitlestring.dtx
%       makeindex -s gind.ist gettitlestring.idx
%       pdflatex gettitlestring.dtx
%
% Installation:
%    TDS:tex/generic/oberdiek/gettitlestring.sty
%    TDS:doc/latex/oberdiek/gettitlestring.pdf
%    TDS:doc/latex/oberdiek/test/gettitlestring-test1.tex
%    TDS:doc/latex/oberdiek/test/gettitlestring-test2.tex
%    TDS:source/latex/oberdiek/gettitlestring.dtx
%
%<*ignore>
\begingroup
  \catcode123=1 %
  \catcode125=2 %
  \def\x{LaTeX2e}%
\expandafter\endgroup
\ifcase 0\ifx\install y1\fi\expandafter
         \ifx\csname processbatchFile\endcsname\relax\else1\fi
         \ifx\fmtname\x\else 1\fi\relax
\else\csname fi\endcsname
%</ignore>
%<*install>
\input docstrip.tex
\Msg{************************************************************************}
\Msg{* Installation}
\Msg{* Package: gettitlestring 2016/05/16 v1.5 Cleanup title references (HO)}
\Msg{************************************************************************}

\keepsilent
\askforoverwritefalse

\let\MetaPrefix\relax
\preamble

This is a generated file.

Project: gettitlestring
Version: 2016/05/16 v1.5

Copyright (C) 2009, 2010 by
   Heiko Oberdiek <heiko.oberdiek at googlemail.com>

This work may be distributed and/or modified under the
conditions of the LaTeX Project Public License, either
version 1.3c of this license or (at your option) any later
version. This version of this license is in
   http://www.latex-project.org/lppl/lppl-1-3c.txt
and the latest version of this license is in
   http://www.latex-project.org/lppl.txt
and version 1.3 or later is part of all distributions of
LaTeX version 2005/12/01 or later.

This work has the LPPL maintenance status "maintained".

This Current Maintainer of this work is Heiko Oberdiek.

The Base Interpreter refers to any `TeX-Format',
because some files are installed in TDS:tex/generic//.

This work consists of the main source file gettitlestring.dtx
and the derived files
   gettitlestring.sty, gettitlestring.pdf, gettitlestring.ins,
   gettitlestring.drv, gettitlestring-test1.tex,
   gettitlestring-test2.tex.

\endpreamble
\let\MetaPrefix\DoubleperCent

\generate{%
  \file{gettitlestring.ins}{\from{gettitlestring.dtx}{install}}%
  \file{gettitlestring.drv}{\from{gettitlestring.dtx}{driver}}%
  \usedir{tex/generic/oberdiek}%
  \file{gettitlestring.sty}{\from{gettitlestring.dtx}{package}}%
%  \usedir{doc/latex/oberdiek/test}%
%  \file{gettitlestring-test1.tex}{\from{gettitlestring.dtx}{test1}}%
%  \file{gettitlestring-test2.tex}{\from{gettitlestring.dtx}{test2}}%
  \nopreamble
  \nopostamble
%  \usedir{source/latex/oberdiek/catalogue}%
%  \file{gettitlestring.xml}{\from{gettitlestring.dtx}{catalogue}}%
}

\catcode32=13\relax% active space
\let =\space%
\Msg{************************************************************************}
\Msg{*}
\Msg{* To finish the installation you have to move the following}
\Msg{* file into a directory searched by TeX:}
\Msg{*}
\Msg{*     gettitlestring.sty}
\Msg{*}
\Msg{* To produce the documentation run the file `gettitlestring.drv'}
\Msg{* through LaTeX.}
\Msg{*}
\Msg{* Happy TeXing!}
\Msg{*}
\Msg{************************************************************************}

\endbatchfile
%</install>
%<*ignore>
\fi
%</ignore>
%<*driver>
\NeedsTeXFormat{LaTeX2e}
\ProvidesFile{gettitlestring.drv}%
  [2016/05/16 v1.5 Cleanup title references (HO)]%
\documentclass{ltxdoc}
\usepackage{holtxdoc}[2011/11/22]
\begin{document}
  \DocInput{gettitlestring.dtx}%
\end{document}
%</driver>
% \fi
%
%
% \CharacterTable
%  {Upper-case    \A\B\C\D\E\F\G\H\I\J\K\L\M\N\O\P\Q\R\S\T\U\V\W\X\Y\Z
%   Lower-case    \a\b\c\d\e\f\g\h\i\j\k\l\m\n\o\p\q\r\s\t\u\v\w\x\y\z
%   Digits        \0\1\2\3\4\5\6\7\8\9
%   Exclamation   \!     Double quote  \"     Hash (number) \#
%   Dollar        \$     Percent       \%     Ampersand     \&
%   Acute accent  \'     Left paren    \(     Right paren   \)
%   Asterisk      \*     Plus          \+     Comma         \,
%   Minus         \-     Point         \.     Solidus       \/
%   Colon         \:     Semicolon     \;     Less than     \<
%   Equals        \=     Greater than  \>     Question mark \?
%   Commercial at \@     Left bracket  \[     Backslash     \\
%   Right bracket \]     Circumflex    \^     Underscore    \_
%   Grave accent  \`     Left brace    \{     Vertical bar  \|
%   Right brace   \}     Tilde         \~}
%
% \GetFileInfo{gettitlestring.drv}
%
% \title{The \xpackage{gettitlestring} package}
% \date{2016/05/16 v1.5}
% \author{Heiko Oberdiek\thanks
% {Please report any issues at \url{https://github.com/ho-tex/oberdiek/issues}}\\
% \xemail{heiko.oberdiek at googlemail.com}}
%
% \maketitle
%
% \begin{abstract}
% The \LaTeX\ package addresses packages that are dealing with
% references to titles (\cs{section}, \cs{caption}, \dots).
% The package tries to remove \cs{label} and other
% commands from title strings.
% \end{abstract}
%
% \tableofcontents
%
% \section{Documentation}
%
% \subsection{Macros}
%
% \begin{declcs}{GetTitleStringSetup} \M{key value list}
% \end{declcs}
% The options are given as comma separated key value pairs.
% See section \ref{sec:options}.
%
% \begin{declcs}{GetTitleString} \M{text}\\
% \cs{GetTitleStringExpand} \M{text}\\
% \cs{GetTitleStringNonExpand} \M{text}
% \end{declcs}
% Macro \cs{GetTitleString} tries to remove unwanted stuff from \meta{text}
% the result is stored in Macro \cs{GetTitleStringResult}.
% Two methods are available:
% \begin{description}
% \item[\cs{GetTitleStringExpand}:]
% The \meta{text} is expanded in a context where the unwanted
% macros are redefined to remove themselves.
% This is the method used in packages \xpackage{titleref}~\cite{titleref},
% \xpackage{zref-titleref}~\cite{zref}
% or class \xclass{memoir}~\cite{memoir}.
% \cs{protect} is supported, but fragile material might break.
% \item[\cs{GetTitleStringNonExpand}:]
% The \meta{text} is not expanded. Thus the removal of unwanted
% material is more difficult. It is especially removed at the
% start of the \meta{text} and spaces are removed from the end.
% Currently only \cs{label} is removed in the whole string,
% if it is not hidden inside curly braces or part of macro
% definitions. Thus the removal of unwanted stuff might not be
% complete, but fragile material will not break.
% (But the result string can break at a later time, of course).
% \end{description}
% Option \xoption{expand} controls which method is used by
% macro \cs{GetTitleString}.
%
% \begin{declcs}{GetTitleStringDisableCommands} \M{code}
% \end{declcs}
% The \meta{code} is called right before the
% text is expanded in \cs{GetTitleStringExpand}.
% Additional definitions can be given for macros that
% should be removed.
% Keep in mind that expansion means that the definitions
% must work in expandable context. Macros like
% \cs{@ifstar} or \cs{@ifnextchar} or optional arguments
% will not work. The macro names in \meta{code} may contain
% the at sign |@|, it has catcode 11 (letter).
%
% \subsection{Options}\label{sec:options}
%
% \begin{description}
% \item[\xoption{expand}:] Boolean option, takes values |true| or |false|.
% No value means |true|. The option specifies the method to remove
% unwanted stuff from the title string, see below.
% \end{description}
% Options can be set at the following places:
% \begin{itemize}
% \item \cs{usepackage}
% \item Configuration file \xfile{gettitlestring.cfg}.
% \item \cs{GetTitleStringSetup}
% \end{itemize}
%
% \StopEventually{
% }
%
% \section{Implementation}
%    \begin{macrocode}
%<*package>
%    \end{macrocode}
%    Reload check, especially if the package is not used with \LaTeX.
%    \begin{macrocode}
\begingroup\catcode61\catcode48\catcode32=10\relax%
  \catcode13=5 % ^^M
  \endlinechar=13 %
  \catcode35=6 % #
  \catcode39=12 % '
  \catcode44=12 % ,
  \catcode45=12 % -
  \catcode46=12 % .
  \catcode58=12 % :
  \catcode64=11 % @
  \catcode123=1 % {
  \catcode125=2 % }
  \expandafter\let\expandafter\x\csname ver@gettitlestring.sty\endcsname
  \ifx\x\relax % plain-TeX, first loading
  \else
    \def\empty{}%
    \ifx\x\empty % LaTeX, first loading,
      % variable is initialized, but \ProvidesPackage not yet seen
    \else
      \expandafter\ifx\csname PackageInfo\endcsname\relax
        \def\x#1#2{%
          \immediate\write-1{Package #1 Info: #2.}%
        }%
      \else
        \def\x#1#2{\PackageInfo{#1}{#2, stopped}}%
      \fi
      \x{gettitlestring}{The package is already loaded}%
      \aftergroup\endinput
    \fi
  \fi
\endgroup%
%    \end{macrocode}
%    Package identification:
%    \begin{macrocode}
\begingroup\catcode61\catcode48\catcode32=10\relax%
  \catcode13=5 % ^^M
  \endlinechar=13 %
  \catcode35=6 % #
  \catcode39=12 % '
  \catcode40=12 % (
  \catcode41=12 % )
  \catcode44=12 % ,
  \catcode45=12 % -
  \catcode46=12 % .
  \catcode47=12 % /
  \catcode58=12 % :
  \catcode64=11 % @
  \catcode91=12 % [
  \catcode93=12 % ]
  \catcode123=1 % {
  \catcode125=2 % }
  \expandafter\ifx\csname ProvidesPackage\endcsname\relax
    \def\x#1#2#3[#4]{\endgroup
      \immediate\write-1{Package: #3 #4}%
      \xdef#1{#4}%
    }%
  \else
    \def\x#1#2[#3]{\endgroup
      #2[{#3}]%
      \ifx#1\@undefined
        \xdef#1{#3}%
      \fi
      \ifx#1\relax
        \xdef#1{#3}%
      \fi
    }%
  \fi
\expandafter\x\csname ver@gettitlestring.sty\endcsname
\ProvidesPackage{gettitlestring}%
  [2016/05/16 v1.5 Cleanup title references (HO)]%
%    \end{macrocode}
%
%    \begin{macrocode}
\begingroup\catcode61\catcode48\catcode32=10\relax%
  \catcode13=5 % ^^M
  \endlinechar=13 %
  \catcode123=1 % {
  \catcode125=2 % }
  \catcode64=11 % @
  \def\x{\endgroup
    \expandafter\edef\csname GTS@AtEnd\endcsname{%
      \endlinechar=\the\endlinechar\relax
      \catcode13=\the\catcode13\relax
      \catcode32=\the\catcode32\relax
      \catcode35=\the\catcode35\relax
      \catcode61=\the\catcode61\relax
      \catcode64=\the\catcode64\relax
      \catcode123=\the\catcode123\relax
      \catcode125=\the\catcode125\relax
    }%
  }%
\x\catcode61\catcode48\catcode32=10\relax%
\catcode13=5 % ^^M
\endlinechar=13 %
\catcode35=6 % #
\catcode64=11 % @
\catcode123=1 % {
\catcode125=2 % }
\def\TMP@EnsureCode#1#2{%
  \edef\GTS@AtEnd{%
    \GTS@AtEnd
    \catcode#1=\the\catcode#1\relax
  }%
  \catcode#1=#2\relax
}
\TMP@EnsureCode{42}{12}% *
\TMP@EnsureCode{44}{12}% ,
\TMP@EnsureCode{45}{12}% -
\TMP@EnsureCode{46}{12}% .
\TMP@EnsureCode{47}{12}% /
\TMP@EnsureCode{91}{12}% [
\TMP@EnsureCode{93}{12}% ]
\edef\GTS@AtEnd{\GTS@AtEnd\noexpand\endinput}
%    \end{macrocode}
%
% \subsection{Options}
%
%    \begin{macrocode}
\RequirePackage{kvoptions}[2009/07/17]
\SetupKeyvalOptions{%
  family=gettitlestring,%
  prefix=GTS@%
}
\newcommand*{\GetTitleStringSetup}{%
  \setkeys{gettitlestring}%
}
\DeclareBoolOption{expand}
\InputIfFileExists{gettitlestring.cfg}{}{}
\ProcessKeyvalOptions*\relax
%    \end{macrocode}
%
% \subsection{\cs{GetTitleString}}
%
%    \begin{macro}{\GetTitleString}
%    \begin{macrocode}
\newcommand*{\GetTitleString}{%
  \ifGTS@expand
    \expandafter\GetTitleStringExpand
  \else
    \expandafter\GetTitleStringNonExpand
  \fi
}
%    \end{macrocode}
%    \end{macro}
%    \begin{macro}{\GetTitleStringExpand}
%    \begin{macrocode}
\newcommand{\GetTitleStringExpand}[1]{%
  \def\GetTitleStringResult{#1}%
  \begingroup
    \GTS@DisablePredefinedCmds
    \GTS@DisableHook
    \edef\x{\endgroup
      \noexpand\def\noexpand\GetTitleStringResult{%
        \GetTitleStringResult
      }%
    }%
  \x
}
%    \end{macrocode}
%    \end{macro}
%    \begin{macro}{\GetTitleString}
%    \begin{macrocode}
\newcommand{\GetTitleStringNonExpand}[1]{%
  \def\GetTitleStringResult{#1}%
  \global\let\GTS@GlobalString\GetTitleStringResult
  \begingroup
    \GTS@RemoveLeft
    \GTS@RemoveRight
  \endgroup
  \let\GetTitleStringResult\GTS@GlobalString
}
%    \end{macrocode}
%    \end{macro}
%
% \subsubsection{Expand method}
%
%    \begin{macro}{\GTS@DisablePredefinedCmds}
%    \begin{macrocode}
\def\GTS@DisablePredefinedCmds{%
  \let\label\@gobble
  \let\zlabel\@gobble
  \let\zref@label\@gobble
  \let\zref@labelbylist\@gobbletwo
  \let\zref@labelbyprops\@gobbletwo
  \let\index\@gobble
  \let\glossary\@gobble
  \let\markboth\@gobbletwo
  \let\@mkboth\@gobbletwo
  \let\markright\@gobble
  \let\phantomsection\@empty
  \def\addcontentsline{\expandafter\@gobble\@gobbletwo}%
  \let\raggedright\@empty
  \let\raggedleft\@empty
  \let\centering\@empty
  \let\protect\@unexpandable@protect
  \let\enit@format\@empty % package enumitem
}
%    \end{macrocode}
%    \end{macro}
%
%    \begin{macro}{\GTS@DisableHook}
%    \begin{macrocode}
\providecommand*{\GTS@DisableHook}{}
%    \end{macrocode}
%    \end{macro}
%    \begin{macro}{\GetTitleStringDisableCommands}
%    \begin{macrocode}
\def\GetTitleStringDisableCommands{%
  \begingroup
    \makeatletter
    \GTS@DisableCommands
}
%    \end{macrocode}
%    \end{macro}
%    \begin{macro}{\GTS@DisableCommands}
%    \begin{macrocode}
\long\def\GTS@DisableCommands#1{%
    \toks0=\expandafter{\GTS@DisableHook}%
    \toks2={#1}%
    \xdef\GTS@GlobalString{\the\toks0 \the\toks2}%
  \endgroup
  \let\GTS@DisableHook\GTS@GlobalString
}
%    \end{macrocode}
%    \end{macro}
%
% \subsubsection{Non-expand method}
%
%    \begin{macrocode}
\def\GTS@RemoveLeft{%
  \toks@\expandafter\expandafter\expandafter{%
    \expandafter\GTS@Car\GTS@GlobalString{}{}{}{}\GTS@Nil
  }%
  \edef\GTS@Token{\the\toks@}%
  \GTS@PredefinedLeftCmds
  \expandafter\futurelet\expandafter\GTS@Token
  \expandafter\GTS@TestLeftSpace\GTS@GlobalString\GTS@Nil
  \GTS@End
}
\def\GTS@End{}
\long\def\GTS@TestLeft#1#2{%
  \def\GTS@temp{#1}%
  \ifx\GTS@temp\GTS@Token
    \toks@\expandafter\expandafter\expandafter{%
      \expandafter#2\GTS@GlobalString\GTS@Nil
    }%
    \expandafter\GTS@TestLeftEnd
  \fi
}
\long\def\GTS@TestLeftEnd#1\GTS@End{%
  \xdef\GTS@GlobalString{\the\toks@}%
  \GTS@RemoveLeft
}
\long\def\GTS@Car#1#2\GTS@Nil{#1}
\long\def\GTS@Cdr#1#2\GTS@Nil{#2}
\long\def\GTS@CdrTwo#1#2#3\GTS@Nil{#3}
\long\def\GTS@CdrThree#1#2#3#4\GTS@Nil{#4}
\long\def\GTS@CdrFour#1#2#3#4#5\GTS@Nil{#5}
\long\def\GTS@TestLeftSpace#1\GTS@Nil{%
  \ifx\GTS@Token\@sptoken
    \toks@\expandafter{%
      \romannumeral-0\GTS@GlobalString
    }%
    \expandafter\GTS@TestLeftEnd
  \fi
}
%    \end{macrocode}
%    \begin{macro}{\GTS@PredefinedLeftCmds}
%    \begin{macrocode}
\def\GTS@PredefinedLeftCmds{%
  \GTS@TestLeft\Hy@phantomsection\GTS@Cdr
  \GTS@TestLeft\Hy@SectionAnchor\GTS@Cdr
  \GTS@TestLeft\Hy@SectionAnchorHref\GTS@CdrTwo
  \GTS@TestLeft\label\GTS@CdrTwo
  \GTS@TestLeft\zlabel\GTS@CdrTwo
  \GTS@TestLeft\index\GTS@CdrTwo
  \GTS@TestLeft\glossary\GTS@CdrTwo
  \GTS@TestLeft\markboth\GTS@CdrThree
  \GTS@TestLeft\@mkboth\GTS@CdrThree
  \GTS@TestLeft\addcontentsline\GTS@CdrFour
  \GTS@TestLeft\enit@format\GTS@Cdr % package enumitem
}
%    \end{macrocode}
%    \end{macro}
%
%    \begin{macrocode}
\def\GTS@RemoveRight{%
  \toks@{}%
  \expandafter\GTS@TestRightLabel\GTS@GlobalString
      \label{}\GTS@Nil\@nil
  \GTS@RemoveRightSpace
}
\begingroup
  \def\GTS@temp#1{\endgroup
    \def\GTS@RemoveRightSpace{%
      \expandafter\GTS@TestRightSpace\GTS@GlobalString
          \GTS@Nil#1\GTS@Nil\@nil
    }%
  }%
\GTS@temp{ }
\def\GTS@TestRightSpace#1 \GTS@Nil#2\@nil{%
  \ifx\relax#2\relax
  \else
    \gdef\GTS@GlobalString{#1}%
    \expandafter\GTS@RemoveRightSpace
  \fi
}
\def\GTS@TestRightLabel#1\label#2#3\GTS@Nil#4\@nil{%
  \def\GTS@temp{#3}%
  \ifx\GTS@temp\@empty
    \expandafter\gdef\expandafter\GTS@GlobalString\expandafter{%
      \the\toks@
      #1%
    }%
    \expandafter\@gobble
  \else
    \expandafter\@firstofone
  \fi
  {%
    \toks@\expandafter{\the\toks@#1}%
    \GTS@TestRightLabel#3\GTS@Nil\@nil
  }%
}
%    \end{macrocode}
%
%    \begin{macrocode}
\GTS@AtEnd%
%</package>
%    \end{macrocode}
%
% \section{Test}
%
% \subsection{Catcode checks for loading}
%
%    \begin{macrocode}
%<*test1>
%    \end{macrocode}
%    \begin{macrocode}
\catcode`\{=1 %
\catcode`\}=2 %
\catcode`\#=6 %
\catcode`\@=11 %
\expandafter\ifx\csname count@\endcsname\relax
  \countdef\count@=255 %
\fi
\expandafter\ifx\csname @gobble\endcsname\relax
  \long\def\@gobble#1{}%
\fi
\expandafter\ifx\csname @firstofone\endcsname\relax
  \long\def\@firstofone#1{#1}%
\fi
\expandafter\ifx\csname loop\endcsname\relax
  \expandafter\@firstofone
\else
  \expandafter\@gobble
\fi
{%
  \def\loop#1\repeat{%
    \def\body{#1}%
    \iterate
  }%
  \def\iterate{%
    \body
      \let\next\iterate
    \else
      \let\next\relax
    \fi
    \next
  }%
  \let\repeat=\fi
}%
\def\RestoreCatcodes{}
\count@=0 %
\loop
  \edef\RestoreCatcodes{%
    \RestoreCatcodes
    \catcode\the\count@=\the\catcode\count@\relax
  }%
\ifnum\count@<255 %
  \advance\count@ 1 %
\repeat

\def\RangeCatcodeInvalid#1#2{%
  \count@=#1\relax
  \loop
    \catcode\count@=15 %
  \ifnum\count@<#2\relax
    \advance\count@ 1 %
  \repeat
}
\def\RangeCatcodeCheck#1#2#3{%
  \count@=#1\relax
  \loop
    \ifnum#3=\catcode\count@
    \else
      \errmessage{%
        Character \the\count@\space
        with wrong catcode \the\catcode\count@\space
        instead of \number#3%
      }%
    \fi
  \ifnum\count@<#2\relax
    \advance\count@ 1 %
  \repeat
}
\def\space{ }
\expandafter\ifx\csname LoadCommand\endcsname\relax
  \def\LoadCommand{\input gettitlestring.sty\relax}%
\fi
\def\Test{%
  \RangeCatcodeInvalid{0}{47}%
  \RangeCatcodeInvalid{58}{64}%
  \RangeCatcodeInvalid{91}{96}%
  \RangeCatcodeInvalid{123}{255}%
  \catcode`\@=12 %
  \catcode`\\=0 %
  \catcode`\%=14 %
  \LoadCommand
  \RangeCatcodeCheck{0}{36}{15}%
  \RangeCatcodeCheck{37}{37}{14}%
  \RangeCatcodeCheck{38}{47}{15}%
  \RangeCatcodeCheck{48}{57}{12}%
  \RangeCatcodeCheck{58}{63}{15}%
  \RangeCatcodeCheck{64}{64}{12}%
  \RangeCatcodeCheck{65}{90}{11}%
  \RangeCatcodeCheck{91}{91}{15}%
  \RangeCatcodeCheck{92}{92}{0}%
  \RangeCatcodeCheck{93}{96}{15}%
  \RangeCatcodeCheck{97}{122}{11}%
  \RangeCatcodeCheck{123}{255}{15}%
  \RestoreCatcodes
}
\Test
\csname @@end\endcsname
\end
%    \end{macrocode}
%    \begin{macrocode}
%</test1>
%    \end{macrocode}
%
% \subsection{Test of non-expand method}
%
%    \begin{macrocode}
%<*test2>
\NeedsTeXFormat{LaTeX2e}
\documentclass{minimal}
\usepackage{gettitlestring}[2016/05/16]
\usepackage{qstest}
\IncludeTests{*}
\LogTests{log}{*}{*}
\begin{document}
\begin{qstest}{non-expand}{non-expand}
  \def\test#1#2{%
    \sbox0{%
      \GetTitleString{#1}%
      \Expect{#2}*{\GetTitleStringResult}%
    }%
    \Expect{0.0pt}*{\the\wd0}%
  }%
  \test{}{}%
  \test{ }{}%
  \test{ x }{x}%
  \test{ x y }{x y}%
  \test{ \relax}{\relax}%
  \test{\label{f}a}{a}%
  \test{ \label{f}a}{a}%
  \test{\label{f} a}{a}%
  \test{ \label{f} a}{a}%
  \test{a\label{f}}{a}%
  \test{a\label{f} }{a}%
  \test{a \label{f}}{a}%
  \test{a \label{f} }{a}%
  \test{a\label{f}b\label{g}}{ab}%
  \test{a \label{f}b \label{g} }{a b}%
  \test{a\label{f} b \label{g} }{a b}%
\end{qstest}
\end{document}
%</test2>
%    \end{macrocode}
%
% \section{Installation}
%
% \subsection{Download}
%
% \paragraph{Package.} This package is available on
% CTAN\footnote{\CTANpkg{gettitlestring}}:
% \begin{description}
% \item[\CTAN{macros/latex/contrib/oberdiek/gettitlestring.dtx}] The source file.
% \item[\CTAN{macros/latex/contrib/oberdiek/gettitlestring.pdf}] Documentation.
% \end{description}
%
%
% \paragraph{Bundle.} All the packages of the bundle `oberdiek'
% are also available in a TDS compliant ZIP archive. There
% the packages are already unpacked and the documentation files
% are generated. The files and directories obey the TDS standard.
% \begin{description}
% \item[\CTANinstall{install/macros/latex/contrib/oberdiek.tds.zip}]
% \end{description}
% \emph{TDS} refers to the standard ``A Directory Structure
% for \TeX\ Files'' (\CTAN{tds/tds.pdf}). Directories
% with \xfile{texmf} in their name are usually organized this way.
%
% \subsection{Bundle installation}
%
% \paragraph{Unpacking.} Unpack the \xfile{oberdiek.tds.zip} in the
% TDS tree (also known as \xfile{texmf} tree) of your choice.
% Example (linux):
% \begin{quote}
%   |unzip oberdiek.tds.zip -d ~/texmf|
% \end{quote}
%
% \paragraph{Script installation.}
% Check the directory \xfile{TDS:scripts/oberdiek/} for
% scripts that need further installation steps.
% Package \xpackage{attachfile2} comes with the Perl script
% \xfile{pdfatfi.pl} that should be installed in such a way
% that it can be called as \texttt{pdfatfi}.
% Example (linux):
% \begin{quote}
%   |chmod +x scripts/oberdiek/pdfatfi.pl|\\
%   |cp scripts/oberdiek/pdfatfi.pl /usr/local/bin/|
% \end{quote}
%
% \subsection{Package installation}
%
% \paragraph{Unpacking.} The \xfile{.dtx} file is a self-extracting
% \docstrip\ archive. The files are extracted by running the
% \xfile{.dtx} through \plainTeX:
% \begin{quote}
%   \verb|tex gettitlestring.dtx|
% \end{quote}
%
% \paragraph{TDS.} Now the different files must be moved into
% the different directories in your installation TDS tree
% (also known as \xfile{texmf} tree):
% \begin{quote}
% \def\t{^^A
% \begin{tabular}{@{}>{\ttfamily}l@{ $\rightarrow$ }>{\ttfamily}l@{}}
%   gettitlestring.sty & tex/generic/oberdiek/gettitlestring.sty\\
%   gettitlestring.pdf & doc/latex/oberdiek/gettitlestring.pdf\\
%   test/gettitlestring-test1.tex & doc/latex/oberdiek/test/gettitlestring-test1.tex\\
%   test/gettitlestring-test2.tex & doc/latex/oberdiek/test/gettitlestring-test2.tex\\
%   gettitlestring.dtx & source/latex/oberdiek/gettitlestring.dtx\\
% \end{tabular}^^A
% }^^A
% \sbox0{\t}^^A
% \ifdim\wd0>\linewidth
%   \begingroup
%     \advance\linewidth by\leftmargin
%     \advance\linewidth by\rightmargin
%   \edef\x{\endgroup
%     \def\noexpand\lw{\the\linewidth}^^A
%   }\x
%   \def\lwbox{^^A
%     \leavevmode
%     \hbox to \linewidth{^^A
%       \kern-\leftmargin\relax
%       \hss
%       \usebox0
%       \hss
%       \kern-\rightmargin\relax
%     }^^A
%   }^^A
%   \ifdim\wd0>\lw
%     \sbox0{\small\t}^^A
%     \ifdim\wd0>\linewidth
%       \ifdim\wd0>\lw
%         \sbox0{\footnotesize\t}^^A
%         \ifdim\wd0>\linewidth
%           \ifdim\wd0>\lw
%             \sbox0{\scriptsize\t}^^A
%             \ifdim\wd0>\linewidth
%               \ifdim\wd0>\lw
%                 \sbox0{\tiny\t}^^A
%                 \ifdim\wd0>\linewidth
%                   \lwbox
%                 \else
%                   \usebox0
%                 \fi
%               \else
%                 \lwbox
%               \fi
%             \else
%               \usebox0
%             \fi
%           \else
%             \lwbox
%           \fi
%         \else
%           \usebox0
%         \fi
%       \else
%         \lwbox
%       \fi
%     \else
%       \usebox0
%     \fi
%   \else
%     \lwbox
%   \fi
% \else
%   \usebox0
% \fi
% \end{quote}
% If you have a \xfile{docstrip.cfg} that configures and enables \docstrip's
% TDS installing feature, then some files can already be in the right
% place, see the documentation of \docstrip.
%
% \subsection{Refresh file name databases}
%
% If your \TeX~distribution
% (\teTeX, \mikTeX, \dots) relies on file name databases, you must refresh
% these. For example, \teTeX\ users run \verb|texhash| or
% \verb|mktexlsr|.
%
% \subsection{Some details for the interested}
%
% \paragraph{Attached source.}
%
% The PDF documentation on CTAN also includes the
% \xfile{.dtx} source file. It can be extracted by
% AcrobatReader 6 or higher. Another option is \textsf{pdftk},
% e.g. unpack the file into the current directory:
% \begin{quote}
%   \verb|pdftk gettitlestring.pdf unpack_files output .|
% \end{quote}
%
% \paragraph{Unpacking with \LaTeX.}
% The \xfile{.dtx} chooses its action depending on the format:
% \begin{description}
% \item[\plainTeX:] Run \docstrip\ and extract the files.
% \item[\LaTeX:] Generate the documentation.
% \end{description}
% If you insist on using \LaTeX\ for \docstrip\ (really,
% \docstrip\ does not need \LaTeX), then inform the autodetect routine
% about your intention:
% \begin{quote}
%   \verb|latex \let\install=y% \iffalse meta-comment
%
% File: gettitlestring.dtx
% Version: 2016/05/16 v1.5
% Info: Cleanup title references
%
% Copyright (C) 2009, 2010 by
%    Heiko Oberdiek <heiko.oberdiek at googlemail.com>
%    2016
%    https://github.com/ho-tex/oberdiek/issues
%
% This work may be distributed and/or modified under the
% conditions of the LaTeX Project Public License, either
% version 1.3c of this license or (at your option) any later
% version. This version of this license is in
%    http://www.latex-project.org/lppl/lppl-1-3c.txt
% and the latest version of this license is in
%    http://www.latex-project.org/lppl.txt
% and version 1.3 or later is part of all distributions of
% LaTeX version 2005/12/01 or later.
%
% This work has the LPPL maintenance status "maintained".
%
% This Current Maintainer of this work is Heiko Oberdiek.
%
% The Base Interpreter refers to any `TeX-Format',
% because some files are installed in TDS:tex/generic//.
%
% This work consists of the main source file gettitlestring.dtx
% and the derived files
%    gettitlestring.sty, gettitlestring.pdf, gettitlestring.ins,
%    gettitlestring.drv, gettitlestring-test1.tex,
%    gettitlestring-test2.tex.
%
% Distribution:
%    CTAN:macros/latex/contrib/oberdiek/gettitlestring.dtx
%    CTAN:macros/latex/contrib/oberdiek/gettitlestring.pdf
%
% Unpacking:
%    (a) If gettitlestring.ins is present:
%           tex gettitlestring.ins
%    (b) Without gettitlestring.ins:
%           tex gettitlestring.dtx
%    (c) If you insist on using LaTeX
%           latex \let\install=y\input{gettitlestring.dtx}
%        (quote the arguments according to the demands of your shell)
%
% Documentation:
%    (a) If gettitlestring.drv is present:
%           latex gettitlestring.drv
%    (b) Without gettitlestring.drv:
%           latex gettitlestring.dtx; ...
%    The class ltxdoc loads the configuration file ltxdoc.cfg
%    if available. Here you can specify further options, e.g.
%    use A4 as paper format:
%       \PassOptionsToClass{a4paper}{article}
%
%    Programm calls to get the documentation (example):
%       pdflatex gettitlestring.dtx
%       makeindex -s gind.ist gettitlestring.idx
%       pdflatex gettitlestring.dtx
%       makeindex -s gind.ist gettitlestring.idx
%       pdflatex gettitlestring.dtx
%
% Installation:
%    TDS:tex/generic/oberdiek/gettitlestring.sty
%    TDS:doc/latex/oberdiek/gettitlestring.pdf
%    TDS:doc/latex/oberdiek/test/gettitlestring-test1.tex
%    TDS:doc/latex/oberdiek/test/gettitlestring-test2.tex
%    TDS:source/latex/oberdiek/gettitlestring.dtx
%
%<*ignore>
\begingroup
  \catcode123=1 %
  \catcode125=2 %
  \def\x{LaTeX2e}%
\expandafter\endgroup
\ifcase 0\ifx\install y1\fi\expandafter
         \ifx\csname processbatchFile\endcsname\relax\else1\fi
         \ifx\fmtname\x\else 1\fi\relax
\else\csname fi\endcsname
%</ignore>
%<*install>
\input docstrip.tex
\Msg{************************************************************************}
\Msg{* Installation}
\Msg{* Package: gettitlestring 2016/05/16 v1.5 Cleanup title references (HO)}
\Msg{************************************************************************}

\keepsilent
\askforoverwritefalse

\let\MetaPrefix\relax
\preamble

This is a generated file.

Project: gettitlestring
Version: 2016/05/16 v1.5

Copyright (C) 2009, 2010 by
   Heiko Oberdiek <heiko.oberdiek at googlemail.com>

This work may be distributed and/or modified under the
conditions of the LaTeX Project Public License, either
version 1.3c of this license or (at your option) any later
version. This version of this license is in
   http://www.latex-project.org/lppl/lppl-1-3c.txt
and the latest version of this license is in
   http://www.latex-project.org/lppl.txt
and version 1.3 or later is part of all distributions of
LaTeX version 2005/12/01 or later.

This work has the LPPL maintenance status "maintained".

This Current Maintainer of this work is Heiko Oberdiek.

The Base Interpreter refers to any `TeX-Format',
because some files are installed in TDS:tex/generic//.

This work consists of the main source file gettitlestring.dtx
and the derived files
   gettitlestring.sty, gettitlestring.pdf, gettitlestring.ins,
   gettitlestring.drv, gettitlestring-test1.tex,
   gettitlestring-test2.tex.

\endpreamble
\let\MetaPrefix\DoubleperCent

\generate{%
  \file{gettitlestring.ins}{\from{gettitlestring.dtx}{install}}%
  \file{gettitlestring.drv}{\from{gettitlestring.dtx}{driver}}%
  \usedir{tex/generic/oberdiek}%
  \file{gettitlestring.sty}{\from{gettitlestring.dtx}{package}}%
%  \usedir{doc/latex/oberdiek/test}%
%  \file{gettitlestring-test1.tex}{\from{gettitlestring.dtx}{test1}}%
%  \file{gettitlestring-test2.tex}{\from{gettitlestring.dtx}{test2}}%
  \nopreamble
  \nopostamble
%  \usedir{source/latex/oberdiek/catalogue}%
%  \file{gettitlestring.xml}{\from{gettitlestring.dtx}{catalogue}}%
}

\catcode32=13\relax% active space
\let =\space%
\Msg{************************************************************************}
\Msg{*}
\Msg{* To finish the installation you have to move the following}
\Msg{* file into a directory searched by TeX:}
\Msg{*}
\Msg{*     gettitlestring.sty}
\Msg{*}
\Msg{* To produce the documentation run the file `gettitlestring.drv'}
\Msg{* through LaTeX.}
\Msg{*}
\Msg{* Happy TeXing!}
\Msg{*}
\Msg{************************************************************************}

\endbatchfile
%</install>
%<*ignore>
\fi
%</ignore>
%<*driver>
\NeedsTeXFormat{LaTeX2e}
\ProvidesFile{gettitlestring.drv}%
  [2016/05/16 v1.5 Cleanup title references (HO)]%
\documentclass{ltxdoc}
\usepackage{holtxdoc}[2011/11/22]
\begin{document}
  \DocInput{gettitlestring.dtx}%
\end{document}
%</driver>
% \fi
%
%
% \CharacterTable
%  {Upper-case    \A\B\C\D\E\F\G\H\I\J\K\L\M\N\O\P\Q\R\S\T\U\V\W\X\Y\Z
%   Lower-case    \a\b\c\d\e\f\g\h\i\j\k\l\m\n\o\p\q\r\s\t\u\v\w\x\y\z
%   Digits        \0\1\2\3\4\5\6\7\8\9
%   Exclamation   \!     Double quote  \"     Hash (number) \#
%   Dollar        \$     Percent       \%     Ampersand     \&
%   Acute accent  \'     Left paren    \(     Right paren   \)
%   Asterisk      \*     Plus          \+     Comma         \,
%   Minus         \-     Point         \.     Solidus       \/
%   Colon         \:     Semicolon     \;     Less than     \<
%   Equals        \=     Greater than  \>     Question mark \?
%   Commercial at \@     Left bracket  \[     Backslash     \\
%   Right bracket \]     Circumflex    \^     Underscore    \_
%   Grave accent  \`     Left brace    \{     Vertical bar  \|
%   Right brace   \}     Tilde         \~}
%
% \GetFileInfo{gettitlestring.drv}
%
% \title{The \xpackage{gettitlestring} package}
% \date{2016/05/16 v1.5}
% \author{Heiko Oberdiek\thanks
% {Please report any issues at \url{https://github.com/ho-tex/oberdiek/issues}}\\
% \xemail{heiko.oberdiek at googlemail.com}}
%
% \maketitle
%
% \begin{abstract}
% The \LaTeX\ package addresses packages that are dealing with
% references to titles (\cs{section}, \cs{caption}, \dots).
% The package tries to remove \cs{label} and other
% commands from title strings.
% \end{abstract}
%
% \tableofcontents
%
% \section{Documentation}
%
% \subsection{Macros}
%
% \begin{declcs}{GetTitleStringSetup} \M{key value list}
% \end{declcs}
% The options are given as comma separated key value pairs.
% See section \ref{sec:options}.
%
% \begin{declcs}{GetTitleString} \M{text}\\
% \cs{GetTitleStringExpand} \M{text}\\
% \cs{GetTitleStringNonExpand} \M{text}
% \end{declcs}
% Macro \cs{GetTitleString} tries to remove unwanted stuff from \meta{text}
% the result is stored in Macro \cs{GetTitleStringResult}.
% Two methods are available:
% \begin{description}
% \item[\cs{GetTitleStringExpand}:]
% The \meta{text} is expanded in a context where the unwanted
% macros are redefined to remove themselves.
% This is the method used in packages \xpackage{titleref}~\cite{titleref},
% \xpackage{zref-titleref}~\cite{zref}
% or class \xclass{memoir}~\cite{memoir}.
% \cs{protect} is supported, but fragile material might break.
% \item[\cs{GetTitleStringNonExpand}:]
% The \meta{text} is not expanded. Thus the removal of unwanted
% material is more difficult. It is especially removed at the
% start of the \meta{text} and spaces are removed from the end.
% Currently only \cs{label} is removed in the whole string,
% if it is not hidden inside curly braces or part of macro
% definitions. Thus the removal of unwanted stuff might not be
% complete, but fragile material will not break.
% (But the result string can break at a later time, of course).
% \end{description}
% Option \xoption{expand} controls which method is used by
% macro \cs{GetTitleString}.
%
% \begin{declcs}{GetTitleStringDisableCommands} \M{code}
% \end{declcs}
% The \meta{code} is called right before the
% text is expanded in \cs{GetTitleStringExpand}.
% Additional definitions can be given for macros that
% should be removed.
% Keep in mind that expansion means that the definitions
% must work in expandable context. Macros like
% \cs{@ifstar} or \cs{@ifnextchar} or optional arguments
% will not work. The macro names in \meta{code} may contain
% the at sign |@|, it has catcode 11 (letter).
%
% \subsection{Options}\label{sec:options}
%
% \begin{description}
% \item[\xoption{expand}:] Boolean option, takes values |true| or |false|.
% No value means |true|. The option specifies the method to remove
% unwanted stuff from the title string, see below.
% \end{description}
% Options can be set at the following places:
% \begin{itemize}
% \item \cs{usepackage}
% \item Configuration file \xfile{gettitlestring.cfg}.
% \item \cs{GetTitleStringSetup}
% \end{itemize}
%
% \StopEventually{
% }
%
% \section{Implementation}
%    \begin{macrocode}
%<*package>
%    \end{macrocode}
%    Reload check, especially if the package is not used with \LaTeX.
%    \begin{macrocode}
\begingroup\catcode61\catcode48\catcode32=10\relax%
  \catcode13=5 % ^^M
  \endlinechar=13 %
  \catcode35=6 % #
  \catcode39=12 % '
  \catcode44=12 % ,
  \catcode45=12 % -
  \catcode46=12 % .
  \catcode58=12 % :
  \catcode64=11 % @
  \catcode123=1 % {
  \catcode125=2 % }
  \expandafter\let\expandafter\x\csname ver@gettitlestring.sty\endcsname
  \ifx\x\relax % plain-TeX, first loading
  \else
    \def\empty{}%
    \ifx\x\empty % LaTeX, first loading,
      % variable is initialized, but \ProvidesPackage not yet seen
    \else
      \expandafter\ifx\csname PackageInfo\endcsname\relax
        \def\x#1#2{%
          \immediate\write-1{Package #1 Info: #2.}%
        }%
      \else
        \def\x#1#2{\PackageInfo{#1}{#2, stopped}}%
      \fi
      \x{gettitlestring}{The package is already loaded}%
      \aftergroup\endinput
    \fi
  \fi
\endgroup%
%    \end{macrocode}
%    Package identification:
%    \begin{macrocode}
\begingroup\catcode61\catcode48\catcode32=10\relax%
  \catcode13=5 % ^^M
  \endlinechar=13 %
  \catcode35=6 % #
  \catcode39=12 % '
  \catcode40=12 % (
  \catcode41=12 % )
  \catcode44=12 % ,
  \catcode45=12 % -
  \catcode46=12 % .
  \catcode47=12 % /
  \catcode58=12 % :
  \catcode64=11 % @
  \catcode91=12 % [
  \catcode93=12 % ]
  \catcode123=1 % {
  \catcode125=2 % }
  \expandafter\ifx\csname ProvidesPackage\endcsname\relax
    \def\x#1#2#3[#4]{\endgroup
      \immediate\write-1{Package: #3 #4}%
      \xdef#1{#4}%
    }%
  \else
    \def\x#1#2[#3]{\endgroup
      #2[{#3}]%
      \ifx#1\@undefined
        \xdef#1{#3}%
      \fi
      \ifx#1\relax
        \xdef#1{#3}%
      \fi
    }%
  \fi
\expandafter\x\csname ver@gettitlestring.sty\endcsname
\ProvidesPackage{gettitlestring}%
  [2016/05/16 v1.5 Cleanup title references (HO)]%
%    \end{macrocode}
%
%    \begin{macrocode}
\begingroup\catcode61\catcode48\catcode32=10\relax%
  \catcode13=5 % ^^M
  \endlinechar=13 %
  \catcode123=1 % {
  \catcode125=2 % }
  \catcode64=11 % @
  \def\x{\endgroup
    \expandafter\edef\csname GTS@AtEnd\endcsname{%
      \endlinechar=\the\endlinechar\relax
      \catcode13=\the\catcode13\relax
      \catcode32=\the\catcode32\relax
      \catcode35=\the\catcode35\relax
      \catcode61=\the\catcode61\relax
      \catcode64=\the\catcode64\relax
      \catcode123=\the\catcode123\relax
      \catcode125=\the\catcode125\relax
    }%
  }%
\x\catcode61\catcode48\catcode32=10\relax%
\catcode13=5 % ^^M
\endlinechar=13 %
\catcode35=6 % #
\catcode64=11 % @
\catcode123=1 % {
\catcode125=2 % }
\def\TMP@EnsureCode#1#2{%
  \edef\GTS@AtEnd{%
    \GTS@AtEnd
    \catcode#1=\the\catcode#1\relax
  }%
  \catcode#1=#2\relax
}
\TMP@EnsureCode{42}{12}% *
\TMP@EnsureCode{44}{12}% ,
\TMP@EnsureCode{45}{12}% -
\TMP@EnsureCode{46}{12}% .
\TMP@EnsureCode{47}{12}% /
\TMP@EnsureCode{91}{12}% [
\TMP@EnsureCode{93}{12}% ]
\edef\GTS@AtEnd{\GTS@AtEnd\noexpand\endinput}
%    \end{macrocode}
%
% \subsection{Options}
%
%    \begin{macrocode}
\RequirePackage{kvoptions}[2009/07/17]
\SetupKeyvalOptions{%
  family=gettitlestring,%
  prefix=GTS@%
}
\newcommand*{\GetTitleStringSetup}{%
  \setkeys{gettitlestring}%
}
\DeclareBoolOption{expand}
\InputIfFileExists{gettitlestring.cfg}{}{}
\ProcessKeyvalOptions*\relax
%    \end{macrocode}
%
% \subsection{\cs{GetTitleString}}
%
%    \begin{macro}{\GetTitleString}
%    \begin{macrocode}
\newcommand*{\GetTitleString}{%
  \ifGTS@expand
    \expandafter\GetTitleStringExpand
  \else
    \expandafter\GetTitleStringNonExpand
  \fi
}
%    \end{macrocode}
%    \end{macro}
%    \begin{macro}{\GetTitleStringExpand}
%    \begin{macrocode}
\newcommand{\GetTitleStringExpand}[1]{%
  \def\GetTitleStringResult{#1}%
  \begingroup
    \GTS@DisablePredefinedCmds
    \GTS@DisableHook
    \edef\x{\endgroup
      \noexpand\def\noexpand\GetTitleStringResult{%
        \GetTitleStringResult
      }%
    }%
  \x
}
%    \end{macrocode}
%    \end{macro}
%    \begin{macro}{\GetTitleString}
%    \begin{macrocode}
\newcommand{\GetTitleStringNonExpand}[1]{%
  \def\GetTitleStringResult{#1}%
  \global\let\GTS@GlobalString\GetTitleStringResult
  \begingroup
    \GTS@RemoveLeft
    \GTS@RemoveRight
  \endgroup
  \let\GetTitleStringResult\GTS@GlobalString
}
%    \end{macrocode}
%    \end{macro}
%
% \subsubsection{Expand method}
%
%    \begin{macro}{\GTS@DisablePredefinedCmds}
%    \begin{macrocode}
\def\GTS@DisablePredefinedCmds{%
  \let\label\@gobble
  \let\zlabel\@gobble
  \let\zref@label\@gobble
  \let\zref@labelbylist\@gobbletwo
  \let\zref@labelbyprops\@gobbletwo
  \let\index\@gobble
  \let\glossary\@gobble
  \let\markboth\@gobbletwo
  \let\@mkboth\@gobbletwo
  \let\markright\@gobble
  \let\phantomsection\@empty
  \def\addcontentsline{\expandafter\@gobble\@gobbletwo}%
  \let\raggedright\@empty
  \let\raggedleft\@empty
  \let\centering\@empty
  \let\protect\@unexpandable@protect
  \let\enit@format\@empty % package enumitem
}
%    \end{macrocode}
%    \end{macro}
%
%    \begin{macro}{\GTS@DisableHook}
%    \begin{macrocode}
\providecommand*{\GTS@DisableHook}{}
%    \end{macrocode}
%    \end{macro}
%    \begin{macro}{\GetTitleStringDisableCommands}
%    \begin{macrocode}
\def\GetTitleStringDisableCommands{%
  \begingroup
    \makeatletter
    \GTS@DisableCommands
}
%    \end{macrocode}
%    \end{macro}
%    \begin{macro}{\GTS@DisableCommands}
%    \begin{macrocode}
\long\def\GTS@DisableCommands#1{%
    \toks0=\expandafter{\GTS@DisableHook}%
    \toks2={#1}%
    \xdef\GTS@GlobalString{\the\toks0 \the\toks2}%
  \endgroup
  \let\GTS@DisableHook\GTS@GlobalString
}
%    \end{macrocode}
%    \end{macro}
%
% \subsubsection{Non-expand method}
%
%    \begin{macrocode}
\def\GTS@RemoveLeft{%
  \toks@\expandafter\expandafter\expandafter{%
    \expandafter\GTS@Car\GTS@GlobalString{}{}{}{}\GTS@Nil
  }%
  \edef\GTS@Token{\the\toks@}%
  \GTS@PredefinedLeftCmds
  \expandafter\futurelet\expandafter\GTS@Token
  \expandafter\GTS@TestLeftSpace\GTS@GlobalString\GTS@Nil
  \GTS@End
}
\def\GTS@End{}
\long\def\GTS@TestLeft#1#2{%
  \def\GTS@temp{#1}%
  \ifx\GTS@temp\GTS@Token
    \toks@\expandafter\expandafter\expandafter{%
      \expandafter#2\GTS@GlobalString\GTS@Nil
    }%
    \expandafter\GTS@TestLeftEnd
  \fi
}
\long\def\GTS@TestLeftEnd#1\GTS@End{%
  \xdef\GTS@GlobalString{\the\toks@}%
  \GTS@RemoveLeft
}
\long\def\GTS@Car#1#2\GTS@Nil{#1}
\long\def\GTS@Cdr#1#2\GTS@Nil{#2}
\long\def\GTS@CdrTwo#1#2#3\GTS@Nil{#3}
\long\def\GTS@CdrThree#1#2#3#4\GTS@Nil{#4}
\long\def\GTS@CdrFour#1#2#3#4#5\GTS@Nil{#5}
\long\def\GTS@TestLeftSpace#1\GTS@Nil{%
  \ifx\GTS@Token\@sptoken
    \toks@\expandafter{%
      \romannumeral-0\GTS@GlobalString
    }%
    \expandafter\GTS@TestLeftEnd
  \fi
}
%    \end{macrocode}
%    \begin{macro}{\GTS@PredefinedLeftCmds}
%    \begin{macrocode}
\def\GTS@PredefinedLeftCmds{%
  \GTS@TestLeft\Hy@phantomsection\GTS@Cdr
  \GTS@TestLeft\Hy@SectionAnchor\GTS@Cdr
  \GTS@TestLeft\Hy@SectionAnchorHref\GTS@CdrTwo
  \GTS@TestLeft\label\GTS@CdrTwo
  \GTS@TestLeft\zlabel\GTS@CdrTwo
  \GTS@TestLeft\index\GTS@CdrTwo
  \GTS@TestLeft\glossary\GTS@CdrTwo
  \GTS@TestLeft\markboth\GTS@CdrThree
  \GTS@TestLeft\@mkboth\GTS@CdrThree
  \GTS@TestLeft\addcontentsline\GTS@CdrFour
  \GTS@TestLeft\enit@format\GTS@Cdr % package enumitem
}
%    \end{macrocode}
%    \end{macro}
%
%    \begin{macrocode}
\def\GTS@RemoveRight{%
  \toks@{}%
  \expandafter\GTS@TestRightLabel\GTS@GlobalString
      \label{}\GTS@Nil\@nil
  \GTS@RemoveRightSpace
}
\begingroup
  \def\GTS@temp#1{\endgroup
    \def\GTS@RemoveRightSpace{%
      \expandafter\GTS@TestRightSpace\GTS@GlobalString
          \GTS@Nil#1\GTS@Nil\@nil
    }%
  }%
\GTS@temp{ }
\def\GTS@TestRightSpace#1 \GTS@Nil#2\@nil{%
  \ifx\relax#2\relax
  \else
    \gdef\GTS@GlobalString{#1}%
    \expandafter\GTS@RemoveRightSpace
  \fi
}
\def\GTS@TestRightLabel#1\label#2#3\GTS@Nil#4\@nil{%
  \def\GTS@temp{#3}%
  \ifx\GTS@temp\@empty
    \expandafter\gdef\expandafter\GTS@GlobalString\expandafter{%
      \the\toks@
      #1%
    }%
    \expandafter\@gobble
  \else
    \expandafter\@firstofone
  \fi
  {%
    \toks@\expandafter{\the\toks@#1}%
    \GTS@TestRightLabel#3\GTS@Nil\@nil
  }%
}
%    \end{macrocode}
%
%    \begin{macrocode}
\GTS@AtEnd%
%</package>
%    \end{macrocode}
%
% \section{Test}
%
% \subsection{Catcode checks for loading}
%
%    \begin{macrocode}
%<*test1>
%    \end{macrocode}
%    \begin{macrocode}
\catcode`\{=1 %
\catcode`\}=2 %
\catcode`\#=6 %
\catcode`\@=11 %
\expandafter\ifx\csname count@\endcsname\relax
  \countdef\count@=255 %
\fi
\expandafter\ifx\csname @gobble\endcsname\relax
  \long\def\@gobble#1{}%
\fi
\expandafter\ifx\csname @firstofone\endcsname\relax
  \long\def\@firstofone#1{#1}%
\fi
\expandafter\ifx\csname loop\endcsname\relax
  \expandafter\@firstofone
\else
  \expandafter\@gobble
\fi
{%
  \def\loop#1\repeat{%
    \def\body{#1}%
    \iterate
  }%
  \def\iterate{%
    \body
      \let\next\iterate
    \else
      \let\next\relax
    \fi
    \next
  }%
  \let\repeat=\fi
}%
\def\RestoreCatcodes{}
\count@=0 %
\loop
  \edef\RestoreCatcodes{%
    \RestoreCatcodes
    \catcode\the\count@=\the\catcode\count@\relax
  }%
\ifnum\count@<255 %
  \advance\count@ 1 %
\repeat

\def\RangeCatcodeInvalid#1#2{%
  \count@=#1\relax
  \loop
    \catcode\count@=15 %
  \ifnum\count@<#2\relax
    \advance\count@ 1 %
  \repeat
}
\def\RangeCatcodeCheck#1#2#3{%
  \count@=#1\relax
  \loop
    \ifnum#3=\catcode\count@
    \else
      \errmessage{%
        Character \the\count@\space
        with wrong catcode \the\catcode\count@\space
        instead of \number#3%
      }%
    \fi
  \ifnum\count@<#2\relax
    \advance\count@ 1 %
  \repeat
}
\def\space{ }
\expandafter\ifx\csname LoadCommand\endcsname\relax
  \def\LoadCommand{\input gettitlestring.sty\relax}%
\fi
\def\Test{%
  \RangeCatcodeInvalid{0}{47}%
  \RangeCatcodeInvalid{58}{64}%
  \RangeCatcodeInvalid{91}{96}%
  \RangeCatcodeInvalid{123}{255}%
  \catcode`\@=12 %
  \catcode`\\=0 %
  \catcode`\%=14 %
  \LoadCommand
  \RangeCatcodeCheck{0}{36}{15}%
  \RangeCatcodeCheck{37}{37}{14}%
  \RangeCatcodeCheck{38}{47}{15}%
  \RangeCatcodeCheck{48}{57}{12}%
  \RangeCatcodeCheck{58}{63}{15}%
  \RangeCatcodeCheck{64}{64}{12}%
  \RangeCatcodeCheck{65}{90}{11}%
  \RangeCatcodeCheck{91}{91}{15}%
  \RangeCatcodeCheck{92}{92}{0}%
  \RangeCatcodeCheck{93}{96}{15}%
  \RangeCatcodeCheck{97}{122}{11}%
  \RangeCatcodeCheck{123}{255}{15}%
  \RestoreCatcodes
}
\Test
\csname @@end\endcsname
\end
%    \end{macrocode}
%    \begin{macrocode}
%</test1>
%    \end{macrocode}
%
% \subsection{Test of non-expand method}
%
%    \begin{macrocode}
%<*test2>
\NeedsTeXFormat{LaTeX2e}
\documentclass{minimal}
\usepackage{gettitlestring}[2016/05/16]
\usepackage{qstest}
\IncludeTests{*}
\LogTests{log}{*}{*}
\begin{document}
\begin{qstest}{non-expand}{non-expand}
  \def\test#1#2{%
    \sbox0{%
      \GetTitleString{#1}%
      \Expect{#2}*{\GetTitleStringResult}%
    }%
    \Expect{0.0pt}*{\the\wd0}%
  }%
  \test{}{}%
  \test{ }{}%
  \test{ x }{x}%
  \test{ x y }{x y}%
  \test{ \relax}{\relax}%
  \test{\label{f}a}{a}%
  \test{ \label{f}a}{a}%
  \test{\label{f} a}{a}%
  \test{ \label{f} a}{a}%
  \test{a\label{f}}{a}%
  \test{a\label{f} }{a}%
  \test{a \label{f}}{a}%
  \test{a \label{f} }{a}%
  \test{a\label{f}b\label{g}}{ab}%
  \test{a \label{f}b \label{g} }{a b}%
  \test{a\label{f} b \label{g} }{a b}%
\end{qstest}
\end{document}
%</test2>
%    \end{macrocode}
%
% \section{Installation}
%
% \subsection{Download}
%
% \paragraph{Package.} This package is available on
% CTAN\footnote{\CTANpkg{gettitlestring}}:
% \begin{description}
% \item[\CTAN{macros/latex/contrib/oberdiek/gettitlestring.dtx}] The source file.
% \item[\CTAN{macros/latex/contrib/oberdiek/gettitlestring.pdf}] Documentation.
% \end{description}
%
%
% \paragraph{Bundle.} All the packages of the bundle `oberdiek'
% are also available in a TDS compliant ZIP archive. There
% the packages are already unpacked and the documentation files
% are generated. The files and directories obey the TDS standard.
% \begin{description}
% \item[\CTANinstall{install/macros/latex/contrib/oberdiek.tds.zip}]
% \end{description}
% \emph{TDS} refers to the standard ``A Directory Structure
% for \TeX\ Files'' (\CTAN{tds/tds.pdf}). Directories
% with \xfile{texmf} in their name are usually organized this way.
%
% \subsection{Bundle installation}
%
% \paragraph{Unpacking.} Unpack the \xfile{oberdiek.tds.zip} in the
% TDS tree (also known as \xfile{texmf} tree) of your choice.
% Example (linux):
% \begin{quote}
%   |unzip oberdiek.tds.zip -d ~/texmf|
% \end{quote}
%
% \paragraph{Script installation.}
% Check the directory \xfile{TDS:scripts/oberdiek/} for
% scripts that need further installation steps.
% Package \xpackage{attachfile2} comes with the Perl script
% \xfile{pdfatfi.pl} that should be installed in such a way
% that it can be called as \texttt{pdfatfi}.
% Example (linux):
% \begin{quote}
%   |chmod +x scripts/oberdiek/pdfatfi.pl|\\
%   |cp scripts/oberdiek/pdfatfi.pl /usr/local/bin/|
% \end{quote}
%
% \subsection{Package installation}
%
% \paragraph{Unpacking.} The \xfile{.dtx} file is a self-extracting
% \docstrip\ archive. The files are extracted by running the
% \xfile{.dtx} through \plainTeX:
% \begin{quote}
%   \verb|tex gettitlestring.dtx|
% \end{quote}
%
% \paragraph{TDS.} Now the different files must be moved into
% the different directories in your installation TDS tree
% (also known as \xfile{texmf} tree):
% \begin{quote}
% \def\t{^^A
% \begin{tabular}{@{}>{\ttfamily}l@{ $\rightarrow$ }>{\ttfamily}l@{}}
%   gettitlestring.sty & tex/generic/oberdiek/gettitlestring.sty\\
%   gettitlestring.pdf & doc/latex/oberdiek/gettitlestring.pdf\\
%   test/gettitlestring-test1.tex & doc/latex/oberdiek/test/gettitlestring-test1.tex\\
%   test/gettitlestring-test2.tex & doc/latex/oberdiek/test/gettitlestring-test2.tex\\
%   gettitlestring.dtx & source/latex/oberdiek/gettitlestring.dtx\\
% \end{tabular}^^A
% }^^A
% \sbox0{\t}^^A
% \ifdim\wd0>\linewidth
%   \begingroup
%     \advance\linewidth by\leftmargin
%     \advance\linewidth by\rightmargin
%   \edef\x{\endgroup
%     \def\noexpand\lw{\the\linewidth}^^A
%   }\x
%   \def\lwbox{^^A
%     \leavevmode
%     \hbox to \linewidth{^^A
%       \kern-\leftmargin\relax
%       \hss
%       \usebox0
%       \hss
%       \kern-\rightmargin\relax
%     }^^A
%   }^^A
%   \ifdim\wd0>\lw
%     \sbox0{\small\t}^^A
%     \ifdim\wd0>\linewidth
%       \ifdim\wd0>\lw
%         \sbox0{\footnotesize\t}^^A
%         \ifdim\wd0>\linewidth
%           \ifdim\wd0>\lw
%             \sbox0{\scriptsize\t}^^A
%             \ifdim\wd0>\linewidth
%               \ifdim\wd0>\lw
%                 \sbox0{\tiny\t}^^A
%                 \ifdim\wd0>\linewidth
%                   \lwbox
%                 \else
%                   \usebox0
%                 \fi
%               \else
%                 \lwbox
%               \fi
%             \else
%               \usebox0
%             \fi
%           \else
%             \lwbox
%           \fi
%         \else
%           \usebox0
%         \fi
%       \else
%         \lwbox
%       \fi
%     \else
%       \usebox0
%     \fi
%   \else
%     \lwbox
%   \fi
% \else
%   \usebox0
% \fi
% \end{quote}
% If you have a \xfile{docstrip.cfg} that configures and enables \docstrip's
% TDS installing feature, then some files can already be in the right
% place, see the documentation of \docstrip.
%
% \subsection{Refresh file name databases}
%
% If your \TeX~distribution
% (\teTeX, \mikTeX, \dots) relies on file name databases, you must refresh
% these. For example, \teTeX\ users run \verb|texhash| or
% \verb|mktexlsr|.
%
% \subsection{Some details for the interested}
%
% \paragraph{Attached source.}
%
% The PDF documentation on CTAN also includes the
% \xfile{.dtx} source file. It can be extracted by
% AcrobatReader 6 or higher. Another option is \textsf{pdftk},
% e.g. unpack the file into the current directory:
% \begin{quote}
%   \verb|pdftk gettitlestring.pdf unpack_files output .|
% \end{quote}
%
% \paragraph{Unpacking with \LaTeX.}
% The \xfile{.dtx} chooses its action depending on the format:
% \begin{description}
% \item[\plainTeX:] Run \docstrip\ and extract the files.
% \item[\LaTeX:] Generate the documentation.
% \end{description}
% If you insist on using \LaTeX\ for \docstrip\ (really,
% \docstrip\ does not need \LaTeX), then inform the autodetect routine
% about your intention:
% \begin{quote}
%   \verb|latex \let\install=y\input{gettitlestring.dtx}|
% \end{quote}
% Do not forget to quote the argument according to the demands
% of your shell.
%
% \paragraph{Generating the documentation.}
% You can use both the \xfile{.dtx} or the \xfile{.drv} to generate
% the documentation. The process can be configured by the
% configuration file \xfile{ltxdoc.cfg}. For instance, put this
% line into this file, if you want to have A4 as paper format:
% \begin{quote}
%   \verb|\PassOptionsToClass{a4paper}{article}|
% \end{quote}
% An example follows how to generate the
% documentation with pdf\LaTeX:
% \begin{quote}
%\begin{verbatim}
%pdflatex gettitlestring.dtx
%makeindex -s gind.ist gettitlestring.idx
%pdflatex gettitlestring.dtx
%makeindex -s gind.ist gettitlestring.idx
%pdflatex gettitlestring.dtx
%\end{verbatim}
% \end{quote}
%
% \begin{thebibliography}{9}
%
% \bibitem{memoir}
% Peter Wilson, Lars Madsen:
% \textit{The Memoir Class};
% 2009/11/17 v1.61803398c;
% \CTANpkg{memoir}
%
% \bibitem{titleref}
% Donald Arsenau:
% \textit{Titleref.sty};
% 2001/04/05 ver 3.1;
% \CTAN{macros/latex/contrib/misc/titleref.sty}
%
% \bibitem{zref}
% Heiko Oberdiek:
% \textit{The \xpackage{zref} package};
% 2009/12/08 v2.7;
% \CTAN{macros/latex/contrib/oberdiek/zref.pdf}
%
% \end{thebibliography}
%
% \begin{History}
%   \begin{Version}{2009/12/08 v1.0}
%   \item
%     The first version.
%   \end{Version}
%   \begin{Version}{2009/12/12 v1.1}
%   \item
%     Short info shortened.
%   \end{Version}
%   \begin{Version}{2009/12/13 v1.2}
%   \item
%     Forgotten third argument for \cs{InputIfFileExists} added.
%   \end{Version}
%   \begin{Version}{2009/12/18 v1.3}
%   \item
%     \cs{Hy@SectionAnchorHref} added for filtering
%     (hyperref 2009/12/18 v6.79w).
%   \end{Version}
%   \begin{Version}{2010/12/03 v1.4}
%   \item
%     Support of package \xpackage{enumitem}: removing
%     \cs{enit@format} from title string (problem report by GL).
%   \end{Version}
%   \begin{Version}{2016/05/16 v1.5}
%   \item
%     Documentation updates.
%   \end{Version}
% \end{History}
%
% \PrintIndex
%
% \Finale
\endinput
|
% \end{quote}
% Do not forget to quote the argument according to the demands
% of your shell.
%
% \paragraph{Generating the documentation.}
% You can use both the \xfile{.dtx} or the \xfile{.drv} to generate
% the documentation. The process can be configured by the
% configuration file \xfile{ltxdoc.cfg}. For instance, put this
% line into this file, if you want to have A4 as paper format:
% \begin{quote}
%   \verb|\PassOptionsToClass{a4paper}{article}|
% \end{quote}
% An example follows how to generate the
% documentation with pdf\LaTeX:
% \begin{quote}
%\begin{verbatim}
%pdflatex gettitlestring.dtx
%makeindex -s gind.ist gettitlestring.idx
%pdflatex gettitlestring.dtx
%makeindex -s gind.ist gettitlestring.idx
%pdflatex gettitlestring.dtx
%\end{verbatim}
% \end{quote}
%
% \begin{thebibliography}{9}
%
% \bibitem{memoir}
% Peter Wilson, Lars Madsen:
% \textit{The Memoir Class};
% 2009/11/17 v1.61803398c;
% \CTANpkg{memoir}
%
% \bibitem{titleref}
% Donald Arsenau:
% \textit{Titleref.sty};
% 2001/04/05 ver 3.1;
% \CTAN{macros/latex/contrib/misc/titleref.sty}
%
% \bibitem{zref}
% Heiko Oberdiek:
% \textit{The \xpackage{zref} package};
% 2009/12/08 v2.7;
% \CTAN{macros/latex/contrib/oberdiek/zref.pdf}
%
% \end{thebibliography}
%
% \begin{History}
%   \begin{Version}{2009/12/08 v1.0}
%   \item
%     The first version.
%   \end{Version}
%   \begin{Version}{2009/12/12 v1.1}
%   \item
%     Short info shortened.
%   \end{Version}
%   \begin{Version}{2009/12/13 v1.2}
%   \item
%     Forgotten third argument for \cs{InputIfFileExists} added.
%   \end{Version}
%   \begin{Version}{2009/12/18 v1.3}
%   \item
%     \cs{Hy@SectionAnchorHref} added for filtering
%     (hyperref 2009/12/18 v6.79w).
%   \end{Version}
%   \begin{Version}{2010/12/03 v1.4}
%   \item
%     Support of package \xpackage{enumitem}: removing
%     \cs{enit@format} from title string (problem report by GL).
%   \end{Version}
%   \begin{Version}{2016/05/16 v1.5}
%   \item
%     Documentation updates.
%   \end{Version}
% \end{History}
%
% \PrintIndex
%
% \Finale
\endinput
|
% \end{quote}
% Do not forget to quote the argument according to the demands
% of your shell.
%
% \paragraph{Generating the documentation.}
% You can use both the \xfile{.dtx} or the \xfile{.drv} to generate
% the documentation. The process can be configured by the
% configuration file \xfile{ltxdoc.cfg}. For instance, put this
% line into this file, if you want to have A4 as paper format:
% \begin{quote}
%   \verb|\PassOptionsToClass{a4paper}{article}|
% \end{quote}
% An example follows how to generate the
% documentation with pdf\LaTeX:
% \begin{quote}
%\begin{verbatim}
%pdflatex gettitlestring.dtx
%makeindex -s gind.ist gettitlestring.idx
%pdflatex gettitlestring.dtx
%makeindex -s gind.ist gettitlestring.idx
%pdflatex gettitlestring.dtx
%\end{verbatim}
% \end{quote}
%
% \begin{thebibliography}{9}
%
% \bibitem{memoir}
% Peter Wilson, Lars Madsen:
% \textit{The Memoir Class};
% 2009/11/17 v1.61803398c;
% \CTANpkg{memoir}
%
% \bibitem{titleref}
% Donald Arsenau:
% \textit{Titleref.sty};
% 2001/04/05 ver 3.1;
% \CTAN{macros/latex/contrib/misc/titleref.sty}
%
% \bibitem{zref}
% Heiko Oberdiek:
% \textit{The \xpackage{zref} package};
% 2009/12/08 v2.7;
% \CTAN{macros/latex/contrib/oberdiek/zref.pdf}
%
% \end{thebibliography}
%
% \begin{History}
%   \begin{Version}{2009/12/08 v1.0}
%   \item
%     The first version.
%   \end{Version}
%   \begin{Version}{2009/12/12 v1.1}
%   \item
%     Short info shortened.
%   \end{Version}
%   \begin{Version}{2009/12/13 v1.2}
%   \item
%     Forgotten third argument for \cs{InputIfFileExists} added.
%   \end{Version}
%   \begin{Version}{2009/12/18 v1.3}
%   \item
%     \cs{Hy@SectionAnchorHref} added for filtering
%     (hyperref 2009/12/18 v6.79w).
%   \end{Version}
%   \begin{Version}{2010/12/03 v1.4}
%   \item
%     Support of package \xpackage{enumitem}: removing
%     \cs{enit@format} from title string (problem report by GL).
%   \end{Version}
%   \begin{Version}{2016/05/16 v1.5}
%   \item
%     Documentation updates.
%   \end{Version}
% \end{History}
%
% \PrintIndex
%
% \Finale
\endinput
|
% \end{quote}
% Do not forget to quote the argument according to the demands
% of your shell.
%
% \paragraph{Generating the documentation.}
% You can use both the \xfile{.dtx} or the \xfile{.drv} to generate
% the documentation. The process can be configured by the
% configuration file \xfile{ltxdoc.cfg}. For instance, put this
% line into this file, if you want to have A4 as paper format:
% \begin{quote}
%   \verb|\PassOptionsToClass{a4paper}{article}|
% \end{quote}
% An example follows how to generate the
% documentation with pdf\LaTeX:
% \begin{quote}
%\begin{verbatim}
%pdflatex gettitlestring.dtx
%makeindex -s gind.ist gettitlestring.idx
%pdflatex gettitlestring.dtx
%makeindex -s gind.ist gettitlestring.idx
%pdflatex gettitlestring.dtx
%\end{verbatim}
% \end{quote}
%
% \begin{thebibliography}{9}
%
% \bibitem{memoir}
% Peter Wilson, Lars Madsen:
% \textit{The Memoir Class};
% 2009/11/17 v1.61803398c;
% \CTANpkg{memoir}
%
% \bibitem{titleref}
% Donald Arsenau:
% \textit{Titleref.sty};
% 2001/04/05 ver 3.1;
% \CTANpkg{titleref}
%
% \bibitem{zref}
% Heiko Oberdiek:
% \textit{The \xpackage{zref} package};
% 2009/12/08 v2.7;
% \CTANpkg{zref}
%
% \end{thebibliography}
%
% \begin{History}
%   \begin{Version}{2009/12/08 v1.0}
%   \item
%     The first version.
%   \end{Version}
%   \begin{Version}{2009/12/12 v1.1}
%   \item
%     Short info shortened.
%   \end{Version}
%   \begin{Version}{2009/12/13 v1.2}
%   \item
%     Forgotten third argument for \cs{InputIfFileExists} added.
%   \end{Version}
%   \begin{Version}{2009/12/18 v1.3}
%   \item
%     \cs{Hy@SectionAnchorHref} added for filtering
%     (hyperref 2009/12/18 v6.79w).
%   \end{Version}
%   \begin{Version}{2010/12/03 v1.4}
%   \item
%     Support of package \xpackage{enumitem}: removing
%     \cs{enit@format} from title string (problem report by GL).
%   \end{Version}
%   \begin{Version}{2016/05/16 v1.5}
%   \item
%     Documentation updates.
%   \end{Version}
% \end{History}
%
% \PrintIndex
%
% \Finale
\endinput
