% \iffalse meta-comment
%
% File: setouterhbox.dtx
% Version: 2016/05/16 v1.8
% Info: Set hbox in outer horizontal mode
%
% Copyright (C)
%    2005-2007 Heiko Oberdiek
%    2016-2019 Oberdiek Package Support Group
%    https://github.com/ho-tex/oberdiek/issues
%
% This work may be distributed and/or modified under the
% conditions of the LaTeX Project Public License, either
% version 1.3c of this license or (at your option) any later
% version. This version of this license is in
%    https://www.latex-project.org/lppl/lppl-1-3c.txt
% and the latest version of this license is in
%    https://www.latex-project.org/lppl.txt
% and version 1.3 or later is part of all distributions of
% LaTeX version 2005/12/01 or later.
%
% This work has the LPPL maintenance status "maintained".
%
% The Current Maintainers of this work are
% Heiko Oberdiek and the Oberdiek Package Support Group
% https://github.com/ho-tex/oberdiek/issues
%
% The Base Interpreter refers to any `TeX-Format',
% because some files are installed in TDS:tex/generic//.
%
% This work consists of the main source file setouterhbox.dtx
% and the derived files
%    setouterhbox.sty, setouterhbox.pdf, setouterhbox.ins, setouterhbox.drv,
%    setouterhbox-example.tex, setouterhbox-test1.tex,
%    setouterhbox-test2.tex.
%
% Distribution:
%    CTAN:macros/latex/contrib/oberdiek/setouterhbox.dtx
%    CTAN:macros/latex/contrib/oberdiek/setouterhbox.pdf
%
% Unpacking:
%    (a) If setouterhbox.ins is present:
%           tex setouterhbox.ins
%    (b) Without setouterhbox.ins:
%           tex setouterhbox.dtx
%    (c) If you insist on using LaTeX
%           latex \let\install=y% \iffalse meta-comment
%
% File: setouterhbox.dtx
% Version: 2016/05/16 v1.8
% Info: Set hbox in outer horizontal mode
%
% Copyright (C)
%    2005-2007 Heiko Oberdiek
%    2016-2019 Oberdiek Package Support Group
%    https://github.com/ho-tex/oberdiek/issues
%
% This work may be distributed and/or modified under the
% conditions of the LaTeX Project Public License, either
% version 1.3c of this license or (at your option) any later
% version. This version of this license is in
%    https://www.latex-project.org/lppl/lppl-1-3c.txt
% and the latest version of this license is in
%    https://www.latex-project.org/lppl.txt
% and version 1.3 or later is part of all distributions of
% LaTeX version 2005/12/01 or later.
%
% This work has the LPPL maintenance status "maintained".
%
% The Current Maintainers of this work are
% Heiko Oberdiek and the Oberdiek Package Support Group
% https://github.com/ho-tex/oberdiek/issues
%
% The Base Interpreter refers to any `TeX-Format',
% because some files are installed in TDS:tex/generic//.
%
% This work consists of the main source file setouterhbox.dtx
% and the derived files
%    setouterhbox.sty, setouterhbox.pdf, setouterhbox.ins, setouterhbox.drv,
%    setouterhbox-example.tex, setouterhbox-test1.tex,
%    setouterhbox-test2.tex.
%
% Distribution:
%    CTAN:macros/latex/contrib/oberdiek/setouterhbox.dtx
%    CTAN:macros/latex/contrib/oberdiek/setouterhbox.pdf
%
% Unpacking:
%    (a) If setouterhbox.ins is present:
%           tex setouterhbox.ins
%    (b) Without setouterhbox.ins:
%           tex setouterhbox.dtx
%    (c) If you insist on using LaTeX
%           latex \let\install=y% \iffalse meta-comment
%
% File: setouterhbox.dtx
% Version: 2016/05/16 v1.8
% Info: Set hbox in outer horizontal mode
%
% Copyright (C)
%    2005-2007 Heiko Oberdiek
%    2016-2019 Oberdiek Package Support Group
%    https://github.com/ho-tex/oberdiek/issues
%
% This work may be distributed and/or modified under the
% conditions of the LaTeX Project Public License, either
% version 1.3c of this license or (at your option) any later
% version. This version of this license is in
%    https://www.latex-project.org/lppl/lppl-1-3c.txt
% and the latest version of this license is in
%    https://www.latex-project.org/lppl.txt
% and version 1.3 or later is part of all distributions of
% LaTeX version 2005/12/01 or later.
%
% This work has the LPPL maintenance status "maintained".
%
% The Current Maintainers of this work are
% Heiko Oberdiek and the Oberdiek Package Support Group
% https://github.com/ho-tex/oberdiek/issues
%
% The Base Interpreter refers to any `TeX-Format',
% because some files are installed in TDS:tex/generic//.
%
% This work consists of the main source file setouterhbox.dtx
% and the derived files
%    setouterhbox.sty, setouterhbox.pdf, setouterhbox.ins, setouterhbox.drv,
%    setouterhbox-example.tex, setouterhbox-test1.tex,
%    setouterhbox-test2.tex.
%
% Distribution:
%    CTAN:macros/latex/contrib/oberdiek/setouterhbox.dtx
%    CTAN:macros/latex/contrib/oberdiek/setouterhbox.pdf
%
% Unpacking:
%    (a) If setouterhbox.ins is present:
%           tex setouterhbox.ins
%    (b) Without setouterhbox.ins:
%           tex setouterhbox.dtx
%    (c) If you insist on using LaTeX
%           latex \let\install=y% \iffalse meta-comment
%
% File: setouterhbox.dtx
% Version: 2016/05/16 v1.8
% Info: Set hbox in outer horizontal mode
%
% Copyright (C)
%    2005-2007 Heiko Oberdiek
%    2016-2019 Oberdiek Package Support Group
%    https://github.com/ho-tex/oberdiek/issues
%
% This work may be distributed and/or modified under the
% conditions of the LaTeX Project Public License, either
% version 1.3c of this license or (at your option) any later
% version. This version of this license is in
%    https://www.latex-project.org/lppl/lppl-1-3c.txt
% and the latest version of this license is in
%    https://www.latex-project.org/lppl.txt
% and version 1.3 or later is part of all distributions of
% LaTeX version 2005/12/01 or later.
%
% This work has the LPPL maintenance status "maintained".
%
% The Current Maintainers of this work are
% Heiko Oberdiek and the Oberdiek Package Support Group
% https://github.com/ho-tex/oberdiek/issues
%
% The Base Interpreter refers to any `TeX-Format',
% because some files are installed in TDS:tex/generic//.
%
% This work consists of the main source file setouterhbox.dtx
% and the derived files
%    setouterhbox.sty, setouterhbox.pdf, setouterhbox.ins, setouterhbox.drv,
%    setouterhbox-example.tex, setouterhbox-test1.tex,
%    setouterhbox-test2.tex.
%
% Distribution:
%    CTAN:macros/latex/contrib/oberdiek/setouterhbox.dtx
%    CTAN:macros/latex/contrib/oberdiek/setouterhbox.pdf
%
% Unpacking:
%    (a) If setouterhbox.ins is present:
%           tex setouterhbox.ins
%    (b) Without setouterhbox.ins:
%           tex setouterhbox.dtx
%    (c) If you insist on using LaTeX
%           latex \let\install=y\input{setouterhbox.dtx}
%        (quote the arguments according to the demands of your shell)
%
% Documentation:
%    (a) If setouterhbox.drv is present:
%           latex setouterhbox.drv
%    (b) Without setouterhbox.drv:
%           latex setouterhbox.dtx; ...
%    The class ltxdoc loads the configuration file ltxdoc.cfg
%    if available. Here you can specify further options, e.g.
%    use A4 as paper format:
%       \PassOptionsToClass{a4paper}{article}
%
%    Programm calls to get the documentation (example):
%       pdflatex setouterhbox.dtx
%       makeindex -s gind.ist setouterhbox.idx
%       pdflatex setouterhbox.dtx
%       makeindex -s gind.ist setouterhbox.idx
%       pdflatex setouterhbox.dtx
%
% Installation:
%    TDS:tex/generic/oberdiek/setouterhbox.sty
%    TDS:doc/latex/oberdiek/setouterhbox.pdf
%    TDS:doc/latex/oberdiek/setouterhbox-example.tex
%    TDS:doc/latex/oberdiek/test/setouterhbox-test1.tex
%    TDS:doc/latex/oberdiek/test/setouterhbox-test2.tex
%    TDS:source/latex/oberdiek/setouterhbox.dtx
%
%<*ignore>
\begingroup
  \catcode123=1 %
  \catcode125=2 %
  \def\x{LaTeX2e}%
\expandafter\endgroup
\ifcase 0\ifx\install y1\fi\expandafter
         \ifx\csname processbatchFile\endcsname\relax\else1\fi
         \ifx\fmtname\x\else 1\fi\relax
\else\csname fi\endcsname
%</ignore>
%<*install>
\input docstrip.tex
\Msg{************************************************************************}
\Msg{* Installation}
\Msg{* Package: setouterhbox 2016/05/16 v1.8 Set hbox in outer horizontal mode (HO)}
\Msg{************************************************************************}

\keepsilent
\askforoverwritefalse

\let\MetaPrefix\relax
\preamble

This is a generated file.

Project: setouterhbox
Version: 2016/05/16 v1.8

Copyright (C)
   2005-2007 Heiko Oberdiek
   2016-2019 Oberdiek Package Support Group

This work may be distributed and/or modified under the
conditions of the LaTeX Project Public License, either
version 1.3c of this license or (at your option) any later
version. This version of this license is in
   https://www.latex-project.org/lppl/lppl-1-3c.txt
and the latest version of this license is in
   https://www.latex-project.org/lppl.txt
and version 1.3 or later is part of all distributions of
LaTeX version 2005/12/01 or later.

This work has the LPPL maintenance status "maintained".

The Current Maintainers of this work are
Heiko Oberdiek and the Oberdiek Package Support Group
https://github.com/ho-tex/oberdiek/issues


The Base Interpreter refers to any `TeX-Format',
because some files are installed in TDS:tex/generic//.

This work consists of the main source file setouterhbox.dtx
and the derived files
   setouterhbox.sty, setouterhbox.pdf, setouterhbox.ins, setouterhbox.drv,
   setouterhbox-example.tex, setouterhbox-test1.tex,
   setouterhbox-test2.tex.

\endpreamble
\let\MetaPrefix\DoubleperCent

\generate{%
  \file{setouterhbox.ins}{\from{setouterhbox.dtx}{install}}%
  \file{setouterhbox.drv}{\from{setouterhbox.dtx}{driver}}%
  \usedir{tex/generic/oberdiek}%
  \file{setouterhbox.sty}{\from{setouterhbox.dtx}{package}}%
  \usedir{doc/latex/oberdiek}%
  \file{setouterhbox-example.tex}{\from{setouterhbox.dtx}{example}}%
%  \usedir{doc/latex/oberdiek/test}%
%  \file{setouterhbox-test1.tex}{\from{setouterhbox.dtx}{test1}}%
%  \file{setouterhbox-test2.tex}{\from{setouterhbox.dtx}{test2}}%
  \nopreamble
  \nopostamble
%  \usedir{source/latex/oberdiek/catalogue}%
%  \file{setouterhbox.xml}{\from{setouterhbox.dtx}{catalogue}}%
}

\catcode32=13\relax% active space
\let =\space%
\Msg{************************************************************************}
\Msg{*}
\Msg{* To finish the installation you have to move the following}
\Msg{* file into a directory searched by TeX:}
\Msg{*}
\Msg{*     setouterhbox.sty}
\Msg{*}
\Msg{* To produce the documentation run the file `setouterhbox.drv'}
\Msg{* through LaTeX.}
\Msg{*}
\Msg{* Happy TeXing!}
\Msg{*}
\Msg{************************************************************************}

\endbatchfile
%</install>
%<*ignore>
\fi
%</ignore>
%<*driver>
\NeedsTeXFormat{LaTeX2e}
\ProvidesFile{setouterhbox.drv}%
  [2016/05/16 v1.8 Set hbox in outer horizontal mode (HO)]%
\documentclass{ltxdoc}
\usepackage{holtxdoc}[2011/11/22]
\begin{document}
  \DocInput{setouterhbox.dtx}%
\end{document}
%</driver>
% \fi
%
%
% \CharacterTable
%  {Upper-case    \A\B\C\D\E\F\G\H\I\J\K\L\M\N\O\P\Q\R\S\T\U\V\W\X\Y\Z
%   Lower-case    \a\b\c\d\e\f\g\h\i\j\k\l\m\n\o\p\q\r\s\t\u\v\w\x\y\z
%   Digits        \0\1\2\3\4\5\6\7\8\9
%   Exclamation   \!     Double quote  \"     Hash (number) \#
%   Dollar        \$     Percent       \%     Ampersand     \&
%   Acute accent  \'     Left paren    \(     Right paren   \)
%   Asterisk      \*     Plus          \+     Comma         \,
%   Minus         \-     Point         \.     Solidus       \/
%   Colon         \:     Semicolon     \;     Less than     \<
%   Equals        \=     Greater than  \>     Question mark \?
%   Commercial at \@     Left bracket  \[     Backslash     \\
%   Right bracket \]     Circumflex    \^     Underscore    \_
%   Grave accent  \`     Left brace    \{     Vertical bar  \|
%   Right brace   \}     Tilde         \~}
%
% \GetFileInfo{setouterhbox.drv}
%
% \title{The \xpackage{setouterhbox} package}
% \date{2016/05/16 v1.8}
% \author{Heiko Oberdiek\thanks
% {Please report any issues at \url{https://github.com/ho-tex/oberdiek/issues}}}
%
% \maketitle
%
% \begin{abstract}
% If math stuff is set in an \cs{hbox}, then TeX
% performs some optimization and omits the implicite
% penalties \cs{binoppenalty} and \cs{relpenalty}.
% This packages tries to put stuff into an \cs{hbox}
% without getting lost of those penalties.
% \end{abstract}
%
% \tableofcontents
%
% \section{Documentation}
%
% \subsection{Introduction}
%
% There is a situation in \xpackage{hyperref}'s driver for dvips
% where the user wants to have links that can be broken across
% lines. However dvips doesn't support the feature. With option
% \xoption{breaklinks} \xpackage{hyperref} sets the links as
% usual, put them in a box and write the link data with
% box dimensions into the appropriate \cs{special}s.
% Then, however, it does not set the complete unbreakable
% box, but it unwrappes the material inside to allow line
% breaks. Of course line breaking and glue setting will falsify
% the link dimensions, but line breaking was more important
% for the user.
%
% \subsection{Acknowledgement}
%
% Jonathan Fine, Donald Arsenau and me discussed the problem
% in the newsgroup \xnewsgroup{comp.text.tex} where Damian
% Menscher has started the thread, see \cite{newsstart}.
%
% The discussion was productive and generated many ideas
% and code examples. In order to have a more permanent
% result I wrote this package and tried to implement
% most of the ideas, a kind of summary of the discussion.
% Thus I want and have to thank Jonathan Fine and Donald Arsenau
% very much.
%
% Two weeks later David Kastrup (posting in
% \xnewsgroup{comp.text.tex}, \cite{kastrup})
% remembered an old article of Michael Downes (\cite{downes})
% in TUGboat, where Michael Downes already presented the
% method we discuss here. Nowadays we have \eTeX\ that extends
% the tool set of a \TeX\ macro programmer. Especially useful
% \eTeX\ was in this package for detecting and dealing with
% errorneous situations.
%
% However also nowadays a perfect solution for the problem
% is still missing at macro level. Probably someone has
% to go deep in the internals of the \TeX\ compiler to
% implement a switch that let penalties stay where otherwise
% \TeX\ would remove them for optimization reasons.
%
% \subsection{Usage}
%
% \paragraph{Package loading.}
% \LaTeX: as usually:
% \begin{quote}
%   |\usepackage{setouterhbox}|
% \end{quote}
% The package can also be included directly, thus \plainTeX\ users
% write:
% \begin{quote}
%   |\input setouterhbox.sty|
% \end{quote}
%
% \paragraph{Register allocation.}
% The material will be put into a box, thus we need to know these
% box number. If you need to allocate a new box register:
% \begin{description}
%  \item[\LaTeX:] |\newsavebox{\|\meta{name}|}|
%  \item[\plainTeX:] |\newbox\|\meta{name}
% \end{description}
% Then |\|\meta{name} is a command that held the box number.
%
% \paragraph{Box wrapping.}
% \LaTeX\ users put the material in the box with an environment
% similar to \texttt{lrbox}. The environment \texttt{setouterhbox}
% uses the same syntax and offers the same features, such
% as verbatim stuff inside:
% \begin{quote}
%  |\begin{setouterhbox}{|\meta{box number}|}|\dots
%  |\end{setouterhbox}|
% \end{quote}
% Users with \plainTeX\ do not have environments, they use instead:
% \begin{quote}
%   |\setouterhbox{|\meta{box number}|}|\dots|\endsetouterhbox|
% \end{quote}
% In both cases the material is put into an \cs{hbox} and assigned
% to the given box, denoted by \meta{box number}. Note the
% assignment is local, the same way \texttt{lrbox} behaves.
%
% \paragraph{Unwrapping.}
% The box material is ready for unwrapping:
% \begin{quote}
%   |\unhbox|\meta{box number}
% \end{quote}
%
% \subsection{Option \xoption{hyperref}}
%
% Package url uses math mode for typesetting urls.
% Break points are inserted by \cs{binoppenalty} and
% \cs{relpenalty}. Unhappily these break points are
% removed, if \xpackage{hyperref}
% is used with option {breaklinks}
% and drivers that depend on \xoption{pdfmark}:
% \xoption{dvips}, \xoption{vtexpdfmark}, \xoption{textures},
% and \xoption{dvipsone}.
% Thus the option \xoption{hyperref} enables the method
% of this package to avoid the removal of \cs{relpenalty}
% and \cs{binoppenalty}. Thus you get more break points.
% However, the link areas are still wrong for these
% drivers, because they are not supporting broken
% links.
%
% Note, you need version 2006/08/16 v6.75c of package \xpackage{hyperref},
% because starting with this version the necessary hook is provided
% that package \xpackage{setouterhbox} uses.
% \begin{quote}
%   |\usepackage[|\dots|]{hyperref}[2006/08/16]|\\
%   |\usepackage[hyperref]{setouterhbox}|
% \end{quote}
% Package order does not matter.
%
% \subsection{Example}
%
%    \begin{macrocode}
%<*example>
\documentclass[a5paper]{article}
\usepackage{url}[2005/06/27]
\usepackage{setouterhbox}

\newsavebox{\testbox}

\setlength{\parindent}{0pt}
\setlength{\parskip}{2em}

\begin{document}
\raggedright

\url{http://this.is.a.very.long.host.name/followed/%
by/a/very_long_long_long_path.html}%

\sbox\testbox{%
  \url{http://this.is.a.very.long.host.name/followed/%
  by/a/very_long_long_long_path.html}%
}%
\unhbox\testbox

\begin{setouterhbox}{\testbox}%
  \url{http://this.is.a.very.long.host.name/followed/%
  by/a/very_long_long_long_path.html}%
\end{setouterhbox}
\unhbox\testbox

\end{document}
%</example>
%    \end{macrocode}
%
% \StopEventually{
% }
%
% \section{Implementation}
%
% Internal macros are prefixed by \cs{setouterhbox}, |@| is
% not used inside names, thus we do not need to care of its
% catcode if we are not using it as \LaTeX\ package.
%
% \subsection{Package start stuff}
%
%    \begin{macrocode}
%<*package>
%    \end{macrocode}
%
% Prevent reloading more than one, necessary for \plainTeX:
%    Reload check, especially if the package is not used with \LaTeX.
%    \begin{macrocode}
\begingroup\catcode61\catcode48\catcode32=10\relax%
  \catcode13=5 % ^^M
  \endlinechar=13 %
  \catcode35=6 % #
  \catcode39=12 % '
  \catcode44=12 % ,
  \catcode45=12 % -
  \catcode46=12 % .
  \catcode58=12 % :
  \catcode64=11 % @
  \catcode123=1 % {
  \catcode125=2 % }
  \expandafter\let\expandafter\x\csname ver@setouterhbox.sty\endcsname
  \ifx\x\relax % plain-TeX, first loading
  \else
    \def\empty{}%
    \ifx\x\empty % LaTeX, first loading,
      % variable is initialized, but \ProvidesPackage not yet seen
    \else
      \expandafter\ifx\csname PackageInfo\endcsname\relax
        \def\x#1#2{%
          \immediate\write-1{Package #1 Info: #2.}%
        }%
      \else
        \def\x#1#2{\PackageInfo{#1}{#2, stopped}}%
      \fi
      \x{setouterhbox}{The package is already loaded}%
      \aftergroup\endinput
    \fi
  \fi
\endgroup%
%    \end{macrocode}
%    Package identification:
%    \begin{macrocode}
\begingroup\catcode61\catcode48\catcode32=10\relax%
  \catcode13=5 % ^^M
  \endlinechar=13 %
  \catcode35=6 % #
  \catcode39=12 % '
  \catcode40=12 % (
  \catcode41=12 % )
  \catcode44=12 % ,
  \catcode45=12 % -
  \catcode46=12 % .
  \catcode47=12 % /
  \catcode58=12 % :
  \catcode64=11 % @
  \catcode91=12 % [
  \catcode93=12 % ]
  \catcode123=1 % {
  \catcode125=2 % }
  \expandafter\ifx\csname ProvidesPackage\endcsname\relax
    \def\x#1#2#3[#4]{\endgroup
      \immediate\write-1{Package: #3 #4}%
      \xdef#1{#4}%
    }%
  \else
    \def\x#1#2[#3]{\endgroup
      #2[{#3}]%
      \ifx#1\@undefined
        \xdef#1{#3}%
      \fi
      \ifx#1\relax
        \xdef#1{#3}%
      \fi
    }%
  \fi
\expandafter\x\csname ver@setouterhbox.sty\endcsname
\ProvidesPackage{setouterhbox}%
  [2016/05/16 v1.8 Set hbox in outer horizontal mode (HO)]%
%    \end{macrocode}
%
%    \begin{macrocode}
\begingroup\catcode61\catcode48\catcode32=10\relax%
  \catcode13=5 % ^^M
  \endlinechar=13 %
  \catcode123=1 % {
  \catcode125=2 % }
  \catcode64=11 % @
  \def\x{\endgroup
    \expandafter\edef\csname setouterhboxAtEnd\endcsname{%
      \endlinechar=\the\endlinechar\relax
      \catcode13=\the\catcode13\relax
      \catcode32=\the\catcode32\relax
      \catcode35=\the\catcode35\relax
      \catcode61=\the\catcode61\relax
      \catcode64=\the\catcode64\relax
      \catcode123=\the\catcode123\relax
      \catcode125=\the\catcode125\relax
    }%
  }%
\x\catcode61\catcode48\catcode32=10\relax%
\catcode13=5 % ^^M
\endlinechar=13 %
\catcode35=6 % #
\catcode64=11 % @
\catcode123=1 % {
\catcode125=2 % }
\def\TMP@EnsureCode#1#2{%
  \edef\setouterhboxAtEnd{%
    \setouterhboxAtEnd
    \catcode#1=\the\catcode#1\relax
  }%
  \catcode#1=#2\relax
}
\TMP@EnsureCode{40}{12}% (
\TMP@EnsureCode{41}{12}% )
\TMP@EnsureCode{44}{12}% ,
\TMP@EnsureCode{45}{12}% -
\TMP@EnsureCode{46}{12}% .
\TMP@EnsureCode{47}{12}% /
\TMP@EnsureCode{58}{12}% :
\TMP@EnsureCode{60}{12}% <
\TMP@EnsureCode{62}{12}% >
\TMP@EnsureCode{91}{12}% [
\TMP@EnsureCode{93}{12}% ]
\TMP@EnsureCode{96}{12}% `
\edef\setouterhboxAtEnd{\setouterhboxAtEnd\noexpand\endinput}
%    \end{macrocode}
%
% \subsection{Interface macros}
%
%    \begin{macro}{\setouterhboxBox}
% The method requires a global box assignment. To be on the
% safe side, a new box register is allocated for this
% global box assignment.
%    \begin{macrocode}
\newbox\setouterhboxBox
%    \end{macrocode}
%    \end{macro}
%
%    \begin{macro}{\setouterhboxFailure}
% Error message for both \plainTeX\ and \LaTeX
%    \begin{macrocode}
\begingroup\expandafter\expandafter\expandafter\endgroup
\expandafter\ifx\csname RequirePackage\endcsname\relax
  \input infwarerr.sty\relax
\else
  \RequirePackage{infwarerr}[2016/05/16]%
\fi
\edef\setouterhboxFailure#1#2{%
  \expandafter\noexpand\csname @PackageError\endcsname
      {setouterhbox}{#1}{#2}%
}
%    \end{macrocode}
%    \end{macro}
%
% \subsection{Main part}
%
% eTeX provides much better means for checking
% error conditions. Thus lines marked by "E" are executed
% if eTeX is available, otherwise the lines marked by "T" are
% used.
%    \begin{macrocode}
\begingroup\expandafter\expandafter\expandafter\endgroup
\expandafter\ifx\csname lastnodetype\endcsname\relax
  \catcode`T=9 % ignore
  \catcode`E=14 % comment
\else
  \catcode`T=14 % comment
  \catcode`E=9 % ignore
\fi
%    \end{macrocode}
%
%    \begin{macro}{\setouterhboxRemove}
% Remove all kern, glue, and penalty nodes;
% poor man's version, if \eTeX\ is not available
%    \begin{macrocode}
\def\setouterhboxRemove{%
E \ifnum\lastnodetype<11 %
E   \else
E   \ifnum\lastnodetype>13 %
E   \else
      \unskip\unkern\unpenalty
E     \expandafter\expandafter\expandafter\setouterhboxRemove
E   \fi
E \fi
}%
%    \end{macrocode}
%    \end{macro}
%
%    \begin{macro}{\setouterhbox}
% Passing the box contents by macro parameter would prevent
% catcode changes in the box contents like by \cs{verb}.
% Also \cs{bgroup} and \cs{egroup} does not work, because stuff
% has to be added at the begin and end of the box, thus
% the syntax
% |\setouterhbox{|\meta{box number}|}|\dots|\endsetouterhbox|
% is used. Also we automatically get an environment \texttt{setouterhbox}
% if \LaTeX\ is used.
%    \begin{macrocode}
\def\setouterhbox#1{%
  \begingroup
    \def\setouterhboxNum{#1}%
    \setbox0\vbox\bgroup
T     \kern.123pt\relax % marker
T     \kern0pt\relax % removed by \setouterhboxRemove
      \begingroup
        \everypar{}%
        \noindent
}
%    \end{macrocode}
%    \end{macro}
%    \begin{macro}{\endsetouterhbox}
% Most of the work is done in the end part, thus the heart of
% the method follows:
%    \begin{macrocode}
\def\endsetouterhbox{%
      \endgroup
%    \end{macrocode}
% Omit the first pass to get the penalties
% of the second pass.
%    \begin{macrocode}
      \pretolerance-1 %
%    \end{macrocode}
%  We don't want a third pass with \cs{emergencystretch}.
%    \begin{macrocode}
      \tolerance10000 %
      \hsize\maxdimen
%    \end{macrocode}
% Line is not underfull:
%    \begin{macrocode}
      \parfillskip 0pt plus 1filll\relax
      \leftskip0pt\relax
%    \end{macrocode}
% Suppress underful \cs{hbox} warnings,
% is explicit line breaks are used.
%    \begin{macrocode}
      \rightskip0pt plus 1fil\relax
      \everypar{}%
%    \end{macrocode}
% Ensure that there is a paragraph and
% prevents \cs{endgraph} from eating terminal glue:
%    \begin{macrocode}
      \kern0pt%
      \endgraf
      \setouterhboxRemove
E     \ifnum\lastnodetype=1 %
E       \global\setbox\setouterhboxBox\lastbox
E       \loop
E         \setouterhboxRemove
E       \ifnum\lastnodetype=1 %
E         \setbox0=\lastbox
E         \global\setbox\setouterhboxBox=\hbox{%
E           \unhbox0 %
%    \end{macrocode}
% Remove \cs{rightskip}, a penalty with -10000 is part of the previous line.
%    \begin{macrocode}
E           \unskip
E           \unhbox\setouterhboxBox
E         }%
E       \repeat
E     \else
E       \setouterhboxFailure{%
E         Something is wrong%
E       }{%
E         Could not find expected line.%
E         \MessageBreak
E         (\string\lastnodetype: \number\lastnodetype, expected: 1)%
E       }%
E     \fi
E     \setouterhboxRemove
T     \global\setbox\setouterhboxBox\lastbox
T     \loop
T       \setouterhboxRemove
T       \setbox0=\lastbox
T     \ifcase\ifvoid0 1\else0\fi
T       \global\setbox\setouterhboxBox=\hbox{%
T         \unhbox0 %
%    \end{macrocode}
% Remove \cs{rightskip}, a penalty with -10000 is part of the previous line.
%    \begin{macrocode}
T         \unskip
T         \unhbox\setouterhboxBox
T       }%
T     \repeat
T     \ifdim.123pt=\lastkern
T     \else
T       \setouterhboxFailure{%
T         Something is wrong%
T       }{%
T         Unexpected stuff was detected before the line.%
T       }%
T     \fi
T   \egroup
T   \ifcase \ifnum\wd0=0 \else 1\fi
T           \ifdim\ht0=.123pt \else 1\fi
T           \ifnum\dp0=0 \else 1\fi
T           0 %
E     \ifnum\lastnodetype=-1 %
%    \end{macrocode}
% There was just one line that we have caught.
%    \begin{macrocode}
      \else
        \setouterhboxFailure{%
            Something is wrong%
        }{%
            After fetching the line there is more unexpected stuff.%
E           \MessageBreak
E           (\string\lastnodetype: \number\lastnodetype, expected: -1)%
        }%
      \fi
E   \egroup
  \expandafter\endgroup
  \expandafter\setouterhboxFinish\expandafter{%
    \number\setouterhboxNum
  }%
}
%    \end{macrocode}
%    \end{macro}
%
% \subsection{Environment support}
%
% Check \cs{@currenvir} for the case that \cs{setouterhbox}
% was called as environment. Then the box assignment
% must be put after the \cs{endgroup} of |\end{|\dots|}|.
%    \begin{macrocode}
\def\setouterhboxCurr{setouterhbox}
\def\setouterhboxLast#1{%
  \setbox#1\hbox{%
    \unhbox\setouterhboxBox
    \unskip % remove \rightskip glue
    \unskip % remove \parfillskip glue
    \unpenalty % remove paragraph ending \penalty 10000
    \unkern % remove explicit kern inserted above
  }%
}
%    \end{macrocode}
%    \begin{macro}{\setouterhboxFinish}
% |#1| is an explicit number.
%    \begin{macrocode}
\def\setouterhboxFinish#1{%
  \begingroup\expandafter\expandafter\expandafter\endgroup
  \expandafter\ifx\csname @currenvir\endcsname\setouterhboxCurr
    \aftergroup\setouterhboxLast
    \aftergroup{%
    \setouterhboxAfter #1\NIL
    \aftergroup}%
  \else
    \setouterhboxLast{#1}%
  \fi
}
%    \end{macrocode}
%    \end{macro}
%    \begin{macro}{\setouterhboxAfter}
% |#1| is an explicit number.
%    \begin{macrocode}
\def\setouterhboxAfter#1#2\NIL{%
  \aftergroup#1%
  \ifx\\#2\\%
  \else
    \setouterhboxReturnAfterFi{%
      \setouterhboxAfter#2\NIL
    }%
  \fi
}
%    \end{macrocode}
%    \end{macro}
%    \begin{macro}{\setouterhboxReturnAfterFi}
% A utility macro to get tail recursion.
%    \begin{macrocode}
\long\def\setouterhboxReturnAfterFi#1\fi{\fi#1}
%    \end{macrocode}
%    \end{macro}
% Restore catcodes we have need to distinguish between
% the implementation with and without \eTeX.
%    \begin{macrocode}
\catcode69=11\relax % E
\catcode84=11\relax % T
%    \end{macrocode}
%
% \subsection{Option \xoption{hyperref}}
%    \begin{macrocode}
\begingroup
  \def\x{LaTeX2e}%
\expandafter\endgroup
\ifx\x\fmtname
\else
  \expandafter\setouterhboxAtEnd
\fi%
%    \end{macrocode}
%    \begin{macro}{\Hy@setouterhbox}
% \cs{Hy@setouterhbox} is the internal hook that \xpackage{hyperref}
% uses since 2006/02/12 v6.75a.
%    \begin{macrocode}
\DeclareOption{hyperref}{%
  \long\def\Hy@setouterhbox#1#2{%
    \setouterhbox{#1}#2\endsetouterhbox
  }%
}
%    \end{macrocode}
%    \end{macro}
%    \begin{macrocode}
\ProcessOptions\relax
%    \end{macrocode}
%
%    \begin{macrocode}
\setouterhboxAtEnd%
%</package>
%    \end{macrocode}
%
% \section{Test}
%
% \subsection{Catcode checks for loading}
%
%    \begin{macrocode}
%<*test1>
%    \end{macrocode}
%    \begin{macrocode}
\catcode`\{=1 %
\catcode`\}=2 %
\catcode`\#=6 %
\catcode`\@=11 %
\expandafter\ifx\csname count@\endcsname\relax
  \countdef\count@=255 %
\fi
\expandafter\ifx\csname @gobble\endcsname\relax
  \long\def\@gobble#1{}%
\fi
\expandafter\ifx\csname @firstofone\endcsname\relax
  \long\def\@firstofone#1{#1}%
\fi
\expandafter\ifx\csname loop\endcsname\relax
  \expandafter\@firstofone
\else
  \expandafter\@gobble
\fi
{%
  \def\loop#1\repeat{%
    \def\body{#1}%
    \iterate
  }%
  \def\iterate{%
    \body
      \let\next\iterate
    \else
      \let\next\relax
    \fi
    \next
  }%
  \let\repeat=\fi
}%
\def\RestoreCatcodes{}
\count@=0 %
\loop
  \edef\RestoreCatcodes{%
    \RestoreCatcodes
    \catcode\the\count@=\the\catcode\count@\relax
  }%
\ifnum\count@<255 %
  \advance\count@ 1 %
\repeat

\def\RangeCatcodeInvalid#1#2{%
  \count@=#1\relax
  \loop
    \catcode\count@=15 %
  \ifnum\count@<#2\relax
    \advance\count@ 1 %
  \repeat
}
\def\RangeCatcodeCheck#1#2#3{%
  \count@=#1\relax
  \loop
    \ifnum#3=\catcode\count@
    \else
      \errmessage{%
        Character \the\count@\space
        with wrong catcode \the\catcode\count@\space
        instead of \number#3%
      }%
    \fi
  \ifnum\count@<#2\relax
    \advance\count@ 1 %
  \repeat
}
\def\space{ }
\expandafter\ifx\csname LoadCommand\endcsname\relax
  \def\LoadCommand{\input setouterhbox.sty\relax}%
\fi
\def\Test{%
  \RangeCatcodeInvalid{0}{47}%
  \RangeCatcodeInvalid{58}{64}%
  \RangeCatcodeInvalid{91}{96}%
  \RangeCatcodeInvalid{123}{255}%
  \catcode`\@=12 %
  \catcode`\\=0 %
  \catcode`\%=14 %
  \LoadCommand
  \RangeCatcodeCheck{0}{36}{15}%
  \RangeCatcodeCheck{37}{37}{14}%
  \RangeCatcodeCheck{38}{47}{15}%
  \RangeCatcodeCheck{48}{57}{12}%
  \RangeCatcodeCheck{58}{63}{15}%
  \RangeCatcodeCheck{64}{64}{12}%
  \RangeCatcodeCheck{65}{90}{11}%
  \RangeCatcodeCheck{91}{91}{15}%
  \RangeCatcodeCheck{92}{92}{0}%
  \RangeCatcodeCheck{93}{96}{15}%
  \RangeCatcodeCheck{97}{122}{11}%
  \RangeCatcodeCheck{123}{255}{15}%
  \RestoreCatcodes
}
\Test
\csname @@end\endcsname
\end
%    \end{macrocode}
%    \begin{macrocode}
%</test1>
%    \end{macrocode}
%
% \subsection{Test with package \xpackage{url}}
%
%    \begin{macrocode}
%<*test2>
\nofiles
\documentclass[a5paper]{article}
\usepackage{url}[2005/06/27]
\usepackage{setouterhbox}

\newsavebox{\testbox}

\setlength{\parindent}{0pt}
\setlength{\parskip}{2em}

\begin{document}
\raggedright

\url{http://this.is.a.very.long.host.name/followed/%
by/a/very_long_long_long_path.html}%

\sbox\testbox{%
  \url{http://this.is.a.very.long.host.name/followed/%
  by/a/very_long_long_long_path.html}%
}%
\unhbox\testbox

\begin{setouterhbox}{\testbox}%
  \url{http://this.is.a.very.long.host.name/followed/%
  by/a/very_long_long_long_path.html}%
\end{setouterhbox}
\unhbox\testbox

\end{document}
%</test2>
%    \end{macrocode}
%
% \section{Installation}
%
% \subsection{Download}
%
% \paragraph{Package.} This package is available on
% CTAN\footnote{\CTANpkg{setouterhbox}}:
% \begin{description}
% \item[\CTAN{macros/latex/contrib/oberdiek/setouterhbox.dtx}] The source file.
% \item[\CTAN{macros/latex/contrib/oberdiek/setouterhbox.pdf}] Documentation.
% \end{description}
%
%
% \paragraph{Bundle.} All the packages of the bundle `oberdiek'
% are also available in a TDS compliant ZIP archive. There
% the packages are already unpacked and the documentation files
% are generated. The files and directories obey the TDS standard.
% \begin{description}
% \item[\CTANinstall{install/macros/latex/contrib/oberdiek.tds.zip}]
% \end{description}
% \emph{TDS} refers to the standard ``A Directory Structure
% for \TeX\ Files'' (\CTAN{tds/tds.pdf}). Directories
% with \xfile{texmf} in their name are usually organized this way.
%
% \subsection{Bundle installation}
%
% \paragraph{Unpacking.} Unpack the \xfile{oberdiek.tds.zip} in the
% TDS tree (also known as \xfile{texmf} tree) of your choice.
% Example (linux):
% \begin{quote}
%   |unzip oberdiek.tds.zip -d ~/texmf|
% \end{quote}
%
% \subsection{Package installation}
%
% \paragraph{Unpacking.} The \xfile{.dtx} file is a self-extracting
% \docstrip\ archive. The files are extracted by running the
% \xfile{.dtx} through \plainTeX:
% \begin{quote}
%   \verb|tex setouterhbox.dtx|
% \end{quote}
%
% \paragraph{TDS.} Now the different files must be moved into
% the different directories in your installation TDS tree
% (also known as \xfile{texmf} tree):
% \begin{quote}
% \def\t{^^A
% \begin{tabular}{@{}>{\ttfamily}l@{ $\rightarrow$ }>{\ttfamily}l@{}}
%   setouterhbox.sty & tex/generic/oberdiek/setouterhbox.sty\\
%   setouterhbox.pdf & doc/latex/oberdiek/setouterhbox.pdf\\
%   setouterhbox-example.tex & doc/latex/oberdiek/setouterhbox-example.tex\\
%   test/setouterhbox-test1.tex & doc/latex/oberdiek/test/setouterhbox-test1.tex\\
%   test/setouterhbox-test2.tex & doc/latex/oberdiek/test/setouterhbox-test2.tex\\
%   setouterhbox.dtx & source/latex/oberdiek/setouterhbox.dtx\\
% \end{tabular}^^A
% }^^A
% \sbox0{\t}^^A
% \ifdim\wd0>\linewidth
%   \begingroup
%     \advance\linewidth by\leftmargin
%     \advance\linewidth by\rightmargin
%   \edef\x{\endgroup
%     \def\noexpand\lw{\the\linewidth}^^A
%   }\x
%   \def\lwbox{^^A
%     \leavevmode
%     \hbox to \linewidth{^^A
%       \kern-\leftmargin\relax
%       \hss
%       \usebox0
%       \hss
%       \kern-\rightmargin\relax
%     }^^A
%   }^^A
%   \ifdim\wd0>\lw
%     \sbox0{\small\t}^^A
%     \ifdim\wd0>\linewidth
%       \ifdim\wd0>\lw
%         \sbox0{\footnotesize\t}^^A
%         \ifdim\wd0>\linewidth
%           \ifdim\wd0>\lw
%             \sbox0{\scriptsize\t}^^A
%             \ifdim\wd0>\linewidth
%               \ifdim\wd0>\lw
%                 \sbox0{\tiny\t}^^A
%                 \ifdim\wd0>\linewidth
%                   \lwbox
%                 \else
%                   \usebox0
%                 \fi
%               \else
%                 \lwbox
%               \fi
%             \else
%               \usebox0
%             \fi
%           \else
%             \lwbox
%           \fi
%         \else
%           \usebox0
%         \fi
%       \else
%         \lwbox
%       \fi
%     \else
%       \usebox0
%     \fi
%   \else
%     \lwbox
%   \fi
% \else
%   \usebox0
% \fi
% \end{quote}
% If you have a \xfile{docstrip.cfg} that configures and enables \docstrip's
% TDS installing feature, then some files can already be in the right
% place, see the documentation of \docstrip.
%
% \subsection{Refresh file name databases}
%
% If your \TeX~distribution
% (\TeX\,Live, \mikTeX, \dots) relies on file name databases, you must refresh
% these. For example, \TeX\,Live\ users run \verb|texhash| or
% \verb|mktexlsr|.
%
% \subsection{Some details for the interested}
%
% \paragraph{Unpacking with \LaTeX.}
% The \xfile{.dtx} chooses its action depending on the format:
% \begin{description}
% \item[\plainTeX:] Run \docstrip\ and extract the files.
% \item[\LaTeX:] Generate the documentation.
% \end{description}
% If you insist on using \LaTeX\ for \docstrip\ (really,
% \docstrip\ does not need \LaTeX), then inform the autodetect routine
% about your intention:
% \begin{quote}
%   \verb|latex \let\install=y\input{setouterhbox.dtx}|
% \end{quote}
% Do not forget to quote the argument according to the demands
% of your shell.
%
% \paragraph{Generating the documentation.}
% You can use both the \xfile{.dtx} or the \xfile{.drv} to generate
% the documentation. The process can be configured by the
% configuration file \xfile{ltxdoc.cfg}. For instance, put this
% line into this file, if you want to have A4 as paper format:
% \begin{quote}
%   \verb|\PassOptionsToClass{a4paper}{article}|
% \end{quote}
% An example follows how to generate the
% documentation with pdf\LaTeX:
% \begin{quote}
%\begin{verbatim}
%pdflatex setouterhbox.dtx
%makeindex -s gind.ist setouterhbox.idx
%pdflatex setouterhbox.dtx
%makeindex -s gind.ist setouterhbox.idx
%pdflatex setouterhbox.dtx
%\end{verbatim}
% \end{quote}
%
% \begin{thebibliography}{9}
%
% \bibitem{newsstart}
%   Damian Menscher, \Newsgroup{comp.text.tex},
%   \textit{overlong lines in List of Figures},
%   \nolinkurl{<dh058t$qbd$1@news.ks.uiuc.edu>},
%   23rd September 2005.
%   \url{https://groups.google.com/group/comp.text.tex/msg/79648d4cf1f8bc13}
%
% \bibitem{kastrup}
%   David Kastrup, \Newsgroup{comp.text.tex},
%   \textit{Re: ANN: outerhbox.sty -- collect horizontal material,
%   for unboxing into a paragraph},
%   \nolinkurl{<85y855lrx3.fsf@lola.goethe.zz>},
%   7th October 2005.
%   \url{https://groups.google.com/group/comp.text.tex/msg/7cf0a345ef932e52}
%
% \bibitem{downes}
%   Michael Downes, \textit{Line breaking in \cs{unhbox}ed Text},
%   TUGboat 11 (1990), pp. 605--612.
%
% \bibitem{hyperref}
%   Sebastian Rahtz, Heiko Oberdiek:
%   \textit{The \xpackage{hyperref} package};
%   2006/08/16 v6.75c;
%   \CTANpkg{hyperref}.
%
% \end{thebibliography}
%
% \begin{History}
%   \begin{Version}{2005/10/05 v1.0}
%   \item
%     First version.
%   \end{Version}
%   \begin{Version}{2005/10/07 v1.1}
%   \item
%     Option \xoption{hyperref} added.
%   \end{Version}
%   \begin{Version}{2005/10/18 v1.2}
%   \item
%     Support for explicit line breaks added.
%   \end{Version}
%   \begin{Version}{2006/02/12 v1.3}
%   \item
%     DTX format.
%   \item
%     Documentation extended.
%   \end{Version}
%   \begin{Version}{2006/08/26 v1.4}
%   \item
%     Date of hyperref updated.
%   \end{Version}
%   \begin{Version}{2007/04/26 v1.5}
%   \item
%     Use of package \xpackage{infwarerr}.
%   \end{Version}
%   \begin{Version}{2007/05/17 v1.6}
%   \item
%     Standard header part for generic files.
%   \end{Version}
%   \begin{Version}{2007/09/09 v1.7}
%   \item
%     Catcode section added.
%   \end{Version}
%   \begin{Version}{2016/05/16 v1.8}
%   \item
%     Documentation updates.
%   \end{Version}
% \end{History}
%
% \PrintIndex
%
% \Finale
\endinput

%        (quote the arguments according to the demands of your shell)
%
% Documentation:
%    (a) If setouterhbox.drv is present:
%           latex setouterhbox.drv
%    (b) Without setouterhbox.drv:
%           latex setouterhbox.dtx; ...
%    The class ltxdoc loads the configuration file ltxdoc.cfg
%    if available. Here you can specify further options, e.g.
%    use A4 as paper format:
%       \PassOptionsToClass{a4paper}{article}
%
%    Programm calls to get the documentation (example):
%       pdflatex setouterhbox.dtx
%       makeindex -s gind.ist setouterhbox.idx
%       pdflatex setouterhbox.dtx
%       makeindex -s gind.ist setouterhbox.idx
%       pdflatex setouterhbox.dtx
%
% Installation:
%    TDS:tex/generic/oberdiek/setouterhbox.sty
%    TDS:doc/latex/oberdiek/setouterhbox.pdf
%    TDS:doc/latex/oberdiek/setouterhbox-example.tex
%    TDS:doc/latex/oberdiek/test/setouterhbox-test1.tex
%    TDS:doc/latex/oberdiek/test/setouterhbox-test2.tex
%    TDS:source/latex/oberdiek/setouterhbox.dtx
%
%<*ignore>
\begingroup
  \catcode123=1 %
  \catcode125=2 %
  \def\x{LaTeX2e}%
\expandafter\endgroup
\ifcase 0\ifx\install y1\fi\expandafter
         \ifx\csname processbatchFile\endcsname\relax\else1\fi
         \ifx\fmtname\x\else 1\fi\relax
\else\csname fi\endcsname
%</ignore>
%<*install>
\input docstrip.tex
\Msg{************************************************************************}
\Msg{* Installation}
\Msg{* Package: setouterhbox 2016/05/16 v1.8 Set hbox in outer horizontal mode (HO)}
\Msg{************************************************************************}

\keepsilent
\askforoverwritefalse

\let\MetaPrefix\relax
\preamble

This is a generated file.

Project: setouterhbox
Version: 2016/05/16 v1.8

Copyright (C)
   2005-2007 Heiko Oberdiek
   2016-2019 Oberdiek Package Support Group

This work may be distributed and/or modified under the
conditions of the LaTeX Project Public License, either
version 1.3c of this license or (at your option) any later
version. This version of this license is in
   https://www.latex-project.org/lppl/lppl-1-3c.txt
and the latest version of this license is in
   https://www.latex-project.org/lppl.txt
and version 1.3 or later is part of all distributions of
LaTeX version 2005/12/01 or later.

This work has the LPPL maintenance status "maintained".

The Current Maintainers of this work are
Heiko Oberdiek and the Oberdiek Package Support Group
https://github.com/ho-tex/oberdiek/issues


The Base Interpreter refers to any `TeX-Format',
because some files are installed in TDS:tex/generic//.

This work consists of the main source file setouterhbox.dtx
and the derived files
   setouterhbox.sty, setouterhbox.pdf, setouterhbox.ins, setouterhbox.drv,
   setouterhbox-example.tex, setouterhbox-test1.tex,
   setouterhbox-test2.tex.

\endpreamble
\let\MetaPrefix\DoubleperCent

\generate{%
  \file{setouterhbox.ins}{\from{setouterhbox.dtx}{install}}%
  \file{setouterhbox.drv}{\from{setouterhbox.dtx}{driver}}%
  \usedir{tex/generic/oberdiek}%
  \file{setouterhbox.sty}{\from{setouterhbox.dtx}{package}}%
  \usedir{doc/latex/oberdiek}%
  \file{setouterhbox-example.tex}{\from{setouterhbox.dtx}{example}}%
%  \usedir{doc/latex/oberdiek/test}%
%  \file{setouterhbox-test1.tex}{\from{setouterhbox.dtx}{test1}}%
%  \file{setouterhbox-test2.tex}{\from{setouterhbox.dtx}{test2}}%
  \nopreamble
  \nopostamble
%  \usedir{source/latex/oberdiek/catalogue}%
%  \file{setouterhbox.xml}{\from{setouterhbox.dtx}{catalogue}}%
}

\catcode32=13\relax% active space
\let =\space%
\Msg{************************************************************************}
\Msg{*}
\Msg{* To finish the installation you have to move the following}
\Msg{* file into a directory searched by TeX:}
\Msg{*}
\Msg{*     setouterhbox.sty}
\Msg{*}
\Msg{* To produce the documentation run the file `setouterhbox.drv'}
\Msg{* through LaTeX.}
\Msg{*}
\Msg{* Happy TeXing!}
\Msg{*}
\Msg{************************************************************************}

\endbatchfile
%</install>
%<*ignore>
\fi
%</ignore>
%<*driver>
\NeedsTeXFormat{LaTeX2e}
\ProvidesFile{setouterhbox.drv}%
  [2016/05/16 v1.8 Set hbox in outer horizontal mode (HO)]%
\documentclass{ltxdoc}
\usepackage{holtxdoc}[2011/11/22]
\begin{document}
  \DocInput{setouterhbox.dtx}%
\end{document}
%</driver>
% \fi
%
%
% \CharacterTable
%  {Upper-case    \A\B\C\D\E\F\G\H\I\J\K\L\M\N\O\P\Q\R\S\T\U\V\W\X\Y\Z
%   Lower-case    \a\b\c\d\e\f\g\h\i\j\k\l\m\n\o\p\q\r\s\t\u\v\w\x\y\z
%   Digits        \0\1\2\3\4\5\6\7\8\9
%   Exclamation   \!     Double quote  \"     Hash (number) \#
%   Dollar        \$     Percent       \%     Ampersand     \&
%   Acute accent  \'     Left paren    \(     Right paren   \)
%   Asterisk      \*     Plus          \+     Comma         \,
%   Minus         \-     Point         \.     Solidus       \/
%   Colon         \:     Semicolon     \;     Less than     \<
%   Equals        \=     Greater than  \>     Question mark \?
%   Commercial at \@     Left bracket  \[     Backslash     \\
%   Right bracket \]     Circumflex    \^     Underscore    \_
%   Grave accent  \`     Left brace    \{     Vertical bar  \|
%   Right brace   \}     Tilde         \~}
%
% \GetFileInfo{setouterhbox.drv}
%
% \title{The \xpackage{setouterhbox} package}
% \date{2016/05/16 v1.8}
% \author{Heiko Oberdiek\thanks
% {Please report any issues at \url{https://github.com/ho-tex/oberdiek/issues}}}
%
% \maketitle
%
% \begin{abstract}
% If math stuff is set in an \cs{hbox}, then TeX
% performs some optimization and omits the implicite
% penalties \cs{binoppenalty} and \cs{relpenalty}.
% This packages tries to put stuff into an \cs{hbox}
% without getting lost of those penalties.
% \end{abstract}
%
% \tableofcontents
%
% \section{Documentation}
%
% \subsection{Introduction}
%
% There is a situation in \xpackage{hyperref}'s driver for dvips
% where the user wants to have links that can be broken across
% lines. However dvips doesn't support the feature. With option
% \xoption{breaklinks} \xpackage{hyperref} sets the links as
% usual, put them in a box and write the link data with
% box dimensions into the appropriate \cs{special}s.
% Then, however, it does not set the complete unbreakable
% box, but it unwrappes the material inside to allow line
% breaks. Of course line breaking and glue setting will falsify
% the link dimensions, but line breaking was more important
% for the user.
%
% \subsection{Acknowledgement}
%
% Jonathan Fine, Donald Arsenau and me discussed the problem
% in the newsgroup \xnewsgroup{comp.text.tex} where Damian
% Menscher has started the thread, see \cite{newsstart}.
%
% The discussion was productive and generated many ideas
% and code examples. In order to have a more permanent
% result I wrote this package and tried to implement
% most of the ideas, a kind of summary of the discussion.
% Thus I want and have to thank Jonathan Fine and Donald Arsenau
% very much.
%
% Two weeks later David Kastrup (posting in
% \xnewsgroup{comp.text.tex}, \cite{kastrup})
% remembered an old article of Michael Downes (\cite{downes})
% in TUGboat, where Michael Downes already presented the
% method we discuss here. Nowadays we have \eTeX\ that extends
% the tool set of a \TeX\ macro programmer. Especially useful
% \eTeX\ was in this package for detecting and dealing with
% errorneous situations.
%
% However also nowadays a perfect solution for the problem
% is still missing at macro level. Probably someone has
% to go deep in the internals of the \TeX\ compiler to
% implement a switch that let penalties stay where otherwise
% \TeX\ would remove them for optimization reasons.
%
% \subsection{Usage}
%
% \paragraph{Package loading.}
% \LaTeX: as usually:
% \begin{quote}
%   |\usepackage{setouterhbox}|
% \end{quote}
% The package can also be included directly, thus \plainTeX\ users
% write:
% \begin{quote}
%   |\input setouterhbox.sty|
% \end{quote}
%
% \paragraph{Register allocation.}
% The material will be put into a box, thus we need to know these
% box number. If you need to allocate a new box register:
% \begin{description}
%  \item[\LaTeX:] |\newsavebox{\|\meta{name}|}|
%  \item[\plainTeX:] |\newbox\|\meta{name}
% \end{description}
% Then |\|\meta{name} is a command that held the box number.
%
% \paragraph{Box wrapping.}
% \LaTeX\ users put the material in the box with an environment
% similar to \texttt{lrbox}. The environment \texttt{setouterhbox}
% uses the same syntax and offers the same features, such
% as verbatim stuff inside:
% \begin{quote}
%  |\begin{setouterhbox}{|\meta{box number}|}|\dots
%  |\end{setouterhbox}|
% \end{quote}
% Users with \plainTeX\ do not have environments, they use instead:
% \begin{quote}
%   |\setouterhbox{|\meta{box number}|}|\dots|\endsetouterhbox|
% \end{quote}
% In both cases the material is put into an \cs{hbox} and assigned
% to the given box, denoted by \meta{box number}. Note the
% assignment is local, the same way \texttt{lrbox} behaves.
%
% \paragraph{Unwrapping.}
% The box material is ready for unwrapping:
% \begin{quote}
%   |\unhbox|\meta{box number}
% \end{quote}
%
% \subsection{Option \xoption{hyperref}}
%
% Package url uses math mode for typesetting urls.
% Break points are inserted by \cs{binoppenalty} and
% \cs{relpenalty}. Unhappily these break points are
% removed, if \xpackage{hyperref}
% is used with option {breaklinks}
% and drivers that depend on \xoption{pdfmark}:
% \xoption{dvips}, \xoption{vtexpdfmark}, \xoption{textures},
% and \xoption{dvipsone}.
% Thus the option \xoption{hyperref} enables the method
% of this package to avoid the removal of \cs{relpenalty}
% and \cs{binoppenalty}. Thus you get more break points.
% However, the link areas are still wrong for these
% drivers, because they are not supporting broken
% links.
%
% Note, you need version 2006/08/16 v6.75c of package \xpackage{hyperref},
% because starting with this version the necessary hook is provided
% that package \xpackage{setouterhbox} uses.
% \begin{quote}
%   |\usepackage[|\dots|]{hyperref}[2006/08/16]|\\
%   |\usepackage[hyperref]{setouterhbox}|
% \end{quote}
% Package order does not matter.
%
% \subsection{Example}
%
%    \begin{macrocode}
%<*example>
\documentclass[a5paper]{article}
\usepackage{url}[2005/06/27]
\usepackage{setouterhbox}

\newsavebox{\testbox}

\setlength{\parindent}{0pt}
\setlength{\parskip}{2em}

\begin{document}
\raggedright

\url{http://this.is.a.very.long.host.name/followed/%
by/a/very_long_long_long_path.html}%

\sbox\testbox{%
  \url{http://this.is.a.very.long.host.name/followed/%
  by/a/very_long_long_long_path.html}%
}%
\unhbox\testbox

\begin{setouterhbox}{\testbox}%
  \url{http://this.is.a.very.long.host.name/followed/%
  by/a/very_long_long_long_path.html}%
\end{setouterhbox}
\unhbox\testbox

\end{document}
%</example>
%    \end{macrocode}
%
% \StopEventually{
% }
%
% \section{Implementation}
%
% Internal macros are prefixed by \cs{setouterhbox}, |@| is
% not used inside names, thus we do not need to care of its
% catcode if we are not using it as \LaTeX\ package.
%
% \subsection{Package start stuff}
%
%    \begin{macrocode}
%<*package>
%    \end{macrocode}
%
% Prevent reloading more than one, necessary for \plainTeX:
%    Reload check, especially if the package is not used with \LaTeX.
%    \begin{macrocode}
\begingroup\catcode61\catcode48\catcode32=10\relax%
  \catcode13=5 % ^^M
  \endlinechar=13 %
  \catcode35=6 % #
  \catcode39=12 % '
  \catcode44=12 % ,
  \catcode45=12 % -
  \catcode46=12 % .
  \catcode58=12 % :
  \catcode64=11 % @
  \catcode123=1 % {
  \catcode125=2 % }
  \expandafter\let\expandafter\x\csname ver@setouterhbox.sty\endcsname
  \ifx\x\relax % plain-TeX, first loading
  \else
    \def\empty{}%
    \ifx\x\empty % LaTeX, first loading,
      % variable is initialized, but \ProvidesPackage not yet seen
    \else
      \expandafter\ifx\csname PackageInfo\endcsname\relax
        \def\x#1#2{%
          \immediate\write-1{Package #1 Info: #2.}%
        }%
      \else
        \def\x#1#2{\PackageInfo{#1}{#2, stopped}}%
      \fi
      \x{setouterhbox}{The package is already loaded}%
      \aftergroup\endinput
    \fi
  \fi
\endgroup%
%    \end{macrocode}
%    Package identification:
%    \begin{macrocode}
\begingroup\catcode61\catcode48\catcode32=10\relax%
  \catcode13=5 % ^^M
  \endlinechar=13 %
  \catcode35=6 % #
  \catcode39=12 % '
  \catcode40=12 % (
  \catcode41=12 % )
  \catcode44=12 % ,
  \catcode45=12 % -
  \catcode46=12 % .
  \catcode47=12 % /
  \catcode58=12 % :
  \catcode64=11 % @
  \catcode91=12 % [
  \catcode93=12 % ]
  \catcode123=1 % {
  \catcode125=2 % }
  \expandafter\ifx\csname ProvidesPackage\endcsname\relax
    \def\x#1#2#3[#4]{\endgroup
      \immediate\write-1{Package: #3 #4}%
      \xdef#1{#4}%
    }%
  \else
    \def\x#1#2[#3]{\endgroup
      #2[{#3}]%
      \ifx#1\@undefined
        \xdef#1{#3}%
      \fi
      \ifx#1\relax
        \xdef#1{#3}%
      \fi
    }%
  \fi
\expandafter\x\csname ver@setouterhbox.sty\endcsname
\ProvidesPackage{setouterhbox}%
  [2016/05/16 v1.8 Set hbox in outer horizontal mode (HO)]%
%    \end{macrocode}
%
%    \begin{macrocode}
\begingroup\catcode61\catcode48\catcode32=10\relax%
  \catcode13=5 % ^^M
  \endlinechar=13 %
  \catcode123=1 % {
  \catcode125=2 % }
  \catcode64=11 % @
  \def\x{\endgroup
    \expandafter\edef\csname setouterhboxAtEnd\endcsname{%
      \endlinechar=\the\endlinechar\relax
      \catcode13=\the\catcode13\relax
      \catcode32=\the\catcode32\relax
      \catcode35=\the\catcode35\relax
      \catcode61=\the\catcode61\relax
      \catcode64=\the\catcode64\relax
      \catcode123=\the\catcode123\relax
      \catcode125=\the\catcode125\relax
    }%
  }%
\x\catcode61\catcode48\catcode32=10\relax%
\catcode13=5 % ^^M
\endlinechar=13 %
\catcode35=6 % #
\catcode64=11 % @
\catcode123=1 % {
\catcode125=2 % }
\def\TMP@EnsureCode#1#2{%
  \edef\setouterhboxAtEnd{%
    \setouterhboxAtEnd
    \catcode#1=\the\catcode#1\relax
  }%
  \catcode#1=#2\relax
}
\TMP@EnsureCode{40}{12}% (
\TMP@EnsureCode{41}{12}% )
\TMP@EnsureCode{44}{12}% ,
\TMP@EnsureCode{45}{12}% -
\TMP@EnsureCode{46}{12}% .
\TMP@EnsureCode{47}{12}% /
\TMP@EnsureCode{58}{12}% :
\TMP@EnsureCode{60}{12}% <
\TMP@EnsureCode{62}{12}% >
\TMP@EnsureCode{91}{12}% [
\TMP@EnsureCode{93}{12}% ]
\TMP@EnsureCode{96}{12}% `
\edef\setouterhboxAtEnd{\setouterhboxAtEnd\noexpand\endinput}
%    \end{macrocode}
%
% \subsection{Interface macros}
%
%    \begin{macro}{\setouterhboxBox}
% The method requires a global box assignment. To be on the
% safe side, a new box register is allocated for this
% global box assignment.
%    \begin{macrocode}
\newbox\setouterhboxBox
%    \end{macrocode}
%    \end{macro}
%
%    \begin{macro}{\setouterhboxFailure}
% Error message for both \plainTeX\ and \LaTeX
%    \begin{macrocode}
\begingroup\expandafter\expandafter\expandafter\endgroup
\expandafter\ifx\csname RequirePackage\endcsname\relax
  \input infwarerr.sty\relax
\else
  \RequirePackage{infwarerr}[2016/05/16]%
\fi
\edef\setouterhboxFailure#1#2{%
  \expandafter\noexpand\csname @PackageError\endcsname
      {setouterhbox}{#1}{#2}%
}
%    \end{macrocode}
%    \end{macro}
%
% \subsection{Main part}
%
% eTeX provides much better means for checking
% error conditions. Thus lines marked by "E" are executed
% if eTeX is available, otherwise the lines marked by "T" are
% used.
%    \begin{macrocode}
\begingroup\expandafter\expandafter\expandafter\endgroup
\expandafter\ifx\csname lastnodetype\endcsname\relax
  \catcode`T=9 % ignore
  \catcode`E=14 % comment
\else
  \catcode`T=14 % comment
  \catcode`E=9 % ignore
\fi
%    \end{macrocode}
%
%    \begin{macro}{\setouterhboxRemove}
% Remove all kern, glue, and penalty nodes;
% poor man's version, if \eTeX\ is not available
%    \begin{macrocode}
\def\setouterhboxRemove{%
E \ifnum\lastnodetype<11 %
E   \else
E   \ifnum\lastnodetype>13 %
E   \else
      \unskip\unkern\unpenalty
E     \expandafter\expandafter\expandafter\setouterhboxRemove
E   \fi
E \fi
}%
%    \end{macrocode}
%    \end{macro}
%
%    \begin{macro}{\setouterhbox}
% Passing the box contents by macro parameter would prevent
% catcode changes in the box contents like by \cs{verb}.
% Also \cs{bgroup} and \cs{egroup} does not work, because stuff
% has to be added at the begin and end of the box, thus
% the syntax
% |\setouterhbox{|\meta{box number}|}|\dots|\endsetouterhbox|
% is used. Also we automatically get an environment \texttt{setouterhbox}
% if \LaTeX\ is used.
%    \begin{macrocode}
\def\setouterhbox#1{%
  \begingroup
    \def\setouterhboxNum{#1}%
    \setbox0\vbox\bgroup
T     \kern.123pt\relax % marker
T     \kern0pt\relax % removed by \setouterhboxRemove
      \begingroup
        \everypar{}%
        \noindent
}
%    \end{macrocode}
%    \end{macro}
%    \begin{macro}{\endsetouterhbox}
% Most of the work is done in the end part, thus the heart of
% the method follows:
%    \begin{macrocode}
\def\endsetouterhbox{%
      \endgroup
%    \end{macrocode}
% Omit the first pass to get the penalties
% of the second pass.
%    \begin{macrocode}
      \pretolerance-1 %
%    \end{macrocode}
%  We don't want a third pass with \cs{emergencystretch}.
%    \begin{macrocode}
      \tolerance10000 %
      \hsize\maxdimen
%    \end{macrocode}
% Line is not underfull:
%    \begin{macrocode}
      \parfillskip 0pt plus 1filll\relax
      \leftskip0pt\relax
%    \end{macrocode}
% Suppress underful \cs{hbox} warnings,
% is explicit line breaks are used.
%    \begin{macrocode}
      \rightskip0pt plus 1fil\relax
      \everypar{}%
%    \end{macrocode}
% Ensure that there is a paragraph and
% prevents \cs{endgraph} from eating terminal glue:
%    \begin{macrocode}
      \kern0pt%
      \endgraf
      \setouterhboxRemove
E     \ifnum\lastnodetype=1 %
E       \global\setbox\setouterhboxBox\lastbox
E       \loop
E         \setouterhboxRemove
E       \ifnum\lastnodetype=1 %
E         \setbox0=\lastbox
E         \global\setbox\setouterhboxBox=\hbox{%
E           \unhbox0 %
%    \end{macrocode}
% Remove \cs{rightskip}, a penalty with -10000 is part of the previous line.
%    \begin{macrocode}
E           \unskip
E           \unhbox\setouterhboxBox
E         }%
E       \repeat
E     \else
E       \setouterhboxFailure{%
E         Something is wrong%
E       }{%
E         Could not find expected line.%
E         \MessageBreak
E         (\string\lastnodetype: \number\lastnodetype, expected: 1)%
E       }%
E     \fi
E     \setouterhboxRemove
T     \global\setbox\setouterhboxBox\lastbox
T     \loop
T       \setouterhboxRemove
T       \setbox0=\lastbox
T     \ifcase\ifvoid0 1\else0\fi
T       \global\setbox\setouterhboxBox=\hbox{%
T         \unhbox0 %
%    \end{macrocode}
% Remove \cs{rightskip}, a penalty with -10000 is part of the previous line.
%    \begin{macrocode}
T         \unskip
T         \unhbox\setouterhboxBox
T       }%
T     \repeat
T     \ifdim.123pt=\lastkern
T     \else
T       \setouterhboxFailure{%
T         Something is wrong%
T       }{%
T         Unexpected stuff was detected before the line.%
T       }%
T     \fi
T   \egroup
T   \ifcase \ifnum\wd0=0 \else 1\fi
T           \ifdim\ht0=.123pt \else 1\fi
T           \ifnum\dp0=0 \else 1\fi
T           0 %
E     \ifnum\lastnodetype=-1 %
%    \end{macrocode}
% There was just one line that we have caught.
%    \begin{macrocode}
      \else
        \setouterhboxFailure{%
            Something is wrong%
        }{%
            After fetching the line there is more unexpected stuff.%
E           \MessageBreak
E           (\string\lastnodetype: \number\lastnodetype, expected: -1)%
        }%
      \fi
E   \egroup
  \expandafter\endgroup
  \expandafter\setouterhboxFinish\expandafter{%
    \number\setouterhboxNum
  }%
}
%    \end{macrocode}
%    \end{macro}
%
% \subsection{Environment support}
%
% Check \cs{@currenvir} for the case that \cs{setouterhbox}
% was called as environment. Then the box assignment
% must be put after the \cs{endgroup} of |\end{|\dots|}|.
%    \begin{macrocode}
\def\setouterhboxCurr{setouterhbox}
\def\setouterhboxLast#1{%
  \setbox#1\hbox{%
    \unhbox\setouterhboxBox
    \unskip % remove \rightskip glue
    \unskip % remove \parfillskip glue
    \unpenalty % remove paragraph ending \penalty 10000
    \unkern % remove explicit kern inserted above
  }%
}
%    \end{macrocode}
%    \begin{macro}{\setouterhboxFinish}
% |#1| is an explicit number.
%    \begin{macrocode}
\def\setouterhboxFinish#1{%
  \begingroup\expandafter\expandafter\expandafter\endgroup
  \expandafter\ifx\csname @currenvir\endcsname\setouterhboxCurr
    \aftergroup\setouterhboxLast
    \aftergroup{%
    \setouterhboxAfter #1\NIL
    \aftergroup}%
  \else
    \setouterhboxLast{#1}%
  \fi
}
%    \end{macrocode}
%    \end{macro}
%    \begin{macro}{\setouterhboxAfter}
% |#1| is an explicit number.
%    \begin{macrocode}
\def\setouterhboxAfter#1#2\NIL{%
  \aftergroup#1%
  \ifx\\#2\\%
  \else
    \setouterhboxReturnAfterFi{%
      \setouterhboxAfter#2\NIL
    }%
  \fi
}
%    \end{macrocode}
%    \end{macro}
%    \begin{macro}{\setouterhboxReturnAfterFi}
% A utility macro to get tail recursion.
%    \begin{macrocode}
\long\def\setouterhboxReturnAfterFi#1\fi{\fi#1}
%    \end{macrocode}
%    \end{macro}
% Restore catcodes we have need to distinguish between
% the implementation with and without \eTeX.
%    \begin{macrocode}
\catcode69=11\relax % E
\catcode84=11\relax % T
%    \end{macrocode}
%
% \subsection{Option \xoption{hyperref}}
%    \begin{macrocode}
\begingroup
  \def\x{LaTeX2e}%
\expandafter\endgroup
\ifx\x\fmtname
\else
  \expandafter\setouterhboxAtEnd
\fi%
%    \end{macrocode}
%    \begin{macro}{\Hy@setouterhbox}
% \cs{Hy@setouterhbox} is the internal hook that \xpackage{hyperref}
% uses since 2006/02/12 v6.75a.
%    \begin{macrocode}
\DeclareOption{hyperref}{%
  \long\def\Hy@setouterhbox#1#2{%
    \setouterhbox{#1}#2\endsetouterhbox
  }%
}
%    \end{macrocode}
%    \end{macro}
%    \begin{macrocode}
\ProcessOptions\relax
%    \end{macrocode}
%
%    \begin{macrocode}
\setouterhboxAtEnd%
%</package>
%    \end{macrocode}
%
% \section{Test}
%
% \subsection{Catcode checks for loading}
%
%    \begin{macrocode}
%<*test1>
%    \end{macrocode}
%    \begin{macrocode}
\catcode`\{=1 %
\catcode`\}=2 %
\catcode`\#=6 %
\catcode`\@=11 %
\expandafter\ifx\csname count@\endcsname\relax
  \countdef\count@=255 %
\fi
\expandafter\ifx\csname @gobble\endcsname\relax
  \long\def\@gobble#1{}%
\fi
\expandafter\ifx\csname @firstofone\endcsname\relax
  \long\def\@firstofone#1{#1}%
\fi
\expandafter\ifx\csname loop\endcsname\relax
  \expandafter\@firstofone
\else
  \expandafter\@gobble
\fi
{%
  \def\loop#1\repeat{%
    \def\body{#1}%
    \iterate
  }%
  \def\iterate{%
    \body
      \let\next\iterate
    \else
      \let\next\relax
    \fi
    \next
  }%
  \let\repeat=\fi
}%
\def\RestoreCatcodes{}
\count@=0 %
\loop
  \edef\RestoreCatcodes{%
    \RestoreCatcodes
    \catcode\the\count@=\the\catcode\count@\relax
  }%
\ifnum\count@<255 %
  \advance\count@ 1 %
\repeat

\def\RangeCatcodeInvalid#1#2{%
  \count@=#1\relax
  \loop
    \catcode\count@=15 %
  \ifnum\count@<#2\relax
    \advance\count@ 1 %
  \repeat
}
\def\RangeCatcodeCheck#1#2#3{%
  \count@=#1\relax
  \loop
    \ifnum#3=\catcode\count@
    \else
      \errmessage{%
        Character \the\count@\space
        with wrong catcode \the\catcode\count@\space
        instead of \number#3%
      }%
    \fi
  \ifnum\count@<#2\relax
    \advance\count@ 1 %
  \repeat
}
\def\space{ }
\expandafter\ifx\csname LoadCommand\endcsname\relax
  \def\LoadCommand{\input setouterhbox.sty\relax}%
\fi
\def\Test{%
  \RangeCatcodeInvalid{0}{47}%
  \RangeCatcodeInvalid{58}{64}%
  \RangeCatcodeInvalid{91}{96}%
  \RangeCatcodeInvalid{123}{255}%
  \catcode`\@=12 %
  \catcode`\\=0 %
  \catcode`\%=14 %
  \LoadCommand
  \RangeCatcodeCheck{0}{36}{15}%
  \RangeCatcodeCheck{37}{37}{14}%
  \RangeCatcodeCheck{38}{47}{15}%
  \RangeCatcodeCheck{48}{57}{12}%
  \RangeCatcodeCheck{58}{63}{15}%
  \RangeCatcodeCheck{64}{64}{12}%
  \RangeCatcodeCheck{65}{90}{11}%
  \RangeCatcodeCheck{91}{91}{15}%
  \RangeCatcodeCheck{92}{92}{0}%
  \RangeCatcodeCheck{93}{96}{15}%
  \RangeCatcodeCheck{97}{122}{11}%
  \RangeCatcodeCheck{123}{255}{15}%
  \RestoreCatcodes
}
\Test
\csname @@end\endcsname
\end
%    \end{macrocode}
%    \begin{macrocode}
%</test1>
%    \end{macrocode}
%
% \subsection{Test with package \xpackage{url}}
%
%    \begin{macrocode}
%<*test2>
\nofiles
\documentclass[a5paper]{article}
\usepackage{url}[2005/06/27]
\usepackage{setouterhbox}

\newsavebox{\testbox}

\setlength{\parindent}{0pt}
\setlength{\parskip}{2em}

\begin{document}
\raggedright

\url{http://this.is.a.very.long.host.name/followed/%
by/a/very_long_long_long_path.html}%

\sbox\testbox{%
  \url{http://this.is.a.very.long.host.name/followed/%
  by/a/very_long_long_long_path.html}%
}%
\unhbox\testbox

\begin{setouterhbox}{\testbox}%
  \url{http://this.is.a.very.long.host.name/followed/%
  by/a/very_long_long_long_path.html}%
\end{setouterhbox}
\unhbox\testbox

\end{document}
%</test2>
%    \end{macrocode}
%
% \section{Installation}
%
% \subsection{Download}
%
% \paragraph{Package.} This package is available on
% CTAN\footnote{\CTANpkg{setouterhbox}}:
% \begin{description}
% \item[\CTAN{macros/latex/contrib/oberdiek/setouterhbox.dtx}] The source file.
% \item[\CTAN{macros/latex/contrib/oberdiek/setouterhbox.pdf}] Documentation.
% \end{description}
%
%
% \paragraph{Bundle.} All the packages of the bundle `oberdiek'
% are also available in a TDS compliant ZIP archive. There
% the packages are already unpacked and the documentation files
% are generated. The files and directories obey the TDS standard.
% \begin{description}
% \item[\CTANinstall{install/macros/latex/contrib/oberdiek.tds.zip}]
% \end{description}
% \emph{TDS} refers to the standard ``A Directory Structure
% for \TeX\ Files'' (\CTAN{tds/tds.pdf}). Directories
% with \xfile{texmf} in their name are usually organized this way.
%
% \subsection{Bundle installation}
%
% \paragraph{Unpacking.} Unpack the \xfile{oberdiek.tds.zip} in the
% TDS tree (also known as \xfile{texmf} tree) of your choice.
% Example (linux):
% \begin{quote}
%   |unzip oberdiek.tds.zip -d ~/texmf|
% \end{quote}
%
% \subsection{Package installation}
%
% \paragraph{Unpacking.} The \xfile{.dtx} file is a self-extracting
% \docstrip\ archive. The files are extracted by running the
% \xfile{.dtx} through \plainTeX:
% \begin{quote}
%   \verb|tex setouterhbox.dtx|
% \end{quote}
%
% \paragraph{TDS.} Now the different files must be moved into
% the different directories in your installation TDS tree
% (also known as \xfile{texmf} tree):
% \begin{quote}
% \def\t{^^A
% \begin{tabular}{@{}>{\ttfamily}l@{ $\rightarrow$ }>{\ttfamily}l@{}}
%   setouterhbox.sty & tex/generic/oberdiek/setouterhbox.sty\\
%   setouterhbox.pdf & doc/latex/oberdiek/setouterhbox.pdf\\
%   setouterhbox-example.tex & doc/latex/oberdiek/setouterhbox-example.tex\\
%   test/setouterhbox-test1.tex & doc/latex/oberdiek/test/setouterhbox-test1.tex\\
%   test/setouterhbox-test2.tex & doc/latex/oberdiek/test/setouterhbox-test2.tex\\
%   setouterhbox.dtx & source/latex/oberdiek/setouterhbox.dtx\\
% \end{tabular}^^A
% }^^A
% \sbox0{\t}^^A
% \ifdim\wd0>\linewidth
%   \begingroup
%     \advance\linewidth by\leftmargin
%     \advance\linewidth by\rightmargin
%   \edef\x{\endgroup
%     \def\noexpand\lw{\the\linewidth}^^A
%   }\x
%   \def\lwbox{^^A
%     \leavevmode
%     \hbox to \linewidth{^^A
%       \kern-\leftmargin\relax
%       \hss
%       \usebox0
%       \hss
%       \kern-\rightmargin\relax
%     }^^A
%   }^^A
%   \ifdim\wd0>\lw
%     \sbox0{\small\t}^^A
%     \ifdim\wd0>\linewidth
%       \ifdim\wd0>\lw
%         \sbox0{\footnotesize\t}^^A
%         \ifdim\wd0>\linewidth
%           \ifdim\wd0>\lw
%             \sbox0{\scriptsize\t}^^A
%             \ifdim\wd0>\linewidth
%               \ifdim\wd0>\lw
%                 \sbox0{\tiny\t}^^A
%                 \ifdim\wd0>\linewidth
%                   \lwbox
%                 \else
%                   \usebox0
%                 \fi
%               \else
%                 \lwbox
%               \fi
%             \else
%               \usebox0
%             \fi
%           \else
%             \lwbox
%           \fi
%         \else
%           \usebox0
%         \fi
%       \else
%         \lwbox
%       \fi
%     \else
%       \usebox0
%     \fi
%   \else
%     \lwbox
%   \fi
% \else
%   \usebox0
% \fi
% \end{quote}
% If you have a \xfile{docstrip.cfg} that configures and enables \docstrip's
% TDS installing feature, then some files can already be in the right
% place, see the documentation of \docstrip.
%
% \subsection{Refresh file name databases}
%
% If your \TeX~distribution
% (\TeX\,Live, \mikTeX, \dots) relies on file name databases, you must refresh
% these. For example, \TeX\,Live\ users run \verb|texhash| or
% \verb|mktexlsr|.
%
% \subsection{Some details for the interested}
%
% \paragraph{Unpacking with \LaTeX.}
% The \xfile{.dtx} chooses its action depending on the format:
% \begin{description}
% \item[\plainTeX:] Run \docstrip\ and extract the files.
% \item[\LaTeX:] Generate the documentation.
% \end{description}
% If you insist on using \LaTeX\ for \docstrip\ (really,
% \docstrip\ does not need \LaTeX), then inform the autodetect routine
% about your intention:
% \begin{quote}
%   \verb|latex \let\install=y% \iffalse meta-comment
%
% File: setouterhbox.dtx
% Version: 2016/05/16 v1.8
% Info: Set hbox in outer horizontal mode
%
% Copyright (C)
%    2005-2007 Heiko Oberdiek
%    2016-2019 Oberdiek Package Support Group
%    https://github.com/ho-tex/oberdiek/issues
%
% This work may be distributed and/or modified under the
% conditions of the LaTeX Project Public License, either
% version 1.3c of this license or (at your option) any later
% version. This version of this license is in
%    https://www.latex-project.org/lppl/lppl-1-3c.txt
% and the latest version of this license is in
%    https://www.latex-project.org/lppl.txt
% and version 1.3 or later is part of all distributions of
% LaTeX version 2005/12/01 or later.
%
% This work has the LPPL maintenance status "maintained".
%
% The Current Maintainers of this work are
% Heiko Oberdiek and the Oberdiek Package Support Group
% https://github.com/ho-tex/oberdiek/issues
%
% The Base Interpreter refers to any `TeX-Format',
% because some files are installed in TDS:tex/generic//.
%
% This work consists of the main source file setouterhbox.dtx
% and the derived files
%    setouterhbox.sty, setouterhbox.pdf, setouterhbox.ins, setouterhbox.drv,
%    setouterhbox-example.tex, setouterhbox-test1.tex,
%    setouterhbox-test2.tex.
%
% Distribution:
%    CTAN:macros/latex/contrib/oberdiek/setouterhbox.dtx
%    CTAN:macros/latex/contrib/oberdiek/setouterhbox.pdf
%
% Unpacking:
%    (a) If setouterhbox.ins is present:
%           tex setouterhbox.ins
%    (b) Without setouterhbox.ins:
%           tex setouterhbox.dtx
%    (c) If you insist on using LaTeX
%           latex \let\install=y\input{setouterhbox.dtx}
%        (quote the arguments according to the demands of your shell)
%
% Documentation:
%    (a) If setouterhbox.drv is present:
%           latex setouterhbox.drv
%    (b) Without setouterhbox.drv:
%           latex setouterhbox.dtx; ...
%    The class ltxdoc loads the configuration file ltxdoc.cfg
%    if available. Here you can specify further options, e.g.
%    use A4 as paper format:
%       \PassOptionsToClass{a4paper}{article}
%
%    Programm calls to get the documentation (example):
%       pdflatex setouterhbox.dtx
%       makeindex -s gind.ist setouterhbox.idx
%       pdflatex setouterhbox.dtx
%       makeindex -s gind.ist setouterhbox.idx
%       pdflatex setouterhbox.dtx
%
% Installation:
%    TDS:tex/generic/oberdiek/setouterhbox.sty
%    TDS:doc/latex/oberdiek/setouterhbox.pdf
%    TDS:doc/latex/oberdiek/setouterhbox-example.tex
%    TDS:doc/latex/oberdiek/test/setouterhbox-test1.tex
%    TDS:doc/latex/oberdiek/test/setouterhbox-test2.tex
%    TDS:source/latex/oberdiek/setouterhbox.dtx
%
%<*ignore>
\begingroup
  \catcode123=1 %
  \catcode125=2 %
  \def\x{LaTeX2e}%
\expandafter\endgroup
\ifcase 0\ifx\install y1\fi\expandafter
         \ifx\csname processbatchFile\endcsname\relax\else1\fi
         \ifx\fmtname\x\else 1\fi\relax
\else\csname fi\endcsname
%</ignore>
%<*install>
\input docstrip.tex
\Msg{************************************************************************}
\Msg{* Installation}
\Msg{* Package: setouterhbox 2016/05/16 v1.8 Set hbox in outer horizontal mode (HO)}
\Msg{************************************************************************}

\keepsilent
\askforoverwritefalse

\let\MetaPrefix\relax
\preamble

This is a generated file.

Project: setouterhbox
Version: 2016/05/16 v1.8

Copyright (C)
   2005-2007 Heiko Oberdiek
   2016-2019 Oberdiek Package Support Group

This work may be distributed and/or modified under the
conditions of the LaTeX Project Public License, either
version 1.3c of this license or (at your option) any later
version. This version of this license is in
   https://www.latex-project.org/lppl/lppl-1-3c.txt
and the latest version of this license is in
   https://www.latex-project.org/lppl.txt
and version 1.3 or later is part of all distributions of
LaTeX version 2005/12/01 or later.

This work has the LPPL maintenance status "maintained".

The Current Maintainers of this work are
Heiko Oberdiek and the Oberdiek Package Support Group
https://github.com/ho-tex/oberdiek/issues


The Base Interpreter refers to any `TeX-Format',
because some files are installed in TDS:tex/generic//.

This work consists of the main source file setouterhbox.dtx
and the derived files
   setouterhbox.sty, setouterhbox.pdf, setouterhbox.ins, setouterhbox.drv,
   setouterhbox-example.tex, setouterhbox-test1.tex,
   setouterhbox-test2.tex.

\endpreamble
\let\MetaPrefix\DoubleperCent

\generate{%
  \file{setouterhbox.ins}{\from{setouterhbox.dtx}{install}}%
  \file{setouterhbox.drv}{\from{setouterhbox.dtx}{driver}}%
  \usedir{tex/generic/oberdiek}%
  \file{setouterhbox.sty}{\from{setouterhbox.dtx}{package}}%
  \usedir{doc/latex/oberdiek}%
  \file{setouterhbox-example.tex}{\from{setouterhbox.dtx}{example}}%
%  \usedir{doc/latex/oberdiek/test}%
%  \file{setouterhbox-test1.tex}{\from{setouterhbox.dtx}{test1}}%
%  \file{setouterhbox-test2.tex}{\from{setouterhbox.dtx}{test2}}%
  \nopreamble
  \nopostamble
%  \usedir{source/latex/oberdiek/catalogue}%
%  \file{setouterhbox.xml}{\from{setouterhbox.dtx}{catalogue}}%
}

\catcode32=13\relax% active space
\let =\space%
\Msg{************************************************************************}
\Msg{*}
\Msg{* To finish the installation you have to move the following}
\Msg{* file into a directory searched by TeX:}
\Msg{*}
\Msg{*     setouterhbox.sty}
\Msg{*}
\Msg{* To produce the documentation run the file `setouterhbox.drv'}
\Msg{* through LaTeX.}
\Msg{*}
\Msg{* Happy TeXing!}
\Msg{*}
\Msg{************************************************************************}

\endbatchfile
%</install>
%<*ignore>
\fi
%</ignore>
%<*driver>
\NeedsTeXFormat{LaTeX2e}
\ProvidesFile{setouterhbox.drv}%
  [2016/05/16 v1.8 Set hbox in outer horizontal mode (HO)]%
\documentclass{ltxdoc}
\usepackage{holtxdoc}[2011/11/22]
\begin{document}
  \DocInput{setouterhbox.dtx}%
\end{document}
%</driver>
% \fi
%
%
% \CharacterTable
%  {Upper-case    \A\B\C\D\E\F\G\H\I\J\K\L\M\N\O\P\Q\R\S\T\U\V\W\X\Y\Z
%   Lower-case    \a\b\c\d\e\f\g\h\i\j\k\l\m\n\o\p\q\r\s\t\u\v\w\x\y\z
%   Digits        \0\1\2\3\4\5\6\7\8\9
%   Exclamation   \!     Double quote  \"     Hash (number) \#
%   Dollar        \$     Percent       \%     Ampersand     \&
%   Acute accent  \'     Left paren    \(     Right paren   \)
%   Asterisk      \*     Plus          \+     Comma         \,
%   Minus         \-     Point         \.     Solidus       \/
%   Colon         \:     Semicolon     \;     Less than     \<
%   Equals        \=     Greater than  \>     Question mark \?
%   Commercial at \@     Left bracket  \[     Backslash     \\
%   Right bracket \]     Circumflex    \^     Underscore    \_
%   Grave accent  \`     Left brace    \{     Vertical bar  \|
%   Right brace   \}     Tilde         \~}
%
% \GetFileInfo{setouterhbox.drv}
%
% \title{The \xpackage{setouterhbox} package}
% \date{2016/05/16 v1.8}
% \author{Heiko Oberdiek\thanks
% {Please report any issues at \url{https://github.com/ho-tex/oberdiek/issues}}}
%
% \maketitle
%
% \begin{abstract}
% If math stuff is set in an \cs{hbox}, then TeX
% performs some optimization and omits the implicite
% penalties \cs{binoppenalty} and \cs{relpenalty}.
% This packages tries to put stuff into an \cs{hbox}
% without getting lost of those penalties.
% \end{abstract}
%
% \tableofcontents
%
% \section{Documentation}
%
% \subsection{Introduction}
%
% There is a situation in \xpackage{hyperref}'s driver for dvips
% where the user wants to have links that can be broken across
% lines. However dvips doesn't support the feature. With option
% \xoption{breaklinks} \xpackage{hyperref} sets the links as
% usual, put them in a box and write the link data with
% box dimensions into the appropriate \cs{special}s.
% Then, however, it does not set the complete unbreakable
% box, but it unwrappes the material inside to allow line
% breaks. Of course line breaking and glue setting will falsify
% the link dimensions, but line breaking was more important
% for the user.
%
% \subsection{Acknowledgement}
%
% Jonathan Fine, Donald Arsenau and me discussed the problem
% in the newsgroup \xnewsgroup{comp.text.tex} where Damian
% Menscher has started the thread, see \cite{newsstart}.
%
% The discussion was productive and generated many ideas
% and code examples. In order to have a more permanent
% result I wrote this package and tried to implement
% most of the ideas, a kind of summary of the discussion.
% Thus I want and have to thank Jonathan Fine and Donald Arsenau
% very much.
%
% Two weeks later David Kastrup (posting in
% \xnewsgroup{comp.text.tex}, \cite{kastrup})
% remembered an old article of Michael Downes (\cite{downes})
% in TUGboat, where Michael Downes already presented the
% method we discuss here. Nowadays we have \eTeX\ that extends
% the tool set of a \TeX\ macro programmer. Especially useful
% \eTeX\ was in this package for detecting and dealing with
% errorneous situations.
%
% However also nowadays a perfect solution for the problem
% is still missing at macro level. Probably someone has
% to go deep in the internals of the \TeX\ compiler to
% implement a switch that let penalties stay where otherwise
% \TeX\ would remove them for optimization reasons.
%
% \subsection{Usage}
%
% \paragraph{Package loading.}
% \LaTeX: as usually:
% \begin{quote}
%   |\usepackage{setouterhbox}|
% \end{quote}
% The package can also be included directly, thus \plainTeX\ users
% write:
% \begin{quote}
%   |\input setouterhbox.sty|
% \end{quote}
%
% \paragraph{Register allocation.}
% The material will be put into a box, thus we need to know these
% box number. If you need to allocate a new box register:
% \begin{description}
%  \item[\LaTeX:] |\newsavebox{\|\meta{name}|}|
%  \item[\plainTeX:] |\newbox\|\meta{name}
% \end{description}
% Then |\|\meta{name} is a command that held the box number.
%
% \paragraph{Box wrapping.}
% \LaTeX\ users put the material in the box with an environment
% similar to \texttt{lrbox}. The environment \texttt{setouterhbox}
% uses the same syntax and offers the same features, such
% as verbatim stuff inside:
% \begin{quote}
%  |\begin{setouterhbox}{|\meta{box number}|}|\dots
%  |\end{setouterhbox}|
% \end{quote}
% Users with \plainTeX\ do not have environments, they use instead:
% \begin{quote}
%   |\setouterhbox{|\meta{box number}|}|\dots|\endsetouterhbox|
% \end{quote}
% In both cases the material is put into an \cs{hbox} and assigned
% to the given box, denoted by \meta{box number}. Note the
% assignment is local, the same way \texttt{lrbox} behaves.
%
% \paragraph{Unwrapping.}
% The box material is ready for unwrapping:
% \begin{quote}
%   |\unhbox|\meta{box number}
% \end{quote}
%
% \subsection{Option \xoption{hyperref}}
%
% Package url uses math mode for typesetting urls.
% Break points are inserted by \cs{binoppenalty} and
% \cs{relpenalty}. Unhappily these break points are
% removed, if \xpackage{hyperref}
% is used with option {breaklinks}
% and drivers that depend on \xoption{pdfmark}:
% \xoption{dvips}, \xoption{vtexpdfmark}, \xoption{textures},
% and \xoption{dvipsone}.
% Thus the option \xoption{hyperref} enables the method
% of this package to avoid the removal of \cs{relpenalty}
% and \cs{binoppenalty}. Thus you get more break points.
% However, the link areas are still wrong for these
% drivers, because they are not supporting broken
% links.
%
% Note, you need version 2006/08/16 v6.75c of package \xpackage{hyperref},
% because starting with this version the necessary hook is provided
% that package \xpackage{setouterhbox} uses.
% \begin{quote}
%   |\usepackage[|\dots|]{hyperref}[2006/08/16]|\\
%   |\usepackage[hyperref]{setouterhbox}|
% \end{quote}
% Package order does not matter.
%
% \subsection{Example}
%
%    \begin{macrocode}
%<*example>
\documentclass[a5paper]{article}
\usepackage{url}[2005/06/27]
\usepackage{setouterhbox}

\newsavebox{\testbox}

\setlength{\parindent}{0pt}
\setlength{\parskip}{2em}

\begin{document}
\raggedright

\url{http://this.is.a.very.long.host.name/followed/%
by/a/very_long_long_long_path.html}%

\sbox\testbox{%
  \url{http://this.is.a.very.long.host.name/followed/%
  by/a/very_long_long_long_path.html}%
}%
\unhbox\testbox

\begin{setouterhbox}{\testbox}%
  \url{http://this.is.a.very.long.host.name/followed/%
  by/a/very_long_long_long_path.html}%
\end{setouterhbox}
\unhbox\testbox

\end{document}
%</example>
%    \end{macrocode}
%
% \StopEventually{
% }
%
% \section{Implementation}
%
% Internal macros are prefixed by \cs{setouterhbox}, |@| is
% not used inside names, thus we do not need to care of its
% catcode if we are not using it as \LaTeX\ package.
%
% \subsection{Package start stuff}
%
%    \begin{macrocode}
%<*package>
%    \end{macrocode}
%
% Prevent reloading more than one, necessary for \plainTeX:
%    Reload check, especially if the package is not used with \LaTeX.
%    \begin{macrocode}
\begingroup\catcode61\catcode48\catcode32=10\relax%
  \catcode13=5 % ^^M
  \endlinechar=13 %
  \catcode35=6 % #
  \catcode39=12 % '
  \catcode44=12 % ,
  \catcode45=12 % -
  \catcode46=12 % .
  \catcode58=12 % :
  \catcode64=11 % @
  \catcode123=1 % {
  \catcode125=2 % }
  \expandafter\let\expandafter\x\csname ver@setouterhbox.sty\endcsname
  \ifx\x\relax % plain-TeX, first loading
  \else
    \def\empty{}%
    \ifx\x\empty % LaTeX, first loading,
      % variable is initialized, but \ProvidesPackage not yet seen
    \else
      \expandafter\ifx\csname PackageInfo\endcsname\relax
        \def\x#1#2{%
          \immediate\write-1{Package #1 Info: #2.}%
        }%
      \else
        \def\x#1#2{\PackageInfo{#1}{#2, stopped}}%
      \fi
      \x{setouterhbox}{The package is already loaded}%
      \aftergroup\endinput
    \fi
  \fi
\endgroup%
%    \end{macrocode}
%    Package identification:
%    \begin{macrocode}
\begingroup\catcode61\catcode48\catcode32=10\relax%
  \catcode13=5 % ^^M
  \endlinechar=13 %
  \catcode35=6 % #
  \catcode39=12 % '
  \catcode40=12 % (
  \catcode41=12 % )
  \catcode44=12 % ,
  \catcode45=12 % -
  \catcode46=12 % .
  \catcode47=12 % /
  \catcode58=12 % :
  \catcode64=11 % @
  \catcode91=12 % [
  \catcode93=12 % ]
  \catcode123=1 % {
  \catcode125=2 % }
  \expandafter\ifx\csname ProvidesPackage\endcsname\relax
    \def\x#1#2#3[#4]{\endgroup
      \immediate\write-1{Package: #3 #4}%
      \xdef#1{#4}%
    }%
  \else
    \def\x#1#2[#3]{\endgroup
      #2[{#3}]%
      \ifx#1\@undefined
        \xdef#1{#3}%
      \fi
      \ifx#1\relax
        \xdef#1{#3}%
      \fi
    }%
  \fi
\expandafter\x\csname ver@setouterhbox.sty\endcsname
\ProvidesPackage{setouterhbox}%
  [2016/05/16 v1.8 Set hbox in outer horizontal mode (HO)]%
%    \end{macrocode}
%
%    \begin{macrocode}
\begingroup\catcode61\catcode48\catcode32=10\relax%
  \catcode13=5 % ^^M
  \endlinechar=13 %
  \catcode123=1 % {
  \catcode125=2 % }
  \catcode64=11 % @
  \def\x{\endgroup
    \expandafter\edef\csname setouterhboxAtEnd\endcsname{%
      \endlinechar=\the\endlinechar\relax
      \catcode13=\the\catcode13\relax
      \catcode32=\the\catcode32\relax
      \catcode35=\the\catcode35\relax
      \catcode61=\the\catcode61\relax
      \catcode64=\the\catcode64\relax
      \catcode123=\the\catcode123\relax
      \catcode125=\the\catcode125\relax
    }%
  }%
\x\catcode61\catcode48\catcode32=10\relax%
\catcode13=5 % ^^M
\endlinechar=13 %
\catcode35=6 % #
\catcode64=11 % @
\catcode123=1 % {
\catcode125=2 % }
\def\TMP@EnsureCode#1#2{%
  \edef\setouterhboxAtEnd{%
    \setouterhboxAtEnd
    \catcode#1=\the\catcode#1\relax
  }%
  \catcode#1=#2\relax
}
\TMP@EnsureCode{40}{12}% (
\TMP@EnsureCode{41}{12}% )
\TMP@EnsureCode{44}{12}% ,
\TMP@EnsureCode{45}{12}% -
\TMP@EnsureCode{46}{12}% .
\TMP@EnsureCode{47}{12}% /
\TMP@EnsureCode{58}{12}% :
\TMP@EnsureCode{60}{12}% <
\TMP@EnsureCode{62}{12}% >
\TMP@EnsureCode{91}{12}% [
\TMP@EnsureCode{93}{12}% ]
\TMP@EnsureCode{96}{12}% `
\edef\setouterhboxAtEnd{\setouterhboxAtEnd\noexpand\endinput}
%    \end{macrocode}
%
% \subsection{Interface macros}
%
%    \begin{macro}{\setouterhboxBox}
% The method requires a global box assignment. To be on the
% safe side, a new box register is allocated for this
% global box assignment.
%    \begin{macrocode}
\newbox\setouterhboxBox
%    \end{macrocode}
%    \end{macro}
%
%    \begin{macro}{\setouterhboxFailure}
% Error message for both \plainTeX\ and \LaTeX
%    \begin{macrocode}
\begingroup\expandafter\expandafter\expandafter\endgroup
\expandafter\ifx\csname RequirePackage\endcsname\relax
  \input infwarerr.sty\relax
\else
  \RequirePackage{infwarerr}[2016/05/16]%
\fi
\edef\setouterhboxFailure#1#2{%
  \expandafter\noexpand\csname @PackageError\endcsname
      {setouterhbox}{#1}{#2}%
}
%    \end{macrocode}
%    \end{macro}
%
% \subsection{Main part}
%
% eTeX provides much better means for checking
% error conditions. Thus lines marked by "E" are executed
% if eTeX is available, otherwise the lines marked by "T" are
% used.
%    \begin{macrocode}
\begingroup\expandafter\expandafter\expandafter\endgroup
\expandafter\ifx\csname lastnodetype\endcsname\relax
  \catcode`T=9 % ignore
  \catcode`E=14 % comment
\else
  \catcode`T=14 % comment
  \catcode`E=9 % ignore
\fi
%    \end{macrocode}
%
%    \begin{macro}{\setouterhboxRemove}
% Remove all kern, glue, and penalty nodes;
% poor man's version, if \eTeX\ is not available
%    \begin{macrocode}
\def\setouterhboxRemove{%
E \ifnum\lastnodetype<11 %
E   \else
E   \ifnum\lastnodetype>13 %
E   \else
      \unskip\unkern\unpenalty
E     \expandafter\expandafter\expandafter\setouterhboxRemove
E   \fi
E \fi
}%
%    \end{macrocode}
%    \end{macro}
%
%    \begin{macro}{\setouterhbox}
% Passing the box contents by macro parameter would prevent
% catcode changes in the box contents like by \cs{verb}.
% Also \cs{bgroup} and \cs{egroup} does not work, because stuff
% has to be added at the begin and end of the box, thus
% the syntax
% |\setouterhbox{|\meta{box number}|}|\dots|\endsetouterhbox|
% is used. Also we automatically get an environment \texttt{setouterhbox}
% if \LaTeX\ is used.
%    \begin{macrocode}
\def\setouterhbox#1{%
  \begingroup
    \def\setouterhboxNum{#1}%
    \setbox0\vbox\bgroup
T     \kern.123pt\relax % marker
T     \kern0pt\relax % removed by \setouterhboxRemove
      \begingroup
        \everypar{}%
        \noindent
}
%    \end{macrocode}
%    \end{macro}
%    \begin{macro}{\endsetouterhbox}
% Most of the work is done in the end part, thus the heart of
% the method follows:
%    \begin{macrocode}
\def\endsetouterhbox{%
      \endgroup
%    \end{macrocode}
% Omit the first pass to get the penalties
% of the second pass.
%    \begin{macrocode}
      \pretolerance-1 %
%    \end{macrocode}
%  We don't want a third pass with \cs{emergencystretch}.
%    \begin{macrocode}
      \tolerance10000 %
      \hsize\maxdimen
%    \end{macrocode}
% Line is not underfull:
%    \begin{macrocode}
      \parfillskip 0pt plus 1filll\relax
      \leftskip0pt\relax
%    \end{macrocode}
% Suppress underful \cs{hbox} warnings,
% is explicit line breaks are used.
%    \begin{macrocode}
      \rightskip0pt plus 1fil\relax
      \everypar{}%
%    \end{macrocode}
% Ensure that there is a paragraph and
% prevents \cs{endgraph} from eating terminal glue:
%    \begin{macrocode}
      \kern0pt%
      \endgraf
      \setouterhboxRemove
E     \ifnum\lastnodetype=1 %
E       \global\setbox\setouterhboxBox\lastbox
E       \loop
E         \setouterhboxRemove
E       \ifnum\lastnodetype=1 %
E         \setbox0=\lastbox
E         \global\setbox\setouterhboxBox=\hbox{%
E           \unhbox0 %
%    \end{macrocode}
% Remove \cs{rightskip}, a penalty with -10000 is part of the previous line.
%    \begin{macrocode}
E           \unskip
E           \unhbox\setouterhboxBox
E         }%
E       \repeat
E     \else
E       \setouterhboxFailure{%
E         Something is wrong%
E       }{%
E         Could not find expected line.%
E         \MessageBreak
E         (\string\lastnodetype: \number\lastnodetype, expected: 1)%
E       }%
E     \fi
E     \setouterhboxRemove
T     \global\setbox\setouterhboxBox\lastbox
T     \loop
T       \setouterhboxRemove
T       \setbox0=\lastbox
T     \ifcase\ifvoid0 1\else0\fi
T       \global\setbox\setouterhboxBox=\hbox{%
T         \unhbox0 %
%    \end{macrocode}
% Remove \cs{rightskip}, a penalty with -10000 is part of the previous line.
%    \begin{macrocode}
T         \unskip
T         \unhbox\setouterhboxBox
T       }%
T     \repeat
T     \ifdim.123pt=\lastkern
T     \else
T       \setouterhboxFailure{%
T         Something is wrong%
T       }{%
T         Unexpected stuff was detected before the line.%
T       }%
T     \fi
T   \egroup
T   \ifcase \ifnum\wd0=0 \else 1\fi
T           \ifdim\ht0=.123pt \else 1\fi
T           \ifnum\dp0=0 \else 1\fi
T           0 %
E     \ifnum\lastnodetype=-1 %
%    \end{macrocode}
% There was just one line that we have caught.
%    \begin{macrocode}
      \else
        \setouterhboxFailure{%
            Something is wrong%
        }{%
            After fetching the line there is more unexpected stuff.%
E           \MessageBreak
E           (\string\lastnodetype: \number\lastnodetype, expected: -1)%
        }%
      \fi
E   \egroup
  \expandafter\endgroup
  \expandafter\setouterhboxFinish\expandafter{%
    \number\setouterhboxNum
  }%
}
%    \end{macrocode}
%    \end{macro}
%
% \subsection{Environment support}
%
% Check \cs{@currenvir} for the case that \cs{setouterhbox}
% was called as environment. Then the box assignment
% must be put after the \cs{endgroup} of |\end{|\dots|}|.
%    \begin{macrocode}
\def\setouterhboxCurr{setouterhbox}
\def\setouterhboxLast#1{%
  \setbox#1\hbox{%
    \unhbox\setouterhboxBox
    \unskip % remove \rightskip glue
    \unskip % remove \parfillskip glue
    \unpenalty % remove paragraph ending \penalty 10000
    \unkern % remove explicit kern inserted above
  }%
}
%    \end{macrocode}
%    \begin{macro}{\setouterhboxFinish}
% |#1| is an explicit number.
%    \begin{macrocode}
\def\setouterhboxFinish#1{%
  \begingroup\expandafter\expandafter\expandafter\endgroup
  \expandafter\ifx\csname @currenvir\endcsname\setouterhboxCurr
    \aftergroup\setouterhboxLast
    \aftergroup{%
    \setouterhboxAfter #1\NIL
    \aftergroup}%
  \else
    \setouterhboxLast{#1}%
  \fi
}
%    \end{macrocode}
%    \end{macro}
%    \begin{macro}{\setouterhboxAfter}
% |#1| is an explicit number.
%    \begin{macrocode}
\def\setouterhboxAfter#1#2\NIL{%
  \aftergroup#1%
  \ifx\\#2\\%
  \else
    \setouterhboxReturnAfterFi{%
      \setouterhboxAfter#2\NIL
    }%
  \fi
}
%    \end{macrocode}
%    \end{macro}
%    \begin{macro}{\setouterhboxReturnAfterFi}
% A utility macro to get tail recursion.
%    \begin{macrocode}
\long\def\setouterhboxReturnAfterFi#1\fi{\fi#1}
%    \end{macrocode}
%    \end{macro}
% Restore catcodes we have need to distinguish between
% the implementation with and without \eTeX.
%    \begin{macrocode}
\catcode69=11\relax % E
\catcode84=11\relax % T
%    \end{macrocode}
%
% \subsection{Option \xoption{hyperref}}
%    \begin{macrocode}
\begingroup
  \def\x{LaTeX2e}%
\expandafter\endgroup
\ifx\x\fmtname
\else
  \expandafter\setouterhboxAtEnd
\fi%
%    \end{macrocode}
%    \begin{macro}{\Hy@setouterhbox}
% \cs{Hy@setouterhbox} is the internal hook that \xpackage{hyperref}
% uses since 2006/02/12 v6.75a.
%    \begin{macrocode}
\DeclareOption{hyperref}{%
  \long\def\Hy@setouterhbox#1#2{%
    \setouterhbox{#1}#2\endsetouterhbox
  }%
}
%    \end{macrocode}
%    \end{macro}
%    \begin{macrocode}
\ProcessOptions\relax
%    \end{macrocode}
%
%    \begin{macrocode}
\setouterhboxAtEnd%
%</package>
%    \end{macrocode}
%
% \section{Test}
%
% \subsection{Catcode checks for loading}
%
%    \begin{macrocode}
%<*test1>
%    \end{macrocode}
%    \begin{macrocode}
\catcode`\{=1 %
\catcode`\}=2 %
\catcode`\#=6 %
\catcode`\@=11 %
\expandafter\ifx\csname count@\endcsname\relax
  \countdef\count@=255 %
\fi
\expandafter\ifx\csname @gobble\endcsname\relax
  \long\def\@gobble#1{}%
\fi
\expandafter\ifx\csname @firstofone\endcsname\relax
  \long\def\@firstofone#1{#1}%
\fi
\expandafter\ifx\csname loop\endcsname\relax
  \expandafter\@firstofone
\else
  \expandafter\@gobble
\fi
{%
  \def\loop#1\repeat{%
    \def\body{#1}%
    \iterate
  }%
  \def\iterate{%
    \body
      \let\next\iterate
    \else
      \let\next\relax
    \fi
    \next
  }%
  \let\repeat=\fi
}%
\def\RestoreCatcodes{}
\count@=0 %
\loop
  \edef\RestoreCatcodes{%
    \RestoreCatcodes
    \catcode\the\count@=\the\catcode\count@\relax
  }%
\ifnum\count@<255 %
  \advance\count@ 1 %
\repeat

\def\RangeCatcodeInvalid#1#2{%
  \count@=#1\relax
  \loop
    \catcode\count@=15 %
  \ifnum\count@<#2\relax
    \advance\count@ 1 %
  \repeat
}
\def\RangeCatcodeCheck#1#2#3{%
  \count@=#1\relax
  \loop
    \ifnum#3=\catcode\count@
    \else
      \errmessage{%
        Character \the\count@\space
        with wrong catcode \the\catcode\count@\space
        instead of \number#3%
      }%
    \fi
  \ifnum\count@<#2\relax
    \advance\count@ 1 %
  \repeat
}
\def\space{ }
\expandafter\ifx\csname LoadCommand\endcsname\relax
  \def\LoadCommand{\input setouterhbox.sty\relax}%
\fi
\def\Test{%
  \RangeCatcodeInvalid{0}{47}%
  \RangeCatcodeInvalid{58}{64}%
  \RangeCatcodeInvalid{91}{96}%
  \RangeCatcodeInvalid{123}{255}%
  \catcode`\@=12 %
  \catcode`\\=0 %
  \catcode`\%=14 %
  \LoadCommand
  \RangeCatcodeCheck{0}{36}{15}%
  \RangeCatcodeCheck{37}{37}{14}%
  \RangeCatcodeCheck{38}{47}{15}%
  \RangeCatcodeCheck{48}{57}{12}%
  \RangeCatcodeCheck{58}{63}{15}%
  \RangeCatcodeCheck{64}{64}{12}%
  \RangeCatcodeCheck{65}{90}{11}%
  \RangeCatcodeCheck{91}{91}{15}%
  \RangeCatcodeCheck{92}{92}{0}%
  \RangeCatcodeCheck{93}{96}{15}%
  \RangeCatcodeCheck{97}{122}{11}%
  \RangeCatcodeCheck{123}{255}{15}%
  \RestoreCatcodes
}
\Test
\csname @@end\endcsname
\end
%    \end{macrocode}
%    \begin{macrocode}
%</test1>
%    \end{macrocode}
%
% \subsection{Test with package \xpackage{url}}
%
%    \begin{macrocode}
%<*test2>
\nofiles
\documentclass[a5paper]{article}
\usepackage{url}[2005/06/27]
\usepackage{setouterhbox}

\newsavebox{\testbox}

\setlength{\parindent}{0pt}
\setlength{\parskip}{2em}

\begin{document}
\raggedright

\url{http://this.is.a.very.long.host.name/followed/%
by/a/very_long_long_long_path.html}%

\sbox\testbox{%
  \url{http://this.is.a.very.long.host.name/followed/%
  by/a/very_long_long_long_path.html}%
}%
\unhbox\testbox

\begin{setouterhbox}{\testbox}%
  \url{http://this.is.a.very.long.host.name/followed/%
  by/a/very_long_long_long_path.html}%
\end{setouterhbox}
\unhbox\testbox

\end{document}
%</test2>
%    \end{macrocode}
%
% \section{Installation}
%
% \subsection{Download}
%
% \paragraph{Package.} This package is available on
% CTAN\footnote{\CTANpkg{setouterhbox}}:
% \begin{description}
% \item[\CTAN{macros/latex/contrib/oberdiek/setouterhbox.dtx}] The source file.
% \item[\CTAN{macros/latex/contrib/oberdiek/setouterhbox.pdf}] Documentation.
% \end{description}
%
%
% \paragraph{Bundle.} All the packages of the bundle `oberdiek'
% are also available in a TDS compliant ZIP archive. There
% the packages are already unpacked and the documentation files
% are generated. The files and directories obey the TDS standard.
% \begin{description}
% \item[\CTANinstall{install/macros/latex/contrib/oberdiek.tds.zip}]
% \end{description}
% \emph{TDS} refers to the standard ``A Directory Structure
% for \TeX\ Files'' (\CTAN{tds/tds.pdf}). Directories
% with \xfile{texmf} in their name are usually organized this way.
%
% \subsection{Bundle installation}
%
% \paragraph{Unpacking.} Unpack the \xfile{oberdiek.tds.zip} in the
% TDS tree (also known as \xfile{texmf} tree) of your choice.
% Example (linux):
% \begin{quote}
%   |unzip oberdiek.tds.zip -d ~/texmf|
% \end{quote}
%
% \subsection{Package installation}
%
% \paragraph{Unpacking.} The \xfile{.dtx} file is a self-extracting
% \docstrip\ archive. The files are extracted by running the
% \xfile{.dtx} through \plainTeX:
% \begin{quote}
%   \verb|tex setouterhbox.dtx|
% \end{quote}
%
% \paragraph{TDS.} Now the different files must be moved into
% the different directories in your installation TDS tree
% (also known as \xfile{texmf} tree):
% \begin{quote}
% \def\t{^^A
% \begin{tabular}{@{}>{\ttfamily}l@{ $\rightarrow$ }>{\ttfamily}l@{}}
%   setouterhbox.sty & tex/generic/oberdiek/setouterhbox.sty\\
%   setouterhbox.pdf & doc/latex/oberdiek/setouterhbox.pdf\\
%   setouterhbox-example.tex & doc/latex/oberdiek/setouterhbox-example.tex\\
%   test/setouterhbox-test1.tex & doc/latex/oberdiek/test/setouterhbox-test1.tex\\
%   test/setouterhbox-test2.tex & doc/latex/oberdiek/test/setouterhbox-test2.tex\\
%   setouterhbox.dtx & source/latex/oberdiek/setouterhbox.dtx\\
% \end{tabular}^^A
% }^^A
% \sbox0{\t}^^A
% \ifdim\wd0>\linewidth
%   \begingroup
%     \advance\linewidth by\leftmargin
%     \advance\linewidth by\rightmargin
%   \edef\x{\endgroup
%     \def\noexpand\lw{\the\linewidth}^^A
%   }\x
%   \def\lwbox{^^A
%     \leavevmode
%     \hbox to \linewidth{^^A
%       \kern-\leftmargin\relax
%       \hss
%       \usebox0
%       \hss
%       \kern-\rightmargin\relax
%     }^^A
%   }^^A
%   \ifdim\wd0>\lw
%     \sbox0{\small\t}^^A
%     \ifdim\wd0>\linewidth
%       \ifdim\wd0>\lw
%         \sbox0{\footnotesize\t}^^A
%         \ifdim\wd0>\linewidth
%           \ifdim\wd0>\lw
%             \sbox0{\scriptsize\t}^^A
%             \ifdim\wd0>\linewidth
%               \ifdim\wd0>\lw
%                 \sbox0{\tiny\t}^^A
%                 \ifdim\wd0>\linewidth
%                   \lwbox
%                 \else
%                   \usebox0
%                 \fi
%               \else
%                 \lwbox
%               \fi
%             \else
%               \usebox0
%             \fi
%           \else
%             \lwbox
%           \fi
%         \else
%           \usebox0
%         \fi
%       \else
%         \lwbox
%       \fi
%     \else
%       \usebox0
%     \fi
%   \else
%     \lwbox
%   \fi
% \else
%   \usebox0
% \fi
% \end{quote}
% If you have a \xfile{docstrip.cfg} that configures and enables \docstrip's
% TDS installing feature, then some files can already be in the right
% place, see the documentation of \docstrip.
%
% \subsection{Refresh file name databases}
%
% If your \TeX~distribution
% (\TeX\,Live, \mikTeX, \dots) relies on file name databases, you must refresh
% these. For example, \TeX\,Live\ users run \verb|texhash| or
% \verb|mktexlsr|.
%
% \subsection{Some details for the interested}
%
% \paragraph{Unpacking with \LaTeX.}
% The \xfile{.dtx} chooses its action depending on the format:
% \begin{description}
% \item[\plainTeX:] Run \docstrip\ and extract the files.
% \item[\LaTeX:] Generate the documentation.
% \end{description}
% If you insist on using \LaTeX\ for \docstrip\ (really,
% \docstrip\ does not need \LaTeX), then inform the autodetect routine
% about your intention:
% \begin{quote}
%   \verb|latex \let\install=y\input{setouterhbox.dtx}|
% \end{quote}
% Do not forget to quote the argument according to the demands
% of your shell.
%
% \paragraph{Generating the documentation.}
% You can use both the \xfile{.dtx} or the \xfile{.drv} to generate
% the documentation. The process can be configured by the
% configuration file \xfile{ltxdoc.cfg}. For instance, put this
% line into this file, if you want to have A4 as paper format:
% \begin{quote}
%   \verb|\PassOptionsToClass{a4paper}{article}|
% \end{quote}
% An example follows how to generate the
% documentation with pdf\LaTeX:
% \begin{quote}
%\begin{verbatim}
%pdflatex setouterhbox.dtx
%makeindex -s gind.ist setouterhbox.idx
%pdflatex setouterhbox.dtx
%makeindex -s gind.ist setouterhbox.idx
%pdflatex setouterhbox.dtx
%\end{verbatim}
% \end{quote}
%
% \begin{thebibliography}{9}
%
% \bibitem{newsstart}
%   Damian Menscher, \Newsgroup{comp.text.tex},
%   \textit{overlong lines in List of Figures},
%   \nolinkurl{<dh058t$qbd$1@news.ks.uiuc.edu>},
%   23rd September 2005.
%   \url{https://groups.google.com/group/comp.text.tex/msg/79648d4cf1f8bc13}
%
% \bibitem{kastrup}
%   David Kastrup, \Newsgroup{comp.text.tex},
%   \textit{Re: ANN: outerhbox.sty -- collect horizontal material,
%   for unboxing into a paragraph},
%   \nolinkurl{<85y855lrx3.fsf@lola.goethe.zz>},
%   7th October 2005.
%   \url{https://groups.google.com/group/comp.text.tex/msg/7cf0a345ef932e52}
%
% \bibitem{downes}
%   Michael Downes, \textit{Line breaking in \cs{unhbox}ed Text},
%   TUGboat 11 (1990), pp. 605--612.
%
% \bibitem{hyperref}
%   Sebastian Rahtz, Heiko Oberdiek:
%   \textit{The \xpackage{hyperref} package};
%   2006/08/16 v6.75c;
%   \CTANpkg{hyperref}.
%
% \end{thebibliography}
%
% \begin{History}
%   \begin{Version}{2005/10/05 v1.0}
%   \item
%     First version.
%   \end{Version}
%   \begin{Version}{2005/10/07 v1.1}
%   \item
%     Option \xoption{hyperref} added.
%   \end{Version}
%   \begin{Version}{2005/10/18 v1.2}
%   \item
%     Support for explicit line breaks added.
%   \end{Version}
%   \begin{Version}{2006/02/12 v1.3}
%   \item
%     DTX format.
%   \item
%     Documentation extended.
%   \end{Version}
%   \begin{Version}{2006/08/26 v1.4}
%   \item
%     Date of hyperref updated.
%   \end{Version}
%   \begin{Version}{2007/04/26 v1.5}
%   \item
%     Use of package \xpackage{infwarerr}.
%   \end{Version}
%   \begin{Version}{2007/05/17 v1.6}
%   \item
%     Standard header part for generic files.
%   \end{Version}
%   \begin{Version}{2007/09/09 v1.7}
%   \item
%     Catcode section added.
%   \end{Version}
%   \begin{Version}{2016/05/16 v1.8}
%   \item
%     Documentation updates.
%   \end{Version}
% \end{History}
%
% \PrintIndex
%
% \Finale
\endinput
|
% \end{quote}
% Do not forget to quote the argument according to the demands
% of your shell.
%
% \paragraph{Generating the documentation.}
% You can use both the \xfile{.dtx} or the \xfile{.drv} to generate
% the documentation. The process can be configured by the
% configuration file \xfile{ltxdoc.cfg}. For instance, put this
% line into this file, if you want to have A4 as paper format:
% \begin{quote}
%   \verb|\PassOptionsToClass{a4paper}{article}|
% \end{quote}
% An example follows how to generate the
% documentation with pdf\LaTeX:
% \begin{quote}
%\begin{verbatim}
%pdflatex setouterhbox.dtx
%makeindex -s gind.ist setouterhbox.idx
%pdflatex setouterhbox.dtx
%makeindex -s gind.ist setouterhbox.idx
%pdflatex setouterhbox.dtx
%\end{verbatim}
% \end{quote}
%
% \begin{thebibliography}{9}
%
% \bibitem{newsstart}
%   Damian Menscher, \Newsgroup{comp.text.tex},
%   \textit{overlong lines in List of Figures},
%   \nolinkurl{<dh058t$qbd$1@news.ks.uiuc.edu>},
%   23rd September 2005.
%   \url{https://groups.google.com/group/comp.text.tex/msg/79648d4cf1f8bc13}
%
% \bibitem{kastrup}
%   David Kastrup, \Newsgroup{comp.text.tex},
%   \textit{Re: ANN: outerhbox.sty -- collect horizontal material,
%   for unboxing into a paragraph},
%   \nolinkurl{<85y855lrx3.fsf@lola.goethe.zz>},
%   7th October 2005.
%   \url{https://groups.google.com/group/comp.text.tex/msg/7cf0a345ef932e52}
%
% \bibitem{downes}
%   Michael Downes, \textit{Line breaking in \cs{unhbox}ed Text},
%   TUGboat 11 (1990), pp. 605--612.
%
% \bibitem{hyperref}
%   Sebastian Rahtz, Heiko Oberdiek:
%   \textit{The \xpackage{hyperref} package};
%   2006/08/16 v6.75c;
%   \CTANpkg{hyperref}.
%
% \end{thebibliography}
%
% \begin{History}
%   \begin{Version}{2005/10/05 v1.0}
%   \item
%     First version.
%   \end{Version}
%   \begin{Version}{2005/10/07 v1.1}
%   \item
%     Option \xoption{hyperref} added.
%   \end{Version}
%   \begin{Version}{2005/10/18 v1.2}
%   \item
%     Support for explicit line breaks added.
%   \end{Version}
%   \begin{Version}{2006/02/12 v1.3}
%   \item
%     DTX format.
%   \item
%     Documentation extended.
%   \end{Version}
%   \begin{Version}{2006/08/26 v1.4}
%   \item
%     Date of hyperref updated.
%   \end{Version}
%   \begin{Version}{2007/04/26 v1.5}
%   \item
%     Use of package \xpackage{infwarerr}.
%   \end{Version}
%   \begin{Version}{2007/05/17 v1.6}
%   \item
%     Standard header part for generic files.
%   \end{Version}
%   \begin{Version}{2007/09/09 v1.7}
%   \item
%     Catcode section added.
%   \end{Version}
%   \begin{Version}{2016/05/16 v1.8}
%   \item
%     Documentation updates.
%   \end{Version}
% \end{History}
%
% \PrintIndex
%
% \Finale
\endinput

%        (quote the arguments according to the demands of your shell)
%
% Documentation:
%    (a) If setouterhbox.drv is present:
%           latex setouterhbox.drv
%    (b) Without setouterhbox.drv:
%           latex setouterhbox.dtx; ...
%    The class ltxdoc loads the configuration file ltxdoc.cfg
%    if available. Here you can specify further options, e.g.
%    use A4 as paper format:
%       \PassOptionsToClass{a4paper}{article}
%
%    Programm calls to get the documentation (example):
%       pdflatex setouterhbox.dtx
%       makeindex -s gind.ist setouterhbox.idx
%       pdflatex setouterhbox.dtx
%       makeindex -s gind.ist setouterhbox.idx
%       pdflatex setouterhbox.dtx
%
% Installation:
%    TDS:tex/generic/oberdiek/setouterhbox.sty
%    TDS:doc/latex/oberdiek/setouterhbox.pdf
%    TDS:doc/latex/oberdiek/setouterhbox-example.tex
%    TDS:doc/latex/oberdiek/test/setouterhbox-test1.tex
%    TDS:doc/latex/oberdiek/test/setouterhbox-test2.tex
%    TDS:source/latex/oberdiek/setouterhbox.dtx
%
%<*ignore>
\begingroup
  \catcode123=1 %
  \catcode125=2 %
  \def\x{LaTeX2e}%
\expandafter\endgroup
\ifcase 0\ifx\install y1\fi\expandafter
         \ifx\csname processbatchFile\endcsname\relax\else1\fi
         \ifx\fmtname\x\else 1\fi\relax
\else\csname fi\endcsname
%</ignore>
%<*install>
\input docstrip.tex
\Msg{************************************************************************}
\Msg{* Installation}
\Msg{* Package: setouterhbox 2016/05/16 v1.8 Set hbox in outer horizontal mode (HO)}
\Msg{************************************************************************}

\keepsilent
\askforoverwritefalse

\let\MetaPrefix\relax
\preamble

This is a generated file.

Project: setouterhbox
Version: 2016/05/16 v1.8

Copyright (C)
   2005-2007 Heiko Oberdiek
   2016-2019 Oberdiek Package Support Group

This work may be distributed and/or modified under the
conditions of the LaTeX Project Public License, either
version 1.3c of this license or (at your option) any later
version. This version of this license is in
   https://www.latex-project.org/lppl/lppl-1-3c.txt
and the latest version of this license is in
   https://www.latex-project.org/lppl.txt
and version 1.3 or later is part of all distributions of
LaTeX version 2005/12/01 or later.

This work has the LPPL maintenance status "maintained".

The Current Maintainers of this work are
Heiko Oberdiek and the Oberdiek Package Support Group
https://github.com/ho-tex/oberdiek/issues


The Base Interpreter refers to any `TeX-Format',
because some files are installed in TDS:tex/generic//.

This work consists of the main source file setouterhbox.dtx
and the derived files
   setouterhbox.sty, setouterhbox.pdf, setouterhbox.ins, setouterhbox.drv,
   setouterhbox-example.tex, setouterhbox-test1.tex,
   setouterhbox-test2.tex.

\endpreamble
\let\MetaPrefix\DoubleperCent

\generate{%
  \file{setouterhbox.ins}{\from{setouterhbox.dtx}{install}}%
  \file{setouterhbox.drv}{\from{setouterhbox.dtx}{driver}}%
  \usedir{tex/generic/oberdiek}%
  \file{setouterhbox.sty}{\from{setouterhbox.dtx}{package}}%
  \usedir{doc/latex/oberdiek}%
  \file{setouterhbox-example.tex}{\from{setouterhbox.dtx}{example}}%
%  \usedir{doc/latex/oberdiek/test}%
%  \file{setouterhbox-test1.tex}{\from{setouterhbox.dtx}{test1}}%
%  \file{setouterhbox-test2.tex}{\from{setouterhbox.dtx}{test2}}%
  \nopreamble
  \nopostamble
%  \usedir{source/latex/oberdiek/catalogue}%
%  \file{setouterhbox.xml}{\from{setouterhbox.dtx}{catalogue}}%
}

\catcode32=13\relax% active space
\let =\space%
\Msg{************************************************************************}
\Msg{*}
\Msg{* To finish the installation you have to move the following}
\Msg{* file into a directory searched by TeX:}
\Msg{*}
\Msg{*     setouterhbox.sty}
\Msg{*}
\Msg{* To produce the documentation run the file `setouterhbox.drv'}
\Msg{* through LaTeX.}
\Msg{*}
\Msg{* Happy TeXing!}
\Msg{*}
\Msg{************************************************************************}

\endbatchfile
%</install>
%<*ignore>
\fi
%</ignore>
%<*driver>
\NeedsTeXFormat{LaTeX2e}
\ProvidesFile{setouterhbox.drv}%
  [2016/05/16 v1.8 Set hbox in outer horizontal mode (HO)]%
\documentclass{ltxdoc}
\usepackage{holtxdoc}[2011/11/22]
\begin{document}
  \DocInput{setouterhbox.dtx}%
\end{document}
%</driver>
% \fi
%
%
% \CharacterTable
%  {Upper-case    \A\B\C\D\E\F\G\H\I\J\K\L\M\N\O\P\Q\R\S\T\U\V\W\X\Y\Z
%   Lower-case    \a\b\c\d\e\f\g\h\i\j\k\l\m\n\o\p\q\r\s\t\u\v\w\x\y\z
%   Digits        \0\1\2\3\4\5\6\7\8\9
%   Exclamation   \!     Double quote  \"     Hash (number) \#
%   Dollar        \$     Percent       \%     Ampersand     \&
%   Acute accent  \'     Left paren    \(     Right paren   \)
%   Asterisk      \*     Plus          \+     Comma         \,
%   Minus         \-     Point         \.     Solidus       \/
%   Colon         \:     Semicolon     \;     Less than     \<
%   Equals        \=     Greater than  \>     Question mark \?
%   Commercial at \@     Left bracket  \[     Backslash     \\
%   Right bracket \]     Circumflex    \^     Underscore    \_
%   Grave accent  \`     Left brace    \{     Vertical bar  \|
%   Right brace   \}     Tilde         \~}
%
% \GetFileInfo{setouterhbox.drv}
%
% \title{The \xpackage{setouterhbox} package}
% \date{2016/05/16 v1.8}
% \author{Heiko Oberdiek\thanks
% {Please report any issues at \url{https://github.com/ho-tex/oberdiek/issues}}}
%
% \maketitle
%
% \begin{abstract}
% If math stuff is set in an \cs{hbox}, then TeX
% performs some optimization and omits the implicite
% penalties \cs{binoppenalty} and \cs{relpenalty}.
% This packages tries to put stuff into an \cs{hbox}
% without getting lost of those penalties.
% \end{abstract}
%
% \tableofcontents
%
% \section{Documentation}
%
% \subsection{Introduction}
%
% There is a situation in \xpackage{hyperref}'s driver for dvips
% where the user wants to have links that can be broken across
% lines. However dvips doesn't support the feature. With option
% \xoption{breaklinks} \xpackage{hyperref} sets the links as
% usual, put them in a box and write the link data with
% box dimensions into the appropriate \cs{special}s.
% Then, however, it does not set the complete unbreakable
% box, but it unwrappes the material inside to allow line
% breaks. Of course line breaking and glue setting will falsify
% the link dimensions, but line breaking was more important
% for the user.
%
% \subsection{Acknowledgement}
%
% Jonathan Fine, Donald Arsenau and me discussed the problem
% in the newsgroup \xnewsgroup{comp.text.tex} where Damian
% Menscher has started the thread, see \cite{newsstart}.
%
% The discussion was productive and generated many ideas
% and code examples. In order to have a more permanent
% result I wrote this package and tried to implement
% most of the ideas, a kind of summary of the discussion.
% Thus I want and have to thank Jonathan Fine and Donald Arsenau
% very much.
%
% Two weeks later David Kastrup (posting in
% \xnewsgroup{comp.text.tex}, \cite{kastrup})
% remembered an old article of Michael Downes (\cite{downes})
% in TUGboat, where Michael Downes already presented the
% method we discuss here. Nowadays we have \eTeX\ that extends
% the tool set of a \TeX\ macro programmer. Especially useful
% \eTeX\ was in this package for detecting and dealing with
% errorneous situations.
%
% However also nowadays a perfect solution for the problem
% is still missing at macro level. Probably someone has
% to go deep in the internals of the \TeX\ compiler to
% implement a switch that let penalties stay where otherwise
% \TeX\ would remove them for optimization reasons.
%
% \subsection{Usage}
%
% \paragraph{Package loading.}
% \LaTeX: as usually:
% \begin{quote}
%   |\usepackage{setouterhbox}|
% \end{quote}
% The package can also be included directly, thus \plainTeX\ users
% write:
% \begin{quote}
%   |\input setouterhbox.sty|
% \end{quote}
%
% \paragraph{Register allocation.}
% The material will be put into a box, thus we need to know these
% box number. If you need to allocate a new box register:
% \begin{description}
%  \item[\LaTeX:] |\newsavebox{\|\meta{name}|}|
%  \item[\plainTeX:] |\newbox\|\meta{name}
% \end{description}
% Then |\|\meta{name} is a command that held the box number.
%
% \paragraph{Box wrapping.}
% \LaTeX\ users put the material in the box with an environment
% similar to \texttt{lrbox}. The environment \texttt{setouterhbox}
% uses the same syntax and offers the same features, such
% as verbatim stuff inside:
% \begin{quote}
%  |\begin{setouterhbox}{|\meta{box number}|}|\dots
%  |\end{setouterhbox}|
% \end{quote}
% Users with \plainTeX\ do not have environments, they use instead:
% \begin{quote}
%   |\setouterhbox{|\meta{box number}|}|\dots|\endsetouterhbox|
% \end{quote}
% In both cases the material is put into an \cs{hbox} and assigned
% to the given box, denoted by \meta{box number}. Note the
% assignment is local, the same way \texttt{lrbox} behaves.
%
% \paragraph{Unwrapping.}
% The box material is ready for unwrapping:
% \begin{quote}
%   |\unhbox|\meta{box number}
% \end{quote}
%
% \subsection{Option \xoption{hyperref}}
%
% Package url uses math mode for typesetting urls.
% Break points are inserted by \cs{binoppenalty} and
% \cs{relpenalty}. Unhappily these break points are
% removed, if \xpackage{hyperref}
% is used with option {breaklinks}
% and drivers that depend on \xoption{pdfmark}:
% \xoption{dvips}, \xoption{vtexpdfmark}, \xoption{textures},
% and \xoption{dvipsone}.
% Thus the option \xoption{hyperref} enables the method
% of this package to avoid the removal of \cs{relpenalty}
% and \cs{binoppenalty}. Thus you get more break points.
% However, the link areas are still wrong for these
% drivers, because they are not supporting broken
% links.
%
% Note, you need version 2006/08/16 v6.75c of package \xpackage{hyperref},
% because starting with this version the necessary hook is provided
% that package \xpackage{setouterhbox} uses.
% \begin{quote}
%   |\usepackage[|\dots|]{hyperref}[2006/08/16]|\\
%   |\usepackage[hyperref]{setouterhbox}|
% \end{quote}
% Package order does not matter.
%
% \subsection{Example}
%
%    \begin{macrocode}
%<*example>
\documentclass[a5paper]{article}
\usepackage{url}[2005/06/27]
\usepackage{setouterhbox}

\newsavebox{\testbox}

\setlength{\parindent}{0pt}
\setlength{\parskip}{2em}

\begin{document}
\raggedright

\url{http://this.is.a.very.long.host.name/followed/%
by/a/very_long_long_long_path.html}%

\sbox\testbox{%
  \url{http://this.is.a.very.long.host.name/followed/%
  by/a/very_long_long_long_path.html}%
}%
\unhbox\testbox

\begin{setouterhbox}{\testbox}%
  \url{http://this.is.a.very.long.host.name/followed/%
  by/a/very_long_long_long_path.html}%
\end{setouterhbox}
\unhbox\testbox

\end{document}
%</example>
%    \end{macrocode}
%
% \StopEventually{
% }
%
% \section{Implementation}
%
% Internal macros are prefixed by \cs{setouterhbox}, |@| is
% not used inside names, thus we do not need to care of its
% catcode if we are not using it as \LaTeX\ package.
%
% \subsection{Package start stuff}
%
%    \begin{macrocode}
%<*package>
%    \end{macrocode}
%
% Prevent reloading more than one, necessary for \plainTeX:
%    Reload check, especially if the package is not used with \LaTeX.
%    \begin{macrocode}
\begingroup\catcode61\catcode48\catcode32=10\relax%
  \catcode13=5 % ^^M
  \endlinechar=13 %
  \catcode35=6 % #
  \catcode39=12 % '
  \catcode44=12 % ,
  \catcode45=12 % -
  \catcode46=12 % .
  \catcode58=12 % :
  \catcode64=11 % @
  \catcode123=1 % {
  \catcode125=2 % }
  \expandafter\let\expandafter\x\csname ver@setouterhbox.sty\endcsname
  \ifx\x\relax % plain-TeX, first loading
  \else
    \def\empty{}%
    \ifx\x\empty % LaTeX, first loading,
      % variable is initialized, but \ProvidesPackage not yet seen
    \else
      \expandafter\ifx\csname PackageInfo\endcsname\relax
        \def\x#1#2{%
          \immediate\write-1{Package #1 Info: #2.}%
        }%
      \else
        \def\x#1#2{\PackageInfo{#1}{#2, stopped}}%
      \fi
      \x{setouterhbox}{The package is already loaded}%
      \aftergroup\endinput
    \fi
  \fi
\endgroup%
%    \end{macrocode}
%    Package identification:
%    \begin{macrocode}
\begingroup\catcode61\catcode48\catcode32=10\relax%
  \catcode13=5 % ^^M
  \endlinechar=13 %
  \catcode35=6 % #
  \catcode39=12 % '
  \catcode40=12 % (
  \catcode41=12 % )
  \catcode44=12 % ,
  \catcode45=12 % -
  \catcode46=12 % .
  \catcode47=12 % /
  \catcode58=12 % :
  \catcode64=11 % @
  \catcode91=12 % [
  \catcode93=12 % ]
  \catcode123=1 % {
  \catcode125=2 % }
  \expandafter\ifx\csname ProvidesPackage\endcsname\relax
    \def\x#1#2#3[#4]{\endgroup
      \immediate\write-1{Package: #3 #4}%
      \xdef#1{#4}%
    }%
  \else
    \def\x#1#2[#3]{\endgroup
      #2[{#3}]%
      \ifx#1\@undefined
        \xdef#1{#3}%
      \fi
      \ifx#1\relax
        \xdef#1{#3}%
      \fi
    }%
  \fi
\expandafter\x\csname ver@setouterhbox.sty\endcsname
\ProvidesPackage{setouterhbox}%
  [2016/05/16 v1.8 Set hbox in outer horizontal mode (HO)]%
%    \end{macrocode}
%
%    \begin{macrocode}
\begingroup\catcode61\catcode48\catcode32=10\relax%
  \catcode13=5 % ^^M
  \endlinechar=13 %
  \catcode123=1 % {
  \catcode125=2 % }
  \catcode64=11 % @
  \def\x{\endgroup
    \expandafter\edef\csname setouterhboxAtEnd\endcsname{%
      \endlinechar=\the\endlinechar\relax
      \catcode13=\the\catcode13\relax
      \catcode32=\the\catcode32\relax
      \catcode35=\the\catcode35\relax
      \catcode61=\the\catcode61\relax
      \catcode64=\the\catcode64\relax
      \catcode123=\the\catcode123\relax
      \catcode125=\the\catcode125\relax
    }%
  }%
\x\catcode61\catcode48\catcode32=10\relax%
\catcode13=5 % ^^M
\endlinechar=13 %
\catcode35=6 % #
\catcode64=11 % @
\catcode123=1 % {
\catcode125=2 % }
\def\TMP@EnsureCode#1#2{%
  \edef\setouterhboxAtEnd{%
    \setouterhboxAtEnd
    \catcode#1=\the\catcode#1\relax
  }%
  \catcode#1=#2\relax
}
\TMP@EnsureCode{40}{12}% (
\TMP@EnsureCode{41}{12}% )
\TMP@EnsureCode{44}{12}% ,
\TMP@EnsureCode{45}{12}% -
\TMP@EnsureCode{46}{12}% .
\TMP@EnsureCode{47}{12}% /
\TMP@EnsureCode{58}{12}% :
\TMP@EnsureCode{60}{12}% <
\TMP@EnsureCode{62}{12}% >
\TMP@EnsureCode{91}{12}% [
\TMP@EnsureCode{93}{12}% ]
\TMP@EnsureCode{96}{12}% `
\edef\setouterhboxAtEnd{\setouterhboxAtEnd\noexpand\endinput}
%    \end{macrocode}
%
% \subsection{Interface macros}
%
%    \begin{macro}{\setouterhboxBox}
% The method requires a global box assignment. To be on the
% safe side, a new box register is allocated for this
% global box assignment.
%    \begin{macrocode}
\newbox\setouterhboxBox
%    \end{macrocode}
%    \end{macro}
%
%    \begin{macro}{\setouterhboxFailure}
% Error message for both \plainTeX\ and \LaTeX
%    \begin{macrocode}
\begingroup\expandafter\expandafter\expandafter\endgroup
\expandafter\ifx\csname RequirePackage\endcsname\relax
  \input infwarerr.sty\relax
\else
  \RequirePackage{infwarerr}[2016/05/16]%
\fi
\edef\setouterhboxFailure#1#2{%
  \expandafter\noexpand\csname @PackageError\endcsname
      {setouterhbox}{#1}{#2}%
}
%    \end{macrocode}
%    \end{macro}
%
% \subsection{Main part}
%
% eTeX provides much better means for checking
% error conditions. Thus lines marked by "E" are executed
% if eTeX is available, otherwise the lines marked by "T" are
% used.
%    \begin{macrocode}
\begingroup\expandafter\expandafter\expandafter\endgroup
\expandafter\ifx\csname lastnodetype\endcsname\relax
  \catcode`T=9 % ignore
  \catcode`E=14 % comment
\else
  \catcode`T=14 % comment
  \catcode`E=9 % ignore
\fi
%    \end{macrocode}
%
%    \begin{macro}{\setouterhboxRemove}
% Remove all kern, glue, and penalty nodes;
% poor man's version, if \eTeX\ is not available
%    \begin{macrocode}
\def\setouterhboxRemove{%
E \ifnum\lastnodetype<11 %
E   \else
E   \ifnum\lastnodetype>13 %
E   \else
      \unskip\unkern\unpenalty
E     \expandafter\expandafter\expandafter\setouterhboxRemove
E   \fi
E \fi
}%
%    \end{macrocode}
%    \end{macro}
%
%    \begin{macro}{\setouterhbox}
% Passing the box contents by macro parameter would prevent
% catcode changes in the box contents like by \cs{verb}.
% Also \cs{bgroup} and \cs{egroup} does not work, because stuff
% has to be added at the begin and end of the box, thus
% the syntax
% |\setouterhbox{|\meta{box number}|}|\dots|\endsetouterhbox|
% is used. Also we automatically get an environment \texttt{setouterhbox}
% if \LaTeX\ is used.
%    \begin{macrocode}
\def\setouterhbox#1{%
  \begingroup
    \def\setouterhboxNum{#1}%
    \setbox0\vbox\bgroup
T     \kern.123pt\relax % marker
T     \kern0pt\relax % removed by \setouterhboxRemove
      \begingroup
        \everypar{}%
        \noindent
}
%    \end{macrocode}
%    \end{macro}
%    \begin{macro}{\endsetouterhbox}
% Most of the work is done in the end part, thus the heart of
% the method follows:
%    \begin{macrocode}
\def\endsetouterhbox{%
      \endgroup
%    \end{macrocode}
% Omit the first pass to get the penalties
% of the second pass.
%    \begin{macrocode}
      \pretolerance-1 %
%    \end{macrocode}
%  We don't want a third pass with \cs{emergencystretch}.
%    \begin{macrocode}
      \tolerance10000 %
      \hsize\maxdimen
%    \end{macrocode}
% Line is not underfull:
%    \begin{macrocode}
      \parfillskip 0pt plus 1filll\relax
      \leftskip0pt\relax
%    \end{macrocode}
% Suppress underful \cs{hbox} warnings,
% is explicit line breaks are used.
%    \begin{macrocode}
      \rightskip0pt plus 1fil\relax
      \everypar{}%
%    \end{macrocode}
% Ensure that there is a paragraph and
% prevents \cs{endgraph} from eating terminal glue:
%    \begin{macrocode}
      \kern0pt%
      \endgraf
      \setouterhboxRemove
E     \ifnum\lastnodetype=1 %
E       \global\setbox\setouterhboxBox\lastbox
E       \loop
E         \setouterhboxRemove
E       \ifnum\lastnodetype=1 %
E         \setbox0=\lastbox
E         \global\setbox\setouterhboxBox=\hbox{%
E           \unhbox0 %
%    \end{macrocode}
% Remove \cs{rightskip}, a penalty with -10000 is part of the previous line.
%    \begin{macrocode}
E           \unskip
E           \unhbox\setouterhboxBox
E         }%
E       \repeat
E     \else
E       \setouterhboxFailure{%
E         Something is wrong%
E       }{%
E         Could not find expected line.%
E         \MessageBreak
E         (\string\lastnodetype: \number\lastnodetype, expected: 1)%
E       }%
E     \fi
E     \setouterhboxRemove
T     \global\setbox\setouterhboxBox\lastbox
T     \loop
T       \setouterhboxRemove
T       \setbox0=\lastbox
T     \ifcase\ifvoid0 1\else0\fi
T       \global\setbox\setouterhboxBox=\hbox{%
T         \unhbox0 %
%    \end{macrocode}
% Remove \cs{rightskip}, a penalty with -10000 is part of the previous line.
%    \begin{macrocode}
T         \unskip
T         \unhbox\setouterhboxBox
T       }%
T     \repeat
T     \ifdim.123pt=\lastkern
T     \else
T       \setouterhboxFailure{%
T         Something is wrong%
T       }{%
T         Unexpected stuff was detected before the line.%
T       }%
T     \fi
T   \egroup
T   \ifcase \ifnum\wd0=0 \else 1\fi
T           \ifdim\ht0=.123pt \else 1\fi
T           \ifnum\dp0=0 \else 1\fi
T           0 %
E     \ifnum\lastnodetype=-1 %
%    \end{macrocode}
% There was just one line that we have caught.
%    \begin{macrocode}
      \else
        \setouterhboxFailure{%
            Something is wrong%
        }{%
            After fetching the line there is more unexpected stuff.%
E           \MessageBreak
E           (\string\lastnodetype: \number\lastnodetype, expected: -1)%
        }%
      \fi
E   \egroup
  \expandafter\endgroup
  \expandafter\setouterhboxFinish\expandafter{%
    \number\setouterhboxNum
  }%
}
%    \end{macrocode}
%    \end{macro}
%
% \subsection{Environment support}
%
% Check \cs{@currenvir} for the case that \cs{setouterhbox}
% was called as environment. Then the box assignment
% must be put after the \cs{endgroup} of |\end{|\dots|}|.
%    \begin{macrocode}
\def\setouterhboxCurr{setouterhbox}
\def\setouterhboxLast#1{%
  \setbox#1\hbox{%
    \unhbox\setouterhboxBox
    \unskip % remove \rightskip glue
    \unskip % remove \parfillskip glue
    \unpenalty % remove paragraph ending \penalty 10000
    \unkern % remove explicit kern inserted above
  }%
}
%    \end{macrocode}
%    \begin{macro}{\setouterhboxFinish}
% |#1| is an explicit number.
%    \begin{macrocode}
\def\setouterhboxFinish#1{%
  \begingroup\expandafter\expandafter\expandafter\endgroup
  \expandafter\ifx\csname @currenvir\endcsname\setouterhboxCurr
    \aftergroup\setouterhboxLast
    \aftergroup{%
    \setouterhboxAfter #1\NIL
    \aftergroup}%
  \else
    \setouterhboxLast{#1}%
  \fi
}
%    \end{macrocode}
%    \end{macro}
%    \begin{macro}{\setouterhboxAfter}
% |#1| is an explicit number.
%    \begin{macrocode}
\def\setouterhboxAfter#1#2\NIL{%
  \aftergroup#1%
  \ifx\\#2\\%
  \else
    \setouterhboxReturnAfterFi{%
      \setouterhboxAfter#2\NIL
    }%
  \fi
}
%    \end{macrocode}
%    \end{macro}
%    \begin{macro}{\setouterhboxReturnAfterFi}
% A utility macro to get tail recursion.
%    \begin{macrocode}
\long\def\setouterhboxReturnAfterFi#1\fi{\fi#1}
%    \end{macrocode}
%    \end{macro}
% Restore catcodes we have need to distinguish between
% the implementation with and without \eTeX.
%    \begin{macrocode}
\catcode69=11\relax % E
\catcode84=11\relax % T
%    \end{macrocode}
%
% \subsection{Option \xoption{hyperref}}
%    \begin{macrocode}
\begingroup
  \def\x{LaTeX2e}%
\expandafter\endgroup
\ifx\x\fmtname
\else
  \expandafter\setouterhboxAtEnd
\fi%
%    \end{macrocode}
%    \begin{macro}{\Hy@setouterhbox}
% \cs{Hy@setouterhbox} is the internal hook that \xpackage{hyperref}
% uses since 2006/02/12 v6.75a.
%    \begin{macrocode}
\DeclareOption{hyperref}{%
  \long\def\Hy@setouterhbox#1#2{%
    \setouterhbox{#1}#2\endsetouterhbox
  }%
}
%    \end{macrocode}
%    \end{macro}
%    \begin{macrocode}
\ProcessOptions\relax
%    \end{macrocode}
%
%    \begin{macrocode}
\setouterhboxAtEnd%
%</package>
%    \end{macrocode}
%
% \section{Test}
%
% \subsection{Catcode checks for loading}
%
%    \begin{macrocode}
%<*test1>
%    \end{macrocode}
%    \begin{macrocode}
\catcode`\{=1 %
\catcode`\}=2 %
\catcode`\#=6 %
\catcode`\@=11 %
\expandafter\ifx\csname count@\endcsname\relax
  \countdef\count@=255 %
\fi
\expandafter\ifx\csname @gobble\endcsname\relax
  \long\def\@gobble#1{}%
\fi
\expandafter\ifx\csname @firstofone\endcsname\relax
  \long\def\@firstofone#1{#1}%
\fi
\expandafter\ifx\csname loop\endcsname\relax
  \expandafter\@firstofone
\else
  \expandafter\@gobble
\fi
{%
  \def\loop#1\repeat{%
    \def\body{#1}%
    \iterate
  }%
  \def\iterate{%
    \body
      \let\next\iterate
    \else
      \let\next\relax
    \fi
    \next
  }%
  \let\repeat=\fi
}%
\def\RestoreCatcodes{}
\count@=0 %
\loop
  \edef\RestoreCatcodes{%
    \RestoreCatcodes
    \catcode\the\count@=\the\catcode\count@\relax
  }%
\ifnum\count@<255 %
  \advance\count@ 1 %
\repeat

\def\RangeCatcodeInvalid#1#2{%
  \count@=#1\relax
  \loop
    \catcode\count@=15 %
  \ifnum\count@<#2\relax
    \advance\count@ 1 %
  \repeat
}
\def\RangeCatcodeCheck#1#2#3{%
  \count@=#1\relax
  \loop
    \ifnum#3=\catcode\count@
    \else
      \errmessage{%
        Character \the\count@\space
        with wrong catcode \the\catcode\count@\space
        instead of \number#3%
      }%
    \fi
  \ifnum\count@<#2\relax
    \advance\count@ 1 %
  \repeat
}
\def\space{ }
\expandafter\ifx\csname LoadCommand\endcsname\relax
  \def\LoadCommand{\input setouterhbox.sty\relax}%
\fi
\def\Test{%
  \RangeCatcodeInvalid{0}{47}%
  \RangeCatcodeInvalid{58}{64}%
  \RangeCatcodeInvalid{91}{96}%
  \RangeCatcodeInvalid{123}{255}%
  \catcode`\@=12 %
  \catcode`\\=0 %
  \catcode`\%=14 %
  \LoadCommand
  \RangeCatcodeCheck{0}{36}{15}%
  \RangeCatcodeCheck{37}{37}{14}%
  \RangeCatcodeCheck{38}{47}{15}%
  \RangeCatcodeCheck{48}{57}{12}%
  \RangeCatcodeCheck{58}{63}{15}%
  \RangeCatcodeCheck{64}{64}{12}%
  \RangeCatcodeCheck{65}{90}{11}%
  \RangeCatcodeCheck{91}{91}{15}%
  \RangeCatcodeCheck{92}{92}{0}%
  \RangeCatcodeCheck{93}{96}{15}%
  \RangeCatcodeCheck{97}{122}{11}%
  \RangeCatcodeCheck{123}{255}{15}%
  \RestoreCatcodes
}
\Test
\csname @@end\endcsname
\end
%    \end{macrocode}
%    \begin{macrocode}
%</test1>
%    \end{macrocode}
%
% \subsection{Test with package \xpackage{url}}
%
%    \begin{macrocode}
%<*test2>
\nofiles
\documentclass[a5paper]{article}
\usepackage{url}[2005/06/27]
\usepackage{setouterhbox}

\newsavebox{\testbox}

\setlength{\parindent}{0pt}
\setlength{\parskip}{2em}

\begin{document}
\raggedright

\url{http://this.is.a.very.long.host.name/followed/%
by/a/very_long_long_long_path.html}%

\sbox\testbox{%
  \url{http://this.is.a.very.long.host.name/followed/%
  by/a/very_long_long_long_path.html}%
}%
\unhbox\testbox

\begin{setouterhbox}{\testbox}%
  \url{http://this.is.a.very.long.host.name/followed/%
  by/a/very_long_long_long_path.html}%
\end{setouterhbox}
\unhbox\testbox

\end{document}
%</test2>
%    \end{macrocode}
%
% \section{Installation}
%
% \subsection{Download}
%
% \paragraph{Package.} This package is available on
% CTAN\footnote{\CTANpkg{setouterhbox}}:
% \begin{description}
% \item[\CTAN{macros/latex/contrib/oberdiek/setouterhbox.dtx}] The source file.
% \item[\CTAN{macros/latex/contrib/oberdiek/setouterhbox.pdf}] Documentation.
% \end{description}
%
%
% \paragraph{Bundle.} All the packages of the bundle `oberdiek'
% are also available in a TDS compliant ZIP archive. There
% the packages are already unpacked and the documentation files
% are generated. The files and directories obey the TDS standard.
% \begin{description}
% \item[\CTANinstall{install/macros/latex/contrib/oberdiek.tds.zip}]
% \end{description}
% \emph{TDS} refers to the standard ``A Directory Structure
% for \TeX\ Files'' (\CTAN{tds/tds.pdf}). Directories
% with \xfile{texmf} in their name are usually organized this way.
%
% \subsection{Bundle installation}
%
% \paragraph{Unpacking.} Unpack the \xfile{oberdiek.tds.zip} in the
% TDS tree (also known as \xfile{texmf} tree) of your choice.
% Example (linux):
% \begin{quote}
%   |unzip oberdiek.tds.zip -d ~/texmf|
% \end{quote}
%
% \subsection{Package installation}
%
% \paragraph{Unpacking.} The \xfile{.dtx} file is a self-extracting
% \docstrip\ archive. The files are extracted by running the
% \xfile{.dtx} through \plainTeX:
% \begin{quote}
%   \verb|tex setouterhbox.dtx|
% \end{quote}
%
% \paragraph{TDS.} Now the different files must be moved into
% the different directories in your installation TDS tree
% (also known as \xfile{texmf} tree):
% \begin{quote}
% \def\t{^^A
% \begin{tabular}{@{}>{\ttfamily}l@{ $\rightarrow$ }>{\ttfamily}l@{}}
%   setouterhbox.sty & tex/generic/oberdiek/setouterhbox.sty\\
%   setouterhbox.pdf & doc/latex/oberdiek/setouterhbox.pdf\\
%   setouterhbox-example.tex & doc/latex/oberdiek/setouterhbox-example.tex\\
%   test/setouterhbox-test1.tex & doc/latex/oberdiek/test/setouterhbox-test1.tex\\
%   test/setouterhbox-test2.tex & doc/latex/oberdiek/test/setouterhbox-test2.tex\\
%   setouterhbox.dtx & source/latex/oberdiek/setouterhbox.dtx\\
% \end{tabular}^^A
% }^^A
% \sbox0{\t}^^A
% \ifdim\wd0>\linewidth
%   \begingroup
%     \advance\linewidth by\leftmargin
%     \advance\linewidth by\rightmargin
%   \edef\x{\endgroup
%     \def\noexpand\lw{\the\linewidth}^^A
%   }\x
%   \def\lwbox{^^A
%     \leavevmode
%     \hbox to \linewidth{^^A
%       \kern-\leftmargin\relax
%       \hss
%       \usebox0
%       \hss
%       \kern-\rightmargin\relax
%     }^^A
%   }^^A
%   \ifdim\wd0>\lw
%     \sbox0{\small\t}^^A
%     \ifdim\wd0>\linewidth
%       \ifdim\wd0>\lw
%         \sbox0{\footnotesize\t}^^A
%         \ifdim\wd0>\linewidth
%           \ifdim\wd0>\lw
%             \sbox0{\scriptsize\t}^^A
%             \ifdim\wd0>\linewidth
%               \ifdim\wd0>\lw
%                 \sbox0{\tiny\t}^^A
%                 \ifdim\wd0>\linewidth
%                   \lwbox
%                 \else
%                   \usebox0
%                 \fi
%               \else
%                 \lwbox
%               \fi
%             \else
%               \usebox0
%             \fi
%           \else
%             \lwbox
%           \fi
%         \else
%           \usebox0
%         \fi
%       \else
%         \lwbox
%       \fi
%     \else
%       \usebox0
%     \fi
%   \else
%     \lwbox
%   \fi
% \else
%   \usebox0
% \fi
% \end{quote}
% If you have a \xfile{docstrip.cfg} that configures and enables \docstrip's
% TDS installing feature, then some files can already be in the right
% place, see the documentation of \docstrip.
%
% \subsection{Refresh file name databases}
%
% If your \TeX~distribution
% (\TeX\,Live, \mikTeX, \dots) relies on file name databases, you must refresh
% these. For example, \TeX\,Live\ users run \verb|texhash| or
% \verb|mktexlsr|.
%
% \subsection{Some details for the interested}
%
% \paragraph{Unpacking with \LaTeX.}
% The \xfile{.dtx} chooses its action depending on the format:
% \begin{description}
% \item[\plainTeX:] Run \docstrip\ and extract the files.
% \item[\LaTeX:] Generate the documentation.
% \end{description}
% If you insist on using \LaTeX\ for \docstrip\ (really,
% \docstrip\ does not need \LaTeX), then inform the autodetect routine
% about your intention:
% \begin{quote}
%   \verb|latex \let\install=y% \iffalse meta-comment
%
% File: setouterhbox.dtx
% Version: 2016/05/16 v1.8
% Info: Set hbox in outer horizontal mode
%
% Copyright (C)
%    2005-2007 Heiko Oberdiek
%    2016-2019 Oberdiek Package Support Group
%    https://github.com/ho-tex/oberdiek/issues
%
% This work may be distributed and/or modified under the
% conditions of the LaTeX Project Public License, either
% version 1.3c of this license or (at your option) any later
% version. This version of this license is in
%    https://www.latex-project.org/lppl/lppl-1-3c.txt
% and the latest version of this license is in
%    https://www.latex-project.org/lppl.txt
% and version 1.3 or later is part of all distributions of
% LaTeX version 2005/12/01 or later.
%
% This work has the LPPL maintenance status "maintained".
%
% The Current Maintainers of this work are
% Heiko Oberdiek and the Oberdiek Package Support Group
% https://github.com/ho-tex/oberdiek/issues
%
% The Base Interpreter refers to any `TeX-Format',
% because some files are installed in TDS:tex/generic//.
%
% This work consists of the main source file setouterhbox.dtx
% and the derived files
%    setouterhbox.sty, setouterhbox.pdf, setouterhbox.ins, setouterhbox.drv,
%    setouterhbox-example.tex, setouterhbox-test1.tex,
%    setouterhbox-test2.tex.
%
% Distribution:
%    CTAN:macros/latex/contrib/oberdiek/setouterhbox.dtx
%    CTAN:macros/latex/contrib/oberdiek/setouterhbox.pdf
%
% Unpacking:
%    (a) If setouterhbox.ins is present:
%           tex setouterhbox.ins
%    (b) Without setouterhbox.ins:
%           tex setouterhbox.dtx
%    (c) If you insist on using LaTeX
%           latex \let\install=y% \iffalse meta-comment
%
% File: setouterhbox.dtx
% Version: 2016/05/16 v1.8
% Info: Set hbox in outer horizontal mode
%
% Copyright (C)
%    2005-2007 Heiko Oberdiek
%    2016-2019 Oberdiek Package Support Group
%    https://github.com/ho-tex/oberdiek/issues
%
% This work may be distributed and/or modified under the
% conditions of the LaTeX Project Public License, either
% version 1.3c of this license or (at your option) any later
% version. This version of this license is in
%    https://www.latex-project.org/lppl/lppl-1-3c.txt
% and the latest version of this license is in
%    https://www.latex-project.org/lppl.txt
% and version 1.3 or later is part of all distributions of
% LaTeX version 2005/12/01 or later.
%
% This work has the LPPL maintenance status "maintained".
%
% The Current Maintainers of this work are
% Heiko Oberdiek and the Oberdiek Package Support Group
% https://github.com/ho-tex/oberdiek/issues
%
% The Base Interpreter refers to any `TeX-Format',
% because some files are installed in TDS:tex/generic//.
%
% This work consists of the main source file setouterhbox.dtx
% and the derived files
%    setouterhbox.sty, setouterhbox.pdf, setouterhbox.ins, setouterhbox.drv,
%    setouterhbox-example.tex, setouterhbox-test1.tex,
%    setouterhbox-test2.tex.
%
% Distribution:
%    CTAN:macros/latex/contrib/oberdiek/setouterhbox.dtx
%    CTAN:macros/latex/contrib/oberdiek/setouterhbox.pdf
%
% Unpacking:
%    (a) If setouterhbox.ins is present:
%           tex setouterhbox.ins
%    (b) Without setouterhbox.ins:
%           tex setouterhbox.dtx
%    (c) If you insist on using LaTeX
%           latex \let\install=y\input{setouterhbox.dtx}
%        (quote the arguments according to the demands of your shell)
%
% Documentation:
%    (a) If setouterhbox.drv is present:
%           latex setouterhbox.drv
%    (b) Without setouterhbox.drv:
%           latex setouterhbox.dtx; ...
%    The class ltxdoc loads the configuration file ltxdoc.cfg
%    if available. Here you can specify further options, e.g.
%    use A4 as paper format:
%       \PassOptionsToClass{a4paper}{article}
%
%    Programm calls to get the documentation (example):
%       pdflatex setouterhbox.dtx
%       makeindex -s gind.ist setouterhbox.idx
%       pdflatex setouterhbox.dtx
%       makeindex -s gind.ist setouterhbox.idx
%       pdflatex setouterhbox.dtx
%
% Installation:
%    TDS:tex/generic/oberdiek/setouterhbox.sty
%    TDS:doc/latex/oberdiek/setouterhbox.pdf
%    TDS:doc/latex/oberdiek/setouterhbox-example.tex
%    TDS:doc/latex/oberdiek/test/setouterhbox-test1.tex
%    TDS:doc/latex/oberdiek/test/setouterhbox-test2.tex
%    TDS:source/latex/oberdiek/setouterhbox.dtx
%
%<*ignore>
\begingroup
  \catcode123=1 %
  \catcode125=2 %
  \def\x{LaTeX2e}%
\expandafter\endgroup
\ifcase 0\ifx\install y1\fi\expandafter
         \ifx\csname processbatchFile\endcsname\relax\else1\fi
         \ifx\fmtname\x\else 1\fi\relax
\else\csname fi\endcsname
%</ignore>
%<*install>
\input docstrip.tex
\Msg{************************************************************************}
\Msg{* Installation}
\Msg{* Package: setouterhbox 2016/05/16 v1.8 Set hbox in outer horizontal mode (HO)}
\Msg{************************************************************************}

\keepsilent
\askforoverwritefalse

\let\MetaPrefix\relax
\preamble

This is a generated file.

Project: setouterhbox
Version: 2016/05/16 v1.8

Copyright (C)
   2005-2007 Heiko Oberdiek
   2016-2019 Oberdiek Package Support Group

This work may be distributed and/or modified under the
conditions of the LaTeX Project Public License, either
version 1.3c of this license or (at your option) any later
version. This version of this license is in
   https://www.latex-project.org/lppl/lppl-1-3c.txt
and the latest version of this license is in
   https://www.latex-project.org/lppl.txt
and version 1.3 or later is part of all distributions of
LaTeX version 2005/12/01 or later.

This work has the LPPL maintenance status "maintained".

The Current Maintainers of this work are
Heiko Oberdiek and the Oberdiek Package Support Group
https://github.com/ho-tex/oberdiek/issues


The Base Interpreter refers to any `TeX-Format',
because some files are installed in TDS:tex/generic//.

This work consists of the main source file setouterhbox.dtx
and the derived files
   setouterhbox.sty, setouterhbox.pdf, setouterhbox.ins, setouterhbox.drv,
   setouterhbox-example.tex, setouterhbox-test1.tex,
   setouterhbox-test2.tex.

\endpreamble
\let\MetaPrefix\DoubleperCent

\generate{%
  \file{setouterhbox.ins}{\from{setouterhbox.dtx}{install}}%
  \file{setouterhbox.drv}{\from{setouterhbox.dtx}{driver}}%
  \usedir{tex/generic/oberdiek}%
  \file{setouterhbox.sty}{\from{setouterhbox.dtx}{package}}%
  \usedir{doc/latex/oberdiek}%
  \file{setouterhbox-example.tex}{\from{setouterhbox.dtx}{example}}%
%  \usedir{doc/latex/oberdiek/test}%
%  \file{setouterhbox-test1.tex}{\from{setouterhbox.dtx}{test1}}%
%  \file{setouterhbox-test2.tex}{\from{setouterhbox.dtx}{test2}}%
  \nopreamble
  \nopostamble
%  \usedir{source/latex/oberdiek/catalogue}%
%  \file{setouterhbox.xml}{\from{setouterhbox.dtx}{catalogue}}%
}

\catcode32=13\relax% active space
\let =\space%
\Msg{************************************************************************}
\Msg{*}
\Msg{* To finish the installation you have to move the following}
\Msg{* file into a directory searched by TeX:}
\Msg{*}
\Msg{*     setouterhbox.sty}
\Msg{*}
\Msg{* To produce the documentation run the file `setouterhbox.drv'}
\Msg{* through LaTeX.}
\Msg{*}
\Msg{* Happy TeXing!}
\Msg{*}
\Msg{************************************************************************}

\endbatchfile
%</install>
%<*ignore>
\fi
%</ignore>
%<*driver>
\NeedsTeXFormat{LaTeX2e}
\ProvidesFile{setouterhbox.drv}%
  [2016/05/16 v1.8 Set hbox in outer horizontal mode (HO)]%
\documentclass{ltxdoc}
\usepackage{holtxdoc}[2011/11/22]
\begin{document}
  \DocInput{setouterhbox.dtx}%
\end{document}
%</driver>
% \fi
%
%
% \CharacterTable
%  {Upper-case    \A\B\C\D\E\F\G\H\I\J\K\L\M\N\O\P\Q\R\S\T\U\V\W\X\Y\Z
%   Lower-case    \a\b\c\d\e\f\g\h\i\j\k\l\m\n\o\p\q\r\s\t\u\v\w\x\y\z
%   Digits        \0\1\2\3\4\5\6\7\8\9
%   Exclamation   \!     Double quote  \"     Hash (number) \#
%   Dollar        \$     Percent       \%     Ampersand     \&
%   Acute accent  \'     Left paren    \(     Right paren   \)
%   Asterisk      \*     Plus          \+     Comma         \,
%   Minus         \-     Point         \.     Solidus       \/
%   Colon         \:     Semicolon     \;     Less than     \<
%   Equals        \=     Greater than  \>     Question mark \?
%   Commercial at \@     Left bracket  \[     Backslash     \\
%   Right bracket \]     Circumflex    \^     Underscore    \_
%   Grave accent  \`     Left brace    \{     Vertical bar  \|
%   Right brace   \}     Tilde         \~}
%
% \GetFileInfo{setouterhbox.drv}
%
% \title{The \xpackage{setouterhbox} package}
% \date{2016/05/16 v1.8}
% \author{Heiko Oberdiek\thanks
% {Please report any issues at \url{https://github.com/ho-tex/oberdiek/issues}}}
%
% \maketitle
%
% \begin{abstract}
% If math stuff is set in an \cs{hbox}, then TeX
% performs some optimization and omits the implicite
% penalties \cs{binoppenalty} and \cs{relpenalty}.
% This packages tries to put stuff into an \cs{hbox}
% without getting lost of those penalties.
% \end{abstract}
%
% \tableofcontents
%
% \section{Documentation}
%
% \subsection{Introduction}
%
% There is a situation in \xpackage{hyperref}'s driver for dvips
% where the user wants to have links that can be broken across
% lines. However dvips doesn't support the feature. With option
% \xoption{breaklinks} \xpackage{hyperref} sets the links as
% usual, put them in a box and write the link data with
% box dimensions into the appropriate \cs{special}s.
% Then, however, it does not set the complete unbreakable
% box, but it unwrappes the material inside to allow line
% breaks. Of course line breaking and glue setting will falsify
% the link dimensions, but line breaking was more important
% for the user.
%
% \subsection{Acknowledgement}
%
% Jonathan Fine, Donald Arsenau and me discussed the problem
% in the newsgroup \xnewsgroup{comp.text.tex} where Damian
% Menscher has started the thread, see \cite{newsstart}.
%
% The discussion was productive and generated many ideas
% and code examples. In order to have a more permanent
% result I wrote this package and tried to implement
% most of the ideas, a kind of summary of the discussion.
% Thus I want and have to thank Jonathan Fine and Donald Arsenau
% very much.
%
% Two weeks later David Kastrup (posting in
% \xnewsgroup{comp.text.tex}, \cite{kastrup})
% remembered an old article of Michael Downes (\cite{downes})
% in TUGboat, where Michael Downes already presented the
% method we discuss here. Nowadays we have \eTeX\ that extends
% the tool set of a \TeX\ macro programmer. Especially useful
% \eTeX\ was in this package for detecting and dealing with
% errorneous situations.
%
% However also nowadays a perfect solution for the problem
% is still missing at macro level. Probably someone has
% to go deep in the internals of the \TeX\ compiler to
% implement a switch that let penalties stay where otherwise
% \TeX\ would remove them for optimization reasons.
%
% \subsection{Usage}
%
% \paragraph{Package loading.}
% \LaTeX: as usually:
% \begin{quote}
%   |\usepackage{setouterhbox}|
% \end{quote}
% The package can also be included directly, thus \plainTeX\ users
% write:
% \begin{quote}
%   |\input setouterhbox.sty|
% \end{quote}
%
% \paragraph{Register allocation.}
% The material will be put into a box, thus we need to know these
% box number. If you need to allocate a new box register:
% \begin{description}
%  \item[\LaTeX:] |\newsavebox{\|\meta{name}|}|
%  \item[\plainTeX:] |\newbox\|\meta{name}
% \end{description}
% Then |\|\meta{name} is a command that held the box number.
%
% \paragraph{Box wrapping.}
% \LaTeX\ users put the material in the box with an environment
% similar to \texttt{lrbox}. The environment \texttt{setouterhbox}
% uses the same syntax and offers the same features, such
% as verbatim stuff inside:
% \begin{quote}
%  |\begin{setouterhbox}{|\meta{box number}|}|\dots
%  |\end{setouterhbox}|
% \end{quote}
% Users with \plainTeX\ do not have environments, they use instead:
% \begin{quote}
%   |\setouterhbox{|\meta{box number}|}|\dots|\endsetouterhbox|
% \end{quote}
% In both cases the material is put into an \cs{hbox} and assigned
% to the given box, denoted by \meta{box number}. Note the
% assignment is local, the same way \texttt{lrbox} behaves.
%
% \paragraph{Unwrapping.}
% The box material is ready for unwrapping:
% \begin{quote}
%   |\unhbox|\meta{box number}
% \end{quote}
%
% \subsection{Option \xoption{hyperref}}
%
% Package url uses math mode for typesetting urls.
% Break points are inserted by \cs{binoppenalty} and
% \cs{relpenalty}. Unhappily these break points are
% removed, if \xpackage{hyperref}
% is used with option {breaklinks}
% and drivers that depend on \xoption{pdfmark}:
% \xoption{dvips}, \xoption{vtexpdfmark}, \xoption{textures},
% and \xoption{dvipsone}.
% Thus the option \xoption{hyperref} enables the method
% of this package to avoid the removal of \cs{relpenalty}
% and \cs{binoppenalty}. Thus you get more break points.
% However, the link areas are still wrong for these
% drivers, because they are not supporting broken
% links.
%
% Note, you need version 2006/08/16 v6.75c of package \xpackage{hyperref},
% because starting with this version the necessary hook is provided
% that package \xpackage{setouterhbox} uses.
% \begin{quote}
%   |\usepackage[|\dots|]{hyperref}[2006/08/16]|\\
%   |\usepackage[hyperref]{setouterhbox}|
% \end{quote}
% Package order does not matter.
%
% \subsection{Example}
%
%    \begin{macrocode}
%<*example>
\documentclass[a5paper]{article}
\usepackage{url}[2005/06/27]
\usepackage{setouterhbox}

\newsavebox{\testbox}

\setlength{\parindent}{0pt}
\setlength{\parskip}{2em}

\begin{document}
\raggedright

\url{http://this.is.a.very.long.host.name/followed/%
by/a/very_long_long_long_path.html}%

\sbox\testbox{%
  \url{http://this.is.a.very.long.host.name/followed/%
  by/a/very_long_long_long_path.html}%
}%
\unhbox\testbox

\begin{setouterhbox}{\testbox}%
  \url{http://this.is.a.very.long.host.name/followed/%
  by/a/very_long_long_long_path.html}%
\end{setouterhbox}
\unhbox\testbox

\end{document}
%</example>
%    \end{macrocode}
%
% \StopEventually{
% }
%
% \section{Implementation}
%
% Internal macros are prefixed by \cs{setouterhbox}, |@| is
% not used inside names, thus we do not need to care of its
% catcode if we are not using it as \LaTeX\ package.
%
% \subsection{Package start stuff}
%
%    \begin{macrocode}
%<*package>
%    \end{macrocode}
%
% Prevent reloading more than one, necessary for \plainTeX:
%    Reload check, especially if the package is not used with \LaTeX.
%    \begin{macrocode}
\begingroup\catcode61\catcode48\catcode32=10\relax%
  \catcode13=5 % ^^M
  \endlinechar=13 %
  \catcode35=6 % #
  \catcode39=12 % '
  \catcode44=12 % ,
  \catcode45=12 % -
  \catcode46=12 % .
  \catcode58=12 % :
  \catcode64=11 % @
  \catcode123=1 % {
  \catcode125=2 % }
  \expandafter\let\expandafter\x\csname ver@setouterhbox.sty\endcsname
  \ifx\x\relax % plain-TeX, first loading
  \else
    \def\empty{}%
    \ifx\x\empty % LaTeX, first loading,
      % variable is initialized, but \ProvidesPackage not yet seen
    \else
      \expandafter\ifx\csname PackageInfo\endcsname\relax
        \def\x#1#2{%
          \immediate\write-1{Package #1 Info: #2.}%
        }%
      \else
        \def\x#1#2{\PackageInfo{#1}{#2, stopped}}%
      \fi
      \x{setouterhbox}{The package is already loaded}%
      \aftergroup\endinput
    \fi
  \fi
\endgroup%
%    \end{macrocode}
%    Package identification:
%    \begin{macrocode}
\begingroup\catcode61\catcode48\catcode32=10\relax%
  \catcode13=5 % ^^M
  \endlinechar=13 %
  \catcode35=6 % #
  \catcode39=12 % '
  \catcode40=12 % (
  \catcode41=12 % )
  \catcode44=12 % ,
  \catcode45=12 % -
  \catcode46=12 % .
  \catcode47=12 % /
  \catcode58=12 % :
  \catcode64=11 % @
  \catcode91=12 % [
  \catcode93=12 % ]
  \catcode123=1 % {
  \catcode125=2 % }
  \expandafter\ifx\csname ProvidesPackage\endcsname\relax
    \def\x#1#2#3[#4]{\endgroup
      \immediate\write-1{Package: #3 #4}%
      \xdef#1{#4}%
    }%
  \else
    \def\x#1#2[#3]{\endgroup
      #2[{#3}]%
      \ifx#1\@undefined
        \xdef#1{#3}%
      \fi
      \ifx#1\relax
        \xdef#1{#3}%
      \fi
    }%
  \fi
\expandafter\x\csname ver@setouterhbox.sty\endcsname
\ProvidesPackage{setouterhbox}%
  [2016/05/16 v1.8 Set hbox in outer horizontal mode (HO)]%
%    \end{macrocode}
%
%    \begin{macrocode}
\begingroup\catcode61\catcode48\catcode32=10\relax%
  \catcode13=5 % ^^M
  \endlinechar=13 %
  \catcode123=1 % {
  \catcode125=2 % }
  \catcode64=11 % @
  \def\x{\endgroup
    \expandafter\edef\csname setouterhboxAtEnd\endcsname{%
      \endlinechar=\the\endlinechar\relax
      \catcode13=\the\catcode13\relax
      \catcode32=\the\catcode32\relax
      \catcode35=\the\catcode35\relax
      \catcode61=\the\catcode61\relax
      \catcode64=\the\catcode64\relax
      \catcode123=\the\catcode123\relax
      \catcode125=\the\catcode125\relax
    }%
  }%
\x\catcode61\catcode48\catcode32=10\relax%
\catcode13=5 % ^^M
\endlinechar=13 %
\catcode35=6 % #
\catcode64=11 % @
\catcode123=1 % {
\catcode125=2 % }
\def\TMP@EnsureCode#1#2{%
  \edef\setouterhboxAtEnd{%
    \setouterhboxAtEnd
    \catcode#1=\the\catcode#1\relax
  }%
  \catcode#1=#2\relax
}
\TMP@EnsureCode{40}{12}% (
\TMP@EnsureCode{41}{12}% )
\TMP@EnsureCode{44}{12}% ,
\TMP@EnsureCode{45}{12}% -
\TMP@EnsureCode{46}{12}% .
\TMP@EnsureCode{47}{12}% /
\TMP@EnsureCode{58}{12}% :
\TMP@EnsureCode{60}{12}% <
\TMP@EnsureCode{62}{12}% >
\TMP@EnsureCode{91}{12}% [
\TMP@EnsureCode{93}{12}% ]
\TMP@EnsureCode{96}{12}% `
\edef\setouterhboxAtEnd{\setouterhboxAtEnd\noexpand\endinput}
%    \end{macrocode}
%
% \subsection{Interface macros}
%
%    \begin{macro}{\setouterhboxBox}
% The method requires a global box assignment. To be on the
% safe side, a new box register is allocated for this
% global box assignment.
%    \begin{macrocode}
\newbox\setouterhboxBox
%    \end{macrocode}
%    \end{macro}
%
%    \begin{macro}{\setouterhboxFailure}
% Error message for both \plainTeX\ and \LaTeX
%    \begin{macrocode}
\begingroup\expandafter\expandafter\expandafter\endgroup
\expandafter\ifx\csname RequirePackage\endcsname\relax
  \input infwarerr.sty\relax
\else
  \RequirePackage{infwarerr}[2016/05/16]%
\fi
\edef\setouterhboxFailure#1#2{%
  \expandafter\noexpand\csname @PackageError\endcsname
      {setouterhbox}{#1}{#2}%
}
%    \end{macrocode}
%    \end{macro}
%
% \subsection{Main part}
%
% eTeX provides much better means for checking
% error conditions. Thus lines marked by "E" are executed
% if eTeX is available, otherwise the lines marked by "T" are
% used.
%    \begin{macrocode}
\begingroup\expandafter\expandafter\expandafter\endgroup
\expandafter\ifx\csname lastnodetype\endcsname\relax
  \catcode`T=9 % ignore
  \catcode`E=14 % comment
\else
  \catcode`T=14 % comment
  \catcode`E=9 % ignore
\fi
%    \end{macrocode}
%
%    \begin{macro}{\setouterhboxRemove}
% Remove all kern, glue, and penalty nodes;
% poor man's version, if \eTeX\ is not available
%    \begin{macrocode}
\def\setouterhboxRemove{%
E \ifnum\lastnodetype<11 %
E   \else
E   \ifnum\lastnodetype>13 %
E   \else
      \unskip\unkern\unpenalty
E     \expandafter\expandafter\expandafter\setouterhboxRemove
E   \fi
E \fi
}%
%    \end{macrocode}
%    \end{macro}
%
%    \begin{macro}{\setouterhbox}
% Passing the box contents by macro parameter would prevent
% catcode changes in the box contents like by \cs{verb}.
% Also \cs{bgroup} and \cs{egroup} does not work, because stuff
% has to be added at the begin and end of the box, thus
% the syntax
% |\setouterhbox{|\meta{box number}|}|\dots|\endsetouterhbox|
% is used. Also we automatically get an environment \texttt{setouterhbox}
% if \LaTeX\ is used.
%    \begin{macrocode}
\def\setouterhbox#1{%
  \begingroup
    \def\setouterhboxNum{#1}%
    \setbox0\vbox\bgroup
T     \kern.123pt\relax % marker
T     \kern0pt\relax % removed by \setouterhboxRemove
      \begingroup
        \everypar{}%
        \noindent
}
%    \end{macrocode}
%    \end{macro}
%    \begin{macro}{\endsetouterhbox}
% Most of the work is done in the end part, thus the heart of
% the method follows:
%    \begin{macrocode}
\def\endsetouterhbox{%
      \endgroup
%    \end{macrocode}
% Omit the first pass to get the penalties
% of the second pass.
%    \begin{macrocode}
      \pretolerance-1 %
%    \end{macrocode}
%  We don't want a third pass with \cs{emergencystretch}.
%    \begin{macrocode}
      \tolerance10000 %
      \hsize\maxdimen
%    \end{macrocode}
% Line is not underfull:
%    \begin{macrocode}
      \parfillskip 0pt plus 1filll\relax
      \leftskip0pt\relax
%    \end{macrocode}
% Suppress underful \cs{hbox} warnings,
% is explicit line breaks are used.
%    \begin{macrocode}
      \rightskip0pt plus 1fil\relax
      \everypar{}%
%    \end{macrocode}
% Ensure that there is a paragraph and
% prevents \cs{endgraph} from eating terminal glue:
%    \begin{macrocode}
      \kern0pt%
      \endgraf
      \setouterhboxRemove
E     \ifnum\lastnodetype=1 %
E       \global\setbox\setouterhboxBox\lastbox
E       \loop
E         \setouterhboxRemove
E       \ifnum\lastnodetype=1 %
E         \setbox0=\lastbox
E         \global\setbox\setouterhboxBox=\hbox{%
E           \unhbox0 %
%    \end{macrocode}
% Remove \cs{rightskip}, a penalty with -10000 is part of the previous line.
%    \begin{macrocode}
E           \unskip
E           \unhbox\setouterhboxBox
E         }%
E       \repeat
E     \else
E       \setouterhboxFailure{%
E         Something is wrong%
E       }{%
E         Could not find expected line.%
E         \MessageBreak
E         (\string\lastnodetype: \number\lastnodetype, expected: 1)%
E       }%
E     \fi
E     \setouterhboxRemove
T     \global\setbox\setouterhboxBox\lastbox
T     \loop
T       \setouterhboxRemove
T       \setbox0=\lastbox
T     \ifcase\ifvoid0 1\else0\fi
T       \global\setbox\setouterhboxBox=\hbox{%
T         \unhbox0 %
%    \end{macrocode}
% Remove \cs{rightskip}, a penalty with -10000 is part of the previous line.
%    \begin{macrocode}
T         \unskip
T         \unhbox\setouterhboxBox
T       }%
T     \repeat
T     \ifdim.123pt=\lastkern
T     \else
T       \setouterhboxFailure{%
T         Something is wrong%
T       }{%
T         Unexpected stuff was detected before the line.%
T       }%
T     \fi
T   \egroup
T   \ifcase \ifnum\wd0=0 \else 1\fi
T           \ifdim\ht0=.123pt \else 1\fi
T           \ifnum\dp0=0 \else 1\fi
T           0 %
E     \ifnum\lastnodetype=-1 %
%    \end{macrocode}
% There was just one line that we have caught.
%    \begin{macrocode}
      \else
        \setouterhboxFailure{%
            Something is wrong%
        }{%
            After fetching the line there is more unexpected stuff.%
E           \MessageBreak
E           (\string\lastnodetype: \number\lastnodetype, expected: -1)%
        }%
      \fi
E   \egroup
  \expandafter\endgroup
  \expandafter\setouterhboxFinish\expandafter{%
    \number\setouterhboxNum
  }%
}
%    \end{macrocode}
%    \end{macro}
%
% \subsection{Environment support}
%
% Check \cs{@currenvir} for the case that \cs{setouterhbox}
% was called as environment. Then the box assignment
% must be put after the \cs{endgroup} of |\end{|\dots|}|.
%    \begin{macrocode}
\def\setouterhboxCurr{setouterhbox}
\def\setouterhboxLast#1{%
  \setbox#1\hbox{%
    \unhbox\setouterhboxBox
    \unskip % remove \rightskip glue
    \unskip % remove \parfillskip glue
    \unpenalty % remove paragraph ending \penalty 10000
    \unkern % remove explicit kern inserted above
  }%
}
%    \end{macrocode}
%    \begin{macro}{\setouterhboxFinish}
% |#1| is an explicit number.
%    \begin{macrocode}
\def\setouterhboxFinish#1{%
  \begingroup\expandafter\expandafter\expandafter\endgroup
  \expandafter\ifx\csname @currenvir\endcsname\setouterhboxCurr
    \aftergroup\setouterhboxLast
    \aftergroup{%
    \setouterhboxAfter #1\NIL
    \aftergroup}%
  \else
    \setouterhboxLast{#1}%
  \fi
}
%    \end{macrocode}
%    \end{macro}
%    \begin{macro}{\setouterhboxAfter}
% |#1| is an explicit number.
%    \begin{macrocode}
\def\setouterhboxAfter#1#2\NIL{%
  \aftergroup#1%
  \ifx\\#2\\%
  \else
    \setouterhboxReturnAfterFi{%
      \setouterhboxAfter#2\NIL
    }%
  \fi
}
%    \end{macrocode}
%    \end{macro}
%    \begin{macro}{\setouterhboxReturnAfterFi}
% A utility macro to get tail recursion.
%    \begin{macrocode}
\long\def\setouterhboxReturnAfterFi#1\fi{\fi#1}
%    \end{macrocode}
%    \end{macro}
% Restore catcodes we have need to distinguish between
% the implementation with and without \eTeX.
%    \begin{macrocode}
\catcode69=11\relax % E
\catcode84=11\relax % T
%    \end{macrocode}
%
% \subsection{Option \xoption{hyperref}}
%    \begin{macrocode}
\begingroup
  \def\x{LaTeX2e}%
\expandafter\endgroup
\ifx\x\fmtname
\else
  \expandafter\setouterhboxAtEnd
\fi%
%    \end{macrocode}
%    \begin{macro}{\Hy@setouterhbox}
% \cs{Hy@setouterhbox} is the internal hook that \xpackage{hyperref}
% uses since 2006/02/12 v6.75a.
%    \begin{macrocode}
\DeclareOption{hyperref}{%
  \long\def\Hy@setouterhbox#1#2{%
    \setouterhbox{#1}#2\endsetouterhbox
  }%
}
%    \end{macrocode}
%    \end{macro}
%    \begin{macrocode}
\ProcessOptions\relax
%    \end{macrocode}
%
%    \begin{macrocode}
\setouterhboxAtEnd%
%</package>
%    \end{macrocode}
%
% \section{Test}
%
% \subsection{Catcode checks for loading}
%
%    \begin{macrocode}
%<*test1>
%    \end{macrocode}
%    \begin{macrocode}
\catcode`\{=1 %
\catcode`\}=2 %
\catcode`\#=6 %
\catcode`\@=11 %
\expandafter\ifx\csname count@\endcsname\relax
  \countdef\count@=255 %
\fi
\expandafter\ifx\csname @gobble\endcsname\relax
  \long\def\@gobble#1{}%
\fi
\expandafter\ifx\csname @firstofone\endcsname\relax
  \long\def\@firstofone#1{#1}%
\fi
\expandafter\ifx\csname loop\endcsname\relax
  \expandafter\@firstofone
\else
  \expandafter\@gobble
\fi
{%
  \def\loop#1\repeat{%
    \def\body{#1}%
    \iterate
  }%
  \def\iterate{%
    \body
      \let\next\iterate
    \else
      \let\next\relax
    \fi
    \next
  }%
  \let\repeat=\fi
}%
\def\RestoreCatcodes{}
\count@=0 %
\loop
  \edef\RestoreCatcodes{%
    \RestoreCatcodes
    \catcode\the\count@=\the\catcode\count@\relax
  }%
\ifnum\count@<255 %
  \advance\count@ 1 %
\repeat

\def\RangeCatcodeInvalid#1#2{%
  \count@=#1\relax
  \loop
    \catcode\count@=15 %
  \ifnum\count@<#2\relax
    \advance\count@ 1 %
  \repeat
}
\def\RangeCatcodeCheck#1#2#3{%
  \count@=#1\relax
  \loop
    \ifnum#3=\catcode\count@
    \else
      \errmessage{%
        Character \the\count@\space
        with wrong catcode \the\catcode\count@\space
        instead of \number#3%
      }%
    \fi
  \ifnum\count@<#2\relax
    \advance\count@ 1 %
  \repeat
}
\def\space{ }
\expandafter\ifx\csname LoadCommand\endcsname\relax
  \def\LoadCommand{\input setouterhbox.sty\relax}%
\fi
\def\Test{%
  \RangeCatcodeInvalid{0}{47}%
  \RangeCatcodeInvalid{58}{64}%
  \RangeCatcodeInvalid{91}{96}%
  \RangeCatcodeInvalid{123}{255}%
  \catcode`\@=12 %
  \catcode`\\=0 %
  \catcode`\%=14 %
  \LoadCommand
  \RangeCatcodeCheck{0}{36}{15}%
  \RangeCatcodeCheck{37}{37}{14}%
  \RangeCatcodeCheck{38}{47}{15}%
  \RangeCatcodeCheck{48}{57}{12}%
  \RangeCatcodeCheck{58}{63}{15}%
  \RangeCatcodeCheck{64}{64}{12}%
  \RangeCatcodeCheck{65}{90}{11}%
  \RangeCatcodeCheck{91}{91}{15}%
  \RangeCatcodeCheck{92}{92}{0}%
  \RangeCatcodeCheck{93}{96}{15}%
  \RangeCatcodeCheck{97}{122}{11}%
  \RangeCatcodeCheck{123}{255}{15}%
  \RestoreCatcodes
}
\Test
\csname @@end\endcsname
\end
%    \end{macrocode}
%    \begin{macrocode}
%</test1>
%    \end{macrocode}
%
% \subsection{Test with package \xpackage{url}}
%
%    \begin{macrocode}
%<*test2>
\nofiles
\documentclass[a5paper]{article}
\usepackage{url}[2005/06/27]
\usepackage{setouterhbox}

\newsavebox{\testbox}

\setlength{\parindent}{0pt}
\setlength{\parskip}{2em}

\begin{document}
\raggedright

\url{http://this.is.a.very.long.host.name/followed/%
by/a/very_long_long_long_path.html}%

\sbox\testbox{%
  \url{http://this.is.a.very.long.host.name/followed/%
  by/a/very_long_long_long_path.html}%
}%
\unhbox\testbox

\begin{setouterhbox}{\testbox}%
  \url{http://this.is.a.very.long.host.name/followed/%
  by/a/very_long_long_long_path.html}%
\end{setouterhbox}
\unhbox\testbox

\end{document}
%</test2>
%    \end{macrocode}
%
% \section{Installation}
%
% \subsection{Download}
%
% \paragraph{Package.} This package is available on
% CTAN\footnote{\CTANpkg{setouterhbox}}:
% \begin{description}
% \item[\CTAN{macros/latex/contrib/oberdiek/setouterhbox.dtx}] The source file.
% \item[\CTAN{macros/latex/contrib/oberdiek/setouterhbox.pdf}] Documentation.
% \end{description}
%
%
% \paragraph{Bundle.} All the packages of the bundle `oberdiek'
% are also available in a TDS compliant ZIP archive. There
% the packages are already unpacked and the documentation files
% are generated. The files and directories obey the TDS standard.
% \begin{description}
% \item[\CTANinstall{install/macros/latex/contrib/oberdiek.tds.zip}]
% \end{description}
% \emph{TDS} refers to the standard ``A Directory Structure
% for \TeX\ Files'' (\CTAN{tds/tds.pdf}). Directories
% with \xfile{texmf} in their name are usually organized this way.
%
% \subsection{Bundle installation}
%
% \paragraph{Unpacking.} Unpack the \xfile{oberdiek.tds.zip} in the
% TDS tree (also known as \xfile{texmf} tree) of your choice.
% Example (linux):
% \begin{quote}
%   |unzip oberdiek.tds.zip -d ~/texmf|
% \end{quote}
%
% \subsection{Package installation}
%
% \paragraph{Unpacking.} The \xfile{.dtx} file is a self-extracting
% \docstrip\ archive. The files are extracted by running the
% \xfile{.dtx} through \plainTeX:
% \begin{quote}
%   \verb|tex setouterhbox.dtx|
% \end{quote}
%
% \paragraph{TDS.} Now the different files must be moved into
% the different directories in your installation TDS tree
% (also known as \xfile{texmf} tree):
% \begin{quote}
% \def\t{^^A
% \begin{tabular}{@{}>{\ttfamily}l@{ $\rightarrow$ }>{\ttfamily}l@{}}
%   setouterhbox.sty & tex/generic/oberdiek/setouterhbox.sty\\
%   setouterhbox.pdf & doc/latex/oberdiek/setouterhbox.pdf\\
%   setouterhbox-example.tex & doc/latex/oberdiek/setouterhbox-example.tex\\
%   test/setouterhbox-test1.tex & doc/latex/oberdiek/test/setouterhbox-test1.tex\\
%   test/setouterhbox-test2.tex & doc/latex/oberdiek/test/setouterhbox-test2.tex\\
%   setouterhbox.dtx & source/latex/oberdiek/setouterhbox.dtx\\
% \end{tabular}^^A
% }^^A
% \sbox0{\t}^^A
% \ifdim\wd0>\linewidth
%   \begingroup
%     \advance\linewidth by\leftmargin
%     \advance\linewidth by\rightmargin
%   \edef\x{\endgroup
%     \def\noexpand\lw{\the\linewidth}^^A
%   }\x
%   \def\lwbox{^^A
%     \leavevmode
%     \hbox to \linewidth{^^A
%       \kern-\leftmargin\relax
%       \hss
%       \usebox0
%       \hss
%       \kern-\rightmargin\relax
%     }^^A
%   }^^A
%   \ifdim\wd0>\lw
%     \sbox0{\small\t}^^A
%     \ifdim\wd0>\linewidth
%       \ifdim\wd0>\lw
%         \sbox0{\footnotesize\t}^^A
%         \ifdim\wd0>\linewidth
%           \ifdim\wd0>\lw
%             \sbox0{\scriptsize\t}^^A
%             \ifdim\wd0>\linewidth
%               \ifdim\wd0>\lw
%                 \sbox0{\tiny\t}^^A
%                 \ifdim\wd0>\linewidth
%                   \lwbox
%                 \else
%                   \usebox0
%                 \fi
%               \else
%                 \lwbox
%               \fi
%             \else
%               \usebox0
%             \fi
%           \else
%             \lwbox
%           \fi
%         \else
%           \usebox0
%         \fi
%       \else
%         \lwbox
%       \fi
%     \else
%       \usebox0
%     \fi
%   \else
%     \lwbox
%   \fi
% \else
%   \usebox0
% \fi
% \end{quote}
% If you have a \xfile{docstrip.cfg} that configures and enables \docstrip's
% TDS installing feature, then some files can already be in the right
% place, see the documentation of \docstrip.
%
% \subsection{Refresh file name databases}
%
% If your \TeX~distribution
% (\TeX\,Live, \mikTeX, \dots) relies on file name databases, you must refresh
% these. For example, \TeX\,Live\ users run \verb|texhash| or
% \verb|mktexlsr|.
%
% \subsection{Some details for the interested}
%
% \paragraph{Unpacking with \LaTeX.}
% The \xfile{.dtx} chooses its action depending on the format:
% \begin{description}
% \item[\plainTeX:] Run \docstrip\ and extract the files.
% \item[\LaTeX:] Generate the documentation.
% \end{description}
% If you insist on using \LaTeX\ for \docstrip\ (really,
% \docstrip\ does not need \LaTeX), then inform the autodetect routine
% about your intention:
% \begin{quote}
%   \verb|latex \let\install=y\input{setouterhbox.dtx}|
% \end{quote}
% Do not forget to quote the argument according to the demands
% of your shell.
%
% \paragraph{Generating the documentation.}
% You can use both the \xfile{.dtx} or the \xfile{.drv} to generate
% the documentation. The process can be configured by the
% configuration file \xfile{ltxdoc.cfg}. For instance, put this
% line into this file, if you want to have A4 as paper format:
% \begin{quote}
%   \verb|\PassOptionsToClass{a4paper}{article}|
% \end{quote}
% An example follows how to generate the
% documentation with pdf\LaTeX:
% \begin{quote}
%\begin{verbatim}
%pdflatex setouterhbox.dtx
%makeindex -s gind.ist setouterhbox.idx
%pdflatex setouterhbox.dtx
%makeindex -s gind.ist setouterhbox.idx
%pdflatex setouterhbox.dtx
%\end{verbatim}
% \end{quote}
%
% \begin{thebibliography}{9}
%
% \bibitem{newsstart}
%   Damian Menscher, \Newsgroup{comp.text.tex},
%   \textit{overlong lines in List of Figures},
%   \nolinkurl{<dh058t$qbd$1@news.ks.uiuc.edu>},
%   23rd September 2005.
%   \url{https://groups.google.com/group/comp.text.tex/msg/79648d4cf1f8bc13}
%
% \bibitem{kastrup}
%   David Kastrup, \Newsgroup{comp.text.tex},
%   \textit{Re: ANN: outerhbox.sty -- collect horizontal material,
%   for unboxing into a paragraph},
%   \nolinkurl{<85y855lrx3.fsf@lola.goethe.zz>},
%   7th October 2005.
%   \url{https://groups.google.com/group/comp.text.tex/msg/7cf0a345ef932e52}
%
% \bibitem{downes}
%   Michael Downes, \textit{Line breaking in \cs{unhbox}ed Text},
%   TUGboat 11 (1990), pp. 605--612.
%
% \bibitem{hyperref}
%   Sebastian Rahtz, Heiko Oberdiek:
%   \textit{The \xpackage{hyperref} package};
%   2006/08/16 v6.75c;
%   \CTANpkg{hyperref}.
%
% \end{thebibliography}
%
% \begin{History}
%   \begin{Version}{2005/10/05 v1.0}
%   \item
%     First version.
%   \end{Version}
%   \begin{Version}{2005/10/07 v1.1}
%   \item
%     Option \xoption{hyperref} added.
%   \end{Version}
%   \begin{Version}{2005/10/18 v1.2}
%   \item
%     Support for explicit line breaks added.
%   \end{Version}
%   \begin{Version}{2006/02/12 v1.3}
%   \item
%     DTX format.
%   \item
%     Documentation extended.
%   \end{Version}
%   \begin{Version}{2006/08/26 v1.4}
%   \item
%     Date of hyperref updated.
%   \end{Version}
%   \begin{Version}{2007/04/26 v1.5}
%   \item
%     Use of package \xpackage{infwarerr}.
%   \end{Version}
%   \begin{Version}{2007/05/17 v1.6}
%   \item
%     Standard header part for generic files.
%   \end{Version}
%   \begin{Version}{2007/09/09 v1.7}
%   \item
%     Catcode section added.
%   \end{Version}
%   \begin{Version}{2016/05/16 v1.8}
%   \item
%     Documentation updates.
%   \end{Version}
% \end{History}
%
% \PrintIndex
%
% \Finale
\endinput

%        (quote the arguments according to the demands of your shell)
%
% Documentation:
%    (a) If setouterhbox.drv is present:
%           latex setouterhbox.drv
%    (b) Without setouterhbox.drv:
%           latex setouterhbox.dtx; ...
%    The class ltxdoc loads the configuration file ltxdoc.cfg
%    if available. Here you can specify further options, e.g.
%    use A4 as paper format:
%       \PassOptionsToClass{a4paper}{article}
%
%    Programm calls to get the documentation (example):
%       pdflatex setouterhbox.dtx
%       makeindex -s gind.ist setouterhbox.idx
%       pdflatex setouterhbox.dtx
%       makeindex -s gind.ist setouterhbox.idx
%       pdflatex setouterhbox.dtx
%
% Installation:
%    TDS:tex/generic/oberdiek/setouterhbox.sty
%    TDS:doc/latex/oberdiek/setouterhbox.pdf
%    TDS:doc/latex/oberdiek/setouterhbox-example.tex
%    TDS:doc/latex/oberdiek/test/setouterhbox-test1.tex
%    TDS:doc/latex/oberdiek/test/setouterhbox-test2.tex
%    TDS:source/latex/oberdiek/setouterhbox.dtx
%
%<*ignore>
\begingroup
  \catcode123=1 %
  \catcode125=2 %
  \def\x{LaTeX2e}%
\expandafter\endgroup
\ifcase 0\ifx\install y1\fi\expandafter
         \ifx\csname processbatchFile\endcsname\relax\else1\fi
         \ifx\fmtname\x\else 1\fi\relax
\else\csname fi\endcsname
%</ignore>
%<*install>
\input docstrip.tex
\Msg{************************************************************************}
\Msg{* Installation}
\Msg{* Package: setouterhbox 2016/05/16 v1.8 Set hbox in outer horizontal mode (HO)}
\Msg{************************************************************************}

\keepsilent
\askforoverwritefalse

\let\MetaPrefix\relax
\preamble

This is a generated file.

Project: setouterhbox
Version: 2016/05/16 v1.8

Copyright (C)
   2005-2007 Heiko Oberdiek
   2016-2019 Oberdiek Package Support Group

This work may be distributed and/or modified under the
conditions of the LaTeX Project Public License, either
version 1.3c of this license or (at your option) any later
version. This version of this license is in
   https://www.latex-project.org/lppl/lppl-1-3c.txt
and the latest version of this license is in
   https://www.latex-project.org/lppl.txt
and version 1.3 or later is part of all distributions of
LaTeX version 2005/12/01 or later.

This work has the LPPL maintenance status "maintained".

The Current Maintainers of this work are
Heiko Oberdiek and the Oberdiek Package Support Group
https://github.com/ho-tex/oberdiek/issues


The Base Interpreter refers to any `TeX-Format',
because some files are installed in TDS:tex/generic//.

This work consists of the main source file setouterhbox.dtx
and the derived files
   setouterhbox.sty, setouterhbox.pdf, setouterhbox.ins, setouterhbox.drv,
   setouterhbox-example.tex, setouterhbox-test1.tex,
   setouterhbox-test2.tex.

\endpreamble
\let\MetaPrefix\DoubleperCent

\generate{%
  \file{setouterhbox.ins}{\from{setouterhbox.dtx}{install}}%
  \file{setouterhbox.drv}{\from{setouterhbox.dtx}{driver}}%
  \usedir{tex/generic/oberdiek}%
  \file{setouterhbox.sty}{\from{setouterhbox.dtx}{package}}%
  \usedir{doc/latex/oberdiek}%
  \file{setouterhbox-example.tex}{\from{setouterhbox.dtx}{example}}%
%  \usedir{doc/latex/oberdiek/test}%
%  \file{setouterhbox-test1.tex}{\from{setouterhbox.dtx}{test1}}%
%  \file{setouterhbox-test2.tex}{\from{setouterhbox.dtx}{test2}}%
  \nopreamble
  \nopostamble
%  \usedir{source/latex/oberdiek/catalogue}%
%  \file{setouterhbox.xml}{\from{setouterhbox.dtx}{catalogue}}%
}

\catcode32=13\relax% active space
\let =\space%
\Msg{************************************************************************}
\Msg{*}
\Msg{* To finish the installation you have to move the following}
\Msg{* file into a directory searched by TeX:}
\Msg{*}
\Msg{*     setouterhbox.sty}
\Msg{*}
\Msg{* To produce the documentation run the file `setouterhbox.drv'}
\Msg{* through LaTeX.}
\Msg{*}
\Msg{* Happy TeXing!}
\Msg{*}
\Msg{************************************************************************}

\endbatchfile
%</install>
%<*ignore>
\fi
%</ignore>
%<*driver>
\NeedsTeXFormat{LaTeX2e}
\ProvidesFile{setouterhbox.drv}%
  [2016/05/16 v1.8 Set hbox in outer horizontal mode (HO)]%
\documentclass{ltxdoc}
\usepackage{holtxdoc}[2011/11/22]
\begin{document}
  \DocInput{setouterhbox.dtx}%
\end{document}
%</driver>
% \fi
%
%
% \CharacterTable
%  {Upper-case    \A\B\C\D\E\F\G\H\I\J\K\L\M\N\O\P\Q\R\S\T\U\V\W\X\Y\Z
%   Lower-case    \a\b\c\d\e\f\g\h\i\j\k\l\m\n\o\p\q\r\s\t\u\v\w\x\y\z
%   Digits        \0\1\2\3\4\5\6\7\8\9
%   Exclamation   \!     Double quote  \"     Hash (number) \#
%   Dollar        \$     Percent       \%     Ampersand     \&
%   Acute accent  \'     Left paren    \(     Right paren   \)
%   Asterisk      \*     Plus          \+     Comma         \,
%   Minus         \-     Point         \.     Solidus       \/
%   Colon         \:     Semicolon     \;     Less than     \<
%   Equals        \=     Greater than  \>     Question mark \?
%   Commercial at \@     Left bracket  \[     Backslash     \\
%   Right bracket \]     Circumflex    \^     Underscore    \_
%   Grave accent  \`     Left brace    \{     Vertical bar  \|
%   Right brace   \}     Tilde         \~}
%
% \GetFileInfo{setouterhbox.drv}
%
% \title{The \xpackage{setouterhbox} package}
% \date{2016/05/16 v1.8}
% \author{Heiko Oberdiek\thanks
% {Please report any issues at \url{https://github.com/ho-tex/oberdiek/issues}}}
%
% \maketitle
%
% \begin{abstract}
% If math stuff is set in an \cs{hbox}, then TeX
% performs some optimization and omits the implicite
% penalties \cs{binoppenalty} and \cs{relpenalty}.
% This packages tries to put stuff into an \cs{hbox}
% without getting lost of those penalties.
% \end{abstract}
%
% \tableofcontents
%
% \section{Documentation}
%
% \subsection{Introduction}
%
% There is a situation in \xpackage{hyperref}'s driver for dvips
% where the user wants to have links that can be broken across
% lines. However dvips doesn't support the feature. With option
% \xoption{breaklinks} \xpackage{hyperref} sets the links as
% usual, put them in a box and write the link data with
% box dimensions into the appropriate \cs{special}s.
% Then, however, it does not set the complete unbreakable
% box, but it unwrappes the material inside to allow line
% breaks. Of course line breaking and glue setting will falsify
% the link dimensions, but line breaking was more important
% for the user.
%
% \subsection{Acknowledgement}
%
% Jonathan Fine, Donald Arsenau and me discussed the problem
% in the newsgroup \xnewsgroup{comp.text.tex} where Damian
% Menscher has started the thread, see \cite{newsstart}.
%
% The discussion was productive and generated many ideas
% and code examples. In order to have a more permanent
% result I wrote this package and tried to implement
% most of the ideas, a kind of summary of the discussion.
% Thus I want and have to thank Jonathan Fine and Donald Arsenau
% very much.
%
% Two weeks later David Kastrup (posting in
% \xnewsgroup{comp.text.tex}, \cite{kastrup})
% remembered an old article of Michael Downes (\cite{downes})
% in TUGboat, where Michael Downes already presented the
% method we discuss here. Nowadays we have \eTeX\ that extends
% the tool set of a \TeX\ macro programmer. Especially useful
% \eTeX\ was in this package for detecting and dealing with
% errorneous situations.
%
% However also nowadays a perfect solution for the problem
% is still missing at macro level. Probably someone has
% to go deep in the internals of the \TeX\ compiler to
% implement a switch that let penalties stay where otherwise
% \TeX\ would remove them for optimization reasons.
%
% \subsection{Usage}
%
% \paragraph{Package loading.}
% \LaTeX: as usually:
% \begin{quote}
%   |\usepackage{setouterhbox}|
% \end{quote}
% The package can also be included directly, thus \plainTeX\ users
% write:
% \begin{quote}
%   |\input setouterhbox.sty|
% \end{quote}
%
% \paragraph{Register allocation.}
% The material will be put into a box, thus we need to know these
% box number. If you need to allocate a new box register:
% \begin{description}
%  \item[\LaTeX:] |\newsavebox{\|\meta{name}|}|
%  \item[\plainTeX:] |\newbox\|\meta{name}
% \end{description}
% Then |\|\meta{name} is a command that held the box number.
%
% \paragraph{Box wrapping.}
% \LaTeX\ users put the material in the box with an environment
% similar to \texttt{lrbox}. The environment \texttt{setouterhbox}
% uses the same syntax and offers the same features, such
% as verbatim stuff inside:
% \begin{quote}
%  |\begin{setouterhbox}{|\meta{box number}|}|\dots
%  |\end{setouterhbox}|
% \end{quote}
% Users with \plainTeX\ do not have environments, they use instead:
% \begin{quote}
%   |\setouterhbox{|\meta{box number}|}|\dots|\endsetouterhbox|
% \end{quote}
% In both cases the material is put into an \cs{hbox} and assigned
% to the given box, denoted by \meta{box number}. Note the
% assignment is local, the same way \texttt{lrbox} behaves.
%
% \paragraph{Unwrapping.}
% The box material is ready for unwrapping:
% \begin{quote}
%   |\unhbox|\meta{box number}
% \end{quote}
%
% \subsection{Option \xoption{hyperref}}
%
% Package url uses math mode for typesetting urls.
% Break points are inserted by \cs{binoppenalty} and
% \cs{relpenalty}. Unhappily these break points are
% removed, if \xpackage{hyperref}
% is used with option {breaklinks}
% and drivers that depend on \xoption{pdfmark}:
% \xoption{dvips}, \xoption{vtexpdfmark}, \xoption{textures},
% and \xoption{dvipsone}.
% Thus the option \xoption{hyperref} enables the method
% of this package to avoid the removal of \cs{relpenalty}
% and \cs{binoppenalty}. Thus you get more break points.
% However, the link areas are still wrong for these
% drivers, because they are not supporting broken
% links.
%
% Note, you need version 2006/08/16 v6.75c of package \xpackage{hyperref},
% because starting with this version the necessary hook is provided
% that package \xpackage{setouterhbox} uses.
% \begin{quote}
%   |\usepackage[|\dots|]{hyperref}[2006/08/16]|\\
%   |\usepackage[hyperref]{setouterhbox}|
% \end{quote}
% Package order does not matter.
%
% \subsection{Example}
%
%    \begin{macrocode}
%<*example>
\documentclass[a5paper]{article}
\usepackage{url}[2005/06/27]
\usepackage{setouterhbox}

\newsavebox{\testbox}

\setlength{\parindent}{0pt}
\setlength{\parskip}{2em}

\begin{document}
\raggedright

\url{http://this.is.a.very.long.host.name/followed/%
by/a/very_long_long_long_path.html}%

\sbox\testbox{%
  \url{http://this.is.a.very.long.host.name/followed/%
  by/a/very_long_long_long_path.html}%
}%
\unhbox\testbox

\begin{setouterhbox}{\testbox}%
  \url{http://this.is.a.very.long.host.name/followed/%
  by/a/very_long_long_long_path.html}%
\end{setouterhbox}
\unhbox\testbox

\end{document}
%</example>
%    \end{macrocode}
%
% \StopEventually{
% }
%
% \section{Implementation}
%
% Internal macros are prefixed by \cs{setouterhbox}, |@| is
% not used inside names, thus we do not need to care of its
% catcode if we are not using it as \LaTeX\ package.
%
% \subsection{Package start stuff}
%
%    \begin{macrocode}
%<*package>
%    \end{macrocode}
%
% Prevent reloading more than one, necessary for \plainTeX:
%    Reload check, especially if the package is not used with \LaTeX.
%    \begin{macrocode}
\begingroup\catcode61\catcode48\catcode32=10\relax%
  \catcode13=5 % ^^M
  \endlinechar=13 %
  \catcode35=6 % #
  \catcode39=12 % '
  \catcode44=12 % ,
  \catcode45=12 % -
  \catcode46=12 % .
  \catcode58=12 % :
  \catcode64=11 % @
  \catcode123=1 % {
  \catcode125=2 % }
  \expandafter\let\expandafter\x\csname ver@setouterhbox.sty\endcsname
  \ifx\x\relax % plain-TeX, first loading
  \else
    \def\empty{}%
    \ifx\x\empty % LaTeX, first loading,
      % variable is initialized, but \ProvidesPackage not yet seen
    \else
      \expandafter\ifx\csname PackageInfo\endcsname\relax
        \def\x#1#2{%
          \immediate\write-1{Package #1 Info: #2.}%
        }%
      \else
        \def\x#1#2{\PackageInfo{#1}{#2, stopped}}%
      \fi
      \x{setouterhbox}{The package is already loaded}%
      \aftergroup\endinput
    \fi
  \fi
\endgroup%
%    \end{macrocode}
%    Package identification:
%    \begin{macrocode}
\begingroup\catcode61\catcode48\catcode32=10\relax%
  \catcode13=5 % ^^M
  \endlinechar=13 %
  \catcode35=6 % #
  \catcode39=12 % '
  \catcode40=12 % (
  \catcode41=12 % )
  \catcode44=12 % ,
  \catcode45=12 % -
  \catcode46=12 % .
  \catcode47=12 % /
  \catcode58=12 % :
  \catcode64=11 % @
  \catcode91=12 % [
  \catcode93=12 % ]
  \catcode123=1 % {
  \catcode125=2 % }
  \expandafter\ifx\csname ProvidesPackage\endcsname\relax
    \def\x#1#2#3[#4]{\endgroup
      \immediate\write-1{Package: #3 #4}%
      \xdef#1{#4}%
    }%
  \else
    \def\x#1#2[#3]{\endgroup
      #2[{#3}]%
      \ifx#1\@undefined
        \xdef#1{#3}%
      \fi
      \ifx#1\relax
        \xdef#1{#3}%
      \fi
    }%
  \fi
\expandafter\x\csname ver@setouterhbox.sty\endcsname
\ProvidesPackage{setouterhbox}%
  [2016/05/16 v1.8 Set hbox in outer horizontal mode (HO)]%
%    \end{macrocode}
%
%    \begin{macrocode}
\begingroup\catcode61\catcode48\catcode32=10\relax%
  \catcode13=5 % ^^M
  \endlinechar=13 %
  \catcode123=1 % {
  \catcode125=2 % }
  \catcode64=11 % @
  \def\x{\endgroup
    \expandafter\edef\csname setouterhboxAtEnd\endcsname{%
      \endlinechar=\the\endlinechar\relax
      \catcode13=\the\catcode13\relax
      \catcode32=\the\catcode32\relax
      \catcode35=\the\catcode35\relax
      \catcode61=\the\catcode61\relax
      \catcode64=\the\catcode64\relax
      \catcode123=\the\catcode123\relax
      \catcode125=\the\catcode125\relax
    }%
  }%
\x\catcode61\catcode48\catcode32=10\relax%
\catcode13=5 % ^^M
\endlinechar=13 %
\catcode35=6 % #
\catcode64=11 % @
\catcode123=1 % {
\catcode125=2 % }
\def\TMP@EnsureCode#1#2{%
  \edef\setouterhboxAtEnd{%
    \setouterhboxAtEnd
    \catcode#1=\the\catcode#1\relax
  }%
  \catcode#1=#2\relax
}
\TMP@EnsureCode{40}{12}% (
\TMP@EnsureCode{41}{12}% )
\TMP@EnsureCode{44}{12}% ,
\TMP@EnsureCode{45}{12}% -
\TMP@EnsureCode{46}{12}% .
\TMP@EnsureCode{47}{12}% /
\TMP@EnsureCode{58}{12}% :
\TMP@EnsureCode{60}{12}% <
\TMP@EnsureCode{62}{12}% >
\TMP@EnsureCode{91}{12}% [
\TMP@EnsureCode{93}{12}% ]
\TMP@EnsureCode{96}{12}% `
\edef\setouterhboxAtEnd{\setouterhboxAtEnd\noexpand\endinput}
%    \end{macrocode}
%
% \subsection{Interface macros}
%
%    \begin{macro}{\setouterhboxBox}
% The method requires a global box assignment. To be on the
% safe side, a new box register is allocated for this
% global box assignment.
%    \begin{macrocode}
\newbox\setouterhboxBox
%    \end{macrocode}
%    \end{macro}
%
%    \begin{macro}{\setouterhboxFailure}
% Error message for both \plainTeX\ and \LaTeX
%    \begin{macrocode}
\begingroup\expandafter\expandafter\expandafter\endgroup
\expandafter\ifx\csname RequirePackage\endcsname\relax
  \input infwarerr.sty\relax
\else
  \RequirePackage{infwarerr}[2016/05/16]%
\fi
\edef\setouterhboxFailure#1#2{%
  \expandafter\noexpand\csname @PackageError\endcsname
      {setouterhbox}{#1}{#2}%
}
%    \end{macrocode}
%    \end{macro}
%
% \subsection{Main part}
%
% eTeX provides much better means for checking
% error conditions. Thus lines marked by "E" are executed
% if eTeX is available, otherwise the lines marked by "T" are
% used.
%    \begin{macrocode}
\begingroup\expandafter\expandafter\expandafter\endgroup
\expandafter\ifx\csname lastnodetype\endcsname\relax
  \catcode`T=9 % ignore
  \catcode`E=14 % comment
\else
  \catcode`T=14 % comment
  \catcode`E=9 % ignore
\fi
%    \end{macrocode}
%
%    \begin{macro}{\setouterhboxRemove}
% Remove all kern, glue, and penalty nodes;
% poor man's version, if \eTeX\ is not available
%    \begin{macrocode}
\def\setouterhboxRemove{%
E \ifnum\lastnodetype<11 %
E   \else
E   \ifnum\lastnodetype>13 %
E   \else
      \unskip\unkern\unpenalty
E     \expandafter\expandafter\expandafter\setouterhboxRemove
E   \fi
E \fi
}%
%    \end{macrocode}
%    \end{macro}
%
%    \begin{macro}{\setouterhbox}
% Passing the box contents by macro parameter would prevent
% catcode changes in the box contents like by \cs{verb}.
% Also \cs{bgroup} and \cs{egroup} does not work, because stuff
% has to be added at the begin and end of the box, thus
% the syntax
% |\setouterhbox{|\meta{box number}|}|\dots|\endsetouterhbox|
% is used. Also we automatically get an environment \texttt{setouterhbox}
% if \LaTeX\ is used.
%    \begin{macrocode}
\def\setouterhbox#1{%
  \begingroup
    \def\setouterhboxNum{#1}%
    \setbox0\vbox\bgroup
T     \kern.123pt\relax % marker
T     \kern0pt\relax % removed by \setouterhboxRemove
      \begingroup
        \everypar{}%
        \noindent
}
%    \end{macrocode}
%    \end{macro}
%    \begin{macro}{\endsetouterhbox}
% Most of the work is done in the end part, thus the heart of
% the method follows:
%    \begin{macrocode}
\def\endsetouterhbox{%
      \endgroup
%    \end{macrocode}
% Omit the first pass to get the penalties
% of the second pass.
%    \begin{macrocode}
      \pretolerance-1 %
%    \end{macrocode}
%  We don't want a third pass with \cs{emergencystretch}.
%    \begin{macrocode}
      \tolerance10000 %
      \hsize\maxdimen
%    \end{macrocode}
% Line is not underfull:
%    \begin{macrocode}
      \parfillskip 0pt plus 1filll\relax
      \leftskip0pt\relax
%    \end{macrocode}
% Suppress underful \cs{hbox} warnings,
% is explicit line breaks are used.
%    \begin{macrocode}
      \rightskip0pt plus 1fil\relax
      \everypar{}%
%    \end{macrocode}
% Ensure that there is a paragraph and
% prevents \cs{endgraph} from eating terminal glue:
%    \begin{macrocode}
      \kern0pt%
      \endgraf
      \setouterhboxRemove
E     \ifnum\lastnodetype=1 %
E       \global\setbox\setouterhboxBox\lastbox
E       \loop
E         \setouterhboxRemove
E       \ifnum\lastnodetype=1 %
E         \setbox0=\lastbox
E         \global\setbox\setouterhboxBox=\hbox{%
E           \unhbox0 %
%    \end{macrocode}
% Remove \cs{rightskip}, a penalty with -10000 is part of the previous line.
%    \begin{macrocode}
E           \unskip
E           \unhbox\setouterhboxBox
E         }%
E       \repeat
E     \else
E       \setouterhboxFailure{%
E         Something is wrong%
E       }{%
E         Could not find expected line.%
E         \MessageBreak
E         (\string\lastnodetype: \number\lastnodetype, expected: 1)%
E       }%
E     \fi
E     \setouterhboxRemove
T     \global\setbox\setouterhboxBox\lastbox
T     \loop
T       \setouterhboxRemove
T       \setbox0=\lastbox
T     \ifcase\ifvoid0 1\else0\fi
T       \global\setbox\setouterhboxBox=\hbox{%
T         \unhbox0 %
%    \end{macrocode}
% Remove \cs{rightskip}, a penalty with -10000 is part of the previous line.
%    \begin{macrocode}
T         \unskip
T         \unhbox\setouterhboxBox
T       }%
T     \repeat
T     \ifdim.123pt=\lastkern
T     \else
T       \setouterhboxFailure{%
T         Something is wrong%
T       }{%
T         Unexpected stuff was detected before the line.%
T       }%
T     \fi
T   \egroup
T   \ifcase \ifnum\wd0=0 \else 1\fi
T           \ifdim\ht0=.123pt \else 1\fi
T           \ifnum\dp0=0 \else 1\fi
T           0 %
E     \ifnum\lastnodetype=-1 %
%    \end{macrocode}
% There was just one line that we have caught.
%    \begin{macrocode}
      \else
        \setouterhboxFailure{%
            Something is wrong%
        }{%
            After fetching the line there is more unexpected stuff.%
E           \MessageBreak
E           (\string\lastnodetype: \number\lastnodetype, expected: -1)%
        }%
      \fi
E   \egroup
  \expandafter\endgroup
  \expandafter\setouterhboxFinish\expandafter{%
    \number\setouterhboxNum
  }%
}
%    \end{macrocode}
%    \end{macro}
%
% \subsection{Environment support}
%
% Check \cs{@currenvir} for the case that \cs{setouterhbox}
% was called as environment. Then the box assignment
% must be put after the \cs{endgroup} of |\end{|\dots|}|.
%    \begin{macrocode}
\def\setouterhboxCurr{setouterhbox}
\def\setouterhboxLast#1{%
  \setbox#1\hbox{%
    \unhbox\setouterhboxBox
    \unskip % remove \rightskip glue
    \unskip % remove \parfillskip glue
    \unpenalty % remove paragraph ending \penalty 10000
    \unkern % remove explicit kern inserted above
  }%
}
%    \end{macrocode}
%    \begin{macro}{\setouterhboxFinish}
% |#1| is an explicit number.
%    \begin{macrocode}
\def\setouterhboxFinish#1{%
  \begingroup\expandafter\expandafter\expandafter\endgroup
  \expandafter\ifx\csname @currenvir\endcsname\setouterhboxCurr
    \aftergroup\setouterhboxLast
    \aftergroup{%
    \setouterhboxAfter #1\NIL
    \aftergroup}%
  \else
    \setouterhboxLast{#1}%
  \fi
}
%    \end{macrocode}
%    \end{macro}
%    \begin{macro}{\setouterhboxAfter}
% |#1| is an explicit number.
%    \begin{macrocode}
\def\setouterhboxAfter#1#2\NIL{%
  \aftergroup#1%
  \ifx\\#2\\%
  \else
    \setouterhboxReturnAfterFi{%
      \setouterhboxAfter#2\NIL
    }%
  \fi
}
%    \end{macrocode}
%    \end{macro}
%    \begin{macro}{\setouterhboxReturnAfterFi}
% A utility macro to get tail recursion.
%    \begin{macrocode}
\long\def\setouterhboxReturnAfterFi#1\fi{\fi#1}
%    \end{macrocode}
%    \end{macro}
% Restore catcodes we have need to distinguish between
% the implementation with and without \eTeX.
%    \begin{macrocode}
\catcode69=11\relax % E
\catcode84=11\relax % T
%    \end{macrocode}
%
% \subsection{Option \xoption{hyperref}}
%    \begin{macrocode}
\begingroup
  \def\x{LaTeX2e}%
\expandafter\endgroup
\ifx\x\fmtname
\else
  \expandafter\setouterhboxAtEnd
\fi%
%    \end{macrocode}
%    \begin{macro}{\Hy@setouterhbox}
% \cs{Hy@setouterhbox} is the internal hook that \xpackage{hyperref}
% uses since 2006/02/12 v6.75a.
%    \begin{macrocode}
\DeclareOption{hyperref}{%
  \long\def\Hy@setouterhbox#1#2{%
    \setouterhbox{#1}#2\endsetouterhbox
  }%
}
%    \end{macrocode}
%    \end{macro}
%    \begin{macrocode}
\ProcessOptions\relax
%    \end{macrocode}
%
%    \begin{macrocode}
\setouterhboxAtEnd%
%</package>
%    \end{macrocode}
%
% \section{Test}
%
% \subsection{Catcode checks for loading}
%
%    \begin{macrocode}
%<*test1>
%    \end{macrocode}
%    \begin{macrocode}
\catcode`\{=1 %
\catcode`\}=2 %
\catcode`\#=6 %
\catcode`\@=11 %
\expandafter\ifx\csname count@\endcsname\relax
  \countdef\count@=255 %
\fi
\expandafter\ifx\csname @gobble\endcsname\relax
  \long\def\@gobble#1{}%
\fi
\expandafter\ifx\csname @firstofone\endcsname\relax
  \long\def\@firstofone#1{#1}%
\fi
\expandafter\ifx\csname loop\endcsname\relax
  \expandafter\@firstofone
\else
  \expandafter\@gobble
\fi
{%
  \def\loop#1\repeat{%
    \def\body{#1}%
    \iterate
  }%
  \def\iterate{%
    \body
      \let\next\iterate
    \else
      \let\next\relax
    \fi
    \next
  }%
  \let\repeat=\fi
}%
\def\RestoreCatcodes{}
\count@=0 %
\loop
  \edef\RestoreCatcodes{%
    \RestoreCatcodes
    \catcode\the\count@=\the\catcode\count@\relax
  }%
\ifnum\count@<255 %
  \advance\count@ 1 %
\repeat

\def\RangeCatcodeInvalid#1#2{%
  \count@=#1\relax
  \loop
    \catcode\count@=15 %
  \ifnum\count@<#2\relax
    \advance\count@ 1 %
  \repeat
}
\def\RangeCatcodeCheck#1#2#3{%
  \count@=#1\relax
  \loop
    \ifnum#3=\catcode\count@
    \else
      \errmessage{%
        Character \the\count@\space
        with wrong catcode \the\catcode\count@\space
        instead of \number#3%
      }%
    \fi
  \ifnum\count@<#2\relax
    \advance\count@ 1 %
  \repeat
}
\def\space{ }
\expandafter\ifx\csname LoadCommand\endcsname\relax
  \def\LoadCommand{\input setouterhbox.sty\relax}%
\fi
\def\Test{%
  \RangeCatcodeInvalid{0}{47}%
  \RangeCatcodeInvalid{58}{64}%
  \RangeCatcodeInvalid{91}{96}%
  \RangeCatcodeInvalid{123}{255}%
  \catcode`\@=12 %
  \catcode`\\=0 %
  \catcode`\%=14 %
  \LoadCommand
  \RangeCatcodeCheck{0}{36}{15}%
  \RangeCatcodeCheck{37}{37}{14}%
  \RangeCatcodeCheck{38}{47}{15}%
  \RangeCatcodeCheck{48}{57}{12}%
  \RangeCatcodeCheck{58}{63}{15}%
  \RangeCatcodeCheck{64}{64}{12}%
  \RangeCatcodeCheck{65}{90}{11}%
  \RangeCatcodeCheck{91}{91}{15}%
  \RangeCatcodeCheck{92}{92}{0}%
  \RangeCatcodeCheck{93}{96}{15}%
  \RangeCatcodeCheck{97}{122}{11}%
  \RangeCatcodeCheck{123}{255}{15}%
  \RestoreCatcodes
}
\Test
\csname @@end\endcsname
\end
%    \end{macrocode}
%    \begin{macrocode}
%</test1>
%    \end{macrocode}
%
% \subsection{Test with package \xpackage{url}}
%
%    \begin{macrocode}
%<*test2>
\nofiles
\documentclass[a5paper]{article}
\usepackage{url}[2005/06/27]
\usepackage{setouterhbox}

\newsavebox{\testbox}

\setlength{\parindent}{0pt}
\setlength{\parskip}{2em}

\begin{document}
\raggedright

\url{http://this.is.a.very.long.host.name/followed/%
by/a/very_long_long_long_path.html}%

\sbox\testbox{%
  \url{http://this.is.a.very.long.host.name/followed/%
  by/a/very_long_long_long_path.html}%
}%
\unhbox\testbox

\begin{setouterhbox}{\testbox}%
  \url{http://this.is.a.very.long.host.name/followed/%
  by/a/very_long_long_long_path.html}%
\end{setouterhbox}
\unhbox\testbox

\end{document}
%</test2>
%    \end{macrocode}
%
% \section{Installation}
%
% \subsection{Download}
%
% \paragraph{Package.} This package is available on
% CTAN\footnote{\CTANpkg{setouterhbox}}:
% \begin{description}
% \item[\CTAN{macros/latex/contrib/oberdiek/setouterhbox.dtx}] The source file.
% \item[\CTAN{macros/latex/contrib/oberdiek/setouterhbox.pdf}] Documentation.
% \end{description}
%
%
% \paragraph{Bundle.} All the packages of the bundle `oberdiek'
% are also available in a TDS compliant ZIP archive. There
% the packages are already unpacked and the documentation files
% are generated. The files and directories obey the TDS standard.
% \begin{description}
% \item[\CTANinstall{install/macros/latex/contrib/oberdiek.tds.zip}]
% \end{description}
% \emph{TDS} refers to the standard ``A Directory Structure
% for \TeX\ Files'' (\CTAN{tds/tds.pdf}). Directories
% with \xfile{texmf} in their name are usually organized this way.
%
% \subsection{Bundle installation}
%
% \paragraph{Unpacking.} Unpack the \xfile{oberdiek.tds.zip} in the
% TDS tree (also known as \xfile{texmf} tree) of your choice.
% Example (linux):
% \begin{quote}
%   |unzip oberdiek.tds.zip -d ~/texmf|
% \end{quote}
%
% \subsection{Package installation}
%
% \paragraph{Unpacking.} The \xfile{.dtx} file is a self-extracting
% \docstrip\ archive. The files are extracted by running the
% \xfile{.dtx} through \plainTeX:
% \begin{quote}
%   \verb|tex setouterhbox.dtx|
% \end{quote}
%
% \paragraph{TDS.} Now the different files must be moved into
% the different directories in your installation TDS tree
% (also known as \xfile{texmf} tree):
% \begin{quote}
% \def\t{^^A
% \begin{tabular}{@{}>{\ttfamily}l@{ $\rightarrow$ }>{\ttfamily}l@{}}
%   setouterhbox.sty & tex/generic/oberdiek/setouterhbox.sty\\
%   setouterhbox.pdf & doc/latex/oberdiek/setouterhbox.pdf\\
%   setouterhbox-example.tex & doc/latex/oberdiek/setouterhbox-example.tex\\
%   test/setouterhbox-test1.tex & doc/latex/oberdiek/test/setouterhbox-test1.tex\\
%   test/setouterhbox-test2.tex & doc/latex/oberdiek/test/setouterhbox-test2.tex\\
%   setouterhbox.dtx & source/latex/oberdiek/setouterhbox.dtx\\
% \end{tabular}^^A
% }^^A
% \sbox0{\t}^^A
% \ifdim\wd0>\linewidth
%   \begingroup
%     \advance\linewidth by\leftmargin
%     \advance\linewidth by\rightmargin
%   \edef\x{\endgroup
%     \def\noexpand\lw{\the\linewidth}^^A
%   }\x
%   \def\lwbox{^^A
%     \leavevmode
%     \hbox to \linewidth{^^A
%       \kern-\leftmargin\relax
%       \hss
%       \usebox0
%       \hss
%       \kern-\rightmargin\relax
%     }^^A
%   }^^A
%   \ifdim\wd0>\lw
%     \sbox0{\small\t}^^A
%     \ifdim\wd0>\linewidth
%       \ifdim\wd0>\lw
%         \sbox0{\footnotesize\t}^^A
%         \ifdim\wd0>\linewidth
%           \ifdim\wd0>\lw
%             \sbox0{\scriptsize\t}^^A
%             \ifdim\wd0>\linewidth
%               \ifdim\wd0>\lw
%                 \sbox0{\tiny\t}^^A
%                 \ifdim\wd0>\linewidth
%                   \lwbox
%                 \else
%                   \usebox0
%                 \fi
%               \else
%                 \lwbox
%               \fi
%             \else
%               \usebox0
%             \fi
%           \else
%             \lwbox
%           \fi
%         \else
%           \usebox0
%         \fi
%       \else
%         \lwbox
%       \fi
%     \else
%       \usebox0
%     \fi
%   \else
%     \lwbox
%   \fi
% \else
%   \usebox0
% \fi
% \end{quote}
% If you have a \xfile{docstrip.cfg} that configures and enables \docstrip's
% TDS installing feature, then some files can already be in the right
% place, see the documentation of \docstrip.
%
% \subsection{Refresh file name databases}
%
% If your \TeX~distribution
% (\TeX\,Live, \mikTeX, \dots) relies on file name databases, you must refresh
% these. For example, \TeX\,Live\ users run \verb|texhash| or
% \verb|mktexlsr|.
%
% \subsection{Some details for the interested}
%
% \paragraph{Unpacking with \LaTeX.}
% The \xfile{.dtx} chooses its action depending on the format:
% \begin{description}
% \item[\plainTeX:] Run \docstrip\ and extract the files.
% \item[\LaTeX:] Generate the documentation.
% \end{description}
% If you insist on using \LaTeX\ for \docstrip\ (really,
% \docstrip\ does not need \LaTeX), then inform the autodetect routine
% about your intention:
% \begin{quote}
%   \verb|latex \let\install=y% \iffalse meta-comment
%
% File: setouterhbox.dtx
% Version: 2016/05/16 v1.8
% Info: Set hbox in outer horizontal mode
%
% Copyright (C)
%    2005-2007 Heiko Oberdiek
%    2016-2019 Oberdiek Package Support Group
%    https://github.com/ho-tex/oberdiek/issues
%
% This work may be distributed and/or modified under the
% conditions of the LaTeX Project Public License, either
% version 1.3c of this license or (at your option) any later
% version. This version of this license is in
%    https://www.latex-project.org/lppl/lppl-1-3c.txt
% and the latest version of this license is in
%    https://www.latex-project.org/lppl.txt
% and version 1.3 or later is part of all distributions of
% LaTeX version 2005/12/01 or later.
%
% This work has the LPPL maintenance status "maintained".
%
% The Current Maintainers of this work are
% Heiko Oberdiek and the Oberdiek Package Support Group
% https://github.com/ho-tex/oberdiek/issues
%
% The Base Interpreter refers to any `TeX-Format',
% because some files are installed in TDS:tex/generic//.
%
% This work consists of the main source file setouterhbox.dtx
% and the derived files
%    setouterhbox.sty, setouterhbox.pdf, setouterhbox.ins, setouterhbox.drv,
%    setouterhbox-example.tex, setouterhbox-test1.tex,
%    setouterhbox-test2.tex.
%
% Distribution:
%    CTAN:macros/latex/contrib/oberdiek/setouterhbox.dtx
%    CTAN:macros/latex/contrib/oberdiek/setouterhbox.pdf
%
% Unpacking:
%    (a) If setouterhbox.ins is present:
%           tex setouterhbox.ins
%    (b) Without setouterhbox.ins:
%           tex setouterhbox.dtx
%    (c) If you insist on using LaTeX
%           latex \let\install=y\input{setouterhbox.dtx}
%        (quote the arguments according to the demands of your shell)
%
% Documentation:
%    (a) If setouterhbox.drv is present:
%           latex setouterhbox.drv
%    (b) Without setouterhbox.drv:
%           latex setouterhbox.dtx; ...
%    The class ltxdoc loads the configuration file ltxdoc.cfg
%    if available. Here you can specify further options, e.g.
%    use A4 as paper format:
%       \PassOptionsToClass{a4paper}{article}
%
%    Programm calls to get the documentation (example):
%       pdflatex setouterhbox.dtx
%       makeindex -s gind.ist setouterhbox.idx
%       pdflatex setouterhbox.dtx
%       makeindex -s gind.ist setouterhbox.idx
%       pdflatex setouterhbox.dtx
%
% Installation:
%    TDS:tex/generic/oberdiek/setouterhbox.sty
%    TDS:doc/latex/oberdiek/setouterhbox.pdf
%    TDS:doc/latex/oberdiek/setouterhbox-example.tex
%    TDS:doc/latex/oberdiek/test/setouterhbox-test1.tex
%    TDS:doc/latex/oberdiek/test/setouterhbox-test2.tex
%    TDS:source/latex/oberdiek/setouterhbox.dtx
%
%<*ignore>
\begingroup
  \catcode123=1 %
  \catcode125=2 %
  \def\x{LaTeX2e}%
\expandafter\endgroup
\ifcase 0\ifx\install y1\fi\expandafter
         \ifx\csname processbatchFile\endcsname\relax\else1\fi
         \ifx\fmtname\x\else 1\fi\relax
\else\csname fi\endcsname
%</ignore>
%<*install>
\input docstrip.tex
\Msg{************************************************************************}
\Msg{* Installation}
\Msg{* Package: setouterhbox 2016/05/16 v1.8 Set hbox in outer horizontal mode (HO)}
\Msg{************************************************************************}

\keepsilent
\askforoverwritefalse

\let\MetaPrefix\relax
\preamble

This is a generated file.

Project: setouterhbox
Version: 2016/05/16 v1.8

Copyright (C)
   2005-2007 Heiko Oberdiek
   2016-2019 Oberdiek Package Support Group

This work may be distributed and/or modified under the
conditions of the LaTeX Project Public License, either
version 1.3c of this license or (at your option) any later
version. This version of this license is in
   https://www.latex-project.org/lppl/lppl-1-3c.txt
and the latest version of this license is in
   https://www.latex-project.org/lppl.txt
and version 1.3 or later is part of all distributions of
LaTeX version 2005/12/01 or later.

This work has the LPPL maintenance status "maintained".

The Current Maintainers of this work are
Heiko Oberdiek and the Oberdiek Package Support Group
https://github.com/ho-tex/oberdiek/issues


The Base Interpreter refers to any `TeX-Format',
because some files are installed in TDS:tex/generic//.

This work consists of the main source file setouterhbox.dtx
and the derived files
   setouterhbox.sty, setouterhbox.pdf, setouterhbox.ins, setouterhbox.drv,
   setouterhbox-example.tex, setouterhbox-test1.tex,
   setouterhbox-test2.tex.

\endpreamble
\let\MetaPrefix\DoubleperCent

\generate{%
  \file{setouterhbox.ins}{\from{setouterhbox.dtx}{install}}%
  \file{setouterhbox.drv}{\from{setouterhbox.dtx}{driver}}%
  \usedir{tex/generic/oberdiek}%
  \file{setouterhbox.sty}{\from{setouterhbox.dtx}{package}}%
  \usedir{doc/latex/oberdiek}%
  \file{setouterhbox-example.tex}{\from{setouterhbox.dtx}{example}}%
%  \usedir{doc/latex/oberdiek/test}%
%  \file{setouterhbox-test1.tex}{\from{setouterhbox.dtx}{test1}}%
%  \file{setouterhbox-test2.tex}{\from{setouterhbox.dtx}{test2}}%
  \nopreamble
  \nopostamble
%  \usedir{source/latex/oberdiek/catalogue}%
%  \file{setouterhbox.xml}{\from{setouterhbox.dtx}{catalogue}}%
}

\catcode32=13\relax% active space
\let =\space%
\Msg{************************************************************************}
\Msg{*}
\Msg{* To finish the installation you have to move the following}
\Msg{* file into a directory searched by TeX:}
\Msg{*}
\Msg{*     setouterhbox.sty}
\Msg{*}
\Msg{* To produce the documentation run the file `setouterhbox.drv'}
\Msg{* through LaTeX.}
\Msg{*}
\Msg{* Happy TeXing!}
\Msg{*}
\Msg{************************************************************************}

\endbatchfile
%</install>
%<*ignore>
\fi
%</ignore>
%<*driver>
\NeedsTeXFormat{LaTeX2e}
\ProvidesFile{setouterhbox.drv}%
  [2016/05/16 v1.8 Set hbox in outer horizontal mode (HO)]%
\documentclass{ltxdoc}
\usepackage{holtxdoc}[2011/11/22]
\begin{document}
  \DocInput{setouterhbox.dtx}%
\end{document}
%</driver>
% \fi
%
%
% \CharacterTable
%  {Upper-case    \A\B\C\D\E\F\G\H\I\J\K\L\M\N\O\P\Q\R\S\T\U\V\W\X\Y\Z
%   Lower-case    \a\b\c\d\e\f\g\h\i\j\k\l\m\n\o\p\q\r\s\t\u\v\w\x\y\z
%   Digits        \0\1\2\3\4\5\6\7\8\9
%   Exclamation   \!     Double quote  \"     Hash (number) \#
%   Dollar        \$     Percent       \%     Ampersand     \&
%   Acute accent  \'     Left paren    \(     Right paren   \)
%   Asterisk      \*     Plus          \+     Comma         \,
%   Minus         \-     Point         \.     Solidus       \/
%   Colon         \:     Semicolon     \;     Less than     \<
%   Equals        \=     Greater than  \>     Question mark \?
%   Commercial at \@     Left bracket  \[     Backslash     \\
%   Right bracket \]     Circumflex    \^     Underscore    \_
%   Grave accent  \`     Left brace    \{     Vertical bar  \|
%   Right brace   \}     Tilde         \~}
%
% \GetFileInfo{setouterhbox.drv}
%
% \title{The \xpackage{setouterhbox} package}
% \date{2016/05/16 v1.8}
% \author{Heiko Oberdiek\thanks
% {Please report any issues at \url{https://github.com/ho-tex/oberdiek/issues}}}
%
% \maketitle
%
% \begin{abstract}
% If math stuff is set in an \cs{hbox}, then TeX
% performs some optimization and omits the implicite
% penalties \cs{binoppenalty} and \cs{relpenalty}.
% This packages tries to put stuff into an \cs{hbox}
% without getting lost of those penalties.
% \end{abstract}
%
% \tableofcontents
%
% \section{Documentation}
%
% \subsection{Introduction}
%
% There is a situation in \xpackage{hyperref}'s driver for dvips
% where the user wants to have links that can be broken across
% lines. However dvips doesn't support the feature. With option
% \xoption{breaklinks} \xpackage{hyperref} sets the links as
% usual, put them in a box and write the link data with
% box dimensions into the appropriate \cs{special}s.
% Then, however, it does not set the complete unbreakable
% box, but it unwrappes the material inside to allow line
% breaks. Of course line breaking and glue setting will falsify
% the link dimensions, but line breaking was more important
% for the user.
%
% \subsection{Acknowledgement}
%
% Jonathan Fine, Donald Arsenau and me discussed the problem
% in the newsgroup \xnewsgroup{comp.text.tex} where Damian
% Menscher has started the thread, see \cite{newsstart}.
%
% The discussion was productive and generated many ideas
% and code examples. In order to have a more permanent
% result I wrote this package and tried to implement
% most of the ideas, a kind of summary of the discussion.
% Thus I want and have to thank Jonathan Fine and Donald Arsenau
% very much.
%
% Two weeks later David Kastrup (posting in
% \xnewsgroup{comp.text.tex}, \cite{kastrup})
% remembered an old article of Michael Downes (\cite{downes})
% in TUGboat, where Michael Downes already presented the
% method we discuss here. Nowadays we have \eTeX\ that extends
% the tool set of a \TeX\ macro programmer. Especially useful
% \eTeX\ was in this package for detecting and dealing with
% errorneous situations.
%
% However also nowadays a perfect solution for the problem
% is still missing at macro level. Probably someone has
% to go deep in the internals of the \TeX\ compiler to
% implement a switch that let penalties stay where otherwise
% \TeX\ would remove them for optimization reasons.
%
% \subsection{Usage}
%
% \paragraph{Package loading.}
% \LaTeX: as usually:
% \begin{quote}
%   |\usepackage{setouterhbox}|
% \end{quote}
% The package can also be included directly, thus \plainTeX\ users
% write:
% \begin{quote}
%   |\input setouterhbox.sty|
% \end{quote}
%
% \paragraph{Register allocation.}
% The material will be put into a box, thus we need to know these
% box number. If you need to allocate a new box register:
% \begin{description}
%  \item[\LaTeX:] |\newsavebox{\|\meta{name}|}|
%  \item[\plainTeX:] |\newbox\|\meta{name}
% \end{description}
% Then |\|\meta{name} is a command that held the box number.
%
% \paragraph{Box wrapping.}
% \LaTeX\ users put the material in the box with an environment
% similar to \texttt{lrbox}. The environment \texttt{setouterhbox}
% uses the same syntax and offers the same features, such
% as verbatim stuff inside:
% \begin{quote}
%  |\begin{setouterhbox}{|\meta{box number}|}|\dots
%  |\end{setouterhbox}|
% \end{quote}
% Users with \plainTeX\ do not have environments, they use instead:
% \begin{quote}
%   |\setouterhbox{|\meta{box number}|}|\dots|\endsetouterhbox|
% \end{quote}
% In both cases the material is put into an \cs{hbox} and assigned
% to the given box, denoted by \meta{box number}. Note the
% assignment is local, the same way \texttt{lrbox} behaves.
%
% \paragraph{Unwrapping.}
% The box material is ready for unwrapping:
% \begin{quote}
%   |\unhbox|\meta{box number}
% \end{quote}
%
% \subsection{Option \xoption{hyperref}}
%
% Package url uses math mode for typesetting urls.
% Break points are inserted by \cs{binoppenalty} and
% \cs{relpenalty}. Unhappily these break points are
% removed, if \xpackage{hyperref}
% is used with option {breaklinks}
% and drivers that depend on \xoption{pdfmark}:
% \xoption{dvips}, \xoption{vtexpdfmark}, \xoption{textures},
% and \xoption{dvipsone}.
% Thus the option \xoption{hyperref} enables the method
% of this package to avoid the removal of \cs{relpenalty}
% and \cs{binoppenalty}. Thus you get more break points.
% However, the link areas are still wrong for these
% drivers, because they are not supporting broken
% links.
%
% Note, you need version 2006/08/16 v6.75c of package \xpackage{hyperref},
% because starting with this version the necessary hook is provided
% that package \xpackage{setouterhbox} uses.
% \begin{quote}
%   |\usepackage[|\dots|]{hyperref}[2006/08/16]|\\
%   |\usepackage[hyperref]{setouterhbox}|
% \end{quote}
% Package order does not matter.
%
% \subsection{Example}
%
%    \begin{macrocode}
%<*example>
\documentclass[a5paper]{article}
\usepackage{url}[2005/06/27]
\usepackage{setouterhbox}

\newsavebox{\testbox}

\setlength{\parindent}{0pt}
\setlength{\parskip}{2em}

\begin{document}
\raggedright

\url{http://this.is.a.very.long.host.name/followed/%
by/a/very_long_long_long_path.html}%

\sbox\testbox{%
  \url{http://this.is.a.very.long.host.name/followed/%
  by/a/very_long_long_long_path.html}%
}%
\unhbox\testbox

\begin{setouterhbox}{\testbox}%
  \url{http://this.is.a.very.long.host.name/followed/%
  by/a/very_long_long_long_path.html}%
\end{setouterhbox}
\unhbox\testbox

\end{document}
%</example>
%    \end{macrocode}
%
% \StopEventually{
% }
%
% \section{Implementation}
%
% Internal macros are prefixed by \cs{setouterhbox}, |@| is
% not used inside names, thus we do not need to care of its
% catcode if we are not using it as \LaTeX\ package.
%
% \subsection{Package start stuff}
%
%    \begin{macrocode}
%<*package>
%    \end{macrocode}
%
% Prevent reloading more than one, necessary for \plainTeX:
%    Reload check, especially if the package is not used with \LaTeX.
%    \begin{macrocode}
\begingroup\catcode61\catcode48\catcode32=10\relax%
  \catcode13=5 % ^^M
  \endlinechar=13 %
  \catcode35=6 % #
  \catcode39=12 % '
  \catcode44=12 % ,
  \catcode45=12 % -
  \catcode46=12 % .
  \catcode58=12 % :
  \catcode64=11 % @
  \catcode123=1 % {
  \catcode125=2 % }
  \expandafter\let\expandafter\x\csname ver@setouterhbox.sty\endcsname
  \ifx\x\relax % plain-TeX, first loading
  \else
    \def\empty{}%
    \ifx\x\empty % LaTeX, first loading,
      % variable is initialized, but \ProvidesPackage not yet seen
    \else
      \expandafter\ifx\csname PackageInfo\endcsname\relax
        \def\x#1#2{%
          \immediate\write-1{Package #1 Info: #2.}%
        }%
      \else
        \def\x#1#2{\PackageInfo{#1}{#2, stopped}}%
      \fi
      \x{setouterhbox}{The package is already loaded}%
      \aftergroup\endinput
    \fi
  \fi
\endgroup%
%    \end{macrocode}
%    Package identification:
%    \begin{macrocode}
\begingroup\catcode61\catcode48\catcode32=10\relax%
  \catcode13=5 % ^^M
  \endlinechar=13 %
  \catcode35=6 % #
  \catcode39=12 % '
  \catcode40=12 % (
  \catcode41=12 % )
  \catcode44=12 % ,
  \catcode45=12 % -
  \catcode46=12 % .
  \catcode47=12 % /
  \catcode58=12 % :
  \catcode64=11 % @
  \catcode91=12 % [
  \catcode93=12 % ]
  \catcode123=1 % {
  \catcode125=2 % }
  \expandafter\ifx\csname ProvidesPackage\endcsname\relax
    \def\x#1#2#3[#4]{\endgroup
      \immediate\write-1{Package: #3 #4}%
      \xdef#1{#4}%
    }%
  \else
    \def\x#1#2[#3]{\endgroup
      #2[{#3}]%
      \ifx#1\@undefined
        \xdef#1{#3}%
      \fi
      \ifx#1\relax
        \xdef#1{#3}%
      \fi
    }%
  \fi
\expandafter\x\csname ver@setouterhbox.sty\endcsname
\ProvidesPackage{setouterhbox}%
  [2016/05/16 v1.8 Set hbox in outer horizontal mode (HO)]%
%    \end{macrocode}
%
%    \begin{macrocode}
\begingroup\catcode61\catcode48\catcode32=10\relax%
  \catcode13=5 % ^^M
  \endlinechar=13 %
  \catcode123=1 % {
  \catcode125=2 % }
  \catcode64=11 % @
  \def\x{\endgroup
    \expandafter\edef\csname setouterhboxAtEnd\endcsname{%
      \endlinechar=\the\endlinechar\relax
      \catcode13=\the\catcode13\relax
      \catcode32=\the\catcode32\relax
      \catcode35=\the\catcode35\relax
      \catcode61=\the\catcode61\relax
      \catcode64=\the\catcode64\relax
      \catcode123=\the\catcode123\relax
      \catcode125=\the\catcode125\relax
    }%
  }%
\x\catcode61\catcode48\catcode32=10\relax%
\catcode13=5 % ^^M
\endlinechar=13 %
\catcode35=6 % #
\catcode64=11 % @
\catcode123=1 % {
\catcode125=2 % }
\def\TMP@EnsureCode#1#2{%
  \edef\setouterhboxAtEnd{%
    \setouterhboxAtEnd
    \catcode#1=\the\catcode#1\relax
  }%
  \catcode#1=#2\relax
}
\TMP@EnsureCode{40}{12}% (
\TMP@EnsureCode{41}{12}% )
\TMP@EnsureCode{44}{12}% ,
\TMP@EnsureCode{45}{12}% -
\TMP@EnsureCode{46}{12}% .
\TMP@EnsureCode{47}{12}% /
\TMP@EnsureCode{58}{12}% :
\TMP@EnsureCode{60}{12}% <
\TMP@EnsureCode{62}{12}% >
\TMP@EnsureCode{91}{12}% [
\TMP@EnsureCode{93}{12}% ]
\TMP@EnsureCode{96}{12}% `
\edef\setouterhboxAtEnd{\setouterhboxAtEnd\noexpand\endinput}
%    \end{macrocode}
%
% \subsection{Interface macros}
%
%    \begin{macro}{\setouterhboxBox}
% The method requires a global box assignment. To be on the
% safe side, a new box register is allocated for this
% global box assignment.
%    \begin{macrocode}
\newbox\setouterhboxBox
%    \end{macrocode}
%    \end{macro}
%
%    \begin{macro}{\setouterhboxFailure}
% Error message for both \plainTeX\ and \LaTeX
%    \begin{macrocode}
\begingroup\expandafter\expandafter\expandafter\endgroup
\expandafter\ifx\csname RequirePackage\endcsname\relax
  \input infwarerr.sty\relax
\else
  \RequirePackage{infwarerr}[2016/05/16]%
\fi
\edef\setouterhboxFailure#1#2{%
  \expandafter\noexpand\csname @PackageError\endcsname
      {setouterhbox}{#1}{#2}%
}
%    \end{macrocode}
%    \end{macro}
%
% \subsection{Main part}
%
% eTeX provides much better means for checking
% error conditions. Thus lines marked by "E" are executed
% if eTeX is available, otherwise the lines marked by "T" are
% used.
%    \begin{macrocode}
\begingroup\expandafter\expandafter\expandafter\endgroup
\expandafter\ifx\csname lastnodetype\endcsname\relax
  \catcode`T=9 % ignore
  \catcode`E=14 % comment
\else
  \catcode`T=14 % comment
  \catcode`E=9 % ignore
\fi
%    \end{macrocode}
%
%    \begin{macro}{\setouterhboxRemove}
% Remove all kern, glue, and penalty nodes;
% poor man's version, if \eTeX\ is not available
%    \begin{macrocode}
\def\setouterhboxRemove{%
E \ifnum\lastnodetype<11 %
E   \else
E   \ifnum\lastnodetype>13 %
E   \else
      \unskip\unkern\unpenalty
E     \expandafter\expandafter\expandafter\setouterhboxRemove
E   \fi
E \fi
}%
%    \end{macrocode}
%    \end{macro}
%
%    \begin{macro}{\setouterhbox}
% Passing the box contents by macro parameter would prevent
% catcode changes in the box contents like by \cs{verb}.
% Also \cs{bgroup} and \cs{egroup} does not work, because stuff
% has to be added at the begin and end of the box, thus
% the syntax
% |\setouterhbox{|\meta{box number}|}|\dots|\endsetouterhbox|
% is used. Also we automatically get an environment \texttt{setouterhbox}
% if \LaTeX\ is used.
%    \begin{macrocode}
\def\setouterhbox#1{%
  \begingroup
    \def\setouterhboxNum{#1}%
    \setbox0\vbox\bgroup
T     \kern.123pt\relax % marker
T     \kern0pt\relax % removed by \setouterhboxRemove
      \begingroup
        \everypar{}%
        \noindent
}
%    \end{macrocode}
%    \end{macro}
%    \begin{macro}{\endsetouterhbox}
% Most of the work is done in the end part, thus the heart of
% the method follows:
%    \begin{macrocode}
\def\endsetouterhbox{%
      \endgroup
%    \end{macrocode}
% Omit the first pass to get the penalties
% of the second pass.
%    \begin{macrocode}
      \pretolerance-1 %
%    \end{macrocode}
%  We don't want a third pass with \cs{emergencystretch}.
%    \begin{macrocode}
      \tolerance10000 %
      \hsize\maxdimen
%    \end{macrocode}
% Line is not underfull:
%    \begin{macrocode}
      \parfillskip 0pt plus 1filll\relax
      \leftskip0pt\relax
%    \end{macrocode}
% Suppress underful \cs{hbox} warnings,
% is explicit line breaks are used.
%    \begin{macrocode}
      \rightskip0pt plus 1fil\relax
      \everypar{}%
%    \end{macrocode}
% Ensure that there is a paragraph and
% prevents \cs{endgraph} from eating terminal glue:
%    \begin{macrocode}
      \kern0pt%
      \endgraf
      \setouterhboxRemove
E     \ifnum\lastnodetype=1 %
E       \global\setbox\setouterhboxBox\lastbox
E       \loop
E         \setouterhboxRemove
E       \ifnum\lastnodetype=1 %
E         \setbox0=\lastbox
E         \global\setbox\setouterhboxBox=\hbox{%
E           \unhbox0 %
%    \end{macrocode}
% Remove \cs{rightskip}, a penalty with -10000 is part of the previous line.
%    \begin{macrocode}
E           \unskip
E           \unhbox\setouterhboxBox
E         }%
E       \repeat
E     \else
E       \setouterhboxFailure{%
E         Something is wrong%
E       }{%
E         Could not find expected line.%
E         \MessageBreak
E         (\string\lastnodetype: \number\lastnodetype, expected: 1)%
E       }%
E     \fi
E     \setouterhboxRemove
T     \global\setbox\setouterhboxBox\lastbox
T     \loop
T       \setouterhboxRemove
T       \setbox0=\lastbox
T     \ifcase\ifvoid0 1\else0\fi
T       \global\setbox\setouterhboxBox=\hbox{%
T         \unhbox0 %
%    \end{macrocode}
% Remove \cs{rightskip}, a penalty with -10000 is part of the previous line.
%    \begin{macrocode}
T         \unskip
T         \unhbox\setouterhboxBox
T       }%
T     \repeat
T     \ifdim.123pt=\lastkern
T     \else
T       \setouterhboxFailure{%
T         Something is wrong%
T       }{%
T         Unexpected stuff was detected before the line.%
T       }%
T     \fi
T   \egroup
T   \ifcase \ifnum\wd0=0 \else 1\fi
T           \ifdim\ht0=.123pt \else 1\fi
T           \ifnum\dp0=0 \else 1\fi
T           0 %
E     \ifnum\lastnodetype=-1 %
%    \end{macrocode}
% There was just one line that we have caught.
%    \begin{macrocode}
      \else
        \setouterhboxFailure{%
            Something is wrong%
        }{%
            After fetching the line there is more unexpected stuff.%
E           \MessageBreak
E           (\string\lastnodetype: \number\lastnodetype, expected: -1)%
        }%
      \fi
E   \egroup
  \expandafter\endgroup
  \expandafter\setouterhboxFinish\expandafter{%
    \number\setouterhboxNum
  }%
}
%    \end{macrocode}
%    \end{macro}
%
% \subsection{Environment support}
%
% Check \cs{@currenvir} for the case that \cs{setouterhbox}
% was called as environment. Then the box assignment
% must be put after the \cs{endgroup} of |\end{|\dots|}|.
%    \begin{macrocode}
\def\setouterhboxCurr{setouterhbox}
\def\setouterhboxLast#1{%
  \setbox#1\hbox{%
    \unhbox\setouterhboxBox
    \unskip % remove \rightskip glue
    \unskip % remove \parfillskip glue
    \unpenalty % remove paragraph ending \penalty 10000
    \unkern % remove explicit kern inserted above
  }%
}
%    \end{macrocode}
%    \begin{macro}{\setouterhboxFinish}
% |#1| is an explicit number.
%    \begin{macrocode}
\def\setouterhboxFinish#1{%
  \begingroup\expandafter\expandafter\expandafter\endgroup
  \expandafter\ifx\csname @currenvir\endcsname\setouterhboxCurr
    \aftergroup\setouterhboxLast
    \aftergroup{%
    \setouterhboxAfter #1\NIL
    \aftergroup}%
  \else
    \setouterhboxLast{#1}%
  \fi
}
%    \end{macrocode}
%    \end{macro}
%    \begin{macro}{\setouterhboxAfter}
% |#1| is an explicit number.
%    \begin{macrocode}
\def\setouterhboxAfter#1#2\NIL{%
  \aftergroup#1%
  \ifx\\#2\\%
  \else
    \setouterhboxReturnAfterFi{%
      \setouterhboxAfter#2\NIL
    }%
  \fi
}
%    \end{macrocode}
%    \end{macro}
%    \begin{macro}{\setouterhboxReturnAfterFi}
% A utility macro to get tail recursion.
%    \begin{macrocode}
\long\def\setouterhboxReturnAfterFi#1\fi{\fi#1}
%    \end{macrocode}
%    \end{macro}
% Restore catcodes we have need to distinguish between
% the implementation with and without \eTeX.
%    \begin{macrocode}
\catcode69=11\relax % E
\catcode84=11\relax % T
%    \end{macrocode}
%
% \subsection{Option \xoption{hyperref}}
%    \begin{macrocode}
\begingroup
  \def\x{LaTeX2e}%
\expandafter\endgroup
\ifx\x\fmtname
\else
  \expandafter\setouterhboxAtEnd
\fi%
%    \end{macrocode}
%    \begin{macro}{\Hy@setouterhbox}
% \cs{Hy@setouterhbox} is the internal hook that \xpackage{hyperref}
% uses since 2006/02/12 v6.75a.
%    \begin{macrocode}
\DeclareOption{hyperref}{%
  \long\def\Hy@setouterhbox#1#2{%
    \setouterhbox{#1}#2\endsetouterhbox
  }%
}
%    \end{macrocode}
%    \end{macro}
%    \begin{macrocode}
\ProcessOptions\relax
%    \end{macrocode}
%
%    \begin{macrocode}
\setouterhboxAtEnd%
%</package>
%    \end{macrocode}
%
% \section{Test}
%
% \subsection{Catcode checks for loading}
%
%    \begin{macrocode}
%<*test1>
%    \end{macrocode}
%    \begin{macrocode}
\catcode`\{=1 %
\catcode`\}=2 %
\catcode`\#=6 %
\catcode`\@=11 %
\expandafter\ifx\csname count@\endcsname\relax
  \countdef\count@=255 %
\fi
\expandafter\ifx\csname @gobble\endcsname\relax
  \long\def\@gobble#1{}%
\fi
\expandafter\ifx\csname @firstofone\endcsname\relax
  \long\def\@firstofone#1{#1}%
\fi
\expandafter\ifx\csname loop\endcsname\relax
  \expandafter\@firstofone
\else
  \expandafter\@gobble
\fi
{%
  \def\loop#1\repeat{%
    \def\body{#1}%
    \iterate
  }%
  \def\iterate{%
    \body
      \let\next\iterate
    \else
      \let\next\relax
    \fi
    \next
  }%
  \let\repeat=\fi
}%
\def\RestoreCatcodes{}
\count@=0 %
\loop
  \edef\RestoreCatcodes{%
    \RestoreCatcodes
    \catcode\the\count@=\the\catcode\count@\relax
  }%
\ifnum\count@<255 %
  \advance\count@ 1 %
\repeat

\def\RangeCatcodeInvalid#1#2{%
  \count@=#1\relax
  \loop
    \catcode\count@=15 %
  \ifnum\count@<#2\relax
    \advance\count@ 1 %
  \repeat
}
\def\RangeCatcodeCheck#1#2#3{%
  \count@=#1\relax
  \loop
    \ifnum#3=\catcode\count@
    \else
      \errmessage{%
        Character \the\count@\space
        with wrong catcode \the\catcode\count@\space
        instead of \number#3%
      }%
    \fi
  \ifnum\count@<#2\relax
    \advance\count@ 1 %
  \repeat
}
\def\space{ }
\expandafter\ifx\csname LoadCommand\endcsname\relax
  \def\LoadCommand{\input setouterhbox.sty\relax}%
\fi
\def\Test{%
  \RangeCatcodeInvalid{0}{47}%
  \RangeCatcodeInvalid{58}{64}%
  \RangeCatcodeInvalid{91}{96}%
  \RangeCatcodeInvalid{123}{255}%
  \catcode`\@=12 %
  \catcode`\\=0 %
  \catcode`\%=14 %
  \LoadCommand
  \RangeCatcodeCheck{0}{36}{15}%
  \RangeCatcodeCheck{37}{37}{14}%
  \RangeCatcodeCheck{38}{47}{15}%
  \RangeCatcodeCheck{48}{57}{12}%
  \RangeCatcodeCheck{58}{63}{15}%
  \RangeCatcodeCheck{64}{64}{12}%
  \RangeCatcodeCheck{65}{90}{11}%
  \RangeCatcodeCheck{91}{91}{15}%
  \RangeCatcodeCheck{92}{92}{0}%
  \RangeCatcodeCheck{93}{96}{15}%
  \RangeCatcodeCheck{97}{122}{11}%
  \RangeCatcodeCheck{123}{255}{15}%
  \RestoreCatcodes
}
\Test
\csname @@end\endcsname
\end
%    \end{macrocode}
%    \begin{macrocode}
%</test1>
%    \end{macrocode}
%
% \subsection{Test with package \xpackage{url}}
%
%    \begin{macrocode}
%<*test2>
\nofiles
\documentclass[a5paper]{article}
\usepackage{url}[2005/06/27]
\usepackage{setouterhbox}

\newsavebox{\testbox}

\setlength{\parindent}{0pt}
\setlength{\parskip}{2em}

\begin{document}
\raggedright

\url{http://this.is.a.very.long.host.name/followed/%
by/a/very_long_long_long_path.html}%

\sbox\testbox{%
  \url{http://this.is.a.very.long.host.name/followed/%
  by/a/very_long_long_long_path.html}%
}%
\unhbox\testbox

\begin{setouterhbox}{\testbox}%
  \url{http://this.is.a.very.long.host.name/followed/%
  by/a/very_long_long_long_path.html}%
\end{setouterhbox}
\unhbox\testbox

\end{document}
%</test2>
%    \end{macrocode}
%
% \section{Installation}
%
% \subsection{Download}
%
% \paragraph{Package.} This package is available on
% CTAN\footnote{\CTANpkg{setouterhbox}}:
% \begin{description}
% \item[\CTAN{macros/latex/contrib/oberdiek/setouterhbox.dtx}] The source file.
% \item[\CTAN{macros/latex/contrib/oberdiek/setouterhbox.pdf}] Documentation.
% \end{description}
%
%
% \paragraph{Bundle.} All the packages of the bundle `oberdiek'
% are also available in a TDS compliant ZIP archive. There
% the packages are already unpacked and the documentation files
% are generated. The files and directories obey the TDS standard.
% \begin{description}
% \item[\CTANinstall{install/macros/latex/contrib/oberdiek.tds.zip}]
% \end{description}
% \emph{TDS} refers to the standard ``A Directory Structure
% for \TeX\ Files'' (\CTAN{tds/tds.pdf}). Directories
% with \xfile{texmf} in their name are usually organized this way.
%
% \subsection{Bundle installation}
%
% \paragraph{Unpacking.} Unpack the \xfile{oberdiek.tds.zip} in the
% TDS tree (also known as \xfile{texmf} tree) of your choice.
% Example (linux):
% \begin{quote}
%   |unzip oberdiek.tds.zip -d ~/texmf|
% \end{quote}
%
% \subsection{Package installation}
%
% \paragraph{Unpacking.} The \xfile{.dtx} file is a self-extracting
% \docstrip\ archive. The files are extracted by running the
% \xfile{.dtx} through \plainTeX:
% \begin{quote}
%   \verb|tex setouterhbox.dtx|
% \end{quote}
%
% \paragraph{TDS.} Now the different files must be moved into
% the different directories in your installation TDS tree
% (also known as \xfile{texmf} tree):
% \begin{quote}
% \def\t{^^A
% \begin{tabular}{@{}>{\ttfamily}l@{ $\rightarrow$ }>{\ttfamily}l@{}}
%   setouterhbox.sty & tex/generic/oberdiek/setouterhbox.sty\\
%   setouterhbox.pdf & doc/latex/oberdiek/setouterhbox.pdf\\
%   setouterhbox-example.tex & doc/latex/oberdiek/setouterhbox-example.tex\\
%   test/setouterhbox-test1.tex & doc/latex/oberdiek/test/setouterhbox-test1.tex\\
%   test/setouterhbox-test2.tex & doc/latex/oberdiek/test/setouterhbox-test2.tex\\
%   setouterhbox.dtx & source/latex/oberdiek/setouterhbox.dtx\\
% \end{tabular}^^A
% }^^A
% \sbox0{\t}^^A
% \ifdim\wd0>\linewidth
%   \begingroup
%     \advance\linewidth by\leftmargin
%     \advance\linewidth by\rightmargin
%   \edef\x{\endgroup
%     \def\noexpand\lw{\the\linewidth}^^A
%   }\x
%   \def\lwbox{^^A
%     \leavevmode
%     \hbox to \linewidth{^^A
%       \kern-\leftmargin\relax
%       \hss
%       \usebox0
%       \hss
%       \kern-\rightmargin\relax
%     }^^A
%   }^^A
%   \ifdim\wd0>\lw
%     \sbox0{\small\t}^^A
%     \ifdim\wd0>\linewidth
%       \ifdim\wd0>\lw
%         \sbox0{\footnotesize\t}^^A
%         \ifdim\wd0>\linewidth
%           \ifdim\wd0>\lw
%             \sbox0{\scriptsize\t}^^A
%             \ifdim\wd0>\linewidth
%               \ifdim\wd0>\lw
%                 \sbox0{\tiny\t}^^A
%                 \ifdim\wd0>\linewidth
%                   \lwbox
%                 \else
%                   \usebox0
%                 \fi
%               \else
%                 \lwbox
%               \fi
%             \else
%               \usebox0
%             \fi
%           \else
%             \lwbox
%           \fi
%         \else
%           \usebox0
%         \fi
%       \else
%         \lwbox
%       \fi
%     \else
%       \usebox0
%     \fi
%   \else
%     \lwbox
%   \fi
% \else
%   \usebox0
% \fi
% \end{quote}
% If you have a \xfile{docstrip.cfg} that configures and enables \docstrip's
% TDS installing feature, then some files can already be in the right
% place, see the documentation of \docstrip.
%
% \subsection{Refresh file name databases}
%
% If your \TeX~distribution
% (\TeX\,Live, \mikTeX, \dots) relies on file name databases, you must refresh
% these. For example, \TeX\,Live\ users run \verb|texhash| or
% \verb|mktexlsr|.
%
% \subsection{Some details for the interested}
%
% \paragraph{Unpacking with \LaTeX.}
% The \xfile{.dtx} chooses its action depending on the format:
% \begin{description}
% \item[\plainTeX:] Run \docstrip\ and extract the files.
% \item[\LaTeX:] Generate the documentation.
% \end{description}
% If you insist on using \LaTeX\ for \docstrip\ (really,
% \docstrip\ does not need \LaTeX), then inform the autodetect routine
% about your intention:
% \begin{quote}
%   \verb|latex \let\install=y\input{setouterhbox.dtx}|
% \end{quote}
% Do not forget to quote the argument according to the demands
% of your shell.
%
% \paragraph{Generating the documentation.}
% You can use both the \xfile{.dtx} or the \xfile{.drv} to generate
% the documentation. The process can be configured by the
% configuration file \xfile{ltxdoc.cfg}. For instance, put this
% line into this file, if you want to have A4 as paper format:
% \begin{quote}
%   \verb|\PassOptionsToClass{a4paper}{article}|
% \end{quote}
% An example follows how to generate the
% documentation with pdf\LaTeX:
% \begin{quote}
%\begin{verbatim}
%pdflatex setouterhbox.dtx
%makeindex -s gind.ist setouterhbox.idx
%pdflatex setouterhbox.dtx
%makeindex -s gind.ist setouterhbox.idx
%pdflatex setouterhbox.dtx
%\end{verbatim}
% \end{quote}
%
% \begin{thebibliography}{9}
%
% \bibitem{newsstart}
%   Damian Menscher, \Newsgroup{comp.text.tex},
%   \textit{overlong lines in List of Figures},
%   \nolinkurl{<dh058t$qbd$1@news.ks.uiuc.edu>},
%   23rd September 2005.
%   \url{https://groups.google.com/group/comp.text.tex/msg/79648d4cf1f8bc13}
%
% \bibitem{kastrup}
%   David Kastrup, \Newsgroup{comp.text.tex},
%   \textit{Re: ANN: outerhbox.sty -- collect horizontal material,
%   for unboxing into a paragraph},
%   \nolinkurl{<85y855lrx3.fsf@lola.goethe.zz>},
%   7th October 2005.
%   \url{https://groups.google.com/group/comp.text.tex/msg/7cf0a345ef932e52}
%
% \bibitem{downes}
%   Michael Downes, \textit{Line breaking in \cs{unhbox}ed Text},
%   TUGboat 11 (1990), pp. 605--612.
%
% \bibitem{hyperref}
%   Sebastian Rahtz, Heiko Oberdiek:
%   \textit{The \xpackage{hyperref} package};
%   2006/08/16 v6.75c;
%   \CTANpkg{hyperref}.
%
% \end{thebibliography}
%
% \begin{History}
%   \begin{Version}{2005/10/05 v1.0}
%   \item
%     First version.
%   \end{Version}
%   \begin{Version}{2005/10/07 v1.1}
%   \item
%     Option \xoption{hyperref} added.
%   \end{Version}
%   \begin{Version}{2005/10/18 v1.2}
%   \item
%     Support for explicit line breaks added.
%   \end{Version}
%   \begin{Version}{2006/02/12 v1.3}
%   \item
%     DTX format.
%   \item
%     Documentation extended.
%   \end{Version}
%   \begin{Version}{2006/08/26 v1.4}
%   \item
%     Date of hyperref updated.
%   \end{Version}
%   \begin{Version}{2007/04/26 v1.5}
%   \item
%     Use of package \xpackage{infwarerr}.
%   \end{Version}
%   \begin{Version}{2007/05/17 v1.6}
%   \item
%     Standard header part for generic files.
%   \end{Version}
%   \begin{Version}{2007/09/09 v1.7}
%   \item
%     Catcode section added.
%   \end{Version}
%   \begin{Version}{2016/05/16 v1.8}
%   \item
%     Documentation updates.
%   \end{Version}
% \end{History}
%
% \PrintIndex
%
% \Finale
\endinput
|
% \end{quote}
% Do not forget to quote the argument according to the demands
% of your shell.
%
% \paragraph{Generating the documentation.}
% You can use both the \xfile{.dtx} or the \xfile{.drv} to generate
% the documentation. The process can be configured by the
% configuration file \xfile{ltxdoc.cfg}. For instance, put this
% line into this file, if you want to have A4 as paper format:
% \begin{quote}
%   \verb|\PassOptionsToClass{a4paper}{article}|
% \end{quote}
% An example follows how to generate the
% documentation with pdf\LaTeX:
% \begin{quote}
%\begin{verbatim}
%pdflatex setouterhbox.dtx
%makeindex -s gind.ist setouterhbox.idx
%pdflatex setouterhbox.dtx
%makeindex -s gind.ist setouterhbox.idx
%pdflatex setouterhbox.dtx
%\end{verbatim}
% \end{quote}
%
% \begin{thebibliography}{9}
%
% \bibitem{newsstart}
%   Damian Menscher, \Newsgroup{comp.text.tex},
%   \textit{overlong lines in List of Figures},
%   \nolinkurl{<dh058t$qbd$1@news.ks.uiuc.edu>},
%   23rd September 2005.
%   \url{https://groups.google.com/group/comp.text.tex/msg/79648d4cf1f8bc13}
%
% \bibitem{kastrup}
%   David Kastrup, \Newsgroup{comp.text.tex},
%   \textit{Re: ANN: outerhbox.sty -- collect horizontal material,
%   for unboxing into a paragraph},
%   \nolinkurl{<85y855lrx3.fsf@lola.goethe.zz>},
%   7th October 2005.
%   \url{https://groups.google.com/group/comp.text.tex/msg/7cf0a345ef932e52}
%
% \bibitem{downes}
%   Michael Downes, \textit{Line breaking in \cs{unhbox}ed Text},
%   TUGboat 11 (1990), pp. 605--612.
%
% \bibitem{hyperref}
%   Sebastian Rahtz, Heiko Oberdiek:
%   \textit{The \xpackage{hyperref} package};
%   2006/08/16 v6.75c;
%   \CTANpkg{hyperref}.
%
% \end{thebibliography}
%
% \begin{History}
%   \begin{Version}{2005/10/05 v1.0}
%   \item
%     First version.
%   \end{Version}
%   \begin{Version}{2005/10/07 v1.1}
%   \item
%     Option \xoption{hyperref} added.
%   \end{Version}
%   \begin{Version}{2005/10/18 v1.2}
%   \item
%     Support for explicit line breaks added.
%   \end{Version}
%   \begin{Version}{2006/02/12 v1.3}
%   \item
%     DTX format.
%   \item
%     Documentation extended.
%   \end{Version}
%   \begin{Version}{2006/08/26 v1.4}
%   \item
%     Date of hyperref updated.
%   \end{Version}
%   \begin{Version}{2007/04/26 v1.5}
%   \item
%     Use of package \xpackage{infwarerr}.
%   \end{Version}
%   \begin{Version}{2007/05/17 v1.6}
%   \item
%     Standard header part for generic files.
%   \end{Version}
%   \begin{Version}{2007/09/09 v1.7}
%   \item
%     Catcode section added.
%   \end{Version}
%   \begin{Version}{2016/05/16 v1.8}
%   \item
%     Documentation updates.
%   \end{Version}
% \end{History}
%
% \PrintIndex
%
% \Finale
\endinput
|
% \end{quote}
% Do not forget to quote the argument according to the demands
% of your shell.
%
% \paragraph{Generating the documentation.}
% You can use both the \xfile{.dtx} or the \xfile{.drv} to generate
% the documentation. The process can be configured by the
% configuration file \xfile{ltxdoc.cfg}. For instance, put this
% line into this file, if you want to have A4 as paper format:
% \begin{quote}
%   \verb|\PassOptionsToClass{a4paper}{article}|
% \end{quote}
% An example follows how to generate the
% documentation with pdf\LaTeX:
% \begin{quote}
%\begin{verbatim}
%pdflatex setouterhbox.dtx
%makeindex -s gind.ist setouterhbox.idx
%pdflatex setouterhbox.dtx
%makeindex -s gind.ist setouterhbox.idx
%pdflatex setouterhbox.dtx
%\end{verbatim}
% \end{quote}
%
% \begin{thebibliography}{9}
%
% \bibitem{newsstart}
%   Damian Menscher, \Newsgroup{comp.text.tex},
%   \textit{overlong lines in List of Figures},
%   \nolinkurl{<dh058t$qbd$1@news.ks.uiuc.edu>},
%   23rd September 2005.
%   \url{https://groups.google.com/group/comp.text.tex/msg/79648d4cf1f8bc13}
%
% \bibitem{kastrup}
%   David Kastrup, \Newsgroup{comp.text.tex},
%   \textit{Re: ANN: outerhbox.sty -- collect horizontal material,
%   for unboxing into a paragraph},
%   \nolinkurl{<85y855lrx3.fsf@lola.goethe.zz>},
%   7th October 2005.
%   \url{https://groups.google.com/group/comp.text.tex/msg/7cf0a345ef932e52}
%
% \bibitem{downes}
%   Michael Downes, \textit{Line breaking in \cs{unhbox}ed Text},
%   TUGboat 11 (1990), pp. 605--612.
%
% \bibitem{hyperref}
%   Sebastian Rahtz, Heiko Oberdiek:
%   \textit{The \xpackage{hyperref} package};
%   2006/08/16 v6.75c;
%   \CTANpkg{hyperref}.
%
% \end{thebibliography}
%
% \begin{History}
%   \begin{Version}{2005/10/05 v1.0}
%   \item
%     First version.
%   \end{Version}
%   \begin{Version}{2005/10/07 v1.1}
%   \item
%     Option \xoption{hyperref} added.
%   \end{Version}
%   \begin{Version}{2005/10/18 v1.2}
%   \item
%     Support for explicit line breaks added.
%   \end{Version}
%   \begin{Version}{2006/02/12 v1.3}
%   \item
%     DTX format.
%   \item
%     Documentation extended.
%   \end{Version}
%   \begin{Version}{2006/08/26 v1.4}
%   \item
%     Date of hyperref updated.
%   \end{Version}
%   \begin{Version}{2007/04/26 v1.5}
%   \item
%     Use of package \xpackage{infwarerr}.
%   \end{Version}
%   \begin{Version}{2007/05/17 v1.6}
%   \item
%     Standard header part for generic files.
%   \end{Version}
%   \begin{Version}{2007/09/09 v1.7}
%   \item
%     Catcode section added.
%   \end{Version}
%   \begin{Version}{2016/05/16 v1.8}
%   \item
%     Documentation updates.
%   \end{Version}
% \end{History}
%
% \PrintIndex
%
% \Finale
\endinput

%        (quote the arguments according to the demands of your shell)
%
% Documentation:
%    (a) If setouterhbox.drv is present:
%           latex setouterhbox.drv
%    (b) Without setouterhbox.drv:
%           latex setouterhbox.dtx; ...
%    The class ltxdoc loads the configuration file ltxdoc.cfg
%    if available. Here you can specify further options, e.g.
%    use A4 as paper format:
%       \PassOptionsToClass{a4paper}{article}
%
%    Programm calls to get the documentation (example):
%       pdflatex setouterhbox.dtx
%       makeindex -s gind.ist setouterhbox.idx
%       pdflatex setouterhbox.dtx
%       makeindex -s gind.ist setouterhbox.idx
%       pdflatex setouterhbox.dtx
%
% Installation:
%    TDS:tex/generic/oberdiek/setouterhbox.sty
%    TDS:doc/latex/oberdiek/setouterhbox.pdf
%    TDS:doc/latex/oberdiek/setouterhbox-example.tex
%    TDS:source/latex/oberdiek/setouterhbox.dtx
%
%<*ignore>
\begingroup
  \catcode123=1 %
  \catcode125=2 %
  \def\x{LaTeX2e}%
\expandafter\endgroup
\ifcase 0\ifx\install y1\fi\expandafter
         \ifx\csname processbatchFile\endcsname\relax\else1\fi
         \ifx\fmtname\x\else 1\fi\relax
\else\csname fi\endcsname
%</ignore>
%<*install>
\input docstrip.tex
\Msg{************************************************************************}
\Msg{* Installation}
\Msg{* Package: setouterhbox 2016/05/16 v1.8 Set hbox in outer horizontal mode (HO)}
\Msg{************************************************************************}

\keepsilent
\askforoverwritefalse

\let\MetaPrefix\relax
\preamble

This is a generated file.

Project: setouterhbox
Version: 2016/05/16 v1.8

Copyright (C)
   2005-2007 Heiko Oberdiek
   2016-2019 Oberdiek Package Support Group

This work may be distributed and/or modified under the
conditions of the LaTeX Project Public License, either
version 1.3c of this license or (at your option) any later
version. This version of this license is in
   https://www.latex-project.org/lppl/lppl-1-3c.txt
and the latest version of this license is in
   https://www.latex-project.org/lppl.txt
and version 1.3 or later is part of all distributions of
LaTeX version 2005/12/01 or later.

This work has the LPPL maintenance status "maintained".

The Current Maintainers of this work are
Heiko Oberdiek and the Oberdiek Package Support Group
https://github.com/ho-tex/oberdiek/issues


The Base Interpreter refers to any `TeX-Format',
because some files are installed in TDS:tex/generic//.

This work consists of the main source file setouterhbox.dtx
and the derived files
   setouterhbox.sty, setouterhbox.pdf, setouterhbox.ins, setouterhbox.drv,
   setouterhbox-example.tex, setouterhbox-test1.tex,
   setouterhbox-test2.tex.

\endpreamble
\let\MetaPrefix\DoubleperCent

\generate{%
  \file{setouterhbox.ins}{\from{setouterhbox.dtx}{install}}%
  \file{setouterhbox.drv}{\from{setouterhbox.dtx}{driver}}%
  \usedir{tex/generic/oberdiek}%
  \file{setouterhbox.sty}{\from{setouterhbox.dtx}{package}}%
  \usedir{doc/latex/oberdiek}%
  \file{setouterhbox-example.tex}{\from{setouterhbox.dtx}{example}}%
%  \usedir{doc/latex/oberdiek/test}%
%  \file{setouterhbox-test1.tex}{\from{setouterhbox.dtx}{test1}}%
%  \file{setouterhbox-test2.tex}{\from{setouterhbox.dtx}{test2}}%
}

\catcode32=13\relax% active space
\let =\space%
\Msg{************************************************************************}
\Msg{*}
\Msg{* To finish the installation you have to move the following}
\Msg{* file into a directory searched by TeX:}
\Msg{*}
\Msg{*     setouterhbox.sty}
\Msg{*}
\Msg{* To produce the documentation run the file `setouterhbox.drv'}
\Msg{* through LaTeX.}
\Msg{*}
\Msg{* Happy TeXing!}
\Msg{*}
\Msg{************************************************************************}

\endbatchfile
%</install>
%<*ignore>
\fi
%</ignore>
%<*driver>
\NeedsTeXFormat{LaTeX2e}
\ProvidesFile{setouterhbox.drv}%
  [2016/05/16 v1.8 Set hbox in outer horizontal mode (HO)]%
\documentclass{ltxdoc}
\usepackage{holtxdoc}[2011/11/22]
\begin{document}
  \DocInput{setouterhbox.dtx}%
\end{document}
%</driver>
% \fi
%
%
%
% \GetFileInfo{setouterhbox.drv}
%
% \title{The \xpackage{setouterhbox} package}
% \date{2016/05/16 v1.8}
% \author{Heiko Oberdiek\thanks
% {Please report any issues at \url{https://github.com/ho-tex/oberdiek/issues}}}
%
% \maketitle
%
% \begin{abstract}
% If math stuff is set in an \cs{hbox}, then TeX
% performs some optimization and omits the implicite
% penalties \cs{binoppenalty} and \cs{relpenalty}.
% This packages tries to put stuff into an \cs{hbox}
% without getting lost of those penalties.
% \end{abstract}
%
% \tableofcontents
%
% \section{Documentation}
%
% \subsection{Introduction}
%
% There is a situation in \xpackage{hyperref}'s driver for dvips
% where the user wants to have links that can be broken across
% lines. However dvips doesn't support the feature. With option
% \xoption{breaklinks} \xpackage{hyperref} sets the links as
% usual, put them in a box and write the link data with
% box dimensions into the appropriate \cs{special}s.
% Then, however, it does not set the complete unbreakable
% box, but it unwrappes the material inside to allow line
% breaks. Of course line breaking and glue setting will falsify
% the link dimensions, but line breaking was more important
% for the user.
%
% \subsection{Acknowledgement}
%
% Jonathan Fine, Donald Arsenau and me discussed the problem
% in the newsgroup \xnewsgroup{comp.text.tex} where Damian
% Menscher has started the thread, see \cite{newsstart}.
%
% The discussion was productive and generated many ideas
% and code examples. In order to have a more permanent
% result I wrote this package and tried to implement
% most of the ideas, a kind of summary of the discussion.
% Thus I want and have to thank Jonathan Fine and Donald Arsenau
% very much.
%
% Two weeks later David Kastrup (posting in
% \xnewsgroup{comp.text.tex}, \cite{kastrup})
% remembered an old article of Michael Downes (\cite{downes})
% in TUGboat, where Michael Downes already presented the
% method we discuss here. Nowadays we have \eTeX\ that extends
% the tool set of a \TeX\ macro programmer. Especially useful
% \eTeX\ was in this package for detecting and dealing with
% errorneous situations.
%
% However also nowadays a perfect solution for the problem
% is still missing at macro level. Probably someone has
% to go deep in the internals of the \TeX\ compiler to
% implement a switch that let penalties stay where otherwise
% \TeX\ would remove them for optimization reasons.
%
% \subsection{Usage}
%
% \paragraph{Package loading.}
% \LaTeX: as usually:
% \begin{quote}
%   |\usepackage{setouterhbox}|
% \end{quote}
% The package can also be included directly, thus \plainTeX\ users
% write:
% \begin{quote}
%   |\input setouterhbox.sty|
% \end{quote}
%
% \paragraph{Register allocation.}
% The material will be put into a box, thus we need to know these
% box number. If you need to allocate a new box register:
% \begin{description}
%  \item[\LaTeX:] |\newsavebox{\|\meta{name}|}|
%  \item[\plainTeX:] |\newbox\|\meta{name}
% \end{description}
% Then |\|\meta{name} is a command that held the box number.
%
% \paragraph{Box wrapping.}
% \LaTeX\ users put the material in the box with an environment
% similar to \texttt{lrbox}. The environment \texttt{setouterhbox}
% uses the same syntax and offers the same features, such
% as verbatim stuff inside:
% \begin{quote}
%  |\begin{setouterhbox}{|\meta{box number}|}|\dots
%  |\end{setouterhbox}|
% \end{quote}
% Users with \plainTeX\ do not have environments, they use instead:
% \begin{quote}
%   |\setouterhbox{|\meta{box number}|}|\dots|\endsetouterhbox|
% \end{quote}
% In both cases the material is put into an \cs{hbox} and assigned
% to the given box, denoted by \meta{box number}. Note the
% assignment is local, the same way \texttt{lrbox} behaves.
%
% \paragraph{Unwrapping.}
% The box material is ready for unwrapping:
% \begin{quote}
%   |\unhbox|\meta{box number}
% \end{quote}
%
% \subsection{Option \xoption{hyperref}}
%
% Package url uses math mode for typesetting urls.
% Break points are inserted by \cs{binoppenalty} and
% \cs{relpenalty}. Unhappily these break points are
% removed, if \xpackage{hyperref}
% is used with option {breaklinks}
% and drivers that depend on \xoption{pdfmark}:
% \xoption{dvips}, \xoption{vtexpdfmark}, \xoption{textures},
% and \xoption{dvipsone}.
% Thus the option \xoption{hyperref} enables the method
% of this package to avoid the removal of \cs{relpenalty}
% and \cs{binoppenalty}. Thus you get more break points.
% However, the link areas are still wrong for these
% drivers, because they are not supporting broken
% links.
%
% Note, you need version 2006/08/16 v6.75c of package \xpackage{hyperref},
% because starting with this version the necessary hook is provided
% that package \xpackage{setouterhbox} uses.
% \begin{quote}
%   |\usepackage[|\dots|]{hyperref}[2006/08/16]|\\
%   |\usepackage[hyperref]{setouterhbox}|
% \end{quote}
% Package order does not matter.
%
% \subsection{Example}
%
%    \begin{macrocode}
%<*example>
\documentclass[a5paper]{article}
\usepackage{url}[2005/06/27]
\usepackage{setouterhbox}

\newsavebox{\testbox}

\setlength{\parindent}{0pt}
\setlength{\parskip}{2em}

\begin{document}
\raggedright

\url{http://this.is.a.very.long.host.name/followed/%
by/a/very_long_long_long_path.html}%

\sbox\testbox{%
  \url{http://this.is.a.very.long.host.name/followed/%
  by/a/very_long_long_long_path.html}%
}%
\unhbox\testbox

\begin{setouterhbox}{\testbox}%
  \url{http://this.is.a.very.long.host.name/followed/%
  by/a/very_long_long_long_path.html}%
\end{setouterhbox}
\unhbox\testbox

\end{document}
%</example>
%    \end{macrocode}
%
% \StopEventually{
% }
%
% \section{Implementation}
%
% Internal macros are prefixed by \cs{setouterhbox}, |@| is
% not used inside names, thus we do not need to care of its
% catcode if we are not using it as \LaTeX\ package.
%
% \subsection{Package start stuff}
%
%    \begin{macrocode}
%<*package>
%    \end{macrocode}
%
% Prevent reloading more than one, necessary for \plainTeX:
%    Reload check, especially if the package is not used with \LaTeX.
%    \begin{macrocode}
\begingroup\catcode61\catcode48\catcode32=10\relax%
  \catcode13=5 % ^^M
  \endlinechar=13 %
  \catcode35=6 % #
  \catcode39=12 % '
  \catcode44=12 % ,
  \catcode45=12 % -
  \catcode46=12 % .
  \catcode58=12 % :
  \catcode64=11 % @
  \catcode123=1 % {
  \catcode125=2 % }
  \expandafter\let\expandafter\x\csname ver@setouterhbox.sty\endcsname
  \ifx\x\relax % plain-TeX, first loading
  \else
    \def\empty{}%
    \ifx\x\empty % LaTeX, first loading,
      % variable is initialized, but \ProvidesPackage not yet seen
    \else
      \expandafter\ifx\csname PackageInfo\endcsname\relax
        \def\x#1#2{%
          \immediate\write-1{Package #1 Info: #2.}%
        }%
      \else
        \def\x#1#2{\PackageInfo{#1}{#2, stopped}}%
      \fi
      \x{setouterhbox}{The package is already loaded}%
      \aftergroup\endinput
    \fi
  \fi
\endgroup%
%    \end{macrocode}
%    Package identification:
%    \begin{macrocode}
\begingroup\catcode61\catcode48\catcode32=10\relax%
  \catcode13=5 % ^^M
  \endlinechar=13 %
  \catcode35=6 % #
  \catcode39=12 % '
  \catcode40=12 % (
  \catcode41=12 % )
  \catcode44=12 % ,
  \catcode45=12 % -
  \catcode46=12 % .
  \catcode47=12 % /
  \catcode58=12 % :
  \catcode64=11 % @
  \catcode91=12 % [
  \catcode93=12 % ]
  \catcode123=1 % {
  \catcode125=2 % }
  \expandafter\ifx\csname ProvidesPackage\endcsname\relax
    \def\x#1#2#3[#4]{\endgroup
      \immediate\write-1{Package: #3 #4}%
      \xdef#1{#4}%
    }%
  \else
    \def\x#1#2[#3]{\endgroup
      #2[{#3}]%
      \ifx#1\@undefined
        \xdef#1{#3}%
      \fi
      \ifx#1\relax
        \xdef#1{#3}%
      \fi
    }%
  \fi
\expandafter\x\csname ver@setouterhbox.sty\endcsname
\ProvidesPackage{setouterhbox}%
  [2016/05/16 v1.8 Set hbox in outer horizontal mode (HO)]%
%    \end{macrocode}
%
%    \begin{macrocode}
\begingroup\catcode61\catcode48\catcode32=10\relax%
  \catcode13=5 % ^^M
  \endlinechar=13 %
  \catcode123=1 % {
  \catcode125=2 % }
  \catcode64=11 % @
  \def\x{\endgroup
    \expandafter\edef\csname setouterhboxAtEnd\endcsname{%
      \endlinechar=\the\endlinechar\relax
      \catcode13=\the\catcode13\relax
      \catcode32=\the\catcode32\relax
      \catcode35=\the\catcode35\relax
      \catcode61=\the\catcode61\relax
      \catcode64=\the\catcode64\relax
      \catcode123=\the\catcode123\relax
      \catcode125=\the\catcode125\relax
    }%
  }%
\x\catcode61\catcode48\catcode32=10\relax%
\catcode13=5 % ^^M
\endlinechar=13 %
\catcode35=6 % #
\catcode64=11 % @
\catcode123=1 % {
\catcode125=2 % }
\def\TMP@EnsureCode#1#2{%
  \edef\setouterhboxAtEnd{%
    \setouterhboxAtEnd
    \catcode#1=\the\catcode#1\relax
  }%
  \catcode#1=#2\relax
}
\TMP@EnsureCode{40}{12}% (
\TMP@EnsureCode{41}{12}% )
\TMP@EnsureCode{44}{12}% ,
\TMP@EnsureCode{45}{12}% -
\TMP@EnsureCode{46}{12}% .
\TMP@EnsureCode{47}{12}% /
\TMP@EnsureCode{58}{12}% :
\TMP@EnsureCode{60}{12}% <
\TMP@EnsureCode{62}{12}% >
\TMP@EnsureCode{91}{12}% [
\TMP@EnsureCode{93}{12}% ]
\TMP@EnsureCode{96}{12}% `
\edef\setouterhboxAtEnd{\setouterhboxAtEnd\noexpand\endinput}
%    \end{macrocode}
%
% \subsection{Interface macros}
%
%    \begin{macro}{\setouterhboxBox}
% The method requires a global box assignment. To be on the
% safe side, a new box register is allocated for this
% global box assignment.
%    \begin{macrocode}
\newbox\setouterhboxBox
%    \end{macrocode}
%    \end{macro}
%
%    \begin{macro}{\setouterhboxFailure}
% Error message for both \plainTeX\ and \LaTeX
%    \begin{macrocode}
\begingroup\expandafter\expandafter\expandafter\endgroup
\expandafter\ifx\csname RequirePackage\endcsname\relax
  \input infwarerr.sty\relax
\else
  \RequirePackage{infwarerr}[2016/05/16]%
\fi
\edef\setouterhboxFailure#1#2{%
  \expandafter\noexpand\csname @PackageError\endcsname
      {setouterhbox}{#1}{#2}%
}
%    \end{macrocode}
%    \end{macro}
%
% \subsection{Main part}
%
% eTeX provides much better means for checking
% error conditions. Thus lines marked by "E" are executed
% if eTeX is available, otherwise the lines marked by "T" are
% used.
%    \begin{macrocode}
\begingroup\expandafter\expandafter\expandafter\endgroup
\expandafter\ifx\csname lastnodetype\endcsname\relax
  \catcode`T=9 % ignore
  \catcode`E=14 % comment
\else
  \catcode`T=14 % comment
  \catcode`E=9 % ignore
\fi
%    \end{macrocode}
%
%    \begin{macro}{\setouterhboxRemove}
% Remove all kern, glue, and penalty nodes;
% poor man's version, if \eTeX\ is not available
%    \begin{macrocode}
\def\setouterhboxRemove{%
E \ifnum\lastnodetype<11 %
E   \else
E   \ifnum\lastnodetype>13 %
E   \else
      \unskip\unkern\unpenalty
E     \expandafter\expandafter\expandafter\setouterhboxRemove
E   \fi
E \fi
}%
%    \end{macrocode}
%    \end{macro}
%
%    \begin{macro}{\setouterhbox}
% Passing the box contents by macro parameter would prevent
% catcode changes in the box contents like by \cs{verb}.
% Also \cs{bgroup} and \cs{egroup} does not work, because stuff
% has to be added at the begin and end of the box, thus
% the syntax
% |\setouterhbox{|\meta{box number}|}|\dots|\endsetouterhbox|
% is used. Also we automatically get an environment \texttt{setouterhbox}
% if \LaTeX\ is used.
%    \begin{macrocode}
\def\setouterhbox#1{%
  \begingroup
    \def\setouterhboxNum{#1}%
    \setbox0\vbox\bgroup
T     \kern.123pt\relax % marker
T     \kern0pt\relax % removed by \setouterhboxRemove
      \begingroup
        \everypar{}%
        \noindent
}
%    \end{macrocode}
%    \end{macro}
%    \begin{macro}{\endsetouterhbox}
% Most of the work is done in the end part, thus the heart of
% the method follows:
%    \begin{macrocode}
\def\endsetouterhbox{%
      \endgroup
%    \end{macrocode}
% Omit the first pass to get the penalties
% of the second pass.
%    \begin{macrocode}
      \pretolerance-1 %
%    \end{macrocode}
%  We don't want a third pass with \cs{emergencystretch}.
%    \begin{macrocode}
      \tolerance10000 %
      \hsize\maxdimen
%    \end{macrocode}
% Line is not underfull:
%    \begin{macrocode}
      \parfillskip 0pt plus 1filll\relax
      \leftskip0pt\relax
%    \end{macrocode}
% Suppress underful \cs{hbox} warnings,
% is explicit line breaks are used.
%    \begin{macrocode}
      \rightskip0pt plus 1fil\relax
      \everypar{}%
%    \end{macrocode}
% Ensure that there is a paragraph and
% prevents \cs{endgraph} from eating terminal glue:
%    \begin{macrocode}
      \kern0pt%
      \endgraf
      \setouterhboxRemove
E     \ifnum\lastnodetype=1 %
E       \global\setbox\setouterhboxBox\lastbox
E       \loop
E         \setouterhboxRemove
E       \ifnum\lastnodetype=1 %
E         \setbox0=\lastbox
E         \global\setbox\setouterhboxBox=\hbox{%
E           \unhbox0 %
%    \end{macrocode}
% Remove \cs{rightskip}, a penalty with -10000 is part of the previous line.
%    \begin{macrocode}
E           \unskip
E           \unhbox\setouterhboxBox
E         }%
E       \repeat
E     \else
E       \setouterhboxFailure{%
E         Something is wrong%
E       }{%
E         Could not find expected line.%
E         \MessageBreak
E         (\string\lastnodetype: \number\lastnodetype, expected: 1)%
E       }%
E     \fi
E     \setouterhboxRemove
T     \global\setbox\setouterhboxBox\lastbox
T     \loop
T       \setouterhboxRemove
T       \setbox0=\lastbox
T     \ifcase\ifvoid0 1\else0\fi
T       \global\setbox\setouterhboxBox=\hbox{%
T         \unhbox0 %
%    \end{macrocode}
% Remove \cs{rightskip}, a penalty with -10000 is part of the previous line.
%    \begin{macrocode}
T         \unskip
T         \unhbox\setouterhboxBox
T       }%
T     \repeat
T     \ifdim.123pt=\lastkern
T     \else
T       \setouterhboxFailure{%
T         Something is wrong%
T       }{%
T         Unexpected stuff was detected before the line.%
T       }%
T     \fi
T   \egroup
T   \ifcase \ifnum\wd0=0 \else 1\fi
T           \ifdim\ht0=.123pt \else 1\fi
T           \ifnum\dp0=0 \else 1\fi
T           0 %
E     \ifnum\lastnodetype=-1 %
%    \end{macrocode}
% There was just one line that we have caught.
%    \begin{macrocode}
      \else
        \setouterhboxFailure{%
            Something is wrong%
        }{%
            After fetching the line there is more unexpected stuff.%
E           \MessageBreak
E           (\string\lastnodetype: \number\lastnodetype, expected: -1)%
        }%
      \fi
E   \egroup
  \expandafter\endgroup
  \expandafter\setouterhboxFinish\expandafter{%
    \number\setouterhboxNum
  }%
}
%    \end{macrocode}
%    \end{macro}
%
% \subsection{Environment support}
%
% Check \cs{@currenvir} for the case that \cs{setouterhbox}
% was called as environment. Then the box assignment
% must be put after the \cs{endgroup} of |\end{|\dots|}|.
%    \begin{macrocode}
\def\setouterhboxCurr{setouterhbox}
\def\setouterhboxLast#1{%
  \setbox#1\hbox{%
    \unhbox\setouterhboxBox
    \unskip % remove \rightskip glue
    \unskip % remove \parfillskip glue
    \unpenalty % remove paragraph ending \penalty 10000
    \unkern % remove explicit kern inserted above
  }%
}
%    \end{macrocode}
%    \begin{macro}{\setouterhboxFinish}
% |#1| is an explicit number.
%    \begin{macrocode}
\def\setouterhboxFinish#1{%
  \begingroup\expandafter\expandafter\expandafter\endgroup
  \expandafter\ifx\csname @currenvir\endcsname\setouterhboxCurr
    \aftergroup\setouterhboxLast
    \aftergroup{%
    \setouterhboxAfter #1\NIL
    \aftergroup}%
  \else
    \setouterhboxLast{#1}%
  \fi
}
%    \end{macrocode}
%    \end{macro}
%    \begin{macro}{\setouterhboxAfter}
% |#1| is an explicit number.
%    \begin{macrocode}
\def\setouterhboxAfter#1#2\NIL{%
  \aftergroup#1%
  \ifx\\#2\\%
  \else
    \setouterhboxReturnAfterFi{%
      \setouterhboxAfter#2\NIL
    }%
  \fi
}
%    \end{macrocode}
%    \end{macro}
%    \begin{macro}{\setouterhboxReturnAfterFi}
% A utility macro to get tail recursion.
%    \begin{macrocode}
\long\def\setouterhboxReturnAfterFi#1\fi{\fi#1}
%    \end{macrocode}
%    \end{macro}
% Restore catcodes we have need to distinguish between
% the implementation with and without \eTeX.
%    \begin{macrocode}
\catcode69=11\relax % E
\catcode84=11\relax % T
%    \end{macrocode}
%
% \subsection{Option \xoption{hyperref}}
%    \begin{macrocode}
\begingroup
  \def\x{LaTeX2e}%
\expandafter\endgroup
\ifx\x\fmtname
\else
  \expandafter\setouterhboxAtEnd
\fi%
%    \end{macrocode}
%    \begin{macro}{\Hy@setouterhbox}
% \cs{Hy@setouterhbox} is the internal hook that \xpackage{hyperref}
% uses since 2006/02/12 v6.75a.
%    \begin{macrocode}
\DeclareOption{hyperref}{%
  \long\def\Hy@setouterhbox#1#2{%
    \setouterhbox{#1}#2\endsetouterhbox
  }%
}
%    \end{macrocode}
%    \end{macro}
%    \begin{macrocode}
\ProcessOptions\relax
%    \end{macrocode}
%
%    \begin{macrocode}
\setouterhboxAtEnd%
%</package>
%    \end{macrocode}
%% \section{Installation}
%
% \subsection{Download}
%
% \paragraph{Package.} This package is available on
% CTAN\footnote{\CTANpkg{setouterhbox}}:
% \begin{description}
% \item[\CTAN{macros/latex/contrib/oberdiek/setouterhbox.dtx}] The source file.
% \item[\CTAN{macros/latex/contrib/oberdiek/setouterhbox.pdf}] Documentation.
% \end{description}
%
%
% \paragraph{Bundle.} All the packages of the bundle `oberdiek'
% are also available in a TDS compliant ZIP archive. There
% the packages are already unpacked and the documentation files
% are generated. The files and directories obey the TDS standard.
% \begin{description}
% \item[\CTANinstall{install/macros/latex/contrib/oberdiek.tds.zip}]
% \end{description}
% \emph{TDS} refers to the standard ``A Directory Structure
% for \TeX\ Files'' (\CTANpkg{tds}). Directories
% with \xfile{texmf} in their name are usually organized this way.
%
% \subsection{Bundle installation}
%
% \paragraph{Unpacking.} Unpack the \xfile{oberdiek.tds.zip} in the
% TDS tree (also known as \xfile{texmf} tree) of your choice.
% Example (linux):
% \begin{quote}
%   |unzip oberdiek.tds.zip -d ~/texmf|
% \end{quote}
%
% \subsection{Package installation}
%
% \paragraph{Unpacking.} The \xfile{.dtx} file is a self-extracting
% \docstrip\ archive. The files are extracted by running the
% \xfile{.dtx} through \plainTeX:
% \begin{quote}
%   \verb|tex setouterhbox.dtx|
% \end{quote}
%
% \paragraph{TDS.} Now the different files must be moved into
% the different directories in your installation TDS tree
% (also known as \xfile{texmf} tree):
% \begin{quote}
% \def\t{^^A
% \begin{tabular}{@{}>{\ttfamily}l@{ $\rightarrow$ }>{\ttfamily}l@{}}
%   setouterhbox.sty & tex/generic/oberdiek/setouterhbox.sty\\
%   setouterhbox.pdf & doc/latex/oberdiek/setouterhbox.pdf\\
%   setouterhbox-example.tex & doc/latex/oberdiek/setouterhbox-example.tex\\
%   setouterhbox.dtx & source/latex/oberdiek/setouterhbox.dtx\\
% \end{tabular}^^A
% }^^A
% \sbox0{\t}^^A
% \ifdim\wd0>\linewidth
%   \begingroup
%     \advance\linewidth by\leftmargin
%     \advance\linewidth by\rightmargin
%   \edef\x{\endgroup
%     \def\noexpand\lw{\the\linewidth}^^A
%   }\x
%   \def\lwbox{^^A
%     \leavevmode
%     \hbox to \linewidth{^^A
%       \kern-\leftmargin\relax
%       \hss
%       \usebox0
%       \hss
%       \kern-\rightmargin\relax
%     }^^A
%   }^^A
%   \ifdim\wd0>\lw
%     \sbox0{\small\t}^^A
%     \ifdim\wd0>\linewidth
%       \ifdim\wd0>\lw
%         \sbox0{\footnotesize\t}^^A
%         \ifdim\wd0>\linewidth
%           \ifdim\wd0>\lw
%             \sbox0{\scriptsize\t}^^A
%             \ifdim\wd0>\linewidth
%               \ifdim\wd0>\lw
%                 \sbox0{\tiny\t}^^A
%                 \ifdim\wd0>\linewidth
%                   \lwbox
%                 \else
%                   \usebox0
%                 \fi
%               \else
%                 \lwbox
%               \fi
%             \else
%               \usebox0
%             \fi
%           \else
%             \lwbox
%           \fi
%         \else
%           \usebox0
%         \fi
%       \else
%         \lwbox
%       \fi
%     \else
%       \usebox0
%     \fi
%   \else
%     \lwbox
%   \fi
% \else
%   \usebox0
% \fi
% \end{quote}
% If you have a \xfile{docstrip.cfg} that configures and enables \docstrip's
% TDS installing feature, then some files can already be in the right
% place, see the documentation of \docstrip.
%
% \subsection{Refresh file name databases}
%
% If your \TeX~distribution
% (\TeX\,Live, \mikTeX, \dots) relies on file name databases, you must refresh
% these. For example, \TeX\,Live\ users run \verb|texhash| or
% \verb|mktexlsr|.
%
% \subsection{Some details for the interested}
%
% \paragraph{Unpacking with \LaTeX.}
% The \xfile{.dtx} chooses its action depending on the format:
% \begin{description}
% \item[\plainTeX:] Run \docstrip\ and extract the files.
% \item[\LaTeX:] Generate the documentation.
% \end{description}
% If you insist on using \LaTeX\ for \docstrip\ (really,
% \docstrip\ does not need \LaTeX), then inform the autodetect routine
% about your intention:
% \begin{quote}
%   \verb|latex \let\install=y% \iffalse meta-comment
%
% File: setouterhbox.dtx
% Version: 2016/05/16 v1.8
% Info: Set hbox in outer horizontal mode
%
% Copyright (C)
%    2005-2007 Heiko Oberdiek
%    2016-2019 Oberdiek Package Support Group
%    https://github.com/ho-tex/oberdiek/issues
%
% This work may be distributed and/or modified under the
% conditions of the LaTeX Project Public License, either
% version 1.3c of this license or (at your option) any later
% version. This version of this license is in
%    https://www.latex-project.org/lppl/lppl-1-3c.txt
% and the latest version of this license is in
%    https://www.latex-project.org/lppl.txt
% and version 1.3 or later is part of all distributions of
% LaTeX version 2005/12/01 or later.
%
% This work has the LPPL maintenance status "maintained".
%
% The Current Maintainers of this work are
% Heiko Oberdiek and the Oberdiek Package Support Group
% https://github.com/ho-tex/oberdiek/issues
%
% The Base Interpreter refers to any `TeX-Format',
% because some files are installed in TDS:tex/generic//.
%
% This work consists of the main source file setouterhbox.dtx
% and the derived files
%    setouterhbox.sty, setouterhbox.pdf, setouterhbox.ins, setouterhbox.drv,
%    setouterhbox-example.tex, setouterhbox-test1.tex,
%    setouterhbox-test2.tex.
%
% Distribution:
%    CTAN:macros/latex/contrib/oberdiek/setouterhbox.dtx
%    CTAN:macros/latex/contrib/oberdiek/setouterhbox.pdf
%
% Unpacking:
%    (a) If setouterhbox.ins is present:
%           tex setouterhbox.ins
%    (b) Without setouterhbox.ins:
%           tex setouterhbox.dtx
%    (c) If you insist on using LaTeX
%           latex \let\install=y% \iffalse meta-comment
%
% File: setouterhbox.dtx
% Version: 2016/05/16 v1.8
% Info: Set hbox in outer horizontal mode
%
% Copyright (C)
%    2005-2007 Heiko Oberdiek
%    2016-2019 Oberdiek Package Support Group
%    https://github.com/ho-tex/oberdiek/issues
%
% This work may be distributed and/or modified under the
% conditions of the LaTeX Project Public License, either
% version 1.3c of this license or (at your option) any later
% version. This version of this license is in
%    https://www.latex-project.org/lppl/lppl-1-3c.txt
% and the latest version of this license is in
%    https://www.latex-project.org/lppl.txt
% and version 1.3 or later is part of all distributions of
% LaTeX version 2005/12/01 or later.
%
% This work has the LPPL maintenance status "maintained".
%
% The Current Maintainers of this work are
% Heiko Oberdiek and the Oberdiek Package Support Group
% https://github.com/ho-tex/oberdiek/issues
%
% The Base Interpreter refers to any `TeX-Format',
% because some files are installed in TDS:tex/generic//.
%
% This work consists of the main source file setouterhbox.dtx
% and the derived files
%    setouterhbox.sty, setouterhbox.pdf, setouterhbox.ins, setouterhbox.drv,
%    setouterhbox-example.tex, setouterhbox-test1.tex,
%    setouterhbox-test2.tex.
%
% Distribution:
%    CTAN:macros/latex/contrib/oberdiek/setouterhbox.dtx
%    CTAN:macros/latex/contrib/oberdiek/setouterhbox.pdf
%
% Unpacking:
%    (a) If setouterhbox.ins is present:
%           tex setouterhbox.ins
%    (b) Without setouterhbox.ins:
%           tex setouterhbox.dtx
%    (c) If you insist on using LaTeX
%           latex \let\install=y% \iffalse meta-comment
%
% File: setouterhbox.dtx
% Version: 2016/05/16 v1.8
% Info: Set hbox in outer horizontal mode
%
% Copyright (C)
%    2005-2007 Heiko Oberdiek
%    2016-2019 Oberdiek Package Support Group
%    https://github.com/ho-tex/oberdiek/issues
%
% This work may be distributed and/or modified under the
% conditions of the LaTeX Project Public License, either
% version 1.3c of this license or (at your option) any later
% version. This version of this license is in
%    https://www.latex-project.org/lppl/lppl-1-3c.txt
% and the latest version of this license is in
%    https://www.latex-project.org/lppl.txt
% and version 1.3 or later is part of all distributions of
% LaTeX version 2005/12/01 or later.
%
% This work has the LPPL maintenance status "maintained".
%
% The Current Maintainers of this work are
% Heiko Oberdiek and the Oberdiek Package Support Group
% https://github.com/ho-tex/oberdiek/issues
%
% The Base Interpreter refers to any `TeX-Format',
% because some files are installed in TDS:tex/generic//.
%
% This work consists of the main source file setouterhbox.dtx
% and the derived files
%    setouterhbox.sty, setouterhbox.pdf, setouterhbox.ins, setouterhbox.drv,
%    setouterhbox-example.tex, setouterhbox-test1.tex,
%    setouterhbox-test2.tex.
%
% Distribution:
%    CTAN:macros/latex/contrib/oberdiek/setouterhbox.dtx
%    CTAN:macros/latex/contrib/oberdiek/setouterhbox.pdf
%
% Unpacking:
%    (a) If setouterhbox.ins is present:
%           tex setouterhbox.ins
%    (b) Without setouterhbox.ins:
%           tex setouterhbox.dtx
%    (c) If you insist on using LaTeX
%           latex \let\install=y\input{setouterhbox.dtx}
%        (quote the arguments according to the demands of your shell)
%
% Documentation:
%    (a) If setouterhbox.drv is present:
%           latex setouterhbox.drv
%    (b) Without setouterhbox.drv:
%           latex setouterhbox.dtx; ...
%    The class ltxdoc loads the configuration file ltxdoc.cfg
%    if available. Here you can specify further options, e.g.
%    use A4 as paper format:
%       \PassOptionsToClass{a4paper}{article}
%
%    Programm calls to get the documentation (example):
%       pdflatex setouterhbox.dtx
%       makeindex -s gind.ist setouterhbox.idx
%       pdflatex setouterhbox.dtx
%       makeindex -s gind.ist setouterhbox.idx
%       pdflatex setouterhbox.dtx
%
% Installation:
%    TDS:tex/generic/oberdiek/setouterhbox.sty
%    TDS:doc/latex/oberdiek/setouterhbox.pdf
%    TDS:doc/latex/oberdiek/setouterhbox-example.tex
%    TDS:doc/latex/oberdiek/test/setouterhbox-test1.tex
%    TDS:doc/latex/oberdiek/test/setouterhbox-test2.tex
%    TDS:source/latex/oberdiek/setouterhbox.dtx
%
%<*ignore>
\begingroup
  \catcode123=1 %
  \catcode125=2 %
  \def\x{LaTeX2e}%
\expandafter\endgroup
\ifcase 0\ifx\install y1\fi\expandafter
         \ifx\csname processbatchFile\endcsname\relax\else1\fi
         \ifx\fmtname\x\else 1\fi\relax
\else\csname fi\endcsname
%</ignore>
%<*install>
\input docstrip.tex
\Msg{************************************************************************}
\Msg{* Installation}
\Msg{* Package: setouterhbox 2016/05/16 v1.8 Set hbox in outer horizontal mode (HO)}
\Msg{************************************************************************}

\keepsilent
\askforoverwritefalse

\let\MetaPrefix\relax
\preamble

This is a generated file.

Project: setouterhbox
Version: 2016/05/16 v1.8

Copyright (C)
   2005-2007 Heiko Oberdiek
   2016-2019 Oberdiek Package Support Group

This work may be distributed and/or modified under the
conditions of the LaTeX Project Public License, either
version 1.3c of this license or (at your option) any later
version. This version of this license is in
   https://www.latex-project.org/lppl/lppl-1-3c.txt
and the latest version of this license is in
   https://www.latex-project.org/lppl.txt
and version 1.3 or later is part of all distributions of
LaTeX version 2005/12/01 or later.

This work has the LPPL maintenance status "maintained".

The Current Maintainers of this work are
Heiko Oberdiek and the Oberdiek Package Support Group
https://github.com/ho-tex/oberdiek/issues


The Base Interpreter refers to any `TeX-Format',
because some files are installed in TDS:tex/generic//.

This work consists of the main source file setouterhbox.dtx
and the derived files
   setouterhbox.sty, setouterhbox.pdf, setouterhbox.ins, setouterhbox.drv,
   setouterhbox-example.tex, setouterhbox-test1.tex,
   setouterhbox-test2.tex.

\endpreamble
\let\MetaPrefix\DoubleperCent

\generate{%
  \file{setouterhbox.ins}{\from{setouterhbox.dtx}{install}}%
  \file{setouterhbox.drv}{\from{setouterhbox.dtx}{driver}}%
  \usedir{tex/generic/oberdiek}%
  \file{setouterhbox.sty}{\from{setouterhbox.dtx}{package}}%
  \usedir{doc/latex/oberdiek}%
  \file{setouterhbox-example.tex}{\from{setouterhbox.dtx}{example}}%
%  \usedir{doc/latex/oberdiek/test}%
%  \file{setouterhbox-test1.tex}{\from{setouterhbox.dtx}{test1}}%
%  \file{setouterhbox-test2.tex}{\from{setouterhbox.dtx}{test2}}%
  \nopreamble
  \nopostamble
%  \usedir{source/latex/oberdiek/catalogue}%
%  \file{setouterhbox.xml}{\from{setouterhbox.dtx}{catalogue}}%
}

\catcode32=13\relax% active space
\let =\space%
\Msg{************************************************************************}
\Msg{*}
\Msg{* To finish the installation you have to move the following}
\Msg{* file into a directory searched by TeX:}
\Msg{*}
\Msg{*     setouterhbox.sty}
\Msg{*}
\Msg{* To produce the documentation run the file `setouterhbox.drv'}
\Msg{* through LaTeX.}
\Msg{*}
\Msg{* Happy TeXing!}
\Msg{*}
\Msg{************************************************************************}

\endbatchfile
%</install>
%<*ignore>
\fi
%</ignore>
%<*driver>
\NeedsTeXFormat{LaTeX2e}
\ProvidesFile{setouterhbox.drv}%
  [2016/05/16 v1.8 Set hbox in outer horizontal mode (HO)]%
\documentclass{ltxdoc}
\usepackage{holtxdoc}[2011/11/22]
\begin{document}
  \DocInput{setouterhbox.dtx}%
\end{document}
%</driver>
% \fi
%
%
% \CharacterTable
%  {Upper-case    \A\B\C\D\E\F\G\H\I\J\K\L\M\N\O\P\Q\R\S\T\U\V\W\X\Y\Z
%   Lower-case    \a\b\c\d\e\f\g\h\i\j\k\l\m\n\o\p\q\r\s\t\u\v\w\x\y\z
%   Digits        \0\1\2\3\4\5\6\7\8\9
%   Exclamation   \!     Double quote  \"     Hash (number) \#
%   Dollar        \$     Percent       \%     Ampersand     \&
%   Acute accent  \'     Left paren    \(     Right paren   \)
%   Asterisk      \*     Plus          \+     Comma         \,
%   Minus         \-     Point         \.     Solidus       \/
%   Colon         \:     Semicolon     \;     Less than     \<
%   Equals        \=     Greater than  \>     Question mark \?
%   Commercial at \@     Left bracket  \[     Backslash     \\
%   Right bracket \]     Circumflex    \^     Underscore    \_
%   Grave accent  \`     Left brace    \{     Vertical bar  \|
%   Right brace   \}     Tilde         \~}
%
% \GetFileInfo{setouterhbox.drv}
%
% \title{The \xpackage{setouterhbox} package}
% \date{2016/05/16 v1.8}
% \author{Heiko Oberdiek\thanks
% {Please report any issues at \url{https://github.com/ho-tex/oberdiek/issues}}}
%
% \maketitle
%
% \begin{abstract}
% If math stuff is set in an \cs{hbox}, then TeX
% performs some optimization and omits the implicite
% penalties \cs{binoppenalty} and \cs{relpenalty}.
% This packages tries to put stuff into an \cs{hbox}
% without getting lost of those penalties.
% \end{abstract}
%
% \tableofcontents
%
% \section{Documentation}
%
% \subsection{Introduction}
%
% There is a situation in \xpackage{hyperref}'s driver for dvips
% where the user wants to have links that can be broken across
% lines. However dvips doesn't support the feature. With option
% \xoption{breaklinks} \xpackage{hyperref} sets the links as
% usual, put them in a box and write the link data with
% box dimensions into the appropriate \cs{special}s.
% Then, however, it does not set the complete unbreakable
% box, but it unwrappes the material inside to allow line
% breaks. Of course line breaking and glue setting will falsify
% the link dimensions, but line breaking was more important
% for the user.
%
% \subsection{Acknowledgement}
%
% Jonathan Fine, Donald Arsenau and me discussed the problem
% in the newsgroup \xnewsgroup{comp.text.tex} where Damian
% Menscher has started the thread, see \cite{newsstart}.
%
% The discussion was productive and generated many ideas
% and code examples. In order to have a more permanent
% result I wrote this package and tried to implement
% most of the ideas, a kind of summary of the discussion.
% Thus I want and have to thank Jonathan Fine and Donald Arsenau
% very much.
%
% Two weeks later David Kastrup (posting in
% \xnewsgroup{comp.text.tex}, \cite{kastrup})
% remembered an old article of Michael Downes (\cite{downes})
% in TUGboat, where Michael Downes already presented the
% method we discuss here. Nowadays we have \eTeX\ that extends
% the tool set of a \TeX\ macro programmer. Especially useful
% \eTeX\ was in this package for detecting and dealing with
% errorneous situations.
%
% However also nowadays a perfect solution for the problem
% is still missing at macro level. Probably someone has
% to go deep in the internals of the \TeX\ compiler to
% implement a switch that let penalties stay where otherwise
% \TeX\ would remove them for optimization reasons.
%
% \subsection{Usage}
%
% \paragraph{Package loading.}
% \LaTeX: as usually:
% \begin{quote}
%   |\usepackage{setouterhbox}|
% \end{quote}
% The package can also be included directly, thus \plainTeX\ users
% write:
% \begin{quote}
%   |\input setouterhbox.sty|
% \end{quote}
%
% \paragraph{Register allocation.}
% The material will be put into a box, thus we need to know these
% box number. If you need to allocate a new box register:
% \begin{description}
%  \item[\LaTeX:] |\newsavebox{\|\meta{name}|}|
%  \item[\plainTeX:] |\newbox\|\meta{name}
% \end{description}
% Then |\|\meta{name} is a command that held the box number.
%
% \paragraph{Box wrapping.}
% \LaTeX\ users put the material in the box with an environment
% similar to \texttt{lrbox}. The environment \texttt{setouterhbox}
% uses the same syntax and offers the same features, such
% as verbatim stuff inside:
% \begin{quote}
%  |\begin{setouterhbox}{|\meta{box number}|}|\dots
%  |\end{setouterhbox}|
% \end{quote}
% Users with \plainTeX\ do not have environments, they use instead:
% \begin{quote}
%   |\setouterhbox{|\meta{box number}|}|\dots|\endsetouterhbox|
% \end{quote}
% In both cases the material is put into an \cs{hbox} and assigned
% to the given box, denoted by \meta{box number}. Note the
% assignment is local, the same way \texttt{lrbox} behaves.
%
% \paragraph{Unwrapping.}
% The box material is ready for unwrapping:
% \begin{quote}
%   |\unhbox|\meta{box number}
% \end{quote}
%
% \subsection{Option \xoption{hyperref}}
%
% Package url uses math mode for typesetting urls.
% Break points are inserted by \cs{binoppenalty} and
% \cs{relpenalty}. Unhappily these break points are
% removed, if \xpackage{hyperref}
% is used with option {breaklinks}
% and drivers that depend on \xoption{pdfmark}:
% \xoption{dvips}, \xoption{vtexpdfmark}, \xoption{textures},
% and \xoption{dvipsone}.
% Thus the option \xoption{hyperref} enables the method
% of this package to avoid the removal of \cs{relpenalty}
% and \cs{binoppenalty}. Thus you get more break points.
% However, the link areas are still wrong for these
% drivers, because they are not supporting broken
% links.
%
% Note, you need version 2006/08/16 v6.75c of package \xpackage{hyperref},
% because starting with this version the necessary hook is provided
% that package \xpackage{setouterhbox} uses.
% \begin{quote}
%   |\usepackage[|\dots|]{hyperref}[2006/08/16]|\\
%   |\usepackage[hyperref]{setouterhbox}|
% \end{quote}
% Package order does not matter.
%
% \subsection{Example}
%
%    \begin{macrocode}
%<*example>
\documentclass[a5paper]{article}
\usepackage{url}[2005/06/27]
\usepackage{setouterhbox}

\newsavebox{\testbox}

\setlength{\parindent}{0pt}
\setlength{\parskip}{2em}

\begin{document}
\raggedright

\url{http://this.is.a.very.long.host.name/followed/%
by/a/very_long_long_long_path.html}%

\sbox\testbox{%
  \url{http://this.is.a.very.long.host.name/followed/%
  by/a/very_long_long_long_path.html}%
}%
\unhbox\testbox

\begin{setouterhbox}{\testbox}%
  \url{http://this.is.a.very.long.host.name/followed/%
  by/a/very_long_long_long_path.html}%
\end{setouterhbox}
\unhbox\testbox

\end{document}
%</example>
%    \end{macrocode}
%
% \StopEventually{
% }
%
% \section{Implementation}
%
% Internal macros are prefixed by \cs{setouterhbox}, |@| is
% not used inside names, thus we do not need to care of its
% catcode if we are not using it as \LaTeX\ package.
%
% \subsection{Package start stuff}
%
%    \begin{macrocode}
%<*package>
%    \end{macrocode}
%
% Prevent reloading more than one, necessary for \plainTeX:
%    Reload check, especially if the package is not used with \LaTeX.
%    \begin{macrocode}
\begingroup\catcode61\catcode48\catcode32=10\relax%
  \catcode13=5 % ^^M
  \endlinechar=13 %
  \catcode35=6 % #
  \catcode39=12 % '
  \catcode44=12 % ,
  \catcode45=12 % -
  \catcode46=12 % .
  \catcode58=12 % :
  \catcode64=11 % @
  \catcode123=1 % {
  \catcode125=2 % }
  \expandafter\let\expandafter\x\csname ver@setouterhbox.sty\endcsname
  \ifx\x\relax % plain-TeX, first loading
  \else
    \def\empty{}%
    \ifx\x\empty % LaTeX, first loading,
      % variable is initialized, but \ProvidesPackage not yet seen
    \else
      \expandafter\ifx\csname PackageInfo\endcsname\relax
        \def\x#1#2{%
          \immediate\write-1{Package #1 Info: #2.}%
        }%
      \else
        \def\x#1#2{\PackageInfo{#1}{#2, stopped}}%
      \fi
      \x{setouterhbox}{The package is already loaded}%
      \aftergroup\endinput
    \fi
  \fi
\endgroup%
%    \end{macrocode}
%    Package identification:
%    \begin{macrocode}
\begingroup\catcode61\catcode48\catcode32=10\relax%
  \catcode13=5 % ^^M
  \endlinechar=13 %
  \catcode35=6 % #
  \catcode39=12 % '
  \catcode40=12 % (
  \catcode41=12 % )
  \catcode44=12 % ,
  \catcode45=12 % -
  \catcode46=12 % .
  \catcode47=12 % /
  \catcode58=12 % :
  \catcode64=11 % @
  \catcode91=12 % [
  \catcode93=12 % ]
  \catcode123=1 % {
  \catcode125=2 % }
  \expandafter\ifx\csname ProvidesPackage\endcsname\relax
    \def\x#1#2#3[#4]{\endgroup
      \immediate\write-1{Package: #3 #4}%
      \xdef#1{#4}%
    }%
  \else
    \def\x#1#2[#3]{\endgroup
      #2[{#3}]%
      \ifx#1\@undefined
        \xdef#1{#3}%
      \fi
      \ifx#1\relax
        \xdef#1{#3}%
      \fi
    }%
  \fi
\expandafter\x\csname ver@setouterhbox.sty\endcsname
\ProvidesPackage{setouterhbox}%
  [2016/05/16 v1.8 Set hbox in outer horizontal mode (HO)]%
%    \end{macrocode}
%
%    \begin{macrocode}
\begingroup\catcode61\catcode48\catcode32=10\relax%
  \catcode13=5 % ^^M
  \endlinechar=13 %
  \catcode123=1 % {
  \catcode125=2 % }
  \catcode64=11 % @
  \def\x{\endgroup
    \expandafter\edef\csname setouterhboxAtEnd\endcsname{%
      \endlinechar=\the\endlinechar\relax
      \catcode13=\the\catcode13\relax
      \catcode32=\the\catcode32\relax
      \catcode35=\the\catcode35\relax
      \catcode61=\the\catcode61\relax
      \catcode64=\the\catcode64\relax
      \catcode123=\the\catcode123\relax
      \catcode125=\the\catcode125\relax
    }%
  }%
\x\catcode61\catcode48\catcode32=10\relax%
\catcode13=5 % ^^M
\endlinechar=13 %
\catcode35=6 % #
\catcode64=11 % @
\catcode123=1 % {
\catcode125=2 % }
\def\TMP@EnsureCode#1#2{%
  \edef\setouterhboxAtEnd{%
    \setouterhboxAtEnd
    \catcode#1=\the\catcode#1\relax
  }%
  \catcode#1=#2\relax
}
\TMP@EnsureCode{40}{12}% (
\TMP@EnsureCode{41}{12}% )
\TMP@EnsureCode{44}{12}% ,
\TMP@EnsureCode{45}{12}% -
\TMP@EnsureCode{46}{12}% .
\TMP@EnsureCode{47}{12}% /
\TMP@EnsureCode{58}{12}% :
\TMP@EnsureCode{60}{12}% <
\TMP@EnsureCode{62}{12}% >
\TMP@EnsureCode{91}{12}% [
\TMP@EnsureCode{93}{12}% ]
\TMP@EnsureCode{96}{12}% `
\edef\setouterhboxAtEnd{\setouterhboxAtEnd\noexpand\endinput}
%    \end{macrocode}
%
% \subsection{Interface macros}
%
%    \begin{macro}{\setouterhboxBox}
% The method requires a global box assignment. To be on the
% safe side, a new box register is allocated for this
% global box assignment.
%    \begin{macrocode}
\newbox\setouterhboxBox
%    \end{macrocode}
%    \end{macro}
%
%    \begin{macro}{\setouterhboxFailure}
% Error message for both \plainTeX\ and \LaTeX
%    \begin{macrocode}
\begingroup\expandafter\expandafter\expandafter\endgroup
\expandafter\ifx\csname RequirePackage\endcsname\relax
  \input infwarerr.sty\relax
\else
  \RequirePackage{infwarerr}[2016/05/16]%
\fi
\edef\setouterhboxFailure#1#2{%
  \expandafter\noexpand\csname @PackageError\endcsname
      {setouterhbox}{#1}{#2}%
}
%    \end{macrocode}
%    \end{macro}
%
% \subsection{Main part}
%
% eTeX provides much better means for checking
% error conditions. Thus lines marked by "E" are executed
% if eTeX is available, otherwise the lines marked by "T" are
% used.
%    \begin{macrocode}
\begingroup\expandafter\expandafter\expandafter\endgroup
\expandafter\ifx\csname lastnodetype\endcsname\relax
  \catcode`T=9 % ignore
  \catcode`E=14 % comment
\else
  \catcode`T=14 % comment
  \catcode`E=9 % ignore
\fi
%    \end{macrocode}
%
%    \begin{macro}{\setouterhboxRemove}
% Remove all kern, glue, and penalty nodes;
% poor man's version, if \eTeX\ is not available
%    \begin{macrocode}
\def\setouterhboxRemove{%
E \ifnum\lastnodetype<11 %
E   \else
E   \ifnum\lastnodetype>13 %
E   \else
      \unskip\unkern\unpenalty
E     \expandafter\expandafter\expandafter\setouterhboxRemove
E   \fi
E \fi
}%
%    \end{macrocode}
%    \end{macro}
%
%    \begin{macro}{\setouterhbox}
% Passing the box contents by macro parameter would prevent
% catcode changes in the box contents like by \cs{verb}.
% Also \cs{bgroup} and \cs{egroup} does not work, because stuff
% has to be added at the begin and end of the box, thus
% the syntax
% |\setouterhbox{|\meta{box number}|}|\dots|\endsetouterhbox|
% is used. Also we automatically get an environment \texttt{setouterhbox}
% if \LaTeX\ is used.
%    \begin{macrocode}
\def\setouterhbox#1{%
  \begingroup
    \def\setouterhboxNum{#1}%
    \setbox0\vbox\bgroup
T     \kern.123pt\relax % marker
T     \kern0pt\relax % removed by \setouterhboxRemove
      \begingroup
        \everypar{}%
        \noindent
}
%    \end{macrocode}
%    \end{macro}
%    \begin{macro}{\endsetouterhbox}
% Most of the work is done in the end part, thus the heart of
% the method follows:
%    \begin{macrocode}
\def\endsetouterhbox{%
      \endgroup
%    \end{macrocode}
% Omit the first pass to get the penalties
% of the second pass.
%    \begin{macrocode}
      \pretolerance-1 %
%    \end{macrocode}
%  We don't want a third pass with \cs{emergencystretch}.
%    \begin{macrocode}
      \tolerance10000 %
      \hsize\maxdimen
%    \end{macrocode}
% Line is not underfull:
%    \begin{macrocode}
      \parfillskip 0pt plus 1filll\relax
      \leftskip0pt\relax
%    \end{macrocode}
% Suppress underful \cs{hbox} warnings,
% is explicit line breaks are used.
%    \begin{macrocode}
      \rightskip0pt plus 1fil\relax
      \everypar{}%
%    \end{macrocode}
% Ensure that there is a paragraph and
% prevents \cs{endgraph} from eating terminal glue:
%    \begin{macrocode}
      \kern0pt%
      \endgraf
      \setouterhboxRemove
E     \ifnum\lastnodetype=1 %
E       \global\setbox\setouterhboxBox\lastbox
E       \loop
E         \setouterhboxRemove
E       \ifnum\lastnodetype=1 %
E         \setbox0=\lastbox
E         \global\setbox\setouterhboxBox=\hbox{%
E           \unhbox0 %
%    \end{macrocode}
% Remove \cs{rightskip}, a penalty with -10000 is part of the previous line.
%    \begin{macrocode}
E           \unskip
E           \unhbox\setouterhboxBox
E         }%
E       \repeat
E     \else
E       \setouterhboxFailure{%
E         Something is wrong%
E       }{%
E         Could not find expected line.%
E         \MessageBreak
E         (\string\lastnodetype: \number\lastnodetype, expected: 1)%
E       }%
E     \fi
E     \setouterhboxRemove
T     \global\setbox\setouterhboxBox\lastbox
T     \loop
T       \setouterhboxRemove
T       \setbox0=\lastbox
T     \ifcase\ifvoid0 1\else0\fi
T       \global\setbox\setouterhboxBox=\hbox{%
T         \unhbox0 %
%    \end{macrocode}
% Remove \cs{rightskip}, a penalty with -10000 is part of the previous line.
%    \begin{macrocode}
T         \unskip
T         \unhbox\setouterhboxBox
T       }%
T     \repeat
T     \ifdim.123pt=\lastkern
T     \else
T       \setouterhboxFailure{%
T         Something is wrong%
T       }{%
T         Unexpected stuff was detected before the line.%
T       }%
T     \fi
T   \egroup
T   \ifcase \ifnum\wd0=0 \else 1\fi
T           \ifdim\ht0=.123pt \else 1\fi
T           \ifnum\dp0=0 \else 1\fi
T           0 %
E     \ifnum\lastnodetype=-1 %
%    \end{macrocode}
% There was just one line that we have caught.
%    \begin{macrocode}
      \else
        \setouterhboxFailure{%
            Something is wrong%
        }{%
            After fetching the line there is more unexpected stuff.%
E           \MessageBreak
E           (\string\lastnodetype: \number\lastnodetype, expected: -1)%
        }%
      \fi
E   \egroup
  \expandafter\endgroup
  \expandafter\setouterhboxFinish\expandafter{%
    \number\setouterhboxNum
  }%
}
%    \end{macrocode}
%    \end{macro}
%
% \subsection{Environment support}
%
% Check \cs{@currenvir} for the case that \cs{setouterhbox}
% was called as environment. Then the box assignment
% must be put after the \cs{endgroup} of |\end{|\dots|}|.
%    \begin{macrocode}
\def\setouterhboxCurr{setouterhbox}
\def\setouterhboxLast#1{%
  \setbox#1\hbox{%
    \unhbox\setouterhboxBox
    \unskip % remove \rightskip glue
    \unskip % remove \parfillskip glue
    \unpenalty % remove paragraph ending \penalty 10000
    \unkern % remove explicit kern inserted above
  }%
}
%    \end{macrocode}
%    \begin{macro}{\setouterhboxFinish}
% |#1| is an explicit number.
%    \begin{macrocode}
\def\setouterhboxFinish#1{%
  \begingroup\expandafter\expandafter\expandafter\endgroup
  \expandafter\ifx\csname @currenvir\endcsname\setouterhboxCurr
    \aftergroup\setouterhboxLast
    \aftergroup{%
    \setouterhboxAfter #1\NIL
    \aftergroup}%
  \else
    \setouterhboxLast{#1}%
  \fi
}
%    \end{macrocode}
%    \end{macro}
%    \begin{macro}{\setouterhboxAfter}
% |#1| is an explicit number.
%    \begin{macrocode}
\def\setouterhboxAfter#1#2\NIL{%
  \aftergroup#1%
  \ifx\\#2\\%
  \else
    \setouterhboxReturnAfterFi{%
      \setouterhboxAfter#2\NIL
    }%
  \fi
}
%    \end{macrocode}
%    \end{macro}
%    \begin{macro}{\setouterhboxReturnAfterFi}
% A utility macro to get tail recursion.
%    \begin{macrocode}
\long\def\setouterhboxReturnAfterFi#1\fi{\fi#1}
%    \end{macrocode}
%    \end{macro}
% Restore catcodes we have need to distinguish between
% the implementation with and without \eTeX.
%    \begin{macrocode}
\catcode69=11\relax % E
\catcode84=11\relax % T
%    \end{macrocode}
%
% \subsection{Option \xoption{hyperref}}
%    \begin{macrocode}
\begingroup
  \def\x{LaTeX2e}%
\expandafter\endgroup
\ifx\x\fmtname
\else
  \expandafter\setouterhboxAtEnd
\fi%
%    \end{macrocode}
%    \begin{macro}{\Hy@setouterhbox}
% \cs{Hy@setouterhbox} is the internal hook that \xpackage{hyperref}
% uses since 2006/02/12 v6.75a.
%    \begin{macrocode}
\DeclareOption{hyperref}{%
  \long\def\Hy@setouterhbox#1#2{%
    \setouterhbox{#1}#2\endsetouterhbox
  }%
}
%    \end{macrocode}
%    \end{macro}
%    \begin{macrocode}
\ProcessOptions\relax
%    \end{macrocode}
%
%    \begin{macrocode}
\setouterhboxAtEnd%
%</package>
%    \end{macrocode}
%
% \section{Test}
%
% \subsection{Catcode checks for loading}
%
%    \begin{macrocode}
%<*test1>
%    \end{macrocode}
%    \begin{macrocode}
\catcode`\{=1 %
\catcode`\}=2 %
\catcode`\#=6 %
\catcode`\@=11 %
\expandafter\ifx\csname count@\endcsname\relax
  \countdef\count@=255 %
\fi
\expandafter\ifx\csname @gobble\endcsname\relax
  \long\def\@gobble#1{}%
\fi
\expandafter\ifx\csname @firstofone\endcsname\relax
  \long\def\@firstofone#1{#1}%
\fi
\expandafter\ifx\csname loop\endcsname\relax
  \expandafter\@firstofone
\else
  \expandafter\@gobble
\fi
{%
  \def\loop#1\repeat{%
    \def\body{#1}%
    \iterate
  }%
  \def\iterate{%
    \body
      \let\next\iterate
    \else
      \let\next\relax
    \fi
    \next
  }%
  \let\repeat=\fi
}%
\def\RestoreCatcodes{}
\count@=0 %
\loop
  \edef\RestoreCatcodes{%
    \RestoreCatcodes
    \catcode\the\count@=\the\catcode\count@\relax
  }%
\ifnum\count@<255 %
  \advance\count@ 1 %
\repeat

\def\RangeCatcodeInvalid#1#2{%
  \count@=#1\relax
  \loop
    \catcode\count@=15 %
  \ifnum\count@<#2\relax
    \advance\count@ 1 %
  \repeat
}
\def\RangeCatcodeCheck#1#2#3{%
  \count@=#1\relax
  \loop
    \ifnum#3=\catcode\count@
    \else
      \errmessage{%
        Character \the\count@\space
        with wrong catcode \the\catcode\count@\space
        instead of \number#3%
      }%
    \fi
  \ifnum\count@<#2\relax
    \advance\count@ 1 %
  \repeat
}
\def\space{ }
\expandafter\ifx\csname LoadCommand\endcsname\relax
  \def\LoadCommand{\input setouterhbox.sty\relax}%
\fi
\def\Test{%
  \RangeCatcodeInvalid{0}{47}%
  \RangeCatcodeInvalid{58}{64}%
  \RangeCatcodeInvalid{91}{96}%
  \RangeCatcodeInvalid{123}{255}%
  \catcode`\@=12 %
  \catcode`\\=0 %
  \catcode`\%=14 %
  \LoadCommand
  \RangeCatcodeCheck{0}{36}{15}%
  \RangeCatcodeCheck{37}{37}{14}%
  \RangeCatcodeCheck{38}{47}{15}%
  \RangeCatcodeCheck{48}{57}{12}%
  \RangeCatcodeCheck{58}{63}{15}%
  \RangeCatcodeCheck{64}{64}{12}%
  \RangeCatcodeCheck{65}{90}{11}%
  \RangeCatcodeCheck{91}{91}{15}%
  \RangeCatcodeCheck{92}{92}{0}%
  \RangeCatcodeCheck{93}{96}{15}%
  \RangeCatcodeCheck{97}{122}{11}%
  \RangeCatcodeCheck{123}{255}{15}%
  \RestoreCatcodes
}
\Test
\csname @@end\endcsname
\end
%    \end{macrocode}
%    \begin{macrocode}
%</test1>
%    \end{macrocode}
%
% \subsection{Test with package \xpackage{url}}
%
%    \begin{macrocode}
%<*test2>
\nofiles
\documentclass[a5paper]{article}
\usepackage{url}[2005/06/27]
\usepackage{setouterhbox}

\newsavebox{\testbox}

\setlength{\parindent}{0pt}
\setlength{\parskip}{2em}

\begin{document}
\raggedright

\url{http://this.is.a.very.long.host.name/followed/%
by/a/very_long_long_long_path.html}%

\sbox\testbox{%
  \url{http://this.is.a.very.long.host.name/followed/%
  by/a/very_long_long_long_path.html}%
}%
\unhbox\testbox

\begin{setouterhbox}{\testbox}%
  \url{http://this.is.a.very.long.host.name/followed/%
  by/a/very_long_long_long_path.html}%
\end{setouterhbox}
\unhbox\testbox

\end{document}
%</test2>
%    \end{macrocode}
%
% \section{Installation}
%
% \subsection{Download}
%
% \paragraph{Package.} This package is available on
% CTAN\footnote{\CTANpkg{setouterhbox}}:
% \begin{description}
% \item[\CTAN{macros/latex/contrib/oberdiek/setouterhbox.dtx}] The source file.
% \item[\CTAN{macros/latex/contrib/oberdiek/setouterhbox.pdf}] Documentation.
% \end{description}
%
%
% \paragraph{Bundle.} All the packages of the bundle `oberdiek'
% are also available in a TDS compliant ZIP archive. There
% the packages are already unpacked and the documentation files
% are generated. The files and directories obey the TDS standard.
% \begin{description}
% \item[\CTANinstall{install/macros/latex/contrib/oberdiek.tds.zip}]
% \end{description}
% \emph{TDS} refers to the standard ``A Directory Structure
% for \TeX\ Files'' (\CTAN{tds/tds.pdf}). Directories
% with \xfile{texmf} in their name are usually organized this way.
%
% \subsection{Bundle installation}
%
% \paragraph{Unpacking.} Unpack the \xfile{oberdiek.tds.zip} in the
% TDS tree (also known as \xfile{texmf} tree) of your choice.
% Example (linux):
% \begin{quote}
%   |unzip oberdiek.tds.zip -d ~/texmf|
% \end{quote}
%
% \subsection{Package installation}
%
% \paragraph{Unpacking.} The \xfile{.dtx} file is a self-extracting
% \docstrip\ archive. The files are extracted by running the
% \xfile{.dtx} through \plainTeX:
% \begin{quote}
%   \verb|tex setouterhbox.dtx|
% \end{quote}
%
% \paragraph{TDS.} Now the different files must be moved into
% the different directories in your installation TDS tree
% (also known as \xfile{texmf} tree):
% \begin{quote}
% \def\t{^^A
% \begin{tabular}{@{}>{\ttfamily}l@{ $\rightarrow$ }>{\ttfamily}l@{}}
%   setouterhbox.sty & tex/generic/oberdiek/setouterhbox.sty\\
%   setouterhbox.pdf & doc/latex/oberdiek/setouterhbox.pdf\\
%   setouterhbox-example.tex & doc/latex/oberdiek/setouterhbox-example.tex\\
%   test/setouterhbox-test1.tex & doc/latex/oberdiek/test/setouterhbox-test1.tex\\
%   test/setouterhbox-test2.tex & doc/latex/oberdiek/test/setouterhbox-test2.tex\\
%   setouterhbox.dtx & source/latex/oberdiek/setouterhbox.dtx\\
% \end{tabular}^^A
% }^^A
% \sbox0{\t}^^A
% \ifdim\wd0>\linewidth
%   \begingroup
%     \advance\linewidth by\leftmargin
%     \advance\linewidth by\rightmargin
%   \edef\x{\endgroup
%     \def\noexpand\lw{\the\linewidth}^^A
%   }\x
%   \def\lwbox{^^A
%     \leavevmode
%     \hbox to \linewidth{^^A
%       \kern-\leftmargin\relax
%       \hss
%       \usebox0
%       \hss
%       \kern-\rightmargin\relax
%     }^^A
%   }^^A
%   \ifdim\wd0>\lw
%     \sbox0{\small\t}^^A
%     \ifdim\wd0>\linewidth
%       \ifdim\wd0>\lw
%         \sbox0{\footnotesize\t}^^A
%         \ifdim\wd0>\linewidth
%           \ifdim\wd0>\lw
%             \sbox0{\scriptsize\t}^^A
%             \ifdim\wd0>\linewidth
%               \ifdim\wd0>\lw
%                 \sbox0{\tiny\t}^^A
%                 \ifdim\wd0>\linewidth
%                   \lwbox
%                 \else
%                   \usebox0
%                 \fi
%               \else
%                 \lwbox
%               \fi
%             \else
%               \usebox0
%             \fi
%           \else
%             \lwbox
%           \fi
%         \else
%           \usebox0
%         \fi
%       \else
%         \lwbox
%       \fi
%     \else
%       \usebox0
%     \fi
%   \else
%     \lwbox
%   \fi
% \else
%   \usebox0
% \fi
% \end{quote}
% If you have a \xfile{docstrip.cfg} that configures and enables \docstrip's
% TDS installing feature, then some files can already be in the right
% place, see the documentation of \docstrip.
%
% \subsection{Refresh file name databases}
%
% If your \TeX~distribution
% (\TeX\,Live, \mikTeX, \dots) relies on file name databases, you must refresh
% these. For example, \TeX\,Live\ users run \verb|texhash| or
% \verb|mktexlsr|.
%
% \subsection{Some details for the interested}
%
% \paragraph{Unpacking with \LaTeX.}
% The \xfile{.dtx} chooses its action depending on the format:
% \begin{description}
% \item[\plainTeX:] Run \docstrip\ and extract the files.
% \item[\LaTeX:] Generate the documentation.
% \end{description}
% If you insist on using \LaTeX\ for \docstrip\ (really,
% \docstrip\ does not need \LaTeX), then inform the autodetect routine
% about your intention:
% \begin{quote}
%   \verb|latex \let\install=y\input{setouterhbox.dtx}|
% \end{quote}
% Do not forget to quote the argument according to the demands
% of your shell.
%
% \paragraph{Generating the documentation.}
% You can use both the \xfile{.dtx} or the \xfile{.drv} to generate
% the documentation. The process can be configured by the
% configuration file \xfile{ltxdoc.cfg}. For instance, put this
% line into this file, if you want to have A4 as paper format:
% \begin{quote}
%   \verb|\PassOptionsToClass{a4paper}{article}|
% \end{quote}
% An example follows how to generate the
% documentation with pdf\LaTeX:
% \begin{quote}
%\begin{verbatim}
%pdflatex setouterhbox.dtx
%makeindex -s gind.ist setouterhbox.idx
%pdflatex setouterhbox.dtx
%makeindex -s gind.ist setouterhbox.idx
%pdflatex setouterhbox.dtx
%\end{verbatim}
% \end{quote}
%
% \begin{thebibliography}{9}
%
% \bibitem{newsstart}
%   Damian Menscher, \Newsgroup{comp.text.tex},
%   \textit{overlong lines in List of Figures},
%   \nolinkurl{<dh058t$qbd$1@news.ks.uiuc.edu>},
%   23rd September 2005.
%   \url{https://groups.google.com/group/comp.text.tex/msg/79648d4cf1f8bc13}
%
% \bibitem{kastrup}
%   David Kastrup, \Newsgroup{comp.text.tex},
%   \textit{Re: ANN: outerhbox.sty -- collect horizontal material,
%   for unboxing into a paragraph},
%   \nolinkurl{<85y855lrx3.fsf@lola.goethe.zz>},
%   7th October 2005.
%   \url{https://groups.google.com/group/comp.text.tex/msg/7cf0a345ef932e52}
%
% \bibitem{downes}
%   Michael Downes, \textit{Line breaking in \cs{unhbox}ed Text},
%   TUGboat 11 (1990), pp. 605--612.
%
% \bibitem{hyperref}
%   Sebastian Rahtz, Heiko Oberdiek:
%   \textit{The \xpackage{hyperref} package};
%   2006/08/16 v6.75c;
%   \CTANpkg{hyperref}.
%
% \end{thebibliography}
%
% \begin{History}
%   \begin{Version}{2005/10/05 v1.0}
%   \item
%     First version.
%   \end{Version}
%   \begin{Version}{2005/10/07 v1.1}
%   \item
%     Option \xoption{hyperref} added.
%   \end{Version}
%   \begin{Version}{2005/10/18 v1.2}
%   \item
%     Support for explicit line breaks added.
%   \end{Version}
%   \begin{Version}{2006/02/12 v1.3}
%   \item
%     DTX format.
%   \item
%     Documentation extended.
%   \end{Version}
%   \begin{Version}{2006/08/26 v1.4}
%   \item
%     Date of hyperref updated.
%   \end{Version}
%   \begin{Version}{2007/04/26 v1.5}
%   \item
%     Use of package \xpackage{infwarerr}.
%   \end{Version}
%   \begin{Version}{2007/05/17 v1.6}
%   \item
%     Standard header part for generic files.
%   \end{Version}
%   \begin{Version}{2007/09/09 v1.7}
%   \item
%     Catcode section added.
%   \end{Version}
%   \begin{Version}{2016/05/16 v1.8}
%   \item
%     Documentation updates.
%   \end{Version}
% \end{History}
%
% \PrintIndex
%
% \Finale
\endinput

%        (quote the arguments according to the demands of your shell)
%
% Documentation:
%    (a) If setouterhbox.drv is present:
%           latex setouterhbox.drv
%    (b) Without setouterhbox.drv:
%           latex setouterhbox.dtx; ...
%    The class ltxdoc loads the configuration file ltxdoc.cfg
%    if available. Here you can specify further options, e.g.
%    use A4 as paper format:
%       \PassOptionsToClass{a4paper}{article}
%
%    Programm calls to get the documentation (example):
%       pdflatex setouterhbox.dtx
%       makeindex -s gind.ist setouterhbox.idx
%       pdflatex setouterhbox.dtx
%       makeindex -s gind.ist setouterhbox.idx
%       pdflatex setouterhbox.dtx
%
% Installation:
%    TDS:tex/generic/oberdiek/setouterhbox.sty
%    TDS:doc/latex/oberdiek/setouterhbox.pdf
%    TDS:doc/latex/oberdiek/setouterhbox-example.tex
%    TDS:doc/latex/oberdiek/test/setouterhbox-test1.tex
%    TDS:doc/latex/oberdiek/test/setouterhbox-test2.tex
%    TDS:source/latex/oberdiek/setouterhbox.dtx
%
%<*ignore>
\begingroup
  \catcode123=1 %
  \catcode125=2 %
  \def\x{LaTeX2e}%
\expandafter\endgroup
\ifcase 0\ifx\install y1\fi\expandafter
         \ifx\csname processbatchFile\endcsname\relax\else1\fi
         \ifx\fmtname\x\else 1\fi\relax
\else\csname fi\endcsname
%</ignore>
%<*install>
\input docstrip.tex
\Msg{************************************************************************}
\Msg{* Installation}
\Msg{* Package: setouterhbox 2016/05/16 v1.8 Set hbox in outer horizontal mode (HO)}
\Msg{************************************************************************}

\keepsilent
\askforoverwritefalse

\let\MetaPrefix\relax
\preamble

This is a generated file.

Project: setouterhbox
Version: 2016/05/16 v1.8

Copyright (C)
   2005-2007 Heiko Oberdiek
   2016-2019 Oberdiek Package Support Group

This work may be distributed and/or modified under the
conditions of the LaTeX Project Public License, either
version 1.3c of this license or (at your option) any later
version. This version of this license is in
   https://www.latex-project.org/lppl/lppl-1-3c.txt
and the latest version of this license is in
   https://www.latex-project.org/lppl.txt
and version 1.3 or later is part of all distributions of
LaTeX version 2005/12/01 or later.

This work has the LPPL maintenance status "maintained".

The Current Maintainers of this work are
Heiko Oberdiek and the Oberdiek Package Support Group
https://github.com/ho-tex/oberdiek/issues


The Base Interpreter refers to any `TeX-Format',
because some files are installed in TDS:tex/generic//.

This work consists of the main source file setouterhbox.dtx
and the derived files
   setouterhbox.sty, setouterhbox.pdf, setouterhbox.ins, setouterhbox.drv,
   setouterhbox-example.tex, setouterhbox-test1.tex,
   setouterhbox-test2.tex.

\endpreamble
\let\MetaPrefix\DoubleperCent

\generate{%
  \file{setouterhbox.ins}{\from{setouterhbox.dtx}{install}}%
  \file{setouterhbox.drv}{\from{setouterhbox.dtx}{driver}}%
  \usedir{tex/generic/oberdiek}%
  \file{setouterhbox.sty}{\from{setouterhbox.dtx}{package}}%
  \usedir{doc/latex/oberdiek}%
  \file{setouterhbox-example.tex}{\from{setouterhbox.dtx}{example}}%
%  \usedir{doc/latex/oberdiek/test}%
%  \file{setouterhbox-test1.tex}{\from{setouterhbox.dtx}{test1}}%
%  \file{setouterhbox-test2.tex}{\from{setouterhbox.dtx}{test2}}%
  \nopreamble
  \nopostamble
%  \usedir{source/latex/oberdiek/catalogue}%
%  \file{setouterhbox.xml}{\from{setouterhbox.dtx}{catalogue}}%
}

\catcode32=13\relax% active space
\let =\space%
\Msg{************************************************************************}
\Msg{*}
\Msg{* To finish the installation you have to move the following}
\Msg{* file into a directory searched by TeX:}
\Msg{*}
\Msg{*     setouterhbox.sty}
\Msg{*}
\Msg{* To produce the documentation run the file `setouterhbox.drv'}
\Msg{* through LaTeX.}
\Msg{*}
\Msg{* Happy TeXing!}
\Msg{*}
\Msg{************************************************************************}

\endbatchfile
%</install>
%<*ignore>
\fi
%</ignore>
%<*driver>
\NeedsTeXFormat{LaTeX2e}
\ProvidesFile{setouterhbox.drv}%
  [2016/05/16 v1.8 Set hbox in outer horizontal mode (HO)]%
\documentclass{ltxdoc}
\usepackage{holtxdoc}[2011/11/22]
\begin{document}
  \DocInput{setouterhbox.dtx}%
\end{document}
%</driver>
% \fi
%
%
% \CharacterTable
%  {Upper-case    \A\B\C\D\E\F\G\H\I\J\K\L\M\N\O\P\Q\R\S\T\U\V\W\X\Y\Z
%   Lower-case    \a\b\c\d\e\f\g\h\i\j\k\l\m\n\o\p\q\r\s\t\u\v\w\x\y\z
%   Digits        \0\1\2\3\4\5\6\7\8\9
%   Exclamation   \!     Double quote  \"     Hash (number) \#
%   Dollar        \$     Percent       \%     Ampersand     \&
%   Acute accent  \'     Left paren    \(     Right paren   \)
%   Asterisk      \*     Plus          \+     Comma         \,
%   Minus         \-     Point         \.     Solidus       \/
%   Colon         \:     Semicolon     \;     Less than     \<
%   Equals        \=     Greater than  \>     Question mark \?
%   Commercial at \@     Left bracket  \[     Backslash     \\
%   Right bracket \]     Circumflex    \^     Underscore    \_
%   Grave accent  \`     Left brace    \{     Vertical bar  \|
%   Right brace   \}     Tilde         \~}
%
% \GetFileInfo{setouterhbox.drv}
%
% \title{The \xpackage{setouterhbox} package}
% \date{2016/05/16 v1.8}
% \author{Heiko Oberdiek\thanks
% {Please report any issues at \url{https://github.com/ho-tex/oberdiek/issues}}}
%
% \maketitle
%
% \begin{abstract}
% If math stuff is set in an \cs{hbox}, then TeX
% performs some optimization and omits the implicite
% penalties \cs{binoppenalty} and \cs{relpenalty}.
% This packages tries to put stuff into an \cs{hbox}
% without getting lost of those penalties.
% \end{abstract}
%
% \tableofcontents
%
% \section{Documentation}
%
% \subsection{Introduction}
%
% There is a situation in \xpackage{hyperref}'s driver for dvips
% where the user wants to have links that can be broken across
% lines. However dvips doesn't support the feature. With option
% \xoption{breaklinks} \xpackage{hyperref} sets the links as
% usual, put them in a box and write the link data with
% box dimensions into the appropriate \cs{special}s.
% Then, however, it does not set the complete unbreakable
% box, but it unwrappes the material inside to allow line
% breaks. Of course line breaking and glue setting will falsify
% the link dimensions, but line breaking was more important
% for the user.
%
% \subsection{Acknowledgement}
%
% Jonathan Fine, Donald Arsenau and me discussed the problem
% in the newsgroup \xnewsgroup{comp.text.tex} where Damian
% Menscher has started the thread, see \cite{newsstart}.
%
% The discussion was productive and generated many ideas
% and code examples. In order to have a more permanent
% result I wrote this package and tried to implement
% most of the ideas, a kind of summary of the discussion.
% Thus I want and have to thank Jonathan Fine and Donald Arsenau
% very much.
%
% Two weeks later David Kastrup (posting in
% \xnewsgroup{comp.text.tex}, \cite{kastrup})
% remembered an old article of Michael Downes (\cite{downes})
% in TUGboat, where Michael Downes already presented the
% method we discuss here. Nowadays we have \eTeX\ that extends
% the tool set of a \TeX\ macro programmer. Especially useful
% \eTeX\ was in this package for detecting and dealing with
% errorneous situations.
%
% However also nowadays a perfect solution for the problem
% is still missing at macro level. Probably someone has
% to go deep in the internals of the \TeX\ compiler to
% implement a switch that let penalties stay where otherwise
% \TeX\ would remove them for optimization reasons.
%
% \subsection{Usage}
%
% \paragraph{Package loading.}
% \LaTeX: as usually:
% \begin{quote}
%   |\usepackage{setouterhbox}|
% \end{quote}
% The package can also be included directly, thus \plainTeX\ users
% write:
% \begin{quote}
%   |\input setouterhbox.sty|
% \end{quote}
%
% \paragraph{Register allocation.}
% The material will be put into a box, thus we need to know these
% box number. If you need to allocate a new box register:
% \begin{description}
%  \item[\LaTeX:] |\newsavebox{\|\meta{name}|}|
%  \item[\plainTeX:] |\newbox\|\meta{name}
% \end{description}
% Then |\|\meta{name} is a command that held the box number.
%
% \paragraph{Box wrapping.}
% \LaTeX\ users put the material in the box with an environment
% similar to \texttt{lrbox}. The environment \texttt{setouterhbox}
% uses the same syntax and offers the same features, such
% as verbatim stuff inside:
% \begin{quote}
%  |\begin{setouterhbox}{|\meta{box number}|}|\dots
%  |\end{setouterhbox}|
% \end{quote}
% Users with \plainTeX\ do not have environments, they use instead:
% \begin{quote}
%   |\setouterhbox{|\meta{box number}|}|\dots|\endsetouterhbox|
% \end{quote}
% In both cases the material is put into an \cs{hbox} and assigned
% to the given box, denoted by \meta{box number}. Note the
% assignment is local, the same way \texttt{lrbox} behaves.
%
% \paragraph{Unwrapping.}
% The box material is ready for unwrapping:
% \begin{quote}
%   |\unhbox|\meta{box number}
% \end{quote}
%
% \subsection{Option \xoption{hyperref}}
%
% Package url uses math mode for typesetting urls.
% Break points are inserted by \cs{binoppenalty} and
% \cs{relpenalty}. Unhappily these break points are
% removed, if \xpackage{hyperref}
% is used with option {breaklinks}
% and drivers that depend on \xoption{pdfmark}:
% \xoption{dvips}, \xoption{vtexpdfmark}, \xoption{textures},
% and \xoption{dvipsone}.
% Thus the option \xoption{hyperref} enables the method
% of this package to avoid the removal of \cs{relpenalty}
% and \cs{binoppenalty}. Thus you get more break points.
% However, the link areas are still wrong for these
% drivers, because they are not supporting broken
% links.
%
% Note, you need version 2006/08/16 v6.75c of package \xpackage{hyperref},
% because starting with this version the necessary hook is provided
% that package \xpackage{setouterhbox} uses.
% \begin{quote}
%   |\usepackage[|\dots|]{hyperref}[2006/08/16]|\\
%   |\usepackage[hyperref]{setouterhbox}|
% \end{quote}
% Package order does not matter.
%
% \subsection{Example}
%
%    \begin{macrocode}
%<*example>
\documentclass[a5paper]{article}
\usepackage{url}[2005/06/27]
\usepackage{setouterhbox}

\newsavebox{\testbox}

\setlength{\parindent}{0pt}
\setlength{\parskip}{2em}

\begin{document}
\raggedright

\url{http://this.is.a.very.long.host.name/followed/%
by/a/very_long_long_long_path.html}%

\sbox\testbox{%
  \url{http://this.is.a.very.long.host.name/followed/%
  by/a/very_long_long_long_path.html}%
}%
\unhbox\testbox

\begin{setouterhbox}{\testbox}%
  \url{http://this.is.a.very.long.host.name/followed/%
  by/a/very_long_long_long_path.html}%
\end{setouterhbox}
\unhbox\testbox

\end{document}
%</example>
%    \end{macrocode}
%
% \StopEventually{
% }
%
% \section{Implementation}
%
% Internal macros are prefixed by \cs{setouterhbox}, |@| is
% not used inside names, thus we do not need to care of its
% catcode if we are not using it as \LaTeX\ package.
%
% \subsection{Package start stuff}
%
%    \begin{macrocode}
%<*package>
%    \end{macrocode}
%
% Prevent reloading more than one, necessary for \plainTeX:
%    Reload check, especially if the package is not used with \LaTeX.
%    \begin{macrocode}
\begingroup\catcode61\catcode48\catcode32=10\relax%
  \catcode13=5 % ^^M
  \endlinechar=13 %
  \catcode35=6 % #
  \catcode39=12 % '
  \catcode44=12 % ,
  \catcode45=12 % -
  \catcode46=12 % .
  \catcode58=12 % :
  \catcode64=11 % @
  \catcode123=1 % {
  \catcode125=2 % }
  \expandafter\let\expandafter\x\csname ver@setouterhbox.sty\endcsname
  \ifx\x\relax % plain-TeX, first loading
  \else
    \def\empty{}%
    \ifx\x\empty % LaTeX, first loading,
      % variable is initialized, but \ProvidesPackage not yet seen
    \else
      \expandafter\ifx\csname PackageInfo\endcsname\relax
        \def\x#1#2{%
          \immediate\write-1{Package #1 Info: #2.}%
        }%
      \else
        \def\x#1#2{\PackageInfo{#1}{#2, stopped}}%
      \fi
      \x{setouterhbox}{The package is already loaded}%
      \aftergroup\endinput
    \fi
  \fi
\endgroup%
%    \end{macrocode}
%    Package identification:
%    \begin{macrocode}
\begingroup\catcode61\catcode48\catcode32=10\relax%
  \catcode13=5 % ^^M
  \endlinechar=13 %
  \catcode35=6 % #
  \catcode39=12 % '
  \catcode40=12 % (
  \catcode41=12 % )
  \catcode44=12 % ,
  \catcode45=12 % -
  \catcode46=12 % .
  \catcode47=12 % /
  \catcode58=12 % :
  \catcode64=11 % @
  \catcode91=12 % [
  \catcode93=12 % ]
  \catcode123=1 % {
  \catcode125=2 % }
  \expandafter\ifx\csname ProvidesPackage\endcsname\relax
    \def\x#1#2#3[#4]{\endgroup
      \immediate\write-1{Package: #3 #4}%
      \xdef#1{#4}%
    }%
  \else
    \def\x#1#2[#3]{\endgroup
      #2[{#3}]%
      \ifx#1\@undefined
        \xdef#1{#3}%
      \fi
      \ifx#1\relax
        \xdef#1{#3}%
      \fi
    }%
  \fi
\expandafter\x\csname ver@setouterhbox.sty\endcsname
\ProvidesPackage{setouterhbox}%
  [2016/05/16 v1.8 Set hbox in outer horizontal mode (HO)]%
%    \end{macrocode}
%
%    \begin{macrocode}
\begingroup\catcode61\catcode48\catcode32=10\relax%
  \catcode13=5 % ^^M
  \endlinechar=13 %
  \catcode123=1 % {
  \catcode125=2 % }
  \catcode64=11 % @
  \def\x{\endgroup
    \expandafter\edef\csname setouterhboxAtEnd\endcsname{%
      \endlinechar=\the\endlinechar\relax
      \catcode13=\the\catcode13\relax
      \catcode32=\the\catcode32\relax
      \catcode35=\the\catcode35\relax
      \catcode61=\the\catcode61\relax
      \catcode64=\the\catcode64\relax
      \catcode123=\the\catcode123\relax
      \catcode125=\the\catcode125\relax
    }%
  }%
\x\catcode61\catcode48\catcode32=10\relax%
\catcode13=5 % ^^M
\endlinechar=13 %
\catcode35=6 % #
\catcode64=11 % @
\catcode123=1 % {
\catcode125=2 % }
\def\TMP@EnsureCode#1#2{%
  \edef\setouterhboxAtEnd{%
    \setouterhboxAtEnd
    \catcode#1=\the\catcode#1\relax
  }%
  \catcode#1=#2\relax
}
\TMP@EnsureCode{40}{12}% (
\TMP@EnsureCode{41}{12}% )
\TMP@EnsureCode{44}{12}% ,
\TMP@EnsureCode{45}{12}% -
\TMP@EnsureCode{46}{12}% .
\TMP@EnsureCode{47}{12}% /
\TMP@EnsureCode{58}{12}% :
\TMP@EnsureCode{60}{12}% <
\TMP@EnsureCode{62}{12}% >
\TMP@EnsureCode{91}{12}% [
\TMP@EnsureCode{93}{12}% ]
\TMP@EnsureCode{96}{12}% `
\edef\setouterhboxAtEnd{\setouterhboxAtEnd\noexpand\endinput}
%    \end{macrocode}
%
% \subsection{Interface macros}
%
%    \begin{macro}{\setouterhboxBox}
% The method requires a global box assignment. To be on the
% safe side, a new box register is allocated for this
% global box assignment.
%    \begin{macrocode}
\newbox\setouterhboxBox
%    \end{macrocode}
%    \end{macro}
%
%    \begin{macro}{\setouterhboxFailure}
% Error message for both \plainTeX\ and \LaTeX
%    \begin{macrocode}
\begingroup\expandafter\expandafter\expandafter\endgroup
\expandafter\ifx\csname RequirePackage\endcsname\relax
  \input infwarerr.sty\relax
\else
  \RequirePackage{infwarerr}[2016/05/16]%
\fi
\edef\setouterhboxFailure#1#2{%
  \expandafter\noexpand\csname @PackageError\endcsname
      {setouterhbox}{#1}{#2}%
}
%    \end{macrocode}
%    \end{macro}
%
% \subsection{Main part}
%
% eTeX provides much better means for checking
% error conditions. Thus lines marked by "E" are executed
% if eTeX is available, otherwise the lines marked by "T" are
% used.
%    \begin{macrocode}
\begingroup\expandafter\expandafter\expandafter\endgroup
\expandafter\ifx\csname lastnodetype\endcsname\relax
  \catcode`T=9 % ignore
  \catcode`E=14 % comment
\else
  \catcode`T=14 % comment
  \catcode`E=9 % ignore
\fi
%    \end{macrocode}
%
%    \begin{macro}{\setouterhboxRemove}
% Remove all kern, glue, and penalty nodes;
% poor man's version, if \eTeX\ is not available
%    \begin{macrocode}
\def\setouterhboxRemove{%
E \ifnum\lastnodetype<11 %
E   \else
E   \ifnum\lastnodetype>13 %
E   \else
      \unskip\unkern\unpenalty
E     \expandafter\expandafter\expandafter\setouterhboxRemove
E   \fi
E \fi
}%
%    \end{macrocode}
%    \end{macro}
%
%    \begin{macro}{\setouterhbox}
% Passing the box contents by macro parameter would prevent
% catcode changes in the box contents like by \cs{verb}.
% Also \cs{bgroup} and \cs{egroup} does not work, because stuff
% has to be added at the begin and end of the box, thus
% the syntax
% |\setouterhbox{|\meta{box number}|}|\dots|\endsetouterhbox|
% is used. Also we automatically get an environment \texttt{setouterhbox}
% if \LaTeX\ is used.
%    \begin{macrocode}
\def\setouterhbox#1{%
  \begingroup
    \def\setouterhboxNum{#1}%
    \setbox0\vbox\bgroup
T     \kern.123pt\relax % marker
T     \kern0pt\relax % removed by \setouterhboxRemove
      \begingroup
        \everypar{}%
        \noindent
}
%    \end{macrocode}
%    \end{macro}
%    \begin{macro}{\endsetouterhbox}
% Most of the work is done in the end part, thus the heart of
% the method follows:
%    \begin{macrocode}
\def\endsetouterhbox{%
      \endgroup
%    \end{macrocode}
% Omit the first pass to get the penalties
% of the second pass.
%    \begin{macrocode}
      \pretolerance-1 %
%    \end{macrocode}
%  We don't want a third pass with \cs{emergencystretch}.
%    \begin{macrocode}
      \tolerance10000 %
      \hsize\maxdimen
%    \end{macrocode}
% Line is not underfull:
%    \begin{macrocode}
      \parfillskip 0pt plus 1filll\relax
      \leftskip0pt\relax
%    \end{macrocode}
% Suppress underful \cs{hbox} warnings,
% is explicit line breaks are used.
%    \begin{macrocode}
      \rightskip0pt plus 1fil\relax
      \everypar{}%
%    \end{macrocode}
% Ensure that there is a paragraph and
% prevents \cs{endgraph} from eating terminal glue:
%    \begin{macrocode}
      \kern0pt%
      \endgraf
      \setouterhboxRemove
E     \ifnum\lastnodetype=1 %
E       \global\setbox\setouterhboxBox\lastbox
E       \loop
E         \setouterhboxRemove
E       \ifnum\lastnodetype=1 %
E         \setbox0=\lastbox
E         \global\setbox\setouterhboxBox=\hbox{%
E           \unhbox0 %
%    \end{macrocode}
% Remove \cs{rightskip}, a penalty with -10000 is part of the previous line.
%    \begin{macrocode}
E           \unskip
E           \unhbox\setouterhboxBox
E         }%
E       \repeat
E     \else
E       \setouterhboxFailure{%
E         Something is wrong%
E       }{%
E         Could not find expected line.%
E         \MessageBreak
E         (\string\lastnodetype: \number\lastnodetype, expected: 1)%
E       }%
E     \fi
E     \setouterhboxRemove
T     \global\setbox\setouterhboxBox\lastbox
T     \loop
T       \setouterhboxRemove
T       \setbox0=\lastbox
T     \ifcase\ifvoid0 1\else0\fi
T       \global\setbox\setouterhboxBox=\hbox{%
T         \unhbox0 %
%    \end{macrocode}
% Remove \cs{rightskip}, a penalty with -10000 is part of the previous line.
%    \begin{macrocode}
T         \unskip
T         \unhbox\setouterhboxBox
T       }%
T     \repeat
T     \ifdim.123pt=\lastkern
T     \else
T       \setouterhboxFailure{%
T         Something is wrong%
T       }{%
T         Unexpected stuff was detected before the line.%
T       }%
T     \fi
T   \egroup
T   \ifcase \ifnum\wd0=0 \else 1\fi
T           \ifdim\ht0=.123pt \else 1\fi
T           \ifnum\dp0=0 \else 1\fi
T           0 %
E     \ifnum\lastnodetype=-1 %
%    \end{macrocode}
% There was just one line that we have caught.
%    \begin{macrocode}
      \else
        \setouterhboxFailure{%
            Something is wrong%
        }{%
            After fetching the line there is more unexpected stuff.%
E           \MessageBreak
E           (\string\lastnodetype: \number\lastnodetype, expected: -1)%
        }%
      \fi
E   \egroup
  \expandafter\endgroup
  \expandafter\setouterhboxFinish\expandafter{%
    \number\setouterhboxNum
  }%
}
%    \end{macrocode}
%    \end{macro}
%
% \subsection{Environment support}
%
% Check \cs{@currenvir} for the case that \cs{setouterhbox}
% was called as environment. Then the box assignment
% must be put after the \cs{endgroup} of |\end{|\dots|}|.
%    \begin{macrocode}
\def\setouterhboxCurr{setouterhbox}
\def\setouterhboxLast#1{%
  \setbox#1\hbox{%
    \unhbox\setouterhboxBox
    \unskip % remove \rightskip glue
    \unskip % remove \parfillskip glue
    \unpenalty % remove paragraph ending \penalty 10000
    \unkern % remove explicit kern inserted above
  }%
}
%    \end{macrocode}
%    \begin{macro}{\setouterhboxFinish}
% |#1| is an explicit number.
%    \begin{macrocode}
\def\setouterhboxFinish#1{%
  \begingroup\expandafter\expandafter\expandafter\endgroup
  \expandafter\ifx\csname @currenvir\endcsname\setouterhboxCurr
    \aftergroup\setouterhboxLast
    \aftergroup{%
    \setouterhboxAfter #1\NIL
    \aftergroup}%
  \else
    \setouterhboxLast{#1}%
  \fi
}
%    \end{macrocode}
%    \end{macro}
%    \begin{macro}{\setouterhboxAfter}
% |#1| is an explicit number.
%    \begin{macrocode}
\def\setouterhboxAfter#1#2\NIL{%
  \aftergroup#1%
  \ifx\\#2\\%
  \else
    \setouterhboxReturnAfterFi{%
      \setouterhboxAfter#2\NIL
    }%
  \fi
}
%    \end{macrocode}
%    \end{macro}
%    \begin{macro}{\setouterhboxReturnAfterFi}
% A utility macro to get tail recursion.
%    \begin{macrocode}
\long\def\setouterhboxReturnAfterFi#1\fi{\fi#1}
%    \end{macrocode}
%    \end{macro}
% Restore catcodes we have need to distinguish between
% the implementation with and without \eTeX.
%    \begin{macrocode}
\catcode69=11\relax % E
\catcode84=11\relax % T
%    \end{macrocode}
%
% \subsection{Option \xoption{hyperref}}
%    \begin{macrocode}
\begingroup
  \def\x{LaTeX2e}%
\expandafter\endgroup
\ifx\x\fmtname
\else
  \expandafter\setouterhboxAtEnd
\fi%
%    \end{macrocode}
%    \begin{macro}{\Hy@setouterhbox}
% \cs{Hy@setouterhbox} is the internal hook that \xpackage{hyperref}
% uses since 2006/02/12 v6.75a.
%    \begin{macrocode}
\DeclareOption{hyperref}{%
  \long\def\Hy@setouterhbox#1#2{%
    \setouterhbox{#1}#2\endsetouterhbox
  }%
}
%    \end{macrocode}
%    \end{macro}
%    \begin{macrocode}
\ProcessOptions\relax
%    \end{macrocode}
%
%    \begin{macrocode}
\setouterhboxAtEnd%
%</package>
%    \end{macrocode}
%
% \section{Test}
%
% \subsection{Catcode checks for loading}
%
%    \begin{macrocode}
%<*test1>
%    \end{macrocode}
%    \begin{macrocode}
\catcode`\{=1 %
\catcode`\}=2 %
\catcode`\#=6 %
\catcode`\@=11 %
\expandafter\ifx\csname count@\endcsname\relax
  \countdef\count@=255 %
\fi
\expandafter\ifx\csname @gobble\endcsname\relax
  \long\def\@gobble#1{}%
\fi
\expandafter\ifx\csname @firstofone\endcsname\relax
  \long\def\@firstofone#1{#1}%
\fi
\expandafter\ifx\csname loop\endcsname\relax
  \expandafter\@firstofone
\else
  \expandafter\@gobble
\fi
{%
  \def\loop#1\repeat{%
    \def\body{#1}%
    \iterate
  }%
  \def\iterate{%
    \body
      \let\next\iterate
    \else
      \let\next\relax
    \fi
    \next
  }%
  \let\repeat=\fi
}%
\def\RestoreCatcodes{}
\count@=0 %
\loop
  \edef\RestoreCatcodes{%
    \RestoreCatcodes
    \catcode\the\count@=\the\catcode\count@\relax
  }%
\ifnum\count@<255 %
  \advance\count@ 1 %
\repeat

\def\RangeCatcodeInvalid#1#2{%
  \count@=#1\relax
  \loop
    \catcode\count@=15 %
  \ifnum\count@<#2\relax
    \advance\count@ 1 %
  \repeat
}
\def\RangeCatcodeCheck#1#2#3{%
  \count@=#1\relax
  \loop
    \ifnum#3=\catcode\count@
    \else
      \errmessage{%
        Character \the\count@\space
        with wrong catcode \the\catcode\count@\space
        instead of \number#3%
      }%
    \fi
  \ifnum\count@<#2\relax
    \advance\count@ 1 %
  \repeat
}
\def\space{ }
\expandafter\ifx\csname LoadCommand\endcsname\relax
  \def\LoadCommand{\input setouterhbox.sty\relax}%
\fi
\def\Test{%
  \RangeCatcodeInvalid{0}{47}%
  \RangeCatcodeInvalid{58}{64}%
  \RangeCatcodeInvalid{91}{96}%
  \RangeCatcodeInvalid{123}{255}%
  \catcode`\@=12 %
  \catcode`\\=0 %
  \catcode`\%=14 %
  \LoadCommand
  \RangeCatcodeCheck{0}{36}{15}%
  \RangeCatcodeCheck{37}{37}{14}%
  \RangeCatcodeCheck{38}{47}{15}%
  \RangeCatcodeCheck{48}{57}{12}%
  \RangeCatcodeCheck{58}{63}{15}%
  \RangeCatcodeCheck{64}{64}{12}%
  \RangeCatcodeCheck{65}{90}{11}%
  \RangeCatcodeCheck{91}{91}{15}%
  \RangeCatcodeCheck{92}{92}{0}%
  \RangeCatcodeCheck{93}{96}{15}%
  \RangeCatcodeCheck{97}{122}{11}%
  \RangeCatcodeCheck{123}{255}{15}%
  \RestoreCatcodes
}
\Test
\csname @@end\endcsname
\end
%    \end{macrocode}
%    \begin{macrocode}
%</test1>
%    \end{macrocode}
%
% \subsection{Test with package \xpackage{url}}
%
%    \begin{macrocode}
%<*test2>
\nofiles
\documentclass[a5paper]{article}
\usepackage{url}[2005/06/27]
\usepackage{setouterhbox}

\newsavebox{\testbox}

\setlength{\parindent}{0pt}
\setlength{\parskip}{2em}

\begin{document}
\raggedright

\url{http://this.is.a.very.long.host.name/followed/%
by/a/very_long_long_long_path.html}%

\sbox\testbox{%
  \url{http://this.is.a.very.long.host.name/followed/%
  by/a/very_long_long_long_path.html}%
}%
\unhbox\testbox

\begin{setouterhbox}{\testbox}%
  \url{http://this.is.a.very.long.host.name/followed/%
  by/a/very_long_long_long_path.html}%
\end{setouterhbox}
\unhbox\testbox

\end{document}
%</test2>
%    \end{macrocode}
%
% \section{Installation}
%
% \subsection{Download}
%
% \paragraph{Package.} This package is available on
% CTAN\footnote{\CTANpkg{setouterhbox}}:
% \begin{description}
% \item[\CTAN{macros/latex/contrib/oberdiek/setouterhbox.dtx}] The source file.
% \item[\CTAN{macros/latex/contrib/oberdiek/setouterhbox.pdf}] Documentation.
% \end{description}
%
%
% \paragraph{Bundle.} All the packages of the bundle `oberdiek'
% are also available in a TDS compliant ZIP archive. There
% the packages are already unpacked and the documentation files
% are generated. The files and directories obey the TDS standard.
% \begin{description}
% \item[\CTANinstall{install/macros/latex/contrib/oberdiek.tds.zip}]
% \end{description}
% \emph{TDS} refers to the standard ``A Directory Structure
% for \TeX\ Files'' (\CTAN{tds/tds.pdf}). Directories
% with \xfile{texmf} in their name are usually organized this way.
%
% \subsection{Bundle installation}
%
% \paragraph{Unpacking.} Unpack the \xfile{oberdiek.tds.zip} in the
% TDS tree (also known as \xfile{texmf} tree) of your choice.
% Example (linux):
% \begin{quote}
%   |unzip oberdiek.tds.zip -d ~/texmf|
% \end{quote}
%
% \subsection{Package installation}
%
% \paragraph{Unpacking.} The \xfile{.dtx} file is a self-extracting
% \docstrip\ archive. The files are extracted by running the
% \xfile{.dtx} through \plainTeX:
% \begin{quote}
%   \verb|tex setouterhbox.dtx|
% \end{quote}
%
% \paragraph{TDS.} Now the different files must be moved into
% the different directories in your installation TDS tree
% (also known as \xfile{texmf} tree):
% \begin{quote}
% \def\t{^^A
% \begin{tabular}{@{}>{\ttfamily}l@{ $\rightarrow$ }>{\ttfamily}l@{}}
%   setouterhbox.sty & tex/generic/oberdiek/setouterhbox.sty\\
%   setouterhbox.pdf & doc/latex/oberdiek/setouterhbox.pdf\\
%   setouterhbox-example.tex & doc/latex/oberdiek/setouterhbox-example.tex\\
%   test/setouterhbox-test1.tex & doc/latex/oberdiek/test/setouterhbox-test1.tex\\
%   test/setouterhbox-test2.tex & doc/latex/oberdiek/test/setouterhbox-test2.tex\\
%   setouterhbox.dtx & source/latex/oberdiek/setouterhbox.dtx\\
% \end{tabular}^^A
% }^^A
% \sbox0{\t}^^A
% \ifdim\wd0>\linewidth
%   \begingroup
%     \advance\linewidth by\leftmargin
%     \advance\linewidth by\rightmargin
%   \edef\x{\endgroup
%     \def\noexpand\lw{\the\linewidth}^^A
%   }\x
%   \def\lwbox{^^A
%     \leavevmode
%     \hbox to \linewidth{^^A
%       \kern-\leftmargin\relax
%       \hss
%       \usebox0
%       \hss
%       \kern-\rightmargin\relax
%     }^^A
%   }^^A
%   \ifdim\wd0>\lw
%     \sbox0{\small\t}^^A
%     \ifdim\wd0>\linewidth
%       \ifdim\wd0>\lw
%         \sbox0{\footnotesize\t}^^A
%         \ifdim\wd0>\linewidth
%           \ifdim\wd0>\lw
%             \sbox0{\scriptsize\t}^^A
%             \ifdim\wd0>\linewidth
%               \ifdim\wd0>\lw
%                 \sbox0{\tiny\t}^^A
%                 \ifdim\wd0>\linewidth
%                   \lwbox
%                 \else
%                   \usebox0
%                 \fi
%               \else
%                 \lwbox
%               \fi
%             \else
%               \usebox0
%             \fi
%           \else
%             \lwbox
%           \fi
%         \else
%           \usebox0
%         \fi
%       \else
%         \lwbox
%       \fi
%     \else
%       \usebox0
%     \fi
%   \else
%     \lwbox
%   \fi
% \else
%   \usebox0
% \fi
% \end{quote}
% If you have a \xfile{docstrip.cfg} that configures and enables \docstrip's
% TDS installing feature, then some files can already be in the right
% place, see the documentation of \docstrip.
%
% \subsection{Refresh file name databases}
%
% If your \TeX~distribution
% (\TeX\,Live, \mikTeX, \dots) relies on file name databases, you must refresh
% these. For example, \TeX\,Live\ users run \verb|texhash| or
% \verb|mktexlsr|.
%
% \subsection{Some details for the interested}
%
% \paragraph{Unpacking with \LaTeX.}
% The \xfile{.dtx} chooses its action depending on the format:
% \begin{description}
% \item[\plainTeX:] Run \docstrip\ and extract the files.
% \item[\LaTeX:] Generate the documentation.
% \end{description}
% If you insist on using \LaTeX\ for \docstrip\ (really,
% \docstrip\ does not need \LaTeX), then inform the autodetect routine
% about your intention:
% \begin{quote}
%   \verb|latex \let\install=y% \iffalse meta-comment
%
% File: setouterhbox.dtx
% Version: 2016/05/16 v1.8
% Info: Set hbox in outer horizontal mode
%
% Copyright (C)
%    2005-2007 Heiko Oberdiek
%    2016-2019 Oberdiek Package Support Group
%    https://github.com/ho-tex/oberdiek/issues
%
% This work may be distributed and/or modified under the
% conditions of the LaTeX Project Public License, either
% version 1.3c of this license or (at your option) any later
% version. This version of this license is in
%    https://www.latex-project.org/lppl/lppl-1-3c.txt
% and the latest version of this license is in
%    https://www.latex-project.org/lppl.txt
% and version 1.3 or later is part of all distributions of
% LaTeX version 2005/12/01 or later.
%
% This work has the LPPL maintenance status "maintained".
%
% The Current Maintainers of this work are
% Heiko Oberdiek and the Oberdiek Package Support Group
% https://github.com/ho-tex/oberdiek/issues
%
% The Base Interpreter refers to any `TeX-Format',
% because some files are installed in TDS:tex/generic//.
%
% This work consists of the main source file setouterhbox.dtx
% and the derived files
%    setouterhbox.sty, setouterhbox.pdf, setouterhbox.ins, setouterhbox.drv,
%    setouterhbox-example.tex, setouterhbox-test1.tex,
%    setouterhbox-test2.tex.
%
% Distribution:
%    CTAN:macros/latex/contrib/oberdiek/setouterhbox.dtx
%    CTAN:macros/latex/contrib/oberdiek/setouterhbox.pdf
%
% Unpacking:
%    (a) If setouterhbox.ins is present:
%           tex setouterhbox.ins
%    (b) Without setouterhbox.ins:
%           tex setouterhbox.dtx
%    (c) If you insist on using LaTeX
%           latex \let\install=y\input{setouterhbox.dtx}
%        (quote the arguments according to the demands of your shell)
%
% Documentation:
%    (a) If setouterhbox.drv is present:
%           latex setouterhbox.drv
%    (b) Without setouterhbox.drv:
%           latex setouterhbox.dtx; ...
%    The class ltxdoc loads the configuration file ltxdoc.cfg
%    if available. Here you can specify further options, e.g.
%    use A4 as paper format:
%       \PassOptionsToClass{a4paper}{article}
%
%    Programm calls to get the documentation (example):
%       pdflatex setouterhbox.dtx
%       makeindex -s gind.ist setouterhbox.idx
%       pdflatex setouterhbox.dtx
%       makeindex -s gind.ist setouterhbox.idx
%       pdflatex setouterhbox.dtx
%
% Installation:
%    TDS:tex/generic/oberdiek/setouterhbox.sty
%    TDS:doc/latex/oberdiek/setouterhbox.pdf
%    TDS:doc/latex/oberdiek/setouterhbox-example.tex
%    TDS:doc/latex/oberdiek/test/setouterhbox-test1.tex
%    TDS:doc/latex/oberdiek/test/setouterhbox-test2.tex
%    TDS:source/latex/oberdiek/setouterhbox.dtx
%
%<*ignore>
\begingroup
  \catcode123=1 %
  \catcode125=2 %
  \def\x{LaTeX2e}%
\expandafter\endgroup
\ifcase 0\ifx\install y1\fi\expandafter
         \ifx\csname processbatchFile\endcsname\relax\else1\fi
         \ifx\fmtname\x\else 1\fi\relax
\else\csname fi\endcsname
%</ignore>
%<*install>
\input docstrip.tex
\Msg{************************************************************************}
\Msg{* Installation}
\Msg{* Package: setouterhbox 2016/05/16 v1.8 Set hbox in outer horizontal mode (HO)}
\Msg{************************************************************************}

\keepsilent
\askforoverwritefalse

\let\MetaPrefix\relax
\preamble

This is a generated file.

Project: setouterhbox
Version: 2016/05/16 v1.8

Copyright (C)
   2005-2007 Heiko Oberdiek
   2016-2019 Oberdiek Package Support Group

This work may be distributed and/or modified under the
conditions of the LaTeX Project Public License, either
version 1.3c of this license or (at your option) any later
version. This version of this license is in
   https://www.latex-project.org/lppl/lppl-1-3c.txt
and the latest version of this license is in
   https://www.latex-project.org/lppl.txt
and version 1.3 or later is part of all distributions of
LaTeX version 2005/12/01 or later.

This work has the LPPL maintenance status "maintained".

The Current Maintainers of this work are
Heiko Oberdiek and the Oberdiek Package Support Group
https://github.com/ho-tex/oberdiek/issues


The Base Interpreter refers to any `TeX-Format',
because some files are installed in TDS:tex/generic//.

This work consists of the main source file setouterhbox.dtx
and the derived files
   setouterhbox.sty, setouterhbox.pdf, setouterhbox.ins, setouterhbox.drv,
   setouterhbox-example.tex, setouterhbox-test1.tex,
   setouterhbox-test2.tex.

\endpreamble
\let\MetaPrefix\DoubleperCent

\generate{%
  \file{setouterhbox.ins}{\from{setouterhbox.dtx}{install}}%
  \file{setouterhbox.drv}{\from{setouterhbox.dtx}{driver}}%
  \usedir{tex/generic/oberdiek}%
  \file{setouterhbox.sty}{\from{setouterhbox.dtx}{package}}%
  \usedir{doc/latex/oberdiek}%
  \file{setouterhbox-example.tex}{\from{setouterhbox.dtx}{example}}%
%  \usedir{doc/latex/oberdiek/test}%
%  \file{setouterhbox-test1.tex}{\from{setouterhbox.dtx}{test1}}%
%  \file{setouterhbox-test2.tex}{\from{setouterhbox.dtx}{test2}}%
  \nopreamble
  \nopostamble
%  \usedir{source/latex/oberdiek/catalogue}%
%  \file{setouterhbox.xml}{\from{setouterhbox.dtx}{catalogue}}%
}

\catcode32=13\relax% active space
\let =\space%
\Msg{************************************************************************}
\Msg{*}
\Msg{* To finish the installation you have to move the following}
\Msg{* file into a directory searched by TeX:}
\Msg{*}
\Msg{*     setouterhbox.sty}
\Msg{*}
\Msg{* To produce the documentation run the file `setouterhbox.drv'}
\Msg{* through LaTeX.}
\Msg{*}
\Msg{* Happy TeXing!}
\Msg{*}
\Msg{************************************************************************}

\endbatchfile
%</install>
%<*ignore>
\fi
%</ignore>
%<*driver>
\NeedsTeXFormat{LaTeX2e}
\ProvidesFile{setouterhbox.drv}%
  [2016/05/16 v1.8 Set hbox in outer horizontal mode (HO)]%
\documentclass{ltxdoc}
\usepackage{holtxdoc}[2011/11/22]
\begin{document}
  \DocInput{setouterhbox.dtx}%
\end{document}
%</driver>
% \fi
%
%
% \CharacterTable
%  {Upper-case    \A\B\C\D\E\F\G\H\I\J\K\L\M\N\O\P\Q\R\S\T\U\V\W\X\Y\Z
%   Lower-case    \a\b\c\d\e\f\g\h\i\j\k\l\m\n\o\p\q\r\s\t\u\v\w\x\y\z
%   Digits        \0\1\2\3\4\5\6\7\8\9
%   Exclamation   \!     Double quote  \"     Hash (number) \#
%   Dollar        \$     Percent       \%     Ampersand     \&
%   Acute accent  \'     Left paren    \(     Right paren   \)
%   Asterisk      \*     Plus          \+     Comma         \,
%   Minus         \-     Point         \.     Solidus       \/
%   Colon         \:     Semicolon     \;     Less than     \<
%   Equals        \=     Greater than  \>     Question mark \?
%   Commercial at \@     Left bracket  \[     Backslash     \\
%   Right bracket \]     Circumflex    \^     Underscore    \_
%   Grave accent  \`     Left brace    \{     Vertical bar  \|
%   Right brace   \}     Tilde         \~}
%
% \GetFileInfo{setouterhbox.drv}
%
% \title{The \xpackage{setouterhbox} package}
% \date{2016/05/16 v1.8}
% \author{Heiko Oberdiek\thanks
% {Please report any issues at \url{https://github.com/ho-tex/oberdiek/issues}}}
%
% \maketitle
%
% \begin{abstract}
% If math stuff is set in an \cs{hbox}, then TeX
% performs some optimization and omits the implicite
% penalties \cs{binoppenalty} and \cs{relpenalty}.
% This packages tries to put stuff into an \cs{hbox}
% without getting lost of those penalties.
% \end{abstract}
%
% \tableofcontents
%
% \section{Documentation}
%
% \subsection{Introduction}
%
% There is a situation in \xpackage{hyperref}'s driver for dvips
% where the user wants to have links that can be broken across
% lines. However dvips doesn't support the feature. With option
% \xoption{breaklinks} \xpackage{hyperref} sets the links as
% usual, put them in a box and write the link data with
% box dimensions into the appropriate \cs{special}s.
% Then, however, it does not set the complete unbreakable
% box, but it unwrappes the material inside to allow line
% breaks. Of course line breaking and glue setting will falsify
% the link dimensions, but line breaking was more important
% for the user.
%
% \subsection{Acknowledgement}
%
% Jonathan Fine, Donald Arsenau and me discussed the problem
% in the newsgroup \xnewsgroup{comp.text.tex} where Damian
% Menscher has started the thread, see \cite{newsstart}.
%
% The discussion was productive and generated many ideas
% and code examples. In order to have a more permanent
% result I wrote this package and tried to implement
% most of the ideas, a kind of summary of the discussion.
% Thus I want and have to thank Jonathan Fine and Donald Arsenau
% very much.
%
% Two weeks later David Kastrup (posting in
% \xnewsgroup{comp.text.tex}, \cite{kastrup})
% remembered an old article of Michael Downes (\cite{downes})
% in TUGboat, where Michael Downes already presented the
% method we discuss here. Nowadays we have \eTeX\ that extends
% the tool set of a \TeX\ macro programmer. Especially useful
% \eTeX\ was in this package for detecting and dealing with
% errorneous situations.
%
% However also nowadays a perfect solution for the problem
% is still missing at macro level. Probably someone has
% to go deep in the internals of the \TeX\ compiler to
% implement a switch that let penalties stay where otherwise
% \TeX\ would remove them for optimization reasons.
%
% \subsection{Usage}
%
% \paragraph{Package loading.}
% \LaTeX: as usually:
% \begin{quote}
%   |\usepackage{setouterhbox}|
% \end{quote}
% The package can also be included directly, thus \plainTeX\ users
% write:
% \begin{quote}
%   |\input setouterhbox.sty|
% \end{quote}
%
% \paragraph{Register allocation.}
% The material will be put into a box, thus we need to know these
% box number. If you need to allocate a new box register:
% \begin{description}
%  \item[\LaTeX:] |\newsavebox{\|\meta{name}|}|
%  \item[\plainTeX:] |\newbox\|\meta{name}
% \end{description}
% Then |\|\meta{name} is a command that held the box number.
%
% \paragraph{Box wrapping.}
% \LaTeX\ users put the material in the box with an environment
% similar to \texttt{lrbox}. The environment \texttt{setouterhbox}
% uses the same syntax and offers the same features, such
% as verbatim stuff inside:
% \begin{quote}
%  |\begin{setouterhbox}{|\meta{box number}|}|\dots
%  |\end{setouterhbox}|
% \end{quote}
% Users with \plainTeX\ do not have environments, they use instead:
% \begin{quote}
%   |\setouterhbox{|\meta{box number}|}|\dots|\endsetouterhbox|
% \end{quote}
% In both cases the material is put into an \cs{hbox} and assigned
% to the given box, denoted by \meta{box number}. Note the
% assignment is local, the same way \texttt{lrbox} behaves.
%
% \paragraph{Unwrapping.}
% The box material is ready for unwrapping:
% \begin{quote}
%   |\unhbox|\meta{box number}
% \end{quote}
%
% \subsection{Option \xoption{hyperref}}
%
% Package url uses math mode for typesetting urls.
% Break points are inserted by \cs{binoppenalty} and
% \cs{relpenalty}. Unhappily these break points are
% removed, if \xpackage{hyperref}
% is used with option {breaklinks}
% and drivers that depend on \xoption{pdfmark}:
% \xoption{dvips}, \xoption{vtexpdfmark}, \xoption{textures},
% and \xoption{dvipsone}.
% Thus the option \xoption{hyperref} enables the method
% of this package to avoid the removal of \cs{relpenalty}
% and \cs{binoppenalty}. Thus you get more break points.
% However, the link areas are still wrong for these
% drivers, because they are not supporting broken
% links.
%
% Note, you need version 2006/08/16 v6.75c of package \xpackage{hyperref},
% because starting with this version the necessary hook is provided
% that package \xpackage{setouterhbox} uses.
% \begin{quote}
%   |\usepackage[|\dots|]{hyperref}[2006/08/16]|\\
%   |\usepackage[hyperref]{setouterhbox}|
% \end{quote}
% Package order does not matter.
%
% \subsection{Example}
%
%    \begin{macrocode}
%<*example>
\documentclass[a5paper]{article}
\usepackage{url}[2005/06/27]
\usepackage{setouterhbox}

\newsavebox{\testbox}

\setlength{\parindent}{0pt}
\setlength{\parskip}{2em}

\begin{document}
\raggedright

\url{http://this.is.a.very.long.host.name/followed/%
by/a/very_long_long_long_path.html}%

\sbox\testbox{%
  \url{http://this.is.a.very.long.host.name/followed/%
  by/a/very_long_long_long_path.html}%
}%
\unhbox\testbox

\begin{setouterhbox}{\testbox}%
  \url{http://this.is.a.very.long.host.name/followed/%
  by/a/very_long_long_long_path.html}%
\end{setouterhbox}
\unhbox\testbox

\end{document}
%</example>
%    \end{macrocode}
%
% \StopEventually{
% }
%
% \section{Implementation}
%
% Internal macros are prefixed by \cs{setouterhbox}, |@| is
% not used inside names, thus we do not need to care of its
% catcode if we are not using it as \LaTeX\ package.
%
% \subsection{Package start stuff}
%
%    \begin{macrocode}
%<*package>
%    \end{macrocode}
%
% Prevent reloading more than one, necessary for \plainTeX:
%    Reload check, especially if the package is not used with \LaTeX.
%    \begin{macrocode}
\begingroup\catcode61\catcode48\catcode32=10\relax%
  \catcode13=5 % ^^M
  \endlinechar=13 %
  \catcode35=6 % #
  \catcode39=12 % '
  \catcode44=12 % ,
  \catcode45=12 % -
  \catcode46=12 % .
  \catcode58=12 % :
  \catcode64=11 % @
  \catcode123=1 % {
  \catcode125=2 % }
  \expandafter\let\expandafter\x\csname ver@setouterhbox.sty\endcsname
  \ifx\x\relax % plain-TeX, first loading
  \else
    \def\empty{}%
    \ifx\x\empty % LaTeX, first loading,
      % variable is initialized, but \ProvidesPackage not yet seen
    \else
      \expandafter\ifx\csname PackageInfo\endcsname\relax
        \def\x#1#2{%
          \immediate\write-1{Package #1 Info: #2.}%
        }%
      \else
        \def\x#1#2{\PackageInfo{#1}{#2, stopped}}%
      \fi
      \x{setouterhbox}{The package is already loaded}%
      \aftergroup\endinput
    \fi
  \fi
\endgroup%
%    \end{macrocode}
%    Package identification:
%    \begin{macrocode}
\begingroup\catcode61\catcode48\catcode32=10\relax%
  \catcode13=5 % ^^M
  \endlinechar=13 %
  \catcode35=6 % #
  \catcode39=12 % '
  \catcode40=12 % (
  \catcode41=12 % )
  \catcode44=12 % ,
  \catcode45=12 % -
  \catcode46=12 % .
  \catcode47=12 % /
  \catcode58=12 % :
  \catcode64=11 % @
  \catcode91=12 % [
  \catcode93=12 % ]
  \catcode123=1 % {
  \catcode125=2 % }
  \expandafter\ifx\csname ProvidesPackage\endcsname\relax
    \def\x#1#2#3[#4]{\endgroup
      \immediate\write-1{Package: #3 #4}%
      \xdef#1{#4}%
    }%
  \else
    \def\x#1#2[#3]{\endgroup
      #2[{#3}]%
      \ifx#1\@undefined
        \xdef#1{#3}%
      \fi
      \ifx#1\relax
        \xdef#1{#3}%
      \fi
    }%
  \fi
\expandafter\x\csname ver@setouterhbox.sty\endcsname
\ProvidesPackage{setouterhbox}%
  [2016/05/16 v1.8 Set hbox in outer horizontal mode (HO)]%
%    \end{macrocode}
%
%    \begin{macrocode}
\begingroup\catcode61\catcode48\catcode32=10\relax%
  \catcode13=5 % ^^M
  \endlinechar=13 %
  \catcode123=1 % {
  \catcode125=2 % }
  \catcode64=11 % @
  \def\x{\endgroup
    \expandafter\edef\csname setouterhboxAtEnd\endcsname{%
      \endlinechar=\the\endlinechar\relax
      \catcode13=\the\catcode13\relax
      \catcode32=\the\catcode32\relax
      \catcode35=\the\catcode35\relax
      \catcode61=\the\catcode61\relax
      \catcode64=\the\catcode64\relax
      \catcode123=\the\catcode123\relax
      \catcode125=\the\catcode125\relax
    }%
  }%
\x\catcode61\catcode48\catcode32=10\relax%
\catcode13=5 % ^^M
\endlinechar=13 %
\catcode35=6 % #
\catcode64=11 % @
\catcode123=1 % {
\catcode125=2 % }
\def\TMP@EnsureCode#1#2{%
  \edef\setouterhboxAtEnd{%
    \setouterhboxAtEnd
    \catcode#1=\the\catcode#1\relax
  }%
  \catcode#1=#2\relax
}
\TMP@EnsureCode{40}{12}% (
\TMP@EnsureCode{41}{12}% )
\TMP@EnsureCode{44}{12}% ,
\TMP@EnsureCode{45}{12}% -
\TMP@EnsureCode{46}{12}% .
\TMP@EnsureCode{47}{12}% /
\TMP@EnsureCode{58}{12}% :
\TMP@EnsureCode{60}{12}% <
\TMP@EnsureCode{62}{12}% >
\TMP@EnsureCode{91}{12}% [
\TMP@EnsureCode{93}{12}% ]
\TMP@EnsureCode{96}{12}% `
\edef\setouterhboxAtEnd{\setouterhboxAtEnd\noexpand\endinput}
%    \end{macrocode}
%
% \subsection{Interface macros}
%
%    \begin{macro}{\setouterhboxBox}
% The method requires a global box assignment. To be on the
% safe side, a new box register is allocated for this
% global box assignment.
%    \begin{macrocode}
\newbox\setouterhboxBox
%    \end{macrocode}
%    \end{macro}
%
%    \begin{macro}{\setouterhboxFailure}
% Error message for both \plainTeX\ and \LaTeX
%    \begin{macrocode}
\begingroup\expandafter\expandafter\expandafter\endgroup
\expandafter\ifx\csname RequirePackage\endcsname\relax
  \input infwarerr.sty\relax
\else
  \RequirePackage{infwarerr}[2016/05/16]%
\fi
\edef\setouterhboxFailure#1#2{%
  \expandafter\noexpand\csname @PackageError\endcsname
      {setouterhbox}{#1}{#2}%
}
%    \end{macrocode}
%    \end{macro}
%
% \subsection{Main part}
%
% eTeX provides much better means for checking
% error conditions. Thus lines marked by "E" are executed
% if eTeX is available, otherwise the lines marked by "T" are
% used.
%    \begin{macrocode}
\begingroup\expandafter\expandafter\expandafter\endgroup
\expandafter\ifx\csname lastnodetype\endcsname\relax
  \catcode`T=9 % ignore
  \catcode`E=14 % comment
\else
  \catcode`T=14 % comment
  \catcode`E=9 % ignore
\fi
%    \end{macrocode}
%
%    \begin{macro}{\setouterhboxRemove}
% Remove all kern, glue, and penalty nodes;
% poor man's version, if \eTeX\ is not available
%    \begin{macrocode}
\def\setouterhboxRemove{%
E \ifnum\lastnodetype<11 %
E   \else
E   \ifnum\lastnodetype>13 %
E   \else
      \unskip\unkern\unpenalty
E     \expandafter\expandafter\expandafter\setouterhboxRemove
E   \fi
E \fi
}%
%    \end{macrocode}
%    \end{macro}
%
%    \begin{macro}{\setouterhbox}
% Passing the box contents by macro parameter would prevent
% catcode changes in the box contents like by \cs{verb}.
% Also \cs{bgroup} and \cs{egroup} does not work, because stuff
% has to be added at the begin and end of the box, thus
% the syntax
% |\setouterhbox{|\meta{box number}|}|\dots|\endsetouterhbox|
% is used. Also we automatically get an environment \texttt{setouterhbox}
% if \LaTeX\ is used.
%    \begin{macrocode}
\def\setouterhbox#1{%
  \begingroup
    \def\setouterhboxNum{#1}%
    \setbox0\vbox\bgroup
T     \kern.123pt\relax % marker
T     \kern0pt\relax % removed by \setouterhboxRemove
      \begingroup
        \everypar{}%
        \noindent
}
%    \end{macrocode}
%    \end{macro}
%    \begin{macro}{\endsetouterhbox}
% Most of the work is done in the end part, thus the heart of
% the method follows:
%    \begin{macrocode}
\def\endsetouterhbox{%
      \endgroup
%    \end{macrocode}
% Omit the first pass to get the penalties
% of the second pass.
%    \begin{macrocode}
      \pretolerance-1 %
%    \end{macrocode}
%  We don't want a third pass with \cs{emergencystretch}.
%    \begin{macrocode}
      \tolerance10000 %
      \hsize\maxdimen
%    \end{macrocode}
% Line is not underfull:
%    \begin{macrocode}
      \parfillskip 0pt plus 1filll\relax
      \leftskip0pt\relax
%    \end{macrocode}
% Suppress underful \cs{hbox} warnings,
% is explicit line breaks are used.
%    \begin{macrocode}
      \rightskip0pt plus 1fil\relax
      \everypar{}%
%    \end{macrocode}
% Ensure that there is a paragraph and
% prevents \cs{endgraph} from eating terminal glue:
%    \begin{macrocode}
      \kern0pt%
      \endgraf
      \setouterhboxRemove
E     \ifnum\lastnodetype=1 %
E       \global\setbox\setouterhboxBox\lastbox
E       \loop
E         \setouterhboxRemove
E       \ifnum\lastnodetype=1 %
E         \setbox0=\lastbox
E         \global\setbox\setouterhboxBox=\hbox{%
E           \unhbox0 %
%    \end{macrocode}
% Remove \cs{rightskip}, a penalty with -10000 is part of the previous line.
%    \begin{macrocode}
E           \unskip
E           \unhbox\setouterhboxBox
E         }%
E       \repeat
E     \else
E       \setouterhboxFailure{%
E         Something is wrong%
E       }{%
E         Could not find expected line.%
E         \MessageBreak
E         (\string\lastnodetype: \number\lastnodetype, expected: 1)%
E       }%
E     \fi
E     \setouterhboxRemove
T     \global\setbox\setouterhboxBox\lastbox
T     \loop
T       \setouterhboxRemove
T       \setbox0=\lastbox
T     \ifcase\ifvoid0 1\else0\fi
T       \global\setbox\setouterhboxBox=\hbox{%
T         \unhbox0 %
%    \end{macrocode}
% Remove \cs{rightskip}, a penalty with -10000 is part of the previous line.
%    \begin{macrocode}
T         \unskip
T         \unhbox\setouterhboxBox
T       }%
T     \repeat
T     \ifdim.123pt=\lastkern
T     \else
T       \setouterhboxFailure{%
T         Something is wrong%
T       }{%
T         Unexpected stuff was detected before the line.%
T       }%
T     \fi
T   \egroup
T   \ifcase \ifnum\wd0=0 \else 1\fi
T           \ifdim\ht0=.123pt \else 1\fi
T           \ifnum\dp0=0 \else 1\fi
T           0 %
E     \ifnum\lastnodetype=-1 %
%    \end{macrocode}
% There was just one line that we have caught.
%    \begin{macrocode}
      \else
        \setouterhboxFailure{%
            Something is wrong%
        }{%
            After fetching the line there is more unexpected stuff.%
E           \MessageBreak
E           (\string\lastnodetype: \number\lastnodetype, expected: -1)%
        }%
      \fi
E   \egroup
  \expandafter\endgroup
  \expandafter\setouterhboxFinish\expandafter{%
    \number\setouterhboxNum
  }%
}
%    \end{macrocode}
%    \end{macro}
%
% \subsection{Environment support}
%
% Check \cs{@currenvir} for the case that \cs{setouterhbox}
% was called as environment. Then the box assignment
% must be put after the \cs{endgroup} of |\end{|\dots|}|.
%    \begin{macrocode}
\def\setouterhboxCurr{setouterhbox}
\def\setouterhboxLast#1{%
  \setbox#1\hbox{%
    \unhbox\setouterhboxBox
    \unskip % remove \rightskip glue
    \unskip % remove \parfillskip glue
    \unpenalty % remove paragraph ending \penalty 10000
    \unkern % remove explicit kern inserted above
  }%
}
%    \end{macrocode}
%    \begin{macro}{\setouterhboxFinish}
% |#1| is an explicit number.
%    \begin{macrocode}
\def\setouterhboxFinish#1{%
  \begingroup\expandafter\expandafter\expandafter\endgroup
  \expandafter\ifx\csname @currenvir\endcsname\setouterhboxCurr
    \aftergroup\setouterhboxLast
    \aftergroup{%
    \setouterhboxAfter #1\NIL
    \aftergroup}%
  \else
    \setouterhboxLast{#1}%
  \fi
}
%    \end{macrocode}
%    \end{macro}
%    \begin{macro}{\setouterhboxAfter}
% |#1| is an explicit number.
%    \begin{macrocode}
\def\setouterhboxAfter#1#2\NIL{%
  \aftergroup#1%
  \ifx\\#2\\%
  \else
    \setouterhboxReturnAfterFi{%
      \setouterhboxAfter#2\NIL
    }%
  \fi
}
%    \end{macrocode}
%    \end{macro}
%    \begin{macro}{\setouterhboxReturnAfterFi}
% A utility macro to get tail recursion.
%    \begin{macrocode}
\long\def\setouterhboxReturnAfterFi#1\fi{\fi#1}
%    \end{macrocode}
%    \end{macro}
% Restore catcodes we have need to distinguish between
% the implementation with and without \eTeX.
%    \begin{macrocode}
\catcode69=11\relax % E
\catcode84=11\relax % T
%    \end{macrocode}
%
% \subsection{Option \xoption{hyperref}}
%    \begin{macrocode}
\begingroup
  \def\x{LaTeX2e}%
\expandafter\endgroup
\ifx\x\fmtname
\else
  \expandafter\setouterhboxAtEnd
\fi%
%    \end{macrocode}
%    \begin{macro}{\Hy@setouterhbox}
% \cs{Hy@setouterhbox} is the internal hook that \xpackage{hyperref}
% uses since 2006/02/12 v6.75a.
%    \begin{macrocode}
\DeclareOption{hyperref}{%
  \long\def\Hy@setouterhbox#1#2{%
    \setouterhbox{#1}#2\endsetouterhbox
  }%
}
%    \end{macrocode}
%    \end{macro}
%    \begin{macrocode}
\ProcessOptions\relax
%    \end{macrocode}
%
%    \begin{macrocode}
\setouterhboxAtEnd%
%</package>
%    \end{macrocode}
%
% \section{Test}
%
% \subsection{Catcode checks for loading}
%
%    \begin{macrocode}
%<*test1>
%    \end{macrocode}
%    \begin{macrocode}
\catcode`\{=1 %
\catcode`\}=2 %
\catcode`\#=6 %
\catcode`\@=11 %
\expandafter\ifx\csname count@\endcsname\relax
  \countdef\count@=255 %
\fi
\expandafter\ifx\csname @gobble\endcsname\relax
  \long\def\@gobble#1{}%
\fi
\expandafter\ifx\csname @firstofone\endcsname\relax
  \long\def\@firstofone#1{#1}%
\fi
\expandafter\ifx\csname loop\endcsname\relax
  \expandafter\@firstofone
\else
  \expandafter\@gobble
\fi
{%
  \def\loop#1\repeat{%
    \def\body{#1}%
    \iterate
  }%
  \def\iterate{%
    \body
      \let\next\iterate
    \else
      \let\next\relax
    \fi
    \next
  }%
  \let\repeat=\fi
}%
\def\RestoreCatcodes{}
\count@=0 %
\loop
  \edef\RestoreCatcodes{%
    \RestoreCatcodes
    \catcode\the\count@=\the\catcode\count@\relax
  }%
\ifnum\count@<255 %
  \advance\count@ 1 %
\repeat

\def\RangeCatcodeInvalid#1#2{%
  \count@=#1\relax
  \loop
    \catcode\count@=15 %
  \ifnum\count@<#2\relax
    \advance\count@ 1 %
  \repeat
}
\def\RangeCatcodeCheck#1#2#3{%
  \count@=#1\relax
  \loop
    \ifnum#3=\catcode\count@
    \else
      \errmessage{%
        Character \the\count@\space
        with wrong catcode \the\catcode\count@\space
        instead of \number#3%
      }%
    \fi
  \ifnum\count@<#2\relax
    \advance\count@ 1 %
  \repeat
}
\def\space{ }
\expandafter\ifx\csname LoadCommand\endcsname\relax
  \def\LoadCommand{\input setouterhbox.sty\relax}%
\fi
\def\Test{%
  \RangeCatcodeInvalid{0}{47}%
  \RangeCatcodeInvalid{58}{64}%
  \RangeCatcodeInvalid{91}{96}%
  \RangeCatcodeInvalid{123}{255}%
  \catcode`\@=12 %
  \catcode`\\=0 %
  \catcode`\%=14 %
  \LoadCommand
  \RangeCatcodeCheck{0}{36}{15}%
  \RangeCatcodeCheck{37}{37}{14}%
  \RangeCatcodeCheck{38}{47}{15}%
  \RangeCatcodeCheck{48}{57}{12}%
  \RangeCatcodeCheck{58}{63}{15}%
  \RangeCatcodeCheck{64}{64}{12}%
  \RangeCatcodeCheck{65}{90}{11}%
  \RangeCatcodeCheck{91}{91}{15}%
  \RangeCatcodeCheck{92}{92}{0}%
  \RangeCatcodeCheck{93}{96}{15}%
  \RangeCatcodeCheck{97}{122}{11}%
  \RangeCatcodeCheck{123}{255}{15}%
  \RestoreCatcodes
}
\Test
\csname @@end\endcsname
\end
%    \end{macrocode}
%    \begin{macrocode}
%</test1>
%    \end{macrocode}
%
% \subsection{Test with package \xpackage{url}}
%
%    \begin{macrocode}
%<*test2>
\nofiles
\documentclass[a5paper]{article}
\usepackage{url}[2005/06/27]
\usepackage{setouterhbox}

\newsavebox{\testbox}

\setlength{\parindent}{0pt}
\setlength{\parskip}{2em}

\begin{document}
\raggedright

\url{http://this.is.a.very.long.host.name/followed/%
by/a/very_long_long_long_path.html}%

\sbox\testbox{%
  \url{http://this.is.a.very.long.host.name/followed/%
  by/a/very_long_long_long_path.html}%
}%
\unhbox\testbox

\begin{setouterhbox}{\testbox}%
  \url{http://this.is.a.very.long.host.name/followed/%
  by/a/very_long_long_long_path.html}%
\end{setouterhbox}
\unhbox\testbox

\end{document}
%</test2>
%    \end{macrocode}
%
% \section{Installation}
%
% \subsection{Download}
%
% \paragraph{Package.} This package is available on
% CTAN\footnote{\CTANpkg{setouterhbox}}:
% \begin{description}
% \item[\CTAN{macros/latex/contrib/oberdiek/setouterhbox.dtx}] The source file.
% \item[\CTAN{macros/latex/contrib/oberdiek/setouterhbox.pdf}] Documentation.
% \end{description}
%
%
% \paragraph{Bundle.} All the packages of the bundle `oberdiek'
% are also available in a TDS compliant ZIP archive. There
% the packages are already unpacked and the documentation files
% are generated. The files and directories obey the TDS standard.
% \begin{description}
% \item[\CTANinstall{install/macros/latex/contrib/oberdiek.tds.zip}]
% \end{description}
% \emph{TDS} refers to the standard ``A Directory Structure
% for \TeX\ Files'' (\CTAN{tds/tds.pdf}). Directories
% with \xfile{texmf} in their name are usually organized this way.
%
% \subsection{Bundle installation}
%
% \paragraph{Unpacking.} Unpack the \xfile{oberdiek.tds.zip} in the
% TDS tree (also known as \xfile{texmf} tree) of your choice.
% Example (linux):
% \begin{quote}
%   |unzip oberdiek.tds.zip -d ~/texmf|
% \end{quote}
%
% \subsection{Package installation}
%
% \paragraph{Unpacking.} The \xfile{.dtx} file is a self-extracting
% \docstrip\ archive. The files are extracted by running the
% \xfile{.dtx} through \plainTeX:
% \begin{quote}
%   \verb|tex setouterhbox.dtx|
% \end{quote}
%
% \paragraph{TDS.} Now the different files must be moved into
% the different directories in your installation TDS tree
% (also known as \xfile{texmf} tree):
% \begin{quote}
% \def\t{^^A
% \begin{tabular}{@{}>{\ttfamily}l@{ $\rightarrow$ }>{\ttfamily}l@{}}
%   setouterhbox.sty & tex/generic/oberdiek/setouterhbox.sty\\
%   setouterhbox.pdf & doc/latex/oberdiek/setouterhbox.pdf\\
%   setouterhbox-example.tex & doc/latex/oberdiek/setouterhbox-example.tex\\
%   test/setouterhbox-test1.tex & doc/latex/oberdiek/test/setouterhbox-test1.tex\\
%   test/setouterhbox-test2.tex & doc/latex/oberdiek/test/setouterhbox-test2.tex\\
%   setouterhbox.dtx & source/latex/oberdiek/setouterhbox.dtx\\
% \end{tabular}^^A
% }^^A
% \sbox0{\t}^^A
% \ifdim\wd0>\linewidth
%   \begingroup
%     \advance\linewidth by\leftmargin
%     \advance\linewidth by\rightmargin
%   \edef\x{\endgroup
%     \def\noexpand\lw{\the\linewidth}^^A
%   }\x
%   \def\lwbox{^^A
%     \leavevmode
%     \hbox to \linewidth{^^A
%       \kern-\leftmargin\relax
%       \hss
%       \usebox0
%       \hss
%       \kern-\rightmargin\relax
%     }^^A
%   }^^A
%   \ifdim\wd0>\lw
%     \sbox0{\small\t}^^A
%     \ifdim\wd0>\linewidth
%       \ifdim\wd0>\lw
%         \sbox0{\footnotesize\t}^^A
%         \ifdim\wd0>\linewidth
%           \ifdim\wd0>\lw
%             \sbox0{\scriptsize\t}^^A
%             \ifdim\wd0>\linewidth
%               \ifdim\wd0>\lw
%                 \sbox0{\tiny\t}^^A
%                 \ifdim\wd0>\linewidth
%                   \lwbox
%                 \else
%                   \usebox0
%                 \fi
%               \else
%                 \lwbox
%               \fi
%             \else
%               \usebox0
%             \fi
%           \else
%             \lwbox
%           \fi
%         \else
%           \usebox0
%         \fi
%       \else
%         \lwbox
%       \fi
%     \else
%       \usebox0
%     \fi
%   \else
%     \lwbox
%   \fi
% \else
%   \usebox0
% \fi
% \end{quote}
% If you have a \xfile{docstrip.cfg} that configures and enables \docstrip's
% TDS installing feature, then some files can already be in the right
% place, see the documentation of \docstrip.
%
% \subsection{Refresh file name databases}
%
% If your \TeX~distribution
% (\TeX\,Live, \mikTeX, \dots) relies on file name databases, you must refresh
% these. For example, \TeX\,Live\ users run \verb|texhash| or
% \verb|mktexlsr|.
%
% \subsection{Some details for the interested}
%
% \paragraph{Unpacking with \LaTeX.}
% The \xfile{.dtx} chooses its action depending on the format:
% \begin{description}
% \item[\plainTeX:] Run \docstrip\ and extract the files.
% \item[\LaTeX:] Generate the documentation.
% \end{description}
% If you insist on using \LaTeX\ for \docstrip\ (really,
% \docstrip\ does not need \LaTeX), then inform the autodetect routine
% about your intention:
% \begin{quote}
%   \verb|latex \let\install=y\input{setouterhbox.dtx}|
% \end{quote}
% Do not forget to quote the argument according to the demands
% of your shell.
%
% \paragraph{Generating the documentation.}
% You can use both the \xfile{.dtx} or the \xfile{.drv} to generate
% the documentation. The process can be configured by the
% configuration file \xfile{ltxdoc.cfg}. For instance, put this
% line into this file, if you want to have A4 as paper format:
% \begin{quote}
%   \verb|\PassOptionsToClass{a4paper}{article}|
% \end{quote}
% An example follows how to generate the
% documentation with pdf\LaTeX:
% \begin{quote}
%\begin{verbatim}
%pdflatex setouterhbox.dtx
%makeindex -s gind.ist setouterhbox.idx
%pdflatex setouterhbox.dtx
%makeindex -s gind.ist setouterhbox.idx
%pdflatex setouterhbox.dtx
%\end{verbatim}
% \end{quote}
%
% \begin{thebibliography}{9}
%
% \bibitem{newsstart}
%   Damian Menscher, \Newsgroup{comp.text.tex},
%   \textit{overlong lines in List of Figures},
%   \nolinkurl{<dh058t$qbd$1@news.ks.uiuc.edu>},
%   23rd September 2005.
%   \url{https://groups.google.com/group/comp.text.tex/msg/79648d4cf1f8bc13}
%
% \bibitem{kastrup}
%   David Kastrup, \Newsgroup{comp.text.tex},
%   \textit{Re: ANN: outerhbox.sty -- collect horizontal material,
%   for unboxing into a paragraph},
%   \nolinkurl{<85y855lrx3.fsf@lola.goethe.zz>},
%   7th October 2005.
%   \url{https://groups.google.com/group/comp.text.tex/msg/7cf0a345ef932e52}
%
% \bibitem{downes}
%   Michael Downes, \textit{Line breaking in \cs{unhbox}ed Text},
%   TUGboat 11 (1990), pp. 605--612.
%
% \bibitem{hyperref}
%   Sebastian Rahtz, Heiko Oberdiek:
%   \textit{The \xpackage{hyperref} package};
%   2006/08/16 v6.75c;
%   \CTANpkg{hyperref}.
%
% \end{thebibliography}
%
% \begin{History}
%   \begin{Version}{2005/10/05 v1.0}
%   \item
%     First version.
%   \end{Version}
%   \begin{Version}{2005/10/07 v1.1}
%   \item
%     Option \xoption{hyperref} added.
%   \end{Version}
%   \begin{Version}{2005/10/18 v1.2}
%   \item
%     Support for explicit line breaks added.
%   \end{Version}
%   \begin{Version}{2006/02/12 v1.3}
%   \item
%     DTX format.
%   \item
%     Documentation extended.
%   \end{Version}
%   \begin{Version}{2006/08/26 v1.4}
%   \item
%     Date of hyperref updated.
%   \end{Version}
%   \begin{Version}{2007/04/26 v1.5}
%   \item
%     Use of package \xpackage{infwarerr}.
%   \end{Version}
%   \begin{Version}{2007/05/17 v1.6}
%   \item
%     Standard header part for generic files.
%   \end{Version}
%   \begin{Version}{2007/09/09 v1.7}
%   \item
%     Catcode section added.
%   \end{Version}
%   \begin{Version}{2016/05/16 v1.8}
%   \item
%     Documentation updates.
%   \end{Version}
% \end{History}
%
% \PrintIndex
%
% \Finale
\endinput
|
% \end{quote}
% Do not forget to quote the argument according to the demands
% of your shell.
%
% \paragraph{Generating the documentation.}
% You can use both the \xfile{.dtx} or the \xfile{.drv} to generate
% the documentation. The process can be configured by the
% configuration file \xfile{ltxdoc.cfg}. For instance, put this
% line into this file, if you want to have A4 as paper format:
% \begin{quote}
%   \verb|\PassOptionsToClass{a4paper}{article}|
% \end{quote}
% An example follows how to generate the
% documentation with pdf\LaTeX:
% \begin{quote}
%\begin{verbatim}
%pdflatex setouterhbox.dtx
%makeindex -s gind.ist setouterhbox.idx
%pdflatex setouterhbox.dtx
%makeindex -s gind.ist setouterhbox.idx
%pdflatex setouterhbox.dtx
%\end{verbatim}
% \end{quote}
%
% \begin{thebibliography}{9}
%
% \bibitem{newsstart}
%   Damian Menscher, \Newsgroup{comp.text.tex},
%   \textit{overlong lines in List of Figures},
%   \nolinkurl{<dh058t$qbd$1@news.ks.uiuc.edu>},
%   23rd September 2005.
%   \url{https://groups.google.com/group/comp.text.tex/msg/79648d4cf1f8bc13}
%
% \bibitem{kastrup}
%   David Kastrup, \Newsgroup{comp.text.tex},
%   \textit{Re: ANN: outerhbox.sty -- collect horizontal material,
%   for unboxing into a paragraph},
%   \nolinkurl{<85y855lrx3.fsf@lola.goethe.zz>},
%   7th October 2005.
%   \url{https://groups.google.com/group/comp.text.tex/msg/7cf0a345ef932e52}
%
% \bibitem{downes}
%   Michael Downes, \textit{Line breaking in \cs{unhbox}ed Text},
%   TUGboat 11 (1990), pp. 605--612.
%
% \bibitem{hyperref}
%   Sebastian Rahtz, Heiko Oberdiek:
%   \textit{The \xpackage{hyperref} package};
%   2006/08/16 v6.75c;
%   \CTANpkg{hyperref}.
%
% \end{thebibliography}
%
% \begin{History}
%   \begin{Version}{2005/10/05 v1.0}
%   \item
%     First version.
%   \end{Version}
%   \begin{Version}{2005/10/07 v1.1}
%   \item
%     Option \xoption{hyperref} added.
%   \end{Version}
%   \begin{Version}{2005/10/18 v1.2}
%   \item
%     Support for explicit line breaks added.
%   \end{Version}
%   \begin{Version}{2006/02/12 v1.3}
%   \item
%     DTX format.
%   \item
%     Documentation extended.
%   \end{Version}
%   \begin{Version}{2006/08/26 v1.4}
%   \item
%     Date of hyperref updated.
%   \end{Version}
%   \begin{Version}{2007/04/26 v1.5}
%   \item
%     Use of package \xpackage{infwarerr}.
%   \end{Version}
%   \begin{Version}{2007/05/17 v1.6}
%   \item
%     Standard header part for generic files.
%   \end{Version}
%   \begin{Version}{2007/09/09 v1.7}
%   \item
%     Catcode section added.
%   \end{Version}
%   \begin{Version}{2016/05/16 v1.8}
%   \item
%     Documentation updates.
%   \end{Version}
% \end{History}
%
% \PrintIndex
%
% \Finale
\endinput

%        (quote the arguments according to the demands of your shell)
%
% Documentation:
%    (a) If setouterhbox.drv is present:
%           latex setouterhbox.drv
%    (b) Without setouterhbox.drv:
%           latex setouterhbox.dtx; ...
%    The class ltxdoc loads the configuration file ltxdoc.cfg
%    if available. Here you can specify further options, e.g.
%    use A4 as paper format:
%       \PassOptionsToClass{a4paper}{article}
%
%    Programm calls to get the documentation (example):
%       pdflatex setouterhbox.dtx
%       makeindex -s gind.ist setouterhbox.idx
%       pdflatex setouterhbox.dtx
%       makeindex -s gind.ist setouterhbox.idx
%       pdflatex setouterhbox.dtx
%
% Installation:
%    TDS:tex/generic/oberdiek/setouterhbox.sty
%    TDS:doc/latex/oberdiek/setouterhbox.pdf
%    TDS:doc/latex/oberdiek/setouterhbox-example.tex
%    TDS:doc/latex/oberdiek/test/setouterhbox-test1.tex
%    TDS:doc/latex/oberdiek/test/setouterhbox-test2.tex
%    TDS:source/latex/oberdiek/setouterhbox.dtx
%
%<*ignore>
\begingroup
  \catcode123=1 %
  \catcode125=2 %
  \def\x{LaTeX2e}%
\expandafter\endgroup
\ifcase 0\ifx\install y1\fi\expandafter
         \ifx\csname processbatchFile\endcsname\relax\else1\fi
         \ifx\fmtname\x\else 1\fi\relax
\else\csname fi\endcsname
%</ignore>
%<*install>
\input docstrip.tex
\Msg{************************************************************************}
\Msg{* Installation}
\Msg{* Package: setouterhbox 2016/05/16 v1.8 Set hbox in outer horizontal mode (HO)}
\Msg{************************************************************************}

\keepsilent
\askforoverwritefalse

\let\MetaPrefix\relax
\preamble

This is a generated file.

Project: setouterhbox
Version: 2016/05/16 v1.8

Copyright (C)
   2005-2007 Heiko Oberdiek
   2016-2019 Oberdiek Package Support Group

This work may be distributed and/or modified under the
conditions of the LaTeX Project Public License, either
version 1.3c of this license or (at your option) any later
version. This version of this license is in
   https://www.latex-project.org/lppl/lppl-1-3c.txt
and the latest version of this license is in
   https://www.latex-project.org/lppl.txt
and version 1.3 or later is part of all distributions of
LaTeX version 2005/12/01 or later.

This work has the LPPL maintenance status "maintained".

The Current Maintainers of this work are
Heiko Oberdiek and the Oberdiek Package Support Group
https://github.com/ho-tex/oberdiek/issues


The Base Interpreter refers to any `TeX-Format',
because some files are installed in TDS:tex/generic//.

This work consists of the main source file setouterhbox.dtx
and the derived files
   setouterhbox.sty, setouterhbox.pdf, setouterhbox.ins, setouterhbox.drv,
   setouterhbox-example.tex, setouterhbox-test1.tex,
   setouterhbox-test2.tex.

\endpreamble
\let\MetaPrefix\DoubleperCent

\generate{%
  \file{setouterhbox.ins}{\from{setouterhbox.dtx}{install}}%
  \file{setouterhbox.drv}{\from{setouterhbox.dtx}{driver}}%
  \usedir{tex/generic/oberdiek}%
  \file{setouterhbox.sty}{\from{setouterhbox.dtx}{package}}%
  \usedir{doc/latex/oberdiek}%
  \file{setouterhbox-example.tex}{\from{setouterhbox.dtx}{example}}%
%  \usedir{doc/latex/oberdiek/test}%
%  \file{setouterhbox-test1.tex}{\from{setouterhbox.dtx}{test1}}%
%  \file{setouterhbox-test2.tex}{\from{setouterhbox.dtx}{test2}}%
  \nopreamble
  \nopostamble
%  \usedir{source/latex/oberdiek/catalogue}%
%  \file{setouterhbox.xml}{\from{setouterhbox.dtx}{catalogue}}%
}

\catcode32=13\relax% active space
\let =\space%
\Msg{************************************************************************}
\Msg{*}
\Msg{* To finish the installation you have to move the following}
\Msg{* file into a directory searched by TeX:}
\Msg{*}
\Msg{*     setouterhbox.sty}
\Msg{*}
\Msg{* To produce the documentation run the file `setouterhbox.drv'}
\Msg{* through LaTeX.}
\Msg{*}
\Msg{* Happy TeXing!}
\Msg{*}
\Msg{************************************************************************}

\endbatchfile
%</install>
%<*ignore>
\fi
%</ignore>
%<*driver>
\NeedsTeXFormat{LaTeX2e}
\ProvidesFile{setouterhbox.drv}%
  [2016/05/16 v1.8 Set hbox in outer horizontal mode (HO)]%
\documentclass{ltxdoc}
\usepackage{holtxdoc}[2011/11/22]
\begin{document}
  \DocInput{setouterhbox.dtx}%
\end{document}
%</driver>
% \fi
%
%
% \CharacterTable
%  {Upper-case    \A\B\C\D\E\F\G\H\I\J\K\L\M\N\O\P\Q\R\S\T\U\V\W\X\Y\Z
%   Lower-case    \a\b\c\d\e\f\g\h\i\j\k\l\m\n\o\p\q\r\s\t\u\v\w\x\y\z
%   Digits        \0\1\2\3\4\5\6\7\8\9
%   Exclamation   \!     Double quote  \"     Hash (number) \#
%   Dollar        \$     Percent       \%     Ampersand     \&
%   Acute accent  \'     Left paren    \(     Right paren   \)
%   Asterisk      \*     Plus          \+     Comma         \,
%   Minus         \-     Point         \.     Solidus       \/
%   Colon         \:     Semicolon     \;     Less than     \<
%   Equals        \=     Greater than  \>     Question mark \?
%   Commercial at \@     Left bracket  \[     Backslash     \\
%   Right bracket \]     Circumflex    \^     Underscore    \_
%   Grave accent  \`     Left brace    \{     Vertical bar  \|
%   Right brace   \}     Tilde         \~}
%
% \GetFileInfo{setouterhbox.drv}
%
% \title{The \xpackage{setouterhbox} package}
% \date{2016/05/16 v1.8}
% \author{Heiko Oberdiek\thanks
% {Please report any issues at \url{https://github.com/ho-tex/oberdiek/issues}}}
%
% \maketitle
%
% \begin{abstract}
% If math stuff is set in an \cs{hbox}, then TeX
% performs some optimization and omits the implicite
% penalties \cs{binoppenalty} and \cs{relpenalty}.
% This packages tries to put stuff into an \cs{hbox}
% without getting lost of those penalties.
% \end{abstract}
%
% \tableofcontents
%
% \section{Documentation}
%
% \subsection{Introduction}
%
% There is a situation in \xpackage{hyperref}'s driver for dvips
% where the user wants to have links that can be broken across
% lines. However dvips doesn't support the feature. With option
% \xoption{breaklinks} \xpackage{hyperref} sets the links as
% usual, put them in a box and write the link data with
% box dimensions into the appropriate \cs{special}s.
% Then, however, it does not set the complete unbreakable
% box, but it unwrappes the material inside to allow line
% breaks. Of course line breaking and glue setting will falsify
% the link dimensions, but line breaking was more important
% for the user.
%
% \subsection{Acknowledgement}
%
% Jonathan Fine, Donald Arsenau and me discussed the problem
% in the newsgroup \xnewsgroup{comp.text.tex} where Damian
% Menscher has started the thread, see \cite{newsstart}.
%
% The discussion was productive and generated many ideas
% and code examples. In order to have a more permanent
% result I wrote this package and tried to implement
% most of the ideas, a kind of summary of the discussion.
% Thus I want and have to thank Jonathan Fine and Donald Arsenau
% very much.
%
% Two weeks later David Kastrup (posting in
% \xnewsgroup{comp.text.tex}, \cite{kastrup})
% remembered an old article of Michael Downes (\cite{downes})
% in TUGboat, where Michael Downes already presented the
% method we discuss here. Nowadays we have \eTeX\ that extends
% the tool set of a \TeX\ macro programmer. Especially useful
% \eTeX\ was in this package for detecting and dealing with
% errorneous situations.
%
% However also nowadays a perfect solution for the problem
% is still missing at macro level. Probably someone has
% to go deep in the internals of the \TeX\ compiler to
% implement a switch that let penalties stay where otherwise
% \TeX\ would remove them for optimization reasons.
%
% \subsection{Usage}
%
% \paragraph{Package loading.}
% \LaTeX: as usually:
% \begin{quote}
%   |\usepackage{setouterhbox}|
% \end{quote}
% The package can also be included directly, thus \plainTeX\ users
% write:
% \begin{quote}
%   |\input setouterhbox.sty|
% \end{quote}
%
% \paragraph{Register allocation.}
% The material will be put into a box, thus we need to know these
% box number. If you need to allocate a new box register:
% \begin{description}
%  \item[\LaTeX:] |\newsavebox{\|\meta{name}|}|
%  \item[\plainTeX:] |\newbox\|\meta{name}
% \end{description}
% Then |\|\meta{name} is a command that held the box number.
%
% \paragraph{Box wrapping.}
% \LaTeX\ users put the material in the box with an environment
% similar to \texttt{lrbox}. The environment \texttt{setouterhbox}
% uses the same syntax and offers the same features, such
% as verbatim stuff inside:
% \begin{quote}
%  |\begin{setouterhbox}{|\meta{box number}|}|\dots
%  |\end{setouterhbox}|
% \end{quote}
% Users with \plainTeX\ do not have environments, they use instead:
% \begin{quote}
%   |\setouterhbox{|\meta{box number}|}|\dots|\endsetouterhbox|
% \end{quote}
% In both cases the material is put into an \cs{hbox} and assigned
% to the given box, denoted by \meta{box number}. Note the
% assignment is local, the same way \texttt{lrbox} behaves.
%
% \paragraph{Unwrapping.}
% The box material is ready for unwrapping:
% \begin{quote}
%   |\unhbox|\meta{box number}
% \end{quote}
%
% \subsection{Option \xoption{hyperref}}
%
% Package url uses math mode for typesetting urls.
% Break points are inserted by \cs{binoppenalty} and
% \cs{relpenalty}. Unhappily these break points are
% removed, if \xpackage{hyperref}
% is used with option {breaklinks}
% and drivers that depend on \xoption{pdfmark}:
% \xoption{dvips}, \xoption{vtexpdfmark}, \xoption{textures},
% and \xoption{dvipsone}.
% Thus the option \xoption{hyperref} enables the method
% of this package to avoid the removal of \cs{relpenalty}
% and \cs{binoppenalty}. Thus you get more break points.
% However, the link areas are still wrong for these
% drivers, because they are not supporting broken
% links.
%
% Note, you need version 2006/08/16 v6.75c of package \xpackage{hyperref},
% because starting with this version the necessary hook is provided
% that package \xpackage{setouterhbox} uses.
% \begin{quote}
%   |\usepackage[|\dots|]{hyperref}[2006/08/16]|\\
%   |\usepackage[hyperref]{setouterhbox}|
% \end{quote}
% Package order does not matter.
%
% \subsection{Example}
%
%    \begin{macrocode}
%<*example>
\documentclass[a5paper]{article}
\usepackage{url}[2005/06/27]
\usepackage{setouterhbox}

\newsavebox{\testbox}

\setlength{\parindent}{0pt}
\setlength{\parskip}{2em}

\begin{document}
\raggedright

\url{http://this.is.a.very.long.host.name/followed/%
by/a/very_long_long_long_path.html}%

\sbox\testbox{%
  \url{http://this.is.a.very.long.host.name/followed/%
  by/a/very_long_long_long_path.html}%
}%
\unhbox\testbox

\begin{setouterhbox}{\testbox}%
  \url{http://this.is.a.very.long.host.name/followed/%
  by/a/very_long_long_long_path.html}%
\end{setouterhbox}
\unhbox\testbox

\end{document}
%</example>
%    \end{macrocode}
%
% \StopEventually{
% }
%
% \section{Implementation}
%
% Internal macros are prefixed by \cs{setouterhbox}, |@| is
% not used inside names, thus we do not need to care of its
% catcode if we are not using it as \LaTeX\ package.
%
% \subsection{Package start stuff}
%
%    \begin{macrocode}
%<*package>
%    \end{macrocode}
%
% Prevent reloading more than one, necessary for \plainTeX:
%    Reload check, especially if the package is not used with \LaTeX.
%    \begin{macrocode}
\begingroup\catcode61\catcode48\catcode32=10\relax%
  \catcode13=5 % ^^M
  \endlinechar=13 %
  \catcode35=6 % #
  \catcode39=12 % '
  \catcode44=12 % ,
  \catcode45=12 % -
  \catcode46=12 % .
  \catcode58=12 % :
  \catcode64=11 % @
  \catcode123=1 % {
  \catcode125=2 % }
  \expandafter\let\expandafter\x\csname ver@setouterhbox.sty\endcsname
  \ifx\x\relax % plain-TeX, first loading
  \else
    \def\empty{}%
    \ifx\x\empty % LaTeX, first loading,
      % variable is initialized, but \ProvidesPackage not yet seen
    \else
      \expandafter\ifx\csname PackageInfo\endcsname\relax
        \def\x#1#2{%
          \immediate\write-1{Package #1 Info: #2.}%
        }%
      \else
        \def\x#1#2{\PackageInfo{#1}{#2, stopped}}%
      \fi
      \x{setouterhbox}{The package is already loaded}%
      \aftergroup\endinput
    \fi
  \fi
\endgroup%
%    \end{macrocode}
%    Package identification:
%    \begin{macrocode}
\begingroup\catcode61\catcode48\catcode32=10\relax%
  \catcode13=5 % ^^M
  \endlinechar=13 %
  \catcode35=6 % #
  \catcode39=12 % '
  \catcode40=12 % (
  \catcode41=12 % )
  \catcode44=12 % ,
  \catcode45=12 % -
  \catcode46=12 % .
  \catcode47=12 % /
  \catcode58=12 % :
  \catcode64=11 % @
  \catcode91=12 % [
  \catcode93=12 % ]
  \catcode123=1 % {
  \catcode125=2 % }
  \expandafter\ifx\csname ProvidesPackage\endcsname\relax
    \def\x#1#2#3[#4]{\endgroup
      \immediate\write-1{Package: #3 #4}%
      \xdef#1{#4}%
    }%
  \else
    \def\x#1#2[#3]{\endgroup
      #2[{#3}]%
      \ifx#1\@undefined
        \xdef#1{#3}%
      \fi
      \ifx#1\relax
        \xdef#1{#3}%
      \fi
    }%
  \fi
\expandafter\x\csname ver@setouterhbox.sty\endcsname
\ProvidesPackage{setouterhbox}%
  [2016/05/16 v1.8 Set hbox in outer horizontal mode (HO)]%
%    \end{macrocode}
%
%    \begin{macrocode}
\begingroup\catcode61\catcode48\catcode32=10\relax%
  \catcode13=5 % ^^M
  \endlinechar=13 %
  \catcode123=1 % {
  \catcode125=2 % }
  \catcode64=11 % @
  \def\x{\endgroup
    \expandafter\edef\csname setouterhboxAtEnd\endcsname{%
      \endlinechar=\the\endlinechar\relax
      \catcode13=\the\catcode13\relax
      \catcode32=\the\catcode32\relax
      \catcode35=\the\catcode35\relax
      \catcode61=\the\catcode61\relax
      \catcode64=\the\catcode64\relax
      \catcode123=\the\catcode123\relax
      \catcode125=\the\catcode125\relax
    }%
  }%
\x\catcode61\catcode48\catcode32=10\relax%
\catcode13=5 % ^^M
\endlinechar=13 %
\catcode35=6 % #
\catcode64=11 % @
\catcode123=1 % {
\catcode125=2 % }
\def\TMP@EnsureCode#1#2{%
  \edef\setouterhboxAtEnd{%
    \setouterhboxAtEnd
    \catcode#1=\the\catcode#1\relax
  }%
  \catcode#1=#2\relax
}
\TMP@EnsureCode{40}{12}% (
\TMP@EnsureCode{41}{12}% )
\TMP@EnsureCode{44}{12}% ,
\TMP@EnsureCode{45}{12}% -
\TMP@EnsureCode{46}{12}% .
\TMP@EnsureCode{47}{12}% /
\TMP@EnsureCode{58}{12}% :
\TMP@EnsureCode{60}{12}% <
\TMP@EnsureCode{62}{12}% >
\TMP@EnsureCode{91}{12}% [
\TMP@EnsureCode{93}{12}% ]
\TMP@EnsureCode{96}{12}% `
\edef\setouterhboxAtEnd{\setouterhboxAtEnd\noexpand\endinput}
%    \end{macrocode}
%
% \subsection{Interface macros}
%
%    \begin{macro}{\setouterhboxBox}
% The method requires a global box assignment. To be on the
% safe side, a new box register is allocated for this
% global box assignment.
%    \begin{macrocode}
\newbox\setouterhboxBox
%    \end{macrocode}
%    \end{macro}
%
%    \begin{macro}{\setouterhboxFailure}
% Error message for both \plainTeX\ and \LaTeX
%    \begin{macrocode}
\begingroup\expandafter\expandafter\expandafter\endgroup
\expandafter\ifx\csname RequirePackage\endcsname\relax
  \input infwarerr.sty\relax
\else
  \RequirePackage{infwarerr}[2016/05/16]%
\fi
\edef\setouterhboxFailure#1#2{%
  \expandafter\noexpand\csname @PackageError\endcsname
      {setouterhbox}{#1}{#2}%
}
%    \end{macrocode}
%    \end{macro}
%
% \subsection{Main part}
%
% eTeX provides much better means for checking
% error conditions. Thus lines marked by "E" are executed
% if eTeX is available, otherwise the lines marked by "T" are
% used.
%    \begin{macrocode}
\begingroup\expandafter\expandafter\expandafter\endgroup
\expandafter\ifx\csname lastnodetype\endcsname\relax
  \catcode`T=9 % ignore
  \catcode`E=14 % comment
\else
  \catcode`T=14 % comment
  \catcode`E=9 % ignore
\fi
%    \end{macrocode}
%
%    \begin{macro}{\setouterhboxRemove}
% Remove all kern, glue, and penalty nodes;
% poor man's version, if \eTeX\ is not available
%    \begin{macrocode}
\def\setouterhboxRemove{%
E \ifnum\lastnodetype<11 %
E   \else
E   \ifnum\lastnodetype>13 %
E   \else
      \unskip\unkern\unpenalty
E     \expandafter\expandafter\expandafter\setouterhboxRemove
E   \fi
E \fi
}%
%    \end{macrocode}
%    \end{macro}
%
%    \begin{macro}{\setouterhbox}
% Passing the box contents by macro parameter would prevent
% catcode changes in the box contents like by \cs{verb}.
% Also \cs{bgroup} and \cs{egroup} does not work, because stuff
% has to be added at the begin and end of the box, thus
% the syntax
% |\setouterhbox{|\meta{box number}|}|\dots|\endsetouterhbox|
% is used. Also we automatically get an environment \texttt{setouterhbox}
% if \LaTeX\ is used.
%    \begin{macrocode}
\def\setouterhbox#1{%
  \begingroup
    \def\setouterhboxNum{#1}%
    \setbox0\vbox\bgroup
T     \kern.123pt\relax % marker
T     \kern0pt\relax % removed by \setouterhboxRemove
      \begingroup
        \everypar{}%
        \noindent
}
%    \end{macrocode}
%    \end{macro}
%    \begin{macro}{\endsetouterhbox}
% Most of the work is done in the end part, thus the heart of
% the method follows:
%    \begin{macrocode}
\def\endsetouterhbox{%
      \endgroup
%    \end{macrocode}
% Omit the first pass to get the penalties
% of the second pass.
%    \begin{macrocode}
      \pretolerance-1 %
%    \end{macrocode}
%  We don't want a third pass with \cs{emergencystretch}.
%    \begin{macrocode}
      \tolerance10000 %
      \hsize\maxdimen
%    \end{macrocode}
% Line is not underfull:
%    \begin{macrocode}
      \parfillskip 0pt plus 1filll\relax
      \leftskip0pt\relax
%    \end{macrocode}
% Suppress underful \cs{hbox} warnings,
% is explicit line breaks are used.
%    \begin{macrocode}
      \rightskip0pt plus 1fil\relax
      \everypar{}%
%    \end{macrocode}
% Ensure that there is a paragraph and
% prevents \cs{endgraph} from eating terminal glue:
%    \begin{macrocode}
      \kern0pt%
      \endgraf
      \setouterhboxRemove
E     \ifnum\lastnodetype=1 %
E       \global\setbox\setouterhboxBox\lastbox
E       \loop
E         \setouterhboxRemove
E       \ifnum\lastnodetype=1 %
E         \setbox0=\lastbox
E         \global\setbox\setouterhboxBox=\hbox{%
E           \unhbox0 %
%    \end{macrocode}
% Remove \cs{rightskip}, a penalty with -10000 is part of the previous line.
%    \begin{macrocode}
E           \unskip
E           \unhbox\setouterhboxBox
E         }%
E       \repeat
E     \else
E       \setouterhboxFailure{%
E         Something is wrong%
E       }{%
E         Could not find expected line.%
E         \MessageBreak
E         (\string\lastnodetype: \number\lastnodetype, expected: 1)%
E       }%
E     \fi
E     \setouterhboxRemove
T     \global\setbox\setouterhboxBox\lastbox
T     \loop
T       \setouterhboxRemove
T       \setbox0=\lastbox
T     \ifcase\ifvoid0 1\else0\fi
T       \global\setbox\setouterhboxBox=\hbox{%
T         \unhbox0 %
%    \end{macrocode}
% Remove \cs{rightskip}, a penalty with -10000 is part of the previous line.
%    \begin{macrocode}
T         \unskip
T         \unhbox\setouterhboxBox
T       }%
T     \repeat
T     \ifdim.123pt=\lastkern
T     \else
T       \setouterhboxFailure{%
T         Something is wrong%
T       }{%
T         Unexpected stuff was detected before the line.%
T       }%
T     \fi
T   \egroup
T   \ifcase \ifnum\wd0=0 \else 1\fi
T           \ifdim\ht0=.123pt \else 1\fi
T           \ifnum\dp0=0 \else 1\fi
T           0 %
E     \ifnum\lastnodetype=-1 %
%    \end{macrocode}
% There was just one line that we have caught.
%    \begin{macrocode}
      \else
        \setouterhboxFailure{%
            Something is wrong%
        }{%
            After fetching the line there is more unexpected stuff.%
E           \MessageBreak
E           (\string\lastnodetype: \number\lastnodetype, expected: -1)%
        }%
      \fi
E   \egroup
  \expandafter\endgroup
  \expandafter\setouterhboxFinish\expandafter{%
    \number\setouterhboxNum
  }%
}
%    \end{macrocode}
%    \end{macro}
%
% \subsection{Environment support}
%
% Check \cs{@currenvir} for the case that \cs{setouterhbox}
% was called as environment. Then the box assignment
% must be put after the \cs{endgroup} of |\end{|\dots|}|.
%    \begin{macrocode}
\def\setouterhboxCurr{setouterhbox}
\def\setouterhboxLast#1{%
  \setbox#1\hbox{%
    \unhbox\setouterhboxBox
    \unskip % remove \rightskip glue
    \unskip % remove \parfillskip glue
    \unpenalty % remove paragraph ending \penalty 10000
    \unkern % remove explicit kern inserted above
  }%
}
%    \end{macrocode}
%    \begin{macro}{\setouterhboxFinish}
% |#1| is an explicit number.
%    \begin{macrocode}
\def\setouterhboxFinish#1{%
  \begingroup\expandafter\expandafter\expandafter\endgroup
  \expandafter\ifx\csname @currenvir\endcsname\setouterhboxCurr
    \aftergroup\setouterhboxLast
    \aftergroup{%
    \setouterhboxAfter #1\NIL
    \aftergroup}%
  \else
    \setouterhboxLast{#1}%
  \fi
}
%    \end{macrocode}
%    \end{macro}
%    \begin{macro}{\setouterhboxAfter}
% |#1| is an explicit number.
%    \begin{macrocode}
\def\setouterhboxAfter#1#2\NIL{%
  \aftergroup#1%
  \ifx\\#2\\%
  \else
    \setouterhboxReturnAfterFi{%
      \setouterhboxAfter#2\NIL
    }%
  \fi
}
%    \end{macrocode}
%    \end{macro}
%    \begin{macro}{\setouterhboxReturnAfterFi}
% A utility macro to get tail recursion.
%    \begin{macrocode}
\long\def\setouterhboxReturnAfterFi#1\fi{\fi#1}
%    \end{macrocode}
%    \end{macro}
% Restore catcodes we have need to distinguish between
% the implementation with and without \eTeX.
%    \begin{macrocode}
\catcode69=11\relax % E
\catcode84=11\relax % T
%    \end{macrocode}
%
% \subsection{Option \xoption{hyperref}}
%    \begin{macrocode}
\begingroup
  \def\x{LaTeX2e}%
\expandafter\endgroup
\ifx\x\fmtname
\else
  \expandafter\setouterhboxAtEnd
\fi%
%    \end{macrocode}
%    \begin{macro}{\Hy@setouterhbox}
% \cs{Hy@setouterhbox} is the internal hook that \xpackage{hyperref}
% uses since 2006/02/12 v6.75a.
%    \begin{macrocode}
\DeclareOption{hyperref}{%
  \long\def\Hy@setouterhbox#1#2{%
    \setouterhbox{#1}#2\endsetouterhbox
  }%
}
%    \end{macrocode}
%    \end{macro}
%    \begin{macrocode}
\ProcessOptions\relax
%    \end{macrocode}
%
%    \begin{macrocode}
\setouterhboxAtEnd%
%</package>
%    \end{macrocode}
%
% \section{Test}
%
% \subsection{Catcode checks for loading}
%
%    \begin{macrocode}
%<*test1>
%    \end{macrocode}
%    \begin{macrocode}
\catcode`\{=1 %
\catcode`\}=2 %
\catcode`\#=6 %
\catcode`\@=11 %
\expandafter\ifx\csname count@\endcsname\relax
  \countdef\count@=255 %
\fi
\expandafter\ifx\csname @gobble\endcsname\relax
  \long\def\@gobble#1{}%
\fi
\expandafter\ifx\csname @firstofone\endcsname\relax
  \long\def\@firstofone#1{#1}%
\fi
\expandafter\ifx\csname loop\endcsname\relax
  \expandafter\@firstofone
\else
  \expandafter\@gobble
\fi
{%
  \def\loop#1\repeat{%
    \def\body{#1}%
    \iterate
  }%
  \def\iterate{%
    \body
      \let\next\iterate
    \else
      \let\next\relax
    \fi
    \next
  }%
  \let\repeat=\fi
}%
\def\RestoreCatcodes{}
\count@=0 %
\loop
  \edef\RestoreCatcodes{%
    \RestoreCatcodes
    \catcode\the\count@=\the\catcode\count@\relax
  }%
\ifnum\count@<255 %
  \advance\count@ 1 %
\repeat

\def\RangeCatcodeInvalid#1#2{%
  \count@=#1\relax
  \loop
    \catcode\count@=15 %
  \ifnum\count@<#2\relax
    \advance\count@ 1 %
  \repeat
}
\def\RangeCatcodeCheck#1#2#3{%
  \count@=#1\relax
  \loop
    \ifnum#3=\catcode\count@
    \else
      \errmessage{%
        Character \the\count@\space
        with wrong catcode \the\catcode\count@\space
        instead of \number#3%
      }%
    \fi
  \ifnum\count@<#2\relax
    \advance\count@ 1 %
  \repeat
}
\def\space{ }
\expandafter\ifx\csname LoadCommand\endcsname\relax
  \def\LoadCommand{\input setouterhbox.sty\relax}%
\fi
\def\Test{%
  \RangeCatcodeInvalid{0}{47}%
  \RangeCatcodeInvalid{58}{64}%
  \RangeCatcodeInvalid{91}{96}%
  \RangeCatcodeInvalid{123}{255}%
  \catcode`\@=12 %
  \catcode`\\=0 %
  \catcode`\%=14 %
  \LoadCommand
  \RangeCatcodeCheck{0}{36}{15}%
  \RangeCatcodeCheck{37}{37}{14}%
  \RangeCatcodeCheck{38}{47}{15}%
  \RangeCatcodeCheck{48}{57}{12}%
  \RangeCatcodeCheck{58}{63}{15}%
  \RangeCatcodeCheck{64}{64}{12}%
  \RangeCatcodeCheck{65}{90}{11}%
  \RangeCatcodeCheck{91}{91}{15}%
  \RangeCatcodeCheck{92}{92}{0}%
  \RangeCatcodeCheck{93}{96}{15}%
  \RangeCatcodeCheck{97}{122}{11}%
  \RangeCatcodeCheck{123}{255}{15}%
  \RestoreCatcodes
}
\Test
\csname @@end\endcsname
\end
%    \end{macrocode}
%    \begin{macrocode}
%</test1>
%    \end{macrocode}
%
% \subsection{Test with package \xpackage{url}}
%
%    \begin{macrocode}
%<*test2>
\nofiles
\documentclass[a5paper]{article}
\usepackage{url}[2005/06/27]
\usepackage{setouterhbox}

\newsavebox{\testbox}

\setlength{\parindent}{0pt}
\setlength{\parskip}{2em}

\begin{document}
\raggedright

\url{http://this.is.a.very.long.host.name/followed/%
by/a/very_long_long_long_path.html}%

\sbox\testbox{%
  \url{http://this.is.a.very.long.host.name/followed/%
  by/a/very_long_long_long_path.html}%
}%
\unhbox\testbox

\begin{setouterhbox}{\testbox}%
  \url{http://this.is.a.very.long.host.name/followed/%
  by/a/very_long_long_long_path.html}%
\end{setouterhbox}
\unhbox\testbox

\end{document}
%</test2>
%    \end{macrocode}
%
% \section{Installation}
%
% \subsection{Download}
%
% \paragraph{Package.} This package is available on
% CTAN\footnote{\CTANpkg{setouterhbox}}:
% \begin{description}
% \item[\CTAN{macros/latex/contrib/oberdiek/setouterhbox.dtx}] The source file.
% \item[\CTAN{macros/latex/contrib/oberdiek/setouterhbox.pdf}] Documentation.
% \end{description}
%
%
% \paragraph{Bundle.} All the packages of the bundle `oberdiek'
% are also available in a TDS compliant ZIP archive. There
% the packages are already unpacked and the documentation files
% are generated. The files and directories obey the TDS standard.
% \begin{description}
% \item[\CTANinstall{install/macros/latex/contrib/oberdiek.tds.zip}]
% \end{description}
% \emph{TDS} refers to the standard ``A Directory Structure
% for \TeX\ Files'' (\CTAN{tds/tds.pdf}). Directories
% with \xfile{texmf} in their name are usually organized this way.
%
% \subsection{Bundle installation}
%
% \paragraph{Unpacking.} Unpack the \xfile{oberdiek.tds.zip} in the
% TDS tree (also known as \xfile{texmf} tree) of your choice.
% Example (linux):
% \begin{quote}
%   |unzip oberdiek.tds.zip -d ~/texmf|
% \end{quote}
%
% \subsection{Package installation}
%
% \paragraph{Unpacking.} The \xfile{.dtx} file is a self-extracting
% \docstrip\ archive. The files are extracted by running the
% \xfile{.dtx} through \plainTeX:
% \begin{quote}
%   \verb|tex setouterhbox.dtx|
% \end{quote}
%
% \paragraph{TDS.} Now the different files must be moved into
% the different directories in your installation TDS tree
% (also known as \xfile{texmf} tree):
% \begin{quote}
% \def\t{^^A
% \begin{tabular}{@{}>{\ttfamily}l@{ $\rightarrow$ }>{\ttfamily}l@{}}
%   setouterhbox.sty & tex/generic/oberdiek/setouterhbox.sty\\
%   setouterhbox.pdf & doc/latex/oberdiek/setouterhbox.pdf\\
%   setouterhbox-example.tex & doc/latex/oberdiek/setouterhbox-example.tex\\
%   test/setouterhbox-test1.tex & doc/latex/oberdiek/test/setouterhbox-test1.tex\\
%   test/setouterhbox-test2.tex & doc/latex/oberdiek/test/setouterhbox-test2.tex\\
%   setouterhbox.dtx & source/latex/oberdiek/setouterhbox.dtx\\
% \end{tabular}^^A
% }^^A
% \sbox0{\t}^^A
% \ifdim\wd0>\linewidth
%   \begingroup
%     \advance\linewidth by\leftmargin
%     \advance\linewidth by\rightmargin
%   \edef\x{\endgroup
%     \def\noexpand\lw{\the\linewidth}^^A
%   }\x
%   \def\lwbox{^^A
%     \leavevmode
%     \hbox to \linewidth{^^A
%       \kern-\leftmargin\relax
%       \hss
%       \usebox0
%       \hss
%       \kern-\rightmargin\relax
%     }^^A
%   }^^A
%   \ifdim\wd0>\lw
%     \sbox0{\small\t}^^A
%     \ifdim\wd0>\linewidth
%       \ifdim\wd0>\lw
%         \sbox0{\footnotesize\t}^^A
%         \ifdim\wd0>\linewidth
%           \ifdim\wd0>\lw
%             \sbox0{\scriptsize\t}^^A
%             \ifdim\wd0>\linewidth
%               \ifdim\wd0>\lw
%                 \sbox0{\tiny\t}^^A
%                 \ifdim\wd0>\linewidth
%                   \lwbox
%                 \else
%                   \usebox0
%                 \fi
%               \else
%                 \lwbox
%               \fi
%             \else
%               \usebox0
%             \fi
%           \else
%             \lwbox
%           \fi
%         \else
%           \usebox0
%         \fi
%       \else
%         \lwbox
%       \fi
%     \else
%       \usebox0
%     \fi
%   \else
%     \lwbox
%   \fi
% \else
%   \usebox0
% \fi
% \end{quote}
% If you have a \xfile{docstrip.cfg} that configures and enables \docstrip's
% TDS installing feature, then some files can already be in the right
% place, see the documentation of \docstrip.
%
% \subsection{Refresh file name databases}
%
% If your \TeX~distribution
% (\TeX\,Live, \mikTeX, \dots) relies on file name databases, you must refresh
% these. For example, \TeX\,Live\ users run \verb|texhash| or
% \verb|mktexlsr|.
%
% \subsection{Some details for the interested}
%
% \paragraph{Unpacking with \LaTeX.}
% The \xfile{.dtx} chooses its action depending on the format:
% \begin{description}
% \item[\plainTeX:] Run \docstrip\ and extract the files.
% \item[\LaTeX:] Generate the documentation.
% \end{description}
% If you insist on using \LaTeX\ for \docstrip\ (really,
% \docstrip\ does not need \LaTeX), then inform the autodetect routine
% about your intention:
% \begin{quote}
%   \verb|latex \let\install=y% \iffalse meta-comment
%
% File: setouterhbox.dtx
% Version: 2016/05/16 v1.8
% Info: Set hbox in outer horizontal mode
%
% Copyright (C)
%    2005-2007 Heiko Oberdiek
%    2016-2019 Oberdiek Package Support Group
%    https://github.com/ho-tex/oberdiek/issues
%
% This work may be distributed and/or modified under the
% conditions of the LaTeX Project Public License, either
% version 1.3c of this license or (at your option) any later
% version. This version of this license is in
%    https://www.latex-project.org/lppl/lppl-1-3c.txt
% and the latest version of this license is in
%    https://www.latex-project.org/lppl.txt
% and version 1.3 or later is part of all distributions of
% LaTeX version 2005/12/01 or later.
%
% This work has the LPPL maintenance status "maintained".
%
% The Current Maintainers of this work are
% Heiko Oberdiek and the Oberdiek Package Support Group
% https://github.com/ho-tex/oberdiek/issues
%
% The Base Interpreter refers to any `TeX-Format',
% because some files are installed in TDS:tex/generic//.
%
% This work consists of the main source file setouterhbox.dtx
% and the derived files
%    setouterhbox.sty, setouterhbox.pdf, setouterhbox.ins, setouterhbox.drv,
%    setouterhbox-example.tex, setouterhbox-test1.tex,
%    setouterhbox-test2.tex.
%
% Distribution:
%    CTAN:macros/latex/contrib/oberdiek/setouterhbox.dtx
%    CTAN:macros/latex/contrib/oberdiek/setouterhbox.pdf
%
% Unpacking:
%    (a) If setouterhbox.ins is present:
%           tex setouterhbox.ins
%    (b) Without setouterhbox.ins:
%           tex setouterhbox.dtx
%    (c) If you insist on using LaTeX
%           latex \let\install=y% \iffalse meta-comment
%
% File: setouterhbox.dtx
% Version: 2016/05/16 v1.8
% Info: Set hbox in outer horizontal mode
%
% Copyright (C)
%    2005-2007 Heiko Oberdiek
%    2016-2019 Oberdiek Package Support Group
%    https://github.com/ho-tex/oberdiek/issues
%
% This work may be distributed and/or modified under the
% conditions of the LaTeX Project Public License, either
% version 1.3c of this license or (at your option) any later
% version. This version of this license is in
%    https://www.latex-project.org/lppl/lppl-1-3c.txt
% and the latest version of this license is in
%    https://www.latex-project.org/lppl.txt
% and version 1.3 or later is part of all distributions of
% LaTeX version 2005/12/01 or later.
%
% This work has the LPPL maintenance status "maintained".
%
% The Current Maintainers of this work are
% Heiko Oberdiek and the Oberdiek Package Support Group
% https://github.com/ho-tex/oberdiek/issues
%
% The Base Interpreter refers to any `TeX-Format',
% because some files are installed in TDS:tex/generic//.
%
% This work consists of the main source file setouterhbox.dtx
% and the derived files
%    setouterhbox.sty, setouterhbox.pdf, setouterhbox.ins, setouterhbox.drv,
%    setouterhbox-example.tex, setouterhbox-test1.tex,
%    setouterhbox-test2.tex.
%
% Distribution:
%    CTAN:macros/latex/contrib/oberdiek/setouterhbox.dtx
%    CTAN:macros/latex/contrib/oberdiek/setouterhbox.pdf
%
% Unpacking:
%    (a) If setouterhbox.ins is present:
%           tex setouterhbox.ins
%    (b) Without setouterhbox.ins:
%           tex setouterhbox.dtx
%    (c) If you insist on using LaTeX
%           latex \let\install=y\input{setouterhbox.dtx}
%        (quote the arguments according to the demands of your shell)
%
% Documentation:
%    (a) If setouterhbox.drv is present:
%           latex setouterhbox.drv
%    (b) Without setouterhbox.drv:
%           latex setouterhbox.dtx; ...
%    The class ltxdoc loads the configuration file ltxdoc.cfg
%    if available. Here you can specify further options, e.g.
%    use A4 as paper format:
%       \PassOptionsToClass{a4paper}{article}
%
%    Programm calls to get the documentation (example):
%       pdflatex setouterhbox.dtx
%       makeindex -s gind.ist setouterhbox.idx
%       pdflatex setouterhbox.dtx
%       makeindex -s gind.ist setouterhbox.idx
%       pdflatex setouterhbox.dtx
%
% Installation:
%    TDS:tex/generic/oberdiek/setouterhbox.sty
%    TDS:doc/latex/oberdiek/setouterhbox.pdf
%    TDS:doc/latex/oberdiek/setouterhbox-example.tex
%    TDS:doc/latex/oberdiek/test/setouterhbox-test1.tex
%    TDS:doc/latex/oberdiek/test/setouterhbox-test2.tex
%    TDS:source/latex/oberdiek/setouterhbox.dtx
%
%<*ignore>
\begingroup
  \catcode123=1 %
  \catcode125=2 %
  \def\x{LaTeX2e}%
\expandafter\endgroup
\ifcase 0\ifx\install y1\fi\expandafter
         \ifx\csname processbatchFile\endcsname\relax\else1\fi
         \ifx\fmtname\x\else 1\fi\relax
\else\csname fi\endcsname
%</ignore>
%<*install>
\input docstrip.tex
\Msg{************************************************************************}
\Msg{* Installation}
\Msg{* Package: setouterhbox 2016/05/16 v1.8 Set hbox in outer horizontal mode (HO)}
\Msg{************************************************************************}

\keepsilent
\askforoverwritefalse

\let\MetaPrefix\relax
\preamble

This is a generated file.

Project: setouterhbox
Version: 2016/05/16 v1.8

Copyright (C)
   2005-2007 Heiko Oberdiek
   2016-2019 Oberdiek Package Support Group

This work may be distributed and/or modified under the
conditions of the LaTeX Project Public License, either
version 1.3c of this license or (at your option) any later
version. This version of this license is in
   https://www.latex-project.org/lppl/lppl-1-3c.txt
and the latest version of this license is in
   https://www.latex-project.org/lppl.txt
and version 1.3 or later is part of all distributions of
LaTeX version 2005/12/01 or later.

This work has the LPPL maintenance status "maintained".

The Current Maintainers of this work are
Heiko Oberdiek and the Oberdiek Package Support Group
https://github.com/ho-tex/oberdiek/issues


The Base Interpreter refers to any `TeX-Format',
because some files are installed in TDS:tex/generic//.

This work consists of the main source file setouterhbox.dtx
and the derived files
   setouterhbox.sty, setouterhbox.pdf, setouterhbox.ins, setouterhbox.drv,
   setouterhbox-example.tex, setouterhbox-test1.tex,
   setouterhbox-test2.tex.

\endpreamble
\let\MetaPrefix\DoubleperCent

\generate{%
  \file{setouterhbox.ins}{\from{setouterhbox.dtx}{install}}%
  \file{setouterhbox.drv}{\from{setouterhbox.dtx}{driver}}%
  \usedir{tex/generic/oberdiek}%
  \file{setouterhbox.sty}{\from{setouterhbox.dtx}{package}}%
  \usedir{doc/latex/oberdiek}%
  \file{setouterhbox-example.tex}{\from{setouterhbox.dtx}{example}}%
%  \usedir{doc/latex/oberdiek/test}%
%  \file{setouterhbox-test1.tex}{\from{setouterhbox.dtx}{test1}}%
%  \file{setouterhbox-test2.tex}{\from{setouterhbox.dtx}{test2}}%
  \nopreamble
  \nopostamble
%  \usedir{source/latex/oberdiek/catalogue}%
%  \file{setouterhbox.xml}{\from{setouterhbox.dtx}{catalogue}}%
}

\catcode32=13\relax% active space
\let =\space%
\Msg{************************************************************************}
\Msg{*}
\Msg{* To finish the installation you have to move the following}
\Msg{* file into a directory searched by TeX:}
\Msg{*}
\Msg{*     setouterhbox.sty}
\Msg{*}
\Msg{* To produce the documentation run the file `setouterhbox.drv'}
\Msg{* through LaTeX.}
\Msg{*}
\Msg{* Happy TeXing!}
\Msg{*}
\Msg{************************************************************************}

\endbatchfile
%</install>
%<*ignore>
\fi
%</ignore>
%<*driver>
\NeedsTeXFormat{LaTeX2e}
\ProvidesFile{setouterhbox.drv}%
  [2016/05/16 v1.8 Set hbox in outer horizontal mode (HO)]%
\documentclass{ltxdoc}
\usepackage{holtxdoc}[2011/11/22]
\begin{document}
  \DocInput{setouterhbox.dtx}%
\end{document}
%</driver>
% \fi
%
%
% \CharacterTable
%  {Upper-case    \A\B\C\D\E\F\G\H\I\J\K\L\M\N\O\P\Q\R\S\T\U\V\W\X\Y\Z
%   Lower-case    \a\b\c\d\e\f\g\h\i\j\k\l\m\n\o\p\q\r\s\t\u\v\w\x\y\z
%   Digits        \0\1\2\3\4\5\6\7\8\9
%   Exclamation   \!     Double quote  \"     Hash (number) \#
%   Dollar        \$     Percent       \%     Ampersand     \&
%   Acute accent  \'     Left paren    \(     Right paren   \)
%   Asterisk      \*     Plus          \+     Comma         \,
%   Minus         \-     Point         \.     Solidus       \/
%   Colon         \:     Semicolon     \;     Less than     \<
%   Equals        \=     Greater than  \>     Question mark \?
%   Commercial at \@     Left bracket  \[     Backslash     \\
%   Right bracket \]     Circumflex    \^     Underscore    \_
%   Grave accent  \`     Left brace    \{     Vertical bar  \|
%   Right brace   \}     Tilde         \~}
%
% \GetFileInfo{setouterhbox.drv}
%
% \title{The \xpackage{setouterhbox} package}
% \date{2016/05/16 v1.8}
% \author{Heiko Oberdiek\thanks
% {Please report any issues at \url{https://github.com/ho-tex/oberdiek/issues}}}
%
% \maketitle
%
% \begin{abstract}
% If math stuff is set in an \cs{hbox}, then TeX
% performs some optimization and omits the implicite
% penalties \cs{binoppenalty} and \cs{relpenalty}.
% This packages tries to put stuff into an \cs{hbox}
% without getting lost of those penalties.
% \end{abstract}
%
% \tableofcontents
%
% \section{Documentation}
%
% \subsection{Introduction}
%
% There is a situation in \xpackage{hyperref}'s driver for dvips
% where the user wants to have links that can be broken across
% lines. However dvips doesn't support the feature. With option
% \xoption{breaklinks} \xpackage{hyperref} sets the links as
% usual, put them in a box and write the link data with
% box dimensions into the appropriate \cs{special}s.
% Then, however, it does not set the complete unbreakable
% box, but it unwrappes the material inside to allow line
% breaks. Of course line breaking and glue setting will falsify
% the link dimensions, but line breaking was more important
% for the user.
%
% \subsection{Acknowledgement}
%
% Jonathan Fine, Donald Arsenau and me discussed the problem
% in the newsgroup \xnewsgroup{comp.text.tex} where Damian
% Menscher has started the thread, see \cite{newsstart}.
%
% The discussion was productive and generated many ideas
% and code examples. In order to have a more permanent
% result I wrote this package and tried to implement
% most of the ideas, a kind of summary of the discussion.
% Thus I want and have to thank Jonathan Fine and Donald Arsenau
% very much.
%
% Two weeks later David Kastrup (posting in
% \xnewsgroup{comp.text.tex}, \cite{kastrup})
% remembered an old article of Michael Downes (\cite{downes})
% in TUGboat, where Michael Downes already presented the
% method we discuss here. Nowadays we have \eTeX\ that extends
% the tool set of a \TeX\ macro programmer. Especially useful
% \eTeX\ was in this package for detecting and dealing with
% errorneous situations.
%
% However also nowadays a perfect solution for the problem
% is still missing at macro level. Probably someone has
% to go deep in the internals of the \TeX\ compiler to
% implement a switch that let penalties stay where otherwise
% \TeX\ would remove them for optimization reasons.
%
% \subsection{Usage}
%
% \paragraph{Package loading.}
% \LaTeX: as usually:
% \begin{quote}
%   |\usepackage{setouterhbox}|
% \end{quote}
% The package can also be included directly, thus \plainTeX\ users
% write:
% \begin{quote}
%   |\input setouterhbox.sty|
% \end{quote}
%
% \paragraph{Register allocation.}
% The material will be put into a box, thus we need to know these
% box number. If you need to allocate a new box register:
% \begin{description}
%  \item[\LaTeX:] |\newsavebox{\|\meta{name}|}|
%  \item[\plainTeX:] |\newbox\|\meta{name}
% \end{description}
% Then |\|\meta{name} is a command that held the box number.
%
% \paragraph{Box wrapping.}
% \LaTeX\ users put the material in the box with an environment
% similar to \texttt{lrbox}. The environment \texttt{setouterhbox}
% uses the same syntax and offers the same features, such
% as verbatim stuff inside:
% \begin{quote}
%  |\begin{setouterhbox}{|\meta{box number}|}|\dots
%  |\end{setouterhbox}|
% \end{quote}
% Users with \plainTeX\ do not have environments, they use instead:
% \begin{quote}
%   |\setouterhbox{|\meta{box number}|}|\dots|\endsetouterhbox|
% \end{quote}
% In both cases the material is put into an \cs{hbox} and assigned
% to the given box, denoted by \meta{box number}. Note the
% assignment is local, the same way \texttt{lrbox} behaves.
%
% \paragraph{Unwrapping.}
% The box material is ready for unwrapping:
% \begin{quote}
%   |\unhbox|\meta{box number}
% \end{quote}
%
% \subsection{Option \xoption{hyperref}}
%
% Package url uses math mode for typesetting urls.
% Break points are inserted by \cs{binoppenalty} and
% \cs{relpenalty}. Unhappily these break points are
% removed, if \xpackage{hyperref}
% is used with option {breaklinks}
% and drivers that depend on \xoption{pdfmark}:
% \xoption{dvips}, \xoption{vtexpdfmark}, \xoption{textures},
% and \xoption{dvipsone}.
% Thus the option \xoption{hyperref} enables the method
% of this package to avoid the removal of \cs{relpenalty}
% and \cs{binoppenalty}. Thus you get more break points.
% However, the link areas are still wrong for these
% drivers, because they are not supporting broken
% links.
%
% Note, you need version 2006/08/16 v6.75c of package \xpackage{hyperref},
% because starting with this version the necessary hook is provided
% that package \xpackage{setouterhbox} uses.
% \begin{quote}
%   |\usepackage[|\dots|]{hyperref}[2006/08/16]|\\
%   |\usepackage[hyperref]{setouterhbox}|
% \end{quote}
% Package order does not matter.
%
% \subsection{Example}
%
%    \begin{macrocode}
%<*example>
\documentclass[a5paper]{article}
\usepackage{url}[2005/06/27]
\usepackage{setouterhbox}

\newsavebox{\testbox}

\setlength{\parindent}{0pt}
\setlength{\parskip}{2em}

\begin{document}
\raggedright

\url{http://this.is.a.very.long.host.name/followed/%
by/a/very_long_long_long_path.html}%

\sbox\testbox{%
  \url{http://this.is.a.very.long.host.name/followed/%
  by/a/very_long_long_long_path.html}%
}%
\unhbox\testbox

\begin{setouterhbox}{\testbox}%
  \url{http://this.is.a.very.long.host.name/followed/%
  by/a/very_long_long_long_path.html}%
\end{setouterhbox}
\unhbox\testbox

\end{document}
%</example>
%    \end{macrocode}
%
% \StopEventually{
% }
%
% \section{Implementation}
%
% Internal macros are prefixed by \cs{setouterhbox}, |@| is
% not used inside names, thus we do not need to care of its
% catcode if we are not using it as \LaTeX\ package.
%
% \subsection{Package start stuff}
%
%    \begin{macrocode}
%<*package>
%    \end{macrocode}
%
% Prevent reloading more than one, necessary for \plainTeX:
%    Reload check, especially if the package is not used with \LaTeX.
%    \begin{macrocode}
\begingroup\catcode61\catcode48\catcode32=10\relax%
  \catcode13=5 % ^^M
  \endlinechar=13 %
  \catcode35=6 % #
  \catcode39=12 % '
  \catcode44=12 % ,
  \catcode45=12 % -
  \catcode46=12 % .
  \catcode58=12 % :
  \catcode64=11 % @
  \catcode123=1 % {
  \catcode125=2 % }
  \expandafter\let\expandafter\x\csname ver@setouterhbox.sty\endcsname
  \ifx\x\relax % plain-TeX, first loading
  \else
    \def\empty{}%
    \ifx\x\empty % LaTeX, first loading,
      % variable is initialized, but \ProvidesPackage not yet seen
    \else
      \expandafter\ifx\csname PackageInfo\endcsname\relax
        \def\x#1#2{%
          \immediate\write-1{Package #1 Info: #2.}%
        }%
      \else
        \def\x#1#2{\PackageInfo{#1}{#2, stopped}}%
      \fi
      \x{setouterhbox}{The package is already loaded}%
      \aftergroup\endinput
    \fi
  \fi
\endgroup%
%    \end{macrocode}
%    Package identification:
%    \begin{macrocode}
\begingroup\catcode61\catcode48\catcode32=10\relax%
  \catcode13=5 % ^^M
  \endlinechar=13 %
  \catcode35=6 % #
  \catcode39=12 % '
  \catcode40=12 % (
  \catcode41=12 % )
  \catcode44=12 % ,
  \catcode45=12 % -
  \catcode46=12 % .
  \catcode47=12 % /
  \catcode58=12 % :
  \catcode64=11 % @
  \catcode91=12 % [
  \catcode93=12 % ]
  \catcode123=1 % {
  \catcode125=2 % }
  \expandafter\ifx\csname ProvidesPackage\endcsname\relax
    \def\x#1#2#3[#4]{\endgroup
      \immediate\write-1{Package: #3 #4}%
      \xdef#1{#4}%
    }%
  \else
    \def\x#1#2[#3]{\endgroup
      #2[{#3}]%
      \ifx#1\@undefined
        \xdef#1{#3}%
      \fi
      \ifx#1\relax
        \xdef#1{#3}%
      \fi
    }%
  \fi
\expandafter\x\csname ver@setouterhbox.sty\endcsname
\ProvidesPackage{setouterhbox}%
  [2016/05/16 v1.8 Set hbox in outer horizontal mode (HO)]%
%    \end{macrocode}
%
%    \begin{macrocode}
\begingroup\catcode61\catcode48\catcode32=10\relax%
  \catcode13=5 % ^^M
  \endlinechar=13 %
  \catcode123=1 % {
  \catcode125=2 % }
  \catcode64=11 % @
  \def\x{\endgroup
    \expandafter\edef\csname setouterhboxAtEnd\endcsname{%
      \endlinechar=\the\endlinechar\relax
      \catcode13=\the\catcode13\relax
      \catcode32=\the\catcode32\relax
      \catcode35=\the\catcode35\relax
      \catcode61=\the\catcode61\relax
      \catcode64=\the\catcode64\relax
      \catcode123=\the\catcode123\relax
      \catcode125=\the\catcode125\relax
    }%
  }%
\x\catcode61\catcode48\catcode32=10\relax%
\catcode13=5 % ^^M
\endlinechar=13 %
\catcode35=6 % #
\catcode64=11 % @
\catcode123=1 % {
\catcode125=2 % }
\def\TMP@EnsureCode#1#2{%
  \edef\setouterhboxAtEnd{%
    \setouterhboxAtEnd
    \catcode#1=\the\catcode#1\relax
  }%
  \catcode#1=#2\relax
}
\TMP@EnsureCode{40}{12}% (
\TMP@EnsureCode{41}{12}% )
\TMP@EnsureCode{44}{12}% ,
\TMP@EnsureCode{45}{12}% -
\TMP@EnsureCode{46}{12}% .
\TMP@EnsureCode{47}{12}% /
\TMP@EnsureCode{58}{12}% :
\TMP@EnsureCode{60}{12}% <
\TMP@EnsureCode{62}{12}% >
\TMP@EnsureCode{91}{12}% [
\TMP@EnsureCode{93}{12}% ]
\TMP@EnsureCode{96}{12}% `
\edef\setouterhboxAtEnd{\setouterhboxAtEnd\noexpand\endinput}
%    \end{macrocode}
%
% \subsection{Interface macros}
%
%    \begin{macro}{\setouterhboxBox}
% The method requires a global box assignment. To be on the
% safe side, a new box register is allocated for this
% global box assignment.
%    \begin{macrocode}
\newbox\setouterhboxBox
%    \end{macrocode}
%    \end{macro}
%
%    \begin{macro}{\setouterhboxFailure}
% Error message for both \plainTeX\ and \LaTeX
%    \begin{macrocode}
\begingroup\expandafter\expandafter\expandafter\endgroup
\expandafter\ifx\csname RequirePackage\endcsname\relax
  \input infwarerr.sty\relax
\else
  \RequirePackage{infwarerr}[2016/05/16]%
\fi
\edef\setouterhboxFailure#1#2{%
  \expandafter\noexpand\csname @PackageError\endcsname
      {setouterhbox}{#1}{#2}%
}
%    \end{macrocode}
%    \end{macro}
%
% \subsection{Main part}
%
% eTeX provides much better means for checking
% error conditions. Thus lines marked by "E" are executed
% if eTeX is available, otherwise the lines marked by "T" are
% used.
%    \begin{macrocode}
\begingroup\expandafter\expandafter\expandafter\endgroup
\expandafter\ifx\csname lastnodetype\endcsname\relax
  \catcode`T=9 % ignore
  \catcode`E=14 % comment
\else
  \catcode`T=14 % comment
  \catcode`E=9 % ignore
\fi
%    \end{macrocode}
%
%    \begin{macro}{\setouterhboxRemove}
% Remove all kern, glue, and penalty nodes;
% poor man's version, if \eTeX\ is not available
%    \begin{macrocode}
\def\setouterhboxRemove{%
E \ifnum\lastnodetype<11 %
E   \else
E   \ifnum\lastnodetype>13 %
E   \else
      \unskip\unkern\unpenalty
E     \expandafter\expandafter\expandafter\setouterhboxRemove
E   \fi
E \fi
}%
%    \end{macrocode}
%    \end{macro}
%
%    \begin{macro}{\setouterhbox}
% Passing the box contents by macro parameter would prevent
% catcode changes in the box contents like by \cs{verb}.
% Also \cs{bgroup} and \cs{egroup} does not work, because stuff
% has to be added at the begin and end of the box, thus
% the syntax
% |\setouterhbox{|\meta{box number}|}|\dots|\endsetouterhbox|
% is used. Also we automatically get an environment \texttt{setouterhbox}
% if \LaTeX\ is used.
%    \begin{macrocode}
\def\setouterhbox#1{%
  \begingroup
    \def\setouterhboxNum{#1}%
    \setbox0\vbox\bgroup
T     \kern.123pt\relax % marker
T     \kern0pt\relax % removed by \setouterhboxRemove
      \begingroup
        \everypar{}%
        \noindent
}
%    \end{macrocode}
%    \end{macro}
%    \begin{macro}{\endsetouterhbox}
% Most of the work is done in the end part, thus the heart of
% the method follows:
%    \begin{macrocode}
\def\endsetouterhbox{%
      \endgroup
%    \end{macrocode}
% Omit the first pass to get the penalties
% of the second pass.
%    \begin{macrocode}
      \pretolerance-1 %
%    \end{macrocode}
%  We don't want a third pass with \cs{emergencystretch}.
%    \begin{macrocode}
      \tolerance10000 %
      \hsize\maxdimen
%    \end{macrocode}
% Line is not underfull:
%    \begin{macrocode}
      \parfillskip 0pt plus 1filll\relax
      \leftskip0pt\relax
%    \end{macrocode}
% Suppress underful \cs{hbox} warnings,
% is explicit line breaks are used.
%    \begin{macrocode}
      \rightskip0pt plus 1fil\relax
      \everypar{}%
%    \end{macrocode}
% Ensure that there is a paragraph and
% prevents \cs{endgraph} from eating terminal glue:
%    \begin{macrocode}
      \kern0pt%
      \endgraf
      \setouterhboxRemove
E     \ifnum\lastnodetype=1 %
E       \global\setbox\setouterhboxBox\lastbox
E       \loop
E         \setouterhboxRemove
E       \ifnum\lastnodetype=1 %
E         \setbox0=\lastbox
E         \global\setbox\setouterhboxBox=\hbox{%
E           \unhbox0 %
%    \end{macrocode}
% Remove \cs{rightskip}, a penalty with -10000 is part of the previous line.
%    \begin{macrocode}
E           \unskip
E           \unhbox\setouterhboxBox
E         }%
E       \repeat
E     \else
E       \setouterhboxFailure{%
E         Something is wrong%
E       }{%
E         Could not find expected line.%
E         \MessageBreak
E         (\string\lastnodetype: \number\lastnodetype, expected: 1)%
E       }%
E     \fi
E     \setouterhboxRemove
T     \global\setbox\setouterhboxBox\lastbox
T     \loop
T       \setouterhboxRemove
T       \setbox0=\lastbox
T     \ifcase\ifvoid0 1\else0\fi
T       \global\setbox\setouterhboxBox=\hbox{%
T         \unhbox0 %
%    \end{macrocode}
% Remove \cs{rightskip}, a penalty with -10000 is part of the previous line.
%    \begin{macrocode}
T         \unskip
T         \unhbox\setouterhboxBox
T       }%
T     \repeat
T     \ifdim.123pt=\lastkern
T     \else
T       \setouterhboxFailure{%
T         Something is wrong%
T       }{%
T         Unexpected stuff was detected before the line.%
T       }%
T     \fi
T   \egroup
T   \ifcase \ifnum\wd0=0 \else 1\fi
T           \ifdim\ht0=.123pt \else 1\fi
T           \ifnum\dp0=0 \else 1\fi
T           0 %
E     \ifnum\lastnodetype=-1 %
%    \end{macrocode}
% There was just one line that we have caught.
%    \begin{macrocode}
      \else
        \setouterhboxFailure{%
            Something is wrong%
        }{%
            After fetching the line there is more unexpected stuff.%
E           \MessageBreak
E           (\string\lastnodetype: \number\lastnodetype, expected: -1)%
        }%
      \fi
E   \egroup
  \expandafter\endgroup
  \expandafter\setouterhboxFinish\expandafter{%
    \number\setouterhboxNum
  }%
}
%    \end{macrocode}
%    \end{macro}
%
% \subsection{Environment support}
%
% Check \cs{@currenvir} for the case that \cs{setouterhbox}
% was called as environment. Then the box assignment
% must be put after the \cs{endgroup} of |\end{|\dots|}|.
%    \begin{macrocode}
\def\setouterhboxCurr{setouterhbox}
\def\setouterhboxLast#1{%
  \setbox#1\hbox{%
    \unhbox\setouterhboxBox
    \unskip % remove \rightskip glue
    \unskip % remove \parfillskip glue
    \unpenalty % remove paragraph ending \penalty 10000
    \unkern % remove explicit kern inserted above
  }%
}
%    \end{macrocode}
%    \begin{macro}{\setouterhboxFinish}
% |#1| is an explicit number.
%    \begin{macrocode}
\def\setouterhboxFinish#1{%
  \begingroup\expandafter\expandafter\expandafter\endgroup
  \expandafter\ifx\csname @currenvir\endcsname\setouterhboxCurr
    \aftergroup\setouterhboxLast
    \aftergroup{%
    \setouterhboxAfter #1\NIL
    \aftergroup}%
  \else
    \setouterhboxLast{#1}%
  \fi
}
%    \end{macrocode}
%    \end{macro}
%    \begin{macro}{\setouterhboxAfter}
% |#1| is an explicit number.
%    \begin{macrocode}
\def\setouterhboxAfter#1#2\NIL{%
  \aftergroup#1%
  \ifx\\#2\\%
  \else
    \setouterhboxReturnAfterFi{%
      \setouterhboxAfter#2\NIL
    }%
  \fi
}
%    \end{macrocode}
%    \end{macro}
%    \begin{macro}{\setouterhboxReturnAfterFi}
% A utility macro to get tail recursion.
%    \begin{macrocode}
\long\def\setouterhboxReturnAfterFi#1\fi{\fi#1}
%    \end{macrocode}
%    \end{macro}
% Restore catcodes we have need to distinguish between
% the implementation with and without \eTeX.
%    \begin{macrocode}
\catcode69=11\relax % E
\catcode84=11\relax % T
%    \end{macrocode}
%
% \subsection{Option \xoption{hyperref}}
%    \begin{macrocode}
\begingroup
  \def\x{LaTeX2e}%
\expandafter\endgroup
\ifx\x\fmtname
\else
  \expandafter\setouterhboxAtEnd
\fi%
%    \end{macrocode}
%    \begin{macro}{\Hy@setouterhbox}
% \cs{Hy@setouterhbox} is the internal hook that \xpackage{hyperref}
% uses since 2006/02/12 v6.75a.
%    \begin{macrocode}
\DeclareOption{hyperref}{%
  \long\def\Hy@setouterhbox#1#2{%
    \setouterhbox{#1}#2\endsetouterhbox
  }%
}
%    \end{macrocode}
%    \end{macro}
%    \begin{macrocode}
\ProcessOptions\relax
%    \end{macrocode}
%
%    \begin{macrocode}
\setouterhboxAtEnd%
%</package>
%    \end{macrocode}
%
% \section{Test}
%
% \subsection{Catcode checks for loading}
%
%    \begin{macrocode}
%<*test1>
%    \end{macrocode}
%    \begin{macrocode}
\catcode`\{=1 %
\catcode`\}=2 %
\catcode`\#=6 %
\catcode`\@=11 %
\expandafter\ifx\csname count@\endcsname\relax
  \countdef\count@=255 %
\fi
\expandafter\ifx\csname @gobble\endcsname\relax
  \long\def\@gobble#1{}%
\fi
\expandafter\ifx\csname @firstofone\endcsname\relax
  \long\def\@firstofone#1{#1}%
\fi
\expandafter\ifx\csname loop\endcsname\relax
  \expandafter\@firstofone
\else
  \expandafter\@gobble
\fi
{%
  \def\loop#1\repeat{%
    \def\body{#1}%
    \iterate
  }%
  \def\iterate{%
    \body
      \let\next\iterate
    \else
      \let\next\relax
    \fi
    \next
  }%
  \let\repeat=\fi
}%
\def\RestoreCatcodes{}
\count@=0 %
\loop
  \edef\RestoreCatcodes{%
    \RestoreCatcodes
    \catcode\the\count@=\the\catcode\count@\relax
  }%
\ifnum\count@<255 %
  \advance\count@ 1 %
\repeat

\def\RangeCatcodeInvalid#1#2{%
  \count@=#1\relax
  \loop
    \catcode\count@=15 %
  \ifnum\count@<#2\relax
    \advance\count@ 1 %
  \repeat
}
\def\RangeCatcodeCheck#1#2#3{%
  \count@=#1\relax
  \loop
    \ifnum#3=\catcode\count@
    \else
      \errmessage{%
        Character \the\count@\space
        with wrong catcode \the\catcode\count@\space
        instead of \number#3%
      }%
    \fi
  \ifnum\count@<#2\relax
    \advance\count@ 1 %
  \repeat
}
\def\space{ }
\expandafter\ifx\csname LoadCommand\endcsname\relax
  \def\LoadCommand{\input setouterhbox.sty\relax}%
\fi
\def\Test{%
  \RangeCatcodeInvalid{0}{47}%
  \RangeCatcodeInvalid{58}{64}%
  \RangeCatcodeInvalid{91}{96}%
  \RangeCatcodeInvalid{123}{255}%
  \catcode`\@=12 %
  \catcode`\\=0 %
  \catcode`\%=14 %
  \LoadCommand
  \RangeCatcodeCheck{0}{36}{15}%
  \RangeCatcodeCheck{37}{37}{14}%
  \RangeCatcodeCheck{38}{47}{15}%
  \RangeCatcodeCheck{48}{57}{12}%
  \RangeCatcodeCheck{58}{63}{15}%
  \RangeCatcodeCheck{64}{64}{12}%
  \RangeCatcodeCheck{65}{90}{11}%
  \RangeCatcodeCheck{91}{91}{15}%
  \RangeCatcodeCheck{92}{92}{0}%
  \RangeCatcodeCheck{93}{96}{15}%
  \RangeCatcodeCheck{97}{122}{11}%
  \RangeCatcodeCheck{123}{255}{15}%
  \RestoreCatcodes
}
\Test
\csname @@end\endcsname
\end
%    \end{macrocode}
%    \begin{macrocode}
%</test1>
%    \end{macrocode}
%
% \subsection{Test with package \xpackage{url}}
%
%    \begin{macrocode}
%<*test2>
\nofiles
\documentclass[a5paper]{article}
\usepackage{url}[2005/06/27]
\usepackage{setouterhbox}

\newsavebox{\testbox}

\setlength{\parindent}{0pt}
\setlength{\parskip}{2em}

\begin{document}
\raggedright

\url{http://this.is.a.very.long.host.name/followed/%
by/a/very_long_long_long_path.html}%

\sbox\testbox{%
  \url{http://this.is.a.very.long.host.name/followed/%
  by/a/very_long_long_long_path.html}%
}%
\unhbox\testbox

\begin{setouterhbox}{\testbox}%
  \url{http://this.is.a.very.long.host.name/followed/%
  by/a/very_long_long_long_path.html}%
\end{setouterhbox}
\unhbox\testbox

\end{document}
%</test2>
%    \end{macrocode}
%
% \section{Installation}
%
% \subsection{Download}
%
% \paragraph{Package.} This package is available on
% CTAN\footnote{\CTANpkg{setouterhbox}}:
% \begin{description}
% \item[\CTAN{macros/latex/contrib/oberdiek/setouterhbox.dtx}] The source file.
% \item[\CTAN{macros/latex/contrib/oberdiek/setouterhbox.pdf}] Documentation.
% \end{description}
%
%
% \paragraph{Bundle.} All the packages of the bundle `oberdiek'
% are also available in a TDS compliant ZIP archive. There
% the packages are already unpacked and the documentation files
% are generated. The files and directories obey the TDS standard.
% \begin{description}
% \item[\CTANinstall{install/macros/latex/contrib/oberdiek.tds.zip}]
% \end{description}
% \emph{TDS} refers to the standard ``A Directory Structure
% for \TeX\ Files'' (\CTAN{tds/tds.pdf}). Directories
% with \xfile{texmf} in their name are usually organized this way.
%
% \subsection{Bundle installation}
%
% \paragraph{Unpacking.} Unpack the \xfile{oberdiek.tds.zip} in the
% TDS tree (also known as \xfile{texmf} tree) of your choice.
% Example (linux):
% \begin{quote}
%   |unzip oberdiek.tds.zip -d ~/texmf|
% \end{quote}
%
% \subsection{Package installation}
%
% \paragraph{Unpacking.} The \xfile{.dtx} file is a self-extracting
% \docstrip\ archive. The files are extracted by running the
% \xfile{.dtx} through \plainTeX:
% \begin{quote}
%   \verb|tex setouterhbox.dtx|
% \end{quote}
%
% \paragraph{TDS.} Now the different files must be moved into
% the different directories in your installation TDS tree
% (also known as \xfile{texmf} tree):
% \begin{quote}
% \def\t{^^A
% \begin{tabular}{@{}>{\ttfamily}l@{ $\rightarrow$ }>{\ttfamily}l@{}}
%   setouterhbox.sty & tex/generic/oberdiek/setouterhbox.sty\\
%   setouterhbox.pdf & doc/latex/oberdiek/setouterhbox.pdf\\
%   setouterhbox-example.tex & doc/latex/oberdiek/setouterhbox-example.tex\\
%   test/setouterhbox-test1.tex & doc/latex/oberdiek/test/setouterhbox-test1.tex\\
%   test/setouterhbox-test2.tex & doc/latex/oberdiek/test/setouterhbox-test2.tex\\
%   setouterhbox.dtx & source/latex/oberdiek/setouterhbox.dtx\\
% \end{tabular}^^A
% }^^A
% \sbox0{\t}^^A
% \ifdim\wd0>\linewidth
%   \begingroup
%     \advance\linewidth by\leftmargin
%     \advance\linewidth by\rightmargin
%   \edef\x{\endgroup
%     \def\noexpand\lw{\the\linewidth}^^A
%   }\x
%   \def\lwbox{^^A
%     \leavevmode
%     \hbox to \linewidth{^^A
%       \kern-\leftmargin\relax
%       \hss
%       \usebox0
%       \hss
%       \kern-\rightmargin\relax
%     }^^A
%   }^^A
%   \ifdim\wd0>\lw
%     \sbox0{\small\t}^^A
%     \ifdim\wd0>\linewidth
%       \ifdim\wd0>\lw
%         \sbox0{\footnotesize\t}^^A
%         \ifdim\wd0>\linewidth
%           \ifdim\wd0>\lw
%             \sbox0{\scriptsize\t}^^A
%             \ifdim\wd0>\linewidth
%               \ifdim\wd0>\lw
%                 \sbox0{\tiny\t}^^A
%                 \ifdim\wd0>\linewidth
%                   \lwbox
%                 \else
%                   \usebox0
%                 \fi
%               \else
%                 \lwbox
%               \fi
%             \else
%               \usebox0
%             \fi
%           \else
%             \lwbox
%           \fi
%         \else
%           \usebox0
%         \fi
%       \else
%         \lwbox
%       \fi
%     \else
%       \usebox0
%     \fi
%   \else
%     \lwbox
%   \fi
% \else
%   \usebox0
% \fi
% \end{quote}
% If you have a \xfile{docstrip.cfg} that configures and enables \docstrip's
% TDS installing feature, then some files can already be in the right
% place, see the documentation of \docstrip.
%
% \subsection{Refresh file name databases}
%
% If your \TeX~distribution
% (\TeX\,Live, \mikTeX, \dots) relies on file name databases, you must refresh
% these. For example, \TeX\,Live\ users run \verb|texhash| or
% \verb|mktexlsr|.
%
% \subsection{Some details for the interested}
%
% \paragraph{Unpacking with \LaTeX.}
% The \xfile{.dtx} chooses its action depending on the format:
% \begin{description}
% \item[\plainTeX:] Run \docstrip\ and extract the files.
% \item[\LaTeX:] Generate the documentation.
% \end{description}
% If you insist on using \LaTeX\ for \docstrip\ (really,
% \docstrip\ does not need \LaTeX), then inform the autodetect routine
% about your intention:
% \begin{quote}
%   \verb|latex \let\install=y\input{setouterhbox.dtx}|
% \end{quote}
% Do not forget to quote the argument according to the demands
% of your shell.
%
% \paragraph{Generating the documentation.}
% You can use both the \xfile{.dtx} or the \xfile{.drv} to generate
% the documentation. The process can be configured by the
% configuration file \xfile{ltxdoc.cfg}. For instance, put this
% line into this file, if you want to have A4 as paper format:
% \begin{quote}
%   \verb|\PassOptionsToClass{a4paper}{article}|
% \end{quote}
% An example follows how to generate the
% documentation with pdf\LaTeX:
% \begin{quote}
%\begin{verbatim}
%pdflatex setouterhbox.dtx
%makeindex -s gind.ist setouterhbox.idx
%pdflatex setouterhbox.dtx
%makeindex -s gind.ist setouterhbox.idx
%pdflatex setouterhbox.dtx
%\end{verbatim}
% \end{quote}
%
% \begin{thebibliography}{9}
%
% \bibitem{newsstart}
%   Damian Menscher, \Newsgroup{comp.text.tex},
%   \textit{overlong lines in List of Figures},
%   \nolinkurl{<dh058t$qbd$1@news.ks.uiuc.edu>},
%   23rd September 2005.
%   \url{https://groups.google.com/group/comp.text.tex/msg/79648d4cf1f8bc13}
%
% \bibitem{kastrup}
%   David Kastrup, \Newsgroup{comp.text.tex},
%   \textit{Re: ANN: outerhbox.sty -- collect horizontal material,
%   for unboxing into a paragraph},
%   \nolinkurl{<85y855lrx3.fsf@lola.goethe.zz>},
%   7th October 2005.
%   \url{https://groups.google.com/group/comp.text.tex/msg/7cf0a345ef932e52}
%
% \bibitem{downes}
%   Michael Downes, \textit{Line breaking in \cs{unhbox}ed Text},
%   TUGboat 11 (1990), pp. 605--612.
%
% \bibitem{hyperref}
%   Sebastian Rahtz, Heiko Oberdiek:
%   \textit{The \xpackage{hyperref} package};
%   2006/08/16 v6.75c;
%   \CTANpkg{hyperref}.
%
% \end{thebibliography}
%
% \begin{History}
%   \begin{Version}{2005/10/05 v1.0}
%   \item
%     First version.
%   \end{Version}
%   \begin{Version}{2005/10/07 v1.1}
%   \item
%     Option \xoption{hyperref} added.
%   \end{Version}
%   \begin{Version}{2005/10/18 v1.2}
%   \item
%     Support for explicit line breaks added.
%   \end{Version}
%   \begin{Version}{2006/02/12 v1.3}
%   \item
%     DTX format.
%   \item
%     Documentation extended.
%   \end{Version}
%   \begin{Version}{2006/08/26 v1.4}
%   \item
%     Date of hyperref updated.
%   \end{Version}
%   \begin{Version}{2007/04/26 v1.5}
%   \item
%     Use of package \xpackage{infwarerr}.
%   \end{Version}
%   \begin{Version}{2007/05/17 v1.6}
%   \item
%     Standard header part for generic files.
%   \end{Version}
%   \begin{Version}{2007/09/09 v1.7}
%   \item
%     Catcode section added.
%   \end{Version}
%   \begin{Version}{2016/05/16 v1.8}
%   \item
%     Documentation updates.
%   \end{Version}
% \end{History}
%
% \PrintIndex
%
% \Finale
\endinput

%        (quote the arguments according to the demands of your shell)
%
% Documentation:
%    (a) If setouterhbox.drv is present:
%           latex setouterhbox.drv
%    (b) Without setouterhbox.drv:
%           latex setouterhbox.dtx; ...
%    The class ltxdoc loads the configuration file ltxdoc.cfg
%    if available. Here you can specify further options, e.g.
%    use A4 as paper format:
%       \PassOptionsToClass{a4paper}{article}
%
%    Programm calls to get the documentation (example):
%       pdflatex setouterhbox.dtx
%       makeindex -s gind.ist setouterhbox.idx
%       pdflatex setouterhbox.dtx
%       makeindex -s gind.ist setouterhbox.idx
%       pdflatex setouterhbox.dtx
%
% Installation:
%    TDS:tex/generic/oberdiek/setouterhbox.sty
%    TDS:doc/latex/oberdiek/setouterhbox.pdf
%    TDS:doc/latex/oberdiek/setouterhbox-example.tex
%    TDS:doc/latex/oberdiek/test/setouterhbox-test1.tex
%    TDS:doc/latex/oberdiek/test/setouterhbox-test2.tex
%    TDS:source/latex/oberdiek/setouterhbox.dtx
%
%<*ignore>
\begingroup
  \catcode123=1 %
  \catcode125=2 %
  \def\x{LaTeX2e}%
\expandafter\endgroup
\ifcase 0\ifx\install y1\fi\expandafter
         \ifx\csname processbatchFile\endcsname\relax\else1\fi
         \ifx\fmtname\x\else 1\fi\relax
\else\csname fi\endcsname
%</ignore>
%<*install>
\input docstrip.tex
\Msg{************************************************************************}
\Msg{* Installation}
\Msg{* Package: setouterhbox 2016/05/16 v1.8 Set hbox in outer horizontal mode (HO)}
\Msg{************************************************************************}

\keepsilent
\askforoverwritefalse

\let\MetaPrefix\relax
\preamble

This is a generated file.

Project: setouterhbox
Version: 2016/05/16 v1.8

Copyright (C)
   2005-2007 Heiko Oberdiek
   2016-2019 Oberdiek Package Support Group

This work may be distributed and/or modified under the
conditions of the LaTeX Project Public License, either
version 1.3c of this license or (at your option) any later
version. This version of this license is in
   https://www.latex-project.org/lppl/lppl-1-3c.txt
and the latest version of this license is in
   https://www.latex-project.org/lppl.txt
and version 1.3 or later is part of all distributions of
LaTeX version 2005/12/01 or later.

This work has the LPPL maintenance status "maintained".

The Current Maintainers of this work are
Heiko Oberdiek and the Oberdiek Package Support Group
https://github.com/ho-tex/oberdiek/issues


The Base Interpreter refers to any `TeX-Format',
because some files are installed in TDS:tex/generic//.

This work consists of the main source file setouterhbox.dtx
and the derived files
   setouterhbox.sty, setouterhbox.pdf, setouterhbox.ins, setouterhbox.drv,
   setouterhbox-example.tex, setouterhbox-test1.tex,
   setouterhbox-test2.tex.

\endpreamble
\let\MetaPrefix\DoubleperCent

\generate{%
  \file{setouterhbox.ins}{\from{setouterhbox.dtx}{install}}%
  \file{setouterhbox.drv}{\from{setouterhbox.dtx}{driver}}%
  \usedir{tex/generic/oberdiek}%
  \file{setouterhbox.sty}{\from{setouterhbox.dtx}{package}}%
  \usedir{doc/latex/oberdiek}%
  \file{setouterhbox-example.tex}{\from{setouterhbox.dtx}{example}}%
%  \usedir{doc/latex/oberdiek/test}%
%  \file{setouterhbox-test1.tex}{\from{setouterhbox.dtx}{test1}}%
%  \file{setouterhbox-test2.tex}{\from{setouterhbox.dtx}{test2}}%
  \nopreamble
  \nopostamble
%  \usedir{source/latex/oberdiek/catalogue}%
%  \file{setouterhbox.xml}{\from{setouterhbox.dtx}{catalogue}}%
}

\catcode32=13\relax% active space
\let =\space%
\Msg{************************************************************************}
\Msg{*}
\Msg{* To finish the installation you have to move the following}
\Msg{* file into a directory searched by TeX:}
\Msg{*}
\Msg{*     setouterhbox.sty}
\Msg{*}
\Msg{* To produce the documentation run the file `setouterhbox.drv'}
\Msg{* through LaTeX.}
\Msg{*}
\Msg{* Happy TeXing!}
\Msg{*}
\Msg{************************************************************************}

\endbatchfile
%</install>
%<*ignore>
\fi
%</ignore>
%<*driver>
\NeedsTeXFormat{LaTeX2e}
\ProvidesFile{setouterhbox.drv}%
  [2016/05/16 v1.8 Set hbox in outer horizontal mode (HO)]%
\documentclass{ltxdoc}
\usepackage{holtxdoc}[2011/11/22]
\begin{document}
  \DocInput{setouterhbox.dtx}%
\end{document}
%</driver>
% \fi
%
%
% \CharacterTable
%  {Upper-case    \A\B\C\D\E\F\G\H\I\J\K\L\M\N\O\P\Q\R\S\T\U\V\W\X\Y\Z
%   Lower-case    \a\b\c\d\e\f\g\h\i\j\k\l\m\n\o\p\q\r\s\t\u\v\w\x\y\z
%   Digits        \0\1\2\3\4\5\6\7\8\9
%   Exclamation   \!     Double quote  \"     Hash (number) \#
%   Dollar        \$     Percent       \%     Ampersand     \&
%   Acute accent  \'     Left paren    \(     Right paren   \)
%   Asterisk      \*     Plus          \+     Comma         \,
%   Minus         \-     Point         \.     Solidus       \/
%   Colon         \:     Semicolon     \;     Less than     \<
%   Equals        \=     Greater than  \>     Question mark \?
%   Commercial at \@     Left bracket  \[     Backslash     \\
%   Right bracket \]     Circumflex    \^     Underscore    \_
%   Grave accent  \`     Left brace    \{     Vertical bar  \|
%   Right brace   \}     Tilde         \~}
%
% \GetFileInfo{setouterhbox.drv}
%
% \title{The \xpackage{setouterhbox} package}
% \date{2016/05/16 v1.8}
% \author{Heiko Oberdiek\thanks
% {Please report any issues at \url{https://github.com/ho-tex/oberdiek/issues}}}
%
% \maketitle
%
% \begin{abstract}
% If math stuff is set in an \cs{hbox}, then TeX
% performs some optimization and omits the implicite
% penalties \cs{binoppenalty} and \cs{relpenalty}.
% This packages tries to put stuff into an \cs{hbox}
% without getting lost of those penalties.
% \end{abstract}
%
% \tableofcontents
%
% \section{Documentation}
%
% \subsection{Introduction}
%
% There is a situation in \xpackage{hyperref}'s driver for dvips
% where the user wants to have links that can be broken across
% lines. However dvips doesn't support the feature. With option
% \xoption{breaklinks} \xpackage{hyperref} sets the links as
% usual, put them in a box and write the link data with
% box dimensions into the appropriate \cs{special}s.
% Then, however, it does not set the complete unbreakable
% box, but it unwrappes the material inside to allow line
% breaks. Of course line breaking and glue setting will falsify
% the link dimensions, but line breaking was more important
% for the user.
%
% \subsection{Acknowledgement}
%
% Jonathan Fine, Donald Arsenau and me discussed the problem
% in the newsgroup \xnewsgroup{comp.text.tex} where Damian
% Menscher has started the thread, see \cite{newsstart}.
%
% The discussion was productive and generated many ideas
% and code examples. In order to have a more permanent
% result I wrote this package and tried to implement
% most of the ideas, a kind of summary of the discussion.
% Thus I want and have to thank Jonathan Fine and Donald Arsenau
% very much.
%
% Two weeks later David Kastrup (posting in
% \xnewsgroup{comp.text.tex}, \cite{kastrup})
% remembered an old article of Michael Downes (\cite{downes})
% in TUGboat, where Michael Downes already presented the
% method we discuss here. Nowadays we have \eTeX\ that extends
% the tool set of a \TeX\ macro programmer. Especially useful
% \eTeX\ was in this package for detecting and dealing with
% errorneous situations.
%
% However also nowadays a perfect solution for the problem
% is still missing at macro level. Probably someone has
% to go deep in the internals of the \TeX\ compiler to
% implement a switch that let penalties stay where otherwise
% \TeX\ would remove them for optimization reasons.
%
% \subsection{Usage}
%
% \paragraph{Package loading.}
% \LaTeX: as usually:
% \begin{quote}
%   |\usepackage{setouterhbox}|
% \end{quote}
% The package can also be included directly, thus \plainTeX\ users
% write:
% \begin{quote}
%   |\input setouterhbox.sty|
% \end{quote}
%
% \paragraph{Register allocation.}
% The material will be put into a box, thus we need to know these
% box number. If you need to allocate a new box register:
% \begin{description}
%  \item[\LaTeX:] |\newsavebox{\|\meta{name}|}|
%  \item[\plainTeX:] |\newbox\|\meta{name}
% \end{description}
% Then |\|\meta{name} is a command that held the box number.
%
% \paragraph{Box wrapping.}
% \LaTeX\ users put the material in the box with an environment
% similar to \texttt{lrbox}. The environment \texttt{setouterhbox}
% uses the same syntax and offers the same features, such
% as verbatim stuff inside:
% \begin{quote}
%  |\begin{setouterhbox}{|\meta{box number}|}|\dots
%  |\end{setouterhbox}|
% \end{quote}
% Users with \plainTeX\ do not have environments, they use instead:
% \begin{quote}
%   |\setouterhbox{|\meta{box number}|}|\dots|\endsetouterhbox|
% \end{quote}
% In both cases the material is put into an \cs{hbox} and assigned
% to the given box, denoted by \meta{box number}. Note the
% assignment is local, the same way \texttt{lrbox} behaves.
%
% \paragraph{Unwrapping.}
% The box material is ready for unwrapping:
% \begin{quote}
%   |\unhbox|\meta{box number}
% \end{quote}
%
% \subsection{Option \xoption{hyperref}}
%
% Package url uses math mode for typesetting urls.
% Break points are inserted by \cs{binoppenalty} and
% \cs{relpenalty}. Unhappily these break points are
% removed, if \xpackage{hyperref}
% is used with option {breaklinks}
% and drivers that depend on \xoption{pdfmark}:
% \xoption{dvips}, \xoption{vtexpdfmark}, \xoption{textures},
% and \xoption{dvipsone}.
% Thus the option \xoption{hyperref} enables the method
% of this package to avoid the removal of \cs{relpenalty}
% and \cs{binoppenalty}. Thus you get more break points.
% However, the link areas are still wrong for these
% drivers, because they are not supporting broken
% links.
%
% Note, you need version 2006/08/16 v6.75c of package \xpackage{hyperref},
% because starting with this version the necessary hook is provided
% that package \xpackage{setouterhbox} uses.
% \begin{quote}
%   |\usepackage[|\dots|]{hyperref}[2006/08/16]|\\
%   |\usepackage[hyperref]{setouterhbox}|
% \end{quote}
% Package order does not matter.
%
% \subsection{Example}
%
%    \begin{macrocode}
%<*example>
\documentclass[a5paper]{article}
\usepackage{url}[2005/06/27]
\usepackage{setouterhbox}

\newsavebox{\testbox}

\setlength{\parindent}{0pt}
\setlength{\parskip}{2em}

\begin{document}
\raggedright

\url{http://this.is.a.very.long.host.name/followed/%
by/a/very_long_long_long_path.html}%

\sbox\testbox{%
  \url{http://this.is.a.very.long.host.name/followed/%
  by/a/very_long_long_long_path.html}%
}%
\unhbox\testbox

\begin{setouterhbox}{\testbox}%
  \url{http://this.is.a.very.long.host.name/followed/%
  by/a/very_long_long_long_path.html}%
\end{setouterhbox}
\unhbox\testbox

\end{document}
%</example>
%    \end{macrocode}
%
% \StopEventually{
% }
%
% \section{Implementation}
%
% Internal macros are prefixed by \cs{setouterhbox}, |@| is
% not used inside names, thus we do not need to care of its
% catcode if we are not using it as \LaTeX\ package.
%
% \subsection{Package start stuff}
%
%    \begin{macrocode}
%<*package>
%    \end{macrocode}
%
% Prevent reloading more than one, necessary for \plainTeX:
%    Reload check, especially if the package is not used with \LaTeX.
%    \begin{macrocode}
\begingroup\catcode61\catcode48\catcode32=10\relax%
  \catcode13=5 % ^^M
  \endlinechar=13 %
  \catcode35=6 % #
  \catcode39=12 % '
  \catcode44=12 % ,
  \catcode45=12 % -
  \catcode46=12 % .
  \catcode58=12 % :
  \catcode64=11 % @
  \catcode123=1 % {
  \catcode125=2 % }
  \expandafter\let\expandafter\x\csname ver@setouterhbox.sty\endcsname
  \ifx\x\relax % plain-TeX, first loading
  \else
    \def\empty{}%
    \ifx\x\empty % LaTeX, first loading,
      % variable is initialized, but \ProvidesPackage not yet seen
    \else
      \expandafter\ifx\csname PackageInfo\endcsname\relax
        \def\x#1#2{%
          \immediate\write-1{Package #1 Info: #2.}%
        }%
      \else
        \def\x#1#2{\PackageInfo{#1}{#2, stopped}}%
      \fi
      \x{setouterhbox}{The package is already loaded}%
      \aftergroup\endinput
    \fi
  \fi
\endgroup%
%    \end{macrocode}
%    Package identification:
%    \begin{macrocode}
\begingroup\catcode61\catcode48\catcode32=10\relax%
  \catcode13=5 % ^^M
  \endlinechar=13 %
  \catcode35=6 % #
  \catcode39=12 % '
  \catcode40=12 % (
  \catcode41=12 % )
  \catcode44=12 % ,
  \catcode45=12 % -
  \catcode46=12 % .
  \catcode47=12 % /
  \catcode58=12 % :
  \catcode64=11 % @
  \catcode91=12 % [
  \catcode93=12 % ]
  \catcode123=1 % {
  \catcode125=2 % }
  \expandafter\ifx\csname ProvidesPackage\endcsname\relax
    \def\x#1#2#3[#4]{\endgroup
      \immediate\write-1{Package: #3 #4}%
      \xdef#1{#4}%
    }%
  \else
    \def\x#1#2[#3]{\endgroup
      #2[{#3}]%
      \ifx#1\@undefined
        \xdef#1{#3}%
      \fi
      \ifx#1\relax
        \xdef#1{#3}%
      \fi
    }%
  \fi
\expandafter\x\csname ver@setouterhbox.sty\endcsname
\ProvidesPackage{setouterhbox}%
  [2016/05/16 v1.8 Set hbox in outer horizontal mode (HO)]%
%    \end{macrocode}
%
%    \begin{macrocode}
\begingroup\catcode61\catcode48\catcode32=10\relax%
  \catcode13=5 % ^^M
  \endlinechar=13 %
  \catcode123=1 % {
  \catcode125=2 % }
  \catcode64=11 % @
  \def\x{\endgroup
    \expandafter\edef\csname setouterhboxAtEnd\endcsname{%
      \endlinechar=\the\endlinechar\relax
      \catcode13=\the\catcode13\relax
      \catcode32=\the\catcode32\relax
      \catcode35=\the\catcode35\relax
      \catcode61=\the\catcode61\relax
      \catcode64=\the\catcode64\relax
      \catcode123=\the\catcode123\relax
      \catcode125=\the\catcode125\relax
    }%
  }%
\x\catcode61\catcode48\catcode32=10\relax%
\catcode13=5 % ^^M
\endlinechar=13 %
\catcode35=6 % #
\catcode64=11 % @
\catcode123=1 % {
\catcode125=2 % }
\def\TMP@EnsureCode#1#2{%
  \edef\setouterhboxAtEnd{%
    \setouterhboxAtEnd
    \catcode#1=\the\catcode#1\relax
  }%
  \catcode#1=#2\relax
}
\TMP@EnsureCode{40}{12}% (
\TMP@EnsureCode{41}{12}% )
\TMP@EnsureCode{44}{12}% ,
\TMP@EnsureCode{45}{12}% -
\TMP@EnsureCode{46}{12}% .
\TMP@EnsureCode{47}{12}% /
\TMP@EnsureCode{58}{12}% :
\TMP@EnsureCode{60}{12}% <
\TMP@EnsureCode{62}{12}% >
\TMP@EnsureCode{91}{12}% [
\TMP@EnsureCode{93}{12}% ]
\TMP@EnsureCode{96}{12}% `
\edef\setouterhboxAtEnd{\setouterhboxAtEnd\noexpand\endinput}
%    \end{macrocode}
%
% \subsection{Interface macros}
%
%    \begin{macro}{\setouterhboxBox}
% The method requires a global box assignment. To be on the
% safe side, a new box register is allocated for this
% global box assignment.
%    \begin{macrocode}
\newbox\setouterhboxBox
%    \end{macrocode}
%    \end{macro}
%
%    \begin{macro}{\setouterhboxFailure}
% Error message for both \plainTeX\ and \LaTeX
%    \begin{macrocode}
\begingroup\expandafter\expandafter\expandafter\endgroup
\expandafter\ifx\csname RequirePackage\endcsname\relax
  \input infwarerr.sty\relax
\else
  \RequirePackage{infwarerr}[2016/05/16]%
\fi
\edef\setouterhboxFailure#1#2{%
  \expandafter\noexpand\csname @PackageError\endcsname
      {setouterhbox}{#1}{#2}%
}
%    \end{macrocode}
%    \end{macro}
%
% \subsection{Main part}
%
% eTeX provides much better means for checking
% error conditions. Thus lines marked by "E" are executed
% if eTeX is available, otherwise the lines marked by "T" are
% used.
%    \begin{macrocode}
\begingroup\expandafter\expandafter\expandafter\endgroup
\expandafter\ifx\csname lastnodetype\endcsname\relax
  \catcode`T=9 % ignore
  \catcode`E=14 % comment
\else
  \catcode`T=14 % comment
  \catcode`E=9 % ignore
\fi
%    \end{macrocode}
%
%    \begin{macro}{\setouterhboxRemove}
% Remove all kern, glue, and penalty nodes;
% poor man's version, if \eTeX\ is not available
%    \begin{macrocode}
\def\setouterhboxRemove{%
E \ifnum\lastnodetype<11 %
E   \else
E   \ifnum\lastnodetype>13 %
E   \else
      \unskip\unkern\unpenalty
E     \expandafter\expandafter\expandafter\setouterhboxRemove
E   \fi
E \fi
}%
%    \end{macrocode}
%    \end{macro}
%
%    \begin{macro}{\setouterhbox}
% Passing the box contents by macro parameter would prevent
% catcode changes in the box contents like by \cs{verb}.
% Also \cs{bgroup} and \cs{egroup} does not work, because stuff
% has to be added at the begin and end of the box, thus
% the syntax
% |\setouterhbox{|\meta{box number}|}|\dots|\endsetouterhbox|
% is used. Also we automatically get an environment \texttt{setouterhbox}
% if \LaTeX\ is used.
%    \begin{macrocode}
\def\setouterhbox#1{%
  \begingroup
    \def\setouterhboxNum{#1}%
    \setbox0\vbox\bgroup
T     \kern.123pt\relax % marker
T     \kern0pt\relax % removed by \setouterhboxRemove
      \begingroup
        \everypar{}%
        \noindent
}
%    \end{macrocode}
%    \end{macro}
%    \begin{macro}{\endsetouterhbox}
% Most of the work is done in the end part, thus the heart of
% the method follows:
%    \begin{macrocode}
\def\endsetouterhbox{%
      \endgroup
%    \end{macrocode}
% Omit the first pass to get the penalties
% of the second pass.
%    \begin{macrocode}
      \pretolerance-1 %
%    \end{macrocode}
%  We don't want a third pass with \cs{emergencystretch}.
%    \begin{macrocode}
      \tolerance10000 %
      \hsize\maxdimen
%    \end{macrocode}
% Line is not underfull:
%    \begin{macrocode}
      \parfillskip 0pt plus 1filll\relax
      \leftskip0pt\relax
%    \end{macrocode}
% Suppress underful \cs{hbox} warnings,
% is explicit line breaks are used.
%    \begin{macrocode}
      \rightskip0pt plus 1fil\relax
      \everypar{}%
%    \end{macrocode}
% Ensure that there is a paragraph and
% prevents \cs{endgraph} from eating terminal glue:
%    \begin{macrocode}
      \kern0pt%
      \endgraf
      \setouterhboxRemove
E     \ifnum\lastnodetype=1 %
E       \global\setbox\setouterhboxBox\lastbox
E       \loop
E         \setouterhboxRemove
E       \ifnum\lastnodetype=1 %
E         \setbox0=\lastbox
E         \global\setbox\setouterhboxBox=\hbox{%
E           \unhbox0 %
%    \end{macrocode}
% Remove \cs{rightskip}, a penalty with -10000 is part of the previous line.
%    \begin{macrocode}
E           \unskip
E           \unhbox\setouterhboxBox
E         }%
E       \repeat
E     \else
E       \setouterhboxFailure{%
E         Something is wrong%
E       }{%
E         Could not find expected line.%
E         \MessageBreak
E         (\string\lastnodetype: \number\lastnodetype, expected: 1)%
E       }%
E     \fi
E     \setouterhboxRemove
T     \global\setbox\setouterhboxBox\lastbox
T     \loop
T       \setouterhboxRemove
T       \setbox0=\lastbox
T     \ifcase\ifvoid0 1\else0\fi
T       \global\setbox\setouterhboxBox=\hbox{%
T         \unhbox0 %
%    \end{macrocode}
% Remove \cs{rightskip}, a penalty with -10000 is part of the previous line.
%    \begin{macrocode}
T         \unskip
T         \unhbox\setouterhboxBox
T       }%
T     \repeat
T     \ifdim.123pt=\lastkern
T     \else
T       \setouterhboxFailure{%
T         Something is wrong%
T       }{%
T         Unexpected stuff was detected before the line.%
T       }%
T     \fi
T   \egroup
T   \ifcase \ifnum\wd0=0 \else 1\fi
T           \ifdim\ht0=.123pt \else 1\fi
T           \ifnum\dp0=0 \else 1\fi
T           0 %
E     \ifnum\lastnodetype=-1 %
%    \end{macrocode}
% There was just one line that we have caught.
%    \begin{macrocode}
      \else
        \setouterhboxFailure{%
            Something is wrong%
        }{%
            After fetching the line there is more unexpected stuff.%
E           \MessageBreak
E           (\string\lastnodetype: \number\lastnodetype, expected: -1)%
        }%
      \fi
E   \egroup
  \expandafter\endgroup
  \expandafter\setouterhboxFinish\expandafter{%
    \number\setouterhboxNum
  }%
}
%    \end{macrocode}
%    \end{macro}
%
% \subsection{Environment support}
%
% Check \cs{@currenvir} for the case that \cs{setouterhbox}
% was called as environment. Then the box assignment
% must be put after the \cs{endgroup} of |\end{|\dots|}|.
%    \begin{macrocode}
\def\setouterhboxCurr{setouterhbox}
\def\setouterhboxLast#1{%
  \setbox#1\hbox{%
    \unhbox\setouterhboxBox
    \unskip % remove \rightskip glue
    \unskip % remove \parfillskip glue
    \unpenalty % remove paragraph ending \penalty 10000
    \unkern % remove explicit kern inserted above
  }%
}
%    \end{macrocode}
%    \begin{macro}{\setouterhboxFinish}
% |#1| is an explicit number.
%    \begin{macrocode}
\def\setouterhboxFinish#1{%
  \begingroup\expandafter\expandafter\expandafter\endgroup
  \expandafter\ifx\csname @currenvir\endcsname\setouterhboxCurr
    \aftergroup\setouterhboxLast
    \aftergroup{%
    \setouterhboxAfter #1\NIL
    \aftergroup}%
  \else
    \setouterhboxLast{#1}%
  \fi
}
%    \end{macrocode}
%    \end{macro}
%    \begin{macro}{\setouterhboxAfter}
% |#1| is an explicit number.
%    \begin{macrocode}
\def\setouterhboxAfter#1#2\NIL{%
  \aftergroup#1%
  \ifx\\#2\\%
  \else
    \setouterhboxReturnAfterFi{%
      \setouterhboxAfter#2\NIL
    }%
  \fi
}
%    \end{macrocode}
%    \end{macro}
%    \begin{macro}{\setouterhboxReturnAfterFi}
% A utility macro to get tail recursion.
%    \begin{macrocode}
\long\def\setouterhboxReturnAfterFi#1\fi{\fi#1}
%    \end{macrocode}
%    \end{macro}
% Restore catcodes we have need to distinguish between
% the implementation with and without \eTeX.
%    \begin{macrocode}
\catcode69=11\relax % E
\catcode84=11\relax % T
%    \end{macrocode}
%
% \subsection{Option \xoption{hyperref}}
%    \begin{macrocode}
\begingroup
  \def\x{LaTeX2e}%
\expandafter\endgroup
\ifx\x\fmtname
\else
  \expandafter\setouterhboxAtEnd
\fi%
%    \end{macrocode}
%    \begin{macro}{\Hy@setouterhbox}
% \cs{Hy@setouterhbox} is the internal hook that \xpackage{hyperref}
% uses since 2006/02/12 v6.75a.
%    \begin{macrocode}
\DeclareOption{hyperref}{%
  \long\def\Hy@setouterhbox#1#2{%
    \setouterhbox{#1}#2\endsetouterhbox
  }%
}
%    \end{macrocode}
%    \end{macro}
%    \begin{macrocode}
\ProcessOptions\relax
%    \end{macrocode}
%
%    \begin{macrocode}
\setouterhboxAtEnd%
%</package>
%    \end{macrocode}
%
% \section{Test}
%
% \subsection{Catcode checks for loading}
%
%    \begin{macrocode}
%<*test1>
%    \end{macrocode}
%    \begin{macrocode}
\catcode`\{=1 %
\catcode`\}=2 %
\catcode`\#=6 %
\catcode`\@=11 %
\expandafter\ifx\csname count@\endcsname\relax
  \countdef\count@=255 %
\fi
\expandafter\ifx\csname @gobble\endcsname\relax
  \long\def\@gobble#1{}%
\fi
\expandafter\ifx\csname @firstofone\endcsname\relax
  \long\def\@firstofone#1{#1}%
\fi
\expandafter\ifx\csname loop\endcsname\relax
  \expandafter\@firstofone
\else
  \expandafter\@gobble
\fi
{%
  \def\loop#1\repeat{%
    \def\body{#1}%
    \iterate
  }%
  \def\iterate{%
    \body
      \let\next\iterate
    \else
      \let\next\relax
    \fi
    \next
  }%
  \let\repeat=\fi
}%
\def\RestoreCatcodes{}
\count@=0 %
\loop
  \edef\RestoreCatcodes{%
    \RestoreCatcodes
    \catcode\the\count@=\the\catcode\count@\relax
  }%
\ifnum\count@<255 %
  \advance\count@ 1 %
\repeat

\def\RangeCatcodeInvalid#1#2{%
  \count@=#1\relax
  \loop
    \catcode\count@=15 %
  \ifnum\count@<#2\relax
    \advance\count@ 1 %
  \repeat
}
\def\RangeCatcodeCheck#1#2#3{%
  \count@=#1\relax
  \loop
    \ifnum#3=\catcode\count@
    \else
      \errmessage{%
        Character \the\count@\space
        with wrong catcode \the\catcode\count@\space
        instead of \number#3%
      }%
    \fi
  \ifnum\count@<#2\relax
    \advance\count@ 1 %
  \repeat
}
\def\space{ }
\expandafter\ifx\csname LoadCommand\endcsname\relax
  \def\LoadCommand{\input setouterhbox.sty\relax}%
\fi
\def\Test{%
  \RangeCatcodeInvalid{0}{47}%
  \RangeCatcodeInvalid{58}{64}%
  \RangeCatcodeInvalid{91}{96}%
  \RangeCatcodeInvalid{123}{255}%
  \catcode`\@=12 %
  \catcode`\\=0 %
  \catcode`\%=14 %
  \LoadCommand
  \RangeCatcodeCheck{0}{36}{15}%
  \RangeCatcodeCheck{37}{37}{14}%
  \RangeCatcodeCheck{38}{47}{15}%
  \RangeCatcodeCheck{48}{57}{12}%
  \RangeCatcodeCheck{58}{63}{15}%
  \RangeCatcodeCheck{64}{64}{12}%
  \RangeCatcodeCheck{65}{90}{11}%
  \RangeCatcodeCheck{91}{91}{15}%
  \RangeCatcodeCheck{92}{92}{0}%
  \RangeCatcodeCheck{93}{96}{15}%
  \RangeCatcodeCheck{97}{122}{11}%
  \RangeCatcodeCheck{123}{255}{15}%
  \RestoreCatcodes
}
\Test
\csname @@end\endcsname
\end
%    \end{macrocode}
%    \begin{macrocode}
%</test1>
%    \end{macrocode}
%
% \subsection{Test with package \xpackage{url}}
%
%    \begin{macrocode}
%<*test2>
\nofiles
\documentclass[a5paper]{article}
\usepackage{url}[2005/06/27]
\usepackage{setouterhbox}

\newsavebox{\testbox}

\setlength{\parindent}{0pt}
\setlength{\parskip}{2em}

\begin{document}
\raggedright

\url{http://this.is.a.very.long.host.name/followed/%
by/a/very_long_long_long_path.html}%

\sbox\testbox{%
  \url{http://this.is.a.very.long.host.name/followed/%
  by/a/very_long_long_long_path.html}%
}%
\unhbox\testbox

\begin{setouterhbox}{\testbox}%
  \url{http://this.is.a.very.long.host.name/followed/%
  by/a/very_long_long_long_path.html}%
\end{setouterhbox}
\unhbox\testbox

\end{document}
%</test2>
%    \end{macrocode}
%
% \section{Installation}
%
% \subsection{Download}
%
% \paragraph{Package.} This package is available on
% CTAN\footnote{\CTANpkg{setouterhbox}}:
% \begin{description}
% \item[\CTAN{macros/latex/contrib/oberdiek/setouterhbox.dtx}] The source file.
% \item[\CTAN{macros/latex/contrib/oberdiek/setouterhbox.pdf}] Documentation.
% \end{description}
%
%
% \paragraph{Bundle.} All the packages of the bundle `oberdiek'
% are also available in a TDS compliant ZIP archive. There
% the packages are already unpacked and the documentation files
% are generated. The files and directories obey the TDS standard.
% \begin{description}
% \item[\CTANinstall{install/macros/latex/contrib/oberdiek.tds.zip}]
% \end{description}
% \emph{TDS} refers to the standard ``A Directory Structure
% for \TeX\ Files'' (\CTAN{tds/tds.pdf}). Directories
% with \xfile{texmf} in their name are usually organized this way.
%
% \subsection{Bundle installation}
%
% \paragraph{Unpacking.} Unpack the \xfile{oberdiek.tds.zip} in the
% TDS tree (also known as \xfile{texmf} tree) of your choice.
% Example (linux):
% \begin{quote}
%   |unzip oberdiek.tds.zip -d ~/texmf|
% \end{quote}
%
% \subsection{Package installation}
%
% \paragraph{Unpacking.} The \xfile{.dtx} file is a self-extracting
% \docstrip\ archive. The files are extracted by running the
% \xfile{.dtx} through \plainTeX:
% \begin{quote}
%   \verb|tex setouterhbox.dtx|
% \end{quote}
%
% \paragraph{TDS.} Now the different files must be moved into
% the different directories in your installation TDS tree
% (also known as \xfile{texmf} tree):
% \begin{quote}
% \def\t{^^A
% \begin{tabular}{@{}>{\ttfamily}l@{ $\rightarrow$ }>{\ttfamily}l@{}}
%   setouterhbox.sty & tex/generic/oberdiek/setouterhbox.sty\\
%   setouterhbox.pdf & doc/latex/oberdiek/setouterhbox.pdf\\
%   setouterhbox-example.tex & doc/latex/oberdiek/setouterhbox-example.tex\\
%   test/setouterhbox-test1.tex & doc/latex/oberdiek/test/setouterhbox-test1.tex\\
%   test/setouterhbox-test2.tex & doc/latex/oberdiek/test/setouterhbox-test2.tex\\
%   setouterhbox.dtx & source/latex/oberdiek/setouterhbox.dtx\\
% \end{tabular}^^A
% }^^A
% \sbox0{\t}^^A
% \ifdim\wd0>\linewidth
%   \begingroup
%     \advance\linewidth by\leftmargin
%     \advance\linewidth by\rightmargin
%   \edef\x{\endgroup
%     \def\noexpand\lw{\the\linewidth}^^A
%   }\x
%   \def\lwbox{^^A
%     \leavevmode
%     \hbox to \linewidth{^^A
%       \kern-\leftmargin\relax
%       \hss
%       \usebox0
%       \hss
%       \kern-\rightmargin\relax
%     }^^A
%   }^^A
%   \ifdim\wd0>\lw
%     \sbox0{\small\t}^^A
%     \ifdim\wd0>\linewidth
%       \ifdim\wd0>\lw
%         \sbox0{\footnotesize\t}^^A
%         \ifdim\wd0>\linewidth
%           \ifdim\wd0>\lw
%             \sbox0{\scriptsize\t}^^A
%             \ifdim\wd0>\linewidth
%               \ifdim\wd0>\lw
%                 \sbox0{\tiny\t}^^A
%                 \ifdim\wd0>\linewidth
%                   \lwbox
%                 \else
%                   \usebox0
%                 \fi
%               \else
%                 \lwbox
%               \fi
%             \else
%               \usebox0
%             \fi
%           \else
%             \lwbox
%           \fi
%         \else
%           \usebox0
%         \fi
%       \else
%         \lwbox
%       \fi
%     \else
%       \usebox0
%     \fi
%   \else
%     \lwbox
%   \fi
% \else
%   \usebox0
% \fi
% \end{quote}
% If you have a \xfile{docstrip.cfg} that configures and enables \docstrip's
% TDS installing feature, then some files can already be in the right
% place, see the documentation of \docstrip.
%
% \subsection{Refresh file name databases}
%
% If your \TeX~distribution
% (\TeX\,Live, \mikTeX, \dots) relies on file name databases, you must refresh
% these. For example, \TeX\,Live\ users run \verb|texhash| or
% \verb|mktexlsr|.
%
% \subsection{Some details for the interested}
%
% \paragraph{Unpacking with \LaTeX.}
% The \xfile{.dtx} chooses its action depending on the format:
% \begin{description}
% \item[\plainTeX:] Run \docstrip\ and extract the files.
% \item[\LaTeX:] Generate the documentation.
% \end{description}
% If you insist on using \LaTeX\ for \docstrip\ (really,
% \docstrip\ does not need \LaTeX), then inform the autodetect routine
% about your intention:
% \begin{quote}
%   \verb|latex \let\install=y% \iffalse meta-comment
%
% File: setouterhbox.dtx
% Version: 2016/05/16 v1.8
% Info: Set hbox in outer horizontal mode
%
% Copyright (C)
%    2005-2007 Heiko Oberdiek
%    2016-2019 Oberdiek Package Support Group
%    https://github.com/ho-tex/oberdiek/issues
%
% This work may be distributed and/or modified under the
% conditions of the LaTeX Project Public License, either
% version 1.3c of this license or (at your option) any later
% version. This version of this license is in
%    https://www.latex-project.org/lppl/lppl-1-3c.txt
% and the latest version of this license is in
%    https://www.latex-project.org/lppl.txt
% and version 1.3 or later is part of all distributions of
% LaTeX version 2005/12/01 or later.
%
% This work has the LPPL maintenance status "maintained".
%
% The Current Maintainers of this work are
% Heiko Oberdiek and the Oberdiek Package Support Group
% https://github.com/ho-tex/oberdiek/issues
%
% The Base Interpreter refers to any `TeX-Format',
% because some files are installed in TDS:tex/generic//.
%
% This work consists of the main source file setouterhbox.dtx
% and the derived files
%    setouterhbox.sty, setouterhbox.pdf, setouterhbox.ins, setouterhbox.drv,
%    setouterhbox-example.tex, setouterhbox-test1.tex,
%    setouterhbox-test2.tex.
%
% Distribution:
%    CTAN:macros/latex/contrib/oberdiek/setouterhbox.dtx
%    CTAN:macros/latex/contrib/oberdiek/setouterhbox.pdf
%
% Unpacking:
%    (a) If setouterhbox.ins is present:
%           tex setouterhbox.ins
%    (b) Without setouterhbox.ins:
%           tex setouterhbox.dtx
%    (c) If you insist on using LaTeX
%           latex \let\install=y\input{setouterhbox.dtx}
%        (quote the arguments according to the demands of your shell)
%
% Documentation:
%    (a) If setouterhbox.drv is present:
%           latex setouterhbox.drv
%    (b) Without setouterhbox.drv:
%           latex setouterhbox.dtx; ...
%    The class ltxdoc loads the configuration file ltxdoc.cfg
%    if available. Here you can specify further options, e.g.
%    use A4 as paper format:
%       \PassOptionsToClass{a4paper}{article}
%
%    Programm calls to get the documentation (example):
%       pdflatex setouterhbox.dtx
%       makeindex -s gind.ist setouterhbox.idx
%       pdflatex setouterhbox.dtx
%       makeindex -s gind.ist setouterhbox.idx
%       pdflatex setouterhbox.dtx
%
% Installation:
%    TDS:tex/generic/oberdiek/setouterhbox.sty
%    TDS:doc/latex/oberdiek/setouterhbox.pdf
%    TDS:doc/latex/oberdiek/setouterhbox-example.tex
%    TDS:doc/latex/oberdiek/test/setouterhbox-test1.tex
%    TDS:doc/latex/oberdiek/test/setouterhbox-test2.tex
%    TDS:source/latex/oberdiek/setouterhbox.dtx
%
%<*ignore>
\begingroup
  \catcode123=1 %
  \catcode125=2 %
  \def\x{LaTeX2e}%
\expandafter\endgroup
\ifcase 0\ifx\install y1\fi\expandafter
         \ifx\csname processbatchFile\endcsname\relax\else1\fi
         \ifx\fmtname\x\else 1\fi\relax
\else\csname fi\endcsname
%</ignore>
%<*install>
\input docstrip.tex
\Msg{************************************************************************}
\Msg{* Installation}
\Msg{* Package: setouterhbox 2016/05/16 v1.8 Set hbox in outer horizontal mode (HO)}
\Msg{************************************************************************}

\keepsilent
\askforoverwritefalse

\let\MetaPrefix\relax
\preamble

This is a generated file.

Project: setouterhbox
Version: 2016/05/16 v1.8

Copyright (C)
   2005-2007 Heiko Oberdiek
   2016-2019 Oberdiek Package Support Group

This work may be distributed and/or modified under the
conditions of the LaTeX Project Public License, either
version 1.3c of this license or (at your option) any later
version. This version of this license is in
   https://www.latex-project.org/lppl/lppl-1-3c.txt
and the latest version of this license is in
   https://www.latex-project.org/lppl.txt
and version 1.3 or later is part of all distributions of
LaTeX version 2005/12/01 or later.

This work has the LPPL maintenance status "maintained".

The Current Maintainers of this work are
Heiko Oberdiek and the Oberdiek Package Support Group
https://github.com/ho-tex/oberdiek/issues


The Base Interpreter refers to any `TeX-Format',
because some files are installed in TDS:tex/generic//.

This work consists of the main source file setouterhbox.dtx
and the derived files
   setouterhbox.sty, setouterhbox.pdf, setouterhbox.ins, setouterhbox.drv,
   setouterhbox-example.tex, setouterhbox-test1.tex,
   setouterhbox-test2.tex.

\endpreamble
\let\MetaPrefix\DoubleperCent

\generate{%
  \file{setouterhbox.ins}{\from{setouterhbox.dtx}{install}}%
  \file{setouterhbox.drv}{\from{setouterhbox.dtx}{driver}}%
  \usedir{tex/generic/oberdiek}%
  \file{setouterhbox.sty}{\from{setouterhbox.dtx}{package}}%
  \usedir{doc/latex/oberdiek}%
  \file{setouterhbox-example.tex}{\from{setouterhbox.dtx}{example}}%
%  \usedir{doc/latex/oberdiek/test}%
%  \file{setouterhbox-test1.tex}{\from{setouterhbox.dtx}{test1}}%
%  \file{setouterhbox-test2.tex}{\from{setouterhbox.dtx}{test2}}%
  \nopreamble
  \nopostamble
%  \usedir{source/latex/oberdiek/catalogue}%
%  \file{setouterhbox.xml}{\from{setouterhbox.dtx}{catalogue}}%
}

\catcode32=13\relax% active space
\let =\space%
\Msg{************************************************************************}
\Msg{*}
\Msg{* To finish the installation you have to move the following}
\Msg{* file into a directory searched by TeX:}
\Msg{*}
\Msg{*     setouterhbox.sty}
\Msg{*}
\Msg{* To produce the documentation run the file `setouterhbox.drv'}
\Msg{* through LaTeX.}
\Msg{*}
\Msg{* Happy TeXing!}
\Msg{*}
\Msg{************************************************************************}

\endbatchfile
%</install>
%<*ignore>
\fi
%</ignore>
%<*driver>
\NeedsTeXFormat{LaTeX2e}
\ProvidesFile{setouterhbox.drv}%
  [2016/05/16 v1.8 Set hbox in outer horizontal mode (HO)]%
\documentclass{ltxdoc}
\usepackage{holtxdoc}[2011/11/22]
\begin{document}
  \DocInput{setouterhbox.dtx}%
\end{document}
%</driver>
% \fi
%
%
% \CharacterTable
%  {Upper-case    \A\B\C\D\E\F\G\H\I\J\K\L\M\N\O\P\Q\R\S\T\U\V\W\X\Y\Z
%   Lower-case    \a\b\c\d\e\f\g\h\i\j\k\l\m\n\o\p\q\r\s\t\u\v\w\x\y\z
%   Digits        \0\1\2\3\4\5\6\7\8\9
%   Exclamation   \!     Double quote  \"     Hash (number) \#
%   Dollar        \$     Percent       \%     Ampersand     \&
%   Acute accent  \'     Left paren    \(     Right paren   \)
%   Asterisk      \*     Plus          \+     Comma         \,
%   Minus         \-     Point         \.     Solidus       \/
%   Colon         \:     Semicolon     \;     Less than     \<
%   Equals        \=     Greater than  \>     Question mark \?
%   Commercial at \@     Left bracket  \[     Backslash     \\
%   Right bracket \]     Circumflex    \^     Underscore    \_
%   Grave accent  \`     Left brace    \{     Vertical bar  \|
%   Right brace   \}     Tilde         \~}
%
% \GetFileInfo{setouterhbox.drv}
%
% \title{The \xpackage{setouterhbox} package}
% \date{2016/05/16 v1.8}
% \author{Heiko Oberdiek\thanks
% {Please report any issues at \url{https://github.com/ho-tex/oberdiek/issues}}}
%
% \maketitle
%
% \begin{abstract}
% If math stuff is set in an \cs{hbox}, then TeX
% performs some optimization and omits the implicite
% penalties \cs{binoppenalty} and \cs{relpenalty}.
% This packages tries to put stuff into an \cs{hbox}
% without getting lost of those penalties.
% \end{abstract}
%
% \tableofcontents
%
% \section{Documentation}
%
% \subsection{Introduction}
%
% There is a situation in \xpackage{hyperref}'s driver for dvips
% where the user wants to have links that can be broken across
% lines. However dvips doesn't support the feature. With option
% \xoption{breaklinks} \xpackage{hyperref} sets the links as
% usual, put them in a box and write the link data with
% box dimensions into the appropriate \cs{special}s.
% Then, however, it does not set the complete unbreakable
% box, but it unwrappes the material inside to allow line
% breaks. Of course line breaking and glue setting will falsify
% the link dimensions, but line breaking was more important
% for the user.
%
% \subsection{Acknowledgement}
%
% Jonathan Fine, Donald Arsenau and me discussed the problem
% in the newsgroup \xnewsgroup{comp.text.tex} where Damian
% Menscher has started the thread, see \cite{newsstart}.
%
% The discussion was productive and generated many ideas
% and code examples. In order to have a more permanent
% result I wrote this package and tried to implement
% most of the ideas, a kind of summary of the discussion.
% Thus I want and have to thank Jonathan Fine and Donald Arsenau
% very much.
%
% Two weeks later David Kastrup (posting in
% \xnewsgroup{comp.text.tex}, \cite{kastrup})
% remembered an old article of Michael Downes (\cite{downes})
% in TUGboat, where Michael Downes already presented the
% method we discuss here. Nowadays we have \eTeX\ that extends
% the tool set of a \TeX\ macro programmer. Especially useful
% \eTeX\ was in this package for detecting and dealing with
% errorneous situations.
%
% However also nowadays a perfect solution for the problem
% is still missing at macro level. Probably someone has
% to go deep in the internals of the \TeX\ compiler to
% implement a switch that let penalties stay where otherwise
% \TeX\ would remove them for optimization reasons.
%
% \subsection{Usage}
%
% \paragraph{Package loading.}
% \LaTeX: as usually:
% \begin{quote}
%   |\usepackage{setouterhbox}|
% \end{quote}
% The package can also be included directly, thus \plainTeX\ users
% write:
% \begin{quote}
%   |\input setouterhbox.sty|
% \end{quote}
%
% \paragraph{Register allocation.}
% The material will be put into a box, thus we need to know these
% box number. If you need to allocate a new box register:
% \begin{description}
%  \item[\LaTeX:] |\newsavebox{\|\meta{name}|}|
%  \item[\plainTeX:] |\newbox\|\meta{name}
% \end{description}
% Then |\|\meta{name} is a command that held the box number.
%
% \paragraph{Box wrapping.}
% \LaTeX\ users put the material in the box with an environment
% similar to \texttt{lrbox}. The environment \texttt{setouterhbox}
% uses the same syntax and offers the same features, such
% as verbatim stuff inside:
% \begin{quote}
%  |\begin{setouterhbox}{|\meta{box number}|}|\dots
%  |\end{setouterhbox}|
% \end{quote}
% Users with \plainTeX\ do not have environments, they use instead:
% \begin{quote}
%   |\setouterhbox{|\meta{box number}|}|\dots|\endsetouterhbox|
% \end{quote}
% In both cases the material is put into an \cs{hbox} and assigned
% to the given box, denoted by \meta{box number}. Note the
% assignment is local, the same way \texttt{lrbox} behaves.
%
% \paragraph{Unwrapping.}
% The box material is ready for unwrapping:
% \begin{quote}
%   |\unhbox|\meta{box number}
% \end{quote}
%
% \subsection{Option \xoption{hyperref}}
%
% Package url uses math mode for typesetting urls.
% Break points are inserted by \cs{binoppenalty} and
% \cs{relpenalty}. Unhappily these break points are
% removed, if \xpackage{hyperref}
% is used with option {breaklinks}
% and drivers that depend on \xoption{pdfmark}:
% \xoption{dvips}, \xoption{vtexpdfmark}, \xoption{textures},
% and \xoption{dvipsone}.
% Thus the option \xoption{hyperref} enables the method
% of this package to avoid the removal of \cs{relpenalty}
% and \cs{binoppenalty}. Thus you get more break points.
% However, the link areas are still wrong for these
% drivers, because they are not supporting broken
% links.
%
% Note, you need version 2006/08/16 v6.75c of package \xpackage{hyperref},
% because starting with this version the necessary hook is provided
% that package \xpackage{setouterhbox} uses.
% \begin{quote}
%   |\usepackage[|\dots|]{hyperref}[2006/08/16]|\\
%   |\usepackage[hyperref]{setouterhbox}|
% \end{quote}
% Package order does not matter.
%
% \subsection{Example}
%
%    \begin{macrocode}
%<*example>
\documentclass[a5paper]{article}
\usepackage{url}[2005/06/27]
\usepackage{setouterhbox}

\newsavebox{\testbox}

\setlength{\parindent}{0pt}
\setlength{\parskip}{2em}

\begin{document}
\raggedright

\url{http://this.is.a.very.long.host.name/followed/%
by/a/very_long_long_long_path.html}%

\sbox\testbox{%
  \url{http://this.is.a.very.long.host.name/followed/%
  by/a/very_long_long_long_path.html}%
}%
\unhbox\testbox

\begin{setouterhbox}{\testbox}%
  \url{http://this.is.a.very.long.host.name/followed/%
  by/a/very_long_long_long_path.html}%
\end{setouterhbox}
\unhbox\testbox

\end{document}
%</example>
%    \end{macrocode}
%
% \StopEventually{
% }
%
% \section{Implementation}
%
% Internal macros are prefixed by \cs{setouterhbox}, |@| is
% not used inside names, thus we do not need to care of its
% catcode if we are not using it as \LaTeX\ package.
%
% \subsection{Package start stuff}
%
%    \begin{macrocode}
%<*package>
%    \end{macrocode}
%
% Prevent reloading more than one, necessary for \plainTeX:
%    Reload check, especially if the package is not used with \LaTeX.
%    \begin{macrocode}
\begingroup\catcode61\catcode48\catcode32=10\relax%
  \catcode13=5 % ^^M
  \endlinechar=13 %
  \catcode35=6 % #
  \catcode39=12 % '
  \catcode44=12 % ,
  \catcode45=12 % -
  \catcode46=12 % .
  \catcode58=12 % :
  \catcode64=11 % @
  \catcode123=1 % {
  \catcode125=2 % }
  \expandafter\let\expandafter\x\csname ver@setouterhbox.sty\endcsname
  \ifx\x\relax % plain-TeX, first loading
  \else
    \def\empty{}%
    \ifx\x\empty % LaTeX, first loading,
      % variable is initialized, but \ProvidesPackage not yet seen
    \else
      \expandafter\ifx\csname PackageInfo\endcsname\relax
        \def\x#1#2{%
          \immediate\write-1{Package #1 Info: #2.}%
        }%
      \else
        \def\x#1#2{\PackageInfo{#1}{#2, stopped}}%
      \fi
      \x{setouterhbox}{The package is already loaded}%
      \aftergroup\endinput
    \fi
  \fi
\endgroup%
%    \end{macrocode}
%    Package identification:
%    \begin{macrocode}
\begingroup\catcode61\catcode48\catcode32=10\relax%
  \catcode13=5 % ^^M
  \endlinechar=13 %
  \catcode35=6 % #
  \catcode39=12 % '
  \catcode40=12 % (
  \catcode41=12 % )
  \catcode44=12 % ,
  \catcode45=12 % -
  \catcode46=12 % .
  \catcode47=12 % /
  \catcode58=12 % :
  \catcode64=11 % @
  \catcode91=12 % [
  \catcode93=12 % ]
  \catcode123=1 % {
  \catcode125=2 % }
  \expandafter\ifx\csname ProvidesPackage\endcsname\relax
    \def\x#1#2#3[#4]{\endgroup
      \immediate\write-1{Package: #3 #4}%
      \xdef#1{#4}%
    }%
  \else
    \def\x#1#2[#3]{\endgroup
      #2[{#3}]%
      \ifx#1\@undefined
        \xdef#1{#3}%
      \fi
      \ifx#1\relax
        \xdef#1{#3}%
      \fi
    }%
  \fi
\expandafter\x\csname ver@setouterhbox.sty\endcsname
\ProvidesPackage{setouterhbox}%
  [2016/05/16 v1.8 Set hbox in outer horizontal mode (HO)]%
%    \end{macrocode}
%
%    \begin{macrocode}
\begingroup\catcode61\catcode48\catcode32=10\relax%
  \catcode13=5 % ^^M
  \endlinechar=13 %
  \catcode123=1 % {
  \catcode125=2 % }
  \catcode64=11 % @
  \def\x{\endgroup
    \expandafter\edef\csname setouterhboxAtEnd\endcsname{%
      \endlinechar=\the\endlinechar\relax
      \catcode13=\the\catcode13\relax
      \catcode32=\the\catcode32\relax
      \catcode35=\the\catcode35\relax
      \catcode61=\the\catcode61\relax
      \catcode64=\the\catcode64\relax
      \catcode123=\the\catcode123\relax
      \catcode125=\the\catcode125\relax
    }%
  }%
\x\catcode61\catcode48\catcode32=10\relax%
\catcode13=5 % ^^M
\endlinechar=13 %
\catcode35=6 % #
\catcode64=11 % @
\catcode123=1 % {
\catcode125=2 % }
\def\TMP@EnsureCode#1#2{%
  \edef\setouterhboxAtEnd{%
    \setouterhboxAtEnd
    \catcode#1=\the\catcode#1\relax
  }%
  \catcode#1=#2\relax
}
\TMP@EnsureCode{40}{12}% (
\TMP@EnsureCode{41}{12}% )
\TMP@EnsureCode{44}{12}% ,
\TMP@EnsureCode{45}{12}% -
\TMP@EnsureCode{46}{12}% .
\TMP@EnsureCode{47}{12}% /
\TMP@EnsureCode{58}{12}% :
\TMP@EnsureCode{60}{12}% <
\TMP@EnsureCode{62}{12}% >
\TMP@EnsureCode{91}{12}% [
\TMP@EnsureCode{93}{12}% ]
\TMP@EnsureCode{96}{12}% `
\edef\setouterhboxAtEnd{\setouterhboxAtEnd\noexpand\endinput}
%    \end{macrocode}
%
% \subsection{Interface macros}
%
%    \begin{macro}{\setouterhboxBox}
% The method requires a global box assignment. To be on the
% safe side, a new box register is allocated for this
% global box assignment.
%    \begin{macrocode}
\newbox\setouterhboxBox
%    \end{macrocode}
%    \end{macro}
%
%    \begin{macro}{\setouterhboxFailure}
% Error message for both \plainTeX\ and \LaTeX
%    \begin{macrocode}
\begingroup\expandafter\expandafter\expandafter\endgroup
\expandafter\ifx\csname RequirePackage\endcsname\relax
  \input infwarerr.sty\relax
\else
  \RequirePackage{infwarerr}[2016/05/16]%
\fi
\edef\setouterhboxFailure#1#2{%
  \expandafter\noexpand\csname @PackageError\endcsname
      {setouterhbox}{#1}{#2}%
}
%    \end{macrocode}
%    \end{macro}
%
% \subsection{Main part}
%
% eTeX provides much better means for checking
% error conditions. Thus lines marked by "E" are executed
% if eTeX is available, otherwise the lines marked by "T" are
% used.
%    \begin{macrocode}
\begingroup\expandafter\expandafter\expandafter\endgroup
\expandafter\ifx\csname lastnodetype\endcsname\relax
  \catcode`T=9 % ignore
  \catcode`E=14 % comment
\else
  \catcode`T=14 % comment
  \catcode`E=9 % ignore
\fi
%    \end{macrocode}
%
%    \begin{macro}{\setouterhboxRemove}
% Remove all kern, glue, and penalty nodes;
% poor man's version, if \eTeX\ is not available
%    \begin{macrocode}
\def\setouterhboxRemove{%
E \ifnum\lastnodetype<11 %
E   \else
E   \ifnum\lastnodetype>13 %
E   \else
      \unskip\unkern\unpenalty
E     \expandafter\expandafter\expandafter\setouterhboxRemove
E   \fi
E \fi
}%
%    \end{macrocode}
%    \end{macro}
%
%    \begin{macro}{\setouterhbox}
% Passing the box contents by macro parameter would prevent
% catcode changes in the box contents like by \cs{verb}.
% Also \cs{bgroup} and \cs{egroup} does not work, because stuff
% has to be added at the begin and end of the box, thus
% the syntax
% |\setouterhbox{|\meta{box number}|}|\dots|\endsetouterhbox|
% is used. Also we automatically get an environment \texttt{setouterhbox}
% if \LaTeX\ is used.
%    \begin{macrocode}
\def\setouterhbox#1{%
  \begingroup
    \def\setouterhboxNum{#1}%
    \setbox0\vbox\bgroup
T     \kern.123pt\relax % marker
T     \kern0pt\relax % removed by \setouterhboxRemove
      \begingroup
        \everypar{}%
        \noindent
}
%    \end{macrocode}
%    \end{macro}
%    \begin{macro}{\endsetouterhbox}
% Most of the work is done in the end part, thus the heart of
% the method follows:
%    \begin{macrocode}
\def\endsetouterhbox{%
      \endgroup
%    \end{macrocode}
% Omit the first pass to get the penalties
% of the second pass.
%    \begin{macrocode}
      \pretolerance-1 %
%    \end{macrocode}
%  We don't want a third pass with \cs{emergencystretch}.
%    \begin{macrocode}
      \tolerance10000 %
      \hsize\maxdimen
%    \end{macrocode}
% Line is not underfull:
%    \begin{macrocode}
      \parfillskip 0pt plus 1filll\relax
      \leftskip0pt\relax
%    \end{macrocode}
% Suppress underful \cs{hbox} warnings,
% is explicit line breaks are used.
%    \begin{macrocode}
      \rightskip0pt plus 1fil\relax
      \everypar{}%
%    \end{macrocode}
% Ensure that there is a paragraph and
% prevents \cs{endgraph} from eating terminal glue:
%    \begin{macrocode}
      \kern0pt%
      \endgraf
      \setouterhboxRemove
E     \ifnum\lastnodetype=1 %
E       \global\setbox\setouterhboxBox\lastbox
E       \loop
E         \setouterhboxRemove
E       \ifnum\lastnodetype=1 %
E         \setbox0=\lastbox
E         \global\setbox\setouterhboxBox=\hbox{%
E           \unhbox0 %
%    \end{macrocode}
% Remove \cs{rightskip}, a penalty with -10000 is part of the previous line.
%    \begin{macrocode}
E           \unskip
E           \unhbox\setouterhboxBox
E         }%
E       \repeat
E     \else
E       \setouterhboxFailure{%
E         Something is wrong%
E       }{%
E         Could not find expected line.%
E         \MessageBreak
E         (\string\lastnodetype: \number\lastnodetype, expected: 1)%
E       }%
E     \fi
E     \setouterhboxRemove
T     \global\setbox\setouterhboxBox\lastbox
T     \loop
T       \setouterhboxRemove
T       \setbox0=\lastbox
T     \ifcase\ifvoid0 1\else0\fi
T       \global\setbox\setouterhboxBox=\hbox{%
T         \unhbox0 %
%    \end{macrocode}
% Remove \cs{rightskip}, a penalty with -10000 is part of the previous line.
%    \begin{macrocode}
T         \unskip
T         \unhbox\setouterhboxBox
T       }%
T     \repeat
T     \ifdim.123pt=\lastkern
T     \else
T       \setouterhboxFailure{%
T         Something is wrong%
T       }{%
T         Unexpected stuff was detected before the line.%
T       }%
T     \fi
T   \egroup
T   \ifcase \ifnum\wd0=0 \else 1\fi
T           \ifdim\ht0=.123pt \else 1\fi
T           \ifnum\dp0=0 \else 1\fi
T           0 %
E     \ifnum\lastnodetype=-1 %
%    \end{macrocode}
% There was just one line that we have caught.
%    \begin{macrocode}
      \else
        \setouterhboxFailure{%
            Something is wrong%
        }{%
            After fetching the line there is more unexpected stuff.%
E           \MessageBreak
E           (\string\lastnodetype: \number\lastnodetype, expected: -1)%
        }%
      \fi
E   \egroup
  \expandafter\endgroup
  \expandafter\setouterhboxFinish\expandafter{%
    \number\setouterhboxNum
  }%
}
%    \end{macrocode}
%    \end{macro}
%
% \subsection{Environment support}
%
% Check \cs{@currenvir} for the case that \cs{setouterhbox}
% was called as environment. Then the box assignment
% must be put after the \cs{endgroup} of |\end{|\dots|}|.
%    \begin{macrocode}
\def\setouterhboxCurr{setouterhbox}
\def\setouterhboxLast#1{%
  \setbox#1\hbox{%
    \unhbox\setouterhboxBox
    \unskip % remove \rightskip glue
    \unskip % remove \parfillskip glue
    \unpenalty % remove paragraph ending \penalty 10000
    \unkern % remove explicit kern inserted above
  }%
}
%    \end{macrocode}
%    \begin{macro}{\setouterhboxFinish}
% |#1| is an explicit number.
%    \begin{macrocode}
\def\setouterhboxFinish#1{%
  \begingroup\expandafter\expandafter\expandafter\endgroup
  \expandafter\ifx\csname @currenvir\endcsname\setouterhboxCurr
    \aftergroup\setouterhboxLast
    \aftergroup{%
    \setouterhboxAfter #1\NIL
    \aftergroup}%
  \else
    \setouterhboxLast{#1}%
  \fi
}
%    \end{macrocode}
%    \end{macro}
%    \begin{macro}{\setouterhboxAfter}
% |#1| is an explicit number.
%    \begin{macrocode}
\def\setouterhboxAfter#1#2\NIL{%
  \aftergroup#1%
  \ifx\\#2\\%
  \else
    \setouterhboxReturnAfterFi{%
      \setouterhboxAfter#2\NIL
    }%
  \fi
}
%    \end{macrocode}
%    \end{macro}
%    \begin{macro}{\setouterhboxReturnAfterFi}
% A utility macro to get tail recursion.
%    \begin{macrocode}
\long\def\setouterhboxReturnAfterFi#1\fi{\fi#1}
%    \end{macrocode}
%    \end{macro}
% Restore catcodes we have need to distinguish between
% the implementation with and without \eTeX.
%    \begin{macrocode}
\catcode69=11\relax % E
\catcode84=11\relax % T
%    \end{macrocode}
%
% \subsection{Option \xoption{hyperref}}
%    \begin{macrocode}
\begingroup
  \def\x{LaTeX2e}%
\expandafter\endgroup
\ifx\x\fmtname
\else
  \expandafter\setouterhboxAtEnd
\fi%
%    \end{macrocode}
%    \begin{macro}{\Hy@setouterhbox}
% \cs{Hy@setouterhbox} is the internal hook that \xpackage{hyperref}
% uses since 2006/02/12 v6.75a.
%    \begin{macrocode}
\DeclareOption{hyperref}{%
  \long\def\Hy@setouterhbox#1#2{%
    \setouterhbox{#1}#2\endsetouterhbox
  }%
}
%    \end{macrocode}
%    \end{macro}
%    \begin{macrocode}
\ProcessOptions\relax
%    \end{macrocode}
%
%    \begin{macrocode}
\setouterhboxAtEnd%
%</package>
%    \end{macrocode}
%
% \section{Test}
%
% \subsection{Catcode checks for loading}
%
%    \begin{macrocode}
%<*test1>
%    \end{macrocode}
%    \begin{macrocode}
\catcode`\{=1 %
\catcode`\}=2 %
\catcode`\#=6 %
\catcode`\@=11 %
\expandafter\ifx\csname count@\endcsname\relax
  \countdef\count@=255 %
\fi
\expandafter\ifx\csname @gobble\endcsname\relax
  \long\def\@gobble#1{}%
\fi
\expandafter\ifx\csname @firstofone\endcsname\relax
  \long\def\@firstofone#1{#1}%
\fi
\expandafter\ifx\csname loop\endcsname\relax
  \expandafter\@firstofone
\else
  \expandafter\@gobble
\fi
{%
  \def\loop#1\repeat{%
    \def\body{#1}%
    \iterate
  }%
  \def\iterate{%
    \body
      \let\next\iterate
    \else
      \let\next\relax
    \fi
    \next
  }%
  \let\repeat=\fi
}%
\def\RestoreCatcodes{}
\count@=0 %
\loop
  \edef\RestoreCatcodes{%
    \RestoreCatcodes
    \catcode\the\count@=\the\catcode\count@\relax
  }%
\ifnum\count@<255 %
  \advance\count@ 1 %
\repeat

\def\RangeCatcodeInvalid#1#2{%
  \count@=#1\relax
  \loop
    \catcode\count@=15 %
  \ifnum\count@<#2\relax
    \advance\count@ 1 %
  \repeat
}
\def\RangeCatcodeCheck#1#2#3{%
  \count@=#1\relax
  \loop
    \ifnum#3=\catcode\count@
    \else
      \errmessage{%
        Character \the\count@\space
        with wrong catcode \the\catcode\count@\space
        instead of \number#3%
      }%
    \fi
  \ifnum\count@<#2\relax
    \advance\count@ 1 %
  \repeat
}
\def\space{ }
\expandafter\ifx\csname LoadCommand\endcsname\relax
  \def\LoadCommand{\input setouterhbox.sty\relax}%
\fi
\def\Test{%
  \RangeCatcodeInvalid{0}{47}%
  \RangeCatcodeInvalid{58}{64}%
  \RangeCatcodeInvalid{91}{96}%
  \RangeCatcodeInvalid{123}{255}%
  \catcode`\@=12 %
  \catcode`\\=0 %
  \catcode`\%=14 %
  \LoadCommand
  \RangeCatcodeCheck{0}{36}{15}%
  \RangeCatcodeCheck{37}{37}{14}%
  \RangeCatcodeCheck{38}{47}{15}%
  \RangeCatcodeCheck{48}{57}{12}%
  \RangeCatcodeCheck{58}{63}{15}%
  \RangeCatcodeCheck{64}{64}{12}%
  \RangeCatcodeCheck{65}{90}{11}%
  \RangeCatcodeCheck{91}{91}{15}%
  \RangeCatcodeCheck{92}{92}{0}%
  \RangeCatcodeCheck{93}{96}{15}%
  \RangeCatcodeCheck{97}{122}{11}%
  \RangeCatcodeCheck{123}{255}{15}%
  \RestoreCatcodes
}
\Test
\csname @@end\endcsname
\end
%    \end{macrocode}
%    \begin{macrocode}
%</test1>
%    \end{macrocode}
%
% \subsection{Test with package \xpackage{url}}
%
%    \begin{macrocode}
%<*test2>
\nofiles
\documentclass[a5paper]{article}
\usepackage{url}[2005/06/27]
\usepackage{setouterhbox}

\newsavebox{\testbox}

\setlength{\parindent}{0pt}
\setlength{\parskip}{2em}

\begin{document}
\raggedright

\url{http://this.is.a.very.long.host.name/followed/%
by/a/very_long_long_long_path.html}%

\sbox\testbox{%
  \url{http://this.is.a.very.long.host.name/followed/%
  by/a/very_long_long_long_path.html}%
}%
\unhbox\testbox

\begin{setouterhbox}{\testbox}%
  \url{http://this.is.a.very.long.host.name/followed/%
  by/a/very_long_long_long_path.html}%
\end{setouterhbox}
\unhbox\testbox

\end{document}
%</test2>
%    \end{macrocode}
%
% \section{Installation}
%
% \subsection{Download}
%
% \paragraph{Package.} This package is available on
% CTAN\footnote{\CTANpkg{setouterhbox}}:
% \begin{description}
% \item[\CTAN{macros/latex/contrib/oberdiek/setouterhbox.dtx}] The source file.
% \item[\CTAN{macros/latex/contrib/oberdiek/setouterhbox.pdf}] Documentation.
% \end{description}
%
%
% \paragraph{Bundle.} All the packages of the bundle `oberdiek'
% are also available in a TDS compliant ZIP archive. There
% the packages are already unpacked and the documentation files
% are generated. The files and directories obey the TDS standard.
% \begin{description}
% \item[\CTANinstall{install/macros/latex/contrib/oberdiek.tds.zip}]
% \end{description}
% \emph{TDS} refers to the standard ``A Directory Structure
% for \TeX\ Files'' (\CTAN{tds/tds.pdf}). Directories
% with \xfile{texmf} in their name are usually organized this way.
%
% \subsection{Bundle installation}
%
% \paragraph{Unpacking.} Unpack the \xfile{oberdiek.tds.zip} in the
% TDS tree (also known as \xfile{texmf} tree) of your choice.
% Example (linux):
% \begin{quote}
%   |unzip oberdiek.tds.zip -d ~/texmf|
% \end{quote}
%
% \subsection{Package installation}
%
% \paragraph{Unpacking.} The \xfile{.dtx} file is a self-extracting
% \docstrip\ archive. The files are extracted by running the
% \xfile{.dtx} through \plainTeX:
% \begin{quote}
%   \verb|tex setouterhbox.dtx|
% \end{quote}
%
% \paragraph{TDS.} Now the different files must be moved into
% the different directories in your installation TDS tree
% (also known as \xfile{texmf} tree):
% \begin{quote}
% \def\t{^^A
% \begin{tabular}{@{}>{\ttfamily}l@{ $\rightarrow$ }>{\ttfamily}l@{}}
%   setouterhbox.sty & tex/generic/oberdiek/setouterhbox.sty\\
%   setouterhbox.pdf & doc/latex/oberdiek/setouterhbox.pdf\\
%   setouterhbox-example.tex & doc/latex/oberdiek/setouterhbox-example.tex\\
%   test/setouterhbox-test1.tex & doc/latex/oberdiek/test/setouterhbox-test1.tex\\
%   test/setouterhbox-test2.tex & doc/latex/oberdiek/test/setouterhbox-test2.tex\\
%   setouterhbox.dtx & source/latex/oberdiek/setouterhbox.dtx\\
% \end{tabular}^^A
% }^^A
% \sbox0{\t}^^A
% \ifdim\wd0>\linewidth
%   \begingroup
%     \advance\linewidth by\leftmargin
%     \advance\linewidth by\rightmargin
%   \edef\x{\endgroup
%     \def\noexpand\lw{\the\linewidth}^^A
%   }\x
%   \def\lwbox{^^A
%     \leavevmode
%     \hbox to \linewidth{^^A
%       \kern-\leftmargin\relax
%       \hss
%       \usebox0
%       \hss
%       \kern-\rightmargin\relax
%     }^^A
%   }^^A
%   \ifdim\wd0>\lw
%     \sbox0{\small\t}^^A
%     \ifdim\wd0>\linewidth
%       \ifdim\wd0>\lw
%         \sbox0{\footnotesize\t}^^A
%         \ifdim\wd0>\linewidth
%           \ifdim\wd0>\lw
%             \sbox0{\scriptsize\t}^^A
%             \ifdim\wd0>\linewidth
%               \ifdim\wd0>\lw
%                 \sbox0{\tiny\t}^^A
%                 \ifdim\wd0>\linewidth
%                   \lwbox
%                 \else
%                   \usebox0
%                 \fi
%               \else
%                 \lwbox
%               \fi
%             \else
%               \usebox0
%             \fi
%           \else
%             \lwbox
%           \fi
%         \else
%           \usebox0
%         \fi
%       \else
%         \lwbox
%       \fi
%     \else
%       \usebox0
%     \fi
%   \else
%     \lwbox
%   \fi
% \else
%   \usebox0
% \fi
% \end{quote}
% If you have a \xfile{docstrip.cfg} that configures and enables \docstrip's
% TDS installing feature, then some files can already be in the right
% place, see the documentation of \docstrip.
%
% \subsection{Refresh file name databases}
%
% If your \TeX~distribution
% (\TeX\,Live, \mikTeX, \dots) relies on file name databases, you must refresh
% these. For example, \TeX\,Live\ users run \verb|texhash| or
% \verb|mktexlsr|.
%
% \subsection{Some details for the interested}
%
% \paragraph{Unpacking with \LaTeX.}
% The \xfile{.dtx} chooses its action depending on the format:
% \begin{description}
% \item[\plainTeX:] Run \docstrip\ and extract the files.
% \item[\LaTeX:] Generate the documentation.
% \end{description}
% If you insist on using \LaTeX\ for \docstrip\ (really,
% \docstrip\ does not need \LaTeX), then inform the autodetect routine
% about your intention:
% \begin{quote}
%   \verb|latex \let\install=y\input{setouterhbox.dtx}|
% \end{quote}
% Do not forget to quote the argument according to the demands
% of your shell.
%
% \paragraph{Generating the documentation.}
% You can use both the \xfile{.dtx} or the \xfile{.drv} to generate
% the documentation. The process can be configured by the
% configuration file \xfile{ltxdoc.cfg}. For instance, put this
% line into this file, if you want to have A4 as paper format:
% \begin{quote}
%   \verb|\PassOptionsToClass{a4paper}{article}|
% \end{quote}
% An example follows how to generate the
% documentation with pdf\LaTeX:
% \begin{quote}
%\begin{verbatim}
%pdflatex setouterhbox.dtx
%makeindex -s gind.ist setouterhbox.idx
%pdflatex setouterhbox.dtx
%makeindex -s gind.ist setouterhbox.idx
%pdflatex setouterhbox.dtx
%\end{verbatim}
% \end{quote}
%
% \begin{thebibliography}{9}
%
% \bibitem{newsstart}
%   Damian Menscher, \Newsgroup{comp.text.tex},
%   \textit{overlong lines in List of Figures},
%   \nolinkurl{<dh058t$qbd$1@news.ks.uiuc.edu>},
%   23rd September 2005.
%   \url{https://groups.google.com/group/comp.text.tex/msg/79648d4cf1f8bc13}
%
% \bibitem{kastrup}
%   David Kastrup, \Newsgroup{comp.text.tex},
%   \textit{Re: ANN: outerhbox.sty -- collect horizontal material,
%   for unboxing into a paragraph},
%   \nolinkurl{<85y855lrx3.fsf@lola.goethe.zz>},
%   7th October 2005.
%   \url{https://groups.google.com/group/comp.text.tex/msg/7cf0a345ef932e52}
%
% \bibitem{downes}
%   Michael Downes, \textit{Line breaking in \cs{unhbox}ed Text},
%   TUGboat 11 (1990), pp. 605--612.
%
% \bibitem{hyperref}
%   Sebastian Rahtz, Heiko Oberdiek:
%   \textit{The \xpackage{hyperref} package};
%   2006/08/16 v6.75c;
%   \CTANpkg{hyperref}.
%
% \end{thebibliography}
%
% \begin{History}
%   \begin{Version}{2005/10/05 v1.0}
%   \item
%     First version.
%   \end{Version}
%   \begin{Version}{2005/10/07 v1.1}
%   \item
%     Option \xoption{hyperref} added.
%   \end{Version}
%   \begin{Version}{2005/10/18 v1.2}
%   \item
%     Support for explicit line breaks added.
%   \end{Version}
%   \begin{Version}{2006/02/12 v1.3}
%   \item
%     DTX format.
%   \item
%     Documentation extended.
%   \end{Version}
%   \begin{Version}{2006/08/26 v1.4}
%   \item
%     Date of hyperref updated.
%   \end{Version}
%   \begin{Version}{2007/04/26 v1.5}
%   \item
%     Use of package \xpackage{infwarerr}.
%   \end{Version}
%   \begin{Version}{2007/05/17 v1.6}
%   \item
%     Standard header part for generic files.
%   \end{Version}
%   \begin{Version}{2007/09/09 v1.7}
%   \item
%     Catcode section added.
%   \end{Version}
%   \begin{Version}{2016/05/16 v1.8}
%   \item
%     Documentation updates.
%   \end{Version}
% \end{History}
%
% \PrintIndex
%
% \Finale
\endinput
|
% \end{quote}
% Do not forget to quote the argument according to the demands
% of your shell.
%
% \paragraph{Generating the documentation.}
% You can use both the \xfile{.dtx} or the \xfile{.drv} to generate
% the documentation. The process can be configured by the
% configuration file \xfile{ltxdoc.cfg}. For instance, put this
% line into this file, if you want to have A4 as paper format:
% \begin{quote}
%   \verb|\PassOptionsToClass{a4paper}{article}|
% \end{quote}
% An example follows how to generate the
% documentation with pdf\LaTeX:
% \begin{quote}
%\begin{verbatim}
%pdflatex setouterhbox.dtx
%makeindex -s gind.ist setouterhbox.idx
%pdflatex setouterhbox.dtx
%makeindex -s gind.ist setouterhbox.idx
%pdflatex setouterhbox.dtx
%\end{verbatim}
% \end{quote}
%
% \begin{thebibliography}{9}
%
% \bibitem{newsstart}
%   Damian Menscher, \Newsgroup{comp.text.tex},
%   \textit{overlong lines in List of Figures},
%   \nolinkurl{<dh058t$qbd$1@news.ks.uiuc.edu>},
%   23rd September 2005.
%   \url{https://groups.google.com/group/comp.text.tex/msg/79648d4cf1f8bc13}
%
% \bibitem{kastrup}
%   David Kastrup, \Newsgroup{comp.text.tex},
%   \textit{Re: ANN: outerhbox.sty -- collect horizontal material,
%   for unboxing into a paragraph},
%   \nolinkurl{<85y855lrx3.fsf@lola.goethe.zz>},
%   7th October 2005.
%   \url{https://groups.google.com/group/comp.text.tex/msg/7cf0a345ef932e52}
%
% \bibitem{downes}
%   Michael Downes, \textit{Line breaking in \cs{unhbox}ed Text},
%   TUGboat 11 (1990), pp. 605--612.
%
% \bibitem{hyperref}
%   Sebastian Rahtz, Heiko Oberdiek:
%   \textit{The \xpackage{hyperref} package};
%   2006/08/16 v6.75c;
%   \CTANpkg{hyperref}.
%
% \end{thebibliography}
%
% \begin{History}
%   \begin{Version}{2005/10/05 v1.0}
%   \item
%     First version.
%   \end{Version}
%   \begin{Version}{2005/10/07 v1.1}
%   \item
%     Option \xoption{hyperref} added.
%   \end{Version}
%   \begin{Version}{2005/10/18 v1.2}
%   \item
%     Support for explicit line breaks added.
%   \end{Version}
%   \begin{Version}{2006/02/12 v1.3}
%   \item
%     DTX format.
%   \item
%     Documentation extended.
%   \end{Version}
%   \begin{Version}{2006/08/26 v1.4}
%   \item
%     Date of hyperref updated.
%   \end{Version}
%   \begin{Version}{2007/04/26 v1.5}
%   \item
%     Use of package \xpackage{infwarerr}.
%   \end{Version}
%   \begin{Version}{2007/05/17 v1.6}
%   \item
%     Standard header part for generic files.
%   \end{Version}
%   \begin{Version}{2007/09/09 v1.7}
%   \item
%     Catcode section added.
%   \end{Version}
%   \begin{Version}{2016/05/16 v1.8}
%   \item
%     Documentation updates.
%   \end{Version}
% \end{History}
%
% \PrintIndex
%
% \Finale
\endinput
|
% \end{quote}
% Do not forget to quote the argument according to the demands
% of your shell.
%
% \paragraph{Generating the documentation.}
% You can use both the \xfile{.dtx} or the \xfile{.drv} to generate
% the documentation. The process can be configured by the
% configuration file \xfile{ltxdoc.cfg}. For instance, put this
% line into this file, if you want to have A4 as paper format:
% \begin{quote}
%   \verb|\PassOptionsToClass{a4paper}{article}|
% \end{quote}
% An example follows how to generate the
% documentation with pdf\LaTeX:
% \begin{quote}
%\begin{verbatim}
%pdflatex setouterhbox.dtx
%makeindex -s gind.ist setouterhbox.idx
%pdflatex setouterhbox.dtx
%makeindex -s gind.ist setouterhbox.idx
%pdflatex setouterhbox.dtx
%\end{verbatim}
% \end{quote}
%
% \begin{thebibliography}{9}
%
% \bibitem{newsstart}
%   Damian Menscher, \Newsgroup{comp.text.tex},
%   \textit{overlong lines in List of Figures},
%   \nolinkurl{<dh058t$qbd$1@news.ks.uiuc.edu>},
%   23rd September 2005.
%   \url{https://groups.google.com/group/comp.text.tex/msg/79648d4cf1f8bc13}
%
% \bibitem{kastrup}
%   David Kastrup, \Newsgroup{comp.text.tex},
%   \textit{Re: ANN: outerhbox.sty -- collect horizontal material,
%   for unboxing into a paragraph},
%   \nolinkurl{<85y855lrx3.fsf@lola.goethe.zz>},
%   7th October 2005.
%   \url{https://groups.google.com/group/comp.text.tex/msg/7cf0a345ef932e52}
%
% \bibitem{downes}
%   Michael Downes, \textit{Line breaking in \cs{unhbox}ed Text},
%   TUGboat 11 (1990), pp. 605--612.
%
% \bibitem{hyperref}
%   Sebastian Rahtz, Heiko Oberdiek:
%   \textit{The \xpackage{hyperref} package};
%   2006/08/16 v6.75c;
%   \CTANpkg{hyperref}.
%
% \end{thebibliography}
%
% \begin{History}
%   \begin{Version}{2005/10/05 v1.0}
%   \item
%     First version.
%   \end{Version}
%   \begin{Version}{2005/10/07 v1.1}
%   \item
%     Option \xoption{hyperref} added.
%   \end{Version}
%   \begin{Version}{2005/10/18 v1.2}
%   \item
%     Support for explicit line breaks added.
%   \end{Version}
%   \begin{Version}{2006/02/12 v1.3}
%   \item
%     DTX format.
%   \item
%     Documentation extended.
%   \end{Version}
%   \begin{Version}{2006/08/26 v1.4}
%   \item
%     Date of hyperref updated.
%   \end{Version}
%   \begin{Version}{2007/04/26 v1.5}
%   \item
%     Use of package \xpackage{infwarerr}.
%   \end{Version}
%   \begin{Version}{2007/05/17 v1.6}
%   \item
%     Standard header part for generic files.
%   \end{Version}
%   \begin{Version}{2007/09/09 v1.7}
%   \item
%     Catcode section added.
%   \end{Version}
%   \begin{Version}{2016/05/16 v1.8}
%   \item
%     Documentation updates.
%   \end{Version}
% \end{History}
%
% \PrintIndex
%
% \Finale
\endinput
|
% \end{quote}
% Do not forget to quote the argument according to the demands
% of your shell.
%
% \paragraph{Generating the documentation.}
% You can use both the \xfile{.dtx} or the \xfile{.drv} to generate
% the documentation. The process can be configured by the
% configuration file \xfile{ltxdoc.cfg}. For instance, put this
% line into this file, if you want to have A4 as paper format:
% \begin{quote}
%   \verb|\PassOptionsToClass{a4paper}{article}|
% \end{quote}
% An example follows how to generate the
% documentation with pdf\LaTeX:
% \begin{quote}
%\begin{verbatim}
%pdflatex setouterhbox.dtx
%makeindex -s gind.ist setouterhbox.idx
%pdflatex setouterhbox.dtx
%makeindex -s gind.ist setouterhbox.idx
%pdflatex setouterhbox.dtx
%\end{verbatim}
% \end{quote}
%
% \begin{thebibliography}{9}
%
% \bibitem{newsstart}
%   Damian Menscher, \Newsgroup{comp.text.tex},
%   \textit{overlong lines in List of Figures},
%   \nolinkurl{<dh058t$qbd$1@news.ks.uiuc.edu>},
%   23rd September 2005.
%   \url{https://groups.google.com/group/comp.text.tex/msg/79648d4cf1f8bc13}
%
% \bibitem{kastrup}
%   David Kastrup, \Newsgroup{comp.text.tex},
%   \textit{Re: ANN: outerhbox.sty -- collect horizontal material,
%   for unboxing into a paragraph},
%   \nolinkurl{<85y855lrx3.fsf@lola.goethe.zz>},
%   7th October 2005.
%   \url{https://groups.google.com/group/comp.text.tex/msg/7cf0a345ef932e52}
%
% \bibitem{downes}
%   Michael Downes, \textit{Line breaking in \cs{unhbox}ed Text},
%   TUGboat 11 (1990), pp. 605--612.
%
% \bibitem{hyperref}
%   Sebastian Rahtz, Heiko Oberdiek:
%   \textit{The \xpackage{hyperref} package};
%   2006/08/16 v6.75c;
%   \CTANpkg{hyperref}.
%
% \end{thebibliography}
%
% \begin{History}
%   \begin{Version}{2005/10/05 v1.0}
%   \item
%     First version.
%   \end{Version}
%   \begin{Version}{2005/10/07 v1.1}
%   \item
%     Option \xoption{hyperref} added.
%   \end{Version}
%   \begin{Version}{2005/10/18 v1.2}
%   \item
%     Support for explicit line breaks added.
%   \end{Version}
%   \begin{Version}{2006/02/12 v1.3}
%   \item
%     DTX format.
%   \item
%     Documentation extended.
%   \end{Version}
%   \begin{Version}{2006/08/26 v1.4}
%   \item
%     Date of hyperref updated.
%   \end{Version}
%   \begin{Version}{2007/04/26 v1.5}
%   \item
%     Use of package \xpackage{infwarerr}.
%   \end{Version}
%   \begin{Version}{2007/05/17 v1.6}
%   \item
%     Standard header part for generic files.
%   \end{Version}
%   \begin{Version}{2007/09/09 v1.7}
%   \item
%     Catcode section added.
%   \end{Version}
%   \begin{Version}{2016/05/16 v1.8}
%   \item
%     Documentation updates.
%   \end{Version}
% \end{History}
%
% \PrintIndex
%
% \Finale
\endinput
